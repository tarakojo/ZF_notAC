\documentclass{article}

\usepackage{graphicx} 
\usepackage{luacode}  
\usepackage{ascmac}
\usepackage{comment}
\usepackage{amsthm}
\usepackage{amsmath}
\usepackage{newtxtext, newtxmath}

\newtheorem{thm}{Theorem}[section]
\newtheorem{lem}[thm]{Lemma}
\newtheorem{dfn}[thm]{Definition}

%短縮形
\newcommand{\Pbb}{\mathbb{P}}
\newcommand{\Gcal}{\mathcal{G}}
\newcommand{\Fcal}{\mathcal{F}}
\newcommand{\Ncal}{\mathcal{N}}

\begin{document}

\title{MyTitle}
\author{MyName}
\date{\today}
\maketitle

\section{Introduction}

\section{Preliminaries}
In this section, we introduce basic concepts of ZF set theory, forcing, and symmetric extensions.
We use first-order logic with the language of set theory, which consists only of only two relation symbols $\in$ and $=$. 
Formulas involving other mathmatical operators that may appear are considered abbreviations for formulas in this language.
Parentheses in formulas are omitted where no confusion arises.
Unless otherwise stated, "a statement holds" means "the statement holds in ZF".

\subsection{ZF set theory and the axiom of choice}
\begin{dfn}
  The axioms of ZF are the following statements:
  \begin{itemize}
    \item Extensionality: $\forall x \forall y (\forall z (z \in x \leftrightarrow z \in y) \rightarrow x = y)$
    \item Pairing: $\forall x \forall y \exists z \forall w (w \in z \leftrightarrow w = x \lor w = y)$
    \item Union: $\forall x \exists y \forall z (z \in y \leftrightarrow \exists w (z \in w \land w \in x))$
    \item Power set: $\forall x \exists y \forall z (z \in y \leftrightarrow z \subseteq x)$
    \item Infinity: $\exists x (\emptyset \in x \land \forall y (y \in x \rightarrow y \cup \{y\} \in x))$
    \item Regularity: $\forall x(x \neq \emptyset \rightarrow \exists y (y \in x \land y \cap x = \emptyset))$
    \item Infinity: $\forall x (x \neq \emptyset \rightarrow \exists y (y \in x \land \forall z (z \in x \rightarrow z \notin y)))$
    \item Separation: $\forall p \forall x \exists y \forall z (z \in y \leftrightarrow z \in x \land \phi(z, p))$
    \item Replacement: $\forall p (\forall x \forall y \forall z (\phi(x, y, p) \land \phi(x, z, p) \rightarrow y = z) \rightarrow \\ \forall X \exists Y \forall y (y \in Y \leftrightarrow \exists x (x \in X \land \phi(x, y, p))))$
  \end{itemize}
\end{dfn}
Where separation and replacement are axiom schemas, representing infinitely many axioms for each formula $\phi$ with an appropriate arity.
\begin{dfn} 
  \emph{The axiom of choice (AC)} is the following statement: \\
  $\forall x \exists f (\text{"} f \text{ is a function on } x \text{"} \land \forall y (y \in x \rightarrow f(y) \in y))$
\end{dfn}
Where the phrase "f is a function on x" is also considered an abbreviation in the language of set theory.
Theory ZF + AC is denoted by ZFC.
Next, we introduce the well-ordering theorem, as we treat AC in this form.

\begin{dfn}
  We say that a linear ordering $<$ on a set $P$ is a \emph{well-ordering} if, every non-empty subset of $P$, it has a least element.
\end{dfn}

\begin{lem}
  The axiom of choice is equivalent to the well-ordering theorem, which states that every set can be well-ordered.
\end{lem}

\subsection{Forcing}
Forcing is a tequnique used in proving relative consistency and Independence.
We introduce basic concepts of forcing in the context of the transitive countable model (c.t.m.) approach.
In this approach, the relative consistency proof is achieved by using forcing to construct a extended model by adding new sets to an assumed c.t.m.
Let $M$ be a c.t.m. of ZF and $(\Pbb, \leq_{\Pbb})$ be a pre-ordered set in $M$ with a maximum element $1_{\Pbb}$.

\begin{dfn}
  We define $M^{\Pbb}$, the set of \emph{$\Pbb$-names}, by transfinite recursion on ordinals:
  \begin{enumerate}
    \item $M^{\Pbb}_0 = \emptyset$
    \item $M^{\Pbb}_{\alpha + 1} = \mathcal{P}^M(M^{\Pbb}_{\alpha} \times \Pbb)$
    \item $M^{\Pbb}_{\alpha} = \bigcup_{\beta < \alpha} M^{\Pbb}_{\beta}$ for a limit ordinal $\alpha$
    \item $M^{\Pbb} = \bigcup_{\alpha \in \text{Ord}} M^{\Pbb}_{\alpha}$
  \end{enumerate}
\end{dfn}
Where $\mathcal{P}^M$ denotes the power set operation in $M$. 
We often write a $\Pbb$-name with a dot, e.g., $\dot{x}$.

\begin{dfn} %karagilaではタイポがある...
  \,
  \begin{enumerate}
    \item We say that $D \subseteq \Pbb$ is \emph{dense} if, for every $p \in \Pbb$, there exists $q \in D$ such that $q \leq_{\Pbb} p$.
    \item We say that $G \subseteq \Pbb$ is a \emph{filter} if following conditions hold:
      \begin{itemize}
        \item If $p \in G$, $q \in \Pbb$, and $p \leq_{\Pbb} q$, then $q \in G$
        \item If $p, q \in G$, there exists $r \in G$ such that $r \leq_{\Pbb} p$ and $r \leq_{\Pbb} q$
      \end{itemize}
    \item We say that $G \subseteq$ is \emph{generic filter} on $\Pbb$ if $G$ is a filter and for any dense $D \subseteq \Pbb$, $D \cap G \neq \emptyset$.
  \end{enumerate}
\end{dfn}

The following lemma shows that a generic filter actually exists.
\begin{lem} 
  For any $p \in \Pbb$, there exists a generic filter $G$ on $\Pbb$ such that $p \in G$.
\end{lem}

\begin{dfn} 
  Let $G$ be a generic filter on $\Pbb$ and $\dot{x} \in M^{\Pbb}$. We define the \emph{interpretation} of $\dot{x}$ denoted by $\dot{x}_G$ recursively, 
  $\dot{x}_G = \{\dot{y}_G \mid \exists p \in G (\langle \dot{y}, p \rangle \in \dot{x})\}$. 
\end{dfn}

We call a $Pbb$-name whose interpretation is a set $x$ a name of $x$ and denote it by $\dot{x}$.
Note that a single set may have multiple names.

\begin{dfn} 
  Let $G$ be a generic filter on $\Pbb$. We define a \emph{generic extension} $M[G]$ as $\{x_G \mid \dot{x} \in M^{\Pbb}\}$.
\end{dfn}

\begin{thm} 
  Let $G$ be a generic filter on $\Pbb$. Then, $M[G]$ is the smallest c.t.m. of ZF extending $M$ and containing $G$.
\end{thm}

By choosing $\Pbb$ appropriately, we can construct $M[G]$ with various properties.
What holds or does not hold in $M[G]$ can be identified using the forcing relation.

\begin{dfn} [Forcing relation]
  ATODE
\end{dfn}

\subsection{Symmetric extensions}
Symmetric extensions are substructures of generic extensions of a given c.t.m. of ZF 
and are formed by interpreting only the hereditarily symmetric names.
Let $M$ be a c.t.m. of ZF, $(\Pbb, \leq_{\Pbb})$ be a pre-ordered set in $M$ with the maximum element $1_{\Pbb}$.


\begin{dfn} 
  We say that $\pi : \Pbb \rightarrow \Pbb$ is an \emph{automorphism} if for all $p, q \in \Pbb$, $p \leq_{\Pbb} q \Leftrightarrow \pi p \leq_{\Pbb} \pi q$.
  $\pi$ induces an bijection on $\Pbb$-names defined recursively as follows:
  $$ \pi \dot{x} = \{ \langle \pi \dot{y}, \pi p \rangle \mid \langle \dot{y}, p \rangle \in \dot{x} \} $$
\end{dfn}

\begin{dfn} % lara
  Let $\Gcal$ be a group of automorphisms of $\Pbb$. We say that $\Fcal$ is a \emph{normal filter} on $\Gcal$ if the following conditions hold:
  \begin{enumerate}
    \item $\Fcal$ is non-empty family of subgroups of $\Gcal$.
    \item $\Fcal$ is closed under finite intersections and supergroups.
    \item For every $H \in \Fcal$ and $\pi \in \Gcal$, $\pi H \pi^{-1} \in \Fcal$.
  \end{enumerate}
\end{dfn}

We fix a group of automorphisms $\Gcal$ of $\Pbb$ and a normal filter $\Fcal$ on $\Gcal$.
\begin{dfn} 
  For every $\Pbb$-name $\dot{x}$, let $\text{sym}_{\Gcal}(\dot{x}) = \{ \pi \in \Gcal \mid \pi \dot{x} = \dot{x} \}$.
  We say that $\Pbb$-name $\dot{x}$ is \emph{hereditarily $\Fcal$-symmetric} if $\text{sym}_{\Gcal}(\dot{x}) \in \Fcal$.
  $\text{HS}_{\Fcal}$ denotes the set of all hereditarily $\Fcal$-symmetric $\Pbb$-names.
\end{dfn}

\begin{dfn} 
  Let $G$ be a generic filter on $\Pbb$. The set $\text{HS}^{G}_{\Fcal} = \{ \dot{x}_G \mid \dot{x} \in \text{HS}_{\Fcal} \}$ is called a \emph{symmetric extension} of $M$.
\end{dfn}

\begin{thm}
  Let $G$ be a generic filter on $\Pbb$. Then, the symmetric extension $\text{HS}^{G}_{\Fcal}$ is a c.t.m. of ZF and a substructure of $M[G]$.
\end{thm}


\begin{dfn} [HS forcing relation]
  ATODE
\end{dfn}

\subsection{The basic Cohen model} % jech axoim of choice
We construct a symmetric extension $\Ncal$ called the basic Cohen model which is a model of ZF + $\neg$AC using forcing.
Let $M$ be a c.t.m. of ZF, $\Pbb$ be the set of all finite partial functions from $\omega \times \omega$ to $\{0, 1\}$, and $\leq_{\Pbb}$ be $\supseteq$.
Note that the maximum element $1_{\Pbb}$ is the empty set. 
Let $\pi$ be a bijection on $\omega$. $\pi$ induces an automorphism on $\Pbb$ defined as follows:

$$dom(\pi p) = \{ (\pi n, m) \mid (n, m) \in dom(p) \}$$
$$(\pi p)(\pi n, m) = p(n, m)$$

This automorphism further induces an automorphism on $\Pbb$-names. 
Let $\Gcal$ be the group of all such automorphisms.
For every finite $e \subseteq \omega$, let 
$$\text{fix}(e) = \{ \pi \in \Gcal \mid \forall n \in e (\pi n = n) \}$$
Let $\Fcal$ be the set of all subgroups $H$ of $\Gcal$ such that there exists a finite $e \subseteq \omega$ with $\text{fix}(e) \subseteq H$.
Note that $\Fcal$ is a normal filter on $\Gcal$. Let $\Ncal = \text{HS}^{G}_{\Fcal}$.
Since $\Ncal$ is a symmetric extension of $M$, it is a c.t.m. of ZF. 

\begin{thm}
  $\Ncal$ does not satisfy the well-ordering theorem.
\end{thm}

We outline the proof of this theorem as follows. 
For every $n \in \omega$, let $a_n$ be the following real number:
$$a_n = \{ m \in \omega \mid \exists p \in G (p(n, m) = 1) \}$$
Since $a_n$ are pairwise distinct, $A = \{ a_n \mid n \in \omega \}$ is an infinity set.
$A$ and every $a_n$ are elements of $\Ncal$.
$A$ serves as a counterexample to the well-ordering theorem in $\Ncal$.
Suppose for contradiction that $A$ is well-ordered in $\Ncal$, there exists a injection $f$ from $\omega$ to $A$ in $\Ncal$.
Let $\varphi(g,x,y)$ be a formula that represents the relation $g(x) = y$. 
For every $n \in \omega$, There exists $i \in \omega$ such that $N \vDash \varphi(f, i, a_n)$.
Thus there exists $p \in G$ and hereditarily $\Fcal$-symmetric names $\dot{f}, \dot{i}$ and $\dot{a_n}$ for each of $f, i, a_n$ such that $p \Vdash \varphi(\dot{f}, \dot{i}, \dot{a_n})$.
By choosing $n$ and the names appropriately, we can find $\pi \in \Gcal$ such that the following conditions hold:
\begin{enumerate}
  \item $\pi \dot{f} = \dot{f}$
  \item $\pi \dot{i} = \dot{i}$
  \item $\pi n \ne n$
  \item There exists a hereditarily $\Fcal$-symmetric name $\dot{a_{\pi n}}$ of $a_{\pi n}$ such that $\pi \dot{a_n} = \dot{a_{\pi n}}$
\end{enumerate}







\begin{comment}

  理論Tが理論S上で相対的無矛盾であるとは、Tが無矛盾であると仮定したとき、Sも無矛盾であることを意味する。
  
  完全性定理より、ある理論が無矛盾であることを示すには、その理論のモデルが存在することを示せば十分である。
  
  c.t.m.アプローチ
  本研究では、ZFのc.t.m.が存在することを仮定し、その上でZF + $\neg$ACのモデルを構成する。
\end{comment}


\section{Proof Outline}
In this section, we give an outline of a relative consistency proof of the negation of AC with respect to ZF using c.t.m. approach.


\begin{thebibliography}{9}
  \bibitem{jech} Jech, T. (2002). Set Theory: The Third Millennium Edition, Revised and Expanded. Springer.
  \bibitem{karagila} Karagila, A. (2023). Lecture Notes: Forcing \& Symmetric Extensions. % この引用でいい?
  \bibitem{kunen} Kunen, K. (1980). Set Theory: An Introduction to Independence Proofs. North-Holland.
\end{thebibliography}

\end{document}
