\documentclass{report}

\usepackage{luatexja-fontspec}
\usepackage{graphicx} 
\usepackage{luacode}  
\usepackage{ascmac}
\usepackage{comment}
\usepackage{amsthm}
\usepackage{amsmath}
\usepackage{amssymb}
\usepackage{xcolor}
\usepackage{authblk}

\usepackage{mdframed}
% isabelleコードの枠
\newenvironment{isaframe}{\begin{mdframed}[topline=false, rightline=false, bottomline=false]}{\end{mdframed}}

\usepackage{fontspec}
% コード用フォントはとりあえずこれで
% 等幅じゃないけど見やすい
\setmonofont{TeXGyreDejaVuMath-Regular}[Scale=0.75]

% fontspecは先にしないとエラーが出る
\usepackage{newtxtext,newtxmath}

% 参考文献
\usepackage[style=numeric,backend=biber]{biblatex}
\addbibresource{main.bib}

% URL
\usepackage{hyperref}
\hypersetup{luatex, pdfencoding=auto}

% コード表示
\usepackage[outputdir=out]{minted} 

\usepackage{cleveref}

% 定理
\newtheorem{thm}{Theorem}[section]
\newtheorem{lem}[thm]{Lemma}
\newtheorem{dfn}[thm]{Definition}
\newtheorem{cor}[thm]{Corollary}
\newtheorem{prop}[thm]{Proposition}

% 参照
\crefname{section}{Section}{Sections}
\crefname{dfn}{Definition}{Definitions}
\crefname{lem}{Lemma}{Lemmas}
\crefname{thm}{Theorem}{Theorems}
\crefname{cor}{Corollary}{Corollaries}
\crefname{prop}{Proposition}{Propositions} 

 
\title{Formal Proof of the Relative Consistency of ZF$+\neg$AC with Isabelle/ZF}
\author{Daiki Funane}
\affil{Graduate School of Information Sciences, Tohoku University}

\begin{document}

\maketitle

\chapter*{Abstract}
We formalize the relative consistency proof of ZF$+\neg$AC using Isabelle/ZF proof assistant.
Our approach assumes the existence of a transitive countable model of ZF and uses forcing to construct a symmetric extension which is a model of ZF$+\neg$AC.
We show that the symmetric extension satisfies ZF$+\neg$AC by formalizing a relativized forcing relation based on the formalization of forcing by Gunther et al.

\chapter*{Acknowledgements}
本研究を進めるにあたり、多くの方々にご指導、ご助言をいただきました。
この場を借りて心より御礼申し上げます。

まず、大変多くのご指導を賜りました住井 英二郎教授ならびに織田 幸弘氏に、深く感謝いたします。
研究という活動は初めてでしたが、お二人のご指導のおかげで、何とか本論文を執筆することができました。

また、本論文の副審査委員を務めていただきました、横山 啓太教授、中野 圭介教授に感謝申し上げます。
横山教授には、研究の過程においても数学的な視点からアドバイスをいただきました。

さらに、研究発表をお聞きくださり、建設的なご意見を賜りました薄葉 季路教授、湯山 孝雄氏、谷口 雅弥氏、
そして、ゼミ発表の場で多くの示唆を与えてくださった松田 一孝准教授、西田 雄気助教、Oleg Kiselyov氏に感謝申し上げます。

最後に、家族をはじめとする多くの方々に学生生活を支えていただきましたことに、深く感謝いたします。


\clearpage
\tableofcontents
\clearpage




\chapter{Introduction}
The formalization of mathematics using proof assistants such as Isabelle \cite{paulson_1986}, Coq, and Lean, has been actively conducted, leading to numerous achievements. 
For instance, the proofs of the four color theorem, Kepler's conjecture, and the Feit-Thompson theorem have been formalized using proof assistants, enhancing the reliability of these proofs. 
Additionally, various fields of mathematics, such as number theory, algebra, and topology, are also being formalized.

The independence of the axiom of choice (AC) from Zermelo-Fraenkel set theory (ZF) is a well-known result in the early history of axiomatic set theory,
as well as the independence of the continuum hypothesis (CH) from ZF with AC (ZFC).
Cohen invented the forcing method and proved them in 1963.
Forcing is a powerful tool for exploring models of set theory and was subsequently further sophisticated by other researchers.

Independence proofs of CH from ZFC has been formalized in Isabelle/ZF by Gunther et al. \cite{gunther_independence} and in Lean 3 by Han and van Doorn \cite{flypitch}.
In these studies, forcing methods were formalized, and the independence of CH was proven by 
showing the relative consistency of CH and $\neg$CH with ZFC.

For AC, the relative consistency of AC with ZF has been formalized in Isabelle/ZF by Paulson \cite{paulson_AC_consistency}.
However, the relative consistency of $\neg$AC with ZF has not been formalized.
It can be proven by forcing, but the proof involves complexities that cannot be achieved by simply modifying the proof for CH.

In this work, we formalized the relative consistency proof of $\neg$AC with ZF in Isabelle/ZF. 
This work contributes to the formalization of axiomatic set theory and 
serves as a new example of the formalization using forcing, which is a crucial tool in set theory.
It also may provide insights into how the formalization of axiomatic set theory could be advanced.


\subsubsection{Our Approach}
The primary reason we chose Isabelle/ZF for this formalization is that Isabelle/ZF is a mature proof assistant for ZF set theory,
in particular, the formalization by Gunther et al. \cite{gunther_forcing} is a major advantage for this study.
Although there is also a formalization of forcing in Lean 3 by Han and van Doorn \cite{flypitch}, development for Lean 3 has already ended.

To use Gunther et al.'s formalization, we adopted the c.t.m.\  approach for our proof, as in their independence proof of CH.
Specifically, we assumed the existence of a c.t.m.\  of ZF and constructed a model of ZF+$\neg$AC by forcing.
This model is known as the basic Cohen model, which is a symmetric extension of assumed c.t.m.\ 
Our proof is based on Karagila's lecture note \cite{karagila} and Jech's book \cite{jech_set_theory,jech_AC},
as these resources align well with this approach.

The c.t.m.\  approach is a well-established method for relative consistency proofs in axiomatic set theory.
The relative consistency of $\neg$AC with ZF means that if ZF is consistent, then ZF+$\neg$AC is also consistent.
Assuming the consistency of ZF implies that a model of ZF exists, but strictly speaking, the existence of a c.t.m.\  cannot be derived from this assumption.
However, as explained in \cref{subsection:ctm}, we can prove the relative consistency of $\neg$AC with ZF if 
we can construct a model of ZF+$\neg$AC from a c.t.m.\  of ZF.
This kind of reasoning always arises in the c.t.m.\  approach, and ideally, 
we would like to formalize it as well, but this has not been achieved.


\subsubsection{Related works}
Paulson et al. formalized an extensive part of ZF set theory 
\cite{paulson_datatype_impl,paulson_reflection,paulson_AC_consistency,paulson_cardinal_AC,paulson_datatype}
, including cardinal arithmetic, relativization, the reflection theorem, features for handling inductive definitions, and the relative consistency of AC with ZF.
The proof of the relative consistency was achieved by constructing Gödel's constructible universe.

Building on these results, Gunther et al. formalized forcing and a proof of the independence of CH \cite{gunther_forcing,gunther_independence} in Isabelle/ZF.
In these formalizations, the countable transitive model (c.t.m.) approach was used, following Kunen's book \cite{kunen2011}.

In Lean, Han and van Doorn also formalized forcing and the independence of CH \cite{flypitch} in Lean 3.
using the Boolean-valued model approach, which is another approach to forcing.
Additionally, in Lean 4, Holmes and Wilshaw formalized the complex parts of the consistency proof of Quine's New Foundations \cite{NF_consistency}, ensuring the correctness of the proof.


\subsubsection{Synopsis}
The remainder of this thesis is outlined as follows.
In Chapter 2, we first explain set-theoretic concepts in Section 2.1.
Section 2.2 outlines an informal proof of the relative consistency of $\neg$AC with ZF, which is formalized in Chapter 3.
In Section 2.3, we provide an overview of Isabelle/ZF and highlight key results from prior work that are relevant to this study.
Chapter 3 presents the formalization of the relative consistency proof of ZF+$\neg$AC, building on the contents of Chapter 2.
Finally, Chapter 4 concludes the thesis.

\subsubsection{Repository}
Our source code is available at: \url{https://github.com/tarakojo/ZF_notAC}




\chapter{Preliminaries}\label{chap:preliminaries}



\section{Set-Theoretic Preliminaries}
In this section, we introduce the concepts of set theory used in the formalization of this study.
We discuss in ZFC, within which we explore ZF itself. 
Basically, our definition and theorems combine Kunen \cite{kunen2011}, karagila's lecture note \cite{karagila}, and Jech's book \cite{jech_AC}
adapted to the formalized form in Isabelle/ZF.


\subsection{ZF Set Theory and the Axiom of Choice}
We introduce the axioms of ZF set theory and the axiom of choice (AC).
We use first-order logic with the language of set theory, which consists only of only two relation symbols $\in$ and $=$. 
Formulas involving other mathmatical operators that may appear are considered abbreviations for formulas in this language.
Parentheses in formulas are omitted where no confusion arises.

\begin{dfn}
  The axioms of ZF are the following statements:
  \begin{itemize}
    \item Extensionality: $\forall x \forall y (\forall z (z \in x \leftrightarrow z \in y) \rightarrow x = y)$
    \item Pairing: $\forall x \forall y \exists z \forall w (w \in z \leftrightarrow w = x \lor w = y)$
    \item Union: $\forall x \exists y \forall z (z \in y \leftrightarrow \exists w (z \in w \land w \in x))$
    \item Power set: $\forall x \exists y \forall z (z \in y \leftrightarrow z \subseteq x)$
    \item Infinity: $\exists x (\emptyset \in x \land \forall y (y \in x \rightarrow y \cup \{y\} \in x))$
    \item Regularity: $\forall x(x \neq \emptyset \rightarrow \exists y (y \in x \land y \cap x = \emptyset))$
    \item Infinity: $\forall x (x \neq \emptyset \rightarrow \exists y (y \in x \land \forall z (z \in x \rightarrow z \notin y)))$
    \item Separation: $\forall p \forall x \exists y \forall z (z \in y \leftrightarrow z \in x \land \phi(z, p))$
    \item Replacement: $\forall p (\forall x \forall y \forall z (\phi(x, y, p) \land \phi(x, z, p) \rightarrow y = z) \rightarrow \\ \forall X \exists Y \forall y (y \in Y \leftrightarrow \exists x (x \in X \land \phi(x, y, p))))$
  \end{itemize}
\end{dfn}
Where separation and replacement are axiom schemas, representing infinitely many axioms for each formula $\phi$ with an appropriate arity.
\begin{dfn} 
  \emph{The axiom of choice (AC)} is the following statement: \\
  $\forall x \exists f (\mathrm{"} f \text{ is a function on } x \mathrm{"} \land \forall y (y \in x \rightarrow f(y) \in y))$
\end{dfn}
Where the phrase "f is a function on x" is also considered an abbreviation in the language of set theory.
Theory ZF+AC is denoted by ZFC.
Additionally, we introduce the well-ordering theorem, as we treat AC in this form.

\begin{dfn}
  We say that a linear ordering $<$ on a set $P$ is a \emph{well-ordering} if, every non-empty subset of $P$, it has a least element.
\end{dfn}

\begin{lem}
  The axiom of choice is equivalent to the well-ordering theorem, which states that every set can be well-ordered.
\end{lem}


\subsection{Forcing}
Forcing is a tequnique used in proving relative consistency and Independence.
We introduce basic concepts of forcing in the context of the c.t.m.\  approach.
In this approach, the relative consistency proof is achieved by using forcing to construct a extended model by adding new sets to an assumed c.t.m.\ 
Let $M$ be a c.t.m.\  of ZF and $(\mathbb{P}, \leq_{\mathbb{P}})$ be a notion of forcing, which is a pre-ordered set in $M$ with a maximum element $1_{\mathbb{P}}$.

\begin{dfn}
  We define $M^{\mathbb{P}}$, the set of \emph{$\mathbb{P}$-names}, by transfinite recursion on ordinals:
  \begin{enumerate}
    \item $M^{\mathbb{P}}_0 = \emptyset$
    \item $M^{\mathbb{P}}_{\alpha + 1} = \mathcal{P}^M(M^{\mathbb{P}}_{\alpha} \times \mathbb{P})$
    \item $M^{\mathbb{P}}_{\alpha} = \bigcup_{\beta < \alpha} M^{\mathbb{P}}_{\beta}$ for a limit ordinal $\alpha$
    \item $M^{\mathbb{P}} = \bigcup_{\alpha \in \mathrm{Ord}} M^{\mathbb{P}}_{\alpha}$
  \end{enumerate}
\end{dfn}
Where $\mathcal{P}^M$ denotes the power set operation in $M$. 
We often write a $\mathbb{P}$-name with a dot, e.g., $\dot{x}$.
The least $\alpha$ such that $\dot{x} \in M^{\mathbb{P}}_{\alpha + 1}$ is called the \emph{$\mathbb{P}$-rank} of $\dot{x}$.
This allows us to define functions and relations by recursion on $\mathbb{P}$-names.

\begin{dfn}  
  \,
  \begin{enumerate}
    \item We say that $D \subseteq \mathbb{P}$ is \emph{dense} if, for every $p \in \mathbb{P}$, there exists $q \in D$ such that $q \leq_{\mathbb{P}} p$.
    \item We say that $G \subseteq \mathbb{P}$ is a \emph{filter} if following conditions hold:
      \begin{itemize}
        \item If $p \in G$, $q \in \mathbb{P}$, and $p \leq_{\mathbb{P}} q$, then $q \in G$
        \item If $p, q \in G$, there exists $r \in G$ such that $r \leq_{\mathbb{P}} p$ and $r \leq_{\mathbb{P}} q$
      \end{itemize}
    \item We say that $G \subseteq$ is \emph{generic filter} on $\mathbb{P}$ if $G$ is a filter and for any dense $D \subseteq \mathbb{P}$, $D \cap G \neq \emptyset$.
  \end{enumerate}
\end{dfn}

The following lemma shows that a generic filter actually exists.
\begin{lem} 
  For any $p \in \mathbb{P}$, there exists a generic filter $G$ on $\mathbb{P}$ such that $p \in G$.
\end{lem}

\begin{dfn} 
  Let $G$ be a generic filter on $\mathbb{P}$ and $\dot{x} \in M^{\mathbb{P}}$. We define the \emph{interpretation} of $\dot{x}$ denoted by $\dot{x}^G$ recursively with respect to the $\mathbb{P}$-rank of $\dot{x}$:
  $$\dot{x}^G = \{\dot{y}^G \mid \exists p \in G (\langle \dot{y}, p \rangle \in \dot{x})\}$$ 
\end{dfn}

We call a $\mathbb{P}$-name whose interpretation is a set $x$ a name of $x$ and denote it by $\dot{x}$.
Note that a single set may have multiple names.

\begin{dfn} 
  Let $G$ be a generic filter on $\mathbb{P}$. We define a \emph{generic extension} $M[G]$ as $\{x^G \mid \dot{x} \in M^{\mathbb{P}}\}$.
\end{dfn}

\begin{thm} 
  Let $G$ be a generic filter on $\mathbb{P}$. Then, $M[G]$ is the smallest c.t.m.\  of ZF extending $M$ and containing $G$.
\end{thm}

By choosing $\mathbb{P}$ appropriately, we can construct $M[G]$ with various properties.
What holds or does not hold in $M[G]$ can be identified using the forcing relation.
The forcing relation is defined recursively on formulas in a forcing language.
The forcing language is an extension of the language of set theory by adding the elements of $M^{\mathbb{P}}$ as constants.
\begin{dfn} \label{dfn:forcing_relation}
  We define the \emph{forcing relation} $\Vdash$ for formulas in the forcing language and $p \in \mathbb{P}$ inductively
  \footnote{
    Specifically, the forcing relation is first defined for atomic formulas $\dot{x} = \dot{y}$ and $\dot{x} \in \dot{y}$ by mutual recursion,
    inductively on the $\mathbb{P}$-rank of $\dot{x}$ and $\dot{y}$.
    Then, the definition is extended to all formulas in the forcing language by induction on the complexity of formulas.
  }
  as follows:
  \begin{enumerate}
    \item $p \Vdash \dot{x} = \dot{y} \Leftrightarrow \forall \dot{z} \in \mathrm{dom}(\dot{x}) \cup \mathrm{dom}(\dot{y})\,
\forall q \leq_{\mathbb{P}} p \, (q \Vdash \dot{z} \in \dot{x} \leftrightarrow q \Vdash \dot{z} \in \dot{y})$
    \item $p \Vdash \dot{x} \in \dot{y} \Leftrightarrow \forall q \leq_{\mathbb{P}} p\, \exists r \leq_{\mathbb{P}} q \, 
    \exists s \in \mathbb{P} \, \exists \dot{z} \in M^{\mathbb{P}} \, (\langle \dot{z}, s \rangle \in \dot{y} \land r \leq_{\mathbb{P}} s \land r \Vdash \dot{x} = \dot{z} )$
    \item $p \Vdash \phi \land \psi \Leftrightarrow p \Vdash \phi \land p \Vdash \psi$
    \item $p \Vdash \neg \phi \Leftrightarrow \forall q \leq_{\mathbb{P}} p \, \neg (q \Vdash \phi)$
    \item $p \Vdash \exists x \phi(x) \Leftrightarrow \forall q \leq_{\mathbb{P}} p \, \exists r \leq_{\mathbb{P}} q \, \exists \dot{x} \in M^{\mathbb{P}} \, (r \Vdash \phi(\dot{x}))$
  \end{enumerate}
\end{dfn}
We say that $p$ \emph{forces} $\phi$ if $p \Vdash \phi$.

\begin{thm}[The Truth Lemma] \label{thm:truth_lemma}
  Let $G$ be a generic filter on $\mathbb{P}$, $\varphi$ be a formula, and $\dot{x} \in M^{\mathbb{P}}$, then 
  $$M[G] \vDash \varphi(\dot{x}^G) \Leftrightarrow \exists p \in G \, (p \Vdash \varphi(\dot{x}))$$
\end{thm}

\begin{cor}\label{cor:truth_lemma_cor}
  Let $p \in \mathbb{P}$, $\varphi$ be a formula and $\dot{x} \in M^{\mathbb{P}}$, then
  $$p \Vdash \varphi(\dot{x}) \Leftrightarrow \text{for any generic filter } G \text{ containing } p, M[G] \vDash \varphi(\dot{x}^G)$$
\end{cor}


\subsection{Symmetric Extensions}\label{subsection:symmetrc_extensions}
Symmetric extensions are substructures of generic extensions of a given c.t.m.\  of ZF 
and are formed by interpreting only the hereditarily symmetric names.
Let $M$ be a c.t.m.\  of ZF, $(\mathbb{P}, \leq_{\mathbb{P}})$ be a pre-ordered set in $M$ with the maximum element $1_{\mathbb{P}}$.

\begin{dfn} \label{dfn:automorphism}
  We say that $\pi : \mathbb{P} \rightarrow \mathbb{P}$ is an \emph{automorphism} if for all $p, q \in \mathbb{P}$, $p \leq_{\mathbb{P}} q \Leftrightarrow \pi p \leq_{\mathbb{P}} \pi q$.
  $\pi$ induces an bijection on $\mathbb{P}$-names defined recursively as follows:
  $$ \pi \dot{x} = \{ \langle \pi \dot{y}, \pi p \rangle \mid \langle \dot{y}, p \rangle \in \dot{x} \} $$
\end{dfn}

\begin{dfn} 
  Let $\mathcal{G}$ be a group of automorphisms of $\mathbb{P}$. We say that $\mathcal{F}$ is a \emph{normal filter} on $\mathcal{G}$ if the following conditions hold:
  \begin{enumerate}
    \item $\mathcal{F}$ is non-empty family of subgroups of $\mathcal{G}$.
    \item $\mathcal{F}$ is closed under finite intersections and supergroups.
    \item For every $H \in \mathcal{F}$ and $\pi \in \mathcal{G}$, $\pi H \pi^{-1} \in \mathcal{F}$.
  \end{enumerate}
\end{dfn}

We fix a group of automorphisms $\mathcal{G}$ of $\mathbb{P}$ and a normal filter $\mathcal{F}$ on $\mathcal{G}$.
\begin{dfn} 
  For every $\mathbb{P}$-name $\dot{x}$, let $\mathrm{sym}(\dot{x}) = \{ \pi \in \mathcal{G} \mid \pi \dot{x} = \dot{x} \}$.
  We say that $\mathbb{P}$-name $\dot{x}$ is \emph{symmetric} if $\mathrm{sym}(\dot{x}) \in \mathcal{F}$.
\end{dfn}

\begin{dfn} \label{dfn:hereditarily_symmetric}
  We define the set of all hereditarily symmetric names $\mathrm{HS}$ recursively as follows:
  \begin{enumerate}
    \item $0 \in \mathrm{HS}$
    \item if $\mathrm{dom}(\dot{x}) \subseteq \mathrm{HS}$ and $\dot{x}$ is symmetric, then $\dot{x} \in \mathrm{HS}$
  \end{enumerate}
\end{dfn}

\begin{dfn} 
  Let $G$ be a generic filter on $\mathbb{P}$. The set $\mathrm{HS}^{G} = \{ \dot{x}^G \mid \dot{x} \in \mathrm{HS} \}$ is called a \emph{symmetric extension} of $M$.
\end{dfn}

\begin{thm}
  Let $G$ be a generic filter on $\mathbb{P}$. Then, the symmetric extension $\mathrm{HS}^{G}$ is a c.t.m.\  of ZF and a substructure of $M[G]$.
\end{thm}

\begin{dfn}
  The relativized forcing relation $\Vdash_{\mathrm{HS}}$ is defined as the forcing relation $\Vdash$ with $M^{\mathbb{P}}$ in its definition replaced by $\mathrm{HS}$.
\end{dfn}

The relation $\Vdash_{\mathrm{HS}}$ acts as the forcing relation for symmetric extensions.

\begin{thm}\label{thm:symmetry_truth_lemma}
  Let $G$ be a generic filter on $\mathbb{P}$, $\mathcal{N}$ be a symmetric extension generated by $G$, $\varphi$ be a formula, and $\dot{x} \in \mathrm{HS}$, then
  $$\mathcal{N} \vDash \varphi(\dot{x}) \Leftrightarrow \exists p \in G \, (p \Vdash_{\mathrm{HS}} \varphi(\dot{x}))$$
\end{thm}

\begin{cor}\label{cor:symmetry_truth_lemma_cor}
  Let $p \in \mathbb{P}$, $\varphi$ be a formula, and $\dot{x} \in \mathrm{HS}$, then
  $$p \Vdash_{\mathrm{HS}} \varphi(\dot{x}) \Leftrightarrow \text{for any generic filter } G \text{ containing } p, \mathcal{N} \vDash \varphi(\dot{x}^G)$$
  Where $\mathcal{N}$ is the symmetric extension generated by $G$.
\end{cor}

The following lemmas also holds for the forcing relation $\Vdash$, but we only state them for $\Vdash_{\mathrm{HS}}$.

\begin{lem}\label{lem:strengthening}
  Let $p, q \in \mathbb{P}$, $\varphi$ be a formula. If $q \leq_{\mathbb{P}} p$ and $p \Vdash_{\mathrm{HS}} \varphi$, then $q \Vdash_{\mathrm{HS}} \varphi$.
\end{lem}

\begin{lem}[The Symmetry Lemma]\label{lem:symmetry_lemma}
  Let $\pi$ be an automorphism of $\mathbb{P}$, $\dot{x}_0, \ldots, \dot{x}_n \in \mathrm{HS}$, and $\varphi$ be a formula, then
  $$p \Vdash_{\mathrm{HS}} \varphi(\dot{x}_0, \ldots, \dot{x}_n) \Leftrightarrow \pi p \Vdash_{\mathrm{HS}} \varphi(\pi \dot{x}_0, \ldots, \pi \dot{x}_n)$$
\end{lem}


\subsection{The c.t.m.\  Approach}\label{subsection:ctm}
In this section, we explain the validity of the c.t.m.\  approach, 
which is a method for proving the relative consistency and independence in axiomatic set theory.

The relative consistency of $\neg$AC with ZF means that if ZF is consistent, then ZF+$\neg$AC is also consistent.
When ZF is consistent, we can prove, by the completeness theorem, that ZF has a model.
However, the existence of a c.t.m.\  of ZF cannot be derived from the consistency of ZF.
The statement that ZF has a c.t.m.\  is stronger than the consistency of ZF.
Nonetheless, by assuming the existence of a c.t.m.\  of ZF and then constructing a model of ZF+$\neg$AC by forcing, 
we can prove the relative consistency of $\neg$AC with ZF.
To explain this, we introduce some concepts,
following Jech \cite{jech_set_theory}, chapter 6 and 12.

\begin{prop}[Reflection Principle]
Let $\varphi(x_1, \ldots, x_n)$ be a formula. There is a transitive set $M$ such that 
$$\varphi^M(x_1, \ldots, x_n) \leftrightarrow \varphi(x_1, \ldots, x_n)$$
for every $x_1, \ldots, x_n \in M$.
\end{prop}
Where $\varphi^M$ denotes the relativization of $\varphi$ to $M$.
In other words, $\varphi^M$ is the formula obtained 
by replacing every quantifier $\forall x$ by $\forall x \in M$ and $\exists x$ by $\exists x \in M$.
Note that this is a theorem schema for every formula $\varphi$.

\begin{dfn}
  Let $E$ be a binary relation on a set $P$.
  \begin{itemize}
    \item 
  We say thet $E$ is \emph{well-founded} if every non-empty subset of $P$ has an $E$-minimal element.
  \item
  We say that $E$ is extensional if:
  $$\mathrm{ext}_E(x) \ne \mathrm{ext}_E(y)$$
  whenever $x$ and $y$ are distinct elements of $P$, 
  where 
  $$\mathrm{ext}_E(x) = \{ z \in P \mid z\ E\ x \}$$ 
\end{itemize}
\end{dfn}

\begin{thm}[Mostowski's Collapsing Theorem]
  If $E$ is a well-founded and extensional relation on a set $P$, 
  then there is a transitive set $M$ and an isomorphism $\pi$ between $(P, E)$ and $(M, \in)$.
\end{thm}

\begin{thm}[L\"owenheim-Skolem Theorem]\label{thm:lowenheim_skolem}
  Every infinite model for a countable language has a countable elementary submodel.
\end{thm}

Using these, we can prove the following proposition:
\begin{prop}\label{prop:ctm_finite}
  Let $\Gamma$ be a finite fragment of ZF axioms, including the Axiom of Extensionality,
  then there exists a c.t.m.\ of $\Gamma$.
\end{prop}
\begin{proof}
  Let $\Gamma = \{ \varphi_1, \ldots, \varphi_n \}$ and $\psi = \varphi_1 \land \ldots \land \varphi_n$.
  By reflection principle, there exists a set $M$ such that $\psi^M \leftrightarrow \psi$.
  Then, $\mathcal{M} = (M, \in \mid_{M \times M})$ satisfies $\psi$. 
  In other words, $\mathcal{M}$ is a model of $\Gamma$.
  By \cref{thm:lowenheim_skolem}, there exists a countable elementary submodel $\mathcal{N} = (N, \in \mid_{N \times N})$ of $\mathcal{M}$.  
  $\mathcal{N}$ satisfies the Axiom of Extensionality since $\Gamma$ includes it.
  Additionary, by the Axiom of Regularity, every non-empty subset of $N$ has an $\in$-minimal element.
  Therefore, $\in \mid_{N \times N}$ is well-founded and extensional.
  By Mostowski's Collapsing Theorem, there exists a transitive set $N_0$ and an isomorphism between $(N, \in \mid_{N \times N})$ and $(N_0, \in)$.
  Thus, $N_0$ is a c.t.m.\ of $\Gamma$.
\end{proof}

Here, we use this to explain the validity of the c.t.m.\ approach.
Note that this discussion is not limited to the relative consistency proof of $\neg$AC with ZF;
a similar argument can be applied to other relative consistency proofs using the c.t.m.\ approach.

First, let us assume the existence of a c.t.m.\  of ZF and 
suppose that we can prove the existence of a c.t.m.\ of ZF+$\neg$AC by forcing.
It should also be possible to prove the following by modifying the proof.

\begin{prop}\label{prop:ctm_finite2}
  Let $\Gamma$ be a finite fragment of ZF, 
  there exists a finite fragment $\Delta$ of ZF+$\neg$AC such that
  if a c.t.m.\  of $\Gamma$ exists, then a c.t.m.\  of $\Delta$ also exists.
\end{prop}

This is because, in the original proof, 
the axiom of ZF used to formalize forcing are finite, 
and the proof that a c.t.m.\ of ZF+$\neg$AC satisfies each axiom of ZF+$\neg$AC should 
only require a finite number of axioms of ZF. 

By combining \cref{prop:ctm_finite,prop:ctm_finite2}, 
we can prove the following\footnote{
  Since the Reflection Principle is a theorem schema,
  this is also a theorem schema for every $\Delta$.
}.
\begin{prop} \label{prop:ctm_finite3}
  Let $\Delta$ be a finite fragment of ZF+$\neg$AC,
  there exists a c.t.m.\  of $\Delta$. 
  In other words, $\Delta$ is consistent.
\end{prop}

Here, let us prove the relative consistency of ZF+$\neg$AC by contradiction.
Assume that ZF+$\neg$AC is inconsistent.  
Then, since ZF+$\neg$AC proves $\bot$ (a contradiction),  
there exists a finite fragment $\Delta$ of ZF+$\neg$AC such that $\Delta$ proves $\bot$.  
In other words, $\Delta$ is inconsistent.  
However, by \cref{prop:ctm_finite3}, $\Delta$ is consistent, which leads to a contradiction.
Therefore, ZF+$\neg$AC is consistent.


\section{Outline for the Informal Proof}\label{sec:outline}
We outline an informal proof of the relative consistency of $\neg$AC with ZF which we will formalize in the next chapter.
In this proof, the relative consistency is proved by assuming the existence of a c.t.m.\  of ZF and constructing a model of ZF+$\neg$AC by forcing. 
This model is a symmetric extension called the basic Cohen model.

Let $M$ be a c.t.m.\  of ZF, $\mathbb{P}$ be the set of all finite partial functions from $\omega \times \omega$ to $2 = \{0, 1\}$
that are elements of $M$, and $\leq_{\mathbb{P}}$ be $\supseteq$.
Note that the maximum element $1_{\mathbb{P}}$ is the empty set. 
Let $\pi$ be a bijection on $\omega$. $\pi$ induces an automorphism on $\mathbb{P}$ defined as follows:
$$\mathrm{dom}(\pi p) = \{ (\pi n, m) \mid (n, m) \in \mathrm{dom}(p) \}$$
$$(\pi p)(\pi n, m) = p(n, m)$$
This automorphism further induces an automorphism on $\mathbb{P}$-names. 
Let $\mathcal{G}$ be the group of all such automorphisms.
For every finite $e \subseteq \omega$, let 
$$\mathrm{fix}(e) = \{ \pi \in \mathcal{G} \mid \forall n \in e (\pi n = n) \}$$
Let $\mathcal{F}$ be the set of all subgroups $H$ of $\mathcal{G}$ such that there exists a finite $e \subseteq \omega$ with $\mathrm{fix}(e) \subseteq H$.
Note that $\mathcal{F}$ is a normal filter on $\mathcal{G}$. Let $\mathcal{N} = \mathrm{HS}^{G}$.
Since $\mathcal{N}$ is a symmetric extension of $M$, it is a c.t.m.\  of ZF. 

\begin{thm}
  $\mathcal{N}$ does not satisfy the well-ordering theorem.
\end{thm}

\begin{proof}
We outline the proof of this theorem as follows. 
For every $n \in \omega$, let $a_n$ be the following real number (a subset of $\omega$):
$$a_n = \{ m \in \omega \mid \exists p \in G (p(n, m) = 1) \}$$
Since $a_n$ are pairwise distinct, $A = \{ a_n \mid n \in \omega \}$ is an infinity set.
$A$ and every $a_n$ are elements of $\mathcal{N}$.
$A$ serves as a counterexample to the well-ordering theorem in $\mathcal{N}$.
Suppose for contradiction that $A$ is well-ordered in $\mathcal{N}$, there exists a injection $f$ from $\omega$ to $A$ in $\mathcal{N}$.
Let $\varphi(g,x,y)$ be a formula that represents the relation $g(x) = y$. 
For every $n \in \omega$ with $a_n \in \mathrm{ran}(f)$, there exists $i \in \omega$ such that $N \vDash \varphi(f, i, a_n)$.
Thus there exists $p \in G$ and hereditarily symmetric names $\dot{f}, \dot{i}$ and $\dot{a_n}$ for each of $f, i, a_n$ such that 
$$p \Vdash_{\mathrm{HS}} \varphi(\dot{f}, \dot{i}, \dot{a_n})$$
By choosing $n$ and the names appropriately, we can find a bijection $\pi$ on $\omega$ such that the following conditions are additionally satisfied:
\begin{enumerate}
  \item $\pi \dot{f} = \dot{f}$
  \item $\pi \dot{i} = \dot{i}$
  \item $\pi n \ne n$
  \item There exists a hereditarily symmetric name $\dot{a_{\pi n}}$ of $a_{\pi n}$ such that $\pi \dot{a_n} = \dot{a_{\pi n}}$
  \item There exists $q \in G$ such that $q \leq_{\mathbb{P}} p$ and $q \leq_{\mathbb{P}} \pi p$
\end{enumerate}
Note that some occurence of $\pi$ in above conditions refer to the induced automorphism on $\mathbb{P}$ or $\mathbb{P}$-names..
By \cref{lem:symmetry_lemma}
$$\pi p \Vdash_{\mathrm{HS}} \varphi(\pi \dot{f}, \pi \dot{i}, \pi \dot{a_n})$$
Thus 
$$\pi p \Vdash_{\mathrm{HS}} \varphi(\dot{f}, \dot{i}, \dot{a_{\pi n}})$$
Therefore, by \cref{lem:strengthening}
$$q \Vdash_{\mathrm{HS}} \varphi(\dot{f}, \dot{i}, \dot{a_n}) \mathrm{ and } q \Vdash_{\mathrm{HS}} \varphi(\dot{f}, \dot{i}, \dot{a_{\pi n}})$$
This means that $\mathcal{N} \vDash \varphi(f, i, a_n)$ and $\mathcal{N} \vDash \varphi(f, i, a_{\pi n})$, which implies that $f(i) = a_n$ and $f(i) = a_{\pi n}$.
Since $a_n$ and $a_{\pi n}$ are distinct, this is a contradiction. 
\end{proof}
 


\section{Isabelle/ZF and Formalization in Prior Work}
In this section, we introduce Isabelle/ZF, a proof assistant for ZF set theory, 
and important results from prior work used in the formalization of this study.


\subsection{Isabelle/ZF}
Isabelle \cite{paulson_1986} is a proof assistant developed by Paulson, used for formalizing mathematical proofs and verifying programs. 
Since its release in 1986, it has continued to be developed by many researchers.

One notable example of Isabelle's achievements is the formal verification of the seL4 microkernel \cite{seL4}.
This was a massive project involving over a million lines of Isabelle code. 
More recently, in The ALEXANDRIA Project \cite{alexiandria}, various advanced mathematical results have been formalized in Isabelle.

Isabelle is a tool for proving theorems of the loical system called Pure. 
On top of Pure, other logical systems such as first-order logic (Isabelle/FOL) and Higher-Order Logic (Isabelle/HOL) are built,
and Isabelle allows proving theorems in these logical systems.

Isabelle/ZF is an extension of Isabelle/FOL.
It extends Isabelle/FOL by adding the axioms of ZF, allowing theorems of ZF set theory to be proved.
Additionally, the Axiom of Choice (AC) can be included, enabling theorems of ZFC to be proved in Isabelle/ZF.
Many definitions and theorems regarding ZF, such as ordinals, cardinals, functions, and ordered sets, 
have been formalized and provided as a library.


\subsection{Important Results from Prior Work}


\subsubsection{Recursive Data Types and Functions}
Paulson \cite{paulson_datatype_impl} introduced \texttt{"datatype"} to recursive data types within ZF set theory, 
and "primrec" to define functions on top of these data types. 
This made it possible to define data structures such as lists within ZF set theory.
For example, the set of all lists of elements from a set $A$ is defined in a library of Isabelle/ZF as follows:
\begin{isaframe}
\begin{minted}[breaklines]{isabelle}
consts
    list       :: "i⇒i"

datatype
    "list(A)" = Nil | Cons ("a ∈ A", "l ∈ list(A)")
\end{minted}
\end{isaframe}
Additionally, the function "length" on lists is defined using "primrec" as follows.
\begin{isaframe}
\begin{minted}[breaklines]{isabelle}
consts
    length :: "i⇒i"

primrec
    "length([]) = 0"
    "length(Cons(a,l)) = succ(length(l))"
\end{minted}
\end{isaframe}


\subsubsection{Class and Relativization}
In Isabelle/ZF, a class $C$ is represented as a unary predicate $C(x)$.
The statement $x \in C$ is represented as $C(x)$.
For example, the class of all ordinals $\mathrm{Ord}$ is defined in a library of Isabelle/ZF as \texttt{"Ord"},
and the fact that \texttt{"α"} is an ordinal is represented as \texttt{"Ord(α)"}.

Additionally, the notation \texttt{"∀x[C]. P(x)"} represents that \texttt{"P(x)"} holds for all \texttt{"x"} in the class \texttt{"C"},
and \texttt{"∃x[C]. P(x)"} represents that there exists an \texttt{"x"} in the class \texttt{"C"} such that \texttt{"P(x)"} holds.
Using this notation, we can obtain the relativized statement to a class \texttt{"C"} 
by replacing all \texttt{"∀x"} and \texttt{"∃x"} in a original statement with \texttt{"∀x[C]"} and \texttt{"∃x[C]"}, respectively.


\subsubsection{Internalized Formula and Satisfaction Relation}
Paluson defined formulas encoded as sets in ZF set theory in their work \cite{paulson_AC_consistency},
using \texttt{"datatype"} as follows:
\begin{isaframe}
\begin{minted}[breaklines]{isabelle}
consts   formula :: i
datatype
    "formula" = Member ("x ∈ nat", "y ∈ nat")
                      | Equal  ("x ∈ nat", "y ∈ nat")
                      | Nand ("p ∈ formula", "q ∈ formula")
                      | Forall ("p ∈ formula")
\end{minted}
\end{isaframe}
Note that \texttt{"nat"} is the set of natural numbers.
The unique logical connective in this definition is NAND and 
other logical connectives such as AND, OR, and NOT on formulas are defined in terms of NAND.
Formulas are represented using the de Brujin index, in other words, 
a occurence of a natural number $n$ in a formula represents the $(n - m)$-th free variable when this $n$ occurs under $m$ quantifiers.

Additionally, the satisfaction relation \texttt{"sats(A, φ, env)"} is defined.
This relation represents that \texttt{"A"} satisfies the formula \texttt{"φ"} 
under the environment \texttt{"env"}, where \texttt{"env"} is a list of elements of \texttt{"A"}.
Note that the arity of \texttt{"φ"} must be less than or equal to the length of \texttt{"env"}.

For example, \texttt{"sats(A, Member(0, 1), [a, b])"} is equivalent to \texttt{"a ∈ b"} when \texttt{"[a, b] ∈ list(A)"}, 
and \texttt{"sats(A, Forall(Member(0, 1)), [c])"} is equivalent to \texttt{"∀x ∈ A. c ∈ x"} when \texttt{"[c] ∈ list(A)"}.


\subsubsection{Locales and Internalized ZF Axioms}
Isabelle provides a feature called locales, 
which allows us to work more easily in a specific context by fixing constants and introducing assumptions.
For example, when formalizing group theory, rather than adding the assumption "$G$ is a group" to every theorem,
we can simply declare a locale with the constant $G$ and the assumption "$G$ is a group".
 
Paulson and Gunther et al. declared locales to handle internal models of ZF set theory in their work \cite{paulson_AC_consistency, gunther_forcing}, respectively.
For example, the locale \texttt{"M\_ctm"} by Gunther et al. fixes \texttt{"M"} and \texttt{"enum"},
and introduces assumptions that \texttt{"M"} holds the axioms of ZF set theory, 
\texttt{"M"} is a transitive set, and \texttt{"enum"} is a enumeration of \texttt{"M"}.
Therefore, we can work in the context of a c.t.m.\  of ZF by working within this locale.


\subsubsection{Forcing}
Gunther et al. formalized forcing method in ZF set theory \cite{gunther_forcing}.
Their formalization follows Kunen's book \cite{kunen2011}, which uses the c.t.m.\  approach.

They defined the notion of forcing as the locale \texttt{"forcing\_notion"} 
with the constants \texttt{"P"}, \texttt{"leq"}, and \texttt{"one"}.
\begin{isaframe}
\begin{minted}[breaklines]{isabelle}
locale forcing_notion =
    fixes P leq one
    assumes one_in_P:     "one ∈ P"
      and leq_preord:       "preorder_on(P,leq)"
      and one_max:          "∀p∈P. ⟨p,one⟩∈leq"
\end{minted}
\end{isaframe}

They also declared the locale \texttt{"forcing\_data"} 
by combining the locale \texttt{"forcing\_notion"} with the locale \texttt{"M\_ctm"} and assume some additional properties. 
\begin{isaframe}
\begin{minted}[breaklines]{isabelle}
locale forcing_data = forcing_notion + M_ctm +
    assumes P_in_M:         "P ∈ M"
      and leq_in_M:         "leq ∈ M"
\end{minted}
\end{isaframe}

Within this locale, The predicate \texttt{"p ⊩ φ env"} is defined.
This represents that \texttt{"p"} forces the formula \texttt{"φ"} under the environment \texttt{"env"}.
Additionally, the following
\footnote{
In \cref{dfn:forcing_relation},
we explained that the forcing relation is defined for the formulas in the forcing language.
According to that definition, \texttt{"env"} should be a list of elements of $M^{\mathbb{P}}$;
however, in the following lemmas, \texttt{"env"} is a list of elements of $M$.
In fact, the definition of the forcing relation by Gunther et al. differs from \cref{dfn:forcing_relation}.
They defined the forcing relation not for the formulas in the forcing language 
but for the formulas in the extended language of set theory by adding all elements of $M$ as constants.
Additionally, only NAND is used as the logical connective, and only $\forall$ is used as the quantifier of the formulas.
} is their formalization of \cref{thm:truth_lemma} and \cref{cor:truth_lemma_cor}.
\begin{isaframe}
\begin{minted}[breaklines]{isabelle}
lemma truth_lemma:
    assumes 
        "φ∈formula" "M_generic(G)"
    shows 
        "⋀env. env∈list(M) ⟹ arity(φ)≤length(env) ⟹ 
          (∃p∈G. p ⊩ φ env)   ⟷   M[G], map(val(G),env) ⊨ φ"

lemma definition_of_forcing:
    assumes
        "p∈P" "φ∈formula" "env∈list(M)" "arity(φ)≤length(env)"
    shows
        "(p ⊩ φ env) ⟷
        (∀G. M_generic(G) ∧ p∈G  ⟶  M[G], map(val(G),env) ⊨ φ)"
\end{minted}
\end{isaframe}
Where \texttt{"M\_generic(G)"} represents that \texttt{"G"} is a generic filter on \texttt{"P"} in \texttt{"M"},
\texttt{"val(G)"} is the interpretation by \texttt{"G"},
and \texttt{"map"} is the map function on lists.




\chapter{Formalization of the Proof} 
In this chapter, we present the formalization of the relative consistency proof of $\neg$AC with ZF in Isabelle/ZF.
We follow the outline presented in \cref{sec:outline}.
We choose \emph{ZF-Constructible} by Paulson \cite{paulson_AC_consistency} as Isabelle theory 
\footnote{Note that this does not mean assuming $V=L$.}
, which includes the ZF axioms along with useful definitions and lemmas built on top of them.
We also reuse the formalization of forcing by Gunther et al. \cite{gunther_forcing}.



\section{Symmetric Extensions}


\subsection{Defining Symmetric Extensions}
First, we define symmetric extensions in Isabelle/ZF,
formalizing automorphisms, normal filters, and hereditarily symmetric names.


\subsubsection{$\mathbb{P}$-names}
We begin our work within the locale \texttt{"forcing\_data"} by Gunther et al. \cite{gunther_forcing}, 
in which a c.t.m.\  \texttt{"M"} and a notion of forcing \texttt{"P"} are fixed.
The relation $\leq_{\mathbb{P}}$ is denoted by \texttt{"leq"} and 
a maximum element $1_{\mathbb{P}}$ is denoted by \texttt{"one"}.

Gunther et al. did not provide an excplicit formalization of the set of $\mathbb{P}$-names $M^{\mathbb{P}}$ in their work.
In this study, since we need to consider a subset of $M^{\mathbb{P}}$, we define $M^{\mathbb{P}}$ explicitly.
We define $M^{\mathbb{P}}_{\alpha}$ as \texttt{"P\_set(α)"} and $M^{\mathbb{P}}$ as \texttt{"P\_names"} as follows:

\begin{isaframe}
\begin{minted}[breaklines]{isabelle}
definition HP_set_succ :: "[i, i] ⇒ i" where 
    "HP_set_succ(a, X) ≡ Pow(X × P) ∩ M" 

definition P_set :: "i ⇒ i" where 
    "P_set(a) ≡ transrec2(a, 0, HP_set_succ)" 

definition P_names :: "i" where "P_names ≡ { x ∈ M . ∃ a. Ord(a) ∧ x ∈ P_set(a) }" 
\end{minted}
\end{isaframe} 

Where \texttt{"transrec2(α, x, H)"} is the function defined in \emph{ZF} library
that returns \texttt{x} when \texttt{α} is 0;
when \texttt{α} is a successor ordinal, it returns \texttt{H(β, transrec2(β, x, H))} where \texttt{β} is the predecessor of \texttt{α};
and when \texttt{α} is a limit ordinal, it returns $\bigcup$\texttt{β∈α. transrec2(β, x, H)}.
Note that in Isabelle/ZF, the notation \texttt{"\{ x ∈ A. P(x) \}"} denotes
the set of elements in \texttt{A} that satisfy the predicate \texttt{P}.


\subsubsection{Automorphisms}
To simplify the formalization, we declare and work within the locale \texttt{"forcing\_data\_partial"}, 
where $\mathbb{P}$ is assumed to be a partially ordered set. 
Note that \texttt{"one"} is the unique maximum element of $\mathbb{P}$ in this locale.

\begin{isaframe}
\begin{minted}[breaklines]{isabelle}
locale forcing_data_partial = forcing_data + 
    assumes leq_relation_on_P : "leq ∈ Pow(P × P)" 
    and leq_partial_order : "partial_order_on(P, leq)" 
\end{minted}
\end{isaframe}

We define the set of automorphisms on $\mathbb{P}$, denoted by \texttt{"P\_auto"}, as follows:
\begin{isaframe}
\begin{minted}[breaklines]{isabelle}
definition is_P_auto :: "i ⇒ o" where
    "is_P_auto(π) ≡ π ∈ M ∧ π ∈ bij(P, P) ∧ (∀ p ∈ P. ∀q ∈ P. p ≼ q ⟷ π`p ≼ π`q)"  

definition P_auto where "P_auto ≡ { π ∈ P → P. is_P_auto(π) }" 
\end{minted}
\end{isaframe}

Where \texttt{"bij(A, B)"} is the set of bijections from \texttt{A} to \texttt{B}
and \texttt{"f`x"} denotes the function application of \texttt{f} to \texttt{x}.
We denote the induced automorphism on $\mathbb{P}$-names by \texttt{π} as \texttt{"Pn\_auto(π)"}.
Our definition of \texttt{"Pn\_auto"} satisfies the equality given in \cref{dfn:automorphism}.

\begin{isaframe}
\begin{minted}[breaklines]{isabelle}
lemma Pn_auto : 
    "x ∈ P_names ⟹ Pn_auto(π)`x = { <Pn_auto(π)`y, π`p> . <y, p> ∈ x}"  
\end{minted}
\end{isaframe}


\subsubsection{Groups of Automorphisms and Normal Filters}
First, we define the set of all subgroups of a group of automorphisms \texttt{G} that are in \texttt{"M"}.
\begin{isaframe}
\begin{minted}[breaklines]{isabelle}
  definition is_P_auto_group where 
      "is_P_auto_group(G) ≡ 
        G ⊆ { π ∈ P → P. is_P_auto(π) } 
      ∧ (∀π ∈ G. ∀τ ∈ G. π O τ ∈ G) 
      ∧ (∀π ∈ G. converse(π) ∈ G)"   
  
  definition P_auto_subgroups where 
      "P_auto_subgroups(G) ≡ { H ∈ Pow(G) ∩ M. is_P_auto_group(H) }" 
\end{minted}
\end{isaframe}
Where \texttt{"π O τ"} denotes the composition of \texttt{π} and \texttt{τ}.
Then, we declare a locale \texttt{"M\_symmetric\_system"} that fixes a group of automorphisms \texttt{"𝒢"} and a normal filter \texttt{"ℱ"} on \texttt{"𝒢"}.
\begin{isaframe}
\begin{minted}[breaklines]{isabelle}
locale M_symmetric_system = forcing_data_partial + 
    fixes 𝒢 ℱ 
    assumes 𝒢_in_M : "𝒢 ∈ M"  
    and 𝒢_P_auto_group : "is_P_auto_group(𝒢)"   
    and ℱ_in_M : "ℱ ∈ M"
    and ℱ_subset : "ℱ ⊆ P_auto_subgroups(𝒢)" 
    and ℱ_nonempty : "ℱ ≠ 0" 
    and ℱ_closed_under_intersection : "∀A ∈ ℱ. ∀B ∈ ℱ. A ∩ B ∈ ℱ" 
    and ℱ_closed_under_supergroup : 
        "∀A ∈ ℱ. ∀B ∈ P_auto_subgroups(𝒢). A ⊆ B ⟶ B ∈ ℱ" 
    and ℱ_normal : "∀H ∈ ℱ. ∀π ∈ 𝒢. { π O τ O converse(π). τ ∈ H } ∈ ℱ" 
\end{minted}
\end{isaframe}


\subsubsection{Hereditarily Symmetric Names}
We continue our work within the locale \texttt{"M\_symmetric\_system"}.
We define the set of hereditarily symmetric names \texttt{"HS"} in the same way as \texttt{"P\_names"}.
This approach makes it easier to use induction based on the $\mathbb{P}$-rank of names.

\begin{isaframe}
\begin{minted}[breaklines]{isabelle}
definition sym where "sym(x) ≡ { π ∈ 𝒢. Pn_auto(π)`x = x }"  

definition symmetric where "symmetric(x) ≡ sym(x) ∈ ℱ"  

definition HHS_set_succ where 
  "HHS_set_succ(a, X) ≡ { x ∈ P_set(succ(a)). domain(x) ⊆ X ∧ symmetric(x) }" 

definition HS_set where "HS_set(a) ≡ transrec2(a, 0, HHS_set_succ)"

definition HS where "HS ≡ { x ∈ P_names. ∃a. Ord(a) ∧ x ∈ HS_set(a) }" 
\end{minted}
\end{isaframe}

The following lemma shows our definition of \texttt{"HS"} is equivalent to \cref{dfn:hereditarily_symmetric}.

\begin{isaframe}
\begin{minted}[breaklines]{isabelle}
lemma HS_iff: "x ∈ HS ⟷ x ∈ P_names ∧ domain(x) ⊆ HS ∧ symmetric(x)" 
\end{minted}
\end{isaframe}


\subsubsection{Symmetric Extensions}
Finally, we define symmetric extensions.

\begin{isaframe}
\begin{minted}[breaklines]{isabelle}
definition SymExt where "SymExt(G) ≡ { val(G, x). x ∈ HS }" 
\end{minted}
\end{isaframe}

Where \texttt{"val(G, x)"} is the interpretation $x^G$ formalized by Gunther et al. \cite{gunther_forcing}.

We declare a locale \texttt{"M\_symmetric\_system\_G\_generic"},
which combines the locale \texttt{"M\_symmetric\_system"} and the locale \texttt{"G\_generic"}.
\texttt{"G\_generic"} is a locale, defined by Gunther et al. \cite{gunther_forcing}, fixes a generic filter \texttt{"G"} on \texttt{"P"}.

\begin{isaframe}
\begin{minted}[breaklines]{isabelle}
locale M_symmetric_system_G_generic = M_symmetric_system + G_generic
\end{minted}
\end{isaframe}

The following lemmas can be proved relatively easily in this locale.

\begin{isaframe}
\begin{minted}[breaklines]{isabelle}
lemma M_subset_SymExt : "M ⊆ SymExt(G)" 
lemma SymExt_subset_GenExt : "SymExt(G) ⊆ GenExt(G)" 
lemma Transset_SymExt : "Transset(SymExt(G))" 
lemma SymExt_countable : "nat ≈ SymExt(G)" 
\end{minted}
\end{isaframe}

Where \texttt{"GenExt(G)"} is the generic extension generated by \texttt{"G"}, 
\texttt{"Transset(A)"} denotes that \texttt{"A"} is transitive, and
\texttt{"nat ≈ A"} denotes that \texttt{"A"} has the same cardinality as the set of natural numbers.


\subsubsection{Formulas Representing These Definitions}
In our proofs, we need to use the separation and replacement axioms in M
based on formulas that include the concepts defined in this section.
Therefore, it is necessary to represent these concepts as internalized formulas.

We utilized the automated formula synethisis method by Gunther et al. \cite{gunther_forcing};
however, it could not applied to complex formulas, 
and re-verifying the properties of the derived formulas in a more usable form remained a repititive task.

Example of the formulas we defined includes \texttt{"is\_P\_auto\_group"}, \texttt{"is\_P\_name\_fm"}, \texttt{"is\_Pn\_auto\_fm"}, and \texttt{"is\_HS\_fm"},
whick represent \texttt{"P\_names"}, \texttt{"Pn\_auto"} and \texttt{"HS"} respectively.
The correctness of these formulas is verified by proving the following lemmas. 

\begin{isaframe}
\begin{minted}[breaklines]{isabelle}
lemma sats_is_P_name_fm_iff : 
    fixes env i j x 
    assumes "env ∈ list(M)" "i < length(env)" "j < length(env)" 
                    "nth(i, env) = P" "nth(j, env) = x"  
    shows "sats(M, is_P_name_fm(i, j), env) ⟷ x ∈ P_names" 

lemma sats_is_Pn_auto_fm_iff :
    fixes x π v env i j k l
    assumes "i < length(env)" "j < length(env)" "k < length(env)" "l < length(env)" 
                    "nth(i, env) = P" "nth(j, env) = π" "nth(k, env) = x" "nth(l, env) = v" 
                    "env ∈ list(M)" "π ∈ P_auto" 
    shows "sats(M, is_Pn_auto_fm(i, j, k, l), env) ⟷ x ∈ P_names ∧ v = Pn_auto(π)`x" 

lemma sats_is_HS_fm_iff : 
    fixes x i j env 
    assumes "env ∈ list(M)" "i < length(env)" "j < length(env)" 
                    "nth(i, env) = <ℱ, 𝒢, P, P_auto>" "nth(j, env) = x" 
    shows "sats(M, is_HS_fm(i, j), env) ⟷ x ∈ HS" 
\end{minted}
\end{isaframe}

Note that these formulas take as parameters various elements fixed within the locale, 
such as the notion of forcing \texttt{"P"}, the normal filter \texttt{"ℱ"}, etc.


\subsection{Proving Symmetric Extensions are Models of ZF}

In this section, we prove that symmetric extensions are models of ZF.
Initially, we attempted to prove this based on its proof in Karagila's lecture note \cite{karagila},
but this approach was unsuccessful.
Instead, we prove this by using the relativized forcing relation $\Vdash_{\mathrm{HS}}$.
Karagila's approach is applicable to a more general situation,
without using forcing or generic and symmetric extensions,
whereas our approach is more specific to the context of symmetric extensions.


\subsubsection{Unsuccessful Approach}

In Karagila's lecture note \cite{karagila},
the fact that symmetric extensions are models of ZF is proven by showing that 
they are internal models of the conrresponding generic extension
generated by the same notion of forcing and generic filter,
using the following proposition:

\begin{prop}
If $\mathcal{N}$ is a transitive class that is almost universal and satisfies $\Delta_0$-separation, 
then $\mathcal{N}$ is a inner model of ZF.
\end{prop}

Where a class $\mathcal{N}$ is said to be \emph{almost universal} 
\footnote{Jech \cite{jech_set_theory}, Theorem 13.9.} 
 if every subset $X \subseteq \mathcal{N}$ is included in some $Y \in \mathcal{N}$.
Furthermore, $\mathcal{N}$ is said to satisfy $\Delta_0$-separation if it satisfies the separation axioms for formulas with only bounded quantifiers.

We could prove that the symmetric extension \texttt{"SymExt(G)"} is transitive, almost universal and satisfies $\Delta_0$-separation.
However, we could not prove that \texttt{"SymExt(G)"} is actually a class in \texttt{"GenExt(G)"}.
To do this, we needed to explicitly construct a formula that represents the set of all names \texttt{"P\_names"} on the ground model \texttt{"M"} in \texttt{"GenExt(G)"}.
However, we were unable to construct such a formula.


\subsubsection{The Relativized Forcing Relation $\Vdash_{\mathrm{HS}}$}
Instead, we formalize the relativized forcing relation $\Vdash_{\mathrm{HS}}$ and prove that symmetric extensions are models of ZF using it.
By modifying the formalization of forcing by Gunther et al. \cite{gunther_forcing}, 
we were able to define this relation and prove its properties relatively easily.

Within the locale \texttt{"M\_symmetric\_system"},
we define this relation in the form \texttt{"p ⊩HS φ env"},
which has the same form as the formalization of the original forcing relation \texttt{"p ⊩ φ env"} by Gunther et al.
Our definition satisfies the required properties, such as \cref{thm:symmetry_truth_lemma}, \cref{cor:symmetry_truth_lemma_cor}, and \cref{lem:strengthening}.

\begin{isaframe}
\begin{minted}[breaklines]{isabelle}
lemma HS_truth_lemma:
    assumes 
        "φ∈formula" "M_generic(G)"
    shows 
        "⋀env. env∈list(HS) ⟹ arity(φ)≤length(env) ⟹ 
              (∃p∈G. p ⊩HS φ env) ⟷ SymExt(G), map(val(G),env) ⊨ φ"

lemma definition_of_forcing_HS:
    assumes
        "p∈P" "φ∈formula" "env∈list(HS)" "arity(φ)≤length(env)"
    shows
        "(p ⊩HS φ env) ⟷
              (∀G. M_generic(G) ∧ p∈G  ⟶  SymExt(G), map(val(G),env) ⊨ φ)"

lemma HS_strengthening_lemma:
    assumes 
        "p∈P" "φ∈formula" "r∈P" "r≼p"
    shows
        "⋀env. env∈list(M) ⟹ arity(φ)≤length(env) ⟹ p ⊩HS φ env ⟹ r ⊩HS φ env"
\end{minted}
\end{isaframe}


\subsubsection{The Symmetry Lemma}
Additionally, we prove the symmetry lemma (\cref{lem:symmetry_lemma}).
Similary to the definitions of the forcing relations, this lemma is proven using a complex induction.
First, we prove this lemma for atomic formulas in the forcing language, 
and then extend the proof to other formulas by structual induction on the formula.
For atomic formulas $\dot{x} = \dot{y}$ and $\dot{x} \in \dot{y}$, 
the definition is mutually recursive, so the lemma for these formulas is proven simultaneously
by induction on the $\mathbb{P}$-rank of $\dot{x}$ and $\dot{y}$.
Specifically, we show these induction steps:

\begin{isaframe}
\begin{minted}[breaklines]{isabelle}
  have MEM_step: "⋀a. Ord(a) ⟹ ∀b ∈ a. Q(EQ, b, b) ⟹ ∀b ∈ a. Q(MEM, b, a)"  
  have EQ_step : "⋀a. Ord(a) ⟹ ∀b ∈ a. Q(MEM, b, a) ⟹ Q(EQ, a, a)"   
\end{minted}
\end{isaframe}

Where \texttt{"Q(MEM, α, β)"} and \texttt{"Q(EQ, α, β)"} are predicates 
representing that the symmetriy lemma holds for $\dot{x} \in \dot{y}$ and $\dot{x} = \dot{y}$, respectively,
for any $\dot{x}$ with $\mathbb{P}$-rank at most $\alpha$ and $\dot{y}$ with $\mathbb{P}$-rank at most $\beta$.

The proof for non-atomic formulas is relatively straightforward, and finally, 
this completes the proof of the lemma.

\begin{isaframe}
\begin{minted}[breaklines]{isabelle}
lemma symmetry_lemma:
    fixes φ π  
    assumes "φ ∈ formula" "is_P_auto(π)" "π ∈ 𝒢" 
    shows "⋀env p. env ∈ list(HS) ⟹ arity(φ) ≤ length(env) ⟹ p ∈ P ⟹ 
                p ⊩HS φ env ⟷ π`p ⊩HS φ map(λx. Pn_auto(π)`x, env)" 
\end{minted}
\end{isaframe}


\subsubsection{Separation}
Using the symmetry lemma, we prove the axiom schema of separation for symmetric extensions 
within the locale \texttt{"M\_symmetric\_system\_G\_generic"}.
Our proof is based on Karagila's proof \cite{karagila} of $\Delta_0$-separation for symmetric extensions.
However, since we use the $\Vdash_{\mathrm{HS}}$ relation instead of $\Vdash$,
we can prove separation for any formula, not just $\Delta_0$ formulas.

The separation for the formula $\varphi$ means that 
for any elements $x, p$ in the symmetric extension, the set $X = \{ y \in x \mid \varphi(y, p) \}$ is also an element of it.
To prove this, we show that 
$\dot{X} = \{ \langle y, p \rangle \in \mathrm{dom}(x) \times \mathbb{P} \mid p \Vdash_{\mathrm{HS}} \varphi(\dot{y}, \dot{p}) \}$ is a hereditarily symmetric name of $X$,
i.e., $\dot{X} \in \mathrm{HS}$ and $\dot{X}^G = X$.

\begin{isaframe}
\begin{minted}[breaklines]{isabelle}
lemma sep_forces_pair_in_HS : 
    fixes x env φ
    assumes "x ∈ HS" "env ∈ list(HS)" "φ ∈ formula" "arity(φ) ≤ succ(length(env))" 
    shows "{ <y, p> ∈ domain(x) × P. p ⊩HS φ [y] @ env } ∈ HS"

lemma SymExt_separation : 
    fixes x env φ 
    assumes "x ∈ SymExt(G)" "env ∈ list(SymExt(G))" 
                    "φ ∈ formula" "arity(φ) ≤ succ(length(env))" 
    shows "{ y ∈ x. sats(SymExt(G), φ, [y] @ env) } ∈ SymExt(G)"
\end{minted}
\end{isaframe}


\subsubsection{Replacement}
Now that we have proven separation, to prove the axiom schema of replacement for symmetric extensions,
it is sufficient to show that for any formula $\varphi$ and any elements $x, p$ in a symmetric extension $\mathcal{N}$,
we can find a set $S \in \mathcal{N}$ such that : 
$$\mathcal{N} \vDash \exists z \, \varphi(y, z, p) \text{ iff }
  \mathcal{N} \vDash \exists z \, (z \in S \land \varphi(y, z, p)) \text{ for all } y \in x$$
This is because such a set $S$ contains the image of $x$ under the class function defined by $\varphi$ 
\footnote{If $\varphi$ defines a class function.},
which is the set that would be obtained by applying the replacement axiom.
To prove this, we show the following lemma, which rewrites the above statement in terms of the forcing relation $\Vdash_{\mathrm{HS}}$,
and then prove it.
\begin{isaframe}
\begin{minted}[breaklines]{isabelle}
lemma ex_hs_subset_contains_witnesses : 
    fixes φ env x 
    assumes "φ ∈ formula" "env ∈ list(M)" "arity(φ) ≤ 2 #+ length(env)" "x ∈ M" 
    shows "∃S. S ∈ M ∧ S ⊆ HS ∧ (∀p ∈ G. ∀y ∈ domain(x). 
                (∃z ∈ HS. p ⊩HS φ ([y, z] @ env)) ⟷ (∃z ∈ S. p ⊩HS φ ([y, z] @ env)))" 

  lemma ex_SymExt_elem_contains_witnesses : 
    fixes φ env x 
    assumes "φ ∈ formula" "env ∈ list(SymExt(G))" 
                "arity(φ) ≤ 2 #+ length(env)" "x ∈ SymExt(G)" 
    shows "∃S ∈ SymExt(G). ∀y ∈ x. 
                ((∃z ∈ SymExt(G). sats(SymExt(G), φ, [y, z] @ env)) 
                ⟷ (∃z ∈ S. sats(SymExt(G), φ, [y, z] @ env)))"
\end{minted}
\end{isaframe}

Additionally, the following lemma was helpful in these proofs.
This is also frequently used in the proofs of the axioms.

\begin{isaframe}
\begin{minted}[breaklines]{isabelle}
lemma ex_separation_base : 
    fixes X 
    assumes "X ⊆ HS" "X ∈ M" 
    shows "∃S ∈ SymExt(G). { val(G, x). x ∈ X } ⊆ S" 
\end{minted}
\end{isaframe}
This lemma states that for any set of hereditarily symmetric names $X \in M$, 
there exists a set $S$ in the symmetric extension such that $X^G \subseteq S$.

Then, we prove the axiom schema of replacement for symmetric extensions.
\begin{isaframe}
\begin{minted}[breaklines]{isabelle}
lemma SymExt_replacement :
    fixes φ env
    assumes "φ ∈ formula" "arity(φ) ≤ 2 #+ length(env)" "env ∈ list(SymExt(G))" 
    shows 
          "strong_replacement(##SymExt(G), λx y. sats(SymExt(G), φ, [x, y] @ env))" 
\end{minted}
\end{isaframe}


\subsubsection{Other Axioms}
Other axioms of ZF can be proven relatively easily in symmetric extensions by using separation, replacement and the lemma \texttt{"ex\_separation\_base"}.
For example, the axiom of pairing can be proven as follows. 
Let a, b be elements of a symmetric extension $\mathcal{N}$.
Then, we can take hereditarily symmetric names $\dot{a}, \dot{b}$ of a, b, respectively.
By the lemma \texttt{"ex\_separation\_base"}, we can find a set $S$ in $\mathcal{N}$ such that $\{ \dot{a}, \dot{b} \}^G \subseteq S$.
In other words, $\{ a, b \} \subseteq S$. 
Then by the separation axiom, $\{ x \in S \mid x = a \lor x = b \} = \{ a, b \}$ is an element of $\mathcal{N}$.

Finally we obtain the following lemma, which states that symmetric extensions are models of ZF.
\begin{isaframe}
\begin{minted}[breaklines]{isabelle}
theorem SymExt_sats_ZF : "SymExt(G) ⊨ ZF" 
\end{minted}
\end{isaframe}

Furthermore, since symmetric extensions are transitive, it can be shown that they are 
instances of the locale \texttt{"M\_ZF\_trans"}.
As a result, many results concerning ZF set theory formalized in this locale can now be applied to symmetric extensions.

\begin{isaframe}
\begin{minted}[breaklines]{isabelle}
lemma SymExt_M_ZF_trans : "M_ZF_trans(SymExt(G))"
\end{minted}
\end{isaframe}



\section{The Basic Cohen Model}


\subsection{Defining the Basic Cohen Model}
In this section, we define the basic Cohen model which is a symmetric extension that does not satisfy the axiom of choice,
, as outlined in \cref{sec:outline}.
First, we work within the locale \texttt{"M\_ctm"} which fixes a c.t.m.\  \texttt{"M"} of ZF.


\subsubsection{The Notion of Forcing}
the notion of forcing for the basic Cohen model is defined as the set of finite partial functions from $\omega \times \omega$ to $2$ that are elements of the ground model,
with the order relation given by the superset relation.
We define these as \texttt{"Fn"} and \texttt{"Fn\_leq"} in Isabelle/ZF.

\begin{isaframe}
\begin{minted}[breaklines]{isabelle}
definition Fn where "Fn ≡ { f ∈ Pow((nat × nat) × 2) ∩ M. 
                function(f) ∧ domain(f) ⊆ nat × nat ∧ finite_M(domain(f)) ∧ range(f) ⊆ 2 }"

definition Fn_leq where "Fn_leq ≡ { <f, g> ∈ Fn × Fn. g ⊆ f }" 
\end{minted}
\end{isaframe}
Where \texttt{"nat"} denotes $\omega$ and \texttt{"finite\_M(A)"} denotes that the set \texttt{"A"} is finite.
From the following lemma, this ordered set is an instance of the locale \texttt{"forcing\_data\_partial"}, 
meaning that this is indeed a notion of forcing and forms a partially ordered set with the empty set as the maximum element.
\begin{isaframe}
\begin{minted}[breaklines]{isabelle}
lemma Fn_forcing_data_partial : "forcing_data_partial(Fn, Fn_leq, 0, M, enum)" 
\end{minted}
\end{isaframe}
Where \texttt{"enum"} is the enumeration function of the set \texttt{"M"} fixed in the locale \texttt{"M\_ctm"}.


\subsubsection{The Group of Automorphisms and the Normal Filter}
Next, we define the group of automorphisms \texttt{"Fn\_perms"} on the forcing notion \texttt{"Fn"}.
\begin{isaframe}
\begin{minted}[breaklines]{isabelle}
definition nat_perms where "nat_perms ≡ bij(nat, nat) ∩ M" 
definition Fn_perm where "Fn_perm(f, p) ≡ { <<f`n, m>, l> . <<n, m>, l> ∈ p }" 
definition Fn_perm' where "Fn_perm'(f) ≡ { <p, Fn_perm(f, p)>. p ∈ Fn }" 
definition Fn_perms where "Fn_perms ≡ { Fn_perm'(f). f ∈ nat_perms }" 
\end{minted}
\end{isaframe}
Note that \texttt{"Fn\_perm'(f)"} is the automorphism induced by the bijection \texttt{"f"} on $\omega$,
and \texttt{"Fn\_perm(f, p)"} is the value of the automorphism \texttt{"Fn\_perm'(f)"} on the partial function \texttt{"p"}.
We prove that \texttt{"Fn\_perm'(f)"} is indeed an automorphism on \texttt{"Fn"} and 
\texttt{"Fn\_perms"} is a group of automorphisms on \texttt{"Fn"},
i.e., closed under composition and the inverse operation, and contains the identity function.

\begin{isaframe}
\begin{minted}[breaklines]{isabelle}
lemma Fn_perm'_is_P_auto : 
    fixes f 
    assumes "f ∈ nat_perms" 
    shows "forcing_data_partial.is_P_auto(Fn, Fn_leq, M, Fn_perm'(f))"

lemma Fn_perms_group : 
    "forcing_data_partial.is_P_auto_group(Fn, Fn_leq, M, Fn_perms)" 
\end{minted}
\end{isaframe}

Then, we define the normal filter \texttt{"Fn\_perms\_filter"} on the group of automorphisms \texttt{"Fn\_perms"},
and verify that our definition satisfies the prerequisites of the locale \texttt{"M\_symmetric\_system"}.
This completes both the definition of the basic Cohen model and
the proof that it is a model of ZF.

\begin{isaframe}
\begin{minted}[breaklines]{isabelle}
definition Fn_perms_filter where "Fn_perms_filter ≡ { 
              H ∈ forcing_data_partial.P_auto_subgroups(Fn, Fn_leq, M, Fn_perms).  
              ∃E ∈ Pow(nat) ∩ M. finite_M(E) ∧ Fix(E) ⊆ H }" 

lemma Fn_M_symmetric_system : 
    "M_symmetric_system(Fn, Fn_leq, 0, M, enum, Fn_perms, Fn_perms_filter)" 
\end{minted}
\end{isaframe}


\subsection{Proving the Basic Cohen Model does not Satisfy AC}
Finaly, we prove that the basic Cohen model does not satisfy the axiom of choice.
To do this, we construct a set $A$ that cannot be well-ordered in it.
We work within the locale \texttt{"M\_symmetric\_system"} with parameters, as defined above, 
fixed to generate the basic Cohen model.

We formalize $a_n$ and $A$, introduced in \cref{sec:outline}, 
as \texttt{"binmap\_row(G, n)"} and \texttt{"binmap(G)"}, respectively, for given generic filter \texttt{"G"}.
\begin{isaframe}
\begin{minted}[breaklines]{isabelle}
definition binmap_row where 
    "binmap_row(G, n) ≡ { m ∈ nat. ∃p ∈ G. p`<n, m> = 1 }"

definition binmap where "binmap(G) ≡ { binmap_row(G, n). n ∈ nat }" 
\end{minted}
\end{isaframe}
Then, we prove that $a_n$ are pairwise disjoint, which implies that $A$ is an infinite set.
\begin{isaframe}
\begin{minted}[breaklines]{isabelle}
lemma binmap_row_distinct : 
    fixes G n m 
    assumes "M_generic(G)" "n ∈ nat" "m ∈ nat" "n ≠ m" 
    shows "binmap_row(G, n) ≠ binmap_row(G, m)" 
\end{minted}
\end{isaframe}
Using this lemma, we prove that 
$p \Vdash_{\mathrm{HS}} \text{"} \dot{F} \text{ is a injection from } \dot{\omega} \text{ to } \dot{A} \text{"}$ leads a contradiction 
for any $p \in $ \texttt{"Fn"} and $\dot{F} \in $ \texttt{"HS"}.
This means that there is no injection from $\omega$ to $A$ in the basic Cohen model.
\begin{isaframe}
\begin{minted}[breaklines]{isabelle}
lemma no_injection : 
  fixes F' p0
  assumes "F' ∈ HS" "p0 ∈ Fn"  
                  "ForcesHS(p0, injection_fm(0, 1, 2), [check(nat), binmap', F'])" 
  shows False
\end{minted}
\end{isaframe}
Where \texttt{"ForcesHS(p, φ, env)"} is another notation for \texttt{"p ⊩HS φ env"},
and \texttt{"check(nat)"} and \texttt{"binmap'"} are the hereditarily symmetric names of \texttt{"nat"} and \texttt{"binmap"}, respectively.

Since $A$ is an infinite set, there exists an injection from $\omega$ to $A$.
Therefore, using above lemma, we can show that assuming $A$ is well-orderable leads to a contradiction,

\begin{isaframe}
\begin{minted}[breaklines]{isabelle}
lemma no_wellorder : 
    fixes r G
    assumes 
            "M_generic(G)" "wellordered(##SymExt(G), binmap(G), r)" "r ∈ SymExt(G)" 
    shows False
\end{minted}
\end{isaframe}

Where \texttt{"wellordered(C, A, r)"} denotes that 
the relation \texttt{"r"} is a well-ordering of the set \texttt{"A"} in the class \texttt{"C"}, 
and \texttt{"\#\#"} is an operator that converts a set to a class.
These are defined by Paulson \cite{paulson_AC_consistency}.
Specifically, the definition of \texttt{"wellordered"} is as follows
\footnote{
Note that we can show that 
if \texttt{"wellordered(N, A, r)"} holds for an element \texttt{"A"} in a class \texttt{"N"}, which is a model of ZF,
then \texttt{"r"} is irreflexive and assymetric, i.e., a strict total order.
}:
\begin{isaframe}
\begin{minted}[breaklines]{isabelle}
definition transitive_rel :: "[i⇒o,i,i]⇒o" where
    "transitive_rel(M,A,r) ≡ 
        ∀x[M]. x∈A ⟶ (∀y[M]. y∈A ⟶ (∀z[M]. z∈A ⟶ 
                          ⟨x,y⟩∈r ⟶ ⟨y,z⟩∈r ⟶ ⟨x,z⟩∈r))"
definition linear_rel :: "[i⇒o,i,i]⇒o" where
    "linear_rel(M,A,r) ≡ 
        ∀x[M]. x∈A ⟶ (∀y[M]. y∈A ⟶ ⟨x,y⟩∈r | x=y | ⟨y,x⟩∈r)"

definition wellfounded_on :: "[i⇒o,i,i]⇒o" where
    ― ‹every non-empty SUBSET OF ‹A› has an ‹r›-minimal element›
    "wellfounded_on(M,A,r) ≡ 
        ∀x[M]. x≠0 ⟶ x⊆A ⟶ (∃y[M]. y∈x ∧ ¬(∃z[M]. z∈x ∧ ⟨z,y⟩ ∈ r))"
        
definition wellordered :: "[i⇒o,i,i]⇒o" where
    ― ‹linear and wellfounded on ‹A››
    "wellordered(M,A,r) ≡ 
        transitive_rel(M,A,r) ∧ linear_rel(M,A,r) ∧ wellfounded_on(M,A,r)"
\end{minted}
\end{isaframe}

Thus, the statement "a set $\mathcal{N}$ satisfies well-ordering theorem" can be written as follows:
\begin{isaframe}
\begin{minted}[breaklines]{isabelle}
"∀A ∈ N. ∃r ∈ N. wellordered(##N, A, r)"
\end{minted}
\end{isaframe}

Using the above, we prove the following theorem and complete our relative consistency proof.
This theorem has no additional assumptions from any locale.
\begin{isaframe}
\begin{minted}[breaklines]{isabelle}
theorem ZF_notAC_main_theorem :
    fixes M 
    assumes "nat ≈ M" "M ⊨ ZF" "Transset(M)" 
    shows "∃N. nat ≈ N ∧ N ⊨ ZF ∧ Transset(N) 
                ∧ ¬(∀A ∈ N. ∃r ∈ N. wellordered(##N, A, r))" 
\end{minted}
\end{isaframe}
This theorem states that there exists a c.t.m.\ \texttt{"N"} of ZF+$\neg$AC if there exists a c.t.m.\ \texttt{"M"} of ZF.




\chapter{Conclusion}

We formalized the relative consistency proof of $\neg$AC with ZF
on the basis of the work by Paulson \cite{paulson_AC_consistency} and Gunther et al. \cite{gunther_forcing}.

Our formalization comprises approximately 15,000 lines of code, 
distributed roughly as follows:
\begin{itemize}
  \item Definition of symmetric extensions: 3,000 lines
  \item Proof of symmetric extensions are models of ZF: 5,000 lines
  \item Definition of the basic Cohen model: 2,000 lines
  \item Proof of the basic Cohen model does not satisfy AC: 2,000 lines
  \item Other lemmas : 3,000 lines
\end{itemize}
This is as large as Paluson's formalization of the proof of relative consistency of AC (12,000 lines), 
Gunther et al.'s formalization of coercion (16,000 lines), 
and their proof of independence of CH (12,000 lines).

In our proof, we needed to construct internalized formulas 
representing most of the concepts that belong to the ground model. 
This was to verify that the defined sets truly belong to the ground model 
and to enable tha application of the separation and replacement axioms of the ground model for formulas that include these concepts.

For constructing these formulasm we utilized the sythesis method by Gunther et al. \cite{gunther_forcing};
however, this method could not be applied to complex formulas.
When applied, we also needed to reprove properties of derived formulas in a more usable form.

We also needed to prove that the arity of each defined formula is below its respective threshold.
For complex formulas, Isabelle's automated proof tools were not applicable, 
and we had to prove these properties manually.

\subsubsection{Future Work}

Since these works are repetitive, further automation is desirable.
Ideally, if the arguments of the base model could be handled without explicitly dealing with internalized formulas, 
the amount of work could be reduced when formalizing proofs by using an approach similar to ours.

Additionally, we used the c.t.m.\ approach, and the formalized argument only extends to the construction of a model of ZF+$\neg$AC,
assuming the existence of a c.t.m.\ of ZF.
We did not formalize the justification of the c.t.m.\ approach itself.
Formalizing this part would be a interesting future work.
Ultimately, it would be desirable to derive the relative consistency proof automatically from the assumption of the existence of a c.t.m.; 
however, achieving this within Isabelle/ZF, which is based on first-order logic, seems infeasible.

In addition, formalizing the connections between our approach and alternative approaches to forcing, 
such as Boolean-valued models or syntactic methods, is another topic to explore in the future.

\printbibliography
\end{document}
 