\documentclass[11pt,a4paper]{article}
\usepackage[T1]{fontenc}
\usepackage{isabelle,isabellesym}
\usepackage[numbers]{natbib}

% further packages required for unusual symbols (see also
% isabellesym.sty), use only when needed

\usepackage{amssymb}
  %for \<leadsto>, \<box>, \<diamond>, \<sqsupset>, \<mho>, \<Join>,
  %\<lhd>, \<lesssim>, \<greatersim>, \<lessapprox>, \<greaterapprox>,
  %\<triangleq>, \<yen>, \<lozenge>

%\usepackage{eurosym}
  %for \<euro>

%\usepackage[only,bigsqcap]{stmaryrd}
  %for \<Sqinter>

%\usepackage{eufrak}
  %for \<AA> ... \<ZZ>, \<aa> ... \<zz> (also included in amssymb)

%\usepackage{textcomp}
  %for \<onequarter>, \<onehalf>, \<threequarters>, \<degree>, \<cent>,
  %\<currency>

% this should be the last package used
\usepackage{pdfsetup}

% urls in roman style, theory text in math-similar italics
\urlstyle{rm}
\isabellestyle{it}

% for uniform font size
%\renewcommand{\isastyle}{\isastyleminor}

\renewcommand{\isacharunderscorekeyword}{\mbox{\_}}
\renewcommand{\isacharunderscore}{\mbox{\_}}
\renewcommand{\isasymtturnstile}{\isamath{\Vdash}}
\renewcommand{\isacharminus}{-}
\newcommand{\axiomas}[1]{\mathit{#1}}
\newcommand{\ZFC}{\axiomas{ZFC}}


\begin{document}

\title{Formalization of Forcing in Isabelle/ZF}
\author{Emmanuel Gunther\thanks{Universidad Nacional de C\'ordoba. 
    Facultad de Matem\'atica, Astronom\'{\i}a,  F\'{\i}sica y
    Computaci\'on.}
  \and
  Miguel Pagano\footnotemark[1]
  \and
  Pedro S\'anchez Terraf\footnotemark[1] \thanks{Centro de Investigaci\'on y Estudios de Matem\'atica
    (CIEM-FaMAF), Conicet. C\'ordoba. Argentina.
    Supported by Secyt-UNC project 33620180100465CB.}
}
\maketitle

\begin{abstract}
  We formalize the theory of forcing in the set theory framework of
  Isabelle/ZF. Under the assumption of the existence of a countable
  transitive model of $\ZFC$, we construct a proper generic extension and show
  that the latter also satisfies $\ZFC$.
\end{abstract}


\tableofcontents

% sane default for proof documents
\parindent 0pt\parskip 0.5ex

\section{Introduction}
We formalize the theory of forcing. We work on top of the Isabelle/ZF
framework developed by \citet{DBLP:journals/jar/PaulsonG96}. Our
mechanization is described in more detail in our papers
\cite{2018arXiv180705174G} (LSFA 2018), \cite{2019arXiv190103313G},
and \cite{2020arXiv200109715G} (IJCAR 2020).

\subsection*{Release notes}
\label{sec:release-notes}

We have improved several aspects of our development before submitting
it to the AFP:
\begin{enumerate}
\item Our session \isatt{Forcing} depends on the new release of
  \isatt{ZF-Constructible}.
\item We streamlined the commands for synthesizing renames and formulas.
\item The command that synthesizes formulas produces the lemmas for
  them (the synthesized term is a formula and the equivalence between
  the satisfaction of the synthesized term and the relativized term).
\item Consistently use of structured proofs using Isar (except for one
  coming from a schematic goal command).
\end{enumerate}

A cross-linked HTML version of the development can be found at
\url{https://cs.famaf.unc.edu.ar/~pedro/forcing/}.

% generated text of all theories
%
\begin{isabellebody}%
\setisabellecontext{Forcing{\isacharunderscore}{\kern0pt}Notions}%
%
\isadelimdocument
%
\endisadelimdocument
%
\isatagdocument
%
\isamarkupsection{Forcing notions%
}
\isamarkuptrue%
%
\endisatagdocument
{\isafolddocument}%
%
\isadelimdocument
%
\endisadelimdocument
%
\begin{isamarkuptext}%
This theory defines a locale for forcing notions, that is,
 preorders with a distinguished maximum element.%
\end{isamarkuptext}\isamarkuptrue%
%
\isadelimtheory
%
\endisadelimtheory
%
\isatagtheory
\isacommand{theory}\isamarkupfalse%
\ Forcing{\isacharunderscore}{\kern0pt}Notions\isanewline
\ \ \isakeyword{imports}\ {\isachardoublequoteopen}ZF{\isacharminus}{\kern0pt}Constructible{\isachardot}{\kern0pt}Relative{\isachardoublequoteclose}\isanewline
\isakeyword{begin}%
\endisatagtheory
{\isafoldtheory}%
%
\isadelimtheory
%
\endisadelimtheory
%
\isadelimdocument
%
\endisadelimdocument
%
\isatagdocument
%
\isamarkupsubsection{Basic concepts%
}
\isamarkuptrue%
%
\endisatagdocument
{\isafolddocument}%
%
\isadelimdocument
%
\endisadelimdocument
%
\begin{isamarkuptext}%
We say that two elements $p,q$ are
  \emph{compatible} if they have a lower bound in $P$%
\end{isamarkuptext}\isamarkuptrue%
\isacommand{definition}\isamarkupfalse%
\ compat{\isacharunderscore}{\kern0pt}in\ {\isacharcolon}{\kern0pt}{\isacharcolon}{\kern0pt}\ {\isachardoublequoteopen}i{\isasymRightarrow}i{\isasymRightarrow}i{\isasymRightarrow}i{\isasymRightarrow}o{\isachardoublequoteclose}\ \isakeyword{where}\isanewline
\ \ {\isachardoublequoteopen}compat{\isacharunderscore}{\kern0pt}in{\isacharparenleft}{\kern0pt}A{\isacharcomma}{\kern0pt}r{\isacharcomma}{\kern0pt}p{\isacharcomma}{\kern0pt}q{\isacharparenright}{\kern0pt}\ {\isasymequiv}\ {\isasymexists}d{\isasymin}A\ {\isachardot}{\kern0pt}\ {\isasymlangle}d{\isacharcomma}{\kern0pt}p{\isasymrangle}{\isasymin}r\ {\isasymand}\ {\isasymlangle}d{\isacharcomma}{\kern0pt}q{\isasymrangle}{\isasymin}r{\isachardoublequoteclose}\isanewline
\isanewline
\isacommand{definition}\isamarkupfalse%
\isanewline
\ \ is{\isacharunderscore}{\kern0pt}compat{\isacharunderscore}{\kern0pt}in\ {\isacharcolon}{\kern0pt}{\isacharcolon}{\kern0pt}\ {\isachardoublequoteopen}{\isacharbrackleft}{\kern0pt}i{\isasymRightarrow}o{\isacharcomma}{\kern0pt}i{\isacharcomma}{\kern0pt}i{\isacharcomma}{\kern0pt}i{\isacharcomma}{\kern0pt}i{\isacharbrackright}{\kern0pt}\ {\isasymRightarrow}\ o{\isachardoublequoteclose}\ \isakeyword{where}\isanewline
\ \ {\isachardoublequoteopen}is{\isacharunderscore}{\kern0pt}compat{\isacharunderscore}{\kern0pt}in{\isacharparenleft}{\kern0pt}M{\isacharcomma}{\kern0pt}A{\isacharcomma}{\kern0pt}r{\isacharcomma}{\kern0pt}p{\isacharcomma}{\kern0pt}q{\isacharparenright}{\kern0pt}\ {\isasymequiv}\ {\isasymexists}d{\isacharbrackleft}{\kern0pt}M{\isacharbrackright}{\kern0pt}{\isachardot}{\kern0pt}\ d{\isasymin}A\ {\isasymand}\ {\isacharparenleft}{\kern0pt}{\isasymexists}dp{\isacharbrackleft}{\kern0pt}M{\isacharbrackright}{\kern0pt}{\isachardot}{\kern0pt}\ pair{\isacharparenleft}{\kern0pt}M{\isacharcomma}{\kern0pt}d{\isacharcomma}{\kern0pt}p{\isacharcomma}{\kern0pt}dp{\isacharparenright}{\kern0pt}\ {\isasymand}\ dp{\isasymin}r\ {\isasymand}\ \isanewline
\ \ \ \ \ \ \ \ \ \ \ \ \ \ \ \ \ \ \ \ \ \ \ \ \ \ \ \ \ \ \ \ \ \ \ {\isacharparenleft}{\kern0pt}{\isasymexists}dq{\isacharbrackleft}{\kern0pt}M{\isacharbrackright}{\kern0pt}{\isachardot}{\kern0pt}\ pair{\isacharparenleft}{\kern0pt}M{\isacharcomma}{\kern0pt}d{\isacharcomma}{\kern0pt}q{\isacharcomma}{\kern0pt}dq{\isacharparenright}{\kern0pt}\ {\isasymand}\ dq{\isasymin}r{\isacharparenright}{\kern0pt}{\isacharparenright}{\kern0pt}{\isachardoublequoteclose}\isanewline
\isanewline
\isacommand{lemma}\isamarkupfalse%
\ compat{\isacharunderscore}{\kern0pt}inI\ {\isacharcolon}{\kern0pt}\ \isanewline
\ \ {\isachardoublequoteopen}{\isasymlbrakk}\ d{\isasymin}A\ {\isacharsemicolon}{\kern0pt}\ {\isasymlangle}d{\isacharcomma}{\kern0pt}p{\isasymrangle}{\isasymin}r\ {\isacharsemicolon}{\kern0pt}\ {\isasymlangle}d{\isacharcomma}{\kern0pt}g{\isasymrangle}{\isasymin}r\ {\isasymrbrakk}\ {\isasymLongrightarrow}\ compat{\isacharunderscore}{\kern0pt}in{\isacharparenleft}{\kern0pt}A{\isacharcomma}{\kern0pt}r{\isacharcomma}{\kern0pt}p{\isacharcomma}{\kern0pt}g{\isacharparenright}{\kern0pt}{\isachardoublequoteclose}\isanewline
%
\isadelimproof
\ \ %
\endisadelimproof
%
\isatagproof
\isacommand{by}\isamarkupfalse%
\ {\isacharparenleft}{\kern0pt}auto\ simp\ add{\isacharcolon}{\kern0pt}\ compat{\isacharunderscore}{\kern0pt}in{\isacharunderscore}{\kern0pt}def{\isacharparenright}{\kern0pt}%
\endisatagproof
{\isafoldproof}%
%
\isadelimproof
\isanewline
%
\endisadelimproof
\isanewline
\isacommand{lemma}\isamarkupfalse%
\ refl{\isacharunderscore}{\kern0pt}compat{\isacharcolon}{\kern0pt}\isanewline
\ \ {\isachardoublequoteopen}{\isasymlbrakk}\ refl{\isacharparenleft}{\kern0pt}A{\isacharcomma}{\kern0pt}r{\isacharparenright}{\kern0pt}\ {\isacharsemicolon}{\kern0pt}\ {\isasymlangle}p{\isacharcomma}{\kern0pt}q{\isasymrangle}\ {\isasymin}\ r\ {\isacharbar}{\kern0pt}\ p{\isacharequal}{\kern0pt}q\ {\isacharbar}{\kern0pt}\ {\isasymlangle}q{\isacharcomma}{\kern0pt}p{\isasymrangle}\ {\isasymin}\ r\ {\isacharsemicolon}{\kern0pt}\ p{\isasymin}A\ {\isacharsemicolon}{\kern0pt}\ q{\isasymin}A{\isasymrbrakk}\ {\isasymLongrightarrow}\ compat{\isacharunderscore}{\kern0pt}in{\isacharparenleft}{\kern0pt}A{\isacharcomma}{\kern0pt}r{\isacharcomma}{\kern0pt}p{\isacharcomma}{\kern0pt}q{\isacharparenright}{\kern0pt}{\isachardoublequoteclose}\isanewline
%
\isadelimproof
\ \ %
\endisadelimproof
%
\isatagproof
\isacommand{by}\isamarkupfalse%
\ {\isacharparenleft}{\kern0pt}auto\ simp\ add{\isacharcolon}{\kern0pt}\ refl{\isacharunderscore}{\kern0pt}def\ compat{\isacharunderscore}{\kern0pt}inI{\isacharparenright}{\kern0pt}%
\endisatagproof
{\isafoldproof}%
%
\isadelimproof
\isanewline
%
\endisadelimproof
\isanewline
\isacommand{lemma}\isamarkupfalse%
\ \ chain{\isacharunderscore}{\kern0pt}compat{\isacharcolon}{\kern0pt}\isanewline
\ \ {\isachardoublequoteopen}refl{\isacharparenleft}{\kern0pt}A{\isacharcomma}{\kern0pt}r{\isacharparenright}{\kern0pt}\ {\isasymLongrightarrow}\ linear{\isacharparenleft}{\kern0pt}A{\isacharcomma}{\kern0pt}r{\isacharparenright}{\kern0pt}\ {\isasymLongrightarrow}\ \ {\isacharparenleft}{\kern0pt}{\isasymforall}p{\isasymin}A{\isachardot}{\kern0pt}{\isasymforall}q{\isasymin}A{\isachardot}{\kern0pt}\ compat{\isacharunderscore}{\kern0pt}in{\isacharparenleft}{\kern0pt}A{\isacharcomma}{\kern0pt}r{\isacharcomma}{\kern0pt}p{\isacharcomma}{\kern0pt}q{\isacharparenright}{\kern0pt}{\isacharparenright}{\kern0pt}{\isachardoublequoteclose}\isanewline
%
\isadelimproof
\ \ %
\endisadelimproof
%
\isatagproof
\isacommand{by}\isamarkupfalse%
\ {\isacharparenleft}{\kern0pt}simp\ add{\isacharcolon}{\kern0pt}\ refl{\isacharunderscore}{\kern0pt}compat\ linear{\isacharunderscore}{\kern0pt}def{\isacharparenright}{\kern0pt}%
\endisatagproof
{\isafoldproof}%
%
\isadelimproof
\isanewline
%
\endisadelimproof
\isanewline
\isacommand{lemma}\isamarkupfalse%
\ subset{\isacharunderscore}{\kern0pt}fun{\isacharunderscore}{\kern0pt}image{\isacharcolon}{\kern0pt}\ {\isachardoublequoteopen}f{\isacharcolon}{\kern0pt}N{\isasymrightarrow}P\ {\isasymLongrightarrow}\ f{\isacharbackquote}{\kern0pt}{\isacharbackquote}{\kern0pt}N{\isasymsubseteq}P{\isachardoublequoteclose}\isanewline
%
\isadelimproof
\ \ %
\endisadelimproof
%
\isatagproof
\isacommand{by}\isamarkupfalse%
\ {\isacharparenleft}{\kern0pt}auto\ simp\ add{\isacharcolon}{\kern0pt}\ image{\isacharunderscore}{\kern0pt}fun\ apply{\isacharunderscore}{\kern0pt}funtype{\isacharparenright}{\kern0pt}%
\endisatagproof
{\isafoldproof}%
%
\isadelimproof
\isanewline
%
\endisadelimproof
\isanewline
\isacommand{lemma}\isamarkupfalse%
\ refl{\isacharunderscore}{\kern0pt}monot{\isacharunderscore}{\kern0pt}domain{\isacharcolon}{\kern0pt}\ {\isachardoublequoteopen}refl{\isacharparenleft}{\kern0pt}B{\isacharcomma}{\kern0pt}r{\isacharparenright}{\kern0pt}\ {\isasymLongrightarrow}\ A{\isasymsubseteq}B\ {\isasymLongrightarrow}\ refl{\isacharparenleft}{\kern0pt}A{\isacharcomma}{\kern0pt}r{\isacharparenright}{\kern0pt}{\isachardoublequoteclose}\ \ \isanewline
%
\isadelimproof
\ \ %
\endisadelimproof
%
\isatagproof
\isacommand{unfolding}\isamarkupfalse%
\ refl{\isacharunderscore}{\kern0pt}def\ \isacommand{by}\isamarkupfalse%
\ blast%
\endisatagproof
{\isafoldproof}%
%
\isadelimproof
\isanewline
%
\endisadelimproof
\isanewline
\isacommand{definition}\isamarkupfalse%
\isanewline
\ \ antichain\ {\isacharcolon}{\kern0pt}{\isacharcolon}{\kern0pt}\ {\isachardoublequoteopen}i{\isasymRightarrow}i{\isasymRightarrow}i{\isasymRightarrow}o{\isachardoublequoteclose}\ \isakeyword{where}\isanewline
\ \ {\isachardoublequoteopen}antichain{\isacharparenleft}{\kern0pt}P{\isacharcomma}{\kern0pt}leq{\isacharcomma}{\kern0pt}A{\isacharparenright}{\kern0pt}\ {\isasymequiv}\ A{\isasymsubseteq}P\ {\isasymand}\ {\isacharparenleft}{\kern0pt}{\isasymforall}p{\isasymin}A{\isachardot}{\kern0pt}{\isasymforall}q{\isasymin}A{\isachardot}{\kern0pt}{\isacharparenleft}{\kern0pt}{\isasymnot}\ compat{\isacharunderscore}{\kern0pt}in{\isacharparenleft}{\kern0pt}P{\isacharcomma}{\kern0pt}leq{\isacharcomma}{\kern0pt}p{\isacharcomma}{\kern0pt}q{\isacharparenright}{\kern0pt}{\isacharparenright}{\kern0pt}{\isacharparenright}{\kern0pt}{\isachardoublequoteclose}\isanewline
\isanewline
\isacommand{definition}\isamarkupfalse%
\ \isanewline
\ \ ccc\ {\isacharcolon}{\kern0pt}{\isacharcolon}{\kern0pt}\ {\isachardoublequoteopen}i\ {\isasymRightarrow}\ i\ {\isasymRightarrow}\ o{\isachardoublequoteclose}\ \isakeyword{where}\isanewline
\ \ {\isachardoublequoteopen}ccc{\isacharparenleft}{\kern0pt}P{\isacharcomma}{\kern0pt}leq{\isacharparenright}{\kern0pt}\ {\isasymequiv}\ {\isasymforall}A{\isachardot}{\kern0pt}\ antichain{\isacharparenleft}{\kern0pt}P{\isacharcomma}{\kern0pt}leq{\isacharcomma}{\kern0pt}A{\isacharparenright}{\kern0pt}\ {\isasymlongrightarrow}\ {\isacharbar}{\kern0pt}A{\isacharbar}{\kern0pt}\ {\isasymle}\ nat{\isachardoublequoteclose}\isanewline
\isanewline
\isacommand{locale}\isamarkupfalse%
\ forcing{\isacharunderscore}{\kern0pt}notion\ {\isacharequal}{\kern0pt}\isanewline
\ \ \isakeyword{fixes}\ P\ leq\ one\isanewline
\ \ \isakeyword{assumes}\ one{\isacharunderscore}{\kern0pt}in{\isacharunderscore}{\kern0pt}P{\isacharcolon}{\kern0pt}\ \ \ \ \ \ \ \ \ {\isachardoublequoteopen}one\ {\isasymin}\ P{\isachardoublequoteclose}\isanewline
\ \ \ \ \isakeyword{and}\ leq{\isacharunderscore}{\kern0pt}preord{\isacharcolon}{\kern0pt}\ \ \ \ \ \ \ {\isachardoublequoteopen}preorder{\isacharunderscore}{\kern0pt}on{\isacharparenleft}{\kern0pt}P{\isacharcomma}{\kern0pt}leq{\isacharparenright}{\kern0pt}{\isachardoublequoteclose}\isanewline
\ \ \ \ \isakeyword{and}\ one{\isacharunderscore}{\kern0pt}max{\isacharcolon}{\kern0pt}\ \ \ \ \ \ \ \ \ \ {\isachardoublequoteopen}{\isasymforall}p{\isasymin}P{\isachardot}{\kern0pt}\ {\isasymlangle}p{\isacharcomma}{\kern0pt}one{\isasymrangle}{\isasymin}leq{\isachardoublequoteclose}\isanewline
\isakeyword{begin}\isanewline
\isanewline
\isacommand{abbreviation}\isamarkupfalse%
\ Leq\ {\isacharcolon}{\kern0pt}{\isacharcolon}{\kern0pt}\ {\isachardoublequoteopen}{\isacharbrackleft}{\kern0pt}i{\isacharcomma}{\kern0pt}\ i{\isacharbrackright}{\kern0pt}\ {\isasymRightarrow}\ o{\isachardoublequoteclose}\ \ {\isacharparenleft}{\kern0pt}\isakeyword{infixl}\ {\isachardoublequoteopen}{\isasympreceq}{\isachardoublequoteclose}\ {\isadigit{5}}{\isadigit{0}}{\isacharparenright}{\kern0pt}\isanewline
\ \ \isakeyword{where}\ {\isachardoublequoteopen}x\ {\isasympreceq}\ y\ {\isasymequiv}\ {\isasymlangle}x{\isacharcomma}{\kern0pt}y{\isasymrangle}{\isasymin}leq{\isachardoublequoteclose}\isanewline
\isanewline
\isacommand{lemma}\isamarkupfalse%
\ refl{\isacharunderscore}{\kern0pt}leq{\isacharcolon}{\kern0pt}\isanewline
\ \ {\isachardoublequoteopen}r{\isasymin}P\ {\isasymLongrightarrow}\ r{\isasympreceq}r{\isachardoublequoteclose}\isanewline
%
\isadelimproof
\ \ %
\endisadelimproof
%
\isatagproof
\isacommand{using}\isamarkupfalse%
\ leq{\isacharunderscore}{\kern0pt}preord\ \isacommand{unfolding}\isamarkupfalse%
\ preorder{\isacharunderscore}{\kern0pt}on{\isacharunderscore}{\kern0pt}def\ refl{\isacharunderscore}{\kern0pt}def\ \isacommand{by}\isamarkupfalse%
\ simp%
\endisatagproof
{\isafoldproof}%
%
\isadelimproof
%
\endisadelimproof
%
\begin{isamarkuptext}%
A set $D$ is \emph{dense} if every element $p\in P$ has a lower 
bound in $D$.%
\end{isamarkuptext}\isamarkuptrue%
\isacommand{definition}\isamarkupfalse%
\ \isanewline
\ \ dense\ {\isacharcolon}{\kern0pt}{\isacharcolon}{\kern0pt}\ {\isachardoublequoteopen}i{\isasymRightarrow}o{\isachardoublequoteclose}\ \isakeyword{where}\isanewline
\ \ {\isachardoublequoteopen}dense{\isacharparenleft}{\kern0pt}D{\isacharparenright}{\kern0pt}\ {\isasymequiv}\ {\isasymforall}p{\isasymin}P{\isachardot}{\kern0pt}\ {\isasymexists}d{\isasymin}D\ {\isachardot}{\kern0pt}\ d{\isasympreceq}p{\isachardoublequoteclose}%
\begin{isamarkuptext}%
There is also a weaker definition which asks for 
a lower bound in $D$ only for the elements below some fixed 
element $q$.%
\end{isamarkuptext}\isamarkuptrue%
\isacommand{definition}\isamarkupfalse%
\ \isanewline
\ \ dense{\isacharunderscore}{\kern0pt}below\ {\isacharcolon}{\kern0pt}{\isacharcolon}{\kern0pt}\ {\isachardoublequoteopen}i{\isasymRightarrow}i{\isasymRightarrow}o{\isachardoublequoteclose}\ \isakeyword{where}\isanewline
\ \ {\isachardoublequoteopen}dense{\isacharunderscore}{\kern0pt}below{\isacharparenleft}{\kern0pt}D{\isacharcomma}{\kern0pt}q{\isacharparenright}{\kern0pt}\ {\isasymequiv}\ {\isasymforall}p{\isasymin}P{\isachardot}{\kern0pt}\ p{\isasympreceq}q\ {\isasymlongrightarrow}\ {\isacharparenleft}{\kern0pt}{\isasymexists}d{\isasymin}D{\isachardot}{\kern0pt}\ d{\isasymin}P\ {\isasymand}\ d{\isasympreceq}p{\isacharparenright}{\kern0pt}{\isachardoublequoteclose}\isanewline
\isanewline
\isacommand{lemma}\isamarkupfalse%
\ P{\isacharunderscore}{\kern0pt}dense{\isacharcolon}{\kern0pt}\ {\isachardoublequoteopen}dense{\isacharparenleft}{\kern0pt}P{\isacharparenright}{\kern0pt}{\isachardoublequoteclose}\isanewline
%
\isadelimproof
\ \ %
\endisadelimproof
%
\isatagproof
\isacommand{by}\isamarkupfalse%
\ {\isacharparenleft}{\kern0pt}insert\ leq{\isacharunderscore}{\kern0pt}preord{\isacharcomma}{\kern0pt}\ auto\ simp\ add{\isacharcolon}{\kern0pt}\ preorder{\isacharunderscore}{\kern0pt}on{\isacharunderscore}{\kern0pt}def\ refl{\isacharunderscore}{\kern0pt}def\ dense{\isacharunderscore}{\kern0pt}def{\isacharparenright}{\kern0pt}%
\endisatagproof
{\isafoldproof}%
%
\isadelimproof
\isanewline
%
\endisadelimproof
\isanewline
\isacommand{definition}\isamarkupfalse%
\ \isanewline
\ \ increasing\ {\isacharcolon}{\kern0pt}{\isacharcolon}{\kern0pt}\ {\isachardoublequoteopen}i{\isasymRightarrow}o{\isachardoublequoteclose}\ \isakeyword{where}\isanewline
\ \ {\isachardoublequoteopen}increasing{\isacharparenleft}{\kern0pt}F{\isacharparenright}{\kern0pt}\ {\isasymequiv}\ {\isasymforall}x{\isasymin}F{\isachardot}{\kern0pt}\ {\isasymforall}\ p\ {\isasymin}\ P\ {\isachardot}{\kern0pt}\ x{\isasympreceq}p\ {\isasymlongrightarrow}\ p{\isasymin}F{\isachardoublequoteclose}\isanewline
\isanewline
\isacommand{definition}\isamarkupfalse%
\ \isanewline
\ \ compat\ {\isacharcolon}{\kern0pt}{\isacharcolon}{\kern0pt}\ {\isachardoublequoteopen}i{\isasymRightarrow}i{\isasymRightarrow}o{\isachardoublequoteclose}\ \isakeyword{where}\isanewline
\ \ {\isachardoublequoteopen}compat{\isacharparenleft}{\kern0pt}p{\isacharcomma}{\kern0pt}q{\isacharparenright}{\kern0pt}\ {\isasymequiv}\ compat{\isacharunderscore}{\kern0pt}in{\isacharparenleft}{\kern0pt}P{\isacharcomma}{\kern0pt}leq{\isacharcomma}{\kern0pt}p{\isacharcomma}{\kern0pt}q{\isacharparenright}{\kern0pt}{\isachardoublequoteclose}\isanewline
\isanewline
\isacommand{lemma}\isamarkupfalse%
\ leq{\isacharunderscore}{\kern0pt}transD{\isacharcolon}{\kern0pt}\ \ {\isachardoublequoteopen}a{\isasympreceq}b\ {\isasymLongrightarrow}\ b{\isasympreceq}c\ {\isasymLongrightarrow}\ a\ {\isasymin}\ P{\isasymLongrightarrow}\ b\ {\isasymin}\ P{\isasymLongrightarrow}\ c\ {\isasymin}\ P{\isasymLongrightarrow}\ a{\isasympreceq}c{\isachardoublequoteclose}\isanewline
%
\isadelimproof
\ \ %
\endisadelimproof
%
\isatagproof
\isacommand{using}\isamarkupfalse%
\ leq{\isacharunderscore}{\kern0pt}preord\ trans{\isacharunderscore}{\kern0pt}onD\ \isacommand{unfolding}\isamarkupfalse%
\ preorder{\isacharunderscore}{\kern0pt}on{\isacharunderscore}{\kern0pt}def\ \isacommand{by}\isamarkupfalse%
\ blast%
\endisatagproof
{\isafoldproof}%
%
\isadelimproof
\isanewline
%
\endisadelimproof
\isanewline
\isacommand{lemma}\isamarkupfalse%
\ leq{\isacharunderscore}{\kern0pt}transD{\isacharprime}{\kern0pt}{\isacharcolon}{\kern0pt}\ \ {\isachardoublequoteopen}A{\isasymsubseteq}P\ {\isasymLongrightarrow}\ a{\isasympreceq}b\ {\isasymLongrightarrow}\ b{\isasympreceq}c\ {\isasymLongrightarrow}\ a\ {\isasymin}\ A\ {\isasymLongrightarrow}\ b\ {\isasymin}\ P{\isasymLongrightarrow}\ c\ {\isasymin}\ P{\isasymLongrightarrow}\ a{\isasympreceq}c{\isachardoublequoteclose}\isanewline
%
\isadelimproof
\ \ %
\endisadelimproof
%
\isatagproof
\isacommand{using}\isamarkupfalse%
\ leq{\isacharunderscore}{\kern0pt}preord\ trans{\isacharunderscore}{\kern0pt}onD\ subsetD\ \isacommand{unfolding}\isamarkupfalse%
\ preorder{\isacharunderscore}{\kern0pt}on{\isacharunderscore}{\kern0pt}def\ \isacommand{by}\isamarkupfalse%
\ blast%
\endisatagproof
{\isafoldproof}%
%
\isadelimproof
\isanewline
%
\endisadelimproof
\isanewline
\isanewline
\isacommand{lemma}\isamarkupfalse%
\ leq{\isacharunderscore}{\kern0pt}reflI{\isacharcolon}{\kern0pt}\ {\isachardoublequoteopen}p{\isasymin}P\ {\isasymLongrightarrow}\ p{\isasympreceq}p{\isachardoublequoteclose}\isanewline
%
\isadelimproof
\ \ %
\endisadelimproof
%
\isatagproof
\isacommand{using}\isamarkupfalse%
\ leq{\isacharunderscore}{\kern0pt}preord\ \isacommand{unfolding}\isamarkupfalse%
\ preorder{\isacharunderscore}{\kern0pt}on{\isacharunderscore}{\kern0pt}def\ refl{\isacharunderscore}{\kern0pt}def\ \isacommand{by}\isamarkupfalse%
\ blast%
\endisatagproof
{\isafoldproof}%
%
\isadelimproof
\isanewline
%
\endisadelimproof
\isanewline
\isacommand{lemma}\isamarkupfalse%
\ compatD{\isacharbrackleft}{\kern0pt}dest{\isacharbang}{\kern0pt}{\isacharbrackright}{\kern0pt}{\isacharcolon}{\kern0pt}\ {\isachardoublequoteopen}compat{\isacharparenleft}{\kern0pt}p{\isacharcomma}{\kern0pt}q{\isacharparenright}{\kern0pt}\ {\isasymLongrightarrow}\ {\isasymexists}d{\isasymin}P{\isachardot}{\kern0pt}\ d{\isasympreceq}p\ {\isasymand}\ d{\isasympreceq}q{\isachardoublequoteclose}\isanewline
%
\isadelimproof
\ \ %
\endisadelimproof
%
\isatagproof
\isacommand{unfolding}\isamarkupfalse%
\ compat{\isacharunderscore}{\kern0pt}def\ compat{\isacharunderscore}{\kern0pt}in{\isacharunderscore}{\kern0pt}def\ \isacommand{{\isachardot}{\kern0pt}}\isamarkupfalse%
%
\endisatagproof
{\isafoldproof}%
%
\isadelimproof
\isanewline
%
\endisadelimproof
\isanewline
\isacommand{abbreviation}\isamarkupfalse%
\ Incompatible\ {\isacharcolon}{\kern0pt}{\isacharcolon}{\kern0pt}\ {\isachardoublequoteopen}{\isacharbrackleft}{\kern0pt}i{\isacharcomma}{\kern0pt}\ i{\isacharbrackright}{\kern0pt}\ {\isasymRightarrow}\ o{\isachardoublequoteclose}\ \ {\isacharparenleft}{\kern0pt}\isakeyword{infixl}\ {\isachardoublequoteopen}{\isasymbottom}{\isachardoublequoteclose}\ {\isadigit{5}}{\isadigit{0}}{\isacharparenright}{\kern0pt}\isanewline
\ \ \isakeyword{where}\ {\isachardoublequoteopen}p\ {\isasymbottom}\ q\ {\isasymequiv}\ {\isasymnot}\ compat{\isacharparenleft}{\kern0pt}p{\isacharcomma}{\kern0pt}q{\isacharparenright}{\kern0pt}{\isachardoublequoteclose}\isanewline
\isanewline
\isacommand{lemma}\isamarkupfalse%
\ compatI{\isacharbrackleft}{\kern0pt}intro{\isacharbang}{\kern0pt}{\isacharbrackright}{\kern0pt}{\isacharcolon}{\kern0pt}\ {\isachardoublequoteopen}d{\isasymin}P\ {\isasymLongrightarrow}\ d{\isasympreceq}p\ {\isasymLongrightarrow}\ d{\isasympreceq}q\ {\isasymLongrightarrow}\ compat{\isacharparenleft}{\kern0pt}p{\isacharcomma}{\kern0pt}q{\isacharparenright}{\kern0pt}{\isachardoublequoteclose}\isanewline
%
\isadelimproof
\ \ %
\endisadelimproof
%
\isatagproof
\isacommand{unfolding}\isamarkupfalse%
\ compat{\isacharunderscore}{\kern0pt}def\ compat{\isacharunderscore}{\kern0pt}in{\isacharunderscore}{\kern0pt}def\ \isacommand{by}\isamarkupfalse%
\ blast%
\endisatagproof
{\isafoldproof}%
%
\isadelimproof
\isanewline
%
\endisadelimproof
\isanewline
\isacommand{lemma}\isamarkupfalse%
\ denseD\ {\isacharbrackleft}{\kern0pt}dest{\isacharbrackright}{\kern0pt}{\isacharcolon}{\kern0pt}\ {\isachardoublequoteopen}dense{\isacharparenleft}{\kern0pt}D{\isacharparenright}{\kern0pt}\ {\isasymLongrightarrow}\ p{\isasymin}P\ {\isasymLongrightarrow}\ \ {\isasymexists}d{\isasymin}D{\isachardot}{\kern0pt}\ d{\isasympreceq}\ p{\isachardoublequoteclose}\isanewline
%
\isadelimproof
\ \ %
\endisadelimproof
%
\isatagproof
\isacommand{unfolding}\isamarkupfalse%
\ dense{\isacharunderscore}{\kern0pt}def\ \isacommand{by}\isamarkupfalse%
\ blast%
\endisatagproof
{\isafoldproof}%
%
\isadelimproof
\isanewline
%
\endisadelimproof
\isanewline
\isacommand{lemma}\isamarkupfalse%
\ denseI\ {\isacharbrackleft}{\kern0pt}intro{\isacharbang}{\kern0pt}{\isacharbrackright}{\kern0pt}{\isacharcolon}{\kern0pt}\ {\isachardoublequoteopen}{\isasymlbrakk}\ {\isasymAnd}p{\isachardot}{\kern0pt}\ p{\isasymin}P\ {\isasymLongrightarrow}\ {\isasymexists}d{\isasymin}D{\isachardot}{\kern0pt}\ d{\isasympreceq}\ p\ {\isasymrbrakk}\ {\isasymLongrightarrow}\ dense{\isacharparenleft}{\kern0pt}D{\isacharparenright}{\kern0pt}{\isachardoublequoteclose}\isanewline
%
\isadelimproof
\ \ %
\endisadelimproof
%
\isatagproof
\isacommand{unfolding}\isamarkupfalse%
\ dense{\isacharunderscore}{\kern0pt}def\ \isacommand{by}\isamarkupfalse%
\ blast%
\endisatagproof
{\isafoldproof}%
%
\isadelimproof
\isanewline
%
\endisadelimproof
\isanewline
\isacommand{lemma}\isamarkupfalse%
\ dense{\isacharunderscore}{\kern0pt}belowD\ {\isacharbrackleft}{\kern0pt}dest{\isacharbrackright}{\kern0pt}{\isacharcolon}{\kern0pt}\isanewline
\ \ \isakeyword{assumes}\ {\isachardoublequoteopen}dense{\isacharunderscore}{\kern0pt}below{\isacharparenleft}{\kern0pt}D{\isacharcomma}{\kern0pt}p{\isacharparenright}{\kern0pt}{\isachardoublequoteclose}\ {\isachardoublequoteopen}q{\isasymin}P{\isachardoublequoteclose}\ {\isachardoublequoteopen}q{\isasympreceq}p{\isachardoublequoteclose}\isanewline
\ \ \isakeyword{shows}\ {\isachardoublequoteopen}{\isasymexists}d{\isasymin}D{\isachardot}{\kern0pt}\ d{\isasymin}P\ {\isasymand}\ d{\isasympreceq}q{\isachardoublequoteclose}\isanewline
%
\isadelimproof
\ \ %
\endisadelimproof
%
\isatagproof
\isacommand{using}\isamarkupfalse%
\ assms\ \isacommand{unfolding}\isamarkupfalse%
\ dense{\isacharunderscore}{\kern0pt}below{\isacharunderscore}{\kern0pt}def\ \isacommand{by}\isamarkupfalse%
\ simp%
\endisatagproof
{\isafoldproof}%
%
\isadelimproof
\isanewline
%
\endisadelimproof
\ \ \ \ \isanewline
\isanewline
\isacommand{lemma}\isamarkupfalse%
\ dense{\isacharunderscore}{\kern0pt}belowI\ {\isacharbrackleft}{\kern0pt}intro{\isacharbang}{\kern0pt}{\isacharbrackright}{\kern0pt}{\isacharcolon}{\kern0pt}\ \isanewline
\ \ \isakeyword{assumes}\ {\isachardoublequoteopen}{\isasymAnd}q{\isachardot}{\kern0pt}\ q{\isasymin}P\ {\isasymLongrightarrow}\ q{\isasympreceq}p\ {\isasymLongrightarrow}\ {\isasymexists}d{\isasymin}D{\isachardot}{\kern0pt}\ d{\isasymin}P\ {\isasymand}\ d{\isasympreceq}q{\isachardoublequoteclose}\ \isanewline
\ \ \isakeyword{shows}\ {\isachardoublequoteopen}dense{\isacharunderscore}{\kern0pt}below{\isacharparenleft}{\kern0pt}D{\isacharcomma}{\kern0pt}p{\isacharparenright}{\kern0pt}{\isachardoublequoteclose}\isanewline
%
\isadelimproof
\ \ %
\endisadelimproof
%
\isatagproof
\isacommand{using}\isamarkupfalse%
\ assms\ \isacommand{unfolding}\isamarkupfalse%
\ dense{\isacharunderscore}{\kern0pt}below{\isacharunderscore}{\kern0pt}def\ \isacommand{by}\isamarkupfalse%
\ simp%
\endisatagproof
{\isafoldproof}%
%
\isadelimproof
\isanewline
%
\endisadelimproof
\isanewline
\isacommand{lemma}\isamarkupfalse%
\ dense{\isacharunderscore}{\kern0pt}below{\isacharunderscore}{\kern0pt}cong{\isacharcolon}{\kern0pt}\ {\isachardoublequoteopen}p{\isasymin}P\ {\isasymLongrightarrow}\ D\ {\isacharequal}{\kern0pt}\ D{\isacharprime}{\kern0pt}\ {\isasymLongrightarrow}\ dense{\isacharunderscore}{\kern0pt}below{\isacharparenleft}{\kern0pt}D{\isacharcomma}{\kern0pt}p{\isacharparenright}{\kern0pt}\ {\isasymlongleftrightarrow}\ dense{\isacharunderscore}{\kern0pt}below{\isacharparenleft}{\kern0pt}D{\isacharprime}{\kern0pt}{\isacharcomma}{\kern0pt}p{\isacharparenright}{\kern0pt}{\isachardoublequoteclose}\isanewline
%
\isadelimproof
\ \ %
\endisadelimproof
%
\isatagproof
\isacommand{by}\isamarkupfalse%
\ blast%
\endisatagproof
{\isafoldproof}%
%
\isadelimproof
\isanewline
%
\endisadelimproof
\isanewline
\isacommand{lemma}\isamarkupfalse%
\ dense{\isacharunderscore}{\kern0pt}below{\isacharunderscore}{\kern0pt}cong{\isacharprime}{\kern0pt}{\isacharcolon}{\kern0pt}\ {\isachardoublequoteopen}p{\isasymin}P\ {\isasymLongrightarrow}\ {\isasymlbrakk}{\isasymAnd}x{\isachardot}{\kern0pt}\ x{\isasymin}P\ {\isasymLongrightarrow}\ Q{\isacharparenleft}{\kern0pt}x{\isacharparenright}{\kern0pt}\ {\isasymlongleftrightarrow}\ Q{\isacharprime}{\kern0pt}{\isacharparenleft}{\kern0pt}x{\isacharparenright}{\kern0pt}{\isasymrbrakk}\ {\isasymLongrightarrow}\ \isanewline
\ \ \ \ \ \ \ \ \ \ \ dense{\isacharunderscore}{\kern0pt}below{\isacharparenleft}{\kern0pt}{\isacharbraceleft}{\kern0pt}q{\isasymin}P{\isachardot}{\kern0pt}\ Q{\isacharparenleft}{\kern0pt}q{\isacharparenright}{\kern0pt}{\isacharbraceright}{\kern0pt}{\isacharcomma}{\kern0pt}p{\isacharparenright}{\kern0pt}\ {\isasymlongleftrightarrow}\ dense{\isacharunderscore}{\kern0pt}below{\isacharparenleft}{\kern0pt}{\isacharbraceleft}{\kern0pt}q{\isasymin}P{\isachardot}{\kern0pt}\ Q{\isacharprime}{\kern0pt}{\isacharparenleft}{\kern0pt}q{\isacharparenright}{\kern0pt}{\isacharbraceright}{\kern0pt}{\isacharcomma}{\kern0pt}p{\isacharparenright}{\kern0pt}{\isachardoublequoteclose}\isanewline
%
\isadelimproof
\ \ %
\endisadelimproof
%
\isatagproof
\isacommand{by}\isamarkupfalse%
\ blast%
\endisatagproof
{\isafoldproof}%
%
\isadelimproof
\isanewline
%
\endisadelimproof
\isanewline
\isacommand{lemma}\isamarkupfalse%
\ dense{\isacharunderscore}{\kern0pt}below{\isacharunderscore}{\kern0pt}mono{\isacharcolon}{\kern0pt}\ {\isachardoublequoteopen}p{\isasymin}P\ {\isasymLongrightarrow}\ D\ {\isasymsubseteq}\ D{\isacharprime}{\kern0pt}\ {\isasymLongrightarrow}\ dense{\isacharunderscore}{\kern0pt}below{\isacharparenleft}{\kern0pt}D{\isacharcomma}{\kern0pt}p{\isacharparenright}{\kern0pt}\ {\isasymLongrightarrow}\ dense{\isacharunderscore}{\kern0pt}below{\isacharparenleft}{\kern0pt}D{\isacharprime}{\kern0pt}{\isacharcomma}{\kern0pt}p{\isacharparenright}{\kern0pt}{\isachardoublequoteclose}\isanewline
%
\isadelimproof
\ \ %
\endisadelimproof
%
\isatagproof
\isacommand{by}\isamarkupfalse%
\ blast%
\endisatagproof
{\isafoldproof}%
%
\isadelimproof
\isanewline
%
\endisadelimproof
\isanewline
\isacommand{lemma}\isamarkupfalse%
\ dense{\isacharunderscore}{\kern0pt}below{\isacharunderscore}{\kern0pt}under{\isacharcolon}{\kern0pt}\isanewline
\ \ \isakeyword{assumes}\ {\isachardoublequoteopen}dense{\isacharunderscore}{\kern0pt}below{\isacharparenleft}{\kern0pt}D{\isacharcomma}{\kern0pt}p{\isacharparenright}{\kern0pt}{\isachardoublequoteclose}\ {\isachardoublequoteopen}p{\isasymin}P{\isachardoublequoteclose}\ {\isachardoublequoteopen}q{\isasymin}P{\isachardoublequoteclose}\ {\isachardoublequoteopen}q{\isasympreceq}p{\isachardoublequoteclose}\isanewline
\ \ \isakeyword{shows}\ {\isachardoublequoteopen}dense{\isacharunderscore}{\kern0pt}below{\isacharparenleft}{\kern0pt}D{\isacharcomma}{\kern0pt}q{\isacharparenright}{\kern0pt}{\isachardoublequoteclose}\isanewline
%
\isadelimproof
\ \ %
\endisadelimproof
%
\isatagproof
\isacommand{using}\isamarkupfalse%
\ assms\ leq{\isacharunderscore}{\kern0pt}transD\ \isacommand{by}\isamarkupfalse%
\ blast%
\endisatagproof
{\isafoldproof}%
%
\isadelimproof
\isanewline
%
\endisadelimproof
\isanewline
\isacommand{lemma}\isamarkupfalse%
\ ideal{\isacharunderscore}{\kern0pt}dense{\isacharunderscore}{\kern0pt}below{\isacharcolon}{\kern0pt}\isanewline
\ \ \isakeyword{assumes}\ {\isachardoublequoteopen}{\isasymAnd}q{\isachardot}{\kern0pt}\ q{\isasymin}P\ {\isasymLongrightarrow}\ q{\isasympreceq}p\ {\isasymLongrightarrow}\ q{\isasymin}D{\isachardoublequoteclose}\isanewline
\ \ \isakeyword{shows}\ {\isachardoublequoteopen}dense{\isacharunderscore}{\kern0pt}below{\isacharparenleft}{\kern0pt}D{\isacharcomma}{\kern0pt}p{\isacharparenright}{\kern0pt}{\isachardoublequoteclose}\isanewline
%
\isadelimproof
\ \ %
\endisadelimproof
%
\isatagproof
\isacommand{using}\isamarkupfalse%
\ assms\ leq{\isacharunderscore}{\kern0pt}reflI\ \isacommand{by}\isamarkupfalse%
\ blast%
\endisatagproof
{\isafoldproof}%
%
\isadelimproof
\isanewline
%
\endisadelimproof
\isanewline
\isacommand{lemma}\isamarkupfalse%
\ dense{\isacharunderscore}{\kern0pt}below{\isacharunderscore}{\kern0pt}dense{\isacharunderscore}{\kern0pt}below{\isacharcolon}{\kern0pt}\ \isanewline
\ \ \isakeyword{assumes}\ {\isachardoublequoteopen}dense{\isacharunderscore}{\kern0pt}below{\isacharparenleft}{\kern0pt}{\isacharbraceleft}{\kern0pt}q{\isasymin}P{\isachardot}{\kern0pt}\ dense{\isacharunderscore}{\kern0pt}below{\isacharparenleft}{\kern0pt}D{\isacharcomma}{\kern0pt}q{\isacharparenright}{\kern0pt}{\isacharbraceright}{\kern0pt}{\isacharcomma}{\kern0pt}p{\isacharparenright}{\kern0pt}{\isachardoublequoteclose}\ {\isachardoublequoteopen}p{\isasymin}P{\isachardoublequoteclose}\ \isanewline
\ \ \isakeyword{shows}\ {\isachardoublequoteopen}dense{\isacharunderscore}{\kern0pt}below{\isacharparenleft}{\kern0pt}D{\isacharcomma}{\kern0pt}p{\isacharparenright}{\kern0pt}{\isachardoublequoteclose}\ \ \isanewline
%
\isadelimproof
\ \ %
\endisadelimproof
%
\isatagproof
\isacommand{using}\isamarkupfalse%
\ assms\ leq{\isacharunderscore}{\kern0pt}transD\ leq{\isacharunderscore}{\kern0pt}reflI\ \ \isacommand{by}\isamarkupfalse%
\ blast%
\endisatagproof
{\isafoldproof}%
%
\isadelimproof
\isanewline
%
\endisadelimproof
\ \ \ \ \isanewline
\ \ \ \ \isanewline
\isanewline
\isacommand{definition}\isamarkupfalse%
\isanewline
\ \ antichain\ {\isacharcolon}{\kern0pt}{\isacharcolon}{\kern0pt}\ {\isachardoublequoteopen}i{\isasymRightarrow}o{\isachardoublequoteclose}\ \isakeyword{where}\isanewline
\ \ {\isachardoublequoteopen}antichain{\isacharparenleft}{\kern0pt}A{\isacharparenright}{\kern0pt}\ {\isasymequiv}\ A{\isasymsubseteq}P\ {\isasymand}\ {\isacharparenleft}{\kern0pt}{\isasymforall}p{\isasymin}A{\isachardot}{\kern0pt}{\isasymforall}q{\isasymin}A{\isachardot}{\kern0pt}{\isacharparenleft}{\kern0pt}{\isasymnot}compat{\isacharparenleft}{\kern0pt}p{\isacharcomma}{\kern0pt}q{\isacharparenright}{\kern0pt}{\isacharparenright}{\kern0pt}{\isacharparenright}{\kern0pt}{\isachardoublequoteclose}%
\begin{isamarkuptext}%
A filter is an increasing set $G$ with all its elements 
being compatible in $G$.%
\end{isamarkuptext}\isamarkuptrue%
\isacommand{definition}\isamarkupfalse%
\ \isanewline
\ \ filter\ {\isacharcolon}{\kern0pt}{\isacharcolon}{\kern0pt}\ {\isachardoublequoteopen}i{\isasymRightarrow}o{\isachardoublequoteclose}\ \isakeyword{where}\isanewline
\ \ {\isachardoublequoteopen}filter{\isacharparenleft}{\kern0pt}G{\isacharparenright}{\kern0pt}\ {\isasymequiv}\ G{\isasymsubseteq}P\ {\isasymand}\ increasing{\isacharparenleft}{\kern0pt}G{\isacharparenright}{\kern0pt}\ {\isasymand}\ {\isacharparenleft}{\kern0pt}{\isasymforall}p{\isasymin}G{\isachardot}{\kern0pt}\ {\isasymforall}q{\isasymin}G{\isachardot}{\kern0pt}\ compat{\isacharunderscore}{\kern0pt}in{\isacharparenleft}{\kern0pt}G{\isacharcomma}{\kern0pt}leq{\isacharcomma}{\kern0pt}p{\isacharcomma}{\kern0pt}q{\isacharparenright}{\kern0pt}{\isacharparenright}{\kern0pt}{\isachardoublequoteclose}\isanewline
\isanewline
\isacommand{lemma}\isamarkupfalse%
\ filterD\ {\isacharcolon}{\kern0pt}\ {\isachardoublequoteopen}filter{\isacharparenleft}{\kern0pt}G{\isacharparenright}{\kern0pt}\ {\isasymLongrightarrow}\ x\ {\isasymin}\ G\ {\isasymLongrightarrow}\ x\ {\isasymin}\ P{\isachardoublequoteclose}\isanewline
%
\isadelimproof
\ \ %
\endisadelimproof
%
\isatagproof
\isacommand{by}\isamarkupfalse%
\ {\isacharparenleft}{\kern0pt}auto\ simp\ add\ {\isacharcolon}{\kern0pt}\ subsetD\ filter{\isacharunderscore}{\kern0pt}def{\isacharparenright}{\kern0pt}%
\endisatagproof
{\isafoldproof}%
%
\isadelimproof
\isanewline
%
\endisadelimproof
\isanewline
\isacommand{lemma}\isamarkupfalse%
\ filter{\isacharunderscore}{\kern0pt}leqD\ {\isacharcolon}{\kern0pt}\ {\isachardoublequoteopen}filter{\isacharparenleft}{\kern0pt}G{\isacharparenright}{\kern0pt}\ {\isasymLongrightarrow}\ x\ {\isasymin}\ G\ {\isasymLongrightarrow}\ y\ {\isasymin}\ P\ {\isasymLongrightarrow}\ x{\isasympreceq}y\ {\isasymLongrightarrow}\ y\ {\isasymin}\ G{\isachardoublequoteclose}\isanewline
%
\isadelimproof
\ \ %
\endisadelimproof
%
\isatagproof
\isacommand{by}\isamarkupfalse%
\ {\isacharparenleft}{\kern0pt}simp\ add{\isacharcolon}{\kern0pt}\ filter{\isacharunderscore}{\kern0pt}def\ increasing{\isacharunderscore}{\kern0pt}def{\isacharparenright}{\kern0pt}%
\endisatagproof
{\isafoldproof}%
%
\isadelimproof
\isanewline
%
\endisadelimproof
\isanewline
\isacommand{lemma}\isamarkupfalse%
\ filter{\isacharunderscore}{\kern0pt}imp{\isacharunderscore}{\kern0pt}compat{\isacharcolon}{\kern0pt}\ {\isachardoublequoteopen}filter{\isacharparenleft}{\kern0pt}G{\isacharparenright}{\kern0pt}\ {\isasymLongrightarrow}\ p{\isasymin}G\ {\isasymLongrightarrow}\ q{\isasymin}G\ {\isasymLongrightarrow}\ compat{\isacharparenleft}{\kern0pt}p{\isacharcomma}{\kern0pt}q{\isacharparenright}{\kern0pt}{\isachardoublequoteclose}\isanewline
%
\isadelimproof
\ \ %
\endisadelimproof
%
\isatagproof
\isacommand{unfolding}\isamarkupfalse%
\ filter{\isacharunderscore}{\kern0pt}def\ compat{\isacharunderscore}{\kern0pt}in{\isacharunderscore}{\kern0pt}def\ compat{\isacharunderscore}{\kern0pt}def\ \isacommand{by}\isamarkupfalse%
\ blast%
\endisatagproof
{\isafoldproof}%
%
\isadelimproof
\isanewline
%
\endisadelimproof
\isanewline
\isacommand{lemma}\isamarkupfalse%
\ low{\isacharunderscore}{\kern0pt}bound{\isacharunderscore}{\kern0pt}filter{\isacharcolon}{\kern0pt}\ %
\isamarkupcmt{says the compatibility is attained inside G%
}\isanewline
\ \ \isakeyword{assumes}\ {\isachardoublequoteopen}filter{\isacharparenleft}{\kern0pt}G{\isacharparenright}{\kern0pt}{\isachardoublequoteclose}\ \isakeyword{and}\ {\isachardoublequoteopen}p{\isasymin}G{\isachardoublequoteclose}\ \isakeyword{and}\ {\isachardoublequoteopen}q{\isasymin}G{\isachardoublequoteclose}\isanewline
\ \ \isakeyword{shows}\ {\isachardoublequoteopen}{\isasymexists}r{\isasymin}G{\isachardot}{\kern0pt}\ r{\isasympreceq}p\ {\isasymand}\ r{\isasympreceq}q{\isachardoublequoteclose}\ \isanewline
%
\isadelimproof
\ \ %
\endisadelimproof
%
\isatagproof
\isacommand{using}\isamarkupfalse%
\ assms\ \isanewline
\ \ \isacommand{unfolding}\isamarkupfalse%
\ compat{\isacharunderscore}{\kern0pt}in{\isacharunderscore}{\kern0pt}def\ filter{\isacharunderscore}{\kern0pt}def\ \isacommand{by}\isamarkupfalse%
\ blast%
\endisatagproof
{\isafoldproof}%
%
\isadelimproof
%
\endisadelimproof
%
\begin{isamarkuptext}%
We finally introduce the upward closure of a set
and prove that the closure of $A$ is a filter if its elements are
compatible in $A$.%
\end{isamarkuptext}\isamarkuptrue%
\isacommand{definition}\isamarkupfalse%
\ \ \isanewline
\ \ upclosure\ {\isacharcolon}{\kern0pt}{\isacharcolon}{\kern0pt}\ {\isachardoublequoteopen}i{\isasymRightarrow}i{\isachardoublequoteclose}\ \isakeyword{where}\isanewline
\ \ {\isachardoublequoteopen}upclosure{\isacharparenleft}{\kern0pt}A{\isacharparenright}{\kern0pt}\ {\isasymequiv}\ {\isacharbraceleft}{\kern0pt}p{\isasymin}P{\isachardot}{\kern0pt}{\isasymexists}a{\isasymin}A{\isachardot}{\kern0pt}\ a{\isasympreceq}p{\isacharbraceright}{\kern0pt}{\isachardoublequoteclose}\isanewline
\isanewline
\isacommand{lemma}\isamarkupfalse%
\ \ upclosureI\ {\isacharbrackleft}{\kern0pt}intro{\isacharbrackright}{\kern0pt}\ {\isacharcolon}{\kern0pt}\ {\isachardoublequoteopen}p{\isasymin}P\ {\isasymLongrightarrow}\ a{\isasymin}A\ {\isasymLongrightarrow}\ a{\isasympreceq}p\ {\isasymLongrightarrow}\ p{\isasymin}upclosure{\isacharparenleft}{\kern0pt}A{\isacharparenright}{\kern0pt}{\isachardoublequoteclose}\isanewline
%
\isadelimproof
\ \ %
\endisadelimproof
%
\isatagproof
\isacommand{by}\isamarkupfalse%
\ {\isacharparenleft}{\kern0pt}simp\ add{\isacharcolon}{\kern0pt}upclosure{\isacharunderscore}{\kern0pt}def{\isacharcomma}{\kern0pt}\ auto{\isacharparenright}{\kern0pt}%
\endisatagproof
{\isafoldproof}%
%
\isadelimproof
\isanewline
%
\endisadelimproof
\isanewline
\isacommand{lemma}\isamarkupfalse%
\ \ upclosureE\ {\isacharbrackleft}{\kern0pt}elim{\isacharbrackright}{\kern0pt}\ {\isacharcolon}{\kern0pt}\isanewline
\ \ {\isachardoublequoteopen}p{\isasymin}upclosure{\isacharparenleft}{\kern0pt}A{\isacharparenright}{\kern0pt}\ {\isasymLongrightarrow}\ {\isacharparenleft}{\kern0pt}{\isasymAnd}x\ a{\isachardot}{\kern0pt}\ x{\isasymin}P\ {\isasymLongrightarrow}\ a{\isasymin}A\ {\isasymLongrightarrow}\ a{\isasympreceq}x\ {\isasymLongrightarrow}\ R{\isacharparenright}{\kern0pt}\ {\isasymLongrightarrow}\ R{\isachardoublequoteclose}\isanewline
%
\isadelimproof
\ \ %
\endisadelimproof
%
\isatagproof
\isacommand{by}\isamarkupfalse%
\ {\isacharparenleft}{\kern0pt}auto\ simp\ add{\isacharcolon}{\kern0pt}upclosure{\isacharunderscore}{\kern0pt}def{\isacharparenright}{\kern0pt}%
\endisatagproof
{\isafoldproof}%
%
\isadelimproof
\isanewline
%
\endisadelimproof
\isanewline
\isacommand{lemma}\isamarkupfalse%
\ \ upclosureD\ {\isacharbrackleft}{\kern0pt}dest{\isacharbrackright}{\kern0pt}\ {\isacharcolon}{\kern0pt}\isanewline
\ \ {\isachardoublequoteopen}p{\isasymin}upclosure{\isacharparenleft}{\kern0pt}A{\isacharparenright}{\kern0pt}\ {\isasymLongrightarrow}\ {\isasymexists}a{\isasymin}A{\isachardot}{\kern0pt}{\isacharparenleft}{\kern0pt}a{\isasympreceq}p{\isacharparenright}{\kern0pt}\ {\isasymand}\ p{\isasymin}P{\isachardoublequoteclose}\isanewline
%
\isadelimproof
\ \ %
\endisadelimproof
%
\isatagproof
\isacommand{by}\isamarkupfalse%
\ {\isacharparenleft}{\kern0pt}simp\ add{\isacharcolon}{\kern0pt}upclosure{\isacharunderscore}{\kern0pt}def{\isacharparenright}{\kern0pt}%
\endisatagproof
{\isafoldproof}%
%
\isadelimproof
\isanewline
%
\endisadelimproof
\isanewline
\isacommand{lemma}\isamarkupfalse%
\ upclosure{\isacharunderscore}{\kern0pt}increasing\ {\isacharcolon}{\kern0pt}\isanewline
\ \ \isakeyword{assumes}\ {\isachardoublequoteopen}A{\isasymsubseteq}P{\isachardoublequoteclose}\isanewline
\ \ \isakeyword{shows}\ {\isachardoublequoteopen}increasing{\isacharparenleft}{\kern0pt}upclosure{\isacharparenleft}{\kern0pt}A{\isacharparenright}{\kern0pt}{\isacharparenright}{\kern0pt}{\isachardoublequoteclose}\isanewline
%
\isadelimproof
\ \ %
\endisadelimproof
%
\isatagproof
\isacommand{unfolding}\isamarkupfalse%
\ increasing{\isacharunderscore}{\kern0pt}def\ upclosure{\isacharunderscore}{\kern0pt}def\isanewline
\ \ \isacommand{using}\isamarkupfalse%
\ leq{\isacharunderscore}{\kern0pt}transD{\isacharprime}{\kern0pt}{\isacharbrackleft}{\kern0pt}OF\ {\isacartoucheopen}A{\isasymsubseteq}P{\isacartoucheclose}{\isacharbrackright}{\kern0pt}\ \isacommand{by}\isamarkupfalse%
\ auto%
\endisatagproof
{\isafoldproof}%
%
\isadelimproof
\isanewline
%
\endisadelimproof
\isanewline
\isacommand{lemma}\isamarkupfalse%
\ \ upclosure{\isacharunderscore}{\kern0pt}in{\isacharunderscore}{\kern0pt}P{\isacharcolon}{\kern0pt}\ {\isachardoublequoteopen}A\ {\isasymsubseteq}\ P\ {\isasymLongrightarrow}\ upclosure{\isacharparenleft}{\kern0pt}A{\isacharparenright}{\kern0pt}\ {\isasymsubseteq}\ P{\isachardoublequoteclose}\isanewline
%
\isadelimproof
\ \ %
\endisadelimproof
%
\isatagproof
\isacommand{using}\isamarkupfalse%
\ subsetI\ upclosure{\isacharunderscore}{\kern0pt}def\ \isacommand{by}\isamarkupfalse%
\ simp%
\endisatagproof
{\isafoldproof}%
%
\isadelimproof
\isanewline
%
\endisadelimproof
\isanewline
\isacommand{lemma}\isamarkupfalse%
\ \ A{\isacharunderscore}{\kern0pt}sub{\isacharunderscore}{\kern0pt}upclosure{\isacharcolon}{\kern0pt}\ {\isachardoublequoteopen}A\ {\isasymsubseteq}\ P\ {\isasymLongrightarrow}\ A{\isasymsubseteq}upclosure{\isacharparenleft}{\kern0pt}A{\isacharparenright}{\kern0pt}{\isachardoublequoteclose}\isanewline
%
\isadelimproof
\ \ %
\endisadelimproof
%
\isatagproof
\isacommand{using}\isamarkupfalse%
\ subsetI\ leq{\isacharunderscore}{\kern0pt}preord\ \isanewline
\ \ \isacommand{unfolding}\isamarkupfalse%
\ upclosure{\isacharunderscore}{\kern0pt}def\ preorder{\isacharunderscore}{\kern0pt}on{\isacharunderscore}{\kern0pt}def\ refl{\isacharunderscore}{\kern0pt}def\ \isacommand{by}\isamarkupfalse%
\ auto%
\endisatagproof
{\isafoldproof}%
%
\isadelimproof
\isanewline
%
\endisadelimproof
\isanewline
\isacommand{lemma}\isamarkupfalse%
\ \ elem{\isacharunderscore}{\kern0pt}upclosure{\isacharcolon}{\kern0pt}\ {\isachardoublequoteopen}A{\isasymsubseteq}P\ {\isasymLongrightarrow}\ x{\isasymin}A\ \ {\isasymLongrightarrow}\ x{\isasymin}upclosure{\isacharparenleft}{\kern0pt}A{\isacharparenright}{\kern0pt}{\isachardoublequoteclose}\isanewline
%
\isadelimproof
\ \ %
\endisadelimproof
%
\isatagproof
\isacommand{by}\isamarkupfalse%
\ {\isacharparenleft}{\kern0pt}blast\ dest{\isacharcolon}{\kern0pt}A{\isacharunderscore}{\kern0pt}sub{\isacharunderscore}{\kern0pt}upclosure{\isacharparenright}{\kern0pt}%
\endisatagproof
{\isafoldproof}%
%
\isadelimproof
\isanewline
%
\endisadelimproof
\isanewline
\isacommand{lemma}\isamarkupfalse%
\ \ closure{\isacharunderscore}{\kern0pt}compat{\isacharunderscore}{\kern0pt}filter{\isacharcolon}{\kern0pt}\isanewline
\ \ \isakeyword{assumes}\ {\isachardoublequoteopen}A{\isasymsubseteq}P{\isachardoublequoteclose}\ {\isachardoublequoteopen}{\isacharparenleft}{\kern0pt}{\isasymforall}p{\isasymin}A{\isachardot}{\kern0pt}{\isasymforall}q{\isasymin}A{\isachardot}{\kern0pt}\ compat{\isacharunderscore}{\kern0pt}in{\isacharparenleft}{\kern0pt}A{\isacharcomma}{\kern0pt}leq{\isacharcomma}{\kern0pt}p{\isacharcomma}{\kern0pt}q{\isacharparenright}{\kern0pt}{\isacharparenright}{\kern0pt}{\isachardoublequoteclose}\isanewline
\ \ \isakeyword{shows}\ {\isachardoublequoteopen}filter{\isacharparenleft}{\kern0pt}upclosure{\isacharparenleft}{\kern0pt}A{\isacharparenright}{\kern0pt}{\isacharparenright}{\kern0pt}{\isachardoublequoteclose}\isanewline
%
\isadelimproof
\ \ %
\endisadelimproof
%
\isatagproof
\isacommand{unfolding}\isamarkupfalse%
\ filter{\isacharunderscore}{\kern0pt}def\isanewline
\isacommand{proof}\isamarkupfalse%
{\isacharparenleft}{\kern0pt}auto{\isacharparenright}{\kern0pt}\isanewline
\ \ \isacommand{show}\isamarkupfalse%
\ {\isachardoublequoteopen}increasing{\isacharparenleft}{\kern0pt}upclosure{\isacharparenleft}{\kern0pt}A{\isacharparenright}{\kern0pt}{\isacharparenright}{\kern0pt}{\isachardoublequoteclose}\isanewline
\ \ \ \ \isacommand{using}\isamarkupfalse%
\ assms\ upclosure{\isacharunderscore}{\kern0pt}increasing\ \isacommand{by}\isamarkupfalse%
\ simp\isanewline
\isacommand{next}\isamarkupfalse%
\isanewline
\ \ \isacommand{let}\isamarkupfalse%
\ {\isacharquery}{\kern0pt}UA{\isacharequal}{\kern0pt}{\isachardoublequoteopen}upclosure{\isacharparenleft}{\kern0pt}A{\isacharparenright}{\kern0pt}{\isachardoublequoteclose}\isanewline
\ \ \isacommand{show}\isamarkupfalse%
\ {\isachardoublequoteopen}compat{\isacharunderscore}{\kern0pt}in{\isacharparenleft}{\kern0pt}upclosure{\isacharparenleft}{\kern0pt}A{\isacharparenright}{\kern0pt}{\isacharcomma}{\kern0pt}\ leq{\isacharcomma}{\kern0pt}\ p{\isacharcomma}{\kern0pt}\ q{\isacharparenright}{\kern0pt}{\isachardoublequoteclose}\ \isakeyword{if}\ {\isachardoublequoteopen}p{\isasymin}{\isacharquery}{\kern0pt}UA{\isachardoublequoteclose}\ {\isachardoublequoteopen}q{\isasymin}{\isacharquery}{\kern0pt}UA{\isachardoublequoteclose}\ \isakeyword{for}\ p\ q\isanewline
\ \ \isacommand{proof}\isamarkupfalse%
\ {\isacharminus}{\kern0pt}\isanewline
\ \ \ \ \isacommand{from}\isamarkupfalse%
\ that\isanewline
\ \ \ \ \isacommand{obtain}\isamarkupfalse%
\ a\ b\ \isakeyword{where}\ {\isadigit{1}}{\isacharcolon}{\kern0pt}{\isachardoublequoteopen}a{\isasymin}A{\isachardoublequoteclose}\ {\isachardoublequoteopen}b{\isasymin}A{\isachardoublequoteclose}\ {\isachardoublequoteopen}a{\isasympreceq}p{\isachardoublequoteclose}\ {\isachardoublequoteopen}b{\isasympreceq}q{\isachardoublequoteclose}\ {\isachardoublequoteopen}p{\isasymin}P{\isachardoublequoteclose}\ {\isachardoublequoteopen}q{\isasymin}P{\isachardoublequoteclose}\isanewline
\ \ \ \ \ \ \isacommand{using}\isamarkupfalse%
\ upclosureD{\isacharbrackleft}{\kern0pt}OF\ {\isacartoucheopen}p{\isasymin}{\isacharquery}{\kern0pt}UA{\isacartoucheclose}{\isacharbrackright}{\kern0pt}\ upclosureD{\isacharbrackleft}{\kern0pt}OF\ {\isacartoucheopen}q{\isasymin}{\isacharquery}{\kern0pt}UA{\isacartoucheclose}{\isacharbrackright}{\kern0pt}\ \isacommand{by}\isamarkupfalse%
\ auto\isanewline
\ \ \ \ \isacommand{with}\isamarkupfalse%
\ assms{\isacharparenleft}{\kern0pt}{\isadigit{2}}{\isacharparenright}{\kern0pt}\isanewline
\ \ \ \ \isacommand{obtain}\isamarkupfalse%
\ d\ \isakeyword{where}\ {\isachardoublequoteopen}d{\isasymin}A{\isachardoublequoteclose}\ {\isachardoublequoteopen}d{\isasympreceq}a{\isachardoublequoteclose}\ {\isachardoublequoteopen}d{\isasympreceq}b{\isachardoublequoteclose}\isanewline
\ \ \ \ \ \ \isacommand{unfolding}\isamarkupfalse%
\ compat{\isacharunderscore}{\kern0pt}in{\isacharunderscore}{\kern0pt}def\ \isacommand{by}\isamarkupfalse%
\ auto\isanewline
\ \ \ \ \isacommand{with}\isamarkupfalse%
\ {\isadigit{1}}\isanewline
\ \ \ \ \isacommand{have}\isamarkupfalse%
\ {\isadigit{2}}{\isacharcolon}{\kern0pt}{\isachardoublequoteopen}d{\isasympreceq}p{\isachardoublequoteclose}\ {\isachardoublequoteopen}d{\isasympreceq}q{\isachardoublequoteclose}\ {\isachardoublequoteopen}d{\isasymin}{\isacharquery}{\kern0pt}UA{\isachardoublequoteclose}\isanewline
\ \ \ \ \ \ \isacommand{using}\isamarkupfalse%
\ A{\isacharunderscore}{\kern0pt}sub{\isacharunderscore}{\kern0pt}upclosure{\isacharbrackleft}{\kern0pt}THEN\ subsetD{\isacharbrackright}{\kern0pt}\ {\isacartoucheopen}A{\isasymsubseteq}P{\isacartoucheclose}\isanewline
\ \ \ \ \ \ \ \ leq{\isacharunderscore}{\kern0pt}transD{\isacharprime}{\kern0pt}{\isacharbrackleft}{\kern0pt}of\ A\ d\ a{\isacharbrackright}{\kern0pt}\ leq{\isacharunderscore}{\kern0pt}transD{\isacharprime}{\kern0pt}{\isacharbrackleft}{\kern0pt}of\ A\ d\ b{\isacharbrackright}{\kern0pt}\ \isacommand{by}\isamarkupfalse%
\ auto\isanewline
\ \ \ \ \isacommand{then}\isamarkupfalse%
\isanewline
\ \ \ \ \isacommand{show}\isamarkupfalse%
\ {\isacharquery}{\kern0pt}thesis\ \isacommand{unfolding}\isamarkupfalse%
\ compat{\isacharunderscore}{\kern0pt}in{\isacharunderscore}{\kern0pt}def\ \isacommand{by}\isamarkupfalse%
\ auto\isanewline
\ \ \isacommand{qed}\isamarkupfalse%
\isanewline
\isacommand{qed}\isamarkupfalse%
%
\endisatagproof
{\isafoldproof}%
%
\isadelimproof
\isanewline
%
\endisadelimproof
\isanewline
\isacommand{lemma}\isamarkupfalse%
\ \ aux{\isacharunderscore}{\kern0pt}RS{\isadigit{1}}{\isacharcolon}{\kern0pt}\ \ {\isachardoublequoteopen}f\ {\isasymin}\ N\ {\isasymrightarrow}\ P\ {\isasymLongrightarrow}\ n{\isasymin}N\ {\isasymLongrightarrow}\ f{\isacharbackquote}{\kern0pt}n\ {\isasymin}\ upclosure{\isacharparenleft}{\kern0pt}f\ {\isacharbackquote}{\kern0pt}{\isacharbackquote}{\kern0pt}N{\isacharparenright}{\kern0pt}{\isachardoublequoteclose}\isanewline
%
\isadelimproof
\ \ %
\endisadelimproof
%
\isatagproof
\isacommand{using}\isamarkupfalse%
\ elem{\isacharunderscore}{\kern0pt}upclosure{\isacharbrackleft}{\kern0pt}OF\ subset{\isacharunderscore}{\kern0pt}fun{\isacharunderscore}{\kern0pt}image{\isacharbrackright}{\kern0pt}\ image{\isacharunderscore}{\kern0pt}fun\isanewline
\ \ \isacommand{by}\isamarkupfalse%
\ {\isacharparenleft}{\kern0pt}simp{\isacharcomma}{\kern0pt}\ blast{\isacharparenright}{\kern0pt}%
\endisatagproof
{\isafoldproof}%
%
\isadelimproof
\isanewline
%
\endisadelimproof
\isanewline
\isacommand{lemma}\isamarkupfalse%
\ decr{\isacharunderscore}{\kern0pt}succ{\isacharunderscore}{\kern0pt}decr{\isacharcolon}{\kern0pt}\ \isanewline
\ \ \isakeyword{assumes}\ {\isachardoublequoteopen}f\ {\isasymin}\ nat\ {\isasymrightarrow}\ P{\isachardoublequoteclose}\ {\isachardoublequoteopen}preorder{\isacharunderscore}{\kern0pt}on{\isacharparenleft}{\kern0pt}P{\isacharcomma}{\kern0pt}leq{\isacharparenright}{\kern0pt}{\isachardoublequoteclose}\isanewline
\ \ \ \ {\isachardoublequoteopen}{\isasymforall}n{\isasymin}nat{\isachardot}{\kern0pt}\ \ {\isasymlangle}f\ {\isacharbackquote}{\kern0pt}\ succ{\isacharparenleft}{\kern0pt}n{\isacharparenright}{\kern0pt}{\isacharcomma}{\kern0pt}\ f\ {\isacharbackquote}{\kern0pt}\ n{\isasymrangle}\ {\isasymin}\ leq{\isachardoublequoteclose}\isanewline
\ \ \ \ {\isachardoublequoteopen}m{\isasymin}nat{\isachardoublequoteclose}\isanewline
\ \ \isakeyword{shows}\ {\isachardoublequoteopen}n{\isasymin}nat\ {\isasymLongrightarrow}\ n{\isasymle}m\ {\isasymLongrightarrow}\ {\isasymlangle}f\ {\isacharbackquote}{\kern0pt}\ m{\isacharcomma}{\kern0pt}\ f\ {\isacharbackquote}{\kern0pt}\ n{\isasymrangle}\ {\isasymin}\ leq{\isachardoublequoteclose}\isanewline
%
\isadelimproof
\ \ %
\endisadelimproof
%
\isatagproof
\isacommand{using}\isamarkupfalse%
\ {\isacartoucheopen}m{\isasymin}{\isacharunderscore}{\kern0pt}{\isacartoucheclose}\isanewline
\isacommand{proof}\isamarkupfalse%
{\isacharparenleft}{\kern0pt}induct\ m{\isacharparenright}{\kern0pt}\isanewline
\ \ \isacommand{case}\isamarkupfalse%
\ {\isadigit{0}}\isanewline
\ \ \isacommand{then}\isamarkupfalse%
\ \isacommand{show}\isamarkupfalse%
\ {\isacharquery}{\kern0pt}case\ \isacommand{using}\isamarkupfalse%
\ assms\ leq{\isacharunderscore}{\kern0pt}reflI\ \isacommand{by}\isamarkupfalse%
\ simp\isanewline
\isacommand{next}\isamarkupfalse%
\isanewline
\ \ \isacommand{case}\isamarkupfalse%
\ {\isacharparenleft}{\kern0pt}succ\ x{\isacharparenright}{\kern0pt}\isanewline
\ \ \isacommand{then}\isamarkupfalse%
\isanewline
\ \ \isacommand{have}\isamarkupfalse%
\ {\isadigit{1}}{\isacharcolon}{\kern0pt}{\isachardoublequoteopen}f{\isacharbackquote}{\kern0pt}succ{\isacharparenleft}{\kern0pt}x{\isacharparenright}{\kern0pt}\ {\isasympreceq}\ f{\isacharbackquote}{\kern0pt}x{\isachardoublequoteclose}\ {\isachardoublequoteopen}f{\isacharbackquote}{\kern0pt}n{\isasymin}P{\isachardoublequoteclose}\ {\isachardoublequoteopen}f{\isacharbackquote}{\kern0pt}x{\isasymin}P{\isachardoublequoteclose}\ {\isachardoublequoteopen}f{\isacharbackquote}{\kern0pt}succ{\isacharparenleft}{\kern0pt}x{\isacharparenright}{\kern0pt}{\isasymin}P{\isachardoublequoteclose}\isanewline
\ \ \ \ \isacommand{using}\isamarkupfalse%
\ assms\ \isacommand{by}\isamarkupfalse%
\ simp{\isacharunderscore}{\kern0pt}all\isanewline
\ \ \isacommand{consider}\isamarkupfalse%
\ {\isacharparenleft}{\kern0pt}lt{\isacharparenright}{\kern0pt}\ {\isachardoublequoteopen}n{\isacharless}{\kern0pt}succ{\isacharparenleft}{\kern0pt}x{\isacharparenright}{\kern0pt}{\isachardoublequoteclose}\ {\isacharbar}{\kern0pt}\ {\isacharparenleft}{\kern0pt}eq{\isacharparenright}{\kern0pt}\ {\isachardoublequoteopen}n{\isacharequal}{\kern0pt}succ{\isacharparenleft}{\kern0pt}x{\isacharparenright}{\kern0pt}{\isachardoublequoteclose}\isanewline
\ \ \ \ \isacommand{using}\isamarkupfalse%
\ succ\ le{\isacharunderscore}{\kern0pt}succ{\isacharunderscore}{\kern0pt}iff\ \isacommand{by}\isamarkupfalse%
\ auto\isanewline
\ \ \isacommand{then}\isamarkupfalse%
\ \isanewline
\ \ \isacommand{show}\isamarkupfalse%
\ {\isacharquery}{\kern0pt}case\ \isanewline
\ \ \isacommand{proof}\isamarkupfalse%
{\isacharparenleft}{\kern0pt}cases{\isacharparenright}{\kern0pt}\isanewline
\ \ \ \ \isacommand{case}\isamarkupfalse%
\ lt\isanewline
\ \ \ \ \isacommand{with}\isamarkupfalse%
\ {\isadigit{1}}\ \isacommand{show}\isamarkupfalse%
\ {\isacharquery}{\kern0pt}thesis\ \isacommand{using}\isamarkupfalse%
\ leI\ succ\ leq{\isacharunderscore}{\kern0pt}transD\ \isacommand{by}\isamarkupfalse%
\ auto\isanewline
\ \ \isacommand{next}\isamarkupfalse%
\isanewline
\ \ \ \ \isacommand{case}\isamarkupfalse%
\ eq\isanewline
\ \ \ \ \isacommand{with}\isamarkupfalse%
\ {\isadigit{1}}\ \isacommand{show}\isamarkupfalse%
\ {\isacharquery}{\kern0pt}thesis\ \isacommand{using}\isamarkupfalse%
\ leq{\isacharunderscore}{\kern0pt}reflI\ \isacommand{by}\isamarkupfalse%
\ simp\isanewline
\ \ \isacommand{qed}\isamarkupfalse%
\isanewline
\isacommand{qed}\isamarkupfalse%
%
\endisatagproof
{\isafoldproof}%
%
\isadelimproof
\isanewline
%
\endisadelimproof
\isanewline
\isacommand{lemma}\isamarkupfalse%
\ decr{\isacharunderscore}{\kern0pt}seq{\isacharunderscore}{\kern0pt}linear{\isacharcolon}{\kern0pt}\ \isanewline
\ \ \isakeyword{assumes}\ {\isachardoublequoteopen}refl{\isacharparenleft}{\kern0pt}P{\isacharcomma}{\kern0pt}leq{\isacharparenright}{\kern0pt}{\isachardoublequoteclose}\ {\isachardoublequoteopen}f\ {\isasymin}\ nat\ {\isasymrightarrow}\ P{\isachardoublequoteclose}\isanewline
\ \ \ \ {\isachardoublequoteopen}{\isasymforall}n{\isasymin}nat{\isachardot}{\kern0pt}\ \ {\isasymlangle}f\ {\isacharbackquote}{\kern0pt}\ succ{\isacharparenleft}{\kern0pt}n{\isacharparenright}{\kern0pt}{\isacharcomma}{\kern0pt}\ f\ {\isacharbackquote}{\kern0pt}\ n{\isasymrangle}\ {\isasymin}\ leq{\isachardoublequoteclose}\isanewline
\ \ \ \ {\isachardoublequoteopen}trans{\isacharbrackleft}{\kern0pt}P{\isacharbrackright}{\kern0pt}{\isacharparenleft}{\kern0pt}leq{\isacharparenright}{\kern0pt}{\isachardoublequoteclose}\isanewline
\ \ \isakeyword{shows}\ {\isachardoublequoteopen}linear{\isacharparenleft}{\kern0pt}f\ {\isacharbackquote}{\kern0pt}{\isacharbackquote}{\kern0pt}\ nat{\isacharcomma}{\kern0pt}\ leq{\isacharparenright}{\kern0pt}{\isachardoublequoteclose}\isanewline
%
\isadelimproof
%
\endisadelimproof
%
\isatagproof
\isacommand{proof}\isamarkupfalse%
\ {\isacharminus}{\kern0pt}\isanewline
\ \ \isacommand{have}\isamarkupfalse%
\ {\isachardoublequoteopen}preorder{\isacharunderscore}{\kern0pt}on{\isacharparenleft}{\kern0pt}P{\isacharcomma}{\kern0pt}leq{\isacharparenright}{\kern0pt}{\isachardoublequoteclose}\ \isanewline
\ \ \ \ \isacommand{unfolding}\isamarkupfalse%
\ preorder{\isacharunderscore}{\kern0pt}on{\isacharunderscore}{\kern0pt}def\ \isacommand{using}\isamarkupfalse%
\ assms\ \isacommand{by}\isamarkupfalse%
\ simp\isanewline
\ \ \isacommand{{\isacharbraceleft}{\kern0pt}}\isamarkupfalse%
\isanewline
\ \ \ \ \isacommand{fix}\isamarkupfalse%
\ n\ m\isanewline
\ \ \ \ \isacommand{assume}\isamarkupfalse%
\ {\isachardoublequoteopen}n{\isasymin}nat{\isachardoublequoteclose}\ {\isachardoublequoteopen}m{\isasymin}nat{\isachardoublequoteclose}\isanewline
\ \ \ \ \isacommand{then}\isamarkupfalse%
\isanewline
\ \ \ \ \isacommand{have}\isamarkupfalse%
\ {\isachardoublequoteopen}f{\isacharbackquote}{\kern0pt}m\ {\isasympreceq}\ f{\isacharbackquote}{\kern0pt}n\ {\isasymor}\ f{\isacharbackquote}{\kern0pt}n\ {\isasympreceq}\ f{\isacharbackquote}{\kern0pt}m{\isachardoublequoteclose}\isanewline
\ \ \ \ \isacommand{proof}\isamarkupfalse%
{\isacharparenleft}{\kern0pt}cases\ {\isachardoublequoteopen}m{\isasymle}n{\isachardoublequoteclose}{\isacharparenright}{\kern0pt}\isanewline
\ \ \ \ \ \ \isacommand{case}\isamarkupfalse%
\ True\isanewline
\ \ \ \ \ \ \isacommand{with}\isamarkupfalse%
\ {\isacartoucheopen}n{\isasymin}{\isacharunderscore}{\kern0pt}{\isacartoucheclose}\ {\isacartoucheopen}m{\isasymin}{\isacharunderscore}{\kern0pt}{\isacartoucheclose}\isanewline
\ \ \ \ \ \ \isacommand{show}\isamarkupfalse%
\ {\isacharquery}{\kern0pt}thesis\ \isanewline
\ \ \ \ \ \ \ \ \isacommand{using}\isamarkupfalse%
\ decr{\isacharunderscore}{\kern0pt}succ{\isacharunderscore}{\kern0pt}decr{\isacharbrackleft}{\kern0pt}of\ f\ n\ m{\isacharbrackright}{\kern0pt}\ assms\ leI\ {\isacartoucheopen}preorder{\isacharunderscore}{\kern0pt}on{\isacharparenleft}{\kern0pt}P{\isacharcomma}{\kern0pt}leq{\isacharparenright}{\kern0pt}{\isacartoucheclose}\ \isacommand{by}\isamarkupfalse%
\ simp\isanewline
\ \ \ \ \isacommand{next}\isamarkupfalse%
\isanewline
\ \ \ \ \ \ \isacommand{case}\isamarkupfalse%
\ False\isanewline
\ \ \ \ \ \ \isacommand{with}\isamarkupfalse%
\ {\isacartoucheopen}n{\isasymin}{\isacharunderscore}{\kern0pt}{\isacartoucheclose}\ {\isacartoucheopen}m{\isasymin}{\isacharunderscore}{\kern0pt}{\isacartoucheclose}\isanewline
\ \ \ \ \ \ \isacommand{show}\isamarkupfalse%
\ {\isacharquery}{\kern0pt}thesis\ \isanewline
\ \ \ \ \ \ \ \ \isacommand{using}\isamarkupfalse%
\ decr{\isacharunderscore}{\kern0pt}succ{\isacharunderscore}{\kern0pt}decr{\isacharbrackleft}{\kern0pt}of\ f\ m\ n{\isacharbrackright}{\kern0pt}\ assms\ leI\ not{\isacharunderscore}{\kern0pt}le{\isacharunderscore}{\kern0pt}iff{\isacharunderscore}{\kern0pt}lt\ {\isacartoucheopen}preorder{\isacharunderscore}{\kern0pt}on{\isacharparenleft}{\kern0pt}P{\isacharcomma}{\kern0pt}leq{\isacharparenright}{\kern0pt}{\isacartoucheclose}\ \isacommand{by}\isamarkupfalse%
\ simp\isanewline
\ \ \ \ \isacommand{qed}\isamarkupfalse%
\isanewline
\ \ \isacommand{{\isacharbraceright}{\kern0pt}}\isamarkupfalse%
\isanewline
\ \ \isacommand{then}\isamarkupfalse%
\isanewline
\ \ \isacommand{show}\isamarkupfalse%
\ {\isacharquery}{\kern0pt}thesis\isanewline
\ \ \ \ \isacommand{unfolding}\isamarkupfalse%
\ linear{\isacharunderscore}{\kern0pt}def\ \isacommand{using}\isamarkupfalse%
\ ball{\isacharunderscore}{\kern0pt}image{\isacharunderscore}{\kern0pt}simp\ assms\ \isacommand{by}\isamarkupfalse%
\ auto\isanewline
\isacommand{qed}\isamarkupfalse%
%
\endisatagproof
{\isafoldproof}%
%
\isadelimproof
\isanewline
%
\endisadelimproof
\isanewline
\isacommand{end}\isamarkupfalse%
%
\isadelimdocument
%
\endisadelimdocument
%
\isatagdocument
%
\isamarkupsubsection{Towards Rasiowa-Sikorski Lemma (RSL)%
}
\isamarkuptrue%
%
\endisatagdocument
{\isafolddocument}%
%
\isadelimdocument
%
\endisadelimdocument
\isacommand{locale}\isamarkupfalse%
\ countable{\isacharunderscore}{\kern0pt}generic\ {\isacharequal}{\kern0pt}\ forcing{\isacharunderscore}{\kern0pt}notion\ {\isacharplus}{\kern0pt}\isanewline
\ \ \isakeyword{fixes}\ {\isasymD}\isanewline
\ \ \isakeyword{assumes}\ countable{\isacharunderscore}{\kern0pt}subs{\isacharunderscore}{\kern0pt}of{\isacharunderscore}{\kern0pt}P{\isacharcolon}{\kern0pt}\ \ {\isachardoublequoteopen}{\isasymD}\ {\isasymin}\ nat{\isasymrightarrow}Pow{\isacharparenleft}{\kern0pt}P{\isacharparenright}{\kern0pt}{\isachardoublequoteclose}\isanewline
\ \ \ \ \isakeyword{and}\ \ \ \ \ seq{\isacharunderscore}{\kern0pt}of{\isacharunderscore}{\kern0pt}denses{\isacharcolon}{\kern0pt}\ \ \ \ \ \ \ \ {\isachardoublequoteopen}{\isasymforall}n\ {\isasymin}\ nat{\isachardot}{\kern0pt}\ dense{\isacharparenleft}{\kern0pt}{\isasymD}{\isacharbackquote}{\kern0pt}n{\isacharparenright}{\kern0pt}{\isachardoublequoteclose}\isanewline
\isanewline
\isakeyword{begin}\isanewline
\isanewline
\isacommand{definition}\isamarkupfalse%
\isanewline
\ \ D{\isacharunderscore}{\kern0pt}generic\ {\isacharcolon}{\kern0pt}{\isacharcolon}{\kern0pt}\ {\isachardoublequoteopen}i{\isasymRightarrow}o{\isachardoublequoteclose}\ \isakeyword{where}\isanewline
\ \ {\isachardoublequoteopen}D{\isacharunderscore}{\kern0pt}generic{\isacharparenleft}{\kern0pt}G{\isacharparenright}{\kern0pt}\ {\isasymequiv}\ filter{\isacharparenleft}{\kern0pt}G{\isacharparenright}{\kern0pt}\ {\isasymand}\ {\isacharparenleft}{\kern0pt}{\isasymforall}n{\isasymin}nat{\isachardot}{\kern0pt}{\isacharparenleft}{\kern0pt}{\isasymD}{\isacharbackquote}{\kern0pt}n{\isacharparenright}{\kern0pt}{\isasyminter}G{\isasymnoteq}{\isadigit{0}}{\isacharparenright}{\kern0pt}{\isachardoublequoteclose}%
\begin{isamarkuptext}%
The next lemma identifies a sufficient condition for obtaining
RSL.%
\end{isamarkuptext}\isamarkuptrue%
\isacommand{lemma}\isamarkupfalse%
\ RS{\isacharunderscore}{\kern0pt}sequence{\isacharunderscore}{\kern0pt}imp{\isacharunderscore}{\kern0pt}rasiowa{\isacharunderscore}{\kern0pt}sikorski{\isacharcolon}{\kern0pt}\isanewline
\ \ \isakeyword{assumes}\ \isanewline
\ \ \ \ {\isachardoublequoteopen}p{\isasymin}P{\isachardoublequoteclose}\ {\isachardoublequoteopen}f\ {\isacharcolon}{\kern0pt}\ nat{\isasymrightarrow}P{\isachardoublequoteclose}\ {\isachardoublequoteopen}f\ {\isacharbackquote}{\kern0pt}\ {\isadigit{0}}\ {\isacharequal}{\kern0pt}\ p{\isachardoublequoteclose}\isanewline
\ \ \ \ {\isachardoublequoteopen}{\isasymAnd}n{\isachardot}{\kern0pt}\ n{\isasymin}nat\ {\isasymLongrightarrow}\ f\ {\isacharbackquote}{\kern0pt}\ succ{\isacharparenleft}{\kern0pt}n{\isacharparenright}{\kern0pt}{\isasympreceq}\ f\ {\isacharbackquote}{\kern0pt}\ n\ {\isasymand}\ f\ {\isacharbackquote}{\kern0pt}\ succ{\isacharparenleft}{\kern0pt}n{\isacharparenright}{\kern0pt}\ {\isasymin}\ {\isasymD}\ {\isacharbackquote}{\kern0pt}\ n{\isachardoublequoteclose}\ \isanewline
\ \ \isakeyword{shows}\isanewline
\ \ \ \ {\isachardoublequoteopen}{\isasymexists}G{\isachardot}{\kern0pt}\ p{\isasymin}G\ {\isasymand}\ D{\isacharunderscore}{\kern0pt}generic{\isacharparenleft}{\kern0pt}G{\isacharparenright}{\kern0pt}{\isachardoublequoteclose}\isanewline
%
\isadelimproof
%
\endisadelimproof
%
\isatagproof
\isacommand{proof}\isamarkupfalse%
\ {\isacharminus}{\kern0pt}\isanewline
\ \ \isacommand{note}\isamarkupfalse%
\ assms\isanewline
\ \ \isacommand{moreover}\isamarkupfalse%
\ \isacommand{from}\isamarkupfalse%
\ this\ \isanewline
\ \ \isacommand{have}\isamarkupfalse%
\ {\isachardoublequoteopen}f{\isacharbackquote}{\kern0pt}{\isacharbackquote}{\kern0pt}nat\ \ {\isasymsubseteq}\ P{\isachardoublequoteclose}\isanewline
\ \ \ \ \isacommand{by}\isamarkupfalse%
\ {\isacharparenleft}{\kern0pt}simp\ add{\isacharcolon}{\kern0pt}subset{\isacharunderscore}{\kern0pt}fun{\isacharunderscore}{\kern0pt}image{\isacharparenright}{\kern0pt}\isanewline
\ \ \isacommand{moreover}\isamarkupfalse%
\ \isacommand{from}\isamarkupfalse%
\ calculation\isanewline
\ \ \isacommand{have}\isamarkupfalse%
\ {\isachardoublequoteopen}refl{\isacharparenleft}{\kern0pt}f{\isacharbackquote}{\kern0pt}{\isacharbackquote}{\kern0pt}nat{\isacharcomma}{\kern0pt}\ leq{\isacharparenright}{\kern0pt}\ {\isasymand}\ trans{\isacharbrackleft}{\kern0pt}P{\isacharbrackright}{\kern0pt}{\isacharparenleft}{\kern0pt}leq{\isacharparenright}{\kern0pt}{\isachardoublequoteclose}\isanewline
\ \ \ \ \isacommand{using}\isamarkupfalse%
\ leq{\isacharunderscore}{\kern0pt}preord\ \isacommand{unfolding}\isamarkupfalse%
\ preorder{\isacharunderscore}{\kern0pt}on{\isacharunderscore}{\kern0pt}def\ \isacommand{by}\isamarkupfalse%
\ {\isacharparenleft}{\kern0pt}blast\ intro{\isacharcolon}{\kern0pt}refl{\isacharunderscore}{\kern0pt}monot{\isacharunderscore}{\kern0pt}domain{\isacharparenright}{\kern0pt}\isanewline
\ \ \isacommand{moreover}\isamarkupfalse%
\ \isacommand{from}\isamarkupfalse%
\ calculation\ \isanewline
\ \ \isacommand{have}\isamarkupfalse%
\ {\isachardoublequoteopen}{\isasymforall}n{\isasymin}nat{\isachardot}{\kern0pt}\ \ f\ {\isacharbackquote}{\kern0pt}\ succ{\isacharparenleft}{\kern0pt}n{\isacharparenright}{\kern0pt}{\isasympreceq}\ f\ {\isacharbackquote}{\kern0pt}\ n{\isachardoublequoteclose}\ \isacommand{by}\isamarkupfalse%
\ {\isacharparenleft}{\kern0pt}simp{\isacharparenright}{\kern0pt}\isanewline
\ \ \isacommand{moreover}\isamarkupfalse%
\ \isacommand{from}\isamarkupfalse%
\ calculation\isanewline
\ \ \isacommand{have}\isamarkupfalse%
\ {\isachardoublequoteopen}linear{\isacharparenleft}{\kern0pt}f{\isacharbackquote}{\kern0pt}{\isacharbackquote}{\kern0pt}nat{\isacharcomma}{\kern0pt}\ leq{\isacharparenright}{\kern0pt}{\isachardoublequoteclose}\isanewline
\ \ \ \ \isacommand{using}\isamarkupfalse%
\ leq{\isacharunderscore}{\kern0pt}preord\ \isakeyword{and}\ decr{\isacharunderscore}{\kern0pt}seq{\isacharunderscore}{\kern0pt}linear\ \isacommand{unfolding}\isamarkupfalse%
\ preorder{\isacharunderscore}{\kern0pt}on{\isacharunderscore}{\kern0pt}def\ \isacommand{by}\isamarkupfalse%
\ {\isacharparenleft}{\kern0pt}blast{\isacharparenright}{\kern0pt}\isanewline
\ \ \isacommand{moreover}\isamarkupfalse%
\ \isacommand{from}\isamarkupfalse%
\ calculation\isanewline
\ \ \isacommand{have}\isamarkupfalse%
\ {\isachardoublequoteopen}{\isacharparenleft}{\kern0pt}{\isasymforall}p{\isasymin}f{\isacharbackquote}{\kern0pt}{\isacharbackquote}{\kern0pt}nat{\isachardot}{\kern0pt}{\isasymforall}q{\isasymin}f{\isacharbackquote}{\kern0pt}{\isacharbackquote}{\kern0pt}nat{\isachardot}{\kern0pt}\ compat{\isacharunderscore}{\kern0pt}in{\isacharparenleft}{\kern0pt}f{\isacharbackquote}{\kern0pt}{\isacharbackquote}{\kern0pt}nat{\isacharcomma}{\kern0pt}leq{\isacharcomma}{\kern0pt}p{\isacharcomma}{\kern0pt}q{\isacharparenright}{\kern0pt}{\isacharparenright}{\kern0pt}{\isachardoublequoteclose}\ \ \ \ \ \ \ \ \ \ \ \ \ \isanewline
\ \ \ \ \isacommand{using}\isamarkupfalse%
\ chain{\isacharunderscore}{\kern0pt}compat\ \isacommand{by}\isamarkupfalse%
\ {\isacharparenleft}{\kern0pt}auto{\isacharparenright}{\kern0pt}\isanewline
\ \ \isacommand{ultimately}\isamarkupfalse%
\ \ \isanewline
\ \ \isacommand{have}\isamarkupfalse%
\ {\isachardoublequoteopen}filter{\isacharparenleft}{\kern0pt}upclosure{\isacharparenleft}{\kern0pt}f{\isacharbackquote}{\kern0pt}{\isacharbackquote}{\kern0pt}nat{\isacharparenright}{\kern0pt}{\isacharparenright}{\kern0pt}{\isachardoublequoteclose}\ {\isacharparenleft}{\kern0pt}\isakeyword{is}\ {\isachardoublequoteopen}filter{\isacharparenleft}{\kern0pt}{\isacharquery}{\kern0pt}G{\isacharparenright}{\kern0pt}{\isachardoublequoteclose}{\isacharparenright}{\kern0pt}\isanewline
\ \ \ \ \isacommand{using}\isamarkupfalse%
\ closure{\isacharunderscore}{\kern0pt}compat{\isacharunderscore}{\kern0pt}filter\ \isacommand{by}\isamarkupfalse%
\ simp\isanewline
\ \ \isacommand{moreover}\isamarkupfalse%
\isanewline
\ \ \isacommand{have}\isamarkupfalse%
\ {\isachardoublequoteopen}{\isasymforall}n{\isasymin}nat{\isachardot}{\kern0pt}\ {\isasymD}\ {\isacharbackquote}{\kern0pt}\ n\ {\isasyminter}\ {\isacharquery}{\kern0pt}G\ {\isasymnoteq}\ {\isadigit{0}}{\isachardoublequoteclose}\isanewline
\ \ \isacommand{proof}\isamarkupfalse%
\isanewline
\ \ \ \ \isacommand{fix}\isamarkupfalse%
\ n\isanewline
\ \ \ \ \isacommand{assume}\isamarkupfalse%
\ {\isachardoublequoteopen}n{\isasymin}nat{\isachardoublequoteclose}\isanewline
\ \ \ \ \isacommand{with}\isamarkupfalse%
\ assms\ \isanewline
\ \ \ \ \isacommand{have}\isamarkupfalse%
\ {\isachardoublequoteopen}f{\isacharbackquote}{\kern0pt}succ{\isacharparenleft}{\kern0pt}n{\isacharparenright}{\kern0pt}\ {\isasymin}\ {\isacharquery}{\kern0pt}G\ {\isasymand}\ f{\isacharbackquote}{\kern0pt}succ{\isacharparenleft}{\kern0pt}n{\isacharparenright}{\kern0pt}\ {\isasymin}\ {\isasymD}\ {\isacharbackquote}{\kern0pt}\ n{\isachardoublequoteclose}\isanewline
\ \ \ \ \ \ \isacommand{using}\isamarkupfalse%
\ aux{\isacharunderscore}{\kern0pt}RS{\isadigit{1}}\ \isacommand{by}\isamarkupfalse%
\ simp\isanewline
\ \ \ \ \isacommand{then}\isamarkupfalse%
\ \isanewline
\ \ \ \ \isacommand{show}\isamarkupfalse%
\ {\isachardoublequoteopen}{\isasymD}\ {\isacharbackquote}{\kern0pt}\ n\ {\isasyminter}\ {\isacharquery}{\kern0pt}G\ {\isasymnoteq}\ {\isadigit{0}}{\isachardoublequoteclose}\ \ \isacommand{by}\isamarkupfalse%
\ blast\isanewline
\ \ \isacommand{qed}\isamarkupfalse%
\isanewline
\ \ \isacommand{moreover}\isamarkupfalse%
\ \isacommand{from}\isamarkupfalse%
\ assms\ \isanewline
\ \ \isacommand{have}\isamarkupfalse%
\ {\isachardoublequoteopen}p\ {\isasymin}\ {\isacharquery}{\kern0pt}G{\isachardoublequoteclose}\isanewline
\ \ \ \ \isacommand{using}\isamarkupfalse%
\ aux{\isacharunderscore}{\kern0pt}RS{\isadigit{1}}\ \isacommand{by}\isamarkupfalse%
\ auto\isanewline
\ \ \isacommand{ultimately}\isamarkupfalse%
\ \isanewline
\ \ \isacommand{show}\isamarkupfalse%
\ {\isacharquery}{\kern0pt}thesis\ \isacommand{unfolding}\isamarkupfalse%
\ D{\isacharunderscore}{\kern0pt}generic{\isacharunderscore}{\kern0pt}def\ \isacommand{by}\isamarkupfalse%
\ auto\isanewline
\isacommand{qed}\isamarkupfalse%
%
\endisatagproof
{\isafoldproof}%
%
\isadelimproof
\isanewline
%
\endisadelimproof
\isanewline
\isacommand{end}\isamarkupfalse%
%
\begin{isamarkuptext}%
Now, the following recursive definition will fulfill the 
requirements of lemma \isa{RS{\isacharunderscore}{\kern0pt}sequence{\isacharunderscore}{\kern0pt}imp{\isacharunderscore}{\kern0pt}rasiowa{\isacharunderscore}{\kern0pt}sikorski}%
\end{isamarkuptext}\isamarkuptrue%
\isacommand{consts}\isamarkupfalse%
\ RS{\isacharunderscore}{\kern0pt}seq\ {\isacharcolon}{\kern0pt}{\isacharcolon}{\kern0pt}\ {\isachardoublequoteopen}{\isacharbrackleft}{\kern0pt}i{\isacharcomma}{\kern0pt}i{\isacharcomma}{\kern0pt}i{\isacharcomma}{\kern0pt}i{\isacharcomma}{\kern0pt}i{\isacharcomma}{\kern0pt}i{\isacharbrackright}{\kern0pt}\ {\isasymRightarrow}\ i{\isachardoublequoteclose}\isanewline
\isacommand{primrec}\isamarkupfalse%
\isanewline
\ \ {\isachardoublequoteopen}RS{\isacharunderscore}{\kern0pt}seq{\isacharparenleft}{\kern0pt}{\isadigit{0}}{\isacharcomma}{\kern0pt}P{\isacharcomma}{\kern0pt}leq{\isacharcomma}{\kern0pt}p{\isacharcomma}{\kern0pt}enum{\isacharcomma}{\kern0pt}{\isasymD}{\isacharparenright}{\kern0pt}\ {\isacharequal}{\kern0pt}\ p{\isachardoublequoteclose}\isanewline
\ \ {\isachardoublequoteopen}RS{\isacharunderscore}{\kern0pt}seq{\isacharparenleft}{\kern0pt}succ{\isacharparenleft}{\kern0pt}n{\isacharparenright}{\kern0pt}{\isacharcomma}{\kern0pt}P{\isacharcomma}{\kern0pt}leq{\isacharcomma}{\kern0pt}p{\isacharcomma}{\kern0pt}enum{\isacharcomma}{\kern0pt}{\isasymD}{\isacharparenright}{\kern0pt}\ {\isacharequal}{\kern0pt}\ \isanewline
\ \ \ \ enum{\isacharbackquote}{\kern0pt}{\isacharparenleft}{\kern0pt}{\isasymmu}\ m{\isachardot}{\kern0pt}\ {\isasymlangle}enum{\isacharbackquote}{\kern0pt}m{\isacharcomma}{\kern0pt}\ RS{\isacharunderscore}{\kern0pt}seq{\isacharparenleft}{\kern0pt}n{\isacharcomma}{\kern0pt}P{\isacharcomma}{\kern0pt}leq{\isacharcomma}{\kern0pt}p{\isacharcomma}{\kern0pt}enum{\isacharcomma}{\kern0pt}{\isasymD}{\isacharparenright}{\kern0pt}{\isasymrangle}\ {\isasymin}\ leq\ {\isasymand}\ enum{\isacharbackquote}{\kern0pt}m\ {\isasymin}\ {\isasymD}\ {\isacharbackquote}{\kern0pt}\ n{\isacharparenright}{\kern0pt}{\isachardoublequoteclose}\isanewline
\isanewline
\isacommand{context}\isamarkupfalse%
\ countable{\isacharunderscore}{\kern0pt}generic\isanewline
\isakeyword{begin}\isanewline
\isanewline
\isacommand{lemma}\isamarkupfalse%
\ preimage{\isacharunderscore}{\kern0pt}rangeD{\isacharcolon}{\kern0pt}\isanewline
\ \ \isakeyword{assumes}\ {\isachardoublequoteopen}f{\isasymin}Pi{\isacharparenleft}{\kern0pt}A{\isacharcomma}{\kern0pt}B{\isacharparenright}{\kern0pt}{\isachardoublequoteclose}\ {\isachardoublequoteopen}b\ {\isasymin}\ range{\isacharparenleft}{\kern0pt}f{\isacharparenright}{\kern0pt}{\isachardoublequoteclose}\ \isanewline
\ \ \isakeyword{shows}\ {\isachardoublequoteopen}{\isasymexists}a{\isasymin}A{\isachardot}{\kern0pt}\ f{\isacharbackquote}{\kern0pt}a\ {\isacharequal}{\kern0pt}\ b{\isachardoublequoteclose}\isanewline
%
\isadelimproof
\ \ %
\endisadelimproof
%
\isatagproof
\isacommand{using}\isamarkupfalse%
\ assms\ apply{\isacharunderscore}{\kern0pt}equality{\isacharbrackleft}{\kern0pt}OF\ {\isacharunderscore}{\kern0pt}\ assms{\isacharparenleft}{\kern0pt}{\isadigit{1}}{\isacharparenright}{\kern0pt}{\isacharcomma}{\kern0pt}\ of\ {\isacharunderscore}{\kern0pt}\ b{\isacharbrackright}{\kern0pt}\ domain{\isacharunderscore}{\kern0pt}type{\isacharbrackleft}{\kern0pt}OF\ {\isacharunderscore}{\kern0pt}\ assms{\isacharparenleft}{\kern0pt}{\isadigit{1}}{\isacharparenright}{\kern0pt}{\isacharbrackright}{\kern0pt}\ \isacommand{by}\isamarkupfalse%
\ auto%
\endisatagproof
{\isafoldproof}%
%
\isadelimproof
\isanewline
%
\endisadelimproof
\isanewline
\isacommand{lemma}\isamarkupfalse%
\ countable{\isacharunderscore}{\kern0pt}RS{\isacharunderscore}{\kern0pt}sequence{\isacharunderscore}{\kern0pt}aux{\isacharcolon}{\kern0pt}\isanewline
\ \ \isakeyword{fixes}\ p\ enum\isanewline
\ \ \isakeyword{defines}\ {\isachardoublequoteopen}f{\isacharparenleft}{\kern0pt}n{\isacharparenright}{\kern0pt}\ {\isasymequiv}\ RS{\isacharunderscore}{\kern0pt}seq{\isacharparenleft}{\kern0pt}n{\isacharcomma}{\kern0pt}P{\isacharcomma}{\kern0pt}leq{\isacharcomma}{\kern0pt}p{\isacharcomma}{\kern0pt}enum{\isacharcomma}{\kern0pt}{\isasymD}{\isacharparenright}{\kern0pt}{\isachardoublequoteclose}\isanewline
\ \ \ \ \isakeyword{and}\ \ \ {\isachardoublequoteopen}Q{\isacharparenleft}{\kern0pt}q{\isacharcomma}{\kern0pt}k{\isacharcomma}{\kern0pt}m{\isacharparenright}{\kern0pt}\ {\isasymequiv}\ enum{\isacharbackquote}{\kern0pt}m{\isasympreceq}\ q\ {\isasymand}\ enum{\isacharbackquote}{\kern0pt}m\ {\isasymin}\ {\isasymD}\ {\isacharbackquote}{\kern0pt}\ k{\isachardoublequoteclose}\isanewline
\ \ \isakeyword{assumes}\ {\isachardoublequoteopen}n{\isasymin}nat{\isachardoublequoteclose}\ {\isachardoublequoteopen}p{\isasymin}P{\isachardoublequoteclose}\ {\isachardoublequoteopen}P\ {\isasymsubseteq}\ range{\isacharparenleft}{\kern0pt}enum{\isacharparenright}{\kern0pt}{\isachardoublequoteclose}\ {\isachardoublequoteopen}enum{\isacharcolon}{\kern0pt}nat{\isasymrightarrow}M{\isachardoublequoteclose}\isanewline
\ \ \ \ {\isachardoublequoteopen}{\isasymAnd}x\ k{\isachardot}{\kern0pt}\ x{\isasymin}P\ {\isasymLongrightarrow}\ k{\isasymin}nat\ {\isasymLongrightarrow}\ \ {\isasymexists}q{\isasymin}P{\isachardot}{\kern0pt}\ q{\isasympreceq}\ x\ {\isasymand}\ q\ {\isasymin}\ {\isasymD}\ {\isacharbackquote}{\kern0pt}\ k{\isachardoublequoteclose}\ \isanewline
\ \ \isakeyword{shows}\ \isanewline
\ \ \ \ {\isachardoublequoteopen}f{\isacharparenleft}{\kern0pt}succ{\isacharparenleft}{\kern0pt}n{\isacharparenright}{\kern0pt}{\isacharparenright}{\kern0pt}\ {\isasymin}\ P\ {\isasymand}\ f{\isacharparenleft}{\kern0pt}succ{\isacharparenleft}{\kern0pt}n{\isacharparenright}{\kern0pt}{\isacharparenright}{\kern0pt}{\isasympreceq}\ f{\isacharparenleft}{\kern0pt}n{\isacharparenright}{\kern0pt}\ {\isasymand}\ f{\isacharparenleft}{\kern0pt}succ{\isacharparenleft}{\kern0pt}n{\isacharparenright}{\kern0pt}{\isacharparenright}{\kern0pt}\ {\isasymin}\ {\isasymD}\ {\isacharbackquote}{\kern0pt}\ n{\isachardoublequoteclose}\isanewline
%
\isadelimproof
\ \ %
\endisadelimproof
%
\isatagproof
\isacommand{using}\isamarkupfalse%
\ {\isacartoucheopen}n{\isasymin}nat{\isacartoucheclose}\isanewline
\isacommand{proof}\isamarkupfalse%
\ {\isacharparenleft}{\kern0pt}induct{\isacharparenright}{\kern0pt}\isanewline
\ \ \isacommand{case}\isamarkupfalse%
\ {\isadigit{0}}\isanewline
\ \ \isacommand{from}\isamarkupfalse%
\ assms\ \isanewline
\ \ \isacommand{obtain}\isamarkupfalse%
\ q\ \isakeyword{where}\ {\isachardoublequoteopen}q{\isasymin}P{\isachardoublequoteclose}\ {\isachardoublequoteopen}q{\isasympreceq}\ p{\isachardoublequoteclose}\ {\isachardoublequoteopen}q\ {\isasymin}\ {\isasymD}\ {\isacharbackquote}{\kern0pt}\ {\isadigit{0}}{\isachardoublequoteclose}\ \isacommand{by}\isamarkupfalse%
\ blast\isanewline
\ \ \isacommand{moreover}\isamarkupfalse%
\ \isacommand{from}\isamarkupfalse%
\ this\ \isakeyword{and}\ {\isacartoucheopen}P\ {\isasymsubseteq}\ range{\isacharparenleft}{\kern0pt}enum{\isacharparenright}{\kern0pt}{\isacartoucheclose}\isanewline
\ \ \isacommand{obtain}\isamarkupfalse%
\ m\ \isakeyword{where}\ {\isachardoublequoteopen}m{\isasymin}nat{\isachardoublequoteclose}\ {\isachardoublequoteopen}enum{\isacharbackquote}{\kern0pt}m\ {\isacharequal}{\kern0pt}\ q{\isachardoublequoteclose}\ \isanewline
\ \ \ \ \isacommand{using}\isamarkupfalse%
\ preimage{\isacharunderscore}{\kern0pt}rangeD{\isacharbrackleft}{\kern0pt}OF\ {\isacartoucheopen}enum{\isacharcolon}{\kern0pt}nat{\isasymrightarrow}M{\isacartoucheclose}{\isacharbrackright}{\kern0pt}\ \isacommand{by}\isamarkupfalse%
\ blast\isanewline
\ \ \isacommand{moreover}\isamarkupfalse%
\ \isanewline
\ \ \isacommand{have}\isamarkupfalse%
\ {\isachardoublequoteopen}{\isasymD}{\isacharbackquote}{\kern0pt}{\isadigit{0}}\ {\isasymsubseteq}\ P{\isachardoublequoteclose}\isanewline
\ \ \ \ \isacommand{using}\isamarkupfalse%
\ apply{\isacharunderscore}{\kern0pt}funtype{\isacharbrackleft}{\kern0pt}OF\ countable{\isacharunderscore}{\kern0pt}subs{\isacharunderscore}{\kern0pt}of{\isacharunderscore}{\kern0pt}P{\isacharbrackright}{\kern0pt}\ \isacommand{by}\isamarkupfalse%
\ simp\isanewline
\ \ \isacommand{moreover}\isamarkupfalse%
\ \isacommand{note}\isamarkupfalse%
\ {\isacartoucheopen}p{\isasymin}P{\isacartoucheclose}\isanewline
\ \ \isacommand{ultimately}\isamarkupfalse%
\isanewline
\ \ \isacommand{show}\isamarkupfalse%
\ {\isacharquery}{\kern0pt}case\ \isanewline
\ \ \ \ \isacommand{using}\isamarkupfalse%
\ LeastI{\isacharbrackleft}{\kern0pt}of\ {\isachardoublequoteopen}Q{\isacharparenleft}{\kern0pt}p{\isacharcomma}{\kern0pt}{\isadigit{0}}{\isacharparenright}{\kern0pt}{\isachardoublequoteclose}\ m{\isacharbrackright}{\kern0pt}\ \isacommand{unfolding}\isamarkupfalse%
\ Q{\isacharunderscore}{\kern0pt}def\ f{\isacharunderscore}{\kern0pt}def\ \isacommand{by}\isamarkupfalse%
\ auto\isanewline
\isacommand{next}\isamarkupfalse%
\isanewline
\ \ \isacommand{case}\isamarkupfalse%
\ {\isacharparenleft}{\kern0pt}succ\ n{\isacharparenright}{\kern0pt}\isanewline
\ \ \isacommand{with}\isamarkupfalse%
\ assms\ \isanewline
\ \ \isacommand{obtain}\isamarkupfalse%
\ q\ \isakeyword{where}\ {\isachardoublequoteopen}q{\isasymin}P{\isachardoublequoteclose}\ {\isachardoublequoteopen}q{\isasympreceq}\ f{\isacharparenleft}{\kern0pt}succ{\isacharparenleft}{\kern0pt}n{\isacharparenright}{\kern0pt}{\isacharparenright}{\kern0pt}{\isachardoublequoteclose}\ {\isachardoublequoteopen}q\ {\isasymin}\ {\isasymD}\ {\isacharbackquote}{\kern0pt}\ succ{\isacharparenleft}{\kern0pt}n{\isacharparenright}{\kern0pt}{\isachardoublequoteclose}\ \isacommand{by}\isamarkupfalse%
\ blast\isanewline
\ \ \isacommand{moreover}\isamarkupfalse%
\ \isacommand{from}\isamarkupfalse%
\ this\ \isakeyword{and}\ {\isacartoucheopen}P\ {\isasymsubseteq}\ range{\isacharparenleft}{\kern0pt}enum{\isacharparenright}{\kern0pt}{\isacartoucheclose}\isanewline
\ \ \isacommand{obtain}\isamarkupfalse%
\ m\ \isakeyword{where}\ {\isachardoublequoteopen}m{\isasymin}nat{\isachardoublequoteclose}\ {\isachardoublequoteopen}enum{\isacharbackquote}{\kern0pt}m{\isasympreceq}\ f{\isacharparenleft}{\kern0pt}succ{\isacharparenleft}{\kern0pt}n{\isacharparenright}{\kern0pt}{\isacharparenright}{\kern0pt}{\isachardoublequoteclose}\ {\isachardoublequoteopen}enum{\isacharbackquote}{\kern0pt}m\ {\isasymin}\ {\isasymD}\ {\isacharbackquote}{\kern0pt}\ succ{\isacharparenleft}{\kern0pt}n{\isacharparenright}{\kern0pt}{\isachardoublequoteclose}\isanewline
\ \ \ \ \isacommand{using}\isamarkupfalse%
\ preimage{\isacharunderscore}{\kern0pt}rangeD{\isacharbrackleft}{\kern0pt}OF\ {\isacartoucheopen}enum{\isacharcolon}{\kern0pt}nat{\isasymrightarrow}M{\isacartoucheclose}{\isacharbrackright}{\kern0pt}\ \isacommand{by}\isamarkupfalse%
\ blast\isanewline
\ \ \isacommand{moreover}\isamarkupfalse%
\ \isacommand{note}\isamarkupfalse%
\ succ\isanewline
\ \ \isacommand{moreover}\isamarkupfalse%
\ \isacommand{from}\isamarkupfalse%
\ calculation\isanewline
\ \ \isacommand{have}\isamarkupfalse%
\ {\isachardoublequoteopen}{\isasymD}{\isacharbackquote}{\kern0pt}succ{\isacharparenleft}{\kern0pt}n{\isacharparenright}{\kern0pt}\ {\isasymsubseteq}\ P{\isachardoublequoteclose}\ \isanewline
\ \ \ \ \isacommand{using}\isamarkupfalse%
\ apply{\isacharunderscore}{\kern0pt}funtype{\isacharbrackleft}{\kern0pt}OF\ countable{\isacharunderscore}{\kern0pt}subs{\isacharunderscore}{\kern0pt}of{\isacharunderscore}{\kern0pt}P{\isacharbrackright}{\kern0pt}\ \isacommand{by}\isamarkupfalse%
\ auto\isanewline
\ \ \isacommand{ultimately}\isamarkupfalse%
\isanewline
\ \ \isacommand{show}\isamarkupfalse%
\ {\isacharquery}{\kern0pt}case\isanewline
\ \ \ \ \isacommand{using}\isamarkupfalse%
\ LeastI{\isacharbrackleft}{\kern0pt}of\ {\isachardoublequoteopen}Q{\isacharparenleft}{\kern0pt}f{\isacharparenleft}{\kern0pt}succ{\isacharparenleft}{\kern0pt}n{\isacharparenright}{\kern0pt}{\isacharparenright}{\kern0pt}{\isacharcomma}{\kern0pt}succ{\isacharparenleft}{\kern0pt}n{\isacharparenright}{\kern0pt}{\isacharparenright}{\kern0pt}{\isachardoublequoteclose}\ m{\isacharbrackright}{\kern0pt}\ \isacommand{unfolding}\isamarkupfalse%
\ Q{\isacharunderscore}{\kern0pt}def\ f{\isacharunderscore}{\kern0pt}def\ \isacommand{by}\isamarkupfalse%
\ auto\isanewline
\isacommand{qed}\isamarkupfalse%
%
\endisatagproof
{\isafoldproof}%
%
\isadelimproof
\isanewline
%
\endisadelimproof
\isanewline
\isacommand{lemma}\isamarkupfalse%
\ countable{\isacharunderscore}{\kern0pt}RS{\isacharunderscore}{\kern0pt}sequence{\isacharcolon}{\kern0pt}\isanewline
\ \ \isakeyword{fixes}\ p\ enum\isanewline
\ \ \isakeyword{defines}\ {\isachardoublequoteopen}f\ {\isasymequiv}\ {\isasymlambda}n{\isasymin}nat{\isachardot}{\kern0pt}\ RS{\isacharunderscore}{\kern0pt}seq{\isacharparenleft}{\kern0pt}n{\isacharcomma}{\kern0pt}P{\isacharcomma}{\kern0pt}leq{\isacharcomma}{\kern0pt}p{\isacharcomma}{\kern0pt}enum{\isacharcomma}{\kern0pt}{\isasymD}{\isacharparenright}{\kern0pt}{\isachardoublequoteclose}\isanewline
\ \ \ \ \isakeyword{and}\ \ \ {\isachardoublequoteopen}Q{\isacharparenleft}{\kern0pt}q{\isacharcomma}{\kern0pt}k{\isacharcomma}{\kern0pt}m{\isacharparenright}{\kern0pt}\ {\isasymequiv}\ enum{\isacharbackquote}{\kern0pt}m{\isasympreceq}\ q\ {\isasymand}\ enum{\isacharbackquote}{\kern0pt}m\ {\isasymin}\ {\isasymD}\ {\isacharbackquote}{\kern0pt}\ k{\isachardoublequoteclose}\isanewline
\ \ \isakeyword{assumes}\ {\isachardoublequoteopen}n{\isasymin}nat{\isachardoublequoteclose}\ {\isachardoublequoteopen}p{\isasymin}P{\isachardoublequoteclose}\ {\isachardoublequoteopen}P\ {\isasymsubseteq}\ range{\isacharparenleft}{\kern0pt}enum{\isacharparenright}{\kern0pt}{\isachardoublequoteclose}\ {\isachardoublequoteopen}enum{\isacharcolon}{\kern0pt}nat{\isasymrightarrow}M{\isachardoublequoteclose}\isanewline
\ \ \isakeyword{shows}\ \isanewline
\ \ \ \ {\isachardoublequoteopen}f{\isacharbackquote}{\kern0pt}{\isadigit{0}}\ {\isacharequal}{\kern0pt}\ p{\isachardoublequoteclose}\ {\isachardoublequoteopen}f{\isacharbackquote}{\kern0pt}succ{\isacharparenleft}{\kern0pt}n{\isacharparenright}{\kern0pt}{\isasympreceq}\ f{\isacharbackquote}{\kern0pt}n\ {\isasymand}\ f{\isacharbackquote}{\kern0pt}succ{\isacharparenleft}{\kern0pt}n{\isacharparenright}{\kern0pt}\ {\isasymin}\ {\isasymD}\ {\isacharbackquote}{\kern0pt}\ n{\isachardoublequoteclose}\ {\isachardoublequoteopen}f{\isacharbackquote}{\kern0pt}succ{\isacharparenleft}{\kern0pt}n{\isacharparenright}{\kern0pt}\ {\isasymin}\ P{\isachardoublequoteclose}\isanewline
%
\isadelimproof
%
\endisadelimproof
%
\isatagproof
\isacommand{proof}\isamarkupfalse%
\ {\isacharminus}{\kern0pt}\isanewline
\ \ \isacommand{from}\isamarkupfalse%
\ assms\isanewline
\ \ \isacommand{show}\isamarkupfalse%
\ {\isachardoublequoteopen}f{\isacharbackquote}{\kern0pt}{\isadigit{0}}\ {\isacharequal}{\kern0pt}\ p{\isachardoublequoteclose}\ \isacommand{by}\isamarkupfalse%
\ simp\isanewline
\ \ \isacommand{{\isacharbraceleft}{\kern0pt}}\isamarkupfalse%
\isanewline
\ \ \ \ \isacommand{fix}\isamarkupfalse%
\ x\ k\isanewline
\ \ \ \ \isacommand{assume}\isamarkupfalse%
\ {\isachardoublequoteopen}x{\isasymin}P{\isachardoublequoteclose}\ {\isachardoublequoteopen}k{\isasymin}nat{\isachardoublequoteclose}\isanewline
\ \ \ \ \isacommand{then}\isamarkupfalse%
\isanewline
\ \ \ \ \isacommand{have}\isamarkupfalse%
\ {\isachardoublequoteopen}{\isasymexists}q{\isasymin}P{\isachardot}{\kern0pt}\ q{\isasympreceq}\ x\ {\isasymand}\ q\ {\isasymin}\ {\isasymD}\ {\isacharbackquote}{\kern0pt}\ k{\isachardoublequoteclose}\isanewline
\ \ \ \ \ \ \isacommand{using}\isamarkupfalse%
\ seq{\isacharunderscore}{\kern0pt}of{\isacharunderscore}{\kern0pt}denses\ apply{\isacharunderscore}{\kern0pt}funtype{\isacharbrackleft}{\kern0pt}OF\ countable{\isacharunderscore}{\kern0pt}subs{\isacharunderscore}{\kern0pt}of{\isacharunderscore}{\kern0pt}P{\isacharbrackright}{\kern0pt}\ \isanewline
\ \ \ \ \ \ \isacommand{unfolding}\isamarkupfalse%
\ dense{\isacharunderscore}{\kern0pt}def\ \isacommand{by}\isamarkupfalse%
\ blast\isanewline
\ \ \isacommand{{\isacharbraceright}{\kern0pt}}\isamarkupfalse%
\isanewline
\ \ \isacommand{with}\isamarkupfalse%
\ assms\isanewline
\ \ \isacommand{show}\isamarkupfalse%
\ {\isachardoublequoteopen}f{\isacharbackquote}{\kern0pt}succ{\isacharparenleft}{\kern0pt}n{\isacharparenright}{\kern0pt}{\isasympreceq}\ f{\isacharbackquote}{\kern0pt}n\ {\isasymand}\ f{\isacharbackquote}{\kern0pt}succ{\isacharparenleft}{\kern0pt}n{\isacharparenright}{\kern0pt}\ {\isasymin}\ {\isasymD}\ {\isacharbackquote}{\kern0pt}\ n{\isachardoublequoteclose}\ {\isachardoublequoteopen}f{\isacharbackquote}{\kern0pt}succ{\isacharparenleft}{\kern0pt}n{\isacharparenright}{\kern0pt}{\isasymin}P{\isachardoublequoteclose}\isanewline
\ \ \ \ \isacommand{unfolding}\isamarkupfalse%
\ f{\isacharunderscore}{\kern0pt}def\ \isacommand{using}\isamarkupfalse%
\ countable{\isacharunderscore}{\kern0pt}RS{\isacharunderscore}{\kern0pt}sequence{\isacharunderscore}{\kern0pt}aux\ \isacommand{by}\isamarkupfalse%
\ simp{\isacharunderscore}{\kern0pt}all\isanewline
\isacommand{qed}\isamarkupfalse%
%
\endisatagproof
{\isafoldproof}%
%
\isadelimproof
\isanewline
%
\endisadelimproof
\isanewline
\isacommand{lemma}\isamarkupfalse%
\ RS{\isacharunderscore}{\kern0pt}seq{\isacharunderscore}{\kern0pt}type{\isacharcolon}{\kern0pt}\ \isanewline
\ \ \isakeyword{assumes}\ {\isachardoublequoteopen}n\ {\isasymin}\ nat{\isachardoublequoteclose}\ {\isachardoublequoteopen}p{\isasymin}P{\isachardoublequoteclose}\ {\isachardoublequoteopen}P\ {\isasymsubseteq}\ range{\isacharparenleft}{\kern0pt}enum{\isacharparenright}{\kern0pt}{\isachardoublequoteclose}\ {\isachardoublequoteopen}enum{\isacharcolon}{\kern0pt}nat{\isasymrightarrow}M{\isachardoublequoteclose}\isanewline
\ \ \isakeyword{shows}\ {\isachardoublequoteopen}RS{\isacharunderscore}{\kern0pt}seq{\isacharparenleft}{\kern0pt}n{\isacharcomma}{\kern0pt}P{\isacharcomma}{\kern0pt}leq{\isacharcomma}{\kern0pt}p{\isacharcomma}{\kern0pt}enum{\isacharcomma}{\kern0pt}{\isasymD}{\isacharparenright}{\kern0pt}\ {\isasymin}\ P{\isachardoublequoteclose}\isanewline
%
\isadelimproof
\ \ %
\endisadelimproof
%
\isatagproof
\isacommand{using}\isamarkupfalse%
\ assms\ countable{\isacharunderscore}{\kern0pt}RS{\isacharunderscore}{\kern0pt}sequence{\isacharparenleft}{\kern0pt}{\isadigit{1}}{\isacharcomma}{\kern0pt}{\isadigit{3}}{\isacharparenright}{\kern0pt}\ \ \isanewline
\ \ \isacommand{by}\isamarkupfalse%
\ {\isacharparenleft}{\kern0pt}induct{\isacharsemicolon}{\kern0pt}simp{\isacharparenright}{\kern0pt}%
\endisatagproof
{\isafoldproof}%
%
\isadelimproof
\ \isanewline
%
\endisadelimproof
\isanewline
\isacommand{lemma}\isamarkupfalse%
\ RS{\isacharunderscore}{\kern0pt}seq{\isacharunderscore}{\kern0pt}funtype{\isacharcolon}{\kern0pt}\isanewline
\ \ \isakeyword{assumes}\ {\isachardoublequoteopen}p{\isasymin}P{\isachardoublequoteclose}\ {\isachardoublequoteopen}P\ {\isasymsubseteq}\ range{\isacharparenleft}{\kern0pt}enum{\isacharparenright}{\kern0pt}{\isachardoublequoteclose}\ {\isachardoublequoteopen}enum{\isacharcolon}{\kern0pt}nat{\isasymrightarrow}M{\isachardoublequoteclose}\isanewline
\ \ \isakeyword{shows}\ {\isachardoublequoteopen}{\isacharparenleft}{\kern0pt}{\isasymlambda}n{\isasymin}nat{\isachardot}{\kern0pt}\ RS{\isacharunderscore}{\kern0pt}seq{\isacharparenleft}{\kern0pt}n{\isacharcomma}{\kern0pt}P{\isacharcomma}{\kern0pt}leq{\isacharcomma}{\kern0pt}p{\isacharcomma}{\kern0pt}enum{\isacharcomma}{\kern0pt}{\isasymD}{\isacharparenright}{\kern0pt}{\isacharparenright}{\kern0pt}{\isacharcolon}{\kern0pt}\ nat\ {\isasymrightarrow}\ P{\isachardoublequoteclose}\isanewline
%
\isadelimproof
\ \ %
\endisadelimproof
%
\isatagproof
\isacommand{using}\isamarkupfalse%
\ assms\ lam{\isacharunderscore}{\kern0pt}type\ RS{\isacharunderscore}{\kern0pt}seq{\isacharunderscore}{\kern0pt}type\ \isacommand{by}\isamarkupfalse%
\ auto%
\endisatagproof
{\isafoldproof}%
%
\isadelimproof
\isanewline
%
\endisadelimproof
\isanewline
\isacommand{lemmas}\isamarkupfalse%
\ countable{\isacharunderscore}{\kern0pt}rasiowa{\isacharunderscore}{\kern0pt}sikorski\ {\isacharequal}{\kern0pt}\ \isanewline
\ \ RS{\isacharunderscore}{\kern0pt}sequence{\isacharunderscore}{\kern0pt}imp{\isacharunderscore}{\kern0pt}rasiowa{\isacharunderscore}{\kern0pt}sikorski{\isacharbrackleft}{\kern0pt}OF\ {\isacharunderscore}{\kern0pt}\ RS{\isacharunderscore}{\kern0pt}seq{\isacharunderscore}{\kern0pt}funtype\ countable{\isacharunderscore}{\kern0pt}RS{\isacharunderscore}{\kern0pt}sequence{\isacharparenleft}{\kern0pt}{\isadigit{1}}{\isacharcomma}{\kern0pt}{\isadigit{2}}{\isacharparenright}{\kern0pt}{\isacharbrackright}{\kern0pt}\isanewline
\isacommand{end}\isamarkupfalse%
\ \isanewline
%
\isadelimtheory
\isanewline
%
\endisadelimtheory
%
\isatagtheory
\isacommand{end}\isamarkupfalse%
%
\endisatagtheory
{\isafoldtheory}%
%
\isadelimtheory
%
\endisadelimtheory
%
\end{isabellebody}%
\endinput
%:%file=~/source/repos/ZF-notAC/code/Forcing/Forcing_Notions.thy%:%
%:%11=1%:%
%:%23=2%:%
%:%24=3%:%
%:%32=5%:%
%:%33=5%:%
%:%34=6%:%
%:%35=7%:%
%:%49=9%:%
%:%61=10%:%
%:%62=11%:%
%:%64=12%:%
%:%65=12%:%
%:%66=13%:%
%:%67=14%:%
%:%68=15%:%
%:%69=15%:%
%:%70=16%:%
%:%71=17%:%
%:%72=18%:%
%:%73=19%:%
%:%74=20%:%
%:%75=20%:%
%:%76=21%:%
%:%79=22%:%
%:%83=22%:%
%:%84=22%:%
%:%89=22%:%
%:%92=23%:%
%:%93=24%:%
%:%94=24%:%
%:%95=25%:%
%:%98=26%:%
%:%102=26%:%
%:%103=26%:%
%:%108=26%:%
%:%111=27%:%
%:%112=28%:%
%:%113=28%:%
%:%114=29%:%
%:%117=30%:%
%:%121=30%:%
%:%122=30%:%
%:%127=30%:%
%:%130=31%:%
%:%131=32%:%
%:%132=32%:%
%:%135=33%:%
%:%139=33%:%
%:%140=33%:%
%:%145=33%:%
%:%148=34%:%
%:%149=35%:%
%:%150=35%:%
%:%153=36%:%
%:%157=36%:%
%:%158=36%:%
%:%159=36%:%
%:%164=36%:%
%:%167=37%:%
%:%168=38%:%
%:%169=38%:%
%:%170=39%:%
%:%171=40%:%
%:%172=41%:%
%:%173=42%:%
%:%174=42%:%
%:%175=43%:%
%:%176=44%:%
%:%177=45%:%
%:%178=46%:%
%:%179=46%:%
%:%180=47%:%
%:%181=48%:%
%:%182=49%:%
%:%183=50%:%
%:%184=51%:%
%:%185=52%:%
%:%186=53%:%
%:%187=53%:%
%:%188=54%:%
%:%189=55%:%
%:%190=56%:%
%:%191=56%:%
%:%192=57%:%
%:%195=58%:%
%:%199=58%:%
%:%200=58%:%
%:%201=58%:%
%:%202=58%:%
%:%211=60%:%
%:%212=61%:%
%:%214=62%:%
%:%215=62%:%
%:%216=63%:%
%:%217=64%:%
%:%219=66%:%
%:%220=67%:%
%:%221=68%:%
%:%223=69%:%
%:%224=69%:%
%:%225=70%:%
%:%226=71%:%
%:%227=72%:%
%:%228=73%:%
%:%229=73%:%
%:%232=74%:%
%:%236=74%:%
%:%237=74%:%
%:%242=74%:%
%:%245=75%:%
%:%246=76%:%
%:%247=76%:%
%:%248=77%:%
%:%249=78%:%
%:%250=79%:%
%:%251=80%:%
%:%252=80%:%
%:%253=81%:%
%:%254=82%:%
%:%255=83%:%
%:%256=84%:%
%:%257=84%:%
%:%260=85%:%
%:%264=85%:%
%:%265=85%:%
%:%266=85%:%
%:%267=85%:%
%:%272=85%:%
%:%275=86%:%
%:%276=87%:%
%:%277=87%:%
%:%280=88%:%
%:%284=88%:%
%:%285=88%:%
%:%286=88%:%
%:%287=88%:%
%:%292=88%:%
%:%295=89%:%
%:%296=90%:%
%:%297=91%:%
%:%298=91%:%
%:%301=92%:%
%:%305=92%:%
%:%306=92%:%
%:%307=92%:%
%:%308=92%:%
%:%313=92%:%
%:%316=93%:%
%:%317=94%:%
%:%318=94%:%
%:%321=95%:%
%:%325=95%:%
%:%326=95%:%
%:%332=95%:%
%:%335=96%:%
%:%336=97%:%
%:%337=97%:%
%:%338=98%:%
%:%339=99%:%
%:%340=100%:%
%:%341=100%:%
%:%344=101%:%
%:%348=101%:%
%:%349=101%:%
%:%350=101%:%
%:%355=101%:%
%:%358=102%:%
%:%359=103%:%
%:%360=103%:%
%:%363=104%:%
%:%367=104%:%
%:%368=104%:%
%:%369=104%:%
%:%374=104%:%
%:%377=105%:%
%:%378=106%:%
%:%379=106%:%
%:%382=107%:%
%:%386=107%:%
%:%387=107%:%
%:%388=107%:%
%:%393=107%:%
%:%396=108%:%
%:%397=109%:%
%:%398=109%:%
%:%399=110%:%
%:%400=111%:%
%:%403=112%:%
%:%407=112%:%
%:%408=112%:%
%:%409=112%:%
%:%410=112%:%
%:%415=112%:%
%:%418=113%:%
%:%418=114%:%
%:%419=115%:%
%:%420=116%:%
%:%421=116%:%
%:%422=117%:%
%:%423=118%:%
%:%426=119%:%
%:%430=119%:%
%:%431=119%:%
%:%432=119%:%
%:%433=119%:%
%:%438=119%:%
%:%441=120%:%
%:%442=121%:%
%:%443=121%:%
%:%446=122%:%
%:%450=122%:%
%:%451=122%:%
%:%456=122%:%
%:%459=123%:%
%:%460=124%:%
%:%461=124%:%
%:%462=125%:%
%:%465=126%:%
%:%469=126%:%
%:%470=126%:%
%:%475=126%:%
%:%478=127%:%
%:%479=128%:%
%:%480=128%:%
%:%483=129%:%
%:%487=129%:%
%:%488=129%:%
%:%493=129%:%
%:%496=130%:%
%:%497=131%:%
%:%498=131%:%
%:%499=132%:%
%:%500=133%:%
%:%503=134%:%
%:%507=134%:%
%:%508=134%:%
%:%509=134%:%
%:%514=134%:%
%:%517=135%:%
%:%518=136%:%
%:%519=136%:%
%:%520=137%:%
%:%521=138%:%
%:%524=139%:%
%:%528=139%:%
%:%529=139%:%
%:%530=139%:%
%:%535=139%:%
%:%538=140%:%
%:%539=141%:%
%:%540=141%:%
%:%541=142%:%
%:%542=143%:%
%:%545=144%:%
%:%549=144%:%
%:%550=144%:%
%:%551=144%:%
%:%556=144%:%
%:%559=145%:%
%:%560=146%:%
%:%560=161%:%
%:%561=162%:%
%:%562=163%:%
%:%563=163%:%
%:%564=164%:%
%:%565=165%:%
%:%567=167%:%
%:%568=168%:%
%:%570=169%:%
%:%571=169%:%
%:%572=170%:%
%:%573=171%:%
%:%574=172%:%
%:%575=173%:%
%:%576=173%:%
%:%579=174%:%
%:%583=174%:%
%:%584=174%:%
%:%589=174%:%
%:%592=175%:%
%:%593=176%:%
%:%594=176%:%
%:%597=177%:%
%:%601=177%:%
%:%602=177%:%
%:%607=177%:%
%:%610=178%:%
%:%611=179%:%
%:%612=179%:%
%:%615=180%:%
%:%619=180%:%
%:%620=180%:%
%:%621=180%:%
%:%626=180%:%
%:%629=181%:%
%:%630=182%:%
%:%631=182%:%
%:%632=182%:%
%:%633=182%:%
%:%634=183%:%
%:%635=184%:%
%:%638=185%:%
%:%642=185%:%
%:%643=185%:%
%:%644=186%:%
%:%645=186%:%
%:%646=186%:%
%:%655=188%:%
%:%656=189%:%
%:%657=190%:%
%:%659=191%:%
%:%660=191%:%
%:%661=192%:%
%:%662=193%:%
%:%663=194%:%
%:%664=195%:%
%:%665=195%:%
%:%668=196%:%
%:%672=196%:%
%:%673=196%:%
%:%678=196%:%
%:%681=197%:%
%:%682=198%:%
%:%683=198%:%
%:%684=199%:%
%:%687=200%:%
%:%691=200%:%
%:%692=200%:%
%:%697=200%:%
%:%700=201%:%
%:%701=202%:%
%:%702=202%:%
%:%703=203%:%
%:%706=204%:%
%:%710=204%:%
%:%711=204%:%
%:%716=204%:%
%:%719=205%:%
%:%720=206%:%
%:%721=206%:%
%:%722=207%:%
%:%723=208%:%
%:%726=209%:%
%:%730=209%:%
%:%731=209%:%
%:%732=210%:%
%:%733=210%:%
%:%734=210%:%
%:%739=210%:%
%:%742=211%:%
%:%743=212%:%
%:%744=212%:%
%:%747=213%:%
%:%751=213%:%
%:%752=213%:%
%:%753=213%:%
%:%758=213%:%
%:%761=214%:%
%:%762=215%:%
%:%763=215%:%
%:%766=216%:%
%:%770=216%:%
%:%771=216%:%
%:%772=217%:%
%:%773=217%:%
%:%774=217%:%
%:%779=217%:%
%:%782=218%:%
%:%783=219%:%
%:%784=219%:%
%:%787=220%:%
%:%791=220%:%
%:%792=220%:%
%:%797=220%:%
%:%800=221%:%
%:%801=222%:%
%:%802=222%:%
%:%803=223%:%
%:%804=224%:%
%:%807=225%:%
%:%811=225%:%
%:%812=225%:%
%:%813=226%:%
%:%814=226%:%
%:%815=227%:%
%:%816=227%:%
%:%817=228%:%
%:%818=228%:%
%:%819=228%:%
%:%820=229%:%
%:%821=229%:%
%:%822=230%:%
%:%823=230%:%
%:%824=231%:%
%:%825=231%:%
%:%826=232%:%
%:%827=232%:%
%:%828=233%:%
%:%829=233%:%
%:%830=234%:%
%:%831=234%:%
%:%832=235%:%
%:%833=235%:%
%:%834=235%:%
%:%835=236%:%
%:%836=236%:%
%:%837=237%:%
%:%838=237%:%
%:%839=238%:%
%:%840=238%:%
%:%841=238%:%
%:%842=239%:%
%:%843=239%:%
%:%844=240%:%
%:%845=240%:%
%:%846=241%:%
%:%847=241%:%
%:%848=242%:%
%:%849=242%:%
%:%850=243%:%
%:%851=243%:%
%:%852=244%:%
%:%853=244%:%
%:%854=244%:%
%:%855=244%:%
%:%856=245%:%
%:%857=245%:%
%:%858=246%:%
%:%864=246%:%
%:%867=247%:%
%:%868=248%:%
%:%869=248%:%
%:%872=249%:%
%:%876=249%:%
%:%877=249%:%
%:%878=250%:%
%:%879=250%:%
%:%884=250%:%
%:%887=251%:%
%:%888=252%:%
%:%889=252%:%
%:%890=253%:%
%:%891=254%:%
%:%892=255%:%
%:%893=256%:%
%:%896=257%:%
%:%900=257%:%
%:%901=257%:%
%:%902=258%:%
%:%903=258%:%
%:%904=259%:%
%:%905=259%:%
%:%906=260%:%
%:%907=260%:%
%:%908=260%:%
%:%909=260%:%
%:%910=260%:%
%:%911=261%:%
%:%912=261%:%
%:%913=262%:%
%:%914=262%:%
%:%915=263%:%
%:%916=263%:%
%:%917=264%:%
%:%918=264%:%
%:%919=265%:%
%:%920=265%:%
%:%921=265%:%
%:%922=266%:%
%:%923=266%:%
%:%924=267%:%
%:%925=267%:%
%:%926=267%:%
%:%927=268%:%
%:%928=268%:%
%:%929=269%:%
%:%930=269%:%
%:%931=270%:%
%:%932=270%:%
%:%933=271%:%
%:%934=271%:%
%:%935=272%:%
%:%936=272%:%
%:%937=272%:%
%:%938=272%:%
%:%939=272%:%
%:%940=273%:%
%:%941=273%:%
%:%942=274%:%
%:%943=274%:%
%:%944=275%:%
%:%945=275%:%
%:%946=275%:%
%:%947=275%:%
%:%948=275%:%
%:%949=276%:%
%:%950=276%:%
%:%951=277%:%
%:%957=277%:%
%:%960=278%:%
%:%961=279%:%
%:%962=279%:%
%:%963=280%:%
%:%964=281%:%
%:%965=282%:%
%:%966=283%:%
%:%973=284%:%
%:%974=284%:%
%:%975=285%:%
%:%976=285%:%
%:%977=286%:%
%:%978=286%:%
%:%979=286%:%
%:%980=286%:%
%:%981=287%:%
%:%982=287%:%
%:%983=288%:%
%:%984=288%:%
%:%985=289%:%
%:%986=289%:%
%:%987=290%:%
%:%988=290%:%
%:%989=291%:%
%:%990=291%:%
%:%991=292%:%
%:%992=292%:%
%:%993=293%:%
%:%994=293%:%
%:%995=294%:%
%:%996=294%:%
%:%997=295%:%
%:%998=295%:%
%:%999=296%:%
%:%1000=296%:%
%:%1001=296%:%
%:%1002=297%:%
%:%1003=297%:%
%:%1004=298%:%
%:%1005=298%:%
%:%1006=299%:%
%:%1007=299%:%
%:%1008=300%:%
%:%1009=300%:%
%:%1010=301%:%
%:%1011=301%:%
%:%1012=301%:%
%:%1013=302%:%
%:%1014=302%:%
%:%1015=303%:%
%:%1016=303%:%
%:%1017=304%:%
%:%1018=304%:%
%:%1019=305%:%
%:%1020=305%:%
%:%1021=306%:%
%:%1022=306%:%
%:%1023=306%:%
%:%1024=306%:%
%:%1025=307%:%
%:%1031=307%:%
%:%1034=308%:%
%:%1035=309%:%
%:%1043=311%:%
%:%1053=312%:%
%:%1054=312%:%
%:%1055=313%:%
%:%1056=314%:%
%:%1057=315%:%
%:%1058=316%:%
%:%1059=317%:%
%:%1060=318%:%
%:%1061=319%:%
%:%1062=319%:%
%:%1063=320%:%
%:%1064=321%:%
%:%1066=323%:%
%:%1067=324%:%
%:%1069=325%:%
%:%1070=325%:%
%:%1071=326%:%
%:%1072=327%:%
%:%1073=328%:%
%:%1074=329%:%
%:%1075=330%:%
%:%1082=331%:%
%:%1083=331%:%
%:%1084=332%:%
%:%1085=332%:%
%:%1086=333%:%
%:%1087=333%:%
%:%1088=333%:%
%:%1089=334%:%
%:%1090=334%:%
%:%1091=335%:%
%:%1092=335%:%
%:%1093=336%:%
%:%1094=336%:%
%:%1095=336%:%
%:%1096=337%:%
%:%1097=337%:%
%:%1098=338%:%
%:%1099=338%:%
%:%1100=338%:%
%:%1101=338%:%
%:%1102=339%:%
%:%1103=339%:%
%:%1104=339%:%
%:%1105=340%:%
%:%1106=340%:%
%:%1107=340%:%
%:%1108=341%:%
%:%1109=341%:%
%:%1110=341%:%
%:%1111=342%:%
%:%1112=342%:%
%:%1113=343%:%
%:%1114=343%:%
%:%1115=343%:%
%:%1116=343%:%
%:%1117=344%:%
%:%1118=344%:%
%:%1119=344%:%
%:%1120=345%:%
%:%1121=345%:%
%:%1122=346%:%
%:%1123=346%:%
%:%1124=346%:%
%:%1125=347%:%
%:%1126=347%:%
%:%1127=348%:%
%:%1128=348%:%
%:%1129=349%:%
%:%1130=349%:%
%:%1131=349%:%
%:%1132=350%:%
%:%1133=350%:%
%:%1134=351%:%
%:%1135=351%:%
%:%1136=352%:%
%:%1137=352%:%
%:%1138=353%:%
%:%1139=353%:%
%:%1140=354%:%
%:%1141=354%:%
%:%1142=355%:%
%:%1143=355%:%
%:%1144=356%:%
%:%1145=356%:%
%:%1146=357%:%
%:%1147=357%:%
%:%1148=357%:%
%:%1149=358%:%
%:%1150=358%:%
%:%1151=359%:%
%:%1152=359%:%
%:%1153=359%:%
%:%1154=360%:%
%:%1155=360%:%
%:%1156=361%:%
%:%1157=361%:%
%:%1158=361%:%
%:%1159=362%:%
%:%1160=362%:%
%:%1161=363%:%
%:%1162=363%:%
%:%1163=363%:%
%:%1164=364%:%
%:%1165=364%:%
%:%1166=365%:%
%:%1167=365%:%
%:%1168=365%:%
%:%1169=365%:%
%:%1170=366%:%
%:%1176=366%:%
%:%1179=367%:%
%:%1180=368%:%
%:%1183=370%:%
%:%1184=371%:%
%:%1186=373%:%
%:%1187=373%:%
%:%1188=374%:%
%:%1189=374%:%
%:%1190=375%:%
%:%1191=376%:%
%:%1192=377%:%
%:%1193=378%:%
%:%1194=379%:%
%:%1195=379%:%
%:%1196=380%:%
%:%1197=381%:%
%:%1198=382%:%
%:%1199=382%:%
%:%1200=383%:%
%:%1201=384%:%
%:%1204=385%:%
%:%1208=385%:%
%:%1209=385%:%
%:%1210=385%:%
%:%1215=385%:%
%:%1218=386%:%
%:%1219=387%:%
%:%1220=387%:%
%:%1221=388%:%
%:%1222=389%:%
%:%1223=390%:%
%:%1224=391%:%
%:%1225=392%:%
%:%1226=393%:%
%:%1227=394%:%
%:%1230=395%:%
%:%1234=395%:%
%:%1235=395%:%
%:%1236=396%:%
%:%1237=396%:%
%:%1238=397%:%
%:%1239=397%:%
%:%1240=398%:%
%:%1241=398%:%
%:%1242=399%:%
%:%1243=399%:%
%:%1244=399%:%
%:%1245=400%:%
%:%1246=400%:%
%:%1247=400%:%
%:%1248=401%:%
%:%1249=401%:%
%:%1250=402%:%
%:%1251=402%:%
%:%1252=402%:%
%:%1253=403%:%
%:%1254=403%:%
%:%1255=404%:%
%:%1256=404%:%
%:%1257=405%:%
%:%1258=405%:%
%:%1259=405%:%
%:%1260=406%:%
%:%1261=406%:%
%:%1262=406%:%
%:%1263=407%:%
%:%1264=407%:%
%:%1265=408%:%
%:%1266=408%:%
%:%1267=409%:%
%:%1268=409%:%
%:%1269=409%:%
%:%1270=409%:%
%:%1271=410%:%
%:%1272=410%:%
%:%1273=411%:%
%:%1274=411%:%
%:%1275=412%:%
%:%1276=412%:%
%:%1277=413%:%
%:%1278=413%:%
%:%1279=413%:%
%:%1280=414%:%
%:%1281=414%:%
%:%1282=414%:%
%:%1283=415%:%
%:%1284=415%:%
%:%1285=416%:%
%:%1286=416%:%
%:%1287=416%:%
%:%1288=417%:%
%:%1289=417%:%
%:%1290=417%:%
%:%1291=418%:%
%:%1292=418%:%
%:%1293=418%:%
%:%1294=419%:%
%:%1295=419%:%
%:%1296=420%:%
%:%1297=420%:%
%:%1298=420%:%
%:%1299=421%:%
%:%1300=421%:%
%:%1301=422%:%
%:%1302=422%:%
%:%1303=423%:%
%:%1304=423%:%
%:%1305=423%:%
%:%1306=423%:%
%:%1307=424%:%
%:%1313=424%:%
%:%1316=425%:%
%:%1317=426%:%
%:%1318=426%:%
%:%1319=427%:%
%:%1320=428%:%
%:%1321=429%:%
%:%1322=430%:%
%:%1323=431%:%
%:%1324=432%:%
%:%1331=433%:%
%:%1332=433%:%
%:%1333=434%:%
%:%1334=434%:%
%:%1335=435%:%
%:%1336=435%:%
%:%1337=435%:%
%:%1338=436%:%
%:%1339=436%:%
%:%1340=437%:%
%:%1341=437%:%
%:%1342=438%:%
%:%1343=438%:%
%:%1344=439%:%
%:%1345=439%:%
%:%1346=440%:%
%:%1347=440%:%
%:%1348=441%:%
%:%1349=441%:%
%:%1350=442%:%
%:%1351=442%:%
%:%1352=442%:%
%:%1353=443%:%
%:%1354=443%:%
%:%1355=444%:%
%:%1356=444%:%
%:%1357=445%:%
%:%1358=445%:%
%:%1359=446%:%
%:%1360=446%:%
%:%1361=446%:%
%:%1362=446%:%
%:%1363=447%:%
%:%1369=447%:%
%:%1372=448%:%
%:%1373=449%:%
%:%1374=449%:%
%:%1375=450%:%
%:%1376=451%:%
%:%1379=452%:%
%:%1383=452%:%
%:%1384=452%:%
%:%1385=453%:%
%:%1386=453%:%
%:%1391=453%:%
%:%1394=454%:%
%:%1395=455%:%
%:%1396=455%:%
%:%1397=456%:%
%:%1398=457%:%
%:%1401=458%:%
%:%1405=458%:%
%:%1406=458%:%
%:%1407=458%:%
%:%1412=458%:%
%:%1415=459%:%
%:%1416=460%:%
%:%1417=460%:%
%:%1418=461%:%
%:%1419=462%:%
%:%1420=462%:%
%:%1423=463%:%
%:%1428=464%:%

%
\begin{isabellebody}%
\setisabellecontext{Nat{\isacharunderscore}{\kern0pt}Miscellanea}%
%
\isadelimdocument
%
\endisadelimdocument
%
\isatagdocument
%
\isamarkupsection{Auxiliary results on arithmetic%
}
\isamarkuptrue%
%
\endisatagdocument
{\isafolddocument}%
%
\isadelimdocument
%
\endisadelimdocument
%
\isadelimtheory
%
\endisadelimtheory
%
\isatagtheory
\isacommand{theory}\isamarkupfalse%
\ Nat{\isacharunderscore}{\kern0pt}Miscellanea\ \isakeyword{imports}\ ZF\ \isakeyword{begin}%
\endisatagtheory
{\isafoldtheory}%
%
\isadelimtheory
%
\endisadelimtheory
%
\begin{isamarkuptext}%
Most of these results will get used at some point for the
calculation of arities.%
\end{isamarkuptext}\isamarkuptrue%
\isacommand{lemmas}\isamarkupfalse%
\ nat{\isacharunderscore}{\kern0pt}succI\ {\isacharequal}{\kern0pt}\ \ Ord{\isacharunderscore}{\kern0pt}succ{\isacharunderscore}{\kern0pt}mem{\isacharunderscore}{\kern0pt}iff\ {\isacharbrackleft}{\kern0pt}THEN\ iffD{\isadigit{2}}{\isacharcomma}{\kern0pt}OF\ nat{\isacharunderscore}{\kern0pt}into{\isacharunderscore}{\kern0pt}Ord{\isacharbrackright}{\kern0pt}\isanewline
\isanewline
\isacommand{lemma}\isamarkupfalse%
\ nat{\isacharunderscore}{\kern0pt}succD\ {\isacharcolon}{\kern0pt}\ {\isachardoublequoteopen}m\ {\isasymin}\ nat\ {\isasymLongrightarrow}\ \ succ{\isacharparenleft}{\kern0pt}n{\isacharparenright}{\kern0pt}\ {\isasymin}\ succ{\isacharparenleft}{\kern0pt}m{\isacharparenright}{\kern0pt}\ {\isasymLongrightarrow}\ n\ {\isasymin}\ m{\isachardoublequoteclose}\isanewline
%
\isadelimproof
\ \ %
\endisadelimproof
%
\isatagproof
\isacommand{by}\isamarkupfalse%
\ {\isacharparenleft}{\kern0pt}drule{\isacharunderscore}{\kern0pt}tac\ j{\isacharequal}{\kern0pt}{\isachardoublequoteopen}succ{\isacharparenleft}{\kern0pt}m{\isacharparenright}{\kern0pt}{\isachardoublequoteclose}\ \isakeyword{in}\ ltI{\isacharcomma}{\kern0pt}auto\ elim{\isacharcolon}{\kern0pt}ltD{\isacharparenright}{\kern0pt}%
\endisatagproof
{\isafoldproof}%
%
\isadelimproof
\isanewline
%
\endisadelimproof
\isanewline
\isacommand{lemmas}\isamarkupfalse%
\ zero{\isacharunderscore}{\kern0pt}in\ {\isacharequal}{\kern0pt}\ \ ltD\ {\isacharbrackleft}{\kern0pt}OF\ nat{\isacharunderscore}{\kern0pt}{\isadigit{0}}{\isacharunderscore}{\kern0pt}le{\isacharbrackright}{\kern0pt}\isanewline
\isanewline
\isacommand{lemma}\isamarkupfalse%
\ in{\isacharunderscore}{\kern0pt}n{\isacharunderscore}{\kern0pt}in{\isacharunderscore}{\kern0pt}nat\ {\isacharcolon}{\kern0pt}\ \ {\isachardoublequoteopen}m\ {\isasymin}\ nat\ {\isasymLongrightarrow}\ n\ {\isasymin}\ m\ {\isasymLongrightarrow}\ n\ {\isasymin}\ nat{\isachardoublequoteclose}\isanewline
%
\isadelimproof
\ \ %
\endisadelimproof
%
\isatagproof
\isacommand{by}\isamarkupfalse%
{\isacharparenleft}{\kern0pt}drule\ ltI{\isacharbrackleft}{\kern0pt}of\ {\isachardoublequoteopen}n{\isachardoublequoteclose}{\isacharbrackright}{\kern0pt}{\isacharcomma}{\kern0pt}auto\ simp\ add{\isacharcolon}{\kern0pt}\ lt{\isacharunderscore}{\kern0pt}nat{\isacharunderscore}{\kern0pt}in{\isacharunderscore}{\kern0pt}nat{\isacharparenright}{\kern0pt}%
\endisatagproof
{\isafoldproof}%
%
\isadelimproof
\isanewline
%
\endisadelimproof
\isanewline
\isacommand{lemma}\isamarkupfalse%
\ in{\isacharunderscore}{\kern0pt}succ{\isacharunderscore}{\kern0pt}in{\isacharunderscore}{\kern0pt}nat\ {\isacharcolon}{\kern0pt}\ {\isachardoublequoteopen}m\ {\isasymin}\ nat\ {\isasymLongrightarrow}\ n\ {\isasymin}\ succ{\isacharparenleft}{\kern0pt}m{\isacharparenright}{\kern0pt}\ {\isasymLongrightarrow}\ n\ {\isasymin}\ nat{\isachardoublequoteclose}\isanewline
%
\isadelimproof
\ \ %
\endisadelimproof
%
\isatagproof
\isacommand{by}\isamarkupfalse%
{\isacharparenleft}{\kern0pt}auto\ simp\ add{\isacharcolon}{\kern0pt}in{\isacharunderscore}{\kern0pt}n{\isacharunderscore}{\kern0pt}in{\isacharunderscore}{\kern0pt}nat{\isacharparenright}{\kern0pt}%
\endisatagproof
{\isafoldproof}%
%
\isadelimproof
\isanewline
%
\endisadelimproof
\isanewline
\isacommand{lemma}\isamarkupfalse%
\ ltI{\isacharunderscore}{\kern0pt}neg\ {\isacharcolon}{\kern0pt}\ {\isachardoublequoteopen}x\ {\isasymin}\ nat\ {\isasymLongrightarrow}\ j\ {\isasymle}\ x\ {\isasymLongrightarrow}\ j\ {\isasymnoteq}\ x\ {\isasymLongrightarrow}\ j\ {\isacharless}{\kern0pt}\ x{\isachardoublequoteclose}\isanewline
%
\isadelimproof
\ \ %
\endisadelimproof
%
\isatagproof
\isacommand{by}\isamarkupfalse%
\ {\isacharparenleft}{\kern0pt}simp\ add{\isacharcolon}{\kern0pt}\ le{\isacharunderscore}{\kern0pt}iff{\isacharparenright}{\kern0pt}%
\endisatagproof
{\isafoldproof}%
%
\isadelimproof
\isanewline
%
\endisadelimproof
\isanewline
\isacommand{lemma}\isamarkupfalse%
\ succ{\isacharunderscore}{\kern0pt}pred{\isacharunderscore}{\kern0pt}eq\ \ {\isacharcolon}{\kern0pt}\ \ {\isachardoublequoteopen}m\ {\isasymin}\ nat\ {\isasymLongrightarrow}\ m\ {\isasymnoteq}\ {\isadigit{0}}\ \ {\isasymLongrightarrow}\ succ{\isacharparenleft}{\kern0pt}pred{\isacharparenleft}{\kern0pt}m{\isacharparenright}{\kern0pt}{\isacharparenright}{\kern0pt}\ {\isacharequal}{\kern0pt}\ m{\isachardoublequoteclose}\isanewline
%
\isadelimproof
\ \ %
\endisadelimproof
%
\isatagproof
\isacommand{by}\isamarkupfalse%
\ {\isacharparenleft}{\kern0pt}auto\ elim{\isacharcolon}{\kern0pt}\ natE{\isacharparenright}{\kern0pt}%
\endisatagproof
{\isafoldproof}%
%
\isadelimproof
\isanewline
%
\endisadelimproof
\isanewline
\isacommand{lemma}\isamarkupfalse%
\ succ{\isacharunderscore}{\kern0pt}ltI\ {\isacharcolon}{\kern0pt}\ {\isachardoublequoteopen}succ{\isacharparenleft}{\kern0pt}j{\isacharparenright}{\kern0pt}\ {\isacharless}{\kern0pt}\ n\ {\isasymLongrightarrow}\ j\ {\isacharless}{\kern0pt}\ n{\isachardoublequoteclose}\isanewline
%
\isadelimproof
\ \ %
\endisadelimproof
%
\isatagproof
\isacommand{by}\isamarkupfalse%
\ {\isacharparenleft}{\kern0pt}simp\ add{\isacharcolon}{\kern0pt}\ succ{\isacharunderscore}{\kern0pt}leE{\isacharbrackleft}{\kern0pt}OF\ leI{\isacharbrackright}{\kern0pt}{\isacharparenright}{\kern0pt}%
\endisatagproof
{\isafoldproof}%
%
\isadelimproof
\isanewline
%
\endisadelimproof
\isanewline
\isacommand{lemma}\isamarkupfalse%
\ succ{\isacharunderscore}{\kern0pt}In\ {\isacharcolon}{\kern0pt}\ {\isachardoublequoteopen}n\ {\isasymin}\ nat\ {\isasymLongrightarrow}\ succ{\isacharparenleft}{\kern0pt}j{\isacharparenright}{\kern0pt}\ {\isasymin}\ n\ {\isasymLongrightarrow}\ j\ {\isasymin}\ n{\isachardoublequoteclose}\isanewline
%
\isadelimproof
\ \ %
\endisadelimproof
%
\isatagproof
\isacommand{by}\isamarkupfalse%
\ {\isacharparenleft}{\kern0pt}rule\ succ{\isacharunderscore}{\kern0pt}ltI{\isacharbrackleft}{\kern0pt}THEN\ ltD{\isacharbrackright}{\kern0pt}{\isacharcomma}{\kern0pt}\ auto\ intro{\isacharcolon}{\kern0pt}\ ltI{\isacharparenright}{\kern0pt}%
\endisatagproof
{\isafoldproof}%
%
\isadelimproof
\isanewline
%
\endisadelimproof
\isanewline
\isacommand{lemmas}\isamarkupfalse%
\ succ{\isacharunderscore}{\kern0pt}leD\ {\isacharequal}{\kern0pt}\ succ{\isacharunderscore}{\kern0pt}leE{\isacharbrackleft}{\kern0pt}OF\ leI{\isacharbrackright}{\kern0pt}\isanewline
\isanewline
\isacommand{lemma}\isamarkupfalse%
\ succpred{\isacharunderscore}{\kern0pt}leI\ {\isacharcolon}{\kern0pt}\ {\isachardoublequoteopen}n\ {\isasymin}\ nat\ {\isasymLongrightarrow}\ \ n\ {\isasymle}\ succ{\isacharparenleft}{\kern0pt}pred{\isacharparenleft}{\kern0pt}n{\isacharparenright}{\kern0pt}{\isacharparenright}{\kern0pt}{\isachardoublequoteclose}\isanewline
%
\isadelimproof
\ \ %
\endisadelimproof
%
\isatagproof
\isacommand{by}\isamarkupfalse%
\ {\isacharparenleft}{\kern0pt}auto\ elim{\isacharcolon}{\kern0pt}\ natE{\isacharparenright}{\kern0pt}%
\endisatagproof
{\isafoldproof}%
%
\isadelimproof
\isanewline
%
\endisadelimproof
\isanewline
\isacommand{lemma}\isamarkupfalse%
\ succpred{\isacharunderscore}{\kern0pt}n{\isadigit{0}}\ {\isacharcolon}{\kern0pt}\ {\isachardoublequoteopen}succ{\isacharparenleft}{\kern0pt}n{\isacharparenright}{\kern0pt}\ {\isasymin}\ p\ {\isasymLongrightarrow}\ p{\isasymnoteq}{\isadigit{0}}{\isachardoublequoteclose}\isanewline
%
\isadelimproof
\ \ %
\endisadelimproof
%
\isatagproof
\isacommand{by}\isamarkupfalse%
\ {\isacharparenleft}{\kern0pt}auto{\isacharparenright}{\kern0pt}%
\endisatagproof
{\isafoldproof}%
%
\isadelimproof
\isanewline
%
\endisadelimproof
\isanewline
\isanewline
\isacommand{lemma}\isamarkupfalse%
\ funcI\ {\isacharcolon}{\kern0pt}\ {\isachardoublequoteopen}f\ {\isasymin}\ A\ {\isasymrightarrow}\ B\ {\isasymLongrightarrow}\ a\ {\isasymin}\ A\ {\isasymLongrightarrow}\ b{\isacharequal}{\kern0pt}\ f\ {\isacharbackquote}{\kern0pt}\ a\ {\isasymLongrightarrow}\ {\isasymlangle}a{\isacharcomma}{\kern0pt}\ b{\isasymrangle}\ {\isasymin}\ f{\isachardoublequoteclose}\isanewline
%
\isadelimproof
\ \ %
\endisadelimproof
%
\isatagproof
\isacommand{by}\isamarkupfalse%
{\isacharparenleft}{\kern0pt}simp{\isacharunderscore}{\kern0pt}all\ add{\isacharcolon}{\kern0pt}\ apply{\isacharunderscore}{\kern0pt}Pair{\isacharparenright}{\kern0pt}%
\endisatagproof
{\isafoldproof}%
%
\isadelimproof
\isanewline
%
\endisadelimproof
\isanewline
\isacommand{lemmas}\isamarkupfalse%
\ natEin\ {\isacharequal}{\kern0pt}\ natE\ {\isacharbrackleft}{\kern0pt}OF\ lt{\isacharunderscore}{\kern0pt}nat{\isacharunderscore}{\kern0pt}in{\isacharunderscore}{\kern0pt}nat{\isacharbrackright}{\kern0pt}\isanewline
\isanewline
\isacommand{lemma}\isamarkupfalse%
\ succ{\isacharunderscore}{\kern0pt}in\ {\isacharcolon}{\kern0pt}\ {\isachardoublequoteopen}succ{\isacharparenleft}{\kern0pt}x{\isacharparenright}{\kern0pt}\ {\isasymle}\ y\ \ {\isasymLongrightarrow}\ x\ {\isasymin}\ y{\isachardoublequoteclose}\isanewline
%
\isadelimproof
\ \ %
\endisadelimproof
%
\isatagproof
\isacommand{by}\isamarkupfalse%
\ {\isacharparenleft}{\kern0pt}auto\ dest{\isacharcolon}{\kern0pt}ltD{\isacharparenright}{\kern0pt}%
\endisatagproof
{\isafoldproof}%
%
\isadelimproof
\isanewline
%
\endisadelimproof
\isanewline
\isacommand{lemmas}\isamarkupfalse%
\ Un{\isacharunderscore}{\kern0pt}least{\isacharunderscore}{\kern0pt}lt{\isacharunderscore}{\kern0pt}iffn\ {\isacharequal}{\kern0pt}\ \ Un{\isacharunderscore}{\kern0pt}least{\isacharunderscore}{\kern0pt}lt{\isacharunderscore}{\kern0pt}iff\ {\isacharbrackleft}{\kern0pt}OF\ nat{\isacharunderscore}{\kern0pt}into{\isacharunderscore}{\kern0pt}Ord\ nat{\isacharunderscore}{\kern0pt}into{\isacharunderscore}{\kern0pt}Ord{\isacharbrackright}{\kern0pt}\isanewline
\isanewline
\isacommand{lemma}\isamarkupfalse%
\ pred{\isacharunderscore}{\kern0pt}le{\isadigit{2}}\ {\isacharcolon}{\kern0pt}\ {\isachardoublequoteopen}n{\isasymin}\ nat\ {\isasymLongrightarrow}\ m\ {\isasymin}\ nat\ {\isasymLongrightarrow}\ pred{\isacharparenleft}{\kern0pt}n{\isacharparenright}{\kern0pt}\ {\isasymle}\ m\ {\isasymLongrightarrow}\ n\ {\isasymle}\ succ{\isacharparenleft}{\kern0pt}m{\isacharparenright}{\kern0pt}{\isachardoublequoteclose}\isanewline
%
\isadelimproof
\ \ %
\endisadelimproof
%
\isatagproof
\isacommand{by}\isamarkupfalse%
{\isacharparenleft}{\kern0pt}subgoal{\isacharunderscore}{\kern0pt}tac\ {\isachardoublequoteopen}n{\isasymin}nat{\isachardoublequoteclose}{\isacharcomma}{\kern0pt}rule{\isacharunderscore}{\kern0pt}tac\ n{\isacharequal}{\kern0pt}{\isachardoublequoteopen}n{\isachardoublequoteclose}\ \isakeyword{in}\ natE{\isacharcomma}{\kern0pt}auto{\isacharparenright}{\kern0pt}%
\endisatagproof
{\isafoldproof}%
%
\isadelimproof
\isanewline
%
\endisadelimproof
\isanewline
\isacommand{lemma}\isamarkupfalse%
\ pred{\isacharunderscore}{\kern0pt}le\ {\isacharcolon}{\kern0pt}\ {\isachardoublequoteopen}n{\isasymin}\ nat\ {\isasymLongrightarrow}\ m\ {\isasymin}\ nat\ {\isasymLongrightarrow}\ n\ {\isasymle}\ succ{\isacharparenleft}{\kern0pt}m{\isacharparenright}{\kern0pt}\ {\isasymLongrightarrow}\ pred{\isacharparenleft}{\kern0pt}n{\isacharparenright}{\kern0pt}\ {\isasymle}\ m{\isachardoublequoteclose}\isanewline
%
\isadelimproof
\ \ %
\endisadelimproof
%
\isatagproof
\isacommand{by}\isamarkupfalse%
{\isacharparenleft}{\kern0pt}subgoal{\isacharunderscore}{\kern0pt}tac\ {\isachardoublequoteopen}pred{\isacharparenleft}{\kern0pt}n{\isacharparenright}{\kern0pt}{\isasymin}nat{\isachardoublequoteclose}{\isacharcomma}{\kern0pt}rule{\isacharunderscore}{\kern0pt}tac\ n{\isacharequal}{\kern0pt}{\isachardoublequoteopen}n{\isachardoublequoteclose}\ \isakeyword{in}\ natE{\isacharcomma}{\kern0pt}auto{\isacharparenright}{\kern0pt}%
\endisatagproof
{\isafoldproof}%
%
\isadelimproof
\isanewline
%
\endisadelimproof
\isanewline
\isacommand{lemma}\isamarkupfalse%
\ Un{\isacharunderscore}{\kern0pt}leD{\isadigit{1}}\ {\isacharcolon}{\kern0pt}\ {\isachardoublequoteopen}Ord{\isacharparenleft}{\kern0pt}i{\isacharparenright}{\kern0pt}{\isasymLongrightarrow}\ Ord{\isacharparenleft}{\kern0pt}j{\isacharparenright}{\kern0pt}{\isasymLongrightarrow}\ Ord{\isacharparenleft}{\kern0pt}k{\isacharparenright}{\kern0pt}{\isasymLongrightarrow}\ \ i\ {\isasymunion}\ j\ {\isasymle}\ k\ {\isasymLongrightarrow}\ i\ {\isasymle}\ k{\isachardoublequoteclose}\ \ \ \isanewline
%
\isadelimproof
\ \ %
\endisadelimproof
%
\isatagproof
\isacommand{by}\isamarkupfalse%
\ {\isacharparenleft}{\kern0pt}rule\ Un{\isacharunderscore}{\kern0pt}least{\isacharunderscore}{\kern0pt}lt{\isacharunderscore}{\kern0pt}iff{\isacharbrackleft}{\kern0pt}THEN\ iffD{\isadigit{1}}{\isacharbrackleft}{\kern0pt}THEN\ conjunct{\isadigit{1}}{\isacharbrackright}{\kern0pt}{\isacharbrackright}{\kern0pt}{\isacharcomma}{\kern0pt}simp{\isacharunderscore}{\kern0pt}all{\isacharparenright}{\kern0pt}%
\endisatagproof
{\isafoldproof}%
%
\isadelimproof
\isanewline
%
\endisadelimproof
\isanewline
\isacommand{lemma}\isamarkupfalse%
\ Un{\isacharunderscore}{\kern0pt}leD{\isadigit{2}}\ {\isacharcolon}{\kern0pt}\ {\isachardoublequoteopen}Ord{\isacharparenleft}{\kern0pt}i{\isacharparenright}{\kern0pt}{\isasymLongrightarrow}\ Ord{\isacharparenleft}{\kern0pt}j{\isacharparenright}{\kern0pt}{\isasymLongrightarrow}\ Ord{\isacharparenleft}{\kern0pt}k{\isacharparenright}{\kern0pt}{\isasymLongrightarrow}\ \ i\ {\isasymunion}\ j\ {\isasymle}k\ {\isasymLongrightarrow}\ j\ {\isasymle}\ k{\isachardoublequoteclose}\ \ \ \isanewline
%
\isadelimproof
\ \ %
\endisadelimproof
%
\isatagproof
\isacommand{by}\isamarkupfalse%
\ {\isacharparenleft}{\kern0pt}rule\ Un{\isacharunderscore}{\kern0pt}least{\isacharunderscore}{\kern0pt}lt{\isacharunderscore}{\kern0pt}iff{\isacharbrackleft}{\kern0pt}THEN\ iffD{\isadigit{1}}{\isacharbrackleft}{\kern0pt}THEN\ conjunct{\isadigit{2}}{\isacharbrackright}{\kern0pt}{\isacharbrackright}{\kern0pt}{\isacharcomma}{\kern0pt}simp{\isacharunderscore}{\kern0pt}all{\isacharparenright}{\kern0pt}%
\endisatagproof
{\isafoldproof}%
%
\isadelimproof
\isanewline
%
\endisadelimproof
\isanewline
\isacommand{lemma}\isamarkupfalse%
\ gt{\isadigit{1}}\ {\isacharcolon}{\kern0pt}\ {\isachardoublequoteopen}n\ {\isasymin}\ nat\ {\isasymLongrightarrow}\ i\ {\isasymin}\ n\ {\isasymLongrightarrow}\ i\ {\isasymnoteq}\ {\isadigit{0}}\ {\isasymLongrightarrow}\ i\ {\isasymnoteq}\ {\isadigit{1}}\ {\isasymLongrightarrow}\ {\isadigit{1}}{\isacharless}{\kern0pt}i{\isachardoublequoteclose}\isanewline
%
\isadelimproof
\ \ %
\endisadelimproof
%
\isatagproof
\isacommand{by}\isamarkupfalse%
{\isacharparenleft}{\kern0pt}rule{\isacharunderscore}{\kern0pt}tac\ n{\isacharequal}{\kern0pt}{\isachardoublequoteopen}i{\isachardoublequoteclose}\ \isakeyword{in}\ natE{\isacharcomma}{\kern0pt}erule\ in{\isacharunderscore}{\kern0pt}n{\isacharunderscore}{\kern0pt}in{\isacharunderscore}{\kern0pt}nat{\isacharcomma}{\kern0pt}auto\ intro{\isacharcolon}{\kern0pt}\ Ord{\isacharunderscore}{\kern0pt}{\isadigit{0}}{\isacharunderscore}{\kern0pt}lt{\isacharparenright}{\kern0pt}%
\endisatagproof
{\isafoldproof}%
%
\isadelimproof
\isanewline
%
\endisadelimproof
\isanewline
\isacommand{lemma}\isamarkupfalse%
\ pred{\isacharunderscore}{\kern0pt}mono\ {\isacharcolon}{\kern0pt}\ {\isachardoublequoteopen}m\ {\isasymin}\ nat\ {\isasymLongrightarrow}\ n\ {\isasymle}\ m\ {\isasymLongrightarrow}\ pred{\isacharparenleft}{\kern0pt}n{\isacharparenright}{\kern0pt}\ {\isasymle}\ pred{\isacharparenleft}{\kern0pt}m{\isacharparenright}{\kern0pt}{\isachardoublequoteclose}\isanewline
%
\isadelimproof
\ \ %
\endisadelimproof
%
\isatagproof
\isacommand{by}\isamarkupfalse%
{\isacharparenleft}{\kern0pt}rule{\isacharunderscore}{\kern0pt}tac\ n{\isacharequal}{\kern0pt}{\isachardoublequoteopen}n{\isachardoublequoteclose}\ \isakeyword{in}\ natE{\isacharcomma}{\kern0pt}auto\ simp\ add{\isacharcolon}{\kern0pt}le{\isacharunderscore}{\kern0pt}in{\isacharunderscore}{\kern0pt}nat{\isacharcomma}{\kern0pt}erule{\isacharunderscore}{\kern0pt}tac\ n{\isacharequal}{\kern0pt}{\isachardoublequoteopen}m{\isachardoublequoteclose}\ \isakeyword{in}\ natE{\isacharcomma}{\kern0pt}auto{\isacharparenright}{\kern0pt}%
\endisatagproof
{\isafoldproof}%
%
\isadelimproof
\isanewline
%
\endisadelimproof
\isanewline
\isacommand{lemma}\isamarkupfalse%
\ succ{\isacharunderscore}{\kern0pt}mono\ {\isacharcolon}{\kern0pt}\ {\isachardoublequoteopen}m\ {\isasymin}\ nat\ {\isasymLongrightarrow}\ n\ {\isasymle}\ m\ {\isasymLongrightarrow}\ succ{\isacharparenleft}{\kern0pt}n{\isacharparenright}{\kern0pt}\ {\isasymle}\ succ{\isacharparenleft}{\kern0pt}m{\isacharparenright}{\kern0pt}{\isachardoublequoteclose}\isanewline
%
\isadelimproof
\ \ %
\endisadelimproof
%
\isatagproof
\isacommand{by}\isamarkupfalse%
\ auto%
\endisatagproof
{\isafoldproof}%
%
\isadelimproof
\isanewline
%
\endisadelimproof
\isanewline
\isacommand{lemma}\isamarkupfalse%
\ pred{\isadigit{2}}{\isacharunderscore}{\kern0pt}Un{\isacharcolon}{\kern0pt}\ \isanewline
\ \ \isakeyword{assumes}\ {\isachardoublequoteopen}j\ {\isasymin}\ nat{\isachardoublequoteclose}\ {\isachardoublequoteopen}m\ {\isasymle}\ j{\isachardoublequoteclose}\ {\isachardoublequoteopen}n\ {\isasymle}\ j{\isachardoublequoteclose}\ \isanewline
\ \ \isakeyword{shows}\ {\isachardoublequoteopen}pred{\isacharparenleft}{\kern0pt}pred{\isacharparenleft}{\kern0pt}m\ {\isasymunion}\ n{\isacharparenright}{\kern0pt}{\isacharparenright}{\kern0pt}\ {\isasymle}\ pred{\isacharparenleft}{\kern0pt}pred{\isacharparenleft}{\kern0pt}j{\isacharparenright}{\kern0pt}{\isacharparenright}{\kern0pt}{\isachardoublequoteclose}\ \isanewline
%
\isadelimproof
\ \ %
\endisadelimproof
%
\isatagproof
\isacommand{using}\isamarkupfalse%
\ assms\ pred{\isacharunderscore}{\kern0pt}mono{\isacharbrackleft}{\kern0pt}of\ {\isachardoublequoteopen}j{\isachardoublequoteclose}{\isacharbrackright}{\kern0pt}\ le{\isacharunderscore}{\kern0pt}in{\isacharunderscore}{\kern0pt}nat\ Un{\isacharunderscore}{\kern0pt}least{\isacharunderscore}{\kern0pt}lt\ pred{\isacharunderscore}{\kern0pt}mono\ \isacommand{by}\isamarkupfalse%
\ simp%
\endisatagproof
{\isafoldproof}%
%
\isadelimproof
\isanewline
%
\endisadelimproof
\isanewline
\isacommand{lemma}\isamarkupfalse%
\ nat{\isacharunderscore}{\kern0pt}union{\isacharunderscore}{\kern0pt}abs{\isadigit{1}}\ {\isacharcolon}{\kern0pt}\ \isanewline
\ \ {\isachardoublequoteopen}{\isasymlbrakk}\ Ord{\isacharparenleft}{\kern0pt}i{\isacharparenright}{\kern0pt}\ {\isacharsemicolon}{\kern0pt}\ Ord{\isacharparenleft}{\kern0pt}j{\isacharparenright}{\kern0pt}\ {\isacharsemicolon}{\kern0pt}\ i\ {\isasymle}\ j\ {\isasymrbrakk}\ {\isasymLongrightarrow}\ i\ {\isasymunion}\ j\ {\isacharequal}{\kern0pt}\ j{\isachardoublequoteclose}\isanewline
%
\isadelimproof
\ \ %
\endisadelimproof
%
\isatagproof
\isacommand{by}\isamarkupfalse%
\ {\isacharparenleft}{\kern0pt}rule\ Un{\isacharunderscore}{\kern0pt}absorb{\isadigit{1}}{\isacharcomma}{\kern0pt}erule\ le{\isacharunderscore}{\kern0pt}imp{\isacharunderscore}{\kern0pt}subset{\isacharparenright}{\kern0pt}%
\endisatagproof
{\isafoldproof}%
%
\isadelimproof
\isanewline
%
\endisadelimproof
\isanewline
\isacommand{lemma}\isamarkupfalse%
\ nat{\isacharunderscore}{\kern0pt}union{\isacharunderscore}{\kern0pt}abs{\isadigit{2}}\ {\isacharcolon}{\kern0pt}\ \isanewline
\ \ {\isachardoublequoteopen}{\isasymlbrakk}\ Ord{\isacharparenleft}{\kern0pt}i{\isacharparenright}{\kern0pt}\ {\isacharsemicolon}{\kern0pt}\ Ord{\isacharparenleft}{\kern0pt}j{\isacharparenright}{\kern0pt}\ {\isacharsemicolon}{\kern0pt}\ i\ {\isasymle}\ j\ {\isasymrbrakk}\ {\isasymLongrightarrow}\ j\ {\isasymunion}\ i\ {\isacharequal}{\kern0pt}\ j{\isachardoublequoteclose}\isanewline
%
\isadelimproof
\ \ %
\endisadelimproof
%
\isatagproof
\isacommand{by}\isamarkupfalse%
\ {\isacharparenleft}{\kern0pt}rule\ Un{\isacharunderscore}{\kern0pt}absorb{\isadigit{2}}{\isacharcomma}{\kern0pt}erule\ le{\isacharunderscore}{\kern0pt}imp{\isacharunderscore}{\kern0pt}subset{\isacharparenright}{\kern0pt}%
\endisatagproof
{\isafoldproof}%
%
\isadelimproof
\isanewline
%
\endisadelimproof
\isanewline
\isacommand{lemma}\isamarkupfalse%
\ nat{\isacharunderscore}{\kern0pt}un{\isacharunderscore}{\kern0pt}max\ {\isacharcolon}{\kern0pt}\ {\isachardoublequoteopen}Ord{\isacharparenleft}{\kern0pt}i{\isacharparenright}{\kern0pt}\ {\isasymLongrightarrow}\ Ord{\isacharparenleft}{\kern0pt}j{\isacharparenright}{\kern0pt}\ {\isasymLongrightarrow}\ i\ {\isasymunion}\ j\ {\isacharequal}{\kern0pt}\ max{\isacharparenleft}{\kern0pt}i{\isacharcomma}{\kern0pt}j{\isacharparenright}{\kern0pt}{\isachardoublequoteclose}\isanewline
%
\isadelimproof
\ \ %
\endisadelimproof
%
\isatagproof
\isacommand{using}\isamarkupfalse%
\ max{\isacharunderscore}{\kern0pt}def\ nat{\isacharunderscore}{\kern0pt}union{\isacharunderscore}{\kern0pt}abs{\isadigit{1}}\ not{\isacharunderscore}{\kern0pt}lt{\isacharunderscore}{\kern0pt}iff{\isacharunderscore}{\kern0pt}le\ leI\ nat{\isacharunderscore}{\kern0pt}union{\isacharunderscore}{\kern0pt}abs{\isadigit{2}}\isanewline
\ \ \isacommand{by}\isamarkupfalse%
\ auto%
\endisatagproof
{\isafoldproof}%
%
\isadelimproof
\isanewline
%
\endisadelimproof
\isanewline
\isacommand{lemma}\isamarkupfalse%
\ nat{\isacharunderscore}{\kern0pt}max{\isacharunderscore}{\kern0pt}ty\ {\isacharcolon}{\kern0pt}\ {\isachardoublequoteopen}Ord{\isacharparenleft}{\kern0pt}i{\isacharparenright}{\kern0pt}\ {\isasymLongrightarrow}Ord{\isacharparenleft}{\kern0pt}j{\isacharparenright}{\kern0pt}\ {\isasymLongrightarrow}\ Ord{\isacharparenleft}{\kern0pt}max{\isacharparenleft}{\kern0pt}i{\isacharcomma}{\kern0pt}j{\isacharparenright}{\kern0pt}{\isacharparenright}{\kern0pt}{\isachardoublequoteclose}\isanewline
%
\isadelimproof
\ \ %
\endisadelimproof
%
\isatagproof
\isacommand{unfolding}\isamarkupfalse%
\ max{\isacharunderscore}{\kern0pt}def\ \isacommand{by}\isamarkupfalse%
\ simp%
\endisatagproof
{\isafoldproof}%
%
\isadelimproof
\isanewline
%
\endisadelimproof
\isanewline
\isacommand{lemma}\isamarkupfalse%
\ le{\isacharunderscore}{\kern0pt}not{\isacharunderscore}{\kern0pt}lt{\isacharunderscore}{\kern0pt}nat\ {\isacharcolon}{\kern0pt}\ {\isachardoublequoteopen}Ord{\isacharparenleft}{\kern0pt}p{\isacharparenright}{\kern0pt}\ {\isasymLongrightarrow}\ Ord{\isacharparenleft}{\kern0pt}q{\isacharparenright}{\kern0pt}\ {\isasymLongrightarrow}\ {\isasymnot}\ p{\isasymle}\ q\ {\isasymLongrightarrow}\ q\ {\isasymle}\ p{\isachardoublequoteclose}\ \isanewline
%
\isadelimproof
\ \ %
\endisadelimproof
%
\isatagproof
\isacommand{by}\isamarkupfalse%
\ {\isacharparenleft}{\kern0pt}rule\ ltE{\isacharcomma}{\kern0pt}rule\ not{\isacharunderscore}{\kern0pt}le{\isacharunderscore}{\kern0pt}iff{\isacharunderscore}{\kern0pt}lt{\isacharbrackleft}{\kern0pt}THEN\ iffD{\isadigit{1}}{\isacharbrackright}{\kern0pt}{\isacharcomma}{\kern0pt}auto{\isacharcomma}{\kern0pt}drule\ ltI{\isacharbrackleft}{\kern0pt}of\ q\ p{\isacharbrackright}{\kern0pt}{\isacharcomma}{\kern0pt}auto{\isacharcomma}{\kern0pt}erule\ leI{\isacharparenright}{\kern0pt}%
\endisatagproof
{\isafoldproof}%
%
\isadelimproof
\isanewline
%
\endisadelimproof
\isanewline
\isacommand{lemmas}\isamarkupfalse%
\ nat{\isacharunderscore}{\kern0pt}simp{\isacharunderscore}{\kern0pt}union\ {\isacharequal}{\kern0pt}\ nat{\isacharunderscore}{\kern0pt}un{\isacharunderscore}{\kern0pt}max\ nat{\isacharunderscore}{\kern0pt}max{\isacharunderscore}{\kern0pt}ty\ max{\isacharunderscore}{\kern0pt}def\ \isanewline
\isanewline
\isacommand{lemma}\isamarkupfalse%
\ le{\isacharunderscore}{\kern0pt}succ\ {\isacharcolon}{\kern0pt}\ {\isachardoublequoteopen}x{\isasymin}nat\ {\isasymLongrightarrow}\ x{\isasymle}succ{\isacharparenleft}{\kern0pt}x{\isacharparenright}{\kern0pt}{\isachardoublequoteclose}%
\isadelimproof
\ %
\endisadelimproof
%
\isatagproof
\isacommand{by}\isamarkupfalse%
\ simp%
\endisatagproof
{\isafoldproof}%
%
\isadelimproof
%
\endisadelimproof
\isanewline
\isacommand{lemma}\isamarkupfalse%
\ le{\isacharunderscore}{\kern0pt}pred\ {\isacharcolon}{\kern0pt}\ {\isachardoublequoteopen}x{\isasymin}nat\ {\isasymLongrightarrow}\ pred{\isacharparenleft}{\kern0pt}x{\isacharparenright}{\kern0pt}{\isasymle}x{\isachardoublequoteclose}\ \isanewline
%
\isadelimproof
\ \ %
\endisadelimproof
%
\isatagproof
\isacommand{using}\isamarkupfalse%
\ pred{\isacharunderscore}{\kern0pt}le{\isacharbrackleft}{\kern0pt}OF\ {\isacharunderscore}{\kern0pt}\ {\isacharunderscore}{\kern0pt}\ le{\isacharunderscore}{\kern0pt}succ{\isacharbrackright}{\kern0pt}\ pred{\isacharunderscore}{\kern0pt}succ{\isacharunderscore}{\kern0pt}eq\ \isanewline
\ \ \isacommand{by}\isamarkupfalse%
\ simp%
\endisatagproof
{\isafoldproof}%
%
\isadelimproof
\isanewline
%
\endisadelimproof
\isanewline
\isacommand{lemma}\isamarkupfalse%
\ Un{\isacharunderscore}{\kern0pt}le{\isacharunderscore}{\kern0pt}compat\ {\isacharcolon}{\kern0pt}\ {\isachardoublequoteopen}o\ {\isasymle}\ p\ {\isasymLongrightarrow}\ q\ {\isasymle}\ r\ {\isasymLongrightarrow}\ Ord{\isacharparenleft}{\kern0pt}o{\isacharparenright}{\kern0pt}\ {\isasymLongrightarrow}\ Ord{\isacharparenleft}{\kern0pt}p{\isacharparenright}{\kern0pt}\ {\isasymLongrightarrow}\ Ord{\isacharparenleft}{\kern0pt}q{\isacharparenright}{\kern0pt}\ {\isasymLongrightarrow}\ Ord{\isacharparenleft}{\kern0pt}r{\isacharparenright}{\kern0pt}\ {\isasymLongrightarrow}\ o\ {\isasymunion}\ q\ {\isasymle}\ p\ {\isasymunion}\ r{\isachardoublequoteclose}\isanewline
%
\isadelimproof
\ \ %
\endisadelimproof
%
\isatagproof
\isacommand{using}\isamarkupfalse%
\ le{\isacharunderscore}{\kern0pt}trans{\isacharbrackleft}{\kern0pt}of\ q\ r\ {\isachardoublequoteopen}p{\isasymunion}r{\isachardoublequoteclose}{\isacharcomma}{\kern0pt}OF\ {\isacharunderscore}{\kern0pt}\ Un{\isacharunderscore}{\kern0pt}upper{\isadigit{2}}{\isacharunderscore}{\kern0pt}le{\isacharbrackright}{\kern0pt}\ le{\isacharunderscore}{\kern0pt}trans{\isacharbrackleft}{\kern0pt}of\ o\ p\ {\isachardoublequoteopen}p{\isasymunion}r{\isachardoublequoteclose}{\isacharcomma}{\kern0pt}OF\ {\isacharunderscore}{\kern0pt}\ Un{\isacharunderscore}{\kern0pt}upper{\isadigit{1}}{\isacharunderscore}{\kern0pt}le{\isacharbrackright}{\kern0pt}\isanewline
\ \ \ \ nat{\isacharunderscore}{\kern0pt}simp{\isacharunderscore}{\kern0pt}union\ \isanewline
\ \ \isacommand{by}\isamarkupfalse%
\ auto%
\endisatagproof
{\isafoldproof}%
%
\isadelimproof
\isanewline
%
\endisadelimproof
\isanewline
\isacommand{lemma}\isamarkupfalse%
\ Un{\isacharunderscore}{\kern0pt}le\ {\isacharcolon}{\kern0pt}\ {\isachardoublequoteopen}p\ {\isasymle}\ r\ {\isasymLongrightarrow}\ q\ {\isasymle}\ r\ {\isasymLongrightarrow}\isanewline
\ \ \ \ \ \ \ \ \ \ \ \ \ \ \ Ord{\isacharparenleft}{\kern0pt}p{\isacharparenright}{\kern0pt}\ {\isasymLongrightarrow}\ Ord{\isacharparenleft}{\kern0pt}q{\isacharparenright}{\kern0pt}\ {\isasymLongrightarrow}\ Ord{\isacharparenleft}{\kern0pt}r{\isacharparenright}{\kern0pt}\ {\isasymLongrightarrow}\ \isanewline
\ \ \ \ \ \ \ \ \ \ \ \ \ \ \ \ p\ {\isasymunion}\ q\ {\isasymle}\ r{\isachardoublequoteclose}\isanewline
%
\isadelimproof
\ \ %
\endisadelimproof
%
\isatagproof
\isacommand{using}\isamarkupfalse%
\ nat{\isacharunderscore}{\kern0pt}simp{\isacharunderscore}{\kern0pt}union\ \isacommand{by}\isamarkupfalse%
\ auto%
\endisatagproof
{\isafoldproof}%
%
\isadelimproof
\isanewline
%
\endisadelimproof
\isanewline
\isacommand{lemma}\isamarkupfalse%
\ Un{\isacharunderscore}{\kern0pt}leI{\isadigit{3}}\ {\isacharcolon}{\kern0pt}\ {\isachardoublequoteopen}o\ {\isasymle}\ r\ {\isasymLongrightarrow}\ p\ {\isasymle}\ r\ {\isasymLongrightarrow}\ q\ {\isasymle}\ r\ {\isasymLongrightarrow}\ \isanewline
\ \ \ \ \ \ \ \ \ \ \ \ \ \ \ \ Ord{\isacharparenleft}{\kern0pt}o{\isacharparenright}{\kern0pt}\ {\isasymLongrightarrow}\ Ord{\isacharparenleft}{\kern0pt}p{\isacharparenright}{\kern0pt}\ {\isasymLongrightarrow}\ Ord{\isacharparenleft}{\kern0pt}q{\isacharparenright}{\kern0pt}\ {\isasymLongrightarrow}\ Ord{\isacharparenleft}{\kern0pt}r{\isacharparenright}{\kern0pt}\ {\isasymLongrightarrow}\ \isanewline
\ \ \ \ \ \ \ \ \ \ \ \ \ \ \ \ o\ {\isasymunion}\ p\ {\isasymunion}\ q\ {\isasymle}\ r{\isachardoublequoteclose}\isanewline
%
\isadelimproof
\ \ %
\endisadelimproof
%
\isatagproof
\isacommand{using}\isamarkupfalse%
\ nat{\isacharunderscore}{\kern0pt}simp{\isacharunderscore}{\kern0pt}union\ \isacommand{by}\isamarkupfalse%
\ auto%
\endisatagproof
{\isafoldproof}%
%
\isadelimproof
\isanewline
%
\endisadelimproof
\isanewline
\isacommand{lemma}\isamarkupfalse%
\ diff{\isacharunderscore}{\kern0pt}mono\ {\isacharcolon}{\kern0pt}\isanewline
\ \ \isakeyword{assumes}\ {\isachardoublequoteopen}m\ {\isasymin}\ nat{\isachardoublequoteclose}\ {\isachardoublequoteopen}n{\isasymin}nat{\isachardoublequoteclose}\ {\isachardoublequoteopen}p\ {\isasymin}\ nat{\isachardoublequoteclose}\ {\isachardoublequoteopen}m\ {\isacharless}{\kern0pt}\ n{\isachardoublequoteclose}\ {\isachardoublequoteopen}p{\isasymle}m{\isachardoublequoteclose}\isanewline
\ \ \isakeyword{shows}\ {\isachardoublequoteopen}m{\isacharhash}{\kern0pt}{\isacharminus}{\kern0pt}p\ {\isacharless}{\kern0pt}\ n{\isacharhash}{\kern0pt}{\isacharminus}{\kern0pt}p{\isachardoublequoteclose}\isanewline
%
\isadelimproof
%
\endisadelimproof
%
\isatagproof
\isacommand{proof}\isamarkupfalse%
\ {\isacharminus}{\kern0pt}\isanewline
\ \ \isacommand{from}\isamarkupfalse%
\ assms\isanewline
\ \ \isacommand{have}\isamarkupfalse%
\ {\isachardoublequoteopen}m{\isacharhash}{\kern0pt}{\isacharminus}{\kern0pt}p\ {\isasymin}\ nat{\isachardoublequoteclose}\ {\isachardoublequoteopen}m{\isacharhash}{\kern0pt}{\isacharminus}{\kern0pt}p\ {\isacharhash}{\kern0pt}{\isacharplus}{\kern0pt}p\ {\isacharequal}{\kern0pt}\ m{\isachardoublequoteclose}\isanewline
\ \ \ \ \isacommand{using}\isamarkupfalse%
\ add{\isacharunderscore}{\kern0pt}diff{\isacharunderscore}{\kern0pt}inverse{\isadigit{2}}\ \isacommand{by}\isamarkupfalse%
\ simp{\isacharunderscore}{\kern0pt}all\isanewline
\ \ \isacommand{with}\isamarkupfalse%
\ assms\isanewline
\ \ \isacommand{show}\isamarkupfalse%
\ {\isacharquery}{\kern0pt}thesis\isanewline
\ \ \ \ \isacommand{using}\isamarkupfalse%
\ less{\isacharunderscore}{\kern0pt}diff{\isacharunderscore}{\kern0pt}conv{\isacharbrackleft}{\kern0pt}of\ n\ p\ {\isachardoublequoteopen}m\ {\isacharhash}{\kern0pt}{\isacharminus}{\kern0pt}\ p{\isachardoublequoteclose}{\isacharcomma}{\kern0pt}THEN\ iffD{\isadigit{2}}{\isacharbrackright}{\kern0pt}\ \isacommand{by}\isamarkupfalse%
\ simp\isanewline
\isacommand{qed}\isamarkupfalse%
%
\endisatagproof
{\isafoldproof}%
%
\isadelimproof
\isanewline
%
\endisadelimproof
\isanewline
\isacommand{lemma}\isamarkupfalse%
\ pred{\isacharunderscore}{\kern0pt}Un{\isacharcolon}{\kern0pt}\isanewline
\ \ {\isachardoublequoteopen}x\ {\isasymin}\ nat\ {\isasymLongrightarrow}\ y\ {\isasymin}\ nat\ {\isasymLongrightarrow}\ Arith{\isachardot}{\kern0pt}pred{\isacharparenleft}{\kern0pt}succ{\isacharparenleft}{\kern0pt}x{\isacharparenright}{\kern0pt}\ {\isasymunion}\ y{\isacharparenright}{\kern0pt}\ {\isacharequal}{\kern0pt}\ x\ {\isasymunion}\ Arith{\isachardot}{\kern0pt}pred{\isacharparenleft}{\kern0pt}y{\isacharparenright}{\kern0pt}{\isachardoublequoteclose}\isanewline
\ \ {\isachardoublequoteopen}x\ {\isasymin}\ nat\ {\isasymLongrightarrow}\ y\ {\isasymin}\ nat\ {\isasymLongrightarrow}\ Arith{\isachardot}{\kern0pt}pred{\isacharparenleft}{\kern0pt}x\ {\isasymunion}\ succ{\isacharparenleft}{\kern0pt}y{\isacharparenright}{\kern0pt}{\isacharparenright}{\kern0pt}\ {\isacharequal}{\kern0pt}\ Arith{\isachardot}{\kern0pt}pred{\isacharparenleft}{\kern0pt}x{\isacharparenright}{\kern0pt}\ {\isasymunion}\ y{\isachardoublequoteclose}\isanewline
%
\isadelimproof
\ \ %
\endisadelimproof
%
\isatagproof
\isacommand{using}\isamarkupfalse%
\ pred{\isacharunderscore}{\kern0pt}Un{\isacharunderscore}{\kern0pt}distrib\ pred{\isacharunderscore}{\kern0pt}succ{\isacharunderscore}{\kern0pt}eq\ \isacommand{by}\isamarkupfalse%
\ simp{\isacharunderscore}{\kern0pt}all%
\endisatagproof
{\isafoldproof}%
%
\isadelimproof
\isanewline
%
\endisadelimproof
\isanewline
\isacommand{lemma}\isamarkupfalse%
\ le{\isacharunderscore}{\kern0pt}natI\ {\isacharcolon}{\kern0pt}\ {\isachardoublequoteopen}j\ {\isasymle}\ n\ {\isasymLongrightarrow}\ n\ {\isasymin}\ nat\ {\isasymLongrightarrow}\ j{\isasymin}nat{\isachardoublequoteclose}\isanewline
%
\isadelimproof
\ \ %
\endisadelimproof
%
\isatagproof
\isacommand{by}\isamarkupfalse%
{\isacharparenleft}{\kern0pt}drule\ ltD{\isacharcomma}{\kern0pt}rule\ in{\isacharunderscore}{\kern0pt}n{\isacharunderscore}{\kern0pt}in{\isacharunderscore}{\kern0pt}nat{\isacharcomma}{\kern0pt}rule\ nat{\isacharunderscore}{\kern0pt}succ{\isacharunderscore}{\kern0pt}iff{\isacharbrackleft}{\kern0pt}THEN\ iffD{\isadigit{2}}{\isacharcomma}{\kern0pt}of\ n{\isacharbrackright}{\kern0pt}{\isacharcomma}{\kern0pt}simp{\isacharunderscore}{\kern0pt}all{\isacharparenright}{\kern0pt}%
\endisatagproof
{\isafoldproof}%
%
\isadelimproof
\isanewline
%
\endisadelimproof
\isanewline
\isacommand{lemma}\isamarkupfalse%
\ le{\isacharunderscore}{\kern0pt}natE\ {\isacharcolon}{\kern0pt}\ {\isachardoublequoteopen}n{\isasymin}nat\ {\isasymLongrightarrow}\ j\ {\isacharless}{\kern0pt}\ n\ {\isasymLongrightarrow}\ \ j{\isasymin}n{\isachardoublequoteclose}\isanewline
%
\isadelimproof
\ \ %
\endisadelimproof
%
\isatagproof
\isacommand{by}\isamarkupfalse%
{\isacharparenleft}{\kern0pt}rule\ ltE{\isacharbrackleft}{\kern0pt}of\ j\ n{\isacharbrackright}{\kern0pt}{\isacharcomma}{\kern0pt}simp{\isacharplus}{\kern0pt}{\isacharparenright}{\kern0pt}%
\endisatagproof
{\isafoldproof}%
%
\isadelimproof
\isanewline
%
\endisadelimproof
\isanewline
\isacommand{lemma}\isamarkupfalse%
\ diff{\isacharunderscore}{\kern0pt}cancel\ {\isacharcolon}{\kern0pt}\isanewline
\ \ \isakeyword{assumes}\ {\isachardoublequoteopen}m\ {\isasymin}\ nat{\isachardoublequoteclose}\ {\isachardoublequoteopen}n{\isasymin}nat{\isachardoublequoteclose}\ {\isachardoublequoteopen}m\ {\isacharless}{\kern0pt}\ n{\isachardoublequoteclose}\isanewline
\ \ \isakeyword{shows}\ {\isachardoublequoteopen}m{\isacharhash}{\kern0pt}{\isacharminus}{\kern0pt}n\ {\isacharequal}{\kern0pt}\ {\isadigit{0}}{\isachardoublequoteclose}\isanewline
%
\isadelimproof
\ \ %
\endisadelimproof
%
\isatagproof
\isacommand{using}\isamarkupfalse%
\ assms\ diff{\isacharunderscore}{\kern0pt}is{\isacharunderscore}{\kern0pt}{\isadigit{0}}{\isacharunderscore}{\kern0pt}lemma\ leI\ \isacommand{by}\isamarkupfalse%
\ simp%
\endisatagproof
{\isafoldproof}%
%
\isadelimproof
\isanewline
%
\endisadelimproof
\isanewline
\isacommand{lemma}\isamarkupfalse%
\ leD\ {\isacharcolon}{\kern0pt}\ \isakeyword{assumes}\ {\isachardoublequoteopen}n{\isasymin}nat{\isachardoublequoteclose}\ {\isachardoublequoteopen}j\ {\isasymle}\ n{\isachardoublequoteclose}\isanewline
\ \ \isakeyword{shows}\ {\isachardoublequoteopen}j\ {\isacharless}{\kern0pt}\ n\ {\isacharbar}{\kern0pt}\ j\ {\isacharequal}{\kern0pt}\ n{\isachardoublequoteclose}\isanewline
%
\isadelimproof
\ \ %
\endisadelimproof
%
\isatagproof
\isacommand{using}\isamarkupfalse%
\ leE{\isacharbrackleft}{\kern0pt}OF\ {\isacartoucheopen}j{\isasymle}n{\isacartoucheclose}{\isacharcomma}{\kern0pt}of\ {\isachardoublequoteopen}j{\isacharless}{\kern0pt}n\ {\isacharbar}{\kern0pt}\ j\ {\isacharequal}{\kern0pt}\ n{\isachardoublequoteclose}{\isacharbrackright}{\kern0pt}\ \isacommand{by}\isamarkupfalse%
\ auto%
\endisatagproof
{\isafoldproof}%
%
\isadelimproof
%
\endisadelimproof
%
\isadelimdocument
%
\endisadelimdocument
%
\isatagdocument
%
\isamarkupsubsection{Some results in ordinal arithmetic%
}
\isamarkuptrue%
%
\endisatagdocument
{\isafolddocument}%
%
\isadelimdocument
%
\endisadelimdocument
%
\begin{isamarkuptext}%
The following results are auxiliary to the proof of 
wellfoundedness of the relation \isa{frecR}%
\end{isamarkuptext}\isamarkuptrue%
\isacommand{lemma}\isamarkupfalse%
\ max{\isacharunderscore}{\kern0pt}cong\ {\isacharcolon}{\kern0pt}\isanewline
\ \ \isakeyword{assumes}\ {\isachardoublequoteopen}x\ {\isasymle}\ y{\isachardoublequoteclose}\ {\isachardoublequoteopen}Ord{\isacharparenleft}{\kern0pt}y{\isacharparenright}{\kern0pt}{\isachardoublequoteclose}\ {\isachardoublequoteopen}Ord{\isacharparenleft}{\kern0pt}z{\isacharparenright}{\kern0pt}{\isachardoublequoteclose}\ \isakeyword{shows}\ {\isachardoublequoteopen}max{\isacharparenleft}{\kern0pt}x{\isacharcomma}{\kern0pt}y{\isacharparenright}{\kern0pt}\ {\isasymle}\ max{\isacharparenleft}{\kern0pt}y{\isacharcomma}{\kern0pt}z{\isacharparenright}{\kern0pt}{\isachardoublequoteclose}\isanewline
%
\isadelimproof
\ \ %
\endisadelimproof
%
\isatagproof
\isacommand{using}\isamarkupfalse%
\ assms\ \isanewline
\isacommand{proof}\isamarkupfalse%
\ {\isacharparenleft}{\kern0pt}cases\ {\isachardoublequoteopen}y\ {\isasymle}\ z{\isachardoublequoteclose}{\isacharparenright}{\kern0pt}\isanewline
\ \ \isacommand{case}\isamarkupfalse%
\ True\isanewline
\ \ \isacommand{then}\isamarkupfalse%
\ \isacommand{show}\isamarkupfalse%
\ {\isacharquery}{\kern0pt}thesis\ \isanewline
\ \ \ \ \isacommand{unfolding}\isamarkupfalse%
\ max{\isacharunderscore}{\kern0pt}def\ \isacommand{using}\isamarkupfalse%
\ assms\ \isacommand{by}\isamarkupfalse%
\ simp\isanewline
\isacommand{next}\isamarkupfalse%
\isanewline
\ \ \isacommand{case}\isamarkupfalse%
\ False\isanewline
\ \ \isacommand{then}\isamarkupfalse%
\ \isacommand{have}\isamarkupfalse%
\ {\isachardoublequoteopen}z\ {\isasymle}\ y{\isachardoublequoteclose}\ \ \isacommand{using}\isamarkupfalse%
\ assms\ not{\isacharunderscore}{\kern0pt}le{\isacharunderscore}{\kern0pt}iff{\isacharunderscore}{\kern0pt}lt\ leI\ \isacommand{by}\isamarkupfalse%
\ simp\isanewline
\ \ \isacommand{then}\isamarkupfalse%
\ \isacommand{show}\isamarkupfalse%
\ {\isacharquery}{\kern0pt}thesis\ \isanewline
\ \ \ \ \isacommand{unfolding}\isamarkupfalse%
\ max{\isacharunderscore}{\kern0pt}def\ \isacommand{using}\isamarkupfalse%
\ assms\ \isacommand{by}\isamarkupfalse%
\ simp\ \isanewline
\isacommand{qed}\isamarkupfalse%
%
\endisatagproof
{\isafoldproof}%
%
\isadelimproof
\isanewline
%
\endisadelimproof
\isanewline
\isacommand{lemma}\isamarkupfalse%
\ max{\isacharunderscore}{\kern0pt}commutes\ {\isacharcolon}{\kern0pt}\ \isanewline
\ \ \isakeyword{assumes}\ {\isachardoublequoteopen}Ord{\isacharparenleft}{\kern0pt}x{\isacharparenright}{\kern0pt}{\isachardoublequoteclose}\ {\isachardoublequoteopen}Ord{\isacharparenleft}{\kern0pt}y{\isacharparenright}{\kern0pt}{\isachardoublequoteclose}\isanewline
\ \ \isakeyword{shows}\ {\isachardoublequoteopen}max{\isacharparenleft}{\kern0pt}x{\isacharcomma}{\kern0pt}y{\isacharparenright}{\kern0pt}\ {\isacharequal}{\kern0pt}\ max{\isacharparenleft}{\kern0pt}y{\isacharcomma}{\kern0pt}x{\isacharparenright}{\kern0pt}{\isachardoublequoteclose}\isanewline
%
\isadelimproof
\ \ %
\endisadelimproof
%
\isatagproof
\isacommand{using}\isamarkupfalse%
\ assms\ Un{\isacharunderscore}{\kern0pt}commute\ nat{\isacharunderscore}{\kern0pt}simp{\isacharunderscore}{\kern0pt}union{\isacharparenleft}{\kern0pt}{\isadigit{1}}{\isacharparenright}{\kern0pt}\ nat{\isacharunderscore}{\kern0pt}simp{\isacharunderscore}{\kern0pt}union{\isacharparenleft}{\kern0pt}{\isadigit{1}}{\isacharparenright}{\kern0pt}{\isacharbrackleft}{\kern0pt}symmetric{\isacharbrackright}{\kern0pt}\ \isacommand{by}\isamarkupfalse%
\ auto%
\endisatagproof
{\isafoldproof}%
%
\isadelimproof
\isanewline
%
\endisadelimproof
\isanewline
\isacommand{lemma}\isamarkupfalse%
\ max{\isacharunderscore}{\kern0pt}cong{\isadigit{2}}\ {\isacharcolon}{\kern0pt}\isanewline
\ \ \isakeyword{assumes}\ {\isachardoublequoteopen}x\ {\isasymle}\ y{\isachardoublequoteclose}\ {\isachardoublequoteopen}Ord{\isacharparenleft}{\kern0pt}y{\isacharparenright}{\kern0pt}{\isachardoublequoteclose}\ {\isachardoublequoteopen}Ord{\isacharparenleft}{\kern0pt}z{\isacharparenright}{\kern0pt}{\isachardoublequoteclose}\ {\isachardoublequoteopen}Ord{\isacharparenleft}{\kern0pt}x{\isacharparenright}{\kern0pt}{\isachardoublequoteclose}\ \isanewline
\ \ \isakeyword{shows}\ {\isachardoublequoteopen}max{\isacharparenleft}{\kern0pt}x{\isacharcomma}{\kern0pt}z{\isacharparenright}{\kern0pt}\ {\isasymle}\ max{\isacharparenleft}{\kern0pt}y{\isacharcomma}{\kern0pt}z{\isacharparenright}{\kern0pt}{\isachardoublequoteclose}\isanewline
%
\isadelimproof
%
\endisadelimproof
%
\isatagproof
\isacommand{proof}\isamarkupfalse%
\ {\isacharminus}{\kern0pt}\isanewline
\ \ \isacommand{from}\isamarkupfalse%
\ assms\ \isanewline
\ \ \isacommand{have}\isamarkupfalse%
\ {\isachardoublequoteopen}\ x\ {\isasymunion}\ z\ {\isasymle}\ y\ {\isasymunion}\ z{\isachardoublequoteclose}\isanewline
\ \ \ \ \isacommand{using}\isamarkupfalse%
\ lt{\isacharunderscore}{\kern0pt}Ord\ Ord{\isacharunderscore}{\kern0pt}Un\ Un{\isacharunderscore}{\kern0pt}mono{\isacharbrackleft}{\kern0pt}OF\ \ le{\isacharunderscore}{\kern0pt}imp{\isacharunderscore}{\kern0pt}subset{\isacharbrackleft}{\kern0pt}OF\ {\isacartoucheopen}x{\isasymle}y{\isacartoucheclose}{\isacharbrackright}{\kern0pt}{\isacharbrackright}{\kern0pt}\ \ subset{\isacharunderscore}{\kern0pt}imp{\isacharunderscore}{\kern0pt}le\ \isacommand{by}\isamarkupfalse%
\ auto\isanewline
\ \ \isacommand{then}\isamarkupfalse%
\ \isacommand{show}\isamarkupfalse%
\ {\isacharquery}{\kern0pt}thesis\ \isanewline
\ \ \ \ \isacommand{using}\isamarkupfalse%
\ \ nat{\isacharunderscore}{\kern0pt}simp{\isacharunderscore}{\kern0pt}union\ {\isacartoucheopen}Ord{\isacharparenleft}{\kern0pt}x{\isacharparenright}{\kern0pt}{\isacartoucheclose}\ {\isacartoucheopen}Ord{\isacharparenleft}{\kern0pt}z{\isacharparenright}{\kern0pt}{\isacartoucheclose}\ {\isacartoucheopen}Ord{\isacharparenleft}{\kern0pt}y{\isacharparenright}{\kern0pt}{\isacartoucheclose}\ \isacommand{by}\isamarkupfalse%
\ simp\isanewline
\isacommand{qed}\isamarkupfalse%
%
\endisatagproof
{\isafoldproof}%
%
\isadelimproof
\isanewline
%
\endisadelimproof
\isanewline
\isacommand{lemma}\isamarkupfalse%
\ max{\isacharunderscore}{\kern0pt}D{\isadigit{1}}\ {\isacharcolon}{\kern0pt}\isanewline
\ \ \isakeyword{assumes}\ {\isachardoublequoteopen}x\ {\isacharequal}{\kern0pt}\ y{\isachardoublequoteclose}\ {\isachardoublequoteopen}w\ {\isacharless}{\kern0pt}\ z{\isachardoublequoteclose}\ \ {\isachardoublequoteopen}Ord{\isacharparenleft}{\kern0pt}x{\isacharparenright}{\kern0pt}{\isachardoublequoteclose}\ \ {\isachardoublequoteopen}Ord{\isacharparenleft}{\kern0pt}w{\isacharparenright}{\kern0pt}{\isachardoublequoteclose}\ {\isachardoublequoteopen}Ord{\isacharparenleft}{\kern0pt}z{\isacharparenright}{\kern0pt}{\isachardoublequoteclose}\ {\isachardoublequoteopen}max{\isacharparenleft}{\kern0pt}x{\isacharcomma}{\kern0pt}w{\isacharparenright}{\kern0pt}\ {\isacharequal}{\kern0pt}\ max{\isacharparenleft}{\kern0pt}y{\isacharcomma}{\kern0pt}z{\isacharparenright}{\kern0pt}{\isachardoublequoteclose}\isanewline
\ \ \isakeyword{shows}\ {\isachardoublequoteopen}z{\isasymle}y{\isachardoublequoteclose}\isanewline
%
\isadelimproof
%
\endisadelimproof
%
\isatagproof
\isacommand{proof}\isamarkupfalse%
\ {\isacharminus}{\kern0pt}\isanewline
\ \ \isacommand{from}\isamarkupfalse%
\ assms\isanewline
\ \ \isacommand{have}\isamarkupfalse%
\ {\isachardoublequoteopen}w\ {\isacharless}{\kern0pt}\ \ x\ {\isasymunion}\ w{\isachardoublequoteclose}\ \isacommand{using}\isamarkupfalse%
\ Un{\isacharunderscore}{\kern0pt}upper{\isadigit{2}}{\isacharunderscore}{\kern0pt}lt{\isacharbrackleft}{\kern0pt}OF\ {\isacartoucheopen}w{\isacharless}{\kern0pt}z{\isacartoucheclose}{\isacharbrackright}{\kern0pt}\ assms\ nat{\isacharunderscore}{\kern0pt}simp{\isacharunderscore}{\kern0pt}union\ \isacommand{by}\isamarkupfalse%
\ simp\isanewline
\ \ \isacommand{then}\isamarkupfalse%
\isanewline
\ \ \isacommand{have}\isamarkupfalse%
\ {\isachardoublequoteopen}w\ {\isacharless}{\kern0pt}\ x{\isachardoublequoteclose}\ \isacommand{using}\isamarkupfalse%
\ assms\ lt{\isacharunderscore}{\kern0pt}Un{\isacharunderscore}{\kern0pt}iff{\isacharbrackleft}{\kern0pt}of\ x\ w\ w{\isacharbrackright}{\kern0pt}\ lt{\isacharunderscore}{\kern0pt}not{\isacharunderscore}{\kern0pt}refl\ \isacommand{by}\isamarkupfalse%
\ auto\isanewline
\ \ \isacommand{then}\isamarkupfalse%
\ \isanewline
\ \ \isacommand{have}\isamarkupfalse%
\ {\isachardoublequoteopen}y\ {\isacharequal}{\kern0pt}\ y\ {\isasymunion}\ z{\isachardoublequoteclose}\ \isacommand{using}\isamarkupfalse%
\ assms\ max{\isacharunderscore}{\kern0pt}commutes\ nat{\isacharunderscore}{\kern0pt}simp{\isacharunderscore}{\kern0pt}union\ assms\ leI\ \isacommand{by}\isamarkupfalse%
\ simp\ \isanewline
\ \ \isacommand{then}\isamarkupfalse%
\ \isanewline
\ \ \isacommand{show}\isamarkupfalse%
\ {\isacharquery}{\kern0pt}thesis\ \isacommand{using}\isamarkupfalse%
\ Un{\isacharunderscore}{\kern0pt}leD{\isadigit{2}}\ assms\ \isacommand{by}\isamarkupfalse%
\ simp\isanewline
\isacommand{qed}\isamarkupfalse%
%
\endisatagproof
{\isafoldproof}%
%
\isadelimproof
\isanewline
%
\endisadelimproof
\isanewline
\isacommand{lemma}\isamarkupfalse%
\ max{\isacharunderscore}{\kern0pt}D{\isadigit{2}}\ {\isacharcolon}{\kern0pt}\isanewline
\ \ \isakeyword{assumes}\ {\isachardoublequoteopen}w\ {\isacharequal}{\kern0pt}\ y\ {\isasymor}\ w\ {\isacharequal}{\kern0pt}\ z{\isachardoublequoteclose}\ {\isachardoublequoteopen}x\ {\isacharless}{\kern0pt}\ y{\isachardoublequoteclose}\ \ {\isachardoublequoteopen}Ord{\isacharparenleft}{\kern0pt}x{\isacharparenright}{\kern0pt}{\isachardoublequoteclose}\ \ {\isachardoublequoteopen}Ord{\isacharparenleft}{\kern0pt}w{\isacharparenright}{\kern0pt}{\isachardoublequoteclose}\ {\isachardoublequoteopen}Ord{\isacharparenleft}{\kern0pt}y{\isacharparenright}{\kern0pt}{\isachardoublequoteclose}\ {\isachardoublequoteopen}Ord{\isacharparenleft}{\kern0pt}z{\isacharparenright}{\kern0pt}{\isachardoublequoteclose}\ {\isachardoublequoteopen}max{\isacharparenleft}{\kern0pt}x{\isacharcomma}{\kern0pt}w{\isacharparenright}{\kern0pt}\ {\isacharequal}{\kern0pt}\ max{\isacharparenleft}{\kern0pt}y{\isacharcomma}{\kern0pt}z{\isacharparenright}{\kern0pt}{\isachardoublequoteclose}\isanewline
\ \ \isakeyword{shows}\ {\isachardoublequoteopen}x{\isacharless}{\kern0pt}w{\isachardoublequoteclose}\isanewline
%
\isadelimproof
%
\endisadelimproof
%
\isatagproof
\isacommand{proof}\isamarkupfalse%
\ {\isacharminus}{\kern0pt}\isanewline
\ \ \isacommand{from}\isamarkupfalse%
\ assms\isanewline
\ \ \isacommand{have}\isamarkupfalse%
\ {\isachardoublequoteopen}x\ {\isacharless}{\kern0pt}\ z\ {\isasymunion}\ y{\isachardoublequoteclose}\ \isacommand{using}\isamarkupfalse%
\ Un{\isacharunderscore}{\kern0pt}upper{\isadigit{2}}{\isacharunderscore}{\kern0pt}lt{\isacharbrackleft}{\kern0pt}OF\ {\isacartoucheopen}x{\isacharless}{\kern0pt}y{\isacartoucheclose}{\isacharbrackright}{\kern0pt}\ \isacommand{by}\isamarkupfalse%
\ simp\isanewline
\ \ \isacommand{then}\isamarkupfalse%
\isanewline
\ \ \isacommand{consider}\isamarkupfalse%
\ {\isacharparenleft}{\kern0pt}a{\isacharparenright}{\kern0pt}\ {\isachardoublequoteopen}x\ {\isacharless}{\kern0pt}\ y{\isachardoublequoteclose}\ {\isacharbar}{\kern0pt}\ {\isacharparenleft}{\kern0pt}b{\isacharparenright}{\kern0pt}\ {\isachardoublequoteopen}x\ {\isacharless}{\kern0pt}\ w{\isachardoublequoteclose}\isanewline
\ \ \ \ \isacommand{using}\isamarkupfalse%
\ assms\ nat{\isacharunderscore}{\kern0pt}simp{\isacharunderscore}{\kern0pt}union\ \isacommand{by}\isamarkupfalse%
\ simp\isanewline
\ \ \isacommand{then}\isamarkupfalse%
\ \isacommand{show}\isamarkupfalse%
\ {\isacharquery}{\kern0pt}thesis\ \isacommand{proof}\isamarkupfalse%
\ {\isacharparenleft}{\kern0pt}cases{\isacharparenright}{\kern0pt}\isanewline
\ \ \ \ \isacommand{case}\isamarkupfalse%
\ a\isanewline
\ \ \ \ \isacommand{consider}\isamarkupfalse%
\ {\isacharparenleft}{\kern0pt}c{\isacharparenright}{\kern0pt}\ {\isachardoublequoteopen}w\ {\isacharequal}{\kern0pt}\ y{\isachardoublequoteclose}\ {\isacharbar}{\kern0pt}\ {\isacharparenleft}{\kern0pt}d{\isacharparenright}{\kern0pt}\ {\isachardoublequoteopen}w\ {\isacharequal}{\kern0pt}\ z{\isachardoublequoteclose}\ \isanewline
\ \ \ \ \ \ \isacommand{using}\isamarkupfalse%
\ assms\ \isacommand{by}\isamarkupfalse%
\ auto\isanewline
\ \ \ \ \isacommand{then}\isamarkupfalse%
\ \isacommand{show}\isamarkupfalse%
\ {\isacharquery}{\kern0pt}thesis\ \isacommand{proof}\isamarkupfalse%
\ {\isacharparenleft}{\kern0pt}cases{\isacharparenright}{\kern0pt}\isanewline
\ \ \ \ \ \ \isacommand{case}\isamarkupfalse%
\ c\isanewline
\ \ \ \ \ \ \isacommand{with}\isamarkupfalse%
\ a\ \isacommand{show}\isamarkupfalse%
\ {\isacharquery}{\kern0pt}thesis\ \isacommand{by}\isamarkupfalse%
\ simp\isanewline
\ \ \ \ \isacommand{next}\isamarkupfalse%
\isanewline
\ \ \ \ \ \ \isacommand{case}\isamarkupfalse%
\ d\isanewline
\ \ \ \ \ \ \isacommand{with}\isamarkupfalse%
\ a\isanewline
\ \ \ \ \ \ \isacommand{show}\isamarkupfalse%
\ {\isacharquery}{\kern0pt}thesis\ \isanewline
\ \ \ \ \ \ \isacommand{proof}\isamarkupfalse%
\ {\isacharparenleft}{\kern0pt}cases\ {\isachardoublequoteopen}y\ {\isacharless}{\kern0pt}w{\isachardoublequoteclose}{\isacharparenright}{\kern0pt}\isanewline
\ \ \ \ \ \ \ \ \isacommand{case}\isamarkupfalse%
\ True\ \ \ \ \ \ \ \isanewline
\ \ \ \ \ \ \ \ \isacommand{then}\isamarkupfalse%
\ \isacommand{show}\isamarkupfalse%
\ {\isacharquery}{\kern0pt}thesis\ \isacommand{using}\isamarkupfalse%
\ lt{\isacharunderscore}{\kern0pt}trans{\isacharbrackleft}{\kern0pt}OF\ {\isacartoucheopen}x{\isacharless}{\kern0pt}y{\isacartoucheclose}{\isacharbrackright}{\kern0pt}\ \isacommand{by}\isamarkupfalse%
\ simp\isanewline
\ \ \ \ \ \ \isacommand{next}\isamarkupfalse%
\isanewline
\ \ \ \ \ \ \ \ \isacommand{case}\isamarkupfalse%
\ False\isanewline
\ \ \ \ \ \ \ \ \isacommand{then}\isamarkupfalse%
\isanewline
\ \ \ \ \ \ \ \ \isacommand{have}\isamarkupfalse%
\ {\isachardoublequoteopen}w\ {\isasymle}\ y{\isachardoublequoteclose}\ \isanewline
\ \ \ \ \ \ \ \ \ \ \isacommand{using}\isamarkupfalse%
\ not{\isacharunderscore}{\kern0pt}lt{\isacharunderscore}{\kern0pt}iff{\isacharunderscore}{\kern0pt}le{\isacharbrackleft}{\kern0pt}OF\ assms{\isacharparenleft}{\kern0pt}{\isadigit{5}}{\isacharparenright}{\kern0pt}\ assms{\isacharparenleft}{\kern0pt}{\isadigit{4}}{\isacharparenright}{\kern0pt}{\isacharbrackright}{\kern0pt}\ \isacommand{by}\isamarkupfalse%
\ simp\isanewline
\ \ \ \ \ \ \ \ \isacommand{with}\isamarkupfalse%
\ {\isacartoucheopen}w{\isacharequal}{\kern0pt}z{\isacartoucheclose}\isanewline
\ \ \ \ \ \ \ \ \isacommand{have}\isamarkupfalse%
\ {\isachardoublequoteopen}max{\isacharparenleft}{\kern0pt}z{\isacharcomma}{\kern0pt}y{\isacharparenright}{\kern0pt}\ {\isacharequal}{\kern0pt}\ y{\isachardoublequoteclose}\ \ \isacommand{unfolding}\isamarkupfalse%
\ max{\isacharunderscore}{\kern0pt}def\ \isacommand{using}\isamarkupfalse%
\ assms\ \isacommand{by}\isamarkupfalse%
\ simp\isanewline
\ \ \ \ \ \ \ \ \isacommand{with}\isamarkupfalse%
\ assms\isanewline
\ \ \ \ \ \ \ \ \isacommand{have}\isamarkupfalse%
\ {\isachardoublequoteopen}{\isachardot}{\kern0pt}{\isachardot}{\kern0pt}{\isachardot}{\kern0pt}\ {\isacharequal}{\kern0pt}\ x\ {\isasymunion}\ w{\isachardoublequoteclose}\ \isacommand{using}\isamarkupfalse%
\ nat{\isacharunderscore}{\kern0pt}simp{\isacharunderscore}{\kern0pt}union\ max{\isacharunderscore}{\kern0pt}commutes\ \ \isacommand{by}\isamarkupfalse%
\ simp\isanewline
\ \ \ \ \ \ \ \ \isacommand{then}\isamarkupfalse%
\ \isacommand{show}\isamarkupfalse%
\ {\isacharquery}{\kern0pt}thesis\ \isacommand{using}\isamarkupfalse%
\ le{\isacharunderscore}{\kern0pt}Un{\isacharunderscore}{\kern0pt}iff\ assms\ \isacommand{by}\isamarkupfalse%
\ blast\isanewline
\ \ \ \ \ \ \isacommand{qed}\isamarkupfalse%
\isanewline
\ \ \ \ \isacommand{qed}\isamarkupfalse%
\isanewline
\ \ \isacommand{next}\isamarkupfalse%
\isanewline
\ \ \ \ \isacommand{case}\isamarkupfalse%
\ b\isanewline
\ \ \ \ \isacommand{then}\isamarkupfalse%
\ \isacommand{show}\isamarkupfalse%
\ {\isacharquery}{\kern0pt}thesis\ \isacommand{{\isachardot}{\kern0pt}}\isamarkupfalse%
\isanewline
\ \ \isacommand{qed}\isamarkupfalse%
\isanewline
\isacommand{qed}\isamarkupfalse%
%
\endisatagproof
{\isafoldproof}%
%
\isadelimproof
\isanewline
%
\endisadelimproof
\isanewline
\isacommand{lemma}\isamarkupfalse%
\ oadd{\isacharunderscore}{\kern0pt}lt{\isacharunderscore}{\kern0pt}mono{\isadigit{2}}\ {\isacharcolon}{\kern0pt}\isanewline
\ \ \isakeyword{assumes}\ \ {\isachardoublequoteopen}Ord{\isacharparenleft}{\kern0pt}n{\isacharparenright}{\kern0pt}{\isachardoublequoteclose}\ {\isachardoublequoteopen}Ord{\isacharparenleft}{\kern0pt}{\isasymalpha}{\isacharparenright}{\kern0pt}{\isachardoublequoteclose}\ {\isachardoublequoteopen}Ord{\isacharparenleft}{\kern0pt}{\isasymbeta}{\isacharparenright}{\kern0pt}{\isachardoublequoteclose}\ {\isachardoublequoteopen}{\isasymalpha}\ {\isacharless}{\kern0pt}\ {\isasymbeta}{\isachardoublequoteclose}\ {\isachardoublequoteopen}x\ {\isacharless}{\kern0pt}\ n{\isachardoublequoteclose}\ {\isachardoublequoteopen}y\ {\isacharless}{\kern0pt}\ n{\isachardoublequoteclose}\ {\isachardoublequoteopen}{\isadigit{0}}\ {\isacharless}{\kern0pt}n{\isachardoublequoteclose}\isanewline
\ \ \isakeyword{shows}\ {\isachardoublequoteopen}n\ {\isacharasterisk}{\kern0pt}{\isacharasterisk}{\kern0pt}\ {\isasymalpha}\ {\isacharplus}{\kern0pt}{\isacharplus}{\kern0pt}\ x\ {\isacharless}{\kern0pt}\ n\ {\isacharasterisk}{\kern0pt}{\isacharasterisk}{\kern0pt}{\isasymbeta}\ {\isacharplus}{\kern0pt}{\isacharplus}{\kern0pt}\ y{\isachardoublequoteclose}\isanewline
%
\isadelimproof
%
\endisadelimproof
%
\isatagproof
\isacommand{proof}\isamarkupfalse%
\ {\isacharminus}{\kern0pt}\isanewline
\ \ \isacommand{consider}\isamarkupfalse%
\ {\isacharparenleft}{\kern0pt}{\isadigit{0}}{\isacharparenright}{\kern0pt}\ {\isachardoublequoteopen}{\isasymbeta}{\isacharequal}{\kern0pt}{\isadigit{0}}{\isachardoublequoteclose}\ {\isacharbar}{\kern0pt}\ {\isacharparenleft}{\kern0pt}s{\isacharparenright}{\kern0pt}\ {\isasymgamma}\ \isakeyword{where}\ \ {\isachardoublequoteopen}Ord{\isacharparenleft}{\kern0pt}{\isasymgamma}{\isacharparenright}{\kern0pt}{\isachardoublequoteclose}\ {\isachardoublequoteopen}{\isasymbeta}\ {\isacharequal}{\kern0pt}\ succ{\isacharparenleft}{\kern0pt}{\isasymgamma}{\isacharparenright}{\kern0pt}{\isachardoublequoteclose}\ {\isacharbar}{\kern0pt}\ {\isacharparenleft}{\kern0pt}l{\isacharparenright}{\kern0pt}\ {\isachardoublequoteopen}Limit{\isacharparenleft}{\kern0pt}{\isasymbeta}{\isacharparenright}{\kern0pt}{\isachardoublequoteclose}\isanewline
\ \ \ \ \isacommand{using}\isamarkupfalse%
\ Ord{\isacharunderscore}{\kern0pt}cases{\isacharbrackleft}{\kern0pt}OF\ {\isacartoucheopen}Ord{\isacharparenleft}{\kern0pt}{\isasymbeta}{\isacharparenright}{\kern0pt}{\isacartoucheclose}{\isacharcomma}{\kern0pt}of\ {\isacharquery}{\kern0pt}thesis{\isacharbrackright}{\kern0pt}\ \isacommand{by}\isamarkupfalse%
\ force\isanewline
\ \ \isacommand{then}\isamarkupfalse%
\ \isacommand{show}\isamarkupfalse%
\ {\isacharquery}{\kern0pt}thesis\ \isanewline
\ \ \isacommand{proof}\isamarkupfalse%
\ cases\isanewline
\ \ \ \ \isacommand{case}\isamarkupfalse%
\ {\isadigit{0}}\isanewline
\ \ \ \ \isacommand{then}\isamarkupfalse%
\ \isacommand{show}\isamarkupfalse%
\ {\isacharquery}{\kern0pt}thesis\ \isacommand{using}\isamarkupfalse%
\ {\isacartoucheopen}{\isasymalpha}{\isacharless}{\kern0pt}{\isasymbeta}{\isacartoucheclose}\ \isacommand{by}\isamarkupfalse%
\ auto\isanewline
\ \ \isacommand{next}\isamarkupfalse%
\isanewline
\ \ \ \ \isacommand{case}\isamarkupfalse%
\ s\isanewline
\ \ \ \ \isacommand{then}\isamarkupfalse%
\isanewline
\ \ \ \ \isacommand{have}\isamarkupfalse%
\ {\isachardoublequoteopen}{\isasymalpha}{\isasymle}{\isasymgamma}{\isachardoublequoteclose}\ \isacommand{using}\isamarkupfalse%
\ {\isacartoucheopen}{\isasymalpha}{\isacharless}{\kern0pt}{\isasymbeta}{\isacartoucheclose}\ \isacommand{using}\isamarkupfalse%
\ leI\ \isacommand{by}\isamarkupfalse%
\ auto\isanewline
\ \ \ \ \isacommand{then}\isamarkupfalse%
\isanewline
\ \ \ \ \isacommand{have}\isamarkupfalse%
\ {\isachardoublequoteopen}n\ {\isacharasterisk}{\kern0pt}{\isacharasterisk}{\kern0pt}\ {\isasymalpha}\ {\isasymle}\ n\ {\isacharasterisk}{\kern0pt}{\isacharasterisk}{\kern0pt}\ {\isasymgamma}{\isachardoublequoteclose}\ \isacommand{using}\isamarkupfalse%
\ omult{\isacharunderscore}{\kern0pt}le{\isacharunderscore}{\kern0pt}mono{\isacharbrackleft}{\kern0pt}OF\ {\isacharunderscore}{\kern0pt}\ {\isacartoucheopen}{\isasymalpha}{\isasymle}{\isasymgamma}{\isacartoucheclose}{\isacharbrackright}{\kern0pt}\ {\isacartoucheopen}Ord{\isacharparenleft}{\kern0pt}n{\isacharparenright}{\kern0pt}{\isacartoucheclose}\ \isacommand{by}\isamarkupfalse%
\ simp\isanewline
\ \ \ \ \isacommand{then}\isamarkupfalse%
\isanewline
\ \ \ \ \isacommand{have}\isamarkupfalse%
\ {\isachardoublequoteopen}n\ {\isacharasterisk}{\kern0pt}{\isacharasterisk}{\kern0pt}\ {\isasymalpha}\ {\isacharplus}{\kern0pt}{\isacharplus}{\kern0pt}\ x\ {\isacharless}{\kern0pt}\ n\ {\isacharasterisk}{\kern0pt}{\isacharasterisk}{\kern0pt}\ {\isasymgamma}\ {\isacharplus}{\kern0pt}{\isacharplus}{\kern0pt}\ n{\isachardoublequoteclose}\ \isacommand{using}\isamarkupfalse%
\ oadd{\isacharunderscore}{\kern0pt}lt{\isacharunderscore}{\kern0pt}mono{\isacharbrackleft}{\kern0pt}OF\ {\isacharunderscore}{\kern0pt}\ {\isacartoucheopen}x{\isacharless}{\kern0pt}n{\isacartoucheclose}{\isacharbrackright}{\kern0pt}\ \isacommand{by}\isamarkupfalse%
\ simp\isanewline
\ \ \ \ \isacommand{also}\isamarkupfalse%
\isanewline
\ \ \ \ \isacommand{have}\isamarkupfalse%
\ {\isachardoublequoteopen}{\isachardot}{\kern0pt}{\isachardot}{\kern0pt}{\isachardot}{\kern0pt}\ {\isacharequal}{\kern0pt}\ n\ {\isacharasterisk}{\kern0pt}{\isacharasterisk}{\kern0pt}\ {\isasymbeta}{\isachardoublequoteclose}\ \isacommand{using}\isamarkupfalse%
\ {\isacartoucheopen}{\isasymbeta}{\isacharequal}{\kern0pt}succ{\isacharparenleft}{\kern0pt}{\isacharunderscore}{\kern0pt}{\isacharparenright}{\kern0pt}{\isacartoucheclose}\ omult{\isacharunderscore}{\kern0pt}succ\ {\isacartoucheopen}Ord{\isacharparenleft}{\kern0pt}{\isasymbeta}{\isacharparenright}{\kern0pt}{\isacartoucheclose}\ {\isacartoucheopen}Ord{\isacharparenleft}{\kern0pt}n{\isacharparenright}{\kern0pt}{\isacartoucheclose}\ \isacommand{by}\isamarkupfalse%
\ simp\isanewline
\ \ \ \ \isacommand{finally}\isamarkupfalse%
\isanewline
\ \ \ \ \isacommand{have}\isamarkupfalse%
\ {\isachardoublequoteopen}n\ {\isacharasterisk}{\kern0pt}{\isacharasterisk}{\kern0pt}\ {\isasymalpha}\ {\isacharplus}{\kern0pt}{\isacharplus}{\kern0pt}\ x\ {\isacharless}{\kern0pt}\ n\ {\isacharasterisk}{\kern0pt}{\isacharasterisk}{\kern0pt}\ {\isasymbeta}{\isachardoublequoteclose}\ \isacommand{by}\isamarkupfalse%
\ auto\isanewline
\ \ \ \ \isacommand{then}\isamarkupfalse%
\isanewline
\ \ \ \ \isacommand{show}\isamarkupfalse%
\ {\isacharquery}{\kern0pt}thesis\ \isacommand{using}\isamarkupfalse%
\ oadd{\isacharunderscore}{\kern0pt}le{\isacharunderscore}{\kern0pt}self\ {\isacartoucheopen}Ord{\isacharparenleft}{\kern0pt}{\isasymbeta}{\isacharparenright}{\kern0pt}{\isacartoucheclose}\ lt{\isacharunderscore}{\kern0pt}trans{\isadigit{2}}\ {\isacartoucheopen}Ord{\isacharparenleft}{\kern0pt}n{\isacharparenright}{\kern0pt}{\isacartoucheclose}\ \isacommand{by}\isamarkupfalse%
\ auto\isanewline
\ \ \isacommand{next}\isamarkupfalse%
\isanewline
\ \ \ \ \isacommand{case}\isamarkupfalse%
\ l\isanewline
\ \ \ \ \isacommand{have}\isamarkupfalse%
\ {\isachardoublequoteopen}Ord{\isacharparenleft}{\kern0pt}x{\isacharparenright}{\kern0pt}{\isachardoublequoteclose}\ \isacommand{using}\isamarkupfalse%
\ {\isacartoucheopen}x{\isacharless}{\kern0pt}n{\isacartoucheclose}\ lt{\isacharunderscore}{\kern0pt}Ord\ \isacommand{by}\isamarkupfalse%
\ simp\isanewline
\ \ \ \ \isacommand{with}\isamarkupfalse%
\ l\isanewline
\ \ \ \ \isacommand{have}\isamarkupfalse%
\ {\isachardoublequoteopen}succ{\isacharparenleft}{\kern0pt}{\isasymalpha}{\isacharparenright}{\kern0pt}\ {\isacharless}{\kern0pt}\ {\isasymbeta}{\isachardoublequoteclose}\ \isacommand{using}\isamarkupfalse%
\ Limit{\isacharunderscore}{\kern0pt}has{\isacharunderscore}{\kern0pt}succ\ {\isacartoucheopen}{\isasymalpha}{\isacharless}{\kern0pt}{\isasymbeta}{\isacartoucheclose}\ \isacommand{by}\isamarkupfalse%
\ simp\isanewline
\ \ \ \ \isacommand{have}\isamarkupfalse%
\ {\isachardoublequoteopen}n\ {\isacharasterisk}{\kern0pt}{\isacharasterisk}{\kern0pt}\ {\isasymalpha}\ {\isacharplus}{\kern0pt}{\isacharplus}{\kern0pt}\ x\ {\isacharless}{\kern0pt}\ n\ {\isacharasterisk}{\kern0pt}{\isacharasterisk}{\kern0pt}\ {\isasymalpha}\ {\isacharplus}{\kern0pt}{\isacharplus}{\kern0pt}\ n{\isachardoublequoteclose}\ \isanewline
\ \ \ \ \ \ \isacommand{using}\isamarkupfalse%
\ oadd{\isacharunderscore}{\kern0pt}lt{\isacharunderscore}{\kern0pt}mono{\isacharbrackleft}{\kern0pt}OF\ le{\isacharunderscore}{\kern0pt}refl{\isacharbrackleft}{\kern0pt}OF\ Ord{\isacharunderscore}{\kern0pt}omult{\isacharbrackleft}{\kern0pt}OF\ {\isacharunderscore}{\kern0pt}\ {\isacartoucheopen}Ord{\isacharparenleft}{\kern0pt}{\isasymalpha}{\isacharparenright}{\kern0pt}{\isacartoucheclose}{\isacharbrackright}{\kern0pt}{\isacharbrackright}{\kern0pt}\ {\isacartoucheopen}x{\isacharless}{\kern0pt}n{\isacartoucheclose}{\isacharbrackright}{\kern0pt}\ {\isacartoucheopen}Ord{\isacharparenleft}{\kern0pt}n{\isacharparenright}{\kern0pt}{\isacartoucheclose}\ \isacommand{by}\isamarkupfalse%
\ simp\isanewline
\ \ \ \ \isacommand{also}\isamarkupfalse%
\isanewline
\ \ \ \ \isacommand{have}\isamarkupfalse%
\ {\isachardoublequoteopen}{\isachardot}{\kern0pt}{\isachardot}{\kern0pt}{\isachardot}{\kern0pt}\ {\isacharequal}{\kern0pt}\ n\ {\isacharasterisk}{\kern0pt}{\isacharasterisk}{\kern0pt}\ succ{\isacharparenleft}{\kern0pt}{\isasymalpha}{\isacharparenright}{\kern0pt}{\isachardoublequoteclose}\ \isacommand{using}\isamarkupfalse%
\ omult{\isacharunderscore}{\kern0pt}succ\ {\isacartoucheopen}Ord{\isacharparenleft}{\kern0pt}{\isasymalpha}{\isacharparenright}{\kern0pt}{\isacartoucheclose}\ {\isacartoucheopen}Ord{\isacharparenleft}{\kern0pt}n{\isacharparenright}{\kern0pt}{\isacartoucheclose}\ \isacommand{by}\isamarkupfalse%
\ simp\isanewline
\ \ \ \ \isacommand{finally}\isamarkupfalse%
\isanewline
\ \ \ \ \isacommand{have}\isamarkupfalse%
\ {\isachardoublequoteopen}n\ {\isacharasterisk}{\kern0pt}{\isacharasterisk}{\kern0pt}\ {\isasymalpha}\ {\isacharplus}{\kern0pt}{\isacharplus}{\kern0pt}\ x\ {\isacharless}{\kern0pt}\ n\ {\isacharasterisk}{\kern0pt}{\isacharasterisk}{\kern0pt}\ succ{\isacharparenleft}{\kern0pt}{\isasymalpha}{\isacharparenright}{\kern0pt}{\isachardoublequoteclose}\ \isacommand{by}\isamarkupfalse%
\ simp\ \isanewline
\ \ \ \ \isacommand{with}\isamarkupfalse%
\ {\isacartoucheopen}succ{\isacharparenleft}{\kern0pt}{\isasymalpha}{\isacharparenright}{\kern0pt}\ {\isacharless}{\kern0pt}\ {\isasymbeta}{\isacartoucheclose}\isanewline
\ \ \ \ \isacommand{have}\isamarkupfalse%
\ {\isachardoublequoteopen}n\ {\isacharasterisk}{\kern0pt}{\isacharasterisk}{\kern0pt}\ {\isasymalpha}\ {\isacharplus}{\kern0pt}{\isacharplus}{\kern0pt}\ x\ {\isacharless}{\kern0pt}\ n\ {\isacharasterisk}{\kern0pt}{\isacharasterisk}{\kern0pt}\ {\isasymbeta}{\isachardoublequoteclose}\ \isacommand{using}\isamarkupfalse%
\ lt{\isacharunderscore}{\kern0pt}trans\ omult{\isacharunderscore}{\kern0pt}lt{\isacharunderscore}{\kern0pt}mono\ {\isacartoucheopen}Ord{\isacharparenleft}{\kern0pt}n{\isacharparenright}{\kern0pt}{\isacartoucheclose}\ {\isacartoucheopen}{\isadigit{0}}{\isacharless}{\kern0pt}n{\isacartoucheclose}\ \ \isacommand{by}\isamarkupfalse%
\ auto\ \ \ \ \ \ \isanewline
\ \ \ \ \isacommand{then}\isamarkupfalse%
\ \isacommand{show}\isamarkupfalse%
\ {\isacharquery}{\kern0pt}thesis\ \isacommand{using}\isamarkupfalse%
\ oadd{\isacharunderscore}{\kern0pt}le{\isacharunderscore}{\kern0pt}self\ {\isacartoucheopen}Ord{\isacharparenleft}{\kern0pt}{\isasymbeta}{\isacharparenright}{\kern0pt}{\isacartoucheclose}\ lt{\isacharunderscore}{\kern0pt}trans{\isadigit{2}}\ {\isacartoucheopen}Ord{\isacharparenleft}{\kern0pt}n{\isacharparenright}{\kern0pt}{\isacartoucheclose}\ \isacommand{by}\isamarkupfalse%
\ auto\isanewline
\ \ \isacommand{qed}\isamarkupfalse%
\isanewline
\isacommand{qed}\isamarkupfalse%
%
\endisatagproof
{\isafoldproof}%
%
\isadelimproof
\isanewline
%
\endisadelimproof
%
\isadelimtheory
%
\endisadelimtheory
%
\isatagtheory
\isacommand{end}\isamarkupfalse%
%
\endisatagtheory
{\isafoldtheory}%
%
\isadelimtheory
%
\endisadelimtheory
%
\end{isabellebody}%
\endinput
%:%file=~/source/repos/ZF-notAC/code/Forcing/Nat_Miscellanea.thy%:%
%:%11=1%:%
%:%27=2%:%
%:%28=2%:%
%:%37=4%:%
%:%38=5%:%
%:%40=6%:%
%:%41=6%:%
%:%42=7%:%
%:%43=8%:%
%:%44=8%:%
%:%47=9%:%
%:%51=9%:%
%:%52=9%:%
%:%57=9%:%
%:%60=10%:%
%:%61=11%:%
%:%62=11%:%
%:%63=12%:%
%:%64=13%:%
%:%65=13%:%
%:%68=14%:%
%:%72=14%:%
%:%73=14%:%
%:%78=14%:%
%:%81=15%:%
%:%82=16%:%
%:%83=16%:%
%:%86=17%:%
%:%90=17%:%
%:%91=17%:%
%:%96=17%:%
%:%99=18%:%
%:%100=19%:%
%:%101=19%:%
%:%104=20%:%
%:%108=20%:%
%:%109=20%:%
%:%114=20%:%
%:%117=21%:%
%:%118=22%:%
%:%119=22%:%
%:%122=23%:%
%:%126=23%:%
%:%127=23%:%
%:%132=23%:%
%:%135=24%:%
%:%136=25%:%
%:%137=25%:%
%:%140=26%:%
%:%144=26%:%
%:%145=26%:%
%:%150=26%:%
%:%153=27%:%
%:%154=28%:%
%:%155=28%:%
%:%158=29%:%
%:%162=29%:%
%:%163=29%:%
%:%168=29%:%
%:%171=30%:%
%:%172=31%:%
%:%173=31%:%
%:%174=32%:%
%:%175=33%:%
%:%176=33%:%
%:%179=34%:%
%:%183=34%:%
%:%184=34%:%
%:%189=34%:%
%:%192=35%:%
%:%193=36%:%
%:%194=36%:%
%:%197=37%:%
%:%201=37%:%
%:%202=37%:%
%:%207=37%:%
%:%210=38%:%
%:%211=39%:%
%:%212=40%:%
%:%213=40%:%
%:%216=41%:%
%:%220=41%:%
%:%221=41%:%
%:%226=41%:%
%:%229=42%:%
%:%230=43%:%
%:%231=43%:%
%:%232=44%:%
%:%233=45%:%
%:%234=45%:%
%:%237=46%:%
%:%241=46%:%
%:%242=46%:%
%:%247=46%:%
%:%250=47%:%
%:%251=48%:%
%:%252=48%:%
%:%253=49%:%
%:%254=50%:%
%:%255=50%:%
%:%258=51%:%
%:%262=51%:%
%:%263=51%:%
%:%268=51%:%
%:%271=52%:%
%:%272=53%:%
%:%273=53%:%
%:%276=54%:%
%:%280=54%:%
%:%281=54%:%
%:%286=54%:%
%:%289=55%:%
%:%290=56%:%
%:%291=56%:%
%:%294=57%:%
%:%298=57%:%
%:%299=57%:%
%:%304=57%:%
%:%307=58%:%
%:%308=59%:%
%:%309=59%:%
%:%312=60%:%
%:%316=60%:%
%:%317=60%:%
%:%322=60%:%
%:%325=61%:%
%:%326=62%:%
%:%327=62%:%
%:%330=63%:%
%:%334=63%:%
%:%335=63%:%
%:%340=63%:%
%:%343=64%:%
%:%344=65%:%
%:%345=65%:%
%:%348=66%:%
%:%352=66%:%
%:%353=66%:%
%:%358=66%:%
%:%361=67%:%
%:%362=68%:%
%:%363=68%:%
%:%366=69%:%
%:%370=69%:%
%:%371=69%:%
%:%376=69%:%
%:%379=70%:%
%:%380=71%:%
%:%381=71%:%
%:%382=72%:%
%:%383=73%:%
%:%386=74%:%
%:%390=74%:%
%:%391=74%:%
%:%392=74%:%
%:%397=74%:%
%:%400=75%:%
%:%401=76%:%
%:%402=76%:%
%:%403=77%:%
%:%406=78%:%
%:%410=78%:%
%:%411=78%:%
%:%416=78%:%
%:%419=79%:%
%:%420=80%:%
%:%421=80%:%
%:%422=81%:%
%:%425=82%:%
%:%429=82%:%
%:%430=82%:%
%:%435=82%:%
%:%438=83%:%
%:%439=84%:%
%:%440=84%:%
%:%443=85%:%
%:%447=85%:%
%:%448=85%:%
%:%449=86%:%
%:%450=86%:%
%:%455=86%:%
%:%458=87%:%
%:%459=88%:%
%:%460=88%:%
%:%463=89%:%
%:%467=89%:%
%:%468=89%:%
%:%469=89%:%
%:%474=89%:%
%:%477=90%:%
%:%478=91%:%
%:%479=91%:%
%:%482=92%:%
%:%486=92%:%
%:%487=92%:%
%:%492=92%:%
%:%495=93%:%
%:%496=94%:%
%:%497=94%:%
%:%498=95%:%
%:%499=96%:%
%:%500=96%:%
%:%502=96%:%
%:%506=96%:%
%:%507=96%:%
%:%514=96%:%
%:%515=97%:%
%:%516=97%:%
%:%519=98%:%
%:%523=98%:%
%:%524=98%:%
%:%525=99%:%
%:%526=99%:%
%:%531=99%:%
%:%534=100%:%
%:%535=101%:%
%:%536=101%:%
%:%539=102%:%
%:%543=102%:%
%:%544=102%:%
%:%545=103%:%
%:%546=104%:%
%:%547=104%:%
%:%552=104%:%
%:%555=105%:%
%:%556=106%:%
%:%557=106%:%
%:%559=108%:%
%:%562=109%:%
%:%566=109%:%
%:%567=109%:%
%:%568=109%:%
%:%573=109%:%
%:%576=110%:%
%:%577=111%:%
%:%578=111%:%
%:%580=113%:%
%:%583=114%:%
%:%587=114%:%
%:%588=114%:%
%:%589=114%:%
%:%594=114%:%
%:%597=115%:%
%:%598=116%:%
%:%599=116%:%
%:%600=117%:%
%:%601=118%:%
%:%608=119%:%
%:%609=119%:%
%:%610=120%:%
%:%611=120%:%
%:%612=121%:%
%:%613=121%:%
%:%614=122%:%
%:%615=122%:%
%:%616=122%:%
%:%617=123%:%
%:%618=123%:%
%:%619=124%:%
%:%620=124%:%
%:%621=125%:%
%:%622=125%:%
%:%623=125%:%
%:%624=126%:%
%:%630=126%:%
%:%633=127%:%
%:%634=128%:%
%:%635=128%:%
%:%636=129%:%
%:%637=130%:%
%:%640=131%:%
%:%644=131%:%
%:%645=131%:%
%:%646=131%:%
%:%651=131%:%
%:%654=132%:%
%:%655=133%:%
%:%656=133%:%
%:%659=134%:%
%:%663=134%:%
%:%664=134%:%
%:%669=134%:%
%:%672=135%:%
%:%673=136%:%
%:%674=136%:%
%:%677=137%:%
%:%681=137%:%
%:%682=137%:%
%:%687=137%:%
%:%690=138%:%
%:%691=139%:%
%:%692=139%:%
%:%693=140%:%
%:%694=141%:%
%:%697=142%:%
%:%701=142%:%
%:%702=142%:%
%:%703=142%:%
%:%708=142%:%
%:%711=143%:%
%:%712=144%:%
%:%713=144%:%
%:%714=145%:%
%:%717=146%:%
%:%721=146%:%
%:%722=146%:%
%:%723=146%:%
%:%737=148%:%
%:%749=149%:%
%:%750=150%:%
%:%752=152%:%
%:%753=152%:%
%:%754=153%:%
%:%757=154%:%
%:%761=154%:%
%:%762=154%:%
%:%763=155%:%
%:%764=155%:%
%:%765=156%:%
%:%766=156%:%
%:%767=157%:%
%:%768=157%:%
%:%769=157%:%
%:%770=158%:%
%:%771=158%:%
%:%772=158%:%
%:%773=158%:%
%:%774=159%:%
%:%775=159%:%
%:%776=160%:%
%:%777=160%:%
%:%778=161%:%
%:%779=161%:%
%:%780=161%:%
%:%781=161%:%
%:%782=161%:%
%:%783=162%:%
%:%784=162%:%
%:%785=162%:%
%:%786=163%:%
%:%787=163%:%
%:%788=163%:%
%:%789=163%:%
%:%790=164%:%
%:%796=164%:%
%:%799=165%:%
%:%800=166%:%
%:%801=166%:%
%:%802=167%:%
%:%803=168%:%
%:%806=169%:%
%:%810=169%:%
%:%811=169%:%
%:%812=169%:%
%:%817=169%:%
%:%820=170%:%
%:%821=171%:%
%:%822=171%:%
%:%823=172%:%
%:%824=173%:%
%:%831=174%:%
%:%832=174%:%
%:%833=175%:%
%:%834=175%:%
%:%835=176%:%
%:%836=176%:%
%:%837=177%:%
%:%838=177%:%
%:%839=177%:%
%:%840=178%:%
%:%841=178%:%
%:%842=178%:%
%:%843=179%:%
%:%844=179%:%
%:%845=179%:%
%:%846=180%:%
%:%852=180%:%
%:%855=181%:%
%:%856=182%:%
%:%857=182%:%
%:%858=183%:%
%:%859=184%:%
%:%866=185%:%
%:%867=185%:%
%:%868=186%:%
%:%869=186%:%
%:%870=187%:%
%:%871=187%:%
%:%872=187%:%
%:%873=187%:%
%:%874=188%:%
%:%875=188%:%
%:%876=189%:%
%:%877=189%:%
%:%878=189%:%
%:%879=189%:%
%:%880=190%:%
%:%881=190%:%
%:%882=191%:%
%:%883=191%:%
%:%884=191%:%
%:%885=191%:%
%:%886=192%:%
%:%887=192%:%
%:%888=193%:%
%:%889=193%:%
%:%890=193%:%
%:%891=193%:%
%:%892=194%:%
%:%898=194%:%
%:%901=195%:%
%:%902=196%:%
%:%903=196%:%
%:%904=197%:%
%:%905=198%:%
%:%912=199%:%
%:%913=199%:%
%:%914=200%:%
%:%915=200%:%
%:%916=201%:%
%:%917=201%:%
%:%918=201%:%
%:%919=201%:%
%:%920=202%:%
%:%921=202%:%
%:%922=203%:%
%:%923=203%:%
%:%924=204%:%
%:%925=204%:%
%:%926=204%:%
%:%927=205%:%
%:%928=205%:%
%:%929=205%:%
%:%930=205%:%
%:%931=206%:%
%:%932=206%:%
%:%933=207%:%
%:%934=207%:%
%:%935=208%:%
%:%936=208%:%
%:%937=208%:%
%:%938=209%:%
%:%939=209%:%
%:%940=209%:%
%:%941=209%:%
%:%942=210%:%
%:%943=210%:%
%:%944=211%:%
%:%945=211%:%
%:%946=211%:%
%:%947=211%:%
%:%948=212%:%
%:%949=212%:%
%:%950=213%:%
%:%951=213%:%
%:%952=214%:%
%:%953=214%:%
%:%954=215%:%
%:%955=215%:%
%:%956=216%:%
%:%957=216%:%
%:%958=217%:%
%:%959=217%:%
%:%960=218%:%
%:%961=218%:%
%:%962=218%:%
%:%963=218%:%
%:%964=218%:%
%:%965=219%:%
%:%966=219%:%
%:%967=220%:%
%:%968=220%:%
%:%969=221%:%
%:%970=221%:%
%:%971=222%:%
%:%972=222%:%
%:%973=223%:%
%:%974=223%:%
%:%975=223%:%
%:%976=224%:%
%:%977=224%:%
%:%978=225%:%
%:%979=225%:%
%:%980=225%:%
%:%981=225%:%
%:%982=225%:%
%:%983=226%:%
%:%984=226%:%
%:%985=227%:%
%:%986=227%:%
%:%987=227%:%
%:%988=227%:%
%:%989=228%:%
%:%990=228%:%
%:%991=228%:%
%:%992=228%:%
%:%993=228%:%
%:%994=229%:%
%:%995=229%:%
%:%996=230%:%
%:%997=230%:%
%:%998=231%:%
%:%999=231%:%
%:%1000=232%:%
%:%1001=232%:%
%:%1002=233%:%
%:%1003=233%:%
%:%1004=233%:%
%:%1005=233%:%
%:%1006=234%:%
%:%1007=234%:%
%:%1008=235%:%
%:%1014=235%:%
%:%1017=236%:%
%:%1018=237%:%
%:%1019=237%:%
%:%1020=238%:%
%:%1021=239%:%
%:%1028=240%:%
%:%1029=240%:%
%:%1030=241%:%
%:%1031=241%:%
%:%1032=242%:%
%:%1033=242%:%
%:%1034=242%:%
%:%1035=243%:%
%:%1036=243%:%
%:%1037=243%:%
%:%1038=244%:%
%:%1039=244%:%
%:%1040=245%:%
%:%1041=245%:%
%:%1042=246%:%
%:%1043=246%:%
%:%1044=246%:%
%:%1045=246%:%
%:%1046=246%:%
%:%1047=247%:%
%:%1048=247%:%
%:%1049=248%:%
%:%1050=248%:%
%:%1051=249%:%
%:%1052=249%:%
%:%1053=250%:%
%:%1054=250%:%
%:%1055=250%:%
%:%1056=250%:%
%:%1057=250%:%
%:%1058=251%:%
%:%1059=251%:%
%:%1060=252%:%
%:%1061=252%:%
%:%1062=252%:%
%:%1063=252%:%
%:%1064=253%:%
%:%1065=253%:%
%:%1066=254%:%
%:%1067=254%:%
%:%1068=254%:%
%:%1069=254%:%
%:%1070=255%:%
%:%1071=255%:%
%:%1072=256%:%
%:%1073=256%:%
%:%1074=256%:%
%:%1075=256%:%
%:%1076=257%:%
%:%1077=257%:%
%:%1078=258%:%
%:%1079=258%:%
%:%1080=258%:%
%:%1081=259%:%
%:%1082=259%:%
%:%1083=260%:%
%:%1084=260%:%
%:%1085=260%:%
%:%1086=260%:%
%:%1087=261%:%
%:%1088=261%:%
%:%1089=262%:%
%:%1090=262%:%
%:%1091=263%:%
%:%1092=263%:%
%:%1093=263%:%
%:%1094=263%:%
%:%1095=264%:%
%:%1096=264%:%
%:%1097=265%:%
%:%1098=265%:%
%:%1099=265%:%
%:%1100=265%:%
%:%1101=266%:%
%:%1102=266%:%
%:%1103=267%:%
%:%1104=267%:%
%:%1105=267%:%
%:%1106=268%:%
%:%1107=268%:%
%:%1108=269%:%
%:%1109=269%:%
%:%1110=269%:%
%:%1111=269%:%
%:%1112=270%:%
%:%1113=270%:%
%:%1114=271%:%
%:%1115=271%:%
%:%1116=271%:%
%:%1117=272%:%
%:%1118=272%:%
%:%1119=273%:%
%:%1120=273%:%
%:%1121=273%:%
%:%1122=273%:%
%:%1123=274%:%
%:%1124=274%:%
%:%1125=274%:%
%:%1126=274%:%
%:%1127=274%:%
%:%1128=275%:%
%:%1129=275%:%
%:%1130=276%:%
%:%1136=276%:%
%:%1145=277%:%

%
\begin{isabellebody}%
\setisabellecontext{Internalizations}%
%
\isadelimdocument
%
\endisadelimdocument
%
\isatagdocument
%
\isamarkupsection{Aids to internalize formulas%
}
\isamarkuptrue%
%
\endisatagdocument
{\isafolddocument}%
%
\isadelimdocument
%
\endisadelimdocument
%
\isadelimtheory
%
\endisadelimtheory
%
\isatagtheory
\isacommand{theory}\isamarkupfalse%
\ Internalizations\isanewline
\ \ \isakeyword{imports}\ \isanewline
\ \ \ \ {\isachardoublequoteopen}ZF{\isacharminus}{\kern0pt}Constructible{\isachardot}{\kern0pt}DPow{\isacharunderscore}{\kern0pt}absolute{\isachardoublequoteclose}\ \isanewline
\isakeyword{begin}%
\endisatagtheory
{\isafoldtheory}%
%
\isadelimtheory
%
\endisadelimtheory
%
\begin{isamarkuptext}%
We found it useful to have slightly different versions of some 
results in ZF-Constructible:%
\end{isamarkuptext}\isamarkuptrue%
\isacommand{lemma}\isamarkupfalse%
\ nth{\isacharunderscore}{\kern0pt}closed\ {\isacharcolon}{\kern0pt}\isanewline
\ \ \isakeyword{assumes}\ {\isachardoublequoteopen}{\isadigit{0}}{\isasymin}A{\isachardoublequoteclose}\ {\isachardoublequoteopen}env{\isasymin}list{\isacharparenleft}{\kern0pt}A{\isacharparenright}{\kern0pt}{\isachardoublequoteclose}\isanewline
\ \ \isakeyword{shows}\ {\isachardoublequoteopen}nth{\isacharparenleft}{\kern0pt}n{\isacharcomma}{\kern0pt}env{\isacharparenright}{\kern0pt}{\isasymin}A{\isachardoublequoteclose}\ \isanewline
%
\isadelimproof
\ \ %
\endisadelimproof
%
\isatagproof
\isacommand{using}\isamarkupfalse%
\ assms{\isacharparenleft}{\kern0pt}{\isadigit{2}}{\isacharcomma}{\kern0pt}{\isadigit{1}}{\isacharparenright}{\kern0pt}\ \isacommand{unfolding}\isamarkupfalse%
\ nth{\isacharunderscore}{\kern0pt}def\ \isacommand{by}\isamarkupfalse%
\ {\isacharparenleft}{\kern0pt}induct\ env{\isacharsemicolon}{\kern0pt}\ simp{\isacharparenright}{\kern0pt}%
\endisatagproof
{\isafoldproof}%
%
\isadelimproof
\isanewline
%
\endisadelimproof
\isanewline
\isacommand{lemmas}\isamarkupfalse%
\ FOL{\isacharunderscore}{\kern0pt}sats{\isacharunderscore}{\kern0pt}iff\ {\isacharequal}{\kern0pt}\ sats{\isacharunderscore}{\kern0pt}Nand{\isacharunderscore}{\kern0pt}iff\ sats{\isacharunderscore}{\kern0pt}Forall{\isacharunderscore}{\kern0pt}iff\ sats{\isacharunderscore}{\kern0pt}Neg{\isacharunderscore}{\kern0pt}iff\ sats{\isacharunderscore}{\kern0pt}And{\isacharunderscore}{\kern0pt}iff\isanewline
\ \ sats{\isacharunderscore}{\kern0pt}Or{\isacharunderscore}{\kern0pt}iff\ sats{\isacharunderscore}{\kern0pt}Implies{\isacharunderscore}{\kern0pt}iff\ sats{\isacharunderscore}{\kern0pt}Iff{\isacharunderscore}{\kern0pt}iff\ sats{\isacharunderscore}{\kern0pt}Exists{\isacharunderscore}{\kern0pt}iff\ \isanewline
\isanewline
\isacommand{lemma}\isamarkupfalse%
\ nth{\isacharunderscore}{\kern0pt}ConsI{\isacharcolon}{\kern0pt}\ {\isachardoublequoteopen}{\isasymlbrakk}nth{\isacharparenleft}{\kern0pt}n{\isacharcomma}{\kern0pt}l{\isacharparenright}{\kern0pt}\ {\isacharequal}{\kern0pt}\ x{\isacharsemicolon}{\kern0pt}\ n\ {\isasymin}\ nat{\isasymrbrakk}\ {\isasymLongrightarrow}\ nth{\isacharparenleft}{\kern0pt}succ{\isacharparenleft}{\kern0pt}n{\isacharparenright}{\kern0pt}{\isacharcomma}{\kern0pt}\ Cons{\isacharparenleft}{\kern0pt}a{\isacharcomma}{\kern0pt}l{\isacharparenright}{\kern0pt}{\isacharparenright}{\kern0pt}\ {\isacharequal}{\kern0pt}\ x{\isachardoublequoteclose}\isanewline
%
\isadelimproof
%
\endisadelimproof
%
\isatagproof
\isacommand{by}\isamarkupfalse%
\ simp%
\endisatagproof
{\isafoldproof}%
%
\isadelimproof
\isanewline
%
\endisadelimproof
\isanewline
\isacommand{lemmas}\isamarkupfalse%
\ nth{\isacharunderscore}{\kern0pt}rules\ {\isacharequal}{\kern0pt}\ nth{\isacharunderscore}{\kern0pt}{\isadigit{0}}\ nth{\isacharunderscore}{\kern0pt}ConsI\ nat{\isacharunderscore}{\kern0pt}{\isadigit{0}}I\ nat{\isacharunderscore}{\kern0pt}succI\isanewline
\isacommand{lemmas}\isamarkupfalse%
\ sep{\isacharunderscore}{\kern0pt}rules\ {\isacharequal}{\kern0pt}\ nth{\isacharunderscore}{\kern0pt}{\isadigit{0}}\ nth{\isacharunderscore}{\kern0pt}ConsI\ FOL{\isacharunderscore}{\kern0pt}iff{\isacharunderscore}{\kern0pt}sats\ function{\isacharunderscore}{\kern0pt}iff{\isacharunderscore}{\kern0pt}sats\isanewline
\ \ \ \ \ \ \ \ \ \ \ \ \ \ \ \ \ \ \ fun{\isacharunderscore}{\kern0pt}plus{\isacharunderscore}{\kern0pt}iff{\isacharunderscore}{\kern0pt}sats\ successor{\isacharunderscore}{\kern0pt}iff{\isacharunderscore}{\kern0pt}sats\isanewline
\ \ \ \ \ \ \ \ \ \ \ \ \ \ \ \ \ \ \ \ omega{\isacharunderscore}{\kern0pt}iff{\isacharunderscore}{\kern0pt}sats\ FOL{\isacharunderscore}{\kern0pt}sats{\isacharunderscore}{\kern0pt}iff\ Replace{\isacharunderscore}{\kern0pt}iff{\isacharunderscore}{\kern0pt}sats%
\begin{isamarkuptext}%
Also a different compilation of lemmas (term\isa{sep{\isacharunderscore}{\kern0pt}rules}) used in formula
 synthesis%
\end{isamarkuptext}\isamarkuptrue%
\isacommand{lemmas}\isamarkupfalse%
\ fm{\isacharunderscore}{\kern0pt}defs\ {\isacharequal}{\kern0pt}\ omega{\isacharunderscore}{\kern0pt}fm{\isacharunderscore}{\kern0pt}def\ limit{\isacharunderscore}{\kern0pt}ordinal{\isacharunderscore}{\kern0pt}fm{\isacharunderscore}{\kern0pt}def\ empty{\isacharunderscore}{\kern0pt}fm{\isacharunderscore}{\kern0pt}def\ typed{\isacharunderscore}{\kern0pt}function{\isacharunderscore}{\kern0pt}fm{\isacharunderscore}{\kern0pt}def\isanewline
\ \ \ \ \ \ \ \ \ \ \ \ \ \ \ \ \ pair{\isacharunderscore}{\kern0pt}fm{\isacharunderscore}{\kern0pt}def\ upair{\isacharunderscore}{\kern0pt}fm{\isacharunderscore}{\kern0pt}def\ domain{\isacharunderscore}{\kern0pt}fm{\isacharunderscore}{\kern0pt}def\ function{\isacharunderscore}{\kern0pt}fm{\isacharunderscore}{\kern0pt}def\ succ{\isacharunderscore}{\kern0pt}fm{\isacharunderscore}{\kern0pt}def\isanewline
\ \ \ \ \ \ \ \ \ \ \ \ \ \ \ \ \ cons{\isacharunderscore}{\kern0pt}fm{\isacharunderscore}{\kern0pt}def\ fun{\isacharunderscore}{\kern0pt}apply{\isacharunderscore}{\kern0pt}fm{\isacharunderscore}{\kern0pt}def\ image{\isacharunderscore}{\kern0pt}fm{\isacharunderscore}{\kern0pt}def\ big{\isacharunderscore}{\kern0pt}union{\isacharunderscore}{\kern0pt}fm{\isacharunderscore}{\kern0pt}def\ union{\isacharunderscore}{\kern0pt}fm{\isacharunderscore}{\kern0pt}def\isanewline
\ \ \ \ \ \ \ \ \ \ \ \ \ \ \ \ \ relation{\isacharunderscore}{\kern0pt}fm{\isacharunderscore}{\kern0pt}def\ composition{\isacharunderscore}{\kern0pt}fm{\isacharunderscore}{\kern0pt}def\ field{\isacharunderscore}{\kern0pt}fm{\isacharunderscore}{\kern0pt}def\ ordinal{\isacharunderscore}{\kern0pt}fm{\isacharunderscore}{\kern0pt}def\ range{\isacharunderscore}{\kern0pt}fm{\isacharunderscore}{\kern0pt}def\isanewline
\ \ \ \ \ \ \ \ \ \ \ \ \ \ \ \ \ transset{\isacharunderscore}{\kern0pt}fm{\isacharunderscore}{\kern0pt}def\ subset{\isacharunderscore}{\kern0pt}fm{\isacharunderscore}{\kern0pt}def\ Replace{\isacharunderscore}{\kern0pt}fm{\isacharunderscore}{\kern0pt}def\isanewline
\isanewline
%
\isadelimtheory
\isanewline
%
\endisadelimtheory
%
\isatagtheory
\isacommand{end}\isamarkupfalse%
%
\endisatagtheory
{\isafoldtheory}%
%
\isadelimtheory
%
\endisadelimtheory
%
\end{isabellebody}%
\endinput
%:%file=~/source/repos/ZF-notAC/code/Forcing/Internalizations.thy%:%
%:%11=1%:%
%:%27=2%:%
%:%28=2%:%
%:%29=3%:%
%:%30=4%:%
%:%31=5%:%
%:%40=7%:%
%:%41=8%:%
%:%43=9%:%
%:%44=9%:%
%:%45=10%:%
%:%46=11%:%
%:%49=12%:%
%:%53=12%:%
%:%54=12%:%
%:%55=12%:%
%:%56=12%:%
%:%61=12%:%
%:%64=13%:%
%:%65=14%:%
%:%66=14%:%
%:%67=15%:%
%:%68=16%:%
%:%69=17%:%
%:%70=17%:%
%:%77=18%:%
%:%78=18%:%
%:%83=18%:%
%:%86=19%:%
%:%87=20%:%
%:%88=20%:%
%:%89=21%:%
%:%90=21%:%
%:%91=22%:%
%:%92=23%:%
%:%94=25%:%
%:%95=26%:%
%:%97=27%:%
%:%98=27%:%
%:%99=28%:%
%:%100=29%:%
%:%101=30%:%
%:%102=31%:%
%:%103=32%:%
%:%106=33%:%
%:%111=34%:%

%
\begin{isabellebody}%
\setisabellecontext{Recursion{\isacharunderscore}{\kern0pt}Thms}%
%
\isadelimdocument
%
\endisadelimdocument
%
\isatagdocument
%
\isamarkupsection{Some enhanced theorems on recursion%
}
\isamarkuptrue%
%
\endisatagdocument
{\isafolddocument}%
%
\isadelimdocument
%
\endisadelimdocument
%
\isadelimtheory
%
\endisadelimtheory
%
\isatagtheory
\isacommand{theory}\isamarkupfalse%
\ Recursion{\isacharunderscore}{\kern0pt}Thms\ \isakeyword{imports}\ ZF{\isachardot}{\kern0pt}Epsilon\ \isakeyword{begin}%
\endisatagtheory
{\isafoldtheory}%
%
\isadelimtheory
%
\endisadelimtheory
%
\begin{isamarkuptext}%
We prove results concerning definitions by well-founded
recursion on some relation \isa{R} and its transitive closure
\isa{R{\isacharcircum}{\kern0pt}{\isacharasterisk}{\kern0pt}}%
\end{isamarkuptext}\isamarkuptrue%
\isacommand{lemma}\isamarkupfalse%
\ fld{\isacharunderscore}{\kern0pt}restrict{\isacharunderscore}{\kern0pt}eq\ {\isacharcolon}{\kern0pt}\ {\isachardoublequoteopen}a\ {\isasymin}\ A\ {\isasymLongrightarrow}\ {\isacharparenleft}{\kern0pt}r\ {\isasyminter}\ A{\isasymtimes}A{\isacharparenright}{\kern0pt}{\isacharminus}{\kern0pt}{\isacharbackquote}{\kern0pt}{\isacharbackquote}{\kern0pt}{\isacharbraceleft}{\kern0pt}a{\isacharbraceright}{\kern0pt}\ {\isacharequal}{\kern0pt}\ {\isacharparenleft}{\kern0pt}r{\isacharminus}{\kern0pt}{\isacharbackquote}{\kern0pt}{\isacharbackquote}{\kern0pt}{\isacharbraceleft}{\kern0pt}a{\isacharbraceright}{\kern0pt}\ {\isasyminter}\ A{\isacharparenright}{\kern0pt}{\isachardoublequoteclose}\isanewline
%
\isadelimproof
\ \ %
\endisadelimproof
%
\isatagproof
\isacommand{by}\isamarkupfalse%
{\isacharparenleft}{\kern0pt}force{\isacharparenright}{\kern0pt}%
\endisatagproof
{\isafoldproof}%
%
\isadelimproof
\isanewline
%
\endisadelimproof
\isanewline
\isacommand{lemma}\isamarkupfalse%
\ fld{\isacharunderscore}{\kern0pt}restrict{\isacharunderscore}{\kern0pt}mono\ {\isacharcolon}{\kern0pt}\ {\isachardoublequoteopen}relation{\isacharparenleft}{\kern0pt}r{\isacharparenright}{\kern0pt}\ {\isasymLongrightarrow}\ A\ {\isasymsubseteq}\ B\ {\isasymLongrightarrow}\ r\ {\isasyminter}\ A{\isasymtimes}A\ {\isasymsubseteq}\ r\ {\isasyminter}\ B{\isasymtimes}B{\isachardoublequoteclose}\isanewline
%
\isadelimproof
\ \ %
\endisadelimproof
%
\isatagproof
\isacommand{by}\isamarkupfalse%
{\isacharparenleft}{\kern0pt}auto{\isacharparenright}{\kern0pt}%
\endisatagproof
{\isafoldproof}%
%
\isadelimproof
\isanewline
%
\endisadelimproof
\isanewline
\isacommand{lemma}\isamarkupfalse%
\ fld{\isacharunderscore}{\kern0pt}restrict{\isacharunderscore}{\kern0pt}dom\ {\isacharcolon}{\kern0pt}\isanewline
\ \ \isakeyword{assumes}\ {\isachardoublequoteopen}relation{\isacharparenleft}{\kern0pt}r{\isacharparenright}{\kern0pt}{\isachardoublequoteclose}\ {\isachardoublequoteopen}domain{\isacharparenleft}{\kern0pt}r{\isacharparenright}{\kern0pt}\ {\isasymsubseteq}\ A{\isachardoublequoteclose}\ {\isachardoublequoteopen}range{\isacharparenleft}{\kern0pt}r{\isacharparenright}{\kern0pt}{\isasymsubseteq}\ A{\isachardoublequoteclose}\isanewline
\ \ \isakeyword{shows}\ {\isachardoublequoteopen}r{\isasyminter}\ A{\isasymtimes}A\ {\isacharequal}{\kern0pt}\ r{\isachardoublequoteclose}\isanewline
%
\isadelimproof
%
\endisadelimproof
%
\isatagproof
\isacommand{proof}\isamarkupfalse%
\ {\isacharparenleft}{\kern0pt}rule\ equalityI{\isacharcomma}{\kern0pt}blast{\isacharcomma}{\kern0pt}rule\ subsetI{\isacharparenright}{\kern0pt}\isanewline
\ \ \isacommand{{\isacharbraceleft}{\kern0pt}}\isamarkupfalse%
\ \isacommand{fix}\isamarkupfalse%
\ x\isanewline
\ \ \ \ \isacommand{assume}\isamarkupfalse%
\ xr{\isacharcolon}{\kern0pt}\ {\isachardoublequoteopen}x\ {\isasymin}\ r{\isachardoublequoteclose}\isanewline
\ \ \ \ \isacommand{from}\isamarkupfalse%
\ xr\ assms\ \isacommand{have}\isamarkupfalse%
\ {\isachardoublequoteopen}{\isasymexists}\ a\ b\ {\isachardot}{\kern0pt}\ x\ {\isacharequal}{\kern0pt}\ {\isasymlangle}a{\isacharcomma}{\kern0pt}b{\isasymrangle}{\isachardoublequoteclose}\ \isacommand{by}\isamarkupfalse%
\ {\isacharparenleft}{\kern0pt}simp\ add{\isacharcolon}{\kern0pt}\ relation{\isacharunderscore}{\kern0pt}def{\isacharparenright}{\kern0pt}\isanewline
\ \ \ \ \isacommand{then}\isamarkupfalse%
\ \isacommand{obtain}\isamarkupfalse%
\ a\ b\ \isakeyword{where}\ {\isachardoublequoteopen}{\isasymlangle}a{\isacharcomma}{\kern0pt}b{\isasymrangle}\ {\isasymin}\ r{\isachardoublequoteclose}\ {\isachardoublequoteopen}{\isasymlangle}a{\isacharcomma}{\kern0pt}b{\isasymrangle}\ {\isasymin}\ r{\isasyminter}A{\isasymtimes}A{\isachardoublequoteclose}\ {\isachardoublequoteopen}x\ {\isasymin}\ r{\isasyminter}A{\isasymtimes}A{\isachardoublequoteclose}\isanewline
\ \ \ \ \ \ \isacommand{using}\isamarkupfalse%
\ assms\ xr\isanewline
\ \ \ \ \ \ \isacommand{by}\isamarkupfalse%
\ force\isanewline
\ \ \ \ \isacommand{then}\isamarkupfalse%
\ \isacommand{have}\isamarkupfalse%
\ {\isachardoublequoteopen}x{\isasymin}\ r\ {\isasyminter}\ A{\isasymtimes}A{\isachardoublequoteclose}\ \isacommand{by}\isamarkupfalse%
\ simp\isanewline
\ \ \isacommand{{\isacharbraceright}{\kern0pt}}\isamarkupfalse%
\isanewline
\ \ \isacommand{then}\isamarkupfalse%
\ \isacommand{show}\isamarkupfalse%
\ {\isachardoublequoteopen}x\ {\isasymin}\ r\ {\isasymLongrightarrow}\ x{\isasymin}\ r{\isasyminter}A{\isasymtimes}A{\isachardoublequoteclose}\ \isakeyword{for}\ x\ \isacommand{{\isachardot}{\kern0pt}}\isamarkupfalse%
\isanewline
\isacommand{qed}\isamarkupfalse%
%
\endisatagproof
{\isafoldproof}%
%
\isadelimproof
\isanewline
%
\endisadelimproof
\isanewline
\isacommand{definition}\isamarkupfalse%
\ tr{\isacharunderscore}{\kern0pt}down\ {\isacharcolon}{\kern0pt}{\isacharcolon}{\kern0pt}\ {\isachardoublequoteopen}{\isacharbrackleft}{\kern0pt}i{\isacharcomma}{\kern0pt}i{\isacharbrackright}{\kern0pt}\ {\isasymRightarrow}\ i{\isachardoublequoteclose}\isanewline
\ \ \isakeyword{where}\ {\isachardoublequoteopen}tr{\isacharunderscore}{\kern0pt}down{\isacharparenleft}{\kern0pt}r{\isacharcomma}{\kern0pt}a{\isacharparenright}{\kern0pt}\ {\isacharequal}{\kern0pt}\ {\isacharparenleft}{\kern0pt}r{\isacharcircum}{\kern0pt}{\isacharplus}{\kern0pt}{\isacharparenright}{\kern0pt}{\isacharminus}{\kern0pt}{\isacharbackquote}{\kern0pt}{\isacharbackquote}{\kern0pt}{\isacharbraceleft}{\kern0pt}a{\isacharbraceright}{\kern0pt}{\isachardoublequoteclose}\isanewline
\isanewline
\isacommand{lemma}\isamarkupfalse%
\ tr{\isacharunderscore}{\kern0pt}downD\ {\isacharcolon}{\kern0pt}\ {\isachardoublequoteopen}x\ {\isasymin}\ tr{\isacharunderscore}{\kern0pt}down{\isacharparenleft}{\kern0pt}r{\isacharcomma}{\kern0pt}a{\isacharparenright}{\kern0pt}\ {\isasymLongrightarrow}\ {\isasymlangle}x{\isacharcomma}{\kern0pt}a{\isasymrangle}\ {\isasymin}\ r{\isacharcircum}{\kern0pt}{\isacharplus}{\kern0pt}{\isachardoublequoteclose}\isanewline
%
\isadelimproof
\ \ %
\endisadelimproof
%
\isatagproof
\isacommand{by}\isamarkupfalse%
\ {\isacharparenleft}{\kern0pt}simp\ add{\isacharcolon}{\kern0pt}\ tr{\isacharunderscore}{\kern0pt}down{\isacharunderscore}{\kern0pt}def\ vimage{\isacharunderscore}{\kern0pt}singleton{\isacharunderscore}{\kern0pt}iff{\isacharparenright}{\kern0pt}%
\endisatagproof
{\isafoldproof}%
%
\isadelimproof
\isanewline
%
\endisadelimproof
\isanewline
\isacommand{lemma}\isamarkupfalse%
\ pred{\isacharunderscore}{\kern0pt}down\ {\isacharcolon}{\kern0pt}\ {\isachardoublequoteopen}relation{\isacharparenleft}{\kern0pt}r{\isacharparenright}{\kern0pt}\ {\isasymLongrightarrow}\ r{\isacharminus}{\kern0pt}{\isacharbackquote}{\kern0pt}{\isacharbackquote}{\kern0pt}{\isacharbraceleft}{\kern0pt}a{\isacharbraceright}{\kern0pt}\ {\isasymsubseteq}\ tr{\isacharunderscore}{\kern0pt}down{\isacharparenleft}{\kern0pt}r{\isacharcomma}{\kern0pt}a{\isacharparenright}{\kern0pt}{\isachardoublequoteclose}\isanewline
%
\isadelimproof
\ \ %
\endisadelimproof
%
\isatagproof
\isacommand{by}\isamarkupfalse%
{\isacharparenleft}{\kern0pt}simp\ add{\isacharcolon}{\kern0pt}\ tr{\isacharunderscore}{\kern0pt}down{\isacharunderscore}{\kern0pt}def\ vimage{\isacharunderscore}{\kern0pt}mono\ r{\isacharunderscore}{\kern0pt}subset{\isacharunderscore}{\kern0pt}trancl{\isacharparenright}{\kern0pt}%
\endisatagproof
{\isafoldproof}%
%
\isadelimproof
\isanewline
%
\endisadelimproof
\isanewline
\isacommand{lemma}\isamarkupfalse%
\ tr{\isacharunderscore}{\kern0pt}down{\isacharunderscore}{\kern0pt}mono\ {\isacharcolon}{\kern0pt}\ {\isachardoublequoteopen}relation{\isacharparenleft}{\kern0pt}r{\isacharparenright}{\kern0pt}\ {\isasymLongrightarrow}\ x\ {\isasymin}\ r{\isacharminus}{\kern0pt}{\isacharbackquote}{\kern0pt}{\isacharbackquote}{\kern0pt}{\isacharbraceleft}{\kern0pt}a{\isacharbraceright}{\kern0pt}\ {\isasymLongrightarrow}\ tr{\isacharunderscore}{\kern0pt}down{\isacharparenleft}{\kern0pt}r{\isacharcomma}{\kern0pt}x{\isacharparenright}{\kern0pt}\ {\isasymsubseteq}\ tr{\isacharunderscore}{\kern0pt}down{\isacharparenleft}{\kern0pt}r{\isacharcomma}{\kern0pt}a{\isacharparenright}{\kern0pt}{\isachardoublequoteclose}\isanewline
%
\isadelimproof
\ \ %
\endisadelimproof
%
\isatagproof
\isacommand{by}\isamarkupfalse%
{\isacharparenleft}{\kern0pt}rule\ subsetI{\isacharcomma}{\kern0pt}simp\ add{\isacharcolon}{\kern0pt}tr{\isacharunderscore}{\kern0pt}down{\isacharunderscore}{\kern0pt}def{\isacharcomma}{\kern0pt}auto\ dest{\isacharcolon}{\kern0pt}\ underD{\isacharcomma}{\kern0pt}force\ simp\ add{\isacharcolon}{\kern0pt}\ underI\ r{\isacharunderscore}{\kern0pt}into{\isacharunderscore}{\kern0pt}trancl\ trancl{\isacharunderscore}{\kern0pt}trans{\isacharparenright}{\kern0pt}%
\endisatagproof
{\isafoldproof}%
%
\isadelimproof
\isanewline
%
\endisadelimproof
\isanewline
\isacommand{lemma}\isamarkupfalse%
\ rest{\isacharunderscore}{\kern0pt}eq\ {\isacharcolon}{\kern0pt}\isanewline
\ \ \isakeyword{assumes}\ {\isachardoublequoteopen}relation{\isacharparenleft}{\kern0pt}r{\isacharparenright}{\kern0pt}{\isachardoublequoteclose}\ \isakeyword{and}\ {\isachardoublequoteopen}r{\isacharminus}{\kern0pt}{\isacharbackquote}{\kern0pt}{\isacharbackquote}{\kern0pt}{\isacharbraceleft}{\kern0pt}a{\isacharbraceright}{\kern0pt}\ {\isasymsubseteq}\ B{\isachardoublequoteclose}\ \isakeyword{and}\ {\isachardoublequoteopen}a\ {\isasymin}\ B{\isachardoublequoteclose}\isanewline
\ \ \isakeyword{shows}\ {\isachardoublequoteopen}r{\isacharminus}{\kern0pt}{\isacharbackquote}{\kern0pt}{\isacharbackquote}{\kern0pt}{\isacharbraceleft}{\kern0pt}a{\isacharbraceright}{\kern0pt}\ {\isacharequal}{\kern0pt}\ {\isacharparenleft}{\kern0pt}r{\isasyminter}B{\isasymtimes}B{\isacharparenright}{\kern0pt}{\isacharminus}{\kern0pt}{\isacharbackquote}{\kern0pt}{\isacharbackquote}{\kern0pt}{\isacharbraceleft}{\kern0pt}a{\isacharbraceright}{\kern0pt}{\isachardoublequoteclose}\isanewline
%
\isadelimproof
%
\endisadelimproof
%
\isatagproof
\isacommand{proof}\isamarkupfalse%
\ {\isacharparenleft}{\kern0pt}intro\ equalityI\ subsetI{\isacharparenright}{\kern0pt}\isanewline
\ \ \isacommand{fix}\isamarkupfalse%
\ x\isanewline
\ \ \isacommand{assume}\isamarkupfalse%
\ {\isachardoublequoteopen}x\ {\isasymin}\ r{\isacharminus}{\kern0pt}{\isacharbackquote}{\kern0pt}{\isacharbackquote}{\kern0pt}{\isacharbraceleft}{\kern0pt}a{\isacharbraceright}{\kern0pt}{\isachardoublequoteclose}\isanewline
\ \ \isacommand{then}\isamarkupfalse%
\isanewline
\ \ \isacommand{have}\isamarkupfalse%
\ {\isachardoublequoteopen}x\ {\isasymin}\ B{\isachardoublequoteclose}\ \isacommand{using}\isamarkupfalse%
\ assms\ \isacommand{by}\isamarkupfalse%
\ {\isacharparenleft}{\kern0pt}simp\ add{\isacharcolon}{\kern0pt}\ subsetD{\isacharparenright}{\kern0pt}\isanewline
\ \ \isacommand{from}\isamarkupfalse%
\ {\isacartoucheopen}x{\isasymin}\ r{\isacharminus}{\kern0pt}{\isacharbackquote}{\kern0pt}{\isacharbackquote}{\kern0pt}{\isacharbraceleft}{\kern0pt}a{\isacharbraceright}{\kern0pt}{\isacartoucheclose}\isanewline
\ \ \isacommand{have}\isamarkupfalse%
\ {\isachardoublequoteopen}{\isasymlangle}x{\isacharcomma}{\kern0pt}a{\isasymrangle}\ {\isasymin}\ r{\isachardoublequoteclose}\ \isacommand{using}\isamarkupfalse%
\ underD\ \isacommand{by}\isamarkupfalse%
\ simp\isanewline
\ \ \isacommand{then}\isamarkupfalse%
\isanewline
\ \ \isacommand{show}\isamarkupfalse%
\ {\isachardoublequoteopen}x\ {\isasymin}\ {\isacharparenleft}{\kern0pt}r{\isasyminter}B{\isasymtimes}B{\isacharparenright}{\kern0pt}{\isacharminus}{\kern0pt}{\isacharbackquote}{\kern0pt}{\isacharbackquote}{\kern0pt}{\isacharbraceleft}{\kern0pt}a{\isacharbraceright}{\kern0pt}{\isachardoublequoteclose}\ \isacommand{using}\isamarkupfalse%
\ {\isacartoucheopen}x{\isasymin}B{\isacartoucheclose}\ {\isacartoucheopen}a{\isasymin}B{\isacartoucheclose}\ underI\ \isacommand{by}\isamarkupfalse%
\ simp\isanewline
\isacommand{next}\isamarkupfalse%
\isanewline
\ \ \isacommand{from}\isamarkupfalse%
\ assms\isanewline
\ \ \isacommand{show}\isamarkupfalse%
\ {\isachardoublequoteopen}x\ {\isasymin}\ r\ {\isacharminus}{\kern0pt}{\isacharbackquote}{\kern0pt}{\isacharbackquote}{\kern0pt}\ {\isacharbraceleft}{\kern0pt}a{\isacharbraceright}{\kern0pt}{\isachardoublequoteclose}\ \isakeyword{if}\ \ {\isachardoublequoteopen}x\ {\isasymin}\ {\isacharparenleft}{\kern0pt}r\ {\isasyminter}\ B{\isasymtimes}B{\isacharparenright}{\kern0pt}\ {\isacharminus}{\kern0pt}{\isacharbackquote}{\kern0pt}{\isacharbackquote}{\kern0pt}\ {\isacharbraceleft}{\kern0pt}a{\isacharbraceright}{\kern0pt}{\isachardoublequoteclose}\ \isakeyword{for}\ x\isanewline
\ \ \ \ \isacommand{using}\isamarkupfalse%
\ vimage{\isacharunderscore}{\kern0pt}mono\ that\ \isacommand{by}\isamarkupfalse%
\ auto\isanewline
\isacommand{qed}\isamarkupfalse%
%
\endisatagproof
{\isafoldproof}%
%
\isadelimproof
\isanewline
%
\endisadelimproof
\isanewline
\isacommand{lemma}\isamarkupfalse%
\ wfrec{\isacharunderscore}{\kern0pt}restr{\isacharunderscore}{\kern0pt}eq\ {\isacharcolon}{\kern0pt}\ {\isachardoublequoteopen}r{\isacharprime}{\kern0pt}\ {\isacharequal}{\kern0pt}\ r\ {\isasyminter}\ A{\isasymtimes}A\ {\isasymLongrightarrow}\ wfrec{\isacharbrackleft}{\kern0pt}A{\isacharbrackright}{\kern0pt}{\isacharparenleft}{\kern0pt}r{\isacharcomma}{\kern0pt}a{\isacharcomma}{\kern0pt}H{\isacharparenright}{\kern0pt}\ {\isacharequal}{\kern0pt}\ wfrec{\isacharparenleft}{\kern0pt}r{\isacharprime}{\kern0pt}{\isacharcomma}{\kern0pt}a{\isacharcomma}{\kern0pt}H{\isacharparenright}{\kern0pt}{\isachardoublequoteclose}\isanewline
%
\isadelimproof
\ \ %
\endisadelimproof
%
\isatagproof
\isacommand{by}\isamarkupfalse%
{\isacharparenleft}{\kern0pt}simp\ add{\isacharcolon}{\kern0pt}wfrec{\isacharunderscore}{\kern0pt}on{\isacharunderscore}{\kern0pt}def{\isacharparenright}{\kern0pt}%
\endisatagproof
{\isafoldproof}%
%
\isadelimproof
\isanewline
%
\endisadelimproof
\isanewline
\isacommand{lemma}\isamarkupfalse%
\ wfrec{\isacharunderscore}{\kern0pt}restr\ {\isacharcolon}{\kern0pt}\isanewline
\ \ \isakeyword{assumes}\ rr{\isacharcolon}{\kern0pt}\ {\isachardoublequoteopen}relation{\isacharparenleft}{\kern0pt}r{\isacharparenright}{\kern0pt}{\isachardoublequoteclose}\ \isakeyword{and}\ wfr{\isacharcolon}{\kern0pt}{\isachardoublequoteopen}wf{\isacharparenleft}{\kern0pt}r{\isacharparenright}{\kern0pt}{\isachardoublequoteclose}\isanewline
\ \ \isakeyword{shows}\ \ {\isachardoublequoteopen}a\ {\isasymin}\ A\ {\isasymLongrightarrow}\ tr{\isacharunderscore}{\kern0pt}down{\isacharparenleft}{\kern0pt}r{\isacharcomma}{\kern0pt}a{\isacharparenright}{\kern0pt}\ {\isasymsubseteq}\ A\ {\isasymLongrightarrow}\ wfrec{\isacharparenleft}{\kern0pt}r{\isacharcomma}{\kern0pt}a{\isacharcomma}{\kern0pt}H{\isacharparenright}{\kern0pt}\ {\isacharequal}{\kern0pt}\ wfrec{\isacharbrackleft}{\kern0pt}A{\isacharbrackright}{\kern0pt}{\isacharparenleft}{\kern0pt}r{\isacharcomma}{\kern0pt}a{\isacharcomma}{\kern0pt}H{\isacharparenright}{\kern0pt}{\isachardoublequoteclose}\isanewline
%
\isadelimproof
%
\endisadelimproof
%
\isatagproof
\isacommand{proof}\isamarkupfalse%
\ {\isacharparenleft}{\kern0pt}induct\ a\ arbitrary{\isacharcolon}{\kern0pt}A\ rule{\isacharcolon}{\kern0pt}wf{\isacharunderscore}{\kern0pt}induct{\isacharunderscore}{\kern0pt}raw{\isacharbrackleft}{\kern0pt}OF\ wfr{\isacharbrackright}{\kern0pt}\ {\isacharparenright}{\kern0pt}\isanewline
\ \ \isacommand{case}\isamarkupfalse%
\ {\isacharparenleft}{\kern0pt}{\isadigit{1}}\ a{\isacharparenright}{\kern0pt}\isanewline
\ \ \isacommand{have}\isamarkupfalse%
\ wfRa\ {\isacharcolon}{\kern0pt}\ {\isachardoublequoteopen}wf{\isacharbrackleft}{\kern0pt}A{\isacharbrackright}{\kern0pt}{\isacharparenleft}{\kern0pt}r{\isacharparenright}{\kern0pt}{\isachardoublequoteclose}\isanewline
\ \ \ \ \isacommand{using}\isamarkupfalse%
\ wf{\isacharunderscore}{\kern0pt}subset\ wfr\ wf{\isacharunderscore}{\kern0pt}on{\isacharunderscore}{\kern0pt}def\ Int{\isacharunderscore}{\kern0pt}lower{\isadigit{1}}\ \isacommand{by}\isamarkupfalse%
\ simp\isanewline
\ \ \isacommand{from}\isamarkupfalse%
\ pred{\isacharunderscore}{\kern0pt}down\ rr\isanewline
\ \ \isacommand{have}\isamarkupfalse%
\ {\isachardoublequoteopen}r\ {\isacharminus}{\kern0pt}{\isacharbackquote}{\kern0pt}{\isacharbackquote}{\kern0pt}\ {\isacharbraceleft}{\kern0pt}a{\isacharbraceright}{\kern0pt}\ {\isasymsubseteq}\ tr{\isacharunderscore}{\kern0pt}down{\isacharparenleft}{\kern0pt}r{\isacharcomma}{\kern0pt}\ a{\isacharparenright}{\kern0pt}{\isachardoublequoteclose}\ \isacommand{{\isachardot}{\kern0pt}}\isamarkupfalse%
\isanewline
\ \ \isacommand{with}\isamarkupfalse%
\ {\isadigit{1}}\isanewline
\ \ \isacommand{have}\isamarkupfalse%
\ {\isachardoublequoteopen}r{\isacharminus}{\kern0pt}{\isacharbackquote}{\kern0pt}{\isacharbackquote}{\kern0pt}{\isacharbraceleft}{\kern0pt}a{\isacharbraceright}{\kern0pt}\ {\isasymsubseteq}\ A{\isachardoublequoteclose}\ \isacommand{by}\isamarkupfalse%
\ {\isacharparenleft}{\kern0pt}force\ simp\ add{\isacharcolon}{\kern0pt}\ subset{\isacharunderscore}{\kern0pt}trans{\isacharparenright}{\kern0pt}\isanewline
\ \ \isacommand{{\isacharbraceleft}{\kern0pt}}\isamarkupfalse%
\isanewline
\ \ \ \ \isacommand{fix}\isamarkupfalse%
\ x\isanewline
\ \ \ \ \isacommand{assume}\isamarkupfalse%
\ x{\isacharunderscore}{\kern0pt}a\ {\isacharcolon}{\kern0pt}\ {\isachardoublequoteopen}x\ {\isasymin}\ r{\isacharminus}{\kern0pt}{\isacharbackquote}{\kern0pt}{\isacharbackquote}{\kern0pt}{\isacharbraceleft}{\kern0pt}a{\isacharbraceright}{\kern0pt}{\isachardoublequoteclose}\isanewline
\ \ \ \ \isacommand{with}\isamarkupfalse%
\ {\isacartoucheopen}r{\isacharminus}{\kern0pt}{\isacharbackquote}{\kern0pt}{\isacharbackquote}{\kern0pt}{\isacharbraceleft}{\kern0pt}a{\isacharbraceright}{\kern0pt}\ {\isasymsubseteq}\ A{\isacartoucheclose}\isanewline
\ \ \ \ \isacommand{have}\isamarkupfalse%
\ {\isachardoublequoteopen}x\ {\isasymin}\ A{\isachardoublequoteclose}\ \isacommand{{\isachardot}{\kern0pt}{\isachardot}{\kern0pt}}\isamarkupfalse%
\isanewline
\ \ \ \ \isacommand{from}\isamarkupfalse%
\ pred{\isacharunderscore}{\kern0pt}down\ rr\isanewline
\ \ \ \ \isacommand{have}\isamarkupfalse%
\ b\ {\isacharcolon}{\kern0pt}\ {\isachardoublequoteopen}r\ {\isacharminus}{\kern0pt}{\isacharbackquote}{\kern0pt}{\isacharbackquote}{\kern0pt}{\isacharbraceleft}{\kern0pt}x{\isacharbraceright}{\kern0pt}\ {\isasymsubseteq}\ tr{\isacharunderscore}{\kern0pt}down{\isacharparenleft}{\kern0pt}r{\isacharcomma}{\kern0pt}x{\isacharparenright}{\kern0pt}{\isachardoublequoteclose}\ \isacommand{{\isachardot}{\kern0pt}}\isamarkupfalse%
\isanewline
\ \ \ \ \isacommand{then}\isamarkupfalse%
\isanewline
\ \ \ \ \isacommand{have}\isamarkupfalse%
\ {\isachardoublequoteopen}tr{\isacharunderscore}{\kern0pt}down{\isacharparenleft}{\kern0pt}r{\isacharcomma}{\kern0pt}x{\isacharparenright}{\kern0pt}\ {\isasymsubseteq}\ tr{\isacharunderscore}{\kern0pt}down{\isacharparenleft}{\kern0pt}r{\isacharcomma}{\kern0pt}a{\isacharparenright}{\kern0pt}{\isachardoublequoteclose}\isanewline
\ \ \ \ \ \ \isacommand{using}\isamarkupfalse%
\ tr{\isacharunderscore}{\kern0pt}down{\isacharunderscore}{\kern0pt}mono\ x{\isacharunderscore}{\kern0pt}a\ rr\ \isacommand{by}\isamarkupfalse%
\ simp\isanewline
\ \ \ \ \isacommand{with}\isamarkupfalse%
\ {\isadigit{1}}\isanewline
\ \ \ \ \isacommand{have}\isamarkupfalse%
\ {\isachardoublequoteopen}tr{\isacharunderscore}{\kern0pt}down{\isacharparenleft}{\kern0pt}r{\isacharcomma}{\kern0pt}x{\isacharparenright}{\kern0pt}\ {\isasymsubseteq}\ A{\isachardoublequoteclose}\ \isacommand{using}\isamarkupfalse%
\ subset{\isacharunderscore}{\kern0pt}trans\ \isacommand{by}\isamarkupfalse%
\ force\isanewline
\ \ \ \ \isacommand{have}\isamarkupfalse%
\ {\isachardoublequoteopen}{\isasymlangle}x{\isacharcomma}{\kern0pt}a{\isasymrangle}\ {\isasymin}\ r{\isachardoublequoteclose}\ \isacommand{using}\isamarkupfalse%
\ x{\isacharunderscore}{\kern0pt}a\ \ underD\ \isacommand{by}\isamarkupfalse%
\ simp\isanewline
\ \ \ \ \isacommand{with}\isamarkupfalse%
\ {\isadigit{1}}\ {\isacartoucheopen}tr{\isacharunderscore}{\kern0pt}down{\isacharparenleft}{\kern0pt}r{\isacharcomma}{\kern0pt}x{\isacharparenright}{\kern0pt}\ {\isasymsubseteq}\ A{\isacartoucheclose}\ {\isacartoucheopen}x\ {\isasymin}\ A{\isacartoucheclose}\isanewline
\ \ \ \ \isacommand{have}\isamarkupfalse%
\ {\isachardoublequoteopen}wfrec{\isacharparenleft}{\kern0pt}r{\isacharcomma}{\kern0pt}x{\isacharcomma}{\kern0pt}H{\isacharparenright}{\kern0pt}\ {\isacharequal}{\kern0pt}\ wfrec{\isacharbrackleft}{\kern0pt}A{\isacharbrackright}{\kern0pt}{\isacharparenleft}{\kern0pt}r{\isacharcomma}{\kern0pt}x{\isacharcomma}{\kern0pt}H{\isacharparenright}{\kern0pt}{\isachardoublequoteclose}\ \isacommand{by}\isamarkupfalse%
\ simp\isanewline
\ \ \isacommand{{\isacharbraceright}{\kern0pt}}\isamarkupfalse%
\isanewline
\ \ \isacommand{then}\isamarkupfalse%
\isanewline
\ \ \isacommand{have}\isamarkupfalse%
\ {\isachardoublequoteopen}x{\isasymin}\ r{\isacharminus}{\kern0pt}{\isacharbackquote}{\kern0pt}{\isacharbackquote}{\kern0pt}{\isacharbraceleft}{\kern0pt}a{\isacharbraceright}{\kern0pt}\ {\isasymLongrightarrow}\ wfrec{\isacharparenleft}{\kern0pt}r{\isacharcomma}{\kern0pt}x{\isacharcomma}{\kern0pt}H{\isacharparenright}{\kern0pt}\ {\isacharequal}{\kern0pt}\ \ wfrec{\isacharbrackleft}{\kern0pt}A{\isacharbrackright}{\kern0pt}{\isacharparenleft}{\kern0pt}r{\isacharcomma}{\kern0pt}x{\isacharcomma}{\kern0pt}H{\isacharparenright}{\kern0pt}{\isachardoublequoteclose}\ \isakeyword{for}\ x\ \ \isacommand{{\isachardot}{\kern0pt}}\isamarkupfalse%
\isanewline
\ \ \isacommand{then}\isamarkupfalse%
\isanewline
\ \ \isacommand{have}\isamarkupfalse%
\ Eq{\isadigit{1}}\ {\isacharcolon}{\kern0pt}{\isachardoublequoteopen}{\isacharparenleft}{\kern0pt}{\isasymlambda}\ x\ {\isasymin}\ r{\isacharminus}{\kern0pt}{\isacharbackquote}{\kern0pt}{\isacharbackquote}{\kern0pt}{\isacharbraceleft}{\kern0pt}a{\isacharbraceright}{\kern0pt}\ {\isachardot}{\kern0pt}\ wfrec{\isacharparenleft}{\kern0pt}r{\isacharcomma}{\kern0pt}x{\isacharcomma}{\kern0pt}H{\isacharparenright}{\kern0pt}{\isacharparenright}{\kern0pt}\ {\isacharequal}{\kern0pt}\ {\isacharparenleft}{\kern0pt}{\isasymlambda}\ x\ {\isasymin}\ r{\isacharminus}{\kern0pt}{\isacharbackquote}{\kern0pt}{\isacharbackquote}{\kern0pt}{\isacharbraceleft}{\kern0pt}a{\isacharbraceright}{\kern0pt}\ {\isachardot}{\kern0pt}\ wfrec{\isacharbrackleft}{\kern0pt}A{\isacharbrackright}{\kern0pt}{\isacharparenleft}{\kern0pt}r{\isacharcomma}{\kern0pt}x{\isacharcomma}{\kern0pt}H{\isacharparenright}{\kern0pt}{\isacharparenright}{\kern0pt}{\isachardoublequoteclose}\isanewline
\ \ \ \ \isacommand{using}\isamarkupfalse%
\ lam{\isacharunderscore}{\kern0pt}cong\ \isacommand{by}\isamarkupfalse%
\ simp\isanewline
\isanewline
\ \ \isacommand{from}\isamarkupfalse%
\ assms\isanewline
\ \ \isacommand{have}\isamarkupfalse%
\ {\isachardoublequoteopen}wfrec{\isacharparenleft}{\kern0pt}r{\isacharcomma}{\kern0pt}a{\isacharcomma}{\kern0pt}H{\isacharparenright}{\kern0pt}\ {\isacharequal}{\kern0pt}\ H{\isacharparenleft}{\kern0pt}a{\isacharcomma}{\kern0pt}{\isasymlambda}\ x\ {\isasymin}\ r{\isacharminus}{\kern0pt}{\isacharbackquote}{\kern0pt}{\isacharbackquote}{\kern0pt}{\isacharbraceleft}{\kern0pt}a{\isacharbraceright}{\kern0pt}\ {\isachardot}{\kern0pt}\ wfrec{\isacharparenleft}{\kern0pt}r{\isacharcomma}{\kern0pt}x{\isacharcomma}{\kern0pt}H{\isacharparenright}{\kern0pt}{\isacharparenright}{\kern0pt}{\isachardoublequoteclose}\ \isacommand{by}\isamarkupfalse%
\ {\isacharparenleft}{\kern0pt}simp\ add{\isacharcolon}{\kern0pt}wfrec{\isacharparenright}{\kern0pt}\isanewline
\ \ \isacommand{also}\isamarkupfalse%
\isanewline
\ \ \isacommand{have}\isamarkupfalse%
\ {\isachardoublequoteopen}{\isachardot}{\kern0pt}{\isachardot}{\kern0pt}{\isachardot}{\kern0pt}\ {\isacharequal}{\kern0pt}\ H{\isacharparenleft}{\kern0pt}a{\isacharcomma}{\kern0pt}{\isasymlambda}\ x\ {\isasymin}\ r{\isacharminus}{\kern0pt}{\isacharbackquote}{\kern0pt}{\isacharbackquote}{\kern0pt}{\isacharbraceleft}{\kern0pt}a{\isacharbraceright}{\kern0pt}\ {\isachardot}{\kern0pt}\ wfrec{\isacharbrackleft}{\kern0pt}A{\isacharbrackright}{\kern0pt}{\isacharparenleft}{\kern0pt}r{\isacharcomma}{\kern0pt}x{\isacharcomma}{\kern0pt}H{\isacharparenright}{\kern0pt}{\isacharparenright}{\kern0pt}{\isachardoublequoteclose}\isanewline
\ \ \ \ \isacommand{using}\isamarkupfalse%
\ assms\ Eq{\isadigit{1}}\ \isacommand{by}\isamarkupfalse%
\ simp\isanewline
\ \ \isacommand{also}\isamarkupfalse%
\ \isacommand{from}\isamarkupfalse%
\ {\isadigit{1}}\ {\isacartoucheopen}r{\isacharminus}{\kern0pt}{\isacharbackquote}{\kern0pt}{\isacharbackquote}{\kern0pt}{\isacharbraceleft}{\kern0pt}a{\isacharbraceright}{\kern0pt}\ {\isasymsubseteq}\ A{\isacartoucheclose}\isanewline
\ \ \isacommand{have}\isamarkupfalse%
\ {\isachardoublequoteopen}{\isachardot}{\kern0pt}{\isachardot}{\kern0pt}{\isachardot}{\kern0pt}\ {\isacharequal}{\kern0pt}\ H{\isacharparenleft}{\kern0pt}a{\isacharcomma}{\kern0pt}{\isasymlambda}\ x\ {\isasymin}\ {\isacharparenleft}{\kern0pt}r{\isasyminter}A{\isasymtimes}A{\isacharparenright}{\kern0pt}{\isacharminus}{\kern0pt}{\isacharbackquote}{\kern0pt}{\isacharbackquote}{\kern0pt}{\isacharbraceleft}{\kern0pt}a{\isacharbraceright}{\kern0pt}\ {\isachardot}{\kern0pt}\ wfrec{\isacharbrackleft}{\kern0pt}A{\isacharbrackright}{\kern0pt}{\isacharparenleft}{\kern0pt}r{\isacharcomma}{\kern0pt}x{\isacharcomma}{\kern0pt}H{\isacharparenright}{\kern0pt}{\isacharparenright}{\kern0pt}{\isachardoublequoteclose}\isanewline
\ \ \ \ \isacommand{using}\isamarkupfalse%
\ assms\ rest{\isacharunderscore}{\kern0pt}eq\ \ \isacommand{by}\isamarkupfalse%
\ simp\isanewline
\ \ \isacommand{also}\isamarkupfalse%
\ \isacommand{from}\isamarkupfalse%
\ {\isacartoucheopen}a{\isasymin}A{\isacartoucheclose}\isanewline
\ \ \isacommand{have}\isamarkupfalse%
\ {\isachardoublequoteopen}{\isachardot}{\kern0pt}{\isachardot}{\kern0pt}{\isachardot}{\kern0pt}\ {\isacharequal}{\kern0pt}\ H{\isacharparenleft}{\kern0pt}a{\isacharcomma}{\kern0pt}{\isasymlambda}\ x\ {\isasymin}\ {\isacharparenleft}{\kern0pt}r{\isacharminus}{\kern0pt}{\isacharbackquote}{\kern0pt}{\isacharbackquote}{\kern0pt}{\isacharbraceleft}{\kern0pt}a{\isacharbraceright}{\kern0pt}{\isacharparenright}{\kern0pt}{\isasyminter}A\ {\isachardot}{\kern0pt}\ wfrec{\isacharbrackleft}{\kern0pt}A{\isacharbrackright}{\kern0pt}{\isacharparenleft}{\kern0pt}r{\isacharcomma}{\kern0pt}x{\isacharcomma}{\kern0pt}H{\isacharparenright}{\kern0pt}{\isacharparenright}{\kern0pt}{\isachardoublequoteclose}\isanewline
\ \ \ \ \isacommand{using}\isamarkupfalse%
\ fld{\isacharunderscore}{\kern0pt}restrict{\isacharunderscore}{\kern0pt}eq\ \isacommand{by}\isamarkupfalse%
\ simp\isanewline
\ \ \isacommand{also}\isamarkupfalse%
\ \isacommand{from}\isamarkupfalse%
\ {\isacartoucheopen}a{\isasymin}A{\isacartoucheclose}\ {\isacartoucheopen}wf{\isacharbrackleft}{\kern0pt}A{\isacharbrackright}{\kern0pt}{\isacharparenleft}{\kern0pt}r{\isacharparenright}{\kern0pt}{\isacartoucheclose}\isanewline
\ \ \isacommand{have}\isamarkupfalse%
\ {\isachardoublequoteopen}{\isachardot}{\kern0pt}{\isachardot}{\kern0pt}{\isachardot}{\kern0pt}\ {\isacharequal}{\kern0pt}\ wfrec{\isacharbrackleft}{\kern0pt}A{\isacharbrackright}{\kern0pt}{\isacharparenleft}{\kern0pt}r{\isacharcomma}{\kern0pt}a{\isacharcomma}{\kern0pt}H{\isacharparenright}{\kern0pt}{\isachardoublequoteclose}\ \isacommand{using}\isamarkupfalse%
\ wfrec{\isacharunderscore}{\kern0pt}on\ \isacommand{by}\isamarkupfalse%
\ simp\isanewline
\ \ \isacommand{finally}\isamarkupfalse%
\ \isacommand{show}\isamarkupfalse%
\ {\isacharquery}{\kern0pt}case\ \isacommand{{\isachardot}{\kern0pt}}\isamarkupfalse%
\isanewline
\isacommand{qed}\isamarkupfalse%
%
\endisatagproof
{\isafoldproof}%
%
\isadelimproof
\isanewline
%
\endisadelimproof
\isanewline
\isacommand{lemmas}\isamarkupfalse%
\ wfrec{\isacharunderscore}{\kern0pt}tr{\isacharunderscore}{\kern0pt}down\ {\isacharequal}{\kern0pt}\ wfrec{\isacharunderscore}{\kern0pt}restr{\isacharbrackleft}{\kern0pt}OF\ {\isacharunderscore}{\kern0pt}\ {\isacharunderscore}{\kern0pt}\ {\isacharunderscore}{\kern0pt}\ subset{\isacharunderscore}{\kern0pt}refl{\isacharbrackright}{\kern0pt}\isanewline
\isanewline
\isacommand{lemma}\isamarkupfalse%
\ wfrec{\isacharunderscore}{\kern0pt}trans{\isacharunderscore}{\kern0pt}restr\ {\isacharcolon}{\kern0pt}\ {\isachardoublequoteopen}relation{\isacharparenleft}{\kern0pt}r{\isacharparenright}{\kern0pt}\ {\isasymLongrightarrow}\ wf{\isacharparenleft}{\kern0pt}r{\isacharparenright}{\kern0pt}\ {\isasymLongrightarrow}\ trans{\isacharparenleft}{\kern0pt}r{\isacharparenright}{\kern0pt}\ {\isasymLongrightarrow}\ r{\isacharminus}{\kern0pt}{\isacharbackquote}{\kern0pt}{\isacharbackquote}{\kern0pt}{\isacharbraceleft}{\kern0pt}a{\isacharbraceright}{\kern0pt}{\isasymsubseteq}A\ {\isasymLongrightarrow}\ a\ {\isasymin}\ A\ {\isasymLongrightarrow}\isanewline
\ \ wfrec{\isacharparenleft}{\kern0pt}r{\isacharcomma}{\kern0pt}\ a{\isacharcomma}{\kern0pt}\ H{\isacharparenright}{\kern0pt}\ {\isacharequal}{\kern0pt}\ wfrec{\isacharbrackleft}{\kern0pt}A{\isacharbrackright}{\kern0pt}{\isacharparenleft}{\kern0pt}r{\isacharcomma}{\kern0pt}\ a{\isacharcomma}{\kern0pt}\ H{\isacharparenright}{\kern0pt}{\isachardoublequoteclose}\isanewline
%
\isadelimproof
\ \ %
\endisadelimproof
%
\isatagproof
\isacommand{by}\isamarkupfalse%
{\isacharparenleft}{\kern0pt}subgoal{\isacharunderscore}{\kern0pt}tac\ {\isachardoublequoteopen}tr{\isacharunderscore}{\kern0pt}down{\isacharparenleft}{\kern0pt}r{\isacharcomma}{\kern0pt}a{\isacharparenright}{\kern0pt}\ {\isasymsubseteq}\ A{\isachardoublequoteclose}{\isacharcomma}{\kern0pt}auto\ simp\ add\ {\isacharcolon}{\kern0pt}\ wfrec{\isacharunderscore}{\kern0pt}restr\ tr{\isacharunderscore}{\kern0pt}down{\isacharunderscore}{\kern0pt}def\ trancl{\isacharunderscore}{\kern0pt}eq{\isacharunderscore}{\kern0pt}r{\isacharparenright}{\kern0pt}%
\endisatagproof
{\isafoldproof}%
%
\isadelimproof
\isanewline
%
\endisadelimproof
\isanewline
\isanewline
\isacommand{lemma}\isamarkupfalse%
\ field{\isacharunderscore}{\kern0pt}trancl\ {\isacharcolon}{\kern0pt}\ {\isachardoublequoteopen}field{\isacharparenleft}{\kern0pt}r{\isacharcircum}{\kern0pt}{\isacharplus}{\kern0pt}{\isacharparenright}{\kern0pt}\ {\isacharequal}{\kern0pt}\ field{\isacharparenleft}{\kern0pt}r{\isacharparenright}{\kern0pt}{\isachardoublequoteclose}\isanewline
%
\isadelimproof
\ \ %
\endisadelimproof
%
\isatagproof
\isacommand{by}\isamarkupfalse%
\ {\isacharparenleft}{\kern0pt}blast\ intro{\isacharcolon}{\kern0pt}\ r{\isacharunderscore}{\kern0pt}into{\isacharunderscore}{\kern0pt}trancl\ dest{\isacharbang}{\kern0pt}{\isacharcolon}{\kern0pt}\ trancl{\isacharunderscore}{\kern0pt}type\ {\isacharbrackleft}{\kern0pt}THEN\ subsetD{\isacharbrackright}{\kern0pt}{\isacharparenright}{\kern0pt}%
\endisatagproof
{\isafoldproof}%
%
\isadelimproof
\isanewline
%
\endisadelimproof
\isanewline
\isacommand{definition}\isamarkupfalse%
\isanewline
\ \ Rrel\ {\isacharcolon}{\kern0pt}{\isacharcolon}{\kern0pt}\ {\isachardoublequoteopen}{\isacharbrackleft}{\kern0pt}i{\isasymRightarrow}i{\isasymRightarrow}o{\isacharcomma}{\kern0pt}i{\isacharbrackright}{\kern0pt}\ {\isasymRightarrow}\ i{\isachardoublequoteclose}\ \isakeyword{where}\isanewline
\ \ {\isachardoublequoteopen}Rrel{\isacharparenleft}{\kern0pt}R{\isacharcomma}{\kern0pt}A{\isacharparenright}{\kern0pt}\ {\isasymequiv}\ {\isacharbraceleft}{\kern0pt}z{\isasymin}A{\isasymtimes}A{\isachardot}{\kern0pt}\ {\isasymexists}x\ y{\isachardot}{\kern0pt}\ z\ {\isacharequal}{\kern0pt}\ {\isasymlangle}x{\isacharcomma}{\kern0pt}\ y{\isasymrangle}\ {\isasymand}\ R{\isacharparenleft}{\kern0pt}x{\isacharcomma}{\kern0pt}y{\isacharparenright}{\kern0pt}{\isacharbraceright}{\kern0pt}{\isachardoublequoteclose}\isanewline
\isanewline
\isacommand{lemma}\isamarkupfalse%
\ RrelI\ {\isacharcolon}{\kern0pt}\ {\isachardoublequoteopen}x\ {\isasymin}\ A\ {\isasymLongrightarrow}\ y\ {\isasymin}\ A\ {\isasymLongrightarrow}\ R{\isacharparenleft}{\kern0pt}x{\isacharcomma}{\kern0pt}y{\isacharparenright}{\kern0pt}\ {\isasymLongrightarrow}\ {\isasymlangle}x{\isacharcomma}{\kern0pt}y{\isasymrangle}\ {\isasymin}\ Rrel{\isacharparenleft}{\kern0pt}R{\isacharcomma}{\kern0pt}A{\isacharparenright}{\kern0pt}{\isachardoublequoteclose}\isanewline
%
\isadelimproof
\ \ %
\endisadelimproof
%
\isatagproof
\isacommand{unfolding}\isamarkupfalse%
\ Rrel{\isacharunderscore}{\kern0pt}def\ \isacommand{by}\isamarkupfalse%
\ simp%
\endisatagproof
{\isafoldproof}%
%
\isadelimproof
\isanewline
%
\endisadelimproof
\isanewline
\isacommand{lemma}\isamarkupfalse%
\ Rrel{\isacharunderscore}{\kern0pt}mem{\isacharcolon}{\kern0pt}\ {\isachardoublequoteopen}Rrel{\isacharparenleft}{\kern0pt}mem{\isacharcomma}{\kern0pt}x{\isacharparenright}{\kern0pt}\ {\isacharequal}{\kern0pt}\ Memrel{\isacharparenleft}{\kern0pt}x{\isacharparenright}{\kern0pt}{\isachardoublequoteclose}\isanewline
%
\isadelimproof
\ \ %
\endisadelimproof
%
\isatagproof
\isacommand{unfolding}\isamarkupfalse%
\ Rrel{\isacharunderscore}{\kern0pt}def\ Memrel{\isacharunderscore}{\kern0pt}def\ \isacommand{{\isachardot}{\kern0pt}{\isachardot}{\kern0pt}}\isamarkupfalse%
%
\endisatagproof
{\isafoldproof}%
%
\isadelimproof
\isanewline
%
\endisadelimproof
\isanewline
\isacommand{lemma}\isamarkupfalse%
\ relation{\isacharunderscore}{\kern0pt}Rrel{\isacharcolon}{\kern0pt}\ {\isachardoublequoteopen}relation{\isacharparenleft}{\kern0pt}Rrel{\isacharparenleft}{\kern0pt}R{\isacharcomma}{\kern0pt}d{\isacharparenright}{\kern0pt}{\isacharparenright}{\kern0pt}{\isachardoublequoteclose}\isanewline
%
\isadelimproof
\ \ %
\endisadelimproof
%
\isatagproof
\isacommand{unfolding}\isamarkupfalse%
\ Rrel{\isacharunderscore}{\kern0pt}def\ relation{\isacharunderscore}{\kern0pt}def\ \isacommand{by}\isamarkupfalse%
\ simp%
\endisatagproof
{\isafoldproof}%
%
\isadelimproof
\isanewline
%
\endisadelimproof
\isanewline
\isacommand{lemma}\isamarkupfalse%
\ field{\isacharunderscore}{\kern0pt}Rrel{\isacharcolon}{\kern0pt}\ {\isachardoublequoteopen}field{\isacharparenleft}{\kern0pt}Rrel{\isacharparenleft}{\kern0pt}R{\isacharcomma}{\kern0pt}d{\isacharparenright}{\kern0pt}{\isacharparenright}{\kern0pt}\ {\isasymsubseteq}\ \ d{\isachardoublequoteclose}\isanewline
%
\isadelimproof
\ \ %
\endisadelimproof
%
\isatagproof
\isacommand{unfolding}\isamarkupfalse%
\ Rrel{\isacharunderscore}{\kern0pt}def\ \isacommand{by}\isamarkupfalse%
\ auto%
\endisatagproof
{\isafoldproof}%
%
\isadelimproof
\isanewline
%
\endisadelimproof
\isanewline
\isacommand{lemma}\isamarkupfalse%
\ Rrel{\isacharunderscore}{\kern0pt}mono\ {\isacharcolon}{\kern0pt}\ {\isachardoublequoteopen}A\ {\isasymsubseteq}\ B\ {\isasymLongrightarrow}\ Rrel{\isacharparenleft}{\kern0pt}R{\isacharcomma}{\kern0pt}A{\isacharparenright}{\kern0pt}\ {\isasymsubseteq}\ Rrel{\isacharparenleft}{\kern0pt}R{\isacharcomma}{\kern0pt}B{\isacharparenright}{\kern0pt}{\isachardoublequoteclose}\isanewline
%
\isadelimproof
\ \ %
\endisadelimproof
%
\isatagproof
\isacommand{unfolding}\isamarkupfalse%
\ Rrel{\isacharunderscore}{\kern0pt}def\ \isacommand{by}\isamarkupfalse%
\ blast%
\endisatagproof
{\isafoldproof}%
%
\isadelimproof
\isanewline
%
\endisadelimproof
\isanewline
\isacommand{lemma}\isamarkupfalse%
\ Rrel{\isacharunderscore}{\kern0pt}restr{\isacharunderscore}{\kern0pt}eq\ {\isacharcolon}{\kern0pt}\ {\isachardoublequoteopen}Rrel{\isacharparenleft}{\kern0pt}R{\isacharcomma}{\kern0pt}A{\isacharparenright}{\kern0pt}\ {\isasyminter}\ B{\isasymtimes}B\ {\isacharequal}{\kern0pt}\ Rrel{\isacharparenleft}{\kern0pt}R{\isacharcomma}{\kern0pt}A{\isasyminter}B{\isacharparenright}{\kern0pt}{\isachardoublequoteclose}\isanewline
%
\isadelimproof
\ \ %
\endisadelimproof
%
\isatagproof
\isacommand{unfolding}\isamarkupfalse%
\ Rrel{\isacharunderscore}{\kern0pt}def\ \isacommand{by}\isamarkupfalse%
\ blast%
\endisatagproof
{\isafoldproof}%
%
\isadelimproof
\isanewline
%
\endisadelimproof
\isanewline
\isanewline
\isacommand{lemma}\isamarkupfalse%
\ field{\isacharunderscore}{\kern0pt}Memrel\ {\isacharcolon}{\kern0pt}\ {\isachardoublequoteopen}field{\isacharparenleft}{\kern0pt}Memrel{\isacharparenleft}{\kern0pt}A{\isacharparenright}{\kern0pt}{\isacharparenright}{\kern0pt}\ {\isasymsubseteq}\ A{\isachardoublequoteclose}\isanewline
\ \ \isanewline
%
\isadelimproof
\ \ %
\endisadelimproof
%
\isatagproof
\isacommand{using}\isamarkupfalse%
\ Rrel{\isacharunderscore}{\kern0pt}mem\ field{\isacharunderscore}{\kern0pt}Rrel\ \isacommand{by}\isamarkupfalse%
\ blast%
\endisatagproof
{\isafoldproof}%
%
\isadelimproof
\isanewline
%
\endisadelimproof
\isanewline
\isacommand{lemma}\isamarkupfalse%
\ restrict{\isacharunderscore}{\kern0pt}trancl{\isacharunderscore}{\kern0pt}Rrel{\isacharcolon}{\kern0pt}\isanewline
\ \ \isakeyword{assumes}\ {\isachardoublequoteopen}R{\isacharparenleft}{\kern0pt}w{\isacharcomma}{\kern0pt}y{\isacharparenright}{\kern0pt}{\isachardoublequoteclose}\isanewline
\ \ \isakeyword{shows}\ {\isachardoublequoteopen}restrict{\isacharparenleft}{\kern0pt}f{\isacharcomma}{\kern0pt}Rrel{\isacharparenleft}{\kern0pt}R{\isacharcomma}{\kern0pt}d{\isacharparenright}{\kern0pt}{\isacharminus}{\kern0pt}{\isacharbackquote}{\kern0pt}{\isacharbackquote}{\kern0pt}{\isacharbraceleft}{\kern0pt}y{\isacharbraceright}{\kern0pt}{\isacharparenright}{\kern0pt}{\isacharbackquote}{\kern0pt}w\isanewline
\ \ \ \ \ \ \ {\isacharequal}{\kern0pt}\ restrict{\isacharparenleft}{\kern0pt}f{\isacharcomma}{\kern0pt}{\isacharparenleft}{\kern0pt}Rrel{\isacharparenleft}{\kern0pt}R{\isacharcomma}{\kern0pt}d{\isacharparenright}{\kern0pt}{\isacharcircum}{\kern0pt}{\isacharplus}{\kern0pt}{\isacharparenright}{\kern0pt}{\isacharminus}{\kern0pt}{\isacharbackquote}{\kern0pt}{\isacharbackquote}{\kern0pt}{\isacharbraceleft}{\kern0pt}y{\isacharbraceright}{\kern0pt}{\isacharparenright}{\kern0pt}{\isacharbackquote}{\kern0pt}w{\isachardoublequoteclose}\isanewline
%
\isadelimproof
%
\endisadelimproof
%
\isatagproof
\isacommand{proof}\isamarkupfalse%
\ {\isacharparenleft}{\kern0pt}cases\ {\isachardoublequoteopen}y{\isasymin}d{\isachardoublequoteclose}{\isacharparenright}{\kern0pt}\isanewline
\ \ \isacommand{let}\isamarkupfalse%
\ {\isacharquery}{\kern0pt}r{\isacharequal}{\kern0pt}{\isachardoublequoteopen}Rrel{\isacharparenleft}{\kern0pt}R{\isacharcomma}{\kern0pt}d{\isacharparenright}{\kern0pt}{\isachardoublequoteclose}\ \isakeyword{and}\ {\isacharquery}{\kern0pt}s{\isacharequal}{\kern0pt}{\isachardoublequoteopen}{\isacharparenleft}{\kern0pt}Rrel{\isacharparenleft}{\kern0pt}R{\isacharcomma}{\kern0pt}d{\isacharparenright}{\kern0pt}{\isacharparenright}{\kern0pt}{\isacharcircum}{\kern0pt}{\isacharplus}{\kern0pt}{\isachardoublequoteclose}\isanewline
\ \ \isacommand{case}\isamarkupfalse%
\ True\isanewline
\ \ \isacommand{show}\isamarkupfalse%
\ {\isacharquery}{\kern0pt}thesis\isanewline
\ \ \isacommand{proof}\isamarkupfalse%
\ {\isacharparenleft}{\kern0pt}cases\ {\isachardoublequoteopen}w{\isasymin}d{\isachardoublequoteclose}{\isacharparenright}{\kern0pt}\isanewline
\ \ \ \ \isacommand{case}\isamarkupfalse%
\ True\isanewline
\ \ \ \ \isacommand{with}\isamarkupfalse%
\ {\isacartoucheopen}y{\isasymin}d{\isacartoucheclose}\ assms\isanewline
\ \ \ \ \isacommand{have}\isamarkupfalse%
\ {\isachardoublequoteopen}{\isasymlangle}w{\isacharcomma}{\kern0pt}y{\isasymrangle}{\isasymin}{\isacharquery}{\kern0pt}r{\isachardoublequoteclose}\isanewline
\ \ \ \ \ \ \isacommand{unfolding}\isamarkupfalse%
\ Rrel{\isacharunderscore}{\kern0pt}def\ \isacommand{by}\isamarkupfalse%
\ blast\isanewline
\ \ \ \ \isacommand{then}\isamarkupfalse%
\isanewline
\ \ \ \ \isacommand{have}\isamarkupfalse%
\ {\isachardoublequoteopen}{\isasymlangle}w{\isacharcomma}{\kern0pt}y{\isasymrangle}{\isasymin}{\isacharquery}{\kern0pt}s{\isachardoublequoteclose}\isanewline
\ \ \ \ \ \ \isacommand{using}\isamarkupfalse%
\ r{\isacharunderscore}{\kern0pt}subset{\isacharunderscore}{\kern0pt}trancl{\isacharbrackleft}{\kern0pt}of\ {\isacharquery}{\kern0pt}r{\isacharbrackright}{\kern0pt}\ relation{\isacharunderscore}{\kern0pt}Rrel{\isacharbrackleft}{\kern0pt}of\ R\ d{\isacharbrackright}{\kern0pt}\ \isacommand{by}\isamarkupfalse%
\ blast\isanewline
\ \ \ \ \isacommand{with}\isamarkupfalse%
\ {\isacartoucheopen}{\isasymlangle}w{\isacharcomma}{\kern0pt}y{\isasymrangle}{\isasymin}{\isacharquery}{\kern0pt}r{\isacartoucheclose}\isanewline
\ \ \ \ \isacommand{have}\isamarkupfalse%
\ {\isachardoublequoteopen}w{\isasymin}{\isacharquery}{\kern0pt}r{\isacharminus}{\kern0pt}{\isacharbackquote}{\kern0pt}{\isacharbackquote}{\kern0pt}{\isacharbraceleft}{\kern0pt}y{\isacharbraceright}{\kern0pt}{\isachardoublequoteclose}\ {\isachardoublequoteopen}w{\isasymin}{\isacharquery}{\kern0pt}s{\isacharminus}{\kern0pt}{\isacharbackquote}{\kern0pt}{\isacharbackquote}{\kern0pt}{\isacharbraceleft}{\kern0pt}y{\isacharbraceright}{\kern0pt}{\isachardoublequoteclose}\isanewline
\ \ \ \ \ \ \isacommand{using}\isamarkupfalse%
\ vimage{\isacharunderscore}{\kern0pt}singleton{\isacharunderscore}{\kern0pt}iff\ \isacommand{by}\isamarkupfalse%
\ simp{\isacharunderscore}{\kern0pt}all\isanewline
\ \ \ \ \isacommand{then}\isamarkupfalse%
\isanewline
\ \ \ \ \isacommand{show}\isamarkupfalse%
\ {\isacharquery}{\kern0pt}thesis\ \isacommand{by}\isamarkupfalse%
\ simp\isanewline
\ \ \isacommand{next}\isamarkupfalse%
\isanewline
\ \ \ \ \isacommand{case}\isamarkupfalse%
\ False\isanewline
\ \ \ \ \isacommand{then}\isamarkupfalse%
\isanewline
\ \ \ \ \isacommand{have}\isamarkupfalse%
\ {\isachardoublequoteopen}w{\isasymnotin}domain{\isacharparenleft}{\kern0pt}restrict{\isacharparenleft}{\kern0pt}f{\isacharcomma}{\kern0pt}{\isacharquery}{\kern0pt}r{\isacharminus}{\kern0pt}{\isacharbackquote}{\kern0pt}{\isacharbackquote}{\kern0pt}{\isacharbraceleft}{\kern0pt}y{\isacharbraceright}{\kern0pt}{\isacharparenright}{\kern0pt}{\isacharparenright}{\kern0pt}{\isachardoublequoteclose}\isanewline
\ \ \ \ \ \ \isacommand{using}\isamarkupfalse%
\ subsetD{\isacharbrackleft}{\kern0pt}OF\ field{\isacharunderscore}{\kern0pt}Rrel{\isacharbrackleft}{\kern0pt}of\ R\ d{\isacharbrackright}{\kern0pt}{\isacharbrackright}{\kern0pt}\ \isacommand{by}\isamarkupfalse%
\ auto\isanewline
\ \ \ \ \isacommand{moreover}\isamarkupfalse%
\ \isacommand{from}\isamarkupfalse%
\ {\isacartoucheopen}w{\isasymnotin}d{\isacartoucheclose}\isanewline
\ \ \ \ \isacommand{have}\isamarkupfalse%
\ {\isachardoublequoteopen}w{\isasymnotin}domain{\isacharparenleft}{\kern0pt}restrict{\isacharparenleft}{\kern0pt}f{\isacharcomma}{\kern0pt}{\isacharquery}{\kern0pt}s{\isacharminus}{\kern0pt}{\isacharbackquote}{\kern0pt}{\isacharbackquote}{\kern0pt}{\isacharbraceleft}{\kern0pt}y{\isacharbraceright}{\kern0pt}{\isacharparenright}{\kern0pt}{\isacharparenright}{\kern0pt}{\isachardoublequoteclose}\isanewline
\ \ \ \ \ \ \isacommand{using}\isamarkupfalse%
\ subsetD{\isacharbrackleft}{\kern0pt}OF\ field{\isacharunderscore}{\kern0pt}Rrel{\isacharbrackleft}{\kern0pt}of\ R\ d{\isacharbrackright}{\kern0pt}{\isacharcomma}{\kern0pt}\ of\ w{\isacharbrackright}{\kern0pt}\ field{\isacharunderscore}{\kern0pt}trancl{\isacharbrackleft}{\kern0pt}of\ {\isacharquery}{\kern0pt}r{\isacharbrackright}{\kern0pt}\isanewline
\ \ \ \ \ \ \ \ fieldI{\isadigit{1}}{\isacharbrackleft}{\kern0pt}of\ w\ y\ {\isacharquery}{\kern0pt}s{\isacharbrackright}{\kern0pt}\ \isacommand{by}\isamarkupfalse%
\ auto\isanewline
\ \ \ \ \isacommand{ultimately}\isamarkupfalse%
\isanewline
\ \ \ \ \isacommand{have}\isamarkupfalse%
\ {\isachardoublequoteopen}restrict{\isacharparenleft}{\kern0pt}f{\isacharcomma}{\kern0pt}{\isacharquery}{\kern0pt}r{\isacharminus}{\kern0pt}{\isacharbackquote}{\kern0pt}{\isacharbackquote}{\kern0pt}{\isacharbraceleft}{\kern0pt}y{\isacharbraceright}{\kern0pt}{\isacharparenright}{\kern0pt}{\isacharbackquote}{\kern0pt}w\ {\isacharequal}{\kern0pt}\ {\isadigit{0}}{\isachardoublequoteclose}\ {\isachardoublequoteopen}restrict{\isacharparenleft}{\kern0pt}f{\isacharcomma}{\kern0pt}{\isacharquery}{\kern0pt}s{\isacharminus}{\kern0pt}{\isacharbackquote}{\kern0pt}{\isacharbackquote}{\kern0pt}{\isacharbraceleft}{\kern0pt}y{\isacharbraceright}{\kern0pt}{\isacharparenright}{\kern0pt}{\isacharbackquote}{\kern0pt}w\ {\isacharequal}{\kern0pt}\ {\isadigit{0}}{\isachardoublequoteclose}\isanewline
\ \ \ \ \ \ \isacommand{unfolding}\isamarkupfalse%
\ apply{\isacharunderscore}{\kern0pt}def\ \isacommand{by}\isamarkupfalse%
\ auto\isanewline
\ \ \ \ \isacommand{then}\isamarkupfalse%
\ \isacommand{show}\isamarkupfalse%
\ {\isacharquery}{\kern0pt}thesis\ \isacommand{by}\isamarkupfalse%
\ simp\isanewline
\ \ \isacommand{qed}\isamarkupfalse%
\isanewline
\isacommand{next}\isamarkupfalse%
\isanewline
\ \ \isacommand{let}\isamarkupfalse%
\ {\isacharquery}{\kern0pt}r{\isacharequal}{\kern0pt}{\isachardoublequoteopen}Rrel{\isacharparenleft}{\kern0pt}R{\isacharcomma}{\kern0pt}d{\isacharparenright}{\kern0pt}{\isachardoublequoteclose}\isanewline
\ \ \isacommand{let}\isamarkupfalse%
\ {\isacharquery}{\kern0pt}s{\isacharequal}{\kern0pt}{\isachardoublequoteopen}{\isacharquery}{\kern0pt}r{\isacharcircum}{\kern0pt}{\isacharplus}{\kern0pt}{\isachardoublequoteclose}\isanewline
\ \ \isacommand{case}\isamarkupfalse%
\ False\isanewline
\ \ \isacommand{then}\isamarkupfalse%
\isanewline
\ \ \isacommand{have}\isamarkupfalse%
\ {\isachardoublequoteopen}{\isacharquery}{\kern0pt}r{\isacharminus}{\kern0pt}{\isacharbackquote}{\kern0pt}{\isacharbackquote}{\kern0pt}{\isacharbraceleft}{\kern0pt}y{\isacharbraceright}{\kern0pt}{\isacharequal}{\kern0pt}{\isadigit{0}}{\isachardoublequoteclose}\isanewline
\ \ \ \ \isacommand{unfolding}\isamarkupfalse%
\ Rrel{\isacharunderscore}{\kern0pt}def\ \isacommand{by}\isamarkupfalse%
\ blast\isanewline
\ \ \isacommand{then}\isamarkupfalse%
\isanewline
\ \ \isacommand{have}\isamarkupfalse%
\ {\isachardoublequoteopen}w{\isasymnotin}{\isacharquery}{\kern0pt}r{\isacharminus}{\kern0pt}{\isacharbackquote}{\kern0pt}{\isacharbackquote}{\kern0pt}{\isacharbraceleft}{\kern0pt}y{\isacharbraceright}{\kern0pt}{\isachardoublequoteclose}\ \isacommand{by}\isamarkupfalse%
\ simp\isanewline
\ \ \isacommand{with}\isamarkupfalse%
\ {\isacartoucheopen}y{\isasymnotin}d{\isacartoucheclose}\ assms\isanewline
\ \ \isacommand{have}\isamarkupfalse%
\ {\isachardoublequoteopen}y{\isasymnotin}field{\isacharparenleft}{\kern0pt}{\isacharquery}{\kern0pt}s{\isacharparenright}{\kern0pt}{\isachardoublequoteclose}\isanewline
\ \ \ \ \isacommand{using}\isamarkupfalse%
\ field{\isacharunderscore}{\kern0pt}trancl\ subsetD{\isacharbrackleft}{\kern0pt}OF\ field{\isacharunderscore}{\kern0pt}Rrel{\isacharbrackleft}{\kern0pt}of\ R\ d{\isacharbrackright}{\kern0pt}{\isacharbrackright}{\kern0pt}\ \isacommand{by}\isamarkupfalse%
\ force\isanewline
\ \ \isacommand{then}\isamarkupfalse%
\isanewline
\ \ \isacommand{have}\isamarkupfalse%
\ {\isachardoublequoteopen}w{\isasymnotin}{\isacharquery}{\kern0pt}s{\isacharminus}{\kern0pt}{\isacharbackquote}{\kern0pt}{\isacharbackquote}{\kern0pt}{\isacharbraceleft}{\kern0pt}y{\isacharbraceright}{\kern0pt}{\isachardoublequoteclose}\isanewline
\ \ \ \ \isacommand{using}\isamarkupfalse%
\ vimage{\isacharunderscore}{\kern0pt}singleton{\isacharunderscore}{\kern0pt}iff\ \isacommand{by}\isamarkupfalse%
\ blast\isanewline
\ \ \isacommand{with}\isamarkupfalse%
\ {\isacartoucheopen}w{\isasymnotin}{\isacharquery}{\kern0pt}r{\isacharminus}{\kern0pt}{\isacharbackquote}{\kern0pt}{\isacharbackquote}{\kern0pt}{\isacharbraceleft}{\kern0pt}y{\isacharbraceright}{\kern0pt}{\isacartoucheclose}\isanewline
\ \ \isacommand{show}\isamarkupfalse%
\ {\isacharquery}{\kern0pt}thesis\ \isacommand{by}\isamarkupfalse%
\ simp\isanewline
\isacommand{qed}\isamarkupfalse%
%
\endisatagproof
{\isafoldproof}%
%
\isadelimproof
\isanewline
%
\endisadelimproof
\isanewline
\isacommand{lemma}\isamarkupfalse%
\ restrict{\isacharunderscore}{\kern0pt}trans{\isacharunderscore}{\kern0pt}eq{\isacharcolon}{\kern0pt}\isanewline
\ \ \isakeyword{assumes}\ {\isachardoublequoteopen}w\ {\isasymin}\ y{\isachardoublequoteclose}\isanewline
\ \ \isakeyword{shows}\ {\isachardoublequoteopen}restrict{\isacharparenleft}{\kern0pt}f{\isacharcomma}{\kern0pt}Memrel{\isacharparenleft}{\kern0pt}eclose{\isacharparenleft}{\kern0pt}{\isacharbraceleft}{\kern0pt}x{\isacharbraceright}{\kern0pt}{\isacharparenright}{\kern0pt}{\isacharparenright}{\kern0pt}{\isacharminus}{\kern0pt}{\isacharbackquote}{\kern0pt}{\isacharbackquote}{\kern0pt}{\isacharbraceleft}{\kern0pt}y{\isacharbraceright}{\kern0pt}{\isacharparenright}{\kern0pt}{\isacharbackquote}{\kern0pt}w\isanewline
\ \ \ \ \ \ \ {\isacharequal}{\kern0pt}\ restrict{\isacharparenleft}{\kern0pt}f{\isacharcomma}{\kern0pt}{\isacharparenleft}{\kern0pt}Memrel{\isacharparenleft}{\kern0pt}eclose{\isacharparenleft}{\kern0pt}{\isacharbraceleft}{\kern0pt}x{\isacharbraceright}{\kern0pt}{\isacharparenright}{\kern0pt}{\isacharparenright}{\kern0pt}{\isacharcircum}{\kern0pt}{\isacharplus}{\kern0pt}{\isacharparenright}{\kern0pt}{\isacharminus}{\kern0pt}{\isacharbackquote}{\kern0pt}{\isacharbackquote}{\kern0pt}{\isacharbraceleft}{\kern0pt}y{\isacharbraceright}{\kern0pt}{\isacharparenright}{\kern0pt}{\isacharbackquote}{\kern0pt}w{\isachardoublequoteclose}\isanewline
%
\isadelimproof
\ \ %
\endisadelimproof
%
\isatagproof
\isacommand{using}\isamarkupfalse%
\ assms\ restrict{\isacharunderscore}{\kern0pt}trancl{\isacharunderscore}{\kern0pt}Rrel{\isacharbrackleft}{\kern0pt}of\ mem\ {\isacharbrackright}{\kern0pt}\ Rrel{\isacharunderscore}{\kern0pt}mem\ \isacommand{by}\isamarkupfalse%
\ {\isacharparenleft}{\kern0pt}simp{\isacharparenright}{\kern0pt}%
\endisatagproof
{\isafoldproof}%
%
\isadelimproof
\isanewline
%
\endisadelimproof
\isanewline
\isacommand{lemma}\isamarkupfalse%
\ wf{\isacharunderscore}{\kern0pt}eq{\isacharunderscore}{\kern0pt}trancl{\isacharcolon}{\kern0pt}\isanewline
\ \ \isakeyword{assumes}\ {\isachardoublequoteopen}{\isasymAnd}\ f\ y\ {\isachardot}{\kern0pt}\ H{\isacharparenleft}{\kern0pt}y{\isacharcomma}{\kern0pt}restrict{\isacharparenleft}{\kern0pt}f{\isacharcomma}{\kern0pt}R{\isacharminus}{\kern0pt}{\isacharbackquote}{\kern0pt}{\isacharbackquote}{\kern0pt}{\isacharbraceleft}{\kern0pt}y{\isacharbraceright}{\kern0pt}{\isacharparenright}{\kern0pt}{\isacharparenright}{\kern0pt}\ {\isacharequal}{\kern0pt}\ H{\isacharparenleft}{\kern0pt}y{\isacharcomma}{\kern0pt}restrict{\isacharparenleft}{\kern0pt}f{\isacharcomma}{\kern0pt}R{\isacharcircum}{\kern0pt}{\isacharplus}{\kern0pt}{\isacharminus}{\kern0pt}{\isacharbackquote}{\kern0pt}{\isacharbackquote}{\kern0pt}{\isacharbraceleft}{\kern0pt}y{\isacharbraceright}{\kern0pt}{\isacharparenright}{\kern0pt}{\isacharparenright}{\kern0pt}{\isachardoublequoteclose}\isanewline
\ \ \isakeyword{shows}\ \ {\isachardoublequoteopen}wfrec{\isacharparenleft}{\kern0pt}R{\isacharcomma}{\kern0pt}\ x{\isacharcomma}{\kern0pt}\ H{\isacharparenright}{\kern0pt}\ {\isacharequal}{\kern0pt}\ wfrec{\isacharparenleft}{\kern0pt}R{\isacharcircum}{\kern0pt}{\isacharplus}{\kern0pt}{\isacharcomma}{\kern0pt}\ x{\isacharcomma}{\kern0pt}\ H{\isacharparenright}{\kern0pt}{\isachardoublequoteclose}\ {\isacharparenleft}{\kern0pt}\isakeyword{is}\ {\isachardoublequoteopen}wfrec{\isacharparenleft}{\kern0pt}{\isacharquery}{\kern0pt}r{\isacharcomma}{\kern0pt}{\isacharunderscore}{\kern0pt}{\isacharcomma}{\kern0pt}{\isacharunderscore}{\kern0pt}{\isacharparenright}{\kern0pt}\ {\isacharequal}{\kern0pt}\ wfrec{\isacharparenleft}{\kern0pt}{\isacharquery}{\kern0pt}r{\isacharprime}{\kern0pt}{\isacharcomma}{\kern0pt}{\isacharunderscore}{\kern0pt}{\isacharcomma}{\kern0pt}{\isacharunderscore}{\kern0pt}{\isacharparenright}{\kern0pt}{\isachardoublequoteclose}{\isacharparenright}{\kern0pt}\isanewline
%
\isadelimproof
%
\endisadelimproof
%
\isatagproof
\isacommand{proof}\isamarkupfalse%
\ {\isacharminus}{\kern0pt}\isanewline
\ \ \isacommand{have}\isamarkupfalse%
\ {\isachardoublequoteopen}wfrec{\isacharparenleft}{\kern0pt}R{\isacharcomma}{\kern0pt}\ x{\isacharcomma}{\kern0pt}\ H{\isacharparenright}{\kern0pt}\ {\isacharequal}{\kern0pt}\ wftrec{\isacharparenleft}{\kern0pt}{\isacharquery}{\kern0pt}r{\isacharcircum}{\kern0pt}{\isacharplus}{\kern0pt}{\isacharcomma}{\kern0pt}\ x{\isacharcomma}{\kern0pt}\ {\isasymlambda}y\ f{\isachardot}{\kern0pt}\ H{\isacharparenleft}{\kern0pt}y{\isacharcomma}{\kern0pt}\ restrict{\isacharparenleft}{\kern0pt}f{\isacharcomma}{\kern0pt}{\isacharquery}{\kern0pt}r{\isacharminus}{\kern0pt}{\isacharbackquote}{\kern0pt}{\isacharbackquote}{\kern0pt}{\isacharbraceleft}{\kern0pt}y{\isacharbraceright}{\kern0pt}{\isacharparenright}{\kern0pt}{\isacharparenright}{\kern0pt}{\isacharparenright}{\kern0pt}{\isachardoublequoteclose}\isanewline
\ \ \ \ \isacommand{unfolding}\isamarkupfalse%
\ wfrec{\isacharunderscore}{\kern0pt}def\ \isacommand{{\isachardot}{\kern0pt}{\isachardot}{\kern0pt}}\isamarkupfalse%
\isanewline
\ \ \isacommand{also}\isamarkupfalse%
\isanewline
\ \ \isacommand{have}\isamarkupfalse%
\ {\isachardoublequoteopen}\ {\isachardot}{\kern0pt}{\isachardot}{\kern0pt}{\isachardot}{\kern0pt}\ {\isacharequal}{\kern0pt}\ wftrec{\isacharparenleft}{\kern0pt}{\isacharquery}{\kern0pt}r{\isacharcircum}{\kern0pt}{\isacharplus}{\kern0pt}{\isacharcomma}{\kern0pt}\ x{\isacharcomma}{\kern0pt}\ {\isasymlambda}y\ f{\isachardot}{\kern0pt}\ H{\isacharparenleft}{\kern0pt}y{\isacharcomma}{\kern0pt}\ restrict{\isacharparenleft}{\kern0pt}f{\isacharcomma}{\kern0pt}{\isacharparenleft}{\kern0pt}{\isacharquery}{\kern0pt}r{\isacharcircum}{\kern0pt}{\isacharplus}{\kern0pt}{\isacharparenright}{\kern0pt}{\isacharminus}{\kern0pt}{\isacharbackquote}{\kern0pt}{\isacharbackquote}{\kern0pt}{\isacharbraceleft}{\kern0pt}y{\isacharbraceright}{\kern0pt}{\isacharparenright}{\kern0pt}{\isacharparenright}{\kern0pt}{\isacharparenright}{\kern0pt}{\isachardoublequoteclose}\isanewline
\ \ \ \ \isacommand{using}\isamarkupfalse%
\ assms\ \isacommand{by}\isamarkupfalse%
\ simp\isanewline
\ \ \isacommand{also}\isamarkupfalse%
\isanewline
\ \ \isacommand{have}\isamarkupfalse%
\ {\isachardoublequoteopen}\ {\isachardot}{\kern0pt}{\isachardot}{\kern0pt}{\isachardot}{\kern0pt}\ {\isacharequal}{\kern0pt}\ \ wfrec{\isacharparenleft}{\kern0pt}{\isacharquery}{\kern0pt}r{\isacharcircum}{\kern0pt}{\isacharplus}{\kern0pt}{\isacharcomma}{\kern0pt}\ x{\isacharcomma}{\kern0pt}\ H{\isacharparenright}{\kern0pt}{\isachardoublequoteclose}\isanewline
\ \ \ \ \isacommand{unfolding}\isamarkupfalse%
\ wfrec{\isacharunderscore}{\kern0pt}def\ \isacommand{using}\isamarkupfalse%
\ trancl{\isacharunderscore}{\kern0pt}eq{\isacharunderscore}{\kern0pt}r{\isacharbrackleft}{\kern0pt}OF\ relation{\isacharunderscore}{\kern0pt}trancl\ trans{\isacharunderscore}{\kern0pt}trancl{\isacharbrackright}{\kern0pt}\ \isacommand{by}\isamarkupfalse%
\ simp\isanewline
\ \ \isacommand{finally}\isamarkupfalse%
\isanewline
\ \ \isacommand{show}\isamarkupfalse%
\ {\isacharquery}{\kern0pt}thesis\ \isacommand{{\isachardot}{\kern0pt}}\isamarkupfalse%
\isanewline
\isacommand{qed}\isamarkupfalse%
%
\endisatagproof
{\isafoldproof}%
%
\isadelimproof
\isanewline
%
\endisadelimproof
%
\isadelimtheory
\isanewline
%
\endisadelimtheory
%
\isatagtheory
\isacommand{end}\isamarkupfalse%
%
\endisatagtheory
{\isafoldtheory}%
%
\isadelimtheory
%
\endisadelimtheory
%
\end{isabellebody}%
\endinput
%:%file=~/source/repos/ZF-notAC/code/Forcing/Recursion_Thms.thy%:%
%:%11=1%:%
%:%27=3%:%
%:%28=3%:%
%:%37=5%:%
%:%38=6%:%
%:%41=10%:%
%:%42=10%:%
%:%45=11%:%
%:%49=11%:%
%:%50=11%:%
%:%55=11%:%
%:%58=12%:%
%:%59=13%:%
%:%60=13%:%
%:%63=14%:%
%:%67=14%:%
%:%68=14%:%
%:%73=14%:%
%:%76=15%:%
%:%77=16%:%
%:%78=16%:%
%:%79=17%:%
%:%80=18%:%
%:%87=19%:%
%:%88=19%:%
%:%89=20%:%
%:%90=20%:%
%:%91=20%:%
%:%92=21%:%
%:%93=21%:%
%:%94=22%:%
%:%95=22%:%
%:%96=22%:%
%:%97=22%:%
%:%98=23%:%
%:%99=23%:%
%:%100=23%:%
%:%101=24%:%
%:%102=24%:%
%:%103=25%:%
%:%104=25%:%
%:%105=26%:%
%:%106=26%:%
%:%107=26%:%
%:%108=26%:%
%:%109=27%:%
%:%110=27%:%
%:%111=28%:%
%:%112=28%:%
%:%113=28%:%
%:%114=28%:%
%:%115=29%:%
%:%121=29%:%
%:%124=30%:%
%:%125=31%:%
%:%126=31%:%
%:%127=32%:%
%:%128=33%:%
%:%129=34%:%
%:%130=34%:%
%:%133=35%:%
%:%137=35%:%
%:%138=35%:%
%:%143=35%:%
%:%146=36%:%
%:%147=37%:%
%:%148=37%:%
%:%151=38%:%
%:%155=38%:%
%:%156=38%:%
%:%161=38%:%
%:%164=39%:%
%:%165=40%:%
%:%166=40%:%
%:%169=41%:%
%:%173=41%:%
%:%174=41%:%
%:%179=41%:%
%:%182=42%:%
%:%183=43%:%
%:%184=43%:%
%:%185=44%:%
%:%186=45%:%
%:%193=46%:%
%:%194=46%:%
%:%195=47%:%
%:%196=47%:%
%:%197=48%:%
%:%198=48%:%
%:%199=49%:%
%:%200=49%:%
%:%201=50%:%
%:%202=50%:%
%:%203=50%:%
%:%204=50%:%
%:%205=51%:%
%:%206=51%:%
%:%207=52%:%
%:%208=52%:%
%:%209=52%:%
%:%210=52%:%
%:%211=53%:%
%:%212=53%:%
%:%213=54%:%
%:%214=54%:%
%:%215=54%:%
%:%216=54%:%
%:%217=55%:%
%:%218=55%:%
%:%219=56%:%
%:%220=56%:%
%:%221=57%:%
%:%222=57%:%
%:%223=58%:%
%:%224=58%:%
%:%225=58%:%
%:%226=59%:%
%:%232=59%:%
%:%235=60%:%
%:%236=61%:%
%:%237=61%:%
%:%240=62%:%
%:%244=62%:%
%:%245=62%:%
%:%250=62%:%
%:%253=63%:%
%:%254=64%:%
%:%255=64%:%
%:%256=65%:%
%:%257=66%:%
%:%264=67%:%
%:%265=67%:%
%:%266=68%:%
%:%267=68%:%
%:%268=69%:%
%:%269=69%:%
%:%270=70%:%
%:%271=70%:%
%:%272=70%:%
%:%273=71%:%
%:%274=71%:%
%:%275=72%:%
%:%276=72%:%
%:%277=72%:%
%:%278=73%:%
%:%279=73%:%
%:%280=74%:%
%:%281=74%:%
%:%282=74%:%
%:%283=75%:%
%:%284=75%:%
%:%285=76%:%
%:%286=76%:%
%:%287=77%:%
%:%288=77%:%
%:%289=78%:%
%:%290=78%:%
%:%291=79%:%
%:%292=79%:%
%:%293=79%:%
%:%294=80%:%
%:%295=80%:%
%:%296=81%:%
%:%297=81%:%
%:%298=81%:%
%:%299=82%:%
%:%300=82%:%
%:%301=83%:%
%:%302=83%:%
%:%303=84%:%
%:%304=84%:%
%:%305=84%:%
%:%306=85%:%
%:%307=85%:%
%:%308=86%:%
%:%309=86%:%
%:%310=86%:%
%:%311=86%:%
%:%312=87%:%
%:%313=87%:%
%:%314=87%:%
%:%315=87%:%
%:%316=88%:%
%:%317=88%:%
%:%318=89%:%
%:%319=89%:%
%:%320=89%:%
%:%321=90%:%
%:%322=90%:%
%:%323=91%:%
%:%324=91%:%
%:%325=92%:%
%:%326=92%:%
%:%327=92%:%
%:%328=93%:%
%:%329=93%:%
%:%330=94%:%
%:%331=94%:%
%:%332=95%:%
%:%333=95%:%
%:%334=95%:%
%:%335=96%:%
%:%336=97%:%
%:%337=97%:%
%:%338=98%:%
%:%339=98%:%
%:%340=98%:%
%:%341=99%:%
%:%342=99%:%
%:%343=100%:%
%:%344=100%:%
%:%345=101%:%
%:%346=101%:%
%:%347=101%:%
%:%348=102%:%
%:%349=102%:%
%:%350=102%:%
%:%351=103%:%
%:%352=103%:%
%:%353=104%:%
%:%354=104%:%
%:%355=104%:%
%:%356=105%:%
%:%357=105%:%
%:%358=105%:%
%:%359=106%:%
%:%360=106%:%
%:%361=107%:%
%:%362=107%:%
%:%363=107%:%
%:%364=108%:%
%:%365=108%:%
%:%366=108%:%
%:%367=109%:%
%:%368=109%:%
%:%369=109%:%
%:%370=109%:%
%:%371=110%:%
%:%372=110%:%
%:%373=110%:%
%:%374=110%:%
%:%375=111%:%
%:%381=111%:%
%:%384=112%:%
%:%385=113%:%
%:%386=113%:%
%:%387=114%:%
%:%388=115%:%
%:%389=115%:%
%:%390=116%:%
%:%393=117%:%
%:%397=117%:%
%:%398=117%:%
%:%403=117%:%
%:%406=118%:%
%:%407=119%:%
%:%408=120%:%
%:%409=120%:%
%:%412=121%:%
%:%416=121%:%
%:%417=121%:%
%:%422=121%:%
%:%425=122%:%
%:%426=123%:%
%:%427=123%:%
%:%428=124%:%
%:%429=125%:%
%:%430=126%:%
%:%431=127%:%
%:%432=127%:%
%:%435=128%:%
%:%439=128%:%
%:%440=128%:%
%:%441=128%:%
%:%446=128%:%
%:%449=129%:%
%:%450=130%:%
%:%451=130%:%
%:%454=131%:%
%:%458=131%:%
%:%459=131%:%
%:%465=131%:%
%:%468=132%:%
%:%469=133%:%
%:%470=133%:%
%:%473=134%:%
%:%477=134%:%
%:%478=134%:%
%:%479=134%:%
%:%484=134%:%
%:%487=135%:%
%:%488=136%:%
%:%489=136%:%
%:%492=137%:%
%:%496=137%:%
%:%497=137%:%
%:%498=137%:%
%:%503=137%:%
%:%506=138%:%
%:%507=139%:%
%:%508=139%:%
%:%511=140%:%
%:%515=140%:%
%:%516=140%:%
%:%517=140%:%
%:%522=140%:%
%:%525=141%:%
%:%526=142%:%
%:%527=142%:%
%:%530=143%:%
%:%534=143%:%
%:%535=143%:%
%:%536=143%:%
%:%541=143%:%
%:%544=144%:%
%:%545=145%:%
%:%546=146%:%
%:%547=146%:%
%:%548=147%:%
%:%551=148%:%
%:%555=148%:%
%:%556=148%:%
%:%557=148%:%
%:%562=148%:%
%:%565=149%:%
%:%566=150%:%
%:%567=150%:%
%:%568=151%:%
%:%569=152%:%
%:%570=153%:%
%:%577=154%:%
%:%578=154%:%
%:%579=155%:%
%:%580=155%:%
%:%581=156%:%
%:%582=156%:%
%:%583=157%:%
%:%584=157%:%
%:%585=158%:%
%:%586=158%:%
%:%587=159%:%
%:%588=159%:%
%:%589=160%:%
%:%590=160%:%
%:%591=161%:%
%:%592=161%:%
%:%593=162%:%
%:%594=162%:%
%:%595=162%:%
%:%596=163%:%
%:%597=163%:%
%:%598=164%:%
%:%599=164%:%
%:%600=165%:%
%:%601=165%:%
%:%602=165%:%
%:%603=166%:%
%:%604=166%:%
%:%605=167%:%
%:%606=167%:%
%:%607=168%:%
%:%608=168%:%
%:%609=168%:%
%:%610=169%:%
%:%611=169%:%
%:%612=170%:%
%:%613=170%:%
%:%614=170%:%
%:%615=171%:%
%:%616=171%:%
%:%617=172%:%
%:%618=172%:%
%:%619=173%:%
%:%620=173%:%
%:%621=174%:%
%:%622=174%:%
%:%623=175%:%
%:%624=175%:%
%:%625=175%:%
%:%626=176%:%
%:%627=176%:%
%:%628=176%:%
%:%629=177%:%
%:%630=177%:%
%:%631=178%:%
%:%632=178%:%
%:%633=179%:%
%:%634=179%:%
%:%635=180%:%
%:%636=180%:%
%:%637=181%:%
%:%638=181%:%
%:%639=182%:%
%:%640=182%:%
%:%641=182%:%
%:%642=183%:%
%:%643=183%:%
%:%644=183%:%
%:%645=183%:%
%:%646=184%:%
%:%647=184%:%
%:%648=185%:%
%:%649=185%:%
%:%650=186%:%
%:%651=186%:%
%:%652=187%:%
%:%653=187%:%
%:%654=188%:%
%:%655=188%:%
%:%656=189%:%
%:%657=189%:%
%:%658=190%:%
%:%659=190%:%
%:%660=191%:%
%:%661=191%:%
%:%662=191%:%
%:%663=192%:%
%:%664=192%:%
%:%665=193%:%
%:%666=193%:%
%:%667=193%:%
%:%668=194%:%
%:%669=194%:%
%:%670=195%:%
%:%671=195%:%
%:%672=196%:%
%:%673=196%:%
%:%674=196%:%
%:%675=197%:%
%:%676=197%:%
%:%677=198%:%
%:%678=198%:%
%:%679=199%:%
%:%680=199%:%
%:%681=199%:%
%:%682=200%:%
%:%683=200%:%
%:%684=201%:%
%:%685=201%:%
%:%686=201%:%
%:%687=202%:%
%:%693=202%:%
%:%696=203%:%
%:%697=204%:%
%:%698=204%:%
%:%699=205%:%
%:%700=206%:%
%:%701=207%:%
%:%704=208%:%
%:%708=208%:%
%:%709=208%:%
%:%710=208%:%
%:%715=208%:%
%:%718=209%:%
%:%719=210%:%
%:%720=210%:%
%:%721=211%:%
%:%722=212%:%
%:%729=213%:%
%:%730=213%:%
%:%731=214%:%
%:%732=214%:%
%:%733=215%:%
%:%734=215%:%
%:%735=215%:%
%:%736=216%:%
%:%737=216%:%
%:%738=217%:%
%:%739=217%:%
%:%740=218%:%
%:%741=218%:%
%:%742=218%:%
%:%743=219%:%
%:%744=219%:%
%:%745=220%:%
%:%746=220%:%
%:%747=221%:%
%:%748=221%:%
%:%749=221%:%
%:%750=221%:%
%:%751=222%:%
%:%752=222%:%
%:%753=223%:%
%:%754=223%:%
%:%755=223%:%
%:%756=224%:%
%:%762=224%:%
%:%767=225%:%
%:%772=226%:%

%
\begin{isabellebody}%
\setisabellecontext{Relative{\isacharunderscore}{\kern0pt}Univ}%
%
\isadelimdocument
%
\endisadelimdocument
%
\isatagdocument
%
\isamarkupsection{Relativization of the cumulative hierarchy%
}
\isamarkuptrue%
%
\endisatagdocument
{\isafolddocument}%
%
\isadelimdocument
%
\endisadelimdocument
%
\isadelimtheory
%
\endisadelimtheory
%
\isatagtheory
\isacommand{theory}\isamarkupfalse%
\ Relative{\isacharunderscore}{\kern0pt}Univ\isanewline
\ \ \isakeyword{imports}\isanewline
\ \ \ \ {\isachardoublequoteopen}ZF{\isacharminus}{\kern0pt}Constructible{\isachardot}{\kern0pt}Rank{\isachardoublequoteclose}\isanewline
\ \ \ \ Internalizations\isanewline
\ \ \ \ Recursion{\isacharunderscore}{\kern0pt}Thms\isanewline
\isanewline
\isakeyword{begin}%
\endisatagtheory
{\isafoldtheory}%
%
\isadelimtheory
\isanewline
%
\endisadelimtheory
\isanewline
\isacommand{lemma}\isamarkupfalse%
\ {\isacharparenleft}{\kern0pt}\isakeyword{in}\ M{\isacharunderscore}{\kern0pt}trivial{\isacharparenright}{\kern0pt}\ powerset{\isacharunderscore}{\kern0pt}abs{\isacharprime}{\kern0pt}\ {\isacharbrackleft}{\kern0pt}simp{\isacharbrackright}{\kern0pt}{\isacharcolon}{\kern0pt}\ \isanewline
\ \ \isakeyword{assumes}\isanewline
\ \ \ \ {\isachardoublequoteopen}M{\isacharparenleft}{\kern0pt}x{\isacharparenright}{\kern0pt}{\isachardoublequoteclose}\ {\isachardoublequoteopen}M{\isacharparenleft}{\kern0pt}y{\isacharparenright}{\kern0pt}{\isachardoublequoteclose}\isanewline
\ \ \isakeyword{shows}\isanewline
\ \ \ \ {\isachardoublequoteopen}powerset{\isacharparenleft}{\kern0pt}M{\isacharcomma}{\kern0pt}x{\isacharcomma}{\kern0pt}y{\isacharparenright}{\kern0pt}\ {\isasymlongleftrightarrow}\ y\ {\isacharequal}{\kern0pt}\ {\isacharbraceleft}{\kern0pt}a{\isasymin}Pow{\isacharparenleft}{\kern0pt}x{\isacharparenright}{\kern0pt}\ {\isachardot}{\kern0pt}\ M{\isacharparenleft}{\kern0pt}a{\isacharparenright}{\kern0pt}{\isacharbraceright}{\kern0pt}{\isachardoublequoteclose}\isanewline
%
\isadelimproof
\ \ %
\endisadelimproof
%
\isatagproof
\isacommand{using}\isamarkupfalse%
\ powerset{\isacharunderscore}{\kern0pt}abs\ assms\ \isacommand{by}\isamarkupfalse%
\ simp%
\endisatagproof
{\isafoldproof}%
%
\isadelimproof
\isanewline
%
\endisadelimproof
\isanewline
\isacommand{lemma}\isamarkupfalse%
\ Collect{\isacharunderscore}{\kern0pt}inter{\isacharunderscore}{\kern0pt}Transset{\isacharcolon}{\kern0pt}\isanewline
\ \ \isakeyword{assumes}\ \isanewline
\ \ \ \ {\isachardoublequoteopen}Transset{\isacharparenleft}{\kern0pt}M{\isacharparenright}{\kern0pt}{\isachardoublequoteclose}\ {\isachardoublequoteopen}b\ {\isasymin}\ M{\isachardoublequoteclose}\isanewline
\ \ \isakeyword{shows}\isanewline
\ \ \ \ {\isachardoublequoteopen}{\isacharbraceleft}{\kern0pt}x{\isasymin}b\ {\isachardot}{\kern0pt}\ P{\isacharparenleft}{\kern0pt}x{\isacharparenright}{\kern0pt}{\isacharbraceright}{\kern0pt}\ {\isacharequal}{\kern0pt}\ {\isacharbraceleft}{\kern0pt}x{\isasymin}b\ {\isachardot}{\kern0pt}\ P{\isacharparenleft}{\kern0pt}x{\isacharparenright}{\kern0pt}{\isacharbraceright}{\kern0pt}\ {\isasyminter}\ M{\isachardoublequoteclose}\isanewline
%
\isadelimproof
\ \ \ \ %
\endisadelimproof
%
\isatagproof
\isacommand{using}\isamarkupfalse%
\ assms\ \isacommand{unfolding}\isamarkupfalse%
\ Transset{\isacharunderscore}{\kern0pt}def\isanewline
\ \ \isacommand{by}\isamarkupfalse%
\ {\isacharparenleft}{\kern0pt}auto{\isacharparenright}{\kern0pt}%
\endisatagproof
{\isafoldproof}%
%
\isadelimproof
\ \ \isanewline
%
\endisadelimproof
\isanewline
\isacommand{lemma}\isamarkupfalse%
\ {\isacharparenleft}{\kern0pt}\isakeyword{in}\ M{\isacharunderscore}{\kern0pt}trivial{\isacharparenright}{\kern0pt}\ family{\isacharunderscore}{\kern0pt}union{\isacharunderscore}{\kern0pt}closed{\isacharcolon}{\kern0pt}\ {\isachardoublequoteopen}{\isasymlbrakk}strong{\isacharunderscore}{\kern0pt}replacement{\isacharparenleft}{\kern0pt}M{\isacharcomma}{\kern0pt}\ {\isasymlambda}x\ y{\isachardot}{\kern0pt}\ y\ {\isacharequal}{\kern0pt}\ f{\isacharparenleft}{\kern0pt}x{\isacharparenright}{\kern0pt}{\isacharparenright}{\kern0pt}{\isacharsemicolon}{\kern0pt}\ M{\isacharparenleft}{\kern0pt}A{\isacharparenright}{\kern0pt}{\isacharsemicolon}{\kern0pt}\ {\isasymforall}x{\isasymin}A{\isachardot}{\kern0pt}\ M{\isacharparenleft}{\kern0pt}f{\isacharparenleft}{\kern0pt}x{\isacharparenright}{\kern0pt}{\isacharparenright}{\kern0pt}{\isasymrbrakk}\isanewline
\ \ \ \ \ \ {\isasymLongrightarrow}\ M{\isacharparenleft}{\kern0pt}{\isasymUnion}x{\isasymin}A{\isachardot}{\kern0pt}\ f{\isacharparenleft}{\kern0pt}x{\isacharparenright}{\kern0pt}{\isacharparenright}{\kern0pt}{\isachardoublequoteclose}\isanewline
%
\isadelimproof
\ \ %
\endisadelimproof
%
\isatagproof
\isacommand{using}\isamarkupfalse%
\ RepFun{\isacharunderscore}{\kern0pt}closed\ \isacommand{{\isachardot}{\kern0pt}{\isachardot}{\kern0pt}}\isamarkupfalse%
%
\endisatagproof
{\isafoldproof}%
%
\isadelimproof
\isanewline
%
\endisadelimproof
\isanewline
\isanewline
\isanewline
\isacommand{definition}\isamarkupfalse%
\isanewline
\ \ HVfrom\ {\isacharcolon}{\kern0pt}{\isacharcolon}{\kern0pt}\ {\isachardoublequoteopen}{\isacharbrackleft}{\kern0pt}i{\isasymRightarrow}o{\isacharcomma}{\kern0pt}i{\isacharcomma}{\kern0pt}i{\isacharcomma}{\kern0pt}i{\isacharbrackright}{\kern0pt}\ {\isasymRightarrow}\ i{\isachardoublequoteclose}\ \isakeyword{where}\isanewline
\ \ {\isachardoublequoteopen}HVfrom{\isacharparenleft}{\kern0pt}M{\isacharcomma}{\kern0pt}A{\isacharcomma}{\kern0pt}x{\isacharcomma}{\kern0pt}f{\isacharparenright}{\kern0pt}\ {\isasymequiv}\ A\ {\isasymunion}\ {\isacharparenleft}{\kern0pt}{\isasymUnion}y{\isasymin}x{\isachardot}{\kern0pt}\ {\isacharbraceleft}{\kern0pt}a{\isasymin}Pow{\isacharparenleft}{\kern0pt}f{\isacharbackquote}{\kern0pt}y{\isacharparenright}{\kern0pt}{\isachardot}{\kern0pt}\ M{\isacharparenleft}{\kern0pt}a{\isacharparenright}{\kern0pt}{\isacharbraceright}{\kern0pt}{\isacharparenright}{\kern0pt}{\isachardoublequoteclose}\isanewline
\isanewline
\isanewline
\isacommand{definition}\isamarkupfalse%
\isanewline
\ \ is{\isacharunderscore}{\kern0pt}powapply\ {\isacharcolon}{\kern0pt}{\isacharcolon}{\kern0pt}\ {\isachardoublequoteopen}{\isacharbrackleft}{\kern0pt}i{\isasymRightarrow}o{\isacharcomma}{\kern0pt}i{\isacharcomma}{\kern0pt}i{\isacharcomma}{\kern0pt}i{\isacharbrackright}{\kern0pt}\ {\isasymRightarrow}\ o{\isachardoublequoteclose}\ \isakeyword{where}\isanewline
\ \ {\isachardoublequoteopen}is{\isacharunderscore}{\kern0pt}powapply{\isacharparenleft}{\kern0pt}M{\isacharcomma}{\kern0pt}f{\isacharcomma}{\kern0pt}y{\isacharcomma}{\kern0pt}z{\isacharparenright}{\kern0pt}\ {\isasymequiv}\ M{\isacharparenleft}{\kern0pt}z{\isacharparenright}{\kern0pt}\ {\isasymand}\ {\isacharparenleft}{\kern0pt}{\isasymexists}fy{\isacharbrackleft}{\kern0pt}M{\isacharbrackright}{\kern0pt}{\isachardot}{\kern0pt}\ fun{\isacharunderscore}{\kern0pt}apply{\isacharparenleft}{\kern0pt}M{\isacharcomma}{\kern0pt}f{\isacharcomma}{\kern0pt}y{\isacharcomma}{\kern0pt}fy{\isacharparenright}{\kern0pt}\ {\isasymand}\ powerset{\isacharparenleft}{\kern0pt}M{\isacharcomma}{\kern0pt}fy{\isacharcomma}{\kern0pt}z{\isacharparenright}{\kern0pt}{\isacharparenright}{\kern0pt}{\isachardoublequoteclose}\isanewline
\isanewline
\isanewline
\isacommand{lemma}\isamarkupfalse%
\ is{\isacharunderscore}{\kern0pt}powapply{\isacharunderscore}{\kern0pt}closed{\isacharcolon}{\kern0pt}\ {\isachardoublequoteopen}is{\isacharunderscore}{\kern0pt}powapply{\isacharparenleft}{\kern0pt}M{\isacharcomma}{\kern0pt}f{\isacharcomma}{\kern0pt}y{\isacharcomma}{\kern0pt}z{\isacharparenright}{\kern0pt}\ {\isasymLongrightarrow}\ M{\isacharparenleft}{\kern0pt}z{\isacharparenright}{\kern0pt}{\isachardoublequoteclose}\isanewline
%
\isadelimproof
\ \ %
\endisadelimproof
%
\isatagproof
\isacommand{unfolding}\isamarkupfalse%
\ is{\isacharunderscore}{\kern0pt}powapply{\isacharunderscore}{\kern0pt}def\ \isacommand{by}\isamarkupfalse%
\ simp%
\endisatagproof
{\isafoldproof}%
%
\isadelimproof
\isanewline
%
\endisadelimproof
\isanewline
\isanewline
\isacommand{definition}\isamarkupfalse%
\isanewline
\ \ is{\isacharunderscore}{\kern0pt}HVfrom\ {\isacharcolon}{\kern0pt}{\isacharcolon}{\kern0pt}\ {\isachardoublequoteopen}{\isacharbrackleft}{\kern0pt}i{\isasymRightarrow}o{\isacharcomma}{\kern0pt}i{\isacharcomma}{\kern0pt}i{\isacharcomma}{\kern0pt}i{\isacharcomma}{\kern0pt}i{\isacharbrackright}{\kern0pt}\ {\isasymRightarrow}\ o{\isachardoublequoteclose}\ \isakeyword{where}\isanewline
\ \ {\isachardoublequoteopen}is{\isacharunderscore}{\kern0pt}HVfrom{\isacharparenleft}{\kern0pt}M{\isacharcomma}{\kern0pt}A{\isacharcomma}{\kern0pt}x{\isacharcomma}{\kern0pt}f{\isacharcomma}{\kern0pt}h{\isacharparenright}{\kern0pt}\ {\isasymequiv}\ {\isasymexists}U{\isacharbrackleft}{\kern0pt}M{\isacharbrackright}{\kern0pt}{\isachardot}{\kern0pt}\ {\isasymexists}R{\isacharbrackleft}{\kern0pt}M{\isacharbrackright}{\kern0pt}{\isachardot}{\kern0pt}\ \ union{\isacharparenleft}{\kern0pt}M{\isacharcomma}{\kern0pt}A{\isacharcomma}{\kern0pt}U{\isacharcomma}{\kern0pt}h{\isacharparenright}{\kern0pt}\ \isanewline
\ \ \ \ \ \ \ \ {\isasymand}\ big{\isacharunderscore}{\kern0pt}union{\isacharparenleft}{\kern0pt}M{\isacharcomma}{\kern0pt}R{\isacharcomma}{\kern0pt}U{\isacharparenright}{\kern0pt}\ {\isasymand}\ is{\isacharunderscore}{\kern0pt}Replace{\isacharparenleft}{\kern0pt}M{\isacharcomma}{\kern0pt}x{\isacharcomma}{\kern0pt}is{\isacharunderscore}{\kern0pt}powapply{\isacharparenleft}{\kern0pt}M{\isacharcomma}{\kern0pt}f{\isacharparenright}{\kern0pt}{\isacharcomma}{\kern0pt}R{\isacharparenright}{\kern0pt}{\isachardoublequoteclose}\ \isanewline
\isanewline
\isanewline
\isacommand{definition}\isamarkupfalse%
\isanewline
\ \ is{\isacharunderscore}{\kern0pt}Vfrom\ {\isacharcolon}{\kern0pt}{\isacharcolon}{\kern0pt}\ {\isachardoublequoteopen}{\isacharbrackleft}{\kern0pt}i{\isasymRightarrow}o{\isacharcomma}{\kern0pt}i{\isacharcomma}{\kern0pt}i{\isacharcomma}{\kern0pt}i{\isacharbrackright}{\kern0pt}\ {\isasymRightarrow}\ o{\isachardoublequoteclose}\ \isakeyword{where}\isanewline
\ \ {\isachardoublequoteopen}is{\isacharunderscore}{\kern0pt}Vfrom{\isacharparenleft}{\kern0pt}M{\isacharcomma}{\kern0pt}A{\isacharcomma}{\kern0pt}i{\isacharcomma}{\kern0pt}V{\isacharparenright}{\kern0pt}\ {\isasymequiv}\ is{\isacharunderscore}{\kern0pt}transrec{\isacharparenleft}{\kern0pt}M{\isacharcomma}{\kern0pt}is{\isacharunderscore}{\kern0pt}HVfrom{\isacharparenleft}{\kern0pt}M{\isacharcomma}{\kern0pt}A{\isacharparenright}{\kern0pt}{\isacharcomma}{\kern0pt}i{\isacharcomma}{\kern0pt}V{\isacharparenright}{\kern0pt}{\isachardoublequoteclose}\isanewline
\isanewline
\isacommand{definition}\isamarkupfalse%
\isanewline
\ \ is{\isacharunderscore}{\kern0pt}Vset\ {\isacharcolon}{\kern0pt}{\isacharcolon}{\kern0pt}\ {\isachardoublequoteopen}{\isacharbrackleft}{\kern0pt}i{\isasymRightarrow}o{\isacharcomma}{\kern0pt}i{\isacharcomma}{\kern0pt}i{\isacharbrackright}{\kern0pt}\ {\isasymRightarrow}\ o{\isachardoublequoteclose}\ \isakeyword{where}\isanewline
\ \ {\isachardoublequoteopen}is{\isacharunderscore}{\kern0pt}Vset{\isacharparenleft}{\kern0pt}M{\isacharcomma}{\kern0pt}i{\isacharcomma}{\kern0pt}V{\isacharparenright}{\kern0pt}\ {\isasymequiv}\ {\isasymexists}z{\isacharbrackleft}{\kern0pt}M{\isacharbrackright}{\kern0pt}{\isachardot}{\kern0pt}\ empty{\isacharparenleft}{\kern0pt}M{\isacharcomma}{\kern0pt}z{\isacharparenright}{\kern0pt}\ {\isasymand}\ is{\isacharunderscore}{\kern0pt}Vfrom{\isacharparenleft}{\kern0pt}M{\isacharcomma}{\kern0pt}z{\isacharcomma}{\kern0pt}i{\isacharcomma}{\kern0pt}V{\isacharparenright}{\kern0pt}{\isachardoublequoteclose}%
\isadelimdocument
%
\endisadelimdocument
%
\isatagdocument
%
\isamarkupsubsection{Formula synthesis%
}
\isamarkuptrue%
%
\endisatagdocument
{\isafolddocument}%
%
\isadelimdocument
%
\endisadelimdocument
\isacommand{schematic{\isacharunderscore}{\kern0pt}goal}\isamarkupfalse%
\ sats{\isacharunderscore}{\kern0pt}is{\isacharunderscore}{\kern0pt}powapply{\isacharunderscore}{\kern0pt}fm{\isacharunderscore}{\kern0pt}auto{\isacharcolon}{\kern0pt}\isanewline
\ \ \isakeyword{assumes}\isanewline
\ \ \ \ {\isachardoublequoteopen}f{\isasymin}nat{\isachardoublequoteclose}\ {\isachardoublequoteopen}y{\isasymin}nat{\isachardoublequoteclose}\ {\isachardoublequoteopen}z{\isasymin}nat{\isachardoublequoteclose}\ {\isachardoublequoteopen}env{\isasymin}list{\isacharparenleft}{\kern0pt}A{\isacharparenright}{\kern0pt}{\isachardoublequoteclose}\ {\isachardoublequoteopen}{\isadigit{0}}{\isasymin}A{\isachardoublequoteclose}\isanewline
\ \ \isakeyword{shows}\isanewline
\ \ \ \ {\isachardoublequoteopen}is{\isacharunderscore}{\kern0pt}powapply{\isacharparenleft}{\kern0pt}{\isacharhash}{\kern0pt}{\isacharhash}{\kern0pt}A{\isacharcomma}{\kern0pt}nth{\isacharparenleft}{\kern0pt}f{\isacharcomma}{\kern0pt}\ env{\isacharparenright}{\kern0pt}{\isacharcomma}{\kern0pt}nth{\isacharparenleft}{\kern0pt}y{\isacharcomma}{\kern0pt}\ env{\isacharparenright}{\kern0pt}{\isacharcomma}{\kern0pt}nth{\isacharparenleft}{\kern0pt}z{\isacharcomma}{\kern0pt}\ env{\isacharparenright}{\kern0pt}{\isacharparenright}{\kern0pt}\isanewline
\ \ \ \ {\isasymlongleftrightarrow}\ sats{\isacharparenleft}{\kern0pt}A{\isacharcomma}{\kern0pt}{\isacharquery}{\kern0pt}ipa{\isacharunderscore}{\kern0pt}fm{\isacharparenleft}{\kern0pt}f{\isacharcomma}{\kern0pt}y{\isacharcomma}{\kern0pt}z{\isacharparenright}{\kern0pt}{\isacharcomma}{\kern0pt}env{\isacharparenright}{\kern0pt}{\isachardoublequoteclose}\isanewline
%
\isadelimproof
\ \ %
\endisadelimproof
%
\isatagproof
\isacommand{unfolding}\isamarkupfalse%
\ is{\isacharunderscore}{\kern0pt}powapply{\isacharunderscore}{\kern0pt}def\ is{\isacharunderscore}{\kern0pt}Collect{\isacharunderscore}{\kern0pt}def\ powerset{\isacharunderscore}{\kern0pt}def\ subset{\isacharunderscore}{\kern0pt}def\isanewline
\ \ \isacommand{using}\isamarkupfalse%
\ nth{\isacharunderscore}{\kern0pt}closed\ assms\isanewline
\ \ \ \isacommand{by}\isamarkupfalse%
\ {\isacharparenleft}{\kern0pt}simp{\isacharparenright}{\kern0pt}\ {\isacharparenleft}{\kern0pt}rule\ sep{\isacharunderscore}{\kern0pt}rules\ \ {\isacharbar}{\kern0pt}\ simp{\isacharparenright}{\kern0pt}{\isacharplus}{\kern0pt}%
\endisatagproof
{\isafoldproof}%
%
\isadelimproof
\isanewline
%
\endisadelimproof
\isanewline
\isacommand{schematic{\isacharunderscore}{\kern0pt}goal}\isamarkupfalse%
\ is{\isacharunderscore}{\kern0pt}powapply{\isacharunderscore}{\kern0pt}iff{\isacharunderscore}{\kern0pt}sats{\isacharcolon}{\kern0pt}\isanewline
\ \ \isakeyword{assumes}\isanewline
\ \ \ \ {\isachardoublequoteopen}nth{\isacharparenleft}{\kern0pt}f{\isacharcomma}{\kern0pt}env{\isacharparenright}{\kern0pt}\ {\isacharequal}{\kern0pt}\ ff{\isachardoublequoteclose}\ {\isachardoublequoteopen}nth{\isacharparenleft}{\kern0pt}y{\isacharcomma}{\kern0pt}env{\isacharparenright}{\kern0pt}\ {\isacharequal}{\kern0pt}\ yy{\isachardoublequoteclose}\ {\isachardoublequoteopen}nth{\isacharparenleft}{\kern0pt}z{\isacharcomma}{\kern0pt}env{\isacharparenright}{\kern0pt}\ {\isacharequal}{\kern0pt}\ zz{\isachardoublequoteclose}\ {\isachardoublequoteopen}{\isadigit{0}}{\isasymin}A{\isachardoublequoteclose}\isanewline
\ \ \ \ {\isachardoublequoteopen}f\ {\isasymin}\ nat{\isachardoublequoteclose}\ \ {\isachardoublequoteopen}y\ {\isasymin}\ nat{\isachardoublequoteclose}\ {\isachardoublequoteopen}z\ {\isasymin}\ nat{\isachardoublequoteclose}\ {\isachardoublequoteopen}env\ {\isasymin}\ list{\isacharparenleft}{\kern0pt}A{\isacharparenright}{\kern0pt}{\isachardoublequoteclose}\isanewline
\ \ \isakeyword{shows}\isanewline
\ \ \ \ \ \ \ {\isachardoublequoteopen}is{\isacharunderscore}{\kern0pt}powapply{\isacharparenleft}{\kern0pt}{\isacharhash}{\kern0pt}{\isacharhash}{\kern0pt}A{\isacharcomma}{\kern0pt}ff{\isacharcomma}{\kern0pt}yy{\isacharcomma}{\kern0pt}zz{\isacharparenright}{\kern0pt}\ {\isasymlongleftrightarrow}\ sats{\isacharparenleft}{\kern0pt}A{\isacharcomma}{\kern0pt}\ {\isacharquery}{\kern0pt}is{\isacharunderscore}{\kern0pt}one{\isacharunderscore}{\kern0pt}fm{\isacharparenleft}{\kern0pt}a{\isacharcomma}{\kern0pt}r{\isacharparenright}{\kern0pt}{\isacharcomma}{\kern0pt}\ env{\isacharparenright}{\kern0pt}{\isachardoublequoteclose}\isanewline
%
\isadelimproof
\ \ %
\endisadelimproof
%
\isatagproof
\isacommand{unfolding}\isamarkupfalse%
\ {\isacartoucheopen}nth{\isacharparenleft}{\kern0pt}f{\isacharcomma}{\kern0pt}env{\isacharparenright}{\kern0pt}\ {\isacharequal}{\kern0pt}\ ff{\isacartoucheclose}{\isacharbrackleft}{\kern0pt}symmetric{\isacharbrackright}{\kern0pt}\ {\isacartoucheopen}nth{\isacharparenleft}{\kern0pt}y{\isacharcomma}{\kern0pt}env{\isacharparenright}{\kern0pt}\ {\isacharequal}{\kern0pt}\ yy{\isacartoucheclose}{\isacharbrackleft}{\kern0pt}symmetric{\isacharbrackright}{\kern0pt}\isanewline
\ \ \ \ {\isacartoucheopen}nth{\isacharparenleft}{\kern0pt}z{\isacharcomma}{\kern0pt}env{\isacharparenright}{\kern0pt}\ {\isacharequal}{\kern0pt}\ zz{\isacartoucheclose}{\isacharbrackleft}{\kern0pt}symmetric{\isacharbrackright}{\kern0pt}\isanewline
\ \ \isacommand{by}\isamarkupfalse%
\ {\isacharparenleft}{\kern0pt}rule\ sats{\isacharunderscore}{\kern0pt}is{\isacharunderscore}{\kern0pt}powapply{\isacharunderscore}{\kern0pt}fm{\isacharunderscore}{\kern0pt}auto{\isacharparenleft}{\kern0pt}{\isadigit{1}}{\isacharparenright}{\kern0pt}{\isacharsemicolon}{\kern0pt}\ simp\ add{\isacharcolon}{\kern0pt}assms{\isacharparenright}{\kern0pt}%
\endisatagproof
{\isafoldproof}%
%
\isadelimproof
\isanewline
%
\endisadelimproof
\isanewline
\isanewline
\isacommand{definition}\isamarkupfalse%
\isanewline
\ \ Hrank\ {\isacharcolon}{\kern0pt}{\isacharcolon}{\kern0pt}\ {\isachardoublequoteopen}{\isacharbrackleft}{\kern0pt}i{\isacharcomma}{\kern0pt}i{\isacharbrackright}{\kern0pt}\ {\isasymRightarrow}\ i{\isachardoublequoteclose}\ \isakeyword{where}\isanewline
\ \ {\isachardoublequoteopen}Hrank{\isacharparenleft}{\kern0pt}x{\isacharcomma}{\kern0pt}f{\isacharparenright}{\kern0pt}\ {\isacharequal}{\kern0pt}\ {\isacharparenleft}{\kern0pt}{\isasymUnion}y{\isasymin}x{\isachardot}{\kern0pt}\ succ{\isacharparenleft}{\kern0pt}f{\isacharbackquote}{\kern0pt}y{\isacharparenright}{\kern0pt}{\isacharparenright}{\kern0pt}{\isachardoublequoteclose}\isanewline
\isanewline
\isacommand{definition}\isamarkupfalse%
\isanewline
\ \ PHrank\ {\isacharcolon}{\kern0pt}{\isacharcolon}{\kern0pt}\ {\isachardoublequoteopen}{\isacharbrackleft}{\kern0pt}i{\isasymRightarrow}o{\isacharcomma}{\kern0pt}i{\isacharcomma}{\kern0pt}i{\isacharcomma}{\kern0pt}i{\isacharbrackright}{\kern0pt}\ {\isasymRightarrow}\ o{\isachardoublequoteclose}\ \isakeyword{where}\isanewline
\ \ {\isachardoublequoteopen}PHrank{\isacharparenleft}{\kern0pt}M{\isacharcomma}{\kern0pt}f{\isacharcomma}{\kern0pt}y{\isacharcomma}{\kern0pt}z{\isacharparenright}{\kern0pt}\ {\isasymequiv}\ M{\isacharparenleft}{\kern0pt}z{\isacharparenright}{\kern0pt}\ {\isasymand}\ {\isacharparenleft}{\kern0pt}{\isasymexists}fy{\isacharbrackleft}{\kern0pt}M{\isacharbrackright}{\kern0pt}{\isachardot}{\kern0pt}\ fun{\isacharunderscore}{\kern0pt}apply{\isacharparenleft}{\kern0pt}M{\isacharcomma}{\kern0pt}f{\isacharcomma}{\kern0pt}y{\isacharcomma}{\kern0pt}fy{\isacharparenright}{\kern0pt}\ {\isasymand}\ successor{\isacharparenleft}{\kern0pt}M{\isacharcomma}{\kern0pt}fy{\isacharcomma}{\kern0pt}z{\isacharparenright}{\kern0pt}{\isacharparenright}{\kern0pt}{\isachardoublequoteclose}\isanewline
\isanewline
\isacommand{definition}\isamarkupfalse%
\isanewline
\ \ is{\isacharunderscore}{\kern0pt}Hrank\ {\isacharcolon}{\kern0pt}{\isacharcolon}{\kern0pt}\ {\isachardoublequoteopen}{\isacharbrackleft}{\kern0pt}i{\isasymRightarrow}o{\isacharcomma}{\kern0pt}i{\isacharcomma}{\kern0pt}i{\isacharcomma}{\kern0pt}i{\isacharbrackright}{\kern0pt}\ {\isasymRightarrow}\ o{\isachardoublequoteclose}\ \isakeyword{where}\isanewline
\ \ {\isachardoublequoteopen}is{\isacharunderscore}{\kern0pt}Hrank{\isacharparenleft}{\kern0pt}M{\isacharcomma}{\kern0pt}x{\isacharcomma}{\kern0pt}f{\isacharcomma}{\kern0pt}hc{\isacharparenright}{\kern0pt}\ {\isasymequiv}\ {\isacharparenleft}{\kern0pt}{\isasymexists}R{\isacharbrackleft}{\kern0pt}M{\isacharbrackright}{\kern0pt}{\isachardot}{\kern0pt}\ big{\isacharunderscore}{\kern0pt}union{\isacharparenleft}{\kern0pt}M{\isacharcomma}{\kern0pt}R{\isacharcomma}{\kern0pt}hc{\isacharparenright}{\kern0pt}\ {\isasymand}is{\isacharunderscore}{\kern0pt}Replace{\isacharparenleft}{\kern0pt}M{\isacharcomma}{\kern0pt}x{\isacharcomma}{\kern0pt}PHrank{\isacharparenleft}{\kern0pt}M{\isacharcomma}{\kern0pt}f{\isacharparenright}{\kern0pt}{\isacharcomma}{\kern0pt}R{\isacharparenright}{\kern0pt}{\isacharparenright}{\kern0pt}\ {\isachardoublequoteclose}\isanewline
\isanewline
\isacommand{definition}\isamarkupfalse%
\isanewline
\ \ rrank\ {\isacharcolon}{\kern0pt}{\isacharcolon}{\kern0pt}\ {\isachardoublequoteopen}i\ {\isasymRightarrow}\ i{\isachardoublequoteclose}\ \isakeyword{where}\isanewline
\ \ {\isachardoublequoteopen}rrank{\isacharparenleft}{\kern0pt}a{\isacharparenright}{\kern0pt}\ {\isasymequiv}\ Memrel{\isacharparenleft}{\kern0pt}eclose{\isacharparenleft}{\kern0pt}{\isacharbraceleft}{\kern0pt}a{\isacharbraceright}{\kern0pt}{\isacharparenright}{\kern0pt}{\isacharparenright}{\kern0pt}{\isacharcircum}{\kern0pt}{\isacharplus}{\kern0pt}{\isachardoublequoteclose}\ \isanewline
\isanewline
\isacommand{lemma}\isamarkupfalse%
\ {\isacharparenleft}{\kern0pt}\isakeyword{in}\ M{\isacharunderscore}{\kern0pt}eclose{\isacharparenright}{\kern0pt}\ wf{\isacharunderscore}{\kern0pt}rrank\ {\isacharcolon}{\kern0pt}\ {\isachardoublequoteopen}M{\isacharparenleft}{\kern0pt}x{\isacharparenright}{\kern0pt}\ {\isasymLongrightarrow}\ wf{\isacharparenleft}{\kern0pt}rrank{\isacharparenleft}{\kern0pt}x{\isacharparenright}{\kern0pt}{\isacharparenright}{\kern0pt}{\isachardoublequoteclose}\ \isanewline
%
\isadelimproof
\ \ %
\endisadelimproof
%
\isatagproof
\isacommand{unfolding}\isamarkupfalse%
\ rrank{\isacharunderscore}{\kern0pt}def\ \isacommand{using}\isamarkupfalse%
\ wf{\isacharunderscore}{\kern0pt}trancl{\isacharbrackleft}{\kern0pt}OF\ wf{\isacharunderscore}{\kern0pt}Memrel{\isacharbrackright}{\kern0pt}\ \isacommand{{\isachardot}{\kern0pt}}\isamarkupfalse%
%
\endisatagproof
{\isafoldproof}%
%
\isadelimproof
\isanewline
%
\endisadelimproof
\isanewline
\isacommand{lemma}\isamarkupfalse%
\ {\isacharparenleft}{\kern0pt}\isakeyword{in}\ M{\isacharunderscore}{\kern0pt}eclose{\isacharparenright}{\kern0pt}\ trans{\isacharunderscore}{\kern0pt}rrank\ {\isacharcolon}{\kern0pt}\ {\isachardoublequoteopen}M{\isacharparenleft}{\kern0pt}x{\isacharparenright}{\kern0pt}\ {\isasymLongrightarrow}\ trans{\isacharparenleft}{\kern0pt}rrank{\isacharparenleft}{\kern0pt}x{\isacharparenright}{\kern0pt}{\isacharparenright}{\kern0pt}{\isachardoublequoteclose}\isanewline
%
\isadelimproof
\ \ %
\endisadelimproof
%
\isatagproof
\isacommand{unfolding}\isamarkupfalse%
\ rrank{\isacharunderscore}{\kern0pt}def\ \isacommand{using}\isamarkupfalse%
\ trans{\isacharunderscore}{\kern0pt}trancl\ \isacommand{{\isachardot}{\kern0pt}}\isamarkupfalse%
%
\endisatagproof
{\isafoldproof}%
%
\isadelimproof
\isanewline
%
\endisadelimproof
\isanewline
\isacommand{lemma}\isamarkupfalse%
\ {\isacharparenleft}{\kern0pt}\isakeyword{in}\ M{\isacharunderscore}{\kern0pt}eclose{\isacharparenright}{\kern0pt}\ relation{\isacharunderscore}{\kern0pt}rrank\ {\isacharcolon}{\kern0pt}\ {\isachardoublequoteopen}M{\isacharparenleft}{\kern0pt}x{\isacharparenright}{\kern0pt}\ {\isasymLongrightarrow}\ relation{\isacharparenleft}{\kern0pt}rrank{\isacharparenleft}{\kern0pt}x{\isacharparenright}{\kern0pt}{\isacharparenright}{\kern0pt}{\isachardoublequoteclose}\ \isanewline
%
\isadelimproof
\ \ %
\endisadelimproof
%
\isatagproof
\isacommand{unfolding}\isamarkupfalse%
\ rrank{\isacharunderscore}{\kern0pt}def\ \isacommand{using}\isamarkupfalse%
\ relation{\isacharunderscore}{\kern0pt}trancl\ \isacommand{{\isachardot}{\kern0pt}}\isamarkupfalse%
%
\endisatagproof
{\isafoldproof}%
%
\isadelimproof
\isanewline
%
\endisadelimproof
\isanewline
\isacommand{lemma}\isamarkupfalse%
\ {\isacharparenleft}{\kern0pt}\isakeyword{in}\ M{\isacharunderscore}{\kern0pt}eclose{\isacharparenright}{\kern0pt}\ rrank{\isacharunderscore}{\kern0pt}in{\isacharunderscore}{\kern0pt}M\ {\isacharcolon}{\kern0pt}\ {\isachardoublequoteopen}M{\isacharparenleft}{\kern0pt}x{\isacharparenright}{\kern0pt}\ {\isasymLongrightarrow}\ M{\isacharparenleft}{\kern0pt}rrank{\isacharparenleft}{\kern0pt}x{\isacharparenright}{\kern0pt}{\isacharparenright}{\kern0pt}{\isachardoublequoteclose}\ \isanewline
%
\isadelimproof
\ \ %
\endisadelimproof
%
\isatagproof
\isacommand{unfolding}\isamarkupfalse%
\ rrank{\isacharunderscore}{\kern0pt}def\ \isacommand{by}\isamarkupfalse%
\ simp%
\endisatagproof
{\isafoldproof}%
%
\isadelimproof
%
\endisadelimproof
%
\isadelimdocument
%
\endisadelimdocument
%
\isatagdocument
%
\isamarkupsubsection{Absoluteness results%
}
\isamarkuptrue%
%
\endisatagdocument
{\isafolddocument}%
%
\isadelimdocument
%
\endisadelimdocument
\isacommand{locale}\isamarkupfalse%
\ M{\isacharunderscore}{\kern0pt}eclose{\isacharunderscore}{\kern0pt}pow\ {\isacharequal}{\kern0pt}\ M{\isacharunderscore}{\kern0pt}eclose\ {\isacharplus}{\kern0pt}\ \isanewline
\ \ \isakeyword{assumes}\isanewline
\ \ \ \ power{\isacharunderscore}{\kern0pt}ax\ {\isacharcolon}{\kern0pt}\ {\isachardoublequoteopen}power{\isacharunderscore}{\kern0pt}ax{\isacharparenleft}{\kern0pt}M{\isacharparenright}{\kern0pt}{\isachardoublequoteclose}\ \isakeyword{and}\isanewline
\ \ \ \ powapply{\isacharunderscore}{\kern0pt}replacement\ {\isacharcolon}{\kern0pt}\ {\isachardoublequoteopen}M{\isacharparenleft}{\kern0pt}f{\isacharparenright}{\kern0pt}\ {\isasymLongrightarrow}\ strong{\isacharunderscore}{\kern0pt}replacement{\isacharparenleft}{\kern0pt}M{\isacharcomma}{\kern0pt}is{\isacharunderscore}{\kern0pt}powapply{\isacharparenleft}{\kern0pt}M{\isacharcomma}{\kern0pt}f{\isacharparenright}{\kern0pt}{\isacharparenright}{\kern0pt}{\isachardoublequoteclose}\ \isakeyword{and}\isanewline
\ \ \ \ HVfrom{\isacharunderscore}{\kern0pt}replacement\ {\isacharcolon}{\kern0pt}\ {\isachardoublequoteopen}{\isasymlbrakk}\ M{\isacharparenleft}{\kern0pt}i{\isacharparenright}{\kern0pt}\ {\isacharsemicolon}{\kern0pt}\ M{\isacharparenleft}{\kern0pt}A{\isacharparenright}{\kern0pt}\ {\isasymrbrakk}\ {\isasymLongrightarrow}\ \isanewline
\ \ \ \ \ \ \ \ \ \ \ \ \ \ \ \ \ \ \ \ \ \ \ \ \ \ transrec{\isacharunderscore}{\kern0pt}replacement{\isacharparenleft}{\kern0pt}M{\isacharcomma}{\kern0pt}is{\isacharunderscore}{\kern0pt}HVfrom{\isacharparenleft}{\kern0pt}M{\isacharcomma}{\kern0pt}A{\isacharparenright}{\kern0pt}{\isacharcomma}{\kern0pt}i{\isacharparenright}{\kern0pt}{\isachardoublequoteclose}\ \isakeyword{and}\isanewline
\ \ \ \ PHrank{\isacharunderscore}{\kern0pt}replacement\ {\isacharcolon}{\kern0pt}\ {\isachardoublequoteopen}M{\isacharparenleft}{\kern0pt}f{\isacharparenright}{\kern0pt}\ {\isasymLongrightarrow}\ strong{\isacharunderscore}{\kern0pt}replacement{\isacharparenleft}{\kern0pt}M{\isacharcomma}{\kern0pt}PHrank{\isacharparenleft}{\kern0pt}M{\isacharcomma}{\kern0pt}f{\isacharparenright}{\kern0pt}{\isacharparenright}{\kern0pt}{\isachardoublequoteclose}\ \isakeyword{and}\isanewline
\ \ \ \ is{\isacharunderscore}{\kern0pt}Hrank{\isacharunderscore}{\kern0pt}replacement\ {\isacharcolon}{\kern0pt}\ {\isachardoublequoteopen}M{\isacharparenleft}{\kern0pt}x{\isacharparenright}{\kern0pt}\ {\isasymLongrightarrow}\ wfrec{\isacharunderscore}{\kern0pt}replacement{\isacharparenleft}{\kern0pt}M{\isacharcomma}{\kern0pt}is{\isacharunderscore}{\kern0pt}Hrank{\isacharparenleft}{\kern0pt}M{\isacharparenright}{\kern0pt}{\isacharcomma}{\kern0pt}rrank{\isacharparenleft}{\kern0pt}x{\isacharparenright}{\kern0pt}{\isacharparenright}{\kern0pt}{\isachardoublequoteclose}\isanewline
\isanewline
\isakeyword{begin}\isanewline
\isanewline
\isacommand{lemma}\isamarkupfalse%
\ is{\isacharunderscore}{\kern0pt}powapply{\isacharunderscore}{\kern0pt}abs{\isacharcolon}{\kern0pt}\ {\isachardoublequoteopen}{\isasymlbrakk}M{\isacharparenleft}{\kern0pt}f{\isacharparenright}{\kern0pt}{\isacharsemicolon}{\kern0pt}\ M{\isacharparenleft}{\kern0pt}y{\isacharparenright}{\kern0pt}{\isasymrbrakk}\ {\isasymLongrightarrow}\ is{\isacharunderscore}{\kern0pt}powapply{\isacharparenleft}{\kern0pt}M{\isacharcomma}{\kern0pt}f{\isacharcomma}{\kern0pt}y{\isacharcomma}{\kern0pt}z{\isacharparenright}{\kern0pt}\ {\isasymlongleftrightarrow}\ M{\isacharparenleft}{\kern0pt}z{\isacharparenright}{\kern0pt}\ {\isasymand}\ z\ {\isacharequal}{\kern0pt}\ {\isacharbraceleft}{\kern0pt}x{\isasymin}Pow{\isacharparenleft}{\kern0pt}f{\isacharbackquote}{\kern0pt}y{\isacharparenright}{\kern0pt}{\isachardot}{\kern0pt}\ M{\isacharparenleft}{\kern0pt}x{\isacharparenright}{\kern0pt}{\isacharbraceright}{\kern0pt}{\isachardoublequoteclose}\isanewline
%
\isadelimproof
\ \ %
\endisadelimproof
%
\isatagproof
\isacommand{unfolding}\isamarkupfalse%
\ is{\isacharunderscore}{\kern0pt}powapply{\isacharunderscore}{\kern0pt}def\ \isacommand{by}\isamarkupfalse%
\ simp%
\endisatagproof
{\isafoldproof}%
%
\isadelimproof
\isanewline
%
\endisadelimproof
\isanewline
\isacommand{lemma}\isamarkupfalse%
\ {\isachardoublequoteopen}{\isasymlbrakk}M{\isacharparenleft}{\kern0pt}A{\isacharparenright}{\kern0pt}{\isacharsemicolon}{\kern0pt}\ M{\isacharparenleft}{\kern0pt}x{\isacharparenright}{\kern0pt}{\isacharsemicolon}{\kern0pt}\ M{\isacharparenleft}{\kern0pt}f{\isacharparenright}{\kern0pt}{\isacharsemicolon}{\kern0pt}\ M{\isacharparenleft}{\kern0pt}h{\isacharparenright}{\kern0pt}\ {\isasymrbrakk}\ {\isasymLongrightarrow}\ \isanewline
\ \ \ \ \ \ is{\isacharunderscore}{\kern0pt}HVfrom{\isacharparenleft}{\kern0pt}M{\isacharcomma}{\kern0pt}A{\isacharcomma}{\kern0pt}x{\isacharcomma}{\kern0pt}f{\isacharcomma}{\kern0pt}h{\isacharparenright}{\kern0pt}\ {\isasymlongleftrightarrow}\ \isanewline
\ \ \ \ \ \ {\isacharparenleft}{\kern0pt}{\isasymexists}R{\isacharbrackleft}{\kern0pt}M{\isacharbrackright}{\kern0pt}{\isachardot}{\kern0pt}\ h\ {\isacharequal}{\kern0pt}\ A\ {\isasymunion}\ {\isasymUnion}R\ {\isasymand}\ is{\isacharunderscore}{\kern0pt}Replace{\isacharparenleft}{\kern0pt}M{\isacharcomma}{\kern0pt}\ x{\isacharcomma}{\kern0pt}{\isasymlambda}x\ y{\isachardot}{\kern0pt}\ y\ {\isacharequal}{\kern0pt}\ {\isacharbraceleft}{\kern0pt}x\ {\isasymin}\ Pow{\isacharparenleft}{\kern0pt}f\ {\isacharbackquote}{\kern0pt}\ x{\isacharparenright}{\kern0pt}\ {\isachardot}{\kern0pt}\ M{\isacharparenleft}{\kern0pt}x{\isacharparenright}{\kern0pt}{\isacharbraceright}{\kern0pt}{\isacharcomma}{\kern0pt}\ R{\isacharparenright}{\kern0pt}{\isacharparenright}{\kern0pt}{\isachardoublequoteclose}\isanewline
%
\isadelimproof
\ \ %
\endisadelimproof
%
\isatagproof
\isacommand{using}\isamarkupfalse%
\ is{\isacharunderscore}{\kern0pt}powapply{\isacharunderscore}{\kern0pt}abs\ \isacommand{unfolding}\isamarkupfalse%
\ is{\isacharunderscore}{\kern0pt}HVfrom{\isacharunderscore}{\kern0pt}def\ \isacommand{by}\isamarkupfalse%
\ auto%
\endisatagproof
{\isafoldproof}%
%
\isadelimproof
\isanewline
%
\endisadelimproof
\isanewline
\isacommand{lemma}\isamarkupfalse%
\ Replace{\isacharunderscore}{\kern0pt}is{\isacharunderscore}{\kern0pt}powapply{\isacharcolon}{\kern0pt}\isanewline
\ \ \isakeyword{assumes}\isanewline
\ \ \ \ {\isachardoublequoteopen}M{\isacharparenleft}{\kern0pt}R{\isacharparenright}{\kern0pt}{\isachardoublequoteclose}\ {\isachardoublequoteopen}M{\isacharparenleft}{\kern0pt}A{\isacharparenright}{\kern0pt}{\isachardoublequoteclose}\ {\isachardoublequoteopen}M{\isacharparenleft}{\kern0pt}f{\isacharparenright}{\kern0pt}{\isachardoublequoteclose}\ \isanewline
\ \ \isakeyword{shows}\isanewline
\ \ {\isachardoublequoteopen}is{\isacharunderscore}{\kern0pt}Replace{\isacharparenleft}{\kern0pt}M{\isacharcomma}{\kern0pt}\ A{\isacharcomma}{\kern0pt}\ is{\isacharunderscore}{\kern0pt}powapply{\isacharparenleft}{\kern0pt}M{\isacharcomma}{\kern0pt}\ f{\isacharparenright}{\kern0pt}{\isacharcomma}{\kern0pt}\ R{\isacharparenright}{\kern0pt}\ {\isasymlongleftrightarrow}\ R\ {\isacharequal}{\kern0pt}\ Replace{\isacharparenleft}{\kern0pt}A{\isacharcomma}{\kern0pt}is{\isacharunderscore}{\kern0pt}powapply{\isacharparenleft}{\kern0pt}M{\isacharcomma}{\kern0pt}f{\isacharparenright}{\kern0pt}{\isacharparenright}{\kern0pt}{\isachardoublequoteclose}\isanewline
%
\isadelimproof
%
\endisadelimproof
%
\isatagproof
\isacommand{proof}\isamarkupfalse%
\ {\isacharminus}{\kern0pt}\isanewline
\ \ \isacommand{have}\isamarkupfalse%
\ {\isachardoublequoteopen}univalent{\isacharparenleft}{\kern0pt}M{\isacharcomma}{\kern0pt}A{\isacharcomma}{\kern0pt}is{\isacharunderscore}{\kern0pt}powapply{\isacharparenleft}{\kern0pt}M{\isacharcomma}{\kern0pt}f{\isacharparenright}{\kern0pt}{\isacharparenright}{\kern0pt}{\isachardoublequoteclose}\ \isanewline
\ \ \ \ \isacommand{using}\isamarkupfalse%
\ {\isacartoucheopen}M{\isacharparenleft}{\kern0pt}A{\isacharparenright}{\kern0pt}{\isacartoucheclose}\ {\isacartoucheopen}M{\isacharparenleft}{\kern0pt}f{\isacharparenright}{\kern0pt}{\isacartoucheclose}\ \isacommand{unfolding}\isamarkupfalse%
\ univalent{\isacharunderscore}{\kern0pt}def\ is{\isacharunderscore}{\kern0pt}powapply{\isacharunderscore}{\kern0pt}def\ \isacommand{by}\isamarkupfalse%
\ simp\isanewline
\ \ \isacommand{moreover}\isamarkupfalse%
\isanewline
\ \ \isacommand{have}\isamarkupfalse%
\ {\isachardoublequoteopen}{\isasymAnd}x\ y{\isachardot}{\kern0pt}\ {\isasymlbrakk}\ x{\isasymin}A{\isacharsemicolon}{\kern0pt}\ is{\isacharunderscore}{\kern0pt}powapply{\isacharparenleft}{\kern0pt}M{\isacharcomma}{\kern0pt}f{\isacharcomma}{\kern0pt}x{\isacharcomma}{\kern0pt}y{\isacharparenright}{\kern0pt}\ {\isasymrbrakk}\ {\isasymLongrightarrow}\ M{\isacharparenleft}{\kern0pt}y{\isacharparenright}{\kern0pt}{\isachardoublequoteclose}\isanewline
\ \ \ \ \isacommand{using}\isamarkupfalse%
\ {\isacartoucheopen}M{\isacharparenleft}{\kern0pt}A{\isacharparenright}{\kern0pt}{\isacartoucheclose}\ {\isacartoucheopen}M{\isacharparenleft}{\kern0pt}f{\isacharparenright}{\kern0pt}{\isacartoucheclose}\ \isacommand{unfolding}\isamarkupfalse%
\ is{\isacharunderscore}{\kern0pt}powapply{\isacharunderscore}{\kern0pt}def\ \isacommand{by}\isamarkupfalse%
\ simp\isanewline
\ \ \isacommand{ultimately}\isamarkupfalse%
\isanewline
\ \ \isacommand{show}\isamarkupfalse%
\ {\isacharquery}{\kern0pt}thesis\ \isacommand{using}\isamarkupfalse%
\ {\isacartoucheopen}M{\isacharparenleft}{\kern0pt}A{\isacharparenright}{\kern0pt}{\isacartoucheclose}\ {\isacartoucheopen}M{\isacharparenleft}{\kern0pt}R{\isacharparenright}{\kern0pt}{\isacartoucheclose}\ Replace{\isacharunderscore}{\kern0pt}abs\ \isacommand{by}\isamarkupfalse%
\ simp\isanewline
\isacommand{qed}\isamarkupfalse%
%
\endisatagproof
{\isafoldproof}%
%
\isadelimproof
\isanewline
%
\endisadelimproof
\isanewline
\isacommand{lemma}\isamarkupfalse%
\ powapply{\isacharunderscore}{\kern0pt}closed{\isacharcolon}{\kern0pt}\isanewline
\ \ {\isachardoublequoteopen}{\isasymlbrakk}\ M{\isacharparenleft}{\kern0pt}y{\isacharparenright}{\kern0pt}\ {\isacharsemicolon}{\kern0pt}\ M{\isacharparenleft}{\kern0pt}f{\isacharparenright}{\kern0pt}\ {\isasymrbrakk}\ {\isasymLongrightarrow}\ M{\isacharparenleft}{\kern0pt}{\isacharbraceleft}{\kern0pt}x\ {\isasymin}\ Pow{\isacharparenleft}{\kern0pt}f\ {\isacharbackquote}{\kern0pt}\ y{\isacharparenright}{\kern0pt}\ {\isachardot}{\kern0pt}\ M{\isacharparenleft}{\kern0pt}x{\isacharparenright}{\kern0pt}{\isacharbraceright}{\kern0pt}{\isacharparenright}{\kern0pt}{\isachardoublequoteclose}\isanewline
%
\isadelimproof
\ \ %
\endisadelimproof
%
\isatagproof
\isacommand{using}\isamarkupfalse%
\ apply{\isacharunderscore}{\kern0pt}closed\ power{\isacharunderscore}{\kern0pt}ax\ \isacommand{unfolding}\isamarkupfalse%
\ power{\isacharunderscore}{\kern0pt}ax{\isacharunderscore}{\kern0pt}def\ \isacommand{by}\isamarkupfalse%
\ simp%
\endisatagproof
{\isafoldproof}%
%
\isadelimproof
\isanewline
%
\endisadelimproof
\isanewline
\isacommand{lemma}\isamarkupfalse%
\ RepFun{\isacharunderscore}{\kern0pt}is{\isacharunderscore}{\kern0pt}powapply{\isacharcolon}{\kern0pt}\isanewline
\ \ \isakeyword{assumes}\isanewline
\ \ \ \ {\isachardoublequoteopen}M{\isacharparenleft}{\kern0pt}R{\isacharparenright}{\kern0pt}{\isachardoublequoteclose}\ {\isachardoublequoteopen}M{\isacharparenleft}{\kern0pt}A{\isacharparenright}{\kern0pt}{\isachardoublequoteclose}\ {\isachardoublequoteopen}M{\isacharparenleft}{\kern0pt}f{\isacharparenright}{\kern0pt}{\isachardoublequoteclose}\ \isanewline
\ \ \isakeyword{shows}\isanewline
\ \ {\isachardoublequoteopen}Replace{\isacharparenleft}{\kern0pt}A{\isacharcomma}{\kern0pt}is{\isacharunderscore}{\kern0pt}powapply{\isacharparenleft}{\kern0pt}M{\isacharcomma}{\kern0pt}f{\isacharparenright}{\kern0pt}{\isacharparenright}{\kern0pt}\ {\isacharequal}{\kern0pt}\ RepFun{\isacharparenleft}{\kern0pt}A{\isacharcomma}{\kern0pt}{\isasymlambda}y{\isachardot}{\kern0pt}{\isacharbraceleft}{\kern0pt}x{\isasymin}Pow{\isacharparenleft}{\kern0pt}f{\isacharbackquote}{\kern0pt}y{\isacharparenright}{\kern0pt}{\isachardot}{\kern0pt}\ M{\isacharparenleft}{\kern0pt}x{\isacharparenright}{\kern0pt}{\isacharbraceright}{\kern0pt}{\isacharparenright}{\kern0pt}{\isachardoublequoteclose}\isanewline
%
\isadelimproof
%
\endisadelimproof
%
\isatagproof
\isacommand{proof}\isamarkupfalse%
\ {\isacharminus}{\kern0pt}\isanewline
\ \ \isacommand{have}\isamarkupfalse%
\ {\isachardoublequoteopen}{\isacharbraceleft}{\kern0pt}y\ {\isachardot}{\kern0pt}\ x\ {\isasymin}\ A{\isacharcomma}{\kern0pt}\ M{\isacharparenleft}{\kern0pt}y{\isacharparenright}{\kern0pt}\ {\isasymand}\ y\ {\isacharequal}{\kern0pt}\ {\isacharbraceleft}{\kern0pt}x\ {\isasymin}\ Pow{\isacharparenleft}{\kern0pt}f\ {\isacharbackquote}{\kern0pt}\ x{\isacharparenright}{\kern0pt}\ {\isachardot}{\kern0pt}\ M{\isacharparenleft}{\kern0pt}x{\isacharparenright}{\kern0pt}{\isacharbraceright}{\kern0pt}{\isacharbraceright}{\kern0pt}\ {\isacharequal}{\kern0pt}\ {\isacharbraceleft}{\kern0pt}y\ {\isachardot}{\kern0pt}\ x\ {\isasymin}\ A{\isacharcomma}{\kern0pt}\ y\ {\isacharequal}{\kern0pt}\ {\isacharbraceleft}{\kern0pt}x\ {\isasymin}\ Pow{\isacharparenleft}{\kern0pt}f\ {\isacharbackquote}{\kern0pt}\ x{\isacharparenright}{\kern0pt}\ {\isachardot}{\kern0pt}\ M{\isacharparenleft}{\kern0pt}x{\isacharparenright}{\kern0pt}{\isacharbraceright}{\kern0pt}{\isacharbraceright}{\kern0pt}{\isachardoublequoteclose}\isanewline
\ \ \ \ \isacommand{using}\isamarkupfalse%
\ assms\ powapply{\isacharunderscore}{\kern0pt}closed\ transM{\isacharbrackleft}{\kern0pt}of\ {\isacharunderscore}{\kern0pt}\ A{\isacharbrackright}{\kern0pt}\ \isacommand{by}\isamarkupfalse%
\ blast\isanewline
\ \ \isacommand{also}\isamarkupfalse%
\isanewline
\ \ \isacommand{have}\isamarkupfalse%
\ {\isachardoublequoteopen}\ {\isachardot}{\kern0pt}{\isachardot}{\kern0pt}{\isachardot}{\kern0pt}\ {\isacharequal}{\kern0pt}\ {\isacharbraceleft}{\kern0pt}{\isacharbraceleft}{\kern0pt}x\ {\isasymin}\ Pow{\isacharparenleft}{\kern0pt}f\ {\isacharbackquote}{\kern0pt}\ y{\isacharparenright}{\kern0pt}\ {\isachardot}{\kern0pt}\ M{\isacharparenleft}{\kern0pt}x{\isacharparenright}{\kern0pt}{\isacharbraceright}{\kern0pt}\ {\isachardot}{\kern0pt}\ y\ {\isasymin}\ A{\isacharbraceright}{\kern0pt}{\isachardoublequoteclose}\ \isacommand{by}\isamarkupfalse%
\ auto\isanewline
\ \ \isacommand{finally}\isamarkupfalse%
\ \isanewline
\ \ \isacommand{show}\isamarkupfalse%
\ {\isacharquery}{\kern0pt}thesis\ \isacommand{using}\isamarkupfalse%
\ assms\ is{\isacharunderscore}{\kern0pt}powapply{\isacharunderscore}{\kern0pt}abs\ transM{\isacharbrackleft}{\kern0pt}of\ {\isacharunderscore}{\kern0pt}\ A{\isacharbrackright}{\kern0pt}\ \isacommand{by}\isamarkupfalse%
\ simp\isanewline
\isacommand{qed}\isamarkupfalse%
%
\endisatagproof
{\isafoldproof}%
%
\isadelimproof
\isanewline
%
\endisadelimproof
\isanewline
\isacommand{lemma}\isamarkupfalse%
\ RepFun{\isacharunderscore}{\kern0pt}powapply{\isacharunderscore}{\kern0pt}closed{\isacharcolon}{\kern0pt}\isanewline
\ \ \isakeyword{assumes}\ \isanewline
\ \ \ \ {\isachardoublequoteopen}M{\isacharparenleft}{\kern0pt}f{\isacharparenright}{\kern0pt}{\isachardoublequoteclose}\ {\isachardoublequoteopen}M{\isacharparenleft}{\kern0pt}A{\isacharparenright}{\kern0pt}{\isachardoublequoteclose}\isanewline
\ \ \isakeyword{shows}\ \isanewline
\ \ \ \ {\isachardoublequoteopen}M{\isacharparenleft}{\kern0pt}Replace{\isacharparenleft}{\kern0pt}A{\isacharcomma}{\kern0pt}is{\isacharunderscore}{\kern0pt}powapply{\isacharparenleft}{\kern0pt}M{\isacharcomma}{\kern0pt}f{\isacharparenright}{\kern0pt}{\isacharparenright}{\kern0pt}{\isacharparenright}{\kern0pt}{\isachardoublequoteclose}\isanewline
%
\isadelimproof
%
\endisadelimproof
%
\isatagproof
\isacommand{proof}\isamarkupfalse%
\ {\isacharminus}{\kern0pt}\isanewline
\ \ \isacommand{have}\isamarkupfalse%
\ {\isachardoublequoteopen}univalent{\isacharparenleft}{\kern0pt}M{\isacharcomma}{\kern0pt}A{\isacharcomma}{\kern0pt}is{\isacharunderscore}{\kern0pt}powapply{\isacharparenleft}{\kern0pt}M{\isacharcomma}{\kern0pt}f{\isacharparenright}{\kern0pt}{\isacharparenright}{\kern0pt}{\isachardoublequoteclose}\ \isanewline
\ \ \ \ \isacommand{using}\isamarkupfalse%
\ {\isacartoucheopen}M{\isacharparenleft}{\kern0pt}A{\isacharparenright}{\kern0pt}{\isacartoucheclose}\ {\isacartoucheopen}M{\isacharparenleft}{\kern0pt}f{\isacharparenright}{\kern0pt}{\isacartoucheclose}\ \isacommand{unfolding}\isamarkupfalse%
\ univalent{\isacharunderscore}{\kern0pt}def\ is{\isacharunderscore}{\kern0pt}powapply{\isacharunderscore}{\kern0pt}def\ \isacommand{by}\isamarkupfalse%
\ simp\isanewline
\ \ \isacommand{moreover}\isamarkupfalse%
\isanewline
\ \ \isacommand{have}\isamarkupfalse%
\ {\isachardoublequoteopen}{\isasymlbrakk}\ x{\isasymin}A\ {\isacharsemicolon}{\kern0pt}\ is{\isacharunderscore}{\kern0pt}powapply{\isacharparenleft}{\kern0pt}M{\isacharcomma}{\kern0pt}f{\isacharcomma}{\kern0pt}x{\isacharcomma}{\kern0pt}y{\isacharparenright}{\kern0pt}\ {\isasymrbrakk}\ {\isasymLongrightarrow}\ M{\isacharparenleft}{\kern0pt}y{\isacharparenright}{\kern0pt}{\isachardoublequoteclose}\ \isakeyword{for}\ x\ y\isanewline
\ \ \ \ \isacommand{using}\isamarkupfalse%
\ assms\ \isacommand{unfolding}\isamarkupfalse%
\ is{\isacharunderscore}{\kern0pt}powapply{\isacharunderscore}{\kern0pt}def\ \isacommand{by}\isamarkupfalse%
\ simp\isanewline
\ \ \isacommand{ultimately}\isamarkupfalse%
\isanewline
\ \ \isacommand{show}\isamarkupfalse%
\ {\isacharquery}{\kern0pt}thesis\ \isacommand{using}\isamarkupfalse%
\ assms\ powapply{\isacharunderscore}{\kern0pt}replacement\ \isacommand{by}\isamarkupfalse%
\ simp\isanewline
\isacommand{qed}\isamarkupfalse%
%
\endisatagproof
{\isafoldproof}%
%
\isadelimproof
\isanewline
%
\endisadelimproof
\isanewline
\isacommand{lemma}\isamarkupfalse%
\ Union{\isacharunderscore}{\kern0pt}powapply{\isacharunderscore}{\kern0pt}closed{\isacharcolon}{\kern0pt}\isanewline
\ \ \isakeyword{assumes}\ \isanewline
\ \ \ \ {\isachardoublequoteopen}M{\isacharparenleft}{\kern0pt}x{\isacharparenright}{\kern0pt}{\isachardoublequoteclose}\ {\isachardoublequoteopen}M{\isacharparenleft}{\kern0pt}f{\isacharparenright}{\kern0pt}{\isachardoublequoteclose}\isanewline
\ \ \isakeyword{shows}\ \isanewline
\ \ \ \ {\isachardoublequoteopen}M{\isacharparenleft}{\kern0pt}{\isasymUnion}y{\isasymin}x{\isachardot}{\kern0pt}\ {\isacharbraceleft}{\kern0pt}a{\isasymin}Pow{\isacharparenleft}{\kern0pt}f{\isacharbackquote}{\kern0pt}y{\isacharparenright}{\kern0pt}{\isachardot}{\kern0pt}\ M{\isacharparenleft}{\kern0pt}a{\isacharparenright}{\kern0pt}{\isacharbraceright}{\kern0pt}{\isacharparenright}{\kern0pt}{\isachardoublequoteclose}\isanewline
%
\isadelimproof
%
\endisadelimproof
%
\isatagproof
\isacommand{proof}\isamarkupfalse%
\ {\isacharminus}{\kern0pt}\isanewline
\ \ \isacommand{have}\isamarkupfalse%
\ {\isachardoublequoteopen}M{\isacharparenleft}{\kern0pt}{\isacharbraceleft}{\kern0pt}a{\isasymin}Pow{\isacharparenleft}{\kern0pt}f{\isacharbackquote}{\kern0pt}y{\isacharparenright}{\kern0pt}{\isachardot}{\kern0pt}\ M{\isacharparenleft}{\kern0pt}a{\isacharparenright}{\kern0pt}{\isacharbraceright}{\kern0pt}{\isacharparenright}{\kern0pt}{\isachardoublequoteclose}\ \isakeyword{if}\ {\isachardoublequoteopen}y{\isasymin}x{\isachardoublequoteclose}\ \isakeyword{for}\ y\isanewline
\ \ \ \ \isacommand{using}\isamarkupfalse%
\ that\ assms\ transM{\isacharbrackleft}{\kern0pt}of\ {\isacharunderscore}{\kern0pt}\ x{\isacharbrackright}{\kern0pt}\ powapply{\isacharunderscore}{\kern0pt}closed\ \isacommand{by}\isamarkupfalse%
\ simp\isanewline
\ \ \isacommand{then}\isamarkupfalse%
\isanewline
\ \ \isacommand{have}\isamarkupfalse%
\ {\isachardoublequoteopen}M{\isacharparenleft}{\kern0pt}{\isacharbraceleft}{\kern0pt}{\isacharbraceleft}{\kern0pt}a{\isasymin}Pow{\isacharparenleft}{\kern0pt}f{\isacharbackquote}{\kern0pt}y{\isacharparenright}{\kern0pt}{\isachardot}{\kern0pt}\ M{\isacharparenleft}{\kern0pt}a{\isacharparenright}{\kern0pt}{\isacharbraceright}{\kern0pt}{\isachardot}{\kern0pt}\ y{\isasymin}x{\isacharbraceright}{\kern0pt}{\isacharparenright}{\kern0pt}{\isachardoublequoteclose}\isanewline
\ \ \ \ \isacommand{using}\isamarkupfalse%
\ assms\ transM{\isacharbrackleft}{\kern0pt}of\ {\isacharunderscore}{\kern0pt}\ x{\isacharbrackright}{\kern0pt}\ \ RepFun{\isacharunderscore}{\kern0pt}powapply{\isacharunderscore}{\kern0pt}closed\ RepFun{\isacharunderscore}{\kern0pt}is{\isacharunderscore}{\kern0pt}powapply\ \isacommand{by}\isamarkupfalse%
\ simp\isanewline
\ \ \isacommand{then}\isamarkupfalse%
\ \isacommand{show}\isamarkupfalse%
\ {\isacharquery}{\kern0pt}thesis\ \isacommand{using}\isamarkupfalse%
\ assms\ \isacommand{by}\isamarkupfalse%
\ simp\isanewline
\isacommand{qed}\isamarkupfalse%
%
\endisatagproof
{\isafoldproof}%
%
\isadelimproof
\isanewline
%
\endisadelimproof
\isanewline
\isacommand{lemma}\isamarkupfalse%
\ relation{\isadigit{2}}{\isacharunderscore}{\kern0pt}HVfrom{\isacharcolon}{\kern0pt}\ {\isachardoublequoteopen}M{\isacharparenleft}{\kern0pt}A{\isacharparenright}{\kern0pt}\ {\isasymLongrightarrow}\ relation{\isadigit{2}}{\isacharparenleft}{\kern0pt}M{\isacharcomma}{\kern0pt}is{\isacharunderscore}{\kern0pt}HVfrom{\isacharparenleft}{\kern0pt}M{\isacharcomma}{\kern0pt}A{\isacharparenright}{\kern0pt}{\isacharcomma}{\kern0pt}HVfrom{\isacharparenleft}{\kern0pt}M{\isacharcomma}{\kern0pt}A{\isacharparenright}{\kern0pt}{\isacharparenright}{\kern0pt}{\isachardoublequoteclose}\isanewline
%
\isadelimproof
\ \ \ \ %
\endisadelimproof
%
\isatagproof
\isacommand{unfolding}\isamarkupfalse%
\ is{\isacharunderscore}{\kern0pt}HVfrom{\isacharunderscore}{\kern0pt}def\ HVfrom{\isacharunderscore}{\kern0pt}def\ relation{\isadigit{2}}{\isacharunderscore}{\kern0pt}def\isanewline
\ \ \ \ \isacommand{using}\isamarkupfalse%
\ Replace{\isacharunderscore}{\kern0pt}is{\isacharunderscore}{\kern0pt}powapply\ RepFun{\isacharunderscore}{\kern0pt}is{\isacharunderscore}{\kern0pt}powapply\ \isanewline
\ \ \ \ \ \ \ \ \ \ Union{\isacharunderscore}{\kern0pt}powapply{\isacharunderscore}{\kern0pt}closed\ RepFun{\isacharunderscore}{\kern0pt}powapply{\isacharunderscore}{\kern0pt}closed\ \isacommand{by}\isamarkupfalse%
\ auto%
\endisatagproof
{\isafoldproof}%
%
\isadelimproof
\isanewline
%
\endisadelimproof
\isanewline
\isacommand{lemma}\isamarkupfalse%
\ HVfrom{\isacharunderscore}{\kern0pt}closed\ {\isacharcolon}{\kern0pt}\ \isanewline
\ \ {\isachardoublequoteopen}M{\isacharparenleft}{\kern0pt}A{\isacharparenright}{\kern0pt}\ {\isasymLongrightarrow}\ {\isasymforall}x{\isacharbrackleft}{\kern0pt}M{\isacharbrackright}{\kern0pt}{\isachardot}{\kern0pt}\ {\isasymforall}g{\isacharbrackleft}{\kern0pt}M{\isacharbrackright}{\kern0pt}{\isachardot}{\kern0pt}\ function{\isacharparenleft}{\kern0pt}g{\isacharparenright}{\kern0pt}\ {\isasymlongrightarrow}\ M{\isacharparenleft}{\kern0pt}HVfrom{\isacharparenleft}{\kern0pt}M{\isacharcomma}{\kern0pt}A{\isacharcomma}{\kern0pt}x{\isacharcomma}{\kern0pt}g{\isacharparenright}{\kern0pt}{\isacharparenright}{\kern0pt}{\isachardoublequoteclose}\isanewline
%
\isadelimproof
\ \ %
\endisadelimproof
%
\isatagproof
\isacommand{unfolding}\isamarkupfalse%
\ HVfrom{\isacharunderscore}{\kern0pt}def\ \isacommand{using}\isamarkupfalse%
\ Union{\isacharunderscore}{\kern0pt}powapply{\isacharunderscore}{\kern0pt}closed\ \isacommand{by}\isamarkupfalse%
\ simp%
\endisatagproof
{\isafoldproof}%
%
\isadelimproof
\isanewline
%
\endisadelimproof
\isanewline
\isacommand{lemma}\isamarkupfalse%
\ transrec{\isacharunderscore}{\kern0pt}HVfrom{\isacharcolon}{\kern0pt}\isanewline
\ \ \isakeyword{assumes}\ {\isachardoublequoteopen}M{\isacharparenleft}{\kern0pt}A{\isacharparenright}{\kern0pt}{\isachardoublequoteclose}\isanewline
\ \ \isakeyword{shows}\ {\isachardoublequoteopen}Ord{\isacharparenleft}{\kern0pt}i{\isacharparenright}{\kern0pt}\ {\isasymLongrightarrow}\ {\isacharbraceleft}{\kern0pt}x{\isasymin}Vfrom{\isacharparenleft}{\kern0pt}A{\isacharcomma}{\kern0pt}i{\isacharparenright}{\kern0pt}{\isachardot}{\kern0pt}\ M{\isacharparenleft}{\kern0pt}x{\isacharparenright}{\kern0pt}{\isacharbraceright}{\kern0pt}\ {\isacharequal}{\kern0pt}\ transrec{\isacharparenleft}{\kern0pt}i{\isacharcomma}{\kern0pt}HVfrom{\isacharparenleft}{\kern0pt}M{\isacharcomma}{\kern0pt}A{\isacharparenright}{\kern0pt}{\isacharparenright}{\kern0pt}{\isachardoublequoteclose}\isanewline
%
\isadelimproof
%
\endisadelimproof
%
\isatagproof
\isacommand{proof}\isamarkupfalse%
\ {\isacharparenleft}{\kern0pt}induct\ rule{\isacharcolon}{\kern0pt}trans{\isacharunderscore}{\kern0pt}induct{\isacharparenright}{\kern0pt}\isanewline
\ \ \isacommand{case}\isamarkupfalse%
\ {\isacharparenleft}{\kern0pt}step\ i{\isacharparenright}{\kern0pt}\isanewline
\ \ \isacommand{have}\isamarkupfalse%
\ {\isachardoublequoteopen}Vfrom{\isacharparenleft}{\kern0pt}A{\isacharcomma}{\kern0pt}i{\isacharparenright}{\kern0pt}\ {\isacharequal}{\kern0pt}\ A\ {\isasymunion}\ {\isacharparenleft}{\kern0pt}{\isasymUnion}y{\isasymin}i{\isachardot}{\kern0pt}\ Pow{\isacharparenleft}{\kern0pt}{\isacharparenleft}{\kern0pt}{\isasymlambda}x{\isasymin}i{\isachardot}{\kern0pt}\ Vfrom{\isacharparenleft}{\kern0pt}A{\isacharcomma}{\kern0pt}\ x{\isacharparenright}{\kern0pt}{\isacharparenright}{\kern0pt}\ {\isacharbackquote}{\kern0pt}\ y{\isacharparenright}{\kern0pt}{\isacharparenright}{\kern0pt}{\isachardoublequoteclose}\isanewline
\ \ \ \ \isacommand{using}\isamarkupfalse%
\ def{\isacharunderscore}{\kern0pt}transrec{\isacharbrackleft}{\kern0pt}OF\ Vfrom{\isacharunderscore}{\kern0pt}def{\isacharcomma}{\kern0pt}\ of\ A\ i{\isacharbrackright}{\kern0pt}\ \isacommand{by}\isamarkupfalse%
\ simp\isanewline
\ \ \isacommand{then}\isamarkupfalse%
\ \isanewline
\ \ \isacommand{have}\isamarkupfalse%
\ {\isachardoublequoteopen}Vfrom{\isacharparenleft}{\kern0pt}A{\isacharcomma}{\kern0pt}i{\isacharparenright}{\kern0pt}\ {\isacharequal}{\kern0pt}\ A\ {\isasymunion}\ {\isacharparenleft}{\kern0pt}{\isasymUnion}y{\isasymin}i{\isachardot}{\kern0pt}\ Pow{\isacharparenleft}{\kern0pt}Vfrom{\isacharparenleft}{\kern0pt}A{\isacharcomma}{\kern0pt}\ y{\isacharparenright}{\kern0pt}{\isacharparenright}{\kern0pt}{\isacharparenright}{\kern0pt}{\isachardoublequoteclose}\isanewline
\ \ \ \ \isacommand{by}\isamarkupfalse%
\ simp\isanewline
\ \ \isacommand{then}\isamarkupfalse%
\isanewline
\ \ \isacommand{have}\isamarkupfalse%
\ {\isachardoublequoteopen}{\isacharbraceleft}{\kern0pt}x{\isasymin}Vfrom{\isacharparenleft}{\kern0pt}A{\isacharcomma}{\kern0pt}i{\isacharparenright}{\kern0pt}{\isachardot}{\kern0pt}\ M{\isacharparenleft}{\kern0pt}x{\isacharparenright}{\kern0pt}{\isacharbraceright}{\kern0pt}\ {\isacharequal}{\kern0pt}\ {\isacharbraceleft}{\kern0pt}x{\isasymin}A{\isachardot}{\kern0pt}\ M{\isacharparenleft}{\kern0pt}x{\isacharparenright}{\kern0pt}{\isacharbraceright}{\kern0pt}\ {\isasymunion}\ {\isacharparenleft}{\kern0pt}{\isasymUnion}y{\isasymin}i{\isachardot}{\kern0pt}\ {\isacharbraceleft}{\kern0pt}x{\isasymin}Pow{\isacharparenleft}{\kern0pt}Vfrom{\isacharparenleft}{\kern0pt}A{\isacharcomma}{\kern0pt}\ y{\isacharparenright}{\kern0pt}{\isacharparenright}{\kern0pt}{\isachardot}{\kern0pt}\ M{\isacharparenleft}{\kern0pt}x{\isacharparenright}{\kern0pt}{\isacharbraceright}{\kern0pt}{\isacharparenright}{\kern0pt}{\isachardoublequoteclose}\isanewline
\ \ \ \ \isacommand{by}\isamarkupfalse%
\ auto\isanewline
\ \ \isacommand{with}\isamarkupfalse%
\ {\isacartoucheopen}M{\isacharparenleft}{\kern0pt}A{\isacharparenright}{\kern0pt}{\isacartoucheclose}\isanewline
\ \ \isacommand{have}\isamarkupfalse%
\ {\isachardoublequoteopen}{\isacharbraceleft}{\kern0pt}x{\isasymin}Vfrom{\isacharparenleft}{\kern0pt}A{\isacharcomma}{\kern0pt}i{\isacharparenright}{\kern0pt}{\isachardot}{\kern0pt}\ M{\isacharparenleft}{\kern0pt}x{\isacharparenright}{\kern0pt}{\isacharbraceright}{\kern0pt}\ {\isacharequal}{\kern0pt}\ A\ {\isasymunion}\ {\isacharparenleft}{\kern0pt}{\isasymUnion}y{\isasymin}i{\isachardot}{\kern0pt}\ {\isacharbraceleft}{\kern0pt}x{\isasymin}Pow{\isacharparenleft}{\kern0pt}Vfrom{\isacharparenleft}{\kern0pt}A{\isacharcomma}{\kern0pt}\ y{\isacharparenright}{\kern0pt}{\isacharparenright}{\kern0pt}{\isachardot}{\kern0pt}\ M{\isacharparenleft}{\kern0pt}x{\isacharparenright}{\kern0pt}{\isacharbraceright}{\kern0pt}{\isacharparenright}{\kern0pt}{\isachardoublequoteclose}\ \isanewline
\ \ \ \ \isacommand{by}\isamarkupfalse%
\ {\isacharparenleft}{\kern0pt}auto\ intro{\isacharcolon}{\kern0pt}transM{\isacharparenright}{\kern0pt}\isanewline
\ \ \isacommand{also}\isamarkupfalse%
\isanewline
\ \ \isacommand{have}\isamarkupfalse%
\ {\isachardoublequoteopen}{\isachardot}{\kern0pt}{\isachardot}{\kern0pt}{\isachardot}{\kern0pt}\ {\isacharequal}{\kern0pt}\ A\ {\isasymunion}\ {\isacharparenleft}{\kern0pt}{\isasymUnion}y{\isasymin}i{\isachardot}{\kern0pt}\ {\isacharbraceleft}{\kern0pt}x{\isasymin}Pow{\isacharparenleft}{\kern0pt}{\isacharbraceleft}{\kern0pt}z{\isasymin}Vfrom{\isacharparenleft}{\kern0pt}A{\isacharcomma}{\kern0pt}y{\isacharparenright}{\kern0pt}{\isachardot}{\kern0pt}\ M{\isacharparenleft}{\kern0pt}z{\isacharparenright}{\kern0pt}{\isacharbraceright}{\kern0pt}{\isacharparenright}{\kern0pt}{\isachardot}{\kern0pt}\ M{\isacharparenleft}{\kern0pt}x{\isacharparenright}{\kern0pt}{\isacharbraceright}{\kern0pt}{\isacharparenright}{\kern0pt}{\isachardoublequoteclose}\ \isanewline
\ \ \isacommand{proof}\isamarkupfalse%
\ {\isacharminus}{\kern0pt}\isanewline
\ \ \ \ \isacommand{have}\isamarkupfalse%
\ {\isachardoublequoteopen}{\isacharbraceleft}{\kern0pt}x{\isasymin}Pow{\isacharparenleft}{\kern0pt}Vfrom{\isacharparenleft}{\kern0pt}A{\isacharcomma}{\kern0pt}\ y{\isacharparenright}{\kern0pt}{\isacharparenright}{\kern0pt}{\isachardot}{\kern0pt}\ M{\isacharparenleft}{\kern0pt}x{\isacharparenright}{\kern0pt}{\isacharbraceright}{\kern0pt}\ {\isacharequal}{\kern0pt}\ {\isacharbraceleft}{\kern0pt}x{\isasymin}Pow{\isacharparenleft}{\kern0pt}{\isacharbraceleft}{\kern0pt}z{\isasymin}Vfrom{\isacharparenleft}{\kern0pt}A{\isacharcomma}{\kern0pt}y{\isacharparenright}{\kern0pt}{\isachardot}{\kern0pt}\ M{\isacharparenleft}{\kern0pt}z{\isacharparenright}{\kern0pt}{\isacharbraceright}{\kern0pt}{\isacharparenright}{\kern0pt}{\isachardot}{\kern0pt}\ M{\isacharparenleft}{\kern0pt}x{\isacharparenright}{\kern0pt}{\isacharbraceright}{\kern0pt}{\isachardoublequoteclose}\isanewline
\ \ \ \ \ \ \isakeyword{if}\ {\isachardoublequoteopen}y{\isasymin}i{\isachardoublequoteclose}\ \isakeyword{for}\ y\ \isacommand{by}\isamarkupfalse%
\ {\isacharparenleft}{\kern0pt}auto\ intro{\isacharcolon}{\kern0pt}transM{\isacharparenright}{\kern0pt}\isanewline
\ \ \ \ \isacommand{then}\isamarkupfalse%
\isanewline
\ \ \ \ \isacommand{show}\isamarkupfalse%
\ {\isacharquery}{\kern0pt}thesis\ \isacommand{by}\isamarkupfalse%
\ simp\isanewline
\ \ \isacommand{qed}\isamarkupfalse%
\isanewline
\ \ \isacommand{also}\isamarkupfalse%
\ \isacommand{from}\isamarkupfalse%
\ step\ \isanewline
\ \ \isacommand{have}\isamarkupfalse%
\ {\isachardoublequoteopen}\ {\isachardot}{\kern0pt}{\isachardot}{\kern0pt}{\isachardot}{\kern0pt}\ {\isacharequal}{\kern0pt}\ A\ {\isasymunion}\ {\isacharparenleft}{\kern0pt}{\isasymUnion}y{\isasymin}i{\isachardot}{\kern0pt}\ {\isacharbraceleft}{\kern0pt}x{\isasymin}Pow{\isacharparenleft}{\kern0pt}transrec{\isacharparenleft}{\kern0pt}y{\isacharcomma}{\kern0pt}\ HVfrom{\isacharparenleft}{\kern0pt}M{\isacharcomma}{\kern0pt}\ A{\isacharparenright}{\kern0pt}{\isacharparenright}{\kern0pt}{\isacharparenright}{\kern0pt}{\isachardot}{\kern0pt}\ M{\isacharparenleft}{\kern0pt}x{\isacharparenright}{\kern0pt}{\isacharbraceright}{\kern0pt}{\isacharparenright}{\kern0pt}{\isachardoublequoteclose}\ \isacommand{by}\isamarkupfalse%
\ auto\isanewline
\ \ \isacommand{also}\isamarkupfalse%
\isanewline
\ \ \isacommand{have}\isamarkupfalse%
\ {\isachardoublequoteopen}\ {\isachardot}{\kern0pt}{\isachardot}{\kern0pt}{\isachardot}{\kern0pt}\ {\isacharequal}{\kern0pt}\ transrec{\isacharparenleft}{\kern0pt}i{\isacharcomma}{\kern0pt}\ HVfrom{\isacharparenleft}{\kern0pt}M{\isacharcomma}{\kern0pt}\ A{\isacharparenright}{\kern0pt}{\isacharparenright}{\kern0pt}{\isachardoublequoteclose}\isanewline
\ \ \ \ \isacommand{using}\isamarkupfalse%
\ def{\isacharunderscore}{\kern0pt}transrec{\isacharbrackleft}{\kern0pt}of\ {\isachardoublequoteopen}{\isasymlambda}y{\isachardot}{\kern0pt}\ transrec{\isacharparenleft}{\kern0pt}y{\isacharcomma}{\kern0pt}\ HVfrom{\isacharparenleft}{\kern0pt}M{\isacharcomma}{\kern0pt}\ A{\isacharparenright}{\kern0pt}{\isacharparenright}{\kern0pt}{\isachardoublequoteclose}\ {\isachardoublequoteopen}HVfrom{\isacharparenleft}{\kern0pt}M{\isacharcomma}{\kern0pt}\ A{\isacharparenright}{\kern0pt}{\isachardoublequoteclose}\ i{\isacharcomma}{\kern0pt}symmetric{\isacharbrackright}{\kern0pt}\ \isanewline
\ \ \ \ \isacommand{unfolding}\isamarkupfalse%
\ HVfrom{\isacharunderscore}{\kern0pt}def\ \isacommand{by}\isamarkupfalse%
\ simp\isanewline
\ \ \isacommand{finally}\isamarkupfalse%
\isanewline
\ \ \isacommand{show}\isamarkupfalse%
\ {\isacharquery}{\kern0pt}case\ \isacommand{{\isachardot}{\kern0pt}}\isamarkupfalse%
\isanewline
\isacommand{qed}\isamarkupfalse%
%
\endisatagproof
{\isafoldproof}%
%
\isadelimproof
\isanewline
%
\endisadelimproof
\isanewline
\isacommand{lemma}\isamarkupfalse%
\ Vfrom{\isacharunderscore}{\kern0pt}abs{\isacharcolon}{\kern0pt}\ {\isachardoublequoteopen}{\isasymlbrakk}\ M{\isacharparenleft}{\kern0pt}A{\isacharparenright}{\kern0pt}{\isacharsemicolon}{\kern0pt}\ M{\isacharparenleft}{\kern0pt}i{\isacharparenright}{\kern0pt}{\isacharsemicolon}{\kern0pt}\ M{\isacharparenleft}{\kern0pt}V{\isacharparenright}{\kern0pt}{\isacharsemicolon}{\kern0pt}\ Ord{\isacharparenleft}{\kern0pt}i{\isacharparenright}{\kern0pt}\ {\isasymrbrakk}\ {\isasymLongrightarrow}\ is{\isacharunderscore}{\kern0pt}Vfrom{\isacharparenleft}{\kern0pt}M{\isacharcomma}{\kern0pt}A{\isacharcomma}{\kern0pt}i{\isacharcomma}{\kern0pt}V{\isacharparenright}{\kern0pt}\ {\isasymlongleftrightarrow}\ V\ {\isacharequal}{\kern0pt}\ {\isacharbraceleft}{\kern0pt}x{\isasymin}Vfrom{\isacharparenleft}{\kern0pt}A{\isacharcomma}{\kern0pt}i{\isacharparenright}{\kern0pt}{\isachardot}{\kern0pt}\ M{\isacharparenleft}{\kern0pt}x{\isacharparenright}{\kern0pt}{\isacharbraceright}{\kern0pt}{\isachardoublequoteclose}\isanewline
%
\isadelimproof
\ \ %
\endisadelimproof
%
\isatagproof
\isacommand{unfolding}\isamarkupfalse%
\ is{\isacharunderscore}{\kern0pt}Vfrom{\isacharunderscore}{\kern0pt}def\isanewline
\ \ \isacommand{using}\isamarkupfalse%
\ relation{\isadigit{2}}{\isacharunderscore}{\kern0pt}HVfrom\ HVfrom{\isacharunderscore}{\kern0pt}closed\ HVfrom{\isacharunderscore}{\kern0pt}replacement\ \isanewline
\ \ \ \ transrec{\isacharunderscore}{\kern0pt}abs{\isacharbrackleft}{\kern0pt}of\ {\isachardoublequoteopen}is{\isacharunderscore}{\kern0pt}HVfrom{\isacharparenleft}{\kern0pt}M{\isacharcomma}{\kern0pt}A{\isacharparenright}{\kern0pt}{\isachardoublequoteclose}\ i\ {\isachardoublequoteopen}HVfrom{\isacharparenleft}{\kern0pt}M{\isacharcomma}{\kern0pt}A{\isacharparenright}{\kern0pt}{\isachardoublequoteclose}{\isacharbrackright}{\kern0pt}\ transrec{\isacharunderscore}{\kern0pt}HVfrom\ \isacommand{by}\isamarkupfalse%
\ simp%
\endisatagproof
{\isafoldproof}%
%
\isadelimproof
\isanewline
%
\endisadelimproof
\isanewline
\isacommand{lemma}\isamarkupfalse%
\ Vfrom{\isacharunderscore}{\kern0pt}closed{\isacharcolon}{\kern0pt}\ {\isachardoublequoteopen}{\isasymlbrakk}\ M{\isacharparenleft}{\kern0pt}A{\isacharparenright}{\kern0pt}{\isacharsemicolon}{\kern0pt}\ M{\isacharparenleft}{\kern0pt}i{\isacharparenright}{\kern0pt}{\isacharsemicolon}{\kern0pt}\ Ord{\isacharparenleft}{\kern0pt}i{\isacharparenright}{\kern0pt}\ {\isasymrbrakk}\ {\isasymLongrightarrow}\ M{\isacharparenleft}{\kern0pt}{\isacharbraceleft}{\kern0pt}x{\isasymin}Vfrom{\isacharparenleft}{\kern0pt}A{\isacharcomma}{\kern0pt}i{\isacharparenright}{\kern0pt}{\isachardot}{\kern0pt}\ M{\isacharparenleft}{\kern0pt}x{\isacharparenright}{\kern0pt}{\isacharbraceright}{\kern0pt}{\isacharparenright}{\kern0pt}{\isachardoublequoteclose}\isanewline
%
\isadelimproof
\ \ %
\endisadelimproof
%
\isatagproof
\isacommand{unfolding}\isamarkupfalse%
\ is{\isacharunderscore}{\kern0pt}Vfrom{\isacharunderscore}{\kern0pt}def\isanewline
\ \ \isacommand{using}\isamarkupfalse%
\ relation{\isadigit{2}}{\isacharunderscore}{\kern0pt}HVfrom\ HVfrom{\isacharunderscore}{\kern0pt}closed\ HVfrom{\isacharunderscore}{\kern0pt}replacement\ \isanewline
\ \ \ \ transrec{\isacharunderscore}{\kern0pt}closed{\isacharbrackleft}{\kern0pt}of\ {\isachardoublequoteopen}is{\isacharunderscore}{\kern0pt}HVfrom{\isacharparenleft}{\kern0pt}M{\isacharcomma}{\kern0pt}A{\isacharparenright}{\kern0pt}{\isachardoublequoteclose}\ i\ {\isachardoublequoteopen}HVfrom{\isacharparenleft}{\kern0pt}M{\isacharcomma}{\kern0pt}A{\isacharparenright}{\kern0pt}{\isachardoublequoteclose}{\isacharbrackright}{\kern0pt}\ transrec{\isacharunderscore}{\kern0pt}HVfrom\ \isacommand{by}\isamarkupfalse%
\ simp%
\endisatagproof
{\isafoldproof}%
%
\isadelimproof
\isanewline
%
\endisadelimproof
\isanewline
\isacommand{lemma}\isamarkupfalse%
\ Vset{\isacharunderscore}{\kern0pt}abs{\isacharcolon}{\kern0pt}\ {\isachardoublequoteopen}{\isasymlbrakk}\ M{\isacharparenleft}{\kern0pt}i{\isacharparenright}{\kern0pt}{\isacharsemicolon}{\kern0pt}\ M{\isacharparenleft}{\kern0pt}V{\isacharparenright}{\kern0pt}{\isacharsemicolon}{\kern0pt}\ Ord{\isacharparenleft}{\kern0pt}i{\isacharparenright}{\kern0pt}\ {\isasymrbrakk}\ {\isasymLongrightarrow}\ is{\isacharunderscore}{\kern0pt}Vset{\isacharparenleft}{\kern0pt}M{\isacharcomma}{\kern0pt}i{\isacharcomma}{\kern0pt}V{\isacharparenright}{\kern0pt}\ {\isasymlongleftrightarrow}\ V\ {\isacharequal}{\kern0pt}\ {\isacharbraceleft}{\kern0pt}x{\isasymin}Vset{\isacharparenleft}{\kern0pt}i{\isacharparenright}{\kern0pt}{\isachardot}{\kern0pt}\ M{\isacharparenleft}{\kern0pt}x{\isacharparenright}{\kern0pt}{\isacharbraceright}{\kern0pt}{\isachardoublequoteclose}\isanewline
%
\isadelimproof
\ \ %
\endisadelimproof
%
\isatagproof
\isacommand{using}\isamarkupfalse%
\ Vfrom{\isacharunderscore}{\kern0pt}abs\ \isacommand{unfolding}\isamarkupfalse%
\ is{\isacharunderscore}{\kern0pt}Vset{\isacharunderscore}{\kern0pt}def\ \isacommand{by}\isamarkupfalse%
\ simp%
\endisatagproof
{\isafoldproof}%
%
\isadelimproof
\isanewline
%
\endisadelimproof
\isanewline
\isacommand{lemma}\isamarkupfalse%
\ Vset{\isacharunderscore}{\kern0pt}closed{\isacharcolon}{\kern0pt}\ {\isachardoublequoteopen}{\isasymlbrakk}\ M{\isacharparenleft}{\kern0pt}i{\isacharparenright}{\kern0pt}{\isacharsemicolon}{\kern0pt}\ Ord{\isacharparenleft}{\kern0pt}i{\isacharparenright}{\kern0pt}\ {\isasymrbrakk}\ {\isasymLongrightarrow}\ M{\isacharparenleft}{\kern0pt}{\isacharbraceleft}{\kern0pt}x{\isasymin}Vset{\isacharparenleft}{\kern0pt}i{\isacharparenright}{\kern0pt}{\isachardot}{\kern0pt}\ M{\isacharparenleft}{\kern0pt}x{\isacharparenright}{\kern0pt}{\isacharbraceright}{\kern0pt}{\isacharparenright}{\kern0pt}{\isachardoublequoteclose}\isanewline
%
\isadelimproof
\ \ %
\endisadelimproof
%
\isatagproof
\isacommand{using}\isamarkupfalse%
\ Vfrom{\isacharunderscore}{\kern0pt}closed\ \isacommand{unfolding}\isamarkupfalse%
\ is{\isacharunderscore}{\kern0pt}Vset{\isacharunderscore}{\kern0pt}def\ \isacommand{by}\isamarkupfalse%
\ simp%
\endisatagproof
{\isafoldproof}%
%
\isadelimproof
\isanewline
%
\endisadelimproof
\isanewline
\isacommand{lemma}\isamarkupfalse%
\ Hrank{\isacharunderscore}{\kern0pt}trancl{\isacharcolon}{\kern0pt}{\isachardoublequoteopen}Hrank{\isacharparenleft}{\kern0pt}y{\isacharcomma}{\kern0pt}\ restrict{\isacharparenleft}{\kern0pt}f{\isacharcomma}{\kern0pt}Memrel{\isacharparenleft}{\kern0pt}eclose{\isacharparenleft}{\kern0pt}{\isacharbraceleft}{\kern0pt}x{\isacharbraceright}{\kern0pt}{\isacharparenright}{\kern0pt}{\isacharparenright}{\kern0pt}{\isacharminus}{\kern0pt}{\isacharbackquote}{\kern0pt}{\isacharbackquote}{\kern0pt}{\isacharbraceleft}{\kern0pt}y{\isacharbraceright}{\kern0pt}{\isacharparenright}{\kern0pt}{\isacharparenright}{\kern0pt}\isanewline
\ \ \ \ \ \ \ \ \ \ \ \ \ \ \ \ \ \ {\isacharequal}{\kern0pt}\ Hrank{\isacharparenleft}{\kern0pt}y{\isacharcomma}{\kern0pt}\ restrict{\isacharparenleft}{\kern0pt}f{\isacharcomma}{\kern0pt}{\isacharparenleft}{\kern0pt}Memrel{\isacharparenleft}{\kern0pt}eclose{\isacharparenleft}{\kern0pt}{\isacharbraceleft}{\kern0pt}x{\isacharbraceright}{\kern0pt}{\isacharparenright}{\kern0pt}{\isacharparenright}{\kern0pt}{\isacharcircum}{\kern0pt}{\isacharplus}{\kern0pt}{\isacharparenright}{\kern0pt}{\isacharminus}{\kern0pt}{\isacharbackquote}{\kern0pt}{\isacharbackquote}{\kern0pt}{\isacharbraceleft}{\kern0pt}y{\isacharbraceright}{\kern0pt}{\isacharparenright}{\kern0pt}{\isacharparenright}{\kern0pt}{\isachardoublequoteclose}\isanewline
%
\isadelimproof
\ \ %
\endisadelimproof
%
\isatagproof
\isacommand{unfolding}\isamarkupfalse%
\ Hrank{\isacharunderscore}{\kern0pt}def\isanewline
\ \ \isacommand{using}\isamarkupfalse%
\ restrict{\isacharunderscore}{\kern0pt}trans{\isacharunderscore}{\kern0pt}eq\ \isacommand{by}\isamarkupfalse%
\ simp%
\endisatagproof
{\isafoldproof}%
%
\isadelimproof
\isanewline
%
\endisadelimproof
\isanewline
\isacommand{lemma}\isamarkupfalse%
\ rank{\isacharunderscore}{\kern0pt}trancl{\isacharcolon}{\kern0pt}\ {\isachardoublequoteopen}rank{\isacharparenleft}{\kern0pt}x{\isacharparenright}{\kern0pt}\ {\isacharequal}{\kern0pt}\ wfrec{\isacharparenleft}{\kern0pt}rrank{\isacharparenleft}{\kern0pt}x{\isacharparenright}{\kern0pt}{\isacharcomma}{\kern0pt}\ x{\isacharcomma}{\kern0pt}\ Hrank{\isacharparenright}{\kern0pt}{\isachardoublequoteclose}\isanewline
%
\isadelimproof
%
\endisadelimproof
%
\isatagproof
\isacommand{proof}\isamarkupfalse%
\ {\isacharminus}{\kern0pt}\isanewline
\ \ \isacommand{have}\isamarkupfalse%
\ {\isachardoublequoteopen}rank{\isacharparenleft}{\kern0pt}x{\isacharparenright}{\kern0pt}\ {\isacharequal}{\kern0pt}\ \ wfrec{\isacharparenleft}{\kern0pt}Memrel{\isacharparenleft}{\kern0pt}eclose{\isacharparenleft}{\kern0pt}{\isacharbraceleft}{\kern0pt}x{\isacharbraceright}{\kern0pt}{\isacharparenright}{\kern0pt}{\isacharparenright}{\kern0pt}{\isacharcomma}{\kern0pt}\ x{\isacharcomma}{\kern0pt}\ Hrank{\isacharparenright}{\kern0pt}{\isachardoublequoteclose}\isanewline
\ \ \ \ {\isacharparenleft}{\kern0pt}\isakeyword{is}\ {\isachardoublequoteopen}{\isacharunderscore}{\kern0pt}\ {\isacharequal}{\kern0pt}\ wfrec{\isacharparenleft}{\kern0pt}{\isacharquery}{\kern0pt}r{\isacharcomma}{\kern0pt}{\isacharunderscore}{\kern0pt}{\isacharcomma}{\kern0pt}{\isacharunderscore}{\kern0pt}{\isacharparenright}{\kern0pt}{\isachardoublequoteclose}{\isacharparenright}{\kern0pt}\isanewline
\ \ \ \ \isacommand{unfolding}\isamarkupfalse%
\ rank{\isacharunderscore}{\kern0pt}def\ transrec{\isacharunderscore}{\kern0pt}def\ Hrank{\isacharunderscore}{\kern0pt}def\ \isacommand{by}\isamarkupfalse%
\ simp\isanewline
\ \ \isacommand{also}\isamarkupfalse%
\isanewline
\ \ \isacommand{have}\isamarkupfalse%
\ {\isachardoublequoteopen}\ {\isachardot}{\kern0pt}{\isachardot}{\kern0pt}{\isachardot}{\kern0pt}\ {\isacharequal}{\kern0pt}\ wftrec{\isacharparenleft}{\kern0pt}{\isacharquery}{\kern0pt}r{\isacharcircum}{\kern0pt}{\isacharplus}{\kern0pt}{\isacharcomma}{\kern0pt}\ x{\isacharcomma}{\kern0pt}\ {\isasymlambda}y\ f{\isachardot}{\kern0pt}\ Hrank{\isacharparenleft}{\kern0pt}y{\isacharcomma}{\kern0pt}\ restrict{\isacharparenleft}{\kern0pt}f{\isacharcomma}{\kern0pt}{\isacharquery}{\kern0pt}r{\isacharminus}{\kern0pt}{\isacharbackquote}{\kern0pt}{\isacharbackquote}{\kern0pt}{\isacharbraceleft}{\kern0pt}y{\isacharbraceright}{\kern0pt}{\isacharparenright}{\kern0pt}{\isacharparenright}{\kern0pt}{\isacharparenright}{\kern0pt}{\isachardoublequoteclose}\isanewline
\ \ \ \ \isacommand{unfolding}\isamarkupfalse%
\ wfrec{\isacharunderscore}{\kern0pt}def\ \isacommand{{\isachardot}{\kern0pt}{\isachardot}{\kern0pt}}\isamarkupfalse%
\isanewline
\ \ \isacommand{also}\isamarkupfalse%
\isanewline
\ \ \isacommand{have}\isamarkupfalse%
\ {\isachardoublequoteopen}\ {\isachardot}{\kern0pt}{\isachardot}{\kern0pt}{\isachardot}{\kern0pt}\ {\isacharequal}{\kern0pt}\ wftrec{\isacharparenleft}{\kern0pt}{\isacharquery}{\kern0pt}r{\isacharcircum}{\kern0pt}{\isacharplus}{\kern0pt}{\isacharcomma}{\kern0pt}\ x{\isacharcomma}{\kern0pt}\ {\isasymlambda}y\ f{\isachardot}{\kern0pt}\ Hrank{\isacharparenleft}{\kern0pt}y{\isacharcomma}{\kern0pt}\ restrict{\isacharparenleft}{\kern0pt}f{\isacharcomma}{\kern0pt}{\isacharparenleft}{\kern0pt}{\isacharquery}{\kern0pt}r{\isacharcircum}{\kern0pt}{\isacharplus}{\kern0pt}{\isacharparenright}{\kern0pt}{\isacharminus}{\kern0pt}{\isacharbackquote}{\kern0pt}{\isacharbackquote}{\kern0pt}{\isacharbraceleft}{\kern0pt}y{\isacharbraceright}{\kern0pt}{\isacharparenright}{\kern0pt}{\isacharparenright}{\kern0pt}{\isacharparenright}{\kern0pt}{\isachardoublequoteclose}\isanewline
\ \ \ \ \isacommand{using}\isamarkupfalse%
\ Hrank{\isacharunderscore}{\kern0pt}trancl\ \isacommand{by}\isamarkupfalse%
\ simp\isanewline
\ \ \isacommand{also}\isamarkupfalse%
\isanewline
\ \ \isacommand{have}\isamarkupfalse%
\ {\isachardoublequoteopen}\ {\isachardot}{\kern0pt}{\isachardot}{\kern0pt}{\isachardot}{\kern0pt}\ {\isacharequal}{\kern0pt}\ \ wfrec{\isacharparenleft}{\kern0pt}{\isacharquery}{\kern0pt}r{\isacharcircum}{\kern0pt}{\isacharplus}{\kern0pt}{\isacharcomma}{\kern0pt}\ x{\isacharcomma}{\kern0pt}\ Hrank{\isacharparenright}{\kern0pt}{\isachardoublequoteclose}\isanewline
\ \ \ \ \isacommand{unfolding}\isamarkupfalse%
\ wfrec{\isacharunderscore}{\kern0pt}def\ \isacommand{using}\isamarkupfalse%
\ trancl{\isacharunderscore}{\kern0pt}eq{\isacharunderscore}{\kern0pt}r{\isacharbrackleft}{\kern0pt}OF\ relation{\isacharunderscore}{\kern0pt}trancl\ trans{\isacharunderscore}{\kern0pt}trancl{\isacharbrackright}{\kern0pt}\ \isacommand{by}\isamarkupfalse%
\ simp\isanewline
\ \ \isacommand{finally}\isamarkupfalse%
\isanewline
\ \ \isacommand{show}\isamarkupfalse%
\ {\isacharquery}{\kern0pt}thesis\ \isacommand{unfolding}\isamarkupfalse%
\ rrank{\isacharunderscore}{\kern0pt}def\ \isacommand{{\isachardot}{\kern0pt}}\isamarkupfalse%
\isanewline
\isacommand{qed}\isamarkupfalse%
%
\endisatagproof
{\isafoldproof}%
%
\isadelimproof
\isanewline
%
\endisadelimproof
\isanewline
\isacommand{lemma}\isamarkupfalse%
\ univ{\isacharunderscore}{\kern0pt}PHrank\ {\isacharcolon}{\kern0pt}\ {\isachardoublequoteopen}{\isasymlbrakk}\ M{\isacharparenleft}{\kern0pt}z{\isacharparenright}{\kern0pt}\ {\isacharsemicolon}{\kern0pt}\ M{\isacharparenleft}{\kern0pt}f{\isacharparenright}{\kern0pt}\ {\isasymrbrakk}\ {\isasymLongrightarrow}\ univalent{\isacharparenleft}{\kern0pt}M{\isacharcomma}{\kern0pt}z{\isacharcomma}{\kern0pt}PHrank{\isacharparenleft}{\kern0pt}M{\isacharcomma}{\kern0pt}f{\isacharparenright}{\kern0pt}{\isacharparenright}{\kern0pt}{\isachardoublequoteclose}\ \isanewline
%
\isadelimproof
\ \ %
\endisadelimproof
%
\isatagproof
\isacommand{unfolding}\isamarkupfalse%
\ univalent{\isacharunderscore}{\kern0pt}def\ PHrank{\isacharunderscore}{\kern0pt}def\ \isacommand{by}\isamarkupfalse%
\ simp%
\endisatagproof
{\isafoldproof}%
%
\isadelimproof
\isanewline
%
\endisadelimproof
\isanewline
\isanewline
\isacommand{lemma}\isamarkupfalse%
\ PHrank{\isacharunderscore}{\kern0pt}abs\ {\isacharcolon}{\kern0pt}\isanewline
\ \ \ \ {\isachardoublequoteopen}{\isasymlbrakk}\ M{\isacharparenleft}{\kern0pt}f{\isacharparenright}{\kern0pt}\ {\isacharsemicolon}{\kern0pt}\ M{\isacharparenleft}{\kern0pt}y{\isacharparenright}{\kern0pt}\ {\isasymrbrakk}\ {\isasymLongrightarrow}\ PHrank{\isacharparenleft}{\kern0pt}M{\isacharcomma}{\kern0pt}f{\isacharcomma}{\kern0pt}y{\isacharcomma}{\kern0pt}z{\isacharparenright}{\kern0pt}\ {\isasymlongleftrightarrow}\ M{\isacharparenleft}{\kern0pt}z{\isacharparenright}{\kern0pt}\ {\isasymand}\ z\ {\isacharequal}{\kern0pt}\ succ{\isacharparenleft}{\kern0pt}f{\isacharbackquote}{\kern0pt}y{\isacharparenright}{\kern0pt}{\isachardoublequoteclose}\isanewline
%
\isadelimproof
\ \ %
\endisadelimproof
%
\isatagproof
\isacommand{unfolding}\isamarkupfalse%
\ PHrank{\isacharunderscore}{\kern0pt}def\ \isacommand{by}\isamarkupfalse%
\ simp%
\endisatagproof
{\isafoldproof}%
%
\isadelimproof
\isanewline
%
\endisadelimproof
\isanewline
\isacommand{lemma}\isamarkupfalse%
\ PHrank{\isacharunderscore}{\kern0pt}closed\ {\isacharcolon}{\kern0pt}\ {\isachardoublequoteopen}PHrank{\isacharparenleft}{\kern0pt}M{\isacharcomma}{\kern0pt}f{\isacharcomma}{\kern0pt}y{\isacharcomma}{\kern0pt}z{\isacharparenright}{\kern0pt}\ {\isasymLongrightarrow}\ M{\isacharparenleft}{\kern0pt}z{\isacharparenright}{\kern0pt}{\isachardoublequoteclose}\ \isanewline
%
\isadelimproof
\ \ %
\endisadelimproof
%
\isatagproof
\isacommand{unfolding}\isamarkupfalse%
\ PHrank{\isacharunderscore}{\kern0pt}def\ \isacommand{by}\isamarkupfalse%
\ simp%
\endisatagproof
{\isafoldproof}%
%
\isadelimproof
\isanewline
%
\endisadelimproof
\isanewline
\isacommand{lemma}\isamarkupfalse%
\ Replace{\isacharunderscore}{\kern0pt}PHrank{\isacharunderscore}{\kern0pt}abs{\isacharcolon}{\kern0pt}\isanewline
\ \ \isakeyword{assumes}\isanewline
\ \ \ \ {\isachardoublequoteopen}M{\isacharparenleft}{\kern0pt}z{\isacharparenright}{\kern0pt}{\isachardoublequoteclose}\ {\isachardoublequoteopen}M{\isacharparenleft}{\kern0pt}f{\isacharparenright}{\kern0pt}{\isachardoublequoteclose}\ {\isachardoublequoteopen}M{\isacharparenleft}{\kern0pt}hr{\isacharparenright}{\kern0pt}{\isachardoublequoteclose}\ \isanewline
\ \ \isakeyword{shows}\isanewline
\ \ \ \ {\isachardoublequoteopen}is{\isacharunderscore}{\kern0pt}Replace{\isacharparenleft}{\kern0pt}M{\isacharcomma}{\kern0pt}z{\isacharcomma}{\kern0pt}PHrank{\isacharparenleft}{\kern0pt}M{\isacharcomma}{\kern0pt}f{\isacharparenright}{\kern0pt}{\isacharcomma}{\kern0pt}hr{\isacharparenright}{\kern0pt}\ {\isasymlongleftrightarrow}\ hr\ {\isacharequal}{\kern0pt}\ Replace{\isacharparenleft}{\kern0pt}z{\isacharcomma}{\kern0pt}PHrank{\isacharparenleft}{\kern0pt}M{\isacharcomma}{\kern0pt}f{\isacharparenright}{\kern0pt}{\isacharparenright}{\kern0pt}{\isachardoublequoteclose}\ \isanewline
%
\isadelimproof
%
\endisadelimproof
%
\isatagproof
\isacommand{proof}\isamarkupfalse%
\ {\isacharminus}{\kern0pt}\isanewline
\ \ \isacommand{have}\isamarkupfalse%
\ {\isachardoublequoteopen}{\isasymAnd}x\ y{\isachardot}{\kern0pt}\ {\isasymlbrakk}x{\isasymin}z{\isacharsemicolon}{\kern0pt}\ PHrank{\isacharparenleft}{\kern0pt}M{\isacharcomma}{\kern0pt}f{\isacharcomma}{\kern0pt}x{\isacharcomma}{\kern0pt}y{\isacharparenright}{\kern0pt}\ {\isasymrbrakk}\ {\isasymLongrightarrow}\ M{\isacharparenleft}{\kern0pt}y{\isacharparenright}{\kern0pt}{\isachardoublequoteclose}\isanewline
\ \ \ \ \isacommand{using}\isamarkupfalse%
\ {\isacartoucheopen}M{\isacharparenleft}{\kern0pt}z{\isacharparenright}{\kern0pt}{\isacartoucheclose}\ {\isacartoucheopen}M{\isacharparenleft}{\kern0pt}f{\isacharparenright}{\kern0pt}{\isacartoucheclose}\ \isacommand{unfolding}\isamarkupfalse%
\ PHrank{\isacharunderscore}{\kern0pt}def\ \isacommand{by}\isamarkupfalse%
\ simp\isanewline
\ \ \isacommand{then}\isamarkupfalse%
\isanewline
\ \ \isacommand{show}\isamarkupfalse%
\ {\isacharquery}{\kern0pt}thesis\ \isacommand{using}\isamarkupfalse%
\ {\isacartoucheopen}M{\isacharparenleft}{\kern0pt}z{\isacharparenright}{\kern0pt}{\isacartoucheclose}\ {\isacartoucheopen}M{\isacharparenleft}{\kern0pt}hr{\isacharparenright}{\kern0pt}{\isacartoucheclose}\ {\isacartoucheopen}M{\isacharparenleft}{\kern0pt}f{\isacharparenright}{\kern0pt}{\isacartoucheclose}\ univ{\isacharunderscore}{\kern0pt}PHrank\ Replace{\isacharunderscore}{\kern0pt}abs\ \isacommand{by}\isamarkupfalse%
\ simp\isanewline
\isacommand{qed}\isamarkupfalse%
%
\endisatagproof
{\isafoldproof}%
%
\isadelimproof
\isanewline
%
\endisadelimproof
\isanewline
\isacommand{lemma}\isamarkupfalse%
\ RepFun{\isacharunderscore}{\kern0pt}PHrank{\isacharcolon}{\kern0pt}\isanewline
\ \ \isakeyword{assumes}\isanewline
\ \ \ \ {\isachardoublequoteopen}M{\isacharparenleft}{\kern0pt}R{\isacharparenright}{\kern0pt}{\isachardoublequoteclose}\ {\isachardoublequoteopen}M{\isacharparenleft}{\kern0pt}A{\isacharparenright}{\kern0pt}{\isachardoublequoteclose}\ {\isachardoublequoteopen}M{\isacharparenleft}{\kern0pt}f{\isacharparenright}{\kern0pt}{\isachardoublequoteclose}\ \isanewline
\ \ \isakeyword{shows}\isanewline
\ \ {\isachardoublequoteopen}Replace{\isacharparenleft}{\kern0pt}A{\isacharcomma}{\kern0pt}PHrank{\isacharparenleft}{\kern0pt}M{\isacharcomma}{\kern0pt}f{\isacharparenright}{\kern0pt}{\isacharparenright}{\kern0pt}\ {\isacharequal}{\kern0pt}\ RepFun{\isacharparenleft}{\kern0pt}A{\isacharcomma}{\kern0pt}{\isasymlambda}y{\isachardot}{\kern0pt}\ succ{\isacharparenleft}{\kern0pt}f{\isacharbackquote}{\kern0pt}y{\isacharparenright}{\kern0pt}{\isacharparenright}{\kern0pt}{\isachardoublequoteclose}\isanewline
%
\isadelimproof
%
\endisadelimproof
%
\isatagproof
\isacommand{proof}\isamarkupfalse%
\ {\isacharminus}{\kern0pt}\isanewline
\ \ \isacommand{have}\isamarkupfalse%
\ {\isachardoublequoteopen}{\isacharbraceleft}{\kern0pt}z\ {\isachardot}{\kern0pt}\ y\ {\isasymin}\ A{\isacharcomma}{\kern0pt}\ M{\isacharparenleft}{\kern0pt}z{\isacharparenright}{\kern0pt}\ {\isasymand}\ z\ {\isacharequal}{\kern0pt}\ succ{\isacharparenleft}{\kern0pt}f{\isacharbackquote}{\kern0pt}y{\isacharparenright}{\kern0pt}{\isacharbraceright}{\kern0pt}\ {\isacharequal}{\kern0pt}\ {\isacharbraceleft}{\kern0pt}z\ {\isachardot}{\kern0pt}\ y\ {\isasymin}\ A{\isacharcomma}{\kern0pt}\ z\ {\isacharequal}{\kern0pt}\ succ{\isacharparenleft}{\kern0pt}f{\isacharbackquote}{\kern0pt}y{\isacharparenright}{\kern0pt}{\isacharbraceright}{\kern0pt}{\isachardoublequoteclose}\ \isanewline
\ \ \ \ \isacommand{using}\isamarkupfalse%
\ assms\ PHrank{\isacharunderscore}{\kern0pt}closed\ transM{\isacharbrackleft}{\kern0pt}of\ {\isacharunderscore}{\kern0pt}\ A{\isacharbrackright}{\kern0pt}\ \isacommand{by}\isamarkupfalse%
\ blast\isanewline
\ \ \isacommand{also}\isamarkupfalse%
\isanewline
\ \ \isacommand{have}\isamarkupfalse%
\ {\isachardoublequoteopen}\ {\isachardot}{\kern0pt}{\isachardot}{\kern0pt}{\isachardot}{\kern0pt}\ {\isacharequal}{\kern0pt}\ {\isacharbraceleft}{\kern0pt}succ{\isacharparenleft}{\kern0pt}f{\isacharbackquote}{\kern0pt}y{\isacharparenright}{\kern0pt}\ {\isachardot}{\kern0pt}\ y\ {\isasymin}\ A{\isacharbraceright}{\kern0pt}{\isachardoublequoteclose}\ \isacommand{by}\isamarkupfalse%
\ auto\isanewline
\ \ \isacommand{finally}\isamarkupfalse%
\ \isanewline
\ \ \isacommand{show}\isamarkupfalse%
\ {\isacharquery}{\kern0pt}thesis\ \isacommand{using}\isamarkupfalse%
\ assms\ PHrank{\isacharunderscore}{\kern0pt}abs\ transM{\isacharbrackleft}{\kern0pt}of\ {\isacharunderscore}{\kern0pt}\ A{\isacharbrackright}{\kern0pt}\ \isacommand{by}\isamarkupfalse%
\ simp\isanewline
\isacommand{qed}\isamarkupfalse%
%
\endisatagproof
{\isafoldproof}%
%
\isadelimproof
\isanewline
%
\endisadelimproof
\isanewline
\isacommand{lemma}\isamarkupfalse%
\ RepFun{\isacharunderscore}{\kern0pt}PHrank{\isacharunderscore}{\kern0pt}closed\ {\isacharcolon}{\kern0pt}\isanewline
\ \ \isakeyword{assumes}\isanewline
\ \ \ \ {\isachardoublequoteopen}M{\isacharparenleft}{\kern0pt}f{\isacharparenright}{\kern0pt}{\isachardoublequoteclose}\ {\isachardoublequoteopen}M{\isacharparenleft}{\kern0pt}A{\isacharparenright}{\kern0pt}{\isachardoublequoteclose}\ \isanewline
\ \ \isakeyword{shows}\isanewline
\ \ \ \ {\isachardoublequoteopen}M{\isacharparenleft}{\kern0pt}Replace{\isacharparenleft}{\kern0pt}A{\isacharcomma}{\kern0pt}PHrank{\isacharparenleft}{\kern0pt}M{\isacharcomma}{\kern0pt}f{\isacharparenright}{\kern0pt}{\isacharparenright}{\kern0pt}{\isacharparenright}{\kern0pt}{\isachardoublequoteclose}\isanewline
%
\isadelimproof
%
\endisadelimproof
%
\isatagproof
\isacommand{proof}\isamarkupfalse%
\ {\isacharminus}{\kern0pt}\isanewline
\ \ \isacommand{have}\isamarkupfalse%
\ {\isachardoublequoteopen}{\isasymlbrakk}\ x{\isasymin}A\ {\isacharsemicolon}{\kern0pt}\ PHrank{\isacharparenleft}{\kern0pt}M{\isacharcomma}{\kern0pt}f{\isacharcomma}{\kern0pt}x{\isacharcomma}{\kern0pt}y{\isacharparenright}{\kern0pt}\ {\isasymrbrakk}\ {\isasymLongrightarrow}\ M{\isacharparenleft}{\kern0pt}y{\isacharparenright}{\kern0pt}{\isachardoublequoteclose}\ \isakeyword{for}\ x\ y\isanewline
\ \ \ \ \isacommand{using}\isamarkupfalse%
\ assms\ \isacommand{unfolding}\isamarkupfalse%
\ PHrank{\isacharunderscore}{\kern0pt}def\ \isacommand{by}\isamarkupfalse%
\ simp\isanewline
\ \ \isacommand{with}\isamarkupfalse%
\ univ{\isacharunderscore}{\kern0pt}PHrank\isanewline
\ \ \isacommand{show}\isamarkupfalse%
\ {\isacharquery}{\kern0pt}thesis\ \isacommand{using}\isamarkupfalse%
\ assms\ PHrank{\isacharunderscore}{\kern0pt}replacement\ \isacommand{by}\isamarkupfalse%
\ simp\isanewline
\isacommand{qed}\isamarkupfalse%
%
\endisatagproof
{\isafoldproof}%
%
\isadelimproof
\isanewline
%
\endisadelimproof
\isanewline
\isacommand{lemma}\isamarkupfalse%
\ relation{\isadigit{2}}{\isacharunderscore}{\kern0pt}Hrank\ {\isacharcolon}{\kern0pt}\isanewline
\ \ {\isachardoublequoteopen}relation{\isadigit{2}}{\isacharparenleft}{\kern0pt}M{\isacharcomma}{\kern0pt}is{\isacharunderscore}{\kern0pt}Hrank{\isacharparenleft}{\kern0pt}M{\isacharparenright}{\kern0pt}{\isacharcomma}{\kern0pt}Hrank{\isacharparenright}{\kern0pt}{\isachardoublequoteclose}\isanewline
%
\isadelimproof
\ \ %
\endisadelimproof
%
\isatagproof
\isacommand{unfolding}\isamarkupfalse%
\ is{\isacharunderscore}{\kern0pt}Hrank{\isacharunderscore}{\kern0pt}def\ Hrank{\isacharunderscore}{\kern0pt}def\ relation{\isadigit{2}}{\isacharunderscore}{\kern0pt}def\isanewline
\ \ \isacommand{using}\isamarkupfalse%
\ Replace{\isacharunderscore}{\kern0pt}PHrank{\isacharunderscore}{\kern0pt}abs\ RepFun{\isacharunderscore}{\kern0pt}PHrank\ RepFun{\isacharunderscore}{\kern0pt}PHrank{\isacharunderscore}{\kern0pt}closed\ \isacommand{by}\isamarkupfalse%
\ auto%
\endisatagproof
{\isafoldproof}%
%
\isadelimproof
\isanewline
%
\endisadelimproof
\isanewline
\isanewline
\isacommand{lemma}\isamarkupfalse%
\ Union{\isacharunderscore}{\kern0pt}PHrank{\isacharunderscore}{\kern0pt}closed{\isacharcolon}{\kern0pt}\isanewline
\ \ \isakeyword{assumes}\ \isanewline
\ \ \ \ {\isachardoublequoteopen}M{\isacharparenleft}{\kern0pt}x{\isacharparenright}{\kern0pt}{\isachardoublequoteclose}\ {\isachardoublequoteopen}M{\isacharparenleft}{\kern0pt}f{\isacharparenright}{\kern0pt}{\isachardoublequoteclose}\isanewline
\ \ \isakeyword{shows}\ \isanewline
\ \ \ \ {\isachardoublequoteopen}M{\isacharparenleft}{\kern0pt}{\isasymUnion}y{\isasymin}x{\isachardot}{\kern0pt}\ succ{\isacharparenleft}{\kern0pt}f{\isacharbackquote}{\kern0pt}y{\isacharparenright}{\kern0pt}{\isacharparenright}{\kern0pt}{\isachardoublequoteclose}\isanewline
%
\isadelimproof
%
\endisadelimproof
%
\isatagproof
\isacommand{proof}\isamarkupfalse%
\ {\isacharminus}{\kern0pt}\isanewline
\ \ \isacommand{have}\isamarkupfalse%
\ {\isachardoublequoteopen}M{\isacharparenleft}{\kern0pt}succ{\isacharparenleft}{\kern0pt}f{\isacharbackquote}{\kern0pt}y{\isacharparenright}{\kern0pt}{\isacharparenright}{\kern0pt}{\isachardoublequoteclose}\ \isakeyword{if}\ {\isachardoublequoteopen}y{\isasymin}x{\isachardoublequoteclose}\ \isakeyword{for}\ y\isanewline
\ \ \ \ \isacommand{using}\isamarkupfalse%
\ that\ assms\ transM{\isacharbrackleft}{\kern0pt}of\ {\isacharunderscore}{\kern0pt}\ x{\isacharbrackright}{\kern0pt}\ \isacommand{by}\isamarkupfalse%
\ simp\isanewline
\ \ \isacommand{then}\isamarkupfalse%
\isanewline
\ \ \isacommand{have}\isamarkupfalse%
\ {\isachardoublequoteopen}M{\isacharparenleft}{\kern0pt}{\isacharbraceleft}{\kern0pt}succ{\isacharparenleft}{\kern0pt}f{\isacharbackquote}{\kern0pt}y{\isacharparenright}{\kern0pt}{\isachardot}{\kern0pt}\ y{\isasymin}x{\isacharbraceright}{\kern0pt}{\isacharparenright}{\kern0pt}{\isachardoublequoteclose}\isanewline
\ \ \ \ \isacommand{using}\isamarkupfalse%
\ assms\ transM{\isacharbrackleft}{\kern0pt}of\ {\isacharunderscore}{\kern0pt}\ x{\isacharbrackright}{\kern0pt}\ \ RepFun{\isacharunderscore}{\kern0pt}PHrank{\isacharunderscore}{\kern0pt}closed\ RepFun{\isacharunderscore}{\kern0pt}PHrank\ \isacommand{by}\isamarkupfalse%
\ simp\isanewline
\ \ \isacommand{then}\isamarkupfalse%
\ \isacommand{show}\isamarkupfalse%
\ {\isacharquery}{\kern0pt}thesis\ \isacommand{using}\isamarkupfalse%
\ assms\ \isacommand{by}\isamarkupfalse%
\ simp\isanewline
\isacommand{qed}\isamarkupfalse%
%
\endisatagproof
{\isafoldproof}%
%
\isadelimproof
\isanewline
%
\endisadelimproof
\isanewline
\isacommand{lemma}\isamarkupfalse%
\ is{\isacharunderscore}{\kern0pt}Hrank{\isacharunderscore}{\kern0pt}closed\ {\isacharcolon}{\kern0pt}\ \isanewline
\ \ {\isachardoublequoteopen}M{\isacharparenleft}{\kern0pt}A{\isacharparenright}{\kern0pt}\ {\isasymLongrightarrow}\ {\isasymforall}x{\isacharbrackleft}{\kern0pt}M{\isacharbrackright}{\kern0pt}{\isachardot}{\kern0pt}\ {\isasymforall}g{\isacharbrackleft}{\kern0pt}M{\isacharbrackright}{\kern0pt}{\isachardot}{\kern0pt}\ function{\isacharparenleft}{\kern0pt}g{\isacharparenright}{\kern0pt}\ {\isasymlongrightarrow}\ M{\isacharparenleft}{\kern0pt}Hrank{\isacharparenleft}{\kern0pt}x{\isacharcomma}{\kern0pt}g{\isacharparenright}{\kern0pt}{\isacharparenright}{\kern0pt}{\isachardoublequoteclose}\isanewline
%
\isadelimproof
\ \ %
\endisadelimproof
%
\isatagproof
\isacommand{unfolding}\isamarkupfalse%
\ Hrank{\isacharunderscore}{\kern0pt}def\ \isacommand{using}\isamarkupfalse%
\ RepFun{\isacharunderscore}{\kern0pt}PHrank{\isacharunderscore}{\kern0pt}closed\ Union{\isacharunderscore}{\kern0pt}PHrank{\isacharunderscore}{\kern0pt}closed\ \isacommand{by}\isamarkupfalse%
\ simp%
\endisatagproof
{\isafoldproof}%
%
\isadelimproof
\isanewline
%
\endisadelimproof
\isanewline
\isacommand{lemma}\isamarkupfalse%
\ rank{\isacharunderscore}{\kern0pt}closed{\isacharcolon}{\kern0pt}\ {\isachardoublequoteopen}M{\isacharparenleft}{\kern0pt}a{\isacharparenright}{\kern0pt}\ {\isasymLongrightarrow}\ M{\isacharparenleft}{\kern0pt}rank{\isacharparenleft}{\kern0pt}a{\isacharparenright}{\kern0pt}{\isacharparenright}{\kern0pt}{\isachardoublequoteclose}\isanewline
%
\isadelimproof
\ \ %
\endisadelimproof
%
\isatagproof
\isacommand{unfolding}\isamarkupfalse%
\ rank{\isacharunderscore}{\kern0pt}trancl\ \isanewline
\ \ \isacommand{using}\isamarkupfalse%
\ relation{\isadigit{2}}{\isacharunderscore}{\kern0pt}Hrank\ is{\isacharunderscore}{\kern0pt}Hrank{\isacharunderscore}{\kern0pt}closed\ is{\isacharunderscore}{\kern0pt}Hrank{\isacharunderscore}{\kern0pt}replacement\ \isanewline
\ \ \ \ \ \ \ \ wf{\isacharunderscore}{\kern0pt}rrank\ relation{\isacharunderscore}{\kern0pt}rrank\ trans{\isacharunderscore}{\kern0pt}rrank\ rrank{\isacharunderscore}{\kern0pt}in{\isacharunderscore}{\kern0pt}M\ \isanewline
\ \ \ \ \ \ \ \ \ trans{\isacharunderscore}{\kern0pt}wfrec{\isacharunderscore}{\kern0pt}closed{\isacharbrackleft}{\kern0pt}of\ {\isachardoublequoteopen}rrank{\isacharparenleft}{\kern0pt}a{\isacharparenright}{\kern0pt}{\isachardoublequoteclose}\ a\ {\isachardoublequoteopen}is{\isacharunderscore}{\kern0pt}Hrank{\isacharparenleft}{\kern0pt}M{\isacharparenright}{\kern0pt}{\isachardoublequoteclose}{\isacharbrackright}{\kern0pt}\ \isacommand{by}\isamarkupfalse%
\ simp%
\endisatagproof
{\isafoldproof}%
%
\isadelimproof
\isanewline
%
\endisadelimproof
\isanewline
\isanewline
\isacommand{lemma}\isamarkupfalse%
\ M{\isacharunderscore}{\kern0pt}into{\isacharunderscore}{\kern0pt}Vset{\isacharcolon}{\kern0pt}\isanewline
\ \ \isakeyword{assumes}\ {\isachardoublequoteopen}M{\isacharparenleft}{\kern0pt}a{\isacharparenright}{\kern0pt}{\isachardoublequoteclose}\isanewline
\ \ \isakeyword{shows}\ {\isachardoublequoteopen}{\isasymexists}i{\isacharbrackleft}{\kern0pt}M{\isacharbrackright}{\kern0pt}{\isachardot}{\kern0pt}\ {\isasymexists}V{\isacharbrackleft}{\kern0pt}M{\isacharbrackright}{\kern0pt}{\isachardot}{\kern0pt}\ ordinal{\isacharparenleft}{\kern0pt}M{\isacharcomma}{\kern0pt}i{\isacharparenright}{\kern0pt}\ {\isasymand}\ is{\isacharunderscore}{\kern0pt}Vfrom{\isacharparenleft}{\kern0pt}M{\isacharcomma}{\kern0pt}{\isadigit{0}}{\isacharcomma}{\kern0pt}i{\isacharcomma}{\kern0pt}V{\isacharparenright}{\kern0pt}\ {\isasymand}\ a{\isasymin}V{\isachardoublequoteclose}\isanewline
%
\isadelimproof
%
\endisadelimproof
%
\isatagproof
\isacommand{proof}\isamarkupfalse%
\ {\isacharminus}{\kern0pt}\isanewline
\ \ \isacommand{let}\isamarkupfalse%
\ {\isacharquery}{\kern0pt}i{\isacharequal}{\kern0pt}{\isachardoublequoteopen}succ{\isacharparenleft}{\kern0pt}rank{\isacharparenleft}{\kern0pt}a{\isacharparenright}{\kern0pt}{\isacharparenright}{\kern0pt}{\isachardoublequoteclose}\isanewline
\ \ \isacommand{from}\isamarkupfalse%
\ assms\isanewline
\ \ \isacommand{have}\isamarkupfalse%
\ {\isachardoublequoteopen}a{\isasymin}{\isacharbraceleft}{\kern0pt}x{\isasymin}Vfrom{\isacharparenleft}{\kern0pt}{\isadigit{0}}{\isacharcomma}{\kern0pt}{\isacharquery}{\kern0pt}i{\isacharparenright}{\kern0pt}{\isachardot}{\kern0pt}\ M{\isacharparenleft}{\kern0pt}x{\isacharparenright}{\kern0pt}{\isacharbraceright}{\kern0pt}{\isachardoublequoteclose}\ {\isacharparenleft}{\kern0pt}\isakeyword{is}\ {\isachardoublequoteopen}a{\isasymin}{\isacharquery}{\kern0pt}V{\isachardoublequoteclose}{\isacharparenright}{\kern0pt}\isanewline
\ \ \ \ \isacommand{using}\isamarkupfalse%
\ Vset{\isacharunderscore}{\kern0pt}Ord{\isacharunderscore}{\kern0pt}rank{\isacharunderscore}{\kern0pt}iff\ \isacommand{by}\isamarkupfalse%
\ simp\isanewline
\ \ \isacommand{moreover}\isamarkupfalse%
\ \isacommand{from}\isamarkupfalse%
\ assms\isanewline
\ \ \isacommand{have}\isamarkupfalse%
\ {\isachardoublequoteopen}M{\isacharparenleft}{\kern0pt}{\isacharquery}{\kern0pt}i{\isacharparenright}{\kern0pt}{\isachardoublequoteclose}\isanewline
\ \ \ \ \isacommand{using}\isamarkupfalse%
\ rank{\isacharunderscore}{\kern0pt}closed\ \isacommand{by}\isamarkupfalse%
\ simp\isanewline
\ \ \isacommand{moreover}\isamarkupfalse%
\ \isanewline
\ \ \isacommand{note}\isamarkupfalse%
\ {\isacartoucheopen}M{\isacharparenleft}{\kern0pt}a{\isacharparenright}{\kern0pt}{\isacartoucheclose}\isanewline
\ \ \isacommand{moreover}\isamarkupfalse%
\ \isacommand{from}\isamarkupfalse%
\ calculation\isanewline
\ \ \isacommand{have}\isamarkupfalse%
\ {\isachardoublequoteopen}M{\isacharparenleft}{\kern0pt}{\isacharquery}{\kern0pt}V{\isacharparenright}{\kern0pt}{\isachardoublequoteclose}\isanewline
\ \ \ \ \isacommand{using}\isamarkupfalse%
\ Vfrom{\isacharunderscore}{\kern0pt}closed\ \isacommand{by}\isamarkupfalse%
\ simp\isanewline
\ \ \isacommand{moreover}\isamarkupfalse%
\ \isacommand{from}\isamarkupfalse%
\ calculation\isanewline
\ \ \isacommand{have}\isamarkupfalse%
\ {\isachardoublequoteopen}ordinal{\isacharparenleft}{\kern0pt}M{\isacharcomma}{\kern0pt}{\isacharquery}{\kern0pt}i{\isacharparenright}{\kern0pt}\ {\isasymand}\ is{\isacharunderscore}{\kern0pt}Vfrom{\isacharparenleft}{\kern0pt}M{\isacharcomma}{\kern0pt}\ {\isadigit{0}}{\isacharcomma}{\kern0pt}\ {\isacharquery}{\kern0pt}i{\isacharcomma}{\kern0pt}\ {\isacharquery}{\kern0pt}V{\isacharparenright}{\kern0pt}\ {\isasymand}\ a\ {\isasymin}\ {\isacharquery}{\kern0pt}V{\isachardoublequoteclose}\isanewline
\ \ \ \ \isacommand{using}\isamarkupfalse%
\ Ord{\isacharunderscore}{\kern0pt}rank\ Vfrom{\isacharunderscore}{\kern0pt}abs\ \isacommand{by}\isamarkupfalse%
\ simp\ \isanewline
\ \ \isacommand{ultimately}\isamarkupfalse%
\isanewline
\ \ \isacommand{show}\isamarkupfalse%
\ {\isacharquery}{\kern0pt}thesis\ \isacommand{by}\isamarkupfalse%
\ blast\isanewline
\isacommand{qed}\isamarkupfalse%
%
\endisatagproof
{\isafoldproof}%
%
\isadelimproof
\isanewline
%
\endisadelimproof
\isanewline
\isacommand{end}\isamarkupfalse%
\isanewline
%
\isadelimtheory
%
\endisadelimtheory
%
\isatagtheory
\isacommand{end}\isamarkupfalse%
%
\endisatagtheory
{\isafoldtheory}%
%
\isadelimtheory
%
\endisadelimtheory
%
\end{isabellebody}%
\endinput
%:%file=~/source/repos/ZF-notAC/code/Forcing/Relative_Univ.thy%:%
%:%11=1%:%
%:%27=2%:%
%:%28=2%:%
%:%29=3%:%
%:%30=4%:%
%:%31=5%:%
%:%32=6%:%
%:%33=7%:%
%:%34=8%:%
%:%39=8%:%
%:%42=9%:%
%:%43=10%:%
%:%44=10%:%
%:%45=11%:%
%:%46=12%:%
%:%47=13%:%
%:%48=14%:%
%:%51=15%:%
%:%55=15%:%
%:%56=15%:%
%:%57=15%:%
%:%62=15%:%
%:%65=16%:%
%:%66=17%:%
%:%67=17%:%
%:%68=18%:%
%:%69=19%:%
%:%70=20%:%
%:%71=21%:%
%:%74=22%:%
%:%78=22%:%
%:%79=22%:%
%:%80=22%:%
%:%81=23%:%
%:%82=23%:%
%:%87=23%:%
%:%90=24%:%
%:%91=25%:%
%:%92=25%:%
%:%93=26%:%
%:%96=27%:%
%:%100=27%:%
%:%101=27%:%
%:%107=27%:%
%:%110=28%:%
%:%111=29%:%
%:%112=31%:%
%:%113=32%:%
%:%114=32%:%
%:%115=33%:%
%:%116=34%:%
%:%117=35%:%
%:%118=36%:%
%:%119=37%:%
%:%120=37%:%
%:%121=38%:%
%:%122=39%:%
%:%123=40%:%
%:%124=41%:%
%:%125=42%:%
%:%126=42%:%
%:%129=43%:%
%:%133=43%:%
%:%134=43%:%
%:%135=43%:%
%:%140=43%:%
%:%143=44%:%
%:%144=45%:%
%:%145=46%:%
%:%146=46%:%
%:%147=47%:%
%:%148=48%:%
%:%149=49%:%
%:%150=50%:%
%:%151=51%:%
%:%152=52%:%
%:%153=52%:%
%:%154=53%:%
%:%155=54%:%
%:%156=55%:%
%:%157=56%:%
%:%158=56%:%
%:%159=57%:%
%:%160=58%:%
%:%167=61%:%
%:%177=63%:%
%:%178=63%:%
%:%179=64%:%
%:%180=65%:%
%:%181=66%:%
%:%182=67%:%
%:%183=68%:%
%:%186=69%:%
%:%190=69%:%
%:%191=69%:%
%:%192=70%:%
%:%193=70%:%
%:%194=71%:%
%:%195=71%:%
%:%200=71%:%
%:%203=72%:%
%:%204=73%:%
%:%205=73%:%
%:%206=74%:%
%:%207=75%:%
%:%208=76%:%
%:%209=77%:%
%:%210=78%:%
%:%213=79%:%
%:%217=79%:%
%:%218=79%:%
%:%219=80%:%
%:%220=81%:%
%:%221=81%:%
%:%226=81%:%
%:%229=82%:%
%:%230=83%:%
%:%231=84%:%
%:%232=84%:%
%:%233=85%:%
%:%234=86%:%
%:%235=87%:%
%:%236=88%:%
%:%237=88%:%
%:%238=89%:%
%:%239=90%:%
%:%240=91%:%
%:%241=92%:%
%:%242=92%:%
%:%243=93%:%
%:%244=94%:%
%:%245=95%:%
%:%246=96%:%
%:%247=96%:%
%:%248=97%:%
%:%249=98%:%
%:%250=99%:%
%:%251=100%:%
%:%252=100%:%
%:%255=101%:%
%:%259=101%:%
%:%260=101%:%
%:%261=101%:%
%:%267=101%:%
%:%270=102%:%
%:%271=103%:%
%:%272=103%:%
%:%275=104%:%
%:%279=104%:%
%:%280=104%:%
%:%281=104%:%
%:%287=104%:%
%:%290=105%:%
%:%291=106%:%
%:%292=106%:%
%:%295=107%:%
%:%299=107%:%
%:%300=107%:%
%:%301=107%:%
%:%307=107%:%
%:%310=108%:%
%:%311=109%:%
%:%312=109%:%
%:%315=110%:%
%:%319=110%:%
%:%320=110%:%
%:%321=110%:%
%:%335=113%:%
%:%345=115%:%
%:%346=115%:%
%:%347=116%:%
%:%348=117%:%
%:%349=118%:%
%:%350=119%:%
%:%351=120%:%
%:%352=121%:%
%:%353=122%:%
%:%354=123%:%
%:%355=124%:%
%:%356=125%:%
%:%357=126%:%
%:%358=126%:%
%:%361=127%:%
%:%365=127%:%
%:%366=127%:%
%:%367=127%:%
%:%372=127%:%
%:%375=128%:%
%:%376=129%:%
%:%377=129%:%
%:%379=131%:%
%:%382=132%:%
%:%386=132%:%
%:%387=132%:%
%:%388=132%:%
%:%389=132%:%
%:%394=132%:%
%:%397=133%:%
%:%398=134%:%
%:%399=134%:%
%:%400=135%:%
%:%401=136%:%
%:%402=137%:%
%:%403=138%:%
%:%410=139%:%
%:%411=139%:%
%:%412=140%:%
%:%413=140%:%
%:%414=141%:%
%:%415=141%:%
%:%416=141%:%
%:%417=141%:%
%:%418=142%:%
%:%419=142%:%
%:%420=143%:%
%:%421=143%:%
%:%422=144%:%
%:%423=144%:%
%:%424=144%:%
%:%425=144%:%
%:%426=145%:%
%:%427=145%:%
%:%428=146%:%
%:%429=146%:%
%:%430=146%:%
%:%431=146%:%
%:%432=147%:%
%:%438=147%:%
%:%441=148%:%
%:%442=149%:%
%:%443=149%:%
%:%444=150%:%
%:%447=151%:%
%:%451=151%:%
%:%452=151%:%
%:%453=151%:%
%:%454=151%:%
%:%459=151%:%
%:%462=152%:%
%:%463=153%:%
%:%464=153%:%
%:%465=154%:%
%:%466=155%:%
%:%467=156%:%
%:%468=157%:%
%:%475=158%:%
%:%476=158%:%
%:%477=159%:%
%:%478=159%:%
%:%479=160%:%
%:%480=160%:%
%:%481=160%:%
%:%482=161%:%
%:%483=161%:%
%:%484=162%:%
%:%485=162%:%
%:%486=162%:%
%:%487=163%:%
%:%488=163%:%
%:%489=164%:%
%:%490=164%:%
%:%491=164%:%
%:%492=164%:%
%:%493=165%:%
%:%499=165%:%
%:%502=166%:%
%:%503=167%:%
%:%504=167%:%
%:%505=168%:%
%:%506=169%:%
%:%507=170%:%
%:%508=171%:%
%:%515=172%:%
%:%516=172%:%
%:%517=173%:%
%:%518=173%:%
%:%519=174%:%
%:%520=174%:%
%:%521=174%:%
%:%522=174%:%
%:%523=175%:%
%:%524=175%:%
%:%525=176%:%
%:%526=176%:%
%:%527=177%:%
%:%528=177%:%
%:%529=177%:%
%:%530=177%:%
%:%531=178%:%
%:%532=178%:%
%:%533=179%:%
%:%534=179%:%
%:%535=179%:%
%:%536=179%:%
%:%537=180%:%
%:%543=180%:%
%:%546=181%:%
%:%547=182%:%
%:%548=182%:%
%:%549=183%:%
%:%550=184%:%
%:%551=185%:%
%:%552=186%:%
%:%559=187%:%
%:%560=187%:%
%:%561=188%:%
%:%562=188%:%
%:%563=189%:%
%:%564=189%:%
%:%565=189%:%
%:%566=190%:%
%:%567=190%:%
%:%568=191%:%
%:%569=191%:%
%:%570=192%:%
%:%571=192%:%
%:%572=192%:%
%:%573=193%:%
%:%574=193%:%
%:%575=193%:%
%:%576=193%:%
%:%577=193%:%
%:%578=194%:%
%:%584=194%:%
%:%587=195%:%
%:%588=196%:%
%:%589=196%:%
%:%592=197%:%
%:%596=197%:%
%:%597=197%:%
%:%598=198%:%
%:%599=198%:%
%:%600=199%:%
%:%601=199%:%
%:%606=199%:%
%:%609=200%:%
%:%610=201%:%
%:%611=201%:%
%:%612=202%:%
%:%615=203%:%
%:%619=203%:%
%:%620=203%:%
%:%621=203%:%
%:%622=203%:%
%:%627=203%:%
%:%630=204%:%
%:%631=205%:%
%:%632=205%:%
%:%633=206%:%
%:%634=207%:%
%:%641=208%:%
%:%642=208%:%
%:%643=209%:%
%:%644=209%:%
%:%645=210%:%
%:%646=210%:%
%:%647=211%:%
%:%648=211%:%
%:%649=211%:%
%:%650=212%:%
%:%651=212%:%
%:%652=213%:%
%:%653=213%:%
%:%654=214%:%
%:%655=214%:%
%:%656=215%:%
%:%657=215%:%
%:%658=216%:%
%:%659=216%:%
%:%660=217%:%
%:%661=217%:%
%:%662=218%:%
%:%663=218%:%
%:%664=219%:%
%:%665=219%:%
%:%666=220%:%
%:%667=220%:%
%:%668=221%:%
%:%669=221%:%
%:%670=222%:%
%:%671=222%:%
%:%672=223%:%
%:%673=223%:%
%:%674=224%:%
%:%675=224%:%
%:%676=225%:%
%:%677=225%:%
%:%678=226%:%
%:%679=226%:%
%:%680=227%:%
%:%681=227%:%
%:%682=227%:%
%:%683=228%:%
%:%684=228%:%
%:%685=229%:%
%:%686=229%:%
%:%687=229%:%
%:%688=230%:%
%:%689=230%:%
%:%690=230%:%
%:%691=231%:%
%:%692=231%:%
%:%693=232%:%
%:%694=232%:%
%:%695=233%:%
%:%696=233%:%
%:%697=234%:%
%:%698=234%:%
%:%699=234%:%
%:%700=235%:%
%:%701=235%:%
%:%702=236%:%
%:%703=236%:%
%:%704=236%:%
%:%705=237%:%
%:%711=237%:%
%:%714=238%:%
%:%715=239%:%
%:%716=239%:%
%:%719=240%:%
%:%723=240%:%
%:%724=240%:%
%:%725=241%:%
%:%726=241%:%
%:%727=242%:%
%:%728=242%:%
%:%733=242%:%
%:%736=243%:%
%:%737=244%:%
%:%738=244%:%
%:%741=245%:%
%:%745=245%:%
%:%746=245%:%
%:%747=246%:%
%:%748=246%:%
%:%749=247%:%
%:%750=247%:%
%:%755=247%:%
%:%758=248%:%
%:%759=249%:%
%:%760=249%:%
%:%763=250%:%
%:%767=250%:%
%:%768=250%:%
%:%769=250%:%
%:%770=250%:%
%:%775=250%:%
%:%778=251%:%
%:%779=252%:%
%:%780=252%:%
%:%783=253%:%
%:%787=253%:%
%:%788=253%:%
%:%789=253%:%
%:%790=253%:%
%:%795=253%:%
%:%798=254%:%
%:%799=255%:%
%:%800=255%:%
%:%801=256%:%
%:%804=257%:%
%:%808=257%:%
%:%809=257%:%
%:%810=258%:%
%:%811=258%:%
%:%812=258%:%
%:%817=258%:%
%:%820=259%:%
%:%821=260%:%
%:%822=260%:%
%:%829=261%:%
%:%830=261%:%
%:%831=262%:%
%:%832=262%:%
%:%833=263%:%
%:%834=264%:%
%:%835=264%:%
%:%836=264%:%
%:%837=265%:%
%:%838=265%:%
%:%839=266%:%
%:%840=266%:%
%:%841=267%:%
%:%842=267%:%
%:%843=267%:%
%:%844=268%:%
%:%845=268%:%
%:%846=269%:%
%:%847=269%:%
%:%848=270%:%
%:%849=270%:%
%:%850=270%:%
%:%851=271%:%
%:%852=271%:%
%:%853=272%:%
%:%854=272%:%
%:%855=273%:%
%:%856=273%:%
%:%857=273%:%
%:%858=273%:%
%:%859=274%:%
%:%860=274%:%
%:%861=275%:%
%:%862=275%:%
%:%863=275%:%
%:%864=275%:%
%:%865=276%:%
%:%871=276%:%
%:%874=277%:%
%:%875=278%:%
%:%876=278%:%
%:%879=279%:%
%:%883=279%:%
%:%884=279%:%
%:%885=279%:%
%:%890=279%:%
%:%893=280%:%
%:%894=281%:%
%:%895=282%:%
%:%896=282%:%
%:%897=283%:%
%:%900=284%:%
%:%904=284%:%
%:%905=284%:%
%:%906=284%:%
%:%911=284%:%
%:%914=285%:%
%:%915=286%:%
%:%916=286%:%
%:%919=287%:%
%:%923=287%:%
%:%924=287%:%
%:%925=287%:%
%:%930=287%:%
%:%933=288%:%
%:%934=289%:%
%:%935=289%:%
%:%936=290%:%
%:%937=291%:%
%:%938=292%:%
%:%939=293%:%
%:%946=294%:%
%:%947=294%:%
%:%948=295%:%
%:%949=295%:%
%:%950=296%:%
%:%951=296%:%
%:%952=296%:%
%:%953=296%:%
%:%954=297%:%
%:%955=297%:%
%:%956=298%:%
%:%957=298%:%
%:%958=298%:%
%:%959=298%:%
%:%960=299%:%
%:%966=299%:%
%:%969=300%:%
%:%970=301%:%
%:%971=301%:%
%:%972=302%:%
%:%973=303%:%
%:%974=304%:%
%:%975=305%:%
%:%982=306%:%
%:%983=306%:%
%:%984=307%:%
%:%985=307%:%
%:%986=308%:%
%:%987=308%:%
%:%988=308%:%
%:%989=309%:%
%:%990=309%:%
%:%991=310%:%
%:%992=310%:%
%:%993=310%:%
%:%994=311%:%
%:%995=311%:%
%:%996=312%:%
%:%997=312%:%
%:%998=312%:%
%:%999=312%:%
%:%1000=313%:%
%:%1006=313%:%
%:%1009=314%:%
%:%1010=315%:%
%:%1011=315%:%
%:%1012=316%:%
%:%1013=317%:%
%:%1014=318%:%
%:%1015=319%:%
%:%1022=320%:%
%:%1023=320%:%
%:%1024=321%:%
%:%1025=321%:%
%:%1026=322%:%
%:%1027=322%:%
%:%1028=322%:%
%:%1029=322%:%
%:%1030=323%:%
%:%1031=323%:%
%:%1032=324%:%
%:%1033=324%:%
%:%1034=324%:%
%:%1035=324%:%
%:%1036=325%:%
%:%1042=325%:%
%:%1045=326%:%
%:%1046=327%:%
%:%1047=327%:%
%:%1048=328%:%
%:%1051=329%:%
%:%1055=329%:%
%:%1056=329%:%
%:%1057=330%:%
%:%1058=330%:%
%:%1059=330%:%
%:%1064=330%:%
%:%1067=331%:%
%:%1068=332%:%
%:%1069=333%:%
%:%1070=333%:%
%:%1071=334%:%
%:%1072=335%:%
%:%1073=336%:%
%:%1074=337%:%
%:%1081=338%:%
%:%1082=338%:%
%:%1083=339%:%
%:%1084=339%:%
%:%1085=340%:%
%:%1086=340%:%
%:%1087=340%:%
%:%1088=341%:%
%:%1089=341%:%
%:%1090=342%:%
%:%1091=342%:%
%:%1092=343%:%
%:%1093=343%:%
%:%1094=343%:%
%:%1095=344%:%
%:%1096=344%:%
%:%1097=344%:%
%:%1098=344%:%
%:%1099=344%:%
%:%1100=345%:%
%:%1106=345%:%
%:%1109=346%:%
%:%1110=347%:%
%:%1111=347%:%
%:%1112=348%:%
%:%1115=349%:%
%:%1119=349%:%
%:%1120=349%:%
%:%1121=349%:%
%:%1122=349%:%
%:%1127=349%:%
%:%1130=350%:%
%:%1131=351%:%
%:%1132=351%:%
%:%1135=352%:%
%:%1139=352%:%
%:%1140=352%:%
%:%1141=353%:%
%:%1142=353%:%
%:%1143=354%:%
%:%1144=355%:%
%:%1145=355%:%
%:%1150=355%:%
%:%1153=356%:%
%:%1154=357%:%
%:%1155=358%:%
%:%1156=358%:%
%:%1157=359%:%
%:%1158=360%:%
%:%1165=361%:%
%:%1166=361%:%
%:%1167=362%:%
%:%1168=362%:%
%:%1169=363%:%
%:%1170=363%:%
%:%1171=364%:%
%:%1172=364%:%
%:%1173=365%:%
%:%1174=365%:%
%:%1175=365%:%
%:%1176=366%:%
%:%1177=366%:%
%:%1178=366%:%
%:%1179=367%:%
%:%1180=367%:%
%:%1181=368%:%
%:%1182=368%:%
%:%1183=368%:%
%:%1184=369%:%
%:%1185=369%:%
%:%1186=370%:%
%:%1187=370%:%
%:%1188=371%:%
%:%1189=371%:%
%:%1190=371%:%
%:%1191=372%:%
%:%1192=372%:%
%:%1193=373%:%
%:%1194=373%:%
%:%1195=373%:%
%:%1196=374%:%
%:%1197=374%:%
%:%1198=374%:%
%:%1199=375%:%
%:%1200=375%:%
%:%1201=376%:%
%:%1202=376%:%
%:%1203=376%:%
%:%1204=377%:%
%:%1205=377%:%
%:%1206=378%:%
%:%1207=378%:%
%:%1208=378%:%
%:%1209=379%:%
%:%1215=379%:%
%:%1218=380%:%
%:%1219=381%:%
%:%1220=381%:%
%:%1227=382%:%

%
\begin{isabellebody}%
\setisabellecontext{Utils}%
%
\isadelimtheory
%
\endisadelimtheory
%
\isatagtheory
\isacommand{theory}\isamarkupfalse%
\ Utils\isanewline
\ \ \isakeyword{imports}\ {\isachardoublequoteopen}ZF{\isacharminus}{\kern0pt}Constructible{\isachardot}{\kern0pt}Formula{\isachardoublequoteclose}\isanewline
\isakeyword{begin}%
\endisatagtheory
{\isafoldtheory}%
%
\isadelimtheory
\isanewline
%
\endisadelimtheory
%
\isadelimML
\isanewline
%
\endisadelimML
%
\isatagML
\isacommand{ML{\isacharunderscore}{\kern0pt}file}\isamarkupfalse%
\ {\isacartoucheopen}utils{\isachardot}{\kern0pt}ML{\isacartoucheclose}%
\endisatagML
{\isafoldML}%
%
\isadelimML
\isanewline
%
\endisadelimML
%
\isadelimtheory
\isanewline
%
\endisadelimtheory
%
\isatagtheory
\isacommand{end}\isamarkupfalse%
%
\endisatagtheory
{\isafoldtheory}%
%
\isadelimtheory
%
\endisadelimtheory
%
\end{isabellebody}%
\endinput
%:%file=~/source/repos/ZF-notAC/code/Forcing/Utils.thy%:%
%:%10=1%:%
%:%11=1%:%
%:%12=2%:%
%:%13=3%:%
%:%18=3%:%
%:%23=4%:%
%:%28=5%:%
%:%29=5%:%
%:%34=5%:%
%:%39=6%:%
%:%44=7%:%

%
\begin{isabellebody}%
\setisabellecontext{Synthetic{\isacharunderscore}{\kern0pt}Definition}%
%
\isadelimdocument
%
\endisadelimdocument
%
\isatagdocument
%
\isamarkupsection{Automatic synthesis of formulas%
}
\isamarkuptrue%
%
\endisatagdocument
{\isafolddocument}%
%
\isadelimdocument
%
\endisadelimdocument
%
\isadelimtheory
%
\endisadelimtheory
%
\isatagtheory
\isacommand{theory}\isamarkupfalse%
\ Synthetic{\isacharunderscore}{\kern0pt}Definition\isanewline
\ \ \isakeyword{imports}\ Utils\isanewline
\ \ \isakeyword{keywords}\ {\isachardoublequoteopen}synthesize{\isachardoublequoteclose}\ {\isacharcolon}{\kern0pt}{\isacharcolon}{\kern0pt}\ thy{\isacharunderscore}{\kern0pt}decl\ {\isacharpercent}{\kern0pt}\ {\isachardoublequoteopen}ML{\isachardoublequoteclose}\isanewline
\ \ \ \ \isakeyword{and}\ {\isachardoublequoteopen}synthesize{\isacharunderscore}{\kern0pt}notc{\isachardoublequoteclose}\ {\isacharcolon}{\kern0pt}{\isacharcolon}{\kern0pt}\ thy{\isacharunderscore}{\kern0pt}decl\ {\isacharpercent}{\kern0pt}\ {\isachardoublequoteopen}ML{\isachardoublequoteclose}\isanewline
\ \ \ \ \isakeyword{and}\ {\isachardoublequoteopen}from{\isacharunderscore}{\kern0pt}schematic{\isachardoublequoteclose}\isanewline
\isakeyword{begin}%
\endisatagtheory
{\isafoldtheory}%
%
\isadelimtheory
\isanewline
%
\endisadelimtheory
%
\isadelimML
\isanewline
%
\endisadelimML
%
\isatagML
\isacommand{ML}\isamarkupfalse%
{\isacartoucheopen}\isanewline
val\ {\isachardollar}{\kern0pt}{\isacharbackquote}{\kern0pt}\ {\isacharequal}{\kern0pt}\ curry\ {\isacharparenleft}{\kern0pt}{\isacharparenleft}{\kern0pt}op\ {\isachardollar}{\kern0pt}{\isacharparenright}{\kern0pt}\ o\ swap{\isacharparenright}{\kern0pt}\isanewline
infix\ {\isachardollar}{\kern0pt}{\isacharbackquote}{\kern0pt}\isanewline
\isanewline
fun\ pair\ f\ g\ x\ {\isacharequal}{\kern0pt}\ {\isacharparenleft}{\kern0pt}f\ x{\isacharcomma}{\kern0pt}\ g\ x{\isacharparenright}{\kern0pt}\isanewline
\isanewline
fun\ print{\isacharunderscore}{\kern0pt}theorem\ pos\ {\isacharparenleft}{\kern0pt}thms{\isacharcomma}{\kern0pt}\ lthy{\isacharparenright}{\kern0pt}\ {\isacharequal}{\kern0pt}\isanewline
\ \ {\isacharparenleft}{\kern0pt}Proof{\isacharunderscore}{\kern0pt}Display{\isachardot}{\kern0pt}print{\isacharunderscore}{\kern0pt}theorem\ pos\ lthy\ thms{\isacharsemicolon}{\kern0pt}\ lthy{\isacharparenright}{\kern0pt}\isanewline
\isanewline
fun\ prove{\isacharunderscore}{\kern0pt}tc{\isacharunderscore}{\kern0pt}form\ goal\ thms\ ctxt\ {\isacharequal}{\kern0pt}\isanewline
\ \ Goal{\isachardot}{\kern0pt}prove\ ctxt\ {\isacharbrackleft}{\kern0pt}{\isacharbrackright}{\kern0pt}\ {\isacharbrackleft}{\kern0pt}{\isacharbrackright}{\kern0pt}\ goal\isanewline
\ \ \ \ {\isacharparenleft}{\kern0pt}fn\ {\isacharbraceleft}{\kern0pt}context\ {\isacharequal}{\kern0pt}\ ctxt{\isacharprime}{\kern0pt}{\isacharcomma}{\kern0pt}\ {\isachardot}{\kern0pt}{\isachardot}{\kern0pt}{\isachardot}{\kern0pt}{\isacharbraceright}{\kern0pt}\ {\isacharequal}{\kern0pt}{\isachargreater}{\kern0pt}\isanewline
\ \ \ \ \ \ rewrite{\isacharunderscore}{\kern0pt}goal{\isacharunderscore}{\kern0pt}tac\ ctxt{\isacharprime}{\kern0pt}\ thms\ {\isadigit{1}}\isanewline
\ \ \ \ \ \ THEN\ TypeCheck{\isachardot}{\kern0pt}typecheck{\isacharunderscore}{\kern0pt}tac\ ctxt{\isacharprime}{\kern0pt}{\isacharparenright}{\kern0pt}\isanewline
\isanewline
fun\ prove{\isacharunderscore}{\kern0pt}sats\ goal\ thms\ thm{\isacharunderscore}{\kern0pt}auto\ ctxt\ {\isacharequal}{\kern0pt}\isanewline
\ \ Goal{\isachardot}{\kern0pt}prove\ ctxt\ {\isacharbrackleft}{\kern0pt}{\isacharbrackright}{\kern0pt}\ {\isacharbrackleft}{\kern0pt}{\isacharbrackright}{\kern0pt}\ goal\isanewline
\ \ \ \ {\isacharparenleft}{\kern0pt}fn\ {\isacharbraceleft}{\kern0pt}context\ {\isacharequal}{\kern0pt}\ ctxt{\isacharprime}{\kern0pt}{\isacharcomma}{\kern0pt}\ {\isachardot}{\kern0pt}{\isachardot}{\kern0pt}{\isachardot}{\kern0pt}{\isacharbraceright}{\kern0pt}\ {\isacharequal}{\kern0pt}{\isachargreater}{\kern0pt}\isanewline
\ \ \ \ \ \ let\ val\ ctxt{\isacharprime}{\kern0pt}{\isacharprime}{\kern0pt}\ {\isacharequal}{\kern0pt}\ ctxt{\isacharprime}{\kern0pt}\ {\isacharbar}{\kern0pt}{\isachargreater}{\kern0pt}\ Simplifier{\isachardot}{\kern0pt}add{\isacharunderscore}{\kern0pt}simp\ {\isacharparenleft}{\kern0pt}thm{\isacharunderscore}{\kern0pt}auto\ {\isacharbar}{\kern0pt}{\isachargreater}{\kern0pt}\ hd{\isacharparenright}{\kern0pt}\ in\isanewline
\ \ \ \ \ \ \ \ rewrite{\isacharunderscore}{\kern0pt}goal{\isacharunderscore}{\kern0pt}tac\ ctxt{\isacharprime}{\kern0pt}{\isacharprime}{\kern0pt}\ thms\ {\isadigit{1}}\isanewline
\ \ \ \ \ \ \ \ THEN\ PARALLEL{\isacharunderscore}{\kern0pt}ALLGOALS\ {\isacharparenleft}{\kern0pt}asm{\isacharunderscore}{\kern0pt}simp{\isacharunderscore}{\kern0pt}tac\ ctxt{\isacharprime}{\kern0pt}{\isacharprime}{\kern0pt}{\isacharparenright}{\kern0pt}\isanewline
\ \ \ \ \ \ \ \ THEN\ TypeCheck{\isachardot}{\kern0pt}typecheck{\isacharunderscore}{\kern0pt}tac\ ctxt{\isacharprime}{\kern0pt}{\isacharprime}{\kern0pt}\isanewline
\ \ \ \ \ \ end{\isacharparenright}{\kern0pt}\isanewline
\isanewline
fun\ is{\isacharunderscore}{\kern0pt}mem\ \isactrlConstUNDERSCORE {\isasymopen}mem\ for\ {\isacharunderscore}{\kern0pt}\ {\isacharunderscore}{\kern0pt}{\isasymclose}\ {\isacharequal}{\kern0pt}\ true\isanewline
\ \ {\isacharbar}{\kern0pt}\ is{\isacharunderscore}{\kern0pt}mem\ {\isacharunderscore}{\kern0pt}\ {\isacharequal}{\kern0pt}\ false\isanewline
\isanewline
fun\ synth{\isacharunderscore}{\kern0pt}thm{\isacharunderscore}{\kern0pt}sats\ def{\isacharunderscore}{\kern0pt}name\ term\ lhs\ set\ env\ hyps\ vars\ vs\ pos\ thm{\isacharunderscore}{\kern0pt}auto\ lthy\ {\isacharequal}{\kern0pt}\isanewline
let\ val\ {\isacharparenleft}{\kern0pt}{\isacharunderscore}{\kern0pt}{\isacharcomma}{\kern0pt}tm{\isacharcomma}{\kern0pt}ctxt{\isadigit{1}}{\isacharparenright}{\kern0pt}\ {\isacharequal}{\kern0pt}\ Utils{\isachardot}{\kern0pt}thm{\isacharunderscore}{\kern0pt}concl{\isacharunderscore}{\kern0pt}tm\ lthy\ term\isanewline
\ \ \ \ val\ {\isacharparenleft}{\kern0pt}thm{\isacharunderscore}{\kern0pt}refs{\isacharcomma}{\kern0pt}ctxt{\isadigit{2}}{\isacharparenright}{\kern0pt}\ {\isacharequal}{\kern0pt}\ Variable{\isachardot}{\kern0pt}import\ true\ {\isacharbrackleft}{\kern0pt}Proof{\isacharunderscore}{\kern0pt}Context{\isachardot}{\kern0pt}get{\isacharunderscore}{\kern0pt}thm\ lthy\ term{\isacharbrackright}{\kern0pt}\ ctxt{\isadigit{1}}\ {\isacharbar}{\kern0pt}{\isachargreater}{\kern0pt}{\isachargreater}{\kern0pt}\ {\isacharhash}{\kern0pt}{\isadigit{2}}\isanewline
\ \ \ \ val\ vs{\isacharprime}{\kern0pt}\ {\isacharequal}{\kern0pt}\ map\ {\isacharparenleft}{\kern0pt}Thm{\isachardot}{\kern0pt}term{\isacharunderscore}{\kern0pt}of\ o\ {\isacharhash}{\kern0pt}{\isadigit{2}}{\isacharparenright}{\kern0pt}\ vs\isanewline
\ \ \ \ val\ vars{\isacharprime}{\kern0pt}\ {\isacharequal}{\kern0pt}\ map\ {\isacharparenleft}{\kern0pt}Thm{\isachardot}{\kern0pt}term{\isacharunderscore}{\kern0pt}of\ o\ {\isacharhash}{\kern0pt}{\isadigit{2}}{\isacharparenright}{\kern0pt}\ vars\isanewline
\ \ \ \ val\ r{\isacharunderscore}{\kern0pt}tm\ {\isacharequal}{\kern0pt}\ tm\ {\isacharbar}{\kern0pt}{\isachargreater}{\kern0pt}\ Utils{\isachardot}{\kern0pt}dest{\isacharunderscore}{\kern0pt}lhs{\isacharunderscore}{\kern0pt}def\ {\isacharbar}{\kern0pt}{\isachargreater}{\kern0pt}\ fold\ {\isacharparenleft}{\kern0pt}op\ {\isachardollar}{\kern0pt}{\isacharbackquote}{\kern0pt}{\isacharparenright}{\kern0pt}\ vs{\isacharprime}{\kern0pt}\isanewline
\ \ \ \ val\ sats\ {\isacharequal}{\kern0pt}\ \isactrlConst {\isasymopen}apply\ for\ \isactrlConst {\isasymopen}satisfies\ for\ set\ r{\isacharunderscore}{\kern0pt}tm{\isasymclose}\ env{\isasymclose}\isanewline
\ \ \ \ val\ rhs\ {\isacharequal}{\kern0pt}\ \isactrlConst {\isasymopen}IFOL{\isachardot}{\kern0pt}eq\ \isactrlType {\isasymopen}i{\isasymclose}\ for\ sats\ \isactrlConst {\isasymopen}succ\ for\ \isactrlConst {\isasymopen}zero{\isasymclose}{\isasymclose}{\isasymclose}\isanewline
\ \ \ \ val\ concl\ {\isacharequal}{\kern0pt}\ \isactrlConst {\isasymopen}iff\ for\ lhs\ rhs{\isasymclose}\isanewline
\ \ \ \ val\ g{\isacharunderscore}{\kern0pt}iff\ {\isacharequal}{\kern0pt}\ Logic{\isachardot}{\kern0pt}list{\isacharunderscore}{\kern0pt}implies{\isacharparenleft}{\kern0pt}hyps{\isacharcomma}{\kern0pt}\ Utils{\isachardot}{\kern0pt}tp\ concl{\isacharparenright}{\kern0pt}\isanewline
\ \ \ \ val\ thm\ {\isacharequal}{\kern0pt}\ prove{\isacharunderscore}{\kern0pt}sats\ g{\isacharunderscore}{\kern0pt}iff\ thm{\isacharunderscore}{\kern0pt}refs\ thm{\isacharunderscore}{\kern0pt}auto\ ctxt{\isadigit{2}}\isanewline
\ \ \ \ val\ name\ {\isacharequal}{\kern0pt}\ Binding{\isachardot}{\kern0pt}name\ {\isacharparenleft}{\kern0pt}def{\isacharunderscore}{\kern0pt}name\ {\isacharcircum}{\kern0pt}\ {\isachardoublequote}{\kern0pt}{\isacharunderscore}{\kern0pt}iff{\isacharunderscore}{\kern0pt}sats{\isachardoublequote}{\kern0pt}{\isacharparenright}{\kern0pt}\isanewline
\ \ \ \ val\ thm\ {\isacharequal}{\kern0pt}\ Utils{\isachardot}{\kern0pt}fix{\isacharunderscore}{\kern0pt}vars\ thm\ {\isacharparenleft}{\kern0pt}map\ {\isacharparenleft}{\kern0pt}{\isacharhash}{\kern0pt}{\isadigit{1}}\ o\ dest{\isacharunderscore}{\kern0pt}Free{\isacharparenright}{\kern0pt}\ vars{\isacharprime}{\kern0pt}{\isacharparenright}{\kern0pt}\ lthy\isanewline
\ in\isanewline
\ \ \ Local{\isacharunderscore}{\kern0pt}Theory{\isachardot}{\kern0pt}note\ {\isacharparenleft}{\kern0pt}{\isacharparenleft}{\kern0pt}name{\isacharcomma}{\kern0pt}\ {\isacharbrackleft}{\kern0pt}{\isacharbrackright}{\kern0pt}{\isacharparenright}{\kern0pt}{\isacharcomma}{\kern0pt}\ {\isacharbrackleft}{\kern0pt}thm{\isacharbrackright}{\kern0pt}{\isacharparenright}{\kern0pt}\ lthy\ {\isacharbar}{\kern0pt}{\isachargreater}{\kern0pt}\ print{\isacharunderscore}{\kern0pt}theorem\ pos\isanewline
\ end\isanewline
\isanewline
fun\ synth{\isacharunderscore}{\kern0pt}thm{\isacharunderscore}{\kern0pt}tc\ def{\isacharunderscore}{\kern0pt}name\ term\ hyps\ vars\ pos\ lthy\ {\isacharequal}{\kern0pt}\isanewline
let\ val\ {\isacharparenleft}{\kern0pt}{\isacharunderscore}{\kern0pt}{\isacharcomma}{\kern0pt}tm{\isacharcomma}{\kern0pt}ctxt{\isadigit{1}}{\isacharparenright}{\kern0pt}\ {\isacharequal}{\kern0pt}\ Utils{\isachardot}{\kern0pt}thm{\isacharunderscore}{\kern0pt}concl{\isacharunderscore}{\kern0pt}tm\ lthy\ term\isanewline
\ \ \ \ val\ {\isacharparenleft}{\kern0pt}thm{\isacharunderscore}{\kern0pt}refs{\isacharcomma}{\kern0pt}ctxt{\isadigit{2}}{\isacharparenright}{\kern0pt}\ {\isacharequal}{\kern0pt}\ Variable{\isachardot}{\kern0pt}import\ true\ {\isacharbrackleft}{\kern0pt}Proof{\isacharunderscore}{\kern0pt}Context{\isachardot}{\kern0pt}get{\isacharunderscore}{\kern0pt}thm\ lthy\ term{\isacharbrackright}{\kern0pt}\ ctxt{\isadigit{1}}\isanewline
\ \ \ \ \ \ \ \ \ \ \ \ \ \ \ \ \ \ \ \ {\isacharbar}{\kern0pt}{\isachargreater}{\kern0pt}{\isachargreater}{\kern0pt}\ {\isacharhash}{\kern0pt}{\isadigit{2}}\isanewline
\ \ \ \ val\ vars{\isacharprime}{\kern0pt}\ {\isacharequal}{\kern0pt}\ map\ {\isacharparenleft}{\kern0pt}Thm{\isachardot}{\kern0pt}term{\isacharunderscore}{\kern0pt}of\ o\ {\isacharhash}{\kern0pt}{\isadigit{2}}{\isacharparenright}{\kern0pt}\ vars\isanewline
\ \ \ \ val\ tc{\isacharunderscore}{\kern0pt}attrib\ {\isacharequal}{\kern0pt}\ {\isacharat}{\kern0pt}{\isacharbraceleft}{\kern0pt}attributes\ {\isacharbrackleft}{\kern0pt}TC{\isacharbrackright}{\kern0pt}{\isacharbraceright}{\kern0pt}\isanewline
\ \ \ \ val\ r{\isacharunderscore}{\kern0pt}tm\ {\isacharequal}{\kern0pt}\ tm\ {\isacharbar}{\kern0pt}{\isachargreater}{\kern0pt}\ Utils{\isachardot}{\kern0pt}dest{\isacharunderscore}{\kern0pt}lhs{\isacharunderscore}{\kern0pt}def\ {\isacharbar}{\kern0pt}{\isachargreater}{\kern0pt}\ fold\ {\isacharparenleft}{\kern0pt}op\ {\isachardollar}{\kern0pt}{\isacharbackquote}{\kern0pt}{\isacharparenright}{\kern0pt}\ vars{\isacharprime}{\kern0pt}\isanewline
\ \ \ \ val\ concl\ {\isacharequal}{\kern0pt}\ \isactrlConst {\isasymopen}mem\ for\ r{\isacharunderscore}{\kern0pt}tm\ \isactrlConst {\isasymopen}formula{\isasymclose}{\isasymclose}\isanewline
\ \ \ \ val\ g\ {\isacharequal}{\kern0pt}\ Logic{\isachardot}{\kern0pt}list{\isacharunderscore}{\kern0pt}implies{\isacharparenleft}{\kern0pt}hyps{\isacharcomma}{\kern0pt}\ Utils{\isachardot}{\kern0pt}tp\ concl{\isacharparenright}{\kern0pt}\isanewline
\ \ \ \ val\ thm\ {\isacharequal}{\kern0pt}\ prove{\isacharunderscore}{\kern0pt}tc{\isacharunderscore}{\kern0pt}form\ g\ thm{\isacharunderscore}{\kern0pt}refs\ ctxt{\isadigit{2}}\isanewline
\ \ \ \ val\ name\ {\isacharequal}{\kern0pt}\ Binding{\isachardot}{\kern0pt}name\ {\isacharparenleft}{\kern0pt}def{\isacharunderscore}{\kern0pt}name\ {\isacharcircum}{\kern0pt}\ {\isachardoublequote}{\kern0pt}{\isacharunderscore}{\kern0pt}type{\isachardoublequote}{\kern0pt}{\isacharparenright}{\kern0pt}\isanewline
\ \ \ \ val\ thm\ {\isacharequal}{\kern0pt}\ Utils{\isachardot}{\kern0pt}fix{\isacharunderscore}{\kern0pt}vars\ thm\ {\isacharparenleft}{\kern0pt}map\ {\isacharparenleft}{\kern0pt}{\isacharhash}{\kern0pt}{\isadigit{1}}\ o\ dest{\isacharunderscore}{\kern0pt}Free{\isacharparenright}{\kern0pt}\ vars{\isacharprime}{\kern0pt}{\isacharparenright}{\kern0pt}\ ctxt{\isadigit{2}}\isanewline
\ in\isanewline
\ \ \ Local{\isacharunderscore}{\kern0pt}Theory{\isachardot}{\kern0pt}note\ {\isacharparenleft}{\kern0pt}{\isacharparenleft}{\kern0pt}name{\isacharcomma}{\kern0pt}\ tc{\isacharunderscore}{\kern0pt}attrib{\isacharparenright}{\kern0pt}{\isacharcomma}{\kern0pt}\ {\isacharbrackleft}{\kern0pt}thm{\isacharbrackright}{\kern0pt}{\isacharparenright}{\kern0pt}\ lthy\ {\isacharbar}{\kern0pt}{\isachargreater}{\kern0pt}\ print{\isacharunderscore}{\kern0pt}theorem\ pos\isanewline
\ end\isanewline
\isanewline
\isanewline
fun\ synthetic{\isacharunderscore}{\kern0pt}def\ def{\isacharunderscore}{\kern0pt}name\ thmref\ pos\ tc\ auto\ thy\ {\isacharequal}{\kern0pt}\isanewline
\ \ let\isanewline
\ \ \ \ val\ {\isacharparenleft}{\kern0pt}thm{\isacharunderscore}{\kern0pt}ref{\isacharcomma}{\kern0pt}{\isacharunderscore}{\kern0pt}{\isacharparenright}{\kern0pt}\ {\isacharequal}{\kern0pt}\ thmref\ {\isacharbar}{\kern0pt}{\isachargreater}{\kern0pt}{\isachargreater}{\kern0pt}\ Facts{\isachardot}{\kern0pt}ref{\isacharunderscore}{\kern0pt}name\isanewline
\ \ \ \ val\ thm\ {\isacharequal}{\kern0pt}\ Proof{\isacharunderscore}{\kern0pt}Context{\isachardot}{\kern0pt}get{\isacharunderscore}{\kern0pt}thm\ thy\ thm{\isacharunderscore}{\kern0pt}ref{\isacharsemicolon}{\kern0pt}\isanewline
\ \ \ \ val\ thm{\isacharunderscore}{\kern0pt}vars\ {\isacharequal}{\kern0pt}\ rev\ {\isacharparenleft}{\kern0pt}Term{\isachardot}{\kern0pt}add{\isacharunderscore}{\kern0pt}vars\ {\isacharparenleft}{\kern0pt}Thm{\isachardot}{\kern0pt}full{\isacharunderscore}{\kern0pt}prop{\isacharunderscore}{\kern0pt}of\ thm{\isacharparenright}{\kern0pt}\ {\isacharbrackleft}{\kern0pt}{\isacharbrackright}{\kern0pt}{\isacharparenright}{\kern0pt}{\isacharsemicolon}{\kern0pt}\isanewline
\ \ \ \ val\ {\isacharparenleft}{\kern0pt}{\isacharparenleft}{\kern0pt}{\isacharparenleft}{\kern0pt}{\isacharunderscore}{\kern0pt}{\isacharcomma}{\kern0pt}inst{\isacharparenright}{\kern0pt}{\isacharcomma}{\kern0pt}thm{\isacharunderscore}{\kern0pt}tms{\isacharparenright}{\kern0pt}{\isacharcomma}{\kern0pt}{\isacharunderscore}{\kern0pt}{\isacharparenright}{\kern0pt}\ {\isacharequal}{\kern0pt}\ Variable{\isachardot}{\kern0pt}import\ true\ {\isacharbrackleft}{\kern0pt}thm{\isacharbrackright}{\kern0pt}\ thy\isanewline
\ \ \ \ val\ vars\ {\isacharequal}{\kern0pt}\ map\ {\isacharparenleft}{\kern0pt}fn\ v\ {\isacharequal}{\kern0pt}{\isachargreater}{\kern0pt}\ {\isacharparenleft}{\kern0pt}v{\isacharcomma}{\kern0pt}\ the\ {\isacharparenleft}{\kern0pt}Vars{\isachardot}{\kern0pt}lookup\ inst\ v{\isacharparenright}{\kern0pt}{\isacharparenright}{\kern0pt}{\isacharparenright}{\kern0pt}\ thm{\isacharunderscore}{\kern0pt}vars{\isacharsemicolon}{\kern0pt}\isanewline
\ \ \ \ val\ {\isacharparenleft}{\kern0pt}tm{\isacharcomma}{\kern0pt}hyps{\isacharparenright}{\kern0pt}\ {\isacharequal}{\kern0pt}\ thm{\isacharunderscore}{\kern0pt}tms\ {\isacharbar}{\kern0pt}{\isachargreater}{\kern0pt}\ hd\ {\isacharbar}{\kern0pt}{\isachargreater}{\kern0pt}\ pair\ Thm{\isachardot}{\kern0pt}concl{\isacharunderscore}{\kern0pt}of\ Thm{\isachardot}{\kern0pt}prems{\isacharunderscore}{\kern0pt}of\isanewline
\ \ \ \ val\ {\isacharparenleft}{\kern0pt}lhs{\isacharcomma}{\kern0pt}rhs{\isacharparenright}{\kern0pt}\ {\isacharequal}{\kern0pt}\ tm\ {\isacharbar}{\kern0pt}{\isachargreater}{\kern0pt}\ Utils{\isachardot}{\kern0pt}dest{\isacharunderscore}{\kern0pt}iff{\isacharunderscore}{\kern0pt}tms\ o\ Utils{\isachardot}{\kern0pt}dest{\isacharunderscore}{\kern0pt}trueprop\isanewline
\ \ \ \ val\ {\isacharparenleft}{\kern0pt}{\isacharparenleft}{\kern0pt}set{\isacharcomma}{\kern0pt}t{\isacharparenright}{\kern0pt}{\isacharcomma}{\kern0pt}env{\isacharparenright}{\kern0pt}\ {\isacharequal}{\kern0pt}\ rhs\ {\isacharbar}{\kern0pt}{\isachargreater}{\kern0pt}\ Utils{\isachardot}{\kern0pt}dest{\isacharunderscore}{\kern0pt}sats{\isacharunderscore}{\kern0pt}frm\isanewline
\ \ \ \ fun\ relevant\ ts\ \isactrlConstUNDERSCORE {\isasymopen}mem\ for\ t\ {\isacharunderscore}{\kern0pt}{\isasymclose}\ {\isacharequal}{\kern0pt}\ not\ {\isacharparenleft}{\kern0pt}Term{\isachardot}{\kern0pt}is{\isacharunderscore}{\kern0pt}Free\ t{\isacharparenright}{\kern0pt}\ orelse\isanewline
\ \ \ \ \ \ \ \ \ \ member\ {\isacharparenleft}{\kern0pt}op\ {\isacharequal}{\kern0pt}{\isacharparenright}{\kern0pt}\ ts\ {\isacharparenleft}{\kern0pt}t\ {\isacharbar}{\kern0pt}{\isachargreater}{\kern0pt}\ Term{\isachardot}{\kern0pt}dest{\isacharunderscore}{\kern0pt}Free\ {\isacharbar}{\kern0pt}{\isachargreater}{\kern0pt}\ {\isacharhash}{\kern0pt}{\isadigit{1}}{\isacharparenright}{\kern0pt}\isanewline
\ \ \ \ \ \ {\isacharbar}{\kern0pt}\ relevant\ {\isacharunderscore}{\kern0pt}\ {\isacharunderscore}{\kern0pt}\ {\isacharequal}{\kern0pt}\ false\isanewline
\ \ \ \ val\ t{\isacharunderscore}{\kern0pt}vars\ {\isacharequal}{\kern0pt}\ sort{\isacharunderscore}{\kern0pt}strings\ {\isacharparenleft}{\kern0pt}Term{\isachardot}{\kern0pt}add{\isacharunderscore}{\kern0pt}free{\isacharunderscore}{\kern0pt}names\ t\ {\isacharbrackleft}{\kern0pt}{\isacharbrackright}{\kern0pt}{\isacharparenright}{\kern0pt}\isanewline
\ \ \ \ val\ vs\ {\isacharequal}{\kern0pt}\ filter\ {\isacharparenleft}{\kern0pt}member\ {\isacharparenleft}{\kern0pt}op\ {\isacharequal}{\kern0pt}{\isacharparenright}{\kern0pt}\ t{\isacharunderscore}{\kern0pt}vars\ o\ {\isacharhash}{\kern0pt}{\isadigit{1}}\ o\ {\isacharhash}{\kern0pt}{\isadigit{1}}\ o\ {\isacharhash}{\kern0pt}{\isadigit{1}}{\isacharparenright}{\kern0pt}\ vars\isanewline
\ \ \ \ val\ at\ {\isacharequal}{\kern0pt}\ fold{\isacharunderscore}{\kern0pt}rev\ {\isacharparenleft}{\kern0pt}lambda\ o\ Thm{\isachardot}{\kern0pt}term{\isacharunderscore}{\kern0pt}of\ o\ {\isacharhash}{\kern0pt}{\isadigit{2}}{\isacharparenright}{\kern0pt}\ vs\ t\isanewline
\ \ \ \ val\ hyps{\isacharprime}{\kern0pt}\ {\isacharequal}{\kern0pt}\ filter\ {\isacharparenleft}{\kern0pt}relevant\ t{\isacharunderscore}{\kern0pt}vars\ o\ Utils{\isachardot}{\kern0pt}dest{\isacharunderscore}{\kern0pt}trueprop{\isacharparenright}{\kern0pt}\ hyps\isanewline
\ \ in\isanewline
\ \ \ \ Local{\isacharunderscore}{\kern0pt}Theory{\isachardot}{\kern0pt}define\ {\isacharparenleft}{\kern0pt}{\isacharparenleft}{\kern0pt}Binding{\isachardot}{\kern0pt}name\ def{\isacharunderscore}{\kern0pt}name{\isacharcomma}{\kern0pt}\ NoSyn{\isacharparenright}{\kern0pt}{\isacharcomma}{\kern0pt}\isanewline
\ \ \ \ \ \ \ \ \ \ \ \ \ \ \ \ \ \ \ \ \ \ \ \ {\isacharparenleft}{\kern0pt}{\isacharparenleft}{\kern0pt}Binding{\isachardot}{\kern0pt}name\ {\isacharparenleft}{\kern0pt}def{\isacharunderscore}{\kern0pt}name\ {\isacharcircum}{\kern0pt}\ {\isachardoublequote}{\kern0pt}{\isacharunderscore}{\kern0pt}def{\isachardoublequote}{\kern0pt}{\isacharparenright}{\kern0pt}{\isacharcomma}{\kern0pt}\ {\isacharbrackleft}{\kern0pt}{\isacharbrackright}{\kern0pt}{\isacharparenright}{\kern0pt}{\isacharcomma}{\kern0pt}\ at{\isacharparenright}{\kern0pt}{\isacharparenright}{\kern0pt}\ thy\ {\isacharbar}{\kern0pt}{\isachargreater}{\kern0pt}\ {\isacharhash}{\kern0pt}{\isadigit{2}}\ {\isacharbar}{\kern0pt}{\isachargreater}{\kern0pt}\isanewline
\ \ \ \ {\isacharparenleft}{\kern0pt}if\ tc\ then\ synth{\isacharunderscore}{\kern0pt}thm{\isacharunderscore}{\kern0pt}tc\ def{\isacharunderscore}{\kern0pt}name\ {\isacharparenleft}{\kern0pt}def{\isacharunderscore}{\kern0pt}name\ {\isacharcircum}{\kern0pt}\ {\isachardoublequote}{\kern0pt}{\isacharunderscore}{\kern0pt}def{\isachardoublequote}{\kern0pt}{\isacharparenright}{\kern0pt}\ hyps{\isacharprime}{\kern0pt}\ vs\ pos\ else\ I{\isacharparenright}{\kern0pt}\ {\isacharbar}{\kern0pt}{\isachargreater}{\kern0pt}\isanewline
\ \ \ \ {\isacharparenleft}{\kern0pt}if\ auto\ then\ synth{\isacharunderscore}{\kern0pt}thm{\isacharunderscore}{\kern0pt}sats\ def{\isacharunderscore}{\kern0pt}name\ {\isacharparenleft}{\kern0pt}def{\isacharunderscore}{\kern0pt}name\ {\isacharcircum}{\kern0pt}\ {\isachardoublequote}{\kern0pt}{\isacharunderscore}{\kern0pt}def{\isachardoublequote}{\kern0pt}{\isacharparenright}{\kern0pt}\ lhs\ set\ env\ hyps\ vars\ vs\ pos\ thm{\isacharunderscore}{\kern0pt}tms\ else\ I{\isacharparenright}{\kern0pt}\isanewline
\isanewline
end\isanewline
{\isacartoucheclose}\isanewline
\isacommand{ML}\isamarkupfalse%
{\isacartoucheopen}\isanewline
\isanewline
local\isanewline
\ \ val\ synth{\isacharunderscore}{\kern0pt}constdecl\ {\isacharequal}{\kern0pt}\isanewline
\ \ \ \ \ \ \ Parse{\isachardot}{\kern0pt}position\ {\isacharparenleft}{\kern0pt}Parse{\isachardot}{\kern0pt}string\ {\isacharminus}{\kern0pt}{\isacharminus}{\kern0pt}\ {\isacharparenleft}{\kern0pt}{\isacharparenleft}{\kern0pt}Parse{\isachardot}{\kern0pt}{\isachardollar}{\kern0pt}{\isachardollar}{\kern0pt}{\isachardollar}{\kern0pt}\ {\isachardoublequote}{\kern0pt}from{\isacharunderscore}{\kern0pt}schematic{\isachardoublequote}{\kern0pt}\ {\isacharbar}{\kern0pt}{\isacharminus}{\kern0pt}{\isacharminus}{\kern0pt}\ Parse{\isachardot}{\kern0pt}thm{\isacharparenright}{\kern0pt}{\isacharparenright}{\kern0pt}{\isacharparenright}{\kern0pt}{\isacharsemicolon}{\kern0pt}\isanewline
\isanewline
\ \ val\ {\isacharunderscore}{\kern0pt}\ {\isacharequal}{\kern0pt}\isanewline
\ \ \ \ \ Outer{\isacharunderscore}{\kern0pt}Syntax{\isachardot}{\kern0pt}local{\isacharunderscore}{\kern0pt}theory\ \isactrlcommandUNDERSCOREkeyword {\isasymopen}synthesize{\isasymclose}\ {\isachardoublequote}{\kern0pt}ML\ setup\ for\ synthetic\ definitions{\isachardoublequote}{\kern0pt}\isanewline
\ \ \ \ \ \ \ {\isacharparenleft}{\kern0pt}synth{\isacharunderscore}{\kern0pt}constdecl\ {\isachargreater}{\kern0pt}{\isachargreater}{\kern0pt}\ {\isacharparenleft}{\kern0pt}fn\ {\isacharparenleft}{\kern0pt}{\isacharparenleft}{\kern0pt}bndg{\isacharcomma}{\kern0pt}thm{\isacharparenright}{\kern0pt}{\isacharcomma}{\kern0pt}p{\isacharparenright}{\kern0pt}\ {\isacharequal}{\kern0pt}{\isachargreater}{\kern0pt}\ synthetic{\isacharunderscore}{\kern0pt}def\ bndg\ thm\ p\ true\ true{\isacharparenright}{\kern0pt}{\isacharparenright}{\kern0pt}\isanewline
\isanewline
\ \ val\ {\isacharunderscore}{\kern0pt}\ {\isacharequal}{\kern0pt}\isanewline
\ \ \ \ \ Outer{\isacharunderscore}{\kern0pt}Syntax{\isachardot}{\kern0pt}local{\isacharunderscore}{\kern0pt}theory\ \isactrlcommandUNDERSCOREkeyword {\isasymopen}synthesize{\isacharunderscore}{\kern0pt}notc{\isasymclose}\ {\isachardoublequote}{\kern0pt}ML\ setup\ for\ synthetic\ definitions{\isachardoublequote}{\kern0pt}\isanewline
\ \ \ \ \ \ \ {\isacharparenleft}{\kern0pt}synth{\isacharunderscore}{\kern0pt}constdecl\ {\isachargreater}{\kern0pt}{\isachargreater}{\kern0pt}\ {\isacharparenleft}{\kern0pt}fn\ {\isacharparenleft}{\kern0pt}{\isacharparenleft}{\kern0pt}bndg{\isacharcomma}{\kern0pt}thm{\isacharparenright}{\kern0pt}{\isacharcomma}{\kern0pt}p{\isacharparenright}{\kern0pt}\ {\isacharequal}{\kern0pt}{\isachargreater}{\kern0pt}\ synthetic{\isacharunderscore}{\kern0pt}def\ bndg\ thm\ p\ false\ false{\isacharparenright}{\kern0pt}{\isacharparenright}{\kern0pt}\isanewline
\isanewline
in\isanewline
\isanewline
end\isanewline
{\isacartoucheclose}%
\endisatagML
{\isafoldML}%
%
\isadelimML
%
\endisadelimML
%
\begin{isamarkuptext}%
The \isatt{s{\kern0pt}y{\kern0pt}n{\kern0pt}t{\kern0pt}h{\kern0pt}e{\kern0pt}t{\kern0pt}i{\kern0pt}c{\kern0pt}{\char`\_}{\kern0pt}d{\kern0pt}e{\kern0pt}f{\kern0pt}} function extracts definitions from
schematic goals. A new definition is added to the context.%
\end{isamarkuptext}\isamarkuptrue%
%
\isadelimtheory
%
\endisadelimtheory
%
\isatagtheory
\isacommand{end}\isamarkupfalse%
%
\endisatagtheory
{\isafoldtheory}%
%
\isadelimtheory
%
\endisadelimtheory
%
\end{isabellebody}%
\endinput
%:%file=~/source/repos/ZF-notAC/code/Forcing/Synthetic_Definition.thy%:%
%:%11=1%:%
%:%27=3%:%
%:%28=3%:%
%:%29=4%:%
%:%30=5%:%
%:%31=6%:%
%:%32=7%:%
%:%33=8%:%
%:%38=8%:%
%:%43=9%:%
%:%48=10%:%
%:%49=10%:%
%:%134=95%:%
%:%135=96%:%
%:%136=96%:%
%:%162=114%:%
%:%163=115%:%
%:%171=127%:%

%
\begin{isabellebody}%
\setisabellecontext{Interface}%
%
\isadelimdocument
%
\endisadelimdocument
%
\isatagdocument
%
\isamarkupsection{Interface between set models and Constructibility%
}
\isamarkuptrue%
%
\endisatagdocument
{\isafolddocument}%
%
\isadelimdocument
%
\endisadelimdocument
%
\begin{isamarkuptext}%
This theory provides an interface between Paulson's
relativization results and set models of ZFC. In particular,
it is used to prove that the locale \isa{forcing{\isacharunderscore}{\kern0pt}data} is
a sublocale of all relevant locales in ZF-Constructibility
(\isa{M{\isacharunderscore}{\kern0pt}trivial}, \isa{M{\isacharunderscore}{\kern0pt}basic}, \isa{M{\isacharunderscore}{\kern0pt}eclose}, etc).%
\end{isamarkuptext}\isamarkuptrue%
%
\isadelimtheory
%
\endisadelimtheory
%
\isatagtheory
\isacommand{theory}\isamarkupfalse%
\ Interface\isanewline
\ \ \isakeyword{imports}\isanewline
\ \ \ \ Nat{\isacharunderscore}{\kern0pt}Miscellanea\isanewline
\ \ \ \ Relative{\isacharunderscore}{\kern0pt}Univ\isanewline
\ \ \ \ Synthetic{\isacharunderscore}{\kern0pt}Definition\isanewline
\isakeyword{begin}%
\endisatagtheory
{\isafoldtheory}%
%
\isadelimtheory
%
\endisadelimtheory
\isanewline
\isanewline
\isacommand{syntax}\isamarkupfalse%
\isanewline
\ \ {\isachardoublequoteopen}{\isacharunderscore}{\kern0pt}sats{\isachardoublequoteclose}\ \ {\isacharcolon}{\kern0pt}{\isacharcolon}{\kern0pt}\ {\isachardoublequoteopen}{\isacharbrackleft}{\kern0pt}i{\isacharcomma}{\kern0pt}\ i{\isacharcomma}{\kern0pt}\ i{\isacharbrackright}{\kern0pt}\ {\isasymRightarrow}\ o{\isachardoublequoteclose}\ \ {\isacharparenleft}{\kern0pt}{\isachardoublequoteopen}{\isacharparenleft}{\kern0pt}{\isacharunderscore}{\kern0pt}{\isacharcomma}{\kern0pt}\ {\isacharunderscore}{\kern0pt}\ {\isasymTurnstile}\ {\isacharunderscore}{\kern0pt}{\isacharparenright}{\kern0pt}{\isachardoublequoteclose}\ {\isacharbrackleft}{\kern0pt}{\isadigit{3}}{\isadigit{6}}{\isacharcomma}{\kern0pt}{\isadigit{3}}{\isadigit{6}}{\isacharcomma}{\kern0pt}{\isadigit{3}}{\isadigit{6}}{\isacharbrackright}{\kern0pt}\ {\isadigit{6}}{\isadigit{0}}{\isacharparenright}{\kern0pt}\isanewline
\isacommand{translations}\isamarkupfalse%
\isanewline
\ \ {\isachardoublequoteopen}{\isacharparenleft}{\kern0pt}M{\isacharcomma}{\kern0pt}env\ {\isasymTurnstile}\ {\isasymphi}{\isacharparenright}{\kern0pt}{\isachardoublequoteclose}\ {\isasymrightleftharpoons}\ {\isachardoublequoteopen}CONST\ sats{\isacharparenleft}{\kern0pt}M{\isacharcomma}{\kern0pt}{\isasymphi}{\isacharcomma}{\kern0pt}env{\isacharparenright}{\kern0pt}{\isachardoublequoteclose}\isanewline
\isanewline
\isacommand{abbreviation}\isamarkupfalse%
\isanewline
\ \ dec{\isadigit{1}}{\isadigit{0}}\ \ {\isacharcolon}{\kern0pt}{\isacharcolon}{\kern0pt}\ i\ \ \ {\isacharparenleft}{\kern0pt}{\isachardoublequoteopen}{\isadigit{1}}{\isadigit{0}}{\isachardoublequoteclose}{\isacharparenright}{\kern0pt}\ \isakeyword{where}\ {\isachardoublequoteopen}{\isadigit{1}}{\isadigit{0}}\ {\isasymequiv}\ succ{\isacharparenleft}{\kern0pt}{\isadigit{9}}{\isacharparenright}{\kern0pt}{\isachardoublequoteclose}\isanewline
\isanewline
\isacommand{abbreviation}\isamarkupfalse%
\isanewline
\ \ dec{\isadigit{1}}{\isadigit{1}}\ \ {\isacharcolon}{\kern0pt}{\isacharcolon}{\kern0pt}\ i\ \ \ {\isacharparenleft}{\kern0pt}{\isachardoublequoteopen}{\isadigit{1}}{\isadigit{1}}{\isachardoublequoteclose}{\isacharparenright}{\kern0pt}\ \isakeyword{where}\ {\isachardoublequoteopen}{\isadigit{1}}{\isadigit{1}}\ {\isasymequiv}\ succ{\isacharparenleft}{\kern0pt}{\isadigit{1}}{\isadigit{0}}{\isacharparenright}{\kern0pt}{\isachardoublequoteclose}\isanewline
\isanewline
\isacommand{abbreviation}\isamarkupfalse%
\isanewline
\ \ dec{\isadigit{1}}{\isadigit{2}}\ \ {\isacharcolon}{\kern0pt}{\isacharcolon}{\kern0pt}\ i\ \ \ {\isacharparenleft}{\kern0pt}{\isachardoublequoteopen}{\isadigit{1}}{\isadigit{2}}{\isachardoublequoteclose}{\isacharparenright}{\kern0pt}\ \isakeyword{where}\ {\isachardoublequoteopen}{\isadigit{1}}{\isadigit{2}}\ {\isasymequiv}\ succ{\isacharparenleft}{\kern0pt}{\isadigit{1}}{\isadigit{1}}{\isacharparenright}{\kern0pt}{\isachardoublequoteclose}\isanewline
\isanewline
\isacommand{abbreviation}\isamarkupfalse%
\isanewline
\ \ dec{\isadigit{1}}{\isadigit{3}}\ \ {\isacharcolon}{\kern0pt}{\isacharcolon}{\kern0pt}\ i\ \ \ {\isacharparenleft}{\kern0pt}{\isachardoublequoteopen}{\isadigit{1}}{\isadigit{3}}{\isachardoublequoteclose}{\isacharparenright}{\kern0pt}\ \isakeyword{where}\ {\isachardoublequoteopen}{\isadigit{1}}{\isadigit{3}}\ {\isasymequiv}\ succ{\isacharparenleft}{\kern0pt}{\isadigit{1}}{\isadigit{2}}{\isacharparenright}{\kern0pt}{\isachardoublequoteclose}\isanewline
\isanewline
\isacommand{abbreviation}\isamarkupfalse%
\isanewline
\ \ dec{\isadigit{1}}{\isadigit{4}}\ \ {\isacharcolon}{\kern0pt}{\isacharcolon}{\kern0pt}\ i\ \ \ {\isacharparenleft}{\kern0pt}{\isachardoublequoteopen}{\isadigit{1}}{\isadigit{4}}{\isachardoublequoteclose}{\isacharparenright}{\kern0pt}\ \isakeyword{where}\ {\isachardoublequoteopen}{\isadigit{1}}{\isadigit{4}}\ {\isasymequiv}\ succ{\isacharparenleft}{\kern0pt}{\isadigit{1}}{\isadigit{3}}{\isacharparenright}{\kern0pt}{\isachardoublequoteclose}\isanewline
\isanewline
\isanewline
\isacommand{definition}\isamarkupfalse%
\isanewline
\ \ infinity{\isacharunderscore}{\kern0pt}ax\ {\isacharcolon}{\kern0pt}{\isacharcolon}{\kern0pt}\ {\isachardoublequoteopen}{\isacharparenleft}{\kern0pt}i\ {\isasymRightarrow}\ o{\isacharparenright}{\kern0pt}\ {\isasymRightarrow}\ o{\isachardoublequoteclose}\ \isakeyword{where}\isanewline
\ \ {\isachardoublequoteopen}infinity{\isacharunderscore}{\kern0pt}ax{\isacharparenleft}{\kern0pt}M{\isacharparenright}{\kern0pt}\ {\isasymequiv}\isanewline
\ \ \ \ \ \ {\isacharparenleft}{\kern0pt}{\isasymexists}I{\isacharbrackleft}{\kern0pt}M{\isacharbrackright}{\kern0pt}{\isachardot}{\kern0pt}\ {\isacharparenleft}{\kern0pt}{\isasymexists}z{\isacharbrackleft}{\kern0pt}M{\isacharbrackright}{\kern0pt}{\isachardot}{\kern0pt}\ empty{\isacharparenleft}{\kern0pt}M{\isacharcomma}{\kern0pt}z{\isacharparenright}{\kern0pt}\ {\isasymand}\ z{\isasymin}I{\isacharparenright}{\kern0pt}\ {\isasymand}\ {\isacharparenleft}{\kern0pt}{\isasymforall}y{\isacharbrackleft}{\kern0pt}M{\isacharbrackright}{\kern0pt}{\isachardot}{\kern0pt}\ y{\isasymin}I\ {\isasymlongrightarrow}\ {\isacharparenleft}{\kern0pt}{\isasymexists}sy{\isacharbrackleft}{\kern0pt}M{\isacharbrackright}{\kern0pt}{\isachardot}{\kern0pt}\ successor{\isacharparenleft}{\kern0pt}M{\isacharcomma}{\kern0pt}y{\isacharcomma}{\kern0pt}sy{\isacharparenright}{\kern0pt}\ {\isasymand}\ sy{\isasymin}I{\isacharparenright}{\kern0pt}{\isacharparenright}{\kern0pt}{\isacharparenright}{\kern0pt}{\isachardoublequoteclose}\isanewline
\isanewline
\isacommand{definition}\isamarkupfalse%
\isanewline
\ \ choice{\isacharunderscore}{\kern0pt}ax\ {\isacharcolon}{\kern0pt}{\isacharcolon}{\kern0pt}\ {\isachardoublequoteopen}{\isacharparenleft}{\kern0pt}i{\isasymRightarrow}o{\isacharparenright}{\kern0pt}\ {\isasymRightarrow}\ o{\isachardoublequoteclose}\ \isakeyword{where}\isanewline
\ \ {\isachardoublequoteopen}choice{\isacharunderscore}{\kern0pt}ax{\isacharparenleft}{\kern0pt}M{\isacharparenright}{\kern0pt}\ {\isasymequiv}\ {\isasymforall}x{\isacharbrackleft}{\kern0pt}M{\isacharbrackright}{\kern0pt}{\isachardot}{\kern0pt}\ {\isasymexists}a{\isacharbrackleft}{\kern0pt}M{\isacharbrackright}{\kern0pt}{\isachardot}{\kern0pt}\ {\isasymexists}f{\isacharbrackleft}{\kern0pt}M{\isacharbrackright}{\kern0pt}{\isachardot}{\kern0pt}\ ordinal{\isacharparenleft}{\kern0pt}M{\isacharcomma}{\kern0pt}a{\isacharparenright}{\kern0pt}\ {\isasymand}\ surjection{\isacharparenleft}{\kern0pt}M{\isacharcomma}{\kern0pt}a{\isacharcomma}{\kern0pt}x{\isacharcomma}{\kern0pt}f{\isacharparenright}{\kern0pt}{\isachardoublequoteclose}\isanewline
\isanewline
\isacommand{context}\isamarkupfalse%
\ M{\isacharunderscore}{\kern0pt}basic\ \isakeyword{begin}\isanewline
\isanewline
\isacommand{lemma}\isamarkupfalse%
\ choice{\isacharunderscore}{\kern0pt}ax{\isacharunderscore}{\kern0pt}abs\ {\isacharcolon}{\kern0pt}\isanewline
\ \ {\isachardoublequoteopen}choice{\isacharunderscore}{\kern0pt}ax{\isacharparenleft}{\kern0pt}M{\isacharparenright}{\kern0pt}\ {\isasymlongleftrightarrow}\ {\isacharparenleft}{\kern0pt}{\isasymforall}x{\isacharbrackleft}{\kern0pt}M{\isacharbrackright}{\kern0pt}{\isachardot}{\kern0pt}\ {\isasymexists}a{\isacharbrackleft}{\kern0pt}M{\isacharbrackright}{\kern0pt}{\isachardot}{\kern0pt}\ {\isasymexists}f{\isacharbrackleft}{\kern0pt}M{\isacharbrackright}{\kern0pt}{\isachardot}{\kern0pt}\ Ord{\isacharparenleft}{\kern0pt}a{\isacharparenright}{\kern0pt}\ {\isasymand}\ f\ {\isasymin}\ surj{\isacharparenleft}{\kern0pt}a{\isacharcomma}{\kern0pt}x{\isacharparenright}{\kern0pt}{\isacharparenright}{\kern0pt}{\isachardoublequoteclose}\isanewline
%
\isadelimproof
\ \ %
\endisadelimproof
%
\isatagproof
\isacommand{unfolding}\isamarkupfalse%
\ choice{\isacharunderscore}{\kern0pt}ax{\isacharunderscore}{\kern0pt}def\isanewline
\ \ \isacommand{by}\isamarkupfalse%
\ {\isacharparenleft}{\kern0pt}simp{\isacharparenright}{\kern0pt}%
\endisatagproof
{\isafoldproof}%
%
\isadelimproof
\isanewline
%
\endisadelimproof
\isanewline
\isacommand{end}\isamarkupfalse%
\ \isanewline
\isanewline
\isacommand{definition}\isamarkupfalse%
\isanewline
\ \ wellfounded{\isacharunderscore}{\kern0pt}trancl\ {\isacharcolon}{\kern0pt}{\isacharcolon}{\kern0pt}\ {\isachardoublequoteopen}{\isacharbrackleft}{\kern0pt}i{\isacharequal}{\kern0pt}{\isachargreater}{\kern0pt}o{\isacharcomma}{\kern0pt}i{\isacharcomma}{\kern0pt}i{\isacharcomma}{\kern0pt}i{\isacharbrackright}{\kern0pt}\ {\isacharequal}{\kern0pt}{\isachargreater}{\kern0pt}\ o{\isachardoublequoteclose}\ \isakeyword{where}\isanewline
\ \ {\isachardoublequoteopen}wellfounded{\isacharunderscore}{\kern0pt}trancl{\isacharparenleft}{\kern0pt}M{\isacharcomma}{\kern0pt}Z{\isacharcomma}{\kern0pt}r{\isacharcomma}{\kern0pt}p{\isacharparenright}{\kern0pt}\ {\isasymequiv}\isanewline
\ \ \ \ \ \ {\isasymexists}w{\isacharbrackleft}{\kern0pt}M{\isacharbrackright}{\kern0pt}{\isachardot}{\kern0pt}\ {\isasymexists}wx{\isacharbrackleft}{\kern0pt}M{\isacharbrackright}{\kern0pt}{\isachardot}{\kern0pt}\ {\isasymexists}rp{\isacharbrackleft}{\kern0pt}M{\isacharbrackright}{\kern0pt}{\isachardot}{\kern0pt}\isanewline
\ \ \ \ \ \ \ \ \ \ \ \ \ \ \ w\ {\isasymin}\ Z\ {\isacharampersand}{\kern0pt}\ pair{\isacharparenleft}{\kern0pt}M{\isacharcomma}{\kern0pt}w{\isacharcomma}{\kern0pt}p{\isacharcomma}{\kern0pt}wx{\isacharparenright}{\kern0pt}\ {\isacharampersand}{\kern0pt}\ tran{\isacharunderscore}{\kern0pt}closure{\isacharparenleft}{\kern0pt}M{\isacharcomma}{\kern0pt}r{\isacharcomma}{\kern0pt}rp{\isacharparenright}{\kern0pt}\ {\isacharampersand}{\kern0pt}\ wx\ {\isasymin}\ rp{\isachardoublequoteclose}\isanewline
\isanewline
\isacommand{lemma}\isamarkupfalse%
\ empty{\isacharunderscore}{\kern0pt}intf\ {\isacharcolon}{\kern0pt}\isanewline
\ \ {\isachardoublequoteopen}infinity{\isacharunderscore}{\kern0pt}ax{\isacharparenleft}{\kern0pt}M{\isacharparenright}{\kern0pt}\ {\isasymLongrightarrow}\isanewline
\ \ {\isacharparenleft}{\kern0pt}{\isasymexists}z{\isacharbrackleft}{\kern0pt}M{\isacharbrackright}{\kern0pt}{\isachardot}{\kern0pt}\ empty{\isacharparenleft}{\kern0pt}M{\isacharcomma}{\kern0pt}z{\isacharparenright}{\kern0pt}{\isacharparenright}{\kern0pt}{\isachardoublequoteclose}\isanewline
%
\isadelimproof
\ \ %
\endisadelimproof
%
\isatagproof
\isacommand{by}\isamarkupfalse%
\ {\isacharparenleft}{\kern0pt}auto\ simp\ add{\isacharcolon}{\kern0pt}\ empty{\isacharunderscore}{\kern0pt}def\ infinity{\isacharunderscore}{\kern0pt}ax{\isacharunderscore}{\kern0pt}def{\isacharparenright}{\kern0pt}%
\endisatagproof
{\isafoldproof}%
%
\isadelimproof
\isanewline
%
\endisadelimproof
\isanewline
\isacommand{lemma}\isamarkupfalse%
\ Transset{\isacharunderscore}{\kern0pt}intf\ {\isacharcolon}{\kern0pt}\isanewline
\ \ {\isachardoublequoteopen}Transset{\isacharparenleft}{\kern0pt}M{\isacharparenright}{\kern0pt}\ {\isasymLongrightarrow}\ \ y{\isasymin}x\ {\isasymLongrightarrow}\ x\ {\isasymin}\ M\ {\isasymLongrightarrow}\ y\ {\isasymin}\ M{\isachardoublequoteclose}\isanewline
%
\isadelimproof
\ \ %
\endisadelimproof
%
\isatagproof
\isacommand{by}\isamarkupfalse%
\ {\isacharparenleft}{\kern0pt}simp\ add{\isacharcolon}{\kern0pt}\ Transset{\isacharunderscore}{\kern0pt}def{\isacharcomma}{\kern0pt}auto{\isacharparenright}{\kern0pt}%
\endisatagproof
{\isafoldproof}%
%
\isadelimproof
\isanewline
%
\endisadelimproof
\isanewline
\isacommand{locale}\isamarkupfalse%
\ M{\isacharunderscore}{\kern0pt}ZF{\isacharunderscore}{\kern0pt}trans\ {\isacharequal}{\kern0pt}\isanewline
\ \ \isakeyword{fixes}\ M\isanewline
\ \ \isakeyword{assumes}\isanewline
\ \ \ \ upair{\isacharunderscore}{\kern0pt}ax{\isacharcolon}{\kern0pt}\ \ \ \ \ \ \ \ \ {\isachardoublequoteopen}upair{\isacharunderscore}{\kern0pt}ax{\isacharparenleft}{\kern0pt}{\isacharhash}{\kern0pt}{\isacharhash}{\kern0pt}M{\isacharparenright}{\kern0pt}{\isachardoublequoteclose}\isanewline
\ \ \ \ \isakeyword{and}\ Union{\isacharunderscore}{\kern0pt}ax{\isacharcolon}{\kern0pt}\ \ \ \ \ \ \ \ \ {\isachardoublequoteopen}Union{\isacharunderscore}{\kern0pt}ax{\isacharparenleft}{\kern0pt}{\isacharhash}{\kern0pt}{\isacharhash}{\kern0pt}M{\isacharparenright}{\kern0pt}{\isachardoublequoteclose}\isanewline
\ \ \ \ \isakeyword{and}\ power{\isacharunderscore}{\kern0pt}ax{\isacharcolon}{\kern0pt}\ \ \ \ \ \ \ \ \ {\isachardoublequoteopen}power{\isacharunderscore}{\kern0pt}ax{\isacharparenleft}{\kern0pt}{\isacharhash}{\kern0pt}{\isacharhash}{\kern0pt}M{\isacharparenright}{\kern0pt}{\isachardoublequoteclose}\isanewline
\ \ \ \ \isakeyword{and}\ extensionality{\isacharcolon}{\kern0pt}\ \ \ {\isachardoublequoteopen}extensionality{\isacharparenleft}{\kern0pt}{\isacharhash}{\kern0pt}{\isacharhash}{\kern0pt}M{\isacharparenright}{\kern0pt}{\isachardoublequoteclose}\isanewline
\ \ \ \ \isakeyword{and}\ foundation{\isacharunderscore}{\kern0pt}ax{\isacharcolon}{\kern0pt}\ \ \ \ {\isachardoublequoteopen}foundation{\isacharunderscore}{\kern0pt}ax{\isacharparenleft}{\kern0pt}{\isacharhash}{\kern0pt}{\isacharhash}{\kern0pt}M{\isacharparenright}{\kern0pt}{\isachardoublequoteclose}\isanewline
\ \ \ \ \isakeyword{and}\ infinity{\isacharunderscore}{\kern0pt}ax{\isacharcolon}{\kern0pt}\ \ \ \ \ \ {\isachardoublequoteopen}infinity{\isacharunderscore}{\kern0pt}ax{\isacharparenleft}{\kern0pt}{\isacharhash}{\kern0pt}{\isacharhash}{\kern0pt}M{\isacharparenright}{\kern0pt}{\isachardoublequoteclose}\isanewline
\ \ \ \ \isakeyword{and}\ separation{\isacharunderscore}{\kern0pt}ax{\isacharcolon}{\kern0pt}\ \ \ \ {\isachardoublequoteopen}{\isasymphi}{\isasymin}formula\ {\isasymLongrightarrow}\ env{\isasymin}list{\isacharparenleft}{\kern0pt}M{\isacharparenright}{\kern0pt}\ {\isasymLongrightarrow}\ arity{\isacharparenleft}{\kern0pt}{\isasymphi}{\isacharparenright}{\kern0pt}\ {\isasymle}\ {\isadigit{1}}\ {\isacharhash}{\kern0pt}{\isacharplus}{\kern0pt}\ length{\isacharparenleft}{\kern0pt}env{\isacharparenright}{\kern0pt}\ {\isasymLongrightarrow}\isanewline
\ \ \ \ \ \ \ \ \ \ \ \ \ \ \ \ \ \ \ \ separation{\isacharparenleft}{\kern0pt}{\isacharhash}{\kern0pt}{\isacharhash}{\kern0pt}M{\isacharcomma}{\kern0pt}{\isasymlambda}x{\isachardot}{\kern0pt}\ sats{\isacharparenleft}{\kern0pt}M{\isacharcomma}{\kern0pt}{\isasymphi}{\isacharcomma}{\kern0pt}{\isacharbrackleft}{\kern0pt}x{\isacharbrackright}{\kern0pt}\ {\isacharat}{\kern0pt}\ env{\isacharparenright}{\kern0pt}{\isacharparenright}{\kern0pt}{\isachardoublequoteclose}\isanewline
\ \ \ \ \isakeyword{and}\ replacement{\isacharunderscore}{\kern0pt}ax{\isacharcolon}{\kern0pt}\ \ \ {\isachardoublequoteopen}{\isasymphi}{\isasymin}formula\ {\isasymLongrightarrow}\ env{\isasymin}list{\isacharparenleft}{\kern0pt}M{\isacharparenright}{\kern0pt}\ {\isasymLongrightarrow}\ arity{\isacharparenleft}{\kern0pt}{\isasymphi}{\isacharparenright}{\kern0pt}\ {\isasymle}\ {\isadigit{2}}\ {\isacharhash}{\kern0pt}{\isacharplus}{\kern0pt}\ length{\isacharparenleft}{\kern0pt}env{\isacharparenright}{\kern0pt}\ {\isasymLongrightarrow}\isanewline
\ \ \ \ \ \ \ \ \ \ \ \ \ \ \ \ \ \ \ \ strong{\isacharunderscore}{\kern0pt}replacement{\isacharparenleft}{\kern0pt}{\isacharhash}{\kern0pt}{\isacharhash}{\kern0pt}M{\isacharcomma}{\kern0pt}{\isasymlambda}x\ y{\isachardot}{\kern0pt}\ sats{\isacharparenleft}{\kern0pt}M{\isacharcomma}{\kern0pt}{\isasymphi}{\isacharcomma}{\kern0pt}{\isacharbrackleft}{\kern0pt}x{\isacharcomma}{\kern0pt}y{\isacharbrackright}{\kern0pt}\ {\isacharat}{\kern0pt}\ env{\isacharparenright}{\kern0pt}{\isacharparenright}{\kern0pt}{\isachardoublequoteclose}\isanewline
\ \ \ \ \isakeyword{and}\ trans{\isacharunderscore}{\kern0pt}M{\isacharcolon}{\kern0pt}\ \ \ \ \ \ \ \ \ \ {\isachardoublequoteopen}Transset{\isacharparenleft}{\kern0pt}M{\isacharparenright}{\kern0pt}{\isachardoublequoteclose}\isanewline
\isakeyword{begin}\isanewline
\isanewline
\isanewline
\isacommand{lemma}\isamarkupfalse%
\ TranssetI\ {\isacharcolon}{\kern0pt}\isanewline
\ \ {\isachardoublequoteopen}{\isacharparenleft}{\kern0pt}{\isasymAnd}y\ x{\isachardot}{\kern0pt}\ y{\isasymin}x\ {\isasymLongrightarrow}\ x{\isasymin}M\ {\isasymLongrightarrow}\ y{\isasymin}M{\isacharparenright}{\kern0pt}\ {\isasymLongrightarrow}\ Transset{\isacharparenleft}{\kern0pt}M{\isacharparenright}{\kern0pt}{\isachardoublequoteclose}\isanewline
%
\isadelimproof
\ \ %
\endisadelimproof
%
\isatagproof
\isacommand{by}\isamarkupfalse%
\ {\isacharparenleft}{\kern0pt}auto\ simp\ add{\isacharcolon}{\kern0pt}\ Transset{\isacharunderscore}{\kern0pt}def{\isacharparenright}{\kern0pt}%
\endisatagproof
{\isafoldproof}%
%
\isadelimproof
\isanewline
%
\endisadelimproof
\isanewline
\isacommand{lemma}\isamarkupfalse%
\ zero{\isacharunderscore}{\kern0pt}in{\isacharunderscore}{\kern0pt}M{\isacharcolon}{\kern0pt}\ \ {\isachardoublequoteopen}{\isadigit{0}}\ {\isasymin}\ M{\isachardoublequoteclose}\isanewline
%
\isadelimproof
%
\endisadelimproof
%
\isatagproof
\isacommand{proof}\isamarkupfalse%
\ {\isacharminus}{\kern0pt}\isanewline
\ \ \isacommand{from}\isamarkupfalse%
\ infinity{\isacharunderscore}{\kern0pt}ax\ \isacommand{have}\isamarkupfalse%
\isanewline
\ \ \ \ {\isachardoublequoteopen}{\isacharparenleft}{\kern0pt}{\isasymexists}z{\isacharbrackleft}{\kern0pt}{\isacharhash}{\kern0pt}{\isacharhash}{\kern0pt}M{\isacharbrackright}{\kern0pt}{\isachardot}{\kern0pt}\ empty{\isacharparenleft}{\kern0pt}{\isacharhash}{\kern0pt}{\isacharhash}{\kern0pt}M{\isacharcomma}{\kern0pt}z{\isacharparenright}{\kern0pt}{\isacharparenright}{\kern0pt}{\isachardoublequoteclose}\isanewline
\ \ \ \ \isacommand{by}\isamarkupfalse%
\ {\isacharparenleft}{\kern0pt}rule\ empty{\isacharunderscore}{\kern0pt}intf{\isacharparenright}{\kern0pt}\isanewline
\ \ \isacommand{then}\isamarkupfalse%
\ \isacommand{obtain}\isamarkupfalse%
\ z\ \isakeyword{where}\isanewline
\ \ \ \ zm{\isacharcolon}{\kern0pt}\ {\isachardoublequoteopen}empty{\isacharparenleft}{\kern0pt}{\isacharhash}{\kern0pt}{\isacharhash}{\kern0pt}M{\isacharcomma}{\kern0pt}z{\isacharparenright}{\kern0pt}{\isachardoublequoteclose}\ \ {\isachardoublequoteopen}z{\isasymin}M{\isachardoublequoteclose}\isanewline
\ \ \ \ \isacommand{by}\isamarkupfalse%
\ auto\isanewline
\ \ \isacommand{with}\isamarkupfalse%
\ trans{\isacharunderscore}{\kern0pt}M\ \isacommand{have}\isamarkupfalse%
\ {\isachardoublequoteopen}z{\isacharequal}{\kern0pt}{\isadigit{0}}{\isachardoublequoteclose}\isanewline
\ \ \ \ \isacommand{by}\isamarkupfalse%
\ {\isacharparenleft}{\kern0pt}simp\ \ add{\isacharcolon}{\kern0pt}\ empty{\isacharunderscore}{\kern0pt}def{\isacharcomma}{\kern0pt}\ blast\ intro{\isacharcolon}{\kern0pt}\ Transset{\isacharunderscore}{\kern0pt}intf\ {\isacharparenright}{\kern0pt}\isanewline
\ \ \isacommand{with}\isamarkupfalse%
\ zm\ \isacommand{show}\isamarkupfalse%
\ {\isacharquery}{\kern0pt}thesis\isanewline
\ \ \ \ \isacommand{by}\isamarkupfalse%
\ simp\isanewline
\isacommand{qed}\isamarkupfalse%
%
\endisatagproof
{\isafoldproof}%
%
\isadelimproof
%
\endisadelimproof
%
\isadelimdocument
%
\endisadelimdocument
%
\isatagdocument
%
\isamarkupsubsection{Interface with \isa{M{\isacharunderscore}{\kern0pt}trivial}%
}
\isamarkuptrue%
%
\endisatagdocument
{\isafolddocument}%
%
\isadelimdocument
%
\endisadelimdocument
\isacommand{lemma}\isamarkupfalse%
\ mtrans\ {\isacharcolon}{\kern0pt}\isanewline
\ \ {\isachardoublequoteopen}M{\isacharunderscore}{\kern0pt}trans{\isacharparenleft}{\kern0pt}{\isacharhash}{\kern0pt}{\isacharhash}{\kern0pt}M{\isacharparenright}{\kern0pt}{\isachardoublequoteclose}\isanewline
%
\isadelimproof
\ \ %
\endisadelimproof
%
\isatagproof
\isacommand{using}\isamarkupfalse%
\ Transset{\isacharunderscore}{\kern0pt}intf{\isacharbrackleft}{\kern0pt}OF\ trans{\isacharunderscore}{\kern0pt}M{\isacharbrackright}{\kern0pt}\ zero{\isacharunderscore}{\kern0pt}in{\isacharunderscore}{\kern0pt}M\ exI{\isacharbrackleft}{\kern0pt}of\ {\isachardoublequoteopen}{\isasymlambda}x{\isachardot}{\kern0pt}\ x{\isasymin}M{\isachardoublequoteclose}{\isacharbrackright}{\kern0pt}\isanewline
\ \ \isacommand{by}\isamarkupfalse%
\ unfold{\isacharunderscore}{\kern0pt}locales\ auto%
\endisatagproof
{\isafoldproof}%
%
\isadelimproof
\isanewline
%
\endisadelimproof
\isanewline
\isanewline
\isacommand{lemma}\isamarkupfalse%
\ mtriv\ {\isacharcolon}{\kern0pt}\isanewline
\ \ {\isachardoublequoteopen}M{\isacharunderscore}{\kern0pt}trivial{\isacharparenleft}{\kern0pt}{\isacharhash}{\kern0pt}{\isacharhash}{\kern0pt}M{\isacharparenright}{\kern0pt}{\isachardoublequoteclose}\isanewline
%
\isadelimproof
\ \ %
\endisadelimproof
%
\isatagproof
\isacommand{using}\isamarkupfalse%
\ trans{\isacharunderscore}{\kern0pt}M\ M{\isacharunderscore}{\kern0pt}trivial{\isachardot}{\kern0pt}intro\ mtrans\ M{\isacharunderscore}{\kern0pt}trivial{\isacharunderscore}{\kern0pt}axioms{\isachardot}{\kern0pt}intro\ upair{\isacharunderscore}{\kern0pt}ax\ Union{\isacharunderscore}{\kern0pt}ax\isanewline
\ \ \isacommand{by}\isamarkupfalse%
\ simp%
\endisatagproof
{\isafoldproof}%
%
\isadelimproof
\isanewline
%
\endisadelimproof
\isanewline
\isacommand{end}\isamarkupfalse%
\isanewline
\isanewline
\isacommand{sublocale}\isamarkupfalse%
\ M{\isacharunderscore}{\kern0pt}ZF{\isacharunderscore}{\kern0pt}trans\ {\isasymsubseteq}\ M{\isacharunderscore}{\kern0pt}trivial\ {\isachardoublequoteopen}{\isacharhash}{\kern0pt}{\isacharhash}{\kern0pt}M{\isachardoublequoteclose}\isanewline
%
\isadelimproof
\ \ %
\endisadelimproof
%
\isatagproof
\isacommand{by}\isamarkupfalse%
\ {\isacharparenleft}{\kern0pt}rule\ mtriv{\isacharparenright}{\kern0pt}%
\endisatagproof
{\isafoldproof}%
%
\isadelimproof
\isanewline
%
\endisadelimproof
\isanewline
\isacommand{context}\isamarkupfalse%
\ M{\isacharunderscore}{\kern0pt}ZF{\isacharunderscore}{\kern0pt}trans\isanewline
\isakeyword{begin}%
\isadelimdocument
%
\endisadelimdocument
%
\isatagdocument
%
\isamarkupsubsection{Interface with \isa{M{\isacharunderscore}{\kern0pt}basic}%
}
\isamarkuptrue%
%
\endisatagdocument
{\isafolddocument}%
%
\isadelimdocument
%
\endisadelimdocument
\isacommand{schematic{\isacharunderscore}{\kern0pt}goal}\isamarkupfalse%
\ inter{\isacharunderscore}{\kern0pt}fm{\isacharunderscore}{\kern0pt}auto{\isacharcolon}{\kern0pt}\isanewline
\ \ \isakeyword{assumes}\isanewline
\ \ \ \ {\isachardoublequoteopen}nth{\isacharparenleft}{\kern0pt}i{\isacharcomma}{\kern0pt}env{\isacharparenright}{\kern0pt}\ {\isacharequal}{\kern0pt}\ x{\isachardoublequoteclose}\ {\isachardoublequoteopen}nth{\isacharparenleft}{\kern0pt}j{\isacharcomma}{\kern0pt}env{\isacharparenright}{\kern0pt}\ {\isacharequal}{\kern0pt}\ B{\isachardoublequoteclose}\isanewline
\ \ \ \ {\isachardoublequoteopen}i\ {\isasymin}\ nat{\isachardoublequoteclose}\ {\isachardoublequoteopen}j\ {\isasymin}\ nat{\isachardoublequoteclose}\ {\isachardoublequoteopen}env\ {\isasymin}\ list{\isacharparenleft}{\kern0pt}A{\isacharparenright}{\kern0pt}{\isachardoublequoteclose}\isanewline
\ \ \isakeyword{shows}\isanewline
\ \ \ \ {\isachardoublequoteopen}{\isacharparenleft}{\kern0pt}{\isasymforall}y{\isasymin}A\ {\isachardot}{\kern0pt}\ y{\isasymin}B\ {\isasymlongrightarrow}\ x{\isasymin}y{\isacharparenright}{\kern0pt}\ {\isasymlongleftrightarrow}\ sats{\isacharparenleft}{\kern0pt}A{\isacharcomma}{\kern0pt}{\isacharquery}{\kern0pt}ifm{\isacharparenleft}{\kern0pt}i{\isacharcomma}{\kern0pt}j{\isacharparenright}{\kern0pt}{\isacharcomma}{\kern0pt}env{\isacharparenright}{\kern0pt}{\isachardoublequoteclose}\isanewline
%
\isadelimproof
\ \ %
\endisadelimproof
%
\isatagproof
\isacommand{by}\isamarkupfalse%
\ {\isacharparenleft}{\kern0pt}insert\ assms\ {\isacharsemicolon}{\kern0pt}\ {\isacharparenleft}{\kern0pt}rule\ sep{\isacharunderscore}{\kern0pt}rules\ {\isacharbar}{\kern0pt}\ simp{\isacharparenright}{\kern0pt}{\isacharplus}{\kern0pt}{\isacharparenright}{\kern0pt}%
\endisatagproof
{\isafoldproof}%
%
\isadelimproof
\isanewline
%
\endisadelimproof
\isanewline
\isacommand{lemma}\isamarkupfalse%
\ inter{\isacharunderscore}{\kern0pt}sep{\isacharunderscore}{\kern0pt}intf\ {\isacharcolon}{\kern0pt}\isanewline
\ \ \isakeyword{assumes}\isanewline
\ \ \ \ {\isachardoublequoteopen}A{\isasymin}M{\isachardoublequoteclose}\isanewline
\ \ \isakeyword{shows}\isanewline
\ \ \ \ {\isachardoublequoteopen}separation{\isacharparenleft}{\kern0pt}{\isacharhash}{\kern0pt}{\isacharhash}{\kern0pt}M{\isacharcomma}{\kern0pt}{\isasymlambda}x\ {\isachardot}{\kern0pt}\ {\isasymforall}y{\isasymin}M\ {\isachardot}{\kern0pt}\ y{\isasymin}A\ {\isasymlongrightarrow}\ x{\isasymin}y{\isacharparenright}{\kern0pt}{\isachardoublequoteclose}\isanewline
%
\isadelimproof
%
\endisadelimproof
%
\isatagproof
\isacommand{proof}\isamarkupfalse%
\ {\isacharminus}{\kern0pt}\isanewline
\ \ \isacommand{obtain}\isamarkupfalse%
\ ifm\ \isakeyword{where}\isanewline
\ \ \ \ fmsats{\isacharcolon}{\kern0pt}{\isachardoublequoteopen}{\isasymAnd}env{\isachardot}{\kern0pt}\ env{\isasymin}list{\isacharparenleft}{\kern0pt}M{\isacharparenright}{\kern0pt}\ {\isasymLongrightarrow}\ {\isacharparenleft}{\kern0pt}{\isasymforall}\ y{\isasymin}M{\isachardot}{\kern0pt}\ y{\isasymin}{\isacharparenleft}{\kern0pt}nth{\isacharparenleft}{\kern0pt}{\isadigit{1}}{\isacharcomma}{\kern0pt}env{\isacharparenright}{\kern0pt}{\isacharparenright}{\kern0pt}\ {\isasymlongrightarrow}\ nth{\isacharparenleft}{\kern0pt}{\isadigit{0}}{\isacharcomma}{\kern0pt}env{\isacharparenright}{\kern0pt}{\isasymin}y{\isacharparenright}{\kern0pt}\isanewline
\ \ \ \ {\isasymlongleftrightarrow}\ sats{\isacharparenleft}{\kern0pt}M{\isacharcomma}{\kern0pt}ifm{\isacharparenleft}{\kern0pt}{\isadigit{0}}{\isacharcomma}{\kern0pt}{\isadigit{1}}{\isacharparenright}{\kern0pt}{\isacharcomma}{\kern0pt}env{\isacharparenright}{\kern0pt}{\isachardoublequoteclose}\isanewline
\ \ \ \ \isakeyword{and}\isanewline
\ \ \ \ {\isachardoublequoteopen}ifm{\isacharparenleft}{\kern0pt}{\isadigit{0}}{\isacharcomma}{\kern0pt}{\isadigit{1}}{\isacharparenright}{\kern0pt}\ {\isasymin}\ formula{\isachardoublequoteclose}\isanewline
\ \ \ \ \isakeyword{and}\isanewline
\ \ \ \ {\isachardoublequoteopen}arity{\isacharparenleft}{\kern0pt}ifm{\isacharparenleft}{\kern0pt}{\isadigit{0}}{\isacharcomma}{\kern0pt}{\isadigit{1}}{\isacharparenright}{\kern0pt}{\isacharparenright}{\kern0pt}\ {\isacharequal}{\kern0pt}\ {\isadigit{2}}{\isachardoublequoteclose}\isanewline
\ \ \ \ \isacommand{using}\isamarkupfalse%
\ {\isacartoucheopen}A{\isasymin}M{\isacartoucheclose}\ inter{\isacharunderscore}{\kern0pt}fm{\isacharunderscore}{\kern0pt}auto\isanewline
\ \ \ \ \isacommand{by}\isamarkupfalse%
\ {\isacharparenleft}{\kern0pt}simp\ del{\isacharcolon}{\kern0pt}FOL{\isacharunderscore}{\kern0pt}sats{\isacharunderscore}{\kern0pt}iff\ add{\isacharcolon}{\kern0pt}\ nat{\isacharunderscore}{\kern0pt}simp{\isacharunderscore}{\kern0pt}union{\isacharparenright}{\kern0pt}\isanewline
\ \ \isacommand{then}\isamarkupfalse%
\isanewline
\ \ \isacommand{have}\isamarkupfalse%
\ {\isachardoublequoteopen}{\isasymforall}a{\isasymin}M{\isachardot}{\kern0pt}\ separation{\isacharparenleft}{\kern0pt}{\isacharhash}{\kern0pt}{\isacharhash}{\kern0pt}M{\isacharcomma}{\kern0pt}\ {\isasymlambda}x{\isachardot}{\kern0pt}\ sats{\isacharparenleft}{\kern0pt}M{\isacharcomma}{\kern0pt}ifm{\isacharparenleft}{\kern0pt}{\isadigit{0}}{\isacharcomma}{\kern0pt}{\isadigit{1}}{\isacharparenright}{\kern0pt}\ {\isacharcomma}{\kern0pt}\ {\isacharbrackleft}{\kern0pt}x{\isacharcomma}{\kern0pt}\ a{\isacharbrackright}{\kern0pt}{\isacharparenright}{\kern0pt}{\isacharparenright}{\kern0pt}{\isachardoublequoteclose}\isanewline
\ \ \ \ \isacommand{using}\isamarkupfalse%
\ separation{\isacharunderscore}{\kern0pt}ax\ \isacommand{by}\isamarkupfalse%
\ simp\isanewline
\ \ \isacommand{moreover}\isamarkupfalse%
\isanewline
\ \ \isacommand{have}\isamarkupfalse%
\ {\isachardoublequoteopen}{\isacharparenleft}{\kern0pt}{\isasymforall}y{\isasymin}M\ {\isachardot}{\kern0pt}\ y{\isasymin}a\ {\isasymlongrightarrow}\ x{\isasymin}y{\isacharparenright}{\kern0pt}\ {\isasymlongleftrightarrow}\ sats{\isacharparenleft}{\kern0pt}M{\isacharcomma}{\kern0pt}ifm{\isacharparenleft}{\kern0pt}{\isadigit{0}}{\isacharcomma}{\kern0pt}{\isadigit{1}}{\isacharparenright}{\kern0pt}{\isacharcomma}{\kern0pt}{\isacharbrackleft}{\kern0pt}x{\isacharcomma}{\kern0pt}a{\isacharbrackright}{\kern0pt}{\isacharparenright}{\kern0pt}{\isachardoublequoteclose}\isanewline
\ \ \ \ \isakeyword{if}\ {\isachardoublequoteopen}a{\isasymin}M{\isachardoublequoteclose}\ {\isachardoublequoteopen}x{\isasymin}M{\isachardoublequoteclose}\ \isakeyword{for}\ a\ x\isanewline
\ \ \ \ \isacommand{using}\isamarkupfalse%
\ that\ fmsats{\isacharbrackleft}{\kern0pt}of\ {\isachardoublequoteopen}{\isacharbrackleft}{\kern0pt}x{\isacharcomma}{\kern0pt}a{\isacharbrackright}{\kern0pt}{\isachardoublequoteclose}{\isacharbrackright}{\kern0pt}\ \isacommand{by}\isamarkupfalse%
\ simp\isanewline
\ \ \isacommand{ultimately}\isamarkupfalse%
\isanewline
\ \ \isacommand{have}\isamarkupfalse%
\ {\isachardoublequoteopen}{\isasymforall}a{\isasymin}M{\isachardot}{\kern0pt}\ separation{\isacharparenleft}{\kern0pt}{\isacharhash}{\kern0pt}{\isacharhash}{\kern0pt}M{\isacharcomma}{\kern0pt}\ {\isasymlambda}x\ {\isachardot}{\kern0pt}\ {\isasymforall}y{\isasymin}M\ {\isachardot}{\kern0pt}\ y{\isasymin}a\ {\isasymlongrightarrow}\ x{\isasymin}y{\isacharparenright}{\kern0pt}{\isachardoublequoteclose}\isanewline
\ \ \ \ \isacommand{unfolding}\isamarkupfalse%
\ separation{\isacharunderscore}{\kern0pt}def\ \isacommand{by}\isamarkupfalse%
\ simp\isanewline
\ \ \isacommand{with}\isamarkupfalse%
\ {\isacartoucheopen}A{\isasymin}M{\isacartoucheclose}\ \isacommand{show}\isamarkupfalse%
\ {\isacharquery}{\kern0pt}thesis\ \isacommand{by}\isamarkupfalse%
\ simp\isanewline
\isacommand{qed}\isamarkupfalse%
%
\endisatagproof
{\isafoldproof}%
%
\isadelimproof
\isanewline
%
\endisadelimproof
\isanewline
\isanewline
\isanewline
\isacommand{schematic{\isacharunderscore}{\kern0pt}goal}\isamarkupfalse%
\ diff{\isacharunderscore}{\kern0pt}fm{\isacharunderscore}{\kern0pt}auto{\isacharcolon}{\kern0pt}\isanewline
\ \ \isakeyword{assumes}\isanewline
\ \ \ \ {\isachardoublequoteopen}nth{\isacharparenleft}{\kern0pt}i{\isacharcomma}{\kern0pt}env{\isacharparenright}{\kern0pt}\ {\isacharequal}{\kern0pt}\ x{\isachardoublequoteclose}\ {\isachardoublequoteopen}nth{\isacharparenleft}{\kern0pt}j{\isacharcomma}{\kern0pt}env{\isacharparenright}{\kern0pt}\ {\isacharequal}{\kern0pt}\ B{\isachardoublequoteclose}\isanewline
\ \ \ \ {\isachardoublequoteopen}i\ {\isasymin}\ nat{\isachardoublequoteclose}\ {\isachardoublequoteopen}j\ {\isasymin}\ nat{\isachardoublequoteclose}\ {\isachardoublequoteopen}env\ {\isasymin}\ list{\isacharparenleft}{\kern0pt}A{\isacharparenright}{\kern0pt}{\isachardoublequoteclose}\isanewline
\ \ \isakeyword{shows}\isanewline
\ \ \ \ {\isachardoublequoteopen}x{\isasymnotin}B\ {\isasymlongleftrightarrow}\ sats{\isacharparenleft}{\kern0pt}A{\isacharcomma}{\kern0pt}{\isacharquery}{\kern0pt}dfm{\isacharparenleft}{\kern0pt}i{\isacharcomma}{\kern0pt}j{\isacharparenright}{\kern0pt}{\isacharcomma}{\kern0pt}env{\isacharparenright}{\kern0pt}{\isachardoublequoteclose}\isanewline
%
\isadelimproof
\ \ %
\endisadelimproof
%
\isatagproof
\isacommand{by}\isamarkupfalse%
\ {\isacharparenleft}{\kern0pt}insert\ assms\ {\isacharsemicolon}{\kern0pt}\ {\isacharparenleft}{\kern0pt}rule\ sep{\isacharunderscore}{\kern0pt}rules\ {\isacharbar}{\kern0pt}\ simp{\isacharparenright}{\kern0pt}{\isacharplus}{\kern0pt}{\isacharparenright}{\kern0pt}%
\endisatagproof
{\isafoldproof}%
%
\isadelimproof
\isanewline
%
\endisadelimproof
\isanewline
\isacommand{lemma}\isamarkupfalse%
\ diff{\isacharunderscore}{\kern0pt}sep{\isacharunderscore}{\kern0pt}intf\ {\isacharcolon}{\kern0pt}\isanewline
\ \ \isakeyword{assumes}\isanewline
\ \ \ \ {\isachardoublequoteopen}B{\isasymin}M{\isachardoublequoteclose}\isanewline
\ \ \isakeyword{shows}\isanewline
\ \ \ \ {\isachardoublequoteopen}separation{\isacharparenleft}{\kern0pt}{\isacharhash}{\kern0pt}{\isacharhash}{\kern0pt}M{\isacharcomma}{\kern0pt}{\isasymlambda}x\ {\isachardot}{\kern0pt}\ x{\isasymnotin}B{\isacharparenright}{\kern0pt}{\isachardoublequoteclose}\isanewline
%
\isadelimproof
%
\endisadelimproof
%
\isatagproof
\isacommand{proof}\isamarkupfalse%
\ {\isacharminus}{\kern0pt}\isanewline
\ \ \isacommand{obtain}\isamarkupfalse%
\ dfm\ \isakeyword{where}\isanewline
\ \ \ \ fmsats{\isacharcolon}{\kern0pt}{\isachardoublequoteopen}{\isasymAnd}env{\isachardot}{\kern0pt}\ env{\isasymin}list{\isacharparenleft}{\kern0pt}M{\isacharparenright}{\kern0pt}\ {\isasymLongrightarrow}\ nth{\isacharparenleft}{\kern0pt}{\isadigit{0}}{\isacharcomma}{\kern0pt}env{\isacharparenright}{\kern0pt}{\isasymnotin}nth{\isacharparenleft}{\kern0pt}{\isadigit{1}}{\isacharcomma}{\kern0pt}env{\isacharparenright}{\kern0pt}\isanewline
\ \ \ \ {\isasymlongleftrightarrow}\ sats{\isacharparenleft}{\kern0pt}M{\isacharcomma}{\kern0pt}dfm{\isacharparenleft}{\kern0pt}{\isadigit{0}}{\isacharcomma}{\kern0pt}{\isadigit{1}}{\isacharparenright}{\kern0pt}{\isacharcomma}{\kern0pt}env{\isacharparenright}{\kern0pt}{\isachardoublequoteclose}\isanewline
\ \ \ \ \isakeyword{and}\isanewline
\ \ \ \ {\isachardoublequoteopen}dfm{\isacharparenleft}{\kern0pt}{\isadigit{0}}{\isacharcomma}{\kern0pt}{\isadigit{1}}{\isacharparenright}{\kern0pt}\ {\isasymin}\ formula{\isachardoublequoteclose}\isanewline
\ \ \ \ \isakeyword{and}\isanewline
\ \ \ \ {\isachardoublequoteopen}arity{\isacharparenleft}{\kern0pt}dfm{\isacharparenleft}{\kern0pt}{\isadigit{0}}{\isacharcomma}{\kern0pt}{\isadigit{1}}{\isacharparenright}{\kern0pt}{\isacharparenright}{\kern0pt}\ {\isacharequal}{\kern0pt}\ {\isadigit{2}}{\isachardoublequoteclose}\isanewline
\ \ \ \ \isacommand{using}\isamarkupfalse%
\ {\isacartoucheopen}B{\isasymin}M{\isacartoucheclose}\ diff{\isacharunderscore}{\kern0pt}fm{\isacharunderscore}{\kern0pt}auto\isanewline
\ \ \ \ \isacommand{by}\isamarkupfalse%
\ {\isacharparenleft}{\kern0pt}simp\ del{\isacharcolon}{\kern0pt}FOL{\isacharunderscore}{\kern0pt}sats{\isacharunderscore}{\kern0pt}iff\ add{\isacharcolon}{\kern0pt}\ nat{\isacharunderscore}{\kern0pt}simp{\isacharunderscore}{\kern0pt}union{\isacharparenright}{\kern0pt}\isanewline
\ \ \isacommand{then}\isamarkupfalse%
\isanewline
\ \ \isacommand{have}\isamarkupfalse%
\ {\isachardoublequoteopen}{\isasymforall}b{\isasymin}M{\isachardot}{\kern0pt}\ separation{\isacharparenleft}{\kern0pt}{\isacharhash}{\kern0pt}{\isacharhash}{\kern0pt}M{\isacharcomma}{\kern0pt}\ {\isasymlambda}x{\isachardot}{\kern0pt}\ sats{\isacharparenleft}{\kern0pt}M{\isacharcomma}{\kern0pt}dfm{\isacharparenleft}{\kern0pt}{\isadigit{0}}{\isacharcomma}{\kern0pt}{\isadigit{1}}{\isacharparenright}{\kern0pt}\ {\isacharcomma}{\kern0pt}\ {\isacharbrackleft}{\kern0pt}x{\isacharcomma}{\kern0pt}\ b{\isacharbrackright}{\kern0pt}{\isacharparenright}{\kern0pt}{\isacharparenright}{\kern0pt}{\isachardoublequoteclose}\isanewline
\ \ \ \ \isacommand{using}\isamarkupfalse%
\ separation{\isacharunderscore}{\kern0pt}ax\ \isacommand{by}\isamarkupfalse%
\ simp\isanewline
\ \ \isacommand{moreover}\isamarkupfalse%
\isanewline
\ \ \isacommand{have}\isamarkupfalse%
\ {\isachardoublequoteopen}x{\isasymnotin}b\ {\isasymlongleftrightarrow}\ sats{\isacharparenleft}{\kern0pt}M{\isacharcomma}{\kern0pt}dfm{\isacharparenleft}{\kern0pt}{\isadigit{0}}{\isacharcomma}{\kern0pt}{\isadigit{1}}{\isacharparenright}{\kern0pt}{\isacharcomma}{\kern0pt}{\isacharbrackleft}{\kern0pt}x{\isacharcomma}{\kern0pt}b{\isacharbrackright}{\kern0pt}{\isacharparenright}{\kern0pt}{\isachardoublequoteclose}\isanewline
\ \ \ \ \isakeyword{if}\ {\isachardoublequoteopen}b{\isasymin}M{\isachardoublequoteclose}\ {\isachardoublequoteopen}x{\isasymin}M{\isachardoublequoteclose}\ \isakeyword{for}\ b\ x\isanewline
\ \ \ \ \isacommand{using}\isamarkupfalse%
\ that\ fmsats{\isacharbrackleft}{\kern0pt}of\ {\isachardoublequoteopen}{\isacharbrackleft}{\kern0pt}x{\isacharcomma}{\kern0pt}b{\isacharbrackright}{\kern0pt}{\isachardoublequoteclose}{\isacharbrackright}{\kern0pt}\ \isacommand{by}\isamarkupfalse%
\ simp\isanewline
\ \ \isacommand{ultimately}\isamarkupfalse%
\isanewline
\ \ \isacommand{have}\isamarkupfalse%
\ {\isachardoublequoteopen}{\isasymforall}b{\isasymin}M{\isachardot}{\kern0pt}\ separation{\isacharparenleft}{\kern0pt}{\isacharhash}{\kern0pt}{\isacharhash}{\kern0pt}M{\isacharcomma}{\kern0pt}\ {\isasymlambda}x\ {\isachardot}{\kern0pt}\ x{\isasymnotin}b{\isacharparenright}{\kern0pt}{\isachardoublequoteclose}\isanewline
\ \ \ \ \isacommand{unfolding}\isamarkupfalse%
\ separation{\isacharunderscore}{\kern0pt}def\ \isacommand{by}\isamarkupfalse%
\ simp\isanewline
\ \ \isacommand{with}\isamarkupfalse%
\ {\isacartoucheopen}B{\isasymin}M{\isacartoucheclose}\ \isacommand{show}\isamarkupfalse%
\ {\isacharquery}{\kern0pt}thesis\ \isacommand{by}\isamarkupfalse%
\ simp\isanewline
\isacommand{qed}\isamarkupfalse%
%
\endisatagproof
{\isafoldproof}%
%
\isadelimproof
\isanewline
%
\endisadelimproof
\isanewline
\isacommand{schematic{\isacharunderscore}{\kern0pt}goal}\isamarkupfalse%
\ cprod{\isacharunderscore}{\kern0pt}fm{\isacharunderscore}{\kern0pt}auto{\isacharcolon}{\kern0pt}\isanewline
\ \ \isakeyword{assumes}\isanewline
\ \ \ \ {\isachardoublequoteopen}nth{\isacharparenleft}{\kern0pt}i{\isacharcomma}{\kern0pt}env{\isacharparenright}{\kern0pt}\ {\isacharequal}{\kern0pt}\ z{\isachardoublequoteclose}\ {\isachardoublequoteopen}nth{\isacharparenleft}{\kern0pt}j{\isacharcomma}{\kern0pt}env{\isacharparenright}{\kern0pt}\ {\isacharequal}{\kern0pt}\ B{\isachardoublequoteclose}\ {\isachardoublequoteopen}nth{\isacharparenleft}{\kern0pt}h{\isacharcomma}{\kern0pt}env{\isacharparenright}{\kern0pt}\ {\isacharequal}{\kern0pt}\ C{\isachardoublequoteclose}\isanewline
\ \ \ \ {\isachardoublequoteopen}i\ {\isasymin}\ nat{\isachardoublequoteclose}\ {\isachardoublequoteopen}j\ {\isasymin}\ nat{\isachardoublequoteclose}\ {\isachardoublequoteopen}h\ {\isasymin}\ nat{\isachardoublequoteclose}\ {\isachardoublequoteopen}env\ {\isasymin}\ list{\isacharparenleft}{\kern0pt}A{\isacharparenright}{\kern0pt}{\isachardoublequoteclose}\isanewline
\ \ \isakeyword{shows}\isanewline
\ \ \ \ {\isachardoublequoteopen}{\isacharparenleft}{\kern0pt}{\isasymexists}x{\isasymin}A{\isachardot}{\kern0pt}\ x{\isasymin}B\ {\isasymand}\ {\isacharparenleft}{\kern0pt}{\isasymexists}y{\isasymin}A{\isachardot}{\kern0pt}\ y{\isasymin}C\ {\isasymand}\ pair{\isacharparenleft}{\kern0pt}{\isacharhash}{\kern0pt}{\isacharhash}{\kern0pt}A{\isacharcomma}{\kern0pt}x{\isacharcomma}{\kern0pt}y{\isacharcomma}{\kern0pt}z{\isacharparenright}{\kern0pt}{\isacharparenright}{\kern0pt}{\isacharparenright}{\kern0pt}\ {\isasymlongleftrightarrow}\ sats{\isacharparenleft}{\kern0pt}A{\isacharcomma}{\kern0pt}{\isacharquery}{\kern0pt}cpfm{\isacharparenleft}{\kern0pt}i{\isacharcomma}{\kern0pt}j{\isacharcomma}{\kern0pt}h{\isacharparenright}{\kern0pt}{\isacharcomma}{\kern0pt}env{\isacharparenright}{\kern0pt}{\isachardoublequoteclose}\isanewline
%
\isadelimproof
\ \ %
\endisadelimproof
%
\isatagproof
\isacommand{by}\isamarkupfalse%
\ {\isacharparenleft}{\kern0pt}insert\ assms\ {\isacharsemicolon}{\kern0pt}\ {\isacharparenleft}{\kern0pt}rule\ sep{\isacharunderscore}{\kern0pt}rules\ {\isacharbar}{\kern0pt}\ simp{\isacharparenright}{\kern0pt}{\isacharplus}{\kern0pt}{\isacharparenright}{\kern0pt}%
\endisatagproof
{\isafoldproof}%
%
\isadelimproof
\isanewline
%
\endisadelimproof
\isanewline
\isanewline
\isacommand{lemma}\isamarkupfalse%
\ cartprod{\isacharunderscore}{\kern0pt}sep{\isacharunderscore}{\kern0pt}intf\ {\isacharcolon}{\kern0pt}\isanewline
\ \ \isakeyword{assumes}\isanewline
\ \ \ \ {\isachardoublequoteopen}A{\isasymin}M{\isachardoublequoteclose}\isanewline
\ \ \ \ \isakeyword{and}\isanewline
\ \ \ \ {\isachardoublequoteopen}B{\isasymin}M{\isachardoublequoteclose}\isanewline
\ \ \isakeyword{shows}\isanewline
\ \ \ \ {\isachardoublequoteopen}separation{\isacharparenleft}{\kern0pt}{\isacharhash}{\kern0pt}{\isacharhash}{\kern0pt}M{\isacharcomma}{\kern0pt}{\isasymlambda}z{\isachardot}{\kern0pt}\ {\isasymexists}x{\isasymin}M{\isachardot}{\kern0pt}\ x{\isasymin}A\ {\isasymand}\ {\isacharparenleft}{\kern0pt}{\isasymexists}y{\isasymin}M{\isachardot}{\kern0pt}\ y{\isasymin}B\ {\isasymand}\ pair{\isacharparenleft}{\kern0pt}{\isacharhash}{\kern0pt}{\isacharhash}{\kern0pt}M{\isacharcomma}{\kern0pt}x{\isacharcomma}{\kern0pt}y{\isacharcomma}{\kern0pt}z{\isacharparenright}{\kern0pt}{\isacharparenright}{\kern0pt}{\isacharparenright}{\kern0pt}{\isachardoublequoteclose}\isanewline
%
\isadelimproof
%
\endisadelimproof
%
\isatagproof
\isacommand{proof}\isamarkupfalse%
\ {\isacharminus}{\kern0pt}\isanewline
\ \ \isacommand{obtain}\isamarkupfalse%
\ cpfm\ \isakeyword{where}\isanewline
\ \ \ \ fmsats{\isacharcolon}{\kern0pt}{\isachardoublequoteopen}{\isasymAnd}env{\isachardot}{\kern0pt}\ env{\isasymin}list{\isacharparenleft}{\kern0pt}M{\isacharparenright}{\kern0pt}\ {\isasymLongrightarrow}\isanewline
\ \ \ \ {\isacharparenleft}{\kern0pt}{\isasymexists}x{\isasymin}M{\isachardot}{\kern0pt}\ x{\isasymin}nth{\isacharparenleft}{\kern0pt}{\isadigit{1}}{\isacharcomma}{\kern0pt}env{\isacharparenright}{\kern0pt}\ {\isasymand}\ {\isacharparenleft}{\kern0pt}{\isasymexists}y{\isasymin}M{\isachardot}{\kern0pt}\ y{\isasymin}nth{\isacharparenleft}{\kern0pt}{\isadigit{2}}{\isacharcomma}{\kern0pt}env{\isacharparenright}{\kern0pt}\ {\isasymand}\ pair{\isacharparenleft}{\kern0pt}{\isacharhash}{\kern0pt}{\isacharhash}{\kern0pt}M{\isacharcomma}{\kern0pt}x{\isacharcomma}{\kern0pt}y{\isacharcomma}{\kern0pt}nth{\isacharparenleft}{\kern0pt}{\isadigit{0}}{\isacharcomma}{\kern0pt}env{\isacharparenright}{\kern0pt}{\isacharparenright}{\kern0pt}{\isacharparenright}{\kern0pt}{\isacharparenright}{\kern0pt}\isanewline
\ \ \ \ {\isasymlongleftrightarrow}\ sats{\isacharparenleft}{\kern0pt}M{\isacharcomma}{\kern0pt}cpfm{\isacharparenleft}{\kern0pt}{\isadigit{0}}{\isacharcomma}{\kern0pt}{\isadigit{1}}{\isacharcomma}{\kern0pt}{\isadigit{2}}{\isacharparenright}{\kern0pt}{\isacharcomma}{\kern0pt}env{\isacharparenright}{\kern0pt}{\isachardoublequoteclose}\isanewline
\ \ \ \ \isakeyword{and}\isanewline
\ \ \ \ {\isachardoublequoteopen}cpfm{\isacharparenleft}{\kern0pt}{\isadigit{0}}{\isacharcomma}{\kern0pt}{\isadigit{1}}{\isacharcomma}{\kern0pt}{\isadigit{2}}{\isacharparenright}{\kern0pt}\ {\isasymin}\ formula{\isachardoublequoteclose}\isanewline
\ \ \ \ \isakeyword{and}\isanewline
\ \ \ \ {\isachardoublequoteopen}arity{\isacharparenleft}{\kern0pt}cpfm{\isacharparenleft}{\kern0pt}{\isadigit{0}}{\isacharcomma}{\kern0pt}{\isadigit{1}}{\isacharcomma}{\kern0pt}{\isadigit{2}}{\isacharparenright}{\kern0pt}{\isacharparenright}{\kern0pt}\ {\isacharequal}{\kern0pt}\ {\isadigit{3}}{\isachardoublequoteclose}\isanewline
\ \ \ \ \isacommand{using}\isamarkupfalse%
\ cprod{\isacharunderscore}{\kern0pt}fm{\isacharunderscore}{\kern0pt}auto\ \isacommand{by}\isamarkupfalse%
\ {\isacharparenleft}{\kern0pt}simp\ del{\isacharcolon}{\kern0pt}FOL{\isacharunderscore}{\kern0pt}sats{\isacharunderscore}{\kern0pt}iff\ add{\isacharcolon}{\kern0pt}\ fm{\isacharunderscore}{\kern0pt}defs\ nat{\isacharunderscore}{\kern0pt}simp{\isacharunderscore}{\kern0pt}union{\isacharparenright}{\kern0pt}\isanewline
\ \ \isacommand{then}\isamarkupfalse%
\isanewline
\ \ \isacommand{have}\isamarkupfalse%
\ {\isachardoublequoteopen}{\isasymforall}a{\isasymin}M{\isachardot}{\kern0pt}\ {\isasymforall}b{\isasymin}M{\isachardot}{\kern0pt}\ separation{\isacharparenleft}{\kern0pt}{\isacharhash}{\kern0pt}{\isacharhash}{\kern0pt}M{\isacharcomma}{\kern0pt}\ {\isasymlambda}z{\isachardot}{\kern0pt}\ sats{\isacharparenleft}{\kern0pt}M{\isacharcomma}{\kern0pt}cpfm{\isacharparenleft}{\kern0pt}{\isadigit{0}}{\isacharcomma}{\kern0pt}{\isadigit{1}}{\isacharcomma}{\kern0pt}{\isadigit{2}}{\isacharparenright}{\kern0pt}\ {\isacharcomma}{\kern0pt}\ {\isacharbrackleft}{\kern0pt}z{\isacharcomma}{\kern0pt}\ a{\isacharcomma}{\kern0pt}\ b{\isacharbrackright}{\kern0pt}{\isacharparenright}{\kern0pt}{\isacharparenright}{\kern0pt}{\isachardoublequoteclose}\isanewline
\ \ \ \ \isacommand{using}\isamarkupfalse%
\ separation{\isacharunderscore}{\kern0pt}ax\ \isacommand{by}\isamarkupfalse%
\ simp\isanewline
\ \ \isacommand{moreover}\isamarkupfalse%
\isanewline
\ \ \isacommand{have}\isamarkupfalse%
\ {\isachardoublequoteopen}{\isacharparenleft}{\kern0pt}{\isasymexists}x{\isasymin}M{\isachardot}{\kern0pt}\ x{\isasymin}a\ {\isasymand}\ {\isacharparenleft}{\kern0pt}{\isasymexists}y{\isasymin}M{\isachardot}{\kern0pt}\ y{\isasymin}b\ {\isasymand}\ pair{\isacharparenleft}{\kern0pt}{\isacharhash}{\kern0pt}{\isacharhash}{\kern0pt}M{\isacharcomma}{\kern0pt}x{\isacharcomma}{\kern0pt}y{\isacharcomma}{\kern0pt}z{\isacharparenright}{\kern0pt}{\isacharparenright}{\kern0pt}{\isacharparenright}{\kern0pt}\ {\isasymlongleftrightarrow}\ sats{\isacharparenleft}{\kern0pt}M{\isacharcomma}{\kern0pt}cpfm{\isacharparenleft}{\kern0pt}{\isadigit{0}}{\isacharcomma}{\kern0pt}{\isadigit{1}}{\isacharcomma}{\kern0pt}{\isadigit{2}}{\isacharparenright}{\kern0pt}{\isacharcomma}{\kern0pt}{\isacharbrackleft}{\kern0pt}z{\isacharcomma}{\kern0pt}a{\isacharcomma}{\kern0pt}b{\isacharbrackright}{\kern0pt}{\isacharparenright}{\kern0pt}{\isachardoublequoteclose}\isanewline
\ \ \ \ \isakeyword{if}\ {\isachardoublequoteopen}a{\isasymin}M{\isachardoublequoteclose}\ {\isachardoublequoteopen}b{\isasymin}M{\isachardoublequoteclose}\ {\isachardoublequoteopen}z{\isasymin}M{\isachardoublequoteclose}\ \isakeyword{for}\ a\ b\ z\isanewline
\ \ \ \ \isacommand{using}\isamarkupfalse%
\ that\ fmsats{\isacharbrackleft}{\kern0pt}of\ {\isachardoublequoteopen}{\isacharbrackleft}{\kern0pt}z{\isacharcomma}{\kern0pt}a{\isacharcomma}{\kern0pt}b{\isacharbrackright}{\kern0pt}{\isachardoublequoteclose}{\isacharbrackright}{\kern0pt}\ \isacommand{by}\isamarkupfalse%
\ simp\isanewline
\ \ \isacommand{ultimately}\isamarkupfalse%
\isanewline
\ \ \isacommand{have}\isamarkupfalse%
\ {\isachardoublequoteopen}{\isasymforall}a{\isasymin}M{\isachardot}{\kern0pt}\ {\isasymforall}b{\isasymin}M{\isachardot}{\kern0pt}\ separation{\isacharparenleft}{\kern0pt}{\isacharhash}{\kern0pt}{\isacharhash}{\kern0pt}M{\isacharcomma}{\kern0pt}\ {\isasymlambda}z\ {\isachardot}{\kern0pt}\ {\isacharparenleft}{\kern0pt}{\isasymexists}x{\isasymin}M{\isachardot}{\kern0pt}\ x{\isasymin}a\ {\isasymand}\ {\isacharparenleft}{\kern0pt}{\isasymexists}y{\isasymin}M{\isachardot}{\kern0pt}\ y{\isasymin}b\ {\isasymand}\ pair{\isacharparenleft}{\kern0pt}{\isacharhash}{\kern0pt}{\isacharhash}{\kern0pt}M{\isacharcomma}{\kern0pt}x{\isacharcomma}{\kern0pt}y{\isacharcomma}{\kern0pt}z{\isacharparenright}{\kern0pt}{\isacharparenright}{\kern0pt}{\isacharparenright}{\kern0pt}{\isacharparenright}{\kern0pt}{\isachardoublequoteclose}\isanewline
\ \ \ \ \isacommand{unfolding}\isamarkupfalse%
\ separation{\isacharunderscore}{\kern0pt}def\ \isacommand{by}\isamarkupfalse%
\ simp\isanewline
\ \ \isacommand{with}\isamarkupfalse%
\ {\isacartoucheopen}A{\isasymin}M{\isacartoucheclose}\ {\isacartoucheopen}B{\isasymin}M{\isacartoucheclose}\ \isacommand{show}\isamarkupfalse%
\ {\isacharquery}{\kern0pt}thesis\ \isacommand{by}\isamarkupfalse%
\ simp\isanewline
\isacommand{qed}\isamarkupfalse%
%
\endisatagproof
{\isafoldproof}%
%
\isadelimproof
\isanewline
%
\endisadelimproof
\isanewline
\isacommand{schematic{\isacharunderscore}{\kern0pt}goal}\isamarkupfalse%
\ im{\isacharunderscore}{\kern0pt}fm{\isacharunderscore}{\kern0pt}auto{\isacharcolon}{\kern0pt}\isanewline
\ \ \isakeyword{assumes}\isanewline
\ \ \ \ {\isachardoublequoteopen}nth{\isacharparenleft}{\kern0pt}i{\isacharcomma}{\kern0pt}env{\isacharparenright}{\kern0pt}\ {\isacharequal}{\kern0pt}\ y{\isachardoublequoteclose}\ {\isachardoublequoteopen}nth{\isacharparenleft}{\kern0pt}j{\isacharcomma}{\kern0pt}env{\isacharparenright}{\kern0pt}\ {\isacharequal}{\kern0pt}\ r{\isachardoublequoteclose}\ {\isachardoublequoteopen}nth{\isacharparenleft}{\kern0pt}h{\isacharcomma}{\kern0pt}env{\isacharparenright}{\kern0pt}\ {\isacharequal}{\kern0pt}\ B{\isachardoublequoteclose}\isanewline
\ \ \ \ {\isachardoublequoteopen}i\ {\isasymin}\ nat{\isachardoublequoteclose}\ {\isachardoublequoteopen}j\ {\isasymin}\ nat{\isachardoublequoteclose}\ {\isachardoublequoteopen}h\ {\isasymin}\ nat{\isachardoublequoteclose}\ {\isachardoublequoteopen}env\ {\isasymin}\ list{\isacharparenleft}{\kern0pt}A{\isacharparenright}{\kern0pt}{\isachardoublequoteclose}\isanewline
\ \ \isakeyword{shows}\isanewline
\ \ \ \ {\isachardoublequoteopen}{\isacharparenleft}{\kern0pt}{\isasymexists}p{\isasymin}A{\isachardot}{\kern0pt}\ p{\isasymin}r\ {\isacharampersand}{\kern0pt}\ {\isacharparenleft}{\kern0pt}{\isasymexists}x{\isasymin}A{\isachardot}{\kern0pt}\ x{\isasymin}B\ {\isacharampersand}{\kern0pt}\ pair{\isacharparenleft}{\kern0pt}{\isacharhash}{\kern0pt}{\isacharhash}{\kern0pt}A{\isacharcomma}{\kern0pt}x{\isacharcomma}{\kern0pt}y{\isacharcomma}{\kern0pt}p{\isacharparenright}{\kern0pt}{\isacharparenright}{\kern0pt}{\isacharparenright}{\kern0pt}\ {\isasymlongleftrightarrow}\ sats{\isacharparenleft}{\kern0pt}A{\isacharcomma}{\kern0pt}{\isacharquery}{\kern0pt}imfm{\isacharparenleft}{\kern0pt}i{\isacharcomma}{\kern0pt}j{\isacharcomma}{\kern0pt}h{\isacharparenright}{\kern0pt}{\isacharcomma}{\kern0pt}env{\isacharparenright}{\kern0pt}{\isachardoublequoteclose}\isanewline
%
\isadelimproof
\ \ %
\endisadelimproof
%
\isatagproof
\isacommand{by}\isamarkupfalse%
\ {\isacharparenleft}{\kern0pt}insert\ assms\ {\isacharsemicolon}{\kern0pt}\ {\isacharparenleft}{\kern0pt}rule\ sep{\isacharunderscore}{\kern0pt}rules\ {\isacharbar}{\kern0pt}\ simp{\isacharparenright}{\kern0pt}{\isacharplus}{\kern0pt}{\isacharparenright}{\kern0pt}%
\endisatagproof
{\isafoldproof}%
%
\isadelimproof
\isanewline
%
\endisadelimproof
\isanewline
\isacommand{lemma}\isamarkupfalse%
\ image{\isacharunderscore}{\kern0pt}sep{\isacharunderscore}{\kern0pt}intf\ {\isacharcolon}{\kern0pt}\isanewline
\ \ \isakeyword{assumes}\isanewline
\ \ \ \ {\isachardoublequoteopen}A{\isasymin}M{\isachardoublequoteclose}\isanewline
\ \ \ \ \isakeyword{and}\isanewline
\ \ \ \ {\isachardoublequoteopen}r{\isasymin}M{\isachardoublequoteclose}\isanewline
\ \ \isakeyword{shows}\isanewline
\ \ \ \ {\isachardoublequoteopen}separation{\isacharparenleft}{\kern0pt}{\isacharhash}{\kern0pt}{\isacharhash}{\kern0pt}M{\isacharcomma}{\kern0pt}\ {\isasymlambda}y{\isachardot}{\kern0pt}\ {\isasymexists}p{\isasymin}M{\isachardot}{\kern0pt}\ p{\isasymin}r\ {\isacharampersand}{\kern0pt}\ {\isacharparenleft}{\kern0pt}{\isasymexists}x{\isasymin}M{\isachardot}{\kern0pt}\ x{\isasymin}A\ {\isacharampersand}{\kern0pt}\ pair{\isacharparenleft}{\kern0pt}{\isacharhash}{\kern0pt}{\isacharhash}{\kern0pt}M{\isacharcomma}{\kern0pt}x{\isacharcomma}{\kern0pt}y{\isacharcomma}{\kern0pt}p{\isacharparenright}{\kern0pt}{\isacharparenright}{\kern0pt}{\isacharparenright}{\kern0pt}{\isachardoublequoteclose}\isanewline
%
\isadelimproof
%
\endisadelimproof
%
\isatagproof
\isacommand{proof}\isamarkupfalse%
\ {\isacharminus}{\kern0pt}\isanewline
\ \ \isacommand{obtain}\isamarkupfalse%
\ imfm\ \isakeyword{where}\isanewline
\ \ \ \ fmsats{\isacharcolon}{\kern0pt}{\isachardoublequoteopen}{\isasymAnd}env{\isachardot}{\kern0pt}\ env{\isasymin}list{\isacharparenleft}{\kern0pt}M{\isacharparenright}{\kern0pt}\ {\isasymLongrightarrow}\isanewline
\ \ \ \ {\isacharparenleft}{\kern0pt}{\isasymexists}p{\isasymin}M{\isachardot}{\kern0pt}\ p{\isasymin}nth{\isacharparenleft}{\kern0pt}{\isadigit{1}}{\isacharcomma}{\kern0pt}env{\isacharparenright}{\kern0pt}\ {\isacharampersand}{\kern0pt}\ {\isacharparenleft}{\kern0pt}{\isasymexists}x{\isasymin}M{\isachardot}{\kern0pt}\ x{\isasymin}nth{\isacharparenleft}{\kern0pt}{\isadigit{2}}{\isacharcomma}{\kern0pt}env{\isacharparenright}{\kern0pt}\ {\isacharampersand}{\kern0pt}\ pair{\isacharparenleft}{\kern0pt}{\isacharhash}{\kern0pt}{\isacharhash}{\kern0pt}M{\isacharcomma}{\kern0pt}x{\isacharcomma}{\kern0pt}nth{\isacharparenleft}{\kern0pt}{\isadigit{0}}{\isacharcomma}{\kern0pt}env{\isacharparenright}{\kern0pt}{\isacharcomma}{\kern0pt}p{\isacharparenright}{\kern0pt}{\isacharparenright}{\kern0pt}{\isacharparenright}{\kern0pt}\isanewline
\ \ \ \ {\isasymlongleftrightarrow}\ sats{\isacharparenleft}{\kern0pt}M{\isacharcomma}{\kern0pt}imfm{\isacharparenleft}{\kern0pt}{\isadigit{0}}{\isacharcomma}{\kern0pt}{\isadigit{1}}{\isacharcomma}{\kern0pt}{\isadigit{2}}{\isacharparenright}{\kern0pt}{\isacharcomma}{\kern0pt}env{\isacharparenright}{\kern0pt}{\isachardoublequoteclose}\isanewline
\ \ \ \ \isakeyword{and}\isanewline
\ \ \ \ {\isachardoublequoteopen}imfm{\isacharparenleft}{\kern0pt}{\isadigit{0}}{\isacharcomma}{\kern0pt}{\isadigit{1}}{\isacharcomma}{\kern0pt}{\isadigit{2}}{\isacharparenright}{\kern0pt}\ {\isasymin}\ formula{\isachardoublequoteclose}\isanewline
\ \ \ \ \isakeyword{and}\isanewline
\ \ \ \ {\isachardoublequoteopen}arity{\isacharparenleft}{\kern0pt}imfm{\isacharparenleft}{\kern0pt}{\isadigit{0}}{\isacharcomma}{\kern0pt}{\isadigit{1}}{\isacharcomma}{\kern0pt}{\isadigit{2}}{\isacharparenright}{\kern0pt}{\isacharparenright}{\kern0pt}\ {\isacharequal}{\kern0pt}\ {\isadigit{3}}{\isachardoublequoteclose}\isanewline
\ \ \ \ \isacommand{using}\isamarkupfalse%
\ im{\isacharunderscore}{\kern0pt}fm{\isacharunderscore}{\kern0pt}auto\ \isacommand{by}\isamarkupfalse%
\ {\isacharparenleft}{\kern0pt}simp\ del{\isacharcolon}{\kern0pt}FOL{\isacharunderscore}{\kern0pt}sats{\isacharunderscore}{\kern0pt}iff\ pair{\isacharunderscore}{\kern0pt}abs\ add{\isacharcolon}{\kern0pt}\ fm{\isacharunderscore}{\kern0pt}defs\ nat{\isacharunderscore}{\kern0pt}simp{\isacharunderscore}{\kern0pt}union{\isacharparenright}{\kern0pt}\isanewline
\ \ \isacommand{then}\isamarkupfalse%
\isanewline
\ \ \isacommand{have}\isamarkupfalse%
\ {\isachardoublequoteopen}{\isasymforall}r{\isasymin}M{\isachardot}{\kern0pt}\ {\isasymforall}a{\isasymin}M{\isachardot}{\kern0pt}\ separation{\isacharparenleft}{\kern0pt}{\isacharhash}{\kern0pt}{\isacharhash}{\kern0pt}M{\isacharcomma}{\kern0pt}\ {\isasymlambda}y{\isachardot}{\kern0pt}\ sats{\isacharparenleft}{\kern0pt}M{\isacharcomma}{\kern0pt}imfm{\isacharparenleft}{\kern0pt}{\isadigit{0}}{\isacharcomma}{\kern0pt}{\isadigit{1}}{\isacharcomma}{\kern0pt}{\isadigit{2}}{\isacharparenright}{\kern0pt}\ {\isacharcomma}{\kern0pt}\ {\isacharbrackleft}{\kern0pt}y{\isacharcomma}{\kern0pt}r{\isacharcomma}{\kern0pt}a{\isacharbrackright}{\kern0pt}{\isacharparenright}{\kern0pt}{\isacharparenright}{\kern0pt}{\isachardoublequoteclose}\isanewline
\ \ \ \ \isacommand{using}\isamarkupfalse%
\ separation{\isacharunderscore}{\kern0pt}ax\ \isacommand{by}\isamarkupfalse%
\ simp\isanewline
\ \ \isacommand{moreover}\isamarkupfalse%
\isanewline
\ \ \isacommand{have}\isamarkupfalse%
\ {\isachardoublequoteopen}{\isacharparenleft}{\kern0pt}{\isasymexists}p{\isasymin}M{\isachardot}{\kern0pt}\ p{\isasymin}k\ {\isacharampersand}{\kern0pt}\ {\isacharparenleft}{\kern0pt}{\isasymexists}x{\isasymin}M{\isachardot}{\kern0pt}\ x{\isasymin}a\ {\isacharampersand}{\kern0pt}\ pair{\isacharparenleft}{\kern0pt}{\isacharhash}{\kern0pt}{\isacharhash}{\kern0pt}M{\isacharcomma}{\kern0pt}x{\isacharcomma}{\kern0pt}y{\isacharcomma}{\kern0pt}p{\isacharparenright}{\kern0pt}{\isacharparenright}{\kern0pt}{\isacharparenright}{\kern0pt}\ {\isasymlongleftrightarrow}\ sats{\isacharparenleft}{\kern0pt}M{\isacharcomma}{\kern0pt}imfm{\isacharparenleft}{\kern0pt}{\isadigit{0}}{\isacharcomma}{\kern0pt}{\isadigit{1}}{\isacharcomma}{\kern0pt}{\isadigit{2}}{\isacharparenright}{\kern0pt}{\isacharcomma}{\kern0pt}{\isacharbrackleft}{\kern0pt}y{\isacharcomma}{\kern0pt}k{\isacharcomma}{\kern0pt}a{\isacharbrackright}{\kern0pt}{\isacharparenright}{\kern0pt}{\isachardoublequoteclose}\isanewline
\ \ \ \ \isakeyword{if}\ {\isachardoublequoteopen}k{\isasymin}M{\isachardoublequoteclose}\ {\isachardoublequoteopen}a{\isasymin}M{\isachardoublequoteclose}\ {\isachardoublequoteopen}y{\isasymin}M{\isachardoublequoteclose}\ \isakeyword{for}\ k\ a\ y\isanewline
\ \ \ \ \isacommand{using}\isamarkupfalse%
\ that\ fmsats{\isacharbrackleft}{\kern0pt}of\ {\isachardoublequoteopen}{\isacharbrackleft}{\kern0pt}y{\isacharcomma}{\kern0pt}k{\isacharcomma}{\kern0pt}a{\isacharbrackright}{\kern0pt}{\isachardoublequoteclose}{\isacharbrackright}{\kern0pt}\ \isacommand{by}\isamarkupfalse%
\ simp\isanewline
\ \ \isacommand{ultimately}\isamarkupfalse%
\isanewline
\ \ \isacommand{have}\isamarkupfalse%
\ {\isachardoublequoteopen}{\isasymforall}k{\isasymin}M{\isachardot}{\kern0pt}\ {\isasymforall}a{\isasymin}M{\isachardot}{\kern0pt}\ separation{\isacharparenleft}{\kern0pt}{\isacharhash}{\kern0pt}{\isacharhash}{\kern0pt}M{\isacharcomma}{\kern0pt}\ {\isasymlambda}y\ {\isachardot}{\kern0pt}\ {\isasymexists}p{\isasymin}M{\isachardot}{\kern0pt}\ p{\isasymin}k\ {\isacharampersand}{\kern0pt}\ {\isacharparenleft}{\kern0pt}{\isasymexists}x{\isasymin}M{\isachardot}{\kern0pt}\ x{\isasymin}a\ {\isacharampersand}{\kern0pt}\ pair{\isacharparenleft}{\kern0pt}{\isacharhash}{\kern0pt}{\isacharhash}{\kern0pt}M{\isacharcomma}{\kern0pt}x{\isacharcomma}{\kern0pt}y{\isacharcomma}{\kern0pt}p{\isacharparenright}{\kern0pt}{\isacharparenright}{\kern0pt}{\isacharparenright}{\kern0pt}{\isachardoublequoteclose}\isanewline
\ \ \ \ \isacommand{unfolding}\isamarkupfalse%
\ separation{\isacharunderscore}{\kern0pt}def\ \isacommand{by}\isamarkupfalse%
\ simp\isanewline
\ \ \isacommand{with}\isamarkupfalse%
\ {\isacartoucheopen}r{\isasymin}M{\isacartoucheclose}\ {\isacartoucheopen}A{\isasymin}M{\isacartoucheclose}\ \isacommand{show}\isamarkupfalse%
\ {\isacharquery}{\kern0pt}thesis\ \isacommand{by}\isamarkupfalse%
\ simp\isanewline
\isacommand{qed}\isamarkupfalse%
%
\endisatagproof
{\isafoldproof}%
%
\isadelimproof
\isanewline
%
\endisadelimproof
\isanewline
\isacommand{schematic{\isacharunderscore}{\kern0pt}goal}\isamarkupfalse%
\ con{\isacharunderscore}{\kern0pt}fm{\isacharunderscore}{\kern0pt}auto{\isacharcolon}{\kern0pt}\isanewline
\ \ \isakeyword{assumes}\isanewline
\ \ \ \ {\isachardoublequoteopen}nth{\isacharparenleft}{\kern0pt}i{\isacharcomma}{\kern0pt}env{\isacharparenright}{\kern0pt}\ {\isacharequal}{\kern0pt}\ z{\isachardoublequoteclose}\ {\isachardoublequoteopen}nth{\isacharparenleft}{\kern0pt}j{\isacharcomma}{\kern0pt}env{\isacharparenright}{\kern0pt}\ {\isacharequal}{\kern0pt}\ R{\isachardoublequoteclose}\isanewline
\ \ \ \ {\isachardoublequoteopen}i\ {\isasymin}\ nat{\isachardoublequoteclose}\ {\isachardoublequoteopen}j\ {\isasymin}\ nat{\isachardoublequoteclose}\ {\isachardoublequoteopen}env\ {\isasymin}\ list{\isacharparenleft}{\kern0pt}A{\isacharparenright}{\kern0pt}{\isachardoublequoteclose}\isanewline
\ \ \isakeyword{shows}\isanewline
\ \ \ \ {\isachardoublequoteopen}{\isacharparenleft}{\kern0pt}{\isasymexists}p{\isasymin}A{\isachardot}{\kern0pt}\ p{\isasymin}R\ {\isacharampersand}{\kern0pt}\ {\isacharparenleft}{\kern0pt}{\isasymexists}x{\isasymin}A{\isachardot}{\kern0pt}{\isasymexists}y{\isasymin}A{\isachardot}{\kern0pt}\ pair{\isacharparenleft}{\kern0pt}{\isacharhash}{\kern0pt}{\isacharhash}{\kern0pt}A{\isacharcomma}{\kern0pt}x{\isacharcomma}{\kern0pt}y{\isacharcomma}{\kern0pt}p{\isacharparenright}{\kern0pt}\ {\isacharampersand}{\kern0pt}\ pair{\isacharparenleft}{\kern0pt}{\isacharhash}{\kern0pt}{\isacharhash}{\kern0pt}A{\isacharcomma}{\kern0pt}y{\isacharcomma}{\kern0pt}x{\isacharcomma}{\kern0pt}z{\isacharparenright}{\kern0pt}{\isacharparenright}{\kern0pt}{\isacharparenright}{\kern0pt}\isanewline
\ \ {\isasymlongleftrightarrow}\ sats{\isacharparenleft}{\kern0pt}A{\isacharcomma}{\kern0pt}{\isacharquery}{\kern0pt}cfm{\isacharparenleft}{\kern0pt}i{\isacharcomma}{\kern0pt}j{\isacharparenright}{\kern0pt}{\isacharcomma}{\kern0pt}env{\isacharparenright}{\kern0pt}{\isachardoublequoteclose}\isanewline
%
\isadelimproof
\ \ %
\endisadelimproof
%
\isatagproof
\isacommand{by}\isamarkupfalse%
\ {\isacharparenleft}{\kern0pt}insert\ assms\ {\isacharsemicolon}{\kern0pt}\ {\isacharparenleft}{\kern0pt}rule\ sep{\isacharunderscore}{\kern0pt}rules\ {\isacharbar}{\kern0pt}\ simp{\isacharparenright}{\kern0pt}{\isacharplus}{\kern0pt}{\isacharparenright}{\kern0pt}%
\endisatagproof
{\isafoldproof}%
%
\isadelimproof
\isanewline
%
\endisadelimproof
\isanewline
\isanewline
\isacommand{lemma}\isamarkupfalse%
\ converse{\isacharunderscore}{\kern0pt}sep{\isacharunderscore}{\kern0pt}intf\ {\isacharcolon}{\kern0pt}\isanewline
\ \ \isakeyword{assumes}\isanewline
\ \ \ \ {\isachardoublequoteopen}R{\isasymin}M{\isachardoublequoteclose}\isanewline
\ \ \isakeyword{shows}\isanewline
\ \ \ \ {\isachardoublequoteopen}separation{\isacharparenleft}{\kern0pt}{\isacharhash}{\kern0pt}{\isacharhash}{\kern0pt}M{\isacharcomma}{\kern0pt}{\isasymlambda}z{\isachardot}{\kern0pt}\ {\isasymexists}p{\isasymin}M{\isachardot}{\kern0pt}\ p{\isasymin}R\ {\isacharampersand}{\kern0pt}\ {\isacharparenleft}{\kern0pt}{\isasymexists}x{\isasymin}M{\isachardot}{\kern0pt}{\isasymexists}y{\isasymin}M{\isachardot}{\kern0pt}\ pair{\isacharparenleft}{\kern0pt}{\isacharhash}{\kern0pt}{\isacharhash}{\kern0pt}M{\isacharcomma}{\kern0pt}x{\isacharcomma}{\kern0pt}y{\isacharcomma}{\kern0pt}p{\isacharparenright}{\kern0pt}\ {\isacharampersand}{\kern0pt}\ pair{\isacharparenleft}{\kern0pt}{\isacharhash}{\kern0pt}{\isacharhash}{\kern0pt}M{\isacharcomma}{\kern0pt}y{\isacharcomma}{\kern0pt}x{\isacharcomma}{\kern0pt}z{\isacharparenright}{\kern0pt}{\isacharparenright}{\kern0pt}{\isacharparenright}{\kern0pt}{\isachardoublequoteclose}\isanewline
%
\isadelimproof
%
\endisadelimproof
%
\isatagproof
\isacommand{proof}\isamarkupfalse%
\ {\isacharminus}{\kern0pt}\isanewline
\ \ \isacommand{obtain}\isamarkupfalse%
\ cfm\ \isakeyword{where}\isanewline
\ \ \ \ fmsats{\isacharcolon}{\kern0pt}{\isachardoublequoteopen}{\isasymAnd}env{\isachardot}{\kern0pt}\ env{\isasymin}list{\isacharparenleft}{\kern0pt}M{\isacharparenright}{\kern0pt}\ {\isasymLongrightarrow}\isanewline
\ \ \ \ {\isacharparenleft}{\kern0pt}{\isasymexists}p{\isasymin}M{\isachardot}{\kern0pt}\ p{\isasymin}nth{\isacharparenleft}{\kern0pt}{\isadigit{1}}{\isacharcomma}{\kern0pt}env{\isacharparenright}{\kern0pt}\ {\isacharampersand}{\kern0pt}\ {\isacharparenleft}{\kern0pt}{\isasymexists}x{\isasymin}M{\isachardot}{\kern0pt}{\isasymexists}y{\isasymin}M{\isachardot}{\kern0pt}\ pair{\isacharparenleft}{\kern0pt}{\isacharhash}{\kern0pt}{\isacharhash}{\kern0pt}M{\isacharcomma}{\kern0pt}x{\isacharcomma}{\kern0pt}y{\isacharcomma}{\kern0pt}p{\isacharparenright}{\kern0pt}\ {\isacharampersand}{\kern0pt}\ pair{\isacharparenleft}{\kern0pt}{\isacharhash}{\kern0pt}{\isacharhash}{\kern0pt}M{\isacharcomma}{\kern0pt}y{\isacharcomma}{\kern0pt}x{\isacharcomma}{\kern0pt}nth{\isacharparenleft}{\kern0pt}{\isadigit{0}}{\isacharcomma}{\kern0pt}env{\isacharparenright}{\kern0pt}{\isacharparenright}{\kern0pt}{\isacharparenright}{\kern0pt}{\isacharparenright}{\kern0pt}\isanewline
\ \ \ \ {\isasymlongleftrightarrow}\ sats{\isacharparenleft}{\kern0pt}M{\isacharcomma}{\kern0pt}cfm{\isacharparenleft}{\kern0pt}{\isadigit{0}}{\isacharcomma}{\kern0pt}{\isadigit{1}}{\isacharparenright}{\kern0pt}{\isacharcomma}{\kern0pt}env{\isacharparenright}{\kern0pt}{\isachardoublequoteclose}\isanewline
\ \ \ \ \isakeyword{and}\isanewline
\ \ \ \ {\isachardoublequoteopen}cfm{\isacharparenleft}{\kern0pt}{\isadigit{0}}{\isacharcomma}{\kern0pt}{\isadigit{1}}{\isacharparenright}{\kern0pt}\ {\isasymin}\ formula{\isachardoublequoteclose}\isanewline
\ \ \ \ \isakeyword{and}\isanewline
\ \ \ \ {\isachardoublequoteopen}arity{\isacharparenleft}{\kern0pt}cfm{\isacharparenleft}{\kern0pt}{\isadigit{0}}{\isacharcomma}{\kern0pt}{\isadigit{1}}{\isacharparenright}{\kern0pt}{\isacharparenright}{\kern0pt}\ {\isacharequal}{\kern0pt}\ {\isadigit{2}}{\isachardoublequoteclose}\isanewline
\ \ \ \ \isacommand{using}\isamarkupfalse%
\ con{\isacharunderscore}{\kern0pt}fm{\isacharunderscore}{\kern0pt}auto\ \isacommand{by}\isamarkupfalse%
\ {\isacharparenleft}{\kern0pt}simp\ del{\isacharcolon}{\kern0pt}FOL{\isacharunderscore}{\kern0pt}sats{\isacharunderscore}{\kern0pt}iff\ pair{\isacharunderscore}{\kern0pt}abs\ add{\isacharcolon}{\kern0pt}\ fm{\isacharunderscore}{\kern0pt}defs\ nat{\isacharunderscore}{\kern0pt}simp{\isacharunderscore}{\kern0pt}union{\isacharparenright}{\kern0pt}\isanewline
\ \ \isacommand{then}\isamarkupfalse%
\isanewline
\ \ \isacommand{have}\isamarkupfalse%
\ {\isachardoublequoteopen}{\isasymforall}r{\isasymin}M{\isachardot}{\kern0pt}\ separation{\isacharparenleft}{\kern0pt}{\isacharhash}{\kern0pt}{\isacharhash}{\kern0pt}M{\isacharcomma}{\kern0pt}\ {\isasymlambda}z{\isachardot}{\kern0pt}\ sats{\isacharparenleft}{\kern0pt}M{\isacharcomma}{\kern0pt}cfm{\isacharparenleft}{\kern0pt}{\isadigit{0}}{\isacharcomma}{\kern0pt}{\isadigit{1}}{\isacharparenright}{\kern0pt}\ {\isacharcomma}{\kern0pt}\ {\isacharbrackleft}{\kern0pt}z{\isacharcomma}{\kern0pt}r{\isacharbrackright}{\kern0pt}{\isacharparenright}{\kern0pt}{\isacharparenright}{\kern0pt}{\isachardoublequoteclose}\isanewline
\ \ \ \ \isacommand{using}\isamarkupfalse%
\ separation{\isacharunderscore}{\kern0pt}ax\ \isacommand{by}\isamarkupfalse%
\ simp\isanewline
\ \ \isacommand{moreover}\isamarkupfalse%
\isanewline
\ \ \isacommand{have}\isamarkupfalse%
\ {\isachardoublequoteopen}{\isacharparenleft}{\kern0pt}{\isasymexists}p{\isasymin}M{\isachardot}{\kern0pt}\ p{\isasymin}r\ {\isacharampersand}{\kern0pt}\ {\isacharparenleft}{\kern0pt}{\isasymexists}x{\isasymin}M{\isachardot}{\kern0pt}{\isasymexists}y{\isasymin}M{\isachardot}{\kern0pt}\ pair{\isacharparenleft}{\kern0pt}{\isacharhash}{\kern0pt}{\isacharhash}{\kern0pt}M{\isacharcomma}{\kern0pt}x{\isacharcomma}{\kern0pt}y{\isacharcomma}{\kern0pt}p{\isacharparenright}{\kern0pt}\ {\isacharampersand}{\kern0pt}\ pair{\isacharparenleft}{\kern0pt}{\isacharhash}{\kern0pt}{\isacharhash}{\kern0pt}M{\isacharcomma}{\kern0pt}y{\isacharcomma}{\kern0pt}x{\isacharcomma}{\kern0pt}z{\isacharparenright}{\kern0pt}{\isacharparenright}{\kern0pt}{\isacharparenright}{\kern0pt}\ {\isasymlongleftrightarrow}\isanewline
\ \ \ \ \ \ \ \ \ \ sats{\isacharparenleft}{\kern0pt}M{\isacharcomma}{\kern0pt}cfm{\isacharparenleft}{\kern0pt}{\isadigit{0}}{\isacharcomma}{\kern0pt}{\isadigit{1}}{\isacharparenright}{\kern0pt}{\isacharcomma}{\kern0pt}{\isacharbrackleft}{\kern0pt}z{\isacharcomma}{\kern0pt}r{\isacharbrackright}{\kern0pt}{\isacharparenright}{\kern0pt}{\isachardoublequoteclose}\isanewline
\ \ \ \ \isakeyword{if}\ {\isachardoublequoteopen}z{\isasymin}M{\isachardoublequoteclose}\ {\isachardoublequoteopen}r{\isasymin}M{\isachardoublequoteclose}\ \isakeyword{for}\ z\ r\isanewline
\ \ \ \ \isacommand{using}\isamarkupfalse%
\ that\ fmsats{\isacharbrackleft}{\kern0pt}of\ {\isachardoublequoteopen}{\isacharbrackleft}{\kern0pt}z{\isacharcomma}{\kern0pt}r{\isacharbrackright}{\kern0pt}{\isachardoublequoteclose}{\isacharbrackright}{\kern0pt}\ \isacommand{by}\isamarkupfalse%
\ simp\isanewline
\ \ \isacommand{ultimately}\isamarkupfalse%
\isanewline
\ \ \isacommand{have}\isamarkupfalse%
\ {\isachardoublequoteopen}{\isasymforall}r{\isasymin}M{\isachardot}{\kern0pt}\ separation{\isacharparenleft}{\kern0pt}{\isacharhash}{\kern0pt}{\isacharhash}{\kern0pt}M{\isacharcomma}{\kern0pt}\ {\isasymlambda}z\ {\isachardot}{\kern0pt}\ {\isasymexists}p{\isasymin}M{\isachardot}{\kern0pt}\ p{\isasymin}r\ {\isacharampersand}{\kern0pt}\ {\isacharparenleft}{\kern0pt}{\isasymexists}x{\isasymin}M{\isachardot}{\kern0pt}{\isasymexists}y{\isasymin}M{\isachardot}{\kern0pt}\ pair{\isacharparenleft}{\kern0pt}{\isacharhash}{\kern0pt}{\isacharhash}{\kern0pt}M{\isacharcomma}{\kern0pt}x{\isacharcomma}{\kern0pt}y{\isacharcomma}{\kern0pt}p{\isacharparenright}{\kern0pt}\ {\isacharampersand}{\kern0pt}\ pair{\isacharparenleft}{\kern0pt}{\isacharhash}{\kern0pt}{\isacharhash}{\kern0pt}M{\isacharcomma}{\kern0pt}y{\isacharcomma}{\kern0pt}x{\isacharcomma}{\kern0pt}z{\isacharparenright}{\kern0pt}{\isacharparenright}{\kern0pt}{\isacharparenright}{\kern0pt}{\isachardoublequoteclose}\isanewline
\ \ \ \ \isacommand{unfolding}\isamarkupfalse%
\ separation{\isacharunderscore}{\kern0pt}def\ \isacommand{by}\isamarkupfalse%
\ simp\isanewline
\ \ \isacommand{with}\isamarkupfalse%
\ {\isacartoucheopen}R{\isasymin}M{\isacartoucheclose}\ \isacommand{show}\isamarkupfalse%
\ {\isacharquery}{\kern0pt}thesis\ \isacommand{by}\isamarkupfalse%
\ simp\isanewline
\isacommand{qed}\isamarkupfalse%
%
\endisatagproof
{\isafoldproof}%
%
\isadelimproof
\isanewline
%
\endisadelimproof
\isanewline
\isanewline
\isacommand{schematic{\isacharunderscore}{\kern0pt}goal}\isamarkupfalse%
\ rest{\isacharunderscore}{\kern0pt}fm{\isacharunderscore}{\kern0pt}auto{\isacharcolon}{\kern0pt}\isanewline
\ \ \isakeyword{assumes}\isanewline
\ \ \ \ {\isachardoublequoteopen}nth{\isacharparenleft}{\kern0pt}i{\isacharcomma}{\kern0pt}env{\isacharparenright}{\kern0pt}\ {\isacharequal}{\kern0pt}\ z{\isachardoublequoteclose}\ {\isachardoublequoteopen}nth{\isacharparenleft}{\kern0pt}j{\isacharcomma}{\kern0pt}env{\isacharparenright}{\kern0pt}\ {\isacharequal}{\kern0pt}\ C{\isachardoublequoteclose}\isanewline
\ \ \ \ {\isachardoublequoteopen}i\ {\isasymin}\ nat{\isachardoublequoteclose}\ {\isachardoublequoteopen}j\ {\isasymin}\ nat{\isachardoublequoteclose}\ {\isachardoublequoteopen}env\ {\isasymin}\ list{\isacharparenleft}{\kern0pt}A{\isacharparenright}{\kern0pt}{\isachardoublequoteclose}\isanewline
\ \ \isakeyword{shows}\isanewline
\ \ \ \ {\isachardoublequoteopen}{\isacharparenleft}{\kern0pt}{\isasymexists}x{\isasymin}A{\isachardot}{\kern0pt}\ x{\isasymin}C\ {\isacharampersand}{\kern0pt}\ {\isacharparenleft}{\kern0pt}{\isasymexists}y{\isasymin}A{\isachardot}{\kern0pt}\ pair{\isacharparenleft}{\kern0pt}{\isacharhash}{\kern0pt}{\isacharhash}{\kern0pt}A{\isacharcomma}{\kern0pt}x{\isacharcomma}{\kern0pt}y{\isacharcomma}{\kern0pt}z{\isacharparenright}{\kern0pt}{\isacharparenright}{\kern0pt}{\isacharparenright}{\kern0pt}\isanewline
\ \ {\isasymlongleftrightarrow}\ sats{\isacharparenleft}{\kern0pt}A{\isacharcomma}{\kern0pt}{\isacharquery}{\kern0pt}rfm{\isacharparenleft}{\kern0pt}i{\isacharcomma}{\kern0pt}j{\isacharparenright}{\kern0pt}{\isacharcomma}{\kern0pt}env{\isacharparenright}{\kern0pt}{\isachardoublequoteclose}\isanewline
%
\isadelimproof
\ \ %
\endisadelimproof
%
\isatagproof
\isacommand{by}\isamarkupfalse%
\ {\isacharparenleft}{\kern0pt}insert\ assms\ {\isacharsemicolon}{\kern0pt}\ {\isacharparenleft}{\kern0pt}rule\ sep{\isacharunderscore}{\kern0pt}rules\ {\isacharbar}{\kern0pt}\ simp{\isacharparenright}{\kern0pt}{\isacharplus}{\kern0pt}{\isacharparenright}{\kern0pt}%
\endisatagproof
{\isafoldproof}%
%
\isadelimproof
\isanewline
%
\endisadelimproof
\isanewline
\isanewline
\isacommand{lemma}\isamarkupfalse%
\ restrict{\isacharunderscore}{\kern0pt}sep{\isacharunderscore}{\kern0pt}intf\ {\isacharcolon}{\kern0pt}\isanewline
\ \ \isakeyword{assumes}\isanewline
\ \ \ \ {\isachardoublequoteopen}A{\isasymin}M{\isachardoublequoteclose}\isanewline
\ \ \isakeyword{shows}\isanewline
\ \ \ \ {\isachardoublequoteopen}separation{\isacharparenleft}{\kern0pt}{\isacharhash}{\kern0pt}{\isacharhash}{\kern0pt}M{\isacharcomma}{\kern0pt}{\isasymlambda}z{\isachardot}{\kern0pt}\ {\isasymexists}x{\isasymin}M{\isachardot}{\kern0pt}\ x{\isasymin}A\ {\isacharampersand}{\kern0pt}\ {\isacharparenleft}{\kern0pt}{\isasymexists}y{\isasymin}M{\isachardot}{\kern0pt}\ pair{\isacharparenleft}{\kern0pt}{\isacharhash}{\kern0pt}{\isacharhash}{\kern0pt}M{\isacharcomma}{\kern0pt}x{\isacharcomma}{\kern0pt}y{\isacharcomma}{\kern0pt}z{\isacharparenright}{\kern0pt}{\isacharparenright}{\kern0pt}{\isacharparenright}{\kern0pt}{\isachardoublequoteclose}\isanewline
%
\isadelimproof
%
\endisadelimproof
%
\isatagproof
\isacommand{proof}\isamarkupfalse%
\ {\isacharminus}{\kern0pt}\isanewline
\ \ \isacommand{obtain}\isamarkupfalse%
\ rfm\ \isakeyword{where}\isanewline
\ \ \ \ fmsats{\isacharcolon}{\kern0pt}{\isachardoublequoteopen}{\isasymAnd}env{\isachardot}{\kern0pt}\ env{\isasymin}list{\isacharparenleft}{\kern0pt}M{\isacharparenright}{\kern0pt}\ {\isasymLongrightarrow}\isanewline
\ \ \ \ {\isacharparenleft}{\kern0pt}{\isasymexists}x{\isasymin}M{\isachardot}{\kern0pt}\ x{\isasymin}nth{\isacharparenleft}{\kern0pt}{\isadigit{1}}{\isacharcomma}{\kern0pt}env{\isacharparenright}{\kern0pt}\ {\isacharampersand}{\kern0pt}\ {\isacharparenleft}{\kern0pt}{\isasymexists}y{\isasymin}M{\isachardot}{\kern0pt}\ pair{\isacharparenleft}{\kern0pt}{\isacharhash}{\kern0pt}{\isacharhash}{\kern0pt}M{\isacharcomma}{\kern0pt}x{\isacharcomma}{\kern0pt}y{\isacharcomma}{\kern0pt}nth{\isacharparenleft}{\kern0pt}{\isadigit{0}}{\isacharcomma}{\kern0pt}env{\isacharparenright}{\kern0pt}{\isacharparenright}{\kern0pt}{\isacharparenright}{\kern0pt}{\isacharparenright}{\kern0pt}\isanewline
\ \ \ \ {\isasymlongleftrightarrow}\ sats{\isacharparenleft}{\kern0pt}M{\isacharcomma}{\kern0pt}rfm{\isacharparenleft}{\kern0pt}{\isadigit{0}}{\isacharcomma}{\kern0pt}{\isadigit{1}}{\isacharparenright}{\kern0pt}{\isacharcomma}{\kern0pt}env{\isacharparenright}{\kern0pt}{\isachardoublequoteclose}\isanewline
\ \ \ \ \isakeyword{and}\isanewline
\ \ \ \ {\isachardoublequoteopen}rfm{\isacharparenleft}{\kern0pt}{\isadigit{0}}{\isacharcomma}{\kern0pt}{\isadigit{1}}{\isacharparenright}{\kern0pt}\ {\isasymin}\ formula{\isachardoublequoteclose}\isanewline
\ \ \ \ \isakeyword{and}\isanewline
\ \ \ \ {\isachardoublequoteopen}arity{\isacharparenleft}{\kern0pt}rfm{\isacharparenleft}{\kern0pt}{\isadigit{0}}{\isacharcomma}{\kern0pt}{\isadigit{1}}{\isacharparenright}{\kern0pt}{\isacharparenright}{\kern0pt}\ {\isacharequal}{\kern0pt}\ {\isadigit{2}}{\isachardoublequoteclose}\isanewline
\ \ \ \ \isacommand{using}\isamarkupfalse%
\ rest{\isacharunderscore}{\kern0pt}fm{\isacharunderscore}{\kern0pt}auto\ \isacommand{by}\isamarkupfalse%
\ {\isacharparenleft}{\kern0pt}simp\ del{\isacharcolon}{\kern0pt}FOL{\isacharunderscore}{\kern0pt}sats{\isacharunderscore}{\kern0pt}iff\ pair{\isacharunderscore}{\kern0pt}abs\ add{\isacharcolon}{\kern0pt}\ fm{\isacharunderscore}{\kern0pt}defs\ nat{\isacharunderscore}{\kern0pt}simp{\isacharunderscore}{\kern0pt}union{\isacharparenright}{\kern0pt}\isanewline
\ \ \isacommand{then}\isamarkupfalse%
\isanewline
\ \ \isacommand{have}\isamarkupfalse%
\ {\isachardoublequoteopen}{\isasymforall}a{\isasymin}M{\isachardot}{\kern0pt}\ separation{\isacharparenleft}{\kern0pt}{\isacharhash}{\kern0pt}{\isacharhash}{\kern0pt}M{\isacharcomma}{\kern0pt}\ {\isasymlambda}z{\isachardot}{\kern0pt}\ sats{\isacharparenleft}{\kern0pt}M{\isacharcomma}{\kern0pt}rfm{\isacharparenleft}{\kern0pt}{\isadigit{0}}{\isacharcomma}{\kern0pt}{\isadigit{1}}{\isacharparenright}{\kern0pt}\ {\isacharcomma}{\kern0pt}\ {\isacharbrackleft}{\kern0pt}z{\isacharcomma}{\kern0pt}a{\isacharbrackright}{\kern0pt}{\isacharparenright}{\kern0pt}{\isacharparenright}{\kern0pt}{\isachardoublequoteclose}\isanewline
\ \ \ \ \isacommand{using}\isamarkupfalse%
\ separation{\isacharunderscore}{\kern0pt}ax\ \isacommand{by}\isamarkupfalse%
\ simp\isanewline
\ \ \isacommand{moreover}\isamarkupfalse%
\isanewline
\ \ \isacommand{have}\isamarkupfalse%
\ {\isachardoublequoteopen}{\isacharparenleft}{\kern0pt}{\isasymexists}x{\isasymin}M{\isachardot}{\kern0pt}\ x{\isasymin}a\ {\isacharampersand}{\kern0pt}\ {\isacharparenleft}{\kern0pt}{\isasymexists}y{\isasymin}M{\isachardot}{\kern0pt}\ pair{\isacharparenleft}{\kern0pt}{\isacharhash}{\kern0pt}{\isacharhash}{\kern0pt}M{\isacharcomma}{\kern0pt}x{\isacharcomma}{\kern0pt}y{\isacharcomma}{\kern0pt}z{\isacharparenright}{\kern0pt}{\isacharparenright}{\kern0pt}{\isacharparenright}{\kern0pt}\ {\isasymlongleftrightarrow}\isanewline
\ \ \ \ \ \ \ \ \ \ sats{\isacharparenleft}{\kern0pt}M{\isacharcomma}{\kern0pt}rfm{\isacharparenleft}{\kern0pt}{\isadigit{0}}{\isacharcomma}{\kern0pt}{\isadigit{1}}{\isacharparenright}{\kern0pt}{\isacharcomma}{\kern0pt}{\isacharbrackleft}{\kern0pt}z{\isacharcomma}{\kern0pt}a{\isacharbrackright}{\kern0pt}{\isacharparenright}{\kern0pt}{\isachardoublequoteclose}\isanewline
\ \ \ \ \isakeyword{if}\ {\isachardoublequoteopen}z{\isasymin}M{\isachardoublequoteclose}\ {\isachardoublequoteopen}a{\isasymin}M{\isachardoublequoteclose}\ \isakeyword{for}\ z\ a\isanewline
\ \ \ \ \isacommand{using}\isamarkupfalse%
\ that\ fmsats{\isacharbrackleft}{\kern0pt}of\ {\isachardoublequoteopen}{\isacharbrackleft}{\kern0pt}z{\isacharcomma}{\kern0pt}a{\isacharbrackright}{\kern0pt}{\isachardoublequoteclose}{\isacharbrackright}{\kern0pt}\ \isacommand{by}\isamarkupfalse%
\ simp\isanewline
\ \ \isacommand{ultimately}\isamarkupfalse%
\isanewline
\ \ \isacommand{have}\isamarkupfalse%
\ {\isachardoublequoteopen}{\isasymforall}a{\isasymin}M{\isachardot}{\kern0pt}\ separation{\isacharparenleft}{\kern0pt}{\isacharhash}{\kern0pt}{\isacharhash}{\kern0pt}M{\isacharcomma}{\kern0pt}\ {\isasymlambda}z\ {\isachardot}{\kern0pt}\ {\isasymexists}x{\isasymin}M{\isachardot}{\kern0pt}\ x{\isasymin}a\ {\isacharampersand}{\kern0pt}\ {\isacharparenleft}{\kern0pt}{\isasymexists}y{\isasymin}M{\isachardot}{\kern0pt}\ pair{\isacharparenleft}{\kern0pt}{\isacharhash}{\kern0pt}{\isacharhash}{\kern0pt}M{\isacharcomma}{\kern0pt}x{\isacharcomma}{\kern0pt}y{\isacharcomma}{\kern0pt}z{\isacharparenright}{\kern0pt}{\isacharparenright}{\kern0pt}{\isacharparenright}{\kern0pt}{\isachardoublequoteclose}\isanewline
\ \ \ \ \isacommand{unfolding}\isamarkupfalse%
\ separation{\isacharunderscore}{\kern0pt}def\ \isacommand{by}\isamarkupfalse%
\ simp\isanewline
\ \ \isacommand{with}\isamarkupfalse%
\ {\isacartoucheopen}A{\isasymin}M{\isacartoucheclose}\ \isacommand{show}\isamarkupfalse%
\ {\isacharquery}{\kern0pt}thesis\ \isacommand{by}\isamarkupfalse%
\ simp\isanewline
\isacommand{qed}\isamarkupfalse%
%
\endisatagproof
{\isafoldproof}%
%
\isadelimproof
\isanewline
%
\endisadelimproof
\isanewline
\isacommand{schematic{\isacharunderscore}{\kern0pt}goal}\isamarkupfalse%
\ comp{\isacharunderscore}{\kern0pt}fm{\isacharunderscore}{\kern0pt}auto{\isacharcolon}{\kern0pt}\isanewline
\ \ \isakeyword{assumes}\isanewline
\ \ \ \ {\isachardoublequoteopen}nth{\isacharparenleft}{\kern0pt}i{\isacharcomma}{\kern0pt}env{\isacharparenright}{\kern0pt}\ {\isacharequal}{\kern0pt}\ xz{\isachardoublequoteclose}\ {\isachardoublequoteopen}nth{\isacharparenleft}{\kern0pt}j{\isacharcomma}{\kern0pt}env{\isacharparenright}{\kern0pt}\ {\isacharequal}{\kern0pt}\ S{\isachardoublequoteclose}\ {\isachardoublequoteopen}nth{\isacharparenleft}{\kern0pt}h{\isacharcomma}{\kern0pt}env{\isacharparenright}{\kern0pt}\ {\isacharequal}{\kern0pt}\ R{\isachardoublequoteclose}\isanewline
\ \ \ \ {\isachardoublequoteopen}i\ {\isasymin}\ nat{\isachardoublequoteclose}\ {\isachardoublequoteopen}j\ {\isasymin}\ nat{\isachardoublequoteclose}\ {\isachardoublequoteopen}h\ {\isasymin}\ nat{\isachardoublequoteclose}\ {\isachardoublequoteopen}env\ {\isasymin}\ list{\isacharparenleft}{\kern0pt}A{\isacharparenright}{\kern0pt}{\isachardoublequoteclose}\isanewline
\ \ \isakeyword{shows}\isanewline
\ \ \ \ {\isachardoublequoteopen}{\isacharparenleft}{\kern0pt}{\isasymexists}x{\isasymin}A{\isachardot}{\kern0pt}\ {\isasymexists}y{\isasymin}A{\isachardot}{\kern0pt}\ {\isasymexists}z{\isasymin}A{\isachardot}{\kern0pt}\ {\isasymexists}xy{\isasymin}A{\isachardot}{\kern0pt}\ {\isasymexists}yz{\isasymin}A{\isachardot}{\kern0pt}\isanewline
\ \ \ \ \ \ \ \ \ \ \ \ \ \ pair{\isacharparenleft}{\kern0pt}{\isacharhash}{\kern0pt}{\isacharhash}{\kern0pt}A{\isacharcomma}{\kern0pt}x{\isacharcomma}{\kern0pt}z{\isacharcomma}{\kern0pt}xz{\isacharparenright}{\kern0pt}\ {\isacharampersand}{\kern0pt}\ pair{\isacharparenleft}{\kern0pt}{\isacharhash}{\kern0pt}{\isacharhash}{\kern0pt}A{\isacharcomma}{\kern0pt}x{\isacharcomma}{\kern0pt}y{\isacharcomma}{\kern0pt}xy{\isacharparenright}{\kern0pt}\ {\isacharampersand}{\kern0pt}\ pair{\isacharparenleft}{\kern0pt}{\isacharhash}{\kern0pt}{\isacharhash}{\kern0pt}A{\isacharcomma}{\kern0pt}y{\isacharcomma}{\kern0pt}z{\isacharcomma}{\kern0pt}yz{\isacharparenright}{\kern0pt}\ {\isacharampersand}{\kern0pt}\ xy{\isasymin}S\ {\isacharampersand}{\kern0pt}\ yz{\isasymin}R{\isacharparenright}{\kern0pt}\isanewline
\ \ {\isasymlongleftrightarrow}\ sats{\isacharparenleft}{\kern0pt}A{\isacharcomma}{\kern0pt}{\isacharquery}{\kern0pt}cfm{\isacharparenleft}{\kern0pt}i{\isacharcomma}{\kern0pt}j{\isacharcomma}{\kern0pt}h{\isacharparenright}{\kern0pt}{\isacharcomma}{\kern0pt}env{\isacharparenright}{\kern0pt}{\isachardoublequoteclose}\isanewline
%
\isadelimproof
\ \ %
\endisadelimproof
%
\isatagproof
\isacommand{by}\isamarkupfalse%
\ {\isacharparenleft}{\kern0pt}insert\ assms\ {\isacharsemicolon}{\kern0pt}\ {\isacharparenleft}{\kern0pt}rule\ sep{\isacharunderscore}{\kern0pt}rules\ {\isacharbar}{\kern0pt}\ simp{\isacharparenright}{\kern0pt}{\isacharplus}{\kern0pt}{\isacharparenright}{\kern0pt}%
\endisatagproof
{\isafoldproof}%
%
\isadelimproof
\isanewline
%
\endisadelimproof
\isanewline
\isanewline
\isacommand{lemma}\isamarkupfalse%
\ comp{\isacharunderscore}{\kern0pt}sep{\isacharunderscore}{\kern0pt}intf\ {\isacharcolon}{\kern0pt}\isanewline
\ \ \isakeyword{assumes}\isanewline
\ \ \ \ {\isachardoublequoteopen}R{\isasymin}M{\isachardoublequoteclose}\isanewline
\ \ \ \ \isakeyword{and}\isanewline
\ \ \ \ {\isachardoublequoteopen}S{\isasymin}M{\isachardoublequoteclose}\isanewline
\ \ \isakeyword{shows}\isanewline
\ \ \ \ {\isachardoublequoteopen}separation{\isacharparenleft}{\kern0pt}{\isacharhash}{\kern0pt}{\isacharhash}{\kern0pt}M{\isacharcomma}{\kern0pt}{\isasymlambda}xz{\isachardot}{\kern0pt}\ {\isasymexists}x{\isasymin}M{\isachardot}{\kern0pt}\ {\isasymexists}y{\isasymin}M{\isachardot}{\kern0pt}\ {\isasymexists}z{\isasymin}M{\isachardot}{\kern0pt}\ {\isasymexists}xy{\isasymin}M{\isachardot}{\kern0pt}\ {\isasymexists}yz{\isasymin}M{\isachardot}{\kern0pt}\isanewline
\ \ \ \ \ \ \ \ \ \ \ \ \ \ pair{\isacharparenleft}{\kern0pt}{\isacharhash}{\kern0pt}{\isacharhash}{\kern0pt}M{\isacharcomma}{\kern0pt}x{\isacharcomma}{\kern0pt}z{\isacharcomma}{\kern0pt}xz{\isacharparenright}{\kern0pt}\ {\isacharampersand}{\kern0pt}\ pair{\isacharparenleft}{\kern0pt}{\isacharhash}{\kern0pt}{\isacharhash}{\kern0pt}M{\isacharcomma}{\kern0pt}x{\isacharcomma}{\kern0pt}y{\isacharcomma}{\kern0pt}xy{\isacharparenright}{\kern0pt}\ {\isacharampersand}{\kern0pt}\ pair{\isacharparenleft}{\kern0pt}{\isacharhash}{\kern0pt}{\isacharhash}{\kern0pt}M{\isacharcomma}{\kern0pt}y{\isacharcomma}{\kern0pt}z{\isacharcomma}{\kern0pt}yz{\isacharparenright}{\kern0pt}\ {\isacharampersand}{\kern0pt}\ xy{\isasymin}S\ {\isacharampersand}{\kern0pt}\ yz{\isasymin}R{\isacharparenright}{\kern0pt}{\isachardoublequoteclose}\isanewline
%
\isadelimproof
%
\endisadelimproof
%
\isatagproof
\isacommand{proof}\isamarkupfalse%
\ {\isacharminus}{\kern0pt}\isanewline
\ \ \isacommand{obtain}\isamarkupfalse%
\ cfm\ \isakeyword{where}\isanewline
\ \ \ \ fmsats{\isacharcolon}{\kern0pt}{\isachardoublequoteopen}{\isasymAnd}env{\isachardot}{\kern0pt}\ env{\isasymin}list{\isacharparenleft}{\kern0pt}M{\isacharparenright}{\kern0pt}\ {\isasymLongrightarrow}\isanewline
\ \ \ \ {\isacharparenleft}{\kern0pt}{\isasymexists}x{\isasymin}M{\isachardot}{\kern0pt}\ {\isasymexists}y{\isasymin}M{\isachardot}{\kern0pt}\ {\isasymexists}z{\isasymin}M{\isachardot}{\kern0pt}\ {\isasymexists}xy{\isasymin}M{\isachardot}{\kern0pt}\ {\isasymexists}yz{\isasymin}M{\isachardot}{\kern0pt}\ pair{\isacharparenleft}{\kern0pt}{\isacharhash}{\kern0pt}{\isacharhash}{\kern0pt}M{\isacharcomma}{\kern0pt}x{\isacharcomma}{\kern0pt}z{\isacharcomma}{\kern0pt}nth{\isacharparenleft}{\kern0pt}{\isadigit{0}}{\isacharcomma}{\kern0pt}env{\isacharparenright}{\kern0pt}{\isacharparenright}{\kern0pt}\ {\isacharampersand}{\kern0pt}\isanewline
\ \ \ \ \ \ \ \ \ \ \ \ pair{\isacharparenleft}{\kern0pt}{\isacharhash}{\kern0pt}{\isacharhash}{\kern0pt}M{\isacharcomma}{\kern0pt}x{\isacharcomma}{\kern0pt}y{\isacharcomma}{\kern0pt}xy{\isacharparenright}{\kern0pt}\ {\isacharampersand}{\kern0pt}\ pair{\isacharparenleft}{\kern0pt}{\isacharhash}{\kern0pt}{\isacharhash}{\kern0pt}M{\isacharcomma}{\kern0pt}y{\isacharcomma}{\kern0pt}z{\isacharcomma}{\kern0pt}yz{\isacharparenright}{\kern0pt}\ {\isacharampersand}{\kern0pt}\ xy{\isasymin}nth{\isacharparenleft}{\kern0pt}{\isadigit{1}}{\isacharcomma}{\kern0pt}env{\isacharparenright}{\kern0pt}\ {\isacharampersand}{\kern0pt}\ yz{\isasymin}nth{\isacharparenleft}{\kern0pt}{\isadigit{2}}{\isacharcomma}{\kern0pt}env{\isacharparenright}{\kern0pt}{\isacharparenright}{\kern0pt}\isanewline
\ \ \ \ {\isasymlongleftrightarrow}\ sats{\isacharparenleft}{\kern0pt}M{\isacharcomma}{\kern0pt}cfm{\isacharparenleft}{\kern0pt}{\isadigit{0}}{\isacharcomma}{\kern0pt}{\isadigit{1}}{\isacharcomma}{\kern0pt}{\isadigit{2}}{\isacharparenright}{\kern0pt}{\isacharcomma}{\kern0pt}env{\isacharparenright}{\kern0pt}{\isachardoublequoteclose}\isanewline
\ \ \ \ \isakeyword{and}\isanewline
\ \ \ \ {\isachardoublequoteopen}cfm{\isacharparenleft}{\kern0pt}{\isadigit{0}}{\isacharcomma}{\kern0pt}{\isadigit{1}}{\isacharcomma}{\kern0pt}{\isadigit{2}}{\isacharparenright}{\kern0pt}\ {\isasymin}\ formula{\isachardoublequoteclose}\isanewline
\ \ \ \ \isakeyword{and}\isanewline
\ \ \ \ {\isachardoublequoteopen}arity{\isacharparenleft}{\kern0pt}cfm{\isacharparenleft}{\kern0pt}{\isadigit{0}}{\isacharcomma}{\kern0pt}{\isadigit{1}}{\isacharcomma}{\kern0pt}{\isadigit{2}}{\isacharparenright}{\kern0pt}{\isacharparenright}{\kern0pt}\ {\isacharequal}{\kern0pt}\ {\isadigit{3}}{\isachardoublequoteclose}\isanewline
\ \ \ \ \isacommand{using}\isamarkupfalse%
\ comp{\isacharunderscore}{\kern0pt}fm{\isacharunderscore}{\kern0pt}auto\ \isacommand{by}\isamarkupfalse%
\ {\isacharparenleft}{\kern0pt}simp\ del{\isacharcolon}{\kern0pt}FOL{\isacharunderscore}{\kern0pt}sats{\isacharunderscore}{\kern0pt}iff\ pair{\isacharunderscore}{\kern0pt}abs\ add{\isacharcolon}{\kern0pt}\ fm{\isacharunderscore}{\kern0pt}defs\ nat{\isacharunderscore}{\kern0pt}simp{\isacharunderscore}{\kern0pt}union{\isacharparenright}{\kern0pt}\isanewline
\ \ \isacommand{then}\isamarkupfalse%
\isanewline
\ \ \isacommand{have}\isamarkupfalse%
\ {\isachardoublequoteopen}{\isasymforall}r{\isasymin}M{\isachardot}{\kern0pt}\ {\isasymforall}s{\isasymin}M{\isachardot}{\kern0pt}\ separation{\isacharparenleft}{\kern0pt}{\isacharhash}{\kern0pt}{\isacharhash}{\kern0pt}M{\isacharcomma}{\kern0pt}\ {\isasymlambda}y{\isachardot}{\kern0pt}\ sats{\isacharparenleft}{\kern0pt}M{\isacharcomma}{\kern0pt}cfm{\isacharparenleft}{\kern0pt}{\isadigit{0}}{\isacharcomma}{\kern0pt}{\isadigit{1}}{\isacharcomma}{\kern0pt}{\isadigit{2}}{\isacharparenright}{\kern0pt}\ {\isacharcomma}{\kern0pt}\ {\isacharbrackleft}{\kern0pt}y{\isacharcomma}{\kern0pt}s{\isacharcomma}{\kern0pt}r{\isacharbrackright}{\kern0pt}{\isacharparenright}{\kern0pt}{\isacharparenright}{\kern0pt}{\isachardoublequoteclose}\isanewline
\ \ \ \ \isacommand{using}\isamarkupfalse%
\ separation{\isacharunderscore}{\kern0pt}ax\ \isacommand{by}\isamarkupfalse%
\ simp\isanewline
\ \ \isacommand{moreover}\isamarkupfalse%
\isanewline
\ \ \isacommand{have}\isamarkupfalse%
\ {\isachardoublequoteopen}{\isacharparenleft}{\kern0pt}{\isasymexists}x{\isasymin}M{\isachardot}{\kern0pt}\ {\isasymexists}y{\isasymin}M{\isachardot}{\kern0pt}\ {\isasymexists}z{\isasymin}M{\isachardot}{\kern0pt}\ {\isasymexists}xy{\isasymin}M{\isachardot}{\kern0pt}\ {\isasymexists}yz{\isasymin}M{\isachardot}{\kern0pt}\isanewline
\ \ \ \ \ \ \ \ \ \ \ \ \ \ pair{\isacharparenleft}{\kern0pt}{\isacharhash}{\kern0pt}{\isacharhash}{\kern0pt}M{\isacharcomma}{\kern0pt}x{\isacharcomma}{\kern0pt}z{\isacharcomma}{\kern0pt}xz{\isacharparenright}{\kern0pt}\ {\isacharampersand}{\kern0pt}\ pair{\isacharparenleft}{\kern0pt}{\isacharhash}{\kern0pt}{\isacharhash}{\kern0pt}M{\isacharcomma}{\kern0pt}x{\isacharcomma}{\kern0pt}y{\isacharcomma}{\kern0pt}xy{\isacharparenright}{\kern0pt}\ {\isacharampersand}{\kern0pt}\ pair{\isacharparenleft}{\kern0pt}{\isacharhash}{\kern0pt}{\isacharhash}{\kern0pt}M{\isacharcomma}{\kern0pt}y{\isacharcomma}{\kern0pt}z{\isacharcomma}{\kern0pt}yz{\isacharparenright}{\kern0pt}\ {\isacharampersand}{\kern0pt}\ xy{\isasymin}s\ {\isacharampersand}{\kern0pt}\ yz{\isasymin}r{\isacharparenright}{\kern0pt}\isanewline
\ \ \ \ \ \ \ \ \ \ {\isasymlongleftrightarrow}\ sats{\isacharparenleft}{\kern0pt}M{\isacharcomma}{\kern0pt}cfm{\isacharparenleft}{\kern0pt}{\isadigit{0}}{\isacharcomma}{\kern0pt}{\isadigit{1}}{\isacharcomma}{\kern0pt}{\isadigit{2}}{\isacharparenright}{\kern0pt}\ {\isacharcomma}{\kern0pt}\ {\isacharbrackleft}{\kern0pt}xz{\isacharcomma}{\kern0pt}s{\isacharcomma}{\kern0pt}r{\isacharbrackright}{\kern0pt}{\isacharparenright}{\kern0pt}{\isachardoublequoteclose}\isanewline
\ \ \ \ \isakeyword{if}\ {\isachardoublequoteopen}xz{\isasymin}M{\isachardoublequoteclose}\ {\isachardoublequoteopen}s{\isasymin}M{\isachardoublequoteclose}\ {\isachardoublequoteopen}r{\isasymin}M{\isachardoublequoteclose}\ \isakeyword{for}\ xz\ s\ r\isanewline
\ \ \ \ \isacommand{using}\isamarkupfalse%
\ that\ fmsats{\isacharbrackleft}{\kern0pt}of\ {\isachardoublequoteopen}{\isacharbrackleft}{\kern0pt}xz{\isacharcomma}{\kern0pt}s{\isacharcomma}{\kern0pt}r{\isacharbrackright}{\kern0pt}{\isachardoublequoteclose}{\isacharbrackright}{\kern0pt}\ \isacommand{by}\isamarkupfalse%
\ simp\isanewline
\ \ \isacommand{ultimately}\isamarkupfalse%
\isanewline
\ \ \isacommand{have}\isamarkupfalse%
\ {\isachardoublequoteopen}{\isasymforall}s{\isasymin}M{\isachardot}{\kern0pt}\ {\isasymforall}r{\isasymin}M{\isachardot}{\kern0pt}\ separation{\isacharparenleft}{\kern0pt}{\isacharhash}{\kern0pt}{\isacharhash}{\kern0pt}M{\isacharcomma}{\kern0pt}\ {\isasymlambda}xz\ {\isachardot}{\kern0pt}\ {\isasymexists}x{\isasymin}M{\isachardot}{\kern0pt}\ {\isasymexists}y{\isasymin}M{\isachardot}{\kern0pt}\ {\isasymexists}z{\isasymin}M{\isachardot}{\kern0pt}\ {\isasymexists}xy{\isasymin}M{\isachardot}{\kern0pt}\ {\isasymexists}yz{\isasymin}M{\isachardot}{\kern0pt}\isanewline
\ \ \ \ \ \ \ \ \ \ \ \ \ \ pair{\isacharparenleft}{\kern0pt}{\isacharhash}{\kern0pt}{\isacharhash}{\kern0pt}M{\isacharcomma}{\kern0pt}x{\isacharcomma}{\kern0pt}z{\isacharcomma}{\kern0pt}xz{\isacharparenright}{\kern0pt}\ {\isacharampersand}{\kern0pt}\ pair{\isacharparenleft}{\kern0pt}{\isacharhash}{\kern0pt}{\isacharhash}{\kern0pt}M{\isacharcomma}{\kern0pt}x{\isacharcomma}{\kern0pt}y{\isacharcomma}{\kern0pt}xy{\isacharparenright}{\kern0pt}\ {\isacharampersand}{\kern0pt}\ pair{\isacharparenleft}{\kern0pt}{\isacharhash}{\kern0pt}{\isacharhash}{\kern0pt}M{\isacharcomma}{\kern0pt}y{\isacharcomma}{\kern0pt}z{\isacharcomma}{\kern0pt}yz{\isacharparenright}{\kern0pt}\ {\isacharampersand}{\kern0pt}\ xy{\isasymin}s\ {\isacharampersand}{\kern0pt}\ yz{\isasymin}r{\isacharparenright}{\kern0pt}{\isachardoublequoteclose}\isanewline
\ \ \ \ \isacommand{unfolding}\isamarkupfalse%
\ separation{\isacharunderscore}{\kern0pt}def\ \isacommand{by}\isamarkupfalse%
\ simp\isanewline
\ \ \isacommand{with}\isamarkupfalse%
\ {\isacartoucheopen}S{\isasymin}M{\isacartoucheclose}\ {\isacartoucheopen}R{\isasymin}M{\isacartoucheclose}\ \isacommand{show}\isamarkupfalse%
\ {\isacharquery}{\kern0pt}thesis\ \isacommand{by}\isamarkupfalse%
\ simp\isanewline
\isacommand{qed}\isamarkupfalse%
%
\endisatagproof
{\isafoldproof}%
%
\isadelimproof
\isanewline
%
\endisadelimproof
\isanewline
\isanewline
\isacommand{schematic{\isacharunderscore}{\kern0pt}goal}\isamarkupfalse%
\ pred{\isacharunderscore}{\kern0pt}fm{\isacharunderscore}{\kern0pt}auto{\isacharcolon}{\kern0pt}\isanewline
\ \ \isakeyword{assumes}\isanewline
\ \ \ \ {\isachardoublequoteopen}nth{\isacharparenleft}{\kern0pt}i{\isacharcomma}{\kern0pt}env{\isacharparenright}{\kern0pt}\ {\isacharequal}{\kern0pt}\ y{\isachardoublequoteclose}\ {\isachardoublequoteopen}nth{\isacharparenleft}{\kern0pt}j{\isacharcomma}{\kern0pt}env{\isacharparenright}{\kern0pt}\ {\isacharequal}{\kern0pt}\ R{\isachardoublequoteclose}\ {\isachardoublequoteopen}nth{\isacharparenleft}{\kern0pt}h{\isacharcomma}{\kern0pt}env{\isacharparenright}{\kern0pt}\ {\isacharequal}{\kern0pt}\ X{\isachardoublequoteclose}\isanewline
\ \ \ \ {\isachardoublequoteopen}i\ {\isasymin}\ nat{\isachardoublequoteclose}\ {\isachardoublequoteopen}j\ {\isasymin}\ nat{\isachardoublequoteclose}\ {\isachardoublequoteopen}h\ {\isasymin}\ nat{\isachardoublequoteclose}\ {\isachardoublequoteopen}env\ {\isasymin}\ list{\isacharparenleft}{\kern0pt}A{\isacharparenright}{\kern0pt}{\isachardoublequoteclose}\isanewline
\ \ \isakeyword{shows}\isanewline
\ \ \ \ {\isachardoublequoteopen}{\isacharparenleft}{\kern0pt}{\isasymexists}p{\isasymin}A{\isachardot}{\kern0pt}\ p{\isasymin}R\ {\isacharampersand}{\kern0pt}\ pair{\isacharparenleft}{\kern0pt}{\isacharhash}{\kern0pt}{\isacharhash}{\kern0pt}A{\isacharcomma}{\kern0pt}y{\isacharcomma}{\kern0pt}X{\isacharcomma}{\kern0pt}p{\isacharparenright}{\kern0pt}{\isacharparenright}{\kern0pt}\ {\isasymlongleftrightarrow}\ sats{\isacharparenleft}{\kern0pt}A{\isacharcomma}{\kern0pt}{\isacharquery}{\kern0pt}pfm{\isacharparenleft}{\kern0pt}i{\isacharcomma}{\kern0pt}j{\isacharcomma}{\kern0pt}h{\isacharparenright}{\kern0pt}{\isacharcomma}{\kern0pt}env{\isacharparenright}{\kern0pt}{\isachardoublequoteclose}\isanewline
%
\isadelimproof
\ \ %
\endisadelimproof
%
\isatagproof
\isacommand{by}\isamarkupfalse%
\ {\isacharparenleft}{\kern0pt}insert\ assms\ {\isacharsemicolon}{\kern0pt}\ {\isacharparenleft}{\kern0pt}rule\ sep{\isacharunderscore}{\kern0pt}rules\ {\isacharbar}{\kern0pt}\ simp{\isacharparenright}{\kern0pt}{\isacharplus}{\kern0pt}{\isacharparenright}{\kern0pt}%
\endisatagproof
{\isafoldproof}%
%
\isadelimproof
\isanewline
%
\endisadelimproof
\isanewline
\isanewline
\isacommand{lemma}\isamarkupfalse%
\ pred{\isacharunderscore}{\kern0pt}sep{\isacharunderscore}{\kern0pt}intf{\isacharcolon}{\kern0pt}\isanewline
\ \ \isakeyword{assumes}\isanewline
\ \ \ \ {\isachardoublequoteopen}R{\isasymin}M{\isachardoublequoteclose}\isanewline
\ \ \ \ \isakeyword{and}\isanewline
\ \ \ \ {\isachardoublequoteopen}X{\isasymin}M{\isachardoublequoteclose}\isanewline
\ \ \isakeyword{shows}\isanewline
\ \ \ \ {\isachardoublequoteopen}separation{\isacharparenleft}{\kern0pt}{\isacharhash}{\kern0pt}{\isacharhash}{\kern0pt}M{\isacharcomma}{\kern0pt}\ {\isasymlambda}y{\isachardot}{\kern0pt}\ {\isasymexists}p{\isasymin}M{\isachardot}{\kern0pt}\ p{\isasymin}R\ {\isacharampersand}{\kern0pt}\ pair{\isacharparenleft}{\kern0pt}{\isacharhash}{\kern0pt}{\isacharhash}{\kern0pt}M{\isacharcomma}{\kern0pt}y{\isacharcomma}{\kern0pt}X{\isacharcomma}{\kern0pt}p{\isacharparenright}{\kern0pt}{\isacharparenright}{\kern0pt}{\isachardoublequoteclose}\isanewline
%
\isadelimproof
%
\endisadelimproof
%
\isatagproof
\isacommand{proof}\isamarkupfalse%
\ {\isacharminus}{\kern0pt}\isanewline
\ \ \isacommand{obtain}\isamarkupfalse%
\ pfm\ \isakeyword{where}\isanewline
\ \ \ \ fmsats{\isacharcolon}{\kern0pt}{\isachardoublequoteopen}{\isasymAnd}env{\isachardot}{\kern0pt}\ env{\isasymin}list{\isacharparenleft}{\kern0pt}M{\isacharparenright}{\kern0pt}\ {\isasymLongrightarrow}\isanewline
\ \ \ \ {\isacharparenleft}{\kern0pt}{\isasymexists}p{\isasymin}M{\isachardot}{\kern0pt}\ p{\isasymin}nth{\isacharparenleft}{\kern0pt}{\isadigit{1}}{\isacharcomma}{\kern0pt}env{\isacharparenright}{\kern0pt}\ {\isacharampersand}{\kern0pt}\ pair{\isacharparenleft}{\kern0pt}{\isacharhash}{\kern0pt}{\isacharhash}{\kern0pt}M{\isacharcomma}{\kern0pt}nth{\isacharparenleft}{\kern0pt}{\isadigit{0}}{\isacharcomma}{\kern0pt}env{\isacharparenright}{\kern0pt}{\isacharcomma}{\kern0pt}nth{\isacharparenleft}{\kern0pt}{\isadigit{2}}{\isacharcomma}{\kern0pt}env{\isacharparenright}{\kern0pt}{\isacharcomma}{\kern0pt}p{\isacharparenright}{\kern0pt}{\isacharparenright}{\kern0pt}\ {\isasymlongleftrightarrow}\ sats{\isacharparenleft}{\kern0pt}M{\isacharcomma}{\kern0pt}pfm{\isacharparenleft}{\kern0pt}{\isadigit{0}}{\isacharcomma}{\kern0pt}{\isadigit{1}}{\isacharcomma}{\kern0pt}{\isadigit{2}}{\isacharparenright}{\kern0pt}{\isacharcomma}{\kern0pt}env{\isacharparenright}{\kern0pt}{\isachardoublequoteclose}\isanewline
\ \ \ \ \isakeyword{and}\isanewline
\ \ \ \ {\isachardoublequoteopen}pfm{\isacharparenleft}{\kern0pt}{\isadigit{0}}{\isacharcomma}{\kern0pt}{\isadigit{1}}{\isacharcomma}{\kern0pt}{\isadigit{2}}{\isacharparenright}{\kern0pt}\ {\isasymin}\ formula{\isachardoublequoteclose}\isanewline
\ \ \ \ \isakeyword{and}\isanewline
\ \ \ \ {\isachardoublequoteopen}arity{\isacharparenleft}{\kern0pt}pfm{\isacharparenleft}{\kern0pt}{\isadigit{0}}{\isacharcomma}{\kern0pt}{\isadigit{1}}{\isacharcomma}{\kern0pt}{\isadigit{2}}{\isacharparenright}{\kern0pt}{\isacharparenright}{\kern0pt}\ {\isacharequal}{\kern0pt}\ {\isadigit{3}}{\isachardoublequoteclose}\isanewline
\ \ \ \ \isacommand{using}\isamarkupfalse%
\ pred{\isacharunderscore}{\kern0pt}fm{\isacharunderscore}{\kern0pt}auto\ \isacommand{by}\isamarkupfalse%
\ {\isacharparenleft}{\kern0pt}simp\ del{\isacharcolon}{\kern0pt}FOL{\isacharunderscore}{\kern0pt}sats{\isacharunderscore}{\kern0pt}iff\ pair{\isacharunderscore}{\kern0pt}abs\ add{\isacharcolon}{\kern0pt}\ fm{\isacharunderscore}{\kern0pt}defs\ nat{\isacharunderscore}{\kern0pt}simp{\isacharunderscore}{\kern0pt}union{\isacharparenright}{\kern0pt}\isanewline
\ \ \isacommand{then}\isamarkupfalse%
\isanewline
\ \ \isacommand{have}\isamarkupfalse%
\ {\isachardoublequoteopen}{\isasymforall}x{\isasymin}M{\isachardot}{\kern0pt}\ {\isasymforall}r{\isasymin}M{\isachardot}{\kern0pt}\ separation{\isacharparenleft}{\kern0pt}{\isacharhash}{\kern0pt}{\isacharhash}{\kern0pt}M{\isacharcomma}{\kern0pt}\ {\isasymlambda}y{\isachardot}{\kern0pt}\ sats{\isacharparenleft}{\kern0pt}M{\isacharcomma}{\kern0pt}pfm{\isacharparenleft}{\kern0pt}{\isadigit{0}}{\isacharcomma}{\kern0pt}{\isadigit{1}}{\isacharcomma}{\kern0pt}{\isadigit{2}}{\isacharparenright}{\kern0pt}\ {\isacharcomma}{\kern0pt}\ {\isacharbrackleft}{\kern0pt}y{\isacharcomma}{\kern0pt}r{\isacharcomma}{\kern0pt}x{\isacharbrackright}{\kern0pt}{\isacharparenright}{\kern0pt}{\isacharparenright}{\kern0pt}{\isachardoublequoteclose}\isanewline
\ \ \ \ \isacommand{using}\isamarkupfalse%
\ separation{\isacharunderscore}{\kern0pt}ax\ \isacommand{by}\isamarkupfalse%
\ simp\isanewline
\ \ \isacommand{moreover}\isamarkupfalse%
\isanewline
\ \ \isacommand{have}\isamarkupfalse%
\ {\isachardoublequoteopen}{\isacharparenleft}{\kern0pt}{\isasymexists}p{\isasymin}M{\isachardot}{\kern0pt}\ p{\isasymin}r\ {\isacharampersand}{\kern0pt}\ pair{\isacharparenleft}{\kern0pt}{\isacharhash}{\kern0pt}{\isacharhash}{\kern0pt}M{\isacharcomma}{\kern0pt}y{\isacharcomma}{\kern0pt}x{\isacharcomma}{\kern0pt}p{\isacharparenright}{\kern0pt}{\isacharparenright}{\kern0pt}\isanewline
\ \ \ \ \ \ \ \ \ \ {\isasymlongleftrightarrow}\ sats{\isacharparenleft}{\kern0pt}M{\isacharcomma}{\kern0pt}pfm{\isacharparenleft}{\kern0pt}{\isadigit{0}}{\isacharcomma}{\kern0pt}{\isadigit{1}}{\isacharcomma}{\kern0pt}{\isadigit{2}}{\isacharparenright}{\kern0pt}\ {\isacharcomma}{\kern0pt}\ {\isacharbrackleft}{\kern0pt}y{\isacharcomma}{\kern0pt}r{\isacharcomma}{\kern0pt}x{\isacharbrackright}{\kern0pt}{\isacharparenright}{\kern0pt}{\isachardoublequoteclose}\isanewline
\ \ \ \ \isakeyword{if}\ {\isachardoublequoteopen}y{\isasymin}M{\isachardoublequoteclose}\ {\isachardoublequoteopen}r{\isasymin}M{\isachardoublequoteclose}\ {\isachardoublequoteopen}x{\isasymin}M{\isachardoublequoteclose}\ \isakeyword{for}\ y\ x\ r\isanewline
\ \ \ \ \isacommand{using}\isamarkupfalse%
\ that\ fmsats{\isacharbrackleft}{\kern0pt}of\ {\isachardoublequoteopen}{\isacharbrackleft}{\kern0pt}y{\isacharcomma}{\kern0pt}r{\isacharcomma}{\kern0pt}x{\isacharbrackright}{\kern0pt}{\isachardoublequoteclose}{\isacharbrackright}{\kern0pt}\ \isacommand{by}\isamarkupfalse%
\ simp\isanewline
\ \ \isacommand{ultimately}\isamarkupfalse%
\isanewline
\ \ \isacommand{have}\isamarkupfalse%
\ {\isachardoublequoteopen}{\isasymforall}x{\isasymin}M{\isachardot}{\kern0pt}\ {\isasymforall}r{\isasymin}M{\isachardot}{\kern0pt}\ separation{\isacharparenleft}{\kern0pt}{\isacharhash}{\kern0pt}{\isacharhash}{\kern0pt}M{\isacharcomma}{\kern0pt}\ {\isasymlambda}\ y\ {\isachardot}{\kern0pt}\ {\isasymexists}p{\isasymin}M{\isachardot}{\kern0pt}\ p{\isasymin}r\ {\isacharampersand}{\kern0pt}\ pair{\isacharparenleft}{\kern0pt}{\isacharhash}{\kern0pt}{\isacharhash}{\kern0pt}M{\isacharcomma}{\kern0pt}y{\isacharcomma}{\kern0pt}x{\isacharcomma}{\kern0pt}p{\isacharparenright}{\kern0pt}{\isacharparenright}{\kern0pt}{\isachardoublequoteclose}\isanewline
\ \ \ \ \isacommand{unfolding}\isamarkupfalse%
\ separation{\isacharunderscore}{\kern0pt}def\ \isacommand{by}\isamarkupfalse%
\ simp\isanewline
\ \ \isacommand{with}\isamarkupfalse%
\ {\isacartoucheopen}X{\isasymin}M{\isacartoucheclose}\ {\isacartoucheopen}R{\isasymin}M{\isacartoucheclose}\ \isacommand{show}\isamarkupfalse%
\ {\isacharquery}{\kern0pt}thesis\ \isacommand{by}\isamarkupfalse%
\ simp\isanewline
\isacommand{qed}\isamarkupfalse%
%
\endisatagproof
{\isafoldproof}%
%
\isadelimproof
\isanewline
%
\endisadelimproof
\isanewline
\isanewline
\isacommand{schematic{\isacharunderscore}{\kern0pt}goal}\isamarkupfalse%
\ mem{\isacharunderscore}{\kern0pt}fm{\isacharunderscore}{\kern0pt}auto{\isacharcolon}{\kern0pt}\isanewline
\ \ \isakeyword{assumes}\isanewline
\ \ \ \ {\isachardoublequoteopen}nth{\isacharparenleft}{\kern0pt}i{\isacharcomma}{\kern0pt}env{\isacharparenright}{\kern0pt}\ {\isacharequal}{\kern0pt}\ z{\isachardoublequoteclose}\ {\isachardoublequoteopen}i\ {\isasymin}\ nat{\isachardoublequoteclose}\ {\isachardoublequoteopen}env\ {\isasymin}\ list{\isacharparenleft}{\kern0pt}A{\isacharparenright}{\kern0pt}{\isachardoublequoteclose}\isanewline
\ \ \isakeyword{shows}\isanewline
\ \ \ \ {\isachardoublequoteopen}{\isacharparenleft}{\kern0pt}{\isasymexists}x{\isasymin}A{\isachardot}{\kern0pt}\ {\isasymexists}y{\isasymin}A{\isachardot}{\kern0pt}\ pair{\isacharparenleft}{\kern0pt}{\isacharhash}{\kern0pt}{\isacharhash}{\kern0pt}A{\isacharcomma}{\kern0pt}x{\isacharcomma}{\kern0pt}y{\isacharcomma}{\kern0pt}z{\isacharparenright}{\kern0pt}\ {\isacharampersand}{\kern0pt}\ x\ {\isasymin}\ y{\isacharparenright}{\kern0pt}\ {\isasymlongleftrightarrow}\ sats{\isacharparenleft}{\kern0pt}A{\isacharcomma}{\kern0pt}{\isacharquery}{\kern0pt}mfm{\isacharparenleft}{\kern0pt}i{\isacharparenright}{\kern0pt}{\isacharcomma}{\kern0pt}env{\isacharparenright}{\kern0pt}{\isachardoublequoteclose}\isanewline
%
\isadelimproof
\ \ %
\endisadelimproof
%
\isatagproof
\isacommand{by}\isamarkupfalse%
\ {\isacharparenleft}{\kern0pt}insert\ assms\ {\isacharsemicolon}{\kern0pt}\ {\isacharparenleft}{\kern0pt}rule\ sep{\isacharunderscore}{\kern0pt}rules\ {\isacharbar}{\kern0pt}\ simp{\isacharparenright}{\kern0pt}{\isacharplus}{\kern0pt}{\isacharparenright}{\kern0pt}%
\endisatagproof
{\isafoldproof}%
%
\isadelimproof
\isanewline
%
\endisadelimproof
\isanewline
\isacommand{lemma}\isamarkupfalse%
\ memrel{\isacharunderscore}{\kern0pt}sep{\isacharunderscore}{\kern0pt}intf{\isacharcolon}{\kern0pt}\isanewline
\ \ {\isachardoublequoteopen}separation{\isacharparenleft}{\kern0pt}{\isacharhash}{\kern0pt}{\isacharhash}{\kern0pt}M{\isacharcomma}{\kern0pt}\ {\isasymlambda}z{\isachardot}{\kern0pt}\ {\isasymexists}x{\isasymin}M{\isachardot}{\kern0pt}\ {\isasymexists}y{\isasymin}M{\isachardot}{\kern0pt}\ pair{\isacharparenleft}{\kern0pt}{\isacharhash}{\kern0pt}{\isacharhash}{\kern0pt}M{\isacharcomma}{\kern0pt}x{\isacharcomma}{\kern0pt}y{\isacharcomma}{\kern0pt}z{\isacharparenright}{\kern0pt}\ {\isacharampersand}{\kern0pt}\ x\ {\isasymin}\ y{\isacharparenright}{\kern0pt}{\isachardoublequoteclose}\isanewline
%
\isadelimproof
%
\endisadelimproof
%
\isatagproof
\isacommand{proof}\isamarkupfalse%
\ {\isacharminus}{\kern0pt}\isanewline
\ \ \isacommand{obtain}\isamarkupfalse%
\ mfm\ \isakeyword{where}\isanewline
\ \ \ \ fmsats{\isacharcolon}{\kern0pt}{\isachardoublequoteopen}{\isasymAnd}env{\isachardot}{\kern0pt}\ env{\isasymin}list{\isacharparenleft}{\kern0pt}M{\isacharparenright}{\kern0pt}\ {\isasymLongrightarrow}\isanewline
\ \ \ \ {\isacharparenleft}{\kern0pt}{\isasymexists}x{\isasymin}M{\isachardot}{\kern0pt}\ {\isasymexists}y{\isasymin}M{\isachardot}{\kern0pt}\ pair{\isacharparenleft}{\kern0pt}{\isacharhash}{\kern0pt}{\isacharhash}{\kern0pt}M{\isacharcomma}{\kern0pt}x{\isacharcomma}{\kern0pt}y{\isacharcomma}{\kern0pt}nth{\isacharparenleft}{\kern0pt}{\isadigit{0}}{\isacharcomma}{\kern0pt}env{\isacharparenright}{\kern0pt}{\isacharparenright}{\kern0pt}\ {\isacharampersand}{\kern0pt}\ x\ {\isasymin}\ y{\isacharparenright}{\kern0pt}\ {\isasymlongleftrightarrow}\ sats{\isacharparenleft}{\kern0pt}M{\isacharcomma}{\kern0pt}mfm{\isacharparenleft}{\kern0pt}{\isadigit{0}}{\isacharparenright}{\kern0pt}{\isacharcomma}{\kern0pt}env{\isacharparenright}{\kern0pt}{\isachardoublequoteclose}\isanewline
\ \ \ \ \isakeyword{and}\isanewline
\ \ \ \ {\isachardoublequoteopen}mfm{\isacharparenleft}{\kern0pt}{\isadigit{0}}{\isacharparenright}{\kern0pt}\ {\isasymin}\ formula{\isachardoublequoteclose}\isanewline
\ \ \ \ \isakeyword{and}\isanewline
\ \ \ \ {\isachardoublequoteopen}arity{\isacharparenleft}{\kern0pt}mfm{\isacharparenleft}{\kern0pt}{\isadigit{0}}{\isacharparenright}{\kern0pt}{\isacharparenright}{\kern0pt}\ {\isacharequal}{\kern0pt}\ {\isadigit{1}}{\isachardoublequoteclose}\isanewline
\ \ \ \ \isacommand{using}\isamarkupfalse%
\ mem{\isacharunderscore}{\kern0pt}fm{\isacharunderscore}{\kern0pt}auto\ \isacommand{by}\isamarkupfalse%
\ {\isacharparenleft}{\kern0pt}simp\ del{\isacharcolon}{\kern0pt}FOL{\isacharunderscore}{\kern0pt}sats{\isacharunderscore}{\kern0pt}iff\ pair{\isacharunderscore}{\kern0pt}abs\ add{\isacharcolon}{\kern0pt}\ fm{\isacharunderscore}{\kern0pt}defs\ nat{\isacharunderscore}{\kern0pt}simp{\isacharunderscore}{\kern0pt}union{\isacharparenright}{\kern0pt}\isanewline
\ \ \isacommand{then}\isamarkupfalse%
\isanewline
\ \ \isacommand{have}\isamarkupfalse%
\ {\isachardoublequoteopen}separation{\isacharparenleft}{\kern0pt}{\isacharhash}{\kern0pt}{\isacharhash}{\kern0pt}M{\isacharcomma}{\kern0pt}\ {\isasymlambda}z{\isachardot}{\kern0pt}\ sats{\isacharparenleft}{\kern0pt}M{\isacharcomma}{\kern0pt}mfm{\isacharparenleft}{\kern0pt}{\isadigit{0}}{\isacharparenright}{\kern0pt}\ {\isacharcomma}{\kern0pt}\ {\isacharbrackleft}{\kern0pt}z{\isacharbrackright}{\kern0pt}{\isacharparenright}{\kern0pt}{\isacharparenright}{\kern0pt}{\isachardoublequoteclose}\isanewline
\ \ \ \ \isacommand{using}\isamarkupfalse%
\ separation{\isacharunderscore}{\kern0pt}ax\ \isacommand{by}\isamarkupfalse%
\ simp\isanewline
\ \ \isacommand{moreover}\isamarkupfalse%
\isanewline
\ \ \isacommand{have}\isamarkupfalse%
\ {\isachardoublequoteopen}{\isacharparenleft}{\kern0pt}{\isasymexists}x{\isasymin}M{\isachardot}{\kern0pt}\ {\isasymexists}y{\isasymin}M{\isachardot}{\kern0pt}\ pair{\isacharparenleft}{\kern0pt}{\isacharhash}{\kern0pt}{\isacharhash}{\kern0pt}M{\isacharcomma}{\kern0pt}x{\isacharcomma}{\kern0pt}y{\isacharcomma}{\kern0pt}z{\isacharparenright}{\kern0pt}\ {\isacharampersand}{\kern0pt}\ x\ {\isasymin}\ y{\isacharparenright}{\kern0pt}\ {\isasymlongleftrightarrow}\ sats{\isacharparenleft}{\kern0pt}M{\isacharcomma}{\kern0pt}mfm{\isacharparenleft}{\kern0pt}{\isadigit{0}}{\isacharparenright}{\kern0pt}{\isacharcomma}{\kern0pt}{\isacharbrackleft}{\kern0pt}z{\isacharbrackright}{\kern0pt}{\isacharparenright}{\kern0pt}{\isachardoublequoteclose}\isanewline
\ \ \ \ \isakeyword{if}\ {\isachardoublequoteopen}z{\isasymin}M{\isachardoublequoteclose}\ \isakeyword{for}\ z\isanewline
\ \ \ \ \isacommand{using}\isamarkupfalse%
\ that\ fmsats{\isacharbrackleft}{\kern0pt}of\ {\isachardoublequoteopen}{\isacharbrackleft}{\kern0pt}z{\isacharbrackright}{\kern0pt}{\isachardoublequoteclose}{\isacharbrackright}{\kern0pt}\ \isacommand{by}\isamarkupfalse%
\ simp\isanewline
\ \ \isacommand{ultimately}\isamarkupfalse%
\isanewline
\ \ \isacommand{have}\isamarkupfalse%
\ {\isachardoublequoteopen}separation{\isacharparenleft}{\kern0pt}{\isacharhash}{\kern0pt}{\isacharhash}{\kern0pt}M{\isacharcomma}{\kern0pt}\ {\isasymlambda}z\ {\isachardot}{\kern0pt}\ {\isasymexists}x{\isasymin}M{\isachardot}{\kern0pt}\ {\isasymexists}y{\isasymin}M{\isachardot}{\kern0pt}\ pair{\isacharparenleft}{\kern0pt}{\isacharhash}{\kern0pt}{\isacharhash}{\kern0pt}M{\isacharcomma}{\kern0pt}x{\isacharcomma}{\kern0pt}y{\isacharcomma}{\kern0pt}z{\isacharparenright}{\kern0pt}\ {\isacharampersand}{\kern0pt}\ x\ {\isasymin}\ y{\isacharparenright}{\kern0pt}{\isachardoublequoteclose}\isanewline
\ \ \ \ \isacommand{unfolding}\isamarkupfalse%
\ separation{\isacharunderscore}{\kern0pt}def\ \isacommand{by}\isamarkupfalse%
\ simp\isanewline
\ \ \isacommand{then}\isamarkupfalse%
\ \isacommand{show}\isamarkupfalse%
\ {\isacharquery}{\kern0pt}thesis\ \isacommand{by}\isamarkupfalse%
\ simp\isanewline
\isacommand{qed}\isamarkupfalse%
%
\endisatagproof
{\isafoldproof}%
%
\isadelimproof
\isanewline
%
\endisadelimproof
\isanewline
\isacommand{schematic{\isacharunderscore}{\kern0pt}goal}\isamarkupfalse%
\ recfun{\isacharunderscore}{\kern0pt}fm{\isacharunderscore}{\kern0pt}auto{\isacharcolon}{\kern0pt}\isanewline
\ \ \isakeyword{assumes}\isanewline
\ \ \ \ {\isachardoublequoteopen}nth{\isacharparenleft}{\kern0pt}i{\isadigit{1}}{\isacharcomma}{\kern0pt}env{\isacharparenright}{\kern0pt}\ {\isacharequal}{\kern0pt}\ x{\isachardoublequoteclose}\ {\isachardoublequoteopen}nth{\isacharparenleft}{\kern0pt}i{\isadigit{2}}{\isacharcomma}{\kern0pt}env{\isacharparenright}{\kern0pt}\ {\isacharequal}{\kern0pt}\ r{\isachardoublequoteclose}\ {\isachardoublequoteopen}nth{\isacharparenleft}{\kern0pt}i{\isadigit{3}}{\isacharcomma}{\kern0pt}env{\isacharparenright}{\kern0pt}\ {\isacharequal}{\kern0pt}\ f{\isachardoublequoteclose}\ {\isachardoublequoteopen}nth{\isacharparenleft}{\kern0pt}i{\isadigit{4}}{\isacharcomma}{\kern0pt}env{\isacharparenright}{\kern0pt}\ {\isacharequal}{\kern0pt}\ g{\isachardoublequoteclose}\ {\isachardoublequoteopen}nth{\isacharparenleft}{\kern0pt}i{\isadigit{5}}{\isacharcomma}{\kern0pt}env{\isacharparenright}{\kern0pt}\ {\isacharequal}{\kern0pt}\ a{\isachardoublequoteclose}\isanewline
\ \ \ \ {\isachardoublequoteopen}nth{\isacharparenleft}{\kern0pt}i{\isadigit{6}}{\isacharcomma}{\kern0pt}env{\isacharparenright}{\kern0pt}\ {\isacharequal}{\kern0pt}\ b{\isachardoublequoteclose}\ {\isachardoublequoteopen}i{\isadigit{1}}{\isasymin}nat{\isachardoublequoteclose}\ {\isachardoublequoteopen}i{\isadigit{2}}{\isasymin}nat{\isachardoublequoteclose}\ {\isachardoublequoteopen}i{\isadigit{3}}{\isasymin}nat{\isachardoublequoteclose}\ {\isachardoublequoteopen}i{\isadigit{4}}{\isasymin}nat{\isachardoublequoteclose}\ {\isachardoublequoteopen}i{\isadigit{5}}{\isasymin}nat{\isachardoublequoteclose}\ {\isachardoublequoteopen}i{\isadigit{6}}{\isasymin}nat{\isachardoublequoteclose}\ {\isachardoublequoteopen}env\ {\isasymin}\ list{\isacharparenleft}{\kern0pt}A{\isacharparenright}{\kern0pt}{\isachardoublequoteclose}\isanewline
\ \ \isakeyword{shows}\isanewline
\ \ \ \ {\isachardoublequoteopen}{\isacharparenleft}{\kern0pt}{\isasymexists}xa{\isasymin}A{\isachardot}{\kern0pt}\ {\isasymexists}xb{\isasymin}A{\isachardot}{\kern0pt}\ pair{\isacharparenleft}{\kern0pt}{\isacharhash}{\kern0pt}{\isacharhash}{\kern0pt}A{\isacharcomma}{\kern0pt}x{\isacharcomma}{\kern0pt}a{\isacharcomma}{\kern0pt}xa{\isacharparenright}{\kern0pt}\ {\isacharampersand}{\kern0pt}\ xa\ {\isasymin}\ r\ {\isacharampersand}{\kern0pt}\ pair{\isacharparenleft}{\kern0pt}{\isacharhash}{\kern0pt}{\isacharhash}{\kern0pt}A{\isacharcomma}{\kern0pt}x{\isacharcomma}{\kern0pt}b{\isacharcomma}{\kern0pt}xb{\isacharparenright}{\kern0pt}\ {\isacharampersand}{\kern0pt}\ xb\ {\isasymin}\ r\ {\isacharampersand}{\kern0pt}\isanewline
\ \ \ \ \ \ \ \ \ \ \ \ \ \ \ \ \ \ {\isacharparenleft}{\kern0pt}{\isasymexists}fx{\isasymin}A{\isachardot}{\kern0pt}\ {\isasymexists}gx{\isasymin}A{\isachardot}{\kern0pt}\ fun{\isacharunderscore}{\kern0pt}apply{\isacharparenleft}{\kern0pt}{\isacharhash}{\kern0pt}{\isacharhash}{\kern0pt}A{\isacharcomma}{\kern0pt}f{\isacharcomma}{\kern0pt}x{\isacharcomma}{\kern0pt}fx{\isacharparenright}{\kern0pt}\ {\isacharampersand}{\kern0pt}\ fun{\isacharunderscore}{\kern0pt}apply{\isacharparenleft}{\kern0pt}{\isacharhash}{\kern0pt}{\isacharhash}{\kern0pt}A{\isacharcomma}{\kern0pt}g{\isacharcomma}{\kern0pt}x{\isacharcomma}{\kern0pt}gx{\isacharparenright}{\kern0pt}\ {\isacharampersand}{\kern0pt}\ fx\ {\isasymnoteq}\ gx{\isacharparenright}{\kern0pt}{\isacharparenright}{\kern0pt}\isanewline
\ \ \ \ {\isasymlongleftrightarrow}\ sats{\isacharparenleft}{\kern0pt}A{\isacharcomma}{\kern0pt}{\isacharquery}{\kern0pt}rffm{\isacharparenleft}{\kern0pt}i{\isadigit{1}}{\isacharcomma}{\kern0pt}i{\isadigit{2}}{\isacharcomma}{\kern0pt}i{\isadigit{3}}{\isacharcomma}{\kern0pt}i{\isadigit{4}}{\isacharcomma}{\kern0pt}i{\isadigit{5}}{\isacharcomma}{\kern0pt}i{\isadigit{6}}{\isacharparenright}{\kern0pt}{\isacharcomma}{\kern0pt}env{\isacharparenright}{\kern0pt}{\isachardoublequoteclose}\isanewline
%
\isadelimproof
\ \ %
\endisadelimproof
%
\isatagproof
\isacommand{by}\isamarkupfalse%
\ {\isacharparenleft}{\kern0pt}insert\ assms\ {\isacharsemicolon}{\kern0pt}\ {\isacharparenleft}{\kern0pt}rule\ sep{\isacharunderscore}{\kern0pt}rules\ {\isacharbar}{\kern0pt}\ simp{\isacharparenright}{\kern0pt}{\isacharplus}{\kern0pt}{\isacharparenright}{\kern0pt}%
\endisatagproof
{\isafoldproof}%
%
\isadelimproof
\isanewline
%
\endisadelimproof
\isanewline
\isanewline
\isacommand{lemma}\isamarkupfalse%
\ is{\isacharunderscore}{\kern0pt}recfun{\isacharunderscore}{\kern0pt}sep{\isacharunderscore}{\kern0pt}intf\ {\isacharcolon}{\kern0pt}\isanewline
\ \ \isakeyword{assumes}\isanewline
\ \ \ \ {\isachardoublequoteopen}r{\isasymin}M{\isachardoublequoteclose}\ {\isachardoublequoteopen}f{\isasymin}M{\isachardoublequoteclose}\ {\isachardoublequoteopen}g{\isasymin}M{\isachardoublequoteclose}\ {\isachardoublequoteopen}a{\isasymin}M{\isachardoublequoteclose}\ {\isachardoublequoteopen}b{\isasymin}M{\isachardoublequoteclose}\isanewline
\ \ \isakeyword{shows}\isanewline
\ \ \ \ {\isachardoublequoteopen}separation{\isacharparenleft}{\kern0pt}{\isacharhash}{\kern0pt}{\isacharhash}{\kern0pt}M{\isacharcomma}{\kern0pt}{\isasymlambda}x{\isachardot}{\kern0pt}\ {\isasymexists}xa{\isasymin}M{\isachardot}{\kern0pt}\ {\isasymexists}xb{\isasymin}M{\isachardot}{\kern0pt}\isanewline
\ \ \ \ \ \ \ \ \ \ \ \ \ \ \ \ \ \ \ \ pair{\isacharparenleft}{\kern0pt}{\isacharhash}{\kern0pt}{\isacharhash}{\kern0pt}M{\isacharcomma}{\kern0pt}x{\isacharcomma}{\kern0pt}a{\isacharcomma}{\kern0pt}xa{\isacharparenright}{\kern0pt}\ {\isacharampersand}{\kern0pt}\ xa\ {\isasymin}\ r\ {\isacharampersand}{\kern0pt}\ pair{\isacharparenleft}{\kern0pt}{\isacharhash}{\kern0pt}{\isacharhash}{\kern0pt}M{\isacharcomma}{\kern0pt}x{\isacharcomma}{\kern0pt}b{\isacharcomma}{\kern0pt}xb{\isacharparenright}{\kern0pt}\ {\isacharampersand}{\kern0pt}\ xb\ {\isasymin}\ r\ {\isacharampersand}{\kern0pt}\isanewline
\ \ \ \ \ \ \ \ \ \ \ \ \ \ \ \ \ \ \ \ {\isacharparenleft}{\kern0pt}{\isasymexists}fx{\isasymin}M{\isachardot}{\kern0pt}\ {\isasymexists}gx{\isasymin}M{\isachardot}{\kern0pt}\ fun{\isacharunderscore}{\kern0pt}apply{\isacharparenleft}{\kern0pt}{\isacharhash}{\kern0pt}{\isacharhash}{\kern0pt}M{\isacharcomma}{\kern0pt}f{\isacharcomma}{\kern0pt}x{\isacharcomma}{\kern0pt}fx{\isacharparenright}{\kern0pt}\ {\isacharampersand}{\kern0pt}\ fun{\isacharunderscore}{\kern0pt}apply{\isacharparenleft}{\kern0pt}{\isacharhash}{\kern0pt}{\isacharhash}{\kern0pt}M{\isacharcomma}{\kern0pt}g{\isacharcomma}{\kern0pt}x{\isacharcomma}{\kern0pt}gx{\isacharparenright}{\kern0pt}\ {\isacharampersand}{\kern0pt}\isanewline
\ \ \ \ \ \ \ \ \ \ \ \ \ \ \ \ \ \ \ \ \ \ \ \ \ \ \ \ \ \ \ \ \ \ \ \ \ fx\ {\isasymnoteq}\ gx{\isacharparenright}{\kern0pt}{\isacharparenright}{\kern0pt}{\isachardoublequoteclose}\isanewline
%
\isadelimproof
%
\endisadelimproof
%
\isatagproof
\isacommand{proof}\isamarkupfalse%
\ {\isacharminus}{\kern0pt}\isanewline
\ \ \isacommand{obtain}\isamarkupfalse%
\ rffm\ \isakeyword{where}\isanewline
\ \ \ \ fmsats{\isacharcolon}{\kern0pt}{\isachardoublequoteopen}{\isasymAnd}env{\isachardot}{\kern0pt}\ env{\isasymin}list{\isacharparenleft}{\kern0pt}M{\isacharparenright}{\kern0pt}\ {\isasymLongrightarrow}\isanewline
\ \ \ \ {\isacharparenleft}{\kern0pt}{\isasymexists}xa{\isasymin}M{\isachardot}{\kern0pt}\ {\isasymexists}xb{\isasymin}M{\isachardot}{\kern0pt}\ pair{\isacharparenleft}{\kern0pt}{\isacharhash}{\kern0pt}{\isacharhash}{\kern0pt}M{\isacharcomma}{\kern0pt}nth{\isacharparenleft}{\kern0pt}{\isadigit{0}}{\isacharcomma}{\kern0pt}env{\isacharparenright}{\kern0pt}{\isacharcomma}{\kern0pt}nth{\isacharparenleft}{\kern0pt}{\isadigit{4}}{\isacharcomma}{\kern0pt}env{\isacharparenright}{\kern0pt}{\isacharcomma}{\kern0pt}xa{\isacharparenright}{\kern0pt}\ {\isacharampersand}{\kern0pt}\ xa\ {\isasymin}\ nth{\isacharparenleft}{\kern0pt}{\isadigit{1}}{\isacharcomma}{\kern0pt}env{\isacharparenright}{\kern0pt}\ {\isacharampersand}{\kern0pt}\isanewline
\ \ \ \ pair{\isacharparenleft}{\kern0pt}{\isacharhash}{\kern0pt}{\isacharhash}{\kern0pt}M{\isacharcomma}{\kern0pt}nth{\isacharparenleft}{\kern0pt}{\isadigit{0}}{\isacharcomma}{\kern0pt}env{\isacharparenright}{\kern0pt}{\isacharcomma}{\kern0pt}nth{\isacharparenleft}{\kern0pt}{\isadigit{5}}{\isacharcomma}{\kern0pt}env{\isacharparenright}{\kern0pt}{\isacharcomma}{\kern0pt}xb{\isacharparenright}{\kern0pt}\ {\isacharampersand}{\kern0pt}\ xb\ {\isasymin}\ nth{\isacharparenleft}{\kern0pt}{\isadigit{1}}{\isacharcomma}{\kern0pt}env{\isacharparenright}{\kern0pt}\ {\isacharampersand}{\kern0pt}\ {\isacharparenleft}{\kern0pt}{\isasymexists}fx{\isasymin}M{\isachardot}{\kern0pt}\ {\isasymexists}gx{\isasymin}M{\isachardot}{\kern0pt}\isanewline
\ \ \ \ fun{\isacharunderscore}{\kern0pt}apply{\isacharparenleft}{\kern0pt}{\isacharhash}{\kern0pt}{\isacharhash}{\kern0pt}M{\isacharcomma}{\kern0pt}nth{\isacharparenleft}{\kern0pt}{\isadigit{2}}{\isacharcomma}{\kern0pt}env{\isacharparenright}{\kern0pt}{\isacharcomma}{\kern0pt}nth{\isacharparenleft}{\kern0pt}{\isadigit{0}}{\isacharcomma}{\kern0pt}env{\isacharparenright}{\kern0pt}{\isacharcomma}{\kern0pt}fx{\isacharparenright}{\kern0pt}\ {\isacharampersand}{\kern0pt}\ fun{\isacharunderscore}{\kern0pt}apply{\isacharparenleft}{\kern0pt}{\isacharhash}{\kern0pt}{\isacharhash}{\kern0pt}M{\isacharcomma}{\kern0pt}nth{\isacharparenleft}{\kern0pt}{\isadigit{3}}{\isacharcomma}{\kern0pt}env{\isacharparenright}{\kern0pt}{\isacharcomma}{\kern0pt}nth{\isacharparenleft}{\kern0pt}{\isadigit{0}}{\isacharcomma}{\kern0pt}env{\isacharparenright}{\kern0pt}{\isacharcomma}{\kern0pt}gx{\isacharparenright}{\kern0pt}\ {\isacharampersand}{\kern0pt}\ fx\ {\isasymnoteq}\ gx{\isacharparenright}{\kern0pt}{\isacharparenright}{\kern0pt}\isanewline
\ \ \ \ {\isasymlongleftrightarrow}\ sats{\isacharparenleft}{\kern0pt}M{\isacharcomma}{\kern0pt}rffm{\isacharparenleft}{\kern0pt}{\isadigit{0}}{\isacharcomma}{\kern0pt}{\isadigit{1}}{\isacharcomma}{\kern0pt}{\isadigit{2}}{\isacharcomma}{\kern0pt}{\isadigit{3}}{\isacharcomma}{\kern0pt}{\isadigit{4}}{\isacharcomma}{\kern0pt}{\isadigit{5}}{\isacharparenright}{\kern0pt}{\isacharcomma}{\kern0pt}env{\isacharparenright}{\kern0pt}{\isachardoublequoteclose}\isanewline
\ \ \ \ \isakeyword{and}\isanewline
\ \ \ \ {\isachardoublequoteopen}rffm{\isacharparenleft}{\kern0pt}{\isadigit{0}}{\isacharcomma}{\kern0pt}{\isadigit{1}}{\isacharcomma}{\kern0pt}{\isadigit{2}}{\isacharcomma}{\kern0pt}{\isadigit{3}}{\isacharcomma}{\kern0pt}{\isadigit{4}}{\isacharcomma}{\kern0pt}{\isadigit{5}}{\isacharparenright}{\kern0pt}\ {\isasymin}\ formula{\isachardoublequoteclose}\isanewline
\ \ \ \ \isakeyword{and}\isanewline
\ \ \ \ {\isachardoublequoteopen}arity{\isacharparenleft}{\kern0pt}rffm{\isacharparenleft}{\kern0pt}{\isadigit{0}}{\isacharcomma}{\kern0pt}{\isadigit{1}}{\isacharcomma}{\kern0pt}{\isadigit{2}}{\isacharcomma}{\kern0pt}{\isadigit{3}}{\isacharcomma}{\kern0pt}{\isadigit{4}}{\isacharcomma}{\kern0pt}{\isadigit{5}}{\isacharparenright}{\kern0pt}{\isacharparenright}{\kern0pt}\ {\isacharequal}{\kern0pt}\ {\isadigit{6}}{\isachardoublequoteclose}\isanewline
\ \ \ \ \isacommand{using}\isamarkupfalse%
\ recfun{\isacharunderscore}{\kern0pt}fm{\isacharunderscore}{\kern0pt}auto\ \isacommand{by}\isamarkupfalse%
\ {\isacharparenleft}{\kern0pt}simp\ del{\isacharcolon}{\kern0pt}FOL{\isacharunderscore}{\kern0pt}sats{\isacharunderscore}{\kern0pt}iff\ pair{\isacharunderscore}{\kern0pt}abs\ add{\isacharcolon}{\kern0pt}\ fm{\isacharunderscore}{\kern0pt}defs\ nat{\isacharunderscore}{\kern0pt}simp{\isacharunderscore}{\kern0pt}union{\isacharparenright}{\kern0pt}\isanewline
\ \ \isacommand{then}\isamarkupfalse%
\isanewline
\ \ \isacommand{have}\isamarkupfalse%
\ {\isachardoublequoteopen}{\isasymforall}a{\isadigit{1}}{\isasymin}M{\isachardot}{\kern0pt}\ {\isasymforall}a{\isadigit{2}}{\isasymin}M{\isachardot}{\kern0pt}\ {\isasymforall}a{\isadigit{3}}{\isasymin}M{\isachardot}{\kern0pt}\ {\isasymforall}a{\isadigit{4}}{\isasymin}M{\isachardot}{\kern0pt}\ {\isasymforall}a{\isadigit{5}}{\isasymin}M{\isachardot}{\kern0pt}\isanewline
\ \ \ \ \ \ \ \ separation{\isacharparenleft}{\kern0pt}{\isacharhash}{\kern0pt}{\isacharhash}{\kern0pt}M{\isacharcomma}{\kern0pt}\ {\isasymlambda}x{\isachardot}{\kern0pt}\ sats{\isacharparenleft}{\kern0pt}M{\isacharcomma}{\kern0pt}rffm{\isacharparenleft}{\kern0pt}{\isadigit{0}}{\isacharcomma}{\kern0pt}{\isadigit{1}}{\isacharcomma}{\kern0pt}{\isadigit{2}}{\isacharcomma}{\kern0pt}{\isadigit{3}}{\isacharcomma}{\kern0pt}{\isadigit{4}}{\isacharcomma}{\kern0pt}{\isadigit{5}}{\isacharparenright}{\kern0pt}\ {\isacharcomma}{\kern0pt}\ {\isacharbrackleft}{\kern0pt}x{\isacharcomma}{\kern0pt}a{\isadigit{1}}{\isacharcomma}{\kern0pt}a{\isadigit{2}}{\isacharcomma}{\kern0pt}a{\isadigit{3}}{\isacharcomma}{\kern0pt}a{\isadigit{4}}{\isacharcomma}{\kern0pt}a{\isadigit{5}}{\isacharbrackright}{\kern0pt}{\isacharparenright}{\kern0pt}{\isacharparenright}{\kern0pt}{\isachardoublequoteclose}\isanewline
\ \ \ \ \isacommand{using}\isamarkupfalse%
\ separation{\isacharunderscore}{\kern0pt}ax\ \isacommand{by}\isamarkupfalse%
\ simp\isanewline
\ \ \isacommand{moreover}\isamarkupfalse%
\isanewline
\ \ \isacommand{have}\isamarkupfalse%
\ {\isachardoublequoteopen}{\isacharparenleft}{\kern0pt}{\isasymexists}xa{\isasymin}M{\isachardot}{\kern0pt}\ {\isasymexists}xb{\isasymin}M{\isachardot}{\kern0pt}\ pair{\isacharparenleft}{\kern0pt}{\isacharhash}{\kern0pt}{\isacharhash}{\kern0pt}M{\isacharcomma}{\kern0pt}x{\isacharcomma}{\kern0pt}a{\isadigit{4}}{\isacharcomma}{\kern0pt}xa{\isacharparenright}{\kern0pt}\ {\isacharampersand}{\kern0pt}\ xa\ {\isasymin}\ a{\isadigit{1}}\ {\isacharampersand}{\kern0pt}\ pair{\isacharparenleft}{\kern0pt}{\isacharhash}{\kern0pt}{\isacharhash}{\kern0pt}M{\isacharcomma}{\kern0pt}x{\isacharcomma}{\kern0pt}a{\isadigit{5}}{\isacharcomma}{\kern0pt}xb{\isacharparenright}{\kern0pt}\ {\isacharampersand}{\kern0pt}\ xb\ {\isasymin}\ a{\isadigit{1}}\ {\isacharampersand}{\kern0pt}\isanewline
\ \ \ \ \ \ \ \ \ \ {\isacharparenleft}{\kern0pt}{\isasymexists}fx{\isasymin}M{\isachardot}{\kern0pt}\ {\isasymexists}gx{\isasymin}M{\isachardot}{\kern0pt}\ fun{\isacharunderscore}{\kern0pt}apply{\isacharparenleft}{\kern0pt}{\isacharhash}{\kern0pt}{\isacharhash}{\kern0pt}M{\isacharcomma}{\kern0pt}a{\isadigit{2}}{\isacharcomma}{\kern0pt}x{\isacharcomma}{\kern0pt}fx{\isacharparenright}{\kern0pt}\ {\isacharampersand}{\kern0pt}\ fun{\isacharunderscore}{\kern0pt}apply{\isacharparenleft}{\kern0pt}{\isacharhash}{\kern0pt}{\isacharhash}{\kern0pt}M{\isacharcomma}{\kern0pt}a{\isadigit{3}}{\isacharcomma}{\kern0pt}x{\isacharcomma}{\kern0pt}gx{\isacharparenright}{\kern0pt}\ {\isacharampersand}{\kern0pt}\ fx\ {\isasymnoteq}\ gx{\isacharparenright}{\kern0pt}{\isacharparenright}{\kern0pt}\isanewline
\ \ \ \ \ \ \ \ \ \ {\isasymlongleftrightarrow}\ sats{\isacharparenleft}{\kern0pt}M{\isacharcomma}{\kern0pt}rffm{\isacharparenleft}{\kern0pt}{\isadigit{0}}{\isacharcomma}{\kern0pt}{\isadigit{1}}{\isacharcomma}{\kern0pt}{\isadigit{2}}{\isacharcomma}{\kern0pt}{\isadigit{3}}{\isacharcomma}{\kern0pt}{\isadigit{4}}{\isacharcomma}{\kern0pt}{\isadigit{5}}{\isacharparenright}{\kern0pt}\ {\isacharcomma}{\kern0pt}\ {\isacharbrackleft}{\kern0pt}x{\isacharcomma}{\kern0pt}a{\isadigit{1}}{\isacharcomma}{\kern0pt}a{\isadigit{2}}{\isacharcomma}{\kern0pt}a{\isadigit{3}}{\isacharcomma}{\kern0pt}a{\isadigit{4}}{\isacharcomma}{\kern0pt}a{\isadigit{5}}{\isacharbrackright}{\kern0pt}{\isacharparenright}{\kern0pt}{\isachardoublequoteclose}\isanewline
\ \ \ \ \isakeyword{if}\ {\isachardoublequoteopen}x{\isasymin}M{\isachardoublequoteclose}\ {\isachardoublequoteopen}a{\isadigit{1}}{\isasymin}M{\isachardoublequoteclose}\ {\isachardoublequoteopen}a{\isadigit{2}}{\isasymin}M{\isachardoublequoteclose}\ {\isachardoublequoteopen}a{\isadigit{3}}{\isasymin}M{\isachardoublequoteclose}\ {\isachardoublequoteopen}a{\isadigit{4}}{\isasymin}M{\isachardoublequoteclose}\ {\isachardoublequoteopen}a{\isadigit{5}}{\isasymin}M{\isachardoublequoteclose}\ \ \isakeyword{for}\ x\ a{\isadigit{1}}\ a{\isadigit{2}}\ a{\isadigit{3}}\ a{\isadigit{4}}\ a{\isadigit{5}}\isanewline
\ \ \ \ \isacommand{using}\isamarkupfalse%
\ that\ fmsats{\isacharbrackleft}{\kern0pt}of\ {\isachardoublequoteopen}{\isacharbrackleft}{\kern0pt}x{\isacharcomma}{\kern0pt}a{\isadigit{1}}{\isacharcomma}{\kern0pt}a{\isadigit{2}}{\isacharcomma}{\kern0pt}a{\isadigit{3}}{\isacharcomma}{\kern0pt}a{\isadigit{4}}{\isacharcomma}{\kern0pt}a{\isadigit{5}}{\isacharbrackright}{\kern0pt}{\isachardoublequoteclose}{\isacharbrackright}{\kern0pt}\ \isacommand{by}\isamarkupfalse%
\ simp\isanewline
\ \ \isacommand{ultimately}\isamarkupfalse%
\isanewline
\ \ \isacommand{have}\isamarkupfalse%
\ {\isachardoublequoteopen}{\isasymforall}a{\isadigit{1}}{\isasymin}M{\isachardot}{\kern0pt}\ {\isasymforall}a{\isadigit{2}}{\isasymin}M{\isachardot}{\kern0pt}\ {\isasymforall}a{\isadigit{3}}{\isasymin}M{\isachardot}{\kern0pt}\ {\isasymforall}a{\isadigit{4}}{\isasymin}M{\isachardot}{\kern0pt}\ {\isasymforall}a{\isadigit{5}}{\isasymin}M{\isachardot}{\kern0pt}\ separation{\isacharparenleft}{\kern0pt}{\isacharhash}{\kern0pt}{\isacharhash}{\kern0pt}M{\isacharcomma}{\kern0pt}\ {\isasymlambda}\ x\ {\isachardot}{\kern0pt}\isanewline
\ \ \ \ \ \ \ \ \ \ {\isasymexists}xa{\isasymin}M{\isachardot}{\kern0pt}\ {\isasymexists}xb{\isasymin}M{\isachardot}{\kern0pt}\ pair{\isacharparenleft}{\kern0pt}{\isacharhash}{\kern0pt}{\isacharhash}{\kern0pt}M{\isacharcomma}{\kern0pt}x{\isacharcomma}{\kern0pt}a{\isadigit{4}}{\isacharcomma}{\kern0pt}xa{\isacharparenright}{\kern0pt}\ {\isacharampersand}{\kern0pt}\ xa\ {\isasymin}\ a{\isadigit{1}}\ {\isacharampersand}{\kern0pt}\ pair{\isacharparenleft}{\kern0pt}{\isacharhash}{\kern0pt}{\isacharhash}{\kern0pt}M{\isacharcomma}{\kern0pt}x{\isacharcomma}{\kern0pt}a{\isadigit{5}}{\isacharcomma}{\kern0pt}xb{\isacharparenright}{\kern0pt}\ {\isacharampersand}{\kern0pt}\ xb\ {\isasymin}\ a{\isadigit{1}}\ {\isacharampersand}{\kern0pt}\isanewline
\ \ \ \ \ \ \ \ \ \ {\isacharparenleft}{\kern0pt}{\isasymexists}fx{\isasymin}M{\isachardot}{\kern0pt}\ {\isasymexists}gx{\isasymin}M{\isachardot}{\kern0pt}\ fun{\isacharunderscore}{\kern0pt}apply{\isacharparenleft}{\kern0pt}{\isacharhash}{\kern0pt}{\isacharhash}{\kern0pt}M{\isacharcomma}{\kern0pt}a{\isadigit{2}}{\isacharcomma}{\kern0pt}x{\isacharcomma}{\kern0pt}fx{\isacharparenright}{\kern0pt}\ {\isacharampersand}{\kern0pt}\ fun{\isacharunderscore}{\kern0pt}apply{\isacharparenleft}{\kern0pt}{\isacharhash}{\kern0pt}{\isacharhash}{\kern0pt}M{\isacharcomma}{\kern0pt}a{\isadigit{3}}{\isacharcomma}{\kern0pt}x{\isacharcomma}{\kern0pt}gx{\isacharparenright}{\kern0pt}\ {\isacharampersand}{\kern0pt}\ fx\ {\isasymnoteq}\ gx{\isacharparenright}{\kern0pt}{\isacharparenright}{\kern0pt}{\isachardoublequoteclose}\isanewline
\ \ \ \ \isacommand{unfolding}\isamarkupfalse%
\ separation{\isacharunderscore}{\kern0pt}def\ \isacommand{by}\isamarkupfalse%
\ simp\isanewline
\ \ \isacommand{with}\isamarkupfalse%
\ {\isacartoucheopen}r{\isasymin}M{\isacartoucheclose}\ {\isacartoucheopen}f{\isasymin}M{\isacartoucheclose}\ {\isacartoucheopen}g{\isasymin}M{\isacartoucheclose}\ {\isacartoucheopen}a{\isasymin}M{\isacartoucheclose}\ {\isacartoucheopen}b{\isasymin}M{\isacartoucheclose}\ \isacommand{show}\isamarkupfalse%
\ {\isacharquery}{\kern0pt}thesis\ \isacommand{by}\isamarkupfalse%
\ simp\isanewline
\isacommand{qed}\isamarkupfalse%
%
\endisatagproof
{\isafoldproof}%
%
\isadelimproof
\isanewline
%
\endisadelimproof
\isanewline
\isanewline
\isanewline
\isanewline
\isacommand{schematic{\isacharunderscore}{\kern0pt}goal}\isamarkupfalse%
\ funsp{\isacharunderscore}{\kern0pt}fm{\isacharunderscore}{\kern0pt}auto{\isacharcolon}{\kern0pt}\isanewline
\ \ \isakeyword{assumes}\isanewline
\ \ \ \ {\isachardoublequoteopen}nth{\isacharparenleft}{\kern0pt}i{\isacharcomma}{\kern0pt}env{\isacharparenright}{\kern0pt}\ {\isacharequal}{\kern0pt}\ p{\isachardoublequoteclose}\ {\isachardoublequoteopen}nth{\isacharparenleft}{\kern0pt}j{\isacharcomma}{\kern0pt}env{\isacharparenright}{\kern0pt}\ {\isacharequal}{\kern0pt}\ z{\isachardoublequoteclose}\ {\isachardoublequoteopen}nth{\isacharparenleft}{\kern0pt}h{\isacharcomma}{\kern0pt}env{\isacharparenright}{\kern0pt}\ {\isacharequal}{\kern0pt}\ n{\isachardoublequoteclose}\isanewline
\ \ \ \ {\isachardoublequoteopen}i\ {\isasymin}\ nat{\isachardoublequoteclose}\ {\isachardoublequoteopen}j\ {\isasymin}\ nat{\isachardoublequoteclose}\ {\isachardoublequoteopen}h\ {\isasymin}\ nat{\isachardoublequoteclose}\ {\isachardoublequoteopen}env\ {\isasymin}\ list{\isacharparenleft}{\kern0pt}A{\isacharparenright}{\kern0pt}{\isachardoublequoteclose}\isanewline
\ \ \isakeyword{shows}\isanewline
\ \ \ \ {\isachardoublequoteopen}{\isacharparenleft}{\kern0pt}{\isasymexists}f{\isasymin}A{\isachardot}{\kern0pt}\ {\isasymexists}b{\isasymin}A{\isachardot}{\kern0pt}\ {\isasymexists}nb{\isasymin}A{\isachardot}{\kern0pt}\ {\isasymexists}cnbf{\isasymin}A{\isachardot}{\kern0pt}\ pair{\isacharparenleft}{\kern0pt}{\isacharhash}{\kern0pt}{\isacharhash}{\kern0pt}A{\isacharcomma}{\kern0pt}f{\isacharcomma}{\kern0pt}b{\isacharcomma}{\kern0pt}p{\isacharparenright}{\kern0pt}\ {\isacharampersand}{\kern0pt}\ pair{\isacharparenleft}{\kern0pt}{\isacharhash}{\kern0pt}{\isacharhash}{\kern0pt}A{\isacharcomma}{\kern0pt}n{\isacharcomma}{\kern0pt}b{\isacharcomma}{\kern0pt}nb{\isacharparenright}{\kern0pt}\ {\isacharampersand}{\kern0pt}\ is{\isacharunderscore}{\kern0pt}cons{\isacharparenleft}{\kern0pt}{\isacharhash}{\kern0pt}{\isacharhash}{\kern0pt}A{\isacharcomma}{\kern0pt}nb{\isacharcomma}{\kern0pt}f{\isacharcomma}{\kern0pt}cnbf{\isacharparenright}{\kern0pt}\ {\isacharampersand}{\kern0pt}\isanewline
\ \ \ \ upair{\isacharparenleft}{\kern0pt}{\isacharhash}{\kern0pt}{\isacharhash}{\kern0pt}A{\isacharcomma}{\kern0pt}cnbf{\isacharcomma}{\kern0pt}cnbf{\isacharcomma}{\kern0pt}z{\isacharparenright}{\kern0pt}{\isacharparenright}{\kern0pt}\ {\isasymlongleftrightarrow}\ sats{\isacharparenleft}{\kern0pt}A{\isacharcomma}{\kern0pt}{\isacharquery}{\kern0pt}fsfm{\isacharparenleft}{\kern0pt}i{\isacharcomma}{\kern0pt}j{\isacharcomma}{\kern0pt}h{\isacharparenright}{\kern0pt}{\isacharcomma}{\kern0pt}env{\isacharparenright}{\kern0pt}{\isachardoublequoteclose}\isanewline
%
\isadelimproof
\ \ %
\endisadelimproof
%
\isatagproof
\isacommand{by}\isamarkupfalse%
\ {\isacharparenleft}{\kern0pt}insert\ assms\ {\isacharsemicolon}{\kern0pt}\ {\isacharparenleft}{\kern0pt}rule\ sep{\isacharunderscore}{\kern0pt}rules\ {\isacharbar}{\kern0pt}\ simp{\isacharparenright}{\kern0pt}{\isacharplus}{\kern0pt}{\isacharparenright}{\kern0pt}%
\endisatagproof
{\isafoldproof}%
%
\isadelimproof
\isanewline
%
\endisadelimproof
\isanewline
\isanewline
\isacommand{lemma}\isamarkupfalse%
\ funspace{\isacharunderscore}{\kern0pt}succ{\isacharunderscore}{\kern0pt}rep{\isacharunderscore}{\kern0pt}intf\ {\isacharcolon}{\kern0pt}\isanewline
\ \ \isakeyword{assumes}\isanewline
\ \ \ \ {\isachardoublequoteopen}n{\isasymin}M{\isachardoublequoteclose}\isanewline
\ \ \isakeyword{shows}\isanewline
\ \ \ \ {\isachardoublequoteopen}strong{\isacharunderscore}{\kern0pt}replacement{\isacharparenleft}{\kern0pt}{\isacharhash}{\kern0pt}{\isacharhash}{\kern0pt}M{\isacharcomma}{\kern0pt}\isanewline
\ \ \ \ \ \ \ \ \ \ {\isasymlambda}p\ z{\isachardot}{\kern0pt}\ {\isasymexists}f{\isasymin}M{\isachardot}{\kern0pt}\ {\isasymexists}b{\isasymin}M{\isachardot}{\kern0pt}\ {\isasymexists}nb{\isasymin}M{\isachardot}{\kern0pt}\ {\isasymexists}cnbf{\isasymin}M{\isachardot}{\kern0pt}\isanewline
\ \ \ \ \ \ \ \ \ \ \ \ \ \ \ \ pair{\isacharparenleft}{\kern0pt}{\isacharhash}{\kern0pt}{\isacharhash}{\kern0pt}M{\isacharcomma}{\kern0pt}f{\isacharcomma}{\kern0pt}b{\isacharcomma}{\kern0pt}p{\isacharparenright}{\kern0pt}\ {\isacharampersand}{\kern0pt}\ pair{\isacharparenleft}{\kern0pt}{\isacharhash}{\kern0pt}{\isacharhash}{\kern0pt}M{\isacharcomma}{\kern0pt}n{\isacharcomma}{\kern0pt}b{\isacharcomma}{\kern0pt}nb{\isacharparenright}{\kern0pt}\ {\isacharampersand}{\kern0pt}\ is{\isacharunderscore}{\kern0pt}cons{\isacharparenleft}{\kern0pt}{\isacharhash}{\kern0pt}{\isacharhash}{\kern0pt}M{\isacharcomma}{\kern0pt}nb{\isacharcomma}{\kern0pt}f{\isacharcomma}{\kern0pt}cnbf{\isacharparenright}{\kern0pt}\ {\isacharampersand}{\kern0pt}\isanewline
\ \ \ \ \ \ \ \ \ \ \ \ \ \ \ \ upair{\isacharparenleft}{\kern0pt}{\isacharhash}{\kern0pt}{\isacharhash}{\kern0pt}M{\isacharcomma}{\kern0pt}cnbf{\isacharcomma}{\kern0pt}cnbf{\isacharcomma}{\kern0pt}z{\isacharparenright}{\kern0pt}{\isacharparenright}{\kern0pt}{\isachardoublequoteclose}\isanewline
%
\isadelimproof
%
\endisadelimproof
%
\isatagproof
\isacommand{proof}\isamarkupfalse%
\ {\isacharminus}{\kern0pt}\isanewline
\ \ \isacommand{obtain}\isamarkupfalse%
\ fsfm\ \isakeyword{where}\isanewline
\ \ \ \ fmsats{\isacharcolon}{\kern0pt}{\isachardoublequoteopen}env{\isasymin}list{\isacharparenleft}{\kern0pt}M{\isacharparenright}{\kern0pt}\ {\isasymLongrightarrow}\isanewline
\ \ \ \ {\isacharparenleft}{\kern0pt}{\isasymexists}f{\isasymin}M{\isachardot}{\kern0pt}\ {\isasymexists}b{\isasymin}M{\isachardot}{\kern0pt}\ {\isasymexists}nb{\isasymin}M{\isachardot}{\kern0pt}\ {\isasymexists}cnbf{\isasymin}M{\isachardot}{\kern0pt}\ pair{\isacharparenleft}{\kern0pt}{\isacharhash}{\kern0pt}{\isacharhash}{\kern0pt}M{\isacharcomma}{\kern0pt}f{\isacharcomma}{\kern0pt}b{\isacharcomma}{\kern0pt}nth{\isacharparenleft}{\kern0pt}{\isadigit{0}}{\isacharcomma}{\kern0pt}env{\isacharparenright}{\kern0pt}{\isacharparenright}{\kern0pt}\ {\isacharampersand}{\kern0pt}\ pair{\isacharparenleft}{\kern0pt}{\isacharhash}{\kern0pt}{\isacharhash}{\kern0pt}M{\isacharcomma}{\kern0pt}nth{\isacharparenleft}{\kern0pt}{\isadigit{2}}{\isacharcomma}{\kern0pt}env{\isacharparenright}{\kern0pt}{\isacharcomma}{\kern0pt}b{\isacharcomma}{\kern0pt}nb{\isacharparenright}{\kern0pt}\isanewline
\ \ \ \ \ \ {\isacharampersand}{\kern0pt}\ is{\isacharunderscore}{\kern0pt}cons{\isacharparenleft}{\kern0pt}{\isacharhash}{\kern0pt}{\isacharhash}{\kern0pt}M{\isacharcomma}{\kern0pt}nb{\isacharcomma}{\kern0pt}f{\isacharcomma}{\kern0pt}cnbf{\isacharparenright}{\kern0pt}\ {\isacharampersand}{\kern0pt}\ upair{\isacharparenleft}{\kern0pt}{\isacharhash}{\kern0pt}{\isacharhash}{\kern0pt}M{\isacharcomma}{\kern0pt}cnbf{\isacharcomma}{\kern0pt}cnbf{\isacharcomma}{\kern0pt}nth{\isacharparenleft}{\kern0pt}{\isadigit{1}}{\isacharcomma}{\kern0pt}env{\isacharparenright}{\kern0pt}{\isacharparenright}{\kern0pt}{\isacharparenright}{\kern0pt}\isanewline
\ \ \ \ {\isasymlongleftrightarrow}\ sats{\isacharparenleft}{\kern0pt}M{\isacharcomma}{\kern0pt}fsfm{\isacharparenleft}{\kern0pt}{\isadigit{0}}{\isacharcomma}{\kern0pt}{\isadigit{1}}{\isacharcomma}{\kern0pt}{\isadigit{2}}{\isacharparenright}{\kern0pt}{\isacharcomma}{\kern0pt}env{\isacharparenright}{\kern0pt}{\isachardoublequoteclose}\isanewline
\ \ \ \ \isakeyword{and}\ {\isachardoublequoteopen}fsfm{\isacharparenleft}{\kern0pt}{\isadigit{0}}{\isacharcomma}{\kern0pt}{\isadigit{1}}{\isacharcomma}{\kern0pt}{\isadigit{2}}{\isacharparenright}{\kern0pt}\ {\isasymin}\ formula{\isachardoublequoteclose}\ \isakeyword{and}\ {\isachardoublequoteopen}arity{\isacharparenleft}{\kern0pt}fsfm{\isacharparenleft}{\kern0pt}{\isadigit{0}}{\isacharcomma}{\kern0pt}{\isadigit{1}}{\isacharcomma}{\kern0pt}{\isadigit{2}}{\isacharparenright}{\kern0pt}{\isacharparenright}{\kern0pt}\ {\isacharequal}{\kern0pt}\ {\isadigit{3}}{\isachardoublequoteclose}\ \isakeyword{for}\ env\isanewline
\ \ \ \ \isacommand{using}\isamarkupfalse%
\ funsp{\isacharunderscore}{\kern0pt}fm{\isacharunderscore}{\kern0pt}auto{\isacharbrackleft}{\kern0pt}of\ concl{\isacharcolon}{\kern0pt}M{\isacharbrackright}{\kern0pt}\ \isacommand{by}\isamarkupfalse%
\ {\isacharparenleft}{\kern0pt}simp\ del{\isacharcolon}{\kern0pt}FOL{\isacharunderscore}{\kern0pt}sats{\isacharunderscore}{\kern0pt}iff\ pair{\isacharunderscore}{\kern0pt}abs\ add{\isacharcolon}{\kern0pt}\ fm{\isacharunderscore}{\kern0pt}defs\ nat{\isacharunderscore}{\kern0pt}simp{\isacharunderscore}{\kern0pt}union{\isacharparenright}{\kern0pt}\isanewline
\ \ \isacommand{then}\isamarkupfalse%
\isanewline
\ \ \isacommand{have}\isamarkupfalse%
\ {\isachardoublequoteopen}{\isasymforall}n{\isadigit{0}}{\isasymin}M{\isachardot}{\kern0pt}\ strong{\isacharunderscore}{\kern0pt}replacement{\isacharparenleft}{\kern0pt}{\isacharhash}{\kern0pt}{\isacharhash}{\kern0pt}M{\isacharcomma}{\kern0pt}\ {\isasymlambda}p\ z{\isachardot}{\kern0pt}\ sats{\isacharparenleft}{\kern0pt}M{\isacharcomma}{\kern0pt}fsfm{\isacharparenleft}{\kern0pt}{\isadigit{0}}{\isacharcomma}{\kern0pt}{\isadigit{1}}{\isacharcomma}{\kern0pt}{\isadigit{2}}{\isacharparenright}{\kern0pt}\ {\isacharcomma}{\kern0pt}\ {\isacharbrackleft}{\kern0pt}p{\isacharcomma}{\kern0pt}z{\isacharcomma}{\kern0pt}n{\isadigit{0}}{\isacharbrackright}{\kern0pt}{\isacharparenright}{\kern0pt}{\isacharparenright}{\kern0pt}{\isachardoublequoteclose}\isanewline
\ \ \ \ \isacommand{using}\isamarkupfalse%
\ replacement{\isacharunderscore}{\kern0pt}ax\ \isacommand{by}\isamarkupfalse%
\ simp\isanewline
\ \ \isacommand{moreover}\isamarkupfalse%
\isanewline
\ \ \isacommand{have}\isamarkupfalse%
\ {\isachardoublequoteopen}{\isacharparenleft}{\kern0pt}{\isasymexists}f{\isasymin}M{\isachardot}{\kern0pt}\ {\isasymexists}b{\isasymin}M{\isachardot}{\kern0pt}\ {\isasymexists}nb{\isasymin}M{\isachardot}{\kern0pt}\ {\isasymexists}cnbf{\isasymin}M{\isachardot}{\kern0pt}\ pair{\isacharparenleft}{\kern0pt}{\isacharhash}{\kern0pt}{\isacharhash}{\kern0pt}M{\isacharcomma}{\kern0pt}f{\isacharcomma}{\kern0pt}b{\isacharcomma}{\kern0pt}p{\isacharparenright}{\kern0pt}\ {\isacharampersand}{\kern0pt}\ pair{\isacharparenleft}{\kern0pt}{\isacharhash}{\kern0pt}{\isacharhash}{\kern0pt}M{\isacharcomma}{\kern0pt}n{\isadigit{0}}{\isacharcomma}{\kern0pt}b{\isacharcomma}{\kern0pt}nb{\isacharparenright}{\kern0pt}\ {\isacharampersand}{\kern0pt}\isanewline
\ \ \ \ \ \ \ \ \ \ is{\isacharunderscore}{\kern0pt}cons{\isacharparenleft}{\kern0pt}{\isacharhash}{\kern0pt}{\isacharhash}{\kern0pt}M{\isacharcomma}{\kern0pt}nb{\isacharcomma}{\kern0pt}f{\isacharcomma}{\kern0pt}cnbf{\isacharparenright}{\kern0pt}\ {\isacharampersand}{\kern0pt}\ upair{\isacharparenleft}{\kern0pt}{\isacharhash}{\kern0pt}{\isacharhash}{\kern0pt}M{\isacharcomma}{\kern0pt}cnbf{\isacharcomma}{\kern0pt}cnbf{\isacharcomma}{\kern0pt}z{\isacharparenright}{\kern0pt}{\isacharparenright}{\kern0pt}\isanewline
\ \ \ \ \ \ \ \ \ \ {\isasymlongleftrightarrow}\ sats{\isacharparenleft}{\kern0pt}M{\isacharcomma}{\kern0pt}fsfm{\isacharparenleft}{\kern0pt}{\isadigit{0}}{\isacharcomma}{\kern0pt}{\isadigit{1}}{\isacharcomma}{\kern0pt}{\isadigit{2}}{\isacharparenright}{\kern0pt}\ {\isacharcomma}{\kern0pt}\ {\isacharbrackleft}{\kern0pt}p{\isacharcomma}{\kern0pt}z{\isacharcomma}{\kern0pt}n{\isadigit{0}}{\isacharbrackright}{\kern0pt}{\isacharparenright}{\kern0pt}{\isachardoublequoteclose}\isanewline
\ \ \ \ \isakeyword{if}\ {\isachardoublequoteopen}p{\isasymin}M{\isachardoublequoteclose}\ {\isachardoublequoteopen}z{\isasymin}M{\isachardoublequoteclose}\ {\isachardoublequoteopen}n{\isadigit{0}}{\isasymin}M{\isachardoublequoteclose}\ \isakeyword{for}\ p\ z\ n{\isadigit{0}}\isanewline
\ \ \ \ \isacommand{using}\isamarkupfalse%
\ that\ fmsats{\isacharbrackleft}{\kern0pt}of\ {\isachardoublequoteopen}{\isacharbrackleft}{\kern0pt}p{\isacharcomma}{\kern0pt}z{\isacharcomma}{\kern0pt}n{\isadigit{0}}{\isacharbrackright}{\kern0pt}{\isachardoublequoteclose}{\isacharbrackright}{\kern0pt}\ \isacommand{by}\isamarkupfalse%
\ simp\isanewline
\ \ \isacommand{ultimately}\isamarkupfalse%
\isanewline
\ \ \isacommand{have}\isamarkupfalse%
\ {\isachardoublequoteopen}{\isasymforall}n{\isadigit{0}}{\isasymin}M{\isachardot}{\kern0pt}\ strong{\isacharunderscore}{\kern0pt}replacement{\isacharparenleft}{\kern0pt}{\isacharhash}{\kern0pt}{\isacharhash}{\kern0pt}M{\isacharcomma}{\kern0pt}\ {\isasymlambda}\ p\ z{\isachardot}{\kern0pt}\isanewline
\ \ \ \ \ \ \ \ \ \ {\isasymexists}f{\isasymin}M{\isachardot}{\kern0pt}\ {\isasymexists}b{\isasymin}M{\isachardot}{\kern0pt}\ {\isasymexists}nb{\isasymin}M{\isachardot}{\kern0pt}\ {\isasymexists}cnbf{\isasymin}M{\isachardot}{\kern0pt}\ pair{\isacharparenleft}{\kern0pt}{\isacharhash}{\kern0pt}{\isacharhash}{\kern0pt}M{\isacharcomma}{\kern0pt}f{\isacharcomma}{\kern0pt}b{\isacharcomma}{\kern0pt}p{\isacharparenright}{\kern0pt}\ {\isacharampersand}{\kern0pt}\ pair{\isacharparenleft}{\kern0pt}{\isacharhash}{\kern0pt}{\isacharhash}{\kern0pt}M{\isacharcomma}{\kern0pt}n{\isadigit{0}}{\isacharcomma}{\kern0pt}b{\isacharcomma}{\kern0pt}nb{\isacharparenright}{\kern0pt}\ {\isacharampersand}{\kern0pt}\isanewline
\ \ \ \ \ \ \ \ \ \ is{\isacharunderscore}{\kern0pt}cons{\isacharparenleft}{\kern0pt}{\isacharhash}{\kern0pt}{\isacharhash}{\kern0pt}M{\isacharcomma}{\kern0pt}nb{\isacharcomma}{\kern0pt}f{\isacharcomma}{\kern0pt}cnbf{\isacharparenright}{\kern0pt}\ {\isacharampersand}{\kern0pt}\ upair{\isacharparenleft}{\kern0pt}{\isacharhash}{\kern0pt}{\isacharhash}{\kern0pt}M{\isacharcomma}{\kern0pt}cnbf{\isacharcomma}{\kern0pt}cnbf{\isacharcomma}{\kern0pt}z{\isacharparenright}{\kern0pt}{\isacharparenright}{\kern0pt}{\isachardoublequoteclose}\isanewline
\ \ \ \ \isacommand{unfolding}\isamarkupfalse%
\ strong{\isacharunderscore}{\kern0pt}replacement{\isacharunderscore}{\kern0pt}def\ univalent{\isacharunderscore}{\kern0pt}def\ \isacommand{by}\isamarkupfalse%
\ simp\isanewline
\ \ \isacommand{with}\isamarkupfalse%
\ {\isacartoucheopen}n{\isasymin}M{\isacartoucheclose}\ \isacommand{show}\isamarkupfalse%
\ {\isacharquery}{\kern0pt}thesis\ \isacommand{by}\isamarkupfalse%
\ simp\isanewline
\isacommand{qed}\isamarkupfalse%
%
\endisatagproof
{\isafoldproof}%
%
\isadelimproof
\isanewline
%
\endisadelimproof
\isanewline
\isanewline
\isanewline
\isanewline
\isacommand{lemmas}\isamarkupfalse%
\ M{\isacharunderscore}{\kern0pt}basic{\isacharunderscore}{\kern0pt}sep{\isacharunderscore}{\kern0pt}instances\ {\isacharequal}{\kern0pt}\isanewline
\ \ inter{\isacharunderscore}{\kern0pt}sep{\isacharunderscore}{\kern0pt}intf\ diff{\isacharunderscore}{\kern0pt}sep{\isacharunderscore}{\kern0pt}intf\ cartprod{\isacharunderscore}{\kern0pt}sep{\isacharunderscore}{\kern0pt}intf\isanewline
\ \ image{\isacharunderscore}{\kern0pt}sep{\isacharunderscore}{\kern0pt}intf\ converse{\isacharunderscore}{\kern0pt}sep{\isacharunderscore}{\kern0pt}intf\ restrict{\isacharunderscore}{\kern0pt}sep{\isacharunderscore}{\kern0pt}intf\isanewline
\ \ pred{\isacharunderscore}{\kern0pt}sep{\isacharunderscore}{\kern0pt}intf\ memrel{\isacharunderscore}{\kern0pt}sep{\isacharunderscore}{\kern0pt}intf\ comp{\isacharunderscore}{\kern0pt}sep{\isacharunderscore}{\kern0pt}intf\ is{\isacharunderscore}{\kern0pt}recfun{\isacharunderscore}{\kern0pt}sep{\isacharunderscore}{\kern0pt}intf\isanewline
\isanewline
\isacommand{lemma}\isamarkupfalse%
\ mbasic\ {\isacharcolon}{\kern0pt}\ {\isachardoublequoteopen}M{\isacharunderscore}{\kern0pt}basic{\isacharparenleft}{\kern0pt}{\isacharhash}{\kern0pt}{\isacharhash}{\kern0pt}M{\isacharparenright}{\kern0pt}{\isachardoublequoteclose}\isanewline
%
\isadelimproof
\ \ %
\endisadelimproof
%
\isatagproof
\isacommand{using}\isamarkupfalse%
\ trans{\isacharunderscore}{\kern0pt}M\ zero{\isacharunderscore}{\kern0pt}in{\isacharunderscore}{\kern0pt}M\ power{\isacharunderscore}{\kern0pt}ax\ M{\isacharunderscore}{\kern0pt}basic{\isacharunderscore}{\kern0pt}sep{\isacharunderscore}{\kern0pt}instances\ funspace{\isacharunderscore}{\kern0pt}succ{\isacharunderscore}{\kern0pt}rep{\isacharunderscore}{\kern0pt}intf\ mtriv\isanewline
\ \ \isacommand{by}\isamarkupfalse%
\ unfold{\isacharunderscore}{\kern0pt}locales\ auto%
\endisatagproof
{\isafoldproof}%
%
\isadelimproof
\isanewline
%
\endisadelimproof
\isanewline
\isacommand{end}\isamarkupfalse%
\isanewline
\isanewline
\isacommand{sublocale}\isamarkupfalse%
\ M{\isacharunderscore}{\kern0pt}ZF{\isacharunderscore}{\kern0pt}trans\ {\isasymsubseteq}\ M{\isacharunderscore}{\kern0pt}basic\ {\isachardoublequoteopen}{\isacharhash}{\kern0pt}{\isacharhash}{\kern0pt}M{\isachardoublequoteclose}\isanewline
%
\isadelimproof
\ \ %
\endisadelimproof
%
\isatagproof
\isacommand{by}\isamarkupfalse%
\ {\isacharparenleft}{\kern0pt}rule\ mbasic{\isacharparenright}{\kern0pt}%
\endisatagproof
{\isafoldproof}%
%
\isadelimproof
%
\endisadelimproof
%
\isadelimdocument
%
\endisadelimdocument
%
\isatagdocument
%
\isamarkupsubsection{Interface with \isa{M{\isacharunderscore}{\kern0pt}trancl}%
}
\isamarkuptrue%
%
\endisatagdocument
{\isafolddocument}%
%
\isadelimdocument
%
\endisadelimdocument
\isacommand{schematic{\isacharunderscore}{\kern0pt}goal}\isamarkupfalse%
\ rtran{\isacharunderscore}{\kern0pt}closure{\isacharunderscore}{\kern0pt}mem{\isacharunderscore}{\kern0pt}auto{\isacharcolon}{\kern0pt}\isanewline
\ \ \isakeyword{assumes}\isanewline
\ \ \ \ {\isachardoublequoteopen}nth{\isacharparenleft}{\kern0pt}i{\isacharcomma}{\kern0pt}env{\isacharparenright}{\kern0pt}\ {\isacharequal}{\kern0pt}\ p{\isachardoublequoteclose}\ {\isachardoublequoteopen}nth{\isacharparenleft}{\kern0pt}j{\isacharcomma}{\kern0pt}env{\isacharparenright}{\kern0pt}\ {\isacharequal}{\kern0pt}\ r{\isachardoublequoteclose}\ \ {\isachardoublequoteopen}nth{\isacharparenleft}{\kern0pt}k{\isacharcomma}{\kern0pt}env{\isacharparenright}{\kern0pt}\ {\isacharequal}{\kern0pt}\ B{\isachardoublequoteclose}\isanewline
\ \ \ \ {\isachardoublequoteopen}i\ {\isasymin}\ nat{\isachardoublequoteclose}\ {\isachardoublequoteopen}j\ {\isasymin}\ nat{\isachardoublequoteclose}\ {\isachardoublequoteopen}k\ {\isasymin}\ nat{\isachardoublequoteclose}\ {\isachardoublequoteopen}env\ {\isasymin}\ list{\isacharparenleft}{\kern0pt}A{\isacharparenright}{\kern0pt}{\isachardoublequoteclose}\isanewline
\ \ \isakeyword{shows}\isanewline
\ \ \ \ {\isachardoublequoteopen}rtran{\isacharunderscore}{\kern0pt}closure{\isacharunderscore}{\kern0pt}mem{\isacharparenleft}{\kern0pt}{\isacharhash}{\kern0pt}{\isacharhash}{\kern0pt}A{\isacharcomma}{\kern0pt}B{\isacharcomma}{\kern0pt}r{\isacharcomma}{\kern0pt}p{\isacharparenright}{\kern0pt}\ {\isasymlongleftrightarrow}\ sats{\isacharparenleft}{\kern0pt}A{\isacharcomma}{\kern0pt}{\isacharquery}{\kern0pt}rcfm{\isacharparenleft}{\kern0pt}i{\isacharcomma}{\kern0pt}j{\isacharcomma}{\kern0pt}k{\isacharparenright}{\kern0pt}{\isacharcomma}{\kern0pt}env{\isacharparenright}{\kern0pt}{\isachardoublequoteclose}\isanewline
%
\isadelimproof
\ \ %
\endisadelimproof
%
\isatagproof
\isacommand{unfolding}\isamarkupfalse%
\ rtran{\isacharunderscore}{\kern0pt}closure{\isacharunderscore}{\kern0pt}mem{\isacharunderscore}{\kern0pt}def\isanewline
\ \ \isacommand{by}\isamarkupfalse%
\ {\isacharparenleft}{\kern0pt}insert\ assms\ {\isacharsemicolon}{\kern0pt}\ {\isacharparenleft}{\kern0pt}rule\ sep{\isacharunderscore}{\kern0pt}rules\ {\isacharbar}{\kern0pt}\ simp{\isacharparenright}{\kern0pt}{\isacharplus}{\kern0pt}{\isacharparenright}{\kern0pt}%
\endisatagproof
{\isafoldproof}%
%
\isadelimproof
\isanewline
%
\endisadelimproof
\isanewline
\isanewline
\isacommand{lemma}\isamarkupfalse%
\ {\isacharparenleft}{\kern0pt}\isakeyword{in}\ M{\isacharunderscore}{\kern0pt}ZF{\isacharunderscore}{\kern0pt}trans{\isacharparenright}{\kern0pt}\ rtrancl{\isacharunderscore}{\kern0pt}separation{\isacharunderscore}{\kern0pt}intf{\isacharcolon}{\kern0pt}\isanewline
\ \ \isakeyword{assumes}\isanewline
\ \ \ \ {\isachardoublequoteopen}r{\isasymin}M{\isachardoublequoteclose}\isanewline
\ \ \ \ \isakeyword{and}\isanewline
\ \ \ \ {\isachardoublequoteopen}A{\isasymin}M{\isachardoublequoteclose}\isanewline
\ \ \isakeyword{shows}\isanewline
\ \ \ \ {\isachardoublequoteopen}separation\ {\isacharparenleft}{\kern0pt}{\isacharhash}{\kern0pt}{\isacharhash}{\kern0pt}M{\isacharcomma}{\kern0pt}\ rtran{\isacharunderscore}{\kern0pt}closure{\isacharunderscore}{\kern0pt}mem{\isacharparenleft}{\kern0pt}{\isacharhash}{\kern0pt}{\isacharhash}{\kern0pt}M{\isacharcomma}{\kern0pt}A{\isacharcomma}{\kern0pt}r{\isacharparenright}{\kern0pt}{\isacharparenright}{\kern0pt}{\isachardoublequoteclose}\isanewline
%
\isadelimproof
%
\endisadelimproof
%
\isatagproof
\isacommand{proof}\isamarkupfalse%
\ {\isacharminus}{\kern0pt}\isanewline
\ \ \isacommand{obtain}\isamarkupfalse%
\ rcfm\ \isakeyword{where}\isanewline
\ \ \ \ fmsats{\isacharcolon}{\kern0pt}{\isachardoublequoteopen}{\isasymAnd}env{\isachardot}{\kern0pt}\ env{\isasymin}list{\isacharparenleft}{\kern0pt}M{\isacharparenright}{\kern0pt}\ {\isasymLongrightarrow}\isanewline
\ \ \ \ {\isacharparenleft}{\kern0pt}rtran{\isacharunderscore}{\kern0pt}closure{\isacharunderscore}{\kern0pt}mem{\isacharparenleft}{\kern0pt}{\isacharhash}{\kern0pt}{\isacharhash}{\kern0pt}M{\isacharcomma}{\kern0pt}nth{\isacharparenleft}{\kern0pt}{\isadigit{2}}{\isacharcomma}{\kern0pt}env{\isacharparenright}{\kern0pt}{\isacharcomma}{\kern0pt}nth{\isacharparenleft}{\kern0pt}{\isadigit{1}}{\isacharcomma}{\kern0pt}env{\isacharparenright}{\kern0pt}{\isacharcomma}{\kern0pt}nth{\isacharparenleft}{\kern0pt}{\isadigit{0}}{\isacharcomma}{\kern0pt}env{\isacharparenright}{\kern0pt}{\isacharparenright}{\kern0pt}{\isacharparenright}{\kern0pt}\ {\isasymlongleftrightarrow}\ sats{\isacharparenleft}{\kern0pt}M{\isacharcomma}{\kern0pt}rcfm{\isacharparenleft}{\kern0pt}{\isadigit{0}}{\isacharcomma}{\kern0pt}{\isadigit{1}}{\isacharcomma}{\kern0pt}{\isadigit{2}}{\isacharparenright}{\kern0pt}{\isacharcomma}{\kern0pt}env{\isacharparenright}{\kern0pt}{\isachardoublequoteclose}\isanewline
\ \ \ \ \isakeyword{and}\isanewline
\ \ \ \ {\isachardoublequoteopen}rcfm{\isacharparenleft}{\kern0pt}{\isadigit{0}}{\isacharcomma}{\kern0pt}{\isadigit{1}}{\isacharcomma}{\kern0pt}{\isadigit{2}}{\isacharparenright}{\kern0pt}\ {\isasymin}\ formula{\isachardoublequoteclose}\isanewline
\ \ \ \ \isakeyword{and}\isanewline
\ \ \ \ {\isachardoublequoteopen}arity{\isacharparenleft}{\kern0pt}rcfm{\isacharparenleft}{\kern0pt}{\isadigit{0}}{\isacharcomma}{\kern0pt}{\isadigit{1}}{\isacharcomma}{\kern0pt}{\isadigit{2}}{\isacharparenright}{\kern0pt}{\isacharparenright}{\kern0pt}\ {\isacharequal}{\kern0pt}\ {\isadigit{3}}{\isachardoublequoteclose}\isanewline
\ \ \ \ \isacommand{using}\isamarkupfalse%
\ rtran{\isacharunderscore}{\kern0pt}closure{\isacharunderscore}{\kern0pt}mem{\isacharunderscore}{\kern0pt}auto\ \isacommand{by}\isamarkupfalse%
\ {\isacharparenleft}{\kern0pt}simp\ del{\isacharcolon}{\kern0pt}FOL{\isacharunderscore}{\kern0pt}sats{\isacharunderscore}{\kern0pt}iff\ pair{\isacharunderscore}{\kern0pt}abs\ add{\isacharcolon}{\kern0pt}\ fm{\isacharunderscore}{\kern0pt}defs\ nat{\isacharunderscore}{\kern0pt}simp{\isacharunderscore}{\kern0pt}union{\isacharparenright}{\kern0pt}\isanewline
\ \ \isacommand{then}\isamarkupfalse%
\isanewline
\ \ \isacommand{have}\isamarkupfalse%
\ {\isachardoublequoteopen}{\isasymforall}x{\isasymin}M{\isachardot}{\kern0pt}\ {\isasymforall}a{\isasymin}M{\isachardot}{\kern0pt}\ separation{\isacharparenleft}{\kern0pt}{\isacharhash}{\kern0pt}{\isacharhash}{\kern0pt}M{\isacharcomma}{\kern0pt}\ {\isasymlambda}y{\isachardot}{\kern0pt}\ sats{\isacharparenleft}{\kern0pt}M{\isacharcomma}{\kern0pt}rcfm{\isacharparenleft}{\kern0pt}{\isadigit{0}}{\isacharcomma}{\kern0pt}{\isadigit{1}}{\isacharcomma}{\kern0pt}{\isadigit{2}}{\isacharparenright}{\kern0pt}\ {\isacharcomma}{\kern0pt}\ {\isacharbrackleft}{\kern0pt}y{\isacharcomma}{\kern0pt}x{\isacharcomma}{\kern0pt}a{\isacharbrackright}{\kern0pt}{\isacharparenright}{\kern0pt}{\isacharparenright}{\kern0pt}{\isachardoublequoteclose}\isanewline
\ \ \ \ \isacommand{using}\isamarkupfalse%
\ separation{\isacharunderscore}{\kern0pt}ax\ \isacommand{by}\isamarkupfalse%
\ simp\isanewline
\ \ \isacommand{moreover}\isamarkupfalse%
\isanewline
\ \ \isacommand{have}\isamarkupfalse%
\ {\isachardoublequoteopen}{\isacharparenleft}{\kern0pt}rtran{\isacharunderscore}{\kern0pt}closure{\isacharunderscore}{\kern0pt}mem{\isacharparenleft}{\kern0pt}{\isacharhash}{\kern0pt}{\isacharhash}{\kern0pt}M{\isacharcomma}{\kern0pt}a{\isacharcomma}{\kern0pt}x{\isacharcomma}{\kern0pt}y{\isacharparenright}{\kern0pt}{\isacharparenright}{\kern0pt}\isanewline
\ \ \ \ \ \ \ \ \ \ {\isasymlongleftrightarrow}\ sats{\isacharparenleft}{\kern0pt}M{\isacharcomma}{\kern0pt}rcfm{\isacharparenleft}{\kern0pt}{\isadigit{0}}{\isacharcomma}{\kern0pt}{\isadigit{1}}{\isacharcomma}{\kern0pt}{\isadigit{2}}{\isacharparenright}{\kern0pt}\ {\isacharcomma}{\kern0pt}\ {\isacharbrackleft}{\kern0pt}y{\isacharcomma}{\kern0pt}x{\isacharcomma}{\kern0pt}a{\isacharbrackright}{\kern0pt}{\isacharparenright}{\kern0pt}{\isachardoublequoteclose}\isanewline
\ \ \ \ \isakeyword{if}\ {\isachardoublequoteopen}y{\isasymin}M{\isachardoublequoteclose}\ {\isachardoublequoteopen}x{\isasymin}M{\isachardoublequoteclose}\ {\isachardoublequoteopen}a{\isasymin}M{\isachardoublequoteclose}\ \isakeyword{for}\ y\ x\ a\isanewline
\ \ \ \ \isacommand{using}\isamarkupfalse%
\ that\ fmsats{\isacharbrackleft}{\kern0pt}of\ {\isachardoublequoteopen}{\isacharbrackleft}{\kern0pt}y{\isacharcomma}{\kern0pt}x{\isacharcomma}{\kern0pt}a{\isacharbrackright}{\kern0pt}{\isachardoublequoteclose}{\isacharbrackright}{\kern0pt}\ \isacommand{by}\isamarkupfalse%
\ simp\isanewline
\ \ \isacommand{ultimately}\isamarkupfalse%
\isanewline
\ \ \isacommand{have}\isamarkupfalse%
\ {\isachardoublequoteopen}{\isasymforall}x{\isasymin}M{\isachardot}{\kern0pt}\ {\isasymforall}a{\isasymin}M{\isachardot}{\kern0pt}\ separation{\isacharparenleft}{\kern0pt}{\isacharhash}{\kern0pt}{\isacharhash}{\kern0pt}M{\isacharcomma}{\kern0pt}\ rtran{\isacharunderscore}{\kern0pt}closure{\isacharunderscore}{\kern0pt}mem{\isacharparenleft}{\kern0pt}{\isacharhash}{\kern0pt}{\isacharhash}{\kern0pt}M{\isacharcomma}{\kern0pt}a{\isacharcomma}{\kern0pt}x{\isacharparenright}{\kern0pt}{\isacharparenright}{\kern0pt}{\isachardoublequoteclose}\isanewline
\ \ \ \ \isacommand{unfolding}\isamarkupfalse%
\ separation{\isacharunderscore}{\kern0pt}def\ \isacommand{by}\isamarkupfalse%
\ simp\isanewline
\ \ \isacommand{with}\isamarkupfalse%
\ {\isacartoucheopen}r{\isasymin}M{\isacartoucheclose}\ {\isacartoucheopen}A{\isasymin}M{\isacartoucheclose}\ \isacommand{show}\isamarkupfalse%
\ {\isacharquery}{\kern0pt}thesis\ \isacommand{by}\isamarkupfalse%
\ simp\isanewline
\isacommand{qed}\isamarkupfalse%
%
\endisatagproof
{\isafoldproof}%
%
\isadelimproof
\isanewline
%
\endisadelimproof
\isanewline
\isacommand{schematic{\isacharunderscore}{\kern0pt}goal}\isamarkupfalse%
\ rtran{\isacharunderscore}{\kern0pt}closure{\isacharunderscore}{\kern0pt}fm{\isacharunderscore}{\kern0pt}auto{\isacharcolon}{\kern0pt}\isanewline
\ \ \isakeyword{assumes}\isanewline
\ \ \ \ {\isachardoublequoteopen}nth{\isacharparenleft}{\kern0pt}i{\isacharcomma}{\kern0pt}env{\isacharparenright}{\kern0pt}\ {\isacharequal}{\kern0pt}\ r{\isachardoublequoteclose}\ {\isachardoublequoteopen}nth{\isacharparenleft}{\kern0pt}j{\isacharcomma}{\kern0pt}env{\isacharparenright}{\kern0pt}\ {\isacharequal}{\kern0pt}\ rp{\isachardoublequoteclose}\isanewline
\ \ \ \ {\isachardoublequoteopen}i\ {\isasymin}\ nat{\isachardoublequoteclose}\ {\isachardoublequoteopen}j\ {\isasymin}\ nat{\isachardoublequoteclose}\ {\isachardoublequoteopen}env\ {\isasymin}\ list{\isacharparenleft}{\kern0pt}A{\isacharparenright}{\kern0pt}{\isachardoublequoteclose}\isanewline
\ \ \isakeyword{shows}\isanewline
\ \ \ \ {\isachardoublequoteopen}rtran{\isacharunderscore}{\kern0pt}closure{\isacharparenleft}{\kern0pt}{\isacharhash}{\kern0pt}{\isacharhash}{\kern0pt}A{\isacharcomma}{\kern0pt}r{\isacharcomma}{\kern0pt}rp{\isacharparenright}{\kern0pt}\ {\isasymlongleftrightarrow}\ sats{\isacharparenleft}{\kern0pt}A{\isacharcomma}{\kern0pt}{\isacharquery}{\kern0pt}rtc{\isacharparenleft}{\kern0pt}i{\isacharcomma}{\kern0pt}j{\isacharparenright}{\kern0pt}{\isacharcomma}{\kern0pt}env{\isacharparenright}{\kern0pt}{\isachardoublequoteclose}\isanewline
%
\isadelimproof
\ \ %
\endisadelimproof
%
\isatagproof
\isacommand{unfolding}\isamarkupfalse%
\ rtran{\isacharunderscore}{\kern0pt}closure{\isacharunderscore}{\kern0pt}def\isanewline
\ \ \isacommand{by}\isamarkupfalse%
\ {\isacharparenleft}{\kern0pt}insert\ assms\ {\isacharsemicolon}{\kern0pt}\ {\isacharparenleft}{\kern0pt}rule\ sep{\isacharunderscore}{\kern0pt}rules\ rtran{\isacharunderscore}{\kern0pt}closure{\isacharunderscore}{\kern0pt}mem{\isacharunderscore}{\kern0pt}auto\ {\isacharbar}{\kern0pt}\ simp{\isacharparenright}{\kern0pt}{\isacharplus}{\kern0pt}{\isacharparenright}{\kern0pt}%
\endisatagproof
{\isafoldproof}%
%
\isadelimproof
\isanewline
%
\endisadelimproof
\isanewline
\isacommand{schematic{\isacharunderscore}{\kern0pt}goal}\isamarkupfalse%
\ trans{\isacharunderscore}{\kern0pt}closure{\isacharunderscore}{\kern0pt}fm{\isacharunderscore}{\kern0pt}auto{\isacharcolon}{\kern0pt}\isanewline
\ \ \isakeyword{assumes}\isanewline
\ \ \ \ {\isachardoublequoteopen}nth{\isacharparenleft}{\kern0pt}i{\isacharcomma}{\kern0pt}env{\isacharparenright}{\kern0pt}\ {\isacharequal}{\kern0pt}\ r{\isachardoublequoteclose}\ {\isachardoublequoteopen}nth{\isacharparenleft}{\kern0pt}j{\isacharcomma}{\kern0pt}env{\isacharparenright}{\kern0pt}\ {\isacharequal}{\kern0pt}\ rp{\isachardoublequoteclose}\isanewline
\ \ \ \ {\isachardoublequoteopen}i\ {\isasymin}\ nat{\isachardoublequoteclose}\ {\isachardoublequoteopen}j\ {\isasymin}\ nat{\isachardoublequoteclose}\ {\isachardoublequoteopen}env\ {\isasymin}\ list{\isacharparenleft}{\kern0pt}A{\isacharparenright}{\kern0pt}{\isachardoublequoteclose}\isanewline
\ \ \isakeyword{shows}\isanewline
\ \ \ \ {\isachardoublequoteopen}tran{\isacharunderscore}{\kern0pt}closure{\isacharparenleft}{\kern0pt}{\isacharhash}{\kern0pt}{\isacharhash}{\kern0pt}A{\isacharcomma}{\kern0pt}r{\isacharcomma}{\kern0pt}rp{\isacharparenright}{\kern0pt}\ {\isasymlongleftrightarrow}\ sats{\isacharparenleft}{\kern0pt}A{\isacharcomma}{\kern0pt}{\isacharquery}{\kern0pt}tc{\isacharparenleft}{\kern0pt}i{\isacharcomma}{\kern0pt}j{\isacharparenright}{\kern0pt}{\isacharcomma}{\kern0pt}env{\isacharparenright}{\kern0pt}{\isachardoublequoteclose}\isanewline
%
\isadelimproof
\ \ %
\endisadelimproof
%
\isatagproof
\isacommand{unfolding}\isamarkupfalse%
\ tran{\isacharunderscore}{\kern0pt}closure{\isacharunderscore}{\kern0pt}def\isanewline
\ \ \isacommand{by}\isamarkupfalse%
\ {\isacharparenleft}{\kern0pt}insert\ assms\ {\isacharsemicolon}{\kern0pt}\ {\isacharparenleft}{\kern0pt}rule\ sep{\isacharunderscore}{\kern0pt}rules\ rtran{\isacharunderscore}{\kern0pt}closure{\isacharunderscore}{\kern0pt}fm{\isacharunderscore}{\kern0pt}auto\ {\isacharbar}{\kern0pt}\ simp{\isacharparenright}{\kern0pt}{\isacharparenright}{\kern0pt}{\isacharplus}{\kern0pt}%
\endisatagproof
{\isafoldproof}%
%
\isadelimproof
\isanewline
%
\endisadelimproof
%
\isadelimML
\isanewline
%
\endisadelimML
%
\isatagML
\isacommand{synthesize}\isamarkupfalse%
\ {\isachardoublequoteopen}trans{\isacharunderscore}{\kern0pt}closure{\isacharunderscore}{\kern0pt}fm{\isachardoublequoteclose}\ \isakeyword{from{\isacharunderscore}{\kern0pt}schematic}\ trans{\isacharunderscore}{\kern0pt}closure{\isacharunderscore}{\kern0pt}fm{\isacharunderscore}{\kern0pt}auto%
\endisatagML
{\isafoldML}%
%
\isadelimML
\isanewline
%
\endisadelimML
\isanewline
\isacommand{schematic{\isacharunderscore}{\kern0pt}goal}\isamarkupfalse%
\ wellfounded{\isacharunderscore}{\kern0pt}trancl{\isacharunderscore}{\kern0pt}fm{\isacharunderscore}{\kern0pt}auto{\isacharcolon}{\kern0pt}\isanewline
\ \ \isakeyword{assumes}\isanewline
\ \ \ \ {\isachardoublequoteopen}nth{\isacharparenleft}{\kern0pt}i{\isacharcomma}{\kern0pt}env{\isacharparenright}{\kern0pt}\ {\isacharequal}{\kern0pt}\ p{\isachardoublequoteclose}\ {\isachardoublequoteopen}nth{\isacharparenleft}{\kern0pt}j{\isacharcomma}{\kern0pt}env{\isacharparenright}{\kern0pt}\ {\isacharequal}{\kern0pt}\ r{\isachardoublequoteclose}\ \ {\isachardoublequoteopen}nth{\isacharparenleft}{\kern0pt}k{\isacharcomma}{\kern0pt}env{\isacharparenright}{\kern0pt}\ {\isacharequal}{\kern0pt}\ B{\isachardoublequoteclose}\isanewline
\ \ \ \ {\isachardoublequoteopen}i\ {\isasymin}\ nat{\isachardoublequoteclose}\ {\isachardoublequoteopen}j\ {\isasymin}\ nat{\isachardoublequoteclose}\ {\isachardoublequoteopen}k\ {\isasymin}\ nat{\isachardoublequoteclose}\ {\isachardoublequoteopen}env\ {\isasymin}\ list{\isacharparenleft}{\kern0pt}A{\isacharparenright}{\kern0pt}{\isachardoublequoteclose}\isanewline
\ \ \isakeyword{shows}\isanewline
\ \ \ \ {\isachardoublequoteopen}wellfounded{\isacharunderscore}{\kern0pt}trancl{\isacharparenleft}{\kern0pt}{\isacharhash}{\kern0pt}{\isacharhash}{\kern0pt}A{\isacharcomma}{\kern0pt}B{\isacharcomma}{\kern0pt}r{\isacharcomma}{\kern0pt}p{\isacharparenright}{\kern0pt}\ {\isasymlongleftrightarrow}\ sats{\isacharparenleft}{\kern0pt}A{\isacharcomma}{\kern0pt}{\isacharquery}{\kern0pt}wtf{\isacharparenleft}{\kern0pt}i{\isacharcomma}{\kern0pt}j{\isacharcomma}{\kern0pt}k{\isacharparenright}{\kern0pt}{\isacharcomma}{\kern0pt}env{\isacharparenright}{\kern0pt}{\isachardoublequoteclose}\isanewline
%
\isadelimproof
\ \ %
\endisadelimproof
%
\isatagproof
\isacommand{unfolding}\isamarkupfalse%
\ \ wellfounded{\isacharunderscore}{\kern0pt}trancl{\isacharunderscore}{\kern0pt}def\isanewline
\ \ \isacommand{by}\isamarkupfalse%
\ {\isacharparenleft}{\kern0pt}insert\ assms\ {\isacharsemicolon}{\kern0pt}\ {\isacharparenleft}{\kern0pt}rule\ sep{\isacharunderscore}{\kern0pt}rules\ trans{\isacharunderscore}{\kern0pt}closure{\isacharunderscore}{\kern0pt}fm{\isacharunderscore}{\kern0pt}iff{\isacharunderscore}{\kern0pt}sats\ {\isacharbar}{\kern0pt}\ simp{\isacharparenright}{\kern0pt}{\isacharplus}{\kern0pt}{\isacharparenright}{\kern0pt}%
\endisatagproof
{\isafoldproof}%
%
\isadelimproof
\isanewline
%
\endisadelimproof
\isanewline
\isacommand{lemma}\isamarkupfalse%
\ {\isacharparenleft}{\kern0pt}\isakeyword{in}\ M{\isacharunderscore}{\kern0pt}ZF{\isacharunderscore}{\kern0pt}trans{\isacharparenright}{\kern0pt}\ wftrancl{\isacharunderscore}{\kern0pt}separation{\isacharunderscore}{\kern0pt}intf{\isacharcolon}{\kern0pt}\isanewline
\ \ \isakeyword{assumes}\isanewline
\ \ \ \ {\isachardoublequoteopen}r{\isasymin}M{\isachardoublequoteclose}\isanewline
\ \ \ \ \isakeyword{and}\isanewline
\ \ \ \ {\isachardoublequoteopen}Z{\isasymin}M{\isachardoublequoteclose}\isanewline
\ \ \isakeyword{shows}\isanewline
\ \ \ \ {\isachardoublequoteopen}separation\ {\isacharparenleft}{\kern0pt}{\isacharhash}{\kern0pt}{\isacharhash}{\kern0pt}M{\isacharcomma}{\kern0pt}\ wellfounded{\isacharunderscore}{\kern0pt}trancl{\isacharparenleft}{\kern0pt}{\isacharhash}{\kern0pt}{\isacharhash}{\kern0pt}M{\isacharcomma}{\kern0pt}Z{\isacharcomma}{\kern0pt}r{\isacharparenright}{\kern0pt}{\isacharparenright}{\kern0pt}{\isachardoublequoteclose}\isanewline
%
\isadelimproof
%
\endisadelimproof
%
\isatagproof
\isacommand{proof}\isamarkupfalse%
\ {\isacharminus}{\kern0pt}\isanewline
\ \ \isacommand{obtain}\isamarkupfalse%
\ rcfm\ \isakeyword{where}\isanewline
\ \ \ \ fmsats{\isacharcolon}{\kern0pt}{\isachardoublequoteopen}{\isasymAnd}env{\isachardot}{\kern0pt}\ env{\isasymin}list{\isacharparenleft}{\kern0pt}M{\isacharparenright}{\kern0pt}\ {\isasymLongrightarrow}\isanewline
\ \ \ \ {\isacharparenleft}{\kern0pt}wellfounded{\isacharunderscore}{\kern0pt}trancl{\isacharparenleft}{\kern0pt}{\isacharhash}{\kern0pt}{\isacharhash}{\kern0pt}M{\isacharcomma}{\kern0pt}nth{\isacharparenleft}{\kern0pt}{\isadigit{2}}{\isacharcomma}{\kern0pt}env{\isacharparenright}{\kern0pt}{\isacharcomma}{\kern0pt}nth{\isacharparenleft}{\kern0pt}{\isadigit{1}}{\isacharcomma}{\kern0pt}env{\isacharparenright}{\kern0pt}{\isacharcomma}{\kern0pt}nth{\isacharparenleft}{\kern0pt}{\isadigit{0}}{\isacharcomma}{\kern0pt}env{\isacharparenright}{\kern0pt}{\isacharparenright}{\kern0pt}{\isacharparenright}{\kern0pt}\ {\isasymlongleftrightarrow}\ sats{\isacharparenleft}{\kern0pt}M{\isacharcomma}{\kern0pt}rcfm{\isacharparenleft}{\kern0pt}{\isadigit{0}}{\isacharcomma}{\kern0pt}{\isadigit{1}}{\isacharcomma}{\kern0pt}{\isadigit{2}}{\isacharparenright}{\kern0pt}{\isacharcomma}{\kern0pt}env{\isacharparenright}{\kern0pt}{\isachardoublequoteclose}\isanewline
\ \ \ \ \isakeyword{and}\isanewline
\ \ \ \ {\isachardoublequoteopen}rcfm{\isacharparenleft}{\kern0pt}{\isadigit{0}}{\isacharcomma}{\kern0pt}{\isadigit{1}}{\isacharcomma}{\kern0pt}{\isadigit{2}}{\isacharparenright}{\kern0pt}\ {\isasymin}\ formula{\isachardoublequoteclose}\isanewline
\ \ \ \ \isakeyword{and}\isanewline
\ \ \ \ {\isachardoublequoteopen}arity{\isacharparenleft}{\kern0pt}rcfm{\isacharparenleft}{\kern0pt}{\isadigit{0}}{\isacharcomma}{\kern0pt}{\isadigit{1}}{\isacharcomma}{\kern0pt}{\isadigit{2}}{\isacharparenright}{\kern0pt}{\isacharparenright}{\kern0pt}\ {\isacharequal}{\kern0pt}\ {\isadigit{3}}{\isachardoublequoteclose}\isanewline
\ \ \ \ \isacommand{using}\isamarkupfalse%
\ wellfounded{\isacharunderscore}{\kern0pt}trancl{\isacharunderscore}{\kern0pt}fm{\isacharunderscore}{\kern0pt}auto{\isacharbrackleft}{\kern0pt}of\ concl{\isacharcolon}{\kern0pt}M\ {\isachardoublequoteopen}nth{\isacharparenleft}{\kern0pt}{\isadigit{2}}{\isacharcomma}{\kern0pt}{\isacharunderscore}{\kern0pt}{\isacharparenright}{\kern0pt}{\isachardoublequoteclose}{\isacharbrackright}{\kern0pt}\ \isacommand{unfolding}\isamarkupfalse%
\ fm{\isacharunderscore}{\kern0pt}defs\ trans{\isacharunderscore}{\kern0pt}closure{\isacharunderscore}{\kern0pt}fm{\isacharunderscore}{\kern0pt}def\isanewline
\ \ \ \ \isacommand{by}\isamarkupfalse%
\ {\isacharparenleft}{\kern0pt}simp\ del{\isacharcolon}{\kern0pt}FOL{\isacharunderscore}{\kern0pt}sats{\isacharunderscore}{\kern0pt}iff\ pair{\isacharunderscore}{\kern0pt}abs\ add{\isacharcolon}{\kern0pt}\ fm{\isacharunderscore}{\kern0pt}defs\ nat{\isacharunderscore}{\kern0pt}simp{\isacharunderscore}{\kern0pt}union{\isacharparenright}{\kern0pt}\isanewline
\ \ \isacommand{then}\isamarkupfalse%
\isanewline
\ \ \isacommand{have}\isamarkupfalse%
\ {\isachardoublequoteopen}{\isasymforall}x{\isasymin}M{\isachardot}{\kern0pt}\ {\isasymforall}z{\isasymin}M{\isachardot}{\kern0pt}\ separation{\isacharparenleft}{\kern0pt}{\isacharhash}{\kern0pt}{\isacharhash}{\kern0pt}M{\isacharcomma}{\kern0pt}\ {\isasymlambda}y{\isachardot}{\kern0pt}\ sats{\isacharparenleft}{\kern0pt}M{\isacharcomma}{\kern0pt}rcfm{\isacharparenleft}{\kern0pt}{\isadigit{0}}{\isacharcomma}{\kern0pt}{\isadigit{1}}{\isacharcomma}{\kern0pt}{\isadigit{2}}{\isacharparenright}{\kern0pt}\ {\isacharcomma}{\kern0pt}\ {\isacharbrackleft}{\kern0pt}y{\isacharcomma}{\kern0pt}x{\isacharcomma}{\kern0pt}z{\isacharbrackright}{\kern0pt}{\isacharparenright}{\kern0pt}{\isacharparenright}{\kern0pt}{\isachardoublequoteclose}\isanewline
\ \ \ \ \isacommand{using}\isamarkupfalse%
\ separation{\isacharunderscore}{\kern0pt}ax\ \isacommand{by}\isamarkupfalse%
\ simp\isanewline
\ \ \isacommand{moreover}\isamarkupfalse%
\isanewline
\ \ \isacommand{have}\isamarkupfalse%
\ {\isachardoublequoteopen}{\isacharparenleft}{\kern0pt}wellfounded{\isacharunderscore}{\kern0pt}trancl{\isacharparenleft}{\kern0pt}{\isacharhash}{\kern0pt}{\isacharhash}{\kern0pt}M{\isacharcomma}{\kern0pt}z{\isacharcomma}{\kern0pt}x{\isacharcomma}{\kern0pt}y{\isacharparenright}{\kern0pt}{\isacharparenright}{\kern0pt}\isanewline
\ \ \ \ \ \ \ \ \ \ {\isasymlongleftrightarrow}\ sats{\isacharparenleft}{\kern0pt}M{\isacharcomma}{\kern0pt}rcfm{\isacharparenleft}{\kern0pt}{\isadigit{0}}{\isacharcomma}{\kern0pt}{\isadigit{1}}{\isacharcomma}{\kern0pt}{\isadigit{2}}{\isacharparenright}{\kern0pt}\ {\isacharcomma}{\kern0pt}\ {\isacharbrackleft}{\kern0pt}y{\isacharcomma}{\kern0pt}x{\isacharcomma}{\kern0pt}z{\isacharbrackright}{\kern0pt}{\isacharparenright}{\kern0pt}{\isachardoublequoteclose}\isanewline
\ \ \ \ \isakeyword{if}\ {\isachardoublequoteopen}y{\isasymin}M{\isachardoublequoteclose}\ {\isachardoublequoteopen}x{\isasymin}M{\isachardoublequoteclose}\ {\isachardoublequoteopen}z{\isasymin}M{\isachardoublequoteclose}\ \isakeyword{for}\ y\ x\ z\isanewline
\ \ \ \ \isacommand{using}\isamarkupfalse%
\ that\ fmsats{\isacharbrackleft}{\kern0pt}of\ {\isachardoublequoteopen}{\isacharbrackleft}{\kern0pt}y{\isacharcomma}{\kern0pt}x{\isacharcomma}{\kern0pt}z{\isacharbrackright}{\kern0pt}{\isachardoublequoteclose}{\isacharbrackright}{\kern0pt}\ \isacommand{by}\isamarkupfalse%
\ simp\isanewline
\ \ \isacommand{ultimately}\isamarkupfalse%
\isanewline
\ \ \isacommand{have}\isamarkupfalse%
\ {\isachardoublequoteopen}{\isasymforall}x{\isasymin}M{\isachardot}{\kern0pt}\ {\isasymforall}z{\isasymin}M{\isachardot}{\kern0pt}\ separation{\isacharparenleft}{\kern0pt}{\isacharhash}{\kern0pt}{\isacharhash}{\kern0pt}M{\isacharcomma}{\kern0pt}\ wellfounded{\isacharunderscore}{\kern0pt}trancl{\isacharparenleft}{\kern0pt}{\isacharhash}{\kern0pt}{\isacharhash}{\kern0pt}M{\isacharcomma}{\kern0pt}z{\isacharcomma}{\kern0pt}x{\isacharparenright}{\kern0pt}{\isacharparenright}{\kern0pt}{\isachardoublequoteclose}\isanewline
\ \ \ \ \isacommand{unfolding}\isamarkupfalse%
\ separation{\isacharunderscore}{\kern0pt}def\ \isacommand{by}\isamarkupfalse%
\ simp\isanewline
\ \ \isacommand{with}\isamarkupfalse%
\ {\isacartoucheopen}r{\isasymin}M{\isacartoucheclose}\ {\isacartoucheopen}Z{\isasymin}M{\isacartoucheclose}\ \isacommand{show}\isamarkupfalse%
\ {\isacharquery}{\kern0pt}thesis\ \isacommand{by}\isamarkupfalse%
\ simp\isanewline
\isacommand{qed}\isamarkupfalse%
%
\endisatagproof
{\isafoldproof}%
%
\isadelimproof
\isanewline
%
\endisadelimproof
\isanewline
\isanewline
\isanewline
\isacommand{lemma}\isamarkupfalse%
\ {\isacharparenleft}{\kern0pt}\isakeyword{in}\ M{\isacharunderscore}{\kern0pt}ZF{\isacharunderscore}{\kern0pt}trans{\isacharparenright}{\kern0pt}\ finite{\isacharunderscore}{\kern0pt}sep{\isacharunderscore}{\kern0pt}intf{\isacharcolon}{\kern0pt}\isanewline
\ \ {\isachardoublequoteopen}separation{\isacharparenleft}{\kern0pt}{\isacharhash}{\kern0pt}{\isacharhash}{\kern0pt}M{\isacharcomma}{\kern0pt}\ {\isasymlambda}x{\isachardot}{\kern0pt}\ x{\isasymin}nat{\isacharparenright}{\kern0pt}{\isachardoublequoteclose}\isanewline
%
\isadelimproof
%
\endisadelimproof
%
\isatagproof
\isacommand{proof}\isamarkupfalse%
\ {\isacharminus}{\kern0pt}\isanewline
\ \ \isacommand{have}\isamarkupfalse%
\ {\isachardoublequoteopen}arity{\isacharparenleft}{\kern0pt}finite{\isacharunderscore}{\kern0pt}ordinal{\isacharunderscore}{\kern0pt}fm{\isacharparenleft}{\kern0pt}{\isadigit{0}}{\isacharparenright}{\kern0pt}{\isacharparenright}{\kern0pt}\ {\isacharequal}{\kern0pt}\ {\isadigit{1}}\ {\isachardoublequoteclose}\isanewline
\ \ \ \ \isacommand{unfolding}\isamarkupfalse%
\ finite{\isacharunderscore}{\kern0pt}ordinal{\isacharunderscore}{\kern0pt}fm{\isacharunderscore}{\kern0pt}def\ limit{\isacharunderscore}{\kern0pt}ordinal{\isacharunderscore}{\kern0pt}fm{\isacharunderscore}{\kern0pt}def\ empty{\isacharunderscore}{\kern0pt}fm{\isacharunderscore}{\kern0pt}def\ succ{\isacharunderscore}{\kern0pt}fm{\isacharunderscore}{\kern0pt}def\ cons{\isacharunderscore}{\kern0pt}fm{\isacharunderscore}{\kern0pt}def\isanewline
\ \ \ \ \ \ union{\isacharunderscore}{\kern0pt}fm{\isacharunderscore}{\kern0pt}def\ upair{\isacharunderscore}{\kern0pt}fm{\isacharunderscore}{\kern0pt}def\isanewline
\ \ \ \ \isacommand{by}\isamarkupfalse%
\ {\isacharparenleft}{\kern0pt}simp\ add{\isacharcolon}{\kern0pt}\ nat{\isacharunderscore}{\kern0pt}union{\isacharunderscore}{\kern0pt}abs{\isadigit{1}}\ Un{\isacharunderscore}{\kern0pt}commute{\isacharparenright}{\kern0pt}\isanewline
\ \ \isacommand{with}\isamarkupfalse%
\ separation{\isacharunderscore}{\kern0pt}ax\isanewline
\ \ \isacommand{have}\isamarkupfalse%
\ {\isachardoublequoteopen}{\isacharparenleft}{\kern0pt}{\isasymforall}v{\isasymin}M{\isachardot}{\kern0pt}\ separation{\isacharparenleft}{\kern0pt}{\isacharhash}{\kern0pt}{\isacharhash}{\kern0pt}M{\isacharcomma}{\kern0pt}{\isasymlambda}x{\isachardot}{\kern0pt}\ sats{\isacharparenleft}{\kern0pt}M{\isacharcomma}{\kern0pt}finite{\isacharunderscore}{\kern0pt}ordinal{\isacharunderscore}{\kern0pt}fm{\isacharparenleft}{\kern0pt}{\isadigit{0}}{\isacharparenright}{\kern0pt}{\isacharcomma}{\kern0pt}{\isacharbrackleft}{\kern0pt}x{\isacharcomma}{\kern0pt}v{\isacharbrackright}{\kern0pt}{\isacharparenright}{\kern0pt}{\isacharparenright}{\kern0pt}{\isacharparenright}{\kern0pt}{\isachardoublequoteclose}\isanewline
\ \ \ \ \isacommand{by}\isamarkupfalse%
\ simp\isanewline
\ \ \isacommand{then}\isamarkupfalse%
\ \isacommand{have}\isamarkupfalse%
\ {\isachardoublequoteopen}{\isacharparenleft}{\kern0pt}{\isasymforall}v{\isasymin}M{\isachardot}{\kern0pt}\ separation{\isacharparenleft}{\kern0pt}{\isacharhash}{\kern0pt}{\isacharhash}{\kern0pt}M{\isacharcomma}{\kern0pt}finite{\isacharunderscore}{\kern0pt}ordinal{\isacharparenleft}{\kern0pt}{\isacharhash}{\kern0pt}{\isacharhash}{\kern0pt}M{\isacharparenright}{\kern0pt}{\isacharparenright}{\kern0pt}{\isacharparenright}{\kern0pt}{\isachardoublequoteclose}\isanewline
\ \ \ \ \isacommand{unfolding}\isamarkupfalse%
\ separation{\isacharunderscore}{\kern0pt}def\ \isacommand{by}\isamarkupfalse%
\ simp\isanewline
\ \ \isacommand{then}\isamarkupfalse%
\ \isacommand{have}\isamarkupfalse%
\ {\isachardoublequoteopen}separation{\isacharparenleft}{\kern0pt}{\isacharhash}{\kern0pt}{\isacharhash}{\kern0pt}M{\isacharcomma}{\kern0pt}finite{\isacharunderscore}{\kern0pt}ordinal{\isacharparenleft}{\kern0pt}{\isacharhash}{\kern0pt}{\isacharhash}{\kern0pt}M{\isacharparenright}{\kern0pt}{\isacharparenright}{\kern0pt}{\isachardoublequoteclose}\isanewline
\ \ \ \ \isacommand{using}\isamarkupfalse%
\ zero{\isacharunderscore}{\kern0pt}in{\isacharunderscore}{\kern0pt}M\ \isacommand{by}\isamarkupfalse%
\ auto\isanewline
\ \ \isacommand{then}\isamarkupfalse%
\ \isacommand{show}\isamarkupfalse%
\ {\isacharquery}{\kern0pt}thesis\ \isacommand{unfolding}\isamarkupfalse%
\ separation{\isacharunderscore}{\kern0pt}def\ \isacommand{by}\isamarkupfalse%
\ simp\isanewline
\isacommand{qed}\isamarkupfalse%
%
\endisatagproof
{\isafoldproof}%
%
\isadelimproof
\isanewline
%
\endisadelimproof
\isanewline
\isanewline
\isacommand{lemma}\isamarkupfalse%
\ {\isacharparenleft}{\kern0pt}\isakeyword{in}\ M{\isacharunderscore}{\kern0pt}ZF{\isacharunderscore}{\kern0pt}trans{\isacharparenright}{\kern0pt}\ nat{\isacharunderscore}{\kern0pt}subset{\isacharunderscore}{\kern0pt}I{\isacharprime}{\kern0pt}\ {\isacharcolon}{\kern0pt}\isanewline
\ \ {\isachardoublequoteopen}{\isasymlbrakk}\ I{\isasymin}M\ {\isacharsemicolon}{\kern0pt}\ {\isadigit{0}}{\isasymin}I\ {\isacharsemicolon}{\kern0pt}\ {\isasymAnd}x{\isachardot}{\kern0pt}\ x{\isasymin}I\ {\isasymLongrightarrow}\ succ{\isacharparenleft}{\kern0pt}x{\isacharparenright}{\kern0pt}{\isasymin}I\ {\isasymrbrakk}\ {\isasymLongrightarrow}\ nat\ {\isasymsubseteq}\ I{\isachardoublequoteclose}\isanewline
%
\isadelimproof
\ \ %
\endisadelimproof
%
\isatagproof
\isacommand{by}\isamarkupfalse%
\ {\isacharparenleft}{\kern0pt}rule\ subsetI{\isacharcomma}{\kern0pt}induct{\isacharunderscore}{\kern0pt}tac\ x{\isacharcomma}{\kern0pt}simp{\isacharplus}{\kern0pt}{\isacharparenright}{\kern0pt}%
\endisatagproof
{\isafoldproof}%
%
\isadelimproof
\isanewline
%
\endisadelimproof
\isanewline
\isanewline
\isacommand{lemma}\isamarkupfalse%
\ {\isacharparenleft}{\kern0pt}\isakeyword{in}\ M{\isacharunderscore}{\kern0pt}ZF{\isacharunderscore}{\kern0pt}trans{\isacharparenright}{\kern0pt}\ nat{\isacharunderscore}{\kern0pt}subset{\isacharunderscore}{\kern0pt}I\ {\isacharcolon}{\kern0pt}\isanewline
\ \ {\isachardoublequoteopen}{\isasymexists}I{\isasymin}M{\isachardot}{\kern0pt}\ nat\ {\isasymsubseteq}\ I{\isachardoublequoteclose}\isanewline
%
\isadelimproof
%
\endisadelimproof
%
\isatagproof
\isacommand{proof}\isamarkupfalse%
\ {\isacharminus}{\kern0pt}\isanewline
\ \ \isacommand{have}\isamarkupfalse%
\ {\isachardoublequoteopen}{\isasymexists}I{\isasymin}M{\isachardot}{\kern0pt}\ {\isadigit{0}}{\isasymin}I\ {\isasymand}\ {\isacharparenleft}{\kern0pt}{\isasymforall}x{\isasymin}M{\isachardot}{\kern0pt}\ x{\isasymin}I\ {\isasymlongrightarrow}\ succ{\isacharparenleft}{\kern0pt}x{\isacharparenright}{\kern0pt}{\isasymin}I{\isacharparenright}{\kern0pt}{\isachardoublequoteclose}\isanewline
\ \ \ \ \isacommand{using}\isamarkupfalse%
\ infinity{\isacharunderscore}{\kern0pt}ax\ \isacommand{unfolding}\isamarkupfalse%
\ infinity{\isacharunderscore}{\kern0pt}ax{\isacharunderscore}{\kern0pt}def\ \isacommand{by}\isamarkupfalse%
\ auto\isanewline
\ \ \isacommand{then}\isamarkupfalse%
\ \isacommand{obtain}\isamarkupfalse%
\ I\ \isakeyword{where}\isanewline
\ \ \ \ {\isachardoublequoteopen}I{\isasymin}M{\isachardoublequoteclose}\ {\isachardoublequoteopen}{\isadigit{0}}{\isasymin}I{\isachardoublequoteclose}\ {\isachardoublequoteopen}{\isacharparenleft}{\kern0pt}{\isasymforall}x{\isasymin}M{\isachardot}{\kern0pt}\ x{\isasymin}I\ {\isasymlongrightarrow}\ succ{\isacharparenleft}{\kern0pt}x{\isacharparenright}{\kern0pt}{\isasymin}I{\isacharparenright}{\kern0pt}{\isachardoublequoteclose}\isanewline
\ \ \ \ \isacommand{by}\isamarkupfalse%
\ auto\isanewline
\ \ \isacommand{then}\isamarkupfalse%
\ \isacommand{have}\isamarkupfalse%
\ {\isachardoublequoteopen}{\isasymAnd}x{\isachardot}{\kern0pt}\ x{\isasymin}I\ {\isasymLongrightarrow}\ succ{\isacharparenleft}{\kern0pt}x{\isacharparenright}{\kern0pt}{\isasymin}I{\isachardoublequoteclose}\isanewline
\ \ \ \ \isacommand{using}\isamarkupfalse%
\ Transset{\isacharunderscore}{\kern0pt}intf{\isacharbrackleft}{\kern0pt}OF\ trans{\isacharunderscore}{\kern0pt}M{\isacharbrackright}{\kern0pt}\ \ \isacommand{by}\isamarkupfalse%
\ simp\isanewline
\ \ \isacommand{then}\isamarkupfalse%
\ \isacommand{have}\isamarkupfalse%
\ {\isachardoublequoteopen}nat{\isasymsubseteq}I{\isachardoublequoteclose}\isanewline
\ \ \ \ \isacommand{using}\isamarkupfalse%
\ \ {\isacartoucheopen}I{\isasymin}M{\isacartoucheclose}\ {\isacartoucheopen}{\isadigit{0}}{\isasymin}I{\isacartoucheclose}\ nat{\isacharunderscore}{\kern0pt}subset{\isacharunderscore}{\kern0pt}I{\isacharprime}{\kern0pt}\ \isacommand{by}\isamarkupfalse%
\ simp\isanewline
\ \ \isacommand{then}\isamarkupfalse%
\ \isacommand{show}\isamarkupfalse%
\ {\isacharquery}{\kern0pt}thesis\ \isacommand{using}\isamarkupfalse%
\ {\isacartoucheopen}I{\isasymin}M{\isacartoucheclose}\ \isacommand{by}\isamarkupfalse%
\ auto\isanewline
\isacommand{qed}\isamarkupfalse%
%
\endisatagproof
{\isafoldproof}%
%
\isadelimproof
\isanewline
%
\endisadelimproof
\isanewline
\isacommand{lemma}\isamarkupfalse%
\ {\isacharparenleft}{\kern0pt}\isakeyword{in}\ M{\isacharunderscore}{\kern0pt}ZF{\isacharunderscore}{\kern0pt}trans{\isacharparenright}{\kern0pt}\ nat{\isacharunderscore}{\kern0pt}in{\isacharunderscore}{\kern0pt}M\ {\isacharcolon}{\kern0pt}\isanewline
\ \ {\isachardoublequoteopen}nat\ {\isasymin}\ M{\isachardoublequoteclose}\isanewline
%
\isadelimproof
%
\endisadelimproof
%
\isatagproof
\isacommand{proof}\isamarkupfalse%
\ {\isacharminus}{\kern0pt}\isanewline
\ \ \isacommand{have}\isamarkupfalse%
\ {\isadigit{1}}{\isacharcolon}{\kern0pt}{\isachardoublequoteopen}{\isacharbraceleft}{\kern0pt}x{\isasymin}B\ {\isachardot}{\kern0pt}\ x{\isasymin}A{\isacharbraceright}{\kern0pt}{\isacharequal}{\kern0pt}A{\isachardoublequoteclose}\ \isakeyword{if}\ {\isachardoublequoteopen}A{\isasymsubseteq}B{\isachardoublequoteclose}\ \isakeyword{for}\ A\ B\isanewline
\ \ \ \ \isacommand{using}\isamarkupfalse%
\ that\ \isacommand{by}\isamarkupfalse%
\ auto\isanewline
\ \ \isacommand{obtain}\isamarkupfalse%
\ I\ \isakeyword{where}\isanewline
\ \ \ \ {\isachardoublequoteopen}I{\isasymin}M{\isachardoublequoteclose}\ {\isachardoublequoteopen}nat{\isasymsubseteq}I{\isachardoublequoteclose}\isanewline
\ \ \ \ \isacommand{using}\isamarkupfalse%
\ nat{\isacharunderscore}{\kern0pt}subset{\isacharunderscore}{\kern0pt}I\ \isacommand{by}\isamarkupfalse%
\ auto\isanewline
\ \ \isacommand{then}\isamarkupfalse%
\ \isacommand{have}\isamarkupfalse%
\ {\isachardoublequoteopen}{\isacharbraceleft}{\kern0pt}x{\isasymin}I\ {\isachardot}{\kern0pt}\ x{\isasymin}nat{\isacharbraceright}{\kern0pt}\ {\isasymin}\ M{\isachardoublequoteclose}\isanewline
\ \ \ \ \isacommand{using}\isamarkupfalse%
\ finite{\isacharunderscore}{\kern0pt}sep{\isacharunderscore}{\kern0pt}intf\ separation{\isacharunderscore}{\kern0pt}closed{\isacharbrackleft}{\kern0pt}of\ {\isachardoublequoteopen}{\isasymlambda}x\ {\isachardot}{\kern0pt}\ x{\isasymin}nat{\isachardoublequoteclose}{\isacharbrackright}{\kern0pt}\ \isacommand{by}\isamarkupfalse%
\ simp\isanewline
\ \ \isacommand{then}\isamarkupfalse%
\ \isacommand{show}\isamarkupfalse%
\ {\isacharquery}{\kern0pt}thesis\isanewline
\ \ \ \ \isacommand{using}\isamarkupfalse%
\ {\isacartoucheopen}nat{\isasymsubseteq}I{\isacartoucheclose}\ {\isadigit{1}}\ \isacommand{by}\isamarkupfalse%
\ simp\isanewline
\isacommand{qed}\isamarkupfalse%
%
\endisatagproof
{\isafoldproof}%
%
\isadelimproof
\isanewline
%
\endisadelimproof
\ \ \isanewline
\isanewline
\isanewline
\isacommand{lemma}\isamarkupfalse%
\ {\isacharparenleft}{\kern0pt}\isakeyword{in}\ M{\isacharunderscore}{\kern0pt}ZF{\isacharunderscore}{\kern0pt}trans{\isacharparenright}{\kern0pt}\ mtrancl\ {\isacharcolon}{\kern0pt}\ {\isachardoublequoteopen}M{\isacharunderscore}{\kern0pt}trancl{\isacharparenleft}{\kern0pt}{\isacharhash}{\kern0pt}{\isacharhash}{\kern0pt}M{\isacharparenright}{\kern0pt}{\isachardoublequoteclose}\isanewline
%
\isadelimproof
\ \ %
\endisadelimproof
%
\isatagproof
\isacommand{using}\isamarkupfalse%
\ \ mbasic\ rtrancl{\isacharunderscore}{\kern0pt}separation{\isacharunderscore}{\kern0pt}intf\ wftrancl{\isacharunderscore}{\kern0pt}separation{\isacharunderscore}{\kern0pt}intf\ nat{\isacharunderscore}{\kern0pt}in{\isacharunderscore}{\kern0pt}M\isanewline
\ \ \ \ wellfounded{\isacharunderscore}{\kern0pt}trancl{\isacharunderscore}{\kern0pt}def\isanewline
\ \ \isacommand{by}\isamarkupfalse%
\ unfold{\isacharunderscore}{\kern0pt}locales\ auto%
\endisatagproof
{\isafoldproof}%
%
\isadelimproof
\isanewline
%
\endisadelimproof
\isanewline
\isacommand{sublocale}\isamarkupfalse%
\ M{\isacharunderscore}{\kern0pt}ZF{\isacharunderscore}{\kern0pt}trans\ {\isasymsubseteq}\ M{\isacharunderscore}{\kern0pt}trancl\ {\isachardoublequoteopen}{\isacharhash}{\kern0pt}{\isacharhash}{\kern0pt}M{\isachardoublequoteclose}\isanewline
%
\isadelimproof
\ \ %
\endisadelimproof
%
\isatagproof
\isacommand{by}\isamarkupfalse%
\ {\isacharparenleft}{\kern0pt}rule\ mtrancl{\isacharparenright}{\kern0pt}%
\endisatagproof
{\isafoldproof}%
%
\isadelimproof
%
\endisadelimproof
%
\isadelimdocument
%
\endisadelimdocument
%
\isatagdocument
%
\isamarkupsubsection{Interface with \isa{M{\isacharunderscore}{\kern0pt}eclose}%
}
\isamarkuptrue%
%
\endisatagdocument
{\isafolddocument}%
%
\isadelimdocument
%
\endisadelimdocument
\isacommand{lemma}\isamarkupfalse%
\ repl{\isacharunderscore}{\kern0pt}sats{\isacharcolon}{\kern0pt}\isanewline
\ \ \isakeyword{assumes}\isanewline
\ \ \ \ sat{\isacharcolon}{\kern0pt}{\isachardoublequoteopen}{\isasymAnd}x\ z{\isachardot}{\kern0pt}\ x{\isasymin}M\ {\isasymLongrightarrow}\ z{\isasymin}M\ {\isasymLongrightarrow}\ sats{\isacharparenleft}{\kern0pt}M{\isacharcomma}{\kern0pt}{\isasymphi}{\isacharcomma}{\kern0pt}Cons{\isacharparenleft}{\kern0pt}x{\isacharcomma}{\kern0pt}Cons{\isacharparenleft}{\kern0pt}z{\isacharcomma}{\kern0pt}env{\isacharparenright}{\kern0pt}{\isacharparenright}{\kern0pt}{\isacharparenright}{\kern0pt}\ {\isasymlongleftrightarrow}\ P{\isacharparenleft}{\kern0pt}x{\isacharcomma}{\kern0pt}z{\isacharparenright}{\kern0pt}{\isachardoublequoteclose}\isanewline
\ \ \isakeyword{shows}\isanewline
\ \ \ \ {\isachardoublequoteopen}strong{\isacharunderscore}{\kern0pt}replacement{\isacharparenleft}{\kern0pt}{\isacharhash}{\kern0pt}{\isacharhash}{\kern0pt}M{\isacharcomma}{\kern0pt}{\isasymlambda}x\ z{\isachardot}{\kern0pt}\ sats{\isacharparenleft}{\kern0pt}M{\isacharcomma}{\kern0pt}{\isasymphi}{\isacharcomma}{\kern0pt}Cons{\isacharparenleft}{\kern0pt}x{\isacharcomma}{\kern0pt}Cons{\isacharparenleft}{\kern0pt}z{\isacharcomma}{\kern0pt}env{\isacharparenright}{\kern0pt}{\isacharparenright}{\kern0pt}{\isacharparenright}{\kern0pt}{\isacharparenright}{\kern0pt}\ {\isasymlongleftrightarrow}\isanewline
\ \ \ strong{\isacharunderscore}{\kern0pt}replacement{\isacharparenleft}{\kern0pt}{\isacharhash}{\kern0pt}{\isacharhash}{\kern0pt}M{\isacharcomma}{\kern0pt}P{\isacharparenright}{\kern0pt}{\isachardoublequoteclose}\isanewline
%
\isadelimproof
\ \ %
\endisadelimproof
%
\isatagproof
\isacommand{by}\isamarkupfalse%
\ {\isacharparenleft}{\kern0pt}rule\ strong{\isacharunderscore}{\kern0pt}replacement{\isacharunderscore}{\kern0pt}cong{\isacharcomma}{\kern0pt}simp\ add{\isacharcolon}{\kern0pt}sat{\isacharparenright}{\kern0pt}%
\endisatagproof
{\isafoldproof}%
%
\isadelimproof
\isanewline
%
\endisadelimproof
\isanewline
\isacommand{lemma}\isamarkupfalse%
\ {\isacharparenleft}{\kern0pt}\isakeyword{in}\ M{\isacharunderscore}{\kern0pt}ZF{\isacharunderscore}{\kern0pt}trans{\isacharparenright}{\kern0pt}\ nat{\isacharunderscore}{\kern0pt}trans{\isacharunderscore}{\kern0pt}M\ {\isacharcolon}{\kern0pt}\isanewline
\ \ {\isachardoublequoteopen}n{\isasymin}M{\isachardoublequoteclose}\ \isakeyword{if}\ {\isachardoublequoteopen}n{\isasymin}nat{\isachardoublequoteclose}\ \isakeyword{for}\ n\isanewline
%
\isadelimproof
\ \ %
\endisadelimproof
%
\isatagproof
\isacommand{using}\isamarkupfalse%
\ that\ nat{\isacharunderscore}{\kern0pt}in{\isacharunderscore}{\kern0pt}M\ Transset{\isacharunderscore}{\kern0pt}intf{\isacharbrackleft}{\kern0pt}OF\ trans{\isacharunderscore}{\kern0pt}M{\isacharbrackright}{\kern0pt}\ \isacommand{by}\isamarkupfalse%
\ simp%
\endisatagproof
{\isafoldproof}%
%
\isadelimproof
\isanewline
%
\endisadelimproof
\isanewline
\isacommand{lemma}\isamarkupfalse%
\ {\isacharparenleft}{\kern0pt}\isakeyword{in}\ M{\isacharunderscore}{\kern0pt}ZF{\isacharunderscore}{\kern0pt}trans{\isacharparenright}{\kern0pt}\ list{\isacharunderscore}{\kern0pt}repl{\isadigit{1}}{\isacharunderscore}{\kern0pt}intf{\isacharcolon}{\kern0pt}\isanewline
\ \ \isakeyword{assumes}\isanewline
\ \ \ \ {\isachardoublequoteopen}A{\isasymin}M{\isachardoublequoteclose}\isanewline
\ \ \isakeyword{shows}\isanewline
\ \ \ \ {\isachardoublequoteopen}iterates{\isacharunderscore}{\kern0pt}replacement{\isacharparenleft}{\kern0pt}{\isacharhash}{\kern0pt}{\isacharhash}{\kern0pt}M{\isacharcomma}{\kern0pt}\ is{\isacharunderscore}{\kern0pt}list{\isacharunderscore}{\kern0pt}functor{\isacharparenleft}{\kern0pt}{\isacharhash}{\kern0pt}{\isacharhash}{\kern0pt}M{\isacharcomma}{\kern0pt}A{\isacharparenright}{\kern0pt}{\isacharcomma}{\kern0pt}\ {\isadigit{0}}{\isacharparenright}{\kern0pt}{\isachardoublequoteclose}\isanewline
%
\isadelimproof
%
\endisadelimproof
%
\isatagproof
\isacommand{proof}\isamarkupfalse%
\ {\isacharminus}{\kern0pt}\isanewline
\ \ \isacommand{{\isacharbraceleft}{\kern0pt}}\isamarkupfalse%
\isanewline
\ \ \ \ \isacommand{fix}\isamarkupfalse%
\ n\isanewline
\ \ \ \ \isacommand{assume}\isamarkupfalse%
\ {\isachardoublequoteopen}n{\isasymin}nat{\isachardoublequoteclose}\isanewline
\ \ \ \ \isacommand{have}\isamarkupfalse%
\ {\isachardoublequoteopen}succ{\isacharparenleft}{\kern0pt}n{\isacharparenright}{\kern0pt}{\isasymin}M{\isachardoublequoteclose}\isanewline
\ \ \ \ \ \ \isacommand{using}\isamarkupfalse%
\ {\isacartoucheopen}n{\isasymin}nat{\isacartoucheclose}\ nat{\isacharunderscore}{\kern0pt}trans{\isacharunderscore}{\kern0pt}M\ \isacommand{by}\isamarkupfalse%
\ simp\isanewline
\ \ \ \ \isacommand{then}\isamarkupfalse%
\ \isacommand{have}\isamarkupfalse%
\ {\isadigit{1}}{\isacharcolon}{\kern0pt}{\isachardoublequoteopen}Memrel{\isacharparenleft}{\kern0pt}succ{\isacharparenleft}{\kern0pt}n{\isacharparenright}{\kern0pt}{\isacharparenright}{\kern0pt}{\isasymin}M{\isachardoublequoteclose}\isanewline
\ \ \ \ \ \ \isacommand{using}\isamarkupfalse%
\ {\isacartoucheopen}n{\isasymin}nat{\isacartoucheclose}\ Memrel{\isacharunderscore}{\kern0pt}closed\ \isacommand{by}\isamarkupfalse%
\ simp\isanewline
\ \ \ \ \isacommand{have}\isamarkupfalse%
\ {\isachardoublequoteopen}{\isadigit{0}}{\isasymin}M{\isachardoublequoteclose}\isanewline
\ \ \ \ \ \ \isacommand{using}\isamarkupfalse%
\ \ nat{\isacharunderscore}{\kern0pt}{\isadigit{0}}I\ nat{\isacharunderscore}{\kern0pt}trans{\isacharunderscore}{\kern0pt}M\ \isacommand{by}\isamarkupfalse%
\ simp\isanewline
\ \ \ \ \isacommand{then}\isamarkupfalse%
\ \isacommand{have}\isamarkupfalse%
\ {\isachardoublequoteopen}is{\isacharunderscore}{\kern0pt}list{\isacharunderscore}{\kern0pt}functor{\isacharparenleft}{\kern0pt}{\isacharhash}{\kern0pt}{\isacharhash}{\kern0pt}M{\isacharcomma}{\kern0pt}\ A{\isacharcomma}{\kern0pt}\ a{\isacharcomma}{\kern0pt}\ b{\isacharparenright}{\kern0pt}\isanewline
\ \ \ \ \ \ \ {\isasymlongleftrightarrow}\ sats{\isacharparenleft}{\kern0pt}M{\isacharcomma}{\kern0pt}\ list{\isacharunderscore}{\kern0pt}functor{\isacharunderscore}{\kern0pt}fm{\isacharparenleft}{\kern0pt}{\isadigit{1}}{\isadigit{3}}{\isacharcomma}{\kern0pt}{\isadigit{1}}{\isacharcomma}{\kern0pt}{\isadigit{0}}{\isacharparenright}{\kern0pt}{\isacharcomma}{\kern0pt}\ {\isacharbrackleft}{\kern0pt}b{\isacharcomma}{\kern0pt}a{\isacharcomma}{\kern0pt}c{\isacharcomma}{\kern0pt}d{\isacharcomma}{\kern0pt}a{\isadigit{0}}{\isacharcomma}{\kern0pt}a{\isadigit{1}}{\isacharcomma}{\kern0pt}a{\isadigit{2}}{\isacharcomma}{\kern0pt}a{\isadigit{3}}{\isacharcomma}{\kern0pt}a{\isadigit{4}}{\isacharcomma}{\kern0pt}y{\isacharcomma}{\kern0pt}x{\isacharcomma}{\kern0pt}z{\isacharcomma}{\kern0pt}Memrel{\isacharparenleft}{\kern0pt}succ{\isacharparenleft}{\kern0pt}n{\isacharparenright}{\kern0pt}{\isacharparenright}{\kern0pt}{\isacharcomma}{\kern0pt}A{\isacharcomma}{\kern0pt}{\isadigit{0}}{\isacharbrackright}{\kern0pt}{\isacharparenright}{\kern0pt}{\isachardoublequoteclose}\isanewline
\ \ \ \ \ \ \isakeyword{if}\ {\isachardoublequoteopen}a{\isasymin}M{\isachardoublequoteclose}\ {\isachardoublequoteopen}b{\isasymin}M{\isachardoublequoteclose}\ {\isachardoublequoteopen}c{\isasymin}M{\isachardoublequoteclose}\ {\isachardoublequoteopen}d{\isasymin}M{\isachardoublequoteclose}\ {\isachardoublequoteopen}a{\isadigit{0}}{\isasymin}M{\isachardoublequoteclose}\ {\isachardoublequoteopen}a{\isadigit{1}}{\isasymin}M{\isachardoublequoteclose}\ {\isachardoublequoteopen}a{\isadigit{2}}{\isasymin}M{\isachardoublequoteclose}\ {\isachardoublequoteopen}a{\isadigit{3}}{\isasymin}M{\isachardoublequoteclose}\ {\isachardoublequoteopen}a{\isadigit{4}}{\isasymin}M{\isachardoublequoteclose}\ {\isachardoublequoteopen}y{\isasymin}M{\isachardoublequoteclose}\ {\isachardoublequoteopen}x{\isasymin}M{\isachardoublequoteclose}\ {\isachardoublequoteopen}z{\isasymin}M{\isachardoublequoteclose}\isanewline
\ \ \ \ \ \ \isakeyword{for}\ a\ b\ c\ d\ a{\isadigit{0}}\ a{\isadigit{1}}\ a{\isadigit{2}}\ a{\isadigit{3}}\ a{\isadigit{4}}\ y\ x\ z\isanewline
\ \ \ \ \ \ \isacommand{using}\isamarkupfalse%
\ that\ {\isadigit{1}}\ {\isacartoucheopen}A{\isasymin}M{\isacartoucheclose}\ list{\isacharunderscore}{\kern0pt}functor{\isacharunderscore}{\kern0pt}iff{\isacharunderscore}{\kern0pt}sats\ \isacommand{by}\isamarkupfalse%
\ simp\isanewline
\ \ \ \ \isacommand{then}\isamarkupfalse%
\ \isacommand{have}\isamarkupfalse%
\ {\isachardoublequoteopen}sats{\isacharparenleft}{\kern0pt}M{\isacharcomma}{\kern0pt}\ iterates{\isacharunderscore}{\kern0pt}MH{\isacharunderscore}{\kern0pt}fm{\isacharparenleft}{\kern0pt}list{\isacharunderscore}{\kern0pt}functor{\isacharunderscore}{\kern0pt}fm{\isacharparenleft}{\kern0pt}{\isadigit{1}}{\isadigit{3}}{\isacharcomma}{\kern0pt}{\isadigit{1}}{\isacharcomma}{\kern0pt}{\isadigit{0}}{\isacharparenright}{\kern0pt}{\isacharcomma}{\kern0pt}{\isadigit{1}}{\isadigit{0}}{\isacharcomma}{\kern0pt}{\isadigit{2}}{\isacharcomma}{\kern0pt}{\isadigit{1}}{\isacharcomma}{\kern0pt}{\isadigit{0}}{\isacharparenright}{\kern0pt}{\isacharcomma}{\kern0pt}\ {\isacharbrackleft}{\kern0pt}a{\isadigit{0}}{\isacharcomma}{\kern0pt}a{\isadigit{1}}{\isacharcomma}{\kern0pt}a{\isadigit{2}}{\isacharcomma}{\kern0pt}a{\isadigit{3}}{\isacharcomma}{\kern0pt}a{\isadigit{4}}{\isacharcomma}{\kern0pt}y{\isacharcomma}{\kern0pt}x{\isacharcomma}{\kern0pt}z{\isacharcomma}{\kern0pt}Memrel{\isacharparenleft}{\kern0pt}succ{\isacharparenleft}{\kern0pt}n{\isacharparenright}{\kern0pt}{\isacharparenright}{\kern0pt}{\isacharcomma}{\kern0pt}A{\isacharcomma}{\kern0pt}{\isadigit{0}}{\isacharbrackright}{\kern0pt}{\isacharparenright}{\kern0pt}\isanewline
\ \ \ \ \ \ \ \ {\isasymlongleftrightarrow}\ iterates{\isacharunderscore}{\kern0pt}MH{\isacharparenleft}{\kern0pt}{\isacharhash}{\kern0pt}{\isacharhash}{\kern0pt}M{\isacharcomma}{\kern0pt}is{\isacharunderscore}{\kern0pt}list{\isacharunderscore}{\kern0pt}functor{\isacharparenleft}{\kern0pt}{\isacharhash}{\kern0pt}{\isacharhash}{\kern0pt}M{\isacharcomma}{\kern0pt}A{\isacharparenright}{\kern0pt}{\isacharcomma}{\kern0pt}{\isadigit{0}}{\isacharcomma}{\kern0pt}a{\isadigit{2}}{\isacharcomma}{\kern0pt}\ a{\isadigit{1}}{\isacharcomma}{\kern0pt}\ a{\isadigit{0}}{\isacharparenright}{\kern0pt}{\isachardoublequoteclose}\isanewline
\ \ \ \ \ \ \isakeyword{if}\ {\isachardoublequoteopen}a{\isadigit{0}}{\isasymin}M{\isachardoublequoteclose}\ {\isachardoublequoteopen}a{\isadigit{1}}{\isasymin}M{\isachardoublequoteclose}\ {\isachardoublequoteopen}a{\isadigit{2}}{\isasymin}M{\isachardoublequoteclose}\ {\isachardoublequoteopen}a{\isadigit{3}}{\isasymin}M{\isachardoublequoteclose}\ {\isachardoublequoteopen}a{\isadigit{4}}{\isasymin}M{\isachardoublequoteclose}\ {\isachardoublequoteopen}y{\isasymin}M{\isachardoublequoteclose}\ {\isachardoublequoteopen}x{\isasymin}M{\isachardoublequoteclose}\ {\isachardoublequoteopen}z{\isasymin}M{\isachardoublequoteclose}\isanewline
\ \ \ \ \ \ \isakeyword{for}\ a{\isadigit{0}}\ a{\isadigit{1}}\ a{\isadigit{2}}\ a{\isadigit{3}}\ a{\isadigit{4}}\ y\ x\ z\isanewline
\ \ \ \ \ \ \isacommand{using}\isamarkupfalse%
\ that\ sats{\isacharunderscore}{\kern0pt}iterates{\isacharunderscore}{\kern0pt}MH{\isacharunderscore}{\kern0pt}fm{\isacharbrackleft}{\kern0pt}of\ M\ {\isachardoublequoteopen}is{\isacharunderscore}{\kern0pt}list{\isacharunderscore}{\kern0pt}functor{\isacharparenleft}{\kern0pt}{\isacharhash}{\kern0pt}{\isacharhash}{\kern0pt}M{\isacharcomma}{\kern0pt}A{\isacharparenright}{\kern0pt}{\isachardoublequoteclose}\ {\isacharunderscore}{\kern0pt}{\isacharbrackright}{\kern0pt}\ {\isadigit{1}}\ {\isacartoucheopen}{\isadigit{0}}{\isasymin}M{\isacartoucheclose}\ {\isacartoucheopen}A{\isasymin}M{\isacartoucheclose}\ \ \isacommand{by}\isamarkupfalse%
\ simp\isanewline
\ \ \ \ \isacommand{then}\isamarkupfalse%
\ \isacommand{have}\isamarkupfalse%
\ {\isadigit{2}}{\isacharcolon}{\kern0pt}{\isachardoublequoteopen}sats{\isacharparenleft}{\kern0pt}M{\isacharcomma}{\kern0pt}\ is{\isacharunderscore}{\kern0pt}wfrec{\isacharunderscore}{\kern0pt}fm{\isacharparenleft}{\kern0pt}iterates{\isacharunderscore}{\kern0pt}MH{\isacharunderscore}{\kern0pt}fm{\isacharparenleft}{\kern0pt}list{\isacharunderscore}{\kern0pt}functor{\isacharunderscore}{\kern0pt}fm{\isacharparenleft}{\kern0pt}{\isadigit{1}}{\isadigit{3}}{\isacharcomma}{\kern0pt}{\isadigit{1}}{\isacharcomma}{\kern0pt}{\isadigit{0}}{\isacharparenright}{\kern0pt}{\isacharcomma}{\kern0pt}{\isadigit{1}}{\isadigit{0}}{\isacharcomma}{\kern0pt}{\isadigit{2}}{\isacharcomma}{\kern0pt}{\isadigit{1}}{\isacharcomma}{\kern0pt}{\isadigit{0}}{\isacharparenright}{\kern0pt}{\isacharcomma}{\kern0pt}{\isadigit{3}}{\isacharcomma}{\kern0pt}{\isadigit{1}}{\isacharcomma}{\kern0pt}{\isadigit{0}}{\isacharparenright}{\kern0pt}{\isacharcomma}{\kern0pt}\isanewline
\ \ \ \ \ \ \ \ \ \ \ \ \ \ \ \ \ \ \ \ \ \ \ \ \ \ \ \ {\isacharbrackleft}{\kern0pt}y{\isacharcomma}{\kern0pt}x{\isacharcomma}{\kern0pt}z{\isacharcomma}{\kern0pt}Memrel{\isacharparenleft}{\kern0pt}succ{\isacharparenleft}{\kern0pt}n{\isacharparenright}{\kern0pt}{\isacharparenright}{\kern0pt}{\isacharcomma}{\kern0pt}A{\isacharcomma}{\kern0pt}{\isadigit{0}}{\isacharbrackright}{\kern0pt}{\isacharparenright}{\kern0pt}\isanewline
\ \ \ \ \ \ \ \ {\isasymlongleftrightarrow}\isanewline
\ \ \ \ \ \ \ \ is{\isacharunderscore}{\kern0pt}wfrec{\isacharparenleft}{\kern0pt}{\isacharhash}{\kern0pt}{\isacharhash}{\kern0pt}M{\isacharcomma}{\kern0pt}\ iterates{\isacharunderscore}{\kern0pt}MH{\isacharparenleft}{\kern0pt}{\isacharhash}{\kern0pt}{\isacharhash}{\kern0pt}M{\isacharcomma}{\kern0pt}is{\isacharunderscore}{\kern0pt}list{\isacharunderscore}{\kern0pt}functor{\isacharparenleft}{\kern0pt}{\isacharhash}{\kern0pt}{\isacharhash}{\kern0pt}M{\isacharcomma}{\kern0pt}A{\isacharparenright}{\kern0pt}{\isacharcomma}{\kern0pt}{\isadigit{0}}{\isacharparenright}{\kern0pt}\ {\isacharcomma}{\kern0pt}\ Memrel{\isacharparenleft}{\kern0pt}succ{\isacharparenleft}{\kern0pt}n{\isacharparenright}{\kern0pt}{\isacharparenright}{\kern0pt}{\isacharcomma}{\kern0pt}\ x{\isacharcomma}{\kern0pt}\ y{\isacharparenright}{\kern0pt}{\isachardoublequoteclose}\isanewline
\ \ \ \ \ \ \isakeyword{if}\ {\isachardoublequoteopen}y{\isasymin}M{\isachardoublequoteclose}\ {\isachardoublequoteopen}x{\isasymin}M{\isachardoublequoteclose}\ {\isachardoublequoteopen}z{\isasymin}M{\isachardoublequoteclose}\ \isakeyword{for}\ y\ x\ z\isanewline
\ \ \ \ \ \ \isacommand{using}\isamarkupfalse%
\ \ that\ sats{\isacharunderscore}{\kern0pt}is{\isacharunderscore}{\kern0pt}wfrec{\isacharunderscore}{\kern0pt}fm\ {\isadigit{1}}\ {\isacartoucheopen}{\isadigit{0}}{\isasymin}M{\isacartoucheclose}\ {\isacartoucheopen}A{\isasymin}M{\isacartoucheclose}\ \isacommand{by}\isamarkupfalse%
\ simp\isanewline
\ \ \ \ \isacommand{let}\isamarkupfalse%
\isanewline
\ \ \ \ \ \ {\isacharquery}{\kern0pt}f{\isacharequal}{\kern0pt}{\isachardoublequoteopen}Exists{\isacharparenleft}{\kern0pt}And{\isacharparenleft}{\kern0pt}pair{\isacharunderscore}{\kern0pt}fm{\isacharparenleft}{\kern0pt}{\isadigit{1}}{\isacharcomma}{\kern0pt}{\isadigit{0}}{\isacharcomma}{\kern0pt}{\isadigit{2}}{\isacharparenright}{\kern0pt}{\isacharcomma}{\kern0pt}\isanewline
\ \ \ \ \ \ \ \ \ \ \ \ \ \ \ is{\isacharunderscore}{\kern0pt}wfrec{\isacharunderscore}{\kern0pt}fm{\isacharparenleft}{\kern0pt}iterates{\isacharunderscore}{\kern0pt}MH{\isacharunderscore}{\kern0pt}fm{\isacharparenleft}{\kern0pt}list{\isacharunderscore}{\kern0pt}functor{\isacharunderscore}{\kern0pt}fm{\isacharparenleft}{\kern0pt}{\isadigit{1}}{\isadigit{3}}{\isacharcomma}{\kern0pt}{\isadigit{1}}{\isacharcomma}{\kern0pt}{\isadigit{0}}{\isacharparenright}{\kern0pt}{\isacharcomma}{\kern0pt}{\isadigit{1}}{\isadigit{0}}{\isacharcomma}{\kern0pt}{\isadigit{2}}{\isacharcomma}{\kern0pt}{\isadigit{1}}{\isacharcomma}{\kern0pt}{\isadigit{0}}{\isacharparenright}{\kern0pt}{\isacharcomma}{\kern0pt}{\isadigit{3}}{\isacharcomma}{\kern0pt}{\isadigit{1}}{\isacharcomma}{\kern0pt}{\isadigit{0}}{\isacharparenright}{\kern0pt}{\isacharparenright}{\kern0pt}{\isacharparenright}{\kern0pt}{\isachardoublequoteclose}\isanewline
\ \ \ \ \isacommand{have}\isamarkupfalse%
\ satsf{\isacharcolon}{\kern0pt}{\isachardoublequoteopen}sats{\isacharparenleft}{\kern0pt}M{\isacharcomma}{\kern0pt}\ {\isacharquery}{\kern0pt}f{\isacharcomma}{\kern0pt}\ {\isacharbrackleft}{\kern0pt}x{\isacharcomma}{\kern0pt}z{\isacharcomma}{\kern0pt}Memrel{\isacharparenleft}{\kern0pt}succ{\isacharparenleft}{\kern0pt}n{\isacharparenright}{\kern0pt}{\isacharparenright}{\kern0pt}{\isacharcomma}{\kern0pt}A{\isacharcomma}{\kern0pt}{\isadigit{0}}{\isacharbrackright}{\kern0pt}{\isacharparenright}{\kern0pt}\isanewline
\ \ \ \ \ \ \ \ {\isasymlongleftrightarrow}\isanewline
\ \ \ \ \ \ \ \ {\isacharparenleft}{\kern0pt}{\isasymexists}y{\isasymin}M{\isachardot}{\kern0pt}\ pair{\isacharparenleft}{\kern0pt}{\isacharhash}{\kern0pt}{\isacharhash}{\kern0pt}M{\isacharcomma}{\kern0pt}x{\isacharcomma}{\kern0pt}y{\isacharcomma}{\kern0pt}z{\isacharparenright}{\kern0pt}\ {\isacharampersand}{\kern0pt}\isanewline
\ \ \ \ \ \ \ \ is{\isacharunderscore}{\kern0pt}wfrec{\isacharparenleft}{\kern0pt}{\isacharhash}{\kern0pt}{\isacharhash}{\kern0pt}M{\isacharcomma}{\kern0pt}\ iterates{\isacharunderscore}{\kern0pt}MH{\isacharparenleft}{\kern0pt}{\isacharhash}{\kern0pt}{\isacharhash}{\kern0pt}M{\isacharcomma}{\kern0pt}is{\isacharunderscore}{\kern0pt}list{\isacharunderscore}{\kern0pt}functor{\isacharparenleft}{\kern0pt}{\isacharhash}{\kern0pt}{\isacharhash}{\kern0pt}M{\isacharcomma}{\kern0pt}A{\isacharparenright}{\kern0pt}{\isacharcomma}{\kern0pt}{\isadigit{0}}{\isacharparenright}{\kern0pt}\ {\isacharcomma}{\kern0pt}\ Memrel{\isacharparenleft}{\kern0pt}succ{\isacharparenleft}{\kern0pt}n{\isacharparenright}{\kern0pt}{\isacharparenright}{\kern0pt}{\isacharcomma}{\kern0pt}\ x{\isacharcomma}{\kern0pt}\ y{\isacharparenright}{\kern0pt}{\isacharparenright}{\kern0pt}{\isachardoublequoteclose}\isanewline
\ \ \ \ \ \ \isakeyword{if}\ {\isachardoublequoteopen}x{\isasymin}M{\isachardoublequoteclose}\ {\isachardoublequoteopen}z{\isasymin}M{\isachardoublequoteclose}\ \isakeyword{for}\ x\ z\isanewline
\ \ \ \ \ \ \isacommand{using}\isamarkupfalse%
\ that\ {\isadigit{2}}\ {\isadigit{1}}\ {\isacartoucheopen}{\isadigit{0}}{\isasymin}M{\isacartoucheclose}\ {\isacartoucheopen}A{\isasymin}M{\isacartoucheclose}\ \isacommand{by}\isamarkupfalse%
\ {\isacharparenleft}{\kern0pt}simp\ del{\isacharcolon}{\kern0pt}pair{\isacharunderscore}{\kern0pt}abs{\isacharparenright}{\kern0pt}\isanewline
\ \ \ \ \isacommand{have}\isamarkupfalse%
\ {\isachardoublequoteopen}arity{\isacharparenleft}{\kern0pt}{\isacharquery}{\kern0pt}f{\isacharparenright}{\kern0pt}\ {\isacharequal}{\kern0pt}\ {\isadigit{5}}{\isachardoublequoteclose}\isanewline
\ \ \ \ \ \ \isacommand{unfolding}\isamarkupfalse%
\ iterates{\isacharunderscore}{\kern0pt}MH{\isacharunderscore}{\kern0pt}fm{\isacharunderscore}{\kern0pt}def\ is{\isacharunderscore}{\kern0pt}wfrec{\isacharunderscore}{\kern0pt}fm{\isacharunderscore}{\kern0pt}def\ is{\isacharunderscore}{\kern0pt}recfun{\isacharunderscore}{\kern0pt}fm{\isacharunderscore}{\kern0pt}def\ is{\isacharunderscore}{\kern0pt}nat{\isacharunderscore}{\kern0pt}case{\isacharunderscore}{\kern0pt}fm{\isacharunderscore}{\kern0pt}def\isanewline
\ \ \ \ \ \ \ \ restriction{\isacharunderscore}{\kern0pt}fm{\isacharunderscore}{\kern0pt}def\ list{\isacharunderscore}{\kern0pt}functor{\isacharunderscore}{\kern0pt}fm{\isacharunderscore}{\kern0pt}def\ number{\isadigit{1}}{\isacharunderscore}{\kern0pt}fm{\isacharunderscore}{\kern0pt}def\ cartprod{\isacharunderscore}{\kern0pt}fm{\isacharunderscore}{\kern0pt}def\isanewline
\ \ \ \ \ \ \ \ sum{\isacharunderscore}{\kern0pt}fm{\isacharunderscore}{\kern0pt}def\ quasinat{\isacharunderscore}{\kern0pt}fm{\isacharunderscore}{\kern0pt}def\ pre{\isacharunderscore}{\kern0pt}image{\isacharunderscore}{\kern0pt}fm{\isacharunderscore}{\kern0pt}def\ fm{\isacharunderscore}{\kern0pt}defs\isanewline
\ \ \ \ \ \ \isacommand{by}\isamarkupfalse%
\ {\isacharparenleft}{\kern0pt}simp\ add{\isacharcolon}{\kern0pt}nat{\isacharunderscore}{\kern0pt}simp{\isacharunderscore}{\kern0pt}union{\isacharparenright}{\kern0pt}\isanewline
\ \ \ \ \isacommand{then}\isamarkupfalse%
\isanewline
\ \ \ \ \isacommand{have}\isamarkupfalse%
\ {\isachardoublequoteopen}strong{\isacharunderscore}{\kern0pt}replacement{\isacharparenleft}{\kern0pt}{\isacharhash}{\kern0pt}{\isacharhash}{\kern0pt}M{\isacharcomma}{\kern0pt}{\isasymlambda}x\ z{\isachardot}{\kern0pt}\ sats{\isacharparenleft}{\kern0pt}M{\isacharcomma}{\kern0pt}{\isacharquery}{\kern0pt}f{\isacharcomma}{\kern0pt}{\isacharbrackleft}{\kern0pt}x{\isacharcomma}{\kern0pt}z{\isacharcomma}{\kern0pt}Memrel{\isacharparenleft}{\kern0pt}succ{\isacharparenleft}{\kern0pt}n{\isacharparenright}{\kern0pt}{\isacharparenright}{\kern0pt}{\isacharcomma}{\kern0pt}A{\isacharcomma}{\kern0pt}{\isadigit{0}}{\isacharbrackright}{\kern0pt}{\isacharparenright}{\kern0pt}{\isacharparenright}{\kern0pt}{\isachardoublequoteclose}\isanewline
\ \ \ \ \ \ \isacommand{using}\isamarkupfalse%
\ replacement{\isacharunderscore}{\kern0pt}ax\ {\isadigit{1}}\ {\isacartoucheopen}A{\isasymin}M{\isacartoucheclose}\ {\isacartoucheopen}{\isadigit{0}}{\isasymin}M{\isacartoucheclose}\ \isacommand{by}\isamarkupfalse%
\ simp\isanewline
\ \ \ \ \isacommand{then}\isamarkupfalse%
\isanewline
\ \ \ \ \isacommand{have}\isamarkupfalse%
\ {\isachardoublequoteopen}strong{\isacharunderscore}{\kern0pt}replacement{\isacharparenleft}{\kern0pt}{\isacharhash}{\kern0pt}{\isacharhash}{\kern0pt}M{\isacharcomma}{\kern0pt}{\isasymlambda}x\ z{\isachardot}{\kern0pt}\isanewline
\ \ \ \ \ \ \ \ \ \ {\isasymexists}y{\isasymin}M{\isachardot}{\kern0pt}\ pair{\isacharparenleft}{\kern0pt}{\isacharhash}{\kern0pt}{\isacharhash}{\kern0pt}M{\isacharcomma}{\kern0pt}x{\isacharcomma}{\kern0pt}y{\isacharcomma}{\kern0pt}z{\isacharparenright}{\kern0pt}\ {\isacharampersand}{\kern0pt}\ is{\isacharunderscore}{\kern0pt}wfrec{\isacharparenleft}{\kern0pt}{\isacharhash}{\kern0pt}{\isacharhash}{\kern0pt}M{\isacharcomma}{\kern0pt}\ iterates{\isacharunderscore}{\kern0pt}MH{\isacharparenleft}{\kern0pt}{\isacharhash}{\kern0pt}{\isacharhash}{\kern0pt}M{\isacharcomma}{\kern0pt}is{\isacharunderscore}{\kern0pt}list{\isacharunderscore}{\kern0pt}functor{\isacharparenleft}{\kern0pt}{\isacharhash}{\kern0pt}{\isacharhash}{\kern0pt}M{\isacharcomma}{\kern0pt}A{\isacharparenright}{\kern0pt}{\isacharcomma}{\kern0pt}{\isadigit{0}}{\isacharparenright}{\kern0pt}\ {\isacharcomma}{\kern0pt}\isanewline
\ \ \ \ \ \ \ \ \ \ \ \ \ \ \ \ Memrel{\isacharparenleft}{\kern0pt}succ{\isacharparenleft}{\kern0pt}n{\isacharparenright}{\kern0pt}{\isacharparenright}{\kern0pt}{\isacharcomma}{\kern0pt}\ x{\isacharcomma}{\kern0pt}\ y{\isacharparenright}{\kern0pt}{\isacharparenright}{\kern0pt}{\isachardoublequoteclose}\isanewline
\ \ \ \ \ \ \isacommand{using}\isamarkupfalse%
\ repl{\isacharunderscore}{\kern0pt}sats{\isacharbrackleft}{\kern0pt}of\ M\ {\isacharquery}{\kern0pt}f\ {\isachardoublequoteopen}{\isacharbrackleft}{\kern0pt}Memrel{\isacharparenleft}{\kern0pt}succ{\isacharparenleft}{\kern0pt}n{\isacharparenright}{\kern0pt}{\isacharparenright}{\kern0pt}{\isacharcomma}{\kern0pt}A{\isacharcomma}{\kern0pt}{\isadigit{0}}{\isacharbrackright}{\kern0pt}{\isachardoublequoteclose}{\isacharbrackright}{\kern0pt}\ \ satsf\ \isacommand{by}\isamarkupfalse%
\ {\isacharparenleft}{\kern0pt}simp\ del{\isacharcolon}{\kern0pt}pair{\isacharunderscore}{\kern0pt}abs{\isacharparenright}{\kern0pt}\isanewline
\ \ \isacommand{{\isacharbraceright}{\kern0pt}}\isamarkupfalse%
\isanewline
\ \ \isacommand{then}\isamarkupfalse%
\isanewline
\ \ \isacommand{show}\isamarkupfalse%
\ {\isacharquery}{\kern0pt}thesis\ \isacommand{unfolding}\isamarkupfalse%
\ iterates{\isacharunderscore}{\kern0pt}replacement{\isacharunderscore}{\kern0pt}def\ wfrec{\isacharunderscore}{\kern0pt}replacement{\isacharunderscore}{\kern0pt}def\ \isacommand{by}\isamarkupfalse%
\ simp\isanewline
\isacommand{qed}\isamarkupfalse%
%
\endisatagproof
{\isafoldproof}%
%
\isadelimproof
\isanewline
%
\endisadelimproof
\isanewline
\isanewline
\isanewline
\isanewline
\isacommand{lemma}\isamarkupfalse%
\ {\isacharparenleft}{\kern0pt}\isakeyword{in}\ M{\isacharunderscore}{\kern0pt}ZF{\isacharunderscore}{\kern0pt}trans{\isacharparenright}{\kern0pt}\ iterates{\isacharunderscore}{\kern0pt}repl{\isacharunderscore}{\kern0pt}intf\ {\isacharcolon}{\kern0pt}\isanewline
\ \ \isakeyword{assumes}\isanewline
\ \ \ \ {\isachardoublequoteopen}v{\isasymin}M{\isachardoublequoteclose}\ \isakeyword{and}\isanewline
\ \ \ \ isfm{\isacharcolon}{\kern0pt}{\isachardoublequoteopen}is{\isacharunderscore}{\kern0pt}F{\isacharunderscore}{\kern0pt}fm\ {\isasymin}\ formula{\isachardoublequoteclose}\ \isakeyword{and}\isanewline
\ \ \ \ arty{\isacharcolon}{\kern0pt}{\isachardoublequoteopen}arity{\isacharparenleft}{\kern0pt}is{\isacharunderscore}{\kern0pt}F{\isacharunderscore}{\kern0pt}fm{\isacharparenright}{\kern0pt}{\isacharequal}{\kern0pt}{\isadigit{2}}{\isachardoublequoteclose}\ \isakeyword{and}\isanewline
\ \ \ \ satsf{\isacharcolon}{\kern0pt}\ {\isachardoublequoteopen}{\isasymAnd}a\ b\ env{\isacharprime}{\kern0pt}{\isachardot}{\kern0pt}\ {\isasymlbrakk}\ a{\isasymin}M\ {\isacharsemicolon}{\kern0pt}\ b{\isasymin}M\ {\isacharsemicolon}{\kern0pt}\ env{\isacharprime}{\kern0pt}{\isasymin}list{\isacharparenleft}{\kern0pt}M{\isacharparenright}{\kern0pt}\ {\isasymrbrakk}\isanewline
\ \ \ \ \ \ \ \ \ \ \ \ \ \ {\isasymLongrightarrow}\ is{\isacharunderscore}{\kern0pt}F{\isacharparenleft}{\kern0pt}a{\isacharcomma}{\kern0pt}b{\isacharparenright}{\kern0pt}\ {\isasymlongleftrightarrow}\ sats{\isacharparenleft}{\kern0pt}M{\isacharcomma}{\kern0pt}\ is{\isacharunderscore}{\kern0pt}F{\isacharunderscore}{\kern0pt}fm{\isacharcomma}{\kern0pt}\ {\isacharbrackleft}{\kern0pt}b{\isacharcomma}{\kern0pt}a{\isacharbrackright}{\kern0pt}{\isacharat}{\kern0pt}env{\isacharprime}{\kern0pt}{\isacharparenright}{\kern0pt}{\isachardoublequoteclose}\isanewline
\ \ \isakeyword{shows}\isanewline
\ \ \ \ {\isachardoublequoteopen}iterates{\isacharunderscore}{\kern0pt}replacement{\isacharparenleft}{\kern0pt}{\isacharhash}{\kern0pt}{\isacharhash}{\kern0pt}M{\isacharcomma}{\kern0pt}is{\isacharunderscore}{\kern0pt}F{\isacharcomma}{\kern0pt}v{\isacharparenright}{\kern0pt}{\isachardoublequoteclose}\isanewline
%
\isadelimproof
%
\endisadelimproof
%
\isatagproof
\isacommand{proof}\isamarkupfalse%
\ {\isacharminus}{\kern0pt}\isanewline
\ \ \isacommand{{\isacharbraceleft}{\kern0pt}}\isamarkupfalse%
\isanewline
\ \ \ \ \isacommand{fix}\isamarkupfalse%
\ n\isanewline
\ \ \ \ \isacommand{assume}\isamarkupfalse%
\ {\isachardoublequoteopen}n{\isasymin}nat{\isachardoublequoteclose}\isanewline
\ \ \ \ \isacommand{have}\isamarkupfalse%
\ {\isachardoublequoteopen}succ{\isacharparenleft}{\kern0pt}n{\isacharparenright}{\kern0pt}{\isasymin}M{\isachardoublequoteclose}\isanewline
\ \ \ \ \ \ \isacommand{using}\isamarkupfalse%
\ {\isacartoucheopen}n{\isasymin}nat{\isacartoucheclose}\ nat{\isacharunderscore}{\kern0pt}trans{\isacharunderscore}{\kern0pt}M\ \isacommand{by}\isamarkupfalse%
\ simp\isanewline
\ \ \ \ \isacommand{then}\isamarkupfalse%
\ \isacommand{have}\isamarkupfalse%
\ {\isadigit{1}}{\isacharcolon}{\kern0pt}{\isachardoublequoteopen}Memrel{\isacharparenleft}{\kern0pt}succ{\isacharparenleft}{\kern0pt}n{\isacharparenright}{\kern0pt}{\isacharparenright}{\kern0pt}{\isasymin}M{\isachardoublequoteclose}\isanewline
\ \ \ \ \ \ \isacommand{using}\isamarkupfalse%
\ {\isacartoucheopen}n{\isasymin}nat{\isacartoucheclose}\ Memrel{\isacharunderscore}{\kern0pt}closed\ \isacommand{by}\isamarkupfalse%
\ simp\isanewline
\ \ \ \ \isacommand{{\isacharbraceleft}{\kern0pt}}\isamarkupfalse%
\isanewline
\ \ \ \ \ \ \isacommand{fix}\isamarkupfalse%
\ a{\isadigit{0}}\ a{\isadigit{1}}\ a{\isadigit{2}}\ a{\isadigit{3}}\ a{\isadigit{4}}\ y\ x\ z\isanewline
\ \ \ \ \ \ \isacommand{assume}\isamarkupfalse%
\ as{\isacharcolon}{\kern0pt}{\isachardoublequoteopen}a{\isadigit{0}}{\isasymin}M{\isachardoublequoteclose}\ {\isachardoublequoteopen}a{\isadigit{1}}{\isasymin}M{\isachardoublequoteclose}\ {\isachardoublequoteopen}a{\isadigit{2}}{\isasymin}M{\isachardoublequoteclose}\ {\isachardoublequoteopen}a{\isadigit{3}}{\isasymin}M{\isachardoublequoteclose}\ {\isachardoublequoteopen}a{\isadigit{4}}{\isasymin}M{\isachardoublequoteclose}\ {\isachardoublequoteopen}y{\isasymin}M{\isachardoublequoteclose}\ {\isachardoublequoteopen}x{\isasymin}M{\isachardoublequoteclose}\ {\isachardoublequoteopen}z{\isasymin}M{\isachardoublequoteclose}\isanewline
\ \ \ \ \ \ \isacommand{have}\isamarkupfalse%
\ {\isachardoublequoteopen}sats{\isacharparenleft}{\kern0pt}M{\isacharcomma}{\kern0pt}\ is{\isacharunderscore}{\kern0pt}F{\isacharunderscore}{\kern0pt}fm{\isacharcomma}{\kern0pt}\ Cons{\isacharparenleft}{\kern0pt}b{\isacharcomma}{\kern0pt}Cons{\isacharparenleft}{\kern0pt}a{\isacharcomma}{\kern0pt}Cons{\isacharparenleft}{\kern0pt}c{\isacharcomma}{\kern0pt}Cons{\isacharparenleft}{\kern0pt}d{\isacharcomma}{\kern0pt}{\isacharbrackleft}{\kern0pt}a{\isadigit{0}}{\isacharcomma}{\kern0pt}a{\isadigit{1}}{\isacharcomma}{\kern0pt}a{\isadigit{2}}{\isacharcomma}{\kern0pt}a{\isadigit{3}}{\isacharcomma}{\kern0pt}a{\isadigit{4}}{\isacharcomma}{\kern0pt}y{\isacharcomma}{\kern0pt}x{\isacharcomma}{\kern0pt}z{\isacharcomma}{\kern0pt}Memrel{\isacharparenleft}{\kern0pt}succ{\isacharparenleft}{\kern0pt}n{\isacharparenright}{\kern0pt}{\isacharparenright}{\kern0pt}{\isacharcomma}{\kern0pt}v{\isacharbrackright}{\kern0pt}{\isacharparenright}{\kern0pt}{\isacharparenright}{\kern0pt}{\isacharparenright}{\kern0pt}{\isacharparenright}{\kern0pt}{\isacharparenright}{\kern0pt}\isanewline
\ \ \ \ \ \ \ \ \ \ {\isasymlongleftrightarrow}\ is{\isacharunderscore}{\kern0pt}F{\isacharparenleft}{\kern0pt}a{\isacharcomma}{\kern0pt}b{\isacharparenright}{\kern0pt}{\isachardoublequoteclose}\isanewline
\ \ \ \ \ \ \ \ \isakeyword{if}\ {\isachardoublequoteopen}a{\isasymin}M{\isachardoublequoteclose}\ {\isachardoublequoteopen}b{\isasymin}M{\isachardoublequoteclose}\ {\isachardoublequoteopen}c{\isasymin}M{\isachardoublequoteclose}\ {\isachardoublequoteopen}d{\isasymin}M{\isachardoublequoteclose}\ \isakeyword{for}\ a\ b\ c\ d\isanewline
\ \ \ \ \ \ \ \ \isacommand{using}\isamarkupfalse%
\ as\ that\ {\isadigit{1}}\ satsf{\isacharbrackleft}{\kern0pt}of\ a\ b\ {\isachardoublequoteopen}{\isacharbrackleft}{\kern0pt}c{\isacharcomma}{\kern0pt}d{\isacharcomma}{\kern0pt}a{\isadigit{0}}{\isacharcomma}{\kern0pt}a{\isadigit{1}}{\isacharcomma}{\kern0pt}a{\isadigit{2}}{\isacharcomma}{\kern0pt}a{\isadigit{3}}{\isacharcomma}{\kern0pt}a{\isadigit{4}}{\isacharcomma}{\kern0pt}y{\isacharcomma}{\kern0pt}x{\isacharcomma}{\kern0pt}z{\isacharcomma}{\kern0pt}Memrel{\isacharparenleft}{\kern0pt}succ{\isacharparenleft}{\kern0pt}n{\isacharparenright}{\kern0pt}{\isacharparenright}{\kern0pt}{\isacharcomma}{\kern0pt}v{\isacharbrackright}{\kern0pt}{\isachardoublequoteclose}{\isacharbrackright}{\kern0pt}\ {\isacartoucheopen}v{\isasymin}M{\isacartoucheclose}\ \isacommand{by}\isamarkupfalse%
\ simp\isanewline
\ \ \ \ \ \ \isacommand{then}\isamarkupfalse%
\isanewline
\ \ \ \ \ \ \isacommand{have}\isamarkupfalse%
\ {\isachardoublequoteopen}sats{\isacharparenleft}{\kern0pt}M{\isacharcomma}{\kern0pt}\ iterates{\isacharunderscore}{\kern0pt}MH{\isacharunderscore}{\kern0pt}fm{\isacharparenleft}{\kern0pt}is{\isacharunderscore}{\kern0pt}F{\isacharunderscore}{\kern0pt}fm{\isacharcomma}{\kern0pt}{\isadigit{9}}{\isacharcomma}{\kern0pt}{\isadigit{2}}{\isacharcomma}{\kern0pt}{\isadigit{1}}{\isacharcomma}{\kern0pt}{\isadigit{0}}{\isacharparenright}{\kern0pt}{\isacharcomma}{\kern0pt}\ {\isacharbrackleft}{\kern0pt}a{\isadigit{0}}{\isacharcomma}{\kern0pt}a{\isadigit{1}}{\isacharcomma}{\kern0pt}a{\isadigit{2}}{\isacharcomma}{\kern0pt}a{\isadigit{3}}{\isacharcomma}{\kern0pt}a{\isadigit{4}}{\isacharcomma}{\kern0pt}y{\isacharcomma}{\kern0pt}x{\isacharcomma}{\kern0pt}z{\isacharcomma}{\kern0pt}Memrel{\isacharparenleft}{\kern0pt}succ{\isacharparenleft}{\kern0pt}n{\isacharparenright}{\kern0pt}{\isacharparenright}{\kern0pt}{\isacharcomma}{\kern0pt}v{\isacharbrackright}{\kern0pt}{\isacharparenright}{\kern0pt}\isanewline
\ \ \ \ \ \ \ \ \ \ {\isasymlongleftrightarrow}\ iterates{\isacharunderscore}{\kern0pt}MH{\isacharparenleft}{\kern0pt}{\isacharhash}{\kern0pt}{\isacharhash}{\kern0pt}M{\isacharcomma}{\kern0pt}is{\isacharunderscore}{\kern0pt}F{\isacharcomma}{\kern0pt}v{\isacharcomma}{\kern0pt}a{\isadigit{2}}{\isacharcomma}{\kern0pt}\ a{\isadigit{1}}{\isacharcomma}{\kern0pt}\ a{\isadigit{0}}{\isacharparenright}{\kern0pt}{\isachardoublequoteclose}\isanewline
\ \ \ \ \ \ \ \ \isacommand{using}\isamarkupfalse%
\ as\isanewline
\ \ \ \ \ \ \ \ \ \ sats{\isacharunderscore}{\kern0pt}iterates{\isacharunderscore}{\kern0pt}MH{\isacharunderscore}{\kern0pt}fm{\isacharbrackleft}{\kern0pt}of\ M\ {\isachardoublequoteopen}is{\isacharunderscore}{\kern0pt}F{\isachardoublequoteclose}\ {\isachardoublequoteopen}is{\isacharunderscore}{\kern0pt}F{\isacharunderscore}{\kern0pt}fm{\isachardoublequoteclose}{\isacharbrackright}{\kern0pt}\ {\isadigit{1}}\ {\isacartoucheopen}v{\isasymin}M{\isacartoucheclose}\ \isacommand{by}\isamarkupfalse%
\ simp\isanewline
\ \ \ \ \isacommand{{\isacharbraceright}{\kern0pt}}\isamarkupfalse%
\isanewline
\ \ \ \ \isacommand{then}\isamarkupfalse%
\ \isacommand{have}\isamarkupfalse%
\ {\isadigit{2}}{\isacharcolon}{\kern0pt}{\isachardoublequoteopen}sats{\isacharparenleft}{\kern0pt}M{\isacharcomma}{\kern0pt}\ is{\isacharunderscore}{\kern0pt}wfrec{\isacharunderscore}{\kern0pt}fm{\isacharparenleft}{\kern0pt}iterates{\isacharunderscore}{\kern0pt}MH{\isacharunderscore}{\kern0pt}fm{\isacharparenleft}{\kern0pt}is{\isacharunderscore}{\kern0pt}F{\isacharunderscore}{\kern0pt}fm{\isacharcomma}{\kern0pt}{\isadigit{9}}{\isacharcomma}{\kern0pt}{\isadigit{2}}{\isacharcomma}{\kern0pt}{\isadigit{1}}{\isacharcomma}{\kern0pt}{\isadigit{0}}{\isacharparenright}{\kern0pt}{\isacharcomma}{\kern0pt}{\isadigit{3}}{\isacharcomma}{\kern0pt}{\isadigit{1}}{\isacharcomma}{\kern0pt}{\isadigit{0}}{\isacharparenright}{\kern0pt}{\isacharcomma}{\kern0pt}\isanewline
\ \ \ \ \ \ \ \ \ \ \ \ \ \ \ \ \ \ \ \ \ \ \ \ \ \ \ \ {\isacharbrackleft}{\kern0pt}y{\isacharcomma}{\kern0pt}x{\isacharcomma}{\kern0pt}z{\isacharcomma}{\kern0pt}Memrel{\isacharparenleft}{\kern0pt}succ{\isacharparenleft}{\kern0pt}n{\isacharparenright}{\kern0pt}{\isacharparenright}{\kern0pt}{\isacharcomma}{\kern0pt}v{\isacharbrackright}{\kern0pt}{\isacharparenright}{\kern0pt}\isanewline
\ \ \ \ \ \ \ \ {\isasymlongleftrightarrow}\isanewline
\ \ \ \ \ \ \ \ is{\isacharunderscore}{\kern0pt}wfrec{\isacharparenleft}{\kern0pt}{\isacharhash}{\kern0pt}{\isacharhash}{\kern0pt}M{\isacharcomma}{\kern0pt}\ iterates{\isacharunderscore}{\kern0pt}MH{\isacharparenleft}{\kern0pt}{\isacharhash}{\kern0pt}{\isacharhash}{\kern0pt}M{\isacharcomma}{\kern0pt}is{\isacharunderscore}{\kern0pt}F{\isacharcomma}{\kern0pt}v{\isacharparenright}{\kern0pt}{\isacharcomma}{\kern0pt}Memrel{\isacharparenleft}{\kern0pt}succ{\isacharparenleft}{\kern0pt}n{\isacharparenright}{\kern0pt}{\isacharparenright}{\kern0pt}{\isacharcomma}{\kern0pt}\ x{\isacharcomma}{\kern0pt}\ y{\isacharparenright}{\kern0pt}{\isachardoublequoteclose}\isanewline
\ \ \ \ \ \ \isakeyword{if}\ {\isachardoublequoteopen}y{\isasymin}M{\isachardoublequoteclose}\ {\isachardoublequoteopen}x{\isasymin}M{\isachardoublequoteclose}\ {\isachardoublequoteopen}z{\isasymin}M{\isachardoublequoteclose}\ \isakeyword{for}\ y\ x\ z\isanewline
\ \ \ \ \ \ \isacommand{using}\isamarkupfalse%
\ \ that\ sats{\isacharunderscore}{\kern0pt}is{\isacharunderscore}{\kern0pt}wfrec{\isacharunderscore}{\kern0pt}fm\ {\isadigit{1}}\ {\isacartoucheopen}v{\isasymin}M{\isacartoucheclose}\ \isacommand{by}\isamarkupfalse%
\ simp\isanewline
\ \ \ \ \isacommand{let}\isamarkupfalse%
\isanewline
\ \ \ \ \ \ {\isacharquery}{\kern0pt}f{\isacharequal}{\kern0pt}{\isachardoublequoteopen}Exists{\isacharparenleft}{\kern0pt}And{\isacharparenleft}{\kern0pt}pair{\isacharunderscore}{\kern0pt}fm{\isacharparenleft}{\kern0pt}{\isadigit{1}}{\isacharcomma}{\kern0pt}{\isadigit{0}}{\isacharcomma}{\kern0pt}{\isadigit{2}}{\isacharparenright}{\kern0pt}{\isacharcomma}{\kern0pt}\isanewline
\ \ \ \ \ \ \ \ \ \ \ \ \ \ \ is{\isacharunderscore}{\kern0pt}wfrec{\isacharunderscore}{\kern0pt}fm{\isacharparenleft}{\kern0pt}iterates{\isacharunderscore}{\kern0pt}MH{\isacharunderscore}{\kern0pt}fm{\isacharparenleft}{\kern0pt}is{\isacharunderscore}{\kern0pt}F{\isacharunderscore}{\kern0pt}fm{\isacharcomma}{\kern0pt}{\isadigit{9}}{\isacharcomma}{\kern0pt}{\isadigit{2}}{\isacharcomma}{\kern0pt}{\isadigit{1}}{\isacharcomma}{\kern0pt}{\isadigit{0}}{\isacharparenright}{\kern0pt}{\isacharcomma}{\kern0pt}{\isadigit{3}}{\isacharcomma}{\kern0pt}{\isadigit{1}}{\isacharcomma}{\kern0pt}{\isadigit{0}}{\isacharparenright}{\kern0pt}{\isacharparenright}{\kern0pt}{\isacharparenright}{\kern0pt}{\isachardoublequoteclose}\isanewline
\ \ \ \ \isacommand{have}\isamarkupfalse%
\ satsf{\isacharcolon}{\kern0pt}{\isachardoublequoteopen}sats{\isacharparenleft}{\kern0pt}M{\isacharcomma}{\kern0pt}\ {\isacharquery}{\kern0pt}f{\isacharcomma}{\kern0pt}\ {\isacharbrackleft}{\kern0pt}x{\isacharcomma}{\kern0pt}z{\isacharcomma}{\kern0pt}Memrel{\isacharparenleft}{\kern0pt}succ{\isacharparenleft}{\kern0pt}n{\isacharparenright}{\kern0pt}{\isacharparenright}{\kern0pt}{\isacharcomma}{\kern0pt}v{\isacharbrackright}{\kern0pt}{\isacharparenright}{\kern0pt}\isanewline
\ \ \ \ \ \ \ \ {\isasymlongleftrightarrow}\isanewline
\ \ \ \ \ \ \ \ {\isacharparenleft}{\kern0pt}{\isasymexists}y{\isasymin}M{\isachardot}{\kern0pt}\ pair{\isacharparenleft}{\kern0pt}{\isacharhash}{\kern0pt}{\isacharhash}{\kern0pt}M{\isacharcomma}{\kern0pt}x{\isacharcomma}{\kern0pt}y{\isacharcomma}{\kern0pt}z{\isacharparenright}{\kern0pt}\ {\isacharampersand}{\kern0pt}\isanewline
\ \ \ \ \ \ \ \ is{\isacharunderscore}{\kern0pt}wfrec{\isacharparenleft}{\kern0pt}{\isacharhash}{\kern0pt}{\isacharhash}{\kern0pt}M{\isacharcomma}{\kern0pt}\ iterates{\isacharunderscore}{\kern0pt}MH{\isacharparenleft}{\kern0pt}{\isacharhash}{\kern0pt}{\isacharhash}{\kern0pt}M{\isacharcomma}{\kern0pt}is{\isacharunderscore}{\kern0pt}F{\isacharcomma}{\kern0pt}v{\isacharparenright}{\kern0pt}\ {\isacharcomma}{\kern0pt}\ Memrel{\isacharparenleft}{\kern0pt}succ{\isacharparenleft}{\kern0pt}n{\isacharparenright}{\kern0pt}{\isacharparenright}{\kern0pt}{\isacharcomma}{\kern0pt}\ x{\isacharcomma}{\kern0pt}\ y{\isacharparenright}{\kern0pt}{\isacharparenright}{\kern0pt}{\isachardoublequoteclose}\isanewline
\ \ \ \ \ \ \isakeyword{if}\ {\isachardoublequoteopen}x{\isasymin}M{\isachardoublequoteclose}\ {\isachardoublequoteopen}z{\isasymin}M{\isachardoublequoteclose}\ \isakeyword{for}\ x\ z\isanewline
\ \ \ \ \ \ \isacommand{using}\isamarkupfalse%
\ that\ {\isadigit{2}}\ {\isadigit{1}}\ {\isacartoucheopen}v{\isasymin}M{\isacartoucheclose}\ \isacommand{by}\isamarkupfalse%
\ {\isacharparenleft}{\kern0pt}simp\ del{\isacharcolon}{\kern0pt}pair{\isacharunderscore}{\kern0pt}abs{\isacharparenright}{\kern0pt}\isanewline
\ \ \ \ \isacommand{have}\isamarkupfalse%
\ {\isachardoublequoteopen}arity{\isacharparenleft}{\kern0pt}{\isacharquery}{\kern0pt}f{\isacharparenright}{\kern0pt}\ {\isacharequal}{\kern0pt}\ {\isadigit{4}}{\isachardoublequoteclose}\isanewline
\ \ \ \ \ \ \isacommand{unfolding}\isamarkupfalse%
\ iterates{\isacharunderscore}{\kern0pt}MH{\isacharunderscore}{\kern0pt}fm{\isacharunderscore}{\kern0pt}def\ is{\isacharunderscore}{\kern0pt}wfrec{\isacharunderscore}{\kern0pt}fm{\isacharunderscore}{\kern0pt}def\ is{\isacharunderscore}{\kern0pt}recfun{\isacharunderscore}{\kern0pt}fm{\isacharunderscore}{\kern0pt}def\ is{\isacharunderscore}{\kern0pt}nat{\isacharunderscore}{\kern0pt}case{\isacharunderscore}{\kern0pt}fm{\isacharunderscore}{\kern0pt}def\isanewline
\ \ \ \ \ \ \ \ restriction{\isacharunderscore}{\kern0pt}fm{\isacharunderscore}{\kern0pt}def\ pre{\isacharunderscore}{\kern0pt}image{\isacharunderscore}{\kern0pt}fm{\isacharunderscore}{\kern0pt}def\ quasinat{\isacharunderscore}{\kern0pt}fm{\isacharunderscore}{\kern0pt}def\ fm{\isacharunderscore}{\kern0pt}defs\isanewline
\ \ \ \ \ \ \isacommand{using}\isamarkupfalse%
\ arty\ \isacommand{by}\isamarkupfalse%
\ {\isacharparenleft}{\kern0pt}simp\ add{\isacharcolon}{\kern0pt}nat{\isacharunderscore}{\kern0pt}simp{\isacharunderscore}{\kern0pt}union{\isacharparenright}{\kern0pt}\isanewline
\ \ \ \ \isacommand{then}\isamarkupfalse%
\isanewline
\ \ \ \ \isacommand{have}\isamarkupfalse%
\ {\isachardoublequoteopen}strong{\isacharunderscore}{\kern0pt}replacement{\isacharparenleft}{\kern0pt}{\isacharhash}{\kern0pt}{\isacharhash}{\kern0pt}M{\isacharcomma}{\kern0pt}{\isasymlambda}x\ z{\isachardot}{\kern0pt}\ sats{\isacharparenleft}{\kern0pt}M{\isacharcomma}{\kern0pt}{\isacharquery}{\kern0pt}f{\isacharcomma}{\kern0pt}{\isacharbrackleft}{\kern0pt}x{\isacharcomma}{\kern0pt}z{\isacharcomma}{\kern0pt}Memrel{\isacharparenleft}{\kern0pt}succ{\isacharparenleft}{\kern0pt}n{\isacharparenright}{\kern0pt}{\isacharparenright}{\kern0pt}{\isacharcomma}{\kern0pt}v{\isacharbrackright}{\kern0pt}{\isacharparenright}{\kern0pt}{\isacharparenright}{\kern0pt}{\isachardoublequoteclose}\isanewline
\ \ \ \ \ \ \isacommand{using}\isamarkupfalse%
\ replacement{\isacharunderscore}{\kern0pt}ax\ {\isadigit{1}}\ {\isacartoucheopen}v{\isasymin}M{\isacartoucheclose}\ {\isacartoucheopen}is{\isacharunderscore}{\kern0pt}F{\isacharunderscore}{\kern0pt}fm{\isasymin}formula{\isacartoucheclose}\ \isacommand{by}\isamarkupfalse%
\ simp\isanewline
\ \ \ \ \isacommand{then}\isamarkupfalse%
\isanewline
\ \ \ \ \isacommand{have}\isamarkupfalse%
\ {\isachardoublequoteopen}strong{\isacharunderscore}{\kern0pt}replacement{\isacharparenleft}{\kern0pt}{\isacharhash}{\kern0pt}{\isacharhash}{\kern0pt}M{\isacharcomma}{\kern0pt}{\isasymlambda}x\ z{\isachardot}{\kern0pt}\isanewline
\ \ \ \ \ \ \ \ \ \ {\isasymexists}y{\isasymin}M{\isachardot}{\kern0pt}\ pair{\isacharparenleft}{\kern0pt}{\isacharhash}{\kern0pt}{\isacharhash}{\kern0pt}M{\isacharcomma}{\kern0pt}x{\isacharcomma}{\kern0pt}y{\isacharcomma}{\kern0pt}z{\isacharparenright}{\kern0pt}\ {\isacharampersand}{\kern0pt}\ is{\isacharunderscore}{\kern0pt}wfrec{\isacharparenleft}{\kern0pt}{\isacharhash}{\kern0pt}{\isacharhash}{\kern0pt}M{\isacharcomma}{\kern0pt}\ iterates{\isacharunderscore}{\kern0pt}MH{\isacharparenleft}{\kern0pt}{\isacharhash}{\kern0pt}{\isacharhash}{\kern0pt}M{\isacharcomma}{\kern0pt}is{\isacharunderscore}{\kern0pt}F{\isacharcomma}{\kern0pt}v{\isacharparenright}{\kern0pt}\ {\isacharcomma}{\kern0pt}\isanewline
\ \ \ \ \ \ \ \ \ \ \ \ \ \ \ \ Memrel{\isacharparenleft}{\kern0pt}succ{\isacharparenleft}{\kern0pt}n{\isacharparenright}{\kern0pt}{\isacharparenright}{\kern0pt}{\isacharcomma}{\kern0pt}\ x{\isacharcomma}{\kern0pt}\ y{\isacharparenright}{\kern0pt}{\isacharparenright}{\kern0pt}{\isachardoublequoteclose}\isanewline
\ \ \ \ \ \ \isacommand{using}\isamarkupfalse%
\ repl{\isacharunderscore}{\kern0pt}sats{\isacharbrackleft}{\kern0pt}of\ M\ {\isacharquery}{\kern0pt}f\ {\isachardoublequoteopen}{\isacharbrackleft}{\kern0pt}Memrel{\isacharparenleft}{\kern0pt}succ{\isacharparenleft}{\kern0pt}n{\isacharparenright}{\kern0pt}{\isacharparenright}{\kern0pt}{\isacharcomma}{\kern0pt}v{\isacharbrackright}{\kern0pt}{\isachardoublequoteclose}{\isacharbrackright}{\kern0pt}\ \ satsf\ \isacommand{by}\isamarkupfalse%
\ {\isacharparenleft}{\kern0pt}simp\ del{\isacharcolon}{\kern0pt}pair{\isacharunderscore}{\kern0pt}abs{\isacharparenright}{\kern0pt}\isanewline
\ \ \isacommand{{\isacharbraceright}{\kern0pt}}\isamarkupfalse%
\isanewline
\ \ \isacommand{then}\isamarkupfalse%
\isanewline
\ \ \isacommand{show}\isamarkupfalse%
\ {\isacharquery}{\kern0pt}thesis\ \isacommand{unfolding}\isamarkupfalse%
\ iterates{\isacharunderscore}{\kern0pt}replacement{\isacharunderscore}{\kern0pt}def\ wfrec{\isacharunderscore}{\kern0pt}replacement{\isacharunderscore}{\kern0pt}def\ \isacommand{by}\isamarkupfalse%
\ simp\isanewline
\isacommand{qed}\isamarkupfalse%
%
\endisatagproof
{\isafoldproof}%
%
\isadelimproof
\isanewline
%
\endisadelimproof
\isanewline
\isacommand{lemma}\isamarkupfalse%
\ {\isacharparenleft}{\kern0pt}\isakeyword{in}\ M{\isacharunderscore}{\kern0pt}ZF{\isacharunderscore}{\kern0pt}trans{\isacharparenright}{\kern0pt}\ formula{\isacharunderscore}{\kern0pt}repl{\isadigit{1}}{\isacharunderscore}{\kern0pt}intf\ {\isacharcolon}{\kern0pt}\isanewline
\ \ {\isachardoublequoteopen}iterates{\isacharunderscore}{\kern0pt}replacement{\isacharparenleft}{\kern0pt}{\isacharhash}{\kern0pt}{\isacharhash}{\kern0pt}M{\isacharcomma}{\kern0pt}\ is{\isacharunderscore}{\kern0pt}formula{\isacharunderscore}{\kern0pt}functor{\isacharparenleft}{\kern0pt}{\isacharhash}{\kern0pt}{\isacharhash}{\kern0pt}M{\isacharparenright}{\kern0pt}{\isacharcomma}{\kern0pt}\ {\isadigit{0}}{\isacharparenright}{\kern0pt}{\isachardoublequoteclose}\isanewline
%
\isadelimproof
%
\endisadelimproof
%
\isatagproof
\isacommand{proof}\isamarkupfalse%
\ {\isacharminus}{\kern0pt}\isanewline
\ \ \isacommand{have}\isamarkupfalse%
\ {\isachardoublequoteopen}{\isadigit{0}}{\isasymin}M{\isachardoublequoteclose}\isanewline
\ \ \ \ \isacommand{using}\isamarkupfalse%
\ \ nat{\isacharunderscore}{\kern0pt}{\isadigit{0}}I\ nat{\isacharunderscore}{\kern0pt}trans{\isacharunderscore}{\kern0pt}M\ \isacommand{by}\isamarkupfalse%
\ simp\isanewline
\ \ \isacommand{have}\isamarkupfalse%
\ {\isadigit{1}}{\isacharcolon}{\kern0pt}{\isachardoublequoteopen}arity{\isacharparenleft}{\kern0pt}formula{\isacharunderscore}{\kern0pt}functor{\isacharunderscore}{\kern0pt}fm{\isacharparenleft}{\kern0pt}{\isadigit{1}}{\isacharcomma}{\kern0pt}{\isadigit{0}}{\isacharparenright}{\kern0pt}{\isacharparenright}{\kern0pt}\ {\isacharequal}{\kern0pt}\ {\isadigit{2}}{\isachardoublequoteclose}\isanewline
\ \ \ \ \isacommand{unfolding}\isamarkupfalse%
\ formula{\isacharunderscore}{\kern0pt}functor{\isacharunderscore}{\kern0pt}fm{\isacharunderscore}{\kern0pt}def\ fm{\isacharunderscore}{\kern0pt}defs\ sum{\isacharunderscore}{\kern0pt}fm{\isacharunderscore}{\kern0pt}def\ cartprod{\isacharunderscore}{\kern0pt}fm{\isacharunderscore}{\kern0pt}def\ number{\isadigit{1}}{\isacharunderscore}{\kern0pt}fm{\isacharunderscore}{\kern0pt}def\isanewline
\ \ \ \ \isacommand{by}\isamarkupfalse%
\ {\isacharparenleft}{\kern0pt}simp\ add{\isacharcolon}{\kern0pt}nat{\isacharunderscore}{\kern0pt}simp{\isacharunderscore}{\kern0pt}union{\isacharparenright}{\kern0pt}\isanewline
\ \ \isacommand{have}\isamarkupfalse%
\ {\isadigit{2}}{\isacharcolon}{\kern0pt}{\isachardoublequoteopen}formula{\isacharunderscore}{\kern0pt}functor{\isacharunderscore}{\kern0pt}fm{\isacharparenleft}{\kern0pt}{\isadigit{1}}{\isacharcomma}{\kern0pt}{\isadigit{0}}{\isacharparenright}{\kern0pt}{\isasymin}formula{\isachardoublequoteclose}\ \isacommand{by}\isamarkupfalse%
\ simp\isanewline
\ \ \isacommand{have}\isamarkupfalse%
\ {\isachardoublequoteopen}is{\isacharunderscore}{\kern0pt}formula{\isacharunderscore}{\kern0pt}functor{\isacharparenleft}{\kern0pt}{\isacharhash}{\kern0pt}{\isacharhash}{\kern0pt}M{\isacharcomma}{\kern0pt}a{\isacharcomma}{\kern0pt}b{\isacharparenright}{\kern0pt}\ {\isasymlongleftrightarrow}\isanewline
\ \ \ \ \ \ \ \ sats{\isacharparenleft}{\kern0pt}M{\isacharcomma}{\kern0pt}\ formula{\isacharunderscore}{\kern0pt}functor{\isacharunderscore}{\kern0pt}fm{\isacharparenleft}{\kern0pt}{\isadigit{1}}{\isacharcomma}{\kern0pt}{\isadigit{0}}{\isacharparenright}{\kern0pt}{\isacharcomma}{\kern0pt}\ {\isacharbrackleft}{\kern0pt}b{\isacharcomma}{\kern0pt}a{\isacharbrackright}{\kern0pt}{\isacharparenright}{\kern0pt}{\isachardoublequoteclose}\isanewline
\ \ \ \ \isakeyword{if}\ {\isachardoublequoteopen}a{\isasymin}M{\isachardoublequoteclose}\ {\isachardoublequoteopen}b{\isasymin}M{\isachardoublequoteclose}\ \ \isakeyword{for}\ a\ b\isanewline
\ \ \ \ \isacommand{using}\isamarkupfalse%
\ that\ \isacommand{by}\isamarkupfalse%
\ simp\isanewline
\ \ \isacommand{then}\isamarkupfalse%
\ \isacommand{show}\isamarkupfalse%
\ {\isacharquery}{\kern0pt}thesis\ \isacommand{using}\isamarkupfalse%
\ {\isacartoucheopen}{\isadigit{0}}{\isasymin}M{\isacartoucheclose}\ {\isadigit{1}}\ {\isadigit{2}}\ iterates{\isacharunderscore}{\kern0pt}repl{\isacharunderscore}{\kern0pt}intf\ \isacommand{by}\isamarkupfalse%
\ simp\isanewline
\isacommand{qed}\isamarkupfalse%
%
\endisatagproof
{\isafoldproof}%
%
\isadelimproof
\isanewline
%
\endisadelimproof
\isanewline
\isacommand{lemma}\isamarkupfalse%
\ {\isacharparenleft}{\kern0pt}\isakeyword{in}\ M{\isacharunderscore}{\kern0pt}ZF{\isacharunderscore}{\kern0pt}trans{\isacharparenright}{\kern0pt}\ nth{\isacharunderscore}{\kern0pt}repl{\isacharunderscore}{\kern0pt}intf{\isacharcolon}{\kern0pt}\isanewline
\ \ \isakeyword{assumes}\isanewline
\ \ \ \ {\isachardoublequoteopen}l\ {\isasymin}\ M{\isachardoublequoteclose}\isanewline
\ \ \isakeyword{shows}\isanewline
\ \ \ \ {\isachardoublequoteopen}iterates{\isacharunderscore}{\kern0pt}replacement{\isacharparenleft}{\kern0pt}{\isacharhash}{\kern0pt}{\isacharhash}{\kern0pt}M{\isacharcomma}{\kern0pt}{\isasymlambda}l{\isacharprime}{\kern0pt}\ t{\isachardot}{\kern0pt}\ is{\isacharunderscore}{\kern0pt}tl{\isacharparenleft}{\kern0pt}{\isacharhash}{\kern0pt}{\isacharhash}{\kern0pt}M{\isacharcomma}{\kern0pt}l{\isacharprime}{\kern0pt}{\isacharcomma}{\kern0pt}t{\isacharparenright}{\kern0pt}{\isacharcomma}{\kern0pt}l{\isacharparenright}{\kern0pt}{\isachardoublequoteclose}\isanewline
%
\isadelimproof
%
\endisadelimproof
%
\isatagproof
\isacommand{proof}\isamarkupfalse%
\ {\isacharminus}{\kern0pt}\isanewline
\ \ \isacommand{have}\isamarkupfalse%
\ {\isadigit{1}}{\isacharcolon}{\kern0pt}{\isachardoublequoteopen}arity{\isacharparenleft}{\kern0pt}tl{\isacharunderscore}{\kern0pt}fm{\isacharparenleft}{\kern0pt}{\isadigit{1}}{\isacharcomma}{\kern0pt}{\isadigit{0}}{\isacharparenright}{\kern0pt}{\isacharparenright}{\kern0pt}\ {\isacharequal}{\kern0pt}\ {\isadigit{2}}{\isachardoublequoteclose}\isanewline
\ \ \ \ \isacommand{unfolding}\isamarkupfalse%
\ tl{\isacharunderscore}{\kern0pt}fm{\isacharunderscore}{\kern0pt}def\ fm{\isacharunderscore}{\kern0pt}defs\ quasilist{\isacharunderscore}{\kern0pt}fm{\isacharunderscore}{\kern0pt}def\ Cons{\isacharunderscore}{\kern0pt}fm{\isacharunderscore}{\kern0pt}def\ Nil{\isacharunderscore}{\kern0pt}fm{\isacharunderscore}{\kern0pt}def\ Inr{\isacharunderscore}{\kern0pt}fm{\isacharunderscore}{\kern0pt}def\ number{\isadigit{1}}{\isacharunderscore}{\kern0pt}fm{\isacharunderscore}{\kern0pt}def\isanewline
\ \ \ \ \ \ Inl{\isacharunderscore}{\kern0pt}fm{\isacharunderscore}{\kern0pt}def\ \isacommand{by}\isamarkupfalse%
\ {\isacharparenleft}{\kern0pt}simp\ add{\isacharcolon}{\kern0pt}nat{\isacharunderscore}{\kern0pt}simp{\isacharunderscore}{\kern0pt}union{\isacharparenright}{\kern0pt}\isanewline
\ \ \isacommand{have}\isamarkupfalse%
\ {\isadigit{2}}{\isacharcolon}{\kern0pt}{\isachardoublequoteopen}tl{\isacharunderscore}{\kern0pt}fm{\isacharparenleft}{\kern0pt}{\isadigit{1}}{\isacharcomma}{\kern0pt}{\isadigit{0}}{\isacharparenright}{\kern0pt}{\isasymin}formula{\isachardoublequoteclose}\ \isacommand{by}\isamarkupfalse%
\ simp\isanewline
\ \ \isacommand{have}\isamarkupfalse%
\ {\isachardoublequoteopen}is{\isacharunderscore}{\kern0pt}tl{\isacharparenleft}{\kern0pt}{\isacharhash}{\kern0pt}{\isacharhash}{\kern0pt}M{\isacharcomma}{\kern0pt}a{\isacharcomma}{\kern0pt}b{\isacharparenright}{\kern0pt}\ {\isasymlongleftrightarrow}\ sats{\isacharparenleft}{\kern0pt}M{\isacharcomma}{\kern0pt}\ tl{\isacharunderscore}{\kern0pt}fm{\isacharparenleft}{\kern0pt}{\isadigit{1}}{\isacharcomma}{\kern0pt}{\isadigit{0}}{\isacharparenright}{\kern0pt}{\isacharcomma}{\kern0pt}\ {\isacharbrackleft}{\kern0pt}b{\isacharcomma}{\kern0pt}a{\isacharbrackright}{\kern0pt}{\isacharparenright}{\kern0pt}{\isachardoublequoteclose}\isanewline
\ \ \ \ \isakeyword{if}\ {\isachardoublequoteopen}a{\isasymin}M{\isachardoublequoteclose}\ {\isachardoublequoteopen}b{\isasymin}M{\isachardoublequoteclose}\ \isakeyword{for}\ a\ b\isanewline
\ \ \ \ \isacommand{using}\isamarkupfalse%
\ that\ \isacommand{by}\isamarkupfalse%
\ simp\isanewline
\ \ \isacommand{then}\isamarkupfalse%
\ \isacommand{show}\isamarkupfalse%
\ {\isacharquery}{\kern0pt}thesis\ \isacommand{using}\isamarkupfalse%
\ {\isacartoucheopen}l{\isasymin}M{\isacartoucheclose}\ {\isadigit{1}}\ {\isadigit{2}}\ iterates{\isacharunderscore}{\kern0pt}repl{\isacharunderscore}{\kern0pt}intf\ \isacommand{by}\isamarkupfalse%
\ simp\isanewline
\isacommand{qed}\isamarkupfalse%
%
\endisatagproof
{\isafoldproof}%
%
\isadelimproof
\isanewline
%
\endisadelimproof
\isanewline
\isanewline
\isacommand{lemma}\isamarkupfalse%
\ {\isacharparenleft}{\kern0pt}\isakeyword{in}\ M{\isacharunderscore}{\kern0pt}ZF{\isacharunderscore}{\kern0pt}trans{\isacharparenright}{\kern0pt}\ eclose{\isacharunderscore}{\kern0pt}repl{\isadigit{1}}{\isacharunderscore}{\kern0pt}intf{\isacharcolon}{\kern0pt}\isanewline
\ \ \isakeyword{assumes}\isanewline
\ \ \ \ {\isachardoublequoteopen}A{\isasymin}M{\isachardoublequoteclose}\isanewline
\ \ \isakeyword{shows}\isanewline
\ \ \ \ {\isachardoublequoteopen}iterates{\isacharunderscore}{\kern0pt}replacement{\isacharparenleft}{\kern0pt}{\isacharhash}{\kern0pt}{\isacharhash}{\kern0pt}M{\isacharcomma}{\kern0pt}\ big{\isacharunderscore}{\kern0pt}union{\isacharparenleft}{\kern0pt}{\isacharhash}{\kern0pt}{\isacharhash}{\kern0pt}M{\isacharparenright}{\kern0pt}{\isacharcomma}{\kern0pt}\ A{\isacharparenright}{\kern0pt}{\isachardoublequoteclose}\isanewline
%
\isadelimproof
%
\endisadelimproof
%
\isatagproof
\isacommand{proof}\isamarkupfalse%
\ {\isacharminus}{\kern0pt}\isanewline
\ \ \isacommand{have}\isamarkupfalse%
\ {\isadigit{1}}{\isacharcolon}{\kern0pt}{\isachardoublequoteopen}arity{\isacharparenleft}{\kern0pt}big{\isacharunderscore}{\kern0pt}union{\isacharunderscore}{\kern0pt}fm{\isacharparenleft}{\kern0pt}{\isadigit{1}}{\isacharcomma}{\kern0pt}{\isadigit{0}}{\isacharparenright}{\kern0pt}{\isacharparenright}{\kern0pt}\ {\isacharequal}{\kern0pt}\ {\isadigit{2}}{\isachardoublequoteclose}\isanewline
\ \ \ \ \isacommand{unfolding}\isamarkupfalse%
\ big{\isacharunderscore}{\kern0pt}union{\isacharunderscore}{\kern0pt}fm{\isacharunderscore}{\kern0pt}def\ fm{\isacharunderscore}{\kern0pt}defs\ \isacommand{by}\isamarkupfalse%
\ {\isacharparenleft}{\kern0pt}simp\ add{\isacharcolon}{\kern0pt}nat{\isacharunderscore}{\kern0pt}simp{\isacharunderscore}{\kern0pt}union{\isacharparenright}{\kern0pt}\isanewline
\ \ \isacommand{have}\isamarkupfalse%
\ {\isadigit{2}}{\isacharcolon}{\kern0pt}{\isachardoublequoteopen}big{\isacharunderscore}{\kern0pt}union{\isacharunderscore}{\kern0pt}fm{\isacharparenleft}{\kern0pt}{\isadigit{1}}{\isacharcomma}{\kern0pt}{\isadigit{0}}{\isacharparenright}{\kern0pt}{\isasymin}formula{\isachardoublequoteclose}\ \isacommand{by}\isamarkupfalse%
\ simp\isanewline
\ \ \isacommand{have}\isamarkupfalse%
\ {\isachardoublequoteopen}big{\isacharunderscore}{\kern0pt}union{\isacharparenleft}{\kern0pt}{\isacharhash}{\kern0pt}{\isacharhash}{\kern0pt}M{\isacharcomma}{\kern0pt}a{\isacharcomma}{\kern0pt}b{\isacharparenright}{\kern0pt}\ {\isasymlongleftrightarrow}\ sats{\isacharparenleft}{\kern0pt}M{\isacharcomma}{\kern0pt}\ big{\isacharunderscore}{\kern0pt}union{\isacharunderscore}{\kern0pt}fm{\isacharparenleft}{\kern0pt}{\isadigit{1}}{\isacharcomma}{\kern0pt}{\isadigit{0}}{\isacharparenright}{\kern0pt}{\isacharcomma}{\kern0pt}\ {\isacharbrackleft}{\kern0pt}b{\isacharcomma}{\kern0pt}a{\isacharbrackright}{\kern0pt}{\isacharparenright}{\kern0pt}{\isachardoublequoteclose}\isanewline
\ \ \ \ \isakeyword{if}\ {\isachardoublequoteopen}a{\isasymin}M{\isachardoublequoteclose}\ {\isachardoublequoteopen}b{\isasymin}M{\isachardoublequoteclose}\ \isakeyword{for}\ a\ b\isanewline
\ \ \ \ \isacommand{using}\isamarkupfalse%
\ that\ \isacommand{by}\isamarkupfalse%
\ simp\isanewline
\ \ \isacommand{then}\isamarkupfalse%
\ \isacommand{show}\isamarkupfalse%
\ {\isacharquery}{\kern0pt}thesis\ \isacommand{using}\isamarkupfalse%
\ {\isacartoucheopen}A{\isasymin}M{\isacartoucheclose}\ {\isadigit{1}}\ {\isadigit{2}}\ iterates{\isacharunderscore}{\kern0pt}repl{\isacharunderscore}{\kern0pt}intf\ \isacommand{by}\isamarkupfalse%
\ simp\isanewline
\isacommand{qed}\isamarkupfalse%
%
\endisatagproof
{\isafoldproof}%
%
\isadelimproof
\isanewline
%
\endisadelimproof
\isanewline
\isanewline
\isacommand{lemma}\isamarkupfalse%
\ {\isacharparenleft}{\kern0pt}\isakeyword{in}\ M{\isacharunderscore}{\kern0pt}ZF{\isacharunderscore}{\kern0pt}trans{\isacharparenright}{\kern0pt}\ list{\isacharunderscore}{\kern0pt}repl{\isadigit{2}}{\isacharunderscore}{\kern0pt}intf{\isacharcolon}{\kern0pt}\isanewline
\ \ \isakeyword{assumes}\isanewline
\ \ \ \ {\isachardoublequoteopen}A{\isasymin}M{\isachardoublequoteclose}\isanewline
\ \ \isakeyword{shows}\isanewline
\ \ \ \ {\isachardoublequoteopen}strong{\isacharunderscore}{\kern0pt}replacement{\isacharparenleft}{\kern0pt}{\isacharhash}{\kern0pt}{\isacharhash}{\kern0pt}M{\isacharcomma}{\kern0pt}{\isasymlambda}n\ y{\isachardot}{\kern0pt}\ n{\isasymin}nat\ {\isacharampersand}{\kern0pt}\ is{\isacharunderscore}{\kern0pt}iterates{\isacharparenleft}{\kern0pt}{\isacharhash}{\kern0pt}{\isacharhash}{\kern0pt}M{\isacharcomma}{\kern0pt}\ is{\isacharunderscore}{\kern0pt}list{\isacharunderscore}{\kern0pt}functor{\isacharparenleft}{\kern0pt}{\isacharhash}{\kern0pt}{\isacharhash}{\kern0pt}M{\isacharcomma}{\kern0pt}A{\isacharparenright}{\kern0pt}{\isacharcomma}{\kern0pt}\ {\isadigit{0}}{\isacharcomma}{\kern0pt}\ n{\isacharcomma}{\kern0pt}\ y{\isacharparenright}{\kern0pt}{\isacharparenright}{\kern0pt}{\isachardoublequoteclose}\isanewline
%
\isadelimproof
%
\endisadelimproof
%
\isatagproof
\isacommand{proof}\isamarkupfalse%
\ {\isacharminus}{\kern0pt}\isanewline
\ \ \isacommand{have}\isamarkupfalse%
\ {\isachardoublequoteopen}{\isadigit{0}}{\isasymin}M{\isachardoublequoteclose}\isanewline
\ \ \ \ \isacommand{using}\isamarkupfalse%
\ \ nat{\isacharunderscore}{\kern0pt}{\isadigit{0}}I\ nat{\isacharunderscore}{\kern0pt}trans{\isacharunderscore}{\kern0pt}M\ \isacommand{by}\isamarkupfalse%
\ simp\isanewline
\ \ \isacommand{have}\isamarkupfalse%
\ {\isachardoublequoteopen}is{\isacharunderscore}{\kern0pt}list{\isacharunderscore}{\kern0pt}functor{\isacharparenleft}{\kern0pt}{\isacharhash}{\kern0pt}{\isacharhash}{\kern0pt}M{\isacharcomma}{\kern0pt}A{\isacharcomma}{\kern0pt}a{\isacharcomma}{\kern0pt}b{\isacharparenright}{\kern0pt}\ {\isasymlongleftrightarrow}\isanewline
\ \ \ \ \ \ \ \ sats{\isacharparenleft}{\kern0pt}M{\isacharcomma}{\kern0pt}list{\isacharunderscore}{\kern0pt}functor{\isacharunderscore}{\kern0pt}fm{\isacharparenleft}{\kern0pt}{\isadigit{1}}{\isadigit{3}}{\isacharcomma}{\kern0pt}{\isadigit{1}}{\isacharcomma}{\kern0pt}{\isadigit{0}}{\isacharparenright}{\kern0pt}{\isacharcomma}{\kern0pt}{\isacharbrackleft}{\kern0pt}b{\isacharcomma}{\kern0pt}a{\isacharcomma}{\kern0pt}c{\isacharcomma}{\kern0pt}d{\isacharcomma}{\kern0pt}e{\isacharcomma}{\kern0pt}f{\isacharcomma}{\kern0pt}g{\isacharcomma}{\kern0pt}h{\isacharcomma}{\kern0pt}i{\isacharcomma}{\kern0pt}j{\isacharcomma}{\kern0pt}k{\isacharcomma}{\kern0pt}n{\isacharcomma}{\kern0pt}y{\isacharcomma}{\kern0pt}A{\isacharcomma}{\kern0pt}{\isadigit{0}}{\isacharcomma}{\kern0pt}nat{\isacharbrackright}{\kern0pt}{\isacharparenright}{\kern0pt}{\isachardoublequoteclose}\isanewline
\ \ \ \ \isakeyword{if}\ {\isachardoublequoteopen}a{\isasymin}M{\isachardoublequoteclose}\ {\isachardoublequoteopen}b{\isasymin}M{\isachardoublequoteclose}\ {\isachardoublequoteopen}c{\isasymin}M{\isachardoublequoteclose}\ {\isachardoublequoteopen}d{\isasymin}M{\isachardoublequoteclose}\ {\isachardoublequoteopen}e{\isasymin}M{\isachardoublequoteclose}\ {\isachardoublequoteopen}f{\isasymin}M{\isachardoublequoteclose}{\isachardoublequoteopen}g{\isasymin}M{\isachardoublequoteclose}{\isachardoublequoteopen}h{\isasymin}M{\isachardoublequoteclose}{\isachardoublequoteopen}i{\isasymin}M{\isachardoublequoteclose}{\isachardoublequoteopen}j{\isasymin}M{\isachardoublequoteclose}\ {\isachardoublequoteopen}k{\isasymin}M{\isachardoublequoteclose}\ {\isachardoublequoteopen}n{\isasymin}M{\isachardoublequoteclose}\ {\isachardoublequoteopen}y{\isasymin}M{\isachardoublequoteclose}\isanewline
\ \ \ \ \isakeyword{for}\ a\ b\ c\ d\ e\ f\ g\ h\ i\ j\ k\ n\ y\isanewline
\ \ \ \ \isacommand{using}\isamarkupfalse%
\ that\ {\isacartoucheopen}{\isadigit{0}}{\isasymin}M{\isacartoucheclose}\ nat{\isacharunderscore}{\kern0pt}in{\isacharunderscore}{\kern0pt}M\ {\isacartoucheopen}A{\isasymin}M{\isacartoucheclose}\ \isacommand{by}\isamarkupfalse%
\ simp\isanewline
\ \ \isacommand{then}\isamarkupfalse%
\isanewline
\ \ \isacommand{have}\isamarkupfalse%
\ {\isadigit{1}}{\isacharcolon}{\kern0pt}{\isachardoublequoteopen}sats{\isacharparenleft}{\kern0pt}M{\isacharcomma}{\kern0pt}\ is{\isacharunderscore}{\kern0pt}iterates{\isacharunderscore}{\kern0pt}fm{\isacharparenleft}{\kern0pt}list{\isacharunderscore}{\kern0pt}functor{\isacharunderscore}{\kern0pt}fm{\isacharparenleft}{\kern0pt}{\isadigit{1}}{\isadigit{3}}{\isacharcomma}{\kern0pt}{\isadigit{1}}{\isacharcomma}{\kern0pt}{\isadigit{0}}{\isacharparenright}{\kern0pt}{\isacharcomma}{\kern0pt}{\isadigit{3}}{\isacharcomma}{\kern0pt}{\isadigit{0}}{\isacharcomma}{\kern0pt}{\isadigit{1}}{\isacharparenright}{\kern0pt}{\isacharcomma}{\kern0pt}{\isacharbrackleft}{\kern0pt}n{\isacharcomma}{\kern0pt}y{\isacharcomma}{\kern0pt}A{\isacharcomma}{\kern0pt}{\isadigit{0}}{\isacharcomma}{\kern0pt}nat{\isacharbrackright}{\kern0pt}\ {\isacharparenright}{\kern0pt}\ {\isasymlongleftrightarrow}\isanewline
\ \ \ \ \ \ \ \ \ \ \ is{\isacharunderscore}{\kern0pt}iterates{\isacharparenleft}{\kern0pt}{\isacharhash}{\kern0pt}{\isacharhash}{\kern0pt}M{\isacharcomma}{\kern0pt}\ is{\isacharunderscore}{\kern0pt}list{\isacharunderscore}{\kern0pt}functor{\isacharparenleft}{\kern0pt}{\isacharhash}{\kern0pt}{\isacharhash}{\kern0pt}M{\isacharcomma}{\kern0pt}A{\isacharparenright}{\kern0pt}{\isacharcomma}{\kern0pt}\ {\isadigit{0}}{\isacharcomma}{\kern0pt}\ n\ {\isacharcomma}{\kern0pt}\ y{\isacharparenright}{\kern0pt}{\isachardoublequoteclose}\isanewline
\ \ \ \ \isakeyword{if}\ {\isachardoublequoteopen}n{\isasymin}M{\isachardoublequoteclose}\ {\isachardoublequoteopen}y{\isasymin}M{\isachardoublequoteclose}\ \isakeyword{for}\ n\ y\isanewline
\ \ \ \ \isacommand{using}\isamarkupfalse%
\ that\ {\isacartoucheopen}{\isadigit{0}}{\isasymin}M{\isacartoucheclose}\ {\isacartoucheopen}A{\isasymin}M{\isacartoucheclose}\ nat{\isacharunderscore}{\kern0pt}in{\isacharunderscore}{\kern0pt}M\isanewline
\ \ \ \ \ \ sats{\isacharunderscore}{\kern0pt}is{\isacharunderscore}{\kern0pt}iterates{\isacharunderscore}{\kern0pt}fm{\isacharbrackleft}{\kern0pt}of\ M\ {\isachardoublequoteopen}is{\isacharunderscore}{\kern0pt}list{\isacharunderscore}{\kern0pt}functor{\isacharparenleft}{\kern0pt}{\isacharhash}{\kern0pt}{\isacharhash}{\kern0pt}M{\isacharcomma}{\kern0pt}A{\isacharparenright}{\kern0pt}{\isachardoublequoteclose}{\isacharbrackright}{\kern0pt}\ \isacommand{by}\isamarkupfalse%
\ simp\isanewline
\ \ \isacommand{let}\isamarkupfalse%
\ {\isacharquery}{\kern0pt}f\ {\isacharequal}{\kern0pt}\ {\isachardoublequoteopen}And{\isacharparenleft}{\kern0pt}Member{\isacharparenleft}{\kern0pt}{\isadigit{0}}{\isacharcomma}{\kern0pt}{\isadigit{4}}{\isacharparenright}{\kern0pt}{\isacharcomma}{\kern0pt}is{\isacharunderscore}{\kern0pt}iterates{\isacharunderscore}{\kern0pt}fm{\isacharparenleft}{\kern0pt}list{\isacharunderscore}{\kern0pt}functor{\isacharunderscore}{\kern0pt}fm{\isacharparenleft}{\kern0pt}{\isadigit{1}}{\isadigit{3}}{\isacharcomma}{\kern0pt}{\isadigit{1}}{\isacharcomma}{\kern0pt}{\isadigit{0}}{\isacharparenright}{\kern0pt}{\isacharcomma}{\kern0pt}{\isadigit{3}}{\isacharcomma}{\kern0pt}{\isadigit{0}}{\isacharcomma}{\kern0pt}{\isadigit{1}}{\isacharparenright}{\kern0pt}{\isacharparenright}{\kern0pt}{\isachardoublequoteclose}\isanewline
\ \ \isacommand{have}\isamarkupfalse%
\ satsf{\isacharcolon}{\kern0pt}{\isachardoublequoteopen}sats{\isacharparenleft}{\kern0pt}M{\isacharcomma}{\kern0pt}\ {\isacharquery}{\kern0pt}f{\isacharcomma}{\kern0pt}{\isacharbrackleft}{\kern0pt}n{\isacharcomma}{\kern0pt}y{\isacharcomma}{\kern0pt}A{\isacharcomma}{\kern0pt}{\isadigit{0}}{\isacharcomma}{\kern0pt}nat{\isacharbrackright}{\kern0pt}\ {\isacharparenright}{\kern0pt}\ {\isasymlongleftrightarrow}\isanewline
\ \ \ \ \ \ \ \ n{\isasymin}nat\ {\isacharampersand}{\kern0pt}\ is{\isacharunderscore}{\kern0pt}iterates{\isacharparenleft}{\kern0pt}{\isacharhash}{\kern0pt}{\isacharhash}{\kern0pt}M{\isacharcomma}{\kern0pt}\ is{\isacharunderscore}{\kern0pt}list{\isacharunderscore}{\kern0pt}functor{\isacharparenleft}{\kern0pt}{\isacharhash}{\kern0pt}{\isacharhash}{\kern0pt}M{\isacharcomma}{\kern0pt}A{\isacharparenright}{\kern0pt}{\isacharcomma}{\kern0pt}\ {\isadigit{0}}{\isacharcomma}{\kern0pt}\ n{\isacharcomma}{\kern0pt}\ y{\isacharparenright}{\kern0pt}{\isachardoublequoteclose}\isanewline
\ \ \ \ \isakeyword{if}\ {\isachardoublequoteopen}n{\isasymin}M{\isachardoublequoteclose}\ {\isachardoublequoteopen}y{\isasymin}M{\isachardoublequoteclose}\ \isakeyword{for}\ n\ y\isanewline
\ \ \ \ \isacommand{using}\isamarkupfalse%
\ that\ {\isacartoucheopen}{\isadigit{0}}{\isasymin}M{\isacartoucheclose}\ {\isacartoucheopen}A{\isasymin}M{\isacartoucheclose}\ nat{\isacharunderscore}{\kern0pt}in{\isacharunderscore}{\kern0pt}M\ {\isadigit{1}}\ \isacommand{by}\isamarkupfalse%
\ simp\isanewline
\ \ \isacommand{have}\isamarkupfalse%
\ {\isachardoublequoteopen}arity{\isacharparenleft}{\kern0pt}{\isacharquery}{\kern0pt}f{\isacharparenright}{\kern0pt}\ {\isacharequal}{\kern0pt}\ {\isadigit{5}}{\isachardoublequoteclose}\isanewline
\ \ \ \ \isacommand{unfolding}\isamarkupfalse%
\ is{\isacharunderscore}{\kern0pt}iterates{\isacharunderscore}{\kern0pt}fm{\isacharunderscore}{\kern0pt}def\ restriction{\isacharunderscore}{\kern0pt}fm{\isacharunderscore}{\kern0pt}def\ list{\isacharunderscore}{\kern0pt}functor{\isacharunderscore}{\kern0pt}fm{\isacharunderscore}{\kern0pt}def\ number{\isadigit{1}}{\isacharunderscore}{\kern0pt}fm{\isacharunderscore}{\kern0pt}def\ Memrel{\isacharunderscore}{\kern0pt}fm{\isacharunderscore}{\kern0pt}def\isanewline
\ \ \ \ \ \ cartprod{\isacharunderscore}{\kern0pt}fm{\isacharunderscore}{\kern0pt}def\ sum{\isacharunderscore}{\kern0pt}fm{\isacharunderscore}{\kern0pt}def\ quasinat{\isacharunderscore}{\kern0pt}fm{\isacharunderscore}{\kern0pt}def\ pre{\isacharunderscore}{\kern0pt}image{\isacharunderscore}{\kern0pt}fm{\isacharunderscore}{\kern0pt}def\ fm{\isacharunderscore}{\kern0pt}defs\ is{\isacharunderscore}{\kern0pt}wfrec{\isacharunderscore}{\kern0pt}fm{\isacharunderscore}{\kern0pt}def\isanewline
\ \ \ \ \ \ is{\isacharunderscore}{\kern0pt}recfun{\isacharunderscore}{\kern0pt}fm{\isacharunderscore}{\kern0pt}def\ iterates{\isacharunderscore}{\kern0pt}MH{\isacharunderscore}{\kern0pt}fm{\isacharunderscore}{\kern0pt}def\ is{\isacharunderscore}{\kern0pt}nat{\isacharunderscore}{\kern0pt}case{\isacharunderscore}{\kern0pt}fm{\isacharunderscore}{\kern0pt}def\isanewline
\ \ \ \ \isacommand{by}\isamarkupfalse%
\ {\isacharparenleft}{\kern0pt}simp\ add{\isacharcolon}{\kern0pt}nat{\isacharunderscore}{\kern0pt}simp{\isacharunderscore}{\kern0pt}union{\isacharparenright}{\kern0pt}\isanewline
\ \ \isacommand{then}\isamarkupfalse%
\isanewline
\ \ \isacommand{have}\isamarkupfalse%
\ {\isachardoublequoteopen}strong{\isacharunderscore}{\kern0pt}replacement{\isacharparenleft}{\kern0pt}{\isacharhash}{\kern0pt}{\isacharhash}{\kern0pt}M{\isacharcomma}{\kern0pt}{\isasymlambda}n\ y{\isachardot}{\kern0pt}\ sats{\isacharparenleft}{\kern0pt}M{\isacharcomma}{\kern0pt}{\isacharquery}{\kern0pt}f{\isacharcomma}{\kern0pt}{\isacharbrackleft}{\kern0pt}n{\isacharcomma}{\kern0pt}y{\isacharcomma}{\kern0pt}A{\isacharcomma}{\kern0pt}{\isadigit{0}}{\isacharcomma}{\kern0pt}nat{\isacharbrackright}{\kern0pt}{\isacharparenright}{\kern0pt}{\isacharparenright}{\kern0pt}{\isachardoublequoteclose}\isanewline
\ \ \ \ \isacommand{using}\isamarkupfalse%
\ replacement{\isacharunderscore}{\kern0pt}ax\ {\isadigit{1}}\ nat{\isacharunderscore}{\kern0pt}in{\isacharunderscore}{\kern0pt}M\ {\isacartoucheopen}A{\isasymin}M{\isacartoucheclose}\ {\isacartoucheopen}{\isadigit{0}}{\isasymin}M{\isacartoucheclose}\ \isacommand{by}\isamarkupfalse%
\ simp\isanewline
\ \ \isacommand{then}\isamarkupfalse%
\isanewline
\ \ \isacommand{show}\isamarkupfalse%
\ {\isacharquery}{\kern0pt}thesis\ \isacommand{using}\isamarkupfalse%
\ repl{\isacharunderscore}{\kern0pt}sats{\isacharbrackleft}{\kern0pt}of\ M\ {\isacharquery}{\kern0pt}f\ {\isachardoublequoteopen}{\isacharbrackleft}{\kern0pt}A{\isacharcomma}{\kern0pt}{\isadigit{0}}{\isacharcomma}{\kern0pt}nat{\isacharbrackright}{\kern0pt}{\isachardoublequoteclose}{\isacharbrackright}{\kern0pt}\ \ satsf\ \ \isacommand{by}\isamarkupfalse%
\ simp\isanewline
\isacommand{qed}\isamarkupfalse%
%
\endisatagproof
{\isafoldproof}%
%
\isadelimproof
\isanewline
%
\endisadelimproof
\isanewline
\isacommand{lemma}\isamarkupfalse%
\ {\isacharparenleft}{\kern0pt}\isakeyword{in}\ M{\isacharunderscore}{\kern0pt}ZF{\isacharunderscore}{\kern0pt}trans{\isacharparenright}{\kern0pt}\ formula{\isacharunderscore}{\kern0pt}repl{\isadigit{2}}{\isacharunderscore}{\kern0pt}intf{\isacharcolon}{\kern0pt}\isanewline
\ \ {\isachardoublequoteopen}strong{\isacharunderscore}{\kern0pt}replacement{\isacharparenleft}{\kern0pt}{\isacharhash}{\kern0pt}{\isacharhash}{\kern0pt}M{\isacharcomma}{\kern0pt}{\isasymlambda}n\ y{\isachardot}{\kern0pt}\ n{\isasymin}nat\ {\isacharampersand}{\kern0pt}\ is{\isacharunderscore}{\kern0pt}iterates{\isacharparenleft}{\kern0pt}{\isacharhash}{\kern0pt}{\isacharhash}{\kern0pt}M{\isacharcomma}{\kern0pt}\ is{\isacharunderscore}{\kern0pt}formula{\isacharunderscore}{\kern0pt}functor{\isacharparenleft}{\kern0pt}{\isacharhash}{\kern0pt}{\isacharhash}{\kern0pt}M{\isacharparenright}{\kern0pt}{\isacharcomma}{\kern0pt}\ {\isadigit{0}}{\isacharcomma}{\kern0pt}\ n{\isacharcomma}{\kern0pt}\ y{\isacharparenright}{\kern0pt}{\isacharparenright}{\kern0pt}{\isachardoublequoteclose}\isanewline
%
\isadelimproof
%
\endisadelimproof
%
\isatagproof
\isacommand{proof}\isamarkupfalse%
\ {\isacharminus}{\kern0pt}\isanewline
\ \ \isacommand{have}\isamarkupfalse%
\ {\isachardoublequoteopen}{\isadigit{0}}{\isasymin}M{\isachardoublequoteclose}\isanewline
\ \ \ \ \isacommand{using}\isamarkupfalse%
\ \ nat{\isacharunderscore}{\kern0pt}{\isadigit{0}}I\ nat{\isacharunderscore}{\kern0pt}trans{\isacharunderscore}{\kern0pt}M\ \isacommand{by}\isamarkupfalse%
\ simp\isanewline
\ \ \isacommand{have}\isamarkupfalse%
\ {\isachardoublequoteopen}is{\isacharunderscore}{\kern0pt}formula{\isacharunderscore}{\kern0pt}functor{\isacharparenleft}{\kern0pt}{\isacharhash}{\kern0pt}{\isacharhash}{\kern0pt}M{\isacharcomma}{\kern0pt}a{\isacharcomma}{\kern0pt}b{\isacharparenright}{\kern0pt}\ {\isasymlongleftrightarrow}\isanewline
\ \ \ \ \ \ \ \ sats{\isacharparenleft}{\kern0pt}M{\isacharcomma}{\kern0pt}formula{\isacharunderscore}{\kern0pt}functor{\isacharunderscore}{\kern0pt}fm{\isacharparenleft}{\kern0pt}{\isadigit{1}}{\isacharcomma}{\kern0pt}{\isadigit{0}}{\isacharparenright}{\kern0pt}{\isacharcomma}{\kern0pt}{\isacharbrackleft}{\kern0pt}b{\isacharcomma}{\kern0pt}a{\isacharcomma}{\kern0pt}c{\isacharcomma}{\kern0pt}d{\isacharcomma}{\kern0pt}e{\isacharcomma}{\kern0pt}f{\isacharcomma}{\kern0pt}g{\isacharcomma}{\kern0pt}h{\isacharcomma}{\kern0pt}i{\isacharcomma}{\kern0pt}j{\isacharcomma}{\kern0pt}k{\isacharcomma}{\kern0pt}n{\isacharcomma}{\kern0pt}y{\isacharcomma}{\kern0pt}{\isadigit{0}}{\isacharcomma}{\kern0pt}nat{\isacharbrackright}{\kern0pt}{\isacharparenright}{\kern0pt}{\isachardoublequoteclose}\isanewline
\ \ \ \ \isakeyword{if}\ {\isachardoublequoteopen}a{\isasymin}M{\isachardoublequoteclose}\ {\isachardoublequoteopen}b{\isasymin}M{\isachardoublequoteclose}\ {\isachardoublequoteopen}c{\isasymin}M{\isachardoublequoteclose}\ {\isachardoublequoteopen}d{\isasymin}M{\isachardoublequoteclose}\ {\isachardoublequoteopen}e{\isasymin}M{\isachardoublequoteclose}\ {\isachardoublequoteopen}f{\isasymin}M{\isachardoublequoteclose}{\isachardoublequoteopen}g{\isasymin}M{\isachardoublequoteclose}{\isachardoublequoteopen}h{\isasymin}M{\isachardoublequoteclose}{\isachardoublequoteopen}i{\isasymin}M{\isachardoublequoteclose}{\isachardoublequoteopen}j{\isasymin}M{\isachardoublequoteclose}\ {\isachardoublequoteopen}k{\isasymin}M{\isachardoublequoteclose}\ {\isachardoublequoteopen}n{\isasymin}M{\isachardoublequoteclose}\ {\isachardoublequoteopen}y{\isasymin}M{\isachardoublequoteclose}\isanewline
\ \ \ \ \isakeyword{for}\ a\ b\ c\ d\ e\ f\ g\ h\ i\ j\ k\ n\ y\isanewline
\ \ \ \ \isacommand{using}\isamarkupfalse%
\ that\ {\isacartoucheopen}{\isadigit{0}}{\isasymin}M{\isacartoucheclose}\ nat{\isacharunderscore}{\kern0pt}in{\isacharunderscore}{\kern0pt}M\ \isacommand{by}\isamarkupfalse%
\ simp\isanewline
\ \ \isacommand{then}\isamarkupfalse%
\isanewline
\ \ \isacommand{have}\isamarkupfalse%
\ {\isadigit{1}}{\isacharcolon}{\kern0pt}{\isachardoublequoteopen}sats{\isacharparenleft}{\kern0pt}M{\isacharcomma}{\kern0pt}\ is{\isacharunderscore}{\kern0pt}iterates{\isacharunderscore}{\kern0pt}fm{\isacharparenleft}{\kern0pt}formula{\isacharunderscore}{\kern0pt}functor{\isacharunderscore}{\kern0pt}fm{\isacharparenleft}{\kern0pt}{\isadigit{1}}{\isacharcomma}{\kern0pt}{\isadigit{0}}{\isacharparenright}{\kern0pt}{\isacharcomma}{\kern0pt}{\isadigit{2}}{\isacharcomma}{\kern0pt}{\isadigit{0}}{\isacharcomma}{\kern0pt}{\isadigit{1}}{\isacharparenright}{\kern0pt}{\isacharcomma}{\kern0pt}{\isacharbrackleft}{\kern0pt}n{\isacharcomma}{\kern0pt}y{\isacharcomma}{\kern0pt}{\isadigit{0}}{\isacharcomma}{\kern0pt}nat{\isacharbrackright}{\kern0pt}\ {\isacharparenright}{\kern0pt}\ {\isasymlongleftrightarrow}\isanewline
\ \ \ \ \ \ \ \ \ \ \ is{\isacharunderscore}{\kern0pt}iterates{\isacharparenleft}{\kern0pt}{\isacharhash}{\kern0pt}{\isacharhash}{\kern0pt}M{\isacharcomma}{\kern0pt}\ is{\isacharunderscore}{\kern0pt}formula{\isacharunderscore}{\kern0pt}functor{\isacharparenleft}{\kern0pt}{\isacharhash}{\kern0pt}{\isacharhash}{\kern0pt}M{\isacharparenright}{\kern0pt}{\isacharcomma}{\kern0pt}\ {\isadigit{0}}{\isacharcomma}{\kern0pt}\ n\ {\isacharcomma}{\kern0pt}\ y{\isacharparenright}{\kern0pt}{\isachardoublequoteclose}\isanewline
\ \ \ \ \isakeyword{if}\ {\isachardoublequoteopen}n{\isasymin}M{\isachardoublequoteclose}\ {\isachardoublequoteopen}y{\isasymin}M{\isachardoublequoteclose}\ \isakeyword{for}\ n\ y\isanewline
\ \ \ \ \isacommand{using}\isamarkupfalse%
\ that\ {\isacartoucheopen}{\isadigit{0}}{\isasymin}M{\isacartoucheclose}\ nat{\isacharunderscore}{\kern0pt}in{\isacharunderscore}{\kern0pt}M\isanewline
\ \ \ \ \ \ sats{\isacharunderscore}{\kern0pt}is{\isacharunderscore}{\kern0pt}iterates{\isacharunderscore}{\kern0pt}fm{\isacharbrackleft}{\kern0pt}of\ M\ {\isachardoublequoteopen}is{\isacharunderscore}{\kern0pt}formula{\isacharunderscore}{\kern0pt}functor{\isacharparenleft}{\kern0pt}{\isacharhash}{\kern0pt}{\isacharhash}{\kern0pt}M{\isacharparenright}{\kern0pt}{\isachardoublequoteclose}{\isacharbrackright}{\kern0pt}\ \isacommand{by}\isamarkupfalse%
\ simp\isanewline
\ \ \isacommand{let}\isamarkupfalse%
\ {\isacharquery}{\kern0pt}f\ {\isacharequal}{\kern0pt}\ {\isachardoublequoteopen}And{\isacharparenleft}{\kern0pt}Member{\isacharparenleft}{\kern0pt}{\isadigit{0}}{\isacharcomma}{\kern0pt}{\isadigit{3}}{\isacharparenright}{\kern0pt}{\isacharcomma}{\kern0pt}is{\isacharunderscore}{\kern0pt}iterates{\isacharunderscore}{\kern0pt}fm{\isacharparenleft}{\kern0pt}formula{\isacharunderscore}{\kern0pt}functor{\isacharunderscore}{\kern0pt}fm{\isacharparenleft}{\kern0pt}{\isadigit{1}}{\isacharcomma}{\kern0pt}{\isadigit{0}}{\isacharparenright}{\kern0pt}{\isacharcomma}{\kern0pt}{\isadigit{2}}{\isacharcomma}{\kern0pt}{\isadigit{0}}{\isacharcomma}{\kern0pt}{\isadigit{1}}{\isacharparenright}{\kern0pt}{\isacharparenright}{\kern0pt}{\isachardoublequoteclose}\isanewline
\ \ \isacommand{have}\isamarkupfalse%
\ satsf{\isacharcolon}{\kern0pt}{\isachardoublequoteopen}sats{\isacharparenleft}{\kern0pt}M{\isacharcomma}{\kern0pt}\ {\isacharquery}{\kern0pt}f{\isacharcomma}{\kern0pt}{\isacharbrackleft}{\kern0pt}n{\isacharcomma}{\kern0pt}y{\isacharcomma}{\kern0pt}{\isadigit{0}}{\isacharcomma}{\kern0pt}nat{\isacharbrackright}{\kern0pt}\ {\isacharparenright}{\kern0pt}\ {\isasymlongleftrightarrow}\isanewline
\ \ \ \ \ \ \ \ n{\isasymin}nat\ {\isacharampersand}{\kern0pt}\ is{\isacharunderscore}{\kern0pt}iterates{\isacharparenleft}{\kern0pt}{\isacharhash}{\kern0pt}{\isacharhash}{\kern0pt}M{\isacharcomma}{\kern0pt}\ is{\isacharunderscore}{\kern0pt}formula{\isacharunderscore}{\kern0pt}functor{\isacharparenleft}{\kern0pt}{\isacharhash}{\kern0pt}{\isacharhash}{\kern0pt}M{\isacharparenright}{\kern0pt}{\isacharcomma}{\kern0pt}\ {\isadigit{0}}{\isacharcomma}{\kern0pt}\ n{\isacharcomma}{\kern0pt}\ y{\isacharparenright}{\kern0pt}{\isachardoublequoteclose}\isanewline
\ \ \ \ \isakeyword{if}\ {\isachardoublequoteopen}n{\isasymin}M{\isachardoublequoteclose}\ {\isachardoublequoteopen}y{\isasymin}M{\isachardoublequoteclose}\ \isakeyword{for}\ n\ y\isanewline
\ \ \ \ \isacommand{using}\isamarkupfalse%
\ that\ {\isacartoucheopen}{\isadigit{0}}{\isasymin}M{\isacartoucheclose}\ nat{\isacharunderscore}{\kern0pt}in{\isacharunderscore}{\kern0pt}M\ {\isadigit{1}}\ \isacommand{by}\isamarkupfalse%
\ simp\isanewline
\ \ \isacommand{have}\isamarkupfalse%
\ artyf{\isacharcolon}{\kern0pt}{\isachardoublequoteopen}arity{\isacharparenleft}{\kern0pt}{\isacharquery}{\kern0pt}f{\isacharparenright}{\kern0pt}\ {\isacharequal}{\kern0pt}\ {\isadigit{4}}{\isachardoublequoteclose}\isanewline
\ \ \ \ \isacommand{unfolding}\isamarkupfalse%
\ is{\isacharunderscore}{\kern0pt}iterates{\isacharunderscore}{\kern0pt}fm{\isacharunderscore}{\kern0pt}def\ formula{\isacharunderscore}{\kern0pt}functor{\isacharunderscore}{\kern0pt}fm{\isacharunderscore}{\kern0pt}def\ fm{\isacharunderscore}{\kern0pt}defs\ sum{\isacharunderscore}{\kern0pt}fm{\isacharunderscore}{\kern0pt}def\ quasinat{\isacharunderscore}{\kern0pt}fm{\isacharunderscore}{\kern0pt}def\isanewline
\ \ \ \ \ \ cartprod{\isacharunderscore}{\kern0pt}fm{\isacharunderscore}{\kern0pt}def\ number{\isadigit{1}}{\isacharunderscore}{\kern0pt}fm{\isacharunderscore}{\kern0pt}def\ Memrel{\isacharunderscore}{\kern0pt}fm{\isacharunderscore}{\kern0pt}def\ ordinal{\isacharunderscore}{\kern0pt}fm{\isacharunderscore}{\kern0pt}def\ transset{\isacharunderscore}{\kern0pt}fm{\isacharunderscore}{\kern0pt}def\isanewline
\ \ \ \ \ \ is{\isacharunderscore}{\kern0pt}wfrec{\isacharunderscore}{\kern0pt}fm{\isacharunderscore}{\kern0pt}def\ is{\isacharunderscore}{\kern0pt}recfun{\isacharunderscore}{\kern0pt}fm{\isacharunderscore}{\kern0pt}def\ iterates{\isacharunderscore}{\kern0pt}MH{\isacharunderscore}{\kern0pt}fm{\isacharunderscore}{\kern0pt}def\ is{\isacharunderscore}{\kern0pt}nat{\isacharunderscore}{\kern0pt}case{\isacharunderscore}{\kern0pt}fm{\isacharunderscore}{\kern0pt}def\ subset{\isacharunderscore}{\kern0pt}fm{\isacharunderscore}{\kern0pt}def\isanewline
\ \ \ \ \ \ pre{\isacharunderscore}{\kern0pt}image{\isacharunderscore}{\kern0pt}fm{\isacharunderscore}{\kern0pt}def\ restriction{\isacharunderscore}{\kern0pt}fm{\isacharunderscore}{\kern0pt}def\isanewline
\ \ \ \ \isacommand{by}\isamarkupfalse%
\ {\isacharparenleft}{\kern0pt}simp\ add{\isacharcolon}{\kern0pt}nat{\isacharunderscore}{\kern0pt}simp{\isacharunderscore}{\kern0pt}union{\isacharparenright}{\kern0pt}\isanewline
\ \ \isacommand{then}\isamarkupfalse%
\isanewline
\ \ \isacommand{have}\isamarkupfalse%
\ {\isachardoublequoteopen}strong{\isacharunderscore}{\kern0pt}replacement{\isacharparenleft}{\kern0pt}{\isacharhash}{\kern0pt}{\isacharhash}{\kern0pt}M{\isacharcomma}{\kern0pt}{\isasymlambda}n\ y{\isachardot}{\kern0pt}\ sats{\isacharparenleft}{\kern0pt}M{\isacharcomma}{\kern0pt}{\isacharquery}{\kern0pt}f{\isacharcomma}{\kern0pt}{\isacharbrackleft}{\kern0pt}n{\isacharcomma}{\kern0pt}y{\isacharcomma}{\kern0pt}{\isadigit{0}}{\isacharcomma}{\kern0pt}nat{\isacharbrackright}{\kern0pt}{\isacharparenright}{\kern0pt}{\isacharparenright}{\kern0pt}{\isachardoublequoteclose}\isanewline
\ \ \ \ \isacommand{using}\isamarkupfalse%
\ replacement{\isacharunderscore}{\kern0pt}ax\ {\isadigit{1}}\ artyf\ {\isacartoucheopen}{\isadigit{0}}{\isasymin}M{\isacartoucheclose}\ nat{\isacharunderscore}{\kern0pt}in{\isacharunderscore}{\kern0pt}M\ \isacommand{by}\isamarkupfalse%
\ simp\isanewline
\ \ \isacommand{then}\isamarkupfalse%
\isanewline
\ \ \isacommand{show}\isamarkupfalse%
\ {\isacharquery}{\kern0pt}thesis\ \isacommand{using}\isamarkupfalse%
\ repl{\isacharunderscore}{\kern0pt}sats{\isacharbrackleft}{\kern0pt}of\ M\ {\isacharquery}{\kern0pt}f\ {\isachardoublequoteopen}{\isacharbrackleft}{\kern0pt}{\isadigit{0}}{\isacharcomma}{\kern0pt}nat{\isacharbrackright}{\kern0pt}{\isachardoublequoteclose}{\isacharbrackright}{\kern0pt}\ \ satsf\ \ \isacommand{by}\isamarkupfalse%
\ simp\isanewline
\isacommand{qed}\isamarkupfalse%
%
\endisatagproof
{\isafoldproof}%
%
\isadelimproof
\isanewline
%
\endisadelimproof
\isanewline
\isanewline
\isanewline
\isanewline
\isacommand{lemma}\isamarkupfalse%
\ {\isacharparenleft}{\kern0pt}\isakeyword{in}\ M{\isacharunderscore}{\kern0pt}ZF{\isacharunderscore}{\kern0pt}trans{\isacharparenright}{\kern0pt}\ eclose{\isacharunderscore}{\kern0pt}repl{\isadigit{2}}{\isacharunderscore}{\kern0pt}intf{\isacharcolon}{\kern0pt}\isanewline
\ \ \isakeyword{assumes}\isanewline
\ \ \ \ {\isachardoublequoteopen}A{\isasymin}M{\isachardoublequoteclose}\isanewline
\ \ \isakeyword{shows}\isanewline
\ \ \ \ {\isachardoublequoteopen}strong{\isacharunderscore}{\kern0pt}replacement{\isacharparenleft}{\kern0pt}{\isacharhash}{\kern0pt}{\isacharhash}{\kern0pt}M{\isacharcomma}{\kern0pt}{\isasymlambda}n\ y{\isachardot}{\kern0pt}\ n{\isasymin}nat\ {\isacharampersand}{\kern0pt}\ is{\isacharunderscore}{\kern0pt}iterates{\isacharparenleft}{\kern0pt}{\isacharhash}{\kern0pt}{\isacharhash}{\kern0pt}M{\isacharcomma}{\kern0pt}\ big{\isacharunderscore}{\kern0pt}union{\isacharparenleft}{\kern0pt}{\isacharhash}{\kern0pt}{\isacharhash}{\kern0pt}M{\isacharparenright}{\kern0pt}{\isacharcomma}{\kern0pt}\ A{\isacharcomma}{\kern0pt}\ n{\isacharcomma}{\kern0pt}\ y{\isacharparenright}{\kern0pt}{\isacharparenright}{\kern0pt}{\isachardoublequoteclose}\isanewline
%
\isadelimproof
%
\endisadelimproof
%
\isatagproof
\isacommand{proof}\isamarkupfalse%
\ {\isacharminus}{\kern0pt}\isanewline
\ \ \isacommand{have}\isamarkupfalse%
\ {\isachardoublequoteopen}big{\isacharunderscore}{\kern0pt}union{\isacharparenleft}{\kern0pt}{\isacharhash}{\kern0pt}{\isacharhash}{\kern0pt}M{\isacharcomma}{\kern0pt}a{\isacharcomma}{\kern0pt}b{\isacharparenright}{\kern0pt}\ {\isasymlongleftrightarrow}\isanewline
\ \ \ \ \ \ \ \ sats{\isacharparenleft}{\kern0pt}M{\isacharcomma}{\kern0pt}big{\isacharunderscore}{\kern0pt}union{\isacharunderscore}{\kern0pt}fm{\isacharparenleft}{\kern0pt}{\isadigit{1}}{\isacharcomma}{\kern0pt}{\isadigit{0}}{\isacharparenright}{\kern0pt}{\isacharcomma}{\kern0pt}{\isacharbrackleft}{\kern0pt}b{\isacharcomma}{\kern0pt}a{\isacharcomma}{\kern0pt}c{\isacharcomma}{\kern0pt}d{\isacharcomma}{\kern0pt}e{\isacharcomma}{\kern0pt}f{\isacharcomma}{\kern0pt}g{\isacharcomma}{\kern0pt}h{\isacharcomma}{\kern0pt}i{\isacharcomma}{\kern0pt}j{\isacharcomma}{\kern0pt}k{\isacharcomma}{\kern0pt}n{\isacharcomma}{\kern0pt}y{\isacharcomma}{\kern0pt}A{\isacharcomma}{\kern0pt}nat{\isacharbrackright}{\kern0pt}{\isacharparenright}{\kern0pt}{\isachardoublequoteclose}\isanewline
\ \ \ \ \isakeyword{if}\ {\isachardoublequoteopen}a{\isasymin}M{\isachardoublequoteclose}\ {\isachardoublequoteopen}b{\isasymin}M{\isachardoublequoteclose}\ {\isachardoublequoteopen}c{\isasymin}M{\isachardoublequoteclose}\ {\isachardoublequoteopen}d{\isasymin}M{\isachardoublequoteclose}\ {\isachardoublequoteopen}e{\isasymin}M{\isachardoublequoteclose}\ {\isachardoublequoteopen}f{\isasymin}M{\isachardoublequoteclose}{\isachardoublequoteopen}g{\isasymin}M{\isachardoublequoteclose}{\isachardoublequoteopen}h{\isasymin}M{\isachardoublequoteclose}{\isachardoublequoteopen}i{\isasymin}M{\isachardoublequoteclose}{\isachardoublequoteopen}j{\isasymin}M{\isachardoublequoteclose}\ {\isachardoublequoteopen}k{\isasymin}M{\isachardoublequoteclose}\ {\isachardoublequoteopen}n{\isasymin}M{\isachardoublequoteclose}\ {\isachardoublequoteopen}y{\isasymin}M{\isachardoublequoteclose}\isanewline
\ \ \ \ \isakeyword{for}\ a\ b\ c\ d\ e\ f\ g\ h\ i\ j\ k\ n\ y\isanewline
\ \ \ \ \isacommand{using}\isamarkupfalse%
\ that\ {\isacartoucheopen}A{\isasymin}M{\isacartoucheclose}\ nat{\isacharunderscore}{\kern0pt}in{\isacharunderscore}{\kern0pt}M\ \isacommand{by}\isamarkupfalse%
\ simp\isanewline
\ \ \isacommand{then}\isamarkupfalse%
\isanewline
\ \ \isacommand{have}\isamarkupfalse%
\ {\isadigit{1}}{\isacharcolon}{\kern0pt}{\isachardoublequoteopen}sats{\isacharparenleft}{\kern0pt}M{\isacharcomma}{\kern0pt}\ is{\isacharunderscore}{\kern0pt}iterates{\isacharunderscore}{\kern0pt}fm{\isacharparenleft}{\kern0pt}big{\isacharunderscore}{\kern0pt}union{\isacharunderscore}{\kern0pt}fm{\isacharparenleft}{\kern0pt}{\isadigit{1}}{\isacharcomma}{\kern0pt}{\isadigit{0}}{\isacharparenright}{\kern0pt}{\isacharcomma}{\kern0pt}{\isadigit{2}}{\isacharcomma}{\kern0pt}{\isadigit{0}}{\isacharcomma}{\kern0pt}{\isadigit{1}}{\isacharparenright}{\kern0pt}{\isacharcomma}{\kern0pt}{\isacharbrackleft}{\kern0pt}n{\isacharcomma}{\kern0pt}y{\isacharcomma}{\kern0pt}A{\isacharcomma}{\kern0pt}nat{\isacharbrackright}{\kern0pt}\ {\isacharparenright}{\kern0pt}\ {\isasymlongleftrightarrow}\isanewline
\ \ \ \ \ \ \ \ \ \ \ is{\isacharunderscore}{\kern0pt}iterates{\isacharparenleft}{\kern0pt}{\isacharhash}{\kern0pt}{\isacharhash}{\kern0pt}M{\isacharcomma}{\kern0pt}\ big{\isacharunderscore}{\kern0pt}union{\isacharparenleft}{\kern0pt}{\isacharhash}{\kern0pt}{\isacharhash}{\kern0pt}M{\isacharparenright}{\kern0pt}{\isacharcomma}{\kern0pt}\ A{\isacharcomma}{\kern0pt}\ n\ {\isacharcomma}{\kern0pt}\ y{\isacharparenright}{\kern0pt}{\isachardoublequoteclose}\isanewline
\ \ \ \ \isakeyword{if}\ {\isachardoublequoteopen}n{\isasymin}M{\isachardoublequoteclose}\ {\isachardoublequoteopen}y{\isasymin}M{\isachardoublequoteclose}\ \isakeyword{for}\ n\ y\isanewline
\ \ \ \ \isacommand{using}\isamarkupfalse%
\ that\ {\isacartoucheopen}A{\isasymin}M{\isacartoucheclose}\ nat{\isacharunderscore}{\kern0pt}in{\isacharunderscore}{\kern0pt}M\isanewline
\ \ \ \ \ \ sats{\isacharunderscore}{\kern0pt}is{\isacharunderscore}{\kern0pt}iterates{\isacharunderscore}{\kern0pt}fm{\isacharbrackleft}{\kern0pt}of\ M\ {\isachardoublequoteopen}big{\isacharunderscore}{\kern0pt}union{\isacharparenleft}{\kern0pt}{\isacharhash}{\kern0pt}{\isacharhash}{\kern0pt}M{\isacharparenright}{\kern0pt}{\isachardoublequoteclose}{\isacharbrackright}{\kern0pt}\ \isacommand{by}\isamarkupfalse%
\ simp\isanewline
\ \ \isacommand{let}\isamarkupfalse%
\ {\isacharquery}{\kern0pt}f\ {\isacharequal}{\kern0pt}\ {\isachardoublequoteopen}And{\isacharparenleft}{\kern0pt}Member{\isacharparenleft}{\kern0pt}{\isadigit{0}}{\isacharcomma}{\kern0pt}{\isadigit{3}}{\isacharparenright}{\kern0pt}{\isacharcomma}{\kern0pt}is{\isacharunderscore}{\kern0pt}iterates{\isacharunderscore}{\kern0pt}fm{\isacharparenleft}{\kern0pt}big{\isacharunderscore}{\kern0pt}union{\isacharunderscore}{\kern0pt}fm{\isacharparenleft}{\kern0pt}{\isadigit{1}}{\isacharcomma}{\kern0pt}{\isadigit{0}}{\isacharparenright}{\kern0pt}{\isacharcomma}{\kern0pt}{\isadigit{2}}{\isacharcomma}{\kern0pt}{\isadigit{0}}{\isacharcomma}{\kern0pt}{\isadigit{1}}{\isacharparenright}{\kern0pt}{\isacharparenright}{\kern0pt}{\isachardoublequoteclose}\isanewline
\ \ \isacommand{have}\isamarkupfalse%
\ satsf{\isacharcolon}{\kern0pt}{\isachardoublequoteopen}sats{\isacharparenleft}{\kern0pt}M{\isacharcomma}{\kern0pt}\ {\isacharquery}{\kern0pt}f{\isacharcomma}{\kern0pt}{\isacharbrackleft}{\kern0pt}n{\isacharcomma}{\kern0pt}y{\isacharcomma}{\kern0pt}A{\isacharcomma}{\kern0pt}nat{\isacharbrackright}{\kern0pt}\ {\isacharparenright}{\kern0pt}\ {\isasymlongleftrightarrow}\isanewline
\ \ \ \ \ \ \ \ n{\isasymin}nat\ {\isacharampersand}{\kern0pt}\ is{\isacharunderscore}{\kern0pt}iterates{\isacharparenleft}{\kern0pt}{\isacharhash}{\kern0pt}{\isacharhash}{\kern0pt}M{\isacharcomma}{\kern0pt}\ big{\isacharunderscore}{\kern0pt}union{\isacharparenleft}{\kern0pt}{\isacharhash}{\kern0pt}{\isacharhash}{\kern0pt}M{\isacharparenright}{\kern0pt}{\isacharcomma}{\kern0pt}\ A{\isacharcomma}{\kern0pt}\ n{\isacharcomma}{\kern0pt}\ y{\isacharparenright}{\kern0pt}{\isachardoublequoteclose}\isanewline
\ \ \ \ \isakeyword{if}\ {\isachardoublequoteopen}n{\isasymin}M{\isachardoublequoteclose}\ {\isachardoublequoteopen}y{\isasymin}M{\isachardoublequoteclose}\ \isakeyword{for}\ n\ y\isanewline
\ \ \ \ \isacommand{using}\isamarkupfalse%
\ that\ {\isacartoucheopen}A{\isasymin}M{\isacartoucheclose}\ nat{\isacharunderscore}{\kern0pt}in{\isacharunderscore}{\kern0pt}M\ {\isadigit{1}}\ \isacommand{by}\isamarkupfalse%
\ simp\isanewline
\ \ \isacommand{have}\isamarkupfalse%
\ artyf{\isacharcolon}{\kern0pt}{\isachardoublequoteopen}arity{\isacharparenleft}{\kern0pt}{\isacharquery}{\kern0pt}f{\isacharparenright}{\kern0pt}\ {\isacharequal}{\kern0pt}\ {\isadigit{4}}{\isachardoublequoteclose}\isanewline
\ \ \ \ \isacommand{unfolding}\isamarkupfalse%
\ is{\isacharunderscore}{\kern0pt}iterates{\isacharunderscore}{\kern0pt}fm{\isacharunderscore}{\kern0pt}def\ formula{\isacharunderscore}{\kern0pt}functor{\isacharunderscore}{\kern0pt}fm{\isacharunderscore}{\kern0pt}def\ fm{\isacharunderscore}{\kern0pt}defs\ sum{\isacharunderscore}{\kern0pt}fm{\isacharunderscore}{\kern0pt}def\ quasinat{\isacharunderscore}{\kern0pt}fm{\isacharunderscore}{\kern0pt}def\isanewline
\ \ \ \ \ \ cartprod{\isacharunderscore}{\kern0pt}fm{\isacharunderscore}{\kern0pt}def\ number{\isadigit{1}}{\isacharunderscore}{\kern0pt}fm{\isacharunderscore}{\kern0pt}def\ Memrel{\isacharunderscore}{\kern0pt}fm{\isacharunderscore}{\kern0pt}def\ ordinal{\isacharunderscore}{\kern0pt}fm{\isacharunderscore}{\kern0pt}def\ transset{\isacharunderscore}{\kern0pt}fm{\isacharunderscore}{\kern0pt}def\isanewline
\ \ \ \ \ \ is{\isacharunderscore}{\kern0pt}wfrec{\isacharunderscore}{\kern0pt}fm{\isacharunderscore}{\kern0pt}def\ is{\isacharunderscore}{\kern0pt}recfun{\isacharunderscore}{\kern0pt}fm{\isacharunderscore}{\kern0pt}def\ iterates{\isacharunderscore}{\kern0pt}MH{\isacharunderscore}{\kern0pt}fm{\isacharunderscore}{\kern0pt}def\ is{\isacharunderscore}{\kern0pt}nat{\isacharunderscore}{\kern0pt}case{\isacharunderscore}{\kern0pt}fm{\isacharunderscore}{\kern0pt}def\ subset{\isacharunderscore}{\kern0pt}fm{\isacharunderscore}{\kern0pt}def\isanewline
\ \ \ \ \ \ pre{\isacharunderscore}{\kern0pt}image{\isacharunderscore}{\kern0pt}fm{\isacharunderscore}{\kern0pt}def\ restriction{\isacharunderscore}{\kern0pt}fm{\isacharunderscore}{\kern0pt}def\isanewline
\ \ \ \ \isacommand{by}\isamarkupfalse%
\ {\isacharparenleft}{\kern0pt}simp\ add{\isacharcolon}{\kern0pt}nat{\isacharunderscore}{\kern0pt}simp{\isacharunderscore}{\kern0pt}union{\isacharparenright}{\kern0pt}\isanewline
\ \ \isacommand{then}\isamarkupfalse%
\isanewline
\ \ \isacommand{have}\isamarkupfalse%
\ {\isachardoublequoteopen}strong{\isacharunderscore}{\kern0pt}replacement{\isacharparenleft}{\kern0pt}{\isacharhash}{\kern0pt}{\isacharhash}{\kern0pt}M{\isacharcomma}{\kern0pt}{\isasymlambda}n\ y{\isachardot}{\kern0pt}\ sats{\isacharparenleft}{\kern0pt}M{\isacharcomma}{\kern0pt}{\isacharquery}{\kern0pt}f{\isacharcomma}{\kern0pt}{\isacharbrackleft}{\kern0pt}n{\isacharcomma}{\kern0pt}y{\isacharcomma}{\kern0pt}A{\isacharcomma}{\kern0pt}nat{\isacharbrackright}{\kern0pt}{\isacharparenright}{\kern0pt}{\isacharparenright}{\kern0pt}{\isachardoublequoteclose}\isanewline
\ \ \ \ \isacommand{using}\isamarkupfalse%
\ replacement{\isacharunderscore}{\kern0pt}ax\ {\isadigit{1}}\ artyf\ {\isacartoucheopen}A{\isasymin}M{\isacartoucheclose}\ nat{\isacharunderscore}{\kern0pt}in{\isacharunderscore}{\kern0pt}M\ \isacommand{by}\isamarkupfalse%
\ simp\isanewline
\ \ \isacommand{then}\isamarkupfalse%
\isanewline
\ \ \isacommand{show}\isamarkupfalse%
\ {\isacharquery}{\kern0pt}thesis\ \isacommand{using}\isamarkupfalse%
\ repl{\isacharunderscore}{\kern0pt}sats{\isacharbrackleft}{\kern0pt}of\ M\ {\isacharquery}{\kern0pt}f\ {\isachardoublequoteopen}{\isacharbrackleft}{\kern0pt}A{\isacharcomma}{\kern0pt}nat{\isacharbrackright}{\kern0pt}{\isachardoublequoteclose}{\isacharbrackright}{\kern0pt}\ \ satsf\ \ \isacommand{by}\isamarkupfalse%
\ simp\isanewline
\isacommand{qed}\isamarkupfalse%
%
\endisatagproof
{\isafoldproof}%
%
\isadelimproof
\isanewline
%
\endisadelimproof
\isanewline
\isacommand{lemma}\isamarkupfalse%
\ {\isacharparenleft}{\kern0pt}\isakeyword{in}\ M{\isacharunderscore}{\kern0pt}ZF{\isacharunderscore}{\kern0pt}trans{\isacharparenright}{\kern0pt}\ mdatatypes\ {\isacharcolon}{\kern0pt}\ {\isachardoublequoteopen}M{\isacharunderscore}{\kern0pt}datatypes{\isacharparenleft}{\kern0pt}{\isacharhash}{\kern0pt}{\isacharhash}{\kern0pt}M{\isacharparenright}{\kern0pt}{\isachardoublequoteclose}\isanewline
%
\isadelimproof
\ \ %
\endisadelimproof
%
\isatagproof
\isacommand{using}\isamarkupfalse%
\ \ mtrancl\ list{\isacharunderscore}{\kern0pt}repl{\isadigit{1}}{\isacharunderscore}{\kern0pt}intf\ list{\isacharunderscore}{\kern0pt}repl{\isadigit{2}}{\isacharunderscore}{\kern0pt}intf\ formula{\isacharunderscore}{\kern0pt}repl{\isadigit{1}}{\isacharunderscore}{\kern0pt}intf\isanewline
\ \ \ \ formula{\isacharunderscore}{\kern0pt}repl{\isadigit{2}}{\isacharunderscore}{\kern0pt}intf\ nth{\isacharunderscore}{\kern0pt}repl{\isacharunderscore}{\kern0pt}intf\isanewline
\ \ \isacommand{by}\isamarkupfalse%
\ unfold{\isacharunderscore}{\kern0pt}locales\ auto%
\endisatagproof
{\isafoldproof}%
%
\isadelimproof
\isanewline
%
\endisadelimproof
\isanewline
\isacommand{sublocale}\isamarkupfalse%
\ M{\isacharunderscore}{\kern0pt}ZF{\isacharunderscore}{\kern0pt}trans\ {\isasymsubseteq}\ M{\isacharunderscore}{\kern0pt}datatypes\ {\isachardoublequoteopen}{\isacharhash}{\kern0pt}{\isacharhash}{\kern0pt}M{\isachardoublequoteclose}\isanewline
%
\isadelimproof
\ \ %
\endisadelimproof
%
\isatagproof
\isacommand{by}\isamarkupfalse%
\ {\isacharparenleft}{\kern0pt}rule\ mdatatypes{\isacharparenright}{\kern0pt}%
\endisatagproof
{\isafoldproof}%
%
\isadelimproof
\isanewline
%
\endisadelimproof
\isanewline
\isacommand{lemma}\isamarkupfalse%
\ {\isacharparenleft}{\kern0pt}\isakeyword{in}\ M{\isacharunderscore}{\kern0pt}ZF{\isacharunderscore}{\kern0pt}trans{\isacharparenright}{\kern0pt}\ meclose\ {\isacharcolon}{\kern0pt}\ {\isachardoublequoteopen}M{\isacharunderscore}{\kern0pt}eclose{\isacharparenleft}{\kern0pt}{\isacharhash}{\kern0pt}{\isacharhash}{\kern0pt}M{\isacharparenright}{\kern0pt}{\isachardoublequoteclose}\isanewline
%
\isadelimproof
\ \ %
\endisadelimproof
%
\isatagproof
\isacommand{using}\isamarkupfalse%
\ mdatatypes\ eclose{\isacharunderscore}{\kern0pt}repl{\isadigit{1}}{\isacharunderscore}{\kern0pt}intf\ eclose{\isacharunderscore}{\kern0pt}repl{\isadigit{2}}{\isacharunderscore}{\kern0pt}intf\isanewline
\ \ \isacommand{by}\isamarkupfalse%
\ unfold{\isacharunderscore}{\kern0pt}locales\ auto%
\endisatagproof
{\isafoldproof}%
%
\isadelimproof
\isanewline
%
\endisadelimproof
\isanewline
\isacommand{sublocale}\isamarkupfalse%
\ M{\isacharunderscore}{\kern0pt}ZF{\isacharunderscore}{\kern0pt}trans\ {\isasymsubseteq}\ M{\isacharunderscore}{\kern0pt}eclose\ {\isachardoublequoteopen}{\isacharhash}{\kern0pt}{\isacharhash}{\kern0pt}M{\isachardoublequoteclose}\isanewline
%
\isadelimproof
\ \ %
\endisadelimproof
%
\isatagproof
\isacommand{by}\isamarkupfalse%
\ {\isacharparenleft}{\kern0pt}rule\ meclose{\isacharparenright}{\kern0pt}%
\endisatagproof
{\isafoldproof}%
%
\isadelimproof
\isanewline
%
\endisadelimproof
\isanewline
\isanewline
\isanewline
\isanewline
\isacommand{definition}\isamarkupfalse%
\isanewline
\ \ powerset{\isacharunderscore}{\kern0pt}fm\ {\isacharcolon}{\kern0pt}{\isacharcolon}{\kern0pt}\ {\isachardoublequoteopen}{\isacharbrackleft}{\kern0pt}i{\isacharcomma}{\kern0pt}i{\isacharbrackright}{\kern0pt}\ {\isasymRightarrow}\ i{\isachardoublequoteclose}\ \isakeyword{where}\isanewline
\ \ {\isachardoublequoteopen}powerset{\isacharunderscore}{\kern0pt}fm{\isacharparenleft}{\kern0pt}A{\isacharcomma}{\kern0pt}z{\isacharparenright}{\kern0pt}\ {\isasymequiv}\ Forall{\isacharparenleft}{\kern0pt}Iff{\isacharparenleft}{\kern0pt}Member{\isacharparenleft}{\kern0pt}{\isadigit{0}}{\isacharcomma}{\kern0pt}succ{\isacharparenleft}{\kern0pt}z{\isacharparenright}{\kern0pt}{\isacharparenright}{\kern0pt}{\isacharcomma}{\kern0pt}subset{\isacharunderscore}{\kern0pt}fm{\isacharparenleft}{\kern0pt}{\isadigit{0}}{\isacharcomma}{\kern0pt}succ{\isacharparenleft}{\kern0pt}A{\isacharparenright}{\kern0pt}{\isacharparenright}{\kern0pt}{\isacharparenright}{\kern0pt}{\isacharparenright}{\kern0pt}{\isachardoublequoteclose}\isanewline
\isanewline
\isacommand{lemma}\isamarkupfalse%
\ powerset{\isacharunderscore}{\kern0pt}type\ {\isacharbrackleft}{\kern0pt}TC{\isacharbrackright}{\kern0pt}{\isacharcolon}{\kern0pt}\isanewline
\ \ {\isachardoublequoteopen}{\isasymlbrakk}\ x\ {\isasymin}\ nat{\isacharsemicolon}{\kern0pt}\ y\ {\isasymin}\ nat\ {\isasymrbrakk}\ {\isasymLongrightarrow}\ powerset{\isacharunderscore}{\kern0pt}fm{\isacharparenleft}{\kern0pt}x{\isacharcomma}{\kern0pt}y{\isacharparenright}{\kern0pt}\ {\isasymin}\ formula{\isachardoublequoteclose}\isanewline
%
\isadelimproof
\ \ %
\endisadelimproof
%
\isatagproof
\isacommand{by}\isamarkupfalse%
\ {\isacharparenleft}{\kern0pt}simp\ add{\isacharcolon}{\kern0pt}powerset{\isacharunderscore}{\kern0pt}fm{\isacharunderscore}{\kern0pt}def{\isacharparenright}{\kern0pt}%
\endisatagproof
{\isafoldproof}%
%
\isadelimproof
\isanewline
%
\endisadelimproof
\isanewline
\isacommand{definition}\isamarkupfalse%
\isanewline
\ \ is{\isacharunderscore}{\kern0pt}powapply{\isacharunderscore}{\kern0pt}fm\ {\isacharcolon}{\kern0pt}{\isacharcolon}{\kern0pt}\ {\isachardoublequoteopen}{\isacharbrackleft}{\kern0pt}i{\isacharcomma}{\kern0pt}i{\isacharcomma}{\kern0pt}i{\isacharbrackright}{\kern0pt}\ {\isasymRightarrow}\ i{\isachardoublequoteclose}\ \isakeyword{where}\isanewline
\ \ {\isachardoublequoteopen}is{\isacharunderscore}{\kern0pt}powapply{\isacharunderscore}{\kern0pt}fm{\isacharparenleft}{\kern0pt}f{\isacharcomma}{\kern0pt}y{\isacharcomma}{\kern0pt}z{\isacharparenright}{\kern0pt}\ {\isasymequiv}\isanewline
\ \ \ \ \ \ Exists{\isacharparenleft}{\kern0pt}And{\isacharparenleft}{\kern0pt}fun{\isacharunderscore}{\kern0pt}apply{\isacharunderscore}{\kern0pt}fm{\isacharparenleft}{\kern0pt}succ{\isacharparenleft}{\kern0pt}f{\isacharparenright}{\kern0pt}{\isacharcomma}{\kern0pt}\ succ{\isacharparenleft}{\kern0pt}y{\isacharparenright}{\kern0pt}{\isacharcomma}{\kern0pt}\ {\isadigit{0}}{\isacharparenright}{\kern0pt}{\isacharcomma}{\kern0pt}\isanewline
\ \ \ \ \ \ \ \ \ \ \ \ Forall{\isacharparenleft}{\kern0pt}Iff{\isacharparenleft}{\kern0pt}Member{\isacharparenleft}{\kern0pt}{\isadigit{0}}{\isacharcomma}{\kern0pt}\ succ{\isacharparenleft}{\kern0pt}succ{\isacharparenleft}{\kern0pt}z{\isacharparenright}{\kern0pt}{\isacharparenright}{\kern0pt}{\isacharparenright}{\kern0pt}{\isacharcomma}{\kern0pt}\isanewline
\ \ \ \ \ \ \ \ \ \ \ \ Forall{\isacharparenleft}{\kern0pt}Implies{\isacharparenleft}{\kern0pt}Member{\isacharparenleft}{\kern0pt}{\isadigit{0}}{\isacharcomma}{\kern0pt}\ {\isadigit{1}}{\isacharparenright}{\kern0pt}{\isacharcomma}{\kern0pt}\ Member{\isacharparenleft}{\kern0pt}{\isadigit{0}}{\isacharcomma}{\kern0pt}\ {\isadigit{2}}{\isacharparenright}{\kern0pt}{\isacharparenright}{\kern0pt}{\isacharparenright}{\kern0pt}{\isacharparenright}{\kern0pt}{\isacharparenright}{\kern0pt}{\isacharparenright}{\kern0pt}{\isacharparenright}{\kern0pt}{\isachardoublequoteclose}\isanewline
\isanewline
\isacommand{lemma}\isamarkupfalse%
\ is{\isacharunderscore}{\kern0pt}powapply{\isacharunderscore}{\kern0pt}type\ {\isacharbrackleft}{\kern0pt}TC{\isacharbrackright}{\kern0pt}\ {\isacharcolon}{\kern0pt}\isanewline
\ \ {\isachardoublequoteopen}{\isasymlbrakk}f{\isasymin}nat\ {\isacharsemicolon}{\kern0pt}\ y{\isasymin}nat{\isacharsemicolon}{\kern0pt}\ z{\isasymin}nat{\isasymrbrakk}\ {\isasymLongrightarrow}\ is{\isacharunderscore}{\kern0pt}powapply{\isacharunderscore}{\kern0pt}fm{\isacharparenleft}{\kern0pt}f{\isacharcomma}{\kern0pt}y{\isacharcomma}{\kern0pt}z{\isacharparenright}{\kern0pt}{\isasymin}formula{\isachardoublequoteclose}\isanewline
%
\isadelimproof
\ \ %
\endisadelimproof
%
\isatagproof
\isacommand{unfolding}\isamarkupfalse%
\ is{\isacharunderscore}{\kern0pt}powapply{\isacharunderscore}{\kern0pt}fm{\isacharunderscore}{\kern0pt}def\ \isacommand{by}\isamarkupfalse%
\ simp%
\endisatagproof
{\isafoldproof}%
%
\isadelimproof
\isanewline
%
\endisadelimproof
\isanewline
\isacommand{lemma}\isamarkupfalse%
\ sats{\isacharunderscore}{\kern0pt}is{\isacharunderscore}{\kern0pt}powapply{\isacharunderscore}{\kern0pt}fm\ {\isacharcolon}{\kern0pt}\isanewline
\ \ \isakeyword{assumes}\isanewline
\ \ \ \ {\isachardoublequoteopen}f{\isasymin}nat{\isachardoublequoteclose}\ {\isachardoublequoteopen}y{\isasymin}nat{\isachardoublequoteclose}\ {\isachardoublequoteopen}z{\isasymin}nat{\isachardoublequoteclose}\ {\isachardoublequoteopen}env{\isasymin}list{\isacharparenleft}{\kern0pt}A{\isacharparenright}{\kern0pt}{\isachardoublequoteclose}\ {\isachardoublequoteopen}{\isadigit{0}}{\isasymin}A{\isachardoublequoteclose}\isanewline
\ \ \isakeyword{shows}\isanewline
\ \ \ \ {\isachardoublequoteopen}is{\isacharunderscore}{\kern0pt}powapply{\isacharparenleft}{\kern0pt}{\isacharhash}{\kern0pt}{\isacharhash}{\kern0pt}A{\isacharcomma}{\kern0pt}nth{\isacharparenleft}{\kern0pt}f{\isacharcomma}{\kern0pt}\ env{\isacharparenright}{\kern0pt}{\isacharcomma}{\kern0pt}nth{\isacharparenleft}{\kern0pt}y{\isacharcomma}{\kern0pt}\ env{\isacharparenright}{\kern0pt}{\isacharcomma}{\kern0pt}nth{\isacharparenleft}{\kern0pt}z{\isacharcomma}{\kern0pt}\ env{\isacharparenright}{\kern0pt}{\isacharparenright}{\kern0pt}\isanewline
\ \ \ \ {\isasymlongleftrightarrow}\ sats{\isacharparenleft}{\kern0pt}A{\isacharcomma}{\kern0pt}is{\isacharunderscore}{\kern0pt}powapply{\isacharunderscore}{\kern0pt}fm{\isacharparenleft}{\kern0pt}f{\isacharcomma}{\kern0pt}y{\isacharcomma}{\kern0pt}z{\isacharparenright}{\kern0pt}{\isacharcomma}{\kern0pt}env{\isacharparenright}{\kern0pt}{\isachardoublequoteclose}\isanewline
%
\isadelimproof
\ \ %
\endisadelimproof
%
\isatagproof
\isacommand{unfolding}\isamarkupfalse%
\ is{\isacharunderscore}{\kern0pt}powapply{\isacharunderscore}{\kern0pt}def\ is{\isacharunderscore}{\kern0pt}powapply{\isacharunderscore}{\kern0pt}fm{\isacharunderscore}{\kern0pt}def\ is{\isacharunderscore}{\kern0pt}Collect{\isacharunderscore}{\kern0pt}def\ powerset{\isacharunderscore}{\kern0pt}def\ subset{\isacharunderscore}{\kern0pt}def\isanewline
\ \ \isacommand{using}\isamarkupfalse%
\ nth{\isacharunderscore}{\kern0pt}closed\ assms\ \isacommand{by}\isamarkupfalse%
\ simp%
\endisatagproof
{\isafoldproof}%
%
\isadelimproof
\isanewline
%
\endisadelimproof
\isanewline
\isanewline
\isacommand{lemma}\isamarkupfalse%
\ {\isacharparenleft}{\kern0pt}\isakeyword{in}\ M{\isacharunderscore}{\kern0pt}ZF{\isacharunderscore}{\kern0pt}trans{\isacharparenright}{\kern0pt}\ powapply{\isacharunderscore}{\kern0pt}repl\ {\isacharcolon}{\kern0pt}\isanewline
\ \ \isakeyword{assumes}\isanewline
\ \ \ \ {\isachardoublequoteopen}f{\isasymin}M{\isachardoublequoteclose}\isanewline
\ \ \isakeyword{shows}\isanewline
\ \ \ \ {\isachardoublequoteopen}strong{\isacharunderscore}{\kern0pt}replacement{\isacharparenleft}{\kern0pt}{\isacharhash}{\kern0pt}{\isacharhash}{\kern0pt}M{\isacharcomma}{\kern0pt}is{\isacharunderscore}{\kern0pt}powapply{\isacharparenleft}{\kern0pt}{\isacharhash}{\kern0pt}{\isacharhash}{\kern0pt}M{\isacharcomma}{\kern0pt}f{\isacharparenright}{\kern0pt}{\isacharparenright}{\kern0pt}{\isachardoublequoteclose}\isanewline
%
\isadelimproof
%
\endisadelimproof
%
\isatagproof
\isacommand{proof}\isamarkupfalse%
\ {\isacharminus}{\kern0pt}\isanewline
\ \ \isacommand{have}\isamarkupfalse%
\ {\isachardoublequoteopen}arity{\isacharparenleft}{\kern0pt}is{\isacharunderscore}{\kern0pt}powapply{\isacharunderscore}{\kern0pt}fm{\isacharparenleft}{\kern0pt}{\isadigit{2}}{\isacharcomma}{\kern0pt}{\isadigit{0}}{\isacharcomma}{\kern0pt}{\isadigit{1}}{\isacharparenright}{\kern0pt}{\isacharparenright}{\kern0pt}\ {\isacharequal}{\kern0pt}\ {\isadigit{3}}{\isachardoublequoteclose}\isanewline
\ \ \ \ \isacommand{unfolding}\isamarkupfalse%
\ is{\isacharunderscore}{\kern0pt}powapply{\isacharunderscore}{\kern0pt}fm{\isacharunderscore}{\kern0pt}def\isanewline
\ \ \ \ \isacommand{by}\isamarkupfalse%
\ {\isacharparenleft}{\kern0pt}simp\ add{\isacharcolon}{\kern0pt}\ fm{\isacharunderscore}{\kern0pt}defs\ nat{\isacharunderscore}{\kern0pt}simp{\isacharunderscore}{\kern0pt}union{\isacharparenright}{\kern0pt}\isanewline
\ \ \isacommand{then}\isamarkupfalse%
\isanewline
\ \ \isacommand{have}\isamarkupfalse%
\ {\isachardoublequoteopen}{\isasymforall}f{\isadigit{0}}{\isasymin}M{\isachardot}{\kern0pt}\ strong{\isacharunderscore}{\kern0pt}replacement{\isacharparenleft}{\kern0pt}{\isacharhash}{\kern0pt}{\isacharhash}{\kern0pt}M{\isacharcomma}{\kern0pt}\ {\isasymlambda}p\ z{\isachardot}{\kern0pt}\ sats{\isacharparenleft}{\kern0pt}M{\isacharcomma}{\kern0pt}is{\isacharunderscore}{\kern0pt}powapply{\isacharunderscore}{\kern0pt}fm{\isacharparenleft}{\kern0pt}{\isadigit{2}}{\isacharcomma}{\kern0pt}{\isadigit{0}}{\isacharcomma}{\kern0pt}{\isadigit{1}}{\isacharparenright}{\kern0pt}\ {\isacharcomma}{\kern0pt}\ {\isacharbrackleft}{\kern0pt}p{\isacharcomma}{\kern0pt}z{\isacharcomma}{\kern0pt}f{\isadigit{0}}{\isacharbrackright}{\kern0pt}{\isacharparenright}{\kern0pt}{\isacharparenright}{\kern0pt}{\isachardoublequoteclose}\isanewline
\ \ \ \ \isacommand{using}\isamarkupfalse%
\ replacement{\isacharunderscore}{\kern0pt}ax\ \isacommand{by}\isamarkupfalse%
\ simp\isanewline
\ \ \isacommand{moreover}\isamarkupfalse%
\isanewline
\ \ \isacommand{have}\isamarkupfalse%
\ {\isachardoublequoteopen}is{\isacharunderscore}{\kern0pt}powapply{\isacharparenleft}{\kern0pt}{\isacharhash}{\kern0pt}{\isacharhash}{\kern0pt}M{\isacharcomma}{\kern0pt}f{\isadigit{0}}{\isacharcomma}{\kern0pt}p{\isacharcomma}{\kern0pt}z{\isacharparenright}{\kern0pt}\ {\isasymlongleftrightarrow}\ sats{\isacharparenleft}{\kern0pt}M{\isacharcomma}{\kern0pt}is{\isacharunderscore}{\kern0pt}powapply{\isacharunderscore}{\kern0pt}fm{\isacharparenleft}{\kern0pt}{\isadigit{2}}{\isacharcomma}{\kern0pt}{\isadigit{0}}{\isacharcomma}{\kern0pt}{\isadigit{1}}{\isacharparenright}{\kern0pt}\ {\isacharcomma}{\kern0pt}\ {\isacharbrackleft}{\kern0pt}p{\isacharcomma}{\kern0pt}z{\isacharcomma}{\kern0pt}f{\isadigit{0}}{\isacharbrackright}{\kern0pt}{\isacharparenright}{\kern0pt}{\isachardoublequoteclose}\isanewline
\ \ \ \ \isakeyword{if}\ {\isachardoublequoteopen}p{\isasymin}M{\isachardoublequoteclose}\ {\isachardoublequoteopen}z{\isasymin}M{\isachardoublequoteclose}\ {\isachardoublequoteopen}f{\isadigit{0}}{\isasymin}M{\isachardoublequoteclose}\ \isakeyword{for}\ p\ z\ f{\isadigit{0}}\isanewline
\ \ \ \ \isacommand{using}\isamarkupfalse%
\ that\ zero{\isacharunderscore}{\kern0pt}in{\isacharunderscore}{\kern0pt}M\ sats{\isacharunderscore}{\kern0pt}is{\isacharunderscore}{\kern0pt}powapply{\isacharunderscore}{\kern0pt}fm{\isacharbrackleft}{\kern0pt}of\ {\isadigit{2}}\ {\isadigit{0}}\ {\isadigit{1}}\ {\isachardoublequoteopen}{\isacharbrackleft}{\kern0pt}p{\isacharcomma}{\kern0pt}z{\isacharcomma}{\kern0pt}f{\isadigit{0}}{\isacharbrackright}{\kern0pt}{\isachardoublequoteclose}\ M{\isacharbrackright}{\kern0pt}\ \isacommand{by}\isamarkupfalse%
\ simp\isanewline
\ \ \isacommand{ultimately}\isamarkupfalse%
\isanewline
\ \ \isacommand{have}\isamarkupfalse%
\ {\isachardoublequoteopen}{\isasymforall}f{\isadigit{0}}{\isasymin}M{\isachardot}{\kern0pt}\ strong{\isacharunderscore}{\kern0pt}replacement{\isacharparenleft}{\kern0pt}{\isacharhash}{\kern0pt}{\isacharhash}{\kern0pt}M{\isacharcomma}{\kern0pt}\ is{\isacharunderscore}{\kern0pt}powapply{\isacharparenleft}{\kern0pt}{\isacharhash}{\kern0pt}{\isacharhash}{\kern0pt}M{\isacharcomma}{\kern0pt}f{\isadigit{0}}{\isacharparenright}{\kern0pt}{\isacharparenright}{\kern0pt}{\isachardoublequoteclose}\isanewline
\ \ \ \ \isacommand{unfolding}\isamarkupfalse%
\ strong{\isacharunderscore}{\kern0pt}replacement{\isacharunderscore}{\kern0pt}def\ univalent{\isacharunderscore}{\kern0pt}def\ \isacommand{by}\isamarkupfalse%
\ simp\isanewline
\ \ \isacommand{with}\isamarkupfalse%
\ {\isacartoucheopen}f{\isasymin}M{\isacartoucheclose}\ \isacommand{show}\isamarkupfalse%
\ {\isacharquery}{\kern0pt}thesis\ \isacommand{by}\isamarkupfalse%
\ simp\isanewline
\isacommand{qed}\isamarkupfalse%
%
\endisatagproof
{\isafoldproof}%
%
\isadelimproof
\isanewline
%
\endisadelimproof
\isanewline
\isanewline
\isanewline
\isacommand{definition}\isamarkupfalse%
\isanewline
\ \ PHrank{\isacharunderscore}{\kern0pt}fm\ {\isacharcolon}{\kern0pt}{\isacharcolon}{\kern0pt}\ {\isachardoublequoteopen}{\isacharbrackleft}{\kern0pt}i{\isacharcomma}{\kern0pt}i{\isacharcomma}{\kern0pt}i{\isacharbrackright}{\kern0pt}\ {\isasymRightarrow}\ i{\isachardoublequoteclose}\ \isakeyword{where}\isanewline
\ \ {\isachardoublequoteopen}PHrank{\isacharunderscore}{\kern0pt}fm{\isacharparenleft}{\kern0pt}f{\isacharcomma}{\kern0pt}y{\isacharcomma}{\kern0pt}z{\isacharparenright}{\kern0pt}\ {\isasymequiv}\ Exists{\isacharparenleft}{\kern0pt}And{\isacharparenleft}{\kern0pt}fun{\isacharunderscore}{\kern0pt}apply{\isacharunderscore}{\kern0pt}fm{\isacharparenleft}{\kern0pt}succ{\isacharparenleft}{\kern0pt}f{\isacharparenright}{\kern0pt}{\isacharcomma}{\kern0pt}succ{\isacharparenleft}{\kern0pt}y{\isacharparenright}{\kern0pt}{\isacharcomma}{\kern0pt}{\isadigit{0}}{\isacharparenright}{\kern0pt}\isanewline
\ \ \ \ \ \ \ \ \ \ \ \ \ \ \ \ \ \ \ \ \ \ \ \ \ \ \ \ \ \ \ \ \ {\isacharcomma}{\kern0pt}succ{\isacharunderscore}{\kern0pt}fm{\isacharparenleft}{\kern0pt}{\isadigit{0}}{\isacharcomma}{\kern0pt}succ{\isacharparenleft}{\kern0pt}z{\isacharparenright}{\kern0pt}{\isacharparenright}{\kern0pt}{\isacharparenright}{\kern0pt}{\isacharparenright}{\kern0pt}{\isachardoublequoteclose}\isanewline
\isanewline
\isacommand{lemma}\isamarkupfalse%
\ PHrank{\isacharunderscore}{\kern0pt}type\ {\isacharbrackleft}{\kern0pt}TC{\isacharbrackright}{\kern0pt}{\isacharcolon}{\kern0pt}\isanewline
\ \ {\isachardoublequoteopen}{\isasymlbrakk}\ x\ {\isasymin}\ nat{\isacharsemicolon}{\kern0pt}\ y\ {\isasymin}\ nat{\isacharsemicolon}{\kern0pt}\ z\ {\isasymin}\ nat\ {\isasymrbrakk}\ {\isasymLongrightarrow}\ PHrank{\isacharunderscore}{\kern0pt}fm{\isacharparenleft}{\kern0pt}x{\isacharcomma}{\kern0pt}y{\isacharcomma}{\kern0pt}z{\isacharparenright}{\kern0pt}\ {\isasymin}\ formula{\isachardoublequoteclose}\isanewline
%
\isadelimproof
\ \ %
\endisadelimproof
%
\isatagproof
\isacommand{by}\isamarkupfalse%
\ {\isacharparenleft}{\kern0pt}simp\ add{\isacharcolon}{\kern0pt}PHrank{\isacharunderscore}{\kern0pt}fm{\isacharunderscore}{\kern0pt}def{\isacharparenright}{\kern0pt}%
\endisatagproof
{\isafoldproof}%
%
\isadelimproof
\isanewline
%
\endisadelimproof
\isanewline
\isanewline
\isacommand{lemma}\isamarkupfalse%
\ {\isacharparenleft}{\kern0pt}\isakeyword{in}\ M{\isacharunderscore}{\kern0pt}ZF{\isacharunderscore}{\kern0pt}trans{\isacharparenright}{\kern0pt}\ sats{\isacharunderscore}{\kern0pt}PHrank{\isacharunderscore}{\kern0pt}fm\ {\isacharbrackleft}{\kern0pt}simp{\isacharbrackright}{\kern0pt}{\isacharcolon}{\kern0pt}\isanewline
\ \ {\isachardoublequoteopen}{\isasymlbrakk}\ x\ {\isasymin}\ nat{\isacharsemicolon}{\kern0pt}\ y\ {\isasymin}\ nat{\isacharsemicolon}{\kern0pt}\ z\ {\isasymin}\ nat{\isacharsemicolon}{\kern0pt}\ \ env\ {\isasymin}\ list{\isacharparenleft}{\kern0pt}M{\isacharparenright}{\kern0pt}\ {\isasymrbrakk}\ \isanewline
\ \ \ \ {\isasymLongrightarrow}\ sats{\isacharparenleft}{\kern0pt}M{\isacharcomma}{\kern0pt}PHrank{\isacharunderscore}{\kern0pt}fm{\isacharparenleft}{\kern0pt}x{\isacharcomma}{\kern0pt}y{\isacharcomma}{\kern0pt}z{\isacharparenright}{\kern0pt}{\isacharcomma}{\kern0pt}env{\isacharparenright}{\kern0pt}\ {\isasymlongleftrightarrow}\isanewline
\ \ \ \ \ \ \ \ PHrank{\isacharparenleft}{\kern0pt}{\isacharhash}{\kern0pt}{\isacharhash}{\kern0pt}M{\isacharcomma}{\kern0pt}nth{\isacharparenleft}{\kern0pt}x{\isacharcomma}{\kern0pt}env{\isacharparenright}{\kern0pt}{\isacharcomma}{\kern0pt}nth{\isacharparenleft}{\kern0pt}y{\isacharcomma}{\kern0pt}env{\isacharparenright}{\kern0pt}{\isacharcomma}{\kern0pt}nth{\isacharparenleft}{\kern0pt}z{\isacharcomma}{\kern0pt}env{\isacharparenright}{\kern0pt}{\isacharparenright}{\kern0pt}{\isachardoublequoteclose}\isanewline
%
\isadelimproof
\ \ %
\endisadelimproof
%
\isatagproof
\isacommand{using}\isamarkupfalse%
\ zero{\isacharunderscore}{\kern0pt}in{\isacharunderscore}{\kern0pt}M\ Internalizations{\isachardot}{\kern0pt}nth{\isacharunderscore}{\kern0pt}closed\ \isacommand{by}\isamarkupfalse%
\ {\isacharparenleft}{\kern0pt}simp\ add{\isacharcolon}{\kern0pt}\ PHrank{\isacharunderscore}{\kern0pt}def\ PHrank{\isacharunderscore}{\kern0pt}fm{\isacharunderscore}{\kern0pt}def{\isacharparenright}{\kern0pt}%
\endisatagproof
{\isafoldproof}%
%
\isadelimproof
\isanewline
%
\endisadelimproof
\isanewline
\isanewline
\isacommand{lemma}\isamarkupfalse%
\ {\isacharparenleft}{\kern0pt}\isakeyword{in}\ M{\isacharunderscore}{\kern0pt}ZF{\isacharunderscore}{\kern0pt}trans{\isacharparenright}{\kern0pt}\ phrank{\isacharunderscore}{\kern0pt}repl\ {\isacharcolon}{\kern0pt}\isanewline
\ \ \isakeyword{assumes}\isanewline
\ \ \ \ {\isachardoublequoteopen}f{\isasymin}M{\isachardoublequoteclose}\isanewline
\ \ \isakeyword{shows}\isanewline
\ \ \ \ {\isachardoublequoteopen}strong{\isacharunderscore}{\kern0pt}replacement{\isacharparenleft}{\kern0pt}{\isacharhash}{\kern0pt}{\isacharhash}{\kern0pt}M{\isacharcomma}{\kern0pt}PHrank{\isacharparenleft}{\kern0pt}{\isacharhash}{\kern0pt}{\isacharhash}{\kern0pt}M{\isacharcomma}{\kern0pt}f{\isacharparenright}{\kern0pt}{\isacharparenright}{\kern0pt}{\isachardoublequoteclose}\isanewline
%
\isadelimproof
%
\endisadelimproof
%
\isatagproof
\isacommand{proof}\isamarkupfalse%
\ {\isacharminus}{\kern0pt}\isanewline
\ \ \isacommand{have}\isamarkupfalse%
\ {\isachardoublequoteopen}arity{\isacharparenleft}{\kern0pt}PHrank{\isacharunderscore}{\kern0pt}fm{\isacharparenleft}{\kern0pt}{\isadigit{2}}{\isacharcomma}{\kern0pt}{\isadigit{0}}{\isacharcomma}{\kern0pt}{\isadigit{1}}{\isacharparenright}{\kern0pt}{\isacharparenright}{\kern0pt}\ {\isacharequal}{\kern0pt}\ {\isadigit{3}}{\isachardoublequoteclose}\isanewline
\ \ \ \ \isacommand{unfolding}\isamarkupfalse%
\ PHrank{\isacharunderscore}{\kern0pt}fm{\isacharunderscore}{\kern0pt}def\isanewline
\ \ \ \ \isacommand{by}\isamarkupfalse%
\ {\isacharparenleft}{\kern0pt}simp\ add{\isacharcolon}{\kern0pt}\ fm{\isacharunderscore}{\kern0pt}defs\ nat{\isacharunderscore}{\kern0pt}simp{\isacharunderscore}{\kern0pt}union{\isacharparenright}{\kern0pt}\isanewline
\ \ \isacommand{then}\isamarkupfalse%
\isanewline
\ \ \isacommand{have}\isamarkupfalse%
\ {\isachardoublequoteopen}{\isasymforall}f{\isadigit{0}}{\isasymin}M{\isachardot}{\kern0pt}\ strong{\isacharunderscore}{\kern0pt}replacement{\isacharparenleft}{\kern0pt}{\isacharhash}{\kern0pt}{\isacharhash}{\kern0pt}M{\isacharcomma}{\kern0pt}\ {\isasymlambda}p\ z{\isachardot}{\kern0pt}\ sats{\isacharparenleft}{\kern0pt}M{\isacharcomma}{\kern0pt}PHrank{\isacharunderscore}{\kern0pt}fm{\isacharparenleft}{\kern0pt}{\isadigit{2}}{\isacharcomma}{\kern0pt}{\isadigit{0}}{\isacharcomma}{\kern0pt}{\isadigit{1}}{\isacharparenright}{\kern0pt}\ {\isacharcomma}{\kern0pt}\ {\isacharbrackleft}{\kern0pt}p{\isacharcomma}{\kern0pt}z{\isacharcomma}{\kern0pt}f{\isadigit{0}}{\isacharbrackright}{\kern0pt}{\isacharparenright}{\kern0pt}{\isacharparenright}{\kern0pt}{\isachardoublequoteclose}\isanewline
\ \ \ \ \isacommand{using}\isamarkupfalse%
\ replacement{\isacharunderscore}{\kern0pt}ax\ \isacommand{by}\isamarkupfalse%
\ simp\isanewline
\ \ \isacommand{then}\isamarkupfalse%
\isanewline
\ \ \isacommand{have}\isamarkupfalse%
\ {\isachardoublequoteopen}{\isasymforall}f{\isadigit{0}}{\isasymin}M{\isachardot}{\kern0pt}\ strong{\isacharunderscore}{\kern0pt}replacement{\isacharparenleft}{\kern0pt}{\isacharhash}{\kern0pt}{\isacharhash}{\kern0pt}M{\isacharcomma}{\kern0pt}\ PHrank{\isacharparenleft}{\kern0pt}{\isacharhash}{\kern0pt}{\isacharhash}{\kern0pt}M{\isacharcomma}{\kern0pt}f{\isadigit{0}}{\isacharparenright}{\kern0pt}{\isacharparenright}{\kern0pt}{\isachardoublequoteclose}\isanewline
\ \ \ \ \isacommand{unfolding}\isamarkupfalse%
\ strong{\isacharunderscore}{\kern0pt}replacement{\isacharunderscore}{\kern0pt}def\ univalent{\isacharunderscore}{\kern0pt}def\ \isacommand{by}\isamarkupfalse%
\ simp\isanewline
\ \ \isacommand{with}\isamarkupfalse%
\ {\isacartoucheopen}f{\isasymin}M{\isacartoucheclose}\ \isacommand{show}\isamarkupfalse%
\ {\isacharquery}{\kern0pt}thesis\ \isacommand{by}\isamarkupfalse%
\ simp\isanewline
\isacommand{qed}\isamarkupfalse%
%
\endisatagproof
{\isafoldproof}%
%
\isadelimproof
\isanewline
%
\endisadelimproof
\isanewline
\isanewline
\isanewline
\isacommand{definition}\isamarkupfalse%
\isanewline
\ \ is{\isacharunderscore}{\kern0pt}Hrank{\isacharunderscore}{\kern0pt}fm\ {\isacharcolon}{\kern0pt}{\isacharcolon}{\kern0pt}\ {\isachardoublequoteopen}{\isacharbrackleft}{\kern0pt}i{\isacharcomma}{\kern0pt}i{\isacharcomma}{\kern0pt}i{\isacharbrackright}{\kern0pt}\ {\isasymRightarrow}\ i{\isachardoublequoteclose}\ \isakeyword{where}\isanewline
\ \ {\isachardoublequoteopen}is{\isacharunderscore}{\kern0pt}Hrank{\isacharunderscore}{\kern0pt}fm{\isacharparenleft}{\kern0pt}x{\isacharcomma}{\kern0pt}f{\isacharcomma}{\kern0pt}hc{\isacharparenright}{\kern0pt}\ {\isasymequiv}\ Exists{\isacharparenleft}{\kern0pt}And{\isacharparenleft}{\kern0pt}big{\isacharunderscore}{\kern0pt}union{\isacharunderscore}{\kern0pt}fm{\isacharparenleft}{\kern0pt}{\isadigit{0}}{\isacharcomma}{\kern0pt}succ{\isacharparenleft}{\kern0pt}hc{\isacharparenright}{\kern0pt}{\isacharparenright}{\kern0pt}{\isacharcomma}{\kern0pt}\isanewline
\ \ \ \ \ \ \ \ \ \ \ \ \ \ \ \ \ \ \ \ \ \ \ \ \ \ \ \ \ \ \ \ Replace{\isacharunderscore}{\kern0pt}fm{\isacharparenleft}{\kern0pt}succ{\isacharparenleft}{\kern0pt}x{\isacharparenright}{\kern0pt}{\isacharcomma}{\kern0pt}PHrank{\isacharunderscore}{\kern0pt}fm{\isacharparenleft}{\kern0pt}succ{\isacharparenleft}{\kern0pt}succ{\isacharparenleft}{\kern0pt}succ{\isacharparenleft}{\kern0pt}f{\isacharparenright}{\kern0pt}{\isacharparenright}{\kern0pt}{\isacharparenright}{\kern0pt}{\isacharcomma}{\kern0pt}{\isadigit{0}}{\isacharcomma}{\kern0pt}{\isadigit{1}}{\isacharparenright}{\kern0pt}{\isacharcomma}{\kern0pt}{\isadigit{0}}{\isacharparenright}{\kern0pt}{\isacharparenright}{\kern0pt}{\isacharparenright}{\kern0pt}{\isachardoublequoteclose}\isanewline
\isanewline
\isacommand{lemma}\isamarkupfalse%
\ is{\isacharunderscore}{\kern0pt}Hrank{\isacharunderscore}{\kern0pt}type\ {\isacharbrackleft}{\kern0pt}TC{\isacharbrackright}{\kern0pt}{\isacharcolon}{\kern0pt}\isanewline
\ \ {\isachardoublequoteopen}{\isasymlbrakk}\ x\ {\isasymin}\ nat{\isacharsemicolon}{\kern0pt}\ y\ {\isasymin}\ nat{\isacharsemicolon}{\kern0pt}\ z\ {\isasymin}\ nat\ {\isasymrbrakk}\ {\isasymLongrightarrow}\ is{\isacharunderscore}{\kern0pt}Hrank{\isacharunderscore}{\kern0pt}fm{\isacharparenleft}{\kern0pt}x{\isacharcomma}{\kern0pt}y{\isacharcomma}{\kern0pt}z{\isacharparenright}{\kern0pt}\ {\isasymin}\ formula{\isachardoublequoteclose}\isanewline
%
\isadelimproof
\ \ %
\endisadelimproof
%
\isatagproof
\isacommand{by}\isamarkupfalse%
\ {\isacharparenleft}{\kern0pt}simp\ add{\isacharcolon}{\kern0pt}is{\isacharunderscore}{\kern0pt}Hrank{\isacharunderscore}{\kern0pt}fm{\isacharunderscore}{\kern0pt}def{\isacharparenright}{\kern0pt}%
\endisatagproof
{\isafoldproof}%
%
\isadelimproof
\isanewline
%
\endisadelimproof
\isanewline
\isacommand{lemma}\isamarkupfalse%
\ {\isacharparenleft}{\kern0pt}\isakeyword{in}\ M{\isacharunderscore}{\kern0pt}ZF{\isacharunderscore}{\kern0pt}trans{\isacharparenright}{\kern0pt}\ sats{\isacharunderscore}{\kern0pt}is{\isacharunderscore}{\kern0pt}Hrank{\isacharunderscore}{\kern0pt}fm\ {\isacharbrackleft}{\kern0pt}simp{\isacharbrackright}{\kern0pt}{\isacharcolon}{\kern0pt}\isanewline
\ \ {\isachardoublequoteopen}{\isasymlbrakk}\ x\ {\isasymin}\ nat{\isacharsemicolon}{\kern0pt}\ y\ {\isasymin}\ nat{\isacharsemicolon}{\kern0pt}\ z\ {\isasymin}\ nat{\isacharsemicolon}{\kern0pt}\ env\ {\isasymin}\ list{\isacharparenleft}{\kern0pt}M{\isacharparenright}{\kern0pt}{\isasymrbrakk}\isanewline
\ \ \ \ {\isasymLongrightarrow}\ sats{\isacharparenleft}{\kern0pt}M{\isacharcomma}{\kern0pt}is{\isacharunderscore}{\kern0pt}Hrank{\isacharunderscore}{\kern0pt}fm{\isacharparenleft}{\kern0pt}x{\isacharcomma}{\kern0pt}y{\isacharcomma}{\kern0pt}z{\isacharparenright}{\kern0pt}{\isacharcomma}{\kern0pt}env{\isacharparenright}{\kern0pt}\ {\isasymlongleftrightarrow}\isanewline
\ \ \ \ \ \ \ \ is{\isacharunderscore}{\kern0pt}Hrank{\isacharparenleft}{\kern0pt}{\isacharhash}{\kern0pt}{\isacharhash}{\kern0pt}M{\isacharcomma}{\kern0pt}nth{\isacharparenleft}{\kern0pt}x{\isacharcomma}{\kern0pt}env{\isacharparenright}{\kern0pt}{\isacharcomma}{\kern0pt}nth{\isacharparenleft}{\kern0pt}y{\isacharcomma}{\kern0pt}env{\isacharparenright}{\kern0pt}{\isacharcomma}{\kern0pt}nth{\isacharparenleft}{\kern0pt}z{\isacharcomma}{\kern0pt}env{\isacharparenright}{\kern0pt}{\isacharparenright}{\kern0pt}{\isachardoublequoteclose}\isanewline
%
\isadelimproof
\ \ %
\endisadelimproof
%
\isatagproof
\isacommand{using}\isamarkupfalse%
\ zero{\isacharunderscore}{\kern0pt}in{\isacharunderscore}{\kern0pt}M\ is{\isacharunderscore}{\kern0pt}Hrank{\isacharunderscore}{\kern0pt}def\ is{\isacharunderscore}{\kern0pt}Hrank{\isacharunderscore}{\kern0pt}fm{\isacharunderscore}{\kern0pt}def\ sats{\isacharunderscore}{\kern0pt}Replace{\isacharunderscore}{\kern0pt}fm\isanewline
\ \ \isacommand{by}\isamarkupfalse%
\ simp%
\endisatagproof
{\isafoldproof}%
%
\isadelimproof
\isanewline
%
\endisadelimproof
\isanewline
\isanewline
\isacommand{lemma}\isamarkupfalse%
\ {\isacharparenleft}{\kern0pt}\isakeyword{in}\ M{\isacharunderscore}{\kern0pt}ZF{\isacharunderscore}{\kern0pt}trans{\isacharparenright}{\kern0pt}\ wfrec{\isacharunderscore}{\kern0pt}rank\ {\isacharcolon}{\kern0pt}\isanewline
\ \ \isakeyword{assumes}\isanewline
\ \ \ \ {\isachardoublequoteopen}X{\isasymin}M{\isachardoublequoteclose}\isanewline
\ \ \isakeyword{shows}\isanewline
\ \ \ \ {\isachardoublequoteopen}wfrec{\isacharunderscore}{\kern0pt}replacement{\isacharparenleft}{\kern0pt}{\isacharhash}{\kern0pt}{\isacharhash}{\kern0pt}M{\isacharcomma}{\kern0pt}is{\isacharunderscore}{\kern0pt}Hrank{\isacharparenleft}{\kern0pt}{\isacharhash}{\kern0pt}{\isacharhash}{\kern0pt}M{\isacharparenright}{\kern0pt}{\isacharcomma}{\kern0pt}rrank{\isacharparenleft}{\kern0pt}X{\isacharparenright}{\kern0pt}{\isacharparenright}{\kern0pt}{\isachardoublequoteclose}\isanewline
%
\isadelimproof
%
\endisadelimproof
%
\isatagproof
\isacommand{proof}\isamarkupfalse%
\ {\isacharminus}{\kern0pt}\isanewline
\ \ \isacommand{have}\isamarkupfalse%
\isanewline
\ \ \ \ {\isachardoublequoteopen}is{\isacharunderscore}{\kern0pt}Hrank{\isacharparenleft}{\kern0pt}{\isacharhash}{\kern0pt}{\isacharhash}{\kern0pt}M{\isacharcomma}{\kern0pt}a{\isadigit{2}}{\isacharcomma}{\kern0pt}\ a{\isadigit{1}}{\isacharcomma}{\kern0pt}\ a{\isadigit{0}}{\isacharparenright}{\kern0pt}\ {\isasymlongleftrightarrow}\isanewline
\ \ \ \ \ \ \ \ \ \ \ \ \ sats{\isacharparenleft}{\kern0pt}M{\isacharcomma}{\kern0pt}\ is{\isacharunderscore}{\kern0pt}Hrank{\isacharunderscore}{\kern0pt}fm{\isacharparenleft}{\kern0pt}{\isadigit{2}}{\isacharcomma}{\kern0pt}{\isadigit{1}}{\isacharcomma}{\kern0pt}{\isadigit{0}}{\isacharparenright}{\kern0pt}{\isacharcomma}{\kern0pt}\ {\isacharbrackleft}{\kern0pt}a{\isadigit{0}}{\isacharcomma}{\kern0pt}a{\isadigit{1}}{\isacharcomma}{\kern0pt}a{\isadigit{2}}{\isacharcomma}{\kern0pt}a{\isadigit{3}}{\isacharcomma}{\kern0pt}a{\isadigit{4}}{\isacharcomma}{\kern0pt}y{\isacharcomma}{\kern0pt}x{\isacharcomma}{\kern0pt}z{\isacharcomma}{\kern0pt}rrank{\isacharparenleft}{\kern0pt}X{\isacharparenright}{\kern0pt}{\isacharbrackright}{\kern0pt}{\isacharparenright}{\kern0pt}{\isachardoublequoteclose}\isanewline
\ \ \ \ \isakeyword{if}\ {\isachardoublequoteopen}a{\isadigit{4}}{\isasymin}M{\isachardoublequoteclose}\ {\isachardoublequoteopen}a{\isadigit{3}}{\isasymin}M{\isachardoublequoteclose}\ {\isachardoublequoteopen}a{\isadigit{2}}{\isasymin}M{\isachardoublequoteclose}\ {\isachardoublequoteopen}a{\isadigit{1}}{\isasymin}M{\isachardoublequoteclose}\ {\isachardoublequoteopen}a{\isadigit{0}}{\isasymin}M{\isachardoublequoteclose}\ {\isachardoublequoteopen}y{\isasymin}M{\isachardoublequoteclose}\ {\isachardoublequoteopen}x{\isasymin}M{\isachardoublequoteclose}\ {\isachardoublequoteopen}z{\isasymin}M{\isachardoublequoteclose}\ \isakeyword{for}\ a{\isadigit{4}}\ a{\isadigit{3}}\ a{\isadigit{2}}\ a{\isadigit{1}}\ a{\isadigit{0}}\ y\ x\ z\isanewline
\ \ \ \ \isacommand{using}\isamarkupfalse%
\ that\ rrank{\isacharunderscore}{\kern0pt}in{\isacharunderscore}{\kern0pt}M\ {\isacartoucheopen}X{\isasymin}M{\isacartoucheclose}\ \isacommand{by}\isamarkupfalse%
\ simp\isanewline
\ \ \isacommand{then}\isamarkupfalse%
\isanewline
\ \ \isacommand{have}\isamarkupfalse%
\isanewline
\ \ \ \ {\isadigit{1}}{\isacharcolon}{\kern0pt}{\isachardoublequoteopen}sats{\isacharparenleft}{\kern0pt}M{\isacharcomma}{\kern0pt}\ is{\isacharunderscore}{\kern0pt}wfrec{\isacharunderscore}{\kern0pt}fm{\isacharparenleft}{\kern0pt}is{\isacharunderscore}{\kern0pt}Hrank{\isacharunderscore}{\kern0pt}fm{\isacharparenleft}{\kern0pt}{\isadigit{2}}{\isacharcomma}{\kern0pt}{\isadigit{1}}{\isacharcomma}{\kern0pt}{\isadigit{0}}{\isacharparenright}{\kern0pt}{\isacharcomma}{\kern0pt}{\isadigit{3}}{\isacharcomma}{\kern0pt}{\isadigit{1}}{\isacharcomma}{\kern0pt}{\isadigit{0}}{\isacharparenright}{\kern0pt}{\isacharcomma}{\kern0pt}{\isacharbrackleft}{\kern0pt}y{\isacharcomma}{\kern0pt}x{\isacharcomma}{\kern0pt}z{\isacharcomma}{\kern0pt}rrank{\isacharparenleft}{\kern0pt}X{\isacharparenright}{\kern0pt}{\isacharbrackright}{\kern0pt}{\isacharparenright}{\kern0pt}\isanewline
\ \ {\isasymlongleftrightarrow}\ is{\isacharunderscore}{\kern0pt}wfrec{\isacharparenleft}{\kern0pt}{\isacharhash}{\kern0pt}{\isacharhash}{\kern0pt}M{\isacharcomma}{\kern0pt}\ is{\isacharunderscore}{\kern0pt}Hrank{\isacharparenleft}{\kern0pt}{\isacharhash}{\kern0pt}{\isacharhash}{\kern0pt}M{\isacharparenright}{\kern0pt}\ {\isacharcomma}{\kern0pt}rrank{\isacharparenleft}{\kern0pt}X{\isacharparenright}{\kern0pt}{\isacharcomma}{\kern0pt}\ x{\isacharcomma}{\kern0pt}\ y{\isacharparenright}{\kern0pt}{\isachardoublequoteclose}\isanewline
\ \ \ \ \isakeyword{if}\ {\isachardoublequoteopen}y{\isasymin}M{\isachardoublequoteclose}\ {\isachardoublequoteopen}x{\isasymin}M{\isachardoublequoteclose}\ {\isachardoublequoteopen}z{\isasymin}M{\isachardoublequoteclose}\ \isakeyword{for}\ y\ x\ z\isanewline
\ \ \ \ \isacommand{using}\isamarkupfalse%
\ that\ {\isacartoucheopen}X{\isasymin}M{\isacartoucheclose}\ rrank{\isacharunderscore}{\kern0pt}in{\isacharunderscore}{\kern0pt}M\ sats{\isacharunderscore}{\kern0pt}is{\isacharunderscore}{\kern0pt}wfrec{\isacharunderscore}{\kern0pt}fm\ \isacommand{by}\isamarkupfalse%
\ simp\isanewline
\ \ \isacommand{let}\isamarkupfalse%
\isanewline
\ \ \ \ {\isacharquery}{\kern0pt}f{\isacharequal}{\kern0pt}{\isachardoublequoteopen}Exists{\isacharparenleft}{\kern0pt}And{\isacharparenleft}{\kern0pt}pair{\isacharunderscore}{\kern0pt}fm{\isacharparenleft}{\kern0pt}{\isadigit{1}}{\isacharcomma}{\kern0pt}{\isadigit{0}}{\isacharcomma}{\kern0pt}{\isadigit{2}}{\isacharparenright}{\kern0pt}{\isacharcomma}{\kern0pt}is{\isacharunderscore}{\kern0pt}wfrec{\isacharunderscore}{\kern0pt}fm{\isacharparenleft}{\kern0pt}is{\isacharunderscore}{\kern0pt}Hrank{\isacharunderscore}{\kern0pt}fm{\isacharparenleft}{\kern0pt}{\isadigit{2}}{\isacharcomma}{\kern0pt}{\isadigit{1}}{\isacharcomma}{\kern0pt}{\isadigit{0}}{\isacharparenright}{\kern0pt}{\isacharcomma}{\kern0pt}{\isadigit{3}}{\isacharcomma}{\kern0pt}{\isadigit{1}}{\isacharcomma}{\kern0pt}{\isadigit{0}}{\isacharparenright}{\kern0pt}{\isacharparenright}{\kern0pt}{\isacharparenright}{\kern0pt}{\isachardoublequoteclose}\isanewline
\ \ \isacommand{have}\isamarkupfalse%
\ satsf{\isacharcolon}{\kern0pt}{\isachardoublequoteopen}sats{\isacharparenleft}{\kern0pt}M{\isacharcomma}{\kern0pt}\ {\isacharquery}{\kern0pt}f{\isacharcomma}{\kern0pt}\ {\isacharbrackleft}{\kern0pt}x{\isacharcomma}{\kern0pt}z{\isacharcomma}{\kern0pt}rrank{\isacharparenleft}{\kern0pt}X{\isacharparenright}{\kern0pt}{\isacharbrackright}{\kern0pt}{\isacharparenright}{\kern0pt}\isanewline
\ \ \ \ \ \ \ \ \ \ \ \ \ \ {\isasymlongleftrightarrow}\ {\isacharparenleft}{\kern0pt}{\isasymexists}y{\isasymin}M{\isachardot}{\kern0pt}\ pair{\isacharparenleft}{\kern0pt}{\isacharhash}{\kern0pt}{\isacharhash}{\kern0pt}M{\isacharcomma}{\kern0pt}x{\isacharcomma}{\kern0pt}y{\isacharcomma}{\kern0pt}z{\isacharparenright}{\kern0pt}\ {\isacharampersand}{\kern0pt}\ is{\isacharunderscore}{\kern0pt}wfrec{\isacharparenleft}{\kern0pt}{\isacharhash}{\kern0pt}{\isacharhash}{\kern0pt}M{\isacharcomma}{\kern0pt}\ is{\isacharunderscore}{\kern0pt}Hrank{\isacharparenleft}{\kern0pt}{\isacharhash}{\kern0pt}{\isacharhash}{\kern0pt}M{\isacharparenright}{\kern0pt}\ {\isacharcomma}{\kern0pt}\ rrank{\isacharparenleft}{\kern0pt}X{\isacharparenright}{\kern0pt}{\isacharcomma}{\kern0pt}\ x{\isacharcomma}{\kern0pt}\ y{\isacharparenright}{\kern0pt}{\isacharparenright}{\kern0pt}{\isachardoublequoteclose}\isanewline
\ \ \ \ \isakeyword{if}\ {\isachardoublequoteopen}x{\isasymin}M{\isachardoublequoteclose}\ {\isachardoublequoteopen}z{\isasymin}M{\isachardoublequoteclose}\ \isakeyword{for}\ x\ z\isanewline
\ \ \ \ \isacommand{using}\isamarkupfalse%
\ that\ {\isadigit{1}}\ {\isacartoucheopen}X{\isasymin}M{\isacartoucheclose}\ rrank{\isacharunderscore}{\kern0pt}in{\isacharunderscore}{\kern0pt}M\ \isacommand{by}\isamarkupfalse%
\ {\isacharparenleft}{\kern0pt}simp\ del{\isacharcolon}{\kern0pt}pair{\isacharunderscore}{\kern0pt}abs{\isacharparenright}{\kern0pt}\isanewline
\ \ \isacommand{have}\isamarkupfalse%
\ {\isachardoublequoteopen}arity{\isacharparenleft}{\kern0pt}{\isacharquery}{\kern0pt}f{\isacharparenright}{\kern0pt}\ {\isacharequal}{\kern0pt}\ {\isadigit{3}}{\isachardoublequoteclose}\isanewline
\ \ \ \ \isacommand{unfolding}\isamarkupfalse%
\ is{\isacharunderscore}{\kern0pt}wfrec{\isacharunderscore}{\kern0pt}fm{\isacharunderscore}{\kern0pt}def\ is{\isacharunderscore}{\kern0pt}recfun{\isacharunderscore}{\kern0pt}fm{\isacharunderscore}{\kern0pt}def\ is{\isacharunderscore}{\kern0pt}nat{\isacharunderscore}{\kern0pt}case{\isacharunderscore}{\kern0pt}fm{\isacharunderscore}{\kern0pt}def\ is{\isacharunderscore}{\kern0pt}Hrank{\isacharunderscore}{\kern0pt}fm{\isacharunderscore}{\kern0pt}def\ PHrank{\isacharunderscore}{\kern0pt}fm{\isacharunderscore}{\kern0pt}def\isanewline
\ \ \ \ \ \ restriction{\isacharunderscore}{\kern0pt}fm{\isacharunderscore}{\kern0pt}def\ list{\isacharunderscore}{\kern0pt}functor{\isacharunderscore}{\kern0pt}fm{\isacharunderscore}{\kern0pt}def\ number{\isadigit{1}}{\isacharunderscore}{\kern0pt}fm{\isacharunderscore}{\kern0pt}def\ cartprod{\isacharunderscore}{\kern0pt}fm{\isacharunderscore}{\kern0pt}def\isanewline
\ \ \ \ \ \ sum{\isacharunderscore}{\kern0pt}fm{\isacharunderscore}{\kern0pt}def\ quasinat{\isacharunderscore}{\kern0pt}fm{\isacharunderscore}{\kern0pt}def\ pre{\isacharunderscore}{\kern0pt}image{\isacharunderscore}{\kern0pt}fm{\isacharunderscore}{\kern0pt}def\ fm{\isacharunderscore}{\kern0pt}defs\isanewline
\ \ \ \ \isacommand{by}\isamarkupfalse%
\ {\isacharparenleft}{\kern0pt}simp\ add{\isacharcolon}{\kern0pt}nat{\isacharunderscore}{\kern0pt}simp{\isacharunderscore}{\kern0pt}union{\isacharparenright}{\kern0pt}\isanewline
\ \ \isacommand{then}\isamarkupfalse%
\isanewline
\ \ \isacommand{have}\isamarkupfalse%
\ {\isachardoublequoteopen}strong{\isacharunderscore}{\kern0pt}replacement{\isacharparenleft}{\kern0pt}{\isacharhash}{\kern0pt}{\isacharhash}{\kern0pt}M{\isacharcomma}{\kern0pt}{\isasymlambda}x\ z{\isachardot}{\kern0pt}\ sats{\isacharparenleft}{\kern0pt}M{\isacharcomma}{\kern0pt}{\isacharquery}{\kern0pt}f{\isacharcomma}{\kern0pt}{\isacharbrackleft}{\kern0pt}x{\isacharcomma}{\kern0pt}z{\isacharcomma}{\kern0pt}rrank{\isacharparenleft}{\kern0pt}X{\isacharparenright}{\kern0pt}{\isacharbrackright}{\kern0pt}{\isacharparenright}{\kern0pt}{\isacharparenright}{\kern0pt}{\isachardoublequoteclose}\isanewline
\ \ \ \ \isacommand{using}\isamarkupfalse%
\ replacement{\isacharunderscore}{\kern0pt}ax\ {\isadigit{1}}\ {\isacartoucheopen}X{\isasymin}M{\isacartoucheclose}\ rrank{\isacharunderscore}{\kern0pt}in{\isacharunderscore}{\kern0pt}M\ \isacommand{by}\isamarkupfalse%
\ simp\isanewline
\ \ \isacommand{then}\isamarkupfalse%
\isanewline
\ \ \isacommand{have}\isamarkupfalse%
\ {\isachardoublequoteopen}strong{\isacharunderscore}{\kern0pt}replacement{\isacharparenleft}{\kern0pt}{\isacharhash}{\kern0pt}{\isacharhash}{\kern0pt}M{\isacharcomma}{\kern0pt}{\isasymlambda}x\ z{\isachardot}{\kern0pt}\isanewline
\ \ \ \ \ \ \ \ \ \ {\isasymexists}y{\isasymin}M{\isachardot}{\kern0pt}\ pair{\isacharparenleft}{\kern0pt}{\isacharhash}{\kern0pt}{\isacharhash}{\kern0pt}M{\isacharcomma}{\kern0pt}x{\isacharcomma}{\kern0pt}y{\isacharcomma}{\kern0pt}z{\isacharparenright}{\kern0pt}\ {\isacharampersand}{\kern0pt}\ is{\isacharunderscore}{\kern0pt}wfrec{\isacharparenleft}{\kern0pt}{\isacharhash}{\kern0pt}{\isacharhash}{\kern0pt}M{\isacharcomma}{\kern0pt}\ is{\isacharunderscore}{\kern0pt}Hrank{\isacharparenleft}{\kern0pt}{\isacharhash}{\kern0pt}{\isacharhash}{\kern0pt}M{\isacharparenright}{\kern0pt}\ {\isacharcomma}{\kern0pt}\ rrank{\isacharparenleft}{\kern0pt}X{\isacharparenright}{\kern0pt}{\isacharcomma}{\kern0pt}\ x{\isacharcomma}{\kern0pt}\ y{\isacharparenright}{\kern0pt}{\isacharparenright}{\kern0pt}{\isachardoublequoteclose}\isanewline
\ \ \ \ \isacommand{using}\isamarkupfalse%
\ repl{\isacharunderscore}{\kern0pt}sats{\isacharbrackleft}{\kern0pt}of\ M\ {\isacharquery}{\kern0pt}f\ {\isachardoublequoteopen}{\isacharbrackleft}{\kern0pt}rrank{\isacharparenleft}{\kern0pt}X{\isacharparenright}{\kern0pt}{\isacharbrackright}{\kern0pt}{\isachardoublequoteclose}{\isacharbrackright}{\kern0pt}\ \ satsf\ \isacommand{by}\isamarkupfalse%
\ {\isacharparenleft}{\kern0pt}simp\ del{\isacharcolon}{\kern0pt}pair{\isacharunderscore}{\kern0pt}abs{\isacharparenright}{\kern0pt}\isanewline
\ \ \isacommand{then}\isamarkupfalse%
\isanewline
\ \ \isacommand{show}\isamarkupfalse%
\ {\isacharquery}{\kern0pt}thesis\ \isacommand{unfolding}\isamarkupfalse%
\ wfrec{\isacharunderscore}{\kern0pt}replacement{\isacharunderscore}{\kern0pt}def\ \ \isacommand{by}\isamarkupfalse%
\ simp\isanewline
\isacommand{qed}\isamarkupfalse%
%
\endisatagproof
{\isafoldproof}%
%
\isadelimproof
\isanewline
%
\endisadelimproof
\isanewline
\isanewline
\isacommand{definition}\isamarkupfalse%
\isanewline
\ \ is{\isacharunderscore}{\kern0pt}HVfrom{\isacharunderscore}{\kern0pt}fm\ {\isacharcolon}{\kern0pt}{\isacharcolon}{\kern0pt}\ {\isachardoublequoteopen}{\isacharbrackleft}{\kern0pt}i{\isacharcomma}{\kern0pt}i{\isacharcomma}{\kern0pt}i{\isacharcomma}{\kern0pt}i{\isacharbrackright}{\kern0pt}\ {\isasymRightarrow}\ i{\isachardoublequoteclose}\ \isakeyword{where}\isanewline
\ \ {\isachardoublequoteopen}is{\isacharunderscore}{\kern0pt}HVfrom{\isacharunderscore}{\kern0pt}fm{\isacharparenleft}{\kern0pt}A{\isacharcomma}{\kern0pt}x{\isacharcomma}{\kern0pt}f{\isacharcomma}{\kern0pt}h{\isacharparenright}{\kern0pt}\ {\isasymequiv}\ Exists{\isacharparenleft}{\kern0pt}Exists{\isacharparenleft}{\kern0pt}And{\isacharparenleft}{\kern0pt}union{\isacharunderscore}{\kern0pt}fm{\isacharparenleft}{\kern0pt}A\ {\isacharhash}{\kern0pt}{\isacharplus}{\kern0pt}\ {\isadigit{2}}{\isacharcomma}{\kern0pt}{\isadigit{1}}{\isacharcomma}{\kern0pt}h\ {\isacharhash}{\kern0pt}{\isacharplus}{\kern0pt}\ {\isadigit{2}}{\isacharparenright}{\kern0pt}{\isacharcomma}{\kern0pt}\isanewline
\ \ \ \ \ \ \ \ \ \ \ \ \ \ \ \ \ \ \ \ \ \ \ \ \ \ \ \ And{\isacharparenleft}{\kern0pt}big{\isacharunderscore}{\kern0pt}union{\isacharunderscore}{\kern0pt}fm{\isacharparenleft}{\kern0pt}{\isadigit{0}}{\isacharcomma}{\kern0pt}{\isadigit{1}}{\isacharparenright}{\kern0pt}{\isacharcomma}{\kern0pt}\isanewline
\ \ \ \ \ \ \ \ \ \ \ \ \ \ \ \ \ \ \ \ \ \ \ \ \ \ \ \ Replace{\isacharunderscore}{\kern0pt}fm{\isacharparenleft}{\kern0pt}x\ {\isacharhash}{\kern0pt}{\isacharplus}{\kern0pt}\ {\isadigit{2}}{\isacharcomma}{\kern0pt}is{\isacharunderscore}{\kern0pt}powapply{\isacharunderscore}{\kern0pt}fm{\isacharparenleft}{\kern0pt}f\ {\isacharhash}{\kern0pt}{\isacharplus}{\kern0pt}\ {\isadigit{4}}{\isacharcomma}{\kern0pt}{\isadigit{0}}{\isacharcomma}{\kern0pt}{\isadigit{1}}{\isacharparenright}{\kern0pt}{\isacharcomma}{\kern0pt}{\isadigit{0}}{\isacharparenright}{\kern0pt}{\isacharparenright}{\kern0pt}{\isacharparenright}{\kern0pt}{\isacharparenright}{\kern0pt}{\isacharparenright}{\kern0pt}{\isachardoublequoteclose}\isanewline
\isanewline
\isacommand{lemma}\isamarkupfalse%
\ is{\isacharunderscore}{\kern0pt}HVfrom{\isacharunderscore}{\kern0pt}type\ {\isacharbrackleft}{\kern0pt}TC{\isacharbrackright}{\kern0pt}{\isacharcolon}{\kern0pt}\isanewline
\ \ {\isachardoublequoteopen}{\isasymlbrakk}\ A{\isasymin}nat{\isacharsemicolon}{\kern0pt}\ x\ {\isasymin}\ nat{\isacharsemicolon}{\kern0pt}\ f\ {\isasymin}\ nat{\isacharsemicolon}{\kern0pt}\ h\ {\isasymin}\ nat\ {\isasymrbrakk}\ {\isasymLongrightarrow}\ is{\isacharunderscore}{\kern0pt}HVfrom{\isacharunderscore}{\kern0pt}fm{\isacharparenleft}{\kern0pt}A{\isacharcomma}{\kern0pt}x{\isacharcomma}{\kern0pt}f{\isacharcomma}{\kern0pt}h{\isacharparenright}{\kern0pt}\ {\isasymin}\ formula{\isachardoublequoteclose}\isanewline
%
\isadelimproof
\ \ %
\endisadelimproof
%
\isatagproof
\isacommand{by}\isamarkupfalse%
\ {\isacharparenleft}{\kern0pt}simp\ add{\isacharcolon}{\kern0pt}is{\isacharunderscore}{\kern0pt}HVfrom{\isacharunderscore}{\kern0pt}fm{\isacharunderscore}{\kern0pt}def{\isacharparenright}{\kern0pt}%
\endisatagproof
{\isafoldproof}%
%
\isadelimproof
\isanewline
%
\endisadelimproof
\isanewline
\isacommand{lemma}\isamarkupfalse%
\ sats{\isacharunderscore}{\kern0pt}is{\isacharunderscore}{\kern0pt}HVfrom{\isacharunderscore}{\kern0pt}fm\ {\isacharcolon}{\kern0pt}\isanewline
\ \ {\isachardoublequoteopen}{\isasymlbrakk}\ a{\isasymin}nat{\isacharsemicolon}{\kern0pt}\ x\ {\isasymin}\ nat{\isacharsemicolon}{\kern0pt}\ f\ {\isasymin}\ nat{\isacharsemicolon}{\kern0pt}\ h\ {\isasymin}\ nat{\isacharsemicolon}{\kern0pt}\ env\ {\isasymin}\ list{\isacharparenleft}{\kern0pt}A{\isacharparenright}{\kern0pt}{\isacharsemicolon}{\kern0pt}\ {\isadigit{0}}{\isasymin}A{\isasymrbrakk}\isanewline
\ \ \ \ {\isasymLongrightarrow}\ sats{\isacharparenleft}{\kern0pt}A{\isacharcomma}{\kern0pt}is{\isacharunderscore}{\kern0pt}HVfrom{\isacharunderscore}{\kern0pt}fm{\isacharparenleft}{\kern0pt}a{\isacharcomma}{\kern0pt}x{\isacharcomma}{\kern0pt}f{\isacharcomma}{\kern0pt}h{\isacharparenright}{\kern0pt}{\isacharcomma}{\kern0pt}env{\isacharparenright}{\kern0pt}\ {\isasymlongleftrightarrow}\isanewline
\ \ \ \ \ \ \ \ is{\isacharunderscore}{\kern0pt}HVfrom{\isacharparenleft}{\kern0pt}{\isacharhash}{\kern0pt}{\isacharhash}{\kern0pt}A{\isacharcomma}{\kern0pt}nth{\isacharparenleft}{\kern0pt}a{\isacharcomma}{\kern0pt}env{\isacharparenright}{\kern0pt}{\isacharcomma}{\kern0pt}nth{\isacharparenleft}{\kern0pt}x{\isacharcomma}{\kern0pt}env{\isacharparenright}{\kern0pt}{\isacharcomma}{\kern0pt}nth{\isacharparenleft}{\kern0pt}f{\isacharcomma}{\kern0pt}env{\isacharparenright}{\kern0pt}{\isacharcomma}{\kern0pt}nth{\isacharparenleft}{\kern0pt}h{\isacharcomma}{\kern0pt}env{\isacharparenright}{\kern0pt}{\isacharparenright}{\kern0pt}{\isachardoublequoteclose}\isanewline
%
\isadelimproof
\ \ %
\endisadelimproof
%
\isatagproof
\isacommand{using}\isamarkupfalse%
\ is{\isacharunderscore}{\kern0pt}HVfrom{\isacharunderscore}{\kern0pt}def\ is{\isacharunderscore}{\kern0pt}HVfrom{\isacharunderscore}{\kern0pt}fm{\isacharunderscore}{\kern0pt}def\ sats{\isacharunderscore}{\kern0pt}Replace{\isacharunderscore}{\kern0pt}fm{\isacharbrackleft}{\kern0pt}OF\ sats{\isacharunderscore}{\kern0pt}is{\isacharunderscore}{\kern0pt}powapply{\isacharunderscore}{\kern0pt}fm{\isacharbrackright}{\kern0pt}\isanewline
\ \ \isacommand{by}\isamarkupfalse%
\ simp%
\endisatagproof
{\isafoldproof}%
%
\isadelimproof
\isanewline
%
\endisadelimproof
\isanewline
\isacommand{lemma}\isamarkupfalse%
\ is{\isacharunderscore}{\kern0pt}HVfrom{\isacharunderscore}{\kern0pt}iff{\isacharunderscore}{\kern0pt}sats{\isacharcolon}{\kern0pt}\isanewline
\ \ \isakeyword{assumes}\isanewline
\ \ \ \ {\isachardoublequoteopen}nth{\isacharparenleft}{\kern0pt}a{\isacharcomma}{\kern0pt}env{\isacharparenright}{\kern0pt}\ {\isacharequal}{\kern0pt}\ aa{\isachardoublequoteclose}\ {\isachardoublequoteopen}nth{\isacharparenleft}{\kern0pt}x{\isacharcomma}{\kern0pt}env{\isacharparenright}{\kern0pt}\ {\isacharequal}{\kern0pt}\ xx{\isachardoublequoteclose}\ {\isachardoublequoteopen}nth{\isacharparenleft}{\kern0pt}f{\isacharcomma}{\kern0pt}env{\isacharparenright}{\kern0pt}\ {\isacharequal}{\kern0pt}\ ff{\isachardoublequoteclose}\ {\isachardoublequoteopen}nth{\isacharparenleft}{\kern0pt}h{\isacharcomma}{\kern0pt}env{\isacharparenright}{\kern0pt}\ {\isacharequal}{\kern0pt}\ hh{\isachardoublequoteclose}\isanewline
\ \ \ \ {\isachardoublequoteopen}a{\isasymin}nat{\isachardoublequoteclose}\ {\isachardoublequoteopen}x{\isasymin}nat{\isachardoublequoteclose}\ {\isachardoublequoteopen}f{\isasymin}nat{\isachardoublequoteclose}\ {\isachardoublequoteopen}h{\isasymin}nat{\isachardoublequoteclose}\ {\isachardoublequoteopen}env{\isasymin}list{\isacharparenleft}{\kern0pt}A{\isacharparenright}{\kern0pt}{\isachardoublequoteclose}\ {\isachardoublequoteopen}{\isadigit{0}}{\isasymin}A{\isachardoublequoteclose}\isanewline
\ \ \isakeyword{shows}\isanewline
\ \ \ \ {\isachardoublequoteopen}is{\isacharunderscore}{\kern0pt}HVfrom{\isacharparenleft}{\kern0pt}{\isacharhash}{\kern0pt}{\isacharhash}{\kern0pt}A{\isacharcomma}{\kern0pt}aa{\isacharcomma}{\kern0pt}xx{\isacharcomma}{\kern0pt}ff{\isacharcomma}{\kern0pt}hh{\isacharparenright}{\kern0pt}\ {\isasymlongleftrightarrow}\ sats{\isacharparenleft}{\kern0pt}A{\isacharcomma}{\kern0pt}\ is{\isacharunderscore}{\kern0pt}HVfrom{\isacharunderscore}{\kern0pt}fm{\isacharparenleft}{\kern0pt}a{\isacharcomma}{\kern0pt}x{\isacharcomma}{\kern0pt}f{\isacharcomma}{\kern0pt}h{\isacharparenright}{\kern0pt}{\isacharcomma}{\kern0pt}\ env{\isacharparenright}{\kern0pt}{\isachardoublequoteclose}\isanewline
%
\isadelimproof
\ \ %
\endisadelimproof
%
\isatagproof
\isacommand{using}\isamarkupfalse%
\ assms\ sats{\isacharunderscore}{\kern0pt}is{\isacharunderscore}{\kern0pt}HVfrom{\isacharunderscore}{\kern0pt}fm\ \isacommand{by}\isamarkupfalse%
\ simp%
\endisatagproof
{\isafoldproof}%
%
\isadelimproof
\isanewline
%
\endisadelimproof
\isanewline
\isanewline
\isacommand{schematic{\isacharunderscore}{\kern0pt}goal}\isamarkupfalse%
\ sats{\isacharunderscore}{\kern0pt}is{\isacharunderscore}{\kern0pt}Vset{\isacharunderscore}{\kern0pt}fm{\isacharunderscore}{\kern0pt}auto{\isacharcolon}{\kern0pt}\isanewline
\ \ \isakeyword{assumes}\isanewline
\ \ \ \ {\isachardoublequoteopen}i{\isasymin}nat{\isachardoublequoteclose}\ {\isachardoublequoteopen}v{\isasymin}nat{\isachardoublequoteclose}\ {\isachardoublequoteopen}env{\isasymin}list{\isacharparenleft}{\kern0pt}A{\isacharparenright}{\kern0pt}{\isachardoublequoteclose}\ {\isachardoublequoteopen}{\isadigit{0}}{\isasymin}A{\isachardoublequoteclose}\isanewline
\ \ \ \ {\isachardoublequoteopen}i\ {\isacharless}{\kern0pt}\ length{\isacharparenleft}{\kern0pt}env{\isacharparenright}{\kern0pt}{\isachardoublequoteclose}\ {\isachardoublequoteopen}v\ {\isacharless}{\kern0pt}\ length{\isacharparenleft}{\kern0pt}env{\isacharparenright}{\kern0pt}{\isachardoublequoteclose}\isanewline
\ \ \isakeyword{shows}\isanewline
\ \ \ \ {\isachardoublequoteopen}is{\isacharunderscore}{\kern0pt}Vset{\isacharparenleft}{\kern0pt}{\isacharhash}{\kern0pt}{\isacharhash}{\kern0pt}A{\isacharcomma}{\kern0pt}nth{\isacharparenleft}{\kern0pt}i{\isacharcomma}{\kern0pt}\ env{\isacharparenright}{\kern0pt}{\isacharcomma}{\kern0pt}nth{\isacharparenleft}{\kern0pt}v{\isacharcomma}{\kern0pt}\ env{\isacharparenright}{\kern0pt}{\isacharparenright}{\kern0pt}\isanewline
\ \ \ \ {\isasymlongleftrightarrow}\ sats{\isacharparenleft}{\kern0pt}A{\isacharcomma}{\kern0pt}{\isacharquery}{\kern0pt}ivs{\isacharunderscore}{\kern0pt}fm{\isacharparenleft}{\kern0pt}i{\isacharcomma}{\kern0pt}v{\isacharparenright}{\kern0pt}{\isacharcomma}{\kern0pt}env{\isacharparenright}{\kern0pt}{\isachardoublequoteclose}\isanewline
%
\isadelimproof
\ \ %
\endisadelimproof
%
\isatagproof
\isacommand{unfolding}\isamarkupfalse%
\ is{\isacharunderscore}{\kern0pt}Vset{\isacharunderscore}{\kern0pt}def\ is{\isacharunderscore}{\kern0pt}Vfrom{\isacharunderscore}{\kern0pt}def\isanewline
\ \ \isacommand{by}\isamarkupfalse%
\ {\isacharparenleft}{\kern0pt}insert\ assms{\isacharsemicolon}{\kern0pt}\ {\isacharparenleft}{\kern0pt}rule\ sep{\isacharunderscore}{\kern0pt}rules\ is{\isacharunderscore}{\kern0pt}HVfrom{\isacharunderscore}{\kern0pt}iff{\isacharunderscore}{\kern0pt}sats\ is{\isacharunderscore}{\kern0pt}transrec{\isacharunderscore}{\kern0pt}iff{\isacharunderscore}{\kern0pt}sats\ {\isacharbar}{\kern0pt}\ simp{\isacharparenright}{\kern0pt}{\isacharplus}{\kern0pt}{\isacharparenright}{\kern0pt}%
\endisatagproof
{\isafoldproof}%
%
\isadelimproof
\isanewline
%
\endisadelimproof
\isanewline
\isacommand{schematic{\isacharunderscore}{\kern0pt}goal}\isamarkupfalse%
\ is{\isacharunderscore}{\kern0pt}Vset{\isacharunderscore}{\kern0pt}iff{\isacharunderscore}{\kern0pt}sats{\isacharcolon}{\kern0pt}\isanewline
\ \ \isakeyword{assumes}\isanewline
\ \ \ \ {\isachardoublequoteopen}nth{\isacharparenleft}{\kern0pt}i{\isacharcomma}{\kern0pt}env{\isacharparenright}{\kern0pt}\ {\isacharequal}{\kern0pt}\ ii{\isachardoublequoteclose}\ {\isachardoublequoteopen}nth{\isacharparenleft}{\kern0pt}v{\isacharcomma}{\kern0pt}env{\isacharparenright}{\kern0pt}\ {\isacharequal}{\kern0pt}\ vv{\isachardoublequoteclose}\isanewline
\ \ \ \ {\isachardoublequoteopen}i{\isasymin}nat{\isachardoublequoteclose}\ {\isachardoublequoteopen}v{\isasymin}nat{\isachardoublequoteclose}\ {\isachardoublequoteopen}env{\isasymin}list{\isacharparenleft}{\kern0pt}A{\isacharparenright}{\kern0pt}{\isachardoublequoteclose}\ {\isachardoublequoteopen}{\isadigit{0}}{\isasymin}A{\isachardoublequoteclose}\isanewline
\ \ \ \ {\isachardoublequoteopen}i\ {\isacharless}{\kern0pt}\ length{\isacharparenleft}{\kern0pt}env{\isacharparenright}{\kern0pt}{\isachardoublequoteclose}\ {\isachardoublequoteopen}v\ {\isacharless}{\kern0pt}\ length{\isacharparenleft}{\kern0pt}env{\isacharparenright}{\kern0pt}{\isachardoublequoteclose}\isanewline
\ \ \isakeyword{shows}\isanewline
\ \ \ \ {\isachardoublequoteopen}is{\isacharunderscore}{\kern0pt}Vset{\isacharparenleft}{\kern0pt}{\isacharhash}{\kern0pt}{\isacharhash}{\kern0pt}A{\isacharcomma}{\kern0pt}ii{\isacharcomma}{\kern0pt}vv{\isacharparenright}{\kern0pt}\ {\isasymlongleftrightarrow}\ sats{\isacharparenleft}{\kern0pt}A{\isacharcomma}{\kern0pt}\ {\isacharquery}{\kern0pt}ivs{\isacharunderscore}{\kern0pt}fm{\isacharparenleft}{\kern0pt}i{\isacharcomma}{\kern0pt}v{\isacharparenright}{\kern0pt}{\isacharcomma}{\kern0pt}\ env{\isacharparenright}{\kern0pt}{\isachardoublequoteclose}\isanewline
%
\isadelimproof
\ \ %
\endisadelimproof
%
\isatagproof
\isacommand{unfolding}\isamarkupfalse%
\ {\isacartoucheopen}nth{\isacharparenleft}{\kern0pt}i{\isacharcomma}{\kern0pt}env{\isacharparenright}{\kern0pt}\ {\isacharequal}{\kern0pt}\ ii{\isacartoucheclose}{\isacharbrackleft}{\kern0pt}symmetric{\isacharbrackright}{\kern0pt}\ {\isacartoucheopen}nth{\isacharparenleft}{\kern0pt}v{\isacharcomma}{\kern0pt}env{\isacharparenright}{\kern0pt}\ {\isacharequal}{\kern0pt}\ vv{\isacartoucheclose}{\isacharbrackleft}{\kern0pt}symmetric{\isacharbrackright}{\kern0pt}\isanewline
\ \ \isacommand{by}\isamarkupfalse%
\ {\isacharparenleft}{\kern0pt}rule\ sats{\isacharunderscore}{\kern0pt}is{\isacharunderscore}{\kern0pt}Vset{\isacharunderscore}{\kern0pt}fm{\isacharunderscore}{\kern0pt}auto{\isacharparenleft}{\kern0pt}{\isadigit{1}}{\isacharparenright}{\kern0pt}{\isacharsemicolon}{\kern0pt}\ simp\ add{\isacharcolon}{\kern0pt}assms{\isacharparenright}{\kern0pt}%
\endisatagproof
{\isafoldproof}%
%
\isadelimproof
\isanewline
%
\endisadelimproof
\isanewline
\isanewline
\isacommand{lemma}\isamarkupfalse%
\ {\isacharparenleft}{\kern0pt}\isakeyword{in}\ M{\isacharunderscore}{\kern0pt}ZF{\isacharunderscore}{\kern0pt}trans{\isacharparenright}{\kern0pt}\ memrel{\isacharunderscore}{\kern0pt}eclose{\isacharunderscore}{\kern0pt}sing\ {\isacharcolon}{\kern0pt}\isanewline
\ \ {\isachardoublequoteopen}a{\isasymin}M\ {\isasymLongrightarrow}\ {\isasymexists}sa{\isasymin}M{\isachardot}{\kern0pt}\ {\isasymexists}esa{\isasymin}M{\isachardot}{\kern0pt}\ {\isasymexists}mesa{\isasymin}M{\isachardot}{\kern0pt}\isanewline
\ \ \ \ \ \ \ upair{\isacharparenleft}{\kern0pt}{\isacharhash}{\kern0pt}{\isacharhash}{\kern0pt}M{\isacharcomma}{\kern0pt}a{\isacharcomma}{\kern0pt}a{\isacharcomma}{\kern0pt}sa{\isacharparenright}{\kern0pt}\ {\isacharampersand}{\kern0pt}\ is{\isacharunderscore}{\kern0pt}eclose{\isacharparenleft}{\kern0pt}{\isacharhash}{\kern0pt}{\isacharhash}{\kern0pt}M{\isacharcomma}{\kern0pt}sa{\isacharcomma}{\kern0pt}esa{\isacharparenright}{\kern0pt}\ {\isacharampersand}{\kern0pt}\ membership{\isacharparenleft}{\kern0pt}{\isacharhash}{\kern0pt}{\isacharhash}{\kern0pt}M{\isacharcomma}{\kern0pt}esa{\isacharcomma}{\kern0pt}mesa{\isacharparenright}{\kern0pt}{\isachardoublequoteclose}\isanewline
%
\isadelimproof
\ \ %
\endisadelimproof
%
\isatagproof
\isacommand{using}\isamarkupfalse%
\ upair{\isacharunderscore}{\kern0pt}ax\ eclose{\isacharunderscore}{\kern0pt}closed\ Memrel{\isacharunderscore}{\kern0pt}closed\ \isacommand{unfolding}\isamarkupfalse%
\ upair{\isacharunderscore}{\kern0pt}ax{\isacharunderscore}{\kern0pt}def\isanewline
\ \ \isacommand{by}\isamarkupfalse%
\ {\isacharparenleft}{\kern0pt}simp\ del{\isacharcolon}{\kern0pt}upair{\isacharunderscore}{\kern0pt}abs{\isacharparenright}{\kern0pt}%
\endisatagproof
{\isafoldproof}%
%
\isadelimproof
\isanewline
%
\endisadelimproof
\isanewline
\isanewline
\isacommand{lemma}\isamarkupfalse%
\ {\isacharparenleft}{\kern0pt}\isakeyword{in}\ M{\isacharunderscore}{\kern0pt}ZF{\isacharunderscore}{\kern0pt}trans{\isacharparenright}{\kern0pt}\ trans{\isacharunderscore}{\kern0pt}repl{\isacharunderscore}{\kern0pt}HVFrom\ {\isacharcolon}{\kern0pt}\isanewline
\ \ \isakeyword{assumes}\isanewline
\ \ \ \ {\isachardoublequoteopen}A{\isasymin}M{\isachardoublequoteclose}\ {\isachardoublequoteopen}i{\isasymin}M{\isachardoublequoteclose}\isanewline
\ \ \isakeyword{shows}\isanewline
\ \ \ \ {\isachardoublequoteopen}transrec{\isacharunderscore}{\kern0pt}replacement{\isacharparenleft}{\kern0pt}{\isacharhash}{\kern0pt}{\isacharhash}{\kern0pt}M{\isacharcomma}{\kern0pt}is{\isacharunderscore}{\kern0pt}HVfrom{\isacharparenleft}{\kern0pt}{\isacharhash}{\kern0pt}{\isacharhash}{\kern0pt}M{\isacharcomma}{\kern0pt}A{\isacharparenright}{\kern0pt}{\isacharcomma}{\kern0pt}i{\isacharparenright}{\kern0pt}{\isachardoublequoteclose}\isanewline
%
\isadelimproof
%
\endisadelimproof
%
\isatagproof
\isacommand{proof}\isamarkupfalse%
\ {\isacharminus}{\kern0pt}\isanewline
\ \ \isacommand{{\isacharbraceleft}{\kern0pt}}\isamarkupfalse%
\ \isacommand{fix}\isamarkupfalse%
\ mesa\isanewline
\ \ \ \ \isacommand{assume}\isamarkupfalse%
\ {\isachardoublequoteopen}mesa{\isasymin}M{\isachardoublequoteclose}\isanewline
\ \ \ \ \isacommand{have}\isamarkupfalse%
\isanewline
\ \ \ \ \ \ {\isadigit{0}}{\isacharcolon}{\kern0pt}{\isachardoublequoteopen}is{\isacharunderscore}{\kern0pt}HVfrom{\isacharparenleft}{\kern0pt}{\isacharhash}{\kern0pt}{\isacharhash}{\kern0pt}M{\isacharcomma}{\kern0pt}A{\isacharcomma}{\kern0pt}a{\isadigit{2}}{\isacharcomma}{\kern0pt}\ a{\isadigit{1}}{\isacharcomma}{\kern0pt}\ a{\isadigit{0}}{\isacharparenright}{\kern0pt}\ {\isasymlongleftrightarrow}\isanewline
\ \ \ \ \ \ sats{\isacharparenleft}{\kern0pt}M{\isacharcomma}{\kern0pt}\ is{\isacharunderscore}{\kern0pt}HVfrom{\isacharunderscore}{\kern0pt}fm{\isacharparenleft}{\kern0pt}{\isadigit{8}}{\isacharcomma}{\kern0pt}{\isadigit{2}}{\isacharcomma}{\kern0pt}{\isadigit{1}}{\isacharcomma}{\kern0pt}{\isadigit{0}}{\isacharparenright}{\kern0pt}{\isacharcomma}{\kern0pt}\ {\isacharbrackleft}{\kern0pt}a{\isadigit{0}}{\isacharcomma}{\kern0pt}a{\isadigit{1}}{\isacharcomma}{\kern0pt}a{\isadigit{2}}{\isacharcomma}{\kern0pt}a{\isadigit{3}}{\isacharcomma}{\kern0pt}a{\isadigit{4}}{\isacharcomma}{\kern0pt}y{\isacharcomma}{\kern0pt}x{\isacharcomma}{\kern0pt}z{\isacharcomma}{\kern0pt}A{\isacharcomma}{\kern0pt}mesa{\isacharbrackright}{\kern0pt}{\isacharparenright}{\kern0pt}{\isachardoublequoteclose}\isanewline
\ \ \ \ \ \ \isakeyword{if}\ {\isachardoublequoteopen}a{\isadigit{4}}{\isasymin}M{\isachardoublequoteclose}\ {\isachardoublequoteopen}a{\isadigit{3}}{\isasymin}M{\isachardoublequoteclose}\ {\isachardoublequoteopen}a{\isadigit{2}}{\isasymin}M{\isachardoublequoteclose}\ {\isachardoublequoteopen}a{\isadigit{1}}{\isasymin}M{\isachardoublequoteclose}\ {\isachardoublequoteopen}a{\isadigit{0}}{\isasymin}M{\isachardoublequoteclose}\ {\isachardoublequoteopen}y{\isasymin}M{\isachardoublequoteclose}\ {\isachardoublequoteopen}x{\isasymin}M{\isachardoublequoteclose}\ {\isachardoublequoteopen}z{\isasymin}M{\isachardoublequoteclose}\ \isakeyword{for}\ a{\isadigit{4}}\ a{\isadigit{3}}\ a{\isadigit{2}}\ a{\isadigit{1}}\ a{\isadigit{0}}\ y\ x\ z\isanewline
\ \ \ \ \ \ \isacommand{using}\isamarkupfalse%
\ that\ zero{\isacharunderscore}{\kern0pt}in{\isacharunderscore}{\kern0pt}M\ sats{\isacharunderscore}{\kern0pt}is{\isacharunderscore}{\kern0pt}HVfrom{\isacharunderscore}{\kern0pt}fm\ {\isacartoucheopen}mesa{\isasymin}M{\isacartoucheclose}\ {\isacartoucheopen}A{\isasymin}M{\isacartoucheclose}\ \isacommand{by}\isamarkupfalse%
\ simp\isanewline
\ \ \ \ \isacommand{have}\isamarkupfalse%
\isanewline
\ \ \ \ \ \ {\isadigit{1}}{\isacharcolon}{\kern0pt}{\isachardoublequoteopen}sats{\isacharparenleft}{\kern0pt}M{\isacharcomma}{\kern0pt}\ is{\isacharunderscore}{\kern0pt}wfrec{\isacharunderscore}{\kern0pt}fm{\isacharparenleft}{\kern0pt}is{\isacharunderscore}{\kern0pt}HVfrom{\isacharunderscore}{\kern0pt}fm{\isacharparenleft}{\kern0pt}{\isadigit{8}}{\isacharcomma}{\kern0pt}{\isadigit{2}}{\isacharcomma}{\kern0pt}{\isadigit{1}}{\isacharcomma}{\kern0pt}{\isadigit{0}}{\isacharparenright}{\kern0pt}{\isacharcomma}{\kern0pt}{\isadigit{4}}{\isacharcomma}{\kern0pt}{\isadigit{1}}{\isacharcomma}{\kern0pt}{\isadigit{0}}{\isacharparenright}{\kern0pt}{\isacharcomma}{\kern0pt}{\isacharbrackleft}{\kern0pt}y{\isacharcomma}{\kern0pt}x{\isacharcomma}{\kern0pt}z{\isacharcomma}{\kern0pt}A{\isacharcomma}{\kern0pt}mesa{\isacharbrackright}{\kern0pt}{\isacharparenright}{\kern0pt}\isanewline
\ \ \ \ \ \ \ \ {\isasymlongleftrightarrow}\ is{\isacharunderscore}{\kern0pt}wfrec{\isacharparenleft}{\kern0pt}{\isacharhash}{\kern0pt}{\isacharhash}{\kern0pt}M{\isacharcomma}{\kern0pt}\ is{\isacharunderscore}{\kern0pt}HVfrom{\isacharparenleft}{\kern0pt}{\isacharhash}{\kern0pt}{\isacharhash}{\kern0pt}M{\isacharcomma}{\kern0pt}A{\isacharparenright}{\kern0pt}{\isacharcomma}{\kern0pt}mesa{\isacharcomma}{\kern0pt}\ x{\isacharcomma}{\kern0pt}\ y{\isacharparenright}{\kern0pt}{\isachardoublequoteclose}\isanewline
\ \ \ \ \ \ \isakeyword{if}\ {\isachardoublequoteopen}y{\isasymin}M{\isachardoublequoteclose}\ {\isachardoublequoteopen}x{\isasymin}M{\isachardoublequoteclose}\ {\isachardoublequoteopen}z{\isasymin}M{\isachardoublequoteclose}\ \isakeyword{for}\ y\ x\ z\isanewline
\ \ \ \ \ \ \isacommand{using}\isamarkupfalse%
\ that\ {\isacartoucheopen}A{\isasymin}M{\isacartoucheclose}\ {\isacartoucheopen}mesa{\isasymin}M{\isacartoucheclose}\ sats{\isacharunderscore}{\kern0pt}is{\isacharunderscore}{\kern0pt}wfrec{\isacharunderscore}{\kern0pt}fm{\isacharbrackleft}{\kern0pt}OF\ {\isadigit{0}}{\isacharbrackright}{\kern0pt}\ \isacommand{by}\isamarkupfalse%
\ simp\isanewline
\ \ \ \ \isacommand{let}\isamarkupfalse%
\isanewline
\ \ \ \ \ \ {\isacharquery}{\kern0pt}f{\isacharequal}{\kern0pt}{\isachardoublequoteopen}Exists{\isacharparenleft}{\kern0pt}And{\isacharparenleft}{\kern0pt}pair{\isacharunderscore}{\kern0pt}fm{\isacharparenleft}{\kern0pt}{\isadigit{1}}{\isacharcomma}{\kern0pt}{\isadigit{0}}{\isacharcomma}{\kern0pt}{\isadigit{2}}{\isacharparenright}{\kern0pt}{\isacharcomma}{\kern0pt}is{\isacharunderscore}{\kern0pt}wfrec{\isacharunderscore}{\kern0pt}fm{\isacharparenleft}{\kern0pt}is{\isacharunderscore}{\kern0pt}HVfrom{\isacharunderscore}{\kern0pt}fm{\isacharparenleft}{\kern0pt}{\isadigit{8}}{\isacharcomma}{\kern0pt}{\isadigit{2}}{\isacharcomma}{\kern0pt}{\isadigit{1}}{\isacharcomma}{\kern0pt}{\isadigit{0}}{\isacharparenright}{\kern0pt}{\isacharcomma}{\kern0pt}{\isadigit{4}}{\isacharcomma}{\kern0pt}{\isadigit{1}}{\isacharcomma}{\kern0pt}{\isadigit{0}}{\isacharparenright}{\kern0pt}{\isacharparenright}{\kern0pt}{\isacharparenright}{\kern0pt}{\isachardoublequoteclose}\isanewline
\ \ \ \ \isacommand{have}\isamarkupfalse%
\ satsf{\isacharcolon}{\kern0pt}{\isachardoublequoteopen}sats{\isacharparenleft}{\kern0pt}M{\isacharcomma}{\kern0pt}\ {\isacharquery}{\kern0pt}f{\isacharcomma}{\kern0pt}\ {\isacharbrackleft}{\kern0pt}x{\isacharcomma}{\kern0pt}z{\isacharcomma}{\kern0pt}A{\isacharcomma}{\kern0pt}mesa{\isacharbrackright}{\kern0pt}{\isacharparenright}{\kern0pt}\isanewline
\ \ \ \ \ \ \ \ \ \ \ \ \ \ {\isasymlongleftrightarrow}\ {\isacharparenleft}{\kern0pt}{\isasymexists}y{\isasymin}M{\isachardot}{\kern0pt}\ pair{\isacharparenleft}{\kern0pt}{\isacharhash}{\kern0pt}{\isacharhash}{\kern0pt}M{\isacharcomma}{\kern0pt}x{\isacharcomma}{\kern0pt}y{\isacharcomma}{\kern0pt}z{\isacharparenright}{\kern0pt}\ {\isacharampersand}{\kern0pt}\ is{\isacharunderscore}{\kern0pt}wfrec{\isacharparenleft}{\kern0pt}{\isacharhash}{\kern0pt}{\isacharhash}{\kern0pt}M{\isacharcomma}{\kern0pt}\ is{\isacharunderscore}{\kern0pt}HVfrom{\isacharparenleft}{\kern0pt}{\isacharhash}{\kern0pt}{\isacharhash}{\kern0pt}M{\isacharcomma}{\kern0pt}A{\isacharparenright}{\kern0pt}\ {\isacharcomma}{\kern0pt}\ mesa{\isacharcomma}{\kern0pt}\ x{\isacharcomma}{\kern0pt}\ y{\isacharparenright}{\kern0pt}{\isacharparenright}{\kern0pt}{\isachardoublequoteclose}\isanewline
\ \ \ \ \ \ \isakeyword{if}\ {\isachardoublequoteopen}x{\isasymin}M{\isachardoublequoteclose}\ {\isachardoublequoteopen}z{\isasymin}M{\isachardoublequoteclose}\ \isakeyword{for}\ x\ z\isanewline
\ \ \ \ \ \ \isacommand{using}\isamarkupfalse%
\ that\ {\isadigit{1}}\ {\isacartoucheopen}A{\isasymin}M{\isacartoucheclose}\ {\isacartoucheopen}mesa{\isasymin}M{\isacartoucheclose}\ \isacommand{by}\isamarkupfalse%
\ {\isacharparenleft}{\kern0pt}simp\ del{\isacharcolon}{\kern0pt}pair{\isacharunderscore}{\kern0pt}abs{\isacharparenright}{\kern0pt}\isanewline
\ \ \ \ \isacommand{have}\isamarkupfalse%
\ {\isachardoublequoteopen}arity{\isacharparenleft}{\kern0pt}{\isacharquery}{\kern0pt}f{\isacharparenright}{\kern0pt}\ {\isacharequal}{\kern0pt}\ {\isadigit{4}}{\isachardoublequoteclose}\isanewline
\ \ \ \ \ \ \isacommand{unfolding}\isamarkupfalse%
\ is{\isacharunderscore}{\kern0pt}HVfrom{\isacharunderscore}{\kern0pt}fm{\isacharunderscore}{\kern0pt}def\ is{\isacharunderscore}{\kern0pt}wfrec{\isacharunderscore}{\kern0pt}fm{\isacharunderscore}{\kern0pt}def\ is{\isacharunderscore}{\kern0pt}recfun{\isacharunderscore}{\kern0pt}fm{\isacharunderscore}{\kern0pt}def\ is{\isacharunderscore}{\kern0pt}nat{\isacharunderscore}{\kern0pt}case{\isacharunderscore}{\kern0pt}fm{\isacharunderscore}{\kern0pt}def\isanewline
\ \ \ \ \ \ \ \ restriction{\isacharunderscore}{\kern0pt}fm{\isacharunderscore}{\kern0pt}def\ list{\isacharunderscore}{\kern0pt}functor{\isacharunderscore}{\kern0pt}fm{\isacharunderscore}{\kern0pt}def\ number{\isadigit{1}}{\isacharunderscore}{\kern0pt}fm{\isacharunderscore}{\kern0pt}def\ cartprod{\isacharunderscore}{\kern0pt}fm{\isacharunderscore}{\kern0pt}def\isanewline
\ \ \ \ \ \ \ \ is{\isacharunderscore}{\kern0pt}powapply{\isacharunderscore}{\kern0pt}fm{\isacharunderscore}{\kern0pt}def\ sum{\isacharunderscore}{\kern0pt}fm{\isacharunderscore}{\kern0pt}def\ quasinat{\isacharunderscore}{\kern0pt}fm{\isacharunderscore}{\kern0pt}def\ pre{\isacharunderscore}{\kern0pt}image{\isacharunderscore}{\kern0pt}fm{\isacharunderscore}{\kern0pt}def\ fm{\isacharunderscore}{\kern0pt}defs\isanewline
\ \ \ \ \ \ \isacommand{by}\isamarkupfalse%
\ {\isacharparenleft}{\kern0pt}simp\ add{\isacharcolon}{\kern0pt}nat{\isacharunderscore}{\kern0pt}simp{\isacharunderscore}{\kern0pt}union{\isacharparenright}{\kern0pt}\isanewline
\ \ \ \ \isacommand{then}\isamarkupfalse%
\isanewline
\ \ \ \ \isacommand{have}\isamarkupfalse%
\ {\isachardoublequoteopen}strong{\isacharunderscore}{\kern0pt}replacement{\isacharparenleft}{\kern0pt}{\isacharhash}{\kern0pt}{\isacharhash}{\kern0pt}M{\isacharcomma}{\kern0pt}{\isasymlambda}x\ z{\isachardot}{\kern0pt}\ sats{\isacharparenleft}{\kern0pt}M{\isacharcomma}{\kern0pt}{\isacharquery}{\kern0pt}f{\isacharcomma}{\kern0pt}{\isacharbrackleft}{\kern0pt}x{\isacharcomma}{\kern0pt}z{\isacharcomma}{\kern0pt}A{\isacharcomma}{\kern0pt}mesa{\isacharbrackright}{\kern0pt}{\isacharparenright}{\kern0pt}{\isacharparenright}{\kern0pt}{\isachardoublequoteclose}\isanewline
\ \ \ \ \ \ \isacommand{using}\isamarkupfalse%
\ replacement{\isacharunderscore}{\kern0pt}ax\ {\isadigit{1}}\ {\isacartoucheopen}A{\isasymin}M{\isacartoucheclose}\ {\isacartoucheopen}mesa{\isasymin}M{\isacartoucheclose}\ \isacommand{by}\isamarkupfalse%
\ simp\isanewline
\ \ \ \ \isacommand{then}\isamarkupfalse%
\isanewline
\ \ \ \ \isacommand{have}\isamarkupfalse%
\ {\isachardoublequoteopen}strong{\isacharunderscore}{\kern0pt}replacement{\isacharparenleft}{\kern0pt}{\isacharhash}{\kern0pt}{\isacharhash}{\kern0pt}M{\isacharcomma}{\kern0pt}{\isasymlambda}x\ z{\isachardot}{\kern0pt}\isanewline
\ \ \ \ \ \ \ \ \ \ {\isasymexists}y{\isasymin}M{\isachardot}{\kern0pt}\ pair{\isacharparenleft}{\kern0pt}{\isacharhash}{\kern0pt}{\isacharhash}{\kern0pt}M{\isacharcomma}{\kern0pt}x{\isacharcomma}{\kern0pt}y{\isacharcomma}{\kern0pt}z{\isacharparenright}{\kern0pt}\ {\isacharampersand}{\kern0pt}\ is{\isacharunderscore}{\kern0pt}wfrec{\isacharparenleft}{\kern0pt}{\isacharhash}{\kern0pt}{\isacharhash}{\kern0pt}M{\isacharcomma}{\kern0pt}\ is{\isacharunderscore}{\kern0pt}HVfrom{\isacharparenleft}{\kern0pt}{\isacharhash}{\kern0pt}{\isacharhash}{\kern0pt}M{\isacharcomma}{\kern0pt}A{\isacharparenright}{\kern0pt}\ {\isacharcomma}{\kern0pt}\ mesa{\isacharcomma}{\kern0pt}\ x{\isacharcomma}{\kern0pt}\ y{\isacharparenright}{\kern0pt}{\isacharparenright}{\kern0pt}{\isachardoublequoteclose}\isanewline
\ \ \ \ \ \ \isacommand{using}\isamarkupfalse%
\ repl{\isacharunderscore}{\kern0pt}sats{\isacharbrackleft}{\kern0pt}of\ M\ {\isacharquery}{\kern0pt}f\ {\isachardoublequoteopen}{\isacharbrackleft}{\kern0pt}A{\isacharcomma}{\kern0pt}mesa{\isacharbrackright}{\kern0pt}{\isachardoublequoteclose}{\isacharbrackright}{\kern0pt}\ \ satsf\ \isacommand{by}\isamarkupfalse%
\ {\isacharparenleft}{\kern0pt}simp\ del{\isacharcolon}{\kern0pt}pair{\isacharunderscore}{\kern0pt}abs{\isacharparenright}{\kern0pt}\isanewline
\ \ \ \ \isacommand{then}\isamarkupfalse%
\isanewline
\ \ \ \ \isacommand{have}\isamarkupfalse%
\ {\isachardoublequoteopen}wfrec{\isacharunderscore}{\kern0pt}replacement{\isacharparenleft}{\kern0pt}{\isacharhash}{\kern0pt}{\isacharhash}{\kern0pt}M{\isacharcomma}{\kern0pt}is{\isacharunderscore}{\kern0pt}HVfrom{\isacharparenleft}{\kern0pt}{\isacharhash}{\kern0pt}{\isacharhash}{\kern0pt}M{\isacharcomma}{\kern0pt}A{\isacharparenright}{\kern0pt}{\isacharcomma}{\kern0pt}mesa{\isacharparenright}{\kern0pt}{\isachardoublequoteclose}\isanewline
\ \ \ \ \ \ \isacommand{unfolding}\isamarkupfalse%
\ wfrec{\isacharunderscore}{\kern0pt}replacement{\isacharunderscore}{\kern0pt}def\ \ \isacommand{by}\isamarkupfalse%
\ simp\isanewline
\ \ \isacommand{{\isacharbraceright}{\kern0pt}}\isamarkupfalse%
\isanewline
\ \ \isacommand{then}\isamarkupfalse%
\ \isacommand{show}\isamarkupfalse%
\ {\isacharquery}{\kern0pt}thesis\ \isacommand{unfolding}\isamarkupfalse%
\ transrec{\isacharunderscore}{\kern0pt}replacement{\isacharunderscore}{\kern0pt}def\isanewline
\ \ \ \ \isacommand{using}\isamarkupfalse%
\ {\isacartoucheopen}i{\isasymin}M{\isacartoucheclose}\ memrel{\isacharunderscore}{\kern0pt}eclose{\isacharunderscore}{\kern0pt}sing\ \isacommand{by}\isamarkupfalse%
\ simp\isanewline
\isacommand{qed}\isamarkupfalse%
%
\endisatagproof
{\isafoldproof}%
%
\isadelimproof
\isanewline
%
\endisadelimproof
\isanewline
\isanewline
\isacommand{lemma}\isamarkupfalse%
\ {\isacharparenleft}{\kern0pt}\isakeyword{in}\ M{\isacharunderscore}{\kern0pt}ZF{\isacharunderscore}{\kern0pt}trans{\isacharparenright}{\kern0pt}\ meclose{\isacharunderscore}{\kern0pt}pow\ {\isacharcolon}{\kern0pt}\ {\isachardoublequoteopen}M{\isacharunderscore}{\kern0pt}eclose{\isacharunderscore}{\kern0pt}pow{\isacharparenleft}{\kern0pt}{\isacharhash}{\kern0pt}{\isacharhash}{\kern0pt}M{\isacharparenright}{\kern0pt}{\isachardoublequoteclose}\isanewline
%
\isadelimproof
\ \ %
\endisadelimproof
%
\isatagproof
\isacommand{using}\isamarkupfalse%
\ meclose\ power{\isacharunderscore}{\kern0pt}ax\ powapply{\isacharunderscore}{\kern0pt}repl\ phrank{\isacharunderscore}{\kern0pt}repl\ trans{\isacharunderscore}{\kern0pt}repl{\isacharunderscore}{\kern0pt}HVFrom\ wfrec{\isacharunderscore}{\kern0pt}rank\isanewline
\ \ \isacommand{by}\isamarkupfalse%
\ unfold{\isacharunderscore}{\kern0pt}locales\ auto%
\endisatagproof
{\isafoldproof}%
%
\isadelimproof
\isanewline
%
\endisadelimproof
\isanewline
\isacommand{sublocale}\isamarkupfalse%
\ M{\isacharunderscore}{\kern0pt}ZF{\isacharunderscore}{\kern0pt}trans\ {\isasymsubseteq}\ M{\isacharunderscore}{\kern0pt}eclose{\isacharunderscore}{\kern0pt}pow\ {\isachardoublequoteopen}{\isacharhash}{\kern0pt}{\isacharhash}{\kern0pt}M{\isachardoublequoteclose}\isanewline
%
\isadelimproof
\ \ %
\endisadelimproof
%
\isatagproof
\isacommand{by}\isamarkupfalse%
\ {\isacharparenleft}{\kern0pt}rule\ meclose{\isacharunderscore}{\kern0pt}pow{\isacharparenright}{\kern0pt}%
\endisatagproof
{\isafoldproof}%
%
\isadelimproof
\isanewline
%
\endisadelimproof
\isanewline
\isacommand{lemma}\isamarkupfalse%
\ {\isacharparenleft}{\kern0pt}\isakeyword{in}\ M{\isacharunderscore}{\kern0pt}ZF{\isacharunderscore}{\kern0pt}trans{\isacharparenright}{\kern0pt}\ repl{\isacharunderscore}{\kern0pt}gen\ {\isacharcolon}{\kern0pt}\isanewline
\ \ \isakeyword{assumes}\isanewline
\ \ \ \ f{\isacharunderscore}{\kern0pt}abs{\isacharcolon}{\kern0pt}\ {\isachardoublequoteopen}{\isasymAnd}x\ y{\isachardot}{\kern0pt}\ {\isasymlbrakk}\ x{\isasymin}M{\isacharsemicolon}{\kern0pt}\ y{\isasymin}M\ {\isasymrbrakk}\ {\isasymLongrightarrow}\ is{\isacharunderscore}{\kern0pt}F{\isacharparenleft}{\kern0pt}{\isacharhash}{\kern0pt}{\isacharhash}{\kern0pt}M{\isacharcomma}{\kern0pt}x{\isacharcomma}{\kern0pt}y{\isacharparenright}{\kern0pt}\ {\isasymlongleftrightarrow}\ y\ {\isacharequal}{\kern0pt}\ f{\isacharparenleft}{\kern0pt}x{\isacharparenright}{\kern0pt}{\isachardoublequoteclose}\isanewline
\ \ \ \ \isakeyword{and}\isanewline
\ \ \ \ f{\isacharunderscore}{\kern0pt}sats{\isacharcolon}{\kern0pt}\ {\isachardoublequoteopen}{\isasymAnd}x\ y{\isachardot}{\kern0pt}\ {\isasymlbrakk}x{\isasymin}M\ {\isacharsemicolon}{\kern0pt}\ y{\isasymin}M\ {\isasymrbrakk}\ {\isasymLongrightarrow}\isanewline
\ \ \ \ \ \ \ \ \ \ \ \ \ sats{\isacharparenleft}{\kern0pt}M{\isacharcomma}{\kern0pt}f{\isacharunderscore}{\kern0pt}fm{\isacharcomma}{\kern0pt}Cons{\isacharparenleft}{\kern0pt}x{\isacharcomma}{\kern0pt}Cons{\isacharparenleft}{\kern0pt}y{\isacharcomma}{\kern0pt}env{\isacharparenright}{\kern0pt}{\isacharparenright}{\kern0pt}{\isacharparenright}{\kern0pt}\ {\isasymlongleftrightarrow}\ is{\isacharunderscore}{\kern0pt}F{\isacharparenleft}{\kern0pt}{\isacharhash}{\kern0pt}{\isacharhash}{\kern0pt}M{\isacharcomma}{\kern0pt}x{\isacharcomma}{\kern0pt}y{\isacharparenright}{\kern0pt}{\isachardoublequoteclose}\isanewline
\ \ \ \ \isakeyword{and}\isanewline
\ \ \ \ f{\isacharunderscore}{\kern0pt}form{\isacharcolon}{\kern0pt}\ {\isachardoublequoteopen}f{\isacharunderscore}{\kern0pt}fm\ {\isasymin}\ formula{\isachardoublequoteclose}\isanewline
\ \ \ \ \isakeyword{and}\isanewline
\ \ \ \ f{\isacharunderscore}{\kern0pt}arty{\isacharcolon}{\kern0pt}\ {\isachardoublequoteopen}arity{\isacharparenleft}{\kern0pt}f{\isacharunderscore}{\kern0pt}fm{\isacharparenright}{\kern0pt}\ {\isacharequal}{\kern0pt}\ {\isadigit{2}}{\isachardoublequoteclose}\isanewline
\ \ \ \ \isakeyword{and}\isanewline
\ \ \ \ {\isachardoublequoteopen}env{\isasymin}list{\isacharparenleft}{\kern0pt}M{\isacharparenright}{\kern0pt}{\isachardoublequoteclose}\isanewline
\ \ \isakeyword{shows}\isanewline
\ \ \ \ {\isachardoublequoteopen}strong{\isacharunderscore}{\kern0pt}replacement{\isacharparenleft}{\kern0pt}{\isacharhash}{\kern0pt}{\isacharhash}{\kern0pt}M{\isacharcomma}{\kern0pt}\ {\isasymlambda}x\ y{\isachardot}{\kern0pt}\ y\ {\isacharequal}{\kern0pt}\ f{\isacharparenleft}{\kern0pt}x{\isacharparenright}{\kern0pt}{\isacharparenright}{\kern0pt}{\isachardoublequoteclose}\isanewline
%
\isadelimproof
%
\endisadelimproof
%
\isatagproof
\isacommand{proof}\isamarkupfalse%
\ {\isacharminus}{\kern0pt}\isanewline
\ \ \isacommand{have}\isamarkupfalse%
\ {\isachardoublequoteopen}sats{\isacharparenleft}{\kern0pt}M{\isacharcomma}{\kern0pt}f{\isacharunderscore}{\kern0pt}fm{\isacharcomma}{\kern0pt}{\isacharbrackleft}{\kern0pt}x{\isacharcomma}{\kern0pt}y{\isacharbrackright}{\kern0pt}{\isacharat}{\kern0pt}env{\isacharparenright}{\kern0pt}\ {\isasymlongleftrightarrow}\ is{\isacharunderscore}{\kern0pt}F{\isacharparenleft}{\kern0pt}{\isacharhash}{\kern0pt}{\isacharhash}{\kern0pt}M{\isacharcomma}{\kern0pt}x{\isacharcomma}{\kern0pt}y{\isacharparenright}{\kern0pt}{\isachardoublequoteclose}\ \isakeyword{if}\ {\isachardoublequoteopen}x{\isasymin}M{\isachardoublequoteclose}\ {\isachardoublequoteopen}y{\isasymin}M{\isachardoublequoteclose}\ \isakeyword{for}\ x\ y\isanewline
\ \ \ \ \isacommand{using}\isamarkupfalse%
\ that\ f{\isacharunderscore}{\kern0pt}sats{\isacharbrackleft}{\kern0pt}of\ x\ y{\isacharbrackright}{\kern0pt}\ \isacommand{by}\isamarkupfalse%
\ simp\isanewline
\ \ \isacommand{moreover}\isamarkupfalse%
\isanewline
\ \ \isacommand{from}\isamarkupfalse%
\ f{\isacharunderscore}{\kern0pt}form\ f{\isacharunderscore}{\kern0pt}arty\isanewline
\ \ \isacommand{have}\isamarkupfalse%
\ {\isachardoublequoteopen}strong{\isacharunderscore}{\kern0pt}replacement{\isacharparenleft}{\kern0pt}{\isacharhash}{\kern0pt}{\isacharhash}{\kern0pt}M{\isacharcomma}{\kern0pt}\ {\isasymlambda}x\ y{\isachardot}{\kern0pt}\ sats{\isacharparenleft}{\kern0pt}M{\isacharcomma}{\kern0pt}f{\isacharunderscore}{\kern0pt}fm{\isacharcomma}{\kern0pt}{\isacharbrackleft}{\kern0pt}x{\isacharcomma}{\kern0pt}y{\isacharbrackright}{\kern0pt}{\isacharat}{\kern0pt}env{\isacharparenright}{\kern0pt}{\isacharparenright}{\kern0pt}{\isachardoublequoteclose}\isanewline
\ \ \ \ \isacommand{using}\isamarkupfalse%
\ {\isacartoucheopen}env{\isasymin}list{\isacharparenleft}{\kern0pt}M{\isacharparenright}{\kern0pt}{\isacartoucheclose}\ replacement{\isacharunderscore}{\kern0pt}ax\ \isacommand{by}\isamarkupfalse%
\ simp\isanewline
\ \ \isacommand{ultimately}\isamarkupfalse%
\isanewline
\ \ \isacommand{have}\isamarkupfalse%
\ {\isachardoublequoteopen}strong{\isacharunderscore}{\kern0pt}replacement{\isacharparenleft}{\kern0pt}{\isacharhash}{\kern0pt}{\isacharhash}{\kern0pt}M{\isacharcomma}{\kern0pt}\ is{\isacharunderscore}{\kern0pt}F{\isacharparenleft}{\kern0pt}{\isacharhash}{\kern0pt}{\isacharhash}{\kern0pt}M{\isacharparenright}{\kern0pt}{\isacharparenright}{\kern0pt}{\isachardoublequoteclose}\isanewline
\ \ \ \ \isacommand{using}\isamarkupfalse%
\ strong{\isacharunderscore}{\kern0pt}replacement{\isacharunderscore}{\kern0pt}cong{\isacharbrackleft}{\kern0pt}of\ {\isachardoublequoteopen}{\isacharhash}{\kern0pt}{\isacharhash}{\kern0pt}M{\isachardoublequoteclose}\ {\isachardoublequoteopen}{\isasymlambda}x\ y{\isachardot}{\kern0pt}\ sats{\isacharparenleft}{\kern0pt}M{\isacharcomma}{\kern0pt}f{\isacharunderscore}{\kern0pt}fm{\isacharcomma}{\kern0pt}{\isacharbrackleft}{\kern0pt}x{\isacharcomma}{\kern0pt}y{\isacharbrackright}{\kern0pt}{\isacharat}{\kern0pt}env{\isacharparenright}{\kern0pt}{\isachardoublequoteclose}\ {\isachardoublequoteopen}is{\isacharunderscore}{\kern0pt}F{\isacharparenleft}{\kern0pt}{\isacharhash}{\kern0pt}{\isacharhash}{\kern0pt}M{\isacharparenright}{\kern0pt}{\isachardoublequoteclose}{\isacharbrackright}{\kern0pt}\ \isacommand{by}\isamarkupfalse%
\ simp\isanewline
\ \ \isacommand{with}\isamarkupfalse%
\ f{\isacharunderscore}{\kern0pt}abs\ \isacommand{show}\isamarkupfalse%
\ {\isacharquery}{\kern0pt}thesis\isanewline
\ \ \ \ \isacommand{using}\isamarkupfalse%
\ strong{\isacharunderscore}{\kern0pt}replacement{\isacharunderscore}{\kern0pt}cong{\isacharbrackleft}{\kern0pt}of\ {\isachardoublequoteopen}{\isacharhash}{\kern0pt}{\isacharhash}{\kern0pt}M{\isachardoublequoteclose}\ {\isachardoublequoteopen}is{\isacharunderscore}{\kern0pt}F{\isacharparenleft}{\kern0pt}{\isacharhash}{\kern0pt}{\isacharhash}{\kern0pt}M{\isacharparenright}{\kern0pt}{\isachardoublequoteclose}\ {\isachardoublequoteopen}{\isasymlambda}x\ y{\isachardot}{\kern0pt}\ y\ {\isacharequal}{\kern0pt}\ f{\isacharparenleft}{\kern0pt}x{\isacharparenright}{\kern0pt}{\isachardoublequoteclose}{\isacharbrackright}{\kern0pt}\ \isacommand{by}\isamarkupfalse%
\ simp\isanewline
\isacommand{qed}\isamarkupfalse%
%
\endisatagproof
{\isafoldproof}%
%
\isadelimproof
\isanewline
%
\endisadelimproof
\isanewline
\isanewline
\isacommand{lemma}\isamarkupfalse%
\ {\isacharparenleft}{\kern0pt}\isakeyword{in}\ M{\isacharunderscore}{\kern0pt}ZF{\isacharunderscore}{\kern0pt}trans{\isacharparenright}{\kern0pt}\ sep{\isacharunderscore}{\kern0pt}in{\isacharunderscore}{\kern0pt}M\ {\isacharcolon}{\kern0pt}\isanewline
\ \ \isakeyword{assumes}\isanewline
\ \ \ \ {\isachardoublequoteopen}{\isasymphi}\ {\isasymin}\ formula{\isachardoublequoteclose}\ {\isachardoublequoteopen}env{\isasymin}list{\isacharparenleft}{\kern0pt}M{\isacharparenright}{\kern0pt}{\isachardoublequoteclose}\isanewline
\ \ \ \ {\isachardoublequoteopen}arity{\isacharparenleft}{\kern0pt}{\isasymphi}{\isacharparenright}{\kern0pt}\ {\isasymle}\ {\isadigit{1}}\ {\isacharhash}{\kern0pt}{\isacharplus}{\kern0pt}\ length{\isacharparenleft}{\kern0pt}env{\isacharparenright}{\kern0pt}{\isachardoublequoteclose}\ {\isachardoublequoteopen}A{\isasymin}M{\isachardoublequoteclose}\ \isakeyword{and}\isanewline
\ \ \ \ satsQ{\isacharcolon}{\kern0pt}\ {\isachardoublequoteopen}{\isasymAnd}x{\isachardot}{\kern0pt}\ x{\isasymin}M\ {\isasymLongrightarrow}\ sats{\isacharparenleft}{\kern0pt}M{\isacharcomma}{\kern0pt}{\isasymphi}{\isacharcomma}{\kern0pt}{\isacharbrackleft}{\kern0pt}x{\isacharbrackright}{\kern0pt}{\isacharat}{\kern0pt}env{\isacharparenright}{\kern0pt}\ {\isasymlongleftrightarrow}\ Q{\isacharparenleft}{\kern0pt}x{\isacharparenright}{\kern0pt}{\isachardoublequoteclose}\isanewline
\ \ \isakeyword{shows}\isanewline
\ \ \ \ {\isachardoublequoteopen}{\isacharbraceleft}{\kern0pt}y{\isasymin}A\ {\isachardot}{\kern0pt}\ Q{\isacharparenleft}{\kern0pt}y{\isacharparenright}{\kern0pt}{\isacharbraceright}{\kern0pt}{\isasymin}M{\isachardoublequoteclose}\isanewline
%
\isadelimproof
%
\endisadelimproof
%
\isatagproof
\isacommand{proof}\isamarkupfalse%
\ {\isacharminus}{\kern0pt}\isanewline
\ \ \isacommand{have}\isamarkupfalse%
\ {\isachardoublequoteopen}separation{\isacharparenleft}{\kern0pt}{\isacharhash}{\kern0pt}{\isacharhash}{\kern0pt}M{\isacharcomma}{\kern0pt}{\isasymlambda}x{\isachardot}{\kern0pt}\ sats{\isacharparenleft}{\kern0pt}M{\isacharcomma}{\kern0pt}{\isasymphi}{\isacharcomma}{\kern0pt}{\isacharbrackleft}{\kern0pt}x{\isacharbrackright}{\kern0pt}\ {\isacharat}{\kern0pt}\ env{\isacharparenright}{\kern0pt}{\isacharparenright}{\kern0pt}{\isachardoublequoteclose}\isanewline
\ \ \ \ \isacommand{using}\isamarkupfalse%
\ assms\ separation{\isacharunderscore}{\kern0pt}ax\ \isacommand{by}\isamarkupfalse%
\ simp\isanewline
\ \ \isacommand{then}\isamarkupfalse%
\ \isacommand{show}\isamarkupfalse%
\ {\isacharquery}{\kern0pt}thesis\ \isacommand{using}\isamarkupfalse%
\isanewline
\ \ \ \ \ \ {\isacartoucheopen}A{\isasymin}M{\isacartoucheclose}\ satsQ\ trans{\isacharunderscore}{\kern0pt}M\isanewline
\ \ \ \ \ \ separation{\isacharunderscore}{\kern0pt}cong{\isacharbrackleft}{\kern0pt}of\ {\isachardoublequoteopen}{\isacharhash}{\kern0pt}{\isacharhash}{\kern0pt}M{\isachardoublequoteclose}\ {\isachardoublequoteopen}{\isasymlambda}y{\isachardot}{\kern0pt}\ sats{\isacharparenleft}{\kern0pt}M{\isacharcomma}{\kern0pt}{\isasymphi}{\isacharcomma}{\kern0pt}{\isacharbrackleft}{\kern0pt}y{\isacharbrackright}{\kern0pt}{\isacharat}{\kern0pt}env{\isacharparenright}{\kern0pt}{\isachardoublequoteclose}\ {\isachardoublequoteopen}Q{\isachardoublequoteclose}{\isacharbrackright}{\kern0pt}\isanewline
\ \ \ \ \ \ separation{\isacharunderscore}{\kern0pt}closed\ \ \isacommand{by}\isamarkupfalse%
\ simp\isanewline
\isacommand{qed}\isamarkupfalse%
%
\endisatagproof
{\isafoldproof}%
%
\isadelimproof
\isanewline
%
\endisadelimproof
%
\isadelimtheory
\isanewline
%
\endisadelimtheory
%
\isatagtheory
\isacommand{end}\isamarkupfalse%
%
\endisatagtheory
{\isafoldtheory}%
%
\isadelimtheory
%
\endisadelimtheory
%
\end{isabellebody}%
\endinput
%:%file=~/source/repos/ZF-notAC/code/Forcing/Interface.thy%:%
%:%11=1%:%
%:%23=3%:%
%:%24=4%:%
%:%25=5%:%
%:%26=6%:%
%:%27=7%:%
%:%35=9%:%
%:%36=9%:%
%:%37=10%:%
%:%38=11%:%
%:%39=12%:%
%:%40=13%:%
%:%41=14%:%
%:%48=14%:%
%:%49=15%:%
%:%50=16%:%
%:%51=16%:%
%:%52=17%:%
%:%53=18%:%
%:%54=18%:%
%:%55=19%:%
%:%56=20%:%
%:%57=21%:%
%:%58=21%:%
%:%59=22%:%
%:%60=23%:%
%:%61=24%:%
%:%62=24%:%
%:%63=25%:%
%:%64=26%:%
%:%65=27%:%
%:%66=27%:%
%:%67=28%:%
%:%68=29%:%
%:%69=30%:%
%:%70=30%:%
%:%71=31%:%
%:%72=32%:%
%:%73=33%:%
%:%74=33%:%
%:%75=34%:%
%:%76=35%:%
%:%77=36%:%
%:%78=37%:%
%:%79=37%:%
%:%80=38%:%
%:%81=39%:%
%:%82=40%:%
%:%83=41%:%
%:%84=42%:%
%:%85=42%:%
%:%86=43%:%
%:%87=44%:%
%:%88=45%:%
%:%89=46%:%
%:%90=46%:%
%:%91=47%:%
%:%92=48%:%
%:%93=48%:%
%:%94=49%:%
%:%97=50%:%
%:%101=50%:%
%:%102=50%:%
%:%103=51%:%
%:%104=51%:%
%:%109=51%:%
%:%112=52%:%
%:%113=53%:%
%:%114=53%:%
%:%115=54%:%
%:%116=55%:%
%:%117=55%:%
%:%118=56%:%
%:%119=57%:%
%:%121=59%:%
%:%122=60%:%
%:%123=61%:%
%:%124=61%:%
%:%125=62%:%
%:%126=63%:%
%:%129=64%:%
%:%133=64%:%
%:%134=64%:%
%:%139=64%:%
%:%142=65%:%
%:%143=66%:%
%:%144=66%:%
%:%145=67%:%
%:%148=68%:%
%:%152=68%:%
%:%153=68%:%
%:%158=68%:%
%:%161=69%:%
%:%162=70%:%
%:%163=70%:%
%:%164=71%:%
%:%165=72%:%
%:%166=73%:%
%:%167=74%:%
%:%168=75%:%
%:%169=76%:%
%:%170=77%:%
%:%171=78%:%
%:%172=79%:%
%:%173=80%:%
%:%174=81%:%
%:%175=82%:%
%:%176=83%:%
%:%177=84%:%
%:%178=85%:%
%:%179=86%:%
%:%180=87%:%
%:%181=87%:%
%:%182=88%:%
%:%185=89%:%
%:%189=89%:%
%:%190=89%:%
%:%195=89%:%
%:%198=90%:%
%:%199=91%:%
%:%200=91%:%
%:%207=92%:%
%:%208=92%:%
%:%209=93%:%
%:%210=93%:%
%:%211=93%:%
%:%212=94%:%
%:%213=95%:%
%:%214=95%:%
%:%215=96%:%
%:%216=96%:%
%:%217=96%:%
%:%218=97%:%
%:%219=98%:%
%:%220=98%:%
%:%221=99%:%
%:%222=99%:%
%:%223=99%:%
%:%224=100%:%
%:%225=100%:%
%:%226=101%:%
%:%227=101%:%
%:%228=101%:%
%:%229=102%:%
%:%230=102%:%
%:%231=103%:%
%:%246=105%:%
%:%256=106%:%
%:%257=106%:%
%:%258=107%:%
%:%261=108%:%
%:%265=108%:%
%:%266=108%:%
%:%267=109%:%
%:%268=109%:%
%:%273=109%:%
%:%276=110%:%
%:%277=111%:%
%:%278=112%:%
%:%279=112%:%
%:%280=113%:%
%:%283=114%:%
%:%287=114%:%
%:%288=114%:%
%:%289=115%:%
%:%290=115%:%
%:%295=115%:%
%:%298=116%:%
%:%299=117%:%
%:%300=117%:%
%:%301=118%:%
%:%302=119%:%
%:%303=119%:%
%:%306=120%:%
%:%310=120%:%
%:%311=120%:%
%:%316=120%:%
%:%319=121%:%
%:%320=122%:%
%:%321=122%:%
%:%322=123%:%
%:%329=125%:%
%:%339=128%:%
%:%340=128%:%
%:%341=129%:%
%:%342=130%:%
%:%343=131%:%
%:%344=132%:%
%:%345=133%:%
%:%348=134%:%
%:%352=134%:%
%:%353=134%:%
%:%358=134%:%
%:%361=135%:%
%:%362=136%:%
%:%363=136%:%
%:%364=137%:%
%:%365=138%:%
%:%366=139%:%
%:%367=140%:%
%:%374=141%:%
%:%375=141%:%
%:%376=142%:%
%:%377=142%:%
%:%378=143%:%
%:%379=144%:%
%:%380=145%:%
%:%381=146%:%
%:%382=147%:%
%:%383=148%:%
%:%384=149%:%
%:%385=149%:%
%:%386=150%:%
%:%387=150%:%
%:%388=151%:%
%:%389=151%:%
%:%390=152%:%
%:%391=152%:%
%:%392=153%:%
%:%393=153%:%
%:%394=153%:%
%:%395=154%:%
%:%396=154%:%
%:%397=155%:%
%:%398=155%:%
%:%399=156%:%
%:%400=157%:%
%:%401=157%:%
%:%402=157%:%
%:%403=158%:%
%:%404=158%:%
%:%405=159%:%
%:%406=159%:%
%:%407=160%:%
%:%408=160%:%
%:%409=160%:%
%:%410=161%:%
%:%411=161%:%
%:%412=161%:%
%:%413=161%:%
%:%414=162%:%
%:%420=162%:%
%:%423=163%:%
%:%424=164%:%
%:%425=165%:%
%:%426=166%:%
%:%427=166%:%
%:%428=167%:%
%:%429=168%:%
%:%430=169%:%
%:%431=170%:%
%:%432=171%:%
%:%435=172%:%
%:%439=172%:%
%:%440=172%:%
%:%445=172%:%
%:%448=173%:%
%:%449=174%:%
%:%450=174%:%
%:%451=175%:%
%:%452=176%:%
%:%453=177%:%
%:%454=178%:%
%:%461=179%:%
%:%462=179%:%
%:%463=180%:%
%:%464=180%:%
%:%465=181%:%
%:%466=182%:%
%:%467=183%:%
%:%468=184%:%
%:%469=185%:%
%:%470=186%:%
%:%471=187%:%
%:%472=187%:%
%:%473=188%:%
%:%474=188%:%
%:%475=189%:%
%:%476=189%:%
%:%477=190%:%
%:%478=190%:%
%:%479=191%:%
%:%480=191%:%
%:%481=191%:%
%:%482=192%:%
%:%483=192%:%
%:%484=193%:%
%:%485=193%:%
%:%486=194%:%
%:%487=195%:%
%:%488=195%:%
%:%489=195%:%
%:%490=196%:%
%:%491=196%:%
%:%492=197%:%
%:%493=197%:%
%:%494=198%:%
%:%495=198%:%
%:%496=198%:%
%:%497=199%:%
%:%498=199%:%
%:%499=199%:%
%:%500=199%:%
%:%501=200%:%
%:%507=200%:%
%:%510=201%:%
%:%511=202%:%
%:%512=202%:%
%:%513=203%:%
%:%514=204%:%
%:%515=205%:%
%:%516=206%:%
%:%517=207%:%
%:%520=208%:%
%:%524=208%:%
%:%525=208%:%
%:%530=208%:%
%:%533=209%:%
%:%534=210%:%
%:%535=211%:%
%:%536=211%:%
%:%537=212%:%
%:%538=213%:%
%:%539=214%:%
%:%540=215%:%
%:%541=216%:%
%:%542=217%:%
%:%549=218%:%
%:%550=218%:%
%:%551=219%:%
%:%552=219%:%
%:%553=220%:%
%:%555=222%:%
%:%556=223%:%
%:%557=224%:%
%:%558=225%:%
%:%559=226%:%
%:%560=227%:%
%:%561=227%:%
%:%562=227%:%
%:%563=228%:%
%:%564=228%:%
%:%565=229%:%
%:%566=229%:%
%:%567=230%:%
%:%568=230%:%
%:%569=230%:%
%:%570=231%:%
%:%571=231%:%
%:%572=232%:%
%:%573=232%:%
%:%574=233%:%
%:%575=234%:%
%:%576=234%:%
%:%577=234%:%
%:%578=235%:%
%:%579=235%:%
%:%580=236%:%
%:%581=236%:%
%:%582=237%:%
%:%583=237%:%
%:%584=237%:%
%:%585=238%:%
%:%586=238%:%
%:%587=238%:%
%:%588=238%:%
%:%589=239%:%
%:%595=239%:%
%:%598=240%:%
%:%599=241%:%
%:%600=241%:%
%:%601=242%:%
%:%602=243%:%
%:%603=244%:%
%:%604=245%:%
%:%605=246%:%
%:%608=247%:%
%:%612=247%:%
%:%613=247%:%
%:%618=247%:%
%:%621=248%:%
%:%622=249%:%
%:%623=249%:%
%:%624=250%:%
%:%625=251%:%
%:%626=252%:%
%:%627=253%:%
%:%628=254%:%
%:%629=255%:%
%:%636=256%:%
%:%637=256%:%
%:%638=257%:%
%:%639=257%:%
%:%640=258%:%
%:%642=260%:%
%:%643=261%:%
%:%644=262%:%
%:%645=263%:%
%:%646=264%:%
%:%647=265%:%
%:%648=265%:%
%:%649=265%:%
%:%650=266%:%
%:%651=266%:%
%:%652=267%:%
%:%653=267%:%
%:%654=268%:%
%:%655=268%:%
%:%656=268%:%
%:%657=269%:%
%:%658=269%:%
%:%659=270%:%
%:%660=270%:%
%:%661=271%:%
%:%662=272%:%
%:%663=272%:%
%:%664=272%:%
%:%665=273%:%
%:%666=273%:%
%:%667=274%:%
%:%668=274%:%
%:%669=275%:%
%:%670=275%:%
%:%671=275%:%
%:%672=276%:%
%:%673=276%:%
%:%674=276%:%
%:%675=276%:%
%:%676=277%:%
%:%682=277%:%
%:%685=278%:%
%:%686=279%:%
%:%687=279%:%
%:%688=280%:%
%:%689=281%:%
%:%690=282%:%
%:%691=283%:%
%:%692=284%:%
%:%693=285%:%
%:%696=286%:%
%:%700=286%:%
%:%701=286%:%
%:%706=286%:%
%:%709=287%:%
%:%710=288%:%
%:%711=289%:%
%:%712=289%:%
%:%713=290%:%
%:%714=291%:%
%:%715=292%:%
%:%716=293%:%
%:%723=294%:%
%:%724=294%:%
%:%725=295%:%
%:%726=295%:%
%:%727=296%:%
%:%729=298%:%
%:%730=299%:%
%:%731=300%:%
%:%732=301%:%
%:%733=302%:%
%:%734=303%:%
%:%735=303%:%
%:%736=303%:%
%:%737=304%:%
%:%738=304%:%
%:%739=305%:%
%:%740=305%:%
%:%741=306%:%
%:%742=306%:%
%:%743=306%:%
%:%744=307%:%
%:%745=307%:%
%:%746=308%:%
%:%747=308%:%
%:%748=309%:%
%:%749=310%:%
%:%750=311%:%
%:%751=311%:%
%:%752=311%:%
%:%753=312%:%
%:%754=312%:%
%:%755=313%:%
%:%756=313%:%
%:%757=314%:%
%:%758=314%:%
%:%759=314%:%
%:%760=315%:%
%:%761=315%:%
%:%762=315%:%
%:%763=315%:%
%:%764=316%:%
%:%770=316%:%
%:%773=317%:%
%:%774=318%:%
%:%775=319%:%
%:%776=319%:%
%:%777=320%:%
%:%778=321%:%
%:%779=322%:%
%:%780=323%:%
%:%781=324%:%
%:%782=325%:%
%:%785=326%:%
%:%789=326%:%
%:%790=326%:%
%:%795=326%:%
%:%798=327%:%
%:%799=328%:%
%:%800=329%:%
%:%801=329%:%
%:%802=330%:%
%:%803=331%:%
%:%804=332%:%
%:%805=333%:%
%:%812=334%:%
%:%813=334%:%
%:%814=335%:%
%:%815=335%:%
%:%816=336%:%
%:%818=338%:%
%:%819=339%:%
%:%820=340%:%
%:%821=341%:%
%:%822=342%:%
%:%823=343%:%
%:%824=343%:%
%:%825=343%:%
%:%826=344%:%
%:%827=344%:%
%:%828=345%:%
%:%829=345%:%
%:%830=346%:%
%:%831=346%:%
%:%832=346%:%
%:%833=347%:%
%:%834=347%:%
%:%835=348%:%
%:%836=348%:%
%:%837=349%:%
%:%838=350%:%
%:%839=351%:%
%:%840=351%:%
%:%841=351%:%
%:%842=352%:%
%:%843=352%:%
%:%844=353%:%
%:%845=353%:%
%:%846=354%:%
%:%847=354%:%
%:%848=354%:%
%:%849=355%:%
%:%850=355%:%
%:%851=355%:%
%:%852=355%:%
%:%853=356%:%
%:%859=356%:%
%:%862=357%:%
%:%863=358%:%
%:%864=358%:%
%:%865=359%:%
%:%866=360%:%
%:%867=361%:%
%:%868=362%:%
%:%869=363%:%
%:%871=365%:%
%:%874=366%:%
%:%878=366%:%
%:%879=366%:%
%:%884=366%:%
%:%887=367%:%
%:%888=368%:%
%:%889=369%:%
%:%890=369%:%
%:%891=370%:%
%:%892=371%:%
%:%893=372%:%
%:%894=373%:%
%:%895=374%:%
%:%896=375%:%
%:%897=376%:%
%:%904=377%:%
%:%905=377%:%
%:%906=378%:%
%:%907=378%:%
%:%908=379%:%
%:%911=382%:%
%:%912=383%:%
%:%913=384%:%
%:%914=385%:%
%:%915=386%:%
%:%916=387%:%
%:%917=387%:%
%:%918=387%:%
%:%919=388%:%
%:%920=388%:%
%:%921=389%:%
%:%922=389%:%
%:%923=390%:%
%:%924=390%:%
%:%925=390%:%
%:%926=391%:%
%:%927=391%:%
%:%928=392%:%
%:%929=392%:%
%:%931=394%:%
%:%932=395%:%
%:%933=396%:%
%:%934=396%:%
%:%935=396%:%
%:%936=397%:%
%:%937=397%:%
%:%938=398%:%
%:%939=398%:%
%:%940=399%:%
%:%941=400%:%
%:%942=400%:%
%:%943=400%:%
%:%944=401%:%
%:%945=401%:%
%:%946=401%:%
%:%947=401%:%
%:%948=402%:%
%:%954=402%:%
%:%957=403%:%
%:%958=404%:%
%:%959=405%:%
%:%960=405%:%
%:%961=406%:%
%:%962=407%:%
%:%963=408%:%
%:%964=409%:%
%:%965=410%:%
%:%968=411%:%
%:%972=411%:%
%:%973=411%:%
%:%978=411%:%
%:%981=412%:%
%:%982=413%:%
%:%983=414%:%
%:%984=414%:%
%:%985=415%:%
%:%986=416%:%
%:%987=417%:%
%:%988=418%:%
%:%989=419%:%
%:%990=420%:%
%:%997=421%:%
%:%998=421%:%
%:%999=422%:%
%:%1000=422%:%
%:%1001=423%:%
%:%1002=424%:%
%:%1003=425%:%
%:%1004=426%:%
%:%1005=427%:%
%:%1006=428%:%
%:%1007=429%:%
%:%1008=429%:%
%:%1009=429%:%
%:%1010=430%:%
%:%1011=430%:%
%:%1012=431%:%
%:%1013=431%:%
%:%1014=432%:%
%:%1015=432%:%
%:%1016=432%:%
%:%1017=433%:%
%:%1018=433%:%
%:%1019=434%:%
%:%1020=434%:%
%:%1021=435%:%
%:%1022=436%:%
%:%1023=437%:%
%:%1024=437%:%
%:%1025=437%:%
%:%1026=438%:%
%:%1027=438%:%
%:%1028=439%:%
%:%1029=439%:%
%:%1030=440%:%
%:%1031=440%:%
%:%1032=440%:%
%:%1033=441%:%
%:%1034=441%:%
%:%1035=441%:%
%:%1036=441%:%
%:%1037=442%:%
%:%1043=442%:%
%:%1046=443%:%
%:%1047=446%:%
%:%1048=447%:%
%:%1049=447%:%
%:%1050=448%:%
%:%1051=449%:%
%:%1052=450%:%
%:%1053=451%:%
%:%1056=452%:%
%:%1060=452%:%
%:%1061=452%:%
%:%1066=452%:%
%:%1069=453%:%
%:%1070=454%:%
%:%1071=454%:%
%:%1072=455%:%
%:%1079=456%:%
%:%1080=456%:%
%:%1081=457%:%
%:%1082=457%:%
%:%1083=458%:%
%:%1084=459%:%
%:%1085=460%:%
%:%1086=461%:%
%:%1087=462%:%
%:%1088=463%:%
%:%1089=464%:%
%:%1090=464%:%
%:%1091=464%:%
%:%1092=465%:%
%:%1093=465%:%
%:%1094=466%:%
%:%1095=466%:%
%:%1096=467%:%
%:%1097=467%:%
%:%1098=467%:%
%:%1099=468%:%
%:%1100=468%:%
%:%1101=469%:%
%:%1102=469%:%
%:%1103=470%:%
%:%1104=471%:%
%:%1105=471%:%
%:%1106=471%:%
%:%1107=472%:%
%:%1108=472%:%
%:%1109=473%:%
%:%1110=473%:%
%:%1111=474%:%
%:%1112=474%:%
%:%1113=474%:%
%:%1114=475%:%
%:%1115=475%:%
%:%1116=475%:%
%:%1117=475%:%
%:%1118=476%:%
%:%1124=476%:%
%:%1127=477%:%
%:%1128=478%:%
%:%1129=478%:%
%:%1130=479%:%
%:%1131=480%:%
%:%1132=481%:%
%:%1133=482%:%
%:%1134=483%:%
%:%1136=485%:%
%:%1139=486%:%
%:%1143=486%:%
%:%1144=486%:%
%:%1149=486%:%
%:%1152=487%:%
%:%1153=488%:%
%:%1154=489%:%
%:%1155=489%:%
%:%1156=490%:%
%:%1157=491%:%
%:%1158=492%:%
%:%1159=493%:%
%:%1162=496%:%
%:%1169=497%:%
%:%1170=497%:%
%:%1171=498%:%
%:%1172=498%:%
%:%1173=499%:%
%:%1177=503%:%
%:%1178=504%:%
%:%1179=505%:%
%:%1180=506%:%
%:%1181=507%:%
%:%1182=508%:%
%:%1183=508%:%
%:%1184=508%:%
%:%1185=509%:%
%:%1186=509%:%
%:%1187=510%:%
%:%1188=510%:%
%:%1189=511%:%
%:%1190=512%:%
%:%1191=512%:%
%:%1192=512%:%
%:%1193=513%:%
%:%1194=513%:%
%:%1195=514%:%
%:%1196=514%:%
%:%1198=516%:%
%:%1199=517%:%
%:%1200=518%:%
%:%1201=518%:%
%:%1202=518%:%
%:%1203=519%:%
%:%1204=519%:%
%:%1205=520%:%
%:%1206=520%:%
%:%1208=522%:%
%:%1209=523%:%
%:%1210=523%:%
%:%1211=523%:%
%:%1212=524%:%
%:%1213=524%:%
%:%1214=524%:%
%:%1215=524%:%
%:%1216=525%:%
%:%1222=525%:%
%:%1225=526%:%
%:%1226=527%:%
%:%1227=528%:%
%:%1228=529%:%
%:%1229=530%:%
%:%1230=530%:%
%:%1231=531%:%
%:%1232=532%:%
%:%1233=533%:%
%:%1234=534%:%
%:%1235=535%:%
%:%1236=536%:%
%:%1239=537%:%
%:%1243=537%:%
%:%1244=537%:%
%:%1249=537%:%
%:%1252=538%:%
%:%1253=539%:%
%:%1254=540%:%
%:%1255=540%:%
%:%1256=541%:%
%:%1257=542%:%
%:%1258=543%:%
%:%1259=544%:%
%:%1262=547%:%
%:%1269=548%:%
%:%1270=548%:%
%:%1271=549%:%
%:%1272=549%:%
%:%1273=550%:%
%:%1276=553%:%
%:%1277=554%:%
%:%1278=555%:%
%:%1279=555%:%
%:%1280=555%:%
%:%1281=556%:%
%:%1282=556%:%
%:%1283=557%:%
%:%1284=557%:%
%:%1285=558%:%
%:%1286=558%:%
%:%1287=558%:%
%:%1288=559%:%
%:%1289=559%:%
%:%1290=560%:%
%:%1291=560%:%
%:%1293=562%:%
%:%1294=563%:%
%:%1295=564%:%
%:%1296=564%:%
%:%1297=564%:%
%:%1298=565%:%
%:%1299=565%:%
%:%1300=566%:%
%:%1301=566%:%
%:%1303=568%:%
%:%1304=569%:%
%:%1305=569%:%
%:%1306=569%:%
%:%1307=570%:%
%:%1308=570%:%
%:%1309=570%:%
%:%1310=570%:%
%:%1311=571%:%
%:%1317=571%:%
%:%1320=572%:%
%:%1321=573%:%
%:%1322=574%:%
%:%1323=575%:%
%:%1324=576%:%
%:%1325=576%:%
%:%1326=577%:%
%:%1327=578%:%
%:%1328=579%:%
%:%1329=580%:%
%:%1330=581%:%
%:%1331=581%:%
%:%1334=582%:%
%:%1338=582%:%
%:%1339=582%:%
%:%1340=583%:%
%:%1341=583%:%
%:%1346=583%:%
%:%1349=584%:%
%:%1350=585%:%
%:%1351=585%:%
%:%1352=586%:%
%:%1353=587%:%
%:%1354=587%:%
%:%1357=588%:%
%:%1361=588%:%
%:%1362=588%:%
%:%1376=590%:%
%:%1386=593%:%
%:%1387=593%:%
%:%1388=594%:%
%:%1389=595%:%
%:%1390=596%:%
%:%1391=597%:%
%:%1392=598%:%
%:%1395=599%:%
%:%1399=599%:%
%:%1400=599%:%
%:%1401=600%:%
%:%1402=600%:%
%:%1407=600%:%
%:%1410=601%:%
%:%1411=602%:%
%:%1412=603%:%
%:%1413=603%:%
%:%1414=604%:%
%:%1415=605%:%
%:%1416=606%:%
%:%1417=607%:%
%:%1418=608%:%
%:%1419=609%:%
%:%1426=610%:%
%:%1427=610%:%
%:%1428=611%:%
%:%1429=611%:%
%:%1430=612%:%
%:%1431=613%:%
%:%1432=614%:%
%:%1433=615%:%
%:%1434=616%:%
%:%1435=617%:%
%:%1436=618%:%
%:%1437=618%:%
%:%1438=618%:%
%:%1439=619%:%
%:%1440=619%:%
%:%1441=620%:%
%:%1442=620%:%
%:%1443=621%:%
%:%1444=621%:%
%:%1445=621%:%
%:%1446=622%:%
%:%1447=622%:%
%:%1448=623%:%
%:%1449=623%:%
%:%1450=624%:%
%:%1451=625%:%
%:%1452=626%:%
%:%1453=626%:%
%:%1454=626%:%
%:%1455=627%:%
%:%1456=627%:%
%:%1457=628%:%
%:%1458=628%:%
%:%1459=629%:%
%:%1460=629%:%
%:%1461=629%:%
%:%1462=630%:%
%:%1463=630%:%
%:%1464=630%:%
%:%1465=630%:%
%:%1466=631%:%
%:%1472=631%:%
%:%1475=632%:%
%:%1476=633%:%
%:%1477=633%:%
%:%1478=634%:%
%:%1479=635%:%
%:%1480=636%:%
%:%1481=637%:%
%:%1482=638%:%
%:%1485=639%:%
%:%1489=639%:%
%:%1490=639%:%
%:%1491=640%:%
%:%1492=640%:%
%:%1497=640%:%
%:%1500=641%:%
%:%1501=642%:%
%:%1502=642%:%
%:%1503=643%:%
%:%1504=644%:%
%:%1505=645%:%
%:%1506=646%:%
%:%1507=647%:%
%:%1510=648%:%
%:%1514=648%:%
%:%1515=648%:%
%:%1516=649%:%
%:%1517=649%:%
%:%1522=649%:%
%:%1527=650%:%
%:%1532=651%:%
%:%1533=651%:%
%:%1538=651%:%
%:%1541=652%:%
%:%1542=653%:%
%:%1543=653%:%
%:%1544=654%:%
%:%1545=655%:%
%:%1546=656%:%
%:%1547=657%:%
%:%1548=658%:%
%:%1551=659%:%
%:%1555=659%:%
%:%1556=659%:%
%:%1557=660%:%
%:%1558=660%:%
%:%1563=660%:%
%:%1566=661%:%
%:%1567=662%:%
%:%1568=662%:%
%:%1569=663%:%
%:%1570=664%:%
%:%1571=665%:%
%:%1572=666%:%
%:%1573=667%:%
%:%1574=668%:%
%:%1581=669%:%
%:%1582=669%:%
%:%1583=670%:%
%:%1584=670%:%
%:%1585=671%:%
%:%1586=672%:%
%:%1587=673%:%
%:%1588=674%:%
%:%1589=675%:%
%:%1590=676%:%
%:%1591=677%:%
%:%1592=677%:%
%:%1593=677%:%
%:%1594=678%:%
%:%1595=678%:%
%:%1596=679%:%
%:%1597=679%:%
%:%1598=680%:%
%:%1599=680%:%
%:%1600=681%:%
%:%1601=681%:%
%:%1602=681%:%
%:%1603=682%:%
%:%1604=682%:%
%:%1605=683%:%
%:%1606=683%:%
%:%1607=684%:%
%:%1608=685%:%
%:%1609=686%:%
%:%1610=686%:%
%:%1611=686%:%
%:%1612=687%:%
%:%1613=687%:%
%:%1614=688%:%
%:%1615=688%:%
%:%1616=689%:%
%:%1617=689%:%
%:%1618=689%:%
%:%1619=690%:%
%:%1620=690%:%
%:%1621=690%:%
%:%1622=690%:%
%:%1623=691%:%
%:%1629=691%:%
%:%1632=692%:%
%:%1633=693%:%
%:%1634=694%:%
%:%1635=695%:%
%:%1636=695%:%
%:%1637=696%:%
%:%1644=697%:%
%:%1645=697%:%
%:%1646=698%:%
%:%1647=698%:%
%:%1648=699%:%
%:%1649=699%:%
%:%1650=700%:%
%:%1651=701%:%
%:%1652=701%:%
%:%1653=702%:%
%:%1654=702%:%
%:%1655=703%:%
%:%1656=703%:%
%:%1657=704%:%
%:%1658=704%:%
%:%1659=705%:%
%:%1660=705%:%
%:%1661=705%:%
%:%1662=706%:%
%:%1663=706%:%
%:%1664=706%:%
%:%1665=707%:%
%:%1666=707%:%
%:%1667=707%:%
%:%1668=708%:%
%:%1669=708%:%
%:%1670=708%:%
%:%1671=709%:%
%:%1672=709%:%
%:%1673=709%:%
%:%1674=709%:%
%:%1675=709%:%
%:%1676=710%:%
%:%1682=710%:%
%:%1685=711%:%
%:%1686=712%:%
%:%1687=713%:%
%:%1688=713%:%
%:%1689=714%:%
%:%1692=715%:%
%:%1696=715%:%
%:%1697=715%:%
%:%1702=715%:%
%:%1705=716%:%
%:%1706=717%:%
%:%1707=718%:%
%:%1708=718%:%
%:%1709=719%:%
%:%1716=720%:%
%:%1717=720%:%
%:%1718=721%:%
%:%1719=721%:%
%:%1720=722%:%
%:%1721=722%:%
%:%1722=722%:%
%:%1723=722%:%
%:%1724=723%:%
%:%1725=723%:%
%:%1726=723%:%
%:%1727=724%:%
%:%1728=725%:%
%:%1729=725%:%
%:%1730=726%:%
%:%1731=726%:%
%:%1732=726%:%
%:%1733=727%:%
%:%1734=727%:%
%:%1735=727%:%
%:%1736=728%:%
%:%1737=728%:%
%:%1738=728%:%
%:%1739=729%:%
%:%1740=729%:%
%:%1741=729%:%
%:%1742=730%:%
%:%1743=730%:%
%:%1744=730%:%
%:%1745=730%:%
%:%1746=730%:%
%:%1747=731%:%
%:%1753=731%:%
%:%1756=732%:%
%:%1757=733%:%
%:%1758=733%:%
%:%1759=734%:%
%:%1766=735%:%
%:%1767=735%:%
%:%1768=736%:%
%:%1769=736%:%
%:%1770=737%:%
%:%1771=737%:%
%:%1772=737%:%
%:%1773=738%:%
%:%1774=738%:%
%:%1775=739%:%
%:%1776=740%:%
%:%1777=740%:%
%:%1778=740%:%
%:%1779=741%:%
%:%1780=741%:%
%:%1781=741%:%
%:%1782=742%:%
%:%1783=742%:%
%:%1784=742%:%
%:%1785=743%:%
%:%1786=743%:%
%:%1787=743%:%
%:%1788=744%:%
%:%1789=744%:%
%:%1790=744%:%
%:%1791=745%:%
%:%1797=745%:%
%:%1800=746%:%
%:%1801=747%:%
%:%1802=748%:%
%:%1803=749%:%
%:%1804=749%:%
%:%1807=750%:%
%:%1811=750%:%
%:%1812=750%:%
%:%1813=751%:%
%:%1814=752%:%
%:%1815=752%:%
%:%1820=752%:%
%:%1823=753%:%
%:%1824=754%:%
%:%1825=754%:%
%:%1828=755%:%
%:%1832=755%:%
%:%1833=755%:%
%:%1847=757%:%
%:%1857=759%:%
%:%1858=759%:%
%:%1859=760%:%
%:%1860=761%:%
%:%1861=762%:%
%:%1862=763%:%
%:%1863=764%:%
%:%1866=765%:%
%:%1870=765%:%
%:%1871=765%:%
%:%1876=765%:%
%:%1879=766%:%
%:%1880=767%:%
%:%1881=767%:%
%:%1882=768%:%
%:%1885=769%:%
%:%1889=769%:%
%:%1890=769%:%
%:%1891=769%:%
%:%1896=769%:%
%:%1899=770%:%
%:%1900=771%:%
%:%1901=771%:%
%:%1902=772%:%
%:%1903=773%:%
%:%1904=774%:%
%:%1905=775%:%
%:%1912=776%:%
%:%1913=776%:%
%:%1914=777%:%
%:%1915=777%:%
%:%1916=778%:%
%:%1917=778%:%
%:%1918=779%:%
%:%1919=779%:%
%:%1920=780%:%
%:%1921=780%:%
%:%1922=781%:%
%:%1923=781%:%
%:%1924=781%:%
%:%1925=782%:%
%:%1926=782%:%
%:%1927=782%:%
%:%1928=783%:%
%:%1929=783%:%
%:%1930=783%:%
%:%1931=784%:%
%:%1932=784%:%
%:%1933=785%:%
%:%1934=785%:%
%:%1935=785%:%
%:%1936=786%:%
%:%1937=786%:%
%:%1938=786%:%
%:%1939=787%:%
%:%1940=788%:%
%:%1941=789%:%
%:%1942=790%:%
%:%1943=790%:%
%:%1944=790%:%
%:%1945=791%:%
%:%1946=791%:%
%:%1947=791%:%
%:%1948=792%:%
%:%1949=793%:%
%:%1950=794%:%
%:%1951=795%:%
%:%1952=795%:%
%:%1953=795%:%
%:%1954=796%:%
%:%1955=796%:%
%:%1956=796%:%
%:%1959=799%:%
%:%1960=800%:%
%:%1961=801%:%
%:%1962=801%:%
%:%1963=801%:%
%:%1964=802%:%
%:%1965=802%:%
%:%1966=803%:%
%:%1967=804%:%
%:%1968=805%:%
%:%1969=805%:%
%:%1972=808%:%
%:%1973=809%:%
%:%1974=810%:%
%:%1975=810%:%
%:%1976=810%:%
%:%1977=811%:%
%:%1978=811%:%
%:%1979=812%:%
%:%1980=812%:%
%:%1981=813%:%
%:%1982=814%:%
%:%1983=815%:%
%:%1984=815%:%
%:%1985=816%:%
%:%1986=816%:%
%:%1987=817%:%
%:%1988=817%:%
%:%1989=818%:%
%:%1990=818%:%
%:%1991=818%:%
%:%1992=819%:%
%:%1993=819%:%
%:%1994=820%:%
%:%1995=820%:%
%:%1997=822%:%
%:%1998=823%:%
%:%1999=823%:%
%:%2000=823%:%
%:%2001=824%:%
%:%2002=824%:%
%:%2003=825%:%
%:%2004=825%:%
%:%2005=826%:%
%:%2006=826%:%
%:%2007=826%:%
%:%2008=826%:%
%:%2009=827%:%
%:%2015=827%:%
%:%2018=828%:%
%:%2019=829%:%
%:%2020=830%:%
%:%2021=831%:%
%:%2022=832%:%
%:%2023=832%:%
%:%2024=833%:%
%:%2025=834%:%
%:%2026=835%:%
%:%2027=836%:%
%:%2028=837%:%
%:%2029=838%:%
%:%2030=839%:%
%:%2031=840%:%
%:%2038=841%:%
%:%2039=841%:%
%:%2040=842%:%
%:%2041=842%:%
%:%2042=843%:%
%:%2043=843%:%
%:%2044=844%:%
%:%2045=844%:%
%:%2046=845%:%
%:%2047=845%:%
%:%2048=846%:%
%:%2049=846%:%
%:%2050=846%:%
%:%2051=847%:%
%:%2052=847%:%
%:%2053=847%:%
%:%2054=848%:%
%:%2055=848%:%
%:%2056=848%:%
%:%2057=849%:%
%:%2058=849%:%
%:%2059=850%:%
%:%2060=850%:%
%:%2061=851%:%
%:%2062=851%:%
%:%2063=852%:%
%:%2064=852%:%
%:%2065=853%:%
%:%2066=854%:%
%:%2067=855%:%
%:%2068=855%:%
%:%2069=855%:%
%:%2070=856%:%
%:%2071=856%:%
%:%2072=857%:%
%:%2073=857%:%
%:%2074=858%:%
%:%2075=859%:%
%:%2076=859%:%
%:%2077=860%:%
%:%2078=860%:%
%:%2079=861%:%
%:%2080=861%:%
%:%2081=862%:%
%:%2082=862%:%
%:%2083=862%:%
%:%2086=865%:%
%:%2087=866%:%
%:%2088=867%:%
%:%2089=867%:%
%:%2090=867%:%
%:%2091=868%:%
%:%2092=868%:%
%:%2093=869%:%
%:%2094=870%:%
%:%2095=871%:%
%:%2096=871%:%
%:%2099=874%:%
%:%2100=875%:%
%:%2101=876%:%
%:%2102=876%:%
%:%2103=876%:%
%:%2104=877%:%
%:%2105=877%:%
%:%2106=878%:%
%:%2107=878%:%
%:%2108=879%:%
%:%2109=880%:%
%:%2110=880%:%
%:%2111=880%:%
%:%2112=881%:%
%:%2113=881%:%
%:%2114=882%:%
%:%2115=882%:%
%:%2116=883%:%
%:%2117=883%:%
%:%2118=883%:%
%:%2119=884%:%
%:%2120=884%:%
%:%2121=885%:%
%:%2122=885%:%
%:%2124=887%:%
%:%2125=888%:%
%:%2126=888%:%
%:%2127=888%:%
%:%2128=889%:%
%:%2129=889%:%
%:%2130=890%:%
%:%2131=890%:%
%:%2132=891%:%
%:%2133=891%:%
%:%2134=891%:%
%:%2135=891%:%
%:%2136=892%:%
%:%2142=892%:%
%:%2145=893%:%
%:%2146=894%:%
%:%2147=894%:%
%:%2148=895%:%
%:%2155=896%:%
%:%2156=896%:%
%:%2157=897%:%
%:%2158=897%:%
%:%2159=898%:%
%:%2160=898%:%
%:%2161=898%:%
%:%2162=899%:%
%:%2163=899%:%
%:%2164=900%:%
%:%2165=900%:%
%:%2166=901%:%
%:%2167=901%:%
%:%2168=902%:%
%:%2169=902%:%
%:%2170=902%:%
%:%2171=903%:%
%:%2172=903%:%
%:%2173=904%:%
%:%2174=905%:%
%:%2175=906%:%
%:%2176=906%:%
%:%2177=906%:%
%:%2178=907%:%
%:%2179=907%:%
%:%2180=907%:%
%:%2181=907%:%
%:%2182=907%:%
%:%2183=908%:%
%:%2189=908%:%
%:%2192=909%:%
%:%2193=910%:%
%:%2194=910%:%
%:%2195=911%:%
%:%2196=912%:%
%:%2197=913%:%
%:%2198=914%:%
%:%2205=915%:%
%:%2206=915%:%
%:%2207=916%:%
%:%2208=916%:%
%:%2209=917%:%
%:%2210=917%:%
%:%2211=918%:%
%:%2212=918%:%
%:%2213=919%:%
%:%2214=919%:%
%:%2215=919%:%
%:%2216=920%:%
%:%2217=920%:%
%:%2218=921%:%
%:%2219=922%:%
%:%2220=922%:%
%:%2221=922%:%
%:%2222=923%:%
%:%2223=923%:%
%:%2224=923%:%
%:%2225=923%:%
%:%2226=923%:%
%:%2227=924%:%
%:%2233=924%:%
%:%2236=925%:%
%:%2237=926%:%
%:%2238=927%:%
%:%2239=927%:%
%:%2240=928%:%
%:%2241=929%:%
%:%2242=930%:%
%:%2243=931%:%
%:%2250=932%:%
%:%2251=932%:%
%:%2252=933%:%
%:%2253=933%:%
%:%2254=934%:%
%:%2255=934%:%
%:%2256=934%:%
%:%2257=935%:%
%:%2258=935%:%
%:%2259=935%:%
%:%2260=936%:%
%:%2261=936%:%
%:%2262=937%:%
%:%2263=938%:%
%:%2264=938%:%
%:%2265=938%:%
%:%2266=939%:%
%:%2267=939%:%
%:%2268=939%:%
%:%2269=939%:%
%:%2270=939%:%
%:%2271=940%:%
%:%2277=940%:%
%:%2280=941%:%
%:%2281=947%:%
%:%2282=948%:%
%:%2283=948%:%
%:%2284=949%:%
%:%2285=950%:%
%:%2286=951%:%
%:%2287=952%:%
%:%2294=953%:%
%:%2295=953%:%
%:%2296=954%:%
%:%2297=954%:%
%:%2298=955%:%
%:%2299=955%:%
%:%2300=955%:%
%:%2301=956%:%
%:%2302=956%:%
%:%2303=957%:%
%:%2304=958%:%
%:%2305=959%:%
%:%2306=960%:%
%:%2307=960%:%
%:%2308=960%:%
%:%2309=961%:%
%:%2310=961%:%
%:%2311=962%:%
%:%2312=962%:%
%:%2313=963%:%
%:%2314=964%:%
%:%2315=965%:%
%:%2316=965%:%
%:%2317=966%:%
%:%2318=966%:%
%:%2319=967%:%
%:%2320=967%:%
%:%2321=968%:%
%:%2322=968%:%
%:%2323=969%:%
%:%2324=970%:%
%:%2325=971%:%
%:%2326=971%:%
%:%2327=971%:%
%:%2328=972%:%
%:%2329=972%:%
%:%2330=973%:%
%:%2331=973%:%
%:%2332=974%:%
%:%2333=975%:%
%:%2334=976%:%
%:%2335=976%:%
%:%2336=977%:%
%:%2337=977%:%
%:%2338=978%:%
%:%2339=978%:%
%:%2340=979%:%
%:%2341=979%:%
%:%2342=979%:%
%:%2343=980%:%
%:%2344=980%:%
%:%2345=981%:%
%:%2346=981%:%
%:%2347=981%:%
%:%2348=981%:%
%:%2349=982%:%
%:%2355=982%:%
%:%2358=983%:%
%:%2359=984%:%
%:%2360=984%:%
%:%2361=985%:%
%:%2368=986%:%
%:%2369=986%:%
%:%2370=987%:%
%:%2371=987%:%
%:%2372=988%:%
%:%2373=988%:%
%:%2374=988%:%
%:%2375=989%:%
%:%2376=989%:%
%:%2377=990%:%
%:%2378=991%:%
%:%2379=992%:%
%:%2380=993%:%
%:%2381=993%:%
%:%2382=993%:%
%:%2383=994%:%
%:%2384=994%:%
%:%2385=995%:%
%:%2386=995%:%
%:%2387=996%:%
%:%2388=997%:%
%:%2389=998%:%
%:%2390=998%:%
%:%2391=999%:%
%:%2392=999%:%
%:%2393=1000%:%
%:%2394=1000%:%
%:%2395=1001%:%
%:%2396=1001%:%
%:%2397=1002%:%
%:%2398=1003%:%
%:%2399=1004%:%
%:%2400=1004%:%
%:%2401=1004%:%
%:%2402=1005%:%
%:%2403=1005%:%
%:%2404=1006%:%
%:%2405=1006%:%
%:%2406=1007%:%
%:%2407=1008%:%
%:%2408=1009%:%
%:%2409=1010%:%
%:%2410=1010%:%
%:%2411=1011%:%
%:%2412=1011%:%
%:%2413=1012%:%
%:%2414=1012%:%
%:%2415=1013%:%
%:%2416=1013%:%
%:%2417=1013%:%
%:%2418=1014%:%
%:%2419=1014%:%
%:%2420=1015%:%
%:%2421=1015%:%
%:%2422=1015%:%
%:%2423=1015%:%
%:%2424=1016%:%
%:%2430=1016%:%
%:%2433=1017%:%
%:%2434=1018%:%
%:%2435=1022%:%
%:%2436=1023%:%
%:%2437=1024%:%
%:%2438=1024%:%
%:%2439=1025%:%
%:%2440=1026%:%
%:%2441=1027%:%
%:%2442=1028%:%
%:%2449=1029%:%
%:%2450=1029%:%
%:%2451=1030%:%
%:%2452=1030%:%
%:%2453=1031%:%
%:%2454=1032%:%
%:%2455=1033%:%
%:%2456=1034%:%
%:%2457=1034%:%
%:%2458=1034%:%
%:%2459=1035%:%
%:%2460=1035%:%
%:%2461=1036%:%
%:%2462=1036%:%
%:%2463=1037%:%
%:%2464=1038%:%
%:%2465=1039%:%
%:%2466=1039%:%
%:%2467=1040%:%
%:%2468=1040%:%
%:%2469=1041%:%
%:%2470=1041%:%
%:%2471=1042%:%
%:%2472=1042%:%
%:%2473=1043%:%
%:%2474=1044%:%
%:%2475=1045%:%
%:%2476=1045%:%
%:%2477=1045%:%
%:%2478=1046%:%
%:%2479=1046%:%
%:%2480=1047%:%
%:%2481=1047%:%
%:%2482=1048%:%
%:%2483=1049%:%
%:%2484=1050%:%
%:%2485=1051%:%
%:%2486=1051%:%
%:%2487=1052%:%
%:%2488=1052%:%
%:%2489=1053%:%
%:%2490=1053%:%
%:%2491=1054%:%
%:%2492=1054%:%
%:%2493=1054%:%
%:%2494=1055%:%
%:%2495=1055%:%
%:%2496=1056%:%
%:%2497=1056%:%
%:%2498=1056%:%
%:%2499=1056%:%
%:%2500=1057%:%
%:%2506=1057%:%
%:%2509=1058%:%
%:%2510=1059%:%
%:%2511=1059%:%
%:%2514=1060%:%
%:%2518=1060%:%
%:%2519=1060%:%
%:%2520=1061%:%
%:%2521=1062%:%
%:%2522=1062%:%
%:%2527=1062%:%
%:%2530=1063%:%
%:%2531=1064%:%
%:%2532=1064%:%
%:%2535=1065%:%
%:%2539=1065%:%
%:%2540=1065%:%
%:%2545=1065%:%
%:%2548=1066%:%
%:%2549=1067%:%
%:%2550=1067%:%
%:%2553=1068%:%
%:%2557=1068%:%
%:%2558=1068%:%
%:%2559=1069%:%
%:%2560=1069%:%
%:%2565=1069%:%
%:%2568=1070%:%
%:%2569=1071%:%
%:%2570=1071%:%
%:%2573=1072%:%
%:%2577=1072%:%
%:%2578=1072%:%
%:%2583=1072%:%
%:%2586=1073%:%
%:%2587=1074%:%
%:%2588=1075%:%
%:%2589=1076%:%
%:%2590=1077%:%
%:%2591=1077%:%
%:%2592=1078%:%
%:%2593=1079%:%
%:%2594=1080%:%
%:%2595=1081%:%
%:%2596=1081%:%
%:%2597=1082%:%
%:%2600=1083%:%
%:%2604=1083%:%
%:%2605=1083%:%
%:%2610=1083%:%
%:%2613=1084%:%
%:%2614=1085%:%
%:%2615=1085%:%
%:%2616=1086%:%
%:%2617=1087%:%
%:%2620=1090%:%
%:%2621=1091%:%
%:%2622=1092%:%
%:%2623=1092%:%
%:%2624=1093%:%
%:%2627=1094%:%
%:%2631=1094%:%
%:%2632=1094%:%
%:%2633=1094%:%
%:%2638=1094%:%
%:%2641=1095%:%
%:%2642=1096%:%
%:%2643=1096%:%
%:%2644=1097%:%
%:%2645=1098%:%
%:%2646=1099%:%
%:%2647=1100%:%
%:%2648=1101%:%
%:%2651=1102%:%
%:%2655=1102%:%
%:%2656=1102%:%
%:%2657=1103%:%
%:%2658=1103%:%
%:%2659=1103%:%
%:%2664=1103%:%
%:%2667=1104%:%
%:%2668=1105%:%
%:%2669=1106%:%
%:%2670=1106%:%
%:%2671=1107%:%
%:%2672=1108%:%
%:%2673=1109%:%
%:%2674=1110%:%
%:%2681=1111%:%
%:%2682=1111%:%
%:%2683=1112%:%
%:%2684=1112%:%
%:%2685=1113%:%
%:%2686=1113%:%
%:%2687=1114%:%
%:%2688=1114%:%
%:%2689=1115%:%
%:%2690=1115%:%
%:%2691=1116%:%
%:%2692=1116%:%
%:%2693=1117%:%
%:%2694=1117%:%
%:%2695=1117%:%
%:%2696=1118%:%
%:%2697=1118%:%
%:%2698=1119%:%
%:%2699=1119%:%
%:%2700=1120%:%
%:%2701=1121%:%
%:%2702=1121%:%
%:%2703=1121%:%
%:%2704=1122%:%
%:%2705=1122%:%
%:%2706=1123%:%
%:%2707=1123%:%
%:%2708=1124%:%
%:%2709=1124%:%
%:%2710=1124%:%
%:%2711=1125%:%
%:%2712=1125%:%
%:%2713=1125%:%
%:%2714=1125%:%
%:%2715=1126%:%
%:%2721=1126%:%
%:%2724=1127%:%
%:%2725=1128%:%
%:%2726=1129%:%
%:%2727=1130%:%
%:%2728=1130%:%
%:%2729=1131%:%
%:%2730=1132%:%
%:%2731=1133%:%
%:%2732=1134%:%
%:%2733=1135%:%
%:%2734=1135%:%
%:%2735=1136%:%
%:%2738=1137%:%
%:%2742=1137%:%
%:%2743=1137%:%
%:%2748=1137%:%
%:%2751=1138%:%
%:%2752=1139%:%
%:%2753=1140%:%
%:%2754=1140%:%
%:%2755=1141%:%
%:%2757=1143%:%
%:%2760=1144%:%
%:%2764=1144%:%
%:%2765=1144%:%
%:%2766=1144%:%
%:%2771=1144%:%
%:%2774=1145%:%
%:%2775=1146%:%
%:%2776=1147%:%
%:%2777=1147%:%
%:%2778=1148%:%
%:%2779=1149%:%
%:%2780=1150%:%
%:%2781=1151%:%
%:%2788=1152%:%
%:%2789=1152%:%
%:%2790=1153%:%
%:%2791=1153%:%
%:%2792=1154%:%
%:%2793=1154%:%
%:%2794=1155%:%
%:%2795=1155%:%
%:%2796=1156%:%
%:%2797=1156%:%
%:%2798=1157%:%
%:%2799=1157%:%
%:%2800=1158%:%
%:%2801=1158%:%
%:%2802=1158%:%
%:%2803=1159%:%
%:%2804=1159%:%
%:%2805=1160%:%
%:%2806=1160%:%
%:%2807=1161%:%
%:%2808=1161%:%
%:%2809=1161%:%
%:%2810=1162%:%
%:%2811=1162%:%
%:%2812=1162%:%
%:%2813=1162%:%
%:%2814=1163%:%
%:%2820=1163%:%
%:%2823=1164%:%
%:%2824=1165%:%
%:%2825=1166%:%
%:%2826=1167%:%
%:%2827=1167%:%
%:%2828=1168%:%
%:%2829=1169%:%
%:%2830=1170%:%
%:%2831=1171%:%
%:%2832=1172%:%
%:%2833=1172%:%
%:%2834=1173%:%
%:%2837=1174%:%
%:%2841=1174%:%
%:%2842=1174%:%
%:%2847=1174%:%
%:%2850=1175%:%
%:%2851=1176%:%
%:%2852=1176%:%
%:%2853=1177%:%
%:%2855=1179%:%
%:%2858=1180%:%
%:%2862=1180%:%
%:%2863=1180%:%
%:%2864=1181%:%
%:%2865=1181%:%
%:%2870=1181%:%
%:%2873=1182%:%
%:%2874=1183%:%
%:%2875=1184%:%
%:%2876=1184%:%
%:%2877=1185%:%
%:%2878=1186%:%
%:%2879=1187%:%
%:%2880=1188%:%
%:%2887=1189%:%
%:%2888=1189%:%
%:%2889=1190%:%
%:%2890=1190%:%
%:%2891=1191%:%
%:%2892=1192%:%
%:%2893=1193%:%
%:%2894=1194%:%
%:%2895=1194%:%
%:%2896=1194%:%
%:%2897=1195%:%
%:%2898=1195%:%
%:%2899=1196%:%
%:%2900=1196%:%
%:%2901=1197%:%
%:%2902=1198%:%
%:%2903=1199%:%
%:%2904=1200%:%
%:%2905=1200%:%
%:%2906=1200%:%
%:%2907=1201%:%
%:%2908=1201%:%
%:%2909=1202%:%
%:%2910=1203%:%
%:%2911=1203%:%
%:%2912=1204%:%
%:%2913=1205%:%
%:%2914=1206%:%
%:%2915=1206%:%
%:%2916=1206%:%
%:%2917=1207%:%
%:%2918=1207%:%
%:%2919=1208%:%
%:%2920=1208%:%
%:%2921=1209%:%
%:%2922=1210%:%
%:%2923=1211%:%
%:%2924=1211%:%
%:%2925=1212%:%
%:%2926=1212%:%
%:%2927=1213%:%
%:%2928=1213%:%
%:%2929=1214%:%
%:%2930=1214%:%
%:%2931=1214%:%
%:%2932=1215%:%
%:%2933=1215%:%
%:%2934=1216%:%
%:%2935=1216%:%
%:%2936=1217%:%
%:%2937=1218%:%
%:%2938=1218%:%
%:%2939=1218%:%
%:%2940=1219%:%
%:%2941=1219%:%
%:%2942=1220%:%
%:%2943=1220%:%
%:%2944=1220%:%
%:%2945=1220%:%
%:%2946=1221%:%
%:%2952=1221%:%
%:%2955=1222%:%
%:%2956=1224%:%
%:%2957=1225%:%
%:%2958=1225%:%
%:%2959=1226%:%
%:%2960=1227%:%
%:%2962=1229%:%
%:%2963=1230%:%
%:%2964=1231%:%
%:%2965=1231%:%
%:%2966=1232%:%
%:%2969=1233%:%
%:%2973=1233%:%
%:%2974=1233%:%
%:%2979=1233%:%
%:%2982=1234%:%
%:%2983=1235%:%
%:%2984=1235%:%
%:%2985=1236%:%
%:%2987=1238%:%
%:%2990=1239%:%
%:%2994=1239%:%
%:%2995=1239%:%
%:%2996=1240%:%
%:%2997=1240%:%
%:%3002=1240%:%
%:%3005=1241%:%
%:%3006=1242%:%
%:%3007=1242%:%
%:%3008=1243%:%
%:%3009=1244%:%
%:%3010=1245%:%
%:%3011=1246%:%
%:%3012=1247%:%
%:%3015=1248%:%
%:%3019=1248%:%
%:%3020=1248%:%
%:%3021=1248%:%
%:%3026=1248%:%
%:%3029=1249%:%
%:%3030=1250%:%
%:%3031=1251%:%
%:%3032=1251%:%
%:%3033=1252%:%
%:%3034=1253%:%
%:%3035=1254%:%
%:%3036=1255%:%
%:%3037=1256%:%
%:%3038=1257%:%
%:%3041=1258%:%
%:%3045=1258%:%
%:%3046=1258%:%
%:%3047=1259%:%
%:%3048=1259%:%
%:%3053=1259%:%
%:%3056=1260%:%
%:%3057=1261%:%
%:%3058=1261%:%
%:%3059=1262%:%
%:%3060=1263%:%
%:%3061=1264%:%
%:%3062=1265%:%
%:%3063=1266%:%
%:%3064=1267%:%
%:%3067=1268%:%
%:%3071=1268%:%
%:%3072=1268%:%
%:%3073=1269%:%
%:%3074=1269%:%
%:%3079=1269%:%
%:%3082=1270%:%
%:%3083=1271%:%
%:%3084=1272%:%
%:%3085=1272%:%
%:%3086=1273%:%
%:%3087=1274%:%
%:%3090=1275%:%
%:%3094=1275%:%
%:%3095=1275%:%
%:%3096=1275%:%
%:%3097=1276%:%
%:%3098=1276%:%
%:%3103=1276%:%
%:%3106=1277%:%
%:%3107=1278%:%
%:%3108=1279%:%
%:%3109=1279%:%
%:%3110=1280%:%
%:%3111=1281%:%
%:%3112=1282%:%
%:%3113=1283%:%
%:%3120=1284%:%
%:%3121=1284%:%
%:%3122=1285%:%
%:%3123=1285%:%
%:%3124=1285%:%
%:%3125=1286%:%
%:%3126=1286%:%
%:%3127=1287%:%
%:%3128=1287%:%
%:%3129=1288%:%
%:%3130=1289%:%
%:%3131=1290%:%
%:%3132=1291%:%
%:%3133=1291%:%
%:%3134=1291%:%
%:%3135=1292%:%
%:%3136=1292%:%
%:%3137=1293%:%
%:%3138=1294%:%
%:%3139=1295%:%
%:%3140=1296%:%
%:%3141=1296%:%
%:%3142=1296%:%
%:%3143=1297%:%
%:%3144=1297%:%
%:%3145=1298%:%
%:%3146=1299%:%
%:%3147=1299%:%
%:%3148=1300%:%
%:%3149=1301%:%
%:%3150=1302%:%
%:%3151=1302%:%
%:%3152=1302%:%
%:%3153=1303%:%
%:%3154=1303%:%
%:%3155=1304%:%
%:%3156=1304%:%
%:%3157=1305%:%
%:%3158=1306%:%
%:%3159=1307%:%
%:%3160=1307%:%
%:%3161=1308%:%
%:%3162=1308%:%
%:%3163=1309%:%
%:%3164=1309%:%
%:%3165=1310%:%
%:%3166=1310%:%
%:%3167=1310%:%
%:%3168=1311%:%
%:%3169=1311%:%
%:%3170=1312%:%
%:%3171=1312%:%
%:%3172=1313%:%
%:%3173=1314%:%
%:%3174=1314%:%
%:%3175=1314%:%
%:%3176=1315%:%
%:%3177=1315%:%
%:%3178=1316%:%
%:%3179=1316%:%
%:%3180=1317%:%
%:%3181=1317%:%
%:%3182=1317%:%
%:%3183=1318%:%
%:%3184=1318%:%
%:%3185=1319%:%
%:%3186=1319%:%
%:%3187=1319%:%
%:%3188=1319%:%
%:%3189=1320%:%
%:%3190=1320%:%
%:%3191=1320%:%
%:%3192=1321%:%
%:%3198=1321%:%
%:%3201=1322%:%
%:%3202=1323%:%
%:%3203=1324%:%
%:%3204=1324%:%
%:%3207=1325%:%
%:%3211=1325%:%
%:%3212=1325%:%
%:%3213=1326%:%
%:%3214=1326%:%
%:%3219=1326%:%
%:%3222=1327%:%
%:%3223=1328%:%
%:%3224=1328%:%
%:%3227=1329%:%
%:%3231=1329%:%
%:%3232=1329%:%
%:%3237=1329%:%
%:%3240=1330%:%
%:%3241=1331%:%
%:%3242=1331%:%
%:%3243=1332%:%
%:%3244=1333%:%
%:%3245=1334%:%
%:%3246=1335%:%
%:%3247=1336%:%
%:%3248=1337%:%
%:%3249=1338%:%
%:%3250=1339%:%
%:%3251=1340%:%
%:%3252=1341%:%
%:%3253=1342%:%
%:%3254=1343%:%
%:%3255=1344%:%
%:%3262=1345%:%
%:%3263=1345%:%
%:%3264=1346%:%
%:%3265=1346%:%
%:%3266=1347%:%
%:%3267=1347%:%
%:%3268=1347%:%
%:%3269=1348%:%
%:%3270=1348%:%
%:%3271=1349%:%
%:%3272=1349%:%
%:%3273=1350%:%
%:%3274=1350%:%
%:%3275=1351%:%
%:%3276=1351%:%
%:%3277=1351%:%
%:%3278=1352%:%
%:%3279=1352%:%
%:%3280=1353%:%
%:%3281=1353%:%
%:%3282=1354%:%
%:%3283=1354%:%
%:%3284=1354%:%
%:%3285=1355%:%
%:%3286=1355%:%
%:%3287=1355%:%
%:%3288=1356%:%
%:%3289=1356%:%
%:%3290=1356%:%
%:%3291=1357%:%
%:%3297=1357%:%
%:%3300=1358%:%
%:%3301=1359%:%
%:%3302=1360%:%
%:%3303=1360%:%
%:%3304=1361%:%
%:%3305=1362%:%
%:%3306=1363%:%
%:%3307=1364%:%
%:%3308=1365%:%
%:%3309=1366%:%
%:%3316=1367%:%
%:%3317=1367%:%
%:%3318=1368%:%
%:%3319=1368%:%
%:%3320=1369%:%
%:%3321=1369%:%
%:%3322=1369%:%
%:%3323=1370%:%
%:%3324=1370%:%
%:%3325=1370%:%
%:%3326=1370%:%
%:%3327=1371%:%
%:%3328=1372%:%
%:%3329=1373%:%
%:%3330=1373%:%
%:%3331=1374%:%
%:%3337=1374%:%
%:%3342=1375%:%
%:%3347=1376%:%

%
\begin{isabellebody}%
\setisabellecontext{Forcing{\isacharunderscore}{\kern0pt}Data}%
%
\isadelimdocument
%
\endisadelimdocument
%
\isatagdocument
%
\isamarkupsection{Transitive set models of ZF%
}
\isamarkuptrue%
%
\endisatagdocument
{\isafolddocument}%
%
\isadelimdocument
%
\endisadelimdocument
%
\begin{isamarkuptext}%
This theory defines the locale \isa{M{\isacharunderscore}{\kern0pt}ZF{\isacharunderscore}{\kern0pt}trans} for
transitive models of ZF, and the associated \isa{forcing{\isacharunderscore}{\kern0pt}data}
 that adds a forcing notion%
\end{isamarkuptext}\isamarkuptrue%
%
\isadelimtheory
%
\endisadelimtheory
%
\isatagtheory
\isacommand{theory}\isamarkupfalse%
\ Forcing{\isacharunderscore}{\kern0pt}Data\isanewline
\ \ \isakeyword{imports}\ \ \isanewline
\ \ \ \ Forcing{\isacharunderscore}{\kern0pt}Notions\ \isanewline
\ \ \ \ Interface\isanewline
\isanewline
\isakeyword{begin}%
\endisatagtheory
{\isafoldtheory}%
%
\isadelimtheory
%
\endisadelimtheory
\isanewline
\isanewline
\isacommand{lemma}\isamarkupfalse%
\ Transset{\isacharunderscore}{\kern0pt}M\ {\isacharcolon}{\kern0pt}\isanewline
\ \ {\isachardoublequoteopen}Transset{\isacharparenleft}{\kern0pt}M{\isacharparenright}{\kern0pt}\ {\isasymLongrightarrow}\ \ y{\isasymin}x\ {\isasymLongrightarrow}\ x\ {\isasymin}\ M\ {\isasymLongrightarrow}\ y\ {\isasymin}\ M{\isachardoublequoteclose}\isanewline
%
\isadelimproof
\ \ %
\endisadelimproof
%
\isatagproof
\isacommand{by}\isamarkupfalse%
\ {\isacharparenleft}{\kern0pt}simp\ add{\isacharcolon}{\kern0pt}\ Transset{\isacharunderscore}{\kern0pt}def{\isacharcomma}{\kern0pt}auto{\isacharparenright}{\kern0pt}%
\endisatagproof
{\isafoldproof}%
%
\isadelimproof
\ \ \isanewline
%
\endisadelimproof
\isanewline
\isanewline
\isacommand{locale}\isamarkupfalse%
\ M{\isacharunderscore}{\kern0pt}ZF\ {\isacharequal}{\kern0pt}\ \isanewline
\ \ \isakeyword{fixes}\ M\ \isanewline
\ \ \isakeyword{assumes}\ \isanewline
\ \ \ \ upair{\isacharunderscore}{\kern0pt}ax{\isacharcolon}{\kern0pt}\ \ \ \ \ \ \ \ \ {\isachardoublequoteopen}upair{\isacharunderscore}{\kern0pt}ax{\isacharparenleft}{\kern0pt}{\isacharhash}{\kern0pt}{\isacharhash}{\kern0pt}M{\isacharparenright}{\kern0pt}{\isachardoublequoteclose}\isanewline
\ \ \ \ \isakeyword{and}\ Union{\isacharunderscore}{\kern0pt}ax{\isacharcolon}{\kern0pt}\ \ \ \ \ \ \ \ \ {\isachardoublequoteopen}Union{\isacharunderscore}{\kern0pt}ax{\isacharparenleft}{\kern0pt}{\isacharhash}{\kern0pt}{\isacharhash}{\kern0pt}M{\isacharparenright}{\kern0pt}{\isachardoublequoteclose}\isanewline
\ \ \ \ \isakeyword{and}\ power{\isacharunderscore}{\kern0pt}ax{\isacharcolon}{\kern0pt}\ \ \ \ \ \ \ \ \ {\isachardoublequoteopen}power{\isacharunderscore}{\kern0pt}ax{\isacharparenleft}{\kern0pt}{\isacharhash}{\kern0pt}{\isacharhash}{\kern0pt}M{\isacharparenright}{\kern0pt}{\isachardoublequoteclose}\isanewline
\ \ \ \ \isakeyword{and}\ extensionality{\isacharcolon}{\kern0pt}\ \ \ {\isachardoublequoteopen}extensionality{\isacharparenleft}{\kern0pt}{\isacharhash}{\kern0pt}{\isacharhash}{\kern0pt}M{\isacharparenright}{\kern0pt}{\isachardoublequoteclose}\isanewline
\ \ \ \ \isakeyword{and}\ foundation{\isacharunderscore}{\kern0pt}ax{\isacharcolon}{\kern0pt}\ \ \ \ {\isachardoublequoteopen}foundation{\isacharunderscore}{\kern0pt}ax{\isacharparenleft}{\kern0pt}{\isacharhash}{\kern0pt}{\isacharhash}{\kern0pt}M{\isacharparenright}{\kern0pt}{\isachardoublequoteclose}\isanewline
\ \ \ \ \isakeyword{and}\ infinity{\isacharunderscore}{\kern0pt}ax{\isacharcolon}{\kern0pt}\ \ \ \ \ \ {\isachardoublequoteopen}infinity{\isacharunderscore}{\kern0pt}ax{\isacharparenleft}{\kern0pt}{\isacharhash}{\kern0pt}{\isacharhash}{\kern0pt}M{\isacharparenright}{\kern0pt}{\isachardoublequoteclose}\isanewline
\ \ \ \ \isakeyword{and}\ separation{\isacharunderscore}{\kern0pt}ax{\isacharcolon}{\kern0pt}\ \ \ \ {\isachardoublequoteopen}{\isasymphi}{\isasymin}formula\ {\isasymLongrightarrow}\ env{\isasymin}list{\isacharparenleft}{\kern0pt}M{\isacharparenright}{\kern0pt}\ {\isasymLongrightarrow}\ arity{\isacharparenleft}{\kern0pt}{\isasymphi}{\isacharparenright}{\kern0pt}\ {\isasymle}\ {\isadigit{1}}\ {\isacharhash}{\kern0pt}{\isacharplus}{\kern0pt}\ length{\isacharparenleft}{\kern0pt}env{\isacharparenright}{\kern0pt}\ {\isasymLongrightarrow}\isanewline
\ \ \ \ \ \ \ \ \ \ \ \ \ \ \ \ \ \ \ \ separation{\isacharparenleft}{\kern0pt}{\isacharhash}{\kern0pt}{\isacharhash}{\kern0pt}M{\isacharcomma}{\kern0pt}{\isasymlambda}x{\isachardot}{\kern0pt}\ sats{\isacharparenleft}{\kern0pt}M{\isacharcomma}{\kern0pt}{\isasymphi}{\isacharcomma}{\kern0pt}{\isacharbrackleft}{\kern0pt}x{\isacharbrackright}{\kern0pt}\ {\isacharat}{\kern0pt}\ env{\isacharparenright}{\kern0pt}{\isacharparenright}{\kern0pt}{\isachardoublequoteclose}\ \isanewline
\ \ \ \ \isakeyword{and}\ replacement{\isacharunderscore}{\kern0pt}ax{\isacharcolon}{\kern0pt}\ \ \ {\isachardoublequoteopen}{\isasymphi}{\isasymin}formula\ {\isasymLongrightarrow}\ env{\isasymin}list{\isacharparenleft}{\kern0pt}M{\isacharparenright}{\kern0pt}\ {\isasymLongrightarrow}\ arity{\isacharparenleft}{\kern0pt}{\isasymphi}{\isacharparenright}{\kern0pt}\ {\isasymle}\ {\isadigit{2}}\ {\isacharhash}{\kern0pt}{\isacharplus}{\kern0pt}\ length{\isacharparenleft}{\kern0pt}env{\isacharparenright}{\kern0pt}\ {\isasymLongrightarrow}\ \isanewline
\ \ \ \ \ \ \ \ \ \ \ \ \ \ \ \ \ \ \ \ strong{\isacharunderscore}{\kern0pt}replacement{\isacharparenleft}{\kern0pt}{\isacharhash}{\kern0pt}{\isacharhash}{\kern0pt}M{\isacharcomma}{\kern0pt}{\isasymlambda}x\ y{\isachardot}{\kern0pt}\ sats{\isacharparenleft}{\kern0pt}M{\isacharcomma}{\kern0pt}{\isasymphi}{\isacharcomma}{\kern0pt}{\isacharbrackleft}{\kern0pt}x{\isacharcomma}{\kern0pt}y{\isacharbrackright}{\kern0pt}\ {\isacharat}{\kern0pt}\ env{\isacharparenright}{\kern0pt}{\isacharparenright}{\kern0pt}{\isachardoublequoteclose}\ \isanewline
\isanewline
\isacommand{locale}\isamarkupfalse%
\ M{\isacharunderscore}{\kern0pt}ctm\ {\isacharequal}{\kern0pt}\ M{\isacharunderscore}{\kern0pt}ZF\ {\isacharplus}{\kern0pt}\isanewline
\ \ \isakeyword{fixes}\ enum\isanewline
\ \ \isakeyword{assumes}\ M{\isacharunderscore}{\kern0pt}countable{\isacharcolon}{\kern0pt}\ \ \ \ \ \ {\isachardoublequoteopen}enum{\isasymin}bij{\isacharparenleft}{\kern0pt}nat{\isacharcomma}{\kern0pt}M{\isacharparenright}{\kern0pt}{\isachardoublequoteclose}\isanewline
\ \ \ \ \isakeyword{and}\ trans{\isacharunderscore}{\kern0pt}M{\isacharcolon}{\kern0pt}\ \ \ \ \ \ \ \ \ \ {\isachardoublequoteopen}Transset{\isacharparenleft}{\kern0pt}M{\isacharparenright}{\kern0pt}{\isachardoublequoteclose}\isanewline
\isanewline
\isakeyword{begin}\isanewline
\isacommand{interpretation}\isamarkupfalse%
\ intf{\isacharcolon}{\kern0pt}\ M{\isacharunderscore}{\kern0pt}ZF{\isacharunderscore}{\kern0pt}trans\ {\isachardoublequoteopen}M{\isachardoublequoteclose}\isanewline
%
\isadelimproof
\ \ %
\endisadelimproof
%
\isatagproof
\isacommand{using}\isamarkupfalse%
\ M{\isacharunderscore}{\kern0pt}ZF{\isacharunderscore}{\kern0pt}trans{\isachardot}{\kern0pt}intro\isanewline
\ \ \ \ trans{\isacharunderscore}{\kern0pt}M\ upair{\isacharunderscore}{\kern0pt}ax\ Union{\isacharunderscore}{\kern0pt}ax\ power{\isacharunderscore}{\kern0pt}ax\ extensionality\isanewline
\ \ \ \ foundation{\isacharunderscore}{\kern0pt}ax\ infinity{\isacharunderscore}{\kern0pt}ax\ separation{\isacharunderscore}{\kern0pt}ax{\isacharbrackleft}{\kern0pt}simplified{\isacharbrackright}{\kern0pt}\ \isanewline
\ \ \ \ replacement{\isacharunderscore}{\kern0pt}ax{\isacharbrackleft}{\kern0pt}simplified{\isacharbrackright}{\kern0pt}\isanewline
\ \ \isacommand{by}\isamarkupfalse%
\ simp%
\endisatagproof
{\isafoldproof}%
%
\isadelimproof
\isanewline
%
\endisadelimproof
\isanewline
\isanewline
\isacommand{lemmas}\isamarkupfalse%
\ transitivity\ {\isacharequal}{\kern0pt}\ Transset{\isacharunderscore}{\kern0pt}intf{\isacharbrackleft}{\kern0pt}OF\ trans{\isacharunderscore}{\kern0pt}M{\isacharbrackright}{\kern0pt}\isanewline
\isanewline
\isacommand{lemma}\isamarkupfalse%
\ zero{\isacharunderscore}{\kern0pt}in{\isacharunderscore}{\kern0pt}M{\isacharcolon}{\kern0pt}\ \ {\isachardoublequoteopen}{\isadigit{0}}\ {\isasymin}\ M{\isachardoublequoteclose}\ \isanewline
%
\isadelimproof
\ \ %
\endisadelimproof
%
\isatagproof
\isacommand{by}\isamarkupfalse%
\ {\isacharparenleft}{\kern0pt}rule\ intf{\isachardot}{\kern0pt}zero{\isacharunderscore}{\kern0pt}in{\isacharunderscore}{\kern0pt}M{\isacharparenright}{\kern0pt}%
\endisatagproof
{\isafoldproof}%
%
\isadelimproof
\isanewline
%
\endisadelimproof
\isanewline
\isacommand{lemma}\isamarkupfalse%
\ tuples{\isacharunderscore}{\kern0pt}in{\isacharunderscore}{\kern0pt}M{\isacharcolon}{\kern0pt}\ {\isachardoublequoteopen}A{\isasymin}M\ {\isasymLongrightarrow}\ B{\isasymin}M\ {\isasymLongrightarrow}\ {\isasymlangle}A{\isacharcomma}{\kern0pt}B{\isasymrangle}{\isasymin}M{\isachardoublequoteclose}\ \isanewline
%
\isadelimproof
\ \ %
\endisadelimproof
%
\isatagproof
\isacommand{by}\isamarkupfalse%
\ {\isacharparenleft}{\kern0pt}simp\ flip{\isacharcolon}{\kern0pt}setclass{\isacharunderscore}{\kern0pt}iff{\isacharparenright}{\kern0pt}%
\endisatagproof
{\isafoldproof}%
%
\isadelimproof
\isanewline
%
\endisadelimproof
\isanewline
\isacommand{lemma}\isamarkupfalse%
\ nat{\isacharunderscore}{\kern0pt}in{\isacharunderscore}{\kern0pt}M\ {\isacharcolon}{\kern0pt}\ {\isachardoublequoteopen}nat\ {\isasymin}\ M{\isachardoublequoteclose}\isanewline
%
\isadelimproof
\ \ %
\endisadelimproof
%
\isatagproof
\isacommand{by}\isamarkupfalse%
\ {\isacharparenleft}{\kern0pt}rule\ intf{\isachardot}{\kern0pt}nat{\isacharunderscore}{\kern0pt}in{\isacharunderscore}{\kern0pt}M{\isacharparenright}{\kern0pt}%
\endisatagproof
{\isafoldproof}%
%
\isadelimproof
\isanewline
%
\endisadelimproof
\isanewline
\isacommand{lemma}\isamarkupfalse%
\ n{\isacharunderscore}{\kern0pt}in{\isacharunderscore}{\kern0pt}M\ {\isacharcolon}{\kern0pt}\ {\isachardoublequoteopen}n{\isasymin}nat\ {\isasymLongrightarrow}\ n{\isasymin}M{\isachardoublequoteclose}\isanewline
%
\isadelimproof
\ \ %
\endisadelimproof
%
\isatagproof
\isacommand{using}\isamarkupfalse%
\ nat{\isacharunderscore}{\kern0pt}in{\isacharunderscore}{\kern0pt}M\ transitivity\ \isacommand{by}\isamarkupfalse%
\ simp%
\endisatagproof
{\isafoldproof}%
%
\isadelimproof
\isanewline
%
\endisadelimproof
\isanewline
\isacommand{lemma}\isamarkupfalse%
\ mtriv{\isacharcolon}{\kern0pt}\ {\isachardoublequoteopen}M{\isacharunderscore}{\kern0pt}trivial{\isacharparenleft}{\kern0pt}{\isacharhash}{\kern0pt}{\isacharhash}{\kern0pt}M{\isacharparenright}{\kern0pt}{\isachardoublequoteclose}\ \isanewline
%
\isadelimproof
\ \ %
\endisadelimproof
%
\isatagproof
\isacommand{by}\isamarkupfalse%
\ {\isacharparenleft}{\kern0pt}rule\ intf{\isachardot}{\kern0pt}mtriv{\isacharparenright}{\kern0pt}%
\endisatagproof
{\isafoldproof}%
%
\isadelimproof
\isanewline
%
\endisadelimproof
\isanewline
\isacommand{lemma}\isamarkupfalse%
\ mtrans{\isacharcolon}{\kern0pt}\ {\isachardoublequoteopen}M{\isacharunderscore}{\kern0pt}trans{\isacharparenleft}{\kern0pt}{\isacharhash}{\kern0pt}{\isacharhash}{\kern0pt}M{\isacharparenright}{\kern0pt}{\isachardoublequoteclose}\isanewline
%
\isadelimproof
\ \ %
\endisadelimproof
%
\isatagproof
\isacommand{by}\isamarkupfalse%
\ {\isacharparenleft}{\kern0pt}rule\ intf{\isachardot}{\kern0pt}mtrans{\isacharparenright}{\kern0pt}%
\endisatagproof
{\isafoldproof}%
%
\isadelimproof
\isanewline
%
\endisadelimproof
\isanewline
\isacommand{lemma}\isamarkupfalse%
\ mbasic{\isacharcolon}{\kern0pt}\ {\isachardoublequoteopen}M{\isacharunderscore}{\kern0pt}basic{\isacharparenleft}{\kern0pt}{\isacharhash}{\kern0pt}{\isacharhash}{\kern0pt}M{\isacharparenright}{\kern0pt}{\isachardoublequoteclose}\isanewline
%
\isadelimproof
\ \ %
\endisadelimproof
%
\isatagproof
\isacommand{by}\isamarkupfalse%
\ {\isacharparenleft}{\kern0pt}rule\ intf{\isachardot}{\kern0pt}mbasic{\isacharparenright}{\kern0pt}%
\endisatagproof
{\isafoldproof}%
%
\isadelimproof
\isanewline
%
\endisadelimproof
\isanewline
\isacommand{lemma}\isamarkupfalse%
\ mtrancl{\isacharcolon}{\kern0pt}\ {\isachardoublequoteopen}M{\isacharunderscore}{\kern0pt}trancl{\isacharparenleft}{\kern0pt}{\isacharhash}{\kern0pt}{\isacharhash}{\kern0pt}M{\isacharparenright}{\kern0pt}{\isachardoublequoteclose}\isanewline
%
\isadelimproof
\ \ %
\endisadelimproof
%
\isatagproof
\isacommand{by}\isamarkupfalse%
\ {\isacharparenleft}{\kern0pt}rule\ intf{\isachardot}{\kern0pt}mtrancl{\isacharparenright}{\kern0pt}%
\endisatagproof
{\isafoldproof}%
%
\isadelimproof
\isanewline
%
\endisadelimproof
\isanewline
\isacommand{lemma}\isamarkupfalse%
\ mdatatypes{\isacharcolon}{\kern0pt}\ {\isachardoublequoteopen}M{\isacharunderscore}{\kern0pt}datatypes{\isacharparenleft}{\kern0pt}{\isacharhash}{\kern0pt}{\isacharhash}{\kern0pt}M{\isacharparenright}{\kern0pt}{\isachardoublequoteclose}\isanewline
%
\isadelimproof
\ \ %
\endisadelimproof
%
\isatagproof
\isacommand{by}\isamarkupfalse%
\ {\isacharparenleft}{\kern0pt}rule\ intf{\isachardot}{\kern0pt}mdatatypes{\isacharparenright}{\kern0pt}%
\endisatagproof
{\isafoldproof}%
%
\isadelimproof
\isanewline
%
\endisadelimproof
\isanewline
\isacommand{lemma}\isamarkupfalse%
\ meclose{\isacharcolon}{\kern0pt}\ {\isachardoublequoteopen}M{\isacharunderscore}{\kern0pt}eclose{\isacharparenleft}{\kern0pt}{\isacharhash}{\kern0pt}{\isacharhash}{\kern0pt}M{\isacharparenright}{\kern0pt}{\isachardoublequoteclose}\isanewline
%
\isadelimproof
\ \ %
\endisadelimproof
%
\isatagproof
\isacommand{by}\isamarkupfalse%
\ {\isacharparenleft}{\kern0pt}rule\ intf{\isachardot}{\kern0pt}meclose{\isacharparenright}{\kern0pt}%
\endisatagproof
{\isafoldproof}%
%
\isadelimproof
\isanewline
%
\endisadelimproof
\isanewline
\isacommand{lemma}\isamarkupfalse%
\ meclose{\isacharunderscore}{\kern0pt}pow{\isacharcolon}{\kern0pt}\ {\isachardoublequoteopen}M{\isacharunderscore}{\kern0pt}eclose{\isacharunderscore}{\kern0pt}pow{\isacharparenleft}{\kern0pt}{\isacharhash}{\kern0pt}{\isacharhash}{\kern0pt}M{\isacharparenright}{\kern0pt}{\isachardoublequoteclose}\isanewline
%
\isadelimproof
\ \ %
\endisadelimproof
%
\isatagproof
\isacommand{by}\isamarkupfalse%
\ {\isacharparenleft}{\kern0pt}rule\ intf{\isachardot}{\kern0pt}meclose{\isacharunderscore}{\kern0pt}pow{\isacharparenright}{\kern0pt}%
\endisatagproof
{\isafoldproof}%
%
\isadelimproof
\isanewline
%
\endisadelimproof
\isanewline
\isanewline
\isanewline
\isacommand{end}\isamarkupfalse%
\ \isanewline
\isanewline
\isanewline
\isacommand{sublocale}\isamarkupfalse%
\ M{\isacharunderscore}{\kern0pt}ctm\ {\isasymsubseteq}\ M{\isacharunderscore}{\kern0pt}trivial\ {\isachardoublequoteopen}{\isacharhash}{\kern0pt}{\isacharhash}{\kern0pt}M{\isachardoublequoteclose}\isanewline
%
\isadelimproof
\ \ %
\endisadelimproof
%
\isatagproof
\isacommand{by}\isamarkupfalse%
\ \ {\isacharparenleft}{\kern0pt}rule\ mtriv{\isacharparenright}{\kern0pt}%
\endisatagproof
{\isafoldproof}%
%
\isadelimproof
\isanewline
%
\endisadelimproof
\isanewline
\isacommand{sublocale}\isamarkupfalse%
\ M{\isacharunderscore}{\kern0pt}ctm\ {\isasymsubseteq}\ M{\isacharunderscore}{\kern0pt}trans\ {\isachardoublequoteopen}{\isacharhash}{\kern0pt}{\isacharhash}{\kern0pt}M{\isachardoublequoteclose}\isanewline
%
\isadelimproof
\ \ %
\endisadelimproof
%
\isatagproof
\isacommand{by}\isamarkupfalse%
\ \ {\isacharparenleft}{\kern0pt}rule\ mtrans{\isacharparenright}{\kern0pt}%
\endisatagproof
{\isafoldproof}%
%
\isadelimproof
\isanewline
%
\endisadelimproof
\isanewline
\isacommand{sublocale}\isamarkupfalse%
\ M{\isacharunderscore}{\kern0pt}ctm\ {\isasymsubseteq}\ M{\isacharunderscore}{\kern0pt}basic\ {\isachardoublequoteopen}{\isacharhash}{\kern0pt}{\isacharhash}{\kern0pt}M{\isachardoublequoteclose}\isanewline
%
\isadelimproof
\ \ %
\endisadelimproof
%
\isatagproof
\isacommand{by}\isamarkupfalse%
\ \ {\isacharparenleft}{\kern0pt}rule\ mbasic{\isacharparenright}{\kern0pt}%
\endisatagproof
{\isafoldproof}%
%
\isadelimproof
\isanewline
%
\endisadelimproof
\isanewline
\isacommand{sublocale}\isamarkupfalse%
\ M{\isacharunderscore}{\kern0pt}ctm\ {\isasymsubseteq}\ M{\isacharunderscore}{\kern0pt}trancl\ {\isachardoublequoteopen}{\isacharhash}{\kern0pt}{\isacharhash}{\kern0pt}M{\isachardoublequoteclose}\isanewline
%
\isadelimproof
\ \ %
\endisadelimproof
%
\isatagproof
\isacommand{by}\isamarkupfalse%
\ \ {\isacharparenleft}{\kern0pt}rule\ mtrancl{\isacharparenright}{\kern0pt}%
\endisatagproof
{\isafoldproof}%
%
\isadelimproof
\isanewline
%
\endisadelimproof
\isanewline
\isacommand{sublocale}\isamarkupfalse%
\ M{\isacharunderscore}{\kern0pt}ctm\ {\isasymsubseteq}\ M{\isacharunderscore}{\kern0pt}datatypes\ {\isachardoublequoteopen}{\isacharhash}{\kern0pt}{\isacharhash}{\kern0pt}M{\isachardoublequoteclose}\isanewline
%
\isadelimproof
\ \ %
\endisadelimproof
%
\isatagproof
\isacommand{by}\isamarkupfalse%
\ \ {\isacharparenleft}{\kern0pt}rule\ mdatatypes{\isacharparenright}{\kern0pt}%
\endisatagproof
{\isafoldproof}%
%
\isadelimproof
\isanewline
%
\endisadelimproof
\isanewline
\isacommand{sublocale}\isamarkupfalse%
\ M{\isacharunderscore}{\kern0pt}ctm\ {\isasymsubseteq}\ M{\isacharunderscore}{\kern0pt}eclose\ {\isachardoublequoteopen}{\isacharhash}{\kern0pt}{\isacharhash}{\kern0pt}M{\isachardoublequoteclose}\isanewline
%
\isadelimproof
\ \ %
\endisadelimproof
%
\isatagproof
\isacommand{by}\isamarkupfalse%
\ \ {\isacharparenleft}{\kern0pt}rule\ meclose{\isacharparenright}{\kern0pt}%
\endisatagproof
{\isafoldproof}%
%
\isadelimproof
\isanewline
%
\endisadelimproof
\isanewline
\isacommand{sublocale}\isamarkupfalse%
\ M{\isacharunderscore}{\kern0pt}ctm\ {\isasymsubseteq}\ M{\isacharunderscore}{\kern0pt}eclose{\isacharunderscore}{\kern0pt}pow\ {\isachardoublequoteopen}{\isacharhash}{\kern0pt}{\isacharhash}{\kern0pt}M{\isachardoublequoteclose}\isanewline
%
\isadelimproof
\ \ %
\endisadelimproof
%
\isatagproof
\isacommand{by}\isamarkupfalse%
\ \ {\isacharparenleft}{\kern0pt}rule\ meclose{\isacharunderscore}{\kern0pt}pow{\isacharparenright}{\kern0pt}%
\endisatagproof
{\isafoldproof}%
%
\isadelimproof
\isanewline
%
\endisadelimproof
\isanewline
\isanewline
\isanewline
\isacommand{context}\isamarkupfalse%
\ M{\isacharunderscore}{\kern0pt}ctm\isanewline
\isakeyword{begin}%
\isadelimdocument
%
\endisadelimdocument
%
\isatagdocument
%
\isamarkupsubsection{\isa{Collects} in $M$%
}
\isamarkuptrue%
%
\endisatagdocument
{\isafolddocument}%
%
\isadelimdocument
%
\endisadelimdocument
\isacommand{lemma}\isamarkupfalse%
\ Collect{\isacharunderscore}{\kern0pt}in{\isacharunderscore}{\kern0pt}M{\isacharunderscore}{\kern0pt}{\isadigit{0}}p\ {\isacharcolon}{\kern0pt}\isanewline
\ \ \isakeyword{assumes}\isanewline
\ \ \ \ Qfm\ {\isacharcolon}{\kern0pt}\ {\isachardoublequoteopen}Q{\isacharunderscore}{\kern0pt}fm\ {\isasymin}\ formula{\isachardoublequoteclose}\ \isakeyword{and}\isanewline
\ \ \ \ Qarty\ {\isacharcolon}{\kern0pt}\ {\isachardoublequoteopen}arity{\isacharparenleft}{\kern0pt}Q{\isacharunderscore}{\kern0pt}fm{\isacharparenright}{\kern0pt}\ {\isacharequal}{\kern0pt}\ {\isadigit{1}}{\isachardoublequoteclose}\ \isakeyword{and}\isanewline
\ \ \ \ Qsats\ {\isacharcolon}{\kern0pt}\ {\isachardoublequoteopen}{\isasymAnd}x{\isachardot}{\kern0pt}\ x{\isasymin}M\ {\isasymLongrightarrow}\ sats{\isacharparenleft}{\kern0pt}M{\isacharcomma}{\kern0pt}Q{\isacharunderscore}{\kern0pt}fm{\isacharcomma}{\kern0pt}{\isacharbrackleft}{\kern0pt}x{\isacharbrackright}{\kern0pt}{\isacharparenright}{\kern0pt}\ {\isasymlongleftrightarrow}\ is{\isacharunderscore}{\kern0pt}Q{\isacharparenleft}{\kern0pt}{\isacharhash}{\kern0pt}{\isacharhash}{\kern0pt}M{\isacharcomma}{\kern0pt}x{\isacharparenright}{\kern0pt}{\isachardoublequoteclose}\ \isakeyword{and}\isanewline
\ \ \ \ Qabs\ \ {\isacharcolon}{\kern0pt}\ {\isachardoublequoteopen}{\isasymAnd}x{\isachardot}{\kern0pt}\ x{\isasymin}M\ {\isasymLongrightarrow}\ is{\isacharunderscore}{\kern0pt}Q{\isacharparenleft}{\kern0pt}{\isacharhash}{\kern0pt}{\isacharhash}{\kern0pt}M{\isacharcomma}{\kern0pt}x{\isacharparenright}{\kern0pt}\ {\isasymlongleftrightarrow}\ Q{\isacharparenleft}{\kern0pt}x{\isacharparenright}{\kern0pt}{\isachardoublequoteclose}\ \isakeyword{and}\isanewline
\ \ \ \ {\isachardoublequoteopen}A{\isasymin}M{\isachardoublequoteclose}\isanewline
\ \ \isakeyword{shows}\isanewline
\ \ \ \ {\isachardoublequoteopen}Collect{\isacharparenleft}{\kern0pt}A{\isacharcomma}{\kern0pt}Q{\isacharparenright}{\kern0pt}{\isasymin}M{\isachardoublequoteclose}\ \isanewline
%
\isadelimproof
%
\endisadelimproof
%
\isatagproof
\isacommand{proof}\isamarkupfalse%
\ {\isacharminus}{\kern0pt}\isanewline
\ \ \isacommand{have}\isamarkupfalse%
\ {\isachardoublequoteopen}z{\isasymin}A\ {\isasymLongrightarrow}\ z{\isasymin}M{\isachardoublequoteclose}\ \isakeyword{for}\ z\isanewline
\ \ \ \ \isacommand{using}\isamarkupfalse%
\ {\isacartoucheopen}A{\isasymin}M{\isacartoucheclose}\ transitivity{\isacharbrackleft}{\kern0pt}of\ z\ A{\isacharbrackright}{\kern0pt}\ \isacommand{by}\isamarkupfalse%
\ simp\isanewline
\ \ \isacommand{then}\isamarkupfalse%
\isanewline
\ \ \isacommand{have}\isamarkupfalse%
\ {\isadigit{1}}{\isacharcolon}{\kern0pt}{\isachardoublequoteopen}Collect{\isacharparenleft}{\kern0pt}A{\isacharcomma}{\kern0pt}is{\isacharunderscore}{\kern0pt}Q{\isacharparenleft}{\kern0pt}{\isacharhash}{\kern0pt}{\isacharhash}{\kern0pt}M{\isacharparenright}{\kern0pt}{\isacharparenright}{\kern0pt}\ {\isacharequal}{\kern0pt}\ Collect{\isacharparenleft}{\kern0pt}A{\isacharcomma}{\kern0pt}Q{\isacharparenright}{\kern0pt}{\isachardoublequoteclose}\ \isanewline
\ \ \ \ \isacommand{using}\isamarkupfalse%
\ Qabs\ Collect{\isacharunderscore}{\kern0pt}cong{\isacharbrackleft}{\kern0pt}of\ {\isachardoublequoteopen}A{\isachardoublequoteclose}\ {\isachardoublequoteopen}A{\isachardoublequoteclose}\ {\isachardoublequoteopen}is{\isacharunderscore}{\kern0pt}Q{\isacharparenleft}{\kern0pt}{\isacharhash}{\kern0pt}{\isacharhash}{\kern0pt}M{\isacharparenright}{\kern0pt}{\isachardoublequoteclose}\ {\isachardoublequoteopen}Q{\isachardoublequoteclose}{\isacharbrackright}{\kern0pt}\ \isacommand{by}\isamarkupfalse%
\ simp\isanewline
\ \ \isacommand{have}\isamarkupfalse%
\ {\isachardoublequoteopen}separation{\isacharparenleft}{\kern0pt}{\isacharhash}{\kern0pt}{\isacharhash}{\kern0pt}M{\isacharcomma}{\kern0pt}is{\isacharunderscore}{\kern0pt}Q{\isacharparenleft}{\kern0pt}{\isacharhash}{\kern0pt}{\isacharhash}{\kern0pt}M{\isacharparenright}{\kern0pt}{\isacharparenright}{\kern0pt}{\isachardoublequoteclose}\isanewline
\ \ \ \ \isacommand{using}\isamarkupfalse%
\ separation{\isacharunderscore}{\kern0pt}ax\ Qsats\ Qarty\ Qfm\isanewline
\ \ \ \ \ \ separation{\isacharunderscore}{\kern0pt}cong{\isacharbrackleft}{\kern0pt}of\ {\isachardoublequoteopen}{\isacharhash}{\kern0pt}{\isacharhash}{\kern0pt}M{\isachardoublequoteclose}\ {\isachardoublequoteopen}{\isasymlambda}y{\isachardot}{\kern0pt}\ sats{\isacharparenleft}{\kern0pt}M{\isacharcomma}{\kern0pt}Q{\isacharunderscore}{\kern0pt}fm{\isacharcomma}{\kern0pt}{\isacharbrackleft}{\kern0pt}y{\isacharbrackright}{\kern0pt}{\isacharparenright}{\kern0pt}{\isachardoublequoteclose}\ {\isachardoublequoteopen}is{\isacharunderscore}{\kern0pt}Q{\isacharparenleft}{\kern0pt}{\isacharhash}{\kern0pt}{\isacharhash}{\kern0pt}M{\isacharparenright}{\kern0pt}{\isachardoublequoteclose}{\isacharbrackright}{\kern0pt}\isanewline
\ \ \ \ \isacommand{by}\isamarkupfalse%
\ simp\isanewline
\ \ \isacommand{then}\isamarkupfalse%
\ \isanewline
\ \ \isacommand{have}\isamarkupfalse%
\ {\isachardoublequoteopen}Collect{\isacharparenleft}{\kern0pt}A{\isacharcomma}{\kern0pt}is{\isacharunderscore}{\kern0pt}Q{\isacharparenleft}{\kern0pt}{\isacharhash}{\kern0pt}{\isacharhash}{\kern0pt}M{\isacharparenright}{\kern0pt}{\isacharparenright}{\kern0pt}{\isasymin}M{\isachardoublequoteclose}\isanewline
\ \ \ \ \isacommand{using}\isamarkupfalse%
\ separation{\isacharunderscore}{\kern0pt}closed\ {\isacartoucheopen}A{\isasymin}M{\isacartoucheclose}\ \isacommand{by}\isamarkupfalse%
\ simp\ \isanewline
\ \ \isacommand{then}\isamarkupfalse%
\isanewline
\ \ \isacommand{show}\isamarkupfalse%
\ {\isacharquery}{\kern0pt}thesis\ \isacommand{using}\isamarkupfalse%
\ {\isadigit{1}}\ \isacommand{by}\isamarkupfalse%
\ simp\isanewline
\isacommand{qed}\isamarkupfalse%
%
\endisatagproof
{\isafoldproof}%
%
\isadelimproof
\isanewline
%
\endisadelimproof
\isanewline
\isacommand{lemma}\isamarkupfalse%
\ Collect{\isacharunderscore}{\kern0pt}in{\isacharunderscore}{\kern0pt}M{\isacharunderscore}{\kern0pt}{\isadigit{2}}p\ {\isacharcolon}{\kern0pt}\isanewline
\ \ \isakeyword{assumes}\isanewline
\ \ \ \ Qfm\ {\isacharcolon}{\kern0pt}\ {\isachardoublequoteopen}Q{\isacharunderscore}{\kern0pt}fm\ {\isasymin}\ formula{\isachardoublequoteclose}\ \isakeyword{and}\isanewline
\ \ \ \ Qarty\ {\isacharcolon}{\kern0pt}\ {\isachardoublequoteopen}arity{\isacharparenleft}{\kern0pt}Q{\isacharunderscore}{\kern0pt}fm{\isacharparenright}{\kern0pt}\ {\isacharequal}{\kern0pt}\ {\isadigit{3}}{\isachardoublequoteclose}\ \isakeyword{and}\isanewline
\ \ \ \ params{\isacharunderscore}{\kern0pt}M\ {\isacharcolon}{\kern0pt}\ {\isachardoublequoteopen}y{\isasymin}M{\isachardoublequoteclose}\ {\isachardoublequoteopen}z{\isasymin}M{\isachardoublequoteclose}\ \isakeyword{and}\isanewline
\ \ \ \ Qsats\ {\isacharcolon}{\kern0pt}\ {\isachardoublequoteopen}{\isasymAnd}x{\isachardot}{\kern0pt}\ x{\isasymin}M\ {\isasymLongrightarrow}\ sats{\isacharparenleft}{\kern0pt}M{\isacharcomma}{\kern0pt}Q{\isacharunderscore}{\kern0pt}fm{\isacharcomma}{\kern0pt}{\isacharbrackleft}{\kern0pt}x{\isacharcomma}{\kern0pt}y{\isacharcomma}{\kern0pt}z{\isacharbrackright}{\kern0pt}{\isacharparenright}{\kern0pt}\ {\isasymlongleftrightarrow}\ is{\isacharunderscore}{\kern0pt}Q{\isacharparenleft}{\kern0pt}{\isacharhash}{\kern0pt}{\isacharhash}{\kern0pt}M{\isacharcomma}{\kern0pt}x{\isacharcomma}{\kern0pt}y{\isacharcomma}{\kern0pt}z{\isacharparenright}{\kern0pt}{\isachardoublequoteclose}\ \isakeyword{and}\isanewline
\ \ \ \ Qabs\ \ {\isacharcolon}{\kern0pt}\ {\isachardoublequoteopen}{\isasymAnd}x{\isachardot}{\kern0pt}\ x{\isasymin}M\ {\isasymLongrightarrow}\ is{\isacharunderscore}{\kern0pt}Q{\isacharparenleft}{\kern0pt}{\isacharhash}{\kern0pt}{\isacharhash}{\kern0pt}M{\isacharcomma}{\kern0pt}x{\isacharcomma}{\kern0pt}y{\isacharcomma}{\kern0pt}z{\isacharparenright}{\kern0pt}\ {\isasymlongleftrightarrow}\ Q{\isacharparenleft}{\kern0pt}x{\isacharcomma}{\kern0pt}y{\isacharcomma}{\kern0pt}z{\isacharparenright}{\kern0pt}{\isachardoublequoteclose}\ \isakeyword{and}\isanewline
\ \ \ \ {\isachardoublequoteopen}A{\isasymin}M{\isachardoublequoteclose}\isanewline
\ \ \isakeyword{shows}\isanewline
\ \ \ \ {\isachardoublequoteopen}Collect{\isacharparenleft}{\kern0pt}A{\isacharcomma}{\kern0pt}{\isasymlambda}x{\isachardot}{\kern0pt}\ Q{\isacharparenleft}{\kern0pt}x{\isacharcomma}{\kern0pt}y{\isacharcomma}{\kern0pt}z{\isacharparenright}{\kern0pt}{\isacharparenright}{\kern0pt}{\isasymin}M{\isachardoublequoteclose}\ \isanewline
%
\isadelimproof
%
\endisadelimproof
%
\isatagproof
\isacommand{proof}\isamarkupfalse%
\ {\isacharminus}{\kern0pt}\isanewline
\ \ \isacommand{have}\isamarkupfalse%
\ {\isachardoublequoteopen}z{\isasymin}A\ {\isasymLongrightarrow}\ z{\isasymin}M{\isachardoublequoteclose}\ \isakeyword{for}\ z\isanewline
\ \ \ \ \isacommand{using}\isamarkupfalse%
\ {\isacartoucheopen}A{\isasymin}M{\isacartoucheclose}\ transitivity{\isacharbrackleft}{\kern0pt}of\ z\ A{\isacharbrackright}{\kern0pt}\ \isacommand{by}\isamarkupfalse%
\ simp\isanewline
\ \ \isacommand{then}\isamarkupfalse%
\isanewline
\ \ \isacommand{have}\isamarkupfalse%
\ {\isadigit{1}}{\isacharcolon}{\kern0pt}{\isachardoublequoteopen}Collect{\isacharparenleft}{\kern0pt}A{\isacharcomma}{\kern0pt}{\isasymlambda}x{\isachardot}{\kern0pt}\ is{\isacharunderscore}{\kern0pt}Q{\isacharparenleft}{\kern0pt}{\isacharhash}{\kern0pt}{\isacharhash}{\kern0pt}M{\isacharcomma}{\kern0pt}x{\isacharcomma}{\kern0pt}y{\isacharcomma}{\kern0pt}z{\isacharparenright}{\kern0pt}{\isacharparenright}{\kern0pt}\ {\isacharequal}{\kern0pt}\ Collect{\isacharparenleft}{\kern0pt}A{\isacharcomma}{\kern0pt}{\isasymlambda}x{\isachardot}{\kern0pt}\ Q{\isacharparenleft}{\kern0pt}x{\isacharcomma}{\kern0pt}y{\isacharcomma}{\kern0pt}z{\isacharparenright}{\kern0pt}{\isacharparenright}{\kern0pt}{\isachardoublequoteclose}\ \isanewline
\ \ \ \ \isacommand{using}\isamarkupfalse%
\ Qabs\ Collect{\isacharunderscore}{\kern0pt}cong{\isacharbrackleft}{\kern0pt}of\ {\isachardoublequoteopen}A{\isachardoublequoteclose}\ {\isachardoublequoteopen}A{\isachardoublequoteclose}\ {\isachardoublequoteopen}{\isasymlambda}x{\isachardot}{\kern0pt}\ is{\isacharunderscore}{\kern0pt}Q{\isacharparenleft}{\kern0pt}{\isacharhash}{\kern0pt}{\isacharhash}{\kern0pt}M{\isacharcomma}{\kern0pt}x{\isacharcomma}{\kern0pt}y{\isacharcomma}{\kern0pt}z{\isacharparenright}{\kern0pt}{\isachardoublequoteclose}\ {\isachardoublequoteopen}{\isasymlambda}x{\isachardot}{\kern0pt}\ Q{\isacharparenleft}{\kern0pt}x{\isacharcomma}{\kern0pt}y{\isacharcomma}{\kern0pt}z{\isacharparenright}{\kern0pt}{\isachardoublequoteclose}{\isacharbrackright}{\kern0pt}\ \isacommand{by}\isamarkupfalse%
\ simp\isanewline
\ \ \isacommand{have}\isamarkupfalse%
\ {\isachardoublequoteopen}separation{\isacharparenleft}{\kern0pt}{\isacharhash}{\kern0pt}{\isacharhash}{\kern0pt}M{\isacharcomma}{\kern0pt}{\isasymlambda}x{\isachardot}{\kern0pt}\ is{\isacharunderscore}{\kern0pt}Q{\isacharparenleft}{\kern0pt}{\isacharhash}{\kern0pt}{\isacharhash}{\kern0pt}M{\isacharcomma}{\kern0pt}x{\isacharcomma}{\kern0pt}y{\isacharcomma}{\kern0pt}z{\isacharparenright}{\kern0pt}{\isacharparenright}{\kern0pt}{\isachardoublequoteclose}\isanewline
\ \ \ \ \isacommand{using}\isamarkupfalse%
\ separation{\isacharunderscore}{\kern0pt}ax\ Qsats\ Qarty\ Qfm\ params{\isacharunderscore}{\kern0pt}M\isanewline
\ \ \ \ \ \ separation{\isacharunderscore}{\kern0pt}cong{\isacharbrackleft}{\kern0pt}of\ {\isachardoublequoteopen}{\isacharhash}{\kern0pt}{\isacharhash}{\kern0pt}M{\isachardoublequoteclose}\ {\isachardoublequoteopen}{\isasymlambda}x{\isachardot}{\kern0pt}\ sats{\isacharparenleft}{\kern0pt}M{\isacharcomma}{\kern0pt}Q{\isacharunderscore}{\kern0pt}fm{\isacharcomma}{\kern0pt}{\isacharbrackleft}{\kern0pt}x{\isacharcomma}{\kern0pt}y{\isacharcomma}{\kern0pt}z{\isacharbrackright}{\kern0pt}{\isacharparenright}{\kern0pt}{\isachardoublequoteclose}\ {\isachardoublequoteopen}{\isasymlambda}x{\isachardot}{\kern0pt}\ is{\isacharunderscore}{\kern0pt}Q{\isacharparenleft}{\kern0pt}{\isacharhash}{\kern0pt}{\isacharhash}{\kern0pt}M{\isacharcomma}{\kern0pt}x{\isacharcomma}{\kern0pt}y{\isacharcomma}{\kern0pt}z{\isacharparenright}{\kern0pt}{\isachardoublequoteclose}{\isacharbrackright}{\kern0pt}\isanewline
\ \ \ \ \isacommand{by}\isamarkupfalse%
\ simp\isanewline
\ \ \isacommand{then}\isamarkupfalse%
\ \isanewline
\ \ \isacommand{have}\isamarkupfalse%
\ {\isachardoublequoteopen}Collect{\isacharparenleft}{\kern0pt}A{\isacharcomma}{\kern0pt}{\isasymlambda}x{\isachardot}{\kern0pt}\ is{\isacharunderscore}{\kern0pt}Q{\isacharparenleft}{\kern0pt}{\isacharhash}{\kern0pt}{\isacharhash}{\kern0pt}M{\isacharcomma}{\kern0pt}x{\isacharcomma}{\kern0pt}y{\isacharcomma}{\kern0pt}z{\isacharparenright}{\kern0pt}{\isacharparenright}{\kern0pt}{\isasymin}M{\isachardoublequoteclose}\isanewline
\ \ \ \ \isacommand{using}\isamarkupfalse%
\ separation{\isacharunderscore}{\kern0pt}closed\ {\isacartoucheopen}A{\isasymin}M{\isacartoucheclose}\ \isacommand{by}\isamarkupfalse%
\ simp\ \isanewline
\ \ \isacommand{then}\isamarkupfalse%
\isanewline
\ \ \isacommand{show}\isamarkupfalse%
\ {\isacharquery}{\kern0pt}thesis\ \isacommand{using}\isamarkupfalse%
\ {\isadigit{1}}\ \isacommand{by}\isamarkupfalse%
\ simp\isanewline
\isacommand{qed}\isamarkupfalse%
%
\endisatagproof
{\isafoldproof}%
%
\isadelimproof
\isanewline
%
\endisadelimproof
\isanewline
\isacommand{lemma}\isamarkupfalse%
\ Collect{\isacharunderscore}{\kern0pt}in{\isacharunderscore}{\kern0pt}M{\isacharunderscore}{\kern0pt}{\isadigit{4}}p\ {\isacharcolon}{\kern0pt}\isanewline
\ \ \isakeyword{assumes}\isanewline
\ \ \ \ Qfm\ {\isacharcolon}{\kern0pt}\ {\isachardoublequoteopen}Q{\isacharunderscore}{\kern0pt}fm\ {\isasymin}\ formula{\isachardoublequoteclose}\ \isakeyword{and}\isanewline
\ \ \ \ Qarty\ {\isacharcolon}{\kern0pt}\ {\isachardoublequoteopen}arity{\isacharparenleft}{\kern0pt}Q{\isacharunderscore}{\kern0pt}fm{\isacharparenright}{\kern0pt}\ {\isacharequal}{\kern0pt}\ {\isadigit{5}}{\isachardoublequoteclose}\ \isakeyword{and}\isanewline
\ \ \ \ params{\isacharunderscore}{\kern0pt}M\ {\isacharcolon}{\kern0pt}\ {\isachardoublequoteopen}a{\isadigit{1}}{\isasymin}M{\isachardoublequoteclose}\ {\isachardoublequoteopen}a{\isadigit{2}}{\isasymin}M{\isachardoublequoteclose}\ {\isachardoublequoteopen}a{\isadigit{3}}{\isasymin}M{\isachardoublequoteclose}\ {\isachardoublequoteopen}a{\isadigit{4}}{\isasymin}M{\isachardoublequoteclose}\ \isakeyword{and}\isanewline
\ \ \ \ Qsats\ {\isacharcolon}{\kern0pt}\ {\isachardoublequoteopen}{\isasymAnd}x{\isachardot}{\kern0pt}\ x{\isasymin}M\ {\isasymLongrightarrow}\ sats{\isacharparenleft}{\kern0pt}M{\isacharcomma}{\kern0pt}Q{\isacharunderscore}{\kern0pt}fm{\isacharcomma}{\kern0pt}{\isacharbrackleft}{\kern0pt}x{\isacharcomma}{\kern0pt}a{\isadigit{1}}{\isacharcomma}{\kern0pt}a{\isadigit{2}}{\isacharcomma}{\kern0pt}a{\isadigit{3}}{\isacharcomma}{\kern0pt}a{\isadigit{4}}{\isacharbrackright}{\kern0pt}{\isacharparenright}{\kern0pt}\ {\isasymlongleftrightarrow}\ is{\isacharunderscore}{\kern0pt}Q{\isacharparenleft}{\kern0pt}{\isacharhash}{\kern0pt}{\isacharhash}{\kern0pt}M{\isacharcomma}{\kern0pt}x{\isacharcomma}{\kern0pt}a{\isadigit{1}}{\isacharcomma}{\kern0pt}a{\isadigit{2}}{\isacharcomma}{\kern0pt}a{\isadigit{3}}{\isacharcomma}{\kern0pt}a{\isadigit{4}}{\isacharparenright}{\kern0pt}{\isachardoublequoteclose}\ \isakeyword{and}\isanewline
\ \ \ \ Qabs\ \ {\isacharcolon}{\kern0pt}\ {\isachardoublequoteopen}{\isasymAnd}x{\isachardot}{\kern0pt}\ x{\isasymin}M\ {\isasymLongrightarrow}\ is{\isacharunderscore}{\kern0pt}Q{\isacharparenleft}{\kern0pt}{\isacharhash}{\kern0pt}{\isacharhash}{\kern0pt}M{\isacharcomma}{\kern0pt}x{\isacharcomma}{\kern0pt}a{\isadigit{1}}{\isacharcomma}{\kern0pt}a{\isadigit{2}}{\isacharcomma}{\kern0pt}a{\isadigit{3}}{\isacharcomma}{\kern0pt}a{\isadigit{4}}{\isacharparenright}{\kern0pt}\ {\isasymlongleftrightarrow}\ Q{\isacharparenleft}{\kern0pt}x{\isacharcomma}{\kern0pt}a{\isadigit{1}}{\isacharcomma}{\kern0pt}a{\isadigit{2}}{\isacharcomma}{\kern0pt}a{\isadigit{3}}{\isacharcomma}{\kern0pt}a{\isadigit{4}}{\isacharparenright}{\kern0pt}{\isachardoublequoteclose}\ \isakeyword{and}\isanewline
\ \ \ \ {\isachardoublequoteopen}A{\isasymin}M{\isachardoublequoteclose}\isanewline
\ \ \isakeyword{shows}\isanewline
\ \ \ \ {\isachardoublequoteopen}Collect{\isacharparenleft}{\kern0pt}A{\isacharcomma}{\kern0pt}{\isasymlambda}x{\isachardot}{\kern0pt}\ Q{\isacharparenleft}{\kern0pt}x{\isacharcomma}{\kern0pt}a{\isadigit{1}}{\isacharcomma}{\kern0pt}a{\isadigit{2}}{\isacharcomma}{\kern0pt}a{\isadigit{3}}{\isacharcomma}{\kern0pt}a{\isadigit{4}}{\isacharparenright}{\kern0pt}{\isacharparenright}{\kern0pt}{\isasymin}M{\isachardoublequoteclose}\ \isanewline
%
\isadelimproof
%
\endisadelimproof
%
\isatagproof
\isacommand{proof}\isamarkupfalse%
\ {\isacharminus}{\kern0pt}\isanewline
\ \ \isacommand{have}\isamarkupfalse%
\ {\isachardoublequoteopen}z{\isasymin}A\ {\isasymLongrightarrow}\ z{\isasymin}M{\isachardoublequoteclose}\ \isakeyword{for}\ z\isanewline
\ \ \ \ \isacommand{using}\isamarkupfalse%
\ {\isacartoucheopen}A{\isasymin}M{\isacartoucheclose}\ transitivity{\isacharbrackleft}{\kern0pt}of\ z\ A{\isacharbrackright}{\kern0pt}\ \isacommand{by}\isamarkupfalse%
\ simp\isanewline
\ \ \isacommand{then}\isamarkupfalse%
\isanewline
\ \ \isacommand{have}\isamarkupfalse%
\ {\isadigit{1}}{\isacharcolon}{\kern0pt}{\isachardoublequoteopen}Collect{\isacharparenleft}{\kern0pt}A{\isacharcomma}{\kern0pt}{\isasymlambda}x{\isachardot}{\kern0pt}\ is{\isacharunderscore}{\kern0pt}Q{\isacharparenleft}{\kern0pt}{\isacharhash}{\kern0pt}{\isacharhash}{\kern0pt}M{\isacharcomma}{\kern0pt}x{\isacharcomma}{\kern0pt}a{\isadigit{1}}{\isacharcomma}{\kern0pt}a{\isadigit{2}}{\isacharcomma}{\kern0pt}a{\isadigit{3}}{\isacharcomma}{\kern0pt}a{\isadigit{4}}{\isacharparenright}{\kern0pt}{\isacharparenright}{\kern0pt}\ {\isacharequal}{\kern0pt}\ Collect{\isacharparenleft}{\kern0pt}A{\isacharcomma}{\kern0pt}{\isasymlambda}x{\isachardot}{\kern0pt}\ Q{\isacharparenleft}{\kern0pt}x{\isacharcomma}{\kern0pt}a{\isadigit{1}}{\isacharcomma}{\kern0pt}a{\isadigit{2}}{\isacharcomma}{\kern0pt}a{\isadigit{3}}{\isacharcomma}{\kern0pt}a{\isadigit{4}}{\isacharparenright}{\kern0pt}{\isacharparenright}{\kern0pt}{\isachardoublequoteclose}\ \isanewline
\ \ \ \ \isacommand{using}\isamarkupfalse%
\ Qabs\ Collect{\isacharunderscore}{\kern0pt}cong{\isacharbrackleft}{\kern0pt}of\ {\isachardoublequoteopen}A{\isachardoublequoteclose}\ {\isachardoublequoteopen}A{\isachardoublequoteclose}\ {\isachardoublequoteopen}{\isasymlambda}x{\isachardot}{\kern0pt}\ is{\isacharunderscore}{\kern0pt}Q{\isacharparenleft}{\kern0pt}{\isacharhash}{\kern0pt}{\isacharhash}{\kern0pt}M{\isacharcomma}{\kern0pt}x{\isacharcomma}{\kern0pt}a{\isadigit{1}}{\isacharcomma}{\kern0pt}a{\isadigit{2}}{\isacharcomma}{\kern0pt}a{\isadigit{3}}{\isacharcomma}{\kern0pt}a{\isadigit{4}}{\isacharparenright}{\kern0pt}{\isachardoublequoteclose}\ {\isachardoublequoteopen}{\isasymlambda}x{\isachardot}{\kern0pt}\ Q{\isacharparenleft}{\kern0pt}x{\isacharcomma}{\kern0pt}a{\isadigit{1}}{\isacharcomma}{\kern0pt}a{\isadigit{2}}{\isacharcomma}{\kern0pt}a{\isadigit{3}}{\isacharcomma}{\kern0pt}a{\isadigit{4}}{\isacharparenright}{\kern0pt}{\isachardoublequoteclose}{\isacharbrackright}{\kern0pt}\ \isanewline
\ \ \ \ \isacommand{by}\isamarkupfalse%
\ simp\isanewline
\ \ \isacommand{have}\isamarkupfalse%
\ {\isachardoublequoteopen}separation{\isacharparenleft}{\kern0pt}{\isacharhash}{\kern0pt}{\isacharhash}{\kern0pt}M{\isacharcomma}{\kern0pt}{\isasymlambda}x{\isachardot}{\kern0pt}\ is{\isacharunderscore}{\kern0pt}Q{\isacharparenleft}{\kern0pt}{\isacharhash}{\kern0pt}{\isacharhash}{\kern0pt}M{\isacharcomma}{\kern0pt}x{\isacharcomma}{\kern0pt}a{\isadigit{1}}{\isacharcomma}{\kern0pt}a{\isadigit{2}}{\isacharcomma}{\kern0pt}a{\isadigit{3}}{\isacharcomma}{\kern0pt}a{\isadigit{4}}{\isacharparenright}{\kern0pt}{\isacharparenright}{\kern0pt}{\isachardoublequoteclose}\isanewline
\ \ \ \ \isacommand{using}\isamarkupfalse%
\ separation{\isacharunderscore}{\kern0pt}ax\ Qsats\ Qarty\ Qfm\ params{\isacharunderscore}{\kern0pt}M\isanewline
\ \ \ \ \ \ separation{\isacharunderscore}{\kern0pt}cong{\isacharbrackleft}{\kern0pt}of\ {\isachardoublequoteopen}{\isacharhash}{\kern0pt}{\isacharhash}{\kern0pt}M{\isachardoublequoteclose}\ {\isachardoublequoteopen}{\isasymlambda}x{\isachardot}{\kern0pt}\ sats{\isacharparenleft}{\kern0pt}M{\isacharcomma}{\kern0pt}Q{\isacharunderscore}{\kern0pt}fm{\isacharcomma}{\kern0pt}{\isacharbrackleft}{\kern0pt}x{\isacharcomma}{\kern0pt}a{\isadigit{1}}{\isacharcomma}{\kern0pt}a{\isadigit{2}}{\isacharcomma}{\kern0pt}a{\isadigit{3}}{\isacharcomma}{\kern0pt}a{\isadigit{4}}{\isacharbrackright}{\kern0pt}{\isacharparenright}{\kern0pt}{\isachardoublequoteclose}\ \isanewline
\ \ \ \ \ \ \ \ {\isachardoublequoteopen}{\isasymlambda}x{\isachardot}{\kern0pt}\ is{\isacharunderscore}{\kern0pt}Q{\isacharparenleft}{\kern0pt}{\isacharhash}{\kern0pt}{\isacharhash}{\kern0pt}M{\isacharcomma}{\kern0pt}x{\isacharcomma}{\kern0pt}a{\isadigit{1}}{\isacharcomma}{\kern0pt}a{\isadigit{2}}{\isacharcomma}{\kern0pt}a{\isadigit{3}}{\isacharcomma}{\kern0pt}a{\isadigit{4}}{\isacharparenright}{\kern0pt}{\isachardoublequoteclose}{\isacharbrackright}{\kern0pt}\isanewline
\ \ \ \ \isacommand{by}\isamarkupfalse%
\ simp\isanewline
\ \ \isacommand{then}\isamarkupfalse%
\ \isanewline
\ \ \isacommand{have}\isamarkupfalse%
\ {\isachardoublequoteopen}Collect{\isacharparenleft}{\kern0pt}A{\isacharcomma}{\kern0pt}{\isasymlambda}x{\isachardot}{\kern0pt}\ is{\isacharunderscore}{\kern0pt}Q{\isacharparenleft}{\kern0pt}{\isacharhash}{\kern0pt}{\isacharhash}{\kern0pt}M{\isacharcomma}{\kern0pt}x{\isacharcomma}{\kern0pt}a{\isadigit{1}}{\isacharcomma}{\kern0pt}a{\isadigit{2}}{\isacharcomma}{\kern0pt}a{\isadigit{3}}{\isacharcomma}{\kern0pt}a{\isadigit{4}}{\isacharparenright}{\kern0pt}{\isacharparenright}{\kern0pt}{\isasymin}M{\isachardoublequoteclose}\isanewline
\ \ \ \ \isacommand{using}\isamarkupfalse%
\ separation{\isacharunderscore}{\kern0pt}closed\ {\isacartoucheopen}A{\isasymin}M{\isacartoucheclose}\ \isacommand{by}\isamarkupfalse%
\ simp\ \isanewline
\ \ \isacommand{then}\isamarkupfalse%
\isanewline
\ \ \isacommand{show}\isamarkupfalse%
\ {\isacharquery}{\kern0pt}thesis\ \isacommand{using}\isamarkupfalse%
\ {\isadigit{1}}\ \isacommand{by}\isamarkupfalse%
\ simp\isanewline
\isacommand{qed}\isamarkupfalse%
%
\endisatagproof
{\isafoldproof}%
%
\isadelimproof
\isanewline
%
\endisadelimproof
\isanewline
\isacommand{lemma}\isamarkupfalse%
\ Repl{\isacharunderscore}{\kern0pt}in{\isacharunderscore}{\kern0pt}M\ {\isacharcolon}{\kern0pt}\isanewline
\ \ \isakeyword{assumes}\isanewline
\ \ \ \ f{\isacharunderscore}{\kern0pt}fm{\isacharcolon}{\kern0pt}\ \ {\isachardoublequoteopen}f{\isacharunderscore}{\kern0pt}fm\ {\isasymin}\ formula{\isachardoublequoteclose}\ \isakeyword{and}\isanewline
\ \ \ \ f{\isacharunderscore}{\kern0pt}ar{\isacharcolon}{\kern0pt}\ \ {\isachardoublequoteopen}arity{\isacharparenleft}{\kern0pt}f{\isacharunderscore}{\kern0pt}fm{\isacharparenright}{\kern0pt}{\isasymle}\ {\isadigit{2}}\ {\isacharhash}{\kern0pt}{\isacharplus}{\kern0pt}\ length{\isacharparenleft}{\kern0pt}env{\isacharparenright}{\kern0pt}{\isachardoublequoteclose}\ \isakeyword{and}\isanewline
\ \ \ \ fsats{\isacharcolon}{\kern0pt}\ {\isachardoublequoteopen}{\isasymAnd}x\ y{\isachardot}{\kern0pt}\ x{\isasymin}M\ {\isasymLongrightarrow}\ y{\isasymin}M\ {\isasymLongrightarrow}\ sats{\isacharparenleft}{\kern0pt}M{\isacharcomma}{\kern0pt}f{\isacharunderscore}{\kern0pt}fm{\isacharcomma}{\kern0pt}{\isacharbrackleft}{\kern0pt}x{\isacharcomma}{\kern0pt}y{\isacharbrackright}{\kern0pt}{\isacharat}{\kern0pt}env{\isacharparenright}{\kern0pt}\ {\isasymlongleftrightarrow}\ is{\isacharunderscore}{\kern0pt}f{\isacharparenleft}{\kern0pt}x{\isacharcomma}{\kern0pt}y{\isacharparenright}{\kern0pt}{\isachardoublequoteclose}\ \isakeyword{and}\isanewline
\ \ \ \ fabs{\isacharcolon}{\kern0pt}\ \ {\isachardoublequoteopen}{\isasymAnd}x\ y{\isachardot}{\kern0pt}\ x{\isasymin}M\ {\isasymLongrightarrow}\ y{\isasymin}M\ {\isasymLongrightarrow}\ is{\isacharunderscore}{\kern0pt}f{\isacharparenleft}{\kern0pt}x{\isacharcomma}{\kern0pt}y{\isacharparenright}{\kern0pt}\ {\isasymlongleftrightarrow}\ y\ {\isacharequal}{\kern0pt}\ f{\isacharparenleft}{\kern0pt}x{\isacharparenright}{\kern0pt}{\isachardoublequoteclose}\ \isakeyword{and}\isanewline
\ \ \ \ fclosed{\isacharcolon}{\kern0pt}\ {\isachardoublequoteopen}{\isasymAnd}x{\isachardot}{\kern0pt}\ x{\isasymin}A\ {\isasymLongrightarrow}\ f{\isacharparenleft}{\kern0pt}x{\isacharparenright}{\kern0pt}\ {\isasymin}\ M{\isachardoublequoteclose}\ \ \isakeyword{and}\isanewline
\ \ \ \ {\isachardoublequoteopen}A{\isasymin}M{\isachardoublequoteclose}\ {\isachardoublequoteopen}env{\isasymin}list{\isacharparenleft}{\kern0pt}M{\isacharparenright}{\kern0pt}{\isachardoublequoteclose}\ \isanewline
\ \ \isakeyword{shows}\ {\isachardoublequoteopen}{\isacharbraceleft}{\kern0pt}f{\isacharparenleft}{\kern0pt}x{\isacharparenright}{\kern0pt}{\isachardot}{\kern0pt}\ x{\isasymin}A{\isacharbraceright}{\kern0pt}{\isasymin}M{\isachardoublequoteclose}\isanewline
%
\isadelimproof
%
\endisadelimproof
%
\isatagproof
\isacommand{proof}\isamarkupfalse%
\ {\isacharminus}{\kern0pt}\isanewline
\ \ \isacommand{have}\isamarkupfalse%
\ {\isachardoublequoteopen}strong{\isacharunderscore}{\kern0pt}replacement{\isacharparenleft}{\kern0pt}{\isacharhash}{\kern0pt}{\isacharhash}{\kern0pt}M{\isacharcomma}{\kern0pt}\ {\isasymlambda}x\ y{\isachardot}{\kern0pt}\ sats{\isacharparenleft}{\kern0pt}M{\isacharcomma}{\kern0pt}f{\isacharunderscore}{\kern0pt}fm{\isacharcomma}{\kern0pt}{\isacharbrackleft}{\kern0pt}x{\isacharcomma}{\kern0pt}y{\isacharbrackright}{\kern0pt}{\isacharat}{\kern0pt}env{\isacharparenright}{\kern0pt}{\isacharparenright}{\kern0pt}{\isachardoublequoteclose}\isanewline
\ \ \ \ \isacommand{using}\isamarkupfalse%
\ replacement{\isacharunderscore}{\kern0pt}ax\ f{\isacharunderscore}{\kern0pt}fm\ f{\isacharunderscore}{\kern0pt}ar\ {\isacartoucheopen}env{\isasymin}list{\isacharparenleft}{\kern0pt}M{\isacharparenright}{\kern0pt}{\isacartoucheclose}\ \isacommand{by}\isamarkupfalse%
\ simp\isanewline
\ \ \isacommand{then}\isamarkupfalse%
\isanewline
\ \ \isacommand{have}\isamarkupfalse%
\ {\isachardoublequoteopen}strong{\isacharunderscore}{\kern0pt}replacement{\isacharparenleft}{\kern0pt}{\isacharhash}{\kern0pt}{\isacharhash}{\kern0pt}M{\isacharcomma}{\kern0pt}\ {\isasymlambda}x\ y{\isachardot}{\kern0pt}\ y\ {\isacharequal}{\kern0pt}\ f{\isacharparenleft}{\kern0pt}x{\isacharparenright}{\kern0pt}{\isacharparenright}{\kern0pt}{\isachardoublequoteclose}\isanewline
\ \ \ \ \isacommand{using}\isamarkupfalse%
\ fsats\ fabs\ \isanewline
\ \ \ \ \ \ strong{\isacharunderscore}{\kern0pt}replacement{\isacharunderscore}{\kern0pt}cong{\isacharbrackleft}{\kern0pt}of\ {\isachardoublequoteopen}{\isacharhash}{\kern0pt}{\isacharhash}{\kern0pt}M{\isachardoublequoteclose}\ {\isachardoublequoteopen}{\isasymlambda}x\ y{\isachardot}{\kern0pt}\ sats{\isacharparenleft}{\kern0pt}M{\isacharcomma}{\kern0pt}f{\isacharunderscore}{\kern0pt}fm{\isacharcomma}{\kern0pt}{\isacharbrackleft}{\kern0pt}x{\isacharcomma}{\kern0pt}y{\isacharbrackright}{\kern0pt}{\isacharat}{\kern0pt}env{\isacharparenright}{\kern0pt}{\isachardoublequoteclose}\ {\isachardoublequoteopen}{\isasymlambda}x\ y{\isachardot}{\kern0pt}\ y\ {\isacharequal}{\kern0pt}\ f{\isacharparenleft}{\kern0pt}x{\isacharparenright}{\kern0pt}{\isachardoublequoteclose}{\isacharbrackright}{\kern0pt}\isanewline
\ \ \ \ \isacommand{by}\isamarkupfalse%
\ simp\isanewline
\ \ \isacommand{then}\isamarkupfalse%
\isanewline
\ \ \isacommand{have}\isamarkupfalse%
\ {\isachardoublequoteopen}{\isacharbraceleft}{\kern0pt}\ y\ {\isachardot}{\kern0pt}\ x{\isasymin}A\ {\isacharcomma}{\kern0pt}\ y\ {\isacharequal}{\kern0pt}\ f{\isacharparenleft}{\kern0pt}x{\isacharparenright}{\kern0pt}\ {\isacharbraceright}{\kern0pt}\ {\isasymin}\ M{\isachardoublequoteclose}\ \isanewline
\ \ \ \ \isacommand{using}\isamarkupfalse%
\ {\isacartoucheopen}A{\isasymin}M{\isacartoucheclose}\ fclosed\ strong{\isacharunderscore}{\kern0pt}replacement{\isacharunderscore}{\kern0pt}closed\ \isacommand{by}\isamarkupfalse%
\ simp\isanewline
\ \ \isacommand{moreover}\isamarkupfalse%
\isanewline
\ \ \isacommand{have}\isamarkupfalse%
\ {\isachardoublequoteopen}{\isacharbraceleft}{\kern0pt}f{\isacharparenleft}{\kern0pt}x{\isacharparenright}{\kern0pt}{\isachardot}{\kern0pt}\ x{\isasymin}A{\isacharbraceright}{\kern0pt}\ {\isacharequal}{\kern0pt}\ {\isacharbraceleft}{\kern0pt}\ y\ {\isachardot}{\kern0pt}\ x{\isasymin}A\ {\isacharcomma}{\kern0pt}\ y\ {\isacharequal}{\kern0pt}\ f{\isacharparenleft}{\kern0pt}x{\isacharparenright}{\kern0pt}\ {\isacharbraceright}{\kern0pt}{\isachardoublequoteclose}\isanewline
\ \ \ \ \isacommand{by}\isamarkupfalse%
\ auto\isanewline
\ \ \isacommand{ultimately}\isamarkupfalse%
\ \isacommand{show}\isamarkupfalse%
\ {\isacharquery}{\kern0pt}thesis\ \isacommand{by}\isamarkupfalse%
\ simp\isanewline
\isacommand{qed}\isamarkupfalse%
%
\endisatagproof
{\isafoldproof}%
%
\isadelimproof
\isanewline
%
\endisadelimproof
\isanewline
\isacommand{end}\isamarkupfalse%
%
\isadelimdocument
%
\endisadelimdocument
%
\isatagdocument
%
\isamarkupsubsection{A forcing locale and generic filters%
}
\isamarkuptrue%
%
\endisatagdocument
{\isafolddocument}%
%
\isadelimdocument
%
\endisadelimdocument
\isacommand{locale}\isamarkupfalse%
\ forcing{\isacharunderscore}{\kern0pt}data\ {\isacharequal}{\kern0pt}\ forcing{\isacharunderscore}{\kern0pt}notion\ {\isacharplus}{\kern0pt}\ M{\isacharunderscore}{\kern0pt}ctm\ {\isacharplus}{\kern0pt}\isanewline
\ \ \isakeyword{assumes}\ P{\isacharunderscore}{\kern0pt}in{\isacharunderscore}{\kern0pt}M{\isacharcolon}{\kern0pt}\ \ \ \ \ \ \ \ \ \ \ {\isachardoublequoteopen}P\ {\isasymin}\ M{\isachardoublequoteclose}\isanewline
\ \ \ \ \isakeyword{and}\ leq{\isacharunderscore}{\kern0pt}in{\isacharunderscore}{\kern0pt}M{\isacharcolon}{\kern0pt}\ \ \ \ \ \ \ \ \ {\isachardoublequoteopen}leq\ {\isasymin}\ M{\isachardoublequoteclose}\isanewline
\isanewline
\isakeyword{begin}\isanewline
\isanewline
\isacommand{lemma}\isamarkupfalse%
\ transD\ {\isacharcolon}{\kern0pt}\ {\isachardoublequoteopen}Transset{\isacharparenleft}{\kern0pt}M{\isacharparenright}{\kern0pt}\ {\isasymLongrightarrow}\ y\ {\isasymin}\ M\ {\isasymLongrightarrow}\ y\ {\isasymsubseteq}\ M{\isachardoublequoteclose}\ \isanewline
%
\isadelimproof
\ \ %
\endisadelimproof
%
\isatagproof
\isacommand{by}\isamarkupfalse%
\ {\isacharparenleft}{\kern0pt}unfold\ Transset{\isacharunderscore}{\kern0pt}def{\isacharcomma}{\kern0pt}\ blast{\isacharparenright}{\kern0pt}%
\endisatagproof
{\isafoldproof}%
%
\isadelimproof
\ \isanewline
%
\endisadelimproof
\isanewline
\isanewline
\isacommand{lemmas}\isamarkupfalse%
\ P{\isacharunderscore}{\kern0pt}sub{\isacharunderscore}{\kern0pt}M\ {\isacharequal}{\kern0pt}\ transD{\isacharbrackleft}{\kern0pt}OF\ trans{\isacharunderscore}{\kern0pt}M\ P{\isacharunderscore}{\kern0pt}in{\isacharunderscore}{\kern0pt}M{\isacharbrackright}{\kern0pt}\isanewline
\isanewline
\isacommand{definition}\isamarkupfalse%
\isanewline
\ \ M{\isacharunderscore}{\kern0pt}generic\ {\isacharcolon}{\kern0pt}{\isacharcolon}{\kern0pt}\ {\isachardoublequoteopen}i{\isasymRightarrow}o{\isachardoublequoteclose}\ \isakeyword{where}\isanewline
\ \ {\isachardoublequoteopen}M{\isacharunderscore}{\kern0pt}generic{\isacharparenleft}{\kern0pt}G{\isacharparenright}{\kern0pt}\ {\isasymequiv}\ filter{\isacharparenleft}{\kern0pt}G{\isacharparenright}{\kern0pt}\ {\isasymand}\ {\isacharparenleft}{\kern0pt}{\isasymforall}D{\isasymin}M{\isachardot}{\kern0pt}\ D{\isasymsubseteq}P\ {\isasymand}\ dense{\isacharparenleft}{\kern0pt}D{\isacharparenright}{\kern0pt}{\isasymlongrightarrow}D{\isasyminter}G{\isasymnoteq}{\isadigit{0}}{\isacharparenright}{\kern0pt}{\isachardoublequoteclose}\isanewline
\isanewline
\isacommand{lemma}\isamarkupfalse%
\ M{\isacharunderscore}{\kern0pt}genericD\ {\isacharbrackleft}{\kern0pt}dest{\isacharbrackright}{\kern0pt}{\isacharcolon}{\kern0pt}\ {\isachardoublequoteopen}M{\isacharunderscore}{\kern0pt}generic{\isacharparenleft}{\kern0pt}G{\isacharparenright}{\kern0pt}\ {\isasymLongrightarrow}\ x{\isasymin}G\ {\isasymLongrightarrow}\ x{\isasymin}P{\isachardoublequoteclose}\isanewline
%
\isadelimproof
\ \ %
\endisadelimproof
%
\isatagproof
\isacommand{unfolding}\isamarkupfalse%
\ M{\isacharunderscore}{\kern0pt}generic{\isacharunderscore}{\kern0pt}def\ \isacommand{by}\isamarkupfalse%
\ {\isacharparenleft}{\kern0pt}blast\ dest{\isacharcolon}{\kern0pt}filterD{\isacharparenright}{\kern0pt}%
\endisatagproof
{\isafoldproof}%
%
\isadelimproof
\isanewline
%
\endisadelimproof
\isanewline
\isacommand{lemma}\isamarkupfalse%
\ M{\isacharunderscore}{\kern0pt}generic{\isacharunderscore}{\kern0pt}leqD\ {\isacharbrackleft}{\kern0pt}dest{\isacharbrackright}{\kern0pt}{\isacharcolon}{\kern0pt}\ {\isachardoublequoteopen}M{\isacharunderscore}{\kern0pt}generic{\isacharparenleft}{\kern0pt}G{\isacharparenright}{\kern0pt}\ {\isasymLongrightarrow}\ p{\isasymin}G\ {\isasymLongrightarrow}\ q{\isasymin}P\ {\isasymLongrightarrow}\ p{\isasympreceq}q\ {\isasymLongrightarrow}\ q{\isasymin}G{\isachardoublequoteclose}\isanewline
%
\isadelimproof
\ \ %
\endisadelimproof
%
\isatagproof
\isacommand{unfolding}\isamarkupfalse%
\ M{\isacharunderscore}{\kern0pt}generic{\isacharunderscore}{\kern0pt}def\ \isacommand{by}\isamarkupfalse%
\ {\isacharparenleft}{\kern0pt}blast\ dest{\isacharcolon}{\kern0pt}filter{\isacharunderscore}{\kern0pt}leqD{\isacharparenright}{\kern0pt}%
\endisatagproof
{\isafoldproof}%
%
\isadelimproof
\isanewline
%
\endisadelimproof
\isanewline
\isacommand{lemma}\isamarkupfalse%
\ M{\isacharunderscore}{\kern0pt}generic{\isacharunderscore}{\kern0pt}compatD\ {\isacharbrackleft}{\kern0pt}dest{\isacharbrackright}{\kern0pt}{\isacharcolon}{\kern0pt}\ {\isachardoublequoteopen}M{\isacharunderscore}{\kern0pt}generic{\isacharparenleft}{\kern0pt}G{\isacharparenright}{\kern0pt}\ {\isasymLongrightarrow}\ p{\isasymin}G\ {\isasymLongrightarrow}\ r{\isasymin}G\ {\isasymLongrightarrow}\ {\isasymexists}q{\isasymin}G{\isachardot}{\kern0pt}\ q{\isasympreceq}p\ {\isasymand}\ q{\isasympreceq}r{\isachardoublequoteclose}\isanewline
%
\isadelimproof
\ \ %
\endisadelimproof
%
\isatagproof
\isacommand{unfolding}\isamarkupfalse%
\ M{\isacharunderscore}{\kern0pt}generic{\isacharunderscore}{\kern0pt}def\ \isacommand{by}\isamarkupfalse%
\ {\isacharparenleft}{\kern0pt}blast\ dest{\isacharcolon}{\kern0pt}low{\isacharunderscore}{\kern0pt}bound{\isacharunderscore}{\kern0pt}filter{\isacharparenright}{\kern0pt}%
\endisatagproof
{\isafoldproof}%
%
\isadelimproof
\isanewline
%
\endisadelimproof
\isanewline
\isacommand{lemma}\isamarkupfalse%
\ M{\isacharunderscore}{\kern0pt}generic{\isacharunderscore}{\kern0pt}denseD\ {\isacharbrackleft}{\kern0pt}dest{\isacharbrackright}{\kern0pt}{\isacharcolon}{\kern0pt}\ {\isachardoublequoteopen}M{\isacharunderscore}{\kern0pt}generic{\isacharparenleft}{\kern0pt}G{\isacharparenright}{\kern0pt}\ {\isasymLongrightarrow}\ dense{\isacharparenleft}{\kern0pt}D{\isacharparenright}{\kern0pt}\ {\isasymLongrightarrow}\ D{\isasymsubseteq}P\ {\isasymLongrightarrow}\ D{\isasymin}M\ {\isasymLongrightarrow}\ {\isasymexists}q{\isasymin}G{\isachardot}{\kern0pt}\ q{\isasymin}D{\isachardoublequoteclose}\isanewline
%
\isadelimproof
\ \ %
\endisadelimproof
%
\isatagproof
\isacommand{unfolding}\isamarkupfalse%
\ M{\isacharunderscore}{\kern0pt}generic{\isacharunderscore}{\kern0pt}def\ \isacommand{by}\isamarkupfalse%
\ blast%
\endisatagproof
{\isafoldproof}%
%
\isadelimproof
\isanewline
%
\endisadelimproof
\isanewline
\isacommand{lemma}\isamarkupfalse%
\ G{\isacharunderscore}{\kern0pt}nonempty{\isacharcolon}{\kern0pt}\ {\isachardoublequoteopen}M{\isacharunderscore}{\kern0pt}generic{\isacharparenleft}{\kern0pt}G{\isacharparenright}{\kern0pt}\ {\isasymLongrightarrow}\ G{\isasymnoteq}{\isadigit{0}}{\isachardoublequoteclose}\isanewline
%
\isadelimproof
%
\endisadelimproof
%
\isatagproof
\isacommand{proof}\isamarkupfalse%
\ {\isacharminus}{\kern0pt}\isanewline
\ \ \isacommand{have}\isamarkupfalse%
\ {\isachardoublequoteopen}P{\isasymsubseteq}P{\isachardoublequoteclose}\ \isacommand{{\isachardot}{\kern0pt}{\isachardot}{\kern0pt}}\isamarkupfalse%
\isanewline
\ \ \isacommand{assume}\isamarkupfalse%
\isanewline
\ \ \ \ {\isachardoublequoteopen}M{\isacharunderscore}{\kern0pt}generic{\isacharparenleft}{\kern0pt}G{\isacharparenright}{\kern0pt}{\isachardoublequoteclose}\isanewline
\ \ \isacommand{with}\isamarkupfalse%
\ P{\isacharunderscore}{\kern0pt}in{\isacharunderscore}{\kern0pt}M\ P{\isacharunderscore}{\kern0pt}dense\ {\isacartoucheopen}P{\isasymsubseteq}P{\isacartoucheclose}\ \isacommand{show}\isamarkupfalse%
\isanewline
\ \ \ \ {\isachardoublequoteopen}G\ {\isasymnoteq}\ {\isadigit{0}}{\isachardoublequoteclose}\isanewline
\ \ \ \ \isacommand{unfolding}\isamarkupfalse%
\ M{\isacharunderscore}{\kern0pt}generic{\isacharunderscore}{\kern0pt}def\ \isacommand{by}\isamarkupfalse%
\ auto\isanewline
\isacommand{qed}\isamarkupfalse%
%
\endisatagproof
{\isafoldproof}%
%
\isadelimproof
\isanewline
%
\endisadelimproof
\isanewline
\isacommand{lemma}\isamarkupfalse%
\ one{\isacharunderscore}{\kern0pt}in{\isacharunderscore}{\kern0pt}G\ {\isacharcolon}{\kern0pt}\ \isanewline
\ \ \isakeyword{assumes}\ {\isachardoublequoteopen}M{\isacharunderscore}{\kern0pt}generic{\isacharparenleft}{\kern0pt}G{\isacharparenright}{\kern0pt}{\isachardoublequoteclose}\isanewline
\ \ \isakeyword{shows}\ \ {\isachardoublequoteopen}one\ {\isasymin}\ G{\isachardoublequoteclose}\ \isanewline
%
\isadelimproof
%
\endisadelimproof
%
\isatagproof
\isacommand{proof}\isamarkupfalse%
\ {\isacharminus}{\kern0pt}\isanewline
\ \ \isacommand{from}\isamarkupfalse%
\ assms\ \isacommand{have}\isamarkupfalse%
\ {\isachardoublequoteopen}G{\isasymsubseteq}P{\isachardoublequoteclose}\ \isanewline
\ \ \ \ \isacommand{unfolding}\isamarkupfalse%
\ M{\isacharunderscore}{\kern0pt}generic{\isacharunderscore}{\kern0pt}def\ \isakeyword{and}\ filter{\isacharunderscore}{\kern0pt}def\ \isacommand{by}\isamarkupfalse%
\ simp\isanewline
\ \ \isacommand{from}\isamarkupfalse%
\ {\isacartoucheopen}M{\isacharunderscore}{\kern0pt}generic{\isacharparenleft}{\kern0pt}G{\isacharparenright}{\kern0pt}{\isacartoucheclose}\ \isacommand{have}\isamarkupfalse%
\ {\isachardoublequoteopen}increasing{\isacharparenleft}{\kern0pt}G{\isacharparenright}{\kern0pt}{\isachardoublequoteclose}\ \isanewline
\ \ \ \ \isacommand{unfolding}\isamarkupfalse%
\ M{\isacharunderscore}{\kern0pt}generic{\isacharunderscore}{\kern0pt}def\ \isakeyword{and}\ filter{\isacharunderscore}{\kern0pt}def\ \isacommand{by}\isamarkupfalse%
\ simp\isanewline
\ \ \isacommand{with}\isamarkupfalse%
\ {\isacartoucheopen}G{\isasymsubseteq}P{\isacartoucheclose}\ \isakeyword{and}\ {\isacartoucheopen}M{\isacharunderscore}{\kern0pt}generic{\isacharparenleft}{\kern0pt}G{\isacharparenright}{\kern0pt}{\isacartoucheclose}\ \isanewline
\ \ \isacommand{show}\isamarkupfalse%
\ {\isacharquery}{\kern0pt}thesis\ \isanewline
\ \ \ \ \isacommand{using}\isamarkupfalse%
\ G{\isacharunderscore}{\kern0pt}nonempty\ \isakeyword{and}\ one{\isacharunderscore}{\kern0pt}in{\isacharunderscore}{\kern0pt}P\ \isakeyword{and}\ one{\isacharunderscore}{\kern0pt}max\ \isanewline
\ \ \ \ \isacommand{unfolding}\isamarkupfalse%
\ increasing{\isacharunderscore}{\kern0pt}def\ \isacommand{by}\isamarkupfalse%
\ blast\isanewline
\isacommand{qed}\isamarkupfalse%
%
\endisatagproof
{\isafoldproof}%
%
\isadelimproof
\isanewline
%
\endisadelimproof
\isanewline
\isacommand{lemma}\isamarkupfalse%
\ G{\isacharunderscore}{\kern0pt}subset{\isacharunderscore}{\kern0pt}M{\isacharcolon}{\kern0pt}\ {\isachardoublequoteopen}M{\isacharunderscore}{\kern0pt}generic{\isacharparenleft}{\kern0pt}G{\isacharparenright}{\kern0pt}\ {\isasymLongrightarrow}\ G\ {\isasymsubseteq}\ M{\isachardoublequoteclose}\isanewline
%
\isadelimproof
\ \ %
\endisadelimproof
%
\isatagproof
\isacommand{using}\isamarkupfalse%
\ transitivity{\isacharbrackleft}{\kern0pt}OF\ {\isacharunderscore}{\kern0pt}\ P{\isacharunderscore}{\kern0pt}in{\isacharunderscore}{\kern0pt}M{\isacharbrackright}{\kern0pt}\ \isacommand{by}\isamarkupfalse%
\ auto%
\endisatagproof
{\isafoldproof}%
%
\isadelimproof
\isanewline
%
\endisadelimproof
\isanewline
\isacommand{declare}\isamarkupfalse%
\ iff{\isacharunderscore}{\kern0pt}trans\ {\isacharbrackleft}{\kern0pt}trans{\isacharbrackright}{\kern0pt}\isanewline
\isanewline
\isacommand{lemma}\isamarkupfalse%
\ generic{\isacharunderscore}{\kern0pt}filter{\isacharunderscore}{\kern0pt}existence{\isacharcolon}{\kern0pt}\ \isanewline
\ \ {\isachardoublequoteopen}p{\isasymin}P\ {\isasymLongrightarrow}\ {\isasymexists}G{\isachardot}{\kern0pt}\ p{\isasymin}G\ {\isasymand}\ M{\isacharunderscore}{\kern0pt}generic{\isacharparenleft}{\kern0pt}G{\isacharparenright}{\kern0pt}{\isachardoublequoteclose}\isanewline
%
\isadelimproof
%
\endisadelimproof
%
\isatagproof
\isacommand{proof}\isamarkupfalse%
\ {\isacharminus}{\kern0pt}\isanewline
\ \ \isacommand{assume}\isamarkupfalse%
\ {\isachardoublequoteopen}p{\isasymin}P{\isachardoublequoteclose}\isanewline
\ \ \isacommand{let}\isamarkupfalse%
\ {\isacharquery}{\kern0pt}D{\isacharequal}{\kern0pt}{\isachardoublequoteopen}{\isasymlambda}n{\isasymin}nat{\isachardot}{\kern0pt}\ {\isacharparenleft}{\kern0pt}if\ {\isacharparenleft}{\kern0pt}enum{\isacharbackquote}{\kern0pt}n{\isasymsubseteq}P\ {\isasymand}\ dense{\isacharparenleft}{\kern0pt}enum{\isacharbackquote}{\kern0pt}n{\isacharparenright}{\kern0pt}{\isacharparenright}{\kern0pt}\ \ then\ enum{\isacharbackquote}{\kern0pt}n\ else\ P{\isacharparenright}{\kern0pt}{\isachardoublequoteclose}\isanewline
\ \ \isacommand{have}\isamarkupfalse%
\ {\isachardoublequoteopen}{\isasymforall}n{\isasymin}nat{\isachardot}{\kern0pt}\ {\isacharquery}{\kern0pt}D{\isacharbackquote}{\kern0pt}n\ {\isasymin}\ Pow{\isacharparenleft}{\kern0pt}P{\isacharparenright}{\kern0pt}{\isachardoublequoteclose}\isanewline
\ \ \ \ \isacommand{by}\isamarkupfalse%
\ auto\isanewline
\ \ \isacommand{then}\isamarkupfalse%
\ \isanewline
\ \ \isacommand{have}\isamarkupfalse%
\ {\isachardoublequoteopen}{\isacharquery}{\kern0pt}D{\isacharcolon}{\kern0pt}nat{\isasymrightarrow}Pow{\isacharparenleft}{\kern0pt}P{\isacharparenright}{\kern0pt}{\isachardoublequoteclose}\isanewline
\ \ \ \ \isacommand{using}\isamarkupfalse%
\ lam{\isacharunderscore}{\kern0pt}type\ \isacommand{by}\isamarkupfalse%
\ auto\isanewline
\ \ \isacommand{have}\isamarkupfalse%
\ Eq{\isadigit{4}}{\isacharcolon}{\kern0pt}\ {\isachardoublequoteopen}{\isasymforall}n{\isasymin}nat{\isachardot}{\kern0pt}\ dense{\isacharparenleft}{\kern0pt}{\isacharquery}{\kern0pt}D{\isacharbackquote}{\kern0pt}n{\isacharparenright}{\kern0pt}{\isachardoublequoteclose}\isanewline
\ \ \isacommand{proof}\isamarkupfalse%
{\isacharparenleft}{\kern0pt}intro\ ballI{\isacharparenright}{\kern0pt}\isanewline
\ \ \ \ \isacommand{fix}\isamarkupfalse%
\ n\isanewline
\ \ \ \ \isacommand{assume}\isamarkupfalse%
\ {\isachardoublequoteopen}n{\isasymin}nat{\isachardoublequoteclose}\isanewline
\ \ \ \ \isacommand{then}\isamarkupfalse%
\isanewline
\ \ \ \ \isacommand{have}\isamarkupfalse%
\ {\isachardoublequoteopen}dense{\isacharparenleft}{\kern0pt}{\isacharquery}{\kern0pt}D{\isacharbackquote}{\kern0pt}n{\isacharparenright}{\kern0pt}\ {\isasymlongleftrightarrow}\ dense{\isacharparenleft}{\kern0pt}if\ enum{\isacharbackquote}{\kern0pt}n\ {\isasymsubseteq}\ P\ {\isasymand}\ dense{\isacharparenleft}{\kern0pt}enum{\isacharbackquote}{\kern0pt}n{\isacharparenright}{\kern0pt}\ then\ enum{\isacharbackquote}{\kern0pt}n\ else\ P{\isacharparenright}{\kern0pt}{\isachardoublequoteclose}\isanewline
\ \ \ \ \ \ \isacommand{by}\isamarkupfalse%
\ simp\isanewline
\ \ \ \ \isacommand{also}\isamarkupfalse%
\ \isanewline
\ \ \ \ \isacommand{have}\isamarkupfalse%
\ {\isachardoublequoteopen}{\isachardot}{\kern0pt}{\isachardot}{\kern0pt}{\isachardot}{\kern0pt}\ {\isasymlongleftrightarrow}\ \ {\isacharparenleft}{\kern0pt}{\isasymnot}{\isacharparenleft}{\kern0pt}enum{\isacharbackquote}{\kern0pt}n\ {\isasymsubseteq}\ P\ {\isasymand}\ dense{\isacharparenleft}{\kern0pt}enum{\isacharbackquote}{\kern0pt}n{\isacharparenright}{\kern0pt}{\isacharparenright}{\kern0pt}\ {\isasymlongrightarrow}\ dense{\isacharparenleft}{\kern0pt}P{\isacharparenright}{\kern0pt}{\isacharparenright}{\kern0pt}\ {\isachardoublequoteclose}\isanewline
\ \ \ \ \ \ \isacommand{using}\isamarkupfalse%
\ split{\isacharunderscore}{\kern0pt}if\ \isacommand{by}\isamarkupfalse%
\ simp\isanewline
\ \ \ \ \isacommand{finally}\isamarkupfalse%
\isanewline
\ \ \ \ \isacommand{show}\isamarkupfalse%
\ {\isachardoublequoteopen}dense{\isacharparenleft}{\kern0pt}{\isacharquery}{\kern0pt}D{\isacharbackquote}{\kern0pt}n{\isacharparenright}{\kern0pt}{\isachardoublequoteclose}\isanewline
\ \ \ \ \ \ \isacommand{using}\isamarkupfalse%
\ P{\isacharunderscore}{\kern0pt}dense\ {\isacartoucheopen}n{\isasymin}nat{\isacartoucheclose}\ \isacommand{by}\isamarkupfalse%
\ auto\isanewline
\ \ \isacommand{qed}\isamarkupfalse%
\isanewline
\ \ \isacommand{from}\isamarkupfalse%
\ {\isacartoucheopen}{\isacharquery}{\kern0pt}D{\isasymin}{\isacharunderscore}{\kern0pt}{\isacartoucheclose}\ \isakeyword{and}\ Eq{\isadigit{4}}\ \isanewline
\ \ \isacommand{interpret}\isamarkupfalse%
\ cg{\isacharcolon}{\kern0pt}\ countable{\isacharunderscore}{\kern0pt}generic\ P\ leq\ one\ {\isacharquery}{\kern0pt}D\ \isanewline
\ \ \ \ \isacommand{by}\isamarkupfalse%
\ {\isacharparenleft}{\kern0pt}unfold{\isacharunderscore}{\kern0pt}locales{\isacharcomma}{\kern0pt}\ auto{\isacharparenright}{\kern0pt}\isanewline
\ \ \isacommand{from}\isamarkupfalse%
\ {\isacartoucheopen}p{\isasymin}P{\isacartoucheclose}\ \isanewline
\ \ \isacommand{obtain}\isamarkupfalse%
\ G\ \isakeyword{where}\ Eq{\isadigit{6}}{\isacharcolon}{\kern0pt}\ {\isachardoublequoteopen}p{\isasymin}G\ {\isasymand}\ filter{\isacharparenleft}{\kern0pt}G{\isacharparenright}{\kern0pt}\ {\isasymand}\ {\isacharparenleft}{\kern0pt}{\isasymforall}n{\isasymin}nat{\isachardot}{\kern0pt}{\isacharparenleft}{\kern0pt}{\isacharquery}{\kern0pt}D{\isacharbackquote}{\kern0pt}n{\isacharparenright}{\kern0pt}{\isasyminter}G{\isasymnoteq}{\isadigit{0}}{\isacharparenright}{\kern0pt}{\isachardoublequoteclose}\isanewline
\ \ \ \ \isacommand{using}\isamarkupfalse%
\ cg{\isachardot}{\kern0pt}countable{\isacharunderscore}{\kern0pt}rasiowa{\isacharunderscore}{\kern0pt}sikorski{\isacharbrackleft}{\kern0pt}\isakeyword{where}\ M{\isacharequal}{\kern0pt}{\isachardoublequoteopen}{\isasymlambda}{\isacharunderscore}{\kern0pt}{\isachardot}{\kern0pt}\ M{\isachardoublequoteclose}{\isacharbrackright}{\kern0pt}\ \ P{\isacharunderscore}{\kern0pt}sub{\isacharunderscore}{\kern0pt}M\isanewline
\ \ \ \ \ \ M{\isacharunderscore}{\kern0pt}countable{\isacharbrackleft}{\kern0pt}THEN\ bij{\isacharunderscore}{\kern0pt}is{\isacharunderscore}{\kern0pt}fun{\isacharbrackright}{\kern0pt}\ M{\isacharunderscore}{\kern0pt}countable{\isacharbrackleft}{\kern0pt}THEN\ bij{\isacharunderscore}{\kern0pt}is{\isacharunderscore}{\kern0pt}surj{\isacharcomma}{\kern0pt}\ THEN\ surj{\isacharunderscore}{\kern0pt}range{\isacharbrackright}{\kern0pt}\ \isanewline
\ \ \ \ \isacommand{unfolding}\isamarkupfalse%
\ cg{\isachardot}{\kern0pt}D{\isacharunderscore}{\kern0pt}generic{\isacharunderscore}{\kern0pt}def\ \isacommand{by}\isamarkupfalse%
\ blast\isanewline
\ \ \isacommand{then}\isamarkupfalse%
\ \isanewline
\ \ \isacommand{have}\isamarkupfalse%
\ Eq{\isadigit{7}}{\isacharcolon}{\kern0pt}\ {\isachardoublequoteopen}{\isacharparenleft}{\kern0pt}{\isasymforall}D{\isasymin}M{\isachardot}{\kern0pt}\ D{\isasymsubseteq}P\ {\isasymand}\ dense{\isacharparenleft}{\kern0pt}D{\isacharparenright}{\kern0pt}{\isasymlongrightarrow}D{\isasyminter}G{\isasymnoteq}{\isadigit{0}}{\isacharparenright}{\kern0pt}{\isachardoublequoteclose}\isanewline
\ \ \isacommand{proof}\isamarkupfalse%
\ {\isacharparenleft}{\kern0pt}intro\ ballI\ impI{\isacharparenright}{\kern0pt}\isanewline
\ \ \ \ \isacommand{fix}\isamarkupfalse%
\ D\isanewline
\ \ \ \ \isacommand{assume}\isamarkupfalse%
\ {\isachardoublequoteopen}D{\isasymin}M{\isachardoublequoteclose}\ \isakeyword{and}\ Eq{\isadigit{9}}{\isacharcolon}{\kern0pt}\ {\isachardoublequoteopen}D\ {\isasymsubseteq}\ P\ {\isasymand}\ dense{\isacharparenleft}{\kern0pt}D{\isacharparenright}{\kern0pt}\ {\isachardoublequoteclose}\ \isanewline
\ \ \ \ \isacommand{have}\isamarkupfalse%
\ {\isachardoublequoteopen}{\isasymforall}y{\isasymin}M{\isachardot}{\kern0pt}\ {\isasymexists}x{\isasymin}nat{\isachardot}{\kern0pt}\ enum{\isacharbackquote}{\kern0pt}x{\isacharequal}{\kern0pt}\ y{\isachardoublequoteclose}\isanewline
\ \ \ \ \ \ \isacommand{using}\isamarkupfalse%
\ M{\isacharunderscore}{\kern0pt}countable\ \isakeyword{and}\ \ bij{\isacharunderscore}{\kern0pt}is{\isacharunderscore}{\kern0pt}surj\ \isacommand{unfolding}\isamarkupfalse%
\ surj{\isacharunderscore}{\kern0pt}def\ \isacommand{by}\isamarkupfalse%
\ {\isacharparenleft}{\kern0pt}simp{\isacharparenright}{\kern0pt}\isanewline
\ \ \ \ \isacommand{with}\isamarkupfalse%
\ {\isacartoucheopen}D{\isasymin}M{\isacartoucheclose}\ \isacommand{obtain}\isamarkupfalse%
\ n\ \isakeyword{where}\ Eq{\isadigit{1}}{\isadigit{0}}{\isacharcolon}{\kern0pt}\ {\isachardoublequoteopen}n{\isasymin}nat\ {\isasymand}\ enum{\isacharbackquote}{\kern0pt}n\ {\isacharequal}{\kern0pt}\ D{\isachardoublequoteclose}\ \isanewline
\ \ \ \ \ \ \isacommand{by}\isamarkupfalse%
\ auto\isanewline
\ \ \ \ \isacommand{with}\isamarkupfalse%
\ Eq{\isadigit{9}}\ \isakeyword{and}\ if{\isacharunderscore}{\kern0pt}P\isanewline
\ \ \ \ \isacommand{have}\isamarkupfalse%
\ {\isachardoublequoteopen}{\isacharquery}{\kern0pt}D{\isacharbackquote}{\kern0pt}n\ {\isacharequal}{\kern0pt}\ D{\isachardoublequoteclose}\ \isacommand{by}\isamarkupfalse%
\ {\isacharparenleft}{\kern0pt}simp{\isacharparenright}{\kern0pt}\isanewline
\ \ \ \ \isacommand{with}\isamarkupfalse%
\ Eq{\isadigit{6}}\ \isakeyword{and}\ Eq{\isadigit{1}}{\isadigit{0}}\ \isanewline
\ \ \ \ \isacommand{show}\isamarkupfalse%
\ {\isachardoublequoteopen}D{\isasyminter}G{\isasymnoteq}{\isadigit{0}}{\isachardoublequoteclose}\ \isacommand{by}\isamarkupfalse%
\ auto\isanewline
\ \ \isacommand{qed}\isamarkupfalse%
\isanewline
\ \ \isacommand{with}\isamarkupfalse%
\ Eq{\isadigit{6}}\ \isanewline
\ \ \isacommand{show}\isamarkupfalse%
\ {\isacharquery}{\kern0pt}thesis\ \isacommand{unfolding}\isamarkupfalse%
\ M{\isacharunderscore}{\kern0pt}generic{\isacharunderscore}{\kern0pt}def\ \isacommand{by}\isamarkupfalse%
\ auto\isanewline
\isacommand{qed}\isamarkupfalse%
%
\endisatagproof
{\isafoldproof}%
%
\isadelimproof
\isanewline
%
\endisadelimproof
\isanewline
\isanewline
\isacommand{lemma}\isamarkupfalse%
\ compat{\isacharunderscore}{\kern0pt}in{\isacharunderscore}{\kern0pt}abs\ {\isacharcolon}{\kern0pt}\isanewline
\ \ \isakeyword{assumes}\isanewline
\ \ \ \ {\isachardoublequoteopen}A{\isasymin}M{\isachardoublequoteclose}\ {\isachardoublequoteopen}r{\isasymin}M{\isachardoublequoteclose}\ {\isachardoublequoteopen}p{\isasymin}M{\isachardoublequoteclose}\ {\isachardoublequoteopen}q{\isasymin}M{\isachardoublequoteclose}\ \isanewline
\ \ \isakeyword{shows}\isanewline
\ \ \ \ {\isachardoublequoteopen}is{\isacharunderscore}{\kern0pt}compat{\isacharunderscore}{\kern0pt}in{\isacharparenleft}{\kern0pt}{\isacharhash}{\kern0pt}{\isacharhash}{\kern0pt}M{\isacharcomma}{\kern0pt}A{\isacharcomma}{\kern0pt}r{\isacharcomma}{\kern0pt}p{\isacharcomma}{\kern0pt}q{\isacharparenright}{\kern0pt}\ {\isasymlongleftrightarrow}\ compat{\isacharunderscore}{\kern0pt}in{\isacharparenleft}{\kern0pt}A{\isacharcomma}{\kern0pt}r{\isacharcomma}{\kern0pt}p{\isacharcomma}{\kern0pt}q{\isacharparenright}{\kern0pt}{\isachardoublequoteclose}\ \isanewline
%
\isadelimproof
%
\endisadelimproof
%
\isatagproof
\isacommand{proof}\isamarkupfalse%
\ {\isacharminus}{\kern0pt}\isanewline
\ \ \isacommand{have}\isamarkupfalse%
\ {\isachardoublequoteopen}d{\isasymin}A\ {\isasymLongrightarrow}\ d{\isasymin}M{\isachardoublequoteclose}\ \isakeyword{for}\ d\isanewline
\ \ \ \ \isacommand{using}\isamarkupfalse%
\ transitivity\ {\isacartoucheopen}A{\isasymin}M{\isacartoucheclose}\ \isacommand{by}\isamarkupfalse%
\ simp\isanewline
\ \ \isacommand{moreover}\isamarkupfalse%
\ \isacommand{from}\isamarkupfalse%
\ this\isanewline
\ \ \isacommand{have}\isamarkupfalse%
\ {\isachardoublequoteopen}d{\isasymin}A\ {\isasymLongrightarrow}\ {\isasymlangle}d{\isacharcomma}{\kern0pt}\ t{\isasymrangle}\ {\isasymin}\ M{\isachardoublequoteclose}\ \isakeyword{if}\ {\isachardoublequoteopen}t{\isasymin}M{\isachardoublequoteclose}\ \isakeyword{for}\ t\ d\isanewline
\ \ \ \ \isacommand{using}\isamarkupfalse%
\ that\ pair{\isacharunderscore}{\kern0pt}in{\isacharunderscore}{\kern0pt}M{\isacharunderscore}{\kern0pt}iff\ \isacommand{by}\isamarkupfalse%
\ simp\isanewline
\ \ \isacommand{ultimately}\isamarkupfalse%
\ \isanewline
\ \ \isacommand{show}\isamarkupfalse%
\ {\isacharquery}{\kern0pt}thesis\isanewline
\ \ \ \ \isacommand{unfolding}\isamarkupfalse%
\ is{\isacharunderscore}{\kern0pt}compat{\isacharunderscore}{\kern0pt}in{\isacharunderscore}{\kern0pt}def\ compat{\isacharunderscore}{\kern0pt}in{\isacharunderscore}{\kern0pt}def\ \isanewline
\ \ \ \ \isacommand{using}\isamarkupfalse%
\ assms\ pair{\isacharunderscore}{\kern0pt}in{\isacharunderscore}{\kern0pt}M{\isacharunderscore}{\kern0pt}iff\ transitivity\ \isacommand{by}\isamarkupfalse%
\ auto\isanewline
\isacommand{qed}\isamarkupfalse%
%
\endisatagproof
{\isafoldproof}%
%
\isadelimproof
\isanewline
%
\endisadelimproof
\isanewline
\isacommand{definition}\isamarkupfalse%
\isanewline
\ \ compat{\isacharunderscore}{\kern0pt}in{\isacharunderscore}{\kern0pt}fm\ {\isacharcolon}{\kern0pt}{\isacharcolon}{\kern0pt}\ {\isachardoublequoteopen}{\isacharbrackleft}{\kern0pt}i{\isacharcomma}{\kern0pt}i{\isacharcomma}{\kern0pt}i{\isacharcomma}{\kern0pt}i{\isacharbrackright}{\kern0pt}\ {\isasymRightarrow}\ i{\isachardoublequoteclose}\ \isakeyword{where}\isanewline
\ \ {\isachardoublequoteopen}compat{\isacharunderscore}{\kern0pt}in{\isacharunderscore}{\kern0pt}fm{\isacharparenleft}{\kern0pt}A{\isacharcomma}{\kern0pt}r{\isacharcomma}{\kern0pt}p{\isacharcomma}{\kern0pt}q{\isacharparenright}{\kern0pt}\ {\isasymequiv}\ \isanewline
\ \ \ \ Exists{\isacharparenleft}{\kern0pt}And{\isacharparenleft}{\kern0pt}Member{\isacharparenleft}{\kern0pt}{\isadigit{0}}{\isacharcomma}{\kern0pt}succ{\isacharparenleft}{\kern0pt}A{\isacharparenright}{\kern0pt}{\isacharparenright}{\kern0pt}{\isacharcomma}{\kern0pt}Exists{\isacharparenleft}{\kern0pt}And{\isacharparenleft}{\kern0pt}pair{\isacharunderscore}{\kern0pt}fm{\isacharparenleft}{\kern0pt}{\isadigit{1}}{\isacharcomma}{\kern0pt}p{\isacharhash}{\kern0pt}{\isacharplus}{\kern0pt}{\isadigit{2}}{\isacharcomma}{\kern0pt}{\isadigit{0}}{\isacharparenright}{\kern0pt}{\isacharcomma}{\kern0pt}\isanewline
\ \ \ \ \ \ \ \ \ \ \ \ \ \ \ \ \ \ \ \ \ \ \ \ \ \ \ \ \ \ \ \ \ \ \ \ \ \ \ \ And{\isacharparenleft}{\kern0pt}Member{\isacharparenleft}{\kern0pt}{\isadigit{0}}{\isacharcomma}{\kern0pt}r{\isacharhash}{\kern0pt}{\isacharplus}{\kern0pt}{\isadigit{2}}{\isacharparenright}{\kern0pt}{\isacharcomma}{\kern0pt}\isanewline
\ \ \ \ \ \ \ \ \ \ \ \ \ \ \ \ \ \ \ \ \ \ \ \ \ \ \ \ \ \ \ \ \ Exists{\isacharparenleft}{\kern0pt}And{\isacharparenleft}{\kern0pt}pair{\isacharunderscore}{\kern0pt}fm{\isacharparenleft}{\kern0pt}{\isadigit{2}}{\isacharcomma}{\kern0pt}q{\isacharhash}{\kern0pt}{\isacharplus}{\kern0pt}{\isadigit{3}}{\isacharcomma}{\kern0pt}{\isadigit{0}}{\isacharparenright}{\kern0pt}{\isacharcomma}{\kern0pt}Member{\isacharparenleft}{\kern0pt}{\isadigit{0}}{\isacharcomma}{\kern0pt}r{\isacharhash}{\kern0pt}{\isacharplus}{\kern0pt}{\isadigit{3}}{\isacharparenright}{\kern0pt}{\isacharparenright}{\kern0pt}{\isacharparenright}{\kern0pt}{\isacharparenright}{\kern0pt}{\isacharparenright}{\kern0pt}{\isacharparenright}{\kern0pt}{\isacharparenright}{\kern0pt}{\isacharparenright}{\kern0pt}{\isachardoublequoteclose}\ \isanewline
\isanewline
\isacommand{lemma}\isamarkupfalse%
\ compat{\isacharunderscore}{\kern0pt}in{\isacharunderscore}{\kern0pt}fm{\isacharunderscore}{\kern0pt}type{\isacharbrackleft}{\kern0pt}TC{\isacharbrackright}{\kern0pt}\ {\isacharcolon}{\kern0pt}\ \isanewline
\ \ {\isachardoublequoteopen}{\isasymlbrakk}\ A{\isasymin}nat{\isacharsemicolon}{\kern0pt}r{\isasymin}nat{\isacharsemicolon}{\kern0pt}p{\isasymin}nat{\isacharsemicolon}{\kern0pt}q{\isasymin}nat{\isasymrbrakk}\ {\isasymLongrightarrow}\ compat{\isacharunderscore}{\kern0pt}in{\isacharunderscore}{\kern0pt}fm{\isacharparenleft}{\kern0pt}A{\isacharcomma}{\kern0pt}r{\isacharcomma}{\kern0pt}p{\isacharcomma}{\kern0pt}q{\isacharparenright}{\kern0pt}{\isasymin}formula{\isachardoublequoteclose}\ \isanewline
%
\isadelimproof
\ \ %
\endisadelimproof
%
\isatagproof
\isacommand{unfolding}\isamarkupfalse%
\ compat{\isacharunderscore}{\kern0pt}in{\isacharunderscore}{\kern0pt}fm{\isacharunderscore}{\kern0pt}def\ \isacommand{by}\isamarkupfalse%
\ simp%
\endisatagproof
{\isafoldproof}%
%
\isadelimproof
\isanewline
%
\endisadelimproof
\isanewline
\isacommand{lemma}\isamarkupfalse%
\ sats{\isacharunderscore}{\kern0pt}compat{\isacharunderscore}{\kern0pt}in{\isacharunderscore}{\kern0pt}fm{\isacharcolon}{\kern0pt}\isanewline
\ \ \isakeyword{assumes}\isanewline
\ \ \ \ {\isachardoublequoteopen}A{\isasymin}nat{\isachardoublequoteclose}\ {\isachardoublequoteopen}r{\isasymin}nat{\isachardoublequoteclose}\ \ {\isachardoublequoteopen}p{\isasymin}nat{\isachardoublequoteclose}\ {\isachardoublequoteopen}q{\isasymin}nat{\isachardoublequoteclose}\ {\isachardoublequoteopen}env{\isasymin}list{\isacharparenleft}{\kern0pt}M{\isacharparenright}{\kern0pt}{\isachardoublequoteclose}\ \ \isanewline
\ \ \isakeyword{shows}\isanewline
\ \ \ \ {\isachardoublequoteopen}sats{\isacharparenleft}{\kern0pt}M{\isacharcomma}{\kern0pt}compat{\isacharunderscore}{\kern0pt}in{\isacharunderscore}{\kern0pt}fm{\isacharparenleft}{\kern0pt}A{\isacharcomma}{\kern0pt}r{\isacharcomma}{\kern0pt}p{\isacharcomma}{\kern0pt}q{\isacharparenright}{\kern0pt}{\isacharcomma}{\kern0pt}env{\isacharparenright}{\kern0pt}\ {\isasymlongleftrightarrow}\ \isanewline
\ \ \ \ \ \ \ \ \ \ \ \ is{\isacharunderscore}{\kern0pt}compat{\isacharunderscore}{\kern0pt}in{\isacharparenleft}{\kern0pt}{\isacharhash}{\kern0pt}{\isacharhash}{\kern0pt}M{\isacharcomma}{\kern0pt}nth{\isacharparenleft}{\kern0pt}A{\isacharcomma}{\kern0pt}\ env{\isacharparenright}{\kern0pt}{\isacharcomma}{\kern0pt}nth{\isacharparenleft}{\kern0pt}r{\isacharcomma}{\kern0pt}\ env{\isacharparenright}{\kern0pt}{\isacharcomma}{\kern0pt}nth{\isacharparenleft}{\kern0pt}p{\isacharcomma}{\kern0pt}\ env{\isacharparenright}{\kern0pt}{\isacharcomma}{\kern0pt}nth{\isacharparenleft}{\kern0pt}q{\isacharcomma}{\kern0pt}\ env{\isacharparenright}{\kern0pt}{\isacharparenright}{\kern0pt}{\isachardoublequoteclose}\isanewline
%
\isadelimproof
\ \ %
\endisadelimproof
%
\isatagproof
\isacommand{unfolding}\isamarkupfalse%
\ compat{\isacharunderscore}{\kern0pt}in{\isacharunderscore}{\kern0pt}fm{\isacharunderscore}{\kern0pt}def\ is{\isacharunderscore}{\kern0pt}compat{\isacharunderscore}{\kern0pt}in{\isacharunderscore}{\kern0pt}def\ \isacommand{using}\isamarkupfalse%
\ assms\ \isacommand{by}\isamarkupfalse%
\ simp%
\endisatagproof
{\isafoldproof}%
%
\isadelimproof
\isanewline
%
\endisadelimproof
\isanewline
\isacommand{end}\isamarkupfalse%
\ \isanewline
%
\isadelimtheory
\isanewline
%
\endisadelimtheory
%
\isatagtheory
\isacommand{end}\isamarkupfalse%
%
\endisatagtheory
{\isafoldtheory}%
%
\isadelimtheory
%
\endisadelimtheory
%
\end{isabellebody}%
\endinput
%:%file=~/source/repos/ZF-notAC/code/Forcing/Forcing_Data.thy%:%
%:%11=1%:%
%:%23=2%:%
%:%24=3%:%
%:%25=4%:%
%:%33=5%:%
%:%34=5%:%
%:%35=6%:%
%:%36=7%:%
%:%37=8%:%
%:%38=9%:%
%:%39=10%:%
%:%46=10%:%
%:%47=11%:%
%:%48=12%:%
%:%49=12%:%
%:%50=13%:%
%:%53=14%:%
%:%57=14%:%
%:%58=14%:%
%:%63=14%:%
%:%66=15%:%
%:%67=16%:%
%:%68=17%:%
%:%69=17%:%
%:%70=18%:%
%:%71=19%:%
%:%72=20%:%
%:%73=21%:%
%:%74=22%:%
%:%75=23%:%
%:%76=24%:%
%:%77=25%:%
%:%78=26%:%
%:%79=27%:%
%:%80=28%:%
%:%81=29%:%
%:%82=30%:%
%:%83=31%:%
%:%84=31%:%
%:%85=32%:%
%:%86=33%:%
%:%87=34%:%
%:%88=35%:%
%:%89=36%:%
%:%90=37%:%
%:%91=37%:%
%:%94=38%:%
%:%98=38%:%
%:%99=38%:%
%:%100=39%:%
%:%101=40%:%
%:%102=41%:%
%:%103=42%:%
%:%104=42%:%
%:%109=42%:%
%:%112=43%:%
%:%113=44%:%
%:%114=45%:%
%:%115=45%:%
%:%116=46%:%
%:%117=47%:%
%:%118=47%:%
%:%121=48%:%
%:%125=48%:%
%:%126=48%:%
%:%131=48%:%
%:%134=49%:%
%:%135=50%:%
%:%136=50%:%
%:%139=51%:%
%:%143=51%:%
%:%144=51%:%
%:%149=51%:%
%:%152=52%:%
%:%153=53%:%
%:%154=53%:%
%:%157=54%:%
%:%161=54%:%
%:%162=54%:%
%:%167=54%:%
%:%170=55%:%
%:%171=56%:%
%:%172=56%:%
%:%175=57%:%
%:%179=57%:%
%:%180=57%:%
%:%181=57%:%
%:%186=57%:%
%:%189=58%:%
%:%190=59%:%
%:%191=59%:%
%:%194=60%:%
%:%198=60%:%
%:%199=60%:%
%:%204=60%:%
%:%207=61%:%
%:%208=62%:%
%:%209=62%:%
%:%212=63%:%
%:%216=63%:%
%:%217=63%:%
%:%222=63%:%
%:%225=64%:%
%:%226=65%:%
%:%227=65%:%
%:%230=66%:%
%:%234=66%:%
%:%235=66%:%
%:%240=66%:%
%:%243=67%:%
%:%244=68%:%
%:%245=68%:%
%:%248=69%:%
%:%252=69%:%
%:%253=69%:%
%:%258=69%:%
%:%261=70%:%
%:%262=71%:%
%:%263=71%:%
%:%266=72%:%
%:%270=72%:%
%:%271=72%:%
%:%276=72%:%
%:%279=73%:%
%:%280=74%:%
%:%281=74%:%
%:%284=75%:%
%:%288=75%:%
%:%289=75%:%
%:%294=75%:%
%:%297=76%:%
%:%298=77%:%
%:%299=77%:%
%:%302=78%:%
%:%306=78%:%
%:%307=78%:%
%:%312=78%:%
%:%315=79%:%
%:%316=80%:%
%:%317=81%:%
%:%318=82%:%
%:%319=82%:%
%:%320=83%:%
%:%321=84%:%
%:%322=85%:%
%:%323=85%:%
%:%326=86%:%
%:%330=86%:%
%:%331=86%:%
%:%336=86%:%
%:%339=87%:%
%:%340=88%:%
%:%341=88%:%
%:%344=89%:%
%:%348=89%:%
%:%349=89%:%
%:%354=89%:%
%:%357=90%:%
%:%358=91%:%
%:%359=91%:%
%:%362=92%:%
%:%366=92%:%
%:%367=92%:%
%:%372=92%:%
%:%375=93%:%
%:%376=94%:%
%:%377=94%:%
%:%380=95%:%
%:%384=95%:%
%:%385=95%:%
%:%390=95%:%
%:%393=96%:%
%:%394=97%:%
%:%395=97%:%
%:%398=98%:%
%:%402=98%:%
%:%403=98%:%
%:%408=98%:%
%:%411=99%:%
%:%412=100%:%
%:%413=100%:%
%:%416=101%:%
%:%420=101%:%
%:%421=101%:%
%:%426=101%:%
%:%429=102%:%
%:%430=103%:%
%:%431=103%:%
%:%434=104%:%
%:%438=104%:%
%:%439=104%:%
%:%444=104%:%
%:%447=105%:%
%:%448=106%:%
%:%449=107%:%
%:%450=108%:%
%:%451=108%:%
%:%452=109%:%
%:%459=111%:%
%:%469=112%:%
%:%470=112%:%
%:%471=113%:%
%:%472=114%:%
%:%473=115%:%
%:%474=116%:%
%:%475=117%:%
%:%476=118%:%
%:%477=119%:%
%:%478=120%:%
%:%485=121%:%
%:%486=121%:%
%:%487=122%:%
%:%488=122%:%
%:%489=123%:%
%:%490=123%:%
%:%491=123%:%
%:%492=124%:%
%:%493=124%:%
%:%494=125%:%
%:%495=125%:%
%:%496=126%:%
%:%497=126%:%
%:%498=126%:%
%:%499=127%:%
%:%500=127%:%
%:%501=128%:%
%:%502=128%:%
%:%503=129%:%
%:%504=130%:%
%:%505=130%:%
%:%506=131%:%
%:%507=131%:%
%:%508=132%:%
%:%509=132%:%
%:%510=133%:%
%:%511=133%:%
%:%512=133%:%
%:%513=134%:%
%:%514=134%:%
%:%515=135%:%
%:%516=135%:%
%:%517=135%:%
%:%518=135%:%
%:%519=136%:%
%:%525=136%:%
%:%528=137%:%
%:%529=138%:%
%:%530=138%:%
%:%531=139%:%
%:%532=140%:%
%:%533=141%:%
%:%534=142%:%
%:%535=143%:%
%:%536=144%:%
%:%537=145%:%
%:%538=146%:%
%:%539=147%:%
%:%546=148%:%
%:%547=148%:%
%:%548=149%:%
%:%549=149%:%
%:%550=150%:%
%:%551=150%:%
%:%552=150%:%
%:%553=151%:%
%:%554=151%:%
%:%555=152%:%
%:%556=152%:%
%:%557=153%:%
%:%558=153%:%
%:%559=153%:%
%:%560=154%:%
%:%561=154%:%
%:%562=155%:%
%:%563=155%:%
%:%564=156%:%
%:%565=157%:%
%:%566=157%:%
%:%567=158%:%
%:%568=158%:%
%:%569=159%:%
%:%570=159%:%
%:%571=160%:%
%:%572=160%:%
%:%573=160%:%
%:%574=161%:%
%:%575=161%:%
%:%576=162%:%
%:%577=162%:%
%:%578=162%:%
%:%579=162%:%
%:%580=163%:%
%:%586=163%:%
%:%589=164%:%
%:%590=165%:%
%:%591=165%:%
%:%592=166%:%
%:%593=167%:%
%:%594=168%:%
%:%595=169%:%
%:%596=170%:%
%:%597=171%:%
%:%598=172%:%
%:%599=173%:%
%:%600=174%:%
%:%607=175%:%
%:%608=175%:%
%:%609=176%:%
%:%610=176%:%
%:%611=177%:%
%:%612=177%:%
%:%613=177%:%
%:%614=178%:%
%:%615=178%:%
%:%616=179%:%
%:%617=179%:%
%:%618=180%:%
%:%619=180%:%
%:%620=181%:%
%:%621=181%:%
%:%622=182%:%
%:%623=182%:%
%:%624=183%:%
%:%625=183%:%
%:%626=184%:%
%:%627=185%:%
%:%628=186%:%
%:%629=186%:%
%:%630=187%:%
%:%631=187%:%
%:%632=188%:%
%:%633=188%:%
%:%634=189%:%
%:%635=189%:%
%:%636=189%:%
%:%637=190%:%
%:%638=190%:%
%:%639=191%:%
%:%640=191%:%
%:%641=191%:%
%:%642=191%:%
%:%643=192%:%
%:%649=192%:%
%:%652=193%:%
%:%653=194%:%
%:%654=194%:%
%:%655=195%:%
%:%656=196%:%
%:%657=197%:%
%:%658=198%:%
%:%659=199%:%
%:%660=200%:%
%:%661=201%:%
%:%662=202%:%
%:%669=203%:%
%:%670=203%:%
%:%671=204%:%
%:%672=204%:%
%:%673=205%:%
%:%674=205%:%
%:%675=205%:%
%:%676=206%:%
%:%677=206%:%
%:%678=207%:%
%:%679=207%:%
%:%680=208%:%
%:%681=208%:%
%:%682=209%:%
%:%683=210%:%
%:%684=210%:%
%:%685=211%:%
%:%686=211%:%
%:%687=212%:%
%:%688=212%:%
%:%689=213%:%
%:%690=213%:%
%:%691=213%:%
%:%692=214%:%
%:%693=214%:%
%:%694=215%:%
%:%695=215%:%
%:%696=216%:%
%:%697=216%:%
%:%698=217%:%
%:%699=217%:%
%:%700=217%:%
%:%701=217%:%
%:%702=218%:%
%:%708=218%:%
%:%711=219%:%
%:%712=220%:%
%:%720=222%:%
%:%730=223%:%
%:%731=223%:%
%:%732=224%:%
%:%733=225%:%
%:%734=226%:%
%:%735=227%:%
%:%736=228%:%
%:%737=229%:%
%:%738=229%:%
%:%741=230%:%
%:%745=230%:%
%:%746=230%:%
%:%751=230%:%
%:%754=231%:%
%:%755=232%:%
%:%756=233%:%
%:%757=233%:%
%:%758=234%:%
%:%759=235%:%
%:%760=235%:%
%:%761=236%:%
%:%762=237%:%
%:%763=238%:%
%:%764=239%:%
%:%765=239%:%
%:%768=240%:%
%:%772=240%:%
%:%773=240%:%
%:%774=240%:%
%:%779=240%:%
%:%782=241%:%
%:%783=242%:%
%:%784=242%:%
%:%787=243%:%
%:%791=243%:%
%:%792=243%:%
%:%793=243%:%
%:%798=243%:%
%:%801=244%:%
%:%802=245%:%
%:%803=245%:%
%:%806=246%:%
%:%810=246%:%
%:%811=246%:%
%:%812=246%:%
%:%817=246%:%
%:%820=247%:%
%:%821=248%:%
%:%822=248%:%
%:%825=249%:%
%:%829=249%:%
%:%830=249%:%
%:%831=249%:%
%:%836=249%:%
%:%839=250%:%
%:%840=251%:%
%:%841=251%:%
%:%848=252%:%
%:%849=252%:%
%:%850=253%:%
%:%851=253%:%
%:%852=253%:%
%:%853=254%:%
%:%854=254%:%
%:%855=255%:%
%:%856=256%:%
%:%857=256%:%
%:%858=256%:%
%:%859=257%:%
%:%860=258%:%
%:%861=258%:%
%:%862=258%:%
%:%863=259%:%
%:%869=259%:%
%:%872=260%:%
%:%873=261%:%
%:%874=261%:%
%:%875=262%:%
%:%876=263%:%
%:%883=264%:%
%:%884=264%:%
%:%885=265%:%
%:%886=265%:%
%:%887=265%:%
%:%888=266%:%
%:%889=266%:%
%:%890=266%:%
%:%891=267%:%
%:%892=267%:%
%:%893=267%:%
%:%894=268%:%
%:%895=268%:%
%:%896=268%:%
%:%897=269%:%
%:%898=269%:%
%:%899=270%:%
%:%900=270%:%
%:%901=271%:%
%:%902=271%:%
%:%903=272%:%
%:%904=272%:%
%:%905=272%:%
%:%906=273%:%
%:%912=273%:%
%:%915=274%:%
%:%916=275%:%
%:%917=275%:%
%:%920=276%:%
%:%924=276%:%
%:%925=276%:%
%:%926=276%:%
%:%931=276%:%
%:%934=277%:%
%:%935=278%:%
%:%936=278%:%
%:%937=279%:%
%:%938=280%:%
%:%939=280%:%
%:%940=281%:%
%:%947=282%:%
%:%948=282%:%
%:%949=283%:%
%:%950=283%:%
%:%951=284%:%
%:%952=284%:%
%:%953=285%:%
%:%954=285%:%
%:%955=286%:%
%:%956=286%:%
%:%957=287%:%
%:%958=287%:%
%:%959=288%:%
%:%960=288%:%
%:%961=289%:%
%:%962=289%:%
%:%963=289%:%
%:%964=290%:%
%:%965=290%:%
%:%966=291%:%
%:%967=291%:%
%:%968=292%:%
%:%969=292%:%
%:%970=293%:%
%:%971=293%:%
%:%972=294%:%
%:%973=294%:%
%:%974=295%:%
%:%975=295%:%
%:%976=296%:%
%:%977=296%:%
%:%978=297%:%
%:%979=297%:%
%:%980=298%:%
%:%981=298%:%
%:%982=299%:%
%:%983=299%:%
%:%984=299%:%
%:%985=300%:%
%:%986=300%:%
%:%987=301%:%
%:%988=301%:%
%:%989=302%:%
%:%990=302%:%
%:%991=302%:%
%:%992=303%:%
%:%993=303%:%
%:%994=304%:%
%:%995=304%:%
%:%996=305%:%
%:%997=305%:%
%:%998=306%:%
%:%999=306%:%
%:%1000=307%:%
%:%1001=307%:%
%:%1002=308%:%
%:%1003=308%:%
%:%1004=309%:%
%:%1005=309%:%
%:%1006=310%:%
%:%1007=311%:%
%:%1008=311%:%
%:%1009=311%:%
%:%1010=312%:%
%:%1011=312%:%
%:%1012=313%:%
%:%1013=313%:%
%:%1014=314%:%
%:%1015=314%:%
%:%1016=315%:%
%:%1017=315%:%
%:%1018=316%:%
%:%1019=316%:%
%:%1020=317%:%
%:%1021=317%:%
%:%1022=318%:%
%:%1023=318%:%
%:%1024=318%:%
%:%1025=318%:%
%:%1026=319%:%
%:%1027=319%:%
%:%1028=319%:%
%:%1029=320%:%
%:%1030=320%:%
%:%1031=321%:%
%:%1032=321%:%
%:%1033=322%:%
%:%1034=322%:%
%:%1035=322%:%
%:%1036=323%:%
%:%1037=323%:%
%:%1038=324%:%
%:%1039=324%:%
%:%1040=324%:%
%:%1041=325%:%
%:%1042=325%:%
%:%1043=326%:%
%:%1044=326%:%
%:%1045=327%:%
%:%1046=327%:%
%:%1047=327%:%
%:%1048=327%:%
%:%1049=328%:%
%:%1055=328%:%
%:%1058=329%:%
%:%1059=330%:%
%:%1060=331%:%
%:%1061=331%:%
%:%1062=332%:%
%:%1063=333%:%
%:%1064=334%:%
%:%1065=335%:%
%:%1072=336%:%
%:%1073=336%:%
%:%1074=337%:%
%:%1075=337%:%
%:%1076=338%:%
%:%1077=338%:%
%:%1078=338%:%
%:%1079=339%:%
%:%1080=339%:%
%:%1081=339%:%
%:%1082=340%:%
%:%1083=340%:%
%:%1084=341%:%
%:%1085=341%:%
%:%1086=341%:%
%:%1087=342%:%
%:%1088=342%:%
%:%1089=343%:%
%:%1090=343%:%
%:%1091=344%:%
%:%1092=344%:%
%:%1093=345%:%
%:%1094=345%:%
%:%1095=345%:%
%:%1096=346%:%
%:%1102=346%:%
%:%1105=347%:%
%:%1106=348%:%
%:%1107=348%:%
%:%1108=349%:%
%:%1109=350%:%
%:%1112=353%:%
%:%1113=354%:%
%:%1114=355%:%
%:%1115=355%:%
%:%1116=356%:%
%:%1119=357%:%
%:%1123=357%:%
%:%1124=357%:%
%:%1125=357%:%
%:%1130=357%:%
%:%1133=358%:%
%:%1134=359%:%
%:%1135=359%:%
%:%1136=360%:%
%:%1137=361%:%
%:%1138=362%:%
%:%1139=363%:%
%:%1140=364%:%
%:%1143=365%:%
%:%1147=365%:%
%:%1148=365%:%
%:%1149=365%:%
%:%1150=365%:%
%:%1155=365%:%
%:%1158=366%:%
%:%1159=367%:%
%:%1160=367%:%
%:%1163=368%:%
%:%1168=369%:%

%
\begin{isabellebody}%
\setisabellecontext{Internal{\isacharunderscore}{\kern0pt}ZFC{\isacharunderscore}{\kern0pt}Axioms}%
%
\isadelimdocument
%
\endisadelimdocument
%
\isatagdocument
%
\isamarkupsection{The ZFC axioms, internalized%
}
\isamarkuptrue%
%
\endisatagdocument
{\isafolddocument}%
%
\isadelimdocument
%
\endisadelimdocument
%
\isadelimtheory
%
\endisadelimtheory
%
\isatagtheory
\isacommand{theory}\isamarkupfalse%
\ Internal{\isacharunderscore}{\kern0pt}ZFC{\isacharunderscore}{\kern0pt}Axioms\isanewline
\ \ \isakeyword{imports}\ \isanewline
\ \ Forcing{\isacharunderscore}{\kern0pt}Data\isanewline
\isanewline
\isakeyword{begin}%
\endisatagtheory
{\isafoldtheory}%
%
\isadelimtheory
\isanewline
%
\endisadelimtheory
\isanewline
\isacommand{schematic{\isacharunderscore}{\kern0pt}goal}\isamarkupfalse%
\ ZF{\isacharunderscore}{\kern0pt}union{\isacharunderscore}{\kern0pt}auto{\isacharcolon}{\kern0pt}\isanewline
\ \ \ \ {\isachardoublequoteopen}Union{\isacharunderscore}{\kern0pt}ax{\isacharparenleft}{\kern0pt}{\isacharhash}{\kern0pt}{\isacharhash}{\kern0pt}A{\isacharparenright}{\kern0pt}\ {\isasymlongleftrightarrow}\ {\isacharparenleft}{\kern0pt}A{\isacharcomma}{\kern0pt}\ {\isacharbrackleft}{\kern0pt}{\isacharbrackright}{\kern0pt}\ {\isasymTurnstile}\ {\isacharquery}{\kern0pt}zfunion{\isacharparenright}{\kern0pt}{\isachardoublequoteclose}\isanewline
%
\isadelimproof
\ \ %
\endisadelimproof
%
\isatagproof
\isacommand{unfolding}\isamarkupfalse%
\ Union{\isacharunderscore}{\kern0pt}ax{\isacharunderscore}{\kern0pt}def\ \isanewline
\ \ \isacommand{by}\isamarkupfalse%
\ {\isacharparenleft}{\kern0pt}{\isacharparenleft}{\kern0pt}rule\ sep{\isacharunderscore}{\kern0pt}rules\ {\isacharbar}{\kern0pt}\ simp{\isacharparenright}{\kern0pt}{\isacharplus}{\kern0pt}{\isacharparenright}{\kern0pt}%
\endisatagproof
{\isafoldproof}%
%
\isadelimproof
\isanewline
%
\endisadelimproof
%
\isadelimML
\isanewline
%
\endisadelimML
%
\isatagML
\isacommand{synthesize}\isamarkupfalse%
\ {\isachardoublequoteopen}ZF{\isacharunderscore}{\kern0pt}union{\isacharunderscore}{\kern0pt}fm{\isachardoublequoteclose}\ \isakeyword{from{\isacharunderscore}{\kern0pt}schematic}\ ZF{\isacharunderscore}{\kern0pt}union{\isacharunderscore}{\kern0pt}auto%
\endisatagML
{\isafoldML}%
%
\isadelimML
\isanewline
%
\endisadelimML
\isanewline
\isacommand{schematic{\isacharunderscore}{\kern0pt}goal}\isamarkupfalse%
\ ZF{\isacharunderscore}{\kern0pt}power{\isacharunderscore}{\kern0pt}auto{\isacharcolon}{\kern0pt}\isanewline
\ \ \ \ {\isachardoublequoteopen}power{\isacharunderscore}{\kern0pt}ax{\isacharparenleft}{\kern0pt}{\isacharhash}{\kern0pt}{\isacharhash}{\kern0pt}A{\isacharparenright}{\kern0pt}\ {\isasymlongleftrightarrow}\ {\isacharparenleft}{\kern0pt}A{\isacharcomma}{\kern0pt}\ {\isacharbrackleft}{\kern0pt}{\isacharbrackright}{\kern0pt}\ {\isasymTurnstile}\ {\isacharquery}{\kern0pt}zfpow{\isacharparenright}{\kern0pt}{\isachardoublequoteclose}\isanewline
%
\isadelimproof
\ \ %
\endisadelimproof
%
\isatagproof
\isacommand{unfolding}\isamarkupfalse%
\ power{\isacharunderscore}{\kern0pt}ax{\isacharunderscore}{\kern0pt}def\ powerset{\isacharunderscore}{\kern0pt}def\ subset{\isacharunderscore}{\kern0pt}def\isanewline
\ \ \isacommand{by}\isamarkupfalse%
\ {\isacharparenleft}{\kern0pt}{\isacharparenleft}{\kern0pt}rule\ sep{\isacharunderscore}{\kern0pt}rules\ {\isacharbar}{\kern0pt}\ simp{\isacharparenright}{\kern0pt}{\isacharplus}{\kern0pt}{\isacharparenright}{\kern0pt}%
\endisatagproof
{\isafoldproof}%
%
\isadelimproof
\isanewline
%
\endisadelimproof
%
\isadelimML
\isanewline
%
\endisadelimML
%
\isatagML
\isacommand{synthesize}\isamarkupfalse%
\ {\isachardoublequoteopen}ZF{\isacharunderscore}{\kern0pt}power{\isacharunderscore}{\kern0pt}fm{\isachardoublequoteclose}\ \isakeyword{from{\isacharunderscore}{\kern0pt}schematic}\ ZF{\isacharunderscore}{\kern0pt}power{\isacharunderscore}{\kern0pt}auto%
\endisatagML
{\isafoldML}%
%
\isadelimML
\isanewline
%
\endisadelimML
\isanewline
\isacommand{schematic{\isacharunderscore}{\kern0pt}goal}\isamarkupfalse%
\ ZF{\isacharunderscore}{\kern0pt}pairing{\isacharunderscore}{\kern0pt}auto{\isacharcolon}{\kern0pt}\isanewline
\ \ \ \ {\isachardoublequoteopen}upair{\isacharunderscore}{\kern0pt}ax{\isacharparenleft}{\kern0pt}{\isacharhash}{\kern0pt}{\isacharhash}{\kern0pt}A{\isacharparenright}{\kern0pt}\ {\isasymlongleftrightarrow}\ {\isacharparenleft}{\kern0pt}A{\isacharcomma}{\kern0pt}\ {\isacharbrackleft}{\kern0pt}{\isacharbrackright}{\kern0pt}\ {\isasymTurnstile}\ {\isacharquery}{\kern0pt}zfpair{\isacharparenright}{\kern0pt}{\isachardoublequoteclose}\isanewline
%
\isadelimproof
\ \ %
\endisadelimproof
%
\isatagproof
\isacommand{unfolding}\isamarkupfalse%
\ upair{\isacharunderscore}{\kern0pt}ax{\isacharunderscore}{\kern0pt}def\ \isanewline
\ \ \isacommand{by}\isamarkupfalse%
\ {\isacharparenleft}{\kern0pt}{\isacharparenleft}{\kern0pt}rule\ sep{\isacharunderscore}{\kern0pt}rules\ {\isacharbar}{\kern0pt}\ simp{\isacharparenright}{\kern0pt}{\isacharplus}{\kern0pt}{\isacharparenright}{\kern0pt}%
\endisatagproof
{\isafoldproof}%
%
\isadelimproof
\isanewline
%
\endisadelimproof
%
\isadelimML
\isanewline
%
\endisadelimML
%
\isatagML
\isacommand{synthesize}\isamarkupfalse%
\ {\isachardoublequoteopen}ZF{\isacharunderscore}{\kern0pt}pairing{\isacharunderscore}{\kern0pt}fm{\isachardoublequoteclose}\ \isakeyword{from{\isacharunderscore}{\kern0pt}schematic}\ ZF{\isacharunderscore}{\kern0pt}pairing{\isacharunderscore}{\kern0pt}auto%
\endisatagML
{\isafoldML}%
%
\isadelimML
\isanewline
%
\endisadelimML
\isanewline
\isacommand{schematic{\isacharunderscore}{\kern0pt}goal}\isamarkupfalse%
\ ZF{\isacharunderscore}{\kern0pt}foundation{\isacharunderscore}{\kern0pt}auto{\isacharcolon}{\kern0pt}\isanewline
\ \ \ \ {\isachardoublequoteopen}foundation{\isacharunderscore}{\kern0pt}ax{\isacharparenleft}{\kern0pt}{\isacharhash}{\kern0pt}{\isacharhash}{\kern0pt}A{\isacharparenright}{\kern0pt}\ {\isasymlongleftrightarrow}\ {\isacharparenleft}{\kern0pt}A{\isacharcomma}{\kern0pt}\ {\isacharbrackleft}{\kern0pt}{\isacharbrackright}{\kern0pt}\ {\isasymTurnstile}\ {\isacharquery}{\kern0pt}zfpow{\isacharparenright}{\kern0pt}{\isachardoublequoteclose}\isanewline
%
\isadelimproof
\ \ %
\endisadelimproof
%
\isatagproof
\isacommand{unfolding}\isamarkupfalse%
\ foundation{\isacharunderscore}{\kern0pt}ax{\isacharunderscore}{\kern0pt}def\ \isanewline
\ \ \isacommand{by}\isamarkupfalse%
\ {\isacharparenleft}{\kern0pt}{\isacharparenleft}{\kern0pt}rule\ sep{\isacharunderscore}{\kern0pt}rules\ {\isacharbar}{\kern0pt}\ simp{\isacharparenright}{\kern0pt}{\isacharplus}{\kern0pt}{\isacharparenright}{\kern0pt}%
\endisatagproof
{\isafoldproof}%
%
\isadelimproof
\isanewline
%
\endisadelimproof
%
\isadelimML
\isanewline
%
\endisadelimML
%
\isatagML
\isacommand{synthesize}\isamarkupfalse%
\ {\isachardoublequoteopen}ZF{\isacharunderscore}{\kern0pt}foundation{\isacharunderscore}{\kern0pt}fm{\isachardoublequoteclose}\ \isakeyword{from{\isacharunderscore}{\kern0pt}schematic}\ ZF{\isacharunderscore}{\kern0pt}foundation{\isacharunderscore}{\kern0pt}auto%
\endisatagML
{\isafoldML}%
%
\isadelimML
\isanewline
%
\endisadelimML
\isanewline
\isacommand{schematic{\isacharunderscore}{\kern0pt}goal}\isamarkupfalse%
\ ZF{\isacharunderscore}{\kern0pt}extensionality{\isacharunderscore}{\kern0pt}auto{\isacharcolon}{\kern0pt}\isanewline
\ \ \ \ {\isachardoublequoteopen}extensionality{\isacharparenleft}{\kern0pt}{\isacharhash}{\kern0pt}{\isacharhash}{\kern0pt}A{\isacharparenright}{\kern0pt}\ {\isasymlongleftrightarrow}\ {\isacharparenleft}{\kern0pt}A{\isacharcomma}{\kern0pt}\ {\isacharbrackleft}{\kern0pt}{\isacharbrackright}{\kern0pt}\ {\isasymTurnstile}\ {\isacharquery}{\kern0pt}zfpow{\isacharparenright}{\kern0pt}{\isachardoublequoteclose}\isanewline
%
\isadelimproof
\ \ %
\endisadelimproof
%
\isatagproof
\isacommand{unfolding}\isamarkupfalse%
\ extensionality{\isacharunderscore}{\kern0pt}def\ \isanewline
\ \ \isacommand{by}\isamarkupfalse%
\ {\isacharparenleft}{\kern0pt}{\isacharparenleft}{\kern0pt}rule\ sep{\isacharunderscore}{\kern0pt}rules\ {\isacharbar}{\kern0pt}\ simp{\isacharparenright}{\kern0pt}{\isacharplus}{\kern0pt}{\isacharparenright}{\kern0pt}%
\endisatagproof
{\isafoldproof}%
%
\isadelimproof
\isanewline
%
\endisadelimproof
%
\isadelimML
\isanewline
%
\endisadelimML
%
\isatagML
\isacommand{synthesize}\isamarkupfalse%
\ {\isachardoublequoteopen}ZF{\isacharunderscore}{\kern0pt}extensionality{\isacharunderscore}{\kern0pt}fm{\isachardoublequoteclose}\ \isakeyword{from{\isacharunderscore}{\kern0pt}schematic}\ ZF{\isacharunderscore}{\kern0pt}extensionality{\isacharunderscore}{\kern0pt}auto%
\endisatagML
{\isafoldML}%
%
\isadelimML
\isanewline
%
\endisadelimML
\isanewline
\isacommand{schematic{\isacharunderscore}{\kern0pt}goal}\isamarkupfalse%
\ ZF{\isacharunderscore}{\kern0pt}infinity{\isacharunderscore}{\kern0pt}auto{\isacharcolon}{\kern0pt}\isanewline
\ \ \ \ {\isachardoublequoteopen}infinity{\isacharunderscore}{\kern0pt}ax{\isacharparenleft}{\kern0pt}{\isacharhash}{\kern0pt}{\isacharhash}{\kern0pt}A{\isacharparenright}{\kern0pt}\ {\isasymlongleftrightarrow}\ {\isacharparenleft}{\kern0pt}A{\isacharcomma}{\kern0pt}\ {\isacharbrackleft}{\kern0pt}{\isacharbrackright}{\kern0pt}\ {\isasymTurnstile}\ {\isacharparenleft}{\kern0pt}{\isacharquery}{\kern0pt}{\isasymphi}{\isacharparenleft}{\kern0pt}i{\isacharcomma}{\kern0pt}j{\isacharcomma}{\kern0pt}h{\isacharparenright}{\kern0pt}{\isacharparenright}{\kern0pt}{\isacharparenright}{\kern0pt}{\isachardoublequoteclose}\isanewline
%
\isadelimproof
\ \ %
\endisadelimproof
%
\isatagproof
\isacommand{unfolding}\isamarkupfalse%
\ infinity{\isacharunderscore}{\kern0pt}ax{\isacharunderscore}{\kern0pt}def\ \isanewline
\ \ \isacommand{by}\isamarkupfalse%
\ {\isacharparenleft}{\kern0pt}{\isacharparenleft}{\kern0pt}rule\ sep{\isacharunderscore}{\kern0pt}rules\ {\isacharbar}{\kern0pt}\ simp{\isacharparenright}{\kern0pt}{\isacharplus}{\kern0pt}{\isacharparenright}{\kern0pt}%
\endisatagproof
{\isafoldproof}%
%
\isadelimproof
\isanewline
%
\endisadelimproof
%
\isadelimML
\isanewline
%
\endisadelimML
%
\isatagML
\isacommand{synthesize}\isamarkupfalse%
\ {\isachardoublequoteopen}ZF{\isacharunderscore}{\kern0pt}infinity{\isacharunderscore}{\kern0pt}fm{\isachardoublequoteclose}\ \isakeyword{from{\isacharunderscore}{\kern0pt}schematic}\ ZF{\isacharunderscore}{\kern0pt}infinity{\isacharunderscore}{\kern0pt}auto%
\endisatagML
{\isafoldML}%
%
\isadelimML
\isanewline
%
\endisadelimML
\isanewline
\isacommand{schematic{\isacharunderscore}{\kern0pt}goal}\isamarkupfalse%
\ ZF{\isacharunderscore}{\kern0pt}choice{\isacharunderscore}{\kern0pt}auto{\isacharcolon}{\kern0pt}\isanewline
\ \ \ \ {\isachardoublequoteopen}choice{\isacharunderscore}{\kern0pt}ax{\isacharparenleft}{\kern0pt}{\isacharhash}{\kern0pt}{\isacharhash}{\kern0pt}A{\isacharparenright}{\kern0pt}\ {\isasymlongleftrightarrow}\ {\isacharparenleft}{\kern0pt}A{\isacharcomma}{\kern0pt}\ {\isacharbrackleft}{\kern0pt}{\isacharbrackright}{\kern0pt}\ {\isasymTurnstile}\ {\isacharparenleft}{\kern0pt}{\isacharquery}{\kern0pt}{\isasymphi}{\isacharparenleft}{\kern0pt}i{\isacharcomma}{\kern0pt}j{\isacharcomma}{\kern0pt}h{\isacharparenright}{\kern0pt}{\isacharparenright}{\kern0pt}{\isacharparenright}{\kern0pt}{\isachardoublequoteclose}\isanewline
%
\isadelimproof
\ \ %
\endisadelimproof
%
\isatagproof
\isacommand{unfolding}\isamarkupfalse%
\ choice{\isacharunderscore}{\kern0pt}ax{\isacharunderscore}{\kern0pt}def\ \isanewline
\ \ \isacommand{by}\isamarkupfalse%
\ {\isacharparenleft}{\kern0pt}{\isacharparenleft}{\kern0pt}rule\ sep{\isacharunderscore}{\kern0pt}rules\ {\isacharbar}{\kern0pt}\ simp{\isacharparenright}{\kern0pt}{\isacharplus}{\kern0pt}{\isacharparenright}{\kern0pt}%
\endisatagproof
{\isafoldproof}%
%
\isadelimproof
\isanewline
%
\endisadelimproof
%
\isadelimML
\isanewline
%
\endisadelimML
%
\isatagML
\isacommand{synthesize}\isamarkupfalse%
\ {\isachardoublequoteopen}ZF{\isacharunderscore}{\kern0pt}choice{\isacharunderscore}{\kern0pt}fm{\isachardoublequoteclose}\ \isakeyword{from{\isacharunderscore}{\kern0pt}schematic}\ ZF{\isacharunderscore}{\kern0pt}choice{\isacharunderscore}{\kern0pt}auto%
\endisatagML
{\isafoldML}%
%
\isadelimML
\isanewline
%
\endisadelimML
\isanewline
\isacommand{syntax}\isamarkupfalse%
\isanewline
\ \ {\isachardoublequoteopen}{\isacharunderscore}{\kern0pt}choice{\isachardoublequoteclose}\ {\isacharcolon}{\kern0pt}{\isacharcolon}{\kern0pt}\ {\isachardoublequoteopen}i{\isachardoublequoteclose}\ \ {\isacharparenleft}{\kern0pt}{\isachardoublequoteopen}AC{\isachardoublequoteclose}{\isacharparenright}{\kern0pt}\isanewline
\isacommand{translations}\isamarkupfalse%
\isanewline
\ \ {\isachardoublequoteopen}AC{\isachardoublequoteclose}\ {\isasymrightharpoonup}\ {\isachardoublequoteopen}CONST\ ZF{\isacharunderscore}{\kern0pt}choice{\isacharunderscore}{\kern0pt}fm{\isachardoublequoteclose}\isanewline
\isanewline
\isacommand{lemmas}\isamarkupfalse%
\ ZFC{\isacharunderscore}{\kern0pt}fm{\isacharunderscore}{\kern0pt}defs\ {\isacharequal}{\kern0pt}\ ZF{\isacharunderscore}{\kern0pt}extensionality{\isacharunderscore}{\kern0pt}fm{\isacharunderscore}{\kern0pt}def\ ZF{\isacharunderscore}{\kern0pt}foundation{\isacharunderscore}{\kern0pt}fm{\isacharunderscore}{\kern0pt}def\ ZF{\isacharunderscore}{\kern0pt}pairing{\isacharunderscore}{\kern0pt}fm{\isacharunderscore}{\kern0pt}def\isanewline
\ \ \ \ \ \ \ \ \ \ \ \ \ \ ZF{\isacharunderscore}{\kern0pt}union{\isacharunderscore}{\kern0pt}fm{\isacharunderscore}{\kern0pt}def\ ZF{\isacharunderscore}{\kern0pt}infinity{\isacharunderscore}{\kern0pt}fm{\isacharunderscore}{\kern0pt}def\ ZF{\isacharunderscore}{\kern0pt}power{\isacharunderscore}{\kern0pt}fm{\isacharunderscore}{\kern0pt}def\ ZF{\isacharunderscore}{\kern0pt}choice{\isacharunderscore}{\kern0pt}fm{\isacharunderscore}{\kern0pt}def\isanewline
\isanewline
\isacommand{lemmas}\isamarkupfalse%
\ ZFC{\isacharunderscore}{\kern0pt}fm{\isacharunderscore}{\kern0pt}sats\ {\isacharequal}{\kern0pt}\ ZF{\isacharunderscore}{\kern0pt}extensionality{\isacharunderscore}{\kern0pt}auto\ ZF{\isacharunderscore}{\kern0pt}foundation{\isacharunderscore}{\kern0pt}auto\ ZF{\isacharunderscore}{\kern0pt}pairing{\isacharunderscore}{\kern0pt}auto\isanewline
\ \ \ \ \ \ \ \ \ \ \ \ \ \ ZF{\isacharunderscore}{\kern0pt}union{\isacharunderscore}{\kern0pt}auto\ ZF{\isacharunderscore}{\kern0pt}infinity{\isacharunderscore}{\kern0pt}auto\ ZF{\isacharunderscore}{\kern0pt}power{\isacharunderscore}{\kern0pt}auto\ ZF{\isacharunderscore}{\kern0pt}choice{\isacharunderscore}{\kern0pt}auto\isanewline
\isanewline
\isacommand{definition}\isamarkupfalse%
\isanewline
\ \ ZF{\isacharunderscore}{\kern0pt}fin\ {\isacharcolon}{\kern0pt}{\isacharcolon}{\kern0pt}\ {\isachardoublequoteopen}i{\isachardoublequoteclose}\ \isakeyword{where}\isanewline
\ \ {\isachardoublequoteopen}ZF{\isacharunderscore}{\kern0pt}fin\ {\isasymequiv}\ {\isacharbraceleft}{\kern0pt}\ ZF{\isacharunderscore}{\kern0pt}extensionality{\isacharunderscore}{\kern0pt}fm{\isacharcomma}{\kern0pt}\ ZF{\isacharunderscore}{\kern0pt}foundation{\isacharunderscore}{\kern0pt}fm{\isacharcomma}{\kern0pt}\ ZF{\isacharunderscore}{\kern0pt}pairing{\isacharunderscore}{\kern0pt}fm{\isacharcomma}{\kern0pt}\isanewline
\ \ \ \ \ \ \ \ \ \ \ \ \ \ ZF{\isacharunderscore}{\kern0pt}union{\isacharunderscore}{\kern0pt}fm{\isacharcomma}{\kern0pt}\ ZF{\isacharunderscore}{\kern0pt}infinity{\isacharunderscore}{\kern0pt}fm{\isacharcomma}{\kern0pt}\ ZF{\isacharunderscore}{\kern0pt}power{\isacharunderscore}{\kern0pt}fm\ {\isacharbraceright}{\kern0pt}{\isachardoublequoteclose}\isanewline
\isanewline
\isacommand{definition}\isamarkupfalse%
\isanewline
\ \ ZFC{\isacharunderscore}{\kern0pt}fin\ {\isacharcolon}{\kern0pt}{\isacharcolon}{\kern0pt}\ {\isachardoublequoteopen}i{\isachardoublequoteclose}\ \isakeyword{where}\isanewline
\ \ {\isachardoublequoteopen}ZFC{\isacharunderscore}{\kern0pt}fin\ {\isasymequiv}\ ZF{\isacharunderscore}{\kern0pt}fin\ {\isasymunion}\ {\isacharbraceleft}{\kern0pt}ZF{\isacharunderscore}{\kern0pt}choice{\isacharunderscore}{\kern0pt}fm{\isacharbraceright}{\kern0pt}{\isachardoublequoteclose}\isanewline
\isanewline
\isacommand{lemma}\isamarkupfalse%
\ ZFC{\isacharunderscore}{\kern0pt}fin{\isacharunderscore}{\kern0pt}type\ {\isacharcolon}{\kern0pt}\ {\isachardoublequoteopen}ZFC{\isacharunderscore}{\kern0pt}fin\ {\isasymsubseteq}\ formula{\isachardoublequoteclose}\isanewline
%
\isadelimproof
\ \ %
\endisadelimproof
%
\isatagproof
\isacommand{unfolding}\isamarkupfalse%
\ ZFC{\isacharunderscore}{\kern0pt}fin{\isacharunderscore}{\kern0pt}def\ ZF{\isacharunderscore}{\kern0pt}fin{\isacharunderscore}{\kern0pt}def\ ZFC{\isacharunderscore}{\kern0pt}fm{\isacharunderscore}{\kern0pt}defs\ \isacommand{by}\isamarkupfalse%
\ {\isacharparenleft}{\kern0pt}auto{\isacharparenright}{\kern0pt}%
\endisatagproof
{\isafoldproof}%
%
\isadelimproof
%
\endisadelimproof
%
\isadelimdocument
%
\endisadelimdocument
%
\isatagdocument
%
\isamarkupsubsection{The Axiom of Separation, internalized%
}
\isamarkuptrue%
%
\endisatagdocument
{\isafolddocument}%
%
\isadelimdocument
%
\endisadelimdocument
\isacommand{lemma}\isamarkupfalse%
\ iterates{\isacharunderscore}{\kern0pt}Forall{\isacharunderscore}{\kern0pt}type\ {\isacharbrackleft}{\kern0pt}TC{\isacharbrackright}{\kern0pt}{\isacharcolon}{\kern0pt}\isanewline
\ \ \ \ \ \ {\isachardoublequoteopen}{\isasymlbrakk}\ n\ {\isasymin}\ nat{\isacharsemicolon}{\kern0pt}\ p\ {\isasymin}\ formula\ {\isasymrbrakk}\ {\isasymLongrightarrow}\ Forall{\isacharcircum}{\kern0pt}n{\isacharparenleft}{\kern0pt}p{\isacharparenright}{\kern0pt}\ {\isasymin}\ formula{\isachardoublequoteclose}\isanewline
%
\isadelimproof
\ \ %
\endisadelimproof
%
\isatagproof
\isacommand{by}\isamarkupfalse%
\ {\isacharparenleft}{\kern0pt}induct\ set{\isacharcolon}{\kern0pt}nat{\isacharcomma}{\kern0pt}\ auto{\isacharparenright}{\kern0pt}%
\endisatagproof
{\isafoldproof}%
%
\isadelimproof
\isanewline
%
\endisadelimproof
\isanewline
\isacommand{lemma}\isamarkupfalse%
\ last{\isacharunderscore}{\kern0pt}init{\isacharunderscore}{\kern0pt}eq\ {\isacharcolon}{\kern0pt}\isanewline
\ \ \isakeyword{assumes}\ {\isachardoublequoteopen}l\ {\isasymin}\ list{\isacharparenleft}{\kern0pt}A{\isacharparenright}{\kern0pt}{\isachardoublequoteclose}\ {\isachardoublequoteopen}length{\isacharparenleft}{\kern0pt}l{\isacharparenright}{\kern0pt}\ {\isacharequal}{\kern0pt}\ succ{\isacharparenleft}{\kern0pt}n{\isacharparenright}{\kern0pt}{\isachardoublequoteclose}\isanewline
\ \ \isakeyword{shows}\ {\isachardoublequoteopen}{\isasymexists}\ a{\isasymin}A{\isachardot}{\kern0pt}\ {\isasymexists}l{\isacharprime}{\kern0pt}{\isasymin}list{\isacharparenleft}{\kern0pt}A{\isacharparenright}{\kern0pt}{\isachardot}{\kern0pt}\ l\ {\isacharequal}{\kern0pt}\ l{\isacharprime}{\kern0pt}{\isacharat}{\kern0pt}{\isacharbrackleft}{\kern0pt}a{\isacharbrackright}{\kern0pt}{\isachardoublequoteclose}\isanewline
%
\isadelimproof
%
\endisadelimproof
%
\isatagproof
\isacommand{proof}\isamarkupfalse%
{\isacharminus}{\kern0pt}\isanewline
\ \ \isacommand{from}\isamarkupfalse%
\ {\isacartoucheopen}l{\isasymin}{\isacharunderscore}{\kern0pt}{\isacartoucheclose}\ {\isacartoucheopen}length{\isacharparenleft}{\kern0pt}{\isacharunderscore}{\kern0pt}{\isacharparenright}{\kern0pt}\ {\isacharequal}{\kern0pt}\ {\isacharunderscore}{\kern0pt}{\isacartoucheclose}\isanewline
\ \ \isacommand{have}\isamarkupfalse%
\ {\isachardoublequoteopen}rev{\isacharparenleft}{\kern0pt}l{\isacharparenright}{\kern0pt}\ {\isasymin}\ list{\isacharparenleft}{\kern0pt}A{\isacharparenright}{\kern0pt}{\isachardoublequoteclose}\ {\isachardoublequoteopen}length{\isacharparenleft}{\kern0pt}rev{\isacharparenleft}{\kern0pt}l{\isacharparenright}{\kern0pt}{\isacharparenright}{\kern0pt}\ {\isacharequal}{\kern0pt}\ succ{\isacharparenleft}{\kern0pt}n{\isacharparenright}{\kern0pt}{\isachardoublequoteclose}\isanewline
\ \ \ \ \isacommand{by}\isamarkupfalse%
\ simp{\isacharunderscore}{\kern0pt}all\isanewline
\ \ \isacommand{then}\isamarkupfalse%
\isanewline
\ \ \isacommand{obtain}\isamarkupfalse%
\ a\ l{\isacharprime}{\kern0pt}\ \isakeyword{where}\ {\isachardoublequoteopen}a{\isasymin}A{\isachardoublequoteclose}\ {\isachardoublequoteopen}l{\isacharprime}{\kern0pt}{\isasymin}list{\isacharparenleft}{\kern0pt}A{\isacharparenright}{\kern0pt}{\isachardoublequoteclose}\ {\isachardoublequoteopen}rev{\isacharparenleft}{\kern0pt}l{\isacharparenright}{\kern0pt}\ {\isacharequal}{\kern0pt}\ Cons{\isacharparenleft}{\kern0pt}a{\isacharcomma}{\kern0pt}l{\isacharprime}{\kern0pt}{\isacharparenright}{\kern0pt}{\isachardoublequoteclose}\isanewline
\ \ \ \ \isacommand{by}\isamarkupfalse%
\ {\isacharparenleft}{\kern0pt}cases{\isacharsemicolon}{\kern0pt}simp{\isacharparenright}{\kern0pt}\isanewline
\ \ \isacommand{then}\isamarkupfalse%
\isanewline
\ \ \isacommand{have}\isamarkupfalse%
\ {\isachardoublequoteopen}l\ {\isacharequal}{\kern0pt}\ rev{\isacharparenleft}{\kern0pt}l{\isacharprime}{\kern0pt}{\isacharparenright}{\kern0pt}\ {\isacharat}{\kern0pt}\ {\isacharbrackleft}{\kern0pt}a{\isacharbrackright}{\kern0pt}{\isachardoublequoteclose}\ {\isachardoublequoteopen}rev{\isacharparenleft}{\kern0pt}l{\isacharprime}{\kern0pt}{\isacharparenright}{\kern0pt}\ {\isasymin}\ list{\isacharparenleft}{\kern0pt}A{\isacharparenright}{\kern0pt}{\isachardoublequoteclose}\isanewline
\ \ \ \ \isacommand{using}\isamarkupfalse%
\ rev{\isacharunderscore}{\kern0pt}rev{\isacharunderscore}{\kern0pt}ident{\isacharbrackleft}{\kern0pt}OF\ {\isacartoucheopen}l{\isasymin}{\isacharunderscore}{\kern0pt}{\isacartoucheclose}{\isacharbrackright}{\kern0pt}\ \isacommand{by}\isamarkupfalse%
\ auto\isanewline
\ \ \isacommand{with}\isamarkupfalse%
\ {\isacartoucheopen}a{\isasymin}{\isacharunderscore}{\kern0pt}{\isacartoucheclose}\isanewline
\ \ \isacommand{show}\isamarkupfalse%
\ {\isacharquery}{\kern0pt}thesis\ \isacommand{by}\isamarkupfalse%
\ blast\isanewline
\isacommand{qed}\isamarkupfalse%
%
\endisatagproof
{\isafoldproof}%
%
\isadelimproof
\isanewline
%
\endisadelimproof
\isanewline
\isacommand{lemma}\isamarkupfalse%
\ take{\isacharunderscore}{\kern0pt}drop{\isacharunderscore}{\kern0pt}eq\ {\isacharcolon}{\kern0pt}\isanewline
\ \ \isakeyword{assumes}\ {\isachardoublequoteopen}l{\isasymin}list{\isacharparenleft}{\kern0pt}M{\isacharparenright}{\kern0pt}{\isachardoublequoteclose}\isanewline
\ \ \isakeyword{shows}\ {\isachardoublequoteopen}{\isasymAnd}\ n\ {\isachardot}{\kern0pt}\ n\ {\isacharless}{\kern0pt}\ succ{\isacharparenleft}{\kern0pt}length{\isacharparenleft}{\kern0pt}l{\isacharparenright}{\kern0pt}{\isacharparenright}{\kern0pt}\ {\isasymLongrightarrow}\ l\ {\isacharequal}{\kern0pt}\ take{\isacharparenleft}{\kern0pt}n{\isacharcomma}{\kern0pt}l{\isacharparenright}{\kern0pt}\ {\isacharat}{\kern0pt}\ drop{\isacharparenleft}{\kern0pt}n{\isacharcomma}{\kern0pt}l{\isacharparenright}{\kern0pt}{\isachardoublequoteclose}\isanewline
%
\isadelimproof
\ \ %
\endisadelimproof
%
\isatagproof
\isacommand{using}\isamarkupfalse%
\ {\isacartoucheopen}l{\isasymin}list{\isacharparenleft}{\kern0pt}M{\isacharparenright}{\kern0pt}{\isacartoucheclose}\isanewline
\isacommand{proof}\isamarkupfalse%
\ induct\isanewline
\ \ \isacommand{case}\isamarkupfalse%
\ Nil\isanewline
\ \ \isacommand{then}\isamarkupfalse%
\ \isacommand{show}\isamarkupfalse%
\ {\isacharquery}{\kern0pt}case\ \isacommand{by}\isamarkupfalse%
\ auto\isanewline
\isacommand{next}\isamarkupfalse%
\isanewline
\ \ \isacommand{case}\isamarkupfalse%
\ {\isacharparenleft}{\kern0pt}Cons\ a\ l{\isacharparenright}{\kern0pt}\isanewline
\ \ \isacommand{then}\isamarkupfalse%
\ \isacommand{show}\isamarkupfalse%
\ {\isacharquery}{\kern0pt}case\isanewline
\ \ \isacommand{proof}\isamarkupfalse%
\ {\isacharminus}{\kern0pt}\isanewline
\ \ \ \ \isacommand{{\isacharbraceleft}{\kern0pt}}\isamarkupfalse%
\isanewline
\ \ \ \ \ \ \isacommand{fix}\isamarkupfalse%
\ i\isanewline
\ \ \ \ \ \ \isacommand{assume}\isamarkupfalse%
\ {\isachardoublequoteopen}i{\isacharless}{\kern0pt}succ{\isacharparenleft}{\kern0pt}succ{\isacharparenleft}{\kern0pt}length{\isacharparenleft}{\kern0pt}l{\isacharparenright}{\kern0pt}{\isacharparenright}{\kern0pt}{\isacharparenright}{\kern0pt}{\isachardoublequoteclose}\isanewline
\ \ \ \ \ \ \isacommand{with}\isamarkupfalse%
\ {\isacartoucheopen}l{\isasymin}list{\isacharparenleft}{\kern0pt}M{\isacharparenright}{\kern0pt}{\isacartoucheclose}\isanewline
\ \ \ \ \ \ \isacommand{consider}\isamarkupfalse%
\ {\isacharparenleft}{\kern0pt}lt{\isacharparenright}{\kern0pt}\ {\isachardoublequoteopen}i\ {\isacharequal}{\kern0pt}\ {\isadigit{0}}{\isachardoublequoteclose}\ {\isacharbar}{\kern0pt}\ {\isacharparenleft}{\kern0pt}eq{\isacharparenright}{\kern0pt}\ {\isachardoublequoteopen}{\isasymexists}k{\isasymin}nat{\isachardot}{\kern0pt}\ i\ {\isacharequal}{\kern0pt}\ succ{\isacharparenleft}{\kern0pt}k{\isacharparenright}{\kern0pt}\ {\isasymand}\ k\ {\isacharless}{\kern0pt}\ succ{\isacharparenleft}{\kern0pt}length{\isacharparenleft}{\kern0pt}l{\isacharparenright}{\kern0pt}{\isacharparenright}{\kern0pt}{\isachardoublequoteclose}\isanewline
\ \ \ \ \ \ \ \ \isacommand{using}\isamarkupfalse%
\ {\isacartoucheopen}l{\isasymin}list{\isacharparenleft}{\kern0pt}M{\isacharparenright}{\kern0pt}{\isacartoucheclose}\ \ le{\isacharunderscore}{\kern0pt}natI\ nat{\isacharunderscore}{\kern0pt}imp{\isacharunderscore}{\kern0pt}quasinat\isanewline
\ \ \ \ \ \ \ \ \isacommand{by}\isamarkupfalse%
\ {\isacharparenleft}{\kern0pt}cases\ rule{\isacharcolon}{\kern0pt}nat{\isacharunderscore}{\kern0pt}cases{\isacharbrackleft}{\kern0pt}of\ i{\isacharbrackright}{\kern0pt}{\isacharsemicolon}{\kern0pt}auto{\isacharparenright}{\kern0pt}\isanewline
\ \ \ \ \ \ \isacommand{then}\isamarkupfalse%
\isanewline
\ \ \ \ \ \ \isacommand{have}\isamarkupfalse%
\ {\isachardoublequoteopen}take{\isacharparenleft}{\kern0pt}i{\isacharcomma}{\kern0pt}Cons{\isacharparenleft}{\kern0pt}a{\isacharcomma}{\kern0pt}l{\isacharparenright}{\kern0pt}{\isacharparenright}{\kern0pt}\ {\isacharat}{\kern0pt}\ drop{\isacharparenleft}{\kern0pt}i{\isacharcomma}{\kern0pt}Cons{\isacharparenleft}{\kern0pt}a{\isacharcomma}{\kern0pt}l{\isacharparenright}{\kern0pt}{\isacharparenright}{\kern0pt}\ {\isacharequal}{\kern0pt}\ Cons{\isacharparenleft}{\kern0pt}a{\isacharcomma}{\kern0pt}l{\isacharparenright}{\kern0pt}{\isachardoublequoteclose}\isanewline
\ \ \ \ \ \ \ \ \isacommand{using}\isamarkupfalse%
\ Cons\isanewline
\ \ \ \ \ \ \ \ \isacommand{by}\isamarkupfalse%
\ {\isacharparenleft}{\kern0pt}cases{\isacharsemicolon}{\kern0pt}auto{\isacharparenright}{\kern0pt}\isanewline
\ \ \ \ \isacommand{{\isacharbraceright}{\kern0pt}}\isamarkupfalse%
\isanewline
\ \ \ \ \isacommand{then}\isamarkupfalse%
\ \isacommand{show}\isamarkupfalse%
\ {\isacharquery}{\kern0pt}thesis\ \isacommand{using}\isamarkupfalse%
\ Cons\ \isacommand{by}\isamarkupfalse%
\ auto\isanewline
\ \ \isacommand{qed}\isamarkupfalse%
\isanewline
\isacommand{qed}\isamarkupfalse%
%
\endisatagproof
{\isafoldproof}%
%
\isadelimproof
\isanewline
%
\endisadelimproof
\isanewline
\isacommand{lemma}\isamarkupfalse%
\ list{\isacharunderscore}{\kern0pt}split\ {\isacharcolon}{\kern0pt}\isanewline
\isakeyword{assumes}\ {\isachardoublequoteopen}n\ {\isasymle}\ succ{\isacharparenleft}{\kern0pt}length{\isacharparenleft}{\kern0pt}rest{\isacharparenright}{\kern0pt}{\isacharparenright}{\kern0pt}{\isachardoublequoteclose}\ {\isachardoublequoteopen}rest\ {\isasymin}\ list{\isacharparenleft}{\kern0pt}M{\isacharparenright}{\kern0pt}{\isachardoublequoteclose}\isanewline
\isakeyword{shows}\ \ {\isachardoublequoteopen}{\isasymexists}re{\isasymin}list{\isacharparenleft}{\kern0pt}M{\isacharparenright}{\kern0pt}{\isachardot}{\kern0pt}\ {\isasymexists}st{\isasymin}list{\isacharparenleft}{\kern0pt}M{\isacharparenright}{\kern0pt}{\isachardot}{\kern0pt}\ rest\ {\isacharequal}{\kern0pt}\ re\ {\isacharat}{\kern0pt}\ st\ {\isasymand}\ length{\isacharparenleft}{\kern0pt}re{\isacharparenright}{\kern0pt}\ {\isacharequal}{\kern0pt}\ pred{\isacharparenleft}{\kern0pt}n{\isacharparenright}{\kern0pt}{\isachardoublequoteclose}\isanewline
%
\isadelimproof
%
\endisadelimproof
%
\isatagproof
\isacommand{proof}\isamarkupfalse%
\ {\isacharminus}{\kern0pt}\isanewline
\ \ \isacommand{from}\isamarkupfalse%
\ assms\isanewline
\ \ \isacommand{have}\isamarkupfalse%
\ {\isachardoublequoteopen}pred{\isacharparenleft}{\kern0pt}n{\isacharparenright}{\kern0pt}\ {\isasymle}\ length{\isacharparenleft}{\kern0pt}rest{\isacharparenright}{\kern0pt}{\isachardoublequoteclose}\isanewline
\ \ \ \ \isacommand{using}\isamarkupfalse%
\ pred{\isacharunderscore}{\kern0pt}mono{\isacharbrackleft}{\kern0pt}OF\ {\isacharunderscore}{\kern0pt}\ {\isacartoucheopen}n{\isasymle}{\isacharunderscore}{\kern0pt}{\isacartoucheclose}{\isacharbrackright}{\kern0pt}\ pred{\isacharunderscore}{\kern0pt}succ{\isacharunderscore}{\kern0pt}eq\ \isacommand{by}\isamarkupfalse%
\ auto\isanewline
\ \ \isacommand{with}\isamarkupfalse%
\ {\isacartoucheopen}rest{\isasymin}{\isacharunderscore}{\kern0pt}{\isacartoucheclose}\isanewline
\ \ \isacommand{have}\isamarkupfalse%
\ {\isachardoublequoteopen}pred{\isacharparenleft}{\kern0pt}n{\isacharparenright}{\kern0pt}{\isasymin}nat{\isachardoublequoteclose}\ {\isachardoublequoteopen}rest\ {\isacharequal}{\kern0pt}\ take{\isacharparenleft}{\kern0pt}pred{\isacharparenleft}{\kern0pt}n{\isacharparenright}{\kern0pt}{\isacharcomma}{\kern0pt}rest{\isacharparenright}{\kern0pt}\ {\isacharat}{\kern0pt}\ drop{\isacharparenleft}{\kern0pt}pred{\isacharparenleft}{\kern0pt}n{\isacharparenright}{\kern0pt}{\isacharcomma}{\kern0pt}rest{\isacharparenright}{\kern0pt}{\isachardoublequoteclose}\ {\isacharparenleft}{\kern0pt}\isakeyword{is}\ {\isachardoublequoteopen}{\isacharunderscore}{\kern0pt}\ {\isacharequal}{\kern0pt}\ {\isacharquery}{\kern0pt}re\ {\isacharat}{\kern0pt}\ {\isacharquery}{\kern0pt}st{\isachardoublequoteclose}{\isacharparenright}{\kern0pt}\isanewline
\ \ \ \ \isacommand{using}\isamarkupfalse%
\ take{\isacharunderscore}{\kern0pt}drop{\isacharunderscore}{\kern0pt}eq{\isacharbrackleft}{\kern0pt}OF\ {\isacartoucheopen}rest{\isasymin}{\isacharunderscore}{\kern0pt}{\isacartoucheclose}{\isacharbrackright}{\kern0pt}\ le{\isacharunderscore}{\kern0pt}natI\ \isacommand{by}\isamarkupfalse%
\ auto\isanewline
\ \ \isacommand{then}\isamarkupfalse%
\isanewline
\ \ \isacommand{have}\isamarkupfalse%
\ {\isachardoublequoteopen}length{\isacharparenleft}{\kern0pt}{\isacharquery}{\kern0pt}re{\isacharparenright}{\kern0pt}\ {\isacharequal}{\kern0pt}\ pred{\isacharparenleft}{\kern0pt}n{\isacharparenright}{\kern0pt}{\isachardoublequoteclose}\ {\isachardoublequoteopen}{\isacharquery}{\kern0pt}re{\isasymin}list{\isacharparenleft}{\kern0pt}M{\isacharparenright}{\kern0pt}{\isachardoublequoteclose}\ {\isachardoublequoteopen}{\isacharquery}{\kern0pt}st{\isasymin}list{\isacharparenleft}{\kern0pt}M{\isacharparenright}{\kern0pt}{\isachardoublequoteclose}\isanewline
\ \ \ \ \isacommand{using}\isamarkupfalse%
\ length{\isacharunderscore}{\kern0pt}take{\isacharbrackleft}{\kern0pt}rule{\isacharunderscore}{\kern0pt}format{\isacharcomma}{\kern0pt}OF\ {\isacharunderscore}{\kern0pt}\ {\isacartoucheopen}pred{\isacharparenleft}{\kern0pt}n{\isacharparenright}{\kern0pt}{\isasymin}{\isacharunderscore}{\kern0pt}{\isacartoucheclose}{\isacharbrackright}{\kern0pt}\ {\isacartoucheopen}pred{\isacharparenleft}{\kern0pt}n{\isacharparenright}{\kern0pt}\ {\isasymle}\ {\isacharunderscore}{\kern0pt}{\isacartoucheclose}\ {\isacartoucheopen}rest{\isasymin}{\isacharunderscore}{\kern0pt}{\isacartoucheclose}\isanewline
\ \ \ \ \isacommand{unfolding}\isamarkupfalse%
\ min{\isacharunderscore}{\kern0pt}def\isanewline
\ \ \ \ \isacommand{by}\isamarkupfalse%
\ auto\isanewline
\ \ \isacommand{then}\isamarkupfalse%
\isanewline
\ \ \isacommand{show}\isamarkupfalse%
\ {\isacharquery}{\kern0pt}thesis\isanewline
\ \ \ \ \isacommand{using}\isamarkupfalse%
\ rev{\isacharunderscore}{\kern0pt}bexI{\isacharbrackleft}{\kern0pt}of\ {\isacharunderscore}{\kern0pt}\ {\isacharunderscore}{\kern0pt}\ {\isachardoublequoteopen}{\isasymlambda}\ re{\isachardot}{\kern0pt}\ {\isasymexists}st{\isasymin}list{\isacharparenleft}{\kern0pt}M{\isacharparenright}{\kern0pt}{\isachardot}{\kern0pt}\ rest\ {\isacharequal}{\kern0pt}\ re\ {\isacharat}{\kern0pt}\ st\ {\isasymand}\ length{\isacharparenleft}{\kern0pt}re{\isacharparenright}{\kern0pt}\ {\isacharequal}{\kern0pt}\ pred{\isacharparenleft}{\kern0pt}n{\isacharparenright}{\kern0pt}{\isachardoublequoteclose}{\isacharbrackright}{\kern0pt}\isanewline
\ \ \ \ \ \ {\isacartoucheopen}length{\isacharparenleft}{\kern0pt}{\isacharquery}{\kern0pt}re{\isacharparenright}{\kern0pt}\ {\isacharequal}{\kern0pt}\ {\isacharunderscore}{\kern0pt}{\isacartoucheclose}\ {\isacartoucheopen}rest\ {\isacharequal}{\kern0pt}\ {\isacharunderscore}{\kern0pt}{\isacartoucheclose}\isanewline
\ \ \ \ \isacommand{by}\isamarkupfalse%
\ auto\isanewline
\isacommand{qed}\isamarkupfalse%
%
\endisatagproof
{\isafoldproof}%
%
\isadelimproof
\isanewline
%
\endisadelimproof
\isanewline
\isacommand{lemma}\isamarkupfalse%
\ sats{\isacharunderscore}{\kern0pt}nForall{\isacharcolon}{\kern0pt}\isanewline
\ \ \isakeyword{assumes}\isanewline
\ \ \ \ {\isachardoublequoteopen}{\isasymphi}\ {\isasymin}\ formula{\isachardoublequoteclose}\isanewline
\ \ \isakeyword{shows}\isanewline
\ \ \ \ {\isachardoublequoteopen}n{\isasymin}nat\ {\isasymLongrightarrow}\ ms\ {\isasymin}\ list{\isacharparenleft}{\kern0pt}M{\isacharparenright}{\kern0pt}\ {\isasymLongrightarrow}\isanewline
\ \ \ \ \ \ \ M{\isacharcomma}{\kern0pt}\ ms\ {\isasymTurnstile}\ {\isacharparenleft}{\kern0pt}Forall{\isacharcircum}{\kern0pt}n{\isacharparenleft}{\kern0pt}{\isasymphi}{\isacharparenright}{\kern0pt}{\isacharparenright}{\kern0pt}\ {\isasymlongleftrightarrow}\isanewline
\ \ \ \ \ \ \ {\isacharparenleft}{\kern0pt}{\isasymforall}rest\ {\isasymin}\ list{\isacharparenleft}{\kern0pt}M{\isacharparenright}{\kern0pt}{\isachardot}{\kern0pt}\ length{\isacharparenleft}{\kern0pt}rest{\isacharparenright}{\kern0pt}\ {\isacharequal}{\kern0pt}\ n\ {\isasymlongrightarrow}\ M{\isacharcomma}{\kern0pt}\ rest\ {\isacharat}{\kern0pt}\ ms\ {\isasymTurnstile}\ {\isasymphi}{\isacharparenright}{\kern0pt}{\isachardoublequoteclose}\isanewline
%
\isadelimproof
%
\endisadelimproof
%
\isatagproof
\isacommand{proof}\isamarkupfalse%
\ {\isacharparenleft}{\kern0pt}induct\ n\ arbitrary{\isacharcolon}{\kern0pt}ms\ set{\isacharcolon}{\kern0pt}nat{\isacharparenright}{\kern0pt}\isanewline
\ \ \isacommand{case}\isamarkupfalse%
\ {\isadigit{0}}\isanewline
\ \ \isacommand{with}\isamarkupfalse%
\ assms\isanewline
\ \ \isacommand{show}\isamarkupfalse%
\ {\isacharquery}{\kern0pt}case\ \isacommand{by}\isamarkupfalse%
\ simp\isanewline
\isacommand{next}\isamarkupfalse%
\isanewline
\ \ \isacommand{case}\isamarkupfalse%
\ {\isacharparenleft}{\kern0pt}succ\ n{\isacharparenright}{\kern0pt}\isanewline
\ \ \isacommand{have}\isamarkupfalse%
\ {\isachardoublequoteopen}{\isacharparenleft}{\kern0pt}{\isasymforall}rest{\isasymin}list{\isacharparenleft}{\kern0pt}M{\isacharparenright}{\kern0pt}{\isachardot}{\kern0pt}\ length{\isacharparenleft}{\kern0pt}rest{\isacharparenright}{\kern0pt}\ {\isacharequal}{\kern0pt}\ succ{\isacharparenleft}{\kern0pt}n{\isacharparenright}{\kern0pt}\ {\isasymlongrightarrow}\ P{\isacharparenleft}{\kern0pt}rest{\isacharcomma}{\kern0pt}n{\isacharparenright}{\kern0pt}{\isacharparenright}{\kern0pt}\ {\isasymlongleftrightarrow}\isanewline
\ \ \ \ \ \ \ \ {\isacharparenleft}{\kern0pt}{\isasymforall}t{\isasymin}M{\isachardot}{\kern0pt}\ {\isasymforall}res{\isasymin}list{\isacharparenleft}{\kern0pt}M{\isacharparenright}{\kern0pt}{\isachardot}{\kern0pt}\ length{\isacharparenleft}{\kern0pt}res{\isacharparenright}{\kern0pt}\ {\isacharequal}{\kern0pt}\ n\ {\isasymlongrightarrow}\ P{\isacharparenleft}{\kern0pt}res\ {\isacharat}{\kern0pt}\ {\isacharbrackleft}{\kern0pt}t{\isacharbrackright}{\kern0pt}{\isacharcomma}{\kern0pt}n{\isacharparenright}{\kern0pt}{\isacharparenright}{\kern0pt}{\isachardoublequoteclose}\isanewline
\ \ \ \ \isakeyword{if}\ {\isachardoublequoteopen}n{\isasymin}nat{\isachardoublequoteclose}\ \isakeyword{for}\ n\ P\isanewline
\ \ \ \ \isacommand{using}\isamarkupfalse%
\ that\ last{\isacharunderscore}{\kern0pt}init{\isacharunderscore}{\kern0pt}eq\ \isacommand{by}\isamarkupfalse%
\ force\isanewline
\ \ \isacommand{from}\isamarkupfalse%
\ this{\isacharbrackleft}{\kern0pt}of\ {\isacharunderscore}{\kern0pt}\ {\isachardoublequoteopen}{\isasymlambda}rest\ {\isacharunderscore}{\kern0pt}{\isachardot}{\kern0pt}\ {\isacharparenleft}{\kern0pt}M{\isacharcomma}{\kern0pt}\ rest\ {\isacharat}{\kern0pt}\ ms\ {\isasymTurnstile}\ {\isasymphi}{\isacharparenright}{\kern0pt}{\isachardoublequoteclose}{\isacharbrackright}{\kern0pt}\ {\isacartoucheopen}n{\isasymin}nat{\isacartoucheclose}\isanewline
\ \ \isacommand{have}\isamarkupfalse%
\ {\isachardoublequoteopen}{\isacharparenleft}{\kern0pt}{\isasymforall}rest{\isasymin}list{\isacharparenleft}{\kern0pt}M{\isacharparenright}{\kern0pt}{\isachardot}{\kern0pt}\ length{\isacharparenleft}{\kern0pt}rest{\isacharparenright}{\kern0pt}\ {\isacharequal}{\kern0pt}\ succ{\isacharparenleft}{\kern0pt}n{\isacharparenright}{\kern0pt}\ {\isasymlongrightarrow}\ M{\isacharcomma}{\kern0pt}\ rest\ {\isacharat}{\kern0pt}\ ms\ {\isasymTurnstile}\ {\isasymphi}{\isacharparenright}{\kern0pt}\ {\isasymlongleftrightarrow}\isanewline
\ \ \ \ \ \ \ \ {\isacharparenleft}{\kern0pt}{\isasymforall}t{\isasymin}M{\isachardot}{\kern0pt}\ {\isasymforall}res{\isasymin}list{\isacharparenleft}{\kern0pt}M{\isacharparenright}{\kern0pt}{\isachardot}{\kern0pt}\ length{\isacharparenleft}{\kern0pt}res{\isacharparenright}{\kern0pt}\ {\isacharequal}{\kern0pt}\ n\ {\isasymlongrightarrow}\ \ M{\isacharcomma}{\kern0pt}\ {\isacharparenleft}{\kern0pt}res\ {\isacharat}{\kern0pt}\ {\isacharbrackleft}{\kern0pt}t{\isacharbrackright}{\kern0pt}{\isacharparenright}{\kern0pt}\ {\isacharat}{\kern0pt}\ ms\ {\isasymTurnstile}\ {\isasymphi}{\isacharparenright}{\kern0pt}{\isachardoublequoteclose}\isanewline
\ \ \ \ \isacommand{by}\isamarkupfalse%
\ simp\isanewline
\ \ \ \ \isacommand{with}\isamarkupfalse%
\ assms\ succ{\isacharparenleft}{\kern0pt}{\isadigit{1}}{\isacharcomma}{\kern0pt}{\isadigit{3}}{\isacharparenright}{\kern0pt}\ succ{\isacharparenleft}{\kern0pt}{\isadigit{2}}{\isacharparenright}{\kern0pt}{\isacharbrackleft}{\kern0pt}of\ {\isachardoublequoteopen}Cons{\isacharparenleft}{\kern0pt}{\isacharunderscore}{\kern0pt}{\isacharcomma}{\kern0pt}ms{\isacharparenright}{\kern0pt}{\isachardoublequoteclose}{\isacharbrackright}{\kern0pt}\isanewline
\ \ \isacommand{show}\isamarkupfalse%
\ {\isacharquery}{\kern0pt}case\isanewline
\ \ \ \ \isacommand{using}\isamarkupfalse%
\ arity{\isacharunderscore}{\kern0pt}sats{\isacharunderscore}{\kern0pt}iff{\isacharbrackleft}{\kern0pt}of\ {\isasymphi}\ {\isacharunderscore}{\kern0pt}\ M\ {\isachardoublequoteopen}Cons{\isacharparenleft}{\kern0pt}{\isacharunderscore}{\kern0pt}{\isacharcomma}{\kern0pt}\ ms\ {\isacharat}{\kern0pt}\ {\isacharunderscore}{\kern0pt}{\isacharparenright}{\kern0pt}{\isachardoublequoteclose}{\isacharbrackright}{\kern0pt}\ app{\isacharunderscore}{\kern0pt}assoc\isanewline
\ \ \ \ \isacommand{by}\isamarkupfalse%
\ {\isacharparenleft}{\kern0pt}simp{\isacharparenright}{\kern0pt}\isanewline
\isacommand{qed}\isamarkupfalse%
%
\endisatagproof
{\isafoldproof}%
%
\isadelimproof
\isanewline
%
\endisadelimproof
\isanewline
\isacommand{definition}\isamarkupfalse%
\isanewline
\ \ sep{\isacharunderscore}{\kern0pt}body{\isacharunderscore}{\kern0pt}fm\ {\isacharcolon}{\kern0pt}{\isacharcolon}{\kern0pt}\ {\isachardoublequoteopen}i\ {\isasymRightarrow}\ i{\isachardoublequoteclose}\ \isakeyword{where}\isanewline
\ \ {\isachardoublequoteopen}sep{\isacharunderscore}{\kern0pt}body{\isacharunderscore}{\kern0pt}fm{\isacharparenleft}{\kern0pt}p{\isacharparenright}{\kern0pt}\ {\isasymequiv}\ Forall{\isacharparenleft}{\kern0pt}Exists{\isacharparenleft}{\kern0pt}Forall{\isacharparenleft}{\kern0pt}\isanewline
\ \ \ \ \ \ \ \ \ \ \ \ \ \ \ \ \ \ \ \ \ \ \ \ \ \ \ Iff{\isacharparenleft}{\kern0pt}Member{\isacharparenleft}{\kern0pt}{\isadigit{0}}{\isacharcomma}{\kern0pt}{\isadigit{1}}{\isacharparenright}{\kern0pt}{\isacharcomma}{\kern0pt}And{\isacharparenleft}{\kern0pt}Member{\isacharparenleft}{\kern0pt}{\isadigit{0}}{\isacharcomma}{\kern0pt}{\isadigit{2}}{\isacharparenright}{\kern0pt}{\isacharcomma}{\kern0pt}\isanewline
\ \ \ \ \ \ \ \ \ \ \ \ \ \ \ \ \ \ \ \ \ \ \ \ \ \ \ \ \ \ \ \ \ \ \ \ incr{\isacharunderscore}{\kern0pt}bv{\isadigit{1}}{\isacharcircum}{\kern0pt}{\isadigit{2}}{\isacharparenleft}{\kern0pt}p{\isacharparenright}{\kern0pt}{\isacharparenright}{\kern0pt}{\isacharparenright}{\kern0pt}{\isacharparenright}{\kern0pt}{\isacharparenright}{\kern0pt}{\isacharparenright}{\kern0pt}{\isachardoublequoteclose}\isanewline
\isanewline
\isacommand{lemma}\isamarkupfalse%
\ sep{\isacharunderscore}{\kern0pt}body{\isacharunderscore}{\kern0pt}fm{\isacharunderscore}{\kern0pt}type\ {\isacharbrackleft}{\kern0pt}TC{\isacharbrackright}{\kern0pt}{\isacharcolon}{\kern0pt}\ {\isachardoublequoteopen}p\ {\isasymin}\ formula\ {\isasymLongrightarrow}\ sep{\isacharunderscore}{\kern0pt}body{\isacharunderscore}{\kern0pt}fm{\isacharparenleft}{\kern0pt}p{\isacharparenright}{\kern0pt}\ {\isasymin}\ formula{\isachardoublequoteclose}\isanewline
%
\isadelimproof
\ \ %
\endisadelimproof
%
\isatagproof
\isacommand{by}\isamarkupfalse%
\ {\isacharparenleft}{\kern0pt}simp\ add{\isacharcolon}{\kern0pt}\ sep{\isacharunderscore}{\kern0pt}body{\isacharunderscore}{\kern0pt}fm{\isacharunderscore}{\kern0pt}def{\isacharparenright}{\kern0pt}%
\endisatagproof
{\isafoldproof}%
%
\isadelimproof
\isanewline
%
\endisadelimproof
\isanewline
\isacommand{lemma}\isamarkupfalse%
\ sats{\isacharunderscore}{\kern0pt}sep{\isacharunderscore}{\kern0pt}body{\isacharunderscore}{\kern0pt}fm{\isacharcolon}{\kern0pt}\ \isanewline
\ \ \isakeyword{assumes}\isanewline
\ \ \ \ {\isachardoublequoteopen}{\isasymphi}\ {\isasymin}\ formula{\isachardoublequoteclose}\ {\isachardoublequoteopen}ms{\isasymin}list{\isacharparenleft}{\kern0pt}M{\isacharparenright}{\kern0pt}{\isachardoublequoteclose}\ {\isachardoublequoteopen}rest{\isasymin}list{\isacharparenleft}{\kern0pt}M{\isacharparenright}{\kern0pt}{\isachardoublequoteclose}\isanewline
\ \ \isakeyword{shows}\isanewline
\ \ \ \ {\isachardoublequoteopen}M{\isacharcomma}{\kern0pt}\ rest\ {\isacharat}{\kern0pt}\ ms\ {\isasymTurnstile}\ sep{\isacharunderscore}{\kern0pt}body{\isacharunderscore}{\kern0pt}fm{\isacharparenleft}{\kern0pt}{\isasymphi}{\isacharparenright}{\kern0pt}\ {\isasymlongleftrightarrow}\ \isanewline
\ \ \ \ \ separation{\isacharparenleft}{\kern0pt}{\isacharhash}{\kern0pt}{\isacharhash}{\kern0pt}M{\isacharcomma}{\kern0pt}{\isasymlambda}x{\isachardot}{\kern0pt}\ M{\isacharcomma}{\kern0pt}\ {\isacharbrackleft}{\kern0pt}x{\isacharbrackright}{\kern0pt}\ {\isacharat}{\kern0pt}\ rest\ {\isacharat}{\kern0pt}\ ms\ {\isasymTurnstile}\ {\isasymphi}{\isacharparenright}{\kern0pt}{\isachardoublequoteclose}\isanewline
%
\isadelimproof
\ \ %
\endisadelimproof
%
\isatagproof
\isacommand{using}\isamarkupfalse%
\ assms\ formula{\isacharunderscore}{\kern0pt}add{\isacharunderscore}{\kern0pt}params{\isadigit{1}}{\isacharbrackleft}{\kern0pt}of\ {\isacharunderscore}{\kern0pt}\ {\isadigit{2}}\ {\isacharunderscore}{\kern0pt}\ {\isacharunderscore}{\kern0pt}\ {\isachardoublequoteopen}{\isacharbrackleft}{\kern0pt}{\isacharunderscore}{\kern0pt}{\isacharcomma}{\kern0pt}{\isacharunderscore}{\kern0pt}{\isacharbrackright}{\kern0pt}{\isachardoublequoteclose}\ {\isacharbrackright}{\kern0pt}\isanewline
\ \ \isacommand{unfolding}\isamarkupfalse%
\ sep{\isacharunderscore}{\kern0pt}body{\isacharunderscore}{\kern0pt}fm{\isacharunderscore}{\kern0pt}def\ separation{\isacharunderscore}{\kern0pt}def\ \isacommand{by}\isamarkupfalse%
\ simp%
\endisatagproof
{\isafoldproof}%
%
\isadelimproof
\isanewline
%
\endisadelimproof
\isanewline
\isacommand{definition}\isamarkupfalse%
\isanewline
\ \ ZF{\isacharunderscore}{\kern0pt}separation{\isacharunderscore}{\kern0pt}fm\ {\isacharcolon}{\kern0pt}{\isacharcolon}{\kern0pt}\ {\isachardoublequoteopen}i\ {\isasymRightarrow}\ i{\isachardoublequoteclose}\ \isakeyword{where}\isanewline
\ \ {\isachardoublequoteopen}ZF{\isacharunderscore}{\kern0pt}separation{\isacharunderscore}{\kern0pt}fm{\isacharparenleft}{\kern0pt}p{\isacharparenright}{\kern0pt}\ {\isasymequiv}\ Forall{\isacharcircum}{\kern0pt}{\isacharparenleft}{\kern0pt}pred{\isacharparenleft}{\kern0pt}arity{\isacharparenleft}{\kern0pt}p{\isacharparenright}{\kern0pt}{\isacharparenright}{\kern0pt}{\isacharparenright}{\kern0pt}{\isacharparenleft}{\kern0pt}sep{\isacharunderscore}{\kern0pt}body{\isacharunderscore}{\kern0pt}fm{\isacharparenleft}{\kern0pt}p{\isacharparenright}{\kern0pt}{\isacharparenright}{\kern0pt}{\isachardoublequoteclose}\isanewline
\isanewline
\isacommand{lemma}\isamarkupfalse%
\ ZF{\isacharunderscore}{\kern0pt}separation{\isacharunderscore}{\kern0pt}fm{\isacharunderscore}{\kern0pt}type\ {\isacharbrackleft}{\kern0pt}TC{\isacharbrackright}{\kern0pt}{\isacharcolon}{\kern0pt}\ {\isachardoublequoteopen}p\ {\isasymin}\ formula\ {\isasymLongrightarrow}\ ZF{\isacharunderscore}{\kern0pt}separation{\isacharunderscore}{\kern0pt}fm{\isacharparenleft}{\kern0pt}p{\isacharparenright}{\kern0pt}\ {\isasymin}\ formula{\isachardoublequoteclose}\isanewline
%
\isadelimproof
\ \ %
\endisadelimproof
%
\isatagproof
\isacommand{by}\isamarkupfalse%
\ {\isacharparenleft}{\kern0pt}simp\ add{\isacharcolon}{\kern0pt}\ ZF{\isacharunderscore}{\kern0pt}separation{\isacharunderscore}{\kern0pt}fm{\isacharunderscore}{\kern0pt}def{\isacharparenright}{\kern0pt}%
\endisatagproof
{\isafoldproof}%
%
\isadelimproof
\isanewline
%
\endisadelimproof
\isanewline
\isacommand{lemma}\isamarkupfalse%
\ sats{\isacharunderscore}{\kern0pt}ZF{\isacharunderscore}{\kern0pt}separation{\isacharunderscore}{\kern0pt}fm{\isacharunderscore}{\kern0pt}iff{\isacharcolon}{\kern0pt}\isanewline
\ \ \isakeyword{assumes}\isanewline
\ \ \ \ {\isachardoublequoteopen}{\isasymphi}{\isasymin}formula{\isachardoublequoteclose}\isanewline
\ \ \isakeyword{shows}\isanewline
\ \ {\isachardoublequoteopen}{\isacharparenleft}{\kern0pt}M{\isacharcomma}{\kern0pt}\ {\isacharbrackleft}{\kern0pt}{\isacharbrackright}{\kern0pt}\ {\isasymTurnstile}\ {\isacharparenleft}{\kern0pt}ZF{\isacharunderscore}{\kern0pt}separation{\isacharunderscore}{\kern0pt}fm{\isacharparenleft}{\kern0pt}{\isasymphi}{\isacharparenright}{\kern0pt}{\isacharparenright}{\kern0pt}{\isacharparenright}{\kern0pt}\isanewline
\ \ \ {\isasymlongleftrightarrow}\isanewline
\ \ \ {\isacharparenleft}{\kern0pt}{\isasymforall}env{\isasymin}list{\isacharparenleft}{\kern0pt}M{\isacharparenright}{\kern0pt}{\isachardot}{\kern0pt}\ arity{\isacharparenleft}{\kern0pt}{\isasymphi}{\isacharparenright}{\kern0pt}\ {\isasymle}\ {\isadigit{1}}\ {\isacharhash}{\kern0pt}{\isacharplus}{\kern0pt}\ length{\isacharparenleft}{\kern0pt}env{\isacharparenright}{\kern0pt}\ {\isasymlongrightarrow}\ \isanewline
\ \ \ \ \ \ separation{\isacharparenleft}{\kern0pt}{\isacharhash}{\kern0pt}{\isacharhash}{\kern0pt}M{\isacharcomma}{\kern0pt}{\isasymlambda}x{\isachardot}{\kern0pt}\ M{\isacharcomma}{\kern0pt}\ {\isacharbrackleft}{\kern0pt}x{\isacharbrackright}{\kern0pt}\ {\isacharat}{\kern0pt}\ env\ {\isasymTurnstile}\ {\isasymphi}{\isacharparenright}{\kern0pt}{\isacharparenright}{\kern0pt}{\isachardoublequoteclose}\isanewline
%
\isadelimproof
%
\endisadelimproof
%
\isatagproof
\isacommand{proof}\isamarkupfalse%
\ {\isacharparenleft}{\kern0pt}intro\ iffI\ ballI\ impI{\isacharparenright}{\kern0pt}\isanewline
\ \ \isacommand{let}\isamarkupfalse%
\ {\isacharquery}{\kern0pt}n{\isacharequal}{\kern0pt}{\isachardoublequoteopen}Arith{\isachardot}{\kern0pt}pred{\isacharparenleft}{\kern0pt}arity{\isacharparenleft}{\kern0pt}{\isasymphi}{\isacharparenright}{\kern0pt}{\isacharparenright}{\kern0pt}{\isachardoublequoteclose}\isanewline
\ \ \isacommand{fix}\isamarkupfalse%
\ env\isanewline
\ \ \isacommand{assume}\isamarkupfalse%
\ {\isachardoublequoteopen}M{\isacharcomma}{\kern0pt}\ {\isacharbrackleft}{\kern0pt}{\isacharbrackright}{\kern0pt}\ {\isasymTurnstile}\ ZF{\isacharunderscore}{\kern0pt}separation{\isacharunderscore}{\kern0pt}fm{\isacharparenleft}{\kern0pt}{\isasymphi}{\isacharparenright}{\kern0pt}{\isachardoublequoteclose}\ \isanewline
\ \ \isacommand{assume}\isamarkupfalse%
\ {\isachardoublequoteopen}arity{\isacharparenleft}{\kern0pt}{\isasymphi}{\isacharparenright}{\kern0pt}\ {\isasymle}\ {\isadigit{1}}\ {\isacharhash}{\kern0pt}{\isacharplus}{\kern0pt}\ length{\isacharparenleft}{\kern0pt}env{\isacharparenright}{\kern0pt}{\isachardoublequoteclose}\ {\isachardoublequoteopen}env{\isasymin}list{\isacharparenleft}{\kern0pt}M{\isacharparenright}{\kern0pt}{\isachardoublequoteclose}\isanewline
\ \ \isacommand{moreover}\isamarkupfalse%
\ \isacommand{from}\isamarkupfalse%
\ this\isanewline
\ \ \isacommand{have}\isamarkupfalse%
\ {\isachardoublequoteopen}arity{\isacharparenleft}{\kern0pt}{\isasymphi}{\isacharparenright}{\kern0pt}\ {\isasymle}\ succ{\isacharparenleft}{\kern0pt}length{\isacharparenleft}{\kern0pt}env{\isacharparenright}{\kern0pt}{\isacharparenright}{\kern0pt}{\isachardoublequoteclose}\ \isacommand{by}\isamarkupfalse%
\ simp\isanewline
\ \ \isacommand{then}\isamarkupfalse%
\isanewline
\ \ \isacommand{obtain}\isamarkupfalse%
\ some\ rest\ \isakeyword{where}\ {\isachardoublequoteopen}some{\isasymin}list{\isacharparenleft}{\kern0pt}M{\isacharparenright}{\kern0pt}{\isachardoublequoteclose}\ {\isachardoublequoteopen}rest{\isasymin}list{\isacharparenleft}{\kern0pt}M{\isacharparenright}{\kern0pt}{\isachardoublequoteclose}\ \isanewline
\ \ \ \ {\isachardoublequoteopen}env\ {\isacharequal}{\kern0pt}\ some\ {\isacharat}{\kern0pt}\ rest{\isachardoublequoteclose}\ {\isachardoublequoteopen}length{\isacharparenleft}{\kern0pt}some{\isacharparenright}{\kern0pt}\ {\isacharequal}{\kern0pt}\ Arith{\isachardot}{\kern0pt}pred{\isacharparenleft}{\kern0pt}arity{\isacharparenleft}{\kern0pt}{\isasymphi}{\isacharparenright}{\kern0pt}{\isacharparenright}{\kern0pt}{\isachardoublequoteclose}\isanewline
\ \ \ \ \isacommand{using}\isamarkupfalse%
\ list{\isacharunderscore}{\kern0pt}split{\isacharbrackleft}{\kern0pt}OF\ {\isacartoucheopen}arity{\isacharparenleft}{\kern0pt}{\isasymphi}{\isacharparenright}{\kern0pt}\ {\isasymle}\ succ{\isacharparenleft}{\kern0pt}{\isacharunderscore}{\kern0pt}{\isacharparenright}{\kern0pt}{\isacartoucheclose}\ {\isacartoucheopen}env{\isasymin}{\isacharunderscore}{\kern0pt}{\isacartoucheclose}{\isacharbrackright}{\kern0pt}\ \isacommand{by}\isamarkupfalse%
\ force\isanewline
\ \ \isacommand{moreover}\isamarkupfalse%
\ \isacommand{from}\isamarkupfalse%
\ {\isacartoucheopen}{\isasymphi}{\isasymin}{\isacharunderscore}{\kern0pt}{\isacartoucheclose}\isanewline
\ \ \isacommand{have}\isamarkupfalse%
\ {\isachardoublequoteopen}arity{\isacharparenleft}{\kern0pt}{\isasymphi}{\isacharparenright}{\kern0pt}\ {\isasymle}\ succ{\isacharparenleft}{\kern0pt}Arith{\isachardot}{\kern0pt}pred{\isacharparenleft}{\kern0pt}arity{\isacharparenleft}{\kern0pt}{\isasymphi}{\isacharparenright}{\kern0pt}{\isacharparenright}{\kern0pt}{\isacharparenright}{\kern0pt}{\isachardoublequoteclose}\isanewline
\ \ \ \isacommand{using}\isamarkupfalse%
\ succpred{\isacharunderscore}{\kern0pt}leI\ \isacommand{by}\isamarkupfalse%
\ simp\isanewline
\ \ \isacommand{moreover}\isamarkupfalse%
\isanewline
\ \ \isacommand{note}\isamarkupfalse%
\ assms\isanewline
\ \ \isacommand{moreover}\isamarkupfalse%
\ \isanewline
\ \ \isacommand{assume}\isamarkupfalse%
\ {\isachardoublequoteopen}M{\isacharcomma}{\kern0pt}\ {\isacharbrackleft}{\kern0pt}{\isacharbrackright}{\kern0pt}\ {\isasymTurnstile}\ ZF{\isacharunderscore}{\kern0pt}separation{\isacharunderscore}{\kern0pt}fm{\isacharparenleft}{\kern0pt}{\isasymphi}{\isacharparenright}{\kern0pt}{\isachardoublequoteclose}\ \isanewline
\ \ \isacommand{moreover}\isamarkupfalse%
\ \isacommand{from}\isamarkupfalse%
\ calculation\isanewline
\ \ \isacommand{have}\isamarkupfalse%
\ {\isachardoublequoteopen}M{\isacharcomma}{\kern0pt}\ some\ {\isasymTurnstile}\ sep{\isacharunderscore}{\kern0pt}body{\isacharunderscore}{\kern0pt}fm{\isacharparenleft}{\kern0pt}{\isasymphi}{\isacharparenright}{\kern0pt}{\isachardoublequoteclose}\isanewline
\ \ \ \ \isacommand{using}\isamarkupfalse%
\ sats{\isacharunderscore}{\kern0pt}nForall{\isacharbrackleft}{\kern0pt}of\ {\isachardoublequoteopen}sep{\isacharunderscore}{\kern0pt}body{\isacharunderscore}{\kern0pt}fm{\isacharparenleft}{\kern0pt}{\isasymphi}{\isacharparenright}{\kern0pt}{\isachardoublequoteclose}\ {\isacharquery}{\kern0pt}n{\isacharbrackright}{\kern0pt}\isanewline
\ \ \ \ \isacommand{unfolding}\isamarkupfalse%
\ ZF{\isacharunderscore}{\kern0pt}separation{\isacharunderscore}{\kern0pt}fm{\isacharunderscore}{\kern0pt}def\ \isacommand{by}\isamarkupfalse%
\ simp\isanewline
\ \ \isacommand{ultimately}\isamarkupfalse%
\isanewline
\ \ \isacommand{show}\isamarkupfalse%
\ {\isachardoublequoteopen}separation{\isacharparenleft}{\kern0pt}{\isacharhash}{\kern0pt}{\isacharhash}{\kern0pt}M{\isacharcomma}{\kern0pt}\ {\isasymlambda}x{\isachardot}{\kern0pt}\ M{\isacharcomma}{\kern0pt}\ {\isacharbrackleft}{\kern0pt}x{\isacharbrackright}{\kern0pt}\ {\isacharat}{\kern0pt}\ env\ {\isasymTurnstile}\ {\isasymphi}{\isacharparenright}{\kern0pt}{\isachardoublequoteclose}\isanewline
\ \ \ \ \isacommand{unfolding}\isamarkupfalse%
\ ZF{\isacharunderscore}{\kern0pt}separation{\isacharunderscore}{\kern0pt}fm{\isacharunderscore}{\kern0pt}def\isanewline
\ \ \ \ \isacommand{using}\isamarkupfalse%
\ sats{\isacharunderscore}{\kern0pt}sep{\isacharunderscore}{\kern0pt}body{\isacharunderscore}{\kern0pt}fm{\isacharbrackleft}{\kern0pt}of\ {\isasymphi}\ {\isachardoublequoteopen}{\isacharbrackleft}{\kern0pt}{\isacharbrackright}{\kern0pt}{\isachardoublequoteclose}\ M\ some{\isacharbrackright}{\kern0pt}\isanewline
\ \ \ \ \ \ arity{\isacharunderscore}{\kern0pt}sats{\isacharunderscore}{\kern0pt}iff{\isacharbrackleft}{\kern0pt}of\ {\isasymphi}\ rest\ M\ {\isachardoublequoteopen}{\isacharbrackleft}{\kern0pt}{\isacharunderscore}{\kern0pt}{\isacharbrackright}{\kern0pt}\ {\isacharat}{\kern0pt}\ some{\isachardoublequoteclose}{\isacharbrackright}{\kern0pt}\isanewline
\ \ \ \ \ \ separation{\isacharunderscore}{\kern0pt}cong{\isacharbrackleft}{\kern0pt}of\ {\isachardoublequoteopen}{\isacharhash}{\kern0pt}{\isacharhash}{\kern0pt}M{\isachardoublequoteclose}\ {\isachardoublequoteopen}{\isasymlambda}x{\isachardot}{\kern0pt}\ M{\isacharcomma}{\kern0pt}\ Cons{\isacharparenleft}{\kern0pt}x{\isacharcomma}{\kern0pt}\ some\ {\isacharat}{\kern0pt}\ rest{\isacharparenright}{\kern0pt}\ {\isasymTurnstile}\ {\isasymphi}{\isachardoublequoteclose}\ {\isacharunderscore}{\kern0pt}\ {\isacharbrackright}{\kern0pt}\isanewline
\ \ \ \ \isacommand{by}\isamarkupfalse%
\ simp\isanewline
\isacommand{next}\isamarkupfalse%
\ %
\isamarkupcmt{almost equal to the previous implication%
}\isanewline
\ \ \isacommand{let}\isamarkupfalse%
\ {\isacharquery}{\kern0pt}n{\isacharequal}{\kern0pt}{\isachardoublequoteopen}Arith{\isachardot}{\kern0pt}pred{\isacharparenleft}{\kern0pt}arity{\isacharparenleft}{\kern0pt}{\isasymphi}{\isacharparenright}{\kern0pt}{\isacharparenright}{\kern0pt}{\isachardoublequoteclose}\isanewline
\ \ \isacommand{assume}\isamarkupfalse%
\ asm{\isacharcolon}{\kern0pt}{\isachardoublequoteopen}{\isasymforall}env{\isasymin}list{\isacharparenleft}{\kern0pt}M{\isacharparenright}{\kern0pt}{\isachardot}{\kern0pt}\ arity{\isacharparenleft}{\kern0pt}{\isasymphi}{\isacharparenright}{\kern0pt}\ {\isasymle}\ {\isadigit{1}}\ {\isacharhash}{\kern0pt}{\isacharplus}{\kern0pt}\ length{\isacharparenleft}{\kern0pt}env{\isacharparenright}{\kern0pt}\ {\isasymlongrightarrow}\ \isanewline
\ \ \ \ separation{\isacharparenleft}{\kern0pt}{\isacharhash}{\kern0pt}{\isacharhash}{\kern0pt}M{\isacharcomma}{\kern0pt}\ {\isasymlambda}x{\isachardot}{\kern0pt}\ M{\isacharcomma}{\kern0pt}\ {\isacharbrackleft}{\kern0pt}x{\isacharbrackright}{\kern0pt}\ {\isacharat}{\kern0pt}\ env\ {\isasymTurnstile}\ {\isasymphi}{\isacharparenright}{\kern0pt}{\isachardoublequoteclose}\isanewline
\ \ \isacommand{{\isacharbraceleft}{\kern0pt}}\isamarkupfalse%
\isanewline
\ \ \ \ \isacommand{fix}\isamarkupfalse%
\ some\isanewline
\ \ \ \ \isacommand{assume}\isamarkupfalse%
\ {\isachardoublequoteopen}some{\isasymin}list{\isacharparenleft}{\kern0pt}M{\isacharparenright}{\kern0pt}{\isachardoublequoteclose}\ {\isachardoublequoteopen}length{\isacharparenleft}{\kern0pt}some{\isacharparenright}{\kern0pt}\ {\isacharequal}{\kern0pt}\ Arith{\isachardot}{\kern0pt}pred{\isacharparenleft}{\kern0pt}arity{\isacharparenleft}{\kern0pt}{\isasymphi}{\isacharparenright}{\kern0pt}{\isacharparenright}{\kern0pt}{\isachardoublequoteclose}\isanewline
\ \ \ \ \isacommand{moreover}\isamarkupfalse%
\isanewline
\ \ \ \ \isacommand{note}\isamarkupfalse%
\ {\isacartoucheopen}{\isasymphi}{\isasymin}{\isacharunderscore}{\kern0pt}{\isacartoucheclose}\isanewline
\ \ \ \ \isacommand{moreover}\isamarkupfalse%
\ \isacommand{from}\isamarkupfalse%
\ calculation\isanewline
\ \ \ \ \isacommand{have}\isamarkupfalse%
\ {\isachardoublequoteopen}arity{\isacharparenleft}{\kern0pt}{\isasymphi}{\isacharparenright}{\kern0pt}\ {\isasymle}\ {\isadigit{1}}\ {\isacharhash}{\kern0pt}{\isacharplus}{\kern0pt}\ length{\isacharparenleft}{\kern0pt}some{\isacharparenright}{\kern0pt}{\isachardoublequoteclose}\ \isanewline
\ \ \ \ \ \ \isacommand{using}\isamarkupfalse%
\ le{\isacharunderscore}{\kern0pt}trans{\isacharbrackleft}{\kern0pt}OF\ succpred{\isacharunderscore}{\kern0pt}leI{\isacharbrackright}{\kern0pt}\ succpred{\isacharunderscore}{\kern0pt}leI\ \isacommand{by}\isamarkupfalse%
\ simp\isanewline
\ \ \ \ \isacommand{moreover}\isamarkupfalse%
\ \isacommand{from}\isamarkupfalse%
\ calculation\ \isakeyword{and}\ asm\isanewline
\ \ \ \ \isacommand{have}\isamarkupfalse%
\ {\isachardoublequoteopen}separation{\isacharparenleft}{\kern0pt}{\isacharhash}{\kern0pt}{\isacharhash}{\kern0pt}M{\isacharcomma}{\kern0pt}\ {\isasymlambda}x{\isachardot}{\kern0pt}\ M{\isacharcomma}{\kern0pt}\ {\isacharbrackleft}{\kern0pt}x{\isacharbrackright}{\kern0pt}\ {\isacharat}{\kern0pt}\ some\ {\isasymTurnstile}\ {\isasymphi}{\isacharparenright}{\kern0pt}{\isachardoublequoteclose}\ \isacommand{by}\isamarkupfalse%
\ blast\isanewline
\ \ \ \ \isacommand{ultimately}\isamarkupfalse%
\isanewline
\ \ \ \ \isacommand{have}\isamarkupfalse%
\ {\isachardoublequoteopen}M{\isacharcomma}{\kern0pt}\ some\ {\isasymTurnstile}\ sep{\isacharunderscore}{\kern0pt}body{\isacharunderscore}{\kern0pt}fm{\isacharparenleft}{\kern0pt}{\isasymphi}{\isacharparenright}{\kern0pt}{\isachardoublequoteclose}\ \isanewline
\ \ \ \ \isacommand{using}\isamarkupfalse%
\ sats{\isacharunderscore}{\kern0pt}sep{\isacharunderscore}{\kern0pt}body{\isacharunderscore}{\kern0pt}fm{\isacharbrackleft}{\kern0pt}of\ {\isasymphi}\ {\isachardoublequoteopen}{\isacharbrackleft}{\kern0pt}{\isacharbrackright}{\kern0pt}{\isachardoublequoteclose}\ M\ some{\isacharbrackright}{\kern0pt}\isanewline
\ \ \ \ \ \ arity{\isacharunderscore}{\kern0pt}sats{\isacharunderscore}{\kern0pt}iff{\isacharbrackleft}{\kern0pt}of\ {\isasymphi}\ {\isacharunderscore}{\kern0pt}\ M\ {\isachardoublequoteopen}{\isacharbrackleft}{\kern0pt}{\isacharunderscore}{\kern0pt}{\isacharcomma}{\kern0pt}{\isacharunderscore}{\kern0pt}{\isacharbrackright}{\kern0pt}\ {\isacharat}{\kern0pt}\ some{\isachardoublequoteclose}{\isacharbrackright}{\kern0pt}\isanewline
\ \ \ \ \ \ strong{\isacharunderscore}{\kern0pt}replacement{\isacharunderscore}{\kern0pt}cong{\isacharbrackleft}{\kern0pt}of\ {\isachardoublequoteopen}{\isacharhash}{\kern0pt}{\isacharhash}{\kern0pt}M{\isachardoublequoteclose}\ {\isachardoublequoteopen}{\isasymlambda}x\ y{\isachardot}{\kern0pt}\ M{\isacharcomma}{\kern0pt}\ Cons{\isacharparenleft}{\kern0pt}x{\isacharcomma}{\kern0pt}\ Cons{\isacharparenleft}{\kern0pt}y{\isacharcomma}{\kern0pt}\ some\ {\isacharat}{\kern0pt}\ {\isacharunderscore}{\kern0pt}{\isacharparenright}{\kern0pt}{\isacharparenright}{\kern0pt}\ {\isasymTurnstile}\ {\isasymphi}{\isachardoublequoteclose}\ {\isacharunderscore}{\kern0pt}\ {\isacharbrackright}{\kern0pt}\isanewline
\ \ \ \ \isacommand{by}\isamarkupfalse%
\ simp\isanewline
\ \ \isacommand{{\isacharbraceright}{\kern0pt}}\isamarkupfalse%
\isanewline
\ \ \isacommand{with}\isamarkupfalse%
\ {\isacartoucheopen}{\isasymphi}{\isasymin}{\isacharunderscore}{\kern0pt}{\isacartoucheclose}\isanewline
\ \ \isacommand{show}\isamarkupfalse%
\ {\isachardoublequoteopen}M{\isacharcomma}{\kern0pt}\ {\isacharbrackleft}{\kern0pt}{\isacharbrackright}{\kern0pt}\ {\isasymTurnstile}\ ZF{\isacharunderscore}{\kern0pt}separation{\isacharunderscore}{\kern0pt}fm{\isacharparenleft}{\kern0pt}{\isasymphi}{\isacharparenright}{\kern0pt}{\isachardoublequoteclose}\isanewline
\ \ \ \ \isacommand{using}\isamarkupfalse%
\ sats{\isacharunderscore}{\kern0pt}nForall{\isacharbrackleft}{\kern0pt}of\ {\isachardoublequoteopen}sep{\isacharunderscore}{\kern0pt}body{\isacharunderscore}{\kern0pt}fm{\isacharparenleft}{\kern0pt}{\isasymphi}{\isacharparenright}{\kern0pt}{\isachardoublequoteclose}\ {\isacharquery}{\kern0pt}n{\isacharbrackright}{\kern0pt}\isanewline
\ \ \ \ \isacommand{unfolding}\isamarkupfalse%
\ ZF{\isacharunderscore}{\kern0pt}separation{\isacharunderscore}{\kern0pt}fm{\isacharunderscore}{\kern0pt}def\isanewline
\ \ \ \ \isacommand{by}\isamarkupfalse%
\ simp\isanewline
\isacommand{qed}\isamarkupfalse%
%
\endisatagproof
{\isafoldproof}%
%
\isadelimproof
%
\endisadelimproof
%
\isadelimdocument
%
\endisadelimdocument
%
\isatagdocument
%
\isamarkupsubsection{The Axiom of Replacement, internalized%
}
\isamarkuptrue%
%
\endisatagdocument
{\isafolddocument}%
%
\isadelimdocument
%
\endisadelimdocument
\isacommand{schematic{\isacharunderscore}{\kern0pt}goal}\isamarkupfalse%
\ sats{\isacharunderscore}{\kern0pt}univalent{\isacharunderscore}{\kern0pt}fm{\isacharunderscore}{\kern0pt}auto{\isacharcolon}{\kern0pt}\isanewline
\ \ \isakeyword{assumes}\ \isanewline
\ \ \ \ \isanewline
\ \ \ \ Q{\isacharunderscore}{\kern0pt}iff{\isacharunderscore}{\kern0pt}sats{\isacharcolon}{\kern0pt}{\isachardoublequoteopen}{\isasymAnd}x\ y\ z{\isachardot}{\kern0pt}\ x\ {\isasymin}\ A\ {\isasymLongrightarrow}\ y\ {\isasymin}\ A\ {\isasymLongrightarrow}\ z{\isasymin}A\ {\isasymLongrightarrow}\ \isanewline
\ \ \ \ \ \ \ \ \ \ \ \ \ \ \ \ \ Q{\isacharparenleft}{\kern0pt}x{\isacharcomma}{\kern0pt}z{\isacharparenright}{\kern0pt}\ {\isasymlongleftrightarrow}\ {\isacharparenleft}{\kern0pt}A{\isacharcomma}{\kern0pt}Cons{\isacharparenleft}{\kern0pt}z{\isacharcomma}{\kern0pt}Cons{\isacharparenleft}{\kern0pt}y{\isacharcomma}{\kern0pt}Cons{\isacharparenleft}{\kern0pt}x{\isacharcomma}{\kern0pt}env{\isacharparenright}{\kern0pt}{\isacharparenright}{\kern0pt}{\isacharparenright}{\kern0pt}\ {\isasymTurnstile}\ Q{\isadigit{1}}{\isacharunderscore}{\kern0pt}fm{\isacharparenright}{\kern0pt}{\isachardoublequoteclose}\isanewline
\ \ \ \ \ \ \ {\isachardoublequoteopen}{\isasymAnd}x\ y\ z{\isachardot}{\kern0pt}\ x\ {\isasymin}\ A\ {\isasymLongrightarrow}\ y\ {\isasymin}\ A\ {\isasymLongrightarrow}\ z{\isasymin}A\ {\isasymLongrightarrow}\ \isanewline
\ \ \ \ \ \ \ \ \ \ \ \ \ \ \ \ \ Q{\isacharparenleft}{\kern0pt}x{\isacharcomma}{\kern0pt}y{\isacharparenright}{\kern0pt}\ {\isasymlongleftrightarrow}\ {\isacharparenleft}{\kern0pt}A{\isacharcomma}{\kern0pt}Cons{\isacharparenleft}{\kern0pt}z{\isacharcomma}{\kern0pt}Cons{\isacharparenleft}{\kern0pt}y{\isacharcomma}{\kern0pt}Cons{\isacharparenleft}{\kern0pt}x{\isacharcomma}{\kern0pt}env{\isacharparenright}{\kern0pt}{\isacharparenright}{\kern0pt}{\isacharparenright}{\kern0pt}\ {\isasymTurnstile}\ Q{\isadigit{2}}{\isacharunderscore}{\kern0pt}fm{\isacharparenright}{\kern0pt}{\isachardoublequoteclose}\isanewline
\ \ \ \ \isakeyword{and}\ \isanewline
\ \ \ \ asms{\isacharcolon}{\kern0pt}\ {\isachardoublequoteopen}nth{\isacharparenleft}{\kern0pt}i{\isacharcomma}{\kern0pt}env{\isacharparenright}{\kern0pt}\ {\isacharequal}{\kern0pt}\ B{\isachardoublequoteclose}\ {\isachardoublequoteopen}i\ {\isasymin}\ nat{\isachardoublequoteclose}\ {\isachardoublequoteopen}env\ {\isasymin}\ list{\isacharparenleft}{\kern0pt}A{\isacharparenright}{\kern0pt}{\isachardoublequoteclose}\isanewline
\ \ \isakeyword{shows}\isanewline
\ \ \ \ {\isachardoublequoteopen}univalent{\isacharparenleft}{\kern0pt}{\isacharhash}{\kern0pt}{\isacharhash}{\kern0pt}A{\isacharcomma}{\kern0pt}B{\isacharcomma}{\kern0pt}Q{\isacharparenright}{\kern0pt}\ {\isasymlongleftrightarrow}\ A{\isacharcomma}{\kern0pt}env\ {\isasymTurnstile}\ {\isacharquery}{\kern0pt}ufm{\isacharparenleft}{\kern0pt}i{\isacharparenright}{\kern0pt}{\isachardoublequoteclose}\isanewline
%
\isadelimproof
\ \ %
\endisadelimproof
%
\isatagproof
\isacommand{unfolding}\isamarkupfalse%
\ univalent{\isacharunderscore}{\kern0pt}def\ \isanewline
\ \ \isacommand{by}\isamarkupfalse%
\ {\isacharparenleft}{\kern0pt}insert\ asms{\isacharsemicolon}{\kern0pt}\ {\isacharparenleft}{\kern0pt}rule\ sep{\isacharunderscore}{\kern0pt}rules\ Q{\isacharunderscore}{\kern0pt}iff{\isacharunderscore}{\kern0pt}sats\ {\isacharbar}{\kern0pt}\ simp{\isacharparenright}{\kern0pt}{\isacharplus}{\kern0pt}{\isacharparenright}{\kern0pt}%
\endisatagproof
{\isafoldproof}%
%
\isadelimproof
\isanewline
%
\endisadelimproof
%
\isadelimML
\ \ \isanewline
%
\endisadelimML
%
\isatagML
\isacommand{synthesize{\isacharunderscore}{\kern0pt}notc}\isamarkupfalse%
\ {\isachardoublequoteopen}univalent{\isacharunderscore}{\kern0pt}fm{\isachardoublequoteclose}\ \isakeyword{from{\isacharunderscore}{\kern0pt}schematic}\ sats{\isacharunderscore}{\kern0pt}univalent{\isacharunderscore}{\kern0pt}fm{\isacharunderscore}{\kern0pt}auto%
\endisatagML
{\isafoldML}%
%
\isadelimML
\isanewline
%
\endisadelimML
\isanewline
\isacommand{lemma}\isamarkupfalse%
\ univalent{\isacharunderscore}{\kern0pt}fm{\isacharunderscore}{\kern0pt}type\ {\isacharbrackleft}{\kern0pt}TC{\isacharbrackright}{\kern0pt}{\isacharcolon}{\kern0pt}\ {\isachardoublequoteopen}q{\isadigit{1}}{\isasymin}\ formula\ {\isasymLongrightarrow}\ q{\isadigit{2}}{\isasymin}formula\ {\isasymLongrightarrow}\ i{\isasymin}nat\ {\isasymLongrightarrow}\ \isanewline
\ \ univalent{\isacharunderscore}{\kern0pt}fm{\isacharparenleft}{\kern0pt}q{\isadigit{2}}{\isacharcomma}{\kern0pt}q{\isadigit{1}}{\isacharcomma}{\kern0pt}i{\isacharparenright}{\kern0pt}\ {\isasymin}formula{\isachardoublequoteclose}\isanewline
%
\isadelimproof
\ \ %
\endisadelimproof
%
\isatagproof
\isacommand{by}\isamarkupfalse%
\ {\isacharparenleft}{\kern0pt}simp\ add{\isacharcolon}{\kern0pt}univalent{\isacharunderscore}{\kern0pt}fm{\isacharunderscore}{\kern0pt}def{\isacharparenright}{\kern0pt}%
\endisatagproof
{\isafoldproof}%
%
\isadelimproof
\isanewline
%
\endisadelimproof
\isanewline
\isacommand{lemma}\isamarkupfalse%
\ sats{\isacharunderscore}{\kern0pt}univalent{\isacharunderscore}{\kern0pt}fm\ {\isacharcolon}{\kern0pt}\isanewline
\ \ \isakeyword{assumes}\isanewline
\ \ \ \ Q{\isacharunderscore}{\kern0pt}iff{\isacharunderscore}{\kern0pt}sats{\isacharcolon}{\kern0pt}{\isachardoublequoteopen}{\isasymAnd}x\ y\ z{\isachardot}{\kern0pt}\ x\ {\isasymin}\ A\ {\isasymLongrightarrow}\ y\ {\isasymin}\ A\ {\isasymLongrightarrow}\ z{\isasymin}A\ {\isasymLongrightarrow}\ \isanewline
\ \ \ \ \ \ \ \ \ \ \ \ \ \ \ \ \ Q{\isacharparenleft}{\kern0pt}x{\isacharcomma}{\kern0pt}z{\isacharparenright}{\kern0pt}\ {\isasymlongleftrightarrow}\ {\isacharparenleft}{\kern0pt}A{\isacharcomma}{\kern0pt}Cons{\isacharparenleft}{\kern0pt}z{\isacharcomma}{\kern0pt}Cons{\isacharparenleft}{\kern0pt}y{\isacharcomma}{\kern0pt}Cons{\isacharparenleft}{\kern0pt}x{\isacharcomma}{\kern0pt}env{\isacharparenright}{\kern0pt}{\isacharparenright}{\kern0pt}{\isacharparenright}{\kern0pt}\ {\isasymTurnstile}\ Q{\isadigit{1}}{\isacharunderscore}{\kern0pt}fm{\isacharparenright}{\kern0pt}{\isachardoublequoteclose}\isanewline
\ \ \ \ \ \ \ {\isachardoublequoteopen}{\isasymAnd}x\ y\ z{\isachardot}{\kern0pt}\ x\ {\isasymin}\ A\ {\isasymLongrightarrow}\ y\ {\isasymin}\ A\ {\isasymLongrightarrow}\ z{\isasymin}A\ {\isasymLongrightarrow}\ \isanewline
\ \ \ \ \ \ \ \ \ \ \ \ \ \ \ \ \ Q{\isacharparenleft}{\kern0pt}x{\isacharcomma}{\kern0pt}y{\isacharparenright}{\kern0pt}\ {\isasymlongleftrightarrow}\ {\isacharparenleft}{\kern0pt}A{\isacharcomma}{\kern0pt}Cons{\isacharparenleft}{\kern0pt}z{\isacharcomma}{\kern0pt}Cons{\isacharparenleft}{\kern0pt}y{\isacharcomma}{\kern0pt}Cons{\isacharparenleft}{\kern0pt}x{\isacharcomma}{\kern0pt}env{\isacharparenright}{\kern0pt}{\isacharparenright}{\kern0pt}{\isacharparenright}{\kern0pt}\ {\isasymTurnstile}\ Q{\isadigit{2}}{\isacharunderscore}{\kern0pt}fm{\isacharparenright}{\kern0pt}{\isachardoublequoteclose}\isanewline
\ \ \ \ \isakeyword{and}\ \isanewline
\ \ \ \ asms{\isacharcolon}{\kern0pt}\ {\isachardoublequoteopen}nth{\isacharparenleft}{\kern0pt}i{\isacharcomma}{\kern0pt}env{\isacharparenright}{\kern0pt}\ {\isacharequal}{\kern0pt}\ B{\isachardoublequoteclose}\ {\isachardoublequoteopen}i\ {\isasymin}\ nat{\isachardoublequoteclose}\ {\isachardoublequoteopen}env\ {\isasymin}\ list{\isacharparenleft}{\kern0pt}A{\isacharparenright}{\kern0pt}{\isachardoublequoteclose}\isanewline
\ \ \isakeyword{shows}\isanewline
\ \ \ \ {\isachardoublequoteopen}A{\isacharcomma}{\kern0pt}env\ {\isasymTurnstile}\ univalent{\isacharunderscore}{\kern0pt}fm{\isacharparenleft}{\kern0pt}Q{\isadigit{1}}{\isacharunderscore}{\kern0pt}fm{\isacharcomma}{\kern0pt}Q{\isadigit{2}}{\isacharunderscore}{\kern0pt}fm{\isacharcomma}{\kern0pt}i{\isacharparenright}{\kern0pt}\ {\isasymlongleftrightarrow}\ univalent{\isacharparenleft}{\kern0pt}{\isacharhash}{\kern0pt}{\isacharhash}{\kern0pt}A{\isacharcomma}{\kern0pt}B{\isacharcomma}{\kern0pt}Q{\isacharparenright}{\kern0pt}{\isachardoublequoteclose}\isanewline
%
\isadelimproof
\ \ %
\endisadelimproof
%
\isatagproof
\isacommand{unfolding}\isamarkupfalse%
\ univalent{\isacharunderscore}{\kern0pt}fm{\isacharunderscore}{\kern0pt}def\ \isacommand{using}\isamarkupfalse%
\ asms\ sats{\isacharunderscore}{\kern0pt}univalent{\isacharunderscore}{\kern0pt}fm{\isacharunderscore}{\kern0pt}auto{\isacharbrackleft}{\kern0pt}OF\ Q{\isacharunderscore}{\kern0pt}iff{\isacharunderscore}{\kern0pt}sats{\isacharbrackright}{\kern0pt}\ \isacommand{by}\isamarkupfalse%
\ simp%
\endisatagproof
{\isafoldproof}%
%
\isadelimproof
\isanewline
%
\endisadelimproof
\isanewline
\isacommand{definition}\isamarkupfalse%
\isanewline
\ \ swap{\isacharunderscore}{\kern0pt}vars\ {\isacharcolon}{\kern0pt}{\isacharcolon}{\kern0pt}\ {\isachardoublequoteopen}i{\isasymRightarrow}i{\isachardoublequoteclose}\ \isakeyword{where}\isanewline
\ \ {\isachardoublequoteopen}swap{\isacharunderscore}{\kern0pt}vars{\isacharparenleft}{\kern0pt}{\isasymphi}{\isacharparenright}{\kern0pt}\ {\isasymequiv}\ \isanewline
\ \ \ \ \ \ Exists{\isacharparenleft}{\kern0pt}Exists{\isacharparenleft}{\kern0pt}And{\isacharparenleft}{\kern0pt}Equal{\isacharparenleft}{\kern0pt}{\isadigit{0}}{\isacharcomma}{\kern0pt}{\isadigit{3}}{\isacharparenright}{\kern0pt}{\isacharcomma}{\kern0pt}And{\isacharparenleft}{\kern0pt}Equal{\isacharparenleft}{\kern0pt}{\isadigit{1}}{\isacharcomma}{\kern0pt}{\isadigit{2}}{\isacharparenright}{\kern0pt}{\isacharcomma}{\kern0pt}iterates{\isacharparenleft}{\kern0pt}{\isasymlambda}p{\isachardot}{\kern0pt}\ incr{\isacharunderscore}{\kern0pt}bv{\isacharparenleft}{\kern0pt}p{\isacharparenright}{\kern0pt}{\isacharbackquote}{\kern0pt}{\isadigit{2}}\ {\isacharcomma}{\kern0pt}\ {\isadigit{2}}{\isacharcomma}{\kern0pt}\ {\isasymphi}{\isacharparenright}{\kern0pt}{\isacharparenright}{\kern0pt}{\isacharparenright}{\kern0pt}{\isacharparenright}{\kern0pt}{\isacharparenright}{\kern0pt}{\isachardoublequoteclose}\ \isanewline
\isanewline
\isacommand{lemma}\isamarkupfalse%
\ swap{\isacharunderscore}{\kern0pt}vars{\isacharunderscore}{\kern0pt}type{\isacharbrackleft}{\kern0pt}TC{\isacharbrackright}{\kern0pt}\ {\isacharcolon}{\kern0pt}\isanewline
\ \ {\isachardoublequoteopen}{\isasymphi}{\isasymin}formula\ {\isasymLongrightarrow}\ swap{\isacharunderscore}{\kern0pt}vars{\isacharparenleft}{\kern0pt}{\isasymphi}{\isacharparenright}{\kern0pt}\ {\isasymin}formula{\isachardoublequoteclose}\ \isanewline
%
\isadelimproof
\ \ %
\endisadelimproof
%
\isatagproof
\isacommand{unfolding}\isamarkupfalse%
\ swap{\isacharunderscore}{\kern0pt}vars{\isacharunderscore}{\kern0pt}def\ \isacommand{by}\isamarkupfalse%
\ simp%
\endisatagproof
{\isafoldproof}%
%
\isadelimproof
\isanewline
%
\endisadelimproof
\isanewline
\isacommand{lemma}\isamarkupfalse%
\ sats{\isacharunderscore}{\kern0pt}swap{\isacharunderscore}{\kern0pt}vars\ {\isacharcolon}{\kern0pt}\isanewline
\ \ {\isachardoublequoteopen}{\isacharbrackleft}{\kern0pt}x{\isacharcomma}{\kern0pt}y{\isacharbrackright}{\kern0pt}\ {\isacharat}{\kern0pt}\ env\ {\isasymin}\ list{\isacharparenleft}{\kern0pt}M{\isacharparenright}{\kern0pt}\ {\isasymLongrightarrow}\ {\isasymphi}{\isasymin}formula\ {\isasymLongrightarrow}\ \isanewline
\ \ \ \ M{\isacharcomma}{\kern0pt}\ {\isacharbrackleft}{\kern0pt}x{\isacharcomma}{\kern0pt}y{\isacharbrackright}{\kern0pt}\ {\isacharat}{\kern0pt}\ env\ {\isasymTurnstile}\ swap{\isacharunderscore}{\kern0pt}vars{\isacharparenleft}{\kern0pt}{\isasymphi}{\isacharparenright}{\kern0pt}{\isasymlongleftrightarrow}\ M{\isacharcomma}{\kern0pt}{\isacharbrackleft}{\kern0pt}y{\isacharcomma}{\kern0pt}x{\isacharbrackright}{\kern0pt}\ {\isacharat}{\kern0pt}\ env\ {\isasymTurnstile}\ {\isasymphi}{\isachardoublequoteclose}\isanewline
%
\isadelimproof
\ \ %
\endisadelimproof
%
\isatagproof
\isacommand{unfolding}\isamarkupfalse%
\ swap{\isacharunderscore}{\kern0pt}vars{\isacharunderscore}{\kern0pt}def\isanewline
\ \ \isacommand{using}\isamarkupfalse%
\ sats{\isacharunderscore}{\kern0pt}incr{\isacharunderscore}{\kern0pt}bv{\isacharunderscore}{\kern0pt}iff\ {\isacharbrackleft}{\kern0pt}of\ {\isacharunderscore}{\kern0pt}\ {\isacharunderscore}{\kern0pt}\ M\ {\isacharunderscore}{\kern0pt}\ {\isachardoublequoteopen}{\isacharbrackleft}{\kern0pt}y{\isacharcomma}{\kern0pt}x{\isacharbrackright}{\kern0pt}{\isachardoublequoteclose}{\isacharbrackright}{\kern0pt}\ \isacommand{by}\isamarkupfalse%
\ simp%
\endisatagproof
{\isafoldproof}%
%
\isadelimproof
\isanewline
%
\endisadelimproof
\isanewline
\isacommand{definition}\isamarkupfalse%
\isanewline
\ \ univalent{\isacharunderscore}{\kern0pt}Q{\isadigit{1}}\ {\isacharcolon}{\kern0pt}{\isacharcolon}{\kern0pt}\ {\isachardoublequoteopen}i\ {\isasymRightarrow}\ i{\isachardoublequoteclose}\ \isakeyword{where}\isanewline
\ \ {\isachardoublequoteopen}univalent{\isacharunderscore}{\kern0pt}Q{\isadigit{1}}{\isacharparenleft}{\kern0pt}{\isasymphi}{\isacharparenright}{\kern0pt}\ {\isasymequiv}\ incr{\isacharunderscore}{\kern0pt}bv{\isadigit{1}}{\isacharparenleft}{\kern0pt}swap{\isacharunderscore}{\kern0pt}vars{\isacharparenleft}{\kern0pt}{\isasymphi}{\isacharparenright}{\kern0pt}{\isacharparenright}{\kern0pt}{\isachardoublequoteclose}\isanewline
\isanewline
\isacommand{definition}\isamarkupfalse%
\isanewline
\ \ univalent{\isacharunderscore}{\kern0pt}Q{\isadigit{2}}\ {\isacharcolon}{\kern0pt}{\isacharcolon}{\kern0pt}\ {\isachardoublequoteopen}i\ {\isasymRightarrow}\ i{\isachardoublequoteclose}\ \isakeyword{where}\isanewline
\ \ {\isachardoublequoteopen}univalent{\isacharunderscore}{\kern0pt}Q{\isadigit{2}}{\isacharparenleft}{\kern0pt}{\isasymphi}{\isacharparenright}{\kern0pt}\ {\isasymequiv}\ incr{\isacharunderscore}{\kern0pt}bv{\isacharparenleft}{\kern0pt}swap{\isacharunderscore}{\kern0pt}vars{\isacharparenleft}{\kern0pt}{\isasymphi}{\isacharparenright}{\kern0pt}{\isacharparenright}{\kern0pt}{\isacharbackquote}{\kern0pt}{\isadigit{0}}{\isachardoublequoteclose}\isanewline
\isanewline
\isacommand{lemma}\isamarkupfalse%
\ univalent{\isacharunderscore}{\kern0pt}Qs{\isacharunderscore}{\kern0pt}type\ {\isacharbrackleft}{\kern0pt}TC{\isacharbrackright}{\kern0pt}{\isacharcolon}{\kern0pt}\ \isanewline
\ \ \isakeyword{assumes}\ {\isachardoublequoteopen}{\isasymphi}{\isasymin}formula{\isachardoublequoteclose}\isanewline
\ \ \isakeyword{shows}\ {\isachardoublequoteopen}univalent{\isacharunderscore}{\kern0pt}Q{\isadigit{1}}{\isacharparenleft}{\kern0pt}{\isasymphi}{\isacharparenright}{\kern0pt}\ {\isasymin}\ formula{\isachardoublequoteclose}\ {\isachardoublequoteopen}univalent{\isacharunderscore}{\kern0pt}Q{\isadigit{2}}{\isacharparenleft}{\kern0pt}{\isasymphi}{\isacharparenright}{\kern0pt}\ {\isasymin}\ formula{\isachardoublequoteclose}\isanewline
%
\isadelimproof
\ \ %
\endisadelimproof
%
\isatagproof
\isacommand{unfolding}\isamarkupfalse%
\ univalent{\isacharunderscore}{\kern0pt}Q{\isadigit{1}}{\isacharunderscore}{\kern0pt}def\ univalent{\isacharunderscore}{\kern0pt}Q{\isadigit{2}}{\isacharunderscore}{\kern0pt}def\ \isacommand{using}\isamarkupfalse%
\ assms\ \isacommand{by}\isamarkupfalse%
\ simp{\isacharunderscore}{\kern0pt}all%
\endisatagproof
{\isafoldproof}%
%
\isadelimproof
\isanewline
%
\endisadelimproof
\isanewline
\isacommand{lemma}\isamarkupfalse%
\ sats{\isacharunderscore}{\kern0pt}univalent{\isacharunderscore}{\kern0pt}fm{\isacharunderscore}{\kern0pt}assm{\isacharcolon}{\kern0pt}\isanewline
\ \ \isakeyword{assumes}\ \isanewline
\ \ \ \ {\isachardoublequoteopen}x\ {\isasymin}\ A{\isachardoublequoteclose}\ {\isachardoublequoteopen}y\ {\isasymin}\ A{\isachardoublequoteclose}\ {\isachardoublequoteopen}z{\isasymin}A{\isachardoublequoteclose}\ {\isachardoublequoteopen}env{\isasymin}\ list{\isacharparenleft}{\kern0pt}A{\isacharparenright}{\kern0pt}{\isachardoublequoteclose}\ {\isachardoublequoteopen}{\isasymphi}\ {\isasymin}\ formula{\isachardoublequoteclose}\isanewline
\ \ \isakeyword{shows}\ \isanewline
\ \ \ \ {\isachardoublequoteopen}{\isacharparenleft}{\kern0pt}A{\isacharcomma}{\kern0pt}\ {\isacharparenleft}{\kern0pt}{\isacharbrackleft}{\kern0pt}x{\isacharcomma}{\kern0pt}z{\isacharbrackright}{\kern0pt}\ {\isacharat}{\kern0pt}\ env{\isacharparenright}{\kern0pt}\ {\isasymTurnstile}\ {\isasymphi}{\isacharparenright}{\kern0pt}\ {\isasymlongleftrightarrow}\ {\isacharparenleft}{\kern0pt}A{\isacharcomma}{\kern0pt}\ Cons{\isacharparenleft}{\kern0pt}z{\isacharcomma}{\kern0pt}Cons{\isacharparenleft}{\kern0pt}y{\isacharcomma}{\kern0pt}Cons{\isacharparenleft}{\kern0pt}x{\isacharcomma}{\kern0pt}env{\isacharparenright}{\kern0pt}{\isacharparenright}{\kern0pt}{\isacharparenright}{\kern0pt}\ {\isasymTurnstile}\ {\isacharparenleft}{\kern0pt}univalent{\isacharunderscore}{\kern0pt}Q{\isadigit{1}}{\isacharparenleft}{\kern0pt}{\isasymphi}{\isacharparenright}{\kern0pt}{\isacharparenright}{\kern0pt}{\isacharparenright}{\kern0pt}{\isachardoublequoteclose}\isanewline
\ \ \ \ {\isachardoublequoteopen}{\isacharparenleft}{\kern0pt}A{\isacharcomma}{\kern0pt}\ {\isacharparenleft}{\kern0pt}{\isacharbrackleft}{\kern0pt}x{\isacharcomma}{\kern0pt}y{\isacharbrackright}{\kern0pt}\ {\isacharat}{\kern0pt}\ env{\isacharparenright}{\kern0pt}\ {\isasymTurnstile}\ {\isasymphi}{\isacharparenright}{\kern0pt}\ {\isasymlongleftrightarrow}\ {\isacharparenleft}{\kern0pt}A{\isacharcomma}{\kern0pt}\ Cons{\isacharparenleft}{\kern0pt}z{\isacharcomma}{\kern0pt}Cons{\isacharparenleft}{\kern0pt}y{\isacharcomma}{\kern0pt}Cons{\isacharparenleft}{\kern0pt}x{\isacharcomma}{\kern0pt}env{\isacharparenright}{\kern0pt}{\isacharparenright}{\kern0pt}{\isacharparenright}{\kern0pt}\ {\isasymTurnstile}\ {\isacharparenleft}{\kern0pt}univalent{\isacharunderscore}{\kern0pt}Q{\isadigit{2}}{\isacharparenleft}{\kern0pt}{\isasymphi}{\isacharparenright}{\kern0pt}{\isacharparenright}{\kern0pt}{\isacharparenright}{\kern0pt}{\isachardoublequoteclose}\isanewline
%
\isadelimproof
\ \ %
\endisadelimproof
%
\isatagproof
\isacommand{unfolding}\isamarkupfalse%
\ univalent{\isacharunderscore}{\kern0pt}Q{\isadigit{1}}{\isacharunderscore}{\kern0pt}def\ univalent{\isacharunderscore}{\kern0pt}Q{\isadigit{2}}{\isacharunderscore}{\kern0pt}def\isanewline
\ \ \isacommand{using}\isamarkupfalse%
\ \isanewline
\ \ \ \ sats{\isacharunderscore}{\kern0pt}incr{\isacharunderscore}{\kern0pt}bv{\isacharunderscore}{\kern0pt}iff{\isacharbrackleft}{\kern0pt}of\ {\isacharunderscore}{\kern0pt}\ {\isacharunderscore}{\kern0pt}\ A\ {\isacharunderscore}{\kern0pt}\ {\isachardoublequoteopen}{\isacharbrackleft}{\kern0pt}{\isacharbrackright}{\kern0pt}{\isachardoublequoteclose}{\isacharbrackright}{\kern0pt}\ %
\isamarkupcmt{simplifies iterates of \isa{{\isasymlambda}x{\isachardot}{\kern0pt}\ incr{\isacharunderscore}{\kern0pt}bv{\isacharparenleft}{\kern0pt}x{\isacharparenright}{\kern0pt}\ {\isacharbackquote}{\kern0pt}\ {\isadigit{0}}}%
}\isanewline
\ \ \ \ sats{\isacharunderscore}{\kern0pt}incr{\isacharunderscore}{\kern0pt}bv{\isadigit{1}}{\isacharunderscore}{\kern0pt}iff{\isacharbrackleft}{\kern0pt}of\ {\isacharunderscore}{\kern0pt}\ {\isachardoublequoteopen}Cons{\isacharparenleft}{\kern0pt}x{\isacharcomma}{\kern0pt}env{\isacharparenright}{\kern0pt}{\isachardoublequoteclose}\ A\ z\ y{\isacharbrackright}{\kern0pt}\ \isanewline
\ \ \ \ sats{\isacharunderscore}{\kern0pt}swap{\isacharunderscore}{\kern0pt}vars\ \ assms\ \isanewline
\ \ \ \isacommand{by}\isamarkupfalse%
\ simp{\isacharunderscore}{\kern0pt}all%
\endisatagproof
{\isafoldproof}%
%
\isadelimproof
\isanewline
%
\endisadelimproof
\isanewline
\isacommand{definition}\isamarkupfalse%
\isanewline
\ \ rep{\isacharunderscore}{\kern0pt}body{\isacharunderscore}{\kern0pt}fm\ {\isacharcolon}{\kern0pt}{\isacharcolon}{\kern0pt}\ {\isachardoublequoteopen}i\ {\isasymRightarrow}\ i{\isachardoublequoteclose}\ \isakeyword{where}\isanewline
\ \ {\isachardoublequoteopen}rep{\isacharunderscore}{\kern0pt}body{\isacharunderscore}{\kern0pt}fm{\isacharparenleft}{\kern0pt}p{\isacharparenright}{\kern0pt}\ {\isasymequiv}\ Forall{\isacharparenleft}{\kern0pt}Implies{\isacharparenleft}{\kern0pt}\isanewline
\ \ \ \ \ \ \ \ univalent{\isacharunderscore}{\kern0pt}fm{\isacharparenleft}{\kern0pt}univalent{\isacharunderscore}{\kern0pt}Q{\isadigit{1}}{\isacharparenleft}{\kern0pt}incr{\isacharunderscore}{\kern0pt}bv{\isacharparenleft}{\kern0pt}p{\isacharparenright}{\kern0pt}{\isacharbackquote}{\kern0pt}{\isadigit{2}}{\isacharparenright}{\kern0pt}{\isacharcomma}{\kern0pt}univalent{\isacharunderscore}{\kern0pt}Q{\isadigit{2}}{\isacharparenleft}{\kern0pt}incr{\isacharunderscore}{\kern0pt}bv{\isacharparenleft}{\kern0pt}p{\isacharparenright}{\kern0pt}{\isacharbackquote}{\kern0pt}{\isadigit{2}}{\isacharparenright}{\kern0pt}{\isacharcomma}{\kern0pt}{\isadigit{0}}{\isacharparenright}{\kern0pt}{\isacharcomma}{\kern0pt}\isanewline
\ \ \ \ \ \ \ \ Exists{\isacharparenleft}{\kern0pt}Forall{\isacharparenleft}{\kern0pt}\isanewline
\ \ \ \ \ \ \ \ \ \ Iff{\isacharparenleft}{\kern0pt}Member{\isacharparenleft}{\kern0pt}{\isadigit{0}}{\isacharcomma}{\kern0pt}{\isadigit{1}}{\isacharparenright}{\kern0pt}{\isacharcomma}{\kern0pt}Exists{\isacharparenleft}{\kern0pt}And{\isacharparenleft}{\kern0pt}Member{\isacharparenleft}{\kern0pt}{\isadigit{0}}{\isacharcomma}{\kern0pt}{\isadigit{3}}{\isacharparenright}{\kern0pt}{\isacharcomma}{\kern0pt}incr{\isacharunderscore}{\kern0pt}bv{\isacharparenleft}{\kern0pt}incr{\isacharunderscore}{\kern0pt}bv{\isacharparenleft}{\kern0pt}p{\isacharparenright}{\kern0pt}{\isacharbackquote}{\kern0pt}{\isadigit{2}}{\isacharparenright}{\kern0pt}{\isacharbackquote}{\kern0pt}{\isadigit{2}}{\isacharparenright}{\kern0pt}{\isacharparenright}{\kern0pt}{\isacharparenright}{\kern0pt}{\isacharparenright}{\kern0pt}{\isacharparenright}{\kern0pt}{\isacharparenright}{\kern0pt}{\isacharparenright}{\kern0pt}{\isachardoublequoteclose}\isanewline
\isanewline
\isacommand{lemma}\isamarkupfalse%
\ rep{\isacharunderscore}{\kern0pt}body{\isacharunderscore}{\kern0pt}fm{\isacharunderscore}{\kern0pt}type\ {\isacharbrackleft}{\kern0pt}TC{\isacharbrackright}{\kern0pt}{\isacharcolon}{\kern0pt}\ {\isachardoublequoteopen}p\ {\isasymin}\ formula\ {\isasymLongrightarrow}\ rep{\isacharunderscore}{\kern0pt}body{\isacharunderscore}{\kern0pt}fm{\isacharparenleft}{\kern0pt}p{\isacharparenright}{\kern0pt}\ {\isasymin}\ formula{\isachardoublequoteclose}\isanewline
%
\isadelimproof
\ \ %
\endisadelimproof
%
\isatagproof
\isacommand{by}\isamarkupfalse%
\ {\isacharparenleft}{\kern0pt}simp\ add{\isacharcolon}{\kern0pt}\ rep{\isacharunderscore}{\kern0pt}body{\isacharunderscore}{\kern0pt}fm{\isacharunderscore}{\kern0pt}def{\isacharparenright}{\kern0pt}%
\endisatagproof
{\isafoldproof}%
%
\isadelimproof
\isanewline
%
\endisadelimproof
\isanewline
\isacommand{lemmas}\isamarkupfalse%
\ ZF{\isacharunderscore}{\kern0pt}replacement{\isacharunderscore}{\kern0pt}simps\ {\isacharequal}{\kern0pt}\ formula{\isacharunderscore}{\kern0pt}add{\isacharunderscore}{\kern0pt}params{\isadigit{1}}{\isacharbrackleft}{\kern0pt}of\ {\isasymphi}\ {\isadigit{2}}\ {\isacharunderscore}{\kern0pt}\ M\ {\isachardoublequoteopen}{\isacharbrackleft}{\kern0pt}{\isacharunderscore}{\kern0pt}{\isacharcomma}{\kern0pt}{\isacharunderscore}{\kern0pt}{\isacharbrackright}{\kern0pt}{\isachardoublequoteclose}\ {\isacharbrackright}{\kern0pt}\isanewline
\ \ sats{\isacharunderscore}{\kern0pt}incr{\isacharunderscore}{\kern0pt}bv{\isacharunderscore}{\kern0pt}iff{\isacharbrackleft}{\kern0pt}of\ {\isacharunderscore}{\kern0pt}\ {\isacharunderscore}{\kern0pt}\ M\ {\isacharunderscore}{\kern0pt}\ {\isachardoublequoteopen}{\isacharbrackleft}{\kern0pt}{\isacharbrackright}{\kern0pt}{\isachardoublequoteclose}{\isacharbrackright}{\kern0pt}\ %
\isamarkupcmt{simplifies iterates of \isa{{\isasymlambda}x{\isachardot}{\kern0pt}\ incr{\isacharunderscore}{\kern0pt}bv{\isacharparenleft}{\kern0pt}x{\isacharparenright}{\kern0pt}\ {\isacharbackquote}{\kern0pt}\ {\isadigit{0}}}%
}\isanewline
\ \ sats{\isacharunderscore}{\kern0pt}incr{\isacharunderscore}{\kern0pt}bv{\isacharunderscore}{\kern0pt}iff{\isacharbrackleft}{\kern0pt}of\ {\isacharunderscore}{\kern0pt}\ {\isacharunderscore}{\kern0pt}\ M\ {\isacharunderscore}{\kern0pt}\ {\isachardoublequoteopen}{\isacharbrackleft}{\kern0pt}{\isacharunderscore}{\kern0pt}{\isacharcomma}{\kern0pt}{\isacharunderscore}{\kern0pt}{\isacharbrackright}{\kern0pt}{\isachardoublequoteclose}{\isacharbrackright}{\kern0pt}%
\isamarkupcmt{simplifies \isa{{\isasymlambda}x{\isachardot}{\kern0pt}\ incr{\isacharunderscore}{\kern0pt}bv{\isacharparenleft}{\kern0pt}x{\isacharparenright}{\kern0pt}\ {\isacharbackquote}{\kern0pt}\ {\isadigit{2}}}%
}\isanewline
\ \ sats{\isacharunderscore}{\kern0pt}incr{\isacharunderscore}{\kern0pt}bv{\isadigit{1}}{\isacharunderscore}{\kern0pt}iff{\isacharbrackleft}{\kern0pt}of\ {\isacharunderscore}{\kern0pt}\ {\isacharunderscore}{\kern0pt}\ M{\isacharbrackright}{\kern0pt}\ sats{\isacharunderscore}{\kern0pt}swap{\isacharunderscore}{\kern0pt}vars\ \isakeyword{for}\ {\isasymphi}\ M\isanewline
\isanewline
\isacommand{lemma}\isamarkupfalse%
\ sats{\isacharunderscore}{\kern0pt}rep{\isacharunderscore}{\kern0pt}body{\isacharunderscore}{\kern0pt}fm{\isacharcolon}{\kern0pt}\isanewline
\ \ \isakeyword{assumes}\isanewline
\ \ \ \ {\isachardoublequoteopen}{\isasymphi}\ {\isasymin}\ formula{\isachardoublequoteclose}\ {\isachardoublequoteopen}ms{\isasymin}list{\isacharparenleft}{\kern0pt}M{\isacharparenright}{\kern0pt}{\isachardoublequoteclose}\ {\isachardoublequoteopen}rest{\isasymin}list{\isacharparenleft}{\kern0pt}M{\isacharparenright}{\kern0pt}{\isachardoublequoteclose}\isanewline
\ \ \isakeyword{shows}\isanewline
\ \ \ \ {\isachardoublequoteopen}M{\isacharcomma}{\kern0pt}\ rest\ {\isacharat}{\kern0pt}\ ms\ {\isasymTurnstile}\ rep{\isacharunderscore}{\kern0pt}body{\isacharunderscore}{\kern0pt}fm{\isacharparenleft}{\kern0pt}{\isasymphi}{\isacharparenright}{\kern0pt}\ {\isasymlongleftrightarrow}\ \isanewline
\ \ \ \ \ strong{\isacharunderscore}{\kern0pt}replacement{\isacharparenleft}{\kern0pt}{\isacharhash}{\kern0pt}{\isacharhash}{\kern0pt}M{\isacharcomma}{\kern0pt}{\isasymlambda}x\ y{\isachardot}{\kern0pt}\ M{\isacharcomma}{\kern0pt}\ {\isacharbrackleft}{\kern0pt}x{\isacharcomma}{\kern0pt}y{\isacharbrackright}{\kern0pt}\ {\isacharat}{\kern0pt}\ rest\ {\isacharat}{\kern0pt}\ ms\ {\isasymTurnstile}\ {\isasymphi}{\isacharparenright}{\kern0pt}{\isachardoublequoteclose}\isanewline
%
\isadelimproof
\ \ %
\endisadelimproof
%
\isatagproof
\isacommand{using}\isamarkupfalse%
\ assms\ ZF{\isacharunderscore}{\kern0pt}replacement{\isacharunderscore}{\kern0pt}simps\ \isanewline
\ \ \isacommand{unfolding}\isamarkupfalse%
\ rep{\isacharunderscore}{\kern0pt}body{\isacharunderscore}{\kern0pt}fm{\isacharunderscore}{\kern0pt}def\ strong{\isacharunderscore}{\kern0pt}replacement{\isacharunderscore}{\kern0pt}def\ univalent{\isacharunderscore}{\kern0pt}def\isanewline
\ \ \isacommand{unfolding}\isamarkupfalse%
\ univalent{\isacharunderscore}{\kern0pt}fm{\isacharunderscore}{\kern0pt}def\ univalent{\isacharunderscore}{\kern0pt}Q{\isadigit{1}}{\isacharunderscore}{\kern0pt}def\ univalent{\isacharunderscore}{\kern0pt}Q{\isadigit{2}}{\isacharunderscore}{\kern0pt}def\isanewline
\ \ \isacommand{by}\isamarkupfalse%
\ simp%
\endisatagproof
{\isafoldproof}%
%
\isadelimproof
\isanewline
%
\endisadelimproof
\isanewline
\isacommand{definition}\isamarkupfalse%
\isanewline
\ \ ZF{\isacharunderscore}{\kern0pt}replacement{\isacharunderscore}{\kern0pt}fm\ {\isacharcolon}{\kern0pt}{\isacharcolon}{\kern0pt}\ {\isachardoublequoteopen}i\ {\isasymRightarrow}\ i{\isachardoublequoteclose}\ \isakeyword{where}\isanewline
\ \ {\isachardoublequoteopen}ZF{\isacharunderscore}{\kern0pt}replacement{\isacharunderscore}{\kern0pt}fm{\isacharparenleft}{\kern0pt}p{\isacharparenright}{\kern0pt}\ {\isasymequiv}\ Forall{\isacharcircum}{\kern0pt}{\isacharparenleft}{\kern0pt}pred{\isacharparenleft}{\kern0pt}pred{\isacharparenleft}{\kern0pt}arity{\isacharparenleft}{\kern0pt}p{\isacharparenright}{\kern0pt}{\isacharparenright}{\kern0pt}{\isacharparenright}{\kern0pt}{\isacharparenright}{\kern0pt}{\isacharparenleft}{\kern0pt}rep{\isacharunderscore}{\kern0pt}body{\isacharunderscore}{\kern0pt}fm{\isacharparenleft}{\kern0pt}p{\isacharparenright}{\kern0pt}{\isacharparenright}{\kern0pt}{\isachardoublequoteclose}\isanewline
\isanewline
\isacommand{lemma}\isamarkupfalse%
\ ZF{\isacharunderscore}{\kern0pt}replacement{\isacharunderscore}{\kern0pt}fm{\isacharunderscore}{\kern0pt}type\ {\isacharbrackleft}{\kern0pt}TC{\isacharbrackright}{\kern0pt}{\isacharcolon}{\kern0pt}\ {\isachardoublequoteopen}p\ {\isasymin}\ formula\ {\isasymLongrightarrow}\ ZF{\isacharunderscore}{\kern0pt}replacement{\isacharunderscore}{\kern0pt}fm{\isacharparenleft}{\kern0pt}p{\isacharparenright}{\kern0pt}\ {\isasymin}\ formula{\isachardoublequoteclose}\isanewline
%
\isadelimproof
\ \ %
\endisadelimproof
%
\isatagproof
\isacommand{by}\isamarkupfalse%
\ {\isacharparenleft}{\kern0pt}simp\ add{\isacharcolon}{\kern0pt}\ ZF{\isacharunderscore}{\kern0pt}replacement{\isacharunderscore}{\kern0pt}fm{\isacharunderscore}{\kern0pt}def{\isacharparenright}{\kern0pt}%
\endisatagproof
{\isafoldproof}%
%
\isadelimproof
\isanewline
%
\endisadelimproof
\isanewline
\isacommand{lemma}\isamarkupfalse%
\ sats{\isacharunderscore}{\kern0pt}ZF{\isacharunderscore}{\kern0pt}replacement{\isacharunderscore}{\kern0pt}fm{\isacharunderscore}{\kern0pt}iff{\isacharcolon}{\kern0pt}\isanewline
\ \ \isakeyword{assumes}\isanewline
\ \ \ \ {\isachardoublequoteopen}{\isasymphi}{\isasymin}formula{\isachardoublequoteclose}\isanewline
\ \ \isakeyword{shows}\isanewline
\ \ {\isachardoublequoteopen}{\isacharparenleft}{\kern0pt}M{\isacharcomma}{\kern0pt}\ {\isacharbrackleft}{\kern0pt}{\isacharbrackright}{\kern0pt}\ {\isasymTurnstile}\ {\isacharparenleft}{\kern0pt}ZF{\isacharunderscore}{\kern0pt}replacement{\isacharunderscore}{\kern0pt}fm{\isacharparenleft}{\kern0pt}{\isasymphi}{\isacharparenright}{\kern0pt}{\isacharparenright}{\kern0pt}{\isacharparenright}{\kern0pt}\isanewline
\ \ \ {\isasymlongleftrightarrow}\isanewline
\ \ \ {\isacharparenleft}{\kern0pt}{\isasymforall}env{\isasymin}list{\isacharparenleft}{\kern0pt}M{\isacharparenright}{\kern0pt}{\isachardot}{\kern0pt}\ arity{\isacharparenleft}{\kern0pt}{\isasymphi}{\isacharparenright}{\kern0pt}\ {\isasymle}\ {\isadigit{2}}\ {\isacharhash}{\kern0pt}{\isacharplus}{\kern0pt}\ length{\isacharparenleft}{\kern0pt}env{\isacharparenright}{\kern0pt}\ {\isasymlongrightarrow}\ \isanewline
\ \ \ \ \ \ strong{\isacharunderscore}{\kern0pt}replacement{\isacharparenleft}{\kern0pt}{\isacharhash}{\kern0pt}{\isacharhash}{\kern0pt}M{\isacharcomma}{\kern0pt}{\isasymlambda}x\ y{\isachardot}{\kern0pt}\ M{\isacharcomma}{\kern0pt}{\isacharbrackleft}{\kern0pt}x{\isacharcomma}{\kern0pt}y{\isacharbrackright}{\kern0pt}\ {\isacharat}{\kern0pt}\ env\ {\isasymTurnstile}\ {\isasymphi}{\isacharparenright}{\kern0pt}{\isacharparenright}{\kern0pt}{\isachardoublequoteclose}\isanewline
%
\isadelimproof
%
\endisadelimproof
%
\isatagproof
\isacommand{proof}\isamarkupfalse%
\ {\isacharparenleft}{\kern0pt}intro\ iffI\ ballI\ impI{\isacharparenright}{\kern0pt}\isanewline
\ \ \isacommand{let}\isamarkupfalse%
\ {\isacharquery}{\kern0pt}n{\isacharequal}{\kern0pt}{\isachardoublequoteopen}Arith{\isachardot}{\kern0pt}pred{\isacharparenleft}{\kern0pt}Arith{\isachardot}{\kern0pt}pred{\isacharparenleft}{\kern0pt}arity{\isacharparenleft}{\kern0pt}{\isasymphi}{\isacharparenright}{\kern0pt}{\isacharparenright}{\kern0pt}{\isacharparenright}{\kern0pt}{\isachardoublequoteclose}\isanewline
\ \ \isacommand{fix}\isamarkupfalse%
\ env\isanewline
\ \ \isacommand{assume}\isamarkupfalse%
\ {\isachardoublequoteopen}M{\isacharcomma}{\kern0pt}\ {\isacharbrackleft}{\kern0pt}{\isacharbrackright}{\kern0pt}\ {\isasymTurnstile}\ ZF{\isacharunderscore}{\kern0pt}replacement{\isacharunderscore}{\kern0pt}fm{\isacharparenleft}{\kern0pt}{\isasymphi}{\isacharparenright}{\kern0pt}{\isachardoublequoteclose}\ {\isachardoublequoteopen}arity{\isacharparenleft}{\kern0pt}{\isasymphi}{\isacharparenright}{\kern0pt}\ {\isasymle}\ {\isadigit{2}}\ {\isacharhash}{\kern0pt}{\isacharplus}{\kern0pt}\ length{\isacharparenleft}{\kern0pt}env{\isacharparenright}{\kern0pt}{\isachardoublequoteclose}\ {\isachardoublequoteopen}env{\isasymin}list{\isacharparenleft}{\kern0pt}M{\isacharparenright}{\kern0pt}{\isachardoublequoteclose}\isanewline
\ \ \isacommand{moreover}\isamarkupfalse%
\ \isacommand{from}\isamarkupfalse%
\ this\isanewline
\ \ \isacommand{have}\isamarkupfalse%
\ {\isachardoublequoteopen}arity{\isacharparenleft}{\kern0pt}{\isasymphi}{\isacharparenright}{\kern0pt}\ {\isasymle}\ succ{\isacharparenleft}{\kern0pt}succ{\isacharparenleft}{\kern0pt}length{\isacharparenleft}{\kern0pt}env{\isacharparenright}{\kern0pt}{\isacharparenright}{\kern0pt}{\isacharparenright}{\kern0pt}{\isachardoublequoteclose}\ \isacommand{by}\isamarkupfalse%
\ {\isacharparenleft}{\kern0pt}simp{\isacharparenright}{\kern0pt}\isanewline
\ \ \isacommand{moreover}\isamarkupfalse%
\ \isacommand{from}\isamarkupfalse%
\ calculation\ \isanewline
\ \ \isacommand{have}\isamarkupfalse%
\ {\isachardoublequoteopen}pred{\isacharparenleft}{\kern0pt}arity{\isacharparenleft}{\kern0pt}{\isasymphi}{\isacharparenright}{\kern0pt}{\isacharparenright}{\kern0pt}\ {\isasymle}\ succ{\isacharparenleft}{\kern0pt}length{\isacharparenleft}{\kern0pt}env{\isacharparenright}{\kern0pt}{\isacharparenright}{\kern0pt}{\isachardoublequoteclose}\isanewline
\ \ \ \ \isacommand{using}\isamarkupfalse%
\ pred{\isacharunderscore}{\kern0pt}mono{\isacharbrackleft}{\kern0pt}OF\ {\isacharunderscore}{\kern0pt}\ {\isacartoucheopen}arity{\isacharparenleft}{\kern0pt}{\isasymphi}{\isacharparenright}{\kern0pt}{\isasymle}succ{\isacharparenleft}{\kern0pt}{\isacharunderscore}{\kern0pt}{\isacharparenright}{\kern0pt}{\isacartoucheclose}{\isacharbrackright}{\kern0pt}\ pred{\isacharunderscore}{\kern0pt}succ{\isacharunderscore}{\kern0pt}eq\ \isacommand{by}\isamarkupfalse%
\ simp\isanewline
\ \ \isacommand{moreover}\isamarkupfalse%
\ \isacommand{from}\isamarkupfalse%
\ calculation\isanewline
\ \ \isacommand{obtain}\isamarkupfalse%
\ some\ rest\ \isakeyword{where}\ {\isachardoublequoteopen}some{\isasymin}list{\isacharparenleft}{\kern0pt}M{\isacharparenright}{\kern0pt}{\isachardoublequoteclose}\ {\isachardoublequoteopen}rest{\isasymin}list{\isacharparenleft}{\kern0pt}M{\isacharparenright}{\kern0pt}{\isachardoublequoteclose}\ \isanewline
\ \ \ \ {\isachardoublequoteopen}env\ {\isacharequal}{\kern0pt}\ some\ {\isacharat}{\kern0pt}\ rest{\isachardoublequoteclose}\ {\isachardoublequoteopen}length{\isacharparenleft}{\kern0pt}some{\isacharparenright}{\kern0pt}\ {\isacharequal}{\kern0pt}\ Arith{\isachardot}{\kern0pt}pred{\isacharparenleft}{\kern0pt}Arith{\isachardot}{\kern0pt}pred{\isacharparenleft}{\kern0pt}arity{\isacharparenleft}{\kern0pt}{\isasymphi}{\isacharparenright}{\kern0pt}{\isacharparenright}{\kern0pt}{\isacharparenright}{\kern0pt}{\isachardoublequoteclose}\ \isanewline
\ \ \ \ \isacommand{using}\isamarkupfalse%
\ list{\isacharunderscore}{\kern0pt}split{\isacharbrackleft}{\kern0pt}OF\ {\isacartoucheopen}pred{\isacharparenleft}{\kern0pt}{\isacharunderscore}{\kern0pt}{\isacharparenright}{\kern0pt}\ {\isasymle}\ {\isacharunderscore}{\kern0pt}{\isacartoucheclose}\ {\isacartoucheopen}env{\isasymin}{\isacharunderscore}{\kern0pt}{\isacartoucheclose}{\isacharbrackright}{\kern0pt}\ \isacommand{by}\isamarkupfalse%
\ auto\isanewline
\ \ \isacommand{moreover}\isamarkupfalse%
\isanewline
\ \ \isacommand{note}\isamarkupfalse%
\ {\isacartoucheopen}{\isasymphi}{\isasymin}{\isacharunderscore}{\kern0pt}{\isacartoucheclose}\isanewline
\ \ \isacommand{moreover}\isamarkupfalse%
\ \isacommand{from}\isamarkupfalse%
\ this\isanewline
\ \ \isacommand{have}\isamarkupfalse%
\ {\isachardoublequoteopen}arity{\isacharparenleft}{\kern0pt}{\isasymphi}{\isacharparenright}{\kern0pt}\ {\isasymle}\ succ{\isacharparenleft}{\kern0pt}succ{\isacharparenleft}{\kern0pt}Arith{\isachardot}{\kern0pt}pred{\isacharparenleft}{\kern0pt}Arith{\isachardot}{\kern0pt}pred{\isacharparenleft}{\kern0pt}arity{\isacharparenleft}{\kern0pt}{\isasymphi}{\isacharparenright}{\kern0pt}{\isacharparenright}{\kern0pt}{\isacharparenright}{\kern0pt}{\isacharparenright}{\kern0pt}{\isacharparenright}{\kern0pt}{\isachardoublequoteclose}\isanewline
\ \ \ \ \isacommand{using}\isamarkupfalse%
\ le{\isacharunderscore}{\kern0pt}trans{\isacharbrackleft}{\kern0pt}OF\ succpred{\isacharunderscore}{\kern0pt}leI{\isacharbrackright}{\kern0pt}\ succpred{\isacharunderscore}{\kern0pt}leI\ \isacommand{by}\isamarkupfalse%
\ simp\isanewline
\ \ \isacommand{moreover}\isamarkupfalse%
\ \isacommand{from}\isamarkupfalse%
\ calculation\isanewline
\ \ \isacommand{have}\isamarkupfalse%
\ {\isachardoublequoteopen}M{\isacharcomma}{\kern0pt}\ some\ {\isasymTurnstile}\ rep{\isacharunderscore}{\kern0pt}body{\isacharunderscore}{\kern0pt}fm{\isacharparenleft}{\kern0pt}{\isasymphi}{\isacharparenright}{\kern0pt}{\isachardoublequoteclose}\isanewline
\ \ \ \ \isacommand{using}\isamarkupfalse%
\ sats{\isacharunderscore}{\kern0pt}nForall{\isacharbrackleft}{\kern0pt}of\ {\isachardoublequoteopen}rep{\isacharunderscore}{\kern0pt}body{\isacharunderscore}{\kern0pt}fm{\isacharparenleft}{\kern0pt}{\isasymphi}{\isacharparenright}{\kern0pt}{\isachardoublequoteclose}\ {\isacharquery}{\kern0pt}n{\isacharbrackright}{\kern0pt}\isanewline
\ \ \ \ \isacommand{unfolding}\isamarkupfalse%
\ ZF{\isacharunderscore}{\kern0pt}replacement{\isacharunderscore}{\kern0pt}fm{\isacharunderscore}{\kern0pt}def\isanewline
\ \ \ \ \isacommand{by}\isamarkupfalse%
\ simp\isanewline
\ \ \isacommand{ultimately}\isamarkupfalse%
\isanewline
\ \ \isacommand{show}\isamarkupfalse%
\ {\isachardoublequoteopen}strong{\isacharunderscore}{\kern0pt}replacement{\isacharparenleft}{\kern0pt}{\isacharhash}{\kern0pt}{\isacharhash}{\kern0pt}M{\isacharcomma}{\kern0pt}\ {\isasymlambda}x\ y{\isachardot}{\kern0pt}\ M{\isacharcomma}{\kern0pt}\ {\isacharbrackleft}{\kern0pt}x{\isacharcomma}{\kern0pt}\ y{\isacharbrackright}{\kern0pt}\ {\isacharat}{\kern0pt}\ env\ {\isasymTurnstile}\ {\isasymphi}{\isacharparenright}{\kern0pt}{\isachardoublequoteclose}\isanewline
\ \ \ \ \isacommand{using}\isamarkupfalse%
\ sats{\isacharunderscore}{\kern0pt}rep{\isacharunderscore}{\kern0pt}body{\isacharunderscore}{\kern0pt}fm{\isacharbrackleft}{\kern0pt}of\ {\isasymphi}\ {\isachardoublequoteopen}{\isacharbrackleft}{\kern0pt}{\isacharbrackright}{\kern0pt}{\isachardoublequoteclose}\ M\ some{\isacharbrackright}{\kern0pt}\isanewline
\ \ \ \ \ \ arity{\isacharunderscore}{\kern0pt}sats{\isacharunderscore}{\kern0pt}iff{\isacharbrackleft}{\kern0pt}of\ {\isasymphi}\ rest\ M\ {\isachardoublequoteopen}{\isacharbrackleft}{\kern0pt}{\isacharunderscore}{\kern0pt}{\isacharcomma}{\kern0pt}{\isacharunderscore}{\kern0pt}{\isacharbrackright}{\kern0pt}\ {\isacharat}{\kern0pt}\ some{\isachardoublequoteclose}{\isacharbrackright}{\kern0pt}\isanewline
\ \ \ \ \ \ strong{\isacharunderscore}{\kern0pt}replacement{\isacharunderscore}{\kern0pt}cong{\isacharbrackleft}{\kern0pt}of\ {\isachardoublequoteopen}{\isacharhash}{\kern0pt}{\isacharhash}{\kern0pt}M{\isachardoublequoteclose}\ {\isachardoublequoteopen}{\isasymlambda}x\ y{\isachardot}{\kern0pt}\ M{\isacharcomma}{\kern0pt}\ Cons{\isacharparenleft}{\kern0pt}x{\isacharcomma}{\kern0pt}\ Cons{\isacharparenleft}{\kern0pt}y{\isacharcomma}{\kern0pt}\ some\ {\isacharat}{\kern0pt}\ rest{\isacharparenright}{\kern0pt}{\isacharparenright}{\kern0pt}\ {\isasymTurnstile}\ {\isasymphi}{\isachardoublequoteclose}\ {\isacharunderscore}{\kern0pt}\ {\isacharbrackright}{\kern0pt}\isanewline
\ \ \ \ \isacommand{by}\isamarkupfalse%
\ simp\isanewline
\isacommand{next}\isamarkupfalse%
\ %
\isamarkupcmt{almost equal to the previous implication%
}\isanewline
\ \ \isacommand{let}\isamarkupfalse%
\ {\isacharquery}{\kern0pt}n{\isacharequal}{\kern0pt}{\isachardoublequoteopen}Arith{\isachardot}{\kern0pt}pred{\isacharparenleft}{\kern0pt}Arith{\isachardot}{\kern0pt}pred{\isacharparenleft}{\kern0pt}arity{\isacharparenleft}{\kern0pt}{\isasymphi}{\isacharparenright}{\kern0pt}{\isacharparenright}{\kern0pt}{\isacharparenright}{\kern0pt}{\isachardoublequoteclose}\isanewline
\ \ \isacommand{assume}\isamarkupfalse%
\ asm{\isacharcolon}{\kern0pt}{\isachardoublequoteopen}{\isasymforall}env{\isasymin}list{\isacharparenleft}{\kern0pt}M{\isacharparenright}{\kern0pt}{\isachardot}{\kern0pt}\ arity{\isacharparenleft}{\kern0pt}{\isasymphi}{\isacharparenright}{\kern0pt}\ {\isasymle}\ {\isadigit{2}}\ {\isacharhash}{\kern0pt}{\isacharplus}{\kern0pt}\ length{\isacharparenleft}{\kern0pt}env{\isacharparenright}{\kern0pt}\ {\isasymlongrightarrow}\ \isanewline
\ \ \ \ strong{\isacharunderscore}{\kern0pt}replacement{\isacharparenleft}{\kern0pt}{\isacharhash}{\kern0pt}{\isacharhash}{\kern0pt}M{\isacharcomma}{\kern0pt}\ {\isasymlambda}x\ y{\isachardot}{\kern0pt}\ M{\isacharcomma}{\kern0pt}\ {\isacharbrackleft}{\kern0pt}x{\isacharcomma}{\kern0pt}\ y{\isacharbrackright}{\kern0pt}\ {\isacharat}{\kern0pt}\ env\ {\isasymTurnstile}\ {\isasymphi}{\isacharparenright}{\kern0pt}{\isachardoublequoteclose}\isanewline
\ \ \isacommand{{\isacharbraceleft}{\kern0pt}}\isamarkupfalse%
\isanewline
\ \ \ \ \isacommand{fix}\isamarkupfalse%
\ some\isanewline
\ \ \ \ \isacommand{assume}\isamarkupfalse%
\ {\isachardoublequoteopen}some{\isasymin}list{\isacharparenleft}{\kern0pt}M{\isacharparenright}{\kern0pt}{\isachardoublequoteclose}\ {\isachardoublequoteopen}length{\isacharparenleft}{\kern0pt}some{\isacharparenright}{\kern0pt}\ {\isacharequal}{\kern0pt}\ Arith{\isachardot}{\kern0pt}pred{\isacharparenleft}{\kern0pt}Arith{\isachardot}{\kern0pt}pred{\isacharparenleft}{\kern0pt}arity{\isacharparenleft}{\kern0pt}{\isasymphi}{\isacharparenright}{\kern0pt}{\isacharparenright}{\kern0pt}{\isacharparenright}{\kern0pt}{\isachardoublequoteclose}\isanewline
\ \ \ \ \isacommand{moreover}\isamarkupfalse%
\isanewline
\ \ \ \ \isacommand{note}\isamarkupfalse%
\ {\isacartoucheopen}{\isasymphi}{\isasymin}{\isacharunderscore}{\kern0pt}{\isacartoucheclose}\isanewline
\ \ \ \ \isacommand{moreover}\isamarkupfalse%
\ \isacommand{from}\isamarkupfalse%
\ calculation\isanewline
\ \ \ \ \isacommand{have}\isamarkupfalse%
\ {\isachardoublequoteopen}arity{\isacharparenleft}{\kern0pt}{\isasymphi}{\isacharparenright}{\kern0pt}\ {\isasymle}\ {\isadigit{2}}\ {\isacharhash}{\kern0pt}{\isacharplus}{\kern0pt}\ length{\isacharparenleft}{\kern0pt}some{\isacharparenright}{\kern0pt}{\isachardoublequoteclose}\ \isanewline
\ \ \ \ \ \ \isacommand{using}\isamarkupfalse%
\ le{\isacharunderscore}{\kern0pt}trans{\isacharbrackleft}{\kern0pt}OF\ succpred{\isacharunderscore}{\kern0pt}leI{\isacharbrackright}{\kern0pt}\ succpred{\isacharunderscore}{\kern0pt}leI\ \isacommand{by}\isamarkupfalse%
\ simp\isanewline
\ \ \ \ \isacommand{moreover}\isamarkupfalse%
\ \isacommand{from}\isamarkupfalse%
\ calculation\ \isakeyword{and}\ asm\isanewline
\ \ \ \ \isacommand{have}\isamarkupfalse%
\ {\isachardoublequoteopen}strong{\isacharunderscore}{\kern0pt}replacement{\isacharparenleft}{\kern0pt}{\isacharhash}{\kern0pt}{\isacharhash}{\kern0pt}M{\isacharcomma}{\kern0pt}\ {\isasymlambda}x\ y{\isachardot}{\kern0pt}\ M{\isacharcomma}{\kern0pt}\ {\isacharbrackleft}{\kern0pt}x{\isacharcomma}{\kern0pt}\ y{\isacharbrackright}{\kern0pt}\ {\isacharat}{\kern0pt}\ some\ {\isasymTurnstile}\ {\isasymphi}{\isacharparenright}{\kern0pt}{\isachardoublequoteclose}\ \isacommand{by}\isamarkupfalse%
\ blast\isanewline
\ \ \ \ \isacommand{ultimately}\isamarkupfalse%
\isanewline
\ \ \ \ \isacommand{have}\isamarkupfalse%
\ {\isachardoublequoteopen}M{\isacharcomma}{\kern0pt}\ some\ {\isasymTurnstile}\ rep{\isacharunderscore}{\kern0pt}body{\isacharunderscore}{\kern0pt}fm{\isacharparenleft}{\kern0pt}{\isasymphi}{\isacharparenright}{\kern0pt}{\isachardoublequoteclose}\ \isanewline
\ \ \ \ \isacommand{using}\isamarkupfalse%
\ sats{\isacharunderscore}{\kern0pt}rep{\isacharunderscore}{\kern0pt}body{\isacharunderscore}{\kern0pt}fm{\isacharbrackleft}{\kern0pt}of\ {\isasymphi}\ {\isachardoublequoteopen}{\isacharbrackleft}{\kern0pt}{\isacharbrackright}{\kern0pt}{\isachardoublequoteclose}\ M\ some{\isacharbrackright}{\kern0pt}\isanewline
\ \ \ \ \ \ arity{\isacharunderscore}{\kern0pt}sats{\isacharunderscore}{\kern0pt}iff{\isacharbrackleft}{\kern0pt}of\ {\isasymphi}\ {\isacharunderscore}{\kern0pt}\ M\ {\isachardoublequoteopen}{\isacharbrackleft}{\kern0pt}{\isacharunderscore}{\kern0pt}{\isacharcomma}{\kern0pt}{\isacharunderscore}{\kern0pt}{\isacharbrackright}{\kern0pt}\ {\isacharat}{\kern0pt}\ some{\isachardoublequoteclose}{\isacharbrackright}{\kern0pt}\isanewline
\ \ \ \ \ \ strong{\isacharunderscore}{\kern0pt}replacement{\isacharunderscore}{\kern0pt}cong{\isacharbrackleft}{\kern0pt}of\ {\isachardoublequoteopen}{\isacharhash}{\kern0pt}{\isacharhash}{\kern0pt}M{\isachardoublequoteclose}\ {\isachardoublequoteopen}{\isasymlambda}x\ y{\isachardot}{\kern0pt}\ M{\isacharcomma}{\kern0pt}\ Cons{\isacharparenleft}{\kern0pt}x{\isacharcomma}{\kern0pt}\ Cons{\isacharparenleft}{\kern0pt}y{\isacharcomma}{\kern0pt}\ some\ {\isacharat}{\kern0pt}\ {\isacharunderscore}{\kern0pt}{\isacharparenright}{\kern0pt}{\isacharparenright}{\kern0pt}\ {\isasymTurnstile}\ {\isasymphi}{\isachardoublequoteclose}\ {\isacharunderscore}{\kern0pt}\ {\isacharbrackright}{\kern0pt}\isanewline
\ \ \ \ \isacommand{by}\isamarkupfalse%
\ simp\isanewline
\ \ \isacommand{{\isacharbraceright}{\kern0pt}}\isamarkupfalse%
\isanewline
\ \ \isacommand{with}\isamarkupfalse%
\ {\isacartoucheopen}{\isasymphi}{\isasymin}{\isacharunderscore}{\kern0pt}{\isacartoucheclose}\isanewline
\ \ \isacommand{show}\isamarkupfalse%
\ {\isachardoublequoteopen}M{\isacharcomma}{\kern0pt}\ {\isacharbrackleft}{\kern0pt}{\isacharbrackright}{\kern0pt}\ {\isasymTurnstile}\ ZF{\isacharunderscore}{\kern0pt}replacement{\isacharunderscore}{\kern0pt}fm{\isacharparenleft}{\kern0pt}{\isasymphi}{\isacharparenright}{\kern0pt}{\isachardoublequoteclose}\isanewline
\ \ \ \ \isacommand{using}\isamarkupfalse%
\ sats{\isacharunderscore}{\kern0pt}nForall{\isacharbrackleft}{\kern0pt}of\ {\isachardoublequoteopen}rep{\isacharunderscore}{\kern0pt}body{\isacharunderscore}{\kern0pt}fm{\isacharparenleft}{\kern0pt}{\isasymphi}{\isacharparenright}{\kern0pt}{\isachardoublequoteclose}\ {\isacharquery}{\kern0pt}n{\isacharbrackright}{\kern0pt}\isanewline
\ \ \ \ \isacommand{unfolding}\isamarkupfalse%
\ ZF{\isacharunderscore}{\kern0pt}replacement{\isacharunderscore}{\kern0pt}fm{\isacharunderscore}{\kern0pt}def\isanewline
\ \ \ \ \isacommand{by}\isamarkupfalse%
\ simp\isanewline
\isacommand{qed}\isamarkupfalse%
%
\endisatagproof
{\isafoldproof}%
%
\isadelimproof
\isanewline
%
\endisadelimproof
\isanewline
\isacommand{definition}\isamarkupfalse%
\isanewline
\ \ ZF{\isacharunderscore}{\kern0pt}inf\ {\isacharcolon}{\kern0pt}{\isacharcolon}{\kern0pt}\ {\isachardoublequoteopen}i{\isachardoublequoteclose}\ \isakeyword{where}\isanewline
\ \ {\isachardoublequoteopen}ZF{\isacharunderscore}{\kern0pt}inf\ {\isasymequiv}\ {\isacharbraceleft}{\kern0pt}ZF{\isacharunderscore}{\kern0pt}separation{\isacharunderscore}{\kern0pt}fm{\isacharparenleft}{\kern0pt}p{\isacharparenright}{\kern0pt}\ {\isachardot}{\kern0pt}\ p\ {\isasymin}\ formula\ {\isacharbraceright}{\kern0pt}\ {\isasymunion}\ {\isacharbraceleft}{\kern0pt}ZF{\isacharunderscore}{\kern0pt}replacement{\isacharunderscore}{\kern0pt}fm{\isacharparenleft}{\kern0pt}p{\isacharparenright}{\kern0pt}\ {\isachardot}{\kern0pt}\ p\ {\isasymin}\ formula\ {\isacharbraceright}{\kern0pt}{\isachardoublequoteclose}\isanewline
\ \ \ \ \ \ \ \ \ \ \ \ \ \ \isanewline
\isacommand{lemma}\isamarkupfalse%
\ Un{\isacharunderscore}{\kern0pt}subset{\isacharunderscore}{\kern0pt}formula{\isacharcolon}{\kern0pt}\ {\isachardoublequoteopen}A{\isasymsubseteq}formula\ {\isasymand}\ B{\isasymsubseteq}formula\ {\isasymLongrightarrow}\ A{\isasymunion}B\ {\isasymsubseteq}\ formula{\isachardoublequoteclose}\isanewline
%
\isadelimproof
\ \ %
\endisadelimproof
%
\isatagproof
\isacommand{by}\isamarkupfalse%
\ auto%
\endisatagproof
{\isafoldproof}%
%
\isadelimproof
\isanewline
%
\endisadelimproof
\ \ \isanewline
\isacommand{lemma}\isamarkupfalse%
\ ZF{\isacharunderscore}{\kern0pt}inf{\isacharunderscore}{\kern0pt}subset{\isacharunderscore}{\kern0pt}formula\ {\isacharcolon}{\kern0pt}\ {\isachardoublequoteopen}ZF{\isacharunderscore}{\kern0pt}inf\ {\isasymsubseteq}\ formula{\isachardoublequoteclose}\isanewline
%
\isadelimproof
\ \ %
\endisadelimproof
%
\isatagproof
\isacommand{unfolding}\isamarkupfalse%
\ ZF{\isacharunderscore}{\kern0pt}inf{\isacharunderscore}{\kern0pt}def\ \isacommand{by}\isamarkupfalse%
\ auto%
\endisatagproof
{\isafoldproof}%
%
\isadelimproof
\isanewline
%
\endisadelimproof
\ \ \ \ \isanewline
\isacommand{definition}\isamarkupfalse%
\isanewline
\ \ ZFC\ {\isacharcolon}{\kern0pt}{\isacharcolon}{\kern0pt}\ {\isachardoublequoteopen}i{\isachardoublequoteclose}\ \isakeyword{where}\isanewline
\ \ {\isachardoublequoteopen}ZFC\ {\isasymequiv}\ ZF{\isacharunderscore}{\kern0pt}inf\ {\isasymunion}\ ZFC{\isacharunderscore}{\kern0pt}fin{\isachardoublequoteclose}\isanewline
\isanewline
\isacommand{definition}\isamarkupfalse%
\isanewline
\ \ ZF\ {\isacharcolon}{\kern0pt}{\isacharcolon}{\kern0pt}\ {\isachardoublequoteopen}i{\isachardoublequoteclose}\ \isakeyword{where}\isanewline
\ \ {\isachardoublequoteopen}ZF\ {\isasymequiv}\ ZF{\isacharunderscore}{\kern0pt}inf\ {\isasymunion}\ ZF{\isacharunderscore}{\kern0pt}fin{\isachardoublequoteclose}\isanewline
\isanewline
\isacommand{definition}\isamarkupfalse%
\ \isanewline
\ \ ZF{\isacharunderscore}{\kern0pt}minus{\isacharunderscore}{\kern0pt}P\ {\isacharcolon}{\kern0pt}{\isacharcolon}{\kern0pt}\ {\isachardoublequoteopen}i{\isachardoublequoteclose}\ \isakeyword{where}\isanewline
\ \ {\isachardoublequoteopen}ZF{\isacharunderscore}{\kern0pt}minus{\isacharunderscore}{\kern0pt}P\ {\isasymequiv}\ ZF\ {\isacharminus}{\kern0pt}\ {\isacharbraceleft}{\kern0pt}\ ZF{\isacharunderscore}{\kern0pt}power{\isacharunderscore}{\kern0pt}fm\ {\isacharbraceright}{\kern0pt}{\isachardoublequoteclose}\isanewline
\isanewline
\isacommand{lemma}\isamarkupfalse%
\ ZFC{\isacharunderscore}{\kern0pt}subset{\isacharunderscore}{\kern0pt}formula{\isacharcolon}{\kern0pt}\ {\isachardoublequoteopen}ZFC\ {\isasymsubseteq}\ formula{\isachardoublequoteclose}\isanewline
%
\isadelimproof
\ \ %
\endisadelimproof
%
\isatagproof
\isacommand{by}\isamarkupfalse%
\ {\isacharparenleft}{\kern0pt}simp\ add{\isacharcolon}{\kern0pt}ZFC{\isacharunderscore}{\kern0pt}def\ Un{\isacharunderscore}{\kern0pt}subset{\isacharunderscore}{\kern0pt}formula\ ZF{\isacharunderscore}{\kern0pt}inf{\isacharunderscore}{\kern0pt}subset{\isacharunderscore}{\kern0pt}formula\ ZFC{\isacharunderscore}{\kern0pt}fin{\isacharunderscore}{\kern0pt}type{\isacharparenright}{\kern0pt}%
\endisatagproof
{\isafoldproof}%
%
\isadelimproof
%
\endisadelimproof
%
\begin{isamarkuptxt}%
Satisfaction of a set of sentences%
\end{isamarkuptxt}\isamarkuptrue%
\isacommand{definition}\isamarkupfalse%
\isanewline
\ \ satT\ {\isacharcolon}{\kern0pt}{\isacharcolon}{\kern0pt}\ {\isachardoublequoteopen}{\isacharbrackleft}{\kern0pt}i{\isacharcomma}{\kern0pt}i{\isacharbrackright}{\kern0pt}\ {\isasymRightarrow}\ o{\isachardoublequoteclose}\ \ {\isacharparenleft}{\kern0pt}{\isachardoublequoteopen}{\isacharunderscore}{\kern0pt}\ {\isasymTurnstile}\ {\isacharunderscore}{\kern0pt}{\isachardoublequoteclose}\ {\isacharbrackleft}{\kern0pt}{\isadigit{3}}{\isadigit{6}}{\isacharcomma}{\kern0pt}{\isadigit{3}}{\isadigit{6}}{\isacharbrackright}{\kern0pt}\ {\isadigit{6}}{\isadigit{0}}{\isacharparenright}{\kern0pt}\ \isakeyword{where}\isanewline
\ \ {\isachardoublequoteopen}A\ {\isasymTurnstile}\ {\isasymPhi}\ \ {\isasymequiv}\ \ {\isasymforall}{\isasymphi}{\isasymin}{\isasymPhi}{\isachardot}{\kern0pt}\ {\isacharparenleft}{\kern0pt}A{\isacharcomma}{\kern0pt}{\isacharbrackleft}{\kern0pt}{\isacharbrackright}{\kern0pt}\ {\isasymTurnstile}\ {\isasymphi}{\isacharparenright}{\kern0pt}{\isachardoublequoteclose}\isanewline
\isanewline
\isacommand{lemma}\isamarkupfalse%
\ satTI\ {\isacharbrackleft}{\kern0pt}intro{\isacharbang}{\kern0pt}{\isacharbrackright}{\kern0pt}{\isacharcolon}{\kern0pt}\ \isanewline
\ \ \isakeyword{assumes}\ {\isachardoublequoteopen}{\isasymAnd}{\isasymphi}{\isachardot}{\kern0pt}\ {\isasymphi}{\isasymin}{\isasymPhi}\ {\isasymLongrightarrow}\ A{\isacharcomma}{\kern0pt}{\isacharbrackleft}{\kern0pt}{\isacharbrackright}{\kern0pt}\ {\isasymTurnstile}\ {\isasymphi}{\isachardoublequoteclose}\isanewline
\ \ \isakeyword{shows}\ {\isachardoublequoteopen}A\ {\isasymTurnstile}\ {\isasymPhi}{\isachardoublequoteclose}\isanewline
%
\isadelimproof
\ \ %
\endisadelimproof
%
\isatagproof
\isacommand{using}\isamarkupfalse%
\ assms\ \isacommand{unfolding}\isamarkupfalse%
\ satT{\isacharunderscore}{\kern0pt}def\ \isacommand{by}\isamarkupfalse%
\ simp%
\endisatagproof
{\isafoldproof}%
%
\isadelimproof
\isanewline
%
\endisadelimproof
\isanewline
\isacommand{lemma}\isamarkupfalse%
\ satTD\ {\isacharbrackleft}{\kern0pt}dest{\isacharbrackright}{\kern0pt}\ {\isacharcolon}{\kern0pt}{\isachardoublequoteopen}A\ {\isasymTurnstile}\ {\isasymPhi}\ {\isasymLongrightarrow}\ \ {\isasymphi}{\isasymin}{\isasymPhi}\ {\isasymLongrightarrow}\ A{\isacharcomma}{\kern0pt}{\isacharbrackleft}{\kern0pt}{\isacharbrackright}{\kern0pt}\ {\isasymTurnstile}\ {\isasymphi}{\isachardoublequoteclose}\isanewline
%
\isadelimproof
\ \ %
\endisadelimproof
%
\isatagproof
\isacommand{unfolding}\isamarkupfalse%
\ satT{\isacharunderscore}{\kern0pt}def\ \isacommand{by}\isamarkupfalse%
\ simp%
\endisatagproof
{\isafoldproof}%
%
\isadelimproof
\isanewline
%
\endisadelimproof
\isanewline
\isacommand{lemma}\isamarkupfalse%
\ sats{\isacharunderscore}{\kern0pt}ZFC{\isacharunderscore}{\kern0pt}iff{\isacharunderscore}{\kern0pt}sats{\isacharunderscore}{\kern0pt}ZF{\isacharunderscore}{\kern0pt}AC{\isacharcolon}{\kern0pt}\ \isanewline
\ \ {\isachardoublequoteopen}{\isacharparenleft}{\kern0pt}N\ {\isasymTurnstile}\ ZFC{\isacharparenright}{\kern0pt}\ {\isasymlongleftrightarrow}\ {\isacharparenleft}{\kern0pt}N\ {\isasymTurnstile}\ ZF{\isacharparenright}{\kern0pt}\ {\isasymand}\ {\isacharparenleft}{\kern0pt}N{\isacharcomma}{\kern0pt}\ {\isacharbrackleft}{\kern0pt}{\isacharbrackright}{\kern0pt}\ {\isasymTurnstile}\ AC{\isacharparenright}{\kern0pt}{\isachardoublequoteclose}\isanewline
%
\isadelimproof
\ \ \ \ %
\endisadelimproof
%
\isatagproof
\isacommand{unfolding}\isamarkupfalse%
\ ZFC{\isacharunderscore}{\kern0pt}def\ ZFC{\isacharunderscore}{\kern0pt}fin{\isacharunderscore}{\kern0pt}def\ ZF{\isacharunderscore}{\kern0pt}def\ \isacommand{by}\isamarkupfalse%
\ auto%
\endisatagproof
{\isafoldproof}%
%
\isadelimproof
\isanewline
%
\endisadelimproof
\isanewline
\isacommand{lemma}\isamarkupfalse%
\ M{\isacharunderscore}{\kern0pt}ZF{\isacharunderscore}{\kern0pt}iff{\isacharunderscore}{\kern0pt}M{\isacharunderscore}{\kern0pt}satT{\isacharcolon}{\kern0pt}\ {\isachardoublequoteopen}M{\isacharunderscore}{\kern0pt}ZF{\isacharparenleft}{\kern0pt}M{\isacharparenright}{\kern0pt}\ {\isasymlongleftrightarrow}\ {\isacharparenleft}{\kern0pt}M\ {\isasymTurnstile}\ ZF{\isacharparenright}{\kern0pt}{\isachardoublequoteclose}\isanewline
%
\isadelimproof
%
\endisadelimproof
%
\isatagproof
\isacommand{proof}\isamarkupfalse%
\isanewline
\ \ \isacommand{assume}\isamarkupfalse%
\ {\isachardoublequoteopen}M\ {\isasymTurnstile}\ ZF{\isachardoublequoteclose}\isanewline
\ \ \isacommand{then}\isamarkupfalse%
\isanewline
\ \ \isacommand{have}\isamarkupfalse%
\ fin{\isacharcolon}{\kern0pt}\ {\isachardoublequoteopen}upair{\isacharunderscore}{\kern0pt}ax{\isacharparenleft}{\kern0pt}{\isacharhash}{\kern0pt}{\isacharhash}{\kern0pt}M{\isacharparenright}{\kern0pt}{\isachardoublequoteclose}\ {\isachardoublequoteopen}Union{\isacharunderscore}{\kern0pt}ax{\isacharparenleft}{\kern0pt}{\isacharhash}{\kern0pt}{\isacharhash}{\kern0pt}M{\isacharparenright}{\kern0pt}{\isachardoublequoteclose}\ {\isachardoublequoteopen}power{\isacharunderscore}{\kern0pt}ax{\isacharparenleft}{\kern0pt}{\isacharhash}{\kern0pt}{\isacharhash}{\kern0pt}M{\isacharparenright}{\kern0pt}{\isachardoublequoteclose}\isanewline
\ \ \ \ {\isachardoublequoteopen}extensionality{\isacharparenleft}{\kern0pt}{\isacharhash}{\kern0pt}{\isacharhash}{\kern0pt}M{\isacharparenright}{\kern0pt}{\isachardoublequoteclose}\ {\isachardoublequoteopen}foundation{\isacharunderscore}{\kern0pt}ax{\isacharparenleft}{\kern0pt}{\isacharhash}{\kern0pt}{\isacharhash}{\kern0pt}M{\isacharparenright}{\kern0pt}{\isachardoublequoteclose}\ {\isachardoublequoteopen}infinity{\isacharunderscore}{\kern0pt}ax{\isacharparenleft}{\kern0pt}{\isacharhash}{\kern0pt}{\isacharhash}{\kern0pt}M{\isacharparenright}{\kern0pt}{\isachardoublequoteclose}\isanewline
\ \ \ \ \isacommand{unfolding}\isamarkupfalse%
\ ZF{\isacharunderscore}{\kern0pt}def\ ZF{\isacharunderscore}{\kern0pt}fin{\isacharunderscore}{\kern0pt}def\ ZFC{\isacharunderscore}{\kern0pt}fm{\isacharunderscore}{\kern0pt}defs\ satT{\isacharunderscore}{\kern0pt}def\isanewline
\ \ \ \ \isacommand{using}\isamarkupfalse%
\ ZFC{\isacharunderscore}{\kern0pt}fm{\isacharunderscore}{\kern0pt}sats{\isacharbrackleft}{\kern0pt}of\ M{\isacharbrackright}{\kern0pt}\ \isacommand{by}\isamarkupfalse%
\ simp{\isacharunderscore}{\kern0pt}all\isanewline
\ \ \isacommand{{\isacharbraceleft}{\kern0pt}}\isamarkupfalse%
\isanewline
\ \ \ \ \isacommand{fix}\isamarkupfalse%
\ {\isasymphi}\ env\isanewline
\ \ \ \ \isacommand{assume}\isamarkupfalse%
\ {\isachardoublequoteopen}{\isasymphi}\ {\isasymin}\ formula{\isachardoublequoteclose}\ {\isachardoublequoteopen}env{\isasymin}list{\isacharparenleft}{\kern0pt}M{\isacharparenright}{\kern0pt}{\isachardoublequoteclose}\ \isanewline
\ \ \ \ \isacommand{moreover}\isamarkupfalse%
\ \isacommand{from}\isamarkupfalse%
\ {\isacartoucheopen}M\ {\isasymTurnstile}\ ZF{\isacartoucheclose}\isanewline
\ \ \ \ \isacommand{have}\isamarkupfalse%
\ {\isachardoublequoteopen}{\isasymforall}p{\isasymin}formula{\isachardot}{\kern0pt}\ {\isacharparenleft}{\kern0pt}M{\isacharcomma}{\kern0pt}\ {\isacharbrackleft}{\kern0pt}{\isacharbrackright}{\kern0pt}\ {\isasymTurnstile}\ {\isacharparenleft}{\kern0pt}ZF{\isacharunderscore}{\kern0pt}separation{\isacharunderscore}{\kern0pt}fm{\isacharparenleft}{\kern0pt}p{\isacharparenright}{\kern0pt}{\isacharparenright}{\kern0pt}{\isacharparenright}{\kern0pt}{\isachardoublequoteclose}\ \isanewline
\ \ \ \ \ \ \ \ \ {\isachardoublequoteopen}{\isasymforall}p{\isasymin}formula{\isachardot}{\kern0pt}\ {\isacharparenleft}{\kern0pt}M{\isacharcomma}{\kern0pt}\ {\isacharbrackleft}{\kern0pt}{\isacharbrackright}{\kern0pt}\ {\isasymTurnstile}\ {\isacharparenleft}{\kern0pt}ZF{\isacharunderscore}{\kern0pt}replacement{\isacharunderscore}{\kern0pt}fm{\isacharparenleft}{\kern0pt}p{\isacharparenright}{\kern0pt}{\isacharparenright}{\kern0pt}{\isacharparenright}{\kern0pt}{\isachardoublequoteclose}\isanewline
\ \ \ \ \ \ \isacommand{unfolding}\isamarkupfalse%
\ ZF{\isacharunderscore}{\kern0pt}def\ ZF{\isacharunderscore}{\kern0pt}inf{\isacharunderscore}{\kern0pt}def\ \isacommand{by}\isamarkupfalse%
\ auto\isanewline
\ \ \ \ \isacommand{moreover}\isamarkupfalse%
\ \isacommand{from}\isamarkupfalse%
\ calculation\isanewline
\ \ \ \ \isacommand{have}\isamarkupfalse%
\ {\isachardoublequoteopen}arity{\isacharparenleft}{\kern0pt}{\isasymphi}{\isacharparenright}{\kern0pt}\ {\isasymle}\ succ{\isacharparenleft}{\kern0pt}length{\isacharparenleft}{\kern0pt}env{\isacharparenright}{\kern0pt}{\isacharparenright}{\kern0pt}\ {\isasymLongrightarrow}\ separation{\isacharparenleft}{\kern0pt}{\isacharhash}{\kern0pt}{\isacharhash}{\kern0pt}M{\isacharcomma}{\kern0pt}\ {\isasymlambda}x{\isachardot}{\kern0pt}\ {\isacharparenleft}{\kern0pt}M{\isacharcomma}{\kern0pt}\ Cons{\isacharparenleft}{\kern0pt}x{\isacharcomma}{\kern0pt}\ env{\isacharparenright}{\kern0pt}\ {\isasymTurnstile}\ {\isasymphi}{\isacharparenright}{\kern0pt}{\isacharparenright}{\kern0pt}{\isachardoublequoteclose}\isanewline
\ \ \ \ \ \ {\isachardoublequoteopen}arity{\isacharparenleft}{\kern0pt}{\isasymphi}{\isacharparenright}{\kern0pt}\ {\isasymle}\ succ{\isacharparenleft}{\kern0pt}succ{\isacharparenleft}{\kern0pt}length{\isacharparenleft}{\kern0pt}env{\isacharparenright}{\kern0pt}{\isacharparenright}{\kern0pt}{\isacharparenright}{\kern0pt}\ {\isasymLongrightarrow}\ strong{\isacharunderscore}{\kern0pt}replacement{\isacharparenleft}{\kern0pt}{\isacharhash}{\kern0pt}{\isacharhash}{\kern0pt}M{\isacharcomma}{\kern0pt}{\isasymlambda}x\ y{\isachardot}{\kern0pt}\ sats{\isacharparenleft}{\kern0pt}M{\isacharcomma}{\kern0pt}{\isasymphi}{\isacharcomma}{\kern0pt}Cons{\isacharparenleft}{\kern0pt}x{\isacharcomma}{\kern0pt}Cons{\isacharparenleft}{\kern0pt}y{\isacharcomma}{\kern0pt}\ env{\isacharparenright}{\kern0pt}{\isacharparenright}{\kern0pt}{\isacharparenright}{\kern0pt}{\isacharparenright}{\kern0pt}{\isachardoublequoteclose}\isanewline
\ \ \ \ \ \ \isacommand{using}\isamarkupfalse%
\ sats{\isacharunderscore}{\kern0pt}ZF{\isacharunderscore}{\kern0pt}separation{\isacharunderscore}{\kern0pt}fm{\isacharunderscore}{\kern0pt}iff\ sats{\isacharunderscore}{\kern0pt}ZF{\isacharunderscore}{\kern0pt}replacement{\isacharunderscore}{\kern0pt}fm{\isacharunderscore}{\kern0pt}iff\ \isacommand{by}\isamarkupfalse%
\ simp{\isacharunderscore}{\kern0pt}all\ \ \isanewline
\ \ \isacommand{{\isacharbraceright}{\kern0pt}}\isamarkupfalse%
\isanewline
\ \ \isacommand{with}\isamarkupfalse%
\ fin\isanewline
\ \ \isacommand{show}\isamarkupfalse%
\ {\isachardoublequoteopen}M{\isacharunderscore}{\kern0pt}ZF{\isacharparenleft}{\kern0pt}M{\isacharparenright}{\kern0pt}{\isachardoublequoteclose}\isanewline
\ \ \ \ \isacommand{unfolding}\isamarkupfalse%
\ M{\isacharunderscore}{\kern0pt}ZF{\isacharunderscore}{\kern0pt}def\ \isacommand{by}\isamarkupfalse%
\ simp\isanewline
\isacommand{next}\isamarkupfalse%
\isanewline
\ \ \isacommand{assume}\isamarkupfalse%
\ {\isacartoucheopen}M{\isacharunderscore}{\kern0pt}ZF{\isacharparenleft}{\kern0pt}M{\isacharparenright}{\kern0pt}{\isacartoucheclose}\isanewline
\ \ \isacommand{then}\isamarkupfalse%
\isanewline
\ \ \isacommand{have}\isamarkupfalse%
\ {\isachardoublequoteopen}M\ {\isasymTurnstile}\ ZF{\isacharunderscore}{\kern0pt}fin{\isachardoublequoteclose}\ \isanewline
\ \ \ \ \isacommand{unfolding}\isamarkupfalse%
\ M{\isacharunderscore}{\kern0pt}ZF{\isacharunderscore}{\kern0pt}def\ ZF{\isacharunderscore}{\kern0pt}fin{\isacharunderscore}{\kern0pt}def\ ZFC{\isacharunderscore}{\kern0pt}fm{\isacharunderscore}{\kern0pt}defs\ satT{\isacharunderscore}{\kern0pt}def\isanewline
\ \ \ \ \isacommand{using}\isamarkupfalse%
\ ZFC{\isacharunderscore}{\kern0pt}fm{\isacharunderscore}{\kern0pt}sats{\isacharbrackleft}{\kern0pt}of\ M{\isacharbrackright}{\kern0pt}\ \isacommand{by}\isamarkupfalse%
\ blast\isanewline
\ \ \isacommand{moreover}\isamarkupfalse%
\ \isacommand{from}\isamarkupfalse%
\ {\isacartoucheopen}M{\isacharunderscore}{\kern0pt}ZF{\isacharparenleft}{\kern0pt}M{\isacharparenright}{\kern0pt}{\isacartoucheclose}\isanewline
\ \ \isacommand{have}\isamarkupfalse%
\ {\isachardoublequoteopen}{\isasymforall}p{\isasymin}formula{\isachardot}{\kern0pt}\ {\isacharparenleft}{\kern0pt}M{\isacharcomma}{\kern0pt}\ {\isacharbrackleft}{\kern0pt}{\isacharbrackright}{\kern0pt}\ {\isasymTurnstile}\ {\isacharparenleft}{\kern0pt}ZF{\isacharunderscore}{\kern0pt}separation{\isacharunderscore}{\kern0pt}fm{\isacharparenleft}{\kern0pt}p{\isacharparenright}{\kern0pt}{\isacharparenright}{\kern0pt}{\isacharparenright}{\kern0pt}{\isachardoublequoteclose}\ \isanewline
\ \ \ \ \ \ \ {\isachardoublequoteopen}{\isasymforall}p{\isasymin}formula{\isachardot}{\kern0pt}\ {\isacharparenleft}{\kern0pt}M{\isacharcomma}{\kern0pt}\ {\isacharbrackleft}{\kern0pt}{\isacharbrackright}{\kern0pt}\ {\isasymTurnstile}\ {\isacharparenleft}{\kern0pt}ZF{\isacharunderscore}{\kern0pt}replacement{\isacharunderscore}{\kern0pt}fm{\isacharparenleft}{\kern0pt}p{\isacharparenright}{\kern0pt}{\isacharparenright}{\kern0pt}{\isacharparenright}{\kern0pt}{\isachardoublequoteclose}\isanewline
\ \ \ \ \isacommand{unfolding}\isamarkupfalse%
\ M{\isacharunderscore}{\kern0pt}ZF{\isacharunderscore}{\kern0pt}def\ \isacommand{using}\isamarkupfalse%
\ sats{\isacharunderscore}{\kern0pt}ZF{\isacharunderscore}{\kern0pt}separation{\isacharunderscore}{\kern0pt}fm{\isacharunderscore}{\kern0pt}iff\ \isanewline
\ \ \ \ \ \ sats{\isacharunderscore}{\kern0pt}ZF{\isacharunderscore}{\kern0pt}replacement{\isacharunderscore}{\kern0pt}fm{\isacharunderscore}{\kern0pt}iff\ \isacommand{by}\isamarkupfalse%
\ simp{\isacharunderscore}{\kern0pt}all\isanewline
\ \ \isacommand{ultimately}\isamarkupfalse%
\isanewline
\ \ \isacommand{show}\isamarkupfalse%
\ {\isachardoublequoteopen}M\ {\isasymTurnstile}\ ZF{\isachardoublequoteclose}\isanewline
\ \ \ \ \isacommand{unfolding}\isamarkupfalse%
\ ZF{\isacharunderscore}{\kern0pt}def\ ZF{\isacharunderscore}{\kern0pt}inf{\isacharunderscore}{\kern0pt}def\ \isacommand{by}\isamarkupfalse%
\ blast\isanewline
\isacommand{qed}\isamarkupfalse%
%
\endisatagproof
{\isafoldproof}%
%
\isadelimproof
\isanewline
%
\endisadelimproof
%
\isadelimtheory
\isanewline
%
\endisadelimtheory
%
\isatagtheory
\isacommand{end}\isamarkupfalse%
%
\endisatagtheory
{\isafoldtheory}%
%
\isadelimtheory
%
\endisadelimtheory
%
\end{isabellebody}%
\endinput
%:%file=~/source/repos/ZF-notAC/code/Forcing/Internal_ZFC_Axioms.thy%:%
%:%11=1%:%
%:%27=2%:%
%:%28=2%:%
%:%29=3%:%
%:%30=4%:%
%:%31=5%:%
%:%32=6%:%
%:%37=6%:%
%:%40=7%:%
%:%41=8%:%
%:%42=8%:%
%:%43=9%:%
%:%46=10%:%
%:%50=10%:%
%:%51=10%:%
%:%52=11%:%
%:%53=11%:%
%:%58=11%:%
%:%63=12%:%
%:%68=13%:%
%:%69=13%:%
%:%74=13%:%
%:%77=14%:%
%:%78=15%:%
%:%79=15%:%
%:%80=16%:%
%:%83=17%:%
%:%87=17%:%
%:%88=17%:%
%:%89=18%:%
%:%90=18%:%
%:%95=18%:%
%:%100=19%:%
%:%105=20%:%
%:%106=20%:%
%:%111=20%:%
%:%114=21%:%
%:%115=22%:%
%:%116=22%:%
%:%117=23%:%
%:%120=24%:%
%:%124=24%:%
%:%125=24%:%
%:%126=25%:%
%:%127=25%:%
%:%132=25%:%
%:%137=26%:%
%:%142=27%:%
%:%143=27%:%
%:%148=27%:%
%:%151=28%:%
%:%152=29%:%
%:%153=29%:%
%:%154=30%:%
%:%157=31%:%
%:%161=31%:%
%:%162=31%:%
%:%163=32%:%
%:%164=32%:%
%:%169=32%:%
%:%174=33%:%
%:%179=34%:%
%:%180=34%:%
%:%185=34%:%
%:%188=35%:%
%:%189=36%:%
%:%190=36%:%
%:%191=37%:%
%:%194=38%:%
%:%198=38%:%
%:%199=38%:%
%:%200=39%:%
%:%201=39%:%
%:%206=39%:%
%:%211=40%:%
%:%216=41%:%
%:%217=41%:%
%:%222=41%:%
%:%225=42%:%
%:%226=43%:%
%:%227=43%:%
%:%228=44%:%
%:%231=45%:%
%:%235=45%:%
%:%236=45%:%
%:%237=46%:%
%:%238=46%:%
%:%243=46%:%
%:%248=47%:%
%:%253=48%:%
%:%254=48%:%
%:%259=48%:%
%:%262=49%:%
%:%263=50%:%
%:%264=50%:%
%:%265=51%:%
%:%268=52%:%
%:%272=52%:%
%:%273=52%:%
%:%274=53%:%
%:%275=53%:%
%:%280=53%:%
%:%285=54%:%
%:%290=55%:%
%:%291=55%:%
%:%296=55%:%
%:%299=56%:%
%:%300=57%:%
%:%301=57%:%
%:%302=58%:%
%:%303=59%:%
%:%304=59%:%
%:%305=60%:%
%:%306=61%:%
%:%307=62%:%
%:%308=62%:%
%:%309=63%:%
%:%310=64%:%
%:%311=65%:%
%:%312=65%:%
%:%313=66%:%
%:%314=67%:%
%:%315=68%:%
%:%316=68%:%
%:%317=69%:%
%:%318=70%:%
%:%319=71%:%
%:%320=72%:%
%:%321=73%:%
%:%322=73%:%
%:%323=74%:%
%:%324=75%:%
%:%325=76%:%
%:%326=77%:%
%:%327=77%:%
%:%330=78%:%
%:%334=78%:%
%:%335=78%:%
%:%336=78%:%
%:%350=80%:%
%:%360=81%:%
%:%361=81%:%
%:%362=82%:%
%:%365=83%:%
%:%369=83%:%
%:%370=83%:%
%:%375=83%:%
%:%378=84%:%
%:%379=85%:%
%:%380=85%:%
%:%381=86%:%
%:%382=87%:%
%:%389=88%:%
%:%390=88%:%
%:%391=89%:%
%:%392=89%:%
%:%393=90%:%
%:%394=90%:%
%:%395=91%:%
%:%396=91%:%
%:%397=92%:%
%:%398=92%:%
%:%399=93%:%
%:%400=93%:%
%:%401=94%:%
%:%402=94%:%
%:%403=95%:%
%:%404=95%:%
%:%405=96%:%
%:%406=96%:%
%:%407=97%:%
%:%408=97%:%
%:%409=97%:%
%:%410=98%:%
%:%411=98%:%
%:%412=99%:%
%:%413=99%:%
%:%414=99%:%
%:%415=100%:%
%:%421=100%:%
%:%424=101%:%
%:%425=102%:%
%:%426=102%:%
%:%427=103%:%
%:%428=104%:%
%:%431=105%:%
%:%435=105%:%
%:%436=105%:%
%:%437=106%:%
%:%438=106%:%
%:%439=107%:%
%:%440=107%:%
%:%441=108%:%
%:%442=108%:%
%:%443=108%:%
%:%444=108%:%
%:%445=109%:%
%:%446=109%:%
%:%447=110%:%
%:%448=110%:%
%:%449=111%:%
%:%450=111%:%
%:%451=111%:%
%:%452=112%:%
%:%453=112%:%
%:%454=113%:%
%:%455=113%:%
%:%456=114%:%
%:%457=114%:%
%:%458=115%:%
%:%459=115%:%
%:%460=116%:%
%:%461=116%:%
%:%462=117%:%
%:%463=117%:%
%:%464=118%:%
%:%465=118%:%
%:%466=119%:%
%:%467=119%:%
%:%468=120%:%
%:%469=120%:%
%:%470=121%:%
%:%471=121%:%
%:%472=122%:%
%:%473=122%:%
%:%474=123%:%
%:%475=123%:%
%:%476=124%:%
%:%477=124%:%
%:%478=125%:%
%:%479=125%:%
%:%480=125%:%
%:%481=125%:%
%:%482=125%:%
%:%483=126%:%
%:%484=126%:%
%:%485=127%:%
%:%491=127%:%
%:%494=128%:%
%:%495=129%:%
%:%496=129%:%
%:%497=130%:%
%:%498=131%:%
%:%505=132%:%
%:%506=132%:%
%:%507=133%:%
%:%508=133%:%
%:%509=134%:%
%:%510=134%:%
%:%511=135%:%
%:%512=135%:%
%:%513=135%:%
%:%514=136%:%
%:%515=136%:%
%:%516=137%:%
%:%517=137%:%
%:%518=138%:%
%:%519=138%:%
%:%520=138%:%
%:%521=139%:%
%:%522=139%:%
%:%523=140%:%
%:%524=140%:%
%:%525=141%:%
%:%526=141%:%
%:%527=142%:%
%:%528=142%:%
%:%529=143%:%
%:%530=143%:%
%:%531=144%:%
%:%532=144%:%
%:%533=145%:%
%:%534=145%:%
%:%535=146%:%
%:%536=146%:%
%:%537=147%:%
%:%538=148%:%
%:%539=148%:%
%:%540=149%:%
%:%546=149%:%
%:%549=150%:%
%:%550=151%:%
%:%551=151%:%
%:%552=152%:%
%:%553=153%:%
%:%554=154%:%
%:%555=155%:%
%:%557=157%:%
%:%564=158%:%
%:%565=158%:%
%:%566=159%:%
%:%567=159%:%
%:%568=160%:%
%:%569=160%:%
%:%570=161%:%
%:%571=161%:%
%:%572=161%:%
%:%573=162%:%
%:%574=162%:%
%:%575=163%:%
%:%576=163%:%
%:%577=164%:%
%:%578=164%:%
%:%579=165%:%
%:%580=166%:%
%:%581=167%:%
%:%582=167%:%
%:%583=167%:%
%:%584=168%:%
%:%585=168%:%
%:%586=169%:%
%:%587=169%:%
%:%588=170%:%
%:%589=171%:%
%:%590=171%:%
%:%591=172%:%
%:%592=172%:%
%:%593=173%:%
%:%594=173%:%
%:%595=174%:%
%:%596=174%:%
%:%597=175%:%
%:%598=175%:%
%:%599=176%:%
%:%605=176%:%
%:%608=177%:%
%:%609=178%:%
%:%610=178%:%
%:%611=179%:%
%:%612=180%:%
%:%614=182%:%
%:%615=183%:%
%:%616=184%:%
%:%617=184%:%
%:%620=185%:%
%:%624=185%:%
%:%625=185%:%
%:%630=185%:%
%:%633=186%:%
%:%634=187%:%
%:%635=187%:%
%:%636=188%:%
%:%637=189%:%
%:%638=190%:%
%:%639=191%:%
%:%640=192%:%
%:%643=193%:%
%:%647=193%:%
%:%648=193%:%
%:%649=194%:%
%:%650=194%:%
%:%651=194%:%
%:%656=194%:%
%:%659=195%:%
%:%660=196%:%
%:%661=196%:%
%:%662=197%:%
%:%663=198%:%
%:%664=199%:%
%:%665=200%:%
%:%666=200%:%
%:%669=201%:%
%:%673=201%:%
%:%674=201%:%
%:%679=201%:%
%:%682=202%:%
%:%683=203%:%
%:%684=203%:%
%:%685=204%:%
%:%686=205%:%
%:%687=206%:%
%:%688=207%:%
%:%691=210%:%
%:%698=211%:%
%:%699=211%:%
%:%700=212%:%
%:%701=212%:%
%:%702=213%:%
%:%703=213%:%
%:%704=214%:%
%:%705=214%:%
%:%706=215%:%
%:%707=215%:%
%:%708=216%:%
%:%709=216%:%
%:%710=216%:%
%:%711=217%:%
%:%712=217%:%
%:%713=217%:%
%:%714=218%:%
%:%715=218%:%
%:%716=219%:%
%:%717=219%:%
%:%718=220%:%
%:%719=221%:%
%:%720=221%:%
%:%721=221%:%
%:%722=222%:%
%:%723=222%:%
%:%724=222%:%
%:%725=223%:%
%:%726=223%:%
%:%727=224%:%
%:%728=224%:%
%:%729=224%:%
%:%730=225%:%
%:%731=225%:%
%:%732=226%:%
%:%733=226%:%
%:%734=227%:%
%:%735=227%:%
%:%736=228%:%
%:%737=228%:%
%:%738=229%:%
%:%739=229%:%
%:%740=229%:%
%:%741=230%:%
%:%742=230%:%
%:%743=231%:%
%:%744=231%:%
%:%745=232%:%
%:%746=232%:%
%:%747=232%:%
%:%748=233%:%
%:%749=233%:%
%:%750=234%:%
%:%751=234%:%
%:%752=235%:%
%:%753=235%:%
%:%754=236%:%
%:%755=236%:%
%:%756=237%:%
%:%757=238%:%
%:%758=239%:%
%:%759=239%:%
%:%760=240%:%
%:%761=240%:%
%:%762=240%:%
%:%763=240%:%
%:%764=241%:%
%:%765=241%:%
%:%766=242%:%
%:%767=242%:%
%:%768=243%:%
%:%769=244%:%
%:%770=244%:%
%:%771=245%:%
%:%772=245%:%
%:%773=246%:%
%:%774=246%:%
%:%775=247%:%
%:%776=247%:%
%:%777=248%:%
%:%778=248%:%
%:%779=249%:%
%:%780=249%:%
%:%781=249%:%
%:%782=250%:%
%:%783=250%:%
%:%784=251%:%
%:%785=251%:%
%:%786=251%:%
%:%787=252%:%
%:%788=252%:%
%:%789=252%:%
%:%790=253%:%
%:%791=253%:%
%:%792=253%:%
%:%793=254%:%
%:%794=254%:%
%:%795=255%:%
%:%796=255%:%
%:%797=256%:%
%:%798=256%:%
%:%799=257%:%
%:%800=258%:%
%:%801=259%:%
%:%802=259%:%
%:%803=260%:%
%:%804=260%:%
%:%805=261%:%
%:%806=261%:%
%:%807=262%:%
%:%808=262%:%
%:%809=263%:%
%:%810=263%:%
%:%811=264%:%
%:%812=264%:%
%:%813=265%:%
%:%814=265%:%
%:%815=266%:%
%:%830=268%:%
%:%840=269%:%
%:%841=269%:%
%:%842=270%:%
%:%843=271%:%
%:%843=273%:%
%:%844=274%:%
%:%845=275%:%
%:%846=276%:%
%:%847=277%:%
%:%848=278%:%
%:%849=279%:%
%:%850=280%:%
%:%851=281%:%
%:%854=282%:%
%:%858=282%:%
%:%859=282%:%
%:%860=283%:%
%:%861=283%:%
%:%866=283%:%
%:%871=284%:%
%:%876=285%:%
%:%877=285%:%
%:%882=285%:%
%:%885=286%:%
%:%886=287%:%
%:%887=287%:%
%:%888=288%:%
%:%891=289%:%
%:%895=289%:%
%:%896=289%:%
%:%901=289%:%
%:%904=290%:%
%:%905=291%:%
%:%906=291%:%
%:%907=292%:%
%:%908=293%:%
%:%909=294%:%
%:%910=295%:%
%:%911=296%:%
%:%912=297%:%
%:%913=298%:%
%:%914=299%:%
%:%915=300%:%
%:%918=301%:%
%:%922=301%:%
%:%923=301%:%
%:%924=301%:%
%:%925=301%:%
%:%930=301%:%
%:%933=302%:%
%:%934=303%:%
%:%935=303%:%
%:%936=304%:%
%:%937=305%:%
%:%938=306%:%
%:%939=307%:%
%:%940=308%:%
%:%941=308%:%
%:%942=309%:%
%:%945=310%:%
%:%949=310%:%
%:%950=310%:%
%:%951=310%:%
%:%956=310%:%
%:%959=311%:%
%:%960=312%:%
%:%961=312%:%
%:%962=313%:%
%:%963=314%:%
%:%966=315%:%
%:%970=315%:%
%:%971=315%:%
%:%972=316%:%
%:%973=316%:%
%:%974=316%:%
%:%979=316%:%
%:%982=317%:%
%:%983=318%:%
%:%984=318%:%
%:%985=319%:%
%:%986=320%:%
%:%987=321%:%
%:%988=322%:%
%:%989=322%:%
%:%990=323%:%
%:%991=324%:%
%:%992=325%:%
%:%993=326%:%
%:%994=326%:%
%:%995=327%:%
%:%996=328%:%
%:%999=329%:%
%:%1003=329%:%
%:%1004=329%:%
%:%1005=329%:%
%:%1006=329%:%
%:%1011=329%:%
%:%1014=330%:%
%:%1015=331%:%
%:%1016=331%:%
%:%1017=332%:%
%:%1018=333%:%
%:%1019=334%:%
%:%1020=335%:%
%:%1021=336%:%
%:%1024=337%:%
%:%1028=337%:%
%:%1029=337%:%
%:%1030=338%:%
%:%1031=338%:%
%:%1032=339%:%
%:%1033=339%:%
%:%1034=339%:%
%:%1035=340%:%
%:%1036=341%:%
%:%1037=342%:%
%:%1038=342%:%
%:%1043=342%:%
%:%1046=343%:%
%:%1047=344%:%
%:%1048=344%:%
%:%1049=345%:%
%:%1050=346%:%
%:%1053=349%:%
%:%1054=350%:%
%:%1055=351%:%
%:%1056=351%:%
%:%1059=352%:%
%:%1063=352%:%
%:%1064=352%:%
%:%1069=352%:%
%:%1072=353%:%
%:%1073=354%:%
%:%1074=354%:%
%:%1075=355%:%
%:%1076=355%:%
%:%1077=355%:%
%:%1078=356%:%
%:%1079=356%:%
%:%1080=356%:%
%:%1081=357%:%
%:%1082=358%:%
%:%1083=359%:%
%:%1084=359%:%
%:%1085=360%:%
%:%1086=361%:%
%:%1087=362%:%
%:%1088=363%:%
%:%1089=364%:%
%:%1092=365%:%
%:%1096=365%:%
%:%1097=365%:%
%:%1098=366%:%
%:%1099=366%:%
%:%1100=367%:%
%:%1101=367%:%
%:%1102=368%:%
%:%1103=368%:%
%:%1108=368%:%
%:%1111=369%:%
%:%1112=370%:%
%:%1113=370%:%
%:%1114=371%:%
%:%1115=372%:%
%:%1116=373%:%
%:%1117=374%:%
%:%1118=374%:%
%:%1121=375%:%
%:%1125=375%:%
%:%1126=375%:%
%:%1131=375%:%
%:%1134=376%:%
%:%1135=377%:%
%:%1136=377%:%
%:%1137=378%:%
%:%1138=379%:%
%:%1139=380%:%
%:%1140=381%:%
%:%1143=384%:%
%:%1150=385%:%
%:%1151=385%:%
%:%1152=386%:%
%:%1153=386%:%
%:%1154=387%:%
%:%1155=387%:%
%:%1156=388%:%
%:%1157=388%:%
%:%1158=389%:%
%:%1159=389%:%
%:%1160=389%:%
%:%1161=390%:%
%:%1162=390%:%
%:%1163=390%:%
%:%1164=391%:%
%:%1165=391%:%
%:%1166=391%:%
%:%1167=392%:%
%:%1168=392%:%
%:%1169=393%:%
%:%1170=393%:%
%:%1171=393%:%
%:%1172=394%:%
%:%1173=394%:%
%:%1174=394%:%
%:%1175=395%:%
%:%1176=395%:%
%:%1177=396%:%
%:%1178=397%:%
%:%1179=397%:%
%:%1180=397%:%
%:%1181=398%:%
%:%1182=398%:%
%:%1183=399%:%
%:%1184=399%:%
%:%1185=400%:%
%:%1186=400%:%
%:%1187=400%:%
%:%1188=401%:%
%:%1189=401%:%
%:%1190=402%:%
%:%1191=402%:%
%:%1192=402%:%
%:%1193=403%:%
%:%1194=403%:%
%:%1195=403%:%
%:%1196=404%:%
%:%1197=404%:%
%:%1198=405%:%
%:%1199=405%:%
%:%1200=406%:%
%:%1201=406%:%
%:%1202=407%:%
%:%1203=407%:%
%:%1204=408%:%
%:%1205=408%:%
%:%1206=409%:%
%:%1207=409%:%
%:%1208=410%:%
%:%1209=410%:%
%:%1210=411%:%
%:%1211=412%:%
%:%1212=413%:%
%:%1213=413%:%
%:%1214=414%:%
%:%1215=414%:%
%:%1216=414%:%
%:%1217=414%:%
%:%1218=415%:%
%:%1219=415%:%
%:%1220=416%:%
%:%1221=416%:%
%:%1222=417%:%
%:%1223=418%:%
%:%1224=418%:%
%:%1225=419%:%
%:%1226=419%:%
%:%1227=420%:%
%:%1228=420%:%
%:%1229=421%:%
%:%1230=421%:%
%:%1231=422%:%
%:%1232=422%:%
%:%1233=423%:%
%:%1234=423%:%
%:%1235=423%:%
%:%1236=424%:%
%:%1237=424%:%
%:%1238=425%:%
%:%1239=425%:%
%:%1240=425%:%
%:%1241=426%:%
%:%1242=426%:%
%:%1243=426%:%
%:%1244=427%:%
%:%1245=427%:%
%:%1246=427%:%
%:%1247=428%:%
%:%1248=428%:%
%:%1249=429%:%
%:%1250=429%:%
%:%1251=430%:%
%:%1252=430%:%
%:%1253=431%:%
%:%1254=432%:%
%:%1255=433%:%
%:%1256=433%:%
%:%1257=434%:%
%:%1258=434%:%
%:%1259=435%:%
%:%1260=435%:%
%:%1261=436%:%
%:%1262=436%:%
%:%1263=437%:%
%:%1264=437%:%
%:%1265=438%:%
%:%1266=438%:%
%:%1267=439%:%
%:%1268=439%:%
%:%1269=440%:%
%:%1275=440%:%
%:%1278=441%:%
%:%1279=442%:%
%:%1280=442%:%
%:%1281=443%:%
%:%1282=444%:%
%:%1283=445%:%
%:%1284=446%:%
%:%1285=446%:%
%:%1288=447%:%
%:%1292=447%:%
%:%1293=447%:%
%:%1298=447%:%
%:%1301=448%:%
%:%1302=449%:%
%:%1303=449%:%
%:%1306=450%:%
%:%1310=450%:%
%:%1311=450%:%
%:%1312=450%:%
%:%1317=450%:%
%:%1320=451%:%
%:%1321=452%:%
%:%1322=452%:%
%:%1323=453%:%
%:%1324=454%:%
%:%1325=455%:%
%:%1326=456%:%
%:%1327=456%:%
%:%1328=457%:%
%:%1329=458%:%
%:%1330=459%:%
%:%1331=460%:%
%:%1332=460%:%
%:%1333=461%:%
%:%1334=462%:%
%:%1335=463%:%
%:%1336=464%:%
%:%1337=464%:%
%:%1340=465%:%
%:%1344=465%:%
%:%1345=465%:%
%:%1354=467%:%
%:%1356=468%:%
%:%1357=468%:%
%:%1358=469%:%
%:%1359=470%:%
%:%1360=471%:%
%:%1361=472%:%
%:%1362=472%:%
%:%1363=473%:%
%:%1364=474%:%
%:%1367=475%:%
%:%1371=475%:%
%:%1372=475%:%
%:%1373=475%:%
%:%1374=475%:%
%:%1379=475%:%
%:%1382=476%:%
%:%1383=477%:%
%:%1384=477%:%
%:%1387=478%:%
%:%1391=478%:%
%:%1392=478%:%
%:%1393=478%:%
%:%1398=478%:%
%:%1401=479%:%
%:%1402=480%:%
%:%1403=480%:%
%:%1404=481%:%
%:%1407=482%:%
%:%1411=482%:%
%:%1412=482%:%
%:%1413=482%:%
%:%1418=482%:%
%:%1421=483%:%
%:%1422=484%:%
%:%1423=484%:%
%:%1430=485%:%
%:%1431=485%:%
%:%1432=486%:%
%:%1433=486%:%
%:%1434=487%:%
%:%1435=487%:%
%:%1436=488%:%
%:%1437=488%:%
%:%1438=489%:%
%:%1439=490%:%
%:%1440=490%:%
%:%1441=491%:%
%:%1442=491%:%
%:%1443=491%:%
%:%1444=492%:%
%:%1445=492%:%
%:%1446=493%:%
%:%1447=493%:%
%:%1448=494%:%
%:%1449=494%:%
%:%1450=495%:%
%:%1451=495%:%
%:%1452=495%:%
%:%1453=496%:%
%:%1454=496%:%
%:%1455=497%:%
%:%1456=498%:%
%:%1457=498%:%
%:%1458=498%:%
%:%1459=499%:%
%:%1460=499%:%
%:%1461=499%:%
%:%1462=500%:%
%:%1463=500%:%
%:%1464=501%:%
%:%1465=502%:%
%:%1466=502%:%
%:%1467=502%:%
%:%1468=503%:%
%:%1469=503%:%
%:%1470=504%:%
%:%1471=504%:%
%:%1472=505%:%
%:%1473=505%:%
%:%1474=506%:%
%:%1475=506%:%
%:%1476=506%:%
%:%1477=507%:%
%:%1478=507%:%
%:%1479=508%:%
%:%1480=508%:%
%:%1481=509%:%
%:%1482=509%:%
%:%1483=510%:%
%:%1484=510%:%
%:%1485=511%:%
%:%1486=511%:%
%:%1487=512%:%
%:%1488=512%:%
%:%1489=512%:%
%:%1490=513%:%
%:%1491=513%:%
%:%1492=513%:%
%:%1493=514%:%
%:%1494=514%:%
%:%1495=515%:%
%:%1496=516%:%
%:%1497=516%:%
%:%1498=516%:%
%:%1499=517%:%
%:%1500=517%:%
%:%1501=518%:%
%:%1502=518%:%
%:%1503=519%:%
%:%1504=519%:%
%:%1505=520%:%
%:%1506=520%:%
%:%1507=520%:%
%:%1508=521%:%
%:%1514=521%:%
%:%1519=522%:%
%:%1524=523%:%

%
\begin{isabellebody}%
\setisabellecontext{Renaming}%
%
\isadelimdocument
%
\endisadelimdocument
%
\isatagdocument
%
\isamarkupsection{Renaming of variables in internalized formulas%
}
\isamarkuptrue%
%
\endisatagdocument
{\isafolddocument}%
%
\isadelimdocument
%
\endisadelimdocument
%
\isadelimtheory
%
\endisadelimtheory
%
\isatagtheory
\isacommand{theory}\isamarkupfalse%
\ Renaming\isanewline
\ \ \isakeyword{imports}\isanewline
\ \ \ \ Nat{\isacharunderscore}{\kern0pt}Miscellanea\isanewline
\ \ \ \ {\isachardoublequoteopen}ZF{\isacharminus}{\kern0pt}Constructible{\isachardot}{\kern0pt}Formula{\isachardoublequoteclose}\isanewline
\isakeyword{begin}%
\endisatagtheory
{\isafoldtheory}%
%
\isadelimtheory
\isanewline
%
\endisadelimtheory
\isanewline
\isacommand{lemma}\isamarkupfalse%
\ app{\isacharunderscore}{\kern0pt}nm\ {\isacharcolon}{\kern0pt}\isanewline
\ \ \isakeyword{assumes}\ {\isachardoublequoteopen}n{\isasymin}nat{\isachardoublequoteclose}\ {\isachardoublequoteopen}m{\isasymin}nat{\isachardoublequoteclose}\ {\isachardoublequoteopen}f{\isasymin}n{\isasymrightarrow}m{\isachardoublequoteclose}\ {\isachardoublequoteopen}x\ {\isasymin}\ nat{\isachardoublequoteclose}\isanewline
\ \ \isakeyword{shows}\ {\isachardoublequoteopen}f{\isacharbackquote}{\kern0pt}x\ {\isasymin}\ nat{\isachardoublequoteclose}\isanewline
%
\isadelimproof
%
\endisadelimproof
%
\isatagproof
\isacommand{proof}\isamarkupfalse%
{\isacharparenleft}{\kern0pt}cases\ {\isachardoublequoteopen}x{\isasymin}n{\isachardoublequoteclose}{\isacharparenright}{\kern0pt}\isanewline
\ \ \isacommand{case}\isamarkupfalse%
\ True\isanewline
\ \ \isacommand{then}\isamarkupfalse%
\ \isacommand{show}\isamarkupfalse%
\ {\isacharquery}{\kern0pt}thesis\ \isacommand{using}\isamarkupfalse%
\ assms\ in{\isacharunderscore}{\kern0pt}n{\isacharunderscore}{\kern0pt}in{\isacharunderscore}{\kern0pt}nat\ apply{\isacharunderscore}{\kern0pt}type\ \isacommand{by}\isamarkupfalse%
\ simp\isanewline
\isacommand{next}\isamarkupfalse%
\isanewline
\ \ \isacommand{case}\isamarkupfalse%
\ False\isanewline
\ \ \isacommand{then}\isamarkupfalse%
\ \isacommand{show}\isamarkupfalse%
\ {\isacharquery}{\kern0pt}thesis\ \isacommand{using}\isamarkupfalse%
\ assms\ apply{\isacharunderscore}{\kern0pt}{\isadigit{0}}\ domain{\isacharunderscore}{\kern0pt}of{\isacharunderscore}{\kern0pt}fun\ \isacommand{by}\isamarkupfalse%
\ simp\isanewline
\isacommand{qed}\isamarkupfalse%
%
\endisatagproof
{\isafoldproof}%
%
\isadelimproof
%
\endisadelimproof
%
\isadelimdocument
%
\endisadelimdocument
%
\isatagdocument
%
\isamarkupsubsection{Renaming of free variables%
}
\isamarkuptrue%
%
\endisatagdocument
{\isafolddocument}%
%
\isadelimdocument
%
\endisadelimdocument
\isacommand{definition}\isamarkupfalse%
\isanewline
\ \ union{\isacharunderscore}{\kern0pt}fun\ {\isacharcolon}{\kern0pt}{\isacharcolon}{\kern0pt}\ {\isachardoublequoteopen}{\isacharbrackleft}{\kern0pt}i{\isacharcomma}{\kern0pt}i{\isacharcomma}{\kern0pt}i{\isacharcomma}{\kern0pt}i{\isacharbrackright}{\kern0pt}\ {\isasymRightarrow}\ i{\isachardoublequoteclose}\ \isakeyword{where}\isanewline
\ \ {\isachardoublequoteopen}union{\isacharunderscore}{\kern0pt}fun{\isacharparenleft}{\kern0pt}f{\isacharcomma}{\kern0pt}g{\isacharcomma}{\kern0pt}m{\isacharcomma}{\kern0pt}p{\isacharparenright}{\kern0pt}\ {\isasymequiv}\ {\isasymlambda}j\ {\isasymin}\ m\ {\isasymunion}\ p\ \ {\isachardot}{\kern0pt}\ if\ j{\isasymin}m\ then\ f{\isacharbackquote}{\kern0pt}j\ else\ g{\isacharbackquote}{\kern0pt}j{\isachardoublequoteclose}\isanewline
\isanewline
\isacommand{lemma}\isamarkupfalse%
\ union{\isacharunderscore}{\kern0pt}fun{\isacharunderscore}{\kern0pt}type{\isacharcolon}{\kern0pt}\isanewline
\ \ \isakeyword{assumes}\ {\isachardoublequoteopen}f\ {\isasymin}\ m\ {\isasymrightarrow}\ n{\isachardoublequoteclose}\isanewline
\ \ \ \ {\isachardoublequoteopen}g\ {\isasymin}\ p\ {\isasymrightarrow}\ q{\isachardoublequoteclose}\isanewline
\ \ \isakeyword{shows}\ {\isachardoublequoteopen}union{\isacharunderscore}{\kern0pt}fun{\isacharparenleft}{\kern0pt}f{\isacharcomma}{\kern0pt}g{\isacharcomma}{\kern0pt}m{\isacharcomma}{\kern0pt}p{\isacharparenright}{\kern0pt}\ {\isasymin}\ m\ {\isasymunion}\ p\ {\isasymrightarrow}\ n\ {\isasymunion}\ q{\isachardoublequoteclose}\isanewline
%
\isadelimproof
%
\endisadelimproof
%
\isatagproof
\isacommand{proof}\isamarkupfalse%
\ {\isacharminus}{\kern0pt}\isanewline
\ \ \isacommand{let}\isamarkupfalse%
\ {\isacharquery}{\kern0pt}h{\isacharequal}{\kern0pt}{\isachardoublequoteopen}union{\isacharunderscore}{\kern0pt}fun{\isacharparenleft}{\kern0pt}f{\isacharcomma}{\kern0pt}g{\isacharcomma}{\kern0pt}m{\isacharcomma}{\kern0pt}p{\isacharparenright}{\kern0pt}{\isachardoublequoteclose}\isanewline
\ \ \isacommand{have}\isamarkupfalse%
\isanewline
\ \ \ \ D{\isacharcolon}{\kern0pt}\ {\isachardoublequoteopen}{\isacharquery}{\kern0pt}h{\isacharbackquote}{\kern0pt}x\ {\isasymin}\ n\ {\isasymunion}\ q{\isachardoublequoteclose}\ \isakeyword{if}\ {\isachardoublequoteopen}x\ {\isasymin}\ m\ {\isasymunion}\ p{\isachardoublequoteclose}\ \isakeyword{for}\ x\isanewline
\ \ \isacommand{proof}\isamarkupfalse%
\ {\isacharparenleft}{\kern0pt}cases\ {\isachardoublequoteopen}x\ {\isasymin}\ m{\isachardoublequoteclose}{\isacharparenright}{\kern0pt}\isanewline
\ \ \ \ \isacommand{case}\isamarkupfalse%
\ True\isanewline
\ \ \ \ \isacommand{then}\isamarkupfalse%
\ \isacommand{have}\isamarkupfalse%
\isanewline
\ \ \ \ \ \ {\isachardoublequoteopen}x\ {\isasymin}\ m\ {\isasymunion}\ p{\isachardoublequoteclose}\ \isacommand{by}\isamarkupfalse%
\ simp\isanewline
\ \ \ \ \isacommand{with}\isamarkupfalse%
\ {\isacartoucheopen}x{\isasymin}m{\isacartoucheclose}\isanewline
\ \ \ \ \isacommand{have}\isamarkupfalse%
\ {\isachardoublequoteopen}{\isacharquery}{\kern0pt}h{\isacharbackquote}{\kern0pt}x\ {\isacharequal}{\kern0pt}\ f{\isacharbackquote}{\kern0pt}x{\isachardoublequoteclose}\isanewline
\ \ \ \ \ \ \isacommand{unfolding}\isamarkupfalse%
\ union{\isacharunderscore}{\kern0pt}fun{\isacharunderscore}{\kern0pt}def\ \ beta\ \isacommand{by}\isamarkupfalse%
\ simp\isanewline
\ \ \ \ \isacommand{with}\isamarkupfalse%
\ {\isacartoucheopen}f\ {\isasymin}\ m\ {\isasymrightarrow}\ n{\isacartoucheclose}\ {\isacartoucheopen}x{\isasymin}m{\isacartoucheclose}\isanewline
\ \ \ \ \isacommand{have}\isamarkupfalse%
\ {\isachardoublequoteopen}{\isacharquery}{\kern0pt}h{\isacharbackquote}{\kern0pt}x\ {\isasymin}\ n{\isachardoublequoteclose}\ \isacommand{by}\isamarkupfalse%
\ simp\isanewline
\ \ \ \ \isacommand{then}\isamarkupfalse%
\ \isacommand{show}\isamarkupfalse%
\ {\isacharquery}{\kern0pt}thesis\ \isacommand{{\isachardot}{\kern0pt}{\isachardot}{\kern0pt}}\isamarkupfalse%
\isanewline
\ \ \isacommand{next}\isamarkupfalse%
\isanewline
\ \ \ \ \isacommand{case}\isamarkupfalse%
\ False\isanewline
\ \ \ \ \isacommand{with}\isamarkupfalse%
\ {\isacartoucheopen}x\ {\isasymin}\ m\ {\isasymunion}\ p{\isacartoucheclose}\isanewline
\ \ \ \ \isacommand{have}\isamarkupfalse%
\ {\isachardoublequoteopen}x\ {\isasymin}\ p{\isachardoublequoteclose}\isanewline
\ \ \ \ \ \ \isacommand{by}\isamarkupfalse%
\ auto\isanewline
\ \ \ \ \isacommand{with}\isamarkupfalse%
\ {\isacartoucheopen}x{\isasymnotin}m{\isacartoucheclose}\isanewline
\ \ \ \ \isacommand{have}\isamarkupfalse%
\ {\isachardoublequoteopen}{\isacharquery}{\kern0pt}h{\isacharbackquote}{\kern0pt}x\ {\isacharequal}{\kern0pt}\ g{\isacharbackquote}{\kern0pt}x{\isachardoublequoteclose}\isanewline
\ \ \ \ \ \ \isacommand{unfolding}\isamarkupfalse%
\ union{\isacharunderscore}{\kern0pt}fun{\isacharunderscore}{\kern0pt}def\ \isacommand{using}\isamarkupfalse%
\ beta\ \isacommand{by}\isamarkupfalse%
\ simp\isanewline
\ \ \ \ \isacommand{with}\isamarkupfalse%
\ {\isacartoucheopen}g\ {\isasymin}\ p\ {\isasymrightarrow}\ q{\isacartoucheclose}\ {\isacartoucheopen}x{\isasymin}p{\isacartoucheclose}\isanewline
\ \ \ \ \isacommand{have}\isamarkupfalse%
\ {\isachardoublequoteopen}{\isacharquery}{\kern0pt}h{\isacharbackquote}{\kern0pt}x\ {\isasymin}\ q{\isachardoublequoteclose}\ \isacommand{by}\isamarkupfalse%
\ simp\isanewline
\ \ \ \ \isacommand{then}\isamarkupfalse%
\ \isacommand{show}\isamarkupfalse%
\ {\isacharquery}{\kern0pt}thesis\ \isacommand{{\isachardot}{\kern0pt}{\isachardot}{\kern0pt}}\isamarkupfalse%
\isanewline
\ \ \isacommand{qed}\isamarkupfalse%
\isanewline
\ \ \isacommand{have}\isamarkupfalse%
\ A{\isacharcolon}{\kern0pt}{\isachardoublequoteopen}function{\isacharparenleft}{\kern0pt}{\isacharquery}{\kern0pt}h{\isacharparenright}{\kern0pt}{\isachardoublequoteclose}\ \isacommand{unfolding}\isamarkupfalse%
\ union{\isacharunderscore}{\kern0pt}fun{\isacharunderscore}{\kern0pt}def\ \isacommand{using}\isamarkupfalse%
\ function{\isacharunderscore}{\kern0pt}lam\ \isacommand{by}\isamarkupfalse%
\ simp\isanewline
\ \ \isacommand{have}\isamarkupfalse%
\ {\isachardoublequoteopen}\ x{\isasymin}\ {\isacharparenleft}{\kern0pt}m\ {\isasymunion}\ p{\isacharparenright}{\kern0pt}\ {\isasymtimes}\ {\isacharparenleft}{\kern0pt}n\ {\isasymunion}\ q{\isacharparenright}{\kern0pt}{\isachardoublequoteclose}\ \isakeyword{if}\ {\isachardoublequoteopen}x{\isasymin}\ {\isacharquery}{\kern0pt}h{\isachardoublequoteclose}\ \isakeyword{for}\ x\isanewline
\ \ \ \ \isacommand{using}\isamarkupfalse%
\ that\ lamE{\isacharbrackleft}{\kern0pt}of\ x\ {\isachardoublequoteopen}m\ {\isasymunion}\ p{\isachardoublequoteclose}\ {\isacharunderscore}{\kern0pt}\ {\isachardoublequoteopen}x\ {\isasymin}\ {\isacharparenleft}{\kern0pt}m\ {\isasymunion}\ p{\isacharparenright}{\kern0pt}\ {\isasymtimes}\ {\isacharparenleft}{\kern0pt}n\ {\isasymunion}\ q{\isacharparenright}{\kern0pt}{\isachardoublequoteclose}{\isacharbrackright}{\kern0pt}\ D\ \isacommand{unfolding}\isamarkupfalse%
\ union{\isacharunderscore}{\kern0pt}fun{\isacharunderscore}{\kern0pt}def\isanewline
\ \ \ \ \isacommand{by}\isamarkupfalse%
\ auto\isanewline
\ \ \isacommand{then}\isamarkupfalse%
\ \isacommand{have}\isamarkupfalse%
\ B{\isacharcolon}{\kern0pt}{\isachardoublequoteopen}{\isacharquery}{\kern0pt}h\ {\isasymsubseteq}\ {\isacharparenleft}{\kern0pt}m\ {\isasymunion}\ p{\isacharparenright}{\kern0pt}\ {\isasymtimes}\ {\isacharparenleft}{\kern0pt}n\ {\isasymunion}\ q{\isacharparenright}{\kern0pt}{\isachardoublequoteclose}\ \isacommand{{\isachardot}{\kern0pt}{\isachardot}{\kern0pt}}\isamarkupfalse%
\isanewline
\ \ \isacommand{have}\isamarkupfalse%
\ {\isachardoublequoteopen}m\ {\isasymunion}\ p\ {\isasymsubseteq}\ domain{\isacharparenleft}{\kern0pt}{\isacharquery}{\kern0pt}h{\isacharparenright}{\kern0pt}{\isachardoublequoteclose}\isanewline
\ \ \ \ \isacommand{unfolding}\isamarkupfalse%
\ union{\isacharunderscore}{\kern0pt}fun{\isacharunderscore}{\kern0pt}def\ \isacommand{using}\isamarkupfalse%
\ domain{\isacharunderscore}{\kern0pt}lam\ \isacommand{by}\isamarkupfalse%
\ simp\isanewline
\ \ \isacommand{with}\isamarkupfalse%
\ A\ B\isanewline
\ \ \isacommand{show}\isamarkupfalse%
\ {\isacharquery}{\kern0pt}thesis\ \isacommand{using}\isamarkupfalse%
\ \ Pi{\isacharunderscore}{\kern0pt}iff\ {\isacharbrackleft}{\kern0pt}THEN\ iffD{\isadigit{2}}{\isacharbrackright}{\kern0pt}\ \isacommand{by}\isamarkupfalse%
\ simp\isanewline
\isacommand{qed}\isamarkupfalse%
%
\endisatagproof
{\isafoldproof}%
%
\isadelimproof
\isanewline
%
\endisadelimproof
\isanewline
\isacommand{lemma}\isamarkupfalse%
\ union{\isacharunderscore}{\kern0pt}fun{\isacharunderscore}{\kern0pt}action\ {\isacharcolon}{\kern0pt}\isanewline
\ \ \isakeyword{assumes}\isanewline
\ \ \ \ {\isachardoublequoteopen}env\ {\isasymin}\ list{\isacharparenleft}{\kern0pt}M{\isacharparenright}{\kern0pt}{\isachardoublequoteclose}\isanewline
\ \ \ \ {\isachardoublequoteopen}env{\isacharprime}{\kern0pt}\ {\isasymin}\ list{\isacharparenleft}{\kern0pt}M{\isacharparenright}{\kern0pt}{\isachardoublequoteclose}\isanewline
\ \ \ \ {\isachardoublequoteopen}length{\isacharparenleft}{\kern0pt}env{\isacharparenright}{\kern0pt}\ {\isacharequal}{\kern0pt}\ m\ {\isasymunion}\ p{\isachardoublequoteclose}\isanewline
\ \ \ \ {\isachardoublequoteopen}{\isasymforall}\ i\ {\isachardot}{\kern0pt}\ i\ {\isasymin}\ m\ {\isasymlongrightarrow}\ \ nth{\isacharparenleft}{\kern0pt}f{\isacharbackquote}{\kern0pt}i{\isacharcomma}{\kern0pt}env{\isacharprime}{\kern0pt}{\isacharparenright}{\kern0pt}\ {\isacharequal}{\kern0pt}\ nth{\isacharparenleft}{\kern0pt}i{\isacharcomma}{\kern0pt}env{\isacharparenright}{\kern0pt}{\isachardoublequoteclose}\isanewline
\ \ \ \ {\isachardoublequoteopen}{\isasymforall}\ j\ {\isachardot}{\kern0pt}\ j\ {\isasymin}\ p\ {\isasymlongrightarrow}\ nth{\isacharparenleft}{\kern0pt}g{\isacharbackquote}{\kern0pt}j{\isacharcomma}{\kern0pt}env{\isacharprime}{\kern0pt}{\isacharparenright}{\kern0pt}\ {\isacharequal}{\kern0pt}\ nth{\isacharparenleft}{\kern0pt}j{\isacharcomma}{\kern0pt}env{\isacharparenright}{\kern0pt}{\isachardoublequoteclose}\isanewline
\ \ \isakeyword{shows}\ {\isachardoublequoteopen}{\isasymforall}\ i\ {\isachardot}{\kern0pt}\ i\ {\isasymin}\ m\ {\isasymunion}\ p\ {\isasymlongrightarrow}\isanewline
\ \ \ \ \ \ \ \ \ \ nth{\isacharparenleft}{\kern0pt}i{\isacharcomma}{\kern0pt}env{\isacharparenright}{\kern0pt}\ {\isacharequal}{\kern0pt}\ nth{\isacharparenleft}{\kern0pt}union{\isacharunderscore}{\kern0pt}fun{\isacharparenleft}{\kern0pt}f{\isacharcomma}{\kern0pt}g{\isacharcomma}{\kern0pt}m{\isacharcomma}{\kern0pt}p{\isacharparenright}{\kern0pt}{\isacharbackquote}{\kern0pt}i{\isacharcomma}{\kern0pt}env{\isacharprime}{\kern0pt}{\isacharparenright}{\kern0pt}{\isachardoublequoteclose}\isanewline
%
\isadelimproof
%
\endisadelimproof
%
\isatagproof
\isacommand{proof}\isamarkupfalse%
\ {\isacharminus}{\kern0pt}\isanewline
\ \ \isacommand{let}\isamarkupfalse%
\ {\isacharquery}{\kern0pt}h\ {\isacharequal}{\kern0pt}\ {\isachardoublequoteopen}union{\isacharunderscore}{\kern0pt}fun{\isacharparenleft}{\kern0pt}f{\isacharcomma}{\kern0pt}g{\isacharcomma}{\kern0pt}m{\isacharcomma}{\kern0pt}p{\isacharparenright}{\kern0pt}{\isachardoublequoteclose}\isanewline
\ \ \isacommand{have}\isamarkupfalse%
\ {\isachardoublequoteopen}nth{\isacharparenleft}{\kern0pt}x{\isacharcomma}{\kern0pt}\ env{\isacharparenright}{\kern0pt}\ {\isacharequal}{\kern0pt}\ nth{\isacharparenleft}{\kern0pt}{\isacharquery}{\kern0pt}h{\isacharbackquote}{\kern0pt}x{\isacharcomma}{\kern0pt}env{\isacharprime}{\kern0pt}{\isacharparenright}{\kern0pt}{\isachardoublequoteclose}\ \isakeyword{if}\ {\isachardoublequoteopen}x\ {\isasymin}\ m\ {\isasymunion}\ p{\isachardoublequoteclose}\ \isakeyword{for}\ x\isanewline
\ \ \ \ \isacommand{using}\isamarkupfalse%
\ that\isanewline
\ \ \isacommand{proof}\isamarkupfalse%
\ {\isacharparenleft}{\kern0pt}cases\ {\isachardoublequoteopen}x{\isasymin}m{\isachardoublequoteclose}{\isacharparenright}{\kern0pt}\isanewline
\ \ \ \ \isacommand{case}\isamarkupfalse%
\ True\isanewline
\ \ \ \ \isacommand{with}\isamarkupfalse%
\ {\isacartoucheopen}x{\isasymin}m{\isacartoucheclose}\isanewline
\ \ \ \ \isacommand{have}\isamarkupfalse%
\ {\isachardoublequoteopen}{\isacharquery}{\kern0pt}h{\isacharbackquote}{\kern0pt}x\ {\isacharequal}{\kern0pt}\ f{\isacharbackquote}{\kern0pt}x{\isachardoublequoteclose}\isanewline
\ \ \ \ \ \ \isacommand{unfolding}\isamarkupfalse%
\ union{\isacharunderscore}{\kern0pt}fun{\isacharunderscore}{\kern0pt}def\ \ beta\ \isacommand{by}\isamarkupfalse%
\ simp\isanewline
\ \ \ \ \isacommand{with}\isamarkupfalse%
\ assms\ {\isacartoucheopen}x{\isasymin}m{\isacartoucheclose}\isanewline
\ \ \ \ \isacommand{have}\isamarkupfalse%
\ {\isachardoublequoteopen}nth{\isacharparenleft}{\kern0pt}x{\isacharcomma}{\kern0pt}env{\isacharparenright}{\kern0pt}\ {\isacharequal}{\kern0pt}\ nth{\isacharparenleft}{\kern0pt}{\isacharquery}{\kern0pt}h{\isacharbackquote}{\kern0pt}x{\isacharcomma}{\kern0pt}env{\isacharprime}{\kern0pt}{\isacharparenright}{\kern0pt}{\isachardoublequoteclose}\ \isacommand{by}\isamarkupfalse%
\ simp\isanewline
\ \ \ \ \isacommand{then}\isamarkupfalse%
\ \isacommand{show}\isamarkupfalse%
\ {\isacharquery}{\kern0pt}thesis\ \isacommand{{\isachardot}{\kern0pt}}\isamarkupfalse%
\isanewline
\ \ \isacommand{next}\isamarkupfalse%
\isanewline
\ \ \ \ \isacommand{case}\isamarkupfalse%
\ False\isanewline
\ \ \ \ \isacommand{with}\isamarkupfalse%
\ {\isacartoucheopen}x\ {\isasymin}\ m\ {\isasymunion}\ p{\isacartoucheclose}\isanewline
\ \ \ \ \isacommand{have}\isamarkupfalse%
\isanewline
\ \ \ \ \ \ {\isachardoublequoteopen}x\ {\isasymin}\ p{\isachardoublequoteclose}\ {\isachardoublequoteopen}x{\isasymnotin}m{\isachardoublequoteclose}\ \ \isacommand{by}\isamarkupfalse%
\ auto\isanewline
\ \ \ \ \isacommand{then}\isamarkupfalse%
\isanewline
\ \ \ \ \isacommand{have}\isamarkupfalse%
\ {\isachardoublequoteopen}{\isacharquery}{\kern0pt}h{\isacharbackquote}{\kern0pt}x\ {\isacharequal}{\kern0pt}\ g{\isacharbackquote}{\kern0pt}x{\isachardoublequoteclose}\isanewline
\ \ \ \ \ \ \isacommand{unfolding}\isamarkupfalse%
\ union{\isacharunderscore}{\kern0pt}fun{\isacharunderscore}{\kern0pt}def\ beta\ \isacommand{by}\isamarkupfalse%
\ simp\isanewline
\ \ \ \ \isacommand{with}\isamarkupfalse%
\ assms\ {\isacartoucheopen}x{\isasymin}p{\isacartoucheclose}\isanewline
\ \ \ \ \isacommand{have}\isamarkupfalse%
\ {\isachardoublequoteopen}nth{\isacharparenleft}{\kern0pt}x{\isacharcomma}{\kern0pt}env{\isacharparenright}{\kern0pt}\ {\isacharequal}{\kern0pt}\ nth{\isacharparenleft}{\kern0pt}{\isacharquery}{\kern0pt}h{\isacharbackquote}{\kern0pt}x{\isacharcomma}{\kern0pt}env{\isacharprime}{\kern0pt}{\isacharparenright}{\kern0pt}{\isachardoublequoteclose}\ \isacommand{by}\isamarkupfalse%
\ simp\isanewline
\ \ \ \ \isacommand{then}\isamarkupfalse%
\ \isacommand{show}\isamarkupfalse%
\ {\isacharquery}{\kern0pt}thesis\ \isacommand{{\isachardot}{\kern0pt}}\isamarkupfalse%
\isanewline
\ \ \isacommand{qed}\isamarkupfalse%
\isanewline
\ \ \isacommand{then}\isamarkupfalse%
\ \isacommand{show}\isamarkupfalse%
\ {\isacharquery}{\kern0pt}thesis\ \isacommand{by}\isamarkupfalse%
\ simp\isanewline
\isacommand{qed}\isamarkupfalse%
%
\endisatagproof
{\isafoldproof}%
%
\isadelimproof
\isanewline
%
\endisadelimproof
\isanewline
\isanewline
\isacommand{lemma}\isamarkupfalse%
\ id{\isacharunderscore}{\kern0pt}fn{\isacharunderscore}{\kern0pt}type\ {\isacharcolon}{\kern0pt}\isanewline
\ \ \isakeyword{assumes}\ {\isachardoublequoteopen}n\ {\isasymin}\ nat{\isachardoublequoteclose}\isanewline
\ \ \isakeyword{shows}\ {\isachardoublequoteopen}id{\isacharparenleft}{\kern0pt}n{\isacharparenright}{\kern0pt}\ {\isasymin}\ n\ {\isasymrightarrow}\ n{\isachardoublequoteclose}\isanewline
%
\isadelimproof
\ \ %
\endisadelimproof
%
\isatagproof
\isacommand{unfolding}\isamarkupfalse%
\ id{\isacharunderscore}{\kern0pt}def\ \isacommand{using}\isamarkupfalse%
\ {\isacartoucheopen}n{\isasymin}nat{\isacartoucheclose}\ \isacommand{by}\isamarkupfalse%
\ simp%
\endisatagproof
{\isafoldproof}%
%
\isadelimproof
\isanewline
%
\endisadelimproof
\isanewline
\isacommand{lemma}\isamarkupfalse%
\ id{\isacharunderscore}{\kern0pt}fn{\isacharunderscore}{\kern0pt}action{\isacharcolon}{\kern0pt}\isanewline
\ \ \isakeyword{assumes}\ {\isachardoublequoteopen}n\ {\isasymin}\ nat{\isachardoublequoteclose}\ {\isachardoublequoteopen}env{\isasymin}list{\isacharparenleft}{\kern0pt}M{\isacharparenright}{\kern0pt}{\isachardoublequoteclose}\isanewline
\ \ \isakeyword{shows}\ {\isachardoublequoteopen}{\isasymAnd}\ j\ {\isachardot}{\kern0pt}\ j\ {\isacharless}{\kern0pt}\ n\ {\isasymLongrightarrow}\ nth{\isacharparenleft}{\kern0pt}j{\isacharcomma}{\kern0pt}env{\isacharparenright}{\kern0pt}\ {\isacharequal}{\kern0pt}\ nth{\isacharparenleft}{\kern0pt}id{\isacharparenleft}{\kern0pt}n{\isacharparenright}{\kern0pt}{\isacharbackquote}{\kern0pt}j{\isacharcomma}{\kern0pt}env{\isacharparenright}{\kern0pt}{\isachardoublequoteclose}\isanewline
%
\isadelimproof
%
\endisadelimproof
%
\isatagproof
\isacommand{proof}\isamarkupfalse%
\ {\isacharminus}{\kern0pt}\isanewline
\ \ \isacommand{show}\isamarkupfalse%
\ {\isachardoublequoteopen}nth{\isacharparenleft}{\kern0pt}j{\isacharcomma}{\kern0pt}env{\isacharparenright}{\kern0pt}\ {\isacharequal}{\kern0pt}\ nth{\isacharparenleft}{\kern0pt}id{\isacharparenleft}{\kern0pt}n{\isacharparenright}{\kern0pt}{\isacharbackquote}{\kern0pt}j{\isacharcomma}{\kern0pt}env{\isacharparenright}{\kern0pt}{\isachardoublequoteclose}\ \isakeyword{if}\ {\isachardoublequoteopen}j\ {\isacharless}{\kern0pt}\ n{\isachardoublequoteclose}\ \isakeyword{for}\ j\ \isacommand{using}\isamarkupfalse%
\ that\ {\isacartoucheopen}n{\isasymin}nat{\isacartoucheclose}\ ltD\ \isacommand{by}\isamarkupfalse%
\ simp\isanewline
\isacommand{qed}\isamarkupfalse%
%
\endisatagproof
{\isafoldproof}%
%
\isadelimproof
\isanewline
%
\endisadelimproof
\isanewline
\isanewline
\isacommand{definition}\isamarkupfalse%
\isanewline
\ \ sum\ {\isacharcolon}{\kern0pt}{\isacharcolon}{\kern0pt}\ {\isachardoublequoteopen}{\isacharbrackleft}{\kern0pt}i{\isacharcomma}{\kern0pt}i{\isacharcomma}{\kern0pt}i{\isacharcomma}{\kern0pt}i{\isacharcomma}{\kern0pt}i{\isacharbrackright}{\kern0pt}\ {\isasymRightarrow}\ i{\isachardoublequoteclose}\ \isakeyword{where}\isanewline
\ \ {\isachardoublequoteopen}sum{\isacharparenleft}{\kern0pt}f{\isacharcomma}{\kern0pt}g{\isacharcomma}{\kern0pt}m{\isacharcomma}{\kern0pt}n{\isacharcomma}{\kern0pt}p{\isacharparenright}{\kern0pt}\ {\isasymequiv}\ {\isasymlambda}j\ {\isasymin}\ m{\isacharhash}{\kern0pt}{\isacharplus}{\kern0pt}p\ \ {\isachardot}{\kern0pt}\ if\ j{\isacharless}{\kern0pt}m\ then\ f{\isacharbackquote}{\kern0pt}j\ else\ {\isacharparenleft}{\kern0pt}g{\isacharbackquote}{\kern0pt}{\isacharparenleft}{\kern0pt}j{\isacharhash}{\kern0pt}{\isacharminus}{\kern0pt}m{\isacharparenright}{\kern0pt}{\isacharparenright}{\kern0pt}{\isacharhash}{\kern0pt}{\isacharplus}{\kern0pt}n{\isachardoublequoteclose}\isanewline
\isanewline
\isacommand{lemma}\isamarkupfalse%
\ sum{\isacharunderscore}{\kern0pt}inl{\isacharcolon}{\kern0pt}\isanewline
\ \ \isakeyword{assumes}\ {\isachardoublequoteopen}m\ {\isasymin}\ nat{\isachardoublequoteclose}\ {\isachardoublequoteopen}n{\isasymin}nat{\isachardoublequoteclose}\isanewline
\ \ \ \ {\isachardoublequoteopen}f\ {\isasymin}\ m{\isasymrightarrow}n{\isachardoublequoteclose}\ {\isachardoublequoteopen}x\ {\isasymin}\ m{\isachardoublequoteclose}\isanewline
\ \ \isakeyword{shows}\ {\isachardoublequoteopen}sum{\isacharparenleft}{\kern0pt}f{\isacharcomma}{\kern0pt}g{\isacharcomma}{\kern0pt}m{\isacharcomma}{\kern0pt}n{\isacharcomma}{\kern0pt}p{\isacharparenright}{\kern0pt}{\isacharbackquote}{\kern0pt}x\ {\isacharequal}{\kern0pt}\ f{\isacharbackquote}{\kern0pt}x{\isachardoublequoteclose}\isanewline
%
\isadelimproof
%
\endisadelimproof
%
\isatagproof
\isacommand{proof}\isamarkupfalse%
\ {\isacharminus}{\kern0pt}\isanewline
\ \ \isacommand{from}\isamarkupfalse%
\ {\isacartoucheopen}m{\isasymin}nat{\isacartoucheclose}\isanewline
\ \ \isacommand{have}\isamarkupfalse%
\ {\isachardoublequoteopen}m{\isasymle}m{\isacharhash}{\kern0pt}{\isacharplus}{\kern0pt}p{\isachardoublequoteclose}\isanewline
\ \ \ \ \isacommand{using}\isamarkupfalse%
\ add{\isacharunderscore}{\kern0pt}le{\isacharunderscore}{\kern0pt}self{\isacharbrackleft}{\kern0pt}of\ m{\isacharbrackright}{\kern0pt}\ \isacommand{by}\isamarkupfalse%
\ simp\isanewline
\ \ \isacommand{with}\isamarkupfalse%
\ assms\isanewline
\ \ \isacommand{have}\isamarkupfalse%
\ {\isachardoublequoteopen}x{\isasymin}m{\isacharhash}{\kern0pt}{\isacharplus}{\kern0pt}p{\isachardoublequoteclose}\isanewline
\ \ \ \ \isacommand{using}\isamarkupfalse%
\ ltI{\isacharbrackleft}{\kern0pt}of\ x\ m{\isacharbrackright}{\kern0pt}\ lt{\isacharunderscore}{\kern0pt}trans{\isadigit{2}}{\isacharbrackleft}{\kern0pt}of\ x\ m\ {\isachardoublequoteopen}m{\isacharhash}{\kern0pt}{\isacharplus}{\kern0pt}p{\isachardoublequoteclose}{\isacharbrackright}{\kern0pt}\ ltD\ \isacommand{by}\isamarkupfalse%
\ simp\isanewline
\ \ \isacommand{from}\isamarkupfalse%
\ assms\isanewline
\ \ \isacommand{have}\isamarkupfalse%
\ {\isachardoublequoteopen}x{\isacharless}{\kern0pt}m{\isachardoublequoteclose}\isanewline
\ \ \ \ \isacommand{using}\isamarkupfalse%
\ ltI\ \isacommand{by}\isamarkupfalse%
\ simp\isanewline
\ \ \isacommand{with}\isamarkupfalse%
\ {\isacartoucheopen}x{\isasymin}m{\isacharhash}{\kern0pt}{\isacharplus}{\kern0pt}p{\isacartoucheclose}\isanewline
\ \ \isacommand{show}\isamarkupfalse%
\ {\isacharquery}{\kern0pt}thesis\ \isacommand{unfolding}\isamarkupfalse%
\ sum{\isacharunderscore}{\kern0pt}def\ \isacommand{by}\isamarkupfalse%
\ simp\isanewline
\isacommand{qed}\isamarkupfalse%
%
\endisatagproof
{\isafoldproof}%
%
\isadelimproof
\isanewline
%
\endisadelimproof
\isanewline
\isacommand{lemma}\isamarkupfalse%
\ sum{\isacharunderscore}{\kern0pt}inr{\isacharcolon}{\kern0pt}\isanewline
\ \ \isakeyword{assumes}\ {\isachardoublequoteopen}m\ {\isasymin}\ nat{\isachardoublequoteclose}\ {\isachardoublequoteopen}n{\isasymin}nat{\isachardoublequoteclose}\ {\isachardoublequoteopen}p{\isasymin}nat{\isachardoublequoteclose}\isanewline
\ \ \ \ {\isachardoublequoteopen}g{\isasymin}p{\isasymrightarrow}q{\isachardoublequoteclose}\ {\isachardoublequoteopen}m\ {\isasymle}\ x{\isachardoublequoteclose}\ {\isachardoublequoteopen}x\ {\isacharless}{\kern0pt}\ m{\isacharhash}{\kern0pt}{\isacharplus}{\kern0pt}p{\isachardoublequoteclose}\isanewline
\ \ \isakeyword{shows}\ {\isachardoublequoteopen}sum{\isacharparenleft}{\kern0pt}f{\isacharcomma}{\kern0pt}g{\isacharcomma}{\kern0pt}m{\isacharcomma}{\kern0pt}n{\isacharcomma}{\kern0pt}p{\isacharparenright}{\kern0pt}{\isacharbackquote}{\kern0pt}x\ {\isacharequal}{\kern0pt}\ g{\isacharbackquote}{\kern0pt}{\isacharparenleft}{\kern0pt}x{\isacharhash}{\kern0pt}{\isacharminus}{\kern0pt}m{\isacharparenright}{\kern0pt}{\isacharhash}{\kern0pt}{\isacharplus}{\kern0pt}n{\isachardoublequoteclose}\isanewline
%
\isadelimproof
%
\endisadelimproof
%
\isatagproof
\isacommand{proof}\isamarkupfalse%
\ {\isacharminus}{\kern0pt}\isanewline
\ \ \isacommand{from}\isamarkupfalse%
\ assms\isanewline
\ \ \isacommand{have}\isamarkupfalse%
\ {\isachardoublequoteopen}x{\isasymin}nat{\isachardoublequoteclose}\isanewline
\ \ \ \ \isacommand{using}\isamarkupfalse%
\ in{\isacharunderscore}{\kern0pt}n{\isacharunderscore}{\kern0pt}in{\isacharunderscore}{\kern0pt}nat{\isacharbrackleft}{\kern0pt}of\ {\isachardoublequoteopen}m{\isacharhash}{\kern0pt}{\isacharplus}{\kern0pt}p{\isachardoublequoteclose}{\isacharbrackright}{\kern0pt}\ ltD\isanewline
\ \ \ \ \isacommand{by}\isamarkupfalse%
\ simp\isanewline
\ \ \isacommand{with}\isamarkupfalse%
\ assms\isanewline
\ \ \isacommand{have}\isamarkupfalse%
\ {\isachardoublequoteopen}{\isasymnot}\ x{\isacharless}{\kern0pt}m{\isachardoublequoteclose}\isanewline
\ \ \ \ \isacommand{using}\isamarkupfalse%
\ not{\isacharunderscore}{\kern0pt}lt{\isacharunderscore}{\kern0pt}iff{\isacharunderscore}{\kern0pt}le{\isacharbrackleft}{\kern0pt}THEN\ iffD{\isadigit{2}}{\isacharbrackright}{\kern0pt}\ \isacommand{by}\isamarkupfalse%
\ simp\isanewline
\ \ \isacommand{from}\isamarkupfalse%
\ assms\isanewline
\ \ \isacommand{have}\isamarkupfalse%
\ {\isachardoublequoteopen}x{\isasymin}m{\isacharhash}{\kern0pt}{\isacharplus}{\kern0pt}p{\isachardoublequoteclose}\isanewline
\ \ \ \ \isacommand{using}\isamarkupfalse%
\ ltD\ \isacommand{by}\isamarkupfalse%
\ simp\isanewline
\ \ \isacommand{with}\isamarkupfalse%
\ {\isacartoucheopen}{\isasymnot}\ x{\isacharless}{\kern0pt}m{\isacartoucheclose}\isanewline
\ \ \isacommand{show}\isamarkupfalse%
\ {\isacharquery}{\kern0pt}thesis\ \isacommand{unfolding}\isamarkupfalse%
\ sum{\isacharunderscore}{\kern0pt}def\ \isacommand{by}\isamarkupfalse%
\ simp\isanewline
\isacommand{qed}\isamarkupfalse%
%
\endisatagproof
{\isafoldproof}%
%
\isadelimproof
\isanewline
%
\endisadelimproof
\isanewline
\isanewline
\isacommand{lemma}\isamarkupfalse%
\ sum{\isacharunderscore}{\kern0pt}action\ {\isacharcolon}{\kern0pt}\isanewline
\ \ \isakeyword{assumes}\ {\isachardoublequoteopen}m\ {\isasymin}\ nat{\isachardoublequoteclose}\ {\isachardoublequoteopen}n{\isasymin}nat{\isachardoublequoteclose}\ {\isachardoublequoteopen}p{\isasymin}nat{\isachardoublequoteclose}\ {\isachardoublequoteopen}q{\isasymin}nat{\isachardoublequoteclose}\isanewline
\ \ \ \ {\isachardoublequoteopen}f\ {\isasymin}\ m{\isasymrightarrow}n{\isachardoublequoteclose}\ {\isachardoublequoteopen}g{\isasymin}p{\isasymrightarrow}q{\isachardoublequoteclose}\isanewline
\ \ \ \ {\isachardoublequoteopen}env\ {\isasymin}\ list{\isacharparenleft}{\kern0pt}M{\isacharparenright}{\kern0pt}{\isachardoublequoteclose}\isanewline
\ \ \ \ {\isachardoublequoteopen}env{\isacharprime}{\kern0pt}\ {\isasymin}\ list{\isacharparenleft}{\kern0pt}M{\isacharparenright}{\kern0pt}{\isachardoublequoteclose}\isanewline
\ \ \ \ {\isachardoublequoteopen}env{\isadigit{1}}\ {\isasymin}\ list{\isacharparenleft}{\kern0pt}M{\isacharparenright}{\kern0pt}{\isachardoublequoteclose}\isanewline
\ \ \ \ {\isachardoublequoteopen}env{\isadigit{2}}\ {\isasymin}\ list{\isacharparenleft}{\kern0pt}M{\isacharparenright}{\kern0pt}{\isachardoublequoteclose}\isanewline
\ \ \ \ {\isachardoublequoteopen}length{\isacharparenleft}{\kern0pt}env{\isacharparenright}{\kern0pt}\ {\isacharequal}{\kern0pt}\ m{\isachardoublequoteclose}\isanewline
\ \ \ \ {\isachardoublequoteopen}length{\isacharparenleft}{\kern0pt}env{\isadigit{1}}{\isacharparenright}{\kern0pt}\ {\isacharequal}{\kern0pt}\ p{\isachardoublequoteclose}\isanewline
\ \ \ \ {\isachardoublequoteopen}length{\isacharparenleft}{\kern0pt}env{\isacharprime}{\kern0pt}{\isacharparenright}{\kern0pt}\ {\isacharequal}{\kern0pt}\ n{\isachardoublequoteclose}\isanewline
\ \ \ \ {\isachardoublequoteopen}{\isasymAnd}\ i\ {\isachardot}{\kern0pt}\ i\ {\isacharless}{\kern0pt}\ m\ {\isasymLongrightarrow}\ nth{\isacharparenleft}{\kern0pt}i{\isacharcomma}{\kern0pt}env{\isacharparenright}{\kern0pt}\ {\isacharequal}{\kern0pt}\ nth{\isacharparenleft}{\kern0pt}f{\isacharbackquote}{\kern0pt}i{\isacharcomma}{\kern0pt}env{\isacharprime}{\kern0pt}{\isacharparenright}{\kern0pt}{\isachardoublequoteclose}\isanewline
\ \ \ \ {\isachardoublequoteopen}{\isasymAnd}\ j{\isachardot}{\kern0pt}\ j\ {\isacharless}{\kern0pt}\ p\ {\isasymLongrightarrow}\ nth{\isacharparenleft}{\kern0pt}j{\isacharcomma}{\kern0pt}env{\isadigit{1}}{\isacharparenright}{\kern0pt}\ {\isacharequal}{\kern0pt}\ nth{\isacharparenleft}{\kern0pt}g{\isacharbackquote}{\kern0pt}j{\isacharcomma}{\kern0pt}env{\isadigit{2}}{\isacharparenright}{\kern0pt}{\isachardoublequoteclose}\isanewline
\ \ \isakeyword{shows}\ {\isachardoublequoteopen}{\isasymforall}\ i\ {\isachardot}{\kern0pt}\ i\ {\isacharless}{\kern0pt}\ m{\isacharhash}{\kern0pt}{\isacharplus}{\kern0pt}p\ {\isasymlongrightarrow}\isanewline
\ \ \ \ \ \ \ \ \ \ nth{\isacharparenleft}{\kern0pt}i{\isacharcomma}{\kern0pt}env{\isacharat}{\kern0pt}env{\isadigit{1}}{\isacharparenright}{\kern0pt}\ {\isacharequal}{\kern0pt}\ nth{\isacharparenleft}{\kern0pt}sum{\isacharparenleft}{\kern0pt}f{\isacharcomma}{\kern0pt}g{\isacharcomma}{\kern0pt}m{\isacharcomma}{\kern0pt}n{\isacharcomma}{\kern0pt}p{\isacharparenright}{\kern0pt}{\isacharbackquote}{\kern0pt}i{\isacharcomma}{\kern0pt}env{\isacharprime}{\kern0pt}{\isacharat}{\kern0pt}env{\isadigit{2}}{\isacharparenright}{\kern0pt}{\isachardoublequoteclose}\isanewline
%
\isadelimproof
%
\endisadelimproof
%
\isatagproof
\isacommand{proof}\isamarkupfalse%
\ {\isacharminus}{\kern0pt}\isanewline
\ \ \isacommand{let}\isamarkupfalse%
\ {\isacharquery}{\kern0pt}h\ {\isacharequal}{\kern0pt}\ {\isachardoublequoteopen}sum{\isacharparenleft}{\kern0pt}f{\isacharcomma}{\kern0pt}g{\isacharcomma}{\kern0pt}m{\isacharcomma}{\kern0pt}n{\isacharcomma}{\kern0pt}p{\isacharparenright}{\kern0pt}{\isachardoublequoteclose}\isanewline
\ \ \isacommand{from}\isamarkupfalse%
\ {\isacartoucheopen}m{\isasymin}nat{\isacartoucheclose}\ {\isacartoucheopen}n{\isasymin}nat{\isacartoucheclose}\ {\isacartoucheopen}q{\isasymin}nat{\isacartoucheclose}\isanewline
\ \ \isacommand{have}\isamarkupfalse%
\ {\isachardoublequoteopen}m{\isasymle}m{\isacharhash}{\kern0pt}{\isacharplus}{\kern0pt}p{\isachardoublequoteclose}\ {\isachardoublequoteopen}n{\isasymle}n{\isacharhash}{\kern0pt}{\isacharplus}{\kern0pt}q{\isachardoublequoteclose}\ {\isachardoublequoteopen}q{\isasymle}n{\isacharhash}{\kern0pt}{\isacharplus}{\kern0pt}q{\isachardoublequoteclose}\isanewline
\ \ \ \ \isacommand{using}\isamarkupfalse%
\ add{\isacharunderscore}{\kern0pt}le{\isacharunderscore}{\kern0pt}self{\isacharbrackleft}{\kern0pt}of\ m{\isacharbrackright}{\kern0pt}\ \ add{\isacharunderscore}{\kern0pt}le{\isacharunderscore}{\kern0pt}self{\isadigit{2}}{\isacharbrackleft}{\kern0pt}of\ n\ q{\isacharbrackright}{\kern0pt}\ \isacommand{by}\isamarkupfalse%
\ simp{\isacharunderscore}{\kern0pt}all\isanewline
\ \ \isacommand{from}\isamarkupfalse%
\ {\isacartoucheopen}p{\isasymin}nat{\isacartoucheclose}\isanewline
\ \ \isacommand{have}\isamarkupfalse%
\ {\isachardoublequoteopen}p\ {\isacharequal}{\kern0pt}\ {\isacharparenleft}{\kern0pt}m{\isacharhash}{\kern0pt}{\isacharplus}{\kern0pt}p{\isacharparenright}{\kern0pt}{\isacharhash}{\kern0pt}{\isacharminus}{\kern0pt}m{\isachardoublequoteclose}\ \isacommand{using}\isamarkupfalse%
\ diff{\isacharunderscore}{\kern0pt}add{\isacharunderscore}{\kern0pt}inverse{\isadigit{2}}\ \isacommand{by}\isamarkupfalse%
\ simp\isanewline
\ \ \isacommand{have}\isamarkupfalse%
\ {\isachardoublequoteopen}nth{\isacharparenleft}{\kern0pt}x{\isacharcomma}{\kern0pt}\ env\ {\isacharat}{\kern0pt}\ env{\isadigit{1}}{\isacharparenright}{\kern0pt}\ {\isacharequal}{\kern0pt}\ nth{\isacharparenleft}{\kern0pt}{\isacharquery}{\kern0pt}h{\isacharbackquote}{\kern0pt}x{\isacharcomma}{\kern0pt}env{\isacharprime}{\kern0pt}{\isacharat}{\kern0pt}env{\isadigit{2}}{\isacharparenright}{\kern0pt}{\isachardoublequoteclose}\ \isakeyword{if}\ {\isachardoublequoteopen}x{\isacharless}{\kern0pt}m{\isacharhash}{\kern0pt}{\isacharplus}{\kern0pt}p{\isachardoublequoteclose}\ \isakeyword{for}\ x\isanewline
\ \ \isacommand{proof}\isamarkupfalse%
\ {\isacharparenleft}{\kern0pt}cases\ {\isachardoublequoteopen}x{\isacharless}{\kern0pt}m{\isachardoublequoteclose}{\isacharparenright}{\kern0pt}\isanewline
\ \ \ \ \isacommand{case}\isamarkupfalse%
\ True\isanewline
\ \ \ \ \isacommand{then}\isamarkupfalse%
\isanewline
\ \ \ \ \isacommand{have}\isamarkupfalse%
\ {\isadigit{2}}{\isacharcolon}{\kern0pt}\ {\isachardoublequoteopen}{\isacharquery}{\kern0pt}h{\isacharbackquote}{\kern0pt}x{\isacharequal}{\kern0pt}\ f{\isacharbackquote}{\kern0pt}x{\isachardoublequoteclose}\ {\isachardoublequoteopen}x{\isasymin}m{\isachardoublequoteclose}\ {\isachardoublequoteopen}f{\isacharbackquote}{\kern0pt}x\ {\isasymin}\ n{\isachardoublequoteclose}\ {\isachardoublequoteopen}x{\isasymin}nat{\isachardoublequoteclose}\isanewline
\ \ \ \ \ \ \isacommand{using}\isamarkupfalse%
\ assms\ sum{\isacharunderscore}{\kern0pt}inl\ ltD\ apply{\isacharunderscore}{\kern0pt}type{\isacharbrackleft}{\kern0pt}of\ f\ m\ {\isacharunderscore}{\kern0pt}\ x{\isacharbrackright}{\kern0pt}\ in{\isacharunderscore}{\kern0pt}n{\isacharunderscore}{\kern0pt}in{\isacharunderscore}{\kern0pt}nat\ \isacommand{by}\isamarkupfalse%
\ simp{\isacharunderscore}{\kern0pt}all\isanewline
\ \ \ \ \isacommand{with}\isamarkupfalse%
\ {\isacartoucheopen}x{\isacharless}{\kern0pt}m{\isacartoucheclose}\ assms\isanewline
\ \ \ \ \isacommand{have}\isamarkupfalse%
\ {\isachardoublequoteopen}f{\isacharbackquote}{\kern0pt}x\ {\isacharless}{\kern0pt}\ n{\isachardoublequoteclose}\ {\isachardoublequoteopen}f{\isacharbackquote}{\kern0pt}x{\isacharless}{\kern0pt}length{\isacharparenleft}{\kern0pt}env{\isacharprime}{\kern0pt}{\isacharparenright}{\kern0pt}{\isachardoublequoteclose}\ \ {\isachardoublequoteopen}f{\isacharbackquote}{\kern0pt}x{\isasymin}nat{\isachardoublequoteclose}\isanewline
\ \ \ \ \ \ \isacommand{using}\isamarkupfalse%
\ ltI\ in{\isacharunderscore}{\kern0pt}n{\isacharunderscore}{\kern0pt}in{\isacharunderscore}{\kern0pt}nat\ \isacommand{by}\isamarkupfalse%
\ simp{\isacharunderscore}{\kern0pt}all\isanewline
\ \ \ \ \isacommand{with}\isamarkupfalse%
\ {\isadigit{2}}\ {\isacartoucheopen}x{\isacharless}{\kern0pt}m{\isacartoucheclose}\ assms\isanewline
\ \ \ \ \isacommand{have}\isamarkupfalse%
\ {\isachardoublequoteopen}nth{\isacharparenleft}{\kern0pt}x{\isacharcomma}{\kern0pt}env{\isacharat}{\kern0pt}env{\isadigit{1}}{\isacharparenright}{\kern0pt}\ {\isacharequal}{\kern0pt}\ nth{\isacharparenleft}{\kern0pt}x{\isacharcomma}{\kern0pt}env{\isacharparenright}{\kern0pt}{\isachardoublequoteclose}\isanewline
\ \ \ \ \ \ \isacommand{using}\isamarkupfalse%
\ nth{\isacharunderscore}{\kern0pt}append{\isacharbrackleft}{\kern0pt}OF\ {\isacartoucheopen}env{\isasymin}list{\isacharparenleft}{\kern0pt}M{\isacharparenright}{\kern0pt}{\isacartoucheclose}{\isacharbrackright}{\kern0pt}\ {\isacartoucheopen}x{\isasymin}nat{\isacartoucheclose}\ \isacommand{by}\isamarkupfalse%
\ simp\isanewline
\ \ \ \ \isacommand{also}\isamarkupfalse%
\isanewline
\ \ \ \ \isacommand{have}\isamarkupfalse%
\isanewline
\ \ \ \ \ \ {\isachardoublequoteopen}{\isachardot}{\kern0pt}{\isachardot}{\kern0pt}{\isachardot}{\kern0pt}\ {\isacharequal}{\kern0pt}\ nth{\isacharparenleft}{\kern0pt}f{\isacharbackquote}{\kern0pt}x{\isacharcomma}{\kern0pt}env{\isacharprime}{\kern0pt}{\isacharparenright}{\kern0pt}{\isachardoublequoteclose}\isanewline
\ \ \ \ \ \ \isacommand{using}\isamarkupfalse%
\ {\isadigit{2}}\ {\isacartoucheopen}x{\isacharless}{\kern0pt}m{\isacartoucheclose}\ assms\ \isacommand{by}\isamarkupfalse%
\ simp\isanewline
\ \ \ \ \isacommand{also}\isamarkupfalse%
\isanewline
\ \ \ \ \isacommand{have}\isamarkupfalse%
\ {\isachardoublequoteopen}{\isachardot}{\kern0pt}{\isachardot}{\kern0pt}{\isachardot}{\kern0pt}\ {\isacharequal}{\kern0pt}\ nth{\isacharparenleft}{\kern0pt}f{\isacharbackquote}{\kern0pt}x{\isacharcomma}{\kern0pt}env{\isacharprime}{\kern0pt}{\isacharat}{\kern0pt}env{\isadigit{2}}{\isacharparenright}{\kern0pt}{\isachardoublequoteclose}\isanewline
\ \ \ \ \ \ \isacommand{using}\isamarkupfalse%
\ nth{\isacharunderscore}{\kern0pt}append{\isacharbrackleft}{\kern0pt}OF\ {\isacartoucheopen}env{\isacharprime}{\kern0pt}{\isasymin}list{\isacharparenleft}{\kern0pt}M{\isacharparenright}{\kern0pt}{\isacartoucheclose}{\isacharbrackright}{\kern0pt}\ {\isacartoucheopen}f{\isacharbackquote}{\kern0pt}x{\isacharless}{\kern0pt}length{\isacharparenleft}{\kern0pt}env{\isacharprime}{\kern0pt}{\isacharparenright}{\kern0pt}{\isacartoucheclose}\ {\isacartoucheopen}f{\isacharbackquote}{\kern0pt}x\ {\isasymin}nat{\isacartoucheclose}\ \isacommand{by}\isamarkupfalse%
\ simp\isanewline
\ \ \ \ \isacommand{also}\isamarkupfalse%
\isanewline
\ \ \ \ \isacommand{have}\isamarkupfalse%
\ {\isachardoublequoteopen}{\isachardot}{\kern0pt}{\isachardot}{\kern0pt}{\isachardot}{\kern0pt}\ {\isacharequal}{\kern0pt}\ nth{\isacharparenleft}{\kern0pt}{\isacharquery}{\kern0pt}h{\isacharbackquote}{\kern0pt}x{\isacharcomma}{\kern0pt}env{\isacharprime}{\kern0pt}{\isacharat}{\kern0pt}env{\isadigit{2}}{\isacharparenright}{\kern0pt}{\isachardoublequoteclose}\isanewline
\ \ \ \ \ \ \isacommand{using}\isamarkupfalse%
\ {\isadigit{2}}\ \isacommand{by}\isamarkupfalse%
\ simp\isanewline
\ \ \ \ \isacommand{finally}\isamarkupfalse%
\isanewline
\ \ \ \ \isacommand{have}\isamarkupfalse%
\ {\isachardoublequoteopen}nth{\isacharparenleft}{\kern0pt}x{\isacharcomma}{\kern0pt}\ env\ {\isacharat}{\kern0pt}\ env{\isadigit{1}}{\isacharparenright}{\kern0pt}\ {\isacharequal}{\kern0pt}\ nth{\isacharparenleft}{\kern0pt}{\isacharquery}{\kern0pt}h{\isacharbackquote}{\kern0pt}x{\isacharcomma}{\kern0pt}env{\isacharprime}{\kern0pt}{\isacharat}{\kern0pt}env{\isadigit{2}}{\isacharparenright}{\kern0pt}{\isachardoublequoteclose}\ \isacommand{{\isachardot}{\kern0pt}}\isamarkupfalse%
\isanewline
\ \ \ \ \isacommand{then}\isamarkupfalse%
\ \isacommand{show}\isamarkupfalse%
\ {\isacharquery}{\kern0pt}thesis\ \isacommand{{\isachardot}{\kern0pt}}\isamarkupfalse%
\isanewline
\ \ \isacommand{next}\isamarkupfalse%
\isanewline
\ \ \ \ \isacommand{case}\isamarkupfalse%
\ False\isanewline
\ \ \ \ \isacommand{have}\isamarkupfalse%
\ {\isachardoublequoteopen}x{\isasymin}nat{\isachardoublequoteclose}\isanewline
\ \ \ \ \ \ \isacommand{using}\isamarkupfalse%
\ that\ in{\isacharunderscore}{\kern0pt}n{\isacharunderscore}{\kern0pt}in{\isacharunderscore}{\kern0pt}nat{\isacharbrackleft}{\kern0pt}of\ {\isachardoublequoteopen}m{\isacharhash}{\kern0pt}{\isacharplus}{\kern0pt}p{\isachardoublequoteclose}\ x{\isacharbrackright}{\kern0pt}\ ltD\ {\isacartoucheopen}p{\isasymin}nat{\isacartoucheclose}\ {\isacartoucheopen}m{\isasymin}nat{\isacartoucheclose}\ \isacommand{by}\isamarkupfalse%
\ simp\isanewline
\ \ \ \ \isacommand{with}\isamarkupfalse%
\ {\isacartoucheopen}length{\isacharparenleft}{\kern0pt}env{\isacharparenright}{\kern0pt}\ {\isacharequal}{\kern0pt}\ m{\isacartoucheclose}\isanewline
\ \ \ \ \isacommand{have}\isamarkupfalse%
\ {\isachardoublequoteopen}m{\isasymle}x{\isachardoublequoteclose}\ {\isachardoublequoteopen}length{\isacharparenleft}{\kern0pt}env{\isacharparenright}{\kern0pt}\ {\isasymle}\ x{\isachardoublequoteclose}\isanewline
\ \ \ \ \ \ \isacommand{using}\isamarkupfalse%
\ not{\isacharunderscore}{\kern0pt}lt{\isacharunderscore}{\kern0pt}iff{\isacharunderscore}{\kern0pt}le\ {\isacartoucheopen}m{\isasymin}nat{\isacartoucheclose}\ {\isacartoucheopen}{\isasymnot}x{\isacharless}{\kern0pt}m{\isacartoucheclose}\ \isacommand{by}\isamarkupfalse%
\ simp{\isacharunderscore}{\kern0pt}all\isanewline
\ \ \ \ \isacommand{with}\isamarkupfalse%
\ {\isacartoucheopen}{\isasymnot}x{\isacharless}{\kern0pt}m{\isacartoucheclose}\ {\isacartoucheopen}length{\isacharparenleft}{\kern0pt}env{\isacharparenright}{\kern0pt}\ {\isacharequal}{\kern0pt}\ m{\isacartoucheclose}\isanewline
\ \ \ \ \isacommand{have}\isamarkupfalse%
\ {\isadigit{2}}\ {\isacharcolon}{\kern0pt}\ {\isachardoublequoteopen}{\isacharquery}{\kern0pt}h{\isacharbackquote}{\kern0pt}x{\isacharequal}{\kern0pt}\ g{\isacharbackquote}{\kern0pt}{\isacharparenleft}{\kern0pt}x{\isacharhash}{\kern0pt}{\isacharminus}{\kern0pt}m{\isacharparenright}{\kern0pt}{\isacharhash}{\kern0pt}{\isacharplus}{\kern0pt}n{\isachardoublequoteclose}\ \ {\isachardoublequoteopen}{\isasymnot}\ x\ {\isacharless}{\kern0pt}length{\isacharparenleft}{\kern0pt}env{\isacharparenright}{\kern0pt}{\isachardoublequoteclose}\isanewline
\ \ \ \ \ \ \isacommand{unfolding}\isamarkupfalse%
\ sum{\isacharunderscore}{\kern0pt}def\isanewline
\ \ \ \ \ \ \isacommand{using}\isamarkupfalse%
\ \ sum{\isacharunderscore}{\kern0pt}inr\ that\ beta\ ltD\ \isacommand{by}\isamarkupfalse%
\ simp{\isacharunderscore}{\kern0pt}all\isanewline
\ \ \ \ \isacommand{from}\isamarkupfalse%
\ assms\ {\isacartoucheopen}x{\isasymin}nat{\isacartoucheclose}\ {\isacartoucheopen}p{\isacharequal}{\kern0pt}m{\isacharhash}{\kern0pt}{\isacharplus}{\kern0pt}p{\isacharhash}{\kern0pt}{\isacharminus}{\kern0pt}m{\isacartoucheclose}\isanewline
\ \ \ \ \isacommand{have}\isamarkupfalse%
\ {\isachardoublequoteopen}x{\isacharhash}{\kern0pt}{\isacharminus}{\kern0pt}m\ {\isacharless}{\kern0pt}\ p{\isachardoublequoteclose}\isanewline
\ \ \ \ \ \ \isacommand{using}\isamarkupfalse%
\ diff{\isacharunderscore}{\kern0pt}mono{\isacharbrackleft}{\kern0pt}OF\ {\isacharunderscore}{\kern0pt}\ {\isacharunderscore}{\kern0pt}\ {\isacharunderscore}{\kern0pt}\ {\isacartoucheopen}x{\isacharless}{\kern0pt}m{\isacharhash}{\kern0pt}{\isacharplus}{\kern0pt}p{\isacartoucheclose}\ {\isacartoucheopen}m{\isasymle}x{\isacartoucheclose}{\isacharbrackright}{\kern0pt}\ \isacommand{by}\isamarkupfalse%
\ simp\isanewline
\ \ \ \ \isacommand{then}\isamarkupfalse%
\ \isacommand{have}\isamarkupfalse%
\ {\isachardoublequoteopen}x{\isacharhash}{\kern0pt}{\isacharminus}{\kern0pt}m{\isasymin}p{\isachardoublequoteclose}\ \isacommand{using}\isamarkupfalse%
\ ltD\ \isacommand{by}\isamarkupfalse%
\ simp\isanewline
\ \ \ \ \isacommand{with}\isamarkupfalse%
\ {\isacartoucheopen}g{\isasymin}p{\isasymrightarrow}q{\isacartoucheclose}\isanewline
\ \ \ \ \isacommand{have}\isamarkupfalse%
\ {\isachardoublequoteopen}g{\isacharbackquote}{\kern0pt}{\isacharparenleft}{\kern0pt}x{\isacharhash}{\kern0pt}{\isacharminus}{\kern0pt}m{\isacharparenright}{\kern0pt}\ {\isasymin}\ q{\isachardoublequoteclose}\ \ \isacommand{by}\isamarkupfalse%
\ simp\isanewline
\ \ \ \ \isacommand{with}\isamarkupfalse%
\ {\isacartoucheopen}q{\isasymin}nat{\isacartoucheclose}\ {\isacartoucheopen}length{\isacharparenleft}{\kern0pt}env{\isacharprime}{\kern0pt}{\isacharparenright}{\kern0pt}\ {\isacharequal}{\kern0pt}\ n{\isacartoucheclose}\isanewline
\ \ \ \ \isacommand{have}\isamarkupfalse%
\ {\isachardoublequoteopen}g{\isacharbackquote}{\kern0pt}{\isacharparenleft}{\kern0pt}x{\isacharhash}{\kern0pt}{\isacharminus}{\kern0pt}m{\isacharparenright}{\kern0pt}\ {\isacharless}{\kern0pt}\ q{\isachardoublequoteclose}\ {\isachardoublequoteopen}g{\isacharbackquote}{\kern0pt}{\isacharparenleft}{\kern0pt}x{\isacharhash}{\kern0pt}{\isacharminus}{\kern0pt}m{\isacharparenright}{\kern0pt}{\isasymin}nat{\isachardoublequoteclose}\ \isacommand{using}\isamarkupfalse%
\ ltI\ in{\isacharunderscore}{\kern0pt}n{\isacharunderscore}{\kern0pt}in{\isacharunderscore}{\kern0pt}nat\ \isacommand{by}\isamarkupfalse%
\ simp{\isacharunderscore}{\kern0pt}all\isanewline
\ \ \ \ \isacommand{with}\isamarkupfalse%
\ {\isacartoucheopen}q{\isasymin}nat{\isacartoucheclose}\ {\isacartoucheopen}n{\isasymin}nat{\isacartoucheclose}\isanewline
\ \ \ \ \isacommand{have}\isamarkupfalse%
\ {\isachardoublequoteopen}{\isacharparenleft}{\kern0pt}g{\isacharbackquote}{\kern0pt}{\isacharparenleft}{\kern0pt}x{\isacharhash}{\kern0pt}{\isacharminus}{\kern0pt}m{\isacharparenright}{\kern0pt}{\isacharparenright}{\kern0pt}{\isacharhash}{\kern0pt}{\isacharplus}{\kern0pt}n\ {\isacharless}{\kern0pt}n{\isacharhash}{\kern0pt}{\isacharplus}{\kern0pt}q{\isachardoublequoteclose}\ {\isachardoublequoteopen}n\ {\isasymle}\ g{\isacharbackquote}{\kern0pt}{\isacharparenleft}{\kern0pt}x{\isacharhash}{\kern0pt}{\isacharminus}{\kern0pt}m{\isacharparenright}{\kern0pt}{\isacharhash}{\kern0pt}{\isacharplus}{\kern0pt}n{\isachardoublequoteclose}\ {\isachardoublequoteopen}{\isasymnot}\ g{\isacharbackquote}{\kern0pt}{\isacharparenleft}{\kern0pt}x{\isacharhash}{\kern0pt}{\isacharminus}{\kern0pt}m{\isacharparenright}{\kern0pt}{\isacharhash}{\kern0pt}{\isacharplus}{\kern0pt}n\ {\isacharless}{\kern0pt}\ length{\isacharparenleft}{\kern0pt}env{\isacharprime}{\kern0pt}{\isacharparenright}{\kern0pt}{\isachardoublequoteclose}\isanewline
\ \ \ \ \ \ \isacommand{using}\isamarkupfalse%
\ add{\isacharunderscore}{\kern0pt}lt{\isacharunderscore}{\kern0pt}mono{\isadigit{1}}{\isacharbrackleft}{\kern0pt}of\ {\isachardoublequoteopen}g{\isacharbackquote}{\kern0pt}{\isacharparenleft}{\kern0pt}x{\isacharhash}{\kern0pt}{\isacharminus}{\kern0pt}m{\isacharparenright}{\kern0pt}{\isachardoublequoteclose}\ {\isacharunderscore}{\kern0pt}\ n{\isacharcomma}{\kern0pt}OF\ {\isacharunderscore}{\kern0pt}\ {\isacartoucheopen}q{\isasymin}nat{\isacartoucheclose}{\isacharbrackright}{\kern0pt}\isanewline
\ \ \ \ \ \ \ \ add{\isacharunderscore}{\kern0pt}le{\isacharunderscore}{\kern0pt}self{\isadigit{2}}{\isacharbrackleft}{\kern0pt}of\ n{\isacharbrackright}{\kern0pt}\ {\isacartoucheopen}length{\isacharparenleft}{\kern0pt}env{\isacharprime}{\kern0pt}{\isacharparenright}{\kern0pt}\ {\isacharequal}{\kern0pt}\ n{\isacartoucheclose}\isanewline
\ \ \ \ \ \ \isacommand{by}\isamarkupfalse%
\ simp{\isacharunderscore}{\kern0pt}all\isanewline
\ \ \ \ \isacommand{from}\isamarkupfalse%
\ assms\ {\isacartoucheopen}{\isasymnot}\ x\ {\isacharless}{\kern0pt}\ length{\isacharparenleft}{\kern0pt}env{\isacharparenright}{\kern0pt}{\isacartoucheclose}\ {\isacartoucheopen}length{\isacharparenleft}{\kern0pt}env{\isacharparenright}{\kern0pt}\ {\isacharequal}{\kern0pt}\ m{\isacartoucheclose}\isanewline
\ \ \ \ \isacommand{have}\isamarkupfalse%
\ {\isachardoublequoteopen}nth{\isacharparenleft}{\kern0pt}x{\isacharcomma}{\kern0pt}env\ {\isacharat}{\kern0pt}\ env{\isadigit{1}}{\isacharparenright}{\kern0pt}\ {\isacharequal}{\kern0pt}\ nth{\isacharparenleft}{\kern0pt}x{\isacharhash}{\kern0pt}{\isacharminus}{\kern0pt}m{\isacharcomma}{\kern0pt}env{\isadigit{1}}{\isacharparenright}{\kern0pt}{\isachardoublequoteclose}\isanewline
\ \ \ \ \ \ \isacommand{using}\isamarkupfalse%
\ nth{\isacharunderscore}{\kern0pt}append{\isacharbrackleft}{\kern0pt}OF\ {\isacartoucheopen}env{\isasymin}list{\isacharparenleft}{\kern0pt}M{\isacharparenright}{\kern0pt}{\isacartoucheclose}\ {\isacartoucheopen}x{\isasymin}nat{\isacartoucheclose}{\isacharbrackright}{\kern0pt}\ \isacommand{by}\isamarkupfalse%
\ simp\isanewline
\ \ \ \ \isacommand{also}\isamarkupfalse%
\isanewline
\ \ \ \ \isacommand{have}\isamarkupfalse%
\ {\isachardoublequoteopen}{\isachardot}{\kern0pt}{\isachardot}{\kern0pt}{\isachardot}{\kern0pt}\ {\isacharequal}{\kern0pt}\ nth{\isacharparenleft}{\kern0pt}g{\isacharbackquote}{\kern0pt}{\isacharparenleft}{\kern0pt}x{\isacharhash}{\kern0pt}{\isacharminus}{\kern0pt}m{\isacharparenright}{\kern0pt}{\isacharcomma}{\kern0pt}env{\isadigit{2}}{\isacharparenright}{\kern0pt}{\isachardoublequoteclose}\isanewline
\ \ \ \ \ \ \isacommand{using}\isamarkupfalse%
\ assms\ {\isacartoucheopen}x{\isacharhash}{\kern0pt}{\isacharminus}{\kern0pt}m\ {\isacharless}{\kern0pt}\ p{\isacartoucheclose}\ \isacommand{by}\isamarkupfalse%
\ simp\isanewline
\ \ \ \ \isacommand{also}\isamarkupfalse%
\isanewline
\ \ \ \ \isacommand{have}\isamarkupfalse%
\ {\isachardoublequoteopen}{\isachardot}{\kern0pt}{\isachardot}{\kern0pt}{\isachardot}{\kern0pt}\ {\isacharequal}{\kern0pt}\ nth{\isacharparenleft}{\kern0pt}{\isacharparenleft}{\kern0pt}g{\isacharbackquote}{\kern0pt}{\isacharparenleft}{\kern0pt}x{\isacharhash}{\kern0pt}{\isacharminus}{\kern0pt}m{\isacharparenright}{\kern0pt}{\isacharhash}{\kern0pt}{\isacharplus}{\kern0pt}n{\isacharparenright}{\kern0pt}{\isacharhash}{\kern0pt}{\isacharminus}{\kern0pt}length{\isacharparenleft}{\kern0pt}env{\isacharprime}{\kern0pt}{\isacharparenright}{\kern0pt}{\isacharcomma}{\kern0pt}env{\isadigit{2}}{\isacharparenright}{\kern0pt}{\isachardoublequoteclose}\isanewline
\ \ \ \ \ \ \isacommand{using}\isamarkupfalse%
\ \ {\isacartoucheopen}length{\isacharparenleft}{\kern0pt}env{\isacharprime}{\kern0pt}{\isacharparenright}{\kern0pt}\ {\isacharequal}{\kern0pt}\ n{\isacartoucheclose}\isanewline
\ \ \ \ \ \ \ \ diff{\isacharunderscore}{\kern0pt}add{\isacharunderscore}{\kern0pt}inverse{\isadigit{2}}\ {\isacartoucheopen}g{\isacharbackquote}{\kern0pt}{\isacharparenleft}{\kern0pt}x{\isacharhash}{\kern0pt}{\isacharminus}{\kern0pt}m{\isacharparenright}{\kern0pt}{\isasymin}nat{\isacartoucheclose}\isanewline
\ \ \ \ \ \ \isacommand{by}\isamarkupfalse%
\ simp\isanewline
\ \ \ \ \isacommand{also}\isamarkupfalse%
\isanewline
\ \ \ \ \isacommand{have}\isamarkupfalse%
\ {\isachardoublequoteopen}{\isachardot}{\kern0pt}{\isachardot}{\kern0pt}{\isachardot}{\kern0pt}\ {\isacharequal}{\kern0pt}\ nth{\isacharparenleft}{\kern0pt}{\isacharparenleft}{\kern0pt}g{\isacharbackquote}{\kern0pt}{\isacharparenleft}{\kern0pt}x{\isacharhash}{\kern0pt}{\isacharminus}{\kern0pt}m{\isacharparenright}{\kern0pt}{\isacharhash}{\kern0pt}{\isacharplus}{\kern0pt}n{\isacharparenright}{\kern0pt}{\isacharcomma}{\kern0pt}env{\isacharprime}{\kern0pt}{\isacharat}{\kern0pt}env{\isadigit{2}}{\isacharparenright}{\kern0pt}{\isachardoublequoteclose}\isanewline
\ \ \ \ \ \ \isacommand{using}\isamarkupfalse%
\ \ nth{\isacharunderscore}{\kern0pt}append{\isacharbrackleft}{\kern0pt}OF\ {\isacartoucheopen}env{\isacharprime}{\kern0pt}{\isasymin}list{\isacharparenleft}{\kern0pt}M{\isacharparenright}{\kern0pt}{\isacartoucheclose}{\isacharbrackright}{\kern0pt}\ {\isacartoucheopen}n{\isasymin}nat{\isacartoucheclose}\ {\isacartoucheopen}{\isasymnot}\ g{\isacharbackquote}{\kern0pt}{\isacharparenleft}{\kern0pt}x{\isacharhash}{\kern0pt}{\isacharminus}{\kern0pt}m{\isacharparenright}{\kern0pt}{\isacharhash}{\kern0pt}{\isacharplus}{\kern0pt}n\ {\isacharless}{\kern0pt}\ length{\isacharparenleft}{\kern0pt}env{\isacharprime}{\kern0pt}{\isacharparenright}{\kern0pt}{\isacartoucheclose}\isanewline
\ \ \ \ \ \ \isacommand{by}\isamarkupfalse%
\ simp\isanewline
\ \ \ \ \isacommand{also}\isamarkupfalse%
\isanewline
\ \ \ \ \isacommand{have}\isamarkupfalse%
\ {\isachardoublequoteopen}{\isachardot}{\kern0pt}{\isachardot}{\kern0pt}{\isachardot}{\kern0pt}\ {\isacharequal}{\kern0pt}\ nth{\isacharparenleft}{\kern0pt}{\isacharquery}{\kern0pt}h{\isacharbackquote}{\kern0pt}x{\isacharcomma}{\kern0pt}env{\isacharprime}{\kern0pt}{\isacharat}{\kern0pt}env{\isadigit{2}}{\isacharparenright}{\kern0pt}{\isachardoublequoteclose}\isanewline
\ \ \ \ \ \ \isacommand{using}\isamarkupfalse%
\ {\isadigit{2}}\ \isacommand{by}\isamarkupfalse%
\ simp\isanewline
\ \ \ \ \isacommand{finally}\isamarkupfalse%
\isanewline
\ \ \ \ \isacommand{have}\isamarkupfalse%
\ {\isachardoublequoteopen}nth{\isacharparenleft}{\kern0pt}x{\isacharcomma}{\kern0pt}\ env\ {\isacharat}{\kern0pt}\ env{\isadigit{1}}{\isacharparenright}{\kern0pt}\ {\isacharequal}{\kern0pt}\ nth{\isacharparenleft}{\kern0pt}{\isacharquery}{\kern0pt}h{\isacharbackquote}{\kern0pt}x{\isacharcomma}{\kern0pt}env{\isacharprime}{\kern0pt}{\isacharat}{\kern0pt}env{\isadigit{2}}{\isacharparenright}{\kern0pt}{\isachardoublequoteclose}\ \isacommand{{\isachardot}{\kern0pt}}\isamarkupfalse%
\isanewline
\ \ \ \ \isacommand{then}\isamarkupfalse%
\ \isacommand{show}\isamarkupfalse%
\ {\isacharquery}{\kern0pt}thesis\ \isacommand{{\isachardot}{\kern0pt}}\isamarkupfalse%
\isanewline
\ \ \isacommand{qed}\isamarkupfalse%
\isanewline
\ \ \isacommand{then}\isamarkupfalse%
\ \isacommand{show}\isamarkupfalse%
\ {\isacharquery}{\kern0pt}thesis\ \isacommand{by}\isamarkupfalse%
\ simp\isanewline
\isacommand{qed}\isamarkupfalse%
%
\endisatagproof
{\isafoldproof}%
%
\isadelimproof
\isanewline
%
\endisadelimproof
\isanewline
\isacommand{lemma}\isamarkupfalse%
\ sum{\isacharunderscore}{\kern0pt}type\ \ {\isacharcolon}{\kern0pt}\isanewline
\ \ \isakeyword{assumes}\ {\isachardoublequoteopen}m\ {\isasymin}\ nat{\isachardoublequoteclose}\ {\isachardoublequoteopen}n{\isasymin}nat{\isachardoublequoteclose}\ {\isachardoublequoteopen}p{\isasymin}nat{\isachardoublequoteclose}\ {\isachardoublequoteopen}q{\isasymin}nat{\isachardoublequoteclose}\isanewline
\ \ \ \ {\isachardoublequoteopen}f\ {\isasymin}\ m{\isasymrightarrow}n{\isachardoublequoteclose}\ {\isachardoublequoteopen}g{\isasymin}p{\isasymrightarrow}q{\isachardoublequoteclose}\isanewline
\ \ \isakeyword{shows}\ {\isachardoublequoteopen}sum{\isacharparenleft}{\kern0pt}f{\isacharcomma}{\kern0pt}g{\isacharcomma}{\kern0pt}m{\isacharcomma}{\kern0pt}n{\isacharcomma}{\kern0pt}p{\isacharparenright}{\kern0pt}\ {\isasymin}\ {\isacharparenleft}{\kern0pt}m{\isacharhash}{\kern0pt}{\isacharplus}{\kern0pt}p{\isacharparenright}{\kern0pt}\ {\isasymrightarrow}\ {\isacharparenleft}{\kern0pt}n{\isacharhash}{\kern0pt}{\isacharplus}{\kern0pt}q{\isacharparenright}{\kern0pt}{\isachardoublequoteclose}\isanewline
%
\isadelimproof
%
\endisadelimproof
%
\isatagproof
\isacommand{proof}\isamarkupfalse%
\ {\isacharminus}{\kern0pt}\isanewline
\ \ \isacommand{let}\isamarkupfalse%
\ {\isacharquery}{\kern0pt}h\ {\isacharequal}{\kern0pt}\ {\isachardoublequoteopen}sum{\isacharparenleft}{\kern0pt}f{\isacharcomma}{\kern0pt}g{\isacharcomma}{\kern0pt}m{\isacharcomma}{\kern0pt}n{\isacharcomma}{\kern0pt}p{\isacharparenright}{\kern0pt}{\isachardoublequoteclose}\isanewline
\ \ \isacommand{from}\isamarkupfalse%
\ {\isacartoucheopen}m{\isasymin}nat{\isacartoucheclose}\ {\isacartoucheopen}n{\isasymin}nat{\isacartoucheclose}\ {\isacartoucheopen}q{\isasymin}nat{\isacartoucheclose}\isanewline
\ \ \isacommand{have}\isamarkupfalse%
\ {\isachardoublequoteopen}m{\isasymle}m{\isacharhash}{\kern0pt}{\isacharplus}{\kern0pt}p{\isachardoublequoteclose}\ {\isachardoublequoteopen}n{\isasymle}n{\isacharhash}{\kern0pt}{\isacharplus}{\kern0pt}q{\isachardoublequoteclose}\ {\isachardoublequoteopen}q{\isasymle}n{\isacharhash}{\kern0pt}{\isacharplus}{\kern0pt}q{\isachardoublequoteclose}\isanewline
\ \ \ \ \isacommand{using}\isamarkupfalse%
\ add{\isacharunderscore}{\kern0pt}le{\isacharunderscore}{\kern0pt}self{\isacharbrackleft}{\kern0pt}of\ m{\isacharbrackright}{\kern0pt}\ \ add{\isacharunderscore}{\kern0pt}le{\isacharunderscore}{\kern0pt}self{\isadigit{2}}{\isacharbrackleft}{\kern0pt}of\ n\ q{\isacharbrackright}{\kern0pt}\ \isacommand{by}\isamarkupfalse%
\ simp{\isacharunderscore}{\kern0pt}all\isanewline
\ \ \isacommand{from}\isamarkupfalse%
\ {\isacartoucheopen}p{\isasymin}nat{\isacartoucheclose}\isanewline
\ \ \isacommand{have}\isamarkupfalse%
\ {\isachardoublequoteopen}p\ {\isacharequal}{\kern0pt}\ {\isacharparenleft}{\kern0pt}m{\isacharhash}{\kern0pt}{\isacharplus}{\kern0pt}p{\isacharparenright}{\kern0pt}{\isacharhash}{\kern0pt}{\isacharminus}{\kern0pt}m{\isachardoublequoteclose}\ \isacommand{using}\isamarkupfalse%
\ diff{\isacharunderscore}{\kern0pt}add{\isacharunderscore}{\kern0pt}inverse{\isadigit{2}}\ \isacommand{by}\isamarkupfalse%
\ simp\isanewline
\ \ \isacommand{{\isacharbraceleft}{\kern0pt}}\isamarkupfalse%
\isacommand{fix}\isamarkupfalse%
\ x\isanewline
\ \ \ \ \isacommand{assume}\isamarkupfalse%
\ {\isadigit{1}}{\isacharcolon}{\kern0pt}\ {\isachardoublequoteopen}x{\isasymin}m{\isacharhash}{\kern0pt}{\isacharplus}{\kern0pt}p{\isachardoublequoteclose}\ {\isachardoublequoteopen}x{\isacharless}{\kern0pt}m{\isachardoublequoteclose}\isanewline
\ \ \ \ \isacommand{with}\isamarkupfalse%
\ {\isadigit{1}}\ \isacommand{have}\isamarkupfalse%
\ {\isachardoublequoteopen}{\isacharquery}{\kern0pt}h{\isacharbackquote}{\kern0pt}x{\isacharequal}{\kern0pt}\ f{\isacharbackquote}{\kern0pt}x{\isachardoublequoteclose}\ {\isachardoublequoteopen}x{\isasymin}m{\isachardoublequoteclose}\isanewline
\ \ \ \ \ \ \isacommand{using}\isamarkupfalse%
\ assms\ sum{\isacharunderscore}{\kern0pt}inl\ ltD\ \isacommand{by}\isamarkupfalse%
\ simp{\isacharunderscore}{\kern0pt}all\isanewline
\ \ \ \ \isacommand{with}\isamarkupfalse%
\ {\isacartoucheopen}f{\isasymin}m{\isasymrightarrow}n{\isacartoucheclose}\isanewline
\ \ \ \ \isacommand{have}\isamarkupfalse%
\ {\isachardoublequoteopen}{\isacharquery}{\kern0pt}h{\isacharbackquote}{\kern0pt}x\ {\isasymin}\ n{\isachardoublequoteclose}\ \isacommand{by}\isamarkupfalse%
\ simp\isanewline
\ \ \ \ \isacommand{with}\isamarkupfalse%
\ {\isacartoucheopen}n{\isasymin}nat{\isacartoucheclose}\ \isacommand{have}\isamarkupfalse%
\ {\isachardoublequoteopen}{\isacharquery}{\kern0pt}h{\isacharbackquote}{\kern0pt}x\ {\isacharless}{\kern0pt}\ n{\isachardoublequoteclose}\ \isacommand{using}\isamarkupfalse%
\ ltI\ \isacommand{by}\isamarkupfalse%
\ simp\isanewline
\ \ \ \ \isacommand{with}\isamarkupfalse%
\ {\isacartoucheopen}n{\isasymle}n{\isacharhash}{\kern0pt}{\isacharplus}{\kern0pt}q{\isacartoucheclose}\isanewline
\ \ \ \ \isacommand{have}\isamarkupfalse%
\ {\isachardoublequoteopen}{\isacharquery}{\kern0pt}h{\isacharbackquote}{\kern0pt}x\ {\isacharless}{\kern0pt}\ n{\isacharhash}{\kern0pt}{\isacharplus}{\kern0pt}q{\isachardoublequoteclose}\ \isacommand{using}\isamarkupfalse%
\ lt{\isacharunderscore}{\kern0pt}trans{\isadigit{2}}\ \isacommand{by}\isamarkupfalse%
\ simp\isanewline
\ \ \ \ \isacommand{then}\isamarkupfalse%
\isanewline
\ \ \ \ \isacommand{have}\isamarkupfalse%
\ {\isachardoublequoteopen}{\isacharquery}{\kern0pt}h{\isacharbackquote}{\kern0pt}x\ {\isasymin}\ n{\isacharhash}{\kern0pt}{\isacharplus}{\kern0pt}q{\isachardoublequoteclose}\ \ \isacommand{using}\isamarkupfalse%
\ ltD\ \isacommand{by}\isamarkupfalse%
\ simp\isanewline
\ \ \isacommand{{\isacharbraceright}{\kern0pt}}\isamarkupfalse%
\isanewline
\ \ \isacommand{then}\isamarkupfalse%
\ \isacommand{have}\isamarkupfalse%
\ {\isadigit{1}}{\isacharcolon}{\kern0pt}{\isachardoublequoteopen}{\isacharquery}{\kern0pt}h{\isacharbackquote}{\kern0pt}x\ {\isasymin}\ n{\isacharhash}{\kern0pt}{\isacharplus}{\kern0pt}q{\isachardoublequoteclose}\ \isakeyword{if}\ {\isachardoublequoteopen}x{\isasymin}m{\isacharhash}{\kern0pt}{\isacharplus}{\kern0pt}p{\isachardoublequoteclose}\ {\isachardoublequoteopen}x{\isacharless}{\kern0pt}m{\isachardoublequoteclose}\ \isakeyword{for}\ x\ \isacommand{using}\isamarkupfalse%
\ that\ \isacommand{{\isachardot}{\kern0pt}}\isamarkupfalse%
\isanewline
\ \ \isacommand{{\isacharbraceleft}{\kern0pt}}\isamarkupfalse%
\isacommand{fix}\isamarkupfalse%
\ x\isanewline
\ \ \ \ \isacommand{assume}\isamarkupfalse%
\ {\isadigit{1}}{\isacharcolon}{\kern0pt}\ {\isachardoublequoteopen}x{\isasymin}m{\isacharhash}{\kern0pt}{\isacharplus}{\kern0pt}p{\isachardoublequoteclose}\ {\isachardoublequoteopen}m{\isasymle}x{\isachardoublequoteclose}\isanewline
\ \ \ \ \isacommand{then}\isamarkupfalse%
\ \isacommand{have}\isamarkupfalse%
\ {\isachardoublequoteopen}x{\isacharless}{\kern0pt}m{\isacharhash}{\kern0pt}{\isacharplus}{\kern0pt}p{\isachardoublequoteclose}\ {\isachardoublequoteopen}x{\isasymin}nat{\isachardoublequoteclose}\ \isacommand{using}\isamarkupfalse%
\ ltI\ in{\isacharunderscore}{\kern0pt}n{\isacharunderscore}{\kern0pt}in{\isacharunderscore}{\kern0pt}nat{\isacharbrackleft}{\kern0pt}of\ {\isachardoublequoteopen}m{\isacharhash}{\kern0pt}{\isacharplus}{\kern0pt}p{\isachardoublequoteclose}{\isacharbrackright}{\kern0pt}\ ltD\ \isacommand{by}\isamarkupfalse%
\ simp{\isacharunderscore}{\kern0pt}all\isanewline
\ \ \ \ \isacommand{with}\isamarkupfalse%
\ {\isadigit{1}}\isanewline
\ \ \ \ \isacommand{have}\isamarkupfalse%
\ {\isadigit{2}}\ {\isacharcolon}{\kern0pt}\ {\isachardoublequoteopen}{\isacharquery}{\kern0pt}h{\isacharbackquote}{\kern0pt}x{\isacharequal}{\kern0pt}\ g{\isacharbackquote}{\kern0pt}{\isacharparenleft}{\kern0pt}x{\isacharhash}{\kern0pt}{\isacharminus}{\kern0pt}m{\isacharparenright}{\kern0pt}{\isacharhash}{\kern0pt}{\isacharplus}{\kern0pt}n{\isachardoublequoteclose}\isanewline
\ \ \ \ \ \ \isacommand{using}\isamarkupfalse%
\ assms\ sum{\isacharunderscore}{\kern0pt}inr\ ltD\ \isacommand{by}\isamarkupfalse%
\ simp{\isacharunderscore}{\kern0pt}all\isanewline
\ \ \ \ \isacommand{from}\isamarkupfalse%
\ assms\ {\isacartoucheopen}x{\isasymin}nat{\isacartoucheclose}\ {\isacartoucheopen}p{\isacharequal}{\kern0pt}m{\isacharhash}{\kern0pt}{\isacharplus}{\kern0pt}p{\isacharhash}{\kern0pt}{\isacharminus}{\kern0pt}m{\isacartoucheclose}\isanewline
\ \ \ \ \isacommand{have}\isamarkupfalse%
\ {\isachardoublequoteopen}x{\isacharhash}{\kern0pt}{\isacharminus}{\kern0pt}m\ {\isacharless}{\kern0pt}\ p{\isachardoublequoteclose}\ \isacommand{using}\isamarkupfalse%
\ diff{\isacharunderscore}{\kern0pt}mono{\isacharbrackleft}{\kern0pt}OF\ {\isacharunderscore}{\kern0pt}\ {\isacharunderscore}{\kern0pt}\ {\isacharunderscore}{\kern0pt}\ {\isacartoucheopen}x{\isacharless}{\kern0pt}m{\isacharhash}{\kern0pt}{\isacharplus}{\kern0pt}p{\isacartoucheclose}\ {\isacartoucheopen}m{\isasymle}x{\isacartoucheclose}{\isacharbrackright}{\kern0pt}\ \isacommand{by}\isamarkupfalse%
\ simp\isanewline
\ \ \ \ \isacommand{then}\isamarkupfalse%
\ \isacommand{have}\isamarkupfalse%
\ {\isachardoublequoteopen}x{\isacharhash}{\kern0pt}{\isacharminus}{\kern0pt}m{\isasymin}p{\isachardoublequoteclose}\ \isacommand{using}\isamarkupfalse%
\ ltD\ \isacommand{by}\isamarkupfalse%
\ simp\isanewline
\ \ \ \ \isacommand{with}\isamarkupfalse%
\ {\isacartoucheopen}g{\isasymin}p{\isasymrightarrow}q{\isacartoucheclose}\isanewline
\ \ \ \ \isacommand{have}\isamarkupfalse%
\ {\isachardoublequoteopen}g{\isacharbackquote}{\kern0pt}{\isacharparenleft}{\kern0pt}x{\isacharhash}{\kern0pt}{\isacharminus}{\kern0pt}m{\isacharparenright}{\kern0pt}\ {\isasymin}\ q{\isachardoublequoteclose}\ \ \isacommand{by}\isamarkupfalse%
\ simp\isanewline
\ \ \ \ \isacommand{with}\isamarkupfalse%
\ {\isacartoucheopen}q{\isasymin}nat{\isacartoucheclose}\ \isacommand{have}\isamarkupfalse%
\ {\isachardoublequoteopen}g{\isacharbackquote}{\kern0pt}{\isacharparenleft}{\kern0pt}x{\isacharhash}{\kern0pt}{\isacharminus}{\kern0pt}m{\isacharparenright}{\kern0pt}\ {\isacharless}{\kern0pt}\ q{\isachardoublequoteclose}\ \isacommand{using}\isamarkupfalse%
\ ltI\ \isacommand{by}\isamarkupfalse%
\ simp\isanewline
\ \ \ \ \isacommand{with}\isamarkupfalse%
\ {\isacartoucheopen}q{\isasymin}nat{\isacartoucheclose}\isanewline
\ \ \ \ \isacommand{have}\isamarkupfalse%
\ {\isachardoublequoteopen}{\isacharparenleft}{\kern0pt}g{\isacharbackquote}{\kern0pt}{\isacharparenleft}{\kern0pt}x{\isacharhash}{\kern0pt}{\isacharminus}{\kern0pt}m{\isacharparenright}{\kern0pt}{\isacharparenright}{\kern0pt}{\isacharhash}{\kern0pt}{\isacharplus}{\kern0pt}n\ {\isacharless}{\kern0pt}n{\isacharhash}{\kern0pt}{\isacharplus}{\kern0pt}q{\isachardoublequoteclose}\ \isacommand{using}\isamarkupfalse%
\ add{\isacharunderscore}{\kern0pt}lt{\isacharunderscore}{\kern0pt}mono{\isadigit{1}}{\isacharbrackleft}{\kern0pt}of\ {\isachardoublequoteopen}g{\isacharbackquote}{\kern0pt}{\isacharparenleft}{\kern0pt}x{\isacharhash}{\kern0pt}{\isacharminus}{\kern0pt}m{\isacharparenright}{\kern0pt}{\isachardoublequoteclose}\ {\isacharunderscore}{\kern0pt}\ n{\isacharcomma}{\kern0pt}OF\ {\isacharunderscore}{\kern0pt}\ {\isacartoucheopen}q{\isasymin}nat{\isacartoucheclose}{\isacharbrackright}{\kern0pt}\ \isacommand{by}\isamarkupfalse%
\ simp\isanewline
\ \ \ \ \isacommand{with}\isamarkupfalse%
\ {\isadigit{2}}\isanewline
\ \ \ \ \isacommand{have}\isamarkupfalse%
\ {\isachardoublequoteopen}{\isacharquery}{\kern0pt}h{\isacharbackquote}{\kern0pt}x\ {\isasymin}\ n{\isacharhash}{\kern0pt}{\isacharplus}{\kern0pt}q{\isachardoublequoteclose}\ \ \isacommand{using}\isamarkupfalse%
\ ltD\ \isacommand{by}\isamarkupfalse%
\ simp\isanewline
\ \ \isacommand{{\isacharbraceright}{\kern0pt}}\isamarkupfalse%
\isanewline
\ \ \isacommand{then}\isamarkupfalse%
\ \isacommand{have}\isamarkupfalse%
\ {\isadigit{2}}{\isacharcolon}{\kern0pt}{\isachardoublequoteopen}{\isacharquery}{\kern0pt}h{\isacharbackquote}{\kern0pt}x\ {\isasymin}\ n{\isacharhash}{\kern0pt}{\isacharplus}{\kern0pt}q{\isachardoublequoteclose}\ \isakeyword{if}\ {\isachardoublequoteopen}x{\isasymin}m{\isacharhash}{\kern0pt}{\isacharplus}{\kern0pt}p{\isachardoublequoteclose}\ {\isachardoublequoteopen}m{\isasymle}x{\isachardoublequoteclose}\ \isakeyword{for}\ x\ \isacommand{using}\isamarkupfalse%
\ that\ \isacommand{{\isachardot}{\kern0pt}}\isamarkupfalse%
\isanewline
\ \ \isacommand{have}\isamarkupfalse%
\isanewline
\ \ \ \ D{\isacharcolon}{\kern0pt}\ {\isachardoublequoteopen}{\isacharquery}{\kern0pt}h{\isacharbackquote}{\kern0pt}x\ {\isasymin}\ n{\isacharhash}{\kern0pt}{\isacharplus}{\kern0pt}q{\isachardoublequoteclose}\ \isakeyword{if}\ {\isachardoublequoteopen}x{\isasymin}m{\isacharhash}{\kern0pt}{\isacharplus}{\kern0pt}p{\isachardoublequoteclose}\ \isakeyword{for}\ x\isanewline
\ \ \ \ \isacommand{using}\isamarkupfalse%
\ that\isanewline
\ \ \isacommand{proof}\isamarkupfalse%
\ {\isacharparenleft}{\kern0pt}cases\ {\isachardoublequoteopen}x{\isacharless}{\kern0pt}m{\isachardoublequoteclose}{\isacharparenright}{\kern0pt}\isanewline
\ \ \ \ \isacommand{case}\isamarkupfalse%
\ True\isanewline
\ \ \ \ \isacommand{then}\isamarkupfalse%
\ \isacommand{show}\isamarkupfalse%
\ {\isacharquery}{\kern0pt}thesis\ \isacommand{using}\isamarkupfalse%
\ {\isadigit{1}}\ that\ \isacommand{by}\isamarkupfalse%
\ simp\isanewline
\ \ \isacommand{next}\isamarkupfalse%
\isanewline
\ \ \ \ \isacommand{case}\isamarkupfalse%
\ False\isanewline
\ \ \ \ \isacommand{with}\isamarkupfalse%
\ {\isacartoucheopen}m{\isasymin}nat{\isacartoucheclose}\ \isacommand{have}\isamarkupfalse%
\ {\isachardoublequoteopen}m{\isasymle}x{\isachardoublequoteclose}\ \isacommand{using}\isamarkupfalse%
\ not{\isacharunderscore}{\kern0pt}lt{\isacharunderscore}{\kern0pt}iff{\isacharunderscore}{\kern0pt}le\ that\ in{\isacharunderscore}{\kern0pt}n{\isacharunderscore}{\kern0pt}in{\isacharunderscore}{\kern0pt}nat{\isacharbrackleft}{\kern0pt}of\ {\isachardoublequoteopen}m{\isacharhash}{\kern0pt}{\isacharplus}{\kern0pt}p{\isachardoublequoteclose}{\isacharbrackright}{\kern0pt}\ \isacommand{by}\isamarkupfalse%
\ simp\isanewline
\ \ \ \ \isacommand{then}\isamarkupfalse%
\ \isacommand{show}\isamarkupfalse%
\ {\isacharquery}{\kern0pt}thesis\ \isacommand{using}\isamarkupfalse%
\ {\isadigit{2}}\ that\ \isacommand{by}\isamarkupfalse%
\ simp\isanewline
\ \ \isacommand{qed}\isamarkupfalse%
\isanewline
\ \ \isacommand{have}\isamarkupfalse%
\ A{\isacharcolon}{\kern0pt}{\isachardoublequoteopen}function{\isacharparenleft}{\kern0pt}{\isacharquery}{\kern0pt}h{\isacharparenright}{\kern0pt}{\isachardoublequoteclose}\ \isacommand{unfolding}\isamarkupfalse%
\ sum{\isacharunderscore}{\kern0pt}def\ \isacommand{using}\isamarkupfalse%
\ function{\isacharunderscore}{\kern0pt}lam\ \isacommand{by}\isamarkupfalse%
\ simp\isanewline
\ \ \isacommand{have}\isamarkupfalse%
\ {\isachardoublequoteopen}\ x{\isasymin}\ {\isacharparenleft}{\kern0pt}m\ {\isacharhash}{\kern0pt}{\isacharplus}{\kern0pt}\ p{\isacharparenright}{\kern0pt}\ {\isasymtimes}\ {\isacharparenleft}{\kern0pt}n\ {\isacharhash}{\kern0pt}{\isacharplus}{\kern0pt}\ q{\isacharparenright}{\kern0pt}{\isachardoublequoteclose}\ \isakeyword{if}\ {\isachardoublequoteopen}x{\isasymin}\ {\isacharquery}{\kern0pt}h{\isachardoublequoteclose}\ \isakeyword{for}\ x\isanewline
\ \ \ \ \isacommand{using}\isamarkupfalse%
\ that\ lamE{\isacharbrackleft}{\kern0pt}of\ x\ {\isachardoublequoteopen}m{\isacharhash}{\kern0pt}{\isacharplus}{\kern0pt}p{\isachardoublequoteclose}\ {\isacharunderscore}{\kern0pt}\ {\isachardoublequoteopen}x\ {\isasymin}\ {\isacharparenleft}{\kern0pt}m\ {\isacharhash}{\kern0pt}{\isacharplus}{\kern0pt}\ p{\isacharparenright}{\kern0pt}\ {\isasymtimes}\ {\isacharparenleft}{\kern0pt}n\ {\isacharhash}{\kern0pt}{\isacharplus}{\kern0pt}\ q{\isacharparenright}{\kern0pt}{\isachardoublequoteclose}{\isacharbrackright}{\kern0pt}\ D\ \isacommand{unfolding}\isamarkupfalse%
\ sum{\isacharunderscore}{\kern0pt}def\isanewline
\ \ \ \ \isacommand{by}\isamarkupfalse%
\ auto\isanewline
\ \ \isacommand{then}\isamarkupfalse%
\ \isacommand{have}\isamarkupfalse%
\ B{\isacharcolon}{\kern0pt}{\isachardoublequoteopen}{\isacharquery}{\kern0pt}h\ {\isasymsubseteq}\ {\isacharparenleft}{\kern0pt}m\ {\isacharhash}{\kern0pt}{\isacharplus}{\kern0pt}\ p{\isacharparenright}{\kern0pt}\ {\isasymtimes}\ {\isacharparenleft}{\kern0pt}n\ {\isacharhash}{\kern0pt}{\isacharplus}{\kern0pt}\ q{\isacharparenright}{\kern0pt}{\isachardoublequoteclose}\ \isacommand{{\isachardot}{\kern0pt}{\isachardot}{\kern0pt}}\isamarkupfalse%
\isanewline
\ \ \isacommand{have}\isamarkupfalse%
\ {\isachardoublequoteopen}m\ {\isacharhash}{\kern0pt}{\isacharplus}{\kern0pt}\ p\ {\isasymsubseteq}\ domain{\isacharparenleft}{\kern0pt}{\isacharquery}{\kern0pt}h{\isacharparenright}{\kern0pt}{\isachardoublequoteclose}\isanewline
\ \ \ \ \isacommand{unfolding}\isamarkupfalse%
\ sum{\isacharunderscore}{\kern0pt}def\ \isacommand{using}\isamarkupfalse%
\ domain{\isacharunderscore}{\kern0pt}lam\ \isacommand{by}\isamarkupfalse%
\ simp\isanewline
\ \ \isacommand{with}\isamarkupfalse%
\ A\ B\isanewline
\ \ \isacommand{show}\isamarkupfalse%
\ {\isacharquery}{\kern0pt}thesis\ \isacommand{using}\isamarkupfalse%
\ \ Pi{\isacharunderscore}{\kern0pt}iff\ {\isacharbrackleft}{\kern0pt}THEN\ iffD{\isadigit{2}}{\isacharbrackright}{\kern0pt}\ \isacommand{by}\isamarkupfalse%
\ simp\isanewline
\isacommand{qed}\isamarkupfalse%
%
\endisatagproof
{\isafoldproof}%
%
\isadelimproof
\isanewline
%
\endisadelimproof
\isanewline
\isacommand{lemma}\isamarkupfalse%
\ sum{\isacharunderscore}{\kern0pt}type{\isacharunderscore}{\kern0pt}id\ {\isacharcolon}{\kern0pt}\isanewline
\ \ \isakeyword{assumes}\isanewline
\ \ \ \ {\isachardoublequoteopen}f\ {\isasymin}\ length{\isacharparenleft}{\kern0pt}env{\isacharparenright}{\kern0pt}{\isasymrightarrow}length{\isacharparenleft}{\kern0pt}env{\isacharprime}{\kern0pt}{\isacharparenright}{\kern0pt}{\isachardoublequoteclose}\isanewline
\ \ \ \ {\isachardoublequoteopen}env\ {\isasymin}\ list{\isacharparenleft}{\kern0pt}M{\isacharparenright}{\kern0pt}{\isachardoublequoteclose}\isanewline
\ \ \ \ {\isachardoublequoteopen}env{\isacharprime}{\kern0pt}\ {\isasymin}\ list{\isacharparenleft}{\kern0pt}M{\isacharparenright}{\kern0pt}{\isachardoublequoteclose}\isanewline
\ \ \ \ {\isachardoublequoteopen}env{\isadigit{1}}\ {\isasymin}\ list{\isacharparenleft}{\kern0pt}M{\isacharparenright}{\kern0pt}{\isachardoublequoteclose}\isanewline
\ \ \isakeyword{shows}\isanewline
\ \ \ \ {\isachardoublequoteopen}sum{\isacharparenleft}{\kern0pt}f{\isacharcomma}{\kern0pt}id{\isacharparenleft}{\kern0pt}length{\isacharparenleft}{\kern0pt}env{\isadigit{1}}{\isacharparenright}{\kern0pt}{\isacharparenright}{\kern0pt}{\isacharcomma}{\kern0pt}length{\isacharparenleft}{\kern0pt}env{\isacharparenright}{\kern0pt}{\isacharcomma}{\kern0pt}length{\isacharparenleft}{\kern0pt}env{\isacharprime}{\kern0pt}{\isacharparenright}{\kern0pt}{\isacharcomma}{\kern0pt}length{\isacharparenleft}{\kern0pt}env{\isadigit{1}}{\isacharparenright}{\kern0pt}{\isacharparenright}{\kern0pt}\ {\isasymin}\isanewline
\ \ \ \ \ \ \ \ {\isacharparenleft}{\kern0pt}length{\isacharparenleft}{\kern0pt}env{\isacharparenright}{\kern0pt}{\isacharhash}{\kern0pt}{\isacharplus}{\kern0pt}length{\isacharparenleft}{\kern0pt}env{\isadigit{1}}{\isacharparenright}{\kern0pt}{\isacharparenright}{\kern0pt}\ {\isasymrightarrow}\ {\isacharparenleft}{\kern0pt}length{\isacharparenleft}{\kern0pt}env{\isacharprime}{\kern0pt}{\isacharparenright}{\kern0pt}{\isacharhash}{\kern0pt}{\isacharplus}{\kern0pt}length{\isacharparenleft}{\kern0pt}env{\isadigit{1}}{\isacharparenright}{\kern0pt}{\isacharparenright}{\kern0pt}{\isachardoublequoteclose}\isanewline
%
\isadelimproof
\ \ %
\endisadelimproof
%
\isatagproof
\isacommand{using}\isamarkupfalse%
\ assms\ length{\isacharunderscore}{\kern0pt}type\ id{\isacharunderscore}{\kern0pt}fn{\isacharunderscore}{\kern0pt}type\ sum{\isacharunderscore}{\kern0pt}type\isanewline
\ \ \isacommand{by}\isamarkupfalse%
\ simp%
\endisatagproof
{\isafoldproof}%
%
\isadelimproof
\isanewline
%
\endisadelimproof
\isanewline
\isacommand{lemma}\isamarkupfalse%
\ sum{\isacharunderscore}{\kern0pt}type{\isacharunderscore}{\kern0pt}id{\isacharunderscore}{\kern0pt}aux{\isadigit{2}}\ {\isacharcolon}{\kern0pt}\isanewline
\ \ \isakeyword{assumes}\isanewline
\ \ \ \ {\isachardoublequoteopen}f\ {\isasymin}\ m{\isasymrightarrow}n{\isachardoublequoteclose}\isanewline
\ \ \ \ {\isachardoublequoteopen}m\ {\isasymin}\ nat{\isachardoublequoteclose}\ {\isachardoublequoteopen}n\ {\isasymin}\ nat{\isachardoublequoteclose}\isanewline
\ \ \ \ {\isachardoublequoteopen}env{\isadigit{1}}\ {\isasymin}\ list{\isacharparenleft}{\kern0pt}M{\isacharparenright}{\kern0pt}{\isachardoublequoteclose}\isanewline
\ \ \isakeyword{shows}\isanewline
\ \ \ \ {\isachardoublequoteopen}sum{\isacharparenleft}{\kern0pt}f{\isacharcomma}{\kern0pt}id{\isacharparenleft}{\kern0pt}length{\isacharparenleft}{\kern0pt}env{\isadigit{1}}{\isacharparenright}{\kern0pt}{\isacharparenright}{\kern0pt}{\isacharcomma}{\kern0pt}m{\isacharcomma}{\kern0pt}n{\isacharcomma}{\kern0pt}length{\isacharparenleft}{\kern0pt}env{\isadigit{1}}{\isacharparenright}{\kern0pt}{\isacharparenright}{\kern0pt}\ {\isasymin}\isanewline
\ \ \ \ \ \ \ \ {\isacharparenleft}{\kern0pt}m{\isacharhash}{\kern0pt}{\isacharplus}{\kern0pt}length{\isacharparenleft}{\kern0pt}env{\isadigit{1}}{\isacharparenright}{\kern0pt}{\isacharparenright}{\kern0pt}\ {\isasymrightarrow}\ {\isacharparenleft}{\kern0pt}n{\isacharhash}{\kern0pt}{\isacharplus}{\kern0pt}length{\isacharparenleft}{\kern0pt}env{\isadigit{1}}{\isacharparenright}{\kern0pt}{\isacharparenright}{\kern0pt}{\isachardoublequoteclose}\isanewline
%
\isadelimproof
\ \ %
\endisadelimproof
%
\isatagproof
\isacommand{using}\isamarkupfalse%
\ assms\ id{\isacharunderscore}{\kern0pt}fn{\isacharunderscore}{\kern0pt}type\ sum{\isacharunderscore}{\kern0pt}type\isanewline
\ \ \isacommand{by}\isamarkupfalse%
\ auto%
\endisatagproof
{\isafoldproof}%
%
\isadelimproof
\isanewline
%
\endisadelimproof
\isanewline
\isacommand{lemma}\isamarkupfalse%
\ sum{\isacharunderscore}{\kern0pt}action{\isacharunderscore}{\kern0pt}id\ {\isacharcolon}{\kern0pt}\isanewline
\ \ \isakeyword{assumes}\isanewline
\ \ \ \ {\isachardoublequoteopen}env\ {\isasymin}\ list{\isacharparenleft}{\kern0pt}M{\isacharparenright}{\kern0pt}{\isachardoublequoteclose}\isanewline
\ \ \ \ {\isachardoublequoteopen}env{\isacharprime}{\kern0pt}\ {\isasymin}\ list{\isacharparenleft}{\kern0pt}M{\isacharparenright}{\kern0pt}{\isachardoublequoteclose}\isanewline
\ \ \ \ {\isachardoublequoteopen}f\ {\isasymin}\ length{\isacharparenleft}{\kern0pt}env{\isacharparenright}{\kern0pt}{\isasymrightarrow}length{\isacharparenleft}{\kern0pt}env{\isacharprime}{\kern0pt}{\isacharparenright}{\kern0pt}{\isachardoublequoteclose}\isanewline
\ \ \ \ {\isachardoublequoteopen}env{\isadigit{1}}\ {\isasymin}\ list{\isacharparenleft}{\kern0pt}M{\isacharparenright}{\kern0pt}{\isachardoublequoteclose}\isanewline
\ \ \ \ {\isachardoublequoteopen}{\isasymAnd}\ i\ {\isachardot}{\kern0pt}\ i\ {\isacharless}{\kern0pt}\ length{\isacharparenleft}{\kern0pt}env{\isacharparenright}{\kern0pt}\ {\isasymLongrightarrow}\ nth{\isacharparenleft}{\kern0pt}i{\isacharcomma}{\kern0pt}env{\isacharparenright}{\kern0pt}\ {\isacharequal}{\kern0pt}\ nth{\isacharparenleft}{\kern0pt}f{\isacharbackquote}{\kern0pt}i{\isacharcomma}{\kern0pt}env{\isacharprime}{\kern0pt}{\isacharparenright}{\kern0pt}{\isachardoublequoteclose}\isanewline
\ \ \isakeyword{shows}\ {\isachardoublequoteopen}{\isasymAnd}\ i\ {\isachardot}{\kern0pt}\ i\ {\isacharless}{\kern0pt}\ length{\isacharparenleft}{\kern0pt}env{\isacharparenright}{\kern0pt}{\isacharhash}{\kern0pt}{\isacharplus}{\kern0pt}length{\isacharparenleft}{\kern0pt}env{\isadigit{1}}{\isacharparenright}{\kern0pt}\ {\isasymLongrightarrow}\isanewline
\ \ \ \ \ \ \ \ \ \ nth{\isacharparenleft}{\kern0pt}i{\isacharcomma}{\kern0pt}env{\isacharat}{\kern0pt}env{\isadigit{1}}{\isacharparenright}{\kern0pt}\ {\isacharequal}{\kern0pt}\ nth{\isacharparenleft}{\kern0pt}sum{\isacharparenleft}{\kern0pt}f{\isacharcomma}{\kern0pt}id{\isacharparenleft}{\kern0pt}length{\isacharparenleft}{\kern0pt}env{\isadigit{1}}{\isacharparenright}{\kern0pt}{\isacharparenright}{\kern0pt}{\isacharcomma}{\kern0pt}length{\isacharparenleft}{\kern0pt}env{\isacharparenright}{\kern0pt}{\isacharcomma}{\kern0pt}length{\isacharparenleft}{\kern0pt}env{\isacharprime}{\kern0pt}{\isacharparenright}{\kern0pt}{\isacharcomma}{\kern0pt}length{\isacharparenleft}{\kern0pt}env{\isadigit{1}}{\isacharparenright}{\kern0pt}{\isacharparenright}{\kern0pt}{\isacharbackquote}{\kern0pt}i{\isacharcomma}{\kern0pt}env{\isacharprime}{\kern0pt}{\isacharat}{\kern0pt}env{\isadigit{1}}{\isacharparenright}{\kern0pt}{\isachardoublequoteclose}\isanewline
%
\isadelimproof
%
\endisadelimproof
%
\isatagproof
\isacommand{proof}\isamarkupfalse%
\ {\isacharminus}{\kern0pt}\isanewline
\ \ \isacommand{from}\isamarkupfalse%
\ assms\isanewline
\ \ \isacommand{have}\isamarkupfalse%
\ {\isachardoublequoteopen}length{\isacharparenleft}{\kern0pt}env{\isacharparenright}{\kern0pt}{\isasymin}nat{\isachardoublequoteclose}\ {\isacharparenleft}{\kern0pt}\isakeyword{is}\ {\isachardoublequoteopen}{\isacharquery}{\kern0pt}m\ {\isasymin}\ {\isacharunderscore}{\kern0pt}{\isachardoublequoteclose}{\isacharparenright}{\kern0pt}\ \isacommand{by}\isamarkupfalse%
\ simp\isanewline
\ \ \isacommand{from}\isamarkupfalse%
\ assms\ \isacommand{have}\isamarkupfalse%
\ {\isachardoublequoteopen}length{\isacharparenleft}{\kern0pt}env{\isacharprime}{\kern0pt}{\isacharparenright}{\kern0pt}{\isasymin}nat{\isachardoublequoteclose}\ {\isacharparenleft}{\kern0pt}\isakeyword{is}\ {\isachardoublequoteopen}{\isacharquery}{\kern0pt}n\ {\isasymin}\ {\isacharunderscore}{\kern0pt}{\isachardoublequoteclose}{\isacharparenright}{\kern0pt}\ \isacommand{by}\isamarkupfalse%
\ simp\isanewline
\ \ \isacommand{from}\isamarkupfalse%
\ assms\ \isacommand{have}\isamarkupfalse%
\ {\isachardoublequoteopen}length{\isacharparenleft}{\kern0pt}env{\isadigit{1}}{\isacharparenright}{\kern0pt}{\isasymin}nat{\isachardoublequoteclose}\ {\isacharparenleft}{\kern0pt}\isakeyword{is}\ {\isachardoublequoteopen}{\isacharquery}{\kern0pt}p\ {\isasymin}\ {\isacharunderscore}{\kern0pt}{\isachardoublequoteclose}{\isacharparenright}{\kern0pt}\ \isacommand{by}\isamarkupfalse%
\ simp\isanewline
\ \ \isacommand{note}\isamarkupfalse%
\ lenv\ {\isacharequal}{\kern0pt}\ id{\isacharunderscore}{\kern0pt}fn{\isacharunderscore}{\kern0pt}action{\isacharbrackleft}{\kern0pt}OF\ {\isacartoucheopen}{\isacharquery}{\kern0pt}p{\isasymin}nat{\isacartoucheclose}\ {\isacartoucheopen}env{\isadigit{1}}{\isasymin}list{\isacharparenleft}{\kern0pt}M{\isacharparenright}{\kern0pt}{\isacartoucheclose}{\isacharbrackright}{\kern0pt}\isanewline
\ \ \isacommand{note}\isamarkupfalse%
\ lenv{\isacharunderscore}{\kern0pt}ty\ {\isacharequal}{\kern0pt}\ id{\isacharunderscore}{\kern0pt}fn{\isacharunderscore}{\kern0pt}type{\isacharbrackleft}{\kern0pt}OF\ {\isacartoucheopen}{\isacharquery}{\kern0pt}p{\isasymin}nat{\isacartoucheclose}{\isacharbrackright}{\kern0pt}\isanewline
\ \ \isacommand{{\isacharbraceleft}{\kern0pt}}\isamarkupfalse%
\isanewline
\ \ \ \ \isacommand{fix}\isamarkupfalse%
\ i\isanewline
\ \ \ \ \isacommand{assume}\isamarkupfalse%
\ {\isachardoublequoteopen}i\ {\isacharless}{\kern0pt}\ length{\isacharparenleft}{\kern0pt}env{\isacharparenright}{\kern0pt}{\isacharhash}{\kern0pt}{\isacharplus}{\kern0pt}length{\isacharparenleft}{\kern0pt}env{\isadigit{1}}{\isacharparenright}{\kern0pt}{\isachardoublequoteclose}\isanewline
\ \ \ \ \isacommand{have}\isamarkupfalse%
\ {\isachardoublequoteopen}nth{\isacharparenleft}{\kern0pt}i{\isacharcomma}{\kern0pt}env{\isacharat}{\kern0pt}env{\isadigit{1}}{\isacharparenright}{\kern0pt}\ {\isacharequal}{\kern0pt}\ nth{\isacharparenleft}{\kern0pt}sum{\isacharparenleft}{\kern0pt}f{\isacharcomma}{\kern0pt}id{\isacharparenleft}{\kern0pt}length{\isacharparenleft}{\kern0pt}env{\isadigit{1}}{\isacharparenright}{\kern0pt}{\isacharparenright}{\kern0pt}{\isacharcomma}{\kern0pt}{\isacharquery}{\kern0pt}m{\isacharcomma}{\kern0pt}{\isacharquery}{\kern0pt}n{\isacharcomma}{\kern0pt}{\isacharquery}{\kern0pt}p{\isacharparenright}{\kern0pt}{\isacharbackquote}{\kern0pt}i{\isacharcomma}{\kern0pt}env{\isacharprime}{\kern0pt}{\isacharat}{\kern0pt}env{\isadigit{1}}{\isacharparenright}{\kern0pt}{\isachardoublequoteclose}\isanewline
\ \ \ \ \ \ \isacommand{using}\isamarkupfalse%
\ sum{\isacharunderscore}{\kern0pt}action{\isacharbrackleft}{\kern0pt}OF\ {\isacartoucheopen}{\isacharquery}{\kern0pt}m{\isasymin}nat{\isacartoucheclose}\ {\isacartoucheopen}{\isacharquery}{\kern0pt}n{\isasymin}nat{\isacartoucheclose}\ {\isacartoucheopen}{\isacharquery}{\kern0pt}p{\isasymin}nat{\isacartoucheclose}\ {\isacartoucheopen}{\isacharquery}{\kern0pt}p{\isasymin}nat{\isacartoucheclose}\ {\isacartoucheopen}f{\isasymin}{\isacharquery}{\kern0pt}m{\isasymrightarrow}{\isacharquery}{\kern0pt}n{\isacartoucheclose}\isanewline
\ \ \ \ \ \ \ \ \ \ lenv{\isacharunderscore}{\kern0pt}ty\ {\isacartoucheopen}env{\isasymin}list{\isacharparenleft}{\kern0pt}M{\isacharparenright}{\kern0pt}{\isacartoucheclose}\ {\isacartoucheopen}env{\isacharprime}{\kern0pt}{\isasymin}list{\isacharparenleft}{\kern0pt}M{\isacharparenright}{\kern0pt}{\isacartoucheclose}\isanewline
\ \ \ \ \ \ \ \ \ \ {\isacartoucheopen}env{\isadigit{1}}{\isasymin}list{\isacharparenleft}{\kern0pt}M{\isacharparenright}{\kern0pt}{\isacartoucheclose}\ {\isacartoucheopen}env{\isadigit{1}}{\isasymin}list{\isacharparenleft}{\kern0pt}M{\isacharparenright}{\kern0pt}{\isacartoucheclose}\ {\isacharunderscore}{\kern0pt}\isanewline
\ \ \ \ \ \ \ \ \ \ {\isacharunderscore}{\kern0pt}\ {\isacharunderscore}{\kern0pt}\ \ assms{\isacharparenleft}{\kern0pt}{\isadigit{5}}{\isacharparenright}{\kern0pt}\ lenv\isanewline
\ \ \ \ \ \ \ \ \ \ {\isacharbrackright}{\kern0pt}\ {\isacartoucheopen}i{\isacharless}{\kern0pt}{\isacharquery}{\kern0pt}m{\isacharhash}{\kern0pt}{\isacharplus}{\kern0pt}length{\isacharparenleft}{\kern0pt}env{\isadigit{1}}{\isacharparenright}{\kern0pt}{\isacartoucheclose}\ \isacommand{by}\isamarkupfalse%
\ simp\isanewline
\ \ \isacommand{{\isacharbraceright}{\kern0pt}}\isamarkupfalse%
\isanewline
\ \ \isacommand{then}\isamarkupfalse%
\ \isacommand{show}\isamarkupfalse%
\ {\isachardoublequoteopen}{\isasymAnd}\ i\ {\isachardot}{\kern0pt}\ i\ {\isacharless}{\kern0pt}\ {\isacharquery}{\kern0pt}m{\isacharhash}{\kern0pt}{\isacharplus}{\kern0pt}length{\isacharparenleft}{\kern0pt}env{\isadigit{1}}{\isacharparenright}{\kern0pt}\ {\isasymLongrightarrow}\isanewline
\ \ \ \ \ \ \ \ \ \ nth{\isacharparenleft}{\kern0pt}i{\isacharcomma}{\kern0pt}env{\isacharat}{\kern0pt}env{\isadigit{1}}{\isacharparenright}{\kern0pt}\ {\isacharequal}{\kern0pt}\ nth{\isacharparenleft}{\kern0pt}sum{\isacharparenleft}{\kern0pt}f{\isacharcomma}{\kern0pt}id{\isacharparenleft}{\kern0pt}{\isacharquery}{\kern0pt}p{\isacharparenright}{\kern0pt}{\isacharcomma}{\kern0pt}{\isacharquery}{\kern0pt}m{\isacharcomma}{\kern0pt}{\isacharquery}{\kern0pt}n{\isacharcomma}{\kern0pt}{\isacharquery}{\kern0pt}p{\isacharparenright}{\kern0pt}{\isacharbackquote}{\kern0pt}i{\isacharcomma}{\kern0pt}env{\isacharprime}{\kern0pt}{\isacharat}{\kern0pt}env{\isadigit{1}}{\isacharparenright}{\kern0pt}{\isachardoublequoteclose}\ \isacommand{by}\isamarkupfalse%
\ simp\isanewline
\isacommand{qed}\isamarkupfalse%
%
\endisatagproof
{\isafoldproof}%
%
\isadelimproof
\isanewline
%
\endisadelimproof
\isanewline
\isacommand{lemma}\isamarkupfalse%
\ sum{\isacharunderscore}{\kern0pt}action{\isacharunderscore}{\kern0pt}id{\isacharunderscore}{\kern0pt}aux\ {\isacharcolon}{\kern0pt}\isanewline
\ \ \isakeyword{assumes}\isanewline
\ \ \ \ {\isachardoublequoteopen}f\ {\isasymin}\ m{\isasymrightarrow}n{\isachardoublequoteclose}\isanewline
\ \ \ \ {\isachardoublequoteopen}env\ {\isasymin}\ list{\isacharparenleft}{\kern0pt}M{\isacharparenright}{\kern0pt}{\isachardoublequoteclose}\isanewline
\ \ \ \ {\isachardoublequoteopen}env{\isacharprime}{\kern0pt}\ {\isasymin}\ list{\isacharparenleft}{\kern0pt}M{\isacharparenright}{\kern0pt}{\isachardoublequoteclose}\isanewline
\ \ \ \ {\isachardoublequoteopen}env{\isadigit{1}}\ {\isasymin}\ list{\isacharparenleft}{\kern0pt}M{\isacharparenright}{\kern0pt}{\isachardoublequoteclose}\isanewline
\ \ \ \ {\isachardoublequoteopen}length{\isacharparenleft}{\kern0pt}env{\isacharparenright}{\kern0pt}\ {\isacharequal}{\kern0pt}\ m{\isachardoublequoteclose}\isanewline
\ \ \ \ {\isachardoublequoteopen}length{\isacharparenleft}{\kern0pt}env{\isacharprime}{\kern0pt}{\isacharparenright}{\kern0pt}\ {\isacharequal}{\kern0pt}\ n{\isachardoublequoteclose}\isanewline
\ \ \ \ {\isachardoublequoteopen}length{\isacharparenleft}{\kern0pt}env{\isadigit{1}}{\isacharparenright}{\kern0pt}\ {\isacharequal}{\kern0pt}\ p{\isachardoublequoteclose}\isanewline
\ \ \ \ {\isachardoublequoteopen}{\isasymAnd}\ i\ {\isachardot}{\kern0pt}\ i\ {\isacharless}{\kern0pt}\ m\ {\isasymLongrightarrow}\ nth{\isacharparenleft}{\kern0pt}i{\isacharcomma}{\kern0pt}env{\isacharparenright}{\kern0pt}\ {\isacharequal}{\kern0pt}\ nth{\isacharparenleft}{\kern0pt}f{\isacharbackquote}{\kern0pt}i{\isacharcomma}{\kern0pt}env{\isacharprime}{\kern0pt}{\isacharparenright}{\kern0pt}{\isachardoublequoteclose}\isanewline
\ \ \isakeyword{shows}\ {\isachardoublequoteopen}{\isasymAnd}\ i\ {\isachardot}{\kern0pt}\ i\ {\isacharless}{\kern0pt}\ m{\isacharhash}{\kern0pt}{\isacharplus}{\kern0pt}length{\isacharparenleft}{\kern0pt}env{\isadigit{1}}{\isacharparenright}{\kern0pt}\ {\isasymLongrightarrow}\isanewline
\ \ \ \ \ \ \ \ \ \ nth{\isacharparenleft}{\kern0pt}i{\isacharcomma}{\kern0pt}env{\isacharat}{\kern0pt}env{\isadigit{1}}{\isacharparenright}{\kern0pt}\ {\isacharequal}{\kern0pt}\ nth{\isacharparenleft}{\kern0pt}sum{\isacharparenleft}{\kern0pt}f{\isacharcomma}{\kern0pt}id{\isacharparenleft}{\kern0pt}length{\isacharparenleft}{\kern0pt}env{\isadigit{1}}{\isacharparenright}{\kern0pt}{\isacharparenright}{\kern0pt}{\isacharcomma}{\kern0pt}m{\isacharcomma}{\kern0pt}n{\isacharcomma}{\kern0pt}length{\isacharparenleft}{\kern0pt}env{\isadigit{1}}{\isacharparenright}{\kern0pt}{\isacharparenright}{\kern0pt}{\isacharbackquote}{\kern0pt}i{\isacharcomma}{\kern0pt}env{\isacharprime}{\kern0pt}{\isacharat}{\kern0pt}env{\isadigit{1}}{\isacharparenright}{\kern0pt}{\isachardoublequoteclose}\isanewline
%
\isadelimproof
\ \ %
\endisadelimproof
%
\isatagproof
\isacommand{using}\isamarkupfalse%
\ assms\ length{\isacharunderscore}{\kern0pt}type\ id{\isacharunderscore}{\kern0pt}fn{\isacharunderscore}{\kern0pt}type\ sum{\isacharunderscore}{\kern0pt}action{\isacharunderscore}{\kern0pt}id\isanewline
\ \ \isacommand{by}\isamarkupfalse%
\ auto%
\endisatagproof
{\isafoldproof}%
%
\isadelimproof
\isanewline
%
\endisadelimproof
\isanewline
\isanewline
\isacommand{definition}\isamarkupfalse%
\isanewline
\ \ sum{\isacharunderscore}{\kern0pt}id\ {\isacharcolon}{\kern0pt}{\isacharcolon}{\kern0pt}\ {\isachardoublequoteopen}{\isacharbrackleft}{\kern0pt}i{\isacharcomma}{\kern0pt}i{\isacharbrackright}{\kern0pt}\ {\isasymRightarrow}\ i{\isachardoublequoteclose}\ \isakeyword{where}\isanewline
\ \ {\isachardoublequoteopen}sum{\isacharunderscore}{\kern0pt}id{\isacharparenleft}{\kern0pt}m{\isacharcomma}{\kern0pt}f{\isacharparenright}{\kern0pt}\ {\isasymequiv}\ sum{\isacharparenleft}{\kern0pt}{\isasymlambda}x{\isasymin}{\isadigit{1}}{\isachardot}{\kern0pt}x{\isacharcomma}{\kern0pt}f{\isacharcomma}{\kern0pt}{\isadigit{1}}{\isacharcomma}{\kern0pt}{\isadigit{1}}{\isacharcomma}{\kern0pt}m{\isacharparenright}{\kern0pt}{\isachardoublequoteclose}\isanewline
\isanewline
\isacommand{lemma}\isamarkupfalse%
\ sum{\isacharunderscore}{\kern0pt}id{\isadigit{0}}\ {\isacharcolon}{\kern0pt}\ {\isachardoublequoteopen}m{\isasymin}nat{\isasymLongrightarrow}sum{\isacharunderscore}{\kern0pt}id{\isacharparenleft}{\kern0pt}m{\isacharcomma}{\kern0pt}f{\isacharparenright}{\kern0pt}{\isacharbackquote}{\kern0pt}{\isadigit{0}}\ {\isacharequal}{\kern0pt}\ {\isadigit{0}}{\isachardoublequoteclose}\isanewline
%
\isadelimproof
\ \ %
\endisadelimproof
%
\isatagproof
\isacommand{by}\isamarkupfalse%
{\isacharparenleft}{\kern0pt}unfold\ sum{\isacharunderscore}{\kern0pt}id{\isacharunderscore}{\kern0pt}def{\isacharcomma}{\kern0pt}subst\ sum{\isacharunderscore}{\kern0pt}inl{\isacharcomma}{\kern0pt}auto{\isacharparenright}{\kern0pt}%
\endisatagproof
{\isafoldproof}%
%
\isadelimproof
\isanewline
%
\endisadelimproof
\isanewline
\isacommand{lemma}\isamarkupfalse%
\ sum{\isacharunderscore}{\kern0pt}idS\ {\isacharcolon}{\kern0pt}\ {\isachardoublequoteopen}p{\isasymin}nat\ {\isasymLongrightarrow}\ q{\isasymin}nat\ {\isasymLongrightarrow}\ f{\isasymin}p{\isasymrightarrow}q\ {\isasymLongrightarrow}\ x\ {\isasymin}\ p\ {\isasymLongrightarrow}\ sum{\isacharunderscore}{\kern0pt}id{\isacharparenleft}{\kern0pt}p{\isacharcomma}{\kern0pt}f{\isacharparenright}{\kern0pt}{\isacharbackquote}{\kern0pt}{\isacharparenleft}{\kern0pt}succ{\isacharparenleft}{\kern0pt}x{\isacharparenright}{\kern0pt}{\isacharparenright}{\kern0pt}\ {\isacharequal}{\kern0pt}\ succ{\isacharparenleft}{\kern0pt}f{\isacharbackquote}{\kern0pt}x{\isacharparenright}{\kern0pt}{\isachardoublequoteclose}\isanewline
%
\isadelimproof
\ \ %
\endisadelimproof
%
\isatagproof
\isacommand{by}\isamarkupfalse%
{\isacharparenleft}{\kern0pt}subgoal{\isacharunderscore}{\kern0pt}tac\ {\isachardoublequoteopen}x{\isasymin}nat{\isachardoublequoteclose}{\isacharcomma}{\kern0pt}unfold\ sum{\isacharunderscore}{\kern0pt}id{\isacharunderscore}{\kern0pt}def{\isacharcomma}{\kern0pt}subst\ sum{\isacharunderscore}{\kern0pt}inr{\isacharcomma}{\kern0pt}\isanewline
\ \ \ \ \ \ simp{\isacharunderscore}{\kern0pt}all\ add{\isacharcolon}{\kern0pt}ltI{\isacharcomma}{\kern0pt}simp{\isacharunderscore}{\kern0pt}all\ add{\isacharcolon}{\kern0pt}\ app{\isacharunderscore}{\kern0pt}nm\ in{\isacharunderscore}{\kern0pt}n{\isacharunderscore}{\kern0pt}in{\isacharunderscore}{\kern0pt}nat{\isacharparenright}{\kern0pt}%
\endisatagproof
{\isafoldproof}%
%
\isadelimproof
\isanewline
%
\endisadelimproof
\isanewline
\isacommand{lemma}\isamarkupfalse%
\ sum{\isacharunderscore}{\kern0pt}id{\isacharunderscore}{\kern0pt}tc{\isacharunderscore}{\kern0pt}aux\ {\isacharcolon}{\kern0pt}\isanewline
\ \ {\isachardoublequoteopen}p\ {\isasymin}\ nat\ {\isasymLongrightarrow}\ \ q\ {\isasymin}\ nat\ {\isasymLongrightarrow}\ f\ {\isasymin}\ p\ {\isasymrightarrow}\ q\ {\isasymLongrightarrow}\ sum{\isacharunderscore}{\kern0pt}id{\isacharparenleft}{\kern0pt}p{\isacharcomma}{\kern0pt}f{\isacharparenright}{\kern0pt}\ {\isasymin}\ {\isadigit{1}}{\isacharhash}{\kern0pt}{\isacharplus}{\kern0pt}p\ {\isasymrightarrow}\ {\isadigit{1}}{\isacharhash}{\kern0pt}{\isacharplus}{\kern0pt}q{\isachardoublequoteclose}\isanewline
%
\isadelimproof
\ \ %
\endisadelimproof
%
\isatagproof
\isacommand{by}\isamarkupfalse%
\ {\isacharparenleft}{\kern0pt}unfold\ sum{\isacharunderscore}{\kern0pt}id{\isacharunderscore}{\kern0pt}def{\isacharcomma}{\kern0pt}rule\ sum{\isacharunderscore}{\kern0pt}type{\isacharcomma}{\kern0pt}simp{\isacharunderscore}{\kern0pt}all{\isacharparenright}{\kern0pt}%
\endisatagproof
{\isafoldproof}%
%
\isadelimproof
\isanewline
%
\endisadelimproof
\isanewline
\isacommand{lemma}\isamarkupfalse%
\ sum{\isacharunderscore}{\kern0pt}id{\isacharunderscore}{\kern0pt}tc\ {\isacharcolon}{\kern0pt}\isanewline
\ \ {\isachardoublequoteopen}n\ {\isasymin}\ nat\ {\isasymLongrightarrow}\ m\ {\isasymin}\ nat\ {\isasymLongrightarrow}\ f\ {\isasymin}\ n\ {\isasymrightarrow}\ m\ {\isasymLongrightarrow}\ sum{\isacharunderscore}{\kern0pt}id{\isacharparenleft}{\kern0pt}n{\isacharcomma}{\kern0pt}f{\isacharparenright}{\kern0pt}\ {\isasymin}\ succ{\isacharparenleft}{\kern0pt}n{\isacharparenright}{\kern0pt}\ {\isasymrightarrow}\ succ{\isacharparenleft}{\kern0pt}m{\isacharparenright}{\kern0pt}{\isachardoublequoteclose}\isanewline
%
\isadelimproof
\ \ %
\endisadelimproof
%
\isatagproof
\isacommand{by}\isamarkupfalse%
{\isacharparenleft}{\kern0pt}rule\ ssubst{\isacharbrackleft}{\kern0pt}of\ \ {\isachardoublequoteopen}succ{\isacharparenleft}{\kern0pt}n{\isacharparenright}{\kern0pt}\ {\isasymrightarrow}\ succ{\isacharparenleft}{\kern0pt}m{\isacharparenright}{\kern0pt}{\isachardoublequoteclose}\ {\isachardoublequoteopen}{\isadigit{1}}{\isacharhash}{\kern0pt}{\isacharplus}{\kern0pt}n\ {\isasymrightarrow}\ {\isadigit{1}}{\isacharhash}{\kern0pt}{\isacharplus}{\kern0pt}m{\isachardoublequoteclose}{\isacharbrackright}{\kern0pt}{\isacharcomma}{\kern0pt}\isanewline
\ \ \ \ \ \ simp{\isacharcomma}{\kern0pt}rule\ sum{\isacharunderscore}{\kern0pt}id{\isacharunderscore}{\kern0pt}tc{\isacharunderscore}{\kern0pt}aux{\isacharcomma}{\kern0pt}simp{\isacharunderscore}{\kern0pt}all{\isacharparenright}{\kern0pt}%
\endisatagproof
{\isafoldproof}%
%
\isadelimproof
%
\endisadelimproof
%
\isadelimdocument
%
\endisadelimdocument
%
\isatagdocument
%
\isamarkupsubsection{Renaming of formulas%
}
\isamarkuptrue%
%
\endisatagdocument
{\isafolddocument}%
%
\isadelimdocument
%
\endisadelimdocument
\isacommand{consts}\isamarkupfalse%
\ \ \ ren\ {\isacharcolon}{\kern0pt}{\isacharcolon}{\kern0pt}\ {\isachardoublequoteopen}i{\isasymRightarrow}i{\isachardoublequoteclose}\isanewline
\isacommand{primrec}\isamarkupfalse%
\isanewline
\ \ {\isachardoublequoteopen}ren{\isacharparenleft}{\kern0pt}Member{\isacharparenleft}{\kern0pt}x{\isacharcomma}{\kern0pt}y{\isacharparenright}{\kern0pt}{\isacharparenright}{\kern0pt}\ {\isacharequal}{\kern0pt}\isanewline
\ \ \ \ \ \ {\isacharparenleft}{\kern0pt}{\isasymlambda}\ n\ {\isasymin}\ nat\ {\isachardot}{\kern0pt}\ {\isasymlambda}\ m\ {\isasymin}\ nat{\isachardot}{\kern0pt}\ {\isasymlambda}f\ {\isasymin}\ n\ {\isasymrightarrow}\ m{\isachardot}{\kern0pt}\ Member\ {\isacharparenleft}{\kern0pt}f{\isacharbackquote}{\kern0pt}x{\isacharcomma}{\kern0pt}\ f{\isacharbackquote}{\kern0pt}y{\isacharparenright}{\kern0pt}{\isacharparenright}{\kern0pt}{\isachardoublequoteclose}\isanewline
\isanewline
{\isachardoublequoteopen}ren{\isacharparenleft}{\kern0pt}Equal{\isacharparenleft}{\kern0pt}x{\isacharcomma}{\kern0pt}y{\isacharparenright}{\kern0pt}{\isacharparenright}{\kern0pt}\ {\isacharequal}{\kern0pt}\isanewline
\ \ \ \ \ \ {\isacharparenleft}{\kern0pt}{\isasymlambda}\ n\ {\isasymin}\ nat\ {\isachardot}{\kern0pt}\ {\isasymlambda}\ m\ {\isasymin}\ nat{\isachardot}{\kern0pt}\ {\isasymlambda}f\ {\isasymin}\ n\ {\isasymrightarrow}\ m{\isachardot}{\kern0pt}\ Equal\ {\isacharparenleft}{\kern0pt}f{\isacharbackquote}{\kern0pt}x{\isacharcomma}{\kern0pt}\ f{\isacharbackquote}{\kern0pt}y{\isacharparenright}{\kern0pt}{\isacharparenright}{\kern0pt}{\isachardoublequoteclose}\isanewline
\isanewline
{\isachardoublequoteopen}ren{\isacharparenleft}{\kern0pt}Nand{\isacharparenleft}{\kern0pt}p{\isacharcomma}{\kern0pt}q{\isacharparenright}{\kern0pt}{\isacharparenright}{\kern0pt}\ {\isacharequal}{\kern0pt}\isanewline
\ \ \ \ \ \ {\isacharparenleft}{\kern0pt}{\isasymlambda}\ n\ {\isasymin}\ nat\ {\isachardot}{\kern0pt}\ {\isasymlambda}\ m\ {\isasymin}\ nat{\isachardot}{\kern0pt}\ {\isasymlambda}f\ {\isasymin}\ n\ {\isasymrightarrow}\ m{\isachardot}{\kern0pt}\ Nand\ {\isacharparenleft}{\kern0pt}ren{\isacharparenleft}{\kern0pt}p{\isacharparenright}{\kern0pt}{\isacharbackquote}{\kern0pt}n{\isacharbackquote}{\kern0pt}m{\isacharbackquote}{\kern0pt}f{\isacharcomma}{\kern0pt}\ ren{\isacharparenleft}{\kern0pt}q{\isacharparenright}{\kern0pt}{\isacharbackquote}{\kern0pt}n{\isacharbackquote}{\kern0pt}m{\isacharbackquote}{\kern0pt}f{\isacharparenright}{\kern0pt}{\isacharparenright}{\kern0pt}{\isachardoublequoteclose}\isanewline
\isanewline
{\isachardoublequoteopen}ren{\isacharparenleft}{\kern0pt}Forall{\isacharparenleft}{\kern0pt}p{\isacharparenright}{\kern0pt}{\isacharparenright}{\kern0pt}\ {\isacharequal}{\kern0pt}\isanewline
\ \ \ \ \ \ {\isacharparenleft}{\kern0pt}{\isasymlambda}\ n\ {\isasymin}\ nat\ {\isachardot}{\kern0pt}\ {\isasymlambda}\ m\ {\isasymin}\ nat{\isachardot}{\kern0pt}\ {\isasymlambda}f\ {\isasymin}\ n\ {\isasymrightarrow}\ m{\isachardot}{\kern0pt}\ Forall\ {\isacharparenleft}{\kern0pt}ren{\isacharparenleft}{\kern0pt}p{\isacharparenright}{\kern0pt}{\isacharbackquote}{\kern0pt}succ{\isacharparenleft}{\kern0pt}n{\isacharparenright}{\kern0pt}{\isacharbackquote}{\kern0pt}succ{\isacharparenleft}{\kern0pt}m{\isacharparenright}{\kern0pt}{\isacharbackquote}{\kern0pt}sum{\isacharunderscore}{\kern0pt}id{\isacharparenleft}{\kern0pt}n{\isacharcomma}{\kern0pt}f{\isacharparenright}{\kern0pt}{\isacharparenright}{\kern0pt}{\isacharparenright}{\kern0pt}{\isachardoublequoteclose}\isanewline
\isanewline
\isacommand{lemma}\isamarkupfalse%
\ arity{\isacharunderscore}{\kern0pt}meml\ {\isacharcolon}{\kern0pt}\ {\isachardoublequoteopen}l\ {\isasymin}\ nat\ {\isasymLongrightarrow}\ Member{\isacharparenleft}{\kern0pt}x{\isacharcomma}{\kern0pt}y{\isacharparenright}{\kern0pt}\ {\isasymin}\ formula\ {\isasymLongrightarrow}\ arity{\isacharparenleft}{\kern0pt}Member{\isacharparenleft}{\kern0pt}x{\isacharcomma}{\kern0pt}y{\isacharparenright}{\kern0pt}{\isacharparenright}{\kern0pt}\ {\isasymle}\ l\ {\isasymLongrightarrow}\ x\ {\isasymin}\ l{\isachardoublequoteclose}\isanewline
%
\isadelimproof
\ \ %
\endisadelimproof
%
\isatagproof
\isacommand{by}\isamarkupfalse%
{\isacharparenleft}{\kern0pt}simp{\isacharcomma}{\kern0pt}rule\ subsetD{\isacharcomma}{\kern0pt}rule\ le{\isacharunderscore}{\kern0pt}imp{\isacharunderscore}{\kern0pt}subset{\isacharcomma}{\kern0pt}assumption{\isacharcomma}{\kern0pt}simp{\isacharparenright}{\kern0pt}%
\endisatagproof
{\isafoldproof}%
%
\isadelimproof
\isanewline
%
\endisadelimproof
\isacommand{lemma}\isamarkupfalse%
\ arity{\isacharunderscore}{\kern0pt}memr\ {\isacharcolon}{\kern0pt}\ {\isachardoublequoteopen}l\ {\isasymin}\ nat\ {\isasymLongrightarrow}\ Member{\isacharparenleft}{\kern0pt}x{\isacharcomma}{\kern0pt}y{\isacharparenright}{\kern0pt}\ {\isasymin}\ formula\ {\isasymLongrightarrow}\ arity{\isacharparenleft}{\kern0pt}Member{\isacharparenleft}{\kern0pt}x{\isacharcomma}{\kern0pt}y{\isacharparenright}{\kern0pt}{\isacharparenright}{\kern0pt}\ {\isasymle}\ l\ {\isasymLongrightarrow}\ y\ {\isasymin}\ l{\isachardoublequoteclose}\isanewline
%
\isadelimproof
\ \ %
\endisadelimproof
%
\isatagproof
\isacommand{by}\isamarkupfalse%
{\isacharparenleft}{\kern0pt}simp{\isacharcomma}{\kern0pt}rule\ subsetD{\isacharcomma}{\kern0pt}rule\ le{\isacharunderscore}{\kern0pt}imp{\isacharunderscore}{\kern0pt}subset{\isacharcomma}{\kern0pt}assumption{\isacharcomma}{\kern0pt}simp{\isacharparenright}{\kern0pt}%
\endisatagproof
{\isafoldproof}%
%
\isadelimproof
\isanewline
%
\endisadelimproof
\isacommand{lemma}\isamarkupfalse%
\ arity{\isacharunderscore}{\kern0pt}eql\ {\isacharcolon}{\kern0pt}\ {\isachardoublequoteopen}l\ {\isasymin}\ nat\ {\isasymLongrightarrow}\ Equal{\isacharparenleft}{\kern0pt}x{\isacharcomma}{\kern0pt}y{\isacharparenright}{\kern0pt}\ {\isasymin}\ formula\ {\isasymLongrightarrow}\ arity{\isacharparenleft}{\kern0pt}Equal{\isacharparenleft}{\kern0pt}x{\isacharcomma}{\kern0pt}y{\isacharparenright}{\kern0pt}{\isacharparenright}{\kern0pt}\ {\isasymle}\ l\ {\isasymLongrightarrow}\ x\ {\isasymin}\ l{\isachardoublequoteclose}\isanewline
%
\isadelimproof
\ \ %
\endisadelimproof
%
\isatagproof
\isacommand{by}\isamarkupfalse%
{\isacharparenleft}{\kern0pt}simp{\isacharcomma}{\kern0pt}rule\ subsetD{\isacharcomma}{\kern0pt}rule\ le{\isacharunderscore}{\kern0pt}imp{\isacharunderscore}{\kern0pt}subset{\isacharcomma}{\kern0pt}assumption{\isacharcomma}{\kern0pt}simp{\isacharparenright}{\kern0pt}%
\endisatagproof
{\isafoldproof}%
%
\isadelimproof
\isanewline
%
\endisadelimproof
\isacommand{lemma}\isamarkupfalse%
\ arity{\isacharunderscore}{\kern0pt}eqr\ {\isacharcolon}{\kern0pt}\ {\isachardoublequoteopen}l\ {\isasymin}\ nat\ {\isasymLongrightarrow}\ Equal{\isacharparenleft}{\kern0pt}x{\isacharcomma}{\kern0pt}y{\isacharparenright}{\kern0pt}\ {\isasymin}\ formula\ {\isasymLongrightarrow}\ arity{\isacharparenleft}{\kern0pt}Equal{\isacharparenleft}{\kern0pt}x{\isacharcomma}{\kern0pt}y{\isacharparenright}{\kern0pt}{\isacharparenright}{\kern0pt}\ {\isasymle}\ l\ {\isasymLongrightarrow}\ y\ {\isasymin}\ l{\isachardoublequoteclose}\isanewline
%
\isadelimproof
\ \ %
\endisadelimproof
%
\isatagproof
\isacommand{by}\isamarkupfalse%
{\isacharparenleft}{\kern0pt}simp{\isacharcomma}{\kern0pt}rule\ subsetD{\isacharcomma}{\kern0pt}rule\ le{\isacharunderscore}{\kern0pt}imp{\isacharunderscore}{\kern0pt}subset{\isacharcomma}{\kern0pt}assumption{\isacharcomma}{\kern0pt}simp{\isacharparenright}{\kern0pt}%
\endisatagproof
{\isafoldproof}%
%
\isadelimproof
\isanewline
%
\endisadelimproof
\isacommand{lemma}\isamarkupfalse%
\ \ nand{\isacharunderscore}{\kern0pt}ar{\isadigit{1}}\ {\isacharcolon}{\kern0pt}\ {\isachardoublequoteopen}p\ {\isasymin}\ formula\ {\isasymLongrightarrow}\ q{\isasymin}formula\ {\isasymLongrightarrow}arity{\isacharparenleft}{\kern0pt}p{\isacharparenright}{\kern0pt}\ {\isasymle}\ arity{\isacharparenleft}{\kern0pt}Nand{\isacharparenleft}{\kern0pt}p{\isacharcomma}{\kern0pt}q{\isacharparenright}{\kern0pt}{\isacharparenright}{\kern0pt}{\isachardoublequoteclose}\isanewline
%
\isadelimproof
\ \ %
\endisadelimproof
%
\isatagproof
\isacommand{by}\isamarkupfalse%
\ {\isacharparenleft}{\kern0pt}simp{\isacharcomma}{\kern0pt}rule\ Un{\isacharunderscore}{\kern0pt}upper{\isadigit{1}}{\isacharunderscore}{\kern0pt}le{\isacharcomma}{\kern0pt}simp{\isacharplus}{\kern0pt}{\isacharparenright}{\kern0pt}%
\endisatagproof
{\isafoldproof}%
%
\isadelimproof
\isanewline
%
\endisadelimproof
\isacommand{lemma}\isamarkupfalse%
\ nand{\isacharunderscore}{\kern0pt}ar{\isadigit{2}}\ {\isacharcolon}{\kern0pt}\ {\isachardoublequoteopen}p\ {\isasymin}\ formula\ {\isasymLongrightarrow}\ q{\isasymin}formula\ {\isasymLongrightarrow}arity{\isacharparenleft}{\kern0pt}q{\isacharparenright}{\kern0pt}\ {\isasymle}\ arity{\isacharparenleft}{\kern0pt}Nand{\isacharparenleft}{\kern0pt}p{\isacharcomma}{\kern0pt}q{\isacharparenright}{\kern0pt}{\isacharparenright}{\kern0pt}{\isachardoublequoteclose}\isanewline
%
\isadelimproof
\ \ %
\endisadelimproof
%
\isatagproof
\isacommand{by}\isamarkupfalse%
\ {\isacharparenleft}{\kern0pt}simp{\isacharcomma}{\kern0pt}rule\ Un{\isacharunderscore}{\kern0pt}upper{\isadigit{2}}{\isacharunderscore}{\kern0pt}le{\isacharcomma}{\kern0pt}simp{\isacharplus}{\kern0pt}{\isacharparenright}{\kern0pt}%
\endisatagproof
{\isafoldproof}%
%
\isadelimproof
\isanewline
%
\endisadelimproof
\isanewline
\isacommand{lemma}\isamarkupfalse%
\ nand{\isacharunderscore}{\kern0pt}ar{\isadigit{1}}D\ {\isacharcolon}{\kern0pt}\ {\isachardoublequoteopen}p\ {\isasymin}\ formula\ {\isasymLongrightarrow}\ q{\isasymin}formula\ {\isasymLongrightarrow}\ arity{\isacharparenleft}{\kern0pt}Nand{\isacharparenleft}{\kern0pt}p{\isacharcomma}{\kern0pt}q{\isacharparenright}{\kern0pt}{\isacharparenright}{\kern0pt}\ {\isasymle}\ n\ {\isasymLongrightarrow}\ arity{\isacharparenleft}{\kern0pt}p{\isacharparenright}{\kern0pt}\ {\isasymle}\ n{\isachardoublequoteclose}\isanewline
%
\isadelimproof
\ \ %
\endisadelimproof
%
\isatagproof
\isacommand{by}\isamarkupfalse%
{\isacharparenleft}{\kern0pt}auto\ simp\ add{\isacharcolon}{\kern0pt}\ \ le{\isacharunderscore}{\kern0pt}trans{\isacharbrackleft}{\kern0pt}OF\ Un{\isacharunderscore}{\kern0pt}upper{\isadigit{1}}{\isacharunderscore}{\kern0pt}le{\isacharbrackleft}{\kern0pt}of\ {\isachardoublequoteopen}arity{\isacharparenleft}{\kern0pt}p{\isacharparenright}{\kern0pt}{\isachardoublequoteclose}\ {\isachardoublequoteopen}arity{\isacharparenleft}{\kern0pt}q{\isacharparenright}{\kern0pt}{\isachardoublequoteclose}{\isacharbrackright}{\kern0pt}{\isacharbrackright}{\kern0pt}{\isacharparenright}{\kern0pt}%
\endisatagproof
{\isafoldproof}%
%
\isadelimproof
\isanewline
%
\endisadelimproof
\isacommand{lemma}\isamarkupfalse%
\ nand{\isacharunderscore}{\kern0pt}ar{\isadigit{2}}D\ {\isacharcolon}{\kern0pt}\ {\isachardoublequoteopen}p\ {\isasymin}\ formula\ {\isasymLongrightarrow}\ q{\isasymin}formula\ {\isasymLongrightarrow}\ arity{\isacharparenleft}{\kern0pt}Nand{\isacharparenleft}{\kern0pt}p{\isacharcomma}{\kern0pt}q{\isacharparenright}{\kern0pt}{\isacharparenright}{\kern0pt}\ {\isasymle}\ n\ {\isasymLongrightarrow}\ arity{\isacharparenleft}{\kern0pt}q{\isacharparenright}{\kern0pt}\ {\isasymle}\ n{\isachardoublequoteclose}\isanewline
%
\isadelimproof
\ \ %
\endisadelimproof
%
\isatagproof
\isacommand{by}\isamarkupfalse%
{\isacharparenleft}{\kern0pt}auto\ simp\ add{\isacharcolon}{\kern0pt}\ \ le{\isacharunderscore}{\kern0pt}trans{\isacharbrackleft}{\kern0pt}OF\ Un{\isacharunderscore}{\kern0pt}upper{\isadigit{2}}{\isacharunderscore}{\kern0pt}le{\isacharbrackleft}{\kern0pt}of\ {\isachardoublequoteopen}arity{\isacharparenleft}{\kern0pt}p{\isacharparenright}{\kern0pt}{\isachardoublequoteclose}\ {\isachardoublequoteopen}arity{\isacharparenleft}{\kern0pt}q{\isacharparenright}{\kern0pt}{\isachardoublequoteclose}{\isacharbrackright}{\kern0pt}{\isacharbrackright}{\kern0pt}{\isacharparenright}{\kern0pt}%
\endisatagproof
{\isafoldproof}%
%
\isadelimproof
\isanewline
%
\endisadelimproof
\isanewline
\isanewline
\isacommand{lemma}\isamarkupfalse%
\ ren{\isacharunderscore}{\kern0pt}tc\ {\isacharcolon}{\kern0pt}\ {\isachardoublequoteopen}p\ {\isasymin}\ formula\ {\isasymLongrightarrow}\isanewline
\ \ {\isacharparenleft}{\kern0pt}{\isasymAnd}\ n\ m\ f\ {\isachardot}{\kern0pt}\ n\ {\isasymin}\ nat\ {\isasymLongrightarrow}\ m\ {\isasymin}\ nat\ {\isasymLongrightarrow}\ f\ {\isasymin}\ n{\isasymrightarrow}m\ {\isasymLongrightarrow}\ \ ren{\isacharparenleft}{\kern0pt}p{\isacharparenright}{\kern0pt}{\isacharbackquote}{\kern0pt}n{\isacharbackquote}{\kern0pt}m{\isacharbackquote}{\kern0pt}f\ {\isasymin}\ formula{\isacharparenright}{\kern0pt}{\isachardoublequoteclose}\isanewline
%
\isadelimproof
\ \ %
\endisadelimproof
%
\isatagproof
\isacommand{by}\isamarkupfalse%
\ {\isacharparenleft}{\kern0pt}induct\ set{\isacharcolon}{\kern0pt}formula{\isacharcomma}{\kern0pt}auto\ simp\ add{\isacharcolon}{\kern0pt}\ app{\isacharunderscore}{\kern0pt}nm\ sum{\isacharunderscore}{\kern0pt}id{\isacharunderscore}{\kern0pt}tc{\isacharparenright}{\kern0pt}%
\endisatagproof
{\isafoldproof}%
%
\isadelimproof
\isanewline
%
\endisadelimproof
\isanewline
\isanewline
\isacommand{lemma}\isamarkupfalse%
\ arity{\isacharunderscore}{\kern0pt}ren\ {\isacharcolon}{\kern0pt}\isanewline
\ \ \isakeyword{fixes}\ {\isachardoublequoteopen}p{\isachardoublequoteclose}\isanewline
\ \ \isakeyword{assumes}\ {\isachardoublequoteopen}p\ {\isasymin}\ formula{\isachardoublequoteclose}\isanewline
\ \ \isakeyword{shows}\ {\isachardoublequoteopen}{\isasymAnd}\ n\ m\ f\ {\isachardot}{\kern0pt}\ n\ {\isasymin}\ nat\ {\isasymLongrightarrow}\ m\ {\isasymin}\ nat\ {\isasymLongrightarrow}\ f\ {\isasymin}\ n{\isasymrightarrow}m\ {\isasymLongrightarrow}\ arity{\isacharparenleft}{\kern0pt}p{\isacharparenright}{\kern0pt}\ {\isasymle}\ n\ {\isasymLongrightarrow}\ arity{\isacharparenleft}{\kern0pt}ren{\isacharparenleft}{\kern0pt}p{\isacharparenright}{\kern0pt}{\isacharbackquote}{\kern0pt}n{\isacharbackquote}{\kern0pt}m{\isacharbackquote}{\kern0pt}f{\isacharparenright}{\kern0pt}{\isasymle}m{\isachardoublequoteclose}\isanewline
%
\isadelimproof
\ \ %
\endisadelimproof
%
\isatagproof
\isacommand{using}\isamarkupfalse%
\ assms\isanewline
\isacommand{proof}\isamarkupfalse%
\ {\isacharparenleft}{\kern0pt}induct\ set{\isacharcolon}{\kern0pt}formula{\isacharparenright}{\kern0pt}\isanewline
\ \ \isacommand{case}\isamarkupfalse%
\ {\isacharparenleft}{\kern0pt}Member\ x\ y{\isacharparenright}{\kern0pt}\isanewline
\ \ \isacommand{then}\isamarkupfalse%
\ \isacommand{have}\isamarkupfalse%
\ {\isachardoublequoteopen}f{\isacharbackquote}{\kern0pt}x\ {\isasymin}\ m{\isachardoublequoteclose}\ {\isachardoublequoteopen}f{\isacharbackquote}{\kern0pt}y\ {\isasymin}\ m{\isachardoublequoteclose}\isanewline
\ \ \ \ \isacommand{using}\isamarkupfalse%
\ Member\ assms\ \isacommand{by}\isamarkupfalse%
\ {\isacharparenleft}{\kern0pt}simp\ add{\isacharcolon}{\kern0pt}\ arity{\isacharunderscore}{\kern0pt}meml\ apply{\isacharunderscore}{\kern0pt}funtype{\isacharcomma}{\kern0pt}simp\ add{\isacharcolon}{\kern0pt}arity{\isacharunderscore}{\kern0pt}memr\ apply{\isacharunderscore}{\kern0pt}funtype{\isacharparenright}{\kern0pt}\isanewline
\ \ \isacommand{then}\isamarkupfalse%
\ \isacommand{show}\isamarkupfalse%
\ {\isacharquery}{\kern0pt}case\ \isacommand{using}\isamarkupfalse%
\ Member\ \isacommand{by}\isamarkupfalse%
\ {\isacharparenleft}{\kern0pt}simp\ add{\isacharcolon}{\kern0pt}\ Un{\isacharunderscore}{\kern0pt}least{\isacharunderscore}{\kern0pt}lt\ ltI{\isacharparenright}{\kern0pt}\isanewline
\isacommand{next}\isamarkupfalse%
\isanewline
\ \ \isacommand{case}\isamarkupfalse%
\ {\isacharparenleft}{\kern0pt}Equal\ x\ y{\isacharparenright}{\kern0pt}\isanewline
\ \ \isacommand{then}\isamarkupfalse%
\ \isacommand{have}\isamarkupfalse%
\ {\isachardoublequoteopen}f{\isacharbackquote}{\kern0pt}x\ {\isasymin}\ m{\isachardoublequoteclose}\ {\isachardoublequoteopen}f{\isacharbackquote}{\kern0pt}y\ {\isasymin}\ m{\isachardoublequoteclose}\isanewline
\ \ \ \ \isacommand{using}\isamarkupfalse%
\ Equal\ assms\ \isacommand{by}\isamarkupfalse%
\ {\isacharparenleft}{\kern0pt}simp\ add{\isacharcolon}{\kern0pt}\ arity{\isacharunderscore}{\kern0pt}eql\ apply{\isacharunderscore}{\kern0pt}funtype{\isacharcomma}{\kern0pt}simp\ add{\isacharcolon}{\kern0pt}arity{\isacharunderscore}{\kern0pt}eqr\ apply{\isacharunderscore}{\kern0pt}funtype{\isacharparenright}{\kern0pt}\isanewline
\ \ \isacommand{then}\isamarkupfalse%
\ \isacommand{show}\isamarkupfalse%
\ {\isacharquery}{\kern0pt}case\ \isacommand{using}\isamarkupfalse%
\ Equal\ \isacommand{by}\isamarkupfalse%
\ {\isacharparenleft}{\kern0pt}simp\ add{\isacharcolon}{\kern0pt}\ Un{\isacharunderscore}{\kern0pt}least{\isacharunderscore}{\kern0pt}lt\ ltI{\isacharparenright}{\kern0pt}\isanewline
\isacommand{next}\isamarkupfalse%
\isanewline
\ \ \isacommand{case}\isamarkupfalse%
\ {\isacharparenleft}{\kern0pt}Nand\ p\ q{\isacharparenright}{\kern0pt}\isanewline
\ \ \isacommand{then}\isamarkupfalse%
\ \isacommand{have}\isamarkupfalse%
\ {\isachardoublequoteopen}arity{\isacharparenleft}{\kern0pt}p{\isacharparenright}{\kern0pt}{\isasymle}arity{\isacharparenleft}{\kern0pt}Nand{\isacharparenleft}{\kern0pt}p{\isacharcomma}{\kern0pt}q{\isacharparenright}{\kern0pt}{\isacharparenright}{\kern0pt}{\isachardoublequoteclose}\isanewline
\ \ \ \ {\isachardoublequoteopen}arity{\isacharparenleft}{\kern0pt}q{\isacharparenright}{\kern0pt}{\isasymle}arity{\isacharparenleft}{\kern0pt}Nand{\isacharparenleft}{\kern0pt}p{\isacharcomma}{\kern0pt}q{\isacharparenright}{\kern0pt}{\isacharparenright}{\kern0pt}{\isachardoublequoteclose}\isanewline
\ \ \ \ \isacommand{by}\isamarkupfalse%
\ {\isacharparenleft}{\kern0pt}subst\ \ nand{\isacharunderscore}{\kern0pt}ar{\isadigit{1}}{\isacharcomma}{\kern0pt}simp{\isacharcomma}{\kern0pt}simp{\isacharcomma}{\kern0pt}simp{\isacharcomma}{\kern0pt}subst\ nand{\isacharunderscore}{\kern0pt}ar{\isadigit{2}}{\isacharcomma}{\kern0pt}simp{\isacharplus}{\kern0pt}{\isacharparenright}{\kern0pt}\isanewline
\ \ \isacommand{then}\isamarkupfalse%
\ \isacommand{have}\isamarkupfalse%
\ {\isachardoublequoteopen}arity{\isacharparenleft}{\kern0pt}p{\isacharparenright}{\kern0pt}{\isasymle}n{\isachardoublequoteclose}\isanewline
\ \ \ \ \isakeyword{and}\ {\isachardoublequoteopen}arity{\isacharparenleft}{\kern0pt}q{\isacharparenright}{\kern0pt}{\isasymle}n{\isachardoublequoteclose}\ \isacommand{using}\isamarkupfalse%
\ Nand\isanewline
\ \ \ \ \isacommand{by}\isamarkupfalse%
\ {\isacharparenleft}{\kern0pt}rule{\isacharunderscore}{\kern0pt}tac\ j{\isacharequal}{\kern0pt}{\isachardoublequoteopen}arity{\isacharparenleft}{\kern0pt}Nand{\isacharparenleft}{\kern0pt}p{\isacharcomma}{\kern0pt}q{\isacharparenright}{\kern0pt}{\isacharparenright}{\kern0pt}{\isachardoublequoteclose}\ \isakeyword{in}\ le{\isacharunderscore}{\kern0pt}trans{\isacharcomma}{\kern0pt}simp{\isacharcomma}{\kern0pt}simp{\isacharparenright}{\kern0pt}{\isacharplus}{\kern0pt}\isanewline
\ \ \isacommand{then}\isamarkupfalse%
\ \isacommand{have}\isamarkupfalse%
\ {\isachardoublequoteopen}arity{\isacharparenleft}{\kern0pt}ren{\isacharparenleft}{\kern0pt}p{\isacharparenright}{\kern0pt}{\isacharbackquote}{\kern0pt}n{\isacharbackquote}{\kern0pt}m{\isacharbackquote}{\kern0pt}f{\isacharparenright}{\kern0pt}\ {\isasymle}\ m{\isachardoublequoteclose}\ \isakeyword{and}\ \ {\isachardoublequoteopen}arity{\isacharparenleft}{\kern0pt}ren{\isacharparenleft}{\kern0pt}q{\isacharparenright}{\kern0pt}{\isacharbackquote}{\kern0pt}n{\isacharbackquote}{\kern0pt}m{\isacharbackquote}{\kern0pt}f{\isacharparenright}{\kern0pt}\ {\isasymle}\ m{\isachardoublequoteclose}\isanewline
\ \ \ \ \isacommand{using}\isamarkupfalse%
\ Nand\ \isacommand{by}\isamarkupfalse%
\ auto\isanewline
\ \ \isacommand{then}\isamarkupfalse%
\ \isacommand{show}\isamarkupfalse%
\ {\isacharquery}{\kern0pt}case\ \isacommand{using}\isamarkupfalse%
\ Nand\ \isacommand{by}\isamarkupfalse%
\ {\isacharparenleft}{\kern0pt}simp\ add{\isacharcolon}{\kern0pt}Un{\isacharunderscore}{\kern0pt}least{\isacharunderscore}{\kern0pt}lt{\isacharparenright}{\kern0pt}\isanewline
\isacommand{next}\isamarkupfalse%
\isanewline
\ \ \isacommand{case}\isamarkupfalse%
\ {\isacharparenleft}{\kern0pt}Forall\ p{\isacharparenright}{\kern0pt}\isanewline
\ \ \isacommand{from}\isamarkupfalse%
\ Forall\ \isacommand{have}\isamarkupfalse%
\ {\isachardoublequoteopen}succ{\isacharparenleft}{\kern0pt}n{\isacharparenright}{\kern0pt}{\isasymin}nat{\isachardoublequoteclose}\ \ {\isachardoublequoteopen}succ{\isacharparenleft}{\kern0pt}m{\isacharparenright}{\kern0pt}{\isasymin}nat{\isachardoublequoteclose}\ \isacommand{by}\isamarkupfalse%
\ auto\isanewline
\ \ \isacommand{from}\isamarkupfalse%
\ Forall\ \isacommand{have}\isamarkupfalse%
\ {\isadigit{2}}{\isacharcolon}{\kern0pt}\ {\isachardoublequoteopen}sum{\isacharunderscore}{\kern0pt}id{\isacharparenleft}{\kern0pt}n{\isacharcomma}{\kern0pt}f{\isacharparenright}{\kern0pt}\ {\isasymin}\ succ{\isacharparenleft}{\kern0pt}n{\isacharparenright}{\kern0pt}{\isasymrightarrow}succ{\isacharparenleft}{\kern0pt}m{\isacharparenright}{\kern0pt}{\isachardoublequoteclose}\ \isacommand{by}\isamarkupfalse%
\ {\isacharparenleft}{\kern0pt}simp\ add{\isacharcolon}{\kern0pt}sum{\isacharunderscore}{\kern0pt}id{\isacharunderscore}{\kern0pt}tc{\isacharparenright}{\kern0pt}\isanewline
\ \ \isacommand{from}\isamarkupfalse%
\ Forall\ \isacommand{have}\isamarkupfalse%
\ {\isadigit{3}}{\isacharcolon}{\kern0pt}{\isachardoublequoteopen}arity{\isacharparenleft}{\kern0pt}p{\isacharparenright}{\kern0pt}\ {\isasymle}\ succ{\isacharparenleft}{\kern0pt}n{\isacharparenright}{\kern0pt}{\isachardoublequoteclose}\ \isacommand{by}\isamarkupfalse%
\ {\isacharparenleft}{\kern0pt}rule{\isacharunderscore}{\kern0pt}tac\ n{\isacharequal}{\kern0pt}{\isachardoublequoteopen}arity{\isacharparenleft}{\kern0pt}p{\isacharparenright}{\kern0pt}{\isachardoublequoteclose}\ \isakeyword{in}\ natE{\isacharcomma}{\kern0pt}simp{\isacharplus}{\kern0pt}{\isacharparenright}{\kern0pt}\isanewline
\ \ \isacommand{then}\isamarkupfalse%
\ \isacommand{have}\isamarkupfalse%
\ {\isachardoublequoteopen}arity{\isacharparenleft}{\kern0pt}ren{\isacharparenleft}{\kern0pt}p{\isacharparenright}{\kern0pt}{\isacharbackquote}{\kern0pt}succ{\isacharparenleft}{\kern0pt}n{\isacharparenright}{\kern0pt}{\isacharbackquote}{\kern0pt}succ{\isacharparenleft}{\kern0pt}m{\isacharparenright}{\kern0pt}{\isacharbackquote}{\kern0pt}sum{\isacharunderscore}{\kern0pt}id{\isacharparenleft}{\kern0pt}n{\isacharcomma}{\kern0pt}f{\isacharparenright}{\kern0pt}{\isacharparenright}{\kern0pt}{\isasymle}succ{\isacharparenleft}{\kern0pt}m{\isacharparenright}{\kern0pt}{\isachardoublequoteclose}\ \isacommand{using}\isamarkupfalse%
\isanewline
\ \ \ \ \ \ Forall\ {\isacartoucheopen}succ{\isacharparenleft}{\kern0pt}n{\isacharparenright}{\kern0pt}{\isasymin}nat{\isacartoucheclose}\ {\isacartoucheopen}succ{\isacharparenleft}{\kern0pt}m{\isacharparenright}{\kern0pt}{\isasymin}nat{\isacartoucheclose}\ {\isadigit{2}}\ \isacommand{by}\isamarkupfalse%
\ force\isanewline
\ \ \isacommand{then}\isamarkupfalse%
\ \isacommand{show}\isamarkupfalse%
\ {\isacharquery}{\kern0pt}case\ \isacommand{using}\isamarkupfalse%
\ Forall\ {\isadigit{2}}\ {\isadigit{3}}\ ren{\isacharunderscore}{\kern0pt}tc\ arity{\isacharunderscore}{\kern0pt}type\ pred{\isacharunderscore}{\kern0pt}le\ \isacommand{by}\isamarkupfalse%
\ auto\isanewline
\isacommand{qed}\isamarkupfalse%
%
\endisatagproof
{\isafoldproof}%
%
\isadelimproof
\isanewline
%
\endisadelimproof
\isanewline
\isacommand{lemma}\isamarkupfalse%
\ arity{\isacharunderscore}{\kern0pt}forallE\ {\isacharcolon}{\kern0pt}\ {\isachardoublequoteopen}p\ {\isasymin}\ formula\ {\isasymLongrightarrow}\ m\ {\isasymin}\ nat\ {\isasymLongrightarrow}\ arity{\isacharparenleft}{\kern0pt}Forall{\isacharparenleft}{\kern0pt}p{\isacharparenright}{\kern0pt}{\isacharparenright}{\kern0pt}\ {\isasymle}\ m\ {\isasymLongrightarrow}\ arity{\isacharparenleft}{\kern0pt}p{\isacharparenright}{\kern0pt}\ {\isasymle}\ succ{\isacharparenleft}{\kern0pt}m{\isacharparenright}{\kern0pt}{\isachardoublequoteclose}\isanewline
%
\isadelimproof
\ \ %
\endisadelimproof
%
\isatagproof
\isacommand{by}\isamarkupfalse%
{\isacharparenleft}{\kern0pt}rule{\isacharunderscore}{\kern0pt}tac\ n{\isacharequal}{\kern0pt}{\isachardoublequoteopen}arity{\isacharparenleft}{\kern0pt}p{\isacharparenright}{\kern0pt}{\isachardoublequoteclose}\ \isakeyword{in}\ natE{\isacharcomma}{\kern0pt}erule\ arity{\isacharunderscore}{\kern0pt}type{\isacharcomma}{\kern0pt}simp{\isacharplus}{\kern0pt}{\isacharparenright}{\kern0pt}%
\endisatagproof
{\isafoldproof}%
%
\isadelimproof
\isanewline
%
\endisadelimproof
\isanewline
\isacommand{lemma}\isamarkupfalse%
\ env{\isacharunderscore}{\kern0pt}coincidence{\isacharunderscore}{\kern0pt}sum{\isacharunderscore}{\kern0pt}id\ {\isacharcolon}{\kern0pt}\isanewline
\ \ \isakeyword{assumes}\ {\isachardoublequoteopen}m\ {\isasymin}\ nat{\isachardoublequoteclose}\ {\isachardoublequoteopen}n\ {\isasymin}\ nat{\isachardoublequoteclose}\isanewline
\ \ \ \ {\isachardoublequoteopen}{\isasymrho}\ {\isasymin}\ list{\isacharparenleft}{\kern0pt}A{\isacharparenright}{\kern0pt}{\isachardoublequoteclose}\ {\isachardoublequoteopen}{\isasymrho}{\isacharprime}{\kern0pt}\ {\isasymin}\ list{\isacharparenleft}{\kern0pt}A{\isacharparenright}{\kern0pt}{\isachardoublequoteclose}\isanewline
\ \ \ \ {\isachardoublequoteopen}f\ {\isasymin}\ n\ {\isasymrightarrow}\ m{\isachardoublequoteclose}\isanewline
\ \ \ \ {\isachardoublequoteopen}{\isasymAnd}\ i\ {\isachardot}{\kern0pt}\ i\ {\isacharless}{\kern0pt}\ n\ {\isasymLongrightarrow}\ nth{\isacharparenleft}{\kern0pt}i{\isacharcomma}{\kern0pt}{\isasymrho}{\isacharparenright}{\kern0pt}\ {\isacharequal}{\kern0pt}\ nth{\isacharparenleft}{\kern0pt}f{\isacharbackquote}{\kern0pt}i{\isacharcomma}{\kern0pt}{\isasymrho}{\isacharprime}{\kern0pt}{\isacharparenright}{\kern0pt}{\isachardoublequoteclose}\isanewline
\ \ \ \ {\isachardoublequoteopen}a\ {\isasymin}\ A{\isachardoublequoteclose}\ {\isachardoublequoteopen}j\ {\isasymin}\ succ{\isacharparenleft}{\kern0pt}n{\isacharparenright}{\kern0pt}{\isachardoublequoteclose}\isanewline
\ \ \isakeyword{shows}\ {\isachardoublequoteopen}nth{\isacharparenleft}{\kern0pt}j{\isacharcomma}{\kern0pt}Cons{\isacharparenleft}{\kern0pt}a{\isacharcomma}{\kern0pt}{\isasymrho}{\isacharparenright}{\kern0pt}{\isacharparenright}{\kern0pt}\ {\isacharequal}{\kern0pt}\ nth{\isacharparenleft}{\kern0pt}sum{\isacharunderscore}{\kern0pt}id{\isacharparenleft}{\kern0pt}n{\isacharcomma}{\kern0pt}f{\isacharparenright}{\kern0pt}{\isacharbackquote}{\kern0pt}j{\isacharcomma}{\kern0pt}Cons{\isacharparenleft}{\kern0pt}a{\isacharcomma}{\kern0pt}{\isasymrho}{\isacharprime}{\kern0pt}{\isacharparenright}{\kern0pt}{\isacharparenright}{\kern0pt}{\isachardoublequoteclose}\isanewline
%
\isadelimproof
%
\endisadelimproof
%
\isatagproof
\isacommand{proof}\isamarkupfalse%
\ {\isacharminus}{\kern0pt}\isanewline
\ \ \isacommand{let}\isamarkupfalse%
\ {\isacharquery}{\kern0pt}g{\isacharequal}{\kern0pt}{\isachardoublequoteopen}sum{\isacharunderscore}{\kern0pt}id{\isacharparenleft}{\kern0pt}n{\isacharcomma}{\kern0pt}f{\isacharparenright}{\kern0pt}{\isachardoublequoteclose}\isanewline
\ \ \isacommand{have}\isamarkupfalse%
\ {\isachardoublequoteopen}succ{\isacharparenleft}{\kern0pt}n{\isacharparenright}{\kern0pt}\ {\isasymin}\ nat{\isachardoublequoteclose}\ \isacommand{using}\isamarkupfalse%
\ {\isacartoucheopen}n{\isasymin}nat{\isacartoucheclose}\ \isacommand{by}\isamarkupfalse%
\ simp\isanewline
\ \ \isacommand{then}\isamarkupfalse%
\ \isacommand{have}\isamarkupfalse%
\ {\isachardoublequoteopen}j\ {\isasymin}\ nat{\isachardoublequoteclose}\ \isacommand{using}\isamarkupfalse%
\ {\isacartoucheopen}j{\isasymin}succ{\isacharparenleft}{\kern0pt}n{\isacharparenright}{\kern0pt}{\isacartoucheclose}\ in{\isacharunderscore}{\kern0pt}n{\isacharunderscore}{\kern0pt}in{\isacharunderscore}{\kern0pt}nat\ \isacommand{by}\isamarkupfalse%
\ blast\isanewline
\ \ \isacommand{then}\isamarkupfalse%
\ \isacommand{have}\isamarkupfalse%
\ {\isachardoublequoteopen}nth{\isacharparenleft}{\kern0pt}j{\isacharcomma}{\kern0pt}Cons{\isacharparenleft}{\kern0pt}a{\isacharcomma}{\kern0pt}{\isasymrho}{\isacharparenright}{\kern0pt}{\isacharparenright}{\kern0pt}\ {\isacharequal}{\kern0pt}\ nth{\isacharparenleft}{\kern0pt}{\isacharquery}{\kern0pt}g{\isacharbackquote}{\kern0pt}j{\isacharcomma}{\kern0pt}Cons{\isacharparenleft}{\kern0pt}a{\isacharcomma}{\kern0pt}{\isasymrho}{\isacharprime}{\kern0pt}{\isacharparenright}{\kern0pt}{\isacharparenright}{\kern0pt}{\isachardoublequoteclose}\isanewline
\ \ \isacommand{proof}\isamarkupfalse%
\ {\isacharparenleft}{\kern0pt}cases\ rule{\isacharcolon}{\kern0pt}natE{\isacharbrackleft}{\kern0pt}OF\ {\isacartoucheopen}j{\isasymin}nat{\isacartoucheclose}{\isacharbrackright}{\kern0pt}{\isacharparenright}{\kern0pt}\isanewline
\ \ \ \ \isacommand{case}\isamarkupfalse%
\ {\isadigit{1}}\isanewline
\ \ \ \ \isacommand{then}\isamarkupfalse%
\ \isacommand{show}\isamarkupfalse%
\ {\isacharquery}{\kern0pt}thesis\ \isacommand{using}\isamarkupfalse%
\ assms\ sum{\isacharunderscore}{\kern0pt}id{\isadigit{0}}\ \isacommand{by}\isamarkupfalse%
\ simp\isanewline
\ \ \isacommand{next}\isamarkupfalse%
\isanewline
\ \ \ \ \isacommand{case}\isamarkupfalse%
\ {\isacharparenleft}{\kern0pt}{\isadigit{2}}\ i{\isacharparenright}{\kern0pt}\isanewline
\ \ \ \ \isacommand{with}\isamarkupfalse%
\ {\isacartoucheopen}j{\isasymin}succ{\isacharparenleft}{\kern0pt}n{\isacharparenright}{\kern0pt}{\isacartoucheclose}\ \isacommand{have}\isamarkupfalse%
\ {\isachardoublequoteopen}succ{\isacharparenleft}{\kern0pt}i{\isacharparenright}{\kern0pt}{\isasymin}succ{\isacharparenleft}{\kern0pt}n{\isacharparenright}{\kern0pt}{\isachardoublequoteclose}\ \isacommand{by}\isamarkupfalse%
\ simp\isanewline
\ \ \ \ \isacommand{with}\isamarkupfalse%
\ {\isacartoucheopen}n{\isasymin}nat{\isacartoucheclose}\ \isacommand{have}\isamarkupfalse%
\ {\isachardoublequoteopen}i\ {\isasymin}\ n{\isachardoublequoteclose}\ \isacommand{using}\isamarkupfalse%
\ nat{\isacharunderscore}{\kern0pt}succD\ assms\ \isacommand{by}\isamarkupfalse%
\ simp\isanewline
\ \ \ \ \isacommand{have}\isamarkupfalse%
\ {\isachardoublequoteopen}f{\isacharbackquote}{\kern0pt}i{\isasymin}m{\isachardoublequoteclose}\ \isacommand{using}\isamarkupfalse%
\ {\isacartoucheopen}f{\isasymin}n{\isasymrightarrow}m{\isacartoucheclose}\ apply{\isacharunderscore}{\kern0pt}type\ {\isacartoucheopen}i{\isasymin}n{\isacartoucheclose}\ \isacommand{by}\isamarkupfalse%
\ simp\isanewline
\ \ \ \ \isacommand{then}\isamarkupfalse%
\ \isacommand{have}\isamarkupfalse%
\ {\isachardoublequoteopen}f{\isacharbackquote}{\kern0pt}i\ {\isasymin}\ nat{\isachardoublequoteclose}\ \isacommand{using}\isamarkupfalse%
\ in{\isacharunderscore}{\kern0pt}n{\isacharunderscore}{\kern0pt}in{\isacharunderscore}{\kern0pt}nat\ {\isacartoucheopen}m{\isasymin}nat{\isacartoucheclose}\ \isacommand{by}\isamarkupfalse%
\ simp\isanewline
\ \ \ \ \isacommand{have}\isamarkupfalse%
\ {\isachardoublequoteopen}nth{\isacharparenleft}{\kern0pt}succ{\isacharparenleft}{\kern0pt}i{\isacharparenright}{\kern0pt}{\isacharcomma}{\kern0pt}Cons{\isacharparenleft}{\kern0pt}a{\isacharcomma}{\kern0pt}{\isasymrho}{\isacharparenright}{\kern0pt}{\isacharparenright}{\kern0pt}\ {\isacharequal}{\kern0pt}\ nth{\isacharparenleft}{\kern0pt}i{\isacharcomma}{\kern0pt}{\isasymrho}{\isacharparenright}{\kern0pt}{\isachardoublequoteclose}\ \isacommand{using}\isamarkupfalse%
\ {\isacartoucheopen}i{\isasymin}nat{\isacartoucheclose}\ \isacommand{by}\isamarkupfalse%
\ simp\isanewline
\ \ \ \ \isacommand{also}\isamarkupfalse%
\ \isacommand{have}\isamarkupfalse%
\ {\isachardoublequoteopen}{\isachardot}{\kern0pt}{\isachardot}{\kern0pt}{\isachardot}{\kern0pt}\ {\isacharequal}{\kern0pt}\ nth{\isacharparenleft}{\kern0pt}f{\isacharbackquote}{\kern0pt}i{\isacharcomma}{\kern0pt}{\isasymrho}{\isacharprime}{\kern0pt}{\isacharparenright}{\kern0pt}{\isachardoublequoteclose}\ \isacommand{using}\isamarkupfalse%
\ assms\ {\isacartoucheopen}i{\isasymin}n{\isacartoucheclose}\ ltI\ \isacommand{by}\isamarkupfalse%
\ simp\isanewline
\ \ \ \ \isacommand{also}\isamarkupfalse%
\ \isacommand{have}\isamarkupfalse%
\ {\isachardoublequoteopen}{\isachardot}{\kern0pt}{\isachardot}{\kern0pt}{\isachardot}{\kern0pt}\ {\isacharequal}{\kern0pt}\ nth{\isacharparenleft}{\kern0pt}succ{\isacharparenleft}{\kern0pt}f{\isacharbackquote}{\kern0pt}i{\isacharparenright}{\kern0pt}{\isacharcomma}{\kern0pt}Cons{\isacharparenleft}{\kern0pt}a{\isacharcomma}{\kern0pt}{\isasymrho}{\isacharprime}{\kern0pt}{\isacharparenright}{\kern0pt}{\isacharparenright}{\kern0pt}{\isachardoublequoteclose}\ \isacommand{using}\isamarkupfalse%
\ {\isacartoucheopen}f{\isacharbackquote}{\kern0pt}i{\isasymin}nat{\isacartoucheclose}\ \isacommand{by}\isamarkupfalse%
\ simp\isanewline
\ \ \ \ \isacommand{also}\isamarkupfalse%
\ \isacommand{have}\isamarkupfalse%
\ {\isachardoublequoteopen}{\isachardot}{\kern0pt}{\isachardot}{\kern0pt}{\isachardot}{\kern0pt}\ {\isacharequal}{\kern0pt}\ nth{\isacharparenleft}{\kern0pt}{\isacharquery}{\kern0pt}g{\isacharbackquote}{\kern0pt}succ{\isacharparenleft}{\kern0pt}i{\isacharparenright}{\kern0pt}{\isacharcomma}{\kern0pt}Cons{\isacharparenleft}{\kern0pt}a{\isacharcomma}{\kern0pt}{\isasymrho}{\isacharprime}{\kern0pt}{\isacharparenright}{\kern0pt}{\isacharparenright}{\kern0pt}{\isachardoublequoteclose}\isanewline
\ \ \ \ \ \ \isacommand{using}\isamarkupfalse%
\ assms\ sum{\isacharunderscore}{\kern0pt}idS{\isacharbrackleft}{\kern0pt}OF\ {\isacartoucheopen}n{\isasymin}nat{\isacartoucheclose}\ {\isacartoucheopen}m{\isasymin}nat{\isacartoucheclose}\ \ {\isacartoucheopen}f{\isasymin}n{\isasymrightarrow}m{\isacartoucheclose}\ {\isacartoucheopen}i\ {\isasymin}\ n{\isacartoucheclose}{\isacharbrackright}{\kern0pt}\ cases\ \isacommand{by}\isamarkupfalse%
\ simp\isanewline
\ \ \ \ \isacommand{finally}\isamarkupfalse%
\ \isacommand{have}\isamarkupfalse%
\ {\isachardoublequoteopen}nth{\isacharparenleft}{\kern0pt}succ{\isacharparenleft}{\kern0pt}i{\isacharparenright}{\kern0pt}{\isacharcomma}{\kern0pt}Cons{\isacharparenleft}{\kern0pt}a{\isacharcomma}{\kern0pt}{\isasymrho}{\isacharparenright}{\kern0pt}{\isacharparenright}{\kern0pt}\ {\isacharequal}{\kern0pt}\ nth{\isacharparenleft}{\kern0pt}{\isacharquery}{\kern0pt}g{\isacharbackquote}{\kern0pt}succ{\isacharparenleft}{\kern0pt}i{\isacharparenright}{\kern0pt}{\isacharcomma}{\kern0pt}Cons{\isacharparenleft}{\kern0pt}a{\isacharcomma}{\kern0pt}{\isasymrho}{\isacharprime}{\kern0pt}{\isacharparenright}{\kern0pt}{\isacharparenright}{\kern0pt}{\isachardoublequoteclose}\ \isacommand{{\isachardot}{\kern0pt}}\isamarkupfalse%
\isanewline
\ \ \ \ \isacommand{then}\isamarkupfalse%
\ \isacommand{show}\isamarkupfalse%
\ {\isacharquery}{\kern0pt}thesis\ \isacommand{using}\isamarkupfalse%
\ {\isacartoucheopen}j{\isacharequal}{\kern0pt}succ{\isacharparenleft}{\kern0pt}i{\isacharparenright}{\kern0pt}{\isacartoucheclose}\ \isacommand{by}\isamarkupfalse%
\ simp\isanewline
\ \ \isacommand{qed}\isamarkupfalse%
\isanewline
\ \ \isacommand{then}\isamarkupfalse%
\ \isacommand{show}\isamarkupfalse%
\ {\isacharquery}{\kern0pt}thesis\ \isacommand{{\isachardot}{\kern0pt}}\isamarkupfalse%
\isanewline
\isacommand{qed}\isamarkupfalse%
%
\endisatagproof
{\isafoldproof}%
%
\isadelimproof
\isanewline
%
\endisadelimproof
\isanewline
\isacommand{lemma}\isamarkupfalse%
\ sats{\isacharunderscore}{\kern0pt}iff{\isacharunderscore}{\kern0pt}sats{\isacharunderscore}{\kern0pt}ren\ {\isacharcolon}{\kern0pt}\isanewline
\ \ \isakeyword{fixes}\ {\isachardoublequoteopen}{\isasymphi}{\isachardoublequoteclose}\isanewline
\ \ \isakeyword{assumes}\ {\isachardoublequoteopen}{\isasymphi}\ {\isasymin}\ formula{\isachardoublequoteclose}\isanewline
\ \ \isakeyword{shows}\ \ {\isachardoublequoteopen}{\isasymlbrakk}\ \ n\ {\isasymin}\ nat\ {\isacharsemicolon}{\kern0pt}\ m\ {\isasymin}\ nat\ {\isacharsemicolon}{\kern0pt}\ {\isasymrho}\ {\isasymin}\ list{\isacharparenleft}{\kern0pt}M{\isacharparenright}{\kern0pt}\ {\isacharsemicolon}{\kern0pt}\ {\isasymrho}{\isacharprime}{\kern0pt}\ {\isasymin}\ list{\isacharparenleft}{\kern0pt}M{\isacharparenright}{\kern0pt}\ {\isacharsemicolon}{\kern0pt}\ f\ {\isasymin}\ n\ {\isasymrightarrow}\ m\ {\isacharsemicolon}{\kern0pt}\isanewline
\ \ \ \ \ \ \ \ \ \ \ \ arity{\isacharparenleft}{\kern0pt}{\isasymphi}{\isacharparenright}{\kern0pt}\ {\isasymle}\ n\ {\isacharsemicolon}{\kern0pt}\isanewline
\ \ \ \ \ \ \ \ \ \ \ \ {\isasymAnd}\ i\ {\isachardot}{\kern0pt}\ i\ {\isacharless}{\kern0pt}\ n\ {\isasymLongrightarrow}\ nth{\isacharparenleft}{\kern0pt}i{\isacharcomma}{\kern0pt}{\isasymrho}{\isacharparenright}{\kern0pt}\ {\isacharequal}{\kern0pt}\ nth{\isacharparenleft}{\kern0pt}f{\isacharbackquote}{\kern0pt}i{\isacharcomma}{\kern0pt}{\isasymrho}{\isacharprime}{\kern0pt}{\isacharparenright}{\kern0pt}\ {\isasymrbrakk}\ {\isasymLongrightarrow}\isanewline
\ \ \ \ \ \ \ \ \ sats{\isacharparenleft}{\kern0pt}M{\isacharcomma}{\kern0pt}{\isasymphi}{\isacharcomma}{\kern0pt}{\isasymrho}{\isacharparenright}{\kern0pt}\ {\isasymlongleftrightarrow}\ sats{\isacharparenleft}{\kern0pt}M{\isacharcomma}{\kern0pt}ren{\isacharparenleft}{\kern0pt}{\isasymphi}{\isacharparenright}{\kern0pt}{\isacharbackquote}{\kern0pt}n{\isacharbackquote}{\kern0pt}m{\isacharbackquote}{\kern0pt}f{\isacharcomma}{\kern0pt}{\isasymrho}{\isacharprime}{\kern0pt}{\isacharparenright}{\kern0pt}{\isachardoublequoteclose}\isanewline
%
\isadelimproof
\ \ %
\endisadelimproof
%
\isatagproof
\isacommand{using}\isamarkupfalse%
\ {\isacartoucheopen}{\isasymphi}\ {\isasymin}\ formula{\isacartoucheclose}\isanewline
\isacommand{proof}\isamarkupfalse%
{\isacharparenleft}{\kern0pt}induct\ {\isasymphi}\ arbitrary{\isacharcolon}{\kern0pt}n\ m\ {\isasymrho}\ {\isasymrho}{\isacharprime}{\kern0pt}\ f{\isacharparenright}{\kern0pt}\isanewline
\ \ \isacommand{case}\isamarkupfalse%
\ {\isacharparenleft}{\kern0pt}Member\ x\ y{\isacharparenright}{\kern0pt}\isanewline
\ \ \isacommand{have}\isamarkupfalse%
\ {\isachardoublequoteopen}ren{\isacharparenleft}{\kern0pt}Member{\isacharparenleft}{\kern0pt}x{\isacharcomma}{\kern0pt}y{\isacharparenright}{\kern0pt}{\isacharparenright}{\kern0pt}{\isacharbackquote}{\kern0pt}n{\isacharbackquote}{\kern0pt}m{\isacharbackquote}{\kern0pt}f\ {\isacharequal}{\kern0pt}\ Member{\isacharparenleft}{\kern0pt}f{\isacharbackquote}{\kern0pt}x{\isacharcomma}{\kern0pt}f{\isacharbackquote}{\kern0pt}y{\isacharparenright}{\kern0pt}{\isachardoublequoteclose}\ \isacommand{using}\isamarkupfalse%
\ Member\ assms\ arity{\isacharunderscore}{\kern0pt}type\ \isacommand{by}\isamarkupfalse%
\ force\isanewline
\ \ \isacommand{moreover}\isamarkupfalse%
\isanewline
\ \ \isacommand{have}\isamarkupfalse%
\ {\isachardoublequoteopen}x\ {\isasymin}\ n{\isachardoublequoteclose}\ \isacommand{using}\isamarkupfalse%
\ Member\ arity{\isacharunderscore}{\kern0pt}meml\ \isacommand{by}\isamarkupfalse%
\ simp\isanewline
\ \ \isacommand{moreover}\isamarkupfalse%
\ \isanewline
\ \ \isacommand{have}\isamarkupfalse%
\ {\isachardoublequoteopen}y\ {\isasymin}\ n{\isachardoublequoteclose}\ \isacommand{using}\isamarkupfalse%
\ Member\ arity{\isacharunderscore}{\kern0pt}memr\ \isacommand{by}\isamarkupfalse%
\ simp\isanewline
\ \ \isacommand{ultimately}\isamarkupfalse%
\isanewline
\ \ \isacommand{show}\isamarkupfalse%
\ {\isacharquery}{\kern0pt}case\ \isacommand{using}\isamarkupfalse%
\ Member\ ltI\ \isacommand{by}\isamarkupfalse%
\ simp\isanewline
\isacommand{next}\isamarkupfalse%
\isanewline
\ \ \isacommand{case}\isamarkupfalse%
\ {\isacharparenleft}{\kern0pt}Equal\ x\ y{\isacharparenright}{\kern0pt}\isanewline
\ \ \isacommand{have}\isamarkupfalse%
\ {\isachardoublequoteopen}ren{\isacharparenleft}{\kern0pt}Equal{\isacharparenleft}{\kern0pt}x{\isacharcomma}{\kern0pt}y{\isacharparenright}{\kern0pt}{\isacharparenright}{\kern0pt}{\isacharbackquote}{\kern0pt}n{\isacharbackquote}{\kern0pt}m{\isacharbackquote}{\kern0pt}f\ {\isacharequal}{\kern0pt}\ Equal{\isacharparenleft}{\kern0pt}f{\isacharbackquote}{\kern0pt}x{\isacharcomma}{\kern0pt}f{\isacharbackquote}{\kern0pt}y{\isacharparenright}{\kern0pt}{\isachardoublequoteclose}\ \isacommand{using}\isamarkupfalse%
\ Equal\ assms\ arity{\isacharunderscore}{\kern0pt}type\ \isacommand{by}\isamarkupfalse%
\ force\isanewline
\ \ \isacommand{moreover}\isamarkupfalse%
\isanewline
\ \ \isacommand{have}\isamarkupfalse%
\ {\isachardoublequoteopen}x\ {\isasymin}\ n{\isachardoublequoteclose}\ \isacommand{using}\isamarkupfalse%
\ Equal\ arity{\isacharunderscore}{\kern0pt}eql\ \isacommand{by}\isamarkupfalse%
\ simp\isanewline
\ \ \isacommand{moreover}\isamarkupfalse%
\isanewline
\ \ \isacommand{have}\isamarkupfalse%
\ {\isachardoublequoteopen}y\ {\isasymin}\ n{\isachardoublequoteclose}\ \isacommand{using}\isamarkupfalse%
\ Equal\ arity{\isacharunderscore}{\kern0pt}eqr\ \isacommand{by}\isamarkupfalse%
\ simp\isanewline
\ \ \isacommand{ultimately}\isamarkupfalse%
\ \isacommand{show}\isamarkupfalse%
\ {\isacharquery}{\kern0pt}case\ \isacommand{using}\isamarkupfalse%
\ Equal\ ltI\ \isacommand{by}\isamarkupfalse%
\ simp\isanewline
\isacommand{next}\isamarkupfalse%
\isanewline
\ \ \isacommand{case}\isamarkupfalse%
\ {\isacharparenleft}{\kern0pt}Nand\ p\ q{\isacharparenright}{\kern0pt}\isanewline
\ \ \isacommand{have}\isamarkupfalse%
\ {\isachardoublequoteopen}ren{\isacharparenleft}{\kern0pt}Nand{\isacharparenleft}{\kern0pt}p{\isacharcomma}{\kern0pt}q{\isacharparenright}{\kern0pt}{\isacharparenright}{\kern0pt}{\isacharbackquote}{\kern0pt}n{\isacharbackquote}{\kern0pt}m{\isacharbackquote}{\kern0pt}f\ {\isacharequal}{\kern0pt}\ Nand{\isacharparenleft}{\kern0pt}ren{\isacharparenleft}{\kern0pt}p{\isacharparenright}{\kern0pt}{\isacharbackquote}{\kern0pt}n{\isacharbackquote}{\kern0pt}m{\isacharbackquote}{\kern0pt}f{\isacharcomma}{\kern0pt}ren{\isacharparenleft}{\kern0pt}q{\isacharparenright}{\kern0pt}{\isacharbackquote}{\kern0pt}n{\isacharbackquote}{\kern0pt}m{\isacharbackquote}{\kern0pt}f{\isacharparenright}{\kern0pt}{\isachardoublequoteclose}\ \isacommand{using}\isamarkupfalse%
\ Nand\ \isacommand{by}\isamarkupfalse%
\ simp\isanewline
\ \ \isacommand{moreover}\isamarkupfalse%
\isanewline
\ \ \isacommand{have}\isamarkupfalse%
\ {\isachardoublequoteopen}arity{\isacharparenleft}{\kern0pt}p{\isacharparenright}{\kern0pt}\ {\isasymle}\ n{\isachardoublequoteclose}\ \isacommand{using}\isamarkupfalse%
\ Nand\ nand{\isacharunderscore}{\kern0pt}ar{\isadigit{1}}D\ \isacommand{by}\isamarkupfalse%
\ simp\isanewline
\ \ \isacommand{moreover}\isamarkupfalse%
\ \isacommand{from}\isamarkupfalse%
\ this\isanewline
\ \ \isacommand{have}\isamarkupfalse%
\ {\isachardoublequoteopen}i\ {\isasymin}\ arity{\isacharparenleft}{\kern0pt}p{\isacharparenright}{\kern0pt}\ {\isasymLongrightarrow}\ i\ {\isasymin}\ n{\isachardoublequoteclose}\ \isakeyword{for}\ i\ \isacommand{using}\isamarkupfalse%
\ subsetD{\isacharbrackleft}{\kern0pt}OF\ le{\isacharunderscore}{\kern0pt}imp{\isacharunderscore}{\kern0pt}subset{\isacharbrackleft}{\kern0pt}OF\ {\isacartoucheopen}arity{\isacharparenleft}{\kern0pt}p{\isacharparenright}{\kern0pt}\ {\isasymle}\ n{\isacartoucheclose}{\isacharbrackright}{\kern0pt}{\isacharbrackright}{\kern0pt}\ \isacommand{by}\isamarkupfalse%
\ simp\isanewline
\ \ \isacommand{moreover}\isamarkupfalse%
\ \isacommand{from}\isamarkupfalse%
\ this\isanewline
\ \ \isacommand{have}\isamarkupfalse%
\ {\isachardoublequoteopen}i\ {\isasymin}\ arity{\isacharparenleft}{\kern0pt}p{\isacharparenright}{\kern0pt}\ {\isasymLongrightarrow}\ nth{\isacharparenleft}{\kern0pt}i{\isacharcomma}{\kern0pt}{\isasymrho}{\isacharparenright}{\kern0pt}\ {\isacharequal}{\kern0pt}\ nth{\isacharparenleft}{\kern0pt}f{\isacharbackquote}{\kern0pt}i{\isacharcomma}{\kern0pt}{\isasymrho}{\isacharprime}{\kern0pt}{\isacharparenright}{\kern0pt}{\isachardoublequoteclose}\ \isakeyword{for}\ i\ \isacommand{using}\isamarkupfalse%
\ Nand\ ltI\ \isacommand{by}\isamarkupfalse%
\ simp\isanewline
\ \ \isacommand{moreover}\isamarkupfalse%
\ \isacommand{from}\isamarkupfalse%
\ this\isanewline
\ \ \isacommand{have}\isamarkupfalse%
\ {\isachardoublequoteopen}sats{\isacharparenleft}{\kern0pt}M{\isacharcomma}{\kern0pt}p{\isacharcomma}{\kern0pt}{\isasymrho}{\isacharparenright}{\kern0pt}\ {\isasymlongleftrightarrow}\ sats{\isacharparenleft}{\kern0pt}M{\isacharcomma}{\kern0pt}ren{\isacharparenleft}{\kern0pt}p{\isacharparenright}{\kern0pt}{\isacharbackquote}{\kern0pt}n{\isacharbackquote}{\kern0pt}m{\isacharbackquote}{\kern0pt}f{\isacharcomma}{\kern0pt}{\isasymrho}{\isacharprime}{\kern0pt}{\isacharparenright}{\kern0pt}{\isachardoublequoteclose}\ \isacommand{using}\isamarkupfalse%
\ {\isacartoucheopen}arity{\isacharparenleft}{\kern0pt}p{\isacharparenright}{\kern0pt}{\isasymle}n{\isacartoucheclose}\ Nand\ \isacommand{by}\isamarkupfalse%
\ simp\isanewline
\ \ \isacommand{have}\isamarkupfalse%
\ {\isachardoublequoteopen}arity{\isacharparenleft}{\kern0pt}q{\isacharparenright}{\kern0pt}\ {\isasymle}\ n{\isachardoublequoteclose}\ \isacommand{using}\isamarkupfalse%
\ Nand\ nand{\isacharunderscore}{\kern0pt}ar{\isadigit{2}}D\ \isacommand{by}\isamarkupfalse%
\ simp\isanewline
\ \ \isacommand{moreover}\isamarkupfalse%
\ \isacommand{from}\isamarkupfalse%
\ this\isanewline
\ \ \isacommand{have}\isamarkupfalse%
\ {\isachardoublequoteopen}i\ {\isasymin}\ arity{\isacharparenleft}{\kern0pt}q{\isacharparenright}{\kern0pt}\ {\isasymLongrightarrow}\ i\ {\isasymin}\ n{\isachardoublequoteclose}\ \isakeyword{for}\ i\ \isacommand{using}\isamarkupfalse%
\ subsetD{\isacharbrackleft}{\kern0pt}OF\ le{\isacharunderscore}{\kern0pt}imp{\isacharunderscore}{\kern0pt}subset{\isacharbrackleft}{\kern0pt}OF\ {\isacartoucheopen}arity{\isacharparenleft}{\kern0pt}q{\isacharparenright}{\kern0pt}\ {\isasymle}\ n{\isacartoucheclose}{\isacharbrackright}{\kern0pt}{\isacharbrackright}{\kern0pt}\ \isacommand{by}\isamarkupfalse%
\ simp\isanewline
\ \ \isacommand{moreover}\isamarkupfalse%
\ \isacommand{from}\isamarkupfalse%
\ this\isanewline
\ \ \isacommand{have}\isamarkupfalse%
\ {\isachardoublequoteopen}i\ {\isasymin}\ arity{\isacharparenleft}{\kern0pt}q{\isacharparenright}{\kern0pt}\ {\isasymLongrightarrow}\ nth{\isacharparenleft}{\kern0pt}i{\isacharcomma}{\kern0pt}{\isasymrho}{\isacharparenright}{\kern0pt}\ {\isacharequal}{\kern0pt}\ nth{\isacharparenleft}{\kern0pt}f{\isacharbackquote}{\kern0pt}i{\isacharcomma}{\kern0pt}{\isasymrho}{\isacharprime}{\kern0pt}{\isacharparenright}{\kern0pt}{\isachardoublequoteclose}\ \isakeyword{for}\ i\ \isacommand{using}\isamarkupfalse%
\ Nand\ ltI\ \isacommand{by}\isamarkupfalse%
\ simp\isanewline
\ \ \isacommand{moreover}\isamarkupfalse%
\ \isacommand{from}\isamarkupfalse%
\ this\isanewline
\ \ \isacommand{have}\isamarkupfalse%
\ {\isachardoublequoteopen}sats{\isacharparenleft}{\kern0pt}M{\isacharcomma}{\kern0pt}q{\isacharcomma}{\kern0pt}{\isasymrho}{\isacharparenright}{\kern0pt}\ {\isasymlongleftrightarrow}\ sats{\isacharparenleft}{\kern0pt}M{\isacharcomma}{\kern0pt}ren{\isacharparenleft}{\kern0pt}q{\isacharparenright}{\kern0pt}{\isacharbackquote}{\kern0pt}n{\isacharbackquote}{\kern0pt}m{\isacharbackquote}{\kern0pt}f{\isacharcomma}{\kern0pt}{\isasymrho}{\isacharprime}{\kern0pt}{\isacharparenright}{\kern0pt}{\isachardoublequoteclose}\ \isacommand{using}\isamarkupfalse%
\ assms\ {\isacartoucheopen}arity{\isacharparenleft}{\kern0pt}q{\isacharparenright}{\kern0pt}{\isasymle}n{\isacartoucheclose}\ Nand\ \isacommand{by}\isamarkupfalse%
\ simp\isanewline
\ \ \isacommand{ultimately}\isamarkupfalse%
\isanewline
\ \ \isacommand{show}\isamarkupfalse%
\ {\isacharquery}{\kern0pt}case\ \isacommand{using}\isamarkupfalse%
\ Nand\ \isacommand{by}\isamarkupfalse%
\ simp\isanewline
\isacommand{next}\isamarkupfalse%
\isanewline
\ \ \isacommand{case}\isamarkupfalse%
\ {\isacharparenleft}{\kern0pt}Forall\ p{\isacharparenright}{\kern0pt}\isanewline
\ \ \isacommand{have}\isamarkupfalse%
\ {\isadigit{0}}{\isacharcolon}{\kern0pt}{\isachardoublequoteopen}ren{\isacharparenleft}{\kern0pt}Forall{\isacharparenleft}{\kern0pt}p{\isacharparenright}{\kern0pt}{\isacharparenright}{\kern0pt}{\isacharbackquote}{\kern0pt}n{\isacharbackquote}{\kern0pt}m{\isacharbackquote}{\kern0pt}f\ {\isacharequal}{\kern0pt}\ Forall{\isacharparenleft}{\kern0pt}ren{\isacharparenleft}{\kern0pt}p{\isacharparenright}{\kern0pt}{\isacharbackquote}{\kern0pt}succ{\isacharparenleft}{\kern0pt}n{\isacharparenright}{\kern0pt}{\isacharbackquote}{\kern0pt}succ{\isacharparenleft}{\kern0pt}m{\isacharparenright}{\kern0pt}{\isacharbackquote}{\kern0pt}sum{\isacharunderscore}{\kern0pt}id{\isacharparenleft}{\kern0pt}n{\isacharcomma}{\kern0pt}f{\isacharparenright}{\kern0pt}{\isacharparenright}{\kern0pt}{\isachardoublequoteclose}\isanewline
\ \ \ \ \isacommand{using}\isamarkupfalse%
\ Forall\ \isacommand{by}\isamarkupfalse%
\ simp\isanewline
\ \ \isacommand{have}\isamarkupfalse%
\ {\isadigit{1}}{\isacharcolon}{\kern0pt}{\isachardoublequoteopen}sum{\isacharunderscore}{\kern0pt}id{\isacharparenleft}{\kern0pt}n{\isacharcomma}{\kern0pt}f{\isacharparenright}{\kern0pt}\ {\isasymin}\ succ{\isacharparenleft}{\kern0pt}n{\isacharparenright}{\kern0pt}\ {\isasymrightarrow}\ succ{\isacharparenleft}{\kern0pt}m{\isacharparenright}{\kern0pt}{\isachardoublequoteclose}\ {\isacharparenleft}{\kern0pt}\isakeyword{is}\ {\isachardoublequoteopen}{\isacharquery}{\kern0pt}g\ {\isasymin}\ {\isacharunderscore}{\kern0pt}{\isachardoublequoteclose}{\isacharparenright}{\kern0pt}\ \isacommand{using}\isamarkupfalse%
\ sum{\isacharunderscore}{\kern0pt}id{\isacharunderscore}{\kern0pt}tc\ Forall\ \isacommand{by}\isamarkupfalse%
\ simp\isanewline
\ \ \isacommand{then}\isamarkupfalse%
\ \isacommand{have}\isamarkupfalse%
\ {\isadigit{2}}{\isacharcolon}{\kern0pt}\ {\isachardoublequoteopen}arity{\isacharparenleft}{\kern0pt}p{\isacharparenright}{\kern0pt}\ {\isasymle}\ succ{\isacharparenleft}{\kern0pt}n{\isacharparenright}{\kern0pt}{\isachardoublequoteclose}\isanewline
\ \ \ \ \isacommand{using}\isamarkupfalse%
\ Forall\ le{\isacharunderscore}{\kern0pt}trans{\isacharbrackleft}{\kern0pt}of\ {\isacharunderscore}{\kern0pt}\ {\isachardoublequoteopen}succ{\isacharparenleft}{\kern0pt}pred{\isacharparenleft}{\kern0pt}arity{\isacharparenleft}{\kern0pt}p{\isacharparenright}{\kern0pt}{\isacharparenright}{\kern0pt}{\isacharparenright}{\kern0pt}{\isachardoublequoteclose}{\isacharbrackright}{\kern0pt}\ succpred{\isacharunderscore}{\kern0pt}leI\ \isacommand{by}\isamarkupfalse%
\ simp\isanewline
\ \ \isacommand{have}\isamarkupfalse%
\ {\isachardoublequoteopen}succ{\isacharparenleft}{\kern0pt}n{\isacharparenright}{\kern0pt}{\isasymin}nat{\isachardoublequoteclose}\ {\isachardoublequoteopen}succ{\isacharparenleft}{\kern0pt}m{\isacharparenright}{\kern0pt}{\isasymin}nat{\isachardoublequoteclose}\ \isacommand{using}\isamarkupfalse%
\ Forall\ \isacommand{by}\isamarkupfalse%
\ auto\isanewline
\ \ \isacommand{then}\isamarkupfalse%
\ \isacommand{have}\isamarkupfalse%
\ A{\isacharcolon}{\kern0pt}{\isachardoublequoteopen}{\isasymAnd}\ j\ {\isachardot}{\kern0pt}j\ {\isacharless}{\kern0pt}\ succ{\isacharparenleft}{\kern0pt}n{\isacharparenright}{\kern0pt}\ {\isasymLongrightarrow}\ nth{\isacharparenleft}{\kern0pt}j{\isacharcomma}{\kern0pt}\ Cons{\isacharparenleft}{\kern0pt}a{\isacharcomma}{\kern0pt}\ {\isasymrho}{\isacharparenright}{\kern0pt}{\isacharparenright}{\kern0pt}\ {\isacharequal}{\kern0pt}\ nth{\isacharparenleft}{\kern0pt}{\isacharquery}{\kern0pt}g{\isacharbackquote}{\kern0pt}j{\isacharcomma}{\kern0pt}\ Cons{\isacharparenleft}{\kern0pt}a{\isacharcomma}{\kern0pt}\ {\isasymrho}{\isacharprime}{\kern0pt}{\isacharparenright}{\kern0pt}{\isacharparenright}{\kern0pt}{\isachardoublequoteclose}\ \isakeyword{if}\ {\isachardoublequoteopen}a{\isasymin}M{\isachardoublequoteclose}\ \isakeyword{for}\ a\isanewline
\ \ \ \ \isacommand{using}\isamarkupfalse%
\ that\ env{\isacharunderscore}{\kern0pt}coincidence{\isacharunderscore}{\kern0pt}sum{\isacharunderscore}{\kern0pt}id\ Forall\ ltD\ \isacommand{by}\isamarkupfalse%
\ force\isanewline
\ \ \isacommand{have}\isamarkupfalse%
\isanewline
\ \ \ \ {\isachardoublequoteopen}sats{\isacharparenleft}{\kern0pt}M{\isacharcomma}{\kern0pt}p{\isacharcomma}{\kern0pt}Cons{\isacharparenleft}{\kern0pt}a{\isacharcomma}{\kern0pt}{\isasymrho}{\isacharparenright}{\kern0pt}{\isacharparenright}{\kern0pt}\ {\isasymlongleftrightarrow}\ sats{\isacharparenleft}{\kern0pt}M{\isacharcomma}{\kern0pt}ren{\isacharparenleft}{\kern0pt}p{\isacharparenright}{\kern0pt}{\isacharbackquote}{\kern0pt}succ{\isacharparenleft}{\kern0pt}n{\isacharparenright}{\kern0pt}{\isacharbackquote}{\kern0pt}succ{\isacharparenleft}{\kern0pt}m{\isacharparenright}{\kern0pt}{\isacharbackquote}{\kern0pt}{\isacharquery}{\kern0pt}g{\isacharcomma}{\kern0pt}Cons{\isacharparenleft}{\kern0pt}a{\isacharcomma}{\kern0pt}{\isasymrho}{\isacharprime}{\kern0pt}{\isacharparenright}{\kern0pt}{\isacharparenright}{\kern0pt}{\isachardoublequoteclose}\ \isakeyword{if}\ {\isachardoublequoteopen}a{\isasymin}M{\isachardoublequoteclose}\ \isakeyword{for}\ a\isanewline
\ \ \isacommand{proof}\isamarkupfalse%
\ {\isacharminus}{\kern0pt}\isanewline
\ \ \ \ \isacommand{have}\isamarkupfalse%
\ C{\isacharcolon}{\kern0pt}{\isachardoublequoteopen}Cons{\isacharparenleft}{\kern0pt}a{\isacharcomma}{\kern0pt}{\isasymrho}{\isacharparenright}{\kern0pt}\ {\isasymin}\ list{\isacharparenleft}{\kern0pt}M{\isacharparenright}{\kern0pt}{\isachardoublequoteclose}\ {\isachardoublequoteopen}Cons{\isacharparenleft}{\kern0pt}a{\isacharcomma}{\kern0pt}{\isasymrho}{\isacharprime}{\kern0pt}{\isacharparenright}{\kern0pt}{\isasymin}list{\isacharparenleft}{\kern0pt}M{\isacharparenright}{\kern0pt}{\isachardoublequoteclose}\ \isacommand{using}\isamarkupfalse%
\ Forall\ that\ \isacommand{by}\isamarkupfalse%
\ auto\isanewline
\ \ \ \ \isacommand{have}\isamarkupfalse%
\ {\isachardoublequoteopen}sats{\isacharparenleft}{\kern0pt}M{\isacharcomma}{\kern0pt}p{\isacharcomma}{\kern0pt}Cons{\isacharparenleft}{\kern0pt}a{\isacharcomma}{\kern0pt}{\isasymrho}{\isacharparenright}{\kern0pt}{\isacharparenright}{\kern0pt}\ {\isasymlongleftrightarrow}\ sats{\isacharparenleft}{\kern0pt}M{\isacharcomma}{\kern0pt}ren{\isacharparenleft}{\kern0pt}p{\isacharparenright}{\kern0pt}{\isacharbackquote}{\kern0pt}succ{\isacharparenleft}{\kern0pt}n{\isacharparenright}{\kern0pt}{\isacharbackquote}{\kern0pt}succ{\isacharparenleft}{\kern0pt}m{\isacharparenright}{\kern0pt}{\isacharbackquote}{\kern0pt}{\isacharquery}{\kern0pt}g{\isacharcomma}{\kern0pt}Cons{\isacharparenleft}{\kern0pt}a{\isacharcomma}{\kern0pt}{\isasymrho}{\isacharprime}{\kern0pt}{\isacharparenright}{\kern0pt}{\isacharparenright}{\kern0pt}{\isachardoublequoteclose}\isanewline
\ \ \ \ \ \ \isacommand{using}\isamarkupfalse%
\ Forall{\isacharparenleft}{\kern0pt}{\isadigit{2}}{\isacharparenright}{\kern0pt}{\isacharbrackleft}{\kern0pt}OF\ {\isacartoucheopen}succ{\isacharparenleft}{\kern0pt}n{\isacharparenright}{\kern0pt}{\isasymin}nat{\isacartoucheclose}\ {\isacartoucheopen}succ{\isacharparenleft}{\kern0pt}m{\isacharparenright}{\kern0pt}{\isasymin}nat{\isacartoucheclose}\ C{\isacharparenleft}{\kern0pt}{\isadigit{1}}{\isacharparenright}{\kern0pt}\ C{\isacharparenleft}{\kern0pt}{\isadigit{2}}{\isacharparenright}{\kern0pt}\ {\isadigit{1}}\ {\isadigit{2}}\ A{\isacharbrackleft}{\kern0pt}OF\ {\isacartoucheopen}a{\isasymin}M{\isacartoucheclose}{\isacharbrackright}{\kern0pt}{\isacharbrackright}{\kern0pt}\ \isacommand{by}\isamarkupfalse%
\ simp\isanewline
\ \ \ \ \isacommand{then}\isamarkupfalse%
\ \isacommand{show}\isamarkupfalse%
\ {\isacharquery}{\kern0pt}thesis\ \isacommand{{\isachardot}{\kern0pt}}\isamarkupfalse%
\isanewline
\ \ \isacommand{qed}\isamarkupfalse%
\isanewline
\ \ \isacommand{then}\isamarkupfalse%
\ \isacommand{show}\isamarkupfalse%
\ {\isacharquery}{\kern0pt}case\ \isacommand{using}\isamarkupfalse%
\ Forall\ {\isadigit{0}}\ {\isadigit{1}}\ {\isadigit{2}}\ \isacommand{by}\isamarkupfalse%
\ simp\isanewline
\isacommand{qed}\isamarkupfalse%
%
\endisatagproof
{\isafoldproof}%
%
\isadelimproof
\isanewline
%
\endisadelimproof
%
\isadelimtheory
\isanewline
%
\endisadelimtheory
%
\isatagtheory
\isacommand{end}\isamarkupfalse%
%
\endisatagtheory
{\isafoldtheory}%
%
\isadelimtheory
%
\endisadelimtheory
%
\end{isabellebody}%
\endinput
%:%file=~/source/repos/ZF-notAC/code/Forcing/Renaming.thy%:%
%:%11=1%:%
%:%27=3%:%
%:%28=3%:%
%:%29=4%:%
%:%30=5%:%
%:%31=6%:%
%:%32=7%:%
%:%37=7%:%
%:%40=8%:%
%:%41=9%:%
%:%42=9%:%
%:%43=10%:%
%:%44=11%:%
%:%51=12%:%
%:%52=12%:%
%:%53=13%:%
%:%54=13%:%
%:%55=14%:%
%:%56=14%:%
%:%57=14%:%
%:%58=14%:%
%:%59=14%:%
%:%60=15%:%
%:%61=15%:%
%:%62=16%:%
%:%63=16%:%
%:%64=17%:%
%:%65=17%:%
%:%66=17%:%
%:%67=17%:%
%:%68=17%:%
%:%69=18%:%
%:%84=20%:%
%:%94=22%:%
%:%95=22%:%
%:%96=23%:%
%:%97=24%:%
%:%98=25%:%
%:%99=26%:%
%:%100=26%:%
%:%101=27%:%
%:%102=28%:%
%:%103=29%:%
%:%110=30%:%
%:%111=30%:%
%:%112=31%:%
%:%113=31%:%
%:%114=32%:%
%:%115=32%:%
%:%116=33%:%
%:%117=34%:%
%:%118=34%:%
%:%119=35%:%
%:%120=35%:%
%:%121=36%:%
%:%122=36%:%
%:%123=36%:%
%:%124=37%:%
%:%125=37%:%
%:%126=38%:%
%:%127=38%:%
%:%128=39%:%
%:%129=39%:%
%:%130=40%:%
%:%131=40%:%
%:%132=40%:%
%:%133=41%:%
%:%134=41%:%
%:%135=42%:%
%:%136=42%:%
%:%137=42%:%
%:%138=43%:%
%:%139=43%:%
%:%140=43%:%
%:%141=43%:%
%:%142=44%:%
%:%143=44%:%
%:%144=45%:%
%:%145=45%:%
%:%146=46%:%
%:%147=46%:%
%:%148=47%:%
%:%149=47%:%
%:%150=48%:%
%:%151=48%:%
%:%152=49%:%
%:%153=49%:%
%:%154=50%:%
%:%155=50%:%
%:%156=51%:%
%:%157=51%:%
%:%158=51%:%
%:%159=51%:%
%:%160=52%:%
%:%161=52%:%
%:%162=53%:%
%:%163=53%:%
%:%164=53%:%
%:%165=54%:%
%:%166=54%:%
%:%167=54%:%
%:%168=54%:%
%:%169=55%:%
%:%170=55%:%
%:%171=56%:%
%:%172=56%:%
%:%173=56%:%
%:%174=56%:%
%:%175=56%:%
%:%176=57%:%
%:%177=57%:%
%:%178=58%:%
%:%179=58%:%
%:%180=58%:%
%:%181=59%:%
%:%182=59%:%
%:%183=60%:%
%:%184=60%:%
%:%185=60%:%
%:%186=60%:%
%:%187=61%:%
%:%188=61%:%
%:%189=62%:%
%:%190=62%:%
%:%191=62%:%
%:%192=62%:%
%:%193=63%:%
%:%194=63%:%
%:%195=64%:%
%:%196=64%:%
%:%197=64%:%
%:%198=64%:%
%:%199=65%:%
%:%205=65%:%
%:%208=66%:%
%:%209=67%:%
%:%210=67%:%
%:%211=68%:%
%:%212=69%:%
%:%213=70%:%
%:%214=71%:%
%:%215=72%:%
%:%216=73%:%
%:%217=74%:%
%:%218=75%:%
%:%225=76%:%
%:%226=76%:%
%:%227=77%:%
%:%228=77%:%
%:%229=78%:%
%:%230=78%:%
%:%231=79%:%
%:%232=79%:%
%:%233=80%:%
%:%234=80%:%
%:%235=81%:%
%:%236=81%:%
%:%237=82%:%
%:%238=82%:%
%:%239=83%:%
%:%240=83%:%
%:%241=84%:%
%:%242=84%:%
%:%243=84%:%
%:%244=85%:%
%:%245=85%:%
%:%246=86%:%
%:%247=86%:%
%:%248=86%:%
%:%249=87%:%
%:%250=87%:%
%:%251=87%:%
%:%252=87%:%
%:%253=88%:%
%:%254=88%:%
%:%255=89%:%
%:%256=89%:%
%:%257=90%:%
%:%258=90%:%
%:%259=91%:%
%:%260=91%:%
%:%261=92%:%
%:%262=92%:%
%:%263=93%:%
%:%264=93%:%
%:%265=94%:%
%:%266=94%:%
%:%267=95%:%
%:%268=95%:%
%:%269=95%:%
%:%270=96%:%
%:%271=96%:%
%:%272=97%:%
%:%273=97%:%
%:%274=97%:%
%:%275=98%:%
%:%276=98%:%
%:%277=98%:%
%:%278=98%:%
%:%279=99%:%
%:%280=99%:%
%:%281=100%:%
%:%282=100%:%
%:%283=100%:%
%:%284=100%:%
%:%285=101%:%
%:%291=101%:%
%:%294=102%:%
%:%295=103%:%
%:%296=104%:%
%:%297=104%:%
%:%298=105%:%
%:%299=106%:%
%:%302=107%:%
%:%306=107%:%
%:%307=107%:%
%:%308=107%:%
%:%309=107%:%
%:%314=107%:%
%:%317=108%:%
%:%318=109%:%
%:%319=109%:%
%:%320=110%:%
%:%321=111%:%
%:%328=112%:%
%:%329=112%:%
%:%330=113%:%
%:%331=113%:%
%:%332=113%:%
%:%333=113%:%
%:%334=114%:%
%:%340=114%:%
%:%343=115%:%
%:%344=116%:%
%:%345=117%:%
%:%346=117%:%
%:%347=118%:%
%:%348=119%:%
%:%349=120%:%
%:%350=121%:%
%:%351=121%:%
%:%352=122%:%
%:%353=123%:%
%:%354=124%:%
%:%361=125%:%
%:%362=125%:%
%:%363=126%:%
%:%364=126%:%
%:%365=127%:%
%:%366=127%:%
%:%367=128%:%
%:%368=128%:%
%:%369=128%:%
%:%370=129%:%
%:%371=129%:%
%:%372=130%:%
%:%373=130%:%
%:%374=131%:%
%:%375=131%:%
%:%376=131%:%
%:%377=132%:%
%:%378=132%:%
%:%379=133%:%
%:%380=133%:%
%:%381=134%:%
%:%382=134%:%
%:%383=134%:%
%:%384=135%:%
%:%385=135%:%
%:%386=136%:%
%:%387=136%:%
%:%388=136%:%
%:%389=136%:%
%:%390=137%:%
%:%396=137%:%
%:%399=138%:%
%:%400=139%:%
%:%401=139%:%
%:%402=140%:%
%:%403=141%:%
%:%404=142%:%
%:%411=143%:%
%:%412=143%:%
%:%413=144%:%
%:%414=144%:%
%:%415=145%:%
%:%416=145%:%
%:%417=146%:%
%:%418=146%:%
%:%419=147%:%
%:%420=147%:%
%:%421=148%:%
%:%422=148%:%
%:%423=149%:%
%:%424=149%:%
%:%425=150%:%
%:%426=150%:%
%:%427=150%:%
%:%428=151%:%
%:%429=151%:%
%:%430=152%:%
%:%431=152%:%
%:%432=153%:%
%:%433=153%:%
%:%434=153%:%
%:%435=154%:%
%:%436=154%:%
%:%437=155%:%
%:%438=155%:%
%:%439=155%:%
%:%440=155%:%
%:%441=156%:%
%:%447=156%:%
%:%450=157%:%
%:%451=158%:%
%:%452=159%:%
%:%453=159%:%
%:%454=160%:%
%:%455=161%:%
%:%456=162%:%
%:%457=163%:%
%:%458=164%:%
%:%459=165%:%
%:%460=166%:%
%:%461=167%:%
%:%462=168%:%
%:%463=169%:%
%:%464=170%:%
%:%465=171%:%
%:%466=172%:%
%:%473=173%:%
%:%474=173%:%
%:%475=174%:%
%:%476=174%:%
%:%477=175%:%
%:%478=175%:%
%:%479=176%:%
%:%480=176%:%
%:%481=177%:%
%:%482=177%:%
%:%483=177%:%
%:%484=178%:%
%:%485=178%:%
%:%486=179%:%
%:%487=179%:%
%:%488=179%:%
%:%489=179%:%
%:%490=180%:%
%:%491=180%:%
%:%492=181%:%
%:%493=181%:%
%:%494=182%:%
%:%495=182%:%
%:%496=183%:%
%:%497=183%:%
%:%498=184%:%
%:%499=184%:%
%:%500=185%:%
%:%501=185%:%
%:%502=185%:%
%:%503=186%:%
%:%504=186%:%
%:%505=187%:%
%:%506=187%:%
%:%507=188%:%
%:%508=188%:%
%:%509=188%:%
%:%510=189%:%
%:%511=189%:%
%:%512=190%:%
%:%513=190%:%
%:%514=191%:%
%:%515=191%:%
%:%516=191%:%
%:%517=192%:%
%:%518=192%:%
%:%519=193%:%
%:%520=193%:%
%:%521=194%:%
%:%522=195%:%
%:%523=195%:%
%:%524=195%:%
%:%525=196%:%
%:%526=196%:%
%:%527=197%:%
%:%528=197%:%
%:%529=198%:%
%:%530=198%:%
%:%531=198%:%
%:%532=199%:%
%:%533=199%:%
%:%534=200%:%
%:%535=200%:%
%:%536=201%:%
%:%537=201%:%
%:%538=201%:%
%:%539=202%:%
%:%540=202%:%
%:%541=203%:%
%:%542=203%:%
%:%543=203%:%
%:%544=204%:%
%:%545=204%:%
%:%546=204%:%
%:%547=204%:%
%:%548=205%:%
%:%549=205%:%
%:%550=206%:%
%:%551=206%:%
%:%552=207%:%
%:%553=207%:%
%:%554=208%:%
%:%555=208%:%
%:%556=208%:%
%:%557=209%:%
%:%558=209%:%
%:%559=210%:%
%:%560=210%:%
%:%561=211%:%
%:%562=211%:%
%:%563=211%:%
%:%564=212%:%
%:%565=212%:%
%:%566=213%:%
%:%567=213%:%
%:%568=214%:%
%:%569=214%:%
%:%570=215%:%
%:%571=215%:%
%:%572=215%:%
%:%573=216%:%
%:%574=216%:%
%:%575=217%:%
%:%576=217%:%
%:%577=218%:%
%:%578=218%:%
%:%579=218%:%
%:%580=219%:%
%:%581=219%:%
%:%582=219%:%
%:%583=219%:%
%:%584=219%:%
%:%585=220%:%
%:%586=220%:%
%:%587=221%:%
%:%588=221%:%
%:%589=221%:%
%:%590=222%:%
%:%591=222%:%
%:%592=223%:%
%:%593=223%:%
%:%594=223%:%
%:%595=223%:%
%:%596=224%:%
%:%597=224%:%
%:%598=225%:%
%:%599=225%:%
%:%600=226%:%
%:%601=226%:%
%:%602=227%:%
%:%603=228%:%
%:%604=228%:%
%:%605=229%:%
%:%606=229%:%
%:%607=230%:%
%:%608=230%:%
%:%609=231%:%
%:%610=231%:%
%:%611=231%:%
%:%612=232%:%
%:%613=232%:%
%:%614=233%:%
%:%615=233%:%
%:%616=234%:%
%:%617=234%:%
%:%618=234%:%
%:%619=235%:%
%:%620=235%:%
%:%621=236%:%
%:%622=236%:%
%:%623=237%:%
%:%624=237%:%
%:%625=238%:%
%:%626=239%:%
%:%627=239%:%
%:%628=240%:%
%:%629=240%:%
%:%630=241%:%
%:%631=241%:%
%:%632=242%:%
%:%633=242%:%
%:%634=243%:%
%:%635=243%:%
%:%636=244%:%
%:%637=244%:%
%:%638=245%:%
%:%639=245%:%
%:%640=246%:%
%:%641=246%:%
%:%642=246%:%
%:%643=247%:%
%:%644=247%:%
%:%645=248%:%
%:%646=248%:%
%:%647=248%:%
%:%648=249%:%
%:%649=249%:%
%:%650=249%:%
%:%651=249%:%
%:%652=250%:%
%:%653=250%:%
%:%654=251%:%
%:%655=251%:%
%:%656=251%:%
%:%657=251%:%
%:%658=252%:%
%:%664=252%:%
%:%667=253%:%
%:%668=254%:%
%:%669=254%:%
%:%670=255%:%
%:%671=256%:%
%:%672=257%:%
%:%679=258%:%
%:%680=258%:%
%:%681=259%:%
%:%682=259%:%
%:%683=260%:%
%:%684=260%:%
%:%685=261%:%
%:%686=261%:%
%:%687=262%:%
%:%688=262%:%
%:%689=262%:%
%:%690=263%:%
%:%691=263%:%
%:%692=264%:%
%:%693=264%:%
%:%694=264%:%
%:%695=264%:%
%:%696=265%:%
%:%697=265%:%
%:%698=265%:%
%:%699=266%:%
%:%700=266%:%
%:%701=267%:%
%:%702=267%:%
%:%703=267%:%
%:%704=268%:%
%:%705=268%:%
%:%706=268%:%
%:%707=269%:%
%:%708=269%:%
%:%709=270%:%
%:%710=270%:%
%:%711=270%:%
%:%712=271%:%
%:%713=271%:%
%:%714=271%:%
%:%715=271%:%
%:%716=271%:%
%:%717=272%:%
%:%718=272%:%
%:%719=273%:%
%:%720=273%:%
%:%721=273%:%
%:%722=273%:%
%:%723=274%:%
%:%724=274%:%
%:%725=275%:%
%:%726=275%:%
%:%727=275%:%
%:%728=275%:%
%:%729=276%:%
%:%730=276%:%
%:%731=277%:%
%:%732=277%:%
%:%733=277%:%
%:%734=277%:%
%:%735=277%:%
%:%736=278%:%
%:%737=278%:%
%:%738=278%:%
%:%739=279%:%
%:%740=279%:%
%:%741=280%:%
%:%742=280%:%
%:%743=280%:%
%:%744=280%:%
%:%745=280%:%
%:%746=281%:%
%:%747=281%:%
%:%748=282%:%
%:%749=282%:%
%:%750=283%:%
%:%751=283%:%
%:%752=283%:%
%:%753=284%:%
%:%754=284%:%
%:%755=285%:%
%:%756=285%:%
%:%757=285%:%
%:%758=285%:%
%:%759=286%:%
%:%760=286%:%
%:%761=286%:%
%:%762=286%:%
%:%763=286%:%
%:%764=287%:%
%:%765=287%:%
%:%766=288%:%
%:%767=288%:%
%:%768=288%:%
%:%769=289%:%
%:%770=289%:%
%:%771=289%:%
%:%772=289%:%
%:%773=289%:%
%:%774=290%:%
%:%775=290%:%
%:%776=291%:%
%:%777=291%:%
%:%778=291%:%
%:%779=291%:%
%:%780=292%:%
%:%781=292%:%
%:%782=293%:%
%:%783=293%:%
%:%784=293%:%
%:%785=293%:%
%:%786=294%:%
%:%787=294%:%
%:%788=295%:%
%:%789=295%:%
%:%790=295%:%
%:%791=295%:%
%:%792=295%:%
%:%793=296%:%
%:%794=296%:%
%:%795=297%:%
%:%796=298%:%
%:%797=298%:%
%:%798=299%:%
%:%799=299%:%
%:%800=300%:%
%:%801=300%:%
%:%802=301%:%
%:%803=301%:%
%:%804=301%:%
%:%805=301%:%
%:%806=301%:%
%:%807=302%:%
%:%808=302%:%
%:%809=303%:%
%:%810=303%:%
%:%811=304%:%
%:%812=304%:%
%:%813=304%:%
%:%814=304%:%
%:%815=304%:%
%:%816=305%:%
%:%817=305%:%
%:%818=305%:%
%:%819=305%:%
%:%820=305%:%
%:%821=306%:%
%:%822=306%:%
%:%823=307%:%
%:%824=307%:%
%:%825=307%:%
%:%826=307%:%
%:%827=307%:%
%:%828=308%:%
%:%829=308%:%
%:%830=309%:%
%:%831=309%:%
%:%832=309%:%
%:%833=310%:%
%:%834=310%:%
%:%835=311%:%
%:%836=311%:%
%:%837=311%:%
%:%838=311%:%
%:%839=312%:%
%:%840=312%:%
%:%841=313%:%
%:%842=313%:%
%:%843=313%:%
%:%844=313%:%
%:%845=314%:%
%:%846=314%:%
%:%847=315%:%
%:%848=315%:%
%:%849=315%:%
%:%850=315%:%
%:%851=316%:%
%:%857=316%:%
%:%860=317%:%
%:%861=318%:%
%:%862=318%:%
%:%863=319%:%
%:%864=320%:%
%:%865=321%:%
%:%866=322%:%
%:%867=323%:%
%:%868=324%:%
%:%869=325%:%
%:%870=326%:%
%:%873=327%:%
%:%877=327%:%
%:%878=327%:%
%:%879=328%:%
%:%880=328%:%
%:%885=328%:%
%:%888=329%:%
%:%889=330%:%
%:%890=330%:%
%:%891=331%:%
%:%892=332%:%
%:%893=333%:%
%:%894=334%:%
%:%895=335%:%
%:%896=336%:%
%:%897=337%:%
%:%900=338%:%
%:%904=338%:%
%:%905=338%:%
%:%906=339%:%
%:%907=339%:%
%:%912=339%:%
%:%915=340%:%
%:%916=341%:%
%:%917=341%:%
%:%918=342%:%
%:%919=343%:%
%:%920=344%:%
%:%921=345%:%
%:%922=346%:%
%:%923=347%:%
%:%924=348%:%
%:%925=349%:%
%:%932=350%:%
%:%933=350%:%
%:%934=351%:%
%:%935=351%:%
%:%936=352%:%
%:%937=352%:%
%:%938=352%:%
%:%939=353%:%
%:%940=353%:%
%:%941=353%:%
%:%942=353%:%
%:%943=354%:%
%:%944=354%:%
%:%945=354%:%
%:%946=354%:%
%:%947=355%:%
%:%948=355%:%
%:%949=356%:%
%:%950=356%:%
%:%951=357%:%
%:%952=357%:%
%:%953=358%:%
%:%954=358%:%
%:%955=359%:%
%:%956=359%:%
%:%957=360%:%
%:%958=360%:%
%:%959=361%:%
%:%960=361%:%
%:%961=362%:%
%:%962=363%:%
%:%963=364%:%
%:%964=365%:%
%:%965=365%:%
%:%966=366%:%
%:%967=366%:%
%:%968=367%:%
%:%969=367%:%
%:%970=367%:%
%:%971=368%:%
%:%972=368%:%
%:%973=369%:%
%:%979=369%:%
%:%982=370%:%
%:%983=371%:%
%:%984=371%:%
%:%985=372%:%
%:%986=373%:%
%:%987=374%:%
%:%988=375%:%
%:%989=376%:%
%:%990=377%:%
%:%991=378%:%
%:%992=379%:%
%:%993=380%:%
%:%994=381%:%
%:%995=382%:%
%:%998=383%:%
%:%1002=383%:%
%:%1003=383%:%
%:%1004=384%:%
%:%1005=384%:%
%:%1010=384%:%
%:%1013=385%:%
%:%1014=386%:%
%:%1015=387%:%
%:%1016=387%:%
%:%1017=388%:%
%:%1018=389%:%
%:%1019=390%:%
%:%1020=391%:%
%:%1021=391%:%
%:%1024=392%:%
%:%1028=392%:%
%:%1029=392%:%
%:%1034=392%:%
%:%1037=393%:%
%:%1038=394%:%
%:%1039=394%:%
%:%1042=395%:%
%:%1046=395%:%
%:%1047=395%:%
%:%1048=396%:%
%:%1053=396%:%
%:%1056=397%:%
%:%1057=398%:%
%:%1058=398%:%
%:%1059=399%:%
%:%1062=400%:%
%:%1066=400%:%
%:%1067=400%:%
%:%1072=400%:%
%:%1075=401%:%
%:%1076=402%:%
%:%1077=402%:%
%:%1078=403%:%
%:%1081=404%:%
%:%1085=404%:%
%:%1086=404%:%
%:%1087=405%:%
%:%1101=407%:%
%:%1111=409%:%
%:%1112=409%:%
%:%1113=410%:%
%:%1114=410%:%
%:%1115=411%:%
%:%1116=412%:%
%:%1117=413%:%
%:%1118=414%:%
%:%1119=415%:%
%:%1120=416%:%
%:%1121=417%:%
%:%1122=418%:%
%:%1123=419%:%
%:%1124=420%:%
%:%1125=421%:%
%:%1126=422%:%
%:%1127=423%:%
%:%1128=423%:%
%:%1131=424%:%
%:%1135=424%:%
%:%1136=424%:%
%:%1141=424%:%
%:%1144=425%:%
%:%1145=425%:%
%:%1148=426%:%
%:%1152=426%:%
%:%1153=426%:%
%:%1158=426%:%
%:%1161=427%:%
%:%1162=427%:%
%:%1165=428%:%
%:%1169=428%:%
%:%1170=428%:%
%:%1175=428%:%
%:%1178=429%:%
%:%1179=429%:%
%:%1182=430%:%
%:%1186=430%:%
%:%1187=430%:%
%:%1192=430%:%
%:%1195=431%:%
%:%1196=431%:%
%:%1199=432%:%
%:%1203=432%:%
%:%1204=432%:%
%:%1209=432%:%
%:%1212=433%:%
%:%1213=433%:%
%:%1216=434%:%
%:%1220=434%:%
%:%1221=434%:%
%:%1226=434%:%
%:%1229=435%:%
%:%1230=436%:%
%:%1231=436%:%
%:%1234=437%:%
%:%1238=437%:%
%:%1239=437%:%
%:%1244=437%:%
%:%1247=438%:%
%:%1248=438%:%
%:%1251=439%:%
%:%1255=439%:%
%:%1256=439%:%
%:%1261=439%:%
%:%1264=440%:%
%:%1265=441%:%
%:%1266=442%:%
%:%1267=442%:%
%:%1268=443%:%
%:%1271=444%:%
%:%1275=444%:%
%:%1276=444%:%
%:%1281=444%:%
%:%1284=445%:%
%:%1285=446%:%
%:%1286=447%:%
%:%1287=447%:%
%:%1288=448%:%
%:%1289=449%:%
%:%1290=450%:%
%:%1293=451%:%
%:%1297=451%:%
%:%1298=451%:%
%:%1299=452%:%
%:%1300=452%:%
%:%1301=453%:%
%:%1302=453%:%
%:%1303=454%:%
%:%1304=454%:%
%:%1305=454%:%
%:%1306=455%:%
%:%1307=455%:%
%:%1308=455%:%
%:%1309=456%:%
%:%1310=456%:%
%:%1311=456%:%
%:%1312=456%:%
%:%1313=456%:%
%:%1314=457%:%
%:%1315=457%:%
%:%1316=458%:%
%:%1317=458%:%
%:%1318=459%:%
%:%1319=459%:%
%:%1320=459%:%
%:%1321=460%:%
%:%1322=460%:%
%:%1323=460%:%
%:%1324=461%:%
%:%1325=461%:%
%:%1326=461%:%
%:%1327=461%:%
%:%1328=461%:%
%:%1329=462%:%
%:%1330=462%:%
%:%1331=463%:%
%:%1332=463%:%
%:%1333=464%:%
%:%1334=464%:%
%:%1335=464%:%
%:%1336=465%:%
%:%1337=466%:%
%:%1338=466%:%
%:%1339=467%:%
%:%1340=467%:%
%:%1341=467%:%
%:%1342=468%:%
%:%1343=468%:%
%:%1344=469%:%
%:%1345=469%:%
%:%1346=470%:%
%:%1347=470%:%
%:%1348=470%:%
%:%1349=471%:%
%:%1350=471%:%
%:%1351=471%:%
%:%1352=472%:%
%:%1353=472%:%
%:%1354=472%:%
%:%1355=472%:%
%:%1356=472%:%
%:%1357=473%:%
%:%1358=473%:%
%:%1359=474%:%
%:%1360=474%:%
%:%1361=475%:%
%:%1362=475%:%
%:%1363=475%:%
%:%1364=475%:%
%:%1365=476%:%
%:%1366=476%:%
%:%1367=476%:%
%:%1368=476%:%
%:%1369=477%:%
%:%1370=477%:%
%:%1371=477%:%
%:%1372=477%:%
%:%1373=478%:%
%:%1374=478%:%
%:%1375=478%:%
%:%1376=478%:%
%:%1377=479%:%
%:%1378=479%:%
%:%1379=480%:%
%:%1380=480%:%
%:%1381=480%:%
%:%1382=480%:%
%:%1383=480%:%
%:%1384=481%:%
%:%1390=481%:%
%:%1393=482%:%
%:%1394=483%:%
%:%1395=483%:%
%:%1398=484%:%
%:%1402=484%:%
%:%1403=484%:%
%:%1408=484%:%
%:%1411=485%:%
%:%1412=486%:%
%:%1413=486%:%
%:%1414=487%:%
%:%1415=488%:%
%:%1416=489%:%
%:%1417=490%:%
%:%1418=491%:%
%:%1419=492%:%
%:%1426=493%:%
%:%1427=493%:%
%:%1428=494%:%
%:%1429=494%:%
%:%1430=495%:%
%:%1431=495%:%
%:%1432=495%:%
%:%1433=495%:%
%:%1434=496%:%
%:%1435=496%:%
%:%1436=496%:%
%:%1437=496%:%
%:%1438=496%:%
%:%1439=497%:%
%:%1440=497%:%
%:%1441=497%:%
%:%1442=498%:%
%:%1443=498%:%
%:%1444=499%:%
%:%1445=499%:%
%:%1446=500%:%
%:%1447=500%:%
%:%1448=500%:%
%:%1449=500%:%
%:%1450=500%:%
%:%1451=501%:%
%:%1452=501%:%
%:%1453=502%:%
%:%1454=502%:%
%:%1455=503%:%
%:%1456=503%:%
%:%1457=503%:%
%:%1458=503%:%
%:%1459=504%:%
%:%1460=504%:%
%:%1461=504%:%
%:%1462=504%:%
%:%1463=504%:%
%:%1464=505%:%
%:%1465=505%:%
%:%1466=505%:%
%:%1467=505%:%
%:%1468=506%:%
%:%1469=506%:%
%:%1470=506%:%
%:%1471=506%:%
%:%1472=506%:%
%:%1473=507%:%
%:%1474=507%:%
%:%1475=507%:%
%:%1476=507%:%
%:%1477=508%:%
%:%1478=508%:%
%:%1479=508%:%
%:%1480=508%:%
%:%1481=508%:%
%:%1482=509%:%
%:%1483=509%:%
%:%1484=509%:%
%:%1485=509%:%
%:%1486=509%:%
%:%1487=510%:%
%:%1488=510%:%
%:%1489=510%:%
%:%1490=511%:%
%:%1491=511%:%
%:%1492=511%:%
%:%1493=512%:%
%:%1494=512%:%
%:%1495=512%:%
%:%1496=512%:%
%:%1497=513%:%
%:%1498=513%:%
%:%1499=513%:%
%:%1500=513%:%
%:%1501=513%:%
%:%1502=514%:%
%:%1503=514%:%
%:%1504=515%:%
%:%1505=515%:%
%:%1506=515%:%
%:%1507=515%:%
%:%1508=516%:%
%:%1514=516%:%
%:%1517=517%:%
%:%1518=518%:%
%:%1519=518%:%
%:%1520=519%:%
%:%1521=520%:%
%:%1522=521%:%
%:%1525=524%:%
%:%1528=525%:%
%:%1532=525%:%
%:%1533=525%:%
%:%1534=526%:%
%:%1535=526%:%
%:%1536=527%:%
%:%1537=527%:%
%:%1538=528%:%
%:%1539=528%:%
%:%1540=528%:%
%:%1541=528%:%
%:%1542=529%:%
%:%1543=529%:%
%:%1544=530%:%
%:%1545=530%:%
%:%1546=530%:%
%:%1547=530%:%
%:%1548=531%:%
%:%1549=531%:%
%:%1550=532%:%
%:%1551=532%:%
%:%1552=532%:%
%:%1553=532%:%
%:%1554=533%:%
%:%1555=533%:%
%:%1556=534%:%
%:%1557=534%:%
%:%1558=534%:%
%:%1559=534%:%
%:%1560=535%:%
%:%1561=535%:%
%:%1562=536%:%
%:%1563=536%:%
%:%1564=537%:%
%:%1565=537%:%
%:%1566=537%:%
%:%1567=537%:%
%:%1568=538%:%
%:%1569=538%:%
%:%1570=539%:%
%:%1571=539%:%
%:%1572=539%:%
%:%1573=539%:%
%:%1574=540%:%
%:%1575=540%:%
%:%1576=541%:%
%:%1577=541%:%
%:%1578=541%:%
%:%1579=541%:%
%:%1580=542%:%
%:%1581=542%:%
%:%1582=542%:%
%:%1583=542%:%
%:%1584=542%:%
%:%1585=543%:%
%:%1586=543%:%
%:%1587=544%:%
%:%1588=544%:%
%:%1589=545%:%
%:%1590=545%:%
%:%1591=545%:%
%:%1592=545%:%
%:%1593=546%:%
%:%1594=546%:%
%:%1595=547%:%
%:%1596=547%:%
%:%1597=547%:%
%:%1598=547%:%
%:%1599=548%:%
%:%1600=548%:%
%:%1601=548%:%
%:%1602=549%:%
%:%1603=549%:%
%:%1604=549%:%
%:%1605=549%:%
%:%1606=550%:%
%:%1607=550%:%
%:%1608=550%:%
%:%1609=551%:%
%:%1610=551%:%
%:%1611=551%:%
%:%1612=551%:%
%:%1613=552%:%
%:%1614=552%:%
%:%1615=552%:%
%:%1616=553%:%
%:%1617=553%:%
%:%1618=553%:%
%:%1619=553%:%
%:%1620=554%:%
%:%1621=554%:%
%:%1622=554%:%
%:%1623=554%:%
%:%1624=555%:%
%:%1625=555%:%
%:%1626=555%:%
%:%1627=556%:%
%:%1628=556%:%
%:%1629=556%:%
%:%1630=556%:%
%:%1631=557%:%
%:%1632=557%:%
%:%1633=557%:%
%:%1634=558%:%
%:%1635=558%:%
%:%1636=558%:%
%:%1637=558%:%
%:%1638=559%:%
%:%1639=559%:%
%:%1640=559%:%
%:%1641=560%:%
%:%1642=560%:%
%:%1643=560%:%
%:%1644=560%:%
%:%1645=561%:%
%:%1646=561%:%
%:%1647=562%:%
%:%1648=562%:%
%:%1649=562%:%
%:%1650=562%:%
%:%1651=563%:%
%:%1652=563%:%
%:%1653=564%:%
%:%1654=564%:%
%:%1655=565%:%
%:%1656=565%:%
%:%1657=566%:%
%:%1658=566%:%
%:%1659=566%:%
%:%1660=567%:%
%:%1661=567%:%
%:%1662=567%:%
%:%1663=567%:%
%:%1664=568%:%
%:%1665=568%:%
%:%1666=568%:%
%:%1667=569%:%
%:%1668=569%:%
%:%1669=569%:%
%:%1670=570%:%
%:%1671=570%:%
%:%1672=570%:%
%:%1673=570%:%
%:%1674=571%:%
%:%1675=571%:%
%:%1676=571%:%
%:%1677=572%:%
%:%1678=572%:%
%:%1679=572%:%
%:%1680=573%:%
%:%1681=573%:%
%:%1682=574%:%
%:%1683=575%:%
%:%1684=575%:%
%:%1685=576%:%
%:%1686=576%:%
%:%1687=576%:%
%:%1688=576%:%
%:%1689=577%:%
%:%1690=577%:%
%:%1691=578%:%
%:%1692=578%:%
%:%1693=578%:%
%:%1694=579%:%
%:%1695=579%:%
%:%1696=579%:%
%:%1697=579%:%
%:%1698=580%:%
%:%1699=580%:%
%:%1700=581%:%
%:%1701=581%:%
%:%1702=581%:%
%:%1703=581%:%
%:%1704=581%:%
%:%1705=582%:%
%:%1711=582%:%
%:%1716=583%:%
%:%1721=584%:%

%
\begin{isabellebody}%
\setisabellecontext{Renaming{\isacharunderscore}{\kern0pt}Auto}%
%
\isadelimtheory
%
\endisadelimtheory
%
\isatagtheory
\isacommand{theory}\isamarkupfalse%
\ Renaming{\isacharunderscore}{\kern0pt}Auto\isanewline
\ \ \isakeyword{imports}\isanewline
\ \ \ \ Renaming\isanewline
\ \ \ \ Utils\isanewline
\ \ \ \ ZF{\isachardot}{\kern0pt}Finite\isanewline
\ \ \ \ ZF{\isachardot}{\kern0pt}List\isanewline
\ \ \isakeyword{keywords}\ {\isachardoublequoteopen}rename{\isachardoublequoteclose}\ {\isacharcolon}{\kern0pt}{\isacharcolon}{\kern0pt}\ thy{\isacharunderscore}{\kern0pt}decl\ {\isacharpercent}{\kern0pt}\ {\isachardoublequoteopen}ML{\isachardoublequoteclose}\isanewline
\ \ \ \ \isakeyword{and}\ {\isachardoublequoteopen}simple{\isacharunderscore}{\kern0pt}rename{\isachardoublequoteclose}\ {\isacharcolon}{\kern0pt}{\isacharcolon}{\kern0pt}\ thy{\isacharunderscore}{\kern0pt}decl\ {\isacharpercent}{\kern0pt}\ {\isachardoublequoteopen}ML{\isachardoublequoteclose}\isanewline
\ \ \ \ \isakeyword{and}\ {\isachardoublequoteopen}src{\isachardoublequoteclose}\isanewline
\ \ \ \ \isakeyword{and}\ {\isachardoublequoteopen}tgt{\isachardoublequoteclose}\isanewline
\ \ \isakeyword{abbrevs}\ {\isachardoublequoteopen}simple{\isacharunderscore}{\kern0pt}rename{\isachardoublequoteclose}\ {\isacharequal}{\kern0pt}\ {\isachardoublequoteopen}{\isachardoublequoteclose}\isanewline
\isakeyword{begin}%
\endisatagtheory
{\isafoldtheory}%
%
\isadelimtheory
\isanewline
%
\endisadelimtheory
\isanewline
\isacommand{lemmas}\isamarkupfalse%
\ app{\isacharunderscore}{\kern0pt}fun\ {\isacharequal}{\kern0pt}\ apply{\isacharunderscore}{\kern0pt}iff{\isacharbrackleft}{\kern0pt}THEN\ iffD{\isadigit{1}}{\isacharbrackright}{\kern0pt}\isanewline
\isacommand{lemmas}\isamarkupfalse%
\ nat{\isacharunderscore}{\kern0pt}succI\ {\isacharequal}{\kern0pt}\ nat{\isacharunderscore}{\kern0pt}succ{\isacharunderscore}{\kern0pt}iff{\isacharbrackleft}{\kern0pt}THEN\ iffD{\isadigit{2}}{\isacharbrackright}{\kern0pt}\isanewline
%
\isadelimML
\isanewline
%
\endisadelimML
%
\isatagML
\isacommand{ML{\isacharunderscore}{\kern0pt}file}\isamarkupfalse%
\ {\isacartoucheopen}renaming{\isachardot}{\kern0pt}ML{\isacartoucheclose}\isanewline
\isacommand{ML}\isamarkupfalse%
{\isacartoucheopen}\isanewline
\ \ fun\ renaming{\isacharunderscore}{\kern0pt}def\ mk{\isacharunderscore}{\kern0pt}ren\ name\ from\ to\ ctxt\ {\isacharequal}{\kern0pt}\isanewline
\ \ \ \ let\ val\ to\ {\isacharequal}{\kern0pt}\ to\ {\isacharbar}{\kern0pt}{\isachargreater}{\kern0pt}\ Syntax{\isachardot}{\kern0pt}read{\isacharunderscore}{\kern0pt}term\ ctxt\isanewline
\ \ \ \ \ \ \ \ val\ from\ {\isacharequal}{\kern0pt}\ from\ {\isacharbar}{\kern0pt}{\isachargreater}{\kern0pt}\ Syntax{\isachardot}{\kern0pt}read{\isacharunderscore}{\kern0pt}term\ ctxt\isanewline
\ \ \ \ \ \ \ \ val\ {\isacharparenleft}{\kern0pt}tc{\isacharunderscore}{\kern0pt}lemma{\isacharcomma}{\kern0pt}action{\isacharunderscore}{\kern0pt}lemma{\isacharcomma}{\kern0pt}fvs{\isacharcomma}{\kern0pt}r{\isacharparenright}{\kern0pt}\ {\isacharequal}{\kern0pt}\ mk{\isacharunderscore}{\kern0pt}ren\ from\ to\ ctxt\isanewline
\ \ \ \ \ \ \ \ val\ {\isacharparenleft}{\kern0pt}tc{\isacharunderscore}{\kern0pt}lemma{\isacharcomma}{\kern0pt}action{\isacharunderscore}{\kern0pt}lemma{\isacharparenright}{\kern0pt}\ {\isacharequal}{\kern0pt}\isanewline
\ \ \ \ \ \ \ \ \ \ {\isacharparenleft}{\kern0pt}Renaming{\isachardot}{\kern0pt}fix{\isacharunderscore}{\kern0pt}vars\ tc{\isacharunderscore}{\kern0pt}lemma\ fvs\ ctxt{\isacharcomma}{\kern0pt}\ Renaming{\isachardot}{\kern0pt}fix{\isacharunderscore}{\kern0pt}vars\ action{\isacharunderscore}{\kern0pt}lemma\ fvs\ ctxt{\isacharparenright}{\kern0pt}\isanewline
\ \ \ \ \ \ \ \ val\ ren{\isacharunderscore}{\kern0pt}fun{\isacharunderscore}{\kern0pt}name\ {\isacharequal}{\kern0pt}\ Binding{\isachardot}{\kern0pt}name\ {\isacharparenleft}{\kern0pt}name\ {\isacharcircum}{\kern0pt}\ {\isachardoublequote}{\kern0pt}{\isacharunderscore}{\kern0pt}fn{\isachardoublequote}{\kern0pt}{\isacharparenright}{\kern0pt}\isanewline
\ \ \ \ \ \ \ \ val\ ren{\isacharunderscore}{\kern0pt}fun{\isacharunderscore}{\kern0pt}def\ {\isacharequal}{\kern0pt}\ \ Binding{\isachardot}{\kern0pt}name\ {\isacharparenleft}{\kern0pt}name\ {\isacharcircum}{\kern0pt}\ {\isachardoublequote}{\kern0pt}{\isacharunderscore}{\kern0pt}fn{\isacharunderscore}{\kern0pt}def{\isachardoublequote}{\kern0pt}{\isacharparenright}{\kern0pt}\isanewline
\ \ \ \ \ \ \ \ val\ ren{\isacharunderscore}{\kern0pt}thm\ {\isacharequal}{\kern0pt}\ Binding{\isachardot}{\kern0pt}name\ {\isacharparenleft}{\kern0pt}name\ {\isacharcircum}{\kern0pt}\ {\isachardoublequote}{\kern0pt}{\isacharunderscore}{\kern0pt}thm{\isachardoublequote}{\kern0pt}{\isacharparenright}{\kern0pt}\isanewline
\ \ \ \ in\isanewline
\ \ \ \ \ \ Local{\isacharunderscore}{\kern0pt}Theory{\isachardot}{\kern0pt}note\ \ \ {\isacharparenleft}{\kern0pt}{\isacharparenleft}{\kern0pt}ren{\isacharunderscore}{\kern0pt}thm{\isacharcomma}{\kern0pt}\ {\isacharbrackleft}{\kern0pt}{\isacharbrackright}{\kern0pt}{\isacharparenright}{\kern0pt}{\isacharcomma}{\kern0pt}\ {\isacharbrackleft}{\kern0pt}tc{\isacharunderscore}{\kern0pt}lemma{\isacharcomma}{\kern0pt}action{\isacharunderscore}{\kern0pt}lemma{\isacharbrackright}{\kern0pt}{\isacharparenright}{\kern0pt}\ ctxt\ {\isacharbar}{\kern0pt}{\isachargreater}{\kern0pt}\ snd\ {\isacharbar}{\kern0pt}{\isachargreater}{\kern0pt}\isanewline
\ \ \ \ \ \ Local{\isacharunderscore}{\kern0pt}Theory{\isachardot}{\kern0pt}define\ {\isacharparenleft}{\kern0pt}{\isacharparenleft}{\kern0pt}ren{\isacharunderscore}{\kern0pt}fun{\isacharunderscore}{\kern0pt}name{\isacharcomma}{\kern0pt}\ NoSyn{\isacharparenright}{\kern0pt}{\isacharcomma}{\kern0pt}\ {\isacharparenleft}{\kern0pt}{\isacharparenleft}{\kern0pt}ren{\isacharunderscore}{\kern0pt}fun{\isacharunderscore}{\kern0pt}def{\isacharcomma}{\kern0pt}\ {\isacharbrackleft}{\kern0pt}{\isacharbrackright}{\kern0pt}{\isacharparenright}{\kern0pt}{\isacharcomma}{\kern0pt}\ r{\isacharparenright}{\kern0pt}{\isacharparenright}{\kern0pt}\ {\isacharbar}{\kern0pt}{\isachargreater}{\kern0pt}\ snd\ \ \ \ \ \ \isanewline
\ \ end{\isacharsemicolon}{\kern0pt}\isanewline
{\isacartoucheclose}\isanewline
\isanewline
\isacommand{ML}\isamarkupfalse%
{\isacartoucheopen}\isanewline
local\isanewline
\isanewline
\ \ val\ ren{\isacharunderscore}{\kern0pt}parser\ {\isacharequal}{\kern0pt}\ Parse{\isachardot}{\kern0pt}position\ {\isacharparenleft}{\kern0pt}Parse{\isachardot}{\kern0pt}string\ {\isacharminus}{\kern0pt}{\isacharminus}{\kern0pt}\isanewline
\ \ \ \ \ \ {\isacharparenleft}{\kern0pt}Parse{\isachardot}{\kern0pt}{\isachardollar}{\kern0pt}{\isachardollar}{\kern0pt}{\isachardollar}{\kern0pt}\ {\isachardoublequote}{\kern0pt}src{\isachardoublequote}{\kern0pt}\ {\isacharbar}{\kern0pt}{\isacharminus}{\kern0pt}{\isacharminus}{\kern0pt}\ Parse{\isachardot}{\kern0pt}string\ {\isacharminus}{\kern0pt}{\isacharminus}{\kern0pt}{\isacharbar}{\kern0pt}\ Parse{\isachardot}{\kern0pt}{\isachardollar}{\kern0pt}{\isachardollar}{\kern0pt}{\isachardollar}{\kern0pt}\ {\isachardoublequote}{\kern0pt}tgt{\isachardoublequote}{\kern0pt}\ {\isacharminus}{\kern0pt}{\isacharminus}{\kern0pt}\ Parse{\isachardot}{\kern0pt}string{\isacharparenright}{\kern0pt}{\isacharparenright}{\kern0pt}{\isacharsemicolon}{\kern0pt}\isanewline
\isanewline
\ \ val\ {\isacharunderscore}{\kern0pt}\ {\isacharequal}{\kern0pt}\isanewline
\ \ \ Outer{\isacharunderscore}{\kern0pt}Syntax{\isachardot}{\kern0pt}local{\isacharunderscore}{\kern0pt}theory\ \isactrlcommandUNDERSCOREkeyword {\isasymopen}rename{\isasymclose}\ {\isachardoublequote}{\kern0pt}ML\ setup\ for\ synthetic\ definitions{\isachardoublequote}{\kern0pt}\isanewline
\ \ \ \ \ {\isacharparenleft}{\kern0pt}ren{\isacharunderscore}{\kern0pt}parser\ {\isachargreater}{\kern0pt}{\isachargreater}{\kern0pt}\ {\isacharparenleft}{\kern0pt}fn\ {\isacharparenleft}{\kern0pt}{\isacharparenleft}{\kern0pt}name{\isacharcomma}{\kern0pt}{\isacharparenleft}{\kern0pt}from{\isacharcomma}{\kern0pt}to{\isacharparenright}{\kern0pt}{\isacharparenright}{\kern0pt}{\isacharcomma}{\kern0pt}{\isacharunderscore}{\kern0pt}{\isacharparenright}{\kern0pt}\ {\isacharequal}{\kern0pt}{\isachargreater}{\kern0pt}\ renaming{\isacharunderscore}{\kern0pt}def\ Renaming{\isachardot}{\kern0pt}sum{\isacharunderscore}{\kern0pt}rename\ name\ from\ to\ {\isacharparenright}{\kern0pt}{\isacharparenright}{\kern0pt}\isanewline
\isanewline
\ \ val\ {\isacharunderscore}{\kern0pt}\ {\isacharequal}{\kern0pt}\isanewline
\ \ \ Outer{\isacharunderscore}{\kern0pt}Syntax{\isachardot}{\kern0pt}local{\isacharunderscore}{\kern0pt}theory\ \isactrlcommandUNDERSCOREkeyword {\isasymopen}simple{\isacharunderscore}{\kern0pt}rename{\isasymclose}\ {\isachardoublequote}{\kern0pt}ML\ setup\ for\ synthetic\ definitions{\isachardoublequote}{\kern0pt}\isanewline
\ \ \ \ \ {\isacharparenleft}{\kern0pt}ren{\isacharunderscore}{\kern0pt}parser\ {\isachargreater}{\kern0pt}{\isachargreater}{\kern0pt}\ {\isacharparenleft}{\kern0pt}fn\ {\isacharparenleft}{\kern0pt}{\isacharparenleft}{\kern0pt}name{\isacharcomma}{\kern0pt}{\isacharparenleft}{\kern0pt}from{\isacharcomma}{\kern0pt}to{\isacharparenright}{\kern0pt}{\isacharparenright}{\kern0pt}{\isacharcomma}{\kern0pt}{\isacharunderscore}{\kern0pt}{\isacharparenright}{\kern0pt}\ {\isacharequal}{\kern0pt}{\isachargreater}{\kern0pt}\ renaming{\isacharunderscore}{\kern0pt}def\ Renaming{\isachardot}{\kern0pt}ren{\isacharunderscore}{\kern0pt}thm\ name\ from\ to\ {\isacharparenright}{\kern0pt}{\isacharparenright}{\kern0pt}\isanewline
\isanewline
in\isanewline
end\isanewline
{\isacartoucheclose}%
\endisatagML
{\isafoldML}%
%
\isadelimML
\isanewline
%
\endisadelimML
%
\isadelimtheory
%
\endisadelimtheory
%
\isatagtheory
\isacommand{end}\isamarkupfalse%
%
\endisatagtheory
{\isafoldtheory}%
%
\isadelimtheory
%
\endisadelimtheory
%
\end{isabellebody}%
\endinput
%:%file=~/source/repos/ZF-notAC/code/Forcing/Renaming_Auto.thy%:%
%:%10=1%:%
%:%11=1%:%
%:%12=2%:%
%:%13=3%:%
%:%14=4%:%
%:%15=5%:%
%:%16=6%:%
%:%17=7%:%
%:%18=8%:%
%:%19=9%:%
%:%20=10%:%
%:%21=11%:%
%:%22=12%:%
%:%27=12%:%
%:%30=13%:%
%:%31=14%:%
%:%32=14%:%
%:%33=15%:%
%:%34=15%:%
%:%37=16%:%
%:%42=17%:%
%:%43=17%:%
%:%44=18%:%
%:%45=18%:%
%:%59=32%:%
%:%60=33%:%
%:%61=34%:%
%:%62=34%:%
%:%83=50%:%
%:%92=51%:%

%
\begin{isabellebody}%
\setisabellecontext{Names}%
%
\isadelimdocument
%
\endisadelimdocument
%
\isatagdocument
%
\isamarkupsection{Names and generic extensions%
}
\isamarkuptrue%
%
\endisatagdocument
{\isafolddocument}%
%
\isadelimdocument
%
\endisadelimdocument
%
\isadelimtheory
%
\endisadelimtheory
%
\isatagtheory
\isacommand{theory}\isamarkupfalse%
\ Names\isanewline
\ \ \isakeyword{imports}\isanewline
\ \ \ \ Forcing{\isacharunderscore}{\kern0pt}Data\isanewline
\ \ \ \ Interface\isanewline
\ \ \ \ Recursion{\isacharunderscore}{\kern0pt}Thms\isanewline
\ \ \ \ Synthetic{\isacharunderscore}{\kern0pt}Definition\isanewline
\isakeyword{begin}%
\endisatagtheory
{\isafoldtheory}%
%
\isadelimtheory
\isanewline
%
\endisadelimtheory
\isanewline
\isacommand{definition}\isamarkupfalse%
\isanewline
\ \ SepReplace\ {\isacharcolon}{\kern0pt}{\isacharcolon}{\kern0pt}\ {\isachardoublequoteopen}{\isacharbrackleft}{\kern0pt}i{\isacharcomma}{\kern0pt}\ i{\isasymRightarrow}i{\isacharcomma}{\kern0pt}\ i{\isasymRightarrow}\ o{\isacharbrackright}{\kern0pt}\ {\isasymRightarrow}\ i{\isachardoublequoteclose}\ \isakeyword{where}\isanewline
\ \ {\isachardoublequoteopen}SepReplace{\isacharparenleft}{\kern0pt}A{\isacharcomma}{\kern0pt}b{\isacharcomma}{\kern0pt}Q{\isacharparenright}{\kern0pt}\ {\isasymequiv}\ {\isacharbraceleft}{\kern0pt}y\ {\isachardot}{\kern0pt}\ x{\isasymin}A{\isacharcomma}{\kern0pt}\ y{\isacharequal}{\kern0pt}b{\isacharparenleft}{\kern0pt}x{\isacharparenright}{\kern0pt}\ {\isasymand}\ Q{\isacharparenleft}{\kern0pt}x{\isacharparenright}{\kern0pt}{\isacharbraceright}{\kern0pt}{\isachardoublequoteclose}\isanewline
\isanewline
\isacommand{syntax}\isamarkupfalse%
\isanewline
\ \ {\isachardoublequoteopen}{\isacharunderscore}{\kern0pt}SepReplace{\isachardoublequoteclose}\ \ {\isacharcolon}{\kern0pt}{\isacharcolon}{\kern0pt}\ {\isachardoublequoteopen}{\isacharbrackleft}{\kern0pt}i{\isacharcomma}{\kern0pt}\ pttrn{\isacharcomma}{\kern0pt}\ i{\isacharcomma}{\kern0pt}\ o{\isacharbrackright}{\kern0pt}\ {\isasymRightarrow}\ i{\isachardoublequoteclose}\ \ {\isacharparenleft}{\kern0pt}{\isachardoublequoteopen}{\isacharparenleft}{\kern0pt}{\isadigit{1}}{\isacharbraceleft}{\kern0pt}{\isacharunderscore}{\kern0pt}\ {\isachardot}{\kern0pt}{\isachardot}{\kern0pt}{\isacharslash}{\kern0pt}\ {\isacharunderscore}{\kern0pt}\ {\isasymin}\ {\isacharunderscore}{\kern0pt}{\isacharcomma}{\kern0pt}\ {\isacharunderscore}{\kern0pt}{\isacharbraceright}{\kern0pt}{\isacharparenright}{\kern0pt}{\isachardoublequoteclose}{\isacharparenright}{\kern0pt}\isanewline
\isacommand{translations}\isamarkupfalse%
\isanewline
\ \ {\isachardoublequoteopen}{\isacharbraceleft}{\kern0pt}b\ {\isachardot}{\kern0pt}{\isachardot}{\kern0pt}\ x{\isasymin}A{\isacharcomma}{\kern0pt}\ Q{\isacharbraceright}{\kern0pt}{\isachardoublequoteclose}\ {\isacharequal}{\kern0pt}{\isachargreater}{\kern0pt}\ {\isachardoublequoteopen}CONST\ SepReplace{\isacharparenleft}{\kern0pt}A{\isacharcomma}{\kern0pt}\ {\isasymlambda}x{\isachardot}{\kern0pt}\ b{\isacharcomma}{\kern0pt}\ {\isasymlambda}x{\isachardot}{\kern0pt}\ Q{\isacharparenright}{\kern0pt}{\isachardoublequoteclose}\isanewline
\isanewline
\isacommand{lemma}\isamarkupfalse%
\ Sep{\isacharunderscore}{\kern0pt}and{\isacharunderscore}{\kern0pt}Replace{\isacharcolon}{\kern0pt}\ {\isachardoublequoteopen}{\isacharbraceleft}{\kern0pt}b{\isacharparenleft}{\kern0pt}x{\isacharparenright}{\kern0pt}\ {\isachardot}{\kern0pt}{\isachardot}{\kern0pt}\ x{\isasymin}A{\isacharcomma}{\kern0pt}\ P{\isacharparenleft}{\kern0pt}x{\isacharparenright}{\kern0pt}\ {\isacharbraceright}{\kern0pt}\ {\isacharequal}{\kern0pt}\ {\isacharbraceleft}{\kern0pt}b{\isacharparenleft}{\kern0pt}x{\isacharparenright}{\kern0pt}\ {\isachardot}{\kern0pt}\ x{\isasymin}{\isacharbraceleft}{\kern0pt}y{\isasymin}A{\isachardot}{\kern0pt}\ P{\isacharparenleft}{\kern0pt}y{\isacharparenright}{\kern0pt}{\isacharbraceright}{\kern0pt}{\isacharbraceright}{\kern0pt}{\isachardoublequoteclose}\isanewline
%
\isadelimproof
\ \ %
\endisadelimproof
%
\isatagproof
\isacommand{by}\isamarkupfalse%
\ {\isacharparenleft}{\kern0pt}auto\ simp\ add{\isacharcolon}{\kern0pt}SepReplace{\isacharunderscore}{\kern0pt}def{\isacharparenright}{\kern0pt}%
\endisatagproof
{\isafoldproof}%
%
\isadelimproof
\isanewline
%
\endisadelimproof
\isanewline
\isacommand{lemma}\isamarkupfalse%
\ SepReplace{\isacharunderscore}{\kern0pt}subset\ {\isacharcolon}{\kern0pt}\ {\isachardoublequoteopen}A{\isasymsubseteq}A{\isacharprime}{\kern0pt}{\isasymLongrightarrow}\ {\isacharbraceleft}{\kern0pt}b\ {\isachardot}{\kern0pt}{\isachardot}{\kern0pt}\ x{\isasymin}A{\isacharcomma}{\kern0pt}\ Q{\isacharbraceright}{\kern0pt}{\isasymsubseteq}{\isacharbraceleft}{\kern0pt}b\ {\isachardot}{\kern0pt}{\isachardot}{\kern0pt}\ x{\isasymin}A{\isacharprime}{\kern0pt}{\isacharcomma}{\kern0pt}\ Q{\isacharbraceright}{\kern0pt}{\isachardoublequoteclose}\isanewline
%
\isadelimproof
\ \ %
\endisadelimproof
%
\isatagproof
\isacommand{by}\isamarkupfalse%
\ {\isacharparenleft}{\kern0pt}auto\ simp\ add{\isacharcolon}{\kern0pt}SepReplace{\isacharunderscore}{\kern0pt}def{\isacharparenright}{\kern0pt}%
\endisatagproof
{\isafoldproof}%
%
\isadelimproof
\isanewline
%
\endisadelimproof
\isanewline
\isacommand{lemma}\isamarkupfalse%
\ SepReplace{\isacharunderscore}{\kern0pt}iff\ {\isacharbrackleft}{\kern0pt}simp{\isacharbrackright}{\kern0pt}{\isacharcolon}{\kern0pt}\ {\isachardoublequoteopen}y{\isasymin}{\isacharbraceleft}{\kern0pt}b{\isacharparenleft}{\kern0pt}x{\isacharparenright}{\kern0pt}\ {\isachardot}{\kern0pt}{\isachardot}{\kern0pt}\ x{\isasymin}A{\isacharcomma}{\kern0pt}\ P{\isacharparenleft}{\kern0pt}x{\isacharparenright}{\kern0pt}{\isacharbraceright}{\kern0pt}\ {\isasymlongleftrightarrow}\ {\isacharparenleft}{\kern0pt}{\isasymexists}x{\isasymin}A{\isachardot}{\kern0pt}\ y{\isacharequal}{\kern0pt}b{\isacharparenleft}{\kern0pt}x{\isacharparenright}{\kern0pt}\ {\isacharampersand}{\kern0pt}\ P{\isacharparenleft}{\kern0pt}x{\isacharparenright}{\kern0pt}{\isacharparenright}{\kern0pt}{\isachardoublequoteclose}\isanewline
%
\isadelimproof
\ \ %
\endisadelimproof
%
\isatagproof
\isacommand{by}\isamarkupfalse%
\ {\isacharparenleft}{\kern0pt}auto\ simp\ add{\isacharcolon}{\kern0pt}SepReplace{\isacharunderscore}{\kern0pt}def{\isacharparenright}{\kern0pt}%
\endisatagproof
{\isafoldproof}%
%
\isadelimproof
\isanewline
%
\endisadelimproof
\isanewline
\isacommand{lemma}\isamarkupfalse%
\ SepReplace{\isacharunderscore}{\kern0pt}dom{\isacharunderscore}{\kern0pt}implies\ {\isacharcolon}{\kern0pt}\isanewline
\ \ {\isachardoublequoteopen}{\isacharparenleft}{\kern0pt}{\isasymAnd}\ x\ {\isachardot}{\kern0pt}\ x\ {\isasymin}A\ {\isasymLongrightarrow}\ b{\isacharparenleft}{\kern0pt}x{\isacharparenright}{\kern0pt}\ {\isacharequal}{\kern0pt}\ b{\isacharprime}{\kern0pt}{\isacharparenleft}{\kern0pt}x{\isacharparenright}{\kern0pt}{\isacharparenright}{\kern0pt}{\isasymLongrightarrow}\ {\isacharbraceleft}{\kern0pt}b{\isacharparenleft}{\kern0pt}x{\isacharparenright}{\kern0pt}\ {\isachardot}{\kern0pt}{\isachardot}{\kern0pt}\ x{\isasymin}A{\isacharcomma}{\kern0pt}\ Q{\isacharparenleft}{\kern0pt}x{\isacharparenright}{\kern0pt}{\isacharbraceright}{\kern0pt}{\isacharequal}{\kern0pt}{\isacharbraceleft}{\kern0pt}b{\isacharprime}{\kern0pt}{\isacharparenleft}{\kern0pt}x{\isacharparenright}{\kern0pt}\ {\isachardot}{\kern0pt}{\isachardot}{\kern0pt}\ x{\isasymin}A{\isacharcomma}{\kern0pt}\ Q{\isacharparenleft}{\kern0pt}x{\isacharparenright}{\kern0pt}{\isacharbraceright}{\kern0pt}{\isachardoublequoteclose}\isanewline
%
\isadelimproof
\ \ %
\endisadelimproof
%
\isatagproof
\isacommand{by}\isamarkupfalse%
\ \ {\isacharparenleft}{\kern0pt}simp\ add{\isacharcolon}{\kern0pt}SepReplace{\isacharunderscore}{\kern0pt}def{\isacharparenright}{\kern0pt}%
\endisatagproof
{\isafoldproof}%
%
\isadelimproof
\isanewline
%
\endisadelimproof
\isanewline
\isacommand{lemma}\isamarkupfalse%
\ SepReplace{\isacharunderscore}{\kern0pt}pred{\isacharunderscore}{\kern0pt}implies\ {\isacharcolon}{\kern0pt}\isanewline
\ \ {\isachardoublequoteopen}{\isasymforall}x{\isachardot}{\kern0pt}\ Q{\isacharparenleft}{\kern0pt}x{\isacharparenright}{\kern0pt}{\isasymlongrightarrow}\ b{\isacharparenleft}{\kern0pt}x{\isacharparenright}{\kern0pt}\ {\isacharequal}{\kern0pt}\ b{\isacharprime}{\kern0pt}{\isacharparenleft}{\kern0pt}x{\isacharparenright}{\kern0pt}{\isasymLongrightarrow}\ {\isacharbraceleft}{\kern0pt}b{\isacharparenleft}{\kern0pt}x{\isacharparenright}{\kern0pt}\ {\isachardot}{\kern0pt}{\isachardot}{\kern0pt}\ x{\isasymin}A{\isacharcomma}{\kern0pt}\ Q{\isacharparenleft}{\kern0pt}x{\isacharparenright}{\kern0pt}{\isacharbraceright}{\kern0pt}{\isacharequal}{\kern0pt}{\isacharbraceleft}{\kern0pt}b{\isacharprime}{\kern0pt}{\isacharparenleft}{\kern0pt}x{\isacharparenright}{\kern0pt}\ {\isachardot}{\kern0pt}{\isachardot}{\kern0pt}\ x{\isasymin}A{\isacharcomma}{\kern0pt}\ Q{\isacharparenleft}{\kern0pt}x{\isacharparenright}{\kern0pt}{\isacharbraceright}{\kern0pt}{\isachardoublequoteclose}\isanewline
%
\isadelimproof
\ \ %
\endisadelimproof
%
\isatagproof
\isacommand{by}\isamarkupfalse%
\ \ {\isacharparenleft}{\kern0pt}force\ simp\ add{\isacharcolon}{\kern0pt}SepReplace{\isacharunderscore}{\kern0pt}def{\isacharparenright}{\kern0pt}%
\endisatagproof
{\isafoldproof}%
%
\isadelimproof
%
\endisadelimproof
%
\isadelimdocument
%
\endisadelimdocument
%
\isatagdocument
%
\isamarkupsubsection{The well-founded relation \isa{ed}%
}
\isamarkuptrue%
%
\endisatagdocument
{\isafolddocument}%
%
\isadelimdocument
%
\endisadelimdocument
\isacommand{lemma}\isamarkupfalse%
\ eclose{\isacharunderscore}{\kern0pt}sing\ {\isacharcolon}{\kern0pt}\ {\isachardoublequoteopen}x\ {\isasymin}\ eclose{\isacharparenleft}{\kern0pt}a{\isacharparenright}{\kern0pt}\ {\isasymLongrightarrow}\ x\ {\isasymin}\ eclose{\isacharparenleft}{\kern0pt}{\isacharbraceleft}{\kern0pt}a{\isacharbraceright}{\kern0pt}{\isacharparenright}{\kern0pt}{\isachardoublequoteclose}\isanewline
%
\isadelimproof
\ \ %
\endisadelimproof
%
\isatagproof
\isacommand{by}\isamarkupfalse%
{\isacharparenleft}{\kern0pt}rule\ subsetD{\isacharbrackleft}{\kern0pt}OF\ mem{\isacharunderscore}{\kern0pt}eclose{\isacharunderscore}{\kern0pt}subset{\isacharbrackright}{\kern0pt}{\isacharcomma}{\kern0pt}simp{\isacharplus}{\kern0pt}{\isacharparenright}{\kern0pt}%
\endisatagproof
{\isafoldproof}%
%
\isadelimproof
\isanewline
%
\endisadelimproof
\isanewline
\isacommand{lemma}\isamarkupfalse%
\ ecloseE\ {\isacharcolon}{\kern0pt}\isanewline
\ \ \isakeyword{assumes}\ \ {\isachardoublequoteopen}x\ {\isasymin}\ eclose{\isacharparenleft}{\kern0pt}A{\isacharparenright}{\kern0pt}{\isachardoublequoteclose}\isanewline
\ \ \isakeyword{shows}\ \ {\isachardoublequoteopen}x\ {\isasymin}\ A\ {\isasymor}\ {\isacharparenleft}{\kern0pt}{\isasymexists}\ B\ {\isasymin}\ A\ {\isachardot}{\kern0pt}\ x\ {\isasymin}\ eclose{\isacharparenleft}{\kern0pt}B{\isacharparenright}{\kern0pt}{\isacharparenright}{\kern0pt}{\isachardoublequoteclose}\isanewline
%
\isadelimproof
\ \ %
\endisadelimproof
%
\isatagproof
\isacommand{using}\isamarkupfalse%
\ assms\isanewline
\isacommand{proof}\isamarkupfalse%
\ {\isacharparenleft}{\kern0pt}induct\ rule{\isacharcolon}{\kern0pt}eclose{\isacharunderscore}{\kern0pt}induct{\isacharunderscore}{\kern0pt}down{\isacharparenright}{\kern0pt}\isanewline
\ \ \isacommand{case}\isamarkupfalse%
\ {\isacharparenleft}{\kern0pt}{\isadigit{1}}\ y{\isacharparenright}{\kern0pt}\isanewline
\ \ \isacommand{then}\isamarkupfalse%
\isanewline
\ \ \isacommand{show}\isamarkupfalse%
\ {\isacharquery}{\kern0pt}case\isanewline
\ \ \ \ \isacommand{using}\isamarkupfalse%
\ arg{\isacharunderscore}{\kern0pt}into{\isacharunderscore}{\kern0pt}eclose\ \isacommand{by}\isamarkupfalse%
\ auto\isanewline
\isacommand{next}\isamarkupfalse%
\isanewline
\ \ \isacommand{case}\isamarkupfalse%
\ {\isacharparenleft}{\kern0pt}{\isadigit{2}}\ y\ z{\isacharparenright}{\kern0pt}\isanewline
\ \ \isacommand{from}\isamarkupfalse%
\ {\isacartoucheopen}y\ {\isasymin}\ A\ {\isasymor}\ {\isacharparenleft}{\kern0pt}{\isasymexists}B{\isasymin}A{\isachardot}{\kern0pt}\ y\ {\isasymin}\ eclose{\isacharparenleft}{\kern0pt}B{\isacharparenright}{\kern0pt}{\isacharparenright}{\kern0pt}{\isacartoucheclose}\isanewline
\ \ \isacommand{consider}\isamarkupfalse%
\ {\isacharparenleft}{\kern0pt}inA{\isacharparenright}{\kern0pt}\ {\isachardoublequoteopen}y\ {\isasymin}\ A{\isachardoublequoteclose}\ {\isacharbar}{\kern0pt}\ {\isacharparenleft}{\kern0pt}exB{\isacharparenright}{\kern0pt}\ {\isachardoublequoteopen}{\isacharparenleft}{\kern0pt}{\isasymexists}B{\isasymin}A{\isachardot}{\kern0pt}\ y\ {\isasymin}\ eclose{\isacharparenleft}{\kern0pt}B{\isacharparenright}{\kern0pt}{\isacharparenright}{\kern0pt}{\isachardoublequoteclose}\isanewline
\ \ \ \ \isacommand{by}\isamarkupfalse%
\ auto\isanewline
\ \ \isacommand{then}\isamarkupfalse%
\ \isacommand{show}\isamarkupfalse%
\ {\isacharquery}{\kern0pt}case\isanewline
\ \ \isacommand{proof}\isamarkupfalse%
\ {\isacharparenleft}{\kern0pt}cases{\isacharparenright}{\kern0pt}\isanewline
\ \ \ \ \isacommand{case}\isamarkupfalse%
\ inA\isanewline
\ \ \ \ \isacommand{then}\isamarkupfalse%
\isanewline
\ \ \ \ \isacommand{show}\isamarkupfalse%
\ {\isacharquery}{\kern0pt}thesis\ \isacommand{using}\isamarkupfalse%
\ {\isadigit{2}}\ arg{\isacharunderscore}{\kern0pt}into{\isacharunderscore}{\kern0pt}eclose\ \isacommand{by}\isamarkupfalse%
\ auto\isanewline
\ \ \isacommand{next}\isamarkupfalse%
\isanewline
\ \ \ \ \isacommand{case}\isamarkupfalse%
\ exB\isanewline
\ \ \ \ \isacommand{then}\isamarkupfalse%
\ \isacommand{obtain}\isamarkupfalse%
\ B\ \isakeyword{where}\ {\isachardoublequoteopen}y\ {\isasymin}\ eclose{\isacharparenleft}{\kern0pt}B{\isacharparenright}{\kern0pt}{\isachardoublequoteclose}\ {\isachardoublequoteopen}B{\isasymin}A{\isachardoublequoteclose}\isanewline
\ \ \ \ \ \ \isacommand{by}\isamarkupfalse%
\ auto\isanewline
\ \ \ \ \isacommand{then}\isamarkupfalse%
\isanewline
\ \ \ \ \isacommand{show}\isamarkupfalse%
\ {\isacharquery}{\kern0pt}thesis\ \isacommand{using}\isamarkupfalse%
\ {\isadigit{2}}\ ecloseD{\isacharbrackleft}{\kern0pt}of\ y\ B\ z{\isacharbrackright}{\kern0pt}\ \isacommand{by}\isamarkupfalse%
\ auto\isanewline
\ \ \isacommand{qed}\isamarkupfalse%
\isanewline
\isacommand{qed}\isamarkupfalse%
%
\endisatagproof
{\isafoldproof}%
%
\isadelimproof
\isanewline
%
\endisadelimproof
\isanewline
\isacommand{lemma}\isamarkupfalse%
\ eclose{\isacharunderscore}{\kern0pt}singE\ {\isacharcolon}{\kern0pt}\ {\isachardoublequoteopen}x\ {\isasymin}\ eclose{\isacharparenleft}{\kern0pt}{\isacharbraceleft}{\kern0pt}a{\isacharbraceright}{\kern0pt}{\isacharparenright}{\kern0pt}\ {\isasymLongrightarrow}\ x\ {\isacharequal}{\kern0pt}\ a\ {\isasymor}\ x\ {\isasymin}\ eclose{\isacharparenleft}{\kern0pt}a{\isacharparenright}{\kern0pt}{\isachardoublequoteclose}\isanewline
%
\isadelimproof
\ \ %
\endisadelimproof
%
\isatagproof
\isacommand{by}\isamarkupfalse%
{\isacharparenleft}{\kern0pt}blast\ dest{\isacharcolon}{\kern0pt}\ ecloseE{\isacharparenright}{\kern0pt}%
\endisatagproof
{\isafoldproof}%
%
\isadelimproof
\isanewline
%
\endisadelimproof
\isanewline
\isacommand{lemma}\isamarkupfalse%
\ in{\isacharunderscore}{\kern0pt}eclose{\isacharunderscore}{\kern0pt}sing\ {\isacharcolon}{\kern0pt}\isanewline
\ \ \isakeyword{assumes}\ {\isachardoublequoteopen}x\ {\isasymin}\ eclose{\isacharparenleft}{\kern0pt}{\isacharbraceleft}{\kern0pt}a{\isacharbraceright}{\kern0pt}{\isacharparenright}{\kern0pt}{\isachardoublequoteclose}\ {\isachardoublequoteopen}a\ {\isasymin}\ eclose{\isacharparenleft}{\kern0pt}z{\isacharparenright}{\kern0pt}{\isachardoublequoteclose}\isanewline
\ \ \isakeyword{shows}\ {\isachardoublequoteopen}x\ {\isasymin}\ eclose{\isacharparenleft}{\kern0pt}{\isacharbraceleft}{\kern0pt}z{\isacharbraceright}{\kern0pt}{\isacharparenright}{\kern0pt}{\isachardoublequoteclose}\isanewline
%
\isadelimproof
%
\endisadelimproof
%
\isatagproof
\isacommand{proof}\isamarkupfalse%
\ {\isacharminus}{\kern0pt}\isanewline
\ \ \isacommand{from}\isamarkupfalse%
\ {\isacartoucheopen}x{\isasymin}eclose{\isacharparenleft}{\kern0pt}{\isacharbraceleft}{\kern0pt}a{\isacharbraceright}{\kern0pt}{\isacharparenright}{\kern0pt}{\isacartoucheclose}\isanewline
\ \ \isacommand{consider}\isamarkupfalse%
\ {\isacharparenleft}{\kern0pt}eq{\isacharparenright}{\kern0pt}\ {\isachardoublequoteopen}x{\isacharequal}{\kern0pt}a{\isachardoublequoteclose}\ {\isacharbar}{\kern0pt}\ {\isacharparenleft}{\kern0pt}lt{\isacharparenright}{\kern0pt}\ {\isachardoublequoteopen}x{\isasymin}eclose{\isacharparenleft}{\kern0pt}a{\isacharparenright}{\kern0pt}{\isachardoublequoteclose}\isanewline
\ \ \ \ \isacommand{using}\isamarkupfalse%
\ eclose{\isacharunderscore}{\kern0pt}singE\ \isacommand{by}\isamarkupfalse%
\ auto\isanewline
\ \ \isacommand{then}\isamarkupfalse%
\isanewline
\ \ \isacommand{show}\isamarkupfalse%
\ {\isacharquery}{\kern0pt}thesis\isanewline
\ \ \ \ \isacommand{using}\isamarkupfalse%
\ eclose{\isacharunderscore}{\kern0pt}sing\ mem{\isacharunderscore}{\kern0pt}eclose{\isacharunderscore}{\kern0pt}trans\ assms\isanewline
\ \ \ \ \isacommand{by}\isamarkupfalse%
\ {\isacharparenleft}{\kern0pt}cases{\isacharcomma}{\kern0pt}\ auto{\isacharparenright}{\kern0pt}\isanewline
\isacommand{qed}\isamarkupfalse%
%
\endisatagproof
{\isafoldproof}%
%
\isadelimproof
\isanewline
%
\endisadelimproof
\isanewline
\isacommand{lemma}\isamarkupfalse%
\ in{\isacharunderscore}{\kern0pt}dom{\isacharunderscore}{\kern0pt}in{\isacharunderscore}{\kern0pt}eclose\ {\isacharcolon}{\kern0pt}\isanewline
\ \ \isakeyword{assumes}\ {\isachardoublequoteopen}x\ {\isasymin}\ domain{\isacharparenleft}{\kern0pt}z{\isacharparenright}{\kern0pt}{\isachardoublequoteclose}\isanewline
\ \ \isakeyword{shows}\ {\isachardoublequoteopen}x\ {\isasymin}\ eclose{\isacharparenleft}{\kern0pt}z{\isacharparenright}{\kern0pt}{\isachardoublequoteclose}\isanewline
%
\isadelimproof
%
\endisadelimproof
%
\isatagproof
\isacommand{proof}\isamarkupfalse%
\ {\isacharminus}{\kern0pt}\isanewline
\ \ \isacommand{from}\isamarkupfalse%
\ assms\isanewline
\ \ \isacommand{obtain}\isamarkupfalse%
\ y\ \isakeyword{where}\ {\isachardoublequoteopen}{\isasymlangle}x{\isacharcomma}{\kern0pt}y{\isasymrangle}\ {\isasymin}\ z{\isachardoublequoteclose}\isanewline
\ \ \ \ \isacommand{unfolding}\isamarkupfalse%
\ domain{\isacharunderscore}{\kern0pt}def\ \isacommand{by}\isamarkupfalse%
\ auto\isanewline
\ \ \isacommand{then}\isamarkupfalse%
\isanewline
\ \ \isacommand{show}\isamarkupfalse%
\ {\isacharquery}{\kern0pt}thesis\isanewline
\ \ \ \ \isacommand{unfolding}\isamarkupfalse%
\ Pair{\isacharunderscore}{\kern0pt}def\isanewline
\ \ \ \ \isacommand{using}\isamarkupfalse%
\ ecloseD{\isacharbrackleft}{\kern0pt}of\ {\isachardoublequoteopen}{\isacharbraceleft}{\kern0pt}x{\isacharcomma}{\kern0pt}x{\isacharbraceright}{\kern0pt}{\isachardoublequoteclose}{\isacharbrackright}{\kern0pt}\ ecloseD{\isacharbrackleft}{\kern0pt}of\ {\isachardoublequoteopen}{\isacharbraceleft}{\kern0pt}{\isacharbraceleft}{\kern0pt}x{\isacharcomma}{\kern0pt}x{\isacharbraceright}{\kern0pt}{\isacharcomma}{\kern0pt}{\isacharbraceleft}{\kern0pt}x{\isacharcomma}{\kern0pt}y{\isacharbraceright}{\kern0pt}{\isacharbraceright}{\kern0pt}{\isachardoublequoteclose}{\isacharbrackright}{\kern0pt}\ arg{\isacharunderscore}{\kern0pt}into{\isacharunderscore}{\kern0pt}eclose\isanewline
\ \ \ \ \isacommand{by}\isamarkupfalse%
\ auto\isanewline
\isacommand{qed}\isamarkupfalse%
%
\endisatagproof
{\isafoldproof}%
%
\isadelimproof
%
\endisadelimproof
%
\begin{isamarkuptext}%
term\isa{ed} is the well-founded relation on which \isa{val} is defined.%
\end{isamarkuptext}\isamarkuptrue%
\isacommand{definition}\isamarkupfalse%
\isanewline
\ \ ed\ {\isacharcolon}{\kern0pt}{\isacharcolon}{\kern0pt}\ {\isachardoublequoteopen}{\isacharbrackleft}{\kern0pt}i{\isacharcomma}{\kern0pt}i{\isacharbrackright}{\kern0pt}\ {\isasymRightarrow}\ o{\isachardoublequoteclose}\ \isakeyword{where}\isanewline
\ \ {\isachardoublequoteopen}ed{\isacharparenleft}{\kern0pt}x{\isacharcomma}{\kern0pt}y{\isacharparenright}{\kern0pt}\ {\isasymequiv}\ x\ {\isasymin}\ domain{\isacharparenleft}{\kern0pt}y{\isacharparenright}{\kern0pt}{\isachardoublequoteclose}\isanewline
\isanewline
\isacommand{definition}\isamarkupfalse%
\isanewline
\ \ edrel\ {\isacharcolon}{\kern0pt}{\isacharcolon}{\kern0pt}\ {\isachardoublequoteopen}i\ {\isasymRightarrow}\ i{\isachardoublequoteclose}\ \isakeyword{where}\isanewline
\ \ {\isachardoublequoteopen}edrel{\isacharparenleft}{\kern0pt}A{\isacharparenright}{\kern0pt}\ {\isasymequiv}\ Rrel{\isacharparenleft}{\kern0pt}ed{\isacharcomma}{\kern0pt}A{\isacharparenright}{\kern0pt}{\isachardoublequoteclose}\isanewline
\isanewline
\isanewline
\isacommand{lemma}\isamarkupfalse%
\ edI{\isacharbrackleft}{\kern0pt}intro{\isacharbang}{\kern0pt}{\isacharbrackright}{\kern0pt}{\isacharcolon}{\kern0pt}\ {\isachardoublequoteopen}t{\isasymin}domain{\isacharparenleft}{\kern0pt}x{\isacharparenright}{\kern0pt}\ {\isasymLongrightarrow}\ ed{\isacharparenleft}{\kern0pt}t{\isacharcomma}{\kern0pt}x{\isacharparenright}{\kern0pt}{\isachardoublequoteclose}\isanewline
%
\isadelimproof
\ \ %
\endisadelimproof
%
\isatagproof
\isacommand{unfolding}\isamarkupfalse%
\ ed{\isacharunderscore}{\kern0pt}def\ \isacommand{{\isachardot}{\kern0pt}}\isamarkupfalse%
%
\endisatagproof
{\isafoldproof}%
%
\isadelimproof
\isanewline
%
\endisadelimproof
\isanewline
\isacommand{lemma}\isamarkupfalse%
\ edD{\isacharbrackleft}{\kern0pt}dest{\isacharbang}{\kern0pt}{\isacharbrackright}{\kern0pt}{\isacharcolon}{\kern0pt}\ {\isachardoublequoteopen}ed{\isacharparenleft}{\kern0pt}t{\isacharcomma}{\kern0pt}x{\isacharparenright}{\kern0pt}\ {\isasymLongrightarrow}\ t{\isasymin}domain{\isacharparenleft}{\kern0pt}x{\isacharparenright}{\kern0pt}{\isachardoublequoteclose}\isanewline
%
\isadelimproof
\ \ %
\endisadelimproof
%
\isatagproof
\isacommand{unfolding}\isamarkupfalse%
\ ed{\isacharunderscore}{\kern0pt}def\ \isacommand{{\isachardot}{\kern0pt}}\isamarkupfalse%
%
\endisatagproof
{\isafoldproof}%
%
\isadelimproof
\isanewline
%
\endisadelimproof
\isanewline
\isanewline
\isacommand{lemma}\isamarkupfalse%
\ rank{\isacharunderscore}{\kern0pt}ed{\isacharcolon}{\kern0pt}\isanewline
\ \ \isakeyword{assumes}\ {\isachardoublequoteopen}ed{\isacharparenleft}{\kern0pt}y{\isacharcomma}{\kern0pt}x{\isacharparenright}{\kern0pt}{\isachardoublequoteclose}\isanewline
\ \ \isakeyword{shows}\ {\isachardoublequoteopen}succ{\isacharparenleft}{\kern0pt}rank{\isacharparenleft}{\kern0pt}y{\isacharparenright}{\kern0pt}{\isacharparenright}{\kern0pt}\ {\isasymle}\ rank{\isacharparenleft}{\kern0pt}x{\isacharparenright}{\kern0pt}{\isachardoublequoteclose}\isanewline
%
\isadelimproof
%
\endisadelimproof
%
\isatagproof
\isacommand{proof}\isamarkupfalse%
\isanewline
\ \ \isacommand{from}\isamarkupfalse%
\ assms\isanewline
\ \ \isacommand{obtain}\isamarkupfalse%
\ p\ \isakeyword{where}\ {\isachardoublequoteopen}{\isasymlangle}y{\isacharcomma}{\kern0pt}p{\isasymrangle}{\isasymin}x{\isachardoublequoteclose}\ \isacommand{by}\isamarkupfalse%
\ auto\isanewline
\ \ \isacommand{moreover}\isamarkupfalse%
\isanewline
\ \ \isacommand{obtain}\isamarkupfalse%
\ s\ \isakeyword{where}\ {\isachardoublequoteopen}y{\isasymin}s{\isachardoublequoteclose}\ {\isachardoublequoteopen}s{\isasymin}{\isasymlangle}y{\isacharcomma}{\kern0pt}p{\isasymrangle}{\isachardoublequoteclose}\ \isacommand{unfolding}\isamarkupfalse%
\ Pair{\isacharunderscore}{\kern0pt}def\ \isacommand{by}\isamarkupfalse%
\ auto\isanewline
\ \ \isacommand{ultimately}\isamarkupfalse%
\isanewline
\ \ \isacommand{have}\isamarkupfalse%
\ {\isachardoublequoteopen}rank{\isacharparenleft}{\kern0pt}y{\isacharparenright}{\kern0pt}\ {\isacharless}{\kern0pt}\ rank{\isacharparenleft}{\kern0pt}s{\isacharparenright}{\kern0pt}{\isachardoublequoteclose}\ {\isachardoublequoteopen}rank{\isacharparenleft}{\kern0pt}s{\isacharparenright}{\kern0pt}\ {\isacharless}{\kern0pt}\ rank{\isacharparenleft}{\kern0pt}{\isasymlangle}y{\isacharcomma}{\kern0pt}p{\isasymrangle}{\isacharparenright}{\kern0pt}{\isachardoublequoteclose}\ {\isachardoublequoteopen}rank{\isacharparenleft}{\kern0pt}{\isasymlangle}y{\isacharcomma}{\kern0pt}p{\isasymrangle}{\isacharparenright}{\kern0pt}\ {\isacharless}{\kern0pt}\ rank{\isacharparenleft}{\kern0pt}x{\isacharparenright}{\kern0pt}{\isachardoublequoteclose}\isanewline
\ \ \ \ \isacommand{using}\isamarkupfalse%
\ rank{\isacharunderscore}{\kern0pt}lt\ \isacommand{by}\isamarkupfalse%
\ blast{\isacharplus}{\kern0pt}\isanewline
\ \ \isacommand{then}\isamarkupfalse%
\isanewline
\ \ \isacommand{show}\isamarkupfalse%
\ {\isachardoublequoteopen}rank{\isacharparenleft}{\kern0pt}y{\isacharparenright}{\kern0pt}\ {\isacharless}{\kern0pt}\ rank{\isacharparenleft}{\kern0pt}x{\isacharparenright}{\kern0pt}{\isachardoublequoteclose}\isanewline
\ \ \ \ \isacommand{using}\isamarkupfalse%
\ lt{\isacharunderscore}{\kern0pt}trans\ \isacommand{by}\isamarkupfalse%
\ blast\isanewline
\isacommand{qed}\isamarkupfalse%
%
\endisatagproof
{\isafoldproof}%
%
\isadelimproof
\isanewline
%
\endisadelimproof
\isanewline
\isacommand{lemma}\isamarkupfalse%
\ edrel{\isacharunderscore}{\kern0pt}dest\ {\isacharbrackleft}{\kern0pt}dest{\isacharbrackright}{\kern0pt}{\isacharcolon}{\kern0pt}\ {\isachardoublequoteopen}x\ {\isasymin}\ edrel{\isacharparenleft}{\kern0pt}A{\isacharparenright}{\kern0pt}\ {\isasymLongrightarrow}\ {\isasymexists}\ a{\isasymin}\ A{\isachardot}{\kern0pt}\ {\isasymexists}\ b\ {\isasymin}\ A{\isachardot}{\kern0pt}\ x\ {\isacharequal}{\kern0pt}{\isasymlangle}a{\isacharcomma}{\kern0pt}b{\isasymrangle}{\isachardoublequoteclose}\isanewline
%
\isadelimproof
\ \ %
\endisadelimproof
%
\isatagproof
\isacommand{by}\isamarkupfalse%
{\isacharparenleft}{\kern0pt}auto\ simp\ add{\isacharcolon}{\kern0pt}ed{\isacharunderscore}{\kern0pt}def\ edrel{\isacharunderscore}{\kern0pt}def\ Rrel{\isacharunderscore}{\kern0pt}def{\isacharparenright}{\kern0pt}%
\endisatagproof
{\isafoldproof}%
%
\isadelimproof
\isanewline
%
\endisadelimproof
\isanewline
\isacommand{lemma}\isamarkupfalse%
\ edrelD\ {\isacharcolon}{\kern0pt}\ {\isachardoublequoteopen}x\ {\isasymin}\ edrel{\isacharparenleft}{\kern0pt}A{\isacharparenright}{\kern0pt}\ {\isasymLongrightarrow}\ {\isasymexists}\ a{\isasymin}\ A{\isachardot}{\kern0pt}\ {\isasymexists}\ b\ {\isasymin}\ A{\isachardot}{\kern0pt}\ x\ {\isacharequal}{\kern0pt}{\isasymlangle}a{\isacharcomma}{\kern0pt}b{\isasymrangle}\ {\isasymand}\ a\ {\isasymin}\ domain{\isacharparenleft}{\kern0pt}b{\isacharparenright}{\kern0pt}{\isachardoublequoteclose}\isanewline
%
\isadelimproof
\ \ %
\endisadelimproof
%
\isatagproof
\isacommand{by}\isamarkupfalse%
{\isacharparenleft}{\kern0pt}auto\ simp\ add{\isacharcolon}{\kern0pt}ed{\isacharunderscore}{\kern0pt}def\ edrel{\isacharunderscore}{\kern0pt}def\ Rrel{\isacharunderscore}{\kern0pt}def{\isacharparenright}{\kern0pt}%
\endisatagproof
{\isafoldproof}%
%
\isadelimproof
\isanewline
%
\endisadelimproof
\isanewline
\isacommand{lemma}\isamarkupfalse%
\ edrelI\ {\isacharbrackleft}{\kern0pt}intro{\isacharbang}{\kern0pt}{\isacharbrackright}{\kern0pt}{\isacharcolon}{\kern0pt}\ {\isachardoublequoteopen}x{\isasymin}A\ {\isasymLongrightarrow}\ y{\isasymin}A\ {\isasymLongrightarrow}\ x\ {\isasymin}\ domain{\isacharparenleft}{\kern0pt}y{\isacharparenright}{\kern0pt}\ {\isasymLongrightarrow}\ {\isasymlangle}x{\isacharcomma}{\kern0pt}y{\isasymrangle}{\isasymin}edrel{\isacharparenleft}{\kern0pt}A{\isacharparenright}{\kern0pt}{\isachardoublequoteclose}\isanewline
%
\isadelimproof
\ \ %
\endisadelimproof
%
\isatagproof
\isacommand{by}\isamarkupfalse%
\ {\isacharparenleft}{\kern0pt}simp\ add{\isacharcolon}{\kern0pt}ed{\isacharunderscore}{\kern0pt}def\ edrel{\isacharunderscore}{\kern0pt}def\ Rrel{\isacharunderscore}{\kern0pt}def{\isacharparenright}{\kern0pt}%
\endisatagproof
{\isafoldproof}%
%
\isadelimproof
\isanewline
%
\endisadelimproof
\isanewline
\isacommand{lemma}\isamarkupfalse%
\ edrel{\isacharunderscore}{\kern0pt}trans{\isacharcolon}{\kern0pt}\ {\isachardoublequoteopen}Transset{\isacharparenleft}{\kern0pt}A{\isacharparenright}{\kern0pt}\ {\isasymLongrightarrow}\ y{\isasymin}A\ {\isasymLongrightarrow}\ x\ {\isasymin}\ domain{\isacharparenleft}{\kern0pt}y{\isacharparenright}{\kern0pt}\ {\isasymLongrightarrow}\ {\isasymlangle}x{\isacharcomma}{\kern0pt}y{\isasymrangle}{\isasymin}edrel{\isacharparenleft}{\kern0pt}A{\isacharparenright}{\kern0pt}{\isachardoublequoteclose}\isanewline
%
\isadelimproof
\ \ %
\endisadelimproof
%
\isatagproof
\isacommand{by}\isamarkupfalse%
\ {\isacharparenleft}{\kern0pt}rule\ edrelI{\isacharcomma}{\kern0pt}\ auto\ simp\ add{\isacharcolon}{\kern0pt}Transset{\isacharunderscore}{\kern0pt}def\ domain{\isacharunderscore}{\kern0pt}def\ Pair{\isacharunderscore}{\kern0pt}def{\isacharparenright}{\kern0pt}%
\endisatagproof
{\isafoldproof}%
%
\isadelimproof
\isanewline
%
\endisadelimproof
\isanewline
\isacommand{lemma}\isamarkupfalse%
\ domain{\isacharunderscore}{\kern0pt}trans{\isacharcolon}{\kern0pt}\ {\isachardoublequoteopen}Transset{\isacharparenleft}{\kern0pt}A{\isacharparenright}{\kern0pt}\ {\isasymLongrightarrow}\ y{\isasymin}A\ {\isasymLongrightarrow}\ x\ {\isasymin}\ domain{\isacharparenleft}{\kern0pt}y{\isacharparenright}{\kern0pt}\ {\isasymLongrightarrow}\ x{\isasymin}A{\isachardoublequoteclose}\isanewline
%
\isadelimproof
\ \ %
\endisadelimproof
%
\isatagproof
\isacommand{by}\isamarkupfalse%
\ {\isacharparenleft}{\kern0pt}auto\ simp\ add{\isacharcolon}{\kern0pt}\ Transset{\isacharunderscore}{\kern0pt}def\ domain{\isacharunderscore}{\kern0pt}def\ Pair{\isacharunderscore}{\kern0pt}def{\isacharparenright}{\kern0pt}%
\endisatagproof
{\isafoldproof}%
%
\isadelimproof
\isanewline
%
\endisadelimproof
\isanewline
\isacommand{lemma}\isamarkupfalse%
\ relation{\isacharunderscore}{\kern0pt}edrel\ {\isacharcolon}{\kern0pt}\ {\isachardoublequoteopen}relation{\isacharparenleft}{\kern0pt}edrel{\isacharparenleft}{\kern0pt}A{\isacharparenright}{\kern0pt}{\isacharparenright}{\kern0pt}{\isachardoublequoteclose}\isanewline
%
\isadelimproof
\ \ %
\endisadelimproof
%
\isatagproof
\isacommand{by}\isamarkupfalse%
{\isacharparenleft}{\kern0pt}auto\ simp\ add{\isacharcolon}{\kern0pt}\ relation{\isacharunderscore}{\kern0pt}def{\isacharparenright}{\kern0pt}%
\endisatagproof
{\isafoldproof}%
%
\isadelimproof
\isanewline
%
\endisadelimproof
\isanewline
\isacommand{lemma}\isamarkupfalse%
\ field{\isacharunderscore}{\kern0pt}edrel\ {\isacharcolon}{\kern0pt}\ {\isachardoublequoteopen}field{\isacharparenleft}{\kern0pt}edrel{\isacharparenleft}{\kern0pt}A{\isacharparenright}{\kern0pt}{\isacharparenright}{\kern0pt}{\isasymsubseteq}A{\isachardoublequoteclose}\isanewline
%
\isadelimproof
\ \ %
\endisadelimproof
%
\isatagproof
\isacommand{by}\isamarkupfalse%
\ blast%
\endisatagproof
{\isafoldproof}%
%
\isadelimproof
\isanewline
%
\endisadelimproof
\isanewline
\isacommand{lemma}\isamarkupfalse%
\ edrel{\isacharunderscore}{\kern0pt}sub{\isacharunderscore}{\kern0pt}memrel{\isacharcolon}{\kern0pt}\ {\isachardoublequoteopen}edrel{\isacharparenleft}{\kern0pt}A{\isacharparenright}{\kern0pt}\ {\isasymsubseteq}\ trancl{\isacharparenleft}{\kern0pt}Memrel{\isacharparenleft}{\kern0pt}eclose{\isacharparenleft}{\kern0pt}A{\isacharparenright}{\kern0pt}{\isacharparenright}{\kern0pt}{\isacharparenright}{\kern0pt}{\isachardoublequoteclose}\isanewline
%
\isadelimproof
%
\endisadelimproof
%
\isatagproof
\isacommand{proof}\isamarkupfalse%
\isanewline
\ \ \isacommand{fix}\isamarkupfalse%
\ z\isanewline
\ \ \isacommand{assume}\isamarkupfalse%
\isanewline
\ \ \ \ {\isachardoublequoteopen}z{\isasymin}edrel{\isacharparenleft}{\kern0pt}A{\isacharparenright}{\kern0pt}{\isachardoublequoteclose}\isanewline
\ \ \isacommand{then}\isamarkupfalse%
\ \isacommand{obtain}\isamarkupfalse%
\ x\ y\ \isakeyword{where}\isanewline
\ \ \ \ Eq{\isadigit{1}}{\isacharcolon}{\kern0pt}\ \ \ {\isachardoublequoteopen}x{\isasymin}A{\isachardoublequoteclose}\ {\isachardoublequoteopen}y{\isasymin}A{\isachardoublequoteclose}\ {\isachardoublequoteopen}z{\isacharequal}{\kern0pt}{\isasymlangle}x{\isacharcomma}{\kern0pt}y{\isasymrangle}{\isachardoublequoteclose}\ {\isachardoublequoteopen}x{\isasymin}domain{\isacharparenleft}{\kern0pt}y{\isacharparenright}{\kern0pt}{\isachardoublequoteclose}\isanewline
\ \ \ \ \isacommand{using}\isamarkupfalse%
\ edrelD\isanewline
\ \ \ \ \isacommand{by}\isamarkupfalse%
\ blast\isanewline
\ \ \isacommand{then}\isamarkupfalse%
\ \isacommand{obtain}\isamarkupfalse%
\ u\ v\ \isakeyword{where}\isanewline
\ \ \ \ Eq{\isadigit{2}}{\isacharcolon}{\kern0pt}\ \ \ {\isachardoublequoteopen}x{\isasymin}u{\isachardoublequoteclose}\ {\isachardoublequoteopen}u{\isasymin}v{\isachardoublequoteclose}\ {\isachardoublequoteopen}v{\isasymin}y{\isachardoublequoteclose}\isanewline
\ \ \ \ \isacommand{unfolding}\isamarkupfalse%
\ domain{\isacharunderscore}{\kern0pt}def\ Pair{\isacharunderscore}{\kern0pt}def\ \isacommand{by}\isamarkupfalse%
\ auto\isanewline
\ \ \isacommand{with}\isamarkupfalse%
\ Eq{\isadigit{1}}\ \isacommand{have}\isamarkupfalse%
\isanewline
\ \ \ \ Eq{\isadigit{3}}{\isacharcolon}{\kern0pt}\ \ \ {\isachardoublequoteopen}x{\isasymin}eclose{\isacharparenleft}{\kern0pt}A{\isacharparenright}{\kern0pt}{\isachardoublequoteclose}\ {\isachardoublequoteopen}y{\isasymin}eclose{\isacharparenleft}{\kern0pt}A{\isacharparenright}{\kern0pt}{\isachardoublequoteclose}\ {\isachardoublequoteopen}u{\isasymin}eclose{\isacharparenleft}{\kern0pt}A{\isacharparenright}{\kern0pt}{\isachardoublequoteclose}\ {\isachardoublequoteopen}v{\isasymin}eclose{\isacharparenleft}{\kern0pt}A{\isacharparenright}{\kern0pt}{\isachardoublequoteclose}\isanewline
\ \ \ \ \isacommand{by}\isamarkupfalse%
\ {\isacharparenleft}{\kern0pt}auto{\isacharcomma}{\kern0pt}\ rule{\isacharunderscore}{\kern0pt}tac\ {\isacharbrackleft}{\kern0pt}{\isadigit{3}}{\isacharminus}{\kern0pt}{\isadigit{4}}{\isacharbrackright}{\kern0pt}\ ecloseD{\isacharcomma}{\kern0pt}\ rule{\isacharunderscore}{\kern0pt}tac\ {\isacharbrackleft}{\kern0pt}{\isadigit{3}}{\isacharbrackright}{\kern0pt}\ ecloseD{\isacharcomma}{\kern0pt}\ simp{\isacharunderscore}{\kern0pt}all\ add{\isacharcolon}{\kern0pt}arg{\isacharunderscore}{\kern0pt}into{\isacharunderscore}{\kern0pt}eclose{\isacharparenright}{\kern0pt}\isanewline
\ \ \isacommand{let}\isamarkupfalse%
\isanewline
\ \ \ \ {\isacharquery}{\kern0pt}r{\isacharequal}{\kern0pt}{\isachardoublequoteopen}trancl{\isacharparenleft}{\kern0pt}Memrel{\isacharparenleft}{\kern0pt}eclose{\isacharparenleft}{\kern0pt}A{\isacharparenright}{\kern0pt}{\isacharparenright}{\kern0pt}{\isacharparenright}{\kern0pt}{\isachardoublequoteclose}\isanewline
\ \ \isacommand{from}\isamarkupfalse%
\ Eq{\isadigit{2}}\ \isakeyword{and}\ Eq{\isadigit{3}}\ \isacommand{have}\isamarkupfalse%
\isanewline
\ \ \ \ {\isachardoublequoteopen}{\isasymlangle}x{\isacharcomma}{\kern0pt}u{\isasymrangle}{\isasymin}{\isacharquery}{\kern0pt}r{\isachardoublequoteclose}\ {\isachardoublequoteopen}{\isasymlangle}u{\isacharcomma}{\kern0pt}v{\isasymrangle}{\isasymin}{\isacharquery}{\kern0pt}r{\isachardoublequoteclose}\ {\isachardoublequoteopen}{\isasymlangle}v{\isacharcomma}{\kern0pt}y{\isasymrangle}{\isasymin}{\isacharquery}{\kern0pt}r{\isachardoublequoteclose}\isanewline
\ \ \ \ \isacommand{by}\isamarkupfalse%
\ {\isacharparenleft}{\kern0pt}auto\ simp\ add{\isacharcolon}{\kern0pt}\ r{\isacharunderscore}{\kern0pt}into{\isacharunderscore}{\kern0pt}trancl{\isacharparenright}{\kern0pt}\isanewline
\ \ \isacommand{then}\isamarkupfalse%
\ \ \isacommand{have}\isamarkupfalse%
\isanewline
\ \ \ \ {\isachardoublequoteopen}{\isasymlangle}x{\isacharcomma}{\kern0pt}y{\isasymrangle}{\isasymin}{\isacharquery}{\kern0pt}r{\isachardoublequoteclose}\isanewline
\ \ \ \ \isacommand{by}\isamarkupfalse%
\ {\isacharparenleft}{\kern0pt}rule{\isacharunderscore}{\kern0pt}tac\ trancl{\isacharunderscore}{\kern0pt}trans{\isacharcomma}{\kern0pt}\ rule{\isacharunderscore}{\kern0pt}tac\ {\isacharbrackleft}{\kern0pt}{\isadigit{2}}{\isacharbrackright}{\kern0pt}\ trancl{\isacharunderscore}{\kern0pt}trans{\isacharcomma}{\kern0pt}\ simp{\isacharparenright}{\kern0pt}\isanewline
\ \ \isacommand{with}\isamarkupfalse%
\ Eq{\isadigit{1}}\ \isacommand{show}\isamarkupfalse%
\ {\isachardoublequoteopen}z{\isasymin}{\isacharquery}{\kern0pt}r{\isachardoublequoteclose}\ \isacommand{by}\isamarkupfalse%
\ simp\isanewline
\isacommand{qed}\isamarkupfalse%
%
\endisatagproof
{\isafoldproof}%
%
\isadelimproof
\isanewline
%
\endisadelimproof
\isanewline
\isacommand{lemma}\isamarkupfalse%
\ wf{\isacharunderscore}{\kern0pt}edrel\ {\isacharcolon}{\kern0pt}\ {\isachardoublequoteopen}wf{\isacharparenleft}{\kern0pt}edrel{\isacharparenleft}{\kern0pt}A{\isacharparenright}{\kern0pt}{\isacharparenright}{\kern0pt}{\isachardoublequoteclose}\isanewline
%
\isadelimproof
\ \ %
\endisadelimproof
%
\isatagproof
\isacommand{using}\isamarkupfalse%
\ wf{\isacharunderscore}{\kern0pt}subset\ {\isacharbrackleft}{\kern0pt}of\ {\isachardoublequoteopen}trancl{\isacharparenleft}{\kern0pt}Memrel{\isacharparenleft}{\kern0pt}eclose{\isacharparenleft}{\kern0pt}A{\isacharparenright}{\kern0pt}{\isacharparenright}{\kern0pt}{\isacharparenright}{\kern0pt}{\isachardoublequoteclose}{\isacharbrackright}{\kern0pt}\ edrel{\isacharunderscore}{\kern0pt}sub{\isacharunderscore}{\kern0pt}memrel\isanewline
\ \ \ \ wf{\isacharunderscore}{\kern0pt}trancl\ wf{\isacharunderscore}{\kern0pt}Memrel\isanewline
\ \ \isacommand{by}\isamarkupfalse%
\ auto%
\endisatagproof
{\isafoldproof}%
%
\isadelimproof
\isanewline
%
\endisadelimproof
\isanewline
\isacommand{lemma}\isamarkupfalse%
\ ed{\isacharunderscore}{\kern0pt}induction{\isacharcolon}{\kern0pt}\isanewline
\ \ \isakeyword{assumes}\ {\isachardoublequoteopen}{\isasymAnd}x{\isachardot}{\kern0pt}\ {\isasymlbrakk}{\isasymAnd}y{\isachardot}{\kern0pt}\ \ ed{\isacharparenleft}{\kern0pt}y{\isacharcomma}{\kern0pt}x{\isacharparenright}{\kern0pt}\ {\isasymLongrightarrow}\ Q{\isacharparenleft}{\kern0pt}y{\isacharparenright}{\kern0pt}\ {\isasymrbrakk}\ {\isasymLongrightarrow}\ Q{\isacharparenleft}{\kern0pt}x{\isacharparenright}{\kern0pt}{\isachardoublequoteclose}\isanewline
\ \ \isakeyword{shows}\ {\isachardoublequoteopen}Q{\isacharparenleft}{\kern0pt}a{\isacharparenright}{\kern0pt}{\isachardoublequoteclose}\isanewline
%
\isadelimproof
%
\endisadelimproof
%
\isatagproof
\isacommand{proof}\isamarkupfalse%
{\isacharparenleft}{\kern0pt}induct\ rule{\isacharcolon}{\kern0pt}\ wf{\isacharunderscore}{\kern0pt}induct{\isadigit{2}}{\isacharbrackleft}{\kern0pt}OF\ wf{\isacharunderscore}{\kern0pt}edrel{\isacharbrackleft}{\kern0pt}of\ {\isachardoublequoteopen}eclose{\isacharparenleft}{\kern0pt}{\isacharbraceleft}{\kern0pt}a{\isacharbraceright}{\kern0pt}{\isacharparenright}{\kern0pt}{\isachardoublequoteclose}{\isacharbrackright}{\kern0pt}\ {\isacharcomma}{\kern0pt}of\ a\ {\isachardoublequoteopen}eclose{\isacharparenleft}{\kern0pt}{\isacharbraceleft}{\kern0pt}a{\isacharbraceright}{\kern0pt}{\isacharparenright}{\kern0pt}{\isachardoublequoteclose}{\isacharbrackright}{\kern0pt}{\isacharparenright}{\kern0pt}\isanewline
\ \ \isacommand{case}\isamarkupfalse%
\ {\isadigit{1}}\isanewline
\ \ \isacommand{then}\isamarkupfalse%
\ \isacommand{show}\isamarkupfalse%
\ {\isacharquery}{\kern0pt}case\ \isacommand{using}\isamarkupfalse%
\ arg{\isacharunderscore}{\kern0pt}into{\isacharunderscore}{\kern0pt}eclose\ \isacommand{by}\isamarkupfalse%
\ simp\isanewline
\isacommand{next}\isamarkupfalse%
\isanewline
\ \ \isacommand{case}\isamarkupfalse%
\ {\isadigit{2}}\isanewline
\ \ \isacommand{then}\isamarkupfalse%
\ \isacommand{show}\isamarkupfalse%
\ {\isacharquery}{\kern0pt}case\ \isacommand{using}\isamarkupfalse%
\ field{\isacharunderscore}{\kern0pt}edrel\ \isacommand{{\isachardot}{\kern0pt}}\isamarkupfalse%
\isanewline
\isacommand{next}\isamarkupfalse%
\isanewline
\ \ \isacommand{case}\isamarkupfalse%
\ {\isacharparenleft}{\kern0pt}{\isadigit{3}}\ x{\isacharparenright}{\kern0pt}\isanewline
\ \ \isacommand{then}\isamarkupfalse%
\isanewline
\ \ \isacommand{show}\isamarkupfalse%
\ {\isacharquery}{\kern0pt}case\isanewline
\ \ \ \ \isacommand{using}\isamarkupfalse%
\ assms{\isacharbrackleft}{\kern0pt}of\ x{\isacharbrackright}{\kern0pt}\ edrelI\ domain{\isacharunderscore}{\kern0pt}trans{\isacharbrackleft}{\kern0pt}OF\ Transset{\isacharunderscore}{\kern0pt}eclose\ {\isadigit{3}}{\isacharparenleft}{\kern0pt}{\isadigit{1}}{\isacharparenright}{\kern0pt}{\isacharbrackright}{\kern0pt}\ \isacommand{by}\isamarkupfalse%
\ blast\isanewline
\isacommand{qed}\isamarkupfalse%
%
\endisatagproof
{\isafoldproof}%
%
\isadelimproof
\isanewline
%
\endisadelimproof
\isanewline
\isacommand{lemma}\isamarkupfalse%
\ dom{\isacharunderscore}{\kern0pt}under{\isacharunderscore}{\kern0pt}edrel{\isacharunderscore}{\kern0pt}eclose{\isacharcolon}{\kern0pt}\ {\isachardoublequoteopen}edrel{\isacharparenleft}{\kern0pt}eclose{\isacharparenleft}{\kern0pt}{\isacharbraceleft}{\kern0pt}x{\isacharbraceright}{\kern0pt}{\isacharparenright}{\kern0pt}{\isacharparenright}{\kern0pt}\ {\isacharminus}{\kern0pt}{\isacharbackquote}{\kern0pt}{\isacharbackquote}{\kern0pt}\ {\isacharbraceleft}{\kern0pt}x{\isacharbraceright}{\kern0pt}\ {\isacharequal}{\kern0pt}\ domain{\isacharparenleft}{\kern0pt}x{\isacharparenright}{\kern0pt}{\isachardoublequoteclose}\isanewline
%
\isadelimproof
%
\endisadelimproof
%
\isatagproof
\isacommand{proof}\isamarkupfalse%
\isanewline
\ \ \isacommand{show}\isamarkupfalse%
\ {\isachardoublequoteopen}edrel{\isacharparenleft}{\kern0pt}eclose{\isacharparenleft}{\kern0pt}{\isacharbraceleft}{\kern0pt}x{\isacharbraceright}{\kern0pt}{\isacharparenright}{\kern0pt}{\isacharparenright}{\kern0pt}\ {\isacharminus}{\kern0pt}{\isacharbackquote}{\kern0pt}{\isacharbackquote}{\kern0pt}\ {\isacharbraceleft}{\kern0pt}x{\isacharbraceright}{\kern0pt}\ {\isasymsubseteq}\ domain{\isacharparenleft}{\kern0pt}x{\isacharparenright}{\kern0pt}{\isachardoublequoteclose}\isanewline
\ \ \ \ \isacommand{unfolding}\isamarkupfalse%
\ edrel{\isacharunderscore}{\kern0pt}def\ Rrel{\isacharunderscore}{\kern0pt}def\ ed{\isacharunderscore}{\kern0pt}def\isanewline
\ \ \ \ \isacommand{by}\isamarkupfalse%
\ auto\isanewline
\isacommand{next}\isamarkupfalse%
\isanewline
\ \ \isacommand{show}\isamarkupfalse%
\ {\isachardoublequoteopen}domain{\isacharparenleft}{\kern0pt}x{\isacharparenright}{\kern0pt}\ {\isasymsubseteq}\ edrel{\isacharparenleft}{\kern0pt}eclose{\isacharparenleft}{\kern0pt}{\isacharbraceleft}{\kern0pt}x{\isacharbraceright}{\kern0pt}{\isacharparenright}{\kern0pt}{\isacharparenright}{\kern0pt}\ {\isacharminus}{\kern0pt}{\isacharbackquote}{\kern0pt}{\isacharbackquote}{\kern0pt}\ {\isacharbraceleft}{\kern0pt}x{\isacharbraceright}{\kern0pt}{\isachardoublequoteclose}\isanewline
\ \ \ \ \isacommand{unfolding}\isamarkupfalse%
\ edrel{\isacharunderscore}{\kern0pt}def\ Rrel{\isacharunderscore}{\kern0pt}def\isanewline
\ \ \ \ \isacommand{using}\isamarkupfalse%
\ in{\isacharunderscore}{\kern0pt}dom{\isacharunderscore}{\kern0pt}in{\isacharunderscore}{\kern0pt}eclose\ eclose{\isacharunderscore}{\kern0pt}sing\ arg{\isacharunderscore}{\kern0pt}into{\isacharunderscore}{\kern0pt}eclose\isanewline
\ \ \ \ \isacommand{by}\isamarkupfalse%
\ blast\isanewline
\isacommand{qed}\isamarkupfalse%
%
\endisatagproof
{\isafoldproof}%
%
\isadelimproof
\isanewline
%
\endisadelimproof
\isanewline
\isacommand{lemma}\isamarkupfalse%
\ ed{\isacharunderscore}{\kern0pt}eclose\ {\isacharcolon}{\kern0pt}\ {\isachardoublequoteopen}{\isasymlangle}y{\isacharcomma}{\kern0pt}z{\isasymrangle}\ {\isasymin}\ edrel{\isacharparenleft}{\kern0pt}A{\isacharparenright}{\kern0pt}\ {\isasymLongrightarrow}\ y\ {\isasymin}\ eclose{\isacharparenleft}{\kern0pt}z{\isacharparenright}{\kern0pt}{\isachardoublequoteclose}\isanewline
%
\isadelimproof
\ \ %
\endisadelimproof
%
\isatagproof
\isacommand{by}\isamarkupfalse%
{\isacharparenleft}{\kern0pt}drule\ edrelD{\isacharcomma}{\kern0pt}auto\ simp\ add{\isacharcolon}{\kern0pt}domain{\isacharunderscore}{\kern0pt}def\ in{\isacharunderscore}{\kern0pt}dom{\isacharunderscore}{\kern0pt}in{\isacharunderscore}{\kern0pt}eclose{\isacharparenright}{\kern0pt}%
\endisatagproof
{\isafoldproof}%
%
\isadelimproof
\isanewline
%
\endisadelimproof
\isanewline
\isacommand{lemma}\isamarkupfalse%
\ tr{\isacharunderscore}{\kern0pt}edrel{\isacharunderscore}{\kern0pt}eclose\ {\isacharcolon}{\kern0pt}\ {\isachardoublequoteopen}{\isasymlangle}y{\isacharcomma}{\kern0pt}z{\isasymrangle}\ {\isasymin}\ edrel{\isacharparenleft}{\kern0pt}eclose{\isacharparenleft}{\kern0pt}{\isacharbraceleft}{\kern0pt}x{\isacharbraceright}{\kern0pt}{\isacharparenright}{\kern0pt}{\isacharparenright}{\kern0pt}{\isacharcircum}{\kern0pt}{\isacharplus}{\kern0pt}\ {\isasymLongrightarrow}\ y\ {\isasymin}\ eclose{\isacharparenleft}{\kern0pt}z{\isacharparenright}{\kern0pt}{\isachardoublequoteclose}\isanewline
%
\isadelimproof
\ \ %
\endisadelimproof
%
\isatagproof
\isacommand{by}\isamarkupfalse%
{\isacharparenleft}{\kern0pt}rule\ trancl{\isacharunderscore}{\kern0pt}induct{\isacharcomma}{\kern0pt}{\isacharparenleft}{\kern0pt}simp\ add{\isacharcolon}{\kern0pt}\ ed{\isacharunderscore}{\kern0pt}eclose\ mem{\isacharunderscore}{\kern0pt}eclose{\isacharunderscore}{\kern0pt}trans{\isacharparenright}{\kern0pt}{\isacharplus}{\kern0pt}{\isacharparenright}{\kern0pt}%
\endisatagproof
{\isafoldproof}%
%
\isadelimproof
\isanewline
%
\endisadelimproof
\isanewline
\isanewline
\isacommand{lemma}\isamarkupfalse%
\ restrict{\isacharunderscore}{\kern0pt}edrel{\isacharunderscore}{\kern0pt}eq\ {\isacharcolon}{\kern0pt}\isanewline
\ \ \isakeyword{assumes}\ {\isachardoublequoteopen}z\ {\isasymin}\ domain{\isacharparenleft}{\kern0pt}x{\isacharparenright}{\kern0pt}{\isachardoublequoteclose}\isanewline
\ \ \isakeyword{shows}\ {\isachardoublequoteopen}edrel{\isacharparenleft}{\kern0pt}eclose{\isacharparenleft}{\kern0pt}{\isacharbraceleft}{\kern0pt}x{\isacharbraceright}{\kern0pt}{\isacharparenright}{\kern0pt}{\isacharparenright}{\kern0pt}\ {\isasyminter}\ eclose{\isacharparenleft}{\kern0pt}{\isacharbraceleft}{\kern0pt}z{\isacharbraceright}{\kern0pt}{\isacharparenright}{\kern0pt}{\isasymtimes}eclose{\isacharparenleft}{\kern0pt}{\isacharbraceleft}{\kern0pt}z{\isacharbraceright}{\kern0pt}{\isacharparenright}{\kern0pt}\ {\isacharequal}{\kern0pt}\ edrel{\isacharparenleft}{\kern0pt}eclose{\isacharparenleft}{\kern0pt}{\isacharbraceleft}{\kern0pt}z{\isacharbraceright}{\kern0pt}{\isacharparenright}{\kern0pt}{\isacharparenright}{\kern0pt}{\isachardoublequoteclose}\isanewline
%
\isadelimproof
%
\endisadelimproof
%
\isatagproof
\isacommand{proof}\isamarkupfalse%
{\isacharparenleft}{\kern0pt}intro\ equalityI\ subsetI{\isacharparenright}{\kern0pt}\isanewline
\ \ \isacommand{let}\isamarkupfalse%
\ {\isacharquery}{\kern0pt}ec{\isacharequal}{\kern0pt}{\isachardoublequoteopen}{\isasymlambda}\ y\ {\isachardot}{\kern0pt}\ edrel{\isacharparenleft}{\kern0pt}eclose{\isacharparenleft}{\kern0pt}{\isacharbraceleft}{\kern0pt}y{\isacharbraceright}{\kern0pt}{\isacharparenright}{\kern0pt}{\isacharparenright}{\kern0pt}{\isachardoublequoteclose}\isanewline
\ \ \isacommand{let}\isamarkupfalse%
\ {\isacharquery}{\kern0pt}ez{\isacharequal}{\kern0pt}{\isachardoublequoteopen}eclose{\isacharparenleft}{\kern0pt}{\isacharbraceleft}{\kern0pt}z{\isacharbraceright}{\kern0pt}{\isacharparenright}{\kern0pt}{\isachardoublequoteclose}\isanewline
\ \ \isacommand{let}\isamarkupfalse%
\ {\isacharquery}{\kern0pt}rr{\isacharequal}{\kern0pt}{\isachardoublequoteopen}{\isacharquery}{\kern0pt}ec{\isacharparenleft}{\kern0pt}x{\isacharparenright}{\kern0pt}\ {\isasyminter}\ {\isacharquery}{\kern0pt}ez\ {\isasymtimes}\ {\isacharquery}{\kern0pt}ez{\isachardoublequoteclose}\isanewline
\ \ \isacommand{fix}\isamarkupfalse%
\ y\isanewline
\ \ \isacommand{assume}\isamarkupfalse%
\ yr{\isacharcolon}{\kern0pt}{\isachardoublequoteopen}y\ {\isasymin}\ {\isacharquery}{\kern0pt}rr{\isachardoublequoteclose}\isanewline
\ \ \isacommand{with}\isamarkupfalse%
\ yr\ \isacommand{obtain}\isamarkupfalse%
\ a\ b\ \isakeyword{where}\ {\isadigit{1}}{\isacharcolon}{\kern0pt}{\isachardoublequoteopen}{\isasymlangle}a{\isacharcomma}{\kern0pt}b{\isasymrangle}\ {\isasymin}\ {\isacharquery}{\kern0pt}rr{\isachardoublequoteclose}\ {\isachardoublequoteopen}a\ {\isasymin}\ {\isacharquery}{\kern0pt}ez{\isachardoublequoteclose}\ {\isachardoublequoteopen}b\ {\isasymin}\ {\isacharquery}{\kern0pt}ez{\isachardoublequoteclose}\ {\isachardoublequoteopen}{\isasymlangle}a{\isacharcomma}{\kern0pt}b{\isasymrangle}\ {\isasymin}\ {\isacharquery}{\kern0pt}ec{\isacharparenleft}{\kern0pt}x{\isacharparenright}{\kern0pt}{\isachardoublequoteclose}\ {\isachardoublequoteopen}y{\isacharequal}{\kern0pt}{\isasymlangle}a{\isacharcomma}{\kern0pt}b{\isasymrangle}{\isachardoublequoteclose}\isanewline
\ \ \ \ \isacommand{by}\isamarkupfalse%
\ blast\isanewline
\ \ \isacommand{moreover}\isamarkupfalse%
\isanewline
\ \ \isacommand{from}\isamarkupfalse%
\ this\isanewline
\ \ \isacommand{have}\isamarkupfalse%
\ {\isachardoublequoteopen}a\ {\isasymin}\ domain{\isacharparenleft}{\kern0pt}b{\isacharparenright}{\kern0pt}{\isachardoublequoteclose}\ \isacommand{using}\isamarkupfalse%
\ edrelD\ \isacommand{by}\isamarkupfalse%
\ blast\isanewline
\ \ \isacommand{ultimately}\isamarkupfalse%
\isanewline
\ \ \isacommand{show}\isamarkupfalse%
\ {\isachardoublequoteopen}y\ {\isasymin}\ edrel{\isacharparenleft}{\kern0pt}eclose{\isacharparenleft}{\kern0pt}{\isacharbraceleft}{\kern0pt}z{\isacharbraceright}{\kern0pt}{\isacharparenright}{\kern0pt}{\isacharparenright}{\kern0pt}{\isachardoublequoteclose}\ \isacommand{by}\isamarkupfalse%
\ blast\isanewline
\isacommand{next}\isamarkupfalse%
\isanewline
\ \ \isacommand{let}\isamarkupfalse%
\ {\isacharquery}{\kern0pt}ec{\isacharequal}{\kern0pt}{\isachardoublequoteopen}{\isasymlambda}\ y\ {\isachardot}{\kern0pt}\ edrel{\isacharparenleft}{\kern0pt}eclose{\isacharparenleft}{\kern0pt}{\isacharbraceleft}{\kern0pt}y{\isacharbraceright}{\kern0pt}{\isacharparenright}{\kern0pt}{\isacharparenright}{\kern0pt}{\isachardoublequoteclose}\isanewline
\ \ \isacommand{let}\isamarkupfalse%
\ {\isacharquery}{\kern0pt}ez{\isacharequal}{\kern0pt}{\isachardoublequoteopen}eclose{\isacharparenleft}{\kern0pt}{\isacharbraceleft}{\kern0pt}z{\isacharbraceright}{\kern0pt}{\isacharparenright}{\kern0pt}{\isachardoublequoteclose}\isanewline
\ \ \isacommand{let}\isamarkupfalse%
\ {\isacharquery}{\kern0pt}rr{\isacharequal}{\kern0pt}{\isachardoublequoteopen}{\isacharquery}{\kern0pt}ec{\isacharparenleft}{\kern0pt}x{\isacharparenright}{\kern0pt}\ {\isasyminter}\ {\isacharquery}{\kern0pt}ez\ {\isasymtimes}\ {\isacharquery}{\kern0pt}ez{\isachardoublequoteclose}\isanewline
\ \ \isacommand{fix}\isamarkupfalse%
\ y\isanewline
\ \ \isacommand{assume}\isamarkupfalse%
\ yr{\isacharcolon}{\kern0pt}{\isachardoublequoteopen}y\ {\isasymin}\ edrel{\isacharparenleft}{\kern0pt}{\isacharquery}{\kern0pt}ez{\isacharparenright}{\kern0pt}{\isachardoublequoteclose}\isanewline
\ \ \isacommand{then}\isamarkupfalse%
\ \isacommand{obtain}\isamarkupfalse%
\ a\ b\ \isakeyword{where}\ {\isachardoublequoteopen}a\ {\isasymin}\ {\isacharquery}{\kern0pt}ez{\isachardoublequoteclose}\ {\isachardoublequoteopen}b\ {\isasymin}\ {\isacharquery}{\kern0pt}ez{\isachardoublequoteclose}\ {\isachardoublequoteopen}y{\isacharequal}{\kern0pt}{\isasymlangle}a{\isacharcomma}{\kern0pt}b{\isasymrangle}{\isachardoublequoteclose}\ {\isachardoublequoteopen}a\ {\isasymin}\ domain{\isacharparenleft}{\kern0pt}b{\isacharparenright}{\kern0pt}{\isachardoublequoteclose}\isanewline
\ \ \ \ \isacommand{using}\isamarkupfalse%
\ edrelD\ \isacommand{by}\isamarkupfalse%
\ blast\isanewline
\ \ \isacommand{moreover}\isamarkupfalse%
\isanewline
\ \ \isacommand{from}\isamarkupfalse%
\ this\ assms\isanewline
\ \ \isacommand{have}\isamarkupfalse%
\ {\isachardoublequoteopen}z\ {\isasymin}\ eclose{\isacharparenleft}{\kern0pt}x{\isacharparenright}{\kern0pt}{\isachardoublequoteclose}\ \isacommand{using}\isamarkupfalse%
\ in{\isacharunderscore}{\kern0pt}dom{\isacharunderscore}{\kern0pt}in{\isacharunderscore}{\kern0pt}eclose\ \isacommand{by}\isamarkupfalse%
\ simp\isanewline
\ \ \isacommand{moreover}\isamarkupfalse%
\isanewline
\ \ \isacommand{from}\isamarkupfalse%
\ assms\ calculation\isanewline
\ \ \isacommand{have}\isamarkupfalse%
\ {\isachardoublequoteopen}a\ {\isasymin}\ eclose{\isacharparenleft}{\kern0pt}{\isacharbraceleft}{\kern0pt}x{\isacharbraceright}{\kern0pt}{\isacharparenright}{\kern0pt}{\isachardoublequoteclose}\ {\isachardoublequoteopen}b\ {\isasymin}\ eclose{\isacharparenleft}{\kern0pt}{\isacharbraceleft}{\kern0pt}x{\isacharbraceright}{\kern0pt}{\isacharparenright}{\kern0pt}{\isachardoublequoteclose}\ \isacommand{using}\isamarkupfalse%
\ in{\isacharunderscore}{\kern0pt}eclose{\isacharunderscore}{\kern0pt}sing\ \isacommand{by}\isamarkupfalse%
\ simp{\isacharunderscore}{\kern0pt}all\isanewline
\ \ \isacommand{moreover}\isamarkupfalse%
\isanewline
\ \ \isacommand{from}\isamarkupfalse%
\ this\ {\isacartoucheopen}a{\isasymin}domain{\isacharparenleft}{\kern0pt}b{\isacharparenright}{\kern0pt}{\isacartoucheclose}\isanewline
\ \ \isacommand{have}\isamarkupfalse%
\ {\isachardoublequoteopen}{\isasymlangle}a{\isacharcomma}{\kern0pt}b{\isasymrangle}\ {\isasymin}\ edrel{\isacharparenleft}{\kern0pt}eclose{\isacharparenleft}{\kern0pt}{\isacharbraceleft}{\kern0pt}x{\isacharbraceright}{\kern0pt}{\isacharparenright}{\kern0pt}{\isacharparenright}{\kern0pt}{\isachardoublequoteclose}\ \isacommand{by}\isamarkupfalse%
\ blast\isanewline
\ \ \isacommand{ultimately}\isamarkupfalse%
\isanewline
\ \ \isacommand{show}\isamarkupfalse%
\ {\isachardoublequoteopen}y\ {\isasymin}\ {\isacharquery}{\kern0pt}rr{\isachardoublequoteclose}\ \isacommand{by}\isamarkupfalse%
\ simp\isanewline
\isacommand{qed}\isamarkupfalse%
%
\endisatagproof
{\isafoldproof}%
%
\isadelimproof
\isanewline
%
\endisadelimproof
\isanewline
\isacommand{lemma}\isamarkupfalse%
\ tr{\isacharunderscore}{\kern0pt}edrel{\isacharunderscore}{\kern0pt}subset\ {\isacharcolon}{\kern0pt}\isanewline
\ \ \isakeyword{assumes}\ {\isachardoublequoteopen}z\ {\isasymin}\ domain{\isacharparenleft}{\kern0pt}x{\isacharparenright}{\kern0pt}{\isachardoublequoteclose}\isanewline
\ \ \isakeyword{shows}\ \ \ {\isachardoublequoteopen}tr{\isacharunderscore}{\kern0pt}down{\isacharparenleft}{\kern0pt}edrel{\isacharparenleft}{\kern0pt}eclose{\isacharparenleft}{\kern0pt}{\isacharbraceleft}{\kern0pt}x{\isacharbraceright}{\kern0pt}{\isacharparenright}{\kern0pt}{\isacharparenright}{\kern0pt}{\isacharcomma}{\kern0pt}z{\isacharparenright}{\kern0pt}\ {\isasymsubseteq}\ eclose{\isacharparenleft}{\kern0pt}{\isacharbraceleft}{\kern0pt}z{\isacharbraceright}{\kern0pt}{\isacharparenright}{\kern0pt}{\isachardoublequoteclose}\isanewline
%
\isadelimproof
%
\endisadelimproof
%
\isatagproof
\isacommand{proof}\isamarkupfalse%
{\isacharparenleft}{\kern0pt}intro\ subsetI{\isacharparenright}{\kern0pt}\isanewline
\ \ \isacommand{let}\isamarkupfalse%
\ {\isacharquery}{\kern0pt}r{\isacharequal}{\kern0pt}{\isachardoublequoteopen}{\isasymlambda}\ x\ {\isachardot}{\kern0pt}\ edrel{\isacharparenleft}{\kern0pt}eclose{\isacharparenleft}{\kern0pt}{\isacharbraceleft}{\kern0pt}x{\isacharbraceright}{\kern0pt}{\isacharparenright}{\kern0pt}{\isacharparenright}{\kern0pt}{\isachardoublequoteclose}\isanewline
\ \ \isacommand{fix}\isamarkupfalse%
\ y\isanewline
\ \ \isacommand{assume}\isamarkupfalse%
\ \ {\isachardoublequoteopen}y\ {\isasymin}\ tr{\isacharunderscore}{\kern0pt}down{\isacharparenleft}{\kern0pt}{\isacharquery}{\kern0pt}r{\isacharparenleft}{\kern0pt}x{\isacharparenright}{\kern0pt}{\isacharcomma}{\kern0pt}z{\isacharparenright}{\kern0pt}{\isachardoublequoteclose}\isanewline
\ \ \isacommand{then}\isamarkupfalse%
\isanewline
\ \ \isacommand{have}\isamarkupfalse%
\ {\isachardoublequoteopen}{\isasymlangle}y{\isacharcomma}{\kern0pt}z{\isasymrangle}\ {\isasymin}\ {\isacharquery}{\kern0pt}r{\isacharparenleft}{\kern0pt}x{\isacharparenright}{\kern0pt}{\isacharcircum}{\kern0pt}{\isacharplus}{\kern0pt}{\isachardoublequoteclose}\ \isacommand{using}\isamarkupfalse%
\ tr{\isacharunderscore}{\kern0pt}downD\ \isacommand{by}\isamarkupfalse%
\ simp\isanewline
\ \ \isacommand{with}\isamarkupfalse%
\ assms\isanewline
\ \ \isacommand{show}\isamarkupfalse%
\ {\isachardoublequoteopen}y\ {\isasymin}\ eclose{\isacharparenleft}{\kern0pt}{\isacharbraceleft}{\kern0pt}z{\isacharbraceright}{\kern0pt}{\isacharparenright}{\kern0pt}{\isachardoublequoteclose}\ \isacommand{using}\isamarkupfalse%
\ tr{\isacharunderscore}{\kern0pt}edrel{\isacharunderscore}{\kern0pt}eclose\ eclose{\isacharunderscore}{\kern0pt}sing\ \isacommand{by}\isamarkupfalse%
\ simp\isanewline
\isacommand{qed}\isamarkupfalse%
%
\endisatagproof
{\isafoldproof}%
%
\isadelimproof
\isanewline
%
\endisadelimproof
\isanewline
\isanewline
\isacommand{context}\isamarkupfalse%
\ M{\isacharunderscore}{\kern0pt}ctm\isanewline
\isakeyword{begin}\isanewline
\isanewline
\isacommand{lemma}\isamarkupfalse%
\ upairM\ {\isacharcolon}{\kern0pt}\ {\isachardoublequoteopen}x\ {\isasymin}\ M\ {\isasymLongrightarrow}\ y\ {\isasymin}\ M\ {\isasymLongrightarrow}\ {\isacharbraceleft}{\kern0pt}x{\isacharcomma}{\kern0pt}y{\isacharbraceright}{\kern0pt}\ {\isasymin}\ M{\isachardoublequoteclose}\isanewline
%
\isadelimproof
\ \ %
\endisadelimproof
%
\isatagproof
\isacommand{by}\isamarkupfalse%
\ {\isacharparenleft}{\kern0pt}simp\ flip{\isacharcolon}{\kern0pt}\ setclass{\isacharunderscore}{\kern0pt}iff{\isacharparenright}{\kern0pt}%
\endisatagproof
{\isafoldproof}%
%
\isadelimproof
\isanewline
%
\endisadelimproof
\isanewline
\isacommand{lemma}\isamarkupfalse%
\ singletonM\ {\isacharcolon}{\kern0pt}\ {\isachardoublequoteopen}a\ {\isasymin}\ M\ {\isasymLongrightarrow}\ {\isacharbraceleft}{\kern0pt}a{\isacharbraceright}{\kern0pt}\ {\isasymin}\ M{\isachardoublequoteclose}\isanewline
%
\isadelimproof
\ \ %
\endisadelimproof
%
\isatagproof
\isacommand{by}\isamarkupfalse%
\ {\isacharparenleft}{\kern0pt}simp\ flip{\isacharcolon}{\kern0pt}\ setclass{\isacharunderscore}{\kern0pt}iff{\isacharparenright}{\kern0pt}%
\endisatagproof
{\isafoldproof}%
%
\isadelimproof
\isanewline
%
\endisadelimproof
\isanewline
\isacommand{lemma}\isamarkupfalse%
\ Rep{\isacharunderscore}{\kern0pt}simp\ {\isacharcolon}{\kern0pt}\ {\isachardoublequoteopen}Replace{\isacharparenleft}{\kern0pt}u{\isacharcomma}{\kern0pt}{\isasymlambda}\ y\ z\ {\isachardot}{\kern0pt}\ z\ {\isacharequal}{\kern0pt}\ f{\isacharparenleft}{\kern0pt}y{\isacharparenright}{\kern0pt}{\isacharparenright}{\kern0pt}\ {\isacharequal}{\kern0pt}\ {\isacharbraceleft}{\kern0pt}\ f{\isacharparenleft}{\kern0pt}y{\isacharparenright}{\kern0pt}\ {\isachardot}{\kern0pt}\ y\ {\isasymin}\ u{\isacharbraceright}{\kern0pt}{\isachardoublequoteclose}\isanewline
%
\isadelimproof
\ \ %
\endisadelimproof
%
\isatagproof
\isacommand{by}\isamarkupfalse%
{\isacharparenleft}{\kern0pt}auto{\isacharparenright}{\kern0pt}%
\endisatagproof
{\isafoldproof}%
%
\isadelimproof
\isanewline
%
\endisadelimproof
\isanewline
\isacommand{end}\isamarkupfalse%
%
\isadelimdocument
%
\endisadelimdocument
%
\isatagdocument
%
\isamarkupsubsection{Values and check-names%
}
\isamarkuptrue%
%
\endisatagdocument
{\isafolddocument}%
%
\isadelimdocument
%
\endisadelimdocument
\isacommand{context}\isamarkupfalse%
\ forcing{\isacharunderscore}{\kern0pt}data\isanewline
\isakeyword{begin}\isanewline
\isanewline
\isacommand{definition}\isamarkupfalse%
\isanewline
\ \ Hcheck\ {\isacharcolon}{\kern0pt}{\isacharcolon}{\kern0pt}\ {\isachardoublequoteopen}{\isacharbrackleft}{\kern0pt}i{\isacharcomma}{\kern0pt}i{\isacharbrackright}{\kern0pt}\ {\isasymRightarrow}\ i{\isachardoublequoteclose}\ \isakeyword{where}\isanewline
\ \ {\isachardoublequoteopen}Hcheck{\isacharparenleft}{\kern0pt}z{\isacharcomma}{\kern0pt}f{\isacharparenright}{\kern0pt}\ \ {\isasymequiv}\ {\isacharbraceleft}{\kern0pt}\ {\isasymlangle}f{\isacharbackquote}{\kern0pt}y{\isacharcomma}{\kern0pt}one{\isasymrangle}\ {\isachardot}{\kern0pt}\ y\ {\isasymin}\ z{\isacharbraceright}{\kern0pt}{\isachardoublequoteclose}\isanewline
\isanewline
\isacommand{definition}\isamarkupfalse%
\isanewline
\ \ check\ {\isacharcolon}{\kern0pt}{\isacharcolon}{\kern0pt}\ {\isachardoublequoteopen}i\ {\isasymRightarrow}\ i{\isachardoublequoteclose}\ \isakeyword{where}\isanewline
\ \ {\isachardoublequoteopen}check{\isacharparenleft}{\kern0pt}x{\isacharparenright}{\kern0pt}\ {\isasymequiv}\ transrec{\isacharparenleft}{\kern0pt}x\ {\isacharcomma}{\kern0pt}\ Hcheck{\isacharparenright}{\kern0pt}{\isachardoublequoteclose}\isanewline
\isanewline
\isacommand{lemma}\isamarkupfalse%
\ checkD{\isacharcolon}{\kern0pt}\isanewline
\ \ {\isachardoublequoteopen}check{\isacharparenleft}{\kern0pt}x{\isacharparenright}{\kern0pt}\ {\isacharequal}{\kern0pt}\ \ wfrec{\isacharparenleft}{\kern0pt}Memrel{\isacharparenleft}{\kern0pt}eclose{\isacharparenleft}{\kern0pt}{\isacharbraceleft}{\kern0pt}x{\isacharbraceright}{\kern0pt}{\isacharparenright}{\kern0pt}{\isacharparenright}{\kern0pt}{\isacharcomma}{\kern0pt}\ x{\isacharcomma}{\kern0pt}\ Hcheck{\isacharparenright}{\kern0pt}{\isachardoublequoteclose}\isanewline
%
\isadelimproof
\ \ %
\endisadelimproof
%
\isatagproof
\isacommand{unfolding}\isamarkupfalse%
\ check{\isacharunderscore}{\kern0pt}def\ transrec{\isacharunderscore}{\kern0pt}def\ \isacommand{{\isachardot}{\kern0pt}{\isachardot}{\kern0pt}}\isamarkupfalse%
%
\endisatagproof
{\isafoldproof}%
%
\isadelimproof
\isanewline
%
\endisadelimproof
\isanewline
\isacommand{definition}\isamarkupfalse%
\isanewline
\ \ rcheck\ {\isacharcolon}{\kern0pt}{\isacharcolon}{\kern0pt}\ {\isachardoublequoteopen}i\ {\isasymRightarrow}\ i{\isachardoublequoteclose}\ \isakeyword{where}\isanewline
\ \ {\isachardoublequoteopen}rcheck{\isacharparenleft}{\kern0pt}x{\isacharparenright}{\kern0pt}\ {\isasymequiv}\ Memrel{\isacharparenleft}{\kern0pt}eclose{\isacharparenleft}{\kern0pt}{\isacharbraceleft}{\kern0pt}x{\isacharbraceright}{\kern0pt}{\isacharparenright}{\kern0pt}{\isacharparenright}{\kern0pt}{\isacharcircum}{\kern0pt}{\isacharplus}{\kern0pt}{\isachardoublequoteclose}\isanewline
\isanewline
\isanewline
\isacommand{lemma}\isamarkupfalse%
\ Hcheck{\isacharunderscore}{\kern0pt}trancl{\isacharcolon}{\kern0pt}{\isachardoublequoteopen}Hcheck{\isacharparenleft}{\kern0pt}y{\isacharcomma}{\kern0pt}\ restrict{\isacharparenleft}{\kern0pt}f{\isacharcomma}{\kern0pt}Memrel{\isacharparenleft}{\kern0pt}eclose{\isacharparenleft}{\kern0pt}{\isacharbraceleft}{\kern0pt}x{\isacharbraceright}{\kern0pt}{\isacharparenright}{\kern0pt}{\isacharparenright}{\kern0pt}{\isacharminus}{\kern0pt}{\isacharbackquote}{\kern0pt}{\isacharbackquote}{\kern0pt}{\isacharbraceleft}{\kern0pt}y{\isacharbraceright}{\kern0pt}{\isacharparenright}{\kern0pt}{\isacharparenright}{\kern0pt}\isanewline
\ \ \ \ \ \ \ \ \ \ \ \ \ \ \ \ \ \ \ {\isacharequal}{\kern0pt}\ Hcheck{\isacharparenleft}{\kern0pt}y{\isacharcomma}{\kern0pt}\ restrict{\isacharparenleft}{\kern0pt}f{\isacharcomma}{\kern0pt}{\isacharparenleft}{\kern0pt}Memrel{\isacharparenleft}{\kern0pt}eclose{\isacharparenleft}{\kern0pt}{\isacharbraceleft}{\kern0pt}x{\isacharbraceright}{\kern0pt}{\isacharparenright}{\kern0pt}{\isacharparenright}{\kern0pt}{\isacharcircum}{\kern0pt}{\isacharplus}{\kern0pt}{\isacharparenright}{\kern0pt}{\isacharminus}{\kern0pt}{\isacharbackquote}{\kern0pt}{\isacharbackquote}{\kern0pt}{\isacharbraceleft}{\kern0pt}y{\isacharbraceright}{\kern0pt}{\isacharparenright}{\kern0pt}{\isacharparenright}{\kern0pt}{\isachardoublequoteclose}\isanewline
%
\isadelimproof
\ \ %
\endisadelimproof
%
\isatagproof
\isacommand{unfolding}\isamarkupfalse%
\ Hcheck{\isacharunderscore}{\kern0pt}def\isanewline
\ \ \isacommand{using}\isamarkupfalse%
\ restrict{\isacharunderscore}{\kern0pt}trans{\isacharunderscore}{\kern0pt}eq\ \isacommand{by}\isamarkupfalse%
\ simp%
\endisatagproof
{\isafoldproof}%
%
\isadelimproof
\isanewline
%
\endisadelimproof
\isanewline
\isacommand{lemma}\isamarkupfalse%
\ check{\isacharunderscore}{\kern0pt}trancl{\isacharcolon}{\kern0pt}\ {\isachardoublequoteopen}check{\isacharparenleft}{\kern0pt}x{\isacharparenright}{\kern0pt}\ {\isacharequal}{\kern0pt}\ wfrec{\isacharparenleft}{\kern0pt}rcheck{\isacharparenleft}{\kern0pt}x{\isacharparenright}{\kern0pt}{\isacharcomma}{\kern0pt}\ x{\isacharcomma}{\kern0pt}\ Hcheck{\isacharparenright}{\kern0pt}{\isachardoublequoteclose}\isanewline
%
\isadelimproof
\ \ %
\endisadelimproof
%
\isatagproof
\isacommand{using}\isamarkupfalse%
\ checkD\ wf{\isacharunderscore}{\kern0pt}eq{\isacharunderscore}{\kern0pt}trancl\ Hcheck{\isacharunderscore}{\kern0pt}trancl\ \isacommand{unfolding}\isamarkupfalse%
\ rcheck{\isacharunderscore}{\kern0pt}def\ \isacommand{by}\isamarkupfalse%
\ simp%
\endisatagproof
{\isafoldproof}%
%
\isadelimproof
\isanewline
%
\endisadelimproof
\isanewline
\isanewline
\isacommand{lemma}\isamarkupfalse%
\ rcheck{\isacharunderscore}{\kern0pt}in{\isacharunderscore}{\kern0pt}M\ {\isacharcolon}{\kern0pt}\isanewline
\ \ {\isachardoublequoteopen}x\ {\isasymin}\ M\ {\isasymLongrightarrow}\ rcheck{\isacharparenleft}{\kern0pt}x{\isacharparenright}{\kern0pt}\ {\isasymin}\ M{\isachardoublequoteclose}\isanewline
%
\isadelimproof
\ \ %
\endisadelimproof
%
\isatagproof
\isacommand{unfolding}\isamarkupfalse%
\ rcheck{\isacharunderscore}{\kern0pt}def\ \isacommand{by}\isamarkupfalse%
\ {\isacharparenleft}{\kern0pt}simp\ flip{\isacharcolon}{\kern0pt}\ setclass{\isacharunderscore}{\kern0pt}iff{\isacharparenright}{\kern0pt}%
\endisatagproof
{\isafoldproof}%
%
\isadelimproof
\isanewline
%
\endisadelimproof
\isanewline
\isanewline
\isacommand{lemma}\isamarkupfalse%
\ \ aux{\isacharunderscore}{\kern0pt}def{\isacharunderscore}{\kern0pt}check{\isacharcolon}{\kern0pt}\ {\isachardoublequoteopen}x\ {\isasymin}\ y\ {\isasymLongrightarrow}\isanewline
\ \ wfrec{\isacharparenleft}{\kern0pt}Memrel{\isacharparenleft}{\kern0pt}eclose{\isacharparenleft}{\kern0pt}{\isacharbraceleft}{\kern0pt}y{\isacharbraceright}{\kern0pt}{\isacharparenright}{\kern0pt}{\isacharparenright}{\kern0pt}{\isacharcomma}{\kern0pt}\ x{\isacharcomma}{\kern0pt}\ Hcheck{\isacharparenright}{\kern0pt}\ {\isacharequal}{\kern0pt}\isanewline
\ \ wfrec{\isacharparenleft}{\kern0pt}Memrel{\isacharparenleft}{\kern0pt}eclose{\isacharparenleft}{\kern0pt}{\isacharbraceleft}{\kern0pt}x{\isacharbraceright}{\kern0pt}{\isacharparenright}{\kern0pt}{\isacharparenright}{\kern0pt}{\isacharcomma}{\kern0pt}\ x{\isacharcomma}{\kern0pt}\ Hcheck{\isacharparenright}{\kern0pt}{\isachardoublequoteclose}\isanewline
%
\isadelimproof
\ \ %
\endisadelimproof
%
\isatagproof
\isacommand{by}\isamarkupfalse%
\ {\isacharparenleft}{\kern0pt}rule\ wfrec{\isacharunderscore}{\kern0pt}eclose{\isacharunderscore}{\kern0pt}eq{\isacharcomma}{\kern0pt}auto\ simp\ add{\isacharcolon}{\kern0pt}\ arg{\isacharunderscore}{\kern0pt}into{\isacharunderscore}{\kern0pt}eclose\ eclose{\isacharunderscore}{\kern0pt}sing{\isacharparenright}{\kern0pt}%
\endisatagproof
{\isafoldproof}%
%
\isadelimproof
\isanewline
%
\endisadelimproof
\isanewline
\isacommand{lemma}\isamarkupfalse%
\ def{\isacharunderscore}{\kern0pt}check\ {\isacharcolon}{\kern0pt}\ {\isachardoublequoteopen}check{\isacharparenleft}{\kern0pt}y{\isacharparenright}{\kern0pt}\ {\isacharequal}{\kern0pt}\ {\isacharbraceleft}{\kern0pt}\ {\isasymlangle}check{\isacharparenleft}{\kern0pt}w{\isacharparenright}{\kern0pt}{\isacharcomma}{\kern0pt}one{\isasymrangle}\ {\isachardot}{\kern0pt}\ w\ {\isasymin}\ y{\isacharbraceright}{\kern0pt}{\isachardoublequoteclose}\isanewline
%
\isadelimproof
%
\endisadelimproof
%
\isatagproof
\isacommand{proof}\isamarkupfalse%
\ {\isacharminus}{\kern0pt}\isanewline
\ \ \isacommand{let}\isamarkupfalse%
\isanewline
\ \ \ \ {\isacharquery}{\kern0pt}r{\isacharequal}{\kern0pt}{\isachardoublequoteopen}{\isasymlambda}y{\isachardot}{\kern0pt}\ Memrel{\isacharparenleft}{\kern0pt}eclose{\isacharparenleft}{\kern0pt}{\isacharbraceleft}{\kern0pt}y{\isacharbraceright}{\kern0pt}{\isacharparenright}{\kern0pt}{\isacharparenright}{\kern0pt}{\isachardoublequoteclose}\isanewline
\ \ \isacommand{have}\isamarkupfalse%
\ wfr{\isacharcolon}{\kern0pt}\ \ \ {\isachardoublequoteopen}{\isasymforall}w\ {\isachardot}{\kern0pt}\ wf{\isacharparenleft}{\kern0pt}{\isacharquery}{\kern0pt}r{\isacharparenleft}{\kern0pt}w{\isacharparenright}{\kern0pt}{\isacharparenright}{\kern0pt}{\isachardoublequoteclose}\ \isanewline
\ \ \ \ \isacommand{using}\isamarkupfalse%
\ wf{\isacharunderscore}{\kern0pt}Memrel\ \isacommand{{\isachardot}{\kern0pt}{\isachardot}{\kern0pt}}\isamarkupfalse%
\isanewline
\ \ \isacommand{then}\isamarkupfalse%
\ \isanewline
\ \ \isacommand{have}\isamarkupfalse%
\ {\isachardoublequoteopen}check{\isacharparenleft}{\kern0pt}y{\isacharparenright}{\kern0pt}{\isacharequal}{\kern0pt}\ Hcheck{\isacharparenleft}{\kern0pt}\ y{\isacharcomma}{\kern0pt}\ {\isasymlambda}x{\isasymin}{\isacharquery}{\kern0pt}r{\isacharparenleft}{\kern0pt}y{\isacharparenright}{\kern0pt}\ {\isacharminus}{\kern0pt}{\isacharbackquote}{\kern0pt}{\isacharbackquote}{\kern0pt}\ {\isacharbraceleft}{\kern0pt}y{\isacharbraceright}{\kern0pt}{\isachardot}{\kern0pt}\ wfrec{\isacharparenleft}{\kern0pt}{\isacharquery}{\kern0pt}r{\isacharparenleft}{\kern0pt}y{\isacharparenright}{\kern0pt}{\isacharcomma}{\kern0pt}\ x{\isacharcomma}{\kern0pt}\ Hcheck{\isacharparenright}{\kern0pt}{\isacharparenright}{\kern0pt}{\isachardoublequoteclose}\isanewline
\ \ \ \ \isacommand{using}\isamarkupfalse%
\ wfrec{\isacharbrackleft}{\kern0pt}of\ {\isachardoublequoteopen}{\isacharquery}{\kern0pt}r{\isacharparenleft}{\kern0pt}y{\isacharparenright}{\kern0pt}{\isachardoublequoteclose}\ y\ {\isachardoublequoteopen}Hcheck{\isachardoublequoteclose}{\isacharbrackright}{\kern0pt}\ checkD\ \isacommand{by}\isamarkupfalse%
\ simp\isanewline
\ \ \isacommand{also}\isamarkupfalse%
\ \isanewline
\ \ \isacommand{have}\isamarkupfalse%
\ {\isachardoublequoteopen}\ {\isachardot}{\kern0pt}{\isachardot}{\kern0pt}{\isachardot}{\kern0pt}\ {\isacharequal}{\kern0pt}\ Hcheck{\isacharparenleft}{\kern0pt}\ y{\isacharcomma}{\kern0pt}\ {\isasymlambda}x{\isasymin}y{\isachardot}{\kern0pt}\ wfrec{\isacharparenleft}{\kern0pt}{\isacharquery}{\kern0pt}r{\isacharparenleft}{\kern0pt}y{\isacharparenright}{\kern0pt}{\isacharcomma}{\kern0pt}\ x{\isacharcomma}{\kern0pt}\ Hcheck{\isacharparenright}{\kern0pt}{\isacharparenright}{\kern0pt}{\isachardoublequoteclose}\isanewline
\ \ \ \ \isacommand{using}\isamarkupfalse%
\ under{\isacharunderscore}{\kern0pt}Memrel{\isacharunderscore}{\kern0pt}eclose\ arg{\isacharunderscore}{\kern0pt}into{\isacharunderscore}{\kern0pt}eclose\ \isacommand{by}\isamarkupfalse%
\ simp\isanewline
\ \ \isacommand{also}\isamarkupfalse%
\ \isanewline
\ \ \isacommand{have}\isamarkupfalse%
\ {\isachardoublequoteopen}\ {\isachardot}{\kern0pt}{\isachardot}{\kern0pt}{\isachardot}{\kern0pt}\ {\isacharequal}{\kern0pt}\ Hcheck{\isacharparenleft}{\kern0pt}\ y{\isacharcomma}{\kern0pt}\ {\isasymlambda}x{\isasymin}y{\isachardot}{\kern0pt}\ check{\isacharparenleft}{\kern0pt}x{\isacharparenright}{\kern0pt}{\isacharparenright}{\kern0pt}{\isachardoublequoteclose}\isanewline
\ \ \ \ \isacommand{using}\isamarkupfalse%
\ aux{\isacharunderscore}{\kern0pt}def{\isacharunderscore}{\kern0pt}check\ checkD\ \isacommand{by}\isamarkupfalse%
\ simp\isanewline
\ \ \isacommand{finally}\isamarkupfalse%
\ \isacommand{show}\isamarkupfalse%
\ {\isacharquery}{\kern0pt}thesis\ \isacommand{using}\isamarkupfalse%
\ Hcheck{\isacharunderscore}{\kern0pt}def\ \isacommand{by}\isamarkupfalse%
\ simp\isanewline
\isacommand{qed}\isamarkupfalse%
%
\endisatagproof
{\isafoldproof}%
%
\isadelimproof
\isanewline
%
\endisadelimproof
\isanewline
\isanewline
\isacommand{lemma}\isamarkupfalse%
\ def{\isacharunderscore}{\kern0pt}checkS\ {\isacharcolon}{\kern0pt}\isanewline
\ \ \isakeyword{fixes}\ n\isanewline
\ \ \isakeyword{assumes}\ {\isachardoublequoteopen}n\ {\isasymin}\ nat{\isachardoublequoteclose}\isanewline
\ \ \isakeyword{shows}\ {\isachardoublequoteopen}check{\isacharparenleft}{\kern0pt}succ{\isacharparenleft}{\kern0pt}n{\isacharparenright}{\kern0pt}{\isacharparenright}{\kern0pt}\ {\isacharequal}{\kern0pt}\ check{\isacharparenleft}{\kern0pt}n{\isacharparenright}{\kern0pt}\ {\isasymunion}\ {\isacharbraceleft}{\kern0pt}{\isasymlangle}check{\isacharparenleft}{\kern0pt}n{\isacharparenright}{\kern0pt}{\isacharcomma}{\kern0pt}one{\isasymrangle}{\isacharbraceright}{\kern0pt}{\isachardoublequoteclose}\isanewline
%
\isadelimproof
%
\endisadelimproof
%
\isatagproof
\isacommand{proof}\isamarkupfalse%
\ {\isacharminus}{\kern0pt}\isanewline
\ \ \isacommand{have}\isamarkupfalse%
\ {\isachardoublequoteopen}check{\isacharparenleft}{\kern0pt}succ{\isacharparenleft}{\kern0pt}n{\isacharparenright}{\kern0pt}{\isacharparenright}{\kern0pt}\ {\isacharequal}{\kern0pt}\ {\isacharbraceleft}{\kern0pt}{\isasymlangle}check{\isacharparenleft}{\kern0pt}i{\isacharparenright}{\kern0pt}{\isacharcomma}{\kern0pt}one{\isasymrangle}\ {\isachardot}{\kern0pt}\ i\ {\isasymin}\ succ{\isacharparenleft}{\kern0pt}n{\isacharparenright}{\kern0pt}{\isacharbraceright}{\kern0pt}\ {\isachardoublequoteclose}\isanewline
\ \ \ \ \isacommand{using}\isamarkupfalse%
\ def{\isacharunderscore}{\kern0pt}check\ \isacommand{by}\isamarkupfalse%
\ blast\isanewline
\ \ \isacommand{also}\isamarkupfalse%
\ \isacommand{have}\isamarkupfalse%
\ {\isachardoublequoteopen}{\isachardot}{\kern0pt}{\isachardot}{\kern0pt}{\isachardot}{\kern0pt}\ {\isacharequal}{\kern0pt}\ {\isacharbraceleft}{\kern0pt}{\isasymlangle}check{\isacharparenleft}{\kern0pt}i{\isacharparenright}{\kern0pt}{\isacharcomma}{\kern0pt}one{\isasymrangle}\ {\isachardot}{\kern0pt}\ i\ {\isasymin}\ n{\isacharbraceright}{\kern0pt}\ {\isasymunion}\ {\isacharbraceleft}{\kern0pt}{\isasymlangle}check{\isacharparenleft}{\kern0pt}n{\isacharparenright}{\kern0pt}{\isacharcomma}{\kern0pt}one{\isasymrangle}{\isacharbraceright}{\kern0pt}{\isachardoublequoteclose}\isanewline
\ \ \ \ \isacommand{by}\isamarkupfalse%
\ blast\isanewline
\ \ \isacommand{also}\isamarkupfalse%
\ \isacommand{have}\isamarkupfalse%
\ {\isachardoublequoteopen}{\isachardot}{\kern0pt}{\isachardot}{\kern0pt}{\isachardot}{\kern0pt}\ {\isacharequal}{\kern0pt}\ check{\isacharparenleft}{\kern0pt}n{\isacharparenright}{\kern0pt}\ {\isasymunion}\ {\isacharbraceleft}{\kern0pt}{\isasymlangle}check{\isacharparenleft}{\kern0pt}n{\isacharparenright}{\kern0pt}{\isacharcomma}{\kern0pt}one{\isasymrangle}{\isacharbraceright}{\kern0pt}{\isachardoublequoteclose}\isanewline
\ \ \ \ \isacommand{using}\isamarkupfalse%
\ def{\isacharunderscore}{\kern0pt}check{\isacharbrackleft}{\kern0pt}of\ n{\isacharcomma}{\kern0pt}symmetric{\isacharbrackright}{\kern0pt}\ \isacommand{by}\isamarkupfalse%
\ simp\isanewline
\ \ \isacommand{finally}\isamarkupfalse%
\ \isacommand{show}\isamarkupfalse%
\ {\isacharquery}{\kern0pt}thesis\ \isacommand{{\isachardot}{\kern0pt}}\isamarkupfalse%
\isanewline
\isacommand{qed}\isamarkupfalse%
%
\endisatagproof
{\isafoldproof}%
%
\isadelimproof
\isanewline
%
\endisadelimproof
\isanewline
\isacommand{lemma}\isamarkupfalse%
\ field{\isacharunderscore}{\kern0pt}Memrel{\isadigit{2}}\ {\isacharcolon}{\kern0pt}\isanewline
\ \ \isakeyword{assumes}\ {\isachardoublequoteopen}x\ {\isasymin}\ M{\isachardoublequoteclose}\isanewline
\ \ \isakeyword{shows}\ {\isachardoublequoteopen}field{\isacharparenleft}{\kern0pt}Memrel{\isacharparenleft}{\kern0pt}eclose{\isacharparenleft}{\kern0pt}{\isacharbraceleft}{\kern0pt}x{\isacharbraceright}{\kern0pt}{\isacharparenright}{\kern0pt}{\isacharparenright}{\kern0pt}{\isacharparenright}{\kern0pt}\ {\isasymsubseteq}\ M{\isachardoublequoteclose}\isanewline
%
\isadelimproof
%
\endisadelimproof
%
\isatagproof
\isacommand{proof}\isamarkupfalse%
\ {\isacharminus}{\kern0pt}\isanewline
\ \ \isacommand{have}\isamarkupfalse%
\ {\isachardoublequoteopen}field{\isacharparenleft}{\kern0pt}Memrel{\isacharparenleft}{\kern0pt}eclose{\isacharparenleft}{\kern0pt}{\isacharbraceleft}{\kern0pt}x{\isacharbraceright}{\kern0pt}{\isacharparenright}{\kern0pt}{\isacharparenright}{\kern0pt}{\isacharparenright}{\kern0pt}\ {\isasymsubseteq}\ eclose{\isacharparenleft}{\kern0pt}{\isacharbraceleft}{\kern0pt}x{\isacharbraceright}{\kern0pt}{\isacharparenright}{\kern0pt}{\isachardoublequoteclose}\ {\isachardoublequoteopen}eclose{\isacharparenleft}{\kern0pt}{\isacharbraceleft}{\kern0pt}x{\isacharbraceright}{\kern0pt}{\isacharparenright}{\kern0pt}\ {\isasymsubseteq}\ M{\isachardoublequoteclose}\isanewline
\ \ \ \ \isacommand{using}\isamarkupfalse%
\ Ordinal{\isachardot}{\kern0pt}Memrel{\isacharunderscore}{\kern0pt}type\ field{\isacharunderscore}{\kern0pt}rel{\isacharunderscore}{\kern0pt}subset\ assms\ eclose{\isacharunderscore}{\kern0pt}least{\isacharbrackleft}{\kern0pt}OF\ trans{\isacharunderscore}{\kern0pt}M{\isacharbrackright}{\kern0pt}\ \isacommand{by}\isamarkupfalse%
\ auto\isanewline
\ \ \isacommand{then}\isamarkupfalse%
\isanewline
\ \ \isacommand{show}\isamarkupfalse%
\ {\isacharquery}{\kern0pt}thesis\ \isacommand{using}\isamarkupfalse%
\ subset{\isacharunderscore}{\kern0pt}trans\ \isacommand{by}\isamarkupfalse%
\ simp\isanewline
\isacommand{qed}\isamarkupfalse%
%
\endisatagproof
{\isafoldproof}%
%
\isadelimproof
\isanewline
%
\endisadelimproof
\isanewline
\isacommand{definition}\isamarkupfalse%
\isanewline
\ \ Hv\ {\isacharcolon}{\kern0pt}{\isacharcolon}{\kern0pt}\ {\isachardoublequoteopen}i{\isasymRightarrow}i{\isasymRightarrow}i{\isasymRightarrow}i{\isachardoublequoteclose}\ \isakeyword{where}\isanewline
\ \ {\isachardoublequoteopen}Hv{\isacharparenleft}{\kern0pt}G{\isacharcomma}{\kern0pt}x{\isacharcomma}{\kern0pt}f{\isacharparenright}{\kern0pt}\ {\isasymequiv}\ {\isacharbraceleft}{\kern0pt}\ f{\isacharbackquote}{\kern0pt}y\ {\isachardot}{\kern0pt}{\isachardot}{\kern0pt}\ y{\isasymin}\ domain{\isacharparenleft}{\kern0pt}x{\isacharparenright}{\kern0pt}{\isacharcomma}{\kern0pt}\ {\isasymexists}p{\isasymin}P{\isachardot}{\kern0pt}\ {\isasymlangle}y{\isacharcomma}{\kern0pt}p{\isasymrangle}\ {\isasymin}\ x\ {\isasymand}\ p\ {\isasymin}\ G\ {\isacharbraceright}{\kern0pt}{\isachardoublequoteclose}%
\begin{isamarkuptext}%
The funcion \isa{val} interprets a name in \isa{M}
according to a (generic) filter \isa{G}. Note the definition
in terms of the well-founded recursor.%
\end{isamarkuptext}\isamarkuptrue%
\isacommand{definition}\isamarkupfalse%
\isanewline
\ \ val\ {\isacharcolon}{\kern0pt}{\isacharcolon}{\kern0pt}\ {\isachardoublequoteopen}i{\isasymRightarrow}i{\isasymRightarrow}i{\isachardoublequoteclose}\ \isakeyword{where}\isanewline
\ \ {\isachardoublequoteopen}val{\isacharparenleft}{\kern0pt}G{\isacharcomma}{\kern0pt}{\isasymtau}{\isacharparenright}{\kern0pt}\ {\isasymequiv}\ wfrec{\isacharparenleft}{\kern0pt}edrel{\isacharparenleft}{\kern0pt}eclose{\isacharparenleft}{\kern0pt}{\isacharbraceleft}{\kern0pt}{\isasymtau}{\isacharbraceright}{\kern0pt}{\isacharparenright}{\kern0pt}{\isacharparenright}{\kern0pt}{\isacharcomma}{\kern0pt}\ {\isasymtau}\ {\isacharcomma}{\kern0pt}Hv{\isacharparenleft}{\kern0pt}G{\isacharparenright}{\kern0pt}{\isacharparenright}{\kern0pt}{\isachardoublequoteclose}\isanewline
\isanewline
\isacommand{lemma}\isamarkupfalse%
\ aux{\isacharunderscore}{\kern0pt}def{\isacharunderscore}{\kern0pt}val{\isacharcolon}{\kern0pt}\isanewline
\ \ \isakeyword{assumes}\ {\isachardoublequoteopen}z\ {\isasymin}\ domain{\isacharparenleft}{\kern0pt}x{\isacharparenright}{\kern0pt}{\isachardoublequoteclose}\isanewline
\ \ \isakeyword{shows}\ {\isachardoublequoteopen}wfrec{\isacharparenleft}{\kern0pt}edrel{\isacharparenleft}{\kern0pt}eclose{\isacharparenleft}{\kern0pt}{\isacharbraceleft}{\kern0pt}x{\isacharbraceright}{\kern0pt}{\isacharparenright}{\kern0pt}{\isacharparenright}{\kern0pt}{\isacharcomma}{\kern0pt}z{\isacharcomma}{\kern0pt}Hv{\isacharparenleft}{\kern0pt}G{\isacharparenright}{\kern0pt}{\isacharparenright}{\kern0pt}\ {\isacharequal}{\kern0pt}\ wfrec{\isacharparenleft}{\kern0pt}edrel{\isacharparenleft}{\kern0pt}eclose{\isacharparenleft}{\kern0pt}{\isacharbraceleft}{\kern0pt}z{\isacharbraceright}{\kern0pt}{\isacharparenright}{\kern0pt}{\isacharparenright}{\kern0pt}{\isacharcomma}{\kern0pt}z{\isacharcomma}{\kern0pt}Hv{\isacharparenleft}{\kern0pt}G{\isacharparenright}{\kern0pt}{\isacharparenright}{\kern0pt}{\isachardoublequoteclose}\isanewline
%
\isadelimproof
%
\endisadelimproof
%
\isatagproof
\isacommand{proof}\isamarkupfalse%
\ {\isacharminus}{\kern0pt}\isanewline
\ \ \isacommand{let}\isamarkupfalse%
\ {\isacharquery}{\kern0pt}r{\isacharequal}{\kern0pt}{\isachardoublequoteopen}{\isasymlambda}x\ {\isachardot}{\kern0pt}\ edrel{\isacharparenleft}{\kern0pt}eclose{\isacharparenleft}{\kern0pt}{\isacharbraceleft}{\kern0pt}x{\isacharbraceright}{\kern0pt}{\isacharparenright}{\kern0pt}{\isacharparenright}{\kern0pt}{\isachardoublequoteclose}\isanewline
\ \ \isacommand{have}\isamarkupfalse%
\ {\isachardoublequoteopen}z{\isasymin}eclose{\isacharparenleft}{\kern0pt}{\isacharbraceleft}{\kern0pt}z{\isacharbraceright}{\kern0pt}{\isacharparenright}{\kern0pt}{\isachardoublequoteclose}\ \isacommand{using}\isamarkupfalse%
\ arg{\isacharunderscore}{\kern0pt}in{\isacharunderscore}{\kern0pt}eclose{\isacharunderscore}{\kern0pt}sing\ \isacommand{{\isachardot}{\kern0pt}}\isamarkupfalse%
\isanewline
\ \ \isacommand{moreover}\isamarkupfalse%
\isanewline
\ \ \isacommand{have}\isamarkupfalse%
\ {\isachardoublequoteopen}relation{\isacharparenleft}{\kern0pt}{\isacharquery}{\kern0pt}r{\isacharparenleft}{\kern0pt}x{\isacharparenright}{\kern0pt}{\isacharparenright}{\kern0pt}{\isachardoublequoteclose}\ \isacommand{using}\isamarkupfalse%
\ relation{\isacharunderscore}{\kern0pt}edrel\ \isacommand{{\isachardot}{\kern0pt}}\isamarkupfalse%
\isanewline
\ \ \isacommand{moreover}\isamarkupfalse%
\isanewline
\ \ \isacommand{have}\isamarkupfalse%
\ {\isachardoublequoteopen}wf{\isacharparenleft}{\kern0pt}{\isacharquery}{\kern0pt}r{\isacharparenleft}{\kern0pt}x{\isacharparenright}{\kern0pt}{\isacharparenright}{\kern0pt}{\isachardoublequoteclose}\ \isacommand{using}\isamarkupfalse%
\ wf{\isacharunderscore}{\kern0pt}edrel\ \isacommand{{\isachardot}{\kern0pt}}\isamarkupfalse%
\isanewline
\ \ \isacommand{moreover}\isamarkupfalse%
\ \isacommand{from}\isamarkupfalse%
\ assms\isanewline
\ \ \isacommand{have}\isamarkupfalse%
\ {\isachardoublequoteopen}tr{\isacharunderscore}{\kern0pt}down{\isacharparenleft}{\kern0pt}{\isacharquery}{\kern0pt}r{\isacharparenleft}{\kern0pt}x{\isacharparenright}{\kern0pt}{\isacharcomma}{\kern0pt}z{\isacharparenright}{\kern0pt}\ {\isasymsubseteq}\ eclose{\isacharparenleft}{\kern0pt}{\isacharbraceleft}{\kern0pt}z{\isacharbraceright}{\kern0pt}{\isacharparenright}{\kern0pt}{\isachardoublequoteclose}\ \isacommand{using}\isamarkupfalse%
\ tr{\isacharunderscore}{\kern0pt}edrel{\isacharunderscore}{\kern0pt}subset\ \isacommand{by}\isamarkupfalse%
\ simp\isanewline
\ \ \isacommand{ultimately}\isamarkupfalse%
\isanewline
\ \ \isacommand{have}\isamarkupfalse%
\ {\isachardoublequoteopen}wfrec{\isacharparenleft}{\kern0pt}{\isacharquery}{\kern0pt}r{\isacharparenleft}{\kern0pt}x{\isacharparenright}{\kern0pt}{\isacharcomma}{\kern0pt}z{\isacharcomma}{\kern0pt}Hv{\isacharparenleft}{\kern0pt}G{\isacharparenright}{\kern0pt}{\isacharparenright}{\kern0pt}\ {\isacharequal}{\kern0pt}\ wfrec{\isacharbrackleft}{\kern0pt}eclose{\isacharparenleft}{\kern0pt}{\isacharbraceleft}{\kern0pt}z{\isacharbraceright}{\kern0pt}{\isacharparenright}{\kern0pt}{\isacharbrackright}{\kern0pt}{\isacharparenleft}{\kern0pt}{\isacharquery}{\kern0pt}r{\isacharparenleft}{\kern0pt}x{\isacharparenright}{\kern0pt}{\isacharcomma}{\kern0pt}z{\isacharcomma}{\kern0pt}Hv{\isacharparenleft}{\kern0pt}G{\isacharparenright}{\kern0pt}{\isacharparenright}{\kern0pt}{\isachardoublequoteclose}\isanewline
\ \ \ \ \isacommand{using}\isamarkupfalse%
\ wfrec{\isacharunderscore}{\kern0pt}restr\ \isacommand{by}\isamarkupfalse%
\ simp\isanewline
\ \ \isacommand{also}\isamarkupfalse%
\ \isacommand{from}\isamarkupfalse%
\ {\isacartoucheopen}z{\isasymin}domain{\isacharparenleft}{\kern0pt}x{\isacharparenright}{\kern0pt}{\isacartoucheclose}\isanewline
\ \ \isacommand{have}\isamarkupfalse%
\ {\isachardoublequoteopen}{\isachardot}{\kern0pt}{\isachardot}{\kern0pt}{\isachardot}{\kern0pt}\ {\isacharequal}{\kern0pt}\ wfrec{\isacharparenleft}{\kern0pt}{\isacharquery}{\kern0pt}r{\isacharparenleft}{\kern0pt}z{\isacharparenright}{\kern0pt}{\isacharcomma}{\kern0pt}z{\isacharcomma}{\kern0pt}Hv{\isacharparenleft}{\kern0pt}G{\isacharparenright}{\kern0pt}{\isacharparenright}{\kern0pt}{\isachardoublequoteclose}\isanewline
\ \ \ \ \isacommand{using}\isamarkupfalse%
\ restrict{\isacharunderscore}{\kern0pt}edrel{\isacharunderscore}{\kern0pt}eq\ wfrec{\isacharunderscore}{\kern0pt}restr{\isacharunderscore}{\kern0pt}eq\ \isacommand{by}\isamarkupfalse%
\ simp\isanewline
\ \ \isacommand{finally}\isamarkupfalse%
\ \isacommand{show}\isamarkupfalse%
\ {\isacharquery}{\kern0pt}thesis\ \isacommand{{\isachardot}{\kern0pt}}\isamarkupfalse%
\isanewline
\isacommand{qed}\isamarkupfalse%
%
\endisatagproof
{\isafoldproof}%
%
\isadelimproof
%
\endisadelimproof
%
\begin{isamarkuptext}%
The next lemma provides the usual recursive expresion for the definition of term\isa{val}.%
\end{isamarkuptext}\isamarkuptrue%
\isacommand{lemma}\isamarkupfalse%
\ def{\isacharunderscore}{\kern0pt}val{\isacharcolon}{\kern0pt}\ \ {\isachardoublequoteopen}val{\isacharparenleft}{\kern0pt}G{\isacharcomma}{\kern0pt}x{\isacharparenright}{\kern0pt}\ {\isacharequal}{\kern0pt}\ {\isacharbraceleft}{\kern0pt}val{\isacharparenleft}{\kern0pt}G{\isacharcomma}{\kern0pt}t{\isacharparenright}{\kern0pt}\ {\isachardot}{\kern0pt}{\isachardot}{\kern0pt}\ t{\isasymin}domain{\isacharparenleft}{\kern0pt}x{\isacharparenright}{\kern0pt}\ {\isacharcomma}{\kern0pt}\ {\isasymexists}p{\isasymin}P\ {\isachardot}{\kern0pt}\ \ {\isasymlangle}t{\isacharcomma}{\kern0pt}p{\isasymrangle}{\isasymin}x\ {\isasymand}\ p\ {\isasymin}\ G\ {\isacharbraceright}{\kern0pt}{\isachardoublequoteclose}\isanewline
%
\isadelimproof
%
\endisadelimproof
%
\isatagproof
\isacommand{proof}\isamarkupfalse%
\ {\isacharminus}{\kern0pt}\isanewline
\ \ \isacommand{let}\isamarkupfalse%
\isanewline
\ \ \ \ {\isacharquery}{\kern0pt}r{\isacharequal}{\kern0pt}{\isachardoublequoteopen}{\isasymlambda}{\isasymtau}\ {\isachardot}{\kern0pt}\ edrel{\isacharparenleft}{\kern0pt}eclose{\isacharparenleft}{\kern0pt}{\isacharbraceleft}{\kern0pt}{\isasymtau}{\isacharbraceright}{\kern0pt}{\isacharparenright}{\kern0pt}{\isacharparenright}{\kern0pt}{\isachardoublequoteclose}\isanewline
\ \ \isacommand{let}\isamarkupfalse%
\isanewline
\ \ \ \ {\isacharquery}{\kern0pt}f{\isacharequal}{\kern0pt}{\isachardoublequoteopen}{\isasymlambda}z{\isasymin}{\isacharquery}{\kern0pt}r{\isacharparenleft}{\kern0pt}x{\isacharparenright}{\kern0pt}{\isacharminus}{\kern0pt}{\isacharbackquote}{\kern0pt}{\isacharbackquote}{\kern0pt}{\isacharbraceleft}{\kern0pt}x{\isacharbraceright}{\kern0pt}{\isachardot}{\kern0pt}\ wfrec{\isacharparenleft}{\kern0pt}{\isacharquery}{\kern0pt}r{\isacharparenleft}{\kern0pt}x{\isacharparenright}{\kern0pt}{\isacharcomma}{\kern0pt}z{\isacharcomma}{\kern0pt}Hv{\isacharparenleft}{\kern0pt}G{\isacharparenright}{\kern0pt}{\isacharparenright}{\kern0pt}{\isachardoublequoteclose}\isanewline
\ \ \isacommand{have}\isamarkupfalse%
\ {\isachardoublequoteopen}{\isasymforall}{\isasymtau}{\isachardot}{\kern0pt}\ wf{\isacharparenleft}{\kern0pt}{\isacharquery}{\kern0pt}r{\isacharparenleft}{\kern0pt}{\isasymtau}{\isacharparenright}{\kern0pt}{\isacharparenright}{\kern0pt}{\isachardoublequoteclose}\ \isacommand{using}\isamarkupfalse%
\ wf{\isacharunderscore}{\kern0pt}edrel\ \isacommand{by}\isamarkupfalse%
\ simp\isanewline
\ \ \isacommand{with}\isamarkupfalse%
\ wfrec\ {\isacharbrackleft}{\kern0pt}of\ {\isacharunderscore}{\kern0pt}\ x{\isacharbrackright}{\kern0pt}\isanewline
\ \ \isacommand{have}\isamarkupfalse%
\ {\isachardoublequoteopen}val{\isacharparenleft}{\kern0pt}G{\isacharcomma}{\kern0pt}x{\isacharparenright}{\kern0pt}\ {\isacharequal}{\kern0pt}\ Hv{\isacharparenleft}{\kern0pt}G{\isacharcomma}{\kern0pt}x{\isacharcomma}{\kern0pt}{\isacharquery}{\kern0pt}f{\isacharparenright}{\kern0pt}{\isachardoublequoteclose}\ \isacommand{using}\isamarkupfalse%
\ val{\isacharunderscore}{\kern0pt}def\ \isacommand{by}\isamarkupfalse%
\ simp\isanewline
\ \ \isacommand{also}\isamarkupfalse%
\isanewline
\ \ \isacommand{have}\isamarkupfalse%
\ {\isachardoublequoteopen}\ {\isachardot}{\kern0pt}{\isachardot}{\kern0pt}{\isachardot}{\kern0pt}\ {\isacharequal}{\kern0pt}\ Hv{\isacharparenleft}{\kern0pt}G{\isacharcomma}{\kern0pt}x{\isacharcomma}{\kern0pt}{\isasymlambda}z{\isasymin}domain{\isacharparenleft}{\kern0pt}x{\isacharparenright}{\kern0pt}{\isachardot}{\kern0pt}\ wfrec{\isacharparenleft}{\kern0pt}{\isacharquery}{\kern0pt}r{\isacharparenleft}{\kern0pt}x{\isacharparenright}{\kern0pt}{\isacharcomma}{\kern0pt}z{\isacharcomma}{\kern0pt}Hv{\isacharparenleft}{\kern0pt}G{\isacharparenright}{\kern0pt}{\isacharparenright}{\kern0pt}{\isacharparenright}{\kern0pt}{\isachardoublequoteclose}\isanewline
\ \ \ \ \isacommand{using}\isamarkupfalse%
\ dom{\isacharunderscore}{\kern0pt}under{\isacharunderscore}{\kern0pt}edrel{\isacharunderscore}{\kern0pt}eclose\ \isacommand{by}\isamarkupfalse%
\ simp\isanewline
\ \ \isacommand{also}\isamarkupfalse%
\isanewline
\ \ \isacommand{have}\isamarkupfalse%
\ {\isachardoublequoteopen}\ {\isachardot}{\kern0pt}{\isachardot}{\kern0pt}{\isachardot}{\kern0pt}\ {\isacharequal}{\kern0pt}\ Hv{\isacharparenleft}{\kern0pt}G{\isacharcomma}{\kern0pt}x{\isacharcomma}{\kern0pt}{\isasymlambda}z{\isasymin}domain{\isacharparenleft}{\kern0pt}x{\isacharparenright}{\kern0pt}{\isachardot}{\kern0pt}\ val{\isacharparenleft}{\kern0pt}G{\isacharcomma}{\kern0pt}z{\isacharparenright}{\kern0pt}{\isacharparenright}{\kern0pt}{\isachardoublequoteclose}\isanewline
\ \ \ \ \isacommand{using}\isamarkupfalse%
\ aux{\isacharunderscore}{\kern0pt}def{\isacharunderscore}{\kern0pt}val\ val{\isacharunderscore}{\kern0pt}def\ \isacommand{by}\isamarkupfalse%
\ simp\isanewline
\ \ \isacommand{finally}\isamarkupfalse%
\isanewline
\ \ \isacommand{show}\isamarkupfalse%
\ {\isacharquery}{\kern0pt}thesis\ \isacommand{using}\isamarkupfalse%
\ Hv{\isacharunderscore}{\kern0pt}def\ SepReplace{\isacharunderscore}{\kern0pt}def\ \isacommand{by}\isamarkupfalse%
\ simp\isanewline
\isacommand{qed}\isamarkupfalse%
%
\endisatagproof
{\isafoldproof}%
%
\isadelimproof
\isanewline
%
\endisadelimproof
\isanewline
\isacommand{lemma}\isamarkupfalse%
\ val{\isacharunderscore}{\kern0pt}mono\ {\isacharcolon}{\kern0pt}\ {\isachardoublequoteopen}x{\isasymsubseteq}y\ {\isasymLongrightarrow}\ val{\isacharparenleft}{\kern0pt}G{\isacharcomma}{\kern0pt}x{\isacharparenright}{\kern0pt}\ {\isasymsubseteq}\ val{\isacharparenleft}{\kern0pt}G{\isacharcomma}{\kern0pt}y{\isacharparenright}{\kern0pt}{\isachardoublequoteclose}\isanewline
%
\isadelimproof
\ \ %
\endisadelimproof
%
\isatagproof
\isacommand{by}\isamarkupfalse%
\ {\isacharparenleft}{\kern0pt}subst\ {\isacharparenleft}{\kern0pt}{\isadigit{1}}\ {\isadigit{2}}{\isacharparenright}{\kern0pt}\ def{\isacharunderscore}{\kern0pt}val{\isacharcomma}{\kern0pt}\ force{\isacharparenright}{\kern0pt}%
\endisatagproof
{\isafoldproof}%
%
\isadelimproof
%
\endisadelimproof
%
\begin{isamarkuptext}%
Check-names are the canonical names for elements of the
ground model. Here we show that this is the case.%
\end{isamarkuptext}\isamarkuptrue%
\isacommand{lemma}\isamarkupfalse%
\ valcheck\ {\isacharcolon}{\kern0pt}\ {\isachardoublequoteopen}one\ {\isasymin}\ G\ {\isasymLongrightarrow}\ \ one\ {\isasymin}\ P\ {\isasymLongrightarrow}\ val{\isacharparenleft}{\kern0pt}G{\isacharcomma}{\kern0pt}check{\isacharparenleft}{\kern0pt}y{\isacharparenright}{\kern0pt}{\isacharparenright}{\kern0pt}\ \ {\isacharequal}{\kern0pt}\ y{\isachardoublequoteclose}\isanewline
%
\isadelimproof
%
\endisadelimproof
%
\isatagproof
\isacommand{proof}\isamarkupfalse%
\ {\isacharparenleft}{\kern0pt}induct\ rule{\isacharcolon}{\kern0pt}eps{\isacharunderscore}{\kern0pt}induct{\isacharparenright}{\kern0pt}\isanewline
\ \ \isacommand{case}\isamarkupfalse%
\ {\isacharparenleft}{\kern0pt}{\isadigit{1}}\ y{\isacharparenright}{\kern0pt}\isanewline
\ \ \isacommand{then}\isamarkupfalse%
\ \isacommand{show}\isamarkupfalse%
\ {\isacharquery}{\kern0pt}case\isanewline
\ \ \isacommand{proof}\isamarkupfalse%
\ {\isacharminus}{\kern0pt}\ \ \ \ \isanewline
\ \ \ \ \isacommand{have}\isamarkupfalse%
\ {\isachardoublequoteopen}check{\isacharparenleft}{\kern0pt}y{\isacharparenright}{\kern0pt}\ {\isacharequal}{\kern0pt}\ {\isacharbraceleft}{\kern0pt}\ {\isasymlangle}check{\isacharparenleft}{\kern0pt}w{\isacharparenright}{\kern0pt}{\isacharcomma}{\kern0pt}\ one{\isasymrangle}\ {\isachardot}{\kern0pt}\ w\ {\isasymin}\ y{\isacharbraceright}{\kern0pt}{\isachardoublequoteclose}\ \ {\isacharparenleft}{\kern0pt}\isakeyword{is}\ {\isachardoublequoteopen}{\isacharunderscore}{\kern0pt}\ {\isacharequal}{\kern0pt}\ {\isacharquery}{\kern0pt}C{\isachardoublequoteclose}{\isacharparenright}{\kern0pt}\ \isanewline
\ \ \ \ \ \ \isacommand{using}\isamarkupfalse%
\ def{\isacharunderscore}{\kern0pt}check\ \isacommand{{\isachardot}{\kern0pt}}\isamarkupfalse%
\isanewline
\ \ \ \ \isacommand{then}\isamarkupfalse%
\isanewline
\ \ \ \ \isacommand{have}\isamarkupfalse%
\ {\isachardoublequoteopen}val{\isacharparenleft}{\kern0pt}G{\isacharcomma}{\kern0pt}check{\isacharparenleft}{\kern0pt}y{\isacharparenright}{\kern0pt}{\isacharparenright}{\kern0pt}\ {\isacharequal}{\kern0pt}\ val{\isacharparenleft}{\kern0pt}G{\isacharcomma}{\kern0pt}\ {\isacharbraceleft}{\kern0pt}{\isasymlangle}check{\isacharparenleft}{\kern0pt}w{\isacharparenright}{\kern0pt}{\isacharcomma}{\kern0pt}\ one{\isasymrangle}\ {\isachardot}{\kern0pt}\ w\ {\isasymin}\ y{\isacharbraceright}{\kern0pt}{\isacharparenright}{\kern0pt}{\isachardoublequoteclose}\isanewline
\ \ \ \ \ \ \isacommand{by}\isamarkupfalse%
\ simp\isanewline
\ \ \ \ \isacommand{also}\isamarkupfalse%
\isanewline
\ \ \ \ \isacommand{have}\isamarkupfalse%
\ {\isachardoublequoteopen}\ {\isachardot}{\kern0pt}{\isachardot}{\kern0pt}{\isachardot}{\kern0pt}\ \ {\isacharequal}{\kern0pt}\ {\isacharbraceleft}{\kern0pt}val{\isacharparenleft}{\kern0pt}G{\isacharcomma}{\kern0pt}t{\isacharparenright}{\kern0pt}\ {\isachardot}{\kern0pt}{\isachardot}{\kern0pt}\ t{\isasymin}domain{\isacharparenleft}{\kern0pt}{\isacharquery}{\kern0pt}C{\isacharparenright}{\kern0pt}\ {\isacharcomma}{\kern0pt}\ {\isasymexists}p{\isasymin}P\ {\isachardot}{\kern0pt}\ \ {\isasymlangle}t{\isacharcomma}{\kern0pt}\ p{\isasymrangle}{\isasymin}{\isacharquery}{\kern0pt}C\ {\isasymand}\ p\ {\isasymin}\ G\ {\isacharbraceright}{\kern0pt}{\isachardoublequoteclose}\isanewline
\ \ \ \ \ \ \isacommand{using}\isamarkupfalse%
\ def{\isacharunderscore}{\kern0pt}val\ \isacommand{by}\isamarkupfalse%
\ blast\isanewline
\ \ \ \ \isacommand{also}\isamarkupfalse%
\isanewline
\ \ \ \ \isacommand{have}\isamarkupfalse%
\ {\isachardoublequoteopen}\ {\isachardot}{\kern0pt}{\isachardot}{\kern0pt}{\isachardot}{\kern0pt}\ {\isacharequal}{\kern0pt}\ \ {\isacharbraceleft}{\kern0pt}val{\isacharparenleft}{\kern0pt}G{\isacharcomma}{\kern0pt}t{\isacharparenright}{\kern0pt}\ {\isachardot}{\kern0pt}{\isachardot}{\kern0pt}\ t{\isasymin}domain{\isacharparenleft}{\kern0pt}{\isacharquery}{\kern0pt}C{\isacharparenright}{\kern0pt}\ {\isacharcomma}{\kern0pt}\ {\isasymexists}w{\isasymin}y{\isachardot}{\kern0pt}\ t{\isacharequal}{\kern0pt}check{\isacharparenleft}{\kern0pt}w{\isacharparenright}{\kern0pt}\ {\isacharbraceright}{\kern0pt}{\isachardoublequoteclose}\isanewline
\ \ \ \ \ \ \isacommand{using}\isamarkupfalse%
\ {\isadigit{1}}\ \isacommand{by}\isamarkupfalse%
\ simp\isanewline
\ \ \ \ \isacommand{also}\isamarkupfalse%
\isanewline
\ \ \ \ \isacommand{have}\isamarkupfalse%
\ {\isachardoublequoteopen}\ {\isachardot}{\kern0pt}{\isachardot}{\kern0pt}{\isachardot}{\kern0pt}\ {\isacharequal}{\kern0pt}\ {\isacharbraceleft}{\kern0pt}val{\isacharparenleft}{\kern0pt}G{\isacharcomma}{\kern0pt}check{\isacharparenleft}{\kern0pt}w{\isacharparenright}{\kern0pt}{\isacharparenright}{\kern0pt}\ {\isachardot}{\kern0pt}\ w{\isasymin}y\ {\isacharbraceright}{\kern0pt}{\isachardoublequoteclose}\isanewline
\ \ \ \ \ \ \isacommand{by}\isamarkupfalse%
\ force\isanewline
\ \ \ \ \isacommand{finally}\isamarkupfalse%
\isanewline
\ \ \ \ \isacommand{show}\isamarkupfalse%
\ {\isachardoublequoteopen}val{\isacharparenleft}{\kern0pt}G{\isacharcomma}{\kern0pt}check{\isacharparenleft}{\kern0pt}y{\isacharparenright}{\kern0pt}{\isacharparenright}{\kern0pt}\ {\isacharequal}{\kern0pt}\ y{\isachardoublequoteclose}\isanewline
\ \ \ \ \ \ \isacommand{using}\isamarkupfalse%
\ {\isadigit{1}}\ \isacommand{by}\isamarkupfalse%
\ simp\isanewline
\ \ \isacommand{qed}\isamarkupfalse%
\isanewline
\isacommand{qed}\isamarkupfalse%
%
\endisatagproof
{\isafoldproof}%
%
\isadelimproof
\isanewline
%
\endisadelimproof
\isanewline
\isacommand{lemma}\isamarkupfalse%
\ val{\isacharunderscore}{\kern0pt}of{\isacharunderscore}{\kern0pt}name\ {\isacharcolon}{\kern0pt}\isanewline
\ \ {\isachardoublequoteopen}val{\isacharparenleft}{\kern0pt}G{\isacharcomma}{\kern0pt}{\isacharbraceleft}{\kern0pt}x{\isasymin}A{\isasymtimes}P{\isachardot}{\kern0pt}\ Q{\isacharparenleft}{\kern0pt}x{\isacharparenright}{\kern0pt}{\isacharbraceright}{\kern0pt}{\isacharparenright}{\kern0pt}\ {\isacharequal}{\kern0pt}\ {\isacharbraceleft}{\kern0pt}val{\isacharparenleft}{\kern0pt}G{\isacharcomma}{\kern0pt}t{\isacharparenright}{\kern0pt}\ {\isachardot}{\kern0pt}{\isachardot}{\kern0pt}\ t{\isasymin}A\ {\isacharcomma}{\kern0pt}\ {\isasymexists}p{\isasymin}P\ {\isachardot}{\kern0pt}\ \ Q{\isacharparenleft}{\kern0pt}{\isasymlangle}t{\isacharcomma}{\kern0pt}p{\isasymrangle}{\isacharparenright}{\kern0pt}\ {\isasymand}\ p\ {\isasymin}\ G\ {\isacharbraceright}{\kern0pt}{\isachardoublequoteclose}\isanewline
%
\isadelimproof
%
\endisadelimproof
%
\isatagproof
\isacommand{proof}\isamarkupfalse%
\ {\isacharminus}{\kern0pt}\isanewline
\ \ \isacommand{let}\isamarkupfalse%
\isanewline
\ \ \ \ {\isacharquery}{\kern0pt}n{\isacharequal}{\kern0pt}{\isachardoublequoteopen}{\isacharbraceleft}{\kern0pt}x{\isasymin}A{\isasymtimes}P{\isachardot}{\kern0pt}\ Q{\isacharparenleft}{\kern0pt}x{\isacharparenright}{\kern0pt}{\isacharbraceright}{\kern0pt}{\isachardoublequoteclose}\ \isakeyword{and}\isanewline
\ \ \ \ {\isacharquery}{\kern0pt}r{\isacharequal}{\kern0pt}{\isachardoublequoteopen}{\isasymlambda}{\isasymtau}\ {\isachardot}{\kern0pt}\ edrel{\isacharparenleft}{\kern0pt}eclose{\isacharparenleft}{\kern0pt}{\isacharbraceleft}{\kern0pt}{\isasymtau}{\isacharbraceright}{\kern0pt}{\isacharparenright}{\kern0pt}{\isacharparenright}{\kern0pt}{\isachardoublequoteclose}\isanewline
\ \ \isacommand{let}\isamarkupfalse%
\isanewline
\ \ \ \ {\isacharquery}{\kern0pt}f{\isacharequal}{\kern0pt}{\isachardoublequoteopen}{\isasymlambda}z{\isasymin}{\isacharquery}{\kern0pt}r{\isacharparenleft}{\kern0pt}{\isacharquery}{\kern0pt}n{\isacharparenright}{\kern0pt}{\isacharminus}{\kern0pt}{\isacharbackquote}{\kern0pt}{\isacharbackquote}{\kern0pt}{\isacharbraceleft}{\kern0pt}{\isacharquery}{\kern0pt}n{\isacharbraceright}{\kern0pt}{\isachardot}{\kern0pt}\ val{\isacharparenleft}{\kern0pt}G{\isacharcomma}{\kern0pt}z{\isacharparenright}{\kern0pt}{\isachardoublequoteclose}\isanewline
\ \ \isacommand{have}\isamarkupfalse%
\isanewline
\ \ \ \ wfR\ {\isacharcolon}{\kern0pt}\ {\isachardoublequoteopen}wf{\isacharparenleft}{\kern0pt}{\isacharquery}{\kern0pt}r{\isacharparenleft}{\kern0pt}{\isasymtau}{\isacharparenright}{\kern0pt}{\isacharparenright}{\kern0pt}{\isachardoublequoteclose}\ \isakeyword{for}\ {\isasymtau}\isanewline
\ \ \ \ \isacommand{by}\isamarkupfalse%
\ {\isacharparenleft}{\kern0pt}simp\ add{\isacharcolon}{\kern0pt}\ wf{\isacharunderscore}{\kern0pt}edrel{\isacharparenright}{\kern0pt}\isanewline
\ \ \isacommand{have}\isamarkupfalse%
\ {\isachardoublequoteopen}domain{\isacharparenleft}{\kern0pt}{\isacharquery}{\kern0pt}n{\isacharparenright}{\kern0pt}\ {\isasymsubseteq}\ A{\isachardoublequoteclose}\ \isacommand{by}\isamarkupfalse%
\ auto\isanewline
\ \ \isacommand{{\isacharbraceleft}{\kern0pt}}\isamarkupfalse%
\ \isacommand{fix}\isamarkupfalse%
\ t\isanewline
\ \ \ \ \isacommand{assume}\isamarkupfalse%
\ H{\isacharcolon}{\kern0pt}{\isachardoublequoteopen}t\ {\isasymin}\ domain{\isacharparenleft}{\kern0pt}{\isacharbraceleft}{\kern0pt}x\ {\isasymin}\ A\ {\isasymtimes}\ P\ {\isachardot}{\kern0pt}\ Q{\isacharparenleft}{\kern0pt}x{\isacharparenright}{\kern0pt}{\isacharbraceright}{\kern0pt}{\isacharparenright}{\kern0pt}{\isachardoublequoteclose}\isanewline
\ \ \ \ \isacommand{then}\isamarkupfalse%
\ \isacommand{have}\isamarkupfalse%
\ {\isachardoublequoteopen}{\isacharquery}{\kern0pt}f\ {\isacharbackquote}{\kern0pt}\ t\ {\isacharequal}{\kern0pt}\ {\isacharparenleft}{\kern0pt}if\ t\ {\isasymin}\ {\isacharquery}{\kern0pt}r{\isacharparenleft}{\kern0pt}{\isacharquery}{\kern0pt}n{\isacharparenright}{\kern0pt}{\isacharminus}{\kern0pt}{\isacharbackquote}{\kern0pt}{\isacharbackquote}{\kern0pt}{\isacharbraceleft}{\kern0pt}{\isacharquery}{\kern0pt}n{\isacharbraceright}{\kern0pt}\ then\ val{\isacharparenleft}{\kern0pt}G{\isacharcomma}{\kern0pt}t{\isacharparenright}{\kern0pt}\ else\ {\isadigit{0}}{\isacharparenright}{\kern0pt}{\isachardoublequoteclose}\isanewline
\ \ \ \ \ \ \isacommand{by}\isamarkupfalse%
\ simp\isanewline
\ \ \ \ \isacommand{moreover}\isamarkupfalse%
\ \isacommand{have}\isamarkupfalse%
\ {\isachardoublequoteopen}{\isachardot}{\kern0pt}{\isachardot}{\kern0pt}{\isachardot}{\kern0pt}\ {\isacharequal}{\kern0pt}\ val{\isacharparenleft}{\kern0pt}G{\isacharcomma}{\kern0pt}t{\isacharparenright}{\kern0pt}{\isachardoublequoteclose}\isanewline
\ \ \ \ \ \ \isacommand{using}\isamarkupfalse%
\ dom{\isacharunderscore}{\kern0pt}under{\isacharunderscore}{\kern0pt}edrel{\isacharunderscore}{\kern0pt}eclose\ H\ if{\isacharunderscore}{\kern0pt}P\ \isacommand{by}\isamarkupfalse%
\ auto\isanewline
\ \ \isacommand{{\isacharbraceright}{\kern0pt}}\isamarkupfalse%
\isanewline
\ \ \isacommand{then}\isamarkupfalse%
\isanewline
\ \ \isacommand{have}\isamarkupfalse%
\ Eq{\isadigit{1}}{\isacharcolon}{\kern0pt}\ {\isachardoublequoteopen}t\ {\isasymin}\ domain{\isacharparenleft}{\kern0pt}{\isacharbraceleft}{\kern0pt}x\ {\isasymin}\ A\ {\isasymtimes}\ P\ {\isachardot}{\kern0pt}\ Q{\isacharparenleft}{\kern0pt}x{\isacharparenright}{\kern0pt}{\isacharbraceright}{\kern0pt}{\isacharparenright}{\kern0pt}\ {\isasymLongrightarrow}\ val{\isacharparenleft}{\kern0pt}G{\isacharcomma}{\kern0pt}t{\isacharparenright}{\kern0pt}\ {\isacharequal}{\kern0pt}\ {\isacharquery}{\kern0pt}f{\isacharbackquote}{\kern0pt}\ t{\isachardoublequoteclose}\ \ \isakeyword{for}\ t\isanewline
\ \ \ \ \isacommand{by}\isamarkupfalse%
\ simp\isanewline
\ \ \isacommand{have}\isamarkupfalse%
\ {\isachardoublequoteopen}val{\isacharparenleft}{\kern0pt}G{\isacharcomma}{\kern0pt}{\isacharquery}{\kern0pt}n{\isacharparenright}{\kern0pt}\ {\isacharequal}{\kern0pt}\ {\isacharbraceleft}{\kern0pt}val{\isacharparenleft}{\kern0pt}G{\isacharcomma}{\kern0pt}t{\isacharparenright}{\kern0pt}\ {\isachardot}{\kern0pt}{\isachardot}{\kern0pt}\ t{\isasymin}domain{\isacharparenleft}{\kern0pt}{\isacharquery}{\kern0pt}n{\isacharparenright}{\kern0pt}{\isacharcomma}{\kern0pt}\ {\isasymexists}p\ {\isasymin}\ P\ {\isachardot}{\kern0pt}\ {\isasymlangle}t{\isacharcomma}{\kern0pt}p{\isasymrangle}\ {\isasymin}\ {\isacharquery}{\kern0pt}n\ {\isasymand}\ p\ {\isasymin}\ G{\isacharbraceright}{\kern0pt}{\isachardoublequoteclose}\isanewline
\ \ \ \ \isacommand{by}\isamarkupfalse%
\ {\isacharparenleft}{\kern0pt}subst\ def{\isacharunderscore}{\kern0pt}val{\isacharcomma}{\kern0pt}simp{\isacharparenright}{\kern0pt}\isanewline
\ \ \isacommand{also}\isamarkupfalse%
\isanewline
\ \ \isacommand{have}\isamarkupfalse%
\ {\isachardoublequoteopen}{\isachardot}{\kern0pt}{\isachardot}{\kern0pt}{\isachardot}{\kern0pt}\ {\isacharequal}{\kern0pt}\ {\isacharbraceleft}{\kern0pt}{\isacharquery}{\kern0pt}f{\isacharbackquote}{\kern0pt}t\ {\isachardot}{\kern0pt}{\isachardot}{\kern0pt}\ t{\isasymin}domain{\isacharparenleft}{\kern0pt}{\isacharquery}{\kern0pt}n{\isacharparenright}{\kern0pt}{\isacharcomma}{\kern0pt}\ {\isasymexists}p{\isasymin}P\ {\isachardot}{\kern0pt}\ {\isasymlangle}t{\isacharcomma}{\kern0pt}p{\isasymrangle}{\isasymin}{\isacharquery}{\kern0pt}n\ {\isasymand}\ p{\isasymin}G{\isacharbraceright}{\kern0pt}{\isachardoublequoteclose}\isanewline
\ \ \ \ \isacommand{unfolding}\isamarkupfalse%
\ Hv{\isacharunderscore}{\kern0pt}def\isanewline
\ \ \ \ \isacommand{by}\isamarkupfalse%
\ {\isacharparenleft}{\kern0pt}subst\ SepReplace{\isacharunderscore}{\kern0pt}dom{\isacharunderscore}{\kern0pt}implies{\isacharcomma}{\kern0pt}auto\ simp\ add{\isacharcolon}{\kern0pt}Eq{\isadigit{1}}{\isacharparenright}{\kern0pt}\isanewline
\ \ \isacommand{also}\isamarkupfalse%
\isanewline
\ \ \isacommand{have}\isamarkupfalse%
\ \ {\isachardoublequoteopen}{\isachardot}{\kern0pt}{\isachardot}{\kern0pt}{\isachardot}{\kern0pt}\ {\isacharequal}{\kern0pt}\ {\isacharbraceleft}{\kern0pt}\ {\isacharparenleft}{\kern0pt}if\ t{\isasymin}{\isacharquery}{\kern0pt}r{\isacharparenleft}{\kern0pt}{\isacharquery}{\kern0pt}n{\isacharparenright}{\kern0pt}{\isacharminus}{\kern0pt}{\isacharbackquote}{\kern0pt}{\isacharbackquote}{\kern0pt}{\isacharbraceleft}{\kern0pt}{\isacharquery}{\kern0pt}n{\isacharbraceright}{\kern0pt}\ then\ val{\isacharparenleft}{\kern0pt}G{\isacharcomma}{\kern0pt}t{\isacharparenright}{\kern0pt}\ else\ {\isadigit{0}}{\isacharparenright}{\kern0pt}\ {\isachardot}{\kern0pt}{\isachardot}{\kern0pt}\ t{\isasymin}domain{\isacharparenleft}{\kern0pt}{\isacharquery}{\kern0pt}n{\isacharparenright}{\kern0pt}{\isacharcomma}{\kern0pt}\ {\isasymexists}p{\isasymin}P\ {\isachardot}{\kern0pt}\ {\isasymlangle}t{\isacharcomma}{\kern0pt}p{\isasymrangle}{\isasymin}{\isacharquery}{\kern0pt}n\ {\isasymand}\ p{\isasymin}G{\isacharbraceright}{\kern0pt}{\isachardoublequoteclose}\isanewline
\ \ \ \ \isacommand{by}\isamarkupfalse%
\ {\isacharparenleft}{\kern0pt}simp{\isacharparenright}{\kern0pt}\isanewline
\ \ \isacommand{also}\isamarkupfalse%
\isanewline
\ \ \isacommand{have}\isamarkupfalse%
\ Eq{\isadigit{2}}{\isacharcolon}{\kern0pt}\ \ {\isachardoublequoteopen}{\isachardot}{\kern0pt}{\isachardot}{\kern0pt}{\isachardot}{\kern0pt}\ {\isacharequal}{\kern0pt}\ {\isacharbraceleft}{\kern0pt}\ val{\isacharparenleft}{\kern0pt}G{\isacharcomma}{\kern0pt}t{\isacharparenright}{\kern0pt}\ {\isachardot}{\kern0pt}{\isachardot}{\kern0pt}\ t{\isasymin}domain{\isacharparenleft}{\kern0pt}{\isacharquery}{\kern0pt}n{\isacharparenright}{\kern0pt}{\isacharcomma}{\kern0pt}\ {\isasymexists}p{\isasymin}P\ {\isachardot}{\kern0pt}\ {\isasymlangle}t{\isacharcomma}{\kern0pt}p{\isasymrangle}{\isasymin}{\isacharquery}{\kern0pt}n\ {\isasymand}\ p{\isasymin}G{\isacharbraceright}{\kern0pt}{\isachardoublequoteclose}\isanewline
\ \ \isacommand{proof}\isamarkupfalse%
\ {\isacharminus}{\kern0pt}\isanewline
\ \ \ \ \isacommand{have}\isamarkupfalse%
\ {\isachardoublequoteopen}domain{\isacharparenleft}{\kern0pt}{\isacharquery}{\kern0pt}n{\isacharparenright}{\kern0pt}\ {\isasymsubseteq}\ {\isacharquery}{\kern0pt}r{\isacharparenleft}{\kern0pt}{\isacharquery}{\kern0pt}n{\isacharparenright}{\kern0pt}{\isacharminus}{\kern0pt}{\isacharbackquote}{\kern0pt}{\isacharbackquote}{\kern0pt}{\isacharbraceleft}{\kern0pt}{\isacharquery}{\kern0pt}n{\isacharbraceright}{\kern0pt}{\isachardoublequoteclose}\isanewline
\ \ \ \ \ \ \isacommand{using}\isamarkupfalse%
\ dom{\isacharunderscore}{\kern0pt}under{\isacharunderscore}{\kern0pt}edrel{\isacharunderscore}{\kern0pt}eclose\ \isacommand{by}\isamarkupfalse%
\ simp\isanewline
\ \ \ \ \isacommand{then}\isamarkupfalse%
\isanewline
\ \ \ \ \isacommand{have}\isamarkupfalse%
\ {\isachardoublequoteopen}{\isasymforall}t{\isasymin}domain{\isacharparenleft}{\kern0pt}{\isacharquery}{\kern0pt}n{\isacharparenright}{\kern0pt}{\isachardot}{\kern0pt}\ {\isacharparenleft}{\kern0pt}if\ t{\isasymin}{\isacharquery}{\kern0pt}r{\isacharparenleft}{\kern0pt}{\isacharquery}{\kern0pt}n{\isacharparenright}{\kern0pt}{\isacharminus}{\kern0pt}{\isacharbackquote}{\kern0pt}{\isacharbackquote}{\kern0pt}{\isacharbraceleft}{\kern0pt}{\isacharquery}{\kern0pt}n{\isacharbraceright}{\kern0pt}\ then\ val{\isacharparenleft}{\kern0pt}G{\isacharcomma}{\kern0pt}t{\isacharparenright}{\kern0pt}\ else\ {\isadigit{0}}{\isacharparenright}{\kern0pt}\ {\isacharequal}{\kern0pt}\ val{\isacharparenleft}{\kern0pt}G{\isacharcomma}{\kern0pt}t{\isacharparenright}{\kern0pt}{\isachardoublequoteclose}\isanewline
\ \ \ \ \ \ \isacommand{by}\isamarkupfalse%
\ auto\isanewline
\ \ \ \ \isacommand{then}\isamarkupfalse%
\isanewline
\ \ \ \ \isacommand{show}\isamarkupfalse%
\ {\isachardoublequoteopen}{\isacharbraceleft}{\kern0pt}\ {\isacharparenleft}{\kern0pt}if\ t{\isasymin}{\isacharquery}{\kern0pt}r{\isacharparenleft}{\kern0pt}{\isacharquery}{\kern0pt}n{\isacharparenright}{\kern0pt}{\isacharminus}{\kern0pt}{\isacharbackquote}{\kern0pt}{\isacharbackquote}{\kern0pt}{\isacharbraceleft}{\kern0pt}{\isacharquery}{\kern0pt}n{\isacharbraceright}{\kern0pt}\ then\ val{\isacharparenleft}{\kern0pt}G{\isacharcomma}{\kern0pt}t{\isacharparenright}{\kern0pt}\ else\ {\isadigit{0}}{\isacharparenright}{\kern0pt}\ {\isachardot}{\kern0pt}{\isachardot}{\kern0pt}\ t{\isasymin}domain{\isacharparenleft}{\kern0pt}{\isacharquery}{\kern0pt}n{\isacharparenright}{\kern0pt}{\isacharcomma}{\kern0pt}\ {\isasymexists}p{\isasymin}P\ {\isachardot}{\kern0pt}\ {\isasymlangle}t{\isacharcomma}{\kern0pt}p{\isasymrangle}{\isasymin}{\isacharquery}{\kern0pt}n\ {\isasymand}\ p{\isasymin}G{\isacharbraceright}{\kern0pt}\ {\isacharequal}{\kern0pt}\isanewline
\ \ \ \ \ \ \ \ \ \ {\isacharbraceleft}{\kern0pt}\ val{\isacharparenleft}{\kern0pt}G{\isacharcomma}{\kern0pt}t{\isacharparenright}{\kern0pt}\ {\isachardot}{\kern0pt}{\isachardot}{\kern0pt}\ t{\isasymin}domain{\isacharparenleft}{\kern0pt}{\isacharquery}{\kern0pt}n{\isacharparenright}{\kern0pt}{\isacharcomma}{\kern0pt}\ {\isasymexists}p{\isasymin}P\ {\isachardot}{\kern0pt}\ {\isasymlangle}t{\isacharcomma}{\kern0pt}p{\isasymrangle}{\isasymin}{\isacharquery}{\kern0pt}n\ {\isasymand}\ p{\isasymin}G{\isacharbraceright}{\kern0pt}{\isachardoublequoteclose}\isanewline
\ \ \ \ \ \ \isacommand{by}\isamarkupfalse%
\ auto\isanewline
\ \ \isacommand{qed}\isamarkupfalse%
\isanewline
\ \ \isacommand{also}\isamarkupfalse%
\isanewline
\ \ \isacommand{have}\isamarkupfalse%
\ {\isachardoublequoteopen}\ {\isachardot}{\kern0pt}{\isachardot}{\kern0pt}{\isachardot}{\kern0pt}\ {\isacharequal}{\kern0pt}\ {\isacharbraceleft}{\kern0pt}\ val{\isacharparenleft}{\kern0pt}G{\isacharcomma}{\kern0pt}t{\isacharparenright}{\kern0pt}\ {\isachardot}{\kern0pt}{\isachardot}{\kern0pt}\ t{\isasymin}A{\isacharcomma}{\kern0pt}\ {\isasymexists}p{\isasymin}P\ {\isachardot}{\kern0pt}\ {\isasymlangle}t{\isacharcomma}{\kern0pt}p{\isasymrangle}{\isasymin}{\isacharquery}{\kern0pt}n\ {\isasymand}\ p{\isasymin}G{\isacharbraceright}{\kern0pt}{\isachardoublequoteclose}\isanewline
\ \ \ \ \isacommand{by}\isamarkupfalse%
\ force\isanewline
\ \ \isacommand{finally}\isamarkupfalse%
\isanewline
\ \ \isacommand{show}\isamarkupfalse%
\ {\isachardoublequoteopen}\ val{\isacharparenleft}{\kern0pt}G{\isacharcomma}{\kern0pt}{\isacharquery}{\kern0pt}n{\isacharparenright}{\kern0pt}\ \ {\isacharequal}{\kern0pt}\ {\isacharbraceleft}{\kern0pt}\ val{\isacharparenleft}{\kern0pt}G{\isacharcomma}{\kern0pt}t{\isacharparenright}{\kern0pt}\ {\isachardot}{\kern0pt}{\isachardot}{\kern0pt}\ t{\isasymin}A{\isacharcomma}{\kern0pt}\ {\isasymexists}p{\isasymin}P\ {\isachardot}{\kern0pt}\ Q{\isacharparenleft}{\kern0pt}{\isasymlangle}t{\isacharcomma}{\kern0pt}p{\isasymrangle}{\isacharparenright}{\kern0pt}\ {\isasymand}\ p{\isasymin}G{\isacharbraceright}{\kern0pt}{\isachardoublequoteclose}\isanewline
\ \ \ \ \isacommand{by}\isamarkupfalse%
\ auto\isanewline
\isacommand{qed}\isamarkupfalse%
%
\endisatagproof
{\isafoldproof}%
%
\isadelimproof
\isanewline
%
\endisadelimproof
\isanewline
\isacommand{lemma}\isamarkupfalse%
\ val{\isacharunderscore}{\kern0pt}of{\isacharunderscore}{\kern0pt}name{\isacharunderscore}{\kern0pt}alt\ {\isacharcolon}{\kern0pt}\isanewline
\ \ {\isachardoublequoteopen}val{\isacharparenleft}{\kern0pt}G{\isacharcomma}{\kern0pt}{\isacharbraceleft}{\kern0pt}x{\isasymin}A{\isasymtimes}P{\isachardot}{\kern0pt}\ Q{\isacharparenleft}{\kern0pt}x{\isacharparenright}{\kern0pt}{\isacharbraceright}{\kern0pt}{\isacharparenright}{\kern0pt}\ {\isacharequal}{\kern0pt}\ {\isacharbraceleft}{\kern0pt}val{\isacharparenleft}{\kern0pt}G{\isacharcomma}{\kern0pt}t{\isacharparenright}{\kern0pt}\ {\isachardot}{\kern0pt}{\isachardot}{\kern0pt}\ t{\isasymin}A\ {\isacharcomma}{\kern0pt}\ {\isasymexists}p{\isasymin}P{\isasyminter}G\ {\isachardot}{\kern0pt}\ \ Q{\isacharparenleft}{\kern0pt}{\isasymlangle}t{\isacharcomma}{\kern0pt}p{\isasymrangle}{\isacharparenright}{\kern0pt}\ {\isacharbraceright}{\kern0pt}{\isachardoublequoteclose}\isanewline
%
\isadelimproof
\ \ %
\endisadelimproof
%
\isatagproof
\isacommand{using}\isamarkupfalse%
\ val{\isacharunderscore}{\kern0pt}of{\isacharunderscore}{\kern0pt}name\ \isacommand{by}\isamarkupfalse%
\ force%
\endisatagproof
{\isafoldproof}%
%
\isadelimproof
\isanewline
%
\endisadelimproof
\isanewline
\isacommand{lemma}\isamarkupfalse%
\ val{\isacharunderscore}{\kern0pt}only{\isacharunderscore}{\kern0pt}names{\isacharcolon}{\kern0pt}\ {\isachardoublequoteopen}val{\isacharparenleft}{\kern0pt}F{\isacharcomma}{\kern0pt}{\isasymtau}{\isacharparenright}{\kern0pt}\ {\isacharequal}{\kern0pt}\ val{\isacharparenleft}{\kern0pt}F{\isacharcomma}{\kern0pt}{\isacharbraceleft}{\kern0pt}x{\isasymin}{\isasymtau}{\isachardot}{\kern0pt}\ {\isasymexists}t{\isasymin}domain{\isacharparenleft}{\kern0pt}{\isasymtau}{\isacharparenright}{\kern0pt}{\isachardot}{\kern0pt}\ {\isasymexists}p{\isasymin}P{\isachardot}{\kern0pt}\ x{\isacharequal}{\kern0pt}{\isasymlangle}t{\isacharcomma}{\kern0pt}p{\isasymrangle}{\isacharbraceright}{\kern0pt}{\isacharparenright}{\kern0pt}{\isachardoublequoteclose}\isanewline
\ \ {\isacharparenleft}{\kern0pt}\isakeyword{is}\ {\isachardoublequoteopen}{\isacharunderscore}{\kern0pt}\ {\isacharequal}{\kern0pt}\ val{\isacharparenleft}{\kern0pt}F{\isacharcomma}{\kern0pt}{\isacharquery}{\kern0pt}name{\isacharparenright}{\kern0pt}{\isachardoublequoteclose}{\isacharparenright}{\kern0pt}\isanewline
%
\isadelimproof
%
\endisadelimproof
%
\isatagproof
\isacommand{proof}\isamarkupfalse%
\ {\isacharminus}{\kern0pt}\isanewline
\ \ \isacommand{have}\isamarkupfalse%
\ {\isachardoublequoteopen}val{\isacharparenleft}{\kern0pt}F{\isacharcomma}{\kern0pt}{\isacharquery}{\kern0pt}name{\isacharparenright}{\kern0pt}\ {\isacharequal}{\kern0pt}\ {\isacharbraceleft}{\kern0pt}val{\isacharparenleft}{\kern0pt}F{\isacharcomma}{\kern0pt}\ t{\isacharparenright}{\kern0pt}{\isachardot}{\kern0pt}{\isachardot}{\kern0pt}\ t{\isasymin}domain{\isacharparenleft}{\kern0pt}{\isacharquery}{\kern0pt}name{\isacharparenright}{\kern0pt}{\isacharcomma}{\kern0pt}\ {\isasymexists}p{\isasymin}P{\isachardot}{\kern0pt}\ {\isasymlangle}t{\isacharcomma}{\kern0pt}\ p{\isasymrangle}\ {\isasymin}\ {\isacharquery}{\kern0pt}name\ {\isasymand}\ p\ {\isasymin}\ F{\isacharbraceright}{\kern0pt}{\isachardoublequoteclose}\isanewline
\ \ \ \ \isacommand{using}\isamarkupfalse%
\ def{\isacharunderscore}{\kern0pt}val\ \isacommand{by}\isamarkupfalse%
\ blast\isanewline
\ \ \isacommand{also}\isamarkupfalse%
\isanewline
\ \ \isacommand{have}\isamarkupfalse%
\ {\isachardoublequoteopen}\ {\isachardot}{\kern0pt}{\isachardot}{\kern0pt}{\isachardot}{\kern0pt}\ {\isacharequal}{\kern0pt}\ {\isacharbraceleft}{\kern0pt}val{\isacharparenleft}{\kern0pt}F{\isacharcomma}{\kern0pt}\ t{\isacharparenright}{\kern0pt}{\isachardot}{\kern0pt}\ t{\isasymin}{\isacharbraceleft}{\kern0pt}y{\isasymin}domain{\isacharparenleft}{\kern0pt}{\isacharquery}{\kern0pt}name{\isacharparenright}{\kern0pt}{\isachardot}{\kern0pt}\ {\isasymexists}p{\isasymin}P{\isachardot}{\kern0pt}\ {\isasymlangle}y{\isacharcomma}{\kern0pt}\ p{\isasymrangle}\ {\isasymin}\ {\isacharquery}{\kern0pt}name\ {\isasymand}\ p\ {\isasymin}\ F{\isacharbraceright}{\kern0pt}{\isacharbraceright}{\kern0pt}{\isachardoublequoteclose}\isanewline
\ \ \ \ \isacommand{using}\isamarkupfalse%
\ Sep{\isacharunderscore}{\kern0pt}and{\isacharunderscore}{\kern0pt}Replace\ \isacommand{by}\isamarkupfalse%
\ simp\isanewline
\ \ \isacommand{also}\isamarkupfalse%
\isanewline
\ \ \isacommand{have}\isamarkupfalse%
\ {\isachardoublequoteopen}\ {\isachardot}{\kern0pt}{\isachardot}{\kern0pt}{\isachardot}{\kern0pt}\ {\isacharequal}{\kern0pt}\ {\isacharbraceleft}{\kern0pt}val{\isacharparenleft}{\kern0pt}F{\isacharcomma}{\kern0pt}\ t{\isacharparenright}{\kern0pt}{\isachardot}{\kern0pt}\ t{\isasymin}{\isacharbraceleft}{\kern0pt}y{\isasymin}domain{\isacharparenleft}{\kern0pt}{\isasymtau}{\isacharparenright}{\kern0pt}{\isachardot}{\kern0pt}\ {\isasymexists}p{\isasymin}P{\isachardot}{\kern0pt}\ {\isasymlangle}y{\isacharcomma}{\kern0pt}\ p{\isasymrangle}\ {\isasymin}\ {\isasymtau}\ {\isasymand}\ p\ {\isasymin}\ F{\isacharbraceright}{\kern0pt}{\isacharbraceright}{\kern0pt}{\isachardoublequoteclose}\isanewline
\ \ \ \ \isacommand{by}\isamarkupfalse%
\ blast\isanewline
\ \ \isacommand{also}\isamarkupfalse%
\isanewline
\ \ \isacommand{have}\isamarkupfalse%
\ {\isachardoublequoteopen}\ {\isachardot}{\kern0pt}{\isachardot}{\kern0pt}{\isachardot}{\kern0pt}\ {\isacharequal}{\kern0pt}\ {\isacharbraceleft}{\kern0pt}val{\isacharparenleft}{\kern0pt}F{\isacharcomma}{\kern0pt}\ t{\isacharparenright}{\kern0pt}{\isachardot}{\kern0pt}{\isachardot}{\kern0pt}\ t{\isasymin}domain{\isacharparenleft}{\kern0pt}{\isasymtau}{\isacharparenright}{\kern0pt}{\isacharcomma}{\kern0pt}\ {\isasymexists}p{\isasymin}P{\isachardot}{\kern0pt}\ {\isasymlangle}t{\isacharcomma}{\kern0pt}\ p{\isasymrangle}\ {\isasymin}\ {\isasymtau}\ {\isasymand}\ p\ {\isasymin}\ F{\isacharbraceright}{\kern0pt}{\isachardoublequoteclose}\isanewline
\ \ \ \ \isacommand{using}\isamarkupfalse%
\ Sep{\isacharunderscore}{\kern0pt}and{\isacharunderscore}{\kern0pt}Replace\ \isacommand{by}\isamarkupfalse%
\ simp\isanewline
\ \ \isacommand{also}\isamarkupfalse%
\isanewline
\ \ \isacommand{have}\isamarkupfalse%
\ {\isachardoublequoteopen}\ {\isachardot}{\kern0pt}{\isachardot}{\kern0pt}{\isachardot}{\kern0pt}\ {\isacharequal}{\kern0pt}\ val{\isacharparenleft}{\kern0pt}F{\isacharcomma}{\kern0pt}\ {\isasymtau}{\isacharparenright}{\kern0pt}{\isachardoublequoteclose}\isanewline
\ \ \ \ \isacommand{using}\isamarkupfalse%
\ def{\isacharunderscore}{\kern0pt}val{\isacharbrackleft}{\kern0pt}symmetric{\isacharbrackright}{\kern0pt}\ \isacommand{by}\isamarkupfalse%
\ blast\isanewline
\ \ \isacommand{finally}\isamarkupfalse%
\isanewline
\ \ \isacommand{show}\isamarkupfalse%
\ {\isacharquery}{\kern0pt}thesis\ \isacommand{{\isachardot}{\kern0pt}{\isachardot}{\kern0pt}}\isamarkupfalse%
\isanewline
\isacommand{qed}\isamarkupfalse%
%
\endisatagproof
{\isafoldproof}%
%
\isadelimproof
\isanewline
%
\endisadelimproof
\isanewline
\isacommand{lemma}\isamarkupfalse%
\ val{\isacharunderscore}{\kern0pt}only{\isacharunderscore}{\kern0pt}pairs{\isacharcolon}{\kern0pt}\ {\isachardoublequoteopen}val{\isacharparenleft}{\kern0pt}F{\isacharcomma}{\kern0pt}{\isasymtau}{\isacharparenright}{\kern0pt}\ {\isacharequal}{\kern0pt}\ val{\isacharparenleft}{\kern0pt}F{\isacharcomma}{\kern0pt}{\isacharbraceleft}{\kern0pt}x{\isasymin}{\isasymtau}{\isachardot}{\kern0pt}\ {\isasymexists}t\ p{\isachardot}{\kern0pt}\ x{\isacharequal}{\kern0pt}{\isasymlangle}t{\isacharcomma}{\kern0pt}p{\isasymrangle}{\isacharbraceright}{\kern0pt}{\isacharparenright}{\kern0pt}{\isachardoublequoteclose}\isanewline
%
\isadelimproof
%
\endisadelimproof
%
\isatagproof
\isacommand{proof}\isamarkupfalse%
\isanewline
\ \ \isacommand{have}\isamarkupfalse%
\ {\isachardoublequoteopen}val{\isacharparenleft}{\kern0pt}F{\isacharcomma}{\kern0pt}{\isasymtau}{\isacharparenright}{\kern0pt}\ {\isacharequal}{\kern0pt}\ val{\isacharparenleft}{\kern0pt}F{\isacharcomma}{\kern0pt}{\isacharbraceleft}{\kern0pt}x{\isasymin}{\isasymtau}{\isachardot}{\kern0pt}\ {\isasymexists}t{\isasymin}domain{\isacharparenleft}{\kern0pt}{\isasymtau}{\isacharparenright}{\kern0pt}{\isachardot}{\kern0pt}\ {\isasymexists}p{\isasymin}P{\isachardot}{\kern0pt}\ x{\isacharequal}{\kern0pt}{\isasymlangle}t{\isacharcomma}{\kern0pt}p{\isasymrangle}{\isacharbraceright}{\kern0pt}{\isacharparenright}{\kern0pt}{\isachardoublequoteclose}\isanewline
\ \ \ \ {\isacharparenleft}{\kern0pt}\isakeyword{is}\ {\isachardoublequoteopen}{\isacharunderscore}{\kern0pt}\ {\isacharequal}{\kern0pt}\ val{\isacharparenleft}{\kern0pt}F{\isacharcomma}{\kern0pt}{\isacharquery}{\kern0pt}name{\isacharparenright}{\kern0pt}{\isachardoublequoteclose}{\isacharparenright}{\kern0pt}\isanewline
\ \ \ \ \isacommand{using}\isamarkupfalse%
\ val{\isacharunderscore}{\kern0pt}only{\isacharunderscore}{\kern0pt}names\ \isacommand{{\isachardot}{\kern0pt}}\isamarkupfalse%
\isanewline
\ \ \isacommand{also}\isamarkupfalse%
\isanewline
\ \ \isacommand{have}\isamarkupfalse%
\ {\isachardoublequoteopen}{\isachardot}{\kern0pt}{\isachardot}{\kern0pt}{\isachardot}{\kern0pt}\ {\isasymsubseteq}\ val{\isacharparenleft}{\kern0pt}F{\isacharcomma}{\kern0pt}{\isacharbraceleft}{\kern0pt}x{\isasymin}{\isasymtau}{\isachardot}{\kern0pt}\ {\isasymexists}t\ p{\isachardot}{\kern0pt}\ x{\isacharequal}{\kern0pt}{\isasymlangle}t{\isacharcomma}{\kern0pt}p{\isasymrangle}{\isacharbraceright}{\kern0pt}{\isacharparenright}{\kern0pt}{\isachardoublequoteclose}\isanewline
\ \ \ \ \isacommand{using}\isamarkupfalse%
\ val{\isacharunderscore}{\kern0pt}mono{\isacharbrackleft}{\kern0pt}of\ {\isacharquery}{\kern0pt}name\ {\isachardoublequoteopen}{\isacharbraceleft}{\kern0pt}x{\isasymin}{\isasymtau}{\isachardot}{\kern0pt}\ {\isasymexists}t\ p{\isachardot}{\kern0pt}\ x{\isacharequal}{\kern0pt}{\isasymlangle}t{\isacharcomma}{\kern0pt}p{\isasymrangle}{\isacharbraceright}{\kern0pt}{\isachardoublequoteclose}{\isacharbrackright}{\kern0pt}\ \isacommand{by}\isamarkupfalse%
\ auto\isanewline
\ \ \isacommand{finally}\isamarkupfalse%
\isanewline
\ \ \isacommand{show}\isamarkupfalse%
\ {\isachardoublequoteopen}val{\isacharparenleft}{\kern0pt}F{\isacharcomma}{\kern0pt}{\isasymtau}{\isacharparenright}{\kern0pt}\ {\isasymsubseteq}\ val{\isacharparenleft}{\kern0pt}F{\isacharcomma}{\kern0pt}{\isacharbraceleft}{\kern0pt}x{\isasymin}{\isasymtau}{\isachardot}{\kern0pt}\ {\isasymexists}t\ p{\isachardot}{\kern0pt}\ x{\isacharequal}{\kern0pt}{\isasymlangle}t{\isacharcomma}{\kern0pt}p{\isasymrangle}{\isacharbraceright}{\kern0pt}{\isacharparenright}{\kern0pt}{\isachardoublequoteclose}\ \isacommand{by}\isamarkupfalse%
\ simp\isanewline
\isacommand{next}\isamarkupfalse%
\isanewline
\ \ \isacommand{show}\isamarkupfalse%
\ {\isachardoublequoteopen}val{\isacharparenleft}{\kern0pt}F{\isacharcomma}{\kern0pt}{\isacharbraceleft}{\kern0pt}x{\isasymin}{\isasymtau}{\isachardot}{\kern0pt}\ {\isasymexists}t\ p{\isachardot}{\kern0pt}\ x{\isacharequal}{\kern0pt}{\isasymlangle}t{\isacharcomma}{\kern0pt}p{\isasymrangle}{\isacharbraceright}{\kern0pt}{\isacharparenright}{\kern0pt}\ {\isasymsubseteq}\ val{\isacharparenleft}{\kern0pt}F{\isacharcomma}{\kern0pt}{\isasymtau}{\isacharparenright}{\kern0pt}{\isachardoublequoteclose}\isanewline
\ \ \ \ \isacommand{using}\isamarkupfalse%
\ val{\isacharunderscore}{\kern0pt}mono{\isacharbrackleft}{\kern0pt}of\ {\isachardoublequoteopen}{\isacharbraceleft}{\kern0pt}x{\isasymin}{\isasymtau}{\isachardot}{\kern0pt}\ {\isasymexists}t\ p{\isachardot}{\kern0pt}\ x{\isacharequal}{\kern0pt}{\isasymlangle}t{\isacharcomma}{\kern0pt}p{\isasymrangle}{\isacharbraceright}{\kern0pt}{\isachardoublequoteclose}{\isacharbrackright}{\kern0pt}\ \isacommand{by}\isamarkupfalse%
\ auto\isanewline
\isacommand{qed}\isamarkupfalse%
%
\endisatagproof
{\isafoldproof}%
%
\isadelimproof
\isanewline
%
\endisadelimproof
\isanewline
\isacommand{lemma}\isamarkupfalse%
\ val{\isacharunderscore}{\kern0pt}subset{\isacharunderscore}{\kern0pt}domain{\isacharunderscore}{\kern0pt}times{\isacharunderscore}{\kern0pt}range{\isacharcolon}{\kern0pt}\ {\isachardoublequoteopen}val{\isacharparenleft}{\kern0pt}F{\isacharcomma}{\kern0pt}{\isasymtau}{\isacharparenright}{\kern0pt}\ {\isasymsubseteq}\ val{\isacharparenleft}{\kern0pt}F{\isacharcomma}{\kern0pt}domain{\isacharparenleft}{\kern0pt}{\isasymtau}{\isacharparenright}{\kern0pt}{\isasymtimes}range{\isacharparenleft}{\kern0pt}{\isasymtau}{\isacharparenright}{\kern0pt}{\isacharparenright}{\kern0pt}{\isachardoublequoteclose}\isanewline
%
\isadelimproof
\ \ %
\endisadelimproof
%
\isatagproof
\isacommand{using}\isamarkupfalse%
\ val{\isacharunderscore}{\kern0pt}only{\isacharunderscore}{\kern0pt}pairs{\isacharbrackleft}{\kern0pt}THEN\ equalityD{\isadigit{1}}{\isacharbrackright}{\kern0pt}\isanewline
\ \ \ \ val{\isacharunderscore}{\kern0pt}mono{\isacharbrackleft}{\kern0pt}of\ {\isachardoublequoteopen}{\isacharbraceleft}{\kern0pt}x\ {\isasymin}\ {\isasymtau}\ {\isachardot}{\kern0pt}\ {\isasymexists}t\ p{\isachardot}{\kern0pt}\ x\ {\isacharequal}{\kern0pt}\ {\isasymlangle}t{\isacharcomma}{\kern0pt}\ p{\isasymrangle}{\isacharbraceright}{\kern0pt}{\isachardoublequoteclose}\ {\isachardoublequoteopen}domain{\isacharparenleft}{\kern0pt}{\isasymtau}{\isacharparenright}{\kern0pt}{\isasymtimes}range{\isacharparenleft}{\kern0pt}{\isasymtau}{\isacharparenright}{\kern0pt}{\isachardoublequoteclose}{\isacharbrackright}{\kern0pt}\ \isacommand{by}\isamarkupfalse%
\ blast%
\endisatagproof
{\isafoldproof}%
%
\isadelimproof
\isanewline
%
\endisadelimproof
\isanewline
\isacommand{lemma}\isamarkupfalse%
\ val{\isacharunderscore}{\kern0pt}subset{\isacharunderscore}{\kern0pt}domain{\isacharunderscore}{\kern0pt}times{\isacharunderscore}{\kern0pt}P{\isacharcolon}{\kern0pt}\ {\isachardoublequoteopen}val{\isacharparenleft}{\kern0pt}F{\isacharcomma}{\kern0pt}{\isasymtau}{\isacharparenright}{\kern0pt}\ {\isasymsubseteq}\ val{\isacharparenleft}{\kern0pt}F{\isacharcomma}{\kern0pt}domain{\isacharparenleft}{\kern0pt}{\isasymtau}{\isacharparenright}{\kern0pt}{\isasymtimes}P{\isacharparenright}{\kern0pt}{\isachardoublequoteclose}\isanewline
%
\isadelimproof
\ \ %
\endisadelimproof
%
\isatagproof
\isacommand{using}\isamarkupfalse%
\ val{\isacharunderscore}{\kern0pt}only{\isacharunderscore}{\kern0pt}names{\isacharbrackleft}{\kern0pt}of\ F\ {\isasymtau}{\isacharbrackright}{\kern0pt}\ val{\isacharunderscore}{\kern0pt}mono{\isacharbrackleft}{\kern0pt}of\ {\isachardoublequoteopen}{\isacharbraceleft}{\kern0pt}x{\isasymin}{\isasymtau}{\isachardot}{\kern0pt}\ {\isasymexists}t{\isasymin}domain{\isacharparenleft}{\kern0pt}{\isasymtau}{\isacharparenright}{\kern0pt}{\isachardot}{\kern0pt}\ {\isasymexists}p{\isasymin}P{\isachardot}{\kern0pt}\ x{\isacharequal}{\kern0pt}{\isasymlangle}t{\isacharcomma}{\kern0pt}p{\isasymrangle}{\isacharbraceright}{\kern0pt}{\isachardoublequoteclose}\ {\isachardoublequoteopen}domain{\isacharparenleft}{\kern0pt}{\isasymtau}{\isacharparenright}{\kern0pt}{\isasymtimes}P{\isachardoublequoteclose}\ F{\isacharbrackright}{\kern0pt}\isanewline
\ \ \isacommand{by}\isamarkupfalse%
\ auto%
\endisatagproof
{\isafoldproof}%
%
\isadelimproof
\isanewline
%
\endisadelimproof
\isanewline
\isacommand{definition}\isamarkupfalse%
\isanewline
\ \ GenExt\ {\isacharcolon}{\kern0pt}{\isacharcolon}{\kern0pt}\ {\isachardoublequoteopen}i{\isasymRightarrow}i{\isachardoublequoteclose}\ \ \ \ \ {\isacharparenleft}{\kern0pt}{\isachardoublequoteopen}M{\isacharbrackleft}{\kern0pt}{\isacharunderscore}{\kern0pt}{\isacharbrackright}{\kern0pt}{\isachardoublequoteclose}{\isacharparenright}{\kern0pt}\isanewline
\ \ \isakeyword{where}\ {\isachardoublequoteopen}GenExt{\isacharparenleft}{\kern0pt}G{\isacharparenright}{\kern0pt}{\isasymequiv}\ {\isacharbraceleft}{\kern0pt}val{\isacharparenleft}{\kern0pt}G{\isacharcomma}{\kern0pt}{\isasymtau}{\isacharparenright}{\kern0pt}{\isachardot}{\kern0pt}\ {\isasymtau}\ {\isasymin}\ M{\isacharbraceright}{\kern0pt}{\isachardoublequoteclose}\isanewline
\isanewline
\isanewline
\isacommand{lemma}\isamarkupfalse%
\ val{\isacharunderscore}{\kern0pt}of{\isacharunderscore}{\kern0pt}elem{\isacharcolon}{\kern0pt}\ {\isachardoublequoteopen}{\isasymlangle}{\isasymtheta}{\isacharcomma}{\kern0pt}p{\isasymrangle}\ {\isasymin}\ {\isasympi}\ {\isasymLongrightarrow}\ p{\isasymin}G\ {\isasymLongrightarrow}\ p{\isasymin}P\ {\isasymLongrightarrow}\ val{\isacharparenleft}{\kern0pt}G{\isacharcomma}{\kern0pt}{\isasymtheta}{\isacharparenright}{\kern0pt}\ {\isasymin}\ val{\isacharparenleft}{\kern0pt}G{\isacharcomma}{\kern0pt}{\isasympi}{\isacharparenright}{\kern0pt}{\isachardoublequoteclose}\isanewline
%
\isadelimproof
%
\endisadelimproof
%
\isatagproof
\isacommand{proof}\isamarkupfalse%
\ {\isacharminus}{\kern0pt}\isanewline
\ \ \isacommand{assume}\isamarkupfalse%
\isanewline
\ \ \ \ {\isachardoublequoteopen}{\isasymlangle}{\isasymtheta}{\isacharcomma}{\kern0pt}p{\isasymrangle}\ {\isasymin}\ {\isasympi}{\isachardoublequoteclose}\isanewline
\ \ \isacommand{then}\isamarkupfalse%
\isanewline
\ \ \isacommand{have}\isamarkupfalse%
\ {\isachardoublequoteopen}{\isasymtheta}{\isasymin}domain{\isacharparenleft}{\kern0pt}{\isasympi}{\isacharparenright}{\kern0pt}{\isachardoublequoteclose}\ \isacommand{by}\isamarkupfalse%
\ auto\isanewline
\ \ \isacommand{assume}\isamarkupfalse%
\ {\isachardoublequoteopen}p{\isasymin}G{\isachardoublequoteclose}\ {\isachardoublequoteopen}p{\isasymin}P{\isachardoublequoteclose}\isanewline
\ \ \isacommand{with}\isamarkupfalse%
\ {\isacartoucheopen}{\isasymtheta}{\isasymin}domain{\isacharparenleft}{\kern0pt}{\isasympi}{\isacharparenright}{\kern0pt}{\isacartoucheclose}\ {\isacartoucheopen}{\isasymlangle}{\isasymtheta}{\isacharcomma}{\kern0pt}p{\isasymrangle}\ {\isasymin}\ {\isasympi}{\isacartoucheclose}\isanewline
\ \ \isacommand{have}\isamarkupfalse%
\ {\isachardoublequoteopen}val{\isacharparenleft}{\kern0pt}G{\isacharcomma}{\kern0pt}{\isasymtheta}{\isacharparenright}{\kern0pt}\ {\isasymin}\ {\isacharbraceleft}{\kern0pt}val{\isacharparenleft}{\kern0pt}G{\isacharcomma}{\kern0pt}t{\isacharparenright}{\kern0pt}\ {\isachardot}{\kern0pt}{\isachardot}{\kern0pt}\ t{\isasymin}domain{\isacharparenleft}{\kern0pt}{\isasympi}{\isacharparenright}{\kern0pt}\ {\isacharcomma}{\kern0pt}\ {\isasymexists}p{\isasymin}P\ {\isachardot}{\kern0pt}\ \ {\isasymlangle}t{\isacharcomma}{\kern0pt}\ p{\isasymrangle}{\isasymin}{\isasympi}\ {\isasymand}\ p\ {\isasymin}\ G\ {\isacharbraceright}{\kern0pt}{\isachardoublequoteclose}\isanewline
\ \ \ \ \isacommand{by}\isamarkupfalse%
\ auto\isanewline
\ \ \isacommand{then}\isamarkupfalse%
\isanewline
\ \ \isacommand{show}\isamarkupfalse%
\ {\isacharquery}{\kern0pt}thesis\ \isacommand{by}\isamarkupfalse%
\ {\isacharparenleft}{\kern0pt}subst\ def{\isacharunderscore}{\kern0pt}val{\isacharparenright}{\kern0pt}\isanewline
\isacommand{qed}\isamarkupfalse%
%
\endisatagproof
{\isafoldproof}%
%
\isadelimproof
\isanewline
%
\endisadelimproof
\isanewline
\isacommand{lemma}\isamarkupfalse%
\ elem{\isacharunderscore}{\kern0pt}of{\isacharunderscore}{\kern0pt}val{\isacharcolon}{\kern0pt}\ {\isachardoublequoteopen}x{\isasymin}val{\isacharparenleft}{\kern0pt}G{\isacharcomma}{\kern0pt}{\isasympi}{\isacharparenright}{\kern0pt}\ {\isasymLongrightarrow}\ {\isasymexists}{\isasymtheta}{\isasymin}domain{\isacharparenleft}{\kern0pt}{\isasympi}{\isacharparenright}{\kern0pt}{\isachardot}{\kern0pt}\ val{\isacharparenleft}{\kern0pt}G{\isacharcomma}{\kern0pt}{\isasymtheta}{\isacharparenright}{\kern0pt}\ {\isacharequal}{\kern0pt}\ x{\isachardoublequoteclose}\isanewline
%
\isadelimproof
\ \ %
\endisadelimproof
%
\isatagproof
\isacommand{by}\isamarkupfalse%
\ {\isacharparenleft}{\kern0pt}subst\ {\isacharparenleft}{\kern0pt}asm{\isacharparenright}{\kern0pt}\ def{\isacharunderscore}{\kern0pt}val{\isacharcomma}{\kern0pt}auto{\isacharparenright}{\kern0pt}%
\endisatagproof
{\isafoldproof}%
%
\isadelimproof
\isanewline
%
\endisadelimproof
\isanewline
\isacommand{lemma}\isamarkupfalse%
\ elem{\isacharunderscore}{\kern0pt}of{\isacharunderscore}{\kern0pt}val{\isacharunderscore}{\kern0pt}pair{\isacharcolon}{\kern0pt}\ {\isachardoublequoteopen}x{\isasymin}val{\isacharparenleft}{\kern0pt}G{\isacharcomma}{\kern0pt}{\isasympi}{\isacharparenright}{\kern0pt}\ {\isasymLongrightarrow}\ {\isasymexists}{\isasymtheta}{\isachardot}{\kern0pt}\ {\isasymexists}p{\isasymin}G{\isachardot}{\kern0pt}\ \ {\isasymlangle}{\isasymtheta}{\isacharcomma}{\kern0pt}p{\isasymrangle}{\isasymin}{\isasympi}\ {\isasymand}\ val{\isacharparenleft}{\kern0pt}G{\isacharcomma}{\kern0pt}{\isasymtheta}{\isacharparenright}{\kern0pt}\ {\isacharequal}{\kern0pt}\ x{\isachardoublequoteclose}\isanewline
%
\isadelimproof
\ \ %
\endisadelimproof
%
\isatagproof
\isacommand{by}\isamarkupfalse%
\ {\isacharparenleft}{\kern0pt}subst\ {\isacharparenleft}{\kern0pt}asm{\isacharparenright}{\kern0pt}\ def{\isacharunderscore}{\kern0pt}val{\isacharcomma}{\kern0pt}auto{\isacharparenright}{\kern0pt}%
\endisatagproof
{\isafoldproof}%
%
\isadelimproof
\isanewline
%
\endisadelimproof
\isanewline
\isacommand{lemma}\isamarkupfalse%
\ elem{\isacharunderscore}{\kern0pt}of{\isacharunderscore}{\kern0pt}val{\isacharunderscore}{\kern0pt}pair{\isacharprime}{\kern0pt}{\isacharcolon}{\kern0pt}\isanewline
\ \ \isakeyword{assumes}\ {\isachardoublequoteopen}{\isasympi}{\isasymin}M{\isachardoublequoteclose}\ {\isachardoublequoteopen}x{\isasymin}val{\isacharparenleft}{\kern0pt}G{\isacharcomma}{\kern0pt}{\isasympi}{\isacharparenright}{\kern0pt}{\isachardoublequoteclose}\isanewline
\ \ \isakeyword{shows}\ {\isachardoublequoteopen}{\isasymexists}{\isasymtheta}{\isasymin}M{\isachardot}{\kern0pt}\ {\isasymexists}p{\isasymin}G{\isachardot}{\kern0pt}\ \ {\isasymlangle}{\isasymtheta}{\isacharcomma}{\kern0pt}p{\isasymrangle}{\isasymin}{\isasympi}\ {\isasymand}\ val{\isacharparenleft}{\kern0pt}G{\isacharcomma}{\kern0pt}{\isasymtheta}{\isacharparenright}{\kern0pt}\ {\isacharequal}{\kern0pt}\ x{\isachardoublequoteclose}\isanewline
%
\isadelimproof
%
\endisadelimproof
%
\isatagproof
\isacommand{proof}\isamarkupfalse%
\ {\isacharminus}{\kern0pt}\isanewline
\ \ \isacommand{from}\isamarkupfalse%
\ assms\isanewline
\ \ \isacommand{obtain}\isamarkupfalse%
\ {\isasymtheta}\ p\ \isakeyword{where}\ {\isachardoublequoteopen}p{\isasymin}G{\isachardoublequoteclose}\ {\isachardoublequoteopen}{\isasymlangle}{\isasymtheta}{\isacharcomma}{\kern0pt}p{\isasymrangle}{\isasymin}{\isasympi}{\isachardoublequoteclose}\ {\isachardoublequoteopen}val{\isacharparenleft}{\kern0pt}G{\isacharcomma}{\kern0pt}{\isasymtheta}{\isacharparenright}{\kern0pt}\ {\isacharequal}{\kern0pt}\ x{\isachardoublequoteclose}\isanewline
\ \ \ \ \isacommand{using}\isamarkupfalse%
\ elem{\isacharunderscore}{\kern0pt}of{\isacharunderscore}{\kern0pt}val{\isacharunderscore}{\kern0pt}pair\ \isacommand{by}\isamarkupfalse%
\ blast\isanewline
\ \ \isacommand{moreover}\isamarkupfalse%
\ \isacommand{from}\isamarkupfalse%
\ this\ {\isacartoucheopen}{\isasympi}{\isasymin}M{\isacartoucheclose}\isanewline
\ \ \isacommand{have}\isamarkupfalse%
\ {\isachardoublequoteopen}{\isasymtheta}{\isasymin}M{\isachardoublequoteclose}\isanewline
\ \ \ \ \isacommand{using}\isamarkupfalse%
\ pair{\isacharunderscore}{\kern0pt}in{\isacharunderscore}{\kern0pt}M{\isacharunderscore}{\kern0pt}iff{\isacharbrackleft}{\kern0pt}THEN\ iffD{\isadigit{1}}{\isacharcomma}{\kern0pt}\ THEN\ conjunct{\isadigit{1}}{\isacharcomma}{\kern0pt}\ simplified{\isacharbrackright}{\kern0pt}\isanewline
\ \ \ \ \ \ transitivity\ \isacommand{by}\isamarkupfalse%
\ blast\isanewline
\ \ \isacommand{ultimately}\isamarkupfalse%
\isanewline
\ \ \isacommand{show}\isamarkupfalse%
\ {\isacharquery}{\kern0pt}thesis\ \isacommand{by}\isamarkupfalse%
\ blast\isanewline
\isacommand{qed}\isamarkupfalse%
%
\endisatagproof
{\isafoldproof}%
%
\isadelimproof
\isanewline
%
\endisadelimproof
\isanewline
\isanewline
\isacommand{lemma}\isamarkupfalse%
\ GenExtD{\isacharcolon}{\kern0pt}\isanewline
\ \ {\isachardoublequoteopen}x\ {\isasymin}\ M{\isacharbrackleft}{\kern0pt}G{\isacharbrackright}{\kern0pt}\ {\isasymLongrightarrow}\ {\isasymexists}{\isasymtau}{\isasymin}M{\isachardot}{\kern0pt}\ x\ {\isacharequal}{\kern0pt}\ val{\isacharparenleft}{\kern0pt}G{\isacharcomma}{\kern0pt}{\isasymtau}{\isacharparenright}{\kern0pt}{\isachardoublequoteclose}\isanewline
%
\isadelimproof
\ \ %
\endisadelimproof
%
\isatagproof
\isacommand{by}\isamarkupfalse%
\ {\isacharparenleft}{\kern0pt}simp\ add{\isacharcolon}{\kern0pt}GenExt{\isacharunderscore}{\kern0pt}def{\isacharparenright}{\kern0pt}%
\endisatagproof
{\isafoldproof}%
%
\isadelimproof
\isanewline
%
\endisadelimproof
\isanewline
\isacommand{lemma}\isamarkupfalse%
\ GenExtI{\isacharcolon}{\kern0pt}\isanewline
\ \ {\isachardoublequoteopen}x\ {\isasymin}\ M\ {\isasymLongrightarrow}\ val{\isacharparenleft}{\kern0pt}G{\isacharcomma}{\kern0pt}x{\isacharparenright}{\kern0pt}\ {\isasymin}\ M{\isacharbrackleft}{\kern0pt}G{\isacharbrackright}{\kern0pt}{\isachardoublequoteclose}\isanewline
%
\isadelimproof
\ \ %
\endisadelimproof
%
\isatagproof
\isacommand{by}\isamarkupfalse%
\ {\isacharparenleft}{\kern0pt}auto\ simp\ add{\isacharcolon}{\kern0pt}\ GenExt{\isacharunderscore}{\kern0pt}def{\isacharparenright}{\kern0pt}%
\endisatagproof
{\isafoldproof}%
%
\isadelimproof
\isanewline
%
\endisadelimproof
\isanewline
\isacommand{lemma}\isamarkupfalse%
\ Transset{\isacharunderscore}{\kern0pt}MG\ {\isacharcolon}{\kern0pt}\ {\isachardoublequoteopen}Transset{\isacharparenleft}{\kern0pt}M{\isacharbrackleft}{\kern0pt}G{\isacharbrackright}{\kern0pt}{\isacharparenright}{\kern0pt}{\isachardoublequoteclose}\isanewline
%
\isadelimproof
%
\endisadelimproof
%
\isatagproof
\isacommand{proof}\isamarkupfalse%
\ {\isacharminus}{\kern0pt}\isanewline
\ \ \isacommand{{\isacharbraceleft}{\kern0pt}}\isamarkupfalse%
\ \isacommand{fix}\isamarkupfalse%
\ vc\ y\isanewline
\ \ \ \ \isacommand{assume}\isamarkupfalse%
\ {\isachardoublequoteopen}vc\ {\isasymin}\ M{\isacharbrackleft}{\kern0pt}G{\isacharbrackright}{\kern0pt}{\isachardoublequoteclose}\ \isakeyword{and}\ {\isachardoublequoteopen}y\ {\isasymin}\ vc{\isachardoublequoteclose}\isanewline
\ \ \ \ \isacommand{then}\isamarkupfalse%
\ \isacommand{obtain}\isamarkupfalse%
\ c\ \isakeyword{where}\ {\isachardoublequoteopen}c{\isasymin}M{\isachardoublequoteclose}\ {\isachardoublequoteopen}val{\isacharparenleft}{\kern0pt}G{\isacharcomma}{\kern0pt}c{\isacharparenright}{\kern0pt}{\isasymin}M{\isacharbrackleft}{\kern0pt}G{\isacharbrackright}{\kern0pt}{\isachardoublequoteclose}\ {\isachardoublequoteopen}y\ {\isasymin}\ val{\isacharparenleft}{\kern0pt}G{\isacharcomma}{\kern0pt}c{\isacharparenright}{\kern0pt}{\isachardoublequoteclose}\isanewline
\ \ \ \ \ \ \isacommand{using}\isamarkupfalse%
\ GenExtD\ \isacommand{by}\isamarkupfalse%
\ auto\isanewline
\ \ \ \ \isacommand{from}\isamarkupfalse%
\ {\isacartoucheopen}y\ {\isasymin}\ val{\isacharparenleft}{\kern0pt}G{\isacharcomma}{\kern0pt}c{\isacharparenright}{\kern0pt}{\isacartoucheclose}\isanewline
\ \ \ \ \isacommand{obtain}\isamarkupfalse%
\ {\isasymtheta}\ \isakeyword{where}\ {\isachardoublequoteopen}{\isasymtheta}{\isasymin}domain{\isacharparenleft}{\kern0pt}c{\isacharparenright}{\kern0pt}{\isachardoublequoteclose}\ {\isachardoublequoteopen}val{\isacharparenleft}{\kern0pt}G{\isacharcomma}{\kern0pt}{\isasymtheta}{\isacharparenright}{\kern0pt}\ {\isacharequal}{\kern0pt}\ y{\isachardoublequoteclose}\isanewline
\ \ \ \ \ \ \isacommand{using}\isamarkupfalse%
\ elem{\isacharunderscore}{\kern0pt}of{\isacharunderscore}{\kern0pt}val\ \isacommand{by}\isamarkupfalse%
\ blast\isanewline
\ \ \ \ \isacommand{with}\isamarkupfalse%
\ trans{\isacharunderscore}{\kern0pt}M\ {\isacartoucheopen}c{\isasymin}M{\isacartoucheclose}\isanewline
\ \ \ \ \isacommand{have}\isamarkupfalse%
\ {\isachardoublequoteopen}y\ {\isasymin}\ M{\isacharbrackleft}{\kern0pt}G{\isacharbrackright}{\kern0pt}{\isachardoublequoteclose}\isanewline
\ \ \ \ \ \ \isacommand{using}\isamarkupfalse%
\ domain{\isacharunderscore}{\kern0pt}trans\ GenExtI\ \isacommand{by}\isamarkupfalse%
\ blast\isanewline
\ \ \isacommand{{\isacharbraceright}{\kern0pt}}\isamarkupfalse%
\isanewline
\ \ \isacommand{then}\isamarkupfalse%
\isanewline
\ \ \isacommand{show}\isamarkupfalse%
\ {\isacharquery}{\kern0pt}thesis\ \isacommand{using}\isamarkupfalse%
\ Transset{\isacharunderscore}{\kern0pt}def\ \isacommand{by}\isamarkupfalse%
\ auto\isanewline
\isacommand{qed}\isamarkupfalse%
%
\endisatagproof
{\isafoldproof}%
%
\isadelimproof
\isanewline
%
\endisadelimproof
\isanewline
\isacommand{lemmas}\isamarkupfalse%
\ transitivity{\isacharunderscore}{\kern0pt}MG\ {\isacharequal}{\kern0pt}\ Transset{\isacharunderscore}{\kern0pt}intf{\isacharbrackleft}{\kern0pt}OF\ Transset{\isacharunderscore}{\kern0pt}MG{\isacharbrackright}{\kern0pt}\isanewline
\isanewline
\isacommand{lemma}\isamarkupfalse%
\ check{\isacharunderscore}{\kern0pt}n{\isacharunderscore}{\kern0pt}M\ {\isacharcolon}{\kern0pt}\isanewline
\ \ \isakeyword{fixes}\ n\isanewline
\ \ \isakeyword{assumes}\ {\isachardoublequoteopen}n\ {\isasymin}\ nat{\isachardoublequoteclose}\isanewline
\ \ \isakeyword{shows}\ {\isachardoublequoteopen}check{\isacharparenleft}{\kern0pt}n{\isacharparenright}{\kern0pt}\ {\isasymin}\ M{\isachardoublequoteclose}\isanewline
%
\isadelimproof
\ \ %
\endisadelimproof
%
\isatagproof
\isacommand{using}\isamarkupfalse%
\ {\isacartoucheopen}n{\isasymin}nat{\isacartoucheclose}\isanewline
\isacommand{proof}\isamarkupfalse%
\ {\isacharparenleft}{\kern0pt}induct\ n{\isacharparenright}{\kern0pt}\isanewline
\ \ \isacommand{case}\isamarkupfalse%
\ {\isadigit{0}}\isanewline
\ \ \isacommand{then}\isamarkupfalse%
\ \isacommand{show}\isamarkupfalse%
\ {\isacharquery}{\kern0pt}case\ \isacommand{using}\isamarkupfalse%
\ zero{\isacharunderscore}{\kern0pt}in{\isacharunderscore}{\kern0pt}M\ \isacommand{by}\isamarkupfalse%
\ {\isacharparenleft}{\kern0pt}subst\ def{\isacharunderscore}{\kern0pt}check{\isacharcomma}{\kern0pt}simp{\isacharparenright}{\kern0pt}\isanewline
\isacommand{next}\isamarkupfalse%
\isanewline
\ \ \isacommand{case}\isamarkupfalse%
\ {\isacharparenleft}{\kern0pt}succ\ x{\isacharparenright}{\kern0pt}\isanewline
\ \ \isacommand{have}\isamarkupfalse%
\ {\isachardoublequoteopen}one\ {\isasymin}\ M{\isachardoublequoteclose}\ \isacommand{using}\isamarkupfalse%
\ one{\isacharunderscore}{\kern0pt}in{\isacharunderscore}{\kern0pt}P\ P{\isacharunderscore}{\kern0pt}sub{\isacharunderscore}{\kern0pt}M\ subsetD\ \isacommand{by}\isamarkupfalse%
\ simp\isanewline
\ \ \isacommand{with}\isamarkupfalse%
\ {\isacartoucheopen}check{\isacharparenleft}{\kern0pt}x{\isacharparenright}{\kern0pt}{\isasymin}M{\isacartoucheclose}\isanewline
\ \ \isacommand{have}\isamarkupfalse%
\ {\isachardoublequoteopen}{\isasymlangle}check{\isacharparenleft}{\kern0pt}x{\isacharparenright}{\kern0pt}{\isacharcomma}{\kern0pt}one{\isasymrangle}\ {\isasymin}\ M{\isachardoublequoteclose}\isanewline
\ \ \ \ \isacommand{using}\isamarkupfalse%
\ tuples{\isacharunderscore}{\kern0pt}in{\isacharunderscore}{\kern0pt}M\ \isacommand{by}\isamarkupfalse%
\ simp\isanewline
\ \ \isacommand{then}\isamarkupfalse%
\isanewline
\ \ \isacommand{have}\isamarkupfalse%
\ {\isachardoublequoteopen}{\isacharbraceleft}{\kern0pt}{\isasymlangle}check{\isacharparenleft}{\kern0pt}x{\isacharparenright}{\kern0pt}{\isacharcomma}{\kern0pt}one{\isasymrangle}{\isacharbraceright}{\kern0pt}\ {\isasymin}\ M{\isachardoublequoteclose}\isanewline
\ \ \ \ \isacommand{using}\isamarkupfalse%
\ singletonM\ \isacommand{by}\isamarkupfalse%
\ simp\isanewline
\ \ \isacommand{with}\isamarkupfalse%
\ {\isacartoucheopen}check{\isacharparenleft}{\kern0pt}x{\isacharparenright}{\kern0pt}{\isasymin}M{\isacartoucheclose}\isanewline
\ \ \isacommand{have}\isamarkupfalse%
\ {\isachardoublequoteopen}check{\isacharparenleft}{\kern0pt}x{\isacharparenright}{\kern0pt}\ {\isasymunion}\ {\isacharbraceleft}{\kern0pt}{\isasymlangle}check{\isacharparenleft}{\kern0pt}x{\isacharparenright}{\kern0pt}{\isacharcomma}{\kern0pt}one{\isasymrangle}{\isacharbraceright}{\kern0pt}\ {\isasymin}\ M{\isachardoublequoteclose}\isanewline
\ \ \ \ \isacommand{using}\isamarkupfalse%
\ Un{\isacharunderscore}{\kern0pt}closed\ \isacommand{by}\isamarkupfalse%
\ simp\isanewline
\ \ \isacommand{then}\isamarkupfalse%
\ \isacommand{show}\isamarkupfalse%
\ {\isacharquery}{\kern0pt}case\ \isacommand{using}\isamarkupfalse%
\ {\isacartoucheopen}x{\isasymin}nat{\isacartoucheclose}\ def{\isacharunderscore}{\kern0pt}checkS\ \isacommand{by}\isamarkupfalse%
\ simp\isanewline
\isacommand{qed}\isamarkupfalse%
%
\endisatagproof
{\isafoldproof}%
%
\isadelimproof
\isanewline
%
\endisadelimproof
\isanewline
\isanewline
\isacommand{definition}\isamarkupfalse%
\isanewline
\ \ PHcheck\ {\isacharcolon}{\kern0pt}{\isacharcolon}{\kern0pt}\ {\isachardoublequoteopen}{\isacharbrackleft}{\kern0pt}i{\isacharcomma}{\kern0pt}i{\isacharcomma}{\kern0pt}i{\isacharcomma}{\kern0pt}i{\isacharbrackright}{\kern0pt}\ {\isasymRightarrow}\ o{\isachardoublequoteclose}\ \isakeyword{where}\isanewline
\ \ {\isachardoublequoteopen}PHcheck{\isacharparenleft}{\kern0pt}o{\isacharcomma}{\kern0pt}f{\isacharcomma}{\kern0pt}y{\isacharcomma}{\kern0pt}p{\isacharparenright}{\kern0pt}\ {\isasymequiv}\ p{\isasymin}M\ {\isasymand}\ {\isacharparenleft}{\kern0pt}{\isasymexists}fy{\isacharbrackleft}{\kern0pt}{\isacharhash}{\kern0pt}{\isacharhash}{\kern0pt}M{\isacharbrackright}{\kern0pt}{\isachardot}{\kern0pt}\ fun{\isacharunderscore}{\kern0pt}apply{\isacharparenleft}{\kern0pt}{\isacharhash}{\kern0pt}{\isacharhash}{\kern0pt}M{\isacharcomma}{\kern0pt}f{\isacharcomma}{\kern0pt}y{\isacharcomma}{\kern0pt}fy{\isacharparenright}{\kern0pt}\ {\isasymand}\ pair{\isacharparenleft}{\kern0pt}{\isacharhash}{\kern0pt}{\isacharhash}{\kern0pt}M{\isacharcomma}{\kern0pt}fy{\isacharcomma}{\kern0pt}o{\isacharcomma}{\kern0pt}p{\isacharparenright}{\kern0pt}{\isacharparenright}{\kern0pt}{\isachardoublequoteclose}\isanewline
\isanewline
\isacommand{definition}\isamarkupfalse%
\isanewline
\ \ is{\isacharunderscore}{\kern0pt}Hcheck\ {\isacharcolon}{\kern0pt}{\isacharcolon}{\kern0pt}\ {\isachardoublequoteopen}{\isacharbrackleft}{\kern0pt}i{\isacharcomma}{\kern0pt}i{\isacharcomma}{\kern0pt}i{\isacharcomma}{\kern0pt}i{\isacharbrackright}{\kern0pt}\ {\isasymRightarrow}\ o{\isachardoublequoteclose}\ \isakeyword{where}\isanewline
\ \ {\isachardoublequoteopen}is{\isacharunderscore}{\kern0pt}Hcheck{\isacharparenleft}{\kern0pt}o{\isacharcomma}{\kern0pt}z{\isacharcomma}{\kern0pt}f{\isacharcomma}{\kern0pt}hc{\isacharparenright}{\kern0pt}\ \ {\isasymequiv}\ is{\isacharunderscore}{\kern0pt}Replace{\isacharparenleft}{\kern0pt}{\isacharhash}{\kern0pt}{\isacharhash}{\kern0pt}M{\isacharcomma}{\kern0pt}z{\isacharcomma}{\kern0pt}PHcheck{\isacharparenleft}{\kern0pt}o{\isacharcomma}{\kern0pt}f{\isacharparenright}{\kern0pt}{\isacharcomma}{\kern0pt}hc{\isacharparenright}{\kern0pt}{\isachardoublequoteclose}\isanewline
\isanewline
\isacommand{lemma}\isamarkupfalse%
\ one{\isacharunderscore}{\kern0pt}in{\isacharunderscore}{\kern0pt}M{\isacharcolon}{\kern0pt}\ {\isachardoublequoteopen}one\ {\isasymin}\ M{\isachardoublequoteclose}\isanewline
%
\isadelimproof
\ \ %
\endisadelimproof
%
\isatagproof
\isacommand{by}\isamarkupfalse%
\ {\isacharparenleft}{\kern0pt}insert\ one{\isacharunderscore}{\kern0pt}in{\isacharunderscore}{\kern0pt}P\ P{\isacharunderscore}{\kern0pt}in{\isacharunderscore}{\kern0pt}M{\isacharcomma}{\kern0pt}\ simp\ add{\isacharcolon}{\kern0pt}\ transitivity{\isacharparenright}{\kern0pt}%
\endisatagproof
{\isafoldproof}%
%
\isadelimproof
\isanewline
%
\endisadelimproof
\isanewline
\isacommand{lemma}\isamarkupfalse%
\ def{\isacharunderscore}{\kern0pt}PHcheck{\isacharcolon}{\kern0pt}\isanewline
\ \ \isakeyword{assumes}\isanewline
\ \ \ \ {\isachardoublequoteopen}z{\isasymin}M{\isachardoublequoteclose}\ {\isachardoublequoteopen}f{\isasymin}M{\isachardoublequoteclose}\isanewline
\ \ \isakeyword{shows}\isanewline
\ \ \ \ {\isachardoublequoteopen}Hcheck{\isacharparenleft}{\kern0pt}z{\isacharcomma}{\kern0pt}f{\isacharparenright}{\kern0pt}\ {\isacharequal}{\kern0pt}\ Replace{\isacharparenleft}{\kern0pt}z{\isacharcomma}{\kern0pt}PHcheck{\isacharparenleft}{\kern0pt}one{\isacharcomma}{\kern0pt}f{\isacharparenright}{\kern0pt}{\isacharparenright}{\kern0pt}{\isachardoublequoteclose}\isanewline
%
\isadelimproof
%
\endisadelimproof
%
\isatagproof
\isacommand{proof}\isamarkupfalse%
\ {\isacharminus}{\kern0pt}\isanewline
\ \ \isacommand{from}\isamarkupfalse%
\ assms\isanewline
\ \ \isacommand{have}\isamarkupfalse%
\ {\isachardoublequoteopen}{\isasymlangle}f{\isacharbackquote}{\kern0pt}x{\isacharcomma}{\kern0pt}one{\isasymrangle}\ {\isasymin}\ M{\isachardoublequoteclose}\ {\isachardoublequoteopen}f{\isacharbackquote}{\kern0pt}x{\isasymin}M{\isachardoublequoteclose}\ \isakeyword{if}\ {\isachardoublequoteopen}x{\isasymin}z{\isachardoublequoteclose}\ \isakeyword{for}\ x\isanewline
\ \ \ \ \isacommand{using}\isamarkupfalse%
\ tuples{\isacharunderscore}{\kern0pt}in{\isacharunderscore}{\kern0pt}M\ one{\isacharunderscore}{\kern0pt}in{\isacharunderscore}{\kern0pt}M\ transitivity\ that\ apply{\isacharunderscore}{\kern0pt}closed\ \isacommand{by}\isamarkupfalse%
\ simp{\isacharunderscore}{\kern0pt}all\isanewline
\ \ \isacommand{then}\isamarkupfalse%
\isanewline
\ \ \isacommand{have}\isamarkupfalse%
\ {\isachardoublequoteopen}{\isacharbraceleft}{\kern0pt}y\ {\isachardot}{\kern0pt}\ x\ {\isasymin}\ z{\isacharcomma}{\kern0pt}\ y\ {\isacharequal}{\kern0pt}\ {\isasymlangle}f\ {\isacharbackquote}{\kern0pt}\ x{\isacharcomma}{\kern0pt}\ one{\isasymrangle}{\isacharbraceright}{\kern0pt}\ {\isacharequal}{\kern0pt}\ \ {\isacharbraceleft}{\kern0pt}y\ {\isachardot}{\kern0pt}\ x\ {\isasymin}\ z{\isacharcomma}{\kern0pt}\ y\ {\isacharequal}{\kern0pt}\ {\isasymlangle}f\ {\isacharbackquote}{\kern0pt}\ x{\isacharcomma}{\kern0pt}\ one{\isasymrangle}\ {\isasymand}\ y{\isasymin}M\ {\isasymand}\ f{\isacharbackquote}{\kern0pt}x{\isasymin}M{\isacharbraceright}{\kern0pt}{\isachardoublequoteclose}\isanewline
\ \ \ \ \isacommand{by}\isamarkupfalse%
\ simp\isanewline
\ \ \isacommand{then}\isamarkupfalse%
\isanewline
\ \ \isacommand{show}\isamarkupfalse%
\ {\isacharquery}{\kern0pt}thesis\isanewline
\ \ \ \ \isacommand{using}\isamarkupfalse%
\ {\isacartoucheopen}z{\isasymin}M{\isacartoucheclose}\ {\isacartoucheopen}f{\isasymin}M{\isacartoucheclose}\ transitivity\isanewline
\ \ \ \ \isacommand{unfolding}\isamarkupfalse%
\ Hcheck{\isacharunderscore}{\kern0pt}def\ PHcheck{\isacharunderscore}{\kern0pt}def\ RepFun{\isacharunderscore}{\kern0pt}def\isanewline
\ \ \ \ \isacommand{by}\isamarkupfalse%
\ auto\isanewline
\isacommand{qed}\isamarkupfalse%
%
\endisatagproof
{\isafoldproof}%
%
\isadelimproof
\isanewline
%
\endisadelimproof
\isanewline
\isanewline
\isacommand{definition}\isamarkupfalse%
\isanewline
\ \ PHcheck{\isacharunderscore}{\kern0pt}fm\ {\isacharcolon}{\kern0pt}{\isacharcolon}{\kern0pt}\ {\isachardoublequoteopen}{\isacharbrackleft}{\kern0pt}i{\isacharcomma}{\kern0pt}i{\isacharcomma}{\kern0pt}i{\isacharcomma}{\kern0pt}i{\isacharbrackright}{\kern0pt}\ {\isasymRightarrow}\ i{\isachardoublequoteclose}\ \isakeyword{where}\isanewline
\ \ {\isachardoublequoteopen}PHcheck{\isacharunderscore}{\kern0pt}fm{\isacharparenleft}{\kern0pt}o{\isacharcomma}{\kern0pt}f{\isacharcomma}{\kern0pt}y{\isacharcomma}{\kern0pt}p{\isacharparenright}{\kern0pt}\ {\isasymequiv}\ Exists{\isacharparenleft}{\kern0pt}And{\isacharparenleft}{\kern0pt}fun{\isacharunderscore}{\kern0pt}apply{\isacharunderscore}{\kern0pt}fm{\isacharparenleft}{\kern0pt}succ{\isacharparenleft}{\kern0pt}f{\isacharparenright}{\kern0pt}{\isacharcomma}{\kern0pt}succ{\isacharparenleft}{\kern0pt}y{\isacharparenright}{\kern0pt}{\isacharcomma}{\kern0pt}{\isadigit{0}}{\isacharparenright}{\kern0pt}\isanewline
\ \ \ \ \ \ \ \ \ \ \ \ \ \ \ \ \ \ \ \ \ \ \ \ \ \ \ \ \ \ \ \ \ {\isacharcomma}{\kern0pt}pair{\isacharunderscore}{\kern0pt}fm{\isacharparenleft}{\kern0pt}{\isadigit{0}}{\isacharcomma}{\kern0pt}succ{\isacharparenleft}{\kern0pt}o{\isacharparenright}{\kern0pt}{\isacharcomma}{\kern0pt}succ{\isacharparenleft}{\kern0pt}p{\isacharparenright}{\kern0pt}{\isacharparenright}{\kern0pt}{\isacharparenright}{\kern0pt}{\isacharparenright}{\kern0pt}{\isachardoublequoteclose}\isanewline
\isanewline
\isacommand{lemma}\isamarkupfalse%
\ PHcheck{\isacharunderscore}{\kern0pt}type\ {\isacharbrackleft}{\kern0pt}TC{\isacharbrackright}{\kern0pt}{\isacharcolon}{\kern0pt}\isanewline
\ \ {\isachardoublequoteopen}{\isasymlbrakk}\ x\ {\isasymin}\ nat{\isacharsemicolon}{\kern0pt}\ y\ {\isasymin}\ nat{\isacharsemicolon}{\kern0pt}\ z\ {\isasymin}\ nat{\isacharsemicolon}{\kern0pt}\ u\ {\isasymin}\ nat\ {\isasymrbrakk}\ {\isasymLongrightarrow}\ PHcheck{\isacharunderscore}{\kern0pt}fm{\isacharparenleft}{\kern0pt}x{\isacharcomma}{\kern0pt}y{\isacharcomma}{\kern0pt}z{\isacharcomma}{\kern0pt}u{\isacharparenright}{\kern0pt}\ {\isasymin}\ formula{\isachardoublequoteclose}\isanewline
%
\isadelimproof
\ \ %
\endisadelimproof
%
\isatagproof
\isacommand{by}\isamarkupfalse%
\ {\isacharparenleft}{\kern0pt}simp\ add{\isacharcolon}{\kern0pt}PHcheck{\isacharunderscore}{\kern0pt}fm{\isacharunderscore}{\kern0pt}def{\isacharparenright}{\kern0pt}%
\endisatagproof
{\isafoldproof}%
%
\isadelimproof
\isanewline
%
\endisadelimproof
\isanewline
\isacommand{lemma}\isamarkupfalse%
\ sats{\isacharunderscore}{\kern0pt}PHcheck{\isacharunderscore}{\kern0pt}fm\ {\isacharbrackleft}{\kern0pt}simp{\isacharbrackright}{\kern0pt}{\isacharcolon}{\kern0pt}\isanewline
\ \ {\isachardoublequoteopen}{\isasymlbrakk}\ x\ {\isasymin}\ nat{\isacharsemicolon}{\kern0pt}\ y\ {\isasymin}\ nat{\isacharsemicolon}{\kern0pt}\ z\ {\isasymin}\ nat{\isacharsemicolon}{\kern0pt}\ u\ {\isasymin}\ nat\ {\isacharsemicolon}{\kern0pt}\ env\ {\isasymin}\ list{\isacharparenleft}{\kern0pt}M{\isacharparenright}{\kern0pt}{\isasymrbrakk}\isanewline
\ \ \ \ {\isasymLongrightarrow}\ sats{\isacharparenleft}{\kern0pt}M{\isacharcomma}{\kern0pt}PHcheck{\isacharunderscore}{\kern0pt}fm{\isacharparenleft}{\kern0pt}x{\isacharcomma}{\kern0pt}y{\isacharcomma}{\kern0pt}z{\isacharcomma}{\kern0pt}u{\isacharparenright}{\kern0pt}{\isacharcomma}{\kern0pt}env{\isacharparenright}{\kern0pt}\ {\isasymlongleftrightarrow}\isanewline
\ \ \ \ \ \ \ \ PHcheck{\isacharparenleft}{\kern0pt}nth{\isacharparenleft}{\kern0pt}x{\isacharcomma}{\kern0pt}env{\isacharparenright}{\kern0pt}{\isacharcomma}{\kern0pt}nth{\isacharparenleft}{\kern0pt}y{\isacharcomma}{\kern0pt}env{\isacharparenright}{\kern0pt}{\isacharcomma}{\kern0pt}nth{\isacharparenleft}{\kern0pt}z{\isacharcomma}{\kern0pt}env{\isacharparenright}{\kern0pt}{\isacharcomma}{\kern0pt}nth{\isacharparenleft}{\kern0pt}u{\isacharcomma}{\kern0pt}env{\isacharparenright}{\kern0pt}{\isacharparenright}{\kern0pt}{\isachardoublequoteclose}\isanewline
%
\isadelimproof
\ \ %
\endisadelimproof
%
\isatagproof
\isacommand{using}\isamarkupfalse%
\ zero{\isacharunderscore}{\kern0pt}in{\isacharunderscore}{\kern0pt}M\ Internalizations{\isachardot}{\kern0pt}nth{\isacharunderscore}{\kern0pt}closed\ \isacommand{by}\isamarkupfalse%
\ {\isacharparenleft}{\kern0pt}simp\ add{\isacharcolon}{\kern0pt}\ PHcheck{\isacharunderscore}{\kern0pt}def\ PHcheck{\isacharunderscore}{\kern0pt}fm{\isacharunderscore}{\kern0pt}def{\isacharparenright}{\kern0pt}%
\endisatagproof
{\isafoldproof}%
%
\isadelimproof
\isanewline
%
\endisadelimproof
\isanewline
\isanewline
\isacommand{definition}\isamarkupfalse%
\isanewline
\ \ is{\isacharunderscore}{\kern0pt}Hcheck{\isacharunderscore}{\kern0pt}fm\ {\isacharcolon}{\kern0pt}{\isacharcolon}{\kern0pt}\ {\isachardoublequoteopen}{\isacharbrackleft}{\kern0pt}i{\isacharcomma}{\kern0pt}i{\isacharcomma}{\kern0pt}i{\isacharcomma}{\kern0pt}i{\isacharbrackright}{\kern0pt}\ {\isasymRightarrow}\ i{\isachardoublequoteclose}\ \isakeyword{where}\isanewline
\ \ {\isachardoublequoteopen}is{\isacharunderscore}{\kern0pt}Hcheck{\isacharunderscore}{\kern0pt}fm{\isacharparenleft}{\kern0pt}o{\isacharcomma}{\kern0pt}z{\isacharcomma}{\kern0pt}f{\isacharcomma}{\kern0pt}hc{\isacharparenright}{\kern0pt}\ {\isasymequiv}\ Replace{\isacharunderscore}{\kern0pt}fm{\isacharparenleft}{\kern0pt}z{\isacharcomma}{\kern0pt}PHcheck{\isacharunderscore}{\kern0pt}fm{\isacharparenleft}{\kern0pt}succ{\isacharparenleft}{\kern0pt}succ{\isacharparenleft}{\kern0pt}o{\isacharparenright}{\kern0pt}{\isacharparenright}{\kern0pt}{\isacharcomma}{\kern0pt}succ{\isacharparenleft}{\kern0pt}succ{\isacharparenleft}{\kern0pt}f{\isacharparenright}{\kern0pt}{\isacharparenright}{\kern0pt}{\isacharcomma}{\kern0pt}{\isadigit{0}}{\isacharcomma}{\kern0pt}{\isadigit{1}}{\isacharparenright}{\kern0pt}{\isacharcomma}{\kern0pt}hc{\isacharparenright}{\kern0pt}{\isachardoublequoteclose}\isanewline
\isanewline
\isacommand{lemma}\isamarkupfalse%
\ is{\isacharunderscore}{\kern0pt}Hcheck{\isacharunderscore}{\kern0pt}type\ {\isacharbrackleft}{\kern0pt}TC{\isacharbrackright}{\kern0pt}{\isacharcolon}{\kern0pt}\isanewline
\ \ {\isachardoublequoteopen}{\isasymlbrakk}\ x\ {\isasymin}\ nat{\isacharsemicolon}{\kern0pt}\ y\ {\isasymin}\ nat{\isacharsemicolon}{\kern0pt}\ z\ {\isasymin}\ nat{\isacharsemicolon}{\kern0pt}\ u\ {\isasymin}\ nat\ {\isasymrbrakk}\ {\isasymLongrightarrow}\ is{\isacharunderscore}{\kern0pt}Hcheck{\isacharunderscore}{\kern0pt}fm{\isacharparenleft}{\kern0pt}x{\isacharcomma}{\kern0pt}y{\isacharcomma}{\kern0pt}z{\isacharcomma}{\kern0pt}u{\isacharparenright}{\kern0pt}\ {\isasymin}\ formula{\isachardoublequoteclose}\isanewline
%
\isadelimproof
\ \ %
\endisadelimproof
%
\isatagproof
\isacommand{by}\isamarkupfalse%
\ {\isacharparenleft}{\kern0pt}simp\ add{\isacharcolon}{\kern0pt}is{\isacharunderscore}{\kern0pt}Hcheck{\isacharunderscore}{\kern0pt}fm{\isacharunderscore}{\kern0pt}def{\isacharparenright}{\kern0pt}%
\endisatagproof
{\isafoldproof}%
%
\isadelimproof
\isanewline
%
\endisadelimproof
\isanewline
\isacommand{lemma}\isamarkupfalse%
\ sats{\isacharunderscore}{\kern0pt}is{\isacharunderscore}{\kern0pt}Hcheck{\isacharunderscore}{\kern0pt}fm\ {\isacharbrackleft}{\kern0pt}simp{\isacharbrackright}{\kern0pt}{\isacharcolon}{\kern0pt}\isanewline
\ \ {\isachardoublequoteopen}{\isasymlbrakk}\ x\ {\isasymin}\ nat{\isacharsemicolon}{\kern0pt}\ y\ {\isasymin}\ nat{\isacharsemicolon}{\kern0pt}\ z\ {\isasymin}\ nat{\isacharsemicolon}{\kern0pt}\ u\ {\isasymin}\ nat\ {\isacharsemicolon}{\kern0pt}\ env\ {\isasymin}\ list{\isacharparenleft}{\kern0pt}M{\isacharparenright}{\kern0pt}{\isasymrbrakk}\isanewline
\ \ \ \ {\isasymLongrightarrow}\ sats{\isacharparenleft}{\kern0pt}M{\isacharcomma}{\kern0pt}is{\isacharunderscore}{\kern0pt}Hcheck{\isacharunderscore}{\kern0pt}fm{\isacharparenleft}{\kern0pt}x{\isacharcomma}{\kern0pt}y{\isacharcomma}{\kern0pt}z{\isacharcomma}{\kern0pt}u{\isacharparenright}{\kern0pt}{\isacharcomma}{\kern0pt}env{\isacharparenright}{\kern0pt}\ {\isasymlongleftrightarrow}\isanewline
\ \ \ \ \ \ \ \ is{\isacharunderscore}{\kern0pt}Hcheck{\isacharparenleft}{\kern0pt}nth{\isacharparenleft}{\kern0pt}x{\isacharcomma}{\kern0pt}env{\isacharparenright}{\kern0pt}{\isacharcomma}{\kern0pt}nth{\isacharparenleft}{\kern0pt}y{\isacharcomma}{\kern0pt}env{\isacharparenright}{\kern0pt}{\isacharcomma}{\kern0pt}nth{\isacharparenleft}{\kern0pt}z{\isacharcomma}{\kern0pt}env{\isacharparenright}{\kern0pt}{\isacharcomma}{\kern0pt}nth{\isacharparenleft}{\kern0pt}u{\isacharcomma}{\kern0pt}env{\isacharparenright}{\kern0pt}{\isacharparenright}{\kern0pt}{\isachardoublequoteclose}\isanewline
%
\isadelimproof
\ \ %
\endisadelimproof
%
\isatagproof
\isacommand{using}\isamarkupfalse%
\ sats{\isacharunderscore}{\kern0pt}Replace{\isacharunderscore}{\kern0pt}fm\ \isacommand{unfolding}\isamarkupfalse%
\ is{\isacharunderscore}{\kern0pt}Hcheck{\isacharunderscore}{\kern0pt}def\ is{\isacharunderscore}{\kern0pt}Hcheck{\isacharunderscore}{\kern0pt}fm{\isacharunderscore}{\kern0pt}def\isanewline
\ \ \isacommand{by}\isamarkupfalse%
\ simp%
\endisatagproof
{\isafoldproof}%
%
\isadelimproof
\isanewline
%
\endisadelimproof
\isanewline
\isanewline
\isanewline
\isacommand{lemma}\isamarkupfalse%
\ wfrec{\isacharunderscore}{\kern0pt}Hcheck\ {\isacharcolon}{\kern0pt}\isanewline
\ \ \isakeyword{assumes}\isanewline
\ \ \ \ {\isachardoublequoteopen}X{\isasymin}M{\isachardoublequoteclose}\isanewline
\ \ \isakeyword{shows}\isanewline
\ \ \ \ {\isachardoublequoteopen}wfrec{\isacharunderscore}{\kern0pt}replacement{\isacharparenleft}{\kern0pt}{\isacharhash}{\kern0pt}{\isacharhash}{\kern0pt}M{\isacharcomma}{\kern0pt}is{\isacharunderscore}{\kern0pt}Hcheck{\isacharparenleft}{\kern0pt}one{\isacharparenright}{\kern0pt}{\isacharcomma}{\kern0pt}rcheck{\isacharparenleft}{\kern0pt}X{\isacharparenright}{\kern0pt}{\isacharparenright}{\kern0pt}{\isachardoublequoteclose}\isanewline
%
\isadelimproof
%
\endisadelimproof
%
\isatagproof
\isacommand{proof}\isamarkupfalse%
\ {\isacharminus}{\kern0pt}\isanewline
\ \ \isacommand{have}\isamarkupfalse%
\ {\isachardoublequoteopen}is{\isacharunderscore}{\kern0pt}Hcheck{\isacharparenleft}{\kern0pt}one{\isacharcomma}{\kern0pt}a{\isacharcomma}{\kern0pt}b{\isacharcomma}{\kern0pt}c{\isacharparenright}{\kern0pt}\ {\isasymlongleftrightarrow}\isanewline
\ \ \ \ \ \ \ \ sats{\isacharparenleft}{\kern0pt}M{\isacharcomma}{\kern0pt}is{\isacharunderscore}{\kern0pt}Hcheck{\isacharunderscore}{\kern0pt}fm{\isacharparenleft}{\kern0pt}{\isadigit{8}}{\isacharcomma}{\kern0pt}{\isadigit{2}}{\isacharcomma}{\kern0pt}{\isadigit{1}}{\isacharcomma}{\kern0pt}{\isadigit{0}}{\isacharparenright}{\kern0pt}{\isacharcomma}{\kern0pt}{\isacharbrackleft}{\kern0pt}c{\isacharcomma}{\kern0pt}b{\isacharcomma}{\kern0pt}a{\isacharcomma}{\kern0pt}d{\isacharcomma}{\kern0pt}e{\isacharcomma}{\kern0pt}y{\isacharcomma}{\kern0pt}x{\isacharcomma}{\kern0pt}z{\isacharcomma}{\kern0pt}one{\isacharcomma}{\kern0pt}rcheck{\isacharparenleft}{\kern0pt}x{\isacharparenright}{\kern0pt}{\isacharbrackright}{\kern0pt}{\isacharparenright}{\kern0pt}{\isachardoublequoteclose}\isanewline
\ \ \ \ \isakeyword{if}\ {\isachardoublequoteopen}a{\isasymin}M{\isachardoublequoteclose}\ {\isachardoublequoteopen}b{\isasymin}M{\isachardoublequoteclose}\ {\isachardoublequoteopen}c{\isasymin}M{\isachardoublequoteclose}\ {\isachardoublequoteopen}d{\isasymin}M{\isachardoublequoteclose}\ {\isachardoublequoteopen}e{\isasymin}M{\isachardoublequoteclose}\ {\isachardoublequoteopen}y{\isasymin}M{\isachardoublequoteclose}\ {\isachardoublequoteopen}x{\isasymin}M{\isachardoublequoteclose}\ {\isachardoublequoteopen}z{\isasymin}M{\isachardoublequoteclose}\isanewline
\ \ \ \ \isakeyword{for}\ a\ b\ c\ d\ e\ y\ x\ z\isanewline
\ \ \ \ \isacommand{using}\isamarkupfalse%
\ that\ one{\isacharunderscore}{\kern0pt}in{\isacharunderscore}{\kern0pt}M\ {\isacartoucheopen}X{\isasymin}M{\isacartoucheclose}\ rcheck{\isacharunderscore}{\kern0pt}in{\isacharunderscore}{\kern0pt}M\ \isacommand{by}\isamarkupfalse%
\ simp\isanewline
\ \ \isacommand{then}\isamarkupfalse%
\ \isacommand{have}\isamarkupfalse%
\ {\isadigit{1}}{\isacharcolon}{\kern0pt}{\isachardoublequoteopen}sats{\isacharparenleft}{\kern0pt}M{\isacharcomma}{\kern0pt}is{\isacharunderscore}{\kern0pt}wfrec{\isacharunderscore}{\kern0pt}fm{\isacharparenleft}{\kern0pt}is{\isacharunderscore}{\kern0pt}Hcheck{\isacharunderscore}{\kern0pt}fm{\isacharparenleft}{\kern0pt}{\isadigit{8}}{\isacharcomma}{\kern0pt}{\isadigit{2}}{\isacharcomma}{\kern0pt}{\isadigit{1}}{\isacharcomma}{\kern0pt}{\isadigit{0}}{\isacharparenright}{\kern0pt}{\isacharcomma}{\kern0pt}{\isadigit{4}}{\isacharcomma}{\kern0pt}{\isadigit{1}}{\isacharcomma}{\kern0pt}{\isadigit{0}}{\isacharparenright}{\kern0pt}{\isacharcomma}{\kern0pt}\isanewline
\ \ \ \ \ \ \ \ \ \ \ \ \ \ \ \ \ \ \ \ {\isacharbrackleft}{\kern0pt}y{\isacharcomma}{\kern0pt}x{\isacharcomma}{\kern0pt}z{\isacharcomma}{\kern0pt}one{\isacharcomma}{\kern0pt}rcheck{\isacharparenleft}{\kern0pt}X{\isacharparenright}{\kern0pt}{\isacharbrackright}{\kern0pt}{\isacharparenright}{\kern0pt}\ {\isasymlongleftrightarrow}\isanewline
\ \ \ \ \ \ \ \ \ \ \ \ is{\isacharunderscore}{\kern0pt}wfrec{\isacharparenleft}{\kern0pt}{\isacharhash}{\kern0pt}{\isacharhash}{\kern0pt}M{\isacharcomma}{\kern0pt}\ is{\isacharunderscore}{\kern0pt}Hcheck{\isacharparenleft}{\kern0pt}one{\isacharparenright}{\kern0pt}{\isacharcomma}{\kern0pt}rcheck{\isacharparenleft}{\kern0pt}X{\isacharparenright}{\kern0pt}{\isacharcomma}{\kern0pt}\ x{\isacharcomma}{\kern0pt}\ y{\isacharparenright}{\kern0pt}{\isachardoublequoteclose}\isanewline
\ \ \ \ \isakeyword{if}\ {\isachardoublequoteopen}x{\isasymin}M{\isachardoublequoteclose}\ {\isachardoublequoteopen}y{\isasymin}M{\isachardoublequoteclose}\ {\isachardoublequoteopen}z{\isasymin}M{\isachardoublequoteclose}\ \isakeyword{for}\ x\ y\ z\isanewline
\ \ \ \ \isacommand{using}\isamarkupfalse%
\ \ that\ sats{\isacharunderscore}{\kern0pt}is{\isacharunderscore}{\kern0pt}wfrec{\isacharunderscore}{\kern0pt}fm\ {\isacartoucheopen}X{\isasymin}M{\isacartoucheclose}\ rcheck{\isacharunderscore}{\kern0pt}in{\isacharunderscore}{\kern0pt}M\ one{\isacharunderscore}{\kern0pt}in{\isacharunderscore}{\kern0pt}M\ \isacommand{by}\isamarkupfalse%
\ simp\isanewline
\ \ \isacommand{let}\isamarkupfalse%
\isanewline
\ \ \ \ {\isacharquery}{\kern0pt}f{\isacharequal}{\kern0pt}{\isachardoublequoteopen}Exists{\isacharparenleft}{\kern0pt}And{\isacharparenleft}{\kern0pt}pair{\isacharunderscore}{\kern0pt}fm{\isacharparenleft}{\kern0pt}{\isadigit{1}}{\isacharcomma}{\kern0pt}{\isadigit{0}}{\isacharcomma}{\kern0pt}{\isadigit{2}}{\isacharparenright}{\kern0pt}{\isacharcomma}{\kern0pt}\isanewline
\ \ \ \ \ \ \ \ \ \ \ \ \ \ \ is{\isacharunderscore}{\kern0pt}wfrec{\isacharunderscore}{\kern0pt}fm{\isacharparenleft}{\kern0pt}is{\isacharunderscore}{\kern0pt}Hcheck{\isacharunderscore}{\kern0pt}fm{\isacharparenleft}{\kern0pt}{\isadigit{8}}{\isacharcomma}{\kern0pt}{\isadigit{2}}{\isacharcomma}{\kern0pt}{\isadigit{1}}{\isacharcomma}{\kern0pt}{\isadigit{0}}{\isacharparenright}{\kern0pt}{\isacharcomma}{\kern0pt}{\isadigit{4}}{\isacharcomma}{\kern0pt}{\isadigit{1}}{\isacharcomma}{\kern0pt}{\isadigit{0}}{\isacharparenright}{\kern0pt}{\isacharparenright}{\kern0pt}{\isacharparenright}{\kern0pt}{\isachardoublequoteclose}\isanewline
\ \ \isacommand{have}\isamarkupfalse%
\ satsf{\isacharcolon}{\kern0pt}{\isachardoublequoteopen}sats{\isacharparenleft}{\kern0pt}M{\isacharcomma}{\kern0pt}\ {\isacharquery}{\kern0pt}f{\isacharcomma}{\kern0pt}\ {\isacharbrackleft}{\kern0pt}x{\isacharcomma}{\kern0pt}z{\isacharcomma}{\kern0pt}one{\isacharcomma}{\kern0pt}rcheck{\isacharparenleft}{\kern0pt}X{\isacharparenright}{\kern0pt}{\isacharbrackright}{\kern0pt}{\isacharparenright}{\kern0pt}\ {\isasymlongleftrightarrow}\isanewline
\ \ \ \ \ \ \ \ \ \ \ \ \ \ {\isacharparenleft}{\kern0pt}{\isasymexists}y{\isasymin}M{\isachardot}{\kern0pt}\ pair{\isacharparenleft}{\kern0pt}{\isacharhash}{\kern0pt}{\isacharhash}{\kern0pt}M{\isacharcomma}{\kern0pt}x{\isacharcomma}{\kern0pt}y{\isacharcomma}{\kern0pt}z{\isacharparenright}{\kern0pt}\ {\isacharampersand}{\kern0pt}\ is{\isacharunderscore}{\kern0pt}wfrec{\isacharparenleft}{\kern0pt}{\isacharhash}{\kern0pt}{\isacharhash}{\kern0pt}M{\isacharcomma}{\kern0pt}\ is{\isacharunderscore}{\kern0pt}Hcheck{\isacharparenleft}{\kern0pt}one{\isacharparenright}{\kern0pt}{\isacharcomma}{\kern0pt}rcheck{\isacharparenleft}{\kern0pt}X{\isacharparenright}{\kern0pt}{\isacharcomma}{\kern0pt}\ x{\isacharcomma}{\kern0pt}\ y{\isacharparenright}{\kern0pt}{\isacharparenright}{\kern0pt}{\isachardoublequoteclose}\isanewline
\ \ \ \ \isakeyword{if}\ {\isachardoublequoteopen}x{\isasymin}M{\isachardoublequoteclose}\ {\isachardoublequoteopen}z{\isasymin}M{\isachardoublequoteclose}\ \isakeyword{for}\ x\ z\isanewline
\ \ \ \ \isacommand{using}\isamarkupfalse%
\ that\ {\isadigit{1}}\ {\isacartoucheopen}X{\isasymin}M{\isacartoucheclose}\ rcheck{\isacharunderscore}{\kern0pt}in{\isacharunderscore}{\kern0pt}M\ one{\isacharunderscore}{\kern0pt}in{\isacharunderscore}{\kern0pt}M\ \isacommand{by}\isamarkupfalse%
\ {\isacharparenleft}{\kern0pt}simp\ del{\isacharcolon}{\kern0pt}pair{\isacharunderscore}{\kern0pt}abs{\isacharparenright}{\kern0pt}\isanewline
\ \ \isacommand{have}\isamarkupfalse%
\ artyf{\isacharcolon}{\kern0pt}{\isachardoublequoteopen}arity{\isacharparenleft}{\kern0pt}{\isacharquery}{\kern0pt}f{\isacharparenright}{\kern0pt}\ {\isacharequal}{\kern0pt}\ {\isadigit{4}}{\isachardoublequoteclose}\isanewline
\ \ \ \ \isacommand{unfolding}\isamarkupfalse%
\ is{\isacharunderscore}{\kern0pt}wfrec{\isacharunderscore}{\kern0pt}fm{\isacharunderscore}{\kern0pt}def\ is{\isacharunderscore}{\kern0pt}Hcheck{\isacharunderscore}{\kern0pt}fm{\isacharunderscore}{\kern0pt}def\ Replace{\isacharunderscore}{\kern0pt}fm{\isacharunderscore}{\kern0pt}def\ PHcheck{\isacharunderscore}{\kern0pt}fm{\isacharunderscore}{\kern0pt}def\isanewline
\ \ \ \ \ \ pair{\isacharunderscore}{\kern0pt}fm{\isacharunderscore}{\kern0pt}def\ upair{\isacharunderscore}{\kern0pt}fm{\isacharunderscore}{\kern0pt}def\ is{\isacharunderscore}{\kern0pt}recfun{\isacharunderscore}{\kern0pt}fm{\isacharunderscore}{\kern0pt}def\ fun{\isacharunderscore}{\kern0pt}apply{\isacharunderscore}{\kern0pt}fm{\isacharunderscore}{\kern0pt}def\ big{\isacharunderscore}{\kern0pt}union{\isacharunderscore}{\kern0pt}fm{\isacharunderscore}{\kern0pt}def\isanewline
\ \ \ \ \ \ pre{\isacharunderscore}{\kern0pt}image{\isacharunderscore}{\kern0pt}fm{\isacharunderscore}{\kern0pt}def\ restriction{\isacharunderscore}{\kern0pt}fm{\isacharunderscore}{\kern0pt}def\ image{\isacharunderscore}{\kern0pt}fm{\isacharunderscore}{\kern0pt}def\isanewline
\ \ \ \ \isacommand{by}\isamarkupfalse%
\ {\isacharparenleft}{\kern0pt}simp\ add{\isacharcolon}{\kern0pt}nat{\isacharunderscore}{\kern0pt}simp{\isacharunderscore}{\kern0pt}union{\isacharparenright}{\kern0pt}\isanewline
\ \ \isacommand{then}\isamarkupfalse%
\isanewline
\ \ \isacommand{have}\isamarkupfalse%
\ {\isachardoublequoteopen}strong{\isacharunderscore}{\kern0pt}replacement{\isacharparenleft}{\kern0pt}{\isacharhash}{\kern0pt}{\isacharhash}{\kern0pt}M{\isacharcomma}{\kern0pt}{\isasymlambda}x\ z{\isachardot}{\kern0pt}\ sats{\isacharparenleft}{\kern0pt}M{\isacharcomma}{\kern0pt}{\isacharquery}{\kern0pt}f{\isacharcomma}{\kern0pt}{\isacharbrackleft}{\kern0pt}x{\isacharcomma}{\kern0pt}z{\isacharcomma}{\kern0pt}one{\isacharcomma}{\kern0pt}rcheck{\isacharparenleft}{\kern0pt}X{\isacharparenright}{\kern0pt}{\isacharbrackright}{\kern0pt}{\isacharparenright}{\kern0pt}{\isacharparenright}{\kern0pt}{\isachardoublequoteclose}\isanewline
\ \ \ \ \isacommand{using}\isamarkupfalse%
\ replacement{\isacharunderscore}{\kern0pt}ax\ {\isadigit{1}}\ artyf\ {\isacartoucheopen}X{\isasymin}M{\isacartoucheclose}\ rcheck{\isacharunderscore}{\kern0pt}in{\isacharunderscore}{\kern0pt}M\ one{\isacharunderscore}{\kern0pt}in{\isacharunderscore}{\kern0pt}M\ \isacommand{by}\isamarkupfalse%
\ simp\isanewline
\ \ \isacommand{then}\isamarkupfalse%
\isanewline
\ \ \isacommand{have}\isamarkupfalse%
\ {\isachardoublequoteopen}strong{\isacharunderscore}{\kern0pt}replacement{\isacharparenleft}{\kern0pt}{\isacharhash}{\kern0pt}{\isacharhash}{\kern0pt}M{\isacharcomma}{\kern0pt}{\isasymlambda}x\ z{\isachardot}{\kern0pt}\isanewline
\ \ \ \ \ \ \ \ \ \ {\isasymexists}y{\isasymin}M{\isachardot}{\kern0pt}\ pair{\isacharparenleft}{\kern0pt}{\isacharhash}{\kern0pt}{\isacharhash}{\kern0pt}M{\isacharcomma}{\kern0pt}x{\isacharcomma}{\kern0pt}y{\isacharcomma}{\kern0pt}z{\isacharparenright}{\kern0pt}\ {\isacharampersand}{\kern0pt}\ is{\isacharunderscore}{\kern0pt}wfrec{\isacharparenleft}{\kern0pt}{\isacharhash}{\kern0pt}{\isacharhash}{\kern0pt}M{\isacharcomma}{\kern0pt}\ is{\isacharunderscore}{\kern0pt}Hcheck{\isacharparenleft}{\kern0pt}one{\isacharparenright}{\kern0pt}{\isacharcomma}{\kern0pt}rcheck{\isacharparenleft}{\kern0pt}X{\isacharparenright}{\kern0pt}{\isacharcomma}{\kern0pt}\ x{\isacharcomma}{\kern0pt}\ y{\isacharparenright}{\kern0pt}{\isacharparenright}{\kern0pt}{\isachardoublequoteclose}\isanewline
\ \ \ \ \isacommand{using}\isamarkupfalse%
\ repl{\isacharunderscore}{\kern0pt}sats{\isacharbrackleft}{\kern0pt}of\ M\ {\isacharquery}{\kern0pt}f\ {\isachardoublequoteopen}{\isacharbrackleft}{\kern0pt}one{\isacharcomma}{\kern0pt}rcheck{\isacharparenleft}{\kern0pt}X{\isacharparenright}{\kern0pt}{\isacharbrackright}{\kern0pt}{\isachardoublequoteclose}{\isacharbrackright}{\kern0pt}\ satsf\ \isacommand{by}\isamarkupfalse%
\ {\isacharparenleft}{\kern0pt}simp\ del{\isacharcolon}{\kern0pt}pair{\isacharunderscore}{\kern0pt}abs{\isacharparenright}{\kern0pt}\isanewline
\ \ \isacommand{then}\isamarkupfalse%
\isanewline
\ \ \isacommand{show}\isamarkupfalse%
\ {\isacharquery}{\kern0pt}thesis\ \isacommand{unfolding}\isamarkupfalse%
\ wfrec{\isacharunderscore}{\kern0pt}replacement{\isacharunderscore}{\kern0pt}def\ \isacommand{by}\isamarkupfalse%
\ simp\isanewline
\isacommand{qed}\isamarkupfalse%
%
\endisatagproof
{\isafoldproof}%
%
\isadelimproof
\isanewline
%
\endisadelimproof
\isanewline
\isacommand{lemma}\isamarkupfalse%
\ repl{\isacharunderscore}{\kern0pt}PHcheck\ {\isacharcolon}{\kern0pt}\isanewline
\ \ \isakeyword{assumes}\isanewline
\ \ \ \ {\isachardoublequoteopen}f{\isasymin}M{\isachardoublequoteclose}\isanewline
\ \ \isakeyword{shows}\isanewline
\ \ \ \ {\isachardoublequoteopen}strong{\isacharunderscore}{\kern0pt}replacement{\isacharparenleft}{\kern0pt}{\isacharhash}{\kern0pt}{\isacharhash}{\kern0pt}M{\isacharcomma}{\kern0pt}PHcheck{\isacharparenleft}{\kern0pt}one{\isacharcomma}{\kern0pt}f{\isacharparenright}{\kern0pt}{\isacharparenright}{\kern0pt}{\isachardoublequoteclose}\isanewline
%
\isadelimproof
%
\endisadelimproof
%
\isatagproof
\isacommand{proof}\isamarkupfalse%
\ {\isacharminus}{\kern0pt}\isanewline
\ \ \isacommand{have}\isamarkupfalse%
\ {\isachardoublequoteopen}arity{\isacharparenleft}{\kern0pt}PHcheck{\isacharunderscore}{\kern0pt}fm{\isacharparenleft}{\kern0pt}{\isadigit{2}}{\isacharcomma}{\kern0pt}{\isadigit{3}}{\isacharcomma}{\kern0pt}{\isadigit{0}}{\isacharcomma}{\kern0pt}{\isadigit{1}}{\isacharparenright}{\kern0pt}{\isacharparenright}{\kern0pt}\ {\isacharequal}{\kern0pt}\ {\isadigit{4}}{\isachardoublequoteclose}\isanewline
\ \ \ \ \isacommand{unfolding}\isamarkupfalse%
\ PHcheck{\isacharunderscore}{\kern0pt}fm{\isacharunderscore}{\kern0pt}def\ fun{\isacharunderscore}{\kern0pt}apply{\isacharunderscore}{\kern0pt}fm{\isacharunderscore}{\kern0pt}def\ big{\isacharunderscore}{\kern0pt}union{\isacharunderscore}{\kern0pt}fm{\isacharunderscore}{\kern0pt}def\ pair{\isacharunderscore}{\kern0pt}fm{\isacharunderscore}{\kern0pt}def\ image{\isacharunderscore}{\kern0pt}fm{\isacharunderscore}{\kern0pt}def\isanewline
\ \ \ \ \ \ upair{\isacharunderscore}{\kern0pt}fm{\isacharunderscore}{\kern0pt}def\isanewline
\ \ \ \ \isacommand{by}\isamarkupfalse%
\ {\isacharparenleft}{\kern0pt}simp\ add{\isacharcolon}{\kern0pt}nat{\isacharunderscore}{\kern0pt}simp{\isacharunderscore}{\kern0pt}union{\isacharparenright}{\kern0pt}\isanewline
\ \ \isacommand{with}\isamarkupfalse%
\ {\isacartoucheopen}f{\isasymin}M{\isacartoucheclose}\isanewline
\ \ \isacommand{have}\isamarkupfalse%
\ {\isachardoublequoteopen}strong{\isacharunderscore}{\kern0pt}replacement{\isacharparenleft}{\kern0pt}{\isacharhash}{\kern0pt}{\isacharhash}{\kern0pt}M{\isacharcomma}{\kern0pt}{\isasymlambda}x\ y{\isachardot}{\kern0pt}\ sats{\isacharparenleft}{\kern0pt}M{\isacharcomma}{\kern0pt}PHcheck{\isacharunderscore}{\kern0pt}fm{\isacharparenleft}{\kern0pt}{\isadigit{2}}{\isacharcomma}{\kern0pt}{\isadigit{3}}{\isacharcomma}{\kern0pt}{\isadigit{0}}{\isacharcomma}{\kern0pt}{\isadigit{1}}{\isacharparenright}{\kern0pt}{\isacharcomma}{\kern0pt}{\isacharbrackleft}{\kern0pt}x{\isacharcomma}{\kern0pt}y{\isacharcomma}{\kern0pt}one{\isacharcomma}{\kern0pt}f{\isacharbrackright}{\kern0pt}{\isacharparenright}{\kern0pt}{\isacharparenright}{\kern0pt}{\isachardoublequoteclose}\isanewline
\ \ \ \ \isacommand{using}\isamarkupfalse%
\ replacement{\isacharunderscore}{\kern0pt}ax\ one{\isacharunderscore}{\kern0pt}in{\isacharunderscore}{\kern0pt}M\ \isacommand{by}\isamarkupfalse%
\ simp\isanewline
\ \ \isacommand{with}\isamarkupfalse%
\ {\isacartoucheopen}f{\isasymin}M{\isacartoucheclose}\isanewline
\ \ \isacommand{show}\isamarkupfalse%
\ {\isacharquery}{\kern0pt}thesis\ \isacommand{using}\isamarkupfalse%
\ one{\isacharunderscore}{\kern0pt}in{\isacharunderscore}{\kern0pt}M\ \isacommand{unfolding}\isamarkupfalse%
\ strong{\isacharunderscore}{\kern0pt}replacement{\isacharunderscore}{\kern0pt}def\ univalent{\isacharunderscore}{\kern0pt}def\ \isacommand{by}\isamarkupfalse%
\ simp\isanewline
\isacommand{qed}\isamarkupfalse%
%
\endisatagproof
{\isafoldproof}%
%
\isadelimproof
\isanewline
%
\endisadelimproof
\isanewline
\isacommand{lemma}\isamarkupfalse%
\ univ{\isacharunderscore}{\kern0pt}PHcheck\ {\isacharcolon}{\kern0pt}\ {\isachardoublequoteopen}{\isasymlbrakk}\ z{\isasymin}M\ {\isacharsemicolon}{\kern0pt}\ f{\isasymin}M\ {\isasymrbrakk}\ {\isasymLongrightarrow}\ univalent{\isacharparenleft}{\kern0pt}{\isacharhash}{\kern0pt}{\isacharhash}{\kern0pt}M{\isacharcomma}{\kern0pt}z{\isacharcomma}{\kern0pt}PHcheck{\isacharparenleft}{\kern0pt}one{\isacharcomma}{\kern0pt}f{\isacharparenright}{\kern0pt}{\isacharparenright}{\kern0pt}{\isachardoublequoteclose}\isanewline
%
\isadelimproof
\ \ %
\endisadelimproof
%
\isatagproof
\isacommand{unfolding}\isamarkupfalse%
\ univalent{\isacharunderscore}{\kern0pt}def\ PHcheck{\isacharunderscore}{\kern0pt}def\ \isacommand{by}\isamarkupfalse%
\ simp%
\endisatagproof
{\isafoldproof}%
%
\isadelimproof
\isanewline
%
\endisadelimproof
\isanewline
\isacommand{lemma}\isamarkupfalse%
\ relation{\isadigit{2}}{\isacharunderscore}{\kern0pt}Hcheck\ {\isacharcolon}{\kern0pt}\isanewline
\ \ {\isachardoublequoteopen}relation{\isadigit{2}}{\isacharparenleft}{\kern0pt}{\isacharhash}{\kern0pt}{\isacharhash}{\kern0pt}M{\isacharcomma}{\kern0pt}is{\isacharunderscore}{\kern0pt}Hcheck{\isacharparenleft}{\kern0pt}one{\isacharparenright}{\kern0pt}{\isacharcomma}{\kern0pt}Hcheck{\isacharparenright}{\kern0pt}{\isachardoublequoteclose}\isanewline
%
\isadelimproof
%
\endisadelimproof
%
\isatagproof
\isacommand{proof}\isamarkupfalse%
\ {\isacharminus}{\kern0pt}\isanewline
\ \ \isacommand{have}\isamarkupfalse%
\ {\isadigit{1}}{\isacharcolon}{\kern0pt}{\isachardoublequoteopen}{\isasymlbrakk}x{\isasymin}z{\isacharsemicolon}{\kern0pt}\ PHcheck{\isacharparenleft}{\kern0pt}one{\isacharcomma}{\kern0pt}f{\isacharcomma}{\kern0pt}x{\isacharcomma}{\kern0pt}y{\isacharparenright}{\kern0pt}\ {\isasymrbrakk}\ {\isasymLongrightarrow}\ {\isacharparenleft}{\kern0pt}{\isacharhash}{\kern0pt}{\isacharhash}{\kern0pt}M{\isacharparenright}{\kern0pt}{\isacharparenleft}{\kern0pt}y{\isacharparenright}{\kern0pt}{\isachardoublequoteclose}\isanewline
\ \ \ \ \isakeyword{if}\ {\isachardoublequoteopen}z{\isasymin}M{\isachardoublequoteclose}\ {\isachardoublequoteopen}f{\isasymin}M{\isachardoublequoteclose}\ \isakeyword{for}\ z\ f\ x\ y\isanewline
\ \ \ \ \isacommand{using}\isamarkupfalse%
\ that\ \isacommand{unfolding}\isamarkupfalse%
\ PHcheck{\isacharunderscore}{\kern0pt}def\ \isacommand{by}\isamarkupfalse%
\ simp\isanewline
\ \ \isacommand{have}\isamarkupfalse%
\ {\isachardoublequoteopen}is{\isacharunderscore}{\kern0pt}Replace{\isacharparenleft}{\kern0pt}{\isacharhash}{\kern0pt}{\isacharhash}{\kern0pt}M{\isacharcomma}{\kern0pt}z{\isacharcomma}{\kern0pt}PHcheck{\isacharparenleft}{\kern0pt}one{\isacharcomma}{\kern0pt}f{\isacharparenright}{\kern0pt}{\isacharcomma}{\kern0pt}hc{\isacharparenright}{\kern0pt}\ {\isasymlongleftrightarrow}\ hc\ {\isacharequal}{\kern0pt}\ Replace{\isacharparenleft}{\kern0pt}z{\isacharcomma}{\kern0pt}PHcheck{\isacharparenleft}{\kern0pt}one{\isacharcomma}{\kern0pt}f{\isacharparenright}{\kern0pt}{\isacharparenright}{\kern0pt}{\isachardoublequoteclose}\isanewline
\ \ \ \ \isakeyword{if}\ {\isachardoublequoteopen}z{\isasymin}M{\isachardoublequoteclose}\ {\isachardoublequoteopen}f{\isasymin}M{\isachardoublequoteclose}\ {\isachardoublequoteopen}hc{\isasymin}M{\isachardoublequoteclose}\ \isakeyword{for}\ z\ f\ hc\isanewline
\ \ \ \ \isacommand{using}\isamarkupfalse%
\ that\ Replace{\isacharunderscore}{\kern0pt}abs{\isacharbrackleft}{\kern0pt}OF\ {\isacharunderscore}{\kern0pt}\ {\isacharunderscore}{\kern0pt}\ univ{\isacharunderscore}{\kern0pt}PHcheck\ {\isadigit{1}}{\isacharbrackright}{\kern0pt}\ \isacommand{by}\isamarkupfalse%
\ simp\isanewline
\ \ \isacommand{with}\isamarkupfalse%
\ def{\isacharunderscore}{\kern0pt}PHcheck\isanewline
\ \ \isacommand{show}\isamarkupfalse%
\ {\isacharquery}{\kern0pt}thesis\isanewline
\ \ \ \ \isacommand{unfolding}\isamarkupfalse%
\ relation{\isadigit{2}}{\isacharunderscore}{\kern0pt}def\ is{\isacharunderscore}{\kern0pt}Hcheck{\isacharunderscore}{\kern0pt}def\ Hcheck{\isacharunderscore}{\kern0pt}def\ \isacommand{by}\isamarkupfalse%
\ simp\isanewline
\isacommand{qed}\isamarkupfalse%
%
\endisatagproof
{\isafoldproof}%
%
\isadelimproof
\isanewline
%
\endisadelimproof
\isanewline
\isacommand{lemma}\isamarkupfalse%
\ PHcheck{\isacharunderscore}{\kern0pt}closed\ {\isacharcolon}{\kern0pt}\isanewline
\ \ {\isachardoublequoteopen}{\isasymlbrakk}z{\isasymin}M\ {\isacharsemicolon}{\kern0pt}\ f{\isasymin}M\ {\isacharsemicolon}{\kern0pt}\ x{\isasymin}z{\isacharsemicolon}{\kern0pt}\ PHcheck{\isacharparenleft}{\kern0pt}one{\isacharcomma}{\kern0pt}f{\isacharcomma}{\kern0pt}x{\isacharcomma}{\kern0pt}y{\isacharparenright}{\kern0pt}\ {\isasymrbrakk}\ {\isasymLongrightarrow}\ {\isacharparenleft}{\kern0pt}{\isacharhash}{\kern0pt}{\isacharhash}{\kern0pt}M{\isacharparenright}{\kern0pt}{\isacharparenleft}{\kern0pt}y{\isacharparenright}{\kern0pt}{\isachardoublequoteclose}\isanewline
%
\isadelimproof
\ \ %
\endisadelimproof
%
\isatagproof
\isacommand{unfolding}\isamarkupfalse%
\ PHcheck{\isacharunderscore}{\kern0pt}def\ \isacommand{by}\isamarkupfalse%
\ simp%
\endisatagproof
{\isafoldproof}%
%
\isadelimproof
\isanewline
%
\endisadelimproof
\isanewline
\isacommand{lemma}\isamarkupfalse%
\ Hcheck{\isacharunderscore}{\kern0pt}closed\ {\isacharcolon}{\kern0pt}\isanewline
\ \ {\isachardoublequoteopen}{\isasymforall}y{\isasymin}M{\isachardot}{\kern0pt}\ {\isasymforall}g{\isasymin}M{\isachardot}{\kern0pt}\ function{\isacharparenleft}{\kern0pt}g{\isacharparenright}{\kern0pt}\ {\isasymlongrightarrow}\ Hcheck{\isacharparenleft}{\kern0pt}y{\isacharcomma}{\kern0pt}g{\isacharparenright}{\kern0pt}{\isasymin}M{\isachardoublequoteclose}\isanewline
%
\isadelimproof
%
\endisadelimproof
%
\isatagproof
\isacommand{proof}\isamarkupfalse%
\ {\isacharminus}{\kern0pt}\isanewline
\ \ \isacommand{have}\isamarkupfalse%
\ {\isachardoublequoteopen}Replace{\isacharparenleft}{\kern0pt}y{\isacharcomma}{\kern0pt}PHcheck{\isacharparenleft}{\kern0pt}one{\isacharcomma}{\kern0pt}f{\isacharparenright}{\kern0pt}{\isacharparenright}{\kern0pt}{\isasymin}M{\isachardoublequoteclose}\ \isakeyword{if}\ {\isachardoublequoteopen}f{\isasymin}M{\isachardoublequoteclose}\ {\isachardoublequoteopen}y{\isasymin}M{\isachardoublequoteclose}\ \isakeyword{for}\ f\ y\isanewline
\ \ \ \ \isacommand{using}\isamarkupfalse%
\ that\ repl{\isacharunderscore}{\kern0pt}PHcheck\ \ PHcheck{\isacharunderscore}{\kern0pt}closed{\isacharbrackleft}{\kern0pt}of\ y\ f{\isacharbrackright}{\kern0pt}\ univ{\isacharunderscore}{\kern0pt}PHcheck\isanewline
\ \ \ \ \ \ strong{\isacharunderscore}{\kern0pt}replacement{\isacharunderscore}{\kern0pt}closed\isanewline
\ \ \ \ \isacommand{by}\isamarkupfalse%
\ {\isacharparenleft}{\kern0pt}simp\ flip{\isacharcolon}{\kern0pt}\ setclass{\isacharunderscore}{\kern0pt}iff{\isacharparenright}{\kern0pt}\isanewline
\ \ \isacommand{then}\isamarkupfalse%
\ \isacommand{show}\isamarkupfalse%
\ {\isacharquery}{\kern0pt}thesis\ \isacommand{using}\isamarkupfalse%
\ def{\isacharunderscore}{\kern0pt}PHcheck\ \isacommand{by}\isamarkupfalse%
\ auto\isanewline
\isacommand{qed}\isamarkupfalse%
%
\endisatagproof
{\isafoldproof}%
%
\isadelimproof
\isanewline
%
\endisadelimproof
\isanewline
\isacommand{lemma}\isamarkupfalse%
\ wf{\isacharunderscore}{\kern0pt}rcheck\ {\isacharcolon}{\kern0pt}\ {\isachardoublequoteopen}x{\isasymin}M\ {\isasymLongrightarrow}\ wf{\isacharparenleft}{\kern0pt}rcheck{\isacharparenleft}{\kern0pt}x{\isacharparenright}{\kern0pt}{\isacharparenright}{\kern0pt}{\isachardoublequoteclose}\isanewline
%
\isadelimproof
\ \ %
\endisadelimproof
%
\isatagproof
\isacommand{unfolding}\isamarkupfalse%
\ rcheck{\isacharunderscore}{\kern0pt}def\ \isacommand{using}\isamarkupfalse%
\ wf{\isacharunderscore}{\kern0pt}trancl{\isacharbrackleft}{\kern0pt}OF\ wf{\isacharunderscore}{\kern0pt}Memrel{\isacharbrackright}{\kern0pt}\ \isacommand{{\isachardot}{\kern0pt}}\isamarkupfalse%
%
\endisatagproof
{\isafoldproof}%
%
\isadelimproof
\isanewline
%
\endisadelimproof
\isanewline
\isacommand{lemma}\isamarkupfalse%
\ trans{\isacharunderscore}{\kern0pt}rcheck\ {\isacharcolon}{\kern0pt}\ {\isachardoublequoteopen}x{\isasymin}M\ {\isasymLongrightarrow}\ trans{\isacharparenleft}{\kern0pt}rcheck{\isacharparenleft}{\kern0pt}x{\isacharparenright}{\kern0pt}{\isacharparenright}{\kern0pt}{\isachardoublequoteclose}\isanewline
%
\isadelimproof
\ \ %
\endisadelimproof
%
\isatagproof
\isacommand{unfolding}\isamarkupfalse%
\ rcheck{\isacharunderscore}{\kern0pt}def\ \isacommand{using}\isamarkupfalse%
\ trans{\isacharunderscore}{\kern0pt}trancl\ \isacommand{{\isachardot}{\kern0pt}}\isamarkupfalse%
%
\endisatagproof
{\isafoldproof}%
%
\isadelimproof
\isanewline
%
\endisadelimproof
\isanewline
\isacommand{lemma}\isamarkupfalse%
\ relation{\isacharunderscore}{\kern0pt}rcheck\ {\isacharcolon}{\kern0pt}\ {\isachardoublequoteopen}x{\isasymin}M\ {\isasymLongrightarrow}\ relation{\isacharparenleft}{\kern0pt}rcheck{\isacharparenleft}{\kern0pt}x{\isacharparenright}{\kern0pt}{\isacharparenright}{\kern0pt}{\isachardoublequoteclose}\isanewline
%
\isadelimproof
\ \ %
\endisadelimproof
%
\isatagproof
\isacommand{unfolding}\isamarkupfalse%
\ rcheck{\isacharunderscore}{\kern0pt}def\ \isacommand{using}\isamarkupfalse%
\ relation{\isacharunderscore}{\kern0pt}trancl\ \isacommand{{\isachardot}{\kern0pt}}\isamarkupfalse%
%
\endisatagproof
{\isafoldproof}%
%
\isadelimproof
\isanewline
%
\endisadelimproof
\isanewline
\isacommand{lemma}\isamarkupfalse%
\ check{\isacharunderscore}{\kern0pt}in{\isacharunderscore}{\kern0pt}M\ {\isacharcolon}{\kern0pt}\ {\isachardoublequoteopen}x{\isasymin}M\ {\isasymLongrightarrow}\ check{\isacharparenleft}{\kern0pt}x{\isacharparenright}{\kern0pt}\ {\isasymin}\ M{\isachardoublequoteclose}\isanewline
%
\isadelimproof
\ \ %
\endisadelimproof
%
\isatagproof
\isacommand{unfolding}\isamarkupfalse%
\ transrec{\isacharunderscore}{\kern0pt}def\isanewline
\ \ \isacommand{using}\isamarkupfalse%
\ wfrec{\isacharunderscore}{\kern0pt}Hcheck{\isacharbrackleft}{\kern0pt}of\ x{\isacharbrackright}{\kern0pt}\ check{\isacharunderscore}{\kern0pt}trancl\ wf{\isacharunderscore}{\kern0pt}rcheck\ trans{\isacharunderscore}{\kern0pt}rcheck\ relation{\isacharunderscore}{\kern0pt}rcheck\ rcheck{\isacharunderscore}{\kern0pt}in{\isacharunderscore}{\kern0pt}M\isanewline
\ \ \ \ Hcheck{\isacharunderscore}{\kern0pt}closed\ relation{\isadigit{2}}{\isacharunderscore}{\kern0pt}Hcheck\ trans{\isacharunderscore}{\kern0pt}wfrec{\isacharunderscore}{\kern0pt}closed{\isacharbrackleft}{\kern0pt}of\ {\isachardoublequoteopen}rcheck{\isacharparenleft}{\kern0pt}x{\isacharparenright}{\kern0pt}{\isachardoublequoteclose}\ x\ {\isachardoublequoteopen}is{\isacharunderscore}{\kern0pt}Hcheck{\isacharparenleft}{\kern0pt}one{\isacharparenright}{\kern0pt}{\isachardoublequoteclose}\ Hcheck{\isacharbrackright}{\kern0pt}\isanewline
\ \ \isacommand{by}\isamarkupfalse%
\ {\isacharparenleft}{\kern0pt}simp\ flip{\isacharcolon}{\kern0pt}\ setclass{\isacharunderscore}{\kern0pt}iff{\isacharparenright}{\kern0pt}%
\endisatagproof
{\isafoldproof}%
%
\isadelimproof
\isanewline
%
\endisadelimproof
\isanewline
\isacommand{end}\isamarkupfalse%
\ \isanewline
\isanewline
\isanewline
\isacommand{definition}\isamarkupfalse%
\isanewline
\ \ is{\isacharunderscore}{\kern0pt}singleton\ {\isacharcolon}{\kern0pt}{\isacharcolon}{\kern0pt}\ {\isachardoublequoteopen}{\isacharbrackleft}{\kern0pt}i{\isasymRightarrow}o{\isacharcomma}{\kern0pt}i{\isacharcomma}{\kern0pt}i{\isacharbrackright}{\kern0pt}\ {\isasymRightarrow}\ o{\isachardoublequoteclose}\ \isakeyword{where}\isanewline
\ \ {\isachardoublequoteopen}is{\isacharunderscore}{\kern0pt}singleton{\isacharparenleft}{\kern0pt}A{\isacharcomma}{\kern0pt}x{\isacharcomma}{\kern0pt}z{\isacharparenright}{\kern0pt}\ {\isasymequiv}\ {\isasymexists}c{\isacharbrackleft}{\kern0pt}A{\isacharbrackright}{\kern0pt}{\isachardot}{\kern0pt}\ empty{\isacharparenleft}{\kern0pt}A{\isacharcomma}{\kern0pt}c{\isacharparenright}{\kern0pt}\ {\isasymand}\ is{\isacharunderscore}{\kern0pt}cons{\isacharparenleft}{\kern0pt}A{\isacharcomma}{\kern0pt}x{\isacharcomma}{\kern0pt}c{\isacharcomma}{\kern0pt}z{\isacharparenright}{\kern0pt}{\isachardoublequoteclose}\isanewline
\isanewline
\isacommand{lemma}\isamarkupfalse%
\ {\isacharparenleft}{\kern0pt}\isakeyword{in}\ M{\isacharunderscore}{\kern0pt}trivial{\isacharparenright}{\kern0pt}\ singleton{\isacharunderscore}{\kern0pt}abs{\isacharbrackleft}{\kern0pt}simp{\isacharbrackright}{\kern0pt}\ {\isacharcolon}{\kern0pt}\ {\isachardoublequoteopen}{\isasymlbrakk}\ M{\isacharparenleft}{\kern0pt}x{\isacharparenright}{\kern0pt}\ {\isacharsemicolon}{\kern0pt}\ M{\isacharparenleft}{\kern0pt}s{\isacharparenright}{\kern0pt}\ {\isasymrbrakk}\ {\isasymLongrightarrow}\ is{\isacharunderscore}{\kern0pt}singleton{\isacharparenleft}{\kern0pt}M{\isacharcomma}{\kern0pt}x{\isacharcomma}{\kern0pt}s{\isacharparenright}{\kern0pt}\ {\isasymlongleftrightarrow}\ s\ {\isacharequal}{\kern0pt}\ {\isacharbraceleft}{\kern0pt}x{\isacharbraceright}{\kern0pt}{\isachardoublequoteclose}\isanewline
%
\isadelimproof
\ \ %
\endisadelimproof
%
\isatagproof
\isacommand{unfolding}\isamarkupfalse%
\ is{\isacharunderscore}{\kern0pt}singleton{\isacharunderscore}{\kern0pt}def\ \isacommand{using}\isamarkupfalse%
\ nonempty\ \isacommand{by}\isamarkupfalse%
\ simp%
\endisatagproof
{\isafoldproof}%
%
\isadelimproof
\isanewline
%
\endisadelimproof
\isanewline
\isacommand{definition}\isamarkupfalse%
\isanewline
\ \ singleton{\isacharunderscore}{\kern0pt}fm\ {\isacharcolon}{\kern0pt}{\isacharcolon}{\kern0pt}\ {\isachardoublequoteopen}{\isacharbrackleft}{\kern0pt}i{\isacharcomma}{\kern0pt}i{\isacharbrackright}{\kern0pt}\ {\isasymRightarrow}\ i{\isachardoublequoteclose}\ \isakeyword{where}\isanewline
\ \ {\isachardoublequoteopen}singleton{\isacharunderscore}{\kern0pt}fm{\isacharparenleft}{\kern0pt}i{\isacharcomma}{\kern0pt}j{\isacharparenright}{\kern0pt}\ {\isasymequiv}\ Exists{\isacharparenleft}{\kern0pt}And{\isacharparenleft}{\kern0pt}empty{\isacharunderscore}{\kern0pt}fm{\isacharparenleft}{\kern0pt}{\isadigit{0}}{\isacharparenright}{\kern0pt}{\isacharcomma}{\kern0pt}\ cons{\isacharunderscore}{\kern0pt}fm{\isacharparenleft}{\kern0pt}succ{\isacharparenleft}{\kern0pt}i{\isacharparenright}{\kern0pt}{\isacharcomma}{\kern0pt}{\isadigit{0}}{\isacharcomma}{\kern0pt}succ{\isacharparenleft}{\kern0pt}j{\isacharparenright}{\kern0pt}{\isacharparenright}{\kern0pt}{\isacharparenright}{\kern0pt}{\isacharparenright}{\kern0pt}{\isachardoublequoteclose}\isanewline
\isanewline
\isacommand{lemma}\isamarkupfalse%
\ singleton{\isacharunderscore}{\kern0pt}type{\isacharbrackleft}{\kern0pt}TC{\isacharbrackright}{\kern0pt}\ {\isacharcolon}{\kern0pt}\ {\isachardoublequoteopen}{\isasymlbrakk}\ x\ {\isasymin}\ nat{\isacharsemicolon}{\kern0pt}\ y\ {\isasymin}\ nat\ {\isasymrbrakk}\ {\isasymLongrightarrow}\ singleton{\isacharunderscore}{\kern0pt}fm{\isacharparenleft}{\kern0pt}x{\isacharcomma}{\kern0pt}y{\isacharparenright}{\kern0pt}\ {\isasymin}\ formula{\isachardoublequoteclose}\isanewline
%
\isadelimproof
\ \ %
\endisadelimproof
%
\isatagproof
\isacommand{unfolding}\isamarkupfalse%
\ singleton{\isacharunderscore}{\kern0pt}fm{\isacharunderscore}{\kern0pt}def\ \isacommand{by}\isamarkupfalse%
\ simp%
\endisatagproof
{\isafoldproof}%
%
\isadelimproof
\isanewline
%
\endisadelimproof
\isanewline
\isacommand{lemma}\isamarkupfalse%
\ is{\isacharunderscore}{\kern0pt}singleton{\isacharunderscore}{\kern0pt}iff{\isacharunderscore}{\kern0pt}sats{\isacharcolon}{\kern0pt}\isanewline
\ \ {\isachardoublequoteopen}{\isasymlbrakk}\ nth{\isacharparenleft}{\kern0pt}i{\isacharcomma}{\kern0pt}env{\isacharparenright}{\kern0pt}\ {\isacharequal}{\kern0pt}\ x{\isacharsemicolon}{\kern0pt}\ nth{\isacharparenleft}{\kern0pt}j{\isacharcomma}{\kern0pt}env{\isacharparenright}{\kern0pt}\ {\isacharequal}{\kern0pt}\ y{\isacharsemicolon}{\kern0pt}\isanewline
\ \ \ \ \ \ \ \ \ \ i\ {\isasymin}\ nat{\isacharsemicolon}{\kern0pt}\ j{\isasymin}nat\ {\isacharsemicolon}{\kern0pt}\ env\ {\isasymin}\ list{\isacharparenleft}{\kern0pt}A{\isacharparenright}{\kern0pt}{\isasymrbrakk}\isanewline
\ \ \ \ \ \ \ {\isasymLongrightarrow}\ is{\isacharunderscore}{\kern0pt}singleton{\isacharparenleft}{\kern0pt}{\isacharhash}{\kern0pt}{\isacharhash}{\kern0pt}A{\isacharcomma}{\kern0pt}x{\isacharcomma}{\kern0pt}y{\isacharparenright}{\kern0pt}\ {\isasymlongleftrightarrow}\ sats{\isacharparenleft}{\kern0pt}A{\isacharcomma}{\kern0pt}\ singleton{\isacharunderscore}{\kern0pt}fm{\isacharparenleft}{\kern0pt}i{\isacharcomma}{\kern0pt}j{\isacharparenright}{\kern0pt}{\isacharcomma}{\kern0pt}\ env{\isacharparenright}{\kern0pt}{\isachardoublequoteclose}\isanewline
%
\isadelimproof
\ \ %
\endisadelimproof
%
\isatagproof
\isacommand{unfolding}\isamarkupfalse%
\ is{\isacharunderscore}{\kern0pt}singleton{\isacharunderscore}{\kern0pt}def\ singleton{\isacharunderscore}{\kern0pt}fm{\isacharunderscore}{\kern0pt}def\ \isacommand{by}\isamarkupfalse%
\ simp%
\endisatagproof
{\isafoldproof}%
%
\isadelimproof
\isanewline
%
\endisadelimproof
\isanewline
\isacommand{context}\isamarkupfalse%
\ forcing{\isacharunderscore}{\kern0pt}data\ \isakeyword{begin}\isanewline
\isanewline
\isanewline
\isacommand{definition}\isamarkupfalse%
\isanewline
\ \ is{\isacharunderscore}{\kern0pt}rcheck\ {\isacharcolon}{\kern0pt}{\isacharcolon}{\kern0pt}\ {\isachardoublequoteopen}{\isacharbrackleft}{\kern0pt}i{\isacharcomma}{\kern0pt}i{\isacharbrackright}{\kern0pt}\ {\isasymRightarrow}\ o{\isachardoublequoteclose}\ \isakeyword{where}\isanewline
\ \ {\isachardoublequoteopen}is{\isacharunderscore}{\kern0pt}rcheck{\isacharparenleft}{\kern0pt}x{\isacharcomma}{\kern0pt}z{\isacharparenright}{\kern0pt}\ {\isasymequiv}\ {\isasymexists}r{\isasymin}M{\isachardot}{\kern0pt}\ tran{\isacharunderscore}{\kern0pt}closure{\isacharparenleft}{\kern0pt}{\isacharhash}{\kern0pt}{\isacharhash}{\kern0pt}M{\isacharcomma}{\kern0pt}r{\isacharcomma}{\kern0pt}z{\isacharparenright}{\kern0pt}\ {\isasymand}\ {\isacharparenleft}{\kern0pt}{\isasymexists}ec{\isasymin}M{\isachardot}{\kern0pt}\ membership{\isacharparenleft}{\kern0pt}{\isacharhash}{\kern0pt}{\isacharhash}{\kern0pt}M{\isacharcomma}{\kern0pt}ec{\isacharcomma}{\kern0pt}r{\isacharparenright}{\kern0pt}\ {\isasymand}\isanewline
\ \ \ \ \ \ \ \ \ \ \ \ \ \ \ \ \ \ \ \ \ \ \ \ \ \ \ {\isacharparenleft}{\kern0pt}{\isasymexists}s{\isasymin}M{\isachardot}{\kern0pt}\ is{\isacharunderscore}{\kern0pt}singleton{\isacharparenleft}{\kern0pt}{\isacharhash}{\kern0pt}{\isacharhash}{\kern0pt}M{\isacharcomma}{\kern0pt}x{\isacharcomma}{\kern0pt}s{\isacharparenright}{\kern0pt}\ {\isasymand}\ \ is{\isacharunderscore}{\kern0pt}eclose{\isacharparenleft}{\kern0pt}{\isacharhash}{\kern0pt}{\isacharhash}{\kern0pt}M{\isacharcomma}{\kern0pt}s{\isacharcomma}{\kern0pt}ec{\isacharparenright}{\kern0pt}{\isacharparenright}{\kern0pt}{\isacharparenright}{\kern0pt}{\isachardoublequoteclose}\isanewline
\isanewline
\isacommand{lemma}\isamarkupfalse%
\ rcheck{\isacharunderscore}{\kern0pt}abs\ {\isacharcolon}{\kern0pt}\isanewline
\ \ {\isachardoublequoteopen}{\isasymlbrakk}\ x{\isasymin}M\ {\isacharsemicolon}{\kern0pt}\ r{\isasymin}M\ {\isasymrbrakk}\ {\isasymLongrightarrow}\ is{\isacharunderscore}{\kern0pt}rcheck{\isacharparenleft}{\kern0pt}x{\isacharcomma}{\kern0pt}r{\isacharparenright}{\kern0pt}\ {\isasymlongleftrightarrow}\ r\ {\isacharequal}{\kern0pt}\ rcheck{\isacharparenleft}{\kern0pt}x{\isacharparenright}{\kern0pt}{\isachardoublequoteclose}\isanewline
%
\isadelimproof
\ \ %
\endisadelimproof
%
\isatagproof
\isacommand{unfolding}\isamarkupfalse%
\ rcheck{\isacharunderscore}{\kern0pt}def\ is{\isacharunderscore}{\kern0pt}rcheck{\isacharunderscore}{\kern0pt}def\isanewline
\ \ \isacommand{using}\isamarkupfalse%
\ singletonM\ trancl{\isacharunderscore}{\kern0pt}closed\ Memrel{\isacharunderscore}{\kern0pt}closed\ eclose{\isacharunderscore}{\kern0pt}closed\ \isacommand{by}\isamarkupfalse%
\ simp%
\endisatagproof
{\isafoldproof}%
%
\isadelimproof
\isanewline
%
\endisadelimproof
\isanewline
\isacommand{schematic{\isacharunderscore}{\kern0pt}goal}\isamarkupfalse%
\ rcheck{\isacharunderscore}{\kern0pt}fm{\isacharunderscore}{\kern0pt}auto{\isacharcolon}{\kern0pt}\isanewline
\ \ \isakeyword{assumes}\isanewline
\ \ \ \ {\isachardoublequoteopen}i\ {\isasymin}\ nat{\isachardoublequoteclose}\ {\isachardoublequoteopen}j\ {\isasymin}\ nat{\isachardoublequoteclose}\ {\isachardoublequoteopen}env\ {\isasymin}\ list{\isacharparenleft}{\kern0pt}M{\isacharparenright}{\kern0pt}{\isachardoublequoteclose}\isanewline
\ \ \isakeyword{shows}\isanewline
\ \ \ \ {\isachardoublequoteopen}is{\isacharunderscore}{\kern0pt}rcheck{\isacharparenleft}{\kern0pt}nth{\isacharparenleft}{\kern0pt}i{\isacharcomma}{\kern0pt}env{\isacharparenright}{\kern0pt}{\isacharcomma}{\kern0pt}nth{\isacharparenleft}{\kern0pt}j{\isacharcomma}{\kern0pt}env{\isacharparenright}{\kern0pt}{\isacharparenright}{\kern0pt}\ {\isasymlongleftrightarrow}\ sats{\isacharparenleft}{\kern0pt}M{\isacharcomma}{\kern0pt}{\isacharquery}{\kern0pt}rch{\isacharparenleft}{\kern0pt}i{\isacharcomma}{\kern0pt}j{\isacharparenright}{\kern0pt}{\isacharcomma}{\kern0pt}env{\isacharparenright}{\kern0pt}{\isachardoublequoteclose}\isanewline
%
\isadelimproof
\ \ %
\endisadelimproof
%
\isatagproof
\isacommand{unfolding}\isamarkupfalse%
\ is{\isacharunderscore}{\kern0pt}rcheck{\isacharunderscore}{\kern0pt}def\isanewline
\ \ \isacommand{by}\isamarkupfalse%
\ {\isacharparenleft}{\kern0pt}insert\ assms\ {\isacharsemicolon}{\kern0pt}\ {\isacharparenleft}{\kern0pt}rule\ sep{\isacharunderscore}{\kern0pt}rules\ is{\isacharunderscore}{\kern0pt}singleton{\isacharunderscore}{\kern0pt}iff{\isacharunderscore}{\kern0pt}sats\ is{\isacharunderscore}{\kern0pt}eclose{\isacharunderscore}{\kern0pt}iff{\isacharunderscore}{\kern0pt}sats\isanewline
\ \ \ \ \ \ \ \ trans{\isacharunderscore}{\kern0pt}closure{\isacharunderscore}{\kern0pt}fm{\isacharunderscore}{\kern0pt}iff{\isacharunderscore}{\kern0pt}sats\ {\isacharbar}{\kern0pt}\ simp{\isacharparenright}{\kern0pt}{\isacharplus}{\kern0pt}{\isacharparenright}{\kern0pt}%
\endisatagproof
{\isafoldproof}%
%
\isadelimproof
\isanewline
%
\endisadelimproof
%
\isadelimML
\isanewline
%
\endisadelimML
%
\isatagML
\isacommand{synthesize}\isamarkupfalse%
\ {\isachardoublequoteopen}rcheck{\isacharunderscore}{\kern0pt}fm{\isachardoublequoteclose}\ \isakeyword{from{\isacharunderscore}{\kern0pt}schematic}\ rcheck{\isacharunderscore}{\kern0pt}fm{\isacharunderscore}{\kern0pt}auto%
\endisatagML
{\isafoldML}%
%
\isadelimML
\isanewline
%
\endisadelimML
\isanewline
\isacommand{definition}\isamarkupfalse%
\isanewline
\ \ is{\isacharunderscore}{\kern0pt}check\ {\isacharcolon}{\kern0pt}{\isacharcolon}{\kern0pt}\ {\isachardoublequoteopen}{\isacharbrackleft}{\kern0pt}i{\isacharcomma}{\kern0pt}i{\isacharbrackright}{\kern0pt}\ {\isasymRightarrow}\ o{\isachardoublequoteclose}\ \isakeyword{where}\isanewline
\ \ {\isachardoublequoteopen}is{\isacharunderscore}{\kern0pt}check{\isacharparenleft}{\kern0pt}x{\isacharcomma}{\kern0pt}z{\isacharparenright}{\kern0pt}\ {\isasymequiv}\ {\isasymexists}rch{\isasymin}M{\isachardot}{\kern0pt}\ is{\isacharunderscore}{\kern0pt}rcheck{\isacharparenleft}{\kern0pt}x{\isacharcomma}{\kern0pt}rch{\isacharparenright}{\kern0pt}\ {\isasymand}\ is{\isacharunderscore}{\kern0pt}wfrec{\isacharparenleft}{\kern0pt}{\isacharhash}{\kern0pt}{\isacharhash}{\kern0pt}M{\isacharcomma}{\kern0pt}is{\isacharunderscore}{\kern0pt}Hcheck{\isacharparenleft}{\kern0pt}one{\isacharparenright}{\kern0pt}{\isacharcomma}{\kern0pt}rch{\isacharcomma}{\kern0pt}x{\isacharcomma}{\kern0pt}z{\isacharparenright}{\kern0pt}{\isachardoublequoteclose}\isanewline
\isanewline
\isacommand{lemma}\isamarkupfalse%
\ check{\isacharunderscore}{\kern0pt}abs\ {\isacharcolon}{\kern0pt}\isanewline
\ \ \isakeyword{assumes}\isanewline
\ \ \ \ {\isachardoublequoteopen}x{\isasymin}M{\isachardoublequoteclose}\ {\isachardoublequoteopen}z{\isasymin}M{\isachardoublequoteclose}\isanewline
\ \ \isakeyword{shows}\isanewline
\ \ \ \ {\isachardoublequoteopen}is{\isacharunderscore}{\kern0pt}check{\isacharparenleft}{\kern0pt}x{\isacharcomma}{\kern0pt}z{\isacharparenright}{\kern0pt}\ {\isasymlongleftrightarrow}\ z\ {\isacharequal}{\kern0pt}\ check{\isacharparenleft}{\kern0pt}x{\isacharparenright}{\kern0pt}{\isachardoublequoteclose}\isanewline
%
\isadelimproof
%
\endisadelimproof
%
\isatagproof
\isacommand{proof}\isamarkupfalse%
\ {\isacharminus}{\kern0pt}\isanewline
\ \ \isacommand{have}\isamarkupfalse%
\isanewline
\ \ \ \ {\isachardoublequoteopen}is{\isacharunderscore}{\kern0pt}check{\isacharparenleft}{\kern0pt}x{\isacharcomma}{\kern0pt}z{\isacharparenright}{\kern0pt}\ {\isasymlongleftrightarrow}\ is{\isacharunderscore}{\kern0pt}wfrec{\isacharparenleft}{\kern0pt}{\isacharhash}{\kern0pt}{\isacharhash}{\kern0pt}M{\isacharcomma}{\kern0pt}is{\isacharunderscore}{\kern0pt}Hcheck{\isacharparenleft}{\kern0pt}one{\isacharparenright}{\kern0pt}{\isacharcomma}{\kern0pt}rcheck{\isacharparenleft}{\kern0pt}x{\isacharparenright}{\kern0pt}{\isacharcomma}{\kern0pt}x{\isacharcomma}{\kern0pt}z{\isacharparenright}{\kern0pt}{\isachardoublequoteclose}\isanewline
\ \ \ \ \isacommand{unfolding}\isamarkupfalse%
\ is{\isacharunderscore}{\kern0pt}check{\isacharunderscore}{\kern0pt}def\ \isacommand{using}\isamarkupfalse%
\ assms\ rcheck{\isacharunderscore}{\kern0pt}abs\ rcheck{\isacharunderscore}{\kern0pt}in{\isacharunderscore}{\kern0pt}M\isanewline
\ \ \ \ \isacommand{unfolding}\isamarkupfalse%
\ check{\isacharunderscore}{\kern0pt}trancl\ is{\isacharunderscore}{\kern0pt}check{\isacharunderscore}{\kern0pt}def\ \isacommand{by}\isamarkupfalse%
\ simp\isanewline
\ \ \isacommand{then}\isamarkupfalse%
\ \isacommand{show}\isamarkupfalse%
\ {\isacharquery}{\kern0pt}thesis\isanewline
\ \ \ \ \isacommand{unfolding}\isamarkupfalse%
\ check{\isacharunderscore}{\kern0pt}trancl\isanewline
\ \ \ \ \isacommand{using}\isamarkupfalse%
\ assms\ wfrec{\isacharunderscore}{\kern0pt}Hcheck{\isacharbrackleft}{\kern0pt}of\ x{\isacharbrackright}{\kern0pt}\ wf{\isacharunderscore}{\kern0pt}rcheck\ trans{\isacharunderscore}{\kern0pt}rcheck\ relation{\isacharunderscore}{\kern0pt}rcheck\ rcheck{\isacharunderscore}{\kern0pt}in{\isacharunderscore}{\kern0pt}M\isanewline
\ \ \ \ \ \ Hcheck{\isacharunderscore}{\kern0pt}closed\ relation{\isadigit{2}}{\isacharunderscore}{\kern0pt}Hcheck\ trans{\isacharunderscore}{\kern0pt}wfrec{\isacharunderscore}{\kern0pt}abs{\isacharbrackleft}{\kern0pt}of\ {\isachardoublequoteopen}rcheck{\isacharparenleft}{\kern0pt}x{\isacharparenright}{\kern0pt}{\isachardoublequoteclose}\ x\ z\ {\isachardoublequoteopen}is{\isacharunderscore}{\kern0pt}Hcheck{\isacharparenleft}{\kern0pt}one{\isacharparenright}{\kern0pt}{\isachardoublequoteclose}\ Hcheck{\isacharbrackright}{\kern0pt}\isanewline
\ \ \ \ \isacommand{by}\isamarkupfalse%
\ {\isacharparenleft}{\kern0pt}simp\ flip{\isacharcolon}{\kern0pt}\ setclass{\isacharunderscore}{\kern0pt}iff{\isacharparenright}{\kern0pt}\isanewline
\isacommand{qed}\isamarkupfalse%
%
\endisatagproof
{\isafoldproof}%
%
\isadelimproof
\isanewline
%
\endisadelimproof
\isanewline
\isanewline
\isacommand{definition}\isamarkupfalse%
\isanewline
\ \ check{\isacharunderscore}{\kern0pt}fm\ {\isacharcolon}{\kern0pt}{\isacharcolon}{\kern0pt}\ {\isachardoublequoteopen}{\isacharbrackleft}{\kern0pt}i{\isacharcomma}{\kern0pt}i{\isacharcomma}{\kern0pt}i{\isacharbrackright}{\kern0pt}\ {\isasymRightarrow}\ i{\isachardoublequoteclose}\ \isakeyword{where}\isanewline
\ \ {\isachardoublequoteopen}check{\isacharunderscore}{\kern0pt}fm{\isacharparenleft}{\kern0pt}x{\isacharcomma}{\kern0pt}o{\isacharcomma}{\kern0pt}z{\isacharparenright}{\kern0pt}\ {\isasymequiv}\ Exists{\isacharparenleft}{\kern0pt}And{\isacharparenleft}{\kern0pt}rcheck{\isacharunderscore}{\kern0pt}fm{\isacharparenleft}{\kern0pt}{\isadigit{1}}{\isacharhash}{\kern0pt}{\isacharplus}{\kern0pt}x{\isacharcomma}{\kern0pt}{\isadigit{0}}{\isacharparenright}{\kern0pt}{\isacharcomma}{\kern0pt}\isanewline
\ \ \ \ \ \ \ \ \ \ \ \ \ \ \ \ \ \ \ \ \ \ is{\isacharunderscore}{\kern0pt}wfrec{\isacharunderscore}{\kern0pt}fm{\isacharparenleft}{\kern0pt}is{\isacharunderscore}{\kern0pt}Hcheck{\isacharunderscore}{\kern0pt}fm{\isacharparenleft}{\kern0pt}{\isadigit{6}}{\isacharhash}{\kern0pt}{\isacharplus}{\kern0pt}o{\isacharcomma}{\kern0pt}{\isadigit{2}}{\isacharcomma}{\kern0pt}{\isadigit{1}}{\isacharcomma}{\kern0pt}{\isadigit{0}}{\isacharparenright}{\kern0pt}{\isacharcomma}{\kern0pt}{\isadigit{0}}{\isacharcomma}{\kern0pt}{\isadigit{1}}{\isacharhash}{\kern0pt}{\isacharplus}{\kern0pt}x{\isacharcomma}{\kern0pt}{\isadigit{1}}{\isacharhash}{\kern0pt}{\isacharplus}{\kern0pt}z{\isacharparenright}{\kern0pt}{\isacharparenright}{\kern0pt}{\isacharparenright}{\kern0pt}{\isachardoublequoteclose}\isanewline
\isanewline
\isacommand{lemma}\isamarkupfalse%
\ check{\isacharunderscore}{\kern0pt}fm{\isacharunderscore}{\kern0pt}type{\isacharbrackleft}{\kern0pt}TC{\isacharbrackright}{\kern0pt}\ {\isacharcolon}{\kern0pt}\isanewline
\ \ {\isachardoublequoteopen}{\isasymlbrakk}x{\isasymin}nat{\isacharsemicolon}{\kern0pt}o{\isasymin}nat{\isacharsemicolon}{\kern0pt}z{\isasymin}nat{\isasymrbrakk}\ {\isasymLongrightarrow}\ check{\isacharunderscore}{\kern0pt}fm{\isacharparenleft}{\kern0pt}x{\isacharcomma}{\kern0pt}o{\isacharcomma}{\kern0pt}z{\isacharparenright}{\kern0pt}{\isasymin}formula{\isachardoublequoteclose}\isanewline
%
\isadelimproof
\ \ %
\endisadelimproof
%
\isatagproof
\isacommand{unfolding}\isamarkupfalse%
\ check{\isacharunderscore}{\kern0pt}fm{\isacharunderscore}{\kern0pt}def\ \isacommand{by}\isamarkupfalse%
\ simp%
\endisatagproof
{\isafoldproof}%
%
\isadelimproof
\isanewline
%
\endisadelimproof
\isanewline
\isacommand{lemma}\isamarkupfalse%
\ sats{\isacharunderscore}{\kern0pt}check{\isacharunderscore}{\kern0pt}fm\ {\isacharcolon}{\kern0pt}\isanewline
\ \ \isakeyword{assumes}\isanewline
\ \ \ \ {\isachardoublequoteopen}nth{\isacharparenleft}{\kern0pt}o{\isacharcomma}{\kern0pt}env{\isacharparenright}{\kern0pt}\ {\isacharequal}{\kern0pt}\ one{\isachardoublequoteclose}\ {\isachardoublequoteopen}x{\isasymin}nat{\isachardoublequoteclose}\ {\isachardoublequoteopen}z{\isasymin}nat{\isachardoublequoteclose}\ {\isachardoublequoteopen}o{\isasymin}nat{\isachardoublequoteclose}\ {\isachardoublequoteopen}env{\isasymin}list{\isacharparenleft}{\kern0pt}M{\isacharparenright}{\kern0pt}{\isachardoublequoteclose}\ {\isachardoublequoteopen}x\ {\isacharless}{\kern0pt}\ length{\isacharparenleft}{\kern0pt}env{\isacharparenright}{\kern0pt}{\isachardoublequoteclose}\ {\isachardoublequoteopen}z\ {\isacharless}{\kern0pt}\ length{\isacharparenleft}{\kern0pt}env{\isacharparenright}{\kern0pt}{\isachardoublequoteclose}\isanewline
\ \ \isakeyword{shows}\isanewline
\ \ \ \ {\isachardoublequoteopen}sats{\isacharparenleft}{\kern0pt}M{\isacharcomma}{\kern0pt}\ check{\isacharunderscore}{\kern0pt}fm{\isacharparenleft}{\kern0pt}x{\isacharcomma}{\kern0pt}o{\isacharcomma}{\kern0pt}z{\isacharparenright}{\kern0pt}{\isacharcomma}{\kern0pt}\ env{\isacharparenright}{\kern0pt}\ {\isasymlongleftrightarrow}\ is{\isacharunderscore}{\kern0pt}check{\isacharparenleft}{\kern0pt}nth{\isacharparenleft}{\kern0pt}x{\isacharcomma}{\kern0pt}env{\isacharparenright}{\kern0pt}{\isacharcomma}{\kern0pt}nth{\isacharparenleft}{\kern0pt}z{\isacharcomma}{\kern0pt}env{\isacharparenright}{\kern0pt}{\isacharparenright}{\kern0pt}{\isachardoublequoteclose}\isanewline
%
\isadelimproof
%
\endisadelimproof
%
\isatagproof
\isacommand{proof}\isamarkupfalse%
\ {\isacharminus}{\kern0pt}\isanewline
\ \ \isacommand{have}\isamarkupfalse%
\ sats{\isacharunderscore}{\kern0pt}is{\isacharunderscore}{\kern0pt}Hcheck{\isacharunderscore}{\kern0pt}fm{\isacharcolon}{\kern0pt}\isanewline
\ \ \ \ {\isachardoublequoteopen}{\isasymAnd}a{\isadigit{0}}\ a{\isadigit{1}}\ a{\isadigit{2}}\ a{\isadigit{3}}\ a{\isadigit{4}}{\isachardot}{\kern0pt}\ {\isasymlbrakk}\ a{\isadigit{0}}{\isasymin}M{\isacharsemicolon}{\kern0pt}\ a{\isadigit{1}}{\isasymin}M{\isacharsemicolon}{\kern0pt}\ a{\isadigit{2}}{\isasymin}M{\isacharsemicolon}{\kern0pt}\ a{\isadigit{3}}{\isasymin}M{\isacharsemicolon}{\kern0pt}\ a{\isadigit{4}}{\isasymin}M\ {\isasymrbrakk}\ {\isasymLongrightarrow}\isanewline
\ \ \ \ \ \ \ \ \ is{\isacharunderscore}{\kern0pt}Hcheck{\isacharparenleft}{\kern0pt}one{\isacharcomma}{\kern0pt}a{\isadigit{2}}{\isacharcomma}{\kern0pt}\ a{\isadigit{1}}{\isacharcomma}{\kern0pt}\ a{\isadigit{0}}{\isacharparenright}{\kern0pt}\ {\isasymlongleftrightarrow}\isanewline
\ \ \ \ \ \ \ \ \ sats{\isacharparenleft}{\kern0pt}M{\isacharcomma}{\kern0pt}\ is{\isacharunderscore}{\kern0pt}Hcheck{\isacharunderscore}{\kern0pt}fm{\isacharparenleft}{\kern0pt}{\isadigit{6}}{\isacharhash}{\kern0pt}{\isacharplus}{\kern0pt}o{\isacharcomma}{\kern0pt}{\isadigit{2}}{\isacharcomma}{\kern0pt}{\isadigit{1}}{\isacharcomma}{\kern0pt}{\isadigit{0}}{\isacharparenright}{\kern0pt}{\isacharcomma}{\kern0pt}\ {\isacharbrackleft}{\kern0pt}a{\isadigit{0}}{\isacharcomma}{\kern0pt}a{\isadigit{1}}{\isacharcomma}{\kern0pt}a{\isadigit{2}}{\isacharcomma}{\kern0pt}a{\isadigit{3}}{\isacharcomma}{\kern0pt}a{\isadigit{4}}{\isacharcomma}{\kern0pt}r{\isacharbrackright}{\kern0pt}{\isacharat}{\kern0pt}env{\isacharparenright}{\kern0pt}{\isachardoublequoteclose}\ \isakeyword{if}\ {\isachardoublequoteopen}r{\isasymin}M{\isachardoublequoteclose}\ \isakeyword{for}\ r\isanewline
\ \ \ \ \isacommand{using}\isamarkupfalse%
\ that\ one{\isacharunderscore}{\kern0pt}in{\isacharunderscore}{\kern0pt}M\ assms\ \ \isacommand{by}\isamarkupfalse%
\ simp\isanewline
\ \ \isacommand{then}\isamarkupfalse%
\isanewline
\ \ \isacommand{have}\isamarkupfalse%
\ {\isachardoublequoteopen}sats{\isacharparenleft}{\kern0pt}M{\isacharcomma}{\kern0pt}\ is{\isacharunderscore}{\kern0pt}wfrec{\isacharunderscore}{\kern0pt}fm{\isacharparenleft}{\kern0pt}is{\isacharunderscore}{\kern0pt}Hcheck{\isacharunderscore}{\kern0pt}fm{\isacharparenleft}{\kern0pt}{\isadigit{6}}{\isacharhash}{\kern0pt}{\isacharplus}{\kern0pt}o{\isacharcomma}{\kern0pt}{\isadigit{2}}{\isacharcomma}{\kern0pt}{\isadigit{1}}{\isacharcomma}{\kern0pt}{\isadigit{0}}{\isacharparenright}{\kern0pt}{\isacharcomma}{\kern0pt}{\isadigit{0}}{\isacharcomma}{\kern0pt}{\isadigit{1}}{\isacharhash}{\kern0pt}{\isacharplus}{\kern0pt}x{\isacharcomma}{\kern0pt}{\isadigit{1}}{\isacharhash}{\kern0pt}{\isacharplus}{\kern0pt}z{\isacharparenright}{\kern0pt}{\isacharcomma}{\kern0pt}Cons{\isacharparenleft}{\kern0pt}r{\isacharcomma}{\kern0pt}env{\isacharparenright}{\kern0pt}{\isacharparenright}{\kern0pt}\isanewline
\ \ \ \ \ \ \ \ {\isasymlongleftrightarrow}\ is{\isacharunderscore}{\kern0pt}wfrec{\isacharparenleft}{\kern0pt}{\isacharhash}{\kern0pt}{\isacharhash}{\kern0pt}M{\isacharcomma}{\kern0pt}is{\isacharunderscore}{\kern0pt}Hcheck{\isacharparenleft}{\kern0pt}one{\isacharparenright}{\kern0pt}{\isacharcomma}{\kern0pt}r{\isacharcomma}{\kern0pt}nth{\isacharparenleft}{\kern0pt}x{\isacharcomma}{\kern0pt}env{\isacharparenright}{\kern0pt}{\isacharcomma}{\kern0pt}nth{\isacharparenleft}{\kern0pt}z{\isacharcomma}{\kern0pt}env{\isacharparenright}{\kern0pt}{\isacharparenright}{\kern0pt}{\isachardoublequoteclose}\ \isakeyword{if}\ {\isachardoublequoteopen}r{\isasymin}M{\isachardoublequoteclose}\ \isakeyword{for}\ r\isanewline
\ \ \ \ \isacommand{using}\isamarkupfalse%
\ that\ assms\ one{\isacharunderscore}{\kern0pt}in{\isacharunderscore}{\kern0pt}M\ sats{\isacharunderscore}{\kern0pt}is{\isacharunderscore}{\kern0pt}wfrec{\isacharunderscore}{\kern0pt}fm\ \isacommand{by}\isamarkupfalse%
\ simp\isanewline
\ \ \isacommand{then}\isamarkupfalse%
\isanewline
\ \ \isacommand{show}\isamarkupfalse%
\ {\isacharquery}{\kern0pt}thesis\ \isacommand{unfolding}\isamarkupfalse%
\ is{\isacharunderscore}{\kern0pt}check{\isacharunderscore}{\kern0pt}def\ check{\isacharunderscore}{\kern0pt}fm{\isacharunderscore}{\kern0pt}def\isanewline
\ \ \ \ \isacommand{using}\isamarkupfalse%
\ assms\ rcheck{\isacharunderscore}{\kern0pt}in{\isacharunderscore}{\kern0pt}M\ one{\isacharunderscore}{\kern0pt}in{\isacharunderscore}{\kern0pt}M\ rcheck{\isacharunderscore}{\kern0pt}fm{\isacharunderscore}{\kern0pt}iff{\isacharunderscore}{\kern0pt}sats{\isacharbrackleft}{\kern0pt}symmetric{\isacharbrackright}{\kern0pt}\ \isacommand{by}\isamarkupfalse%
\ simp\isanewline
\isacommand{qed}\isamarkupfalse%
%
\endisatagproof
{\isafoldproof}%
%
\isadelimproof
\isanewline
%
\endisadelimproof
\isanewline
\isanewline
\isacommand{lemma}\isamarkupfalse%
\ check{\isacharunderscore}{\kern0pt}replacement{\isacharcolon}{\kern0pt}\isanewline
\ \ {\isachardoublequoteopen}{\isacharbraceleft}{\kern0pt}check{\isacharparenleft}{\kern0pt}x{\isacharparenright}{\kern0pt}{\isachardot}{\kern0pt}\ x{\isasymin}P{\isacharbraceright}{\kern0pt}\ {\isasymin}\ M{\isachardoublequoteclose}\isanewline
%
\isadelimproof
%
\endisadelimproof
%
\isatagproof
\isacommand{proof}\isamarkupfalse%
\ {\isacharminus}{\kern0pt}\isanewline
\ \ \isacommand{have}\isamarkupfalse%
\ {\isachardoublequoteopen}arity{\isacharparenleft}{\kern0pt}check{\isacharunderscore}{\kern0pt}fm{\isacharparenleft}{\kern0pt}{\isadigit{0}}{\isacharcomma}{\kern0pt}{\isadigit{2}}{\isacharcomma}{\kern0pt}{\isadigit{1}}{\isacharparenright}{\kern0pt}{\isacharparenright}{\kern0pt}\ {\isacharequal}{\kern0pt}\ {\isadigit{3}}{\isachardoublequoteclose}\isanewline
\ \ \ \ \isacommand{unfolding}\isamarkupfalse%
\ check{\isacharunderscore}{\kern0pt}fm{\isacharunderscore}{\kern0pt}def\ rcheck{\isacharunderscore}{\kern0pt}fm{\isacharunderscore}{\kern0pt}def\ trans{\isacharunderscore}{\kern0pt}closure{\isacharunderscore}{\kern0pt}fm{\isacharunderscore}{\kern0pt}def\ is{\isacharunderscore}{\kern0pt}eclose{\isacharunderscore}{\kern0pt}fm{\isacharunderscore}{\kern0pt}def\ mem{\isacharunderscore}{\kern0pt}eclose{\isacharunderscore}{\kern0pt}fm{\isacharunderscore}{\kern0pt}def\isanewline
\ \ \ \ \ \ is{\isacharunderscore}{\kern0pt}Hcheck{\isacharunderscore}{\kern0pt}fm{\isacharunderscore}{\kern0pt}def\ Replace{\isacharunderscore}{\kern0pt}fm{\isacharunderscore}{\kern0pt}def\ PHcheck{\isacharunderscore}{\kern0pt}fm{\isacharunderscore}{\kern0pt}def\ finite{\isacharunderscore}{\kern0pt}ordinal{\isacharunderscore}{\kern0pt}fm{\isacharunderscore}{\kern0pt}def\ is{\isacharunderscore}{\kern0pt}iterates{\isacharunderscore}{\kern0pt}fm{\isacharunderscore}{\kern0pt}def\isanewline
\ \ \ \ \ \ is{\isacharunderscore}{\kern0pt}wfrec{\isacharunderscore}{\kern0pt}fm{\isacharunderscore}{\kern0pt}def\ is{\isacharunderscore}{\kern0pt}recfun{\isacharunderscore}{\kern0pt}fm{\isacharunderscore}{\kern0pt}def\ restriction{\isacharunderscore}{\kern0pt}fm{\isacharunderscore}{\kern0pt}def\ pre{\isacharunderscore}{\kern0pt}image{\isacharunderscore}{\kern0pt}fm{\isacharunderscore}{\kern0pt}def\ eclose{\isacharunderscore}{\kern0pt}n{\isacharunderscore}{\kern0pt}fm{\isacharunderscore}{\kern0pt}def\isanewline
\ \ \ \ \ \ is{\isacharunderscore}{\kern0pt}nat{\isacharunderscore}{\kern0pt}case{\isacharunderscore}{\kern0pt}fm{\isacharunderscore}{\kern0pt}def\ quasinat{\isacharunderscore}{\kern0pt}fm{\isacharunderscore}{\kern0pt}def\ Memrel{\isacharunderscore}{\kern0pt}fm{\isacharunderscore}{\kern0pt}def\ singleton{\isacharunderscore}{\kern0pt}fm{\isacharunderscore}{\kern0pt}def\ fm{\isacharunderscore}{\kern0pt}defs\ iterates{\isacharunderscore}{\kern0pt}MH{\isacharunderscore}{\kern0pt}fm{\isacharunderscore}{\kern0pt}def\isanewline
\ \ \ \ \isacommand{by}\isamarkupfalse%
\ {\isacharparenleft}{\kern0pt}simp\ add{\isacharcolon}{\kern0pt}nat{\isacharunderscore}{\kern0pt}simp{\isacharunderscore}{\kern0pt}union{\isacharparenright}{\kern0pt}\isanewline
\ \ \isacommand{moreover}\isamarkupfalse%
\isanewline
\ \ \isacommand{have}\isamarkupfalse%
\ {\isachardoublequoteopen}check{\isacharparenleft}{\kern0pt}x{\isacharparenright}{\kern0pt}{\isasymin}M{\isachardoublequoteclose}\ \isakeyword{if}\ {\isachardoublequoteopen}x{\isasymin}P{\isachardoublequoteclose}\ \isakeyword{for}\ x\isanewline
\ \ \ \ \isacommand{using}\isamarkupfalse%
\ that\ Transset{\isacharunderscore}{\kern0pt}intf{\isacharbrackleft}{\kern0pt}of\ M\ x\ P{\isacharbrackright}{\kern0pt}\ trans{\isacharunderscore}{\kern0pt}M\ check{\isacharunderscore}{\kern0pt}in{\isacharunderscore}{\kern0pt}M\ P{\isacharunderscore}{\kern0pt}in{\isacharunderscore}{\kern0pt}M\ \isacommand{by}\isamarkupfalse%
\ simp\isanewline
\ \ \isacommand{ultimately}\isamarkupfalse%
\isanewline
\ \ \isacommand{show}\isamarkupfalse%
\ {\isacharquery}{\kern0pt}thesis\ \isacommand{using}\isamarkupfalse%
\ sats{\isacharunderscore}{\kern0pt}check{\isacharunderscore}{\kern0pt}fm\ check{\isacharunderscore}{\kern0pt}abs\ P{\isacharunderscore}{\kern0pt}in{\isacharunderscore}{\kern0pt}M\ check{\isacharunderscore}{\kern0pt}in{\isacharunderscore}{\kern0pt}M\ one{\isacharunderscore}{\kern0pt}in{\isacharunderscore}{\kern0pt}M\isanewline
\ \ \ \ \ \ Repl{\isacharunderscore}{\kern0pt}in{\isacharunderscore}{\kern0pt}M{\isacharbrackleft}{\kern0pt}of\ {\isachardoublequoteopen}check{\isacharunderscore}{\kern0pt}fm{\isacharparenleft}{\kern0pt}{\isadigit{0}}{\isacharcomma}{\kern0pt}{\isadigit{2}}{\isacharcomma}{\kern0pt}{\isadigit{1}}{\isacharparenright}{\kern0pt}{\isachardoublequoteclose}\ {\isachardoublequoteopen}{\isacharbrackleft}{\kern0pt}one{\isacharbrackright}{\kern0pt}{\isachardoublequoteclose}\ is{\isacharunderscore}{\kern0pt}check\ check{\isacharbrackright}{\kern0pt}\ \isacommand{by}\isamarkupfalse%
\ simp\isanewline
\isacommand{qed}\isamarkupfalse%
%
\endisatagproof
{\isafoldproof}%
%
\isadelimproof
\isanewline
%
\endisadelimproof
\isanewline
\isacommand{lemma}\isamarkupfalse%
\ pair{\isacharunderscore}{\kern0pt}check\ {\isacharcolon}{\kern0pt}\ {\isachardoublequoteopen}{\isasymlbrakk}\ p{\isasymin}M\ {\isacharsemicolon}{\kern0pt}\ y{\isasymin}M\ {\isasymrbrakk}\ \ {\isasymLongrightarrow}\ {\isacharparenleft}{\kern0pt}{\isasymexists}c{\isasymin}M{\isachardot}{\kern0pt}\ is{\isacharunderscore}{\kern0pt}check{\isacharparenleft}{\kern0pt}p{\isacharcomma}{\kern0pt}c{\isacharparenright}{\kern0pt}\ {\isasymand}\ pair{\isacharparenleft}{\kern0pt}{\isacharhash}{\kern0pt}{\isacharhash}{\kern0pt}M{\isacharcomma}{\kern0pt}c{\isacharcomma}{\kern0pt}p{\isacharcomma}{\kern0pt}y{\isacharparenright}{\kern0pt}{\isacharparenright}{\kern0pt}\ {\isasymlongleftrightarrow}\ y\ {\isacharequal}{\kern0pt}\ {\isasymlangle}check{\isacharparenleft}{\kern0pt}p{\isacharparenright}{\kern0pt}{\isacharcomma}{\kern0pt}p{\isasymrangle}{\isachardoublequoteclose}\isanewline
%
\isadelimproof
\ \ %
\endisadelimproof
%
\isatagproof
\isacommand{using}\isamarkupfalse%
\ check{\isacharunderscore}{\kern0pt}abs\ check{\isacharunderscore}{\kern0pt}in{\isacharunderscore}{\kern0pt}M\ tuples{\isacharunderscore}{\kern0pt}in{\isacharunderscore}{\kern0pt}M\ \isacommand{by}\isamarkupfalse%
\ simp%
\endisatagproof
{\isafoldproof}%
%
\isadelimproof
\isanewline
%
\endisadelimproof
\isanewline
\isanewline
\isacommand{lemma}\isamarkupfalse%
\ M{\isacharunderscore}{\kern0pt}subset{\isacharunderscore}{\kern0pt}MG\ {\isacharcolon}{\kern0pt}\ \ {\isachardoublequoteopen}one\ {\isasymin}\ G\ {\isasymLongrightarrow}\ M\ {\isasymsubseteq}\ M{\isacharbrackleft}{\kern0pt}G{\isacharbrackright}{\kern0pt}{\isachardoublequoteclose}\isanewline
%
\isadelimproof
\ \ %
\endisadelimproof
%
\isatagproof
\isacommand{using}\isamarkupfalse%
\ check{\isacharunderscore}{\kern0pt}in{\isacharunderscore}{\kern0pt}M\ one{\isacharunderscore}{\kern0pt}in{\isacharunderscore}{\kern0pt}P\ GenExtI\isanewline
\ \ \isacommand{by}\isamarkupfalse%
\ {\isacharparenleft}{\kern0pt}intro\ subsetI{\isacharcomma}{\kern0pt}\ subst\ valcheck\ {\isacharbrackleft}{\kern0pt}of\ G{\isacharcomma}{\kern0pt}symmetric{\isacharbrackright}{\kern0pt}{\isacharcomma}{\kern0pt}\ auto{\isacharparenright}{\kern0pt}%
\endisatagproof
{\isafoldproof}%
%
\isadelimproof
%
\endisadelimproof
%
\begin{isamarkuptext}%
The name for the generic filter%
\end{isamarkuptext}\isamarkuptrue%
\isacommand{definition}\isamarkupfalse%
\isanewline
\ \ G{\isacharunderscore}{\kern0pt}dot\ {\isacharcolon}{\kern0pt}{\isacharcolon}{\kern0pt}\ {\isachardoublequoteopen}i{\isachardoublequoteclose}\ \isakeyword{where}\isanewline
\ \ {\isachardoublequoteopen}G{\isacharunderscore}{\kern0pt}dot\ {\isasymequiv}\ {\isacharbraceleft}{\kern0pt}{\isasymlangle}check{\isacharparenleft}{\kern0pt}p{\isacharparenright}{\kern0pt}{\isacharcomma}{\kern0pt}p{\isasymrangle}\ {\isachardot}{\kern0pt}\ p{\isasymin}P{\isacharbraceright}{\kern0pt}{\isachardoublequoteclose}\isanewline
\isanewline
\isacommand{lemma}\isamarkupfalse%
\ G{\isacharunderscore}{\kern0pt}dot{\isacharunderscore}{\kern0pt}in{\isacharunderscore}{\kern0pt}M\ {\isacharcolon}{\kern0pt}\isanewline
\ \ {\isachardoublequoteopen}G{\isacharunderscore}{\kern0pt}dot\ {\isasymin}\ M{\isachardoublequoteclose}\isanewline
%
\isadelimproof
%
\endisadelimproof
%
\isatagproof
\isacommand{proof}\isamarkupfalse%
\ {\isacharminus}{\kern0pt}\isanewline
\ \ \isacommand{let}\isamarkupfalse%
\ {\isacharquery}{\kern0pt}is{\isacharunderscore}{\kern0pt}pcheck\ {\isacharequal}{\kern0pt}\ {\isachardoublequoteopen}{\isasymlambda}x\ y{\isachardot}{\kern0pt}\ {\isasymexists}ch{\isasymin}M{\isachardot}{\kern0pt}\ is{\isacharunderscore}{\kern0pt}check{\isacharparenleft}{\kern0pt}x{\isacharcomma}{\kern0pt}ch{\isacharparenright}{\kern0pt}\ {\isasymand}\ pair{\isacharparenleft}{\kern0pt}{\isacharhash}{\kern0pt}{\isacharhash}{\kern0pt}M{\isacharcomma}{\kern0pt}ch{\isacharcomma}{\kern0pt}x{\isacharcomma}{\kern0pt}y{\isacharparenright}{\kern0pt}{\isachardoublequoteclose}\isanewline
\ \ \isacommand{let}\isamarkupfalse%
\ {\isacharquery}{\kern0pt}pcheck{\isacharunderscore}{\kern0pt}fm\ {\isacharequal}{\kern0pt}\ {\isachardoublequoteopen}Exists{\isacharparenleft}{\kern0pt}And{\isacharparenleft}{\kern0pt}check{\isacharunderscore}{\kern0pt}fm{\isacharparenleft}{\kern0pt}{\isadigit{1}}{\isacharcomma}{\kern0pt}{\isadigit{3}}{\isacharcomma}{\kern0pt}{\isadigit{0}}{\isacharparenright}{\kern0pt}{\isacharcomma}{\kern0pt}pair{\isacharunderscore}{\kern0pt}fm{\isacharparenleft}{\kern0pt}{\isadigit{0}}{\isacharcomma}{\kern0pt}{\isadigit{1}}{\isacharcomma}{\kern0pt}{\isadigit{2}}{\isacharparenright}{\kern0pt}{\isacharparenright}{\kern0pt}{\isacharparenright}{\kern0pt}{\isachardoublequoteclose}\isanewline
\ \ \isacommand{have}\isamarkupfalse%
\ {\isachardoublequoteopen}sats{\isacharparenleft}{\kern0pt}M{\isacharcomma}{\kern0pt}{\isacharquery}{\kern0pt}pcheck{\isacharunderscore}{\kern0pt}fm{\isacharcomma}{\kern0pt}{\isacharbrackleft}{\kern0pt}x{\isacharcomma}{\kern0pt}y{\isacharcomma}{\kern0pt}one{\isacharbrackright}{\kern0pt}{\isacharparenright}{\kern0pt}\ {\isasymlongleftrightarrow}\ {\isacharquery}{\kern0pt}is{\isacharunderscore}{\kern0pt}pcheck{\isacharparenleft}{\kern0pt}x{\isacharcomma}{\kern0pt}y{\isacharparenright}{\kern0pt}{\isachardoublequoteclose}\ \isakeyword{if}\ {\isachardoublequoteopen}x{\isasymin}M{\isachardoublequoteclose}\ {\isachardoublequoteopen}y{\isasymin}M{\isachardoublequoteclose}\ \isakeyword{for}\ x\ y\isanewline
\ \ \ \ \isacommand{using}\isamarkupfalse%
\ sats{\isacharunderscore}{\kern0pt}check{\isacharunderscore}{\kern0pt}fm\ that\ one{\isacharunderscore}{\kern0pt}in{\isacharunderscore}{\kern0pt}M\ \isacommand{by}\isamarkupfalse%
\ simp\isanewline
\ \ \isacommand{moreover}\isamarkupfalse%
\isanewline
\ \ \isacommand{have}\isamarkupfalse%
\ {\isachardoublequoteopen}{\isacharquery}{\kern0pt}is{\isacharunderscore}{\kern0pt}pcheck{\isacharparenleft}{\kern0pt}x{\isacharcomma}{\kern0pt}y{\isacharparenright}{\kern0pt}\ {\isasymlongleftrightarrow}\ y\ {\isacharequal}{\kern0pt}\ {\isasymlangle}check{\isacharparenleft}{\kern0pt}x{\isacharparenright}{\kern0pt}{\isacharcomma}{\kern0pt}x{\isasymrangle}{\isachardoublequoteclose}\ \isakeyword{if}\ {\isachardoublequoteopen}x{\isasymin}M{\isachardoublequoteclose}\ {\isachardoublequoteopen}y{\isasymin}M{\isachardoublequoteclose}\ \isakeyword{for}\ x\ y\isanewline
\ \ \ \ \isacommand{using}\isamarkupfalse%
\ that\ check{\isacharunderscore}{\kern0pt}abs\ check{\isacharunderscore}{\kern0pt}in{\isacharunderscore}{\kern0pt}M\ \isacommand{by}\isamarkupfalse%
\ simp\isanewline
\ \ \isacommand{moreover}\isamarkupfalse%
\isanewline
\ \ \isacommand{have}\isamarkupfalse%
\ {\isachardoublequoteopen}{\isacharquery}{\kern0pt}pcheck{\isacharunderscore}{\kern0pt}fm{\isasymin}formula{\isachardoublequoteclose}\ \isacommand{by}\isamarkupfalse%
\ simp\isanewline
\ \ \isacommand{moreover}\isamarkupfalse%
\isanewline
\ \ \isacommand{have}\isamarkupfalse%
\ {\isachardoublequoteopen}arity{\isacharparenleft}{\kern0pt}{\isacharquery}{\kern0pt}pcheck{\isacharunderscore}{\kern0pt}fm{\isacharparenright}{\kern0pt}{\isacharequal}{\kern0pt}{\isadigit{3}}{\isachardoublequoteclose}\isanewline
\ \ \ \ \isacommand{unfolding}\isamarkupfalse%
\ check{\isacharunderscore}{\kern0pt}fm{\isacharunderscore}{\kern0pt}def\ rcheck{\isacharunderscore}{\kern0pt}fm{\isacharunderscore}{\kern0pt}def\ trans{\isacharunderscore}{\kern0pt}closure{\isacharunderscore}{\kern0pt}fm{\isacharunderscore}{\kern0pt}def\ is{\isacharunderscore}{\kern0pt}eclose{\isacharunderscore}{\kern0pt}fm{\isacharunderscore}{\kern0pt}def\ mem{\isacharunderscore}{\kern0pt}eclose{\isacharunderscore}{\kern0pt}fm{\isacharunderscore}{\kern0pt}def\isanewline
\ \ \ \ \ \ is{\isacharunderscore}{\kern0pt}Hcheck{\isacharunderscore}{\kern0pt}fm{\isacharunderscore}{\kern0pt}def\ Replace{\isacharunderscore}{\kern0pt}fm{\isacharunderscore}{\kern0pt}def\ PHcheck{\isacharunderscore}{\kern0pt}fm{\isacharunderscore}{\kern0pt}def\ finite{\isacharunderscore}{\kern0pt}ordinal{\isacharunderscore}{\kern0pt}fm{\isacharunderscore}{\kern0pt}def\ is{\isacharunderscore}{\kern0pt}iterates{\isacharunderscore}{\kern0pt}fm{\isacharunderscore}{\kern0pt}def\isanewline
\ \ \ \ \ \ is{\isacharunderscore}{\kern0pt}wfrec{\isacharunderscore}{\kern0pt}fm{\isacharunderscore}{\kern0pt}def\ is{\isacharunderscore}{\kern0pt}recfun{\isacharunderscore}{\kern0pt}fm{\isacharunderscore}{\kern0pt}def\ restriction{\isacharunderscore}{\kern0pt}fm{\isacharunderscore}{\kern0pt}def\ pre{\isacharunderscore}{\kern0pt}image{\isacharunderscore}{\kern0pt}fm{\isacharunderscore}{\kern0pt}def\ eclose{\isacharunderscore}{\kern0pt}n{\isacharunderscore}{\kern0pt}fm{\isacharunderscore}{\kern0pt}def\isanewline
\ \ \ \ \ \ is{\isacharunderscore}{\kern0pt}nat{\isacharunderscore}{\kern0pt}case{\isacharunderscore}{\kern0pt}fm{\isacharunderscore}{\kern0pt}def\ quasinat{\isacharunderscore}{\kern0pt}fm{\isacharunderscore}{\kern0pt}def\ Memrel{\isacharunderscore}{\kern0pt}fm{\isacharunderscore}{\kern0pt}def\ singleton{\isacharunderscore}{\kern0pt}fm{\isacharunderscore}{\kern0pt}def\ fm{\isacharunderscore}{\kern0pt}defs\ iterates{\isacharunderscore}{\kern0pt}MH{\isacharunderscore}{\kern0pt}fm{\isacharunderscore}{\kern0pt}def\isanewline
\ \ \ \ \isacommand{by}\isamarkupfalse%
\ {\isacharparenleft}{\kern0pt}simp\ add{\isacharcolon}{\kern0pt}nat{\isacharunderscore}{\kern0pt}simp{\isacharunderscore}{\kern0pt}union{\isacharparenright}{\kern0pt}\isanewline
\ \ \isacommand{moreover}\isamarkupfalse%
\isanewline
\ \ \isacommand{from}\isamarkupfalse%
\ P{\isacharunderscore}{\kern0pt}in{\isacharunderscore}{\kern0pt}M\ check{\isacharunderscore}{\kern0pt}in{\isacharunderscore}{\kern0pt}M\ tuples{\isacharunderscore}{\kern0pt}in{\isacharunderscore}{\kern0pt}M\ P{\isacharunderscore}{\kern0pt}sub{\isacharunderscore}{\kern0pt}M\isanewline
\ \ \isacommand{have}\isamarkupfalse%
\ {\isachardoublequoteopen}{\isasymlangle}check{\isacharparenleft}{\kern0pt}p{\isacharparenright}{\kern0pt}{\isacharcomma}{\kern0pt}p{\isasymrangle}\ {\isasymin}\ M{\isachardoublequoteclose}\ \isakeyword{if}\ {\isachardoublequoteopen}p{\isasymin}P{\isachardoublequoteclose}\ \isakeyword{for}\ p\isanewline
\ \ \ \ \isacommand{using}\isamarkupfalse%
\ that\ \isacommand{by}\isamarkupfalse%
\ auto\isanewline
\ \ \isacommand{ultimately}\isamarkupfalse%
\isanewline
\ \ \isacommand{show}\isamarkupfalse%
\ {\isacharquery}{\kern0pt}thesis\isanewline
\ \ \ \ \isacommand{unfolding}\isamarkupfalse%
\ G{\isacharunderscore}{\kern0pt}dot{\isacharunderscore}{\kern0pt}def\isanewline
\ \ \ \ \isacommand{using}\isamarkupfalse%
\ one{\isacharunderscore}{\kern0pt}in{\isacharunderscore}{\kern0pt}M\ P{\isacharunderscore}{\kern0pt}in{\isacharunderscore}{\kern0pt}M\ Repl{\isacharunderscore}{\kern0pt}in{\isacharunderscore}{\kern0pt}M{\isacharbrackleft}{\kern0pt}of\ {\isacharquery}{\kern0pt}pcheck{\isacharunderscore}{\kern0pt}fm\ {\isachardoublequoteopen}{\isacharbrackleft}{\kern0pt}one{\isacharbrackright}{\kern0pt}{\isachardoublequoteclose}{\isacharbrackright}{\kern0pt}\isanewline
\ \ \ \ \isacommand{by}\isamarkupfalse%
\ simp\isanewline
\isacommand{qed}\isamarkupfalse%
%
\endisatagproof
{\isafoldproof}%
%
\isadelimproof
\isanewline
%
\endisadelimproof
\isanewline
\isanewline
\isacommand{lemma}\isamarkupfalse%
\ val{\isacharunderscore}{\kern0pt}G{\isacharunderscore}{\kern0pt}dot\ {\isacharcolon}{\kern0pt}\isanewline
\ \ \isakeyword{assumes}\ {\isachardoublequoteopen}G\ {\isasymsubseteq}\ P{\isachardoublequoteclose}\isanewline
\ \ \ \ {\isachardoublequoteopen}one\ {\isasymin}\ G{\isachardoublequoteclose}\isanewline
\ \ \isakeyword{shows}\ {\isachardoublequoteopen}val{\isacharparenleft}{\kern0pt}G{\isacharcomma}{\kern0pt}G{\isacharunderscore}{\kern0pt}dot{\isacharparenright}{\kern0pt}\ {\isacharequal}{\kern0pt}\ G{\isachardoublequoteclose}\isanewline
%
\isadelimproof
%
\endisadelimproof
%
\isatagproof
\isacommand{proof}\isamarkupfalse%
\ {\isacharparenleft}{\kern0pt}intro\ equalityI\ subsetI{\isacharparenright}{\kern0pt}\isanewline
\ \ \isacommand{fix}\isamarkupfalse%
\ x\isanewline
\ \ \isacommand{assume}\isamarkupfalse%
\ {\isachardoublequoteopen}x{\isasymin}val{\isacharparenleft}{\kern0pt}G{\isacharcomma}{\kern0pt}G{\isacharunderscore}{\kern0pt}dot{\isacharparenright}{\kern0pt}{\isachardoublequoteclose}\isanewline
\ \ \isacommand{then}\isamarkupfalse%
\ \isacommand{obtain}\isamarkupfalse%
\ {\isasymtheta}\ p\ \isakeyword{where}\ {\isachardoublequoteopen}p{\isasymin}G{\isachardoublequoteclose}\ {\isachardoublequoteopen}{\isasymlangle}{\isasymtheta}{\isacharcomma}{\kern0pt}p{\isasymrangle}\ {\isasymin}\ G{\isacharunderscore}{\kern0pt}dot{\isachardoublequoteclose}\ {\isachardoublequoteopen}val{\isacharparenleft}{\kern0pt}G{\isacharcomma}{\kern0pt}{\isasymtheta}{\isacharparenright}{\kern0pt}\ {\isacharequal}{\kern0pt}\ x{\isachardoublequoteclose}\ {\isachardoublequoteopen}{\isasymtheta}\ {\isacharequal}{\kern0pt}\ check{\isacharparenleft}{\kern0pt}p{\isacharparenright}{\kern0pt}{\isachardoublequoteclose}\isanewline
\ \ \ \ \isacommand{unfolding}\isamarkupfalse%
\ G{\isacharunderscore}{\kern0pt}dot{\isacharunderscore}{\kern0pt}def\ \isacommand{using}\isamarkupfalse%
\ elem{\isacharunderscore}{\kern0pt}of{\isacharunderscore}{\kern0pt}val{\isacharunderscore}{\kern0pt}pair\ G{\isacharunderscore}{\kern0pt}dot{\isacharunderscore}{\kern0pt}in{\isacharunderscore}{\kern0pt}M\isanewline
\ \ \ \ \isacommand{by}\isamarkupfalse%
\ force\isanewline
\ \ \isacommand{with}\isamarkupfalse%
\ {\isacartoucheopen}one{\isasymin}G{\isacartoucheclose}\ {\isacartoucheopen}G{\isasymsubseteq}P{\isacartoucheclose}\ \isacommand{show}\isamarkupfalse%
\isanewline
\ \ \ \ {\isachardoublequoteopen}x\ {\isasymin}\ G{\isachardoublequoteclose}\isanewline
\ \ \ \ \isacommand{using}\isamarkupfalse%
\ valcheck\ P{\isacharunderscore}{\kern0pt}sub{\isacharunderscore}{\kern0pt}M\ \isacommand{by}\isamarkupfalse%
\ auto\isanewline
\isacommand{next}\isamarkupfalse%
\isanewline
\ \ \isacommand{fix}\isamarkupfalse%
\ p\isanewline
\ \ \isacommand{assume}\isamarkupfalse%
\ {\isachardoublequoteopen}p{\isasymin}G{\isachardoublequoteclose}\isanewline
\ \ \isacommand{have}\isamarkupfalse%
\ {\isachardoublequoteopen}{\isasymlangle}check{\isacharparenleft}{\kern0pt}q{\isacharparenright}{\kern0pt}{\isacharcomma}{\kern0pt}q{\isasymrangle}\ {\isasymin}\ G{\isacharunderscore}{\kern0pt}dot{\isachardoublequoteclose}\ \isakeyword{if}\ {\isachardoublequoteopen}q{\isasymin}P{\isachardoublequoteclose}\ \isakeyword{for}\ q\isanewline
\ \ \ \ \isacommand{unfolding}\isamarkupfalse%
\ G{\isacharunderscore}{\kern0pt}dot{\isacharunderscore}{\kern0pt}def\ \isacommand{using}\isamarkupfalse%
\ that\ \isacommand{by}\isamarkupfalse%
\ simp\isanewline
\ \ \isacommand{with}\isamarkupfalse%
\ {\isacartoucheopen}p{\isasymin}G{\isacartoucheclose}\ {\isacartoucheopen}G{\isasymsubseteq}P{\isacartoucheclose}\isanewline
\ \ \isacommand{have}\isamarkupfalse%
\ {\isachardoublequoteopen}val{\isacharparenleft}{\kern0pt}G{\isacharcomma}{\kern0pt}check{\isacharparenleft}{\kern0pt}p{\isacharparenright}{\kern0pt}{\isacharparenright}{\kern0pt}\ {\isasymin}\ val{\isacharparenleft}{\kern0pt}G{\isacharcomma}{\kern0pt}G{\isacharunderscore}{\kern0pt}dot{\isacharparenright}{\kern0pt}{\isachardoublequoteclose}\isanewline
\ \ \ \ \isacommand{using}\isamarkupfalse%
\ val{\isacharunderscore}{\kern0pt}of{\isacharunderscore}{\kern0pt}elem\ G{\isacharunderscore}{\kern0pt}dot{\isacharunderscore}{\kern0pt}in{\isacharunderscore}{\kern0pt}M\ \isacommand{by}\isamarkupfalse%
\ blast\isanewline
\ \ \isacommand{with}\isamarkupfalse%
\ {\isacartoucheopen}p{\isasymin}G{\isacartoucheclose}\ {\isacartoucheopen}G{\isasymsubseteq}P{\isacartoucheclose}\ {\isacartoucheopen}one{\isasymin}G{\isacartoucheclose}\isanewline
\ \ \isacommand{show}\isamarkupfalse%
\ {\isachardoublequoteopen}p\ {\isasymin}\ val{\isacharparenleft}{\kern0pt}G{\isacharcomma}{\kern0pt}G{\isacharunderscore}{\kern0pt}dot{\isacharparenright}{\kern0pt}{\isachardoublequoteclose}\isanewline
\ \ \ \ \isacommand{using}\isamarkupfalse%
\ P{\isacharunderscore}{\kern0pt}sub{\isacharunderscore}{\kern0pt}M\ valcheck\ \isacommand{by}\isamarkupfalse%
\ auto\isanewline
\isacommand{qed}\isamarkupfalse%
%
\endisatagproof
{\isafoldproof}%
%
\isadelimproof
\isanewline
%
\endisadelimproof
\isanewline
\isanewline
\isacommand{lemma}\isamarkupfalse%
\ G{\isacharunderscore}{\kern0pt}in{\isacharunderscore}{\kern0pt}Gen{\isacharunderscore}{\kern0pt}Ext\ {\isacharcolon}{\kern0pt}\isanewline
\ \ \isakeyword{assumes}\ {\isachardoublequoteopen}G\ {\isasymsubseteq}\ P{\isachardoublequoteclose}\ \isakeyword{and}\ {\isachardoublequoteopen}one\ {\isasymin}\ G{\isachardoublequoteclose}\isanewline
\ \ \isakeyword{shows}\ \ \ {\isachardoublequoteopen}G\ {\isasymin}\ M{\isacharbrackleft}{\kern0pt}G{\isacharbrackright}{\kern0pt}{\isachardoublequoteclose}\isanewline
%
\isadelimproof
\ \ %
\endisadelimproof
%
\isatagproof
\isacommand{using}\isamarkupfalse%
\ assms\ val{\isacharunderscore}{\kern0pt}G{\isacharunderscore}{\kern0pt}dot\ GenExtI{\isacharbrackleft}{\kern0pt}of\ {\isacharunderscore}{\kern0pt}\ G{\isacharbrackright}{\kern0pt}\ G{\isacharunderscore}{\kern0pt}dot{\isacharunderscore}{\kern0pt}in{\isacharunderscore}{\kern0pt}M\isanewline
\ \ \isacommand{by}\isamarkupfalse%
\ force%
\endisatagproof
{\isafoldproof}%
%
\isadelimproof
\isanewline
%
\endisadelimproof
\isanewline
\isanewline
\isacommand{lemma}\isamarkupfalse%
\ fst{\isacharunderscore}{\kern0pt}snd{\isacharunderscore}{\kern0pt}closed{\isacharcolon}{\kern0pt}\ {\isachardoublequoteopen}p{\isasymin}M\ {\isasymLongrightarrow}\ fst{\isacharparenleft}{\kern0pt}p{\isacharparenright}{\kern0pt}\ {\isasymin}\ M\ {\isasymand}\ snd{\isacharparenleft}{\kern0pt}p{\isacharparenright}{\kern0pt}{\isasymin}\ M{\isachardoublequoteclose}\isanewline
%
\isadelimproof
%
\endisadelimproof
%
\isatagproof
\isacommand{proof}\isamarkupfalse%
\ {\isacharparenleft}{\kern0pt}cases\ {\isachardoublequoteopen}{\isasymexists}a{\isachardot}{\kern0pt}\ {\isasymexists}b{\isachardot}{\kern0pt}\ p\ {\isacharequal}{\kern0pt}\ {\isasymlangle}a{\isacharcomma}{\kern0pt}\ b{\isasymrangle}{\isachardoublequoteclose}{\isacharparenright}{\kern0pt}\isanewline
\ \ \isacommand{case}\isamarkupfalse%
\ False\isanewline
\ \ \isacommand{then}\isamarkupfalse%
\isanewline
\ \ \isacommand{show}\isamarkupfalse%
\ {\isachardoublequoteopen}fst{\isacharparenleft}{\kern0pt}p{\isacharparenright}{\kern0pt}\ {\isasymin}\ M\ {\isasymand}\ snd{\isacharparenleft}{\kern0pt}p{\isacharparenright}{\kern0pt}\ {\isasymin}\ M{\isachardoublequoteclose}\ \isacommand{unfolding}\isamarkupfalse%
\ fst{\isacharunderscore}{\kern0pt}def\ snd{\isacharunderscore}{\kern0pt}def\ \isacommand{using}\isamarkupfalse%
\ zero{\isacharunderscore}{\kern0pt}in{\isacharunderscore}{\kern0pt}M\ \isacommand{by}\isamarkupfalse%
\ auto\isanewline
\isacommand{next}\isamarkupfalse%
\isanewline
\ \ \isacommand{case}\isamarkupfalse%
\ True\isanewline
\ \ \isacommand{then}\isamarkupfalse%
\isanewline
\ \ \isacommand{obtain}\isamarkupfalse%
\ a\ b\ \isakeyword{where}\ {\isachardoublequoteopen}p\ {\isacharequal}{\kern0pt}\ {\isasymlangle}a{\isacharcomma}{\kern0pt}\ b{\isasymrangle}{\isachardoublequoteclose}\ \isacommand{by}\isamarkupfalse%
\ blast\isanewline
\ \ \isacommand{with}\isamarkupfalse%
\ True\isanewline
\ \ \isacommand{have}\isamarkupfalse%
\ {\isachardoublequoteopen}fst{\isacharparenleft}{\kern0pt}p{\isacharparenright}{\kern0pt}\ {\isacharequal}{\kern0pt}\ a{\isachardoublequoteclose}\ {\isachardoublequoteopen}snd{\isacharparenleft}{\kern0pt}p{\isacharparenright}{\kern0pt}\ {\isacharequal}{\kern0pt}\ b{\isachardoublequoteclose}\ \isacommand{unfolding}\isamarkupfalse%
\ fst{\isacharunderscore}{\kern0pt}def\ snd{\isacharunderscore}{\kern0pt}def\ \isacommand{by}\isamarkupfalse%
\ simp{\isacharunderscore}{\kern0pt}all\isanewline
\ \ \isacommand{moreover}\isamarkupfalse%
\isanewline
\ \ \isacommand{assume}\isamarkupfalse%
\ {\isachardoublequoteopen}p{\isasymin}M{\isachardoublequoteclose}\isanewline
\ \ \isacommand{moreover}\isamarkupfalse%
\ \isacommand{from}\isamarkupfalse%
\ this\isanewline
\ \ \isacommand{have}\isamarkupfalse%
\ {\isachardoublequoteopen}a{\isasymin}M{\isachardoublequoteclose}\isanewline
\ \ \ \ \isacommand{unfolding}\isamarkupfalse%
\ {\isacartoucheopen}p\ {\isacharequal}{\kern0pt}\ {\isacharunderscore}{\kern0pt}{\isacartoucheclose}\ Pair{\isacharunderscore}{\kern0pt}def\ \isacommand{by}\isamarkupfalse%
\ {\isacharparenleft}{\kern0pt}force\ intro{\isacharcolon}{\kern0pt}Transset{\isacharunderscore}{\kern0pt}M{\isacharbrackleft}{\kern0pt}OF\ trans{\isacharunderscore}{\kern0pt}M{\isacharbrackright}{\kern0pt}{\isacharparenright}{\kern0pt}\isanewline
\ \ \isacommand{moreover}\isamarkupfalse%
\ \isacommand{from}\isamarkupfalse%
\ \ {\isacartoucheopen}p{\isasymin}M{\isacartoucheclose}\isanewline
\ \ \isacommand{have}\isamarkupfalse%
\ {\isachardoublequoteopen}b{\isasymin}M{\isachardoublequoteclose}\isanewline
\ \ \ \ \isacommand{using}\isamarkupfalse%
\ Transset{\isacharunderscore}{\kern0pt}M{\isacharbrackleft}{\kern0pt}OF\ trans{\isacharunderscore}{\kern0pt}M{\isacharcomma}{\kern0pt}\ of\ {\isachardoublequoteopen}{\isacharbraceleft}{\kern0pt}a{\isacharcomma}{\kern0pt}b{\isacharbraceright}{\kern0pt}{\isachardoublequoteclose}\ p{\isacharbrackright}{\kern0pt}\ Transset{\isacharunderscore}{\kern0pt}M{\isacharbrackleft}{\kern0pt}OF\ trans{\isacharunderscore}{\kern0pt}M{\isacharcomma}{\kern0pt}\ of\ {\isachardoublequoteopen}b{\isachardoublequoteclose}\ {\isachardoublequoteopen}{\isacharbraceleft}{\kern0pt}a{\isacharcomma}{\kern0pt}b{\isacharbraceright}{\kern0pt}{\isachardoublequoteclose}{\isacharbrackright}{\kern0pt}\isanewline
\ \ \ \ \isacommand{unfolding}\isamarkupfalse%
\ {\isacartoucheopen}p\ {\isacharequal}{\kern0pt}\ {\isacharunderscore}{\kern0pt}{\isacartoucheclose}\ Pair{\isacharunderscore}{\kern0pt}def\ \isacommand{by}\isamarkupfalse%
\ {\isacharparenleft}{\kern0pt}simp{\isacharparenright}{\kern0pt}\isanewline
\ \ \isacommand{ultimately}\isamarkupfalse%
\isanewline
\ \ \isacommand{show}\isamarkupfalse%
\ {\isacharquery}{\kern0pt}thesis\ \isacommand{by}\isamarkupfalse%
\ simp\isanewline
\isacommand{qed}\isamarkupfalse%
%
\endisatagproof
{\isafoldproof}%
%
\isadelimproof
\isanewline
%
\endisadelimproof
\isanewline
\isacommand{end}\isamarkupfalse%
\ \isanewline
\isanewline
\isacommand{locale}\isamarkupfalse%
\ G{\isacharunderscore}{\kern0pt}generic\ {\isacharequal}{\kern0pt}\ forcing{\isacharunderscore}{\kern0pt}data\ {\isacharplus}{\kern0pt}\isanewline
\ \ \isakeyword{fixes}\ G\ {\isacharcolon}{\kern0pt}{\isacharcolon}{\kern0pt}\ {\isachardoublequoteopen}i{\isachardoublequoteclose}\isanewline
\ \ \isakeyword{assumes}\ generic\ {\isacharcolon}{\kern0pt}\ {\isachardoublequoteopen}M{\isacharunderscore}{\kern0pt}generic{\isacharparenleft}{\kern0pt}G{\isacharparenright}{\kern0pt}{\isachardoublequoteclose}\isanewline
\isakeyword{begin}\isanewline
\isanewline
\isacommand{lemma}\isamarkupfalse%
\ zero{\isacharunderscore}{\kern0pt}in{\isacharunderscore}{\kern0pt}MG\ {\isacharcolon}{\kern0pt}\isanewline
\ \ {\isachardoublequoteopen}{\isadigit{0}}\ {\isasymin}\ M{\isacharbrackleft}{\kern0pt}G{\isacharbrackright}{\kern0pt}{\isachardoublequoteclose}\isanewline
%
\isadelimproof
%
\endisadelimproof
%
\isatagproof
\isacommand{proof}\isamarkupfalse%
\ {\isacharminus}{\kern0pt}\isanewline
\ \ \isacommand{have}\isamarkupfalse%
\ {\isachardoublequoteopen}{\isadigit{0}}\ {\isacharequal}{\kern0pt}\ val{\isacharparenleft}{\kern0pt}G{\isacharcomma}{\kern0pt}{\isadigit{0}}{\isacharparenright}{\kern0pt}{\isachardoublequoteclose}\isanewline
\ \ \ \ \isacommand{using}\isamarkupfalse%
\ zero{\isacharunderscore}{\kern0pt}in{\isacharunderscore}{\kern0pt}M\ elem{\isacharunderscore}{\kern0pt}of{\isacharunderscore}{\kern0pt}val\ \isacommand{by}\isamarkupfalse%
\ auto\isanewline
\ \ \isacommand{also}\isamarkupfalse%
\ \isanewline
\ \ \isacommand{have}\isamarkupfalse%
\ {\isachardoublequoteopen}{\isachardot}{\kern0pt}{\isachardot}{\kern0pt}{\isachardot}{\kern0pt}\ {\isasymin}\ M{\isacharbrackleft}{\kern0pt}G{\isacharbrackright}{\kern0pt}{\isachardoublequoteclose}\isanewline
\ \ \ \ \isacommand{using}\isamarkupfalse%
\ GenExtI\ zero{\isacharunderscore}{\kern0pt}in{\isacharunderscore}{\kern0pt}M\ \isacommand{by}\isamarkupfalse%
\ simp\isanewline
\ \ \isacommand{finally}\isamarkupfalse%
\ \isacommand{show}\isamarkupfalse%
\ {\isacharquery}{\kern0pt}thesis\ \isacommand{{\isachardot}{\kern0pt}}\isamarkupfalse%
\isanewline
\isacommand{qed}\isamarkupfalse%
%
\endisatagproof
{\isafoldproof}%
%
\isadelimproof
\isanewline
%
\endisadelimproof
\isanewline
\isacommand{lemma}\isamarkupfalse%
\ G{\isacharunderscore}{\kern0pt}nonempty{\isacharcolon}{\kern0pt}\ {\isachardoublequoteopen}G{\isasymnoteq}{\isadigit{0}}{\isachardoublequoteclose}\isanewline
%
\isadelimproof
%
\endisadelimproof
%
\isatagproof
\isacommand{proof}\isamarkupfalse%
\ {\isacharminus}{\kern0pt}\isanewline
\ \ \isacommand{have}\isamarkupfalse%
\ {\isachardoublequoteopen}P{\isasymsubseteq}P{\isachardoublequoteclose}\ \isacommand{{\isachardot}{\kern0pt}{\isachardot}{\kern0pt}}\isamarkupfalse%
\isanewline
\ \ \isacommand{with}\isamarkupfalse%
\ P{\isacharunderscore}{\kern0pt}in{\isacharunderscore}{\kern0pt}M\ P{\isacharunderscore}{\kern0pt}dense\ {\isacartoucheopen}P{\isasymsubseteq}P{\isacartoucheclose}\isanewline
\ \ \isacommand{show}\isamarkupfalse%
\ {\isachardoublequoteopen}G\ {\isasymnoteq}\ {\isadigit{0}}{\isachardoublequoteclose}\isanewline
\ \ \ \ \isacommand{using}\isamarkupfalse%
\ generic\ \isacommand{unfolding}\isamarkupfalse%
\ M{\isacharunderscore}{\kern0pt}generic{\isacharunderscore}{\kern0pt}def\ \isacommand{by}\isamarkupfalse%
\ auto\isanewline
\isacommand{qed}\isamarkupfalse%
%
\endisatagproof
{\isafoldproof}%
%
\isadelimproof
\isanewline
%
\endisadelimproof
\isanewline
\isacommand{end}\isamarkupfalse%
\ \isanewline
%
\isadelimtheory
%
\endisadelimtheory
%
\isatagtheory
\isacommand{end}\isamarkupfalse%
%
\endisatagtheory
{\isafoldtheory}%
%
\isadelimtheory
%
\endisadelimtheory
%
\end{isabellebody}%
\endinput
%:%file=~/source/repos/ZF-notAC/code/Forcing/Names.thy%:%
%:%11=1%:%
%:%27=3%:%
%:%28=3%:%
%:%29=4%:%
%:%30=5%:%
%:%31=6%:%
%:%32=7%:%
%:%33=8%:%
%:%34=9%:%
%:%39=9%:%
%:%42=10%:%
%:%43=11%:%
%:%44=11%:%
%:%45=12%:%
%:%46=13%:%
%:%47=14%:%
%:%48=15%:%
%:%49=15%:%
%:%50=16%:%
%:%51=17%:%
%:%52=17%:%
%:%53=18%:%
%:%54=19%:%
%:%55=20%:%
%:%56=20%:%
%:%59=21%:%
%:%63=21%:%
%:%64=21%:%
%:%69=21%:%
%:%72=22%:%
%:%73=23%:%
%:%74=23%:%
%:%77=24%:%
%:%81=24%:%
%:%82=24%:%
%:%87=24%:%
%:%90=25%:%
%:%91=26%:%
%:%92=26%:%
%:%95=27%:%
%:%99=27%:%
%:%100=27%:%
%:%105=27%:%
%:%108=28%:%
%:%109=29%:%
%:%110=29%:%
%:%111=30%:%
%:%114=31%:%
%:%118=31%:%
%:%119=31%:%
%:%124=31%:%
%:%127=32%:%
%:%128=33%:%
%:%129=33%:%
%:%130=34%:%
%:%133=35%:%
%:%137=35%:%
%:%138=35%:%
%:%152=37%:%
%:%162=39%:%
%:%163=39%:%
%:%166=40%:%
%:%170=40%:%
%:%171=40%:%
%:%176=40%:%
%:%179=41%:%
%:%180=42%:%
%:%181=42%:%
%:%182=43%:%
%:%183=44%:%
%:%186=45%:%
%:%190=45%:%
%:%191=45%:%
%:%192=46%:%
%:%193=46%:%
%:%194=47%:%
%:%195=47%:%
%:%196=48%:%
%:%197=48%:%
%:%198=49%:%
%:%199=49%:%
%:%200=50%:%
%:%201=50%:%
%:%202=50%:%
%:%203=51%:%
%:%204=51%:%
%:%205=52%:%
%:%206=52%:%
%:%207=53%:%
%:%208=53%:%
%:%209=54%:%
%:%210=54%:%
%:%211=55%:%
%:%212=55%:%
%:%213=56%:%
%:%214=56%:%
%:%215=56%:%
%:%216=57%:%
%:%217=57%:%
%:%218=58%:%
%:%219=58%:%
%:%220=59%:%
%:%221=59%:%
%:%222=60%:%
%:%223=60%:%
%:%224=60%:%
%:%225=60%:%
%:%226=61%:%
%:%227=61%:%
%:%228=62%:%
%:%229=62%:%
%:%230=63%:%
%:%231=63%:%
%:%232=63%:%
%:%233=64%:%
%:%234=64%:%
%:%235=65%:%
%:%236=65%:%
%:%237=66%:%
%:%238=66%:%
%:%239=66%:%
%:%240=66%:%
%:%241=67%:%
%:%242=67%:%
%:%243=68%:%
%:%249=68%:%
%:%252=69%:%
%:%253=70%:%
%:%254=70%:%
%:%257=71%:%
%:%261=71%:%
%:%262=71%:%
%:%267=71%:%
%:%270=72%:%
%:%271=73%:%
%:%272=73%:%
%:%273=74%:%
%:%274=75%:%
%:%281=76%:%
%:%282=76%:%
%:%283=77%:%
%:%284=77%:%
%:%285=78%:%
%:%286=78%:%
%:%287=79%:%
%:%288=79%:%
%:%289=79%:%
%:%290=80%:%
%:%291=80%:%
%:%292=81%:%
%:%293=81%:%
%:%294=82%:%
%:%295=82%:%
%:%296=83%:%
%:%297=83%:%
%:%298=84%:%
%:%304=84%:%
%:%307=85%:%
%:%308=86%:%
%:%309=86%:%
%:%310=87%:%
%:%311=88%:%
%:%318=89%:%
%:%319=89%:%
%:%320=90%:%
%:%321=90%:%
%:%322=91%:%
%:%323=91%:%
%:%324=92%:%
%:%325=92%:%
%:%326=92%:%
%:%327=93%:%
%:%328=93%:%
%:%329=94%:%
%:%330=94%:%
%:%331=95%:%
%:%332=95%:%
%:%333=96%:%
%:%334=96%:%
%:%335=97%:%
%:%336=97%:%
%:%337=98%:%
%:%347=100%:%
%:%349=101%:%
%:%350=101%:%
%:%351=102%:%
%:%352=103%:%
%:%353=104%:%
%:%354=105%:%
%:%355=105%:%
%:%356=106%:%
%:%357=107%:%
%:%358=108%:%
%:%359=109%:%
%:%360=110%:%
%:%361=110%:%
%:%364=111%:%
%:%368=111%:%
%:%369=111%:%
%:%375=111%:%
%:%378=112%:%
%:%379=113%:%
%:%380=113%:%
%:%383=114%:%
%:%387=114%:%
%:%388=114%:%
%:%394=114%:%
%:%397=115%:%
%:%398=116%:%
%:%399=117%:%
%:%400=117%:%
%:%401=118%:%
%:%402=119%:%
%:%409=120%:%
%:%410=120%:%
%:%411=121%:%
%:%412=121%:%
%:%413=122%:%
%:%414=122%:%
%:%415=122%:%
%:%416=123%:%
%:%417=123%:%
%:%418=124%:%
%:%419=124%:%
%:%420=124%:%
%:%421=124%:%
%:%422=125%:%
%:%423=125%:%
%:%424=126%:%
%:%425=126%:%
%:%426=127%:%
%:%427=127%:%
%:%428=127%:%
%:%429=128%:%
%:%430=128%:%
%:%431=129%:%
%:%432=129%:%
%:%433=130%:%
%:%434=130%:%
%:%435=130%:%
%:%436=131%:%
%:%442=131%:%
%:%445=132%:%
%:%446=133%:%
%:%447=133%:%
%:%450=134%:%
%:%454=134%:%
%:%455=134%:%
%:%460=134%:%
%:%463=135%:%
%:%464=136%:%
%:%465=136%:%
%:%468=137%:%
%:%472=137%:%
%:%473=137%:%
%:%478=137%:%
%:%481=138%:%
%:%482=139%:%
%:%483=139%:%
%:%486=140%:%
%:%490=140%:%
%:%491=140%:%
%:%496=140%:%
%:%499=141%:%
%:%500=142%:%
%:%501=142%:%
%:%504=143%:%
%:%508=143%:%
%:%509=143%:%
%:%514=143%:%
%:%517=144%:%
%:%518=145%:%
%:%519=145%:%
%:%522=146%:%
%:%526=146%:%
%:%527=146%:%
%:%532=146%:%
%:%535=147%:%
%:%536=148%:%
%:%537=148%:%
%:%540=149%:%
%:%544=149%:%
%:%545=149%:%
%:%550=149%:%
%:%553=150%:%
%:%554=151%:%
%:%555=151%:%
%:%558=152%:%
%:%562=152%:%
%:%563=152%:%
%:%568=152%:%
%:%571=153%:%
%:%572=154%:%
%:%573=154%:%
%:%580=155%:%
%:%581=155%:%
%:%582=156%:%
%:%583=156%:%
%:%584=157%:%
%:%585=157%:%
%:%586=158%:%
%:%587=159%:%
%:%588=159%:%
%:%589=159%:%
%:%590=160%:%
%:%591=161%:%
%:%592=161%:%
%:%593=162%:%
%:%594=162%:%
%:%595=163%:%
%:%596=163%:%
%:%597=163%:%
%:%598=164%:%
%:%599=165%:%
%:%600=165%:%
%:%601=165%:%
%:%602=166%:%
%:%603=166%:%
%:%604=166%:%
%:%605=167%:%
%:%606=168%:%
%:%607=168%:%
%:%608=169%:%
%:%609=169%:%
%:%610=170%:%
%:%611=171%:%
%:%612=171%:%
%:%613=171%:%
%:%614=172%:%
%:%615=173%:%
%:%616=173%:%
%:%617=174%:%
%:%618=174%:%
%:%619=174%:%
%:%620=175%:%
%:%621=176%:%
%:%622=176%:%
%:%623=177%:%
%:%624=177%:%
%:%625=177%:%
%:%626=177%:%
%:%627=178%:%
%:%633=178%:%
%:%636=179%:%
%:%637=180%:%
%:%638=180%:%
%:%641=181%:%
%:%645=181%:%
%:%646=181%:%
%:%647=182%:%
%:%648=183%:%
%:%649=183%:%
%:%654=183%:%
%:%657=184%:%
%:%658=185%:%
%:%659=185%:%
%:%660=186%:%
%:%661=187%:%
%:%668=188%:%
%:%669=188%:%
%:%670=189%:%
%:%671=189%:%
%:%672=190%:%
%:%673=190%:%
%:%674=190%:%
%:%675=190%:%
%:%676=190%:%
%:%677=191%:%
%:%678=191%:%
%:%679=192%:%
%:%680=192%:%
%:%681=193%:%
%:%682=193%:%
%:%683=193%:%
%:%684=193%:%
%:%685=193%:%
%:%686=194%:%
%:%687=194%:%
%:%688=195%:%
%:%689=195%:%
%:%690=196%:%
%:%691=196%:%
%:%692=197%:%
%:%693=197%:%
%:%694=198%:%
%:%695=198%:%
%:%696=198%:%
%:%697=199%:%
%:%703=199%:%
%:%706=200%:%
%:%707=201%:%
%:%708=201%:%
%:%715=202%:%
%:%716=202%:%
%:%717=203%:%
%:%718=203%:%
%:%719=204%:%
%:%720=204%:%
%:%721=205%:%
%:%722=205%:%
%:%723=206%:%
%:%724=206%:%
%:%725=207%:%
%:%726=207%:%
%:%727=208%:%
%:%728=208%:%
%:%729=209%:%
%:%730=209%:%
%:%731=210%:%
%:%732=210%:%
%:%733=211%:%
%:%739=211%:%
%:%742=212%:%
%:%743=213%:%
%:%744=213%:%
%:%747=214%:%
%:%751=214%:%
%:%752=214%:%
%:%757=214%:%
%:%760=215%:%
%:%761=216%:%
%:%762=216%:%
%:%765=217%:%
%:%769=217%:%
%:%770=217%:%
%:%775=217%:%
%:%778=218%:%
%:%779=219%:%
%:%780=220%:%
%:%781=220%:%
%:%782=221%:%
%:%783=222%:%
%:%790=223%:%
%:%791=223%:%
%:%792=224%:%
%:%793=224%:%
%:%794=225%:%
%:%795=225%:%
%:%796=226%:%
%:%797=226%:%
%:%798=227%:%
%:%799=227%:%
%:%800=228%:%
%:%801=228%:%
%:%802=229%:%
%:%803=229%:%
%:%804=229%:%
%:%805=230%:%
%:%806=230%:%
%:%807=231%:%
%:%808=231%:%
%:%809=232%:%
%:%810=232%:%
%:%811=233%:%
%:%812=233%:%
%:%813=233%:%
%:%814=233%:%
%:%815=234%:%
%:%816=234%:%
%:%817=235%:%
%:%818=235%:%
%:%819=235%:%
%:%820=236%:%
%:%821=236%:%
%:%822=237%:%
%:%823=237%:%
%:%824=238%:%
%:%825=238%:%
%:%826=239%:%
%:%827=239%:%
%:%828=240%:%
%:%829=240%:%
%:%830=241%:%
%:%831=241%:%
%:%832=242%:%
%:%833=242%:%
%:%834=242%:%
%:%835=243%:%
%:%836=243%:%
%:%837=243%:%
%:%838=244%:%
%:%839=244%:%
%:%840=245%:%
%:%841=245%:%
%:%842=246%:%
%:%843=246%:%
%:%844=246%:%
%:%845=246%:%
%:%846=247%:%
%:%847=247%:%
%:%848=248%:%
%:%849=248%:%
%:%850=249%:%
%:%851=249%:%
%:%852=249%:%
%:%853=249%:%
%:%854=250%:%
%:%855=250%:%
%:%856=251%:%
%:%857=251%:%
%:%858=252%:%
%:%859=252%:%
%:%860=252%:%
%:%861=253%:%
%:%862=253%:%
%:%863=254%:%
%:%864=254%:%
%:%865=254%:%
%:%866=255%:%
%:%872=255%:%
%:%875=256%:%
%:%876=257%:%
%:%877=257%:%
%:%878=258%:%
%:%879=259%:%
%:%886=260%:%
%:%887=260%:%
%:%888=261%:%
%:%889=261%:%
%:%890=262%:%
%:%891=262%:%
%:%892=263%:%
%:%893=263%:%
%:%894=264%:%
%:%895=264%:%
%:%896=265%:%
%:%897=265%:%
%:%898=265%:%
%:%899=265%:%
%:%900=266%:%
%:%901=266%:%
%:%902=267%:%
%:%903=267%:%
%:%904=267%:%
%:%905=267%:%
%:%906=268%:%
%:%912=268%:%
%:%915=269%:%
%:%916=270%:%
%:%917=271%:%
%:%918=271%:%
%:%919=272%:%
%:%920=273%:%
%:%921=274%:%
%:%922=274%:%
%:%925=275%:%
%:%929=275%:%
%:%930=275%:%
%:%935=275%:%
%:%938=276%:%
%:%939=277%:%
%:%940=277%:%
%:%943=278%:%
%:%947=278%:%
%:%948=278%:%
%:%953=278%:%
%:%956=279%:%
%:%957=280%:%
%:%958=280%:%
%:%961=281%:%
%:%965=281%:%
%:%966=281%:%
%:%971=281%:%
%:%974=282%:%
%:%975=283%:%
%:%983=285%:%
%:%993=286%:%
%:%994=286%:%
%:%995=287%:%
%:%996=288%:%
%:%997=289%:%
%:%998=289%:%
%:%999=290%:%
%:%1000=291%:%
%:%1001=292%:%
%:%1002=293%:%
%:%1003=293%:%
%:%1004=294%:%
%:%1005=295%:%
%:%1006=296%:%
%:%1007=297%:%
%:%1008=297%:%
%:%1009=298%:%
%:%1012=299%:%
%:%1016=299%:%
%:%1017=299%:%
%:%1023=299%:%
%:%1026=300%:%
%:%1027=301%:%
%:%1028=301%:%
%:%1029=302%:%
%:%1030=303%:%
%:%1031=304%:%
%:%1032=305%:%
%:%1033=306%:%
%:%1034=306%:%
%:%1035=307%:%
%:%1038=308%:%
%:%1042=308%:%
%:%1043=308%:%
%:%1044=309%:%
%:%1045=309%:%
%:%1046=309%:%
%:%1051=309%:%
%:%1054=310%:%
%:%1055=311%:%
%:%1056=311%:%
%:%1059=312%:%
%:%1063=312%:%
%:%1064=312%:%
%:%1065=312%:%
%:%1066=312%:%
%:%1071=312%:%
%:%1074=313%:%
%:%1075=314%:%
%:%1076=315%:%
%:%1077=315%:%
%:%1078=316%:%
%:%1081=317%:%
%:%1085=317%:%
%:%1086=317%:%
%:%1087=317%:%
%:%1092=317%:%
%:%1095=318%:%
%:%1096=319%:%
%:%1097=320%:%
%:%1098=320%:%
%:%1100=322%:%
%:%1103=323%:%
%:%1107=323%:%
%:%1108=323%:%
%:%1113=323%:%
%:%1116=324%:%
%:%1117=325%:%
%:%1118=325%:%
%:%1125=326%:%
%:%1126=326%:%
%:%1127=327%:%
%:%1128=327%:%
%:%1129=328%:%
%:%1130=329%:%
%:%1131=329%:%
%:%1132=330%:%
%:%1133=330%:%
%:%1134=330%:%
%:%1135=331%:%
%:%1136=331%:%
%:%1137=332%:%
%:%1138=332%:%
%:%1139=333%:%
%:%1140=333%:%
%:%1141=333%:%
%:%1142=334%:%
%:%1143=334%:%
%:%1144=335%:%
%:%1145=335%:%
%:%1146=336%:%
%:%1147=336%:%
%:%1148=336%:%
%:%1149=337%:%
%:%1150=337%:%
%:%1151=338%:%
%:%1152=338%:%
%:%1153=339%:%
%:%1154=339%:%
%:%1155=339%:%
%:%1156=340%:%
%:%1157=340%:%
%:%1158=340%:%
%:%1159=340%:%
%:%1160=340%:%
%:%1161=341%:%
%:%1167=341%:%
%:%1170=342%:%
%:%1171=343%:%
%:%1172=344%:%
%:%1173=344%:%
%:%1174=345%:%
%:%1175=346%:%
%:%1176=347%:%
%:%1183=348%:%
%:%1184=348%:%
%:%1185=349%:%
%:%1186=349%:%
%:%1187=350%:%
%:%1188=350%:%
%:%1189=350%:%
%:%1190=351%:%
%:%1191=351%:%
%:%1192=351%:%
%:%1193=352%:%
%:%1194=352%:%
%:%1195=353%:%
%:%1196=353%:%
%:%1197=353%:%
%:%1198=354%:%
%:%1199=354%:%
%:%1200=354%:%
%:%1201=355%:%
%:%1202=355%:%
%:%1203=355%:%
%:%1204=355%:%
%:%1205=356%:%
%:%1211=356%:%
%:%1214=357%:%
%:%1215=358%:%
%:%1216=358%:%
%:%1217=359%:%
%:%1218=360%:%
%:%1225=361%:%
%:%1226=361%:%
%:%1227=362%:%
%:%1228=362%:%
%:%1229=363%:%
%:%1230=363%:%
%:%1231=363%:%
%:%1232=364%:%
%:%1233=364%:%
%:%1234=365%:%
%:%1235=365%:%
%:%1236=365%:%
%:%1237=365%:%
%:%1238=366%:%
%:%1244=366%:%
%:%1247=367%:%
%:%1248=368%:%
%:%1249=368%:%
%:%1250=369%:%
%:%1251=370%:%
%:%1253=372%:%
%:%1254=373%:%
%:%1255=374%:%
%:%1257=376%:%
%:%1258=376%:%
%:%1259=377%:%
%:%1260=378%:%
%:%1261=379%:%
%:%1262=380%:%
%:%1263=380%:%
%:%1264=381%:%
%:%1265=382%:%
%:%1272=383%:%
%:%1273=383%:%
%:%1274=384%:%
%:%1275=384%:%
%:%1276=385%:%
%:%1277=385%:%
%:%1278=385%:%
%:%1279=385%:%
%:%1280=386%:%
%:%1281=386%:%
%:%1282=387%:%
%:%1283=387%:%
%:%1284=387%:%
%:%1285=387%:%
%:%1286=388%:%
%:%1287=388%:%
%:%1288=389%:%
%:%1289=389%:%
%:%1290=389%:%
%:%1291=389%:%
%:%1292=390%:%
%:%1293=390%:%
%:%1294=390%:%
%:%1295=391%:%
%:%1296=391%:%
%:%1297=391%:%
%:%1298=391%:%
%:%1299=392%:%
%:%1300=392%:%
%:%1301=393%:%
%:%1302=393%:%
%:%1303=394%:%
%:%1304=394%:%
%:%1305=394%:%
%:%1306=395%:%
%:%1307=395%:%
%:%1308=395%:%
%:%1309=396%:%
%:%1310=396%:%
%:%1311=397%:%
%:%1312=397%:%
%:%1313=397%:%
%:%1314=398%:%
%:%1315=398%:%
%:%1316=398%:%
%:%1317=398%:%
%:%1318=399%:%
%:%1328=401%:%
%:%1330=403%:%
%:%1331=403%:%
%:%1338=404%:%
%:%1339=404%:%
%:%1340=405%:%
%:%1341=405%:%
%:%1342=406%:%
%:%1343=407%:%
%:%1344=407%:%
%:%1345=408%:%
%:%1346=409%:%
%:%1347=409%:%
%:%1348=409%:%
%:%1349=409%:%
%:%1350=410%:%
%:%1351=410%:%
%:%1352=411%:%
%:%1353=411%:%
%:%1354=411%:%
%:%1355=411%:%
%:%1356=412%:%
%:%1357=412%:%
%:%1358=413%:%
%:%1359=413%:%
%:%1360=414%:%
%:%1361=414%:%
%:%1362=414%:%
%:%1363=415%:%
%:%1364=415%:%
%:%1365=416%:%
%:%1366=416%:%
%:%1367=417%:%
%:%1368=417%:%
%:%1369=417%:%
%:%1370=418%:%
%:%1371=418%:%
%:%1372=419%:%
%:%1373=419%:%
%:%1374=419%:%
%:%1375=419%:%
%:%1376=420%:%
%:%1382=420%:%
%:%1385=421%:%
%:%1386=422%:%
%:%1387=422%:%
%:%1390=423%:%
%:%1394=423%:%
%:%1395=423%:%
%:%1404=425%:%
%:%1405=426%:%
%:%1407=428%:%
%:%1408=428%:%
%:%1415=429%:%
%:%1416=429%:%
%:%1417=430%:%
%:%1418=430%:%
%:%1419=431%:%
%:%1420=431%:%
%:%1421=431%:%
%:%1422=432%:%
%:%1423=432%:%
%:%1424=433%:%
%:%1425=433%:%
%:%1426=434%:%
%:%1427=434%:%
%:%1428=434%:%
%:%1429=435%:%
%:%1430=435%:%
%:%1431=436%:%
%:%1432=436%:%
%:%1433=437%:%
%:%1434=437%:%
%:%1435=438%:%
%:%1436=438%:%
%:%1437=439%:%
%:%1438=439%:%
%:%1439=440%:%
%:%1440=440%:%
%:%1441=440%:%
%:%1442=441%:%
%:%1443=441%:%
%:%1444=442%:%
%:%1445=442%:%
%:%1446=443%:%
%:%1447=443%:%
%:%1448=443%:%
%:%1449=444%:%
%:%1450=444%:%
%:%1451=445%:%
%:%1452=445%:%
%:%1453=446%:%
%:%1454=446%:%
%:%1455=447%:%
%:%1456=447%:%
%:%1457=448%:%
%:%1458=448%:%
%:%1459=449%:%
%:%1460=449%:%
%:%1461=449%:%
%:%1462=450%:%
%:%1463=450%:%
%:%1464=451%:%
%:%1470=451%:%
%:%1473=452%:%
%:%1474=453%:%
%:%1475=453%:%
%:%1476=454%:%
%:%1483=455%:%
%:%1484=455%:%
%:%1485=456%:%
%:%1486=456%:%
%:%1487=457%:%
%:%1488=458%:%
%:%1489=459%:%
%:%1490=459%:%
%:%1491=460%:%
%:%1492=461%:%
%:%1493=461%:%
%:%1494=462%:%
%:%1495=463%:%
%:%1496=463%:%
%:%1497=464%:%
%:%1498=464%:%
%:%1499=464%:%
%:%1500=465%:%
%:%1501=465%:%
%:%1502=465%:%
%:%1503=466%:%
%:%1504=466%:%
%:%1505=467%:%
%:%1506=467%:%
%:%1507=467%:%
%:%1508=468%:%
%:%1509=468%:%
%:%1510=469%:%
%:%1511=469%:%
%:%1512=469%:%
%:%1513=470%:%
%:%1514=470%:%
%:%1515=470%:%
%:%1516=471%:%
%:%1517=471%:%
%:%1518=472%:%
%:%1519=472%:%
%:%1520=473%:%
%:%1521=473%:%
%:%1522=474%:%
%:%1523=474%:%
%:%1524=475%:%
%:%1525=475%:%
%:%1526=476%:%
%:%1527=476%:%
%:%1528=477%:%
%:%1529=477%:%
%:%1530=478%:%
%:%1531=478%:%
%:%1532=479%:%
%:%1533=479%:%
%:%1534=480%:%
%:%1535=480%:%
%:%1536=481%:%
%:%1537=481%:%
%:%1538=482%:%
%:%1539=482%:%
%:%1540=483%:%
%:%1541=483%:%
%:%1542=484%:%
%:%1543=484%:%
%:%1544=485%:%
%:%1545=485%:%
%:%1546=486%:%
%:%1547=486%:%
%:%1548=487%:%
%:%1549=487%:%
%:%1550=488%:%
%:%1551=488%:%
%:%1552=488%:%
%:%1553=489%:%
%:%1554=489%:%
%:%1555=490%:%
%:%1556=490%:%
%:%1557=491%:%
%:%1558=491%:%
%:%1559=492%:%
%:%1560=492%:%
%:%1561=493%:%
%:%1562=493%:%
%:%1563=494%:%
%:%1564=495%:%
%:%1565=495%:%
%:%1566=496%:%
%:%1567=496%:%
%:%1568=497%:%
%:%1569=497%:%
%:%1570=498%:%
%:%1571=498%:%
%:%1572=499%:%
%:%1573=499%:%
%:%1574=500%:%
%:%1575=500%:%
%:%1576=501%:%
%:%1577=501%:%
%:%1578=502%:%
%:%1579=502%:%
%:%1580=503%:%
%:%1586=503%:%
%:%1589=504%:%
%:%1590=505%:%
%:%1591=505%:%
%:%1592=506%:%
%:%1595=507%:%
%:%1599=507%:%
%:%1600=507%:%
%:%1601=507%:%
%:%1606=507%:%
%:%1609=508%:%
%:%1610=509%:%
%:%1611=509%:%
%:%1612=510%:%
%:%1619=511%:%
%:%1620=511%:%
%:%1621=512%:%
%:%1622=512%:%
%:%1623=513%:%
%:%1624=513%:%
%:%1625=513%:%
%:%1626=514%:%
%:%1627=514%:%
%:%1628=515%:%
%:%1629=515%:%
%:%1630=516%:%
%:%1631=516%:%
%:%1632=516%:%
%:%1633=517%:%
%:%1634=517%:%
%:%1635=518%:%
%:%1636=518%:%
%:%1637=519%:%
%:%1638=519%:%
%:%1639=520%:%
%:%1640=520%:%
%:%1641=521%:%
%:%1642=521%:%
%:%1643=522%:%
%:%1644=522%:%
%:%1645=522%:%
%:%1646=523%:%
%:%1647=523%:%
%:%1648=524%:%
%:%1649=524%:%
%:%1650=525%:%
%:%1651=525%:%
%:%1652=525%:%
%:%1653=526%:%
%:%1654=526%:%
%:%1655=527%:%
%:%1656=527%:%
%:%1657=527%:%
%:%1658=528%:%
%:%1664=528%:%
%:%1667=529%:%
%:%1668=530%:%
%:%1669=530%:%
%:%1676=531%:%
%:%1677=531%:%
%:%1678=532%:%
%:%1679=532%:%
%:%1680=533%:%
%:%1681=534%:%
%:%1682=534%:%
%:%1683=534%:%
%:%1684=535%:%
%:%1685=535%:%
%:%1686=536%:%
%:%1687=536%:%
%:%1688=537%:%
%:%1689=537%:%
%:%1690=537%:%
%:%1691=538%:%
%:%1692=538%:%
%:%1693=539%:%
%:%1694=539%:%
%:%1695=539%:%
%:%1696=540%:%
%:%1697=540%:%
%:%1698=541%:%
%:%1699=541%:%
%:%1700=542%:%
%:%1701=542%:%
%:%1702=542%:%
%:%1703=543%:%
%:%1709=543%:%
%:%1712=544%:%
%:%1713=545%:%
%:%1714=545%:%
%:%1717=546%:%
%:%1721=546%:%
%:%1722=546%:%
%:%1723=547%:%
%:%1724=547%:%
%:%1729=547%:%
%:%1732=548%:%
%:%1733=549%:%
%:%1734=549%:%
%:%1737=550%:%
%:%1741=550%:%
%:%1742=550%:%
%:%1743=551%:%
%:%1744=551%:%
%:%1749=551%:%
%:%1752=552%:%
%:%1753=553%:%
%:%1754=553%:%
%:%1755=554%:%
%:%1756=555%:%
%:%1757=556%:%
%:%1758=557%:%
%:%1759=558%:%
%:%1760=558%:%
%:%1767=559%:%
%:%1768=559%:%
%:%1769=560%:%
%:%1770=560%:%
%:%1771=561%:%
%:%1772=562%:%
%:%1773=562%:%
%:%1774=563%:%
%:%1775=563%:%
%:%1776=563%:%
%:%1777=564%:%
%:%1778=564%:%
%:%1779=565%:%
%:%1780=565%:%
%:%1781=566%:%
%:%1782=566%:%
%:%1783=567%:%
%:%1784=567%:%
%:%1785=568%:%
%:%1786=568%:%
%:%1787=569%:%
%:%1788=569%:%
%:%1789=569%:%
%:%1790=570%:%
%:%1796=570%:%
%:%1799=571%:%
%:%1800=572%:%
%:%1801=572%:%
%:%1804=573%:%
%:%1808=573%:%
%:%1809=573%:%
%:%1814=573%:%
%:%1817=574%:%
%:%1818=575%:%
%:%1819=575%:%
%:%1822=576%:%
%:%1826=576%:%
%:%1827=576%:%
%:%1832=576%:%
%:%1835=577%:%
%:%1836=578%:%
%:%1837=578%:%
%:%1838=579%:%
%:%1839=580%:%
%:%1846=581%:%
%:%1847=581%:%
%:%1848=582%:%
%:%1849=582%:%
%:%1850=583%:%
%:%1851=583%:%
%:%1852=584%:%
%:%1853=584%:%
%:%1854=584%:%
%:%1855=585%:%
%:%1856=585%:%
%:%1857=585%:%
%:%1858=586%:%
%:%1859=586%:%
%:%1860=587%:%
%:%1861=587%:%
%:%1862=588%:%
%:%1863=588%:%
%:%1864=589%:%
%:%1865=589%:%
%:%1866=590%:%
%:%1867=590%:%
%:%1868=590%:%
%:%1869=591%:%
%:%1875=591%:%
%:%1878=592%:%
%:%1879=593%:%
%:%1880=594%:%
%:%1881=594%:%
%:%1882=595%:%
%:%1885=596%:%
%:%1889=596%:%
%:%1890=596%:%
%:%1895=596%:%
%:%1898=597%:%
%:%1899=598%:%
%:%1900=598%:%
%:%1901=599%:%
%:%1904=600%:%
%:%1908=600%:%
%:%1909=600%:%
%:%1914=600%:%
%:%1917=601%:%
%:%1918=602%:%
%:%1919=602%:%
%:%1926=603%:%
%:%1927=603%:%
%:%1928=604%:%
%:%1929=604%:%
%:%1930=604%:%
%:%1931=605%:%
%:%1932=605%:%
%:%1933=606%:%
%:%1934=606%:%
%:%1935=606%:%
%:%1936=607%:%
%:%1937=607%:%
%:%1938=607%:%
%:%1939=608%:%
%:%1940=608%:%
%:%1941=609%:%
%:%1942=609%:%
%:%1943=610%:%
%:%1944=610%:%
%:%1945=610%:%
%:%1946=611%:%
%:%1947=611%:%
%:%1948=612%:%
%:%1949=612%:%
%:%1950=613%:%
%:%1951=613%:%
%:%1952=613%:%
%:%1953=614%:%
%:%1954=614%:%
%:%1955=615%:%
%:%1956=615%:%
%:%1957=616%:%
%:%1958=616%:%
%:%1959=616%:%
%:%1960=616%:%
%:%1961=617%:%
%:%1967=617%:%
%:%1970=618%:%
%:%1971=619%:%
%:%1972=619%:%
%:%1973=620%:%
%:%1974=621%:%
%:%1975=621%:%
%:%1976=622%:%
%:%1977=623%:%
%:%1978=624%:%
%:%1981=625%:%
%:%1985=625%:%
%:%1986=625%:%
%:%1987=626%:%
%:%1988=626%:%
%:%1989=627%:%
%:%1990=627%:%
%:%1991=628%:%
%:%1992=628%:%
%:%1993=628%:%
%:%1994=628%:%
%:%1995=628%:%
%:%1996=629%:%
%:%1997=629%:%
%:%1998=630%:%
%:%1999=630%:%
%:%2000=631%:%
%:%2001=631%:%
%:%2002=631%:%
%:%2003=631%:%
%:%2004=632%:%
%:%2005=632%:%
%:%2006=633%:%
%:%2007=633%:%
%:%2008=634%:%
%:%2009=634%:%
%:%2010=634%:%
%:%2011=635%:%
%:%2012=635%:%
%:%2013=636%:%
%:%2014=636%:%
%:%2015=637%:%
%:%2016=637%:%
%:%2017=637%:%
%:%2018=638%:%
%:%2019=638%:%
%:%2020=639%:%
%:%2021=639%:%
%:%2022=640%:%
%:%2023=640%:%
%:%2024=640%:%
%:%2025=641%:%
%:%2026=641%:%
%:%2027=641%:%
%:%2028=641%:%
%:%2029=641%:%
%:%2030=642%:%
%:%2036=642%:%
%:%2039=643%:%
%:%2040=644%:%
%:%2041=645%:%
%:%2042=645%:%
%:%2043=646%:%
%:%2044=647%:%
%:%2045=648%:%
%:%2046=649%:%
%:%2047=649%:%
%:%2048=650%:%
%:%2049=651%:%
%:%2050=652%:%
%:%2051=653%:%
%:%2052=653%:%
%:%2055=654%:%
%:%2059=654%:%
%:%2060=654%:%
%:%2065=654%:%
%:%2068=655%:%
%:%2069=656%:%
%:%2070=656%:%
%:%2071=657%:%
%:%2072=658%:%
%:%2073=659%:%
%:%2074=660%:%
%:%2081=661%:%
%:%2082=661%:%
%:%2083=662%:%
%:%2084=662%:%
%:%2085=663%:%
%:%2086=663%:%
%:%2087=664%:%
%:%2088=664%:%
%:%2089=664%:%
%:%2090=665%:%
%:%2091=665%:%
%:%2092=666%:%
%:%2093=666%:%
%:%2094=667%:%
%:%2095=667%:%
%:%2096=668%:%
%:%2097=668%:%
%:%2098=669%:%
%:%2099=669%:%
%:%2100=670%:%
%:%2101=670%:%
%:%2102=671%:%
%:%2103=671%:%
%:%2104=672%:%
%:%2105=672%:%
%:%2106=673%:%
%:%2112=673%:%
%:%2115=674%:%
%:%2116=677%:%
%:%2117=678%:%
%:%2118=678%:%
%:%2119=679%:%
%:%2120=680%:%
%:%2121=681%:%
%:%2122=682%:%
%:%2123=683%:%
%:%2124=683%:%
%:%2125=684%:%
%:%2128=685%:%
%:%2132=685%:%
%:%2133=685%:%
%:%2138=685%:%
%:%2141=686%:%
%:%2142=687%:%
%:%2143=687%:%
%:%2144=688%:%
%:%2146=690%:%
%:%2149=691%:%
%:%2153=691%:%
%:%2154=691%:%
%:%2155=691%:%
%:%2160=691%:%
%:%2163=692%:%
%:%2164=695%:%
%:%2165=696%:%
%:%2166=696%:%
%:%2167=697%:%
%:%2168=698%:%
%:%2169=699%:%
%:%2170=700%:%
%:%2171=700%:%
%:%2172=701%:%
%:%2175=702%:%
%:%2179=702%:%
%:%2180=702%:%
%:%2185=702%:%
%:%2188=703%:%
%:%2189=704%:%
%:%2190=704%:%
%:%2191=705%:%
%:%2193=707%:%
%:%2196=708%:%
%:%2200=708%:%
%:%2201=708%:%
%:%2202=708%:%
%:%2203=709%:%
%:%2204=709%:%
%:%2209=709%:%
%:%2212=710%:%
%:%2213=711%:%
%:%2214=712%:%
%:%2215=713%:%
%:%2216=713%:%
%:%2217=714%:%
%:%2218=715%:%
%:%2219=716%:%
%:%2220=717%:%
%:%2227=718%:%
%:%2228=718%:%
%:%2229=719%:%
%:%2230=719%:%
%:%2231=720%:%
%:%2232=721%:%
%:%2233=722%:%
%:%2234=723%:%
%:%2235=723%:%
%:%2236=723%:%
%:%2237=724%:%
%:%2238=724%:%
%:%2239=724%:%
%:%2241=726%:%
%:%2242=727%:%
%:%2243=728%:%
%:%2244=728%:%
%:%2245=728%:%
%:%2246=729%:%
%:%2247=729%:%
%:%2248=730%:%
%:%2249=731%:%
%:%2250=732%:%
%:%2251=732%:%
%:%2252=733%:%
%:%2253=734%:%
%:%2254=735%:%
%:%2255=735%:%
%:%2256=735%:%
%:%2257=736%:%
%:%2258=736%:%
%:%2259=737%:%
%:%2260=737%:%
%:%2261=738%:%
%:%2262=739%:%
%:%2263=740%:%
%:%2264=740%:%
%:%2265=741%:%
%:%2266=741%:%
%:%2267=742%:%
%:%2268=742%:%
%:%2269=743%:%
%:%2270=743%:%
%:%2271=743%:%
%:%2272=744%:%
%:%2273=744%:%
%:%2274=745%:%
%:%2275=745%:%
%:%2276=746%:%
%:%2277=747%:%
%:%2278=747%:%
%:%2279=747%:%
%:%2280=748%:%
%:%2281=748%:%
%:%2282=749%:%
%:%2283=749%:%
%:%2284=749%:%
%:%2285=749%:%
%:%2286=750%:%
%:%2292=750%:%
%:%2295=751%:%
%:%2296=752%:%
%:%2297=752%:%
%:%2298=753%:%
%:%2299=754%:%
%:%2300=755%:%
%:%2301=756%:%
%:%2308=757%:%
%:%2309=757%:%
%:%2310=758%:%
%:%2311=758%:%
%:%2312=759%:%
%:%2313=759%:%
%:%2314=760%:%
%:%2315=761%:%
%:%2316=761%:%
%:%2317=762%:%
%:%2318=762%:%
%:%2319=763%:%
%:%2320=763%:%
%:%2321=764%:%
%:%2322=764%:%
%:%2323=764%:%
%:%2324=765%:%
%:%2325=765%:%
%:%2326=766%:%
%:%2327=766%:%
%:%2328=766%:%
%:%2329=766%:%
%:%2330=766%:%
%:%2331=767%:%
%:%2337=767%:%
%:%2340=768%:%
%:%2341=769%:%
%:%2342=769%:%
%:%2345=770%:%
%:%2349=770%:%
%:%2350=770%:%
%:%2351=770%:%
%:%2356=770%:%
%:%2359=771%:%
%:%2360=772%:%
%:%2361=772%:%
%:%2362=773%:%
%:%2369=774%:%
%:%2370=774%:%
%:%2371=775%:%
%:%2372=775%:%
%:%2373=776%:%
%:%2374=777%:%
%:%2375=777%:%
%:%2376=777%:%
%:%2377=777%:%
%:%2378=778%:%
%:%2379=778%:%
%:%2380=779%:%
%:%2381=780%:%
%:%2382=780%:%
%:%2383=780%:%
%:%2384=781%:%
%:%2385=781%:%
%:%2386=782%:%
%:%2387=782%:%
%:%2388=783%:%
%:%2389=783%:%
%:%2390=783%:%
%:%2391=784%:%
%:%2397=784%:%
%:%2400=785%:%
%:%2401=786%:%
%:%2402=786%:%
%:%2403=787%:%
%:%2406=788%:%
%:%2410=788%:%
%:%2411=788%:%
%:%2412=788%:%
%:%2417=788%:%
%:%2420=789%:%
%:%2421=790%:%
%:%2422=790%:%
%:%2423=791%:%
%:%2430=792%:%
%:%2431=792%:%
%:%2432=793%:%
%:%2433=793%:%
%:%2434=794%:%
%:%2435=794%:%
%:%2436=795%:%
%:%2437=796%:%
%:%2438=796%:%
%:%2439=797%:%
%:%2440=797%:%
%:%2441=797%:%
%:%2442=797%:%
%:%2443=797%:%
%:%2444=798%:%
%:%2450=798%:%
%:%2453=799%:%
%:%2454=800%:%
%:%2455=800%:%
%:%2458=801%:%
%:%2462=801%:%
%:%2463=801%:%
%:%2464=801%:%
%:%2470=801%:%
%:%2473=802%:%
%:%2474=803%:%
%:%2475=803%:%
%:%2478=804%:%
%:%2482=804%:%
%:%2483=804%:%
%:%2484=804%:%
%:%2490=804%:%
%:%2493=805%:%
%:%2494=806%:%
%:%2495=806%:%
%:%2498=807%:%
%:%2502=807%:%
%:%2503=807%:%
%:%2504=807%:%
%:%2510=807%:%
%:%2513=808%:%
%:%2514=809%:%
%:%2515=809%:%
%:%2518=810%:%
%:%2522=810%:%
%:%2523=810%:%
%:%2524=811%:%
%:%2525=811%:%
%:%2526=812%:%
%:%2527=813%:%
%:%2528=813%:%
%:%2533=813%:%
%:%2536=814%:%
%:%2537=815%:%
%:%2538=815%:%
%:%2539=816%:%
%:%2540=817%:%
%:%2541=818%:%
%:%2542=818%:%
%:%2543=819%:%
%:%2544=820%:%
%:%2545=821%:%
%:%2546=822%:%
%:%2547=822%:%
%:%2550=823%:%
%:%2554=823%:%
%:%2555=823%:%
%:%2556=823%:%
%:%2557=823%:%
%:%2562=823%:%
%:%2565=824%:%
%:%2566=825%:%
%:%2567=825%:%
%:%2568=826%:%
%:%2569=827%:%
%:%2570=828%:%
%:%2571=829%:%
%:%2572=829%:%
%:%2575=830%:%
%:%2579=830%:%
%:%2580=830%:%
%:%2581=830%:%
%:%2586=830%:%
%:%2589=831%:%
%:%2590=832%:%
%:%2591=832%:%
%:%2592=833%:%
%:%2594=835%:%
%:%2597=836%:%
%:%2601=836%:%
%:%2602=836%:%
%:%2603=836%:%
%:%2608=836%:%
%:%2611=837%:%
%:%2612=838%:%
%:%2613=838%:%
%:%2614=839%:%
%:%2615=840%:%
%:%2616=841%:%
%:%2617=841%:%
%:%2618=842%:%
%:%2619=843%:%
%:%2620=844%:%
%:%2621=845%:%
%:%2622=846%:%
%:%2623=846%:%
%:%2624=847%:%
%:%2627=848%:%
%:%2631=848%:%
%:%2632=848%:%
%:%2633=849%:%
%:%2634=849%:%
%:%2635=849%:%
%:%2640=849%:%
%:%2643=850%:%
%:%2644=851%:%
%:%2645=851%:%
%:%2646=852%:%
%:%2647=853%:%
%:%2648=854%:%
%:%2649=855%:%
%:%2652=856%:%
%:%2656=856%:%
%:%2657=856%:%
%:%2658=857%:%
%:%2659=857%:%
%:%2660=858%:%
%:%2665=858%:%
%:%2670=859%:%
%:%2675=860%:%
%:%2676=860%:%
%:%2681=860%:%
%:%2684=861%:%
%:%2685=862%:%
%:%2686=862%:%
%:%2687=863%:%
%:%2688=864%:%
%:%2689=865%:%
%:%2690=866%:%
%:%2691=866%:%
%:%2692=867%:%
%:%2693=868%:%
%:%2694=869%:%
%:%2695=870%:%
%:%2702=871%:%
%:%2703=871%:%
%:%2704=872%:%
%:%2705=872%:%
%:%2706=873%:%
%:%2707=874%:%
%:%2708=874%:%
%:%2709=874%:%
%:%2710=875%:%
%:%2711=875%:%
%:%2712=875%:%
%:%2713=876%:%
%:%2714=876%:%
%:%2715=876%:%
%:%2716=877%:%
%:%2717=877%:%
%:%2718=878%:%
%:%2719=878%:%
%:%2720=879%:%
%:%2721=880%:%
%:%2722=880%:%
%:%2723=881%:%
%:%2729=881%:%
%:%2732=882%:%
%:%2733=883%:%
%:%2734=884%:%
%:%2735=884%:%
%:%2736=885%:%
%:%2737=886%:%
%:%2738=887%:%
%:%2739=888%:%
%:%2740=889%:%
%:%2741=889%:%
%:%2742=890%:%
%:%2745=891%:%
%:%2749=891%:%
%:%2750=891%:%
%:%2751=891%:%
%:%2756=891%:%
%:%2759=892%:%
%:%2760=893%:%
%:%2761=893%:%
%:%2762=894%:%
%:%2763=895%:%
%:%2764=896%:%
%:%2765=897%:%
%:%2772=898%:%
%:%2773=898%:%
%:%2774=899%:%
%:%2775=899%:%
%:%2776=900%:%
%:%2778=902%:%
%:%2779=903%:%
%:%2780=903%:%
%:%2781=903%:%
%:%2782=904%:%
%:%2783=904%:%
%:%2784=905%:%
%:%2785=905%:%
%:%2786=906%:%
%:%2787=907%:%
%:%2788=907%:%
%:%2789=907%:%
%:%2790=908%:%
%:%2791=908%:%
%:%2792=909%:%
%:%2793=909%:%
%:%2794=909%:%
%:%2795=910%:%
%:%2796=910%:%
%:%2797=910%:%
%:%2798=911%:%
%:%2804=911%:%
%:%2807=912%:%
%:%2808=913%:%
%:%2809=914%:%
%:%2810=914%:%
%:%2811=915%:%
%:%2818=916%:%
%:%2819=916%:%
%:%2820=917%:%
%:%2821=917%:%
%:%2822=918%:%
%:%2823=918%:%
%:%2824=919%:%
%:%2825=920%:%
%:%2826=921%:%
%:%2827=922%:%
%:%2828=922%:%
%:%2829=923%:%
%:%2830=923%:%
%:%2831=924%:%
%:%2832=924%:%
%:%2833=925%:%
%:%2834=925%:%
%:%2835=925%:%
%:%2836=926%:%
%:%2837=926%:%
%:%2838=927%:%
%:%2839=927%:%
%:%2840=927%:%
%:%2841=928%:%
%:%2842=928%:%
%:%2843=929%:%
%:%2849=929%:%
%:%2852=930%:%
%:%2853=931%:%
%:%2854=931%:%
%:%2857=932%:%
%:%2861=932%:%
%:%2862=932%:%
%:%2863=932%:%
%:%2868=932%:%
%:%2871=933%:%
%:%2872=934%:%
%:%2873=935%:%
%:%2874=935%:%
%:%2877=936%:%
%:%2881=936%:%
%:%2882=936%:%
%:%2883=937%:%
%:%2884=937%:%
%:%2893=939%:%
%:%2895=940%:%
%:%2896=940%:%
%:%2897=941%:%
%:%2898=942%:%
%:%2899=943%:%
%:%2900=944%:%
%:%2901=944%:%
%:%2902=945%:%
%:%2909=946%:%
%:%2910=946%:%
%:%2911=947%:%
%:%2912=947%:%
%:%2913=948%:%
%:%2914=948%:%
%:%2915=949%:%
%:%2916=949%:%
%:%2917=950%:%
%:%2918=950%:%
%:%2919=950%:%
%:%2920=951%:%
%:%2921=951%:%
%:%2922=952%:%
%:%2923=952%:%
%:%2924=953%:%
%:%2925=953%:%
%:%2926=953%:%
%:%2927=954%:%
%:%2928=954%:%
%:%2929=955%:%
%:%2930=955%:%
%:%2931=955%:%
%:%2932=956%:%
%:%2933=956%:%
%:%2934=957%:%
%:%2935=957%:%
%:%2936=958%:%
%:%2937=958%:%
%:%2938=959%:%
%:%2939=960%:%
%:%2940=961%:%
%:%2941=962%:%
%:%2942=962%:%
%:%2943=963%:%
%:%2944=963%:%
%:%2945=964%:%
%:%2946=964%:%
%:%2947=965%:%
%:%2948=965%:%
%:%2949=966%:%
%:%2950=966%:%
%:%2951=966%:%
%:%2952=967%:%
%:%2953=967%:%
%:%2954=968%:%
%:%2955=968%:%
%:%2956=969%:%
%:%2957=969%:%
%:%2958=970%:%
%:%2959=970%:%
%:%2960=971%:%
%:%2961=971%:%
%:%2962=972%:%
%:%2968=972%:%
%:%2971=973%:%
%:%2972=974%:%
%:%2973=975%:%
%:%2974=975%:%
%:%2975=976%:%
%:%2976=977%:%
%:%2977=978%:%
%:%2984=979%:%
%:%2985=979%:%
%:%2986=980%:%
%:%2987=980%:%
%:%2988=981%:%
%:%2989=981%:%
%:%2990=982%:%
%:%2991=982%:%
%:%2992=982%:%
%:%2993=983%:%
%:%2994=983%:%
%:%2995=983%:%
%:%2996=984%:%
%:%2997=984%:%
%:%2998=985%:%
%:%2999=985%:%
%:%3000=985%:%
%:%3001=986%:%
%:%3002=987%:%
%:%3003=987%:%
%:%3004=987%:%
%:%3005=988%:%
%:%3006=988%:%
%:%3007=989%:%
%:%3008=989%:%
%:%3009=990%:%
%:%3010=990%:%
%:%3011=991%:%
%:%3012=991%:%
%:%3013=992%:%
%:%3014=992%:%
%:%3015=992%:%
%:%3016=992%:%
%:%3017=993%:%
%:%3018=993%:%
%:%3019=994%:%
%:%3020=994%:%
%:%3021=995%:%
%:%3022=995%:%
%:%3023=995%:%
%:%3024=996%:%
%:%3025=996%:%
%:%3026=997%:%
%:%3027=997%:%
%:%3028=998%:%
%:%3029=998%:%
%:%3030=998%:%
%:%3031=999%:%
%:%3037=999%:%
%:%3040=1000%:%
%:%3041=1001%:%
%:%3042=1002%:%
%:%3043=1002%:%
%:%3044=1003%:%
%:%3045=1004%:%
%:%3048=1005%:%
%:%3052=1005%:%
%:%3053=1005%:%
%:%3054=1006%:%
%:%3055=1006%:%
%:%3060=1006%:%
%:%3063=1007%:%
%:%3064=1008%:%
%:%3065=1009%:%
%:%3066=1009%:%
%:%3073=1010%:%
%:%3074=1010%:%
%:%3075=1011%:%
%:%3076=1011%:%
%:%3077=1012%:%
%:%3078=1012%:%
%:%3079=1013%:%
%:%3080=1013%:%
%:%3081=1013%:%
%:%3082=1013%:%
%:%3083=1013%:%
%:%3084=1014%:%
%:%3085=1014%:%
%:%3086=1015%:%
%:%3087=1015%:%
%:%3088=1016%:%
%:%3089=1016%:%
%:%3090=1017%:%
%:%3091=1017%:%
%:%3092=1017%:%
%:%3093=1018%:%
%:%3094=1018%:%
%:%3095=1019%:%
%:%3096=1019%:%
%:%3097=1019%:%
%:%3098=1019%:%
%:%3099=1020%:%
%:%3100=1020%:%
%:%3101=1021%:%
%:%3102=1021%:%
%:%3103=1022%:%
%:%3104=1022%:%
%:%3105=1022%:%
%:%3106=1023%:%
%:%3107=1023%:%
%:%3108=1024%:%
%:%3109=1024%:%
%:%3110=1024%:%
%:%3111=1025%:%
%:%3112=1025%:%
%:%3113=1025%:%
%:%3114=1026%:%
%:%3115=1026%:%
%:%3116=1027%:%
%:%3117=1027%:%
%:%3118=1028%:%
%:%3119=1028%:%
%:%3120=1028%:%
%:%3121=1029%:%
%:%3122=1029%:%
%:%3123=1030%:%
%:%3124=1030%:%
%:%3125=1030%:%
%:%3126=1031%:%
%:%3132=1031%:%
%:%3135=1032%:%
%:%3136=1033%:%
%:%3137=1033%:%
%:%3138=1034%:%
%:%3139=1035%:%
%:%3140=1035%:%
%:%3141=1036%:%
%:%3142=1037%:%
%:%3143=1038%:%
%:%3144=1039%:%
%:%3145=1040%:%
%:%3146=1040%:%
%:%3147=1041%:%
%:%3154=1042%:%
%:%3155=1042%:%
%:%3156=1043%:%
%:%3157=1043%:%
%:%3158=1044%:%
%:%3159=1044%:%
%:%3160=1044%:%
%:%3161=1045%:%
%:%3162=1045%:%
%:%3163=1046%:%
%:%3164=1046%:%
%:%3165=1047%:%
%:%3166=1047%:%
%:%3167=1047%:%
%:%3168=1048%:%
%:%3169=1048%:%
%:%3170=1048%:%
%:%3171=1048%:%
%:%3172=1049%:%
%:%3178=1049%:%
%:%3181=1050%:%
%:%3182=1051%:%
%:%3183=1051%:%
%:%3190=1052%:%
%:%3191=1052%:%
%:%3192=1053%:%
%:%3193=1053%:%
%:%3194=1053%:%
%:%3195=1054%:%
%:%3196=1054%:%
%:%3197=1055%:%
%:%3198=1055%:%
%:%3199=1056%:%
%:%3200=1056%:%
%:%3201=1056%:%
%:%3202=1056%:%
%:%3203=1057%:%
%:%3209=1057%:%
%:%3212=1058%:%
%:%3213=1059%:%
%:%3214=1059%:%
%:%3221=1060%:%

%
\begin{isabellebody}%
\setisabellecontext{FrecR}%
%
\isadelimdocument
%
\endisadelimdocument
%
\isatagdocument
%
\isamarkupsection{Well-founded relation on names%
}
\isamarkuptrue%
%
\endisatagdocument
{\isafolddocument}%
%
\isadelimdocument
%
\endisadelimdocument
%
\isadelimtheory
%
\endisadelimtheory
%
\isatagtheory
\isacommand{theory}\isamarkupfalse%
\ FrecR\ \isakeyword{imports}\ Names\ Synthetic{\isacharunderscore}{\kern0pt}Definition\ \isakeyword{begin}%
\endisatagtheory
{\isafoldtheory}%
%
\isadelimtheory
\isanewline
%
\endisadelimtheory
\isanewline
\isacommand{lemmas}\isamarkupfalse%
\ sep{\isacharunderscore}{\kern0pt}rules{\isacharprime}{\kern0pt}\ {\isacharequal}{\kern0pt}\ nth{\isacharunderscore}{\kern0pt}{\isadigit{0}}\ nth{\isacharunderscore}{\kern0pt}ConsI\ FOL{\isacharunderscore}{\kern0pt}iff{\isacharunderscore}{\kern0pt}sats\ function{\isacharunderscore}{\kern0pt}iff{\isacharunderscore}{\kern0pt}sats\isanewline
\ \ fun{\isacharunderscore}{\kern0pt}plus{\isacharunderscore}{\kern0pt}iff{\isacharunderscore}{\kern0pt}sats\ omega{\isacharunderscore}{\kern0pt}iff{\isacharunderscore}{\kern0pt}sats\ FOL{\isacharunderscore}{\kern0pt}sats{\isacharunderscore}{\kern0pt}iff%
\begin{isamarkuptext}%
\isa{frecR} is the well-founded relation on names that allows
us to define forcing for atomic formulas.%
\end{isamarkuptext}\isamarkuptrue%
\isacommand{definition}\isamarkupfalse%
\isanewline
\ \ is{\isacharunderscore}{\kern0pt}hcomp\ {\isacharcolon}{\kern0pt}{\isacharcolon}{\kern0pt}\ {\isachardoublequoteopen}{\isacharbrackleft}{\kern0pt}i{\isasymRightarrow}o{\isacharcomma}{\kern0pt}i{\isasymRightarrow}i{\isasymRightarrow}o{\isacharcomma}{\kern0pt}i{\isasymRightarrow}i{\isasymRightarrow}o{\isacharcomma}{\kern0pt}i{\isacharcomma}{\kern0pt}i{\isacharbrackright}{\kern0pt}\ {\isasymRightarrow}\ o{\isachardoublequoteclose}\ \isakeyword{where}\isanewline
\ \ {\isachardoublequoteopen}is{\isacharunderscore}{\kern0pt}hcomp{\isacharparenleft}{\kern0pt}M{\isacharcomma}{\kern0pt}is{\isacharunderscore}{\kern0pt}f{\isacharcomma}{\kern0pt}is{\isacharunderscore}{\kern0pt}g{\isacharcomma}{\kern0pt}a{\isacharcomma}{\kern0pt}w{\isacharparenright}{\kern0pt}\ {\isasymequiv}\ {\isasymexists}z{\isacharbrackleft}{\kern0pt}M{\isacharbrackright}{\kern0pt}{\isachardot}{\kern0pt}\ is{\isacharunderscore}{\kern0pt}g{\isacharparenleft}{\kern0pt}a{\isacharcomma}{\kern0pt}z{\isacharparenright}{\kern0pt}\ {\isasymand}\ is{\isacharunderscore}{\kern0pt}f{\isacharparenleft}{\kern0pt}z{\isacharcomma}{\kern0pt}w{\isacharparenright}{\kern0pt}{\isachardoublequoteclose}\ \isanewline
\isanewline
\isacommand{lemma}\isamarkupfalse%
\ {\isacharparenleft}{\kern0pt}\isakeyword{in}\ M{\isacharunderscore}{\kern0pt}trivial{\isacharparenright}{\kern0pt}\ hcomp{\isacharunderscore}{\kern0pt}abs{\isacharcolon}{\kern0pt}\ \isanewline
\ \ \isakeyword{assumes}\isanewline
\ \ \ \ is{\isacharunderscore}{\kern0pt}f{\isacharunderscore}{\kern0pt}abs{\isacharcolon}{\kern0pt}{\isachardoublequoteopen}{\isasymAnd}a\ z{\isachardot}{\kern0pt}\ M{\isacharparenleft}{\kern0pt}a{\isacharparenright}{\kern0pt}\ {\isasymLongrightarrow}\ M{\isacharparenleft}{\kern0pt}z{\isacharparenright}{\kern0pt}\ {\isasymLongrightarrow}\ is{\isacharunderscore}{\kern0pt}f{\isacharparenleft}{\kern0pt}a{\isacharcomma}{\kern0pt}z{\isacharparenright}{\kern0pt}\ {\isasymlongleftrightarrow}\ z\ {\isacharequal}{\kern0pt}\ f{\isacharparenleft}{\kern0pt}a{\isacharparenright}{\kern0pt}{\isachardoublequoteclose}\ \isakeyword{and}\isanewline
\ \ \ \ is{\isacharunderscore}{\kern0pt}g{\isacharunderscore}{\kern0pt}abs{\isacharcolon}{\kern0pt}{\isachardoublequoteopen}{\isasymAnd}a\ z{\isachardot}{\kern0pt}\ M{\isacharparenleft}{\kern0pt}a{\isacharparenright}{\kern0pt}\ {\isasymLongrightarrow}\ M{\isacharparenleft}{\kern0pt}z{\isacharparenright}{\kern0pt}\ {\isasymLongrightarrow}\ is{\isacharunderscore}{\kern0pt}g{\isacharparenleft}{\kern0pt}a{\isacharcomma}{\kern0pt}z{\isacharparenright}{\kern0pt}\ {\isasymlongleftrightarrow}\ z\ {\isacharequal}{\kern0pt}\ g{\isacharparenleft}{\kern0pt}a{\isacharparenright}{\kern0pt}{\isachardoublequoteclose}\ \isakeyword{and}\isanewline
\ \ \ \ g{\isacharunderscore}{\kern0pt}closed{\isacharcolon}{\kern0pt}{\isachardoublequoteopen}{\isasymAnd}a{\isachardot}{\kern0pt}\ M{\isacharparenleft}{\kern0pt}a{\isacharparenright}{\kern0pt}\ {\isasymLongrightarrow}\ M{\isacharparenleft}{\kern0pt}g{\isacharparenleft}{\kern0pt}a{\isacharparenright}{\kern0pt}{\isacharparenright}{\kern0pt}{\isachardoublequoteclose}\ \isanewline
\ \ \ \ {\isachardoublequoteopen}M{\isacharparenleft}{\kern0pt}a{\isacharparenright}{\kern0pt}{\isachardoublequoteclose}\ {\isachardoublequoteopen}M{\isacharparenleft}{\kern0pt}w{\isacharparenright}{\kern0pt}{\isachardoublequoteclose}\ \isanewline
\ \ \isakeyword{shows}\isanewline
\ \ \ \ {\isachardoublequoteopen}is{\isacharunderscore}{\kern0pt}hcomp{\isacharparenleft}{\kern0pt}M{\isacharcomma}{\kern0pt}is{\isacharunderscore}{\kern0pt}f{\isacharcomma}{\kern0pt}is{\isacharunderscore}{\kern0pt}g{\isacharcomma}{\kern0pt}a{\isacharcomma}{\kern0pt}w{\isacharparenright}{\kern0pt}\ {\isasymlongleftrightarrow}\ w\ {\isacharequal}{\kern0pt}\ f{\isacharparenleft}{\kern0pt}g{\isacharparenleft}{\kern0pt}a{\isacharparenright}{\kern0pt}{\isacharparenright}{\kern0pt}{\isachardoublequoteclose}\ \isanewline
%
\isadelimproof
\ \ %
\endisadelimproof
%
\isatagproof
\isacommand{unfolding}\isamarkupfalse%
\ is{\isacharunderscore}{\kern0pt}hcomp{\isacharunderscore}{\kern0pt}def\ \isacommand{using}\isamarkupfalse%
\ assms\ \isacommand{by}\isamarkupfalse%
\ simp%
\endisatagproof
{\isafoldproof}%
%
\isadelimproof
\isanewline
%
\endisadelimproof
\isanewline
\isacommand{definition}\isamarkupfalse%
\isanewline
\ \ hcomp{\isacharunderscore}{\kern0pt}fm\ {\isacharcolon}{\kern0pt}{\isacharcolon}{\kern0pt}\ {\isachardoublequoteopen}{\isacharbrackleft}{\kern0pt}i{\isasymRightarrow}i{\isasymRightarrow}i{\isacharcomma}{\kern0pt}i{\isasymRightarrow}i{\isasymRightarrow}i{\isacharcomma}{\kern0pt}i{\isacharcomma}{\kern0pt}i{\isacharbrackright}{\kern0pt}\ {\isasymRightarrow}\ i{\isachardoublequoteclose}\ \isakeyword{where}\isanewline
\ \ {\isachardoublequoteopen}hcomp{\isacharunderscore}{\kern0pt}fm{\isacharparenleft}{\kern0pt}pf{\isacharcomma}{\kern0pt}pg{\isacharcomma}{\kern0pt}a{\isacharcomma}{\kern0pt}w{\isacharparenright}{\kern0pt}\ {\isasymequiv}\ Exists{\isacharparenleft}{\kern0pt}And{\isacharparenleft}{\kern0pt}pg{\isacharparenleft}{\kern0pt}succ{\isacharparenleft}{\kern0pt}a{\isacharparenright}{\kern0pt}{\isacharcomma}{\kern0pt}{\isadigit{0}}{\isacharparenright}{\kern0pt}{\isacharcomma}{\kern0pt}pf{\isacharparenleft}{\kern0pt}{\isadigit{0}}{\isacharcomma}{\kern0pt}succ{\isacharparenleft}{\kern0pt}w{\isacharparenright}{\kern0pt}{\isacharparenright}{\kern0pt}{\isacharparenright}{\kern0pt}{\isacharparenright}{\kern0pt}{\isachardoublequoteclose}\isanewline
\isanewline
\isacommand{lemma}\isamarkupfalse%
\ sats{\isacharunderscore}{\kern0pt}hcomp{\isacharunderscore}{\kern0pt}fm{\isacharcolon}{\kern0pt}\isanewline
\ \ \isakeyword{assumes}\ \isanewline
\ \ \ \ f{\isacharunderscore}{\kern0pt}iff{\isacharunderscore}{\kern0pt}sats{\isacharcolon}{\kern0pt}{\isachardoublequoteopen}{\isasymAnd}a\ b\ z{\isachardot}{\kern0pt}\ a{\isasymin}nat\ {\isasymLongrightarrow}\ b{\isasymin}nat\ {\isasymLongrightarrow}\ z{\isasymin}M\ {\isasymLongrightarrow}\ \isanewline
\ \ \ \ \ \ \ \ \ \ \ \ \ \ \ \ \ is{\isacharunderscore}{\kern0pt}f{\isacharparenleft}{\kern0pt}nth{\isacharparenleft}{\kern0pt}a{\isacharcomma}{\kern0pt}Cons{\isacharparenleft}{\kern0pt}z{\isacharcomma}{\kern0pt}env{\isacharparenright}{\kern0pt}{\isacharparenright}{\kern0pt}{\isacharcomma}{\kern0pt}nth{\isacharparenleft}{\kern0pt}b{\isacharcomma}{\kern0pt}Cons{\isacharparenleft}{\kern0pt}z{\isacharcomma}{\kern0pt}env{\isacharparenright}{\kern0pt}{\isacharparenright}{\kern0pt}{\isacharparenright}{\kern0pt}\ {\isasymlongleftrightarrow}\ sats{\isacharparenleft}{\kern0pt}M{\isacharcomma}{\kern0pt}pf{\isacharparenleft}{\kern0pt}a{\isacharcomma}{\kern0pt}b{\isacharparenright}{\kern0pt}{\isacharcomma}{\kern0pt}Cons{\isacharparenleft}{\kern0pt}z{\isacharcomma}{\kern0pt}env{\isacharparenright}{\kern0pt}{\isacharparenright}{\kern0pt}{\isachardoublequoteclose}\isanewline
\ \ \ \ \isakeyword{and}\isanewline
\ \ \ \ g{\isacharunderscore}{\kern0pt}iff{\isacharunderscore}{\kern0pt}sats{\isacharcolon}{\kern0pt}{\isachardoublequoteopen}{\isasymAnd}a\ b\ z{\isachardot}{\kern0pt}\ a{\isasymin}nat\ {\isasymLongrightarrow}\ b{\isasymin}nat\ {\isasymLongrightarrow}\ z{\isasymin}M\ {\isasymLongrightarrow}\ \isanewline
\ \ \ \ \ \ \ \ \ \ \ \ \ \ \ \ is{\isacharunderscore}{\kern0pt}g{\isacharparenleft}{\kern0pt}nth{\isacharparenleft}{\kern0pt}a{\isacharcomma}{\kern0pt}Cons{\isacharparenleft}{\kern0pt}z{\isacharcomma}{\kern0pt}env{\isacharparenright}{\kern0pt}{\isacharparenright}{\kern0pt}{\isacharcomma}{\kern0pt}nth{\isacharparenleft}{\kern0pt}b{\isacharcomma}{\kern0pt}Cons{\isacharparenleft}{\kern0pt}z{\isacharcomma}{\kern0pt}env{\isacharparenright}{\kern0pt}{\isacharparenright}{\kern0pt}{\isacharparenright}{\kern0pt}\ {\isasymlongleftrightarrow}\ sats{\isacharparenleft}{\kern0pt}M{\isacharcomma}{\kern0pt}pg{\isacharparenleft}{\kern0pt}a{\isacharcomma}{\kern0pt}b{\isacharparenright}{\kern0pt}{\isacharcomma}{\kern0pt}Cons{\isacharparenleft}{\kern0pt}z{\isacharcomma}{\kern0pt}env{\isacharparenright}{\kern0pt}{\isacharparenright}{\kern0pt}{\isachardoublequoteclose}\isanewline
\ \ \ \ \isakeyword{and}\isanewline
\ \ \ \ {\isachardoublequoteopen}a{\isasymin}nat{\isachardoublequoteclose}\ {\isachardoublequoteopen}w{\isasymin}nat{\isachardoublequoteclose}\ {\isachardoublequoteopen}env{\isasymin}list{\isacharparenleft}{\kern0pt}M{\isacharparenright}{\kern0pt}{\isachardoublequoteclose}\ \isanewline
\ \ \isakeyword{shows}\isanewline
\ \ \ \ {\isachardoublequoteopen}sats{\isacharparenleft}{\kern0pt}M{\isacharcomma}{\kern0pt}hcomp{\isacharunderscore}{\kern0pt}fm{\isacharparenleft}{\kern0pt}pf{\isacharcomma}{\kern0pt}pg{\isacharcomma}{\kern0pt}a{\isacharcomma}{\kern0pt}w{\isacharparenright}{\kern0pt}{\isacharcomma}{\kern0pt}env{\isacharparenright}{\kern0pt}\ {\isasymlongleftrightarrow}\ is{\isacharunderscore}{\kern0pt}hcomp{\isacharparenleft}{\kern0pt}{\isacharhash}{\kern0pt}{\isacharhash}{\kern0pt}M{\isacharcomma}{\kern0pt}is{\isacharunderscore}{\kern0pt}f{\isacharcomma}{\kern0pt}is{\isacharunderscore}{\kern0pt}g{\isacharcomma}{\kern0pt}nth{\isacharparenleft}{\kern0pt}a{\isacharcomma}{\kern0pt}env{\isacharparenright}{\kern0pt}{\isacharcomma}{\kern0pt}nth{\isacharparenleft}{\kern0pt}w{\isacharcomma}{\kern0pt}env{\isacharparenright}{\kern0pt}{\isacharparenright}{\kern0pt}{\isachardoublequoteclose}\ \isanewline
%
\isadelimproof
%
\endisadelimproof
%
\isatagproof
\isacommand{proof}\isamarkupfalse%
\ {\isacharminus}{\kern0pt}\isanewline
\ \ \isacommand{have}\isamarkupfalse%
\ {\isachardoublequoteopen}sats{\isacharparenleft}{\kern0pt}M{\isacharcomma}{\kern0pt}\ pf{\isacharparenleft}{\kern0pt}{\isadigit{0}}{\isacharcomma}{\kern0pt}\ succ{\isacharparenleft}{\kern0pt}w{\isacharparenright}{\kern0pt}{\isacharparenright}{\kern0pt}{\isacharcomma}{\kern0pt}\ Cons{\isacharparenleft}{\kern0pt}x{\isacharcomma}{\kern0pt}\ env{\isacharparenright}{\kern0pt}{\isacharparenright}{\kern0pt}\ {\isasymlongleftrightarrow}\ is{\isacharunderscore}{\kern0pt}f{\isacharparenleft}{\kern0pt}x{\isacharcomma}{\kern0pt}nth{\isacharparenleft}{\kern0pt}w{\isacharcomma}{\kern0pt}env{\isacharparenright}{\kern0pt}{\isacharparenright}{\kern0pt}{\isachardoublequoteclose}\ \isakeyword{if}\ {\isachardoublequoteopen}x{\isasymin}M{\isachardoublequoteclose}\ {\isachardoublequoteopen}w{\isasymin}nat{\isachardoublequoteclose}\ \isakeyword{for}\ x\ w\isanewline
\ \ \ \ \isacommand{using}\isamarkupfalse%
\ f{\isacharunderscore}{\kern0pt}iff{\isacharunderscore}{\kern0pt}sats{\isacharbrackleft}{\kern0pt}of\ {\isadigit{0}}\ {\isachardoublequoteopen}succ{\isacharparenleft}{\kern0pt}w{\isacharparenright}{\kern0pt}{\isachardoublequoteclose}\ x{\isacharbrackright}{\kern0pt}\ that\ \isacommand{by}\isamarkupfalse%
\ simp\isanewline
\ \ \isacommand{moreover}\isamarkupfalse%
\isanewline
\ \ \isacommand{have}\isamarkupfalse%
\ {\isachardoublequoteopen}sats{\isacharparenleft}{\kern0pt}M{\isacharcomma}{\kern0pt}\ pg{\isacharparenleft}{\kern0pt}succ{\isacharparenleft}{\kern0pt}a{\isacharparenright}{\kern0pt}{\isacharcomma}{\kern0pt}\ {\isadigit{0}}{\isacharparenright}{\kern0pt}{\isacharcomma}{\kern0pt}\ Cons{\isacharparenleft}{\kern0pt}x{\isacharcomma}{\kern0pt}\ env{\isacharparenright}{\kern0pt}{\isacharparenright}{\kern0pt}\ {\isasymlongleftrightarrow}\ is{\isacharunderscore}{\kern0pt}g{\isacharparenleft}{\kern0pt}nth{\isacharparenleft}{\kern0pt}a{\isacharcomma}{\kern0pt}env{\isacharparenright}{\kern0pt}{\isacharcomma}{\kern0pt}x{\isacharparenright}{\kern0pt}{\isachardoublequoteclose}\ \isakeyword{if}\ {\isachardoublequoteopen}x{\isasymin}M{\isachardoublequoteclose}\ {\isachardoublequoteopen}a{\isasymin}nat{\isachardoublequoteclose}\ \isakeyword{for}\ x\ a\isanewline
\ \ \ \ \isacommand{using}\isamarkupfalse%
\ g{\isacharunderscore}{\kern0pt}iff{\isacharunderscore}{\kern0pt}sats{\isacharbrackleft}{\kern0pt}of\ {\isachardoublequoteopen}succ{\isacharparenleft}{\kern0pt}a{\isacharparenright}{\kern0pt}{\isachardoublequoteclose}\ {\isadigit{0}}\ x{\isacharbrackright}{\kern0pt}\ that\ \isacommand{by}\isamarkupfalse%
\ simp\isanewline
\ \ \isacommand{ultimately}\isamarkupfalse%
\isanewline
\ \ \isacommand{show}\isamarkupfalse%
\ {\isacharquery}{\kern0pt}thesis\ \isacommand{unfolding}\isamarkupfalse%
\ hcomp{\isacharunderscore}{\kern0pt}fm{\isacharunderscore}{\kern0pt}def\ is{\isacharunderscore}{\kern0pt}hcomp{\isacharunderscore}{\kern0pt}def\ \isacommand{using}\isamarkupfalse%
\ assms\ \isacommand{by}\isamarkupfalse%
\ simp\isanewline
\isacommand{qed}\isamarkupfalse%
%
\endisatagproof
{\isafoldproof}%
%
\isadelimproof
\isanewline
%
\endisadelimproof
\isanewline
\isanewline
\isanewline
\isacommand{definition}\isamarkupfalse%
\isanewline
\ \ ftype\ {\isacharcolon}{\kern0pt}{\isacharcolon}{\kern0pt}\ {\isachardoublequoteopen}i{\isasymRightarrow}i{\isachardoublequoteclose}\ \isakeyword{where}\isanewline
\ \ {\isachardoublequoteopen}ftype\ {\isasymequiv}\ fst{\isachardoublequoteclose}\isanewline
\isanewline
\isacommand{definition}\isamarkupfalse%
\isanewline
\ \ name{\isadigit{1}}\ {\isacharcolon}{\kern0pt}{\isacharcolon}{\kern0pt}\ {\isachardoublequoteopen}i{\isasymRightarrow}i{\isachardoublequoteclose}\ \isakeyword{where}\isanewline
\ \ {\isachardoublequoteopen}name{\isadigit{1}}{\isacharparenleft}{\kern0pt}x{\isacharparenright}{\kern0pt}\ {\isasymequiv}\ fst{\isacharparenleft}{\kern0pt}snd{\isacharparenleft}{\kern0pt}x{\isacharparenright}{\kern0pt}{\isacharparenright}{\kern0pt}{\isachardoublequoteclose}\isanewline
\isanewline
\isacommand{definition}\isamarkupfalse%
\isanewline
\ \ name{\isadigit{2}}\ {\isacharcolon}{\kern0pt}{\isacharcolon}{\kern0pt}\ {\isachardoublequoteopen}i{\isasymRightarrow}i{\isachardoublequoteclose}\ \isakeyword{where}\isanewline
\ \ {\isachardoublequoteopen}name{\isadigit{2}}{\isacharparenleft}{\kern0pt}x{\isacharparenright}{\kern0pt}\ {\isasymequiv}\ fst{\isacharparenleft}{\kern0pt}snd{\isacharparenleft}{\kern0pt}snd{\isacharparenleft}{\kern0pt}x{\isacharparenright}{\kern0pt}{\isacharparenright}{\kern0pt}{\isacharparenright}{\kern0pt}{\isachardoublequoteclose}\isanewline
\isanewline
\isacommand{definition}\isamarkupfalse%
\isanewline
\ \ cond{\isacharunderscore}{\kern0pt}of\ {\isacharcolon}{\kern0pt}{\isacharcolon}{\kern0pt}\ {\isachardoublequoteopen}i{\isasymRightarrow}i{\isachardoublequoteclose}\ \isakeyword{where}\isanewline
\ \ {\isachardoublequoteopen}cond{\isacharunderscore}{\kern0pt}of{\isacharparenleft}{\kern0pt}x{\isacharparenright}{\kern0pt}\ {\isasymequiv}\ snd{\isacharparenleft}{\kern0pt}snd{\isacharparenleft}{\kern0pt}snd{\isacharparenleft}{\kern0pt}{\isacharparenleft}{\kern0pt}x{\isacharparenright}{\kern0pt}{\isacharparenright}{\kern0pt}{\isacharparenright}{\kern0pt}{\isacharparenright}{\kern0pt}{\isachardoublequoteclose}\isanewline
\isanewline
\isacommand{lemma}\isamarkupfalse%
\ components{\isacharunderscore}{\kern0pt}simp{\isacharcolon}{\kern0pt}\isanewline
\ \ {\isachardoublequoteopen}ftype{\isacharparenleft}{\kern0pt}{\isasymlangle}f{\isacharcomma}{\kern0pt}n{\isadigit{1}}{\isacharcomma}{\kern0pt}n{\isadigit{2}}{\isacharcomma}{\kern0pt}c{\isasymrangle}{\isacharparenright}{\kern0pt}\ {\isacharequal}{\kern0pt}\ f{\isachardoublequoteclose}\isanewline
\ \ {\isachardoublequoteopen}name{\isadigit{1}}{\isacharparenleft}{\kern0pt}{\isasymlangle}f{\isacharcomma}{\kern0pt}n{\isadigit{1}}{\isacharcomma}{\kern0pt}n{\isadigit{2}}{\isacharcomma}{\kern0pt}c{\isasymrangle}{\isacharparenright}{\kern0pt}\ {\isacharequal}{\kern0pt}\ n{\isadigit{1}}{\isachardoublequoteclose}\isanewline
\ \ {\isachardoublequoteopen}name{\isadigit{2}}{\isacharparenleft}{\kern0pt}{\isasymlangle}f{\isacharcomma}{\kern0pt}n{\isadigit{1}}{\isacharcomma}{\kern0pt}n{\isadigit{2}}{\isacharcomma}{\kern0pt}c{\isasymrangle}{\isacharparenright}{\kern0pt}\ {\isacharequal}{\kern0pt}\ n{\isadigit{2}}{\isachardoublequoteclose}\isanewline
\ \ {\isachardoublequoteopen}cond{\isacharunderscore}{\kern0pt}of{\isacharparenleft}{\kern0pt}{\isasymlangle}f{\isacharcomma}{\kern0pt}n{\isadigit{1}}{\isacharcomma}{\kern0pt}n{\isadigit{2}}{\isacharcomma}{\kern0pt}c{\isasymrangle}{\isacharparenright}{\kern0pt}\ {\isacharequal}{\kern0pt}\ c{\isachardoublequoteclose}\isanewline
%
\isadelimproof
\ \ %
\endisadelimproof
%
\isatagproof
\isacommand{unfolding}\isamarkupfalse%
\ ftype{\isacharunderscore}{\kern0pt}def\ name{\isadigit{1}}{\isacharunderscore}{\kern0pt}def\ name{\isadigit{2}}{\isacharunderscore}{\kern0pt}def\ cond{\isacharunderscore}{\kern0pt}of{\isacharunderscore}{\kern0pt}def\isanewline
\ \ \isacommand{by}\isamarkupfalse%
\ simp{\isacharunderscore}{\kern0pt}all%
\endisatagproof
{\isafoldproof}%
%
\isadelimproof
\isanewline
%
\endisadelimproof
\isanewline
\isacommand{definition}\isamarkupfalse%
\ eclose{\isacharunderscore}{\kern0pt}n\ {\isacharcolon}{\kern0pt}{\isacharcolon}{\kern0pt}\ {\isachardoublequoteopen}{\isacharbrackleft}{\kern0pt}i{\isasymRightarrow}i{\isacharcomma}{\kern0pt}i{\isacharbrackright}{\kern0pt}\ {\isasymRightarrow}\ i{\isachardoublequoteclose}\ \isakeyword{where}\isanewline
\ \ {\isachardoublequoteopen}eclose{\isacharunderscore}{\kern0pt}n{\isacharparenleft}{\kern0pt}name{\isacharcomma}{\kern0pt}x{\isacharparenright}{\kern0pt}\ {\isacharequal}{\kern0pt}\ eclose{\isacharparenleft}{\kern0pt}{\isacharbraceleft}{\kern0pt}name{\isacharparenleft}{\kern0pt}x{\isacharparenright}{\kern0pt}{\isacharbraceright}{\kern0pt}{\isacharparenright}{\kern0pt}{\isachardoublequoteclose}\isanewline
\isanewline
\isacommand{definition}\isamarkupfalse%
\isanewline
\ \ ecloseN\ {\isacharcolon}{\kern0pt}{\isacharcolon}{\kern0pt}\ {\isachardoublequoteopen}i\ {\isasymRightarrow}\ i{\isachardoublequoteclose}\ \isakeyword{where}\isanewline
\ \ {\isachardoublequoteopen}ecloseN{\isacharparenleft}{\kern0pt}x{\isacharparenright}{\kern0pt}\ {\isacharequal}{\kern0pt}\ eclose{\isacharunderscore}{\kern0pt}n{\isacharparenleft}{\kern0pt}name{\isadigit{1}}{\isacharcomma}{\kern0pt}x{\isacharparenright}{\kern0pt}\ {\isasymunion}\ eclose{\isacharunderscore}{\kern0pt}n{\isacharparenleft}{\kern0pt}name{\isadigit{2}}{\isacharcomma}{\kern0pt}x{\isacharparenright}{\kern0pt}{\isachardoublequoteclose}\isanewline
\isanewline
\isacommand{lemma}\isamarkupfalse%
\ components{\isacharunderscore}{\kern0pt}in{\isacharunderscore}{\kern0pt}eclose\ {\isacharcolon}{\kern0pt}\isanewline
\ \ {\isachardoublequoteopen}n{\isadigit{1}}\ {\isasymin}\ ecloseN{\isacharparenleft}{\kern0pt}{\isasymlangle}f{\isacharcomma}{\kern0pt}n{\isadigit{1}}{\isacharcomma}{\kern0pt}n{\isadigit{2}}{\isacharcomma}{\kern0pt}c{\isasymrangle}{\isacharparenright}{\kern0pt}{\isachardoublequoteclose}\isanewline
\ \ {\isachardoublequoteopen}n{\isadigit{2}}\ {\isasymin}\ ecloseN{\isacharparenleft}{\kern0pt}{\isasymlangle}f{\isacharcomma}{\kern0pt}n{\isadigit{1}}{\isacharcomma}{\kern0pt}n{\isadigit{2}}{\isacharcomma}{\kern0pt}c{\isasymrangle}{\isacharparenright}{\kern0pt}{\isachardoublequoteclose}\isanewline
%
\isadelimproof
\ \ %
\endisadelimproof
%
\isatagproof
\isacommand{unfolding}\isamarkupfalse%
\ ecloseN{\isacharunderscore}{\kern0pt}def\ eclose{\isacharunderscore}{\kern0pt}n{\isacharunderscore}{\kern0pt}def\isanewline
\ \ \isacommand{using}\isamarkupfalse%
\ components{\isacharunderscore}{\kern0pt}simp\ arg{\isacharunderscore}{\kern0pt}into{\isacharunderscore}{\kern0pt}eclose\ \isacommand{by}\isamarkupfalse%
\ auto%
\endisatagproof
{\isafoldproof}%
%
\isadelimproof
\isanewline
%
\endisadelimproof
\isanewline
\isacommand{lemmas}\isamarkupfalse%
\ names{\isacharunderscore}{\kern0pt}simp\ {\isacharequal}{\kern0pt}\ components{\isacharunderscore}{\kern0pt}simp{\isacharparenleft}{\kern0pt}{\isadigit{2}}{\isacharparenright}{\kern0pt}\ components{\isacharunderscore}{\kern0pt}simp{\isacharparenleft}{\kern0pt}{\isadigit{3}}{\isacharparenright}{\kern0pt}\isanewline
\isanewline
\isacommand{lemma}\isamarkupfalse%
\ ecloseNI{\isadigit{1}}\ {\isacharcolon}{\kern0pt}\ \isanewline
\ \ \isakeyword{assumes}\ {\isachardoublequoteopen}x\ {\isasymin}\ eclose{\isacharparenleft}{\kern0pt}n{\isadigit{1}}{\isacharparenright}{\kern0pt}\ {\isasymor}\ x{\isasymin}eclose{\isacharparenleft}{\kern0pt}n{\isadigit{2}}{\isacharparenright}{\kern0pt}{\isachardoublequoteclose}\ \isanewline
\ \ \isakeyword{shows}\ {\isachardoublequoteopen}x\ {\isasymin}\ ecloseN{\isacharparenleft}{\kern0pt}{\isasymlangle}f{\isacharcomma}{\kern0pt}n{\isadigit{1}}{\isacharcomma}{\kern0pt}n{\isadigit{2}}{\isacharcomma}{\kern0pt}c{\isasymrangle}{\isacharparenright}{\kern0pt}{\isachardoublequoteclose}\ \isanewline
%
\isadelimproof
\ \ %
\endisadelimproof
%
\isatagproof
\isacommand{unfolding}\isamarkupfalse%
\ ecloseN{\isacharunderscore}{\kern0pt}def\ eclose{\isacharunderscore}{\kern0pt}n{\isacharunderscore}{\kern0pt}def\isanewline
\ \ \isacommand{using}\isamarkupfalse%
\ assms\ eclose{\isacharunderscore}{\kern0pt}sing\ names{\isacharunderscore}{\kern0pt}simp\isanewline
\ \ \isacommand{by}\isamarkupfalse%
\ auto%
\endisatagproof
{\isafoldproof}%
%
\isadelimproof
\isanewline
%
\endisadelimproof
\isanewline
\isacommand{lemmas}\isamarkupfalse%
\ ecloseNI\ {\isacharequal}{\kern0pt}\ ecloseNI{\isadigit{1}}\isanewline
\isanewline
\isacommand{lemma}\isamarkupfalse%
\ ecloseN{\isacharunderscore}{\kern0pt}mono\ {\isacharcolon}{\kern0pt}\isanewline
\ \ \isakeyword{assumes}\ {\isachardoublequoteopen}u\ {\isasymin}\ ecloseN{\isacharparenleft}{\kern0pt}x{\isacharparenright}{\kern0pt}{\isachardoublequoteclose}\ {\isachardoublequoteopen}name{\isadigit{1}}{\isacharparenleft}{\kern0pt}x{\isacharparenright}{\kern0pt}\ {\isasymin}\ ecloseN{\isacharparenleft}{\kern0pt}y{\isacharparenright}{\kern0pt}{\isachardoublequoteclose}\ {\isachardoublequoteopen}name{\isadigit{2}}{\isacharparenleft}{\kern0pt}x{\isacharparenright}{\kern0pt}\ {\isasymin}\ ecloseN{\isacharparenleft}{\kern0pt}y{\isacharparenright}{\kern0pt}{\isachardoublequoteclose}\isanewline
\ \ \isakeyword{shows}\ {\isachardoublequoteopen}u\ {\isasymin}\ ecloseN{\isacharparenleft}{\kern0pt}y{\isacharparenright}{\kern0pt}{\isachardoublequoteclose}\isanewline
%
\isadelimproof
%
\endisadelimproof
%
\isatagproof
\isacommand{proof}\isamarkupfalse%
\ {\isacharminus}{\kern0pt}\isanewline
\ \ \isacommand{from}\isamarkupfalse%
\ {\isacartoucheopen}u{\isasymin}{\isacharunderscore}{\kern0pt}{\isacartoucheclose}\isanewline
\ \ \isacommand{consider}\isamarkupfalse%
\ {\isacharparenleft}{\kern0pt}a{\isacharparenright}{\kern0pt}\ {\isachardoublequoteopen}u{\isasymin}eclose{\isacharparenleft}{\kern0pt}{\isacharbraceleft}{\kern0pt}name{\isadigit{1}}{\isacharparenleft}{\kern0pt}x{\isacharparenright}{\kern0pt}{\isacharbraceright}{\kern0pt}{\isacharparenright}{\kern0pt}{\isachardoublequoteclose}\ {\isacharbar}{\kern0pt}\ {\isacharparenleft}{\kern0pt}b{\isacharparenright}{\kern0pt}\ {\isachardoublequoteopen}u\ {\isasymin}\ eclose{\isacharparenleft}{\kern0pt}{\isacharbraceleft}{\kern0pt}name{\isadigit{2}}{\isacharparenleft}{\kern0pt}x{\isacharparenright}{\kern0pt}{\isacharbraceright}{\kern0pt}{\isacharparenright}{\kern0pt}{\isachardoublequoteclose}\isanewline
\ \ \ \ \isacommand{unfolding}\isamarkupfalse%
\ ecloseN{\isacharunderscore}{\kern0pt}def\ \ eclose{\isacharunderscore}{\kern0pt}n{\isacharunderscore}{\kern0pt}def\ \isacommand{by}\isamarkupfalse%
\ auto\isanewline
\ \ \isacommand{then}\isamarkupfalse%
\ \isanewline
\ \ \isacommand{show}\isamarkupfalse%
\ {\isacharquery}{\kern0pt}thesis\isanewline
\ \ \isacommand{proof}\isamarkupfalse%
\ cases\isanewline
\ \ \ \ \isacommand{case}\isamarkupfalse%
\ a\isanewline
\ \ \ \ \isacommand{with}\isamarkupfalse%
\ {\isacartoucheopen}name{\isadigit{1}}{\isacharparenleft}{\kern0pt}x{\isacharparenright}{\kern0pt}\ {\isasymin}\ {\isacharunderscore}{\kern0pt}{\isacartoucheclose}\isanewline
\ \ \ \ \isacommand{show}\isamarkupfalse%
\ {\isacharquery}{\kern0pt}thesis\ \isanewline
\ \ \ \ \ \ \isacommand{unfolding}\isamarkupfalse%
\ ecloseN{\isacharunderscore}{\kern0pt}def\ \ eclose{\isacharunderscore}{\kern0pt}n{\isacharunderscore}{\kern0pt}def\isanewline
\ \ \ \ \ \ \isacommand{using}\isamarkupfalse%
\ eclose{\isacharunderscore}{\kern0pt}singE{\isacharbrackleft}{\kern0pt}OF\ a{\isacharbrackright}{\kern0pt}\ mem{\isacharunderscore}{\kern0pt}eclose{\isacharunderscore}{\kern0pt}trans{\isacharbrackleft}{\kern0pt}of\ u\ {\isachardoublequoteopen}name{\isadigit{1}}{\isacharparenleft}{\kern0pt}x{\isacharparenright}{\kern0pt}{\isachardoublequoteclose}\ {\isacharbrackright}{\kern0pt}\ \isacommand{by}\isamarkupfalse%
\ auto\ \isanewline
\ \ \isacommand{next}\isamarkupfalse%
\isanewline
\ \ \ \ \isacommand{case}\isamarkupfalse%
\ b\isanewline
\ \ \ \ \isacommand{with}\isamarkupfalse%
\ {\isacartoucheopen}name{\isadigit{2}}{\isacharparenleft}{\kern0pt}x{\isacharparenright}{\kern0pt}\ {\isasymin}\ {\isacharunderscore}{\kern0pt}{\isacartoucheclose}\isanewline
\ \ \ \ \isacommand{show}\isamarkupfalse%
\ {\isacharquery}{\kern0pt}thesis\ \isanewline
\ \ \ \ \ \ \isacommand{unfolding}\isamarkupfalse%
\ ecloseN{\isacharunderscore}{\kern0pt}def\ eclose{\isacharunderscore}{\kern0pt}n{\isacharunderscore}{\kern0pt}def\isanewline
\ \ \ \ \ \ \isacommand{using}\isamarkupfalse%
\ eclose{\isacharunderscore}{\kern0pt}singE{\isacharbrackleft}{\kern0pt}OF\ b{\isacharbrackright}{\kern0pt}\ mem{\isacharunderscore}{\kern0pt}eclose{\isacharunderscore}{\kern0pt}trans{\isacharbrackleft}{\kern0pt}of\ u\ {\isachardoublequoteopen}name{\isadigit{2}}{\isacharparenleft}{\kern0pt}x{\isacharparenright}{\kern0pt}{\isachardoublequoteclose}{\isacharbrackright}{\kern0pt}\ \isacommand{by}\isamarkupfalse%
\ auto\isanewline
\ \ \isacommand{qed}\isamarkupfalse%
\isanewline
\isacommand{qed}\isamarkupfalse%
%
\endisatagproof
{\isafoldproof}%
%
\isadelimproof
\isanewline
%
\endisadelimproof
\isanewline
\isanewline
\isanewline
\isanewline
\isacommand{definition}\isamarkupfalse%
\isanewline
\ \ is{\isacharunderscore}{\kern0pt}fst\ {\isacharcolon}{\kern0pt}{\isacharcolon}{\kern0pt}\ {\isachardoublequoteopen}{\isacharparenleft}{\kern0pt}i{\isasymRightarrow}o{\isacharparenright}{\kern0pt}{\isasymRightarrow}i{\isasymRightarrow}i{\isasymRightarrow}o{\isachardoublequoteclose}\ \isakeyword{where}\isanewline
\ \ {\isachardoublequoteopen}is{\isacharunderscore}{\kern0pt}fst{\isacharparenleft}{\kern0pt}M{\isacharcomma}{\kern0pt}x{\isacharcomma}{\kern0pt}t{\isacharparenright}{\kern0pt}\ {\isasymequiv}\ {\isacharparenleft}{\kern0pt}{\isasymexists}z{\isacharbrackleft}{\kern0pt}M{\isacharbrackright}{\kern0pt}{\isachardot}{\kern0pt}\ pair{\isacharparenleft}{\kern0pt}M{\isacharcomma}{\kern0pt}t{\isacharcomma}{\kern0pt}z{\isacharcomma}{\kern0pt}x{\isacharparenright}{\kern0pt}{\isacharparenright}{\kern0pt}\ {\isasymor}\ \isanewline
\ \ \ \ \ \ \ \ \ \ \ \ \ \ \ \ \ \ \ \ \ \ \ {\isacharparenleft}{\kern0pt}{\isasymnot}{\isacharparenleft}{\kern0pt}{\isasymexists}z{\isacharbrackleft}{\kern0pt}M{\isacharbrackright}{\kern0pt}{\isachardot}{\kern0pt}\ {\isasymexists}w{\isacharbrackleft}{\kern0pt}M{\isacharbrackright}{\kern0pt}{\isachardot}{\kern0pt}\ pair{\isacharparenleft}{\kern0pt}M{\isacharcomma}{\kern0pt}w{\isacharcomma}{\kern0pt}z{\isacharcomma}{\kern0pt}x{\isacharparenright}{\kern0pt}{\isacharparenright}{\kern0pt}\ {\isasymand}\ empty{\isacharparenleft}{\kern0pt}M{\isacharcomma}{\kern0pt}t{\isacharparenright}{\kern0pt}{\isacharparenright}{\kern0pt}{\isachardoublequoteclose}\isanewline
\isanewline
\isacommand{definition}\isamarkupfalse%
\isanewline
\ \ fst{\isacharunderscore}{\kern0pt}fm\ {\isacharcolon}{\kern0pt}{\isacharcolon}{\kern0pt}\ {\isachardoublequoteopen}{\isacharbrackleft}{\kern0pt}i{\isacharcomma}{\kern0pt}i{\isacharbrackright}{\kern0pt}\ {\isasymRightarrow}\ i{\isachardoublequoteclose}\ \isakeyword{where}\isanewline
\ \ {\isachardoublequoteopen}fst{\isacharunderscore}{\kern0pt}fm{\isacharparenleft}{\kern0pt}x{\isacharcomma}{\kern0pt}t{\isacharparenright}{\kern0pt}\ {\isasymequiv}\ Or{\isacharparenleft}{\kern0pt}Exists{\isacharparenleft}{\kern0pt}pair{\isacharunderscore}{\kern0pt}fm{\isacharparenleft}{\kern0pt}succ{\isacharparenleft}{\kern0pt}t{\isacharparenright}{\kern0pt}{\isacharcomma}{\kern0pt}{\isadigit{0}}{\isacharcomma}{\kern0pt}succ{\isacharparenleft}{\kern0pt}x{\isacharparenright}{\kern0pt}{\isacharparenright}{\kern0pt}{\isacharparenright}{\kern0pt}{\isacharcomma}{\kern0pt}\isanewline
\ \ \ \ \ \ \ \ \ \ \ \ \ \ \ \ \ \ \ And{\isacharparenleft}{\kern0pt}Neg{\isacharparenleft}{\kern0pt}Exists{\isacharparenleft}{\kern0pt}Exists{\isacharparenleft}{\kern0pt}pair{\isacharunderscore}{\kern0pt}fm{\isacharparenleft}{\kern0pt}{\isadigit{0}}{\isacharcomma}{\kern0pt}{\isadigit{1}}{\isacharcomma}{\kern0pt}{\isadigit{2}}\ {\isacharhash}{\kern0pt}{\isacharplus}{\kern0pt}\ x{\isacharparenright}{\kern0pt}{\isacharparenright}{\kern0pt}{\isacharparenright}{\kern0pt}{\isacharparenright}{\kern0pt}{\isacharcomma}{\kern0pt}empty{\isacharunderscore}{\kern0pt}fm{\isacharparenleft}{\kern0pt}t{\isacharparenright}{\kern0pt}{\isacharparenright}{\kern0pt}{\isacharparenright}{\kern0pt}{\isachardoublequoteclose}\isanewline
\isanewline
\isacommand{lemma}\isamarkupfalse%
\ sats{\isacharunderscore}{\kern0pt}fst{\isacharunderscore}{\kern0pt}fm\ {\isacharcolon}{\kern0pt}\isanewline
\ \ {\isachardoublequoteopen}{\isasymlbrakk}\ x\ {\isasymin}\ nat{\isacharsemicolon}{\kern0pt}\ y\ {\isasymin}\ nat{\isacharsemicolon}{\kern0pt}env\ {\isasymin}\ list{\isacharparenleft}{\kern0pt}A{\isacharparenright}{\kern0pt}\ {\isasymrbrakk}\ \isanewline
\ \ \ \ {\isasymLongrightarrow}\ sats{\isacharparenleft}{\kern0pt}A{\isacharcomma}{\kern0pt}\ fst{\isacharunderscore}{\kern0pt}fm{\isacharparenleft}{\kern0pt}x{\isacharcomma}{\kern0pt}y{\isacharparenright}{\kern0pt}{\isacharcomma}{\kern0pt}\ env{\isacharparenright}{\kern0pt}\ {\isasymlongleftrightarrow}\isanewline
\ \ \ \ \ \ \ \ is{\isacharunderscore}{\kern0pt}fst{\isacharparenleft}{\kern0pt}{\isacharhash}{\kern0pt}{\isacharhash}{\kern0pt}A{\isacharcomma}{\kern0pt}\ nth{\isacharparenleft}{\kern0pt}x{\isacharcomma}{\kern0pt}env{\isacharparenright}{\kern0pt}{\isacharcomma}{\kern0pt}\ nth{\isacharparenleft}{\kern0pt}y{\isacharcomma}{\kern0pt}env{\isacharparenright}{\kern0pt}{\isacharparenright}{\kern0pt}{\isachardoublequoteclose}\isanewline
%
\isadelimproof
\ \ %
\endisadelimproof
%
\isatagproof
\isacommand{by}\isamarkupfalse%
\ {\isacharparenleft}{\kern0pt}simp\ add{\isacharcolon}{\kern0pt}\ fst{\isacharunderscore}{\kern0pt}fm{\isacharunderscore}{\kern0pt}def\ is{\isacharunderscore}{\kern0pt}fst{\isacharunderscore}{\kern0pt}def{\isacharparenright}{\kern0pt}%
\endisatagproof
{\isafoldproof}%
%
\isadelimproof
\isanewline
%
\endisadelimproof
\isanewline
\isacommand{definition}\isamarkupfalse%
\ \isanewline
\ \ is{\isacharunderscore}{\kern0pt}ftype\ {\isacharcolon}{\kern0pt}{\isacharcolon}{\kern0pt}\ {\isachardoublequoteopen}{\isacharparenleft}{\kern0pt}i{\isasymRightarrow}o{\isacharparenright}{\kern0pt}{\isasymRightarrow}i{\isasymRightarrow}i{\isasymRightarrow}o{\isachardoublequoteclose}\ \isakeyword{where}\isanewline
\ \ {\isachardoublequoteopen}is{\isacharunderscore}{\kern0pt}ftype\ {\isasymequiv}\ is{\isacharunderscore}{\kern0pt}fst{\isachardoublequoteclose}\ \isanewline
\isanewline
\isacommand{definition}\isamarkupfalse%
\isanewline
\ \ ftype{\isacharunderscore}{\kern0pt}fm\ {\isacharcolon}{\kern0pt}{\isacharcolon}{\kern0pt}\ {\isachardoublequoteopen}{\isacharbrackleft}{\kern0pt}i{\isacharcomma}{\kern0pt}i{\isacharbrackright}{\kern0pt}\ {\isasymRightarrow}\ i{\isachardoublequoteclose}\ \isakeyword{where}\isanewline
\ \ {\isachardoublequoteopen}ftype{\isacharunderscore}{\kern0pt}fm\ {\isasymequiv}\ fst{\isacharunderscore}{\kern0pt}fm{\isachardoublequoteclose}\ \isanewline
\isanewline
\isacommand{lemma}\isamarkupfalse%
\ sats{\isacharunderscore}{\kern0pt}ftype{\isacharunderscore}{\kern0pt}fm\ {\isacharcolon}{\kern0pt}\isanewline
\ \ {\isachardoublequoteopen}{\isasymlbrakk}\ x\ {\isasymin}\ nat{\isacharsemicolon}{\kern0pt}\ y\ {\isasymin}\ nat{\isacharsemicolon}{\kern0pt}env\ {\isasymin}\ list{\isacharparenleft}{\kern0pt}A{\isacharparenright}{\kern0pt}\ {\isasymrbrakk}\ \isanewline
\ \ \ \ {\isasymLongrightarrow}\ sats{\isacharparenleft}{\kern0pt}A{\isacharcomma}{\kern0pt}\ ftype{\isacharunderscore}{\kern0pt}fm{\isacharparenleft}{\kern0pt}x{\isacharcomma}{\kern0pt}y{\isacharparenright}{\kern0pt}{\isacharcomma}{\kern0pt}\ env{\isacharparenright}{\kern0pt}\ {\isasymlongleftrightarrow}\isanewline
\ \ \ \ \ \ \ \ is{\isacharunderscore}{\kern0pt}ftype{\isacharparenleft}{\kern0pt}{\isacharhash}{\kern0pt}{\isacharhash}{\kern0pt}A{\isacharcomma}{\kern0pt}\ nth{\isacharparenleft}{\kern0pt}x{\isacharcomma}{\kern0pt}env{\isacharparenright}{\kern0pt}{\isacharcomma}{\kern0pt}\ nth{\isacharparenleft}{\kern0pt}y{\isacharcomma}{\kern0pt}env{\isacharparenright}{\kern0pt}{\isacharparenright}{\kern0pt}{\isachardoublequoteclose}\isanewline
%
\isadelimproof
\ \ %
\endisadelimproof
%
\isatagproof
\isacommand{unfolding}\isamarkupfalse%
\ ftype{\isacharunderscore}{\kern0pt}fm{\isacharunderscore}{\kern0pt}def\ is{\isacharunderscore}{\kern0pt}ftype{\isacharunderscore}{\kern0pt}def\isanewline
\ \ \isacommand{by}\isamarkupfalse%
\ {\isacharparenleft}{\kern0pt}simp\ add{\isacharcolon}{\kern0pt}sats{\isacharunderscore}{\kern0pt}fst{\isacharunderscore}{\kern0pt}fm{\isacharparenright}{\kern0pt}%
\endisatagproof
{\isafoldproof}%
%
\isadelimproof
\isanewline
%
\endisadelimproof
\isanewline
\isacommand{lemma}\isamarkupfalse%
\ is{\isacharunderscore}{\kern0pt}ftype{\isacharunderscore}{\kern0pt}iff{\isacharunderscore}{\kern0pt}sats{\isacharcolon}{\kern0pt}\isanewline
\ \ \isakeyword{assumes}\isanewline
\ \ \ \ {\isachardoublequoteopen}nth{\isacharparenleft}{\kern0pt}a{\isacharcomma}{\kern0pt}env{\isacharparenright}{\kern0pt}\ {\isacharequal}{\kern0pt}\ aa{\isachardoublequoteclose}\ {\isachardoublequoteopen}nth{\isacharparenleft}{\kern0pt}b{\isacharcomma}{\kern0pt}env{\isacharparenright}{\kern0pt}\ {\isacharequal}{\kern0pt}\ bb{\isachardoublequoteclose}\ {\isachardoublequoteopen}a{\isasymin}nat{\isachardoublequoteclose}\ {\isachardoublequoteopen}b{\isasymin}nat{\isachardoublequoteclose}\ {\isachardoublequoteopen}env\ {\isasymin}\ list{\isacharparenleft}{\kern0pt}A{\isacharparenright}{\kern0pt}{\isachardoublequoteclose}\isanewline
\ \ \isakeyword{shows}\isanewline
\ \ \ \ {\isachardoublequoteopen}is{\isacharunderscore}{\kern0pt}ftype{\isacharparenleft}{\kern0pt}{\isacharhash}{\kern0pt}{\isacharhash}{\kern0pt}A{\isacharcomma}{\kern0pt}aa{\isacharcomma}{\kern0pt}bb{\isacharparenright}{\kern0pt}\ \ {\isasymlongleftrightarrow}\ sats{\isacharparenleft}{\kern0pt}A{\isacharcomma}{\kern0pt}ftype{\isacharunderscore}{\kern0pt}fm{\isacharparenleft}{\kern0pt}a{\isacharcomma}{\kern0pt}b{\isacharparenright}{\kern0pt}{\isacharcomma}{\kern0pt}\ env{\isacharparenright}{\kern0pt}{\isachardoublequoteclose}\isanewline
%
\isadelimproof
\ \ %
\endisadelimproof
%
\isatagproof
\isacommand{using}\isamarkupfalse%
\ assms\isanewline
\ \ \isacommand{by}\isamarkupfalse%
\ {\isacharparenleft}{\kern0pt}simp\ add{\isacharcolon}{\kern0pt}sats{\isacharunderscore}{\kern0pt}ftype{\isacharunderscore}{\kern0pt}fm{\isacharparenright}{\kern0pt}%
\endisatagproof
{\isafoldproof}%
%
\isadelimproof
\isanewline
%
\endisadelimproof
\isanewline
\isacommand{definition}\isamarkupfalse%
\isanewline
\ \ is{\isacharunderscore}{\kern0pt}snd\ {\isacharcolon}{\kern0pt}{\isacharcolon}{\kern0pt}\ {\isachardoublequoteopen}{\isacharparenleft}{\kern0pt}i{\isasymRightarrow}o{\isacharparenright}{\kern0pt}{\isasymRightarrow}i{\isasymRightarrow}i{\isasymRightarrow}o{\isachardoublequoteclose}\ \isakeyword{where}\isanewline
\ \ {\isachardoublequoteopen}is{\isacharunderscore}{\kern0pt}snd{\isacharparenleft}{\kern0pt}M{\isacharcomma}{\kern0pt}x{\isacharcomma}{\kern0pt}t{\isacharparenright}{\kern0pt}\ {\isasymequiv}\ {\isacharparenleft}{\kern0pt}{\isasymexists}z{\isacharbrackleft}{\kern0pt}M{\isacharbrackright}{\kern0pt}{\isachardot}{\kern0pt}\ pair{\isacharparenleft}{\kern0pt}M{\isacharcomma}{\kern0pt}z{\isacharcomma}{\kern0pt}t{\isacharcomma}{\kern0pt}x{\isacharparenright}{\kern0pt}{\isacharparenright}{\kern0pt}\ {\isasymor}\ \isanewline
\ \ \ \ \ \ \ \ \ \ \ \ \ \ \ \ \ \ \ \ \ \ \ {\isacharparenleft}{\kern0pt}{\isasymnot}{\isacharparenleft}{\kern0pt}{\isasymexists}z{\isacharbrackleft}{\kern0pt}M{\isacharbrackright}{\kern0pt}{\isachardot}{\kern0pt}\ {\isasymexists}w{\isacharbrackleft}{\kern0pt}M{\isacharbrackright}{\kern0pt}{\isachardot}{\kern0pt}\ pair{\isacharparenleft}{\kern0pt}M{\isacharcomma}{\kern0pt}z{\isacharcomma}{\kern0pt}w{\isacharcomma}{\kern0pt}x{\isacharparenright}{\kern0pt}{\isacharparenright}{\kern0pt}\ {\isasymand}\ empty{\isacharparenleft}{\kern0pt}M{\isacharcomma}{\kern0pt}t{\isacharparenright}{\kern0pt}{\isacharparenright}{\kern0pt}{\isachardoublequoteclose}\isanewline
\isanewline
\isacommand{definition}\isamarkupfalse%
\isanewline
\ \ snd{\isacharunderscore}{\kern0pt}fm\ {\isacharcolon}{\kern0pt}{\isacharcolon}{\kern0pt}\ {\isachardoublequoteopen}{\isacharbrackleft}{\kern0pt}i{\isacharcomma}{\kern0pt}i{\isacharbrackright}{\kern0pt}\ {\isasymRightarrow}\ i{\isachardoublequoteclose}\ \isakeyword{where}\isanewline
\ \ {\isachardoublequoteopen}snd{\isacharunderscore}{\kern0pt}fm{\isacharparenleft}{\kern0pt}x{\isacharcomma}{\kern0pt}t{\isacharparenright}{\kern0pt}\ {\isasymequiv}\ Or{\isacharparenleft}{\kern0pt}Exists{\isacharparenleft}{\kern0pt}pair{\isacharunderscore}{\kern0pt}fm{\isacharparenleft}{\kern0pt}{\isadigit{0}}{\isacharcomma}{\kern0pt}succ{\isacharparenleft}{\kern0pt}t{\isacharparenright}{\kern0pt}{\isacharcomma}{\kern0pt}succ{\isacharparenleft}{\kern0pt}x{\isacharparenright}{\kern0pt}{\isacharparenright}{\kern0pt}{\isacharparenright}{\kern0pt}{\isacharcomma}{\kern0pt}\isanewline
\ \ \ \ \ \ \ \ \ \ \ \ \ \ \ \ \ \ \ And{\isacharparenleft}{\kern0pt}Neg{\isacharparenleft}{\kern0pt}Exists{\isacharparenleft}{\kern0pt}Exists{\isacharparenleft}{\kern0pt}pair{\isacharunderscore}{\kern0pt}fm{\isacharparenleft}{\kern0pt}{\isadigit{1}}{\isacharcomma}{\kern0pt}{\isadigit{0}}{\isacharcomma}{\kern0pt}{\isadigit{2}}\ {\isacharhash}{\kern0pt}{\isacharplus}{\kern0pt}\ x{\isacharparenright}{\kern0pt}{\isacharparenright}{\kern0pt}{\isacharparenright}{\kern0pt}{\isacharparenright}{\kern0pt}{\isacharcomma}{\kern0pt}empty{\isacharunderscore}{\kern0pt}fm{\isacharparenleft}{\kern0pt}t{\isacharparenright}{\kern0pt}{\isacharparenright}{\kern0pt}{\isacharparenright}{\kern0pt}{\isachardoublequoteclose}\isanewline
\isanewline
\isacommand{lemma}\isamarkupfalse%
\ sats{\isacharunderscore}{\kern0pt}snd{\isacharunderscore}{\kern0pt}fm\ {\isacharcolon}{\kern0pt}\isanewline
\ \ {\isachardoublequoteopen}{\isasymlbrakk}\ x\ {\isasymin}\ nat{\isacharsemicolon}{\kern0pt}\ y\ {\isasymin}\ nat{\isacharsemicolon}{\kern0pt}env\ {\isasymin}\ list{\isacharparenleft}{\kern0pt}A{\isacharparenright}{\kern0pt}\ {\isasymrbrakk}\ \isanewline
\ \ \ \ {\isasymLongrightarrow}\ sats{\isacharparenleft}{\kern0pt}A{\isacharcomma}{\kern0pt}\ snd{\isacharunderscore}{\kern0pt}fm{\isacharparenleft}{\kern0pt}x{\isacharcomma}{\kern0pt}y{\isacharparenright}{\kern0pt}{\isacharcomma}{\kern0pt}\ env{\isacharparenright}{\kern0pt}\ {\isasymlongleftrightarrow}\isanewline
\ \ \ \ \ \ \ \ is{\isacharunderscore}{\kern0pt}snd{\isacharparenleft}{\kern0pt}{\isacharhash}{\kern0pt}{\isacharhash}{\kern0pt}A{\isacharcomma}{\kern0pt}\ nth{\isacharparenleft}{\kern0pt}x{\isacharcomma}{\kern0pt}env{\isacharparenright}{\kern0pt}{\isacharcomma}{\kern0pt}\ nth{\isacharparenleft}{\kern0pt}y{\isacharcomma}{\kern0pt}env{\isacharparenright}{\kern0pt}{\isacharparenright}{\kern0pt}{\isachardoublequoteclose}\isanewline
%
\isadelimproof
\ \ %
\endisadelimproof
%
\isatagproof
\isacommand{by}\isamarkupfalse%
\ {\isacharparenleft}{\kern0pt}simp\ add{\isacharcolon}{\kern0pt}\ snd{\isacharunderscore}{\kern0pt}fm{\isacharunderscore}{\kern0pt}def\ is{\isacharunderscore}{\kern0pt}snd{\isacharunderscore}{\kern0pt}def{\isacharparenright}{\kern0pt}%
\endisatagproof
{\isafoldproof}%
%
\isadelimproof
\isanewline
%
\endisadelimproof
\isanewline
\isacommand{definition}\isamarkupfalse%
\isanewline
\ \ is{\isacharunderscore}{\kern0pt}name{\isadigit{1}}\ {\isacharcolon}{\kern0pt}{\isacharcolon}{\kern0pt}\ {\isachardoublequoteopen}{\isacharparenleft}{\kern0pt}i{\isasymRightarrow}o{\isacharparenright}{\kern0pt}{\isasymRightarrow}i{\isasymRightarrow}i{\isasymRightarrow}o{\isachardoublequoteclose}\ \isakeyword{where}\isanewline
\ \ {\isachardoublequoteopen}is{\isacharunderscore}{\kern0pt}name{\isadigit{1}}{\isacharparenleft}{\kern0pt}M{\isacharcomma}{\kern0pt}x{\isacharcomma}{\kern0pt}t{\isadigit{2}}{\isacharparenright}{\kern0pt}\ {\isasymequiv}\ is{\isacharunderscore}{\kern0pt}hcomp{\isacharparenleft}{\kern0pt}M{\isacharcomma}{\kern0pt}is{\isacharunderscore}{\kern0pt}fst{\isacharparenleft}{\kern0pt}M{\isacharparenright}{\kern0pt}{\isacharcomma}{\kern0pt}is{\isacharunderscore}{\kern0pt}snd{\isacharparenleft}{\kern0pt}M{\isacharparenright}{\kern0pt}{\isacharcomma}{\kern0pt}x{\isacharcomma}{\kern0pt}t{\isadigit{2}}{\isacharparenright}{\kern0pt}{\isachardoublequoteclose}\isanewline
\isanewline
\isacommand{definition}\isamarkupfalse%
\isanewline
\ \ name{\isadigit{1}}{\isacharunderscore}{\kern0pt}fm\ {\isacharcolon}{\kern0pt}{\isacharcolon}{\kern0pt}\ {\isachardoublequoteopen}{\isacharbrackleft}{\kern0pt}i{\isacharcomma}{\kern0pt}i{\isacharbrackright}{\kern0pt}\ {\isasymRightarrow}\ i{\isachardoublequoteclose}\ \isakeyword{where}\isanewline
\ \ {\isachardoublequoteopen}name{\isadigit{1}}{\isacharunderscore}{\kern0pt}fm{\isacharparenleft}{\kern0pt}x{\isacharcomma}{\kern0pt}t{\isacharparenright}{\kern0pt}\ {\isasymequiv}\ hcomp{\isacharunderscore}{\kern0pt}fm{\isacharparenleft}{\kern0pt}fst{\isacharunderscore}{\kern0pt}fm{\isacharcomma}{\kern0pt}snd{\isacharunderscore}{\kern0pt}fm{\isacharcomma}{\kern0pt}x{\isacharcomma}{\kern0pt}t{\isacharparenright}{\kern0pt}{\isachardoublequoteclose}\ \isanewline
\isanewline
\isacommand{lemma}\isamarkupfalse%
\ sats{\isacharunderscore}{\kern0pt}name{\isadigit{1}}{\isacharunderscore}{\kern0pt}fm\ {\isacharcolon}{\kern0pt}\isanewline
\ \ {\isachardoublequoteopen}{\isasymlbrakk}\ x\ {\isasymin}\ nat{\isacharsemicolon}{\kern0pt}\ y\ {\isasymin}\ nat{\isacharsemicolon}{\kern0pt}env\ {\isasymin}\ list{\isacharparenleft}{\kern0pt}A{\isacharparenright}{\kern0pt}\ {\isasymrbrakk}\ \isanewline
\ \ \ \ {\isasymLongrightarrow}\ sats{\isacharparenleft}{\kern0pt}A{\isacharcomma}{\kern0pt}\ name{\isadigit{1}}{\isacharunderscore}{\kern0pt}fm{\isacharparenleft}{\kern0pt}x{\isacharcomma}{\kern0pt}y{\isacharparenright}{\kern0pt}{\isacharcomma}{\kern0pt}\ env{\isacharparenright}{\kern0pt}\ {\isasymlongleftrightarrow}\isanewline
\ \ \ \ \ \ \ \ is{\isacharunderscore}{\kern0pt}name{\isadigit{1}}{\isacharparenleft}{\kern0pt}{\isacharhash}{\kern0pt}{\isacharhash}{\kern0pt}A{\isacharcomma}{\kern0pt}\ nth{\isacharparenleft}{\kern0pt}x{\isacharcomma}{\kern0pt}env{\isacharparenright}{\kern0pt}{\isacharcomma}{\kern0pt}\ nth{\isacharparenleft}{\kern0pt}y{\isacharcomma}{\kern0pt}env{\isacharparenright}{\kern0pt}{\isacharparenright}{\kern0pt}{\isachardoublequoteclose}\isanewline
%
\isadelimproof
\ \ %
\endisadelimproof
%
\isatagproof
\isacommand{unfolding}\isamarkupfalse%
\ name{\isadigit{1}}{\isacharunderscore}{\kern0pt}fm{\isacharunderscore}{\kern0pt}def\ is{\isacharunderscore}{\kern0pt}name{\isadigit{1}}{\isacharunderscore}{\kern0pt}def\ \isacommand{using}\isamarkupfalse%
\ sats{\isacharunderscore}{\kern0pt}fst{\isacharunderscore}{\kern0pt}fm\ sats{\isacharunderscore}{\kern0pt}snd{\isacharunderscore}{\kern0pt}fm\ \isanewline
\ \ \ \ sats{\isacharunderscore}{\kern0pt}hcomp{\isacharunderscore}{\kern0pt}fm{\isacharbrackleft}{\kern0pt}of\ A\ {\isachardoublequoteopen}is{\isacharunderscore}{\kern0pt}fst{\isacharparenleft}{\kern0pt}{\isacharhash}{\kern0pt}{\isacharhash}{\kern0pt}A{\isacharparenright}{\kern0pt}{\isachardoublequoteclose}\ {\isacharunderscore}{\kern0pt}\ fst{\isacharunderscore}{\kern0pt}fm\ {\isachardoublequoteopen}is{\isacharunderscore}{\kern0pt}snd{\isacharparenleft}{\kern0pt}{\isacharhash}{\kern0pt}{\isacharhash}{\kern0pt}A{\isacharparenright}{\kern0pt}{\isachardoublequoteclose}{\isacharbrackright}{\kern0pt}\ \isacommand{by}\isamarkupfalse%
\ simp%
\endisatagproof
{\isafoldproof}%
%
\isadelimproof
\isanewline
%
\endisadelimproof
\isanewline
\isacommand{lemma}\isamarkupfalse%
\ is{\isacharunderscore}{\kern0pt}name{\isadigit{1}}{\isacharunderscore}{\kern0pt}iff{\isacharunderscore}{\kern0pt}sats{\isacharcolon}{\kern0pt}\isanewline
\ \ \isakeyword{assumes}\isanewline
\ \ \ \ {\isachardoublequoteopen}nth{\isacharparenleft}{\kern0pt}a{\isacharcomma}{\kern0pt}env{\isacharparenright}{\kern0pt}\ {\isacharequal}{\kern0pt}\ aa{\isachardoublequoteclose}\ {\isachardoublequoteopen}nth{\isacharparenleft}{\kern0pt}b{\isacharcomma}{\kern0pt}env{\isacharparenright}{\kern0pt}\ {\isacharequal}{\kern0pt}\ bb{\isachardoublequoteclose}\ {\isachardoublequoteopen}a{\isasymin}nat{\isachardoublequoteclose}\ {\isachardoublequoteopen}b{\isasymin}nat{\isachardoublequoteclose}\ {\isachardoublequoteopen}env\ {\isasymin}\ list{\isacharparenleft}{\kern0pt}A{\isacharparenright}{\kern0pt}{\isachardoublequoteclose}\isanewline
\ \ \isakeyword{shows}\isanewline
\ \ \ \ {\isachardoublequoteopen}is{\isacharunderscore}{\kern0pt}name{\isadigit{1}}{\isacharparenleft}{\kern0pt}{\isacharhash}{\kern0pt}{\isacharhash}{\kern0pt}A{\isacharcomma}{\kern0pt}aa{\isacharcomma}{\kern0pt}bb{\isacharparenright}{\kern0pt}\ \ {\isasymlongleftrightarrow}\ sats{\isacharparenleft}{\kern0pt}A{\isacharcomma}{\kern0pt}name{\isadigit{1}}{\isacharunderscore}{\kern0pt}fm{\isacharparenleft}{\kern0pt}a{\isacharcomma}{\kern0pt}b{\isacharparenright}{\kern0pt}{\isacharcomma}{\kern0pt}\ env{\isacharparenright}{\kern0pt}{\isachardoublequoteclose}\isanewline
%
\isadelimproof
\ \ %
\endisadelimproof
%
\isatagproof
\isacommand{using}\isamarkupfalse%
\ assms\isanewline
\ \ \isacommand{by}\isamarkupfalse%
\ {\isacharparenleft}{\kern0pt}simp\ add{\isacharcolon}{\kern0pt}sats{\isacharunderscore}{\kern0pt}name{\isadigit{1}}{\isacharunderscore}{\kern0pt}fm{\isacharparenright}{\kern0pt}%
\endisatagproof
{\isafoldproof}%
%
\isadelimproof
\isanewline
%
\endisadelimproof
\isanewline
\isacommand{definition}\isamarkupfalse%
\isanewline
\ \ is{\isacharunderscore}{\kern0pt}snd{\isacharunderscore}{\kern0pt}snd\ {\isacharcolon}{\kern0pt}{\isacharcolon}{\kern0pt}\ {\isachardoublequoteopen}{\isacharparenleft}{\kern0pt}i{\isasymRightarrow}o{\isacharparenright}{\kern0pt}{\isasymRightarrow}i{\isasymRightarrow}i{\isasymRightarrow}o{\isachardoublequoteclose}\ \isakeyword{where}\isanewline
\ \ {\isachardoublequoteopen}is{\isacharunderscore}{\kern0pt}snd{\isacharunderscore}{\kern0pt}snd{\isacharparenleft}{\kern0pt}M{\isacharcomma}{\kern0pt}x{\isacharcomma}{\kern0pt}t{\isacharparenright}{\kern0pt}\ {\isasymequiv}\ is{\isacharunderscore}{\kern0pt}hcomp{\isacharparenleft}{\kern0pt}M{\isacharcomma}{\kern0pt}is{\isacharunderscore}{\kern0pt}snd{\isacharparenleft}{\kern0pt}M{\isacharparenright}{\kern0pt}{\isacharcomma}{\kern0pt}is{\isacharunderscore}{\kern0pt}snd{\isacharparenleft}{\kern0pt}M{\isacharparenright}{\kern0pt}{\isacharcomma}{\kern0pt}x{\isacharcomma}{\kern0pt}t{\isacharparenright}{\kern0pt}{\isachardoublequoteclose}\isanewline
\isanewline
\isacommand{definition}\isamarkupfalse%
\isanewline
\ \ snd{\isacharunderscore}{\kern0pt}snd{\isacharunderscore}{\kern0pt}fm\ {\isacharcolon}{\kern0pt}{\isacharcolon}{\kern0pt}\ {\isachardoublequoteopen}{\isacharbrackleft}{\kern0pt}i{\isacharcomma}{\kern0pt}i{\isacharbrackright}{\kern0pt}{\isasymRightarrow}i{\isachardoublequoteclose}\ \isakeyword{where}\isanewline
\ \ {\isachardoublequoteopen}snd{\isacharunderscore}{\kern0pt}snd{\isacharunderscore}{\kern0pt}fm{\isacharparenleft}{\kern0pt}x{\isacharcomma}{\kern0pt}t{\isacharparenright}{\kern0pt}\ {\isasymequiv}\ hcomp{\isacharunderscore}{\kern0pt}fm{\isacharparenleft}{\kern0pt}snd{\isacharunderscore}{\kern0pt}fm{\isacharcomma}{\kern0pt}snd{\isacharunderscore}{\kern0pt}fm{\isacharcomma}{\kern0pt}x{\isacharcomma}{\kern0pt}t{\isacharparenright}{\kern0pt}{\isachardoublequoteclose}\isanewline
\isanewline
\isacommand{lemma}\isamarkupfalse%
\ sats{\isacharunderscore}{\kern0pt}snd{\isadigit{2}}{\isacharunderscore}{\kern0pt}fm\ {\isacharcolon}{\kern0pt}\isanewline
\ \ {\isachardoublequoteopen}{\isasymlbrakk}\ x\ {\isasymin}\ nat{\isacharsemicolon}{\kern0pt}\ y\ {\isasymin}\ nat{\isacharsemicolon}{\kern0pt}env\ {\isasymin}\ list{\isacharparenleft}{\kern0pt}A{\isacharparenright}{\kern0pt}\ {\isasymrbrakk}\ \isanewline
\ \ \ \ {\isasymLongrightarrow}\ sats{\isacharparenleft}{\kern0pt}A{\isacharcomma}{\kern0pt}snd{\isacharunderscore}{\kern0pt}snd{\isacharunderscore}{\kern0pt}fm{\isacharparenleft}{\kern0pt}x{\isacharcomma}{\kern0pt}y{\isacharparenright}{\kern0pt}{\isacharcomma}{\kern0pt}\ env{\isacharparenright}{\kern0pt}\ {\isasymlongleftrightarrow}\isanewline
\ \ \ \ \ \ \ \ is{\isacharunderscore}{\kern0pt}snd{\isacharunderscore}{\kern0pt}snd{\isacharparenleft}{\kern0pt}{\isacharhash}{\kern0pt}{\isacharhash}{\kern0pt}A{\isacharcomma}{\kern0pt}\ nth{\isacharparenleft}{\kern0pt}x{\isacharcomma}{\kern0pt}env{\isacharparenright}{\kern0pt}{\isacharcomma}{\kern0pt}\ nth{\isacharparenleft}{\kern0pt}y{\isacharcomma}{\kern0pt}env{\isacharparenright}{\kern0pt}{\isacharparenright}{\kern0pt}{\isachardoublequoteclose}\isanewline
%
\isadelimproof
\ \ %
\endisadelimproof
%
\isatagproof
\isacommand{unfolding}\isamarkupfalse%
\ snd{\isacharunderscore}{\kern0pt}snd{\isacharunderscore}{\kern0pt}fm{\isacharunderscore}{\kern0pt}def\ is{\isacharunderscore}{\kern0pt}snd{\isacharunderscore}{\kern0pt}snd{\isacharunderscore}{\kern0pt}def\ \isacommand{using}\isamarkupfalse%
\ sats{\isacharunderscore}{\kern0pt}snd{\isacharunderscore}{\kern0pt}fm\ \isanewline
\ \ \ \ sats{\isacharunderscore}{\kern0pt}hcomp{\isacharunderscore}{\kern0pt}fm{\isacharbrackleft}{\kern0pt}of\ A\ {\isachardoublequoteopen}is{\isacharunderscore}{\kern0pt}snd{\isacharparenleft}{\kern0pt}{\isacharhash}{\kern0pt}{\isacharhash}{\kern0pt}A{\isacharparenright}{\kern0pt}{\isachardoublequoteclose}\ {\isacharunderscore}{\kern0pt}\ snd{\isacharunderscore}{\kern0pt}fm\ {\isachardoublequoteopen}is{\isacharunderscore}{\kern0pt}snd{\isacharparenleft}{\kern0pt}{\isacharhash}{\kern0pt}{\isacharhash}{\kern0pt}A{\isacharparenright}{\kern0pt}{\isachardoublequoteclose}{\isacharbrackright}{\kern0pt}\ \isacommand{by}\isamarkupfalse%
\ simp%
\endisatagproof
{\isafoldproof}%
%
\isadelimproof
\isanewline
%
\endisadelimproof
\isanewline
\isacommand{definition}\isamarkupfalse%
\isanewline
\ \ is{\isacharunderscore}{\kern0pt}name{\isadigit{2}}\ {\isacharcolon}{\kern0pt}{\isacharcolon}{\kern0pt}\ {\isachardoublequoteopen}{\isacharparenleft}{\kern0pt}i{\isasymRightarrow}o{\isacharparenright}{\kern0pt}{\isasymRightarrow}i{\isasymRightarrow}i{\isasymRightarrow}o{\isachardoublequoteclose}\ \isakeyword{where}\isanewline
\ \ {\isachardoublequoteopen}is{\isacharunderscore}{\kern0pt}name{\isadigit{2}}{\isacharparenleft}{\kern0pt}M{\isacharcomma}{\kern0pt}x{\isacharcomma}{\kern0pt}t{\isadigit{3}}{\isacharparenright}{\kern0pt}\ {\isasymequiv}\ is{\isacharunderscore}{\kern0pt}hcomp{\isacharparenleft}{\kern0pt}M{\isacharcomma}{\kern0pt}is{\isacharunderscore}{\kern0pt}fst{\isacharparenleft}{\kern0pt}M{\isacharparenright}{\kern0pt}{\isacharcomma}{\kern0pt}is{\isacharunderscore}{\kern0pt}snd{\isacharunderscore}{\kern0pt}snd{\isacharparenleft}{\kern0pt}M{\isacharparenright}{\kern0pt}{\isacharcomma}{\kern0pt}x{\isacharcomma}{\kern0pt}t{\isadigit{3}}{\isacharparenright}{\kern0pt}{\isachardoublequoteclose}\isanewline
\isanewline
\isacommand{definition}\isamarkupfalse%
\isanewline
\ \ name{\isadigit{2}}{\isacharunderscore}{\kern0pt}fm\ {\isacharcolon}{\kern0pt}{\isacharcolon}{\kern0pt}\ {\isachardoublequoteopen}{\isacharbrackleft}{\kern0pt}i{\isacharcomma}{\kern0pt}i{\isacharbrackright}{\kern0pt}\ {\isasymRightarrow}\ i{\isachardoublequoteclose}\ \isakeyword{where}\isanewline
\ \ {\isachardoublequoteopen}name{\isadigit{2}}{\isacharunderscore}{\kern0pt}fm{\isacharparenleft}{\kern0pt}x{\isacharcomma}{\kern0pt}t{\isadigit{3}}{\isacharparenright}{\kern0pt}\ {\isasymequiv}\ hcomp{\isacharunderscore}{\kern0pt}fm{\isacharparenleft}{\kern0pt}fst{\isacharunderscore}{\kern0pt}fm{\isacharcomma}{\kern0pt}snd{\isacharunderscore}{\kern0pt}snd{\isacharunderscore}{\kern0pt}fm{\isacharcomma}{\kern0pt}x{\isacharcomma}{\kern0pt}t{\isadigit{3}}{\isacharparenright}{\kern0pt}{\isachardoublequoteclose}\isanewline
\isanewline
\isacommand{lemma}\isamarkupfalse%
\ sats{\isacharunderscore}{\kern0pt}name{\isadigit{2}}{\isacharunderscore}{\kern0pt}fm\ {\isacharcolon}{\kern0pt}\isanewline
\ \ {\isachardoublequoteopen}{\isasymlbrakk}\ x\ {\isasymin}\ nat{\isacharsemicolon}{\kern0pt}\ y\ {\isasymin}\ nat{\isacharsemicolon}{\kern0pt}env\ {\isasymin}\ list{\isacharparenleft}{\kern0pt}A{\isacharparenright}{\kern0pt}\ {\isasymrbrakk}\ \isanewline
\ \ \ \ {\isasymLongrightarrow}\ sats{\isacharparenleft}{\kern0pt}A{\isacharcomma}{\kern0pt}name{\isadigit{2}}{\isacharunderscore}{\kern0pt}fm{\isacharparenleft}{\kern0pt}x{\isacharcomma}{\kern0pt}y{\isacharparenright}{\kern0pt}{\isacharcomma}{\kern0pt}\ env{\isacharparenright}{\kern0pt}\ {\isasymlongleftrightarrow}\isanewline
\ \ \ \ \ \ \ \ is{\isacharunderscore}{\kern0pt}name{\isadigit{2}}{\isacharparenleft}{\kern0pt}{\isacharhash}{\kern0pt}{\isacharhash}{\kern0pt}A{\isacharcomma}{\kern0pt}\ nth{\isacharparenleft}{\kern0pt}x{\isacharcomma}{\kern0pt}env{\isacharparenright}{\kern0pt}{\isacharcomma}{\kern0pt}\ nth{\isacharparenleft}{\kern0pt}y{\isacharcomma}{\kern0pt}env{\isacharparenright}{\kern0pt}{\isacharparenright}{\kern0pt}{\isachardoublequoteclose}\isanewline
%
\isadelimproof
\ \ %
\endisadelimproof
%
\isatagproof
\isacommand{unfolding}\isamarkupfalse%
\ name{\isadigit{2}}{\isacharunderscore}{\kern0pt}fm{\isacharunderscore}{\kern0pt}def\ is{\isacharunderscore}{\kern0pt}name{\isadigit{2}}{\isacharunderscore}{\kern0pt}def\ \isacommand{using}\isamarkupfalse%
\ sats{\isacharunderscore}{\kern0pt}fst{\isacharunderscore}{\kern0pt}fm\ sats{\isacharunderscore}{\kern0pt}snd{\isadigit{2}}{\isacharunderscore}{\kern0pt}fm\isanewline
\ \ \ \ sats{\isacharunderscore}{\kern0pt}hcomp{\isacharunderscore}{\kern0pt}fm{\isacharbrackleft}{\kern0pt}of\ A\ {\isachardoublequoteopen}is{\isacharunderscore}{\kern0pt}fst{\isacharparenleft}{\kern0pt}{\isacharhash}{\kern0pt}{\isacharhash}{\kern0pt}A{\isacharparenright}{\kern0pt}{\isachardoublequoteclose}\ {\isacharunderscore}{\kern0pt}\ fst{\isacharunderscore}{\kern0pt}fm\ {\isachardoublequoteopen}is{\isacharunderscore}{\kern0pt}snd{\isacharunderscore}{\kern0pt}snd{\isacharparenleft}{\kern0pt}{\isacharhash}{\kern0pt}{\isacharhash}{\kern0pt}A{\isacharparenright}{\kern0pt}{\isachardoublequoteclose}{\isacharbrackright}{\kern0pt}\ \isacommand{by}\isamarkupfalse%
\ simp%
\endisatagproof
{\isafoldproof}%
%
\isadelimproof
\isanewline
%
\endisadelimproof
\isanewline
\isacommand{lemma}\isamarkupfalse%
\ is{\isacharunderscore}{\kern0pt}name{\isadigit{2}}{\isacharunderscore}{\kern0pt}iff{\isacharunderscore}{\kern0pt}sats{\isacharcolon}{\kern0pt}\isanewline
\ \ \isakeyword{assumes}\isanewline
\ \ \ \ {\isachardoublequoteopen}nth{\isacharparenleft}{\kern0pt}a{\isacharcomma}{\kern0pt}env{\isacharparenright}{\kern0pt}\ {\isacharequal}{\kern0pt}\ aa{\isachardoublequoteclose}\ {\isachardoublequoteopen}nth{\isacharparenleft}{\kern0pt}b{\isacharcomma}{\kern0pt}env{\isacharparenright}{\kern0pt}\ {\isacharequal}{\kern0pt}\ bb{\isachardoublequoteclose}\ {\isachardoublequoteopen}a{\isasymin}nat{\isachardoublequoteclose}\ {\isachardoublequoteopen}b{\isasymin}nat{\isachardoublequoteclose}\ {\isachardoublequoteopen}env\ {\isasymin}\ list{\isacharparenleft}{\kern0pt}A{\isacharparenright}{\kern0pt}{\isachardoublequoteclose}\isanewline
\ \ \isakeyword{shows}\isanewline
\ \ \ \ {\isachardoublequoteopen}is{\isacharunderscore}{\kern0pt}name{\isadigit{2}}{\isacharparenleft}{\kern0pt}{\isacharhash}{\kern0pt}{\isacharhash}{\kern0pt}A{\isacharcomma}{\kern0pt}aa{\isacharcomma}{\kern0pt}bb{\isacharparenright}{\kern0pt}\ \ {\isasymlongleftrightarrow}\ sats{\isacharparenleft}{\kern0pt}A{\isacharcomma}{\kern0pt}name{\isadigit{2}}{\isacharunderscore}{\kern0pt}fm{\isacharparenleft}{\kern0pt}a{\isacharcomma}{\kern0pt}b{\isacharparenright}{\kern0pt}{\isacharcomma}{\kern0pt}\ env{\isacharparenright}{\kern0pt}{\isachardoublequoteclose}\isanewline
%
\isadelimproof
\ \ %
\endisadelimproof
%
\isatagproof
\isacommand{using}\isamarkupfalse%
\ assms\isanewline
\ \ \isacommand{by}\isamarkupfalse%
\ {\isacharparenleft}{\kern0pt}simp\ add{\isacharcolon}{\kern0pt}sats{\isacharunderscore}{\kern0pt}name{\isadigit{2}}{\isacharunderscore}{\kern0pt}fm{\isacharparenright}{\kern0pt}%
\endisatagproof
{\isafoldproof}%
%
\isadelimproof
\isanewline
%
\endisadelimproof
\isanewline
\isacommand{definition}\isamarkupfalse%
\isanewline
\ \ is{\isacharunderscore}{\kern0pt}cond{\isacharunderscore}{\kern0pt}of\ {\isacharcolon}{\kern0pt}{\isacharcolon}{\kern0pt}\ {\isachardoublequoteopen}{\isacharparenleft}{\kern0pt}i{\isasymRightarrow}o{\isacharparenright}{\kern0pt}{\isasymRightarrow}i{\isasymRightarrow}i{\isasymRightarrow}o{\isachardoublequoteclose}\ \isakeyword{where}\isanewline
\ \ {\isachardoublequoteopen}is{\isacharunderscore}{\kern0pt}cond{\isacharunderscore}{\kern0pt}of{\isacharparenleft}{\kern0pt}M{\isacharcomma}{\kern0pt}x{\isacharcomma}{\kern0pt}t{\isadigit{4}}{\isacharparenright}{\kern0pt}\ {\isasymequiv}\ is{\isacharunderscore}{\kern0pt}hcomp{\isacharparenleft}{\kern0pt}M{\isacharcomma}{\kern0pt}is{\isacharunderscore}{\kern0pt}snd{\isacharparenleft}{\kern0pt}M{\isacharparenright}{\kern0pt}{\isacharcomma}{\kern0pt}is{\isacharunderscore}{\kern0pt}snd{\isacharunderscore}{\kern0pt}snd{\isacharparenleft}{\kern0pt}M{\isacharparenright}{\kern0pt}{\isacharcomma}{\kern0pt}x{\isacharcomma}{\kern0pt}t{\isadigit{4}}{\isacharparenright}{\kern0pt}{\isachardoublequoteclose}\isanewline
\isanewline
\isacommand{definition}\isamarkupfalse%
\isanewline
\ \ cond{\isacharunderscore}{\kern0pt}of{\isacharunderscore}{\kern0pt}fm\ {\isacharcolon}{\kern0pt}{\isacharcolon}{\kern0pt}\ {\isachardoublequoteopen}{\isacharbrackleft}{\kern0pt}i{\isacharcomma}{\kern0pt}i{\isacharbrackright}{\kern0pt}\ {\isasymRightarrow}\ i{\isachardoublequoteclose}\ \isakeyword{where}\isanewline
\ \ {\isachardoublequoteopen}cond{\isacharunderscore}{\kern0pt}of{\isacharunderscore}{\kern0pt}fm{\isacharparenleft}{\kern0pt}x{\isacharcomma}{\kern0pt}t{\isadigit{4}}{\isacharparenright}{\kern0pt}\ {\isasymequiv}\ hcomp{\isacharunderscore}{\kern0pt}fm{\isacharparenleft}{\kern0pt}snd{\isacharunderscore}{\kern0pt}fm{\isacharcomma}{\kern0pt}snd{\isacharunderscore}{\kern0pt}snd{\isacharunderscore}{\kern0pt}fm{\isacharcomma}{\kern0pt}x{\isacharcomma}{\kern0pt}t{\isadigit{4}}{\isacharparenright}{\kern0pt}{\isachardoublequoteclose}\isanewline
\isanewline
\isacommand{lemma}\isamarkupfalse%
\ sats{\isacharunderscore}{\kern0pt}cond{\isacharunderscore}{\kern0pt}of{\isacharunderscore}{\kern0pt}fm\ {\isacharcolon}{\kern0pt}\isanewline
\ \ {\isachardoublequoteopen}{\isasymlbrakk}\ x\ {\isasymin}\ nat{\isacharsemicolon}{\kern0pt}\ y\ {\isasymin}\ nat{\isacharsemicolon}{\kern0pt}env\ {\isasymin}\ list{\isacharparenleft}{\kern0pt}A{\isacharparenright}{\kern0pt}\ {\isasymrbrakk}\ \isanewline
\ \ \ \ {\isasymLongrightarrow}\ sats{\isacharparenleft}{\kern0pt}A{\isacharcomma}{\kern0pt}cond{\isacharunderscore}{\kern0pt}of{\isacharunderscore}{\kern0pt}fm{\isacharparenleft}{\kern0pt}x{\isacharcomma}{\kern0pt}y{\isacharparenright}{\kern0pt}{\isacharcomma}{\kern0pt}\ env{\isacharparenright}{\kern0pt}\ {\isasymlongleftrightarrow}\isanewline
\ \ \ \ \ \ \ \ is{\isacharunderscore}{\kern0pt}cond{\isacharunderscore}{\kern0pt}of{\isacharparenleft}{\kern0pt}{\isacharhash}{\kern0pt}{\isacharhash}{\kern0pt}A{\isacharcomma}{\kern0pt}\ nth{\isacharparenleft}{\kern0pt}x{\isacharcomma}{\kern0pt}env{\isacharparenright}{\kern0pt}{\isacharcomma}{\kern0pt}\ nth{\isacharparenleft}{\kern0pt}y{\isacharcomma}{\kern0pt}env{\isacharparenright}{\kern0pt}{\isacharparenright}{\kern0pt}{\isachardoublequoteclose}\isanewline
%
\isadelimproof
\ \ %
\endisadelimproof
%
\isatagproof
\isacommand{unfolding}\isamarkupfalse%
\ cond{\isacharunderscore}{\kern0pt}of{\isacharunderscore}{\kern0pt}fm{\isacharunderscore}{\kern0pt}def\ is{\isacharunderscore}{\kern0pt}cond{\isacharunderscore}{\kern0pt}of{\isacharunderscore}{\kern0pt}def\ \isacommand{using}\isamarkupfalse%
\ sats{\isacharunderscore}{\kern0pt}snd{\isacharunderscore}{\kern0pt}fm\ sats{\isacharunderscore}{\kern0pt}snd{\isadigit{2}}{\isacharunderscore}{\kern0pt}fm\isanewline
\ \ \ \ sats{\isacharunderscore}{\kern0pt}hcomp{\isacharunderscore}{\kern0pt}fm{\isacharbrackleft}{\kern0pt}of\ A\ {\isachardoublequoteopen}is{\isacharunderscore}{\kern0pt}snd{\isacharparenleft}{\kern0pt}{\isacharhash}{\kern0pt}{\isacharhash}{\kern0pt}A{\isacharparenright}{\kern0pt}{\isachardoublequoteclose}\ {\isacharunderscore}{\kern0pt}\ snd{\isacharunderscore}{\kern0pt}fm\ {\isachardoublequoteopen}is{\isacharunderscore}{\kern0pt}snd{\isacharunderscore}{\kern0pt}snd{\isacharparenleft}{\kern0pt}{\isacharhash}{\kern0pt}{\isacharhash}{\kern0pt}A{\isacharparenright}{\kern0pt}{\isachardoublequoteclose}{\isacharbrackright}{\kern0pt}\ \isacommand{by}\isamarkupfalse%
\ simp%
\endisatagproof
{\isafoldproof}%
%
\isadelimproof
\isanewline
%
\endisadelimproof
\isanewline
\isacommand{lemma}\isamarkupfalse%
\ is{\isacharunderscore}{\kern0pt}cond{\isacharunderscore}{\kern0pt}of{\isacharunderscore}{\kern0pt}iff{\isacharunderscore}{\kern0pt}sats{\isacharcolon}{\kern0pt}\isanewline
\ \ \isakeyword{assumes}\isanewline
\ \ \ \ {\isachardoublequoteopen}nth{\isacharparenleft}{\kern0pt}a{\isacharcomma}{\kern0pt}env{\isacharparenright}{\kern0pt}\ {\isacharequal}{\kern0pt}\ aa{\isachardoublequoteclose}\ {\isachardoublequoteopen}nth{\isacharparenleft}{\kern0pt}b{\isacharcomma}{\kern0pt}env{\isacharparenright}{\kern0pt}\ {\isacharequal}{\kern0pt}\ bb{\isachardoublequoteclose}\ {\isachardoublequoteopen}a{\isasymin}nat{\isachardoublequoteclose}\ {\isachardoublequoteopen}b{\isasymin}nat{\isachardoublequoteclose}\ {\isachardoublequoteopen}env\ {\isasymin}\ list{\isacharparenleft}{\kern0pt}A{\isacharparenright}{\kern0pt}{\isachardoublequoteclose}\isanewline
\ \ \isakeyword{shows}\isanewline
\ \ \ \ {\isachardoublequoteopen}is{\isacharunderscore}{\kern0pt}cond{\isacharunderscore}{\kern0pt}of{\isacharparenleft}{\kern0pt}{\isacharhash}{\kern0pt}{\isacharhash}{\kern0pt}A{\isacharcomma}{\kern0pt}aa{\isacharcomma}{\kern0pt}bb{\isacharparenright}{\kern0pt}\ \ {\isasymlongleftrightarrow}\ sats{\isacharparenleft}{\kern0pt}A{\isacharcomma}{\kern0pt}cond{\isacharunderscore}{\kern0pt}of{\isacharunderscore}{\kern0pt}fm{\isacharparenleft}{\kern0pt}a{\isacharcomma}{\kern0pt}b{\isacharparenright}{\kern0pt}{\isacharcomma}{\kern0pt}\ env{\isacharparenright}{\kern0pt}{\isachardoublequoteclose}\isanewline
%
\isadelimproof
\ \ %
\endisadelimproof
%
\isatagproof
\isacommand{using}\isamarkupfalse%
\ assms\isanewline
\ \ \isacommand{by}\isamarkupfalse%
\ {\isacharparenleft}{\kern0pt}simp\ add{\isacharcolon}{\kern0pt}sats{\isacharunderscore}{\kern0pt}cond{\isacharunderscore}{\kern0pt}of{\isacharunderscore}{\kern0pt}fm{\isacharparenright}{\kern0pt}%
\endisatagproof
{\isafoldproof}%
%
\isadelimproof
\isanewline
%
\endisadelimproof
\isanewline
\isacommand{lemma}\isamarkupfalse%
\ components{\isacharunderscore}{\kern0pt}type{\isacharbrackleft}{\kern0pt}TC{\isacharbrackright}{\kern0pt}\ {\isacharcolon}{\kern0pt}\isanewline
\ \ \isakeyword{assumes}\ {\isachardoublequoteopen}a{\isasymin}nat{\isachardoublequoteclose}\ {\isachardoublequoteopen}b{\isasymin}nat{\isachardoublequoteclose}\isanewline
\ \ \isakeyword{shows}\ \isanewline
\ \ \ \ {\isachardoublequoteopen}ftype{\isacharunderscore}{\kern0pt}fm{\isacharparenleft}{\kern0pt}a{\isacharcomma}{\kern0pt}b{\isacharparenright}{\kern0pt}{\isasymin}formula{\isachardoublequoteclose}\ \isanewline
\ \ \ \ {\isachardoublequoteopen}name{\isadigit{1}}{\isacharunderscore}{\kern0pt}fm{\isacharparenleft}{\kern0pt}a{\isacharcomma}{\kern0pt}b{\isacharparenright}{\kern0pt}{\isasymin}formula{\isachardoublequoteclose}\isanewline
\ \ \ \ {\isachardoublequoteopen}name{\isadigit{2}}{\isacharunderscore}{\kern0pt}fm{\isacharparenleft}{\kern0pt}a{\isacharcomma}{\kern0pt}b{\isacharparenright}{\kern0pt}{\isasymin}formula{\isachardoublequoteclose}\isanewline
\ \ \ \ {\isachardoublequoteopen}cond{\isacharunderscore}{\kern0pt}of{\isacharunderscore}{\kern0pt}fm{\isacharparenleft}{\kern0pt}a{\isacharcomma}{\kern0pt}b{\isacharparenright}{\kern0pt}{\isasymin}formula{\isachardoublequoteclose}\isanewline
%
\isadelimproof
\ \ %
\endisadelimproof
%
\isatagproof
\isacommand{using}\isamarkupfalse%
\ assms\isanewline
\ \ \isacommand{unfolding}\isamarkupfalse%
\ ftype{\isacharunderscore}{\kern0pt}fm{\isacharunderscore}{\kern0pt}def\ fst{\isacharunderscore}{\kern0pt}fm{\isacharunderscore}{\kern0pt}def\ snd{\isacharunderscore}{\kern0pt}fm{\isacharunderscore}{\kern0pt}def\ snd{\isacharunderscore}{\kern0pt}snd{\isacharunderscore}{\kern0pt}fm{\isacharunderscore}{\kern0pt}def\ name{\isadigit{1}}{\isacharunderscore}{\kern0pt}fm{\isacharunderscore}{\kern0pt}def\ name{\isadigit{2}}{\isacharunderscore}{\kern0pt}fm{\isacharunderscore}{\kern0pt}def\ \isanewline
\ \ \ \ cond{\isacharunderscore}{\kern0pt}of{\isacharunderscore}{\kern0pt}fm{\isacharunderscore}{\kern0pt}def\ hcomp{\isacharunderscore}{\kern0pt}fm{\isacharunderscore}{\kern0pt}def\isanewline
\ \ \isacommand{by}\isamarkupfalse%
\ simp{\isacharunderscore}{\kern0pt}all%
\endisatagproof
{\isafoldproof}%
%
\isadelimproof
\isanewline
%
\endisadelimproof
\isanewline
\isacommand{lemmas}\isamarkupfalse%
\ sats{\isacharunderscore}{\kern0pt}components{\isacharunderscore}{\kern0pt}fm{\isacharbrackleft}{\kern0pt}simp{\isacharbrackright}{\kern0pt}\ {\isacharequal}{\kern0pt}\ sats{\isacharunderscore}{\kern0pt}ftype{\isacharunderscore}{\kern0pt}fm\ sats{\isacharunderscore}{\kern0pt}name{\isadigit{1}}{\isacharunderscore}{\kern0pt}fm\ sats{\isacharunderscore}{\kern0pt}name{\isadigit{2}}{\isacharunderscore}{\kern0pt}fm\ sats{\isacharunderscore}{\kern0pt}cond{\isacharunderscore}{\kern0pt}of{\isacharunderscore}{\kern0pt}fm\isanewline
\isanewline
\isacommand{lemmas}\isamarkupfalse%
\ components{\isacharunderscore}{\kern0pt}iff{\isacharunderscore}{\kern0pt}sats\ {\isacharequal}{\kern0pt}\ is{\isacharunderscore}{\kern0pt}ftype{\isacharunderscore}{\kern0pt}iff{\isacharunderscore}{\kern0pt}sats\ is{\isacharunderscore}{\kern0pt}name{\isadigit{1}}{\isacharunderscore}{\kern0pt}iff{\isacharunderscore}{\kern0pt}sats\ is{\isacharunderscore}{\kern0pt}name{\isadigit{2}}{\isacharunderscore}{\kern0pt}iff{\isacharunderscore}{\kern0pt}sats\isanewline
\ \ is{\isacharunderscore}{\kern0pt}cond{\isacharunderscore}{\kern0pt}of{\isacharunderscore}{\kern0pt}iff{\isacharunderscore}{\kern0pt}sats\isanewline
\isanewline
\isacommand{lemmas}\isamarkupfalse%
\ components{\isacharunderscore}{\kern0pt}defs\ {\isacharequal}{\kern0pt}\ fst{\isacharunderscore}{\kern0pt}fm{\isacharunderscore}{\kern0pt}def\ ftype{\isacharunderscore}{\kern0pt}fm{\isacharunderscore}{\kern0pt}def\ snd{\isacharunderscore}{\kern0pt}fm{\isacharunderscore}{\kern0pt}def\ snd{\isacharunderscore}{\kern0pt}snd{\isacharunderscore}{\kern0pt}fm{\isacharunderscore}{\kern0pt}def\ hcomp{\isacharunderscore}{\kern0pt}fm{\isacharunderscore}{\kern0pt}def\isanewline
\ \ name{\isadigit{1}}{\isacharunderscore}{\kern0pt}fm{\isacharunderscore}{\kern0pt}def\ name{\isadigit{2}}{\isacharunderscore}{\kern0pt}fm{\isacharunderscore}{\kern0pt}def\ cond{\isacharunderscore}{\kern0pt}of{\isacharunderscore}{\kern0pt}fm{\isacharunderscore}{\kern0pt}def\isanewline
\isanewline
\isanewline
\isacommand{definition}\isamarkupfalse%
\isanewline
\ \ is{\isacharunderscore}{\kern0pt}eclose{\isacharunderscore}{\kern0pt}n\ {\isacharcolon}{\kern0pt}{\isacharcolon}{\kern0pt}\ {\isachardoublequoteopen}{\isacharbrackleft}{\kern0pt}i{\isasymRightarrow}o{\isacharcomma}{\kern0pt}{\isacharbrackleft}{\kern0pt}i{\isasymRightarrow}o{\isacharcomma}{\kern0pt}i{\isacharcomma}{\kern0pt}i{\isacharbrackright}{\kern0pt}{\isasymRightarrow}o{\isacharcomma}{\kern0pt}i{\isacharcomma}{\kern0pt}i{\isacharbrackright}{\kern0pt}\ {\isasymRightarrow}\ o{\isachardoublequoteclose}\ \isakeyword{where}\isanewline
\ \ {\isachardoublequoteopen}is{\isacharunderscore}{\kern0pt}eclose{\isacharunderscore}{\kern0pt}n{\isacharparenleft}{\kern0pt}N{\isacharcomma}{\kern0pt}is{\isacharunderscore}{\kern0pt}name{\isacharcomma}{\kern0pt}en{\isacharcomma}{\kern0pt}t{\isacharparenright}{\kern0pt}\ {\isasymequiv}\ \isanewline
\ \ \ \ \ \ \ \ {\isasymexists}n{\isadigit{1}}{\isacharbrackleft}{\kern0pt}N{\isacharbrackright}{\kern0pt}{\isachardot}{\kern0pt}{\isasymexists}s{\isadigit{1}}{\isacharbrackleft}{\kern0pt}N{\isacharbrackright}{\kern0pt}{\isachardot}{\kern0pt}\ is{\isacharunderscore}{\kern0pt}name{\isacharparenleft}{\kern0pt}N{\isacharcomma}{\kern0pt}t{\isacharcomma}{\kern0pt}n{\isadigit{1}}{\isacharparenright}{\kern0pt}\ {\isasymand}\ is{\isacharunderscore}{\kern0pt}singleton{\isacharparenleft}{\kern0pt}N{\isacharcomma}{\kern0pt}n{\isadigit{1}}{\isacharcomma}{\kern0pt}s{\isadigit{1}}{\isacharparenright}{\kern0pt}\ {\isasymand}\ is{\isacharunderscore}{\kern0pt}eclose{\isacharparenleft}{\kern0pt}N{\isacharcomma}{\kern0pt}s{\isadigit{1}}{\isacharcomma}{\kern0pt}en{\isacharparenright}{\kern0pt}{\isachardoublequoteclose}\isanewline
\isanewline
\isacommand{definition}\isamarkupfalse%
\ \isanewline
\ \ eclose{\isacharunderscore}{\kern0pt}n{\isadigit{1}}{\isacharunderscore}{\kern0pt}fm\ {\isacharcolon}{\kern0pt}{\isacharcolon}{\kern0pt}\ {\isachardoublequoteopen}{\isacharbrackleft}{\kern0pt}i{\isacharcomma}{\kern0pt}i{\isacharbrackright}{\kern0pt}\ {\isasymRightarrow}\ i{\isachardoublequoteclose}\ \isakeyword{where}\isanewline
\ \ {\isachardoublequoteopen}eclose{\isacharunderscore}{\kern0pt}n{\isadigit{1}}{\isacharunderscore}{\kern0pt}fm{\isacharparenleft}{\kern0pt}m{\isacharcomma}{\kern0pt}t{\isacharparenright}{\kern0pt}\ {\isasymequiv}\ Exists{\isacharparenleft}{\kern0pt}Exists{\isacharparenleft}{\kern0pt}And{\isacharparenleft}{\kern0pt}And{\isacharparenleft}{\kern0pt}name{\isadigit{1}}{\isacharunderscore}{\kern0pt}fm{\isacharparenleft}{\kern0pt}t{\isacharhash}{\kern0pt}{\isacharplus}{\kern0pt}{\isadigit{2}}{\isacharcomma}{\kern0pt}{\isadigit{0}}{\isacharparenright}{\kern0pt}{\isacharcomma}{\kern0pt}singleton{\isacharunderscore}{\kern0pt}fm{\isacharparenleft}{\kern0pt}{\isadigit{0}}{\isacharcomma}{\kern0pt}{\isadigit{1}}{\isacharparenright}{\kern0pt}{\isacharparenright}{\kern0pt}{\isacharcomma}{\kern0pt}\isanewline
\ \ \ \ \ \ \ \ \ \ \ \ \ \ \ \ \ \ \ \ \ \ \ \ \ \ \ \ \ \ \ \ \ \ \ \ \ \ \ is{\isacharunderscore}{\kern0pt}eclose{\isacharunderscore}{\kern0pt}fm{\isacharparenleft}{\kern0pt}{\isadigit{1}}{\isacharcomma}{\kern0pt}m{\isacharhash}{\kern0pt}{\isacharplus}{\kern0pt}{\isadigit{2}}{\isacharparenright}{\kern0pt}{\isacharparenright}{\kern0pt}{\isacharparenright}{\kern0pt}{\isacharparenright}{\kern0pt}{\isachardoublequoteclose}\isanewline
\isanewline
\isacommand{definition}\isamarkupfalse%
\ \isanewline
\ \ eclose{\isacharunderscore}{\kern0pt}n{\isadigit{2}}{\isacharunderscore}{\kern0pt}fm\ {\isacharcolon}{\kern0pt}{\isacharcolon}{\kern0pt}\ {\isachardoublequoteopen}{\isacharbrackleft}{\kern0pt}i{\isacharcomma}{\kern0pt}i{\isacharbrackright}{\kern0pt}\ {\isasymRightarrow}\ i{\isachardoublequoteclose}\ \isakeyword{where}\isanewline
\ \ {\isachardoublequoteopen}eclose{\isacharunderscore}{\kern0pt}n{\isadigit{2}}{\isacharunderscore}{\kern0pt}fm{\isacharparenleft}{\kern0pt}m{\isacharcomma}{\kern0pt}t{\isacharparenright}{\kern0pt}\ {\isasymequiv}\ Exists{\isacharparenleft}{\kern0pt}Exists{\isacharparenleft}{\kern0pt}And{\isacharparenleft}{\kern0pt}And{\isacharparenleft}{\kern0pt}name{\isadigit{2}}{\isacharunderscore}{\kern0pt}fm{\isacharparenleft}{\kern0pt}t{\isacharhash}{\kern0pt}{\isacharplus}{\kern0pt}{\isadigit{2}}{\isacharcomma}{\kern0pt}{\isadigit{0}}{\isacharparenright}{\kern0pt}{\isacharcomma}{\kern0pt}singleton{\isacharunderscore}{\kern0pt}fm{\isacharparenleft}{\kern0pt}{\isadigit{0}}{\isacharcomma}{\kern0pt}{\isadigit{1}}{\isacharparenright}{\kern0pt}{\isacharparenright}{\kern0pt}{\isacharcomma}{\kern0pt}\isanewline
\ \ \ \ \ \ \ \ \ \ \ \ \ \ \ \ \ \ \ \ \ \ \ \ \ \ \ \ \ \ \ \ \ \ \ \ \ \ \ is{\isacharunderscore}{\kern0pt}eclose{\isacharunderscore}{\kern0pt}fm{\isacharparenleft}{\kern0pt}{\isadigit{1}}{\isacharcomma}{\kern0pt}m{\isacharhash}{\kern0pt}{\isacharplus}{\kern0pt}{\isadigit{2}}{\isacharparenright}{\kern0pt}{\isacharparenright}{\kern0pt}{\isacharparenright}{\kern0pt}{\isacharparenright}{\kern0pt}{\isachardoublequoteclose}\isanewline
\isanewline
\isacommand{definition}\isamarkupfalse%
\isanewline
\ \ is{\isacharunderscore}{\kern0pt}ecloseN\ {\isacharcolon}{\kern0pt}{\isacharcolon}{\kern0pt}\ {\isachardoublequoteopen}{\isacharbrackleft}{\kern0pt}i{\isasymRightarrow}o{\isacharcomma}{\kern0pt}i{\isacharcomma}{\kern0pt}i{\isacharbrackright}{\kern0pt}\ {\isasymRightarrow}\ o{\isachardoublequoteclose}\ \isakeyword{where}\isanewline
\ \ {\isachardoublequoteopen}is{\isacharunderscore}{\kern0pt}ecloseN{\isacharparenleft}{\kern0pt}N{\isacharcomma}{\kern0pt}en{\isacharcomma}{\kern0pt}t{\isacharparenright}{\kern0pt}\ {\isasymequiv}\ {\isasymexists}en{\isadigit{1}}{\isacharbrackleft}{\kern0pt}N{\isacharbrackright}{\kern0pt}{\isachardot}{\kern0pt}{\isasymexists}en{\isadigit{2}}{\isacharbrackleft}{\kern0pt}N{\isacharbrackright}{\kern0pt}{\isachardot}{\kern0pt}\isanewline
\ \ \ \ \ \ \ \ \ \ \ \ \ \ \ \ is{\isacharunderscore}{\kern0pt}eclose{\isacharunderscore}{\kern0pt}n{\isacharparenleft}{\kern0pt}N{\isacharcomma}{\kern0pt}is{\isacharunderscore}{\kern0pt}name{\isadigit{1}}{\isacharcomma}{\kern0pt}en{\isadigit{1}}{\isacharcomma}{\kern0pt}t{\isacharparenright}{\kern0pt}\ {\isasymand}\ is{\isacharunderscore}{\kern0pt}eclose{\isacharunderscore}{\kern0pt}n{\isacharparenleft}{\kern0pt}N{\isacharcomma}{\kern0pt}is{\isacharunderscore}{\kern0pt}name{\isadigit{2}}{\isacharcomma}{\kern0pt}en{\isadigit{2}}{\isacharcomma}{\kern0pt}t{\isacharparenright}{\kern0pt}{\isasymand}\isanewline
\ \ \ \ \ \ \ \ \ \ \ \ \ \ \ \ union{\isacharparenleft}{\kern0pt}N{\isacharcomma}{\kern0pt}en{\isadigit{1}}{\isacharcomma}{\kern0pt}en{\isadigit{2}}{\isacharcomma}{\kern0pt}en{\isacharparenright}{\kern0pt}{\isachardoublequoteclose}\isanewline
\isanewline
\isacommand{definition}\isamarkupfalse%
\ \isanewline
\ \ ecloseN{\isacharunderscore}{\kern0pt}fm\ {\isacharcolon}{\kern0pt}{\isacharcolon}{\kern0pt}\ {\isachardoublequoteopen}{\isacharbrackleft}{\kern0pt}i{\isacharcomma}{\kern0pt}i{\isacharbrackright}{\kern0pt}\ {\isasymRightarrow}\ i{\isachardoublequoteclose}\ \isakeyword{where}\isanewline
\ \ {\isachardoublequoteopen}ecloseN{\isacharunderscore}{\kern0pt}fm{\isacharparenleft}{\kern0pt}en{\isacharcomma}{\kern0pt}t{\isacharparenright}{\kern0pt}\ {\isasymequiv}\ Exists{\isacharparenleft}{\kern0pt}Exists{\isacharparenleft}{\kern0pt}And{\isacharparenleft}{\kern0pt}eclose{\isacharunderscore}{\kern0pt}n{\isadigit{1}}{\isacharunderscore}{\kern0pt}fm{\isacharparenleft}{\kern0pt}{\isadigit{1}}{\isacharcomma}{\kern0pt}t{\isacharhash}{\kern0pt}{\isacharplus}{\kern0pt}{\isadigit{2}}{\isacharparenright}{\kern0pt}{\isacharcomma}{\kern0pt}\isanewline
\ \ \ \ \ \ \ \ \ \ \ \ \ \ \ \ \ \ \ \ \ \ \ \ \ \ \ \ And{\isacharparenleft}{\kern0pt}eclose{\isacharunderscore}{\kern0pt}n{\isadigit{2}}{\isacharunderscore}{\kern0pt}fm{\isacharparenleft}{\kern0pt}{\isadigit{0}}{\isacharcomma}{\kern0pt}t{\isacharhash}{\kern0pt}{\isacharplus}{\kern0pt}{\isadigit{2}}{\isacharparenright}{\kern0pt}{\isacharcomma}{\kern0pt}union{\isacharunderscore}{\kern0pt}fm{\isacharparenleft}{\kern0pt}{\isadigit{1}}{\isacharcomma}{\kern0pt}{\isadigit{0}}{\isacharcomma}{\kern0pt}en{\isacharhash}{\kern0pt}{\isacharplus}{\kern0pt}{\isadigit{2}}{\isacharparenright}{\kern0pt}{\isacharparenright}{\kern0pt}{\isacharparenright}{\kern0pt}{\isacharparenright}{\kern0pt}{\isacharparenright}{\kern0pt}{\isachardoublequoteclose}\isanewline
\isacommand{lemma}\isamarkupfalse%
\ ecloseN{\isacharunderscore}{\kern0pt}fm{\isacharunderscore}{\kern0pt}type\ {\isacharbrackleft}{\kern0pt}TC{\isacharbrackright}{\kern0pt}\ {\isacharcolon}{\kern0pt}\isanewline
\ \ {\isachardoublequoteopen}{\isasymlbrakk}\ en\ {\isasymin}\ nat\ {\isacharsemicolon}{\kern0pt}\ t\ {\isasymin}\ nat\ {\isasymrbrakk}\ {\isasymLongrightarrow}\ ecloseN{\isacharunderscore}{\kern0pt}fm{\isacharparenleft}{\kern0pt}en{\isacharcomma}{\kern0pt}t{\isacharparenright}{\kern0pt}\ {\isasymin}\ formula{\isachardoublequoteclose}\isanewline
%
\isadelimproof
\ \ %
\endisadelimproof
%
\isatagproof
\isacommand{unfolding}\isamarkupfalse%
\ ecloseN{\isacharunderscore}{\kern0pt}fm{\isacharunderscore}{\kern0pt}def\ eclose{\isacharunderscore}{\kern0pt}n{\isadigit{1}}{\isacharunderscore}{\kern0pt}fm{\isacharunderscore}{\kern0pt}def\ eclose{\isacharunderscore}{\kern0pt}n{\isadigit{2}}{\isacharunderscore}{\kern0pt}fm{\isacharunderscore}{\kern0pt}def\ \isacommand{by}\isamarkupfalse%
\ simp%
\endisatagproof
{\isafoldproof}%
%
\isadelimproof
\isanewline
%
\endisadelimproof
\isanewline
\isacommand{lemma}\isamarkupfalse%
\ sats{\isacharunderscore}{\kern0pt}ecloseN{\isacharunderscore}{\kern0pt}fm\ {\isacharbrackleft}{\kern0pt}simp{\isacharbrackright}{\kern0pt}{\isacharcolon}{\kern0pt}\isanewline
\ \ {\isachardoublequoteopen}{\isasymlbrakk}\ en\ {\isasymin}\ nat{\isacharsemicolon}{\kern0pt}\ t\ {\isasymin}\ nat\ {\isacharsemicolon}{\kern0pt}\ env\ {\isasymin}\ list{\isacharparenleft}{\kern0pt}A{\isacharparenright}{\kern0pt}\ {\isasymrbrakk}\isanewline
\ \ \ \ {\isasymLongrightarrow}\ sats{\isacharparenleft}{\kern0pt}A{\isacharcomma}{\kern0pt}\ ecloseN{\isacharunderscore}{\kern0pt}fm{\isacharparenleft}{\kern0pt}en{\isacharcomma}{\kern0pt}t{\isacharparenright}{\kern0pt}{\isacharcomma}{\kern0pt}\ env{\isacharparenright}{\kern0pt}\ {\isasymlongleftrightarrow}\ is{\isacharunderscore}{\kern0pt}ecloseN{\isacharparenleft}{\kern0pt}{\isacharhash}{\kern0pt}{\isacharhash}{\kern0pt}A{\isacharcomma}{\kern0pt}nth{\isacharparenleft}{\kern0pt}en{\isacharcomma}{\kern0pt}env{\isacharparenright}{\kern0pt}{\isacharcomma}{\kern0pt}nth{\isacharparenleft}{\kern0pt}t{\isacharcomma}{\kern0pt}env{\isacharparenright}{\kern0pt}{\isacharparenright}{\kern0pt}{\isachardoublequoteclose}\isanewline
%
\isadelimproof
\ \ %
\endisadelimproof
%
\isatagproof
\isacommand{unfolding}\isamarkupfalse%
\ ecloseN{\isacharunderscore}{\kern0pt}fm{\isacharunderscore}{\kern0pt}def\ is{\isacharunderscore}{\kern0pt}ecloseN{\isacharunderscore}{\kern0pt}def\ eclose{\isacharunderscore}{\kern0pt}n{\isadigit{1}}{\isacharunderscore}{\kern0pt}fm{\isacharunderscore}{\kern0pt}def\ eclose{\isacharunderscore}{\kern0pt}n{\isadigit{2}}{\isacharunderscore}{\kern0pt}fm{\isacharunderscore}{\kern0pt}def\ is{\isacharunderscore}{\kern0pt}eclose{\isacharunderscore}{\kern0pt}n{\isacharunderscore}{\kern0pt}def\isanewline
\ \ \isacommand{using}\isamarkupfalse%
\ \ nth{\isacharunderscore}{\kern0pt}{\isadigit{0}}\ nth{\isacharunderscore}{\kern0pt}ConsI\ sats{\isacharunderscore}{\kern0pt}name{\isadigit{1}}{\isacharunderscore}{\kern0pt}fm\ sats{\isacharunderscore}{\kern0pt}name{\isadigit{2}}{\isacharunderscore}{\kern0pt}fm\ \isanewline
\ \ \ \ is{\isacharunderscore}{\kern0pt}singleton{\isacharunderscore}{\kern0pt}iff{\isacharunderscore}{\kern0pt}sats{\isacharbrackleft}{\kern0pt}symmetric{\isacharbrackright}{\kern0pt}\isanewline
\ \ \isacommand{by}\isamarkupfalse%
\ auto%
\endisatagproof
{\isafoldproof}%
%
\isadelimproof
\isanewline
%
\endisadelimproof
\isanewline
\isanewline
\isacommand{definition}\isamarkupfalse%
\isanewline
\ \ frecR\ {\isacharcolon}{\kern0pt}{\isacharcolon}{\kern0pt}\ {\isachardoublequoteopen}i\ {\isasymRightarrow}\ i\ {\isasymRightarrow}\ o{\isachardoublequoteclose}\ \isakeyword{where}\isanewline
\ \ {\isachardoublequoteopen}frecR{\isacharparenleft}{\kern0pt}x{\isacharcomma}{\kern0pt}y{\isacharparenright}{\kern0pt}\ {\isasymequiv}\isanewline
\ \ \ \ {\isacharparenleft}{\kern0pt}ftype{\isacharparenleft}{\kern0pt}x{\isacharparenright}{\kern0pt}\ {\isacharequal}{\kern0pt}\ {\isadigit{1}}\ {\isasymand}\ ftype{\isacharparenleft}{\kern0pt}y{\isacharparenright}{\kern0pt}\ {\isacharequal}{\kern0pt}\ {\isadigit{0}}\ \isanewline
\ \ \ \ \ \ {\isasymand}\ {\isacharparenleft}{\kern0pt}name{\isadigit{1}}{\isacharparenleft}{\kern0pt}x{\isacharparenright}{\kern0pt}\ {\isasymin}\ domain{\isacharparenleft}{\kern0pt}name{\isadigit{1}}{\isacharparenleft}{\kern0pt}y{\isacharparenright}{\kern0pt}{\isacharparenright}{\kern0pt}\ {\isasymunion}\ domain{\isacharparenleft}{\kern0pt}name{\isadigit{2}}{\isacharparenleft}{\kern0pt}y{\isacharparenright}{\kern0pt}{\isacharparenright}{\kern0pt}\ {\isasymand}\ {\isacharparenleft}{\kern0pt}name{\isadigit{2}}{\isacharparenleft}{\kern0pt}x{\isacharparenright}{\kern0pt}\ {\isacharequal}{\kern0pt}\ name{\isadigit{1}}{\isacharparenleft}{\kern0pt}y{\isacharparenright}{\kern0pt}\ {\isasymor}\ name{\isadigit{2}}{\isacharparenleft}{\kern0pt}x{\isacharparenright}{\kern0pt}\ {\isacharequal}{\kern0pt}\ name{\isadigit{2}}{\isacharparenleft}{\kern0pt}y{\isacharparenright}{\kern0pt}{\isacharparenright}{\kern0pt}{\isacharparenright}{\kern0pt}{\isacharparenright}{\kern0pt}\isanewline
\ \ \ {\isasymor}\ {\isacharparenleft}{\kern0pt}ftype{\isacharparenleft}{\kern0pt}x{\isacharparenright}{\kern0pt}\ {\isacharequal}{\kern0pt}\ {\isadigit{0}}\ {\isasymand}\ ftype{\isacharparenleft}{\kern0pt}y{\isacharparenright}{\kern0pt}\ {\isacharequal}{\kern0pt}\ \ {\isadigit{1}}\ {\isasymand}\ name{\isadigit{1}}{\isacharparenleft}{\kern0pt}x{\isacharparenright}{\kern0pt}\ {\isacharequal}{\kern0pt}\ name{\isadigit{1}}{\isacharparenleft}{\kern0pt}y{\isacharparenright}{\kern0pt}\ {\isasymand}\ name{\isadigit{2}}{\isacharparenleft}{\kern0pt}x{\isacharparenright}{\kern0pt}\ {\isasymin}\ domain{\isacharparenleft}{\kern0pt}name{\isadigit{2}}{\isacharparenleft}{\kern0pt}y{\isacharparenright}{\kern0pt}{\isacharparenright}{\kern0pt}{\isacharparenright}{\kern0pt}{\isachardoublequoteclose}\isanewline
\isanewline
\isacommand{lemma}\isamarkupfalse%
\ frecR{\isacharunderscore}{\kern0pt}ftypeD\ {\isacharcolon}{\kern0pt}\isanewline
\ \ \isakeyword{assumes}\ {\isachardoublequoteopen}frecR{\isacharparenleft}{\kern0pt}x{\isacharcomma}{\kern0pt}y{\isacharparenright}{\kern0pt}{\isachardoublequoteclose}\isanewline
\ \ \isakeyword{shows}\ {\isachardoublequoteopen}{\isacharparenleft}{\kern0pt}ftype{\isacharparenleft}{\kern0pt}x{\isacharparenright}{\kern0pt}\ {\isacharequal}{\kern0pt}\ {\isadigit{0}}\ {\isasymand}\ ftype{\isacharparenleft}{\kern0pt}y{\isacharparenright}{\kern0pt}\ {\isacharequal}{\kern0pt}\ {\isadigit{1}}{\isacharparenright}{\kern0pt}\ {\isasymor}\ {\isacharparenleft}{\kern0pt}ftype{\isacharparenleft}{\kern0pt}x{\isacharparenright}{\kern0pt}\ {\isacharequal}{\kern0pt}\ {\isadigit{1}}\ {\isasymand}\ ftype{\isacharparenleft}{\kern0pt}y{\isacharparenright}{\kern0pt}\ {\isacharequal}{\kern0pt}\ {\isadigit{0}}{\isacharparenright}{\kern0pt}{\isachardoublequoteclose}\isanewline
%
\isadelimproof
\ \ %
\endisadelimproof
%
\isatagproof
\isacommand{using}\isamarkupfalse%
\ assms\ \isacommand{unfolding}\isamarkupfalse%
\ frecR{\isacharunderscore}{\kern0pt}def\ \isacommand{by}\isamarkupfalse%
\ auto%
\endisatagproof
{\isafoldproof}%
%
\isadelimproof
\isanewline
%
\endisadelimproof
\isanewline
\isacommand{lemma}\isamarkupfalse%
\ frecRI{\isadigit{1}}{\isacharcolon}{\kern0pt}\ {\isachardoublequoteopen}s\ {\isasymin}\ domain{\isacharparenleft}{\kern0pt}n{\isadigit{1}}{\isacharparenright}{\kern0pt}\ {\isasymor}\ s\ {\isasymin}\ domain{\isacharparenleft}{\kern0pt}n{\isadigit{2}}{\isacharparenright}{\kern0pt}\ {\isasymLongrightarrow}\ frecR{\isacharparenleft}{\kern0pt}{\isasymlangle}{\isadigit{1}}{\isacharcomma}{\kern0pt}\ s{\isacharcomma}{\kern0pt}\ n{\isadigit{1}}{\isacharcomma}{\kern0pt}\ q{\isasymrangle}{\isacharcomma}{\kern0pt}\ {\isasymlangle}{\isadigit{0}}{\isacharcomma}{\kern0pt}\ n{\isadigit{1}}{\isacharcomma}{\kern0pt}\ n{\isadigit{2}}{\isacharcomma}{\kern0pt}\ q{\isacharprime}{\kern0pt}{\isasymrangle}{\isacharparenright}{\kern0pt}{\isachardoublequoteclose}\isanewline
%
\isadelimproof
\ \ %
\endisadelimproof
%
\isatagproof
\isacommand{unfolding}\isamarkupfalse%
\ frecR{\isacharunderscore}{\kern0pt}def\ \isacommand{by}\isamarkupfalse%
\ {\isacharparenleft}{\kern0pt}simp\ add{\isacharcolon}{\kern0pt}components{\isacharunderscore}{\kern0pt}simp{\isacharparenright}{\kern0pt}%
\endisatagproof
{\isafoldproof}%
%
\isadelimproof
\isanewline
%
\endisadelimproof
\isanewline
\isacommand{lemma}\isamarkupfalse%
\ frecRI{\isadigit{1}}{\isacharprime}{\kern0pt}{\isacharcolon}{\kern0pt}\ {\isachardoublequoteopen}s\ {\isasymin}\ domain{\isacharparenleft}{\kern0pt}n{\isadigit{1}}{\isacharparenright}{\kern0pt}\ {\isasymunion}\ domain{\isacharparenleft}{\kern0pt}n{\isadigit{2}}{\isacharparenright}{\kern0pt}\ {\isasymLongrightarrow}\ frecR{\isacharparenleft}{\kern0pt}{\isasymlangle}{\isadigit{1}}{\isacharcomma}{\kern0pt}\ s{\isacharcomma}{\kern0pt}\ n{\isadigit{1}}{\isacharcomma}{\kern0pt}\ q{\isasymrangle}{\isacharcomma}{\kern0pt}\ {\isasymlangle}{\isadigit{0}}{\isacharcomma}{\kern0pt}\ n{\isadigit{1}}{\isacharcomma}{\kern0pt}\ n{\isadigit{2}}{\isacharcomma}{\kern0pt}\ q{\isacharprime}{\kern0pt}{\isasymrangle}{\isacharparenright}{\kern0pt}{\isachardoublequoteclose}\isanewline
%
\isadelimproof
\ \ %
\endisadelimproof
%
\isatagproof
\isacommand{unfolding}\isamarkupfalse%
\ frecR{\isacharunderscore}{\kern0pt}def\ \isacommand{by}\isamarkupfalse%
\ {\isacharparenleft}{\kern0pt}simp\ add{\isacharcolon}{\kern0pt}components{\isacharunderscore}{\kern0pt}simp{\isacharparenright}{\kern0pt}%
\endisatagproof
{\isafoldproof}%
%
\isadelimproof
\isanewline
%
\endisadelimproof
\isanewline
\isacommand{lemma}\isamarkupfalse%
\ frecRI{\isadigit{2}}{\isacharcolon}{\kern0pt}\ {\isachardoublequoteopen}s\ {\isasymin}\ domain{\isacharparenleft}{\kern0pt}n{\isadigit{1}}{\isacharparenright}{\kern0pt}\ {\isasymor}\ s\ {\isasymin}\ domain{\isacharparenleft}{\kern0pt}n{\isadigit{2}}{\isacharparenright}{\kern0pt}\ {\isasymLongrightarrow}\ frecR{\isacharparenleft}{\kern0pt}{\isasymlangle}{\isadigit{1}}{\isacharcomma}{\kern0pt}\ s{\isacharcomma}{\kern0pt}\ n{\isadigit{2}}{\isacharcomma}{\kern0pt}\ q{\isasymrangle}{\isacharcomma}{\kern0pt}\ {\isasymlangle}{\isadigit{0}}{\isacharcomma}{\kern0pt}\ n{\isadigit{1}}{\isacharcomma}{\kern0pt}\ n{\isadigit{2}}{\isacharcomma}{\kern0pt}\ q{\isacharprime}{\kern0pt}{\isasymrangle}{\isacharparenright}{\kern0pt}{\isachardoublequoteclose}\isanewline
%
\isadelimproof
\ \ %
\endisadelimproof
%
\isatagproof
\isacommand{unfolding}\isamarkupfalse%
\ frecR{\isacharunderscore}{\kern0pt}def\ \isacommand{by}\isamarkupfalse%
\ {\isacharparenleft}{\kern0pt}simp\ add{\isacharcolon}{\kern0pt}components{\isacharunderscore}{\kern0pt}simp{\isacharparenright}{\kern0pt}%
\endisatagproof
{\isafoldproof}%
%
\isadelimproof
\isanewline
%
\endisadelimproof
\isanewline
\isacommand{lemma}\isamarkupfalse%
\ frecRI{\isadigit{2}}{\isacharprime}{\kern0pt}{\isacharcolon}{\kern0pt}\ {\isachardoublequoteopen}s\ {\isasymin}\ domain{\isacharparenleft}{\kern0pt}n{\isadigit{1}}{\isacharparenright}{\kern0pt}\ {\isasymunion}\ domain{\isacharparenleft}{\kern0pt}n{\isadigit{2}}{\isacharparenright}{\kern0pt}\ {\isasymLongrightarrow}\ frecR{\isacharparenleft}{\kern0pt}{\isasymlangle}{\isadigit{1}}{\isacharcomma}{\kern0pt}\ s{\isacharcomma}{\kern0pt}\ n{\isadigit{2}}{\isacharcomma}{\kern0pt}\ q{\isasymrangle}{\isacharcomma}{\kern0pt}\ {\isasymlangle}{\isadigit{0}}{\isacharcomma}{\kern0pt}\ n{\isadigit{1}}{\isacharcomma}{\kern0pt}\ n{\isadigit{2}}{\isacharcomma}{\kern0pt}\ q{\isacharprime}{\kern0pt}{\isasymrangle}{\isacharparenright}{\kern0pt}{\isachardoublequoteclose}\isanewline
%
\isadelimproof
\ \ %
\endisadelimproof
%
\isatagproof
\isacommand{unfolding}\isamarkupfalse%
\ frecR{\isacharunderscore}{\kern0pt}def\ \isacommand{by}\isamarkupfalse%
\ {\isacharparenleft}{\kern0pt}simp\ add{\isacharcolon}{\kern0pt}components{\isacharunderscore}{\kern0pt}simp{\isacharparenright}{\kern0pt}%
\endisatagproof
{\isafoldproof}%
%
\isadelimproof
\isanewline
%
\endisadelimproof
\isanewline
\isanewline
\isacommand{lemma}\isamarkupfalse%
\ frecRI{\isadigit{3}}{\isacharcolon}{\kern0pt}\ {\isachardoublequoteopen}{\isasymlangle}s{\isacharcomma}{\kern0pt}\ r{\isasymrangle}\ {\isasymin}\ n{\isadigit{2}}\ {\isasymLongrightarrow}\ frecR{\isacharparenleft}{\kern0pt}{\isasymlangle}{\isadigit{0}}{\isacharcomma}{\kern0pt}\ n{\isadigit{1}}{\isacharcomma}{\kern0pt}\ s{\isacharcomma}{\kern0pt}\ q{\isasymrangle}{\isacharcomma}{\kern0pt}\ {\isasymlangle}{\isadigit{1}}{\isacharcomma}{\kern0pt}\ n{\isadigit{1}}{\isacharcomma}{\kern0pt}\ n{\isadigit{2}}{\isacharcomma}{\kern0pt}\ q{\isacharprime}{\kern0pt}{\isasymrangle}{\isacharparenright}{\kern0pt}{\isachardoublequoteclose}\isanewline
%
\isadelimproof
\ \ %
\endisadelimproof
%
\isatagproof
\isacommand{unfolding}\isamarkupfalse%
\ frecR{\isacharunderscore}{\kern0pt}def\ \isacommand{by}\isamarkupfalse%
\ {\isacharparenleft}{\kern0pt}auto\ simp\ add{\isacharcolon}{\kern0pt}components{\isacharunderscore}{\kern0pt}simp{\isacharparenright}{\kern0pt}%
\endisatagproof
{\isafoldproof}%
%
\isadelimproof
\isanewline
%
\endisadelimproof
\isanewline
\isacommand{lemma}\isamarkupfalse%
\ frecRI{\isadigit{3}}{\isacharprime}{\kern0pt}{\isacharcolon}{\kern0pt}\ {\isachardoublequoteopen}s\ {\isasymin}\ domain{\isacharparenleft}{\kern0pt}n{\isadigit{2}}{\isacharparenright}{\kern0pt}\ {\isasymLongrightarrow}\ frecR{\isacharparenleft}{\kern0pt}{\isasymlangle}{\isadigit{0}}{\isacharcomma}{\kern0pt}\ n{\isadigit{1}}{\isacharcomma}{\kern0pt}\ s{\isacharcomma}{\kern0pt}\ q{\isasymrangle}{\isacharcomma}{\kern0pt}\ {\isasymlangle}{\isadigit{1}}{\isacharcomma}{\kern0pt}\ n{\isadigit{1}}{\isacharcomma}{\kern0pt}\ n{\isadigit{2}}{\isacharcomma}{\kern0pt}\ q{\isacharprime}{\kern0pt}{\isasymrangle}{\isacharparenright}{\kern0pt}{\isachardoublequoteclose}\isanewline
%
\isadelimproof
\ \ %
\endisadelimproof
%
\isatagproof
\isacommand{unfolding}\isamarkupfalse%
\ frecR{\isacharunderscore}{\kern0pt}def\ \isacommand{by}\isamarkupfalse%
\ {\isacharparenleft}{\kern0pt}auto\ simp\ add{\isacharcolon}{\kern0pt}components{\isacharunderscore}{\kern0pt}simp{\isacharparenright}{\kern0pt}%
\endisatagproof
{\isafoldproof}%
%
\isadelimproof
\isanewline
%
\endisadelimproof
\isanewline
\isacommand{lemma}\isamarkupfalse%
\ frecR{\isacharunderscore}{\kern0pt}iff\ {\isacharcolon}{\kern0pt}\isanewline
\ \ {\isachardoublequoteopen}frecR{\isacharparenleft}{\kern0pt}x{\isacharcomma}{\kern0pt}y{\isacharparenright}{\kern0pt}\ {\isasymlongleftrightarrow}\isanewline
\ \ \ \ {\isacharparenleft}{\kern0pt}ftype{\isacharparenleft}{\kern0pt}x{\isacharparenright}{\kern0pt}\ {\isacharequal}{\kern0pt}\ {\isadigit{1}}\ {\isasymand}\ ftype{\isacharparenleft}{\kern0pt}y{\isacharparenright}{\kern0pt}\ {\isacharequal}{\kern0pt}\ {\isadigit{0}}\ \isanewline
\ \ \ \ \ \ {\isasymand}\ {\isacharparenleft}{\kern0pt}name{\isadigit{1}}{\isacharparenleft}{\kern0pt}x{\isacharparenright}{\kern0pt}\ {\isasymin}\ domain{\isacharparenleft}{\kern0pt}name{\isadigit{1}}{\isacharparenleft}{\kern0pt}y{\isacharparenright}{\kern0pt}{\isacharparenright}{\kern0pt}\ {\isasymunion}\ domain{\isacharparenleft}{\kern0pt}name{\isadigit{2}}{\isacharparenleft}{\kern0pt}y{\isacharparenright}{\kern0pt}{\isacharparenright}{\kern0pt}\ {\isasymand}\ {\isacharparenleft}{\kern0pt}name{\isadigit{2}}{\isacharparenleft}{\kern0pt}x{\isacharparenright}{\kern0pt}\ {\isacharequal}{\kern0pt}\ name{\isadigit{1}}{\isacharparenleft}{\kern0pt}y{\isacharparenright}{\kern0pt}\ {\isasymor}\ name{\isadigit{2}}{\isacharparenleft}{\kern0pt}x{\isacharparenright}{\kern0pt}\ {\isacharequal}{\kern0pt}\ name{\isadigit{2}}{\isacharparenleft}{\kern0pt}y{\isacharparenright}{\kern0pt}{\isacharparenright}{\kern0pt}{\isacharparenright}{\kern0pt}{\isacharparenright}{\kern0pt}\isanewline
\ \ \ {\isasymor}\ {\isacharparenleft}{\kern0pt}ftype{\isacharparenleft}{\kern0pt}x{\isacharparenright}{\kern0pt}\ {\isacharequal}{\kern0pt}\ {\isadigit{0}}\ {\isasymand}\ ftype{\isacharparenleft}{\kern0pt}y{\isacharparenright}{\kern0pt}\ {\isacharequal}{\kern0pt}\ \ {\isadigit{1}}\ {\isasymand}\ name{\isadigit{1}}{\isacharparenleft}{\kern0pt}x{\isacharparenright}{\kern0pt}\ {\isacharequal}{\kern0pt}\ name{\isadigit{1}}{\isacharparenleft}{\kern0pt}y{\isacharparenright}{\kern0pt}\ {\isasymand}\ name{\isadigit{2}}{\isacharparenleft}{\kern0pt}x{\isacharparenright}{\kern0pt}\ {\isasymin}\ domain{\isacharparenleft}{\kern0pt}name{\isadigit{2}}{\isacharparenleft}{\kern0pt}y{\isacharparenright}{\kern0pt}{\isacharparenright}{\kern0pt}{\isacharparenright}{\kern0pt}{\isachardoublequoteclose}\isanewline
%
\isadelimproof
\ \ %
\endisadelimproof
%
\isatagproof
\isacommand{unfolding}\isamarkupfalse%
\ frecR{\isacharunderscore}{\kern0pt}def\ \isacommand{{\isachardot}{\kern0pt}{\isachardot}{\kern0pt}}\isamarkupfalse%
%
\endisatagproof
{\isafoldproof}%
%
\isadelimproof
\isanewline
%
\endisadelimproof
\isanewline
\isacommand{lemma}\isamarkupfalse%
\ frecR{\isacharunderscore}{\kern0pt}D{\isadigit{1}}\ {\isacharcolon}{\kern0pt}\isanewline
\ \ {\isachardoublequoteopen}frecR{\isacharparenleft}{\kern0pt}x{\isacharcomma}{\kern0pt}y{\isacharparenright}{\kern0pt}\ {\isasymLongrightarrow}\ ftype{\isacharparenleft}{\kern0pt}y{\isacharparenright}{\kern0pt}\ {\isacharequal}{\kern0pt}\ {\isadigit{0}}\ {\isasymLongrightarrow}\ ftype{\isacharparenleft}{\kern0pt}x{\isacharparenright}{\kern0pt}\ {\isacharequal}{\kern0pt}\ {\isadigit{1}}\ {\isasymand}\ \isanewline
\ \ \ \ \ \ {\isacharparenleft}{\kern0pt}name{\isadigit{1}}{\isacharparenleft}{\kern0pt}x{\isacharparenright}{\kern0pt}\ {\isasymin}\ domain{\isacharparenleft}{\kern0pt}name{\isadigit{1}}{\isacharparenleft}{\kern0pt}y{\isacharparenright}{\kern0pt}{\isacharparenright}{\kern0pt}\ {\isasymunion}\ domain{\isacharparenleft}{\kern0pt}name{\isadigit{2}}{\isacharparenleft}{\kern0pt}y{\isacharparenright}{\kern0pt}{\isacharparenright}{\kern0pt}\ {\isasymand}\ {\isacharparenleft}{\kern0pt}name{\isadigit{2}}{\isacharparenleft}{\kern0pt}x{\isacharparenright}{\kern0pt}\ {\isacharequal}{\kern0pt}\ name{\isadigit{1}}{\isacharparenleft}{\kern0pt}y{\isacharparenright}{\kern0pt}\ {\isasymor}\ name{\isadigit{2}}{\isacharparenleft}{\kern0pt}x{\isacharparenright}{\kern0pt}\ {\isacharequal}{\kern0pt}\ name{\isadigit{2}}{\isacharparenleft}{\kern0pt}y{\isacharparenright}{\kern0pt}{\isacharparenright}{\kern0pt}{\isacharparenright}{\kern0pt}{\isachardoublequoteclose}\isanewline
%
\isadelimproof
\ \ %
\endisadelimproof
%
\isatagproof
\isacommand{using}\isamarkupfalse%
\ frecR{\isacharunderscore}{\kern0pt}iff\isanewline
\ \ \isacommand{by}\isamarkupfalse%
\ auto%
\endisatagproof
{\isafoldproof}%
%
\isadelimproof
\isanewline
%
\endisadelimproof
\isanewline
\isacommand{lemma}\isamarkupfalse%
\ frecR{\isacharunderscore}{\kern0pt}D{\isadigit{2}}\ {\isacharcolon}{\kern0pt}\isanewline
\ \ {\isachardoublequoteopen}frecR{\isacharparenleft}{\kern0pt}x{\isacharcomma}{\kern0pt}y{\isacharparenright}{\kern0pt}\ {\isasymLongrightarrow}\ ftype{\isacharparenleft}{\kern0pt}y{\isacharparenright}{\kern0pt}\ {\isacharequal}{\kern0pt}\ {\isadigit{1}}\ {\isasymLongrightarrow}\ ftype{\isacharparenleft}{\kern0pt}x{\isacharparenright}{\kern0pt}\ {\isacharequal}{\kern0pt}\ {\isadigit{0}}\ {\isasymand}\ \isanewline
\ \ \ \ \ \ ftype{\isacharparenleft}{\kern0pt}x{\isacharparenright}{\kern0pt}\ {\isacharequal}{\kern0pt}\ {\isadigit{0}}\ {\isasymand}\ ftype{\isacharparenleft}{\kern0pt}y{\isacharparenright}{\kern0pt}\ {\isacharequal}{\kern0pt}\ \ {\isadigit{1}}\ {\isasymand}\ name{\isadigit{1}}{\isacharparenleft}{\kern0pt}x{\isacharparenright}{\kern0pt}\ {\isacharequal}{\kern0pt}\ name{\isadigit{1}}{\isacharparenleft}{\kern0pt}y{\isacharparenright}{\kern0pt}\ {\isasymand}\ name{\isadigit{2}}{\isacharparenleft}{\kern0pt}x{\isacharparenright}{\kern0pt}\ {\isasymin}\ domain{\isacharparenleft}{\kern0pt}name{\isadigit{2}}{\isacharparenleft}{\kern0pt}y{\isacharparenright}{\kern0pt}{\isacharparenright}{\kern0pt}{\isachardoublequoteclose}\isanewline
%
\isadelimproof
\ \ %
\endisadelimproof
%
\isatagproof
\isacommand{using}\isamarkupfalse%
\ frecR{\isacharunderscore}{\kern0pt}iff\isanewline
\ \ \isacommand{by}\isamarkupfalse%
\ auto%
\endisatagproof
{\isafoldproof}%
%
\isadelimproof
\isanewline
%
\endisadelimproof
\isanewline
\isacommand{lemma}\isamarkupfalse%
\ frecR{\isacharunderscore}{\kern0pt}DI\ {\isacharcolon}{\kern0pt}\ \isanewline
\ \ \isakeyword{assumes}\ {\isachardoublequoteopen}frecR{\isacharparenleft}{\kern0pt}{\isasymlangle}a{\isacharcomma}{\kern0pt}b{\isacharcomma}{\kern0pt}c{\isacharcomma}{\kern0pt}d{\isasymrangle}{\isacharcomma}{\kern0pt}{\isasymlangle}ftype{\isacharparenleft}{\kern0pt}y{\isacharparenright}{\kern0pt}{\isacharcomma}{\kern0pt}name{\isadigit{1}}{\isacharparenleft}{\kern0pt}y{\isacharparenright}{\kern0pt}{\isacharcomma}{\kern0pt}name{\isadigit{2}}{\isacharparenleft}{\kern0pt}y{\isacharparenright}{\kern0pt}{\isacharcomma}{\kern0pt}cond{\isacharunderscore}{\kern0pt}of{\isacharparenleft}{\kern0pt}y{\isacharparenright}{\kern0pt}{\isasymrangle}{\isacharparenright}{\kern0pt}{\isachardoublequoteclose}\isanewline
\ \ \isakeyword{shows}\ {\isachardoublequoteopen}frecR{\isacharparenleft}{\kern0pt}{\isasymlangle}a{\isacharcomma}{\kern0pt}b{\isacharcomma}{\kern0pt}c{\isacharcomma}{\kern0pt}d{\isasymrangle}{\isacharcomma}{\kern0pt}y{\isacharparenright}{\kern0pt}{\isachardoublequoteclose}\isanewline
%
\isadelimproof
\ \ %
\endisadelimproof
%
\isatagproof
\isacommand{using}\isamarkupfalse%
\ assms\ \isacommand{unfolding}\isamarkupfalse%
\ frecR{\isacharunderscore}{\kern0pt}def\ \isacommand{by}\isamarkupfalse%
\ {\isacharparenleft}{\kern0pt}force\ simp\ add{\isacharcolon}{\kern0pt}components{\isacharunderscore}{\kern0pt}simp{\isacharparenright}{\kern0pt}%
\endisatagproof
{\isafoldproof}%
%
\isadelimproof
\isanewline
%
\endisadelimproof
\isanewline
\isanewline
\isacommand{definition}\isamarkupfalse%
\isanewline
\ \ is{\isacharunderscore}{\kern0pt}frecR\ {\isacharcolon}{\kern0pt}{\isacharcolon}{\kern0pt}\ {\isachardoublequoteopen}{\isacharbrackleft}{\kern0pt}i{\isasymRightarrow}o{\isacharcomma}{\kern0pt}i{\isacharcomma}{\kern0pt}i{\isacharbrackright}{\kern0pt}\ {\isasymRightarrow}\ o{\isachardoublequoteclose}\ \isakeyword{where}\isanewline
\ \ {\isachardoublequoteopen}is{\isacharunderscore}{\kern0pt}frecR{\isacharparenleft}{\kern0pt}M{\isacharcomma}{\kern0pt}x{\isacharcomma}{\kern0pt}y{\isacharparenright}{\kern0pt}\ {\isasymequiv}\ {\isasymexists}\ ftx{\isacharbrackleft}{\kern0pt}M{\isacharbrackright}{\kern0pt}{\isachardot}{\kern0pt}\ {\isasymexists}\ n{\isadigit{1}}x{\isacharbrackleft}{\kern0pt}M{\isacharbrackright}{\kern0pt}{\isachardot}{\kern0pt}\ {\isasymexists}n{\isadigit{2}}x{\isacharbrackleft}{\kern0pt}M{\isacharbrackright}{\kern0pt}{\isachardot}{\kern0pt}\ {\isasymexists}fty{\isacharbrackleft}{\kern0pt}M{\isacharbrackright}{\kern0pt}{\isachardot}{\kern0pt}\ {\isasymexists}n{\isadigit{1}}y{\isacharbrackleft}{\kern0pt}M{\isacharbrackright}{\kern0pt}{\isachardot}{\kern0pt}\ {\isasymexists}n{\isadigit{2}}y{\isacharbrackleft}{\kern0pt}M{\isacharbrackright}{\kern0pt}{\isachardot}{\kern0pt}\ {\isasymexists}dn{\isadigit{1}}{\isacharbrackleft}{\kern0pt}M{\isacharbrackright}{\kern0pt}{\isachardot}{\kern0pt}\ {\isasymexists}dn{\isadigit{2}}{\isacharbrackleft}{\kern0pt}M{\isacharbrackright}{\kern0pt}{\isachardot}{\kern0pt}\isanewline
\ \ is{\isacharunderscore}{\kern0pt}ftype{\isacharparenleft}{\kern0pt}M{\isacharcomma}{\kern0pt}x{\isacharcomma}{\kern0pt}ftx{\isacharparenright}{\kern0pt}\ {\isasymand}\ is{\isacharunderscore}{\kern0pt}name{\isadigit{1}}{\isacharparenleft}{\kern0pt}M{\isacharcomma}{\kern0pt}x{\isacharcomma}{\kern0pt}n{\isadigit{1}}x{\isacharparenright}{\kern0pt}{\isasymand}\ is{\isacharunderscore}{\kern0pt}name{\isadigit{2}}{\isacharparenleft}{\kern0pt}M{\isacharcomma}{\kern0pt}x{\isacharcomma}{\kern0pt}n{\isadigit{2}}x{\isacharparenright}{\kern0pt}\ {\isasymand}\isanewline
\ \ is{\isacharunderscore}{\kern0pt}ftype{\isacharparenleft}{\kern0pt}M{\isacharcomma}{\kern0pt}y{\isacharcomma}{\kern0pt}fty{\isacharparenright}{\kern0pt}\ {\isasymand}\ is{\isacharunderscore}{\kern0pt}name{\isadigit{1}}{\isacharparenleft}{\kern0pt}M{\isacharcomma}{\kern0pt}y{\isacharcomma}{\kern0pt}n{\isadigit{1}}y{\isacharparenright}{\kern0pt}\ {\isasymand}\ is{\isacharunderscore}{\kern0pt}name{\isadigit{2}}{\isacharparenleft}{\kern0pt}M{\isacharcomma}{\kern0pt}y{\isacharcomma}{\kern0pt}n{\isadigit{2}}y{\isacharparenright}{\kern0pt}\isanewline
\ \ \ \ \ \ \ \ \ \ {\isasymand}\ is{\isacharunderscore}{\kern0pt}domain{\isacharparenleft}{\kern0pt}M{\isacharcomma}{\kern0pt}n{\isadigit{1}}y{\isacharcomma}{\kern0pt}dn{\isadigit{1}}{\isacharparenright}{\kern0pt}\ {\isasymand}\ is{\isacharunderscore}{\kern0pt}domain{\isacharparenleft}{\kern0pt}M{\isacharcomma}{\kern0pt}n{\isadigit{2}}y{\isacharcomma}{\kern0pt}dn{\isadigit{2}}{\isacharparenright}{\kern0pt}\ {\isasymand}\ \isanewline
\ \ \ \ \ \ \ \ \ \ {\isacharparenleft}{\kern0pt}\ \ {\isacharparenleft}{\kern0pt}number{\isadigit{1}}{\isacharparenleft}{\kern0pt}M{\isacharcomma}{\kern0pt}ftx{\isacharparenright}{\kern0pt}\ {\isasymand}\ empty{\isacharparenleft}{\kern0pt}M{\isacharcomma}{\kern0pt}fty{\isacharparenright}{\kern0pt}\ {\isasymand}\ {\isacharparenleft}{\kern0pt}n{\isadigit{1}}x\ {\isasymin}\ dn{\isadigit{1}}\ {\isasymor}\ n{\isadigit{1}}x\ {\isasymin}\ dn{\isadigit{2}}{\isacharparenright}{\kern0pt}\ {\isasymand}\ {\isacharparenleft}{\kern0pt}n{\isadigit{2}}x\ {\isacharequal}{\kern0pt}\ n{\isadigit{1}}y\ {\isasymor}\ n{\isadigit{2}}x\ {\isacharequal}{\kern0pt}\ n{\isadigit{2}}y{\isacharparenright}{\kern0pt}{\isacharparenright}{\kern0pt}\isanewline
\ \ \ \ \ \ \ \ \ \ \ {\isasymor}\ {\isacharparenleft}{\kern0pt}empty{\isacharparenleft}{\kern0pt}M{\isacharcomma}{\kern0pt}ftx{\isacharparenright}{\kern0pt}\ {\isasymand}\ number{\isadigit{1}}{\isacharparenleft}{\kern0pt}M{\isacharcomma}{\kern0pt}fty{\isacharparenright}{\kern0pt}\ {\isasymand}\ n{\isadigit{1}}x\ {\isacharequal}{\kern0pt}\ n{\isadigit{1}}y\ {\isasymand}\ n{\isadigit{2}}x\ {\isasymin}\ dn{\isadigit{2}}{\isacharparenright}{\kern0pt}{\isacharparenright}{\kern0pt}{\isachardoublequoteclose}\isanewline
\isanewline
\isacommand{schematic{\isacharunderscore}{\kern0pt}goal}\isamarkupfalse%
\ sats{\isacharunderscore}{\kern0pt}frecR{\isacharunderscore}{\kern0pt}fm{\isacharunderscore}{\kern0pt}auto{\isacharcolon}{\kern0pt}\isanewline
\ \ \isakeyword{assumes}\ \isanewline
\ \ \ \ {\isachardoublequoteopen}i{\isasymin}nat{\isachardoublequoteclose}\ {\isachardoublequoteopen}j{\isasymin}nat{\isachardoublequoteclose}\ {\isachardoublequoteopen}env{\isasymin}list{\isacharparenleft}{\kern0pt}A{\isacharparenright}{\kern0pt}{\isachardoublequoteclose}\ {\isachardoublequoteopen}nth{\isacharparenleft}{\kern0pt}i{\isacharcomma}{\kern0pt}env{\isacharparenright}{\kern0pt}\ {\isacharequal}{\kern0pt}\ a{\isachardoublequoteclose}\ {\isachardoublequoteopen}nth{\isacharparenleft}{\kern0pt}j{\isacharcomma}{\kern0pt}env{\isacharparenright}{\kern0pt}\ {\isacharequal}{\kern0pt}\ b{\isachardoublequoteclose}\isanewline
\ \ \isakeyword{shows}\isanewline
\ \ \ \ {\isachardoublequoteopen}is{\isacharunderscore}{\kern0pt}frecR{\isacharparenleft}{\kern0pt}{\isacharhash}{\kern0pt}{\isacharhash}{\kern0pt}A{\isacharcomma}{\kern0pt}a{\isacharcomma}{\kern0pt}b{\isacharparenright}{\kern0pt}\ {\isasymlongleftrightarrow}\ sats{\isacharparenleft}{\kern0pt}A{\isacharcomma}{\kern0pt}{\isacharquery}{\kern0pt}fr{\isacharunderscore}{\kern0pt}fm{\isacharparenleft}{\kern0pt}i{\isacharcomma}{\kern0pt}j{\isacharparenright}{\kern0pt}{\isacharcomma}{\kern0pt}env{\isacharparenright}{\kern0pt}{\isachardoublequoteclose}\isanewline
%
\isadelimproof
\ \ %
\endisadelimproof
%
\isatagproof
\isacommand{unfolding}\isamarkupfalse%
\ \ is{\isacharunderscore}{\kern0pt}frecR{\isacharunderscore}{\kern0pt}def\ is{\isacharunderscore}{\kern0pt}Collect{\isacharunderscore}{\kern0pt}def\ \ \isanewline
\ \ \isacommand{by}\isamarkupfalse%
\ {\isacharparenleft}{\kern0pt}insert\ assms\ {\isacharsemicolon}{\kern0pt}\ {\isacharparenleft}{\kern0pt}rule\ sep{\isacharunderscore}{\kern0pt}rules{\isacharprime}{\kern0pt}\ cartprod{\isacharunderscore}{\kern0pt}iff{\isacharunderscore}{\kern0pt}sats\ \ components{\isacharunderscore}{\kern0pt}iff{\isacharunderscore}{\kern0pt}sats\isanewline
\ \ \ \ \ \ \ \ {\isacharbar}{\kern0pt}\ simp\ del{\isacharcolon}{\kern0pt}sats{\isacharunderscore}{\kern0pt}cartprod{\isacharunderscore}{\kern0pt}fm{\isacharparenright}{\kern0pt}{\isacharplus}{\kern0pt}{\isacharparenright}{\kern0pt}%
\endisatagproof
{\isafoldproof}%
%
\isadelimproof
\isanewline
%
\endisadelimproof
%
\isadelimML
\isanewline
%
\endisadelimML
%
\isatagML
\isacommand{synthesize}\isamarkupfalse%
\ {\isachardoublequoteopen}frecR{\isacharunderscore}{\kern0pt}fm{\isachardoublequoteclose}\ \isakeyword{from{\isacharunderscore}{\kern0pt}schematic}\ sats{\isacharunderscore}{\kern0pt}frecR{\isacharunderscore}{\kern0pt}fm{\isacharunderscore}{\kern0pt}auto%
\endisatagML
{\isafoldML}%
%
\isadelimML
\isanewline
%
\endisadelimML
\isanewline
\isanewline
\isacommand{lemma}\isamarkupfalse%
\ eq{\isacharunderscore}{\kern0pt}ftypep{\isacharunderscore}{\kern0pt}not{\isacharunderscore}{\kern0pt}frecrR{\isacharcolon}{\kern0pt}\isanewline
\ \ \isakeyword{assumes}\ {\isachardoublequoteopen}ftype{\isacharparenleft}{\kern0pt}x{\isacharparenright}{\kern0pt}\ {\isacharequal}{\kern0pt}\ ftype{\isacharparenleft}{\kern0pt}y{\isacharparenright}{\kern0pt}{\isachardoublequoteclose}\isanewline
\ \ \isakeyword{shows}\ {\isachardoublequoteopen}{\isasymnot}\ frecR{\isacharparenleft}{\kern0pt}x{\isacharcomma}{\kern0pt}y{\isacharparenright}{\kern0pt}{\isachardoublequoteclose}\isanewline
%
\isadelimproof
\ \ %
\endisadelimproof
%
\isatagproof
\isacommand{using}\isamarkupfalse%
\ assms\ frecR{\isacharunderscore}{\kern0pt}ftypeD\ \isacommand{by}\isamarkupfalse%
\ force%
\endisatagproof
{\isafoldproof}%
%
\isadelimproof
\isanewline
%
\endisadelimproof
\isanewline
\isanewline
\isacommand{definition}\isamarkupfalse%
\isanewline
\ \ rank{\isacharunderscore}{\kern0pt}names\ {\isacharcolon}{\kern0pt}{\isacharcolon}{\kern0pt}\ {\isachardoublequoteopen}i\ {\isasymRightarrow}\ i{\isachardoublequoteclose}\ \isakeyword{where}\isanewline
\ \ {\isachardoublequoteopen}rank{\isacharunderscore}{\kern0pt}names{\isacharparenleft}{\kern0pt}x{\isacharparenright}{\kern0pt}\ {\isasymequiv}\ max{\isacharparenleft}{\kern0pt}rank{\isacharparenleft}{\kern0pt}name{\isadigit{1}}{\isacharparenleft}{\kern0pt}x{\isacharparenright}{\kern0pt}{\isacharparenright}{\kern0pt}{\isacharcomma}{\kern0pt}rank{\isacharparenleft}{\kern0pt}name{\isadigit{2}}{\isacharparenleft}{\kern0pt}x{\isacharparenright}{\kern0pt}{\isacharparenright}{\kern0pt}{\isacharparenright}{\kern0pt}{\isachardoublequoteclose}\isanewline
\isanewline
\isacommand{lemma}\isamarkupfalse%
\ rank{\isacharunderscore}{\kern0pt}names{\isacharunderscore}{\kern0pt}types\ {\isacharbrackleft}{\kern0pt}TC{\isacharbrackright}{\kern0pt}{\isacharcolon}{\kern0pt}\ \isanewline
\ \ \isakeyword{shows}\ {\isachardoublequoteopen}Ord{\isacharparenleft}{\kern0pt}rank{\isacharunderscore}{\kern0pt}names{\isacharparenleft}{\kern0pt}x{\isacharparenright}{\kern0pt}{\isacharparenright}{\kern0pt}{\isachardoublequoteclose}\isanewline
%
\isadelimproof
\ \ %
\endisadelimproof
%
\isatagproof
\isacommand{unfolding}\isamarkupfalse%
\ rank{\isacharunderscore}{\kern0pt}names{\isacharunderscore}{\kern0pt}def\ max{\isacharunderscore}{\kern0pt}def\ \isacommand{using}\isamarkupfalse%
\ Ord{\isacharunderscore}{\kern0pt}rank\ Ord{\isacharunderscore}{\kern0pt}Un\ \isacommand{by}\isamarkupfalse%
\ auto%
\endisatagproof
{\isafoldproof}%
%
\isadelimproof
\isanewline
%
\endisadelimproof
\isanewline
\isacommand{definition}\isamarkupfalse%
\isanewline
\ \ mtype{\isacharunderscore}{\kern0pt}form\ {\isacharcolon}{\kern0pt}{\isacharcolon}{\kern0pt}\ {\isachardoublequoteopen}i\ {\isasymRightarrow}\ i{\isachardoublequoteclose}\ \isakeyword{where}\isanewline
\ \ {\isachardoublequoteopen}mtype{\isacharunderscore}{\kern0pt}form{\isacharparenleft}{\kern0pt}x{\isacharparenright}{\kern0pt}\ {\isasymequiv}\ if\ rank{\isacharparenleft}{\kern0pt}name{\isadigit{1}}{\isacharparenleft}{\kern0pt}x{\isacharparenright}{\kern0pt}{\isacharparenright}{\kern0pt}\ {\isacharless}{\kern0pt}\ rank{\isacharparenleft}{\kern0pt}name{\isadigit{2}}{\isacharparenleft}{\kern0pt}x{\isacharparenright}{\kern0pt}{\isacharparenright}{\kern0pt}\ then\ {\isadigit{0}}\ else\ {\isadigit{2}}{\isachardoublequoteclose}\isanewline
\isanewline
\isacommand{definition}\isamarkupfalse%
\isanewline
\ \ type{\isacharunderscore}{\kern0pt}form\ {\isacharcolon}{\kern0pt}{\isacharcolon}{\kern0pt}\ {\isachardoublequoteopen}i\ {\isasymRightarrow}\ i{\isachardoublequoteclose}\ \isakeyword{where}\isanewline
\ \ {\isachardoublequoteopen}type{\isacharunderscore}{\kern0pt}form{\isacharparenleft}{\kern0pt}x{\isacharparenright}{\kern0pt}\ {\isasymequiv}\ if\ ftype{\isacharparenleft}{\kern0pt}x{\isacharparenright}{\kern0pt}\ {\isacharequal}{\kern0pt}\ {\isadigit{0}}\ then\ {\isadigit{1}}\ else\ mtype{\isacharunderscore}{\kern0pt}form{\isacharparenleft}{\kern0pt}x{\isacharparenright}{\kern0pt}{\isachardoublequoteclose}\isanewline
\isanewline
\isacommand{lemma}\isamarkupfalse%
\ type{\isacharunderscore}{\kern0pt}form{\isacharunderscore}{\kern0pt}tc\ {\isacharbrackleft}{\kern0pt}TC{\isacharbrackright}{\kern0pt}{\isacharcolon}{\kern0pt}\ \isanewline
\ \ \isakeyword{shows}\ {\isachardoublequoteopen}type{\isacharunderscore}{\kern0pt}form{\isacharparenleft}{\kern0pt}x{\isacharparenright}{\kern0pt}\ {\isasymin}\ {\isadigit{3}}{\isachardoublequoteclose}\isanewline
%
\isadelimproof
\ \ %
\endisadelimproof
%
\isatagproof
\isacommand{unfolding}\isamarkupfalse%
\ type{\isacharunderscore}{\kern0pt}form{\isacharunderscore}{\kern0pt}def\ mtype{\isacharunderscore}{\kern0pt}form{\isacharunderscore}{\kern0pt}def\ \isacommand{by}\isamarkupfalse%
\ auto%
\endisatagproof
{\isafoldproof}%
%
\isadelimproof
\isanewline
%
\endisadelimproof
\isanewline
\isacommand{lemma}\isamarkupfalse%
\ frecR{\isacharunderscore}{\kern0pt}le{\isacharunderscore}{\kern0pt}rnk{\isacharunderscore}{\kern0pt}names\ {\isacharcolon}{\kern0pt}\isanewline
\ \ \isakeyword{assumes}\ \ {\isachardoublequoteopen}frecR{\isacharparenleft}{\kern0pt}x{\isacharcomma}{\kern0pt}y{\isacharparenright}{\kern0pt}{\isachardoublequoteclose}\isanewline
\ \ \isakeyword{shows}\ {\isachardoublequoteopen}rank{\isacharunderscore}{\kern0pt}names{\isacharparenleft}{\kern0pt}x{\isacharparenright}{\kern0pt}{\isasymle}rank{\isacharunderscore}{\kern0pt}names{\isacharparenleft}{\kern0pt}y{\isacharparenright}{\kern0pt}{\isachardoublequoteclose}\isanewline
%
\isadelimproof
%
\endisadelimproof
%
\isatagproof
\isacommand{proof}\isamarkupfalse%
\ {\isacharminus}{\kern0pt}\isanewline
\ \ \isacommand{obtain}\isamarkupfalse%
\ a\ b\ c\ d\ \ \isakeyword{where}\isanewline
\ \ \ \ H{\isacharcolon}{\kern0pt}\ {\isachardoublequoteopen}a\ {\isacharequal}{\kern0pt}\ name{\isadigit{1}}{\isacharparenleft}{\kern0pt}x{\isacharparenright}{\kern0pt}{\isachardoublequoteclose}\ {\isachardoublequoteopen}b\ {\isacharequal}{\kern0pt}\ name{\isadigit{2}}{\isacharparenleft}{\kern0pt}x{\isacharparenright}{\kern0pt}{\isachardoublequoteclose}\isanewline
\ \ \ \ {\isachardoublequoteopen}c\ {\isacharequal}{\kern0pt}\ name{\isadigit{1}}{\isacharparenleft}{\kern0pt}y{\isacharparenright}{\kern0pt}{\isachardoublequoteclose}\ {\isachardoublequoteopen}d\ {\isacharequal}{\kern0pt}\ name{\isadigit{2}}{\isacharparenleft}{\kern0pt}y{\isacharparenright}{\kern0pt}{\isachardoublequoteclose}\isanewline
\ \ \ \ {\isachardoublequoteopen}{\isacharparenleft}{\kern0pt}a\ {\isasymin}\ domain{\isacharparenleft}{\kern0pt}c{\isacharparenright}{\kern0pt}{\isasymunion}domain{\isacharparenleft}{\kern0pt}d{\isacharparenright}{\kern0pt}\ {\isasymand}\ {\isacharparenleft}{\kern0pt}b{\isacharequal}{\kern0pt}c\ {\isasymor}\ b\ {\isacharequal}{\kern0pt}\ d{\isacharparenright}{\kern0pt}{\isacharparenright}{\kern0pt}\ {\isasymor}\ {\isacharparenleft}{\kern0pt}a\ {\isacharequal}{\kern0pt}\ c\ {\isasymand}\ b\ {\isasymin}\ domain{\isacharparenleft}{\kern0pt}d{\isacharparenright}{\kern0pt}{\isacharparenright}{\kern0pt}{\isachardoublequoteclose}\isanewline
\ \ \ \ \isacommand{using}\isamarkupfalse%
\ assms\ \isacommand{unfolding}\isamarkupfalse%
\ frecR{\isacharunderscore}{\kern0pt}def\ \isacommand{by}\isamarkupfalse%
\ force\isanewline
\ \ \isacommand{then}\isamarkupfalse%
\ \isanewline
\ \ \isacommand{consider}\isamarkupfalse%
\isanewline
\ \ \ \ {\isacharparenleft}{\kern0pt}m{\isacharparenright}{\kern0pt}\ {\isachardoublequoteopen}a\ {\isasymin}\ domain{\isacharparenleft}{\kern0pt}c{\isacharparenright}{\kern0pt}\ {\isasymand}\ {\isacharparenleft}{\kern0pt}b\ {\isacharequal}{\kern0pt}\ c\ {\isasymor}\ b\ {\isacharequal}{\kern0pt}\ d{\isacharparenright}{\kern0pt}\ {\isachardoublequoteclose}\isanewline
\ \ \ \ {\isacharbar}{\kern0pt}\ {\isacharparenleft}{\kern0pt}n{\isacharparenright}{\kern0pt}\ {\isachardoublequoteopen}a\ {\isasymin}\ domain{\isacharparenleft}{\kern0pt}d{\isacharparenright}{\kern0pt}\ {\isasymand}\ {\isacharparenleft}{\kern0pt}b\ {\isacharequal}{\kern0pt}\ c\ {\isasymor}\ b\ {\isacharequal}{\kern0pt}\ d{\isacharparenright}{\kern0pt}{\isachardoublequoteclose}\ \isanewline
\ \ \ \ {\isacharbar}{\kern0pt}\ {\isacharparenleft}{\kern0pt}o{\isacharparenright}{\kern0pt}\ {\isachardoublequoteopen}b\ {\isasymin}\ domain{\isacharparenleft}{\kern0pt}d{\isacharparenright}{\kern0pt}\ {\isasymand}\ a\ {\isacharequal}{\kern0pt}\ c{\isachardoublequoteclose}\isanewline
\ \ \ \ \isacommand{by}\isamarkupfalse%
\ auto\isanewline
\ \ \isacommand{then}\isamarkupfalse%
\ \isacommand{show}\isamarkupfalse%
\ {\isacharquery}{\kern0pt}thesis\ \isacommand{proof}\isamarkupfalse%
{\isacharparenleft}{\kern0pt}cases{\isacharparenright}{\kern0pt}\isanewline
\ \ \ \ \isacommand{case}\isamarkupfalse%
\ m\isanewline
\ \ \ \ \isacommand{then}\isamarkupfalse%
\ \isanewline
\ \ \ \ \isacommand{have}\isamarkupfalse%
\ {\isachardoublequoteopen}rank{\isacharparenleft}{\kern0pt}a{\isacharparenright}{\kern0pt}\ {\isacharless}{\kern0pt}\ rank{\isacharparenleft}{\kern0pt}c{\isacharparenright}{\kern0pt}{\isachardoublequoteclose}\ \isanewline
\ \ \ \ \ \ \isacommand{using}\isamarkupfalse%
\ eclose{\isacharunderscore}{\kern0pt}rank{\isacharunderscore}{\kern0pt}lt\ \ in{\isacharunderscore}{\kern0pt}dom{\isacharunderscore}{\kern0pt}in{\isacharunderscore}{\kern0pt}eclose\ \ \isacommand{by}\isamarkupfalse%
\ simp\isanewline
\ \ \ \ \isacommand{with}\isamarkupfalse%
\ {\isacartoucheopen}rank{\isacharparenleft}{\kern0pt}a{\isacharparenright}{\kern0pt}\ {\isacharless}{\kern0pt}\ rank{\isacharparenleft}{\kern0pt}c{\isacharparenright}{\kern0pt}{\isacartoucheclose}\ H\ m\isanewline
\ \ \ \ \isacommand{show}\isamarkupfalse%
\ {\isacharquery}{\kern0pt}thesis\ \isacommand{unfolding}\isamarkupfalse%
\ rank{\isacharunderscore}{\kern0pt}names{\isacharunderscore}{\kern0pt}def\ \isacommand{using}\isamarkupfalse%
\ Ord{\isacharunderscore}{\kern0pt}rank\ max{\isacharunderscore}{\kern0pt}cong\ max{\isacharunderscore}{\kern0pt}cong{\isadigit{2}}\ leI\ \isacommand{by}\isamarkupfalse%
\ auto\isanewline
\ \ \isacommand{next}\isamarkupfalse%
\isanewline
\ \ \ \ \isacommand{case}\isamarkupfalse%
\ n\isanewline
\ \ \ \ \isacommand{then}\isamarkupfalse%
\isanewline
\ \ \ \ \isacommand{have}\isamarkupfalse%
\ {\isachardoublequoteopen}rank{\isacharparenleft}{\kern0pt}a{\isacharparenright}{\kern0pt}\ {\isacharless}{\kern0pt}\ rank{\isacharparenleft}{\kern0pt}d{\isacharparenright}{\kern0pt}{\isachardoublequoteclose}\isanewline
\ \ \ \ \ \ \isacommand{using}\isamarkupfalse%
\ eclose{\isacharunderscore}{\kern0pt}rank{\isacharunderscore}{\kern0pt}lt\ in{\isacharunderscore}{\kern0pt}dom{\isacharunderscore}{\kern0pt}in{\isacharunderscore}{\kern0pt}eclose\ \ \isacommand{by}\isamarkupfalse%
\ simp\isanewline
\ \ \ \ \isacommand{with}\isamarkupfalse%
\ {\isacartoucheopen}rank{\isacharparenleft}{\kern0pt}a{\isacharparenright}{\kern0pt}\ {\isacharless}{\kern0pt}\ rank{\isacharparenleft}{\kern0pt}d{\isacharparenright}{\kern0pt}{\isacartoucheclose}\ H\ n\isanewline
\ \ \ \ \isacommand{show}\isamarkupfalse%
\ {\isacharquery}{\kern0pt}thesis\ \isacommand{unfolding}\isamarkupfalse%
\ rank{\isacharunderscore}{\kern0pt}names{\isacharunderscore}{\kern0pt}def\ \isanewline
\ \ \ \ \ \ \isacommand{using}\isamarkupfalse%
\ Ord{\isacharunderscore}{\kern0pt}rank\ max{\isacharunderscore}{\kern0pt}cong{\isadigit{2}}\ max{\isacharunderscore}{\kern0pt}cong\ max{\isacharunderscore}{\kern0pt}commutes{\isacharbrackleft}{\kern0pt}of\ {\isachardoublequoteopen}rank{\isacharparenleft}{\kern0pt}c{\isacharparenright}{\kern0pt}{\isachardoublequoteclose}\ {\isachardoublequoteopen}rank{\isacharparenleft}{\kern0pt}d{\isacharparenright}{\kern0pt}{\isachardoublequoteclose}{\isacharbrackright}{\kern0pt}\ leI\ \isacommand{by}\isamarkupfalse%
\ auto\isanewline
\ \ \isacommand{next}\isamarkupfalse%
\isanewline
\ \ \ \ \isacommand{case}\isamarkupfalse%
\ o\isanewline
\ \ \ \ \isacommand{then}\isamarkupfalse%
\isanewline
\ \ \ \ \isacommand{have}\isamarkupfalse%
\ {\isachardoublequoteopen}rank{\isacharparenleft}{\kern0pt}b{\isacharparenright}{\kern0pt}\ {\isacharless}{\kern0pt}\ rank{\isacharparenleft}{\kern0pt}d{\isacharparenright}{\kern0pt}{\isachardoublequoteclose}\ {\isacharparenleft}{\kern0pt}\isakeyword{is}\ {\isachardoublequoteopen}{\isacharquery}{\kern0pt}b\ {\isacharless}{\kern0pt}\ {\isacharquery}{\kern0pt}d{\isachardoublequoteclose}{\isacharparenright}{\kern0pt}\ {\isachardoublequoteopen}rank{\isacharparenleft}{\kern0pt}a{\isacharparenright}{\kern0pt}\ {\isacharequal}{\kern0pt}\ rank{\isacharparenleft}{\kern0pt}c{\isacharparenright}{\kern0pt}{\isachardoublequoteclose}\ {\isacharparenleft}{\kern0pt}\isakeyword{is}\ {\isachardoublequoteopen}{\isacharquery}{\kern0pt}a\ {\isacharequal}{\kern0pt}\ {\isacharunderscore}{\kern0pt}{\isachardoublequoteclose}{\isacharparenright}{\kern0pt}\isanewline
\ \ \ \ \ \ \isacommand{using}\isamarkupfalse%
\ eclose{\isacharunderscore}{\kern0pt}rank{\isacharunderscore}{\kern0pt}lt\ in{\isacharunderscore}{\kern0pt}dom{\isacharunderscore}{\kern0pt}in{\isacharunderscore}{\kern0pt}eclose\ \ \isacommand{by}\isamarkupfalse%
\ simp{\isacharunderscore}{\kern0pt}all\isanewline
\ \ \ \ \isacommand{with}\isamarkupfalse%
\ H\isanewline
\ \ \ \ \isacommand{show}\isamarkupfalse%
\ {\isacharquery}{\kern0pt}thesis\ \isacommand{unfolding}\isamarkupfalse%
\ rank{\isacharunderscore}{\kern0pt}names{\isacharunderscore}{\kern0pt}def\isanewline
\ \ \ \ \ \ \isacommand{using}\isamarkupfalse%
\ Ord{\isacharunderscore}{\kern0pt}rank\ max{\isacharunderscore}{\kern0pt}commutes\ max{\isacharunderscore}{\kern0pt}cong{\isadigit{2}}{\isacharbrackleft}{\kern0pt}OF\ leI{\isacharbrackleft}{\kern0pt}OF\ {\isacartoucheopen}{\isacharquery}{\kern0pt}b\ {\isacharless}{\kern0pt}\ {\isacharquery}{\kern0pt}d{\isacartoucheclose}{\isacharbrackright}{\kern0pt}{\isacharcomma}{\kern0pt}\ of\ {\isacharquery}{\kern0pt}a{\isacharbrackright}{\kern0pt}\ \isacommand{by}\isamarkupfalse%
\ simp\isanewline
\ \ \isacommand{qed}\isamarkupfalse%
\isanewline
\isacommand{qed}\isamarkupfalse%
%
\endisatagproof
{\isafoldproof}%
%
\isadelimproof
\isanewline
%
\endisadelimproof
\isanewline
\isanewline
\isacommand{definition}\isamarkupfalse%
\ \isanewline
\ \ {\isasymGamma}\ {\isacharcolon}{\kern0pt}{\isacharcolon}{\kern0pt}\ {\isachardoublequoteopen}i\ {\isasymRightarrow}\ i{\isachardoublequoteclose}\ \isakeyword{where}\isanewline
\ \ {\isachardoublequoteopen}{\isasymGamma}{\isacharparenleft}{\kern0pt}x{\isacharparenright}{\kern0pt}\ {\isacharequal}{\kern0pt}\ {\isadigit{3}}\ {\isacharasterisk}{\kern0pt}{\isacharasterisk}{\kern0pt}\ rank{\isacharunderscore}{\kern0pt}names{\isacharparenleft}{\kern0pt}x{\isacharparenright}{\kern0pt}\ {\isacharplus}{\kern0pt}{\isacharplus}{\kern0pt}\ type{\isacharunderscore}{\kern0pt}form{\isacharparenleft}{\kern0pt}x{\isacharparenright}{\kern0pt}{\isachardoublequoteclose}\isanewline
\isanewline
\isacommand{lemma}\isamarkupfalse%
\ {\isasymGamma}{\isacharunderscore}{\kern0pt}type\ {\isacharbrackleft}{\kern0pt}TC{\isacharbrackright}{\kern0pt}{\isacharcolon}{\kern0pt}\ \isanewline
\ \ \isakeyword{shows}\ {\isachardoublequoteopen}Ord{\isacharparenleft}{\kern0pt}{\isasymGamma}{\isacharparenleft}{\kern0pt}x{\isacharparenright}{\kern0pt}{\isacharparenright}{\kern0pt}{\isachardoublequoteclose}\isanewline
%
\isadelimproof
\ \ %
\endisadelimproof
%
\isatagproof
\isacommand{unfolding}\isamarkupfalse%
\ {\isasymGamma}{\isacharunderscore}{\kern0pt}def\ \isacommand{by}\isamarkupfalse%
\ simp%
\endisatagproof
{\isafoldproof}%
%
\isadelimproof
\isanewline
%
\endisadelimproof
\isanewline
\isanewline
\isacommand{lemma}\isamarkupfalse%
\ {\isasymGamma}{\isacharunderscore}{\kern0pt}mono\ {\isacharcolon}{\kern0pt}\ \isanewline
\ \ \isakeyword{assumes}\ {\isachardoublequoteopen}frecR{\isacharparenleft}{\kern0pt}x{\isacharcomma}{\kern0pt}y{\isacharparenright}{\kern0pt}{\isachardoublequoteclose}\isanewline
\ \ \isakeyword{shows}\ {\isachardoublequoteopen}{\isasymGamma}{\isacharparenleft}{\kern0pt}x{\isacharparenright}{\kern0pt}\ {\isacharless}{\kern0pt}\ {\isasymGamma}{\isacharparenleft}{\kern0pt}y{\isacharparenright}{\kern0pt}{\isachardoublequoteclose}\isanewline
%
\isadelimproof
%
\endisadelimproof
%
\isatagproof
\isacommand{proof}\isamarkupfalse%
\ {\isacharminus}{\kern0pt}\isanewline
\ \ \isacommand{have}\isamarkupfalse%
\ F{\isacharcolon}{\kern0pt}\ {\isachardoublequoteopen}type{\isacharunderscore}{\kern0pt}form{\isacharparenleft}{\kern0pt}x{\isacharparenright}{\kern0pt}\ {\isacharless}{\kern0pt}\ {\isadigit{3}}{\isachardoublequoteclose}\ {\isachardoublequoteopen}type{\isacharunderscore}{\kern0pt}form{\isacharparenleft}{\kern0pt}y{\isacharparenright}{\kern0pt}\ {\isacharless}{\kern0pt}\ {\isadigit{3}}{\isachardoublequoteclose}\isanewline
\ \ \ \ \isacommand{using}\isamarkupfalse%
\ ltI\ \isacommand{by}\isamarkupfalse%
\ simp{\isacharunderscore}{\kern0pt}all\isanewline
\ \ \isacommand{from}\isamarkupfalse%
\ assms\ \isanewline
\ \ \isacommand{have}\isamarkupfalse%
\ A{\isacharcolon}{\kern0pt}\ {\isachardoublequoteopen}rank{\isacharunderscore}{\kern0pt}names{\isacharparenleft}{\kern0pt}x{\isacharparenright}{\kern0pt}\ {\isasymle}\ rank{\isacharunderscore}{\kern0pt}names{\isacharparenleft}{\kern0pt}y{\isacharparenright}{\kern0pt}{\isachardoublequoteclose}\ {\isacharparenleft}{\kern0pt}\isakeyword{is}\ {\isachardoublequoteopen}{\isacharquery}{\kern0pt}x\ {\isasymle}\ {\isacharquery}{\kern0pt}y{\isachardoublequoteclose}{\isacharparenright}{\kern0pt}\isanewline
\ \ \ \ \isacommand{using}\isamarkupfalse%
\ frecR{\isacharunderscore}{\kern0pt}le{\isacharunderscore}{\kern0pt}rnk{\isacharunderscore}{\kern0pt}names\ \isacommand{by}\isamarkupfalse%
\ simp\isanewline
\ \ \isacommand{then}\isamarkupfalse%
\isanewline
\ \ \isacommand{have}\isamarkupfalse%
\ {\isachardoublequoteopen}Ord{\isacharparenleft}{\kern0pt}{\isacharquery}{\kern0pt}y{\isacharparenright}{\kern0pt}{\isachardoublequoteclose}\ \isacommand{unfolding}\isamarkupfalse%
\ rank{\isacharunderscore}{\kern0pt}names{\isacharunderscore}{\kern0pt}def\ \isacommand{using}\isamarkupfalse%
\ Ord{\isacharunderscore}{\kern0pt}rank\ max{\isacharunderscore}{\kern0pt}def\ \isacommand{by}\isamarkupfalse%
\ simp\isanewline
\ \ \isacommand{note}\isamarkupfalse%
\ leE{\isacharbrackleft}{\kern0pt}OF\ {\isacartoucheopen}{\isacharquery}{\kern0pt}x{\isasymle}{\isacharquery}{\kern0pt}y{\isacartoucheclose}{\isacharbrackright}{\kern0pt}\ \isanewline
\ \ \isacommand{then}\isamarkupfalse%
\isanewline
\ \ \isacommand{show}\isamarkupfalse%
\ {\isacharquery}{\kern0pt}thesis\isanewline
\ \ \isacommand{proof}\isamarkupfalse%
{\isacharparenleft}{\kern0pt}cases{\isacharparenright}{\kern0pt}\isanewline
\ \ \ \ \isacommand{case}\isamarkupfalse%
\ {\isadigit{1}}\isanewline
\ \ \ \ \isacommand{then}\isamarkupfalse%
\ \isanewline
\ \ \ \ \isacommand{show}\isamarkupfalse%
\ {\isacharquery}{\kern0pt}thesis\ \isacommand{unfolding}\isamarkupfalse%
\ {\isasymGamma}{\isacharunderscore}{\kern0pt}def\ \isacommand{using}\isamarkupfalse%
\ oadd{\isacharunderscore}{\kern0pt}lt{\isacharunderscore}{\kern0pt}mono{\isadigit{2}}\ {\isacartoucheopen}{\isacharquery}{\kern0pt}x\ {\isacharless}{\kern0pt}\ {\isacharquery}{\kern0pt}y{\isacartoucheclose}\ F\ \isacommand{by}\isamarkupfalse%
\ auto\isanewline
\ \ \isacommand{next}\isamarkupfalse%
\isanewline
\ \ \ \ \isacommand{case}\isamarkupfalse%
\ {\isadigit{2}}\isanewline
\ \ \ \ \isacommand{consider}\isamarkupfalse%
\ {\isacharparenleft}{\kern0pt}a{\isacharparenright}{\kern0pt}\ {\isachardoublequoteopen}ftype{\isacharparenleft}{\kern0pt}x{\isacharparenright}{\kern0pt}\ {\isacharequal}{\kern0pt}\ {\isadigit{0}}\ {\isasymand}\ ftype{\isacharparenleft}{\kern0pt}y{\isacharparenright}{\kern0pt}\ {\isacharequal}{\kern0pt}\ {\isadigit{1}}{\isachardoublequoteclose}\ {\isacharbar}{\kern0pt}\ {\isacharparenleft}{\kern0pt}b{\isacharparenright}{\kern0pt}\ {\isachardoublequoteopen}ftype{\isacharparenleft}{\kern0pt}x{\isacharparenright}{\kern0pt}\ {\isacharequal}{\kern0pt}\ {\isadigit{1}}\ {\isasymand}\ ftype{\isacharparenleft}{\kern0pt}y{\isacharparenright}{\kern0pt}\ {\isacharequal}{\kern0pt}\ {\isadigit{0}}{\isachardoublequoteclose}\isanewline
\ \ \ \ \ \ \isacommand{using}\isamarkupfalse%
\ \ frecR{\isacharunderscore}{\kern0pt}ftypeD{\isacharbrackleft}{\kern0pt}OF\ {\isacartoucheopen}frecR{\isacharparenleft}{\kern0pt}x{\isacharcomma}{\kern0pt}y{\isacharparenright}{\kern0pt}{\isacartoucheclose}{\isacharbrackright}{\kern0pt}\ \isacommand{by}\isamarkupfalse%
\ auto\isanewline
\ \ \ \ \isacommand{then}\isamarkupfalse%
\ \isacommand{show}\isamarkupfalse%
\ {\isacharquery}{\kern0pt}thesis\ \isacommand{proof}\isamarkupfalse%
{\isacharparenleft}{\kern0pt}cases{\isacharparenright}{\kern0pt}\isanewline
\ \ \ \ \ \ \isacommand{case}\isamarkupfalse%
\ b\isanewline
\ \ \ \ \ \ \isacommand{then}\isamarkupfalse%
\ \isanewline
\ \ \ \ \ \ \isacommand{have}\isamarkupfalse%
\ {\isachardoublequoteopen}type{\isacharunderscore}{\kern0pt}form{\isacharparenleft}{\kern0pt}y{\isacharparenright}{\kern0pt}\ {\isacharequal}{\kern0pt}\ {\isadigit{1}}{\isachardoublequoteclose}\ \isanewline
\ \ \ \ \ \ \ \ \isacommand{using}\isamarkupfalse%
\ type{\isacharunderscore}{\kern0pt}form{\isacharunderscore}{\kern0pt}def\ \isacommand{by}\isamarkupfalse%
\ simp\isanewline
\ \ \ \ \ \ \isacommand{from}\isamarkupfalse%
\ b\isanewline
\ \ \ \ \ \ \isacommand{have}\isamarkupfalse%
\ H{\isacharcolon}{\kern0pt}\ {\isachardoublequoteopen}name{\isadigit{2}}{\isacharparenleft}{\kern0pt}x{\isacharparenright}{\kern0pt}\ {\isacharequal}{\kern0pt}\ name{\isadigit{1}}{\isacharparenleft}{\kern0pt}y{\isacharparenright}{\kern0pt}\ {\isasymor}\ name{\isadigit{2}}{\isacharparenleft}{\kern0pt}x{\isacharparenright}{\kern0pt}\ {\isacharequal}{\kern0pt}\ name{\isadigit{2}}{\isacharparenleft}{\kern0pt}y{\isacharparenright}{\kern0pt}\ {\isachardoublequoteclose}\ {\isacharparenleft}{\kern0pt}\isakeyword{is}\ {\isachardoublequoteopen}{\isacharquery}{\kern0pt}{\isasymtau}\ {\isacharequal}{\kern0pt}\ {\isacharquery}{\kern0pt}{\isasymsigma}{\isacharprime}{\kern0pt}\ {\isasymor}\ {\isacharquery}{\kern0pt}{\isasymtau}\ {\isacharequal}{\kern0pt}\ {\isacharquery}{\kern0pt}{\isasymtau}{\isacharprime}{\kern0pt}{\isachardoublequoteclose}{\isacharparenright}{\kern0pt}\isanewline
\ \ \ \ \ \ \ \ {\isachardoublequoteopen}name{\isadigit{1}}{\isacharparenleft}{\kern0pt}x{\isacharparenright}{\kern0pt}\ {\isasymin}\ domain{\isacharparenleft}{\kern0pt}name{\isadigit{1}}{\isacharparenleft}{\kern0pt}y{\isacharparenright}{\kern0pt}{\isacharparenright}{\kern0pt}\ {\isasymunion}\ domain{\isacharparenleft}{\kern0pt}name{\isadigit{2}}{\isacharparenleft}{\kern0pt}y{\isacharparenright}{\kern0pt}{\isacharparenright}{\kern0pt}{\isachardoublequoteclose}\ \isanewline
\ \ \ \ \ \ \ \ {\isacharparenleft}{\kern0pt}\isakeyword{is}\ {\isachardoublequoteopen}{\isacharquery}{\kern0pt}{\isasymsigma}\ {\isasymin}\ domain{\isacharparenleft}{\kern0pt}{\isacharquery}{\kern0pt}{\isasymsigma}{\isacharprime}{\kern0pt}{\isacharparenright}{\kern0pt}\ {\isasymunion}\ domain{\isacharparenleft}{\kern0pt}{\isacharquery}{\kern0pt}{\isasymtau}{\isacharprime}{\kern0pt}{\isacharparenright}{\kern0pt}{\isachardoublequoteclose}{\isacharparenright}{\kern0pt}\isanewline
\ \ \ \ \ \ \ \ \isacommand{using}\isamarkupfalse%
\ assms\ \isacommand{unfolding}\isamarkupfalse%
\ type{\isacharunderscore}{\kern0pt}form{\isacharunderscore}{\kern0pt}def\ frecR{\isacharunderscore}{\kern0pt}def\ \isacommand{by}\isamarkupfalse%
\ auto\isanewline
\ \ \ \ \ \ \isacommand{then}\isamarkupfalse%
\ \isanewline
\ \ \ \ \ \ \isacommand{have}\isamarkupfalse%
\ E{\isacharcolon}{\kern0pt}\ {\isachardoublequoteopen}rank{\isacharparenleft}{\kern0pt}{\isacharquery}{\kern0pt}{\isasymtau}{\isacharparenright}{\kern0pt}\ {\isacharequal}{\kern0pt}\ rank{\isacharparenleft}{\kern0pt}{\isacharquery}{\kern0pt}{\isasymsigma}{\isacharprime}{\kern0pt}{\isacharparenright}{\kern0pt}\ {\isasymor}\ rank{\isacharparenleft}{\kern0pt}{\isacharquery}{\kern0pt}{\isasymtau}{\isacharparenright}{\kern0pt}\ {\isacharequal}{\kern0pt}\ rank{\isacharparenleft}{\kern0pt}{\isacharquery}{\kern0pt}{\isasymtau}{\isacharprime}{\kern0pt}{\isacharparenright}{\kern0pt}{\isachardoublequoteclose}\ \isacommand{by}\isamarkupfalse%
\ auto\isanewline
\ \ \ \ \ \ \isacommand{from}\isamarkupfalse%
\ H\isanewline
\ \ \ \ \ \ \isacommand{consider}\isamarkupfalse%
\ {\isacharparenleft}{\kern0pt}a{\isacharparenright}{\kern0pt}\ {\isachardoublequoteopen}rank{\isacharparenleft}{\kern0pt}{\isacharquery}{\kern0pt}{\isasymsigma}{\isacharparenright}{\kern0pt}\ {\isacharless}{\kern0pt}\ rank{\isacharparenleft}{\kern0pt}{\isacharquery}{\kern0pt}{\isasymsigma}{\isacharprime}{\kern0pt}{\isacharparenright}{\kern0pt}{\isachardoublequoteclose}\ {\isacharbar}{\kern0pt}\ \ {\isacharparenleft}{\kern0pt}b{\isacharparenright}{\kern0pt}\ {\isachardoublequoteopen}rank{\isacharparenleft}{\kern0pt}{\isacharquery}{\kern0pt}{\isasymsigma}{\isacharparenright}{\kern0pt}\ {\isacharless}{\kern0pt}\ rank{\isacharparenleft}{\kern0pt}{\isacharquery}{\kern0pt}{\isasymtau}{\isacharprime}{\kern0pt}{\isacharparenright}{\kern0pt}{\isachardoublequoteclose}\isanewline
\ \ \ \ \ \ \ \ \isacommand{using}\isamarkupfalse%
\ eclose{\isacharunderscore}{\kern0pt}rank{\isacharunderscore}{\kern0pt}lt\ in{\isacharunderscore}{\kern0pt}dom{\isacharunderscore}{\kern0pt}in{\isacharunderscore}{\kern0pt}eclose\ \isacommand{by}\isamarkupfalse%
\ force\isanewline
\ \ \ \ \ \ \isacommand{then}\isamarkupfalse%
\isanewline
\ \ \ \ \ \ \isacommand{have}\isamarkupfalse%
\ {\isachardoublequoteopen}rank{\isacharparenleft}{\kern0pt}{\isacharquery}{\kern0pt}{\isasymsigma}{\isacharparenright}{\kern0pt}\ {\isacharless}{\kern0pt}\ rank{\isacharparenleft}{\kern0pt}{\isacharquery}{\kern0pt}{\isasymtau}{\isacharparenright}{\kern0pt}{\isachardoublequoteclose}\ \isacommand{proof}\isamarkupfalse%
\ {\isacharparenleft}{\kern0pt}cases{\isacharparenright}{\kern0pt}\isanewline
\ \ \ \ \ \ \ \ \isacommand{case}\isamarkupfalse%
\ a\isanewline
\ \ \ \ \ \ \ \ \isacommand{with}\isamarkupfalse%
\ {\isacartoucheopen}rank{\isacharunderscore}{\kern0pt}names{\isacharparenleft}{\kern0pt}x{\isacharparenright}{\kern0pt}\ {\isacharequal}{\kern0pt}\ rank{\isacharunderscore}{\kern0pt}names{\isacharparenleft}{\kern0pt}y{\isacharparenright}{\kern0pt}\ {\isacartoucheclose}\isanewline
\ \ \ \ \ \ \ \ \isacommand{show}\isamarkupfalse%
\ {\isacharquery}{\kern0pt}thesis\ \isacommand{unfolding}\isamarkupfalse%
\ rank{\isacharunderscore}{\kern0pt}names{\isacharunderscore}{\kern0pt}def\ mtype{\isacharunderscore}{\kern0pt}form{\isacharunderscore}{\kern0pt}def\ type{\isacharunderscore}{\kern0pt}form{\isacharunderscore}{\kern0pt}def\ \isacommand{using}\isamarkupfalse%
\ max{\isacharunderscore}{\kern0pt}D{\isadigit{2}}{\isacharbrackleft}{\kern0pt}OF\ E\ a{\isacharbrackright}{\kern0pt}\isanewline
\ \ \ \ \ \ \ \ \ \ \ \ E\ assms\ Ord{\isacharunderscore}{\kern0pt}rank\ \isacommand{by}\isamarkupfalse%
\ simp\isanewline
\ \ \ \ \ \ \isacommand{next}\isamarkupfalse%
\isanewline
\ \ \ \ \ \ \ \ \isacommand{case}\isamarkupfalse%
\ b\isanewline
\ \ \ \ \ \ \ \ \isacommand{with}\isamarkupfalse%
\ {\isacartoucheopen}rank{\isacharunderscore}{\kern0pt}names{\isacharparenleft}{\kern0pt}x{\isacharparenright}{\kern0pt}\ {\isacharequal}{\kern0pt}\ rank{\isacharunderscore}{\kern0pt}names{\isacharparenleft}{\kern0pt}y{\isacharparenright}{\kern0pt}\ {\isacartoucheclose}\isanewline
\ \ \ \ \ \ \ \ \isacommand{show}\isamarkupfalse%
\ {\isacharquery}{\kern0pt}thesis\ \isacommand{unfolding}\isamarkupfalse%
\ rank{\isacharunderscore}{\kern0pt}names{\isacharunderscore}{\kern0pt}def\ mtype{\isacharunderscore}{\kern0pt}form{\isacharunderscore}{\kern0pt}def\ type{\isacharunderscore}{\kern0pt}form{\isacharunderscore}{\kern0pt}def\ \isanewline
\ \ \ \ \ \ \ \ \ \ \isacommand{using}\isamarkupfalse%
\ max{\isacharunderscore}{\kern0pt}D{\isadigit{2}}{\isacharbrackleft}{\kern0pt}OF\ {\isacharunderscore}{\kern0pt}\ b{\isacharbrackright}{\kern0pt}\ max{\isacharunderscore}{\kern0pt}commutes\ E\ assms\ Ord{\isacharunderscore}{\kern0pt}rank\ disj{\isacharunderscore}{\kern0pt}commute\ \isacommand{by}\isamarkupfalse%
\ auto\isanewline
\ \ \ \ \ \ \isacommand{qed}\isamarkupfalse%
\isanewline
\ \ \ \ \ \ \isacommand{with}\isamarkupfalse%
\ b\isanewline
\ \ \ \ \ \ \isacommand{have}\isamarkupfalse%
\ {\isachardoublequoteopen}type{\isacharunderscore}{\kern0pt}form{\isacharparenleft}{\kern0pt}x{\isacharparenright}{\kern0pt}\ {\isacharequal}{\kern0pt}\ {\isadigit{0}}{\isachardoublequoteclose}\ \isacommand{unfolding}\isamarkupfalse%
\ type{\isacharunderscore}{\kern0pt}form{\isacharunderscore}{\kern0pt}def\ mtype{\isacharunderscore}{\kern0pt}form{\isacharunderscore}{\kern0pt}def\ \isacommand{by}\isamarkupfalse%
\ simp\isanewline
\ \ \ \ \ \ \isacommand{with}\isamarkupfalse%
\ {\isacartoucheopen}rank{\isacharunderscore}{\kern0pt}names{\isacharparenleft}{\kern0pt}x{\isacharparenright}{\kern0pt}\ {\isacharequal}{\kern0pt}\ rank{\isacharunderscore}{\kern0pt}names{\isacharparenleft}{\kern0pt}y{\isacharparenright}{\kern0pt}\ {\isacartoucheclose}\ {\isacartoucheopen}type{\isacharunderscore}{\kern0pt}form{\isacharparenleft}{\kern0pt}y{\isacharparenright}{\kern0pt}\ {\isacharequal}{\kern0pt}\ {\isadigit{1}}{\isacartoucheclose}\ {\isacartoucheopen}type{\isacharunderscore}{\kern0pt}form{\isacharparenleft}{\kern0pt}x{\isacharparenright}{\kern0pt}\ {\isacharequal}{\kern0pt}\ {\isadigit{0}}{\isacartoucheclose}\isanewline
\ \ \ \ \ \ \isacommand{show}\isamarkupfalse%
\ {\isacharquery}{\kern0pt}thesis\ \isanewline
\ \ \ \ \ \ \ \ \isacommand{unfolding}\isamarkupfalse%
\ {\isasymGamma}{\isacharunderscore}{\kern0pt}def\ \isacommand{by}\isamarkupfalse%
\ auto\isanewline
\ \ \ \ \isacommand{next}\isamarkupfalse%
\isanewline
\ \ \ \ \ \ \isacommand{case}\isamarkupfalse%
\ a\isanewline
\ \ \ \ \ \ \isacommand{then}\isamarkupfalse%
\ \isanewline
\ \ \ \ \ \ \isacommand{have}\isamarkupfalse%
\ {\isachardoublequoteopen}name{\isadigit{1}}{\isacharparenleft}{\kern0pt}x{\isacharparenright}{\kern0pt}\ {\isacharequal}{\kern0pt}\ name{\isadigit{1}}{\isacharparenleft}{\kern0pt}y{\isacharparenright}{\kern0pt}{\isachardoublequoteclose}\ {\isacharparenleft}{\kern0pt}\isakeyword{is}\ {\isachardoublequoteopen}{\isacharquery}{\kern0pt}{\isasymsigma}\ {\isacharequal}{\kern0pt}\ {\isacharquery}{\kern0pt}{\isasymsigma}{\isacharprime}{\kern0pt}{\isachardoublequoteclose}{\isacharparenright}{\kern0pt}\ \isanewline
\ \ \ \ \ \ \ \ {\isachardoublequoteopen}name{\isadigit{2}}{\isacharparenleft}{\kern0pt}x{\isacharparenright}{\kern0pt}\ {\isasymin}\ domain{\isacharparenleft}{\kern0pt}name{\isadigit{2}}{\isacharparenleft}{\kern0pt}y{\isacharparenright}{\kern0pt}{\isacharparenright}{\kern0pt}{\isachardoublequoteclose}\ {\isacharparenleft}{\kern0pt}\isakeyword{is}\ {\isachardoublequoteopen}{\isacharquery}{\kern0pt}{\isasymtau}\ {\isasymin}\ domain{\isacharparenleft}{\kern0pt}{\isacharquery}{\kern0pt}{\isasymtau}{\isacharprime}{\kern0pt}{\isacharparenright}{\kern0pt}{\isachardoublequoteclose}{\isacharparenright}{\kern0pt}\isanewline
\ \ \ \ \ \ \ \ {\isachardoublequoteopen}type{\isacharunderscore}{\kern0pt}form{\isacharparenleft}{\kern0pt}x{\isacharparenright}{\kern0pt}\ {\isacharequal}{\kern0pt}\ {\isadigit{1}}{\isachardoublequoteclose}\isanewline
\ \ \ \ \ \ \ \ \isacommand{using}\isamarkupfalse%
\ assms\ \isacommand{unfolding}\isamarkupfalse%
\ type{\isacharunderscore}{\kern0pt}form{\isacharunderscore}{\kern0pt}def\ frecR{\isacharunderscore}{\kern0pt}def\ \isacommand{by}\isamarkupfalse%
\ auto\isanewline
\ \ \ \ \ \ \isacommand{then}\isamarkupfalse%
\isanewline
\ \ \ \ \ \ \isacommand{have}\isamarkupfalse%
\ {\isachardoublequoteopen}rank{\isacharparenleft}{\kern0pt}{\isacharquery}{\kern0pt}{\isasymsigma}{\isacharparenright}{\kern0pt}\ {\isacharequal}{\kern0pt}\ rank{\isacharparenleft}{\kern0pt}{\isacharquery}{\kern0pt}{\isasymsigma}{\isacharprime}{\kern0pt}{\isacharparenright}{\kern0pt}{\isachardoublequoteclose}\ {\isachardoublequoteopen}rank{\isacharparenleft}{\kern0pt}{\isacharquery}{\kern0pt}{\isasymtau}{\isacharparenright}{\kern0pt}\ {\isacharless}{\kern0pt}\ rank{\isacharparenleft}{\kern0pt}{\isacharquery}{\kern0pt}{\isasymtau}{\isacharprime}{\kern0pt}{\isacharparenright}{\kern0pt}{\isachardoublequoteclose}\ \isanewline
\ \ \ \ \ \ \ \ \isacommand{using}\isamarkupfalse%
\ \ eclose{\isacharunderscore}{\kern0pt}rank{\isacharunderscore}{\kern0pt}lt\ in{\isacharunderscore}{\kern0pt}dom{\isacharunderscore}{\kern0pt}in{\isacharunderscore}{\kern0pt}eclose\ \isacommand{by}\isamarkupfalse%
\ simp{\isacharunderscore}{\kern0pt}all\isanewline
\ \ \ \ \ \ \isacommand{with}\isamarkupfalse%
\ {\isacartoucheopen}rank{\isacharunderscore}{\kern0pt}names{\isacharparenleft}{\kern0pt}x{\isacharparenright}{\kern0pt}\ {\isacharequal}{\kern0pt}\ rank{\isacharunderscore}{\kern0pt}names{\isacharparenleft}{\kern0pt}y{\isacharparenright}{\kern0pt}\ {\isacartoucheclose}\ \isanewline
\ \ \ \ \ \ \isacommand{have}\isamarkupfalse%
\ {\isachardoublequoteopen}rank{\isacharparenleft}{\kern0pt}{\isacharquery}{\kern0pt}{\isasymtau}{\isacharprime}{\kern0pt}{\isacharparenright}{\kern0pt}\ {\isasymle}\ rank{\isacharparenleft}{\kern0pt}{\isacharquery}{\kern0pt}{\isasymsigma}{\isacharprime}{\kern0pt}{\isacharparenright}{\kern0pt}{\isachardoublequoteclose}\ \isanewline
\ \ \ \ \ \ \ \ \isacommand{unfolding}\isamarkupfalse%
\ rank{\isacharunderscore}{\kern0pt}names{\isacharunderscore}{\kern0pt}def\ \isacommand{using}\isamarkupfalse%
\ Ord{\isacharunderscore}{\kern0pt}rank\ max{\isacharunderscore}{\kern0pt}D{\isadigit{1}}\ \isacommand{by}\isamarkupfalse%
\ simp\isanewline
\ \ \ \ \ \ \isacommand{with}\isamarkupfalse%
\ a\isanewline
\ \ \ \ \ \ \isacommand{have}\isamarkupfalse%
\ {\isachardoublequoteopen}type{\isacharunderscore}{\kern0pt}form{\isacharparenleft}{\kern0pt}y{\isacharparenright}{\kern0pt}\ {\isacharequal}{\kern0pt}\ {\isadigit{2}}{\isachardoublequoteclose}\isanewline
\ \ \ \ \ \ \ \ \isacommand{unfolding}\isamarkupfalse%
\ type{\isacharunderscore}{\kern0pt}form{\isacharunderscore}{\kern0pt}def\ mtype{\isacharunderscore}{\kern0pt}form{\isacharunderscore}{\kern0pt}def\ \isacommand{using}\isamarkupfalse%
\ not{\isacharunderscore}{\kern0pt}lt{\isacharunderscore}{\kern0pt}iff{\isacharunderscore}{\kern0pt}le\ assms\ \isacommand{by}\isamarkupfalse%
\ simp\isanewline
\ \ \ \ \ \ \isacommand{with}\isamarkupfalse%
\ {\isacartoucheopen}rank{\isacharunderscore}{\kern0pt}names{\isacharparenleft}{\kern0pt}x{\isacharparenright}{\kern0pt}\ {\isacharequal}{\kern0pt}\ rank{\isacharunderscore}{\kern0pt}names{\isacharparenleft}{\kern0pt}y{\isacharparenright}{\kern0pt}\ {\isacartoucheclose}\ {\isacartoucheopen}type{\isacharunderscore}{\kern0pt}form{\isacharparenleft}{\kern0pt}y{\isacharparenright}{\kern0pt}\ {\isacharequal}{\kern0pt}\ {\isadigit{2}}{\isacartoucheclose}\ {\isacartoucheopen}type{\isacharunderscore}{\kern0pt}form{\isacharparenleft}{\kern0pt}x{\isacharparenright}{\kern0pt}\ {\isacharequal}{\kern0pt}\ {\isadigit{1}}{\isacartoucheclose}\isanewline
\ \ \ \ \ \ \isacommand{show}\isamarkupfalse%
\ {\isacharquery}{\kern0pt}thesis\ \isanewline
\ \ \ \ \ \ \ \ \isacommand{unfolding}\isamarkupfalse%
\ {\isasymGamma}{\isacharunderscore}{\kern0pt}def\ \isacommand{by}\isamarkupfalse%
\ auto\isanewline
\ \ \ \ \isacommand{qed}\isamarkupfalse%
\isanewline
\ \ \isacommand{qed}\isamarkupfalse%
\isanewline
\isacommand{qed}\isamarkupfalse%
%
\endisatagproof
{\isafoldproof}%
%
\isadelimproof
\isanewline
%
\endisadelimproof
\isanewline
\isacommand{definition}\isamarkupfalse%
\isanewline
\ \ frecrel\ {\isacharcolon}{\kern0pt}{\isacharcolon}{\kern0pt}\ {\isachardoublequoteopen}i\ {\isasymRightarrow}\ i{\isachardoublequoteclose}\ \isakeyword{where}\isanewline
\ \ {\isachardoublequoteopen}frecrel{\isacharparenleft}{\kern0pt}A{\isacharparenright}{\kern0pt}\ {\isasymequiv}\ Rrel{\isacharparenleft}{\kern0pt}frecR{\isacharcomma}{\kern0pt}A{\isacharparenright}{\kern0pt}{\isachardoublequoteclose}\isanewline
\isanewline
\isacommand{lemma}\isamarkupfalse%
\ frecrelI\ {\isacharcolon}{\kern0pt}\ \isanewline
\ \ \isakeyword{assumes}\ {\isachardoublequoteopen}x\ {\isasymin}\ A{\isachardoublequoteclose}\ {\isachardoublequoteopen}y{\isasymin}A{\isachardoublequoteclose}\ {\isachardoublequoteopen}frecR{\isacharparenleft}{\kern0pt}x{\isacharcomma}{\kern0pt}y{\isacharparenright}{\kern0pt}{\isachardoublequoteclose}\isanewline
\ \ \isakeyword{shows}\ {\isachardoublequoteopen}{\isasymlangle}x{\isacharcomma}{\kern0pt}y{\isasymrangle}{\isasymin}frecrel{\isacharparenleft}{\kern0pt}A{\isacharparenright}{\kern0pt}{\isachardoublequoteclose}\isanewline
%
\isadelimproof
\ \ %
\endisadelimproof
%
\isatagproof
\isacommand{using}\isamarkupfalse%
\ assms\ \isacommand{unfolding}\isamarkupfalse%
\ frecrel{\isacharunderscore}{\kern0pt}def\ Rrel{\isacharunderscore}{\kern0pt}def\ \isacommand{by}\isamarkupfalse%
\ auto%
\endisatagproof
{\isafoldproof}%
%
\isadelimproof
\isanewline
%
\endisadelimproof
\isanewline
\isacommand{lemma}\isamarkupfalse%
\ frecrelD\ {\isacharcolon}{\kern0pt}\isanewline
\ \ \isakeyword{assumes}\ {\isachardoublequoteopen}{\isasymlangle}x{\isacharcomma}{\kern0pt}y{\isasymrangle}\ {\isasymin}\ frecrel{\isacharparenleft}{\kern0pt}A{\isadigit{1}}{\isasymtimes}A{\isadigit{2}}{\isasymtimes}A{\isadigit{3}}{\isasymtimes}A{\isadigit{4}}{\isacharparenright}{\kern0pt}{\isachardoublequoteclose}\isanewline
\ \ \isakeyword{shows}\ {\isachardoublequoteopen}ftype{\isacharparenleft}{\kern0pt}x{\isacharparenright}{\kern0pt}\ {\isasymin}\ A{\isadigit{1}}{\isachardoublequoteclose}\ {\isachardoublequoteopen}ftype{\isacharparenleft}{\kern0pt}x{\isacharparenright}{\kern0pt}\ {\isasymin}\ A{\isadigit{1}}{\isachardoublequoteclose}\isanewline
\ \ \ \ {\isachardoublequoteopen}name{\isadigit{1}}{\isacharparenleft}{\kern0pt}x{\isacharparenright}{\kern0pt}\ {\isasymin}\ A{\isadigit{2}}{\isachardoublequoteclose}\ {\isachardoublequoteopen}name{\isadigit{1}}{\isacharparenleft}{\kern0pt}y{\isacharparenright}{\kern0pt}\ {\isasymin}\ A{\isadigit{2}}{\isachardoublequoteclose}\ {\isachardoublequoteopen}name{\isadigit{2}}{\isacharparenleft}{\kern0pt}x{\isacharparenright}{\kern0pt}\ {\isasymin}\ A{\isadigit{3}}{\isachardoublequoteclose}\ {\isachardoublequoteopen}name{\isadigit{2}}{\isacharparenleft}{\kern0pt}x{\isacharparenright}{\kern0pt}\ {\isasymin}\ A{\isadigit{3}}{\isachardoublequoteclose}\ \isanewline
\ \ \ \ {\isachardoublequoteopen}cond{\isacharunderscore}{\kern0pt}of{\isacharparenleft}{\kern0pt}x{\isacharparenright}{\kern0pt}\ {\isasymin}\ A{\isadigit{4}}{\isachardoublequoteclose}\ {\isachardoublequoteopen}cond{\isacharunderscore}{\kern0pt}of{\isacharparenleft}{\kern0pt}y{\isacharparenright}{\kern0pt}\ {\isasymin}\ A{\isadigit{4}}{\isachardoublequoteclose}\ \isanewline
\ \ \ \ {\isachardoublequoteopen}frecR{\isacharparenleft}{\kern0pt}x{\isacharcomma}{\kern0pt}y{\isacharparenright}{\kern0pt}{\isachardoublequoteclose}\isanewline
%
\isadelimproof
\ \ %
\endisadelimproof
%
\isatagproof
\isacommand{using}\isamarkupfalse%
\ assms\ \isacommand{unfolding}\isamarkupfalse%
\ frecrel{\isacharunderscore}{\kern0pt}def\ Rrel{\isacharunderscore}{\kern0pt}def\ ftype{\isacharunderscore}{\kern0pt}def\ \isacommand{by}\isamarkupfalse%
\ {\isacharparenleft}{\kern0pt}auto\ simp\ add{\isacharcolon}{\kern0pt}components{\isacharunderscore}{\kern0pt}simp{\isacharparenright}{\kern0pt}%
\endisatagproof
{\isafoldproof}%
%
\isadelimproof
\isanewline
%
\endisadelimproof
\isanewline
\isacommand{lemma}\isamarkupfalse%
\ wf{\isacharunderscore}{\kern0pt}frecrel\ {\isacharcolon}{\kern0pt}\ \isanewline
\ \ \isakeyword{shows}\ {\isachardoublequoteopen}wf{\isacharparenleft}{\kern0pt}frecrel{\isacharparenleft}{\kern0pt}A{\isacharparenright}{\kern0pt}{\isacharparenright}{\kern0pt}{\isachardoublequoteclose}\isanewline
%
\isadelimproof
%
\endisadelimproof
%
\isatagproof
\isacommand{proof}\isamarkupfalse%
\ {\isacharminus}{\kern0pt}\isanewline
\ \ \isacommand{have}\isamarkupfalse%
\ {\isachardoublequoteopen}frecrel{\isacharparenleft}{\kern0pt}A{\isacharparenright}{\kern0pt}\ {\isasymsubseteq}\ measure{\isacharparenleft}{\kern0pt}A{\isacharcomma}{\kern0pt}{\isasymGamma}{\isacharparenright}{\kern0pt}{\isachardoublequoteclose}\isanewline
\ \ \ \ \isacommand{unfolding}\isamarkupfalse%
\ frecrel{\isacharunderscore}{\kern0pt}def\ Rrel{\isacharunderscore}{\kern0pt}def\ measure{\isacharunderscore}{\kern0pt}def\isanewline
\ \ \ \ \isacommand{using}\isamarkupfalse%
\ {\isasymGamma}{\isacharunderscore}{\kern0pt}mono\ \isacommand{by}\isamarkupfalse%
\ force\isanewline
\ \ \isacommand{then}\isamarkupfalse%
\ \isacommand{show}\isamarkupfalse%
\ {\isacharquery}{\kern0pt}thesis\ \isacommand{using}\isamarkupfalse%
\ wf{\isacharunderscore}{\kern0pt}subset\ wf{\isacharunderscore}{\kern0pt}measure\ \isacommand{by}\isamarkupfalse%
\ auto\isanewline
\isacommand{qed}\isamarkupfalse%
%
\endisatagproof
{\isafoldproof}%
%
\isadelimproof
\isanewline
%
\endisadelimproof
\isanewline
\isacommand{lemma}\isamarkupfalse%
\ core{\isacharunderscore}{\kern0pt}induction{\isacharunderscore}{\kern0pt}aux{\isacharcolon}{\kern0pt}\isanewline
\ \ \isakeyword{fixes}\ A{\isadigit{1}}\ A{\isadigit{2}}\ {\isacharcolon}{\kern0pt}{\isacharcolon}{\kern0pt}\ {\isachardoublequoteopen}i{\isachardoublequoteclose}\isanewline
\ \ \isakeyword{assumes}\isanewline
\ \ \ \ {\isachardoublequoteopen}Transset{\isacharparenleft}{\kern0pt}A{\isadigit{1}}{\isacharparenright}{\kern0pt}{\isachardoublequoteclose}\isanewline
\ \ \ \ {\isachardoublequoteopen}{\isasymAnd}{\isasymtau}\ {\isasymtheta}\ p{\isachardot}{\kern0pt}\ \ p\ {\isasymin}\ A{\isadigit{2}}\ {\isasymLongrightarrow}\ {\isasymlbrakk}{\isasymAnd}q\ {\isasymsigma}{\isachardot}{\kern0pt}\ {\isasymlbrakk}\ q{\isasymin}A{\isadigit{2}}\ {\isacharsemicolon}{\kern0pt}\ {\isasymsigma}{\isasymin}domain{\isacharparenleft}{\kern0pt}{\isasymtheta}{\isacharparenright}{\kern0pt}{\isasymrbrakk}\ {\isasymLongrightarrow}\ Q{\isacharparenleft}{\kern0pt}{\isadigit{0}}{\isacharcomma}{\kern0pt}{\isasymtau}{\isacharcomma}{\kern0pt}{\isasymsigma}{\isacharcomma}{\kern0pt}q{\isacharparenright}{\kern0pt}{\isasymrbrakk}\ {\isasymLongrightarrow}\ Q{\isacharparenleft}{\kern0pt}{\isadigit{1}}{\isacharcomma}{\kern0pt}{\isasymtau}{\isacharcomma}{\kern0pt}{\isasymtheta}{\isacharcomma}{\kern0pt}p{\isacharparenright}{\kern0pt}{\isachardoublequoteclose}\isanewline
\ \ \ \ {\isachardoublequoteopen}{\isasymAnd}{\isasymtau}\ {\isasymtheta}\ p{\isachardot}{\kern0pt}\ \ p\ {\isasymin}\ A{\isadigit{2}}\ {\isasymLongrightarrow}\ {\isasymlbrakk}{\isasymAnd}q\ {\isasymsigma}{\isachardot}{\kern0pt}\ {\isasymlbrakk}\ q{\isasymin}A{\isadigit{2}}\ {\isacharsemicolon}{\kern0pt}\ {\isasymsigma}{\isasymin}domain{\isacharparenleft}{\kern0pt}{\isasymtau}{\isacharparenright}{\kern0pt}\ {\isasymunion}\ domain{\isacharparenleft}{\kern0pt}{\isasymtheta}{\isacharparenright}{\kern0pt}{\isasymrbrakk}\ {\isasymLongrightarrow}\ Q{\isacharparenleft}{\kern0pt}{\isadigit{1}}{\isacharcomma}{\kern0pt}{\isasymsigma}{\isacharcomma}{\kern0pt}{\isasymtau}{\isacharcomma}{\kern0pt}q{\isacharparenright}{\kern0pt}\ {\isasymand}\ Q{\isacharparenleft}{\kern0pt}{\isadigit{1}}{\isacharcomma}{\kern0pt}{\isasymsigma}{\isacharcomma}{\kern0pt}{\isasymtheta}{\isacharcomma}{\kern0pt}q{\isacharparenright}{\kern0pt}{\isasymrbrakk}\ {\isasymLongrightarrow}\ Q{\isacharparenleft}{\kern0pt}{\isadigit{0}}{\isacharcomma}{\kern0pt}{\isasymtau}{\isacharcomma}{\kern0pt}{\isasymtheta}{\isacharcomma}{\kern0pt}p{\isacharparenright}{\kern0pt}{\isachardoublequoteclose}\isanewline
\ \ \isakeyword{shows}\ {\isachardoublequoteopen}a{\isasymin}{\isadigit{2}}{\isasymtimes}A{\isadigit{1}}{\isasymtimes}A{\isadigit{1}}{\isasymtimes}A{\isadigit{2}}\ {\isasymLongrightarrow}\ Q{\isacharparenleft}{\kern0pt}ftype{\isacharparenleft}{\kern0pt}a{\isacharparenright}{\kern0pt}{\isacharcomma}{\kern0pt}name{\isadigit{1}}{\isacharparenleft}{\kern0pt}a{\isacharparenright}{\kern0pt}{\isacharcomma}{\kern0pt}name{\isadigit{2}}{\isacharparenleft}{\kern0pt}a{\isacharparenright}{\kern0pt}{\isacharcomma}{\kern0pt}cond{\isacharunderscore}{\kern0pt}of{\isacharparenleft}{\kern0pt}a{\isacharparenright}{\kern0pt}{\isacharparenright}{\kern0pt}{\isachardoublequoteclose}\isanewline
%
\isadelimproof
%
\endisadelimproof
%
\isatagproof
\isacommand{proof}\isamarkupfalse%
\ {\isacharparenleft}{\kern0pt}induct\ a\ rule{\isacharcolon}{\kern0pt}wf{\isacharunderscore}{\kern0pt}induct{\isacharbrackleft}{\kern0pt}OF\ wf{\isacharunderscore}{\kern0pt}frecrel{\isacharbrackleft}{\kern0pt}of\ {\isachardoublequoteopen}{\isadigit{2}}{\isasymtimes}A{\isadigit{1}}{\isasymtimes}A{\isadigit{1}}{\isasymtimes}A{\isadigit{2}}{\isachardoublequoteclose}{\isacharbrackright}{\kern0pt}{\isacharbrackright}{\kern0pt}{\isacharparenright}{\kern0pt}\isanewline
\ \ \isacommand{case}\isamarkupfalse%
\ {\isacharparenleft}{\kern0pt}{\isadigit{1}}\ x{\isacharparenright}{\kern0pt}\isanewline
\ \ \isacommand{let}\isamarkupfalse%
\ {\isacharquery}{\kern0pt}{\isasymtau}\ {\isacharequal}{\kern0pt}\ {\isachardoublequoteopen}name{\isadigit{1}}{\isacharparenleft}{\kern0pt}x{\isacharparenright}{\kern0pt}{\isachardoublequoteclose}\ \isanewline
\ \ \isacommand{let}\isamarkupfalse%
\ {\isacharquery}{\kern0pt}{\isasymtheta}\ {\isacharequal}{\kern0pt}\ {\isachardoublequoteopen}name{\isadigit{2}}{\isacharparenleft}{\kern0pt}x{\isacharparenright}{\kern0pt}{\isachardoublequoteclose}\isanewline
\ \ \isacommand{let}\isamarkupfalse%
\ {\isacharquery}{\kern0pt}D\ {\isacharequal}{\kern0pt}\ {\isachardoublequoteopen}{\isadigit{2}}{\isasymtimes}A{\isadigit{1}}{\isasymtimes}A{\isadigit{1}}{\isasymtimes}A{\isadigit{2}}{\isachardoublequoteclose}\isanewline
\ \ \isacommand{assume}\isamarkupfalse%
\ {\isachardoublequoteopen}x\ {\isasymin}\ {\isacharquery}{\kern0pt}D{\isachardoublequoteclose}\isanewline
\ \ \isacommand{then}\isamarkupfalse%
\isanewline
\ \ \isacommand{have}\isamarkupfalse%
\ {\isachardoublequoteopen}cond{\isacharunderscore}{\kern0pt}of{\isacharparenleft}{\kern0pt}x{\isacharparenright}{\kern0pt}{\isasymin}A{\isadigit{2}}{\isachardoublequoteclose}\ \isanewline
\ \ \ \ \isacommand{by}\isamarkupfalse%
\ {\isacharparenleft}{\kern0pt}auto\ simp\ add{\isacharcolon}{\kern0pt}components{\isacharunderscore}{\kern0pt}simp{\isacharparenright}{\kern0pt}\isanewline
\ \ \isacommand{from}\isamarkupfalse%
\ {\isacartoucheopen}x{\isasymin}{\isacharquery}{\kern0pt}D{\isacartoucheclose}\isanewline
\ \ \isacommand{consider}\isamarkupfalse%
\ {\isacharparenleft}{\kern0pt}eq{\isacharparenright}{\kern0pt}\ {\isachardoublequoteopen}ftype{\isacharparenleft}{\kern0pt}x{\isacharparenright}{\kern0pt}{\isacharequal}{\kern0pt}{\isadigit{0}}{\isachardoublequoteclose}\ {\isacharbar}{\kern0pt}\ {\isacharparenleft}{\kern0pt}mem{\isacharparenright}{\kern0pt}\ {\isachardoublequoteopen}ftype{\isacharparenleft}{\kern0pt}x{\isacharparenright}{\kern0pt}{\isacharequal}{\kern0pt}{\isadigit{1}}{\isachardoublequoteclose}\isanewline
\ \ \ \ \isacommand{by}\isamarkupfalse%
\ {\isacharparenleft}{\kern0pt}auto\ simp\ add{\isacharcolon}{\kern0pt}components{\isacharunderscore}{\kern0pt}simp{\isacharparenright}{\kern0pt}\isanewline
\ \ \isacommand{then}\isamarkupfalse%
\ \isanewline
\ \ \isacommand{show}\isamarkupfalse%
\ {\isacharquery}{\kern0pt}case\ \isanewline
\ \ \isacommand{proof}\isamarkupfalse%
\ cases\isanewline
\ \ \ \ \isacommand{case}\isamarkupfalse%
\ eq\isanewline
\ \ \ \ \isacommand{then}\isamarkupfalse%
\ \isanewline
\ \ \ \ \isacommand{have}\isamarkupfalse%
\ {\isachardoublequoteopen}Q{\isacharparenleft}{\kern0pt}{\isadigit{1}}{\isacharcomma}{\kern0pt}\ {\isasymsigma}{\isacharcomma}{\kern0pt}\ {\isacharquery}{\kern0pt}{\isasymtau}{\isacharcomma}{\kern0pt}\ q{\isacharparenright}{\kern0pt}\ {\isasymand}\ Q{\isacharparenleft}{\kern0pt}{\isadigit{1}}{\isacharcomma}{\kern0pt}\ {\isasymsigma}{\isacharcomma}{\kern0pt}\ {\isacharquery}{\kern0pt}{\isasymtheta}{\isacharcomma}{\kern0pt}\ q{\isacharparenright}{\kern0pt}{\isachardoublequoteclose}\ \isakeyword{if}\ {\isachardoublequoteopen}{\isasymsigma}\ {\isasymin}\ domain{\isacharparenleft}{\kern0pt}{\isacharquery}{\kern0pt}{\isasymtau}{\isacharparenright}{\kern0pt}\ {\isasymunion}\ domain{\isacharparenleft}{\kern0pt}{\isacharquery}{\kern0pt}{\isasymtheta}{\isacharparenright}{\kern0pt}{\isachardoublequoteclose}\ \isakeyword{and}\ {\isachardoublequoteopen}q{\isasymin}A{\isadigit{2}}{\isachardoublequoteclose}\ \isakeyword{for}\ q\ {\isasymsigma}\isanewline
\ \ \ \ \isacommand{proof}\isamarkupfalse%
\ {\isacharminus}{\kern0pt}\isanewline
\ \ \ \ \ \ \isacommand{from}\isamarkupfalse%
\ {\isadigit{1}}\isanewline
\ \ \ \ \ \ \isacommand{have}\isamarkupfalse%
\ A{\isacharcolon}{\kern0pt}\ {\isachardoublequoteopen}{\isacharquery}{\kern0pt}{\isasymtau}{\isasymin}A{\isadigit{1}}{\isachardoublequoteclose}\ {\isachardoublequoteopen}{\isacharquery}{\kern0pt}{\isasymtheta}{\isasymin}A{\isadigit{1}}{\isachardoublequoteclose}\ {\isachardoublequoteopen}{\isacharquery}{\kern0pt}{\isasymtau}{\isasymin}eclose{\isacharparenleft}{\kern0pt}A{\isadigit{1}}{\isacharparenright}{\kern0pt}{\isachardoublequoteclose}\ {\isachardoublequoteopen}{\isacharquery}{\kern0pt}{\isasymtheta}{\isasymin}eclose{\isacharparenleft}{\kern0pt}A{\isadigit{1}}{\isacharparenright}{\kern0pt}{\isachardoublequoteclose}\isanewline
\ \ \ \ \ \ \ \ \isacommand{using}\isamarkupfalse%
\ \ arg{\isacharunderscore}{\kern0pt}into{\isacharunderscore}{\kern0pt}eclose\ \isacommand{by}\isamarkupfalse%
\ {\isacharparenleft}{\kern0pt}auto\ simp\ add{\isacharcolon}{\kern0pt}components{\isacharunderscore}{\kern0pt}simp{\isacharparenright}{\kern0pt}\isanewline
\ \ \ \ \ \ \isacommand{with}\isamarkupfalse%
\ \ {\isacartoucheopen}Transset{\isacharparenleft}{\kern0pt}A{\isadigit{1}}{\isacharparenright}{\kern0pt}{\isacartoucheclose}\ that{\isacharparenleft}{\kern0pt}{\isadigit{1}}{\isacharparenright}{\kern0pt}\isanewline
\ \ \ \ \ \ \isacommand{have}\isamarkupfalse%
\ {\isachardoublequoteopen}{\isasymsigma}{\isasymin}eclose{\isacharparenleft}{\kern0pt}{\isacharquery}{\kern0pt}{\isasymtau}{\isacharparenright}{\kern0pt}\ {\isasymunion}\ eclose{\isacharparenleft}{\kern0pt}{\isacharquery}{\kern0pt}{\isasymtheta}{\isacharparenright}{\kern0pt}{\isachardoublequoteclose}\ \isanewline
\ \ \ \ \ \ \ \ \isacommand{using}\isamarkupfalse%
\ in{\isacharunderscore}{\kern0pt}dom{\isacharunderscore}{\kern0pt}in{\isacharunderscore}{\kern0pt}eclose\ \ \isacommand{by}\isamarkupfalse%
\ auto\isanewline
\ \ \ \ \ \ \isacommand{then}\isamarkupfalse%
\isanewline
\ \ \ \ \ \ \isacommand{have}\isamarkupfalse%
\ {\isachardoublequoteopen}{\isasymsigma}{\isasymin}A{\isadigit{1}}{\isachardoublequoteclose}\isanewline
\ \ \ \ \ \ \ \ \isacommand{using}\isamarkupfalse%
\ mem{\isacharunderscore}{\kern0pt}eclose{\isacharunderscore}{\kern0pt}subset{\isacharbrackleft}{\kern0pt}OF\ {\isacartoucheopen}{\isacharquery}{\kern0pt}{\isasymtau}{\isasymin}A{\isadigit{1}}{\isacartoucheclose}{\isacharbrackright}{\kern0pt}\ mem{\isacharunderscore}{\kern0pt}eclose{\isacharunderscore}{\kern0pt}subset{\isacharbrackleft}{\kern0pt}OF\ {\isacartoucheopen}{\isacharquery}{\kern0pt}{\isasymtheta}{\isasymin}A{\isadigit{1}}{\isacartoucheclose}{\isacharbrackright}{\kern0pt}\ \isanewline
\ \ \ \ \ \ \ \ \ \ Transset{\isacharunderscore}{\kern0pt}eclose{\isacharunderscore}{\kern0pt}eq{\isacharunderscore}{\kern0pt}arg{\isacharbrackleft}{\kern0pt}OF\ {\isacartoucheopen}Transset{\isacharparenleft}{\kern0pt}A{\isadigit{1}}{\isacharparenright}{\kern0pt}{\isacartoucheclose}{\isacharbrackright}{\kern0pt}\ \isanewline
\ \ \ \ \ \ \ \ \isacommand{by}\isamarkupfalse%
\ auto\ \ \ \ \ \ \ \ \ \isanewline
\ \ \ \ \ \ \isacommand{with}\isamarkupfalse%
\ {\isacartoucheopen}q{\isasymin}A{\isadigit{2}}{\isacartoucheclose}\ {\isacartoucheopen}{\isacharquery}{\kern0pt}{\isasymtheta}\ {\isasymin}\ A{\isadigit{1}}{\isacartoucheclose}\ {\isacartoucheopen}cond{\isacharunderscore}{\kern0pt}of{\isacharparenleft}{\kern0pt}x{\isacharparenright}{\kern0pt}{\isasymin}A{\isadigit{2}}{\isacartoucheclose}\ {\isacartoucheopen}{\isacharquery}{\kern0pt}{\isasymtau}{\isasymin}A{\isadigit{1}}{\isacartoucheclose}\isanewline
\ \ \ \ \ \ \isacommand{have}\isamarkupfalse%
\ {\isachardoublequoteopen}frecR{\isacharparenleft}{\kern0pt}{\isasymlangle}{\isadigit{1}}{\isacharcomma}{\kern0pt}\ {\isasymsigma}{\isacharcomma}{\kern0pt}\ {\isacharquery}{\kern0pt}{\isasymtau}{\isacharcomma}{\kern0pt}\ q{\isasymrangle}{\isacharcomma}{\kern0pt}\ x{\isacharparenright}{\kern0pt}{\isachardoublequoteclose}\ {\isacharparenleft}{\kern0pt}\isakeyword{is}\ {\isachardoublequoteopen}frecR{\isacharparenleft}{\kern0pt}{\isacharquery}{\kern0pt}T{\isacharcomma}{\kern0pt}{\isacharunderscore}{\kern0pt}{\isacharparenright}{\kern0pt}{\isachardoublequoteclose}{\isacharparenright}{\kern0pt}\isanewline
\ \ \ \ \ \ \ \ {\isachardoublequoteopen}frecR{\isacharparenleft}{\kern0pt}{\isasymlangle}{\isadigit{1}}{\isacharcomma}{\kern0pt}\ {\isasymsigma}{\isacharcomma}{\kern0pt}\ {\isacharquery}{\kern0pt}{\isasymtheta}{\isacharcomma}{\kern0pt}\ q{\isasymrangle}{\isacharcomma}{\kern0pt}\ x{\isacharparenright}{\kern0pt}{\isachardoublequoteclose}\ {\isacharparenleft}{\kern0pt}\isakeyword{is}\ {\isachardoublequoteopen}frecR{\isacharparenleft}{\kern0pt}{\isacharquery}{\kern0pt}U{\isacharcomma}{\kern0pt}{\isacharunderscore}{\kern0pt}{\isacharparenright}{\kern0pt}{\isachardoublequoteclose}{\isacharparenright}{\kern0pt}\isanewline
\ \ \ \ \ \ \ \ \isacommand{using}\isamarkupfalse%
\ \ frecRI{\isadigit{1}}{\isacharprime}{\kern0pt}{\isacharbrackleft}{\kern0pt}OF\ that{\isacharparenleft}{\kern0pt}{\isadigit{1}}{\isacharparenright}{\kern0pt}{\isacharbrackright}{\kern0pt}\ frecR{\isacharunderscore}{\kern0pt}DI\ \ {\isacartoucheopen}ftype{\isacharparenleft}{\kern0pt}x{\isacharparenright}{\kern0pt}\ {\isacharequal}{\kern0pt}\ {\isadigit{0}}{\isacartoucheclose}\ \isanewline
\ \ \ \ \ \ \ \ \ \ frecRI{\isadigit{2}}{\isacharprime}{\kern0pt}{\isacharbrackleft}{\kern0pt}OF\ that{\isacharparenleft}{\kern0pt}{\isadigit{1}}{\isacharparenright}{\kern0pt}{\isacharbrackright}{\kern0pt}\ \isanewline
\ \ \ \ \ \ \ \ \isacommand{by}\isamarkupfalse%
\ {\isacharparenleft}{\kern0pt}auto\ simp\ add{\isacharcolon}{\kern0pt}components{\isacharunderscore}{\kern0pt}simp{\isacharparenright}{\kern0pt}\isanewline
\ \ \ \ \ \ \isacommand{with}\isamarkupfalse%
\ {\isacartoucheopen}x{\isasymin}{\isacharquery}{\kern0pt}D{\isacartoucheclose}\ {\isacartoucheopen}{\isasymsigma}{\isasymin}A{\isadigit{1}}{\isacartoucheclose}\ {\isacartoucheopen}q{\isasymin}A{\isadigit{2}}{\isacartoucheclose}\isanewline
\ \ \ \ \ \ \isacommand{have}\isamarkupfalse%
\ {\isachardoublequoteopen}{\isasymlangle}{\isacharquery}{\kern0pt}T{\isacharcomma}{\kern0pt}x{\isasymrangle}{\isasymin}\ frecrel{\isacharparenleft}{\kern0pt}{\isacharquery}{\kern0pt}D{\isacharparenright}{\kern0pt}{\isachardoublequoteclose}\ {\isachardoublequoteopen}{\isasymlangle}{\isacharquery}{\kern0pt}U{\isacharcomma}{\kern0pt}x{\isasymrangle}{\isasymin}\ frecrel{\isacharparenleft}{\kern0pt}{\isacharquery}{\kern0pt}D{\isacharparenright}{\kern0pt}{\isachardoublequoteclose}\ \isanewline
\ \ \ \ \ \ \ \ \isacommand{using}\isamarkupfalse%
\ frecrelI{\isacharbrackleft}{\kern0pt}of\ {\isacharquery}{\kern0pt}T\ {\isacharquery}{\kern0pt}D\ x{\isacharbrackright}{\kern0pt}\ \ frecrelI{\isacharbrackleft}{\kern0pt}of\ {\isacharquery}{\kern0pt}U\ {\isacharquery}{\kern0pt}D\ x{\isacharbrackright}{\kern0pt}\ \isacommand{by}\isamarkupfalse%
\ {\isacharparenleft}{\kern0pt}auto\ simp\ add{\isacharcolon}{\kern0pt}components{\isacharunderscore}{\kern0pt}simp{\isacharparenright}{\kern0pt}\isanewline
\ \ \ \ \ \ \isacommand{with}\isamarkupfalse%
\ {\isacartoucheopen}q{\isasymin}A{\isadigit{2}}{\isacartoucheclose}\ {\isacartoucheopen}{\isasymsigma}{\isasymin}A{\isadigit{1}}{\isacartoucheclose}\ {\isacartoucheopen}{\isacharquery}{\kern0pt}{\isasymtau}{\isasymin}A{\isadigit{1}}{\isacartoucheclose}\ {\isacartoucheopen}{\isacharquery}{\kern0pt}{\isasymtheta}{\isasymin}A{\isadigit{1}}{\isacartoucheclose}\isanewline
\ \ \ \ \ \ \isacommand{have}\isamarkupfalse%
\ {\isachardoublequoteopen}Q{\isacharparenleft}{\kern0pt}{\isadigit{1}}{\isacharcomma}{\kern0pt}\ {\isasymsigma}{\isacharcomma}{\kern0pt}\ {\isacharquery}{\kern0pt}{\isasymtau}{\isacharcomma}{\kern0pt}\ q{\isacharparenright}{\kern0pt}{\isachardoublequoteclose}\ \isacommand{using}\isamarkupfalse%
\ {\isadigit{1}}\ \isacommand{by}\isamarkupfalse%
\ {\isacharparenleft}{\kern0pt}force\ simp\ add{\isacharcolon}{\kern0pt}components{\isacharunderscore}{\kern0pt}simp{\isacharparenright}{\kern0pt}\isanewline
\ \ \ \ \ \ \isacommand{moreover}\isamarkupfalse%
\ \isacommand{from}\isamarkupfalse%
\ {\isacartoucheopen}q{\isasymin}A{\isadigit{2}}{\isacartoucheclose}\ {\isacartoucheopen}{\isasymsigma}{\isasymin}A{\isadigit{1}}{\isacartoucheclose}\ {\isacartoucheopen}{\isacharquery}{\kern0pt}{\isasymtau}{\isasymin}A{\isadigit{1}}{\isacartoucheclose}\ {\isacartoucheopen}{\isacharquery}{\kern0pt}{\isasymtheta}{\isasymin}A{\isadigit{1}}{\isacartoucheclose}\ {\isacartoucheopen}{\isasymlangle}{\isacharquery}{\kern0pt}U{\isacharcomma}{\kern0pt}x{\isasymrangle}{\isasymin}\ frecrel{\isacharparenleft}{\kern0pt}{\isacharquery}{\kern0pt}D{\isacharparenright}{\kern0pt}{\isacartoucheclose}\isanewline
\ \ \ \ \ \ \isacommand{have}\isamarkupfalse%
\ {\isachardoublequoteopen}Q{\isacharparenleft}{\kern0pt}{\isadigit{1}}{\isacharcomma}{\kern0pt}\ {\isasymsigma}{\isacharcomma}{\kern0pt}\ {\isacharquery}{\kern0pt}{\isasymtheta}{\isacharcomma}{\kern0pt}\ q{\isacharparenright}{\kern0pt}{\isachardoublequoteclose}\ \isacommand{using}\isamarkupfalse%
\ {\isadigit{1}}\ \isacommand{by}\isamarkupfalse%
\ {\isacharparenleft}{\kern0pt}force\ simp\ add{\isacharcolon}{\kern0pt}components{\isacharunderscore}{\kern0pt}simp{\isacharparenright}{\kern0pt}\isanewline
\ \ \ \ \ \ \isacommand{ultimately}\isamarkupfalse%
\isanewline
\ \ \ \ \ \ \isacommand{show}\isamarkupfalse%
\ {\isacharquery}{\kern0pt}thesis\ \isacommand{using}\isamarkupfalse%
\ A\ \isacommand{by}\isamarkupfalse%
\ simp\isanewline
\ \ \ \ \isacommand{qed}\isamarkupfalse%
\isanewline
\ \ \ \ \isacommand{then}\isamarkupfalse%
\ \isacommand{show}\isamarkupfalse%
\ {\isacharquery}{\kern0pt}thesis\ \isacommand{using}\isamarkupfalse%
\ assms{\isacharparenleft}{\kern0pt}{\isadigit{3}}{\isacharparenright}{\kern0pt}\ {\isacartoucheopen}ftype{\isacharparenleft}{\kern0pt}x{\isacharparenright}{\kern0pt}\ {\isacharequal}{\kern0pt}\ {\isadigit{0}}{\isacartoucheclose}\ {\isacartoucheopen}cond{\isacharunderscore}{\kern0pt}of{\isacharparenleft}{\kern0pt}x{\isacharparenright}{\kern0pt}{\isasymin}A{\isadigit{2}}{\isacartoucheclose}\ \isacommand{by}\isamarkupfalse%
\ auto\isanewline
\ \ \isacommand{next}\isamarkupfalse%
\isanewline
\ \ \ \ \isacommand{case}\isamarkupfalse%
\ mem\isanewline
\ \ \ \ \isacommand{have}\isamarkupfalse%
\ {\isachardoublequoteopen}Q{\isacharparenleft}{\kern0pt}{\isadigit{0}}{\isacharcomma}{\kern0pt}\ {\isacharquery}{\kern0pt}{\isasymtau}{\isacharcomma}{\kern0pt}\ \ {\isasymsigma}{\isacharcomma}{\kern0pt}\ q{\isacharparenright}{\kern0pt}{\isachardoublequoteclose}\ \isakeyword{if}\ {\isachardoublequoteopen}{\isasymsigma}\ {\isasymin}\ domain{\isacharparenleft}{\kern0pt}{\isacharquery}{\kern0pt}{\isasymtheta}{\isacharparenright}{\kern0pt}{\isachardoublequoteclose}\ \isakeyword{and}\ {\isachardoublequoteopen}q{\isasymin}A{\isadigit{2}}{\isachardoublequoteclose}\ \isakeyword{for}\ q\ {\isasymsigma}\isanewline
\ \ \ \ \isacommand{proof}\isamarkupfalse%
\ {\isacharminus}{\kern0pt}\isanewline
\ \ \ \ \ \ \isacommand{from}\isamarkupfalse%
\ {\isadigit{1}}\ assms\isanewline
\ \ \ \ \ \ \isacommand{have}\isamarkupfalse%
\ {\isachardoublequoteopen}{\isacharquery}{\kern0pt}{\isasymtau}{\isasymin}A{\isadigit{1}}{\isachardoublequoteclose}\ {\isachardoublequoteopen}{\isacharquery}{\kern0pt}{\isasymtheta}{\isasymin}A{\isadigit{1}}{\isachardoublequoteclose}\ {\isachardoublequoteopen}cond{\isacharunderscore}{\kern0pt}of{\isacharparenleft}{\kern0pt}x{\isacharparenright}{\kern0pt}{\isasymin}A{\isadigit{2}}{\isachardoublequoteclose}\ {\isachardoublequoteopen}{\isacharquery}{\kern0pt}{\isasymtau}{\isasymin}eclose{\isacharparenleft}{\kern0pt}A{\isadigit{1}}{\isacharparenright}{\kern0pt}{\isachardoublequoteclose}\ {\isachardoublequoteopen}{\isacharquery}{\kern0pt}{\isasymtheta}{\isasymin}eclose{\isacharparenleft}{\kern0pt}A{\isadigit{1}}{\isacharparenright}{\kern0pt}{\isachardoublequoteclose}\isanewline
\ \ \ \ \ \ \ \ \isacommand{using}\isamarkupfalse%
\ \ arg{\isacharunderscore}{\kern0pt}into{\isacharunderscore}{\kern0pt}eclose\ \isacommand{by}\isamarkupfalse%
\ {\isacharparenleft}{\kern0pt}auto\ simp\ add{\isacharcolon}{\kern0pt}components{\isacharunderscore}{\kern0pt}simp{\isacharparenright}{\kern0pt}\isanewline
\ \ \ \ \ \ \isacommand{with}\isamarkupfalse%
\ \ {\isacartoucheopen}Transset{\isacharparenleft}{\kern0pt}A{\isadigit{1}}{\isacharparenright}{\kern0pt}{\isacartoucheclose}\ that{\isacharparenleft}{\kern0pt}{\isadigit{1}}{\isacharparenright}{\kern0pt}\isanewline
\ \ \ \ \ \ \isacommand{have}\isamarkupfalse%
\ {\isachardoublequoteopen}{\isasymsigma}{\isasymin}\ eclose{\isacharparenleft}{\kern0pt}{\isacharquery}{\kern0pt}{\isasymtheta}{\isacharparenright}{\kern0pt}{\isachardoublequoteclose}\ \isanewline
\ \ \ \ \ \ \ \ \isacommand{using}\isamarkupfalse%
\ in{\isacharunderscore}{\kern0pt}dom{\isacharunderscore}{\kern0pt}in{\isacharunderscore}{\kern0pt}eclose\ \ \isacommand{by}\isamarkupfalse%
\ auto\isanewline
\ \ \ \ \ \ \isacommand{then}\isamarkupfalse%
\isanewline
\ \ \ \ \ \ \isacommand{have}\isamarkupfalse%
\ {\isachardoublequoteopen}{\isasymsigma}{\isasymin}A{\isadigit{1}}{\isachardoublequoteclose}\isanewline
\ \ \ \ \ \ \ \ \isacommand{using}\isamarkupfalse%
\ mem{\isacharunderscore}{\kern0pt}eclose{\isacharunderscore}{\kern0pt}subset{\isacharbrackleft}{\kern0pt}OF\ {\isacartoucheopen}{\isacharquery}{\kern0pt}{\isasymtheta}{\isasymin}A{\isadigit{1}}{\isacartoucheclose}{\isacharbrackright}{\kern0pt}\ Transset{\isacharunderscore}{\kern0pt}eclose{\isacharunderscore}{\kern0pt}eq{\isacharunderscore}{\kern0pt}arg{\isacharbrackleft}{\kern0pt}OF\ {\isacartoucheopen}Transset{\isacharparenleft}{\kern0pt}A{\isadigit{1}}{\isacharparenright}{\kern0pt}{\isacartoucheclose}{\isacharbrackright}{\kern0pt}\ \isanewline
\ \ \ \ \ \ \ \ \isacommand{by}\isamarkupfalse%
\ auto\ \ \ \ \ \ \ \ \ \isanewline
\ \ \ \ \ \ \isacommand{with}\isamarkupfalse%
\ {\isacartoucheopen}q{\isasymin}A{\isadigit{2}}{\isacartoucheclose}\ {\isacartoucheopen}{\isacharquery}{\kern0pt}{\isasymtheta}\ {\isasymin}\ A{\isadigit{1}}{\isacartoucheclose}\ {\isacartoucheopen}cond{\isacharunderscore}{\kern0pt}of{\isacharparenleft}{\kern0pt}x{\isacharparenright}{\kern0pt}{\isasymin}A{\isadigit{2}}{\isacartoucheclose}\ {\isacartoucheopen}{\isacharquery}{\kern0pt}{\isasymtau}{\isasymin}A{\isadigit{1}}{\isacartoucheclose}\isanewline
\ \ \ \ \ \ \isacommand{have}\isamarkupfalse%
\ {\isachardoublequoteopen}frecR{\isacharparenleft}{\kern0pt}{\isasymlangle}{\isadigit{0}}{\isacharcomma}{\kern0pt}\ {\isacharquery}{\kern0pt}{\isasymtau}{\isacharcomma}{\kern0pt}\ {\isasymsigma}{\isacharcomma}{\kern0pt}\ q{\isasymrangle}{\isacharcomma}{\kern0pt}\ x{\isacharparenright}{\kern0pt}{\isachardoublequoteclose}\ {\isacharparenleft}{\kern0pt}\isakeyword{is}\ {\isachardoublequoteopen}frecR{\isacharparenleft}{\kern0pt}{\isacharquery}{\kern0pt}T{\isacharcomma}{\kern0pt}{\isacharunderscore}{\kern0pt}{\isacharparenright}{\kern0pt}{\isachardoublequoteclose}{\isacharparenright}{\kern0pt}\isanewline
\ \ \ \ \ \ \ \ \isacommand{using}\isamarkupfalse%
\ \ frecRI{\isadigit{3}}{\isacharprime}{\kern0pt}{\isacharbrackleft}{\kern0pt}OF\ that{\isacharparenleft}{\kern0pt}{\isadigit{1}}{\isacharparenright}{\kern0pt}{\isacharbrackright}{\kern0pt}\ frecR{\isacharunderscore}{\kern0pt}DI\ \ {\isacartoucheopen}ftype{\isacharparenleft}{\kern0pt}x{\isacharparenright}{\kern0pt}\ {\isacharequal}{\kern0pt}\ {\isadigit{1}}{\isacartoucheclose}\ \ \ \ \ \ \ \ \ \ \ \ \ \ \ \ \ \isanewline
\ \ \ \ \ \ \ \ \isacommand{by}\isamarkupfalse%
\ {\isacharparenleft}{\kern0pt}auto\ simp\ add{\isacharcolon}{\kern0pt}components{\isacharunderscore}{\kern0pt}simp{\isacharparenright}{\kern0pt}\isanewline
\ \ \ \ \ \ \isacommand{with}\isamarkupfalse%
\ {\isacartoucheopen}x{\isasymin}{\isacharquery}{\kern0pt}D{\isacartoucheclose}\ {\isacartoucheopen}{\isasymsigma}{\isasymin}A{\isadigit{1}}{\isacartoucheclose}\ {\isacartoucheopen}q{\isasymin}A{\isadigit{2}}{\isacartoucheclose}\ {\isacartoucheopen}{\isacharquery}{\kern0pt}{\isasymtau}{\isasymin}A{\isadigit{1}}{\isacartoucheclose}\isanewline
\ \ \ \ \ \ \isacommand{have}\isamarkupfalse%
\ {\isachardoublequoteopen}{\isasymlangle}{\isacharquery}{\kern0pt}T{\isacharcomma}{\kern0pt}x{\isasymrangle}{\isasymin}\ frecrel{\isacharparenleft}{\kern0pt}{\isacharquery}{\kern0pt}D{\isacharparenright}{\kern0pt}{\isachardoublequoteclose}\ {\isachardoublequoteopen}{\isacharquery}{\kern0pt}T{\isasymin}{\isacharquery}{\kern0pt}D{\isachardoublequoteclose}\isanewline
\ \ \ \ \ \ \ \ \isacommand{using}\isamarkupfalse%
\ frecrelI{\isacharbrackleft}{\kern0pt}of\ {\isacharquery}{\kern0pt}T\ {\isacharquery}{\kern0pt}D\ x{\isacharbrackright}{\kern0pt}\ \isacommand{by}\isamarkupfalse%
\ {\isacharparenleft}{\kern0pt}auto\ simp\ add{\isacharcolon}{\kern0pt}components{\isacharunderscore}{\kern0pt}simp{\isacharparenright}{\kern0pt}\isanewline
\ \ \ \ \ \ \isacommand{with}\isamarkupfalse%
\ {\isacartoucheopen}q{\isasymin}A{\isadigit{2}}{\isacartoucheclose}\ {\isacartoucheopen}{\isasymsigma}{\isasymin}A{\isadigit{1}}{\isacartoucheclose}\ {\isacartoucheopen}{\isacharquery}{\kern0pt}{\isasymtau}{\isasymin}A{\isadigit{1}}{\isacartoucheclose}\ {\isacartoucheopen}{\isacharquery}{\kern0pt}{\isasymtheta}{\isasymin}A{\isadigit{1}}{\isacartoucheclose}\ {\isadigit{1}}\isanewline
\ \ \ \ \ \ \isacommand{show}\isamarkupfalse%
\ {\isacharquery}{\kern0pt}thesis\ \isacommand{by}\isamarkupfalse%
\ {\isacharparenleft}{\kern0pt}force\ simp\ add{\isacharcolon}{\kern0pt}components{\isacharunderscore}{\kern0pt}simp{\isacharparenright}{\kern0pt}\isanewline
\ \ \ \ \isacommand{qed}\isamarkupfalse%
\isanewline
\ \ \ \ \isacommand{then}\isamarkupfalse%
\ \isacommand{show}\isamarkupfalse%
\ {\isacharquery}{\kern0pt}thesis\ \isacommand{using}\isamarkupfalse%
\ assms{\isacharparenleft}{\kern0pt}{\isadigit{2}}{\isacharparenright}{\kern0pt}\ {\isacartoucheopen}ftype{\isacharparenleft}{\kern0pt}x{\isacharparenright}{\kern0pt}\ {\isacharequal}{\kern0pt}\ {\isadigit{1}}{\isacartoucheclose}\ {\isacartoucheopen}cond{\isacharunderscore}{\kern0pt}of{\isacharparenleft}{\kern0pt}x{\isacharparenright}{\kern0pt}{\isasymin}A{\isadigit{2}}{\isacartoucheclose}\ \ \isacommand{by}\isamarkupfalse%
\ auto\isanewline
\ \ \isacommand{qed}\isamarkupfalse%
\isanewline
\isacommand{qed}\isamarkupfalse%
%
\endisatagproof
{\isafoldproof}%
%
\isadelimproof
\isanewline
%
\endisadelimproof
\isanewline
\isacommand{lemma}\isamarkupfalse%
\ def{\isacharunderscore}{\kern0pt}frecrel\ {\isacharcolon}{\kern0pt}\ {\isachardoublequoteopen}frecrel{\isacharparenleft}{\kern0pt}A{\isacharparenright}{\kern0pt}\ {\isacharequal}{\kern0pt}\ {\isacharbraceleft}{\kern0pt}z{\isasymin}A{\isasymtimes}A{\isachardot}{\kern0pt}\ {\isasymexists}x\ y{\isachardot}{\kern0pt}\ z\ {\isacharequal}{\kern0pt}\ {\isasymlangle}x{\isacharcomma}{\kern0pt}\ y{\isasymrangle}\ {\isasymand}\ frecR{\isacharparenleft}{\kern0pt}x{\isacharcomma}{\kern0pt}y{\isacharparenright}{\kern0pt}{\isacharbraceright}{\kern0pt}{\isachardoublequoteclose}\isanewline
%
\isadelimproof
\ \ %
\endisadelimproof
%
\isatagproof
\isacommand{unfolding}\isamarkupfalse%
\ frecrel{\isacharunderscore}{\kern0pt}def\ Rrel{\isacharunderscore}{\kern0pt}def\ \isacommand{{\isachardot}{\kern0pt}{\isachardot}{\kern0pt}}\isamarkupfalse%
%
\endisatagproof
{\isafoldproof}%
%
\isadelimproof
\isanewline
%
\endisadelimproof
\isanewline
\isacommand{lemma}\isamarkupfalse%
\ frecrel{\isacharunderscore}{\kern0pt}fst{\isacharunderscore}{\kern0pt}snd{\isacharcolon}{\kern0pt}\isanewline
\ \ {\isachardoublequoteopen}frecrel{\isacharparenleft}{\kern0pt}A{\isacharparenright}{\kern0pt}\ {\isacharequal}{\kern0pt}\ {\isacharbraceleft}{\kern0pt}z\ {\isasymin}\ A{\isasymtimes}A\ {\isachardot}{\kern0pt}\ \isanewline
\ \ \ \ \ \ \ \ \ \ \ \ ftype{\isacharparenleft}{\kern0pt}fst{\isacharparenleft}{\kern0pt}z{\isacharparenright}{\kern0pt}{\isacharparenright}{\kern0pt}\ {\isacharequal}{\kern0pt}\ {\isadigit{1}}\ {\isasymand}\ \isanewline
\ \ \ \ \ \ \ \ ftype{\isacharparenleft}{\kern0pt}snd{\isacharparenleft}{\kern0pt}z{\isacharparenright}{\kern0pt}{\isacharparenright}{\kern0pt}\ {\isacharequal}{\kern0pt}\ {\isadigit{0}}\ {\isasymand}\ name{\isadigit{1}}{\isacharparenleft}{\kern0pt}fst{\isacharparenleft}{\kern0pt}z{\isacharparenright}{\kern0pt}{\isacharparenright}{\kern0pt}\ {\isasymin}\ domain{\isacharparenleft}{\kern0pt}name{\isadigit{1}}{\isacharparenleft}{\kern0pt}snd{\isacharparenleft}{\kern0pt}z{\isacharparenright}{\kern0pt}{\isacharparenright}{\kern0pt}{\isacharparenright}{\kern0pt}\ {\isasymunion}\ domain{\isacharparenleft}{\kern0pt}name{\isadigit{2}}{\isacharparenleft}{\kern0pt}snd{\isacharparenleft}{\kern0pt}z{\isacharparenright}{\kern0pt}{\isacharparenright}{\kern0pt}{\isacharparenright}{\kern0pt}\ {\isasymand}\ \isanewline
\ \ \ \ \ \ \ \ \ \ \ \ {\isacharparenleft}{\kern0pt}name{\isadigit{2}}{\isacharparenleft}{\kern0pt}fst{\isacharparenleft}{\kern0pt}z{\isacharparenright}{\kern0pt}{\isacharparenright}{\kern0pt}\ {\isacharequal}{\kern0pt}\ name{\isadigit{1}}{\isacharparenleft}{\kern0pt}snd{\isacharparenleft}{\kern0pt}z{\isacharparenright}{\kern0pt}{\isacharparenright}{\kern0pt}\ {\isasymor}\ name{\isadigit{2}}{\isacharparenleft}{\kern0pt}fst{\isacharparenleft}{\kern0pt}z{\isacharparenright}{\kern0pt}{\isacharparenright}{\kern0pt}\ {\isacharequal}{\kern0pt}\ name{\isadigit{2}}{\isacharparenleft}{\kern0pt}snd{\isacharparenleft}{\kern0pt}z{\isacharparenright}{\kern0pt}{\isacharparenright}{\kern0pt}{\isacharparenright}{\kern0pt}\ \isanewline
\ \ \ \ \ \ \ \ \ \ {\isasymor}\ {\isacharparenleft}{\kern0pt}ftype{\isacharparenleft}{\kern0pt}fst{\isacharparenleft}{\kern0pt}z{\isacharparenright}{\kern0pt}{\isacharparenright}{\kern0pt}\ {\isacharequal}{\kern0pt}\ {\isadigit{0}}\ {\isasymand}\ \isanewline
\ \ \ \ \ \ \ \ ftype{\isacharparenleft}{\kern0pt}snd{\isacharparenleft}{\kern0pt}z{\isacharparenright}{\kern0pt}{\isacharparenright}{\kern0pt}\ {\isacharequal}{\kern0pt}\ {\isadigit{1}}\ {\isasymand}\ \ name{\isadigit{1}}{\isacharparenleft}{\kern0pt}fst{\isacharparenleft}{\kern0pt}z{\isacharparenright}{\kern0pt}{\isacharparenright}{\kern0pt}\ {\isacharequal}{\kern0pt}\ name{\isadigit{1}}{\isacharparenleft}{\kern0pt}snd{\isacharparenleft}{\kern0pt}z{\isacharparenright}{\kern0pt}{\isacharparenright}{\kern0pt}\ {\isasymand}\ name{\isadigit{2}}{\isacharparenleft}{\kern0pt}fst{\isacharparenleft}{\kern0pt}z{\isacharparenright}{\kern0pt}{\isacharparenright}{\kern0pt}\ {\isasymin}\ domain{\isacharparenleft}{\kern0pt}name{\isadigit{2}}{\isacharparenleft}{\kern0pt}snd{\isacharparenleft}{\kern0pt}z{\isacharparenright}{\kern0pt}{\isacharparenright}{\kern0pt}{\isacharparenright}{\kern0pt}{\isacharparenright}{\kern0pt}{\isacharbraceright}{\kern0pt}{\isachardoublequoteclose}\isanewline
%
\isadelimproof
\ \ %
\endisadelimproof
%
\isatagproof
\isacommand{unfolding}\isamarkupfalse%
\ def{\isacharunderscore}{\kern0pt}frecrel\ frecR{\isacharunderscore}{\kern0pt}def\isanewline
\ \ \isacommand{by}\isamarkupfalse%
\ {\isacharparenleft}{\kern0pt}intro\ equalityI\ subsetI\ CollectI{\isacharsemicolon}{\kern0pt}\ elim\ CollectE{\isacharsemicolon}{\kern0pt}\ auto{\isacharparenright}{\kern0pt}%
\endisatagproof
{\isafoldproof}%
%
\isadelimproof
\isanewline
%
\endisadelimproof
%
\isadelimtheory
\isanewline
%
\endisadelimtheory
%
\isatagtheory
\isacommand{end}\isamarkupfalse%
%
\endisatagtheory
{\isafoldtheory}%
%
\isadelimtheory
%
\endisadelimtheory
%
\end{isabellebody}%
\endinput
%:%file=~/source/repos/ZF-notAC/code/Forcing/FrecR.thy%:%
%:%11=1%:%
%:%27=2%:%
%:%28=2%:%
%:%33=2%:%
%:%36=3%:%
%:%37=4%:%
%:%38=4%:%
%:%39=5%:%
%:%41=7%:%
%:%42=8%:%
%:%44=11%:%
%:%45=11%:%
%:%46=12%:%
%:%47=13%:%
%:%48=14%:%
%:%49=15%:%
%:%50=15%:%
%:%51=16%:%
%:%52=17%:%
%:%53=18%:%
%:%54=19%:%
%:%55=20%:%
%:%56=21%:%
%:%57=22%:%
%:%60=23%:%
%:%64=23%:%
%:%65=23%:%
%:%66=23%:%
%:%67=23%:%
%:%72=23%:%
%:%75=24%:%
%:%76=25%:%
%:%77=25%:%
%:%78=26%:%
%:%79=27%:%
%:%80=28%:%
%:%81=29%:%
%:%82=29%:%
%:%83=30%:%
%:%84=31%:%
%:%85=32%:%
%:%86=33%:%
%:%87=34%:%
%:%88=35%:%
%:%89=36%:%
%:%90=37%:%
%:%91=38%:%
%:%92=39%:%
%:%99=40%:%
%:%100=40%:%
%:%101=41%:%
%:%102=41%:%
%:%103=42%:%
%:%104=42%:%
%:%105=42%:%
%:%106=43%:%
%:%107=43%:%
%:%108=44%:%
%:%109=44%:%
%:%110=45%:%
%:%111=45%:%
%:%112=45%:%
%:%113=46%:%
%:%114=46%:%
%:%115=47%:%
%:%116=47%:%
%:%117=47%:%
%:%118=47%:%
%:%119=47%:%
%:%120=48%:%
%:%126=48%:%
%:%129=49%:%
%:%130=50%:%
%:%131=51%:%
%:%132=52%:%
%:%133=52%:%
%:%134=53%:%
%:%135=54%:%
%:%136=55%:%
%:%137=56%:%
%:%138=56%:%
%:%139=57%:%
%:%140=58%:%
%:%141=59%:%
%:%142=60%:%
%:%143=60%:%
%:%144=61%:%
%:%145=62%:%
%:%146=63%:%
%:%147=64%:%
%:%148=64%:%
%:%149=65%:%
%:%150=66%:%
%:%151=67%:%
%:%152=68%:%
%:%153=68%:%
%:%154=69%:%
%:%155=70%:%
%:%156=71%:%
%:%157=72%:%
%:%160=73%:%
%:%164=73%:%
%:%165=73%:%
%:%166=74%:%
%:%167=74%:%
%:%172=74%:%
%:%175=75%:%
%:%176=76%:%
%:%177=76%:%
%:%178=77%:%
%:%179=78%:%
%:%180=79%:%
%:%181=79%:%
%:%182=80%:%
%:%183=81%:%
%:%184=82%:%
%:%185=83%:%
%:%186=83%:%
%:%187=84%:%
%:%188=85%:%
%:%191=86%:%
%:%195=86%:%
%:%196=86%:%
%:%197=87%:%
%:%198=87%:%
%:%199=87%:%
%:%204=87%:%
%:%207=88%:%
%:%208=89%:%
%:%209=89%:%
%:%210=90%:%
%:%211=91%:%
%:%212=91%:%
%:%213=92%:%
%:%214=93%:%
%:%217=94%:%
%:%221=94%:%
%:%222=94%:%
%:%223=95%:%
%:%224=95%:%
%:%225=96%:%
%:%226=96%:%
%:%231=96%:%
%:%234=97%:%
%:%235=98%:%
%:%236=98%:%
%:%237=99%:%
%:%238=100%:%
%:%239=100%:%
%:%240=101%:%
%:%241=102%:%
%:%248=103%:%
%:%249=103%:%
%:%250=104%:%
%:%251=104%:%
%:%252=105%:%
%:%253=105%:%
%:%254=106%:%
%:%255=106%:%
%:%256=106%:%
%:%257=107%:%
%:%258=107%:%
%:%259=108%:%
%:%260=108%:%
%:%261=109%:%
%:%262=109%:%
%:%263=110%:%
%:%264=110%:%
%:%265=111%:%
%:%266=111%:%
%:%267=112%:%
%:%268=112%:%
%:%269=113%:%
%:%270=113%:%
%:%271=114%:%
%:%272=114%:%
%:%273=114%:%
%:%274=115%:%
%:%275=115%:%
%:%276=116%:%
%:%277=116%:%
%:%278=117%:%
%:%279=117%:%
%:%280=118%:%
%:%281=118%:%
%:%282=119%:%
%:%283=119%:%
%:%284=120%:%
%:%285=120%:%
%:%286=120%:%
%:%287=121%:%
%:%288=121%:%
%:%289=122%:%
%:%295=122%:%
%:%298=123%:%
%:%299=124%:%
%:%300=125%:%
%:%301=126%:%
%:%302=127%:%
%:%303=127%:%
%:%304=128%:%
%:%305=129%:%
%:%306=130%:%
%:%307=131%:%
%:%308=132%:%
%:%309=132%:%
%:%310=133%:%
%:%311=134%:%
%:%312=135%:%
%:%313=136%:%
%:%314=137%:%
%:%315=137%:%
%:%316=138%:%
%:%318=140%:%
%:%321=141%:%
%:%325=141%:%
%:%326=141%:%
%:%331=141%:%
%:%334=142%:%
%:%335=143%:%
%:%336=143%:%
%:%337=144%:%
%:%338=145%:%
%:%339=146%:%
%:%340=147%:%
%:%341=147%:%
%:%342=148%:%
%:%343=149%:%
%:%344=150%:%
%:%345=151%:%
%:%346=151%:%
%:%347=152%:%
%:%349=154%:%
%:%352=155%:%
%:%356=155%:%
%:%357=155%:%
%:%358=156%:%
%:%359=156%:%
%:%364=156%:%
%:%367=157%:%
%:%368=158%:%
%:%369=158%:%
%:%370=159%:%
%:%371=160%:%
%:%372=161%:%
%:%373=162%:%
%:%376=163%:%
%:%380=163%:%
%:%381=163%:%
%:%382=164%:%
%:%383=164%:%
%:%388=164%:%
%:%391=165%:%
%:%392=166%:%
%:%393=166%:%
%:%394=167%:%
%:%395=168%:%
%:%396=169%:%
%:%397=170%:%
%:%398=171%:%
%:%399=171%:%
%:%400=172%:%
%:%401=173%:%
%:%402=174%:%
%:%403=175%:%
%:%404=176%:%
%:%405=176%:%
%:%406=177%:%
%:%408=179%:%
%:%411=180%:%
%:%415=180%:%
%:%416=180%:%
%:%421=180%:%
%:%424=181%:%
%:%425=182%:%
%:%426=182%:%
%:%427=183%:%
%:%428=184%:%
%:%429=185%:%
%:%430=186%:%
%:%431=186%:%
%:%432=187%:%
%:%433=188%:%
%:%434=189%:%
%:%435=190%:%
%:%436=190%:%
%:%437=191%:%
%:%439=193%:%
%:%442=194%:%
%:%446=194%:%
%:%447=194%:%
%:%448=194%:%
%:%449=195%:%
%:%450=195%:%
%:%455=195%:%
%:%458=196%:%
%:%459=197%:%
%:%460=197%:%
%:%461=198%:%
%:%462=199%:%
%:%463=200%:%
%:%464=201%:%
%:%467=202%:%
%:%471=202%:%
%:%472=202%:%
%:%473=203%:%
%:%474=203%:%
%:%479=203%:%
%:%482=204%:%
%:%483=205%:%
%:%484=205%:%
%:%485=206%:%
%:%486=207%:%
%:%487=208%:%
%:%488=209%:%
%:%489=209%:%
%:%490=210%:%
%:%491=211%:%
%:%492=212%:%
%:%493=213%:%
%:%494=213%:%
%:%495=214%:%
%:%497=216%:%
%:%500=217%:%
%:%504=217%:%
%:%505=217%:%
%:%506=217%:%
%:%507=218%:%
%:%508=218%:%
%:%513=218%:%
%:%516=219%:%
%:%517=220%:%
%:%518=220%:%
%:%519=221%:%
%:%520=222%:%
%:%521=223%:%
%:%522=224%:%
%:%523=224%:%
%:%524=225%:%
%:%525=226%:%
%:%526=227%:%
%:%527=228%:%
%:%528=228%:%
%:%529=229%:%
%:%531=231%:%
%:%534=232%:%
%:%538=232%:%
%:%539=232%:%
%:%540=232%:%
%:%541=233%:%
%:%542=233%:%
%:%547=233%:%
%:%550=234%:%
%:%551=235%:%
%:%552=235%:%
%:%553=236%:%
%:%554=237%:%
%:%555=238%:%
%:%556=239%:%
%:%559=240%:%
%:%563=240%:%
%:%564=240%:%
%:%565=241%:%
%:%566=241%:%
%:%571=241%:%
%:%574=242%:%
%:%575=243%:%
%:%576=243%:%
%:%577=244%:%
%:%578=245%:%
%:%579=246%:%
%:%580=247%:%
%:%581=247%:%
%:%582=248%:%
%:%583=249%:%
%:%584=250%:%
%:%585=251%:%
%:%586=251%:%
%:%587=252%:%
%:%589=254%:%
%:%592=255%:%
%:%596=255%:%
%:%597=255%:%
%:%598=255%:%
%:%599=256%:%
%:%600=256%:%
%:%605=256%:%
%:%608=257%:%
%:%609=258%:%
%:%610=258%:%
%:%611=259%:%
%:%612=260%:%
%:%613=261%:%
%:%614=262%:%
%:%617=263%:%
%:%621=263%:%
%:%622=263%:%
%:%623=264%:%
%:%624=264%:%
%:%629=264%:%
%:%632=265%:%
%:%633=266%:%
%:%634=266%:%
%:%635=267%:%
%:%636=268%:%
%:%637=269%:%
%:%638=270%:%
%:%639=271%:%
%:%640=272%:%
%:%643=273%:%
%:%647=273%:%
%:%648=273%:%
%:%649=274%:%
%:%650=274%:%
%:%651=275%:%
%:%652=276%:%
%:%653=276%:%
%:%658=276%:%
%:%661=277%:%
%:%662=278%:%
%:%663=278%:%
%:%664=279%:%
%:%665=280%:%
%:%666=280%:%
%:%667=281%:%
%:%668=282%:%
%:%669=283%:%
%:%670=283%:%
%:%671=284%:%
%:%672=285%:%
%:%673=286%:%
%:%674=287%:%
%:%675=287%:%
%:%676=288%:%
%:%677=289%:%
%:%678=290%:%
%:%679=291%:%
%:%680=292%:%
%:%681=292%:%
%:%682=293%:%
%:%683=294%:%
%:%684=295%:%
%:%685=296%:%
%:%686=297%:%
%:%687=297%:%
%:%688=298%:%
%:%689=299%:%
%:%690=300%:%
%:%691=301%:%
%:%692=302%:%
%:%693=302%:%
%:%694=303%:%
%:%695=304%:%
%:%697=306%:%
%:%698=307%:%
%:%699=308%:%
%:%700=308%:%
%:%701=309%:%
%:%702=310%:%
%:%703=311%:%
%:%704=312%:%
%:%705=312%:%
%:%706=313%:%
%:%709=314%:%
%:%713=314%:%
%:%714=314%:%
%:%715=314%:%
%:%720=314%:%
%:%723=315%:%
%:%724=316%:%
%:%725=316%:%
%:%726=317%:%
%:%727=318%:%
%:%730=319%:%
%:%734=319%:%
%:%735=319%:%
%:%736=320%:%
%:%737=320%:%
%:%738=321%:%
%:%739=322%:%
%:%740=322%:%
%:%745=322%:%
%:%748=323%:%
%:%749=324%:%
%:%750=325%:%
%:%751=325%:%
%:%752=326%:%
%:%753=327%:%
%:%756=330%:%
%:%757=331%:%
%:%758=332%:%
%:%759=332%:%
%:%760=333%:%
%:%761=334%:%
%:%764=335%:%
%:%768=335%:%
%:%769=335%:%
%:%770=335%:%
%:%771=335%:%
%:%776=335%:%
%:%779=336%:%
%:%780=337%:%
%:%781=337%:%
%:%784=338%:%
%:%788=338%:%
%:%789=338%:%
%:%790=338%:%
%:%795=338%:%
%:%798=339%:%
%:%799=340%:%
%:%800=340%:%
%:%803=341%:%
%:%807=341%:%
%:%808=341%:%
%:%809=341%:%
%:%814=341%:%
%:%817=342%:%
%:%818=343%:%
%:%819=343%:%
%:%822=344%:%
%:%826=344%:%
%:%827=344%:%
%:%828=344%:%
%:%833=344%:%
%:%836=345%:%
%:%837=346%:%
%:%838=346%:%
%:%841=347%:%
%:%845=347%:%
%:%846=347%:%
%:%847=347%:%
%:%852=347%:%
%:%855=348%:%
%:%856=349%:%
%:%857=350%:%
%:%858=350%:%
%:%861=351%:%
%:%865=351%:%
%:%866=351%:%
%:%867=351%:%
%:%872=351%:%
%:%875=352%:%
%:%876=353%:%
%:%877=353%:%
%:%880=354%:%
%:%884=354%:%
%:%885=354%:%
%:%886=354%:%
%:%891=354%:%
%:%894=355%:%
%:%895=356%:%
%:%896=356%:%
%:%897=357%:%
%:%900=360%:%
%:%903=361%:%
%:%907=361%:%
%:%908=361%:%
%:%914=361%:%
%:%917=362%:%
%:%918=363%:%
%:%919=363%:%
%:%920=364%:%
%:%921=365%:%
%:%924=366%:%
%:%928=366%:%
%:%929=366%:%
%:%930=367%:%
%:%931=367%:%
%:%936=367%:%
%:%939=368%:%
%:%940=369%:%
%:%941=369%:%
%:%942=370%:%
%:%943=371%:%
%:%946=372%:%
%:%950=372%:%
%:%951=372%:%
%:%952=373%:%
%:%953=373%:%
%:%958=373%:%
%:%961=374%:%
%:%962=375%:%
%:%963=375%:%
%:%964=376%:%
%:%965=377%:%
%:%968=378%:%
%:%972=378%:%
%:%973=378%:%
%:%974=378%:%
%:%975=378%:%
%:%980=378%:%
%:%983=379%:%
%:%984=383%:%
%:%985=384%:%
%:%986=384%:%
%:%987=385%:%
%:%988=386%:%
%:%993=391%:%
%:%994=392%:%
%:%995=393%:%
%:%996=393%:%
%:%997=394%:%
%:%998=395%:%
%:%999=396%:%
%:%1000=397%:%
%:%1003=398%:%
%:%1007=398%:%
%:%1008=398%:%
%:%1009=399%:%
%:%1010=399%:%
%:%1011=400%:%
%:%1016=400%:%
%:%1021=401%:%
%:%1026=402%:%
%:%1027=402%:%
%:%1032=402%:%
%:%1035=403%:%
%:%1036=404%:%
%:%1037=405%:%
%:%1038=405%:%
%:%1039=406%:%
%:%1040=407%:%
%:%1043=408%:%
%:%1047=408%:%
%:%1048=408%:%
%:%1049=408%:%
%:%1054=408%:%
%:%1057=409%:%
%:%1058=410%:%
%:%1059=411%:%
%:%1060=411%:%
%:%1061=412%:%
%:%1062=413%:%
%:%1063=414%:%
%:%1064=415%:%
%:%1065=415%:%
%:%1066=416%:%
%:%1069=417%:%
%:%1073=417%:%
%:%1074=417%:%
%:%1075=417%:%
%:%1076=417%:%
%:%1081=417%:%
%:%1084=418%:%
%:%1085=419%:%
%:%1086=419%:%
%:%1087=420%:%
%:%1088=421%:%
%:%1089=422%:%
%:%1090=423%:%
%:%1091=423%:%
%:%1092=424%:%
%:%1093=425%:%
%:%1094=426%:%
%:%1095=427%:%
%:%1096=427%:%
%:%1097=428%:%
%:%1100=429%:%
%:%1104=429%:%
%:%1105=429%:%
%:%1106=429%:%
%:%1111=429%:%
%:%1114=430%:%
%:%1115=431%:%
%:%1116=431%:%
%:%1117=432%:%
%:%1118=433%:%
%:%1125=434%:%
%:%1126=434%:%
%:%1127=435%:%
%:%1128=435%:%
%:%1129=436%:%
%:%1130=437%:%
%:%1131=438%:%
%:%1132=439%:%
%:%1133=439%:%
%:%1134=439%:%
%:%1135=439%:%
%:%1136=440%:%
%:%1137=440%:%
%:%1138=441%:%
%:%1139=441%:%
%:%1140=442%:%
%:%1141=443%:%
%:%1142=444%:%
%:%1143=445%:%
%:%1144=445%:%
%:%1145=446%:%
%:%1146=446%:%
%:%1147=446%:%
%:%1148=446%:%
%:%1149=447%:%
%:%1150=447%:%
%:%1151=448%:%
%:%1152=448%:%
%:%1153=449%:%
%:%1154=449%:%
%:%1155=450%:%
%:%1156=450%:%
%:%1157=450%:%
%:%1158=451%:%
%:%1159=451%:%
%:%1160=452%:%
%:%1161=452%:%
%:%1162=452%:%
%:%1163=452%:%
%:%1164=452%:%
%:%1165=453%:%
%:%1166=453%:%
%:%1167=454%:%
%:%1168=454%:%
%:%1169=455%:%
%:%1170=455%:%
%:%1171=456%:%
%:%1172=456%:%
%:%1173=457%:%
%:%1174=457%:%
%:%1175=457%:%
%:%1176=458%:%
%:%1177=458%:%
%:%1178=459%:%
%:%1179=459%:%
%:%1180=459%:%
%:%1181=460%:%
%:%1182=460%:%
%:%1183=460%:%
%:%1184=461%:%
%:%1185=461%:%
%:%1186=462%:%
%:%1187=462%:%
%:%1188=463%:%
%:%1189=463%:%
%:%1190=464%:%
%:%1191=464%:%
%:%1192=465%:%
%:%1193=465%:%
%:%1194=465%:%
%:%1195=466%:%
%:%1196=466%:%
%:%1197=467%:%
%:%1198=467%:%
%:%1199=467%:%
%:%1200=468%:%
%:%1201=468%:%
%:%1202=468%:%
%:%1203=469%:%
%:%1204=469%:%
%:%1205=470%:%
%:%1211=470%:%
%:%1214=471%:%
%:%1215=472%:%
%:%1216=473%:%
%:%1217=473%:%
%:%1218=474%:%
%:%1219=475%:%
%:%1220=476%:%
%:%1221=477%:%
%:%1222=477%:%
%:%1223=478%:%
%:%1226=479%:%
%:%1230=479%:%
%:%1231=479%:%
%:%1232=479%:%
%:%1237=479%:%
%:%1240=480%:%
%:%1241=481%:%
%:%1242=482%:%
%:%1243=482%:%
%:%1244=483%:%
%:%1245=484%:%
%:%1252=485%:%
%:%1253=485%:%
%:%1254=486%:%
%:%1255=486%:%
%:%1256=487%:%
%:%1257=487%:%
%:%1258=487%:%
%:%1259=488%:%
%:%1260=488%:%
%:%1261=489%:%
%:%1262=489%:%
%:%1263=490%:%
%:%1264=490%:%
%:%1265=490%:%
%:%1266=491%:%
%:%1267=491%:%
%:%1268=492%:%
%:%1269=492%:%
%:%1270=492%:%
%:%1271=492%:%
%:%1272=492%:%
%:%1273=493%:%
%:%1274=493%:%
%:%1275=494%:%
%:%1276=494%:%
%:%1277=495%:%
%:%1278=495%:%
%:%1279=496%:%
%:%1280=496%:%
%:%1281=497%:%
%:%1282=497%:%
%:%1283=498%:%
%:%1284=498%:%
%:%1285=499%:%
%:%1286=499%:%
%:%1287=499%:%
%:%1288=499%:%
%:%1289=499%:%
%:%1290=500%:%
%:%1291=500%:%
%:%1292=501%:%
%:%1293=501%:%
%:%1294=502%:%
%:%1295=502%:%
%:%1296=503%:%
%:%1297=503%:%
%:%1298=503%:%
%:%1299=504%:%
%:%1300=504%:%
%:%1301=504%:%
%:%1302=504%:%
%:%1303=505%:%
%:%1304=505%:%
%:%1305=506%:%
%:%1306=506%:%
%:%1307=507%:%
%:%1308=507%:%
%:%1309=508%:%
%:%1310=508%:%
%:%1311=508%:%
%:%1312=509%:%
%:%1313=509%:%
%:%1314=510%:%
%:%1315=510%:%
%:%1316=511%:%
%:%1317=512%:%
%:%1318=513%:%
%:%1319=513%:%
%:%1320=513%:%
%:%1321=513%:%
%:%1322=514%:%
%:%1323=514%:%
%:%1324=515%:%
%:%1325=515%:%
%:%1326=515%:%
%:%1327=516%:%
%:%1328=516%:%
%:%1329=517%:%
%:%1330=517%:%
%:%1331=518%:%
%:%1332=518%:%
%:%1333=518%:%
%:%1334=519%:%
%:%1335=519%:%
%:%1336=520%:%
%:%1337=520%:%
%:%1338=520%:%
%:%1339=521%:%
%:%1340=521%:%
%:%1341=522%:%
%:%1342=522%:%
%:%1343=523%:%
%:%1344=523%:%
%:%1345=523%:%
%:%1346=523%:%
%:%1347=524%:%
%:%1348=524%:%
%:%1349=525%:%
%:%1350=525%:%
%:%1351=526%:%
%:%1352=526%:%
%:%1353=527%:%
%:%1354=527%:%
%:%1355=528%:%
%:%1356=528%:%
%:%1357=528%:%
%:%1358=529%:%
%:%1359=529%:%
%:%1360=529%:%
%:%1361=530%:%
%:%1362=530%:%
%:%1363=531%:%
%:%1364=531%:%
%:%1365=532%:%
%:%1366=532%:%
%:%1367=532%:%
%:%1368=532%:%
%:%1369=533%:%
%:%1370=533%:%
%:%1371=534%:%
%:%1372=534%:%
%:%1373=535%:%
%:%1374=535%:%
%:%1375=535%:%
%:%1376=536%:%
%:%1377=536%:%
%:%1378=537%:%
%:%1379=537%:%
%:%1380=538%:%
%:%1381=538%:%
%:%1382=539%:%
%:%1383=539%:%
%:%1384=540%:%
%:%1385=541%:%
%:%1386=542%:%
%:%1387=542%:%
%:%1388=542%:%
%:%1389=542%:%
%:%1390=543%:%
%:%1391=543%:%
%:%1392=544%:%
%:%1393=544%:%
%:%1394=545%:%
%:%1395=545%:%
%:%1396=545%:%
%:%1397=546%:%
%:%1398=546%:%
%:%1399=547%:%
%:%1400=547%:%
%:%1401=548%:%
%:%1402=548%:%
%:%1403=548%:%
%:%1404=548%:%
%:%1405=549%:%
%:%1406=549%:%
%:%1407=550%:%
%:%1408=550%:%
%:%1409=551%:%
%:%1410=551%:%
%:%1411=551%:%
%:%1412=551%:%
%:%1413=552%:%
%:%1414=552%:%
%:%1415=553%:%
%:%1416=553%:%
%:%1417=554%:%
%:%1418=554%:%
%:%1419=554%:%
%:%1420=555%:%
%:%1421=555%:%
%:%1422=556%:%
%:%1423=556%:%
%:%1424=557%:%
%:%1430=557%:%
%:%1433=558%:%
%:%1434=559%:%
%:%1435=559%:%
%:%1436=560%:%
%:%1437=561%:%
%:%1438=562%:%
%:%1439=563%:%
%:%1440=563%:%
%:%1441=564%:%
%:%1442=565%:%
%:%1445=566%:%
%:%1449=566%:%
%:%1450=566%:%
%:%1451=566%:%
%:%1452=566%:%
%:%1457=566%:%
%:%1460=567%:%
%:%1461=568%:%
%:%1462=568%:%
%:%1463=569%:%
%:%1464=570%:%
%:%1465=571%:%
%:%1466=572%:%
%:%1467=573%:%
%:%1470=574%:%
%:%1474=574%:%
%:%1475=574%:%
%:%1476=574%:%
%:%1477=574%:%
%:%1482=574%:%
%:%1485=575%:%
%:%1486=576%:%
%:%1487=576%:%
%:%1488=577%:%
%:%1495=578%:%
%:%1496=578%:%
%:%1497=579%:%
%:%1498=579%:%
%:%1499=580%:%
%:%1500=580%:%
%:%1501=581%:%
%:%1502=581%:%
%:%1503=581%:%
%:%1504=582%:%
%:%1505=582%:%
%:%1506=582%:%
%:%1507=582%:%
%:%1508=582%:%
%:%1509=583%:%
%:%1515=583%:%
%:%1518=584%:%
%:%1519=585%:%
%:%1520=585%:%
%:%1521=586%:%
%:%1522=587%:%
%:%1523=588%:%
%:%1524=589%:%
%:%1525=590%:%
%:%1526=591%:%
%:%1533=592%:%
%:%1534=592%:%
%:%1535=593%:%
%:%1536=593%:%
%:%1537=594%:%
%:%1538=594%:%
%:%1539=595%:%
%:%1540=595%:%
%:%1541=596%:%
%:%1542=596%:%
%:%1543=597%:%
%:%1544=597%:%
%:%1545=598%:%
%:%1546=598%:%
%:%1547=599%:%
%:%1548=599%:%
%:%1549=600%:%
%:%1550=600%:%
%:%1551=601%:%
%:%1552=601%:%
%:%1553=602%:%
%:%1554=602%:%
%:%1555=603%:%
%:%1556=603%:%
%:%1557=604%:%
%:%1558=604%:%
%:%1559=605%:%
%:%1560=605%:%
%:%1561=606%:%
%:%1562=606%:%
%:%1563=607%:%
%:%1564=607%:%
%:%1565=608%:%
%:%1566=608%:%
%:%1567=609%:%
%:%1568=609%:%
%:%1569=610%:%
%:%1570=610%:%
%:%1571=611%:%
%:%1572=611%:%
%:%1573=612%:%
%:%1574=612%:%
%:%1575=613%:%
%:%1576=613%:%
%:%1577=613%:%
%:%1578=614%:%
%:%1579=614%:%
%:%1580=615%:%
%:%1581=615%:%
%:%1582=616%:%
%:%1583=616%:%
%:%1584=616%:%
%:%1585=617%:%
%:%1586=617%:%
%:%1587=618%:%
%:%1588=618%:%
%:%1589=619%:%
%:%1590=619%:%
%:%1591=620%:%
%:%1592=621%:%
%:%1593=621%:%
%:%1594=622%:%
%:%1595=622%:%
%:%1596=623%:%
%:%1597=623%:%
%:%1598=624%:%
%:%1599=625%:%
%:%1600=625%:%
%:%1601=626%:%
%:%1602=627%:%
%:%1603=627%:%
%:%1604=628%:%
%:%1605=628%:%
%:%1606=629%:%
%:%1607=629%:%
%:%1608=630%:%
%:%1609=630%:%
%:%1610=630%:%
%:%1611=631%:%
%:%1612=631%:%
%:%1613=632%:%
%:%1614=632%:%
%:%1615=632%:%
%:%1616=632%:%
%:%1617=633%:%
%:%1618=633%:%
%:%1619=633%:%
%:%1620=634%:%
%:%1621=634%:%
%:%1622=634%:%
%:%1623=634%:%
%:%1624=635%:%
%:%1625=635%:%
%:%1626=636%:%
%:%1627=636%:%
%:%1628=636%:%
%:%1629=636%:%
%:%1630=637%:%
%:%1631=637%:%
%:%1632=638%:%
%:%1633=638%:%
%:%1634=638%:%
%:%1635=638%:%
%:%1636=638%:%
%:%1637=639%:%
%:%1638=639%:%
%:%1639=640%:%
%:%1640=640%:%
%:%1641=641%:%
%:%1642=641%:%
%:%1643=642%:%
%:%1644=642%:%
%:%1645=643%:%
%:%1646=643%:%
%:%1647=644%:%
%:%1648=644%:%
%:%1649=645%:%
%:%1650=645%:%
%:%1651=645%:%
%:%1652=646%:%
%:%1653=646%:%
%:%1654=647%:%
%:%1655=647%:%
%:%1656=648%:%
%:%1657=648%:%
%:%1658=648%:%
%:%1659=649%:%
%:%1660=649%:%
%:%1661=650%:%
%:%1662=650%:%
%:%1663=651%:%
%:%1664=651%:%
%:%1665=652%:%
%:%1666=652%:%
%:%1667=653%:%
%:%1668=653%:%
%:%1669=654%:%
%:%1670=654%:%
%:%1671=655%:%
%:%1672=655%:%
%:%1673=656%:%
%:%1674=656%:%
%:%1675=657%:%
%:%1676=657%:%
%:%1677=658%:%
%:%1678=658%:%
%:%1679=659%:%
%:%1680=659%:%
%:%1681=659%:%
%:%1682=660%:%
%:%1683=660%:%
%:%1684=661%:%
%:%1685=661%:%
%:%1686=661%:%
%:%1687=662%:%
%:%1688=662%:%
%:%1689=663%:%
%:%1690=663%:%
%:%1691=663%:%
%:%1692=663%:%
%:%1693=663%:%
%:%1694=664%:%
%:%1695=664%:%
%:%1696=665%:%
%:%1702=665%:%
%:%1705=666%:%
%:%1706=667%:%
%:%1707=667%:%
%:%1710=668%:%
%:%1714=668%:%
%:%1715=668%:%
%:%1721=668%:%
%:%1724=669%:%
%:%1725=670%:%
%:%1726=670%:%
%:%1727=671%:%
%:%1732=676%:%
%:%1735=677%:%
%:%1739=677%:%
%:%1740=677%:%
%:%1741=678%:%
%:%1742=678%:%
%:%1747=678%:%
%:%1752=679%:%
%:%1757=680%:%

%
\begin{isabellebody}%
\setisabellecontext{Arities}%
%
\isadelimdocument
%
\endisadelimdocument
%
\isatagdocument
%
\isamarkupsection{Arities of internalized formulas%
}
\isamarkuptrue%
%
\endisatagdocument
{\isafolddocument}%
%
\isadelimdocument
%
\endisadelimdocument
%
\isadelimtheory
%
\endisadelimtheory
%
\isatagtheory
\isacommand{theory}\isamarkupfalse%
\ Arities\isanewline
\ \ \isakeyword{imports}\ FrecR\isanewline
\isakeyword{begin}%
\endisatagtheory
{\isafoldtheory}%
%
\isadelimtheory
\isanewline
%
\endisadelimtheory
\isanewline
\isacommand{lemma}\isamarkupfalse%
\ arity{\isacharunderscore}{\kern0pt}upair{\isacharunderscore}{\kern0pt}fm\ {\isacharcolon}{\kern0pt}\ {\isachardoublequoteopen}{\isasymlbrakk}\ \ t{\isadigit{1}}{\isasymin}nat\ {\isacharsemicolon}{\kern0pt}\ t{\isadigit{2}}{\isasymin}nat\ {\isacharsemicolon}{\kern0pt}\ up{\isasymin}nat\ \ {\isasymrbrakk}\ {\isasymLongrightarrow}\ \isanewline
\ \ arity{\isacharparenleft}{\kern0pt}upair{\isacharunderscore}{\kern0pt}fm{\isacharparenleft}{\kern0pt}t{\isadigit{1}}{\isacharcomma}{\kern0pt}t{\isadigit{2}}{\isacharcomma}{\kern0pt}up{\isacharparenright}{\kern0pt}{\isacharparenright}{\kern0pt}\ {\isacharequal}{\kern0pt}\ {\isasymUnion}\ {\isacharbraceleft}{\kern0pt}succ{\isacharparenleft}{\kern0pt}t{\isadigit{1}}{\isacharparenright}{\kern0pt}{\isacharcomma}{\kern0pt}succ{\isacharparenleft}{\kern0pt}t{\isadigit{2}}{\isacharparenright}{\kern0pt}{\isacharcomma}{\kern0pt}succ{\isacharparenleft}{\kern0pt}up{\isacharparenright}{\kern0pt}{\isacharbraceright}{\kern0pt}{\isachardoublequoteclose}\isanewline
%
\isadelimproof
\ \ %
\endisadelimproof
%
\isatagproof
\isacommand{unfolding}\isamarkupfalse%
\ \ upair{\isacharunderscore}{\kern0pt}fm{\isacharunderscore}{\kern0pt}def\isanewline
\ \ \isacommand{using}\isamarkupfalse%
\ nat{\isacharunderscore}{\kern0pt}union{\isacharunderscore}{\kern0pt}abs{\isadigit{1}}\ nat{\isacharunderscore}{\kern0pt}union{\isacharunderscore}{\kern0pt}abs{\isadigit{2}}\ pred{\isacharunderscore}{\kern0pt}Un\ \ \ \isanewline
\ \ \isacommand{by}\isamarkupfalse%
\ auto%
\endisatagproof
{\isafoldproof}%
%
\isadelimproof
\isanewline
%
\endisadelimproof
\isanewline
\isanewline
\isacommand{lemma}\isamarkupfalse%
\ arity{\isacharunderscore}{\kern0pt}pair{\isacharunderscore}{\kern0pt}fm\ {\isacharcolon}{\kern0pt}\ {\isachardoublequoteopen}{\isasymlbrakk}\ \ t{\isadigit{1}}{\isasymin}nat\ {\isacharsemicolon}{\kern0pt}\ t{\isadigit{2}}{\isasymin}nat\ {\isacharsemicolon}{\kern0pt}\ p{\isasymin}nat\ \ {\isasymrbrakk}\ {\isasymLongrightarrow}\ \isanewline
\ \ arity{\isacharparenleft}{\kern0pt}pair{\isacharunderscore}{\kern0pt}fm{\isacharparenleft}{\kern0pt}t{\isadigit{1}}{\isacharcomma}{\kern0pt}t{\isadigit{2}}{\isacharcomma}{\kern0pt}p{\isacharparenright}{\kern0pt}{\isacharparenright}{\kern0pt}\ {\isacharequal}{\kern0pt}\ {\isasymUnion}\ {\isacharbraceleft}{\kern0pt}succ{\isacharparenleft}{\kern0pt}t{\isadigit{1}}{\isacharparenright}{\kern0pt}{\isacharcomma}{\kern0pt}succ{\isacharparenleft}{\kern0pt}t{\isadigit{2}}{\isacharparenright}{\kern0pt}{\isacharcomma}{\kern0pt}succ{\isacharparenleft}{\kern0pt}p{\isacharparenright}{\kern0pt}{\isacharbraceright}{\kern0pt}{\isachardoublequoteclose}\isanewline
%
\isadelimproof
\ \ %
\endisadelimproof
%
\isatagproof
\isacommand{unfolding}\isamarkupfalse%
\ pair{\isacharunderscore}{\kern0pt}fm{\isacharunderscore}{\kern0pt}def\ \isanewline
\ \ \isacommand{using}\isamarkupfalse%
\ arity{\isacharunderscore}{\kern0pt}upair{\isacharunderscore}{\kern0pt}fm\ nat{\isacharunderscore}{\kern0pt}union{\isacharunderscore}{\kern0pt}abs{\isadigit{1}}\ nat{\isacharunderscore}{\kern0pt}union{\isacharunderscore}{\kern0pt}abs{\isadigit{2}}\ pred{\isacharunderscore}{\kern0pt}Un\isanewline
\ \ \isacommand{by}\isamarkupfalse%
\ auto%
\endisatagproof
{\isafoldproof}%
%
\isadelimproof
\isanewline
%
\endisadelimproof
\isanewline
\isacommand{lemma}\isamarkupfalse%
\ arity{\isacharunderscore}{\kern0pt}composition{\isacharunderscore}{\kern0pt}fm\ {\isacharcolon}{\kern0pt}\isanewline
\ \ {\isachardoublequoteopen}{\isasymlbrakk}\ r{\isasymin}nat\ {\isacharsemicolon}{\kern0pt}\ s{\isasymin}nat\ {\isacharsemicolon}{\kern0pt}\ t{\isasymin}nat\ {\isasymrbrakk}\ {\isasymLongrightarrow}\ arity{\isacharparenleft}{\kern0pt}composition{\isacharunderscore}{\kern0pt}fm{\isacharparenleft}{\kern0pt}r{\isacharcomma}{\kern0pt}s{\isacharcomma}{\kern0pt}t{\isacharparenright}{\kern0pt}{\isacharparenright}{\kern0pt}\ {\isacharequal}{\kern0pt}\ {\isasymUnion}\ {\isacharbraceleft}{\kern0pt}succ{\isacharparenleft}{\kern0pt}r{\isacharparenright}{\kern0pt}{\isacharcomma}{\kern0pt}\ succ{\isacharparenleft}{\kern0pt}s{\isacharparenright}{\kern0pt}{\isacharcomma}{\kern0pt}\ succ{\isacharparenleft}{\kern0pt}t{\isacharparenright}{\kern0pt}{\isacharbraceright}{\kern0pt}{\isachardoublequoteclose}\isanewline
%
\isadelimproof
\ \ %
\endisadelimproof
%
\isatagproof
\isacommand{unfolding}\isamarkupfalse%
\ composition{\isacharunderscore}{\kern0pt}fm{\isacharunderscore}{\kern0pt}def\ \ \ \ \isanewline
\ \ \isacommand{using}\isamarkupfalse%
\ arity{\isacharunderscore}{\kern0pt}pair{\isacharunderscore}{\kern0pt}fm\ nat{\isacharunderscore}{\kern0pt}union{\isacharunderscore}{\kern0pt}abs{\isadigit{1}}\ nat{\isacharunderscore}{\kern0pt}union{\isacharunderscore}{\kern0pt}abs{\isadigit{2}}\ pred{\isacharunderscore}{\kern0pt}Un{\isacharunderscore}{\kern0pt}distrib\isanewline
\ \ \isacommand{by}\isamarkupfalse%
\ auto%
\endisatagproof
{\isafoldproof}%
%
\isadelimproof
\isanewline
%
\endisadelimproof
\isanewline
\isacommand{lemma}\isamarkupfalse%
\ arity{\isacharunderscore}{\kern0pt}domain{\isacharunderscore}{\kern0pt}fm\ {\isacharcolon}{\kern0pt}\ \isanewline
\ \ \ \ {\isachardoublequoteopen}{\isasymlbrakk}\ r{\isasymin}nat\ {\isacharsemicolon}{\kern0pt}\ z{\isasymin}nat\ {\isasymrbrakk}\ {\isasymLongrightarrow}\ arity{\isacharparenleft}{\kern0pt}domain{\isacharunderscore}{\kern0pt}fm{\isacharparenleft}{\kern0pt}r{\isacharcomma}{\kern0pt}z{\isacharparenright}{\kern0pt}{\isacharparenright}{\kern0pt}\ {\isacharequal}{\kern0pt}\ succ{\isacharparenleft}{\kern0pt}r{\isacharparenright}{\kern0pt}\ {\isasymunion}\ succ{\isacharparenleft}{\kern0pt}z{\isacharparenright}{\kern0pt}{\isachardoublequoteclose}\isanewline
%
\isadelimproof
\ \ %
\endisadelimproof
%
\isatagproof
\isacommand{unfolding}\isamarkupfalse%
\ domain{\isacharunderscore}{\kern0pt}fm{\isacharunderscore}{\kern0pt}def\ \isanewline
\ \ \isacommand{using}\isamarkupfalse%
\ arity{\isacharunderscore}{\kern0pt}pair{\isacharunderscore}{\kern0pt}fm\ nat{\isacharunderscore}{\kern0pt}union{\isacharunderscore}{\kern0pt}abs{\isadigit{1}}\ nat{\isacharunderscore}{\kern0pt}union{\isacharunderscore}{\kern0pt}abs{\isadigit{2}}\ pred{\isacharunderscore}{\kern0pt}Un{\isacharunderscore}{\kern0pt}distrib\isanewline
\ \ \isacommand{by}\isamarkupfalse%
\ auto%
\endisatagproof
{\isafoldproof}%
%
\isadelimproof
\isanewline
%
\endisadelimproof
\isanewline
\isacommand{lemma}\isamarkupfalse%
\ arity{\isacharunderscore}{\kern0pt}range{\isacharunderscore}{\kern0pt}fm\ {\isacharcolon}{\kern0pt}\ \isanewline
\ \ \ \ {\isachardoublequoteopen}{\isasymlbrakk}\ r{\isasymin}nat\ {\isacharsemicolon}{\kern0pt}\ z{\isasymin}nat\ {\isasymrbrakk}\ {\isasymLongrightarrow}\ arity{\isacharparenleft}{\kern0pt}range{\isacharunderscore}{\kern0pt}fm{\isacharparenleft}{\kern0pt}r{\isacharcomma}{\kern0pt}z{\isacharparenright}{\kern0pt}{\isacharparenright}{\kern0pt}\ {\isacharequal}{\kern0pt}\ succ{\isacharparenleft}{\kern0pt}r{\isacharparenright}{\kern0pt}\ {\isasymunion}\ succ{\isacharparenleft}{\kern0pt}z{\isacharparenright}{\kern0pt}{\isachardoublequoteclose}\isanewline
%
\isadelimproof
\ \ %
\endisadelimproof
%
\isatagproof
\isacommand{unfolding}\isamarkupfalse%
\ range{\isacharunderscore}{\kern0pt}fm{\isacharunderscore}{\kern0pt}def\ \isanewline
\ \ \isacommand{using}\isamarkupfalse%
\ arity{\isacharunderscore}{\kern0pt}pair{\isacharunderscore}{\kern0pt}fm\ nat{\isacharunderscore}{\kern0pt}union{\isacharunderscore}{\kern0pt}abs{\isadigit{1}}\ nat{\isacharunderscore}{\kern0pt}union{\isacharunderscore}{\kern0pt}abs{\isadigit{2}}\ pred{\isacharunderscore}{\kern0pt}Un{\isacharunderscore}{\kern0pt}distrib\isanewline
\ \ \isacommand{by}\isamarkupfalse%
\ auto%
\endisatagproof
{\isafoldproof}%
%
\isadelimproof
\isanewline
%
\endisadelimproof
\isanewline
\isacommand{lemma}\isamarkupfalse%
\ arity{\isacharunderscore}{\kern0pt}union{\isacharunderscore}{\kern0pt}fm\ {\isacharcolon}{\kern0pt}\ \isanewline
\ \ {\isachardoublequoteopen}{\isasymlbrakk}\ x{\isasymin}nat\ {\isacharsemicolon}{\kern0pt}\ y{\isasymin}nat\ {\isacharsemicolon}{\kern0pt}\ z{\isasymin}nat\ {\isasymrbrakk}\ {\isasymLongrightarrow}\ arity{\isacharparenleft}{\kern0pt}union{\isacharunderscore}{\kern0pt}fm{\isacharparenleft}{\kern0pt}x{\isacharcomma}{\kern0pt}y{\isacharcomma}{\kern0pt}z{\isacharparenright}{\kern0pt}{\isacharparenright}{\kern0pt}\ {\isacharequal}{\kern0pt}\ {\isasymUnion}\ {\isacharbraceleft}{\kern0pt}succ{\isacharparenleft}{\kern0pt}x{\isacharparenright}{\kern0pt}{\isacharcomma}{\kern0pt}\ succ{\isacharparenleft}{\kern0pt}y{\isacharparenright}{\kern0pt}{\isacharcomma}{\kern0pt}\ succ{\isacharparenleft}{\kern0pt}z{\isacharparenright}{\kern0pt}{\isacharbraceright}{\kern0pt}{\isachardoublequoteclose}\isanewline
%
\isadelimproof
\ \ %
\endisadelimproof
%
\isatagproof
\isacommand{unfolding}\isamarkupfalse%
\ union{\isacharunderscore}{\kern0pt}fm{\isacharunderscore}{\kern0pt}def\isanewline
\ \ \isacommand{using}\isamarkupfalse%
\ \ nat{\isacharunderscore}{\kern0pt}union{\isacharunderscore}{\kern0pt}abs{\isadigit{1}}\ nat{\isacharunderscore}{\kern0pt}union{\isacharunderscore}{\kern0pt}abs{\isadigit{2}}\ pred{\isacharunderscore}{\kern0pt}Un{\isacharunderscore}{\kern0pt}distrib\isanewline
\ \ \isacommand{by}\isamarkupfalse%
\ auto%
\endisatagproof
{\isafoldproof}%
%
\isadelimproof
\isanewline
%
\endisadelimproof
\isanewline
\isacommand{lemma}\isamarkupfalse%
\ arity{\isacharunderscore}{\kern0pt}image{\isacharunderscore}{\kern0pt}fm\ {\isacharcolon}{\kern0pt}\ \isanewline
\ \ {\isachardoublequoteopen}{\isasymlbrakk}\ x{\isasymin}nat\ {\isacharsemicolon}{\kern0pt}\ y{\isasymin}nat\ {\isacharsemicolon}{\kern0pt}\ z{\isasymin}nat\ {\isasymrbrakk}\ {\isasymLongrightarrow}\ arity{\isacharparenleft}{\kern0pt}image{\isacharunderscore}{\kern0pt}fm{\isacharparenleft}{\kern0pt}x{\isacharcomma}{\kern0pt}y{\isacharcomma}{\kern0pt}z{\isacharparenright}{\kern0pt}{\isacharparenright}{\kern0pt}\ {\isacharequal}{\kern0pt}\ {\isasymUnion}\ {\isacharbraceleft}{\kern0pt}succ{\isacharparenleft}{\kern0pt}x{\isacharparenright}{\kern0pt}{\isacharcomma}{\kern0pt}\ succ{\isacharparenleft}{\kern0pt}y{\isacharparenright}{\kern0pt}{\isacharcomma}{\kern0pt}\ succ{\isacharparenleft}{\kern0pt}z{\isacharparenright}{\kern0pt}{\isacharbraceright}{\kern0pt}{\isachardoublequoteclose}\isanewline
%
\isadelimproof
\ \ %
\endisadelimproof
%
\isatagproof
\isacommand{unfolding}\isamarkupfalse%
\ image{\isacharunderscore}{\kern0pt}fm{\isacharunderscore}{\kern0pt}def\isanewline
\ \ \isacommand{using}\isamarkupfalse%
\ arity{\isacharunderscore}{\kern0pt}pair{\isacharunderscore}{\kern0pt}fm\ \ nat{\isacharunderscore}{\kern0pt}union{\isacharunderscore}{\kern0pt}abs{\isadigit{1}}\ nat{\isacharunderscore}{\kern0pt}union{\isacharunderscore}{\kern0pt}abs{\isadigit{2}}\ pred{\isacharunderscore}{\kern0pt}Un{\isacharunderscore}{\kern0pt}distrib\isanewline
\ \ \isacommand{by}\isamarkupfalse%
\ auto%
\endisatagproof
{\isafoldproof}%
%
\isadelimproof
\isanewline
%
\endisadelimproof
\isanewline
\isacommand{lemma}\isamarkupfalse%
\ arity{\isacharunderscore}{\kern0pt}pre{\isacharunderscore}{\kern0pt}image{\isacharunderscore}{\kern0pt}fm\ {\isacharcolon}{\kern0pt}\ \isanewline
\ \ {\isachardoublequoteopen}{\isasymlbrakk}\ x{\isasymin}nat\ {\isacharsemicolon}{\kern0pt}\ y{\isasymin}nat\ {\isacharsemicolon}{\kern0pt}\ z{\isasymin}nat\ {\isasymrbrakk}\ {\isasymLongrightarrow}\ arity{\isacharparenleft}{\kern0pt}pre{\isacharunderscore}{\kern0pt}image{\isacharunderscore}{\kern0pt}fm{\isacharparenleft}{\kern0pt}x{\isacharcomma}{\kern0pt}y{\isacharcomma}{\kern0pt}z{\isacharparenright}{\kern0pt}{\isacharparenright}{\kern0pt}\ {\isacharequal}{\kern0pt}\ {\isasymUnion}\ {\isacharbraceleft}{\kern0pt}succ{\isacharparenleft}{\kern0pt}x{\isacharparenright}{\kern0pt}{\isacharcomma}{\kern0pt}\ succ{\isacharparenleft}{\kern0pt}y{\isacharparenright}{\kern0pt}{\isacharcomma}{\kern0pt}\ succ{\isacharparenleft}{\kern0pt}z{\isacharparenright}{\kern0pt}{\isacharbraceright}{\kern0pt}{\isachardoublequoteclose}\isanewline
%
\isadelimproof
\ \ %
\endisadelimproof
%
\isatagproof
\isacommand{unfolding}\isamarkupfalse%
\ pre{\isacharunderscore}{\kern0pt}image{\isacharunderscore}{\kern0pt}fm{\isacharunderscore}{\kern0pt}def\isanewline
\ \ \isacommand{using}\isamarkupfalse%
\ arity{\isacharunderscore}{\kern0pt}pair{\isacharunderscore}{\kern0pt}fm\ \ nat{\isacharunderscore}{\kern0pt}union{\isacharunderscore}{\kern0pt}abs{\isadigit{1}}\ nat{\isacharunderscore}{\kern0pt}union{\isacharunderscore}{\kern0pt}abs{\isadigit{2}}\ pred{\isacharunderscore}{\kern0pt}Un{\isacharunderscore}{\kern0pt}distrib\isanewline
\ \ \isacommand{by}\isamarkupfalse%
\ auto%
\endisatagproof
{\isafoldproof}%
%
\isadelimproof
\isanewline
%
\endisadelimproof
\isanewline
\isanewline
\isacommand{lemma}\isamarkupfalse%
\ arity{\isacharunderscore}{\kern0pt}big{\isacharunderscore}{\kern0pt}union{\isacharunderscore}{\kern0pt}fm\ {\isacharcolon}{\kern0pt}\ \isanewline
\ \ {\isachardoublequoteopen}{\isasymlbrakk}\ x{\isasymin}nat\ {\isacharsemicolon}{\kern0pt}\ y{\isasymin}nat\ {\isasymrbrakk}\ {\isasymLongrightarrow}\ arity{\isacharparenleft}{\kern0pt}big{\isacharunderscore}{\kern0pt}union{\isacharunderscore}{\kern0pt}fm{\isacharparenleft}{\kern0pt}x{\isacharcomma}{\kern0pt}y{\isacharparenright}{\kern0pt}{\isacharparenright}{\kern0pt}\ {\isacharequal}{\kern0pt}\ succ{\isacharparenleft}{\kern0pt}x{\isacharparenright}{\kern0pt}\ {\isasymunion}\ succ{\isacharparenleft}{\kern0pt}y{\isacharparenright}{\kern0pt}{\isachardoublequoteclose}\isanewline
%
\isadelimproof
\ \ %
\endisadelimproof
%
\isatagproof
\isacommand{unfolding}\isamarkupfalse%
\ big{\isacharunderscore}{\kern0pt}union{\isacharunderscore}{\kern0pt}fm{\isacharunderscore}{\kern0pt}def\isanewline
\ \ \isacommand{using}\isamarkupfalse%
\ nat{\isacharunderscore}{\kern0pt}union{\isacharunderscore}{\kern0pt}abs{\isadigit{1}}\ nat{\isacharunderscore}{\kern0pt}union{\isacharunderscore}{\kern0pt}abs{\isadigit{2}}\ pred{\isacharunderscore}{\kern0pt}Un{\isacharunderscore}{\kern0pt}distrib\isanewline
\ \ \isacommand{by}\isamarkupfalse%
\ auto%
\endisatagproof
{\isafoldproof}%
%
\isadelimproof
\isanewline
%
\endisadelimproof
\isanewline
\isacommand{lemma}\isamarkupfalse%
\ arity{\isacharunderscore}{\kern0pt}fun{\isacharunderscore}{\kern0pt}apply{\isacharunderscore}{\kern0pt}fm\ {\isacharcolon}{\kern0pt}\ \isanewline
\ \ {\isachardoublequoteopen}{\isasymlbrakk}\ x{\isasymin}nat\ {\isacharsemicolon}{\kern0pt}\ y{\isasymin}nat\ {\isacharsemicolon}{\kern0pt}\ f{\isasymin}nat\ {\isasymrbrakk}\ {\isasymLongrightarrow}\ \isanewline
\ \ \ \ arity{\isacharparenleft}{\kern0pt}fun{\isacharunderscore}{\kern0pt}apply{\isacharunderscore}{\kern0pt}fm{\isacharparenleft}{\kern0pt}f{\isacharcomma}{\kern0pt}x{\isacharcomma}{\kern0pt}y{\isacharparenright}{\kern0pt}{\isacharparenright}{\kern0pt}\ {\isacharequal}{\kern0pt}\ \ succ{\isacharparenleft}{\kern0pt}f{\isacharparenright}{\kern0pt}\ {\isasymunion}\ succ{\isacharparenleft}{\kern0pt}x{\isacharparenright}{\kern0pt}\ {\isasymunion}\ succ{\isacharparenleft}{\kern0pt}y{\isacharparenright}{\kern0pt}{\isachardoublequoteclose}\isanewline
%
\isadelimproof
\ \ %
\endisadelimproof
%
\isatagproof
\isacommand{unfolding}\isamarkupfalse%
\ fun{\isacharunderscore}{\kern0pt}apply{\isacharunderscore}{\kern0pt}fm{\isacharunderscore}{\kern0pt}def\isanewline
\ \ \isacommand{using}\isamarkupfalse%
\ arity{\isacharunderscore}{\kern0pt}upair{\isacharunderscore}{\kern0pt}fm\ arity{\isacharunderscore}{\kern0pt}image{\isacharunderscore}{\kern0pt}fm\ arity{\isacharunderscore}{\kern0pt}big{\isacharunderscore}{\kern0pt}union{\isacharunderscore}{\kern0pt}fm\ nat{\isacharunderscore}{\kern0pt}union{\isacharunderscore}{\kern0pt}abs{\isadigit{2}}\ pred{\isacharunderscore}{\kern0pt}Un{\isacharunderscore}{\kern0pt}distrib\isanewline
\ \ \isacommand{by}\isamarkupfalse%
\ auto%
\endisatagproof
{\isafoldproof}%
%
\isadelimproof
\isanewline
%
\endisadelimproof
\isanewline
\isacommand{lemma}\isamarkupfalse%
\ arity{\isacharunderscore}{\kern0pt}field{\isacharunderscore}{\kern0pt}fm\ {\isacharcolon}{\kern0pt}\ \isanewline
\ \ \ \ {\isachardoublequoteopen}{\isasymlbrakk}\ r{\isasymin}nat\ {\isacharsemicolon}{\kern0pt}\ z{\isasymin}nat\ {\isasymrbrakk}\ {\isasymLongrightarrow}\ arity{\isacharparenleft}{\kern0pt}field{\isacharunderscore}{\kern0pt}fm{\isacharparenleft}{\kern0pt}r{\isacharcomma}{\kern0pt}z{\isacharparenright}{\kern0pt}{\isacharparenright}{\kern0pt}\ {\isacharequal}{\kern0pt}\ succ{\isacharparenleft}{\kern0pt}r{\isacharparenright}{\kern0pt}\ {\isasymunion}\ succ{\isacharparenleft}{\kern0pt}z{\isacharparenright}{\kern0pt}{\isachardoublequoteclose}\isanewline
%
\isadelimproof
\ \ %
\endisadelimproof
%
\isatagproof
\isacommand{unfolding}\isamarkupfalse%
\ field{\isacharunderscore}{\kern0pt}fm{\isacharunderscore}{\kern0pt}def\ \isanewline
\ \ \isacommand{using}\isamarkupfalse%
\ arity{\isacharunderscore}{\kern0pt}pair{\isacharunderscore}{\kern0pt}fm\ arity{\isacharunderscore}{\kern0pt}domain{\isacharunderscore}{\kern0pt}fm\ arity{\isacharunderscore}{\kern0pt}range{\isacharunderscore}{\kern0pt}fm\ arity{\isacharunderscore}{\kern0pt}union{\isacharunderscore}{\kern0pt}fm\ \isanewline
\ \ \ \ nat{\isacharunderscore}{\kern0pt}union{\isacharunderscore}{\kern0pt}abs{\isadigit{1}}\ nat{\isacharunderscore}{\kern0pt}union{\isacharunderscore}{\kern0pt}abs{\isadigit{2}}\ pred{\isacharunderscore}{\kern0pt}Un{\isacharunderscore}{\kern0pt}distrib\isanewline
\ \ \isacommand{by}\isamarkupfalse%
\ auto%
\endisatagproof
{\isafoldproof}%
%
\isadelimproof
\isanewline
%
\endisadelimproof
\isanewline
\isacommand{lemma}\isamarkupfalse%
\ arity{\isacharunderscore}{\kern0pt}empty{\isacharunderscore}{\kern0pt}fm\ {\isacharcolon}{\kern0pt}\ \isanewline
\ \ \ \ {\isachardoublequoteopen}{\isasymlbrakk}\ r{\isasymin}nat\ {\isasymrbrakk}\ {\isasymLongrightarrow}\ arity{\isacharparenleft}{\kern0pt}empty{\isacharunderscore}{\kern0pt}fm{\isacharparenleft}{\kern0pt}r{\isacharparenright}{\kern0pt}{\isacharparenright}{\kern0pt}\ {\isacharequal}{\kern0pt}\ succ{\isacharparenleft}{\kern0pt}r{\isacharparenright}{\kern0pt}{\isachardoublequoteclose}\isanewline
%
\isadelimproof
\ \ %
\endisadelimproof
%
\isatagproof
\isacommand{unfolding}\isamarkupfalse%
\ empty{\isacharunderscore}{\kern0pt}fm{\isacharunderscore}{\kern0pt}def\ \isanewline
\ \ \isacommand{using}\isamarkupfalse%
\ nat{\isacharunderscore}{\kern0pt}union{\isacharunderscore}{\kern0pt}abs{\isadigit{1}}\ nat{\isacharunderscore}{\kern0pt}union{\isacharunderscore}{\kern0pt}abs{\isadigit{2}}\ pred{\isacharunderscore}{\kern0pt}Un{\isacharunderscore}{\kern0pt}distrib\isanewline
\ \ \isacommand{by}\isamarkupfalse%
\ simp%
\endisatagproof
{\isafoldproof}%
%
\isadelimproof
\isanewline
%
\endisadelimproof
\isanewline
\isacommand{lemma}\isamarkupfalse%
\ arity{\isacharunderscore}{\kern0pt}succ{\isacharunderscore}{\kern0pt}fm\ {\isacharcolon}{\kern0pt}\isanewline
\ \ {\isachardoublequoteopen}{\isasymlbrakk}x{\isasymin}nat{\isacharsemicolon}{\kern0pt}y{\isasymin}nat{\isasymrbrakk}\ {\isasymLongrightarrow}\ arity{\isacharparenleft}{\kern0pt}succ{\isacharunderscore}{\kern0pt}fm{\isacharparenleft}{\kern0pt}x{\isacharcomma}{\kern0pt}y{\isacharparenright}{\kern0pt}{\isacharparenright}{\kern0pt}\ {\isacharequal}{\kern0pt}\ succ{\isacharparenleft}{\kern0pt}x{\isacharparenright}{\kern0pt}\ {\isasymunion}\ succ{\isacharparenleft}{\kern0pt}y{\isacharparenright}{\kern0pt}{\isachardoublequoteclose}\isanewline
%
\isadelimproof
\ \ %
\endisadelimproof
%
\isatagproof
\isacommand{unfolding}\isamarkupfalse%
\ succ{\isacharunderscore}{\kern0pt}fm{\isacharunderscore}{\kern0pt}def\ cons{\isacharunderscore}{\kern0pt}fm{\isacharunderscore}{\kern0pt}def\ \isanewline
\ \ \isacommand{using}\isamarkupfalse%
\ arity{\isacharunderscore}{\kern0pt}upair{\isacharunderscore}{\kern0pt}fm\ arity{\isacharunderscore}{\kern0pt}union{\isacharunderscore}{\kern0pt}fm\ nat{\isacharunderscore}{\kern0pt}union{\isacharunderscore}{\kern0pt}abs{\isadigit{2}}\ pred{\isacharunderscore}{\kern0pt}Un{\isacharunderscore}{\kern0pt}distrib\isanewline
\ \ \isacommand{by}\isamarkupfalse%
\ auto%
\endisatagproof
{\isafoldproof}%
%
\isadelimproof
\isanewline
%
\endisadelimproof
\isanewline
\isanewline
\isacommand{lemma}\isamarkupfalse%
\ number{\isadigit{1}}arity{\isacharunderscore}{\kern0pt}{\isacharunderscore}{\kern0pt}fm\ {\isacharcolon}{\kern0pt}\ \isanewline
\ \ \ \ {\isachardoublequoteopen}{\isasymlbrakk}\ r{\isasymin}nat\ {\isasymrbrakk}\ {\isasymLongrightarrow}\ arity{\isacharparenleft}{\kern0pt}number{\isadigit{1}}{\isacharunderscore}{\kern0pt}fm{\isacharparenleft}{\kern0pt}r{\isacharparenright}{\kern0pt}{\isacharparenright}{\kern0pt}\ {\isacharequal}{\kern0pt}\ succ{\isacharparenleft}{\kern0pt}r{\isacharparenright}{\kern0pt}{\isachardoublequoteclose}\isanewline
%
\isadelimproof
\ \ %
\endisadelimproof
%
\isatagproof
\isacommand{unfolding}\isamarkupfalse%
\ number{\isadigit{1}}{\isacharunderscore}{\kern0pt}fm{\isacharunderscore}{\kern0pt}def\ \isanewline
\ \ \isacommand{using}\isamarkupfalse%
\ arity{\isacharunderscore}{\kern0pt}empty{\isacharunderscore}{\kern0pt}fm\ arity{\isacharunderscore}{\kern0pt}succ{\isacharunderscore}{\kern0pt}fm\ nat{\isacharunderscore}{\kern0pt}union{\isacharunderscore}{\kern0pt}abs{\isadigit{1}}\ nat{\isacharunderscore}{\kern0pt}union{\isacharunderscore}{\kern0pt}abs{\isadigit{2}}\ pred{\isacharunderscore}{\kern0pt}Un{\isacharunderscore}{\kern0pt}distrib\isanewline
\ \ \isacommand{by}\isamarkupfalse%
\ simp%
\endisatagproof
{\isafoldproof}%
%
\isadelimproof
\isanewline
%
\endisadelimproof
\isanewline
\isanewline
\isacommand{lemma}\isamarkupfalse%
\ arity{\isacharunderscore}{\kern0pt}function{\isacharunderscore}{\kern0pt}fm\ {\isacharcolon}{\kern0pt}\ \isanewline
\ \ \ \ {\isachardoublequoteopen}{\isasymlbrakk}\ r{\isasymin}nat\ {\isasymrbrakk}\ {\isasymLongrightarrow}\ arity{\isacharparenleft}{\kern0pt}function{\isacharunderscore}{\kern0pt}fm{\isacharparenleft}{\kern0pt}r{\isacharparenright}{\kern0pt}{\isacharparenright}{\kern0pt}\ {\isacharequal}{\kern0pt}\ succ{\isacharparenleft}{\kern0pt}r{\isacharparenright}{\kern0pt}{\isachardoublequoteclose}\isanewline
%
\isadelimproof
\ \ %
\endisadelimproof
%
\isatagproof
\isacommand{unfolding}\isamarkupfalse%
\ function{\isacharunderscore}{\kern0pt}fm{\isacharunderscore}{\kern0pt}def\ \isanewline
\ \ \isacommand{using}\isamarkupfalse%
\ arity{\isacharunderscore}{\kern0pt}pair{\isacharunderscore}{\kern0pt}fm\ nat{\isacharunderscore}{\kern0pt}union{\isacharunderscore}{\kern0pt}abs{\isadigit{1}}\ nat{\isacharunderscore}{\kern0pt}union{\isacharunderscore}{\kern0pt}abs{\isadigit{2}}\ pred{\isacharunderscore}{\kern0pt}Un{\isacharunderscore}{\kern0pt}distrib\isanewline
\ \ \isacommand{by}\isamarkupfalse%
\ simp%
\endisatagproof
{\isafoldproof}%
%
\isadelimproof
\isanewline
%
\endisadelimproof
\isanewline
\isacommand{lemma}\isamarkupfalse%
\ arity{\isacharunderscore}{\kern0pt}relation{\isacharunderscore}{\kern0pt}fm\ {\isacharcolon}{\kern0pt}\ \isanewline
\ \ \ \ {\isachardoublequoteopen}{\isasymlbrakk}\ r{\isasymin}nat\ {\isasymrbrakk}\ {\isasymLongrightarrow}\ arity{\isacharparenleft}{\kern0pt}relation{\isacharunderscore}{\kern0pt}fm{\isacharparenleft}{\kern0pt}r{\isacharparenright}{\kern0pt}{\isacharparenright}{\kern0pt}\ {\isacharequal}{\kern0pt}\ succ{\isacharparenleft}{\kern0pt}r{\isacharparenright}{\kern0pt}{\isachardoublequoteclose}\isanewline
%
\isadelimproof
\ \ %
\endisadelimproof
%
\isatagproof
\isacommand{unfolding}\isamarkupfalse%
\ relation{\isacharunderscore}{\kern0pt}fm{\isacharunderscore}{\kern0pt}def\ \isanewline
\ \ \isacommand{using}\isamarkupfalse%
\ arity{\isacharunderscore}{\kern0pt}pair{\isacharunderscore}{\kern0pt}fm\ nat{\isacharunderscore}{\kern0pt}union{\isacharunderscore}{\kern0pt}abs{\isadigit{1}}\ nat{\isacharunderscore}{\kern0pt}union{\isacharunderscore}{\kern0pt}abs{\isadigit{2}}\ pred{\isacharunderscore}{\kern0pt}Un{\isacharunderscore}{\kern0pt}distrib\isanewline
\ \ \isacommand{by}\isamarkupfalse%
\ simp%
\endisatagproof
{\isafoldproof}%
%
\isadelimproof
\isanewline
%
\endisadelimproof
\isanewline
\isacommand{lemma}\isamarkupfalse%
\ arity{\isacharunderscore}{\kern0pt}restriction{\isacharunderscore}{\kern0pt}fm\ {\isacharcolon}{\kern0pt}\ \isanewline
\ \ \ \ {\isachardoublequoteopen}{\isasymlbrakk}\ r{\isasymin}nat\ {\isacharsemicolon}{\kern0pt}\ z{\isasymin}nat\ {\isacharsemicolon}{\kern0pt}\ A{\isasymin}nat\ {\isasymrbrakk}\ {\isasymLongrightarrow}\ arity{\isacharparenleft}{\kern0pt}restriction{\isacharunderscore}{\kern0pt}fm{\isacharparenleft}{\kern0pt}A{\isacharcomma}{\kern0pt}z{\isacharcomma}{\kern0pt}r{\isacharparenright}{\kern0pt}{\isacharparenright}{\kern0pt}\ {\isacharequal}{\kern0pt}\ succ{\isacharparenleft}{\kern0pt}A{\isacharparenright}{\kern0pt}\ {\isasymunion}\ succ{\isacharparenleft}{\kern0pt}r{\isacharparenright}{\kern0pt}\ {\isasymunion}\ succ{\isacharparenleft}{\kern0pt}z{\isacharparenright}{\kern0pt}{\isachardoublequoteclose}\isanewline
%
\isadelimproof
\ \ %
\endisadelimproof
%
\isatagproof
\isacommand{unfolding}\isamarkupfalse%
\ restriction{\isacharunderscore}{\kern0pt}fm{\isacharunderscore}{\kern0pt}def\ \isanewline
\ \ \isacommand{using}\isamarkupfalse%
\ arity{\isacharunderscore}{\kern0pt}pair{\isacharunderscore}{\kern0pt}fm\ nat{\isacharunderscore}{\kern0pt}union{\isacharunderscore}{\kern0pt}abs{\isadigit{2}}\ pred{\isacharunderscore}{\kern0pt}Un{\isacharunderscore}{\kern0pt}distrib\isanewline
\ \ \isacommand{by}\isamarkupfalse%
\ auto%
\endisatagproof
{\isafoldproof}%
%
\isadelimproof
\isanewline
%
\endisadelimproof
\isanewline
\isacommand{lemma}\isamarkupfalse%
\ arity{\isacharunderscore}{\kern0pt}typed{\isacharunderscore}{\kern0pt}function{\isacharunderscore}{\kern0pt}fm\ {\isacharcolon}{\kern0pt}\ \isanewline
\ \ {\isachardoublequoteopen}{\isasymlbrakk}\ x{\isasymin}nat\ {\isacharsemicolon}{\kern0pt}\ y{\isasymin}nat\ {\isacharsemicolon}{\kern0pt}\ f{\isasymin}nat\ {\isasymrbrakk}\ {\isasymLongrightarrow}\ \isanewline
\ \ \ \ arity{\isacharparenleft}{\kern0pt}typed{\isacharunderscore}{\kern0pt}function{\isacharunderscore}{\kern0pt}fm{\isacharparenleft}{\kern0pt}f{\isacharcomma}{\kern0pt}x{\isacharcomma}{\kern0pt}y{\isacharparenright}{\kern0pt}{\isacharparenright}{\kern0pt}\ {\isacharequal}{\kern0pt}\ {\isasymUnion}\ {\isacharbraceleft}{\kern0pt}succ{\isacharparenleft}{\kern0pt}f{\isacharparenright}{\kern0pt}{\isacharcomma}{\kern0pt}\ succ{\isacharparenleft}{\kern0pt}x{\isacharparenright}{\kern0pt}{\isacharcomma}{\kern0pt}\ succ{\isacharparenleft}{\kern0pt}y{\isacharparenright}{\kern0pt}{\isacharbraceright}{\kern0pt}{\isachardoublequoteclose}\isanewline
%
\isadelimproof
\ \ %
\endisadelimproof
%
\isatagproof
\isacommand{unfolding}\isamarkupfalse%
\ typed{\isacharunderscore}{\kern0pt}function{\isacharunderscore}{\kern0pt}fm{\isacharunderscore}{\kern0pt}def\isanewline
\ \ \isacommand{using}\isamarkupfalse%
\ arity{\isacharunderscore}{\kern0pt}pair{\isacharunderscore}{\kern0pt}fm\ arity{\isacharunderscore}{\kern0pt}relation{\isacharunderscore}{\kern0pt}fm\ arity{\isacharunderscore}{\kern0pt}function{\isacharunderscore}{\kern0pt}fm\ arity{\isacharunderscore}{\kern0pt}domain{\isacharunderscore}{\kern0pt}fm\ \isanewline
\ \ \ \ nat{\isacharunderscore}{\kern0pt}union{\isacharunderscore}{\kern0pt}abs{\isadigit{2}}\ pred{\isacharunderscore}{\kern0pt}Un{\isacharunderscore}{\kern0pt}distrib\isanewline
\ \ \isacommand{by}\isamarkupfalse%
\ auto%
\endisatagproof
{\isafoldproof}%
%
\isadelimproof
\isanewline
%
\endisadelimproof
\isanewline
\isanewline
\isacommand{lemma}\isamarkupfalse%
\ arity{\isacharunderscore}{\kern0pt}subset{\isacharunderscore}{\kern0pt}fm\ {\isacharcolon}{\kern0pt}\ \isanewline
\ \ {\isachardoublequoteopen}{\isasymlbrakk}x{\isasymin}nat\ {\isacharsemicolon}{\kern0pt}\ y{\isasymin}nat{\isasymrbrakk}\ {\isasymLongrightarrow}\ arity{\isacharparenleft}{\kern0pt}subset{\isacharunderscore}{\kern0pt}fm{\isacharparenleft}{\kern0pt}x{\isacharcomma}{\kern0pt}y{\isacharparenright}{\kern0pt}{\isacharparenright}{\kern0pt}\ {\isacharequal}{\kern0pt}\ succ{\isacharparenleft}{\kern0pt}x{\isacharparenright}{\kern0pt}\ {\isasymunion}\ succ{\isacharparenleft}{\kern0pt}y{\isacharparenright}{\kern0pt}{\isachardoublequoteclose}\isanewline
%
\isadelimproof
\ \ %
\endisadelimproof
%
\isatagproof
\isacommand{unfolding}\isamarkupfalse%
\ subset{\isacharunderscore}{\kern0pt}fm{\isacharunderscore}{\kern0pt}def\ \isanewline
\ \ \isacommand{using}\isamarkupfalse%
\ nat{\isacharunderscore}{\kern0pt}union{\isacharunderscore}{\kern0pt}abs{\isadigit{2}}\ pred{\isacharunderscore}{\kern0pt}Un{\isacharunderscore}{\kern0pt}distrib\isanewline
\ \ \isacommand{by}\isamarkupfalse%
\ auto%
\endisatagproof
{\isafoldproof}%
%
\isadelimproof
\isanewline
%
\endisadelimproof
\isanewline
\isacommand{lemma}\isamarkupfalse%
\ arity{\isacharunderscore}{\kern0pt}transset{\isacharunderscore}{\kern0pt}fm\ {\isacharcolon}{\kern0pt}\isanewline
\ \ {\isachardoublequoteopen}{\isasymlbrakk}x{\isasymin}nat{\isasymrbrakk}\ {\isasymLongrightarrow}\ arity{\isacharparenleft}{\kern0pt}transset{\isacharunderscore}{\kern0pt}fm{\isacharparenleft}{\kern0pt}x{\isacharparenright}{\kern0pt}{\isacharparenright}{\kern0pt}\ {\isacharequal}{\kern0pt}\ succ{\isacharparenleft}{\kern0pt}x{\isacharparenright}{\kern0pt}{\isachardoublequoteclose}\isanewline
%
\isadelimproof
\ \ %
\endisadelimproof
%
\isatagproof
\isacommand{unfolding}\isamarkupfalse%
\ transset{\isacharunderscore}{\kern0pt}fm{\isacharunderscore}{\kern0pt}def\ \isanewline
\ \ \isacommand{using}\isamarkupfalse%
\ arity{\isacharunderscore}{\kern0pt}subset{\isacharunderscore}{\kern0pt}fm\ nat{\isacharunderscore}{\kern0pt}union{\isacharunderscore}{\kern0pt}abs{\isadigit{2}}\ pred{\isacharunderscore}{\kern0pt}Un{\isacharunderscore}{\kern0pt}distrib\isanewline
\ \ \isacommand{by}\isamarkupfalse%
\ auto%
\endisatagproof
{\isafoldproof}%
%
\isadelimproof
\isanewline
%
\endisadelimproof
\isanewline
\isacommand{lemma}\isamarkupfalse%
\ arity{\isacharunderscore}{\kern0pt}ordinal{\isacharunderscore}{\kern0pt}fm\ {\isacharcolon}{\kern0pt}\isanewline
\ \ {\isachardoublequoteopen}{\isasymlbrakk}x{\isasymin}nat{\isasymrbrakk}\ {\isasymLongrightarrow}\ arity{\isacharparenleft}{\kern0pt}ordinal{\isacharunderscore}{\kern0pt}fm{\isacharparenleft}{\kern0pt}x{\isacharparenright}{\kern0pt}{\isacharparenright}{\kern0pt}\ {\isacharequal}{\kern0pt}\ succ{\isacharparenleft}{\kern0pt}x{\isacharparenright}{\kern0pt}{\isachardoublequoteclose}\isanewline
%
\isadelimproof
\ \ %
\endisadelimproof
%
\isatagproof
\isacommand{unfolding}\isamarkupfalse%
\ ordinal{\isacharunderscore}{\kern0pt}fm{\isacharunderscore}{\kern0pt}def\ \isanewline
\ \ \isacommand{using}\isamarkupfalse%
\ arity{\isacharunderscore}{\kern0pt}transset{\isacharunderscore}{\kern0pt}fm\ nat{\isacharunderscore}{\kern0pt}union{\isacharunderscore}{\kern0pt}abs{\isadigit{2}}\ pred{\isacharunderscore}{\kern0pt}Un{\isacharunderscore}{\kern0pt}distrib\isanewline
\ \ \isacommand{by}\isamarkupfalse%
\ auto%
\endisatagproof
{\isafoldproof}%
%
\isadelimproof
\isanewline
%
\endisadelimproof
\isanewline
\isacommand{lemma}\isamarkupfalse%
\ arity{\isacharunderscore}{\kern0pt}limit{\isacharunderscore}{\kern0pt}ordinal{\isacharunderscore}{\kern0pt}fm\ {\isacharcolon}{\kern0pt}\isanewline
\ \ {\isachardoublequoteopen}{\isasymlbrakk}x{\isasymin}nat{\isasymrbrakk}\ {\isasymLongrightarrow}\ arity{\isacharparenleft}{\kern0pt}limit{\isacharunderscore}{\kern0pt}ordinal{\isacharunderscore}{\kern0pt}fm{\isacharparenleft}{\kern0pt}x{\isacharparenright}{\kern0pt}{\isacharparenright}{\kern0pt}\ {\isacharequal}{\kern0pt}\ succ{\isacharparenleft}{\kern0pt}x{\isacharparenright}{\kern0pt}{\isachardoublequoteclose}\isanewline
%
\isadelimproof
\ \ %
\endisadelimproof
%
\isatagproof
\isacommand{unfolding}\isamarkupfalse%
\ limit{\isacharunderscore}{\kern0pt}ordinal{\isacharunderscore}{\kern0pt}fm{\isacharunderscore}{\kern0pt}def\ \isanewline
\ \ \isacommand{using}\isamarkupfalse%
\ arity{\isacharunderscore}{\kern0pt}ordinal{\isacharunderscore}{\kern0pt}fm\ arity{\isacharunderscore}{\kern0pt}succ{\isacharunderscore}{\kern0pt}fm\ arity{\isacharunderscore}{\kern0pt}empty{\isacharunderscore}{\kern0pt}fm\ nat{\isacharunderscore}{\kern0pt}union{\isacharunderscore}{\kern0pt}abs{\isadigit{2}}\ pred{\isacharunderscore}{\kern0pt}Un{\isacharunderscore}{\kern0pt}distrib\isanewline
\ \ \isacommand{by}\isamarkupfalse%
\ auto%
\endisatagproof
{\isafoldproof}%
%
\isadelimproof
\isanewline
%
\endisadelimproof
\isanewline
\isacommand{lemma}\isamarkupfalse%
\ arity{\isacharunderscore}{\kern0pt}finite{\isacharunderscore}{\kern0pt}ordinal{\isacharunderscore}{\kern0pt}fm\ {\isacharcolon}{\kern0pt}\isanewline
\ \ {\isachardoublequoteopen}{\isasymlbrakk}x{\isasymin}nat{\isasymrbrakk}\ {\isasymLongrightarrow}\ arity{\isacharparenleft}{\kern0pt}finite{\isacharunderscore}{\kern0pt}ordinal{\isacharunderscore}{\kern0pt}fm{\isacharparenleft}{\kern0pt}x{\isacharparenright}{\kern0pt}{\isacharparenright}{\kern0pt}\ {\isacharequal}{\kern0pt}\ succ{\isacharparenleft}{\kern0pt}x{\isacharparenright}{\kern0pt}{\isachardoublequoteclose}\isanewline
%
\isadelimproof
\ \ %
\endisadelimproof
%
\isatagproof
\isacommand{unfolding}\isamarkupfalse%
\ finite{\isacharunderscore}{\kern0pt}ordinal{\isacharunderscore}{\kern0pt}fm{\isacharunderscore}{\kern0pt}def\ \isanewline
\ \ \isacommand{using}\isamarkupfalse%
\ arity{\isacharunderscore}{\kern0pt}ordinal{\isacharunderscore}{\kern0pt}fm\ arity{\isacharunderscore}{\kern0pt}limit{\isacharunderscore}{\kern0pt}ordinal{\isacharunderscore}{\kern0pt}fm\ arity{\isacharunderscore}{\kern0pt}succ{\isacharunderscore}{\kern0pt}fm\ arity{\isacharunderscore}{\kern0pt}empty{\isacharunderscore}{\kern0pt}fm\ \isanewline
\ \ \ \ nat{\isacharunderscore}{\kern0pt}union{\isacharunderscore}{\kern0pt}abs{\isadigit{2}}\ pred{\isacharunderscore}{\kern0pt}Un{\isacharunderscore}{\kern0pt}distrib\isanewline
\ \ \isacommand{by}\isamarkupfalse%
\ auto%
\endisatagproof
{\isafoldproof}%
%
\isadelimproof
\isanewline
%
\endisadelimproof
\isanewline
\isacommand{lemma}\isamarkupfalse%
\ arity{\isacharunderscore}{\kern0pt}omega{\isacharunderscore}{\kern0pt}fm\ {\isacharcolon}{\kern0pt}\isanewline
\ \ {\isachardoublequoteopen}{\isasymlbrakk}x{\isasymin}nat{\isasymrbrakk}\ {\isasymLongrightarrow}\ arity{\isacharparenleft}{\kern0pt}omega{\isacharunderscore}{\kern0pt}fm{\isacharparenleft}{\kern0pt}x{\isacharparenright}{\kern0pt}{\isacharparenright}{\kern0pt}\ {\isacharequal}{\kern0pt}\ succ{\isacharparenleft}{\kern0pt}x{\isacharparenright}{\kern0pt}{\isachardoublequoteclose}\isanewline
%
\isadelimproof
\ \ %
\endisadelimproof
%
\isatagproof
\isacommand{unfolding}\isamarkupfalse%
\ omega{\isacharunderscore}{\kern0pt}fm{\isacharunderscore}{\kern0pt}def\ \isanewline
\ \ \isacommand{using}\isamarkupfalse%
\ arity{\isacharunderscore}{\kern0pt}limit{\isacharunderscore}{\kern0pt}ordinal{\isacharunderscore}{\kern0pt}fm\ nat{\isacharunderscore}{\kern0pt}union{\isacharunderscore}{\kern0pt}abs{\isadigit{2}}\ pred{\isacharunderscore}{\kern0pt}Un{\isacharunderscore}{\kern0pt}distrib\isanewline
\ \ \isacommand{by}\isamarkupfalse%
\ auto%
\endisatagproof
{\isafoldproof}%
%
\isadelimproof
\isanewline
%
\endisadelimproof
\isanewline
\isacommand{lemma}\isamarkupfalse%
\ arity{\isacharunderscore}{\kern0pt}cartprod{\isacharunderscore}{\kern0pt}fm\ {\isacharcolon}{\kern0pt}\ \isanewline
\ \ {\isachardoublequoteopen}{\isasymlbrakk}\ A{\isasymin}nat\ {\isacharsemicolon}{\kern0pt}\ B{\isasymin}nat\ {\isacharsemicolon}{\kern0pt}\ z{\isasymin}nat\ {\isasymrbrakk}\ {\isasymLongrightarrow}\ arity{\isacharparenleft}{\kern0pt}cartprod{\isacharunderscore}{\kern0pt}fm{\isacharparenleft}{\kern0pt}A{\isacharcomma}{\kern0pt}B{\isacharcomma}{\kern0pt}z{\isacharparenright}{\kern0pt}{\isacharparenright}{\kern0pt}\ {\isacharequal}{\kern0pt}\ succ{\isacharparenleft}{\kern0pt}A{\isacharparenright}{\kern0pt}\ {\isasymunion}\ succ{\isacharparenleft}{\kern0pt}B{\isacharparenright}{\kern0pt}\ {\isasymunion}\ succ{\isacharparenleft}{\kern0pt}z{\isacharparenright}{\kern0pt}{\isachardoublequoteclose}\isanewline
%
\isadelimproof
\ \ %
\endisadelimproof
%
\isatagproof
\isacommand{unfolding}\isamarkupfalse%
\ cartprod{\isacharunderscore}{\kern0pt}fm{\isacharunderscore}{\kern0pt}def\isanewline
\ \ \isacommand{using}\isamarkupfalse%
\ arity{\isacharunderscore}{\kern0pt}pair{\isacharunderscore}{\kern0pt}fm\ nat{\isacharunderscore}{\kern0pt}union{\isacharunderscore}{\kern0pt}abs{\isadigit{2}}\ pred{\isacharunderscore}{\kern0pt}Un{\isacharunderscore}{\kern0pt}distrib\isanewline
\ \ \isacommand{by}\isamarkupfalse%
\ auto%
\endisatagproof
{\isafoldproof}%
%
\isadelimproof
\isanewline
%
\endisadelimproof
\isanewline
\isacommand{lemma}\isamarkupfalse%
\ arity{\isacharunderscore}{\kern0pt}fst{\isacharunderscore}{\kern0pt}fm\ {\isacharcolon}{\kern0pt}\isanewline
\ \ {\isachardoublequoteopen}{\isasymlbrakk}x{\isasymin}nat\ {\isacharsemicolon}{\kern0pt}\ t{\isasymin}nat{\isasymrbrakk}\ {\isasymLongrightarrow}\ arity{\isacharparenleft}{\kern0pt}fst{\isacharunderscore}{\kern0pt}fm{\isacharparenleft}{\kern0pt}x{\isacharcomma}{\kern0pt}t{\isacharparenright}{\kern0pt}{\isacharparenright}{\kern0pt}\ {\isacharequal}{\kern0pt}\ succ{\isacharparenleft}{\kern0pt}x{\isacharparenright}{\kern0pt}\ {\isasymunion}\ succ{\isacharparenleft}{\kern0pt}t{\isacharparenright}{\kern0pt}{\isachardoublequoteclose}\isanewline
%
\isadelimproof
\ \ %
\endisadelimproof
%
\isatagproof
\isacommand{unfolding}\isamarkupfalse%
\ fst{\isacharunderscore}{\kern0pt}fm{\isacharunderscore}{\kern0pt}def\isanewline
\ \ \isacommand{using}\isamarkupfalse%
\ arity{\isacharunderscore}{\kern0pt}pair{\isacharunderscore}{\kern0pt}fm\ arity{\isacharunderscore}{\kern0pt}empty{\isacharunderscore}{\kern0pt}fm\ nat{\isacharunderscore}{\kern0pt}union{\isacharunderscore}{\kern0pt}abs{\isadigit{2}}\ pred{\isacharunderscore}{\kern0pt}Un{\isacharunderscore}{\kern0pt}distrib\isanewline
\ \ \isacommand{by}\isamarkupfalse%
\ auto%
\endisatagproof
{\isafoldproof}%
%
\isadelimproof
\isanewline
%
\endisadelimproof
\isanewline
\isacommand{lemma}\isamarkupfalse%
\ arity{\isacharunderscore}{\kern0pt}snd{\isacharunderscore}{\kern0pt}fm\ {\isacharcolon}{\kern0pt}\isanewline
\ \ {\isachardoublequoteopen}{\isasymlbrakk}x{\isasymin}nat\ {\isacharsemicolon}{\kern0pt}\ t{\isasymin}nat{\isasymrbrakk}\ {\isasymLongrightarrow}\ arity{\isacharparenleft}{\kern0pt}snd{\isacharunderscore}{\kern0pt}fm{\isacharparenleft}{\kern0pt}x{\isacharcomma}{\kern0pt}t{\isacharparenright}{\kern0pt}{\isacharparenright}{\kern0pt}\ {\isacharequal}{\kern0pt}\ succ{\isacharparenleft}{\kern0pt}x{\isacharparenright}{\kern0pt}\ {\isasymunion}\ succ{\isacharparenleft}{\kern0pt}t{\isacharparenright}{\kern0pt}{\isachardoublequoteclose}\isanewline
%
\isadelimproof
\ \ %
\endisadelimproof
%
\isatagproof
\isacommand{unfolding}\isamarkupfalse%
\ snd{\isacharunderscore}{\kern0pt}fm{\isacharunderscore}{\kern0pt}def\isanewline
\ \ \isacommand{using}\isamarkupfalse%
\ arity{\isacharunderscore}{\kern0pt}pair{\isacharunderscore}{\kern0pt}fm\ arity{\isacharunderscore}{\kern0pt}empty{\isacharunderscore}{\kern0pt}fm\ nat{\isacharunderscore}{\kern0pt}union{\isacharunderscore}{\kern0pt}abs{\isadigit{2}}\ pred{\isacharunderscore}{\kern0pt}Un{\isacharunderscore}{\kern0pt}distrib\isanewline
\ \ \isacommand{by}\isamarkupfalse%
\ auto%
\endisatagproof
{\isafoldproof}%
%
\isadelimproof
\isanewline
%
\endisadelimproof
\isanewline
\isacommand{lemma}\isamarkupfalse%
\ arity{\isacharunderscore}{\kern0pt}snd{\isacharunderscore}{\kern0pt}snd{\isacharunderscore}{\kern0pt}fm\ {\isacharcolon}{\kern0pt}\isanewline
\ \ {\isachardoublequoteopen}{\isasymlbrakk}x{\isasymin}nat\ {\isacharsemicolon}{\kern0pt}\ t{\isasymin}nat{\isasymrbrakk}\ {\isasymLongrightarrow}\ arity{\isacharparenleft}{\kern0pt}snd{\isacharunderscore}{\kern0pt}snd{\isacharunderscore}{\kern0pt}fm{\isacharparenleft}{\kern0pt}x{\isacharcomma}{\kern0pt}t{\isacharparenright}{\kern0pt}{\isacharparenright}{\kern0pt}\ {\isacharequal}{\kern0pt}\ succ{\isacharparenleft}{\kern0pt}x{\isacharparenright}{\kern0pt}\ {\isasymunion}\ succ{\isacharparenleft}{\kern0pt}t{\isacharparenright}{\kern0pt}{\isachardoublequoteclose}\isanewline
%
\isadelimproof
\ \ %
\endisadelimproof
%
\isatagproof
\isacommand{unfolding}\isamarkupfalse%
\ snd{\isacharunderscore}{\kern0pt}snd{\isacharunderscore}{\kern0pt}fm{\isacharunderscore}{\kern0pt}def\ hcomp{\isacharunderscore}{\kern0pt}fm{\isacharunderscore}{\kern0pt}def\isanewline
\ \ \isacommand{using}\isamarkupfalse%
\ arity{\isacharunderscore}{\kern0pt}snd{\isacharunderscore}{\kern0pt}fm\ arity{\isacharunderscore}{\kern0pt}empty{\isacharunderscore}{\kern0pt}fm\ nat{\isacharunderscore}{\kern0pt}union{\isacharunderscore}{\kern0pt}abs{\isadigit{2}}\ pred{\isacharunderscore}{\kern0pt}Un{\isacharunderscore}{\kern0pt}distrib\isanewline
\ \ \isacommand{by}\isamarkupfalse%
\ auto%
\endisatagproof
{\isafoldproof}%
%
\isadelimproof
\isanewline
%
\endisadelimproof
\isanewline
\isacommand{lemma}\isamarkupfalse%
\ arity{\isacharunderscore}{\kern0pt}ftype{\isacharunderscore}{\kern0pt}fm\ {\isacharcolon}{\kern0pt}\isanewline
\ \ {\isachardoublequoteopen}{\isasymlbrakk}x{\isasymin}nat\ {\isacharsemicolon}{\kern0pt}\ t{\isasymin}nat{\isasymrbrakk}\ {\isasymLongrightarrow}\ arity{\isacharparenleft}{\kern0pt}ftype{\isacharunderscore}{\kern0pt}fm{\isacharparenleft}{\kern0pt}x{\isacharcomma}{\kern0pt}t{\isacharparenright}{\kern0pt}{\isacharparenright}{\kern0pt}\ {\isacharequal}{\kern0pt}\ succ{\isacharparenleft}{\kern0pt}x{\isacharparenright}{\kern0pt}\ {\isasymunion}\ succ{\isacharparenleft}{\kern0pt}t{\isacharparenright}{\kern0pt}{\isachardoublequoteclose}\isanewline
%
\isadelimproof
\ \ %
\endisadelimproof
%
\isatagproof
\isacommand{unfolding}\isamarkupfalse%
\ ftype{\isacharunderscore}{\kern0pt}fm{\isacharunderscore}{\kern0pt}def\isanewline
\ \ \isacommand{using}\isamarkupfalse%
\ arity{\isacharunderscore}{\kern0pt}fst{\isacharunderscore}{\kern0pt}fm\ \isanewline
\ \ \isacommand{by}\isamarkupfalse%
\ auto%
\endisatagproof
{\isafoldproof}%
%
\isadelimproof
\isanewline
%
\endisadelimproof
\isanewline
\isacommand{lemma}\isamarkupfalse%
\ name{\isadigit{1}}arity{\isacharunderscore}{\kern0pt}{\isacharunderscore}{\kern0pt}fm\ {\isacharcolon}{\kern0pt}\isanewline
\ \ {\isachardoublequoteopen}{\isasymlbrakk}x{\isasymin}nat\ {\isacharsemicolon}{\kern0pt}\ t{\isasymin}nat{\isasymrbrakk}\ {\isasymLongrightarrow}\ arity{\isacharparenleft}{\kern0pt}name{\isadigit{1}}{\isacharunderscore}{\kern0pt}fm{\isacharparenleft}{\kern0pt}x{\isacharcomma}{\kern0pt}t{\isacharparenright}{\kern0pt}{\isacharparenright}{\kern0pt}\ {\isacharequal}{\kern0pt}\ succ{\isacharparenleft}{\kern0pt}x{\isacharparenright}{\kern0pt}\ {\isasymunion}\ succ{\isacharparenleft}{\kern0pt}t{\isacharparenright}{\kern0pt}{\isachardoublequoteclose}\isanewline
%
\isadelimproof
\ \ %
\endisadelimproof
%
\isatagproof
\isacommand{unfolding}\isamarkupfalse%
\ name{\isadigit{1}}{\isacharunderscore}{\kern0pt}fm{\isacharunderscore}{\kern0pt}def\ hcomp{\isacharunderscore}{\kern0pt}fm{\isacharunderscore}{\kern0pt}def\isanewline
\ \ \isacommand{using}\isamarkupfalse%
\ arity{\isacharunderscore}{\kern0pt}fst{\isacharunderscore}{\kern0pt}fm\ arity{\isacharunderscore}{\kern0pt}snd{\isacharunderscore}{\kern0pt}fm\ nat{\isacharunderscore}{\kern0pt}union{\isacharunderscore}{\kern0pt}abs{\isadigit{2}}\ pred{\isacharunderscore}{\kern0pt}Un{\isacharunderscore}{\kern0pt}distrib\isanewline
\ \ \isacommand{by}\isamarkupfalse%
\ auto%
\endisatagproof
{\isafoldproof}%
%
\isadelimproof
\isanewline
%
\endisadelimproof
\isanewline
\isacommand{lemma}\isamarkupfalse%
\ name{\isadigit{2}}arity{\isacharunderscore}{\kern0pt}{\isacharunderscore}{\kern0pt}fm\ {\isacharcolon}{\kern0pt}\isanewline
\ \ {\isachardoublequoteopen}{\isasymlbrakk}x{\isasymin}nat\ {\isacharsemicolon}{\kern0pt}\ t{\isasymin}nat{\isasymrbrakk}\ {\isasymLongrightarrow}\ arity{\isacharparenleft}{\kern0pt}name{\isadigit{2}}{\isacharunderscore}{\kern0pt}fm{\isacharparenleft}{\kern0pt}x{\isacharcomma}{\kern0pt}t{\isacharparenright}{\kern0pt}{\isacharparenright}{\kern0pt}\ {\isacharequal}{\kern0pt}\ succ{\isacharparenleft}{\kern0pt}x{\isacharparenright}{\kern0pt}\ {\isasymunion}\ succ{\isacharparenleft}{\kern0pt}t{\isacharparenright}{\kern0pt}{\isachardoublequoteclose}\isanewline
%
\isadelimproof
\ \ %
\endisadelimproof
%
\isatagproof
\isacommand{unfolding}\isamarkupfalse%
\ name{\isadigit{2}}{\isacharunderscore}{\kern0pt}fm{\isacharunderscore}{\kern0pt}def\ hcomp{\isacharunderscore}{\kern0pt}fm{\isacharunderscore}{\kern0pt}def\isanewline
\ \ \isacommand{using}\isamarkupfalse%
\ arity{\isacharunderscore}{\kern0pt}fst{\isacharunderscore}{\kern0pt}fm\ arity{\isacharunderscore}{\kern0pt}snd{\isacharunderscore}{\kern0pt}snd{\isacharunderscore}{\kern0pt}fm\ nat{\isacharunderscore}{\kern0pt}union{\isacharunderscore}{\kern0pt}abs{\isadigit{2}}\ pred{\isacharunderscore}{\kern0pt}Un{\isacharunderscore}{\kern0pt}distrib\isanewline
\ \ \isacommand{by}\isamarkupfalse%
\ auto%
\endisatagproof
{\isafoldproof}%
%
\isadelimproof
\isanewline
%
\endisadelimproof
\isanewline
\isacommand{lemma}\isamarkupfalse%
\ arity{\isacharunderscore}{\kern0pt}cond{\isacharunderscore}{\kern0pt}of{\isacharunderscore}{\kern0pt}fm\ {\isacharcolon}{\kern0pt}\isanewline
\ \ {\isachardoublequoteopen}{\isasymlbrakk}x{\isasymin}nat\ {\isacharsemicolon}{\kern0pt}\ t{\isasymin}nat{\isasymrbrakk}\ {\isasymLongrightarrow}\ arity{\isacharparenleft}{\kern0pt}cond{\isacharunderscore}{\kern0pt}of{\isacharunderscore}{\kern0pt}fm{\isacharparenleft}{\kern0pt}x{\isacharcomma}{\kern0pt}t{\isacharparenright}{\kern0pt}{\isacharparenright}{\kern0pt}\ {\isacharequal}{\kern0pt}\ succ{\isacharparenleft}{\kern0pt}x{\isacharparenright}{\kern0pt}\ {\isasymunion}\ succ{\isacharparenleft}{\kern0pt}t{\isacharparenright}{\kern0pt}{\isachardoublequoteclose}\isanewline
%
\isadelimproof
\ \ %
\endisadelimproof
%
\isatagproof
\isacommand{unfolding}\isamarkupfalse%
\ cond{\isacharunderscore}{\kern0pt}of{\isacharunderscore}{\kern0pt}fm{\isacharunderscore}{\kern0pt}def\ hcomp{\isacharunderscore}{\kern0pt}fm{\isacharunderscore}{\kern0pt}def\isanewline
\ \ \isacommand{using}\isamarkupfalse%
\ arity{\isacharunderscore}{\kern0pt}snd{\isacharunderscore}{\kern0pt}fm\ arity{\isacharunderscore}{\kern0pt}snd{\isacharunderscore}{\kern0pt}snd{\isacharunderscore}{\kern0pt}fm\ nat{\isacharunderscore}{\kern0pt}union{\isacharunderscore}{\kern0pt}abs{\isadigit{2}}\ pred{\isacharunderscore}{\kern0pt}Un{\isacharunderscore}{\kern0pt}distrib\isanewline
\ \ \isacommand{by}\isamarkupfalse%
\ auto%
\endisatagproof
{\isafoldproof}%
%
\isadelimproof
\isanewline
%
\endisadelimproof
\isanewline
\isacommand{lemma}\isamarkupfalse%
\ arity{\isacharunderscore}{\kern0pt}singleton{\isacharunderscore}{\kern0pt}fm\ {\isacharcolon}{\kern0pt}\isanewline
\ \ {\isachardoublequoteopen}{\isasymlbrakk}x{\isasymin}nat\ {\isacharsemicolon}{\kern0pt}\ t{\isasymin}nat{\isasymrbrakk}\ {\isasymLongrightarrow}\ arity{\isacharparenleft}{\kern0pt}singleton{\isacharunderscore}{\kern0pt}fm{\isacharparenleft}{\kern0pt}x{\isacharcomma}{\kern0pt}t{\isacharparenright}{\kern0pt}{\isacharparenright}{\kern0pt}\ {\isacharequal}{\kern0pt}\ succ{\isacharparenleft}{\kern0pt}x{\isacharparenright}{\kern0pt}\ {\isasymunion}\ succ{\isacharparenleft}{\kern0pt}t{\isacharparenright}{\kern0pt}{\isachardoublequoteclose}\isanewline
%
\isadelimproof
\ \ %
\endisadelimproof
%
\isatagproof
\isacommand{unfolding}\isamarkupfalse%
\ singleton{\isacharunderscore}{\kern0pt}fm{\isacharunderscore}{\kern0pt}def\ cons{\isacharunderscore}{\kern0pt}fm{\isacharunderscore}{\kern0pt}def\isanewline
\ \ \isacommand{using}\isamarkupfalse%
\ arity{\isacharunderscore}{\kern0pt}union{\isacharunderscore}{\kern0pt}fm\ arity{\isacharunderscore}{\kern0pt}upair{\isacharunderscore}{\kern0pt}fm\ arity{\isacharunderscore}{\kern0pt}empty{\isacharunderscore}{\kern0pt}fm\ nat{\isacharunderscore}{\kern0pt}union{\isacharunderscore}{\kern0pt}abs{\isadigit{2}}\ pred{\isacharunderscore}{\kern0pt}Un{\isacharunderscore}{\kern0pt}distrib\isanewline
\ \ \isacommand{by}\isamarkupfalse%
\ auto%
\endisatagproof
{\isafoldproof}%
%
\isadelimproof
\isanewline
%
\endisadelimproof
\isanewline
\isacommand{lemma}\isamarkupfalse%
\ arity{\isacharunderscore}{\kern0pt}Memrel{\isacharunderscore}{\kern0pt}fm\ {\isacharcolon}{\kern0pt}\isanewline
\ \ {\isachardoublequoteopen}{\isasymlbrakk}x{\isasymin}nat\ {\isacharsemicolon}{\kern0pt}\ t{\isasymin}nat{\isasymrbrakk}\ {\isasymLongrightarrow}\ arity{\isacharparenleft}{\kern0pt}Memrel{\isacharunderscore}{\kern0pt}fm{\isacharparenleft}{\kern0pt}x{\isacharcomma}{\kern0pt}t{\isacharparenright}{\kern0pt}{\isacharparenright}{\kern0pt}\ {\isacharequal}{\kern0pt}\ succ{\isacharparenleft}{\kern0pt}x{\isacharparenright}{\kern0pt}\ {\isasymunion}\ succ{\isacharparenleft}{\kern0pt}t{\isacharparenright}{\kern0pt}{\isachardoublequoteclose}\isanewline
%
\isadelimproof
\ \ %
\endisadelimproof
%
\isatagproof
\isacommand{unfolding}\isamarkupfalse%
\ Memrel{\isacharunderscore}{\kern0pt}fm{\isacharunderscore}{\kern0pt}def\ \isanewline
\ \ \isacommand{using}\isamarkupfalse%
\ \ arity{\isacharunderscore}{\kern0pt}pair{\isacharunderscore}{\kern0pt}fm\ nat{\isacharunderscore}{\kern0pt}union{\isacharunderscore}{\kern0pt}abs{\isadigit{2}}\ pred{\isacharunderscore}{\kern0pt}Un{\isacharunderscore}{\kern0pt}distrib\isanewline
\ \ \isacommand{by}\isamarkupfalse%
\ auto%
\endisatagproof
{\isafoldproof}%
%
\isadelimproof
\isanewline
%
\endisadelimproof
\isanewline
\isacommand{lemma}\isamarkupfalse%
\ arity{\isacharunderscore}{\kern0pt}quasinat{\isacharunderscore}{\kern0pt}fm\ {\isacharcolon}{\kern0pt}\isanewline
\ \ {\isachardoublequoteopen}{\isasymlbrakk}x{\isasymin}nat{\isasymrbrakk}\ {\isasymLongrightarrow}\ arity{\isacharparenleft}{\kern0pt}quasinat{\isacharunderscore}{\kern0pt}fm{\isacharparenleft}{\kern0pt}x{\isacharparenright}{\kern0pt}{\isacharparenright}{\kern0pt}\ {\isacharequal}{\kern0pt}\ succ{\isacharparenleft}{\kern0pt}x{\isacharparenright}{\kern0pt}{\isachardoublequoteclose}\isanewline
%
\isadelimproof
\ \ %
\endisadelimproof
%
\isatagproof
\isacommand{unfolding}\isamarkupfalse%
\ quasinat{\isacharunderscore}{\kern0pt}fm{\isacharunderscore}{\kern0pt}def\ cons{\isacharunderscore}{\kern0pt}fm{\isacharunderscore}{\kern0pt}def\ \isanewline
\ \ \isacommand{using}\isamarkupfalse%
\ arity{\isacharunderscore}{\kern0pt}succ{\isacharunderscore}{\kern0pt}fm\ arity{\isacharunderscore}{\kern0pt}empty{\isacharunderscore}{\kern0pt}fm\isanewline
\ \ \ \ nat{\isacharunderscore}{\kern0pt}union{\isacharunderscore}{\kern0pt}abs{\isadigit{2}}\ pred{\isacharunderscore}{\kern0pt}Un{\isacharunderscore}{\kern0pt}distrib\isanewline
\ \ \isacommand{by}\isamarkupfalse%
\ auto%
\endisatagproof
{\isafoldproof}%
%
\isadelimproof
\isanewline
%
\endisadelimproof
\isanewline
\isacommand{lemma}\isamarkupfalse%
\ arity{\isacharunderscore}{\kern0pt}is{\isacharunderscore}{\kern0pt}recfun{\isacharunderscore}{\kern0pt}fm\ {\isacharcolon}{\kern0pt}\isanewline
\ \ {\isachardoublequoteopen}{\isasymlbrakk}p{\isasymin}formula\ {\isacharsemicolon}{\kern0pt}\ v{\isasymin}nat\ {\isacharsemicolon}{\kern0pt}\ n{\isasymin}nat{\isacharsemicolon}{\kern0pt}\ Z{\isasymin}nat{\isacharsemicolon}{\kern0pt}i{\isasymin}nat{\isasymrbrakk}\ {\isasymLongrightarrow}\ \ arity{\isacharparenleft}{\kern0pt}p{\isacharparenright}{\kern0pt}\ {\isacharequal}{\kern0pt}\ i\ {\isasymLongrightarrow}\ \isanewline
\ \ arity{\isacharparenleft}{\kern0pt}is{\isacharunderscore}{\kern0pt}recfun{\isacharunderscore}{\kern0pt}fm{\isacharparenleft}{\kern0pt}p{\isacharcomma}{\kern0pt}v{\isacharcomma}{\kern0pt}n{\isacharcomma}{\kern0pt}Z{\isacharparenright}{\kern0pt}{\isacharparenright}{\kern0pt}\ {\isacharequal}{\kern0pt}\ succ{\isacharparenleft}{\kern0pt}v{\isacharparenright}{\kern0pt}\ {\isasymunion}\ succ{\isacharparenleft}{\kern0pt}n{\isacharparenright}{\kern0pt}\ {\isasymunion}\ succ{\isacharparenleft}{\kern0pt}Z{\isacharparenright}{\kern0pt}\ {\isasymunion}\ pred{\isacharparenleft}{\kern0pt}pred{\isacharparenleft}{\kern0pt}pred{\isacharparenleft}{\kern0pt}pred{\isacharparenleft}{\kern0pt}i{\isacharparenright}{\kern0pt}{\isacharparenright}{\kern0pt}{\isacharparenright}{\kern0pt}{\isacharparenright}{\kern0pt}{\isachardoublequoteclose}\isanewline
%
\isadelimproof
\ \ %
\endisadelimproof
%
\isatagproof
\isacommand{unfolding}\isamarkupfalse%
\ is{\isacharunderscore}{\kern0pt}recfun{\isacharunderscore}{\kern0pt}fm{\isacharunderscore}{\kern0pt}def\isanewline
\ \ \isacommand{using}\isamarkupfalse%
\ arity{\isacharunderscore}{\kern0pt}upair{\isacharunderscore}{\kern0pt}fm\ arity{\isacharunderscore}{\kern0pt}pair{\isacharunderscore}{\kern0pt}fm\ arity{\isacharunderscore}{\kern0pt}pre{\isacharunderscore}{\kern0pt}image{\isacharunderscore}{\kern0pt}fm\ arity{\isacharunderscore}{\kern0pt}restriction{\isacharunderscore}{\kern0pt}fm\isanewline
\ \ \ \ nat{\isacharunderscore}{\kern0pt}union{\isacharunderscore}{\kern0pt}abs{\isadigit{2}}\ pred{\isacharunderscore}{\kern0pt}Un{\isacharunderscore}{\kern0pt}distrib\isanewline
\ \ \isacommand{by}\isamarkupfalse%
\ auto%
\endisatagproof
{\isafoldproof}%
%
\isadelimproof
\isanewline
%
\endisadelimproof
\isanewline
\isacommand{lemma}\isamarkupfalse%
\ arity{\isacharunderscore}{\kern0pt}is{\isacharunderscore}{\kern0pt}wfrec{\isacharunderscore}{\kern0pt}fm\ {\isacharcolon}{\kern0pt}\isanewline
\ \ {\isachardoublequoteopen}{\isasymlbrakk}p{\isasymin}formula\ {\isacharsemicolon}{\kern0pt}\ v{\isasymin}nat\ {\isacharsemicolon}{\kern0pt}\ n{\isasymin}nat{\isacharsemicolon}{\kern0pt}\ Z{\isasymin}nat\ {\isacharsemicolon}{\kern0pt}\ i{\isasymin}nat{\isasymrbrakk}\ {\isasymLongrightarrow}\ arity{\isacharparenleft}{\kern0pt}p{\isacharparenright}{\kern0pt}\ {\isacharequal}{\kern0pt}\ i\ {\isasymLongrightarrow}\ \isanewline
\ \ \ \ arity{\isacharparenleft}{\kern0pt}is{\isacharunderscore}{\kern0pt}wfrec{\isacharunderscore}{\kern0pt}fm{\isacharparenleft}{\kern0pt}p{\isacharcomma}{\kern0pt}v{\isacharcomma}{\kern0pt}n{\isacharcomma}{\kern0pt}Z{\isacharparenright}{\kern0pt}{\isacharparenright}{\kern0pt}\ {\isacharequal}{\kern0pt}\ succ{\isacharparenleft}{\kern0pt}v{\isacharparenright}{\kern0pt}\ {\isasymunion}\ succ{\isacharparenleft}{\kern0pt}n{\isacharparenright}{\kern0pt}\ {\isasymunion}\ succ{\isacharparenleft}{\kern0pt}Z{\isacharparenright}{\kern0pt}\ {\isasymunion}\ pred{\isacharparenleft}{\kern0pt}pred{\isacharparenleft}{\kern0pt}pred{\isacharparenleft}{\kern0pt}pred{\isacharparenleft}{\kern0pt}pred{\isacharparenleft}{\kern0pt}i{\isacharparenright}{\kern0pt}{\isacharparenright}{\kern0pt}{\isacharparenright}{\kern0pt}{\isacharparenright}{\kern0pt}{\isacharparenright}{\kern0pt}{\isachardoublequoteclose}\isanewline
%
\isadelimproof
\ \ %
\endisadelimproof
%
\isatagproof
\isacommand{unfolding}\isamarkupfalse%
\ is{\isacharunderscore}{\kern0pt}wfrec{\isacharunderscore}{\kern0pt}fm{\isacharunderscore}{\kern0pt}def\isanewline
\ \ \isacommand{using}\isamarkupfalse%
\ arity{\isacharunderscore}{\kern0pt}succ{\isacharunderscore}{\kern0pt}fm\ \ arity{\isacharunderscore}{\kern0pt}is{\isacharunderscore}{\kern0pt}recfun{\isacharunderscore}{\kern0pt}fm\ \isanewline
\ \ \ \ \ nat{\isacharunderscore}{\kern0pt}union{\isacharunderscore}{\kern0pt}abs{\isadigit{2}}\ pred{\isacharunderscore}{\kern0pt}Un{\isacharunderscore}{\kern0pt}distrib\isanewline
\ \ \isacommand{by}\isamarkupfalse%
\ auto%
\endisatagproof
{\isafoldproof}%
%
\isadelimproof
\isanewline
%
\endisadelimproof
\isanewline
\isacommand{lemma}\isamarkupfalse%
\ arity{\isacharunderscore}{\kern0pt}is{\isacharunderscore}{\kern0pt}nat{\isacharunderscore}{\kern0pt}case{\isacharunderscore}{\kern0pt}fm\ {\isacharcolon}{\kern0pt}\isanewline
\ \ {\isachardoublequoteopen}{\isasymlbrakk}p{\isasymin}formula\ {\isacharsemicolon}{\kern0pt}\ v{\isasymin}nat\ {\isacharsemicolon}{\kern0pt}\ n{\isasymin}nat{\isacharsemicolon}{\kern0pt}\ Z{\isasymin}nat{\isacharsemicolon}{\kern0pt}\ i{\isasymin}nat{\isasymrbrakk}\ {\isasymLongrightarrow}\ arity{\isacharparenleft}{\kern0pt}p{\isacharparenright}{\kern0pt}\ {\isacharequal}{\kern0pt}\ i\ {\isasymLongrightarrow}\ \isanewline
\ \ \ \ arity{\isacharparenleft}{\kern0pt}is{\isacharunderscore}{\kern0pt}nat{\isacharunderscore}{\kern0pt}case{\isacharunderscore}{\kern0pt}fm{\isacharparenleft}{\kern0pt}v{\isacharcomma}{\kern0pt}p{\isacharcomma}{\kern0pt}n{\isacharcomma}{\kern0pt}Z{\isacharparenright}{\kern0pt}{\isacharparenright}{\kern0pt}\ {\isacharequal}{\kern0pt}\ succ{\isacharparenleft}{\kern0pt}v{\isacharparenright}{\kern0pt}\ {\isasymunion}\ succ{\isacharparenleft}{\kern0pt}n{\isacharparenright}{\kern0pt}\ {\isasymunion}\ succ{\isacharparenleft}{\kern0pt}Z{\isacharparenright}{\kern0pt}\ {\isasymunion}\ pred{\isacharparenleft}{\kern0pt}pred{\isacharparenleft}{\kern0pt}i{\isacharparenright}{\kern0pt}{\isacharparenright}{\kern0pt}{\isachardoublequoteclose}\isanewline
%
\isadelimproof
\ \ %
\endisadelimproof
%
\isatagproof
\isacommand{unfolding}\isamarkupfalse%
\ is{\isacharunderscore}{\kern0pt}nat{\isacharunderscore}{\kern0pt}case{\isacharunderscore}{\kern0pt}fm{\isacharunderscore}{\kern0pt}def\isanewline
\ \ \isacommand{using}\isamarkupfalse%
\ arity{\isacharunderscore}{\kern0pt}succ{\isacharunderscore}{\kern0pt}fm\ arity{\isacharunderscore}{\kern0pt}empty{\isacharunderscore}{\kern0pt}fm\ arity{\isacharunderscore}{\kern0pt}quasinat{\isacharunderscore}{\kern0pt}fm\ \isanewline
\ \ \ \ nat{\isacharunderscore}{\kern0pt}union{\isacharunderscore}{\kern0pt}abs{\isadigit{2}}\ pred{\isacharunderscore}{\kern0pt}Un{\isacharunderscore}{\kern0pt}distrib\isanewline
\ \ \isacommand{by}\isamarkupfalse%
\ auto%
\endisatagproof
{\isafoldproof}%
%
\isadelimproof
\isanewline
%
\endisadelimproof
\isanewline
\isacommand{lemma}\isamarkupfalse%
\ arity{\isacharunderscore}{\kern0pt}iterates{\isacharunderscore}{\kern0pt}MH{\isacharunderscore}{\kern0pt}fm\ {\isacharcolon}{\kern0pt}\isanewline
\ \ \isakeyword{assumes}\ {\isachardoublequoteopen}isF{\isasymin}formula{\isachardoublequoteclose}\ {\isachardoublequoteopen}v{\isasymin}nat{\isachardoublequoteclose}\ {\isachardoublequoteopen}n{\isasymin}nat{\isachardoublequoteclose}\ {\isachardoublequoteopen}g{\isasymin}nat{\isachardoublequoteclose}\ {\isachardoublequoteopen}z{\isasymin}nat{\isachardoublequoteclose}\ {\isachardoublequoteopen}i{\isasymin}nat{\isachardoublequoteclose}\ \isanewline
\ \ \ \ \ \ {\isachardoublequoteopen}arity{\isacharparenleft}{\kern0pt}isF{\isacharparenright}{\kern0pt}\ {\isacharequal}{\kern0pt}\ i{\isachardoublequoteclose}\isanewline
\ \ \ \ \isakeyword{shows}\ {\isachardoublequoteopen}arity{\isacharparenleft}{\kern0pt}iterates{\isacharunderscore}{\kern0pt}MH{\isacharunderscore}{\kern0pt}fm{\isacharparenleft}{\kern0pt}isF{\isacharcomma}{\kern0pt}v{\isacharcomma}{\kern0pt}n{\isacharcomma}{\kern0pt}g{\isacharcomma}{\kern0pt}z{\isacharparenright}{\kern0pt}{\isacharparenright}{\kern0pt}\ {\isacharequal}{\kern0pt}\ \isanewline
\ \ \ \ \ \ \ \ \ \ \ succ{\isacharparenleft}{\kern0pt}v{\isacharparenright}{\kern0pt}\ {\isasymunion}\ succ{\isacharparenleft}{\kern0pt}n{\isacharparenright}{\kern0pt}\ {\isasymunion}\ succ{\isacharparenleft}{\kern0pt}g{\isacharparenright}{\kern0pt}\ {\isasymunion}\ succ{\isacharparenleft}{\kern0pt}z{\isacharparenright}{\kern0pt}\ {\isasymunion}\ pred{\isacharparenleft}{\kern0pt}pred{\isacharparenleft}{\kern0pt}pred{\isacharparenleft}{\kern0pt}pred{\isacharparenleft}{\kern0pt}i{\isacharparenright}{\kern0pt}{\isacharparenright}{\kern0pt}{\isacharparenright}{\kern0pt}{\isacharparenright}{\kern0pt}{\isachardoublequoteclose}\isanewline
%
\isadelimproof
%
\endisadelimproof
%
\isatagproof
\isacommand{proof}\isamarkupfalse%
\ {\isacharminus}{\kern0pt}\isanewline
\ \ \isacommand{let}\isamarkupfalse%
\ {\isacharquery}{\kern0pt}{\isasymphi}\ {\isacharequal}{\kern0pt}\ {\isachardoublequoteopen}Exists{\isacharparenleft}{\kern0pt}And{\isacharparenleft}{\kern0pt}fun{\isacharunderscore}{\kern0pt}apply{\isacharunderscore}{\kern0pt}fm{\isacharparenleft}{\kern0pt}succ{\isacharparenleft}{\kern0pt}succ{\isacharparenleft}{\kern0pt}succ{\isacharparenleft}{\kern0pt}g{\isacharparenright}{\kern0pt}{\isacharparenright}{\kern0pt}{\isacharparenright}{\kern0pt}{\isacharcomma}{\kern0pt}\ {\isadigit{2}}{\isacharcomma}{\kern0pt}\ {\isadigit{0}}{\isacharparenright}{\kern0pt}{\isacharcomma}{\kern0pt}\ Forall{\isacharparenleft}{\kern0pt}Implies{\isacharparenleft}{\kern0pt}Equal{\isacharparenleft}{\kern0pt}{\isadigit{0}}{\isacharcomma}{\kern0pt}\ {\isadigit{2}}{\isacharparenright}{\kern0pt}{\isacharcomma}{\kern0pt}\ isF{\isacharparenright}{\kern0pt}{\isacharparenright}{\kern0pt}{\isacharparenright}{\kern0pt}{\isacharparenright}{\kern0pt}{\isachardoublequoteclose}\isanewline
\ \ \isacommand{let}\isamarkupfalse%
\ {\isacharquery}{\kern0pt}ar\ {\isacharequal}{\kern0pt}\ {\isachardoublequoteopen}succ{\isacharparenleft}{\kern0pt}succ{\isacharparenleft}{\kern0pt}succ{\isacharparenleft}{\kern0pt}g{\isacharparenright}{\kern0pt}{\isacharparenright}{\kern0pt}{\isacharparenright}{\kern0pt}\ {\isasymunion}\ pred{\isacharparenleft}{\kern0pt}pred{\isacharparenleft}{\kern0pt}i{\isacharparenright}{\kern0pt}{\isacharparenright}{\kern0pt}{\isachardoublequoteclose}\isanewline
\ \ \isacommand{from}\isamarkupfalse%
\ assms\isanewline
\ \ \isacommand{have}\isamarkupfalse%
\ {\isachardoublequoteopen}arity{\isacharparenleft}{\kern0pt}{\isacharquery}{\kern0pt}{\isasymphi}{\isacharparenright}{\kern0pt}\ {\isacharequal}{\kern0pt}{\isacharquery}{\kern0pt}ar{\isachardoublequoteclose}\ {\isachardoublequoteopen}{\isacharquery}{\kern0pt}{\isasymphi}{\isasymin}formula{\isachardoublequoteclose}\ \isanewline
\ \ \ \ \isacommand{using}\isamarkupfalse%
\ arity{\isacharunderscore}{\kern0pt}fun{\isacharunderscore}{\kern0pt}apply{\isacharunderscore}{\kern0pt}fm\isanewline
\ \ \ \ nat{\isacharunderscore}{\kern0pt}union{\isacharunderscore}{\kern0pt}abs{\isadigit{1}}\ nat{\isacharunderscore}{\kern0pt}union{\isacharunderscore}{\kern0pt}abs{\isadigit{2}}\ pred{\isacharunderscore}{\kern0pt}Un{\isacharunderscore}{\kern0pt}distrib\ succ{\isacharunderscore}{\kern0pt}Un{\isacharunderscore}{\kern0pt}distrib\ Un{\isacharunderscore}{\kern0pt}assoc{\isacharbrackleft}{\kern0pt}symmetric{\isacharbrackright}{\kern0pt}\isanewline
\ \ \ \ \isacommand{by}\isamarkupfalse%
\ simp{\isacharunderscore}{\kern0pt}all\isanewline
\ \ \isacommand{then}\isamarkupfalse%
\isanewline
\ \ \isacommand{show}\isamarkupfalse%
\ {\isacharquery}{\kern0pt}thesis\isanewline
\ \ \ \ \isacommand{unfolding}\isamarkupfalse%
\ iterates{\isacharunderscore}{\kern0pt}MH{\isacharunderscore}{\kern0pt}fm{\isacharunderscore}{\kern0pt}def\isanewline
\ \ \ \ \isacommand{using}\isamarkupfalse%
\ arity{\isacharunderscore}{\kern0pt}is{\isacharunderscore}{\kern0pt}nat{\isacharunderscore}{\kern0pt}case{\isacharunderscore}{\kern0pt}fm{\isacharbrackleft}{\kern0pt}OF\ {\isacartoucheopen}{\isacharquery}{\kern0pt}{\isasymphi}{\isasymin}{\isacharunderscore}{\kern0pt}{\isacartoucheclose}\ {\isacharunderscore}{\kern0pt}\ {\isacharunderscore}{\kern0pt}\ {\isacharunderscore}{\kern0pt}\ {\isacharunderscore}{\kern0pt}\ {\isacartoucheopen}arity{\isacharparenleft}{\kern0pt}{\isacharquery}{\kern0pt}{\isasymphi}{\isacharparenright}{\kern0pt}\ {\isacharequal}{\kern0pt}\ {\isacharunderscore}{\kern0pt}{\isacartoucheclose}{\isacharbrackright}{\kern0pt}\ assms\ pred{\isacharunderscore}{\kern0pt}succ{\isacharunderscore}{\kern0pt}eq\ pred{\isacharunderscore}{\kern0pt}Un{\isacharunderscore}{\kern0pt}distrib\isanewline
\ \ \ \ \isacommand{by}\isamarkupfalse%
\ auto\isanewline
\isacommand{qed}\isamarkupfalse%
%
\endisatagproof
{\isafoldproof}%
%
\isadelimproof
\isanewline
%
\endisadelimproof
\isanewline
\isacommand{lemma}\isamarkupfalse%
\ arity{\isacharunderscore}{\kern0pt}is{\isacharunderscore}{\kern0pt}iterates{\isacharunderscore}{\kern0pt}fm\ {\isacharcolon}{\kern0pt}\isanewline
\ \ \isakeyword{assumes}\ {\isachardoublequoteopen}p{\isasymin}formula{\isachardoublequoteclose}\ {\isachardoublequoteopen}v{\isasymin}nat{\isachardoublequoteclose}\ {\isachardoublequoteopen}n{\isasymin}nat{\isachardoublequoteclose}\ {\isachardoublequoteopen}Z{\isasymin}nat{\isachardoublequoteclose}\ {\isachardoublequoteopen}i{\isasymin}nat{\isachardoublequoteclose}\ \isanewline
\ \ \ \ {\isachardoublequoteopen}arity{\isacharparenleft}{\kern0pt}p{\isacharparenright}{\kern0pt}\ {\isacharequal}{\kern0pt}\ i{\isachardoublequoteclose}\isanewline
\ \ \isakeyword{shows}\ {\isachardoublequoteopen}arity{\isacharparenleft}{\kern0pt}is{\isacharunderscore}{\kern0pt}iterates{\isacharunderscore}{\kern0pt}fm{\isacharparenleft}{\kern0pt}p{\isacharcomma}{\kern0pt}v{\isacharcomma}{\kern0pt}n{\isacharcomma}{\kern0pt}Z{\isacharparenright}{\kern0pt}{\isacharparenright}{\kern0pt}\ {\isacharequal}{\kern0pt}\ succ{\isacharparenleft}{\kern0pt}v{\isacharparenright}{\kern0pt}\ {\isasymunion}\ succ{\isacharparenleft}{\kern0pt}n{\isacharparenright}{\kern0pt}\ {\isasymunion}\ succ{\isacharparenleft}{\kern0pt}Z{\isacharparenright}{\kern0pt}\ {\isasymunion}\ \isanewline
\ \ \ \ \ \ \ \ \ \ pred{\isacharparenleft}{\kern0pt}pred{\isacharparenleft}{\kern0pt}pred{\isacharparenleft}{\kern0pt}pred{\isacharparenleft}{\kern0pt}pred{\isacharparenleft}{\kern0pt}pred{\isacharparenleft}{\kern0pt}pred{\isacharparenleft}{\kern0pt}pred{\isacharparenleft}{\kern0pt}pred{\isacharparenleft}{\kern0pt}pred{\isacharparenleft}{\kern0pt}pred{\isacharparenleft}{\kern0pt}i{\isacharparenright}{\kern0pt}{\isacharparenright}{\kern0pt}{\isacharparenright}{\kern0pt}{\isacharparenright}{\kern0pt}{\isacharparenright}{\kern0pt}{\isacharparenright}{\kern0pt}{\isacharparenright}{\kern0pt}{\isacharparenright}{\kern0pt}{\isacharparenright}{\kern0pt}{\isacharparenright}{\kern0pt}{\isacharparenright}{\kern0pt}{\isachardoublequoteclose}\isanewline
%
\isadelimproof
%
\endisadelimproof
%
\isatagproof
\isacommand{proof}\isamarkupfalse%
\ {\isacharminus}{\kern0pt}\isanewline
\ \ \isacommand{let}\isamarkupfalse%
\ {\isacharquery}{\kern0pt}{\isasymphi}\ {\isacharequal}{\kern0pt}\ {\isachardoublequoteopen}iterates{\isacharunderscore}{\kern0pt}MH{\isacharunderscore}{\kern0pt}fm{\isacharparenleft}{\kern0pt}p{\isacharcomma}{\kern0pt}\ {\isadigit{7}}{\isacharhash}{\kern0pt}{\isacharplus}{\kern0pt}v{\isacharcomma}{\kern0pt}\ {\isadigit{2}}{\isacharcomma}{\kern0pt}\ {\isadigit{1}}{\isacharcomma}{\kern0pt}\ {\isadigit{0}}{\isacharparenright}{\kern0pt}{\isachardoublequoteclose}\isanewline
\ \ \isacommand{let}\isamarkupfalse%
\ {\isacharquery}{\kern0pt}{\isasympsi}\ {\isacharequal}{\kern0pt}\ {\isachardoublequoteopen}is{\isacharunderscore}{\kern0pt}wfrec{\isacharunderscore}{\kern0pt}fm{\isacharparenleft}{\kern0pt}{\isacharquery}{\kern0pt}{\isasymphi}{\isacharcomma}{\kern0pt}\ {\isadigit{0}}{\isacharcomma}{\kern0pt}\ succ{\isacharparenleft}{\kern0pt}succ{\isacharparenleft}{\kern0pt}n{\isacharparenright}{\kern0pt}{\isacharparenright}{\kern0pt}{\isacharcomma}{\kern0pt}succ{\isacharparenleft}{\kern0pt}succ{\isacharparenleft}{\kern0pt}Z{\isacharparenright}{\kern0pt}{\isacharparenright}{\kern0pt}{\isacharparenright}{\kern0pt}{\isachardoublequoteclose}\isanewline
\ \ \isacommand{from}\isamarkupfalse%
\ {\isacartoucheopen}v{\isasymin}{\isacharunderscore}{\kern0pt}{\isacartoucheclose}\isanewline
\ \ \isacommand{have}\isamarkupfalse%
\ {\isachardoublequoteopen}arity{\isacharparenleft}{\kern0pt}{\isacharquery}{\kern0pt}{\isasymphi}{\isacharparenright}{\kern0pt}\ {\isacharequal}{\kern0pt}\ {\isacharparenleft}{\kern0pt}{\isadigit{8}}{\isacharhash}{\kern0pt}{\isacharplus}{\kern0pt}v{\isacharparenright}{\kern0pt}\ {\isasymunion}\ pred{\isacharparenleft}{\kern0pt}pred{\isacharparenleft}{\kern0pt}pred{\isacharparenleft}{\kern0pt}pred{\isacharparenleft}{\kern0pt}i{\isacharparenright}{\kern0pt}{\isacharparenright}{\kern0pt}{\isacharparenright}{\kern0pt}{\isacharparenright}{\kern0pt}{\isachardoublequoteclose}\ {\isachardoublequoteopen}{\isacharquery}{\kern0pt}{\isasymphi}{\isasymin}formula{\isachardoublequoteclose}\isanewline
\ \ \ \ \isacommand{using}\isamarkupfalse%
\ assms\ arity{\isacharunderscore}{\kern0pt}iterates{\isacharunderscore}{\kern0pt}MH{\isacharunderscore}{\kern0pt}fm\ nat{\isacharunderscore}{\kern0pt}union{\isacharunderscore}{\kern0pt}abs{\isadigit{2}}\isanewline
\ \ \ \ \isacommand{by}\isamarkupfalse%
\ simp{\isacharunderscore}{\kern0pt}all\isanewline
\ \ \isacommand{then}\isamarkupfalse%
\isanewline
\ \ \isacommand{have}\isamarkupfalse%
\ {\isachardoublequoteopen}arity{\isacharparenleft}{\kern0pt}{\isacharquery}{\kern0pt}{\isasympsi}{\isacharparenright}{\kern0pt}\ {\isacharequal}{\kern0pt}\ succ{\isacharparenleft}{\kern0pt}succ{\isacharparenleft}{\kern0pt}succ{\isacharparenleft}{\kern0pt}n{\isacharparenright}{\kern0pt}{\isacharparenright}{\kern0pt}{\isacharparenright}{\kern0pt}\ {\isasymunion}\ succ{\isacharparenleft}{\kern0pt}succ{\isacharparenleft}{\kern0pt}succ{\isacharparenleft}{\kern0pt}Z{\isacharparenright}{\kern0pt}{\isacharparenright}{\kern0pt}{\isacharparenright}{\kern0pt}\ {\isasymunion}\ {\isacharparenleft}{\kern0pt}{\isadigit{3}}{\isacharhash}{\kern0pt}{\isacharplus}{\kern0pt}v{\isacharparenright}{\kern0pt}\ {\isasymunion}\ \isanewline
\ \ \ \ \ \ pred{\isacharparenleft}{\kern0pt}pred{\isacharparenleft}{\kern0pt}pred{\isacharparenleft}{\kern0pt}pred{\isacharparenleft}{\kern0pt}pred{\isacharparenleft}{\kern0pt}pred{\isacharparenleft}{\kern0pt}pred{\isacharparenleft}{\kern0pt}pred{\isacharparenleft}{\kern0pt}pred{\isacharparenleft}{\kern0pt}i{\isacharparenright}{\kern0pt}{\isacharparenright}{\kern0pt}{\isacharparenright}{\kern0pt}{\isacharparenright}{\kern0pt}{\isacharparenright}{\kern0pt}{\isacharparenright}{\kern0pt}{\isacharparenright}{\kern0pt}{\isacharparenright}{\kern0pt}{\isacharparenright}{\kern0pt}{\isachardoublequoteclose}\isanewline
\ \ \ \ \isacommand{using}\isamarkupfalse%
\ assms\ arity{\isacharunderscore}{\kern0pt}is{\isacharunderscore}{\kern0pt}wfrec{\isacharunderscore}{\kern0pt}fm{\isacharbrackleft}{\kern0pt}OF\ {\isacartoucheopen}{\isacharquery}{\kern0pt}{\isasymphi}{\isasymin}{\isacharunderscore}{\kern0pt}{\isacartoucheclose}\ {\isacharunderscore}{\kern0pt}\ {\isacharunderscore}{\kern0pt}\ {\isacharunderscore}{\kern0pt}\ {\isacharunderscore}{\kern0pt}\ {\isacartoucheopen}arity{\isacharparenleft}{\kern0pt}{\isacharquery}{\kern0pt}{\isasymphi}{\isacharparenright}{\kern0pt}\ {\isacharequal}{\kern0pt}\ {\isacharunderscore}{\kern0pt}{\isacartoucheclose}{\isacharbrackright}{\kern0pt}\ nat{\isacharunderscore}{\kern0pt}union{\isacharunderscore}{\kern0pt}abs{\isadigit{1}}\ pred{\isacharunderscore}{\kern0pt}Un{\isacharunderscore}{\kern0pt}distrib\isanewline
\ \ \ \ \isacommand{by}\isamarkupfalse%
\ auto\isanewline
\ \ \isacommand{then}\isamarkupfalse%
\isanewline
\ \ \isacommand{show}\isamarkupfalse%
\ {\isacharquery}{\kern0pt}thesis\isanewline
\ \ \ \ \isacommand{unfolding}\isamarkupfalse%
\ is{\isacharunderscore}{\kern0pt}iterates{\isacharunderscore}{\kern0pt}fm{\isacharunderscore}{\kern0pt}def\ \isanewline
\ \ \ \ \isacommand{using}\isamarkupfalse%
\ arity{\isacharunderscore}{\kern0pt}Memrel{\isacharunderscore}{\kern0pt}fm\ arity{\isacharunderscore}{\kern0pt}succ{\isacharunderscore}{\kern0pt}fm\ assms\ nat{\isacharunderscore}{\kern0pt}union{\isacharunderscore}{\kern0pt}abs{\isadigit{1}}\ pred{\isacharunderscore}{\kern0pt}Un{\isacharunderscore}{\kern0pt}distrib\isanewline
\ \ \ \ \isacommand{by}\isamarkupfalse%
\ auto\isanewline
\isacommand{qed}\isamarkupfalse%
%
\endisatagproof
{\isafoldproof}%
%
\isadelimproof
\isanewline
%
\endisadelimproof
\isanewline
\isacommand{lemma}\isamarkupfalse%
\ arity{\isacharunderscore}{\kern0pt}eclose{\isacharunderscore}{\kern0pt}n{\isacharunderscore}{\kern0pt}fm\ {\isacharcolon}{\kern0pt}\isanewline
\ \ \isakeyword{assumes}\ {\isachardoublequoteopen}A{\isasymin}nat{\isachardoublequoteclose}\ {\isachardoublequoteopen}x{\isasymin}nat{\isachardoublequoteclose}\ {\isachardoublequoteopen}t{\isasymin}nat{\isachardoublequoteclose}\ \isanewline
\ \ \isakeyword{shows}\ {\isachardoublequoteopen}arity{\isacharparenleft}{\kern0pt}eclose{\isacharunderscore}{\kern0pt}n{\isacharunderscore}{\kern0pt}fm{\isacharparenleft}{\kern0pt}A{\isacharcomma}{\kern0pt}x{\isacharcomma}{\kern0pt}t{\isacharparenright}{\kern0pt}{\isacharparenright}{\kern0pt}\ {\isacharequal}{\kern0pt}\ succ{\isacharparenleft}{\kern0pt}A{\isacharparenright}{\kern0pt}\ {\isasymunion}\ succ{\isacharparenleft}{\kern0pt}x{\isacharparenright}{\kern0pt}\ {\isasymunion}\ succ{\isacharparenleft}{\kern0pt}t{\isacharparenright}{\kern0pt}{\isachardoublequoteclose}\isanewline
%
\isadelimproof
%
\endisadelimproof
%
\isatagproof
\isacommand{proof}\isamarkupfalse%
\ {\isacharminus}{\kern0pt}\isanewline
\ \ \isacommand{let}\isamarkupfalse%
\ {\isacharquery}{\kern0pt}{\isasymphi}\ {\isacharequal}{\kern0pt}\ {\isachardoublequoteopen}big{\isacharunderscore}{\kern0pt}union{\isacharunderscore}{\kern0pt}fm{\isacharparenleft}{\kern0pt}{\isadigit{1}}{\isacharcomma}{\kern0pt}{\isadigit{0}}{\isacharparenright}{\kern0pt}{\isachardoublequoteclose}\isanewline
\ \ \isacommand{have}\isamarkupfalse%
\ {\isachardoublequoteopen}arity{\isacharparenleft}{\kern0pt}{\isacharquery}{\kern0pt}{\isasymphi}{\isacharparenright}{\kern0pt}\ {\isacharequal}{\kern0pt}\ {\isadigit{2}}{\isachardoublequoteclose}\ {\isachardoublequoteopen}{\isacharquery}{\kern0pt}{\isasymphi}{\isasymin}formula{\isachardoublequoteclose}\ \isanewline
\ \ \ \ \isacommand{using}\isamarkupfalse%
\ arity{\isacharunderscore}{\kern0pt}big{\isacharunderscore}{\kern0pt}union{\isacharunderscore}{\kern0pt}fm\ nat{\isacharunderscore}{\kern0pt}union{\isacharunderscore}{\kern0pt}abs{\isadigit{2}}\isanewline
\ \ \ \ \isacommand{by}\isamarkupfalse%
\ simp{\isacharunderscore}{\kern0pt}all\isanewline
\ \ \isacommand{with}\isamarkupfalse%
\ assms\isanewline
\ \ \isacommand{show}\isamarkupfalse%
\ {\isacharquery}{\kern0pt}thesis\isanewline
\ \ \ \ \isacommand{unfolding}\isamarkupfalse%
\ eclose{\isacharunderscore}{\kern0pt}n{\isacharunderscore}{\kern0pt}fm{\isacharunderscore}{\kern0pt}def\isanewline
\ \ \ \ \isacommand{using}\isamarkupfalse%
\ arity{\isacharunderscore}{\kern0pt}is{\isacharunderscore}{\kern0pt}iterates{\isacharunderscore}{\kern0pt}fm{\isacharbrackleft}{\kern0pt}OF\ {\isacartoucheopen}{\isacharquery}{\kern0pt}{\isasymphi}{\isasymin}{\isacharunderscore}{\kern0pt}{\isacartoucheclose}\ {\isacharunderscore}{\kern0pt}\ {\isacharunderscore}{\kern0pt}\ {\isacharunderscore}{\kern0pt}{\isacharcomma}{\kern0pt}of\ {\isacharunderscore}{\kern0pt}\ {\isacharunderscore}{\kern0pt}\ {\isacharunderscore}{\kern0pt}\ {\isadigit{2}}{\isacharbrackright}{\kern0pt}\ \isanewline
\ \ \ \ \isacommand{by}\isamarkupfalse%
\ auto\isanewline
\isacommand{qed}\isamarkupfalse%
%
\endisatagproof
{\isafoldproof}%
%
\isadelimproof
\isanewline
%
\endisadelimproof
\isanewline
\isacommand{lemma}\isamarkupfalse%
\ arity{\isacharunderscore}{\kern0pt}mem{\isacharunderscore}{\kern0pt}eclose{\isacharunderscore}{\kern0pt}fm\ {\isacharcolon}{\kern0pt}\isanewline
\ \ \isakeyword{assumes}\ {\isachardoublequoteopen}x{\isasymin}nat{\isachardoublequoteclose}\ {\isachardoublequoteopen}t{\isasymin}nat{\isachardoublequoteclose}\isanewline
\ \ \isakeyword{shows}\ {\isachardoublequoteopen}arity{\isacharparenleft}{\kern0pt}mem{\isacharunderscore}{\kern0pt}eclose{\isacharunderscore}{\kern0pt}fm{\isacharparenleft}{\kern0pt}x{\isacharcomma}{\kern0pt}t{\isacharparenright}{\kern0pt}{\isacharparenright}{\kern0pt}\ {\isacharequal}{\kern0pt}\ succ{\isacharparenleft}{\kern0pt}x{\isacharparenright}{\kern0pt}\ {\isasymunion}\ succ{\isacharparenleft}{\kern0pt}t{\isacharparenright}{\kern0pt}{\isachardoublequoteclose}\isanewline
%
\isadelimproof
%
\endisadelimproof
%
\isatagproof
\isacommand{proof}\isamarkupfalse%
\ {\isacharminus}{\kern0pt}\ \ \isanewline
\ \ \isacommand{let}\isamarkupfalse%
\ {\isacharquery}{\kern0pt}{\isasymphi}{\isacharequal}{\kern0pt}{\isachardoublequoteopen}eclose{\isacharunderscore}{\kern0pt}n{\isacharunderscore}{\kern0pt}fm{\isacharparenleft}{\kern0pt}x\ {\isacharhash}{\kern0pt}{\isacharplus}{\kern0pt}\ {\isadigit{2}}{\isacharcomma}{\kern0pt}\ {\isadigit{1}}{\isacharcomma}{\kern0pt}\ {\isadigit{0}}{\isacharparenright}{\kern0pt}{\isachardoublequoteclose}\isanewline
\ \ \isacommand{from}\isamarkupfalse%
\ {\isacartoucheopen}x{\isasymin}nat{\isacartoucheclose}\isanewline
\ \ \isacommand{have}\isamarkupfalse%
\ {\isachardoublequoteopen}arity{\isacharparenleft}{\kern0pt}{\isacharquery}{\kern0pt}{\isasymphi}{\isacharparenright}{\kern0pt}\ {\isacharequal}{\kern0pt}\ x{\isacharhash}{\kern0pt}{\isacharplus}{\kern0pt}{\isadigit{3}}{\isachardoublequoteclose}\ \isanewline
\ \ \ \ \isacommand{using}\isamarkupfalse%
\ arity{\isacharunderscore}{\kern0pt}eclose{\isacharunderscore}{\kern0pt}n{\isacharunderscore}{\kern0pt}fm\ nat{\isacharunderscore}{\kern0pt}union{\isacharunderscore}{\kern0pt}abs{\isadigit{2}}\ \isanewline
\ \ \ \ \isacommand{by}\isamarkupfalse%
\ simp\isanewline
\ \ \isacommand{with}\isamarkupfalse%
\ assms\isanewline
\ \ \isacommand{show}\isamarkupfalse%
\ {\isacharquery}{\kern0pt}thesis\isanewline
\ \ \ \ \isacommand{unfolding}\isamarkupfalse%
\ mem{\isacharunderscore}{\kern0pt}eclose{\isacharunderscore}{\kern0pt}fm{\isacharunderscore}{\kern0pt}def\ \isanewline
\ \ \ \ \isacommand{using}\isamarkupfalse%
\ arity{\isacharunderscore}{\kern0pt}finite{\isacharunderscore}{\kern0pt}ordinal{\isacharunderscore}{\kern0pt}fm\ nat{\isacharunderscore}{\kern0pt}union{\isacharunderscore}{\kern0pt}abs{\isadigit{2}}\ pred{\isacharunderscore}{\kern0pt}Un{\isacharunderscore}{\kern0pt}distrib\isanewline
\ \ \ \ \isacommand{by}\isamarkupfalse%
\ simp\isanewline
\isacommand{qed}\isamarkupfalse%
%
\endisatagproof
{\isafoldproof}%
%
\isadelimproof
\isanewline
%
\endisadelimproof
\isanewline
\isacommand{lemma}\isamarkupfalse%
\ arity{\isacharunderscore}{\kern0pt}is{\isacharunderscore}{\kern0pt}eclose{\isacharunderscore}{\kern0pt}fm\ {\isacharcolon}{\kern0pt}\isanewline
\ \ {\isachardoublequoteopen}{\isasymlbrakk}x{\isasymin}nat\ {\isacharsemicolon}{\kern0pt}\ t{\isasymin}nat{\isasymrbrakk}\ {\isasymLongrightarrow}\ arity{\isacharparenleft}{\kern0pt}is{\isacharunderscore}{\kern0pt}eclose{\isacharunderscore}{\kern0pt}fm{\isacharparenleft}{\kern0pt}x{\isacharcomma}{\kern0pt}t{\isacharparenright}{\kern0pt}{\isacharparenright}{\kern0pt}\ {\isacharequal}{\kern0pt}\ succ{\isacharparenleft}{\kern0pt}x{\isacharparenright}{\kern0pt}\ {\isasymunion}\ succ{\isacharparenleft}{\kern0pt}t{\isacharparenright}{\kern0pt}{\isachardoublequoteclose}\isanewline
%
\isadelimproof
\ \ %
\endisadelimproof
%
\isatagproof
\isacommand{unfolding}\isamarkupfalse%
\ is{\isacharunderscore}{\kern0pt}eclose{\isacharunderscore}{\kern0pt}fm{\isacharunderscore}{\kern0pt}def\ \isanewline
\ \ \isacommand{using}\isamarkupfalse%
\ arity{\isacharunderscore}{\kern0pt}mem{\isacharunderscore}{\kern0pt}eclose{\isacharunderscore}{\kern0pt}fm\ nat{\isacharunderscore}{\kern0pt}union{\isacharunderscore}{\kern0pt}abs{\isadigit{2}}\ pred{\isacharunderscore}{\kern0pt}Un{\isacharunderscore}{\kern0pt}distrib\isanewline
\ \ \isacommand{by}\isamarkupfalse%
\ auto%
\endisatagproof
{\isafoldproof}%
%
\isadelimproof
\isanewline
%
\endisadelimproof
\isanewline
\isacommand{lemma}\isamarkupfalse%
\ eclose{\isacharunderscore}{\kern0pt}n{\isadigit{1}}arity{\isacharunderscore}{\kern0pt}{\isacharunderscore}{\kern0pt}fm\ {\isacharcolon}{\kern0pt}\isanewline
\ \ {\isachardoublequoteopen}{\isasymlbrakk}x{\isasymin}nat\ {\isacharsemicolon}{\kern0pt}\ t{\isasymin}nat{\isasymrbrakk}\ {\isasymLongrightarrow}\ arity{\isacharparenleft}{\kern0pt}eclose{\isacharunderscore}{\kern0pt}n{\isadigit{1}}{\isacharunderscore}{\kern0pt}fm{\isacharparenleft}{\kern0pt}x{\isacharcomma}{\kern0pt}t{\isacharparenright}{\kern0pt}{\isacharparenright}{\kern0pt}\ {\isacharequal}{\kern0pt}\ succ{\isacharparenleft}{\kern0pt}x{\isacharparenright}{\kern0pt}\ {\isasymunion}\ succ{\isacharparenleft}{\kern0pt}t{\isacharparenright}{\kern0pt}{\isachardoublequoteclose}\isanewline
%
\isadelimproof
\ \ %
\endisadelimproof
%
\isatagproof
\isacommand{unfolding}\isamarkupfalse%
\ eclose{\isacharunderscore}{\kern0pt}n{\isadigit{1}}{\isacharunderscore}{\kern0pt}fm{\isacharunderscore}{\kern0pt}def\ \isanewline
\ \ \isacommand{using}\isamarkupfalse%
\ arity{\isacharunderscore}{\kern0pt}is{\isacharunderscore}{\kern0pt}eclose{\isacharunderscore}{\kern0pt}fm\ arity{\isacharunderscore}{\kern0pt}singleton{\isacharunderscore}{\kern0pt}fm\ name{\isadigit{1}}arity{\isacharunderscore}{\kern0pt}{\isacharunderscore}{\kern0pt}fm\ nat{\isacharunderscore}{\kern0pt}union{\isacharunderscore}{\kern0pt}abs{\isadigit{2}}\ pred{\isacharunderscore}{\kern0pt}Un{\isacharunderscore}{\kern0pt}distrib\isanewline
\ \ \isacommand{by}\isamarkupfalse%
\ auto%
\endisatagproof
{\isafoldproof}%
%
\isadelimproof
\isanewline
%
\endisadelimproof
\isanewline
\isacommand{lemma}\isamarkupfalse%
\ eclose{\isacharunderscore}{\kern0pt}n{\isadigit{2}}arity{\isacharunderscore}{\kern0pt}{\isacharunderscore}{\kern0pt}fm\ {\isacharcolon}{\kern0pt}\isanewline
\ \ {\isachardoublequoteopen}{\isasymlbrakk}x{\isasymin}nat\ {\isacharsemicolon}{\kern0pt}\ t{\isasymin}nat{\isasymrbrakk}\ {\isasymLongrightarrow}\ arity{\isacharparenleft}{\kern0pt}eclose{\isacharunderscore}{\kern0pt}n{\isadigit{2}}{\isacharunderscore}{\kern0pt}fm{\isacharparenleft}{\kern0pt}x{\isacharcomma}{\kern0pt}t{\isacharparenright}{\kern0pt}{\isacharparenright}{\kern0pt}\ {\isacharequal}{\kern0pt}\ succ{\isacharparenleft}{\kern0pt}x{\isacharparenright}{\kern0pt}\ {\isasymunion}\ succ{\isacharparenleft}{\kern0pt}t{\isacharparenright}{\kern0pt}{\isachardoublequoteclose}\isanewline
%
\isadelimproof
\ \ %
\endisadelimproof
%
\isatagproof
\isacommand{unfolding}\isamarkupfalse%
\ eclose{\isacharunderscore}{\kern0pt}n{\isadigit{2}}{\isacharunderscore}{\kern0pt}fm{\isacharunderscore}{\kern0pt}def\ \isanewline
\ \ \isacommand{using}\isamarkupfalse%
\ arity{\isacharunderscore}{\kern0pt}is{\isacharunderscore}{\kern0pt}eclose{\isacharunderscore}{\kern0pt}fm\ arity{\isacharunderscore}{\kern0pt}singleton{\isacharunderscore}{\kern0pt}fm\ name{\isadigit{2}}arity{\isacharunderscore}{\kern0pt}{\isacharunderscore}{\kern0pt}fm\ nat{\isacharunderscore}{\kern0pt}union{\isacharunderscore}{\kern0pt}abs{\isadigit{2}}\ pred{\isacharunderscore}{\kern0pt}Un{\isacharunderscore}{\kern0pt}distrib\isanewline
\ \ \isacommand{by}\isamarkupfalse%
\ auto%
\endisatagproof
{\isafoldproof}%
%
\isadelimproof
\isanewline
%
\endisadelimproof
\isanewline
\isacommand{lemma}\isamarkupfalse%
\ arity{\isacharunderscore}{\kern0pt}ecloseN{\isacharunderscore}{\kern0pt}fm\ {\isacharcolon}{\kern0pt}\isanewline
\ \ {\isachardoublequoteopen}{\isasymlbrakk}x{\isasymin}nat\ {\isacharsemicolon}{\kern0pt}\ t{\isasymin}nat{\isasymrbrakk}\ {\isasymLongrightarrow}\ arity{\isacharparenleft}{\kern0pt}ecloseN{\isacharunderscore}{\kern0pt}fm{\isacharparenleft}{\kern0pt}x{\isacharcomma}{\kern0pt}t{\isacharparenright}{\kern0pt}{\isacharparenright}{\kern0pt}\ {\isacharequal}{\kern0pt}\ succ{\isacharparenleft}{\kern0pt}x{\isacharparenright}{\kern0pt}\ {\isasymunion}\ succ{\isacharparenleft}{\kern0pt}t{\isacharparenright}{\kern0pt}{\isachardoublequoteclose}\isanewline
%
\isadelimproof
\ \ %
\endisadelimproof
%
\isatagproof
\isacommand{unfolding}\isamarkupfalse%
\ ecloseN{\isacharunderscore}{\kern0pt}fm{\isacharunderscore}{\kern0pt}def\ \isanewline
\ \ \isacommand{using}\isamarkupfalse%
\ eclose{\isacharunderscore}{\kern0pt}n{\isadigit{1}}arity{\isacharunderscore}{\kern0pt}{\isacharunderscore}{\kern0pt}fm\ eclose{\isacharunderscore}{\kern0pt}n{\isadigit{2}}arity{\isacharunderscore}{\kern0pt}{\isacharunderscore}{\kern0pt}fm\ arity{\isacharunderscore}{\kern0pt}union{\isacharunderscore}{\kern0pt}fm\ nat{\isacharunderscore}{\kern0pt}union{\isacharunderscore}{\kern0pt}abs{\isadigit{2}}\ pred{\isacharunderscore}{\kern0pt}Un{\isacharunderscore}{\kern0pt}distrib\isanewline
\ \ \isacommand{by}\isamarkupfalse%
\ auto%
\endisatagproof
{\isafoldproof}%
%
\isadelimproof
\isanewline
%
\endisadelimproof
\isanewline
\isacommand{lemma}\isamarkupfalse%
\ arity{\isacharunderscore}{\kern0pt}frecR{\isacharunderscore}{\kern0pt}fm\ {\isacharcolon}{\kern0pt}\isanewline
\ \ {\isachardoublequoteopen}{\isasymlbrakk}a{\isasymin}nat{\isacharsemicolon}{\kern0pt}b{\isasymin}nat{\isasymrbrakk}\ {\isasymLongrightarrow}\ arity{\isacharparenleft}{\kern0pt}frecR{\isacharunderscore}{\kern0pt}fm{\isacharparenleft}{\kern0pt}a{\isacharcomma}{\kern0pt}b{\isacharparenright}{\kern0pt}{\isacharparenright}{\kern0pt}\ {\isacharequal}{\kern0pt}\ succ{\isacharparenleft}{\kern0pt}a{\isacharparenright}{\kern0pt}\ {\isasymunion}\ succ{\isacharparenleft}{\kern0pt}b{\isacharparenright}{\kern0pt}{\isachardoublequoteclose}\isanewline
%
\isadelimproof
\ \ %
\endisadelimproof
%
\isatagproof
\isacommand{unfolding}\isamarkupfalse%
\ frecR{\isacharunderscore}{\kern0pt}fm{\isacharunderscore}{\kern0pt}def\isanewline
\ \ \isacommand{using}\isamarkupfalse%
\ arity{\isacharunderscore}{\kern0pt}ftype{\isacharunderscore}{\kern0pt}fm\ name{\isadigit{1}}arity{\isacharunderscore}{\kern0pt}{\isacharunderscore}{\kern0pt}fm\ name{\isadigit{2}}arity{\isacharunderscore}{\kern0pt}{\isacharunderscore}{\kern0pt}fm\ arity{\isacharunderscore}{\kern0pt}domain{\isacharunderscore}{\kern0pt}fm\ \isanewline
\ \ \ \ \ \ number{\isadigit{1}}arity{\isacharunderscore}{\kern0pt}{\isacharunderscore}{\kern0pt}fm\ arity{\isacharunderscore}{\kern0pt}empty{\isacharunderscore}{\kern0pt}fm\ nat{\isacharunderscore}{\kern0pt}union{\isacharunderscore}{\kern0pt}abs{\isadigit{2}}\ pred{\isacharunderscore}{\kern0pt}Un{\isacharunderscore}{\kern0pt}distrib\isanewline
\ \ \isacommand{by}\isamarkupfalse%
\ auto%
\endisatagproof
{\isafoldproof}%
%
\isadelimproof
\isanewline
%
\endisadelimproof
\isanewline
\isacommand{lemma}\isamarkupfalse%
\ arity{\isacharunderscore}{\kern0pt}Collect{\isacharunderscore}{\kern0pt}fm\ {\isacharcolon}{\kern0pt}\isanewline
\ \ \isakeyword{assumes}\ {\isachardoublequoteopen}x\ {\isasymin}\ nat{\isachardoublequoteclose}\ {\isachardoublequoteopen}y\ {\isasymin}\ nat{\isachardoublequoteclose}\ {\isachardoublequoteopen}p{\isasymin}formula{\isachardoublequoteclose}\ \isanewline
\ \ \isakeyword{shows}\ {\isachardoublequoteopen}arity{\isacharparenleft}{\kern0pt}Collect{\isacharunderscore}{\kern0pt}fm{\isacharparenleft}{\kern0pt}x{\isacharcomma}{\kern0pt}p{\isacharcomma}{\kern0pt}y{\isacharparenright}{\kern0pt}{\isacharparenright}{\kern0pt}\ {\isacharequal}{\kern0pt}\ succ{\isacharparenleft}{\kern0pt}x{\isacharparenright}{\kern0pt}\ {\isasymunion}\ succ{\isacharparenleft}{\kern0pt}y{\isacharparenright}{\kern0pt}\ {\isasymunion}\ pred{\isacharparenleft}{\kern0pt}arity{\isacharparenleft}{\kern0pt}p{\isacharparenright}{\kern0pt}{\isacharparenright}{\kern0pt}{\isachardoublequoteclose}\isanewline
%
\isadelimproof
\ \ %
\endisadelimproof
%
\isatagproof
\isacommand{unfolding}\isamarkupfalse%
\ Collect{\isacharunderscore}{\kern0pt}fm{\isacharunderscore}{\kern0pt}def\isanewline
\ \ \isacommand{using}\isamarkupfalse%
\ assms\ pred{\isacharunderscore}{\kern0pt}Un{\isacharunderscore}{\kern0pt}distrib\isanewline
\ \ \isacommand{by}\isamarkupfalse%
\ auto%
\endisatagproof
{\isafoldproof}%
%
\isadelimproof
\isanewline
%
\endisadelimproof
%
\isadelimtheory
\isanewline
%
\endisadelimtheory
%
\isatagtheory
\isacommand{end}\isamarkupfalse%
%
\endisatagtheory
{\isafoldtheory}%
%
\isadelimtheory
%
\endisadelimtheory
%
\end{isabellebody}%
\endinput
%:%file=~/source/repos/ZF-notAC/code/Forcing/Arities.thy%:%
%:%11=1%:%
%:%27=2%:%
%:%28=2%:%
%:%29=3%:%
%:%30=4%:%
%:%35=4%:%
%:%38=5%:%
%:%39=6%:%
%:%40=6%:%
%:%41=7%:%
%:%44=8%:%
%:%48=8%:%
%:%49=8%:%
%:%50=9%:%
%:%51=9%:%
%:%52=10%:%
%:%53=10%:%
%:%58=10%:%
%:%61=11%:%
%:%62=12%:%
%:%63=13%:%
%:%64=13%:%
%:%65=14%:%
%:%68=15%:%
%:%72=15%:%
%:%73=15%:%
%:%74=16%:%
%:%75=16%:%
%:%76=17%:%
%:%77=17%:%
%:%82=17%:%
%:%85=18%:%
%:%86=19%:%
%:%87=19%:%
%:%88=20%:%
%:%91=21%:%
%:%95=21%:%
%:%96=21%:%
%:%97=22%:%
%:%98=22%:%
%:%99=23%:%
%:%100=23%:%
%:%105=23%:%
%:%108=24%:%
%:%109=25%:%
%:%110=25%:%
%:%111=26%:%
%:%114=27%:%
%:%118=27%:%
%:%119=27%:%
%:%120=28%:%
%:%121=28%:%
%:%122=29%:%
%:%123=29%:%
%:%128=29%:%
%:%131=30%:%
%:%132=31%:%
%:%133=31%:%
%:%134=32%:%
%:%137=33%:%
%:%141=33%:%
%:%142=33%:%
%:%143=34%:%
%:%144=34%:%
%:%145=35%:%
%:%146=35%:%
%:%151=35%:%
%:%154=36%:%
%:%155=37%:%
%:%156=37%:%
%:%157=38%:%
%:%160=39%:%
%:%164=39%:%
%:%165=39%:%
%:%166=40%:%
%:%167=40%:%
%:%168=41%:%
%:%169=41%:%
%:%174=41%:%
%:%177=42%:%
%:%178=43%:%
%:%179=43%:%
%:%180=44%:%
%:%183=45%:%
%:%187=45%:%
%:%188=45%:%
%:%189=46%:%
%:%190=46%:%
%:%191=47%:%
%:%192=47%:%
%:%197=47%:%
%:%200=48%:%
%:%201=49%:%
%:%202=49%:%
%:%203=50%:%
%:%206=51%:%
%:%210=51%:%
%:%211=51%:%
%:%212=52%:%
%:%213=52%:%
%:%214=53%:%
%:%215=53%:%
%:%220=53%:%
%:%223=54%:%
%:%224=55%:%
%:%225=56%:%
%:%226=56%:%
%:%227=57%:%
%:%230=58%:%
%:%234=58%:%
%:%235=58%:%
%:%236=59%:%
%:%237=59%:%
%:%238=60%:%
%:%239=60%:%
%:%244=60%:%
%:%247=61%:%
%:%248=62%:%
%:%249=62%:%
%:%250=63%:%
%:%251=64%:%
%:%254=65%:%
%:%258=65%:%
%:%259=65%:%
%:%260=66%:%
%:%261=66%:%
%:%262=67%:%
%:%263=67%:%
%:%268=67%:%
%:%271=68%:%
%:%272=69%:%
%:%273=69%:%
%:%274=70%:%
%:%277=71%:%
%:%281=71%:%
%:%282=71%:%
%:%283=72%:%
%:%284=72%:%
%:%285=73%:%
%:%286=74%:%
%:%287=74%:%
%:%292=74%:%
%:%295=75%:%
%:%296=76%:%
%:%297=76%:%
%:%298=77%:%
%:%301=78%:%
%:%305=78%:%
%:%306=78%:%
%:%307=79%:%
%:%308=79%:%
%:%309=80%:%
%:%310=80%:%
%:%315=80%:%
%:%318=81%:%
%:%319=82%:%
%:%320=82%:%
%:%321=83%:%
%:%324=84%:%
%:%328=84%:%
%:%329=84%:%
%:%330=85%:%
%:%331=85%:%
%:%332=86%:%
%:%333=86%:%
%:%338=86%:%
%:%341=87%:%
%:%342=88%:%
%:%343=89%:%
%:%344=89%:%
%:%345=90%:%
%:%348=91%:%
%:%352=91%:%
%:%353=91%:%
%:%354=92%:%
%:%355=92%:%
%:%356=93%:%
%:%357=93%:%
%:%362=93%:%
%:%365=94%:%
%:%366=95%:%
%:%367=96%:%
%:%368=96%:%
%:%369=97%:%
%:%372=98%:%
%:%376=98%:%
%:%377=98%:%
%:%378=99%:%
%:%379=99%:%
%:%380=100%:%
%:%381=100%:%
%:%386=100%:%
%:%389=101%:%
%:%390=102%:%
%:%391=102%:%
%:%392=103%:%
%:%395=104%:%
%:%399=104%:%
%:%400=104%:%
%:%401=105%:%
%:%402=105%:%
%:%403=106%:%
%:%404=106%:%
%:%409=106%:%
%:%412=107%:%
%:%413=108%:%
%:%414=108%:%
%:%415=109%:%
%:%418=110%:%
%:%422=110%:%
%:%423=110%:%
%:%424=111%:%
%:%425=111%:%
%:%426=112%:%
%:%427=112%:%
%:%432=112%:%
%:%435=113%:%
%:%436=114%:%
%:%437=114%:%
%:%438=115%:%
%:%439=116%:%
%:%442=117%:%
%:%446=117%:%
%:%447=117%:%
%:%448=118%:%
%:%449=118%:%
%:%450=119%:%
%:%451=120%:%
%:%452=120%:%
%:%457=120%:%
%:%460=121%:%
%:%461=122%:%
%:%462=123%:%
%:%463=123%:%
%:%464=124%:%
%:%467=125%:%
%:%471=125%:%
%:%472=125%:%
%:%473=126%:%
%:%474=126%:%
%:%475=127%:%
%:%476=127%:%
%:%481=127%:%
%:%484=128%:%
%:%485=129%:%
%:%486=129%:%
%:%487=130%:%
%:%490=131%:%
%:%494=131%:%
%:%495=131%:%
%:%496=132%:%
%:%497=132%:%
%:%498=133%:%
%:%499=133%:%
%:%504=133%:%
%:%507=134%:%
%:%508=135%:%
%:%509=135%:%
%:%510=136%:%
%:%513=137%:%
%:%517=137%:%
%:%518=137%:%
%:%519=138%:%
%:%520=138%:%
%:%521=139%:%
%:%522=139%:%
%:%527=139%:%
%:%530=140%:%
%:%531=141%:%
%:%532=141%:%
%:%533=142%:%
%:%536=143%:%
%:%540=143%:%
%:%541=143%:%
%:%542=144%:%
%:%543=144%:%
%:%544=145%:%
%:%545=145%:%
%:%550=145%:%
%:%553=146%:%
%:%554=147%:%
%:%555=147%:%
%:%556=148%:%
%:%559=149%:%
%:%563=149%:%
%:%564=149%:%
%:%565=150%:%
%:%566=150%:%
%:%567=151%:%
%:%568=152%:%
%:%569=152%:%
%:%574=152%:%
%:%577=153%:%
%:%578=154%:%
%:%579=154%:%
%:%580=155%:%
%:%583=156%:%
%:%587=156%:%
%:%588=156%:%
%:%589=157%:%
%:%590=157%:%
%:%591=158%:%
%:%592=158%:%
%:%597=158%:%
%:%600=159%:%
%:%601=160%:%
%:%602=160%:%
%:%603=161%:%
%:%606=162%:%
%:%610=162%:%
%:%611=162%:%
%:%612=163%:%
%:%613=163%:%
%:%614=164%:%
%:%615=164%:%
%:%620=164%:%
%:%623=165%:%
%:%624=166%:%
%:%625=166%:%
%:%626=167%:%
%:%629=168%:%
%:%633=168%:%
%:%634=168%:%
%:%635=169%:%
%:%636=169%:%
%:%637=170%:%
%:%638=170%:%
%:%643=170%:%
%:%646=171%:%
%:%647=172%:%
%:%648=172%:%
%:%649=173%:%
%:%652=174%:%
%:%656=174%:%
%:%657=174%:%
%:%658=175%:%
%:%659=175%:%
%:%660=176%:%
%:%661=176%:%
%:%666=176%:%
%:%669=177%:%
%:%670=178%:%
%:%671=178%:%
%:%672=179%:%
%:%675=180%:%
%:%679=180%:%
%:%680=180%:%
%:%681=181%:%
%:%682=181%:%
%:%683=182%:%
%:%684=182%:%
%:%689=182%:%
%:%692=183%:%
%:%693=184%:%
%:%694=184%:%
%:%695=185%:%
%:%698=186%:%
%:%702=186%:%
%:%703=186%:%
%:%704=187%:%
%:%705=187%:%
%:%706=188%:%
%:%707=188%:%
%:%712=188%:%
%:%715=189%:%
%:%716=190%:%
%:%717=190%:%
%:%718=191%:%
%:%721=192%:%
%:%725=192%:%
%:%726=192%:%
%:%727=193%:%
%:%728=193%:%
%:%729=194%:%
%:%730=194%:%
%:%735=194%:%
%:%738=195%:%
%:%739=196%:%
%:%740=196%:%
%:%741=197%:%
%:%744=198%:%
%:%748=198%:%
%:%749=198%:%
%:%750=199%:%
%:%751=199%:%
%:%752=200%:%
%:%753=200%:%
%:%758=200%:%
%:%761=201%:%
%:%762=202%:%
%:%763=202%:%
%:%764=203%:%
%:%767=204%:%
%:%771=204%:%
%:%772=204%:%
%:%773=205%:%
%:%774=205%:%
%:%775=206%:%
%:%776=206%:%
%:%781=206%:%
%:%784=207%:%
%:%785=208%:%
%:%786=208%:%
%:%787=209%:%
%:%790=210%:%
%:%794=210%:%
%:%795=210%:%
%:%796=211%:%
%:%797=211%:%
%:%798=212%:%
%:%799=212%:%
%:%804=212%:%
%:%807=213%:%
%:%808=214%:%
%:%809=214%:%
%:%810=215%:%
%:%813=216%:%
%:%817=216%:%
%:%818=216%:%
%:%819=217%:%
%:%820=217%:%
%:%821=218%:%
%:%822=218%:%
%:%827=218%:%
%:%830=219%:%
%:%831=220%:%
%:%832=220%:%
%:%833=221%:%
%:%836=222%:%
%:%840=222%:%
%:%841=222%:%
%:%842=223%:%
%:%843=223%:%
%:%844=224%:%
%:%845=225%:%
%:%846=225%:%
%:%851=225%:%
%:%854=226%:%
%:%855=227%:%
%:%856=227%:%
%:%857=228%:%
%:%858=229%:%
%:%861=230%:%
%:%865=230%:%
%:%866=230%:%
%:%867=231%:%
%:%868=231%:%
%:%869=232%:%
%:%870=233%:%
%:%871=233%:%
%:%876=233%:%
%:%879=234%:%
%:%880=235%:%
%:%881=235%:%
%:%882=236%:%
%:%883=237%:%
%:%886=238%:%
%:%890=238%:%
%:%891=238%:%
%:%892=239%:%
%:%893=239%:%
%:%894=240%:%
%:%895=241%:%
%:%896=241%:%
%:%901=241%:%
%:%904=242%:%
%:%905=243%:%
%:%906=243%:%
%:%907=244%:%
%:%908=245%:%
%:%911=246%:%
%:%915=246%:%
%:%916=246%:%
%:%917=247%:%
%:%918=247%:%
%:%919=248%:%
%:%920=249%:%
%:%921=249%:%
%:%926=249%:%
%:%929=250%:%
%:%930=251%:%
%:%931=251%:%
%:%932=252%:%
%:%933=253%:%
%:%934=254%:%
%:%935=255%:%
%:%942=256%:%
%:%943=256%:%
%:%944=257%:%
%:%945=257%:%
%:%946=258%:%
%:%947=258%:%
%:%948=259%:%
%:%949=259%:%
%:%950=260%:%
%:%951=260%:%
%:%952=261%:%
%:%953=261%:%
%:%954=262%:%
%:%955=263%:%
%:%956=263%:%
%:%957=264%:%
%:%958=264%:%
%:%959=265%:%
%:%960=265%:%
%:%961=266%:%
%:%962=266%:%
%:%963=267%:%
%:%964=267%:%
%:%965=268%:%
%:%966=268%:%
%:%967=269%:%
%:%973=269%:%
%:%976=270%:%
%:%977=271%:%
%:%978=271%:%
%:%979=272%:%
%:%980=273%:%
%:%981=274%:%
%:%982=275%:%
%:%989=276%:%
%:%990=276%:%
%:%991=277%:%
%:%992=277%:%
%:%993=278%:%
%:%994=278%:%
%:%995=279%:%
%:%996=279%:%
%:%997=280%:%
%:%998=280%:%
%:%999=281%:%
%:%1000=281%:%
%:%1001=282%:%
%:%1002=282%:%
%:%1003=283%:%
%:%1004=283%:%
%:%1005=284%:%
%:%1006=284%:%
%:%1007=285%:%
%:%1008=286%:%
%:%1009=286%:%
%:%1010=287%:%
%:%1011=287%:%
%:%1012=288%:%
%:%1013=288%:%
%:%1014=289%:%
%:%1015=289%:%
%:%1016=290%:%
%:%1017=290%:%
%:%1018=291%:%
%:%1019=291%:%
%:%1020=292%:%
%:%1021=292%:%
%:%1022=293%:%
%:%1028=293%:%
%:%1031=294%:%
%:%1032=295%:%
%:%1033=295%:%
%:%1034=296%:%
%:%1035=297%:%
%:%1042=298%:%
%:%1043=298%:%
%:%1044=299%:%
%:%1045=299%:%
%:%1046=300%:%
%:%1047=300%:%
%:%1048=301%:%
%:%1049=301%:%
%:%1050=302%:%
%:%1051=302%:%
%:%1052=303%:%
%:%1053=303%:%
%:%1054=304%:%
%:%1055=304%:%
%:%1056=305%:%
%:%1057=305%:%
%:%1058=306%:%
%:%1059=306%:%
%:%1060=307%:%
%:%1061=307%:%
%:%1062=308%:%
%:%1068=308%:%
%:%1071=309%:%
%:%1072=310%:%
%:%1073=310%:%
%:%1074=311%:%
%:%1075=312%:%
%:%1082=313%:%
%:%1083=313%:%
%:%1084=314%:%
%:%1085=314%:%
%:%1086=315%:%
%:%1087=315%:%
%:%1088=316%:%
%:%1089=316%:%
%:%1090=317%:%
%:%1091=317%:%
%:%1092=318%:%
%:%1093=318%:%
%:%1094=319%:%
%:%1095=319%:%
%:%1096=320%:%
%:%1097=320%:%
%:%1098=321%:%
%:%1099=321%:%
%:%1100=322%:%
%:%1101=322%:%
%:%1102=323%:%
%:%1103=323%:%
%:%1104=324%:%
%:%1110=324%:%
%:%1113=325%:%
%:%1114=326%:%
%:%1115=326%:%
%:%1116=327%:%
%:%1119=328%:%
%:%1123=328%:%
%:%1124=328%:%
%:%1125=329%:%
%:%1126=329%:%
%:%1127=330%:%
%:%1128=330%:%
%:%1133=330%:%
%:%1136=331%:%
%:%1137=332%:%
%:%1138=332%:%
%:%1139=333%:%
%:%1142=334%:%
%:%1146=334%:%
%:%1147=334%:%
%:%1148=335%:%
%:%1149=335%:%
%:%1150=336%:%
%:%1151=336%:%
%:%1156=336%:%
%:%1159=337%:%
%:%1160=338%:%
%:%1161=338%:%
%:%1162=339%:%
%:%1165=340%:%
%:%1169=340%:%
%:%1170=340%:%
%:%1171=341%:%
%:%1172=341%:%
%:%1173=342%:%
%:%1174=342%:%
%:%1179=342%:%
%:%1182=343%:%
%:%1183=344%:%
%:%1184=344%:%
%:%1185=345%:%
%:%1188=346%:%
%:%1192=346%:%
%:%1193=346%:%
%:%1194=347%:%
%:%1195=347%:%
%:%1196=348%:%
%:%1197=348%:%
%:%1202=348%:%
%:%1205=349%:%
%:%1206=350%:%
%:%1207=350%:%
%:%1208=351%:%
%:%1211=352%:%
%:%1215=352%:%
%:%1216=352%:%
%:%1217=353%:%
%:%1218=353%:%
%:%1219=354%:%
%:%1220=355%:%
%:%1221=355%:%
%:%1226=355%:%
%:%1229=356%:%
%:%1230=357%:%
%:%1231=357%:%
%:%1232=358%:%
%:%1233=359%:%
%:%1236=360%:%
%:%1240=360%:%
%:%1241=360%:%
%:%1242=361%:%
%:%1243=361%:%
%:%1244=362%:%
%:%1245=362%:%
%:%1250=362%:%
%:%1255=363%:%
%:%1260=364%:%

%
\begin{isabellebody}%
\setisabellecontext{Forces{\isacharunderscore}{\kern0pt}Definition}%
%
\isadelimdocument
%
\endisadelimdocument
%
\isatagdocument
%
\isamarkupsection{The definition of \isa{forces}%
}
\isamarkuptrue%
%
\endisatagdocument
{\isafolddocument}%
%
\isadelimdocument
%
\endisadelimdocument
%
\isadelimtheory
%
\endisadelimtheory
%
\isatagtheory
\isacommand{theory}\isamarkupfalse%
\ Forces{\isacharunderscore}{\kern0pt}Definition\ \isakeyword{imports}\ Arities\ FrecR\ Synthetic{\isacharunderscore}{\kern0pt}Definition\ \isakeyword{begin}%
\endisatagtheory
{\isafoldtheory}%
%
\isadelimtheory
%
\endisadelimtheory
%
\begin{isamarkuptext}%
This is the core of our development.%
\end{isamarkuptext}\isamarkuptrue%
%
\isadelimdocument
%
\endisadelimdocument
%
\isatagdocument
%
\isamarkupsubsection{The relation \isa{frecrel}%
}
\isamarkuptrue%
%
\endisatagdocument
{\isafolddocument}%
%
\isadelimdocument
%
\endisadelimdocument
\isacommand{definition}\isamarkupfalse%
\isanewline
\ \ frecrelP\ {\isacharcolon}{\kern0pt}{\isacharcolon}{\kern0pt}\ {\isachardoublequoteopen}{\isacharbrackleft}{\kern0pt}i{\isasymRightarrow}o{\isacharcomma}{\kern0pt}i{\isacharbrackright}{\kern0pt}\ {\isasymRightarrow}\ o{\isachardoublequoteclose}\ \isakeyword{where}\isanewline
\ \ {\isachardoublequoteopen}frecrelP{\isacharparenleft}{\kern0pt}M{\isacharcomma}{\kern0pt}xy{\isacharparenright}{\kern0pt}\ {\isasymequiv}\ {\isacharparenleft}{\kern0pt}{\isasymexists}x{\isacharbrackleft}{\kern0pt}M{\isacharbrackright}{\kern0pt}{\isachardot}{\kern0pt}\ {\isasymexists}y{\isacharbrackleft}{\kern0pt}M{\isacharbrackright}{\kern0pt}{\isachardot}{\kern0pt}\ pair{\isacharparenleft}{\kern0pt}M{\isacharcomma}{\kern0pt}x{\isacharcomma}{\kern0pt}y{\isacharcomma}{\kern0pt}xy{\isacharparenright}{\kern0pt}\ {\isasymand}\ is{\isacharunderscore}{\kern0pt}frecR{\isacharparenleft}{\kern0pt}M{\isacharcomma}{\kern0pt}x{\isacharcomma}{\kern0pt}y{\isacharparenright}{\kern0pt}{\isacharparenright}{\kern0pt}{\isachardoublequoteclose}\isanewline
\isanewline
\isacommand{definition}\isamarkupfalse%
\isanewline
\ \ frecrelP{\isacharunderscore}{\kern0pt}fm\ {\isacharcolon}{\kern0pt}{\isacharcolon}{\kern0pt}\ {\isachardoublequoteopen}i\ {\isasymRightarrow}\ i{\isachardoublequoteclose}\ \isakeyword{where}\isanewline
\ \ {\isachardoublequoteopen}frecrelP{\isacharunderscore}{\kern0pt}fm{\isacharparenleft}{\kern0pt}a{\isacharparenright}{\kern0pt}\ {\isasymequiv}\ Exists{\isacharparenleft}{\kern0pt}Exists{\isacharparenleft}{\kern0pt}And{\isacharparenleft}{\kern0pt}pair{\isacharunderscore}{\kern0pt}fm{\isacharparenleft}{\kern0pt}{\isadigit{1}}{\isacharcomma}{\kern0pt}{\isadigit{0}}{\isacharcomma}{\kern0pt}a{\isacharhash}{\kern0pt}{\isacharplus}{\kern0pt}{\isadigit{2}}{\isacharparenright}{\kern0pt}{\isacharcomma}{\kern0pt}frecR{\isacharunderscore}{\kern0pt}fm{\isacharparenleft}{\kern0pt}{\isadigit{1}}{\isacharcomma}{\kern0pt}{\isadigit{0}}{\isacharparenright}{\kern0pt}{\isacharparenright}{\kern0pt}{\isacharparenright}{\kern0pt}{\isacharparenright}{\kern0pt}{\isachardoublequoteclose}\isanewline
\isanewline
\isacommand{lemma}\isamarkupfalse%
\ arity{\isacharunderscore}{\kern0pt}frecrelP{\isacharunderscore}{\kern0pt}fm\ {\isacharcolon}{\kern0pt}\isanewline
\ \ {\isachardoublequoteopen}a{\isasymin}nat\ {\isasymLongrightarrow}\ arity{\isacharparenleft}{\kern0pt}frecrelP{\isacharunderscore}{\kern0pt}fm{\isacharparenleft}{\kern0pt}a{\isacharparenright}{\kern0pt}{\isacharparenright}{\kern0pt}\ {\isacharequal}{\kern0pt}\ succ{\isacharparenleft}{\kern0pt}a{\isacharparenright}{\kern0pt}{\isachardoublequoteclose}\isanewline
%
\isadelimproof
\ \ %
\endisadelimproof
%
\isatagproof
\isacommand{unfolding}\isamarkupfalse%
\ frecrelP{\isacharunderscore}{\kern0pt}fm{\isacharunderscore}{\kern0pt}def\isanewline
\ \ \isacommand{using}\isamarkupfalse%
\ arity{\isacharunderscore}{\kern0pt}frecR{\isacharunderscore}{\kern0pt}fm\ arity{\isacharunderscore}{\kern0pt}pair{\isacharunderscore}{\kern0pt}fm\ pred{\isacharunderscore}{\kern0pt}Un{\isacharunderscore}{\kern0pt}distrib\isanewline
\ \ \isacommand{by}\isamarkupfalse%
\ simp%
\endisatagproof
{\isafoldproof}%
%
\isadelimproof
\isanewline
%
\endisadelimproof
\isanewline
\isacommand{lemma}\isamarkupfalse%
\ frecrelP{\isacharunderscore}{\kern0pt}fm{\isacharunderscore}{\kern0pt}type{\isacharbrackleft}{\kern0pt}TC{\isacharbrackright}{\kern0pt}\ {\isacharcolon}{\kern0pt}\isanewline
\ \ {\isachardoublequoteopen}a{\isasymin}nat\ {\isasymLongrightarrow}\ frecrelP{\isacharunderscore}{\kern0pt}fm{\isacharparenleft}{\kern0pt}a{\isacharparenright}{\kern0pt}{\isasymin}formula{\isachardoublequoteclose}\isanewline
%
\isadelimproof
\ \ %
\endisadelimproof
%
\isatagproof
\isacommand{unfolding}\isamarkupfalse%
\ frecrelP{\isacharunderscore}{\kern0pt}fm{\isacharunderscore}{\kern0pt}def\ \isacommand{by}\isamarkupfalse%
\ simp%
\endisatagproof
{\isafoldproof}%
%
\isadelimproof
\isanewline
%
\endisadelimproof
\isanewline
\isacommand{lemma}\isamarkupfalse%
\ sats{\isacharunderscore}{\kern0pt}frecrelP{\isacharunderscore}{\kern0pt}fm\ {\isacharcolon}{\kern0pt}\isanewline
\ \ \isakeyword{assumes}\ {\isachardoublequoteopen}a{\isasymin}nat{\isachardoublequoteclose}\ {\isachardoublequoteopen}env{\isasymin}list{\isacharparenleft}{\kern0pt}A{\isacharparenright}{\kern0pt}{\isachardoublequoteclose}\isanewline
\ \ \isakeyword{shows}\ {\isachardoublequoteopen}sats{\isacharparenleft}{\kern0pt}A{\isacharcomma}{\kern0pt}frecrelP{\isacharunderscore}{\kern0pt}fm{\isacharparenleft}{\kern0pt}a{\isacharparenright}{\kern0pt}{\isacharcomma}{\kern0pt}env{\isacharparenright}{\kern0pt}\ {\isasymlongleftrightarrow}\ frecrelP{\isacharparenleft}{\kern0pt}{\isacharhash}{\kern0pt}{\isacharhash}{\kern0pt}A{\isacharcomma}{\kern0pt}nth{\isacharparenleft}{\kern0pt}a{\isacharcomma}{\kern0pt}\ env{\isacharparenright}{\kern0pt}{\isacharparenright}{\kern0pt}{\isachardoublequoteclose}\isanewline
%
\isadelimproof
\ \ %
\endisadelimproof
%
\isatagproof
\isacommand{unfolding}\isamarkupfalse%
\ frecrelP{\isacharunderscore}{\kern0pt}def\ frecrelP{\isacharunderscore}{\kern0pt}fm{\isacharunderscore}{\kern0pt}def\isanewline
\ \ \isacommand{using}\isamarkupfalse%
\ assms\ \isacommand{by}\isamarkupfalse%
\ {\isacharparenleft}{\kern0pt}auto\ simp\ add{\isacharcolon}{\kern0pt}frecR{\isacharunderscore}{\kern0pt}fm{\isacharunderscore}{\kern0pt}iff{\isacharunderscore}{\kern0pt}sats{\isacharbrackleft}{\kern0pt}symmetric{\isacharbrackright}{\kern0pt}{\isacharparenright}{\kern0pt}%
\endisatagproof
{\isafoldproof}%
%
\isadelimproof
\isanewline
%
\endisadelimproof
\isanewline
\isacommand{lemma}\isamarkupfalse%
\ frecrelP{\isacharunderscore}{\kern0pt}iff{\isacharunderscore}{\kern0pt}sats{\isacharcolon}{\kern0pt}\isanewline
\ \ \isakeyword{assumes}\isanewline
\ \ \ \ {\isachardoublequoteopen}nth{\isacharparenleft}{\kern0pt}a{\isacharcomma}{\kern0pt}env{\isacharparenright}{\kern0pt}\ {\isacharequal}{\kern0pt}\ aa{\isachardoublequoteclose}\ {\isachardoublequoteopen}a{\isasymin}\ nat{\isachardoublequoteclose}\ \ {\isachardoublequoteopen}env\ {\isasymin}\ list{\isacharparenleft}{\kern0pt}A{\isacharparenright}{\kern0pt}{\isachardoublequoteclose}\isanewline
\ \ \isakeyword{shows}\isanewline
\ \ \ \ {\isachardoublequoteopen}frecrelP{\isacharparenleft}{\kern0pt}{\isacharhash}{\kern0pt}{\isacharhash}{\kern0pt}A{\isacharcomma}{\kern0pt}aa{\isacharparenright}{\kern0pt}\ \ {\isasymlongleftrightarrow}\ sats{\isacharparenleft}{\kern0pt}A{\isacharcomma}{\kern0pt}\ frecrelP{\isacharunderscore}{\kern0pt}fm{\isacharparenleft}{\kern0pt}a{\isacharparenright}{\kern0pt}{\isacharcomma}{\kern0pt}\ env{\isacharparenright}{\kern0pt}{\isachardoublequoteclose}\isanewline
%
\isadelimproof
\ \ %
\endisadelimproof
%
\isatagproof
\isacommand{using}\isamarkupfalse%
\ assms\isanewline
\ \ \isacommand{by}\isamarkupfalse%
\ {\isacharparenleft}{\kern0pt}simp\ add{\isacharcolon}{\kern0pt}sats{\isacharunderscore}{\kern0pt}frecrelP{\isacharunderscore}{\kern0pt}fm{\isacharparenright}{\kern0pt}%
\endisatagproof
{\isafoldproof}%
%
\isadelimproof
\isanewline
%
\endisadelimproof
\isanewline
\isacommand{definition}\isamarkupfalse%
\isanewline
\ \ is{\isacharunderscore}{\kern0pt}frecrel\ {\isacharcolon}{\kern0pt}{\isacharcolon}{\kern0pt}\ {\isachardoublequoteopen}{\isacharbrackleft}{\kern0pt}i{\isasymRightarrow}o{\isacharcomma}{\kern0pt}i{\isacharcomma}{\kern0pt}i{\isacharbrackright}{\kern0pt}\ {\isasymRightarrow}\ o{\isachardoublequoteclose}\ \isakeyword{where}\isanewline
\ \ {\isachardoublequoteopen}is{\isacharunderscore}{\kern0pt}frecrel{\isacharparenleft}{\kern0pt}M{\isacharcomma}{\kern0pt}A{\isacharcomma}{\kern0pt}r{\isacharparenright}{\kern0pt}\ {\isasymequiv}\ {\isasymexists}A{\isadigit{2}}{\isacharbrackleft}{\kern0pt}M{\isacharbrackright}{\kern0pt}{\isachardot}{\kern0pt}\ cartprod{\isacharparenleft}{\kern0pt}M{\isacharcomma}{\kern0pt}A{\isacharcomma}{\kern0pt}A{\isacharcomma}{\kern0pt}A{\isadigit{2}}{\isacharparenright}{\kern0pt}\ {\isasymand}\ is{\isacharunderscore}{\kern0pt}Collect{\isacharparenleft}{\kern0pt}M{\isacharcomma}{\kern0pt}A{\isadigit{2}}{\isacharcomma}{\kern0pt}\ frecrelP{\isacharparenleft}{\kern0pt}M{\isacharparenright}{\kern0pt}\ {\isacharcomma}{\kern0pt}r{\isacharparenright}{\kern0pt}{\isachardoublequoteclose}\isanewline
\isanewline
\isacommand{definition}\isamarkupfalse%
\isanewline
\ \ frecrel{\isacharunderscore}{\kern0pt}fm\ {\isacharcolon}{\kern0pt}{\isacharcolon}{\kern0pt}\ {\isachardoublequoteopen}{\isacharbrackleft}{\kern0pt}i{\isacharcomma}{\kern0pt}i{\isacharbrackright}{\kern0pt}\ {\isasymRightarrow}\ i{\isachardoublequoteclose}\ \isakeyword{where}\isanewline
\ \ {\isachardoublequoteopen}frecrel{\isacharunderscore}{\kern0pt}fm{\isacharparenleft}{\kern0pt}a{\isacharcomma}{\kern0pt}r{\isacharparenright}{\kern0pt}\ {\isasymequiv}\ Exists{\isacharparenleft}{\kern0pt}And{\isacharparenleft}{\kern0pt}cartprod{\isacharunderscore}{\kern0pt}fm{\isacharparenleft}{\kern0pt}a{\isacharhash}{\kern0pt}{\isacharplus}{\kern0pt}{\isadigit{1}}{\isacharcomma}{\kern0pt}a{\isacharhash}{\kern0pt}{\isacharplus}{\kern0pt}{\isadigit{1}}{\isacharcomma}{\kern0pt}{\isadigit{0}}{\isacharparenright}{\kern0pt}{\isacharcomma}{\kern0pt}Collect{\isacharunderscore}{\kern0pt}fm{\isacharparenleft}{\kern0pt}{\isadigit{0}}{\isacharcomma}{\kern0pt}frecrelP{\isacharunderscore}{\kern0pt}fm{\isacharparenleft}{\kern0pt}{\isadigit{0}}{\isacharparenright}{\kern0pt}{\isacharcomma}{\kern0pt}r{\isacharhash}{\kern0pt}{\isacharplus}{\kern0pt}{\isadigit{1}}{\isacharparenright}{\kern0pt}{\isacharparenright}{\kern0pt}{\isacharparenright}{\kern0pt}{\isachardoublequoteclose}\isanewline
\isanewline
\isacommand{lemma}\isamarkupfalse%
\ frecrel{\isacharunderscore}{\kern0pt}fm{\isacharunderscore}{\kern0pt}type{\isacharbrackleft}{\kern0pt}TC{\isacharbrackright}{\kern0pt}\ {\isacharcolon}{\kern0pt}\isanewline
\ \ {\isachardoublequoteopen}{\isasymlbrakk}a{\isasymin}nat{\isacharsemicolon}{\kern0pt}b{\isasymin}nat{\isasymrbrakk}\ {\isasymLongrightarrow}\ frecrel{\isacharunderscore}{\kern0pt}fm{\isacharparenleft}{\kern0pt}a{\isacharcomma}{\kern0pt}b{\isacharparenright}{\kern0pt}{\isasymin}formula{\isachardoublequoteclose}\isanewline
%
\isadelimproof
\ \ %
\endisadelimproof
%
\isatagproof
\isacommand{unfolding}\isamarkupfalse%
\ frecrel{\isacharunderscore}{\kern0pt}fm{\isacharunderscore}{\kern0pt}def\ \isacommand{by}\isamarkupfalse%
\ simp%
\endisatagproof
{\isafoldproof}%
%
\isadelimproof
\isanewline
%
\endisadelimproof
\isanewline
\isacommand{lemma}\isamarkupfalse%
\ arity{\isacharunderscore}{\kern0pt}frecrel{\isacharunderscore}{\kern0pt}fm\ {\isacharcolon}{\kern0pt}\isanewline
\ \ \isakeyword{assumes}\ {\isachardoublequoteopen}a{\isasymin}nat{\isachardoublequoteclose}\ \ {\isachardoublequoteopen}b{\isasymin}nat{\isachardoublequoteclose}\isanewline
\ \ \isakeyword{shows}\ {\isachardoublequoteopen}arity{\isacharparenleft}{\kern0pt}frecrel{\isacharunderscore}{\kern0pt}fm{\isacharparenleft}{\kern0pt}a{\isacharcomma}{\kern0pt}b{\isacharparenright}{\kern0pt}{\isacharparenright}{\kern0pt}\ {\isacharequal}{\kern0pt}\ succ{\isacharparenleft}{\kern0pt}a{\isacharparenright}{\kern0pt}\ {\isasymunion}\ succ{\isacharparenleft}{\kern0pt}b{\isacharparenright}{\kern0pt}{\isachardoublequoteclose}\isanewline
%
\isadelimproof
\ \ %
\endisadelimproof
%
\isatagproof
\isacommand{unfolding}\isamarkupfalse%
\ frecrel{\isacharunderscore}{\kern0pt}fm{\isacharunderscore}{\kern0pt}def\isanewline
\ \ \isacommand{using}\isamarkupfalse%
\ assms\ arity{\isacharunderscore}{\kern0pt}Collect{\isacharunderscore}{\kern0pt}fm\ arity{\isacharunderscore}{\kern0pt}cartprod{\isacharunderscore}{\kern0pt}fm\ arity{\isacharunderscore}{\kern0pt}frecrelP{\isacharunderscore}{\kern0pt}fm\ pred{\isacharunderscore}{\kern0pt}Un{\isacharunderscore}{\kern0pt}distrib\isanewline
\ \ \isacommand{by}\isamarkupfalse%
\ auto%
\endisatagproof
{\isafoldproof}%
%
\isadelimproof
\isanewline
%
\endisadelimproof
\isanewline
\isacommand{lemma}\isamarkupfalse%
\ sats{\isacharunderscore}{\kern0pt}frecrel{\isacharunderscore}{\kern0pt}fm\ {\isacharcolon}{\kern0pt}\isanewline
\ \ \isakeyword{assumes}\isanewline
\ \ \ \ {\isachardoublequoteopen}a{\isasymin}nat{\isachardoublequoteclose}\ \ {\isachardoublequoteopen}r{\isasymin}nat{\isachardoublequoteclose}\ {\isachardoublequoteopen}env{\isasymin}list{\isacharparenleft}{\kern0pt}A{\isacharparenright}{\kern0pt}{\isachardoublequoteclose}\isanewline
\ \ \isakeyword{shows}\isanewline
\ \ \ \ {\isachardoublequoteopen}sats{\isacharparenleft}{\kern0pt}A{\isacharcomma}{\kern0pt}frecrel{\isacharunderscore}{\kern0pt}fm{\isacharparenleft}{\kern0pt}a{\isacharcomma}{\kern0pt}r{\isacharparenright}{\kern0pt}{\isacharcomma}{\kern0pt}env{\isacharparenright}{\kern0pt}\isanewline
\ \ \ \ {\isasymlongleftrightarrow}\ is{\isacharunderscore}{\kern0pt}frecrel{\isacharparenleft}{\kern0pt}{\isacharhash}{\kern0pt}{\isacharhash}{\kern0pt}A{\isacharcomma}{\kern0pt}nth{\isacharparenleft}{\kern0pt}a{\isacharcomma}{\kern0pt}\ env{\isacharparenright}{\kern0pt}{\isacharcomma}{\kern0pt}nth{\isacharparenleft}{\kern0pt}r{\isacharcomma}{\kern0pt}\ env{\isacharparenright}{\kern0pt}{\isacharparenright}{\kern0pt}{\isachardoublequoteclose}\isanewline
%
\isadelimproof
\ \ %
\endisadelimproof
%
\isatagproof
\isacommand{unfolding}\isamarkupfalse%
\ is{\isacharunderscore}{\kern0pt}frecrel{\isacharunderscore}{\kern0pt}def\ frecrel{\isacharunderscore}{\kern0pt}fm{\isacharunderscore}{\kern0pt}def\isanewline
\ \ \isacommand{using}\isamarkupfalse%
\ assms\isanewline
\ \ \isacommand{by}\isamarkupfalse%
\ {\isacharparenleft}{\kern0pt}simp\ add{\isacharcolon}{\kern0pt}sats{\isacharunderscore}{\kern0pt}Collect{\isacharunderscore}{\kern0pt}fm\ sats{\isacharunderscore}{\kern0pt}frecrelP{\isacharunderscore}{\kern0pt}fm{\isacharparenright}{\kern0pt}%
\endisatagproof
{\isafoldproof}%
%
\isadelimproof
\isanewline
%
\endisadelimproof
\isanewline
\isacommand{lemma}\isamarkupfalse%
\ is{\isacharunderscore}{\kern0pt}frecrel{\isacharunderscore}{\kern0pt}iff{\isacharunderscore}{\kern0pt}sats{\isacharcolon}{\kern0pt}\isanewline
\ \ \isakeyword{assumes}\isanewline
\ \ \ \ {\isachardoublequoteopen}nth{\isacharparenleft}{\kern0pt}a{\isacharcomma}{\kern0pt}env{\isacharparenright}{\kern0pt}\ {\isacharequal}{\kern0pt}\ aa{\isachardoublequoteclose}\ {\isachardoublequoteopen}nth{\isacharparenleft}{\kern0pt}r{\isacharcomma}{\kern0pt}env{\isacharparenright}{\kern0pt}\ {\isacharequal}{\kern0pt}\ rr{\isachardoublequoteclose}\ {\isachardoublequoteopen}a{\isasymin}\ nat{\isachardoublequoteclose}\ \ {\isachardoublequoteopen}r{\isasymin}\ nat{\isachardoublequoteclose}\ \ {\isachardoublequoteopen}env\ {\isasymin}\ list{\isacharparenleft}{\kern0pt}A{\isacharparenright}{\kern0pt}{\isachardoublequoteclose}\isanewline
\ \ \isakeyword{shows}\isanewline
\ \ \ \ {\isachardoublequoteopen}is{\isacharunderscore}{\kern0pt}frecrel{\isacharparenleft}{\kern0pt}{\isacharhash}{\kern0pt}{\isacharhash}{\kern0pt}A{\isacharcomma}{\kern0pt}\ aa{\isacharcomma}{\kern0pt}rr{\isacharparenright}{\kern0pt}\ {\isasymlongleftrightarrow}\ sats{\isacharparenleft}{\kern0pt}A{\isacharcomma}{\kern0pt}\ frecrel{\isacharunderscore}{\kern0pt}fm{\isacharparenleft}{\kern0pt}a{\isacharcomma}{\kern0pt}r{\isacharparenright}{\kern0pt}{\isacharcomma}{\kern0pt}\ env{\isacharparenright}{\kern0pt}{\isachardoublequoteclose}\isanewline
%
\isadelimproof
\ \ %
\endisadelimproof
%
\isatagproof
\isacommand{using}\isamarkupfalse%
\ assms\isanewline
\ \ \isacommand{by}\isamarkupfalse%
\ {\isacharparenleft}{\kern0pt}simp\ add{\isacharcolon}{\kern0pt}sats{\isacharunderscore}{\kern0pt}frecrel{\isacharunderscore}{\kern0pt}fm{\isacharparenright}{\kern0pt}%
\endisatagproof
{\isafoldproof}%
%
\isadelimproof
\isanewline
%
\endisadelimproof
\isanewline
\isacommand{definition}\isamarkupfalse%
\isanewline
\ \ names{\isacharunderscore}{\kern0pt}below\ {\isacharcolon}{\kern0pt}{\isacharcolon}{\kern0pt}\ {\isachardoublequoteopen}i\ {\isasymRightarrow}\ i\ {\isasymRightarrow}\ i{\isachardoublequoteclose}\ \isakeyword{where}\isanewline
\ \ {\isachardoublequoteopen}names{\isacharunderscore}{\kern0pt}below{\isacharparenleft}{\kern0pt}P{\isacharcomma}{\kern0pt}x{\isacharparenright}{\kern0pt}\ {\isasymequiv}\ {\isadigit{2}}{\isasymtimes}ecloseN{\isacharparenleft}{\kern0pt}x{\isacharparenright}{\kern0pt}{\isasymtimes}ecloseN{\isacharparenleft}{\kern0pt}x{\isacharparenright}{\kern0pt}{\isasymtimes}P{\isachardoublequoteclose}\isanewline
\isanewline
\isacommand{lemma}\isamarkupfalse%
\ names{\isacharunderscore}{\kern0pt}belowsD{\isacharcolon}{\kern0pt}\isanewline
\ \ \isakeyword{assumes}\ {\isachardoublequoteopen}x\ {\isasymin}\ names{\isacharunderscore}{\kern0pt}below{\isacharparenleft}{\kern0pt}P{\isacharcomma}{\kern0pt}z{\isacharparenright}{\kern0pt}{\isachardoublequoteclose}\isanewline
\ \ \isakeyword{obtains}\ f\ n{\isadigit{1}}\ n{\isadigit{2}}\ p\ \isakeyword{where}\isanewline
\ \ \ \ {\isachardoublequoteopen}x\ {\isacharequal}{\kern0pt}\ {\isasymlangle}f{\isacharcomma}{\kern0pt}n{\isadigit{1}}{\isacharcomma}{\kern0pt}n{\isadigit{2}}{\isacharcomma}{\kern0pt}p{\isasymrangle}{\isachardoublequoteclose}\ {\isachardoublequoteopen}f{\isasymin}{\isadigit{2}}{\isachardoublequoteclose}\ {\isachardoublequoteopen}n{\isadigit{1}}{\isasymin}ecloseN{\isacharparenleft}{\kern0pt}z{\isacharparenright}{\kern0pt}{\isachardoublequoteclose}\ {\isachardoublequoteopen}n{\isadigit{2}}{\isasymin}ecloseN{\isacharparenleft}{\kern0pt}z{\isacharparenright}{\kern0pt}{\isachardoublequoteclose}\ {\isachardoublequoteopen}p{\isasymin}P{\isachardoublequoteclose}\isanewline
%
\isadelimproof
\ \ %
\endisadelimproof
%
\isatagproof
\isacommand{using}\isamarkupfalse%
\ assms\ \isacommand{unfolding}\isamarkupfalse%
\ names{\isacharunderscore}{\kern0pt}below{\isacharunderscore}{\kern0pt}def\ \isacommand{by}\isamarkupfalse%
\ auto%
\endisatagproof
{\isafoldproof}%
%
\isadelimproof
\isanewline
%
\endisadelimproof
\isanewline
\isanewline
\isacommand{definition}\isamarkupfalse%
\isanewline
\ \ is{\isacharunderscore}{\kern0pt}names{\isacharunderscore}{\kern0pt}below\ {\isacharcolon}{\kern0pt}{\isacharcolon}{\kern0pt}\ {\isachardoublequoteopen}{\isacharbrackleft}{\kern0pt}i{\isasymRightarrow}o{\isacharcomma}{\kern0pt}i{\isacharcomma}{\kern0pt}i{\isacharcomma}{\kern0pt}i{\isacharbrackright}{\kern0pt}\ {\isasymRightarrow}\ o{\isachardoublequoteclose}\ \isakeyword{where}\isanewline
\ \ {\isachardoublequoteopen}is{\isacharunderscore}{\kern0pt}names{\isacharunderscore}{\kern0pt}below{\isacharparenleft}{\kern0pt}M{\isacharcomma}{\kern0pt}P{\isacharcomma}{\kern0pt}x{\isacharcomma}{\kern0pt}nb{\isacharparenright}{\kern0pt}\ {\isasymequiv}\ {\isasymexists}p{\isadigit{1}}{\isacharbrackleft}{\kern0pt}M{\isacharbrackright}{\kern0pt}{\isachardot}{\kern0pt}\ {\isasymexists}p{\isadigit{0}}{\isacharbrackleft}{\kern0pt}M{\isacharbrackright}{\kern0pt}{\isachardot}{\kern0pt}\ {\isasymexists}t{\isacharbrackleft}{\kern0pt}M{\isacharbrackright}{\kern0pt}{\isachardot}{\kern0pt}\ {\isasymexists}ec{\isacharbrackleft}{\kern0pt}M{\isacharbrackright}{\kern0pt}{\isachardot}{\kern0pt}\isanewline
\ \ \ \ \ \ \ \ \ \ \ \ \ \ is{\isacharunderscore}{\kern0pt}ecloseN{\isacharparenleft}{\kern0pt}M{\isacharcomma}{\kern0pt}ec{\isacharcomma}{\kern0pt}x{\isacharparenright}{\kern0pt}\ {\isasymand}\ number{\isadigit{2}}{\isacharparenleft}{\kern0pt}M{\isacharcomma}{\kern0pt}t{\isacharparenright}{\kern0pt}\ {\isasymand}\ cartprod{\isacharparenleft}{\kern0pt}M{\isacharcomma}{\kern0pt}ec{\isacharcomma}{\kern0pt}P{\isacharcomma}{\kern0pt}p{\isadigit{0}}{\isacharparenright}{\kern0pt}\ {\isasymand}\ cartprod{\isacharparenleft}{\kern0pt}M{\isacharcomma}{\kern0pt}ec{\isacharcomma}{\kern0pt}p{\isadigit{0}}{\isacharcomma}{\kern0pt}p{\isadigit{1}}{\isacharparenright}{\kern0pt}\isanewline
\ \ \ \ \ \ \ \ \ \ \ \ \ \ {\isasymand}\ cartprod{\isacharparenleft}{\kern0pt}M{\isacharcomma}{\kern0pt}t{\isacharcomma}{\kern0pt}p{\isadigit{1}}{\isacharcomma}{\kern0pt}nb{\isacharparenright}{\kern0pt}{\isachardoublequoteclose}\isanewline
\isanewline
\isacommand{definition}\isamarkupfalse%
\isanewline
\ \ number{\isadigit{2}}{\isacharunderscore}{\kern0pt}fm\ {\isacharcolon}{\kern0pt}{\isacharcolon}{\kern0pt}\ {\isachardoublequoteopen}i{\isasymRightarrow}i{\isachardoublequoteclose}\ \isakeyword{where}\isanewline
\ \ {\isachardoublequoteopen}number{\isadigit{2}}{\isacharunderscore}{\kern0pt}fm{\isacharparenleft}{\kern0pt}a{\isacharparenright}{\kern0pt}\ {\isasymequiv}\ Exists{\isacharparenleft}{\kern0pt}And{\isacharparenleft}{\kern0pt}number{\isadigit{1}}{\isacharunderscore}{\kern0pt}fm{\isacharparenleft}{\kern0pt}{\isadigit{0}}{\isacharparenright}{\kern0pt}{\isacharcomma}{\kern0pt}\ succ{\isacharunderscore}{\kern0pt}fm{\isacharparenleft}{\kern0pt}{\isadigit{0}}{\isacharcomma}{\kern0pt}succ{\isacharparenleft}{\kern0pt}a{\isacharparenright}{\kern0pt}{\isacharparenright}{\kern0pt}{\isacharparenright}{\kern0pt}{\isacharparenright}{\kern0pt}{\isachardoublequoteclose}\isanewline
\isanewline
\isacommand{lemma}\isamarkupfalse%
\ number{\isadigit{2}}{\isacharunderscore}{\kern0pt}fm{\isacharunderscore}{\kern0pt}type{\isacharbrackleft}{\kern0pt}TC{\isacharbrackright}{\kern0pt}\ {\isacharcolon}{\kern0pt}\isanewline
\ \ {\isachardoublequoteopen}a{\isasymin}nat\ {\isasymLongrightarrow}\ number{\isadigit{2}}{\isacharunderscore}{\kern0pt}fm{\isacharparenleft}{\kern0pt}a{\isacharparenright}{\kern0pt}\ {\isasymin}\ formula{\isachardoublequoteclose}\isanewline
%
\isadelimproof
\ \ %
\endisadelimproof
%
\isatagproof
\isacommand{unfolding}\isamarkupfalse%
\ number{\isadigit{2}}{\isacharunderscore}{\kern0pt}fm{\isacharunderscore}{\kern0pt}def\ \isacommand{by}\isamarkupfalse%
\ simp%
\endisatagproof
{\isafoldproof}%
%
\isadelimproof
\isanewline
%
\endisadelimproof
\isanewline
\isacommand{lemma}\isamarkupfalse%
\ number{\isadigit{2}}arity{\isacharunderscore}{\kern0pt}{\isacharunderscore}{\kern0pt}fm\ {\isacharcolon}{\kern0pt}\isanewline
\ \ {\isachardoublequoteopen}a{\isasymin}nat\ {\isasymLongrightarrow}\ arity{\isacharparenleft}{\kern0pt}number{\isadigit{2}}{\isacharunderscore}{\kern0pt}fm{\isacharparenleft}{\kern0pt}a{\isacharparenright}{\kern0pt}{\isacharparenright}{\kern0pt}\ {\isacharequal}{\kern0pt}\ succ{\isacharparenleft}{\kern0pt}a{\isacharparenright}{\kern0pt}{\isachardoublequoteclose}\isanewline
%
\isadelimproof
\ \ %
\endisadelimproof
%
\isatagproof
\isacommand{unfolding}\isamarkupfalse%
\ number{\isadigit{2}}{\isacharunderscore}{\kern0pt}fm{\isacharunderscore}{\kern0pt}def\isanewline
\ \ \isacommand{using}\isamarkupfalse%
\ number{\isadigit{1}}arity{\isacharunderscore}{\kern0pt}{\isacharunderscore}{\kern0pt}fm\ arity{\isacharunderscore}{\kern0pt}succ{\isacharunderscore}{\kern0pt}fm\ nat{\isacharunderscore}{\kern0pt}union{\isacharunderscore}{\kern0pt}abs{\isadigit{2}}\ pred{\isacharunderscore}{\kern0pt}Un{\isacharunderscore}{\kern0pt}distrib\isanewline
\ \ \isacommand{by}\isamarkupfalse%
\ simp%
\endisatagproof
{\isafoldproof}%
%
\isadelimproof
\isanewline
%
\endisadelimproof
\isanewline
\isacommand{lemma}\isamarkupfalse%
\ sats{\isacharunderscore}{\kern0pt}number{\isadigit{2}}{\isacharunderscore}{\kern0pt}fm\ {\isacharbrackleft}{\kern0pt}simp{\isacharbrackright}{\kern0pt}{\isacharcolon}{\kern0pt}\isanewline
\ \ {\isachardoublequoteopen}{\isasymlbrakk}\ x\ {\isasymin}\ nat{\isacharsemicolon}{\kern0pt}\ env\ {\isasymin}\ list{\isacharparenleft}{\kern0pt}A{\isacharparenright}{\kern0pt}\ {\isasymrbrakk}\isanewline
\ \ \ \ {\isasymLongrightarrow}\ sats{\isacharparenleft}{\kern0pt}A{\isacharcomma}{\kern0pt}\ number{\isadigit{2}}{\isacharunderscore}{\kern0pt}fm{\isacharparenleft}{\kern0pt}x{\isacharparenright}{\kern0pt}{\isacharcomma}{\kern0pt}\ env{\isacharparenright}{\kern0pt}\ {\isasymlongleftrightarrow}\ number{\isadigit{2}}{\isacharparenleft}{\kern0pt}{\isacharhash}{\kern0pt}{\isacharhash}{\kern0pt}A{\isacharcomma}{\kern0pt}\ nth{\isacharparenleft}{\kern0pt}x{\isacharcomma}{\kern0pt}env{\isacharparenright}{\kern0pt}{\isacharparenright}{\kern0pt}{\isachardoublequoteclose}\isanewline
%
\isadelimproof
\ \ %
\endisadelimproof
%
\isatagproof
\isacommand{by}\isamarkupfalse%
\ {\isacharparenleft}{\kern0pt}simp\ add{\isacharcolon}{\kern0pt}\ number{\isadigit{2}}{\isacharunderscore}{\kern0pt}fm{\isacharunderscore}{\kern0pt}def\ number{\isadigit{2}}{\isacharunderscore}{\kern0pt}def{\isacharparenright}{\kern0pt}%
\endisatagproof
{\isafoldproof}%
%
\isadelimproof
\isanewline
%
\endisadelimproof
\isanewline
\isacommand{definition}\isamarkupfalse%
\isanewline
\ \ is{\isacharunderscore}{\kern0pt}names{\isacharunderscore}{\kern0pt}below{\isacharunderscore}{\kern0pt}fm\ {\isacharcolon}{\kern0pt}{\isacharcolon}{\kern0pt}\ {\isachardoublequoteopen}{\isacharbrackleft}{\kern0pt}i{\isacharcomma}{\kern0pt}i{\isacharcomma}{\kern0pt}i{\isacharbrackright}{\kern0pt}\ {\isasymRightarrow}\ i{\isachardoublequoteclose}\ \isakeyword{where}\isanewline
\ \ {\isachardoublequoteopen}is{\isacharunderscore}{\kern0pt}names{\isacharunderscore}{\kern0pt}below{\isacharunderscore}{\kern0pt}fm{\isacharparenleft}{\kern0pt}P{\isacharcomma}{\kern0pt}x{\isacharcomma}{\kern0pt}nb{\isacharparenright}{\kern0pt}\ {\isasymequiv}\ Exists{\isacharparenleft}{\kern0pt}Exists{\isacharparenleft}{\kern0pt}Exists{\isacharparenleft}{\kern0pt}Exists{\isacharparenleft}{\kern0pt}\isanewline
\ \ \ \ \ \ \ \ \ \ \ \ \ \ \ \ \ \ \ \ And{\isacharparenleft}{\kern0pt}ecloseN{\isacharunderscore}{\kern0pt}fm{\isacharparenleft}{\kern0pt}{\isadigit{0}}{\isacharcomma}{\kern0pt}x\ {\isacharhash}{\kern0pt}{\isacharplus}{\kern0pt}\ {\isadigit{4}}{\isacharparenright}{\kern0pt}{\isacharcomma}{\kern0pt}And{\isacharparenleft}{\kern0pt}number{\isadigit{2}}{\isacharunderscore}{\kern0pt}fm{\isacharparenleft}{\kern0pt}{\isadigit{1}}{\isacharparenright}{\kern0pt}{\isacharcomma}{\kern0pt}\isanewline
\ \ \ \ \ \ \ \ \ \ \ \ \ \ \ \ \ \ \ \ And{\isacharparenleft}{\kern0pt}cartprod{\isacharunderscore}{\kern0pt}fm{\isacharparenleft}{\kern0pt}{\isadigit{0}}{\isacharcomma}{\kern0pt}P\ {\isacharhash}{\kern0pt}{\isacharplus}{\kern0pt}\ {\isadigit{4}}{\isacharcomma}{\kern0pt}{\isadigit{2}}{\isacharparenright}{\kern0pt}{\isacharcomma}{\kern0pt}And{\isacharparenleft}{\kern0pt}cartprod{\isacharunderscore}{\kern0pt}fm{\isacharparenleft}{\kern0pt}{\isadigit{0}}{\isacharcomma}{\kern0pt}{\isadigit{2}}{\isacharcomma}{\kern0pt}{\isadigit{3}}{\isacharparenright}{\kern0pt}{\isacharcomma}{\kern0pt}cartprod{\isacharunderscore}{\kern0pt}fm{\isacharparenleft}{\kern0pt}{\isadigit{1}}{\isacharcomma}{\kern0pt}{\isadigit{3}}{\isacharcomma}{\kern0pt}nb{\isacharhash}{\kern0pt}{\isacharplus}{\kern0pt}{\isadigit{4}}{\isacharparenright}{\kern0pt}{\isacharparenright}{\kern0pt}{\isacharparenright}{\kern0pt}{\isacharparenright}{\kern0pt}{\isacharparenright}{\kern0pt}{\isacharparenright}{\kern0pt}{\isacharparenright}{\kern0pt}{\isacharparenright}{\kern0pt}{\isacharparenright}{\kern0pt}{\isachardoublequoteclose}\isanewline
\isanewline
\isacommand{lemma}\isamarkupfalse%
\ arity{\isacharunderscore}{\kern0pt}is{\isacharunderscore}{\kern0pt}names{\isacharunderscore}{\kern0pt}below{\isacharunderscore}{\kern0pt}fm\ {\isacharcolon}{\kern0pt}\isanewline
\ \ {\isachardoublequoteopen}{\isasymlbrakk}P{\isasymin}nat{\isacharsemicolon}{\kern0pt}x{\isasymin}nat{\isacharsemicolon}{\kern0pt}nb{\isasymin}nat{\isasymrbrakk}\ {\isasymLongrightarrow}\ arity{\isacharparenleft}{\kern0pt}is{\isacharunderscore}{\kern0pt}names{\isacharunderscore}{\kern0pt}below{\isacharunderscore}{\kern0pt}fm{\isacharparenleft}{\kern0pt}P{\isacharcomma}{\kern0pt}x{\isacharcomma}{\kern0pt}nb{\isacharparenright}{\kern0pt}{\isacharparenright}{\kern0pt}\ {\isacharequal}{\kern0pt}\ succ{\isacharparenleft}{\kern0pt}P{\isacharparenright}{\kern0pt}\ {\isasymunion}\ succ{\isacharparenleft}{\kern0pt}x{\isacharparenright}{\kern0pt}\ {\isasymunion}\ succ{\isacharparenleft}{\kern0pt}nb{\isacharparenright}{\kern0pt}{\isachardoublequoteclose}\isanewline
%
\isadelimproof
\ \ %
\endisadelimproof
%
\isatagproof
\isacommand{unfolding}\isamarkupfalse%
\ is{\isacharunderscore}{\kern0pt}names{\isacharunderscore}{\kern0pt}below{\isacharunderscore}{\kern0pt}fm{\isacharunderscore}{\kern0pt}def\isanewline
\ \ \isacommand{using}\isamarkupfalse%
\ arity{\isacharunderscore}{\kern0pt}cartprod{\isacharunderscore}{\kern0pt}fm\ number{\isadigit{2}}arity{\isacharunderscore}{\kern0pt}{\isacharunderscore}{\kern0pt}fm\ arity{\isacharunderscore}{\kern0pt}ecloseN{\isacharunderscore}{\kern0pt}fm\ nat{\isacharunderscore}{\kern0pt}union{\isacharunderscore}{\kern0pt}abs{\isadigit{2}}\ pred{\isacharunderscore}{\kern0pt}Un{\isacharunderscore}{\kern0pt}distrib\isanewline
\ \ \isacommand{by}\isamarkupfalse%
\ auto%
\endisatagproof
{\isafoldproof}%
%
\isadelimproof
\isanewline
%
\endisadelimproof
\isanewline
\isanewline
\isacommand{lemma}\isamarkupfalse%
\ is{\isacharunderscore}{\kern0pt}names{\isacharunderscore}{\kern0pt}below{\isacharunderscore}{\kern0pt}fm{\isacharunderscore}{\kern0pt}type{\isacharbrackleft}{\kern0pt}TC{\isacharbrackright}{\kern0pt}{\isacharcolon}{\kern0pt}\isanewline
\ \ {\isachardoublequoteopen}{\isasymlbrakk}P{\isasymin}nat{\isacharsemicolon}{\kern0pt}x{\isasymin}nat{\isacharsemicolon}{\kern0pt}nb{\isasymin}nat{\isasymrbrakk}\ {\isasymLongrightarrow}\ is{\isacharunderscore}{\kern0pt}names{\isacharunderscore}{\kern0pt}below{\isacharunderscore}{\kern0pt}fm{\isacharparenleft}{\kern0pt}P{\isacharcomma}{\kern0pt}x{\isacharcomma}{\kern0pt}nb{\isacharparenright}{\kern0pt}{\isasymin}formula{\isachardoublequoteclose}\isanewline
%
\isadelimproof
\ \ %
\endisadelimproof
%
\isatagproof
\isacommand{unfolding}\isamarkupfalse%
\ is{\isacharunderscore}{\kern0pt}names{\isacharunderscore}{\kern0pt}below{\isacharunderscore}{\kern0pt}fm{\isacharunderscore}{\kern0pt}def\ \isacommand{by}\isamarkupfalse%
\ simp%
\endisatagproof
{\isafoldproof}%
%
\isadelimproof
\isanewline
%
\endisadelimproof
\isanewline
\isacommand{lemma}\isamarkupfalse%
\ sats{\isacharunderscore}{\kern0pt}is{\isacharunderscore}{\kern0pt}names{\isacharunderscore}{\kern0pt}below{\isacharunderscore}{\kern0pt}fm\ {\isacharcolon}{\kern0pt}\isanewline
\ \ \isakeyword{assumes}\isanewline
\ \ \ \ {\isachardoublequoteopen}P{\isasymin}nat{\isachardoublequoteclose}\ {\isachardoublequoteopen}x{\isasymin}nat{\isachardoublequoteclose}\ {\isachardoublequoteopen}nb{\isasymin}nat{\isachardoublequoteclose}\ {\isachardoublequoteopen}env{\isasymin}list{\isacharparenleft}{\kern0pt}A{\isacharparenright}{\kern0pt}{\isachardoublequoteclose}\isanewline
\ \ \isakeyword{shows}\isanewline
\ \ \ \ {\isachardoublequoteopen}sats{\isacharparenleft}{\kern0pt}A{\isacharcomma}{\kern0pt}is{\isacharunderscore}{\kern0pt}names{\isacharunderscore}{\kern0pt}below{\isacharunderscore}{\kern0pt}fm{\isacharparenleft}{\kern0pt}P{\isacharcomma}{\kern0pt}x{\isacharcomma}{\kern0pt}nb{\isacharparenright}{\kern0pt}{\isacharcomma}{\kern0pt}env{\isacharparenright}{\kern0pt}\isanewline
\ \ \ \ {\isasymlongleftrightarrow}\ is{\isacharunderscore}{\kern0pt}names{\isacharunderscore}{\kern0pt}below{\isacharparenleft}{\kern0pt}{\isacharhash}{\kern0pt}{\isacharhash}{\kern0pt}A{\isacharcomma}{\kern0pt}nth{\isacharparenleft}{\kern0pt}P{\isacharcomma}{\kern0pt}\ env{\isacharparenright}{\kern0pt}{\isacharcomma}{\kern0pt}nth{\isacharparenleft}{\kern0pt}x{\isacharcomma}{\kern0pt}\ env{\isacharparenright}{\kern0pt}{\isacharcomma}{\kern0pt}nth{\isacharparenleft}{\kern0pt}nb{\isacharcomma}{\kern0pt}\ env{\isacharparenright}{\kern0pt}{\isacharparenright}{\kern0pt}{\isachardoublequoteclose}\isanewline
%
\isadelimproof
\ \ %
\endisadelimproof
%
\isatagproof
\isacommand{unfolding}\isamarkupfalse%
\ is{\isacharunderscore}{\kern0pt}names{\isacharunderscore}{\kern0pt}below{\isacharunderscore}{\kern0pt}fm{\isacharunderscore}{\kern0pt}def\ is{\isacharunderscore}{\kern0pt}names{\isacharunderscore}{\kern0pt}below{\isacharunderscore}{\kern0pt}def\ \isacommand{using}\isamarkupfalse%
\ assms\ \isacommand{by}\isamarkupfalse%
\ simp%
\endisatagproof
{\isafoldproof}%
%
\isadelimproof
\isanewline
%
\endisadelimproof
\isanewline
\isacommand{definition}\isamarkupfalse%
\isanewline
\ \ is{\isacharunderscore}{\kern0pt}tuple\ {\isacharcolon}{\kern0pt}{\isacharcolon}{\kern0pt}\ {\isachardoublequoteopen}{\isacharbrackleft}{\kern0pt}i{\isasymRightarrow}o{\isacharcomma}{\kern0pt}i{\isacharcomma}{\kern0pt}i{\isacharcomma}{\kern0pt}i{\isacharcomma}{\kern0pt}i{\isacharcomma}{\kern0pt}i{\isacharbrackright}{\kern0pt}\ {\isasymRightarrow}\ o{\isachardoublequoteclose}\ \isakeyword{where}\isanewline
\ \ {\isachardoublequoteopen}is{\isacharunderscore}{\kern0pt}tuple{\isacharparenleft}{\kern0pt}M{\isacharcomma}{\kern0pt}z{\isacharcomma}{\kern0pt}t{\isadigit{1}}{\isacharcomma}{\kern0pt}t{\isadigit{2}}{\isacharcomma}{\kern0pt}p{\isacharcomma}{\kern0pt}t{\isacharparenright}{\kern0pt}\ {\isasymequiv}\ {\isasymexists}t{\isadigit{1}}t{\isadigit{2}}p{\isacharbrackleft}{\kern0pt}M{\isacharbrackright}{\kern0pt}{\isachardot}{\kern0pt}\ {\isasymexists}t{\isadigit{2}}p{\isacharbrackleft}{\kern0pt}M{\isacharbrackright}{\kern0pt}{\isachardot}{\kern0pt}\ pair{\isacharparenleft}{\kern0pt}M{\isacharcomma}{\kern0pt}t{\isadigit{2}}{\isacharcomma}{\kern0pt}p{\isacharcomma}{\kern0pt}t{\isadigit{2}}p{\isacharparenright}{\kern0pt}\ {\isasymand}\ pair{\isacharparenleft}{\kern0pt}M{\isacharcomma}{\kern0pt}t{\isadigit{1}}{\isacharcomma}{\kern0pt}t{\isadigit{2}}p{\isacharcomma}{\kern0pt}t{\isadigit{1}}t{\isadigit{2}}p{\isacharparenright}{\kern0pt}\ {\isasymand}\isanewline
\ \ \ \ \ \ \ \ \ \ \ \ \ \ \ \ \ \ \ \ \ \ \ \ \ \ \ \ \ \ \ \ \ \ \ \ \ \ \ \ \ \ \ \ \ \ \ \ \ \ pair{\isacharparenleft}{\kern0pt}M{\isacharcomma}{\kern0pt}z{\isacharcomma}{\kern0pt}t{\isadigit{1}}t{\isadigit{2}}p{\isacharcomma}{\kern0pt}t{\isacharparenright}{\kern0pt}{\isachardoublequoteclose}\isanewline
\isanewline
\isanewline
\isacommand{definition}\isamarkupfalse%
\isanewline
\ \ is{\isacharunderscore}{\kern0pt}tuple{\isacharunderscore}{\kern0pt}fm\ {\isacharcolon}{\kern0pt}{\isacharcolon}{\kern0pt}\ {\isachardoublequoteopen}{\isacharbrackleft}{\kern0pt}i{\isacharcomma}{\kern0pt}i{\isacharcomma}{\kern0pt}i{\isacharcomma}{\kern0pt}i{\isacharcomma}{\kern0pt}i{\isacharbrackright}{\kern0pt}\ {\isasymRightarrow}\ i{\isachardoublequoteclose}\ \isakeyword{where}\isanewline
\ \ {\isachardoublequoteopen}is{\isacharunderscore}{\kern0pt}tuple{\isacharunderscore}{\kern0pt}fm{\isacharparenleft}{\kern0pt}z{\isacharcomma}{\kern0pt}t{\isadigit{1}}{\isacharcomma}{\kern0pt}t{\isadigit{2}}{\isacharcomma}{\kern0pt}p{\isacharcomma}{\kern0pt}tup{\isacharparenright}{\kern0pt}\ {\isacharequal}{\kern0pt}\ Exists{\isacharparenleft}{\kern0pt}Exists{\isacharparenleft}{\kern0pt}And{\isacharparenleft}{\kern0pt}pair{\isacharunderscore}{\kern0pt}fm{\isacharparenleft}{\kern0pt}t{\isadigit{2}}\ {\isacharhash}{\kern0pt}{\isacharplus}{\kern0pt}\ {\isadigit{2}}{\isacharcomma}{\kern0pt}p\ {\isacharhash}{\kern0pt}{\isacharplus}{\kern0pt}\ {\isadigit{2}}{\isacharcomma}{\kern0pt}{\isadigit{0}}{\isacharparenright}{\kern0pt}{\isacharcomma}{\kern0pt}\isanewline
\ \ \ \ \ \ \ \ \ \ \ \ \ \ \ \ \ \ \ \ \ \ And{\isacharparenleft}{\kern0pt}pair{\isacharunderscore}{\kern0pt}fm{\isacharparenleft}{\kern0pt}t{\isadigit{1}}\ {\isacharhash}{\kern0pt}{\isacharplus}{\kern0pt}\ {\isadigit{2}}{\isacharcomma}{\kern0pt}{\isadigit{0}}{\isacharcomma}{\kern0pt}{\isadigit{1}}{\isacharparenright}{\kern0pt}{\isacharcomma}{\kern0pt}pair{\isacharunderscore}{\kern0pt}fm{\isacharparenleft}{\kern0pt}z\ {\isacharhash}{\kern0pt}{\isacharplus}{\kern0pt}\ {\isadigit{2}}{\isacharcomma}{\kern0pt}{\isadigit{1}}{\isacharcomma}{\kern0pt}tup\ {\isacharhash}{\kern0pt}{\isacharplus}{\kern0pt}\ {\isadigit{2}}{\isacharparenright}{\kern0pt}{\isacharparenright}{\kern0pt}{\isacharparenright}{\kern0pt}{\isacharparenright}{\kern0pt}{\isacharparenright}{\kern0pt}{\isachardoublequoteclose}\isanewline
\isanewline
\isanewline
\isacommand{lemma}\isamarkupfalse%
\ arity{\isacharunderscore}{\kern0pt}is{\isacharunderscore}{\kern0pt}tuple{\isacharunderscore}{\kern0pt}fm\ {\isacharcolon}{\kern0pt}\ {\isachardoublequoteopen}{\isasymlbrakk}\ z{\isasymin}nat\ {\isacharsemicolon}{\kern0pt}\ t{\isadigit{1}}{\isasymin}nat\ {\isacharsemicolon}{\kern0pt}\ t{\isadigit{2}}{\isasymin}nat\ {\isacharsemicolon}{\kern0pt}\ p{\isasymin}nat\ {\isacharsemicolon}{\kern0pt}\ tup{\isasymin}nat\ {\isasymrbrakk}\ {\isasymLongrightarrow}\isanewline
\ \ arity{\isacharparenleft}{\kern0pt}is{\isacharunderscore}{\kern0pt}tuple{\isacharunderscore}{\kern0pt}fm{\isacharparenleft}{\kern0pt}z{\isacharcomma}{\kern0pt}t{\isadigit{1}}{\isacharcomma}{\kern0pt}t{\isadigit{2}}{\isacharcomma}{\kern0pt}p{\isacharcomma}{\kern0pt}tup{\isacharparenright}{\kern0pt}{\isacharparenright}{\kern0pt}\ {\isacharequal}{\kern0pt}\ {\isasymUnion}\ {\isacharbraceleft}{\kern0pt}succ{\isacharparenleft}{\kern0pt}z{\isacharparenright}{\kern0pt}{\isacharcomma}{\kern0pt}succ{\isacharparenleft}{\kern0pt}t{\isadigit{1}}{\isacharparenright}{\kern0pt}{\isacharcomma}{\kern0pt}succ{\isacharparenleft}{\kern0pt}t{\isadigit{2}}{\isacharparenright}{\kern0pt}{\isacharcomma}{\kern0pt}succ{\isacharparenleft}{\kern0pt}p{\isacharparenright}{\kern0pt}{\isacharcomma}{\kern0pt}succ{\isacharparenleft}{\kern0pt}tup{\isacharparenright}{\kern0pt}{\isacharbraceright}{\kern0pt}{\isachardoublequoteclose}\isanewline
%
\isadelimproof
\ \ %
\endisadelimproof
%
\isatagproof
\isacommand{unfolding}\isamarkupfalse%
\ is{\isacharunderscore}{\kern0pt}tuple{\isacharunderscore}{\kern0pt}fm{\isacharunderscore}{\kern0pt}def\isanewline
\ \ \isacommand{using}\isamarkupfalse%
\ arity{\isacharunderscore}{\kern0pt}pair{\isacharunderscore}{\kern0pt}fm\ nat{\isacharunderscore}{\kern0pt}union{\isacharunderscore}{\kern0pt}abs{\isadigit{1}}\ nat{\isacharunderscore}{\kern0pt}union{\isacharunderscore}{\kern0pt}abs{\isadigit{2}}\ pred{\isacharunderscore}{\kern0pt}Un{\isacharunderscore}{\kern0pt}distrib\isanewline
\ \ \isacommand{by}\isamarkupfalse%
\ auto%
\endisatagproof
{\isafoldproof}%
%
\isadelimproof
\isanewline
%
\endisadelimproof
\isanewline
\isacommand{lemma}\isamarkupfalse%
\ is{\isacharunderscore}{\kern0pt}tuple{\isacharunderscore}{\kern0pt}fm{\isacharunderscore}{\kern0pt}type{\isacharbrackleft}{\kern0pt}TC{\isacharbrackright}{\kern0pt}\ {\isacharcolon}{\kern0pt}\isanewline
\ \ {\isachardoublequoteopen}z{\isasymin}nat\ {\isasymLongrightarrow}\ t{\isadigit{1}}{\isasymin}nat\ {\isasymLongrightarrow}\ t{\isadigit{2}}{\isasymin}nat\ {\isasymLongrightarrow}\ p{\isasymin}nat\ {\isasymLongrightarrow}\ tup{\isasymin}nat\ {\isasymLongrightarrow}\ is{\isacharunderscore}{\kern0pt}tuple{\isacharunderscore}{\kern0pt}fm{\isacharparenleft}{\kern0pt}z{\isacharcomma}{\kern0pt}t{\isadigit{1}}{\isacharcomma}{\kern0pt}t{\isadigit{2}}{\isacharcomma}{\kern0pt}p{\isacharcomma}{\kern0pt}tup{\isacharparenright}{\kern0pt}{\isasymin}formula{\isachardoublequoteclose}\isanewline
%
\isadelimproof
\ \ %
\endisadelimproof
%
\isatagproof
\isacommand{unfolding}\isamarkupfalse%
\ is{\isacharunderscore}{\kern0pt}tuple{\isacharunderscore}{\kern0pt}fm{\isacharunderscore}{\kern0pt}def\ \isacommand{by}\isamarkupfalse%
\ simp%
\endisatagproof
{\isafoldproof}%
%
\isadelimproof
\isanewline
%
\endisadelimproof
\isanewline
\isacommand{lemma}\isamarkupfalse%
\ sats{\isacharunderscore}{\kern0pt}is{\isacharunderscore}{\kern0pt}tuple{\isacharunderscore}{\kern0pt}fm\ {\isacharcolon}{\kern0pt}\isanewline
\ \ \isakeyword{assumes}\isanewline
\ \ \ \ {\isachardoublequoteopen}z{\isasymin}nat{\isachardoublequoteclose}\ \ {\isachardoublequoteopen}t{\isadigit{1}}{\isasymin}nat{\isachardoublequoteclose}\ {\isachardoublequoteopen}t{\isadigit{2}}{\isasymin}nat{\isachardoublequoteclose}\ {\isachardoublequoteopen}p{\isasymin}nat{\isachardoublequoteclose}\ {\isachardoublequoteopen}tup{\isasymin}nat{\isachardoublequoteclose}\ {\isachardoublequoteopen}env{\isasymin}list{\isacharparenleft}{\kern0pt}A{\isacharparenright}{\kern0pt}{\isachardoublequoteclose}\isanewline
\ \ \isakeyword{shows}\isanewline
\ \ \ \ {\isachardoublequoteopen}sats{\isacharparenleft}{\kern0pt}A{\isacharcomma}{\kern0pt}is{\isacharunderscore}{\kern0pt}tuple{\isacharunderscore}{\kern0pt}fm{\isacharparenleft}{\kern0pt}z{\isacharcomma}{\kern0pt}t{\isadigit{1}}{\isacharcomma}{\kern0pt}t{\isadigit{2}}{\isacharcomma}{\kern0pt}p{\isacharcomma}{\kern0pt}tup{\isacharparenright}{\kern0pt}{\isacharcomma}{\kern0pt}env{\isacharparenright}{\kern0pt}\isanewline
\ \ \ \ {\isasymlongleftrightarrow}\ is{\isacharunderscore}{\kern0pt}tuple{\isacharparenleft}{\kern0pt}{\isacharhash}{\kern0pt}{\isacharhash}{\kern0pt}A{\isacharcomma}{\kern0pt}nth{\isacharparenleft}{\kern0pt}z{\isacharcomma}{\kern0pt}\ env{\isacharparenright}{\kern0pt}{\isacharcomma}{\kern0pt}nth{\isacharparenleft}{\kern0pt}t{\isadigit{1}}{\isacharcomma}{\kern0pt}\ env{\isacharparenright}{\kern0pt}{\isacharcomma}{\kern0pt}nth{\isacharparenleft}{\kern0pt}t{\isadigit{2}}{\isacharcomma}{\kern0pt}\ env{\isacharparenright}{\kern0pt}{\isacharcomma}{\kern0pt}nth{\isacharparenleft}{\kern0pt}p{\isacharcomma}{\kern0pt}\ env{\isacharparenright}{\kern0pt}{\isacharcomma}{\kern0pt}nth{\isacharparenleft}{\kern0pt}tup{\isacharcomma}{\kern0pt}\ env{\isacharparenright}{\kern0pt}{\isacharparenright}{\kern0pt}{\isachardoublequoteclose}\isanewline
%
\isadelimproof
\ \ %
\endisadelimproof
%
\isatagproof
\isacommand{unfolding}\isamarkupfalse%
\ is{\isacharunderscore}{\kern0pt}tuple{\isacharunderscore}{\kern0pt}def\ is{\isacharunderscore}{\kern0pt}tuple{\isacharunderscore}{\kern0pt}fm{\isacharunderscore}{\kern0pt}def\ \isacommand{using}\isamarkupfalse%
\ assms\ \isacommand{by}\isamarkupfalse%
\ simp%
\endisatagproof
{\isafoldproof}%
%
\isadelimproof
\isanewline
%
\endisadelimproof
\isanewline
\isacommand{lemma}\isamarkupfalse%
\ is{\isacharunderscore}{\kern0pt}tuple{\isacharunderscore}{\kern0pt}iff{\isacharunderscore}{\kern0pt}sats{\isacharcolon}{\kern0pt}\isanewline
\ \ \isakeyword{assumes}\isanewline
\ \ \ \ {\isachardoublequoteopen}nth{\isacharparenleft}{\kern0pt}a{\isacharcomma}{\kern0pt}env{\isacharparenright}{\kern0pt}\ {\isacharequal}{\kern0pt}\ aa{\isachardoublequoteclose}\ {\isachardoublequoteopen}nth{\isacharparenleft}{\kern0pt}b{\isacharcomma}{\kern0pt}env{\isacharparenright}{\kern0pt}\ {\isacharequal}{\kern0pt}\ bb{\isachardoublequoteclose}\ {\isachardoublequoteopen}nth{\isacharparenleft}{\kern0pt}c{\isacharcomma}{\kern0pt}env{\isacharparenright}{\kern0pt}\ {\isacharequal}{\kern0pt}\ cc{\isachardoublequoteclose}\ {\isachardoublequoteopen}nth{\isacharparenleft}{\kern0pt}d{\isacharcomma}{\kern0pt}env{\isacharparenright}{\kern0pt}\ {\isacharequal}{\kern0pt}\ dd{\isachardoublequoteclose}\ {\isachardoublequoteopen}nth{\isacharparenleft}{\kern0pt}e{\isacharcomma}{\kern0pt}env{\isacharparenright}{\kern0pt}\ {\isacharequal}{\kern0pt}\ ee{\isachardoublequoteclose}\isanewline
\ \ \ \ {\isachardoublequoteopen}a{\isasymin}nat{\isachardoublequoteclose}\ {\isachardoublequoteopen}b{\isasymin}nat{\isachardoublequoteclose}\ {\isachardoublequoteopen}c{\isasymin}nat{\isachardoublequoteclose}\ {\isachardoublequoteopen}d{\isasymin}nat{\isachardoublequoteclose}\ {\isachardoublequoteopen}e{\isasymin}nat{\isachardoublequoteclose}\ \ {\isachardoublequoteopen}env\ {\isasymin}\ list{\isacharparenleft}{\kern0pt}A{\isacharparenright}{\kern0pt}{\isachardoublequoteclose}\isanewline
\ \ \isakeyword{shows}\isanewline
\ \ \ \ {\isachardoublequoteopen}is{\isacharunderscore}{\kern0pt}tuple{\isacharparenleft}{\kern0pt}{\isacharhash}{\kern0pt}{\isacharhash}{\kern0pt}A{\isacharcomma}{\kern0pt}aa{\isacharcomma}{\kern0pt}bb{\isacharcomma}{\kern0pt}cc{\isacharcomma}{\kern0pt}dd{\isacharcomma}{\kern0pt}ee{\isacharparenright}{\kern0pt}\ \ {\isasymlongleftrightarrow}\ sats{\isacharparenleft}{\kern0pt}A{\isacharcomma}{\kern0pt}\ is{\isacharunderscore}{\kern0pt}tuple{\isacharunderscore}{\kern0pt}fm{\isacharparenleft}{\kern0pt}a{\isacharcomma}{\kern0pt}b{\isacharcomma}{\kern0pt}c{\isacharcomma}{\kern0pt}d{\isacharcomma}{\kern0pt}e{\isacharparenright}{\kern0pt}{\isacharcomma}{\kern0pt}\ env{\isacharparenright}{\kern0pt}{\isachardoublequoteclose}\isanewline
%
\isadelimproof
\ \ %
\endisadelimproof
%
\isatagproof
\isacommand{using}\isamarkupfalse%
\ assms\ \isacommand{by}\isamarkupfalse%
\ {\isacharparenleft}{\kern0pt}simp\ add{\isacharcolon}{\kern0pt}\ sats{\isacharunderscore}{\kern0pt}is{\isacharunderscore}{\kern0pt}tuple{\isacharunderscore}{\kern0pt}fm{\isacharparenright}{\kern0pt}%
\endisatagproof
{\isafoldproof}%
%
\isadelimproof
%
\endisadelimproof
%
\isadelimdocument
%
\endisadelimdocument
%
\isatagdocument
%
\isamarkupsubsection{Definition of \isa{forces} for equality and membership%
}
\isamarkuptrue%
%
\endisatagdocument
{\isafolddocument}%
%
\isadelimdocument
%
\endisadelimdocument
\isacommand{definition}\isamarkupfalse%
\isanewline
\ \ eq{\isacharunderscore}{\kern0pt}case\ {\isacharcolon}{\kern0pt}{\isacharcolon}{\kern0pt}\ {\isachardoublequoteopen}{\isacharbrackleft}{\kern0pt}i{\isacharcomma}{\kern0pt}i{\isacharcomma}{\kern0pt}i{\isacharcomma}{\kern0pt}i{\isacharcomma}{\kern0pt}i{\isacharcomma}{\kern0pt}i{\isacharbrackright}{\kern0pt}\ {\isasymRightarrow}\ o{\isachardoublequoteclose}\ \isakeyword{where}\isanewline
\ \ {\isachardoublequoteopen}eq{\isacharunderscore}{\kern0pt}case{\isacharparenleft}{\kern0pt}t{\isadigit{1}}{\isacharcomma}{\kern0pt}t{\isadigit{2}}{\isacharcomma}{\kern0pt}p{\isacharcomma}{\kern0pt}P{\isacharcomma}{\kern0pt}leq{\isacharcomma}{\kern0pt}f{\isacharparenright}{\kern0pt}\ {\isasymequiv}\ {\isasymforall}s{\isachardot}{\kern0pt}\ s{\isasymin}domain{\isacharparenleft}{\kern0pt}t{\isadigit{1}}{\isacharparenright}{\kern0pt}\ {\isasymunion}\ domain{\isacharparenleft}{\kern0pt}t{\isadigit{2}}{\isacharparenright}{\kern0pt}\ {\isasymlongrightarrow}\isanewline
\ \ \ \ \ \ {\isacharparenleft}{\kern0pt}{\isasymforall}q{\isachardot}{\kern0pt}\ q{\isasymin}P\ {\isasymand}\ {\isasymlangle}q{\isacharcomma}{\kern0pt}p{\isasymrangle}{\isasymin}leq\ {\isasymlongrightarrow}\ {\isacharparenleft}{\kern0pt}f{\isacharbackquote}{\kern0pt}{\isasymlangle}{\isadigit{1}}{\isacharcomma}{\kern0pt}s{\isacharcomma}{\kern0pt}t{\isadigit{1}}{\isacharcomma}{\kern0pt}q{\isasymrangle}{\isacharequal}{\kern0pt}{\isadigit{1}}\ \ {\isasymlongleftrightarrow}\ f{\isacharbackquote}{\kern0pt}{\isasymlangle}{\isadigit{1}}{\isacharcomma}{\kern0pt}s{\isacharcomma}{\kern0pt}t{\isadigit{2}}{\isacharcomma}{\kern0pt}q{\isasymrangle}\ {\isacharequal}{\kern0pt}{\isadigit{1}}{\isacharparenright}{\kern0pt}{\isacharparenright}{\kern0pt}{\isachardoublequoteclose}\isanewline
\isanewline
\isanewline
\isacommand{definition}\isamarkupfalse%
\isanewline
\ \ is{\isacharunderscore}{\kern0pt}eq{\isacharunderscore}{\kern0pt}case\ {\isacharcolon}{\kern0pt}{\isacharcolon}{\kern0pt}\ {\isachardoublequoteopen}{\isacharbrackleft}{\kern0pt}i{\isasymRightarrow}o{\isacharcomma}{\kern0pt}i{\isacharcomma}{\kern0pt}i{\isacharcomma}{\kern0pt}i{\isacharcomma}{\kern0pt}i{\isacharcomma}{\kern0pt}i{\isacharcomma}{\kern0pt}i{\isacharbrackright}{\kern0pt}\ {\isasymRightarrow}\ o{\isachardoublequoteclose}\ \isakeyword{where}\isanewline
\ \ {\isachardoublequoteopen}is{\isacharunderscore}{\kern0pt}eq{\isacharunderscore}{\kern0pt}case{\isacharparenleft}{\kern0pt}M{\isacharcomma}{\kern0pt}t{\isadigit{1}}{\isacharcomma}{\kern0pt}t{\isadigit{2}}{\isacharcomma}{\kern0pt}p{\isacharcomma}{\kern0pt}P{\isacharcomma}{\kern0pt}leq{\isacharcomma}{\kern0pt}f{\isacharparenright}{\kern0pt}\ {\isasymequiv}\isanewline
\ \ \ {\isasymforall}s{\isacharbrackleft}{\kern0pt}M{\isacharbrackright}{\kern0pt}{\isachardot}{\kern0pt}\ {\isacharparenleft}{\kern0pt}{\isasymexists}d{\isacharbrackleft}{\kern0pt}M{\isacharbrackright}{\kern0pt}{\isachardot}{\kern0pt}\ is{\isacharunderscore}{\kern0pt}domain{\isacharparenleft}{\kern0pt}M{\isacharcomma}{\kern0pt}t{\isadigit{1}}{\isacharcomma}{\kern0pt}d{\isacharparenright}{\kern0pt}\ {\isasymand}\ s{\isasymin}d{\isacharparenright}{\kern0pt}\ {\isasymor}\ {\isacharparenleft}{\kern0pt}{\isasymexists}d{\isacharbrackleft}{\kern0pt}M{\isacharbrackright}{\kern0pt}{\isachardot}{\kern0pt}\ is{\isacharunderscore}{\kern0pt}domain{\isacharparenleft}{\kern0pt}M{\isacharcomma}{\kern0pt}t{\isadigit{2}}{\isacharcomma}{\kern0pt}d{\isacharparenright}{\kern0pt}\ {\isasymand}\ s{\isasymin}d{\isacharparenright}{\kern0pt}\isanewline
\ \ \ \ \ \ \ {\isasymlongrightarrow}\ {\isacharparenleft}{\kern0pt}{\isasymforall}q{\isacharbrackleft}{\kern0pt}M{\isacharbrackright}{\kern0pt}{\isachardot}{\kern0pt}\ q{\isasymin}P\ {\isasymand}\ {\isacharparenleft}{\kern0pt}{\isasymexists}qp{\isacharbrackleft}{\kern0pt}M{\isacharbrackright}{\kern0pt}{\isachardot}{\kern0pt}\ pair{\isacharparenleft}{\kern0pt}M{\isacharcomma}{\kern0pt}q{\isacharcomma}{\kern0pt}p{\isacharcomma}{\kern0pt}qp{\isacharparenright}{\kern0pt}\ {\isasymand}\ qp{\isasymin}leq{\isacharparenright}{\kern0pt}\ {\isasymlongrightarrow}\isanewline
\ \ \ \ \ \ \ \ \ \ \ \ {\isacharparenleft}{\kern0pt}{\isasymexists}ost{\isadigit{1}}q{\isacharbrackleft}{\kern0pt}M{\isacharbrackright}{\kern0pt}{\isachardot}{\kern0pt}\ {\isasymexists}ost{\isadigit{2}}q{\isacharbrackleft}{\kern0pt}M{\isacharbrackright}{\kern0pt}{\isachardot}{\kern0pt}\ {\isasymexists}o{\isacharbrackleft}{\kern0pt}M{\isacharbrackright}{\kern0pt}{\isachardot}{\kern0pt}\ \ {\isasymexists}vf{\isadigit{1}}{\isacharbrackleft}{\kern0pt}M{\isacharbrackright}{\kern0pt}{\isachardot}{\kern0pt}\ {\isasymexists}vf{\isadigit{2}}{\isacharbrackleft}{\kern0pt}M{\isacharbrackright}{\kern0pt}{\isachardot}{\kern0pt}\isanewline
\ \ \ \ \ \ \ \ \ \ \ \ \ is{\isacharunderscore}{\kern0pt}tuple{\isacharparenleft}{\kern0pt}M{\isacharcomma}{\kern0pt}o{\isacharcomma}{\kern0pt}s{\isacharcomma}{\kern0pt}t{\isadigit{1}}{\isacharcomma}{\kern0pt}q{\isacharcomma}{\kern0pt}ost{\isadigit{1}}q{\isacharparenright}{\kern0pt}\ {\isasymand}\isanewline
\ \ \ \ \ \ \ \ \ \ \ \ \ is{\isacharunderscore}{\kern0pt}tuple{\isacharparenleft}{\kern0pt}M{\isacharcomma}{\kern0pt}o{\isacharcomma}{\kern0pt}s{\isacharcomma}{\kern0pt}t{\isadigit{2}}{\isacharcomma}{\kern0pt}q{\isacharcomma}{\kern0pt}ost{\isadigit{2}}q{\isacharparenright}{\kern0pt}\ {\isasymand}\ number{\isadigit{1}}{\isacharparenleft}{\kern0pt}M{\isacharcomma}{\kern0pt}o{\isacharparenright}{\kern0pt}\ {\isasymand}\isanewline
\ \ \ \ \ \ \ \ \ \ \ \ \ fun{\isacharunderscore}{\kern0pt}apply{\isacharparenleft}{\kern0pt}M{\isacharcomma}{\kern0pt}f{\isacharcomma}{\kern0pt}ost{\isadigit{1}}q{\isacharcomma}{\kern0pt}vf{\isadigit{1}}{\isacharparenright}{\kern0pt}\ {\isasymand}\ fun{\isacharunderscore}{\kern0pt}apply{\isacharparenleft}{\kern0pt}M{\isacharcomma}{\kern0pt}f{\isacharcomma}{\kern0pt}ost{\isadigit{2}}q{\isacharcomma}{\kern0pt}vf{\isadigit{2}}{\isacharparenright}{\kern0pt}\ {\isasymand}\isanewline
\ \ \ \ \ \ \ \ \ \ \ \ \ {\isacharparenleft}{\kern0pt}vf{\isadigit{1}}\ {\isacharequal}{\kern0pt}\ o\ {\isasymlongleftrightarrow}\ vf{\isadigit{2}}\ {\isacharequal}{\kern0pt}\ o{\isacharparenright}{\kern0pt}{\isacharparenright}{\kern0pt}{\isacharparenright}{\kern0pt}{\isachardoublequoteclose}\isanewline
\isanewline
\isanewline
\isacommand{definition}\isamarkupfalse%
\isanewline
\ \ mem{\isacharunderscore}{\kern0pt}case\ {\isacharcolon}{\kern0pt}{\isacharcolon}{\kern0pt}\ {\isachardoublequoteopen}{\isacharbrackleft}{\kern0pt}i{\isacharcomma}{\kern0pt}i{\isacharcomma}{\kern0pt}i{\isacharcomma}{\kern0pt}i{\isacharcomma}{\kern0pt}i{\isacharcomma}{\kern0pt}i{\isacharbrackright}{\kern0pt}\ {\isasymRightarrow}\ o{\isachardoublequoteclose}\ \isakeyword{where}\isanewline
\ \ {\isachardoublequoteopen}mem{\isacharunderscore}{\kern0pt}case{\isacharparenleft}{\kern0pt}t{\isadigit{1}}{\isacharcomma}{\kern0pt}t{\isadigit{2}}{\isacharcomma}{\kern0pt}p{\isacharcomma}{\kern0pt}P{\isacharcomma}{\kern0pt}leq{\isacharcomma}{\kern0pt}f{\isacharparenright}{\kern0pt}\ {\isasymequiv}\ {\isasymforall}v{\isasymin}P{\isachardot}{\kern0pt}\ {\isasymlangle}v{\isacharcomma}{\kern0pt}p{\isasymrangle}{\isasymin}leq\ {\isasymlongrightarrow}\isanewline
\ \ \ \ {\isacharparenleft}{\kern0pt}{\isasymexists}q{\isachardot}{\kern0pt}\ {\isasymexists}s{\isachardot}{\kern0pt}\ {\isasymexists}r{\isachardot}{\kern0pt}\ r{\isasymin}P\ {\isasymand}\ q{\isasymin}P\ {\isasymand}\ {\isasymlangle}q{\isacharcomma}{\kern0pt}v{\isasymrangle}{\isasymin}leq\ {\isasymand}\ {\isasymlangle}s{\isacharcomma}{\kern0pt}r{\isasymrangle}\ {\isasymin}\ t{\isadigit{2}}\ {\isasymand}\ {\isasymlangle}q{\isacharcomma}{\kern0pt}r{\isasymrangle}{\isasymin}leq\ {\isasymand}\ \ f{\isacharbackquote}{\kern0pt}{\isasymlangle}{\isadigit{0}}{\isacharcomma}{\kern0pt}t{\isadigit{1}}{\isacharcomma}{\kern0pt}s{\isacharcomma}{\kern0pt}q{\isasymrangle}\ {\isacharequal}{\kern0pt}\ {\isadigit{1}}{\isacharparenright}{\kern0pt}{\isachardoublequoteclose}\isanewline
\isanewline
\isacommand{definition}\isamarkupfalse%
\isanewline
\ \ is{\isacharunderscore}{\kern0pt}mem{\isacharunderscore}{\kern0pt}case\ {\isacharcolon}{\kern0pt}{\isacharcolon}{\kern0pt}\ {\isachardoublequoteopen}{\isacharbrackleft}{\kern0pt}i{\isasymRightarrow}o{\isacharcomma}{\kern0pt}i{\isacharcomma}{\kern0pt}i{\isacharcomma}{\kern0pt}i{\isacharcomma}{\kern0pt}i{\isacharcomma}{\kern0pt}i{\isacharcomma}{\kern0pt}i{\isacharbrackright}{\kern0pt}\ {\isasymRightarrow}\ o{\isachardoublequoteclose}\ \isakeyword{where}\isanewline
\ \ {\isachardoublequoteopen}is{\isacharunderscore}{\kern0pt}mem{\isacharunderscore}{\kern0pt}case{\isacharparenleft}{\kern0pt}M{\isacharcomma}{\kern0pt}t{\isadigit{1}}{\isacharcomma}{\kern0pt}t{\isadigit{2}}{\isacharcomma}{\kern0pt}p{\isacharcomma}{\kern0pt}P{\isacharcomma}{\kern0pt}leq{\isacharcomma}{\kern0pt}f{\isacharparenright}{\kern0pt}\ {\isasymequiv}\ {\isasymforall}v{\isacharbrackleft}{\kern0pt}M{\isacharbrackright}{\kern0pt}{\isachardot}{\kern0pt}\ {\isasymforall}vp{\isacharbrackleft}{\kern0pt}M{\isacharbrackright}{\kern0pt}{\isachardot}{\kern0pt}\ v{\isasymin}P\ {\isasymand}\ pair{\isacharparenleft}{\kern0pt}M{\isacharcomma}{\kern0pt}v{\isacharcomma}{\kern0pt}p{\isacharcomma}{\kern0pt}vp{\isacharparenright}{\kern0pt}\ {\isasymand}\ vp{\isasymin}leq\ {\isasymlongrightarrow}\isanewline
\ \ \ \ {\isacharparenleft}{\kern0pt}{\isasymexists}q{\isacharbrackleft}{\kern0pt}M{\isacharbrackright}{\kern0pt}{\isachardot}{\kern0pt}\ {\isasymexists}s{\isacharbrackleft}{\kern0pt}M{\isacharbrackright}{\kern0pt}{\isachardot}{\kern0pt}\ {\isasymexists}r{\isacharbrackleft}{\kern0pt}M{\isacharbrackright}{\kern0pt}{\isachardot}{\kern0pt}\ {\isasymexists}qv{\isacharbrackleft}{\kern0pt}M{\isacharbrackright}{\kern0pt}{\isachardot}{\kern0pt}\ {\isasymexists}sr{\isacharbrackleft}{\kern0pt}M{\isacharbrackright}{\kern0pt}{\isachardot}{\kern0pt}\ {\isasymexists}qr{\isacharbrackleft}{\kern0pt}M{\isacharbrackright}{\kern0pt}{\isachardot}{\kern0pt}\ {\isasymexists}z{\isacharbrackleft}{\kern0pt}M{\isacharbrackright}{\kern0pt}{\isachardot}{\kern0pt}\ {\isasymexists}zt{\isadigit{1}}sq{\isacharbrackleft}{\kern0pt}M{\isacharbrackright}{\kern0pt}{\isachardot}{\kern0pt}\ {\isasymexists}o{\isacharbrackleft}{\kern0pt}M{\isacharbrackright}{\kern0pt}{\isachardot}{\kern0pt}\isanewline
\ \ \ \ \ r{\isasymin}\ P\ {\isasymand}\ q{\isasymin}P\ {\isasymand}\ pair{\isacharparenleft}{\kern0pt}M{\isacharcomma}{\kern0pt}q{\isacharcomma}{\kern0pt}v{\isacharcomma}{\kern0pt}qv{\isacharparenright}{\kern0pt}\ {\isasymand}\ pair{\isacharparenleft}{\kern0pt}M{\isacharcomma}{\kern0pt}s{\isacharcomma}{\kern0pt}r{\isacharcomma}{\kern0pt}sr{\isacharparenright}{\kern0pt}\ {\isasymand}\ pair{\isacharparenleft}{\kern0pt}M{\isacharcomma}{\kern0pt}q{\isacharcomma}{\kern0pt}r{\isacharcomma}{\kern0pt}qr{\isacharparenright}{\kern0pt}\ {\isasymand}\isanewline
\ \ \ \ \ empty{\isacharparenleft}{\kern0pt}M{\isacharcomma}{\kern0pt}z{\isacharparenright}{\kern0pt}\ {\isasymand}\ is{\isacharunderscore}{\kern0pt}tuple{\isacharparenleft}{\kern0pt}M{\isacharcomma}{\kern0pt}z{\isacharcomma}{\kern0pt}t{\isadigit{1}}{\isacharcomma}{\kern0pt}s{\isacharcomma}{\kern0pt}q{\isacharcomma}{\kern0pt}zt{\isadigit{1}}sq{\isacharparenright}{\kern0pt}\ {\isasymand}\isanewline
\ \ \ \ \ number{\isadigit{1}}{\isacharparenleft}{\kern0pt}M{\isacharcomma}{\kern0pt}o{\isacharparenright}{\kern0pt}\ {\isasymand}\ qv{\isasymin}leq\ {\isasymand}\ sr{\isasymin}t{\isadigit{2}}\ {\isasymand}\ qr{\isasymin}leq\ {\isasymand}\ fun{\isacharunderscore}{\kern0pt}apply{\isacharparenleft}{\kern0pt}M{\isacharcomma}{\kern0pt}f{\isacharcomma}{\kern0pt}zt{\isadigit{1}}sq{\isacharcomma}{\kern0pt}o{\isacharparenright}{\kern0pt}{\isacharparenright}{\kern0pt}{\isachardoublequoteclose}\isanewline
\isanewline
\isanewline
\isacommand{schematic{\isacharunderscore}{\kern0pt}goal}\isamarkupfalse%
\ sats{\isacharunderscore}{\kern0pt}is{\isacharunderscore}{\kern0pt}mem{\isacharunderscore}{\kern0pt}case{\isacharunderscore}{\kern0pt}fm{\isacharunderscore}{\kern0pt}auto{\isacharcolon}{\kern0pt}\isanewline
\ \ \isakeyword{assumes}\isanewline
\ \ \ \ {\isachardoublequoteopen}n{\isadigit{1}}{\isasymin}nat{\isachardoublequoteclose}\ {\isachardoublequoteopen}n{\isadigit{2}}{\isasymin}nat{\isachardoublequoteclose}\ {\isachardoublequoteopen}p{\isasymin}nat{\isachardoublequoteclose}\ {\isachardoublequoteopen}P{\isasymin}nat{\isachardoublequoteclose}\ {\isachardoublequoteopen}leq{\isasymin}nat{\isachardoublequoteclose}\ {\isachardoublequoteopen}f{\isasymin}nat{\isachardoublequoteclose}\ {\isachardoublequoteopen}env{\isasymin}list{\isacharparenleft}{\kern0pt}A{\isacharparenright}{\kern0pt}{\isachardoublequoteclose}\isanewline
\ \ \isakeyword{shows}\isanewline
\ \ \ \ {\isachardoublequoteopen}is{\isacharunderscore}{\kern0pt}mem{\isacharunderscore}{\kern0pt}case{\isacharparenleft}{\kern0pt}{\isacharhash}{\kern0pt}{\isacharhash}{\kern0pt}A{\isacharcomma}{\kern0pt}\ nth{\isacharparenleft}{\kern0pt}n{\isadigit{1}}{\isacharcomma}{\kern0pt}\ env{\isacharparenright}{\kern0pt}{\isacharcomma}{\kern0pt}nth{\isacharparenleft}{\kern0pt}n{\isadigit{2}}{\isacharcomma}{\kern0pt}\ env{\isacharparenright}{\kern0pt}{\isacharcomma}{\kern0pt}nth{\isacharparenleft}{\kern0pt}p{\isacharcomma}{\kern0pt}\ env{\isacharparenright}{\kern0pt}{\isacharcomma}{\kern0pt}nth{\isacharparenleft}{\kern0pt}P{\isacharcomma}{\kern0pt}\ env{\isacharparenright}{\kern0pt}{\isacharcomma}{\kern0pt}\ nth{\isacharparenleft}{\kern0pt}leq{\isacharcomma}{\kern0pt}\ env{\isacharparenright}{\kern0pt}{\isacharcomma}{\kern0pt}nth{\isacharparenleft}{\kern0pt}f{\isacharcomma}{\kern0pt}env{\isacharparenright}{\kern0pt}{\isacharparenright}{\kern0pt}\isanewline
\ \ \ \ {\isasymlongleftrightarrow}\ sats{\isacharparenleft}{\kern0pt}A{\isacharcomma}{\kern0pt}{\isacharquery}{\kern0pt}imc{\isacharunderscore}{\kern0pt}fm{\isacharparenleft}{\kern0pt}n{\isadigit{1}}{\isacharcomma}{\kern0pt}n{\isadigit{2}}{\isacharcomma}{\kern0pt}p{\isacharcomma}{\kern0pt}P{\isacharcomma}{\kern0pt}leq{\isacharcomma}{\kern0pt}f{\isacharparenright}{\kern0pt}{\isacharcomma}{\kern0pt}env{\isacharparenright}{\kern0pt}{\isachardoublequoteclose}\isanewline
%
\isadelimproof
\ \ %
\endisadelimproof
%
\isatagproof
\isacommand{unfolding}\isamarkupfalse%
\ is{\isacharunderscore}{\kern0pt}mem{\isacharunderscore}{\kern0pt}case{\isacharunderscore}{\kern0pt}def\isanewline
\ \ \isacommand{by}\isamarkupfalse%
\ {\isacharparenleft}{\kern0pt}insert\ assms\ {\isacharsemicolon}{\kern0pt}\ {\isacharparenleft}{\kern0pt}rule\ sep{\isacharunderscore}{\kern0pt}rules{\isacharprime}{\kern0pt}\ \ is{\isacharunderscore}{\kern0pt}tuple{\isacharunderscore}{\kern0pt}iff{\isacharunderscore}{\kern0pt}sats\ {\isacharbar}{\kern0pt}\ simp{\isacharparenright}{\kern0pt}{\isacharplus}{\kern0pt}{\isacharparenright}{\kern0pt}%
\endisatagproof
{\isafoldproof}%
%
\isadelimproof
\isanewline
%
\endisadelimproof
\isanewline
%
\isadelimML
\isanewline
%
\endisadelimML
%
\isatagML
\isacommand{synthesize}\isamarkupfalse%
\ {\isachardoublequoteopen}mem{\isacharunderscore}{\kern0pt}case{\isacharunderscore}{\kern0pt}fm{\isachardoublequoteclose}\ \isakeyword{from{\isacharunderscore}{\kern0pt}schematic}\ sats{\isacharunderscore}{\kern0pt}is{\isacharunderscore}{\kern0pt}mem{\isacharunderscore}{\kern0pt}case{\isacharunderscore}{\kern0pt}fm{\isacharunderscore}{\kern0pt}auto%
\endisatagML
{\isafoldML}%
%
\isadelimML
\isanewline
%
\endisadelimML
\isanewline
\isacommand{lemma}\isamarkupfalse%
\ arity{\isacharunderscore}{\kern0pt}mem{\isacharunderscore}{\kern0pt}case{\isacharunderscore}{\kern0pt}fm\ {\isacharcolon}{\kern0pt}\isanewline
\ \ \isakeyword{assumes}\isanewline
\ \ \ \ {\isachardoublequoteopen}n{\isadigit{1}}{\isasymin}nat{\isachardoublequoteclose}\ {\isachardoublequoteopen}n{\isadigit{2}}{\isasymin}nat{\isachardoublequoteclose}\ {\isachardoublequoteopen}p{\isasymin}nat{\isachardoublequoteclose}\ {\isachardoublequoteopen}P{\isasymin}nat{\isachardoublequoteclose}\ {\isachardoublequoteopen}leq{\isasymin}nat{\isachardoublequoteclose}\ {\isachardoublequoteopen}f{\isasymin}nat{\isachardoublequoteclose}\isanewline
\ \ \isakeyword{shows}\isanewline
\ \ \ \ {\isachardoublequoteopen}arity{\isacharparenleft}{\kern0pt}mem{\isacharunderscore}{\kern0pt}case{\isacharunderscore}{\kern0pt}fm{\isacharparenleft}{\kern0pt}n{\isadigit{1}}{\isacharcomma}{\kern0pt}n{\isadigit{2}}{\isacharcomma}{\kern0pt}p{\isacharcomma}{\kern0pt}P{\isacharcomma}{\kern0pt}leq{\isacharcomma}{\kern0pt}f{\isacharparenright}{\kern0pt}{\isacharparenright}{\kern0pt}\ {\isacharequal}{\kern0pt}\isanewline
\ \ \ \ succ{\isacharparenleft}{\kern0pt}n{\isadigit{1}}{\isacharparenright}{\kern0pt}\ {\isasymunion}\ succ{\isacharparenleft}{\kern0pt}n{\isadigit{2}}{\isacharparenright}{\kern0pt}\ {\isasymunion}\ succ{\isacharparenleft}{\kern0pt}p{\isacharparenright}{\kern0pt}\ {\isasymunion}\ succ{\isacharparenleft}{\kern0pt}P{\isacharparenright}{\kern0pt}\ {\isasymunion}\ succ{\isacharparenleft}{\kern0pt}leq{\isacharparenright}{\kern0pt}\ {\isasymunion}\ succ{\isacharparenleft}{\kern0pt}f{\isacharparenright}{\kern0pt}{\isachardoublequoteclose}\isanewline
%
\isadelimproof
\ \ %
\endisadelimproof
%
\isatagproof
\isacommand{unfolding}\isamarkupfalse%
\ mem{\isacharunderscore}{\kern0pt}case{\isacharunderscore}{\kern0pt}fm{\isacharunderscore}{\kern0pt}def\isanewline
\ \ \isacommand{using}\isamarkupfalse%
\ assms\ arity{\isacharunderscore}{\kern0pt}pair{\isacharunderscore}{\kern0pt}fm\ arity{\isacharunderscore}{\kern0pt}is{\isacharunderscore}{\kern0pt}tuple{\isacharunderscore}{\kern0pt}fm\ number{\isadigit{1}}arity{\isacharunderscore}{\kern0pt}{\isacharunderscore}{\kern0pt}fm\ arity{\isacharunderscore}{\kern0pt}fun{\isacharunderscore}{\kern0pt}apply{\isacharunderscore}{\kern0pt}fm\ arity{\isacharunderscore}{\kern0pt}empty{\isacharunderscore}{\kern0pt}fm\isanewline
\ \ \ \ pred{\isacharunderscore}{\kern0pt}Un{\isacharunderscore}{\kern0pt}distrib\isanewline
\ \ \isacommand{by}\isamarkupfalse%
\ auto%
\endisatagproof
{\isafoldproof}%
%
\isadelimproof
\isanewline
%
\endisadelimproof
\isanewline
\isacommand{schematic{\isacharunderscore}{\kern0pt}goal}\isamarkupfalse%
\ sats{\isacharunderscore}{\kern0pt}is{\isacharunderscore}{\kern0pt}eq{\isacharunderscore}{\kern0pt}case{\isacharunderscore}{\kern0pt}fm{\isacharunderscore}{\kern0pt}auto{\isacharcolon}{\kern0pt}\isanewline
\ \ \isakeyword{assumes}\isanewline
\ \ \ \ {\isachardoublequoteopen}n{\isadigit{1}}{\isasymin}nat{\isachardoublequoteclose}\ {\isachardoublequoteopen}n{\isadigit{2}}{\isasymin}nat{\isachardoublequoteclose}\ {\isachardoublequoteopen}p{\isasymin}nat{\isachardoublequoteclose}\ {\isachardoublequoteopen}P{\isasymin}nat{\isachardoublequoteclose}\ {\isachardoublequoteopen}leq{\isasymin}nat{\isachardoublequoteclose}\ {\isachardoublequoteopen}f{\isasymin}nat{\isachardoublequoteclose}\ {\isachardoublequoteopen}env{\isasymin}list{\isacharparenleft}{\kern0pt}A{\isacharparenright}{\kern0pt}{\isachardoublequoteclose}\isanewline
\ \ \isakeyword{shows}\isanewline
\ \ \ \ {\isachardoublequoteopen}is{\isacharunderscore}{\kern0pt}eq{\isacharunderscore}{\kern0pt}case{\isacharparenleft}{\kern0pt}{\isacharhash}{\kern0pt}{\isacharhash}{\kern0pt}A{\isacharcomma}{\kern0pt}\ nth{\isacharparenleft}{\kern0pt}n{\isadigit{1}}{\isacharcomma}{\kern0pt}\ env{\isacharparenright}{\kern0pt}{\isacharcomma}{\kern0pt}nth{\isacharparenleft}{\kern0pt}n{\isadigit{2}}{\isacharcomma}{\kern0pt}\ env{\isacharparenright}{\kern0pt}{\isacharcomma}{\kern0pt}nth{\isacharparenleft}{\kern0pt}p{\isacharcomma}{\kern0pt}\ env{\isacharparenright}{\kern0pt}{\isacharcomma}{\kern0pt}nth{\isacharparenleft}{\kern0pt}P{\isacharcomma}{\kern0pt}\ env{\isacharparenright}{\kern0pt}{\isacharcomma}{\kern0pt}\ nth{\isacharparenleft}{\kern0pt}leq{\isacharcomma}{\kern0pt}\ env{\isacharparenright}{\kern0pt}{\isacharcomma}{\kern0pt}nth{\isacharparenleft}{\kern0pt}f{\isacharcomma}{\kern0pt}env{\isacharparenright}{\kern0pt}{\isacharparenright}{\kern0pt}\isanewline
\ \ \ \ {\isasymlongleftrightarrow}\ sats{\isacharparenleft}{\kern0pt}A{\isacharcomma}{\kern0pt}{\isacharquery}{\kern0pt}iec{\isacharunderscore}{\kern0pt}fm{\isacharparenleft}{\kern0pt}n{\isadigit{1}}{\isacharcomma}{\kern0pt}n{\isadigit{2}}{\isacharcomma}{\kern0pt}p{\isacharcomma}{\kern0pt}P{\isacharcomma}{\kern0pt}leq{\isacharcomma}{\kern0pt}f{\isacharparenright}{\kern0pt}{\isacharcomma}{\kern0pt}env{\isacharparenright}{\kern0pt}{\isachardoublequoteclose}\isanewline
%
\isadelimproof
\ \ %
\endisadelimproof
%
\isatagproof
\isacommand{unfolding}\isamarkupfalse%
\ is{\isacharunderscore}{\kern0pt}eq{\isacharunderscore}{\kern0pt}case{\isacharunderscore}{\kern0pt}def\isanewline
\ \ \isacommand{by}\isamarkupfalse%
\ {\isacharparenleft}{\kern0pt}insert\ assms\ {\isacharsemicolon}{\kern0pt}\ {\isacharparenleft}{\kern0pt}rule\ sep{\isacharunderscore}{\kern0pt}rules{\isacharprime}{\kern0pt}\ \ is{\isacharunderscore}{\kern0pt}tuple{\isacharunderscore}{\kern0pt}iff{\isacharunderscore}{\kern0pt}sats\ {\isacharbar}{\kern0pt}\ simp{\isacharparenright}{\kern0pt}{\isacharplus}{\kern0pt}{\isacharparenright}{\kern0pt}%
\endisatagproof
{\isafoldproof}%
%
\isadelimproof
\isanewline
%
\endisadelimproof
%
\isadelimML
\isanewline
%
\endisadelimML
%
\isatagML
\isacommand{synthesize}\isamarkupfalse%
\ {\isachardoublequoteopen}eq{\isacharunderscore}{\kern0pt}case{\isacharunderscore}{\kern0pt}fm{\isachardoublequoteclose}\ \isakeyword{from{\isacharunderscore}{\kern0pt}schematic}\ sats{\isacharunderscore}{\kern0pt}is{\isacharunderscore}{\kern0pt}eq{\isacharunderscore}{\kern0pt}case{\isacharunderscore}{\kern0pt}fm{\isacharunderscore}{\kern0pt}auto%
\endisatagML
{\isafoldML}%
%
\isadelimML
\isanewline
%
\endisadelimML
\isanewline
\isacommand{lemma}\isamarkupfalse%
\ arity{\isacharunderscore}{\kern0pt}eq{\isacharunderscore}{\kern0pt}case{\isacharunderscore}{\kern0pt}fm\ {\isacharcolon}{\kern0pt}\isanewline
\ \ \isakeyword{assumes}\isanewline
\ \ \ \ {\isachardoublequoteopen}n{\isadigit{1}}{\isasymin}nat{\isachardoublequoteclose}\ {\isachardoublequoteopen}n{\isadigit{2}}{\isasymin}nat{\isachardoublequoteclose}\ {\isachardoublequoteopen}p{\isasymin}nat{\isachardoublequoteclose}\ {\isachardoublequoteopen}P{\isasymin}nat{\isachardoublequoteclose}\ {\isachardoublequoteopen}leq{\isasymin}nat{\isachardoublequoteclose}\ {\isachardoublequoteopen}f{\isasymin}nat{\isachardoublequoteclose}\isanewline
\ \ \isakeyword{shows}\isanewline
\ \ \ \ {\isachardoublequoteopen}arity{\isacharparenleft}{\kern0pt}eq{\isacharunderscore}{\kern0pt}case{\isacharunderscore}{\kern0pt}fm{\isacharparenleft}{\kern0pt}n{\isadigit{1}}{\isacharcomma}{\kern0pt}n{\isadigit{2}}{\isacharcomma}{\kern0pt}p{\isacharcomma}{\kern0pt}P{\isacharcomma}{\kern0pt}leq{\isacharcomma}{\kern0pt}f{\isacharparenright}{\kern0pt}{\isacharparenright}{\kern0pt}\ {\isacharequal}{\kern0pt}\isanewline
\ \ \ \ succ{\isacharparenleft}{\kern0pt}n{\isadigit{1}}{\isacharparenright}{\kern0pt}\ {\isasymunion}\ succ{\isacharparenleft}{\kern0pt}n{\isadigit{2}}{\isacharparenright}{\kern0pt}\ {\isasymunion}\ succ{\isacharparenleft}{\kern0pt}p{\isacharparenright}{\kern0pt}\ {\isasymunion}\ succ{\isacharparenleft}{\kern0pt}P{\isacharparenright}{\kern0pt}\ {\isasymunion}\ succ{\isacharparenleft}{\kern0pt}leq{\isacharparenright}{\kern0pt}\ {\isasymunion}\ succ{\isacharparenleft}{\kern0pt}f{\isacharparenright}{\kern0pt}{\isachardoublequoteclose}\isanewline
%
\isadelimproof
\ \ %
\endisadelimproof
%
\isatagproof
\isacommand{unfolding}\isamarkupfalse%
\ eq{\isacharunderscore}{\kern0pt}case{\isacharunderscore}{\kern0pt}fm{\isacharunderscore}{\kern0pt}def\isanewline
\ \ \isacommand{using}\isamarkupfalse%
\ assms\ arity{\isacharunderscore}{\kern0pt}pair{\isacharunderscore}{\kern0pt}fm\ arity{\isacharunderscore}{\kern0pt}is{\isacharunderscore}{\kern0pt}tuple{\isacharunderscore}{\kern0pt}fm\ number{\isadigit{1}}arity{\isacharunderscore}{\kern0pt}{\isacharunderscore}{\kern0pt}fm\ arity{\isacharunderscore}{\kern0pt}fun{\isacharunderscore}{\kern0pt}apply{\isacharunderscore}{\kern0pt}fm\ arity{\isacharunderscore}{\kern0pt}empty{\isacharunderscore}{\kern0pt}fm\isanewline
\ \ \ \ arity{\isacharunderscore}{\kern0pt}domain{\isacharunderscore}{\kern0pt}fm\ pred{\isacharunderscore}{\kern0pt}Un{\isacharunderscore}{\kern0pt}distrib\isanewline
\ \ \isacommand{by}\isamarkupfalse%
\ auto%
\endisatagproof
{\isafoldproof}%
%
\isadelimproof
\isanewline
%
\endisadelimproof
\isanewline
\isacommand{definition}\isamarkupfalse%
\isanewline
\ \ Hfrc\ {\isacharcolon}{\kern0pt}{\isacharcolon}{\kern0pt}\ {\isachardoublequoteopen}{\isacharbrackleft}{\kern0pt}i{\isacharcomma}{\kern0pt}i{\isacharcomma}{\kern0pt}i{\isacharcomma}{\kern0pt}i{\isacharbrackright}{\kern0pt}\ {\isasymRightarrow}\ o{\isachardoublequoteclose}\ \isakeyword{where}\isanewline
\ \ {\isachardoublequoteopen}Hfrc{\isacharparenleft}{\kern0pt}P{\isacharcomma}{\kern0pt}leq{\isacharcomma}{\kern0pt}fnnc{\isacharcomma}{\kern0pt}f{\isacharparenright}{\kern0pt}\ {\isasymequiv}\ {\isasymexists}ft{\isachardot}{\kern0pt}\ {\isasymexists}n{\isadigit{1}}{\isachardot}{\kern0pt}\ {\isasymexists}n{\isadigit{2}}{\isachardot}{\kern0pt}\ {\isasymexists}c{\isachardot}{\kern0pt}\ c{\isasymin}P\ {\isasymand}\ fnnc\ {\isacharequal}{\kern0pt}\ {\isasymlangle}ft{\isacharcomma}{\kern0pt}n{\isadigit{1}}{\isacharcomma}{\kern0pt}n{\isadigit{2}}{\isacharcomma}{\kern0pt}c{\isasymrangle}\ {\isasymand}\isanewline
\ \ \ \ \ {\isacharparenleft}{\kern0pt}\ \ ft\ {\isacharequal}{\kern0pt}\ {\isadigit{0}}\ {\isasymand}\ \ eq{\isacharunderscore}{\kern0pt}case{\isacharparenleft}{\kern0pt}n{\isadigit{1}}{\isacharcomma}{\kern0pt}n{\isadigit{2}}{\isacharcomma}{\kern0pt}c{\isacharcomma}{\kern0pt}P{\isacharcomma}{\kern0pt}leq{\isacharcomma}{\kern0pt}f{\isacharparenright}{\kern0pt}\isanewline
\ \ \ \ \ \ {\isasymor}\ ft\ {\isacharequal}{\kern0pt}\ {\isadigit{1}}\ {\isasymand}\ mem{\isacharunderscore}{\kern0pt}case{\isacharparenleft}{\kern0pt}n{\isadigit{1}}{\isacharcomma}{\kern0pt}n{\isadigit{2}}{\isacharcomma}{\kern0pt}c{\isacharcomma}{\kern0pt}P{\isacharcomma}{\kern0pt}leq{\isacharcomma}{\kern0pt}f{\isacharparenright}{\kern0pt}{\isacharparenright}{\kern0pt}{\isachardoublequoteclose}\isanewline
\isanewline
\isacommand{definition}\isamarkupfalse%
\isanewline
\ \ is{\isacharunderscore}{\kern0pt}Hfrc\ {\isacharcolon}{\kern0pt}{\isacharcolon}{\kern0pt}\ {\isachardoublequoteopen}{\isacharbrackleft}{\kern0pt}i{\isasymRightarrow}o{\isacharcomma}{\kern0pt}i{\isacharcomma}{\kern0pt}i{\isacharcomma}{\kern0pt}i{\isacharcomma}{\kern0pt}i{\isacharbrackright}{\kern0pt}\ {\isasymRightarrow}\ o{\isachardoublequoteclose}\ \isakeyword{where}\isanewline
\ \ {\isachardoublequoteopen}is{\isacharunderscore}{\kern0pt}Hfrc{\isacharparenleft}{\kern0pt}M{\isacharcomma}{\kern0pt}P{\isacharcomma}{\kern0pt}leq{\isacharcomma}{\kern0pt}fnnc{\isacharcomma}{\kern0pt}f{\isacharparenright}{\kern0pt}\ {\isasymequiv}\isanewline
\ \ \ \ \ {\isasymexists}ft{\isacharbrackleft}{\kern0pt}M{\isacharbrackright}{\kern0pt}{\isachardot}{\kern0pt}\ {\isasymexists}n{\isadigit{1}}{\isacharbrackleft}{\kern0pt}M{\isacharbrackright}{\kern0pt}{\isachardot}{\kern0pt}\ {\isasymexists}n{\isadigit{2}}{\isacharbrackleft}{\kern0pt}M{\isacharbrackright}{\kern0pt}{\isachardot}{\kern0pt}\ {\isasymexists}co{\isacharbrackleft}{\kern0pt}M{\isacharbrackright}{\kern0pt}{\isachardot}{\kern0pt}\isanewline
\ \ \ \ \ \ co{\isasymin}P\ {\isasymand}\ is{\isacharunderscore}{\kern0pt}tuple{\isacharparenleft}{\kern0pt}M{\isacharcomma}{\kern0pt}ft{\isacharcomma}{\kern0pt}n{\isadigit{1}}{\isacharcomma}{\kern0pt}n{\isadigit{2}}{\isacharcomma}{\kern0pt}co{\isacharcomma}{\kern0pt}fnnc{\isacharparenright}{\kern0pt}\ {\isasymand}\isanewline
\ \ \ \ \ \ {\isacharparenleft}{\kern0pt}\ \ {\isacharparenleft}{\kern0pt}empty{\isacharparenleft}{\kern0pt}M{\isacharcomma}{\kern0pt}ft{\isacharparenright}{\kern0pt}\ {\isasymand}\ is{\isacharunderscore}{\kern0pt}eq{\isacharunderscore}{\kern0pt}case{\isacharparenleft}{\kern0pt}M{\isacharcomma}{\kern0pt}n{\isadigit{1}}{\isacharcomma}{\kern0pt}n{\isadigit{2}}{\isacharcomma}{\kern0pt}co{\isacharcomma}{\kern0pt}P{\isacharcomma}{\kern0pt}leq{\isacharcomma}{\kern0pt}f{\isacharparenright}{\kern0pt}{\isacharparenright}{\kern0pt}\isanewline
\ \ \ \ \ \ \ {\isasymor}\ {\isacharparenleft}{\kern0pt}number{\isadigit{1}}{\isacharparenleft}{\kern0pt}M{\isacharcomma}{\kern0pt}ft{\isacharparenright}{\kern0pt}\ {\isasymand}\ \ is{\isacharunderscore}{\kern0pt}mem{\isacharunderscore}{\kern0pt}case{\isacharparenleft}{\kern0pt}M{\isacharcomma}{\kern0pt}n{\isadigit{1}}{\isacharcomma}{\kern0pt}n{\isadigit{2}}{\isacharcomma}{\kern0pt}co{\isacharcomma}{\kern0pt}P{\isacharcomma}{\kern0pt}leq{\isacharcomma}{\kern0pt}f{\isacharparenright}{\kern0pt}{\isacharparenright}{\kern0pt}{\isacharparenright}{\kern0pt}{\isachardoublequoteclose}\isanewline
\isanewline
\isacommand{definition}\isamarkupfalse%
\isanewline
\ \ Hfrc{\isacharunderscore}{\kern0pt}fm\ {\isacharcolon}{\kern0pt}{\isacharcolon}{\kern0pt}\ {\isachardoublequoteopen}{\isacharbrackleft}{\kern0pt}i{\isacharcomma}{\kern0pt}i{\isacharcomma}{\kern0pt}i{\isacharcomma}{\kern0pt}i{\isacharbrackright}{\kern0pt}\ {\isasymRightarrow}\ i{\isachardoublequoteclose}\ \isakeyword{where}\isanewline
\ \ {\isachardoublequoteopen}Hfrc{\isacharunderscore}{\kern0pt}fm{\isacharparenleft}{\kern0pt}P{\isacharcomma}{\kern0pt}leq{\isacharcomma}{\kern0pt}fnnc{\isacharcomma}{\kern0pt}f{\isacharparenright}{\kern0pt}\ {\isasymequiv}\isanewline
\ \ \ \ Exists{\isacharparenleft}{\kern0pt}Exists{\isacharparenleft}{\kern0pt}Exists{\isacharparenleft}{\kern0pt}Exists{\isacharparenleft}{\kern0pt}\isanewline
\ \ \ \ \ \ And{\isacharparenleft}{\kern0pt}Member{\isacharparenleft}{\kern0pt}{\isadigit{0}}{\isacharcomma}{\kern0pt}P\ {\isacharhash}{\kern0pt}{\isacharplus}{\kern0pt}\ {\isadigit{4}}{\isacharparenright}{\kern0pt}{\isacharcomma}{\kern0pt}And{\isacharparenleft}{\kern0pt}is{\isacharunderscore}{\kern0pt}tuple{\isacharunderscore}{\kern0pt}fm{\isacharparenleft}{\kern0pt}{\isadigit{3}}{\isacharcomma}{\kern0pt}{\isadigit{2}}{\isacharcomma}{\kern0pt}{\isadigit{1}}{\isacharcomma}{\kern0pt}{\isadigit{0}}{\isacharcomma}{\kern0pt}fnnc\ {\isacharhash}{\kern0pt}{\isacharplus}{\kern0pt}\ {\isadigit{4}}{\isacharparenright}{\kern0pt}{\isacharcomma}{\kern0pt}\isanewline
\ \ \ \ \ \ Or{\isacharparenleft}{\kern0pt}And{\isacharparenleft}{\kern0pt}empty{\isacharunderscore}{\kern0pt}fm{\isacharparenleft}{\kern0pt}{\isadigit{3}}{\isacharparenright}{\kern0pt}{\isacharcomma}{\kern0pt}eq{\isacharunderscore}{\kern0pt}case{\isacharunderscore}{\kern0pt}fm{\isacharparenleft}{\kern0pt}{\isadigit{2}}{\isacharcomma}{\kern0pt}{\isadigit{1}}{\isacharcomma}{\kern0pt}{\isadigit{0}}{\isacharcomma}{\kern0pt}P\ {\isacharhash}{\kern0pt}{\isacharplus}{\kern0pt}\ {\isadigit{4}}{\isacharcomma}{\kern0pt}leq\ {\isacharhash}{\kern0pt}{\isacharplus}{\kern0pt}\ {\isadigit{4}}{\isacharcomma}{\kern0pt}f\ {\isacharhash}{\kern0pt}{\isacharplus}{\kern0pt}\ {\isadigit{4}}{\isacharparenright}{\kern0pt}{\isacharparenright}{\kern0pt}{\isacharcomma}{\kern0pt}\isanewline
\ \ \ \ \ \ \ \ \ And{\isacharparenleft}{\kern0pt}number{\isadigit{1}}{\isacharunderscore}{\kern0pt}fm{\isacharparenleft}{\kern0pt}{\isadigit{3}}{\isacharparenright}{\kern0pt}{\isacharcomma}{\kern0pt}mem{\isacharunderscore}{\kern0pt}case{\isacharunderscore}{\kern0pt}fm{\isacharparenleft}{\kern0pt}{\isadigit{2}}{\isacharcomma}{\kern0pt}{\isadigit{1}}{\isacharcomma}{\kern0pt}{\isadigit{0}}{\isacharcomma}{\kern0pt}P\ {\isacharhash}{\kern0pt}{\isacharplus}{\kern0pt}\ {\isadigit{4}}{\isacharcomma}{\kern0pt}leq\ {\isacharhash}{\kern0pt}{\isacharplus}{\kern0pt}\ {\isadigit{4}}{\isacharcomma}{\kern0pt}f\ {\isacharhash}{\kern0pt}{\isacharplus}{\kern0pt}\ {\isadigit{4}}{\isacharparenright}{\kern0pt}{\isacharparenright}{\kern0pt}{\isacharparenright}{\kern0pt}{\isacharparenright}{\kern0pt}{\isacharparenright}{\kern0pt}{\isacharparenright}{\kern0pt}{\isacharparenright}{\kern0pt}{\isacharparenright}{\kern0pt}{\isacharparenright}{\kern0pt}{\isachardoublequoteclose}\isanewline
\isanewline
\isacommand{lemma}\isamarkupfalse%
\ Hfrc{\isacharunderscore}{\kern0pt}fm{\isacharunderscore}{\kern0pt}type{\isacharbrackleft}{\kern0pt}TC{\isacharbrackright}{\kern0pt}\ {\isacharcolon}{\kern0pt}\isanewline
\ \ {\isachardoublequoteopen}{\isasymlbrakk}P{\isasymin}nat{\isacharsemicolon}{\kern0pt}leq{\isasymin}nat{\isacharsemicolon}{\kern0pt}fnnc{\isasymin}nat{\isacharsemicolon}{\kern0pt}f{\isasymin}nat{\isasymrbrakk}\ {\isasymLongrightarrow}\ Hfrc{\isacharunderscore}{\kern0pt}fm{\isacharparenleft}{\kern0pt}P{\isacharcomma}{\kern0pt}leq{\isacharcomma}{\kern0pt}fnnc{\isacharcomma}{\kern0pt}f{\isacharparenright}{\kern0pt}{\isasymin}formula{\isachardoublequoteclose}\isanewline
%
\isadelimproof
\ \ %
\endisadelimproof
%
\isatagproof
\isacommand{unfolding}\isamarkupfalse%
\ Hfrc{\isacharunderscore}{\kern0pt}fm{\isacharunderscore}{\kern0pt}def\ \isacommand{by}\isamarkupfalse%
\ simp%
\endisatagproof
{\isafoldproof}%
%
\isadelimproof
\isanewline
%
\endisadelimproof
\isanewline
\isacommand{lemma}\isamarkupfalse%
\ arity{\isacharunderscore}{\kern0pt}Hfrc{\isacharunderscore}{\kern0pt}fm\ {\isacharcolon}{\kern0pt}\isanewline
\ \ \isakeyword{assumes}\isanewline
\ \ \ \ {\isachardoublequoteopen}P{\isasymin}nat{\isachardoublequoteclose}\ {\isachardoublequoteopen}leq{\isasymin}nat{\isachardoublequoteclose}\ {\isachardoublequoteopen}fnnc{\isasymin}nat{\isachardoublequoteclose}\ {\isachardoublequoteopen}f{\isasymin}nat{\isachardoublequoteclose}\isanewline
\ \ \isakeyword{shows}\isanewline
\ \ \ \ {\isachardoublequoteopen}arity{\isacharparenleft}{\kern0pt}Hfrc{\isacharunderscore}{\kern0pt}fm{\isacharparenleft}{\kern0pt}P{\isacharcomma}{\kern0pt}leq{\isacharcomma}{\kern0pt}fnnc{\isacharcomma}{\kern0pt}f{\isacharparenright}{\kern0pt}{\isacharparenright}{\kern0pt}\ {\isacharequal}{\kern0pt}\ succ{\isacharparenleft}{\kern0pt}P{\isacharparenright}{\kern0pt}\ {\isasymunion}\ succ{\isacharparenleft}{\kern0pt}leq{\isacharparenright}{\kern0pt}\ {\isasymunion}\ succ{\isacharparenleft}{\kern0pt}fnnc{\isacharparenright}{\kern0pt}\ {\isasymunion}\ succ{\isacharparenleft}{\kern0pt}f{\isacharparenright}{\kern0pt}{\isachardoublequoteclose}\isanewline
%
\isadelimproof
\ \ %
\endisadelimproof
%
\isatagproof
\isacommand{unfolding}\isamarkupfalse%
\ Hfrc{\isacharunderscore}{\kern0pt}fm{\isacharunderscore}{\kern0pt}def\isanewline
\ \ \isacommand{using}\isamarkupfalse%
\ assms\ arity{\isacharunderscore}{\kern0pt}is{\isacharunderscore}{\kern0pt}tuple{\isacharunderscore}{\kern0pt}fm\ arity{\isacharunderscore}{\kern0pt}mem{\isacharunderscore}{\kern0pt}case{\isacharunderscore}{\kern0pt}fm\ arity{\isacharunderscore}{\kern0pt}eq{\isacharunderscore}{\kern0pt}case{\isacharunderscore}{\kern0pt}fm\isanewline
\ \ \ \ arity{\isacharunderscore}{\kern0pt}empty{\isacharunderscore}{\kern0pt}fm\ number{\isadigit{1}}arity{\isacharunderscore}{\kern0pt}{\isacharunderscore}{\kern0pt}fm\ pred{\isacharunderscore}{\kern0pt}Un{\isacharunderscore}{\kern0pt}distrib\isanewline
\ \ \isacommand{by}\isamarkupfalse%
\ auto%
\endisatagproof
{\isafoldproof}%
%
\isadelimproof
\isanewline
%
\endisadelimproof
\isanewline
\isacommand{lemma}\isamarkupfalse%
\ sats{\isacharunderscore}{\kern0pt}Hfrc{\isacharunderscore}{\kern0pt}fm{\isacharcolon}{\kern0pt}\isanewline
\ \ \isakeyword{assumes}\isanewline
\ \ \ \ {\isachardoublequoteopen}P{\isasymin}nat{\isachardoublequoteclose}\ {\isachardoublequoteopen}leq{\isasymin}nat{\isachardoublequoteclose}\ {\isachardoublequoteopen}fnnc{\isasymin}nat{\isachardoublequoteclose}\ {\isachardoublequoteopen}f{\isasymin}nat{\isachardoublequoteclose}\ {\isachardoublequoteopen}env{\isasymin}list{\isacharparenleft}{\kern0pt}A{\isacharparenright}{\kern0pt}{\isachardoublequoteclose}\isanewline
\ \ \isakeyword{shows}\isanewline
\ \ \ \ {\isachardoublequoteopen}sats{\isacharparenleft}{\kern0pt}A{\isacharcomma}{\kern0pt}Hfrc{\isacharunderscore}{\kern0pt}fm{\isacharparenleft}{\kern0pt}P{\isacharcomma}{\kern0pt}leq{\isacharcomma}{\kern0pt}fnnc{\isacharcomma}{\kern0pt}f{\isacharparenright}{\kern0pt}{\isacharcomma}{\kern0pt}env{\isacharparenright}{\kern0pt}\isanewline
\ \ \ \ {\isasymlongleftrightarrow}\ is{\isacharunderscore}{\kern0pt}Hfrc{\isacharparenleft}{\kern0pt}{\isacharhash}{\kern0pt}{\isacharhash}{\kern0pt}A{\isacharcomma}{\kern0pt}nth{\isacharparenleft}{\kern0pt}P{\isacharcomma}{\kern0pt}\ env{\isacharparenright}{\kern0pt}{\isacharcomma}{\kern0pt}\ nth{\isacharparenleft}{\kern0pt}leq{\isacharcomma}{\kern0pt}\ env{\isacharparenright}{\kern0pt}{\isacharcomma}{\kern0pt}\ nth{\isacharparenleft}{\kern0pt}fnnc{\isacharcomma}{\kern0pt}\ env{\isacharparenright}{\kern0pt}{\isacharcomma}{\kern0pt}nth{\isacharparenleft}{\kern0pt}f{\isacharcomma}{\kern0pt}\ env{\isacharparenright}{\kern0pt}{\isacharparenright}{\kern0pt}{\isachardoublequoteclose}\isanewline
%
\isadelimproof
\ \ %
\endisadelimproof
%
\isatagproof
\isacommand{unfolding}\isamarkupfalse%
\ is{\isacharunderscore}{\kern0pt}Hfrc{\isacharunderscore}{\kern0pt}def\ Hfrc{\isacharunderscore}{\kern0pt}fm{\isacharunderscore}{\kern0pt}def\isanewline
\ \ \isacommand{using}\isamarkupfalse%
\ assms\ \ \isanewline
\ \ \isacommand{by}\isamarkupfalse%
\ {\isacharparenleft}{\kern0pt}simp\ add{\isacharcolon}{\kern0pt}\ sats{\isacharunderscore}{\kern0pt}is{\isacharunderscore}{\kern0pt}tuple{\isacharunderscore}{\kern0pt}fm\ eq{\isacharunderscore}{\kern0pt}case{\isacharunderscore}{\kern0pt}fm{\isacharunderscore}{\kern0pt}iff{\isacharunderscore}{\kern0pt}sats{\isacharbrackleft}{\kern0pt}symmetric{\isacharbrackright}{\kern0pt}\ mem{\isacharunderscore}{\kern0pt}case{\isacharunderscore}{\kern0pt}fm{\isacharunderscore}{\kern0pt}iff{\isacharunderscore}{\kern0pt}sats{\isacharbrackleft}{\kern0pt}symmetric{\isacharbrackright}{\kern0pt}{\isacharparenright}{\kern0pt}%
\endisatagproof
{\isafoldproof}%
%
\isadelimproof
\isanewline
%
\endisadelimproof
\isanewline
\isacommand{lemma}\isamarkupfalse%
\ Hfrc{\isacharunderscore}{\kern0pt}iff{\isacharunderscore}{\kern0pt}sats{\isacharcolon}{\kern0pt}\isanewline
\ \ \isakeyword{assumes}\isanewline
\ \ \ \ {\isachardoublequoteopen}P{\isasymin}nat{\isachardoublequoteclose}\ {\isachardoublequoteopen}leq{\isasymin}nat{\isachardoublequoteclose}\ {\isachardoublequoteopen}fnnc{\isasymin}nat{\isachardoublequoteclose}\ {\isachardoublequoteopen}f{\isasymin}nat{\isachardoublequoteclose}\ {\isachardoublequoteopen}env{\isasymin}list{\isacharparenleft}{\kern0pt}A{\isacharparenright}{\kern0pt}{\isachardoublequoteclose}\isanewline
\ \ \ \ {\isachardoublequoteopen}nth{\isacharparenleft}{\kern0pt}P{\isacharcomma}{\kern0pt}env{\isacharparenright}{\kern0pt}\ {\isacharequal}{\kern0pt}\ PP{\isachardoublequoteclose}\ \ {\isachardoublequoteopen}nth{\isacharparenleft}{\kern0pt}leq{\isacharcomma}{\kern0pt}env{\isacharparenright}{\kern0pt}\ {\isacharequal}{\kern0pt}\ lleq{\isachardoublequoteclose}\ {\isachardoublequoteopen}nth{\isacharparenleft}{\kern0pt}fnnc{\isacharcomma}{\kern0pt}env{\isacharparenright}{\kern0pt}\ {\isacharequal}{\kern0pt}\ ffnnc{\isachardoublequoteclose}\ {\isachardoublequoteopen}nth{\isacharparenleft}{\kern0pt}f{\isacharcomma}{\kern0pt}env{\isacharparenright}{\kern0pt}\ {\isacharequal}{\kern0pt}\ ff{\isachardoublequoteclose}\isanewline
\ \ \isakeyword{shows}\isanewline
\ \ \ \ {\isachardoublequoteopen}is{\isacharunderscore}{\kern0pt}Hfrc{\isacharparenleft}{\kern0pt}{\isacharhash}{\kern0pt}{\isacharhash}{\kern0pt}A{\isacharcomma}{\kern0pt}\ PP{\isacharcomma}{\kern0pt}\ lleq{\isacharcomma}{\kern0pt}ffnnc{\isacharcomma}{\kern0pt}ff{\isacharparenright}{\kern0pt}\isanewline
\ \ \ \ {\isasymlongleftrightarrow}\ sats{\isacharparenleft}{\kern0pt}A{\isacharcomma}{\kern0pt}Hfrc{\isacharunderscore}{\kern0pt}fm{\isacharparenleft}{\kern0pt}P{\isacharcomma}{\kern0pt}leq{\isacharcomma}{\kern0pt}fnnc{\isacharcomma}{\kern0pt}f{\isacharparenright}{\kern0pt}{\isacharcomma}{\kern0pt}env{\isacharparenright}{\kern0pt}{\isachardoublequoteclose}\isanewline
%
\isadelimproof
\ \ %
\endisadelimproof
%
\isatagproof
\isacommand{using}\isamarkupfalse%
\ assms\isanewline
\ \ \isacommand{by}\isamarkupfalse%
\ {\isacharparenleft}{\kern0pt}simp\ add{\isacharcolon}{\kern0pt}sats{\isacharunderscore}{\kern0pt}Hfrc{\isacharunderscore}{\kern0pt}fm{\isacharparenright}{\kern0pt}%
\endisatagproof
{\isafoldproof}%
%
\isadelimproof
\isanewline
%
\endisadelimproof
\isanewline
\isacommand{definition}\isamarkupfalse%
\isanewline
\ \ is{\isacharunderscore}{\kern0pt}Hfrc{\isacharunderscore}{\kern0pt}at\ {\isacharcolon}{\kern0pt}{\isacharcolon}{\kern0pt}\ {\isachardoublequoteopen}{\isacharbrackleft}{\kern0pt}i{\isasymRightarrow}o{\isacharcomma}{\kern0pt}i{\isacharcomma}{\kern0pt}i{\isacharcomma}{\kern0pt}i{\isacharcomma}{\kern0pt}i{\isacharcomma}{\kern0pt}i{\isacharbrackright}{\kern0pt}\ {\isasymRightarrow}\ o{\isachardoublequoteclose}\ \isakeyword{where}\isanewline
\ \ {\isachardoublequoteopen}is{\isacharunderscore}{\kern0pt}Hfrc{\isacharunderscore}{\kern0pt}at{\isacharparenleft}{\kern0pt}M{\isacharcomma}{\kern0pt}P{\isacharcomma}{\kern0pt}leq{\isacharcomma}{\kern0pt}fnnc{\isacharcomma}{\kern0pt}f{\isacharcomma}{\kern0pt}z{\isacharparenright}{\kern0pt}\ {\isasymequiv}\isanewline
\ \ \ \ \ \ \ \ \ \ \ \ {\isacharparenleft}{\kern0pt}empty{\isacharparenleft}{\kern0pt}M{\isacharcomma}{\kern0pt}z{\isacharparenright}{\kern0pt}\ {\isasymand}\ {\isasymnot}\ is{\isacharunderscore}{\kern0pt}Hfrc{\isacharparenleft}{\kern0pt}M{\isacharcomma}{\kern0pt}P{\isacharcomma}{\kern0pt}leq{\isacharcomma}{\kern0pt}fnnc{\isacharcomma}{\kern0pt}f{\isacharparenright}{\kern0pt}{\isacharparenright}{\kern0pt}\isanewline
\ \ \ \ \ \ \ \ \ \ {\isasymor}\ {\isacharparenleft}{\kern0pt}number{\isadigit{1}}{\isacharparenleft}{\kern0pt}M{\isacharcomma}{\kern0pt}z{\isacharparenright}{\kern0pt}\ {\isasymand}\ is{\isacharunderscore}{\kern0pt}Hfrc{\isacharparenleft}{\kern0pt}M{\isacharcomma}{\kern0pt}P{\isacharcomma}{\kern0pt}leq{\isacharcomma}{\kern0pt}fnnc{\isacharcomma}{\kern0pt}f{\isacharparenright}{\kern0pt}{\isacharparenright}{\kern0pt}{\isachardoublequoteclose}\isanewline
\isanewline
\isacommand{definition}\isamarkupfalse%
\isanewline
\ \ Hfrc{\isacharunderscore}{\kern0pt}at{\isacharunderscore}{\kern0pt}fm\ {\isacharcolon}{\kern0pt}{\isacharcolon}{\kern0pt}\ {\isachardoublequoteopen}{\isacharbrackleft}{\kern0pt}i{\isacharcomma}{\kern0pt}i{\isacharcomma}{\kern0pt}i{\isacharcomma}{\kern0pt}i{\isacharcomma}{\kern0pt}i{\isacharbrackright}{\kern0pt}\ {\isasymRightarrow}\ i{\isachardoublequoteclose}\ \isakeyword{where}\isanewline
\ \ {\isachardoublequoteopen}Hfrc{\isacharunderscore}{\kern0pt}at{\isacharunderscore}{\kern0pt}fm{\isacharparenleft}{\kern0pt}P{\isacharcomma}{\kern0pt}leq{\isacharcomma}{\kern0pt}fnnc{\isacharcomma}{\kern0pt}f{\isacharcomma}{\kern0pt}z{\isacharparenright}{\kern0pt}\ {\isasymequiv}\ Or{\isacharparenleft}{\kern0pt}And{\isacharparenleft}{\kern0pt}empty{\isacharunderscore}{\kern0pt}fm{\isacharparenleft}{\kern0pt}z{\isacharparenright}{\kern0pt}{\isacharcomma}{\kern0pt}Neg{\isacharparenleft}{\kern0pt}Hfrc{\isacharunderscore}{\kern0pt}fm{\isacharparenleft}{\kern0pt}P{\isacharcomma}{\kern0pt}leq{\isacharcomma}{\kern0pt}fnnc{\isacharcomma}{\kern0pt}f{\isacharparenright}{\kern0pt}{\isacharparenright}{\kern0pt}{\isacharparenright}{\kern0pt}{\isacharcomma}{\kern0pt}\isanewline
\ \ \ \ \ \ \ \ \ \ \ \ \ \ \ \ \ \ \ \ \ \ \ \ \ \ \ \ \ \ \ \ \ \ \ \ \ \ And{\isacharparenleft}{\kern0pt}number{\isadigit{1}}{\isacharunderscore}{\kern0pt}fm{\isacharparenleft}{\kern0pt}z{\isacharparenright}{\kern0pt}{\isacharcomma}{\kern0pt}Hfrc{\isacharunderscore}{\kern0pt}fm{\isacharparenleft}{\kern0pt}P{\isacharcomma}{\kern0pt}leq{\isacharcomma}{\kern0pt}fnnc{\isacharcomma}{\kern0pt}f{\isacharparenright}{\kern0pt}{\isacharparenright}{\kern0pt}{\isacharparenright}{\kern0pt}{\isachardoublequoteclose}\isanewline
\isanewline
\isacommand{lemma}\isamarkupfalse%
\ arity{\isacharunderscore}{\kern0pt}Hfrc{\isacharunderscore}{\kern0pt}at{\isacharunderscore}{\kern0pt}fm\ {\isacharcolon}{\kern0pt}\isanewline
\ \ \isakeyword{assumes}\isanewline
\ \ \ \ {\isachardoublequoteopen}P{\isasymin}nat{\isachardoublequoteclose}\ {\isachardoublequoteopen}leq{\isasymin}nat{\isachardoublequoteclose}\ {\isachardoublequoteopen}fnnc{\isasymin}nat{\isachardoublequoteclose}\ {\isachardoublequoteopen}f{\isasymin}nat{\isachardoublequoteclose}\ {\isachardoublequoteopen}z{\isasymin}nat{\isachardoublequoteclose}\isanewline
\ \ \isakeyword{shows}\isanewline
\ \ \ \ {\isachardoublequoteopen}arity{\isacharparenleft}{\kern0pt}Hfrc{\isacharunderscore}{\kern0pt}at{\isacharunderscore}{\kern0pt}fm{\isacharparenleft}{\kern0pt}P{\isacharcomma}{\kern0pt}leq{\isacharcomma}{\kern0pt}fnnc{\isacharcomma}{\kern0pt}f{\isacharcomma}{\kern0pt}z{\isacharparenright}{\kern0pt}{\isacharparenright}{\kern0pt}\ {\isacharequal}{\kern0pt}\ succ{\isacharparenleft}{\kern0pt}P{\isacharparenright}{\kern0pt}\ {\isasymunion}\ succ{\isacharparenleft}{\kern0pt}leq{\isacharparenright}{\kern0pt}\ {\isasymunion}\ succ{\isacharparenleft}{\kern0pt}fnnc{\isacharparenright}{\kern0pt}\ {\isasymunion}\ succ{\isacharparenleft}{\kern0pt}f{\isacharparenright}{\kern0pt}\ {\isasymunion}\ succ{\isacharparenleft}{\kern0pt}z{\isacharparenright}{\kern0pt}{\isachardoublequoteclose}\isanewline
%
\isadelimproof
\ \ %
\endisadelimproof
%
\isatagproof
\isacommand{unfolding}\isamarkupfalse%
\ Hfrc{\isacharunderscore}{\kern0pt}at{\isacharunderscore}{\kern0pt}fm{\isacharunderscore}{\kern0pt}def\isanewline
\ \ \isacommand{using}\isamarkupfalse%
\ assms\ arity{\isacharunderscore}{\kern0pt}Hfrc{\isacharunderscore}{\kern0pt}fm\ arity{\isacharunderscore}{\kern0pt}empty{\isacharunderscore}{\kern0pt}fm\ number{\isadigit{1}}arity{\isacharunderscore}{\kern0pt}{\isacharunderscore}{\kern0pt}fm\ pred{\isacharunderscore}{\kern0pt}Un{\isacharunderscore}{\kern0pt}distrib\isanewline
\ \ \isacommand{by}\isamarkupfalse%
\ auto%
\endisatagproof
{\isafoldproof}%
%
\isadelimproof
\isanewline
%
\endisadelimproof
\isanewline
\isanewline
\isacommand{lemma}\isamarkupfalse%
\ Hfrc{\isacharunderscore}{\kern0pt}at{\isacharunderscore}{\kern0pt}fm{\isacharunderscore}{\kern0pt}type{\isacharbrackleft}{\kern0pt}TC{\isacharbrackright}{\kern0pt}\ {\isacharcolon}{\kern0pt}\isanewline
\ \ {\isachardoublequoteopen}{\isasymlbrakk}P{\isasymin}nat{\isacharsemicolon}{\kern0pt}leq{\isasymin}nat{\isacharsemicolon}{\kern0pt}fnnc{\isasymin}nat{\isacharsemicolon}{\kern0pt}f{\isasymin}nat{\isacharsemicolon}{\kern0pt}z{\isasymin}nat{\isasymrbrakk}\ {\isasymLongrightarrow}\ Hfrc{\isacharunderscore}{\kern0pt}at{\isacharunderscore}{\kern0pt}fm{\isacharparenleft}{\kern0pt}P{\isacharcomma}{\kern0pt}leq{\isacharcomma}{\kern0pt}fnnc{\isacharcomma}{\kern0pt}f{\isacharcomma}{\kern0pt}z{\isacharparenright}{\kern0pt}{\isasymin}formula{\isachardoublequoteclose}\isanewline
%
\isadelimproof
\ \ %
\endisadelimproof
%
\isatagproof
\isacommand{unfolding}\isamarkupfalse%
\ Hfrc{\isacharunderscore}{\kern0pt}at{\isacharunderscore}{\kern0pt}fm{\isacharunderscore}{\kern0pt}def\ \isacommand{by}\isamarkupfalse%
\ simp%
\endisatagproof
{\isafoldproof}%
%
\isadelimproof
\isanewline
%
\endisadelimproof
\isanewline
\isacommand{lemma}\isamarkupfalse%
\ sats{\isacharunderscore}{\kern0pt}Hfrc{\isacharunderscore}{\kern0pt}at{\isacharunderscore}{\kern0pt}fm{\isacharcolon}{\kern0pt}\isanewline
\ \ \isakeyword{assumes}\isanewline
\ \ \ \ {\isachardoublequoteopen}P{\isasymin}nat{\isachardoublequoteclose}\ {\isachardoublequoteopen}leq{\isasymin}nat{\isachardoublequoteclose}\ {\isachardoublequoteopen}fnnc{\isasymin}nat{\isachardoublequoteclose}\ {\isachardoublequoteopen}f{\isasymin}nat{\isachardoublequoteclose}\ {\isachardoublequoteopen}z{\isasymin}nat{\isachardoublequoteclose}\ {\isachardoublequoteopen}env{\isasymin}list{\isacharparenleft}{\kern0pt}A{\isacharparenright}{\kern0pt}{\isachardoublequoteclose}\isanewline
\ \ \isakeyword{shows}\isanewline
\ \ \ \ {\isachardoublequoteopen}sats{\isacharparenleft}{\kern0pt}A{\isacharcomma}{\kern0pt}Hfrc{\isacharunderscore}{\kern0pt}at{\isacharunderscore}{\kern0pt}fm{\isacharparenleft}{\kern0pt}P{\isacharcomma}{\kern0pt}leq{\isacharcomma}{\kern0pt}fnnc{\isacharcomma}{\kern0pt}f{\isacharcomma}{\kern0pt}z{\isacharparenright}{\kern0pt}{\isacharcomma}{\kern0pt}env{\isacharparenright}{\kern0pt}\isanewline
\ \ \ \ {\isasymlongleftrightarrow}\ is{\isacharunderscore}{\kern0pt}Hfrc{\isacharunderscore}{\kern0pt}at{\isacharparenleft}{\kern0pt}{\isacharhash}{\kern0pt}{\isacharhash}{\kern0pt}A{\isacharcomma}{\kern0pt}nth{\isacharparenleft}{\kern0pt}P{\isacharcomma}{\kern0pt}\ env{\isacharparenright}{\kern0pt}{\isacharcomma}{\kern0pt}\ nth{\isacharparenleft}{\kern0pt}leq{\isacharcomma}{\kern0pt}\ env{\isacharparenright}{\kern0pt}{\isacharcomma}{\kern0pt}\ nth{\isacharparenleft}{\kern0pt}fnnc{\isacharcomma}{\kern0pt}\ env{\isacharparenright}{\kern0pt}{\isacharcomma}{\kern0pt}nth{\isacharparenleft}{\kern0pt}f{\isacharcomma}{\kern0pt}\ env{\isacharparenright}{\kern0pt}{\isacharcomma}{\kern0pt}nth{\isacharparenleft}{\kern0pt}z{\isacharcomma}{\kern0pt}\ env{\isacharparenright}{\kern0pt}{\isacharparenright}{\kern0pt}{\isachardoublequoteclose}\isanewline
%
\isadelimproof
\ \ %
\endisadelimproof
%
\isatagproof
\isacommand{unfolding}\isamarkupfalse%
\ is{\isacharunderscore}{\kern0pt}Hfrc{\isacharunderscore}{\kern0pt}at{\isacharunderscore}{\kern0pt}def\ Hfrc{\isacharunderscore}{\kern0pt}at{\isacharunderscore}{\kern0pt}fm{\isacharunderscore}{\kern0pt}def\ \isacommand{using}\isamarkupfalse%
\ assms\ sats{\isacharunderscore}{\kern0pt}Hfrc{\isacharunderscore}{\kern0pt}fm\isanewline
\ \ \isacommand{by}\isamarkupfalse%
\ simp%
\endisatagproof
{\isafoldproof}%
%
\isadelimproof
\isanewline
%
\endisadelimproof
\isanewline
\isacommand{lemma}\isamarkupfalse%
\ is{\isacharunderscore}{\kern0pt}Hfrc{\isacharunderscore}{\kern0pt}at{\isacharunderscore}{\kern0pt}iff{\isacharunderscore}{\kern0pt}sats{\isacharcolon}{\kern0pt}\isanewline
\ \ \isakeyword{assumes}\isanewline
\ \ \ \ {\isachardoublequoteopen}P{\isasymin}nat{\isachardoublequoteclose}\ {\isachardoublequoteopen}leq{\isasymin}nat{\isachardoublequoteclose}\ {\isachardoublequoteopen}fnnc{\isasymin}nat{\isachardoublequoteclose}\ {\isachardoublequoteopen}f{\isasymin}nat{\isachardoublequoteclose}\ {\isachardoublequoteopen}z{\isasymin}nat{\isachardoublequoteclose}\ {\isachardoublequoteopen}env{\isasymin}list{\isacharparenleft}{\kern0pt}A{\isacharparenright}{\kern0pt}{\isachardoublequoteclose}\isanewline
\ \ \ \ {\isachardoublequoteopen}nth{\isacharparenleft}{\kern0pt}P{\isacharcomma}{\kern0pt}env{\isacharparenright}{\kern0pt}\ {\isacharequal}{\kern0pt}\ PP{\isachardoublequoteclose}\ \ {\isachardoublequoteopen}nth{\isacharparenleft}{\kern0pt}leq{\isacharcomma}{\kern0pt}env{\isacharparenright}{\kern0pt}\ {\isacharequal}{\kern0pt}\ lleq{\isachardoublequoteclose}\ {\isachardoublequoteopen}nth{\isacharparenleft}{\kern0pt}fnnc{\isacharcomma}{\kern0pt}env{\isacharparenright}{\kern0pt}\ {\isacharequal}{\kern0pt}\ ffnnc{\isachardoublequoteclose}\isanewline
\ \ \ \ {\isachardoublequoteopen}nth{\isacharparenleft}{\kern0pt}f{\isacharcomma}{\kern0pt}env{\isacharparenright}{\kern0pt}\ {\isacharequal}{\kern0pt}\ ff{\isachardoublequoteclose}\ {\isachardoublequoteopen}nth{\isacharparenleft}{\kern0pt}z{\isacharcomma}{\kern0pt}env{\isacharparenright}{\kern0pt}\ {\isacharequal}{\kern0pt}\ zz{\isachardoublequoteclose}\isanewline
\ \ \isakeyword{shows}\isanewline
\ \ \ \ {\isachardoublequoteopen}is{\isacharunderscore}{\kern0pt}Hfrc{\isacharunderscore}{\kern0pt}at{\isacharparenleft}{\kern0pt}{\isacharhash}{\kern0pt}{\isacharhash}{\kern0pt}A{\isacharcomma}{\kern0pt}\ PP{\isacharcomma}{\kern0pt}\ lleq{\isacharcomma}{\kern0pt}ffnnc{\isacharcomma}{\kern0pt}ff{\isacharcomma}{\kern0pt}zz{\isacharparenright}{\kern0pt}\isanewline
\ \ \ \ {\isasymlongleftrightarrow}\ sats{\isacharparenleft}{\kern0pt}A{\isacharcomma}{\kern0pt}Hfrc{\isacharunderscore}{\kern0pt}at{\isacharunderscore}{\kern0pt}fm{\isacharparenleft}{\kern0pt}P{\isacharcomma}{\kern0pt}leq{\isacharcomma}{\kern0pt}fnnc{\isacharcomma}{\kern0pt}f{\isacharcomma}{\kern0pt}z{\isacharparenright}{\kern0pt}{\isacharcomma}{\kern0pt}env{\isacharparenright}{\kern0pt}{\isachardoublequoteclose}\isanewline
%
\isadelimproof
\ \ %
\endisadelimproof
%
\isatagproof
\isacommand{using}\isamarkupfalse%
\ assms\ \isacommand{by}\isamarkupfalse%
\ {\isacharparenleft}{\kern0pt}simp\ add{\isacharcolon}{\kern0pt}sats{\isacharunderscore}{\kern0pt}Hfrc{\isacharunderscore}{\kern0pt}at{\isacharunderscore}{\kern0pt}fm{\isacharparenright}{\kern0pt}%
\endisatagproof
{\isafoldproof}%
%
\isadelimproof
\isanewline
%
\endisadelimproof
\isanewline
\isacommand{lemma}\isamarkupfalse%
\ arity{\isacharunderscore}{\kern0pt}tran{\isacharunderscore}{\kern0pt}closure{\isacharunderscore}{\kern0pt}fm\ {\isacharcolon}{\kern0pt}\isanewline
\ \ {\isachardoublequoteopen}{\isasymlbrakk}x{\isasymin}nat{\isacharsemicolon}{\kern0pt}f{\isasymin}nat{\isasymrbrakk}\ {\isasymLongrightarrow}\ arity{\isacharparenleft}{\kern0pt}trans{\isacharunderscore}{\kern0pt}closure{\isacharunderscore}{\kern0pt}fm{\isacharparenleft}{\kern0pt}x{\isacharcomma}{\kern0pt}f{\isacharparenright}{\kern0pt}{\isacharparenright}{\kern0pt}\ {\isacharequal}{\kern0pt}\ succ{\isacharparenleft}{\kern0pt}x{\isacharparenright}{\kern0pt}\ {\isasymunion}\ succ{\isacharparenleft}{\kern0pt}f{\isacharparenright}{\kern0pt}{\isachardoublequoteclose}\isanewline
%
\isadelimproof
\ \ %
\endisadelimproof
%
\isatagproof
\isacommand{unfolding}\isamarkupfalse%
\ trans{\isacharunderscore}{\kern0pt}closure{\isacharunderscore}{\kern0pt}fm{\isacharunderscore}{\kern0pt}def\isanewline
\ \ \isacommand{using}\isamarkupfalse%
\ arity{\isacharunderscore}{\kern0pt}omega{\isacharunderscore}{\kern0pt}fm\ arity{\isacharunderscore}{\kern0pt}field{\isacharunderscore}{\kern0pt}fm\ arity{\isacharunderscore}{\kern0pt}typed{\isacharunderscore}{\kern0pt}function{\isacharunderscore}{\kern0pt}fm\ arity{\isacharunderscore}{\kern0pt}pair{\isacharunderscore}{\kern0pt}fm\ arity{\isacharunderscore}{\kern0pt}empty{\isacharunderscore}{\kern0pt}fm\ arity{\isacharunderscore}{\kern0pt}fun{\isacharunderscore}{\kern0pt}apply{\isacharunderscore}{\kern0pt}fm\isanewline
\ \ \ \ arity{\isacharunderscore}{\kern0pt}composition{\isacharunderscore}{\kern0pt}fm\ arity{\isacharunderscore}{\kern0pt}succ{\isacharunderscore}{\kern0pt}fm\ nat{\isacharunderscore}{\kern0pt}union{\isacharunderscore}{\kern0pt}abs{\isadigit{2}}\ pred{\isacharunderscore}{\kern0pt}Un{\isacharunderscore}{\kern0pt}distrib\ \isanewline
\ \ \isacommand{by}\isamarkupfalse%
\ auto%
\endisatagproof
{\isafoldproof}%
%
\isadelimproof
%
\endisadelimproof
%
\isadelimdocument
%
\endisadelimdocument
%
\isatagdocument
%
\isamarkupsubsection{The well-founded relation \isa{forcerel}%
}
\isamarkuptrue%
%
\endisatagdocument
{\isafolddocument}%
%
\isadelimdocument
%
\endisadelimdocument
\isacommand{definition}\isamarkupfalse%
\isanewline
\ \ forcerel\ {\isacharcolon}{\kern0pt}{\isacharcolon}{\kern0pt}\ {\isachardoublequoteopen}i\ {\isasymRightarrow}\ i\ {\isasymRightarrow}\ i{\isachardoublequoteclose}\ \isakeyword{where}\isanewline
\ \ {\isachardoublequoteopen}forcerel{\isacharparenleft}{\kern0pt}P{\isacharcomma}{\kern0pt}x{\isacharparenright}{\kern0pt}\ {\isasymequiv}\ frecrel{\isacharparenleft}{\kern0pt}names{\isacharunderscore}{\kern0pt}below{\isacharparenleft}{\kern0pt}P{\isacharcomma}{\kern0pt}x{\isacharparenright}{\kern0pt}{\isacharparenright}{\kern0pt}{\isacharcircum}{\kern0pt}{\isacharplus}{\kern0pt}{\isachardoublequoteclose}\isanewline
\isanewline
\isacommand{definition}\isamarkupfalse%
\isanewline
\ \ is{\isacharunderscore}{\kern0pt}forcerel\ {\isacharcolon}{\kern0pt}{\isacharcolon}{\kern0pt}\ {\isachardoublequoteopen}{\isacharbrackleft}{\kern0pt}i{\isasymRightarrow}o{\isacharcomma}{\kern0pt}i{\isacharcomma}{\kern0pt}i{\isacharcomma}{\kern0pt}i{\isacharbrackright}{\kern0pt}\ {\isasymRightarrow}\ o{\isachardoublequoteclose}\ \isakeyword{where}\isanewline
\ \ {\isachardoublequoteopen}is{\isacharunderscore}{\kern0pt}forcerel{\isacharparenleft}{\kern0pt}M{\isacharcomma}{\kern0pt}P{\isacharcomma}{\kern0pt}x{\isacharcomma}{\kern0pt}z{\isacharparenright}{\kern0pt}\ {\isasymequiv}\ {\isasymexists}r{\isacharbrackleft}{\kern0pt}M{\isacharbrackright}{\kern0pt}{\isachardot}{\kern0pt}\ {\isasymexists}nb{\isacharbrackleft}{\kern0pt}M{\isacharbrackright}{\kern0pt}{\isachardot}{\kern0pt}\ tran{\isacharunderscore}{\kern0pt}closure{\isacharparenleft}{\kern0pt}M{\isacharcomma}{\kern0pt}r{\isacharcomma}{\kern0pt}z{\isacharparenright}{\kern0pt}\ {\isasymand}\isanewline
\ \ \ \ \ \ \ \ \ \ \ \ \ \ \ \ \ \ \ \ \ \ \ \ {\isacharparenleft}{\kern0pt}is{\isacharunderscore}{\kern0pt}names{\isacharunderscore}{\kern0pt}below{\isacharparenleft}{\kern0pt}M{\isacharcomma}{\kern0pt}P{\isacharcomma}{\kern0pt}x{\isacharcomma}{\kern0pt}nb{\isacharparenright}{\kern0pt}\ {\isasymand}\ is{\isacharunderscore}{\kern0pt}frecrel{\isacharparenleft}{\kern0pt}M{\isacharcomma}{\kern0pt}nb{\isacharcomma}{\kern0pt}r{\isacharparenright}{\kern0pt}{\isacharparenright}{\kern0pt}{\isachardoublequoteclose}\isanewline
\isanewline
\isacommand{definition}\isamarkupfalse%
\isanewline
\ \ forcerel{\isacharunderscore}{\kern0pt}fm\ {\isacharcolon}{\kern0pt}{\isacharcolon}{\kern0pt}\ {\isachardoublequoteopen}i{\isasymRightarrow}\ i\ {\isasymRightarrow}\ i\ {\isasymRightarrow}\ i{\isachardoublequoteclose}\ \isakeyword{where}\isanewline
\ \ {\isachardoublequoteopen}forcerel{\isacharunderscore}{\kern0pt}fm{\isacharparenleft}{\kern0pt}p{\isacharcomma}{\kern0pt}x{\isacharcomma}{\kern0pt}z{\isacharparenright}{\kern0pt}\ {\isasymequiv}\ Exists{\isacharparenleft}{\kern0pt}Exists{\isacharparenleft}{\kern0pt}And{\isacharparenleft}{\kern0pt}trans{\isacharunderscore}{\kern0pt}closure{\isacharunderscore}{\kern0pt}fm{\isacharparenleft}{\kern0pt}{\isadigit{1}}{\isacharcomma}{\kern0pt}\ z{\isacharhash}{\kern0pt}{\isacharplus}{\kern0pt}{\isadigit{2}}{\isacharparenright}{\kern0pt}{\isacharcomma}{\kern0pt}\isanewline
\ \ \ \ \ \ \ \ \ \ \ \ \ \ \ \ \ \ \ \ \ \ \ \ \ \ \ \ \ \ \ \ \ \ \ \ \ \ \ \ And{\isacharparenleft}{\kern0pt}is{\isacharunderscore}{\kern0pt}names{\isacharunderscore}{\kern0pt}below{\isacharunderscore}{\kern0pt}fm{\isacharparenleft}{\kern0pt}p{\isacharhash}{\kern0pt}{\isacharplus}{\kern0pt}{\isadigit{2}}{\isacharcomma}{\kern0pt}x{\isacharhash}{\kern0pt}{\isacharplus}{\kern0pt}{\isadigit{2}}{\isacharcomma}{\kern0pt}{\isadigit{0}}{\isacharparenright}{\kern0pt}{\isacharcomma}{\kern0pt}frecrel{\isacharunderscore}{\kern0pt}fm{\isacharparenleft}{\kern0pt}{\isadigit{0}}{\isacharcomma}{\kern0pt}{\isadigit{1}}{\isacharparenright}{\kern0pt}{\isacharparenright}{\kern0pt}{\isacharparenright}{\kern0pt}{\isacharparenright}{\kern0pt}{\isacharparenright}{\kern0pt}{\isachardoublequoteclose}\isanewline
\isanewline
\isacommand{lemma}\isamarkupfalse%
\ arity{\isacharunderscore}{\kern0pt}forcerel{\isacharunderscore}{\kern0pt}fm{\isacharcolon}{\kern0pt}\isanewline
\ \ {\isachardoublequoteopen}{\isasymlbrakk}p{\isasymin}nat{\isacharsemicolon}{\kern0pt}x{\isasymin}nat{\isacharsemicolon}{\kern0pt}z{\isasymin}nat{\isasymrbrakk}\ {\isasymLongrightarrow}\ arity{\isacharparenleft}{\kern0pt}forcerel{\isacharunderscore}{\kern0pt}fm{\isacharparenleft}{\kern0pt}p{\isacharcomma}{\kern0pt}x{\isacharcomma}{\kern0pt}z{\isacharparenright}{\kern0pt}{\isacharparenright}{\kern0pt}\ {\isacharequal}{\kern0pt}\ succ{\isacharparenleft}{\kern0pt}p{\isacharparenright}{\kern0pt}\ {\isasymunion}\ succ{\isacharparenleft}{\kern0pt}x{\isacharparenright}{\kern0pt}\ {\isasymunion}\ succ{\isacharparenleft}{\kern0pt}z{\isacharparenright}{\kern0pt}{\isachardoublequoteclose}\isanewline
%
\isadelimproof
\ \ %
\endisadelimproof
%
\isatagproof
\isacommand{unfolding}\isamarkupfalse%
\ forcerel{\isacharunderscore}{\kern0pt}fm{\isacharunderscore}{\kern0pt}def\isanewline
\ \ \isacommand{using}\isamarkupfalse%
\ arity{\isacharunderscore}{\kern0pt}frecrel{\isacharunderscore}{\kern0pt}fm\ arity{\isacharunderscore}{\kern0pt}tran{\isacharunderscore}{\kern0pt}closure{\isacharunderscore}{\kern0pt}fm\ arity{\isacharunderscore}{\kern0pt}is{\isacharunderscore}{\kern0pt}names{\isacharunderscore}{\kern0pt}below{\isacharunderscore}{\kern0pt}fm\ pred{\isacharunderscore}{\kern0pt}Un{\isacharunderscore}{\kern0pt}distrib\isanewline
\ \ \isacommand{by}\isamarkupfalse%
\ auto%
\endisatagproof
{\isafoldproof}%
%
\isadelimproof
\isanewline
%
\endisadelimproof
\isanewline
\isacommand{lemma}\isamarkupfalse%
\ forcerel{\isacharunderscore}{\kern0pt}fm{\isacharunderscore}{\kern0pt}type{\isacharbrackleft}{\kern0pt}TC{\isacharbrackright}{\kern0pt}{\isacharcolon}{\kern0pt}\isanewline
\ \ {\isachardoublequoteopen}{\isasymlbrakk}p{\isasymin}nat{\isacharsemicolon}{\kern0pt}x{\isasymin}nat{\isacharsemicolon}{\kern0pt}z{\isasymin}nat{\isasymrbrakk}\ {\isasymLongrightarrow}\ forcerel{\isacharunderscore}{\kern0pt}fm{\isacharparenleft}{\kern0pt}p{\isacharcomma}{\kern0pt}x{\isacharcomma}{\kern0pt}z{\isacharparenright}{\kern0pt}{\isasymin}formula{\isachardoublequoteclose}\isanewline
%
\isadelimproof
\ \ %
\endisadelimproof
%
\isatagproof
\isacommand{unfolding}\isamarkupfalse%
\ forcerel{\isacharunderscore}{\kern0pt}fm{\isacharunderscore}{\kern0pt}def\ \isacommand{by}\isamarkupfalse%
\ simp%
\endisatagproof
{\isafoldproof}%
%
\isadelimproof
\isanewline
%
\endisadelimproof
\isanewline
\isanewline
\isacommand{lemma}\isamarkupfalse%
\ sats{\isacharunderscore}{\kern0pt}forcerel{\isacharunderscore}{\kern0pt}fm{\isacharcolon}{\kern0pt}\isanewline
\ \ \isakeyword{assumes}\isanewline
\ \ \ \ {\isachardoublequoteopen}p{\isasymin}nat{\isachardoublequoteclose}\ {\isachardoublequoteopen}x{\isasymin}nat{\isachardoublequoteclose}\ \ {\isachardoublequoteopen}z{\isasymin}nat{\isachardoublequoteclose}\ {\isachardoublequoteopen}env{\isasymin}list{\isacharparenleft}{\kern0pt}A{\isacharparenright}{\kern0pt}{\isachardoublequoteclose}\isanewline
\ \ \isakeyword{shows}\isanewline
\ \ \ \ {\isachardoublequoteopen}sats{\isacharparenleft}{\kern0pt}A{\isacharcomma}{\kern0pt}forcerel{\isacharunderscore}{\kern0pt}fm{\isacharparenleft}{\kern0pt}p{\isacharcomma}{\kern0pt}x{\isacharcomma}{\kern0pt}z{\isacharparenright}{\kern0pt}{\isacharcomma}{\kern0pt}env{\isacharparenright}{\kern0pt}\ {\isasymlongleftrightarrow}\ is{\isacharunderscore}{\kern0pt}forcerel{\isacharparenleft}{\kern0pt}{\isacharhash}{\kern0pt}{\isacharhash}{\kern0pt}A{\isacharcomma}{\kern0pt}nth{\isacharparenleft}{\kern0pt}p{\isacharcomma}{\kern0pt}env{\isacharparenright}{\kern0pt}{\isacharcomma}{\kern0pt}nth{\isacharparenleft}{\kern0pt}x{\isacharcomma}{\kern0pt}\ env{\isacharparenright}{\kern0pt}{\isacharcomma}{\kern0pt}nth{\isacharparenleft}{\kern0pt}z{\isacharcomma}{\kern0pt}\ env{\isacharparenright}{\kern0pt}{\isacharparenright}{\kern0pt}{\isachardoublequoteclose}\isanewline
%
\isadelimproof
%
\endisadelimproof
%
\isatagproof
\isacommand{proof}\isamarkupfalse%
\ {\isacharminus}{\kern0pt}\isanewline
\ \ \isacommand{have}\isamarkupfalse%
\ {\isachardoublequoteopen}sats{\isacharparenleft}{\kern0pt}A{\isacharcomma}{\kern0pt}trans{\isacharunderscore}{\kern0pt}closure{\isacharunderscore}{\kern0pt}fm{\isacharparenleft}{\kern0pt}{\isadigit{1}}{\isacharcomma}{\kern0pt}z\ {\isacharhash}{\kern0pt}{\isacharplus}{\kern0pt}\ {\isadigit{2}}{\isacharparenright}{\kern0pt}{\isacharcomma}{\kern0pt}Cons{\isacharparenleft}{\kern0pt}nb{\isacharcomma}{\kern0pt}Cons{\isacharparenleft}{\kern0pt}r{\isacharcomma}{\kern0pt}env{\isacharparenright}{\kern0pt}{\isacharparenright}{\kern0pt}{\isacharparenright}{\kern0pt}\ {\isasymlongleftrightarrow}\isanewline
\ \ \ \ \ \ \ \ tran{\isacharunderscore}{\kern0pt}closure{\isacharparenleft}{\kern0pt}{\isacharhash}{\kern0pt}{\isacharhash}{\kern0pt}A{\isacharcomma}{\kern0pt}\ r{\isacharcomma}{\kern0pt}\ nth{\isacharparenleft}{\kern0pt}z{\isacharcomma}{\kern0pt}\ env{\isacharparenright}{\kern0pt}{\isacharparenright}{\kern0pt}{\isachardoublequoteclose}\ \isakeyword{if}\ {\isachardoublequoteopen}r{\isasymin}A{\isachardoublequoteclose}\ {\isachardoublequoteopen}nb{\isasymin}A{\isachardoublequoteclose}\ \isakeyword{for}\ r\ nb\isanewline
\ \ \ \ \isacommand{using}\isamarkupfalse%
\ that\ assms\ trans{\isacharunderscore}{\kern0pt}closure{\isacharunderscore}{\kern0pt}fm{\isacharunderscore}{\kern0pt}iff{\isacharunderscore}{\kern0pt}sats{\isacharbrackleft}{\kern0pt}of\ {\isadigit{1}}\ {\isachardoublequoteopen}{\isacharbrackleft}{\kern0pt}nb{\isacharcomma}{\kern0pt}r{\isacharbrackright}{\kern0pt}{\isacharat}{\kern0pt}env{\isachardoublequoteclose}\ {\isacharunderscore}{\kern0pt}\ {\isachardoublequoteopen}z{\isacharhash}{\kern0pt}{\isacharplus}{\kern0pt}{\isadigit{2}}{\isachardoublequoteclose}{\isacharcomma}{\kern0pt}symmetric{\isacharbrackright}{\kern0pt}\ \isacommand{by}\isamarkupfalse%
\ simp\isanewline
\ \ \isacommand{moreover}\isamarkupfalse%
\isanewline
\ \ \isacommand{have}\isamarkupfalse%
\ {\isachardoublequoteopen}sats{\isacharparenleft}{\kern0pt}A{\isacharcomma}{\kern0pt}\ is{\isacharunderscore}{\kern0pt}names{\isacharunderscore}{\kern0pt}below{\isacharunderscore}{\kern0pt}fm{\isacharparenleft}{\kern0pt}succ{\isacharparenleft}{\kern0pt}succ{\isacharparenleft}{\kern0pt}p{\isacharparenright}{\kern0pt}{\isacharparenright}{\kern0pt}{\isacharcomma}{\kern0pt}\ succ{\isacharparenleft}{\kern0pt}succ{\isacharparenleft}{\kern0pt}x{\isacharparenright}{\kern0pt}{\isacharparenright}{\kern0pt}{\isacharcomma}{\kern0pt}\ {\isadigit{0}}{\isacharparenright}{\kern0pt}{\isacharcomma}{\kern0pt}\ Cons{\isacharparenleft}{\kern0pt}nb{\isacharcomma}{\kern0pt}\ Cons{\isacharparenleft}{\kern0pt}r{\isacharcomma}{\kern0pt}\ env{\isacharparenright}{\kern0pt}{\isacharparenright}{\kern0pt}{\isacharparenright}{\kern0pt}\ {\isasymlongleftrightarrow}\isanewline
\ \ \ \ \ \ \ \ is{\isacharunderscore}{\kern0pt}names{\isacharunderscore}{\kern0pt}below{\isacharparenleft}{\kern0pt}{\isacharhash}{\kern0pt}{\isacharhash}{\kern0pt}A{\isacharcomma}{\kern0pt}\ nth{\isacharparenleft}{\kern0pt}p{\isacharcomma}{\kern0pt}env{\isacharparenright}{\kern0pt}{\isacharcomma}{\kern0pt}\ nth{\isacharparenleft}{\kern0pt}x{\isacharcomma}{\kern0pt}\ env{\isacharparenright}{\kern0pt}{\isacharcomma}{\kern0pt}\ nb{\isacharparenright}{\kern0pt}{\isachardoublequoteclose}\isanewline
\ \ \ \ \isakeyword{if}\ {\isachardoublequoteopen}r{\isasymin}A{\isachardoublequoteclose}\ {\isachardoublequoteopen}nb{\isasymin}A{\isachardoublequoteclose}\ \isakeyword{for}\ nb\ r\isanewline
\ \ \ \ \isacommand{using}\isamarkupfalse%
\ assms\ that\ sats{\isacharunderscore}{\kern0pt}is{\isacharunderscore}{\kern0pt}names{\isacharunderscore}{\kern0pt}below{\isacharunderscore}{\kern0pt}fm{\isacharbrackleft}{\kern0pt}of\ {\isachardoublequoteopen}p\ {\isacharhash}{\kern0pt}{\isacharplus}{\kern0pt}\ {\isadigit{2}}{\isachardoublequoteclose}\ {\isachardoublequoteopen}x\ {\isacharhash}{\kern0pt}{\isacharplus}{\kern0pt}\ {\isadigit{2}}{\isachardoublequoteclose}\ {\isadigit{0}}\ {\isachardoublequoteopen}{\isacharbrackleft}{\kern0pt}nb{\isacharcomma}{\kern0pt}r{\isacharbrackright}{\kern0pt}{\isacharat}{\kern0pt}env{\isachardoublequoteclose}{\isacharbrackright}{\kern0pt}\ \isacommand{by}\isamarkupfalse%
\ simp\isanewline
\ \ \isacommand{moreover}\isamarkupfalse%
\isanewline
\ \ \isacommand{have}\isamarkupfalse%
\ {\isachardoublequoteopen}sats{\isacharparenleft}{\kern0pt}A{\isacharcomma}{\kern0pt}\ frecrel{\isacharunderscore}{\kern0pt}fm{\isacharparenleft}{\kern0pt}{\isadigit{0}}{\isacharcomma}{\kern0pt}\ {\isadigit{1}}{\isacharparenright}{\kern0pt}{\isacharcomma}{\kern0pt}\ Cons{\isacharparenleft}{\kern0pt}nb{\isacharcomma}{\kern0pt}\ Cons{\isacharparenleft}{\kern0pt}r{\isacharcomma}{\kern0pt}\ env{\isacharparenright}{\kern0pt}{\isacharparenright}{\kern0pt}{\isacharparenright}{\kern0pt}\ {\isasymlongleftrightarrow}\isanewline
\ \ \ \ \ \ \ \ is{\isacharunderscore}{\kern0pt}frecrel{\isacharparenleft}{\kern0pt}{\isacharhash}{\kern0pt}{\isacharhash}{\kern0pt}A{\isacharcomma}{\kern0pt}\ nb{\isacharcomma}{\kern0pt}\ r{\isacharparenright}{\kern0pt}{\isachardoublequoteclose}\isanewline
\ \ \ \ \isakeyword{if}\ {\isachardoublequoteopen}r{\isasymin}A{\isachardoublequoteclose}\ {\isachardoublequoteopen}nb{\isasymin}A{\isachardoublequoteclose}\ \isakeyword{for}\ r\ nb\isanewline
\ \ \ \ \isacommand{using}\isamarkupfalse%
\ assms\ that\ sats{\isacharunderscore}{\kern0pt}frecrel{\isacharunderscore}{\kern0pt}fm{\isacharbrackleft}{\kern0pt}of\ {\isadigit{0}}\ {\isadigit{1}}\ {\isachardoublequoteopen}{\isacharbrackleft}{\kern0pt}nb{\isacharcomma}{\kern0pt}r{\isacharbrackright}{\kern0pt}{\isacharat}{\kern0pt}env{\isachardoublequoteclose}{\isacharbrackright}{\kern0pt}\ \isacommand{by}\isamarkupfalse%
\ simp\isanewline
\ \ \isacommand{ultimately}\isamarkupfalse%
\isanewline
\ \ \isacommand{show}\isamarkupfalse%
\ {\isacharquery}{\kern0pt}thesis\ \isacommand{using}\isamarkupfalse%
\ assms\ \isacommand{unfolding}\isamarkupfalse%
\ is{\isacharunderscore}{\kern0pt}forcerel{\isacharunderscore}{\kern0pt}def\ forcerel{\isacharunderscore}{\kern0pt}fm{\isacharunderscore}{\kern0pt}def\ \isacommand{by}\isamarkupfalse%
\ simp\isanewline
\isacommand{qed}\isamarkupfalse%
%
\endisatagproof
{\isafoldproof}%
%
\isadelimproof
%
\endisadelimproof
%
\isadelimdocument
%
\endisadelimdocument
%
\isatagdocument
%
\isamarkupsubsection{\isa{frc{\isacharunderscore}{\kern0pt}at}, forcing for atomic formulas%
}
\isamarkuptrue%
%
\endisatagdocument
{\isafolddocument}%
%
\isadelimdocument
%
\endisadelimdocument
\isacommand{definition}\isamarkupfalse%
\isanewline
\ \ frc{\isacharunderscore}{\kern0pt}at\ {\isacharcolon}{\kern0pt}{\isacharcolon}{\kern0pt}\ {\isachardoublequoteopen}{\isacharbrackleft}{\kern0pt}i{\isacharcomma}{\kern0pt}i{\isacharcomma}{\kern0pt}i{\isacharbrackright}{\kern0pt}\ {\isasymRightarrow}\ i{\isachardoublequoteclose}\ \isakeyword{where}\isanewline
\ \ {\isachardoublequoteopen}frc{\isacharunderscore}{\kern0pt}at{\isacharparenleft}{\kern0pt}P{\isacharcomma}{\kern0pt}leq{\isacharcomma}{\kern0pt}fnnc{\isacharparenright}{\kern0pt}\ {\isasymequiv}\ wfrec{\isacharparenleft}{\kern0pt}frecrel{\isacharparenleft}{\kern0pt}names{\isacharunderscore}{\kern0pt}below{\isacharparenleft}{\kern0pt}P{\isacharcomma}{\kern0pt}fnnc{\isacharparenright}{\kern0pt}{\isacharparenright}{\kern0pt}{\isacharcomma}{\kern0pt}fnnc{\isacharcomma}{\kern0pt}\isanewline
\ \ \ \ \ \ \ \ \ \ \ \ \ \ \ \ \ \ \ \ \ \ \ \ \ \ \ \ \ \ {\isasymlambda}x\ f{\isachardot}{\kern0pt}\ bool{\isacharunderscore}{\kern0pt}of{\isacharunderscore}{\kern0pt}o{\isacharparenleft}{\kern0pt}Hfrc{\isacharparenleft}{\kern0pt}P{\isacharcomma}{\kern0pt}leq{\isacharcomma}{\kern0pt}x{\isacharcomma}{\kern0pt}f{\isacharparenright}{\kern0pt}{\isacharparenright}{\kern0pt}{\isacharparenright}{\kern0pt}{\isachardoublequoteclose}\isanewline
\isanewline
\isacommand{definition}\isamarkupfalse%
\isanewline
\ \ is{\isacharunderscore}{\kern0pt}frc{\isacharunderscore}{\kern0pt}at\ {\isacharcolon}{\kern0pt}{\isacharcolon}{\kern0pt}\ {\isachardoublequoteopen}{\isacharbrackleft}{\kern0pt}i{\isasymRightarrow}o{\isacharcomma}{\kern0pt}i{\isacharcomma}{\kern0pt}i{\isacharcomma}{\kern0pt}i{\isacharcomma}{\kern0pt}i{\isacharbrackright}{\kern0pt}\ {\isasymRightarrow}\ o{\isachardoublequoteclose}\ \isakeyword{where}\isanewline
\ \ {\isachardoublequoteopen}is{\isacharunderscore}{\kern0pt}frc{\isacharunderscore}{\kern0pt}at{\isacharparenleft}{\kern0pt}M{\isacharcomma}{\kern0pt}P{\isacharcomma}{\kern0pt}leq{\isacharcomma}{\kern0pt}x{\isacharcomma}{\kern0pt}z{\isacharparenright}{\kern0pt}\ {\isasymequiv}\ {\isasymexists}r{\isacharbrackleft}{\kern0pt}M{\isacharbrackright}{\kern0pt}{\isachardot}{\kern0pt}\ is{\isacharunderscore}{\kern0pt}forcerel{\isacharparenleft}{\kern0pt}M{\isacharcomma}{\kern0pt}P{\isacharcomma}{\kern0pt}x{\isacharcomma}{\kern0pt}r{\isacharparenright}{\kern0pt}\ {\isasymand}\isanewline
\ \ \ \ \ \ \ \ \ \ \ \ \ \ \ \ \ \ \ \ \ \ \ \ \ \ \ \ \ \ \ \ \ \ \ \ is{\isacharunderscore}{\kern0pt}wfrec{\isacharparenleft}{\kern0pt}M{\isacharcomma}{\kern0pt}is{\isacharunderscore}{\kern0pt}Hfrc{\isacharunderscore}{\kern0pt}at{\isacharparenleft}{\kern0pt}M{\isacharcomma}{\kern0pt}P{\isacharcomma}{\kern0pt}leq{\isacharparenright}{\kern0pt}{\isacharcomma}{\kern0pt}r{\isacharcomma}{\kern0pt}x{\isacharcomma}{\kern0pt}z{\isacharparenright}{\kern0pt}{\isachardoublequoteclose}\isanewline
\isanewline
\isacommand{definition}\isamarkupfalse%
\isanewline
\ \ frc{\isacharunderscore}{\kern0pt}at{\isacharunderscore}{\kern0pt}fm\ {\isacharcolon}{\kern0pt}{\isacharcolon}{\kern0pt}\ {\isachardoublequoteopen}{\isacharbrackleft}{\kern0pt}i{\isacharcomma}{\kern0pt}i{\isacharcomma}{\kern0pt}i{\isacharcomma}{\kern0pt}i{\isacharbrackright}{\kern0pt}\ {\isasymRightarrow}\ i{\isachardoublequoteclose}\ \isakeyword{where}\isanewline
\ \ {\isachardoublequoteopen}frc{\isacharunderscore}{\kern0pt}at{\isacharunderscore}{\kern0pt}fm{\isacharparenleft}{\kern0pt}p{\isacharcomma}{\kern0pt}l{\isacharcomma}{\kern0pt}x{\isacharcomma}{\kern0pt}z{\isacharparenright}{\kern0pt}\ {\isasymequiv}\ Exists{\isacharparenleft}{\kern0pt}And{\isacharparenleft}{\kern0pt}forcerel{\isacharunderscore}{\kern0pt}fm{\isacharparenleft}{\kern0pt}succ{\isacharparenleft}{\kern0pt}p{\isacharparenright}{\kern0pt}{\isacharcomma}{\kern0pt}succ{\isacharparenleft}{\kern0pt}x{\isacharparenright}{\kern0pt}{\isacharcomma}{\kern0pt}{\isadigit{0}}{\isacharparenright}{\kern0pt}{\isacharcomma}{\kern0pt}\isanewline
\ \ \ \ \ \ \ \ \ \ is{\isacharunderscore}{\kern0pt}wfrec{\isacharunderscore}{\kern0pt}fm{\isacharparenleft}{\kern0pt}Hfrc{\isacharunderscore}{\kern0pt}at{\isacharunderscore}{\kern0pt}fm{\isacharparenleft}{\kern0pt}{\isadigit{6}}{\isacharhash}{\kern0pt}{\isacharplus}{\kern0pt}p{\isacharcomma}{\kern0pt}{\isadigit{6}}{\isacharhash}{\kern0pt}{\isacharplus}{\kern0pt}l{\isacharcomma}{\kern0pt}{\isadigit{2}}{\isacharcomma}{\kern0pt}{\isadigit{1}}{\isacharcomma}{\kern0pt}{\isadigit{0}}{\isacharparenright}{\kern0pt}{\isacharcomma}{\kern0pt}{\isadigit{0}}{\isacharcomma}{\kern0pt}succ{\isacharparenleft}{\kern0pt}x{\isacharparenright}{\kern0pt}{\isacharcomma}{\kern0pt}succ{\isacharparenleft}{\kern0pt}z{\isacharparenright}{\kern0pt}{\isacharparenright}{\kern0pt}{\isacharparenright}{\kern0pt}{\isacharparenright}{\kern0pt}{\isachardoublequoteclose}\isanewline
\isanewline
\isacommand{lemma}\isamarkupfalse%
\ frc{\isacharunderscore}{\kern0pt}at{\isacharunderscore}{\kern0pt}fm{\isacharunderscore}{\kern0pt}type\ {\isacharbrackleft}{\kern0pt}TC{\isacharbrackright}{\kern0pt}\ {\isacharcolon}{\kern0pt}\isanewline
\ \ {\isachardoublequoteopen}{\isasymlbrakk}p{\isasymin}nat{\isacharsemicolon}{\kern0pt}l{\isasymin}nat{\isacharsemicolon}{\kern0pt}x{\isasymin}nat{\isacharsemicolon}{\kern0pt}z{\isasymin}nat{\isasymrbrakk}\ {\isasymLongrightarrow}\ frc{\isacharunderscore}{\kern0pt}at{\isacharunderscore}{\kern0pt}fm{\isacharparenleft}{\kern0pt}p{\isacharcomma}{\kern0pt}l{\isacharcomma}{\kern0pt}x{\isacharcomma}{\kern0pt}z{\isacharparenright}{\kern0pt}{\isasymin}formula{\isachardoublequoteclose}\isanewline
%
\isadelimproof
\ \ %
\endisadelimproof
%
\isatagproof
\isacommand{unfolding}\isamarkupfalse%
\ frc{\isacharunderscore}{\kern0pt}at{\isacharunderscore}{\kern0pt}fm{\isacharunderscore}{\kern0pt}def\ \isacommand{by}\isamarkupfalse%
\ simp%
\endisatagproof
{\isafoldproof}%
%
\isadelimproof
\isanewline
%
\endisadelimproof
\isanewline
\isacommand{lemma}\isamarkupfalse%
\ arity{\isacharunderscore}{\kern0pt}frc{\isacharunderscore}{\kern0pt}at{\isacharunderscore}{\kern0pt}fm\ {\isacharcolon}{\kern0pt}\isanewline
\ \ \isakeyword{assumes}\ {\isachardoublequoteopen}p{\isasymin}nat{\isachardoublequoteclose}\ {\isachardoublequoteopen}l{\isasymin}nat{\isachardoublequoteclose}\ {\isachardoublequoteopen}x{\isasymin}nat{\isachardoublequoteclose}\ {\isachardoublequoteopen}z{\isasymin}nat{\isachardoublequoteclose}\isanewline
\ \ \isakeyword{shows}\ {\isachardoublequoteopen}arity{\isacharparenleft}{\kern0pt}frc{\isacharunderscore}{\kern0pt}at{\isacharunderscore}{\kern0pt}fm{\isacharparenleft}{\kern0pt}p{\isacharcomma}{\kern0pt}l{\isacharcomma}{\kern0pt}x{\isacharcomma}{\kern0pt}z{\isacharparenright}{\kern0pt}{\isacharparenright}{\kern0pt}\ {\isacharequal}{\kern0pt}\ succ{\isacharparenleft}{\kern0pt}p{\isacharparenright}{\kern0pt}\ {\isasymunion}\ succ{\isacharparenleft}{\kern0pt}l{\isacharparenright}{\kern0pt}\ {\isasymunion}\ succ{\isacharparenleft}{\kern0pt}x{\isacharparenright}{\kern0pt}\ {\isasymunion}\ succ{\isacharparenleft}{\kern0pt}z{\isacharparenright}{\kern0pt}{\isachardoublequoteclose}\isanewline
%
\isadelimproof
%
\endisadelimproof
%
\isatagproof
\isacommand{proof}\isamarkupfalse%
\ {\isacharminus}{\kern0pt}\isanewline
\ \ \isacommand{let}\isamarkupfalse%
\ {\isacharquery}{\kern0pt}{\isasymphi}\ {\isacharequal}{\kern0pt}\ {\isachardoublequoteopen}Hfrc{\isacharunderscore}{\kern0pt}at{\isacharunderscore}{\kern0pt}fm{\isacharparenleft}{\kern0pt}{\isadigit{6}}\ {\isacharhash}{\kern0pt}{\isacharplus}{\kern0pt}\ p{\isacharcomma}{\kern0pt}\ {\isadigit{6}}\ {\isacharhash}{\kern0pt}{\isacharplus}{\kern0pt}\ l{\isacharcomma}{\kern0pt}\ {\isadigit{2}}{\isacharcomma}{\kern0pt}\ {\isadigit{1}}{\isacharcomma}{\kern0pt}\ {\isadigit{0}}{\isacharparenright}{\kern0pt}{\isachardoublequoteclose}\isanewline
\ \ \isacommand{from}\isamarkupfalse%
\ assms\isanewline
\ \ \isacommand{have}\isamarkupfalse%
\ \ {\isachardoublequoteopen}arity{\isacharparenleft}{\kern0pt}{\isacharquery}{\kern0pt}{\isasymphi}{\isacharparenright}{\kern0pt}\ {\isacharequal}{\kern0pt}\ {\isacharparenleft}{\kern0pt}{\isadigit{7}}{\isacharhash}{\kern0pt}{\isacharplus}{\kern0pt}p{\isacharparenright}{\kern0pt}\ {\isasymunion}\ {\isacharparenleft}{\kern0pt}{\isadigit{7}}{\isacharhash}{\kern0pt}{\isacharplus}{\kern0pt}l{\isacharparenright}{\kern0pt}{\isachardoublequoteclose}\ {\isachardoublequoteopen}{\isacharquery}{\kern0pt}{\isasymphi}\ {\isasymin}\ formula{\isachardoublequoteclose}\isanewline
\ \ \ \ \isacommand{using}\isamarkupfalse%
\ arity{\isacharunderscore}{\kern0pt}Hfrc{\isacharunderscore}{\kern0pt}at{\isacharunderscore}{\kern0pt}fm\ nat{\isacharunderscore}{\kern0pt}simp{\isacharunderscore}{\kern0pt}union\isanewline
\ \ \ \ \isacommand{by}\isamarkupfalse%
\ auto\isanewline
\ \ \isacommand{with}\isamarkupfalse%
\ assms\isanewline
\ \ \isacommand{have}\isamarkupfalse%
\ W{\isacharcolon}{\kern0pt}\ {\isachardoublequoteopen}arity{\isacharparenleft}{\kern0pt}is{\isacharunderscore}{\kern0pt}wfrec{\isacharunderscore}{\kern0pt}fm{\isacharparenleft}{\kern0pt}{\isacharquery}{\kern0pt}{\isasymphi}{\isacharcomma}{\kern0pt}\ {\isadigit{0}}{\isacharcomma}{\kern0pt}\ succ{\isacharparenleft}{\kern0pt}x{\isacharparenright}{\kern0pt}{\isacharcomma}{\kern0pt}\ succ{\isacharparenleft}{\kern0pt}z{\isacharparenright}{\kern0pt}{\isacharparenright}{\kern0pt}{\isacharparenright}{\kern0pt}\ {\isacharequal}{\kern0pt}\ {\isadigit{2}}{\isacharhash}{\kern0pt}{\isacharplus}{\kern0pt}p\ {\isasymunion}\ {\isacharparenleft}{\kern0pt}{\isadigit{2}}{\isacharhash}{\kern0pt}{\isacharplus}{\kern0pt}l{\isacharparenright}{\kern0pt}\ {\isasymunion}\ {\isacharparenleft}{\kern0pt}{\isadigit{2}}{\isacharhash}{\kern0pt}{\isacharplus}{\kern0pt}x{\isacharparenright}{\kern0pt}\ {\isasymunion}\ {\isacharparenleft}{\kern0pt}{\isadigit{2}}{\isacharhash}{\kern0pt}{\isacharplus}{\kern0pt}z{\isacharparenright}{\kern0pt}{\isachardoublequoteclose}\isanewline
\ \ \ \ \isacommand{using}\isamarkupfalse%
\ arity{\isacharunderscore}{\kern0pt}is{\isacharunderscore}{\kern0pt}wfrec{\isacharunderscore}{\kern0pt}fm{\isacharbrackleft}{\kern0pt}OF\ {\isacartoucheopen}{\isacharquery}{\kern0pt}{\isasymphi}{\isasymin}{\isacharunderscore}{\kern0pt}{\isacartoucheclose}\ {\isacharunderscore}{\kern0pt}\ {\isacharunderscore}{\kern0pt}\ {\isacharunderscore}{\kern0pt}\ {\isacharunderscore}{\kern0pt}\ {\isacartoucheopen}arity{\isacharparenleft}{\kern0pt}{\isacharquery}{\kern0pt}{\isasymphi}{\isacharparenright}{\kern0pt}\ {\isacharequal}{\kern0pt}\ {\isacharunderscore}{\kern0pt}{\isacartoucheclose}{\isacharbrackright}{\kern0pt}\ pred{\isacharunderscore}{\kern0pt}Un{\isacharunderscore}{\kern0pt}distrib\ pred{\isacharunderscore}{\kern0pt}succ{\isacharunderscore}{\kern0pt}eq\isanewline
\ \ \ \ \ \ nat{\isacharunderscore}{\kern0pt}union{\isacharunderscore}{\kern0pt}abs{\isadigit{1}}\isanewline
\ \ \ \ \isacommand{by}\isamarkupfalse%
\ auto\isanewline
\ \ \isacommand{from}\isamarkupfalse%
\ assms\isanewline
\ \ \isacommand{have}\isamarkupfalse%
\ {\isachardoublequoteopen}arity{\isacharparenleft}{\kern0pt}forcerel{\isacharunderscore}{\kern0pt}fm{\isacharparenleft}{\kern0pt}succ{\isacharparenleft}{\kern0pt}p{\isacharparenright}{\kern0pt}{\isacharcomma}{\kern0pt}succ{\isacharparenleft}{\kern0pt}x{\isacharparenright}{\kern0pt}{\isacharcomma}{\kern0pt}{\isadigit{0}}{\isacharparenright}{\kern0pt}{\isacharparenright}{\kern0pt}\ {\isacharequal}{\kern0pt}\ succ{\isacharparenleft}{\kern0pt}succ{\isacharparenleft}{\kern0pt}p{\isacharparenright}{\kern0pt}{\isacharparenright}{\kern0pt}\ {\isasymunion}\ succ{\isacharparenleft}{\kern0pt}succ{\isacharparenleft}{\kern0pt}x{\isacharparenright}{\kern0pt}{\isacharparenright}{\kern0pt}{\isachardoublequoteclose}\isanewline
\ \ \ \ \isacommand{using}\isamarkupfalse%
\ arity{\isacharunderscore}{\kern0pt}forcerel{\isacharunderscore}{\kern0pt}fm\ nat{\isacharunderscore}{\kern0pt}simp{\isacharunderscore}{\kern0pt}union\isanewline
\ \ \ \ \isacommand{by}\isamarkupfalse%
\ auto\isanewline
\ \ \isacommand{with}\isamarkupfalse%
\ W\ assms\isanewline
\ \ \isacommand{show}\isamarkupfalse%
\ {\isacharquery}{\kern0pt}thesis\isanewline
\ \ \ \ \isacommand{unfolding}\isamarkupfalse%
\ frc{\isacharunderscore}{\kern0pt}at{\isacharunderscore}{\kern0pt}fm{\isacharunderscore}{\kern0pt}def\isanewline
\ \ \ \ \isacommand{using}\isamarkupfalse%
\ arity{\isacharunderscore}{\kern0pt}forcerel{\isacharunderscore}{\kern0pt}fm\ pred{\isacharunderscore}{\kern0pt}Un{\isacharunderscore}{\kern0pt}distrib\isanewline
\ \ \ \ \isacommand{by}\isamarkupfalse%
\ auto\isanewline
\isacommand{qed}\isamarkupfalse%
%
\endisatagproof
{\isafoldproof}%
%
\isadelimproof
\isanewline
%
\endisadelimproof
\isanewline
\isacommand{lemma}\isamarkupfalse%
\ sats{\isacharunderscore}{\kern0pt}frc{\isacharunderscore}{\kern0pt}at{\isacharunderscore}{\kern0pt}fm\ {\isacharcolon}{\kern0pt}\isanewline
\ \ \isakeyword{assumes}\isanewline
\ \ \ \ {\isachardoublequoteopen}p{\isasymin}nat{\isachardoublequoteclose}\ {\isachardoublequoteopen}l{\isasymin}nat{\isachardoublequoteclose}\ {\isachardoublequoteopen}i{\isasymin}nat{\isachardoublequoteclose}\ {\isachardoublequoteopen}j{\isasymin}nat{\isachardoublequoteclose}\ {\isachardoublequoteopen}env{\isasymin}list{\isacharparenleft}{\kern0pt}A{\isacharparenright}{\kern0pt}{\isachardoublequoteclose}\ {\isachardoublequoteopen}i\ {\isacharless}{\kern0pt}\ length{\isacharparenleft}{\kern0pt}env{\isacharparenright}{\kern0pt}{\isachardoublequoteclose}\ {\isachardoublequoteopen}j\ {\isacharless}{\kern0pt}\ length{\isacharparenleft}{\kern0pt}env{\isacharparenright}{\kern0pt}{\isachardoublequoteclose}\isanewline
\ \ \isakeyword{shows}\isanewline
\ \ \ \ {\isachardoublequoteopen}sats{\isacharparenleft}{\kern0pt}A{\isacharcomma}{\kern0pt}frc{\isacharunderscore}{\kern0pt}at{\isacharunderscore}{\kern0pt}fm{\isacharparenleft}{\kern0pt}p{\isacharcomma}{\kern0pt}l{\isacharcomma}{\kern0pt}i{\isacharcomma}{\kern0pt}j{\isacharparenright}{\kern0pt}{\isacharcomma}{\kern0pt}env{\isacharparenright}{\kern0pt}\ {\isasymlongleftrightarrow}\isanewline
\ \ \ \ \ is{\isacharunderscore}{\kern0pt}frc{\isacharunderscore}{\kern0pt}at{\isacharparenleft}{\kern0pt}{\isacharhash}{\kern0pt}{\isacharhash}{\kern0pt}A{\isacharcomma}{\kern0pt}nth{\isacharparenleft}{\kern0pt}p{\isacharcomma}{\kern0pt}env{\isacharparenright}{\kern0pt}{\isacharcomma}{\kern0pt}nth{\isacharparenleft}{\kern0pt}l{\isacharcomma}{\kern0pt}env{\isacharparenright}{\kern0pt}{\isacharcomma}{\kern0pt}nth{\isacharparenleft}{\kern0pt}i{\isacharcomma}{\kern0pt}env{\isacharparenright}{\kern0pt}{\isacharcomma}{\kern0pt}nth{\isacharparenleft}{\kern0pt}j{\isacharcomma}{\kern0pt}env{\isacharparenright}{\kern0pt}{\isacharparenright}{\kern0pt}{\isachardoublequoteclose}\isanewline
%
\isadelimproof
%
\endisadelimproof
%
\isatagproof
\isacommand{proof}\isamarkupfalse%
\ {\isacharminus}{\kern0pt}\isanewline
\ \ \isacommand{{\isacharbraceleft}{\kern0pt}}\isamarkupfalse%
\isanewline
\ \ \ \ \isacommand{fix}\isamarkupfalse%
\ r\ pp\ ll\isanewline
\ \ \ \ \isacommand{assume}\isamarkupfalse%
\ {\isachardoublequoteopen}r{\isasymin}A{\isachardoublequoteclose}\isanewline
\ \ \ \ \isacommand{have}\isamarkupfalse%
\ {\isadigit{0}}{\isacharcolon}{\kern0pt}{\isachardoublequoteopen}is{\isacharunderscore}{\kern0pt}Hfrc{\isacharunderscore}{\kern0pt}at{\isacharparenleft}{\kern0pt}{\isacharhash}{\kern0pt}{\isacharhash}{\kern0pt}A{\isacharcomma}{\kern0pt}nth{\isacharparenleft}{\kern0pt}p{\isacharcomma}{\kern0pt}env{\isacharparenright}{\kern0pt}{\isacharcomma}{\kern0pt}nth{\isacharparenleft}{\kern0pt}l{\isacharcomma}{\kern0pt}env{\isacharparenright}{\kern0pt}{\isacharcomma}{\kern0pt}a{\isadigit{2}}{\isacharcomma}{\kern0pt}\ a{\isadigit{1}}{\isacharcomma}{\kern0pt}\ a{\isadigit{0}}{\isacharparenright}{\kern0pt}\ {\isasymlongleftrightarrow}\isanewline
\ \ \ \ \ \ \ \ \ sats{\isacharparenleft}{\kern0pt}A{\isacharcomma}{\kern0pt}\ Hfrc{\isacharunderscore}{\kern0pt}at{\isacharunderscore}{\kern0pt}fm{\isacharparenleft}{\kern0pt}{\isadigit{6}}{\isacharhash}{\kern0pt}{\isacharplus}{\kern0pt}p{\isacharcomma}{\kern0pt}{\isadigit{6}}{\isacharhash}{\kern0pt}{\isacharplus}{\kern0pt}l{\isacharcomma}{\kern0pt}{\isadigit{2}}{\isacharcomma}{\kern0pt}{\isadigit{1}}{\isacharcomma}{\kern0pt}{\isadigit{0}}{\isacharparenright}{\kern0pt}{\isacharcomma}{\kern0pt}\ {\isacharbrackleft}{\kern0pt}a{\isadigit{0}}{\isacharcomma}{\kern0pt}a{\isadigit{1}}{\isacharcomma}{\kern0pt}a{\isadigit{2}}{\isacharcomma}{\kern0pt}a{\isadigit{3}}{\isacharcomma}{\kern0pt}a{\isadigit{4}}{\isacharcomma}{\kern0pt}r{\isacharbrackright}{\kern0pt}{\isacharat}{\kern0pt}env{\isacharparenright}{\kern0pt}{\isachardoublequoteclose}\isanewline
\ \ \ \ \ \ \isakeyword{if}\ {\isachardoublequoteopen}a{\isadigit{0}}{\isasymin}A{\isachardoublequoteclose}\ {\isachardoublequoteopen}a{\isadigit{1}}{\isasymin}A{\isachardoublequoteclose}\ {\isachardoublequoteopen}a{\isadigit{2}}{\isasymin}A{\isachardoublequoteclose}\ {\isachardoublequoteopen}a{\isadigit{3}}{\isasymin}A{\isachardoublequoteclose}\ {\isachardoublequoteopen}a{\isadigit{4}}{\isasymin}A{\isachardoublequoteclose}\ \isakeyword{for}\ a{\isadigit{0}}\ a{\isadigit{1}}\ a{\isadigit{2}}\ a{\isadigit{3}}\ a{\isadigit{4}}\isanewline
\ \ \ \ \ \ \isacommand{using}\isamarkupfalse%
\ \ that\ assms\ {\isacartoucheopen}r{\isasymin}A{\isacartoucheclose}\isanewline
\ \ \ \ \ \ \ \ is{\isacharunderscore}{\kern0pt}Hfrc{\isacharunderscore}{\kern0pt}at{\isacharunderscore}{\kern0pt}iff{\isacharunderscore}{\kern0pt}sats{\isacharbrackleft}{\kern0pt}of\ {\isachardoublequoteopen}{\isadigit{6}}{\isacharhash}{\kern0pt}{\isacharplus}{\kern0pt}p{\isachardoublequoteclose}\ {\isachardoublequoteopen}{\isadigit{6}}{\isacharhash}{\kern0pt}{\isacharplus}{\kern0pt}l{\isachardoublequoteclose}\ {\isadigit{2}}\ {\isadigit{1}}\ {\isadigit{0}}\ {\isachardoublequoteopen}{\isacharbrackleft}{\kern0pt}a{\isadigit{0}}{\isacharcomma}{\kern0pt}a{\isadigit{1}}{\isacharcomma}{\kern0pt}a{\isadigit{2}}{\isacharcomma}{\kern0pt}a{\isadigit{3}}{\isacharcomma}{\kern0pt}a{\isadigit{4}}{\isacharcomma}{\kern0pt}r{\isacharbrackright}{\kern0pt}{\isacharat}{\kern0pt}env{\isachardoublequoteclose}\ A{\isacharbrackright}{\kern0pt}\ \ \isacommand{by}\isamarkupfalse%
\ simp\isanewline
\ \ \ \ \isacommand{have}\isamarkupfalse%
\ {\isachardoublequoteopen}sats{\isacharparenleft}{\kern0pt}A{\isacharcomma}{\kern0pt}is{\isacharunderscore}{\kern0pt}wfrec{\isacharunderscore}{\kern0pt}fm{\isacharparenleft}{\kern0pt}Hfrc{\isacharunderscore}{\kern0pt}at{\isacharunderscore}{\kern0pt}fm{\isacharparenleft}{\kern0pt}{\isadigit{6}}\ {\isacharhash}{\kern0pt}{\isacharplus}{\kern0pt}\ p{\isacharcomma}{\kern0pt}\ {\isadigit{6}}\ {\isacharhash}{\kern0pt}{\isacharplus}{\kern0pt}\ l{\isacharcomma}{\kern0pt}\ {\isadigit{2}}{\isacharcomma}{\kern0pt}\ {\isadigit{1}}{\isacharcomma}{\kern0pt}\ {\isadigit{0}}{\isacharparenright}{\kern0pt}{\isacharcomma}{\kern0pt}\ {\isadigit{0}}{\isacharcomma}{\kern0pt}\ succ{\isacharparenleft}{\kern0pt}i{\isacharparenright}{\kern0pt}{\isacharcomma}{\kern0pt}\ succ{\isacharparenleft}{\kern0pt}j{\isacharparenright}{\kern0pt}{\isacharparenright}{\kern0pt}{\isacharcomma}{\kern0pt}{\isacharbrackleft}{\kern0pt}r{\isacharbrackright}{\kern0pt}{\isacharat}{\kern0pt}env{\isacharparenright}{\kern0pt}\ {\isasymlongleftrightarrow}\isanewline
\ \ \ \ \ \ \ \ \ is{\isacharunderscore}{\kern0pt}wfrec{\isacharparenleft}{\kern0pt}{\isacharhash}{\kern0pt}{\isacharhash}{\kern0pt}A{\isacharcomma}{\kern0pt}\ is{\isacharunderscore}{\kern0pt}Hfrc{\isacharunderscore}{\kern0pt}at{\isacharparenleft}{\kern0pt}{\isacharhash}{\kern0pt}{\isacharhash}{\kern0pt}A{\isacharcomma}{\kern0pt}\ nth{\isacharparenleft}{\kern0pt}p{\isacharcomma}{\kern0pt}env{\isacharparenright}{\kern0pt}{\isacharcomma}{\kern0pt}\ nth{\isacharparenleft}{\kern0pt}l{\isacharcomma}{\kern0pt}env{\isacharparenright}{\kern0pt}{\isacharparenright}{\kern0pt}{\isacharcomma}{\kern0pt}\ r{\isacharcomma}{\kern0pt}nth{\isacharparenleft}{\kern0pt}i{\isacharcomma}{\kern0pt}\ env{\isacharparenright}{\kern0pt}{\isacharcomma}{\kern0pt}\ nth{\isacharparenleft}{\kern0pt}j{\isacharcomma}{\kern0pt}\ env{\isacharparenright}{\kern0pt}{\isacharparenright}{\kern0pt}{\isachardoublequoteclose}\isanewline
\ \ \ \ \ \ \isacommand{using}\isamarkupfalse%
\ assms\ {\isacartoucheopen}r{\isasymin}A{\isacartoucheclose}\isanewline
\ \ \ \ \ \ \ \ sats{\isacharunderscore}{\kern0pt}is{\isacharunderscore}{\kern0pt}wfrec{\isacharunderscore}{\kern0pt}fm{\isacharbrackleft}{\kern0pt}OF\ {\isadigit{0}}{\isacharbrackleft}{\kern0pt}simplified{\isacharbrackright}{\kern0pt}{\isacharbrackright}{\kern0pt}\isanewline
\ \ \ \ \ \ \isacommand{by}\isamarkupfalse%
\ simp\isanewline
\ \ \isacommand{{\isacharbraceright}{\kern0pt}}\isamarkupfalse%
\isanewline
\ \ \isacommand{moreover}\isamarkupfalse%
\isanewline
\ \ \isacommand{have}\isamarkupfalse%
\ {\isachardoublequoteopen}sats{\isacharparenleft}{\kern0pt}A{\isacharcomma}{\kern0pt}\ forcerel{\isacharunderscore}{\kern0pt}fm{\isacharparenleft}{\kern0pt}succ{\isacharparenleft}{\kern0pt}p{\isacharparenright}{\kern0pt}{\isacharcomma}{\kern0pt}\ succ{\isacharparenleft}{\kern0pt}i{\isacharparenright}{\kern0pt}{\isacharcomma}{\kern0pt}\ {\isadigit{0}}{\isacharparenright}{\kern0pt}{\isacharcomma}{\kern0pt}\ Cons{\isacharparenleft}{\kern0pt}r{\isacharcomma}{\kern0pt}\ env{\isacharparenright}{\kern0pt}{\isacharparenright}{\kern0pt}\ {\isasymlongleftrightarrow}\isanewline
\ \ \ \ \ \ \ \ is{\isacharunderscore}{\kern0pt}forcerel{\isacharparenleft}{\kern0pt}{\isacharhash}{\kern0pt}{\isacharhash}{\kern0pt}A{\isacharcomma}{\kern0pt}nth{\isacharparenleft}{\kern0pt}p{\isacharcomma}{\kern0pt}env{\isacharparenright}{\kern0pt}{\isacharcomma}{\kern0pt}nth{\isacharparenleft}{\kern0pt}i{\isacharcomma}{\kern0pt}env{\isacharparenright}{\kern0pt}{\isacharcomma}{\kern0pt}r{\isacharparenright}{\kern0pt}{\isachardoublequoteclose}\ \isakeyword{if}\ {\isachardoublequoteopen}r{\isasymin}A{\isachardoublequoteclose}\ \isakeyword{for}\ r\isanewline
\ \ \ \ \isacommand{using}\isamarkupfalse%
\ assms\ sats{\isacharunderscore}{\kern0pt}forcerel{\isacharunderscore}{\kern0pt}fm\ that\ \isacommand{by}\isamarkupfalse%
\ simp\isanewline
\ \ \isacommand{ultimately}\isamarkupfalse%
\isanewline
\ \ \isacommand{show}\isamarkupfalse%
\ {\isacharquery}{\kern0pt}thesis\ \isacommand{unfolding}\isamarkupfalse%
\ is{\isacharunderscore}{\kern0pt}frc{\isacharunderscore}{\kern0pt}at{\isacharunderscore}{\kern0pt}def\ frc{\isacharunderscore}{\kern0pt}at{\isacharunderscore}{\kern0pt}fm{\isacharunderscore}{\kern0pt}def\isanewline
\ \ \ \ \isacommand{using}\isamarkupfalse%
\ assms\ \isacommand{by}\isamarkupfalse%
\ simp\isanewline
\isacommand{qed}\isamarkupfalse%
%
\endisatagproof
{\isafoldproof}%
%
\isadelimproof
\isanewline
%
\endisadelimproof
\isanewline
\isacommand{definition}\isamarkupfalse%
\isanewline
\ \ forces{\isacharunderscore}{\kern0pt}eq{\isacharprime}{\kern0pt}\ {\isacharcolon}{\kern0pt}{\isacharcolon}{\kern0pt}\ {\isachardoublequoteopen}{\isacharbrackleft}{\kern0pt}i{\isacharcomma}{\kern0pt}i{\isacharcomma}{\kern0pt}i{\isacharcomma}{\kern0pt}i{\isacharcomma}{\kern0pt}i{\isacharbrackright}{\kern0pt}\ {\isasymRightarrow}\ o{\isachardoublequoteclose}\ \isakeyword{where}\isanewline
\ \ {\isachardoublequoteopen}forces{\isacharunderscore}{\kern0pt}eq{\isacharprime}{\kern0pt}{\isacharparenleft}{\kern0pt}P{\isacharcomma}{\kern0pt}l{\isacharcomma}{\kern0pt}p{\isacharcomma}{\kern0pt}t{\isadigit{1}}{\isacharcomma}{\kern0pt}t{\isadigit{2}}{\isacharparenright}{\kern0pt}\ {\isasymequiv}\ frc{\isacharunderscore}{\kern0pt}at{\isacharparenleft}{\kern0pt}P{\isacharcomma}{\kern0pt}l{\isacharcomma}{\kern0pt}{\isasymlangle}{\isadigit{0}}{\isacharcomma}{\kern0pt}t{\isadigit{1}}{\isacharcomma}{\kern0pt}t{\isadigit{2}}{\isacharcomma}{\kern0pt}p{\isasymrangle}{\isacharparenright}{\kern0pt}\ {\isacharequal}{\kern0pt}\ {\isadigit{1}}{\isachardoublequoteclose}\isanewline
\isanewline
\isacommand{definition}\isamarkupfalse%
\isanewline
\ \ forces{\isacharunderscore}{\kern0pt}mem{\isacharprime}{\kern0pt}\ {\isacharcolon}{\kern0pt}{\isacharcolon}{\kern0pt}\ {\isachardoublequoteopen}{\isacharbrackleft}{\kern0pt}i{\isacharcomma}{\kern0pt}i{\isacharcomma}{\kern0pt}i{\isacharcomma}{\kern0pt}i{\isacharcomma}{\kern0pt}i{\isacharbrackright}{\kern0pt}\ {\isasymRightarrow}\ o{\isachardoublequoteclose}\ \isakeyword{where}\isanewline
\ \ {\isachardoublequoteopen}forces{\isacharunderscore}{\kern0pt}mem{\isacharprime}{\kern0pt}{\isacharparenleft}{\kern0pt}P{\isacharcomma}{\kern0pt}l{\isacharcomma}{\kern0pt}p{\isacharcomma}{\kern0pt}t{\isadigit{1}}{\isacharcomma}{\kern0pt}t{\isadigit{2}}{\isacharparenright}{\kern0pt}\ {\isasymequiv}\ frc{\isacharunderscore}{\kern0pt}at{\isacharparenleft}{\kern0pt}P{\isacharcomma}{\kern0pt}l{\isacharcomma}{\kern0pt}{\isasymlangle}{\isadigit{1}}{\isacharcomma}{\kern0pt}t{\isadigit{1}}{\isacharcomma}{\kern0pt}t{\isadigit{2}}{\isacharcomma}{\kern0pt}p{\isasymrangle}{\isacharparenright}{\kern0pt}\ {\isacharequal}{\kern0pt}\ {\isadigit{1}}{\isachardoublequoteclose}\isanewline
\isanewline
\isacommand{definition}\isamarkupfalse%
\isanewline
\ \ forces{\isacharunderscore}{\kern0pt}neq{\isacharprime}{\kern0pt}\ {\isacharcolon}{\kern0pt}{\isacharcolon}{\kern0pt}\ {\isachardoublequoteopen}{\isacharbrackleft}{\kern0pt}i{\isacharcomma}{\kern0pt}i{\isacharcomma}{\kern0pt}i{\isacharcomma}{\kern0pt}i{\isacharcomma}{\kern0pt}i{\isacharbrackright}{\kern0pt}\ {\isasymRightarrow}\ o{\isachardoublequoteclose}\ \isakeyword{where}\isanewline
\ \ {\isachardoublequoteopen}forces{\isacharunderscore}{\kern0pt}neq{\isacharprime}{\kern0pt}{\isacharparenleft}{\kern0pt}P{\isacharcomma}{\kern0pt}l{\isacharcomma}{\kern0pt}p{\isacharcomma}{\kern0pt}t{\isadigit{1}}{\isacharcomma}{\kern0pt}t{\isadigit{2}}{\isacharparenright}{\kern0pt}\ {\isasymequiv}\ {\isasymnot}\ {\isacharparenleft}{\kern0pt}{\isasymexists}q{\isasymin}P{\isachardot}{\kern0pt}\ {\isasymlangle}q{\isacharcomma}{\kern0pt}p{\isasymrangle}{\isasymin}l\ {\isasymand}\ forces{\isacharunderscore}{\kern0pt}eq{\isacharprime}{\kern0pt}{\isacharparenleft}{\kern0pt}P{\isacharcomma}{\kern0pt}l{\isacharcomma}{\kern0pt}q{\isacharcomma}{\kern0pt}t{\isadigit{1}}{\isacharcomma}{\kern0pt}t{\isadigit{2}}{\isacharparenright}{\kern0pt}{\isacharparenright}{\kern0pt}{\isachardoublequoteclose}\isanewline
\isanewline
\isacommand{definition}\isamarkupfalse%
\isanewline
\ \ forces{\isacharunderscore}{\kern0pt}nmem{\isacharprime}{\kern0pt}\ {\isacharcolon}{\kern0pt}{\isacharcolon}{\kern0pt}\ {\isachardoublequoteopen}{\isacharbrackleft}{\kern0pt}i{\isacharcomma}{\kern0pt}i{\isacharcomma}{\kern0pt}i{\isacharcomma}{\kern0pt}i{\isacharcomma}{\kern0pt}i{\isacharbrackright}{\kern0pt}\ {\isasymRightarrow}\ o{\isachardoublequoteclose}\ \isakeyword{where}\isanewline
\ \ {\isachardoublequoteopen}forces{\isacharunderscore}{\kern0pt}nmem{\isacharprime}{\kern0pt}{\isacharparenleft}{\kern0pt}P{\isacharcomma}{\kern0pt}l{\isacharcomma}{\kern0pt}p{\isacharcomma}{\kern0pt}t{\isadigit{1}}{\isacharcomma}{\kern0pt}t{\isadigit{2}}{\isacharparenright}{\kern0pt}\ {\isasymequiv}\ {\isasymnot}\ {\isacharparenleft}{\kern0pt}{\isasymexists}q{\isasymin}P{\isachardot}{\kern0pt}\ {\isasymlangle}q{\isacharcomma}{\kern0pt}p{\isasymrangle}{\isasymin}l\ {\isasymand}\ forces{\isacharunderscore}{\kern0pt}mem{\isacharprime}{\kern0pt}{\isacharparenleft}{\kern0pt}P{\isacharcomma}{\kern0pt}l{\isacharcomma}{\kern0pt}q{\isacharcomma}{\kern0pt}t{\isadigit{1}}{\isacharcomma}{\kern0pt}t{\isadigit{2}}{\isacharparenright}{\kern0pt}{\isacharparenright}{\kern0pt}{\isachardoublequoteclose}\isanewline
\isanewline
\isacommand{definition}\isamarkupfalse%
\isanewline
\ \ is{\isacharunderscore}{\kern0pt}forces{\isacharunderscore}{\kern0pt}eq{\isacharprime}{\kern0pt}\ {\isacharcolon}{\kern0pt}{\isacharcolon}{\kern0pt}\ {\isachardoublequoteopen}{\isacharbrackleft}{\kern0pt}i{\isasymRightarrow}o{\isacharcomma}{\kern0pt}i{\isacharcomma}{\kern0pt}i{\isacharcomma}{\kern0pt}i{\isacharcomma}{\kern0pt}i{\isacharcomma}{\kern0pt}i{\isacharbrackright}{\kern0pt}\ {\isasymRightarrow}\ o{\isachardoublequoteclose}\ \isakeyword{where}\isanewline
\ \ {\isachardoublequoteopen}is{\isacharunderscore}{\kern0pt}forces{\isacharunderscore}{\kern0pt}eq{\isacharprime}{\kern0pt}{\isacharparenleft}{\kern0pt}M{\isacharcomma}{\kern0pt}P{\isacharcomma}{\kern0pt}l{\isacharcomma}{\kern0pt}p{\isacharcomma}{\kern0pt}t{\isadigit{1}}{\isacharcomma}{\kern0pt}t{\isadigit{2}}{\isacharparenright}{\kern0pt}\ {\isasymequiv}\ {\isasymexists}o{\isacharbrackleft}{\kern0pt}M{\isacharbrackright}{\kern0pt}{\isachardot}{\kern0pt}\ {\isasymexists}z{\isacharbrackleft}{\kern0pt}M{\isacharbrackright}{\kern0pt}{\isachardot}{\kern0pt}\ {\isasymexists}t{\isacharbrackleft}{\kern0pt}M{\isacharbrackright}{\kern0pt}{\isachardot}{\kern0pt}\ number{\isadigit{1}}{\isacharparenleft}{\kern0pt}M{\isacharcomma}{\kern0pt}o{\isacharparenright}{\kern0pt}\ {\isasymand}\ empty{\isacharparenleft}{\kern0pt}M{\isacharcomma}{\kern0pt}z{\isacharparenright}{\kern0pt}\ {\isasymand}\isanewline
\ \ \ \ \ \ \ \ \ \ \ \ \ \ \ \ \ \ \ \ \ \ \ \ \ \ \ \ \ \ \ \ is{\isacharunderscore}{\kern0pt}tuple{\isacharparenleft}{\kern0pt}M{\isacharcomma}{\kern0pt}z{\isacharcomma}{\kern0pt}t{\isadigit{1}}{\isacharcomma}{\kern0pt}t{\isadigit{2}}{\isacharcomma}{\kern0pt}p{\isacharcomma}{\kern0pt}t{\isacharparenright}{\kern0pt}\ {\isasymand}\ is{\isacharunderscore}{\kern0pt}frc{\isacharunderscore}{\kern0pt}at{\isacharparenleft}{\kern0pt}M{\isacharcomma}{\kern0pt}P{\isacharcomma}{\kern0pt}l{\isacharcomma}{\kern0pt}t{\isacharcomma}{\kern0pt}o{\isacharparenright}{\kern0pt}{\isachardoublequoteclose}\isanewline
\isanewline
\isacommand{definition}\isamarkupfalse%
\isanewline
\ \ is{\isacharunderscore}{\kern0pt}forces{\isacharunderscore}{\kern0pt}mem{\isacharprime}{\kern0pt}\ {\isacharcolon}{\kern0pt}{\isacharcolon}{\kern0pt}\ {\isachardoublequoteopen}{\isacharbrackleft}{\kern0pt}i{\isasymRightarrow}o{\isacharcomma}{\kern0pt}i{\isacharcomma}{\kern0pt}i{\isacharcomma}{\kern0pt}i{\isacharcomma}{\kern0pt}i{\isacharcomma}{\kern0pt}i{\isacharbrackright}{\kern0pt}\ {\isasymRightarrow}\ o{\isachardoublequoteclose}\ \isakeyword{where}\isanewline
\ \ {\isachardoublequoteopen}is{\isacharunderscore}{\kern0pt}forces{\isacharunderscore}{\kern0pt}mem{\isacharprime}{\kern0pt}{\isacharparenleft}{\kern0pt}M{\isacharcomma}{\kern0pt}P{\isacharcomma}{\kern0pt}l{\isacharcomma}{\kern0pt}p{\isacharcomma}{\kern0pt}t{\isadigit{1}}{\isacharcomma}{\kern0pt}t{\isadigit{2}}{\isacharparenright}{\kern0pt}\ {\isasymequiv}\ {\isasymexists}o{\isacharbrackleft}{\kern0pt}M{\isacharbrackright}{\kern0pt}{\isachardot}{\kern0pt}\ {\isasymexists}t{\isacharbrackleft}{\kern0pt}M{\isacharbrackright}{\kern0pt}{\isachardot}{\kern0pt}\ number{\isadigit{1}}{\isacharparenleft}{\kern0pt}M{\isacharcomma}{\kern0pt}o{\isacharparenright}{\kern0pt}\ {\isasymand}\isanewline
\ \ \ \ \ \ \ \ \ \ \ \ \ \ \ \ \ \ \ \ \ \ \ \ \ \ \ \ \ \ \ \ is{\isacharunderscore}{\kern0pt}tuple{\isacharparenleft}{\kern0pt}M{\isacharcomma}{\kern0pt}o{\isacharcomma}{\kern0pt}t{\isadigit{1}}{\isacharcomma}{\kern0pt}t{\isadigit{2}}{\isacharcomma}{\kern0pt}p{\isacharcomma}{\kern0pt}t{\isacharparenright}{\kern0pt}\ {\isasymand}\ is{\isacharunderscore}{\kern0pt}frc{\isacharunderscore}{\kern0pt}at{\isacharparenleft}{\kern0pt}M{\isacharcomma}{\kern0pt}P{\isacharcomma}{\kern0pt}l{\isacharcomma}{\kern0pt}t{\isacharcomma}{\kern0pt}o{\isacharparenright}{\kern0pt}{\isachardoublequoteclose}\isanewline
\isanewline
\isacommand{definition}\isamarkupfalse%
\isanewline
\ \ is{\isacharunderscore}{\kern0pt}forces{\isacharunderscore}{\kern0pt}neq{\isacharprime}{\kern0pt}\ {\isacharcolon}{\kern0pt}{\isacharcolon}{\kern0pt}\ {\isachardoublequoteopen}{\isacharbrackleft}{\kern0pt}i{\isasymRightarrow}o{\isacharcomma}{\kern0pt}i{\isacharcomma}{\kern0pt}i{\isacharcomma}{\kern0pt}i{\isacharcomma}{\kern0pt}i{\isacharcomma}{\kern0pt}i{\isacharbrackright}{\kern0pt}\ {\isasymRightarrow}\ o{\isachardoublequoteclose}\ \isakeyword{where}\isanewline
\ \ {\isachardoublequoteopen}is{\isacharunderscore}{\kern0pt}forces{\isacharunderscore}{\kern0pt}neq{\isacharprime}{\kern0pt}{\isacharparenleft}{\kern0pt}M{\isacharcomma}{\kern0pt}P{\isacharcomma}{\kern0pt}l{\isacharcomma}{\kern0pt}p{\isacharcomma}{\kern0pt}t{\isadigit{1}}{\isacharcomma}{\kern0pt}t{\isadigit{2}}{\isacharparenright}{\kern0pt}\ {\isasymequiv}\isanewline
\ \ \ \ \ \ {\isasymnot}\ {\isacharparenleft}{\kern0pt}{\isasymexists}q{\isacharbrackleft}{\kern0pt}M{\isacharbrackright}{\kern0pt}{\isachardot}{\kern0pt}\ q{\isasymin}P\ {\isasymand}\ {\isacharparenleft}{\kern0pt}{\isasymexists}qp{\isacharbrackleft}{\kern0pt}M{\isacharbrackright}{\kern0pt}{\isachardot}{\kern0pt}\ pair{\isacharparenleft}{\kern0pt}M{\isacharcomma}{\kern0pt}q{\isacharcomma}{\kern0pt}p{\isacharcomma}{\kern0pt}qp{\isacharparenright}{\kern0pt}\ {\isasymand}\ qp{\isasymin}l\ {\isasymand}\ is{\isacharunderscore}{\kern0pt}forces{\isacharunderscore}{\kern0pt}eq{\isacharprime}{\kern0pt}{\isacharparenleft}{\kern0pt}M{\isacharcomma}{\kern0pt}P{\isacharcomma}{\kern0pt}l{\isacharcomma}{\kern0pt}q{\isacharcomma}{\kern0pt}t{\isadigit{1}}{\isacharcomma}{\kern0pt}t{\isadigit{2}}{\isacharparenright}{\kern0pt}{\isacharparenright}{\kern0pt}{\isacharparenright}{\kern0pt}{\isachardoublequoteclose}\isanewline
\isanewline
\isacommand{definition}\isamarkupfalse%
\isanewline
\ \ is{\isacharunderscore}{\kern0pt}forces{\isacharunderscore}{\kern0pt}nmem{\isacharprime}{\kern0pt}\ {\isacharcolon}{\kern0pt}{\isacharcolon}{\kern0pt}\ {\isachardoublequoteopen}{\isacharbrackleft}{\kern0pt}i{\isasymRightarrow}o{\isacharcomma}{\kern0pt}i{\isacharcomma}{\kern0pt}i{\isacharcomma}{\kern0pt}i{\isacharcomma}{\kern0pt}i{\isacharcomma}{\kern0pt}i{\isacharbrackright}{\kern0pt}\ {\isasymRightarrow}\ o{\isachardoublequoteclose}\ \isakeyword{where}\isanewline
\ \ {\isachardoublequoteopen}is{\isacharunderscore}{\kern0pt}forces{\isacharunderscore}{\kern0pt}nmem{\isacharprime}{\kern0pt}{\isacharparenleft}{\kern0pt}M{\isacharcomma}{\kern0pt}P{\isacharcomma}{\kern0pt}l{\isacharcomma}{\kern0pt}p{\isacharcomma}{\kern0pt}t{\isadigit{1}}{\isacharcomma}{\kern0pt}t{\isadigit{2}}{\isacharparenright}{\kern0pt}\ {\isasymequiv}\isanewline
\ \ \ \ \ \ {\isasymnot}\ {\isacharparenleft}{\kern0pt}{\isasymexists}q{\isacharbrackleft}{\kern0pt}M{\isacharbrackright}{\kern0pt}{\isachardot}{\kern0pt}\ {\isasymexists}qp{\isacharbrackleft}{\kern0pt}M{\isacharbrackright}{\kern0pt}{\isachardot}{\kern0pt}\ q{\isasymin}P\ {\isasymand}\ pair{\isacharparenleft}{\kern0pt}M{\isacharcomma}{\kern0pt}q{\isacharcomma}{\kern0pt}p{\isacharcomma}{\kern0pt}qp{\isacharparenright}{\kern0pt}\ {\isasymand}\ qp{\isasymin}l\ {\isasymand}\ is{\isacharunderscore}{\kern0pt}forces{\isacharunderscore}{\kern0pt}mem{\isacharprime}{\kern0pt}{\isacharparenleft}{\kern0pt}M{\isacharcomma}{\kern0pt}P{\isacharcomma}{\kern0pt}l{\isacharcomma}{\kern0pt}q{\isacharcomma}{\kern0pt}t{\isadigit{1}}{\isacharcomma}{\kern0pt}t{\isadigit{2}}{\isacharparenright}{\kern0pt}{\isacharparenright}{\kern0pt}{\isachardoublequoteclose}\isanewline
\isanewline
\isacommand{definition}\isamarkupfalse%
\isanewline
\ \ forces{\isacharunderscore}{\kern0pt}eq{\isacharunderscore}{\kern0pt}fm\ {\isacharcolon}{\kern0pt}{\isacharcolon}{\kern0pt}\ {\isachardoublequoteopen}{\isacharbrackleft}{\kern0pt}i{\isacharcomma}{\kern0pt}i{\isacharcomma}{\kern0pt}i{\isacharcomma}{\kern0pt}i{\isacharcomma}{\kern0pt}i{\isacharbrackright}{\kern0pt}\ {\isasymRightarrow}\ i{\isachardoublequoteclose}\ \isakeyword{where}\isanewline
\ \ {\isachardoublequoteopen}forces{\isacharunderscore}{\kern0pt}eq{\isacharunderscore}{\kern0pt}fm{\isacharparenleft}{\kern0pt}p{\isacharcomma}{\kern0pt}l{\isacharcomma}{\kern0pt}q{\isacharcomma}{\kern0pt}t{\isadigit{1}}{\isacharcomma}{\kern0pt}t{\isadigit{2}}{\isacharparenright}{\kern0pt}\ {\isasymequiv}\isanewline
\ \ \ \ \ Exists{\isacharparenleft}{\kern0pt}Exists{\isacharparenleft}{\kern0pt}Exists{\isacharparenleft}{\kern0pt}And{\isacharparenleft}{\kern0pt}number{\isadigit{1}}{\isacharunderscore}{\kern0pt}fm{\isacharparenleft}{\kern0pt}{\isadigit{2}}{\isacharparenright}{\kern0pt}{\isacharcomma}{\kern0pt}And{\isacharparenleft}{\kern0pt}empty{\isacharunderscore}{\kern0pt}fm{\isacharparenleft}{\kern0pt}{\isadigit{1}}{\isacharparenright}{\kern0pt}{\isacharcomma}{\kern0pt}\isanewline
\ \ \ \ \ \ \ \ \ \ \ \ \ \ And{\isacharparenleft}{\kern0pt}is{\isacharunderscore}{\kern0pt}tuple{\isacharunderscore}{\kern0pt}fm{\isacharparenleft}{\kern0pt}{\isadigit{1}}{\isacharcomma}{\kern0pt}t{\isadigit{1}}{\isacharhash}{\kern0pt}{\isacharplus}{\kern0pt}{\isadigit{3}}{\isacharcomma}{\kern0pt}t{\isadigit{2}}{\isacharhash}{\kern0pt}{\isacharplus}{\kern0pt}{\isadigit{3}}{\isacharcomma}{\kern0pt}q{\isacharhash}{\kern0pt}{\isacharplus}{\kern0pt}{\isadigit{3}}{\isacharcomma}{\kern0pt}{\isadigit{0}}{\isacharparenright}{\kern0pt}{\isacharcomma}{\kern0pt}frc{\isacharunderscore}{\kern0pt}at{\isacharunderscore}{\kern0pt}fm{\isacharparenleft}{\kern0pt}p{\isacharhash}{\kern0pt}{\isacharplus}{\kern0pt}{\isadigit{3}}{\isacharcomma}{\kern0pt}l{\isacharhash}{\kern0pt}{\isacharplus}{\kern0pt}{\isadigit{3}}{\isacharcomma}{\kern0pt}{\isadigit{0}}{\isacharcomma}{\kern0pt}{\isadigit{2}}{\isacharparenright}{\kern0pt}\ {\isacharparenright}{\kern0pt}{\isacharparenright}{\kern0pt}{\isacharparenright}{\kern0pt}{\isacharparenright}{\kern0pt}{\isacharparenright}{\kern0pt}{\isacharparenright}{\kern0pt}{\isachardoublequoteclose}\isanewline
\isanewline
\isacommand{definition}\isamarkupfalse%
\isanewline
\ \ forces{\isacharunderscore}{\kern0pt}mem{\isacharunderscore}{\kern0pt}fm\ {\isacharcolon}{\kern0pt}{\isacharcolon}{\kern0pt}\ {\isachardoublequoteopen}{\isacharbrackleft}{\kern0pt}i{\isacharcomma}{\kern0pt}i{\isacharcomma}{\kern0pt}i{\isacharcomma}{\kern0pt}i{\isacharcomma}{\kern0pt}i{\isacharbrackright}{\kern0pt}\ {\isasymRightarrow}\ i{\isachardoublequoteclose}\ \isakeyword{where}\isanewline
\ \ {\isachardoublequoteopen}forces{\isacharunderscore}{\kern0pt}mem{\isacharunderscore}{\kern0pt}fm{\isacharparenleft}{\kern0pt}p{\isacharcomma}{\kern0pt}l{\isacharcomma}{\kern0pt}q{\isacharcomma}{\kern0pt}t{\isadigit{1}}{\isacharcomma}{\kern0pt}t{\isadigit{2}}{\isacharparenright}{\kern0pt}\ {\isasymequiv}\ Exists{\isacharparenleft}{\kern0pt}Exists{\isacharparenleft}{\kern0pt}And{\isacharparenleft}{\kern0pt}number{\isadigit{1}}{\isacharunderscore}{\kern0pt}fm{\isacharparenleft}{\kern0pt}{\isadigit{1}}{\isacharparenright}{\kern0pt}{\isacharcomma}{\kern0pt}\isanewline
\ \ \ \ \ \ \ \ \ \ \ \ \ \ \ \ \ \ \ \ \ \ \ \ \ \ And{\isacharparenleft}{\kern0pt}is{\isacharunderscore}{\kern0pt}tuple{\isacharunderscore}{\kern0pt}fm{\isacharparenleft}{\kern0pt}{\isadigit{1}}{\isacharcomma}{\kern0pt}t{\isadigit{1}}{\isacharhash}{\kern0pt}{\isacharplus}{\kern0pt}{\isadigit{2}}{\isacharcomma}{\kern0pt}t{\isadigit{2}}{\isacharhash}{\kern0pt}{\isacharplus}{\kern0pt}{\isadigit{2}}{\isacharcomma}{\kern0pt}q{\isacharhash}{\kern0pt}{\isacharplus}{\kern0pt}{\isadigit{2}}{\isacharcomma}{\kern0pt}{\isadigit{0}}{\isacharparenright}{\kern0pt}{\isacharcomma}{\kern0pt}frc{\isacharunderscore}{\kern0pt}at{\isacharunderscore}{\kern0pt}fm{\isacharparenleft}{\kern0pt}p{\isacharhash}{\kern0pt}{\isacharplus}{\kern0pt}{\isadigit{2}}{\isacharcomma}{\kern0pt}l{\isacharhash}{\kern0pt}{\isacharplus}{\kern0pt}{\isadigit{2}}{\isacharcomma}{\kern0pt}{\isadigit{0}}{\isacharcomma}{\kern0pt}{\isadigit{1}}{\isacharparenright}{\kern0pt}{\isacharparenright}{\kern0pt}{\isacharparenright}{\kern0pt}{\isacharparenright}{\kern0pt}{\isacharparenright}{\kern0pt}{\isachardoublequoteclose}\isanewline
\isanewline
\isacommand{definition}\isamarkupfalse%
\isanewline
\ \ forces{\isacharunderscore}{\kern0pt}neq{\isacharunderscore}{\kern0pt}fm\ {\isacharcolon}{\kern0pt}{\isacharcolon}{\kern0pt}\ {\isachardoublequoteopen}{\isacharbrackleft}{\kern0pt}i{\isacharcomma}{\kern0pt}i{\isacharcomma}{\kern0pt}i{\isacharcomma}{\kern0pt}i{\isacharcomma}{\kern0pt}i{\isacharbrackright}{\kern0pt}\ {\isasymRightarrow}\ i{\isachardoublequoteclose}\ \isakeyword{where}\isanewline
\ \ {\isachardoublequoteopen}forces{\isacharunderscore}{\kern0pt}neq{\isacharunderscore}{\kern0pt}fm{\isacharparenleft}{\kern0pt}p{\isacharcomma}{\kern0pt}l{\isacharcomma}{\kern0pt}q{\isacharcomma}{\kern0pt}t{\isadigit{1}}{\isacharcomma}{\kern0pt}t{\isadigit{2}}{\isacharparenright}{\kern0pt}\ {\isasymequiv}\ Neg{\isacharparenleft}{\kern0pt}Exists{\isacharparenleft}{\kern0pt}Exists{\isacharparenleft}{\kern0pt}And{\isacharparenleft}{\kern0pt}Member{\isacharparenleft}{\kern0pt}{\isadigit{1}}{\isacharcomma}{\kern0pt}p{\isacharhash}{\kern0pt}{\isacharplus}{\kern0pt}{\isadigit{2}}{\isacharparenright}{\kern0pt}{\isacharcomma}{\kern0pt}\isanewline
\ \ \ \ \ And{\isacharparenleft}{\kern0pt}pair{\isacharunderscore}{\kern0pt}fm{\isacharparenleft}{\kern0pt}{\isadigit{1}}{\isacharcomma}{\kern0pt}q{\isacharhash}{\kern0pt}{\isacharplus}{\kern0pt}{\isadigit{2}}{\isacharcomma}{\kern0pt}{\isadigit{0}}{\isacharparenright}{\kern0pt}{\isacharcomma}{\kern0pt}And{\isacharparenleft}{\kern0pt}Member{\isacharparenleft}{\kern0pt}{\isadigit{0}}{\isacharcomma}{\kern0pt}l{\isacharhash}{\kern0pt}{\isacharplus}{\kern0pt}{\isadigit{2}}{\isacharparenright}{\kern0pt}{\isacharcomma}{\kern0pt}forces{\isacharunderscore}{\kern0pt}eq{\isacharunderscore}{\kern0pt}fm{\isacharparenleft}{\kern0pt}p{\isacharhash}{\kern0pt}{\isacharplus}{\kern0pt}{\isadigit{2}}{\isacharcomma}{\kern0pt}l{\isacharhash}{\kern0pt}{\isacharplus}{\kern0pt}{\isadigit{2}}{\isacharcomma}{\kern0pt}{\isadigit{1}}{\isacharcomma}{\kern0pt}t{\isadigit{1}}{\isacharhash}{\kern0pt}{\isacharplus}{\kern0pt}{\isadigit{2}}{\isacharcomma}{\kern0pt}t{\isadigit{2}}{\isacharhash}{\kern0pt}{\isacharplus}{\kern0pt}{\isadigit{2}}{\isacharparenright}{\kern0pt}{\isacharparenright}{\kern0pt}{\isacharparenright}{\kern0pt}{\isacharparenright}{\kern0pt}{\isacharparenright}{\kern0pt}{\isacharparenright}{\kern0pt}{\isacharparenright}{\kern0pt}{\isachardoublequoteclose}\isanewline
\isanewline
\isacommand{definition}\isamarkupfalse%
\isanewline
\ \ forces{\isacharunderscore}{\kern0pt}nmem{\isacharunderscore}{\kern0pt}fm\ {\isacharcolon}{\kern0pt}{\isacharcolon}{\kern0pt}\ {\isachardoublequoteopen}{\isacharbrackleft}{\kern0pt}i{\isacharcomma}{\kern0pt}i{\isacharcomma}{\kern0pt}i{\isacharcomma}{\kern0pt}i{\isacharcomma}{\kern0pt}i{\isacharbrackright}{\kern0pt}\ {\isasymRightarrow}\ i{\isachardoublequoteclose}\ \isakeyword{where}\isanewline
\ \ {\isachardoublequoteopen}forces{\isacharunderscore}{\kern0pt}nmem{\isacharunderscore}{\kern0pt}fm{\isacharparenleft}{\kern0pt}p{\isacharcomma}{\kern0pt}l{\isacharcomma}{\kern0pt}q{\isacharcomma}{\kern0pt}t{\isadigit{1}}{\isacharcomma}{\kern0pt}t{\isadigit{2}}{\isacharparenright}{\kern0pt}\ {\isasymequiv}\ Neg{\isacharparenleft}{\kern0pt}Exists{\isacharparenleft}{\kern0pt}Exists{\isacharparenleft}{\kern0pt}And{\isacharparenleft}{\kern0pt}Member{\isacharparenleft}{\kern0pt}{\isadigit{1}}{\isacharcomma}{\kern0pt}p{\isacharhash}{\kern0pt}{\isacharplus}{\kern0pt}{\isadigit{2}}{\isacharparenright}{\kern0pt}{\isacharcomma}{\kern0pt}\isanewline
\ \ \ \ \ And{\isacharparenleft}{\kern0pt}pair{\isacharunderscore}{\kern0pt}fm{\isacharparenleft}{\kern0pt}{\isadigit{1}}{\isacharcomma}{\kern0pt}q{\isacharhash}{\kern0pt}{\isacharplus}{\kern0pt}{\isadigit{2}}{\isacharcomma}{\kern0pt}{\isadigit{0}}{\isacharparenright}{\kern0pt}{\isacharcomma}{\kern0pt}And{\isacharparenleft}{\kern0pt}Member{\isacharparenleft}{\kern0pt}{\isadigit{0}}{\isacharcomma}{\kern0pt}l{\isacharhash}{\kern0pt}{\isacharplus}{\kern0pt}{\isadigit{2}}{\isacharparenright}{\kern0pt}{\isacharcomma}{\kern0pt}forces{\isacharunderscore}{\kern0pt}mem{\isacharunderscore}{\kern0pt}fm{\isacharparenleft}{\kern0pt}p{\isacharhash}{\kern0pt}{\isacharplus}{\kern0pt}{\isadigit{2}}{\isacharcomma}{\kern0pt}l{\isacharhash}{\kern0pt}{\isacharplus}{\kern0pt}{\isadigit{2}}{\isacharcomma}{\kern0pt}{\isadigit{1}}{\isacharcomma}{\kern0pt}t{\isadigit{1}}{\isacharhash}{\kern0pt}{\isacharplus}{\kern0pt}{\isadigit{2}}{\isacharcomma}{\kern0pt}t{\isadigit{2}}{\isacharhash}{\kern0pt}{\isacharplus}{\kern0pt}{\isadigit{2}}{\isacharparenright}{\kern0pt}{\isacharparenright}{\kern0pt}{\isacharparenright}{\kern0pt}{\isacharparenright}{\kern0pt}{\isacharparenright}{\kern0pt}{\isacharparenright}{\kern0pt}{\isacharparenright}{\kern0pt}{\isachardoublequoteclose}\isanewline
\isanewline
\isanewline
\isacommand{lemma}\isamarkupfalse%
\ forces{\isacharunderscore}{\kern0pt}eq{\isacharunderscore}{\kern0pt}fm{\isacharunderscore}{\kern0pt}type\ {\isacharbrackleft}{\kern0pt}TC{\isacharbrackright}{\kern0pt}{\isacharcolon}{\kern0pt}\isanewline
\ \ {\isachardoublequoteopen}{\isasymlbrakk}\ p{\isasymin}nat{\isacharsemicolon}{\kern0pt}l{\isasymin}nat{\isacharsemicolon}{\kern0pt}q{\isasymin}nat{\isacharsemicolon}{\kern0pt}t{\isadigit{1}}{\isasymin}nat{\isacharsemicolon}{\kern0pt}t{\isadigit{2}}{\isasymin}nat{\isasymrbrakk}\ {\isasymLongrightarrow}\ forces{\isacharunderscore}{\kern0pt}eq{\isacharunderscore}{\kern0pt}fm{\isacharparenleft}{\kern0pt}p{\isacharcomma}{\kern0pt}l{\isacharcomma}{\kern0pt}q{\isacharcomma}{\kern0pt}t{\isadigit{1}}{\isacharcomma}{\kern0pt}t{\isadigit{2}}{\isacharparenright}{\kern0pt}\ {\isasymin}\ formula{\isachardoublequoteclose}\isanewline
%
\isadelimproof
\ \ %
\endisadelimproof
%
\isatagproof
\isacommand{unfolding}\isamarkupfalse%
\ forces{\isacharunderscore}{\kern0pt}eq{\isacharunderscore}{\kern0pt}fm{\isacharunderscore}{\kern0pt}def\isanewline
\ \ \isacommand{by}\isamarkupfalse%
\ simp%
\endisatagproof
{\isafoldproof}%
%
\isadelimproof
\isanewline
%
\endisadelimproof
\isanewline
\isacommand{lemma}\isamarkupfalse%
\ forces{\isacharunderscore}{\kern0pt}mem{\isacharunderscore}{\kern0pt}fm{\isacharunderscore}{\kern0pt}type\ {\isacharbrackleft}{\kern0pt}TC{\isacharbrackright}{\kern0pt}{\isacharcolon}{\kern0pt}\isanewline
\ \ {\isachardoublequoteopen}{\isasymlbrakk}\ p{\isasymin}nat{\isacharsemicolon}{\kern0pt}l{\isasymin}nat{\isacharsemicolon}{\kern0pt}q{\isasymin}nat{\isacharsemicolon}{\kern0pt}t{\isadigit{1}}{\isasymin}nat{\isacharsemicolon}{\kern0pt}t{\isadigit{2}}{\isasymin}nat{\isasymrbrakk}\ {\isasymLongrightarrow}\ forces{\isacharunderscore}{\kern0pt}mem{\isacharunderscore}{\kern0pt}fm{\isacharparenleft}{\kern0pt}p{\isacharcomma}{\kern0pt}l{\isacharcomma}{\kern0pt}q{\isacharcomma}{\kern0pt}t{\isadigit{1}}{\isacharcomma}{\kern0pt}t{\isadigit{2}}{\isacharparenright}{\kern0pt}\ {\isasymin}\ formula{\isachardoublequoteclose}\isanewline
%
\isadelimproof
\ \ %
\endisadelimproof
%
\isatagproof
\isacommand{unfolding}\isamarkupfalse%
\ forces{\isacharunderscore}{\kern0pt}mem{\isacharunderscore}{\kern0pt}fm{\isacharunderscore}{\kern0pt}def\isanewline
\ \ \isacommand{by}\isamarkupfalse%
\ simp%
\endisatagproof
{\isafoldproof}%
%
\isadelimproof
\isanewline
%
\endisadelimproof
\isanewline
\isacommand{lemma}\isamarkupfalse%
\ forces{\isacharunderscore}{\kern0pt}neq{\isacharunderscore}{\kern0pt}fm{\isacharunderscore}{\kern0pt}type\ {\isacharbrackleft}{\kern0pt}TC{\isacharbrackright}{\kern0pt}{\isacharcolon}{\kern0pt}\isanewline
\ \ {\isachardoublequoteopen}{\isasymlbrakk}\ p{\isasymin}nat{\isacharsemicolon}{\kern0pt}l{\isasymin}nat{\isacharsemicolon}{\kern0pt}q{\isasymin}nat{\isacharsemicolon}{\kern0pt}t{\isadigit{1}}{\isasymin}nat{\isacharsemicolon}{\kern0pt}t{\isadigit{2}}{\isasymin}nat{\isasymrbrakk}\ {\isasymLongrightarrow}\ forces{\isacharunderscore}{\kern0pt}neq{\isacharunderscore}{\kern0pt}fm{\isacharparenleft}{\kern0pt}p{\isacharcomma}{\kern0pt}l{\isacharcomma}{\kern0pt}q{\isacharcomma}{\kern0pt}t{\isadigit{1}}{\isacharcomma}{\kern0pt}t{\isadigit{2}}{\isacharparenright}{\kern0pt}\ {\isasymin}\ formula{\isachardoublequoteclose}\isanewline
%
\isadelimproof
\ \ %
\endisadelimproof
%
\isatagproof
\isacommand{unfolding}\isamarkupfalse%
\ forces{\isacharunderscore}{\kern0pt}neq{\isacharunderscore}{\kern0pt}fm{\isacharunderscore}{\kern0pt}def\isanewline
\ \ \isacommand{by}\isamarkupfalse%
\ simp%
\endisatagproof
{\isafoldproof}%
%
\isadelimproof
\isanewline
%
\endisadelimproof
\isanewline
\isacommand{lemma}\isamarkupfalse%
\ forces{\isacharunderscore}{\kern0pt}nmem{\isacharunderscore}{\kern0pt}fm{\isacharunderscore}{\kern0pt}type\ {\isacharbrackleft}{\kern0pt}TC{\isacharbrackright}{\kern0pt}{\isacharcolon}{\kern0pt}\isanewline
\ \ {\isachardoublequoteopen}{\isasymlbrakk}\ p{\isasymin}nat{\isacharsemicolon}{\kern0pt}l{\isasymin}nat{\isacharsemicolon}{\kern0pt}q{\isasymin}nat{\isacharsemicolon}{\kern0pt}t{\isadigit{1}}{\isasymin}nat{\isacharsemicolon}{\kern0pt}t{\isadigit{2}}{\isasymin}nat{\isasymrbrakk}\ {\isasymLongrightarrow}\ forces{\isacharunderscore}{\kern0pt}nmem{\isacharunderscore}{\kern0pt}fm{\isacharparenleft}{\kern0pt}p{\isacharcomma}{\kern0pt}l{\isacharcomma}{\kern0pt}q{\isacharcomma}{\kern0pt}t{\isadigit{1}}{\isacharcomma}{\kern0pt}t{\isadigit{2}}{\isacharparenright}{\kern0pt}\ {\isasymin}\ formula{\isachardoublequoteclose}\isanewline
%
\isadelimproof
\ \ %
\endisadelimproof
%
\isatagproof
\isacommand{unfolding}\isamarkupfalse%
\ forces{\isacharunderscore}{\kern0pt}nmem{\isacharunderscore}{\kern0pt}fm{\isacharunderscore}{\kern0pt}def\isanewline
\ \ \isacommand{by}\isamarkupfalse%
\ simp%
\endisatagproof
{\isafoldproof}%
%
\isadelimproof
\isanewline
%
\endisadelimproof
\isanewline
\isacommand{lemma}\isamarkupfalse%
\ arity{\isacharunderscore}{\kern0pt}forces{\isacharunderscore}{\kern0pt}eq{\isacharunderscore}{\kern0pt}fm\ {\isacharcolon}{\kern0pt}\isanewline
\ \ {\isachardoublequoteopen}p{\isasymin}nat\ {\isasymLongrightarrow}\ l{\isasymin}nat\ {\isasymLongrightarrow}\ q{\isasymin}nat\ {\isasymLongrightarrow}\ t{\isadigit{1}}\ {\isasymin}\ nat\ {\isasymLongrightarrow}\ t{\isadigit{2}}\ {\isasymin}\ nat\ {\isasymLongrightarrow}\isanewline
\ \ \ arity{\isacharparenleft}{\kern0pt}forces{\isacharunderscore}{\kern0pt}eq{\isacharunderscore}{\kern0pt}fm{\isacharparenleft}{\kern0pt}p{\isacharcomma}{\kern0pt}l{\isacharcomma}{\kern0pt}q{\isacharcomma}{\kern0pt}t{\isadigit{1}}{\isacharcomma}{\kern0pt}t{\isadigit{2}}{\isacharparenright}{\kern0pt}{\isacharparenright}{\kern0pt}\ {\isacharequal}{\kern0pt}\ succ{\isacharparenleft}{\kern0pt}t{\isadigit{1}}{\isacharparenright}{\kern0pt}\ {\isasymunion}\ succ{\isacharparenleft}{\kern0pt}t{\isadigit{2}}{\isacharparenright}{\kern0pt}\ {\isasymunion}\ succ{\isacharparenleft}{\kern0pt}q{\isacharparenright}{\kern0pt}\ {\isasymunion}\ succ{\isacharparenleft}{\kern0pt}p{\isacharparenright}{\kern0pt}\ {\isasymunion}\ succ{\isacharparenleft}{\kern0pt}l{\isacharparenright}{\kern0pt}{\isachardoublequoteclose}\isanewline
%
\isadelimproof
\ \ %
\endisadelimproof
%
\isatagproof
\isacommand{unfolding}\isamarkupfalse%
\ forces{\isacharunderscore}{\kern0pt}eq{\isacharunderscore}{\kern0pt}fm{\isacharunderscore}{\kern0pt}def\isanewline
\ \ \isacommand{using}\isamarkupfalse%
\ number{\isadigit{1}}arity{\isacharunderscore}{\kern0pt}{\isacharunderscore}{\kern0pt}fm\ arity{\isacharunderscore}{\kern0pt}empty{\isacharunderscore}{\kern0pt}fm\ arity{\isacharunderscore}{\kern0pt}is{\isacharunderscore}{\kern0pt}tuple{\isacharunderscore}{\kern0pt}fm\ arity{\isacharunderscore}{\kern0pt}frc{\isacharunderscore}{\kern0pt}at{\isacharunderscore}{\kern0pt}fm\isanewline
\ \ \ \ pred{\isacharunderscore}{\kern0pt}Un{\isacharunderscore}{\kern0pt}distrib\isanewline
\ \ \isacommand{by}\isamarkupfalse%
\ auto%
\endisatagproof
{\isafoldproof}%
%
\isadelimproof
\isanewline
%
\endisadelimproof
\isanewline
\isacommand{lemma}\isamarkupfalse%
\ arity{\isacharunderscore}{\kern0pt}forces{\isacharunderscore}{\kern0pt}mem{\isacharunderscore}{\kern0pt}fm\ {\isacharcolon}{\kern0pt}\isanewline
\ \ {\isachardoublequoteopen}p{\isasymin}nat\ {\isasymLongrightarrow}\ l{\isasymin}nat\ {\isasymLongrightarrow}\ q{\isasymin}nat\ {\isasymLongrightarrow}\ t{\isadigit{1}}\ {\isasymin}\ nat\ {\isasymLongrightarrow}\ t{\isadigit{2}}\ {\isasymin}\ nat\ {\isasymLongrightarrow}\isanewline
\ \ \ arity{\isacharparenleft}{\kern0pt}forces{\isacharunderscore}{\kern0pt}mem{\isacharunderscore}{\kern0pt}fm{\isacharparenleft}{\kern0pt}p{\isacharcomma}{\kern0pt}l{\isacharcomma}{\kern0pt}q{\isacharcomma}{\kern0pt}t{\isadigit{1}}{\isacharcomma}{\kern0pt}t{\isadigit{2}}{\isacharparenright}{\kern0pt}{\isacharparenright}{\kern0pt}\ {\isacharequal}{\kern0pt}\ succ{\isacharparenleft}{\kern0pt}t{\isadigit{1}}{\isacharparenright}{\kern0pt}\ {\isasymunion}\ succ{\isacharparenleft}{\kern0pt}t{\isadigit{2}}{\isacharparenright}{\kern0pt}\ {\isasymunion}\ succ{\isacharparenleft}{\kern0pt}q{\isacharparenright}{\kern0pt}\ {\isasymunion}\ succ{\isacharparenleft}{\kern0pt}p{\isacharparenright}{\kern0pt}\ {\isasymunion}\ succ{\isacharparenleft}{\kern0pt}l{\isacharparenright}{\kern0pt}{\isachardoublequoteclose}\isanewline
%
\isadelimproof
\ \ %
\endisadelimproof
%
\isatagproof
\isacommand{unfolding}\isamarkupfalse%
\ forces{\isacharunderscore}{\kern0pt}mem{\isacharunderscore}{\kern0pt}fm{\isacharunderscore}{\kern0pt}def\isanewline
\ \ \isacommand{using}\isamarkupfalse%
\ number{\isadigit{1}}arity{\isacharunderscore}{\kern0pt}{\isacharunderscore}{\kern0pt}fm\ arity{\isacharunderscore}{\kern0pt}empty{\isacharunderscore}{\kern0pt}fm\ arity{\isacharunderscore}{\kern0pt}is{\isacharunderscore}{\kern0pt}tuple{\isacharunderscore}{\kern0pt}fm\ arity{\isacharunderscore}{\kern0pt}frc{\isacharunderscore}{\kern0pt}at{\isacharunderscore}{\kern0pt}fm\isanewline
\ \ \ \ pred{\isacharunderscore}{\kern0pt}Un{\isacharunderscore}{\kern0pt}distrib\isanewline
\ \ \isacommand{by}\isamarkupfalse%
\ auto%
\endisatagproof
{\isafoldproof}%
%
\isadelimproof
\isanewline
%
\endisadelimproof
\isanewline
\isacommand{lemma}\isamarkupfalse%
\ sats{\isacharunderscore}{\kern0pt}forces{\isacharunderscore}{\kern0pt}eq{\isacharprime}{\kern0pt}{\isacharunderscore}{\kern0pt}fm{\isacharcolon}{\kern0pt}\isanewline
\ \ \isakeyword{assumes}\ \ {\isachardoublequoteopen}p{\isasymin}nat{\isachardoublequoteclose}\ {\isachardoublequoteopen}l{\isasymin}nat{\isachardoublequoteclose}\ {\isachardoublequoteopen}q{\isasymin}nat{\isachardoublequoteclose}\ {\isachardoublequoteopen}t{\isadigit{1}}{\isasymin}nat{\isachardoublequoteclose}\ {\isachardoublequoteopen}t{\isadigit{2}}{\isasymin}nat{\isachardoublequoteclose}\ \ {\isachardoublequoteopen}env{\isasymin}list{\isacharparenleft}{\kern0pt}M{\isacharparenright}{\kern0pt}{\isachardoublequoteclose}\isanewline
\ \ \isakeyword{shows}\ {\isachardoublequoteopen}sats{\isacharparenleft}{\kern0pt}M{\isacharcomma}{\kern0pt}forces{\isacharunderscore}{\kern0pt}eq{\isacharunderscore}{\kern0pt}fm{\isacharparenleft}{\kern0pt}p{\isacharcomma}{\kern0pt}l{\isacharcomma}{\kern0pt}q{\isacharcomma}{\kern0pt}t{\isadigit{1}}{\isacharcomma}{\kern0pt}t{\isadigit{2}}{\isacharparenright}{\kern0pt}{\isacharcomma}{\kern0pt}env{\isacharparenright}{\kern0pt}\ {\isasymlongleftrightarrow}\isanewline
\ \ \ \ \ \ \ \ \ is{\isacharunderscore}{\kern0pt}forces{\isacharunderscore}{\kern0pt}eq{\isacharprime}{\kern0pt}{\isacharparenleft}{\kern0pt}{\isacharhash}{\kern0pt}{\isacharhash}{\kern0pt}M{\isacharcomma}{\kern0pt}nth{\isacharparenleft}{\kern0pt}p{\isacharcomma}{\kern0pt}env{\isacharparenright}{\kern0pt}{\isacharcomma}{\kern0pt}nth{\isacharparenleft}{\kern0pt}l{\isacharcomma}{\kern0pt}env{\isacharparenright}{\kern0pt}{\isacharcomma}{\kern0pt}nth{\isacharparenleft}{\kern0pt}q{\isacharcomma}{\kern0pt}env{\isacharparenright}{\kern0pt}{\isacharcomma}{\kern0pt}nth{\isacharparenleft}{\kern0pt}t{\isadigit{1}}{\isacharcomma}{\kern0pt}env{\isacharparenright}{\kern0pt}{\isacharcomma}{\kern0pt}nth{\isacharparenleft}{\kern0pt}t{\isadigit{2}}{\isacharcomma}{\kern0pt}env{\isacharparenright}{\kern0pt}{\isacharparenright}{\kern0pt}{\isachardoublequoteclose}\isanewline
%
\isadelimproof
\ \ %
\endisadelimproof
%
\isatagproof
\isacommand{unfolding}\isamarkupfalse%
\ forces{\isacharunderscore}{\kern0pt}eq{\isacharunderscore}{\kern0pt}fm{\isacharunderscore}{\kern0pt}def\ is{\isacharunderscore}{\kern0pt}forces{\isacharunderscore}{\kern0pt}eq{\isacharprime}{\kern0pt}{\isacharunderscore}{\kern0pt}def\ \isacommand{using}\isamarkupfalse%
\ assms\ sats{\isacharunderscore}{\kern0pt}is{\isacharunderscore}{\kern0pt}tuple{\isacharunderscore}{\kern0pt}fm\ \ sats{\isacharunderscore}{\kern0pt}frc{\isacharunderscore}{\kern0pt}at{\isacharunderscore}{\kern0pt}fm\isanewline
\ \ \isacommand{by}\isamarkupfalse%
\ simp%
\endisatagproof
{\isafoldproof}%
%
\isadelimproof
\isanewline
%
\endisadelimproof
\isanewline
\isacommand{lemma}\isamarkupfalse%
\ sats{\isacharunderscore}{\kern0pt}forces{\isacharunderscore}{\kern0pt}mem{\isacharprime}{\kern0pt}{\isacharunderscore}{\kern0pt}fm{\isacharcolon}{\kern0pt}\isanewline
\ \ \isakeyword{assumes}\ \ {\isachardoublequoteopen}p{\isasymin}nat{\isachardoublequoteclose}\ {\isachardoublequoteopen}l{\isasymin}nat{\isachardoublequoteclose}\ {\isachardoublequoteopen}q{\isasymin}nat{\isachardoublequoteclose}\ {\isachardoublequoteopen}t{\isadigit{1}}{\isasymin}nat{\isachardoublequoteclose}\ {\isachardoublequoteopen}t{\isadigit{2}}{\isasymin}nat{\isachardoublequoteclose}\ \ {\isachardoublequoteopen}env{\isasymin}list{\isacharparenleft}{\kern0pt}M{\isacharparenright}{\kern0pt}{\isachardoublequoteclose}\isanewline
\ \ \isakeyword{shows}\ {\isachardoublequoteopen}sats{\isacharparenleft}{\kern0pt}M{\isacharcomma}{\kern0pt}forces{\isacharunderscore}{\kern0pt}mem{\isacharunderscore}{\kern0pt}fm{\isacharparenleft}{\kern0pt}p{\isacharcomma}{\kern0pt}l{\isacharcomma}{\kern0pt}q{\isacharcomma}{\kern0pt}t{\isadigit{1}}{\isacharcomma}{\kern0pt}t{\isadigit{2}}{\isacharparenright}{\kern0pt}{\isacharcomma}{\kern0pt}env{\isacharparenright}{\kern0pt}\ {\isasymlongleftrightarrow}\isanewline
\ \ \ \ \ \ \ \ \ \ \ \ \ is{\isacharunderscore}{\kern0pt}forces{\isacharunderscore}{\kern0pt}mem{\isacharprime}{\kern0pt}{\isacharparenleft}{\kern0pt}{\isacharhash}{\kern0pt}{\isacharhash}{\kern0pt}M{\isacharcomma}{\kern0pt}nth{\isacharparenleft}{\kern0pt}p{\isacharcomma}{\kern0pt}env{\isacharparenright}{\kern0pt}{\isacharcomma}{\kern0pt}nth{\isacharparenleft}{\kern0pt}l{\isacharcomma}{\kern0pt}env{\isacharparenright}{\kern0pt}{\isacharcomma}{\kern0pt}nth{\isacharparenleft}{\kern0pt}q{\isacharcomma}{\kern0pt}env{\isacharparenright}{\kern0pt}{\isacharcomma}{\kern0pt}nth{\isacharparenleft}{\kern0pt}t{\isadigit{1}}{\isacharcomma}{\kern0pt}env{\isacharparenright}{\kern0pt}{\isacharcomma}{\kern0pt}nth{\isacharparenleft}{\kern0pt}t{\isadigit{2}}{\isacharcomma}{\kern0pt}env{\isacharparenright}{\kern0pt}{\isacharparenright}{\kern0pt}{\isachardoublequoteclose}\isanewline
%
\isadelimproof
\ \ %
\endisadelimproof
%
\isatagproof
\isacommand{unfolding}\isamarkupfalse%
\ forces{\isacharunderscore}{\kern0pt}mem{\isacharunderscore}{\kern0pt}fm{\isacharunderscore}{\kern0pt}def\ is{\isacharunderscore}{\kern0pt}forces{\isacharunderscore}{\kern0pt}mem{\isacharprime}{\kern0pt}{\isacharunderscore}{\kern0pt}def\ \isacommand{using}\isamarkupfalse%
\ assms\ sats{\isacharunderscore}{\kern0pt}is{\isacharunderscore}{\kern0pt}tuple{\isacharunderscore}{\kern0pt}fm\ sats{\isacharunderscore}{\kern0pt}frc{\isacharunderscore}{\kern0pt}at{\isacharunderscore}{\kern0pt}fm\isanewline
\ \ \isacommand{by}\isamarkupfalse%
\ simp%
\endisatagproof
{\isafoldproof}%
%
\isadelimproof
\isanewline
%
\endisadelimproof
\isanewline
\isacommand{lemma}\isamarkupfalse%
\ sats{\isacharunderscore}{\kern0pt}forces{\isacharunderscore}{\kern0pt}neq{\isacharprime}{\kern0pt}{\isacharunderscore}{\kern0pt}fm{\isacharcolon}{\kern0pt}\isanewline
\ \ \isakeyword{assumes}\ \ {\isachardoublequoteopen}p{\isasymin}nat{\isachardoublequoteclose}\ {\isachardoublequoteopen}l{\isasymin}nat{\isachardoublequoteclose}\ {\isachardoublequoteopen}q{\isasymin}nat{\isachardoublequoteclose}\ {\isachardoublequoteopen}t{\isadigit{1}}{\isasymin}nat{\isachardoublequoteclose}\ {\isachardoublequoteopen}t{\isadigit{2}}{\isasymin}nat{\isachardoublequoteclose}\ \ {\isachardoublequoteopen}env{\isasymin}list{\isacharparenleft}{\kern0pt}M{\isacharparenright}{\kern0pt}{\isachardoublequoteclose}\isanewline
\ \ \isakeyword{shows}\ {\isachardoublequoteopen}sats{\isacharparenleft}{\kern0pt}M{\isacharcomma}{\kern0pt}forces{\isacharunderscore}{\kern0pt}neq{\isacharunderscore}{\kern0pt}fm{\isacharparenleft}{\kern0pt}p{\isacharcomma}{\kern0pt}l{\isacharcomma}{\kern0pt}q{\isacharcomma}{\kern0pt}t{\isadigit{1}}{\isacharcomma}{\kern0pt}t{\isadigit{2}}{\isacharparenright}{\kern0pt}{\isacharcomma}{\kern0pt}env{\isacharparenright}{\kern0pt}\ {\isasymlongleftrightarrow}\isanewline
\ \ \ \ \ \ \ \ \ \ \ \ \ is{\isacharunderscore}{\kern0pt}forces{\isacharunderscore}{\kern0pt}neq{\isacharprime}{\kern0pt}{\isacharparenleft}{\kern0pt}{\isacharhash}{\kern0pt}{\isacharhash}{\kern0pt}M{\isacharcomma}{\kern0pt}nth{\isacharparenleft}{\kern0pt}p{\isacharcomma}{\kern0pt}env{\isacharparenright}{\kern0pt}{\isacharcomma}{\kern0pt}nth{\isacharparenleft}{\kern0pt}l{\isacharcomma}{\kern0pt}env{\isacharparenright}{\kern0pt}{\isacharcomma}{\kern0pt}nth{\isacharparenleft}{\kern0pt}q{\isacharcomma}{\kern0pt}env{\isacharparenright}{\kern0pt}{\isacharcomma}{\kern0pt}nth{\isacharparenleft}{\kern0pt}t{\isadigit{1}}{\isacharcomma}{\kern0pt}env{\isacharparenright}{\kern0pt}{\isacharcomma}{\kern0pt}nth{\isacharparenleft}{\kern0pt}t{\isadigit{2}}{\isacharcomma}{\kern0pt}env{\isacharparenright}{\kern0pt}{\isacharparenright}{\kern0pt}{\isachardoublequoteclose}\isanewline
%
\isadelimproof
\ \ %
\endisadelimproof
%
\isatagproof
\isacommand{unfolding}\isamarkupfalse%
\ forces{\isacharunderscore}{\kern0pt}neq{\isacharunderscore}{\kern0pt}fm{\isacharunderscore}{\kern0pt}def\ is{\isacharunderscore}{\kern0pt}forces{\isacharunderscore}{\kern0pt}neq{\isacharprime}{\kern0pt}{\isacharunderscore}{\kern0pt}def\isanewline
\ \ \isacommand{using}\isamarkupfalse%
\ assms\ sats{\isacharunderscore}{\kern0pt}forces{\isacharunderscore}{\kern0pt}eq{\isacharprime}{\kern0pt}{\isacharunderscore}{\kern0pt}fm\ sats{\isacharunderscore}{\kern0pt}is{\isacharunderscore}{\kern0pt}tuple{\isacharunderscore}{\kern0pt}fm\ sats{\isacharunderscore}{\kern0pt}frc{\isacharunderscore}{\kern0pt}at{\isacharunderscore}{\kern0pt}fm\isanewline
\ \ \isacommand{by}\isamarkupfalse%
\ simp%
\endisatagproof
{\isafoldproof}%
%
\isadelimproof
\isanewline
%
\endisadelimproof
\isanewline
\isacommand{lemma}\isamarkupfalse%
\ sats{\isacharunderscore}{\kern0pt}forces{\isacharunderscore}{\kern0pt}nmem{\isacharprime}{\kern0pt}{\isacharunderscore}{\kern0pt}fm{\isacharcolon}{\kern0pt}\isanewline
\ \ \isakeyword{assumes}\ \ {\isachardoublequoteopen}p{\isasymin}nat{\isachardoublequoteclose}\ {\isachardoublequoteopen}l{\isasymin}nat{\isachardoublequoteclose}\ {\isachardoublequoteopen}q{\isasymin}nat{\isachardoublequoteclose}\ {\isachardoublequoteopen}t{\isadigit{1}}{\isasymin}nat{\isachardoublequoteclose}\ {\isachardoublequoteopen}t{\isadigit{2}}{\isasymin}nat{\isachardoublequoteclose}\ \ {\isachardoublequoteopen}env{\isasymin}list{\isacharparenleft}{\kern0pt}M{\isacharparenright}{\kern0pt}{\isachardoublequoteclose}\isanewline
\ \ \isakeyword{shows}\ {\isachardoublequoteopen}sats{\isacharparenleft}{\kern0pt}M{\isacharcomma}{\kern0pt}forces{\isacharunderscore}{\kern0pt}nmem{\isacharunderscore}{\kern0pt}fm{\isacharparenleft}{\kern0pt}p{\isacharcomma}{\kern0pt}l{\isacharcomma}{\kern0pt}q{\isacharcomma}{\kern0pt}t{\isadigit{1}}{\isacharcomma}{\kern0pt}t{\isadigit{2}}{\isacharparenright}{\kern0pt}{\isacharcomma}{\kern0pt}env{\isacharparenright}{\kern0pt}\ {\isasymlongleftrightarrow}\isanewline
\ \ \ \ \ \ \ \ \ \ \ \ \ is{\isacharunderscore}{\kern0pt}forces{\isacharunderscore}{\kern0pt}nmem{\isacharprime}{\kern0pt}{\isacharparenleft}{\kern0pt}{\isacharhash}{\kern0pt}{\isacharhash}{\kern0pt}M{\isacharcomma}{\kern0pt}nth{\isacharparenleft}{\kern0pt}p{\isacharcomma}{\kern0pt}env{\isacharparenright}{\kern0pt}{\isacharcomma}{\kern0pt}nth{\isacharparenleft}{\kern0pt}l{\isacharcomma}{\kern0pt}env{\isacharparenright}{\kern0pt}{\isacharcomma}{\kern0pt}nth{\isacharparenleft}{\kern0pt}q{\isacharcomma}{\kern0pt}env{\isacharparenright}{\kern0pt}{\isacharcomma}{\kern0pt}nth{\isacharparenleft}{\kern0pt}t{\isadigit{1}}{\isacharcomma}{\kern0pt}env{\isacharparenright}{\kern0pt}{\isacharcomma}{\kern0pt}nth{\isacharparenleft}{\kern0pt}t{\isadigit{2}}{\isacharcomma}{\kern0pt}env{\isacharparenright}{\kern0pt}{\isacharparenright}{\kern0pt}{\isachardoublequoteclose}\isanewline
%
\isadelimproof
\ \ %
\endisadelimproof
%
\isatagproof
\isacommand{unfolding}\isamarkupfalse%
\ forces{\isacharunderscore}{\kern0pt}nmem{\isacharunderscore}{\kern0pt}fm{\isacharunderscore}{\kern0pt}def\ is{\isacharunderscore}{\kern0pt}forces{\isacharunderscore}{\kern0pt}nmem{\isacharprime}{\kern0pt}{\isacharunderscore}{\kern0pt}def\isanewline
\ \ \isacommand{using}\isamarkupfalse%
\ assms\ sats{\isacharunderscore}{\kern0pt}forces{\isacharunderscore}{\kern0pt}mem{\isacharprime}{\kern0pt}{\isacharunderscore}{\kern0pt}fm\ sats{\isacharunderscore}{\kern0pt}is{\isacharunderscore}{\kern0pt}tuple{\isacharunderscore}{\kern0pt}fm\ sats{\isacharunderscore}{\kern0pt}frc{\isacharunderscore}{\kern0pt}at{\isacharunderscore}{\kern0pt}fm\isanewline
\ \ \isacommand{by}\isamarkupfalse%
\ simp%
\endisatagproof
{\isafoldproof}%
%
\isadelimproof
\isanewline
%
\endisadelimproof
\isanewline
\isacommand{context}\isamarkupfalse%
\ forcing{\isacharunderscore}{\kern0pt}data\isanewline
\isakeyword{begin}\isanewline
\isanewline
\isanewline
\isacommand{lemma}\isamarkupfalse%
\ fst{\isacharunderscore}{\kern0pt}abs\ {\isacharbrackleft}{\kern0pt}simp{\isacharbrackright}{\kern0pt}{\isacharcolon}{\kern0pt}\isanewline
\ \ {\isachardoublequoteopen}{\isasymlbrakk}x{\isasymin}M{\isacharsemicolon}{\kern0pt}\ y{\isasymin}M\ {\isasymrbrakk}\ {\isasymLongrightarrow}\ is{\isacharunderscore}{\kern0pt}fst{\isacharparenleft}{\kern0pt}{\isacharhash}{\kern0pt}{\isacharhash}{\kern0pt}M{\isacharcomma}{\kern0pt}x{\isacharcomma}{\kern0pt}y{\isacharparenright}{\kern0pt}\ {\isasymlongleftrightarrow}\ y\ {\isacharequal}{\kern0pt}\ fst{\isacharparenleft}{\kern0pt}x{\isacharparenright}{\kern0pt}{\isachardoublequoteclose}\isanewline
%
\isadelimproof
\ \ %
\endisadelimproof
%
\isatagproof
\isacommand{unfolding}\isamarkupfalse%
\ fst{\isacharunderscore}{\kern0pt}def\ is{\isacharunderscore}{\kern0pt}fst{\isacharunderscore}{\kern0pt}def\ \isacommand{using}\isamarkupfalse%
\ pair{\isacharunderscore}{\kern0pt}in{\isacharunderscore}{\kern0pt}M{\isacharunderscore}{\kern0pt}iff\ zero{\isacharunderscore}{\kern0pt}in{\isacharunderscore}{\kern0pt}M\isanewline
\ \ \isacommand{by}\isamarkupfalse%
\ {\isacharparenleft}{\kern0pt}auto{\isacharsemicolon}{\kern0pt}rule{\isacharunderscore}{\kern0pt}tac\ the{\isacharunderscore}{\kern0pt}{\isadigit{0}}\ the{\isacharunderscore}{\kern0pt}{\isadigit{0}}{\isacharbrackleft}{\kern0pt}symmetric{\isacharbrackright}{\kern0pt}{\isacharcomma}{\kern0pt}auto{\isacharparenright}{\kern0pt}%
\endisatagproof
{\isafoldproof}%
%
\isadelimproof
\isanewline
%
\endisadelimproof
\isanewline
\isacommand{lemma}\isamarkupfalse%
\ snd{\isacharunderscore}{\kern0pt}abs\ {\isacharbrackleft}{\kern0pt}simp{\isacharbrackright}{\kern0pt}{\isacharcolon}{\kern0pt}\isanewline
\ \ {\isachardoublequoteopen}{\isasymlbrakk}x{\isasymin}M{\isacharsemicolon}{\kern0pt}\ y{\isasymin}M\ {\isasymrbrakk}\ {\isasymLongrightarrow}\ is{\isacharunderscore}{\kern0pt}snd{\isacharparenleft}{\kern0pt}{\isacharhash}{\kern0pt}{\isacharhash}{\kern0pt}M{\isacharcomma}{\kern0pt}x{\isacharcomma}{\kern0pt}y{\isacharparenright}{\kern0pt}\ {\isasymlongleftrightarrow}\ y\ {\isacharequal}{\kern0pt}\ snd{\isacharparenleft}{\kern0pt}x{\isacharparenright}{\kern0pt}{\isachardoublequoteclose}\isanewline
%
\isadelimproof
\ \ %
\endisadelimproof
%
\isatagproof
\isacommand{unfolding}\isamarkupfalse%
\ snd{\isacharunderscore}{\kern0pt}def\ is{\isacharunderscore}{\kern0pt}snd{\isacharunderscore}{\kern0pt}def\ \isacommand{using}\isamarkupfalse%
\ pair{\isacharunderscore}{\kern0pt}in{\isacharunderscore}{\kern0pt}M{\isacharunderscore}{\kern0pt}iff\ zero{\isacharunderscore}{\kern0pt}in{\isacharunderscore}{\kern0pt}M\isanewline
\ \ \isacommand{by}\isamarkupfalse%
\ {\isacharparenleft}{\kern0pt}auto{\isacharsemicolon}{\kern0pt}rule{\isacharunderscore}{\kern0pt}tac\ the{\isacharunderscore}{\kern0pt}{\isadigit{0}}\ the{\isacharunderscore}{\kern0pt}{\isadigit{0}}{\isacharbrackleft}{\kern0pt}symmetric{\isacharbrackright}{\kern0pt}{\isacharcomma}{\kern0pt}auto{\isacharparenright}{\kern0pt}%
\endisatagproof
{\isafoldproof}%
%
\isadelimproof
\isanewline
%
\endisadelimproof
\isanewline
\isacommand{lemma}\isamarkupfalse%
\ ftype{\isacharunderscore}{\kern0pt}abs{\isacharbrackleft}{\kern0pt}simp{\isacharbrackright}{\kern0pt}\ {\isacharcolon}{\kern0pt}\isanewline
\ \ {\isachardoublequoteopen}{\isasymlbrakk}x{\isasymin}M{\isacharsemicolon}{\kern0pt}\ y{\isasymin}M\ {\isasymrbrakk}\ {\isasymLongrightarrow}\ is{\isacharunderscore}{\kern0pt}ftype{\isacharparenleft}{\kern0pt}{\isacharhash}{\kern0pt}{\isacharhash}{\kern0pt}M{\isacharcomma}{\kern0pt}x{\isacharcomma}{\kern0pt}y{\isacharparenright}{\kern0pt}\ {\isasymlongleftrightarrow}\ y\ {\isacharequal}{\kern0pt}\ ftype{\isacharparenleft}{\kern0pt}x{\isacharparenright}{\kern0pt}{\isachardoublequoteclose}%
\isadelimproof
\ %
\endisadelimproof
%
\isatagproof
\isacommand{unfolding}\isamarkupfalse%
\ ftype{\isacharunderscore}{\kern0pt}def\ \ is{\isacharunderscore}{\kern0pt}ftype{\isacharunderscore}{\kern0pt}def\ \isacommand{by}\isamarkupfalse%
\ simp%
\endisatagproof
{\isafoldproof}%
%
\isadelimproof
%
\endisadelimproof
\isanewline
\isanewline
\isacommand{lemma}\isamarkupfalse%
\ name{\isadigit{1}}{\isacharunderscore}{\kern0pt}abs{\isacharbrackleft}{\kern0pt}simp{\isacharbrackright}{\kern0pt}\ {\isacharcolon}{\kern0pt}\isanewline
\ \ {\isachardoublequoteopen}{\isasymlbrakk}x{\isasymin}M{\isacharsemicolon}{\kern0pt}\ y{\isasymin}M\ {\isasymrbrakk}\ {\isasymLongrightarrow}\ is{\isacharunderscore}{\kern0pt}name{\isadigit{1}}{\isacharparenleft}{\kern0pt}{\isacharhash}{\kern0pt}{\isacharhash}{\kern0pt}M{\isacharcomma}{\kern0pt}x{\isacharcomma}{\kern0pt}y{\isacharparenright}{\kern0pt}\ {\isasymlongleftrightarrow}\ y\ {\isacharequal}{\kern0pt}\ name{\isadigit{1}}{\isacharparenleft}{\kern0pt}x{\isacharparenright}{\kern0pt}{\isachardoublequoteclose}\isanewline
%
\isadelimproof
\ \ %
\endisadelimproof
%
\isatagproof
\isacommand{unfolding}\isamarkupfalse%
\ name{\isadigit{1}}{\isacharunderscore}{\kern0pt}def\ is{\isacharunderscore}{\kern0pt}name{\isadigit{1}}{\isacharunderscore}{\kern0pt}def\isanewline
\ \ \isacommand{by}\isamarkupfalse%
\ {\isacharparenleft}{\kern0pt}rule\ hcomp{\isacharunderscore}{\kern0pt}abs{\isacharbrackleft}{\kern0pt}OF\ fst{\isacharunderscore}{\kern0pt}abs{\isacharbrackright}{\kern0pt}{\isacharsemicolon}{\kern0pt}simp{\isacharunderscore}{\kern0pt}all\ add{\isacharcolon}{\kern0pt}fst{\isacharunderscore}{\kern0pt}snd{\isacharunderscore}{\kern0pt}closed{\isacharparenright}{\kern0pt}%
\endisatagproof
{\isafoldproof}%
%
\isadelimproof
\isanewline
%
\endisadelimproof
\isanewline
\isacommand{lemma}\isamarkupfalse%
\ snd{\isacharunderscore}{\kern0pt}snd{\isacharunderscore}{\kern0pt}abs{\isacharcolon}{\kern0pt}\isanewline
\ \ {\isachardoublequoteopen}{\isasymlbrakk}x{\isasymin}M{\isacharsemicolon}{\kern0pt}\ y{\isasymin}M\ {\isasymrbrakk}\ {\isasymLongrightarrow}\ is{\isacharunderscore}{\kern0pt}snd{\isacharunderscore}{\kern0pt}snd{\isacharparenleft}{\kern0pt}{\isacharhash}{\kern0pt}{\isacharhash}{\kern0pt}M{\isacharcomma}{\kern0pt}x{\isacharcomma}{\kern0pt}y{\isacharparenright}{\kern0pt}\ {\isasymlongleftrightarrow}\ y\ {\isacharequal}{\kern0pt}\ snd{\isacharparenleft}{\kern0pt}snd{\isacharparenleft}{\kern0pt}x{\isacharparenright}{\kern0pt}{\isacharparenright}{\kern0pt}{\isachardoublequoteclose}\isanewline
%
\isadelimproof
\ \ %
\endisadelimproof
%
\isatagproof
\isacommand{unfolding}\isamarkupfalse%
\ is{\isacharunderscore}{\kern0pt}snd{\isacharunderscore}{\kern0pt}snd{\isacharunderscore}{\kern0pt}def\isanewline
\ \ \isacommand{by}\isamarkupfalse%
\ {\isacharparenleft}{\kern0pt}rule\ hcomp{\isacharunderscore}{\kern0pt}abs{\isacharbrackleft}{\kern0pt}OF\ snd{\isacharunderscore}{\kern0pt}abs{\isacharbrackright}{\kern0pt}{\isacharsemicolon}{\kern0pt}simp{\isacharunderscore}{\kern0pt}all\ add{\isacharcolon}{\kern0pt}fst{\isacharunderscore}{\kern0pt}snd{\isacharunderscore}{\kern0pt}closed{\isacharparenright}{\kern0pt}%
\endisatagproof
{\isafoldproof}%
%
\isadelimproof
\isanewline
%
\endisadelimproof
\isanewline
\isacommand{lemma}\isamarkupfalse%
\ name{\isadigit{2}}{\isacharunderscore}{\kern0pt}abs{\isacharbrackleft}{\kern0pt}simp{\isacharbrackright}{\kern0pt}{\isacharcolon}{\kern0pt}\isanewline
\ \ {\isachardoublequoteopen}{\isasymlbrakk}x{\isasymin}M{\isacharsemicolon}{\kern0pt}\ y{\isasymin}M\ {\isasymrbrakk}\ {\isasymLongrightarrow}\ is{\isacharunderscore}{\kern0pt}name{\isadigit{2}}{\isacharparenleft}{\kern0pt}{\isacharhash}{\kern0pt}{\isacharhash}{\kern0pt}M{\isacharcomma}{\kern0pt}x{\isacharcomma}{\kern0pt}y{\isacharparenright}{\kern0pt}\ {\isasymlongleftrightarrow}\ y\ {\isacharequal}{\kern0pt}\ name{\isadigit{2}}{\isacharparenleft}{\kern0pt}x{\isacharparenright}{\kern0pt}{\isachardoublequoteclose}\isanewline
%
\isadelimproof
\ \ %
\endisadelimproof
%
\isatagproof
\isacommand{unfolding}\isamarkupfalse%
\ name{\isadigit{2}}{\isacharunderscore}{\kern0pt}def\ is{\isacharunderscore}{\kern0pt}name{\isadigit{2}}{\isacharunderscore}{\kern0pt}def\isanewline
\ \ \isacommand{by}\isamarkupfalse%
\ {\isacharparenleft}{\kern0pt}rule\ hcomp{\isacharunderscore}{\kern0pt}abs{\isacharbrackleft}{\kern0pt}OF\ fst{\isacharunderscore}{\kern0pt}abs\ snd{\isacharunderscore}{\kern0pt}snd{\isacharunderscore}{\kern0pt}abs{\isacharbrackright}{\kern0pt}{\isacharsemicolon}{\kern0pt}simp{\isacharunderscore}{\kern0pt}all\ add{\isacharcolon}{\kern0pt}fst{\isacharunderscore}{\kern0pt}snd{\isacharunderscore}{\kern0pt}closed{\isacharparenright}{\kern0pt}%
\endisatagproof
{\isafoldproof}%
%
\isadelimproof
\isanewline
%
\endisadelimproof
\isanewline
\isacommand{lemma}\isamarkupfalse%
\ cond{\isacharunderscore}{\kern0pt}of{\isacharunderscore}{\kern0pt}abs{\isacharbrackleft}{\kern0pt}simp{\isacharbrackright}{\kern0pt}{\isacharcolon}{\kern0pt}\isanewline
\ \ {\isachardoublequoteopen}{\isasymlbrakk}x{\isasymin}M{\isacharsemicolon}{\kern0pt}\ y{\isasymin}M\ {\isasymrbrakk}\ {\isasymLongrightarrow}\ is{\isacharunderscore}{\kern0pt}cond{\isacharunderscore}{\kern0pt}of{\isacharparenleft}{\kern0pt}{\isacharhash}{\kern0pt}{\isacharhash}{\kern0pt}M{\isacharcomma}{\kern0pt}x{\isacharcomma}{\kern0pt}y{\isacharparenright}{\kern0pt}\ {\isasymlongleftrightarrow}\ y\ {\isacharequal}{\kern0pt}\ cond{\isacharunderscore}{\kern0pt}of{\isacharparenleft}{\kern0pt}x{\isacharparenright}{\kern0pt}{\isachardoublequoteclose}\isanewline
%
\isadelimproof
\ \ %
\endisadelimproof
%
\isatagproof
\isacommand{unfolding}\isamarkupfalse%
\ cond{\isacharunderscore}{\kern0pt}of{\isacharunderscore}{\kern0pt}def\ is{\isacharunderscore}{\kern0pt}cond{\isacharunderscore}{\kern0pt}of{\isacharunderscore}{\kern0pt}def\isanewline
\ \ \isacommand{by}\isamarkupfalse%
\ {\isacharparenleft}{\kern0pt}rule\ hcomp{\isacharunderscore}{\kern0pt}abs{\isacharbrackleft}{\kern0pt}OF\ snd{\isacharunderscore}{\kern0pt}abs\ snd{\isacharunderscore}{\kern0pt}snd{\isacharunderscore}{\kern0pt}abs{\isacharbrackright}{\kern0pt}{\isacharsemicolon}{\kern0pt}simp{\isacharunderscore}{\kern0pt}all\ add{\isacharcolon}{\kern0pt}fst{\isacharunderscore}{\kern0pt}snd{\isacharunderscore}{\kern0pt}closed{\isacharparenright}{\kern0pt}%
\endisatagproof
{\isafoldproof}%
%
\isadelimproof
\isanewline
%
\endisadelimproof
\isanewline
\isacommand{lemma}\isamarkupfalse%
\ tuple{\isacharunderscore}{\kern0pt}abs{\isacharbrackleft}{\kern0pt}simp{\isacharbrackright}{\kern0pt}{\isacharcolon}{\kern0pt}\isanewline
\ \ {\isachardoublequoteopen}{\isasymlbrakk}z{\isasymin}M{\isacharsemicolon}{\kern0pt}t{\isadigit{1}}{\isasymin}M{\isacharsemicolon}{\kern0pt}t{\isadigit{2}}{\isasymin}M{\isacharsemicolon}{\kern0pt}p{\isasymin}M{\isacharsemicolon}{\kern0pt}t{\isasymin}M{\isasymrbrakk}\ {\isasymLongrightarrow}\isanewline
\ \ \ is{\isacharunderscore}{\kern0pt}tuple{\isacharparenleft}{\kern0pt}{\isacharhash}{\kern0pt}{\isacharhash}{\kern0pt}M{\isacharcomma}{\kern0pt}z{\isacharcomma}{\kern0pt}t{\isadigit{1}}{\isacharcomma}{\kern0pt}t{\isadigit{2}}{\isacharcomma}{\kern0pt}p{\isacharcomma}{\kern0pt}t{\isacharparenright}{\kern0pt}\ {\isasymlongleftrightarrow}\ t\ {\isacharequal}{\kern0pt}\ {\isasymlangle}z{\isacharcomma}{\kern0pt}t{\isadigit{1}}{\isacharcomma}{\kern0pt}t{\isadigit{2}}{\isacharcomma}{\kern0pt}p{\isasymrangle}{\isachardoublequoteclose}\isanewline
%
\isadelimproof
\ \ %
\endisadelimproof
%
\isatagproof
\isacommand{unfolding}\isamarkupfalse%
\ is{\isacharunderscore}{\kern0pt}tuple{\isacharunderscore}{\kern0pt}def\ \isacommand{using}\isamarkupfalse%
\ tuples{\isacharunderscore}{\kern0pt}in{\isacharunderscore}{\kern0pt}M\ \isacommand{by}\isamarkupfalse%
\ simp%
\endisatagproof
{\isafoldproof}%
%
\isadelimproof
\isanewline
%
\endisadelimproof
\isanewline
\isacommand{lemma}\isamarkupfalse%
\ oneN{\isacharunderscore}{\kern0pt}in{\isacharunderscore}{\kern0pt}M{\isacharbrackleft}{\kern0pt}simp{\isacharbrackright}{\kern0pt}{\isacharcolon}{\kern0pt}\ {\isachardoublequoteopen}{\isadigit{1}}{\isasymin}M{\isachardoublequoteclose}\isanewline
%
\isadelimproof
\ \ %
\endisadelimproof
%
\isatagproof
\isacommand{by}\isamarkupfalse%
\ {\isacharparenleft}{\kern0pt}simp\ flip{\isacharcolon}{\kern0pt}\ setclass{\isacharunderscore}{\kern0pt}iff{\isacharparenright}{\kern0pt}%
\endisatagproof
{\isafoldproof}%
%
\isadelimproof
\isanewline
%
\endisadelimproof
\isanewline
\isacommand{lemma}\isamarkupfalse%
\ twoN{\isacharunderscore}{\kern0pt}in{\isacharunderscore}{\kern0pt}M\ {\isacharcolon}{\kern0pt}\ {\isachardoublequoteopen}{\isadigit{2}}{\isasymin}M{\isachardoublequoteclose}\isanewline
%
\isadelimproof
\ \ %
\endisadelimproof
%
\isatagproof
\isacommand{by}\isamarkupfalse%
\ {\isacharparenleft}{\kern0pt}simp\ flip{\isacharcolon}{\kern0pt}\ setclass{\isacharunderscore}{\kern0pt}iff{\isacharparenright}{\kern0pt}%
\endisatagproof
{\isafoldproof}%
%
\isadelimproof
\isanewline
%
\endisadelimproof
\isanewline
\isacommand{lemma}\isamarkupfalse%
\ comp{\isacharunderscore}{\kern0pt}in{\isacharunderscore}{\kern0pt}M{\isacharcolon}{\kern0pt}\isanewline
\ \ {\isachardoublequoteopen}p\ {\isasympreceq}\ q\ {\isasymLongrightarrow}\ p{\isasymin}M{\isachardoublequoteclose}\isanewline
\ \ {\isachardoublequoteopen}p\ {\isasympreceq}\ q\ {\isasymLongrightarrow}\ q{\isasymin}M{\isachardoublequoteclose}\isanewline
%
\isadelimproof
\ \ %
\endisadelimproof
%
\isatagproof
\isacommand{using}\isamarkupfalse%
\ leq{\isacharunderscore}{\kern0pt}in{\isacharunderscore}{\kern0pt}M\ transitivity{\isacharbrackleft}{\kern0pt}of\ {\isacharunderscore}{\kern0pt}\ leq{\isacharbrackright}{\kern0pt}\ pair{\isacharunderscore}{\kern0pt}in{\isacharunderscore}{\kern0pt}M{\isacharunderscore}{\kern0pt}iff\ \isacommand{by}\isamarkupfalse%
\ auto%
\endisatagproof
{\isafoldproof}%
%
\isadelimproof
\isanewline
%
\endisadelimproof
\isanewline
\isanewline
\isanewline
\isacommand{lemma}\isamarkupfalse%
\ eq{\isacharunderscore}{\kern0pt}case{\isacharunderscore}{\kern0pt}abs\ {\isacharbrackleft}{\kern0pt}simp{\isacharbrackright}{\kern0pt}{\isacharcolon}{\kern0pt}\isanewline
\ \ \isakeyword{assumes}\isanewline
\ \ \ \ {\isachardoublequoteopen}t{\isadigit{1}}{\isasymin}M{\isachardoublequoteclose}\ {\isachardoublequoteopen}t{\isadigit{2}}{\isasymin}M{\isachardoublequoteclose}\ {\isachardoublequoteopen}p{\isasymin}M{\isachardoublequoteclose}\ {\isachardoublequoteopen}f{\isasymin}M{\isachardoublequoteclose}\isanewline
\ \ \isakeyword{shows}\isanewline
\ \ \ \ {\isachardoublequoteopen}is{\isacharunderscore}{\kern0pt}eq{\isacharunderscore}{\kern0pt}case{\isacharparenleft}{\kern0pt}{\isacharhash}{\kern0pt}{\isacharhash}{\kern0pt}M{\isacharcomma}{\kern0pt}t{\isadigit{1}}{\isacharcomma}{\kern0pt}t{\isadigit{2}}{\isacharcomma}{\kern0pt}p{\isacharcomma}{\kern0pt}P{\isacharcomma}{\kern0pt}leq{\isacharcomma}{\kern0pt}f{\isacharparenright}{\kern0pt}\ {\isasymlongleftrightarrow}\ eq{\isacharunderscore}{\kern0pt}case{\isacharparenleft}{\kern0pt}t{\isadigit{1}}{\isacharcomma}{\kern0pt}t{\isadigit{2}}{\isacharcomma}{\kern0pt}p{\isacharcomma}{\kern0pt}P{\isacharcomma}{\kern0pt}leq{\isacharcomma}{\kern0pt}f{\isacharparenright}{\kern0pt}{\isachardoublequoteclose}\isanewline
%
\isadelimproof
%
\endisadelimproof
%
\isatagproof
\isacommand{proof}\isamarkupfalse%
\ {\isacharminus}{\kern0pt}\isanewline
\ \ \isacommand{have}\isamarkupfalse%
\ {\isachardoublequoteopen}q\ {\isasympreceq}\ p\ {\isasymLongrightarrow}\ q{\isasymin}M{\isachardoublequoteclose}\ \isakeyword{for}\ q\isanewline
\ \ \ \ \isacommand{using}\isamarkupfalse%
\ comp{\isacharunderscore}{\kern0pt}in{\isacharunderscore}{\kern0pt}M\ \isacommand{by}\isamarkupfalse%
\ simp\isanewline
\ \ \isacommand{moreover}\isamarkupfalse%
\isanewline
\ \ \isacommand{have}\isamarkupfalse%
\ {\isachardoublequoteopen}{\isasymlangle}s{\isacharcomma}{\kern0pt}y{\isasymrangle}{\isasymin}t\ {\isasymLongrightarrow}\ s{\isasymin}domain{\isacharparenleft}{\kern0pt}t{\isacharparenright}{\kern0pt}{\isachardoublequoteclose}\ \isakeyword{if}\ {\isachardoublequoteopen}t{\isasymin}M{\isachardoublequoteclose}\ \isakeyword{for}\ s\ y\ t\isanewline
\ \ \ \ \isacommand{using}\isamarkupfalse%
\ that\ \isacommand{unfolding}\isamarkupfalse%
\ domain{\isacharunderscore}{\kern0pt}def\ \isacommand{by}\isamarkupfalse%
\ auto\isanewline
\ \ \isacommand{ultimately}\isamarkupfalse%
\isanewline
\ \ \isacommand{have}\isamarkupfalse%
\isanewline
\ \ \ \ {\isachardoublequoteopen}{\isacharparenleft}{\kern0pt}{\isasymforall}s{\isasymin}M{\isachardot}{\kern0pt}\ s\ {\isasymin}\ domain{\isacharparenleft}{\kern0pt}t{\isadigit{1}}{\isacharparenright}{\kern0pt}\ {\isasymor}\ s\ {\isasymin}\ domain{\isacharparenleft}{\kern0pt}t{\isadigit{2}}{\isacharparenright}{\kern0pt}\ {\isasymlongrightarrow}\ {\isacharparenleft}{\kern0pt}{\isasymforall}q{\isasymin}M{\isachardot}{\kern0pt}\ q{\isasymin}P\ {\isasymand}\ q\ {\isasympreceq}\ p\ {\isasymlongrightarrow}\isanewline
\ \ \ \ \ \ \ \ \ \ \ \ \ \ \ \ \ \ \ \ \ \ \ \ \ \ \ \ \ \ {\isacharparenleft}{\kern0pt}f\ {\isacharbackquote}{\kern0pt}\ {\isasymlangle}{\isadigit{1}}{\isacharcomma}{\kern0pt}\ s{\isacharcomma}{\kern0pt}\ t{\isadigit{1}}{\isacharcomma}{\kern0pt}\ q{\isasymrangle}\ {\isacharequal}{\kern0pt}{\isadigit{1}}\ {\isasymlongleftrightarrow}\ f\ {\isacharbackquote}{\kern0pt}\ {\isasymlangle}{\isadigit{1}}{\isacharcomma}{\kern0pt}\ s{\isacharcomma}{\kern0pt}\ t{\isadigit{2}}{\isacharcomma}{\kern0pt}\ q{\isasymrangle}{\isacharequal}{\kern0pt}{\isadigit{1}}{\isacharparenright}{\kern0pt}{\isacharparenright}{\kern0pt}{\isacharparenright}{\kern0pt}\ {\isasymlongleftrightarrow}\isanewline
\ \ \ \ {\isacharparenleft}{\kern0pt}{\isasymforall}s{\isachardot}{\kern0pt}\ s\ {\isasymin}\ domain{\isacharparenleft}{\kern0pt}t{\isadigit{1}}{\isacharparenright}{\kern0pt}\ {\isasymor}\ s\ {\isasymin}\ domain{\isacharparenleft}{\kern0pt}t{\isadigit{2}}{\isacharparenright}{\kern0pt}\ {\isasymlongrightarrow}\ {\isacharparenleft}{\kern0pt}{\isasymforall}q{\isachardot}{\kern0pt}\ q{\isasymin}P\ {\isasymand}\ q\ {\isasympreceq}\ p\ {\isasymlongrightarrow}\isanewline
\ \ \ \ \ \ \ \ \ \ \ \ \ \ \ \ \ \ \ \ \ \ \ \ \ \ \ \ \ \ \ \ \ \ {\isacharparenleft}{\kern0pt}f\ {\isacharbackquote}{\kern0pt}\ {\isasymlangle}{\isadigit{1}}{\isacharcomma}{\kern0pt}\ s{\isacharcomma}{\kern0pt}\ t{\isadigit{1}}{\isacharcomma}{\kern0pt}\ q{\isasymrangle}\ {\isacharequal}{\kern0pt}{\isadigit{1}}\ {\isasymlongleftrightarrow}\ f\ {\isacharbackquote}{\kern0pt}\ {\isasymlangle}{\isadigit{1}}{\isacharcomma}{\kern0pt}\ s{\isacharcomma}{\kern0pt}\ t{\isadigit{2}}{\isacharcomma}{\kern0pt}\ q{\isasymrangle}{\isacharequal}{\kern0pt}{\isadigit{1}}{\isacharparenright}{\kern0pt}{\isacharparenright}{\kern0pt}{\isacharparenright}{\kern0pt}{\isachardoublequoteclose}\isanewline
\ \ \ \ \isacommand{using}\isamarkupfalse%
\ assms\ domain{\isacharunderscore}{\kern0pt}trans{\isacharbrackleft}{\kern0pt}OF\ trans{\isacharunderscore}{\kern0pt}M{\isacharcomma}{\kern0pt}of\ t{\isadigit{1}}{\isacharbrackright}{\kern0pt}\isanewline
\ \ \ \ \ \ domain{\isacharunderscore}{\kern0pt}trans{\isacharbrackleft}{\kern0pt}OF\ trans{\isacharunderscore}{\kern0pt}M{\isacharcomma}{\kern0pt}of\ t{\isadigit{2}}{\isacharbrackright}{\kern0pt}\ \isacommand{by}\isamarkupfalse%
\ auto\isanewline
\ \ \isacommand{then}\isamarkupfalse%
\ \isacommand{show}\isamarkupfalse%
\ {\isacharquery}{\kern0pt}thesis\isanewline
\ \ \ \ \isacommand{unfolding}\isamarkupfalse%
\ eq{\isacharunderscore}{\kern0pt}case{\isacharunderscore}{\kern0pt}def\ is{\isacharunderscore}{\kern0pt}eq{\isacharunderscore}{\kern0pt}case{\isacharunderscore}{\kern0pt}def\isanewline
\ \ \ \ \isacommand{using}\isamarkupfalse%
\ assms\ pair{\isacharunderscore}{\kern0pt}in{\isacharunderscore}{\kern0pt}M{\isacharunderscore}{\kern0pt}iff\ n{\isacharunderscore}{\kern0pt}in{\isacharunderscore}{\kern0pt}M{\isacharbrackleft}{\kern0pt}of\ {\isadigit{1}}{\isacharbrackright}{\kern0pt}\ domain{\isacharunderscore}{\kern0pt}closed\ tuples{\isacharunderscore}{\kern0pt}in{\isacharunderscore}{\kern0pt}M\isanewline
\ \ \ \ \ \ apply{\isacharunderscore}{\kern0pt}closed\ leq{\isacharunderscore}{\kern0pt}in{\isacharunderscore}{\kern0pt}M\isanewline
\ \ \ \ \isacommand{by}\isamarkupfalse%
\ simp\isanewline
\isacommand{qed}\isamarkupfalse%
%
\endisatagproof
{\isafoldproof}%
%
\isadelimproof
\isanewline
%
\endisadelimproof
\isanewline
\isacommand{lemma}\isamarkupfalse%
\ mem{\isacharunderscore}{\kern0pt}case{\isacharunderscore}{\kern0pt}abs\ {\isacharbrackleft}{\kern0pt}simp{\isacharbrackright}{\kern0pt}{\isacharcolon}{\kern0pt}\isanewline
\ \ \isakeyword{assumes}\isanewline
\ \ \ \ {\isachardoublequoteopen}t{\isadigit{1}}{\isasymin}M{\isachardoublequoteclose}\ {\isachardoublequoteopen}t{\isadigit{2}}{\isasymin}M{\isachardoublequoteclose}\ {\isachardoublequoteopen}p{\isasymin}M{\isachardoublequoteclose}\ {\isachardoublequoteopen}f{\isasymin}M{\isachardoublequoteclose}\isanewline
\ \ \isakeyword{shows}\isanewline
\ \ \ \ {\isachardoublequoteopen}is{\isacharunderscore}{\kern0pt}mem{\isacharunderscore}{\kern0pt}case{\isacharparenleft}{\kern0pt}{\isacharhash}{\kern0pt}{\isacharhash}{\kern0pt}M{\isacharcomma}{\kern0pt}t{\isadigit{1}}{\isacharcomma}{\kern0pt}t{\isadigit{2}}{\isacharcomma}{\kern0pt}p{\isacharcomma}{\kern0pt}P{\isacharcomma}{\kern0pt}leq{\isacharcomma}{\kern0pt}f{\isacharparenright}{\kern0pt}\ {\isasymlongleftrightarrow}\ mem{\isacharunderscore}{\kern0pt}case{\isacharparenleft}{\kern0pt}t{\isadigit{1}}{\isacharcomma}{\kern0pt}t{\isadigit{2}}{\isacharcomma}{\kern0pt}p{\isacharcomma}{\kern0pt}P{\isacharcomma}{\kern0pt}leq{\isacharcomma}{\kern0pt}f{\isacharparenright}{\kern0pt}{\isachardoublequoteclose}\isanewline
%
\isadelimproof
%
\endisadelimproof
%
\isatagproof
\isacommand{proof}\isamarkupfalse%
\isanewline
\ \ \isacommand{{\isacharbraceleft}{\kern0pt}}\isamarkupfalse%
\isanewline
\ \ \ \ \isacommand{fix}\isamarkupfalse%
\ v\isanewline
\ \ \ \ \isacommand{assume}\isamarkupfalse%
\ {\isachardoublequoteopen}v{\isasymin}P{\isachardoublequoteclose}\ {\isachardoublequoteopen}v\ {\isasympreceq}\ p{\isachardoublequoteclose}\ {\isachardoublequoteopen}is{\isacharunderscore}{\kern0pt}mem{\isacharunderscore}{\kern0pt}case{\isacharparenleft}{\kern0pt}{\isacharhash}{\kern0pt}{\isacharhash}{\kern0pt}M{\isacharcomma}{\kern0pt}t{\isadigit{1}}{\isacharcomma}{\kern0pt}t{\isadigit{2}}{\isacharcomma}{\kern0pt}p{\isacharcomma}{\kern0pt}P{\isacharcomma}{\kern0pt}leq{\isacharcomma}{\kern0pt}f{\isacharparenright}{\kern0pt}{\isachardoublequoteclose}\isanewline
\ \ \ \ \isacommand{moreover}\isamarkupfalse%
\isanewline
\ \ \ \ \isacommand{from}\isamarkupfalse%
\ this\isanewline
\ \ \ \ \isacommand{have}\isamarkupfalse%
\ {\isachardoublequoteopen}v{\isasymin}M{\isachardoublequoteclose}\ {\isachardoublequoteopen}{\isasymlangle}v{\isacharcomma}{\kern0pt}p{\isasymrangle}\ {\isasymin}\ M{\isachardoublequoteclose}\ {\isachardoublequoteopen}{\isacharparenleft}{\kern0pt}{\isacharhash}{\kern0pt}{\isacharhash}{\kern0pt}M{\isacharparenright}{\kern0pt}{\isacharparenleft}{\kern0pt}v{\isacharparenright}{\kern0pt}{\isachardoublequoteclose}\isanewline
\ \ \ \ \ \ \isacommand{using}\isamarkupfalse%
\ transitivity{\isacharbrackleft}{\kern0pt}OF\ {\isacharunderscore}{\kern0pt}\ P{\isacharunderscore}{\kern0pt}in{\isacharunderscore}{\kern0pt}M{\isacharcomma}{\kern0pt}of\ v{\isacharbrackright}{\kern0pt}\ transitivity{\isacharbrackleft}{\kern0pt}OF\ {\isacharunderscore}{\kern0pt}\ leq{\isacharunderscore}{\kern0pt}in{\isacharunderscore}{\kern0pt}M{\isacharbrackright}{\kern0pt}\isanewline
\ \ \ \ \ \ \isacommand{by}\isamarkupfalse%
\ simp{\isacharunderscore}{\kern0pt}all\isanewline
\ \ \ \ \isacommand{moreover}\isamarkupfalse%
\isanewline
\ \ \ \ \isacommand{from}\isamarkupfalse%
\ calculation\ assms\isanewline
\ \ \ \ \isacommand{obtain}\isamarkupfalse%
\ q\ r\ s\ \isakeyword{where}\isanewline
\ \ \ \ \ \ {\isachardoublequoteopen}r\ {\isasymin}\ P\ {\isasymand}\ q\ {\isasymin}\ P\ {\isasymand}\ {\isasymlangle}q{\isacharcomma}{\kern0pt}\ v{\isasymrangle}\ {\isasymin}\ M\ {\isasymand}\ {\isasymlangle}s{\isacharcomma}{\kern0pt}\ r{\isasymrangle}\ {\isasymin}\ M\ {\isasymand}\ {\isasymlangle}q{\isacharcomma}{\kern0pt}\ r{\isasymrangle}\ {\isasymin}\ M\ {\isasymand}\ {\isadigit{0}}\ {\isasymin}\ M\ {\isasymand}\isanewline
\ \ \ \ \ \ \ {\isasymlangle}{\isadigit{0}}{\isacharcomma}{\kern0pt}\ t{\isadigit{1}}{\isacharcomma}{\kern0pt}\ s{\isacharcomma}{\kern0pt}\ q{\isasymrangle}\ {\isasymin}\ M\ {\isasymand}\ q\ {\isasympreceq}\ v\ {\isasymand}\ {\isasymlangle}s{\isacharcomma}{\kern0pt}\ r{\isasymrangle}\ {\isasymin}\ t{\isadigit{2}}\ {\isasymand}\ q\ {\isasympreceq}\ r\ {\isasymand}\ f\ {\isacharbackquote}{\kern0pt}\ {\isasymlangle}{\isadigit{0}}{\isacharcomma}{\kern0pt}\ t{\isadigit{1}}{\isacharcomma}{\kern0pt}\ s{\isacharcomma}{\kern0pt}\ q{\isasymrangle}\ {\isacharequal}{\kern0pt}\ {\isadigit{1}}{\isachardoublequoteclose}\isanewline
\ \ \ \ \ \ \isacommand{unfolding}\isamarkupfalse%
\ is{\isacharunderscore}{\kern0pt}mem{\isacharunderscore}{\kern0pt}case{\isacharunderscore}{\kern0pt}def\ \isacommand{by}\isamarkupfalse%
\ auto\isanewline
\ \ \ \ \isacommand{then}\isamarkupfalse%
\isanewline
\ \ \ \ \isacommand{have}\isamarkupfalse%
\ {\isachardoublequoteopen}{\isasymexists}q\ s\ r{\isachardot}{\kern0pt}\ r\ {\isasymin}\ P\ {\isasymand}\ q\ {\isasymin}\ P\ {\isasymand}\ q\ {\isasympreceq}\ v\ {\isasymand}\ {\isasymlangle}s{\isacharcomma}{\kern0pt}\ r{\isasymrangle}\ {\isasymin}\ t{\isadigit{2}}\ {\isasymand}\ q\ {\isasympreceq}\ r\ {\isasymand}\ f\ {\isacharbackquote}{\kern0pt}\ {\isasymlangle}{\isadigit{0}}{\isacharcomma}{\kern0pt}\ t{\isadigit{1}}{\isacharcomma}{\kern0pt}\ s{\isacharcomma}{\kern0pt}\ q{\isasymrangle}\ {\isacharequal}{\kern0pt}\ {\isadigit{1}}{\isachardoublequoteclose}\isanewline
\ \ \ \ \ \ \isacommand{by}\isamarkupfalse%
\ auto\isanewline
\ \ \isacommand{{\isacharbraceright}{\kern0pt}}\isamarkupfalse%
\isanewline
\ \ \isacommand{then}\isamarkupfalse%
\isanewline
\ \ \isacommand{show}\isamarkupfalse%
\ {\isachardoublequoteopen}mem{\isacharunderscore}{\kern0pt}case{\isacharparenleft}{\kern0pt}t{\isadigit{1}}{\isacharcomma}{\kern0pt}\ t{\isadigit{2}}{\isacharcomma}{\kern0pt}\ p{\isacharcomma}{\kern0pt}\ P{\isacharcomma}{\kern0pt}\ leq{\isacharcomma}{\kern0pt}\ f{\isacharparenright}{\kern0pt}{\isachardoublequoteclose}\ \isakeyword{if}\ {\isachardoublequoteopen}is{\isacharunderscore}{\kern0pt}mem{\isacharunderscore}{\kern0pt}case{\isacharparenleft}{\kern0pt}{\isacharhash}{\kern0pt}{\isacharhash}{\kern0pt}M{\isacharcomma}{\kern0pt}\ t{\isadigit{1}}{\isacharcomma}{\kern0pt}\ t{\isadigit{2}}{\isacharcomma}{\kern0pt}\ p{\isacharcomma}{\kern0pt}\ P{\isacharcomma}{\kern0pt}\ leq{\isacharcomma}{\kern0pt}\ f{\isacharparenright}{\kern0pt}{\isachardoublequoteclose}\isanewline
\ \ \ \ \isacommand{unfolding}\isamarkupfalse%
\ mem{\isacharunderscore}{\kern0pt}case{\isacharunderscore}{\kern0pt}def\ \isacommand{using}\isamarkupfalse%
\ that\ assms\ \isacommand{by}\isamarkupfalse%
\ auto\isanewline
\isacommand{next}\isamarkupfalse%
\isanewline
\ \ \isacommand{{\isacharbraceleft}{\kern0pt}}\isamarkupfalse%
\ \isacommand{fix}\isamarkupfalse%
\ v\isanewline
\ \ \ \ \isacommand{assume}\isamarkupfalse%
\ {\isachardoublequoteopen}v\ {\isasymin}\ M{\isachardoublequoteclose}\ {\isachardoublequoteopen}v\ {\isasymin}\ P{\isachardoublequoteclose}\ {\isachardoublequoteopen}{\isasymlangle}v{\isacharcomma}{\kern0pt}\ p{\isasymrangle}\ {\isasymin}\ M{\isachardoublequoteclose}\ {\isachardoublequoteopen}v\ {\isasympreceq}\ p{\isachardoublequoteclose}\ {\isachardoublequoteopen}mem{\isacharunderscore}{\kern0pt}case{\isacharparenleft}{\kern0pt}t{\isadigit{1}}{\isacharcomma}{\kern0pt}\ t{\isadigit{2}}{\isacharcomma}{\kern0pt}\ p{\isacharcomma}{\kern0pt}\ P{\isacharcomma}{\kern0pt}\ leq{\isacharcomma}{\kern0pt}\ f{\isacharparenright}{\kern0pt}{\isachardoublequoteclose}\isanewline
\ \ \ \ \isacommand{moreover}\isamarkupfalse%
\isanewline
\ \ \ \ \isacommand{from}\isamarkupfalse%
\ this\isanewline
\ \ \ \ \isacommand{obtain}\isamarkupfalse%
\ q\ s\ r\ \isakeyword{where}\ {\isachardoublequoteopen}r\ {\isasymin}\ P\ {\isasymand}\ q\ {\isasymin}\ P\ {\isasymand}\ q\ {\isasympreceq}\ v\ {\isasymand}\ {\isasymlangle}s{\isacharcomma}{\kern0pt}\ r{\isasymrangle}\ {\isasymin}\ t{\isadigit{2}}\ {\isasymand}\ q\ {\isasympreceq}\ r\ {\isasymand}\ f\ {\isacharbackquote}{\kern0pt}\ {\isasymlangle}{\isadigit{0}}{\isacharcomma}{\kern0pt}\ t{\isadigit{1}}{\isacharcomma}{\kern0pt}\ s{\isacharcomma}{\kern0pt}\ q{\isasymrangle}\ {\isacharequal}{\kern0pt}\ {\isadigit{1}}{\isachardoublequoteclose}\isanewline
\ \ \ \ \ \ \isacommand{unfolding}\isamarkupfalse%
\ mem{\isacharunderscore}{\kern0pt}case{\isacharunderscore}{\kern0pt}def\ \isacommand{by}\isamarkupfalse%
\ auto\isanewline
\ \ \ \ \isacommand{moreover}\isamarkupfalse%
\isanewline
\ \ \ \ \isacommand{from}\isamarkupfalse%
\ this\ {\isacartoucheopen}t{\isadigit{2}}{\isasymin}M{\isacartoucheclose}\isanewline
\ \ \ \ \isacommand{have}\isamarkupfalse%
\ {\isachardoublequoteopen}r{\isasymin}M{\isachardoublequoteclose}\ {\isachardoublequoteopen}q{\isasymin}M{\isachardoublequoteclose}\ {\isachardoublequoteopen}s{\isasymin}M{\isachardoublequoteclose}\ {\isachardoublequoteopen}r\ {\isasymin}\ P\ {\isasymand}\ q\ {\isasymin}\ P\ {\isasymand}\ q\ {\isasympreceq}\ v\ {\isasymand}\ {\isasymlangle}s{\isacharcomma}{\kern0pt}\ r{\isasymrangle}\ {\isasymin}\ t{\isadigit{2}}\ {\isasymand}\ q\ {\isasympreceq}\ r\ {\isasymand}\ f\ {\isacharbackquote}{\kern0pt}\ {\isasymlangle}{\isadigit{0}}{\isacharcomma}{\kern0pt}\ t{\isadigit{1}}{\isacharcomma}{\kern0pt}\ s{\isacharcomma}{\kern0pt}\ q{\isasymrangle}\ {\isacharequal}{\kern0pt}\ {\isadigit{1}}{\isachardoublequoteclose}\isanewline
\ \ \ \ \ \ \isacommand{using}\isamarkupfalse%
\ transitivity\ P{\isacharunderscore}{\kern0pt}in{\isacharunderscore}{\kern0pt}M\ domain{\isacharunderscore}{\kern0pt}closed{\isacharbrackleft}{\kern0pt}of\ t{\isadigit{2}}{\isacharbrackright}{\kern0pt}\ \isacommand{by}\isamarkupfalse%
\ auto\isanewline
\ \ \ \ \isacommand{moreover}\isamarkupfalse%
\isanewline
\ \ \ \ \isacommand{note}\isamarkupfalse%
\ {\isacartoucheopen}t{\isadigit{1}}{\isasymin}M{\isacartoucheclose}\isanewline
\ \ \ \ \isacommand{ultimately}\isamarkupfalse%
\isanewline
\ \ \ \ \isacommand{have}\isamarkupfalse%
\ {\isachardoublequoteopen}{\isasymexists}q{\isasymin}M\ {\isachardot}{\kern0pt}\ {\isasymexists}s{\isasymin}M{\isachardot}{\kern0pt}\ {\isasymexists}r{\isasymin}M{\isachardot}{\kern0pt}\isanewline
\ \ \ \ \ \ \ \ \ r\ {\isasymin}\ P\ {\isasymand}\ q\ {\isasymin}\ P\ {\isasymand}\ {\isasymlangle}q{\isacharcomma}{\kern0pt}\ v{\isasymrangle}\ {\isasymin}\ M\ {\isasymand}\ {\isasymlangle}s{\isacharcomma}{\kern0pt}\ r{\isasymrangle}\ {\isasymin}\ M\ {\isasymand}\ {\isasymlangle}q{\isacharcomma}{\kern0pt}\ r{\isasymrangle}\ {\isasymin}\ M\ {\isasymand}\ {\isadigit{0}}\ {\isasymin}\ M\ {\isasymand}\isanewline
\ \ \ \ \ \ \ \ \ {\isasymlangle}{\isadigit{0}}{\isacharcomma}{\kern0pt}\ t{\isadigit{1}}{\isacharcomma}{\kern0pt}\ s{\isacharcomma}{\kern0pt}\ q{\isasymrangle}\ {\isasymin}\ M\ {\isasymand}\ q\ {\isasympreceq}\ v\ {\isasymand}\ {\isasymlangle}s{\isacharcomma}{\kern0pt}\ r{\isasymrangle}\ {\isasymin}\ t{\isadigit{2}}\ {\isasymand}\ q\ {\isasympreceq}\ r\ {\isasymand}\ f\ {\isacharbackquote}{\kern0pt}\ {\isasymlangle}{\isadigit{0}}{\isacharcomma}{\kern0pt}\ t{\isadigit{1}}{\isacharcomma}{\kern0pt}\ s{\isacharcomma}{\kern0pt}\ q{\isasymrangle}\ {\isacharequal}{\kern0pt}\ {\isadigit{1}}{\isachardoublequoteclose}\isanewline
\ \ \ \ \ \ \isacommand{using}\isamarkupfalse%
\ tuples{\isacharunderscore}{\kern0pt}in{\isacharunderscore}{\kern0pt}M\ zero{\isacharunderscore}{\kern0pt}in{\isacharunderscore}{\kern0pt}M\ \isacommand{by}\isamarkupfalse%
\ auto\isanewline
\ \ \isacommand{{\isacharbraceright}{\kern0pt}}\isamarkupfalse%
\isanewline
\ \ \isacommand{then}\isamarkupfalse%
\isanewline
\ \ \isacommand{show}\isamarkupfalse%
\ {\isachardoublequoteopen}is{\isacharunderscore}{\kern0pt}mem{\isacharunderscore}{\kern0pt}case{\isacharparenleft}{\kern0pt}{\isacharhash}{\kern0pt}{\isacharhash}{\kern0pt}M{\isacharcomma}{\kern0pt}\ t{\isadigit{1}}{\isacharcomma}{\kern0pt}\ t{\isadigit{2}}{\isacharcomma}{\kern0pt}\ p{\isacharcomma}{\kern0pt}\ P{\isacharcomma}{\kern0pt}\ leq{\isacharcomma}{\kern0pt}\ f{\isacharparenright}{\kern0pt}{\isachardoublequoteclose}\ \isakeyword{if}\ {\isachardoublequoteopen}mem{\isacharunderscore}{\kern0pt}case{\isacharparenleft}{\kern0pt}t{\isadigit{1}}{\isacharcomma}{\kern0pt}\ t{\isadigit{2}}{\isacharcomma}{\kern0pt}\ p{\isacharcomma}{\kern0pt}\ P{\isacharcomma}{\kern0pt}\ leq{\isacharcomma}{\kern0pt}\ f{\isacharparenright}{\kern0pt}{\isachardoublequoteclose}\isanewline
\ \ \ \ \isacommand{unfolding}\isamarkupfalse%
\ is{\isacharunderscore}{\kern0pt}mem{\isacharunderscore}{\kern0pt}case{\isacharunderscore}{\kern0pt}def\ \isacommand{using}\isamarkupfalse%
\ assms\ that\ \isacommand{by}\isamarkupfalse%
\ auto\isanewline
\isacommand{qed}\isamarkupfalse%
%
\endisatagproof
{\isafoldproof}%
%
\isadelimproof
\isanewline
%
\endisadelimproof
\isanewline
\isanewline
\isacommand{lemma}\isamarkupfalse%
\ Hfrc{\isacharunderscore}{\kern0pt}abs{\isacharcolon}{\kern0pt}\isanewline
\ \ {\isachardoublequoteopen}{\isasymlbrakk}fnnc{\isasymin}M{\isacharsemicolon}{\kern0pt}\ f{\isasymin}M{\isasymrbrakk}\ {\isasymLongrightarrow}\isanewline
\ \ \ is{\isacharunderscore}{\kern0pt}Hfrc{\isacharparenleft}{\kern0pt}{\isacharhash}{\kern0pt}{\isacharhash}{\kern0pt}M{\isacharcomma}{\kern0pt}P{\isacharcomma}{\kern0pt}leq{\isacharcomma}{\kern0pt}fnnc{\isacharcomma}{\kern0pt}f{\isacharparenright}{\kern0pt}\ {\isasymlongleftrightarrow}\ Hfrc{\isacharparenleft}{\kern0pt}P{\isacharcomma}{\kern0pt}leq{\isacharcomma}{\kern0pt}fnnc{\isacharcomma}{\kern0pt}f{\isacharparenright}{\kern0pt}{\isachardoublequoteclose}\isanewline
%
\isadelimproof
\ \ %
\endisadelimproof
%
\isatagproof
\isacommand{unfolding}\isamarkupfalse%
\ is{\isacharunderscore}{\kern0pt}Hfrc{\isacharunderscore}{\kern0pt}def\ Hfrc{\isacharunderscore}{\kern0pt}def\ \isacommand{using}\isamarkupfalse%
\ pair{\isacharunderscore}{\kern0pt}in{\isacharunderscore}{\kern0pt}M{\isacharunderscore}{\kern0pt}iff\isanewline
\ \ \isacommand{by}\isamarkupfalse%
\ auto%
\endisatagproof
{\isafoldproof}%
%
\isadelimproof
\isanewline
%
\endisadelimproof
\isanewline
\isacommand{lemma}\isamarkupfalse%
\ Hfrc{\isacharunderscore}{\kern0pt}at{\isacharunderscore}{\kern0pt}abs{\isacharcolon}{\kern0pt}\isanewline
\ \ {\isachardoublequoteopen}{\isasymlbrakk}fnnc{\isasymin}M{\isacharsemicolon}{\kern0pt}\ f{\isasymin}M\ {\isacharsemicolon}{\kern0pt}\ z{\isasymin}M{\isasymrbrakk}\ {\isasymLongrightarrow}\isanewline
\ \ \ is{\isacharunderscore}{\kern0pt}Hfrc{\isacharunderscore}{\kern0pt}at{\isacharparenleft}{\kern0pt}{\isacharhash}{\kern0pt}{\isacharhash}{\kern0pt}M{\isacharcomma}{\kern0pt}P{\isacharcomma}{\kern0pt}leq{\isacharcomma}{\kern0pt}fnnc{\isacharcomma}{\kern0pt}f{\isacharcomma}{\kern0pt}z{\isacharparenright}{\kern0pt}\ {\isasymlongleftrightarrow}\ \ z\ {\isacharequal}{\kern0pt}\ bool{\isacharunderscore}{\kern0pt}of{\isacharunderscore}{\kern0pt}o{\isacharparenleft}{\kern0pt}Hfrc{\isacharparenleft}{\kern0pt}P{\isacharcomma}{\kern0pt}leq{\isacharcomma}{\kern0pt}fnnc{\isacharcomma}{\kern0pt}f{\isacharparenright}{\kern0pt}{\isacharparenright}{\kern0pt}\ {\isachardoublequoteclose}\isanewline
%
\isadelimproof
\ \ %
\endisadelimproof
%
\isatagproof
\isacommand{unfolding}\isamarkupfalse%
\ is{\isacharunderscore}{\kern0pt}Hfrc{\isacharunderscore}{\kern0pt}at{\isacharunderscore}{\kern0pt}def\ \isacommand{using}\isamarkupfalse%
\ Hfrc{\isacharunderscore}{\kern0pt}abs\isanewline
\ \ \isacommand{by}\isamarkupfalse%
\ auto%
\endisatagproof
{\isafoldproof}%
%
\isadelimproof
\isanewline
%
\endisadelimproof
\isanewline
\isacommand{lemma}\isamarkupfalse%
\ components{\isacharunderscore}{\kern0pt}closed\ {\isacharcolon}{\kern0pt}\isanewline
\ \ {\isachardoublequoteopen}x{\isasymin}M\ {\isasymLongrightarrow}\ ftype{\isacharparenleft}{\kern0pt}x{\isacharparenright}{\kern0pt}{\isasymin}M{\isachardoublequoteclose}\isanewline
\ \ {\isachardoublequoteopen}x{\isasymin}M\ {\isasymLongrightarrow}\ name{\isadigit{1}}{\isacharparenleft}{\kern0pt}x{\isacharparenright}{\kern0pt}{\isasymin}M{\isachardoublequoteclose}\isanewline
\ \ {\isachardoublequoteopen}x{\isasymin}M\ {\isasymLongrightarrow}\ name{\isadigit{2}}{\isacharparenleft}{\kern0pt}x{\isacharparenright}{\kern0pt}{\isasymin}M{\isachardoublequoteclose}\isanewline
\ \ {\isachardoublequoteopen}x{\isasymin}M\ {\isasymLongrightarrow}\ cond{\isacharunderscore}{\kern0pt}of{\isacharparenleft}{\kern0pt}x{\isacharparenright}{\kern0pt}{\isasymin}M{\isachardoublequoteclose}\isanewline
%
\isadelimproof
\ \ %
\endisadelimproof
%
\isatagproof
\isacommand{unfolding}\isamarkupfalse%
\ ftype{\isacharunderscore}{\kern0pt}def\ name{\isadigit{1}}{\isacharunderscore}{\kern0pt}def\ name{\isadigit{2}}{\isacharunderscore}{\kern0pt}def\ cond{\isacharunderscore}{\kern0pt}of{\isacharunderscore}{\kern0pt}def\ \isacommand{using}\isamarkupfalse%
\ fst{\isacharunderscore}{\kern0pt}snd{\isacharunderscore}{\kern0pt}closed\ \isacommand{by}\isamarkupfalse%
\ simp{\isacharunderscore}{\kern0pt}all%
\endisatagproof
{\isafoldproof}%
%
\isadelimproof
\isanewline
%
\endisadelimproof
\isanewline
\isacommand{lemma}\isamarkupfalse%
\ ecloseN{\isacharunderscore}{\kern0pt}closed{\isacharcolon}{\kern0pt}\isanewline
\ \ {\isachardoublequoteopen}{\isacharparenleft}{\kern0pt}{\isacharhash}{\kern0pt}{\isacharhash}{\kern0pt}M{\isacharparenright}{\kern0pt}{\isacharparenleft}{\kern0pt}A{\isacharparenright}{\kern0pt}\ {\isasymLongrightarrow}\ {\isacharparenleft}{\kern0pt}{\isacharhash}{\kern0pt}{\isacharhash}{\kern0pt}M{\isacharparenright}{\kern0pt}{\isacharparenleft}{\kern0pt}ecloseN{\isacharparenleft}{\kern0pt}A{\isacharparenright}{\kern0pt}{\isacharparenright}{\kern0pt}{\isachardoublequoteclose}\isanewline
\ \ {\isachardoublequoteopen}{\isacharparenleft}{\kern0pt}{\isacharhash}{\kern0pt}{\isacharhash}{\kern0pt}M{\isacharparenright}{\kern0pt}{\isacharparenleft}{\kern0pt}A{\isacharparenright}{\kern0pt}\ {\isasymLongrightarrow}\ {\isacharparenleft}{\kern0pt}{\isacharhash}{\kern0pt}{\isacharhash}{\kern0pt}M{\isacharparenright}{\kern0pt}{\isacharparenleft}{\kern0pt}eclose{\isacharunderscore}{\kern0pt}n{\isacharparenleft}{\kern0pt}name{\isadigit{1}}{\isacharcomma}{\kern0pt}A{\isacharparenright}{\kern0pt}{\isacharparenright}{\kern0pt}{\isachardoublequoteclose}\isanewline
\ \ {\isachardoublequoteopen}{\isacharparenleft}{\kern0pt}{\isacharhash}{\kern0pt}{\isacharhash}{\kern0pt}M{\isacharparenright}{\kern0pt}{\isacharparenleft}{\kern0pt}A{\isacharparenright}{\kern0pt}\ {\isasymLongrightarrow}\ {\isacharparenleft}{\kern0pt}{\isacharhash}{\kern0pt}{\isacharhash}{\kern0pt}M{\isacharparenright}{\kern0pt}{\isacharparenleft}{\kern0pt}eclose{\isacharunderscore}{\kern0pt}n{\isacharparenleft}{\kern0pt}name{\isadigit{2}}{\isacharcomma}{\kern0pt}A{\isacharparenright}{\kern0pt}{\isacharparenright}{\kern0pt}{\isachardoublequoteclose}\isanewline
%
\isadelimproof
\ \ %
\endisadelimproof
%
\isatagproof
\isacommand{unfolding}\isamarkupfalse%
\ ecloseN{\isacharunderscore}{\kern0pt}def\ eclose{\isacharunderscore}{\kern0pt}n{\isacharunderscore}{\kern0pt}def\isanewline
\ \ \isacommand{using}\isamarkupfalse%
\ components{\isacharunderscore}{\kern0pt}closed\ eclose{\isacharunderscore}{\kern0pt}closed\ singletonM\ Un{\isacharunderscore}{\kern0pt}closed\ \isacommand{by}\isamarkupfalse%
\ auto%
\endisatagproof
{\isafoldproof}%
%
\isadelimproof
\isanewline
%
\endisadelimproof
\isanewline
\isacommand{lemma}\isamarkupfalse%
\ is{\isacharunderscore}{\kern0pt}eclose{\isacharunderscore}{\kern0pt}n{\isacharunderscore}{\kern0pt}abs\ {\isacharcolon}{\kern0pt}\isanewline
\ \ \isakeyword{assumes}\ {\isachardoublequoteopen}x{\isasymin}M{\isachardoublequoteclose}\ {\isachardoublequoteopen}ec{\isasymin}M{\isachardoublequoteclose}\isanewline
\ \ \isakeyword{shows}\ {\isachardoublequoteopen}is{\isacharunderscore}{\kern0pt}eclose{\isacharunderscore}{\kern0pt}n{\isacharparenleft}{\kern0pt}{\isacharhash}{\kern0pt}{\isacharhash}{\kern0pt}M{\isacharcomma}{\kern0pt}is{\isacharunderscore}{\kern0pt}name{\isadigit{1}}{\isacharcomma}{\kern0pt}ec{\isacharcomma}{\kern0pt}x{\isacharparenright}{\kern0pt}\ {\isasymlongleftrightarrow}\ ec\ {\isacharequal}{\kern0pt}\ eclose{\isacharunderscore}{\kern0pt}n{\isacharparenleft}{\kern0pt}name{\isadigit{1}}{\isacharcomma}{\kern0pt}x{\isacharparenright}{\kern0pt}{\isachardoublequoteclose}\isanewline
\ \ \ \ {\isachardoublequoteopen}is{\isacharunderscore}{\kern0pt}eclose{\isacharunderscore}{\kern0pt}n{\isacharparenleft}{\kern0pt}{\isacharhash}{\kern0pt}{\isacharhash}{\kern0pt}M{\isacharcomma}{\kern0pt}is{\isacharunderscore}{\kern0pt}name{\isadigit{2}}{\isacharcomma}{\kern0pt}ec{\isacharcomma}{\kern0pt}x{\isacharparenright}{\kern0pt}\ {\isasymlongleftrightarrow}\ ec\ {\isacharequal}{\kern0pt}\ eclose{\isacharunderscore}{\kern0pt}n{\isacharparenleft}{\kern0pt}name{\isadigit{2}}{\isacharcomma}{\kern0pt}x{\isacharparenright}{\kern0pt}{\isachardoublequoteclose}\isanewline
%
\isadelimproof
\ \ %
\endisadelimproof
%
\isatagproof
\isacommand{unfolding}\isamarkupfalse%
\ is{\isacharunderscore}{\kern0pt}eclose{\isacharunderscore}{\kern0pt}n{\isacharunderscore}{\kern0pt}def\ eclose{\isacharunderscore}{\kern0pt}n{\isacharunderscore}{\kern0pt}def\isanewline
\ \ \isacommand{using}\isamarkupfalse%
\ assms\ name{\isadigit{1}}{\isacharunderscore}{\kern0pt}abs\ name{\isadigit{2}}{\isacharunderscore}{\kern0pt}abs\ eclose{\isacharunderscore}{\kern0pt}abs\ singletonM\ components{\isacharunderscore}{\kern0pt}closed\isanewline
\ \ \isacommand{by}\isamarkupfalse%
\ auto%
\endisatagproof
{\isafoldproof}%
%
\isadelimproof
\isanewline
%
\endisadelimproof
\isanewline
\isanewline
\isacommand{lemma}\isamarkupfalse%
\ is{\isacharunderscore}{\kern0pt}ecloseN{\isacharunderscore}{\kern0pt}abs\ {\isacharcolon}{\kern0pt}\isanewline
\ \ {\isachardoublequoteopen}{\isasymlbrakk}x{\isasymin}M{\isacharsemicolon}{\kern0pt}ec{\isasymin}M{\isasymrbrakk}\ {\isasymLongrightarrow}\ is{\isacharunderscore}{\kern0pt}ecloseN{\isacharparenleft}{\kern0pt}{\isacharhash}{\kern0pt}{\isacharhash}{\kern0pt}M{\isacharcomma}{\kern0pt}ec{\isacharcomma}{\kern0pt}x{\isacharparenright}{\kern0pt}\ {\isasymlongleftrightarrow}\ ec\ {\isacharequal}{\kern0pt}\ ecloseN{\isacharparenleft}{\kern0pt}x{\isacharparenright}{\kern0pt}{\isachardoublequoteclose}\isanewline
%
\isadelimproof
\ \ %
\endisadelimproof
%
\isatagproof
\isacommand{unfolding}\isamarkupfalse%
\ is{\isacharunderscore}{\kern0pt}ecloseN{\isacharunderscore}{\kern0pt}def\ ecloseN{\isacharunderscore}{\kern0pt}def\isanewline
\ \ \isacommand{using}\isamarkupfalse%
\ is{\isacharunderscore}{\kern0pt}eclose{\isacharunderscore}{\kern0pt}n{\isacharunderscore}{\kern0pt}abs\ Un{\isacharunderscore}{\kern0pt}closed\ union{\isacharunderscore}{\kern0pt}abs\ ecloseN{\isacharunderscore}{\kern0pt}closed\isanewline
\ \ \isacommand{by}\isamarkupfalse%
\ auto%
\endisatagproof
{\isafoldproof}%
%
\isadelimproof
\isanewline
%
\endisadelimproof
\isanewline
\isacommand{lemma}\isamarkupfalse%
\ frecR{\isacharunderscore}{\kern0pt}abs\ {\isacharcolon}{\kern0pt}\isanewline
\ \ {\isachardoublequoteopen}x{\isasymin}M\ {\isasymLongrightarrow}\ y{\isasymin}M\ {\isasymLongrightarrow}\ frecR{\isacharparenleft}{\kern0pt}x{\isacharcomma}{\kern0pt}y{\isacharparenright}{\kern0pt}\ {\isasymlongleftrightarrow}\ is{\isacharunderscore}{\kern0pt}frecR{\isacharparenleft}{\kern0pt}{\isacharhash}{\kern0pt}{\isacharhash}{\kern0pt}M{\isacharcomma}{\kern0pt}x{\isacharcomma}{\kern0pt}y{\isacharparenright}{\kern0pt}{\isachardoublequoteclose}\isanewline
%
\isadelimproof
\ \ %
\endisadelimproof
%
\isatagproof
\isacommand{unfolding}\isamarkupfalse%
\ frecR{\isacharunderscore}{\kern0pt}def\ is{\isacharunderscore}{\kern0pt}frecR{\isacharunderscore}{\kern0pt}def\ \isacommand{using}\isamarkupfalse%
\ components{\isacharunderscore}{\kern0pt}closed\ domain{\isacharunderscore}{\kern0pt}closed\ \isacommand{by}\isamarkupfalse%
\ simp%
\endisatagproof
{\isafoldproof}%
%
\isadelimproof
\isanewline
%
\endisadelimproof
\isanewline
\isacommand{lemma}\isamarkupfalse%
\ frecrelP{\isacharunderscore}{\kern0pt}abs\ {\isacharcolon}{\kern0pt}\isanewline
\ \ {\isachardoublequoteopen}z{\isasymin}M\ {\isasymLongrightarrow}\ frecrelP{\isacharparenleft}{\kern0pt}{\isacharhash}{\kern0pt}{\isacharhash}{\kern0pt}M{\isacharcomma}{\kern0pt}z{\isacharparenright}{\kern0pt}\ {\isasymlongleftrightarrow}\ {\isacharparenleft}{\kern0pt}{\isasymexists}x\ y{\isachardot}{\kern0pt}\ z\ {\isacharequal}{\kern0pt}\ {\isasymlangle}x{\isacharcomma}{\kern0pt}y{\isasymrangle}\ {\isasymand}\ frecR{\isacharparenleft}{\kern0pt}x{\isacharcomma}{\kern0pt}y{\isacharparenright}{\kern0pt}{\isacharparenright}{\kern0pt}{\isachardoublequoteclose}\isanewline
%
\isadelimproof
\ \ %
\endisadelimproof
%
\isatagproof
\isacommand{using}\isamarkupfalse%
\ pair{\isacharunderscore}{\kern0pt}in{\isacharunderscore}{\kern0pt}M{\isacharunderscore}{\kern0pt}iff\ frecR{\isacharunderscore}{\kern0pt}abs\ \isacommand{unfolding}\isamarkupfalse%
\ frecrelP{\isacharunderscore}{\kern0pt}def\ \isacommand{by}\isamarkupfalse%
\ auto%
\endisatagproof
{\isafoldproof}%
%
\isadelimproof
\isanewline
%
\endisadelimproof
\isanewline
\isacommand{lemma}\isamarkupfalse%
\ frecrel{\isacharunderscore}{\kern0pt}abs{\isacharcolon}{\kern0pt}\isanewline
\ \ \isakeyword{assumes}\isanewline
\ \ \ \ {\isachardoublequoteopen}A{\isasymin}M{\isachardoublequoteclose}\ {\isachardoublequoteopen}r{\isasymin}M{\isachardoublequoteclose}\isanewline
\ \ \isakeyword{shows}\isanewline
\ \ \ \ {\isachardoublequoteopen}is{\isacharunderscore}{\kern0pt}frecrel{\isacharparenleft}{\kern0pt}{\isacharhash}{\kern0pt}{\isacharhash}{\kern0pt}M{\isacharcomma}{\kern0pt}A{\isacharcomma}{\kern0pt}r{\isacharparenright}{\kern0pt}\ {\isasymlongleftrightarrow}\ \ r\ {\isacharequal}{\kern0pt}\ frecrel{\isacharparenleft}{\kern0pt}A{\isacharparenright}{\kern0pt}{\isachardoublequoteclose}\isanewline
%
\isadelimproof
%
\endisadelimproof
%
\isatagproof
\isacommand{proof}\isamarkupfalse%
\ {\isacharminus}{\kern0pt}\isanewline
\ \ \isacommand{from}\isamarkupfalse%
\ {\isacartoucheopen}A{\isasymin}M{\isacartoucheclose}\isanewline
\ \ \isacommand{have}\isamarkupfalse%
\ {\isachardoublequoteopen}z{\isasymin}M{\isachardoublequoteclose}\ \isakeyword{if}\ {\isachardoublequoteopen}z{\isasymin}A{\isasymtimes}A{\isachardoublequoteclose}\ \isakeyword{for}\ z\isanewline
\ \ \ \ \isacommand{using}\isamarkupfalse%
\ cartprod{\isacharunderscore}{\kern0pt}closed\ transitivity\ that\ \isacommand{by}\isamarkupfalse%
\ simp\isanewline
\ \ \isacommand{then}\isamarkupfalse%
\isanewline
\ \ \isacommand{have}\isamarkupfalse%
\ {\isachardoublequoteopen}Collect{\isacharparenleft}{\kern0pt}A{\isasymtimes}A{\isacharcomma}{\kern0pt}frecrelP{\isacharparenleft}{\kern0pt}{\isacharhash}{\kern0pt}{\isacharhash}{\kern0pt}M{\isacharparenright}{\kern0pt}{\isacharparenright}{\kern0pt}\ {\isacharequal}{\kern0pt}\ Collect{\isacharparenleft}{\kern0pt}A{\isasymtimes}A{\isacharcomma}{\kern0pt}{\isasymlambda}z{\isachardot}{\kern0pt}\ {\isacharparenleft}{\kern0pt}{\isasymexists}x\ y{\isachardot}{\kern0pt}\ z\ {\isacharequal}{\kern0pt}\ {\isasymlangle}x{\isacharcomma}{\kern0pt}y{\isasymrangle}\ {\isasymand}\ frecR{\isacharparenleft}{\kern0pt}x{\isacharcomma}{\kern0pt}y{\isacharparenright}{\kern0pt}{\isacharparenright}{\kern0pt}{\isacharparenright}{\kern0pt}{\isachardoublequoteclose}\isanewline
\ \ \ \ \isacommand{using}\isamarkupfalse%
\ Collect{\isacharunderscore}{\kern0pt}cong{\isacharbrackleft}{\kern0pt}of\ {\isachardoublequoteopen}A{\isasymtimes}A{\isachardoublequoteclose}\ {\isachardoublequoteopen}A{\isasymtimes}A{\isachardoublequoteclose}\ {\isachardoublequoteopen}frecrelP{\isacharparenleft}{\kern0pt}{\isacharhash}{\kern0pt}{\isacharhash}{\kern0pt}M{\isacharparenright}{\kern0pt}{\isachardoublequoteclose}{\isacharbrackright}{\kern0pt}\ assms\ frecrelP{\isacharunderscore}{\kern0pt}abs\ \isacommand{by}\isamarkupfalse%
\ simp\isanewline
\ \ \isacommand{with}\isamarkupfalse%
\ assms\isanewline
\ \ \isacommand{show}\isamarkupfalse%
\ {\isacharquery}{\kern0pt}thesis\ \isacommand{unfolding}\isamarkupfalse%
\ is{\isacharunderscore}{\kern0pt}frecrel{\isacharunderscore}{\kern0pt}def\ def{\isacharunderscore}{\kern0pt}frecrel\ \isacommand{using}\isamarkupfalse%
\ cartprod{\isacharunderscore}{\kern0pt}closed\isanewline
\ \ \ \ \isacommand{by}\isamarkupfalse%
\ simp\isanewline
\isacommand{qed}\isamarkupfalse%
%
\endisatagproof
{\isafoldproof}%
%
\isadelimproof
\isanewline
%
\endisadelimproof
\isanewline
\isacommand{lemma}\isamarkupfalse%
\ frecrel{\isacharunderscore}{\kern0pt}closed{\isacharcolon}{\kern0pt}\isanewline
\ \ \isakeyword{assumes}\isanewline
\ \ \ \ {\isachardoublequoteopen}x{\isasymin}M{\isachardoublequoteclose}\isanewline
\ \ \isakeyword{shows}\isanewline
\ \ \ \ {\isachardoublequoteopen}frecrel{\isacharparenleft}{\kern0pt}x{\isacharparenright}{\kern0pt}{\isasymin}M{\isachardoublequoteclose}\isanewline
%
\isadelimproof
%
\endisadelimproof
%
\isatagproof
\isacommand{proof}\isamarkupfalse%
\ {\isacharminus}{\kern0pt}\isanewline
\ \ \isacommand{have}\isamarkupfalse%
\ {\isachardoublequoteopen}Collect{\isacharparenleft}{\kern0pt}x{\isasymtimes}x{\isacharcomma}{\kern0pt}{\isasymlambda}z{\isachardot}{\kern0pt}\ {\isacharparenleft}{\kern0pt}{\isasymexists}x\ y{\isachardot}{\kern0pt}\ z\ {\isacharequal}{\kern0pt}\ {\isasymlangle}x{\isacharcomma}{\kern0pt}y{\isasymrangle}\ {\isasymand}\ frecR{\isacharparenleft}{\kern0pt}x{\isacharcomma}{\kern0pt}y{\isacharparenright}{\kern0pt}{\isacharparenright}{\kern0pt}{\isacharparenright}{\kern0pt}{\isasymin}M{\isachardoublequoteclose}\isanewline
\ \ \ \ \isacommand{using}\isamarkupfalse%
\ Collect{\isacharunderscore}{\kern0pt}in{\isacharunderscore}{\kern0pt}M{\isacharunderscore}{\kern0pt}{\isadigit{0}}p{\isacharbrackleft}{\kern0pt}of\ {\isachardoublequoteopen}frecrelP{\isacharunderscore}{\kern0pt}fm{\isacharparenleft}{\kern0pt}{\isadigit{0}}{\isacharparenright}{\kern0pt}{\isachardoublequoteclose}{\isacharbrackright}{\kern0pt}\ arity{\isacharunderscore}{\kern0pt}frecrelP{\isacharunderscore}{\kern0pt}fm\ sats{\isacharunderscore}{\kern0pt}frecrelP{\isacharunderscore}{\kern0pt}fm\isanewline
\ \ \ \ \ \ frecrelP{\isacharunderscore}{\kern0pt}abs\ {\isacartoucheopen}x{\isasymin}M{\isacartoucheclose}\ cartprod{\isacharunderscore}{\kern0pt}closed\ \isacommand{by}\isamarkupfalse%
\ simp\isanewline
\ \ \isacommand{then}\isamarkupfalse%
\ \isacommand{show}\isamarkupfalse%
\ {\isacharquery}{\kern0pt}thesis\isanewline
\ \ \ \ \isacommand{unfolding}\isamarkupfalse%
\ frecrel{\isacharunderscore}{\kern0pt}def\ Rrel{\isacharunderscore}{\kern0pt}def\ frecrelP{\isacharunderscore}{\kern0pt}def\ \isacommand{by}\isamarkupfalse%
\ simp\isanewline
\isacommand{qed}\isamarkupfalse%
%
\endisatagproof
{\isafoldproof}%
%
\isadelimproof
\isanewline
%
\endisadelimproof
\isanewline
\isacommand{lemma}\isamarkupfalse%
\ field{\isacharunderscore}{\kern0pt}frecrel\ {\isacharcolon}{\kern0pt}\ {\isachardoublequoteopen}field{\isacharparenleft}{\kern0pt}frecrel{\isacharparenleft}{\kern0pt}names{\isacharunderscore}{\kern0pt}below{\isacharparenleft}{\kern0pt}P{\isacharcomma}{\kern0pt}x{\isacharparenright}{\kern0pt}{\isacharparenright}{\kern0pt}{\isacharparenright}{\kern0pt}\ {\isasymsubseteq}\ names{\isacharunderscore}{\kern0pt}below{\isacharparenleft}{\kern0pt}P{\isacharcomma}{\kern0pt}x{\isacharparenright}{\kern0pt}{\isachardoublequoteclose}\isanewline
%
\isadelimproof
\ \ %
\endisadelimproof
%
\isatagproof
\isacommand{unfolding}\isamarkupfalse%
\ frecrel{\isacharunderscore}{\kern0pt}def\isanewline
\ \ \isacommand{using}\isamarkupfalse%
\ field{\isacharunderscore}{\kern0pt}Rrel\ \isacommand{by}\isamarkupfalse%
\ simp%
\endisatagproof
{\isafoldproof}%
%
\isadelimproof
\isanewline
%
\endisadelimproof
\isanewline
\isacommand{lemma}\isamarkupfalse%
\ forcerelD\ {\isacharcolon}{\kern0pt}\ {\isachardoublequoteopen}uv\ {\isasymin}\ forcerel{\isacharparenleft}{\kern0pt}P{\isacharcomma}{\kern0pt}x{\isacharparenright}{\kern0pt}\ {\isasymLongrightarrow}\ uv{\isasymin}\ names{\isacharunderscore}{\kern0pt}below{\isacharparenleft}{\kern0pt}P{\isacharcomma}{\kern0pt}x{\isacharparenright}{\kern0pt}\ {\isasymtimes}\ names{\isacharunderscore}{\kern0pt}below{\isacharparenleft}{\kern0pt}P{\isacharcomma}{\kern0pt}x{\isacharparenright}{\kern0pt}{\isachardoublequoteclose}\isanewline
%
\isadelimproof
\ \ %
\endisadelimproof
%
\isatagproof
\isacommand{unfolding}\isamarkupfalse%
\ forcerel{\isacharunderscore}{\kern0pt}def\isanewline
\ \ \isacommand{using}\isamarkupfalse%
\ trancl{\isacharunderscore}{\kern0pt}type\ field{\isacharunderscore}{\kern0pt}frecrel\ \isacommand{by}\isamarkupfalse%
\ blast%
\endisatagproof
{\isafoldproof}%
%
\isadelimproof
\isanewline
%
\endisadelimproof
\isanewline
\isacommand{lemma}\isamarkupfalse%
\ wf{\isacharunderscore}{\kern0pt}forcerel\ {\isacharcolon}{\kern0pt}\isanewline
\ \ {\isachardoublequoteopen}wf{\isacharparenleft}{\kern0pt}forcerel{\isacharparenleft}{\kern0pt}P{\isacharcomma}{\kern0pt}x{\isacharparenright}{\kern0pt}{\isacharparenright}{\kern0pt}{\isachardoublequoteclose}\isanewline
%
\isadelimproof
\ \ %
\endisadelimproof
%
\isatagproof
\isacommand{unfolding}\isamarkupfalse%
\ forcerel{\isacharunderscore}{\kern0pt}def\ \isacommand{using}\isamarkupfalse%
\ wf{\isacharunderscore}{\kern0pt}trancl\ wf{\isacharunderscore}{\kern0pt}frecrel\ \isacommand{{\isachardot}{\kern0pt}}\isamarkupfalse%
%
\endisatagproof
{\isafoldproof}%
%
\isadelimproof
\isanewline
%
\endisadelimproof
\isanewline
\isacommand{lemma}\isamarkupfalse%
\ restrict{\isacharunderscore}{\kern0pt}trancl{\isacharunderscore}{\kern0pt}forcerel{\isacharcolon}{\kern0pt}\isanewline
\ \ \isakeyword{assumes}\ {\isachardoublequoteopen}frecR{\isacharparenleft}{\kern0pt}w{\isacharcomma}{\kern0pt}y{\isacharparenright}{\kern0pt}{\isachardoublequoteclose}\isanewline
\ \ \isakeyword{shows}\ {\isachardoublequoteopen}restrict{\isacharparenleft}{\kern0pt}f{\isacharcomma}{\kern0pt}frecrel{\isacharparenleft}{\kern0pt}names{\isacharunderscore}{\kern0pt}below{\isacharparenleft}{\kern0pt}P{\isacharcomma}{\kern0pt}x{\isacharparenright}{\kern0pt}{\isacharparenright}{\kern0pt}{\isacharminus}{\kern0pt}{\isacharbackquote}{\kern0pt}{\isacharbackquote}{\kern0pt}{\isacharbraceleft}{\kern0pt}y{\isacharbraceright}{\kern0pt}{\isacharparenright}{\kern0pt}{\isacharbackquote}{\kern0pt}w\isanewline
\ \ \ \ \ \ \ {\isacharequal}{\kern0pt}\ restrict{\isacharparenleft}{\kern0pt}f{\isacharcomma}{\kern0pt}forcerel{\isacharparenleft}{\kern0pt}P{\isacharcomma}{\kern0pt}x{\isacharparenright}{\kern0pt}{\isacharminus}{\kern0pt}{\isacharbackquote}{\kern0pt}{\isacharbackquote}{\kern0pt}{\isacharbraceleft}{\kern0pt}y{\isacharbraceright}{\kern0pt}{\isacharparenright}{\kern0pt}{\isacharbackquote}{\kern0pt}w{\isachardoublequoteclose}\isanewline
%
\isadelimproof
\ \ %
\endisadelimproof
%
\isatagproof
\isacommand{unfolding}\isamarkupfalse%
\ forcerel{\isacharunderscore}{\kern0pt}def\ frecrel{\isacharunderscore}{\kern0pt}def\ \isacommand{using}\isamarkupfalse%
\ assms\ restrict{\isacharunderscore}{\kern0pt}trancl{\isacharunderscore}{\kern0pt}Rrel{\isacharbrackleft}{\kern0pt}of\ frecR{\isacharbrackright}{\kern0pt}\isanewline
\ \ \isacommand{by}\isamarkupfalse%
\ simp%
\endisatagproof
{\isafoldproof}%
%
\isadelimproof
\isanewline
%
\endisadelimproof
\isanewline
\isacommand{lemma}\isamarkupfalse%
\ names{\isacharunderscore}{\kern0pt}belowI\ {\isacharcolon}{\kern0pt}\isanewline
\ \ \isakeyword{assumes}\ {\isachardoublequoteopen}frecR{\isacharparenleft}{\kern0pt}{\isasymlangle}ft{\isacharcomma}{\kern0pt}n{\isadigit{1}}{\isacharcomma}{\kern0pt}n{\isadigit{2}}{\isacharcomma}{\kern0pt}p{\isasymrangle}{\isacharcomma}{\kern0pt}{\isasymlangle}a{\isacharcomma}{\kern0pt}b{\isacharcomma}{\kern0pt}c{\isacharcomma}{\kern0pt}d{\isasymrangle}{\isacharparenright}{\kern0pt}{\isachardoublequoteclose}\ {\isachardoublequoteopen}p{\isasymin}P{\isachardoublequoteclose}\isanewline
\ \ \isakeyword{shows}\ {\isachardoublequoteopen}{\isasymlangle}ft{\isacharcomma}{\kern0pt}n{\isadigit{1}}{\isacharcomma}{\kern0pt}n{\isadigit{2}}{\isacharcomma}{\kern0pt}p{\isasymrangle}\ {\isasymin}\ names{\isacharunderscore}{\kern0pt}below{\isacharparenleft}{\kern0pt}P{\isacharcomma}{\kern0pt}{\isasymlangle}a{\isacharcomma}{\kern0pt}b{\isacharcomma}{\kern0pt}c{\isacharcomma}{\kern0pt}d{\isasymrangle}{\isacharparenright}{\kern0pt}{\isachardoublequoteclose}\ {\isacharparenleft}{\kern0pt}\isakeyword{is}\ {\isachardoublequoteopen}{\isacharquery}{\kern0pt}x\ {\isasymin}\ names{\isacharunderscore}{\kern0pt}below{\isacharparenleft}{\kern0pt}{\isacharunderscore}{\kern0pt}{\isacharcomma}{\kern0pt}{\isacharquery}{\kern0pt}y{\isacharparenright}{\kern0pt}{\isachardoublequoteclose}{\isacharparenright}{\kern0pt}\isanewline
%
\isadelimproof
%
\endisadelimproof
%
\isatagproof
\isacommand{proof}\isamarkupfalse%
\ {\isacharminus}{\kern0pt}\isanewline
\ \ \isacommand{from}\isamarkupfalse%
\ assms\isanewline
\ \ \isacommand{have}\isamarkupfalse%
\ {\isachardoublequoteopen}ft\ {\isasymin}\ {\isadigit{2}}{\isachardoublequoteclose}\ {\isachardoublequoteopen}a\ {\isasymin}\ {\isadigit{2}}{\isachardoublequoteclose}\isanewline
\ \ \ \ \isacommand{unfolding}\isamarkupfalse%
\ frecR{\isacharunderscore}{\kern0pt}def\ \isacommand{by}\isamarkupfalse%
\ {\isacharparenleft}{\kern0pt}auto\ simp\ add{\isacharcolon}{\kern0pt}components{\isacharunderscore}{\kern0pt}simp{\isacharparenright}{\kern0pt}\isanewline
\ \ \isacommand{from}\isamarkupfalse%
\ assms\isanewline
\ \ \isacommand{consider}\isamarkupfalse%
\ {\isacharparenleft}{\kern0pt}e{\isacharparenright}{\kern0pt}\ {\isachardoublequoteopen}n{\isadigit{1}}\ {\isasymin}\ domain{\isacharparenleft}{\kern0pt}b{\isacharparenright}{\kern0pt}\ {\isasymunion}\ domain{\isacharparenleft}{\kern0pt}c{\isacharparenright}{\kern0pt}\ {\isasymand}\ {\isacharparenleft}{\kern0pt}n{\isadigit{2}}\ {\isacharequal}{\kern0pt}\ b\ {\isasymor}\ n{\isadigit{2}}\ {\isacharequal}{\kern0pt}c{\isacharparenright}{\kern0pt}{\isachardoublequoteclose}\isanewline
\ \ \ \ {\isacharbar}{\kern0pt}\ {\isacharparenleft}{\kern0pt}m{\isacharparenright}{\kern0pt}\ {\isachardoublequoteopen}n{\isadigit{1}}\ {\isacharequal}{\kern0pt}\ b\ {\isasymand}\ n{\isadigit{2}}\ {\isasymin}\ domain{\isacharparenleft}{\kern0pt}c{\isacharparenright}{\kern0pt}{\isachardoublequoteclose}\isanewline
\ \ \ \ \isacommand{unfolding}\isamarkupfalse%
\ frecR{\isacharunderscore}{\kern0pt}def\ \isacommand{by}\isamarkupfalse%
\ {\isacharparenleft}{\kern0pt}auto\ simp\ add{\isacharcolon}{\kern0pt}components{\isacharunderscore}{\kern0pt}simp{\isacharparenright}{\kern0pt}\isanewline
\ \ \isacommand{then}\isamarkupfalse%
\ \isacommand{show}\isamarkupfalse%
\ {\isacharquery}{\kern0pt}thesis\isanewline
\ \ \isacommand{proof}\isamarkupfalse%
\ cases\isanewline
\ \ \ \ \isacommand{case}\isamarkupfalse%
\ e\isanewline
\ \ \ \ \isacommand{then}\isamarkupfalse%
\isanewline
\ \ \ \ \isacommand{have}\isamarkupfalse%
\ {\isachardoublequoteopen}n{\isadigit{1}}\ {\isasymin}\ eclose{\isacharparenleft}{\kern0pt}b{\isacharparenright}{\kern0pt}\ {\isasymor}\ n{\isadigit{1}}\ {\isasymin}\ eclose{\isacharparenleft}{\kern0pt}c{\isacharparenright}{\kern0pt}{\isachardoublequoteclose}\isanewline
\ \ \ \ \ \ \isacommand{using}\isamarkupfalse%
\ Un{\isacharunderscore}{\kern0pt}iff\ in{\isacharunderscore}{\kern0pt}dom{\isacharunderscore}{\kern0pt}in{\isacharunderscore}{\kern0pt}eclose\ \isacommand{by}\isamarkupfalse%
\ auto\isanewline
\ \ \ \ \isacommand{with}\isamarkupfalse%
\ e\isanewline
\ \ \ \ \isacommand{have}\isamarkupfalse%
\ {\isachardoublequoteopen}n{\isadigit{1}}\ {\isasymin}\ ecloseN{\isacharparenleft}{\kern0pt}{\isacharquery}{\kern0pt}y{\isacharparenright}{\kern0pt}{\isachardoublequoteclose}\ {\isachardoublequoteopen}n{\isadigit{2}}\ {\isasymin}\ ecloseN{\isacharparenleft}{\kern0pt}{\isacharquery}{\kern0pt}y{\isacharparenright}{\kern0pt}{\isachardoublequoteclose}\isanewline
\ \ \ \ \ \ \isacommand{using}\isamarkupfalse%
\ ecloseNI\ components{\isacharunderscore}{\kern0pt}in{\isacharunderscore}{\kern0pt}eclose\ \isacommand{by}\isamarkupfalse%
\ auto\isanewline
\ \ \ \ \isacommand{with}\isamarkupfalse%
\ {\isacartoucheopen}ft{\isasymin}{\isadigit{2}}{\isacartoucheclose}\ {\isacartoucheopen}p{\isasymin}P{\isacartoucheclose}\isanewline
\ \ \ \ \isacommand{show}\isamarkupfalse%
\ {\isacharquery}{\kern0pt}thesis\ \isacommand{unfolding}\isamarkupfalse%
\ names{\isacharunderscore}{\kern0pt}below{\isacharunderscore}{\kern0pt}def\ \isacommand{by}\isamarkupfalse%
\ \ auto\isanewline
\ \ \isacommand{next}\isamarkupfalse%
\isanewline
\ \ \ \ \isacommand{case}\isamarkupfalse%
\ m\isanewline
\ \ \ \ \isacommand{then}\isamarkupfalse%
\isanewline
\ \ \ \ \isacommand{have}\isamarkupfalse%
\ {\isachardoublequoteopen}n{\isadigit{1}}\ {\isasymin}\ ecloseN{\isacharparenleft}{\kern0pt}{\isacharquery}{\kern0pt}y{\isacharparenright}{\kern0pt}{\isachardoublequoteclose}\ {\isachardoublequoteopen}n{\isadigit{2}}\ {\isasymin}\ ecloseN{\isacharparenleft}{\kern0pt}{\isacharquery}{\kern0pt}y{\isacharparenright}{\kern0pt}{\isachardoublequoteclose}\isanewline
\ \ \ \ \ \ \isacommand{using}\isamarkupfalse%
\ mem{\isacharunderscore}{\kern0pt}eclose{\isacharunderscore}{\kern0pt}trans\ \ ecloseNI\isanewline
\ \ \ \ \ \ \ \ in{\isacharunderscore}{\kern0pt}dom{\isacharunderscore}{\kern0pt}in{\isacharunderscore}{\kern0pt}eclose\ components{\isacharunderscore}{\kern0pt}in{\isacharunderscore}{\kern0pt}eclose\ \isacommand{by}\isamarkupfalse%
\ auto\isanewline
\ \ \ \ \isacommand{with}\isamarkupfalse%
\ {\isacartoucheopen}ft{\isasymin}{\isadigit{2}}{\isacartoucheclose}\ {\isacartoucheopen}p{\isasymin}P{\isacartoucheclose}\isanewline
\ \ \ \ \isacommand{show}\isamarkupfalse%
\ {\isacharquery}{\kern0pt}thesis\ \isacommand{unfolding}\isamarkupfalse%
\ names{\isacharunderscore}{\kern0pt}below{\isacharunderscore}{\kern0pt}def\isanewline
\ \ \ \ \ \ \isacommand{by}\isamarkupfalse%
\ auto\isanewline
\ \ \isacommand{qed}\isamarkupfalse%
\isanewline
\isacommand{qed}\isamarkupfalse%
%
\endisatagproof
{\isafoldproof}%
%
\isadelimproof
\isanewline
%
\endisadelimproof
\isanewline
\isacommand{lemma}\isamarkupfalse%
\ names{\isacharunderscore}{\kern0pt}below{\isacharunderscore}{\kern0pt}tr\ {\isacharcolon}{\kern0pt}\isanewline
\ \ \isakeyword{assumes}\ {\isachardoublequoteopen}x{\isasymin}\ names{\isacharunderscore}{\kern0pt}below{\isacharparenleft}{\kern0pt}P{\isacharcomma}{\kern0pt}y{\isacharparenright}{\kern0pt}{\isachardoublequoteclose}\isanewline
\ \ \ \ {\isachardoublequoteopen}y{\isasymin}\ names{\isacharunderscore}{\kern0pt}below{\isacharparenleft}{\kern0pt}P{\isacharcomma}{\kern0pt}z{\isacharparenright}{\kern0pt}{\isachardoublequoteclose}\isanewline
\ \ \isakeyword{shows}\ {\isachardoublequoteopen}x{\isasymin}\ names{\isacharunderscore}{\kern0pt}below{\isacharparenleft}{\kern0pt}P{\isacharcomma}{\kern0pt}z{\isacharparenright}{\kern0pt}{\isachardoublequoteclose}\isanewline
%
\isadelimproof
%
\endisadelimproof
%
\isatagproof
\isacommand{proof}\isamarkupfalse%
\ {\isacharminus}{\kern0pt}\isanewline
\ \ \isacommand{let}\isamarkupfalse%
\ {\isacharquery}{\kern0pt}A{\isacharequal}{\kern0pt}{\isachardoublequoteopen}{\isasymlambda}y\ {\isachardot}{\kern0pt}\ names{\isacharunderscore}{\kern0pt}below{\isacharparenleft}{\kern0pt}P{\isacharcomma}{\kern0pt}y{\isacharparenright}{\kern0pt}{\isachardoublequoteclose}\isanewline
\ \ \isacommand{from}\isamarkupfalse%
\ assms\isanewline
\ \ \isacommand{obtain}\isamarkupfalse%
\ fx\ x{\isadigit{1}}\ x{\isadigit{2}}\ px\ \isakeyword{where}\isanewline
\ \ \ \ {\isachardoublequoteopen}x\ {\isacharequal}{\kern0pt}\ {\isasymlangle}fx{\isacharcomma}{\kern0pt}x{\isadigit{1}}{\isacharcomma}{\kern0pt}x{\isadigit{2}}{\isacharcomma}{\kern0pt}px{\isasymrangle}{\isachardoublequoteclose}\ {\isachardoublequoteopen}fx{\isasymin}{\isadigit{2}}{\isachardoublequoteclose}\ {\isachardoublequoteopen}x{\isadigit{1}}{\isasymin}ecloseN{\isacharparenleft}{\kern0pt}y{\isacharparenright}{\kern0pt}{\isachardoublequoteclose}\ {\isachardoublequoteopen}x{\isadigit{2}}{\isasymin}ecloseN{\isacharparenleft}{\kern0pt}y{\isacharparenright}{\kern0pt}{\isachardoublequoteclose}\ {\isachardoublequoteopen}px{\isasymin}P{\isachardoublequoteclose}\isanewline
\ \ \ \ \isacommand{unfolding}\isamarkupfalse%
\ names{\isacharunderscore}{\kern0pt}below{\isacharunderscore}{\kern0pt}def\ \isacommand{by}\isamarkupfalse%
\ auto\isanewline
\ \ \isacommand{from}\isamarkupfalse%
\ assms\isanewline
\ \ \isacommand{obtain}\isamarkupfalse%
\ fy\ y{\isadigit{1}}\ y{\isadigit{2}}\ py\ \isakeyword{where}\isanewline
\ \ \ \ {\isachardoublequoteopen}y\ {\isacharequal}{\kern0pt}\ {\isasymlangle}fy{\isacharcomma}{\kern0pt}y{\isadigit{1}}{\isacharcomma}{\kern0pt}y{\isadigit{2}}{\isacharcomma}{\kern0pt}py{\isasymrangle}{\isachardoublequoteclose}\ {\isachardoublequoteopen}fy{\isasymin}{\isadigit{2}}{\isachardoublequoteclose}\ {\isachardoublequoteopen}y{\isadigit{1}}{\isasymin}ecloseN{\isacharparenleft}{\kern0pt}z{\isacharparenright}{\kern0pt}{\isachardoublequoteclose}\ {\isachardoublequoteopen}y{\isadigit{2}}{\isasymin}ecloseN{\isacharparenleft}{\kern0pt}z{\isacharparenright}{\kern0pt}{\isachardoublequoteclose}\ {\isachardoublequoteopen}py{\isasymin}P{\isachardoublequoteclose}\isanewline
\ \ \ \ \isacommand{unfolding}\isamarkupfalse%
\ names{\isacharunderscore}{\kern0pt}below{\isacharunderscore}{\kern0pt}def\ \isacommand{by}\isamarkupfalse%
\ auto\isanewline
\ \ \isacommand{from}\isamarkupfalse%
\ {\isacartoucheopen}x{\isadigit{1}}{\isasymin}{\isacharunderscore}{\kern0pt}{\isacartoucheclose}\ {\isacartoucheopen}x{\isadigit{2}}{\isasymin}{\isacharunderscore}{\kern0pt}{\isacartoucheclose}\ {\isacartoucheopen}y{\isadigit{1}}{\isasymin}{\isacharunderscore}{\kern0pt}{\isacartoucheclose}\ {\isacartoucheopen}y{\isadigit{2}}{\isasymin}{\isacharunderscore}{\kern0pt}{\isacartoucheclose}\ {\isacartoucheopen}x{\isacharequal}{\kern0pt}{\isacharunderscore}{\kern0pt}{\isacartoucheclose}\ {\isacartoucheopen}y{\isacharequal}{\kern0pt}{\isacharunderscore}{\kern0pt}{\isacartoucheclose}\isanewline
\ \ \isacommand{have}\isamarkupfalse%
\ {\isachardoublequoteopen}x{\isadigit{1}}{\isasymin}ecloseN{\isacharparenleft}{\kern0pt}z{\isacharparenright}{\kern0pt}{\isachardoublequoteclose}\ {\isachardoublequoteopen}x{\isadigit{2}}{\isasymin}ecloseN{\isacharparenleft}{\kern0pt}z{\isacharparenright}{\kern0pt}{\isachardoublequoteclose}\isanewline
\ \ \ \ \isacommand{using}\isamarkupfalse%
\ ecloseN{\isacharunderscore}{\kern0pt}mono\ names{\isacharunderscore}{\kern0pt}simp\ \isacommand{by}\isamarkupfalse%
\ auto\isanewline
\ \ \isacommand{with}\isamarkupfalse%
\ {\isacartoucheopen}fx{\isasymin}{\isadigit{2}}{\isacartoucheclose}\ {\isacartoucheopen}px{\isasymin}P{\isacartoucheclose}\ {\isacartoucheopen}x{\isacharequal}{\kern0pt}{\isacharunderscore}{\kern0pt}{\isacartoucheclose}\isanewline
\ \ \isacommand{have}\isamarkupfalse%
\ {\isachardoublequoteopen}x{\isasymin}{\isacharquery}{\kern0pt}A{\isacharparenleft}{\kern0pt}z{\isacharparenright}{\kern0pt}{\isachardoublequoteclose}\isanewline
\ \ \ \ \isacommand{unfolding}\isamarkupfalse%
\ names{\isacharunderscore}{\kern0pt}below{\isacharunderscore}{\kern0pt}def\ \isacommand{by}\isamarkupfalse%
\ simp\isanewline
\ \ \isacommand{then}\isamarkupfalse%
\ \isacommand{show}\isamarkupfalse%
\ {\isacharquery}{\kern0pt}thesis\ \isacommand{using}\isamarkupfalse%
\ subsetI\ \isacommand{by}\isamarkupfalse%
\ simp\isanewline
\isacommand{qed}\isamarkupfalse%
%
\endisatagproof
{\isafoldproof}%
%
\isadelimproof
\isanewline
%
\endisadelimproof
\isanewline
\isacommand{lemma}\isamarkupfalse%
\ arg{\isacharunderscore}{\kern0pt}into{\isacharunderscore}{\kern0pt}names{\isacharunderscore}{\kern0pt}below{\isadigit{2}}\ {\isacharcolon}{\kern0pt}\isanewline
\ \ \isakeyword{assumes}\ {\isachardoublequoteopen}{\isasymlangle}x{\isacharcomma}{\kern0pt}y{\isasymrangle}\ {\isasymin}\ frecrel{\isacharparenleft}{\kern0pt}names{\isacharunderscore}{\kern0pt}below{\isacharparenleft}{\kern0pt}P{\isacharcomma}{\kern0pt}z{\isacharparenright}{\kern0pt}{\isacharparenright}{\kern0pt}{\isachardoublequoteclose}\isanewline
\ \ \isakeyword{shows}\ \ {\isachardoublequoteopen}x\ {\isasymin}\ names{\isacharunderscore}{\kern0pt}below{\isacharparenleft}{\kern0pt}P{\isacharcomma}{\kern0pt}y{\isacharparenright}{\kern0pt}{\isachardoublequoteclose}\isanewline
%
\isadelimproof
%
\endisadelimproof
%
\isatagproof
\isacommand{proof}\isamarkupfalse%
\ {\isacharminus}{\kern0pt}\isanewline
\ \ \isacommand{{\isacharbraceleft}{\kern0pt}}\isamarkupfalse%
\isanewline
\ \ \ \ \isacommand{from}\isamarkupfalse%
\ assms\isanewline
\ \ \ \ \isacommand{have}\isamarkupfalse%
\ {\isachardoublequoteopen}x{\isasymin}names{\isacharunderscore}{\kern0pt}below{\isacharparenleft}{\kern0pt}P{\isacharcomma}{\kern0pt}z{\isacharparenright}{\kern0pt}{\isachardoublequoteclose}\ {\isachardoublequoteopen}y{\isasymin}names{\isacharunderscore}{\kern0pt}below{\isacharparenleft}{\kern0pt}P{\isacharcomma}{\kern0pt}z{\isacharparenright}{\kern0pt}{\isachardoublequoteclose}\ {\isachardoublequoteopen}frecR{\isacharparenleft}{\kern0pt}x{\isacharcomma}{\kern0pt}y{\isacharparenright}{\kern0pt}{\isachardoublequoteclose}\isanewline
\ \ \ \ \ \ \isacommand{unfolding}\isamarkupfalse%
\ frecrel{\isacharunderscore}{\kern0pt}def\ Rrel{\isacharunderscore}{\kern0pt}def\isanewline
\ \ \ \ \ \ \isacommand{by}\isamarkupfalse%
\ auto\isanewline
\ \ \ \ \isacommand{obtain}\isamarkupfalse%
\ f\ n{\isadigit{1}}\ n{\isadigit{2}}\ p\ \isakeyword{where}\isanewline
\ \ \ \ \ \ {\isachardoublequoteopen}x\ {\isacharequal}{\kern0pt}\ {\isasymlangle}f{\isacharcomma}{\kern0pt}n{\isadigit{1}}{\isacharcomma}{\kern0pt}n{\isadigit{2}}{\isacharcomma}{\kern0pt}p{\isasymrangle}{\isachardoublequoteclose}\ {\isachardoublequoteopen}f{\isasymin}{\isadigit{2}}{\isachardoublequoteclose}\ {\isachardoublequoteopen}n{\isadigit{1}}{\isasymin}ecloseN{\isacharparenleft}{\kern0pt}z{\isacharparenright}{\kern0pt}{\isachardoublequoteclose}\ {\isachardoublequoteopen}n{\isadigit{2}}{\isasymin}ecloseN{\isacharparenleft}{\kern0pt}z{\isacharparenright}{\kern0pt}{\isachardoublequoteclose}\ {\isachardoublequoteopen}p{\isasymin}P{\isachardoublequoteclose}\isanewline
\ \ \ \ \ \ \isacommand{using}\isamarkupfalse%
\ {\isacartoucheopen}x{\isasymin}names{\isacharunderscore}{\kern0pt}below{\isacharparenleft}{\kern0pt}P{\isacharcomma}{\kern0pt}z{\isacharparenright}{\kern0pt}{\isacartoucheclose}\isanewline
\ \ \ \ \ \ \isacommand{unfolding}\isamarkupfalse%
\ names{\isacharunderscore}{\kern0pt}below{\isacharunderscore}{\kern0pt}def\ \isacommand{by}\isamarkupfalse%
\ auto\isanewline
\ \ \ \ \isacommand{moreover}\isamarkupfalse%
\isanewline
\ \ \ \ \isacommand{obtain}\isamarkupfalse%
\ fy\ m{\isadigit{1}}\ m{\isadigit{2}}\ q\ \isakeyword{where}\isanewline
\ \ \ \ \ \ {\isachardoublequoteopen}q{\isasymin}P{\isachardoublequoteclose}\ {\isachardoublequoteopen}y\ {\isacharequal}{\kern0pt}\ {\isasymlangle}fy{\isacharcomma}{\kern0pt}m{\isadigit{1}}{\isacharcomma}{\kern0pt}m{\isadigit{2}}{\isacharcomma}{\kern0pt}q{\isasymrangle}{\isachardoublequoteclose}\isanewline
\ \ \ \ \ \ \isacommand{using}\isamarkupfalse%
\ {\isacartoucheopen}y{\isasymin}names{\isacharunderscore}{\kern0pt}below{\isacharparenleft}{\kern0pt}P{\isacharcomma}{\kern0pt}z{\isacharparenright}{\kern0pt}{\isacartoucheclose}\isanewline
\ \ \ \ \ \ \isacommand{unfolding}\isamarkupfalse%
\ names{\isacharunderscore}{\kern0pt}below{\isacharunderscore}{\kern0pt}def\ \isacommand{by}\isamarkupfalse%
\ auto\isanewline
\ \ \ \ \isacommand{moreover}\isamarkupfalse%
\isanewline
\ \ \ \ \isacommand{note}\isamarkupfalse%
\ {\isacartoucheopen}frecR{\isacharparenleft}{\kern0pt}x{\isacharcomma}{\kern0pt}y{\isacharparenright}{\kern0pt}{\isacartoucheclose}\isanewline
\ \ \ \ \isacommand{ultimately}\isamarkupfalse%
\isanewline
\ \ \ \ \isacommand{have}\isamarkupfalse%
\ {\isachardoublequoteopen}x{\isasymin}names{\isacharunderscore}{\kern0pt}below{\isacharparenleft}{\kern0pt}P{\isacharcomma}{\kern0pt}y{\isacharparenright}{\kern0pt}{\isachardoublequoteclose}\ \isacommand{using}\isamarkupfalse%
\ names{\isacharunderscore}{\kern0pt}belowI\ \isacommand{by}\isamarkupfalse%
\ simp\isanewline
\ \ \isacommand{{\isacharbraceright}{\kern0pt}}\isamarkupfalse%
\isanewline
\ \ \isacommand{then}\isamarkupfalse%
\ \isacommand{show}\isamarkupfalse%
\ {\isacharquery}{\kern0pt}thesis\ \isacommand{{\isachardot}{\kern0pt}}\isamarkupfalse%
\isanewline
\isacommand{qed}\isamarkupfalse%
%
\endisatagproof
{\isafoldproof}%
%
\isadelimproof
\isanewline
%
\endisadelimproof
\isanewline
\isacommand{lemma}\isamarkupfalse%
\ arg{\isacharunderscore}{\kern0pt}into{\isacharunderscore}{\kern0pt}names{\isacharunderscore}{\kern0pt}below\ {\isacharcolon}{\kern0pt}\isanewline
\ \ \isakeyword{assumes}\ {\isachardoublequoteopen}{\isasymlangle}x{\isacharcomma}{\kern0pt}y{\isasymrangle}\ {\isasymin}\ frecrel{\isacharparenleft}{\kern0pt}names{\isacharunderscore}{\kern0pt}below{\isacharparenleft}{\kern0pt}P{\isacharcomma}{\kern0pt}z{\isacharparenright}{\kern0pt}{\isacharparenright}{\kern0pt}{\isachardoublequoteclose}\isanewline
\ \ \isakeyword{shows}\ \ {\isachardoublequoteopen}x\ {\isasymin}\ names{\isacharunderscore}{\kern0pt}below{\isacharparenleft}{\kern0pt}P{\isacharcomma}{\kern0pt}x{\isacharparenright}{\kern0pt}{\isachardoublequoteclose}\isanewline
%
\isadelimproof
%
\endisadelimproof
%
\isatagproof
\isacommand{proof}\isamarkupfalse%
\ {\isacharminus}{\kern0pt}\isanewline
\ \ \isacommand{{\isacharbraceleft}{\kern0pt}}\isamarkupfalse%
\isanewline
\ \ \ \ \isacommand{from}\isamarkupfalse%
\ assms\isanewline
\ \ \ \ \isacommand{have}\isamarkupfalse%
\ {\isachardoublequoteopen}x{\isasymin}names{\isacharunderscore}{\kern0pt}below{\isacharparenleft}{\kern0pt}P{\isacharcomma}{\kern0pt}z{\isacharparenright}{\kern0pt}{\isachardoublequoteclose}\isanewline
\ \ \ \ \ \ \isacommand{unfolding}\isamarkupfalse%
\ frecrel{\isacharunderscore}{\kern0pt}def\ Rrel{\isacharunderscore}{\kern0pt}def\isanewline
\ \ \ \ \ \ \isacommand{by}\isamarkupfalse%
\ auto\isanewline
\ \ \ \ \isacommand{from}\isamarkupfalse%
\ {\isacartoucheopen}x{\isasymin}names{\isacharunderscore}{\kern0pt}below{\isacharparenleft}{\kern0pt}P{\isacharcomma}{\kern0pt}z{\isacharparenright}{\kern0pt}{\isacartoucheclose}\isanewline
\ \ \ \ \isacommand{obtain}\isamarkupfalse%
\ f\ n{\isadigit{1}}\ n{\isadigit{2}}\ p\ \isakeyword{where}\isanewline
\ \ \ \ \ \ {\isachardoublequoteopen}x\ {\isacharequal}{\kern0pt}\ {\isasymlangle}f{\isacharcomma}{\kern0pt}n{\isadigit{1}}{\isacharcomma}{\kern0pt}n{\isadigit{2}}{\isacharcomma}{\kern0pt}p{\isasymrangle}{\isachardoublequoteclose}\ {\isachardoublequoteopen}f{\isasymin}{\isadigit{2}}{\isachardoublequoteclose}\ {\isachardoublequoteopen}n{\isadigit{1}}{\isasymin}ecloseN{\isacharparenleft}{\kern0pt}z{\isacharparenright}{\kern0pt}{\isachardoublequoteclose}\ {\isachardoublequoteopen}n{\isadigit{2}}{\isasymin}ecloseN{\isacharparenleft}{\kern0pt}z{\isacharparenright}{\kern0pt}{\isachardoublequoteclose}\ {\isachardoublequoteopen}p{\isasymin}P{\isachardoublequoteclose}\isanewline
\ \ \ \ \ \ \isacommand{unfolding}\isamarkupfalse%
\ names{\isacharunderscore}{\kern0pt}below{\isacharunderscore}{\kern0pt}def\ \isacommand{by}\isamarkupfalse%
\ auto\isanewline
\ \ \ \ \isacommand{then}\isamarkupfalse%
\isanewline
\ \ \ \ \isacommand{have}\isamarkupfalse%
\ {\isachardoublequoteopen}n{\isadigit{1}}{\isasymin}ecloseN{\isacharparenleft}{\kern0pt}x{\isacharparenright}{\kern0pt}{\isachardoublequoteclose}\ {\isachardoublequoteopen}n{\isadigit{2}}{\isasymin}ecloseN{\isacharparenleft}{\kern0pt}x{\isacharparenright}{\kern0pt}{\isachardoublequoteclose}\isanewline
\ \ \ \ \ \ \isacommand{using}\isamarkupfalse%
\ components{\isacharunderscore}{\kern0pt}in{\isacharunderscore}{\kern0pt}eclose\ \isacommand{by}\isamarkupfalse%
\ simp{\isacharunderscore}{\kern0pt}all\isanewline
\ \ \ \ \isacommand{with}\isamarkupfalse%
\ {\isacartoucheopen}f{\isasymin}{\isadigit{2}}{\isacartoucheclose}\ {\isacartoucheopen}p{\isasymin}P{\isacartoucheclose}\ {\isacartoucheopen}x\ {\isacharequal}{\kern0pt}\ {\isasymlangle}f{\isacharcomma}{\kern0pt}n{\isadigit{1}}{\isacharcomma}{\kern0pt}n{\isadigit{2}}{\isacharcomma}{\kern0pt}p{\isasymrangle}{\isacartoucheclose}\isanewline
\ \ \ \ \isacommand{have}\isamarkupfalse%
\ {\isachardoublequoteopen}x{\isasymin}names{\isacharunderscore}{\kern0pt}below{\isacharparenleft}{\kern0pt}P{\isacharcomma}{\kern0pt}x{\isacharparenright}{\kern0pt}{\isachardoublequoteclose}\isanewline
\ \ \ \ \ \ \isacommand{unfolding}\isamarkupfalse%
\ names{\isacharunderscore}{\kern0pt}below{\isacharunderscore}{\kern0pt}def\ \isacommand{by}\isamarkupfalse%
\ simp\isanewline
\ \ \isacommand{{\isacharbraceright}{\kern0pt}}\isamarkupfalse%
\isanewline
\ \ \isacommand{then}\isamarkupfalse%
\ \isacommand{show}\isamarkupfalse%
\ {\isacharquery}{\kern0pt}thesis\ \isacommand{{\isachardot}{\kern0pt}}\isamarkupfalse%
\isanewline
\isacommand{qed}\isamarkupfalse%
%
\endisatagproof
{\isafoldproof}%
%
\isadelimproof
\isanewline
%
\endisadelimproof
\isanewline
\isacommand{lemma}\isamarkupfalse%
\ forcerel{\isacharunderscore}{\kern0pt}arg{\isacharunderscore}{\kern0pt}into{\isacharunderscore}{\kern0pt}names{\isacharunderscore}{\kern0pt}below\ {\isacharcolon}{\kern0pt}\isanewline
\ \ \isakeyword{assumes}\ {\isachardoublequoteopen}{\isasymlangle}x{\isacharcomma}{\kern0pt}y{\isasymrangle}\ {\isasymin}\ forcerel{\isacharparenleft}{\kern0pt}P{\isacharcomma}{\kern0pt}z{\isacharparenright}{\kern0pt}{\isachardoublequoteclose}\isanewline
\ \ \isakeyword{shows}\ \ {\isachardoublequoteopen}x\ {\isasymin}\ names{\isacharunderscore}{\kern0pt}below{\isacharparenleft}{\kern0pt}P{\isacharcomma}{\kern0pt}x{\isacharparenright}{\kern0pt}{\isachardoublequoteclose}\isanewline
%
\isadelimproof
\ \ %
\endisadelimproof
%
\isatagproof
\isacommand{using}\isamarkupfalse%
\ assms\isanewline
\ \ \isacommand{unfolding}\isamarkupfalse%
\ forcerel{\isacharunderscore}{\kern0pt}def\isanewline
\ \ \isacommand{by}\isamarkupfalse%
{\isacharparenleft}{\kern0pt}rule\ trancl{\isacharunderscore}{\kern0pt}induct{\isacharsemicolon}{\kern0pt}auto\ simp\ add{\isacharcolon}{\kern0pt}\ arg{\isacharunderscore}{\kern0pt}into{\isacharunderscore}{\kern0pt}names{\isacharunderscore}{\kern0pt}below{\isacharparenright}{\kern0pt}%
\endisatagproof
{\isafoldproof}%
%
\isadelimproof
\isanewline
%
\endisadelimproof
\isanewline
\isacommand{lemma}\isamarkupfalse%
\ names{\isacharunderscore}{\kern0pt}below{\isacharunderscore}{\kern0pt}mono\ {\isacharcolon}{\kern0pt}\isanewline
\ \ \isakeyword{assumes}\ {\isachardoublequoteopen}{\isasymlangle}x{\isacharcomma}{\kern0pt}y{\isasymrangle}\ {\isasymin}\ frecrel{\isacharparenleft}{\kern0pt}names{\isacharunderscore}{\kern0pt}below{\isacharparenleft}{\kern0pt}P{\isacharcomma}{\kern0pt}z{\isacharparenright}{\kern0pt}{\isacharparenright}{\kern0pt}{\isachardoublequoteclose}\isanewline
\ \ \isakeyword{shows}\ {\isachardoublequoteopen}names{\isacharunderscore}{\kern0pt}below{\isacharparenleft}{\kern0pt}P{\isacharcomma}{\kern0pt}x{\isacharparenright}{\kern0pt}\ {\isasymsubseteq}\ names{\isacharunderscore}{\kern0pt}below{\isacharparenleft}{\kern0pt}P{\isacharcomma}{\kern0pt}y{\isacharparenright}{\kern0pt}{\isachardoublequoteclose}\isanewline
%
\isadelimproof
%
\endisadelimproof
%
\isatagproof
\isacommand{proof}\isamarkupfalse%
\ {\isacharminus}{\kern0pt}\isanewline
\ \ \isacommand{from}\isamarkupfalse%
\ assms\isanewline
\ \ \isacommand{have}\isamarkupfalse%
\ {\isachardoublequoteopen}x{\isasymin}names{\isacharunderscore}{\kern0pt}below{\isacharparenleft}{\kern0pt}P{\isacharcomma}{\kern0pt}y{\isacharparenright}{\kern0pt}{\isachardoublequoteclose}\isanewline
\ \ \ \ \isacommand{using}\isamarkupfalse%
\ arg{\isacharunderscore}{\kern0pt}into{\isacharunderscore}{\kern0pt}names{\isacharunderscore}{\kern0pt}below{\isadigit{2}}\ \isacommand{by}\isamarkupfalse%
\ simp\isanewline
\ \ \isacommand{then}\isamarkupfalse%
\isanewline
\ \ \isacommand{show}\isamarkupfalse%
\ {\isacharquery}{\kern0pt}thesis\isanewline
\ \ \ \ \isacommand{using}\isamarkupfalse%
\ names{\isacharunderscore}{\kern0pt}below{\isacharunderscore}{\kern0pt}tr\ subsetI\ \isacommand{by}\isamarkupfalse%
\ simp\isanewline
\isacommand{qed}\isamarkupfalse%
%
\endisatagproof
{\isafoldproof}%
%
\isadelimproof
\isanewline
%
\endisadelimproof
\isanewline
\isacommand{lemma}\isamarkupfalse%
\ frecrel{\isacharunderscore}{\kern0pt}mono\ {\isacharcolon}{\kern0pt}\isanewline
\ \ \isakeyword{assumes}\ {\isachardoublequoteopen}{\isasymlangle}x{\isacharcomma}{\kern0pt}y{\isasymrangle}\ {\isasymin}\ frecrel{\isacharparenleft}{\kern0pt}names{\isacharunderscore}{\kern0pt}below{\isacharparenleft}{\kern0pt}P{\isacharcomma}{\kern0pt}z{\isacharparenright}{\kern0pt}{\isacharparenright}{\kern0pt}{\isachardoublequoteclose}\isanewline
\ \ \isakeyword{shows}\ {\isachardoublequoteopen}frecrel{\isacharparenleft}{\kern0pt}names{\isacharunderscore}{\kern0pt}below{\isacharparenleft}{\kern0pt}P{\isacharcomma}{\kern0pt}x{\isacharparenright}{\kern0pt}{\isacharparenright}{\kern0pt}\ {\isasymsubseteq}\ frecrel{\isacharparenleft}{\kern0pt}names{\isacharunderscore}{\kern0pt}below{\isacharparenleft}{\kern0pt}P{\isacharcomma}{\kern0pt}y{\isacharparenright}{\kern0pt}{\isacharparenright}{\kern0pt}{\isachardoublequoteclose}\isanewline
%
\isadelimproof
\ \ %
\endisadelimproof
%
\isatagproof
\isacommand{unfolding}\isamarkupfalse%
\ frecrel{\isacharunderscore}{\kern0pt}def\isanewline
\ \ \isacommand{using}\isamarkupfalse%
\ Rrel{\isacharunderscore}{\kern0pt}mono\ names{\isacharunderscore}{\kern0pt}below{\isacharunderscore}{\kern0pt}mono\ assms\ \isacommand{by}\isamarkupfalse%
\ simp%
\endisatagproof
{\isafoldproof}%
%
\isadelimproof
\isanewline
%
\endisadelimproof
\isanewline
\isacommand{lemma}\isamarkupfalse%
\ forcerel{\isacharunderscore}{\kern0pt}mono{\isadigit{2}}\ {\isacharcolon}{\kern0pt}\isanewline
\ \ \isakeyword{assumes}\ {\isachardoublequoteopen}{\isasymlangle}x{\isacharcomma}{\kern0pt}y{\isasymrangle}\ {\isasymin}\ frecrel{\isacharparenleft}{\kern0pt}names{\isacharunderscore}{\kern0pt}below{\isacharparenleft}{\kern0pt}P{\isacharcomma}{\kern0pt}z{\isacharparenright}{\kern0pt}{\isacharparenright}{\kern0pt}{\isachardoublequoteclose}\isanewline
\ \ \isakeyword{shows}\ {\isachardoublequoteopen}forcerel{\isacharparenleft}{\kern0pt}P{\isacharcomma}{\kern0pt}x{\isacharparenright}{\kern0pt}\ {\isasymsubseteq}\ forcerel{\isacharparenleft}{\kern0pt}P{\isacharcomma}{\kern0pt}y{\isacharparenright}{\kern0pt}{\isachardoublequoteclose}\isanewline
%
\isadelimproof
\ \ %
\endisadelimproof
%
\isatagproof
\isacommand{unfolding}\isamarkupfalse%
\ forcerel{\isacharunderscore}{\kern0pt}def\isanewline
\ \ \isacommand{using}\isamarkupfalse%
\ trancl{\isacharunderscore}{\kern0pt}mono\ frecrel{\isacharunderscore}{\kern0pt}mono\ assms\ \isacommand{by}\isamarkupfalse%
\ simp%
\endisatagproof
{\isafoldproof}%
%
\isadelimproof
\isanewline
%
\endisadelimproof
\isanewline
\isacommand{lemma}\isamarkupfalse%
\ forcerel{\isacharunderscore}{\kern0pt}mono{\isacharunderscore}{\kern0pt}aux\ {\isacharcolon}{\kern0pt}\isanewline
\ \ \isakeyword{assumes}\ {\isachardoublequoteopen}{\isasymlangle}x{\isacharcomma}{\kern0pt}y{\isasymrangle}\ {\isasymin}\ frecrel{\isacharparenleft}{\kern0pt}names{\isacharunderscore}{\kern0pt}below{\isacharparenleft}{\kern0pt}P{\isacharcomma}{\kern0pt}\ w{\isacharparenright}{\kern0pt}{\isacharparenright}{\kern0pt}{\isacharcircum}{\kern0pt}{\isacharplus}{\kern0pt}{\isachardoublequoteclose}\isanewline
\ \ \isakeyword{shows}\ {\isachardoublequoteopen}forcerel{\isacharparenleft}{\kern0pt}P{\isacharcomma}{\kern0pt}x{\isacharparenright}{\kern0pt}\ {\isasymsubseteq}\ forcerel{\isacharparenleft}{\kern0pt}P{\isacharcomma}{\kern0pt}y{\isacharparenright}{\kern0pt}{\isachardoublequoteclose}\isanewline
%
\isadelimproof
\ \ %
\endisadelimproof
%
\isatagproof
\isacommand{using}\isamarkupfalse%
\ assms\isanewline
\ \ \isacommand{by}\isamarkupfalse%
\ {\isacharparenleft}{\kern0pt}rule\ trancl{\isacharunderscore}{\kern0pt}induct{\isacharcomma}{\kern0pt}simp{\isacharunderscore}{\kern0pt}all\ add{\isacharcolon}{\kern0pt}\ subset{\isacharunderscore}{\kern0pt}trans\ forcerel{\isacharunderscore}{\kern0pt}mono{\isadigit{2}}{\isacharparenright}{\kern0pt}%
\endisatagproof
{\isafoldproof}%
%
\isadelimproof
\isanewline
%
\endisadelimproof
\isanewline
\isacommand{lemma}\isamarkupfalse%
\ forcerel{\isacharunderscore}{\kern0pt}mono\ {\isacharcolon}{\kern0pt}\isanewline
\ \ \isakeyword{assumes}\ {\isachardoublequoteopen}{\isasymlangle}x{\isacharcomma}{\kern0pt}y{\isasymrangle}\ {\isasymin}\ forcerel{\isacharparenleft}{\kern0pt}P{\isacharcomma}{\kern0pt}z{\isacharparenright}{\kern0pt}{\isachardoublequoteclose}\isanewline
\ \ \isakeyword{shows}\ {\isachardoublequoteopen}forcerel{\isacharparenleft}{\kern0pt}P{\isacharcomma}{\kern0pt}x{\isacharparenright}{\kern0pt}\ {\isasymsubseteq}\ forcerel{\isacharparenleft}{\kern0pt}P{\isacharcomma}{\kern0pt}y{\isacharparenright}{\kern0pt}{\isachardoublequoteclose}\isanewline
%
\isadelimproof
\ \ %
\endisadelimproof
%
\isatagproof
\isacommand{using}\isamarkupfalse%
\ forcerel{\isacharunderscore}{\kern0pt}mono{\isacharunderscore}{\kern0pt}aux\ assms\ \isacommand{unfolding}\isamarkupfalse%
\ forcerel{\isacharunderscore}{\kern0pt}def\ \isacommand{by}\isamarkupfalse%
\ simp%
\endisatagproof
{\isafoldproof}%
%
\isadelimproof
\isanewline
%
\endisadelimproof
\isanewline
\isacommand{lemma}\isamarkupfalse%
\ aux{\isacharcolon}{\kern0pt}\ {\isachardoublequoteopen}x\ {\isasymin}\ names{\isacharunderscore}{\kern0pt}below{\isacharparenleft}{\kern0pt}P{\isacharcomma}{\kern0pt}\ w{\isacharparenright}{\kern0pt}\ {\isasymLongrightarrow}\ {\isasymlangle}x{\isacharcomma}{\kern0pt}y{\isasymrangle}\ {\isasymin}\ forcerel{\isacharparenleft}{\kern0pt}P{\isacharcomma}{\kern0pt}z{\isacharparenright}{\kern0pt}\ {\isasymLongrightarrow}\isanewline
\ \ {\isacharparenleft}{\kern0pt}y\ {\isasymin}\ names{\isacharunderscore}{\kern0pt}below{\isacharparenleft}{\kern0pt}P{\isacharcomma}{\kern0pt}\ w{\isacharparenright}{\kern0pt}\ {\isasymlongrightarrow}\ {\isasymlangle}x{\isacharcomma}{\kern0pt}y{\isasymrangle}\ {\isasymin}\ forcerel{\isacharparenleft}{\kern0pt}P{\isacharcomma}{\kern0pt}w{\isacharparenright}{\kern0pt}{\isacharparenright}{\kern0pt}{\isachardoublequoteclose}\isanewline
%
\isadelimproof
\ \ %
\endisadelimproof
%
\isatagproof
\isacommand{unfolding}\isamarkupfalse%
\ forcerel{\isacharunderscore}{\kern0pt}def\isanewline
\isacommand{proof}\isamarkupfalse%
{\isacharparenleft}{\kern0pt}rule{\isacharunderscore}{\kern0pt}tac\ a{\isacharequal}{\kern0pt}x\ \isakeyword{and}\ b{\isacharequal}{\kern0pt}y\ \isakeyword{and}\ P{\isacharequal}{\kern0pt}{\isachardoublequoteopen}{\isasymlambda}\ y\ {\isachardot}{\kern0pt}\ y\ {\isasymin}\ names{\isacharunderscore}{\kern0pt}below{\isacharparenleft}{\kern0pt}P{\isacharcomma}{\kern0pt}\ w{\isacharparenright}{\kern0pt}\ {\isasymlongrightarrow}\ {\isasymlangle}x{\isacharcomma}{\kern0pt}y{\isasymrangle}\ {\isasymin}\ frecrel{\isacharparenleft}{\kern0pt}names{\isacharunderscore}{\kern0pt}below{\isacharparenleft}{\kern0pt}P{\isacharcomma}{\kern0pt}w{\isacharparenright}{\kern0pt}{\isacharparenright}{\kern0pt}{\isacharcircum}{\kern0pt}{\isacharplus}{\kern0pt}{\isachardoublequoteclose}\ \isakeyword{in}\ trancl{\isacharunderscore}{\kern0pt}induct{\isacharcomma}{\kern0pt}simp{\isacharparenright}{\kern0pt}\isanewline
\ \ \isacommand{let}\isamarkupfalse%
\ {\isacharquery}{\kern0pt}A{\isacharequal}{\kern0pt}{\isachardoublequoteopen}{\isasymlambda}\ a\ {\isachardot}{\kern0pt}\ names{\isacharunderscore}{\kern0pt}below{\isacharparenleft}{\kern0pt}P{\isacharcomma}{\kern0pt}\ a{\isacharparenright}{\kern0pt}{\isachardoublequoteclose}\isanewline
\ \ \isacommand{let}\isamarkupfalse%
\ {\isacharquery}{\kern0pt}R{\isacharequal}{\kern0pt}{\isachardoublequoteopen}{\isasymlambda}\ a\ {\isachardot}{\kern0pt}\ frecrel{\isacharparenleft}{\kern0pt}{\isacharquery}{\kern0pt}A{\isacharparenleft}{\kern0pt}a{\isacharparenright}{\kern0pt}{\isacharparenright}{\kern0pt}{\isachardoublequoteclose}\isanewline
\ \ \isacommand{let}\isamarkupfalse%
\ {\isacharquery}{\kern0pt}fR{\isacharequal}{\kern0pt}{\isachardoublequoteopen}{\isasymlambda}\ a\ {\isachardot}{\kern0pt}forcerel{\isacharparenleft}{\kern0pt}a{\isacharparenright}{\kern0pt}{\isachardoublequoteclose}\isanewline
\ \ \isacommand{show}\isamarkupfalse%
\ {\isachardoublequoteopen}u{\isasymin}{\isacharquery}{\kern0pt}A{\isacharparenleft}{\kern0pt}w{\isacharparenright}{\kern0pt}\ {\isasymlongrightarrow}\ {\isasymlangle}x{\isacharcomma}{\kern0pt}u{\isasymrangle}{\isasymin}{\isacharquery}{\kern0pt}R{\isacharparenleft}{\kern0pt}w{\isacharparenright}{\kern0pt}{\isacharcircum}{\kern0pt}{\isacharplus}{\kern0pt}{\isachardoublequoteclose}\ \isakeyword{if}\ {\isachardoublequoteopen}x{\isasymin}{\isacharquery}{\kern0pt}A{\isacharparenleft}{\kern0pt}w{\isacharparenright}{\kern0pt}{\isachardoublequoteclose}\ {\isachardoublequoteopen}{\isasymlangle}x{\isacharcomma}{\kern0pt}y{\isasymrangle}{\isasymin}{\isacharquery}{\kern0pt}R{\isacharparenleft}{\kern0pt}z{\isacharparenright}{\kern0pt}{\isacharcircum}{\kern0pt}{\isacharplus}{\kern0pt}{\isachardoublequoteclose}\ {\isachardoublequoteopen}{\isasymlangle}x{\isacharcomma}{\kern0pt}u{\isasymrangle}{\isasymin}{\isacharquery}{\kern0pt}R{\isacharparenleft}{\kern0pt}z{\isacharparenright}{\kern0pt}{\isachardoublequoteclose}\ \ \isakeyword{for}\ \ u\isanewline
\ \ \ \ \isacommand{using}\isamarkupfalse%
\ that\ frecrelD\ frecrelI\ r{\isacharunderscore}{\kern0pt}into{\isacharunderscore}{\kern0pt}trancl\ \isacommand{unfolding}\isamarkupfalse%
\ names{\isacharunderscore}{\kern0pt}below{\isacharunderscore}{\kern0pt}def\ \isacommand{by}\isamarkupfalse%
\ simp\isanewline
\ \ \isacommand{{\isacharbraceleft}{\kern0pt}}\isamarkupfalse%
\isanewline
\ \ \ \ \isacommand{fix}\isamarkupfalse%
\ u\ v\isanewline
\ \ \ \ \isacommand{assume}\isamarkupfalse%
\ {\isachardoublequoteopen}x\ {\isasymin}\ {\isacharquery}{\kern0pt}A{\isacharparenleft}{\kern0pt}w{\isacharparenright}{\kern0pt}{\isachardoublequoteclose}\isanewline
\ \ \ \ \ \ {\isachardoublequoteopen}{\isasymlangle}x{\isacharcomma}{\kern0pt}\ y{\isasymrangle}\ {\isasymin}\ {\isacharquery}{\kern0pt}R{\isacharparenleft}{\kern0pt}z{\isacharparenright}{\kern0pt}{\isacharcircum}{\kern0pt}{\isacharplus}{\kern0pt}{\isachardoublequoteclose}\isanewline
\ \ \ \ \ \ {\isachardoublequoteopen}{\isasymlangle}x{\isacharcomma}{\kern0pt}\ u{\isasymrangle}\ {\isasymin}\ {\isacharquery}{\kern0pt}R{\isacharparenleft}{\kern0pt}z{\isacharparenright}{\kern0pt}{\isacharcircum}{\kern0pt}{\isacharplus}{\kern0pt}{\isachardoublequoteclose}\isanewline
\ \ \ \ \ \ {\isachardoublequoteopen}{\isasymlangle}u{\isacharcomma}{\kern0pt}\ v{\isasymrangle}\ {\isasymin}\ {\isacharquery}{\kern0pt}R{\isacharparenleft}{\kern0pt}z{\isacharparenright}{\kern0pt}{\isachardoublequoteclose}\isanewline
\ \ \ \ \ \ {\isachardoublequoteopen}u\ {\isasymin}\ {\isacharquery}{\kern0pt}A{\isacharparenleft}{\kern0pt}w{\isacharparenright}{\kern0pt}\ {\isasymLongrightarrow}\ {\isasymlangle}x{\isacharcomma}{\kern0pt}\ u{\isasymrangle}\ {\isasymin}\ {\isacharquery}{\kern0pt}R{\isacharparenleft}{\kern0pt}w{\isacharparenright}{\kern0pt}{\isacharcircum}{\kern0pt}{\isacharplus}{\kern0pt}{\isachardoublequoteclose}\isanewline
\ \ \ \ \isacommand{then}\isamarkupfalse%
\isanewline
\ \ \ \ \isacommand{have}\isamarkupfalse%
\ {\isachardoublequoteopen}v\ {\isasymin}\ {\isacharquery}{\kern0pt}A{\isacharparenleft}{\kern0pt}w{\isacharparenright}{\kern0pt}\ {\isasymLongrightarrow}\ {\isasymlangle}x{\isacharcomma}{\kern0pt}\ v{\isasymrangle}\ {\isasymin}\ {\isacharquery}{\kern0pt}R{\isacharparenleft}{\kern0pt}w{\isacharparenright}{\kern0pt}{\isacharcircum}{\kern0pt}{\isacharplus}{\kern0pt}{\isachardoublequoteclose}\isanewline
\ \ \ \ \isacommand{proof}\isamarkupfalse%
\ {\isacharminus}{\kern0pt}\isanewline
\ \ \ \ \ \ \isacommand{assume}\isamarkupfalse%
\ {\isachardoublequoteopen}v\ {\isasymin}{\isacharquery}{\kern0pt}A{\isacharparenleft}{\kern0pt}w{\isacharparenright}{\kern0pt}{\isachardoublequoteclose}\isanewline
\ \ \ \ \ \ \isacommand{from}\isamarkupfalse%
\ {\isacartoucheopen}{\isasymlangle}u{\isacharcomma}{\kern0pt}v{\isasymrangle}{\isasymin}{\isacharunderscore}{\kern0pt}{\isacartoucheclose}\isanewline
\ \ \ \ \ \ \isacommand{have}\isamarkupfalse%
\ {\isachardoublequoteopen}u{\isasymin}{\isacharquery}{\kern0pt}A{\isacharparenleft}{\kern0pt}v{\isacharparenright}{\kern0pt}{\isachardoublequoteclose}\isanewline
\ \ \ \ \ \ \ \ \isacommand{using}\isamarkupfalse%
\ arg{\isacharunderscore}{\kern0pt}into{\isacharunderscore}{\kern0pt}names{\isacharunderscore}{\kern0pt}below{\isadigit{2}}\ \isacommand{by}\isamarkupfalse%
\ simp\isanewline
\ \ \ \ \ \ \isacommand{with}\isamarkupfalse%
\ {\isacartoucheopen}v\ {\isasymin}{\isacharquery}{\kern0pt}A{\isacharparenleft}{\kern0pt}w{\isacharparenright}{\kern0pt}{\isacartoucheclose}\isanewline
\ \ \ \ \ \ \isacommand{have}\isamarkupfalse%
\ {\isachardoublequoteopen}u{\isasymin}{\isacharquery}{\kern0pt}A{\isacharparenleft}{\kern0pt}w{\isacharparenright}{\kern0pt}{\isachardoublequoteclose}\isanewline
\ \ \ \ \ \ \ \ \isacommand{using}\isamarkupfalse%
\ names{\isacharunderscore}{\kern0pt}below{\isacharunderscore}{\kern0pt}tr\ \isacommand{by}\isamarkupfalse%
\ simp\isanewline
\ \ \ \ \ \ \isacommand{with}\isamarkupfalse%
\ {\isacartoucheopen}v{\isasymin}{\isacharunderscore}{\kern0pt}{\isacartoucheclose}\ {\isacartoucheopen}{\isasymlangle}u{\isacharcomma}{\kern0pt}v{\isasymrangle}{\isasymin}{\isacharunderscore}{\kern0pt}{\isacartoucheclose}\isanewline
\ \ \ \ \ \ \isacommand{have}\isamarkupfalse%
\ {\isachardoublequoteopen}{\isasymlangle}u{\isacharcomma}{\kern0pt}v{\isasymrangle}{\isasymin}\ {\isacharquery}{\kern0pt}R{\isacharparenleft}{\kern0pt}w{\isacharparenright}{\kern0pt}{\isachardoublequoteclose}\isanewline
\ \ \ \ \ \ \ \ \isacommand{using}\isamarkupfalse%
\ frecrelD\ frecrelI\ r{\isacharunderscore}{\kern0pt}into{\isacharunderscore}{\kern0pt}trancl\ \isacommand{unfolding}\isamarkupfalse%
\ names{\isacharunderscore}{\kern0pt}below{\isacharunderscore}{\kern0pt}def\ \isacommand{by}\isamarkupfalse%
\ simp\isanewline
\ \ \ \ \ \ \isacommand{with}\isamarkupfalse%
\ {\isacartoucheopen}u\ {\isasymin}\ {\isacharquery}{\kern0pt}A{\isacharparenleft}{\kern0pt}w{\isacharparenright}{\kern0pt}\ {\isasymLongrightarrow}\ {\isasymlangle}x{\isacharcomma}{\kern0pt}\ u{\isasymrangle}\ {\isasymin}\ {\isacharquery}{\kern0pt}R{\isacharparenleft}{\kern0pt}w{\isacharparenright}{\kern0pt}{\isacharcircum}{\kern0pt}{\isacharplus}{\kern0pt}{\isacartoucheclose}\ {\isacartoucheopen}u{\isasymin}{\isacharquery}{\kern0pt}A{\isacharparenleft}{\kern0pt}w{\isacharparenright}{\kern0pt}{\isacartoucheclose}\isanewline
\ \ \ \ \ \ \isacommand{have}\isamarkupfalse%
\ {\isachardoublequoteopen}{\isasymlangle}x{\isacharcomma}{\kern0pt}\ u{\isasymrangle}\ {\isasymin}\ {\isacharquery}{\kern0pt}R{\isacharparenleft}{\kern0pt}w{\isacharparenright}{\kern0pt}{\isacharcircum}{\kern0pt}{\isacharplus}{\kern0pt}{\isachardoublequoteclose}\ \isacommand{by}\isamarkupfalse%
\ simp\isanewline
\ \ \ \ \ \ \isacommand{with}\isamarkupfalse%
\ {\isacartoucheopen}{\isasymlangle}u{\isacharcomma}{\kern0pt}v{\isasymrangle}{\isasymin}\ {\isacharquery}{\kern0pt}R{\isacharparenleft}{\kern0pt}w{\isacharparenright}{\kern0pt}{\isacartoucheclose}\isanewline
\ \ \ \ \ \ \isacommand{show}\isamarkupfalse%
\ {\isachardoublequoteopen}{\isasymlangle}x{\isacharcomma}{\kern0pt}v{\isasymrangle}{\isasymin}\ {\isacharquery}{\kern0pt}R{\isacharparenleft}{\kern0pt}w{\isacharparenright}{\kern0pt}{\isacharcircum}{\kern0pt}{\isacharplus}{\kern0pt}{\isachardoublequoteclose}\ \isacommand{using}\isamarkupfalse%
\ trancl{\isacharunderscore}{\kern0pt}trans\ r{\isacharunderscore}{\kern0pt}into{\isacharunderscore}{\kern0pt}trancl\isanewline
\ \ \ \ \ \ \ \ \isacommand{by}\isamarkupfalse%
\ simp\isanewline
\ \ \ \ \isacommand{qed}\isamarkupfalse%
\isanewline
\ \ \isacommand{{\isacharbraceright}{\kern0pt}}\isamarkupfalse%
\isanewline
\ \ \isacommand{then}\isamarkupfalse%
\ \isacommand{show}\isamarkupfalse%
\ {\isachardoublequoteopen}v\ {\isasymin}\ {\isacharquery}{\kern0pt}A{\isacharparenleft}{\kern0pt}w{\isacharparenright}{\kern0pt}\ {\isasymlongrightarrow}\ {\isasymlangle}x{\isacharcomma}{\kern0pt}\ v{\isasymrangle}\ {\isasymin}\ {\isacharquery}{\kern0pt}R{\isacharparenleft}{\kern0pt}w{\isacharparenright}{\kern0pt}{\isacharcircum}{\kern0pt}{\isacharplus}{\kern0pt}{\isachardoublequoteclose}\isanewline
\ \ \ \ \isakeyword{if}\ {\isachardoublequoteopen}x\ {\isasymin}\ {\isacharquery}{\kern0pt}A{\isacharparenleft}{\kern0pt}w{\isacharparenright}{\kern0pt}{\isachardoublequoteclose}\isanewline
\ \ \ \ \ \ {\isachardoublequoteopen}{\isasymlangle}x{\isacharcomma}{\kern0pt}\ y{\isasymrangle}\ {\isasymin}\ {\isacharquery}{\kern0pt}R{\isacharparenleft}{\kern0pt}z{\isacharparenright}{\kern0pt}{\isacharcircum}{\kern0pt}{\isacharplus}{\kern0pt}{\isachardoublequoteclose}\isanewline
\ \ \ \ \ \ {\isachardoublequoteopen}{\isasymlangle}x{\isacharcomma}{\kern0pt}\ u{\isasymrangle}\ {\isasymin}\ {\isacharquery}{\kern0pt}R{\isacharparenleft}{\kern0pt}z{\isacharparenright}{\kern0pt}{\isacharcircum}{\kern0pt}{\isacharplus}{\kern0pt}{\isachardoublequoteclose}\isanewline
\ \ \ \ \ \ {\isachardoublequoteopen}{\isasymlangle}u{\isacharcomma}{\kern0pt}\ v{\isasymrangle}\ {\isasymin}\ {\isacharquery}{\kern0pt}R{\isacharparenleft}{\kern0pt}z{\isacharparenright}{\kern0pt}{\isachardoublequoteclose}\isanewline
\ \ \ \ \ \ {\isachardoublequoteopen}u\ {\isasymin}\ {\isacharquery}{\kern0pt}A{\isacharparenleft}{\kern0pt}w{\isacharparenright}{\kern0pt}\ {\isasymlongrightarrow}\ {\isasymlangle}x{\isacharcomma}{\kern0pt}\ u{\isasymrangle}\ {\isasymin}\ {\isacharquery}{\kern0pt}R{\isacharparenleft}{\kern0pt}w{\isacharparenright}{\kern0pt}{\isacharcircum}{\kern0pt}{\isacharplus}{\kern0pt}{\isachardoublequoteclose}\ \isakeyword{for}\ u\ v\isanewline
\ \ \ \ \isacommand{using}\isamarkupfalse%
\ that\ \isacommand{by}\isamarkupfalse%
\ simp\isanewline
\isacommand{qed}\isamarkupfalse%
%
\endisatagproof
{\isafoldproof}%
%
\isadelimproof
\isanewline
%
\endisadelimproof
\isanewline
\isacommand{lemma}\isamarkupfalse%
\ forcerel{\isacharunderscore}{\kern0pt}eq\ {\isacharcolon}{\kern0pt}\isanewline
\ \ \isakeyword{assumes}\ {\isachardoublequoteopen}{\isasymlangle}z{\isacharcomma}{\kern0pt}x{\isasymrangle}\ {\isasymin}\ forcerel{\isacharparenleft}{\kern0pt}P{\isacharcomma}{\kern0pt}x{\isacharparenright}{\kern0pt}{\isachardoublequoteclose}\isanewline
\ \ \isakeyword{shows}\ {\isachardoublequoteopen}forcerel{\isacharparenleft}{\kern0pt}P{\isacharcomma}{\kern0pt}z{\isacharparenright}{\kern0pt}\ {\isacharequal}{\kern0pt}\ forcerel{\isacharparenleft}{\kern0pt}P{\isacharcomma}{\kern0pt}x{\isacharparenright}{\kern0pt}\ {\isasyminter}\ names{\isacharunderscore}{\kern0pt}below{\isacharparenleft}{\kern0pt}P{\isacharcomma}{\kern0pt}z{\isacharparenright}{\kern0pt}{\isasymtimes}names{\isacharunderscore}{\kern0pt}below{\isacharparenleft}{\kern0pt}P{\isacharcomma}{\kern0pt}z{\isacharparenright}{\kern0pt}{\isachardoublequoteclose}\isanewline
%
\isadelimproof
\ \ %
\endisadelimproof
%
\isatagproof
\isacommand{using}\isamarkupfalse%
\ assms\ aux\ forcerelD\ forcerel{\isacharunderscore}{\kern0pt}mono{\isacharbrackleft}{\kern0pt}of\ z\ x\ x{\isacharbrackright}{\kern0pt}\ subsetI\isanewline
\ \ \isacommand{by}\isamarkupfalse%
\ auto%
\endisatagproof
{\isafoldproof}%
%
\isadelimproof
\isanewline
%
\endisadelimproof
\isanewline
\isacommand{lemma}\isamarkupfalse%
\ forcerel{\isacharunderscore}{\kern0pt}below{\isacharunderscore}{\kern0pt}aux\ {\isacharcolon}{\kern0pt}\isanewline
\ \ \isakeyword{assumes}\ {\isachardoublequoteopen}{\isasymlangle}z{\isacharcomma}{\kern0pt}x{\isasymrangle}\ {\isasymin}\ forcerel{\isacharparenleft}{\kern0pt}P{\isacharcomma}{\kern0pt}x{\isacharparenright}{\kern0pt}{\isachardoublequoteclose}\ {\isachardoublequoteopen}{\isasymlangle}u{\isacharcomma}{\kern0pt}z{\isasymrangle}\ {\isasymin}\ forcerel{\isacharparenleft}{\kern0pt}P{\isacharcomma}{\kern0pt}x{\isacharparenright}{\kern0pt}{\isachardoublequoteclose}\isanewline
\ \ \isakeyword{shows}\ {\isachardoublequoteopen}u\ {\isasymin}\ names{\isacharunderscore}{\kern0pt}below{\isacharparenleft}{\kern0pt}P{\isacharcomma}{\kern0pt}z{\isacharparenright}{\kern0pt}{\isachardoublequoteclose}\isanewline
%
\isadelimproof
\ \ %
\endisadelimproof
%
\isatagproof
\isacommand{using}\isamarkupfalse%
\ assms{\isacharparenleft}{\kern0pt}{\isadigit{2}}{\isacharparenright}{\kern0pt}\isanewline
\ \ \isacommand{unfolding}\isamarkupfalse%
\ forcerel{\isacharunderscore}{\kern0pt}def\isanewline
\isacommand{proof}\isamarkupfalse%
{\isacharparenleft}{\kern0pt}rule\ trancl{\isacharunderscore}{\kern0pt}induct{\isacharparenright}{\kern0pt}\isanewline
\ \ \isacommand{show}\isamarkupfalse%
\ \ {\isachardoublequoteopen}u\ {\isasymin}\ names{\isacharunderscore}{\kern0pt}below{\isacharparenleft}{\kern0pt}P{\isacharcomma}{\kern0pt}y{\isacharparenright}{\kern0pt}{\isachardoublequoteclose}\ \isakeyword{if}\ {\isachardoublequoteopen}\ {\isasymlangle}u{\isacharcomma}{\kern0pt}\ y{\isasymrangle}\ {\isasymin}\ frecrel{\isacharparenleft}{\kern0pt}names{\isacharunderscore}{\kern0pt}below{\isacharparenleft}{\kern0pt}P{\isacharcomma}{\kern0pt}\ x{\isacharparenright}{\kern0pt}{\isacharparenright}{\kern0pt}{\isachardoublequoteclose}\ \isakeyword{for}\ y\isanewline
\ \ \ \ \isacommand{using}\isamarkupfalse%
\ that\ vimage{\isacharunderscore}{\kern0pt}singleton{\isacharunderscore}{\kern0pt}iff\ arg{\isacharunderscore}{\kern0pt}into{\isacharunderscore}{\kern0pt}names{\isacharunderscore}{\kern0pt}below{\isadigit{2}}\ \isacommand{by}\isamarkupfalse%
\ simp\isanewline
\isacommand{next}\isamarkupfalse%
\isanewline
\ \ \isacommand{show}\isamarkupfalse%
\ {\isachardoublequoteopen}u\ {\isasymin}\ names{\isacharunderscore}{\kern0pt}below{\isacharparenleft}{\kern0pt}P{\isacharcomma}{\kern0pt}z{\isacharparenright}{\kern0pt}{\isachardoublequoteclose}\isanewline
\ \ \ \ \isakeyword{if}\ {\isachardoublequoteopen}{\isasymlangle}u{\isacharcomma}{\kern0pt}\ y{\isasymrangle}\ {\isasymin}\ frecrel{\isacharparenleft}{\kern0pt}names{\isacharunderscore}{\kern0pt}below{\isacharparenleft}{\kern0pt}P{\isacharcomma}{\kern0pt}\ x{\isacharparenright}{\kern0pt}{\isacharparenright}{\kern0pt}{\isacharcircum}{\kern0pt}{\isacharplus}{\kern0pt}{\isachardoublequoteclose}\isanewline
\ \ \ \ \ \ {\isachardoublequoteopen}{\isasymlangle}y{\isacharcomma}{\kern0pt}\ z{\isasymrangle}\ {\isasymin}\ frecrel{\isacharparenleft}{\kern0pt}names{\isacharunderscore}{\kern0pt}below{\isacharparenleft}{\kern0pt}P{\isacharcomma}{\kern0pt}\ x{\isacharparenright}{\kern0pt}{\isacharparenright}{\kern0pt}{\isachardoublequoteclose}\isanewline
\ \ \ \ \ \ {\isachardoublequoteopen}u\ {\isasymin}\ names{\isacharunderscore}{\kern0pt}below{\isacharparenleft}{\kern0pt}P{\isacharcomma}{\kern0pt}\ y{\isacharparenright}{\kern0pt}{\isachardoublequoteclose}\isanewline
\ \ \ \ \isakeyword{for}\ y\ z\isanewline
\ \ \ \ \isacommand{using}\isamarkupfalse%
\ that\ arg{\isacharunderscore}{\kern0pt}into{\isacharunderscore}{\kern0pt}names{\isacharunderscore}{\kern0pt}below{\isadigit{2}}{\isacharbrackleft}{\kern0pt}of\ y\ z\ x{\isacharbrackright}{\kern0pt}\ names{\isacharunderscore}{\kern0pt}below{\isacharunderscore}{\kern0pt}tr\ \isacommand{by}\isamarkupfalse%
\ simp\isanewline
\isacommand{qed}\isamarkupfalse%
%
\endisatagproof
{\isafoldproof}%
%
\isadelimproof
\isanewline
%
\endisadelimproof
\isanewline
\isacommand{lemma}\isamarkupfalse%
\ forcerel{\isacharunderscore}{\kern0pt}below\ {\isacharcolon}{\kern0pt}\isanewline
\ \ \isakeyword{assumes}\ {\isachardoublequoteopen}{\isasymlangle}z{\isacharcomma}{\kern0pt}x{\isasymrangle}\ {\isasymin}\ forcerel{\isacharparenleft}{\kern0pt}P{\isacharcomma}{\kern0pt}x{\isacharparenright}{\kern0pt}{\isachardoublequoteclose}\isanewline
\ \ \isakeyword{shows}\ {\isachardoublequoteopen}forcerel{\isacharparenleft}{\kern0pt}P{\isacharcomma}{\kern0pt}x{\isacharparenright}{\kern0pt}\ {\isacharminus}{\kern0pt}{\isacharbackquote}{\kern0pt}{\isacharbackquote}{\kern0pt}\ {\isacharbraceleft}{\kern0pt}z{\isacharbraceright}{\kern0pt}\ {\isasymsubseteq}\ names{\isacharunderscore}{\kern0pt}below{\isacharparenleft}{\kern0pt}P{\isacharcomma}{\kern0pt}z{\isacharparenright}{\kern0pt}{\isachardoublequoteclose}\isanewline
%
\isadelimproof
\ \ %
\endisadelimproof
%
\isatagproof
\isacommand{using}\isamarkupfalse%
\ vimage{\isacharunderscore}{\kern0pt}singleton{\isacharunderscore}{\kern0pt}iff\ assms\ forcerel{\isacharunderscore}{\kern0pt}below{\isacharunderscore}{\kern0pt}aux\ \isacommand{by}\isamarkupfalse%
\ auto%
\endisatagproof
{\isafoldproof}%
%
\isadelimproof
\isanewline
%
\endisadelimproof
\isanewline
\isacommand{lemma}\isamarkupfalse%
\ relation{\isacharunderscore}{\kern0pt}forcerel\ {\isacharcolon}{\kern0pt}\isanewline
\ \ \isakeyword{shows}\ {\isachardoublequoteopen}relation{\isacharparenleft}{\kern0pt}forcerel{\isacharparenleft}{\kern0pt}P{\isacharcomma}{\kern0pt}z{\isacharparenright}{\kern0pt}{\isacharparenright}{\kern0pt}{\isachardoublequoteclose}\ {\isachardoublequoteopen}trans{\isacharparenleft}{\kern0pt}forcerel{\isacharparenleft}{\kern0pt}P{\isacharcomma}{\kern0pt}z{\isacharparenright}{\kern0pt}{\isacharparenright}{\kern0pt}{\isachardoublequoteclose}\isanewline
%
\isadelimproof
\ \ %
\endisadelimproof
%
\isatagproof
\isacommand{unfolding}\isamarkupfalse%
\ forcerel{\isacharunderscore}{\kern0pt}def\ \isacommand{using}\isamarkupfalse%
\ relation{\isacharunderscore}{\kern0pt}trancl\ trans{\isacharunderscore}{\kern0pt}trancl\ \isacommand{by}\isamarkupfalse%
\ simp{\isacharunderscore}{\kern0pt}all%
\endisatagproof
{\isafoldproof}%
%
\isadelimproof
\isanewline
%
\endisadelimproof
\isanewline
\isacommand{lemma}\isamarkupfalse%
\ Hfrc{\isacharunderscore}{\kern0pt}restrict{\isacharunderscore}{\kern0pt}trancl{\isacharcolon}{\kern0pt}\ {\isachardoublequoteopen}bool{\isacharunderscore}{\kern0pt}of{\isacharunderscore}{\kern0pt}o{\isacharparenleft}{\kern0pt}Hfrc{\isacharparenleft}{\kern0pt}P{\isacharcomma}{\kern0pt}\ leq{\isacharcomma}{\kern0pt}\ y{\isacharcomma}{\kern0pt}\ restrict{\isacharparenleft}{\kern0pt}f{\isacharcomma}{\kern0pt}frecrel{\isacharparenleft}{\kern0pt}names{\isacharunderscore}{\kern0pt}below{\isacharparenleft}{\kern0pt}P{\isacharcomma}{\kern0pt}x{\isacharparenright}{\kern0pt}{\isacharparenright}{\kern0pt}{\isacharminus}{\kern0pt}{\isacharbackquote}{\kern0pt}{\isacharbackquote}{\kern0pt}{\isacharbraceleft}{\kern0pt}y{\isacharbraceright}{\kern0pt}{\isacharparenright}{\kern0pt}{\isacharparenright}{\kern0pt}{\isacharparenright}{\kern0pt}\isanewline
\ \ \ \ \ \ \ \ \ {\isacharequal}{\kern0pt}\ bool{\isacharunderscore}{\kern0pt}of{\isacharunderscore}{\kern0pt}o{\isacharparenleft}{\kern0pt}Hfrc{\isacharparenleft}{\kern0pt}P{\isacharcomma}{\kern0pt}\ leq{\isacharcomma}{\kern0pt}\ y{\isacharcomma}{\kern0pt}\ restrict{\isacharparenleft}{\kern0pt}f{\isacharcomma}{\kern0pt}{\isacharparenleft}{\kern0pt}frecrel{\isacharparenleft}{\kern0pt}names{\isacharunderscore}{\kern0pt}below{\isacharparenleft}{\kern0pt}P{\isacharcomma}{\kern0pt}x{\isacharparenright}{\kern0pt}{\isacharparenright}{\kern0pt}{\isacharcircum}{\kern0pt}{\isacharplus}{\kern0pt}{\isacharparenright}{\kern0pt}{\isacharminus}{\kern0pt}{\isacharbackquote}{\kern0pt}{\isacharbackquote}{\kern0pt}{\isacharbraceleft}{\kern0pt}y{\isacharbraceright}{\kern0pt}{\isacharparenright}{\kern0pt}{\isacharparenright}{\kern0pt}{\isacharparenright}{\kern0pt}{\isachardoublequoteclose}\isanewline
%
\isadelimproof
\ \ %
\endisadelimproof
%
\isatagproof
\isacommand{unfolding}\isamarkupfalse%
\ Hfrc{\isacharunderscore}{\kern0pt}def\ bool{\isacharunderscore}{\kern0pt}of{\isacharunderscore}{\kern0pt}o{\isacharunderscore}{\kern0pt}def\ eq{\isacharunderscore}{\kern0pt}case{\isacharunderscore}{\kern0pt}def\ mem{\isacharunderscore}{\kern0pt}case{\isacharunderscore}{\kern0pt}def\isanewline
\ \ \isacommand{using}\isamarkupfalse%
\ restrict{\isacharunderscore}{\kern0pt}trancl{\isacharunderscore}{\kern0pt}forcerel\ frecRI{\isadigit{1}}\ frecRI{\isadigit{2}}\ frecRI{\isadigit{3}}\isanewline
\ \ \isacommand{unfolding}\isamarkupfalse%
\ forcerel{\isacharunderscore}{\kern0pt}def\isanewline
\ \ \isacommand{by}\isamarkupfalse%
\ simp%
\endisatagproof
{\isafoldproof}%
%
\isadelimproof
\isanewline
%
\endisadelimproof
\isanewline
\isanewline
\isacommand{lemma}\isamarkupfalse%
\ frc{\isacharunderscore}{\kern0pt}at{\isacharunderscore}{\kern0pt}trancl{\isacharcolon}{\kern0pt}\ {\isachardoublequoteopen}frc{\isacharunderscore}{\kern0pt}at{\isacharparenleft}{\kern0pt}P{\isacharcomma}{\kern0pt}leq{\isacharcomma}{\kern0pt}z{\isacharparenright}{\kern0pt}\ {\isacharequal}{\kern0pt}\ wfrec{\isacharparenleft}{\kern0pt}forcerel{\isacharparenleft}{\kern0pt}P{\isacharcomma}{\kern0pt}z{\isacharparenright}{\kern0pt}{\isacharcomma}{\kern0pt}z{\isacharcomma}{\kern0pt}{\isasymlambda}x\ f{\isachardot}{\kern0pt}\ bool{\isacharunderscore}{\kern0pt}of{\isacharunderscore}{\kern0pt}o{\isacharparenleft}{\kern0pt}Hfrc{\isacharparenleft}{\kern0pt}P{\isacharcomma}{\kern0pt}leq{\isacharcomma}{\kern0pt}x{\isacharcomma}{\kern0pt}f{\isacharparenright}{\kern0pt}{\isacharparenright}{\kern0pt}{\isacharparenright}{\kern0pt}{\isachardoublequoteclose}\isanewline
%
\isadelimproof
\ \ %
\endisadelimproof
%
\isatagproof
\isacommand{unfolding}\isamarkupfalse%
\ frc{\isacharunderscore}{\kern0pt}at{\isacharunderscore}{\kern0pt}def\ forcerel{\isacharunderscore}{\kern0pt}def\ \isacommand{using}\isamarkupfalse%
\ wf{\isacharunderscore}{\kern0pt}eq{\isacharunderscore}{\kern0pt}trancl\ Hfrc{\isacharunderscore}{\kern0pt}restrict{\isacharunderscore}{\kern0pt}trancl\ \isacommand{by}\isamarkupfalse%
\ simp%
\endisatagproof
{\isafoldproof}%
%
\isadelimproof
\isanewline
%
\endisadelimproof
\isanewline
\isanewline
\isacommand{lemma}\isamarkupfalse%
\ forcerelI{\isadigit{1}}\ {\isacharcolon}{\kern0pt}\isanewline
\ \ \isakeyword{assumes}\ {\isachardoublequoteopen}n{\isadigit{1}}\ {\isasymin}\ domain{\isacharparenleft}{\kern0pt}b{\isacharparenright}{\kern0pt}\ {\isasymor}\ n{\isadigit{1}}\ {\isasymin}\ domain{\isacharparenleft}{\kern0pt}c{\isacharparenright}{\kern0pt}{\isachardoublequoteclose}\ {\isachardoublequoteopen}p{\isasymin}P{\isachardoublequoteclose}\ {\isachardoublequoteopen}d{\isasymin}P{\isachardoublequoteclose}\isanewline
\ \ \isakeyword{shows}\ {\isachardoublequoteopen}{\isasymlangle}{\isasymlangle}{\isadigit{1}}{\isacharcomma}{\kern0pt}\ n{\isadigit{1}}{\isacharcomma}{\kern0pt}\ b{\isacharcomma}{\kern0pt}\ p{\isasymrangle}{\isacharcomma}{\kern0pt}\ {\isasymlangle}{\isadigit{0}}{\isacharcomma}{\kern0pt}b{\isacharcomma}{\kern0pt}c{\isacharcomma}{\kern0pt}d{\isasymrangle}{\isasymrangle}{\isasymin}\ forcerel{\isacharparenleft}{\kern0pt}P{\isacharcomma}{\kern0pt}{\isasymlangle}{\isadigit{0}}{\isacharcomma}{\kern0pt}b{\isacharcomma}{\kern0pt}c{\isacharcomma}{\kern0pt}d{\isasymrangle}{\isacharparenright}{\kern0pt}{\isachardoublequoteclose}\isanewline
%
\isadelimproof
%
\endisadelimproof
%
\isatagproof
\isacommand{proof}\isamarkupfalse%
\ {\isacharminus}{\kern0pt}\isanewline
\ \ \isacommand{let}\isamarkupfalse%
\ {\isacharquery}{\kern0pt}x{\isacharequal}{\kern0pt}{\isachardoublequoteopen}{\isasymlangle}{\isadigit{1}}{\isacharcomma}{\kern0pt}\ n{\isadigit{1}}{\isacharcomma}{\kern0pt}\ b{\isacharcomma}{\kern0pt}\ p{\isasymrangle}{\isachardoublequoteclose}\isanewline
\ \ \isacommand{let}\isamarkupfalse%
\ {\isacharquery}{\kern0pt}y{\isacharequal}{\kern0pt}{\isachardoublequoteopen}{\isasymlangle}{\isadigit{0}}{\isacharcomma}{\kern0pt}b{\isacharcomma}{\kern0pt}c{\isacharcomma}{\kern0pt}d{\isasymrangle}{\isachardoublequoteclose}\isanewline
\ \ \isacommand{from}\isamarkupfalse%
\ assms\isanewline
\ \ \isacommand{have}\isamarkupfalse%
\ {\isachardoublequoteopen}frecR{\isacharparenleft}{\kern0pt}{\isacharquery}{\kern0pt}x{\isacharcomma}{\kern0pt}{\isacharquery}{\kern0pt}y{\isacharparenright}{\kern0pt}{\isachardoublequoteclose}\isanewline
\ \ \ \ \isacommand{using}\isamarkupfalse%
\ frecRI{\isadigit{1}}\ \isacommand{by}\isamarkupfalse%
\ simp\isanewline
\ \ \isacommand{then}\isamarkupfalse%
\isanewline
\ \ \isacommand{have}\isamarkupfalse%
\ {\isachardoublequoteopen}{\isacharquery}{\kern0pt}x{\isasymin}names{\isacharunderscore}{\kern0pt}below{\isacharparenleft}{\kern0pt}P{\isacharcomma}{\kern0pt}{\isacharquery}{\kern0pt}y{\isacharparenright}{\kern0pt}{\isachardoublequoteclose}\ \ {\isachardoublequoteopen}{\isacharquery}{\kern0pt}y\ {\isasymin}\ names{\isacharunderscore}{\kern0pt}below{\isacharparenleft}{\kern0pt}P{\isacharcomma}{\kern0pt}{\isacharquery}{\kern0pt}y{\isacharparenright}{\kern0pt}{\isachardoublequoteclose}\isanewline
\ \ \ \ \isacommand{using}\isamarkupfalse%
\ names{\isacharunderscore}{\kern0pt}belowI\ \ assms\ components{\isacharunderscore}{\kern0pt}in{\isacharunderscore}{\kern0pt}eclose\isanewline
\ \ \ \ \isacommand{unfolding}\isamarkupfalse%
\ names{\isacharunderscore}{\kern0pt}below{\isacharunderscore}{\kern0pt}def\ \isacommand{by}\isamarkupfalse%
\ auto\isanewline
\ \ \isacommand{with}\isamarkupfalse%
\ {\isacartoucheopen}frecR{\isacharparenleft}{\kern0pt}{\isacharquery}{\kern0pt}x{\isacharcomma}{\kern0pt}{\isacharquery}{\kern0pt}y{\isacharparenright}{\kern0pt}{\isacartoucheclose}\isanewline
\ \ \isacommand{show}\isamarkupfalse%
\ {\isacharquery}{\kern0pt}thesis\isanewline
\ \ \ \ \isacommand{unfolding}\isamarkupfalse%
\ forcerel{\isacharunderscore}{\kern0pt}def\ frecrel{\isacharunderscore}{\kern0pt}def\isanewline
\ \ \ \ \isacommand{using}\isamarkupfalse%
\ subsetD{\isacharbrackleft}{\kern0pt}OF\ r{\isacharunderscore}{\kern0pt}subset{\isacharunderscore}{\kern0pt}trancl{\isacharbrackleft}{\kern0pt}OF\ relation{\isacharunderscore}{\kern0pt}Rrel{\isacharbrackright}{\kern0pt}{\isacharbrackright}{\kern0pt}\ RrelI\isanewline
\ \ \ \ \isacommand{by}\isamarkupfalse%
\ auto\isanewline
\isacommand{qed}\isamarkupfalse%
%
\endisatagproof
{\isafoldproof}%
%
\isadelimproof
\isanewline
%
\endisadelimproof
\isanewline
\isacommand{lemma}\isamarkupfalse%
\ forcerelI{\isadigit{2}}\ {\isacharcolon}{\kern0pt}\isanewline
\ \ \isakeyword{assumes}\ {\isachardoublequoteopen}n{\isadigit{1}}\ {\isasymin}\ domain{\isacharparenleft}{\kern0pt}b{\isacharparenright}{\kern0pt}\ {\isasymor}\ n{\isadigit{1}}\ {\isasymin}\ domain{\isacharparenleft}{\kern0pt}c{\isacharparenright}{\kern0pt}{\isachardoublequoteclose}\ {\isachardoublequoteopen}p{\isasymin}P{\isachardoublequoteclose}\ {\isachardoublequoteopen}d{\isasymin}P{\isachardoublequoteclose}\isanewline
\ \ \isakeyword{shows}\ {\isachardoublequoteopen}{\isasymlangle}{\isasymlangle}{\isadigit{1}}{\isacharcomma}{\kern0pt}\ n{\isadigit{1}}{\isacharcomma}{\kern0pt}\ c{\isacharcomma}{\kern0pt}\ p{\isasymrangle}{\isacharcomma}{\kern0pt}\ {\isasymlangle}{\isadigit{0}}{\isacharcomma}{\kern0pt}b{\isacharcomma}{\kern0pt}c{\isacharcomma}{\kern0pt}d{\isasymrangle}{\isasymrangle}{\isasymin}\ forcerel{\isacharparenleft}{\kern0pt}P{\isacharcomma}{\kern0pt}{\isasymlangle}{\isadigit{0}}{\isacharcomma}{\kern0pt}b{\isacharcomma}{\kern0pt}c{\isacharcomma}{\kern0pt}d{\isasymrangle}{\isacharparenright}{\kern0pt}{\isachardoublequoteclose}\isanewline
%
\isadelimproof
%
\endisadelimproof
%
\isatagproof
\isacommand{proof}\isamarkupfalse%
\ {\isacharminus}{\kern0pt}\isanewline
\ \ \isacommand{let}\isamarkupfalse%
\ {\isacharquery}{\kern0pt}x{\isacharequal}{\kern0pt}{\isachardoublequoteopen}{\isasymlangle}{\isadigit{1}}{\isacharcomma}{\kern0pt}\ n{\isadigit{1}}{\isacharcomma}{\kern0pt}\ c{\isacharcomma}{\kern0pt}\ p{\isasymrangle}{\isachardoublequoteclose}\isanewline
\ \ \isacommand{let}\isamarkupfalse%
\ {\isacharquery}{\kern0pt}y{\isacharequal}{\kern0pt}{\isachardoublequoteopen}{\isasymlangle}{\isadigit{0}}{\isacharcomma}{\kern0pt}b{\isacharcomma}{\kern0pt}c{\isacharcomma}{\kern0pt}d{\isasymrangle}{\isachardoublequoteclose}\isanewline
\ \ \isacommand{from}\isamarkupfalse%
\ assms\isanewline
\ \ \isacommand{have}\isamarkupfalse%
\ {\isachardoublequoteopen}frecR{\isacharparenleft}{\kern0pt}{\isacharquery}{\kern0pt}x{\isacharcomma}{\kern0pt}{\isacharquery}{\kern0pt}y{\isacharparenright}{\kern0pt}{\isachardoublequoteclose}\isanewline
\ \ \ \ \isacommand{using}\isamarkupfalse%
\ frecRI{\isadigit{2}}\ \isacommand{by}\isamarkupfalse%
\ simp\isanewline
\ \ \isacommand{then}\isamarkupfalse%
\isanewline
\ \ \isacommand{have}\isamarkupfalse%
\ {\isachardoublequoteopen}{\isacharquery}{\kern0pt}x{\isasymin}names{\isacharunderscore}{\kern0pt}below{\isacharparenleft}{\kern0pt}P{\isacharcomma}{\kern0pt}{\isacharquery}{\kern0pt}y{\isacharparenright}{\kern0pt}{\isachardoublequoteclose}\ \ {\isachardoublequoteopen}{\isacharquery}{\kern0pt}y\ {\isasymin}\ names{\isacharunderscore}{\kern0pt}below{\isacharparenleft}{\kern0pt}P{\isacharcomma}{\kern0pt}{\isacharquery}{\kern0pt}y{\isacharparenright}{\kern0pt}{\isachardoublequoteclose}\isanewline
\ \ \ \ \isacommand{using}\isamarkupfalse%
\ names{\isacharunderscore}{\kern0pt}belowI\ \ assms\ components{\isacharunderscore}{\kern0pt}in{\isacharunderscore}{\kern0pt}eclose\isanewline
\ \ \ \ \isacommand{unfolding}\isamarkupfalse%
\ names{\isacharunderscore}{\kern0pt}below{\isacharunderscore}{\kern0pt}def\ \isacommand{by}\isamarkupfalse%
\ auto\isanewline
\ \ \isacommand{with}\isamarkupfalse%
\ {\isacartoucheopen}frecR{\isacharparenleft}{\kern0pt}{\isacharquery}{\kern0pt}x{\isacharcomma}{\kern0pt}{\isacharquery}{\kern0pt}y{\isacharparenright}{\kern0pt}{\isacartoucheclose}\isanewline
\ \ \isacommand{show}\isamarkupfalse%
\ {\isacharquery}{\kern0pt}thesis\isanewline
\ \ \ \ \isacommand{unfolding}\isamarkupfalse%
\ forcerel{\isacharunderscore}{\kern0pt}def\ frecrel{\isacharunderscore}{\kern0pt}def\isanewline
\ \ \ \ \isacommand{using}\isamarkupfalse%
\ subsetD{\isacharbrackleft}{\kern0pt}OF\ r{\isacharunderscore}{\kern0pt}subset{\isacharunderscore}{\kern0pt}trancl{\isacharbrackleft}{\kern0pt}OF\ relation{\isacharunderscore}{\kern0pt}Rrel{\isacharbrackright}{\kern0pt}{\isacharbrackright}{\kern0pt}\ RrelI\isanewline
\ \ \ \ \isacommand{by}\isamarkupfalse%
\ auto\isanewline
\isacommand{qed}\isamarkupfalse%
%
\endisatagproof
{\isafoldproof}%
%
\isadelimproof
\isanewline
%
\endisadelimproof
\isanewline
\isacommand{lemma}\isamarkupfalse%
\ forcerelI{\isadigit{3}}\ {\isacharcolon}{\kern0pt}\isanewline
\ \ \isakeyword{assumes}\ {\isachardoublequoteopen}{\isasymlangle}n{\isadigit{2}}{\isacharcomma}{\kern0pt}\ r{\isasymrangle}\ {\isasymin}\ c{\isachardoublequoteclose}\ {\isachardoublequoteopen}p{\isasymin}P{\isachardoublequoteclose}\ {\isachardoublequoteopen}d{\isasymin}P{\isachardoublequoteclose}\ {\isachardoublequoteopen}r\ {\isasymin}\ P{\isachardoublequoteclose}\isanewline
\ \ \isakeyword{shows}\ {\isachardoublequoteopen}{\isasymlangle}{\isasymlangle}{\isadigit{0}}{\isacharcomma}{\kern0pt}\ b{\isacharcomma}{\kern0pt}\ n{\isadigit{2}}{\isacharcomma}{\kern0pt}\ p{\isasymrangle}{\isacharcomma}{\kern0pt}{\isasymlangle}{\isadigit{1}}{\isacharcomma}{\kern0pt}\ b{\isacharcomma}{\kern0pt}\ c{\isacharcomma}{\kern0pt}\ d{\isasymrangle}{\isasymrangle}\ {\isasymin}\ forcerel{\isacharparenleft}{\kern0pt}P{\isacharcomma}{\kern0pt}{\isasymlangle}{\isadigit{1}}{\isacharcomma}{\kern0pt}b{\isacharcomma}{\kern0pt}c{\isacharcomma}{\kern0pt}d{\isasymrangle}{\isacharparenright}{\kern0pt}{\isachardoublequoteclose}\isanewline
%
\isadelimproof
%
\endisadelimproof
%
\isatagproof
\isacommand{proof}\isamarkupfalse%
\ {\isacharminus}{\kern0pt}\isanewline
\ \ \isacommand{let}\isamarkupfalse%
\ {\isacharquery}{\kern0pt}x{\isacharequal}{\kern0pt}{\isachardoublequoteopen}{\isasymlangle}{\isadigit{0}}{\isacharcomma}{\kern0pt}\ b{\isacharcomma}{\kern0pt}\ n{\isadigit{2}}{\isacharcomma}{\kern0pt}\ p{\isasymrangle}{\isachardoublequoteclose}\isanewline
\ \ \isacommand{let}\isamarkupfalse%
\ {\isacharquery}{\kern0pt}y{\isacharequal}{\kern0pt}{\isachardoublequoteopen}{\isasymlangle}{\isadigit{1}}{\isacharcomma}{\kern0pt}\ b{\isacharcomma}{\kern0pt}\ c{\isacharcomma}{\kern0pt}\ d{\isasymrangle}{\isachardoublequoteclose}\isanewline
\ \ \isacommand{from}\isamarkupfalse%
\ assms\isanewline
\ \ \isacommand{have}\isamarkupfalse%
\ {\isachardoublequoteopen}frecR{\isacharparenleft}{\kern0pt}{\isacharquery}{\kern0pt}x{\isacharcomma}{\kern0pt}{\isacharquery}{\kern0pt}y{\isacharparenright}{\kern0pt}{\isachardoublequoteclose}\isanewline
\ \ \ \ \isacommand{using}\isamarkupfalse%
\ assms\ frecRI{\isadigit{3}}\ \isacommand{by}\isamarkupfalse%
\ simp\isanewline
\ \ \isacommand{then}\isamarkupfalse%
\isanewline
\ \ \isacommand{have}\isamarkupfalse%
\ {\isachardoublequoteopen}{\isacharquery}{\kern0pt}x{\isasymin}names{\isacharunderscore}{\kern0pt}below{\isacharparenleft}{\kern0pt}P{\isacharcomma}{\kern0pt}{\isacharquery}{\kern0pt}y{\isacharparenright}{\kern0pt}{\isachardoublequoteclose}\ \ {\isachardoublequoteopen}{\isacharquery}{\kern0pt}y\ {\isasymin}\ names{\isacharunderscore}{\kern0pt}below{\isacharparenleft}{\kern0pt}P{\isacharcomma}{\kern0pt}{\isacharquery}{\kern0pt}y{\isacharparenright}{\kern0pt}{\isachardoublequoteclose}\isanewline
\ \ \ \ \isacommand{using}\isamarkupfalse%
\ names{\isacharunderscore}{\kern0pt}belowI\ \ assms\ components{\isacharunderscore}{\kern0pt}in{\isacharunderscore}{\kern0pt}eclose\isanewline
\ \ \ \ \isacommand{unfolding}\isamarkupfalse%
\ names{\isacharunderscore}{\kern0pt}below{\isacharunderscore}{\kern0pt}def\ \isacommand{by}\isamarkupfalse%
\ auto\isanewline
\ \ \isacommand{with}\isamarkupfalse%
\ {\isacartoucheopen}frecR{\isacharparenleft}{\kern0pt}{\isacharquery}{\kern0pt}x{\isacharcomma}{\kern0pt}{\isacharquery}{\kern0pt}y{\isacharparenright}{\kern0pt}{\isacartoucheclose}\isanewline
\ \ \isacommand{show}\isamarkupfalse%
\ {\isacharquery}{\kern0pt}thesis\isanewline
\ \ \ \ \isacommand{unfolding}\isamarkupfalse%
\ forcerel{\isacharunderscore}{\kern0pt}def\ frecrel{\isacharunderscore}{\kern0pt}def\isanewline
\ \ \ \ \isacommand{using}\isamarkupfalse%
\ subsetD{\isacharbrackleft}{\kern0pt}OF\ r{\isacharunderscore}{\kern0pt}subset{\isacharunderscore}{\kern0pt}trancl{\isacharbrackleft}{\kern0pt}OF\ relation{\isacharunderscore}{\kern0pt}Rrel{\isacharbrackright}{\kern0pt}{\isacharbrackright}{\kern0pt}\ RrelI\isanewline
\ \ \ \ \isacommand{by}\isamarkupfalse%
\ auto\isanewline
\isacommand{qed}\isamarkupfalse%
%
\endisatagproof
{\isafoldproof}%
%
\isadelimproof
\isanewline
%
\endisadelimproof
\isanewline
\isacommand{lemmas}\isamarkupfalse%
\ forcerelI\ {\isacharequal}{\kern0pt}\ forcerelI{\isadigit{1}}{\isacharbrackleft}{\kern0pt}THEN\ vimage{\isacharunderscore}{\kern0pt}singleton{\isacharunderscore}{\kern0pt}iff{\isacharbrackleft}{\kern0pt}THEN\ iffD{\isadigit{2}}{\isacharbrackright}{\kern0pt}{\isacharbrackright}{\kern0pt}\isanewline
\ \ forcerelI{\isadigit{2}}{\isacharbrackleft}{\kern0pt}THEN\ vimage{\isacharunderscore}{\kern0pt}singleton{\isacharunderscore}{\kern0pt}iff{\isacharbrackleft}{\kern0pt}THEN\ iffD{\isadigit{2}}{\isacharbrackright}{\kern0pt}{\isacharbrackright}{\kern0pt}\isanewline
\ \ forcerelI{\isadigit{3}}{\isacharbrackleft}{\kern0pt}THEN\ vimage{\isacharunderscore}{\kern0pt}singleton{\isacharunderscore}{\kern0pt}iff{\isacharbrackleft}{\kern0pt}THEN\ iffD{\isadigit{2}}{\isacharbrackright}{\kern0pt}{\isacharbrackright}{\kern0pt}\isanewline
\isanewline
\isacommand{lemma}\isamarkupfalse%
\ \ aux{\isacharunderscore}{\kern0pt}def{\isacharunderscore}{\kern0pt}frc{\isacharunderscore}{\kern0pt}at{\isacharcolon}{\kern0pt}\isanewline
\ \ \isakeyword{assumes}\ {\isachardoublequoteopen}z\ {\isasymin}\ forcerel{\isacharparenleft}{\kern0pt}P{\isacharcomma}{\kern0pt}x{\isacharparenright}{\kern0pt}\ {\isacharminus}{\kern0pt}{\isacharbackquote}{\kern0pt}{\isacharbackquote}{\kern0pt}\ {\isacharbraceleft}{\kern0pt}x{\isacharbraceright}{\kern0pt}{\isachardoublequoteclose}\isanewline
\ \ \isakeyword{shows}\ {\isachardoublequoteopen}wfrec{\isacharparenleft}{\kern0pt}forcerel{\isacharparenleft}{\kern0pt}P{\isacharcomma}{\kern0pt}x{\isacharparenright}{\kern0pt}{\isacharcomma}{\kern0pt}\ z{\isacharcomma}{\kern0pt}\ H{\isacharparenright}{\kern0pt}\ {\isacharequal}{\kern0pt}\ \ wfrec{\isacharparenleft}{\kern0pt}forcerel{\isacharparenleft}{\kern0pt}P{\isacharcomma}{\kern0pt}z{\isacharparenright}{\kern0pt}{\isacharcomma}{\kern0pt}\ z{\isacharcomma}{\kern0pt}\ H{\isacharparenright}{\kern0pt}{\isachardoublequoteclose}\isanewline
%
\isadelimproof
%
\endisadelimproof
%
\isatagproof
\isacommand{proof}\isamarkupfalse%
\ {\isacharminus}{\kern0pt}\isanewline
\ \ \isacommand{let}\isamarkupfalse%
\ {\isacharquery}{\kern0pt}A{\isacharequal}{\kern0pt}{\isachardoublequoteopen}names{\isacharunderscore}{\kern0pt}below{\isacharparenleft}{\kern0pt}P{\isacharcomma}{\kern0pt}z{\isacharparenright}{\kern0pt}{\isachardoublequoteclose}\isanewline
\ \ \isacommand{from}\isamarkupfalse%
\ assms\isanewline
\ \ \isacommand{have}\isamarkupfalse%
\ {\isachardoublequoteopen}{\isasymlangle}z{\isacharcomma}{\kern0pt}x{\isasymrangle}\ {\isasymin}\ forcerel{\isacharparenleft}{\kern0pt}P{\isacharcomma}{\kern0pt}x{\isacharparenright}{\kern0pt}{\isachardoublequoteclose}\isanewline
\ \ \ \ \isacommand{using}\isamarkupfalse%
\ vimage{\isacharunderscore}{\kern0pt}singleton{\isacharunderscore}{\kern0pt}iff\ \isacommand{by}\isamarkupfalse%
\ simp\isanewline
\ \ \isacommand{then}\isamarkupfalse%
\isanewline
\ \ \isacommand{have}\isamarkupfalse%
\ {\isachardoublequoteopen}z\ {\isasymin}\ {\isacharquery}{\kern0pt}A{\isachardoublequoteclose}\isanewline
\ \ \ \ \isacommand{using}\isamarkupfalse%
\ forcerel{\isacharunderscore}{\kern0pt}arg{\isacharunderscore}{\kern0pt}into{\isacharunderscore}{\kern0pt}names{\isacharunderscore}{\kern0pt}below\ \isacommand{by}\isamarkupfalse%
\ simp\isanewline
\ \ \isacommand{from}\isamarkupfalse%
\ {\isacartoucheopen}{\isasymlangle}z{\isacharcomma}{\kern0pt}x{\isasymrangle}\ {\isasymin}\ forcerel{\isacharparenleft}{\kern0pt}P{\isacharcomma}{\kern0pt}x{\isacharparenright}{\kern0pt}{\isacartoucheclose}\isanewline
\ \ \isacommand{have}\isamarkupfalse%
\ E{\isacharcolon}{\kern0pt}{\isachardoublequoteopen}forcerel{\isacharparenleft}{\kern0pt}P{\isacharcomma}{\kern0pt}z{\isacharparenright}{\kern0pt}\ {\isacharequal}{\kern0pt}\ forcerel{\isacharparenleft}{\kern0pt}P{\isacharcomma}{\kern0pt}x{\isacharparenright}{\kern0pt}\ {\isasyminter}\ {\isacharparenleft}{\kern0pt}{\isacharquery}{\kern0pt}A{\isasymtimes}{\isacharquery}{\kern0pt}A{\isacharparenright}{\kern0pt}{\isachardoublequoteclose}\isanewline
\ \ \ \ {\isachardoublequoteopen}forcerel{\isacharparenleft}{\kern0pt}P{\isacharcomma}{\kern0pt}x{\isacharparenright}{\kern0pt}\ {\isacharminus}{\kern0pt}{\isacharbackquote}{\kern0pt}{\isacharbackquote}{\kern0pt}\ {\isacharbraceleft}{\kern0pt}z{\isacharbraceright}{\kern0pt}\ {\isasymsubseteq}\ {\isacharquery}{\kern0pt}A{\isachardoublequoteclose}\isanewline
\ \ \ \ \isacommand{using}\isamarkupfalse%
\ forcerel{\isacharunderscore}{\kern0pt}eq\ forcerel{\isacharunderscore}{\kern0pt}below\isanewline
\ \ \ \ \isacommand{by}\isamarkupfalse%
\ auto\isanewline
\ \ \isacommand{with}\isamarkupfalse%
\ {\isacartoucheopen}z{\isasymin}{\isacharquery}{\kern0pt}A{\isacartoucheclose}\isanewline
\ \ \isacommand{have}\isamarkupfalse%
\ {\isachardoublequoteopen}wfrec{\isacharparenleft}{\kern0pt}forcerel{\isacharparenleft}{\kern0pt}P{\isacharcomma}{\kern0pt}x{\isacharparenright}{\kern0pt}{\isacharcomma}{\kern0pt}\ z{\isacharcomma}{\kern0pt}\ H{\isacharparenright}{\kern0pt}\ {\isacharequal}{\kern0pt}\ wfrec{\isacharbrackleft}{\kern0pt}{\isacharquery}{\kern0pt}A{\isacharbrackright}{\kern0pt}{\isacharparenleft}{\kern0pt}forcerel{\isacharparenleft}{\kern0pt}P{\isacharcomma}{\kern0pt}x{\isacharparenright}{\kern0pt}{\isacharcomma}{\kern0pt}\ z{\isacharcomma}{\kern0pt}\ H{\isacharparenright}{\kern0pt}{\isachardoublequoteclose}\isanewline
\ \ \ \ \isacommand{using}\isamarkupfalse%
\ wfrec{\isacharunderscore}{\kern0pt}trans{\isacharunderscore}{\kern0pt}restr{\isacharbrackleft}{\kern0pt}OF\ relation{\isacharunderscore}{\kern0pt}forcerel{\isacharparenleft}{\kern0pt}{\isadigit{1}}{\isacharparenright}{\kern0pt}\ wf{\isacharunderscore}{\kern0pt}forcerel\ relation{\isacharunderscore}{\kern0pt}forcerel{\isacharparenleft}{\kern0pt}{\isadigit{2}}{\isacharparenright}{\kern0pt}{\isacharcomma}{\kern0pt}\ of\ x\ z\ {\isacharquery}{\kern0pt}A{\isacharbrackright}{\kern0pt}\isanewline
\ \ \ \ \isacommand{by}\isamarkupfalse%
\ simp\isanewline
\ \ \isacommand{then}\isamarkupfalse%
\ \isacommand{show}\isamarkupfalse%
\ {\isacharquery}{\kern0pt}thesis\isanewline
\ \ \ \ \isacommand{using}\isamarkupfalse%
\ E\ wfrec{\isacharunderscore}{\kern0pt}restr{\isacharunderscore}{\kern0pt}eq\ \isacommand{by}\isamarkupfalse%
\ simp\isanewline
\isacommand{qed}\isamarkupfalse%
%
\endisatagproof
{\isafoldproof}%
%
\isadelimproof
%
\endisadelimproof
%
\isadelimdocument
%
\endisadelimdocument
%
\isatagdocument
%
\isamarkupsubsection{Recursive expression of \isa{frc{\isacharunderscore}{\kern0pt}at}%
}
\isamarkuptrue%
%
\endisatagdocument
{\isafolddocument}%
%
\isadelimdocument
%
\endisadelimdocument
\isacommand{lemma}\isamarkupfalse%
\ def{\isacharunderscore}{\kern0pt}frc{\isacharunderscore}{\kern0pt}at\ {\isacharcolon}{\kern0pt}\isanewline
\ \ \isakeyword{assumes}\ {\isachardoublequoteopen}p{\isasymin}P{\isachardoublequoteclose}\isanewline
\ \ \isakeyword{shows}\isanewline
\ \ \ \ {\isachardoublequoteopen}frc{\isacharunderscore}{\kern0pt}at{\isacharparenleft}{\kern0pt}P{\isacharcomma}{\kern0pt}leq{\isacharcomma}{\kern0pt}{\isasymlangle}ft{\isacharcomma}{\kern0pt}n{\isadigit{1}}{\isacharcomma}{\kern0pt}n{\isadigit{2}}{\isacharcomma}{\kern0pt}p{\isasymrangle}{\isacharparenright}{\kern0pt}\ {\isacharequal}{\kern0pt}\isanewline
\ \ \ bool{\isacharunderscore}{\kern0pt}of{\isacharunderscore}{\kern0pt}o{\isacharparenleft}{\kern0pt}\ p\ {\isasymin}P\ {\isasymand}\isanewline
\ \ {\isacharparenleft}{\kern0pt}\ \ ft\ {\isacharequal}{\kern0pt}\ {\isadigit{0}}\ {\isasymand}\ \ {\isacharparenleft}{\kern0pt}{\isasymforall}s{\isachardot}{\kern0pt}\ s{\isasymin}domain{\isacharparenleft}{\kern0pt}n{\isadigit{1}}{\isacharparenright}{\kern0pt}\ {\isasymunion}\ domain{\isacharparenleft}{\kern0pt}n{\isadigit{2}}{\isacharparenright}{\kern0pt}\ {\isasymlongrightarrow}\isanewline
\ \ \ \ \ \ \ \ {\isacharparenleft}{\kern0pt}{\isasymforall}q{\isachardot}{\kern0pt}\ q{\isasymin}P\ {\isasymand}\ q\ {\isasympreceq}\ p\ {\isasymlongrightarrow}\ {\isacharparenleft}{\kern0pt}frc{\isacharunderscore}{\kern0pt}at{\isacharparenleft}{\kern0pt}P{\isacharcomma}{\kern0pt}leq{\isacharcomma}{\kern0pt}{\isasymlangle}{\isadigit{1}}{\isacharcomma}{\kern0pt}s{\isacharcomma}{\kern0pt}n{\isadigit{1}}{\isacharcomma}{\kern0pt}q{\isasymrangle}{\isacharparenright}{\kern0pt}\ {\isacharequal}{\kern0pt}{\isadigit{1}}\ {\isasymlongleftrightarrow}\ frc{\isacharunderscore}{\kern0pt}at{\isacharparenleft}{\kern0pt}P{\isacharcomma}{\kern0pt}leq{\isacharcomma}{\kern0pt}{\isasymlangle}{\isadigit{1}}{\isacharcomma}{\kern0pt}s{\isacharcomma}{\kern0pt}n{\isadigit{2}}{\isacharcomma}{\kern0pt}q{\isasymrangle}{\isacharparenright}{\kern0pt}\ {\isacharequal}{\kern0pt}{\isadigit{1}}{\isacharparenright}{\kern0pt}{\isacharparenright}{\kern0pt}{\isacharparenright}{\kern0pt}\isanewline
\ \ \ {\isasymor}\ ft\ {\isacharequal}{\kern0pt}\ {\isadigit{1}}\ {\isasymand}\ {\isacharparenleft}{\kern0pt}\ {\isasymforall}v{\isasymin}P{\isachardot}{\kern0pt}\ v\ {\isasympreceq}\ p\ {\isasymlongrightarrow}\isanewline
\ \ \ \ {\isacharparenleft}{\kern0pt}{\isasymexists}q{\isachardot}{\kern0pt}\ {\isasymexists}s{\isachardot}{\kern0pt}\ {\isasymexists}r{\isachardot}{\kern0pt}\ r{\isasymin}P\ {\isasymand}\ q{\isasymin}P\ {\isasymand}\ q\ {\isasympreceq}\ v\ {\isasymand}\ {\isasymlangle}s{\isacharcomma}{\kern0pt}r{\isasymrangle}\ {\isasymin}\ n{\isadigit{2}}\ {\isasymand}\ q\ {\isasympreceq}\ r\ {\isasymand}\ \ frc{\isacharunderscore}{\kern0pt}at{\isacharparenleft}{\kern0pt}P{\isacharcomma}{\kern0pt}leq{\isacharcomma}{\kern0pt}{\isasymlangle}{\isadigit{0}}{\isacharcomma}{\kern0pt}n{\isadigit{1}}{\isacharcomma}{\kern0pt}s{\isacharcomma}{\kern0pt}q{\isasymrangle}{\isacharparenright}{\kern0pt}\ {\isacharequal}{\kern0pt}\ {\isadigit{1}}{\isacharparenright}{\kern0pt}{\isacharparenright}{\kern0pt}{\isacharparenright}{\kern0pt}{\isacharparenright}{\kern0pt}{\isachardoublequoteclose}\isanewline
%
\isadelimproof
%
\endisadelimproof
%
\isatagproof
\isacommand{proof}\isamarkupfalse%
\ {\isacharminus}{\kern0pt}\isanewline
\ \ \isacommand{let}\isamarkupfalse%
\ {\isacharquery}{\kern0pt}r{\isacharequal}{\kern0pt}{\isachardoublequoteopen}{\isasymlambda}y{\isachardot}{\kern0pt}\ forcerel{\isacharparenleft}{\kern0pt}P{\isacharcomma}{\kern0pt}y{\isacharparenright}{\kern0pt}{\isachardoublequoteclose}\ \isakeyword{and}\ {\isacharquery}{\kern0pt}Hf{\isacharequal}{\kern0pt}{\isachardoublequoteopen}{\isasymlambda}x\ f{\isachardot}{\kern0pt}\ bool{\isacharunderscore}{\kern0pt}of{\isacharunderscore}{\kern0pt}o{\isacharparenleft}{\kern0pt}Hfrc{\isacharparenleft}{\kern0pt}P{\isacharcomma}{\kern0pt}leq{\isacharcomma}{\kern0pt}x{\isacharcomma}{\kern0pt}f{\isacharparenright}{\kern0pt}{\isacharparenright}{\kern0pt}{\isachardoublequoteclose}\isanewline
\ \ \isacommand{let}\isamarkupfalse%
\ {\isacharquery}{\kern0pt}t{\isacharequal}{\kern0pt}{\isachardoublequoteopen}{\isasymlambda}y{\isachardot}{\kern0pt}\ {\isacharquery}{\kern0pt}r{\isacharparenleft}{\kern0pt}y{\isacharparenright}{\kern0pt}\ {\isacharminus}{\kern0pt}{\isacharbackquote}{\kern0pt}{\isacharbackquote}{\kern0pt}\ {\isacharbraceleft}{\kern0pt}y{\isacharbraceright}{\kern0pt}{\isachardoublequoteclose}\isanewline
\ \ \isacommand{let}\isamarkupfalse%
\ {\isacharquery}{\kern0pt}arg{\isacharequal}{\kern0pt}{\isachardoublequoteopen}{\isasymlangle}ft{\isacharcomma}{\kern0pt}n{\isadigit{1}}{\isacharcomma}{\kern0pt}n{\isadigit{2}}{\isacharcomma}{\kern0pt}p{\isasymrangle}{\isachardoublequoteclose}\isanewline
\ \ \isacommand{from}\isamarkupfalse%
\ wf{\isacharunderscore}{\kern0pt}forcerel\isanewline
\ \ \isacommand{have}\isamarkupfalse%
\ wfr{\isacharcolon}{\kern0pt}\ {\isachardoublequoteopen}{\isasymforall}w\ {\isachardot}{\kern0pt}\ wf{\isacharparenleft}{\kern0pt}{\isacharquery}{\kern0pt}r{\isacharparenleft}{\kern0pt}w{\isacharparenright}{\kern0pt}{\isacharparenright}{\kern0pt}{\isachardoublequoteclose}\ \isacommand{{\isachardot}{\kern0pt}{\isachardot}{\kern0pt}}\isamarkupfalse%
\isanewline
\ \ \isacommand{with}\isamarkupfalse%
\ wfrec\ {\isacharbrackleft}{\kern0pt}of\ {\isachardoublequoteopen}{\isacharquery}{\kern0pt}r{\isacharparenleft}{\kern0pt}{\isacharquery}{\kern0pt}arg{\isacharparenright}{\kern0pt}{\isachardoublequoteclose}\ {\isacharquery}{\kern0pt}arg\ {\isacharquery}{\kern0pt}Hf{\isacharbrackright}{\kern0pt}\isanewline
\ \ \isacommand{have}\isamarkupfalse%
\ {\isachardoublequoteopen}frc{\isacharunderscore}{\kern0pt}at{\isacharparenleft}{\kern0pt}P{\isacharcomma}{\kern0pt}leq{\isacharcomma}{\kern0pt}{\isacharquery}{\kern0pt}arg{\isacharparenright}{\kern0pt}\ {\isacharequal}{\kern0pt}\ {\isacharquery}{\kern0pt}Hf{\isacharparenleft}{\kern0pt}\ {\isacharquery}{\kern0pt}arg{\isacharcomma}{\kern0pt}\ {\isasymlambda}x{\isasymin}{\isacharquery}{\kern0pt}r{\isacharparenleft}{\kern0pt}{\isacharquery}{\kern0pt}arg{\isacharparenright}{\kern0pt}\ {\isacharminus}{\kern0pt}{\isacharbackquote}{\kern0pt}{\isacharbackquote}{\kern0pt}\ {\isacharbraceleft}{\kern0pt}{\isacharquery}{\kern0pt}arg{\isacharbraceright}{\kern0pt}{\isachardot}{\kern0pt}\ wfrec{\isacharparenleft}{\kern0pt}{\isacharquery}{\kern0pt}r{\isacharparenleft}{\kern0pt}{\isacharquery}{\kern0pt}arg{\isacharparenright}{\kern0pt}{\isacharcomma}{\kern0pt}\ x{\isacharcomma}{\kern0pt}\ {\isacharquery}{\kern0pt}Hf{\isacharparenright}{\kern0pt}{\isacharparenright}{\kern0pt}{\isachardoublequoteclose}\isanewline
\ \ \ \ \isacommand{using}\isamarkupfalse%
\ frc{\isacharunderscore}{\kern0pt}at{\isacharunderscore}{\kern0pt}trancl\ \isacommand{by}\isamarkupfalse%
\ simp\isanewline
\ \ \isacommand{also}\isamarkupfalse%
\isanewline
\ \ \isacommand{have}\isamarkupfalse%
\ {\isachardoublequoteopen}\ {\isachardot}{\kern0pt}{\isachardot}{\kern0pt}{\isachardot}{\kern0pt}\ {\isacharequal}{\kern0pt}\ {\isacharquery}{\kern0pt}Hf{\isacharparenleft}{\kern0pt}\ {\isacharquery}{\kern0pt}arg{\isacharcomma}{\kern0pt}\ {\isasymlambda}x{\isasymin}{\isacharquery}{\kern0pt}r{\isacharparenleft}{\kern0pt}{\isacharquery}{\kern0pt}arg{\isacharparenright}{\kern0pt}\ {\isacharminus}{\kern0pt}{\isacharbackquote}{\kern0pt}{\isacharbackquote}{\kern0pt}\ {\isacharbraceleft}{\kern0pt}{\isacharquery}{\kern0pt}arg{\isacharbraceright}{\kern0pt}{\isachardot}{\kern0pt}\ frc{\isacharunderscore}{\kern0pt}at{\isacharparenleft}{\kern0pt}P{\isacharcomma}{\kern0pt}leq{\isacharcomma}{\kern0pt}x{\isacharparenright}{\kern0pt}{\isacharparenright}{\kern0pt}{\isachardoublequoteclose}\isanewline
\ \ \ \ \isacommand{using}\isamarkupfalse%
\ aux{\isacharunderscore}{\kern0pt}def{\isacharunderscore}{\kern0pt}frc{\isacharunderscore}{\kern0pt}at\ frc{\isacharunderscore}{\kern0pt}at{\isacharunderscore}{\kern0pt}trancl\ \isacommand{by}\isamarkupfalse%
\ simp\isanewline
\ \ \isacommand{finally}\isamarkupfalse%
\isanewline
\ \ \isacommand{show}\isamarkupfalse%
\ {\isacharquery}{\kern0pt}thesis\isanewline
\ \ \ \ \isacommand{unfolding}\isamarkupfalse%
\ Hfrc{\isacharunderscore}{\kern0pt}def\ mem{\isacharunderscore}{\kern0pt}case{\isacharunderscore}{\kern0pt}def\ eq{\isacharunderscore}{\kern0pt}case{\isacharunderscore}{\kern0pt}def\isanewline
\ \ \ \ \isacommand{using}\isamarkupfalse%
\ forcerelI\ \ assms\isanewline
\ \ \ \ \isacommand{by}\isamarkupfalse%
\ auto\isanewline
\isacommand{qed}\isamarkupfalse%
%
\endisatagproof
{\isafoldproof}%
%
\isadelimproof
%
\endisadelimproof
%
\isadelimdocument
%
\endisadelimdocument
%
\isatagdocument
%
\isamarkupsubsection{Absoluteness of \isa{frc{\isacharunderscore}{\kern0pt}at}%
}
\isamarkuptrue%
%
\endisatagdocument
{\isafolddocument}%
%
\isadelimdocument
%
\endisadelimdocument
\isacommand{lemma}\isamarkupfalse%
\ trans{\isacharunderscore}{\kern0pt}forcerel{\isacharunderscore}{\kern0pt}t\ {\isacharcolon}{\kern0pt}\ {\isachardoublequoteopen}trans{\isacharparenleft}{\kern0pt}forcerel{\isacharparenleft}{\kern0pt}P{\isacharcomma}{\kern0pt}x{\isacharparenright}{\kern0pt}{\isacharparenright}{\kern0pt}{\isachardoublequoteclose}\isanewline
%
\isadelimproof
\ \ %
\endisadelimproof
%
\isatagproof
\isacommand{unfolding}\isamarkupfalse%
\ forcerel{\isacharunderscore}{\kern0pt}def\ \isacommand{using}\isamarkupfalse%
\ trans{\isacharunderscore}{\kern0pt}trancl\ \isacommand{{\isachardot}{\kern0pt}}\isamarkupfalse%
%
\endisatagproof
{\isafoldproof}%
%
\isadelimproof
\isanewline
%
\endisadelimproof
\isanewline
\isacommand{lemma}\isamarkupfalse%
\ relation{\isacharunderscore}{\kern0pt}forcerel{\isacharunderscore}{\kern0pt}t\ {\isacharcolon}{\kern0pt}\ {\isachardoublequoteopen}relation{\isacharparenleft}{\kern0pt}forcerel{\isacharparenleft}{\kern0pt}P{\isacharcomma}{\kern0pt}x{\isacharparenright}{\kern0pt}{\isacharparenright}{\kern0pt}{\isachardoublequoteclose}\isanewline
%
\isadelimproof
\ \ %
\endisadelimproof
%
\isatagproof
\isacommand{unfolding}\isamarkupfalse%
\ forcerel{\isacharunderscore}{\kern0pt}def\ \isacommand{using}\isamarkupfalse%
\ relation{\isacharunderscore}{\kern0pt}trancl\ \isacommand{{\isachardot}{\kern0pt}}\isamarkupfalse%
%
\endisatagproof
{\isafoldproof}%
%
\isadelimproof
\isanewline
%
\endisadelimproof
\isanewline
\isanewline
\isacommand{lemma}\isamarkupfalse%
\ forcerel{\isacharunderscore}{\kern0pt}in{\isacharunderscore}{\kern0pt}M\ {\isacharcolon}{\kern0pt}\isanewline
\ \ \isakeyword{assumes}\isanewline
\ \ \ \ {\isachardoublequoteopen}x{\isasymin}M{\isachardoublequoteclose}\isanewline
\ \ \isakeyword{shows}\isanewline
\ \ \ \ {\isachardoublequoteopen}forcerel{\isacharparenleft}{\kern0pt}P{\isacharcomma}{\kern0pt}x{\isacharparenright}{\kern0pt}{\isasymin}M{\isachardoublequoteclose}\isanewline
%
\isadelimproof
\ \ %
\endisadelimproof
%
\isatagproof
\isacommand{unfolding}\isamarkupfalse%
\ forcerel{\isacharunderscore}{\kern0pt}def\ def{\isacharunderscore}{\kern0pt}frecrel\ names{\isacharunderscore}{\kern0pt}below{\isacharunderscore}{\kern0pt}def\isanewline
\isacommand{proof}\isamarkupfalse%
\ {\isacharminus}{\kern0pt}\isanewline
\ \ \isacommand{let}\isamarkupfalse%
\ {\isacharquery}{\kern0pt}Q\ {\isacharequal}{\kern0pt}\ {\isachardoublequoteopen}{\isadigit{2}}\ {\isasymtimes}\ ecloseN{\isacharparenleft}{\kern0pt}x{\isacharparenright}{\kern0pt}\ {\isasymtimes}\ ecloseN{\isacharparenleft}{\kern0pt}x{\isacharparenright}{\kern0pt}\ {\isasymtimes}\ P{\isachardoublequoteclose}\isanewline
\ \ \isacommand{have}\isamarkupfalse%
\ {\isachardoublequoteopen}{\isacharquery}{\kern0pt}Q\ {\isasymtimes}\ {\isacharquery}{\kern0pt}Q\ {\isasymin}\ M{\isachardoublequoteclose}\isanewline
\ \ \ \ \isacommand{using}\isamarkupfalse%
\ {\isacartoucheopen}x{\isasymin}M{\isacartoucheclose}\ P{\isacharunderscore}{\kern0pt}in{\isacharunderscore}{\kern0pt}M\ twoN{\isacharunderscore}{\kern0pt}in{\isacharunderscore}{\kern0pt}M\ ecloseN{\isacharunderscore}{\kern0pt}closed\ cartprod{\isacharunderscore}{\kern0pt}closed\ \isacommand{by}\isamarkupfalse%
\ simp\isanewline
\ \ \isacommand{moreover}\isamarkupfalse%
\isanewline
\ \ \isacommand{have}\isamarkupfalse%
\ {\isachardoublequoteopen}separation{\isacharparenleft}{\kern0pt}{\isacharhash}{\kern0pt}{\isacharhash}{\kern0pt}M{\isacharcomma}{\kern0pt}{\isasymlambda}z{\isachardot}{\kern0pt}\ {\isasymexists}x\ y{\isachardot}{\kern0pt}\ z\ {\isacharequal}{\kern0pt}\ {\isasymlangle}x{\isacharcomma}{\kern0pt}\ y{\isasymrangle}\ {\isasymand}\ frecR{\isacharparenleft}{\kern0pt}x{\isacharcomma}{\kern0pt}\ y{\isacharparenright}{\kern0pt}{\isacharparenright}{\kern0pt}{\isachardoublequoteclose}\isanewline
\ \ \isacommand{proof}\isamarkupfalse%
\ {\isacharminus}{\kern0pt}\isanewline
\ \ \ \ \isacommand{have}\isamarkupfalse%
\ {\isachardoublequoteopen}arity{\isacharparenleft}{\kern0pt}frecrelP{\isacharunderscore}{\kern0pt}fm{\isacharparenleft}{\kern0pt}{\isadigit{0}}{\isacharparenright}{\kern0pt}{\isacharparenright}{\kern0pt}\ {\isacharequal}{\kern0pt}\ {\isadigit{1}}{\isachardoublequoteclose}\isanewline
\ \ \ \ \ \ \isacommand{unfolding}\isamarkupfalse%
\ number{\isadigit{1}}{\isacharunderscore}{\kern0pt}fm{\isacharunderscore}{\kern0pt}def\ frecrelP{\isacharunderscore}{\kern0pt}fm{\isacharunderscore}{\kern0pt}def\isanewline
\ \ \ \ \ \ \isacommand{by}\isamarkupfalse%
\ {\isacharparenleft}{\kern0pt}simp\ del{\isacharcolon}{\kern0pt}FOL{\isacharunderscore}{\kern0pt}sats{\isacharunderscore}{\kern0pt}iff\ pair{\isacharunderscore}{\kern0pt}abs\ empty{\isacharunderscore}{\kern0pt}abs\isanewline
\ \ \ \ \ \ \ \ \ \ add{\isacharcolon}{\kern0pt}\ fm{\isacharunderscore}{\kern0pt}defs\ frecR{\isacharunderscore}{\kern0pt}fm{\isacharunderscore}{\kern0pt}def\ number{\isadigit{1}}{\isacharunderscore}{\kern0pt}fm{\isacharunderscore}{\kern0pt}def\ components{\isacharunderscore}{\kern0pt}defs\ nat{\isacharunderscore}{\kern0pt}simp{\isacharunderscore}{\kern0pt}union{\isacharparenright}{\kern0pt}\isanewline
\ \ \ \ \isacommand{then}\isamarkupfalse%
\isanewline
\ \ \ \ \isacommand{have}\isamarkupfalse%
\ {\isachardoublequoteopen}separation{\isacharparenleft}{\kern0pt}{\isacharhash}{\kern0pt}{\isacharhash}{\kern0pt}M{\isacharcomma}{\kern0pt}\ {\isasymlambda}z{\isachardot}{\kern0pt}\ sats{\isacharparenleft}{\kern0pt}M{\isacharcomma}{\kern0pt}frecrelP{\isacharunderscore}{\kern0pt}fm{\isacharparenleft}{\kern0pt}{\isadigit{0}}{\isacharparenright}{\kern0pt}\ {\isacharcomma}{\kern0pt}\ {\isacharbrackleft}{\kern0pt}z{\isacharbrackright}{\kern0pt}{\isacharparenright}{\kern0pt}{\isacharparenright}{\kern0pt}{\isachardoublequoteclose}\isanewline
\ \ \ \ \ \ \isacommand{using}\isamarkupfalse%
\ separation{\isacharunderscore}{\kern0pt}ax\ \isacommand{by}\isamarkupfalse%
\ simp\isanewline
\ \ \ \ \isacommand{moreover}\isamarkupfalse%
\isanewline
\ \ \ \ \isacommand{have}\isamarkupfalse%
\ {\isachardoublequoteopen}frecrelP{\isacharparenleft}{\kern0pt}{\isacharhash}{\kern0pt}{\isacharhash}{\kern0pt}M{\isacharcomma}{\kern0pt}z{\isacharparenright}{\kern0pt}\ {\isasymlongleftrightarrow}\ sats{\isacharparenleft}{\kern0pt}M{\isacharcomma}{\kern0pt}frecrelP{\isacharunderscore}{\kern0pt}fm{\isacharparenleft}{\kern0pt}{\isadigit{0}}{\isacharparenright}{\kern0pt}{\isacharcomma}{\kern0pt}{\isacharbrackleft}{\kern0pt}z{\isacharbrackright}{\kern0pt}{\isacharparenright}{\kern0pt}{\isachardoublequoteclose}\isanewline
\ \ \ \ \ \ \isakeyword{if}\ {\isachardoublequoteopen}z{\isasymin}M{\isachardoublequoteclose}\ \isakeyword{for}\ z\isanewline
\ \ \ \ \ \ \isacommand{using}\isamarkupfalse%
\ that\ sats{\isacharunderscore}{\kern0pt}frecrelP{\isacharunderscore}{\kern0pt}fm{\isacharbrackleft}{\kern0pt}of\ {\isadigit{0}}\ {\isachardoublequoteopen}{\isacharbrackleft}{\kern0pt}z{\isacharbrackright}{\kern0pt}{\isachardoublequoteclose}{\isacharbrackright}{\kern0pt}\ \isacommand{by}\isamarkupfalse%
\ simp\isanewline
\ \ \ \ \isacommand{ultimately}\isamarkupfalse%
\isanewline
\ \ \ \ \isacommand{have}\isamarkupfalse%
\ {\isachardoublequoteopen}separation{\isacharparenleft}{\kern0pt}{\isacharhash}{\kern0pt}{\isacharhash}{\kern0pt}M{\isacharcomma}{\kern0pt}frecrelP{\isacharparenleft}{\kern0pt}{\isacharhash}{\kern0pt}{\isacharhash}{\kern0pt}M{\isacharparenright}{\kern0pt}{\isacharparenright}{\kern0pt}{\isachardoublequoteclose}\isanewline
\ \ \ \ \ \ \isacommand{unfolding}\isamarkupfalse%
\ separation{\isacharunderscore}{\kern0pt}def\ \isacommand{by}\isamarkupfalse%
\ simp\isanewline
\ \ \ \ \isacommand{then}\isamarkupfalse%
\isanewline
\ \ \ \ \isacommand{show}\isamarkupfalse%
\ {\isacharquery}{\kern0pt}thesis\ \isacommand{using}\isamarkupfalse%
\ frecrelP{\isacharunderscore}{\kern0pt}abs\isanewline
\ \ \ \ \ \ \ \ separation{\isacharunderscore}{\kern0pt}cong{\isacharbrackleft}{\kern0pt}of\ {\isachardoublequoteopen}{\isacharhash}{\kern0pt}{\isacharhash}{\kern0pt}M{\isachardoublequoteclose}\ {\isachardoublequoteopen}frecrelP{\isacharparenleft}{\kern0pt}{\isacharhash}{\kern0pt}{\isacharhash}{\kern0pt}M{\isacharparenright}{\kern0pt}{\isachardoublequoteclose}\ {\isachardoublequoteopen}{\isasymlambda}z{\isachardot}{\kern0pt}\ {\isasymexists}x\ y{\isachardot}{\kern0pt}\ z\ {\isacharequal}{\kern0pt}\ {\isasymlangle}x{\isacharcomma}{\kern0pt}\ y{\isasymrangle}\ {\isasymand}\ frecR{\isacharparenleft}{\kern0pt}x{\isacharcomma}{\kern0pt}\ y{\isacharparenright}{\kern0pt}{\isachardoublequoteclose}{\isacharbrackright}{\kern0pt}\isanewline
\ \ \ \ \ \ \isacommand{by}\isamarkupfalse%
\ simp\isanewline
\ \ \isacommand{qed}\isamarkupfalse%
\isanewline
\ \ \isacommand{ultimately}\isamarkupfalse%
\isanewline
\ \ \isacommand{show}\isamarkupfalse%
\ {\isachardoublequoteopen}{\isacharbraceleft}{\kern0pt}z\ {\isasymin}\ {\isacharquery}{\kern0pt}Q\ {\isasymtimes}\ {\isacharquery}{\kern0pt}Q\ {\isachardot}{\kern0pt}\ {\isasymexists}x\ y{\isachardot}{\kern0pt}\ z\ {\isacharequal}{\kern0pt}\ {\isasymlangle}x{\isacharcomma}{\kern0pt}\ y{\isasymrangle}\ {\isasymand}\ frecR{\isacharparenleft}{\kern0pt}x{\isacharcomma}{\kern0pt}\ y{\isacharparenright}{\kern0pt}{\isacharbraceright}{\kern0pt}{\isacharcircum}{\kern0pt}{\isacharplus}{\kern0pt}\ {\isasymin}\ M{\isachardoublequoteclose}\isanewline
\ \ \ \ \isacommand{using}\isamarkupfalse%
\ separation{\isacharunderscore}{\kern0pt}closed\ frecrelP{\isacharunderscore}{\kern0pt}abs\ trancl{\isacharunderscore}{\kern0pt}closed\ \isacommand{by}\isamarkupfalse%
\ simp\isanewline
\isacommand{qed}\isamarkupfalse%
%
\endisatagproof
{\isafoldproof}%
%
\isadelimproof
\isanewline
%
\endisadelimproof
\isanewline
\isacommand{lemma}\isamarkupfalse%
\ relation{\isadigit{2}}{\isacharunderscore}{\kern0pt}Hfrc{\isacharunderscore}{\kern0pt}at{\isacharunderscore}{\kern0pt}abs{\isacharcolon}{\kern0pt}\isanewline
\ \ {\isachardoublequoteopen}relation{\isadigit{2}}{\isacharparenleft}{\kern0pt}{\isacharhash}{\kern0pt}{\isacharhash}{\kern0pt}M{\isacharcomma}{\kern0pt}is{\isacharunderscore}{\kern0pt}Hfrc{\isacharunderscore}{\kern0pt}at{\isacharparenleft}{\kern0pt}{\isacharhash}{\kern0pt}{\isacharhash}{\kern0pt}M{\isacharcomma}{\kern0pt}P{\isacharcomma}{\kern0pt}leq{\isacharparenright}{\kern0pt}{\isacharcomma}{\kern0pt}{\isasymlambda}x\ f{\isachardot}{\kern0pt}\ bool{\isacharunderscore}{\kern0pt}of{\isacharunderscore}{\kern0pt}o{\isacharparenleft}{\kern0pt}Hfrc{\isacharparenleft}{\kern0pt}P{\isacharcomma}{\kern0pt}leq{\isacharcomma}{\kern0pt}x{\isacharcomma}{\kern0pt}f{\isacharparenright}{\kern0pt}{\isacharparenright}{\kern0pt}{\isacharparenright}{\kern0pt}{\isachardoublequoteclose}\isanewline
%
\isadelimproof
\ \ %
\endisadelimproof
%
\isatagproof
\isacommand{unfolding}\isamarkupfalse%
\ relation{\isadigit{2}}{\isacharunderscore}{\kern0pt}def\ \isacommand{using}\isamarkupfalse%
\ Hfrc{\isacharunderscore}{\kern0pt}at{\isacharunderscore}{\kern0pt}abs\isanewline
\ \ \isacommand{by}\isamarkupfalse%
\ simp%
\endisatagproof
{\isafoldproof}%
%
\isadelimproof
\isanewline
%
\endisadelimproof
\isanewline
\isacommand{lemma}\isamarkupfalse%
\ Hfrc{\isacharunderscore}{\kern0pt}at{\isacharunderscore}{\kern0pt}closed\ {\isacharcolon}{\kern0pt}\isanewline
\ \ {\isachardoublequoteopen}{\isasymforall}x{\isasymin}M{\isachardot}{\kern0pt}\ {\isasymforall}g{\isasymin}M{\isachardot}{\kern0pt}\ function{\isacharparenleft}{\kern0pt}g{\isacharparenright}{\kern0pt}\ {\isasymlongrightarrow}\ bool{\isacharunderscore}{\kern0pt}of{\isacharunderscore}{\kern0pt}o{\isacharparenleft}{\kern0pt}Hfrc{\isacharparenleft}{\kern0pt}P{\isacharcomma}{\kern0pt}leq{\isacharcomma}{\kern0pt}x{\isacharcomma}{\kern0pt}g{\isacharparenright}{\kern0pt}{\isacharparenright}{\kern0pt}{\isasymin}M{\isachardoublequoteclose}\isanewline
%
\isadelimproof
\ \ %
\endisadelimproof
%
\isatagproof
\isacommand{unfolding}\isamarkupfalse%
\ bool{\isacharunderscore}{\kern0pt}of{\isacharunderscore}{\kern0pt}o{\isacharunderscore}{\kern0pt}def\ \isacommand{using}\isamarkupfalse%
\ zero{\isacharunderscore}{\kern0pt}in{\isacharunderscore}{\kern0pt}M\ n{\isacharunderscore}{\kern0pt}in{\isacharunderscore}{\kern0pt}M{\isacharbrackleft}{\kern0pt}of\ {\isadigit{1}}{\isacharbrackright}{\kern0pt}\ \isacommand{by}\isamarkupfalse%
\ simp%
\endisatagproof
{\isafoldproof}%
%
\isadelimproof
\isanewline
%
\endisadelimproof
\isanewline
\isacommand{lemma}\isamarkupfalse%
\ wfrec{\isacharunderscore}{\kern0pt}Hfrc{\isacharunderscore}{\kern0pt}at\ {\isacharcolon}{\kern0pt}\isanewline
\ \ \isakeyword{assumes}\isanewline
\ \ \ \ {\isachardoublequoteopen}X{\isasymin}M{\isachardoublequoteclose}\isanewline
\ \ \isakeyword{shows}\isanewline
\ \ \ \ {\isachardoublequoteopen}wfrec{\isacharunderscore}{\kern0pt}replacement{\isacharparenleft}{\kern0pt}{\isacharhash}{\kern0pt}{\isacharhash}{\kern0pt}M{\isacharcomma}{\kern0pt}is{\isacharunderscore}{\kern0pt}Hfrc{\isacharunderscore}{\kern0pt}at{\isacharparenleft}{\kern0pt}{\isacharhash}{\kern0pt}{\isacharhash}{\kern0pt}M{\isacharcomma}{\kern0pt}P{\isacharcomma}{\kern0pt}leq{\isacharparenright}{\kern0pt}{\isacharcomma}{\kern0pt}forcerel{\isacharparenleft}{\kern0pt}P{\isacharcomma}{\kern0pt}X{\isacharparenright}{\kern0pt}{\isacharparenright}{\kern0pt}{\isachardoublequoteclose}\isanewline
%
\isadelimproof
%
\endisadelimproof
%
\isatagproof
\isacommand{proof}\isamarkupfalse%
\ {\isacharminus}{\kern0pt}\isanewline
\ \ \isacommand{have}\isamarkupfalse%
\ {\isadigit{0}}{\isacharcolon}{\kern0pt}{\isachardoublequoteopen}is{\isacharunderscore}{\kern0pt}Hfrc{\isacharunderscore}{\kern0pt}at{\isacharparenleft}{\kern0pt}{\isacharhash}{\kern0pt}{\isacharhash}{\kern0pt}M{\isacharcomma}{\kern0pt}P{\isacharcomma}{\kern0pt}leq{\isacharcomma}{\kern0pt}a{\isacharcomma}{\kern0pt}b{\isacharcomma}{\kern0pt}c{\isacharparenright}{\kern0pt}\ {\isasymlongleftrightarrow}\isanewline
\ \ \ \ \ \ \ \ sats{\isacharparenleft}{\kern0pt}M{\isacharcomma}{\kern0pt}Hfrc{\isacharunderscore}{\kern0pt}at{\isacharunderscore}{\kern0pt}fm{\isacharparenleft}{\kern0pt}{\isadigit{8}}{\isacharcomma}{\kern0pt}{\isadigit{9}}{\isacharcomma}{\kern0pt}{\isadigit{2}}{\isacharcomma}{\kern0pt}{\isadigit{1}}{\isacharcomma}{\kern0pt}{\isadigit{0}}{\isacharparenright}{\kern0pt}{\isacharcomma}{\kern0pt}{\isacharbrackleft}{\kern0pt}c{\isacharcomma}{\kern0pt}b{\isacharcomma}{\kern0pt}a{\isacharcomma}{\kern0pt}d{\isacharcomma}{\kern0pt}e{\isacharcomma}{\kern0pt}y{\isacharcomma}{\kern0pt}x{\isacharcomma}{\kern0pt}z{\isacharcomma}{\kern0pt}P{\isacharcomma}{\kern0pt}leq{\isacharcomma}{\kern0pt}forcerel{\isacharparenleft}{\kern0pt}P{\isacharcomma}{\kern0pt}X{\isacharparenright}{\kern0pt}{\isacharbrackright}{\kern0pt}{\isacharparenright}{\kern0pt}{\isachardoublequoteclose}\isanewline
\ \ \ \ \isakeyword{if}\ {\isachardoublequoteopen}a{\isasymin}M{\isachardoublequoteclose}\ {\isachardoublequoteopen}b{\isasymin}M{\isachardoublequoteclose}\ {\isachardoublequoteopen}c{\isasymin}M{\isachardoublequoteclose}\ {\isachardoublequoteopen}d{\isasymin}M{\isachardoublequoteclose}\ {\isachardoublequoteopen}e{\isasymin}M{\isachardoublequoteclose}\ {\isachardoublequoteopen}y{\isasymin}M{\isachardoublequoteclose}\ {\isachardoublequoteopen}x{\isasymin}M{\isachardoublequoteclose}\ {\isachardoublequoteopen}z{\isasymin}M{\isachardoublequoteclose}\isanewline
\ \ \ \ \isakeyword{for}\ a\ b\ c\ d\ e\ y\ x\ z\isanewline
\ \ \ \ \isacommand{using}\isamarkupfalse%
\ that\ P{\isacharunderscore}{\kern0pt}in{\isacharunderscore}{\kern0pt}M\ leq{\isacharunderscore}{\kern0pt}in{\isacharunderscore}{\kern0pt}M\ {\isacartoucheopen}X{\isasymin}M{\isacartoucheclose}\ forcerel{\isacharunderscore}{\kern0pt}in{\isacharunderscore}{\kern0pt}M\isanewline
\ \ \ \ \ \ is{\isacharunderscore}{\kern0pt}Hfrc{\isacharunderscore}{\kern0pt}at{\isacharunderscore}{\kern0pt}iff{\isacharunderscore}{\kern0pt}sats{\isacharbrackleft}{\kern0pt}of\ concl{\isacharcolon}{\kern0pt}M\ P\ leq\ a\ b\ c\ {\isadigit{8}}\ {\isadigit{9}}\ {\isadigit{2}}\ {\isadigit{1}}\ {\isadigit{0}}\isanewline
\ \ \ \ \ \ \ \ {\isachardoublequoteopen}{\isacharbrackleft}{\kern0pt}c{\isacharcomma}{\kern0pt}b{\isacharcomma}{\kern0pt}a{\isacharcomma}{\kern0pt}d{\isacharcomma}{\kern0pt}e{\isacharcomma}{\kern0pt}y{\isacharcomma}{\kern0pt}x{\isacharcomma}{\kern0pt}z{\isacharcomma}{\kern0pt}P{\isacharcomma}{\kern0pt}leq{\isacharcomma}{\kern0pt}forcerel{\isacharparenleft}{\kern0pt}P{\isacharcomma}{\kern0pt}X{\isacharparenright}{\kern0pt}{\isacharbrackright}{\kern0pt}{\isachardoublequoteclose}{\isacharbrackright}{\kern0pt}\ \isacommand{by}\isamarkupfalse%
\ simp\isanewline
\ \ \isacommand{have}\isamarkupfalse%
\ {\isadigit{1}}{\isacharcolon}{\kern0pt}{\isachardoublequoteopen}sats{\isacharparenleft}{\kern0pt}M{\isacharcomma}{\kern0pt}is{\isacharunderscore}{\kern0pt}wfrec{\isacharunderscore}{\kern0pt}fm{\isacharparenleft}{\kern0pt}Hfrc{\isacharunderscore}{\kern0pt}at{\isacharunderscore}{\kern0pt}fm{\isacharparenleft}{\kern0pt}{\isadigit{8}}{\isacharcomma}{\kern0pt}{\isadigit{9}}{\isacharcomma}{\kern0pt}{\isadigit{2}}{\isacharcomma}{\kern0pt}{\isadigit{1}}{\isacharcomma}{\kern0pt}{\isadigit{0}}{\isacharparenright}{\kern0pt}{\isacharcomma}{\kern0pt}{\isadigit{5}}{\isacharcomma}{\kern0pt}{\isadigit{1}}{\isacharcomma}{\kern0pt}{\isadigit{0}}{\isacharparenright}{\kern0pt}{\isacharcomma}{\kern0pt}{\isacharbrackleft}{\kern0pt}y{\isacharcomma}{\kern0pt}x{\isacharcomma}{\kern0pt}z{\isacharcomma}{\kern0pt}P{\isacharcomma}{\kern0pt}leq{\isacharcomma}{\kern0pt}forcerel{\isacharparenleft}{\kern0pt}P{\isacharcomma}{\kern0pt}X{\isacharparenright}{\kern0pt}{\isacharbrackright}{\kern0pt}{\isacharparenright}{\kern0pt}\ {\isasymlongleftrightarrow}\isanewline
\ \ \ \ \ \ \ \ \ \ \ \ \ \ \ \ \ \ \ is{\isacharunderscore}{\kern0pt}wfrec{\isacharparenleft}{\kern0pt}{\isacharhash}{\kern0pt}{\isacharhash}{\kern0pt}M{\isacharcomma}{\kern0pt}\ is{\isacharunderscore}{\kern0pt}Hfrc{\isacharunderscore}{\kern0pt}at{\isacharparenleft}{\kern0pt}{\isacharhash}{\kern0pt}{\isacharhash}{\kern0pt}M{\isacharcomma}{\kern0pt}P{\isacharcomma}{\kern0pt}leq{\isacharparenright}{\kern0pt}{\isacharcomma}{\kern0pt}forcerel{\isacharparenleft}{\kern0pt}P{\isacharcomma}{\kern0pt}X{\isacharparenright}{\kern0pt}{\isacharcomma}{\kern0pt}\ x{\isacharcomma}{\kern0pt}\ y{\isacharparenright}{\kern0pt}{\isachardoublequoteclose}\isanewline
\ \ \ \ \isakeyword{if}\ {\isachardoublequoteopen}x{\isasymin}M{\isachardoublequoteclose}\ {\isachardoublequoteopen}y{\isasymin}M{\isachardoublequoteclose}\ {\isachardoublequoteopen}z{\isasymin}M{\isachardoublequoteclose}\ \isakeyword{for}\ x\ y\ z\isanewline
\ \ \ \ \isacommand{using}\isamarkupfalse%
\ \ that\ {\isacartoucheopen}X{\isasymin}M{\isacartoucheclose}\ forcerel{\isacharunderscore}{\kern0pt}in{\isacharunderscore}{\kern0pt}M\ P{\isacharunderscore}{\kern0pt}in{\isacharunderscore}{\kern0pt}M\ leq{\isacharunderscore}{\kern0pt}in{\isacharunderscore}{\kern0pt}M\isanewline
\ \ \ \ \ \ sats{\isacharunderscore}{\kern0pt}is{\isacharunderscore}{\kern0pt}wfrec{\isacharunderscore}{\kern0pt}fm{\isacharbrackleft}{\kern0pt}OF\ {\isadigit{0}}{\isacharbrackright}{\kern0pt}\isanewline
\ \ \ \ \isacommand{by}\isamarkupfalse%
\ simp\isanewline
\ \ \isacommand{let}\isamarkupfalse%
\isanewline
\ \ \ \ {\isacharquery}{\kern0pt}f{\isacharequal}{\kern0pt}{\isachardoublequoteopen}Exists{\isacharparenleft}{\kern0pt}And{\isacharparenleft}{\kern0pt}pair{\isacharunderscore}{\kern0pt}fm{\isacharparenleft}{\kern0pt}{\isadigit{1}}{\isacharcomma}{\kern0pt}{\isadigit{0}}{\isacharcomma}{\kern0pt}{\isadigit{2}}{\isacharparenright}{\kern0pt}{\isacharcomma}{\kern0pt}is{\isacharunderscore}{\kern0pt}wfrec{\isacharunderscore}{\kern0pt}fm{\isacharparenleft}{\kern0pt}Hfrc{\isacharunderscore}{\kern0pt}at{\isacharunderscore}{\kern0pt}fm{\isacharparenleft}{\kern0pt}{\isadigit{8}}{\isacharcomma}{\kern0pt}{\isadigit{9}}{\isacharcomma}{\kern0pt}{\isadigit{2}}{\isacharcomma}{\kern0pt}{\isadigit{1}}{\isacharcomma}{\kern0pt}{\isadigit{0}}{\isacharparenright}{\kern0pt}{\isacharcomma}{\kern0pt}{\isadigit{5}}{\isacharcomma}{\kern0pt}{\isadigit{1}}{\isacharcomma}{\kern0pt}{\isadigit{0}}{\isacharparenright}{\kern0pt}{\isacharparenright}{\kern0pt}{\isacharparenright}{\kern0pt}{\isachardoublequoteclose}\isanewline
\ \ \isacommand{have}\isamarkupfalse%
\ satsf{\isacharcolon}{\kern0pt}{\isachardoublequoteopen}sats{\isacharparenleft}{\kern0pt}M{\isacharcomma}{\kern0pt}\ {\isacharquery}{\kern0pt}f{\isacharcomma}{\kern0pt}\ {\isacharbrackleft}{\kern0pt}x{\isacharcomma}{\kern0pt}z{\isacharcomma}{\kern0pt}P{\isacharcomma}{\kern0pt}leq{\isacharcomma}{\kern0pt}forcerel{\isacharparenleft}{\kern0pt}P{\isacharcomma}{\kern0pt}X{\isacharparenright}{\kern0pt}{\isacharbrackright}{\kern0pt}{\isacharparenright}{\kern0pt}\ {\isasymlongleftrightarrow}\isanewline
\ \ \ \ \ \ \ \ \ \ \ \ \ \ {\isacharparenleft}{\kern0pt}{\isasymexists}y{\isasymin}M{\isachardot}{\kern0pt}\ pair{\isacharparenleft}{\kern0pt}{\isacharhash}{\kern0pt}{\isacharhash}{\kern0pt}M{\isacharcomma}{\kern0pt}x{\isacharcomma}{\kern0pt}y{\isacharcomma}{\kern0pt}z{\isacharparenright}{\kern0pt}\ {\isacharampersand}{\kern0pt}\ is{\isacharunderscore}{\kern0pt}wfrec{\isacharparenleft}{\kern0pt}{\isacharhash}{\kern0pt}{\isacharhash}{\kern0pt}M{\isacharcomma}{\kern0pt}\ is{\isacharunderscore}{\kern0pt}Hfrc{\isacharunderscore}{\kern0pt}at{\isacharparenleft}{\kern0pt}{\isacharhash}{\kern0pt}{\isacharhash}{\kern0pt}M{\isacharcomma}{\kern0pt}P{\isacharcomma}{\kern0pt}leq{\isacharparenright}{\kern0pt}{\isacharcomma}{\kern0pt}forcerel{\isacharparenleft}{\kern0pt}P{\isacharcomma}{\kern0pt}X{\isacharparenright}{\kern0pt}{\isacharcomma}{\kern0pt}\ x{\isacharcomma}{\kern0pt}\ y{\isacharparenright}{\kern0pt}{\isacharparenright}{\kern0pt}{\isachardoublequoteclose}\isanewline
\ \ \ \ \isakeyword{if}\ {\isachardoublequoteopen}x{\isasymin}M{\isachardoublequoteclose}\ {\isachardoublequoteopen}z{\isasymin}M{\isachardoublequoteclose}\ \isakeyword{for}\ x\ z\isanewline
\ \ \ \ \isacommand{using}\isamarkupfalse%
\ that\ {\isadigit{1}}\ {\isacartoucheopen}X{\isasymin}M{\isacartoucheclose}\ forcerel{\isacharunderscore}{\kern0pt}in{\isacharunderscore}{\kern0pt}M\ P{\isacharunderscore}{\kern0pt}in{\isacharunderscore}{\kern0pt}M\ leq{\isacharunderscore}{\kern0pt}in{\isacharunderscore}{\kern0pt}M\ \isacommand{by}\isamarkupfalse%
\ {\isacharparenleft}{\kern0pt}simp\ del{\isacharcolon}{\kern0pt}pair{\isacharunderscore}{\kern0pt}abs{\isacharparenright}{\kern0pt}\isanewline
\ \ \isacommand{have}\isamarkupfalse%
\ artyf{\isacharcolon}{\kern0pt}{\isachardoublequoteopen}arity{\isacharparenleft}{\kern0pt}{\isacharquery}{\kern0pt}f{\isacharparenright}{\kern0pt}\ {\isacharequal}{\kern0pt}\ {\isadigit{5}}{\isachardoublequoteclose}\isanewline
\ \ \ \ \isacommand{unfolding}\isamarkupfalse%
\ is{\isacharunderscore}{\kern0pt}wfrec{\isacharunderscore}{\kern0pt}fm{\isacharunderscore}{\kern0pt}def\ Hfrc{\isacharunderscore}{\kern0pt}at{\isacharunderscore}{\kern0pt}fm{\isacharunderscore}{\kern0pt}def\ Hfrc{\isacharunderscore}{\kern0pt}fm{\isacharunderscore}{\kern0pt}def\ Replace{\isacharunderscore}{\kern0pt}fm{\isacharunderscore}{\kern0pt}def\ PHcheck{\isacharunderscore}{\kern0pt}fm{\isacharunderscore}{\kern0pt}def\isanewline
\ \ \ \ \ \ pair{\isacharunderscore}{\kern0pt}fm{\isacharunderscore}{\kern0pt}def\ upair{\isacharunderscore}{\kern0pt}fm{\isacharunderscore}{\kern0pt}def\ is{\isacharunderscore}{\kern0pt}recfun{\isacharunderscore}{\kern0pt}fm{\isacharunderscore}{\kern0pt}def\ fun{\isacharunderscore}{\kern0pt}apply{\isacharunderscore}{\kern0pt}fm{\isacharunderscore}{\kern0pt}def\ big{\isacharunderscore}{\kern0pt}union{\isacharunderscore}{\kern0pt}fm{\isacharunderscore}{\kern0pt}def\isanewline
\ \ \ \ \ \ pre{\isacharunderscore}{\kern0pt}image{\isacharunderscore}{\kern0pt}fm{\isacharunderscore}{\kern0pt}def\ restriction{\isacharunderscore}{\kern0pt}fm{\isacharunderscore}{\kern0pt}def\ image{\isacharunderscore}{\kern0pt}fm{\isacharunderscore}{\kern0pt}def\ fm{\isacharunderscore}{\kern0pt}defs\ number{\isadigit{1}}{\isacharunderscore}{\kern0pt}fm{\isacharunderscore}{\kern0pt}def\isanewline
\ \ \ \ \ \ eq{\isacharunderscore}{\kern0pt}case{\isacharunderscore}{\kern0pt}fm{\isacharunderscore}{\kern0pt}def\ mem{\isacharunderscore}{\kern0pt}case{\isacharunderscore}{\kern0pt}fm{\isacharunderscore}{\kern0pt}def\ is{\isacharunderscore}{\kern0pt}tuple{\isacharunderscore}{\kern0pt}fm{\isacharunderscore}{\kern0pt}def\isanewline
\ \ \ \ \isacommand{by}\isamarkupfalse%
\ {\isacharparenleft}{\kern0pt}simp\ add{\isacharcolon}{\kern0pt}nat{\isacharunderscore}{\kern0pt}simp{\isacharunderscore}{\kern0pt}union{\isacharparenright}{\kern0pt}\isanewline
\ \ \isacommand{moreover}\isamarkupfalse%
\isanewline
\ \ \isacommand{have}\isamarkupfalse%
\ {\isachardoublequoteopen}{\isacharquery}{\kern0pt}f{\isasymin}formula{\isachardoublequoteclose}\isanewline
\ \ \ \ \isacommand{unfolding}\isamarkupfalse%
\ fm{\isacharunderscore}{\kern0pt}defs\ Hfrc{\isacharunderscore}{\kern0pt}at{\isacharunderscore}{\kern0pt}fm{\isacharunderscore}{\kern0pt}def\ \isacommand{by}\isamarkupfalse%
\ simp\isanewline
\ \ \isacommand{ultimately}\isamarkupfalse%
\isanewline
\ \ \isacommand{have}\isamarkupfalse%
\ {\isachardoublequoteopen}strong{\isacharunderscore}{\kern0pt}replacement{\isacharparenleft}{\kern0pt}{\isacharhash}{\kern0pt}{\isacharhash}{\kern0pt}M{\isacharcomma}{\kern0pt}{\isasymlambda}x\ z{\isachardot}{\kern0pt}\ sats{\isacharparenleft}{\kern0pt}M{\isacharcomma}{\kern0pt}{\isacharquery}{\kern0pt}f{\isacharcomma}{\kern0pt}{\isacharbrackleft}{\kern0pt}x{\isacharcomma}{\kern0pt}z{\isacharcomma}{\kern0pt}P{\isacharcomma}{\kern0pt}leq{\isacharcomma}{\kern0pt}forcerel{\isacharparenleft}{\kern0pt}P{\isacharcomma}{\kern0pt}X{\isacharparenright}{\kern0pt}{\isacharbrackright}{\kern0pt}{\isacharparenright}{\kern0pt}{\isacharparenright}{\kern0pt}{\isachardoublequoteclose}\isanewline
\ \ \ \ \isacommand{using}\isamarkupfalse%
\ replacement{\isacharunderscore}{\kern0pt}ax\ {\isadigit{1}}\ artyf\ {\isacartoucheopen}X{\isasymin}M{\isacartoucheclose}\ forcerel{\isacharunderscore}{\kern0pt}in{\isacharunderscore}{\kern0pt}M\ P{\isacharunderscore}{\kern0pt}in{\isacharunderscore}{\kern0pt}M\ leq{\isacharunderscore}{\kern0pt}in{\isacharunderscore}{\kern0pt}M\ \isacommand{by}\isamarkupfalse%
\ simp\isanewline
\ \ \isacommand{then}\isamarkupfalse%
\isanewline
\ \ \isacommand{have}\isamarkupfalse%
\ {\isachardoublequoteopen}strong{\isacharunderscore}{\kern0pt}replacement{\isacharparenleft}{\kern0pt}{\isacharhash}{\kern0pt}{\isacharhash}{\kern0pt}M{\isacharcomma}{\kern0pt}{\isasymlambda}x\ z{\isachardot}{\kern0pt}\isanewline
\ \ \ \ \ \ \ \ \ \ {\isasymexists}y{\isasymin}M{\isachardot}{\kern0pt}\ pair{\isacharparenleft}{\kern0pt}{\isacharhash}{\kern0pt}{\isacharhash}{\kern0pt}M{\isacharcomma}{\kern0pt}x{\isacharcomma}{\kern0pt}y{\isacharcomma}{\kern0pt}z{\isacharparenright}{\kern0pt}\ {\isacharampersand}{\kern0pt}\ is{\isacharunderscore}{\kern0pt}wfrec{\isacharparenleft}{\kern0pt}{\isacharhash}{\kern0pt}{\isacharhash}{\kern0pt}M{\isacharcomma}{\kern0pt}\ is{\isacharunderscore}{\kern0pt}Hfrc{\isacharunderscore}{\kern0pt}at{\isacharparenleft}{\kern0pt}{\isacharhash}{\kern0pt}{\isacharhash}{\kern0pt}M{\isacharcomma}{\kern0pt}P{\isacharcomma}{\kern0pt}leq{\isacharparenright}{\kern0pt}{\isacharcomma}{\kern0pt}forcerel{\isacharparenleft}{\kern0pt}P{\isacharcomma}{\kern0pt}X{\isacharparenright}{\kern0pt}{\isacharcomma}{\kern0pt}\ x{\isacharcomma}{\kern0pt}\ y{\isacharparenright}{\kern0pt}{\isacharparenright}{\kern0pt}{\isachardoublequoteclose}\isanewline
\ \ \ \ \isacommand{using}\isamarkupfalse%
\ repl{\isacharunderscore}{\kern0pt}sats{\isacharbrackleft}{\kern0pt}of\ M\ {\isacharquery}{\kern0pt}f\ {\isachardoublequoteopen}{\isacharbrackleft}{\kern0pt}P{\isacharcomma}{\kern0pt}leq{\isacharcomma}{\kern0pt}forcerel{\isacharparenleft}{\kern0pt}P{\isacharcomma}{\kern0pt}X{\isacharparenright}{\kern0pt}{\isacharbrackright}{\kern0pt}{\isachardoublequoteclose}{\isacharbrackright}{\kern0pt}\ satsf\ \isacommand{by}\isamarkupfalse%
\ {\isacharparenleft}{\kern0pt}simp\ del{\isacharcolon}{\kern0pt}pair{\isacharunderscore}{\kern0pt}abs{\isacharparenright}{\kern0pt}\isanewline
\ \ \isacommand{then}\isamarkupfalse%
\isanewline
\ \ \isacommand{show}\isamarkupfalse%
\ {\isacharquery}{\kern0pt}thesis\ \isacommand{unfolding}\isamarkupfalse%
\ wfrec{\isacharunderscore}{\kern0pt}replacement{\isacharunderscore}{\kern0pt}def\ \isacommand{by}\isamarkupfalse%
\ simp\isanewline
\isacommand{qed}\isamarkupfalse%
%
\endisatagproof
{\isafoldproof}%
%
\isadelimproof
\isanewline
%
\endisadelimproof
\isanewline
\isacommand{lemma}\isamarkupfalse%
\ names{\isacharunderscore}{\kern0pt}below{\isacharunderscore}{\kern0pt}abs\ {\isacharcolon}{\kern0pt}\isanewline
\ \ {\isachardoublequoteopen}{\isasymlbrakk}Q{\isasymin}M{\isacharsemicolon}{\kern0pt}x{\isasymin}M{\isacharsemicolon}{\kern0pt}nb{\isasymin}M{\isasymrbrakk}\ {\isasymLongrightarrow}\ is{\isacharunderscore}{\kern0pt}names{\isacharunderscore}{\kern0pt}below{\isacharparenleft}{\kern0pt}{\isacharhash}{\kern0pt}{\isacharhash}{\kern0pt}M{\isacharcomma}{\kern0pt}Q{\isacharcomma}{\kern0pt}x{\isacharcomma}{\kern0pt}nb{\isacharparenright}{\kern0pt}\ {\isasymlongleftrightarrow}\ nb\ {\isacharequal}{\kern0pt}\ names{\isacharunderscore}{\kern0pt}below{\isacharparenleft}{\kern0pt}Q{\isacharcomma}{\kern0pt}x{\isacharparenright}{\kern0pt}{\isachardoublequoteclose}\isanewline
%
\isadelimproof
\ \ %
\endisadelimproof
%
\isatagproof
\isacommand{unfolding}\isamarkupfalse%
\ is{\isacharunderscore}{\kern0pt}names{\isacharunderscore}{\kern0pt}below{\isacharunderscore}{\kern0pt}def\ names{\isacharunderscore}{\kern0pt}below{\isacharunderscore}{\kern0pt}def\isanewline
\ \ \isacommand{using}\isamarkupfalse%
\ succ{\isacharunderscore}{\kern0pt}in{\isacharunderscore}{\kern0pt}M{\isacharunderscore}{\kern0pt}iff\ zero{\isacharunderscore}{\kern0pt}in{\isacharunderscore}{\kern0pt}M\ cartprod{\isacharunderscore}{\kern0pt}closed\ is{\isacharunderscore}{\kern0pt}ecloseN{\isacharunderscore}{\kern0pt}abs\ ecloseN{\isacharunderscore}{\kern0pt}closed\isanewline
\ \ \isacommand{by}\isamarkupfalse%
\ auto%
\endisatagproof
{\isafoldproof}%
%
\isadelimproof
\isanewline
%
\endisadelimproof
\isanewline
\isacommand{lemma}\isamarkupfalse%
\ names{\isacharunderscore}{\kern0pt}below{\isacharunderscore}{\kern0pt}closed{\isacharcolon}{\kern0pt}\isanewline
\ \ {\isachardoublequoteopen}{\isasymlbrakk}Q{\isasymin}M{\isacharsemicolon}{\kern0pt}x{\isasymin}M{\isasymrbrakk}\ {\isasymLongrightarrow}\ names{\isacharunderscore}{\kern0pt}below{\isacharparenleft}{\kern0pt}Q{\isacharcomma}{\kern0pt}x{\isacharparenright}{\kern0pt}\ {\isasymin}\ M{\isachardoublequoteclose}\isanewline
%
\isadelimproof
\ \ %
\endisadelimproof
%
\isatagproof
\isacommand{unfolding}\isamarkupfalse%
\ names{\isacharunderscore}{\kern0pt}below{\isacharunderscore}{\kern0pt}def\isanewline
\ \ \isacommand{using}\isamarkupfalse%
\ zero{\isacharunderscore}{\kern0pt}in{\isacharunderscore}{\kern0pt}M\ cartprod{\isacharunderscore}{\kern0pt}closed\ ecloseN{\isacharunderscore}{\kern0pt}closed\ succ{\isacharunderscore}{\kern0pt}in{\isacharunderscore}{\kern0pt}M{\isacharunderscore}{\kern0pt}iff\isanewline
\ \ \isacommand{by}\isamarkupfalse%
\ simp%
\endisatagproof
{\isafoldproof}%
%
\isadelimproof
\isanewline
%
\endisadelimproof
\isanewline
\isacommand{lemma}\isamarkupfalse%
\ {\isachardoublequoteopen}names{\isacharunderscore}{\kern0pt}below{\isacharunderscore}{\kern0pt}productE{\isachardoublequoteclose}\ {\isacharcolon}{\kern0pt}\isanewline
\ \ \isakeyword{assumes}\ {\isachardoublequoteopen}Q\ {\isasymin}\ M{\isachardoublequoteclose}\ {\isachardoublequoteopen}x\ {\isasymin}\ M{\isachardoublequoteclose}\isanewline
\ \ \ \ {\isachardoublequoteopen}{\isasymAnd}A{\isadigit{1}}\ A{\isadigit{2}}\ A{\isadigit{3}}\ A{\isadigit{4}}{\isachardot}{\kern0pt}\ A{\isadigit{1}}\ {\isasymin}\ M\ {\isasymLongrightarrow}\ A{\isadigit{2}}\ {\isasymin}\ M\ {\isasymLongrightarrow}\ A{\isadigit{3}}\ {\isasymin}\ M\ {\isasymLongrightarrow}\ A{\isadigit{4}}\ {\isasymin}\ M\ {\isasymLongrightarrow}\ R{\isacharparenleft}{\kern0pt}A{\isadigit{1}}\ {\isasymtimes}\ A{\isadigit{2}}\ {\isasymtimes}\ A{\isadigit{3}}\ {\isasymtimes}\ A{\isadigit{4}}{\isacharparenright}{\kern0pt}{\isachardoublequoteclose}\isanewline
\ \ \isakeyword{shows}\ {\isachardoublequoteopen}R{\isacharparenleft}{\kern0pt}names{\isacharunderscore}{\kern0pt}below{\isacharparenleft}{\kern0pt}Q{\isacharcomma}{\kern0pt}x{\isacharparenright}{\kern0pt}{\isacharparenright}{\kern0pt}{\isachardoublequoteclose}\isanewline
%
\isadelimproof
\ \ %
\endisadelimproof
%
\isatagproof
\isacommand{unfolding}\isamarkupfalse%
\ names{\isacharunderscore}{\kern0pt}below{\isacharunderscore}{\kern0pt}def\ \isacommand{using}\isamarkupfalse%
\ assms\ zero{\isacharunderscore}{\kern0pt}in{\isacharunderscore}{\kern0pt}M\ ecloseN{\isacharunderscore}{\kern0pt}closed{\isacharbrackleft}{\kern0pt}of\ x{\isacharbrackright}{\kern0pt}\ twoN{\isacharunderscore}{\kern0pt}in{\isacharunderscore}{\kern0pt}M\ \isacommand{by}\isamarkupfalse%
\ auto%
\endisatagproof
{\isafoldproof}%
%
\isadelimproof
\isanewline
%
\endisadelimproof
\isanewline
\isacommand{lemma}\isamarkupfalse%
\ forcerel{\isacharunderscore}{\kern0pt}abs\ {\isacharcolon}{\kern0pt}\isanewline
\ \ {\isachardoublequoteopen}{\isasymlbrakk}x{\isasymin}M{\isacharsemicolon}{\kern0pt}z{\isasymin}M{\isasymrbrakk}\ {\isasymLongrightarrow}\ is{\isacharunderscore}{\kern0pt}forcerel{\isacharparenleft}{\kern0pt}{\isacharhash}{\kern0pt}{\isacharhash}{\kern0pt}M{\isacharcomma}{\kern0pt}P{\isacharcomma}{\kern0pt}x{\isacharcomma}{\kern0pt}z{\isacharparenright}{\kern0pt}\ {\isasymlongleftrightarrow}\ z\ {\isacharequal}{\kern0pt}\ forcerel{\isacharparenleft}{\kern0pt}P{\isacharcomma}{\kern0pt}x{\isacharparenright}{\kern0pt}{\isachardoublequoteclose}\isanewline
%
\isadelimproof
\ \ %
\endisadelimproof
%
\isatagproof
\isacommand{unfolding}\isamarkupfalse%
\ is{\isacharunderscore}{\kern0pt}forcerel{\isacharunderscore}{\kern0pt}def\ forcerel{\isacharunderscore}{\kern0pt}def\isanewline
\ \ \isacommand{using}\isamarkupfalse%
\ frecrel{\isacharunderscore}{\kern0pt}abs\ names{\isacharunderscore}{\kern0pt}below{\isacharunderscore}{\kern0pt}abs\ trancl{\isacharunderscore}{\kern0pt}abs\ P{\isacharunderscore}{\kern0pt}in{\isacharunderscore}{\kern0pt}M\ twoN{\isacharunderscore}{\kern0pt}in{\isacharunderscore}{\kern0pt}M\ ecloseN{\isacharunderscore}{\kern0pt}closed\ names{\isacharunderscore}{\kern0pt}below{\isacharunderscore}{\kern0pt}closed\isanewline
\ \ \ \ names{\isacharunderscore}{\kern0pt}below{\isacharunderscore}{\kern0pt}productE{\isacharbrackleft}{\kern0pt}of\ concl{\isacharcolon}{\kern0pt}{\isachardoublequoteopen}{\isasymlambda}p{\isachardot}{\kern0pt}\ is{\isacharunderscore}{\kern0pt}frecrel{\isacharparenleft}{\kern0pt}{\isacharhash}{\kern0pt}{\isacharhash}{\kern0pt}M{\isacharcomma}{\kern0pt}p{\isacharcomma}{\kern0pt}{\isacharunderscore}{\kern0pt}{\isacharparenright}{\kern0pt}\ {\isasymlongleftrightarrow}\ \ {\isacharunderscore}{\kern0pt}\ {\isacharequal}{\kern0pt}\ frecrel{\isacharparenleft}{\kern0pt}p{\isacharparenright}{\kern0pt}{\isachardoublequoteclose}{\isacharbrackright}{\kern0pt}\ frecrel{\isacharunderscore}{\kern0pt}closed\isanewline
\ \ \isacommand{by}\isamarkupfalse%
\ simp%
\endisatagproof
{\isafoldproof}%
%
\isadelimproof
\isanewline
%
\endisadelimproof
\isanewline
\isacommand{lemma}\isamarkupfalse%
\ frc{\isacharunderscore}{\kern0pt}at{\isacharunderscore}{\kern0pt}abs{\isacharcolon}{\kern0pt}\isanewline
\ \ \isakeyword{assumes}\ {\isachardoublequoteopen}fnnc{\isasymin}M{\isachardoublequoteclose}\ {\isachardoublequoteopen}z{\isasymin}M{\isachardoublequoteclose}\isanewline
\ \ \isakeyword{shows}\ {\isachardoublequoteopen}is{\isacharunderscore}{\kern0pt}frc{\isacharunderscore}{\kern0pt}at{\isacharparenleft}{\kern0pt}{\isacharhash}{\kern0pt}{\isacharhash}{\kern0pt}M{\isacharcomma}{\kern0pt}P{\isacharcomma}{\kern0pt}leq{\isacharcomma}{\kern0pt}fnnc{\isacharcomma}{\kern0pt}z{\isacharparenright}{\kern0pt}\ {\isasymlongleftrightarrow}\ z\ {\isacharequal}{\kern0pt}\ frc{\isacharunderscore}{\kern0pt}at{\isacharparenleft}{\kern0pt}P{\isacharcomma}{\kern0pt}leq{\isacharcomma}{\kern0pt}fnnc{\isacharparenright}{\kern0pt}{\isachardoublequoteclose}\isanewline
%
\isadelimproof
%
\endisadelimproof
%
\isatagproof
\isacommand{proof}\isamarkupfalse%
\ {\isacharminus}{\kern0pt}\isanewline
\ \ \isacommand{from}\isamarkupfalse%
\ assms\isanewline
\ \ \isacommand{have}\isamarkupfalse%
\ {\isachardoublequoteopen}{\isacharparenleft}{\kern0pt}{\isasymexists}r{\isasymin}M{\isachardot}{\kern0pt}\ is{\isacharunderscore}{\kern0pt}forcerel{\isacharparenleft}{\kern0pt}{\isacharhash}{\kern0pt}{\isacharhash}{\kern0pt}M{\isacharcomma}{\kern0pt}P{\isacharcomma}{\kern0pt}fnnc{\isacharcomma}{\kern0pt}\ r{\isacharparenright}{\kern0pt}\ {\isasymand}\ is{\isacharunderscore}{\kern0pt}wfrec{\isacharparenleft}{\kern0pt}{\isacharhash}{\kern0pt}{\isacharhash}{\kern0pt}M{\isacharcomma}{\kern0pt}\ is{\isacharunderscore}{\kern0pt}Hfrc{\isacharunderscore}{\kern0pt}at{\isacharparenleft}{\kern0pt}{\isacharhash}{\kern0pt}{\isacharhash}{\kern0pt}M{\isacharcomma}{\kern0pt}\ P{\isacharcomma}{\kern0pt}\ leq{\isacharparenright}{\kern0pt}{\isacharcomma}{\kern0pt}\ r{\isacharcomma}{\kern0pt}\ fnnc{\isacharcomma}{\kern0pt}\ z{\isacharparenright}{\kern0pt}{\isacharparenright}{\kern0pt}\isanewline
\ \ \ \ \ \ \ \ {\isasymlongleftrightarrow}\ is{\isacharunderscore}{\kern0pt}wfrec{\isacharparenleft}{\kern0pt}{\isacharhash}{\kern0pt}{\isacharhash}{\kern0pt}M{\isacharcomma}{\kern0pt}\ is{\isacharunderscore}{\kern0pt}Hfrc{\isacharunderscore}{\kern0pt}at{\isacharparenleft}{\kern0pt}{\isacharhash}{\kern0pt}{\isacharhash}{\kern0pt}M{\isacharcomma}{\kern0pt}\ P{\isacharcomma}{\kern0pt}\ leq{\isacharparenright}{\kern0pt}{\isacharcomma}{\kern0pt}\ forcerel{\isacharparenleft}{\kern0pt}P{\isacharcomma}{\kern0pt}fnnc{\isacharparenright}{\kern0pt}{\isacharcomma}{\kern0pt}\ fnnc{\isacharcomma}{\kern0pt}\ z{\isacharparenright}{\kern0pt}{\isachardoublequoteclose}\isanewline
\ \ \ \ \isacommand{using}\isamarkupfalse%
\ forcerel{\isacharunderscore}{\kern0pt}abs\ forcerel{\isacharunderscore}{\kern0pt}in{\isacharunderscore}{\kern0pt}M\ \isacommand{by}\isamarkupfalse%
\ simp\isanewline
\ \ \isacommand{then}\isamarkupfalse%
\isanewline
\ \ \isacommand{show}\isamarkupfalse%
\ {\isacharquery}{\kern0pt}thesis\isanewline
\ \ \ \ \isacommand{unfolding}\isamarkupfalse%
\ frc{\isacharunderscore}{\kern0pt}at{\isacharunderscore}{\kern0pt}trancl\ is{\isacharunderscore}{\kern0pt}frc{\isacharunderscore}{\kern0pt}at{\isacharunderscore}{\kern0pt}def\isanewline
\ \ \ \ \isacommand{using}\isamarkupfalse%
\ assms\ wfrec{\isacharunderscore}{\kern0pt}Hfrc{\isacharunderscore}{\kern0pt}at{\isacharbrackleft}{\kern0pt}of\ fnnc{\isacharbrackright}{\kern0pt}\ wf{\isacharunderscore}{\kern0pt}forcerel\ trans{\isacharunderscore}{\kern0pt}forcerel{\isacharunderscore}{\kern0pt}t\ relation{\isacharunderscore}{\kern0pt}forcerel{\isacharunderscore}{\kern0pt}t\ forcerel{\isacharunderscore}{\kern0pt}in{\isacharunderscore}{\kern0pt}M\isanewline
\ \ \ \ \ \ Hfrc{\isacharunderscore}{\kern0pt}at{\isacharunderscore}{\kern0pt}closed\ relation{\isadigit{2}}{\isacharunderscore}{\kern0pt}Hfrc{\isacharunderscore}{\kern0pt}at{\isacharunderscore}{\kern0pt}abs\isanewline
\ \ \ \ \ \ trans{\isacharunderscore}{\kern0pt}wfrec{\isacharunderscore}{\kern0pt}abs{\isacharbrackleft}{\kern0pt}of\ {\isachardoublequoteopen}forcerel{\isacharparenleft}{\kern0pt}P{\isacharcomma}{\kern0pt}fnnc{\isacharparenright}{\kern0pt}{\isachardoublequoteclose}\ fnnc\ z\ {\isachardoublequoteopen}is{\isacharunderscore}{\kern0pt}Hfrc{\isacharunderscore}{\kern0pt}at{\isacharparenleft}{\kern0pt}{\isacharhash}{\kern0pt}{\isacharhash}{\kern0pt}M{\isacharcomma}{\kern0pt}P{\isacharcomma}{\kern0pt}leq{\isacharparenright}{\kern0pt}{\isachardoublequoteclose}\ {\isachardoublequoteopen}{\isasymlambda}x\ f{\isachardot}{\kern0pt}\ bool{\isacharunderscore}{\kern0pt}of{\isacharunderscore}{\kern0pt}o{\isacharparenleft}{\kern0pt}Hfrc{\isacharparenleft}{\kern0pt}P{\isacharcomma}{\kern0pt}leq{\isacharcomma}{\kern0pt}x{\isacharcomma}{\kern0pt}f{\isacharparenright}{\kern0pt}{\isacharparenright}{\kern0pt}{\isachardoublequoteclose}{\isacharbrackright}{\kern0pt}\isanewline
\ \ \ \ \isacommand{by}\isamarkupfalse%
\ {\isacharparenleft}{\kern0pt}simp\ flip{\isacharcolon}{\kern0pt}setclass{\isacharunderscore}{\kern0pt}iff{\isacharparenright}{\kern0pt}\isanewline
\isacommand{qed}\isamarkupfalse%
%
\endisatagproof
{\isafoldproof}%
%
\isadelimproof
\isanewline
%
\endisadelimproof
\isanewline
\isacommand{lemma}\isamarkupfalse%
\ forces{\isacharunderscore}{\kern0pt}eq{\isacharprime}{\kern0pt}{\isacharunderscore}{\kern0pt}abs\ {\isacharcolon}{\kern0pt}\isanewline
\ \ {\isachardoublequoteopen}{\isasymlbrakk}p{\isasymin}M\ {\isacharsemicolon}{\kern0pt}\ t{\isadigit{1}}{\isasymin}M\ {\isacharsemicolon}{\kern0pt}\ t{\isadigit{2}}{\isasymin}M{\isasymrbrakk}\ {\isasymLongrightarrow}\ is{\isacharunderscore}{\kern0pt}forces{\isacharunderscore}{\kern0pt}eq{\isacharprime}{\kern0pt}{\isacharparenleft}{\kern0pt}{\isacharhash}{\kern0pt}{\isacharhash}{\kern0pt}M{\isacharcomma}{\kern0pt}P{\isacharcomma}{\kern0pt}leq{\isacharcomma}{\kern0pt}p{\isacharcomma}{\kern0pt}t{\isadigit{1}}{\isacharcomma}{\kern0pt}t{\isadigit{2}}{\isacharparenright}{\kern0pt}\ {\isasymlongleftrightarrow}\ forces{\isacharunderscore}{\kern0pt}eq{\isacharprime}{\kern0pt}{\isacharparenleft}{\kern0pt}P{\isacharcomma}{\kern0pt}leq{\isacharcomma}{\kern0pt}p{\isacharcomma}{\kern0pt}t{\isadigit{1}}{\isacharcomma}{\kern0pt}t{\isadigit{2}}{\isacharparenright}{\kern0pt}{\isachardoublequoteclose}\isanewline
%
\isadelimproof
\ \ %
\endisadelimproof
%
\isatagproof
\isacommand{unfolding}\isamarkupfalse%
\ is{\isacharunderscore}{\kern0pt}forces{\isacharunderscore}{\kern0pt}eq{\isacharprime}{\kern0pt}{\isacharunderscore}{\kern0pt}def\ forces{\isacharunderscore}{\kern0pt}eq{\isacharprime}{\kern0pt}{\isacharunderscore}{\kern0pt}def\isanewline
\ \ \isacommand{using}\isamarkupfalse%
\ frc{\isacharunderscore}{\kern0pt}at{\isacharunderscore}{\kern0pt}abs\ zero{\isacharunderscore}{\kern0pt}in{\isacharunderscore}{\kern0pt}M\ tuples{\isacharunderscore}{\kern0pt}in{\isacharunderscore}{\kern0pt}M\ \isacommand{by}\isamarkupfalse%
\ auto%
\endisatagproof
{\isafoldproof}%
%
\isadelimproof
\isanewline
%
\endisadelimproof
\isanewline
\isacommand{lemma}\isamarkupfalse%
\ forces{\isacharunderscore}{\kern0pt}mem{\isacharprime}{\kern0pt}{\isacharunderscore}{\kern0pt}abs\ {\isacharcolon}{\kern0pt}\isanewline
\ \ {\isachardoublequoteopen}{\isasymlbrakk}p{\isasymin}M\ {\isacharsemicolon}{\kern0pt}\ t{\isadigit{1}}{\isasymin}M\ {\isacharsemicolon}{\kern0pt}\ t{\isadigit{2}}{\isasymin}M{\isasymrbrakk}\ {\isasymLongrightarrow}\ is{\isacharunderscore}{\kern0pt}forces{\isacharunderscore}{\kern0pt}mem{\isacharprime}{\kern0pt}{\isacharparenleft}{\kern0pt}{\isacharhash}{\kern0pt}{\isacharhash}{\kern0pt}M{\isacharcomma}{\kern0pt}P{\isacharcomma}{\kern0pt}leq{\isacharcomma}{\kern0pt}p{\isacharcomma}{\kern0pt}t{\isadigit{1}}{\isacharcomma}{\kern0pt}t{\isadigit{2}}{\isacharparenright}{\kern0pt}\ {\isasymlongleftrightarrow}\ forces{\isacharunderscore}{\kern0pt}mem{\isacharprime}{\kern0pt}{\isacharparenleft}{\kern0pt}P{\isacharcomma}{\kern0pt}leq{\isacharcomma}{\kern0pt}p{\isacharcomma}{\kern0pt}t{\isadigit{1}}{\isacharcomma}{\kern0pt}t{\isadigit{2}}{\isacharparenright}{\kern0pt}{\isachardoublequoteclose}\isanewline
%
\isadelimproof
\ \ %
\endisadelimproof
%
\isatagproof
\isacommand{unfolding}\isamarkupfalse%
\ is{\isacharunderscore}{\kern0pt}forces{\isacharunderscore}{\kern0pt}mem{\isacharprime}{\kern0pt}{\isacharunderscore}{\kern0pt}def\ forces{\isacharunderscore}{\kern0pt}mem{\isacharprime}{\kern0pt}{\isacharunderscore}{\kern0pt}def\isanewline
\ \ \isacommand{using}\isamarkupfalse%
\ frc{\isacharunderscore}{\kern0pt}at{\isacharunderscore}{\kern0pt}abs\ zero{\isacharunderscore}{\kern0pt}in{\isacharunderscore}{\kern0pt}M\ tuples{\isacharunderscore}{\kern0pt}in{\isacharunderscore}{\kern0pt}M\ \isacommand{by}\isamarkupfalse%
\ auto%
\endisatagproof
{\isafoldproof}%
%
\isadelimproof
\isanewline
%
\endisadelimproof
\isanewline
\isacommand{lemma}\isamarkupfalse%
\ forces{\isacharunderscore}{\kern0pt}neq{\isacharprime}{\kern0pt}{\isacharunderscore}{\kern0pt}abs\ {\isacharcolon}{\kern0pt}\isanewline
\ \ \isakeyword{assumes}\isanewline
\ \ \ \ {\isachardoublequoteopen}p{\isasymin}M{\isachardoublequoteclose}\ {\isachardoublequoteopen}t{\isadigit{1}}{\isasymin}M{\isachardoublequoteclose}\ {\isachardoublequoteopen}t{\isadigit{2}}{\isasymin}M{\isachardoublequoteclose}\isanewline
\ \ \isakeyword{shows}\isanewline
\ \ \ \ {\isachardoublequoteopen}is{\isacharunderscore}{\kern0pt}forces{\isacharunderscore}{\kern0pt}neq{\isacharprime}{\kern0pt}{\isacharparenleft}{\kern0pt}{\isacharhash}{\kern0pt}{\isacharhash}{\kern0pt}M{\isacharcomma}{\kern0pt}P{\isacharcomma}{\kern0pt}leq{\isacharcomma}{\kern0pt}p{\isacharcomma}{\kern0pt}t{\isadigit{1}}{\isacharcomma}{\kern0pt}t{\isadigit{2}}{\isacharparenright}{\kern0pt}\ {\isasymlongleftrightarrow}\ forces{\isacharunderscore}{\kern0pt}neq{\isacharprime}{\kern0pt}{\isacharparenleft}{\kern0pt}P{\isacharcomma}{\kern0pt}leq{\isacharcomma}{\kern0pt}p{\isacharcomma}{\kern0pt}t{\isadigit{1}}{\isacharcomma}{\kern0pt}t{\isadigit{2}}{\isacharparenright}{\kern0pt}{\isachardoublequoteclose}\isanewline
%
\isadelimproof
%
\endisadelimproof
%
\isatagproof
\isacommand{proof}\isamarkupfalse%
\ {\isacharminus}{\kern0pt}\isanewline
\ \ \isacommand{have}\isamarkupfalse%
\ {\isachardoublequoteopen}q{\isasymin}M{\isachardoublequoteclose}\ \isakeyword{if}\ {\isachardoublequoteopen}q{\isasymin}P{\isachardoublequoteclose}\ \isakeyword{for}\ q\isanewline
\ \ \ \ \isacommand{using}\isamarkupfalse%
\ that\ transitivity\ P{\isacharunderscore}{\kern0pt}in{\isacharunderscore}{\kern0pt}M\ \isacommand{by}\isamarkupfalse%
\ simp\isanewline
\ \ \isacommand{then}\isamarkupfalse%
\ \isacommand{show}\isamarkupfalse%
\ {\isacharquery}{\kern0pt}thesis\isanewline
\ \ \ \ \isacommand{unfolding}\isamarkupfalse%
\ is{\isacharunderscore}{\kern0pt}forces{\isacharunderscore}{\kern0pt}neq{\isacharprime}{\kern0pt}{\isacharunderscore}{\kern0pt}def\ forces{\isacharunderscore}{\kern0pt}neq{\isacharprime}{\kern0pt}{\isacharunderscore}{\kern0pt}def\isanewline
\ \ \ \ \isacommand{using}\isamarkupfalse%
\ assms\ forces{\isacharunderscore}{\kern0pt}eq{\isacharprime}{\kern0pt}{\isacharunderscore}{\kern0pt}abs\ pair{\isacharunderscore}{\kern0pt}in{\isacharunderscore}{\kern0pt}M{\isacharunderscore}{\kern0pt}iff\isanewline
\ \ \ \ \isacommand{by}\isamarkupfalse%
\ {\isacharparenleft}{\kern0pt}auto{\isacharcomma}{\kern0pt}blast{\isacharparenright}{\kern0pt}\isanewline
\isacommand{qed}\isamarkupfalse%
%
\endisatagproof
{\isafoldproof}%
%
\isadelimproof
\isanewline
%
\endisadelimproof
\isanewline
\isanewline
\isacommand{lemma}\isamarkupfalse%
\ forces{\isacharunderscore}{\kern0pt}nmem{\isacharprime}{\kern0pt}{\isacharunderscore}{\kern0pt}abs\ {\isacharcolon}{\kern0pt}\isanewline
\ \ \isakeyword{assumes}\isanewline
\ \ \ \ {\isachardoublequoteopen}p{\isasymin}M{\isachardoublequoteclose}\ {\isachardoublequoteopen}t{\isadigit{1}}{\isasymin}M{\isachardoublequoteclose}\ {\isachardoublequoteopen}t{\isadigit{2}}{\isasymin}M{\isachardoublequoteclose}\isanewline
\ \ \isakeyword{shows}\isanewline
\ \ \ \ {\isachardoublequoteopen}is{\isacharunderscore}{\kern0pt}forces{\isacharunderscore}{\kern0pt}nmem{\isacharprime}{\kern0pt}{\isacharparenleft}{\kern0pt}{\isacharhash}{\kern0pt}{\isacharhash}{\kern0pt}M{\isacharcomma}{\kern0pt}P{\isacharcomma}{\kern0pt}leq{\isacharcomma}{\kern0pt}p{\isacharcomma}{\kern0pt}t{\isadigit{1}}{\isacharcomma}{\kern0pt}t{\isadigit{2}}{\isacharparenright}{\kern0pt}\ {\isasymlongleftrightarrow}\ forces{\isacharunderscore}{\kern0pt}nmem{\isacharprime}{\kern0pt}{\isacharparenleft}{\kern0pt}P{\isacharcomma}{\kern0pt}leq{\isacharcomma}{\kern0pt}p{\isacharcomma}{\kern0pt}t{\isadigit{1}}{\isacharcomma}{\kern0pt}t{\isadigit{2}}{\isacharparenright}{\kern0pt}{\isachardoublequoteclose}\isanewline
%
\isadelimproof
%
\endisadelimproof
%
\isatagproof
\isacommand{proof}\isamarkupfalse%
\ {\isacharminus}{\kern0pt}\isanewline
\ \ \isacommand{have}\isamarkupfalse%
\ {\isachardoublequoteopen}q{\isasymin}M{\isachardoublequoteclose}\ \isakeyword{if}\ {\isachardoublequoteopen}q{\isasymin}P{\isachardoublequoteclose}\ \isakeyword{for}\ q\isanewline
\ \ \ \ \isacommand{using}\isamarkupfalse%
\ that\ transitivity\ P{\isacharunderscore}{\kern0pt}in{\isacharunderscore}{\kern0pt}M\ \isacommand{by}\isamarkupfalse%
\ simp\isanewline
\ \ \isacommand{then}\isamarkupfalse%
\ \isacommand{show}\isamarkupfalse%
\ {\isacharquery}{\kern0pt}thesis\isanewline
\ \ \ \ \isacommand{unfolding}\isamarkupfalse%
\ is{\isacharunderscore}{\kern0pt}forces{\isacharunderscore}{\kern0pt}nmem{\isacharprime}{\kern0pt}{\isacharunderscore}{\kern0pt}def\ forces{\isacharunderscore}{\kern0pt}nmem{\isacharprime}{\kern0pt}{\isacharunderscore}{\kern0pt}def\isanewline
\ \ \ \ \isacommand{using}\isamarkupfalse%
\ assms\ forces{\isacharunderscore}{\kern0pt}mem{\isacharprime}{\kern0pt}{\isacharunderscore}{\kern0pt}abs\ pair{\isacharunderscore}{\kern0pt}in{\isacharunderscore}{\kern0pt}M{\isacharunderscore}{\kern0pt}iff\isanewline
\ \ \ \ \isacommand{by}\isamarkupfalse%
\ {\isacharparenleft}{\kern0pt}auto{\isacharcomma}{\kern0pt}blast{\isacharparenright}{\kern0pt}\isanewline
\isacommand{qed}\isamarkupfalse%
%
\endisatagproof
{\isafoldproof}%
%
\isadelimproof
\isanewline
%
\endisadelimproof
\isanewline
\isacommand{end}\isamarkupfalse%
%
\isadelimdocument
%
\endisadelimdocument
%
\isatagdocument
%
\isamarkupsubsection{Forcing for general formulas%
}
\isamarkuptrue%
%
\endisatagdocument
{\isafolddocument}%
%
\isadelimdocument
%
\endisadelimdocument
\isacommand{definition}\isamarkupfalse%
\isanewline
\ \ ren{\isacharunderscore}{\kern0pt}forces{\isacharunderscore}{\kern0pt}nand\ {\isacharcolon}{\kern0pt}{\isacharcolon}{\kern0pt}\ {\isachardoublequoteopen}i{\isasymRightarrow}i{\isachardoublequoteclose}\ \isakeyword{where}\isanewline
\ \ {\isachardoublequoteopen}ren{\isacharunderscore}{\kern0pt}forces{\isacharunderscore}{\kern0pt}nand{\isacharparenleft}{\kern0pt}{\isasymphi}{\isacharparenright}{\kern0pt}\ {\isasymequiv}\ Exists{\isacharparenleft}{\kern0pt}And{\isacharparenleft}{\kern0pt}Equal{\isacharparenleft}{\kern0pt}{\isadigit{0}}{\isacharcomma}{\kern0pt}{\isadigit{1}}{\isacharparenright}{\kern0pt}{\isacharcomma}{\kern0pt}iterates{\isacharparenleft}{\kern0pt}{\isasymlambda}p{\isachardot}{\kern0pt}\ incr{\isacharunderscore}{\kern0pt}bv{\isacharparenleft}{\kern0pt}p{\isacharparenright}{\kern0pt}{\isacharbackquote}{\kern0pt}{\isadigit{1}}\ {\isacharcomma}{\kern0pt}\ {\isadigit{2}}{\isacharcomma}{\kern0pt}\ {\isasymphi}{\isacharparenright}{\kern0pt}{\isacharparenright}{\kern0pt}{\isacharparenright}{\kern0pt}{\isachardoublequoteclose}\isanewline
\isanewline
\isacommand{lemma}\isamarkupfalse%
\ ren{\isacharunderscore}{\kern0pt}forces{\isacharunderscore}{\kern0pt}nand{\isacharunderscore}{\kern0pt}type{\isacharbrackleft}{\kern0pt}TC{\isacharbrackright}{\kern0pt}\ {\isacharcolon}{\kern0pt}\isanewline
\ \ {\isachardoublequoteopen}{\isasymphi}{\isasymin}formula\ {\isasymLongrightarrow}\ ren{\isacharunderscore}{\kern0pt}forces{\isacharunderscore}{\kern0pt}nand{\isacharparenleft}{\kern0pt}{\isasymphi}{\isacharparenright}{\kern0pt}\ {\isasymin}formula{\isachardoublequoteclose}\isanewline
%
\isadelimproof
\ \ %
\endisadelimproof
%
\isatagproof
\isacommand{unfolding}\isamarkupfalse%
\ ren{\isacharunderscore}{\kern0pt}forces{\isacharunderscore}{\kern0pt}nand{\isacharunderscore}{\kern0pt}def\isanewline
\ \ \isacommand{by}\isamarkupfalse%
\ simp%
\endisatagproof
{\isafoldproof}%
%
\isadelimproof
\isanewline
%
\endisadelimproof
\isanewline
\isacommand{lemma}\isamarkupfalse%
\ arity{\isacharunderscore}{\kern0pt}ren{\isacharunderscore}{\kern0pt}forces{\isacharunderscore}{\kern0pt}nand\ {\isacharcolon}{\kern0pt}\isanewline
\ \ \isakeyword{assumes}\ {\isachardoublequoteopen}{\isasymphi}{\isasymin}formula{\isachardoublequoteclose}\isanewline
\ \ \isakeyword{shows}\ {\isachardoublequoteopen}arity{\isacharparenleft}{\kern0pt}ren{\isacharunderscore}{\kern0pt}forces{\isacharunderscore}{\kern0pt}nand{\isacharparenleft}{\kern0pt}{\isasymphi}{\isacharparenright}{\kern0pt}{\isacharparenright}{\kern0pt}\ {\isasymle}\ succ{\isacharparenleft}{\kern0pt}arity{\isacharparenleft}{\kern0pt}{\isasymphi}{\isacharparenright}{\kern0pt}{\isacharparenright}{\kern0pt}{\isachardoublequoteclose}\isanewline
%
\isadelimproof
%
\endisadelimproof
%
\isatagproof
\isacommand{proof}\isamarkupfalse%
\ {\isacharminus}{\kern0pt}\isanewline
\ \ \isacommand{consider}\isamarkupfalse%
\ {\isacharparenleft}{\kern0pt}lt{\isacharparenright}{\kern0pt}\ {\isachardoublequoteopen}{\isadigit{1}}{\isacharless}{\kern0pt}arity{\isacharparenleft}{\kern0pt}{\isasymphi}{\isacharparenright}{\kern0pt}{\isachardoublequoteclose}\ {\isacharbar}{\kern0pt}\ {\isacharparenleft}{\kern0pt}ge{\isacharparenright}{\kern0pt}\ {\isachardoublequoteopen}{\isasymnot}\ {\isadigit{1}}\ {\isacharless}{\kern0pt}\ arity{\isacharparenleft}{\kern0pt}{\isasymphi}{\isacharparenright}{\kern0pt}{\isachardoublequoteclose}\isanewline
\ \ \ \ \isacommand{by}\isamarkupfalse%
\ auto\isanewline
\ \ \isacommand{then}\isamarkupfalse%
\isanewline
\ \ \isacommand{show}\isamarkupfalse%
\ {\isacharquery}{\kern0pt}thesis\isanewline
\ \ \isacommand{proof}\isamarkupfalse%
\ cases\isanewline
\ \ \ \ \isacommand{case}\isamarkupfalse%
\ lt\isanewline
\ \ \ \ \isacommand{with}\isamarkupfalse%
\ {\isacartoucheopen}{\isasymphi}{\isasymin}{\isacharunderscore}{\kern0pt}{\isacartoucheclose}\isanewline
\ \ \ \ \isacommand{have}\isamarkupfalse%
\ {\isachardoublequoteopen}{\isadigit{2}}\ {\isacharless}{\kern0pt}\ succ{\isacharparenleft}{\kern0pt}arity{\isacharparenleft}{\kern0pt}{\isasymphi}{\isacharparenright}{\kern0pt}{\isacharparenright}{\kern0pt}{\isachardoublequoteclose}\ {\isachardoublequoteopen}{\isadigit{2}}{\isacharless}{\kern0pt}arity{\isacharparenleft}{\kern0pt}{\isasymphi}{\isacharparenright}{\kern0pt}{\isacharhash}{\kern0pt}{\isacharplus}{\kern0pt}{\isadigit{2}}{\isachardoublequoteclose}\isanewline
\ \ \ \ \ \ \isacommand{using}\isamarkupfalse%
\ succ{\isacharunderscore}{\kern0pt}ltI\ \isacommand{by}\isamarkupfalse%
\ auto\isanewline
\ \ \ \ \isacommand{with}\isamarkupfalse%
\ {\isacartoucheopen}{\isasymphi}{\isasymin}{\isacharunderscore}{\kern0pt}{\isacartoucheclose}\isanewline
\ \ \ \ \isacommand{have}\isamarkupfalse%
\ {\isachardoublequoteopen}arity{\isacharparenleft}{\kern0pt}iterates{\isacharparenleft}{\kern0pt}{\isasymlambda}p{\isachardot}{\kern0pt}\ incr{\isacharunderscore}{\kern0pt}bv{\isacharparenleft}{\kern0pt}p{\isacharparenright}{\kern0pt}{\isacharbackquote}{\kern0pt}{\isadigit{1}}{\isacharcomma}{\kern0pt}{\isadigit{2}}{\isacharcomma}{\kern0pt}{\isasymphi}{\isacharparenright}{\kern0pt}{\isacharparenright}{\kern0pt}\ {\isacharequal}{\kern0pt}\ {\isadigit{2}}{\isacharhash}{\kern0pt}{\isacharplus}{\kern0pt}arity{\isacharparenleft}{\kern0pt}{\isasymphi}{\isacharparenright}{\kern0pt}{\isachardoublequoteclose}\isanewline
\ \ \ \ \ \ \isacommand{using}\isamarkupfalse%
\ arity{\isacharunderscore}{\kern0pt}incr{\isacharunderscore}{\kern0pt}bv{\isacharunderscore}{\kern0pt}lemma\ lt\isanewline
\ \ \ \ \ \ \isacommand{by}\isamarkupfalse%
\ auto\isanewline
\ \ \ \ \isacommand{with}\isamarkupfalse%
\ {\isacartoucheopen}{\isasymphi}{\isasymin}{\isacharunderscore}{\kern0pt}{\isacartoucheclose}\isanewline
\ \ \ \ \isacommand{show}\isamarkupfalse%
\ {\isacharquery}{\kern0pt}thesis\isanewline
\ \ \ \ \ \ \isacommand{unfolding}\isamarkupfalse%
\ ren{\isacharunderscore}{\kern0pt}forces{\isacharunderscore}{\kern0pt}nand{\isacharunderscore}{\kern0pt}def\isanewline
\ \ \ \ \ \ \isacommand{using}\isamarkupfalse%
\ lt\ pred{\isacharunderscore}{\kern0pt}Un{\isacharunderscore}{\kern0pt}distrib\ nat{\isacharunderscore}{\kern0pt}union{\isacharunderscore}{\kern0pt}abs{\isadigit{1}}\ Un{\isacharunderscore}{\kern0pt}assoc{\isacharbrackleft}{\kern0pt}symmetric{\isacharbrackright}{\kern0pt}\ Un{\isacharunderscore}{\kern0pt}le{\isacharunderscore}{\kern0pt}compat\isanewline
\ \ \ \ \ \ \isacommand{by}\isamarkupfalse%
\ simp\isanewline
\ \ \isacommand{next}\isamarkupfalse%
\isanewline
\ \ \ \ \isacommand{case}\isamarkupfalse%
\ ge\isanewline
\ \ \ \ \isacommand{with}\isamarkupfalse%
\ {\isacartoucheopen}{\isasymphi}{\isasymin}{\isacharunderscore}{\kern0pt}{\isacartoucheclose}\isanewline
\ \ \ \ \isacommand{have}\isamarkupfalse%
\ {\isachardoublequoteopen}arity{\isacharparenleft}{\kern0pt}{\isasymphi}{\isacharparenright}{\kern0pt}\ {\isasymle}\ {\isadigit{1}}{\isachardoublequoteclose}\ {\isachardoublequoteopen}pred{\isacharparenleft}{\kern0pt}arity{\isacharparenleft}{\kern0pt}{\isasymphi}{\isacharparenright}{\kern0pt}{\isacharparenright}{\kern0pt}\ {\isasymle}\ {\isadigit{1}}{\isachardoublequoteclose}\isanewline
\ \ \ \ \ \ \isacommand{using}\isamarkupfalse%
\ not{\isacharunderscore}{\kern0pt}lt{\isacharunderscore}{\kern0pt}iff{\isacharunderscore}{\kern0pt}le\ le{\isacharunderscore}{\kern0pt}trans{\isacharbrackleft}{\kern0pt}OF\ le{\isacharunderscore}{\kern0pt}pred{\isacharbrackright}{\kern0pt}\isanewline
\ \ \ \ \ \ \isacommand{by}\isamarkupfalse%
\ simp{\isacharunderscore}{\kern0pt}all\isanewline
\ \ \ \ \isacommand{with}\isamarkupfalse%
\ {\isacartoucheopen}{\isasymphi}{\isasymin}{\isacharunderscore}{\kern0pt}{\isacartoucheclose}\isanewline
\ \ \ \ \isacommand{have}\isamarkupfalse%
\ {\isachardoublequoteopen}arity{\isacharparenleft}{\kern0pt}iterates{\isacharparenleft}{\kern0pt}{\isasymlambda}p{\isachardot}{\kern0pt}\ incr{\isacharunderscore}{\kern0pt}bv{\isacharparenleft}{\kern0pt}p{\isacharparenright}{\kern0pt}{\isacharbackquote}{\kern0pt}{\isadigit{1}}{\isacharcomma}{\kern0pt}{\isadigit{2}}{\isacharcomma}{\kern0pt}{\isasymphi}{\isacharparenright}{\kern0pt}{\isacharparenright}{\kern0pt}\ {\isacharequal}{\kern0pt}\ {\isacharparenleft}{\kern0pt}arity{\isacharparenleft}{\kern0pt}{\isasymphi}{\isacharparenright}{\kern0pt}{\isacharparenright}{\kern0pt}{\isachardoublequoteclose}\isanewline
\ \ \ \ \ \ \isacommand{using}\isamarkupfalse%
\ arity{\isacharunderscore}{\kern0pt}incr{\isacharunderscore}{\kern0pt}bv{\isacharunderscore}{\kern0pt}lemma\ ge\isanewline
\ \ \ \ \ \ \isacommand{by}\isamarkupfalse%
\ simp\isanewline
\ \ \ \ \isacommand{with}\isamarkupfalse%
\ {\isacartoucheopen}arity{\isacharparenleft}{\kern0pt}{\isasymphi}{\isacharparenright}{\kern0pt}\ {\isasymle}\ {\isadigit{1}}{\isacartoucheclose}\ {\isacartoucheopen}{\isasymphi}{\isasymin}{\isacharunderscore}{\kern0pt}{\isacartoucheclose}\ {\isacartoucheopen}pred{\isacharparenleft}{\kern0pt}{\isacharunderscore}{\kern0pt}{\isacharparenright}{\kern0pt}\ {\isasymle}\ {\isadigit{1}}{\isacartoucheclose}\isanewline
\ \ \ \ \isacommand{show}\isamarkupfalse%
\ {\isacharquery}{\kern0pt}thesis\isanewline
\ \ \ \ \ \ \isacommand{unfolding}\isamarkupfalse%
\ ren{\isacharunderscore}{\kern0pt}forces{\isacharunderscore}{\kern0pt}nand{\isacharunderscore}{\kern0pt}def\isanewline
\ \ \ \ \ \ \isacommand{using}\isamarkupfalse%
\ \ pred{\isacharunderscore}{\kern0pt}Un{\isacharunderscore}{\kern0pt}distrib\ nat{\isacharunderscore}{\kern0pt}union{\isacharunderscore}{\kern0pt}abs{\isadigit{1}}\ Un{\isacharunderscore}{\kern0pt}assoc{\isacharbrackleft}{\kern0pt}symmetric{\isacharbrackright}{\kern0pt}\ nat{\isacharunderscore}{\kern0pt}union{\isacharunderscore}{\kern0pt}abs{\isadigit{2}}\isanewline
\ \ \ \ \ \ \isacommand{by}\isamarkupfalse%
\ simp\isanewline
\ \ \isacommand{qed}\isamarkupfalse%
\isanewline
\isacommand{qed}\isamarkupfalse%
%
\endisatagproof
{\isafoldproof}%
%
\isadelimproof
\isanewline
%
\endisadelimproof
\isanewline
\isacommand{lemma}\isamarkupfalse%
\ sats{\isacharunderscore}{\kern0pt}ren{\isacharunderscore}{\kern0pt}forces{\isacharunderscore}{\kern0pt}nand{\isacharcolon}{\kern0pt}\isanewline
\ \ {\isachardoublequoteopen}{\isacharbrackleft}{\kern0pt}q{\isacharcomma}{\kern0pt}P{\isacharcomma}{\kern0pt}leq{\isacharcomma}{\kern0pt}o{\isacharcomma}{\kern0pt}p{\isacharbrackright}{\kern0pt}\ {\isacharat}{\kern0pt}\ env\ {\isasymin}\ list{\isacharparenleft}{\kern0pt}M{\isacharparenright}{\kern0pt}\ {\isasymLongrightarrow}\ {\isasymphi}{\isasymin}formula\ {\isasymLongrightarrow}\isanewline
\ \ \ sats{\isacharparenleft}{\kern0pt}M{\isacharcomma}{\kern0pt}\ ren{\isacharunderscore}{\kern0pt}forces{\isacharunderscore}{\kern0pt}nand{\isacharparenleft}{\kern0pt}{\isasymphi}{\isacharparenright}{\kern0pt}{\isacharcomma}{\kern0pt}{\isacharbrackleft}{\kern0pt}q{\isacharcomma}{\kern0pt}p{\isacharcomma}{\kern0pt}P{\isacharcomma}{\kern0pt}leq{\isacharcomma}{\kern0pt}o{\isacharbrackright}{\kern0pt}\ {\isacharat}{\kern0pt}\ env{\isacharparenright}{\kern0pt}\ {\isasymlongleftrightarrow}\ sats{\isacharparenleft}{\kern0pt}M{\isacharcomma}{\kern0pt}\ {\isasymphi}{\isacharcomma}{\kern0pt}{\isacharbrackleft}{\kern0pt}q{\isacharcomma}{\kern0pt}P{\isacharcomma}{\kern0pt}leq{\isacharcomma}{\kern0pt}o{\isacharbrackright}{\kern0pt}\ {\isacharat}{\kern0pt}\ env{\isacharparenright}{\kern0pt}{\isachardoublequoteclose}\isanewline
%
\isadelimproof
\ \ %
\endisadelimproof
%
\isatagproof
\isacommand{unfolding}\isamarkupfalse%
\ ren{\isacharunderscore}{\kern0pt}forces{\isacharunderscore}{\kern0pt}nand{\isacharunderscore}{\kern0pt}def\isanewline
\ \ \isacommand{using}\isamarkupfalse%
\ sats{\isacharunderscore}{\kern0pt}incr{\isacharunderscore}{\kern0pt}bv{\isacharunderscore}{\kern0pt}iff\ {\isacharbrackleft}{\kern0pt}of\ {\isacharunderscore}{\kern0pt}\ {\isacharunderscore}{\kern0pt}\ M\ {\isacharunderscore}{\kern0pt}\ {\isachardoublequoteopen}{\isacharbrackleft}{\kern0pt}q{\isacharbrackright}{\kern0pt}{\isachardoublequoteclose}{\isacharbrackright}{\kern0pt}\isanewline
\ \ \isacommand{by}\isamarkupfalse%
\ simp%
\endisatagproof
{\isafoldproof}%
%
\isadelimproof
\isanewline
%
\endisadelimproof
\isanewline
\isanewline
\isacommand{definition}\isamarkupfalse%
\isanewline
\ \ ren{\isacharunderscore}{\kern0pt}forces{\isacharunderscore}{\kern0pt}forall\ {\isacharcolon}{\kern0pt}{\isacharcolon}{\kern0pt}\ {\isachardoublequoteopen}i{\isasymRightarrow}i{\isachardoublequoteclose}\ \isakeyword{where}\isanewline
\ \ {\isachardoublequoteopen}ren{\isacharunderscore}{\kern0pt}forces{\isacharunderscore}{\kern0pt}forall{\isacharparenleft}{\kern0pt}{\isasymphi}{\isacharparenright}{\kern0pt}\ {\isasymequiv}\isanewline
\ \ \ \ \ \ Exists{\isacharparenleft}{\kern0pt}Exists{\isacharparenleft}{\kern0pt}Exists{\isacharparenleft}{\kern0pt}Exists{\isacharparenleft}{\kern0pt}Exists{\isacharparenleft}{\kern0pt}\isanewline
\ \ \ \ \ \ \ \ And{\isacharparenleft}{\kern0pt}Equal{\isacharparenleft}{\kern0pt}{\isadigit{0}}{\isacharcomma}{\kern0pt}{\isadigit{6}}{\isacharparenright}{\kern0pt}{\isacharcomma}{\kern0pt}And{\isacharparenleft}{\kern0pt}Equal{\isacharparenleft}{\kern0pt}{\isadigit{1}}{\isacharcomma}{\kern0pt}{\isadigit{7}}{\isacharparenright}{\kern0pt}{\isacharcomma}{\kern0pt}And{\isacharparenleft}{\kern0pt}Equal{\isacharparenleft}{\kern0pt}{\isadigit{2}}{\isacharcomma}{\kern0pt}{\isadigit{8}}{\isacharparenright}{\kern0pt}{\isacharcomma}{\kern0pt}And{\isacharparenleft}{\kern0pt}Equal{\isacharparenleft}{\kern0pt}{\isadigit{3}}{\isacharcomma}{\kern0pt}{\isadigit{9}}{\isacharparenright}{\kern0pt}{\isacharcomma}{\kern0pt}\isanewline
\ \ \ \ \ \ \ \ And{\isacharparenleft}{\kern0pt}Equal{\isacharparenleft}{\kern0pt}{\isadigit{4}}{\isacharcomma}{\kern0pt}{\isadigit{5}}{\isacharparenright}{\kern0pt}{\isacharcomma}{\kern0pt}iterates{\isacharparenleft}{\kern0pt}{\isasymlambda}p{\isachardot}{\kern0pt}\ incr{\isacharunderscore}{\kern0pt}bv{\isacharparenleft}{\kern0pt}p{\isacharparenright}{\kern0pt}{\isacharbackquote}{\kern0pt}{\isadigit{5}}\ {\isacharcomma}{\kern0pt}\ {\isadigit{5}}{\isacharcomma}{\kern0pt}\ {\isasymphi}{\isacharparenright}{\kern0pt}{\isacharparenright}{\kern0pt}{\isacharparenright}{\kern0pt}{\isacharparenright}{\kern0pt}{\isacharparenright}{\kern0pt}{\isacharparenright}{\kern0pt}{\isacharparenright}{\kern0pt}{\isacharparenright}{\kern0pt}{\isacharparenright}{\kern0pt}{\isacharparenright}{\kern0pt}{\isacharparenright}{\kern0pt}{\isachardoublequoteclose}\isanewline
\isanewline
\isacommand{lemma}\isamarkupfalse%
\ arity{\isacharunderscore}{\kern0pt}ren{\isacharunderscore}{\kern0pt}forces{\isacharunderscore}{\kern0pt}all\ {\isacharcolon}{\kern0pt}\isanewline
\ \ \isakeyword{assumes}\ {\isachardoublequoteopen}{\isasymphi}{\isasymin}formula{\isachardoublequoteclose}\isanewline
\ \ \isakeyword{shows}\ {\isachardoublequoteopen}arity{\isacharparenleft}{\kern0pt}ren{\isacharunderscore}{\kern0pt}forces{\isacharunderscore}{\kern0pt}forall{\isacharparenleft}{\kern0pt}{\isasymphi}{\isacharparenright}{\kern0pt}{\isacharparenright}{\kern0pt}\ {\isacharequal}{\kern0pt}\ {\isadigit{5}}\ {\isasymunion}\ arity{\isacharparenleft}{\kern0pt}{\isasymphi}{\isacharparenright}{\kern0pt}{\isachardoublequoteclose}\isanewline
%
\isadelimproof
%
\endisadelimproof
%
\isatagproof
\isacommand{proof}\isamarkupfalse%
\ {\isacharminus}{\kern0pt}\isanewline
\ \ \isacommand{consider}\isamarkupfalse%
\ {\isacharparenleft}{\kern0pt}lt{\isacharparenright}{\kern0pt}\ {\isachardoublequoteopen}{\isadigit{5}}{\isacharless}{\kern0pt}arity{\isacharparenleft}{\kern0pt}{\isasymphi}{\isacharparenright}{\kern0pt}{\isachardoublequoteclose}\ {\isacharbar}{\kern0pt}\ {\isacharparenleft}{\kern0pt}ge{\isacharparenright}{\kern0pt}\ {\isachardoublequoteopen}{\isasymnot}\ {\isadigit{5}}\ {\isacharless}{\kern0pt}\ arity{\isacharparenleft}{\kern0pt}{\isasymphi}{\isacharparenright}{\kern0pt}{\isachardoublequoteclose}\isanewline
\ \ \ \ \isacommand{by}\isamarkupfalse%
\ auto\isanewline
\ \ \isacommand{then}\isamarkupfalse%
\isanewline
\ \ \isacommand{show}\isamarkupfalse%
\ {\isacharquery}{\kern0pt}thesis\isanewline
\ \ \isacommand{proof}\isamarkupfalse%
\ cases\isanewline
\ \ \ \ \isacommand{case}\isamarkupfalse%
\ lt\isanewline
\ \ \ \ \isacommand{with}\isamarkupfalse%
\ {\isacartoucheopen}{\isasymphi}{\isasymin}{\isacharunderscore}{\kern0pt}{\isacartoucheclose}\isanewline
\ \ \ \ \isacommand{have}\isamarkupfalse%
\ {\isachardoublequoteopen}{\isadigit{5}}\ {\isacharless}{\kern0pt}\ succ{\isacharparenleft}{\kern0pt}arity{\isacharparenleft}{\kern0pt}{\isasymphi}{\isacharparenright}{\kern0pt}{\isacharparenright}{\kern0pt}{\isachardoublequoteclose}\ {\isachardoublequoteopen}{\isadigit{5}}{\isacharless}{\kern0pt}arity{\isacharparenleft}{\kern0pt}{\isasymphi}{\isacharparenright}{\kern0pt}{\isacharhash}{\kern0pt}{\isacharplus}{\kern0pt}{\isadigit{2}}{\isachardoublequoteclose}\ \ {\isachardoublequoteopen}{\isadigit{5}}{\isacharless}{\kern0pt}arity{\isacharparenleft}{\kern0pt}{\isasymphi}{\isacharparenright}{\kern0pt}{\isacharhash}{\kern0pt}{\isacharplus}{\kern0pt}{\isadigit{3}}{\isachardoublequoteclose}\ \ {\isachardoublequoteopen}{\isadigit{5}}{\isacharless}{\kern0pt}arity{\isacharparenleft}{\kern0pt}{\isasymphi}{\isacharparenright}{\kern0pt}{\isacharhash}{\kern0pt}{\isacharplus}{\kern0pt}{\isadigit{4}}{\isachardoublequoteclose}\isanewline
\ \ \ \ \ \ \isacommand{using}\isamarkupfalse%
\ succ{\isacharunderscore}{\kern0pt}ltI\ \isacommand{by}\isamarkupfalse%
\ auto\isanewline
\ \ \ \ \isacommand{with}\isamarkupfalse%
\ {\isacartoucheopen}{\isasymphi}{\isasymin}{\isacharunderscore}{\kern0pt}{\isacartoucheclose}\isanewline
\ \ \ \ \isacommand{have}\isamarkupfalse%
\ {\isachardoublequoteopen}arity{\isacharparenleft}{\kern0pt}iterates{\isacharparenleft}{\kern0pt}{\isasymlambda}p{\isachardot}{\kern0pt}\ incr{\isacharunderscore}{\kern0pt}bv{\isacharparenleft}{\kern0pt}p{\isacharparenright}{\kern0pt}{\isacharbackquote}{\kern0pt}{\isadigit{5}}{\isacharcomma}{\kern0pt}{\isadigit{5}}{\isacharcomma}{\kern0pt}{\isasymphi}{\isacharparenright}{\kern0pt}{\isacharparenright}{\kern0pt}\ {\isacharequal}{\kern0pt}\ {\isadigit{5}}{\isacharhash}{\kern0pt}{\isacharplus}{\kern0pt}arity{\isacharparenleft}{\kern0pt}{\isasymphi}{\isacharparenright}{\kern0pt}{\isachardoublequoteclose}\isanewline
\ \ \ \ \ \ \isacommand{using}\isamarkupfalse%
\ arity{\isacharunderscore}{\kern0pt}incr{\isacharunderscore}{\kern0pt}bv{\isacharunderscore}{\kern0pt}lemma\ lt\isanewline
\ \ \ \ \ \ \isacommand{by}\isamarkupfalse%
\ simp\isanewline
\ \ \ \ \isacommand{with}\isamarkupfalse%
\ {\isacartoucheopen}{\isasymphi}{\isasymin}{\isacharunderscore}{\kern0pt}{\isacartoucheclose}\isanewline
\ \ \ \ \isacommand{show}\isamarkupfalse%
\ {\isacharquery}{\kern0pt}thesis\isanewline
\ \ \ \ \ \ \isacommand{unfolding}\isamarkupfalse%
\ ren{\isacharunderscore}{\kern0pt}forces{\isacharunderscore}{\kern0pt}forall{\isacharunderscore}{\kern0pt}def\isanewline
\ \ \ \ \ \ \isacommand{using}\isamarkupfalse%
\ pred{\isacharunderscore}{\kern0pt}Un{\isacharunderscore}{\kern0pt}distrib\ nat{\isacharunderscore}{\kern0pt}union{\isacharunderscore}{\kern0pt}abs{\isadigit{1}}\ Un{\isacharunderscore}{\kern0pt}assoc{\isacharbrackleft}{\kern0pt}symmetric{\isacharbrackright}{\kern0pt}\ nat{\isacharunderscore}{\kern0pt}union{\isacharunderscore}{\kern0pt}abs{\isadigit{2}}\isanewline
\ \ \ \ \ \ \isacommand{by}\isamarkupfalse%
\ simp\isanewline
\ \ \isacommand{next}\isamarkupfalse%
\isanewline
\ \ \ \ \isacommand{case}\isamarkupfalse%
\ ge\isanewline
\ \ \ \ \isacommand{with}\isamarkupfalse%
\ {\isacartoucheopen}{\isasymphi}{\isasymin}{\isacharunderscore}{\kern0pt}{\isacartoucheclose}\isanewline
\ \ \ \ \isacommand{have}\isamarkupfalse%
\ {\isachardoublequoteopen}arity{\isacharparenleft}{\kern0pt}{\isasymphi}{\isacharparenright}{\kern0pt}\ {\isasymle}\ {\isadigit{5}}{\isachardoublequoteclose}\ {\isachardoublequoteopen}pred{\isacharcircum}{\kern0pt}{\isadigit{5}}{\isacharparenleft}{\kern0pt}arity{\isacharparenleft}{\kern0pt}{\isasymphi}{\isacharparenright}{\kern0pt}{\isacharparenright}{\kern0pt}\ {\isasymle}\ {\isadigit{5}}{\isachardoublequoteclose}\isanewline
\ \ \ \ \ \ \isacommand{using}\isamarkupfalse%
\ not{\isacharunderscore}{\kern0pt}lt{\isacharunderscore}{\kern0pt}iff{\isacharunderscore}{\kern0pt}le\ le{\isacharunderscore}{\kern0pt}trans{\isacharbrackleft}{\kern0pt}OF\ le{\isacharunderscore}{\kern0pt}pred{\isacharbrackright}{\kern0pt}\isanewline
\ \ \ \ \ \ \isacommand{by}\isamarkupfalse%
\ simp{\isacharunderscore}{\kern0pt}all\isanewline
\ \ \ \ \isacommand{with}\isamarkupfalse%
\ {\isacartoucheopen}{\isasymphi}{\isasymin}{\isacharunderscore}{\kern0pt}{\isacartoucheclose}\isanewline
\ \ \ \ \isacommand{have}\isamarkupfalse%
\ {\isachardoublequoteopen}arity{\isacharparenleft}{\kern0pt}iterates{\isacharparenleft}{\kern0pt}{\isasymlambda}p{\isachardot}{\kern0pt}\ incr{\isacharunderscore}{\kern0pt}bv{\isacharparenleft}{\kern0pt}p{\isacharparenright}{\kern0pt}{\isacharbackquote}{\kern0pt}{\isadigit{5}}{\isacharcomma}{\kern0pt}{\isadigit{5}}{\isacharcomma}{\kern0pt}{\isasymphi}{\isacharparenright}{\kern0pt}{\isacharparenright}{\kern0pt}\ {\isacharequal}{\kern0pt}\ arity{\isacharparenleft}{\kern0pt}{\isasymphi}{\isacharparenright}{\kern0pt}{\isachardoublequoteclose}\isanewline
\ \ \ \ \ \ \isacommand{using}\isamarkupfalse%
\ arity{\isacharunderscore}{\kern0pt}incr{\isacharunderscore}{\kern0pt}bv{\isacharunderscore}{\kern0pt}lemma\ ge\isanewline
\ \ \ \ \ \ \isacommand{by}\isamarkupfalse%
\ simp\isanewline
\ \ \ \ \isacommand{with}\isamarkupfalse%
\ {\isacartoucheopen}arity{\isacharparenleft}{\kern0pt}{\isasymphi}{\isacharparenright}{\kern0pt}\ {\isasymle}\ {\isadigit{5}}{\isacartoucheclose}\ {\isacartoucheopen}{\isasymphi}{\isasymin}{\isacharunderscore}{\kern0pt}{\isacartoucheclose}\ {\isacartoucheopen}pred{\isacharcircum}{\kern0pt}{\isadigit{5}}{\isacharparenleft}{\kern0pt}{\isacharunderscore}{\kern0pt}{\isacharparenright}{\kern0pt}\ {\isasymle}\ {\isadigit{5}}{\isacartoucheclose}\isanewline
\ \ \ \ \isacommand{show}\isamarkupfalse%
\ {\isacharquery}{\kern0pt}thesis\isanewline
\ \ \ \ \ \ \isacommand{unfolding}\isamarkupfalse%
\ ren{\isacharunderscore}{\kern0pt}forces{\isacharunderscore}{\kern0pt}forall{\isacharunderscore}{\kern0pt}def\isanewline
\ \ \ \ \ \ \isacommand{using}\isamarkupfalse%
\ \ pred{\isacharunderscore}{\kern0pt}Un{\isacharunderscore}{\kern0pt}distrib\ nat{\isacharunderscore}{\kern0pt}union{\isacharunderscore}{\kern0pt}abs{\isadigit{1}}\ Un{\isacharunderscore}{\kern0pt}assoc{\isacharbrackleft}{\kern0pt}symmetric{\isacharbrackright}{\kern0pt}\ nat{\isacharunderscore}{\kern0pt}union{\isacharunderscore}{\kern0pt}abs{\isadigit{2}}\isanewline
\ \ \ \ \ \ \isacommand{by}\isamarkupfalse%
\ simp\isanewline
\ \ \isacommand{qed}\isamarkupfalse%
\isanewline
\isacommand{qed}\isamarkupfalse%
%
\endisatagproof
{\isafoldproof}%
%
\isadelimproof
\isanewline
%
\endisadelimproof
\isanewline
\isacommand{lemma}\isamarkupfalse%
\ ren{\isacharunderscore}{\kern0pt}forces{\isacharunderscore}{\kern0pt}forall{\isacharunderscore}{\kern0pt}type{\isacharbrackleft}{\kern0pt}TC{\isacharbrackright}{\kern0pt}\ {\isacharcolon}{\kern0pt}\isanewline
\ \ {\isachardoublequoteopen}{\isasymphi}{\isasymin}formula\ {\isasymLongrightarrow}\ ren{\isacharunderscore}{\kern0pt}forces{\isacharunderscore}{\kern0pt}forall{\isacharparenleft}{\kern0pt}{\isasymphi}{\isacharparenright}{\kern0pt}\ {\isasymin}formula{\isachardoublequoteclose}\isanewline
%
\isadelimproof
\ \ %
\endisadelimproof
%
\isatagproof
\isacommand{unfolding}\isamarkupfalse%
\ ren{\isacharunderscore}{\kern0pt}forces{\isacharunderscore}{\kern0pt}forall{\isacharunderscore}{\kern0pt}def\ \isacommand{by}\isamarkupfalse%
\ simp%
\endisatagproof
{\isafoldproof}%
%
\isadelimproof
\isanewline
%
\endisadelimproof
\isanewline
\isacommand{lemma}\isamarkupfalse%
\ sats{\isacharunderscore}{\kern0pt}ren{\isacharunderscore}{\kern0pt}forces{\isacharunderscore}{\kern0pt}forall\ {\isacharcolon}{\kern0pt}\isanewline
\ \ {\isachardoublequoteopen}{\isacharbrackleft}{\kern0pt}x{\isacharcomma}{\kern0pt}P{\isacharcomma}{\kern0pt}leq{\isacharcomma}{\kern0pt}o{\isacharcomma}{\kern0pt}p{\isacharbrackright}{\kern0pt}\ {\isacharat}{\kern0pt}\ env\ {\isasymin}\ list{\isacharparenleft}{\kern0pt}M{\isacharparenright}{\kern0pt}\ {\isasymLongrightarrow}\ {\isasymphi}{\isasymin}formula\ {\isasymLongrightarrow}\isanewline
\ \ \ \ sats{\isacharparenleft}{\kern0pt}M{\isacharcomma}{\kern0pt}\ ren{\isacharunderscore}{\kern0pt}forces{\isacharunderscore}{\kern0pt}forall{\isacharparenleft}{\kern0pt}{\isasymphi}{\isacharparenright}{\kern0pt}{\isacharcomma}{\kern0pt}{\isacharbrackleft}{\kern0pt}x{\isacharcomma}{\kern0pt}p{\isacharcomma}{\kern0pt}P{\isacharcomma}{\kern0pt}leq{\isacharcomma}{\kern0pt}o{\isacharbrackright}{\kern0pt}\ {\isacharat}{\kern0pt}\ env{\isacharparenright}{\kern0pt}\ {\isasymlongleftrightarrow}\ sats{\isacharparenleft}{\kern0pt}M{\isacharcomma}{\kern0pt}\ {\isasymphi}{\isacharcomma}{\kern0pt}{\isacharbrackleft}{\kern0pt}p{\isacharcomma}{\kern0pt}P{\isacharcomma}{\kern0pt}leq{\isacharcomma}{\kern0pt}o{\isacharcomma}{\kern0pt}x{\isacharbrackright}{\kern0pt}\ {\isacharat}{\kern0pt}\ env{\isacharparenright}{\kern0pt}{\isachardoublequoteclose}\isanewline
%
\isadelimproof
\ \ %
\endisadelimproof
%
\isatagproof
\isacommand{unfolding}\isamarkupfalse%
\ ren{\isacharunderscore}{\kern0pt}forces{\isacharunderscore}{\kern0pt}forall{\isacharunderscore}{\kern0pt}def\isanewline
\ \ \isacommand{using}\isamarkupfalse%
\ sats{\isacharunderscore}{\kern0pt}incr{\isacharunderscore}{\kern0pt}bv{\isacharunderscore}{\kern0pt}iff\ {\isacharbrackleft}{\kern0pt}of\ {\isacharunderscore}{\kern0pt}\ {\isacharunderscore}{\kern0pt}\ M\ {\isacharunderscore}{\kern0pt}\ {\isachardoublequoteopen}{\isacharbrackleft}{\kern0pt}p{\isacharcomma}{\kern0pt}P{\isacharcomma}{\kern0pt}leq{\isacharcomma}{\kern0pt}o{\isacharcomma}{\kern0pt}x{\isacharbrackright}{\kern0pt}{\isachardoublequoteclose}{\isacharbrackright}{\kern0pt}\isanewline
\ \ \isacommand{by}\isamarkupfalse%
\ simp%
\endisatagproof
{\isafoldproof}%
%
\isadelimproof
\isanewline
%
\endisadelimproof
\isanewline
\isanewline
\isacommand{definition}\isamarkupfalse%
\isanewline
\ \ is{\isacharunderscore}{\kern0pt}leq\ {\isacharcolon}{\kern0pt}{\isacharcolon}{\kern0pt}\ {\isachardoublequoteopen}{\isacharbrackleft}{\kern0pt}i{\isasymRightarrow}o{\isacharcomma}{\kern0pt}i{\isacharcomma}{\kern0pt}i{\isacharcomma}{\kern0pt}i{\isacharbrackright}{\kern0pt}\ {\isasymRightarrow}\ o{\isachardoublequoteclose}\ \isakeyword{where}\isanewline
\ \ {\isachardoublequoteopen}is{\isacharunderscore}{\kern0pt}leq{\isacharparenleft}{\kern0pt}A{\isacharcomma}{\kern0pt}l{\isacharcomma}{\kern0pt}q{\isacharcomma}{\kern0pt}p{\isacharparenright}{\kern0pt}\ {\isasymequiv}\ {\isasymexists}qp{\isacharbrackleft}{\kern0pt}A{\isacharbrackright}{\kern0pt}{\isachardot}{\kern0pt}\ {\isacharparenleft}{\kern0pt}pair{\isacharparenleft}{\kern0pt}A{\isacharcomma}{\kern0pt}q{\isacharcomma}{\kern0pt}p{\isacharcomma}{\kern0pt}qp{\isacharparenright}{\kern0pt}\ {\isasymand}\ qp{\isasymin}l{\isacharparenright}{\kern0pt}{\isachardoublequoteclose}\isanewline
\isanewline
\isacommand{lemma}\isamarkupfalse%
\ {\isacharparenleft}{\kern0pt}\isakeyword{in}\ forcing{\isacharunderscore}{\kern0pt}data{\isacharparenright}{\kern0pt}\ leq{\isacharunderscore}{\kern0pt}abs{\isacharbrackleft}{\kern0pt}simp{\isacharbrackright}{\kern0pt}{\isacharcolon}{\kern0pt}\isanewline
\ \ {\isachardoublequoteopen}{\isasymlbrakk}\ l{\isasymin}M\ {\isacharsemicolon}{\kern0pt}\ q{\isasymin}M\ {\isacharsemicolon}{\kern0pt}\ p{\isasymin}M\ {\isasymrbrakk}\ {\isasymLongrightarrow}\ is{\isacharunderscore}{\kern0pt}leq{\isacharparenleft}{\kern0pt}{\isacharhash}{\kern0pt}{\isacharhash}{\kern0pt}M{\isacharcomma}{\kern0pt}l{\isacharcomma}{\kern0pt}q{\isacharcomma}{\kern0pt}p{\isacharparenright}{\kern0pt}\ {\isasymlongleftrightarrow}\ {\isasymlangle}q{\isacharcomma}{\kern0pt}p{\isasymrangle}{\isasymin}l{\isachardoublequoteclose}\isanewline
%
\isadelimproof
\ \ %
\endisadelimproof
%
\isatagproof
\isacommand{unfolding}\isamarkupfalse%
\ is{\isacharunderscore}{\kern0pt}leq{\isacharunderscore}{\kern0pt}def\ \isacommand{using}\isamarkupfalse%
\ pair{\isacharunderscore}{\kern0pt}in{\isacharunderscore}{\kern0pt}M{\isacharunderscore}{\kern0pt}iff\ \isacommand{by}\isamarkupfalse%
\ simp%
\endisatagproof
{\isafoldproof}%
%
\isadelimproof
\isanewline
%
\endisadelimproof
\isanewline
\isanewline
\isacommand{definition}\isamarkupfalse%
\isanewline
\ \ leq{\isacharunderscore}{\kern0pt}fm\ {\isacharcolon}{\kern0pt}{\isacharcolon}{\kern0pt}\ {\isachardoublequoteopen}{\isacharbrackleft}{\kern0pt}i{\isacharcomma}{\kern0pt}i{\isacharcomma}{\kern0pt}i{\isacharbrackright}{\kern0pt}\ {\isasymRightarrow}\ i{\isachardoublequoteclose}\ \isakeyword{where}\isanewline
\ \ {\isachardoublequoteopen}leq{\isacharunderscore}{\kern0pt}fm{\isacharparenleft}{\kern0pt}leq{\isacharcomma}{\kern0pt}q{\isacharcomma}{\kern0pt}p{\isacharparenright}{\kern0pt}\ {\isasymequiv}\ Exists{\isacharparenleft}{\kern0pt}And{\isacharparenleft}{\kern0pt}pair{\isacharunderscore}{\kern0pt}fm{\isacharparenleft}{\kern0pt}q{\isacharhash}{\kern0pt}{\isacharplus}{\kern0pt}{\isadigit{1}}{\isacharcomma}{\kern0pt}p{\isacharhash}{\kern0pt}{\isacharplus}{\kern0pt}{\isadigit{1}}{\isacharcomma}{\kern0pt}{\isadigit{0}}{\isacharparenright}{\kern0pt}{\isacharcomma}{\kern0pt}Member{\isacharparenleft}{\kern0pt}{\isadigit{0}}{\isacharcomma}{\kern0pt}leq{\isacharhash}{\kern0pt}{\isacharplus}{\kern0pt}{\isadigit{1}}{\isacharparenright}{\kern0pt}{\isacharparenright}{\kern0pt}{\isacharparenright}{\kern0pt}{\isachardoublequoteclose}\isanewline
\isanewline
\isacommand{lemma}\isamarkupfalse%
\ arity{\isacharunderscore}{\kern0pt}leq{\isacharunderscore}{\kern0pt}fm\ {\isacharcolon}{\kern0pt}\isanewline
\ \ {\isachardoublequoteopen}{\isasymlbrakk}leq{\isasymin}nat{\isacharsemicolon}{\kern0pt}q{\isasymin}nat{\isacharsemicolon}{\kern0pt}p{\isasymin}nat{\isasymrbrakk}\ {\isasymLongrightarrow}\ arity{\isacharparenleft}{\kern0pt}leq{\isacharunderscore}{\kern0pt}fm{\isacharparenleft}{\kern0pt}leq{\isacharcomma}{\kern0pt}q{\isacharcomma}{\kern0pt}p{\isacharparenright}{\kern0pt}{\isacharparenright}{\kern0pt}\ {\isacharequal}{\kern0pt}\ succ{\isacharparenleft}{\kern0pt}q{\isacharparenright}{\kern0pt}\ {\isasymunion}\ succ{\isacharparenleft}{\kern0pt}p{\isacharparenright}{\kern0pt}\ {\isasymunion}\ succ{\isacharparenleft}{\kern0pt}leq{\isacharparenright}{\kern0pt}{\isachardoublequoteclose}\isanewline
%
\isadelimproof
\ \ %
\endisadelimproof
%
\isatagproof
\isacommand{unfolding}\isamarkupfalse%
\ leq{\isacharunderscore}{\kern0pt}fm{\isacharunderscore}{\kern0pt}def\isanewline
\ \ \isacommand{using}\isamarkupfalse%
\ arity{\isacharunderscore}{\kern0pt}pair{\isacharunderscore}{\kern0pt}fm\ pred{\isacharunderscore}{\kern0pt}Un{\isacharunderscore}{\kern0pt}distrib\ nat{\isacharunderscore}{\kern0pt}simp{\isacharunderscore}{\kern0pt}union\isanewline
\ \ \isacommand{by}\isamarkupfalse%
\ auto%
\endisatagproof
{\isafoldproof}%
%
\isadelimproof
\isanewline
%
\endisadelimproof
\isanewline
\isacommand{lemma}\isamarkupfalse%
\ leq{\isacharunderscore}{\kern0pt}fm{\isacharunderscore}{\kern0pt}type{\isacharbrackleft}{\kern0pt}TC{\isacharbrackright}{\kern0pt}\ {\isacharcolon}{\kern0pt}\isanewline
\ \ {\isachardoublequoteopen}{\isasymlbrakk}leq{\isasymin}nat{\isacharsemicolon}{\kern0pt}q{\isasymin}nat{\isacharsemicolon}{\kern0pt}p{\isasymin}nat{\isasymrbrakk}\ {\isasymLongrightarrow}\ leq{\isacharunderscore}{\kern0pt}fm{\isacharparenleft}{\kern0pt}leq{\isacharcomma}{\kern0pt}q{\isacharcomma}{\kern0pt}p{\isacharparenright}{\kern0pt}{\isasymin}formula{\isachardoublequoteclose}\isanewline
%
\isadelimproof
\ \ %
\endisadelimproof
%
\isatagproof
\isacommand{unfolding}\isamarkupfalse%
\ leq{\isacharunderscore}{\kern0pt}fm{\isacharunderscore}{\kern0pt}def\ \isacommand{by}\isamarkupfalse%
\ simp%
\endisatagproof
{\isafoldproof}%
%
\isadelimproof
\isanewline
%
\endisadelimproof
\isanewline
\isacommand{lemma}\isamarkupfalse%
\ sats{\isacharunderscore}{\kern0pt}leq{\isacharunderscore}{\kern0pt}fm\ {\isacharcolon}{\kern0pt}\isanewline
\ \ {\isachardoublequoteopen}{\isasymlbrakk}\ leq{\isasymin}nat{\isacharsemicolon}{\kern0pt}q{\isasymin}nat{\isacharsemicolon}{\kern0pt}p{\isasymin}nat{\isacharsemicolon}{\kern0pt}env{\isasymin}list{\isacharparenleft}{\kern0pt}A{\isacharparenright}{\kern0pt}\ {\isasymrbrakk}\ {\isasymLongrightarrow}\isanewline
\ \ \ \ \ sats{\isacharparenleft}{\kern0pt}A{\isacharcomma}{\kern0pt}leq{\isacharunderscore}{\kern0pt}fm{\isacharparenleft}{\kern0pt}leq{\isacharcomma}{\kern0pt}q{\isacharcomma}{\kern0pt}p{\isacharparenright}{\kern0pt}{\isacharcomma}{\kern0pt}env{\isacharparenright}{\kern0pt}\ {\isasymlongleftrightarrow}\ is{\isacharunderscore}{\kern0pt}leq{\isacharparenleft}{\kern0pt}{\isacharhash}{\kern0pt}{\isacharhash}{\kern0pt}A{\isacharcomma}{\kern0pt}nth{\isacharparenleft}{\kern0pt}leq{\isacharcomma}{\kern0pt}env{\isacharparenright}{\kern0pt}{\isacharcomma}{\kern0pt}nth{\isacharparenleft}{\kern0pt}q{\isacharcomma}{\kern0pt}env{\isacharparenright}{\kern0pt}{\isacharcomma}{\kern0pt}nth{\isacharparenleft}{\kern0pt}p{\isacharcomma}{\kern0pt}env{\isacharparenright}{\kern0pt}{\isacharparenright}{\kern0pt}{\isachardoublequoteclose}\isanewline
%
\isadelimproof
\ \ %
\endisadelimproof
%
\isatagproof
\isacommand{unfolding}\isamarkupfalse%
\ leq{\isacharunderscore}{\kern0pt}fm{\isacharunderscore}{\kern0pt}def\ is{\isacharunderscore}{\kern0pt}leq{\isacharunderscore}{\kern0pt}def\ \isacommand{by}\isamarkupfalse%
\ simp%
\endisatagproof
{\isafoldproof}%
%
\isadelimproof
%
\endisadelimproof
%
\isadelimdocument
%
\endisadelimdocument
%
\isatagdocument
%
\isamarkupsubsubsection{The primitive recursion%
}
\isamarkuptrue%
%
\endisatagdocument
{\isafolddocument}%
%
\isadelimdocument
%
\endisadelimdocument
\isacommand{consts}\isamarkupfalse%
\ forces{\isacharprime}{\kern0pt}\ {\isacharcolon}{\kern0pt}{\isacharcolon}{\kern0pt}\ {\isachardoublequoteopen}i{\isasymRightarrow}i{\isachardoublequoteclose}\isanewline
\isacommand{primrec}\isamarkupfalse%
\isanewline
\ \ {\isachardoublequoteopen}forces{\isacharprime}{\kern0pt}{\isacharparenleft}{\kern0pt}Member{\isacharparenleft}{\kern0pt}x{\isacharcomma}{\kern0pt}y{\isacharparenright}{\kern0pt}{\isacharparenright}{\kern0pt}\ {\isacharequal}{\kern0pt}\ forces{\isacharunderscore}{\kern0pt}mem{\isacharunderscore}{\kern0pt}fm{\isacharparenleft}{\kern0pt}{\isadigit{1}}{\isacharcomma}{\kern0pt}{\isadigit{2}}{\isacharcomma}{\kern0pt}{\isadigit{0}}{\isacharcomma}{\kern0pt}x{\isacharhash}{\kern0pt}{\isacharplus}{\kern0pt}{\isadigit{4}}{\isacharcomma}{\kern0pt}y{\isacharhash}{\kern0pt}{\isacharplus}{\kern0pt}{\isadigit{4}}{\isacharparenright}{\kern0pt}{\isachardoublequoteclose}\isanewline
\ \ {\isachardoublequoteopen}forces{\isacharprime}{\kern0pt}{\isacharparenleft}{\kern0pt}Equal{\isacharparenleft}{\kern0pt}x{\isacharcomma}{\kern0pt}y{\isacharparenright}{\kern0pt}{\isacharparenright}{\kern0pt}\ \ {\isacharequal}{\kern0pt}\ forces{\isacharunderscore}{\kern0pt}eq{\isacharunderscore}{\kern0pt}fm{\isacharparenleft}{\kern0pt}{\isadigit{1}}{\isacharcomma}{\kern0pt}{\isadigit{2}}{\isacharcomma}{\kern0pt}{\isadigit{0}}{\isacharcomma}{\kern0pt}x{\isacharhash}{\kern0pt}{\isacharplus}{\kern0pt}{\isadigit{4}}{\isacharcomma}{\kern0pt}y{\isacharhash}{\kern0pt}{\isacharplus}{\kern0pt}{\isadigit{4}}{\isacharparenright}{\kern0pt}{\isachardoublequoteclose}\isanewline
\ \ {\isachardoublequoteopen}forces{\isacharprime}{\kern0pt}{\isacharparenleft}{\kern0pt}Nand{\isacharparenleft}{\kern0pt}p{\isacharcomma}{\kern0pt}q{\isacharparenright}{\kern0pt}{\isacharparenright}{\kern0pt}\ \ \ {\isacharequal}{\kern0pt}\isanewline
\ \ \ \ \ \ \ \ Neg{\isacharparenleft}{\kern0pt}Exists{\isacharparenleft}{\kern0pt}And{\isacharparenleft}{\kern0pt}Member{\isacharparenleft}{\kern0pt}{\isadigit{0}}{\isacharcomma}{\kern0pt}{\isadigit{2}}{\isacharparenright}{\kern0pt}{\isacharcomma}{\kern0pt}And{\isacharparenleft}{\kern0pt}leq{\isacharunderscore}{\kern0pt}fm{\isacharparenleft}{\kern0pt}{\isadigit{3}}{\isacharcomma}{\kern0pt}{\isadigit{0}}{\isacharcomma}{\kern0pt}{\isadigit{1}}{\isacharparenright}{\kern0pt}{\isacharcomma}{\kern0pt}And{\isacharparenleft}{\kern0pt}ren{\isacharunderscore}{\kern0pt}forces{\isacharunderscore}{\kern0pt}nand{\isacharparenleft}{\kern0pt}forces{\isacharprime}{\kern0pt}{\isacharparenleft}{\kern0pt}p{\isacharparenright}{\kern0pt}{\isacharparenright}{\kern0pt}{\isacharcomma}{\kern0pt}\isanewline
\ \ \ \ \ \ \ \ \ \ \ \ \ \ \ \ \ \ \ \ \ \ \ \ \ \ \ \ \ \ \ \ \ \ \ \ \ \ \ \ \ ren{\isacharunderscore}{\kern0pt}forces{\isacharunderscore}{\kern0pt}nand{\isacharparenleft}{\kern0pt}forces{\isacharprime}{\kern0pt}{\isacharparenleft}{\kern0pt}q{\isacharparenright}{\kern0pt}{\isacharparenright}{\kern0pt}{\isacharparenright}{\kern0pt}{\isacharparenright}{\kern0pt}{\isacharparenright}{\kern0pt}{\isacharparenright}{\kern0pt}{\isacharparenright}{\kern0pt}{\isachardoublequoteclose}\isanewline
\ \ {\isachardoublequoteopen}forces{\isacharprime}{\kern0pt}{\isacharparenleft}{\kern0pt}Forall{\isacharparenleft}{\kern0pt}p{\isacharparenright}{\kern0pt}{\isacharparenright}{\kern0pt}\ \ \ {\isacharequal}{\kern0pt}\ Forall{\isacharparenleft}{\kern0pt}ren{\isacharunderscore}{\kern0pt}forces{\isacharunderscore}{\kern0pt}forall{\isacharparenleft}{\kern0pt}forces{\isacharprime}{\kern0pt}{\isacharparenleft}{\kern0pt}p{\isacharparenright}{\kern0pt}{\isacharparenright}{\kern0pt}{\isacharparenright}{\kern0pt}{\isachardoublequoteclose}\isanewline
\isanewline
\isanewline
\isacommand{definition}\isamarkupfalse%
\isanewline
\ \ forces\ {\isacharcolon}{\kern0pt}{\isacharcolon}{\kern0pt}\ {\isachardoublequoteopen}i{\isasymRightarrow}i{\isachardoublequoteclose}\ \isakeyword{where}\isanewline
\ \ {\isachardoublequoteopen}forces{\isacharparenleft}{\kern0pt}{\isasymphi}{\isacharparenright}{\kern0pt}\ {\isasymequiv}\ And{\isacharparenleft}{\kern0pt}Member{\isacharparenleft}{\kern0pt}{\isadigit{0}}{\isacharcomma}{\kern0pt}{\isadigit{1}}{\isacharparenright}{\kern0pt}{\isacharcomma}{\kern0pt}forces{\isacharprime}{\kern0pt}{\isacharparenleft}{\kern0pt}{\isasymphi}{\isacharparenright}{\kern0pt}{\isacharparenright}{\kern0pt}{\isachardoublequoteclose}\isanewline
\isanewline
\isacommand{lemma}\isamarkupfalse%
\ forces{\isacharprime}{\kern0pt}{\isacharunderscore}{\kern0pt}type\ {\isacharbrackleft}{\kern0pt}TC{\isacharbrackright}{\kern0pt}{\isacharcolon}{\kern0pt}\ \ {\isachardoublequoteopen}{\isasymphi}{\isasymin}formula\ {\isasymLongrightarrow}\ forces{\isacharprime}{\kern0pt}{\isacharparenleft}{\kern0pt}{\isasymphi}{\isacharparenright}{\kern0pt}\ {\isasymin}\ formula{\isachardoublequoteclose}\isanewline
%
\isadelimproof
\ \ %
\endisadelimproof
%
\isatagproof
\isacommand{by}\isamarkupfalse%
\ {\isacharparenleft}{\kern0pt}induct\ {\isasymphi}\ set{\isacharcolon}{\kern0pt}formula{\isacharsemicolon}{\kern0pt}\ simp{\isacharparenright}{\kern0pt}%
\endisatagproof
{\isafoldproof}%
%
\isadelimproof
\isanewline
%
\endisadelimproof
\isanewline
\isacommand{lemma}\isamarkupfalse%
\ forces{\isacharunderscore}{\kern0pt}type{\isacharbrackleft}{\kern0pt}TC{\isacharbrackright}{\kern0pt}\ {\isacharcolon}{\kern0pt}\ {\isachardoublequoteopen}{\isasymphi}{\isasymin}formula\ {\isasymLongrightarrow}\ forces{\isacharparenleft}{\kern0pt}{\isasymphi}{\isacharparenright}{\kern0pt}\ {\isasymin}\ formula{\isachardoublequoteclose}\isanewline
%
\isadelimproof
\ \ %
\endisadelimproof
%
\isatagproof
\isacommand{unfolding}\isamarkupfalse%
\ forces{\isacharunderscore}{\kern0pt}def\ \isacommand{by}\isamarkupfalse%
\ simp%
\endisatagproof
{\isafoldproof}%
%
\isadelimproof
\isanewline
%
\endisadelimproof
\isanewline
\isacommand{context}\isamarkupfalse%
\ forcing{\isacharunderscore}{\kern0pt}data\isanewline
\isakeyword{begin}%
\isadelimdocument
%
\endisadelimdocument
%
\isatagdocument
%
\isamarkupsubsection{Forcing for atomic formulas in context%
}
\isamarkuptrue%
%
\endisatagdocument
{\isafolddocument}%
%
\isadelimdocument
%
\endisadelimdocument
\isacommand{definition}\isamarkupfalse%
\isanewline
\ \ forces{\isacharunderscore}{\kern0pt}eq\ {\isacharcolon}{\kern0pt}{\isacharcolon}{\kern0pt}\ {\isachardoublequoteopen}{\isacharbrackleft}{\kern0pt}i{\isacharcomma}{\kern0pt}i{\isacharcomma}{\kern0pt}i{\isacharbrackright}{\kern0pt}\ {\isasymRightarrow}\ o{\isachardoublequoteclose}\ \isakeyword{where}\isanewline
\ \ {\isachardoublequoteopen}forces{\isacharunderscore}{\kern0pt}eq\ {\isasymequiv}\ forces{\isacharunderscore}{\kern0pt}eq{\isacharprime}{\kern0pt}{\isacharparenleft}{\kern0pt}P{\isacharcomma}{\kern0pt}leq{\isacharparenright}{\kern0pt}{\isachardoublequoteclose}\isanewline
\isanewline
\isacommand{definition}\isamarkupfalse%
\isanewline
\ \ forces{\isacharunderscore}{\kern0pt}mem\ {\isacharcolon}{\kern0pt}{\isacharcolon}{\kern0pt}\ {\isachardoublequoteopen}{\isacharbrackleft}{\kern0pt}i{\isacharcomma}{\kern0pt}i{\isacharcomma}{\kern0pt}i{\isacharbrackright}{\kern0pt}\ {\isasymRightarrow}\ o{\isachardoublequoteclose}\ \isakeyword{where}\isanewline
\ \ {\isachardoublequoteopen}forces{\isacharunderscore}{\kern0pt}mem\ {\isasymequiv}\ forces{\isacharunderscore}{\kern0pt}mem{\isacharprime}{\kern0pt}{\isacharparenleft}{\kern0pt}P{\isacharcomma}{\kern0pt}leq{\isacharparenright}{\kern0pt}{\isachardoublequoteclose}\isanewline
\isanewline
\isanewline
\isacommand{definition}\isamarkupfalse%
\isanewline
\ \ is{\isacharunderscore}{\kern0pt}forces{\isacharunderscore}{\kern0pt}eq\ {\isacharcolon}{\kern0pt}{\isacharcolon}{\kern0pt}\ {\isachardoublequoteopen}{\isacharbrackleft}{\kern0pt}i{\isacharcomma}{\kern0pt}i{\isacharcomma}{\kern0pt}i{\isacharbrackright}{\kern0pt}\ {\isasymRightarrow}\ o{\isachardoublequoteclose}\ \isakeyword{where}\isanewline
\ \ {\isachardoublequoteopen}is{\isacharunderscore}{\kern0pt}forces{\isacharunderscore}{\kern0pt}eq\ {\isasymequiv}\ is{\isacharunderscore}{\kern0pt}forces{\isacharunderscore}{\kern0pt}eq{\isacharprime}{\kern0pt}{\isacharparenleft}{\kern0pt}{\isacharhash}{\kern0pt}{\isacharhash}{\kern0pt}M{\isacharcomma}{\kern0pt}P{\isacharcomma}{\kern0pt}leq{\isacharparenright}{\kern0pt}{\isachardoublequoteclose}\isanewline
\isanewline
\isanewline
\isacommand{definition}\isamarkupfalse%
\isanewline
\ \ is{\isacharunderscore}{\kern0pt}forces{\isacharunderscore}{\kern0pt}mem\ {\isacharcolon}{\kern0pt}{\isacharcolon}{\kern0pt}\ {\isachardoublequoteopen}{\isacharbrackleft}{\kern0pt}i{\isacharcomma}{\kern0pt}i{\isacharcomma}{\kern0pt}i{\isacharbrackright}{\kern0pt}\ {\isasymRightarrow}\ o{\isachardoublequoteclose}\ \isakeyword{where}\isanewline
\ \ {\isachardoublequoteopen}is{\isacharunderscore}{\kern0pt}forces{\isacharunderscore}{\kern0pt}mem\ {\isasymequiv}\ is{\isacharunderscore}{\kern0pt}forces{\isacharunderscore}{\kern0pt}mem{\isacharprime}{\kern0pt}{\isacharparenleft}{\kern0pt}{\isacharhash}{\kern0pt}{\isacharhash}{\kern0pt}M{\isacharcomma}{\kern0pt}P{\isacharcomma}{\kern0pt}leq{\isacharparenright}{\kern0pt}{\isachardoublequoteclose}\isanewline
\isanewline
\isanewline
\isacommand{lemma}\isamarkupfalse%
\ def{\isacharunderscore}{\kern0pt}forces{\isacharunderscore}{\kern0pt}eq{\isacharcolon}{\kern0pt}\ {\isachardoublequoteopen}p{\isasymin}P\ {\isasymLongrightarrow}\ forces{\isacharunderscore}{\kern0pt}eq{\isacharparenleft}{\kern0pt}p{\isacharcomma}{\kern0pt}t{\isadigit{1}}{\isacharcomma}{\kern0pt}t{\isadigit{2}}{\isacharparenright}{\kern0pt}\ {\isasymlongleftrightarrow}\isanewline
\ \ \ \ \ \ {\isacharparenleft}{\kern0pt}{\isasymforall}s{\isasymin}domain{\isacharparenleft}{\kern0pt}t{\isadigit{1}}{\isacharparenright}{\kern0pt}\ {\isasymunion}\ domain{\isacharparenleft}{\kern0pt}t{\isadigit{2}}{\isacharparenright}{\kern0pt}{\isachardot}{\kern0pt}\ {\isasymforall}q{\isachardot}{\kern0pt}\ q{\isasymin}P\ {\isasymand}\ q\ {\isasympreceq}\ p\ {\isasymlongrightarrow}\isanewline
\ \ \ \ \ \ {\isacharparenleft}{\kern0pt}forces{\isacharunderscore}{\kern0pt}mem{\isacharparenleft}{\kern0pt}q{\isacharcomma}{\kern0pt}s{\isacharcomma}{\kern0pt}t{\isadigit{1}}{\isacharparenright}{\kern0pt}\ {\isasymlongleftrightarrow}\ forces{\isacharunderscore}{\kern0pt}mem{\isacharparenleft}{\kern0pt}q{\isacharcomma}{\kern0pt}s{\isacharcomma}{\kern0pt}t{\isadigit{2}}{\isacharparenright}{\kern0pt}{\isacharparenright}{\kern0pt}{\isacharparenright}{\kern0pt}{\isachardoublequoteclose}\isanewline
%
\isadelimproof
\ \ %
\endisadelimproof
%
\isatagproof
\isacommand{unfolding}\isamarkupfalse%
\ forces{\isacharunderscore}{\kern0pt}eq{\isacharunderscore}{\kern0pt}def\ forces{\isacharunderscore}{\kern0pt}mem{\isacharunderscore}{\kern0pt}def\ forces{\isacharunderscore}{\kern0pt}eq{\isacharprime}{\kern0pt}{\isacharunderscore}{\kern0pt}def\ forces{\isacharunderscore}{\kern0pt}mem{\isacharprime}{\kern0pt}{\isacharunderscore}{\kern0pt}def\isanewline
\ \ \isacommand{using}\isamarkupfalse%
\ def{\isacharunderscore}{\kern0pt}frc{\isacharunderscore}{\kern0pt}at{\isacharbrackleft}{\kern0pt}of\ p\ {\isadigit{0}}\ t{\isadigit{1}}\ t{\isadigit{2}}\ {\isacharbrackright}{\kern0pt}\ \ \isacommand{unfolding}\isamarkupfalse%
\ bool{\isacharunderscore}{\kern0pt}of{\isacharunderscore}{\kern0pt}o{\isacharunderscore}{\kern0pt}def\isanewline
\ \ \isacommand{by}\isamarkupfalse%
\ auto%
\endisatagproof
{\isafoldproof}%
%
\isadelimproof
\isanewline
%
\endisadelimproof
\isanewline
\isacommand{lemma}\isamarkupfalse%
\ def{\isacharunderscore}{\kern0pt}forces{\isacharunderscore}{\kern0pt}mem{\isacharcolon}{\kern0pt}\ {\isachardoublequoteopen}p{\isasymin}P\ {\isasymLongrightarrow}\ forces{\isacharunderscore}{\kern0pt}mem{\isacharparenleft}{\kern0pt}p{\isacharcomma}{\kern0pt}t{\isadigit{1}}{\isacharcomma}{\kern0pt}t{\isadigit{2}}{\isacharparenright}{\kern0pt}\ {\isasymlongleftrightarrow}\isanewline
\ \ \ \ \ {\isacharparenleft}{\kern0pt}{\isasymforall}v{\isasymin}P{\isachardot}{\kern0pt}\ v\ {\isasympreceq}\ p\ {\isasymlongrightarrow}\isanewline
\ \ \ \ \ \ {\isacharparenleft}{\kern0pt}{\isasymexists}q{\isachardot}{\kern0pt}\ {\isasymexists}s{\isachardot}{\kern0pt}\ {\isasymexists}r{\isachardot}{\kern0pt}\ r{\isasymin}P\ {\isasymand}\ q{\isasymin}P\ {\isasymand}\ q\ {\isasympreceq}\ v\ {\isasymand}\ {\isasymlangle}s{\isacharcomma}{\kern0pt}r{\isasymrangle}\ {\isasymin}\ t{\isadigit{2}}\ {\isasymand}\ q\ {\isasympreceq}\ r\ {\isasymand}\ forces{\isacharunderscore}{\kern0pt}eq{\isacharparenleft}{\kern0pt}q{\isacharcomma}{\kern0pt}t{\isadigit{1}}{\isacharcomma}{\kern0pt}s{\isacharparenright}{\kern0pt}{\isacharparenright}{\kern0pt}{\isacharparenright}{\kern0pt}{\isachardoublequoteclose}\isanewline
%
\isadelimproof
\ \ %
\endisadelimproof
%
\isatagproof
\isacommand{unfolding}\isamarkupfalse%
\ forces{\isacharunderscore}{\kern0pt}eq{\isacharprime}{\kern0pt}{\isacharunderscore}{\kern0pt}def\ forces{\isacharunderscore}{\kern0pt}mem{\isacharprime}{\kern0pt}{\isacharunderscore}{\kern0pt}def\ forces{\isacharunderscore}{\kern0pt}eq{\isacharunderscore}{\kern0pt}def\ forces{\isacharunderscore}{\kern0pt}mem{\isacharunderscore}{\kern0pt}def\isanewline
\ \ \isacommand{using}\isamarkupfalse%
\ def{\isacharunderscore}{\kern0pt}frc{\isacharunderscore}{\kern0pt}at{\isacharbrackleft}{\kern0pt}of\ p\ {\isadigit{1}}\ t{\isadigit{1}}\ t{\isadigit{2}}{\isacharbrackright}{\kern0pt}\ \ \isacommand{unfolding}\isamarkupfalse%
\ bool{\isacharunderscore}{\kern0pt}of{\isacharunderscore}{\kern0pt}o{\isacharunderscore}{\kern0pt}def\isanewline
\ \ \isacommand{by}\isamarkupfalse%
\ auto%
\endisatagproof
{\isafoldproof}%
%
\isadelimproof
\isanewline
%
\endisadelimproof
\isanewline
\isacommand{lemma}\isamarkupfalse%
\ forces{\isacharunderscore}{\kern0pt}eq{\isacharunderscore}{\kern0pt}abs\ {\isacharcolon}{\kern0pt}\isanewline
\ \ {\isachardoublequoteopen}{\isasymlbrakk}p{\isasymin}M\ {\isacharsemicolon}{\kern0pt}\ t{\isadigit{1}}{\isasymin}M\ {\isacharsemicolon}{\kern0pt}\ t{\isadigit{2}}{\isasymin}M{\isasymrbrakk}\ {\isasymLongrightarrow}\ is{\isacharunderscore}{\kern0pt}forces{\isacharunderscore}{\kern0pt}eq{\isacharparenleft}{\kern0pt}p{\isacharcomma}{\kern0pt}t{\isadigit{1}}{\isacharcomma}{\kern0pt}t{\isadigit{2}}{\isacharparenright}{\kern0pt}\ {\isasymlongleftrightarrow}\ forces{\isacharunderscore}{\kern0pt}eq{\isacharparenleft}{\kern0pt}p{\isacharcomma}{\kern0pt}t{\isadigit{1}}{\isacharcomma}{\kern0pt}t{\isadigit{2}}{\isacharparenright}{\kern0pt}{\isachardoublequoteclose}\isanewline
%
\isadelimproof
\ \ %
\endisadelimproof
%
\isatagproof
\isacommand{unfolding}\isamarkupfalse%
\ is{\isacharunderscore}{\kern0pt}forces{\isacharunderscore}{\kern0pt}eq{\isacharunderscore}{\kern0pt}def\ forces{\isacharunderscore}{\kern0pt}eq{\isacharunderscore}{\kern0pt}def\isanewline
\ \ \isacommand{using}\isamarkupfalse%
\ forces{\isacharunderscore}{\kern0pt}eq{\isacharprime}{\kern0pt}{\isacharunderscore}{\kern0pt}abs\ \isacommand{by}\isamarkupfalse%
\ simp%
\endisatagproof
{\isafoldproof}%
%
\isadelimproof
\isanewline
%
\endisadelimproof
\isanewline
\isacommand{lemma}\isamarkupfalse%
\ forces{\isacharunderscore}{\kern0pt}mem{\isacharunderscore}{\kern0pt}abs\ {\isacharcolon}{\kern0pt}\isanewline
\ \ {\isachardoublequoteopen}{\isasymlbrakk}p{\isasymin}M\ {\isacharsemicolon}{\kern0pt}\ t{\isadigit{1}}{\isasymin}M\ {\isacharsemicolon}{\kern0pt}\ t{\isadigit{2}}{\isasymin}M{\isasymrbrakk}\ {\isasymLongrightarrow}\ is{\isacharunderscore}{\kern0pt}forces{\isacharunderscore}{\kern0pt}mem{\isacharparenleft}{\kern0pt}p{\isacharcomma}{\kern0pt}t{\isadigit{1}}{\isacharcomma}{\kern0pt}t{\isadigit{2}}{\isacharparenright}{\kern0pt}\ {\isasymlongleftrightarrow}\ forces{\isacharunderscore}{\kern0pt}mem{\isacharparenleft}{\kern0pt}p{\isacharcomma}{\kern0pt}t{\isadigit{1}}{\isacharcomma}{\kern0pt}t{\isadigit{2}}{\isacharparenright}{\kern0pt}{\isachardoublequoteclose}\isanewline
%
\isadelimproof
\ \ %
\endisadelimproof
%
\isatagproof
\isacommand{unfolding}\isamarkupfalse%
\ is{\isacharunderscore}{\kern0pt}forces{\isacharunderscore}{\kern0pt}mem{\isacharunderscore}{\kern0pt}def\ forces{\isacharunderscore}{\kern0pt}mem{\isacharunderscore}{\kern0pt}def\isanewline
\ \ \isacommand{using}\isamarkupfalse%
\ forces{\isacharunderscore}{\kern0pt}mem{\isacharprime}{\kern0pt}{\isacharunderscore}{\kern0pt}abs\ \isacommand{by}\isamarkupfalse%
\ simp%
\endisatagproof
{\isafoldproof}%
%
\isadelimproof
\isanewline
%
\endisadelimproof
\isanewline
\isacommand{lemma}\isamarkupfalse%
\ sats{\isacharunderscore}{\kern0pt}forces{\isacharunderscore}{\kern0pt}eq{\isacharunderscore}{\kern0pt}fm{\isacharcolon}{\kern0pt}\isanewline
\ \ \isakeyword{assumes}\ \ {\isachardoublequoteopen}p{\isasymin}nat{\isachardoublequoteclose}\ {\isachardoublequoteopen}l{\isasymin}nat{\isachardoublequoteclose}\ {\isachardoublequoteopen}q{\isasymin}nat{\isachardoublequoteclose}\ {\isachardoublequoteopen}t{\isadigit{1}}{\isasymin}nat{\isachardoublequoteclose}\ {\isachardoublequoteopen}t{\isadigit{2}}{\isasymin}nat{\isachardoublequoteclose}\ \ {\isachardoublequoteopen}env{\isasymin}list{\isacharparenleft}{\kern0pt}M{\isacharparenright}{\kern0pt}{\isachardoublequoteclose}\isanewline
\ \ \ \ {\isachardoublequoteopen}nth{\isacharparenleft}{\kern0pt}p{\isacharcomma}{\kern0pt}env{\isacharparenright}{\kern0pt}{\isacharequal}{\kern0pt}P{\isachardoublequoteclose}\ {\isachardoublequoteopen}nth{\isacharparenleft}{\kern0pt}l{\isacharcomma}{\kern0pt}env{\isacharparenright}{\kern0pt}{\isacharequal}{\kern0pt}leq{\isachardoublequoteclose}\isanewline
\ \ \isakeyword{shows}\ {\isachardoublequoteopen}sats{\isacharparenleft}{\kern0pt}M{\isacharcomma}{\kern0pt}forces{\isacharunderscore}{\kern0pt}eq{\isacharunderscore}{\kern0pt}fm{\isacharparenleft}{\kern0pt}p{\isacharcomma}{\kern0pt}l{\isacharcomma}{\kern0pt}q{\isacharcomma}{\kern0pt}t{\isadigit{1}}{\isacharcomma}{\kern0pt}t{\isadigit{2}}{\isacharparenright}{\kern0pt}{\isacharcomma}{\kern0pt}env{\isacharparenright}{\kern0pt}\ {\isasymlongleftrightarrow}\isanewline
\ \ \ \ \ \ \ \ \ is{\isacharunderscore}{\kern0pt}forces{\isacharunderscore}{\kern0pt}eq{\isacharparenleft}{\kern0pt}nth{\isacharparenleft}{\kern0pt}q{\isacharcomma}{\kern0pt}env{\isacharparenright}{\kern0pt}{\isacharcomma}{\kern0pt}nth{\isacharparenleft}{\kern0pt}t{\isadigit{1}}{\isacharcomma}{\kern0pt}env{\isacharparenright}{\kern0pt}{\isacharcomma}{\kern0pt}nth{\isacharparenleft}{\kern0pt}t{\isadigit{2}}{\isacharcomma}{\kern0pt}env{\isacharparenright}{\kern0pt}{\isacharparenright}{\kern0pt}{\isachardoublequoteclose}\isanewline
%
\isadelimproof
\ \ %
\endisadelimproof
%
\isatagproof
\isacommand{unfolding}\isamarkupfalse%
\ forces{\isacharunderscore}{\kern0pt}eq{\isacharunderscore}{\kern0pt}fm{\isacharunderscore}{\kern0pt}def\ is{\isacharunderscore}{\kern0pt}forces{\isacharunderscore}{\kern0pt}eq{\isacharunderscore}{\kern0pt}def\ is{\isacharunderscore}{\kern0pt}forces{\isacharunderscore}{\kern0pt}eq{\isacharprime}{\kern0pt}{\isacharunderscore}{\kern0pt}def\isanewline
\ \ \isacommand{using}\isamarkupfalse%
\ assms\ sats{\isacharunderscore}{\kern0pt}is{\isacharunderscore}{\kern0pt}tuple{\isacharunderscore}{\kern0pt}fm\ \ sats{\isacharunderscore}{\kern0pt}frc{\isacharunderscore}{\kern0pt}at{\isacharunderscore}{\kern0pt}fm\isanewline
\ \ \isacommand{by}\isamarkupfalse%
\ simp%
\endisatagproof
{\isafoldproof}%
%
\isadelimproof
\isanewline
%
\endisadelimproof
\isanewline
\isacommand{lemma}\isamarkupfalse%
\ sats{\isacharunderscore}{\kern0pt}forces{\isacharunderscore}{\kern0pt}mem{\isacharunderscore}{\kern0pt}fm{\isacharcolon}{\kern0pt}\isanewline
\ \ \isakeyword{assumes}\ \ {\isachardoublequoteopen}p{\isasymin}nat{\isachardoublequoteclose}\ {\isachardoublequoteopen}l{\isasymin}nat{\isachardoublequoteclose}\ {\isachardoublequoteopen}q{\isasymin}nat{\isachardoublequoteclose}\ {\isachardoublequoteopen}t{\isadigit{1}}{\isasymin}nat{\isachardoublequoteclose}\ {\isachardoublequoteopen}t{\isadigit{2}}{\isasymin}nat{\isachardoublequoteclose}\ \ {\isachardoublequoteopen}env{\isasymin}list{\isacharparenleft}{\kern0pt}M{\isacharparenright}{\kern0pt}{\isachardoublequoteclose}\isanewline
\ \ \ \ {\isachardoublequoteopen}nth{\isacharparenleft}{\kern0pt}p{\isacharcomma}{\kern0pt}env{\isacharparenright}{\kern0pt}{\isacharequal}{\kern0pt}P{\isachardoublequoteclose}\ {\isachardoublequoteopen}nth{\isacharparenleft}{\kern0pt}l{\isacharcomma}{\kern0pt}env{\isacharparenright}{\kern0pt}{\isacharequal}{\kern0pt}leq{\isachardoublequoteclose}\isanewline
\ \ \isakeyword{shows}\ {\isachardoublequoteopen}sats{\isacharparenleft}{\kern0pt}M{\isacharcomma}{\kern0pt}forces{\isacharunderscore}{\kern0pt}mem{\isacharunderscore}{\kern0pt}fm{\isacharparenleft}{\kern0pt}p{\isacharcomma}{\kern0pt}l{\isacharcomma}{\kern0pt}q{\isacharcomma}{\kern0pt}t{\isadigit{1}}{\isacharcomma}{\kern0pt}t{\isadigit{2}}{\isacharparenright}{\kern0pt}{\isacharcomma}{\kern0pt}env{\isacharparenright}{\kern0pt}\ {\isasymlongleftrightarrow}\isanewline
\ \ \ \ \ \ \ \ \ \ \ \ \ is{\isacharunderscore}{\kern0pt}forces{\isacharunderscore}{\kern0pt}mem{\isacharparenleft}{\kern0pt}nth{\isacharparenleft}{\kern0pt}q{\isacharcomma}{\kern0pt}env{\isacharparenright}{\kern0pt}{\isacharcomma}{\kern0pt}nth{\isacharparenleft}{\kern0pt}t{\isadigit{1}}{\isacharcomma}{\kern0pt}env{\isacharparenright}{\kern0pt}{\isacharcomma}{\kern0pt}nth{\isacharparenleft}{\kern0pt}t{\isadigit{2}}{\isacharcomma}{\kern0pt}env{\isacharparenright}{\kern0pt}{\isacharparenright}{\kern0pt}{\isachardoublequoteclose}\isanewline
%
\isadelimproof
\ \ %
\endisadelimproof
%
\isatagproof
\isacommand{unfolding}\isamarkupfalse%
\ forces{\isacharunderscore}{\kern0pt}mem{\isacharunderscore}{\kern0pt}fm{\isacharunderscore}{\kern0pt}def\ is{\isacharunderscore}{\kern0pt}forces{\isacharunderscore}{\kern0pt}mem{\isacharunderscore}{\kern0pt}def\ is{\isacharunderscore}{\kern0pt}forces{\isacharunderscore}{\kern0pt}mem{\isacharprime}{\kern0pt}{\isacharunderscore}{\kern0pt}def\isanewline
\ \ \isacommand{using}\isamarkupfalse%
\ assms\ sats{\isacharunderscore}{\kern0pt}is{\isacharunderscore}{\kern0pt}tuple{\isacharunderscore}{\kern0pt}fm\ sats{\isacharunderscore}{\kern0pt}frc{\isacharunderscore}{\kern0pt}at{\isacharunderscore}{\kern0pt}fm\isanewline
\ \ \isacommand{by}\isamarkupfalse%
\ simp%
\endisatagproof
{\isafoldproof}%
%
\isadelimproof
\isanewline
%
\endisadelimproof
\isanewline
\isanewline
\isacommand{definition}\isamarkupfalse%
\isanewline
\ \ forces{\isacharunderscore}{\kern0pt}neq\ {\isacharcolon}{\kern0pt}{\isacharcolon}{\kern0pt}\ {\isachardoublequoteopen}{\isacharbrackleft}{\kern0pt}i{\isacharcomma}{\kern0pt}i{\isacharcomma}{\kern0pt}i{\isacharbrackright}{\kern0pt}\ {\isasymRightarrow}\ o{\isachardoublequoteclose}\ \isakeyword{where}\isanewline
\ \ {\isachardoublequoteopen}forces{\isacharunderscore}{\kern0pt}neq{\isacharparenleft}{\kern0pt}p{\isacharcomma}{\kern0pt}t{\isadigit{1}}{\isacharcomma}{\kern0pt}t{\isadigit{2}}{\isacharparenright}{\kern0pt}\ {\isasymequiv}\ {\isasymnot}\ {\isacharparenleft}{\kern0pt}{\isasymexists}q{\isasymin}P{\isachardot}{\kern0pt}\ q{\isasympreceq}p\ {\isasymand}\ forces{\isacharunderscore}{\kern0pt}eq{\isacharparenleft}{\kern0pt}q{\isacharcomma}{\kern0pt}t{\isadigit{1}}{\isacharcomma}{\kern0pt}t{\isadigit{2}}{\isacharparenright}{\kern0pt}{\isacharparenright}{\kern0pt}{\isachardoublequoteclose}\isanewline
\isanewline
\isacommand{definition}\isamarkupfalse%
\isanewline
\ \ forces{\isacharunderscore}{\kern0pt}nmem\ {\isacharcolon}{\kern0pt}{\isacharcolon}{\kern0pt}\ {\isachardoublequoteopen}{\isacharbrackleft}{\kern0pt}i{\isacharcomma}{\kern0pt}i{\isacharcomma}{\kern0pt}i{\isacharbrackright}{\kern0pt}\ {\isasymRightarrow}\ o{\isachardoublequoteclose}\ \isakeyword{where}\isanewline
\ \ {\isachardoublequoteopen}forces{\isacharunderscore}{\kern0pt}nmem{\isacharparenleft}{\kern0pt}p{\isacharcomma}{\kern0pt}t{\isadigit{1}}{\isacharcomma}{\kern0pt}t{\isadigit{2}}{\isacharparenright}{\kern0pt}\ {\isasymequiv}\ {\isasymnot}\ {\isacharparenleft}{\kern0pt}{\isasymexists}q{\isasymin}P{\isachardot}{\kern0pt}\ q{\isasympreceq}p\ {\isasymand}\ forces{\isacharunderscore}{\kern0pt}mem{\isacharparenleft}{\kern0pt}q{\isacharcomma}{\kern0pt}t{\isadigit{1}}{\isacharcomma}{\kern0pt}t{\isadigit{2}}{\isacharparenright}{\kern0pt}{\isacharparenright}{\kern0pt}{\isachardoublequoteclose}\isanewline
\isanewline
\isanewline
\isacommand{lemma}\isamarkupfalse%
\ forces{\isacharunderscore}{\kern0pt}neq\ {\isacharcolon}{\kern0pt}\isanewline
\ \ {\isachardoublequoteopen}forces{\isacharunderscore}{\kern0pt}neq{\isacharparenleft}{\kern0pt}p{\isacharcomma}{\kern0pt}t{\isadigit{1}}{\isacharcomma}{\kern0pt}t{\isadigit{2}}{\isacharparenright}{\kern0pt}\ {\isasymlongleftrightarrow}\ forces{\isacharunderscore}{\kern0pt}neq{\isacharprime}{\kern0pt}{\isacharparenleft}{\kern0pt}P{\isacharcomma}{\kern0pt}leq{\isacharcomma}{\kern0pt}p{\isacharcomma}{\kern0pt}t{\isadigit{1}}{\isacharcomma}{\kern0pt}t{\isadigit{2}}{\isacharparenright}{\kern0pt}{\isachardoublequoteclose}\isanewline
%
\isadelimproof
\ \ %
\endisadelimproof
%
\isatagproof
\isacommand{unfolding}\isamarkupfalse%
\ forces{\isacharunderscore}{\kern0pt}neq{\isacharunderscore}{\kern0pt}def\ forces{\isacharunderscore}{\kern0pt}neq{\isacharprime}{\kern0pt}{\isacharunderscore}{\kern0pt}def\ forces{\isacharunderscore}{\kern0pt}eq{\isacharunderscore}{\kern0pt}def\ \isacommand{by}\isamarkupfalse%
\ simp%
\endisatagproof
{\isafoldproof}%
%
\isadelimproof
\isanewline
%
\endisadelimproof
\isanewline
\isacommand{lemma}\isamarkupfalse%
\ forces{\isacharunderscore}{\kern0pt}nmem\ {\isacharcolon}{\kern0pt}\isanewline
\ \ {\isachardoublequoteopen}forces{\isacharunderscore}{\kern0pt}nmem{\isacharparenleft}{\kern0pt}p{\isacharcomma}{\kern0pt}t{\isadigit{1}}{\isacharcomma}{\kern0pt}t{\isadigit{2}}{\isacharparenright}{\kern0pt}\ {\isasymlongleftrightarrow}\ forces{\isacharunderscore}{\kern0pt}nmem{\isacharprime}{\kern0pt}{\isacharparenleft}{\kern0pt}P{\isacharcomma}{\kern0pt}leq{\isacharcomma}{\kern0pt}p{\isacharcomma}{\kern0pt}t{\isadigit{1}}{\isacharcomma}{\kern0pt}t{\isadigit{2}}{\isacharparenright}{\kern0pt}{\isachardoublequoteclose}\isanewline
%
\isadelimproof
\ \ %
\endisadelimproof
%
\isatagproof
\isacommand{unfolding}\isamarkupfalse%
\ forces{\isacharunderscore}{\kern0pt}nmem{\isacharunderscore}{\kern0pt}def\ forces{\isacharunderscore}{\kern0pt}nmem{\isacharprime}{\kern0pt}{\isacharunderscore}{\kern0pt}def\ forces{\isacharunderscore}{\kern0pt}mem{\isacharunderscore}{\kern0pt}def\ \isacommand{by}\isamarkupfalse%
\ simp%
\endisatagproof
{\isafoldproof}%
%
\isadelimproof
\isanewline
%
\endisadelimproof
\isanewline
\isanewline
\isacommand{lemma}\isamarkupfalse%
\ sats{\isacharunderscore}{\kern0pt}forces{\isacharunderscore}{\kern0pt}Member\ {\isacharcolon}{\kern0pt}\isanewline
\ \ \isakeyword{assumes}\ \ {\isachardoublequoteopen}x{\isasymin}nat{\isachardoublequoteclose}\ {\isachardoublequoteopen}y{\isasymin}nat{\isachardoublequoteclose}\ {\isachardoublequoteopen}env{\isasymin}list{\isacharparenleft}{\kern0pt}M{\isacharparenright}{\kern0pt}{\isachardoublequoteclose}\isanewline
\ \ \ \ {\isachardoublequoteopen}nth{\isacharparenleft}{\kern0pt}x{\isacharcomma}{\kern0pt}env{\isacharparenright}{\kern0pt}{\isacharequal}{\kern0pt}xx{\isachardoublequoteclose}\ {\isachardoublequoteopen}nth{\isacharparenleft}{\kern0pt}y{\isacharcomma}{\kern0pt}env{\isacharparenright}{\kern0pt}{\isacharequal}{\kern0pt}yy{\isachardoublequoteclose}\ {\isachardoublequoteopen}q{\isasymin}M{\isachardoublequoteclose}\isanewline
\ \ \isakeyword{shows}\ {\isachardoublequoteopen}sats{\isacharparenleft}{\kern0pt}M{\isacharcomma}{\kern0pt}forces{\isacharparenleft}{\kern0pt}Member{\isacharparenleft}{\kern0pt}x{\isacharcomma}{\kern0pt}y{\isacharparenright}{\kern0pt}{\isacharparenright}{\kern0pt}{\isacharcomma}{\kern0pt}{\isacharbrackleft}{\kern0pt}q{\isacharcomma}{\kern0pt}P{\isacharcomma}{\kern0pt}leq{\isacharcomma}{\kern0pt}one{\isacharbrackright}{\kern0pt}{\isacharat}{\kern0pt}env{\isacharparenright}{\kern0pt}\ {\isasymlongleftrightarrow}\isanewline
\ \ \ \ \ \ \ \ \ \ \ \ \ \ \ \ {\isacharparenleft}{\kern0pt}q{\isasymin}P\ {\isasymand}\ is{\isacharunderscore}{\kern0pt}forces{\isacharunderscore}{\kern0pt}mem{\isacharparenleft}{\kern0pt}q{\isacharcomma}{\kern0pt}xx{\isacharcomma}{\kern0pt}yy{\isacharparenright}{\kern0pt}{\isacharparenright}{\kern0pt}{\isachardoublequoteclose}\isanewline
%
\isadelimproof
\ \ %
\endisadelimproof
%
\isatagproof
\isacommand{unfolding}\isamarkupfalse%
\ forces{\isacharunderscore}{\kern0pt}def\isanewline
\ \ \isacommand{using}\isamarkupfalse%
\ assms\ sats{\isacharunderscore}{\kern0pt}forces{\isacharunderscore}{\kern0pt}mem{\isacharunderscore}{\kern0pt}fm\ P{\isacharunderscore}{\kern0pt}in{\isacharunderscore}{\kern0pt}M\ leq{\isacharunderscore}{\kern0pt}in{\isacharunderscore}{\kern0pt}M\ one{\isacharunderscore}{\kern0pt}in{\isacharunderscore}{\kern0pt}M\isanewline
\ \ \isacommand{by}\isamarkupfalse%
\ simp%
\endisatagproof
{\isafoldproof}%
%
\isadelimproof
\isanewline
%
\endisadelimproof
\isanewline
\isacommand{lemma}\isamarkupfalse%
\ sats{\isacharunderscore}{\kern0pt}forces{\isacharunderscore}{\kern0pt}Equal\ {\isacharcolon}{\kern0pt}\isanewline
\ \ \isakeyword{assumes}\ \ {\isachardoublequoteopen}x{\isasymin}nat{\isachardoublequoteclose}\ {\isachardoublequoteopen}y{\isasymin}nat{\isachardoublequoteclose}\ {\isachardoublequoteopen}env{\isasymin}list{\isacharparenleft}{\kern0pt}M{\isacharparenright}{\kern0pt}{\isachardoublequoteclose}\isanewline
\ \ \ \ {\isachardoublequoteopen}nth{\isacharparenleft}{\kern0pt}x{\isacharcomma}{\kern0pt}env{\isacharparenright}{\kern0pt}{\isacharequal}{\kern0pt}xx{\isachardoublequoteclose}\ {\isachardoublequoteopen}nth{\isacharparenleft}{\kern0pt}y{\isacharcomma}{\kern0pt}env{\isacharparenright}{\kern0pt}{\isacharequal}{\kern0pt}yy{\isachardoublequoteclose}\ {\isachardoublequoteopen}q{\isasymin}M{\isachardoublequoteclose}\isanewline
\ \ \isakeyword{shows}\ {\isachardoublequoteopen}sats{\isacharparenleft}{\kern0pt}M{\isacharcomma}{\kern0pt}forces{\isacharparenleft}{\kern0pt}Equal{\isacharparenleft}{\kern0pt}x{\isacharcomma}{\kern0pt}y{\isacharparenright}{\kern0pt}{\isacharparenright}{\kern0pt}{\isacharcomma}{\kern0pt}{\isacharbrackleft}{\kern0pt}q{\isacharcomma}{\kern0pt}P{\isacharcomma}{\kern0pt}leq{\isacharcomma}{\kern0pt}one{\isacharbrackright}{\kern0pt}{\isacharat}{\kern0pt}env{\isacharparenright}{\kern0pt}\ {\isasymlongleftrightarrow}\isanewline
\ \ \ \ \ \ \ \ \ \ \ \ \ \ \ \ {\isacharparenleft}{\kern0pt}q{\isasymin}P\ {\isasymand}\ is{\isacharunderscore}{\kern0pt}forces{\isacharunderscore}{\kern0pt}eq{\isacharparenleft}{\kern0pt}q{\isacharcomma}{\kern0pt}xx{\isacharcomma}{\kern0pt}yy{\isacharparenright}{\kern0pt}{\isacharparenright}{\kern0pt}{\isachardoublequoteclose}\isanewline
%
\isadelimproof
\ \ %
\endisadelimproof
%
\isatagproof
\isacommand{unfolding}\isamarkupfalse%
\ forces{\isacharunderscore}{\kern0pt}def\isanewline
\ \ \isacommand{using}\isamarkupfalse%
\ assms\ sats{\isacharunderscore}{\kern0pt}forces{\isacharunderscore}{\kern0pt}eq{\isacharunderscore}{\kern0pt}fm\ P{\isacharunderscore}{\kern0pt}in{\isacharunderscore}{\kern0pt}M\ leq{\isacharunderscore}{\kern0pt}in{\isacharunderscore}{\kern0pt}M\ one{\isacharunderscore}{\kern0pt}in{\isacharunderscore}{\kern0pt}M\isanewline
\ \ \isacommand{by}\isamarkupfalse%
\ simp%
\endisatagproof
{\isafoldproof}%
%
\isadelimproof
\isanewline
%
\endisadelimproof
\isanewline
\isacommand{lemma}\isamarkupfalse%
\ sats{\isacharunderscore}{\kern0pt}forces{\isacharunderscore}{\kern0pt}Nand\ {\isacharcolon}{\kern0pt}\isanewline
\ \ \isakeyword{assumes}\ \ {\isachardoublequoteopen}{\isasymphi}{\isasymin}formula{\isachardoublequoteclose}\ {\isachardoublequoteopen}{\isasympsi}{\isasymin}formula{\isachardoublequoteclose}\ {\isachardoublequoteopen}env{\isasymin}list{\isacharparenleft}{\kern0pt}M{\isacharparenright}{\kern0pt}{\isachardoublequoteclose}\ {\isachardoublequoteopen}p{\isasymin}M{\isachardoublequoteclose}\isanewline
\ \ \isakeyword{shows}\ {\isachardoublequoteopen}sats{\isacharparenleft}{\kern0pt}M{\isacharcomma}{\kern0pt}forces{\isacharparenleft}{\kern0pt}Nand{\isacharparenleft}{\kern0pt}{\isasymphi}{\isacharcomma}{\kern0pt}{\isasympsi}{\isacharparenright}{\kern0pt}{\isacharparenright}{\kern0pt}{\isacharcomma}{\kern0pt}{\isacharbrackleft}{\kern0pt}p{\isacharcomma}{\kern0pt}P{\isacharcomma}{\kern0pt}leq{\isacharcomma}{\kern0pt}one{\isacharbrackright}{\kern0pt}{\isacharat}{\kern0pt}env{\isacharparenright}{\kern0pt}\ {\isasymlongleftrightarrow}\isanewline
\ \ \ \ \ \ \ \ \ {\isacharparenleft}{\kern0pt}p{\isasymin}P\ {\isasymand}\ {\isasymnot}{\isacharparenleft}{\kern0pt}{\isasymexists}q{\isasymin}M{\isachardot}{\kern0pt}\ q{\isasymin}P\ {\isasymand}\ is{\isacharunderscore}{\kern0pt}leq{\isacharparenleft}{\kern0pt}{\isacharhash}{\kern0pt}{\isacharhash}{\kern0pt}M{\isacharcomma}{\kern0pt}leq{\isacharcomma}{\kern0pt}q{\isacharcomma}{\kern0pt}p{\isacharparenright}{\kern0pt}\ {\isasymand}\isanewline
\ \ \ \ \ \ \ \ \ \ \ \ \ \ \ {\isacharparenleft}{\kern0pt}sats{\isacharparenleft}{\kern0pt}M{\isacharcomma}{\kern0pt}forces{\isacharprime}{\kern0pt}{\isacharparenleft}{\kern0pt}{\isasymphi}{\isacharparenright}{\kern0pt}{\isacharcomma}{\kern0pt}{\isacharbrackleft}{\kern0pt}q{\isacharcomma}{\kern0pt}P{\isacharcomma}{\kern0pt}leq{\isacharcomma}{\kern0pt}one{\isacharbrackright}{\kern0pt}{\isacharat}{\kern0pt}env{\isacharparenright}{\kern0pt}\ {\isasymand}\ sats{\isacharparenleft}{\kern0pt}M{\isacharcomma}{\kern0pt}forces{\isacharprime}{\kern0pt}{\isacharparenleft}{\kern0pt}{\isasympsi}{\isacharparenright}{\kern0pt}{\isacharcomma}{\kern0pt}{\isacharbrackleft}{\kern0pt}q{\isacharcomma}{\kern0pt}P{\isacharcomma}{\kern0pt}leq{\isacharcomma}{\kern0pt}one{\isacharbrackright}{\kern0pt}{\isacharat}{\kern0pt}env{\isacharparenright}{\kern0pt}{\isacharparenright}{\kern0pt}{\isacharparenright}{\kern0pt}{\isacharparenright}{\kern0pt}{\isachardoublequoteclose}\isanewline
%
\isadelimproof
\ \ %
\endisadelimproof
%
\isatagproof
\isacommand{unfolding}\isamarkupfalse%
\ forces{\isacharunderscore}{\kern0pt}def\ \isacommand{using}\isamarkupfalse%
\ sats{\isacharunderscore}{\kern0pt}leq{\isacharunderscore}{\kern0pt}fm\ assms\ sats{\isacharunderscore}{\kern0pt}ren{\isacharunderscore}{\kern0pt}forces{\isacharunderscore}{\kern0pt}nand\ P{\isacharunderscore}{\kern0pt}in{\isacharunderscore}{\kern0pt}M\ leq{\isacharunderscore}{\kern0pt}in{\isacharunderscore}{\kern0pt}M\ one{\isacharunderscore}{\kern0pt}in{\isacharunderscore}{\kern0pt}M\isanewline
\ \ \isacommand{by}\isamarkupfalse%
\ simp%
\endisatagproof
{\isafoldproof}%
%
\isadelimproof
\isanewline
%
\endisadelimproof
\isanewline
\isacommand{lemma}\isamarkupfalse%
\ sats{\isacharunderscore}{\kern0pt}forces{\isacharunderscore}{\kern0pt}Neg\ {\isacharcolon}{\kern0pt}\isanewline
\ \ \isakeyword{assumes}\ \ {\isachardoublequoteopen}{\isasymphi}{\isasymin}formula{\isachardoublequoteclose}\ {\isachardoublequoteopen}env{\isasymin}list{\isacharparenleft}{\kern0pt}M{\isacharparenright}{\kern0pt}{\isachardoublequoteclose}\ {\isachardoublequoteopen}p{\isasymin}M{\isachardoublequoteclose}\isanewline
\ \ \isakeyword{shows}\ {\isachardoublequoteopen}sats{\isacharparenleft}{\kern0pt}M{\isacharcomma}{\kern0pt}forces{\isacharparenleft}{\kern0pt}Neg{\isacharparenleft}{\kern0pt}{\isasymphi}{\isacharparenright}{\kern0pt}{\isacharparenright}{\kern0pt}{\isacharcomma}{\kern0pt}{\isacharbrackleft}{\kern0pt}p{\isacharcomma}{\kern0pt}P{\isacharcomma}{\kern0pt}leq{\isacharcomma}{\kern0pt}one{\isacharbrackright}{\kern0pt}{\isacharat}{\kern0pt}env{\isacharparenright}{\kern0pt}\ {\isasymlongleftrightarrow}\isanewline
\ \ \ \ \ \ \ \ \ {\isacharparenleft}{\kern0pt}p{\isasymin}P\ {\isasymand}\ {\isasymnot}{\isacharparenleft}{\kern0pt}{\isasymexists}q{\isasymin}M{\isachardot}{\kern0pt}\ q{\isasymin}P\ {\isasymand}\ is{\isacharunderscore}{\kern0pt}leq{\isacharparenleft}{\kern0pt}{\isacharhash}{\kern0pt}{\isacharhash}{\kern0pt}M{\isacharcomma}{\kern0pt}leq{\isacharcomma}{\kern0pt}q{\isacharcomma}{\kern0pt}p{\isacharparenright}{\kern0pt}\ {\isasymand}\isanewline
\ \ \ \ \ \ \ \ \ \ \ \ \ \ \ {\isacharparenleft}{\kern0pt}sats{\isacharparenleft}{\kern0pt}M{\isacharcomma}{\kern0pt}forces{\isacharprime}{\kern0pt}{\isacharparenleft}{\kern0pt}{\isasymphi}{\isacharparenright}{\kern0pt}{\isacharcomma}{\kern0pt}{\isacharbrackleft}{\kern0pt}q{\isacharcomma}{\kern0pt}P{\isacharcomma}{\kern0pt}leq{\isacharcomma}{\kern0pt}one{\isacharbrackright}{\kern0pt}{\isacharat}{\kern0pt}env{\isacharparenright}{\kern0pt}{\isacharparenright}{\kern0pt}{\isacharparenright}{\kern0pt}{\isacharparenright}{\kern0pt}{\isachardoublequoteclose}\isanewline
%
\isadelimproof
\ \ %
\endisadelimproof
%
\isatagproof
\isacommand{unfolding}\isamarkupfalse%
\ Neg{\isacharunderscore}{\kern0pt}def\ \isacommand{using}\isamarkupfalse%
\ assms\ sats{\isacharunderscore}{\kern0pt}forces{\isacharunderscore}{\kern0pt}Nand\isanewline
\ \ \isacommand{by}\isamarkupfalse%
\ simp%
\endisatagproof
{\isafoldproof}%
%
\isadelimproof
\isanewline
%
\endisadelimproof
\isanewline
\isacommand{lemma}\isamarkupfalse%
\ sats{\isacharunderscore}{\kern0pt}forces{\isacharunderscore}{\kern0pt}Forall\ {\isacharcolon}{\kern0pt}\isanewline
\ \ \isakeyword{assumes}\ \ {\isachardoublequoteopen}{\isasymphi}{\isasymin}formula{\isachardoublequoteclose}\ {\isachardoublequoteopen}env{\isasymin}list{\isacharparenleft}{\kern0pt}M{\isacharparenright}{\kern0pt}{\isachardoublequoteclose}\ {\isachardoublequoteopen}p{\isasymin}M{\isachardoublequoteclose}\isanewline
\ \ \isakeyword{shows}\ {\isachardoublequoteopen}sats{\isacharparenleft}{\kern0pt}M{\isacharcomma}{\kern0pt}forces{\isacharparenleft}{\kern0pt}Forall{\isacharparenleft}{\kern0pt}{\isasymphi}{\isacharparenright}{\kern0pt}{\isacharparenright}{\kern0pt}{\isacharcomma}{\kern0pt}{\isacharbrackleft}{\kern0pt}p{\isacharcomma}{\kern0pt}P{\isacharcomma}{\kern0pt}leq{\isacharcomma}{\kern0pt}one{\isacharbrackright}{\kern0pt}{\isacharat}{\kern0pt}env{\isacharparenright}{\kern0pt}\ {\isasymlongleftrightarrow}\isanewline
\ \ \ \ \ \ \ \ \ p{\isasymin}P\ {\isasymand}\ {\isacharparenleft}{\kern0pt}{\isasymforall}x{\isasymin}M{\isachardot}{\kern0pt}\ sats{\isacharparenleft}{\kern0pt}M{\isacharcomma}{\kern0pt}forces{\isacharprime}{\kern0pt}{\isacharparenleft}{\kern0pt}{\isasymphi}{\isacharparenright}{\kern0pt}{\isacharcomma}{\kern0pt}{\isacharbrackleft}{\kern0pt}p{\isacharcomma}{\kern0pt}P{\isacharcomma}{\kern0pt}leq{\isacharcomma}{\kern0pt}one{\isacharcomma}{\kern0pt}x{\isacharbrackright}{\kern0pt}{\isacharat}{\kern0pt}env{\isacharparenright}{\kern0pt}{\isacharparenright}{\kern0pt}{\isachardoublequoteclose}\isanewline
%
\isadelimproof
\ \ %
\endisadelimproof
%
\isatagproof
\isacommand{unfolding}\isamarkupfalse%
\ forces{\isacharunderscore}{\kern0pt}def\ \isacommand{using}\isamarkupfalse%
\ assms\ sats{\isacharunderscore}{\kern0pt}ren{\isacharunderscore}{\kern0pt}forces{\isacharunderscore}{\kern0pt}forall\ P{\isacharunderscore}{\kern0pt}in{\isacharunderscore}{\kern0pt}M\ leq{\isacharunderscore}{\kern0pt}in{\isacharunderscore}{\kern0pt}M\ one{\isacharunderscore}{\kern0pt}in{\isacharunderscore}{\kern0pt}M\isanewline
\ \ \isacommand{by}\isamarkupfalse%
\ simp%
\endisatagproof
{\isafoldproof}%
%
\isadelimproof
\isanewline
%
\endisadelimproof
\isanewline
\isacommand{end}\isamarkupfalse%
%
\isadelimdocument
%
\endisadelimdocument
%
\isatagdocument
%
\isamarkupsubsection{The arity of \isa{forces}%
}
\isamarkuptrue%
%
\endisatagdocument
{\isafolddocument}%
%
\isadelimdocument
%
\endisadelimdocument
\isacommand{lemma}\isamarkupfalse%
\ arity{\isacharunderscore}{\kern0pt}forces{\isacharunderscore}{\kern0pt}at{\isacharcolon}{\kern0pt}\isanewline
\ \ \isakeyword{assumes}\ \ {\isachardoublequoteopen}x\ {\isasymin}\ nat{\isachardoublequoteclose}\ {\isachardoublequoteopen}y\ {\isasymin}\ nat{\isachardoublequoteclose}\isanewline
\ \ \isakeyword{shows}\ {\isachardoublequoteopen}arity{\isacharparenleft}{\kern0pt}forces{\isacharparenleft}{\kern0pt}Member{\isacharparenleft}{\kern0pt}x{\isacharcomma}{\kern0pt}\ y{\isacharparenright}{\kern0pt}{\isacharparenright}{\kern0pt}{\isacharparenright}{\kern0pt}\ {\isacharequal}{\kern0pt}\ {\isacharparenleft}{\kern0pt}succ{\isacharparenleft}{\kern0pt}x{\isacharparenright}{\kern0pt}\ {\isasymunion}\ succ{\isacharparenleft}{\kern0pt}y{\isacharparenright}{\kern0pt}{\isacharparenright}{\kern0pt}\ {\isacharhash}{\kern0pt}{\isacharplus}{\kern0pt}\ {\isadigit{4}}{\isachardoublequoteclose}\isanewline
\ \ \ \ {\isachardoublequoteopen}arity{\isacharparenleft}{\kern0pt}forces{\isacharparenleft}{\kern0pt}Equal{\isacharparenleft}{\kern0pt}x{\isacharcomma}{\kern0pt}\ y{\isacharparenright}{\kern0pt}{\isacharparenright}{\kern0pt}{\isacharparenright}{\kern0pt}\ {\isacharequal}{\kern0pt}\ {\isacharparenleft}{\kern0pt}succ{\isacharparenleft}{\kern0pt}x{\isacharparenright}{\kern0pt}\ {\isasymunion}\ succ{\isacharparenleft}{\kern0pt}y{\isacharparenright}{\kern0pt}{\isacharparenright}{\kern0pt}\ {\isacharhash}{\kern0pt}{\isacharplus}{\kern0pt}\ {\isadigit{4}}{\isachardoublequoteclose}\isanewline
%
\isadelimproof
\ \ %
\endisadelimproof
%
\isatagproof
\isacommand{unfolding}\isamarkupfalse%
\ forces{\isacharunderscore}{\kern0pt}def\isanewline
\ \ \isacommand{using}\isamarkupfalse%
\ assms\ arity{\isacharunderscore}{\kern0pt}forces{\isacharunderscore}{\kern0pt}mem{\isacharunderscore}{\kern0pt}fm\ arity{\isacharunderscore}{\kern0pt}forces{\isacharunderscore}{\kern0pt}eq{\isacharunderscore}{\kern0pt}fm\ succ{\isacharunderscore}{\kern0pt}Un{\isacharunderscore}{\kern0pt}distrib\ nat{\isacharunderscore}{\kern0pt}simp{\isacharunderscore}{\kern0pt}union\isanewline
\ \ \isacommand{by}\isamarkupfalse%
\ auto%
\endisatagproof
{\isafoldproof}%
%
\isadelimproof
\isanewline
%
\endisadelimproof
\isanewline
\isacommand{lemma}\isamarkupfalse%
\ arity{\isacharunderscore}{\kern0pt}forces{\isacharprime}{\kern0pt}{\isacharcolon}{\kern0pt}\isanewline
\ \ \isakeyword{assumes}\ {\isachardoublequoteopen}{\isasymphi}{\isasymin}formula{\isachardoublequoteclose}\isanewline
\ \ \isakeyword{shows}\ {\isachardoublequoteopen}arity{\isacharparenleft}{\kern0pt}forces{\isacharprime}{\kern0pt}{\isacharparenleft}{\kern0pt}{\isasymphi}{\isacharparenright}{\kern0pt}{\isacharparenright}{\kern0pt}\ {\isasymle}\ arity{\isacharparenleft}{\kern0pt}{\isasymphi}{\isacharparenright}{\kern0pt}\ {\isacharhash}{\kern0pt}{\isacharplus}{\kern0pt}\ {\isadigit{4}}{\isachardoublequoteclose}\isanewline
%
\isadelimproof
\ \ %
\endisadelimproof
%
\isatagproof
\isacommand{using}\isamarkupfalse%
\ assms\isanewline
\isacommand{proof}\isamarkupfalse%
\ {\isacharparenleft}{\kern0pt}induct\ set{\isacharcolon}{\kern0pt}formula{\isacharparenright}{\kern0pt}\isanewline
\ \ \isacommand{case}\isamarkupfalse%
\ {\isacharparenleft}{\kern0pt}Member\ x\ y{\isacharparenright}{\kern0pt}\isanewline
\ \ \isacommand{then}\isamarkupfalse%
\isanewline
\ \ \isacommand{show}\isamarkupfalse%
\ {\isacharquery}{\kern0pt}case\isanewline
\ \ \ \ \isacommand{using}\isamarkupfalse%
\ arity{\isacharunderscore}{\kern0pt}forces{\isacharunderscore}{\kern0pt}mem{\isacharunderscore}{\kern0pt}fm\ succ{\isacharunderscore}{\kern0pt}Un{\isacharunderscore}{\kern0pt}distrib\ nat{\isacharunderscore}{\kern0pt}simp{\isacharunderscore}{\kern0pt}union\isanewline
\ \ \ \ \isacommand{by}\isamarkupfalse%
\ simp\isanewline
\isacommand{next}\isamarkupfalse%
\isanewline
\ \ \isacommand{case}\isamarkupfalse%
\ {\isacharparenleft}{\kern0pt}Equal\ x\ y{\isacharparenright}{\kern0pt}\isanewline
\ \ \isacommand{then}\isamarkupfalse%
\isanewline
\ \ \isacommand{show}\isamarkupfalse%
\ {\isacharquery}{\kern0pt}case\isanewline
\ \ \ \ \isacommand{using}\isamarkupfalse%
\ arity{\isacharunderscore}{\kern0pt}forces{\isacharunderscore}{\kern0pt}eq{\isacharunderscore}{\kern0pt}fm\ succ{\isacharunderscore}{\kern0pt}Un{\isacharunderscore}{\kern0pt}distrib\ nat{\isacharunderscore}{\kern0pt}simp{\isacharunderscore}{\kern0pt}union\isanewline
\ \ \ \ \isacommand{by}\isamarkupfalse%
\ simp\isanewline
\isacommand{next}\isamarkupfalse%
\isanewline
\ \ \isacommand{case}\isamarkupfalse%
\ {\isacharparenleft}{\kern0pt}Nand\ {\isasymphi}\ {\isasympsi}{\isacharparenright}{\kern0pt}\isanewline
\ \ \isacommand{let}\isamarkupfalse%
\ {\isacharquery}{\kern0pt}{\isasymphi}{\isacharprime}{\kern0pt}\ {\isacharequal}{\kern0pt}\ {\isachardoublequoteopen}ren{\isacharunderscore}{\kern0pt}forces{\isacharunderscore}{\kern0pt}nand{\isacharparenleft}{\kern0pt}forces{\isacharprime}{\kern0pt}{\isacharparenleft}{\kern0pt}{\isasymphi}{\isacharparenright}{\kern0pt}{\isacharparenright}{\kern0pt}{\isachardoublequoteclose}\isanewline
\ \ \isacommand{let}\isamarkupfalse%
\ {\isacharquery}{\kern0pt}{\isasympsi}{\isacharprime}{\kern0pt}\ {\isacharequal}{\kern0pt}\ {\isachardoublequoteopen}ren{\isacharunderscore}{\kern0pt}forces{\isacharunderscore}{\kern0pt}nand{\isacharparenleft}{\kern0pt}forces{\isacharprime}{\kern0pt}{\isacharparenleft}{\kern0pt}{\isasympsi}{\isacharparenright}{\kern0pt}{\isacharparenright}{\kern0pt}{\isachardoublequoteclose}\isanewline
\ \ \isacommand{have}\isamarkupfalse%
\ {\isachardoublequoteopen}arity{\isacharparenleft}{\kern0pt}leq{\isacharunderscore}{\kern0pt}fm{\isacharparenleft}{\kern0pt}{\isadigit{3}}{\isacharcomma}{\kern0pt}\ {\isadigit{0}}{\isacharcomma}{\kern0pt}\ {\isadigit{1}}{\isacharparenright}{\kern0pt}{\isacharparenright}{\kern0pt}\ {\isacharequal}{\kern0pt}\ {\isadigit{4}}{\isachardoublequoteclose}\isanewline
\ \ \ \ \isacommand{using}\isamarkupfalse%
\ arity{\isacharunderscore}{\kern0pt}leq{\isacharunderscore}{\kern0pt}fm\ succ{\isacharunderscore}{\kern0pt}Un{\isacharunderscore}{\kern0pt}distrib\ nat{\isacharunderscore}{\kern0pt}simp{\isacharunderscore}{\kern0pt}union\isanewline
\ \ \ \ \isacommand{by}\isamarkupfalse%
\ simp\isanewline
\ \ \isacommand{have}\isamarkupfalse%
\ {\isachardoublequoteopen}{\isadigit{3}}\ {\isasymle}\ {\isacharparenleft}{\kern0pt}{\isadigit{4}}{\isacharhash}{\kern0pt}{\isacharplus}{\kern0pt}arity{\isacharparenleft}{\kern0pt}{\isasymphi}{\isacharparenright}{\kern0pt}{\isacharparenright}{\kern0pt}\ {\isasymunion}\ {\isacharparenleft}{\kern0pt}{\isadigit{4}}{\isacharhash}{\kern0pt}{\isacharplus}{\kern0pt}arity{\isacharparenleft}{\kern0pt}{\isasympsi}{\isacharparenright}{\kern0pt}{\isacharparenright}{\kern0pt}{\isachardoublequoteclose}\ {\isacharparenleft}{\kern0pt}\isakeyword{is}\ {\isachardoublequoteopen}{\isacharunderscore}{\kern0pt}\ {\isasymle}\ {\isacharquery}{\kern0pt}rhs{\isachardoublequoteclose}{\isacharparenright}{\kern0pt}\isanewline
\ \ \ \ \isacommand{using}\isamarkupfalse%
\ nat{\isacharunderscore}{\kern0pt}simp{\isacharunderscore}{\kern0pt}union\ \isacommand{by}\isamarkupfalse%
\ simp\isanewline
\ \ \isacommand{from}\isamarkupfalse%
\ {\isacartoucheopen}{\isasymphi}{\isasymin}{\isacharunderscore}{\kern0pt}{\isacartoucheclose}\ Nand\isanewline
\ \ \isacommand{have}\isamarkupfalse%
\ {\isachardoublequoteopen}pred{\isacharparenleft}{\kern0pt}arity{\isacharparenleft}{\kern0pt}{\isacharquery}{\kern0pt}{\isasymphi}{\isacharprime}{\kern0pt}{\isacharparenright}{\kern0pt}{\isacharparenright}{\kern0pt}\ {\isasymle}\ {\isacharquery}{\kern0pt}rhs{\isachardoublequoteclose}\ \ {\isachardoublequoteopen}pred{\isacharparenleft}{\kern0pt}arity{\isacharparenleft}{\kern0pt}{\isacharquery}{\kern0pt}{\isasympsi}{\isacharprime}{\kern0pt}{\isacharparenright}{\kern0pt}{\isacharparenright}{\kern0pt}\ {\isasymle}\ {\isacharquery}{\kern0pt}rhs{\isachardoublequoteclose}\isanewline
\ \ \isacommand{proof}\isamarkupfalse%
\ {\isacharminus}{\kern0pt}\isanewline
\ \ \ \ \isacommand{from}\isamarkupfalse%
\ {\isacartoucheopen}{\isasymphi}{\isasymin}{\isacharunderscore}{\kern0pt}{\isacartoucheclose}\ {\isacartoucheopen}{\isasympsi}{\isasymin}{\isacharunderscore}{\kern0pt}{\isacartoucheclose}\isanewline
\ \ \ \ \isacommand{have}\isamarkupfalse%
\ A{\isacharcolon}{\kern0pt}{\isachardoublequoteopen}pred{\isacharparenleft}{\kern0pt}arity{\isacharparenleft}{\kern0pt}{\isacharquery}{\kern0pt}{\isasymphi}{\isacharprime}{\kern0pt}{\isacharparenright}{\kern0pt}{\isacharparenright}{\kern0pt}\ {\isasymle}\ arity{\isacharparenleft}{\kern0pt}forces{\isacharprime}{\kern0pt}{\isacharparenleft}{\kern0pt}{\isasymphi}{\isacharparenright}{\kern0pt}{\isacharparenright}{\kern0pt}{\isachardoublequoteclose}\isanewline
\ \ \ \ \ \ {\isachardoublequoteopen}pred{\isacharparenleft}{\kern0pt}arity{\isacharparenleft}{\kern0pt}{\isacharquery}{\kern0pt}{\isasympsi}{\isacharprime}{\kern0pt}{\isacharparenright}{\kern0pt}{\isacharparenright}{\kern0pt}\ {\isasymle}\ arity{\isacharparenleft}{\kern0pt}forces{\isacharprime}{\kern0pt}{\isacharparenleft}{\kern0pt}{\isasympsi}{\isacharparenright}{\kern0pt}{\isacharparenright}{\kern0pt}{\isachardoublequoteclose}\isanewline
\ \ \ \ \ \ \isacommand{using}\isamarkupfalse%
\ pred{\isacharunderscore}{\kern0pt}mono{\isacharbrackleft}{\kern0pt}OF\ {\isacharunderscore}{\kern0pt}\ \ arity{\isacharunderscore}{\kern0pt}ren{\isacharunderscore}{\kern0pt}forces{\isacharunderscore}{\kern0pt}nand{\isacharbrackright}{\kern0pt}\ pred{\isacharunderscore}{\kern0pt}succ{\isacharunderscore}{\kern0pt}eq\isanewline
\ \ \ \ \ \ \isacommand{by}\isamarkupfalse%
\ simp{\isacharunderscore}{\kern0pt}all\isanewline
\ \ \ \ \isacommand{from}\isamarkupfalse%
\ Nand\isanewline
\ \ \ \ \isacommand{have}\isamarkupfalse%
\ {\isachardoublequoteopen}{\isadigit{3}}\ {\isasymunion}\ arity{\isacharparenleft}{\kern0pt}forces{\isacharprime}{\kern0pt}{\isacharparenleft}{\kern0pt}{\isasymphi}{\isacharparenright}{\kern0pt}{\isacharparenright}{\kern0pt}\ {\isasymle}\ arity{\isacharparenleft}{\kern0pt}{\isasymphi}{\isacharparenright}{\kern0pt}\ {\isacharhash}{\kern0pt}{\isacharplus}{\kern0pt}\ {\isadigit{4}}{\isachardoublequoteclose}\isanewline
\ \ \ \ \ \ {\isachardoublequoteopen}{\isadigit{3}}\ {\isasymunion}\ arity{\isacharparenleft}{\kern0pt}forces{\isacharprime}{\kern0pt}{\isacharparenleft}{\kern0pt}{\isasympsi}{\isacharparenright}{\kern0pt}{\isacharparenright}{\kern0pt}\ {\isasymle}\ arity{\isacharparenleft}{\kern0pt}{\isasympsi}{\isacharparenright}{\kern0pt}\ {\isacharhash}{\kern0pt}{\isacharplus}{\kern0pt}\ {\isadigit{4}}{\isachardoublequoteclose}\isanewline
\ \ \ \ \ \ \isacommand{using}\isamarkupfalse%
\ Un{\isacharunderscore}{\kern0pt}le\ \isacommand{by}\isamarkupfalse%
\ simp{\isacharunderscore}{\kern0pt}all\isanewline
\ \ \ \ \isacommand{with}\isamarkupfalse%
\ Nand\isanewline
\ \ \ \ \isacommand{show}\isamarkupfalse%
\ {\isachardoublequoteopen}pred{\isacharparenleft}{\kern0pt}arity{\isacharparenleft}{\kern0pt}{\isacharquery}{\kern0pt}{\isasymphi}{\isacharprime}{\kern0pt}{\isacharparenright}{\kern0pt}{\isacharparenright}{\kern0pt}\ {\isasymle}\ {\isacharquery}{\kern0pt}rhs{\isachardoublequoteclose}\isanewline
\ \ \ \ \ \ {\isachardoublequoteopen}pred{\isacharparenleft}{\kern0pt}arity{\isacharparenleft}{\kern0pt}{\isacharquery}{\kern0pt}{\isasympsi}{\isacharprime}{\kern0pt}{\isacharparenright}{\kern0pt}{\isacharparenright}{\kern0pt}\ {\isasymle}\ {\isacharquery}{\kern0pt}rhs{\isachardoublequoteclose}\isanewline
\ \ \ \ \ \ \isacommand{using}\isamarkupfalse%
\ le{\isacharunderscore}{\kern0pt}trans{\isacharbrackleft}{\kern0pt}OF\ A{\isacharparenleft}{\kern0pt}{\isadigit{1}}{\isacharparenright}{\kern0pt}{\isacharbrackright}{\kern0pt}\ le{\isacharunderscore}{\kern0pt}trans{\isacharbrackleft}{\kern0pt}OF\ A{\isacharparenleft}{\kern0pt}{\isadigit{2}}{\isacharparenright}{\kern0pt}{\isacharbrackright}{\kern0pt}\ le{\isacharunderscore}{\kern0pt}Un{\isacharunderscore}{\kern0pt}iff\isanewline
\ \ \ \ \ \ \isacommand{by}\isamarkupfalse%
\ simp{\isacharunderscore}{\kern0pt}all\isanewline
\ \ \isacommand{qed}\isamarkupfalse%
\isanewline
\ \ \isacommand{with}\isamarkupfalse%
\ Nand\ {\isacartoucheopen}{\isacharunderscore}{\kern0pt}{\isacharequal}{\kern0pt}{\isadigit{4}}{\isacartoucheclose}\isanewline
\ \ \isacommand{show}\isamarkupfalse%
\ {\isacharquery}{\kern0pt}case\isanewline
\ \ \ \ \isacommand{using}\isamarkupfalse%
\ pred{\isacharunderscore}{\kern0pt}Un{\isacharunderscore}{\kern0pt}distrib\ Un{\isacharunderscore}{\kern0pt}assoc{\isacharbrackleft}{\kern0pt}symmetric{\isacharbrackright}{\kern0pt}\ succ{\isacharunderscore}{\kern0pt}Un{\isacharunderscore}{\kern0pt}distrib\ nat{\isacharunderscore}{\kern0pt}union{\isacharunderscore}{\kern0pt}abs{\isadigit{1}}\ Un{\isacharunderscore}{\kern0pt}leI{\isadigit{3}}{\isacharbrackleft}{\kern0pt}OF\ {\isacartoucheopen}{\isadigit{3}}\ {\isasymle}\ {\isacharquery}{\kern0pt}rhs{\isacartoucheclose}{\isacharbrackright}{\kern0pt}\isanewline
\ \ \ \ \isacommand{by}\isamarkupfalse%
\ simp\isanewline
\isacommand{next}\isamarkupfalse%
\isanewline
\ \ \isacommand{case}\isamarkupfalse%
\ {\isacharparenleft}{\kern0pt}Forall\ {\isasymphi}{\isacharparenright}{\kern0pt}\isanewline
\ \ \isacommand{let}\isamarkupfalse%
\ {\isacharquery}{\kern0pt}{\isasymphi}{\isacharprime}{\kern0pt}\ {\isacharequal}{\kern0pt}\ {\isachardoublequoteopen}ren{\isacharunderscore}{\kern0pt}forces{\isacharunderscore}{\kern0pt}forall{\isacharparenleft}{\kern0pt}forces{\isacharprime}{\kern0pt}{\isacharparenleft}{\kern0pt}{\isasymphi}{\isacharparenright}{\kern0pt}{\isacharparenright}{\kern0pt}{\isachardoublequoteclose}\isanewline
\ \ \isacommand{show}\isamarkupfalse%
\ {\isacharquery}{\kern0pt}case\isanewline
\ \ \isacommand{proof}\isamarkupfalse%
\ {\isacharparenleft}{\kern0pt}cases\ {\isachardoublequoteopen}arity{\isacharparenleft}{\kern0pt}{\isasymphi}{\isacharparenright}{\kern0pt}\ {\isacharequal}{\kern0pt}\ {\isadigit{0}}{\isachardoublequoteclose}{\isacharparenright}{\kern0pt}\isanewline
\ \ \ \ \isacommand{case}\isamarkupfalse%
\ True\isanewline
\ \ \ \ \isacommand{with}\isamarkupfalse%
\ Forall\isanewline
\ \ \ \ \isacommand{show}\isamarkupfalse%
\ {\isacharquery}{\kern0pt}thesis\isanewline
\ \ \ \ \isacommand{proof}\isamarkupfalse%
\ {\isacharminus}{\kern0pt}\isanewline
\ \ \ \ \ \ \isacommand{from}\isamarkupfalse%
\ Forall\ True\isanewline
\ \ \ \ \ \ \isacommand{have}\isamarkupfalse%
\ {\isachardoublequoteopen}arity{\isacharparenleft}{\kern0pt}forces{\isacharprime}{\kern0pt}{\isacharparenleft}{\kern0pt}{\isasymphi}{\isacharparenright}{\kern0pt}{\isacharparenright}{\kern0pt}\ {\isasymle}\ {\isadigit{5}}{\isachardoublequoteclose}\isanewline
\ \ \ \ \ \ \ \ \isacommand{using}\isamarkupfalse%
\ le{\isacharunderscore}{\kern0pt}trans{\isacharbrackleft}{\kern0pt}of\ {\isacharunderscore}{\kern0pt}\ {\isadigit{4}}\ {\isadigit{5}}{\isacharbrackright}{\kern0pt}\ \isacommand{by}\isamarkupfalse%
\ auto\isanewline
\ \ \ \ \ \ \isacommand{with}\isamarkupfalse%
\ {\isacartoucheopen}{\isasymphi}{\isasymin}{\isacharunderscore}{\kern0pt}{\isacartoucheclose}\isanewline
\ \ \ \ \ \ \isacommand{have}\isamarkupfalse%
\ {\isachardoublequoteopen}arity{\isacharparenleft}{\kern0pt}{\isacharquery}{\kern0pt}{\isasymphi}{\isacharprime}{\kern0pt}{\isacharparenright}{\kern0pt}\ {\isasymle}\ {\isadigit{5}}{\isachardoublequoteclose}\isanewline
\ \ \ \ \ \ \ \ \isacommand{using}\isamarkupfalse%
\ arity{\isacharunderscore}{\kern0pt}ren{\isacharunderscore}{\kern0pt}forces{\isacharunderscore}{\kern0pt}all{\isacharbrackleft}{\kern0pt}OF\ forces{\isacharprime}{\kern0pt}{\isacharunderscore}{\kern0pt}type{\isacharbrackleft}{\kern0pt}OF\ {\isacartoucheopen}{\isasymphi}{\isasymin}{\isacharunderscore}{\kern0pt}{\isacartoucheclose}{\isacharbrackright}{\kern0pt}{\isacharbrackright}{\kern0pt}\ nat{\isacharunderscore}{\kern0pt}union{\isacharunderscore}{\kern0pt}abs{\isadigit{2}}\isanewline
\ \ \ \ \ \ \ \ \isacommand{by}\isamarkupfalse%
\ auto\isanewline
\ \ \ \ \ \ \isacommand{with}\isamarkupfalse%
\ Forall\ True\isanewline
\ \ \ \ \ \ \isacommand{show}\isamarkupfalse%
\ {\isacharquery}{\kern0pt}thesis\isanewline
\ \ \ \ \ \ \ \ \isacommand{using}\isamarkupfalse%
\ pred{\isacharunderscore}{\kern0pt}mono{\isacharbrackleft}{\kern0pt}OF\ {\isacharunderscore}{\kern0pt}\ {\isacartoucheopen}arity{\isacharparenleft}{\kern0pt}{\isacharquery}{\kern0pt}{\isasymphi}{\isacharprime}{\kern0pt}{\isacharparenright}{\kern0pt}\ {\isasymle}\ {\isadigit{5}}{\isacartoucheclose}{\isacharbrackright}{\kern0pt}\isanewline
\ \ \ \ \ \ \ \ \isacommand{by}\isamarkupfalse%
\ simp\isanewline
\ \ \ \ \isacommand{qed}\isamarkupfalse%
\isanewline
\ \ \isacommand{next}\isamarkupfalse%
\isanewline
\ \ \ \ \isacommand{case}\isamarkupfalse%
\ False\isanewline
\ \ \ \ \isacommand{with}\isamarkupfalse%
\ Forall\isanewline
\ \ \ \ \isacommand{show}\isamarkupfalse%
\ {\isacharquery}{\kern0pt}thesis\isanewline
\ \ \ \ \isacommand{proof}\isamarkupfalse%
\ {\isacharminus}{\kern0pt}\isanewline
\ \ \ \ \ \ \isacommand{from}\isamarkupfalse%
\ Forall\ False\isanewline
\ \ \ \ \ \ \isacommand{have}\isamarkupfalse%
\ {\isachardoublequoteopen}arity{\isacharparenleft}{\kern0pt}{\isacharquery}{\kern0pt}{\isasymphi}{\isacharprime}{\kern0pt}{\isacharparenright}{\kern0pt}\ {\isacharequal}{\kern0pt}\ {\isadigit{5}}\ {\isasymunion}\ arity{\isacharparenleft}{\kern0pt}forces{\isacharprime}{\kern0pt}{\isacharparenleft}{\kern0pt}{\isasymphi}{\isacharparenright}{\kern0pt}{\isacharparenright}{\kern0pt}{\isachardoublequoteclose}\isanewline
\ \ \ \ \ \ \ \ {\isachardoublequoteopen}arity{\isacharparenleft}{\kern0pt}forces{\isacharprime}{\kern0pt}{\isacharparenleft}{\kern0pt}{\isasymphi}{\isacharparenright}{\kern0pt}{\isacharparenright}{\kern0pt}\ {\isasymle}\ {\isadigit{5}}\ {\isacharhash}{\kern0pt}{\isacharplus}{\kern0pt}\ arity{\isacharparenleft}{\kern0pt}{\isasymphi}{\isacharparenright}{\kern0pt}{\isachardoublequoteclose}\isanewline
\ \ \ \ \ \ \ \ {\isachardoublequoteopen}{\isadigit{4}}\ {\isasymle}\ succ{\isacharparenleft}{\kern0pt}succ{\isacharparenleft}{\kern0pt}succ{\isacharparenleft}{\kern0pt}arity{\isacharparenleft}{\kern0pt}{\isasymphi}{\isacharparenright}{\kern0pt}{\isacharparenright}{\kern0pt}{\isacharparenright}{\kern0pt}{\isacharparenright}{\kern0pt}{\isachardoublequoteclose}\isanewline
\ \ \ \ \ \ \ \ \isacommand{using}\isamarkupfalse%
\ Ord{\isacharunderscore}{\kern0pt}{\isadigit{0}}{\isacharunderscore}{\kern0pt}lt\ arity{\isacharunderscore}{\kern0pt}ren{\isacharunderscore}{\kern0pt}forces{\isacharunderscore}{\kern0pt}all\isanewline
\ \ \ \ \ \ \ \ \ \ le{\isacharunderscore}{\kern0pt}trans{\isacharbrackleft}{\kern0pt}OF\ {\isacharunderscore}{\kern0pt}\ add{\isacharunderscore}{\kern0pt}le{\isacharunderscore}{\kern0pt}mono{\isacharbrackleft}{\kern0pt}of\ {\isadigit{4}}\ {\isadigit{5}}{\isacharcomma}{\kern0pt}\ OF\ {\isacharunderscore}{\kern0pt}\ le{\isacharunderscore}{\kern0pt}refl{\isacharbrackright}{\kern0pt}{\isacharbrackright}{\kern0pt}\isanewline
\ \ \ \ \ \ \ \ \isacommand{by}\isamarkupfalse%
\ auto\isanewline
\ \ \ \ \ \ \isacommand{with}\isamarkupfalse%
\ {\isacartoucheopen}{\isasymphi}{\isasymin}{\isacharunderscore}{\kern0pt}{\isacartoucheclose}\isanewline
\ \ \ \ \ \ \isacommand{have}\isamarkupfalse%
\ {\isachardoublequoteopen}{\isadigit{5}}\ {\isasymunion}\ arity{\isacharparenleft}{\kern0pt}forces{\isacharprime}{\kern0pt}{\isacharparenleft}{\kern0pt}{\isasymphi}{\isacharparenright}{\kern0pt}{\isacharparenright}{\kern0pt}\ {\isasymle}\ {\isadigit{5}}{\isacharhash}{\kern0pt}{\isacharplus}{\kern0pt}arity{\isacharparenleft}{\kern0pt}{\isasymphi}{\isacharparenright}{\kern0pt}{\isachardoublequoteclose}\isanewline
\ \ \ \ \ \ \ \ \isacommand{using}\isamarkupfalse%
\ nat{\isacharunderscore}{\kern0pt}simp{\isacharunderscore}{\kern0pt}union\ \isacommand{by}\isamarkupfalse%
\ auto\isanewline
\ \ \ \ \ \ \isacommand{with}\isamarkupfalse%
\ {\isacartoucheopen}{\isasymphi}{\isasymin}{\isacharunderscore}{\kern0pt}{\isacartoucheclose}\ {\isacartoucheopen}arity{\isacharparenleft}{\kern0pt}{\isacharquery}{\kern0pt}{\isasymphi}{\isacharprime}{\kern0pt}{\isacharparenright}{\kern0pt}\ {\isacharequal}{\kern0pt}\ {\isadigit{5}}\ {\isasymunion}\ {\isacharunderscore}{\kern0pt}{\isacartoucheclose}\isanewline
\ \ \ \ \ \ \isacommand{show}\isamarkupfalse%
\ {\isacharquery}{\kern0pt}thesis\isanewline
\ \ \ \ \ \ \ \ \isacommand{using}\isamarkupfalse%
\ pred{\isacharunderscore}{\kern0pt}Un{\isacharunderscore}{\kern0pt}distrib\ succ{\isacharunderscore}{\kern0pt}pred{\isacharunderscore}{\kern0pt}eq{\isacharbrackleft}{\kern0pt}OF\ {\isacharunderscore}{\kern0pt}\ {\isacartoucheopen}arity{\isacharparenleft}{\kern0pt}{\isasymphi}{\isacharparenright}{\kern0pt}{\isasymnoteq}{\isadigit{0}}{\isacartoucheclose}{\isacharbrackright}{\kern0pt}\isanewline
\ \ \ \ \ \ \ \ \ \ pred{\isacharunderscore}{\kern0pt}mono{\isacharbrackleft}{\kern0pt}OF\ {\isacharunderscore}{\kern0pt}\ Forall{\isacharparenleft}{\kern0pt}{\isadigit{2}}{\isacharparenright}{\kern0pt}{\isacharbrackright}{\kern0pt}\ Un{\isacharunderscore}{\kern0pt}le{\isacharbrackleft}{\kern0pt}OF\ {\isacartoucheopen}{\isadigit{4}}{\isasymle}succ{\isacharparenleft}{\kern0pt}{\isacharunderscore}{\kern0pt}{\isacharparenright}{\kern0pt}{\isacartoucheclose}{\isacharbrackright}{\kern0pt}\isanewline
\ \ \ \ \ \ \ \ \isacommand{by}\isamarkupfalse%
\ simp\isanewline
\ \ \ \ \isacommand{qed}\isamarkupfalse%
\isanewline
\ \ \isacommand{qed}\isamarkupfalse%
\isanewline
\isacommand{qed}\isamarkupfalse%
%
\endisatagproof
{\isafoldproof}%
%
\isadelimproof
\isanewline
%
\endisadelimproof
\isanewline
\isacommand{lemma}\isamarkupfalse%
\ arity{\isacharunderscore}{\kern0pt}forces\ {\isacharcolon}{\kern0pt}\isanewline
\ \ \isakeyword{assumes}\ {\isachardoublequoteopen}{\isasymphi}{\isasymin}formula{\isachardoublequoteclose}\isanewline
\ \ \isakeyword{shows}\ {\isachardoublequoteopen}arity{\isacharparenleft}{\kern0pt}forces{\isacharparenleft}{\kern0pt}{\isasymphi}{\isacharparenright}{\kern0pt}{\isacharparenright}{\kern0pt}\ {\isasymle}\ {\isadigit{4}}{\isacharhash}{\kern0pt}{\isacharplus}{\kern0pt}arity{\isacharparenleft}{\kern0pt}{\isasymphi}{\isacharparenright}{\kern0pt}{\isachardoublequoteclose}\isanewline
%
\isadelimproof
\ \ %
\endisadelimproof
%
\isatagproof
\isacommand{unfolding}\isamarkupfalse%
\ forces{\isacharunderscore}{\kern0pt}def\isanewline
\ \ \isacommand{using}\isamarkupfalse%
\ assms\ arity{\isacharunderscore}{\kern0pt}forces{\isacharprime}{\kern0pt}\ le{\isacharunderscore}{\kern0pt}trans\ nat{\isacharunderscore}{\kern0pt}simp{\isacharunderscore}{\kern0pt}union\ \isacommand{by}\isamarkupfalse%
\ auto%
\endisatagproof
{\isafoldproof}%
%
\isadelimproof
\isanewline
%
\endisadelimproof
\isanewline
\isacommand{lemma}\isamarkupfalse%
\ arity{\isacharunderscore}{\kern0pt}forces{\isacharunderscore}{\kern0pt}le\ {\isacharcolon}{\kern0pt}\isanewline
\ \ \isakeyword{assumes}\ {\isachardoublequoteopen}{\isasymphi}{\isasymin}formula{\isachardoublequoteclose}\ {\isachardoublequoteopen}n{\isasymin}nat{\isachardoublequoteclose}\ {\isachardoublequoteopen}arity{\isacharparenleft}{\kern0pt}{\isasymphi}{\isacharparenright}{\kern0pt}\ {\isasymle}\ n{\isachardoublequoteclose}\isanewline
\ \ \isakeyword{shows}\ {\isachardoublequoteopen}arity{\isacharparenleft}{\kern0pt}forces{\isacharparenleft}{\kern0pt}{\isasymphi}{\isacharparenright}{\kern0pt}{\isacharparenright}{\kern0pt}\ {\isasymle}\ {\isadigit{4}}{\isacharhash}{\kern0pt}{\isacharplus}{\kern0pt}n{\isachardoublequoteclose}\isanewline
%
\isadelimproof
\ \ %
\endisadelimproof
%
\isatagproof
\isacommand{using}\isamarkupfalse%
\ assms\ le{\isacharunderscore}{\kern0pt}trans{\isacharbrackleft}{\kern0pt}OF\ {\isacharunderscore}{\kern0pt}\ add{\isacharunderscore}{\kern0pt}le{\isacharunderscore}{\kern0pt}mono{\isacharbrackleft}{\kern0pt}OF\ le{\isacharunderscore}{\kern0pt}refl{\isacharbrackleft}{\kern0pt}of\ {\isadigit{5}}{\isacharbrackright}{\kern0pt}\ {\isacartoucheopen}arity{\isacharparenleft}{\kern0pt}{\isasymphi}{\isacharparenright}{\kern0pt}{\isasymle}{\isacharunderscore}{\kern0pt}{\isacartoucheclose}{\isacharbrackright}{\kern0pt}{\isacharbrackright}{\kern0pt}\ arity{\isacharunderscore}{\kern0pt}forces\isanewline
\ \ \isacommand{by}\isamarkupfalse%
\ auto%
\endisatagproof
{\isafoldproof}%
%
\isadelimproof
\isanewline
%
\endisadelimproof
%
\isadelimtheory
\isanewline
%
\endisadelimtheory
%
\isatagtheory
\isacommand{end}\isamarkupfalse%
%
\endisatagtheory
{\isafoldtheory}%
%
\isadelimtheory
%
\endisadelimtheory
%
\end{isabellebody}%
\endinput
%:%file=~/source/repos/ZF-notAC/code/Forcing/Forces_Definition.thy%:%
%:%11=1%:%
%:%27=2%:%
%:%28=2%:%
%:%37=4%:%
%:%46=6%:%
%:%56=8%:%
%:%57=8%:%
%:%58=9%:%
%:%59=10%:%
%:%60=11%:%
%:%61=12%:%
%:%62=12%:%
%:%63=13%:%
%:%64=14%:%
%:%65=15%:%
%:%66=16%:%
%:%67=16%:%
%:%68=17%:%
%:%71=18%:%
%:%75=18%:%
%:%76=18%:%
%:%77=19%:%
%:%78=19%:%
%:%79=20%:%
%:%80=20%:%
%:%85=20%:%
%:%88=21%:%
%:%89=22%:%
%:%90=22%:%
%:%91=23%:%
%:%94=24%:%
%:%98=24%:%
%:%99=24%:%
%:%100=24%:%
%:%105=24%:%
%:%108=25%:%
%:%109=26%:%
%:%110=26%:%
%:%111=27%:%
%:%112=28%:%
%:%115=29%:%
%:%119=29%:%
%:%120=29%:%
%:%121=30%:%
%:%122=30%:%
%:%123=30%:%
%:%128=30%:%
%:%131=31%:%
%:%132=32%:%
%:%133=32%:%
%:%134=33%:%
%:%135=34%:%
%:%136=35%:%
%:%137=36%:%
%:%140=37%:%
%:%144=37%:%
%:%145=37%:%
%:%146=38%:%
%:%147=38%:%
%:%152=38%:%
%:%155=39%:%
%:%156=40%:%
%:%157=40%:%
%:%158=41%:%
%:%159=42%:%
%:%160=43%:%
%:%161=44%:%
%:%162=44%:%
%:%163=45%:%
%:%164=46%:%
%:%165=47%:%
%:%166=48%:%
%:%167=48%:%
%:%168=49%:%
%:%171=50%:%
%:%175=50%:%
%:%176=50%:%
%:%177=50%:%
%:%182=50%:%
%:%185=51%:%
%:%186=52%:%
%:%187=52%:%
%:%188=53%:%
%:%189=54%:%
%:%192=55%:%
%:%196=55%:%
%:%197=55%:%
%:%198=56%:%
%:%199=56%:%
%:%200=57%:%
%:%201=57%:%
%:%206=57%:%
%:%209=58%:%
%:%210=59%:%
%:%211=59%:%
%:%212=60%:%
%:%213=61%:%
%:%214=62%:%
%:%215=63%:%
%:%216=64%:%
%:%219=65%:%
%:%223=65%:%
%:%224=65%:%
%:%225=66%:%
%:%226=66%:%
%:%227=67%:%
%:%228=67%:%
%:%233=67%:%
%:%236=68%:%
%:%237=69%:%
%:%238=69%:%
%:%239=70%:%
%:%240=71%:%
%:%241=72%:%
%:%242=73%:%
%:%245=74%:%
%:%249=74%:%
%:%250=74%:%
%:%251=75%:%
%:%252=75%:%
%:%257=75%:%
%:%260=76%:%
%:%261=77%:%
%:%262=77%:%
%:%263=78%:%
%:%264=79%:%
%:%265=80%:%
%:%266=81%:%
%:%267=81%:%
%:%268=82%:%
%:%269=83%:%
%:%270=84%:%
%:%273=85%:%
%:%277=85%:%
%:%278=85%:%
%:%279=85%:%
%:%280=85%:%
%:%285=85%:%
%:%288=86%:%
%:%289=87%:%
%:%290=88%:%
%:%291=88%:%
%:%292=89%:%
%:%293=90%:%
%:%295=92%:%
%:%296=93%:%
%:%297=94%:%
%:%298=94%:%
%:%299=95%:%
%:%300=96%:%
%:%301=97%:%
%:%302=98%:%
%:%303=98%:%
%:%304=99%:%
%:%307=100%:%
%:%311=100%:%
%:%312=100%:%
%:%313=100%:%
%:%318=100%:%
%:%321=101%:%
%:%322=102%:%
%:%323=102%:%
%:%324=103%:%
%:%327=104%:%
%:%331=104%:%
%:%332=104%:%
%:%333=105%:%
%:%334=105%:%
%:%335=106%:%
%:%336=106%:%
%:%341=106%:%
%:%344=107%:%
%:%345=108%:%
%:%346=108%:%
%:%347=109%:%
%:%348=110%:%
%:%351=111%:%
%:%355=111%:%
%:%356=111%:%
%:%361=111%:%
%:%364=112%:%
%:%365=113%:%
%:%366=113%:%
%:%367=114%:%
%:%368=115%:%
%:%370=117%:%
%:%371=118%:%
%:%372=119%:%
%:%373=119%:%
%:%374=120%:%
%:%377=121%:%
%:%381=121%:%
%:%382=121%:%
%:%383=122%:%
%:%384=122%:%
%:%385=123%:%
%:%386=123%:%
%:%391=123%:%
%:%394=124%:%
%:%395=125%:%
%:%396=126%:%
%:%397=126%:%
%:%398=127%:%
%:%401=128%:%
%:%405=128%:%
%:%406=128%:%
%:%407=128%:%
%:%412=128%:%
%:%415=129%:%
%:%416=130%:%
%:%417=130%:%
%:%418=131%:%
%:%419=132%:%
%:%420=133%:%
%:%421=134%:%
%:%422=135%:%
%:%425=136%:%
%:%429=136%:%
%:%430=136%:%
%:%431=136%:%
%:%432=136%:%
%:%437=136%:%
%:%440=137%:%
%:%441=138%:%
%:%442=138%:%
%:%443=139%:%
%:%444=140%:%
%:%445=141%:%
%:%446=142%:%
%:%447=143%:%
%:%448=144%:%
%:%449=144%:%
%:%450=145%:%
%:%451=146%:%
%:%452=147%:%
%:%453=148%:%
%:%454=149%:%
%:%455=150%:%
%:%456=150%:%
%:%457=151%:%
%:%460=152%:%
%:%464=152%:%
%:%465=152%:%
%:%466=153%:%
%:%467=153%:%
%:%468=154%:%
%:%469=154%:%
%:%474=154%:%
%:%477=155%:%
%:%478=156%:%
%:%479=156%:%
%:%480=157%:%
%:%483=158%:%
%:%487=158%:%
%:%488=158%:%
%:%489=158%:%
%:%494=158%:%
%:%497=159%:%
%:%498=160%:%
%:%499=160%:%
%:%500=161%:%
%:%501=162%:%
%:%502=163%:%
%:%503=164%:%
%:%504=165%:%
%:%507=166%:%
%:%511=166%:%
%:%512=166%:%
%:%513=166%:%
%:%514=166%:%
%:%519=166%:%
%:%522=167%:%
%:%523=168%:%
%:%524=168%:%
%:%525=169%:%
%:%526=170%:%
%:%527=171%:%
%:%528=172%:%
%:%529=173%:%
%:%532=174%:%
%:%536=174%:%
%:%537=174%:%
%:%538=174%:%
%:%552=176%:%
%:%562=180%:%
%:%563=180%:%
%:%564=181%:%
%:%565=182%:%
%:%566=183%:%
%:%567=184%:%
%:%568=185%:%
%:%569=186%:%
%:%570=186%:%
%:%571=187%:%
%:%572=188%:%
%:%579=195%:%
%:%580=196%:%
%:%581=198%:%
%:%582=199%:%
%:%583=199%:%
%:%584=200%:%
%:%585=201%:%
%:%586=202%:%
%:%587=203%:%
%:%588=204%:%
%:%589=204%:%
%:%590=205%:%
%:%591=206%:%
%:%595=210%:%
%:%596=211%:%
%:%597=212%:%
%:%598=213%:%
%:%599=213%:%
%:%600=214%:%
%:%601=215%:%
%:%602=216%:%
%:%603=217%:%
%:%604=218%:%
%:%607=219%:%
%:%611=219%:%
%:%612=219%:%
%:%613=220%:%
%:%614=220%:%
%:%619=220%:%
%:%622=221%:%
%:%625=222%:%
%:%630=223%:%
%:%631=223%:%
%:%636=223%:%
%:%639=224%:%
%:%640=225%:%
%:%641=225%:%
%:%642=226%:%
%:%643=227%:%
%:%644=228%:%
%:%645=229%:%
%:%646=230%:%
%:%649=231%:%
%:%653=231%:%
%:%654=231%:%
%:%655=232%:%
%:%656=232%:%
%:%657=233%:%
%:%658=234%:%
%:%659=234%:%
%:%664=234%:%
%:%667=235%:%
%:%668=236%:%
%:%669=236%:%
%:%670=237%:%
%:%671=238%:%
%:%672=239%:%
%:%673=240%:%
%:%674=241%:%
%:%677=242%:%
%:%681=242%:%
%:%682=242%:%
%:%683=243%:%
%:%684=243%:%
%:%689=243%:%
%:%694=244%:%
%:%699=245%:%
%:%700=245%:%
%:%705=245%:%
%:%708=246%:%
%:%709=247%:%
%:%710=247%:%
%:%711=248%:%
%:%712=249%:%
%:%713=250%:%
%:%714=251%:%
%:%715=252%:%
%:%718=253%:%
%:%722=253%:%
%:%723=253%:%
%:%724=254%:%
%:%725=254%:%
%:%726=255%:%
%:%727=256%:%
%:%728=256%:%
%:%733=256%:%
%:%736=257%:%
%:%737=258%:%
%:%738=258%:%
%:%739=259%:%
%:%740=260%:%
%:%742=262%:%
%:%743=263%:%
%:%744=264%:%
%:%745=264%:%
%:%746=265%:%
%:%747=266%:%
%:%751=270%:%
%:%752=271%:%
%:%753=272%:%
%:%754=272%:%
%:%755=273%:%
%:%756=274%:%
%:%760=278%:%
%:%761=279%:%
%:%762=280%:%
%:%763=280%:%
%:%764=281%:%
%:%767=282%:%
%:%771=282%:%
%:%772=282%:%
%:%773=282%:%
%:%778=282%:%
%:%781=283%:%
%:%782=284%:%
%:%783=284%:%
%:%784=285%:%
%:%785=286%:%
%:%786=287%:%
%:%787=288%:%
%:%790=289%:%
%:%794=289%:%
%:%795=289%:%
%:%796=290%:%
%:%797=290%:%
%:%798=291%:%
%:%799=292%:%
%:%800=292%:%
%:%805=292%:%
%:%808=293%:%
%:%809=294%:%
%:%810=294%:%
%:%811=295%:%
%:%812=296%:%
%:%813=297%:%
%:%814=298%:%
%:%815=299%:%
%:%818=300%:%
%:%822=300%:%
%:%823=300%:%
%:%824=301%:%
%:%825=301%:%
%:%826=302%:%
%:%827=302%:%
%:%832=302%:%
%:%835=303%:%
%:%836=304%:%
%:%837=304%:%
%:%838=305%:%
%:%839=306%:%
%:%840=307%:%
%:%841=308%:%
%:%842=309%:%
%:%843=310%:%
%:%846=311%:%
%:%850=311%:%
%:%851=311%:%
%:%852=312%:%
%:%853=312%:%
%:%858=312%:%
%:%861=313%:%
%:%862=314%:%
%:%863=314%:%
%:%864=315%:%
%:%865=316%:%
%:%867=318%:%
%:%868=319%:%
%:%869=320%:%
%:%870=320%:%
%:%871=321%:%
%:%872=322%:%
%:%873=323%:%
%:%874=324%:%
%:%875=325%:%
%:%876=325%:%
%:%877=326%:%
%:%878=327%:%
%:%879=328%:%
%:%880=329%:%
%:%883=330%:%
%:%887=330%:%
%:%888=330%:%
%:%889=331%:%
%:%890=331%:%
%:%891=332%:%
%:%892=332%:%
%:%897=332%:%
%:%900=333%:%
%:%901=334%:%
%:%902=335%:%
%:%903=335%:%
%:%904=336%:%
%:%907=337%:%
%:%911=337%:%
%:%912=337%:%
%:%913=337%:%
%:%918=337%:%
%:%921=338%:%
%:%922=339%:%
%:%923=339%:%
%:%924=340%:%
%:%925=341%:%
%:%926=342%:%
%:%927=343%:%
%:%928=344%:%
%:%931=345%:%
%:%935=345%:%
%:%936=345%:%
%:%937=345%:%
%:%938=346%:%
%:%939=346%:%
%:%944=346%:%
%:%947=347%:%
%:%948=348%:%
%:%949=348%:%
%:%950=349%:%
%:%951=350%:%
%:%952=351%:%
%:%953=352%:%
%:%954=353%:%
%:%955=354%:%
%:%956=355%:%
%:%959=356%:%
%:%963=356%:%
%:%964=356%:%
%:%965=356%:%
%:%970=356%:%
%:%973=357%:%
%:%974=358%:%
%:%975=358%:%
%:%976=359%:%
%:%979=360%:%
%:%983=360%:%
%:%984=360%:%
%:%985=361%:%
%:%986=361%:%
%:%987=362%:%
%:%988=363%:%
%:%989=363%:%
%:%1003=365%:%
%:%1013=366%:%
%:%1014=366%:%
%:%1015=367%:%
%:%1016=368%:%
%:%1017=369%:%
%:%1018=370%:%
%:%1019=370%:%
%:%1020=371%:%
%:%1021=372%:%
%:%1022=373%:%
%:%1023=374%:%
%:%1024=375%:%
%:%1025=375%:%
%:%1026=376%:%
%:%1027=377%:%
%:%1028=378%:%
%:%1029=379%:%
%:%1030=380%:%
%:%1031=380%:%
%:%1032=381%:%
%:%1035=382%:%
%:%1039=382%:%
%:%1040=382%:%
%:%1041=383%:%
%:%1042=383%:%
%:%1043=384%:%
%:%1044=384%:%
%:%1049=384%:%
%:%1052=385%:%
%:%1053=386%:%
%:%1054=386%:%
%:%1055=387%:%
%:%1058=388%:%
%:%1062=388%:%
%:%1063=388%:%
%:%1064=388%:%
%:%1069=388%:%
%:%1072=389%:%
%:%1073=390%:%
%:%1074=391%:%
%:%1075=391%:%
%:%1076=392%:%
%:%1077=393%:%
%:%1078=394%:%
%:%1079=395%:%
%:%1086=396%:%
%:%1087=396%:%
%:%1088=397%:%
%:%1089=397%:%
%:%1090=398%:%
%:%1091=399%:%
%:%1092=399%:%
%:%1093=399%:%
%:%1094=400%:%
%:%1095=400%:%
%:%1096=401%:%
%:%1097=401%:%
%:%1098=402%:%
%:%1099=403%:%
%:%1100=404%:%
%:%1101=404%:%
%:%1102=404%:%
%:%1103=405%:%
%:%1104=405%:%
%:%1105=406%:%
%:%1106=406%:%
%:%1107=407%:%
%:%1108=408%:%
%:%1109=409%:%
%:%1110=409%:%
%:%1111=409%:%
%:%1112=410%:%
%:%1113=410%:%
%:%1114=411%:%
%:%1115=411%:%
%:%1116=411%:%
%:%1117=411%:%
%:%1118=411%:%
%:%1119=412%:%
%:%1134=414%:%
%:%1144=415%:%
%:%1145=415%:%
%:%1146=416%:%
%:%1147=417%:%
%:%1148=418%:%
%:%1149=419%:%
%:%1150=420%:%
%:%1151=420%:%
%:%1152=421%:%
%:%1153=422%:%
%:%1154=423%:%
%:%1155=424%:%
%:%1156=425%:%
%:%1157=425%:%
%:%1158=426%:%
%:%1159=427%:%
%:%1160=428%:%
%:%1161=429%:%
%:%1162=430%:%
%:%1163=430%:%
%:%1164=431%:%
%:%1167=432%:%
%:%1171=432%:%
%:%1172=432%:%
%:%1173=432%:%
%:%1178=432%:%
%:%1181=433%:%
%:%1182=434%:%
%:%1183=434%:%
%:%1184=435%:%
%:%1185=436%:%
%:%1192=437%:%
%:%1193=437%:%
%:%1194=438%:%
%:%1195=438%:%
%:%1196=439%:%
%:%1197=439%:%
%:%1198=440%:%
%:%1199=440%:%
%:%1200=441%:%
%:%1201=441%:%
%:%1202=442%:%
%:%1203=442%:%
%:%1204=443%:%
%:%1205=443%:%
%:%1206=444%:%
%:%1207=444%:%
%:%1208=445%:%
%:%1209=445%:%
%:%1210=446%:%
%:%1211=447%:%
%:%1212=447%:%
%:%1213=448%:%
%:%1214=448%:%
%:%1215=449%:%
%:%1216=449%:%
%:%1217=450%:%
%:%1218=450%:%
%:%1219=451%:%
%:%1220=451%:%
%:%1221=452%:%
%:%1222=452%:%
%:%1223=453%:%
%:%1224=453%:%
%:%1225=454%:%
%:%1226=454%:%
%:%1227=455%:%
%:%1228=455%:%
%:%1229=456%:%
%:%1230=456%:%
%:%1231=457%:%
%:%1237=457%:%
%:%1240=458%:%
%:%1241=459%:%
%:%1242=459%:%
%:%1243=460%:%
%:%1244=461%:%
%:%1245=462%:%
%:%1246=463%:%
%:%1247=464%:%
%:%1254=465%:%
%:%1255=465%:%
%:%1256=466%:%
%:%1257=466%:%
%:%1258=467%:%
%:%1259=467%:%
%:%1260=468%:%
%:%1261=468%:%
%:%1262=469%:%
%:%1263=469%:%
%:%1264=470%:%
%:%1265=471%:%
%:%1266=472%:%
%:%1267=472%:%
%:%1268=473%:%
%:%1269=473%:%
%:%1270=474%:%
%:%1271=474%:%
%:%1272=475%:%
%:%1273=476%:%
%:%1274=476%:%
%:%1275=477%:%
%:%1276=478%:%
%:%1277=478%:%
%:%1278=479%:%
%:%1279=479%:%
%:%1280=480%:%
%:%1281=480%:%
%:%1282=481%:%
%:%1283=481%:%
%:%1284=482%:%
%:%1285=483%:%
%:%1286=483%:%
%:%1287=483%:%
%:%1288=484%:%
%:%1289=484%:%
%:%1290=485%:%
%:%1291=485%:%
%:%1292=485%:%
%:%1293=486%:%
%:%1294=486%:%
%:%1295=486%:%
%:%1296=487%:%
%:%1302=487%:%
%:%1305=488%:%
%:%1306=489%:%
%:%1307=489%:%
%:%1308=490%:%
%:%1309=491%:%
%:%1310=492%:%
%:%1311=493%:%
%:%1312=493%:%
%:%1313=494%:%
%:%1314=495%:%
%:%1315=496%:%
%:%1316=497%:%
%:%1317=497%:%
%:%1318=498%:%
%:%1319=499%:%
%:%1320=500%:%
%:%1321=501%:%
%:%1322=501%:%
%:%1323=502%:%
%:%1324=503%:%
%:%1325=504%:%
%:%1326=505%:%
%:%1327=505%:%
%:%1328=506%:%
%:%1329=507%:%
%:%1330=508%:%
%:%1331=509%:%
%:%1332=510%:%
%:%1333=510%:%
%:%1334=511%:%
%:%1335=512%:%
%:%1336=513%:%
%:%1337=514%:%
%:%1338=515%:%
%:%1339=515%:%
%:%1340=516%:%
%:%1341=517%:%
%:%1342=518%:%
%:%1343=519%:%
%:%1344=520%:%
%:%1345=520%:%
%:%1346=521%:%
%:%1347=522%:%
%:%1348=523%:%
%:%1349=524%:%
%:%1350=525%:%
%:%1351=525%:%
%:%1352=526%:%
%:%1353=527%:%
%:%1355=529%:%
%:%1356=530%:%
%:%1357=531%:%
%:%1358=531%:%
%:%1359=532%:%
%:%1360=533%:%
%:%1361=534%:%
%:%1362=535%:%
%:%1363=536%:%
%:%1364=536%:%
%:%1365=537%:%
%:%1366=538%:%
%:%1367=539%:%
%:%1368=540%:%
%:%1369=541%:%
%:%1370=541%:%
%:%1371=542%:%
%:%1372=543%:%
%:%1373=544%:%
%:%1374=545%:%
%:%1375=546%:%
%:%1376=547%:%
%:%1377=547%:%
%:%1378=548%:%
%:%1381=549%:%
%:%1385=549%:%
%:%1386=549%:%
%:%1387=550%:%
%:%1388=550%:%
%:%1393=550%:%
%:%1396=551%:%
%:%1397=552%:%
%:%1398=552%:%
%:%1399=553%:%
%:%1402=554%:%
%:%1406=554%:%
%:%1407=554%:%
%:%1408=555%:%
%:%1409=555%:%
%:%1414=555%:%
%:%1417=556%:%
%:%1418=557%:%
%:%1419=557%:%
%:%1420=558%:%
%:%1423=559%:%
%:%1427=559%:%
%:%1428=559%:%
%:%1429=560%:%
%:%1430=560%:%
%:%1435=560%:%
%:%1438=561%:%
%:%1439=562%:%
%:%1440=562%:%
%:%1441=563%:%
%:%1444=564%:%
%:%1448=564%:%
%:%1449=564%:%
%:%1450=565%:%
%:%1451=565%:%
%:%1456=565%:%
%:%1459=566%:%
%:%1460=567%:%
%:%1461=567%:%
%:%1462=568%:%
%:%1463=569%:%
%:%1466=570%:%
%:%1470=570%:%
%:%1471=570%:%
%:%1472=571%:%
%:%1473=571%:%
%:%1474=572%:%
%:%1475=573%:%
%:%1476=573%:%
%:%1481=573%:%
%:%1484=574%:%
%:%1485=575%:%
%:%1486=575%:%
%:%1487=576%:%
%:%1488=577%:%
%:%1491=578%:%
%:%1495=578%:%
%:%1496=578%:%
%:%1497=579%:%
%:%1498=579%:%
%:%1499=580%:%
%:%1500=581%:%
%:%1501=581%:%
%:%1506=581%:%
%:%1509=582%:%
%:%1510=583%:%
%:%1511=583%:%
%:%1512=584%:%
%:%1513=585%:%
%:%1514=586%:%
%:%1517=587%:%
%:%1521=587%:%
%:%1522=587%:%
%:%1523=587%:%
%:%1524=588%:%
%:%1525=588%:%
%:%1530=588%:%
%:%1533=589%:%
%:%1534=590%:%
%:%1535=590%:%
%:%1536=591%:%
%:%1537=592%:%
%:%1538=593%:%
%:%1541=594%:%
%:%1545=594%:%
%:%1546=594%:%
%:%1547=594%:%
%:%1548=595%:%
%:%1549=595%:%
%:%1554=595%:%
%:%1557=596%:%
%:%1558=597%:%
%:%1559=597%:%
%:%1560=598%:%
%:%1561=599%:%
%:%1562=600%:%
%:%1565=601%:%
%:%1569=601%:%
%:%1570=601%:%
%:%1571=602%:%
%:%1572=602%:%
%:%1573=603%:%
%:%1574=603%:%
%:%1579=603%:%
%:%1582=604%:%
%:%1583=605%:%
%:%1584=605%:%
%:%1585=606%:%
%:%1586=607%:%
%:%1587=608%:%
%:%1590=609%:%
%:%1594=609%:%
%:%1595=609%:%
%:%1596=610%:%
%:%1597=610%:%
%:%1598=611%:%
%:%1599=611%:%
%:%1604=611%:%
%:%1607=612%:%
%:%1608=613%:%
%:%1609=613%:%
%:%1610=614%:%
%:%1611=615%:%
%:%1612=616%:%
%:%1613=617%:%
%:%1614=617%:%
%:%1615=618%:%
%:%1618=619%:%
%:%1622=619%:%
%:%1623=619%:%
%:%1624=619%:%
%:%1625=620%:%
%:%1626=620%:%
%:%1631=620%:%
%:%1634=621%:%
%:%1635=622%:%
%:%1636=622%:%
%:%1637=623%:%
%:%1640=624%:%
%:%1644=624%:%
%:%1645=624%:%
%:%1646=624%:%
%:%1647=625%:%
%:%1648=625%:%
%:%1653=625%:%
%:%1656=626%:%
%:%1657=627%:%
%:%1658=627%:%
%:%1659=628%:%
%:%1661=628%:%
%:%1665=628%:%
%:%1666=628%:%
%:%1667=628%:%
%:%1674=628%:%
%:%1675=629%:%
%:%1676=630%:%
%:%1677=630%:%
%:%1678=631%:%
%:%1681=632%:%
%:%1685=632%:%
%:%1686=632%:%
%:%1687=633%:%
%:%1688=633%:%
%:%1693=633%:%
%:%1696=634%:%
%:%1697=635%:%
%:%1698=635%:%
%:%1699=636%:%
%:%1702=637%:%
%:%1706=637%:%
%:%1707=637%:%
%:%1708=638%:%
%:%1709=638%:%
%:%1714=638%:%
%:%1717=639%:%
%:%1718=640%:%
%:%1719=640%:%
%:%1720=641%:%
%:%1723=642%:%
%:%1727=642%:%
%:%1728=642%:%
%:%1729=643%:%
%:%1730=643%:%
%:%1735=643%:%
%:%1738=644%:%
%:%1739=645%:%
%:%1740=645%:%
%:%1741=646%:%
%:%1744=647%:%
%:%1748=647%:%
%:%1749=647%:%
%:%1750=648%:%
%:%1751=648%:%
%:%1756=648%:%
%:%1759=649%:%
%:%1760=650%:%
%:%1761=650%:%
%:%1762=651%:%
%:%1763=652%:%
%:%1766=653%:%
%:%1770=653%:%
%:%1771=653%:%
%:%1772=653%:%
%:%1773=653%:%
%:%1778=653%:%
%:%1781=654%:%
%:%1782=655%:%
%:%1783=655%:%
%:%1786=656%:%
%:%1790=656%:%
%:%1791=656%:%
%:%1796=656%:%
%:%1799=657%:%
%:%1800=658%:%
%:%1801=658%:%
%:%1804=659%:%
%:%1808=659%:%
%:%1809=659%:%
%:%1814=659%:%
%:%1817=660%:%
%:%1818=661%:%
%:%1819=661%:%
%:%1820=662%:%
%:%1821=663%:%
%:%1824=664%:%
%:%1828=664%:%
%:%1829=664%:%
%:%1830=664%:%
%:%1835=664%:%
%:%1838=665%:%
%:%1839=666%:%
%:%1840=667%:%
%:%1841=668%:%
%:%1842=668%:%
%:%1843=669%:%
%:%1844=670%:%
%:%1845=671%:%
%:%1846=672%:%
%:%1853=673%:%
%:%1854=673%:%
%:%1855=674%:%
%:%1856=674%:%
%:%1857=675%:%
%:%1858=675%:%
%:%1859=675%:%
%:%1860=676%:%
%:%1861=676%:%
%:%1862=677%:%
%:%1863=677%:%
%:%1864=678%:%
%:%1865=678%:%
%:%1866=678%:%
%:%1867=678%:%
%:%1868=679%:%
%:%1869=679%:%
%:%1870=680%:%
%:%1871=680%:%
%:%1872=681%:%
%:%1875=684%:%
%:%1876=685%:%
%:%1877=685%:%
%:%1878=686%:%
%:%1879=686%:%
%:%1880=687%:%
%:%1881=687%:%
%:%1882=687%:%
%:%1883=688%:%
%:%1884=688%:%
%:%1885=689%:%
%:%1886=689%:%
%:%1887=690%:%
%:%1888=691%:%
%:%1889=691%:%
%:%1890=692%:%
%:%1896=692%:%
%:%1899=693%:%
%:%1900=694%:%
%:%1901=694%:%
%:%1902=695%:%
%:%1903=696%:%
%:%1904=697%:%
%:%1905=698%:%
%:%1912=699%:%
%:%1913=699%:%
%:%1914=700%:%
%:%1915=700%:%
%:%1916=701%:%
%:%1917=701%:%
%:%1918=702%:%
%:%1919=702%:%
%:%1920=703%:%
%:%1921=703%:%
%:%1922=704%:%
%:%1923=704%:%
%:%1924=705%:%
%:%1925=705%:%
%:%1926=706%:%
%:%1927=706%:%
%:%1928=707%:%
%:%1929=707%:%
%:%1930=708%:%
%:%1931=708%:%
%:%1932=709%:%
%:%1933=709%:%
%:%1934=710%:%
%:%1935=710%:%
%:%1936=711%:%
%:%1937=712%:%
%:%1938=713%:%
%:%1939=713%:%
%:%1940=713%:%
%:%1941=714%:%
%:%1942=714%:%
%:%1943=715%:%
%:%1944=715%:%
%:%1945=716%:%
%:%1946=716%:%
%:%1947=717%:%
%:%1948=717%:%
%:%1949=718%:%
%:%1950=718%:%
%:%1951=719%:%
%:%1952=719%:%
%:%1953=720%:%
%:%1954=720%:%
%:%1955=720%:%
%:%1956=720%:%
%:%1957=721%:%
%:%1958=721%:%
%:%1959=722%:%
%:%1960=722%:%
%:%1961=722%:%
%:%1962=723%:%
%:%1963=723%:%
%:%1964=724%:%
%:%1965=724%:%
%:%1966=725%:%
%:%1967=725%:%
%:%1968=726%:%
%:%1969=726%:%
%:%1970=727%:%
%:%1971=727%:%
%:%1972=727%:%
%:%1973=728%:%
%:%1974=728%:%
%:%1975=729%:%
%:%1976=729%:%
%:%1977=730%:%
%:%1978=730%:%
%:%1979=731%:%
%:%1980=731%:%
%:%1981=731%:%
%:%1982=732%:%
%:%1983=732%:%
%:%1984=733%:%
%:%1985=733%:%
%:%1986=734%:%
%:%1987=734%:%
%:%1988=735%:%
%:%1989=735%:%
%:%1991=737%:%
%:%1992=738%:%
%:%1993=738%:%
%:%1994=738%:%
%:%1995=739%:%
%:%1996=739%:%
%:%1997=740%:%
%:%1998=740%:%
%:%1999=741%:%
%:%2000=741%:%
%:%2001=742%:%
%:%2002=742%:%
%:%2003=742%:%
%:%2004=742%:%
%:%2005=743%:%
%:%2011=743%:%
%:%2014=744%:%
%:%2015=745%:%
%:%2016=746%:%
%:%2017=746%:%
%:%2018=747%:%
%:%2019=748%:%
%:%2022=749%:%
%:%2026=749%:%
%:%2027=749%:%
%:%2028=749%:%
%:%2029=750%:%
%:%2030=750%:%
%:%2035=750%:%
%:%2038=751%:%
%:%2039=752%:%
%:%2040=752%:%
%:%2041=753%:%
%:%2042=754%:%
%:%2045=755%:%
%:%2049=755%:%
%:%2050=755%:%
%:%2051=755%:%
%:%2052=756%:%
%:%2053=756%:%
%:%2058=756%:%
%:%2061=757%:%
%:%2062=758%:%
%:%2063=758%:%
%:%2064=759%:%
%:%2065=760%:%
%:%2066=761%:%
%:%2067=762%:%
%:%2070=763%:%
%:%2074=763%:%
%:%2075=763%:%
%:%2076=763%:%
%:%2077=763%:%
%:%2082=763%:%
%:%2085=764%:%
%:%2086=765%:%
%:%2087=765%:%
%:%2088=766%:%
%:%2089=767%:%
%:%2090=768%:%
%:%2093=769%:%
%:%2097=769%:%
%:%2098=769%:%
%:%2099=770%:%
%:%2100=770%:%
%:%2101=770%:%
%:%2106=770%:%
%:%2109=771%:%
%:%2110=772%:%
%:%2111=772%:%
%:%2112=773%:%
%:%2113=774%:%
%:%2114=775%:%
%:%2117=776%:%
%:%2121=776%:%
%:%2122=776%:%
%:%2123=777%:%
%:%2124=777%:%
%:%2125=778%:%
%:%2126=778%:%
%:%2131=778%:%
%:%2134=779%:%
%:%2135=780%:%
%:%2136=781%:%
%:%2137=781%:%
%:%2138=782%:%
%:%2141=783%:%
%:%2145=783%:%
%:%2146=783%:%
%:%2147=784%:%
%:%2148=784%:%
%:%2149=785%:%
%:%2150=785%:%
%:%2155=785%:%
%:%2158=786%:%
%:%2159=787%:%
%:%2160=787%:%
%:%2161=788%:%
%:%2164=789%:%
%:%2168=789%:%
%:%2169=789%:%
%:%2170=789%:%
%:%2171=789%:%
%:%2176=789%:%
%:%2179=790%:%
%:%2180=791%:%
%:%2181=791%:%
%:%2182=792%:%
%:%2185=793%:%
%:%2189=793%:%
%:%2190=793%:%
%:%2191=793%:%
%:%2192=793%:%
%:%2197=793%:%
%:%2200=794%:%
%:%2201=795%:%
%:%2202=795%:%
%:%2203=796%:%
%:%2204=797%:%
%:%2205=798%:%
%:%2206=799%:%
%:%2213=800%:%
%:%2214=800%:%
%:%2215=801%:%
%:%2216=801%:%
%:%2217=802%:%
%:%2218=802%:%
%:%2219=803%:%
%:%2220=803%:%
%:%2221=803%:%
%:%2222=804%:%
%:%2223=804%:%
%:%2224=805%:%
%:%2225=805%:%
%:%2226=806%:%
%:%2227=806%:%
%:%2228=806%:%
%:%2229=807%:%
%:%2230=807%:%
%:%2231=808%:%
%:%2232=808%:%
%:%2233=808%:%
%:%2234=808%:%
%:%2235=809%:%
%:%2236=809%:%
%:%2237=810%:%
%:%2243=810%:%
%:%2246=811%:%
%:%2247=812%:%
%:%2248=812%:%
%:%2249=813%:%
%:%2250=814%:%
%:%2251=815%:%
%:%2252=816%:%
%:%2259=817%:%
%:%2260=817%:%
%:%2261=818%:%
%:%2262=818%:%
%:%2263=819%:%
%:%2264=819%:%
%:%2265=820%:%
%:%2266=820%:%
%:%2267=821%:%
%:%2268=821%:%
%:%2269=821%:%
%:%2270=822%:%
%:%2271=822%:%
%:%2272=822%:%
%:%2273=823%:%
%:%2279=823%:%
%:%2282=824%:%
%:%2283=825%:%
%:%2284=825%:%
%:%2287=826%:%
%:%2291=826%:%
%:%2292=826%:%
%:%2293=827%:%
%:%2294=827%:%
%:%2295=827%:%
%:%2300=827%:%
%:%2303=828%:%
%:%2304=829%:%
%:%2305=829%:%
%:%2308=830%:%
%:%2312=830%:%
%:%2313=830%:%
%:%2314=831%:%
%:%2315=831%:%
%:%2316=831%:%
%:%2321=831%:%
%:%2324=832%:%
%:%2325=833%:%
%:%2326=833%:%
%:%2327=834%:%
%:%2330=835%:%
%:%2334=835%:%
%:%2335=835%:%
%:%2336=835%:%
%:%2342=835%:%
%:%2345=836%:%
%:%2346=837%:%
%:%2347=837%:%
%:%2348=838%:%
%:%2349=839%:%
%:%2350=840%:%
%:%2353=841%:%
%:%2357=841%:%
%:%2358=841%:%
%:%2359=841%:%
%:%2360=842%:%
%:%2361=842%:%
%:%2366=842%:%
%:%2369=843%:%
%:%2370=844%:%
%:%2371=844%:%
%:%2372=845%:%
%:%2373=846%:%
%:%2380=847%:%
%:%2381=847%:%
%:%2382=848%:%
%:%2383=848%:%
%:%2384=849%:%
%:%2385=849%:%
%:%2386=850%:%
%:%2387=850%:%
%:%2388=850%:%
%:%2389=851%:%
%:%2390=851%:%
%:%2391=852%:%
%:%2392=852%:%
%:%2393=853%:%
%:%2394=854%:%
%:%2395=854%:%
%:%2396=854%:%
%:%2397=855%:%
%:%2398=855%:%
%:%2399=855%:%
%:%2400=856%:%
%:%2401=856%:%
%:%2402=857%:%
%:%2403=857%:%
%:%2404=858%:%
%:%2405=858%:%
%:%2406=859%:%
%:%2407=859%:%
%:%2408=860%:%
%:%2409=860%:%
%:%2410=860%:%
%:%2411=861%:%
%:%2412=861%:%
%:%2413=862%:%
%:%2414=862%:%
%:%2415=863%:%
%:%2416=863%:%
%:%2417=863%:%
%:%2418=864%:%
%:%2419=864%:%
%:%2420=865%:%
%:%2421=865%:%
%:%2422=865%:%
%:%2423=865%:%
%:%2424=866%:%
%:%2425=866%:%
%:%2426=867%:%
%:%2427=867%:%
%:%2428=868%:%
%:%2429=868%:%
%:%2430=869%:%
%:%2431=869%:%
%:%2432=870%:%
%:%2433=870%:%
%:%2434=871%:%
%:%2435=871%:%
%:%2436=872%:%
%:%2437=872%:%
%:%2438=873%:%
%:%2439=873%:%
%:%2440=873%:%
%:%2441=874%:%
%:%2442=874%:%
%:%2443=875%:%
%:%2444=875%:%
%:%2445=876%:%
%:%2451=876%:%
%:%2454=877%:%
%:%2455=878%:%
%:%2456=878%:%
%:%2457=879%:%
%:%2458=880%:%
%:%2459=881%:%
%:%2466=882%:%
%:%2467=882%:%
%:%2468=883%:%
%:%2469=883%:%
%:%2470=884%:%
%:%2471=884%:%
%:%2472=885%:%
%:%2473=885%:%
%:%2474=886%:%
%:%2475=887%:%
%:%2476=887%:%
%:%2477=887%:%
%:%2478=888%:%
%:%2479=888%:%
%:%2480=889%:%
%:%2481=889%:%
%:%2482=890%:%
%:%2483=891%:%
%:%2484=891%:%
%:%2485=891%:%
%:%2486=892%:%
%:%2487=892%:%
%:%2488=893%:%
%:%2489=893%:%
%:%2490=894%:%
%:%2491=894%:%
%:%2492=894%:%
%:%2493=895%:%
%:%2494=895%:%
%:%2495=896%:%
%:%2496=896%:%
%:%2497=897%:%
%:%2498=897%:%
%:%2499=897%:%
%:%2500=898%:%
%:%2501=898%:%
%:%2502=898%:%
%:%2503=898%:%
%:%2504=898%:%
%:%2505=899%:%
%:%2511=899%:%
%:%2514=900%:%
%:%2515=901%:%
%:%2516=901%:%
%:%2517=902%:%
%:%2518=903%:%
%:%2525=904%:%
%:%2526=904%:%
%:%2527=905%:%
%:%2528=905%:%
%:%2529=906%:%
%:%2530=906%:%
%:%2531=907%:%
%:%2532=907%:%
%:%2533=908%:%
%:%2534=908%:%
%:%2535=909%:%
%:%2536=909%:%
%:%2537=910%:%
%:%2538=910%:%
%:%2539=911%:%
%:%2540=912%:%
%:%2541=912%:%
%:%2542=913%:%
%:%2543=913%:%
%:%2544=913%:%
%:%2545=914%:%
%:%2546=914%:%
%:%2547=915%:%
%:%2548=915%:%
%:%2549=916%:%
%:%2550=917%:%
%:%2551=917%:%
%:%2552=918%:%
%:%2553=918%:%
%:%2554=918%:%
%:%2555=919%:%
%:%2556=919%:%
%:%2557=920%:%
%:%2558=920%:%
%:%2559=921%:%
%:%2560=921%:%
%:%2561=922%:%
%:%2562=922%:%
%:%2563=922%:%
%:%2564=922%:%
%:%2565=923%:%
%:%2566=923%:%
%:%2567=924%:%
%:%2568=924%:%
%:%2569=924%:%
%:%2570=924%:%
%:%2571=925%:%
%:%2577=925%:%
%:%2580=926%:%
%:%2581=927%:%
%:%2582=927%:%
%:%2583=928%:%
%:%2584=929%:%
%:%2591=930%:%
%:%2592=930%:%
%:%2593=931%:%
%:%2594=931%:%
%:%2595=932%:%
%:%2596=932%:%
%:%2597=933%:%
%:%2598=933%:%
%:%2599=934%:%
%:%2600=934%:%
%:%2601=935%:%
%:%2602=935%:%
%:%2603=936%:%
%:%2604=936%:%
%:%2605=937%:%
%:%2606=937%:%
%:%2607=938%:%
%:%2608=939%:%
%:%2609=939%:%
%:%2610=939%:%
%:%2611=940%:%
%:%2612=940%:%
%:%2613=941%:%
%:%2614=941%:%
%:%2615=942%:%
%:%2616=942%:%
%:%2617=942%:%
%:%2618=943%:%
%:%2619=943%:%
%:%2620=944%:%
%:%2621=944%:%
%:%2622=945%:%
%:%2623=945%:%
%:%2624=945%:%
%:%2625=946%:%
%:%2626=946%:%
%:%2627=947%:%
%:%2628=947%:%
%:%2629=947%:%
%:%2630=947%:%
%:%2631=948%:%
%:%2637=948%:%
%:%2640=949%:%
%:%2641=950%:%
%:%2642=950%:%
%:%2643=951%:%
%:%2644=952%:%
%:%2647=953%:%
%:%2651=953%:%
%:%2652=953%:%
%:%2653=954%:%
%:%2654=954%:%
%:%2655=955%:%
%:%2656=955%:%
%:%2661=955%:%
%:%2664=956%:%
%:%2665=957%:%
%:%2666=957%:%
%:%2667=958%:%
%:%2668=959%:%
%:%2675=960%:%
%:%2676=960%:%
%:%2677=961%:%
%:%2678=961%:%
%:%2679=962%:%
%:%2680=962%:%
%:%2681=963%:%
%:%2682=963%:%
%:%2683=963%:%
%:%2684=964%:%
%:%2685=964%:%
%:%2686=965%:%
%:%2687=965%:%
%:%2688=966%:%
%:%2689=966%:%
%:%2690=966%:%
%:%2691=967%:%
%:%2697=967%:%
%:%2700=968%:%
%:%2701=969%:%
%:%2702=969%:%
%:%2703=970%:%
%:%2704=971%:%
%:%2707=972%:%
%:%2711=972%:%
%:%2712=972%:%
%:%2713=973%:%
%:%2714=973%:%
%:%2715=973%:%
%:%2720=973%:%
%:%2723=974%:%
%:%2724=975%:%
%:%2725=975%:%
%:%2726=976%:%
%:%2727=977%:%
%:%2730=978%:%
%:%2734=978%:%
%:%2735=978%:%
%:%2736=979%:%
%:%2737=979%:%
%:%2738=979%:%
%:%2743=979%:%
%:%2746=980%:%
%:%2747=981%:%
%:%2748=981%:%
%:%2749=982%:%
%:%2750=983%:%
%:%2753=984%:%
%:%2757=984%:%
%:%2758=984%:%
%:%2759=985%:%
%:%2760=985%:%
%:%2765=985%:%
%:%2768=986%:%
%:%2769=987%:%
%:%2770=987%:%
%:%2771=988%:%
%:%2772=989%:%
%:%2775=990%:%
%:%2779=990%:%
%:%2780=990%:%
%:%2781=990%:%
%:%2782=990%:%
%:%2787=990%:%
%:%2790=991%:%
%:%2791=992%:%
%:%2792=992%:%
%:%2793=993%:%
%:%2796=994%:%
%:%2800=994%:%
%:%2801=994%:%
%:%2802=995%:%
%:%2803=995%:%
%:%2804=996%:%
%:%2805=996%:%
%:%2806=997%:%
%:%2807=997%:%
%:%2808=998%:%
%:%2809=998%:%
%:%2810=999%:%
%:%2811=999%:%
%:%2812=1000%:%
%:%2813=1000%:%
%:%2814=1000%:%
%:%2815=1000%:%
%:%2816=1001%:%
%:%2817=1001%:%
%:%2818=1002%:%
%:%2819=1002%:%
%:%2820=1003%:%
%:%2821=1003%:%
%:%2822=1004%:%
%:%2823=1005%:%
%:%2824=1006%:%
%:%2825=1007%:%
%:%2826=1008%:%
%:%2827=1008%:%
%:%2828=1009%:%
%:%2829=1009%:%
%:%2830=1010%:%
%:%2831=1010%:%
%:%2832=1011%:%
%:%2833=1011%:%
%:%2834=1012%:%
%:%2835=1012%:%
%:%2836=1013%:%
%:%2837=1013%:%
%:%2838=1014%:%
%:%2839=1014%:%
%:%2840=1014%:%
%:%2841=1015%:%
%:%2842=1015%:%
%:%2843=1016%:%
%:%2844=1016%:%
%:%2845=1017%:%
%:%2846=1017%:%
%:%2847=1017%:%
%:%2848=1018%:%
%:%2849=1018%:%
%:%2850=1019%:%
%:%2851=1019%:%
%:%2852=1020%:%
%:%2853=1020%:%
%:%2854=1020%:%
%:%2855=1020%:%
%:%2856=1021%:%
%:%2857=1021%:%
%:%2858=1022%:%
%:%2859=1022%:%
%:%2860=1022%:%
%:%2861=1023%:%
%:%2862=1023%:%
%:%2863=1024%:%
%:%2864=1024%:%
%:%2865=1024%:%
%:%2866=1025%:%
%:%2867=1025%:%
%:%2868=1026%:%
%:%2869=1026%:%
%:%2870=1027%:%
%:%2871=1027%:%
%:%2872=1028%:%
%:%2873=1028%:%
%:%2874=1028%:%
%:%2875=1029%:%
%:%2876=1030%:%
%:%2877=1031%:%
%:%2878=1032%:%
%:%2879=1033%:%
%:%2880=1034%:%
%:%2881=1034%:%
%:%2882=1034%:%
%:%2883=1035%:%
%:%2889=1035%:%
%:%2892=1036%:%
%:%2893=1037%:%
%:%2894=1037%:%
%:%2895=1038%:%
%:%2896=1039%:%
%:%2899=1040%:%
%:%2903=1040%:%
%:%2904=1040%:%
%:%2905=1041%:%
%:%2906=1041%:%
%:%2911=1041%:%
%:%2914=1042%:%
%:%2915=1043%:%
%:%2916=1043%:%
%:%2917=1044%:%
%:%2918=1045%:%
%:%2921=1046%:%
%:%2925=1046%:%
%:%2926=1046%:%
%:%2927=1047%:%
%:%2928=1047%:%
%:%2929=1048%:%
%:%2930=1048%:%
%:%2931=1049%:%
%:%2932=1049%:%
%:%2933=1050%:%
%:%2934=1050%:%
%:%2935=1050%:%
%:%2936=1051%:%
%:%2937=1051%:%
%:%2938=1052%:%
%:%2939=1052%:%
%:%2940=1053%:%
%:%2941=1054%:%
%:%2942=1055%:%
%:%2943=1056%:%
%:%2944=1057%:%
%:%2945=1057%:%
%:%2946=1057%:%
%:%2947=1058%:%
%:%2953=1058%:%
%:%2956=1059%:%
%:%2957=1060%:%
%:%2958=1060%:%
%:%2959=1061%:%
%:%2960=1062%:%
%:%2963=1063%:%
%:%2967=1063%:%
%:%2968=1063%:%
%:%2969=1063%:%
%:%2974=1063%:%
%:%2977=1064%:%
%:%2978=1065%:%
%:%2979=1065%:%
%:%2980=1066%:%
%:%2983=1067%:%
%:%2987=1067%:%
%:%2988=1067%:%
%:%2989=1067%:%
%:%2990=1067%:%
%:%2995=1067%:%
%:%2998=1068%:%
%:%2999=1069%:%
%:%3000=1069%:%
%:%3001=1070%:%
%:%3004=1071%:%
%:%3008=1071%:%
%:%3009=1071%:%
%:%3010=1072%:%
%:%3011=1072%:%
%:%3012=1073%:%
%:%3013=1073%:%
%:%3014=1074%:%
%:%3015=1074%:%
%:%3020=1074%:%
%:%3023=1075%:%
%:%3024=1076%:%
%:%3025=1077%:%
%:%3026=1077%:%
%:%3029=1078%:%
%:%3033=1078%:%
%:%3034=1078%:%
%:%3035=1078%:%
%:%3036=1078%:%
%:%3041=1078%:%
%:%3044=1079%:%
%:%3045=1080%:%
%:%3046=1081%:%
%:%3047=1081%:%
%:%3048=1082%:%
%:%3049=1083%:%
%:%3056=1084%:%
%:%3057=1084%:%
%:%3058=1085%:%
%:%3059=1085%:%
%:%3060=1086%:%
%:%3061=1086%:%
%:%3062=1087%:%
%:%3063=1087%:%
%:%3064=1088%:%
%:%3065=1088%:%
%:%3066=1089%:%
%:%3067=1089%:%
%:%3068=1089%:%
%:%3069=1090%:%
%:%3070=1090%:%
%:%3071=1091%:%
%:%3072=1091%:%
%:%3073=1092%:%
%:%3074=1092%:%
%:%3075=1093%:%
%:%3076=1093%:%
%:%3077=1093%:%
%:%3078=1094%:%
%:%3079=1094%:%
%:%3080=1095%:%
%:%3081=1095%:%
%:%3082=1096%:%
%:%3083=1096%:%
%:%3084=1097%:%
%:%3085=1097%:%
%:%3086=1098%:%
%:%3087=1098%:%
%:%3088=1099%:%
%:%3094=1099%:%
%:%3097=1100%:%
%:%3098=1101%:%
%:%3099=1101%:%
%:%3100=1102%:%
%:%3101=1103%:%
%:%3108=1104%:%
%:%3109=1104%:%
%:%3110=1105%:%
%:%3111=1105%:%
%:%3112=1106%:%
%:%3113=1106%:%
%:%3114=1107%:%
%:%3115=1107%:%
%:%3116=1108%:%
%:%3117=1108%:%
%:%3118=1109%:%
%:%3119=1109%:%
%:%3120=1109%:%
%:%3121=1110%:%
%:%3122=1110%:%
%:%3123=1111%:%
%:%3124=1111%:%
%:%3125=1112%:%
%:%3126=1112%:%
%:%3127=1113%:%
%:%3128=1113%:%
%:%3129=1113%:%
%:%3130=1114%:%
%:%3131=1114%:%
%:%3132=1115%:%
%:%3133=1115%:%
%:%3134=1116%:%
%:%3135=1116%:%
%:%3136=1117%:%
%:%3137=1117%:%
%:%3138=1118%:%
%:%3139=1118%:%
%:%3140=1119%:%
%:%3146=1119%:%
%:%3149=1120%:%
%:%3150=1121%:%
%:%3151=1121%:%
%:%3152=1122%:%
%:%3153=1123%:%
%:%3160=1124%:%
%:%3161=1124%:%
%:%3162=1125%:%
%:%3163=1125%:%
%:%3164=1126%:%
%:%3165=1126%:%
%:%3166=1127%:%
%:%3167=1127%:%
%:%3168=1128%:%
%:%3169=1128%:%
%:%3170=1129%:%
%:%3171=1129%:%
%:%3172=1129%:%
%:%3173=1130%:%
%:%3174=1130%:%
%:%3175=1131%:%
%:%3176=1131%:%
%:%3177=1132%:%
%:%3178=1132%:%
%:%3179=1133%:%
%:%3180=1133%:%
%:%3181=1133%:%
%:%3182=1134%:%
%:%3183=1134%:%
%:%3184=1135%:%
%:%3185=1135%:%
%:%3186=1136%:%
%:%3187=1136%:%
%:%3188=1137%:%
%:%3189=1137%:%
%:%3190=1138%:%
%:%3191=1138%:%
%:%3192=1139%:%
%:%3198=1139%:%
%:%3201=1140%:%
%:%3202=1141%:%
%:%3203=1141%:%
%:%3204=1142%:%
%:%3205=1143%:%
%:%3206=1144%:%
%:%3207=1145%:%
%:%3208=1145%:%
%:%3209=1146%:%
%:%3210=1147%:%
%:%3217=1148%:%
%:%3218=1148%:%
%:%3219=1149%:%
%:%3220=1149%:%
%:%3221=1150%:%
%:%3222=1150%:%
%:%3223=1151%:%
%:%3224=1151%:%
%:%3225=1152%:%
%:%3226=1152%:%
%:%3227=1152%:%
%:%3228=1153%:%
%:%3229=1153%:%
%:%3230=1154%:%
%:%3231=1154%:%
%:%3232=1155%:%
%:%3233=1155%:%
%:%3234=1155%:%
%:%3235=1156%:%
%:%3236=1156%:%
%:%3237=1157%:%
%:%3238=1157%:%
%:%3239=1158%:%
%:%3240=1159%:%
%:%3241=1159%:%
%:%3242=1160%:%
%:%3243=1160%:%
%:%3244=1161%:%
%:%3245=1161%:%
%:%3246=1162%:%
%:%3247=1162%:%
%:%3248=1163%:%
%:%3249=1163%:%
%:%3250=1164%:%
%:%3251=1164%:%
%:%3252=1165%:%
%:%3253=1165%:%
%:%3254=1165%:%
%:%3255=1166%:%
%:%3256=1166%:%
%:%3257=1166%:%
%:%3258=1167%:%
%:%3273=1169%:%
%:%3283=1171%:%
%:%3284=1171%:%
%:%3285=1172%:%
%:%3286=1173%:%
%:%3287=1174%:%
%:%3292=1179%:%
%:%3299=1180%:%
%:%3300=1180%:%
%:%3301=1181%:%
%:%3302=1181%:%
%:%3303=1182%:%
%:%3304=1182%:%
%:%3305=1183%:%
%:%3306=1183%:%
%:%3307=1184%:%
%:%3308=1184%:%
%:%3309=1185%:%
%:%3310=1185%:%
%:%3311=1185%:%
%:%3312=1186%:%
%:%3313=1186%:%
%:%3314=1187%:%
%:%3315=1187%:%
%:%3316=1188%:%
%:%3317=1188%:%
%:%3318=1188%:%
%:%3319=1189%:%
%:%3320=1189%:%
%:%3321=1190%:%
%:%3322=1190%:%
%:%3323=1191%:%
%:%3324=1191%:%
%:%3325=1191%:%
%:%3326=1192%:%
%:%3327=1192%:%
%:%3328=1193%:%
%:%3329=1193%:%
%:%3330=1194%:%
%:%3331=1194%:%
%:%3332=1195%:%
%:%3333=1195%:%
%:%3334=1196%:%
%:%3335=1196%:%
%:%3336=1197%:%
%:%3351=1200%:%
%:%3361=1202%:%
%:%3362=1202%:%
%:%3365=1203%:%
%:%3369=1203%:%
%:%3370=1203%:%
%:%3371=1203%:%
%:%3377=1203%:%
%:%3380=1204%:%
%:%3381=1205%:%
%:%3382=1205%:%
%:%3385=1206%:%
%:%3389=1206%:%
%:%3390=1206%:%
%:%3391=1206%:%
%:%3397=1206%:%
%:%3400=1207%:%
%:%3401=1208%:%
%:%3402=1209%:%
%:%3403=1209%:%
%:%3404=1210%:%
%:%3405=1211%:%
%:%3406=1212%:%
%:%3407=1213%:%
%:%3410=1214%:%
%:%3414=1214%:%
%:%3415=1214%:%
%:%3416=1215%:%
%:%3417=1215%:%
%:%3418=1216%:%
%:%3419=1216%:%
%:%3420=1217%:%
%:%3421=1217%:%
%:%3422=1218%:%
%:%3423=1218%:%
%:%3424=1218%:%
%:%3425=1219%:%
%:%3426=1219%:%
%:%3427=1220%:%
%:%3428=1220%:%
%:%3429=1221%:%
%:%3430=1221%:%
%:%3431=1222%:%
%:%3432=1222%:%
%:%3433=1223%:%
%:%3434=1223%:%
%:%3435=1224%:%
%:%3436=1224%:%
%:%3437=1225%:%
%:%3438=1226%:%
%:%3439=1226%:%
%:%3440=1227%:%
%:%3441=1227%:%
%:%3442=1228%:%
%:%3443=1228%:%
%:%3444=1228%:%
%:%3445=1229%:%
%:%3446=1229%:%
%:%3447=1230%:%
%:%3448=1230%:%
%:%3449=1231%:%
%:%3450=1232%:%
%:%3451=1232%:%
%:%3452=1232%:%
%:%3453=1233%:%
%:%3454=1233%:%
%:%3455=1234%:%
%:%3456=1234%:%
%:%3457=1235%:%
%:%3458=1235%:%
%:%3459=1235%:%
%:%3460=1236%:%
%:%3461=1236%:%
%:%3462=1237%:%
%:%3463=1237%:%
%:%3464=1237%:%
%:%3465=1238%:%
%:%3466=1239%:%
%:%3467=1239%:%
%:%3468=1240%:%
%:%3469=1240%:%
%:%3470=1241%:%
%:%3471=1241%:%
%:%3472=1242%:%
%:%3473=1242%:%
%:%3474=1243%:%
%:%3475=1243%:%
%:%3476=1243%:%
%:%3477=1244%:%
%:%3483=1244%:%
%:%3486=1245%:%
%:%3487=1246%:%
%:%3488=1246%:%
%:%3489=1247%:%
%:%3492=1248%:%
%:%3496=1248%:%
%:%3497=1248%:%
%:%3498=1248%:%
%:%3499=1249%:%
%:%3500=1249%:%
%:%3505=1249%:%
%:%3508=1250%:%
%:%3509=1251%:%
%:%3510=1251%:%
%:%3511=1252%:%
%:%3514=1253%:%
%:%3518=1253%:%
%:%3519=1253%:%
%:%3520=1253%:%
%:%3521=1253%:%
%:%3526=1253%:%
%:%3529=1254%:%
%:%3530=1255%:%
%:%3531=1255%:%
%:%3532=1256%:%
%:%3533=1257%:%
%:%3534=1258%:%
%:%3535=1259%:%
%:%3542=1260%:%
%:%3543=1260%:%
%:%3544=1261%:%
%:%3545=1261%:%
%:%3546=1262%:%
%:%3547=1263%:%
%:%3548=1264%:%
%:%3549=1265%:%
%:%3550=1265%:%
%:%3551=1266%:%
%:%3552=1267%:%
%:%3553=1267%:%
%:%3554=1268%:%
%:%3555=1268%:%
%:%3556=1269%:%
%:%3557=1270%:%
%:%3558=1271%:%
%:%3559=1271%:%
%:%3560=1272%:%
%:%3561=1273%:%
%:%3562=1273%:%
%:%3563=1274%:%
%:%3564=1274%:%
%:%3565=1275%:%
%:%3566=1276%:%
%:%3567=1276%:%
%:%3568=1277%:%
%:%3569=1278%:%
%:%3570=1279%:%
%:%3571=1279%:%
%:%3572=1279%:%
%:%3573=1280%:%
%:%3574=1280%:%
%:%3575=1281%:%
%:%3576=1281%:%
%:%3577=1282%:%
%:%3578=1283%:%
%:%3579=1284%:%
%:%3580=1285%:%
%:%3581=1285%:%
%:%3582=1286%:%
%:%3583=1286%:%
%:%3584=1287%:%
%:%3585=1287%:%
%:%3586=1288%:%
%:%3587=1288%:%
%:%3588=1288%:%
%:%3589=1289%:%
%:%3590=1289%:%
%:%3591=1290%:%
%:%3592=1290%:%
%:%3593=1291%:%
%:%3594=1291%:%
%:%3595=1291%:%
%:%3596=1292%:%
%:%3597=1292%:%
%:%3598=1293%:%
%:%3599=1293%:%
%:%3600=1294%:%
%:%3601=1295%:%
%:%3602=1295%:%
%:%3603=1295%:%
%:%3604=1296%:%
%:%3605=1296%:%
%:%3606=1297%:%
%:%3607=1297%:%
%:%3608=1297%:%
%:%3609=1297%:%
%:%3610=1298%:%
%:%3616=1298%:%
%:%3619=1299%:%
%:%3620=1300%:%
%:%3621=1300%:%
%:%3622=1301%:%
%:%3625=1302%:%
%:%3629=1302%:%
%:%3630=1302%:%
%:%3631=1303%:%
%:%3632=1303%:%
%:%3633=1304%:%
%:%3634=1304%:%
%:%3639=1304%:%
%:%3642=1305%:%
%:%3643=1306%:%
%:%3644=1306%:%
%:%3645=1307%:%
%:%3648=1308%:%
%:%3652=1308%:%
%:%3653=1308%:%
%:%3654=1309%:%
%:%3655=1309%:%
%:%3656=1310%:%
%:%3657=1310%:%
%:%3662=1310%:%
%:%3665=1311%:%
%:%3666=1312%:%
%:%3667=1312%:%
%:%3668=1313%:%
%:%3669=1314%:%
%:%3670=1315%:%
%:%3673=1316%:%
%:%3677=1316%:%
%:%3678=1316%:%
%:%3679=1316%:%
%:%3680=1316%:%
%:%3685=1316%:%
%:%3688=1317%:%
%:%3689=1318%:%
%:%3690=1318%:%
%:%3691=1319%:%
%:%3694=1320%:%
%:%3698=1320%:%
%:%3699=1320%:%
%:%3700=1321%:%
%:%3701=1321%:%
%:%3702=1322%:%
%:%3703=1323%:%
%:%3704=1323%:%
%:%3709=1323%:%
%:%3712=1324%:%
%:%3713=1325%:%
%:%3714=1325%:%
%:%3715=1326%:%
%:%3716=1327%:%
%:%3723=1328%:%
%:%3724=1328%:%
%:%3725=1329%:%
%:%3726=1329%:%
%:%3727=1330%:%
%:%3728=1330%:%
%:%3729=1331%:%
%:%3730=1332%:%
%:%3731=1332%:%
%:%3732=1332%:%
%:%3733=1333%:%
%:%3734=1333%:%
%:%3735=1334%:%
%:%3736=1334%:%
%:%3737=1335%:%
%:%3738=1335%:%
%:%3739=1336%:%
%:%3740=1336%:%
%:%3741=1337%:%
%:%3742=1338%:%
%:%3743=1339%:%
%:%3744=1339%:%
%:%3745=1340%:%
%:%3751=1340%:%
%:%3754=1341%:%
%:%3755=1342%:%
%:%3756=1342%:%
%:%3757=1343%:%
%:%3760=1344%:%
%:%3764=1344%:%
%:%3765=1344%:%
%:%3766=1345%:%
%:%3767=1345%:%
%:%3768=1345%:%
%:%3773=1345%:%
%:%3776=1346%:%
%:%3777=1347%:%
%:%3778=1347%:%
%:%3779=1348%:%
%:%3782=1349%:%
%:%3786=1349%:%
%:%3787=1349%:%
%:%3788=1350%:%
%:%3789=1350%:%
%:%3790=1350%:%
%:%3795=1350%:%
%:%3798=1351%:%
%:%3799=1352%:%
%:%3800=1352%:%
%:%3801=1353%:%
%:%3802=1354%:%
%:%3803=1355%:%
%:%3804=1356%:%
%:%3811=1357%:%
%:%3812=1357%:%
%:%3813=1358%:%
%:%3814=1358%:%
%:%3815=1359%:%
%:%3816=1359%:%
%:%3817=1359%:%
%:%3818=1360%:%
%:%3819=1360%:%
%:%3820=1360%:%
%:%3821=1361%:%
%:%3822=1361%:%
%:%3823=1362%:%
%:%3824=1362%:%
%:%3825=1363%:%
%:%3826=1363%:%
%:%3827=1364%:%
%:%3833=1364%:%
%:%3836=1365%:%
%:%3837=1366%:%
%:%3838=1367%:%
%:%3839=1367%:%
%:%3840=1368%:%
%:%3841=1369%:%
%:%3842=1370%:%
%:%3843=1371%:%
%:%3850=1372%:%
%:%3851=1372%:%
%:%3852=1373%:%
%:%3853=1373%:%
%:%3854=1374%:%
%:%3855=1374%:%
%:%3856=1374%:%
%:%3857=1375%:%
%:%3858=1375%:%
%:%3859=1375%:%
%:%3860=1376%:%
%:%3861=1376%:%
%:%3862=1377%:%
%:%3863=1377%:%
%:%3864=1378%:%
%:%3865=1378%:%
%:%3866=1379%:%
%:%3872=1379%:%
%:%3875=1380%:%
%:%3876=1381%:%
%:%3884=1383%:%
%:%3894=1385%:%
%:%3895=1385%:%
%:%3896=1386%:%
%:%3897=1387%:%
%:%3898=1388%:%
%:%3899=1389%:%
%:%3900=1389%:%
%:%3901=1390%:%
%:%3904=1391%:%
%:%3908=1391%:%
%:%3909=1391%:%
%:%3910=1392%:%
%:%3911=1392%:%
%:%3916=1392%:%
%:%3919=1393%:%
%:%3920=1394%:%
%:%3921=1394%:%
%:%3922=1395%:%
%:%3923=1396%:%
%:%3930=1397%:%
%:%3931=1397%:%
%:%3932=1398%:%
%:%3933=1398%:%
%:%3934=1399%:%
%:%3935=1399%:%
%:%3936=1400%:%
%:%3937=1400%:%
%:%3938=1401%:%
%:%3939=1401%:%
%:%3940=1402%:%
%:%3941=1402%:%
%:%3942=1403%:%
%:%3943=1403%:%
%:%3944=1404%:%
%:%3945=1404%:%
%:%3946=1405%:%
%:%3947=1405%:%
%:%3948=1406%:%
%:%3949=1406%:%
%:%3950=1406%:%
%:%3951=1407%:%
%:%3952=1407%:%
%:%3953=1408%:%
%:%3954=1408%:%
%:%3955=1409%:%
%:%3956=1409%:%
%:%3957=1410%:%
%:%3958=1410%:%
%:%3959=1411%:%
%:%3960=1411%:%
%:%3961=1412%:%
%:%3962=1412%:%
%:%3963=1413%:%
%:%3964=1413%:%
%:%3965=1414%:%
%:%3966=1414%:%
%:%3967=1415%:%
%:%3968=1415%:%
%:%3969=1416%:%
%:%3970=1416%:%
%:%3971=1417%:%
%:%3972=1417%:%
%:%3973=1418%:%
%:%3974=1418%:%
%:%3975=1419%:%
%:%3976=1419%:%
%:%3977=1420%:%
%:%3978=1420%:%
%:%3979=1421%:%
%:%3980=1421%:%
%:%3981=1422%:%
%:%3982=1422%:%
%:%3983=1423%:%
%:%3984=1423%:%
%:%3985=1424%:%
%:%3986=1424%:%
%:%3987=1425%:%
%:%3988=1425%:%
%:%3989=1426%:%
%:%3990=1426%:%
%:%3991=1427%:%
%:%3992=1427%:%
%:%3993=1428%:%
%:%3994=1428%:%
%:%3995=1429%:%
%:%3996=1429%:%
%:%3997=1430%:%
%:%3998=1430%:%
%:%3999=1431%:%
%:%4000=1431%:%
%:%4001=1432%:%
%:%4007=1432%:%
%:%4010=1433%:%
%:%4011=1434%:%
%:%4012=1434%:%
%:%4013=1435%:%
%:%4014=1436%:%
%:%4017=1437%:%
%:%4021=1437%:%
%:%4022=1437%:%
%:%4023=1438%:%
%:%4024=1438%:%
%:%4025=1439%:%
%:%4026=1439%:%
%:%4031=1439%:%
%:%4034=1440%:%
%:%4035=1441%:%
%:%4036=1442%:%
%:%4037=1442%:%
%:%4038=1443%:%
%:%4039=1444%:%
%:%4042=1447%:%
%:%4043=1448%:%
%:%4044=1449%:%
%:%4045=1449%:%
%:%4046=1450%:%
%:%4047=1451%:%
%:%4054=1452%:%
%:%4055=1452%:%
%:%4056=1453%:%
%:%4057=1453%:%
%:%4058=1454%:%
%:%4059=1454%:%
%:%4060=1455%:%
%:%4061=1455%:%
%:%4062=1456%:%
%:%4063=1456%:%
%:%4064=1457%:%
%:%4065=1457%:%
%:%4066=1458%:%
%:%4067=1458%:%
%:%4068=1459%:%
%:%4069=1459%:%
%:%4070=1460%:%
%:%4071=1460%:%
%:%4072=1461%:%
%:%4073=1461%:%
%:%4074=1461%:%
%:%4075=1462%:%
%:%4076=1462%:%
%:%4077=1463%:%
%:%4078=1463%:%
%:%4079=1464%:%
%:%4080=1464%:%
%:%4081=1465%:%
%:%4082=1465%:%
%:%4083=1466%:%
%:%4084=1466%:%
%:%4085=1467%:%
%:%4086=1467%:%
%:%4087=1468%:%
%:%4088=1468%:%
%:%4089=1469%:%
%:%4090=1469%:%
%:%4091=1470%:%
%:%4092=1470%:%
%:%4093=1471%:%
%:%4094=1471%:%
%:%4095=1472%:%
%:%4096=1472%:%
%:%4097=1473%:%
%:%4098=1473%:%
%:%4099=1474%:%
%:%4100=1474%:%
%:%4101=1475%:%
%:%4102=1475%:%
%:%4103=1476%:%
%:%4104=1476%:%
%:%4105=1477%:%
%:%4106=1477%:%
%:%4107=1478%:%
%:%4108=1478%:%
%:%4109=1479%:%
%:%4110=1479%:%
%:%4111=1480%:%
%:%4112=1480%:%
%:%4113=1481%:%
%:%4114=1481%:%
%:%4115=1482%:%
%:%4116=1482%:%
%:%4117=1483%:%
%:%4118=1483%:%
%:%4119=1484%:%
%:%4120=1484%:%
%:%4121=1485%:%
%:%4122=1485%:%
%:%4123=1486%:%
%:%4124=1486%:%
%:%4125=1487%:%
%:%4131=1487%:%
%:%4134=1488%:%
%:%4135=1489%:%
%:%4136=1489%:%
%:%4137=1490%:%
%:%4140=1491%:%
%:%4144=1491%:%
%:%4145=1491%:%
%:%4146=1491%:%
%:%4151=1491%:%
%:%4154=1492%:%
%:%4155=1493%:%
%:%4156=1493%:%
%:%4157=1494%:%
%:%4158=1495%:%
%:%4161=1496%:%
%:%4165=1496%:%
%:%4166=1496%:%
%:%4167=1497%:%
%:%4168=1497%:%
%:%4169=1498%:%
%:%4170=1498%:%
%:%4175=1498%:%
%:%4178=1499%:%
%:%4179=1500%:%
%:%4180=1501%:%
%:%4181=1501%:%
%:%4182=1502%:%
%:%4183=1503%:%
%:%4184=1504%:%
%:%4185=1505%:%
%:%4186=1505%:%
%:%4187=1506%:%
%:%4190=1507%:%
%:%4194=1507%:%
%:%4195=1507%:%
%:%4196=1507%:%
%:%4197=1507%:%
%:%4202=1507%:%
%:%4205=1508%:%
%:%4206=1509%:%
%:%4207=1510%:%
%:%4208=1510%:%
%:%4209=1511%:%
%:%4210=1512%:%
%:%4211=1513%:%
%:%4212=1514%:%
%:%4213=1514%:%
%:%4214=1515%:%
%:%4217=1516%:%
%:%4221=1516%:%
%:%4222=1516%:%
%:%4223=1517%:%
%:%4224=1517%:%
%:%4225=1518%:%
%:%4226=1518%:%
%:%4231=1518%:%
%:%4234=1519%:%
%:%4235=1520%:%
%:%4236=1520%:%
%:%4237=1521%:%
%:%4240=1522%:%
%:%4244=1522%:%
%:%4245=1522%:%
%:%4246=1522%:%
%:%4251=1522%:%
%:%4254=1523%:%
%:%4255=1524%:%
%:%4256=1524%:%
%:%4257=1525%:%
%:%4258=1526%:%
%:%4261=1527%:%
%:%4265=1527%:%
%:%4266=1527%:%
%:%4267=1527%:%
%:%4281=1529%:%
%:%4291=1531%:%
%:%4292=1531%:%
%:%4293=1532%:%
%:%4294=1532%:%
%:%4295=1533%:%
%:%4296=1534%:%
%:%4297=1535%:%
%:%4299=1537%:%
%:%4300=1538%:%
%:%4301=1539%:%
%:%4302=1540%:%
%:%4303=1541%:%
%:%4304=1541%:%
%:%4305=1542%:%
%:%4306=1543%:%
%:%4307=1544%:%
%:%4308=1545%:%
%:%4309=1545%:%
%:%4312=1546%:%
%:%4316=1546%:%
%:%4317=1546%:%
%:%4322=1546%:%
%:%4325=1547%:%
%:%4326=1548%:%
%:%4327=1548%:%
%:%4330=1549%:%
%:%4334=1549%:%
%:%4335=1549%:%
%:%4336=1549%:%
%:%4341=1549%:%
%:%4344=1550%:%
%:%4345=1551%:%
%:%4346=1551%:%
%:%4347=1552%:%
%:%4354=1554%:%
%:%4364=1556%:%
%:%4365=1556%:%
%:%4366=1557%:%
%:%4367=1558%:%
%:%4368=1559%:%
%:%4369=1560%:%
%:%4370=1560%:%
%:%4371=1561%:%
%:%4372=1562%:%
%:%4373=1563%:%
%:%4374=1564%:%
%:%4375=1565%:%
%:%4376=1565%:%
%:%4377=1566%:%
%:%4378=1567%:%
%:%4379=1568%:%
%:%4380=1569%:%
%:%4381=1570%:%
%:%4382=1570%:%
%:%4383=1571%:%
%:%4384=1572%:%
%:%4385=1573%:%
%:%4386=1574%:%
%:%4387=1575%:%
%:%4388=1575%:%
%:%4390=1577%:%
%:%4393=1578%:%
%:%4397=1578%:%
%:%4398=1578%:%
%:%4399=1579%:%
%:%4400=1579%:%
%:%4401=1579%:%
%:%4402=1580%:%
%:%4403=1580%:%
%:%4408=1580%:%
%:%4411=1581%:%
%:%4412=1582%:%
%:%4413=1582%:%
%:%4415=1584%:%
%:%4418=1585%:%
%:%4422=1585%:%
%:%4423=1585%:%
%:%4424=1586%:%
%:%4425=1586%:%
%:%4426=1586%:%
%:%4427=1587%:%
%:%4428=1587%:%
%:%4433=1587%:%
%:%4436=1588%:%
%:%4437=1589%:%
%:%4438=1589%:%
%:%4439=1590%:%
%:%4442=1591%:%
%:%4446=1591%:%
%:%4447=1591%:%
%:%4448=1592%:%
%:%4449=1592%:%
%:%4450=1592%:%
%:%4455=1592%:%
%:%4458=1593%:%
%:%4459=1594%:%
%:%4460=1594%:%
%:%4461=1595%:%
%:%4464=1596%:%
%:%4468=1596%:%
%:%4469=1596%:%
%:%4470=1597%:%
%:%4471=1597%:%
%:%4472=1597%:%
%:%4477=1597%:%
%:%4480=1598%:%
%:%4481=1599%:%
%:%4482=1599%:%
%:%4483=1600%:%
%:%4484=1601%:%
%:%4485=1602%:%
%:%4486=1603%:%
%:%4489=1604%:%
%:%4493=1604%:%
%:%4494=1604%:%
%:%4495=1605%:%
%:%4496=1605%:%
%:%4497=1606%:%
%:%4498=1606%:%
%:%4503=1606%:%
%:%4506=1607%:%
%:%4507=1608%:%
%:%4508=1608%:%
%:%4509=1609%:%
%:%4510=1610%:%
%:%4511=1611%:%
%:%4512=1612%:%
%:%4515=1613%:%
%:%4519=1613%:%
%:%4520=1613%:%
%:%4521=1614%:%
%:%4522=1614%:%
%:%4523=1615%:%
%:%4524=1615%:%
%:%4529=1615%:%
%:%4532=1616%:%
%:%4533=1617%:%
%:%4534=1618%:%
%:%4535=1618%:%
%:%4536=1619%:%
%:%4537=1620%:%
%:%4538=1621%:%
%:%4539=1622%:%
%:%4540=1622%:%
%:%4541=1623%:%
%:%4542=1624%:%
%:%4543=1625%:%
%:%4544=1626%:%
%:%4545=1627%:%
%:%4546=1627%:%
%:%4547=1628%:%
%:%4550=1629%:%
%:%4554=1629%:%
%:%4555=1629%:%
%:%4556=1629%:%
%:%4561=1629%:%
%:%4564=1630%:%
%:%4565=1631%:%
%:%4566=1631%:%
%:%4567=1632%:%
%:%4570=1633%:%
%:%4574=1633%:%
%:%4575=1633%:%
%:%4576=1633%:%
%:%4581=1633%:%
%:%4584=1634%:%
%:%4585=1635%:%
%:%4586=1636%:%
%:%4587=1636%:%
%:%4588=1637%:%
%:%4589=1638%:%
%:%4590=1639%:%
%:%4591=1640%:%
%:%4594=1641%:%
%:%4598=1641%:%
%:%4599=1641%:%
%:%4600=1642%:%
%:%4601=1642%:%
%:%4602=1643%:%
%:%4603=1643%:%
%:%4608=1643%:%
%:%4611=1644%:%
%:%4612=1645%:%
%:%4613=1645%:%
%:%4614=1646%:%
%:%4615=1647%:%
%:%4616=1648%:%
%:%4617=1649%:%
%:%4620=1650%:%
%:%4624=1650%:%
%:%4625=1650%:%
%:%4626=1651%:%
%:%4627=1651%:%
%:%4628=1652%:%
%:%4629=1652%:%
%:%4634=1652%:%
%:%4637=1653%:%
%:%4638=1654%:%
%:%4639=1654%:%
%:%4640=1655%:%
%:%4641=1656%:%
%:%4643=1658%:%
%:%4646=1659%:%
%:%4650=1659%:%
%:%4651=1659%:%
%:%4652=1659%:%
%:%4653=1660%:%
%:%4654=1660%:%
%:%4659=1660%:%
%:%4662=1661%:%
%:%4663=1662%:%
%:%4664=1662%:%
%:%4665=1663%:%
%:%4666=1664%:%
%:%4668=1666%:%
%:%4671=1667%:%
%:%4675=1667%:%
%:%4676=1667%:%
%:%4677=1667%:%
%:%4678=1668%:%
%:%4679=1668%:%
%:%4684=1668%:%
%:%4687=1669%:%
%:%4688=1670%:%
%:%4689=1670%:%
%:%4690=1671%:%
%:%4691=1672%:%
%:%4692=1673%:%
%:%4695=1674%:%
%:%4699=1674%:%
%:%4700=1674%:%
%:%4701=1674%:%
%:%4702=1675%:%
%:%4703=1675%:%
%:%4708=1675%:%
%:%4711=1676%:%
%:%4712=1677%:%
%:%4720=1679%:%
%:%4730=1681%:%
%:%4731=1681%:%
%:%4732=1682%:%
%:%4733=1683%:%
%:%4734=1684%:%
%:%4737=1685%:%
%:%4741=1685%:%
%:%4742=1685%:%
%:%4743=1686%:%
%:%4744=1686%:%
%:%4745=1687%:%
%:%4746=1687%:%
%:%4751=1687%:%
%:%4754=1688%:%
%:%4755=1689%:%
%:%4756=1689%:%
%:%4757=1690%:%
%:%4758=1691%:%
%:%4761=1692%:%
%:%4765=1692%:%
%:%4766=1692%:%
%:%4767=1693%:%
%:%4768=1693%:%
%:%4769=1694%:%
%:%4770=1694%:%
%:%4771=1695%:%
%:%4772=1695%:%
%:%4773=1696%:%
%:%4774=1696%:%
%:%4775=1697%:%
%:%4776=1697%:%
%:%4777=1698%:%
%:%4778=1698%:%
%:%4779=1699%:%
%:%4780=1699%:%
%:%4781=1700%:%
%:%4782=1700%:%
%:%4783=1701%:%
%:%4784=1701%:%
%:%4785=1702%:%
%:%4786=1702%:%
%:%4787=1703%:%
%:%4788=1703%:%
%:%4789=1704%:%
%:%4790=1704%:%
%:%4791=1705%:%
%:%4792=1705%:%
%:%4793=1706%:%
%:%4794=1706%:%
%:%4795=1707%:%
%:%4796=1707%:%
%:%4797=1708%:%
%:%4798=1708%:%
%:%4799=1709%:%
%:%4800=1709%:%
%:%4801=1710%:%
%:%4802=1710%:%
%:%4803=1711%:%
%:%4804=1711%:%
%:%4805=1712%:%
%:%4806=1712%:%
%:%4807=1713%:%
%:%4808=1713%:%
%:%4809=1713%:%
%:%4810=1714%:%
%:%4811=1714%:%
%:%4812=1715%:%
%:%4813=1715%:%
%:%4814=1716%:%
%:%4815=1716%:%
%:%4816=1717%:%
%:%4817=1717%:%
%:%4818=1718%:%
%:%4819=1718%:%
%:%4820=1719%:%
%:%4821=1720%:%
%:%4822=1720%:%
%:%4823=1721%:%
%:%4824=1721%:%
%:%4825=1722%:%
%:%4826=1722%:%
%:%4827=1723%:%
%:%4828=1723%:%
%:%4829=1724%:%
%:%4830=1725%:%
%:%4831=1725%:%
%:%4832=1725%:%
%:%4833=1726%:%
%:%4834=1726%:%
%:%4835=1727%:%
%:%4836=1727%:%
%:%4837=1728%:%
%:%4838=1729%:%
%:%4839=1729%:%
%:%4840=1730%:%
%:%4841=1730%:%
%:%4842=1731%:%
%:%4843=1731%:%
%:%4844=1732%:%
%:%4845=1732%:%
%:%4846=1733%:%
%:%4847=1733%:%
%:%4848=1734%:%
%:%4849=1734%:%
%:%4850=1735%:%
%:%4851=1735%:%
%:%4852=1736%:%
%:%4853=1736%:%
%:%4854=1737%:%
%:%4855=1737%:%
%:%4856=1738%:%
%:%4857=1738%:%
%:%4858=1739%:%
%:%4859=1739%:%
%:%4860=1740%:%
%:%4861=1740%:%
%:%4862=1741%:%
%:%4863=1741%:%
%:%4864=1742%:%
%:%4865=1742%:%
%:%4866=1743%:%
%:%4867=1743%:%
%:%4868=1744%:%
%:%4869=1744%:%
%:%4870=1745%:%
%:%4871=1745%:%
%:%4872=1746%:%
%:%4873=1746%:%
%:%4874=1747%:%
%:%4875=1747%:%
%:%4876=1747%:%
%:%4877=1748%:%
%:%4878=1748%:%
%:%4879=1749%:%
%:%4880=1749%:%
%:%4881=1750%:%
%:%4882=1750%:%
%:%4883=1751%:%
%:%4884=1751%:%
%:%4885=1752%:%
%:%4886=1752%:%
%:%4887=1753%:%
%:%4888=1753%:%
%:%4889=1754%:%
%:%4890=1754%:%
%:%4891=1755%:%
%:%4892=1755%:%
%:%4893=1756%:%
%:%4894=1756%:%
%:%4895=1757%:%
%:%4896=1757%:%
%:%4897=1758%:%
%:%4898=1758%:%
%:%4899=1759%:%
%:%4900=1759%:%
%:%4901=1760%:%
%:%4902=1760%:%
%:%4903=1761%:%
%:%4904=1761%:%
%:%4905=1762%:%
%:%4906=1762%:%
%:%4907=1763%:%
%:%4908=1763%:%
%:%4909=1764%:%
%:%4910=1765%:%
%:%4911=1766%:%
%:%4912=1766%:%
%:%4913=1767%:%
%:%4914=1768%:%
%:%4915=1768%:%
%:%4916=1769%:%
%:%4917=1769%:%
%:%4918=1770%:%
%:%4919=1770%:%
%:%4920=1771%:%
%:%4921=1771%:%
%:%4922=1771%:%
%:%4923=1772%:%
%:%4924=1772%:%
%:%4925=1773%:%
%:%4926=1773%:%
%:%4927=1774%:%
%:%4928=1774%:%
%:%4929=1775%:%
%:%4930=1776%:%
%:%4931=1776%:%
%:%4932=1777%:%
%:%4933=1777%:%
%:%4934=1778%:%
%:%4935=1778%:%
%:%4936=1779%:%
%:%4942=1779%:%
%:%4945=1780%:%
%:%4946=1781%:%
%:%4947=1781%:%
%:%4948=1782%:%
%:%4949=1783%:%
%:%4952=1784%:%
%:%4956=1784%:%
%:%4957=1784%:%
%:%4958=1785%:%
%:%4959=1785%:%
%:%4960=1785%:%
%:%4965=1785%:%
%:%4968=1786%:%
%:%4969=1787%:%
%:%4970=1787%:%
%:%4971=1788%:%
%:%4972=1789%:%
%:%4975=1790%:%
%:%4979=1790%:%
%:%4980=1790%:%
%:%4981=1791%:%
%:%4982=1791%:%
%:%4987=1791%:%
%:%4992=1792%:%
%:%4997=1793%:%

%
\begin{isabellebody}%
\setisabellecontext{Forcing{\isacharunderscore}{\kern0pt}Theorems}%
%
\isadelimdocument
%
\endisadelimdocument
%
\isatagdocument
%
\isamarkupsection{The Forcing Theorems%
}
\isamarkuptrue%
%
\endisatagdocument
{\isafolddocument}%
%
\isadelimdocument
%
\endisadelimdocument
%
\isadelimtheory
%
\endisadelimtheory
%
\isatagtheory
\isacommand{theory}\isamarkupfalse%
\ Forcing{\isacharunderscore}{\kern0pt}Theorems\isanewline
\ \ \isakeyword{imports}\isanewline
\ \ \ \ Forces{\isacharunderscore}{\kern0pt}Definition\isanewline
\isanewline
\isakeyword{begin}%
\endisatagtheory
{\isafoldtheory}%
%
\isadelimtheory
\isanewline
%
\endisadelimtheory
\isanewline
\isacommand{context}\isamarkupfalse%
\ forcing{\isacharunderscore}{\kern0pt}data\isanewline
\isakeyword{begin}%
\isadelimdocument
%
\endisadelimdocument
%
\isatagdocument
%
\isamarkupsubsection{The forcing relation in context%
}
\isamarkuptrue%
%
\endisatagdocument
{\isafolddocument}%
%
\isadelimdocument
%
\endisadelimdocument
\isacommand{abbreviation}\isamarkupfalse%
\ Forces\ {\isacharcolon}{\kern0pt}{\isacharcolon}{\kern0pt}\ {\isachardoublequoteopen}{\isacharbrackleft}{\kern0pt}i{\isacharcomma}{\kern0pt}\ i{\isacharcomma}{\kern0pt}\ i{\isacharbrackright}{\kern0pt}\ {\isasymRightarrow}\ o{\isachardoublequoteclose}\ \ {\isacharparenleft}{\kern0pt}{\isachardoublequoteopen}{\isacharunderscore}{\kern0pt}\ {\isasymtturnstile}\ {\isacharunderscore}{\kern0pt}\ {\isacharunderscore}{\kern0pt}{\isachardoublequoteclose}\ {\isacharbrackleft}{\kern0pt}{\isadigit{3}}{\isadigit{6}}{\isacharcomma}{\kern0pt}{\isadigit{3}}{\isadigit{6}}{\isacharcomma}{\kern0pt}{\isadigit{3}}{\isadigit{6}}{\isacharbrackright}{\kern0pt}\ {\isadigit{6}}{\isadigit{0}}{\isacharparenright}{\kern0pt}\ \isakeyword{where}\isanewline
\ \ {\isachardoublequoteopen}p\ {\isasymtturnstile}\ {\isasymphi}\ env\ \ \ {\isasymequiv}\ \ \ M{\isacharcomma}{\kern0pt}\ {\isacharparenleft}{\kern0pt}{\isacharbrackleft}{\kern0pt}p{\isacharcomma}{\kern0pt}P{\isacharcomma}{\kern0pt}leq{\isacharcomma}{\kern0pt}one{\isacharbrackright}{\kern0pt}\ {\isacharat}{\kern0pt}\ env{\isacharparenright}{\kern0pt}\ {\isasymTurnstile}\ forces{\isacharparenleft}{\kern0pt}{\isasymphi}{\isacharparenright}{\kern0pt}{\isachardoublequoteclose}\isanewline
\isanewline
\isacommand{lemma}\isamarkupfalse%
\ Collect{\isacharunderscore}{\kern0pt}forces\ {\isacharcolon}{\kern0pt}\isanewline
\ \ \isakeyword{assumes}\ \isanewline
\ \ \ \ fty{\isacharcolon}{\kern0pt}\ {\isachardoublequoteopen}{\isasymphi}{\isasymin}formula{\isachardoublequoteclose}\ \isakeyword{and}\isanewline
\ \ \ \ far{\isacharcolon}{\kern0pt}\ {\isachardoublequoteopen}arity{\isacharparenleft}{\kern0pt}{\isasymphi}{\isacharparenright}{\kern0pt}{\isasymle}length{\isacharparenleft}{\kern0pt}env{\isacharparenright}{\kern0pt}{\isachardoublequoteclose}\ \isakeyword{and}\isanewline
\ \ \ \ envty{\isacharcolon}{\kern0pt}\ {\isachardoublequoteopen}env{\isasymin}list{\isacharparenleft}{\kern0pt}M{\isacharparenright}{\kern0pt}{\isachardoublequoteclose}\isanewline
\ \ \isakeyword{shows}\isanewline
\ \ \ \ {\isachardoublequoteopen}{\isacharbraceleft}{\kern0pt}p{\isasymin}P\ {\isachardot}{\kern0pt}\ p\ {\isasymtturnstile}\ {\isasymphi}\ env{\isacharbraceright}{\kern0pt}\ {\isasymin}\ M{\isachardoublequoteclose}\isanewline
%
\isadelimproof
%
\endisadelimproof
%
\isatagproof
\isacommand{proof}\isamarkupfalse%
\ {\isacharminus}{\kern0pt}\isanewline
\ \ \isacommand{have}\isamarkupfalse%
\ {\isachardoublequoteopen}z{\isasymin}P\ {\isasymLongrightarrow}\ z{\isasymin}M{\isachardoublequoteclose}\ \isakeyword{for}\ z\isanewline
\ \ \ \ \isacommand{using}\isamarkupfalse%
\ P{\isacharunderscore}{\kern0pt}in{\isacharunderscore}{\kern0pt}M\ transitivity{\isacharbrackleft}{\kern0pt}of\ z\ P{\isacharbrackright}{\kern0pt}\ \isacommand{by}\isamarkupfalse%
\ simp\isanewline
\ \ \isacommand{moreover}\isamarkupfalse%
\isanewline
\ \ \isacommand{have}\isamarkupfalse%
\ {\isachardoublequoteopen}separation{\isacharparenleft}{\kern0pt}{\isacharhash}{\kern0pt}{\isacharhash}{\kern0pt}M{\isacharcomma}{\kern0pt}{\isasymlambda}p{\isachardot}{\kern0pt}\ {\isacharparenleft}{\kern0pt}p\ {\isasymtturnstile}\ {\isasymphi}\ env{\isacharparenright}{\kern0pt}{\isacharparenright}{\kern0pt}{\isachardoublequoteclose}\isanewline
\ \ \ \ \ \ \ \ \isacommand{using}\isamarkupfalse%
\ separation{\isacharunderscore}{\kern0pt}ax\ arity{\isacharunderscore}{\kern0pt}forces\ far\ fty\ P{\isacharunderscore}{\kern0pt}in{\isacharunderscore}{\kern0pt}M\ leq{\isacharunderscore}{\kern0pt}in{\isacharunderscore}{\kern0pt}M\ one{\isacharunderscore}{\kern0pt}in{\isacharunderscore}{\kern0pt}M\ envty\ arity{\isacharunderscore}{\kern0pt}forces{\isacharunderscore}{\kern0pt}le\isanewline
\ \ \ \ \isacommand{by}\isamarkupfalse%
\ simp\isanewline
\ \ \isacommand{then}\isamarkupfalse%
\ \isanewline
\ \ \isacommand{have}\isamarkupfalse%
\ {\isachardoublequoteopen}Collect{\isacharparenleft}{\kern0pt}P{\isacharcomma}{\kern0pt}{\isasymlambda}p{\isachardot}{\kern0pt}\ {\isacharparenleft}{\kern0pt}p\ {\isasymtturnstile}\ {\isasymphi}\ env{\isacharparenright}{\kern0pt}{\isacharparenright}{\kern0pt}{\isasymin}M{\isachardoublequoteclose}\isanewline
\ \ \ \ \isacommand{using}\isamarkupfalse%
\ separation{\isacharunderscore}{\kern0pt}closed\ P{\isacharunderscore}{\kern0pt}in{\isacharunderscore}{\kern0pt}M\ \isacommand{by}\isamarkupfalse%
\ simp\isanewline
\ \ \isacommand{then}\isamarkupfalse%
\ \isacommand{show}\isamarkupfalse%
\ {\isacharquery}{\kern0pt}thesis\ \isacommand{by}\isamarkupfalse%
\ simp\isanewline
\isacommand{qed}\isamarkupfalse%
%
\endisatagproof
{\isafoldproof}%
%
\isadelimproof
\isanewline
%
\endisadelimproof
\isanewline
\isacommand{lemma}\isamarkupfalse%
\ forces{\isacharunderscore}{\kern0pt}mem{\isacharunderscore}{\kern0pt}iff{\isacharunderscore}{\kern0pt}dense{\isacharunderscore}{\kern0pt}below{\isacharcolon}{\kern0pt}\ \ {\isachardoublequoteopen}p{\isasymin}P\ {\isasymLongrightarrow}\ forces{\isacharunderscore}{\kern0pt}mem{\isacharparenleft}{\kern0pt}p{\isacharcomma}{\kern0pt}t{\isadigit{1}}{\isacharcomma}{\kern0pt}t{\isadigit{2}}{\isacharparenright}{\kern0pt}\ {\isasymlongleftrightarrow}\ dense{\isacharunderscore}{\kern0pt}below{\isacharparenleft}{\kern0pt}\isanewline
\ \ \ \ {\isacharbraceleft}{\kern0pt}q{\isasymin}P{\isachardot}{\kern0pt}\ {\isasymexists}s{\isachardot}{\kern0pt}\ {\isasymexists}r{\isachardot}{\kern0pt}\ r{\isasymin}P\ {\isasymand}\ {\isasymlangle}s{\isacharcomma}{\kern0pt}r{\isasymrangle}\ {\isasymin}\ t{\isadigit{2}}\ {\isasymand}\ q{\isasympreceq}r\ {\isasymand}\ forces{\isacharunderscore}{\kern0pt}eq{\isacharparenleft}{\kern0pt}q{\isacharcomma}{\kern0pt}t{\isadigit{1}}{\isacharcomma}{\kern0pt}s{\isacharparenright}{\kern0pt}{\isacharbraceright}{\kern0pt}\isanewline
\ \ \ \ {\isacharcomma}{\kern0pt}p{\isacharparenright}{\kern0pt}{\isachardoublequoteclose}\isanewline
%
\isadelimproof
\ \ %
\endisadelimproof
%
\isatagproof
\isacommand{using}\isamarkupfalse%
\ def{\isacharunderscore}{\kern0pt}forces{\isacharunderscore}{\kern0pt}mem{\isacharbrackleft}{\kern0pt}of\ p\ t{\isadigit{1}}\ t{\isadigit{2}}{\isacharbrackright}{\kern0pt}\ \isacommand{by}\isamarkupfalse%
\ blast%
\endisatagproof
{\isafoldproof}%
%
\isadelimproof
%
\endisadelimproof
%
\isadelimdocument
%
\endisadelimdocument
%
\isatagdocument
%
\isamarkupsubsection{Kunen 2013, Lemma IV.2.37(a)%
}
\isamarkuptrue%
%
\endisatagdocument
{\isafolddocument}%
%
\isadelimdocument
%
\endisadelimdocument
\isacommand{lemma}\isamarkupfalse%
\ strengthening{\isacharunderscore}{\kern0pt}eq{\isacharcolon}{\kern0pt}\isanewline
\ \ \isakeyword{assumes}\ {\isachardoublequoteopen}p{\isasymin}P{\isachardoublequoteclose}\ {\isachardoublequoteopen}r{\isasymin}P{\isachardoublequoteclose}\ {\isachardoublequoteopen}r{\isasympreceq}p{\isachardoublequoteclose}\ {\isachardoublequoteopen}forces{\isacharunderscore}{\kern0pt}eq{\isacharparenleft}{\kern0pt}p{\isacharcomma}{\kern0pt}t{\isadigit{1}}{\isacharcomma}{\kern0pt}t{\isadigit{2}}{\isacharparenright}{\kern0pt}{\isachardoublequoteclose}\isanewline
\ \ \isakeyword{shows}\ {\isachardoublequoteopen}forces{\isacharunderscore}{\kern0pt}eq{\isacharparenleft}{\kern0pt}r{\isacharcomma}{\kern0pt}t{\isadigit{1}}{\isacharcomma}{\kern0pt}t{\isadigit{2}}{\isacharparenright}{\kern0pt}{\isachardoublequoteclose}\isanewline
%
\isadelimproof
\ \ %
\endisadelimproof
%
\isatagproof
\isacommand{using}\isamarkupfalse%
\ assms\ def{\isacharunderscore}{\kern0pt}forces{\isacharunderscore}{\kern0pt}eq{\isacharbrackleft}{\kern0pt}of\ {\isacharunderscore}{\kern0pt}\ t{\isadigit{1}}\ t{\isadigit{2}}{\isacharbrackright}{\kern0pt}\ leq{\isacharunderscore}{\kern0pt}transD\ \isacommand{by}\isamarkupfalse%
\ blast%
\endisatagproof
{\isafoldproof}%
%
\isadelimproof
%
\endisadelimproof
%
\isadelimdocument
%
\endisadelimdocument
%
\isatagdocument
%
\isamarkupsubsection{Kunen 2013, Lemma IV.2.37(a)%
}
\isamarkuptrue%
%
\endisatagdocument
{\isafolddocument}%
%
\isadelimdocument
%
\endisadelimdocument
\isacommand{lemma}\isamarkupfalse%
\ strengthening{\isacharunderscore}{\kern0pt}mem{\isacharcolon}{\kern0pt}\ \isanewline
\ \ \isakeyword{assumes}\ {\isachardoublequoteopen}p{\isasymin}P{\isachardoublequoteclose}\ {\isachardoublequoteopen}r{\isasymin}P{\isachardoublequoteclose}\ {\isachardoublequoteopen}r{\isasympreceq}p{\isachardoublequoteclose}\ {\isachardoublequoteopen}forces{\isacharunderscore}{\kern0pt}mem{\isacharparenleft}{\kern0pt}p{\isacharcomma}{\kern0pt}t{\isadigit{1}}{\isacharcomma}{\kern0pt}t{\isadigit{2}}{\isacharparenright}{\kern0pt}{\isachardoublequoteclose}\isanewline
\ \ \isakeyword{shows}\ {\isachardoublequoteopen}forces{\isacharunderscore}{\kern0pt}mem{\isacharparenleft}{\kern0pt}r{\isacharcomma}{\kern0pt}t{\isadigit{1}}{\isacharcomma}{\kern0pt}t{\isadigit{2}}{\isacharparenright}{\kern0pt}{\isachardoublequoteclose}\isanewline
%
\isadelimproof
\ \ %
\endisadelimproof
%
\isatagproof
\isacommand{using}\isamarkupfalse%
\ assms\ forces{\isacharunderscore}{\kern0pt}mem{\isacharunderscore}{\kern0pt}iff{\isacharunderscore}{\kern0pt}dense{\isacharunderscore}{\kern0pt}below\ dense{\isacharunderscore}{\kern0pt}below{\isacharunderscore}{\kern0pt}under\ \isacommand{by}\isamarkupfalse%
\ auto%
\endisatagproof
{\isafoldproof}%
%
\isadelimproof
%
\endisadelimproof
%
\isadelimdocument
%
\endisadelimdocument
%
\isatagdocument
%
\isamarkupsubsection{Kunen 2013, Lemma IV.2.37(b)%
}
\isamarkuptrue%
%
\endisatagdocument
{\isafolddocument}%
%
\isadelimdocument
%
\endisadelimdocument
\isacommand{lemma}\isamarkupfalse%
\ density{\isacharunderscore}{\kern0pt}mem{\isacharcolon}{\kern0pt}\ \isanewline
\ \ \isakeyword{assumes}\ {\isachardoublequoteopen}p{\isasymin}P{\isachardoublequoteclose}\isanewline
\ \ \isakeyword{shows}\ {\isachardoublequoteopen}forces{\isacharunderscore}{\kern0pt}mem{\isacharparenleft}{\kern0pt}p{\isacharcomma}{\kern0pt}t{\isadigit{1}}{\isacharcomma}{\kern0pt}t{\isadigit{2}}{\isacharparenright}{\kern0pt}\ \ {\isasymlongleftrightarrow}\ dense{\isacharunderscore}{\kern0pt}below{\isacharparenleft}{\kern0pt}{\isacharbraceleft}{\kern0pt}q{\isasymin}P{\isachardot}{\kern0pt}\ forces{\isacharunderscore}{\kern0pt}mem{\isacharparenleft}{\kern0pt}q{\isacharcomma}{\kern0pt}t{\isadigit{1}}{\isacharcomma}{\kern0pt}t{\isadigit{2}}{\isacharparenright}{\kern0pt}{\isacharbraceright}{\kern0pt}{\isacharcomma}{\kern0pt}p{\isacharparenright}{\kern0pt}{\isachardoublequoteclose}\isanewline
%
\isadelimproof
%
\endisadelimproof
%
\isatagproof
\isacommand{proof}\isamarkupfalse%
\isanewline
\ \ \isacommand{assume}\isamarkupfalse%
\ {\isachardoublequoteopen}forces{\isacharunderscore}{\kern0pt}mem{\isacharparenleft}{\kern0pt}p{\isacharcomma}{\kern0pt}t{\isadigit{1}}{\isacharcomma}{\kern0pt}t{\isadigit{2}}{\isacharparenright}{\kern0pt}{\isachardoublequoteclose}\isanewline
\ \ \isacommand{with}\isamarkupfalse%
\ assms\isanewline
\ \ \isacommand{show}\isamarkupfalse%
\ {\isachardoublequoteopen}dense{\isacharunderscore}{\kern0pt}below{\isacharparenleft}{\kern0pt}{\isacharbraceleft}{\kern0pt}q{\isasymin}P{\isachardot}{\kern0pt}\ forces{\isacharunderscore}{\kern0pt}mem{\isacharparenleft}{\kern0pt}q{\isacharcomma}{\kern0pt}t{\isadigit{1}}{\isacharcomma}{\kern0pt}t{\isadigit{2}}{\isacharparenright}{\kern0pt}{\isacharbraceright}{\kern0pt}{\isacharcomma}{\kern0pt}p{\isacharparenright}{\kern0pt}{\isachardoublequoteclose}\isanewline
\ \ \ \ \isacommand{using}\isamarkupfalse%
\ forces{\isacharunderscore}{\kern0pt}mem{\isacharunderscore}{\kern0pt}iff{\isacharunderscore}{\kern0pt}dense{\isacharunderscore}{\kern0pt}below\ strengthening{\isacharunderscore}{\kern0pt}mem{\isacharbrackleft}{\kern0pt}of\ p{\isacharbrackright}{\kern0pt}\ ideal{\isacharunderscore}{\kern0pt}dense{\isacharunderscore}{\kern0pt}below\ \isacommand{by}\isamarkupfalse%
\ auto\isanewline
\isacommand{next}\isamarkupfalse%
\isanewline
\ \ \isacommand{assume}\isamarkupfalse%
\ {\isachardoublequoteopen}dense{\isacharunderscore}{\kern0pt}below{\isacharparenleft}{\kern0pt}{\isacharbraceleft}{\kern0pt}q\ {\isasymin}\ P\ {\isachardot}{\kern0pt}\ forces{\isacharunderscore}{\kern0pt}mem{\isacharparenleft}{\kern0pt}q{\isacharcomma}{\kern0pt}\ t{\isadigit{1}}{\isacharcomma}{\kern0pt}\ t{\isadigit{2}}{\isacharparenright}{\kern0pt}{\isacharbraceright}{\kern0pt}{\isacharcomma}{\kern0pt}\ p{\isacharparenright}{\kern0pt}{\isachardoublequoteclose}\isanewline
\ \ \isacommand{with}\isamarkupfalse%
\ assms\isanewline
\ \ \isacommand{have}\isamarkupfalse%
\ {\isachardoublequoteopen}dense{\isacharunderscore}{\kern0pt}below{\isacharparenleft}{\kern0pt}{\isacharbraceleft}{\kern0pt}q{\isasymin}P{\isachardot}{\kern0pt}\ \isanewline
\ \ \ \ dense{\isacharunderscore}{\kern0pt}below{\isacharparenleft}{\kern0pt}{\isacharbraceleft}{\kern0pt}q{\isacharprime}{\kern0pt}{\isasymin}P{\isachardot}{\kern0pt}\ {\isasymexists}s\ r{\isachardot}{\kern0pt}\ r\ {\isasymin}\ P\ {\isasymand}\ {\isasymlangle}s{\isacharcomma}{\kern0pt}r{\isasymrangle}{\isasymin}t{\isadigit{2}}\ {\isasymand}\ q{\isacharprime}{\kern0pt}{\isasympreceq}r\ {\isasymand}\ forces{\isacharunderscore}{\kern0pt}eq{\isacharparenleft}{\kern0pt}q{\isacharprime}{\kern0pt}{\isacharcomma}{\kern0pt}t{\isadigit{1}}{\isacharcomma}{\kern0pt}s{\isacharparenright}{\kern0pt}{\isacharbraceright}{\kern0pt}{\isacharcomma}{\kern0pt}q{\isacharparenright}{\kern0pt}\isanewline
\ \ \ \ {\isacharbraceright}{\kern0pt}{\isacharcomma}{\kern0pt}p{\isacharparenright}{\kern0pt}{\isachardoublequoteclose}\isanewline
\ \ \ \ \isacommand{using}\isamarkupfalse%
\ forces{\isacharunderscore}{\kern0pt}mem{\isacharunderscore}{\kern0pt}iff{\isacharunderscore}{\kern0pt}dense{\isacharunderscore}{\kern0pt}below\ \isacommand{by}\isamarkupfalse%
\ simp\isanewline
\ \ \isacommand{with}\isamarkupfalse%
\ assms\isanewline
\ \ \isacommand{show}\isamarkupfalse%
\ {\isachardoublequoteopen}forces{\isacharunderscore}{\kern0pt}mem{\isacharparenleft}{\kern0pt}p{\isacharcomma}{\kern0pt}t{\isadigit{1}}{\isacharcomma}{\kern0pt}t{\isadigit{2}}{\isacharparenright}{\kern0pt}{\isachardoublequoteclose}\isanewline
\ \ \ \ \isacommand{using}\isamarkupfalse%
\ dense{\isacharunderscore}{\kern0pt}below{\isacharunderscore}{\kern0pt}dense{\isacharunderscore}{\kern0pt}below\ forces{\isacharunderscore}{\kern0pt}mem{\isacharunderscore}{\kern0pt}iff{\isacharunderscore}{\kern0pt}dense{\isacharunderscore}{\kern0pt}below{\isacharbrackleft}{\kern0pt}of\ p\ t{\isadigit{1}}\ t{\isadigit{2}}{\isacharbrackright}{\kern0pt}\ \isacommand{by}\isamarkupfalse%
\ blast\isanewline
\isacommand{qed}\isamarkupfalse%
%
\endisatagproof
{\isafoldproof}%
%
\isadelimproof
\isanewline
%
\endisadelimproof
\isanewline
\isacommand{lemma}\isamarkupfalse%
\ aux{\isacharunderscore}{\kern0pt}density{\isacharunderscore}{\kern0pt}eq{\isacharcolon}{\kern0pt}\isanewline
\ \ \isakeyword{assumes}\ \isanewline
\ \ \ \ {\isachardoublequoteopen}dense{\isacharunderscore}{\kern0pt}below{\isacharparenleft}{\kern0pt}\isanewline
\ \ \ \ {\isacharbraceleft}{\kern0pt}q{\isacharprime}{\kern0pt}{\isasymin}P{\isachardot}{\kern0pt}\ {\isasymforall}q{\isachardot}{\kern0pt}\ q{\isasymin}P\ {\isasymand}\ q{\isasympreceq}q{\isacharprime}{\kern0pt}\ {\isasymlongrightarrow}\ forces{\isacharunderscore}{\kern0pt}mem{\isacharparenleft}{\kern0pt}q{\isacharcomma}{\kern0pt}s{\isacharcomma}{\kern0pt}t{\isadigit{1}}{\isacharparenright}{\kern0pt}\ {\isasymlongleftrightarrow}\ forces{\isacharunderscore}{\kern0pt}mem{\isacharparenleft}{\kern0pt}q{\isacharcomma}{\kern0pt}s{\isacharcomma}{\kern0pt}t{\isadigit{2}}{\isacharparenright}{\kern0pt}{\isacharbraceright}{\kern0pt}\isanewline
\ \ \ \ {\isacharcomma}{\kern0pt}p{\isacharparenright}{\kern0pt}{\isachardoublequoteclose}\isanewline
\ \ \ \ {\isachardoublequoteopen}forces{\isacharunderscore}{\kern0pt}mem{\isacharparenleft}{\kern0pt}q{\isacharcomma}{\kern0pt}s{\isacharcomma}{\kern0pt}t{\isadigit{1}}{\isacharparenright}{\kern0pt}{\isachardoublequoteclose}\ {\isachardoublequoteopen}q{\isasymin}P{\isachardoublequoteclose}\ {\isachardoublequoteopen}p{\isasymin}P{\isachardoublequoteclose}\ {\isachardoublequoteopen}q{\isasympreceq}p{\isachardoublequoteclose}\isanewline
\ \ \isakeyword{shows}\isanewline
\ \ \ \ {\isachardoublequoteopen}dense{\isacharunderscore}{\kern0pt}below{\isacharparenleft}{\kern0pt}{\isacharbraceleft}{\kern0pt}r{\isasymin}P{\isachardot}{\kern0pt}\ forces{\isacharunderscore}{\kern0pt}mem{\isacharparenleft}{\kern0pt}r{\isacharcomma}{\kern0pt}s{\isacharcomma}{\kern0pt}t{\isadigit{2}}{\isacharparenright}{\kern0pt}{\isacharbraceright}{\kern0pt}{\isacharcomma}{\kern0pt}q{\isacharparenright}{\kern0pt}{\isachardoublequoteclose}\isanewline
%
\isadelimproof
%
\endisadelimproof
%
\isatagproof
\isacommand{proof}\isamarkupfalse%
\isanewline
\ \ \isacommand{fix}\isamarkupfalse%
\ r\isanewline
\ \ \isacommand{assume}\isamarkupfalse%
\ {\isachardoublequoteopen}r{\isasymin}P{\isachardoublequoteclose}\ {\isachardoublequoteopen}r{\isasympreceq}q{\isachardoublequoteclose}\isanewline
\ \ \isacommand{moreover}\isamarkupfalse%
\ \isacommand{from}\isamarkupfalse%
\ this\ \isakeyword{and}\ {\isacartoucheopen}p{\isasymin}P{\isacartoucheclose}\ {\isacartoucheopen}q{\isasympreceq}p{\isacartoucheclose}\ {\isacartoucheopen}q{\isasymin}P{\isacartoucheclose}\isanewline
\ \ \isacommand{have}\isamarkupfalse%
\ {\isachardoublequoteopen}r{\isasympreceq}p{\isachardoublequoteclose}\isanewline
\ \ \ \ \isacommand{using}\isamarkupfalse%
\ leq{\isacharunderscore}{\kern0pt}transD\ \isacommand{by}\isamarkupfalse%
\ simp\isanewline
\ \ \isacommand{moreover}\isamarkupfalse%
\isanewline
\ \ \isacommand{note}\isamarkupfalse%
\ {\isacartoucheopen}forces{\isacharunderscore}{\kern0pt}mem{\isacharparenleft}{\kern0pt}q{\isacharcomma}{\kern0pt}s{\isacharcomma}{\kern0pt}t{\isadigit{1}}{\isacharparenright}{\kern0pt}{\isacartoucheclose}\ {\isacartoucheopen}dense{\isacharunderscore}{\kern0pt}below{\isacharparenleft}{\kern0pt}{\isacharunderscore}{\kern0pt}{\isacharcomma}{\kern0pt}p{\isacharparenright}{\kern0pt}{\isacartoucheclose}\ {\isacartoucheopen}q{\isasymin}P{\isacartoucheclose}\isanewline
\ \ \isacommand{ultimately}\isamarkupfalse%
\isanewline
\ \ \isacommand{obtain}\isamarkupfalse%
\ q{\isadigit{1}}\ \isakeyword{where}\ {\isachardoublequoteopen}q{\isadigit{1}}{\isasympreceq}r{\isachardoublequoteclose}\ {\isachardoublequoteopen}q{\isadigit{1}}{\isasymin}P{\isachardoublequoteclose}\ {\isachardoublequoteopen}forces{\isacharunderscore}{\kern0pt}mem{\isacharparenleft}{\kern0pt}q{\isadigit{1}}{\isacharcomma}{\kern0pt}s{\isacharcomma}{\kern0pt}t{\isadigit{2}}{\isacharparenright}{\kern0pt}{\isachardoublequoteclose}\isanewline
\ \ \ \ \isacommand{using}\isamarkupfalse%
\ strengthening{\isacharunderscore}{\kern0pt}mem{\isacharbrackleft}{\kern0pt}of\ q\ {\isacharunderscore}{\kern0pt}\ s\ t{\isadigit{1}}{\isacharbrackright}{\kern0pt}\ leq{\isacharunderscore}{\kern0pt}reflI\ leq{\isacharunderscore}{\kern0pt}transD{\isacharbrackleft}{\kern0pt}of\ {\isacharunderscore}{\kern0pt}\ r\ q{\isacharbrackright}{\kern0pt}\ \isacommand{by}\isamarkupfalse%
\ blast\isanewline
\ \ \isacommand{then}\isamarkupfalse%
\isanewline
\ \ \isacommand{show}\isamarkupfalse%
\ {\isachardoublequoteopen}{\isasymexists}d{\isasymin}{\isacharbraceleft}{\kern0pt}r\ {\isasymin}\ P\ {\isachardot}{\kern0pt}\ forces{\isacharunderscore}{\kern0pt}mem{\isacharparenleft}{\kern0pt}r{\isacharcomma}{\kern0pt}\ s{\isacharcomma}{\kern0pt}\ t{\isadigit{2}}{\isacharparenright}{\kern0pt}{\isacharbraceright}{\kern0pt}{\isachardot}{\kern0pt}\ d\ {\isasymin}\ P\ {\isasymand}\ d{\isasympreceq}\ r{\isachardoublequoteclose}\isanewline
\ \ \ \ \isacommand{by}\isamarkupfalse%
\ blast\isanewline
\isacommand{qed}\isamarkupfalse%
%
\endisatagproof
{\isafoldproof}%
%
\isadelimproof
\isanewline
%
\endisadelimproof
\isanewline
\isanewline
\isacommand{lemma}\isamarkupfalse%
\ density{\isacharunderscore}{\kern0pt}eq{\isacharcolon}{\kern0pt}\isanewline
\ \ \isakeyword{assumes}\ {\isachardoublequoteopen}p{\isasymin}P{\isachardoublequoteclose}\isanewline
\ \ \isakeyword{shows}\ {\isachardoublequoteopen}forces{\isacharunderscore}{\kern0pt}eq{\isacharparenleft}{\kern0pt}p{\isacharcomma}{\kern0pt}t{\isadigit{1}}{\isacharcomma}{\kern0pt}t{\isadigit{2}}{\isacharparenright}{\kern0pt}\ \ {\isasymlongleftrightarrow}\ dense{\isacharunderscore}{\kern0pt}below{\isacharparenleft}{\kern0pt}{\isacharbraceleft}{\kern0pt}q{\isasymin}P{\isachardot}{\kern0pt}\ forces{\isacharunderscore}{\kern0pt}eq{\isacharparenleft}{\kern0pt}q{\isacharcomma}{\kern0pt}t{\isadigit{1}}{\isacharcomma}{\kern0pt}t{\isadigit{2}}{\isacharparenright}{\kern0pt}{\isacharbraceright}{\kern0pt}{\isacharcomma}{\kern0pt}p{\isacharparenright}{\kern0pt}{\isachardoublequoteclose}\isanewline
%
\isadelimproof
%
\endisadelimproof
%
\isatagproof
\isacommand{proof}\isamarkupfalse%
\isanewline
\ \ \isacommand{assume}\isamarkupfalse%
\ {\isachardoublequoteopen}forces{\isacharunderscore}{\kern0pt}eq{\isacharparenleft}{\kern0pt}p{\isacharcomma}{\kern0pt}t{\isadigit{1}}{\isacharcomma}{\kern0pt}t{\isadigit{2}}{\isacharparenright}{\kern0pt}{\isachardoublequoteclose}\isanewline
\ \ \isacommand{with}\isamarkupfalse%
\ {\isacartoucheopen}p{\isasymin}P{\isacartoucheclose}\isanewline
\ \ \isacommand{show}\isamarkupfalse%
\ {\isachardoublequoteopen}dense{\isacharunderscore}{\kern0pt}below{\isacharparenleft}{\kern0pt}{\isacharbraceleft}{\kern0pt}q{\isasymin}P{\isachardot}{\kern0pt}\ forces{\isacharunderscore}{\kern0pt}eq{\isacharparenleft}{\kern0pt}q{\isacharcomma}{\kern0pt}t{\isadigit{1}}{\isacharcomma}{\kern0pt}t{\isadigit{2}}{\isacharparenright}{\kern0pt}{\isacharbraceright}{\kern0pt}{\isacharcomma}{\kern0pt}p{\isacharparenright}{\kern0pt}{\isachardoublequoteclose}\isanewline
\ \ \ \ \isacommand{using}\isamarkupfalse%
\ strengthening{\isacharunderscore}{\kern0pt}eq\ ideal{\isacharunderscore}{\kern0pt}dense{\isacharunderscore}{\kern0pt}below\ \isacommand{by}\isamarkupfalse%
\ auto\isanewline
\isacommand{next}\isamarkupfalse%
\isanewline
\ \ \isacommand{assume}\isamarkupfalse%
\ {\isachardoublequoteopen}dense{\isacharunderscore}{\kern0pt}below{\isacharparenleft}{\kern0pt}{\isacharbraceleft}{\kern0pt}q{\isasymin}P{\isachardot}{\kern0pt}\ forces{\isacharunderscore}{\kern0pt}eq{\isacharparenleft}{\kern0pt}q{\isacharcomma}{\kern0pt}t{\isadigit{1}}{\isacharcomma}{\kern0pt}t{\isadigit{2}}{\isacharparenright}{\kern0pt}{\isacharbraceright}{\kern0pt}{\isacharcomma}{\kern0pt}p{\isacharparenright}{\kern0pt}{\isachardoublequoteclose}\isanewline
\ \ \isacommand{{\isacharbraceleft}{\kern0pt}}\isamarkupfalse%
\isanewline
\ \ \ \ \isacommand{fix}\isamarkupfalse%
\ s\ q\ \isanewline
\ \ \ \ \isacommand{let}\isamarkupfalse%
\ {\isacharquery}{\kern0pt}D{\isadigit{1}}{\isacharequal}{\kern0pt}{\isachardoublequoteopen}{\isacharbraceleft}{\kern0pt}q{\isacharprime}{\kern0pt}{\isasymin}P{\isachardot}{\kern0pt}\ {\isasymforall}s{\isasymin}domain{\isacharparenleft}{\kern0pt}t{\isadigit{1}}{\isacharparenright}{\kern0pt}\ {\isasymunion}\ domain{\isacharparenleft}{\kern0pt}t{\isadigit{2}}{\isacharparenright}{\kern0pt}{\isachardot}{\kern0pt}\ {\isasymforall}q{\isachardot}{\kern0pt}\ q\ {\isasymin}\ P\ {\isasymand}\ q{\isasympreceq}q{\isacharprime}{\kern0pt}\ {\isasymlongrightarrow}\isanewline
\ \ \ \ \ \ \ \ \ \ \ forces{\isacharunderscore}{\kern0pt}mem{\isacharparenleft}{\kern0pt}q{\isacharcomma}{\kern0pt}s{\isacharcomma}{\kern0pt}t{\isadigit{1}}{\isacharparenright}{\kern0pt}{\isasymlongleftrightarrow}forces{\isacharunderscore}{\kern0pt}mem{\isacharparenleft}{\kern0pt}q{\isacharcomma}{\kern0pt}s{\isacharcomma}{\kern0pt}t{\isadigit{2}}{\isacharparenright}{\kern0pt}{\isacharbraceright}{\kern0pt}{\isachardoublequoteclose}\isanewline
\ \ \ \ \isacommand{let}\isamarkupfalse%
\ {\isacharquery}{\kern0pt}D{\isadigit{2}}{\isacharequal}{\kern0pt}{\isachardoublequoteopen}{\isacharbraceleft}{\kern0pt}q{\isacharprime}{\kern0pt}{\isasymin}P{\isachardot}{\kern0pt}\ {\isasymforall}q{\isachardot}{\kern0pt}\ q{\isasymin}P\ {\isasymand}\ q{\isasympreceq}q{\isacharprime}{\kern0pt}\ {\isasymlongrightarrow}\ forces{\isacharunderscore}{\kern0pt}mem{\isacharparenleft}{\kern0pt}q{\isacharcomma}{\kern0pt}s{\isacharcomma}{\kern0pt}t{\isadigit{1}}{\isacharparenright}{\kern0pt}\ {\isasymlongleftrightarrow}\ forces{\isacharunderscore}{\kern0pt}mem{\isacharparenleft}{\kern0pt}q{\isacharcomma}{\kern0pt}s{\isacharcomma}{\kern0pt}t{\isadigit{2}}{\isacharparenright}{\kern0pt}{\isacharbraceright}{\kern0pt}{\isachardoublequoteclose}\isanewline
\ \ \ \ \isacommand{assume}\isamarkupfalse%
\ {\isachardoublequoteopen}s{\isasymin}domain{\isacharparenleft}{\kern0pt}t{\isadigit{1}}{\isacharparenright}{\kern0pt}\ {\isasymunion}\ domain{\isacharparenleft}{\kern0pt}t{\isadigit{2}}{\isacharparenright}{\kern0pt}{\isachardoublequoteclose}\ \isanewline
\ \ \ \ \isacommand{then}\isamarkupfalse%
\isanewline
\ \ \ \ \isacommand{have}\isamarkupfalse%
\ {\isachardoublequoteopen}{\isacharquery}{\kern0pt}D{\isadigit{1}}{\isasymsubseteq}{\isacharquery}{\kern0pt}D{\isadigit{2}}{\isachardoublequoteclose}\ \isacommand{by}\isamarkupfalse%
\ blast\isanewline
\ \ \ \ \isacommand{with}\isamarkupfalse%
\ {\isacartoucheopen}dense{\isacharunderscore}{\kern0pt}below{\isacharparenleft}{\kern0pt}{\isacharunderscore}{\kern0pt}{\isacharcomma}{\kern0pt}p{\isacharparenright}{\kern0pt}{\isacartoucheclose}\isanewline
\ \ \ \ \isacommand{have}\isamarkupfalse%
\ {\isachardoublequoteopen}dense{\isacharunderscore}{\kern0pt}below{\isacharparenleft}{\kern0pt}{\isacharbraceleft}{\kern0pt}q{\isacharprime}{\kern0pt}{\isasymin}P{\isachardot}{\kern0pt}\ {\isasymforall}s{\isasymin}domain{\isacharparenleft}{\kern0pt}t{\isadigit{1}}{\isacharparenright}{\kern0pt}\ {\isasymunion}\ domain{\isacharparenleft}{\kern0pt}t{\isadigit{2}}{\isacharparenright}{\kern0pt}{\isachardot}{\kern0pt}\ {\isasymforall}q{\isachardot}{\kern0pt}\ q\ {\isasymin}\ P\ {\isasymand}\ q{\isasympreceq}q{\isacharprime}{\kern0pt}\ {\isasymlongrightarrow}\isanewline
\ \ \ \ \ \ \ \ \ \ \ forces{\isacharunderscore}{\kern0pt}mem{\isacharparenleft}{\kern0pt}q{\isacharcomma}{\kern0pt}s{\isacharcomma}{\kern0pt}t{\isadigit{1}}{\isacharparenright}{\kern0pt}{\isasymlongleftrightarrow}forces{\isacharunderscore}{\kern0pt}mem{\isacharparenleft}{\kern0pt}q{\isacharcomma}{\kern0pt}s{\isacharcomma}{\kern0pt}t{\isadigit{2}}{\isacharparenright}{\kern0pt}{\isacharbraceright}{\kern0pt}{\isacharcomma}{\kern0pt}p{\isacharparenright}{\kern0pt}{\isachardoublequoteclose}\isanewline
\ \ \ \ \ \ \isacommand{using}\isamarkupfalse%
\ dense{\isacharunderscore}{\kern0pt}below{\isacharunderscore}{\kern0pt}cong{\isacharprime}{\kern0pt}{\isacharbrackleft}{\kern0pt}OF\ {\isacartoucheopen}p{\isasymin}P{\isacartoucheclose}\ def{\isacharunderscore}{\kern0pt}forces{\isacharunderscore}{\kern0pt}eq{\isacharbrackleft}{\kern0pt}of\ {\isacharunderscore}{\kern0pt}\ t{\isadigit{1}}\ t{\isadigit{2}}{\isacharbrackright}{\kern0pt}{\isacharbrackright}{\kern0pt}\ \isacommand{by}\isamarkupfalse%
\ simp\isanewline
\ \ \ \ \isacommand{with}\isamarkupfalse%
\ {\isacartoucheopen}p{\isasymin}P{\isacartoucheclose}\ {\isacartoucheopen}{\isacharquery}{\kern0pt}D{\isadigit{1}}{\isasymsubseteq}{\isacharquery}{\kern0pt}D{\isadigit{2}}{\isacartoucheclose}\isanewline
\ \ \ \ \isacommand{have}\isamarkupfalse%
\ {\isachardoublequoteopen}dense{\isacharunderscore}{\kern0pt}below{\isacharparenleft}{\kern0pt}{\isacharbraceleft}{\kern0pt}q{\isacharprime}{\kern0pt}{\isasymin}P{\isachardot}{\kern0pt}\ {\isasymforall}q{\isachardot}{\kern0pt}\ q{\isasymin}P\ {\isasymand}\ q{\isasympreceq}q{\isacharprime}{\kern0pt}\ {\isasymlongrightarrow}\ \isanewline
\ \ \ \ \ \ \ \ \ \ \ \ forces{\isacharunderscore}{\kern0pt}mem{\isacharparenleft}{\kern0pt}q{\isacharcomma}{\kern0pt}s{\isacharcomma}{\kern0pt}t{\isadigit{1}}{\isacharparenright}{\kern0pt}\ {\isasymlongleftrightarrow}\ forces{\isacharunderscore}{\kern0pt}mem{\isacharparenleft}{\kern0pt}q{\isacharcomma}{\kern0pt}s{\isacharcomma}{\kern0pt}t{\isadigit{2}}{\isacharparenright}{\kern0pt}{\isacharbraceright}{\kern0pt}{\isacharcomma}{\kern0pt}p{\isacharparenright}{\kern0pt}{\isachardoublequoteclose}\isanewline
\ \ \ \ \ \ \isacommand{using}\isamarkupfalse%
\ dense{\isacharunderscore}{\kern0pt}below{\isacharunderscore}{\kern0pt}mono\ \isacommand{by}\isamarkupfalse%
\ simp\isanewline
\ \ \ \ \isacommand{moreover}\isamarkupfalse%
\ \isacommand{from}\isamarkupfalse%
\ this\ \isanewline
\ \ \ \ \isacommand{have}\isamarkupfalse%
\ \ {\isachardoublequoteopen}dense{\isacharunderscore}{\kern0pt}below{\isacharparenleft}{\kern0pt}{\isacharbraceleft}{\kern0pt}q{\isacharprime}{\kern0pt}{\isasymin}P{\isachardot}{\kern0pt}\ {\isasymforall}q{\isachardot}{\kern0pt}\ q{\isasymin}P\ {\isasymand}\ q{\isasympreceq}q{\isacharprime}{\kern0pt}\ {\isasymlongrightarrow}\ \isanewline
\ \ \ \ \ \ \ \ \ \ \ \ forces{\isacharunderscore}{\kern0pt}mem{\isacharparenleft}{\kern0pt}q{\isacharcomma}{\kern0pt}s{\isacharcomma}{\kern0pt}t{\isadigit{2}}{\isacharparenright}{\kern0pt}\ {\isasymlongleftrightarrow}\ forces{\isacharunderscore}{\kern0pt}mem{\isacharparenleft}{\kern0pt}q{\isacharcomma}{\kern0pt}s{\isacharcomma}{\kern0pt}t{\isadigit{1}}{\isacharparenright}{\kern0pt}{\isacharbraceright}{\kern0pt}{\isacharcomma}{\kern0pt}p{\isacharparenright}{\kern0pt}{\isachardoublequoteclose}\isanewline
\ \ \ \ \ \ \isacommand{by}\isamarkupfalse%
\ blast\isanewline
\ \ \ \ \isacommand{moreover}\isamarkupfalse%
\isanewline
\ \ \ \ \isacommand{assume}\isamarkupfalse%
\ {\isachardoublequoteopen}q\ {\isasymin}\ P{\isachardoublequoteclose}\ {\isachardoublequoteopen}q{\isasympreceq}p{\isachardoublequoteclose}\isanewline
\ \ \ \ \isacommand{moreover}\isamarkupfalse%
\isanewline
\ \ \ \ \isacommand{note}\isamarkupfalse%
\ {\isacartoucheopen}p{\isasymin}P{\isacartoucheclose}\isanewline
\ \ \ \ \isacommand{ultimately}\isamarkupfalse%
\ \isanewline
\ \ \ \ \isacommand{have}\isamarkupfalse%
\ {\isachardoublequoteopen}forces{\isacharunderscore}{\kern0pt}mem{\isacharparenleft}{\kern0pt}q{\isacharcomma}{\kern0pt}s{\isacharcomma}{\kern0pt}t{\isadigit{1}}{\isacharparenright}{\kern0pt}\ {\isasymLongrightarrow}\ dense{\isacharunderscore}{\kern0pt}below{\isacharparenleft}{\kern0pt}{\isacharbraceleft}{\kern0pt}r{\isasymin}P{\isachardot}{\kern0pt}\ forces{\isacharunderscore}{\kern0pt}mem{\isacharparenleft}{\kern0pt}r{\isacharcomma}{\kern0pt}s{\isacharcomma}{\kern0pt}t{\isadigit{2}}{\isacharparenright}{\kern0pt}{\isacharbraceright}{\kern0pt}{\isacharcomma}{\kern0pt}q{\isacharparenright}{\kern0pt}{\isachardoublequoteclose}\isanewline
\ \ \ \ \ \ \ \ \ {\isachardoublequoteopen}forces{\isacharunderscore}{\kern0pt}mem{\isacharparenleft}{\kern0pt}q{\isacharcomma}{\kern0pt}s{\isacharcomma}{\kern0pt}t{\isadigit{2}}{\isacharparenright}{\kern0pt}\ {\isasymLongrightarrow}\ dense{\isacharunderscore}{\kern0pt}below{\isacharparenleft}{\kern0pt}{\isacharbraceleft}{\kern0pt}r{\isasymin}P{\isachardot}{\kern0pt}\ forces{\isacharunderscore}{\kern0pt}mem{\isacharparenleft}{\kern0pt}r{\isacharcomma}{\kern0pt}s{\isacharcomma}{\kern0pt}t{\isadigit{1}}{\isacharparenright}{\kern0pt}{\isacharbraceright}{\kern0pt}{\isacharcomma}{\kern0pt}q{\isacharparenright}{\kern0pt}{\isachardoublequoteclose}\ \isanewline
\ \ \ \ \ \ \isacommand{using}\isamarkupfalse%
\ aux{\isacharunderscore}{\kern0pt}density{\isacharunderscore}{\kern0pt}eq\ \isacommand{by}\isamarkupfalse%
\ simp{\isacharunderscore}{\kern0pt}all\isanewline
\ \ \ \ \isacommand{then}\isamarkupfalse%
\isanewline
\ \ \ \ \isacommand{have}\isamarkupfalse%
\ {\isachardoublequoteopen}forces{\isacharunderscore}{\kern0pt}mem{\isacharparenleft}{\kern0pt}q{\isacharcomma}{\kern0pt}\ s{\isacharcomma}{\kern0pt}\ t{\isadigit{1}}{\isacharparenright}{\kern0pt}\ {\isasymlongleftrightarrow}\ forces{\isacharunderscore}{\kern0pt}mem{\isacharparenleft}{\kern0pt}q{\isacharcomma}{\kern0pt}\ s{\isacharcomma}{\kern0pt}\ t{\isadigit{2}}{\isacharparenright}{\kern0pt}{\isachardoublequoteclose}\isanewline
\ \ \ \ \ \ \isacommand{using}\isamarkupfalse%
\ density{\isacharunderscore}{\kern0pt}mem{\isacharbrackleft}{\kern0pt}OF\ {\isacartoucheopen}q{\isasymin}P{\isacartoucheclose}{\isacharbrackright}{\kern0pt}\ \isacommand{by}\isamarkupfalse%
\ blast\isanewline
\ \ \isacommand{{\isacharbraceright}{\kern0pt}}\isamarkupfalse%
\isanewline
\ \ \isacommand{with}\isamarkupfalse%
\ {\isacartoucheopen}p{\isasymin}P{\isacartoucheclose}\isanewline
\ \ \isacommand{show}\isamarkupfalse%
\ {\isachardoublequoteopen}forces{\isacharunderscore}{\kern0pt}eq{\isacharparenleft}{\kern0pt}p{\isacharcomma}{\kern0pt}t{\isadigit{1}}{\isacharcomma}{\kern0pt}t{\isadigit{2}}{\isacharparenright}{\kern0pt}{\isachardoublequoteclose}\ \isacommand{using}\isamarkupfalse%
\ def{\isacharunderscore}{\kern0pt}forces{\isacharunderscore}{\kern0pt}eq\ \isacommand{by}\isamarkupfalse%
\ blast\isanewline
\isacommand{qed}\isamarkupfalse%
%
\endisatagproof
{\isafoldproof}%
%
\isadelimproof
%
\endisadelimproof
%
\isadelimdocument
%
\endisadelimdocument
%
\isatagdocument
%
\isamarkupsubsection{Kunen 2013, Lemma IV.2.38%
}
\isamarkuptrue%
%
\endisatagdocument
{\isafolddocument}%
%
\isadelimdocument
%
\endisadelimdocument
\isacommand{lemma}\isamarkupfalse%
\ not{\isacharunderscore}{\kern0pt}forces{\isacharunderscore}{\kern0pt}neq{\isacharcolon}{\kern0pt}\isanewline
\ \ \isakeyword{assumes}\ {\isachardoublequoteopen}p{\isasymin}P{\isachardoublequoteclose}\isanewline
\ \ \isakeyword{shows}\ {\isachardoublequoteopen}forces{\isacharunderscore}{\kern0pt}eq{\isacharparenleft}{\kern0pt}p{\isacharcomma}{\kern0pt}t{\isadigit{1}}{\isacharcomma}{\kern0pt}t{\isadigit{2}}{\isacharparenright}{\kern0pt}\ {\isasymlongleftrightarrow}\ {\isasymnot}\ {\isacharparenleft}{\kern0pt}{\isasymexists}q{\isasymin}P{\isachardot}{\kern0pt}\ q{\isasympreceq}p\ {\isasymand}\ forces{\isacharunderscore}{\kern0pt}neq{\isacharparenleft}{\kern0pt}q{\isacharcomma}{\kern0pt}t{\isadigit{1}}{\isacharcomma}{\kern0pt}t{\isadigit{2}}{\isacharparenright}{\kern0pt}{\isacharparenright}{\kern0pt}{\isachardoublequoteclose}\isanewline
%
\isadelimproof
\ \ %
\endisadelimproof
%
\isatagproof
\isacommand{using}\isamarkupfalse%
\ assms\ density{\isacharunderscore}{\kern0pt}eq\ \isacommand{unfolding}\isamarkupfalse%
\ forces{\isacharunderscore}{\kern0pt}neq{\isacharunderscore}{\kern0pt}def\ \isacommand{by}\isamarkupfalse%
\ blast%
\endisatagproof
{\isafoldproof}%
%
\isadelimproof
\isanewline
%
\endisadelimproof
\isanewline
\isanewline
\isacommand{lemma}\isamarkupfalse%
\ not{\isacharunderscore}{\kern0pt}forces{\isacharunderscore}{\kern0pt}nmem{\isacharcolon}{\kern0pt}\isanewline
\ \ \isakeyword{assumes}\ {\isachardoublequoteopen}p{\isasymin}P{\isachardoublequoteclose}\isanewline
\ \ \isakeyword{shows}\ {\isachardoublequoteopen}forces{\isacharunderscore}{\kern0pt}mem{\isacharparenleft}{\kern0pt}p{\isacharcomma}{\kern0pt}t{\isadigit{1}}{\isacharcomma}{\kern0pt}t{\isadigit{2}}{\isacharparenright}{\kern0pt}\ {\isasymlongleftrightarrow}\ {\isasymnot}\ {\isacharparenleft}{\kern0pt}{\isasymexists}q{\isasymin}P{\isachardot}{\kern0pt}\ q{\isasympreceq}p\ {\isasymand}\ forces{\isacharunderscore}{\kern0pt}nmem{\isacharparenleft}{\kern0pt}q{\isacharcomma}{\kern0pt}t{\isadigit{1}}{\isacharcomma}{\kern0pt}t{\isadigit{2}}{\isacharparenright}{\kern0pt}{\isacharparenright}{\kern0pt}{\isachardoublequoteclose}\isanewline
%
\isadelimproof
\ \ %
\endisadelimproof
%
\isatagproof
\isacommand{using}\isamarkupfalse%
\ assms\ density{\isacharunderscore}{\kern0pt}mem\ \isacommand{unfolding}\isamarkupfalse%
\ forces{\isacharunderscore}{\kern0pt}nmem{\isacharunderscore}{\kern0pt}def\ \isacommand{by}\isamarkupfalse%
\ blast%
\endisatagproof
{\isafoldproof}%
%
\isadelimproof
\isanewline
%
\endisadelimproof
\isanewline
\isanewline
\isanewline
\isanewline
\isanewline
\isacommand{lemma}\isamarkupfalse%
\ sats{\isacharunderscore}{\kern0pt}forces{\isacharunderscore}{\kern0pt}Nand{\isacharprime}{\kern0pt}{\isacharcolon}{\kern0pt}\isanewline
\ \ \isakeyword{assumes}\isanewline
\ \ \ \ {\isachardoublequoteopen}p{\isasymin}P{\isachardoublequoteclose}\ {\isachardoublequoteopen}{\isasymphi}{\isasymin}formula{\isachardoublequoteclose}\ {\isachardoublequoteopen}{\isasympsi}{\isasymin}formula{\isachardoublequoteclose}\ {\isachardoublequoteopen}env\ {\isasymin}\ list{\isacharparenleft}{\kern0pt}M{\isacharparenright}{\kern0pt}{\isachardoublequoteclose}\ \isanewline
\ \ \isakeyword{shows}\isanewline
\ \ \ \ {\isachardoublequoteopen}M{\isacharcomma}{\kern0pt}\ {\isacharbrackleft}{\kern0pt}p{\isacharcomma}{\kern0pt}P{\isacharcomma}{\kern0pt}leq{\isacharcomma}{\kern0pt}one{\isacharbrackright}{\kern0pt}\ {\isacharat}{\kern0pt}\ env\ {\isasymTurnstile}\ forces{\isacharparenleft}{\kern0pt}Nand{\isacharparenleft}{\kern0pt}{\isasymphi}{\isacharcomma}{\kern0pt}{\isasympsi}{\isacharparenright}{\kern0pt}{\isacharparenright}{\kern0pt}\ {\isasymlongleftrightarrow}\ \isanewline
\ \ \ \ \ {\isasymnot}{\isacharparenleft}{\kern0pt}{\isasymexists}q{\isasymin}M{\isachardot}{\kern0pt}\ q{\isasymin}P\ {\isasymand}\ is{\isacharunderscore}{\kern0pt}leq{\isacharparenleft}{\kern0pt}{\isacharhash}{\kern0pt}{\isacharhash}{\kern0pt}M{\isacharcomma}{\kern0pt}leq{\isacharcomma}{\kern0pt}q{\isacharcomma}{\kern0pt}p{\isacharparenright}{\kern0pt}\ {\isasymand}\ \isanewline
\ \ \ \ \ \ \ \ \ \ \ M{\isacharcomma}{\kern0pt}\ {\isacharbrackleft}{\kern0pt}q{\isacharcomma}{\kern0pt}P{\isacharcomma}{\kern0pt}leq{\isacharcomma}{\kern0pt}one{\isacharbrackright}{\kern0pt}\ {\isacharat}{\kern0pt}\ env\ {\isasymTurnstile}\ forces{\isacharparenleft}{\kern0pt}{\isasymphi}{\isacharparenright}{\kern0pt}\ {\isasymand}\ \isanewline
\ \ \ \ \ \ \ \ \ \ \ M{\isacharcomma}{\kern0pt}\ {\isacharbrackleft}{\kern0pt}q{\isacharcomma}{\kern0pt}P{\isacharcomma}{\kern0pt}leq{\isacharcomma}{\kern0pt}one{\isacharbrackright}{\kern0pt}\ {\isacharat}{\kern0pt}\ env\ {\isasymTurnstile}\ forces{\isacharparenleft}{\kern0pt}{\isasympsi}{\isacharparenright}{\kern0pt}{\isacharparenright}{\kern0pt}{\isachardoublequoteclose}\isanewline
%
\isadelimproof
\ \ %
\endisadelimproof
%
\isatagproof
\isacommand{using}\isamarkupfalse%
\ assms\ sats{\isacharunderscore}{\kern0pt}forces{\isacharunderscore}{\kern0pt}Nand{\isacharbrackleft}{\kern0pt}OF\ assms{\isacharparenleft}{\kern0pt}{\isadigit{2}}{\isacharminus}{\kern0pt}{\isadigit{4}}{\isacharparenright}{\kern0pt}\ transitivity{\isacharbrackleft}{\kern0pt}OF\ {\isacartoucheopen}p{\isasymin}P{\isacartoucheclose}{\isacharbrackright}{\kern0pt}{\isacharbrackright}{\kern0pt}\isanewline
\ \ P{\isacharunderscore}{\kern0pt}in{\isacharunderscore}{\kern0pt}M\ leq{\isacharunderscore}{\kern0pt}in{\isacharunderscore}{\kern0pt}M\ one{\isacharunderscore}{\kern0pt}in{\isacharunderscore}{\kern0pt}M\ \isacommand{unfolding}\isamarkupfalse%
\ forces{\isacharunderscore}{\kern0pt}def\isanewline
\ \ \isacommand{by}\isamarkupfalse%
\ simp%
\endisatagproof
{\isafoldproof}%
%
\isadelimproof
\isanewline
%
\endisadelimproof
\isanewline
\isacommand{lemma}\isamarkupfalse%
\ sats{\isacharunderscore}{\kern0pt}forces{\isacharunderscore}{\kern0pt}Neg{\isacharprime}{\kern0pt}{\isacharcolon}{\kern0pt}\isanewline
\ \ \isakeyword{assumes}\isanewline
\ \ \ \ {\isachardoublequoteopen}p{\isasymin}P{\isachardoublequoteclose}\ {\isachardoublequoteopen}env\ {\isasymin}\ list{\isacharparenleft}{\kern0pt}M{\isacharparenright}{\kern0pt}{\isachardoublequoteclose}\ {\isachardoublequoteopen}{\isasymphi}{\isasymin}formula{\isachardoublequoteclose}\isanewline
\ \ \isakeyword{shows}\isanewline
\ \ \ \ {\isachardoublequoteopen}M{\isacharcomma}{\kern0pt}\ {\isacharbrackleft}{\kern0pt}p{\isacharcomma}{\kern0pt}P{\isacharcomma}{\kern0pt}leq{\isacharcomma}{\kern0pt}one{\isacharbrackright}{\kern0pt}\ {\isacharat}{\kern0pt}\ env\ {\isasymTurnstile}\ forces{\isacharparenleft}{\kern0pt}Neg{\isacharparenleft}{\kern0pt}{\isasymphi}{\isacharparenright}{\kern0pt}{\isacharparenright}{\kern0pt}\ \ \ {\isasymlongleftrightarrow}\ \isanewline
\ \ \ \ \ {\isasymnot}{\isacharparenleft}{\kern0pt}{\isasymexists}q{\isasymin}M{\isachardot}{\kern0pt}\ q{\isasymin}P\ {\isasymand}\ is{\isacharunderscore}{\kern0pt}leq{\isacharparenleft}{\kern0pt}{\isacharhash}{\kern0pt}{\isacharhash}{\kern0pt}M{\isacharcomma}{\kern0pt}leq{\isacharcomma}{\kern0pt}q{\isacharcomma}{\kern0pt}p{\isacharparenright}{\kern0pt}\ {\isasymand}\ \isanewline
\ \ \ \ \ \ \ \ \ \ M{\isacharcomma}{\kern0pt}\ {\isacharbrackleft}{\kern0pt}q{\isacharcomma}{\kern0pt}P{\isacharcomma}{\kern0pt}leq{\isacharcomma}{\kern0pt}one{\isacharbrackright}{\kern0pt}{\isacharat}{\kern0pt}env\ {\isasymTurnstile}\ forces{\isacharparenleft}{\kern0pt}{\isasymphi}{\isacharparenright}{\kern0pt}{\isacharparenright}{\kern0pt}{\isachardoublequoteclose}\isanewline
%
\isadelimproof
\ \ %
\endisadelimproof
%
\isatagproof
\isacommand{using}\isamarkupfalse%
\ assms\ sats{\isacharunderscore}{\kern0pt}forces{\isacharunderscore}{\kern0pt}Neg\ transitivity\ \isanewline
\ \ P{\isacharunderscore}{\kern0pt}in{\isacharunderscore}{\kern0pt}M\ leq{\isacharunderscore}{\kern0pt}in{\isacharunderscore}{\kern0pt}M\ one{\isacharunderscore}{\kern0pt}in{\isacharunderscore}{\kern0pt}M\ \ \isacommand{unfolding}\isamarkupfalse%
\ forces{\isacharunderscore}{\kern0pt}def\isanewline
\ \ \isacommand{by}\isamarkupfalse%
\ {\isacharparenleft}{\kern0pt}simp{\isacharcomma}{\kern0pt}\ blast{\isacharparenright}{\kern0pt}%
\endisatagproof
{\isafoldproof}%
%
\isadelimproof
\isanewline
%
\endisadelimproof
\isanewline
\isacommand{lemma}\isamarkupfalse%
\ sats{\isacharunderscore}{\kern0pt}forces{\isacharunderscore}{\kern0pt}Forall{\isacharprime}{\kern0pt}{\isacharcolon}{\kern0pt}\isanewline
\ \ \isakeyword{assumes}\isanewline
\ \ \ \ {\isachardoublequoteopen}p{\isasymin}P{\isachardoublequoteclose}\ {\isachardoublequoteopen}env\ {\isasymin}\ list{\isacharparenleft}{\kern0pt}M{\isacharparenright}{\kern0pt}{\isachardoublequoteclose}\ {\isachardoublequoteopen}{\isasymphi}{\isasymin}formula{\isachardoublequoteclose}\isanewline
\ \ \isakeyword{shows}\isanewline
\ \ \ \ {\isachardoublequoteopen}M{\isacharcomma}{\kern0pt}{\isacharbrackleft}{\kern0pt}p{\isacharcomma}{\kern0pt}P{\isacharcomma}{\kern0pt}leq{\isacharcomma}{\kern0pt}one{\isacharbrackright}{\kern0pt}\ {\isacharat}{\kern0pt}\ env\ {\isasymTurnstile}\ forces{\isacharparenleft}{\kern0pt}Forall{\isacharparenleft}{\kern0pt}{\isasymphi}{\isacharparenright}{\kern0pt}{\isacharparenright}{\kern0pt}\ {\isasymlongleftrightarrow}\ \isanewline
\ \ \ \ \ {\isacharparenleft}{\kern0pt}{\isasymforall}x{\isasymin}M{\isachardot}{\kern0pt}\ \ \ M{\isacharcomma}{\kern0pt}\ {\isacharbrackleft}{\kern0pt}p{\isacharcomma}{\kern0pt}P{\isacharcomma}{\kern0pt}leq{\isacharcomma}{\kern0pt}one{\isacharcomma}{\kern0pt}x{\isacharbrackright}{\kern0pt}\ {\isacharat}{\kern0pt}\ env\ {\isasymTurnstile}\ forces{\isacharparenleft}{\kern0pt}{\isasymphi}{\isacharparenright}{\kern0pt}{\isacharparenright}{\kern0pt}{\isachardoublequoteclose}\isanewline
%
\isadelimproof
\ \ %
\endisadelimproof
%
\isatagproof
\isacommand{using}\isamarkupfalse%
\ assms\ sats{\isacharunderscore}{\kern0pt}forces{\isacharunderscore}{\kern0pt}Forall\ transitivity\ \isanewline
\ \ P{\isacharunderscore}{\kern0pt}in{\isacharunderscore}{\kern0pt}M\ leq{\isacharunderscore}{\kern0pt}in{\isacharunderscore}{\kern0pt}M\ one{\isacharunderscore}{\kern0pt}in{\isacharunderscore}{\kern0pt}M\ sats{\isacharunderscore}{\kern0pt}ren{\isacharunderscore}{\kern0pt}forces{\isacharunderscore}{\kern0pt}forall\ \isacommand{unfolding}\isamarkupfalse%
\ forces{\isacharunderscore}{\kern0pt}def\isanewline
\ \ \isacommand{by}\isamarkupfalse%
\ simp%
\endisatagproof
{\isafoldproof}%
%
\isadelimproof
%
\endisadelimproof
%
\isadelimdocument
%
\endisadelimdocument
%
\isatagdocument
%
\isamarkupsubsection{The relation of forcing and atomic formulas%
}
\isamarkuptrue%
%
\endisatagdocument
{\isafolddocument}%
%
\isadelimdocument
%
\endisadelimdocument
\isacommand{lemma}\isamarkupfalse%
\ Forces{\isacharunderscore}{\kern0pt}Equal{\isacharcolon}{\kern0pt}\isanewline
\ \ \isakeyword{assumes}\isanewline
\ \ \ \ {\isachardoublequoteopen}p{\isasymin}P{\isachardoublequoteclose}\ {\isachardoublequoteopen}t{\isadigit{1}}{\isasymin}M{\isachardoublequoteclose}\ {\isachardoublequoteopen}t{\isadigit{2}}{\isasymin}M{\isachardoublequoteclose}\ {\isachardoublequoteopen}env{\isasymin}list{\isacharparenleft}{\kern0pt}M{\isacharparenright}{\kern0pt}{\isachardoublequoteclose}\ {\isachardoublequoteopen}nth{\isacharparenleft}{\kern0pt}n{\isacharcomma}{\kern0pt}env{\isacharparenright}{\kern0pt}\ {\isacharequal}{\kern0pt}\ t{\isadigit{1}}{\isachardoublequoteclose}\ {\isachardoublequoteopen}nth{\isacharparenleft}{\kern0pt}m{\isacharcomma}{\kern0pt}env{\isacharparenright}{\kern0pt}\ {\isacharequal}{\kern0pt}\ t{\isadigit{2}}{\isachardoublequoteclose}\ {\isachardoublequoteopen}n{\isasymin}nat{\isachardoublequoteclose}\ {\isachardoublequoteopen}m{\isasymin}nat{\isachardoublequoteclose}\ \isanewline
\ \ \isakeyword{shows}\isanewline
\ \ \ \ {\isachardoublequoteopen}{\isacharparenleft}{\kern0pt}p\ {\isasymtturnstile}\ Equal{\isacharparenleft}{\kern0pt}n{\isacharcomma}{\kern0pt}m{\isacharparenright}{\kern0pt}\ env{\isacharparenright}{\kern0pt}\ {\isasymlongleftrightarrow}\ forces{\isacharunderscore}{\kern0pt}eq{\isacharparenleft}{\kern0pt}p{\isacharcomma}{\kern0pt}t{\isadigit{1}}{\isacharcomma}{\kern0pt}t{\isadigit{2}}{\isacharparenright}{\kern0pt}{\isachardoublequoteclose}\isanewline
%
\isadelimproof
\ \ \ %
\endisadelimproof
%
\isatagproof
\isacommand{using}\isamarkupfalse%
\ assms\ sats{\isacharunderscore}{\kern0pt}forces{\isacharunderscore}{\kern0pt}Equal\ forces{\isacharunderscore}{\kern0pt}eq{\isacharunderscore}{\kern0pt}abs\ transitivity\ P{\isacharunderscore}{\kern0pt}in{\isacharunderscore}{\kern0pt}M\ \isanewline
\ \ \isacommand{by}\isamarkupfalse%
\ simp%
\endisatagproof
{\isafoldproof}%
%
\isadelimproof
\isanewline
%
\endisadelimproof
\isanewline
\isacommand{lemma}\isamarkupfalse%
\ Forces{\isacharunderscore}{\kern0pt}Member{\isacharcolon}{\kern0pt}\isanewline
\ \ \isakeyword{assumes}\isanewline
\ \ \ \ {\isachardoublequoteopen}p{\isasymin}P{\isachardoublequoteclose}\ {\isachardoublequoteopen}t{\isadigit{1}}{\isasymin}M{\isachardoublequoteclose}\ {\isachardoublequoteopen}t{\isadigit{2}}{\isasymin}M{\isachardoublequoteclose}\ {\isachardoublequoteopen}env{\isasymin}list{\isacharparenleft}{\kern0pt}M{\isacharparenright}{\kern0pt}{\isachardoublequoteclose}\ {\isachardoublequoteopen}nth{\isacharparenleft}{\kern0pt}n{\isacharcomma}{\kern0pt}env{\isacharparenright}{\kern0pt}\ {\isacharequal}{\kern0pt}\ t{\isadigit{1}}{\isachardoublequoteclose}\ {\isachardoublequoteopen}nth{\isacharparenleft}{\kern0pt}m{\isacharcomma}{\kern0pt}env{\isacharparenright}{\kern0pt}\ {\isacharequal}{\kern0pt}\ t{\isadigit{2}}{\isachardoublequoteclose}\ {\isachardoublequoteopen}n{\isasymin}nat{\isachardoublequoteclose}\ {\isachardoublequoteopen}m{\isasymin}nat{\isachardoublequoteclose}\ \isanewline
\ \ \isakeyword{shows}\isanewline
\ \ \ \ {\isachardoublequoteopen}{\isacharparenleft}{\kern0pt}p\ {\isasymtturnstile}\ Member{\isacharparenleft}{\kern0pt}n{\isacharcomma}{\kern0pt}m{\isacharparenright}{\kern0pt}\ env{\isacharparenright}{\kern0pt}\ {\isasymlongleftrightarrow}\ forces{\isacharunderscore}{\kern0pt}mem{\isacharparenleft}{\kern0pt}p{\isacharcomma}{\kern0pt}t{\isadigit{1}}{\isacharcomma}{\kern0pt}t{\isadigit{2}}{\isacharparenright}{\kern0pt}{\isachardoublequoteclose}\isanewline
%
\isadelimproof
\ \ \ %
\endisadelimproof
%
\isatagproof
\isacommand{using}\isamarkupfalse%
\ assms\ sats{\isacharunderscore}{\kern0pt}forces{\isacharunderscore}{\kern0pt}Member\ forces{\isacharunderscore}{\kern0pt}mem{\isacharunderscore}{\kern0pt}abs\ transitivity\ P{\isacharunderscore}{\kern0pt}in{\isacharunderscore}{\kern0pt}M\isanewline
\ \ \isacommand{by}\isamarkupfalse%
\ simp%
\endisatagproof
{\isafoldproof}%
%
\isadelimproof
\isanewline
%
\endisadelimproof
\isanewline
\isacommand{lemma}\isamarkupfalse%
\ Forces{\isacharunderscore}{\kern0pt}Neg{\isacharcolon}{\kern0pt}\isanewline
\ \ \isakeyword{assumes}\isanewline
\ \ \ \ {\isachardoublequoteopen}p{\isasymin}P{\isachardoublequoteclose}\ {\isachardoublequoteopen}env\ {\isasymin}\ list{\isacharparenleft}{\kern0pt}M{\isacharparenright}{\kern0pt}{\isachardoublequoteclose}\ {\isachardoublequoteopen}{\isasymphi}{\isasymin}formula{\isachardoublequoteclose}\ \isanewline
\ \ \isakeyword{shows}\isanewline
\ \ \ \ {\isachardoublequoteopen}{\isacharparenleft}{\kern0pt}p\ {\isasymtturnstile}\ Neg{\isacharparenleft}{\kern0pt}{\isasymphi}{\isacharparenright}{\kern0pt}\ env{\isacharparenright}{\kern0pt}\ {\isasymlongleftrightarrow}\ {\isasymnot}{\isacharparenleft}{\kern0pt}{\isasymexists}q{\isasymin}M{\isachardot}{\kern0pt}\ q{\isasymin}P\ {\isasymand}\ q{\isasympreceq}p\ {\isasymand}\ {\isacharparenleft}{\kern0pt}q\ {\isasymtturnstile}\ {\isasymphi}\ env{\isacharparenright}{\kern0pt}{\isacharparenright}{\kern0pt}{\isachardoublequoteclose}\isanewline
%
\isadelimproof
\ \ \ \ %
\endisadelimproof
%
\isatagproof
\isacommand{using}\isamarkupfalse%
\ assms\ sats{\isacharunderscore}{\kern0pt}forces{\isacharunderscore}{\kern0pt}Neg{\isacharprime}{\kern0pt}\ transitivity\ \isanewline
\ \ P{\isacharunderscore}{\kern0pt}in{\isacharunderscore}{\kern0pt}M\ pair{\isacharunderscore}{\kern0pt}in{\isacharunderscore}{\kern0pt}M{\isacharunderscore}{\kern0pt}iff\ leq{\isacharunderscore}{\kern0pt}in{\isacharunderscore}{\kern0pt}M\ leq{\isacharunderscore}{\kern0pt}abs\ \isacommand{by}\isamarkupfalse%
\ simp%
\endisatagproof
{\isafoldproof}%
%
\isadelimproof
%
\endisadelimproof
%
\isadelimdocument
%
\endisadelimdocument
%
\isatagdocument
%
\isamarkupsubsection{The relation of forcing and connectives%
}
\isamarkuptrue%
%
\endisatagdocument
{\isafolddocument}%
%
\isadelimdocument
%
\endisadelimdocument
\isacommand{lemma}\isamarkupfalse%
\ Forces{\isacharunderscore}{\kern0pt}Nand{\isacharcolon}{\kern0pt}\isanewline
\ \ \isakeyword{assumes}\isanewline
\ \ \ \ {\isachardoublequoteopen}p{\isasymin}P{\isachardoublequoteclose}\ {\isachardoublequoteopen}env\ {\isasymin}\ list{\isacharparenleft}{\kern0pt}M{\isacharparenright}{\kern0pt}{\isachardoublequoteclose}\ {\isachardoublequoteopen}{\isasymphi}{\isasymin}formula{\isachardoublequoteclose}\ {\isachardoublequoteopen}{\isasympsi}{\isasymin}formula{\isachardoublequoteclose}\isanewline
\ \ \isakeyword{shows}\isanewline
\ \ \ \ {\isachardoublequoteopen}{\isacharparenleft}{\kern0pt}p\ {\isasymtturnstile}\ Nand{\isacharparenleft}{\kern0pt}{\isasymphi}{\isacharcomma}{\kern0pt}{\isasympsi}{\isacharparenright}{\kern0pt}\ env{\isacharparenright}{\kern0pt}\ {\isasymlongleftrightarrow}\ {\isasymnot}{\isacharparenleft}{\kern0pt}{\isasymexists}q{\isasymin}M{\isachardot}{\kern0pt}\ q{\isasymin}P\ {\isasymand}\ q{\isasympreceq}p\ {\isasymand}\ {\isacharparenleft}{\kern0pt}q\ {\isasymtturnstile}\ {\isasymphi}\ env{\isacharparenright}{\kern0pt}\ {\isasymand}\ {\isacharparenleft}{\kern0pt}q\ {\isasymtturnstile}\ {\isasympsi}\ env{\isacharparenright}{\kern0pt}{\isacharparenright}{\kern0pt}{\isachardoublequoteclose}\isanewline
%
\isadelimproof
\ \ \ %
\endisadelimproof
%
\isatagproof
\isacommand{using}\isamarkupfalse%
\ assms\ sats{\isacharunderscore}{\kern0pt}forces{\isacharunderscore}{\kern0pt}Nand{\isacharprime}{\kern0pt}\ transitivity\ \isanewline
\ \ P{\isacharunderscore}{\kern0pt}in{\isacharunderscore}{\kern0pt}M\ pair{\isacharunderscore}{\kern0pt}in{\isacharunderscore}{\kern0pt}M{\isacharunderscore}{\kern0pt}iff\ leq{\isacharunderscore}{\kern0pt}in{\isacharunderscore}{\kern0pt}M\ leq{\isacharunderscore}{\kern0pt}abs\ \isacommand{by}\isamarkupfalse%
\ simp%
\endisatagproof
{\isafoldproof}%
%
\isadelimproof
\isanewline
%
\endisadelimproof
\isanewline
\isacommand{lemma}\isamarkupfalse%
\ Forces{\isacharunderscore}{\kern0pt}And{\isacharunderscore}{\kern0pt}aux{\isacharcolon}{\kern0pt}\isanewline
\ \ \isakeyword{assumes}\isanewline
\ \ \ \ {\isachardoublequoteopen}p{\isasymin}P{\isachardoublequoteclose}\ {\isachardoublequoteopen}env\ {\isasymin}\ list{\isacharparenleft}{\kern0pt}M{\isacharparenright}{\kern0pt}{\isachardoublequoteclose}\ {\isachardoublequoteopen}{\isasymphi}{\isasymin}formula{\isachardoublequoteclose}\ {\isachardoublequoteopen}{\isasympsi}{\isasymin}formula{\isachardoublequoteclose}\isanewline
\ \ \isakeyword{shows}\isanewline
\ \ \ \ {\isachardoublequoteopen}p\ {\isasymtturnstile}\ And{\isacharparenleft}{\kern0pt}{\isasymphi}{\isacharcomma}{\kern0pt}{\isasympsi}{\isacharparenright}{\kern0pt}\ env\ \ \ {\isasymlongleftrightarrow}\ \isanewline
\ \ \ \ {\isacharparenleft}{\kern0pt}{\isasymforall}q{\isasymin}M{\isachardot}{\kern0pt}\ q{\isasymin}P\ {\isasymand}\ q{\isasympreceq}p\ {\isasymlongrightarrow}\ {\isacharparenleft}{\kern0pt}{\isasymexists}r{\isasymin}M{\isachardot}{\kern0pt}\ r{\isasymin}P\ {\isasymand}\ r{\isasympreceq}q\ {\isasymand}\ {\isacharparenleft}{\kern0pt}r\ {\isasymtturnstile}\ {\isasymphi}\ env{\isacharparenright}{\kern0pt}\ {\isasymand}\ {\isacharparenleft}{\kern0pt}r\ {\isasymtturnstile}\ {\isasympsi}\ env{\isacharparenright}{\kern0pt}{\isacharparenright}{\kern0pt}{\isacharparenright}{\kern0pt}{\isachardoublequoteclose}\isanewline
%
\isadelimproof
\ \ %
\endisadelimproof
%
\isatagproof
\isacommand{unfolding}\isamarkupfalse%
\ And{\isacharunderscore}{\kern0pt}def\ \isacommand{using}\isamarkupfalse%
\ assms\ Forces{\isacharunderscore}{\kern0pt}Neg\ Forces{\isacharunderscore}{\kern0pt}Nand\ \isacommand{by}\isamarkupfalse%
\ {\isacharparenleft}{\kern0pt}auto\ simp\ only{\isacharcolon}{\kern0pt}{\isacharparenright}{\kern0pt}%
\endisatagproof
{\isafoldproof}%
%
\isadelimproof
\isanewline
%
\endisadelimproof
\isanewline
\isacommand{lemma}\isamarkupfalse%
\ Forces{\isacharunderscore}{\kern0pt}And{\isacharunderscore}{\kern0pt}iff{\isacharunderscore}{\kern0pt}dense{\isacharunderscore}{\kern0pt}below{\isacharcolon}{\kern0pt}\isanewline
\ \ \isakeyword{assumes}\isanewline
\ \ \ \ {\isachardoublequoteopen}p{\isasymin}P{\isachardoublequoteclose}\ {\isachardoublequoteopen}env\ {\isasymin}\ list{\isacharparenleft}{\kern0pt}M{\isacharparenright}{\kern0pt}{\isachardoublequoteclose}\ {\isachardoublequoteopen}{\isasymphi}{\isasymin}formula{\isachardoublequoteclose}\ {\isachardoublequoteopen}{\isasympsi}{\isasymin}formula{\isachardoublequoteclose}\isanewline
\ \ \isakeyword{shows}\isanewline
\ \ \ \ {\isachardoublequoteopen}{\isacharparenleft}{\kern0pt}p\ {\isasymtturnstile}\ And{\isacharparenleft}{\kern0pt}{\isasymphi}{\isacharcomma}{\kern0pt}{\isasympsi}{\isacharparenright}{\kern0pt}\ env{\isacharparenright}{\kern0pt}\ {\isasymlongleftrightarrow}\ dense{\isacharunderscore}{\kern0pt}below{\isacharparenleft}{\kern0pt}{\isacharbraceleft}{\kern0pt}r{\isasymin}P{\isachardot}{\kern0pt}\ {\isacharparenleft}{\kern0pt}r\ {\isasymtturnstile}\ {\isasymphi}\ env{\isacharparenright}{\kern0pt}\ {\isasymand}\ {\isacharparenleft}{\kern0pt}r\ {\isasymtturnstile}\ {\isasympsi}\ env{\isacharparenright}{\kern0pt}\ {\isacharbraceright}{\kern0pt}{\isacharcomma}{\kern0pt}p{\isacharparenright}{\kern0pt}{\isachardoublequoteclose}\isanewline
%
\isadelimproof
\ \ %
\endisadelimproof
%
\isatagproof
\isacommand{unfolding}\isamarkupfalse%
\ dense{\isacharunderscore}{\kern0pt}below{\isacharunderscore}{\kern0pt}def\ \isacommand{using}\isamarkupfalse%
\ Forces{\isacharunderscore}{\kern0pt}And{\isacharunderscore}{\kern0pt}aux\ assms\isanewline
\ \ \ \ \isacommand{by}\isamarkupfalse%
\ {\isacharparenleft}{\kern0pt}auto\ dest{\isacharcolon}{\kern0pt}transitivity{\isacharbrackleft}{\kern0pt}OF\ {\isacharunderscore}{\kern0pt}\ P{\isacharunderscore}{\kern0pt}in{\isacharunderscore}{\kern0pt}M{\isacharbrackright}{\kern0pt}{\isacharsemicolon}{\kern0pt}\ rename{\isacharunderscore}{\kern0pt}tac\ q{\isacharsemicolon}{\kern0pt}\ drule{\isacharunderscore}{\kern0pt}tac\ x{\isacharequal}{\kern0pt}q\ \isakeyword{in}\ bspec{\isacharparenright}{\kern0pt}{\isacharplus}{\kern0pt}%
\endisatagproof
{\isafoldproof}%
%
\isadelimproof
\isanewline
%
\endisadelimproof
\isanewline
\isacommand{lemma}\isamarkupfalse%
\ Forces{\isacharunderscore}{\kern0pt}Forall{\isacharcolon}{\kern0pt}\isanewline
\ \ \isakeyword{assumes}\isanewline
\ \ \ \ {\isachardoublequoteopen}p{\isasymin}P{\isachardoublequoteclose}\ {\isachardoublequoteopen}env\ {\isasymin}\ list{\isacharparenleft}{\kern0pt}M{\isacharparenright}{\kern0pt}{\isachardoublequoteclose}\ {\isachardoublequoteopen}{\isasymphi}{\isasymin}formula{\isachardoublequoteclose}\isanewline
\ \ \isakeyword{shows}\isanewline
\ \ \ \ {\isachardoublequoteopen}{\isacharparenleft}{\kern0pt}p\ {\isasymtturnstile}\ Forall{\isacharparenleft}{\kern0pt}{\isasymphi}{\isacharparenright}{\kern0pt}\ env{\isacharparenright}{\kern0pt}\ {\isasymlongleftrightarrow}\ {\isacharparenleft}{\kern0pt}{\isasymforall}x{\isasymin}M{\isachardot}{\kern0pt}\ {\isacharparenleft}{\kern0pt}p\ {\isasymtturnstile}\ {\isasymphi}\ {\isacharparenleft}{\kern0pt}{\isacharbrackleft}{\kern0pt}x{\isacharbrackright}{\kern0pt}\ {\isacharat}{\kern0pt}\ env{\isacharparenright}{\kern0pt}{\isacharparenright}{\kern0pt}{\isacharparenright}{\kern0pt}{\isachardoublequoteclose}\isanewline
%
\isadelimproof
\ \ \ %
\endisadelimproof
%
\isatagproof
\isacommand{using}\isamarkupfalse%
\ sats{\isacharunderscore}{\kern0pt}forces{\isacharunderscore}{\kern0pt}Forall{\isacharprime}{\kern0pt}\ assms\ \isacommand{by}\isamarkupfalse%
\ simp%
\endisatagproof
{\isafoldproof}%
%
\isadelimproof
\isanewline
%
\endisadelimproof
\isanewline
\isanewline
\isacommand{bundle}\isamarkupfalse%
\ some{\isacharunderscore}{\kern0pt}rules\ {\isacharequal}{\kern0pt}\ \ elem{\isacharunderscore}{\kern0pt}of{\isacharunderscore}{\kern0pt}val{\isacharunderscore}{\kern0pt}pair\ {\isacharbrackleft}{\kern0pt}dest{\isacharbrackright}{\kern0pt}\ SepReplace{\isacharunderscore}{\kern0pt}iff\ {\isacharbrackleft}{\kern0pt}simp\ del{\isacharbrackright}{\kern0pt}\ SepReplace{\isacharunderscore}{\kern0pt}iff{\isacharbrackleft}{\kern0pt}iff{\isacharbrackright}{\kern0pt}\isanewline
\isanewline
\isacommand{context}\isamarkupfalse%
\ \isanewline
\ \ \isakeyword{includes}\ some{\isacharunderscore}{\kern0pt}rules\isanewline
\isakeyword{begin}\isanewline
\isanewline
\isacommand{lemma}\isamarkupfalse%
\ elem{\isacharunderscore}{\kern0pt}of{\isacharunderscore}{\kern0pt}valI{\isacharcolon}{\kern0pt}\ {\isachardoublequoteopen}{\isasymexists}{\isasymtheta}{\isachardot}{\kern0pt}\ {\isasymexists}p{\isasymin}P{\isachardot}{\kern0pt}\ p{\isasymin}G\ {\isasymand}\ {\isasymlangle}{\isasymtheta}{\isacharcomma}{\kern0pt}p{\isasymrangle}{\isasymin}{\isasympi}\ {\isasymand}\ val{\isacharparenleft}{\kern0pt}G{\isacharcomma}{\kern0pt}{\isasymtheta}{\isacharparenright}{\kern0pt}\ {\isacharequal}{\kern0pt}\ x\ {\isasymLongrightarrow}\ x{\isasymin}val{\isacharparenleft}{\kern0pt}G{\isacharcomma}{\kern0pt}{\isasympi}{\isacharparenright}{\kern0pt}{\isachardoublequoteclose}\isanewline
%
\isadelimproof
\ \ %
\endisadelimproof
%
\isatagproof
\isacommand{by}\isamarkupfalse%
\ {\isacharparenleft}{\kern0pt}subst\ def{\isacharunderscore}{\kern0pt}val{\isacharcomma}{\kern0pt}\ auto{\isacharparenright}{\kern0pt}%
\endisatagproof
{\isafoldproof}%
%
\isadelimproof
\isanewline
%
\endisadelimproof
\isanewline
\isacommand{lemma}\isamarkupfalse%
\ GenExtD{\isacharcolon}{\kern0pt}\ {\isachardoublequoteopen}x{\isasymin}M{\isacharbrackleft}{\kern0pt}G{\isacharbrackright}{\kern0pt}\ {\isasymlongleftrightarrow}\ {\isacharparenleft}{\kern0pt}{\isasymexists}{\isasymtau}{\isasymin}M{\isachardot}{\kern0pt}\ x\ {\isacharequal}{\kern0pt}\ val{\isacharparenleft}{\kern0pt}G{\isacharcomma}{\kern0pt}{\isasymtau}{\isacharparenright}{\kern0pt}{\isacharparenright}{\kern0pt}{\isachardoublequoteclose}\isanewline
%
\isadelimproof
\ \ %
\endisadelimproof
%
\isatagproof
\isacommand{unfolding}\isamarkupfalse%
\ GenExt{\isacharunderscore}{\kern0pt}def\ \isacommand{by}\isamarkupfalse%
\ simp%
\endisatagproof
{\isafoldproof}%
%
\isadelimproof
\isanewline
%
\endisadelimproof
\isanewline
\isacommand{lemma}\isamarkupfalse%
\ left{\isacharunderscore}{\kern0pt}in{\isacharunderscore}{\kern0pt}M\ {\isacharcolon}{\kern0pt}\ {\isachardoublequoteopen}tau{\isasymin}M\ {\isasymLongrightarrow}\ {\isasymlangle}a{\isacharcomma}{\kern0pt}b{\isasymrangle}{\isasymin}tau\ {\isasymLongrightarrow}\ a{\isasymin}M{\isachardoublequoteclose}\isanewline
%
\isadelimproof
\ \ %
\endisadelimproof
%
\isatagproof
\isacommand{using}\isamarkupfalse%
\ fst{\isacharunderscore}{\kern0pt}snd{\isacharunderscore}{\kern0pt}closed{\isacharbrackleft}{\kern0pt}of\ {\isachardoublequoteopen}{\isasymlangle}a{\isacharcomma}{\kern0pt}b{\isasymrangle}{\isachardoublequoteclose}{\isacharbrackright}{\kern0pt}\ transitivity\ \isacommand{by}\isamarkupfalse%
\ auto%
\endisatagproof
{\isafoldproof}%
%
\isadelimproof
%
\endisadelimproof
%
\isadelimdocument
%
\endisadelimdocument
%
\isatagdocument
%
\isamarkupsubsection{Kunen 2013, Lemma IV.2.29%
}
\isamarkuptrue%
%
\endisatagdocument
{\isafolddocument}%
%
\isadelimdocument
%
\endisadelimdocument
\isacommand{lemma}\isamarkupfalse%
\ generic{\isacharunderscore}{\kern0pt}inter{\isacharunderscore}{\kern0pt}dense{\isacharunderscore}{\kern0pt}below{\isacharcolon}{\kern0pt}\ \isanewline
\ \ \isakeyword{assumes}\ {\isachardoublequoteopen}D{\isasymin}M{\isachardoublequoteclose}\ {\isachardoublequoteopen}M{\isacharunderscore}{\kern0pt}generic{\isacharparenleft}{\kern0pt}G{\isacharparenright}{\kern0pt}{\isachardoublequoteclose}\ {\isachardoublequoteopen}dense{\isacharunderscore}{\kern0pt}below{\isacharparenleft}{\kern0pt}D{\isacharcomma}{\kern0pt}p{\isacharparenright}{\kern0pt}{\isachardoublequoteclose}\ {\isachardoublequoteopen}p{\isasymin}G{\isachardoublequoteclose}\isanewline
\ \ \isakeyword{shows}\ {\isachardoublequoteopen}D\ {\isasyminter}\ G\ {\isasymnoteq}\ {\isadigit{0}}{\isachardoublequoteclose}\isanewline
%
\isadelimproof
%
\endisadelimproof
%
\isatagproof
\isacommand{proof}\isamarkupfalse%
\ {\isacharminus}{\kern0pt}\isanewline
\ \ \isacommand{let}\isamarkupfalse%
\ {\isacharquery}{\kern0pt}D{\isacharequal}{\kern0pt}{\isachardoublequoteopen}{\isacharbraceleft}{\kern0pt}q{\isasymin}P{\isachardot}{\kern0pt}\ p{\isasymbottom}q\ {\isasymor}\ q{\isasymin}D{\isacharbraceright}{\kern0pt}{\isachardoublequoteclose}\isanewline
\ \ \isacommand{have}\isamarkupfalse%
\ {\isachardoublequoteopen}dense{\isacharparenleft}{\kern0pt}{\isacharquery}{\kern0pt}D{\isacharparenright}{\kern0pt}{\isachardoublequoteclose}\isanewline
\ \ \isacommand{proof}\isamarkupfalse%
\isanewline
\ \ \ \ \isacommand{fix}\isamarkupfalse%
\ r\isanewline
\ \ \ \ \isacommand{assume}\isamarkupfalse%
\ {\isachardoublequoteopen}r{\isasymin}P{\isachardoublequoteclose}\isanewline
\ \ \ \ \isacommand{show}\isamarkupfalse%
\ {\isachardoublequoteopen}{\isasymexists}d{\isasymin}{\isacharbraceleft}{\kern0pt}q\ {\isasymin}\ P\ {\isachardot}{\kern0pt}\ p\ {\isasymbottom}\ q\ {\isasymor}\ q\ {\isasymin}\ D{\isacharbraceright}{\kern0pt}{\isachardot}{\kern0pt}\ d\ {\isasympreceq}\ r{\isachardoublequoteclose}\isanewline
\ \ \ \ \isacommand{proof}\isamarkupfalse%
\ {\isacharparenleft}{\kern0pt}cases\ {\isachardoublequoteopen}p\ {\isasymbottom}\ r{\isachardoublequoteclose}{\isacharparenright}{\kern0pt}\isanewline
\ \ \ \ \ \ \isacommand{case}\isamarkupfalse%
\ True\isanewline
\ \ \ \ \ \ \isacommand{with}\isamarkupfalse%
\ {\isacartoucheopen}r{\isasymin}P{\isacartoucheclose}\isanewline
\ \ \ \ \ \ \ \ \isanewline
\ \ \ \ \ \ \isacommand{show}\isamarkupfalse%
\ {\isacharquery}{\kern0pt}thesis\ \isacommand{using}\isamarkupfalse%
\ leq{\isacharunderscore}{\kern0pt}reflI{\isacharbrackleft}{\kern0pt}of\ r{\isacharbrackright}{\kern0pt}\ \isacommand{by}\isamarkupfalse%
\ {\isacharparenleft}{\kern0pt}intro\ bexI{\isacharparenright}{\kern0pt}\ {\isacharparenleft}{\kern0pt}blast{\isacharplus}{\kern0pt}{\isacharparenright}{\kern0pt}\isanewline
\ \ \ \ \isacommand{next}\isamarkupfalse%
\isanewline
\ \ \ \ \ \ \isacommand{case}\isamarkupfalse%
\ False\isanewline
\ \ \ \ \ \ \isacommand{then}\isamarkupfalse%
\isanewline
\ \ \ \ \ \ \isacommand{obtain}\isamarkupfalse%
\ s\ \isakeyword{where}\ {\isachardoublequoteopen}s{\isasymin}P{\isachardoublequoteclose}\ {\isachardoublequoteopen}s{\isasympreceq}p{\isachardoublequoteclose}\ {\isachardoublequoteopen}s{\isasympreceq}r{\isachardoublequoteclose}\ \isacommand{by}\isamarkupfalse%
\ blast\isanewline
\ \ \ \ \ \ \isacommand{with}\isamarkupfalse%
\ assms\ {\isacartoucheopen}r{\isasymin}P{\isacartoucheclose}\isanewline
\ \ \ \ \ \ \isacommand{show}\isamarkupfalse%
\ {\isacharquery}{\kern0pt}thesis\isanewline
\ \ \ \ \ \ \ \ \isacommand{using}\isamarkupfalse%
\ dense{\isacharunderscore}{\kern0pt}belowD{\isacharbrackleft}{\kern0pt}OF\ assms{\isacharparenleft}{\kern0pt}{\isadigit{3}}{\isacharparenright}{\kern0pt}{\isacharcomma}{\kern0pt}\ of\ s{\isacharbrackright}{\kern0pt}\ leq{\isacharunderscore}{\kern0pt}transD{\isacharbrackleft}{\kern0pt}of\ {\isacharunderscore}{\kern0pt}\ s\ r{\isacharbrackright}{\kern0pt}\isanewline
\ \ \ \ \ \ \ \ \isacommand{by}\isamarkupfalse%
\ blast\isanewline
\ \ \ \ \isacommand{qed}\isamarkupfalse%
\isanewline
\ \ \isacommand{qed}\isamarkupfalse%
\isanewline
\ \ \isacommand{have}\isamarkupfalse%
\ {\isachardoublequoteopen}{\isacharquery}{\kern0pt}D{\isasymsubseteq}P{\isachardoublequoteclose}\ \isacommand{by}\isamarkupfalse%
\ auto\isanewline
\ \ \isanewline
\ \ \isacommand{let}\isamarkupfalse%
\ {\isacharquery}{\kern0pt}d{\isacharunderscore}{\kern0pt}fm{\isacharequal}{\kern0pt}{\isachardoublequoteopen}Or{\isacharparenleft}{\kern0pt}Neg{\isacharparenleft}{\kern0pt}compat{\isacharunderscore}{\kern0pt}in{\isacharunderscore}{\kern0pt}fm{\isacharparenleft}{\kern0pt}{\isadigit{1}}{\isacharcomma}{\kern0pt}{\isadigit{2}}{\isacharcomma}{\kern0pt}{\isadigit{3}}{\isacharcomma}{\kern0pt}{\isadigit{0}}{\isacharparenright}{\kern0pt}{\isacharparenright}{\kern0pt}{\isacharcomma}{\kern0pt}Member{\isacharparenleft}{\kern0pt}{\isadigit{0}}{\isacharcomma}{\kern0pt}{\isadigit{4}}{\isacharparenright}{\kern0pt}{\isacharparenright}{\kern0pt}{\isachardoublequoteclose}\isanewline
\ \ \isacommand{have}\isamarkupfalse%
\ {\isadigit{1}}{\isacharcolon}{\kern0pt}{\isachardoublequoteopen}p{\isasymin}M{\isachardoublequoteclose}\ \isanewline
\ \ \ \ \isacommand{using}\isamarkupfalse%
\ {\isacartoucheopen}M{\isacharunderscore}{\kern0pt}generic{\isacharparenleft}{\kern0pt}G{\isacharparenright}{\kern0pt}{\isacartoucheclose}\ M{\isacharunderscore}{\kern0pt}genericD\ transitivity{\isacharbrackleft}{\kern0pt}OF\ {\isacharunderscore}{\kern0pt}\ P{\isacharunderscore}{\kern0pt}in{\isacharunderscore}{\kern0pt}M{\isacharbrackright}{\kern0pt}\isanewline
\ \ \ \ \ \ \ \ \ \ {\isacartoucheopen}p{\isasymin}G{\isacartoucheclose}\ \isacommand{by}\isamarkupfalse%
\ simp\isanewline
\ \ \isacommand{moreover}\isamarkupfalse%
\isanewline
\ \ \isacommand{have}\isamarkupfalse%
\ {\isachardoublequoteopen}{\isacharquery}{\kern0pt}d{\isacharunderscore}{\kern0pt}fm{\isasymin}formula{\isachardoublequoteclose}\ \isacommand{by}\isamarkupfalse%
\ simp\isanewline
\ \ \isacommand{moreover}\isamarkupfalse%
\isanewline
\ \ \isacommand{have}\isamarkupfalse%
\ {\isachardoublequoteopen}arity{\isacharparenleft}{\kern0pt}{\isacharquery}{\kern0pt}d{\isacharunderscore}{\kern0pt}fm{\isacharparenright}{\kern0pt}\ {\isacharequal}{\kern0pt}\ {\isadigit{5}}{\isachardoublequoteclose}\ \isacommand{unfolding}\isamarkupfalse%
\ compat{\isacharunderscore}{\kern0pt}in{\isacharunderscore}{\kern0pt}fm{\isacharunderscore}{\kern0pt}def\ pair{\isacharunderscore}{\kern0pt}fm{\isacharunderscore}{\kern0pt}def\ upair{\isacharunderscore}{\kern0pt}fm{\isacharunderscore}{\kern0pt}def\isanewline
\ \ \ \ \isacommand{by}\isamarkupfalse%
\ {\isacharparenleft}{\kern0pt}simp\ add{\isacharcolon}{\kern0pt}\ nat{\isacharunderscore}{\kern0pt}union{\isacharunderscore}{\kern0pt}abs{\isadigit{1}}\ Un{\isacharunderscore}{\kern0pt}commute{\isacharparenright}{\kern0pt}\isanewline
\ \ \isacommand{moreover}\isamarkupfalse%
\isanewline
\ \ \isacommand{have}\isamarkupfalse%
\ {\isachardoublequoteopen}{\isacharparenleft}{\kern0pt}M{\isacharcomma}{\kern0pt}\ {\isacharbrackleft}{\kern0pt}q{\isacharcomma}{\kern0pt}P{\isacharcomma}{\kern0pt}leq{\isacharcomma}{\kern0pt}p{\isacharcomma}{\kern0pt}D{\isacharbrackright}{\kern0pt}\ {\isasymTurnstile}\ {\isacharquery}{\kern0pt}d{\isacharunderscore}{\kern0pt}fm{\isacharparenright}{\kern0pt}\ {\isasymlongleftrightarrow}\ {\isacharparenleft}{\kern0pt}{\isasymnot}\ is{\isacharunderscore}{\kern0pt}compat{\isacharunderscore}{\kern0pt}in{\isacharparenleft}{\kern0pt}{\isacharhash}{\kern0pt}{\isacharhash}{\kern0pt}M{\isacharcomma}{\kern0pt}P{\isacharcomma}{\kern0pt}leq{\isacharcomma}{\kern0pt}p{\isacharcomma}{\kern0pt}q{\isacharparenright}{\kern0pt}\ {\isasymor}\ q{\isasymin}D{\isacharparenright}{\kern0pt}{\isachardoublequoteclose}\isanewline
\ \ \ \ \isakeyword{if}\ {\isachardoublequoteopen}q{\isasymin}M{\isachardoublequoteclose}\ \isakeyword{for}\ q\isanewline
\ \ \ \ \isacommand{using}\isamarkupfalse%
\ that\ sats{\isacharunderscore}{\kern0pt}compat{\isacharunderscore}{\kern0pt}in{\isacharunderscore}{\kern0pt}fm\ P{\isacharunderscore}{\kern0pt}in{\isacharunderscore}{\kern0pt}M\ leq{\isacharunderscore}{\kern0pt}in{\isacharunderscore}{\kern0pt}M\ {\isadigit{1}}\ {\isacartoucheopen}D{\isasymin}M{\isacartoucheclose}\ \isacommand{by}\isamarkupfalse%
\ simp\isanewline
\ \ \isacommand{moreover}\isamarkupfalse%
\isanewline
\ \ \isacommand{have}\isamarkupfalse%
\ {\isachardoublequoteopen}{\isacharparenleft}{\kern0pt}{\isasymnot}\ is{\isacharunderscore}{\kern0pt}compat{\isacharunderscore}{\kern0pt}in{\isacharparenleft}{\kern0pt}{\isacharhash}{\kern0pt}{\isacharhash}{\kern0pt}M{\isacharcomma}{\kern0pt}P{\isacharcomma}{\kern0pt}leq{\isacharcomma}{\kern0pt}p{\isacharcomma}{\kern0pt}q{\isacharparenright}{\kern0pt}\ {\isasymor}\ q{\isasymin}D{\isacharparenright}{\kern0pt}\ {\isasymlongleftrightarrow}\ p{\isasymbottom}q\ {\isasymor}\ q{\isasymin}D{\isachardoublequoteclose}\ \isakeyword{if}\ {\isachardoublequoteopen}q{\isasymin}M{\isachardoublequoteclose}\ \isakeyword{for}\ q\isanewline
\ \ \ \ \isacommand{unfolding}\isamarkupfalse%
\ compat{\isacharunderscore}{\kern0pt}def\ \isacommand{using}\isamarkupfalse%
\ that\ compat{\isacharunderscore}{\kern0pt}in{\isacharunderscore}{\kern0pt}abs\ P{\isacharunderscore}{\kern0pt}in{\isacharunderscore}{\kern0pt}M\ leq{\isacharunderscore}{\kern0pt}in{\isacharunderscore}{\kern0pt}M\ {\isadigit{1}}\ \isacommand{by}\isamarkupfalse%
\ simp\isanewline
\ \ \isacommand{ultimately}\isamarkupfalse%
\isanewline
\ \ \isacommand{have}\isamarkupfalse%
\ {\isachardoublequoteopen}{\isacharquery}{\kern0pt}D{\isasymin}M{\isachardoublequoteclose}\ \isacommand{using}\isamarkupfalse%
\ Collect{\isacharunderscore}{\kern0pt}in{\isacharunderscore}{\kern0pt}M{\isacharunderscore}{\kern0pt}{\isadigit{4}}p{\isacharbrackleft}{\kern0pt}of\ {\isacharquery}{\kern0pt}d{\isacharunderscore}{\kern0pt}fm\ {\isacharunderscore}{\kern0pt}\ {\isacharunderscore}{\kern0pt}\ {\isacharunderscore}{\kern0pt}\ {\isacharunderscore}{\kern0pt}\ {\isacharunderscore}{\kern0pt}{\isachardoublequoteopen}{\isasymlambda}x\ y\ z\ w\ h{\isachardot}{\kern0pt}\ w{\isasymbottom}x\ {\isasymor}\ x{\isasymin}h{\isachardoublequoteclose}{\isacharbrackright}{\kern0pt}\ \isanewline
\ \ \ \ \ \ \ \ \ \ \ \ \ \ \ \ \ \ \ \ P{\isacharunderscore}{\kern0pt}in{\isacharunderscore}{\kern0pt}M\ leq{\isacharunderscore}{\kern0pt}in{\isacharunderscore}{\kern0pt}M\ {\isacartoucheopen}D{\isasymin}M{\isacartoucheclose}\ \isacommand{by}\isamarkupfalse%
\ simp\isanewline
\ \ \isacommand{note}\isamarkupfalse%
\ asm\ {\isacharequal}{\kern0pt}\ {\isacartoucheopen}M{\isacharunderscore}{\kern0pt}generic{\isacharparenleft}{\kern0pt}G{\isacharparenright}{\kern0pt}{\isacartoucheclose}\ {\isacartoucheopen}dense{\isacharparenleft}{\kern0pt}{\isacharquery}{\kern0pt}D{\isacharparenright}{\kern0pt}{\isacartoucheclose}\ {\isacartoucheopen}{\isacharquery}{\kern0pt}D{\isasymsubseteq}P{\isacartoucheclose}\ {\isacartoucheopen}{\isacharquery}{\kern0pt}D{\isasymin}M{\isacartoucheclose}\isanewline
\ \ \isacommand{obtain}\isamarkupfalse%
\ x\ \isakeyword{where}\ {\isachardoublequoteopen}x{\isasymin}G{\isachardoublequoteclose}\ {\isachardoublequoteopen}x{\isasymin}{\isacharquery}{\kern0pt}D{\isachardoublequoteclose}\ \isacommand{using}\isamarkupfalse%
\ M{\isacharunderscore}{\kern0pt}generic{\isacharunderscore}{\kern0pt}denseD{\isacharbrackleft}{\kern0pt}OF\ asm{\isacharbrackright}{\kern0pt}\isanewline
\ \ \ \ \isacommand{by}\isamarkupfalse%
\ force\ \isanewline
\ \ \isacommand{moreover}\isamarkupfalse%
\ \isacommand{from}\isamarkupfalse%
\ this\ \isakeyword{and}\ {\isacartoucheopen}M{\isacharunderscore}{\kern0pt}generic{\isacharparenleft}{\kern0pt}G{\isacharparenright}{\kern0pt}{\isacartoucheclose}\isanewline
\ \ \isacommand{have}\isamarkupfalse%
\ {\isachardoublequoteopen}x{\isasymin}D{\isachardoublequoteclose}\isanewline
\ \ \ \ \isacommand{using}\isamarkupfalse%
\ M{\isacharunderscore}{\kern0pt}generic{\isacharunderscore}{\kern0pt}compatD{\isacharbrackleft}{\kern0pt}OF\ {\isacharunderscore}{\kern0pt}\ {\isacartoucheopen}p{\isasymin}G{\isacartoucheclose}{\isacharcomma}{\kern0pt}\ of\ x{\isacharbrackright}{\kern0pt}\isanewline
\ \ \ \ \ \ leq{\isacharunderscore}{\kern0pt}reflI\ compatI{\isacharbrackleft}{\kern0pt}of\ {\isacharunderscore}{\kern0pt}\ p\ x{\isacharbrackright}{\kern0pt}\ \isacommand{by}\isamarkupfalse%
\ force\isanewline
\ \ \isacommand{ultimately}\isamarkupfalse%
\isanewline
\ \ \isacommand{show}\isamarkupfalse%
\ {\isacharquery}{\kern0pt}thesis\ \isacommand{by}\isamarkupfalse%
\ auto\isanewline
\isacommand{qed}\isamarkupfalse%
%
\endisatagproof
{\isafoldproof}%
%
\isadelimproof
%
\endisadelimproof
%
\isadelimdocument
%
\endisadelimdocument
%
\isatagdocument
%
\isamarkupsubsection{Auxiliary results for Lemma IV.2.40(a)%
}
\isamarkuptrue%
%
\endisatagdocument
{\isafolddocument}%
%
\isadelimdocument
%
\endisadelimdocument
\isacommand{lemma}\isamarkupfalse%
\ IV{\isadigit{2}}{\isadigit{4}}{\isadigit{0}}a{\isacharunderscore}{\kern0pt}mem{\isacharunderscore}{\kern0pt}Collect{\isacharcolon}{\kern0pt}\isanewline
\ \ \isakeyword{assumes}\isanewline
\ \ \ \ {\isachardoublequoteopen}{\isasympi}{\isasymin}M{\isachardoublequoteclose}\ {\isachardoublequoteopen}{\isasymtau}{\isasymin}M{\isachardoublequoteclose}\isanewline
\ \ \isakeyword{shows}\isanewline
\ \ \ \ {\isachardoublequoteopen}{\isacharbraceleft}{\kern0pt}q{\isasymin}P{\isachardot}{\kern0pt}\ {\isasymexists}{\isasymsigma}{\isachardot}{\kern0pt}\ {\isasymexists}r{\isachardot}{\kern0pt}\ r{\isasymin}P\ {\isasymand}\ {\isasymlangle}{\isasymsigma}{\isacharcomma}{\kern0pt}r{\isasymrangle}\ {\isasymin}\ {\isasymtau}\ {\isasymand}\ q{\isasympreceq}r\ {\isasymand}\ forces{\isacharunderscore}{\kern0pt}eq{\isacharparenleft}{\kern0pt}q{\isacharcomma}{\kern0pt}{\isasympi}{\isacharcomma}{\kern0pt}{\isasymsigma}{\isacharparenright}{\kern0pt}{\isacharbraceright}{\kern0pt}{\isasymin}M{\isachardoublequoteclose}\isanewline
%
\isadelimproof
%
\endisadelimproof
%
\isatagproof
\isacommand{proof}\isamarkupfalse%
\ {\isacharminus}{\kern0pt}\isanewline
\ \ \isacommand{let}\isamarkupfalse%
\ {\isacharquery}{\kern0pt}rel{\isacharunderscore}{\kern0pt}pred{\isacharequal}{\kern0pt}\ {\isachardoublequoteopen}{\isasymlambda}M\ x\ a{\isadigit{1}}\ a{\isadigit{2}}\ a{\isadigit{3}}\ a{\isadigit{4}}{\isachardot}{\kern0pt}\ {\isasymexists}{\isasymsigma}{\isacharbrackleft}{\kern0pt}M{\isacharbrackright}{\kern0pt}{\isachardot}{\kern0pt}\ {\isasymexists}r{\isacharbrackleft}{\kern0pt}M{\isacharbrackright}{\kern0pt}{\isachardot}{\kern0pt}\ {\isasymexists}{\isasymsigma}r{\isacharbrackleft}{\kern0pt}M{\isacharbrackright}{\kern0pt}{\isachardot}{\kern0pt}\ \isanewline
\ \ \ \ \ \ \ \ \ \ \ \ \ \ \ \ r{\isasymin}a{\isadigit{1}}\ {\isasymand}\ pair{\isacharparenleft}{\kern0pt}M{\isacharcomma}{\kern0pt}{\isasymsigma}{\isacharcomma}{\kern0pt}r{\isacharcomma}{\kern0pt}{\isasymsigma}r{\isacharparenright}{\kern0pt}\ {\isasymand}\ {\isasymsigma}r{\isasymin}a{\isadigit{4}}\ {\isasymand}\ is{\isacharunderscore}{\kern0pt}leq{\isacharparenleft}{\kern0pt}M{\isacharcomma}{\kern0pt}a{\isadigit{2}}{\isacharcomma}{\kern0pt}x{\isacharcomma}{\kern0pt}r{\isacharparenright}{\kern0pt}\ {\isasymand}\ is{\isacharunderscore}{\kern0pt}forces{\isacharunderscore}{\kern0pt}eq{\isacharprime}{\kern0pt}{\isacharparenleft}{\kern0pt}M{\isacharcomma}{\kern0pt}a{\isadigit{1}}{\isacharcomma}{\kern0pt}a{\isadigit{2}}{\isacharcomma}{\kern0pt}x{\isacharcomma}{\kern0pt}a{\isadigit{3}}{\isacharcomma}{\kern0pt}{\isasymsigma}{\isacharparenright}{\kern0pt}{\isachardoublequoteclose}\isanewline
\ \ \isacommand{let}\isamarkupfalse%
\ {\isacharquery}{\kern0pt}{\isasymphi}{\isacharequal}{\kern0pt}{\isachardoublequoteopen}Exists{\isacharparenleft}{\kern0pt}Exists{\isacharparenleft}{\kern0pt}Exists{\isacharparenleft}{\kern0pt}And{\isacharparenleft}{\kern0pt}Member{\isacharparenleft}{\kern0pt}{\isadigit{1}}{\isacharcomma}{\kern0pt}{\isadigit{4}}{\isacharparenright}{\kern0pt}{\isacharcomma}{\kern0pt}And{\isacharparenleft}{\kern0pt}pair{\isacharunderscore}{\kern0pt}fm{\isacharparenleft}{\kern0pt}{\isadigit{2}}{\isacharcomma}{\kern0pt}{\isadigit{1}}{\isacharcomma}{\kern0pt}{\isadigit{0}}{\isacharparenright}{\kern0pt}{\isacharcomma}{\kern0pt}\isanewline
\ \ \ \ \ \ \ \ \ \ And{\isacharparenleft}{\kern0pt}Member{\isacharparenleft}{\kern0pt}{\isadigit{0}}{\isacharcomma}{\kern0pt}{\isadigit{7}}{\isacharparenright}{\kern0pt}{\isacharcomma}{\kern0pt}And{\isacharparenleft}{\kern0pt}leq{\isacharunderscore}{\kern0pt}fm{\isacharparenleft}{\kern0pt}{\isadigit{5}}{\isacharcomma}{\kern0pt}{\isadigit{3}}{\isacharcomma}{\kern0pt}{\isadigit{1}}{\isacharparenright}{\kern0pt}{\isacharcomma}{\kern0pt}forces{\isacharunderscore}{\kern0pt}eq{\isacharunderscore}{\kern0pt}fm{\isacharparenleft}{\kern0pt}{\isadigit{4}}{\isacharcomma}{\kern0pt}{\isadigit{5}}{\isacharcomma}{\kern0pt}{\isadigit{3}}{\isacharcomma}{\kern0pt}{\isadigit{6}}{\isacharcomma}{\kern0pt}{\isadigit{2}}{\isacharparenright}{\kern0pt}{\isacharparenright}{\kern0pt}{\isacharparenright}{\kern0pt}{\isacharparenright}{\kern0pt}{\isacharparenright}{\kern0pt}{\isacharparenright}{\kern0pt}{\isacharparenright}{\kern0pt}{\isacharparenright}{\kern0pt}{\isachardoublequoteclose}\ \isanewline
\ \ \isacommand{have}\isamarkupfalse%
\ {\isachardoublequoteopen}{\isasymsigma}{\isasymin}M\ {\isasymand}\ r{\isasymin}M{\isachardoublequoteclose}\ \isakeyword{if}\ {\isachardoublequoteopen}{\isasymlangle}{\isasymsigma}{\isacharcomma}{\kern0pt}\ r{\isasymrangle}\ {\isasymin}\ {\isasymtau}{\isachardoublequoteclose}\ \ \isakeyword{for}\ {\isasymsigma}\ r\isanewline
\ \ \ \ \isacommand{using}\isamarkupfalse%
\ that\ {\isacartoucheopen}{\isasymtau}{\isasymin}M{\isacartoucheclose}\ pair{\isacharunderscore}{\kern0pt}in{\isacharunderscore}{\kern0pt}M{\isacharunderscore}{\kern0pt}iff\ transitivity{\isacharbrackleft}{\kern0pt}of\ {\isachardoublequoteopen}{\isasymlangle}{\isasymsigma}{\isacharcomma}{\kern0pt}r{\isasymrangle}{\isachardoublequoteclose}\ {\isasymtau}{\isacharbrackright}{\kern0pt}\ \isacommand{by}\isamarkupfalse%
\ simp\isanewline
\ \ \isacommand{then}\isamarkupfalse%
\isanewline
\ \ \isacommand{have}\isamarkupfalse%
\ {\isachardoublequoteopen}{\isacharquery}{\kern0pt}rel{\isacharunderscore}{\kern0pt}pred{\isacharparenleft}{\kern0pt}{\isacharhash}{\kern0pt}{\isacharhash}{\kern0pt}M{\isacharcomma}{\kern0pt}q{\isacharcomma}{\kern0pt}P{\isacharcomma}{\kern0pt}leq{\isacharcomma}{\kern0pt}{\isasympi}{\isacharcomma}{\kern0pt}{\isasymtau}{\isacharparenright}{\kern0pt}\ {\isasymlongleftrightarrow}\ {\isacharparenleft}{\kern0pt}{\isasymexists}{\isasymsigma}{\isachardot}{\kern0pt}\ {\isasymexists}r{\isachardot}{\kern0pt}\ r{\isasymin}P\ {\isasymand}\ {\isasymlangle}{\isasymsigma}{\isacharcomma}{\kern0pt}r{\isasymrangle}\ {\isasymin}\ {\isasymtau}\ {\isasymand}\ q{\isasympreceq}r\ {\isasymand}\ forces{\isacharunderscore}{\kern0pt}eq{\isacharparenleft}{\kern0pt}q{\isacharcomma}{\kern0pt}{\isasympi}{\isacharcomma}{\kern0pt}{\isasymsigma}{\isacharparenright}{\kern0pt}{\isacharparenright}{\kern0pt}{\isachardoublequoteclose}\isanewline
\ \ \ \ \isakeyword{if}\ {\isachardoublequoteopen}q{\isasymin}M{\isachardoublequoteclose}\ \isakeyword{for}\ q\isanewline
\ \ \ \ \isacommand{unfolding}\isamarkupfalse%
\ forces{\isacharunderscore}{\kern0pt}eq{\isacharunderscore}{\kern0pt}def\ \isacommand{using}\isamarkupfalse%
\ assms\ that\ P{\isacharunderscore}{\kern0pt}in{\isacharunderscore}{\kern0pt}M\ leq{\isacharunderscore}{\kern0pt}in{\isacharunderscore}{\kern0pt}M\ leq{\isacharunderscore}{\kern0pt}abs\ forces{\isacharunderscore}{\kern0pt}eq{\isacharprime}{\kern0pt}{\isacharunderscore}{\kern0pt}abs\ pair{\isacharunderscore}{\kern0pt}in{\isacharunderscore}{\kern0pt}M{\isacharunderscore}{\kern0pt}iff\ \isanewline
\ \ \ \ \isacommand{by}\isamarkupfalse%
\ auto\isanewline
\ \ \isacommand{moreover}\isamarkupfalse%
\isanewline
\ \ \isacommand{have}\isamarkupfalse%
\ {\isachardoublequoteopen}{\isacharparenleft}{\kern0pt}M{\isacharcomma}{\kern0pt}\ {\isacharbrackleft}{\kern0pt}q{\isacharcomma}{\kern0pt}P{\isacharcomma}{\kern0pt}leq{\isacharcomma}{\kern0pt}{\isasympi}{\isacharcomma}{\kern0pt}{\isasymtau}{\isacharbrackright}{\kern0pt}\ {\isasymTurnstile}\ {\isacharquery}{\kern0pt}{\isasymphi}{\isacharparenright}{\kern0pt}\ {\isasymlongleftrightarrow}\ {\isacharquery}{\kern0pt}rel{\isacharunderscore}{\kern0pt}pred{\isacharparenleft}{\kern0pt}{\isacharhash}{\kern0pt}{\isacharhash}{\kern0pt}M{\isacharcomma}{\kern0pt}q{\isacharcomma}{\kern0pt}P{\isacharcomma}{\kern0pt}leq{\isacharcomma}{\kern0pt}{\isasympi}{\isacharcomma}{\kern0pt}{\isasymtau}{\isacharparenright}{\kern0pt}{\isachardoublequoteclose}\ \isakeyword{if}\ {\isachardoublequoteopen}q{\isasymin}M{\isachardoublequoteclose}\ \isakeyword{for}\ q\isanewline
\ \ \ \ \isacommand{using}\isamarkupfalse%
\ assms\ that\ sats{\isacharunderscore}{\kern0pt}forces{\isacharunderscore}{\kern0pt}eq{\isacharprime}{\kern0pt}{\isacharunderscore}{\kern0pt}fm\ sats{\isacharunderscore}{\kern0pt}leq{\isacharunderscore}{\kern0pt}fm\ P{\isacharunderscore}{\kern0pt}in{\isacharunderscore}{\kern0pt}M\ leq{\isacharunderscore}{\kern0pt}in{\isacharunderscore}{\kern0pt}M\ \isacommand{by}\isamarkupfalse%
\ simp\isanewline
\ \ \isacommand{moreover}\isamarkupfalse%
\isanewline
\ \ \isacommand{have}\isamarkupfalse%
\ {\isachardoublequoteopen}{\isacharquery}{\kern0pt}{\isasymphi}{\isasymin}formula{\isachardoublequoteclose}\ \isacommand{by}\isamarkupfalse%
\ simp\isanewline
\ \ \isacommand{moreover}\isamarkupfalse%
\isanewline
\ \ \isacommand{have}\isamarkupfalse%
\ {\isachardoublequoteopen}arity{\isacharparenleft}{\kern0pt}{\isacharquery}{\kern0pt}{\isasymphi}{\isacharparenright}{\kern0pt}{\isacharequal}{\kern0pt}{\isadigit{5}}{\isachardoublequoteclose}\ \isanewline
\ \ \ \ \isacommand{unfolding}\isamarkupfalse%
\ leq{\isacharunderscore}{\kern0pt}fm{\isacharunderscore}{\kern0pt}def\ pair{\isacharunderscore}{\kern0pt}fm{\isacharunderscore}{\kern0pt}def\ upair{\isacharunderscore}{\kern0pt}fm{\isacharunderscore}{\kern0pt}def\isanewline
\ \ \ \ \isacommand{using}\isamarkupfalse%
\ arity{\isacharunderscore}{\kern0pt}forces{\isacharunderscore}{\kern0pt}eq{\isacharunderscore}{\kern0pt}fm\ \isacommand{by}\isamarkupfalse%
\ {\isacharparenleft}{\kern0pt}simp\ add{\isacharcolon}{\kern0pt}nat{\isacharunderscore}{\kern0pt}simp{\isacharunderscore}{\kern0pt}union\ Un{\isacharunderscore}{\kern0pt}commute{\isacharparenright}{\kern0pt}\isanewline
\ \ \isacommand{ultimately}\isamarkupfalse%
\isanewline
\ \ \isacommand{show}\isamarkupfalse%
\ {\isacharquery}{\kern0pt}thesis\ \isanewline
\ \ \ \ \isacommand{unfolding}\isamarkupfalse%
\ forces{\isacharunderscore}{\kern0pt}eq{\isacharunderscore}{\kern0pt}def\ \isacommand{using}\isamarkupfalse%
\ P{\isacharunderscore}{\kern0pt}in{\isacharunderscore}{\kern0pt}M\ leq{\isacharunderscore}{\kern0pt}in{\isacharunderscore}{\kern0pt}M\ assms\ \isanewline
\ \ \ \ \ \ \ \ Collect{\isacharunderscore}{\kern0pt}in{\isacharunderscore}{\kern0pt}M{\isacharunderscore}{\kern0pt}{\isadigit{4}}p{\isacharbrackleft}{\kern0pt}of\ {\isacharquery}{\kern0pt}{\isasymphi}\ {\isacharunderscore}{\kern0pt}\ {\isacharunderscore}{\kern0pt}\ {\isacharunderscore}{\kern0pt}\ {\isacharunderscore}{\kern0pt}\ {\isacharunderscore}{\kern0pt}\ \isanewline
\ \ \ \ \ \ \ \ \ \ \ \ {\isachardoublequoteopen}{\isasymlambda}q\ a{\isadigit{1}}\ a{\isadigit{2}}\ a{\isadigit{3}}\ a{\isadigit{4}}{\isachardot}{\kern0pt}\ {\isasymexists}{\isasymsigma}{\isachardot}{\kern0pt}\ {\isasymexists}r{\isachardot}{\kern0pt}\ r{\isasymin}a{\isadigit{1}}\ {\isasymand}\ {\isasymlangle}{\isasymsigma}{\isacharcomma}{\kern0pt}r{\isasymrangle}\ {\isasymin}\ {\isasymtau}\ {\isasymand}\ q{\isasympreceq}r\ {\isasymand}\ forces{\isacharunderscore}{\kern0pt}eq{\isacharprime}{\kern0pt}{\isacharparenleft}{\kern0pt}a{\isadigit{1}}{\isacharcomma}{\kern0pt}a{\isadigit{2}}{\isacharcomma}{\kern0pt}q{\isacharcomma}{\kern0pt}a{\isadigit{3}}{\isacharcomma}{\kern0pt}{\isasymsigma}{\isacharparenright}{\kern0pt}{\isachardoublequoteclose}{\isacharbrackright}{\kern0pt}\ \isacommand{by}\isamarkupfalse%
\ simp\isanewline
\isacommand{qed}\isamarkupfalse%
%
\endisatagproof
{\isafoldproof}%
%
\isadelimproof
\isanewline
%
\endisadelimproof
\isanewline
\isanewline
\isacommand{lemma}\isamarkupfalse%
\ IV{\isadigit{2}}{\isadigit{4}}{\isadigit{0}}a{\isacharunderscore}{\kern0pt}mem{\isacharcolon}{\kern0pt}\isanewline
\ \ \isakeyword{assumes}\isanewline
\ \ \ \ {\isachardoublequoteopen}M{\isacharunderscore}{\kern0pt}generic{\isacharparenleft}{\kern0pt}G{\isacharparenright}{\kern0pt}{\isachardoublequoteclose}\ {\isachardoublequoteopen}p{\isasymin}G{\isachardoublequoteclose}\ {\isachardoublequoteopen}{\isasympi}{\isasymin}M{\isachardoublequoteclose}\ {\isachardoublequoteopen}{\isasymtau}{\isasymin}M{\isachardoublequoteclose}\ {\isachardoublequoteopen}forces{\isacharunderscore}{\kern0pt}mem{\isacharparenleft}{\kern0pt}p{\isacharcomma}{\kern0pt}{\isasympi}{\isacharcomma}{\kern0pt}{\isasymtau}{\isacharparenright}{\kern0pt}{\isachardoublequoteclose}\isanewline
\ \ \ \ {\isachardoublequoteopen}{\isasymAnd}q\ {\isasymsigma}{\isachardot}{\kern0pt}\ q{\isasymin}P\ {\isasymLongrightarrow}\ q{\isasymin}G\ {\isasymLongrightarrow}\ {\isasymsigma}{\isasymin}domain{\isacharparenleft}{\kern0pt}{\isasymtau}{\isacharparenright}{\kern0pt}\ {\isasymLongrightarrow}\ forces{\isacharunderscore}{\kern0pt}eq{\isacharparenleft}{\kern0pt}q{\isacharcomma}{\kern0pt}{\isasympi}{\isacharcomma}{\kern0pt}{\isasymsigma}{\isacharparenright}{\kern0pt}\ {\isasymLongrightarrow}\ \isanewline
\ \ \ \ \ \ val{\isacharparenleft}{\kern0pt}G{\isacharcomma}{\kern0pt}{\isasympi}{\isacharparenright}{\kern0pt}\ {\isacharequal}{\kern0pt}\ val{\isacharparenleft}{\kern0pt}G{\isacharcomma}{\kern0pt}{\isasymsigma}{\isacharparenright}{\kern0pt}{\isachardoublequoteclose}\ \isanewline
\ \ \isakeyword{shows}\isanewline
\ \ \ \ {\isachardoublequoteopen}val{\isacharparenleft}{\kern0pt}G{\isacharcomma}{\kern0pt}{\isasympi}{\isacharparenright}{\kern0pt}{\isasymin}val{\isacharparenleft}{\kern0pt}G{\isacharcomma}{\kern0pt}{\isasymtau}{\isacharparenright}{\kern0pt}{\isachardoublequoteclose}\isanewline
%
\isadelimproof
%
\endisadelimproof
%
\isatagproof
\isacommand{proof}\isamarkupfalse%
\ {\isacharparenleft}{\kern0pt}intro\ elem{\isacharunderscore}{\kern0pt}of{\isacharunderscore}{\kern0pt}valI{\isacharparenright}{\kern0pt}\isanewline
\ \ \isacommand{let}\isamarkupfalse%
\ {\isacharquery}{\kern0pt}D{\isacharequal}{\kern0pt}{\isachardoublequoteopen}{\isacharbraceleft}{\kern0pt}q{\isasymin}P{\isachardot}{\kern0pt}\ {\isasymexists}{\isasymsigma}{\isachardot}{\kern0pt}\ {\isasymexists}r{\isachardot}{\kern0pt}\ r{\isasymin}P\ {\isasymand}\ {\isasymlangle}{\isasymsigma}{\isacharcomma}{\kern0pt}r{\isasymrangle}\ {\isasymin}\ {\isasymtau}\ {\isasymand}\ q{\isasympreceq}r\ {\isasymand}\ forces{\isacharunderscore}{\kern0pt}eq{\isacharparenleft}{\kern0pt}q{\isacharcomma}{\kern0pt}{\isasympi}{\isacharcomma}{\kern0pt}{\isasymsigma}{\isacharparenright}{\kern0pt}{\isacharbraceright}{\kern0pt}{\isachardoublequoteclose}\isanewline
\ \ \isacommand{from}\isamarkupfalse%
\ {\isacartoucheopen}M{\isacharunderscore}{\kern0pt}generic{\isacharparenleft}{\kern0pt}G{\isacharparenright}{\kern0pt}{\isacartoucheclose}\ {\isacartoucheopen}p{\isasymin}G{\isacartoucheclose}\isanewline
\ \ \isacommand{have}\isamarkupfalse%
\ {\isachardoublequoteopen}p{\isasymin}P{\isachardoublequoteclose}\ \isacommand{by}\isamarkupfalse%
\ blast\isanewline
\ \ \isacommand{moreover}\isamarkupfalse%
\isanewline
\ \ \isacommand{note}\isamarkupfalse%
\ {\isacartoucheopen}{\isasympi}{\isasymin}M{\isacartoucheclose}\ {\isacartoucheopen}{\isasymtau}{\isasymin}M{\isacartoucheclose}\isanewline
\ \ \isacommand{ultimately}\isamarkupfalse%
\isanewline
\ \ \isacommand{have}\isamarkupfalse%
\ {\isachardoublequoteopen}{\isacharquery}{\kern0pt}D\ {\isasymin}\ M{\isachardoublequoteclose}\ \isacommand{using}\isamarkupfalse%
\ IV{\isadigit{2}}{\isadigit{4}}{\isadigit{0}}a{\isacharunderscore}{\kern0pt}mem{\isacharunderscore}{\kern0pt}Collect\ \isacommand{by}\isamarkupfalse%
\ simp\isanewline
\ \ \isacommand{moreover}\isamarkupfalse%
\ \isacommand{from}\isamarkupfalse%
\ assms\ {\isacartoucheopen}p{\isasymin}P{\isacartoucheclose}\isanewline
\ \ \isacommand{have}\isamarkupfalse%
\ {\isachardoublequoteopen}dense{\isacharunderscore}{\kern0pt}below{\isacharparenleft}{\kern0pt}{\isacharquery}{\kern0pt}D{\isacharcomma}{\kern0pt}p{\isacharparenright}{\kern0pt}{\isachardoublequoteclose}\isanewline
\ \ \ \ \isacommand{using}\isamarkupfalse%
\ forces{\isacharunderscore}{\kern0pt}mem{\isacharunderscore}{\kern0pt}iff{\isacharunderscore}{\kern0pt}dense{\isacharunderscore}{\kern0pt}below\ \isacommand{by}\isamarkupfalse%
\ simp\isanewline
\ \ \isacommand{moreover}\isamarkupfalse%
\isanewline
\ \ \isacommand{note}\isamarkupfalse%
\ {\isacartoucheopen}M{\isacharunderscore}{\kern0pt}generic{\isacharparenleft}{\kern0pt}G{\isacharparenright}{\kern0pt}{\isacartoucheclose}\ {\isacartoucheopen}p{\isasymin}G{\isacartoucheclose}\isanewline
\ \ \isacommand{ultimately}\isamarkupfalse%
\isanewline
\ \ \isacommand{obtain}\isamarkupfalse%
\ q\ \isakeyword{where}\ {\isachardoublequoteopen}q{\isasymin}G{\isachardoublequoteclose}\ {\isachardoublequoteopen}q{\isasymin}{\isacharquery}{\kern0pt}D{\isachardoublequoteclose}\ \isacommand{using}\isamarkupfalse%
\ generic{\isacharunderscore}{\kern0pt}inter{\isacharunderscore}{\kern0pt}dense{\isacharunderscore}{\kern0pt}below\ \isacommand{by}\isamarkupfalse%
\ blast\isanewline
\ \ \isacommand{then}\isamarkupfalse%
\isanewline
\ \ \isacommand{obtain}\isamarkupfalse%
\ {\isasymsigma}\ r\ \isakeyword{where}\ {\isachardoublequoteopen}r{\isasymin}P{\isachardoublequoteclose}\ {\isachardoublequoteopen}{\isasymlangle}{\isasymsigma}{\isacharcomma}{\kern0pt}r{\isasymrangle}\ {\isasymin}\ {\isasymtau}{\isachardoublequoteclose}\ {\isachardoublequoteopen}q{\isasympreceq}r{\isachardoublequoteclose}\ {\isachardoublequoteopen}forces{\isacharunderscore}{\kern0pt}eq{\isacharparenleft}{\kern0pt}q{\isacharcomma}{\kern0pt}{\isasympi}{\isacharcomma}{\kern0pt}{\isasymsigma}{\isacharparenright}{\kern0pt}{\isachardoublequoteclose}\ \isacommand{by}\isamarkupfalse%
\ blast\isanewline
\ \ \isacommand{moreover}\isamarkupfalse%
\ \isacommand{from}\isamarkupfalse%
\ this\ \isakeyword{and}\ {\isacartoucheopen}q{\isasymin}G{\isacartoucheclose}\ assms\isanewline
\ \ \isacommand{have}\isamarkupfalse%
\ {\isachardoublequoteopen}r\ {\isasymin}\ G{\isachardoublequoteclose}\ {\isachardoublequoteopen}val{\isacharparenleft}{\kern0pt}G{\isacharcomma}{\kern0pt}{\isasympi}{\isacharparenright}{\kern0pt}\ {\isacharequal}{\kern0pt}\ val{\isacharparenleft}{\kern0pt}G{\isacharcomma}{\kern0pt}{\isasymsigma}{\isacharparenright}{\kern0pt}{\isachardoublequoteclose}\ \isacommand{by}\isamarkupfalse%
\ blast{\isacharplus}{\kern0pt}\isanewline
\ \ \isacommand{ultimately}\isamarkupfalse%
\isanewline
\ \ \isacommand{show}\isamarkupfalse%
\ {\isachardoublequoteopen}{\isasymexists}\ {\isasymsigma}{\isachardot}{\kern0pt}\ {\isasymexists}p{\isasymin}P{\isachardot}{\kern0pt}\ p\ {\isasymin}\ G\ {\isasymand}\ {\isasymlangle}{\isasymsigma}{\isacharcomma}{\kern0pt}\ p{\isasymrangle}\ {\isasymin}\ {\isasymtau}\ {\isasymand}\ val{\isacharparenleft}{\kern0pt}G{\isacharcomma}{\kern0pt}\ {\isasymsigma}{\isacharparenright}{\kern0pt}\ {\isacharequal}{\kern0pt}\ val{\isacharparenleft}{\kern0pt}G{\isacharcomma}{\kern0pt}\ {\isasympi}{\isacharparenright}{\kern0pt}{\isachardoublequoteclose}\ \isacommand{by}\isamarkupfalse%
\ auto\isanewline
\isacommand{qed}\isamarkupfalse%
%
\endisatagproof
{\isafoldproof}%
%
\isadelimproof
\isanewline
%
\endisadelimproof
\isanewline
\isanewline
\isacommand{lemma}\isamarkupfalse%
\ refl{\isacharunderscore}{\kern0pt}forces{\isacharunderscore}{\kern0pt}eq{\isacharcolon}{\kern0pt}{\isachardoublequoteopen}p{\isasymin}P\ {\isasymLongrightarrow}\ forces{\isacharunderscore}{\kern0pt}eq{\isacharparenleft}{\kern0pt}p{\isacharcomma}{\kern0pt}x{\isacharcomma}{\kern0pt}x{\isacharparenright}{\kern0pt}{\isachardoublequoteclose}\isanewline
%
\isadelimproof
\ \ %
\endisadelimproof
%
\isatagproof
\isacommand{using}\isamarkupfalse%
\ def{\isacharunderscore}{\kern0pt}forces{\isacharunderscore}{\kern0pt}eq\ \isacommand{by}\isamarkupfalse%
\ simp%
\endisatagproof
{\isafoldproof}%
%
\isadelimproof
\isanewline
%
\endisadelimproof
\isanewline
\isacommand{lemma}\isamarkupfalse%
\ forces{\isacharunderscore}{\kern0pt}memI{\isacharcolon}{\kern0pt}\ {\isachardoublequoteopen}{\isasymlangle}{\isasymsigma}{\isacharcomma}{\kern0pt}r{\isasymrangle}{\isasymin}{\isasymtau}\ {\isasymLongrightarrow}\ p{\isasymin}P\ {\isasymLongrightarrow}\ r{\isasymin}P\ {\isasymLongrightarrow}\ p{\isasympreceq}r\ {\isasymLongrightarrow}\ forces{\isacharunderscore}{\kern0pt}mem{\isacharparenleft}{\kern0pt}p{\isacharcomma}{\kern0pt}{\isasymsigma}{\isacharcomma}{\kern0pt}{\isasymtau}{\isacharparenright}{\kern0pt}{\isachardoublequoteclose}\isanewline
%
\isadelimproof
\ \ %
\endisadelimproof
%
\isatagproof
\isacommand{using}\isamarkupfalse%
\ refl{\isacharunderscore}{\kern0pt}forces{\isacharunderscore}{\kern0pt}eq{\isacharbrackleft}{\kern0pt}of\ {\isacharunderscore}{\kern0pt}\ {\isasymsigma}{\isacharbrackright}{\kern0pt}\ leq{\isacharunderscore}{\kern0pt}transD\ leq{\isacharunderscore}{\kern0pt}reflI\ \isanewline
\ \ \isacommand{by}\isamarkupfalse%
\ {\isacharparenleft}{\kern0pt}blast\ intro{\isacharcolon}{\kern0pt}forces{\isacharunderscore}{\kern0pt}mem{\isacharunderscore}{\kern0pt}iff{\isacharunderscore}{\kern0pt}dense{\isacharunderscore}{\kern0pt}below{\isacharbrackleft}{\kern0pt}THEN\ iffD{\isadigit{2}}{\isacharbrackright}{\kern0pt}{\isacharparenright}{\kern0pt}%
\endisatagproof
{\isafoldproof}%
%
\isadelimproof
\isanewline
%
\endisadelimproof
\isanewline
\isanewline
\isacommand{lemma}\isamarkupfalse%
\ IV{\isadigit{2}}{\isadigit{4}}{\isadigit{0}}a{\isacharunderscore}{\kern0pt}eq{\isacharunderscore}{\kern0pt}{\isadigit{1}}st{\isacharunderscore}{\kern0pt}incl{\isacharcolon}{\kern0pt}\isanewline
\ \ \isakeyword{assumes}\isanewline
\ \ \ \ {\isachardoublequoteopen}M{\isacharunderscore}{\kern0pt}generic{\isacharparenleft}{\kern0pt}G{\isacharparenright}{\kern0pt}{\isachardoublequoteclose}\ {\isachardoublequoteopen}p{\isasymin}G{\isachardoublequoteclose}\ {\isachardoublequoteopen}forces{\isacharunderscore}{\kern0pt}eq{\isacharparenleft}{\kern0pt}p{\isacharcomma}{\kern0pt}{\isasymtau}{\isacharcomma}{\kern0pt}{\isasymtheta}{\isacharparenright}{\kern0pt}{\isachardoublequoteclose}\isanewline
\ \ \ \ \isakeyword{and}\isanewline
\ \ \ \ IH{\isacharcolon}{\kern0pt}{\isachardoublequoteopen}{\isasymAnd}q\ {\isasymsigma}{\isachardot}{\kern0pt}\ q{\isasymin}P\ {\isasymLongrightarrow}\ q{\isasymin}G\ {\isasymLongrightarrow}\ {\isasymsigma}{\isasymin}domain{\isacharparenleft}{\kern0pt}{\isasymtau}{\isacharparenright}{\kern0pt}\ {\isasymunion}\ domain{\isacharparenleft}{\kern0pt}{\isasymtheta}{\isacharparenright}{\kern0pt}\ {\isasymLongrightarrow}\ \isanewline
\ \ \ \ \ \ \ \ {\isacharparenleft}{\kern0pt}forces{\isacharunderscore}{\kern0pt}mem{\isacharparenleft}{\kern0pt}q{\isacharcomma}{\kern0pt}{\isasymsigma}{\isacharcomma}{\kern0pt}{\isasymtau}{\isacharparenright}{\kern0pt}\ {\isasymlongrightarrow}\ val{\isacharparenleft}{\kern0pt}G{\isacharcomma}{\kern0pt}{\isasymsigma}{\isacharparenright}{\kern0pt}\ {\isasymin}\ val{\isacharparenleft}{\kern0pt}G{\isacharcomma}{\kern0pt}{\isasymtau}{\isacharparenright}{\kern0pt}{\isacharparenright}{\kern0pt}\ {\isasymand}\isanewline
\ \ \ \ \ \ \ \ {\isacharparenleft}{\kern0pt}forces{\isacharunderscore}{\kern0pt}mem{\isacharparenleft}{\kern0pt}q{\isacharcomma}{\kern0pt}{\isasymsigma}{\isacharcomma}{\kern0pt}{\isasymtheta}{\isacharparenright}{\kern0pt}\ {\isasymlongrightarrow}\ val{\isacharparenleft}{\kern0pt}G{\isacharcomma}{\kern0pt}{\isasymsigma}{\isacharparenright}{\kern0pt}\ {\isasymin}\ val{\isacharparenleft}{\kern0pt}G{\isacharcomma}{\kern0pt}{\isasymtheta}{\isacharparenright}{\kern0pt}{\isacharparenright}{\kern0pt}{\isachardoublequoteclose}\isanewline
\isanewline
\isanewline
\ \ \isakeyword{shows}\isanewline
\ \ \ \ {\isachardoublequoteopen}val{\isacharparenleft}{\kern0pt}G{\isacharcomma}{\kern0pt}{\isasymtau}{\isacharparenright}{\kern0pt}\ {\isasymsubseteq}\ val{\isacharparenleft}{\kern0pt}G{\isacharcomma}{\kern0pt}{\isasymtheta}{\isacharparenright}{\kern0pt}{\isachardoublequoteclose}\isanewline
%
\isadelimproof
%
\endisadelimproof
%
\isatagproof
\isacommand{proof}\isamarkupfalse%
\isanewline
\ \ \isacommand{fix}\isamarkupfalse%
\ x\isanewline
\ \ \isacommand{assume}\isamarkupfalse%
\ {\isachardoublequoteopen}x{\isasymin}val{\isacharparenleft}{\kern0pt}G{\isacharcomma}{\kern0pt}{\isasymtau}{\isacharparenright}{\kern0pt}{\isachardoublequoteclose}\isanewline
\ \ \isacommand{then}\isamarkupfalse%
\isanewline
\ \ \isacommand{obtain}\isamarkupfalse%
\ {\isasymsigma}\ r\ \isakeyword{where}\ {\isachardoublequoteopen}{\isasymlangle}{\isasymsigma}{\isacharcomma}{\kern0pt}r{\isasymrangle}{\isasymin}{\isasymtau}{\isachardoublequoteclose}\ {\isachardoublequoteopen}r{\isasymin}G{\isachardoublequoteclose}\ {\isachardoublequoteopen}val{\isacharparenleft}{\kern0pt}G{\isacharcomma}{\kern0pt}{\isasymsigma}{\isacharparenright}{\kern0pt}{\isacharequal}{\kern0pt}x{\isachardoublequoteclose}\ \isacommand{by}\isamarkupfalse%
\ blast\isanewline
\ \ \isacommand{moreover}\isamarkupfalse%
\ \isacommand{from}\isamarkupfalse%
\ this\ \isakeyword{and}\ {\isacartoucheopen}p{\isasymin}G{\isacartoucheclose}\ {\isacartoucheopen}M{\isacharunderscore}{\kern0pt}generic{\isacharparenleft}{\kern0pt}G{\isacharparenright}{\kern0pt}{\isacartoucheclose}\isanewline
\ \ \isacommand{obtain}\isamarkupfalse%
\ q\ \isakeyword{where}\ {\isachardoublequoteopen}q{\isasymin}G{\isachardoublequoteclose}\ {\isachardoublequoteopen}q{\isasympreceq}p{\isachardoublequoteclose}\ {\isachardoublequoteopen}q{\isasympreceq}r{\isachardoublequoteclose}\ \isacommand{by}\isamarkupfalse%
\ force\isanewline
\ \ \isacommand{moreover}\isamarkupfalse%
\ \isacommand{from}\isamarkupfalse%
\ this\ \isakeyword{and}\ {\isacartoucheopen}p{\isasymin}G{\isacartoucheclose}\ {\isacartoucheopen}M{\isacharunderscore}{\kern0pt}generic{\isacharparenleft}{\kern0pt}G{\isacharparenright}{\kern0pt}{\isacartoucheclose}\isanewline
\ \ \isacommand{have}\isamarkupfalse%
\ {\isachardoublequoteopen}q{\isasymin}P{\isachardoublequoteclose}\ {\isachardoublequoteopen}p{\isasymin}P{\isachardoublequoteclose}\ \isacommand{by}\isamarkupfalse%
\ blast{\isacharplus}{\kern0pt}\isanewline
\ \ \isacommand{moreover}\isamarkupfalse%
\ \isacommand{from}\isamarkupfalse%
\ calculation\ \isakeyword{and}\ {\isacartoucheopen}M{\isacharunderscore}{\kern0pt}generic{\isacharparenleft}{\kern0pt}G{\isacharparenright}{\kern0pt}{\isacartoucheclose}\isanewline
\ \ \isacommand{have}\isamarkupfalse%
\ {\isachardoublequoteopen}forces{\isacharunderscore}{\kern0pt}mem{\isacharparenleft}{\kern0pt}q{\isacharcomma}{\kern0pt}{\isasymsigma}{\isacharcomma}{\kern0pt}{\isasymtau}{\isacharparenright}{\kern0pt}{\isachardoublequoteclose}\isanewline
\ \ \ \ \isacommand{using}\isamarkupfalse%
\ forces{\isacharunderscore}{\kern0pt}memI\ \isacommand{by}\isamarkupfalse%
\ blast\isanewline
\ \ \isacommand{moreover}\isamarkupfalse%
\isanewline
\ \ \isacommand{note}\isamarkupfalse%
\ {\isacartoucheopen}forces{\isacharunderscore}{\kern0pt}eq{\isacharparenleft}{\kern0pt}p{\isacharcomma}{\kern0pt}{\isasymtau}{\isacharcomma}{\kern0pt}{\isasymtheta}{\isacharparenright}{\kern0pt}{\isacartoucheclose}\isanewline
\ \ \isacommand{ultimately}\isamarkupfalse%
\isanewline
\ \ \isacommand{have}\isamarkupfalse%
\ {\isachardoublequoteopen}forces{\isacharunderscore}{\kern0pt}mem{\isacharparenleft}{\kern0pt}q{\isacharcomma}{\kern0pt}{\isasymsigma}{\isacharcomma}{\kern0pt}{\isasymtheta}{\isacharparenright}{\kern0pt}{\isachardoublequoteclose}\isanewline
\ \ \ \ \isacommand{using}\isamarkupfalse%
\ def{\isacharunderscore}{\kern0pt}forces{\isacharunderscore}{\kern0pt}eq\ \isacommand{by}\isamarkupfalse%
\ blast\isanewline
\ \ \isacommand{with}\isamarkupfalse%
\ {\isacartoucheopen}q{\isasymin}P{\isacartoucheclose}\ {\isacartoucheopen}q{\isasymin}G{\isacartoucheclose}\ IH{\isacharbrackleft}{\kern0pt}of\ q\ {\isasymsigma}{\isacharbrackright}{\kern0pt}\ {\isacartoucheopen}{\isasymlangle}{\isasymsigma}{\isacharcomma}{\kern0pt}r{\isasymrangle}{\isasymin}{\isasymtau}{\isacartoucheclose}\ {\isacartoucheopen}val{\isacharparenleft}{\kern0pt}G{\isacharcomma}{\kern0pt}{\isasymsigma}{\isacharparenright}{\kern0pt}\ {\isacharequal}{\kern0pt}\ x{\isacartoucheclose}\isanewline
\ \ \isacommand{show}\isamarkupfalse%
\ {\isachardoublequoteopen}x{\isasymin}val{\isacharparenleft}{\kern0pt}G{\isacharcomma}{\kern0pt}{\isasymtheta}{\isacharparenright}{\kern0pt}{\isachardoublequoteclose}\ \isacommand{by}\isamarkupfalse%
\ {\isacharparenleft}{\kern0pt}blast{\isacharparenright}{\kern0pt}\isanewline
\isacommand{qed}\isamarkupfalse%
%
\endisatagproof
{\isafoldproof}%
%
\isadelimproof
\isanewline
%
\endisadelimproof
\isanewline
\isanewline
\isacommand{lemma}\isamarkupfalse%
\ IV{\isadigit{2}}{\isadigit{4}}{\isadigit{0}}a{\isacharunderscore}{\kern0pt}eq{\isacharunderscore}{\kern0pt}{\isadigit{2}}nd{\isacharunderscore}{\kern0pt}incl{\isacharcolon}{\kern0pt}\isanewline
\ \ \isakeyword{assumes}\isanewline
\ \ \ \ {\isachardoublequoteopen}M{\isacharunderscore}{\kern0pt}generic{\isacharparenleft}{\kern0pt}G{\isacharparenright}{\kern0pt}{\isachardoublequoteclose}\ {\isachardoublequoteopen}p{\isasymin}G{\isachardoublequoteclose}\ {\isachardoublequoteopen}forces{\isacharunderscore}{\kern0pt}eq{\isacharparenleft}{\kern0pt}p{\isacharcomma}{\kern0pt}{\isasymtau}{\isacharcomma}{\kern0pt}{\isasymtheta}{\isacharparenright}{\kern0pt}{\isachardoublequoteclose}\isanewline
\ \ \ \ \isakeyword{and}\isanewline
\ \ \ \ IH{\isacharcolon}{\kern0pt}{\isachardoublequoteopen}{\isasymAnd}q\ {\isasymsigma}{\isachardot}{\kern0pt}\ q{\isasymin}P\ {\isasymLongrightarrow}\ q{\isasymin}G\ {\isasymLongrightarrow}\ {\isasymsigma}{\isasymin}domain{\isacharparenleft}{\kern0pt}{\isasymtau}{\isacharparenright}{\kern0pt}\ {\isasymunion}\ domain{\isacharparenleft}{\kern0pt}{\isasymtheta}{\isacharparenright}{\kern0pt}\ {\isasymLongrightarrow}\ \isanewline
\ \ \ \ \ \ \ \ {\isacharparenleft}{\kern0pt}forces{\isacharunderscore}{\kern0pt}mem{\isacharparenleft}{\kern0pt}q{\isacharcomma}{\kern0pt}{\isasymsigma}{\isacharcomma}{\kern0pt}{\isasymtau}{\isacharparenright}{\kern0pt}\ {\isasymlongrightarrow}\ val{\isacharparenleft}{\kern0pt}G{\isacharcomma}{\kern0pt}{\isasymsigma}{\isacharparenright}{\kern0pt}\ {\isasymin}\ val{\isacharparenleft}{\kern0pt}G{\isacharcomma}{\kern0pt}{\isasymtau}{\isacharparenright}{\kern0pt}{\isacharparenright}{\kern0pt}\ {\isasymand}\isanewline
\ \ \ \ \ \ \ \ {\isacharparenleft}{\kern0pt}forces{\isacharunderscore}{\kern0pt}mem{\isacharparenleft}{\kern0pt}q{\isacharcomma}{\kern0pt}{\isasymsigma}{\isacharcomma}{\kern0pt}{\isasymtheta}{\isacharparenright}{\kern0pt}\ {\isasymlongrightarrow}\ val{\isacharparenleft}{\kern0pt}G{\isacharcomma}{\kern0pt}{\isasymsigma}{\isacharparenright}{\kern0pt}\ {\isasymin}\ val{\isacharparenleft}{\kern0pt}G{\isacharcomma}{\kern0pt}{\isasymtheta}{\isacharparenright}{\kern0pt}{\isacharparenright}{\kern0pt}{\isachardoublequoteclose}\isanewline
\ \ \isakeyword{shows}\isanewline
\ \ \ \ {\isachardoublequoteopen}val{\isacharparenleft}{\kern0pt}G{\isacharcomma}{\kern0pt}{\isasymtheta}{\isacharparenright}{\kern0pt}\ {\isasymsubseteq}\ val{\isacharparenleft}{\kern0pt}G{\isacharcomma}{\kern0pt}{\isasymtau}{\isacharparenright}{\kern0pt}{\isachardoublequoteclose}\isanewline
%
\isadelimproof
%
\endisadelimproof
%
\isatagproof
\isacommand{proof}\isamarkupfalse%
\isanewline
\ \ \isacommand{fix}\isamarkupfalse%
\ x\isanewline
\ \ \isacommand{assume}\isamarkupfalse%
\ {\isachardoublequoteopen}x{\isasymin}val{\isacharparenleft}{\kern0pt}G{\isacharcomma}{\kern0pt}{\isasymtheta}{\isacharparenright}{\kern0pt}{\isachardoublequoteclose}\isanewline
\ \ \isacommand{then}\isamarkupfalse%
\isanewline
\ \ \isacommand{obtain}\isamarkupfalse%
\ {\isasymsigma}\ r\ \isakeyword{where}\ {\isachardoublequoteopen}{\isasymlangle}{\isasymsigma}{\isacharcomma}{\kern0pt}r{\isasymrangle}{\isasymin}{\isasymtheta}{\isachardoublequoteclose}\ {\isachardoublequoteopen}r{\isasymin}G{\isachardoublequoteclose}\ {\isachardoublequoteopen}val{\isacharparenleft}{\kern0pt}G{\isacharcomma}{\kern0pt}{\isasymsigma}{\isacharparenright}{\kern0pt}{\isacharequal}{\kern0pt}x{\isachardoublequoteclose}\ \isacommand{by}\isamarkupfalse%
\ blast\isanewline
\ \ \isacommand{moreover}\isamarkupfalse%
\ \isacommand{from}\isamarkupfalse%
\ this\ \isakeyword{and}\ {\isacartoucheopen}p{\isasymin}G{\isacartoucheclose}\ {\isacartoucheopen}M{\isacharunderscore}{\kern0pt}generic{\isacharparenleft}{\kern0pt}G{\isacharparenright}{\kern0pt}{\isacartoucheclose}\isanewline
\ \ \isacommand{obtain}\isamarkupfalse%
\ q\ \isakeyword{where}\ {\isachardoublequoteopen}q{\isasymin}G{\isachardoublequoteclose}\ {\isachardoublequoteopen}q{\isasympreceq}p{\isachardoublequoteclose}\ {\isachardoublequoteopen}q{\isasympreceq}r{\isachardoublequoteclose}\ \isacommand{by}\isamarkupfalse%
\ force\isanewline
\ \ \isacommand{moreover}\isamarkupfalse%
\ \isacommand{from}\isamarkupfalse%
\ this\ \isakeyword{and}\ {\isacartoucheopen}p{\isasymin}G{\isacartoucheclose}\ {\isacartoucheopen}M{\isacharunderscore}{\kern0pt}generic{\isacharparenleft}{\kern0pt}G{\isacharparenright}{\kern0pt}{\isacartoucheclose}\isanewline
\ \ \isacommand{have}\isamarkupfalse%
\ {\isachardoublequoteopen}q{\isasymin}P{\isachardoublequoteclose}\ {\isachardoublequoteopen}p{\isasymin}P{\isachardoublequoteclose}\ \isacommand{by}\isamarkupfalse%
\ blast{\isacharplus}{\kern0pt}\isanewline
\ \ \isacommand{moreover}\isamarkupfalse%
\ \isacommand{from}\isamarkupfalse%
\ calculation\ \isakeyword{and}\ {\isacartoucheopen}M{\isacharunderscore}{\kern0pt}generic{\isacharparenleft}{\kern0pt}G{\isacharparenright}{\kern0pt}{\isacartoucheclose}\isanewline
\ \ \isacommand{have}\isamarkupfalse%
\ {\isachardoublequoteopen}forces{\isacharunderscore}{\kern0pt}mem{\isacharparenleft}{\kern0pt}q{\isacharcomma}{\kern0pt}{\isasymsigma}{\isacharcomma}{\kern0pt}{\isasymtheta}{\isacharparenright}{\kern0pt}{\isachardoublequoteclose}\isanewline
\ \ \ \ \isacommand{using}\isamarkupfalse%
\ forces{\isacharunderscore}{\kern0pt}memI\ \isacommand{by}\isamarkupfalse%
\ blast\isanewline
\ \ \isacommand{moreover}\isamarkupfalse%
\isanewline
\ \ \isacommand{note}\isamarkupfalse%
\ {\isacartoucheopen}forces{\isacharunderscore}{\kern0pt}eq{\isacharparenleft}{\kern0pt}p{\isacharcomma}{\kern0pt}{\isasymtau}{\isacharcomma}{\kern0pt}{\isasymtheta}{\isacharparenright}{\kern0pt}{\isacartoucheclose}\isanewline
\ \ \isacommand{ultimately}\isamarkupfalse%
\isanewline
\ \ \isacommand{have}\isamarkupfalse%
\ {\isachardoublequoteopen}forces{\isacharunderscore}{\kern0pt}mem{\isacharparenleft}{\kern0pt}q{\isacharcomma}{\kern0pt}{\isasymsigma}{\isacharcomma}{\kern0pt}{\isasymtau}{\isacharparenright}{\kern0pt}{\isachardoublequoteclose}\isanewline
\ \ \ \ \isacommand{using}\isamarkupfalse%
\ def{\isacharunderscore}{\kern0pt}forces{\isacharunderscore}{\kern0pt}eq\ \isacommand{by}\isamarkupfalse%
\ blast\isanewline
\ \ \isacommand{with}\isamarkupfalse%
\ {\isacartoucheopen}q{\isasymin}P{\isacartoucheclose}\ {\isacartoucheopen}q{\isasymin}G{\isacartoucheclose}\ IH{\isacharbrackleft}{\kern0pt}of\ q\ {\isasymsigma}{\isacharbrackright}{\kern0pt}\ {\isacartoucheopen}{\isasymlangle}{\isasymsigma}{\isacharcomma}{\kern0pt}r{\isasymrangle}{\isasymin}{\isasymtheta}{\isacartoucheclose}\ {\isacartoucheopen}val{\isacharparenleft}{\kern0pt}G{\isacharcomma}{\kern0pt}{\isasymsigma}{\isacharparenright}{\kern0pt}\ {\isacharequal}{\kern0pt}\ x{\isacartoucheclose}\isanewline
\ \ \isacommand{show}\isamarkupfalse%
\ {\isachardoublequoteopen}x{\isasymin}val{\isacharparenleft}{\kern0pt}G{\isacharcomma}{\kern0pt}{\isasymtau}{\isacharparenright}{\kern0pt}{\isachardoublequoteclose}\ \isacommand{by}\isamarkupfalse%
\ {\isacharparenleft}{\kern0pt}blast{\isacharparenright}{\kern0pt}\isanewline
\isacommand{qed}\isamarkupfalse%
%
\endisatagproof
{\isafoldproof}%
%
\isadelimproof
\isanewline
%
\endisadelimproof
\isanewline
\isanewline
\isacommand{lemma}\isamarkupfalse%
\ IV{\isadigit{2}}{\isadigit{4}}{\isadigit{0}}a{\isacharunderscore}{\kern0pt}eq{\isacharcolon}{\kern0pt}\isanewline
\ \ \isakeyword{assumes}\isanewline
\ \ \ \ {\isachardoublequoteopen}M{\isacharunderscore}{\kern0pt}generic{\isacharparenleft}{\kern0pt}G{\isacharparenright}{\kern0pt}{\isachardoublequoteclose}\ {\isachardoublequoteopen}p{\isasymin}G{\isachardoublequoteclose}\ {\isachardoublequoteopen}forces{\isacharunderscore}{\kern0pt}eq{\isacharparenleft}{\kern0pt}p{\isacharcomma}{\kern0pt}{\isasymtau}{\isacharcomma}{\kern0pt}{\isasymtheta}{\isacharparenright}{\kern0pt}{\isachardoublequoteclose}\isanewline
\ \ \ \ \isakeyword{and}\isanewline
\ \ \ \ IH{\isacharcolon}{\kern0pt}{\isachardoublequoteopen}{\isasymAnd}q\ {\isasymsigma}{\isachardot}{\kern0pt}\ q{\isasymin}P\ {\isasymLongrightarrow}\ q{\isasymin}G\ {\isasymLongrightarrow}\ {\isasymsigma}{\isasymin}domain{\isacharparenleft}{\kern0pt}{\isasymtau}{\isacharparenright}{\kern0pt}\ {\isasymunion}\ domain{\isacharparenleft}{\kern0pt}{\isasymtheta}{\isacharparenright}{\kern0pt}\ {\isasymLongrightarrow}\ \isanewline
\ \ \ \ \ \ \ \ {\isacharparenleft}{\kern0pt}forces{\isacharunderscore}{\kern0pt}mem{\isacharparenleft}{\kern0pt}q{\isacharcomma}{\kern0pt}{\isasymsigma}{\isacharcomma}{\kern0pt}{\isasymtau}{\isacharparenright}{\kern0pt}\ {\isasymlongrightarrow}\ val{\isacharparenleft}{\kern0pt}G{\isacharcomma}{\kern0pt}{\isasymsigma}{\isacharparenright}{\kern0pt}\ {\isasymin}\ val{\isacharparenleft}{\kern0pt}G{\isacharcomma}{\kern0pt}{\isasymtau}{\isacharparenright}{\kern0pt}{\isacharparenright}{\kern0pt}\ {\isasymand}\isanewline
\ \ \ \ \ \ \ \ {\isacharparenleft}{\kern0pt}forces{\isacharunderscore}{\kern0pt}mem{\isacharparenleft}{\kern0pt}q{\isacharcomma}{\kern0pt}{\isasymsigma}{\isacharcomma}{\kern0pt}{\isasymtheta}{\isacharparenright}{\kern0pt}\ {\isasymlongrightarrow}\ val{\isacharparenleft}{\kern0pt}G{\isacharcomma}{\kern0pt}{\isasymsigma}{\isacharparenright}{\kern0pt}\ {\isasymin}\ val{\isacharparenleft}{\kern0pt}G{\isacharcomma}{\kern0pt}{\isasymtheta}{\isacharparenright}{\kern0pt}{\isacharparenright}{\kern0pt}{\isachardoublequoteclose}\isanewline
\ \ \isakeyword{shows}\isanewline
\ \ \ \ {\isachardoublequoteopen}val{\isacharparenleft}{\kern0pt}G{\isacharcomma}{\kern0pt}{\isasymtau}{\isacharparenright}{\kern0pt}\ {\isacharequal}{\kern0pt}\ val{\isacharparenleft}{\kern0pt}G{\isacharcomma}{\kern0pt}{\isasymtheta}{\isacharparenright}{\kern0pt}{\isachardoublequoteclose}\isanewline
%
\isadelimproof
\ \ %
\endisadelimproof
%
\isatagproof
\isacommand{using}\isamarkupfalse%
\ IV{\isadigit{2}}{\isadigit{4}}{\isadigit{0}}a{\isacharunderscore}{\kern0pt}eq{\isacharunderscore}{\kern0pt}{\isadigit{1}}st{\isacharunderscore}{\kern0pt}incl{\isacharbrackleft}{\kern0pt}OF\ assms{\isacharbrackright}{\kern0pt}\ IV{\isadigit{2}}{\isadigit{4}}{\isadigit{0}}a{\isacharunderscore}{\kern0pt}eq{\isacharunderscore}{\kern0pt}{\isadigit{2}}nd{\isacharunderscore}{\kern0pt}incl{\isacharbrackleft}{\kern0pt}OF\ assms{\isacharbrackright}{\kern0pt}\ IH\ \isacommand{by}\isamarkupfalse%
\ blast%
\endisatagproof
{\isafoldproof}%
%
\isadelimproof
%
\endisadelimproof
%
\isadelimdocument
%
\endisadelimdocument
%
\isatagdocument
%
\isamarkupsubsection{Induction on names%
}
\isamarkuptrue%
%
\endisatagdocument
{\isafolddocument}%
%
\isadelimdocument
%
\endisadelimdocument
\isacommand{lemma}\isamarkupfalse%
\ core{\isacharunderscore}{\kern0pt}induction{\isacharcolon}{\kern0pt}\isanewline
\ \ \isakeyword{assumes}\isanewline
\ \ \ \ {\isachardoublequoteopen}{\isasymAnd}{\isasymtau}\ {\isasymtheta}\ p{\isachardot}{\kern0pt}\ p\ {\isasymin}\ P\ {\isasymLongrightarrow}\ {\isasymlbrakk}{\isasymAnd}q\ {\isasymsigma}{\isachardot}{\kern0pt}\ {\isasymlbrakk}q{\isasymin}P\ {\isacharsemicolon}{\kern0pt}\ {\isasymsigma}{\isasymin}domain{\isacharparenleft}{\kern0pt}{\isasymtheta}{\isacharparenright}{\kern0pt}{\isasymrbrakk}\ {\isasymLongrightarrow}\ Q{\isacharparenleft}{\kern0pt}{\isadigit{0}}{\isacharcomma}{\kern0pt}{\isasymtau}{\isacharcomma}{\kern0pt}{\isasymsigma}{\isacharcomma}{\kern0pt}q{\isacharparenright}{\kern0pt}{\isasymrbrakk}\ {\isasymLongrightarrow}\ Q{\isacharparenleft}{\kern0pt}{\isadigit{1}}{\isacharcomma}{\kern0pt}{\isasymtau}{\isacharcomma}{\kern0pt}{\isasymtheta}{\isacharcomma}{\kern0pt}p{\isacharparenright}{\kern0pt}{\isachardoublequoteclose}\isanewline
\ \ \ \ {\isachardoublequoteopen}{\isasymAnd}{\isasymtau}\ {\isasymtheta}\ p{\isachardot}{\kern0pt}\ p\ {\isasymin}\ P\ {\isasymLongrightarrow}\ {\isasymlbrakk}{\isasymAnd}q\ {\isasymsigma}{\isachardot}{\kern0pt}\ {\isasymlbrakk}q{\isasymin}P\ {\isacharsemicolon}{\kern0pt}\ {\isasymsigma}{\isasymin}domain{\isacharparenleft}{\kern0pt}{\isasymtau}{\isacharparenright}{\kern0pt}\ {\isasymunion}\ domain{\isacharparenleft}{\kern0pt}{\isasymtheta}{\isacharparenright}{\kern0pt}{\isasymrbrakk}\ {\isasymLongrightarrow}\ Q{\isacharparenleft}{\kern0pt}{\isadigit{1}}{\isacharcomma}{\kern0pt}{\isasymsigma}{\isacharcomma}{\kern0pt}{\isasymtau}{\isacharcomma}{\kern0pt}q{\isacharparenright}{\kern0pt}\ {\isasymand}\ Q{\isacharparenleft}{\kern0pt}{\isadigit{1}}{\isacharcomma}{\kern0pt}{\isasymsigma}{\isacharcomma}{\kern0pt}{\isasymtheta}{\isacharcomma}{\kern0pt}q{\isacharparenright}{\kern0pt}{\isasymrbrakk}\ {\isasymLongrightarrow}\ Q{\isacharparenleft}{\kern0pt}{\isadigit{0}}{\isacharcomma}{\kern0pt}{\isasymtau}{\isacharcomma}{\kern0pt}{\isasymtheta}{\isacharcomma}{\kern0pt}p{\isacharparenright}{\kern0pt}{\isachardoublequoteclose}\isanewline
\ \ \ \ {\isachardoublequoteopen}ft\ {\isasymin}\ {\isadigit{2}}{\isachardoublequoteclose}\ {\isachardoublequoteopen}p\ {\isasymin}\ P{\isachardoublequoteclose}\isanewline
\ \ \isakeyword{shows}\isanewline
\ \ \ \ {\isachardoublequoteopen}Q{\isacharparenleft}{\kern0pt}ft{\isacharcomma}{\kern0pt}{\isasymtau}{\isacharcomma}{\kern0pt}{\isasymtheta}{\isacharcomma}{\kern0pt}p{\isacharparenright}{\kern0pt}{\isachardoublequoteclose}\isanewline
%
\isadelimproof
%
\endisadelimproof
%
\isatagproof
\isacommand{proof}\isamarkupfalse%
\ {\isacharminus}{\kern0pt}\isanewline
\ \ \isacommand{{\isacharbraceleft}{\kern0pt}}\isamarkupfalse%
\isanewline
\ \ \ \ \isacommand{fix}\isamarkupfalse%
\ ft\ p\ {\isasymtau}\ {\isasymtheta}\isanewline
\ \ \ \ \isacommand{have}\isamarkupfalse%
\ {\isachardoublequoteopen}Transset{\isacharparenleft}{\kern0pt}eclose{\isacharparenleft}{\kern0pt}{\isacharbraceleft}{\kern0pt}{\isasymtau}{\isacharcomma}{\kern0pt}{\isasymtheta}{\isacharbraceright}{\kern0pt}{\isacharparenright}{\kern0pt}{\isacharparenright}{\kern0pt}{\isachardoublequoteclose}\ {\isacharparenleft}{\kern0pt}\isakeyword{is}\ {\isachardoublequoteopen}Transset{\isacharparenleft}{\kern0pt}{\isacharquery}{\kern0pt}e{\isacharparenright}{\kern0pt}{\isachardoublequoteclose}{\isacharparenright}{\kern0pt}\ \isanewline
\ \ \ \ \ \ \isacommand{using}\isamarkupfalse%
\ Transset{\isacharunderscore}{\kern0pt}eclose\ \isacommand{by}\isamarkupfalse%
\ simp\isanewline
\ \ \ \ \isacommand{have}\isamarkupfalse%
\ {\isachardoublequoteopen}{\isasymtau}\ {\isasymin}\ {\isacharquery}{\kern0pt}e{\isachardoublequoteclose}\ {\isachardoublequoteopen}{\isasymtheta}\ {\isasymin}\ {\isacharquery}{\kern0pt}e{\isachardoublequoteclose}\ \isanewline
\ \ \ \ \ \ \isacommand{using}\isamarkupfalse%
\ arg{\isacharunderscore}{\kern0pt}into{\isacharunderscore}{\kern0pt}eclose\ \isacommand{by}\isamarkupfalse%
\ simp{\isacharunderscore}{\kern0pt}all\isanewline
\ \ \ \ \isacommand{moreover}\isamarkupfalse%
\isanewline
\ \ \ \ \isacommand{assume}\isamarkupfalse%
\ {\isachardoublequoteopen}ft\ {\isasymin}\ {\isadigit{2}}{\isachardoublequoteclose}\ {\isachardoublequoteopen}p\ {\isasymin}\ P{\isachardoublequoteclose}\isanewline
\ \ \ \ \isacommand{ultimately}\isamarkupfalse%
\isanewline
\ \ \ \ \isacommand{have}\isamarkupfalse%
\ {\isachardoublequoteopen}{\isasymlangle}ft{\isacharcomma}{\kern0pt}{\isasymtau}{\isacharcomma}{\kern0pt}{\isasymtheta}{\isacharcomma}{\kern0pt}p{\isasymrangle}{\isasymin}\ {\isadigit{2}}{\isasymtimes}{\isacharquery}{\kern0pt}e{\isasymtimes}{\isacharquery}{\kern0pt}e{\isasymtimes}P{\isachardoublequoteclose}\ {\isacharparenleft}{\kern0pt}\isakeyword{is}\ {\isachardoublequoteopen}{\isacharquery}{\kern0pt}a{\isasymin}{\isadigit{2}}{\isasymtimes}{\isacharquery}{\kern0pt}e{\isasymtimes}{\isacharquery}{\kern0pt}e{\isasymtimes}P{\isachardoublequoteclose}{\isacharparenright}{\kern0pt}\ \isacommand{by}\isamarkupfalse%
\ simp\isanewline
\ \ \ \ \isacommand{then}\isamarkupfalse%
\ \isanewline
\ \ \ \ \isacommand{have}\isamarkupfalse%
\ {\isachardoublequoteopen}Q{\isacharparenleft}{\kern0pt}ftype{\isacharparenleft}{\kern0pt}{\isacharquery}{\kern0pt}a{\isacharparenright}{\kern0pt}{\isacharcomma}{\kern0pt}\ name{\isadigit{1}}{\isacharparenleft}{\kern0pt}{\isacharquery}{\kern0pt}a{\isacharparenright}{\kern0pt}{\isacharcomma}{\kern0pt}\ name{\isadigit{2}}{\isacharparenleft}{\kern0pt}{\isacharquery}{\kern0pt}a{\isacharparenright}{\kern0pt}{\isacharcomma}{\kern0pt}\ cond{\isacharunderscore}{\kern0pt}of{\isacharparenleft}{\kern0pt}{\isacharquery}{\kern0pt}a{\isacharparenright}{\kern0pt}{\isacharparenright}{\kern0pt}{\isachardoublequoteclose}\isanewline
\ \ \ \ \ \ \isacommand{using}\isamarkupfalse%
\ core{\isacharunderscore}{\kern0pt}induction{\isacharunderscore}{\kern0pt}aux{\isacharbrackleft}{\kern0pt}of\ {\isacharquery}{\kern0pt}e\ P\ Q\ {\isacharquery}{\kern0pt}a{\isacharcomma}{\kern0pt}OF\ {\isacartoucheopen}Transset{\isacharparenleft}{\kern0pt}{\isacharquery}{\kern0pt}e{\isacharparenright}{\kern0pt}{\isacartoucheclose}\ assms{\isacharparenleft}{\kern0pt}{\isadigit{1}}{\isacharcomma}{\kern0pt}{\isadigit{2}}{\isacharparenright}{\kern0pt}\ {\isacartoucheopen}{\isacharquery}{\kern0pt}a{\isasymin}{\isacharunderscore}{\kern0pt}{\isacartoucheclose}{\isacharbrackright}{\kern0pt}\ \isanewline
\ \ \ \ \ \ \isacommand{by}\isamarkupfalse%
\ {\isacharparenleft}{\kern0pt}clarify{\isacharparenright}{\kern0pt}\ {\isacharparenleft}{\kern0pt}blast{\isacharparenright}{\kern0pt}\isanewline
\ \ \ \ \isacommand{then}\isamarkupfalse%
\ \isacommand{have}\isamarkupfalse%
\ {\isachardoublequoteopen}Q{\isacharparenleft}{\kern0pt}ft{\isacharcomma}{\kern0pt}{\isasymtau}{\isacharcomma}{\kern0pt}{\isasymtheta}{\isacharcomma}{\kern0pt}p{\isacharparenright}{\kern0pt}{\isachardoublequoteclose}\ \isacommand{by}\isamarkupfalse%
\ {\isacharparenleft}{\kern0pt}simp\ add{\isacharcolon}{\kern0pt}components{\isacharunderscore}{\kern0pt}simp{\isacharparenright}{\kern0pt}\isanewline
\ \ \isacommand{{\isacharbraceright}{\kern0pt}}\isamarkupfalse%
\isanewline
\ \ \isacommand{then}\isamarkupfalse%
\ \isacommand{show}\isamarkupfalse%
\ {\isacharquery}{\kern0pt}thesis\ \isacommand{using}\isamarkupfalse%
\ assms\ \isacommand{by}\isamarkupfalse%
\ simp\isanewline
\isacommand{qed}\isamarkupfalse%
%
\endisatagproof
{\isafoldproof}%
%
\isadelimproof
\isanewline
%
\endisadelimproof
\isanewline
\isacommand{lemma}\isamarkupfalse%
\ forces{\isacharunderscore}{\kern0pt}induction{\isacharunderscore}{\kern0pt}with{\isacharunderscore}{\kern0pt}conds{\isacharcolon}{\kern0pt}\isanewline
\ \ \isakeyword{assumes}\isanewline
\ \ \ \ {\isachardoublequoteopen}{\isasymAnd}{\isasymtau}\ {\isasymtheta}\ p{\isachardot}{\kern0pt}\ p\ {\isasymin}\ P\ {\isasymLongrightarrow}\ {\isasymlbrakk}{\isasymAnd}q\ {\isasymsigma}{\isachardot}{\kern0pt}\ {\isasymlbrakk}q{\isasymin}P\ {\isacharsemicolon}{\kern0pt}\ {\isasymsigma}{\isasymin}domain{\isacharparenleft}{\kern0pt}{\isasymtheta}{\isacharparenright}{\kern0pt}{\isasymrbrakk}\ {\isasymLongrightarrow}\ Q{\isacharparenleft}{\kern0pt}q{\isacharcomma}{\kern0pt}{\isasymtau}{\isacharcomma}{\kern0pt}{\isasymsigma}{\isacharparenright}{\kern0pt}{\isasymrbrakk}\ {\isasymLongrightarrow}\ R{\isacharparenleft}{\kern0pt}p{\isacharcomma}{\kern0pt}{\isasymtau}{\isacharcomma}{\kern0pt}{\isasymtheta}{\isacharparenright}{\kern0pt}{\isachardoublequoteclose}\isanewline
\ \ \ \ {\isachardoublequoteopen}{\isasymAnd}{\isasymtau}\ {\isasymtheta}\ p{\isachardot}{\kern0pt}\ p\ {\isasymin}\ P\ {\isasymLongrightarrow}\ {\isasymlbrakk}{\isasymAnd}q\ {\isasymsigma}{\isachardot}{\kern0pt}\ {\isasymlbrakk}q{\isasymin}P\ {\isacharsemicolon}{\kern0pt}\ {\isasymsigma}{\isasymin}domain{\isacharparenleft}{\kern0pt}{\isasymtau}{\isacharparenright}{\kern0pt}\ {\isasymunion}\ domain{\isacharparenleft}{\kern0pt}{\isasymtheta}{\isacharparenright}{\kern0pt}{\isasymrbrakk}\ {\isasymLongrightarrow}\ R{\isacharparenleft}{\kern0pt}q{\isacharcomma}{\kern0pt}{\isasymsigma}{\isacharcomma}{\kern0pt}{\isasymtau}{\isacharparenright}{\kern0pt}\ {\isasymand}\ R{\isacharparenleft}{\kern0pt}q{\isacharcomma}{\kern0pt}{\isasymsigma}{\isacharcomma}{\kern0pt}{\isasymtheta}{\isacharparenright}{\kern0pt}{\isasymrbrakk}\ {\isasymLongrightarrow}\ Q{\isacharparenleft}{\kern0pt}p{\isacharcomma}{\kern0pt}{\isasymtau}{\isacharcomma}{\kern0pt}{\isasymtheta}{\isacharparenright}{\kern0pt}{\isachardoublequoteclose}\isanewline
\ \ \ \ {\isachardoublequoteopen}p\ {\isasymin}\ P{\isachardoublequoteclose}\isanewline
\ \ \isakeyword{shows}\isanewline
\ \ \ \ {\isachardoublequoteopen}Q{\isacharparenleft}{\kern0pt}p{\isacharcomma}{\kern0pt}{\isasymtau}{\isacharcomma}{\kern0pt}{\isasymtheta}{\isacharparenright}{\kern0pt}\ {\isasymand}\ R{\isacharparenleft}{\kern0pt}p{\isacharcomma}{\kern0pt}{\isasymtau}{\isacharcomma}{\kern0pt}{\isasymtheta}{\isacharparenright}{\kern0pt}{\isachardoublequoteclose}\isanewline
%
\isadelimproof
%
\endisadelimproof
%
\isatagproof
\isacommand{proof}\isamarkupfalse%
\ {\isacharminus}{\kern0pt}\isanewline
\ \ \isacommand{let}\isamarkupfalse%
\ {\isacharquery}{\kern0pt}Q{\isacharequal}{\kern0pt}{\isachardoublequoteopen}{\isasymlambda}ft\ {\isasymtau}\ {\isasymtheta}\ p{\isachardot}{\kern0pt}\ {\isacharparenleft}{\kern0pt}ft\ {\isacharequal}{\kern0pt}\ {\isadigit{0}}\ {\isasymlongrightarrow}\ Q{\isacharparenleft}{\kern0pt}p{\isacharcomma}{\kern0pt}{\isasymtau}{\isacharcomma}{\kern0pt}{\isasymtheta}{\isacharparenright}{\kern0pt}{\isacharparenright}{\kern0pt}\ {\isasymand}\ {\isacharparenleft}{\kern0pt}ft\ {\isacharequal}{\kern0pt}\ {\isadigit{1}}\ {\isasymlongrightarrow}\ R{\isacharparenleft}{\kern0pt}p{\isacharcomma}{\kern0pt}{\isasymtau}{\isacharcomma}{\kern0pt}{\isasymtheta}{\isacharparenright}{\kern0pt}{\isacharparenright}{\kern0pt}{\isachardoublequoteclose}\isanewline
\ \ \isacommand{from}\isamarkupfalse%
\ assms{\isacharparenleft}{\kern0pt}{\isadigit{1}}{\isacharparenright}{\kern0pt}\isanewline
\ \ \isacommand{have}\isamarkupfalse%
\ {\isachardoublequoteopen}{\isasymAnd}{\isasymtau}\ {\isasymtheta}\ p{\isachardot}{\kern0pt}\ p\ {\isasymin}\ P\ {\isasymLongrightarrow}\ {\isasymlbrakk}{\isasymAnd}q\ {\isasymsigma}{\isachardot}{\kern0pt}\ {\isasymlbrakk}q{\isasymin}P\ {\isacharsemicolon}{\kern0pt}\ {\isasymsigma}{\isasymin}domain{\isacharparenleft}{\kern0pt}{\isasymtheta}{\isacharparenright}{\kern0pt}{\isasymrbrakk}\ {\isasymLongrightarrow}\ {\isacharquery}{\kern0pt}Q{\isacharparenleft}{\kern0pt}{\isadigit{0}}{\isacharcomma}{\kern0pt}{\isasymtau}{\isacharcomma}{\kern0pt}{\isasymsigma}{\isacharcomma}{\kern0pt}q{\isacharparenright}{\kern0pt}{\isasymrbrakk}\ {\isasymLongrightarrow}\ {\isacharquery}{\kern0pt}Q{\isacharparenleft}{\kern0pt}{\isadigit{1}}{\isacharcomma}{\kern0pt}{\isasymtau}{\isacharcomma}{\kern0pt}{\isasymtheta}{\isacharcomma}{\kern0pt}p{\isacharparenright}{\kern0pt}{\isachardoublequoteclose}\isanewline
\ \ \ \ \isacommand{by}\isamarkupfalse%
\ simp\isanewline
\ \ \isacommand{moreover}\isamarkupfalse%
\ \isacommand{from}\isamarkupfalse%
\ assms{\isacharparenleft}{\kern0pt}{\isadigit{2}}{\isacharparenright}{\kern0pt}\isanewline
\ \ \isacommand{have}\isamarkupfalse%
\ {\isachardoublequoteopen}{\isasymAnd}{\isasymtau}\ {\isasymtheta}\ p{\isachardot}{\kern0pt}\ p\ {\isasymin}\ P\ {\isasymLongrightarrow}\ {\isasymlbrakk}{\isasymAnd}q\ {\isasymsigma}{\isachardot}{\kern0pt}\ {\isasymlbrakk}q{\isasymin}P\ {\isacharsemicolon}{\kern0pt}\ {\isasymsigma}{\isasymin}domain{\isacharparenleft}{\kern0pt}{\isasymtau}{\isacharparenright}{\kern0pt}\ {\isasymunion}\ domain{\isacharparenleft}{\kern0pt}{\isasymtheta}{\isacharparenright}{\kern0pt}{\isasymrbrakk}\ {\isasymLongrightarrow}\ {\isacharquery}{\kern0pt}Q{\isacharparenleft}{\kern0pt}{\isadigit{1}}{\isacharcomma}{\kern0pt}{\isasymsigma}{\isacharcomma}{\kern0pt}{\isasymtau}{\isacharcomma}{\kern0pt}q{\isacharparenright}{\kern0pt}\ {\isasymand}\ {\isacharquery}{\kern0pt}Q{\isacharparenleft}{\kern0pt}{\isadigit{1}}{\isacharcomma}{\kern0pt}{\isasymsigma}{\isacharcomma}{\kern0pt}{\isasymtheta}{\isacharcomma}{\kern0pt}q{\isacharparenright}{\kern0pt}{\isasymrbrakk}\ {\isasymLongrightarrow}\ {\isacharquery}{\kern0pt}Q{\isacharparenleft}{\kern0pt}{\isadigit{0}}{\isacharcomma}{\kern0pt}{\isasymtau}{\isacharcomma}{\kern0pt}{\isasymtheta}{\isacharcomma}{\kern0pt}p{\isacharparenright}{\kern0pt}{\isachardoublequoteclose}\isanewline
\ \ \ \ \isacommand{by}\isamarkupfalse%
\ simp\isanewline
\ \ \isacommand{moreover}\isamarkupfalse%
\isanewline
\ \ \isacommand{note}\isamarkupfalse%
\ {\isacartoucheopen}p{\isasymin}P{\isacartoucheclose}\isanewline
\ \ \isacommand{ultimately}\isamarkupfalse%
\isanewline
\ \ \isacommand{have}\isamarkupfalse%
\ {\isachardoublequoteopen}{\isacharquery}{\kern0pt}Q{\isacharparenleft}{\kern0pt}ft{\isacharcomma}{\kern0pt}{\isasymtau}{\isacharcomma}{\kern0pt}{\isasymtheta}{\isacharcomma}{\kern0pt}p{\isacharparenright}{\kern0pt}{\isachardoublequoteclose}\ \isakeyword{if}\ {\isachardoublequoteopen}ft{\isasymin}{\isadigit{2}}{\isachardoublequoteclose}\ \isakeyword{for}\ ft\isanewline
\ \ \ \ \isacommand{by}\isamarkupfalse%
\ {\isacharparenleft}{\kern0pt}rule\ core{\isacharunderscore}{\kern0pt}induction{\isacharbrackleft}{\kern0pt}OF\ {\isacharunderscore}{\kern0pt}\ {\isacharunderscore}{\kern0pt}\ that{\isacharcomma}{\kern0pt}\ of\ {\isacharquery}{\kern0pt}Q{\isacharbrackright}{\kern0pt}{\isacharparenright}{\kern0pt}\isanewline
\ \ \isacommand{then}\isamarkupfalse%
\isanewline
\ \ \isacommand{show}\isamarkupfalse%
\ {\isacharquery}{\kern0pt}thesis\ \isacommand{by}\isamarkupfalse%
\ auto\isanewline
\isacommand{qed}\isamarkupfalse%
%
\endisatagproof
{\isafoldproof}%
%
\isadelimproof
\isanewline
%
\endisadelimproof
\isanewline
\isacommand{lemma}\isamarkupfalse%
\ forces{\isacharunderscore}{\kern0pt}induction{\isacharcolon}{\kern0pt}\isanewline
\ \ \isakeyword{assumes}\isanewline
\ \ \ \ {\isachardoublequoteopen}{\isasymAnd}{\isasymtau}\ {\isasymtheta}{\isachardot}{\kern0pt}\ {\isasymlbrakk}{\isasymAnd}{\isasymsigma}{\isachardot}{\kern0pt}\ {\isasymsigma}{\isasymin}domain{\isacharparenleft}{\kern0pt}{\isasymtheta}{\isacharparenright}{\kern0pt}\ {\isasymLongrightarrow}\ Q{\isacharparenleft}{\kern0pt}{\isasymtau}{\isacharcomma}{\kern0pt}{\isasymsigma}{\isacharparenright}{\kern0pt}{\isasymrbrakk}\ {\isasymLongrightarrow}\ R{\isacharparenleft}{\kern0pt}{\isasymtau}{\isacharcomma}{\kern0pt}{\isasymtheta}{\isacharparenright}{\kern0pt}{\isachardoublequoteclose}\isanewline
\ \ \ \ {\isachardoublequoteopen}{\isasymAnd}{\isasymtau}\ {\isasymtheta}{\isachardot}{\kern0pt}\ {\isasymlbrakk}{\isasymAnd}{\isasymsigma}{\isachardot}{\kern0pt}\ {\isasymsigma}{\isasymin}domain{\isacharparenleft}{\kern0pt}{\isasymtau}{\isacharparenright}{\kern0pt}\ {\isasymunion}\ domain{\isacharparenleft}{\kern0pt}{\isasymtheta}{\isacharparenright}{\kern0pt}\ {\isasymLongrightarrow}\ R{\isacharparenleft}{\kern0pt}{\isasymsigma}{\isacharcomma}{\kern0pt}{\isasymtau}{\isacharparenright}{\kern0pt}\ {\isasymand}\ R{\isacharparenleft}{\kern0pt}{\isasymsigma}{\isacharcomma}{\kern0pt}{\isasymtheta}{\isacharparenright}{\kern0pt}{\isasymrbrakk}\ {\isasymLongrightarrow}\ Q{\isacharparenleft}{\kern0pt}{\isasymtau}{\isacharcomma}{\kern0pt}{\isasymtheta}{\isacharparenright}{\kern0pt}{\isachardoublequoteclose}\isanewline
\ \ \isakeyword{shows}\isanewline
\ \ \ \ {\isachardoublequoteopen}Q{\isacharparenleft}{\kern0pt}{\isasymtau}{\isacharcomma}{\kern0pt}{\isasymtheta}{\isacharparenright}{\kern0pt}\ {\isasymand}\ R{\isacharparenleft}{\kern0pt}{\isasymtau}{\isacharcomma}{\kern0pt}{\isasymtheta}{\isacharparenright}{\kern0pt}{\isachardoublequoteclose}\isanewline
%
\isadelimproof
%
\endisadelimproof
%
\isatagproof
\isacommand{proof}\isamarkupfalse%
\ {\isacharparenleft}{\kern0pt}intro\ forces{\isacharunderscore}{\kern0pt}induction{\isacharunderscore}{\kern0pt}with{\isacharunderscore}{\kern0pt}conds{\isacharbrackleft}{\kern0pt}OF\ {\isacharunderscore}{\kern0pt}\ {\isacharunderscore}{\kern0pt}\ one{\isacharunderscore}{\kern0pt}in{\isacharunderscore}{\kern0pt}P\ {\isacharbrackright}{\kern0pt}{\isacharparenright}{\kern0pt}\isanewline
\ \ \isacommand{fix}\isamarkupfalse%
\ {\isasymtau}\ {\isasymtheta}\ p\ \isanewline
\ \ \isacommand{assume}\isamarkupfalse%
\ {\isachardoublequoteopen}q\ {\isasymin}\ P\ {\isasymLongrightarrow}\ {\isasymsigma}\ {\isasymin}\ domain{\isacharparenleft}{\kern0pt}{\isasymtheta}{\isacharparenright}{\kern0pt}\ {\isasymLongrightarrow}\ Q{\isacharparenleft}{\kern0pt}{\isasymtau}{\isacharcomma}{\kern0pt}\ {\isasymsigma}{\isacharparenright}{\kern0pt}{\isachardoublequoteclose}\ \isakeyword{for}\ q\ {\isasymsigma}\isanewline
\ \ \isacommand{with}\isamarkupfalse%
\ assms{\isacharparenleft}{\kern0pt}{\isadigit{1}}{\isacharparenright}{\kern0pt}\isanewline
\ \ \isacommand{show}\isamarkupfalse%
\ {\isachardoublequoteopen}R{\isacharparenleft}{\kern0pt}{\isasymtau}{\isacharcomma}{\kern0pt}{\isasymtheta}{\isacharparenright}{\kern0pt}{\isachardoublequoteclose}\isanewline
\ \ \ \ \isacommand{using}\isamarkupfalse%
\ one{\isacharunderscore}{\kern0pt}in{\isacharunderscore}{\kern0pt}P\ \isacommand{by}\isamarkupfalse%
\ simp\isanewline
\isacommand{next}\isamarkupfalse%
\isanewline
\ \ \isacommand{fix}\isamarkupfalse%
\ {\isasymtau}\ {\isasymtheta}\ p\ \isanewline
\ \ \ \ \isacommand{assume}\isamarkupfalse%
\ {\isachardoublequoteopen}q\ {\isasymin}\ P\ {\isasymLongrightarrow}\ {\isasymsigma}\ {\isasymin}\ domain{\isacharparenleft}{\kern0pt}{\isasymtau}{\isacharparenright}{\kern0pt}\ {\isasymunion}\ domain{\isacharparenleft}{\kern0pt}{\isasymtheta}{\isacharparenright}{\kern0pt}\ {\isasymLongrightarrow}\ R{\isacharparenleft}{\kern0pt}{\isasymsigma}{\isacharcomma}{\kern0pt}{\isasymtau}{\isacharparenright}{\kern0pt}\ {\isasymand}\ R{\isacharparenleft}{\kern0pt}{\isasymsigma}{\isacharcomma}{\kern0pt}{\isasymtheta}{\isacharparenright}{\kern0pt}{\isachardoublequoteclose}\ \isakeyword{for}\ q\ {\isasymsigma}\isanewline
\ \ \ \ \isacommand{with}\isamarkupfalse%
\ assms{\isacharparenleft}{\kern0pt}{\isadigit{2}}{\isacharparenright}{\kern0pt}\isanewline
\ \ \ \ \isacommand{show}\isamarkupfalse%
\ {\isachardoublequoteopen}Q{\isacharparenleft}{\kern0pt}{\isasymtau}{\isacharcomma}{\kern0pt}{\isasymtheta}{\isacharparenright}{\kern0pt}{\isachardoublequoteclose}\isanewline
\ \ \ \ \ \ \isacommand{using}\isamarkupfalse%
\ one{\isacharunderscore}{\kern0pt}in{\isacharunderscore}{\kern0pt}P\ \isacommand{by}\isamarkupfalse%
\ simp\isanewline
\isacommand{qed}\isamarkupfalse%
%
\endisatagproof
{\isafoldproof}%
%
\isadelimproof
%
\endisadelimproof
%
\isadelimdocument
%
\endisadelimdocument
%
\isatagdocument
%
\isamarkupsubsection{Lemma IV.2.40(a), in full%
}
\isamarkuptrue%
%
\endisatagdocument
{\isafolddocument}%
%
\isadelimdocument
%
\endisadelimdocument
\isacommand{lemma}\isamarkupfalse%
\ IV{\isadigit{2}}{\isadigit{4}}{\isadigit{0}}a{\isacharcolon}{\kern0pt}\isanewline
\ \ \isakeyword{assumes}\isanewline
\ \ \ \ {\isachardoublequoteopen}M{\isacharunderscore}{\kern0pt}generic{\isacharparenleft}{\kern0pt}G{\isacharparenright}{\kern0pt}{\isachardoublequoteclose}\isanewline
\ \ \isakeyword{shows}\ \isanewline
\ \ \ \ {\isachardoublequoteopen}{\isacharparenleft}{\kern0pt}{\isasymtau}{\isasymin}M\ {\isasymlongrightarrow}\ {\isasymtheta}{\isasymin}M\ {\isasymlongrightarrow}\ {\isacharparenleft}{\kern0pt}{\isasymforall}p{\isasymin}G{\isachardot}{\kern0pt}\ forces{\isacharunderscore}{\kern0pt}eq{\isacharparenleft}{\kern0pt}p{\isacharcomma}{\kern0pt}{\isasymtau}{\isacharcomma}{\kern0pt}{\isasymtheta}{\isacharparenright}{\kern0pt}\ {\isasymlongrightarrow}\ val{\isacharparenleft}{\kern0pt}G{\isacharcomma}{\kern0pt}{\isasymtau}{\isacharparenright}{\kern0pt}\ {\isacharequal}{\kern0pt}\ val{\isacharparenleft}{\kern0pt}G{\isacharcomma}{\kern0pt}{\isasymtheta}{\isacharparenright}{\kern0pt}{\isacharparenright}{\kern0pt}{\isacharparenright}{\kern0pt}\ {\isasymand}\ \isanewline
\ \ \ \ \ {\isacharparenleft}{\kern0pt}{\isasymtau}{\isasymin}M\ {\isasymlongrightarrow}\ {\isasymtheta}{\isasymin}M\ {\isasymlongrightarrow}\ {\isacharparenleft}{\kern0pt}{\isasymforall}p{\isasymin}G{\isachardot}{\kern0pt}\ forces{\isacharunderscore}{\kern0pt}mem{\isacharparenleft}{\kern0pt}p{\isacharcomma}{\kern0pt}{\isasymtau}{\isacharcomma}{\kern0pt}{\isasymtheta}{\isacharparenright}{\kern0pt}\ {\isasymlongrightarrow}\ val{\isacharparenleft}{\kern0pt}G{\isacharcomma}{\kern0pt}{\isasymtau}{\isacharparenright}{\kern0pt}\ {\isasymin}\ val{\isacharparenleft}{\kern0pt}G{\isacharcomma}{\kern0pt}{\isasymtheta}{\isacharparenright}{\kern0pt}{\isacharparenright}{\kern0pt}{\isacharparenright}{\kern0pt}{\isachardoublequoteclose}\isanewline
\ \ \ \ {\isacharparenleft}{\kern0pt}\isakeyword{is}\ {\isachardoublequoteopen}{\isacharquery}{\kern0pt}Q{\isacharparenleft}{\kern0pt}{\isasymtau}{\isacharcomma}{\kern0pt}{\isasymtheta}{\isacharparenright}{\kern0pt}\ {\isasymand}\ {\isacharquery}{\kern0pt}R{\isacharparenleft}{\kern0pt}{\isasymtau}{\isacharcomma}{\kern0pt}{\isasymtheta}{\isacharparenright}{\kern0pt}{\isachardoublequoteclose}{\isacharparenright}{\kern0pt}\isanewline
%
\isadelimproof
%
\endisadelimproof
%
\isatagproof
\isacommand{proof}\isamarkupfalse%
\ {\isacharparenleft}{\kern0pt}intro\ forces{\isacharunderscore}{\kern0pt}induction{\isacharbrackleft}{\kern0pt}of\ {\isacharquery}{\kern0pt}Q\ {\isacharquery}{\kern0pt}R{\isacharbrackright}{\kern0pt}\ impI{\isacharparenright}{\kern0pt}\isanewline
\ \ \isacommand{fix}\isamarkupfalse%
\ {\isasymtau}\ {\isasymtheta}\ \isanewline
\ \ \isacommand{assume}\isamarkupfalse%
\ {\isachardoublequoteopen}{\isasymtau}{\isasymin}M{\isachardoublequoteclose}\ {\isachardoublequoteopen}{\isasymtheta}{\isasymin}M{\isachardoublequoteclose}\ \ {\isachardoublequoteopen}{\isasymsigma}{\isasymin}domain{\isacharparenleft}{\kern0pt}{\isasymtheta}{\isacharparenright}{\kern0pt}\ {\isasymLongrightarrow}\ {\isacharquery}{\kern0pt}Q{\isacharparenleft}{\kern0pt}{\isasymtau}{\isacharcomma}{\kern0pt}{\isasymsigma}{\isacharparenright}{\kern0pt}{\isachardoublequoteclose}\ \isakeyword{for}\ {\isasymsigma}\isanewline
\ \ \isacommand{moreover}\isamarkupfalse%
\ \isacommand{from}\isamarkupfalse%
\ this\isanewline
\ \ \isacommand{have}\isamarkupfalse%
\ {\isachardoublequoteopen}{\isasymsigma}{\isasymin}domain{\isacharparenleft}{\kern0pt}{\isasymtheta}{\isacharparenright}{\kern0pt}\ {\isasymLongrightarrow}\ forces{\isacharunderscore}{\kern0pt}eq{\isacharparenleft}{\kern0pt}q{\isacharcomma}{\kern0pt}\ {\isasymtau}{\isacharcomma}{\kern0pt}\ {\isasymsigma}{\isacharparenright}{\kern0pt}\ {\isasymLongrightarrow}\ val{\isacharparenleft}{\kern0pt}G{\isacharcomma}{\kern0pt}\ {\isasymtau}{\isacharparenright}{\kern0pt}\ {\isacharequal}{\kern0pt}\ val{\isacharparenleft}{\kern0pt}G{\isacharcomma}{\kern0pt}\ {\isasymsigma}{\isacharparenright}{\kern0pt}{\isachardoublequoteclose}\ \isanewline
\ \ \ \ \isakeyword{if}\ {\isachardoublequoteopen}q{\isasymin}P{\isachardoublequoteclose}\ {\isachardoublequoteopen}q{\isasymin}G{\isachardoublequoteclose}\ \isakeyword{for}\ q\ {\isasymsigma}\isanewline
\ \ \ \ \isacommand{using}\isamarkupfalse%
\ that\ domain{\isacharunderscore}{\kern0pt}closed{\isacharbrackleft}{\kern0pt}of\ {\isasymtheta}{\isacharbrackright}{\kern0pt}\ transitivity\ \isacommand{by}\isamarkupfalse%
\ auto\isanewline
\ \ \isacommand{moreover}\isamarkupfalse%
\ \isanewline
\ \ \isacommand{note}\isamarkupfalse%
\ assms\isanewline
\ \ \isacommand{ultimately}\isamarkupfalse%
\isanewline
\ \ \isacommand{show}\isamarkupfalse%
\ {\isachardoublequoteopen}{\isasymforall}p{\isasymin}G{\isachardot}{\kern0pt}\ forces{\isacharunderscore}{\kern0pt}mem{\isacharparenleft}{\kern0pt}p{\isacharcomma}{\kern0pt}{\isasymtau}{\isacharcomma}{\kern0pt}{\isasymtheta}{\isacharparenright}{\kern0pt}\ {\isasymlongrightarrow}\ val{\isacharparenleft}{\kern0pt}G{\isacharcomma}{\kern0pt}{\isasymtau}{\isacharparenright}{\kern0pt}\ {\isasymin}\ val{\isacharparenleft}{\kern0pt}G{\isacharcomma}{\kern0pt}{\isasymtheta}{\isacharparenright}{\kern0pt}{\isachardoublequoteclose}\isanewline
\ \ \ \ \isacommand{using}\isamarkupfalse%
\ IV{\isadigit{2}}{\isadigit{4}}{\isadigit{0}}a{\isacharunderscore}{\kern0pt}mem\ domain{\isacharunderscore}{\kern0pt}closed\ transitivity\ \isacommand{by}\isamarkupfalse%
\ {\isacharparenleft}{\kern0pt}simp{\isacharparenright}{\kern0pt}\isanewline
\isacommand{next}\isamarkupfalse%
\isanewline
\ \ \isacommand{fix}\isamarkupfalse%
\ {\isasymtau}\ {\isasymtheta}\ \isanewline
\ \ \isacommand{assume}\isamarkupfalse%
\ {\isachardoublequoteopen}{\isasymtau}{\isasymin}M{\isachardoublequoteclose}\ {\isachardoublequoteopen}{\isasymtheta}{\isasymin}M{\isachardoublequoteclose}\ {\isachardoublequoteopen}{\isasymsigma}\ {\isasymin}\ domain{\isacharparenleft}{\kern0pt}{\isasymtau}{\isacharparenright}{\kern0pt}\ {\isasymunion}\ domain{\isacharparenleft}{\kern0pt}{\isasymtheta}{\isacharparenright}{\kern0pt}\ {\isasymLongrightarrow}\ {\isacharquery}{\kern0pt}R{\isacharparenleft}{\kern0pt}{\isasymsigma}{\isacharcomma}{\kern0pt}{\isasymtau}{\isacharparenright}{\kern0pt}\ {\isasymand}\ {\isacharquery}{\kern0pt}R{\isacharparenleft}{\kern0pt}{\isasymsigma}{\isacharcomma}{\kern0pt}{\isasymtheta}{\isacharparenright}{\kern0pt}{\isachardoublequoteclose}\ \isakeyword{for}\ {\isasymsigma}\isanewline
\ \ \isacommand{moreover}\isamarkupfalse%
\ \isacommand{from}\isamarkupfalse%
\ this\isanewline
\ \ \isacommand{have}\isamarkupfalse%
\ IH{\isacharprime}{\kern0pt}{\isacharcolon}{\kern0pt}{\isachardoublequoteopen}{\isasymsigma}\ {\isasymin}\ domain{\isacharparenleft}{\kern0pt}{\isasymtau}{\isacharparenright}{\kern0pt}\ {\isasymunion}\ domain{\isacharparenleft}{\kern0pt}{\isasymtheta}{\isacharparenright}{\kern0pt}\ {\isasymLongrightarrow}\ q{\isasymin}G\ {\isasymLongrightarrow}\isanewline
\ \ \ \ \ \ \ \ \ \ \ \ {\isacharparenleft}{\kern0pt}forces{\isacharunderscore}{\kern0pt}mem{\isacharparenleft}{\kern0pt}q{\isacharcomma}{\kern0pt}\ {\isasymsigma}{\isacharcomma}{\kern0pt}\ {\isasymtau}{\isacharparenright}{\kern0pt}\ {\isasymlongrightarrow}\ val{\isacharparenleft}{\kern0pt}G{\isacharcomma}{\kern0pt}\ {\isasymsigma}{\isacharparenright}{\kern0pt}\ {\isasymin}\ val{\isacharparenleft}{\kern0pt}G{\isacharcomma}{\kern0pt}\ {\isasymtau}{\isacharparenright}{\kern0pt}{\isacharparenright}{\kern0pt}\ {\isasymand}\ \isanewline
\ \ \ \ \ \ \ \ \ \ \ \ {\isacharparenleft}{\kern0pt}forces{\isacharunderscore}{\kern0pt}mem{\isacharparenleft}{\kern0pt}q{\isacharcomma}{\kern0pt}\ {\isasymsigma}{\isacharcomma}{\kern0pt}\ {\isasymtheta}{\isacharparenright}{\kern0pt}\ {\isasymlongrightarrow}\ val{\isacharparenleft}{\kern0pt}G{\isacharcomma}{\kern0pt}\ {\isasymsigma}{\isacharparenright}{\kern0pt}\ {\isasymin}\ val{\isacharparenleft}{\kern0pt}G{\isacharcomma}{\kern0pt}\ {\isasymtheta}{\isacharparenright}{\kern0pt}{\isacharparenright}{\kern0pt}{\isachardoublequoteclose}\ \isakeyword{for}\ q\ {\isasymsigma}\ \isanewline
\ \ \ \ \isacommand{by}\isamarkupfalse%
\ {\isacharparenleft}{\kern0pt}auto\ intro{\isacharcolon}{\kern0pt}\ \ transitivity{\isacharbrackleft}{\kern0pt}OF\ {\isacharunderscore}{\kern0pt}\ domain{\isacharunderscore}{\kern0pt}closed{\isacharbrackleft}{\kern0pt}simplified{\isacharbrackright}{\kern0pt}{\isacharbrackright}{\kern0pt}{\isacharparenright}{\kern0pt}\isanewline
\ \ \isacommand{ultimately}\isamarkupfalse%
\isanewline
\ \ \isacommand{show}\isamarkupfalse%
\ {\isachardoublequoteopen}{\isasymforall}p{\isasymin}G{\isachardot}{\kern0pt}\ forces{\isacharunderscore}{\kern0pt}eq{\isacharparenleft}{\kern0pt}p{\isacharcomma}{\kern0pt}{\isasymtau}{\isacharcomma}{\kern0pt}{\isasymtheta}{\isacharparenright}{\kern0pt}\ {\isasymlongrightarrow}\ val{\isacharparenleft}{\kern0pt}G{\isacharcomma}{\kern0pt}{\isasymtau}{\isacharparenright}{\kern0pt}\ {\isacharequal}{\kern0pt}\ val{\isacharparenleft}{\kern0pt}G{\isacharcomma}{\kern0pt}{\isasymtheta}{\isacharparenright}{\kern0pt}{\isachardoublequoteclose}\isanewline
\ \ \ \ \isacommand{using}\isamarkupfalse%
\ IV{\isadigit{2}}{\isadigit{4}}{\isadigit{0}}a{\isacharunderscore}{\kern0pt}eq{\isacharbrackleft}{\kern0pt}OF\ assms{\isacharparenleft}{\kern0pt}{\isadigit{1}}{\isacharparenright}{\kern0pt}\ {\isacharunderscore}{\kern0pt}\ {\isacharunderscore}{\kern0pt}\ IH{\isacharprime}{\kern0pt}{\isacharbrackright}{\kern0pt}\ \isacommand{by}\isamarkupfalse%
\ {\isacharparenleft}{\kern0pt}simp{\isacharparenright}{\kern0pt}\isanewline
\isacommand{qed}\isamarkupfalse%
%
\endisatagproof
{\isafoldproof}%
%
\isadelimproof
%
\endisadelimproof
%
\isadelimdocument
%
\endisadelimdocument
%
\isatagdocument
%
\isamarkupsubsection{Lemma IV.2.40(b)%
}
\isamarkuptrue%
%
\endisatagdocument
{\isafolddocument}%
%
\isadelimdocument
%
\endisadelimdocument
\isacommand{lemma}\isamarkupfalse%
\ IV{\isadigit{2}}{\isadigit{4}}{\isadigit{0}}b{\isacharunderscore}{\kern0pt}mem{\isacharcolon}{\kern0pt}\isanewline
\ \ \isakeyword{assumes}\isanewline
\ \ \ \ {\isachardoublequoteopen}M{\isacharunderscore}{\kern0pt}generic{\isacharparenleft}{\kern0pt}G{\isacharparenright}{\kern0pt}{\isachardoublequoteclose}\ {\isachardoublequoteopen}val{\isacharparenleft}{\kern0pt}G{\isacharcomma}{\kern0pt}{\isasympi}{\isacharparenright}{\kern0pt}{\isasymin}val{\isacharparenleft}{\kern0pt}G{\isacharcomma}{\kern0pt}{\isasymtau}{\isacharparenright}{\kern0pt}{\isachardoublequoteclose}\ {\isachardoublequoteopen}{\isasympi}{\isasymin}M{\isachardoublequoteclose}\ {\isachardoublequoteopen}{\isasymtau}{\isasymin}M{\isachardoublequoteclose}\isanewline
\ \ \ \ \isakeyword{and}\isanewline
\ \ \ \ IH{\isacharcolon}{\kern0pt}{\isachardoublequoteopen}{\isasymAnd}{\isasymsigma}{\isachardot}{\kern0pt}\ {\isasymsigma}{\isasymin}domain{\isacharparenleft}{\kern0pt}{\isasymtau}{\isacharparenright}{\kern0pt}\ {\isasymLongrightarrow}\ val{\isacharparenleft}{\kern0pt}G{\isacharcomma}{\kern0pt}{\isasympi}{\isacharparenright}{\kern0pt}\ {\isacharequal}{\kern0pt}\ val{\isacharparenleft}{\kern0pt}G{\isacharcomma}{\kern0pt}{\isasymsigma}{\isacharparenright}{\kern0pt}\ {\isasymLongrightarrow}\ \isanewline
\ \ \ \ \ \ {\isasymexists}p{\isasymin}G{\isachardot}{\kern0pt}\ forces{\isacharunderscore}{\kern0pt}eq{\isacharparenleft}{\kern0pt}p{\isacharcomma}{\kern0pt}{\isasympi}{\isacharcomma}{\kern0pt}{\isasymsigma}{\isacharparenright}{\kern0pt}{\isachardoublequoteclose}\ \isanewline
\ \ \isakeyword{shows}\isanewline
\ \ \ \ {\isachardoublequoteopen}{\isasymexists}p{\isasymin}G{\isachardot}{\kern0pt}\ forces{\isacharunderscore}{\kern0pt}mem{\isacharparenleft}{\kern0pt}p{\isacharcomma}{\kern0pt}{\isasympi}{\isacharcomma}{\kern0pt}{\isasymtau}{\isacharparenright}{\kern0pt}{\isachardoublequoteclose}\isanewline
%
\isadelimproof
%
\endisadelimproof
%
\isatagproof
\isacommand{proof}\isamarkupfalse%
\ {\isacharminus}{\kern0pt}\isanewline
\ \ \isacommand{from}\isamarkupfalse%
\ {\isacartoucheopen}val{\isacharparenleft}{\kern0pt}G{\isacharcomma}{\kern0pt}{\isasympi}{\isacharparenright}{\kern0pt}{\isasymin}val{\isacharparenleft}{\kern0pt}G{\isacharcomma}{\kern0pt}{\isasymtau}{\isacharparenright}{\kern0pt}{\isacartoucheclose}\isanewline
\ \ \isacommand{obtain}\isamarkupfalse%
\ {\isasymsigma}\ r\ \isakeyword{where}\ {\isachardoublequoteopen}r{\isasymin}G{\isachardoublequoteclose}\ {\isachardoublequoteopen}{\isasymlangle}{\isasymsigma}{\isacharcomma}{\kern0pt}r{\isasymrangle}{\isasymin}{\isasymtau}{\isachardoublequoteclose}\ {\isachardoublequoteopen}val{\isacharparenleft}{\kern0pt}G{\isacharcomma}{\kern0pt}{\isasympi}{\isacharparenright}{\kern0pt}\ {\isacharequal}{\kern0pt}\ val{\isacharparenleft}{\kern0pt}G{\isacharcomma}{\kern0pt}{\isasymsigma}{\isacharparenright}{\kern0pt}{\isachardoublequoteclose}\ \isacommand{by}\isamarkupfalse%
\ auto\isanewline
\ \ \isacommand{moreover}\isamarkupfalse%
\ \isacommand{from}\isamarkupfalse%
\ this\ \isakeyword{and}\ IH\isanewline
\ \ \isacommand{obtain}\isamarkupfalse%
\ p{\isacharprime}{\kern0pt}\ \isakeyword{where}\ {\isachardoublequoteopen}p{\isacharprime}{\kern0pt}{\isasymin}G{\isachardoublequoteclose}\ {\isachardoublequoteopen}forces{\isacharunderscore}{\kern0pt}eq{\isacharparenleft}{\kern0pt}p{\isacharprime}{\kern0pt}{\isacharcomma}{\kern0pt}{\isasympi}{\isacharcomma}{\kern0pt}{\isasymsigma}{\isacharparenright}{\kern0pt}{\isachardoublequoteclose}\ \isacommand{by}\isamarkupfalse%
\ blast\isanewline
\ \ \isacommand{moreover}\isamarkupfalse%
\isanewline
\ \ \isacommand{note}\isamarkupfalse%
\ {\isacartoucheopen}M{\isacharunderscore}{\kern0pt}generic{\isacharparenleft}{\kern0pt}G{\isacharparenright}{\kern0pt}{\isacartoucheclose}\isanewline
\ \ \isacommand{ultimately}\isamarkupfalse%
\isanewline
\ \ \isacommand{obtain}\isamarkupfalse%
\ p\ \isakeyword{where}\ {\isachardoublequoteopen}p{\isasympreceq}r{\isachardoublequoteclose}\ {\isachardoublequoteopen}p{\isasymin}G{\isachardoublequoteclose}\ {\isachardoublequoteopen}forces{\isacharunderscore}{\kern0pt}eq{\isacharparenleft}{\kern0pt}p{\isacharcomma}{\kern0pt}{\isasympi}{\isacharcomma}{\kern0pt}{\isasymsigma}{\isacharparenright}{\kern0pt}{\isachardoublequoteclose}\ \isanewline
\ \ \ \ \isacommand{using}\isamarkupfalse%
\ M{\isacharunderscore}{\kern0pt}generic{\isacharunderscore}{\kern0pt}compatD\ strengthening{\isacharunderscore}{\kern0pt}eq{\isacharbrackleft}{\kern0pt}of\ p{\isacharprime}{\kern0pt}{\isacharbrackright}{\kern0pt}\ \isacommand{by}\isamarkupfalse%
\ blast\isanewline
\ \ \isacommand{moreover}\isamarkupfalse%
\ \isanewline
\ \ \isacommand{note}\isamarkupfalse%
\ {\isacartoucheopen}M{\isacharunderscore}{\kern0pt}generic{\isacharparenleft}{\kern0pt}G{\isacharparenright}{\kern0pt}{\isacartoucheclose}\isanewline
\ \ \isacommand{moreover}\isamarkupfalse%
\ \isacommand{from}\isamarkupfalse%
\ calculation\isanewline
\ \ \isacommand{have}\isamarkupfalse%
\ {\isachardoublequoteopen}forces{\isacharunderscore}{\kern0pt}eq{\isacharparenleft}{\kern0pt}q{\isacharcomma}{\kern0pt}{\isasympi}{\isacharcomma}{\kern0pt}{\isasymsigma}{\isacharparenright}{\kern0pt}{\isachardoublequoteclose}\ \isakeyword{if}\ {\isachardoublequoteopen}q{\isasymin}P{\isachardoublequoteclose}\ {\isachardoublequoteopen}q{\isasympreceq}p{\isachardoublequoteclose}\ \isakeyword{for}\ q\isanewline
\ \ \ \ \isacommand{using}\isamarkupfalse%
\ that\ strengthening{\isacharunderscore}{\kern0pt}eq\ \isacommand{by}\isamarkupfalse%
\ blast\isanewline
\ \ \isacommand{moreover}\isamarkupfalse%
\ \isanewline
\ \ \isacommand{note}\isamarkupfalse%
\ {\isacartoucheopen}{\isasymlangle}{\isasymsigma}{\isacharcomma}{\kern0pt}r{\isasymrangle}{\isasymin}{\isasymtau}{\isacartoucheclose}\ {\isacartoucheopen}r{\isasymin}G{\isacartoucheclose}\isanewline
\ \ \isacommand{ultimately}\isamarkupfalse%
\isanewline
\ \ \isacommand{have}\isamarkupfalse%
\ {\isachardoublequoteopen}r{\isasymin}P\ {\isasymand}\ {\isasymlangle}{\isasymsigma}{\isacharcomma}{\kern0pt}r{\isasymrangle}\ {\isasymin}\ {\isasymtau}\ {\isasymand}\ q{\isasympreceq}r\ {\isasymand}\ forces{\isacharunderscore}{\kern0pt}eq{\isacharparenleft}{\kern0pt}q{\isacharcomma}{\kern0pt}{\isasympi}{\isacharcomma}{\kern0pt}{\isasymsigma}{\isacharparenright}{\kern0pt}{\isachardoublequoteclose}\ \isakeyword{if}\ {\isachardoublequoteopen}q{\isasymin}P{\isachardoublequoteclose}\ {\isachardoublequoteopen}q{\isasympreceq}p{\isachardoublequoteclose}\ \isakeyword{for}\ q\isanewline
\ \ \ \ \isacommand{using}\isamarkupfalse%
\ that\ leq{\isacharunderscore}{\kern0pt}transD{\isacharbrackleft}{\kern0pt}of\ {\isacharunderscore}{\kern0pt}\ p\ r{\isacharbrackright}{\kern0pt}\ \isacommand{by}\isamarkupfalse%
\ blast\isanewline
\ \ \isacommand{then}\isamarkupfalse%
\isanewline
\ \ \isacommand{have}\isamarkupfalse%
\ {\isachardoublequoteopen}dense{\isacharunderscore}{\kern0pt}below{\isacharparenleft}{\kern0pt}{\isacharbraceleft}{\kern0pt}q{\isasymin}P{\isachardot}{\kern0pt}\ {\isasymexists}s\ r{\isachardot}{\kern0pt}\ r{\isasymin}P\ {\isasymand}\ {\isasymlangle}s{\isacharcomma}{\kern0pt}r{\isasymrangle}\ {\isasymin}\ {\isasymtau}\ {\isasymand}\ q{\isasympreceq}r\ {\isasymand}\ forces{\isacharunderscore}{\kern0pt}eq{\isacharparenleft}{\kern0pt}q{\isacharcomma}{\kern0pt}{\isasympi}{\isacharcomma}{\kern0pt}s{\isacharparenright}{\kern0pt}{\isacharbraceright}{\kern0pt}{\isacharcomma}{\kern0pt}p{\isacharparenright}{\kern0pt}{\isachardoublequoteclose}\isanewline
\ \ \ \ \isacommand{using}\isamarkupfalse%
\ leq{\isacharunderscore}{\kern0pt}reflI\ \isacommand{by}\isamarkupfalse%
\ blast\isanewline
\ \ \isacommand{moreover}\isamarkupfalse%
\isanewline
\ \ \isacommand{note}\isamarkupfalse%
\ {\isacartoucheopen}M{\isacharunderscore}{\kern0pt}generic{\isacharparenleft}{\kern0pt}G{\isacharparenright}{\kern0pt}{\isacartoucheclose}\ {\isacartoucheopen}p{\isasymin}G{\isacartoucheclose}\isanewline
\ \ \isacommand{moreover}\isamarkupfalse%
\ \isacommand{from}\isamarkupfalse%
\ calculation\isanewline
\ \ \isacommand{have}\isamarkupfalse%
\ {\isachardoublequoteopen}forces{\isacharunderscore}{\kern0pt}mem{\isacharparenleft}{\kern0pt}p{\isacharcomma}{\kern0pt}{\isasympi}{\isacharcomma}{\kern0pt}{\isasymtau}{\isacharparenright}{\kern0pt}{\isachardoublequoteclose}\ \isanewline
\ \ \ \ \isacommand{using}\isamarkupfalse%
\ forces{\isacharunderscore}{\kern0pt}mem{\isacharunderscore}{\kern0pt}iff{\isacharunderscore}{\kern0pt}dense{\isacharunderscore}{\kern0pt}below\ \isacommand{by}\isamarkupfalse%
\ blast\isanewline
\ \ \isacommand{ultimately}\isamarkupfalse%
\isanewline
\ \ \isacommand{show}\isamarkupfalse%
\ {\isacharquery}{\kern0pt}thesis\ \isacommand{by}\isamarkupfalse%
\ blast\isanewline
\isacommand{qed}\isamarkupfalse%
%
\endisatagproof
{\isafoldproof}%
%
\isadelimproof
\isanewline
%
\endisadelimproof
\isanewline
\isacommand{end}\isamarkupfalse%
\ \isanewline
\isanewline
\isacommand{lemma}\isamarkupfalse%
\ Collect{\isacharunderscore}{\kern0pt}forces{\isacharunderscore}{\kern0pt}eq{\isacharunderscore}{\kern0pt}in{\isacharunderscore}{\kern0pt}M{\isacharcolon}{\kern0pt}\isanewline
\ \ \isakeyword{assumes}\ {\isachardoublequoteopen}{\isasymtau}\ {\isasymin}\ M{\isachardoublequoteclose}\ {\isachardoublequoteopen}{\isasymtheta}\ {\isasymin}\ M{\isachardoublequoteclose}\isanewline
\ \ \isakeyword{shows}\ {\isachardoublequoteopen}{\isacharbraceleft}{\kern0pt}p{\isasymin}P{\isachardot}{\kern0pt}\ forces{\isacharunderscore}{\kern0pt}eq{\isacharparenleft}{\kern0pt}p{\isacharcomma}{\kern0pt}{\isasymtau}{\isacharcomma}{\kern0pt}{\isasymtheta}{\isacharparenright}{\kern0pt}{\isacharbraceright}{\kern0pt}\ {\isasymin}\ M{\isachardoublequoteclose}\isanewline
%
\isadelimproof
\ \ %
\endisadelimproof
%
\isatagproof
\isacommand{using}\isamarkupfalse%
\ assms\ Collect{\isacharunderscore}{\kern0pt}in{\isacharunderscore}{\kern0pt}M{\isacharunderscore}{\kern0pt}{\isadigit{4}}p{\isacharbrackleft}{\kern0pt}of\ {\isachardoublequoteopen}forces{\isacharunderscore}{\kern0pt}eq{\isacharunderscore}{\kern0pt}fm{\isacharparenleft}{\kern0pt}{\isadigit{1}}{\isacharcomma}{\kern0pt}{\isadigit{2}}{\isacharcomma}{\kern0pt}{\isadigit{0}}{\isacharcomma}{\kern0pt}{\isadigit{3}}{\isacharcomma}{\kern0pt}{\isadigit{4}}{\isacharparenright}{\kern0pt}{\isachardoublequoteclose}\ P\ leq\ {\isasymtau}\ {\isasymtheta}\ \isanewline
\ \ \ \ \ \ \ \ \ \ \ \ \ \ \ \ \ \ \ \ \ \ \ \ \ \ \ \ \ \ \ \ \ \ {\isachardoublequoteopen}{\isasymlambda}A\ x\ p\ l\ t{\isadigit{1}}\ t{\isadigit{2}}{\isachardot}{\kern0pt}\ is{\isacharunderscore}{\kern0pt}forces{\isacharunderscore}{\kern0pt}eq{\isacharparenleft}{\kern0pt}x{\isacharcomma}{\kern0pt}t{\isadigit{1}}{\isacharcomma}{\kern0pt}t{\isadigit{2}}{\isacharparenright}{\kern0pt}{\isachardoublequoteclose}\isanewline
\ \ \ \ \ \ \ \ \ \ \ \ \ \ \ \ \ \ \ \ \ \ \ \ \ \ \ \ \ \ \ \ \ \ {\isachardoublequoteopen}{\isasymlambda}\ x\ p\ l\ t{\isadigit{1}}\ t{\isadigit{2}}{\isachardot}{\kern0pt}\ forces{\isacharunderscore}{\kern0pt}eq{\isacharparenleft}{\kern0pt}x{\isacharcomma}{\kern0pt}t{\isadigit{1}}{\isacharcomma}{\kern0pt}t{\isadigit{2}}{\isacharparenright}{\kern0pt}{\isachardoublequoteclose}\ P{\isacharbrackright}{\kern0pt}\ \isanewline
\ \ \ \ \ \ \ \ arity{\isacharunderscore}{\kern0pt}forces{\isacharunderscore}{\kern0pt}eq{\isacharunderscore}{\kern0pt}fm\ P{\isacharunderscore}{\kern0pt}in{\isacharunderscore}{\kern0pt}M\ leq{\isacharunderscore}{\kern0pt}in{\isacharunderscore}{\kern0pt}M\ sats{\isacharunderscore}{\kern0pt}forces{\isacharunderscore}{\kern0pt}eq{\isacharunderscore}{\kern0pt}fm\ forces{\isacharunderscore}{\kern0pt}eq{\isacharunderscore}{\kern0pt}abs\ forces{\isacharunderscore}{\kern0pt}eq{\isacharunderscore}{\kern0pt}fm{\isacharunderscore}{\kern0pt}type\ \isanewline
\ \ \isacommand{by}\isamarkupfalse%
\ {\isacharparenleft}{\kern0pt}simp\ add{\isacharcolon}{\kern0pt}\ nat{\isacharunderscore}{\kern0pt}union{\isacharunderscore}{\kern0pt}abs{\isadigit{1}}\ Un{\isacharunderscore}{\kern0pt}commute{\isacharparenright}{\kern0pt}%
\endisatagproof
{\isafoldproof}%
%
\isadelimproof
\isanewline
%
\endisadelimproof
\isanewline
\isacommand{lemma}\isamarkupfalse%
\ IV{\isadigit{2}}{\isadigit{4}}{\isadigit{0}}b{\isacharunderscore}{\kern0pt}eq{\isacharunderscore}{\kern0pt}Collects{\isacharcolon}{\kern0pt}\isanewline
\ \ \isakeyword{assumes}\ {\isachardoublequoteopen}{\isasymtau}\ {\isasymin}\ M{\isachardoublequoteclose}\ {\isachardoublequoteopen}{\isasymtheta}\ {\isasymin}\ M{\isachardoublequoteclose}\isanewline
\ \ \isakeyword{shows}\ {\isachardoublequoteopen}{\isacharbraceleft}{\kern0pt}p{\isasymin}P{\isachardot}{\kern0pt}\ {\isasymexists}{\isasymsigma}{\isasymin}domain{\isacharparenleft}{\kern0pt}{\isasymtau}{\isacharparenright}{\kern0pt}\ {\isasymunion}\ domain{\isacharparenleft}{\kern0pt}{\isasymtheta}{\isacharparenright}{\kern0pt}{\isachardot}{\kern0pt}\ forces{\isacharunderscore}{\kern0pt}mem{\isacharparenleft}{\kern0pt}p{\isacharcomma}{\kern0pt}{\isasymsigma}{\isacharcomma}{\kern0pt}{\isasymtau}{\isacharparenright}{\kern0pt}\ {\isasymand}\ forces{\isacharunderscore}{\kern0pt}nmem{\isacharparenleft}{\kern0pt}p{\isacharcomma}{\kern0pt}{\isasymsigma}{\isacharcomma}{\kern0pt}{\isasymtheta}{\isacharparenright}{\kern0pt}{\isacharbraceright}{\kern0pt}{\isasymin}M{\isachardoublequoteclose}\ \isakeyword{and}\isanewline
\ \ \ \ \ \ \ \ {\isachardoublequoteopen}{\isacharbraceleft}{\kern0pt}p{\isasymin}P{\isachardot}{\kern0pt}\ {\isasymexists}{\isasymsigma}{\isasymin}domain{\isacharparenleft}{\kern0pt}{\isasymtau}{\isacharparenright}{\kern0pt}\ {\isasymunion}\ domain{\isacharparenleft}{\kern0pt}{\isasymtheta}{\isacharparenright}{\kern0pt}{\isachardot}{\kern0pt}\ forces{\isacharunderscore}{\kern0pt}nmem{\isacharparenleft}{\kern0pt}p{\isacharcomma}{\kern0pt}{\isasymsigma}{\isacharcomma}{\kern0pt}{\isasymtau}{\isacharparenright}{\kern0pt}\ {\isasymand}\ forces{\isacharunderscore}{\kern0pt}mem{\isacharparenleft}{\kern0pt}p{\isacharcomma}{\kern0pt}{\isasymsigma}{\isacharcomma}{\kern0pt}{\isasymtheta}{\isacharparenright}{\kern0pt}{\isacharbraceright}{\kern0pt}{\isasymin}M{\isachardoublequoteclose}\isanewline
%
\isadelimproof
%
\endisadelimproof
%
\isatagproof
\isacommand{proof}\isamarkupfalse%
\ {\isacharminus}{\kern0pt}\isanewline
\ \ \isacommand{let}\isamarkupfalse%
\ {\isacharquery}{\kern0pt}rel{\isacharunderscore}{\kern0pt}pred{\isacharequal}{\kern0pt}{\isachardoublequoteopen}{\isasymlambda}M\ x\ a{\isadigit{1}}\ a{\isadigit{2}}\ a{\isadigit{3}}\ a{\isadigit{4}}{\isachardot}{\kern0pt}\ \isanewline
\ \ \ \ \ \ \ \ {\isasymexists}{\isasymsigma}{\isacharbrackleft}{\kern0pt}M{\isacharbrackright}{\kern0pt}{\isachardot}{\kern0pt}\ {\isasymexists}u{\isacharbrackleft}{\kern0pt}M{\isacharbrackright}{\kern0pt}{\isachardot}{\kern0pt}\ {\isasymexists}da{\isadigit{3}}{\isacharbrackleft}{\kern0pt}M{\isacharbrackright}{\kern0pt}{\isachardot}{\kern0pt}\ {\isasymexists}da{\isadigit{4}}{\isacharbrackleft}{\kern0pt}M{\isacharbrackright}{\kern0pt}{\isachardot}{\kern0pt}\ is{\isacharunderscore}{\kern0pt}domain{\isacharparenleft}{\kern0pt}M{\isacharcomma}{\kern0pt}a{\isadigit{3}}{\isacharcomma}{\kern0pt}da{\isadigit{3}}{\isacharparenright}{\kern0pt}\ {\isasymand}\ is{\isacharunderscore}{\kern0pt}domain{\isacharparenleft}{\kern0pt}M{\isacharcomma}{\kern0pt}a{\isadigit{4}}{\isacharcomma}{\kern0pt}da{\isadigit{4}}{\isacharparenright}{\kern0pt}\ {\isasymand}\ \isanewline
\ \ \ \ \ \ \ \ \ \ union{\isacharparenleft}{\kern0pt}M{\isacharcomma}{\kern0pt}da{\isadigit{3}}{\isacharcomma}{\kern0pt}da{\isadigit{4}}{\isacharcomma}{\kern0pt}u{\isacharparenright}{\kern0pt}\ {\isasymand}\ {\isasymsigma}{\isasymin}u\ {\isasymand}\ is{\isacharunderscore}{\kern0pt}forces{\isacharunderscore}{\kern0pt}mem{\isacharprime}{\kern0pt}{\isacharparenleft}{\kern0pt}M{\isacharcomma}{\kern0pt}a{\isadigit{1}}{\isacharcomma}{\kern0pt}a{\isadigit{2}}{\isacharcomma}{\kern0pt}x{\isacharcomma}{\kern0pt}{\isasymsigma}{\isacharcomma}{\kern0pt}a{\isadigit{3}}{\isacharparenright}{\kern0pt}\ {\isasymand}\ \isanewline
\ \ \ \ \ \ \ \ \ \ is{\isacharunderscore}{\kern0pt}forces{\isacharunderscore}{\kern0pt}nmem{\isacharprime}{\kern0pt}{\isacharparenleft}{\kern0pt}M{\isacharcomma}{\kern0pt}a{\isadigit{1}}{\isacharcomma}{\kern0pt}a{\isadigit{2}}{\isacharcomma}{\kern0pt}x{\isacharcomma}{\kern0pt}{\isasymsigma}{\isacharcomma}{\kern0pt}a{\isadigit{4}}{\isacharparenright}{\kern0pt}{\isachardoublequoteclose}\isanewline
\ \ \isacommand{let}\isamarkupfalse%
\ {\isacharquery}{\kern0pt}{\isasymphi}{\isacharequal}{\kern0pt}{\isachardoublequoteopen}Exists{\isacharparenleft}{\kern0pt}Exists{\isacharparenleft}{\kern0pt}Exists{\isacharparenleft}{\kern0pt}Exists{\isacharparenleft}{\kern0pt}And{\isacharparenleft}{\kern0pt}domain{\isacharunderscore}{\kern0pt}fm{\isacharparenleft}{\kern0pt}{\isadigit{7}}{\isacharcomma}{\kern0pt}{\isadigit{1}}{\isacharparenright}{\kern0pt}{\isacharcomma}{\kern0pt}And{\isacharparenleft}{\kern0pt}domain{\isacharunderscore}{\kern0pt}fm{\isacharparenleft}{\kern0pt}{\isadigit{8}}{\isacharcomma}{\kern0pt}{\isadigit{0}}{\isacharparenright}{\kern0pt}{\isacharcomma}{\kern0pt}\isanewline
\ \ \ \ \ \ \ \ \ \ And{\isacharparenleft}{\kern0pt}union{\isacharunderscore}{\kern0pt}fm{\isacharparenleft}{\kern0pt}{\isadigit{1}}{\isacharcomma}{\kern0pt}{\isadigit{0}}{\isacharcomma}{\kern0pt}{\isadigit{2}}{\isacharparenright}{\kern0pt}{\isacharcomma}{\kern0pt}And{\isacharparenleft}{\kern0pt}Member{\isacharparenleft}{\kern0pt}{\isadigit{3}}{\isacharcomma}{\kern0pt}{\isadigit{2}}{\isacharparenright}{\kern0pt}{\isacharcomma}{\kern0pt}And{\isacharparenleft}{\kern0pt}forces{\isacharunderscore}{\kern0pt}mem{\isacharunderscore}{\kern0pt}fm{\isacharparenleft}{\kern0pt}{\isadigit{5}}{\isacharcomma}{\kern0pt}{\isadigit{6}}{\isacharcomma}{\kern0pt}{\isadigit{4}}{\isacharcomma}{\kern0pt}{\isadigit{3}}{\isacharcomma}{\kern0pt}{\isadigit{7}}{\isacharparenright}{\kern0pt}{\isacharcomma}{\kern0pt}\isanewline
\ \ \ \ \ \ \ \ \ \ \ \ \ \ \ \ \ \ \ \ \ \ \ \ \ \ \ \ \ \ forces{\isacharunderscore}{\kern0pt}nmem{\isacharunderscore}{\kern0pt}fm{\isacharparenleft}{\kern0pt}{\isadigit{5}}{\isacharcomma}{\kern0pt}{\isadigit{6}}{\isacharcomma}{\kern0pt}{\isadigit{4}}{\isacharcomma}{\kern0pt}{\isadigit{3}}{\isacharcomma}{\kern0pt}{\isadigit{8}}{\isacharparenright}{\kern0pt}{\isacharparenright}{\kern0pt}{\isacharparenright}{\kern0pt}{\isacharparenright}{\kern0pt}{\isacharparenright}{\kern0pt}{\isacharparenright}{\kern0pt}{\isacharparenright}{\kern0pt}{\isacharparenright}{\kern0pt}{\isacharparenright}{\kern0pt}{\isacharparenright}{\kern0pt}{\isachardoublequoteclose}\ \isanewline
\ \ \isacommand{have}\isamarkupfalse%
\ {\isadigit{1}}{\isacharcolon}{\kern0pt}{\isachardoublequoteopen}{\isasymsigma}{\isasymin}M{\isachardoublequoteclose}\ \isakeyword{if}\ {\isachardoublequoteopen}{\isasymlangle}{\isasymsigma}{\isacharcomma}{\kern0pt}y{\isasymrangle}{\isasymin}{\isasymdelta}{\isachardoublequoteclose}\ {\isachardoublequoteopen}{\isasymdelta}{\isasymin}M{\isachardoublequoteclose}\ \isakeyword{for}\ {\isasymsigma}\ {\isasymdelta}\ y\isanewline
\ \ \ \ \isacommand{using}\isamarkupfalse%
\ that\ pair{\isacharunderscore}{\kern0pt}in{\isacharunderscore}{\kern0pt}M{\isacharunderscore}{\kern0pt}iff\ transitivity{\isacharbrackleft}{\kern0pt}of\ {\isachardoublequoteopen}{\isasymlangle}{\isasymsigma}{\isacharcomma}{\kern0pt}y{\isasymrangle}{\isachardoublequoteclose}\ {\isasymdelta}{\isacharbrackright}{\kern0pt}\ \isacommand{by}\isamarkupfalse%
\ simp\isanewline
\ \ \isacommand{have}\isamarkupfalse%
\ abs{\isadigit{1}}{\isacharcolon}{\kern0pt}{\isachardoublequoteopen}{\isacharquery}{\kern0pt}rel{\isacharunderscore}{\kern0pt}pred{\isacharparenleft}{\kern0pt}{\isacharhash}{\kern0pt}{\isacharhash}{\kern0pt}M{\isacharcomma}{\kern0pt}p{\isacharcomma}{\kern0pt}P{\isacharcomma}{\kern0pt}leq{\isacharcomma}{\kern0pt}{\isasymtau}{\isacharcomma}{\kern0pt}{\isasymtheta}{\isacharparenright}{\kern0pt}\ {\isasymlongleftrightarrow}\ \isanewline
\ \ \ \ \ \ \ \ {\isacharparenleft}{\kern0pt}{\isasymexists}{\isasymsigma}{\isasymin}domain{\isacharparenleft}{\kern0pt}{\isasymtau}{\isacharparenright}{\kern0pt}\ {\isasymunion}\ domain{\isacharparenleft}{\kern0pt}{\isasymtheta}{\isacharparenright}{\kern0pt}{\isachardot}{\kern0pt}\ forces{\isacharunderscore}{\kern0pt}mem{\isacharprime}{\kern0pt}{\isacharparenleft}{\kern0pt}P{\isacharcomma}{\kern0pt}leq{\isacharcomma}{\kern0pt}p{\isacharcomma}{\kern0pt}{\isasymsigma}{\isacharcomma}{\kern0pt}{\isasymtau}{\isacharparenright}{\kern0pt}\ {\isasymand}\ forces{\isacharunderscore}{\kern0pt}nmem{\isacharprime}{\kern0pt}{\isacharparenleft}{\kern0pt}P{\isacharcomma}{\kern0pt}leq{\isacharcomma}{\kern0pt}p{\isacharcomma}{\kern0pt}{\isasymsigma}{\isacharcomma}{\kern0pt}{\isasymtheta}{\isacharparenright}{\kern0pt}{\isacharparenright}{\kern0pt}{\isachardoublequoteclose}\ \isanewline
\ \ \ \ \ \ \ \ \isakeyword{if}\ {\isachardoublequoteopen}p{\isasymin}M{\isachardoublequoteclose}\ \isakeyword{for}\ p\isanewline
\ \ \ \ \isacommand{unfolding}\isamarkupfalse%
\ forces{\isacharunderscore}{\kern0pt}mem{\isacharunderscore}{\kern0pt}def\ forces{\isacharunderscore}{\kern0pt}nmem{\isacharunderscore}{\kern0pt}def\isanewline
\ \ \ \ \isacommand{using}\isamarkupfalse%
\ assms\ that\ forces{\isacharunderscore}{\kern0pt}mem{\isacharprime}{\kern0pt}{\isacharunderscore}{\kern0pt}abs\ forces{\isacharunderscore}{\kern0pt}nmem{\isacharprime}{\kern0pt}{\isacharunderscore}{\kern0pt}abs\ P{\isacharunderscore}{\kern0pt}in{\isacharunderscore}{\kern0pt}M\ leq{\isacharunderscore}{\kern0pt}in{\isacharunderscore}{\kern0pt}M\ \isanewline
\ \ \ \ \ \ domain{\isacharunderscore}{\kern0pt}closed\ Un{\isacharunderscore}{\kern0pt}closed\ \isanewline
\ \ \ \ \isacommand{by}\isamarkupfalse%
\ {\isacharparenleft}{\kern0pt}auto\ simp\ add{\isacharcolon}{\kern0pt}{\isadigit{1}}{\isacharbrackleft}{\kern0pt}of\ {\isacharunderscore}{\kern0pt}\ {\isacharunderscore}{\kern0pt}\ {\isasymtau}{\isacharbrackright}{\kern0pt}\ {\isadigit{1}}{\isacharbrackleft}{\kern0pt}of\ {\isacharunderscore}{\kern0pt}\ {\isacharunderscore}{\kern0pt}\ {\isasymtheta}{\isacharbrackright}{\kern0pt}{\isacharparenright}{\kern0pt}\isanewline
\ \ \isacommand{have}\isamarkupfalse%
\ abs{\isadigit{2}}{\isacharcolon}{\kern0pt}{\isachardoublequoteopen}{\isacharquery}{\kern0pt}rel{\isacharunderscore}{\kern0pt}pred{\isacharparenleft}{\kern0pt}{\isacharhash}{\kern0pt}{\isacharhash}{\kern0pt}M{\isacharcomma}{\kern0pt}p{\isacharcomma}{\kern0pt}P{\isacharcomma}{\kern0pt}leq{\isacharcomma}{\kern0pt}{\isasymtheta}{\isacharcomma}{\kern0pt}{\isasymtau}{\isacharparenright}{\kern0pt}\ {\isasymlongleftrightarrow}\ {\isacharparenleft}{\kern0pt}{\isasymexists}{\isasymsigma}{\isasymin}domain{\isacharparenleft}{\kern0pt}{\isasymtau}{\isacharparenright}{\kern0pt}\ {\isasymunion}\ domain{\isacharparenleft}{\kern0pt}{\isasymtheta}{\isacharparenright}{\kern0pt}{\isachardot}{\kern0pt}\ \isanewline
\ \ \ \ \ \ \ \ forces{\isacharunderscore}{\kern0pt}nmem{\isacharprime}{\kern0pt}{\isacharparenleft}{\kern0pt}P{\isacharcomma}{\kern0pt}leq{\isacharcomma}{\kern0pt}p{\isacharcomma}{\kern0pt}{\isasymsigma}{\isacharcomma}{\kern0pt}{\isasymtau}{\isacharparenright}{\kern0pt}\ {\isasymand}\ forces{\isacharunderscore}{\kern0pt}mem{\isacharprime}{\kern0pt}{\isacharparenleft}{\kern0pt}P{\isacharcomma}{\kern0pt}leq{\isacharcomma}{\kern0pt}p{\isacharcomma}{\kern0pt}{\isasymsigma}{\isacharcomma}{\kern0pt}{\isasymtheta}{\isacharparenright}{\kern0pt}{\isacharparenright}{\kern0pt}{\isachardoublequoteclose}\ \isakeyword{if}\ {\isachardoublequoteopen}p{\isasymin}M{\isachardoublequoteclose}\ \isakeyword{for}\ p\isanewline
\ \ \ \ \isacommand{unfolding}\isamarkupfalse%
\ forces{\isacharunderscore}{\kern0pt}mem{\isacharunderscore}{\kern0pt}def\ forces{\isacharunderscore}{\kern0pt}nmem{\isacharunderscore}{\kern0pt}def\isanewline
\ \ \ \ \isacommand{using}\isamarkupfalse%
\ assms\ that\ forces{\isacharunderscore}{\kern0pt}mem{\isacharprime}{\kern0pt}{\isacharunderscore}{\kern0pt}abs\ forces{\isacharunderscore}{\kern0pt}nmem{\isacharprime}{\kern0pt}{\isacharunderscore}{\kern0pt}abs\ P{\isacharunderscore}{\kern0pt}in{\isacharunderscore}{\kern0pt}M\ leq{\isacharunderscore}{\kern0pt}in{\isacharunderscore}{\kern0pt}M\ \isanewline
\ \ \ \ \ \ domain{\isacharunderscore}{\kern0pt}closed\ Un{\isacharunderscore}{\kern0pt}closed\ \isanewline
\ \ \ \ \isacommand{by}\isamarkupfalse%
\ {\isacharparenleft}{\kern0pt}auto\ simp\ add{\isacharcolon}{\kern0pt}{\isadigit{1}}{\isacharbrackleft}{\kern0pt}of\ {\isacharunderscore}{\kern0pt}\ {\isacharunderscore}{\kern0pt}\ {\isasymtau}{\isacharbrackright}{\kern0pt}\ {\isadigit{1}}{\isacharbrackleft}{\kern0pt}of\ {\isacharunderscore}{\kern0pt}\ {\isacharunderscore}{\kern0pt}\ {\isasymtheta}{\isacharbrackright}{\kern0pt}{\isacharparenright}{\kern0pt}\isanewline
\ \ \isacommand{have}\isamarkupfalse%
\ fsats{\isadigit{1}}{\isacharcolon}{\kern0pt}{\isachardoublequoteopen}{\isacharparenleft}{\kern0pt}M{\isacharcomma}{\kern0pt}{\isacharbrackleft}{\kern0pt}p{\isacharcomma}{\kern0pt}P{\isacharcomma}{\kern0pt}leq{\isacharcomma}{\kern0pt}{\isasymtau}{\isacharcomma}{\kern0pt}{\isasymtheta}{\isacharbrackright}{\kern0pt}\ {\isasymTurnstile}\ {\isacharquery}{\kern0pt}{\isasymphi}{\isacharparenright}{\kern0pt}\ {\isasymlongleftrightarrow}\ {\isacharquery}{\kern0pt}rel{\isacharunderscore}{\kern0pt}pred{\isacharparenleft}{\kern0pt}{\isacharhash}{\kern0pt}{\isacharhash}{\kern0pt}M{\isacharcomma}{\kern0pt}p{\isacharcomma}{\kern0pt}P{\isacharcomma}{\kern0pt}leq{\isacharcomma}{\kern0pt}{\isasymtau}{\isacharcomma}{\kern0pt}{\isasymtheta}{\isacharparenright}{\kern0pt}{\isachardoublequoteclose}\ \isakeyword{if}\ {\isachardoublequoteopen}p{\isasymin}M{\isachardoublequoteclose}\ \isakeyword{for}\ p\isanewline
\ \ \ \ \isacommand{using}\isamarkupfalse%
\ that\ assms\ sats{\isacharunderscore}{\kern0pt}forces{\isacharunderscore}{\kern0pt}mem{\isacharprime}{\kern0pt}{\isacharunderscore}{\kern0pt}fm\ sats{\isacharunderscore}{\kern0pt}forces{\isacharunderscore}{\kern0pt}nmem{\isacharprime}{\kern0pt}{\isacharunderscore}{\kern0pt}fm\ P{\isacharunderscore}{\kern0pt}in{\isacharunderscore}{\kern0pt}M\ leq{\isacharunderscore}{\kern0pt}in{\isacharunderscore}{\kern0pt}M\isanewline
\ \ \ \ \ \ domain{\isacharunderscore}{\kern0pt}closed\ Un{\isacharunderscore}{\kern0pt}closed\ \isacommand{by}\isamarkupfalse%
\ simp\isanewline
\ \ \isacommand{have}\isamarkupfalse%
\ fsats{\isadigit{2}}{\isacharcolon}{\kern0pt}{\isachardoublequoteopen}{\isacharparenleft}{\kern0pt}M{\isacharcomma}{\kern0pt}{\isacharbrackleft}{\kern0pt}p{\isacharcomma}{\kern0pt}P{\isacharcomma}{\kern0pt}leq{\isacharcomma}{\kern0pt}{\isasymtheta}{\isacharcomma}{\kern0pt}{\isasymtau}{\isacharbrackright}{\kern0pt}\ {\isasymTurnstile}\ {\isacharquery}{\kern0pt}{\isasymphi}{\isacharparenright}{\kern0pt}\ {\isasymlongleftrightarrow}\ {\isacharquery}{\kern0pt}rel{\isacharunderscore}{\kern0pt}pred{\isacharparenleft}{\kern0pt}{\isacharhash}{\kern0pt}{\isacharhash}{\kern0pt}M{\isacharcomma}{\kern0pt}p{\isacharcomma}{\kern0pt}P{\isacharcomma}{\kern0pt}leq{\isacharcomma}{\kern0pt}{\isasymtheta}{\isacharcomma}{\kern0pt}{\isasymtau}{\isacharparenright}{\kern0pt}{\isachardoublequoteclose}\ \isakeyword{if}\ {\isachardoublequoteopen}p{\isasymin}M{\isachardoublequoteclose}\ \isakeyword{for}\ p\isanewline
\ \ \ \ \isacommand{using}\isamarkupfalse%
\ that\ assms\ sats{\isacharunderscore}{\kern0pt}forces{\isacharunderscore}{\kern0pt}mem{\isacharprime}{\kern0pt}{\isacharunderscore}{\kern0pt}fm\ sats{\isacharunderscore}{\kern0pt}forces{\isacharunderscore}{\kern0pt}nmem{\isacharprime}{\kern0pt}{\isacharunderscore}{\kern0pt}fm\ P{\isacharunderscore}{\kern0pt}in{\isacharunderscore}{\kern0pt}M\ leq{\isacharunderscore}{\kern0pt}in{\isacharunderscore}{\kern0pt}M\isanewline
\ \ \ \ \ \ domain{\isacharunderscore}{\kern0pt}closed\ Un{\isacharunderscore}{\kern0pt}closed\ \isacommand{by}\isamarkupfalse%
\ simp\isanewline
\ \ \isacommand{have}\isamarkupfalse%
\ fty{\isacharcolon}{\kern0pt}{\isachardoublequoteopen}{\isacharquery}{\kern0pt}{\isasymphi}{\isasymin}formula{\isachardoublequoteclose}\ \isacommand{by}\isamarkupfalse%
\ simp\isanewline
\ \ \isacommand{have}\isamarkupfalse%
\ farit{\isacharcolon}{\kern0pt}{\isachardoublequoteopen}arity{\isacharparenleft}{\kern0pt}{\isacharquery}{\kern0pt}{\isasymphi}{\isacharparenright}{\kern0pt}{\isacharequal}{\kern0pt}{\isadigit{5}}{\isachardoublequoteclose}\isanewline
\ \ \ \ \isacommand{unfolding}\isamarkupfalse%
\ forces{\isacharunderscore}{\kern0pt}nmem{\isacharunderscore}{\kern0pt}fm{\isacharunderscore}{\kern0pt}def\ domain{\isacharunderscore}{\kern0pt}fm{\isacharunderscore}{\kern0pt}def\ pair{\isacharunderscore}{\kern0pt}fm{\isacharunderscore}{\kern0pt}def\ upair{\isacharunderscore}{\kern0pt}fm{\isacharunderscore}{\kern0pt}def\ union{\isacharunderscore}{\kern0pt}fm{\isacharunderscore}{\kern0pt}def\isanewline
\ \ \ \ \isacommand{using}\isamarkupfalse%
\ arity{\isacharunderscore}{\kern0pt}forces{\isacharunderscore}{\kern0pt}mem{\isacharunderscore}{\kern0pt}fm\ \isacommand{by}\isamarkupfalse%
\ {\isacharparenleft}{\kern0pt}simp\ add{\isacharcolon}{\kern0pt}nat{\isacharunderscore}{\kern0pt}simp{\isacharunderscore}{\kern0pt}union\ Un{\isacharunderscore}{\kern0pt}commute{\isacharparenright}{\kern0pt}\isanewline
\ \ \ \ \isacommand{show}\isamarkupfalse%
\ \isanewline
\ \ \ \ {\isachardoublequoteopen}{\isacharbraceleft}{\kern0pt}p\ {\isasymin}\ P\ {\isachardot}{\kern0pt}\ {\isasymexists}{\isasymsigma}{\isasymin}domain{\isacharparenleft}{\kern0pt}{\isasymtau}{\isacharparenright}{\kern0pt}\ {\isasymunion}\ domain{\isacharparenleft}{\kern0pt}{\isasymtheta}{\isacharparenright}{\kern0pt}{\isachardot}{\kern0pt}\ forces{\isacharunderscore}{\kern0pt}mem{\isacharparenleft}{\kern0pt}p{\isacharcomma}{\kern0pt}\ {\isasymsigma}{\isacharcomma}{\kern0pt}\ {\isasymtau}{\isacharparenright}{\kern0pt}\ {\isasymand}\ forces{\isacharunderscore}{\kern0pt}nmem{\isacharparenleft}{\kern0pt}p{\isacharcomma}{\kern0pt}\ {\isasymsigma}{\isacharcomma}{\kern0pt}\ {\isasymtheta}{\isacharparenright}{\kern0pt}{\isacharbraceright}{\kern0pt}\ {\isasymin}\ M{\isachardoublequoteclose}\isanewline
\ \ \ \ \isakeyword{and}\ {\isachardoublequoteopen}{\isacharbraceleft}{\kern0pt}p\ {\isasymin}\ P\ {\isachardot}{\kern0pt}\ {\isasymexists}{\isasymsigma}{\isasymin}domain{\isacharparenleft}{\kern0pt}{\isasymtau}{\isacharparenright}{\kern0pt}\ {\isasymunion}\ domain{\isacharparenleft}{\kern0pt}{\isasymtheta}{\isacharparenright}{\kern0pt}{\isachardot}{\kern0pt}\ forces{\isacharunderscore}{\kern0pt}nmem{\isacharparenleft}{\kern0pt}p{\isacharcomma}{\kern0pt}\ {\isasymsigma}{\isacharcomma}{\kern0pt}\ {\isasymtau}{\isacharparenright}{\kern0pt}\ {\isasymand}\ forces{\isacharunderscore}{\kern0pt}mem{\isacharparenleft}{\kern0pt}p{\isacharcomma}{\kern0pt}\ {\isasymsigma}{\isacharcomma}{\kern0pt}\ {\isasymtheta}{\isacharparenright}{\kern0pt}{\isacharbraceright}{\kern0pt}\ {\isasymin}\ M{\isachardoublequoteclose}\isanewline
\ \ \ \ \isacommand{unfolding}\isamarkupfalse%
\ forces{\isacharunderscore}{\kern0pt}mem{\isacharunderscore}{\kern0pt}def\isanewline
\ \ \ \ \isacommand{using}\isamarkupfalse%
\ abs{\isadigit{1}}\ fty\ fsats{\isadigit{1}}\ farit\ P{\isacharunderscore}{\kern0pt}in{\isacharunderscore}{\kern0pt}M\ leq{\isacharunderscore}{\kern0pt}in{\isacharunderscore}{\kern0pt}M\ assms\ forces{\isacharunderscore}{\kern0pt}nmem\isanewline
\ \ \ \ \ \ \ \ \ \ Collect{\isacharunderscore}{\kern0pt}in{\isacharunderscore}{\kern0pt}M{\isacharunderscore}{\kern0pt}{\isadigit{4}}p{\isacharbrackleft}{\kern0pt}of\ {\isacharquery}{\kern0pt}{\isasymphi}\ {\isacharunderscore}{\kern0pt}\ {\isacharunderscore}{\kern0pt}\ {\isacharunderscore}{\kern0pt}\ {\isacharunderscore}{\kern0pt}\ {\isacharunderscore}{\kern0pt}\ \isanewline
\ \ \ \ \ \ \ \ \ \ {\isachardoublequoteopen}{\isasymlambda}x\ p\ l\ a{\isadigit{1}}\ a{\isadigit{2}}{\isachardot}{\kern0pt}\ {\isacharparenleft}{\kern0pt}{\isasymexists}{\isasymsigma}{\isasymin}domain{\isacharparenleft}{\kern0pt}a{\isadigit{1}}{\isacharparenright}{\kern0pt}\ {\isasymunion}\ domain{\isacharparenleft}{\kern0pt}a{\isadigit{2}}{\isacharparenright}{\kern0pt}{\isachardot}{\kern0pt}\ forces{\isacharunderscore}{\kern0pt}mem{\isacharprime}{\kern0pt}{\isacharparenleft}{\kern0pt}p{\isacharcomma}{\kern0pt}l{\isacharcomma}{\kern0pt}x{\isacharcomma}{\kern0pt}{\isasymsigma}{\isacharcomma}{\kern0pt}a{\isadigit{1}}{\isacharparenright}{\kern0pt}\ {\isasymand}\ \isanewline
\ \ \ \ \ \ \ \ \ \ \ \ \ \ \ \ \ \ \ \ \ \ \ \ \ \ \ \ \ \ \ \ \ \ \ \ \ \ \ \ \ \ \ \ \ \ \ \ \ \ \ \ \ forces{\isacharunderscore}{\kern0pt}nmem{\isacharprime}{\kern0pt}{\isacharparenleft}{\kern0pt}p{\isacharcomma}{\kern0pt}l{\isacharcomma}{\kern0pt}x{\isacharcomma}{\kern0pt}{\isasymsigma}{\isacharcomma}{\kern0pt}a{\isadigit{2}}{\isacharparenright}{\kern0pt}{\isacharparenright}{\kern0pt}{\isachardoublequoteclose}{\isacharbrackright}{\kern0pt}\isanewline
\ \ \ \ \isacommand{using}\isamarkupfalse%
\ abs{\isadigit{2}}\ fty\ fsats{\isadigit{2}}\ farit\ P{\isacharunderscore}{\kern0pt}in{\isacharunderscore}{\kern0pt}M\ leq{\isacharunderscore}{\kern0pt}in{\isacharunderscore}{\kern0pt}M\ assms\ forces{\isacharunderscore}{\kern0pt}nmem\ domain{\isacharunderscore}{\kern0pt}closed\ Un{\isacharunderscore}{\kern0pt}closed\isanewline
\ \ \ \ \ \ \ \ \ \ Collect{\isacharunderscore}{\kern0pt}in{\isacharunderscore}{\kern0pt}M{\isacharunderscore}{\kern0pt}{\isadigit{4}}p{\isacharbrackleft}{\kern0pt}of\ {\isacharquery}{\kern0pt}{\isasymphi}\ P\ leq\ {\isasymtheta}\ {\isasymtau}\ {\isacharquery}{\kern0pt}rel{\isacharunderscore}{\kern0pt}pred\ \isanewline
\ \ \ \ \ \ \ \ \ \ {\isachardoublequoteopen}{\isasymlambda}x\ p\ l\ a{\isadigit{2}}\ a{\isadigit{1}}{\isachardot}{\kern0pt}\ {\isacharparenleft}{\kern0pt}{\isasymexists}{\isasymsigma}{\isasymin}domain{\isacharparenleft}{\kern0pt}a{\isadigit{1}}{\isacharparenright}{\kern0pt}\ {\isasymunion}\ domain{\isacharparenleft}{\kern0pt}a{\isadigit{2}}{\isacharparenright}{\kern0pt}{\isachardot}{\kern0pt}\ forces{\isacharunderscore}{\kern0pt}nmem{\isacharprime}{\kern0pt}{\isacharparenleft}{\kern0pt}p{\isacharcomma}{\kern0pt}l{\isacharcomma}{\kern0pt}x{\isacharcomma}{\kern0pt}{\isasymsigma}{\isacharcomma}{\kern0pt}a{\isadigit{1}}{\isacharparenright}{\kern0pt}\ {\isasymand}\ \isanewline
\ \ \ \ \ \ \ \ \ \ \ \ \ \ \ \ \ \ \ \ \ \ \ \ \ \ \ \ \ \ \ \ \ \ \ \ \ \ \ \ \ \ \ \ \ \ \ \ \ \ \ \ \ forces{\isacharunderscore}{\kern0pt}mem{\isacharprime}{\kern0pt}{\isacharparenleft}{\kern0pt}p{\isacharcomma}{\kern0pt}l{\isacharcomma}{\kern0pt}x{\isacharcomma}{\kern0pt}{\isasymsigma}{\isacharcomma}{\kern0pt}a{\isadigit{2}}{\isacharparenright}{\kern0pt}{\isacharparenright}{\kern0pt}{\isachardoublequoteclose}\ P{\isacharbrackright}{\kern0pt}\ \ \isanewline
\ \ \ \ \isacommand{by}\isamarkupfalse%
\ simp{\isacharunderscore}{\kern0pt}all\isanewline
\isacommand{qed}\isamarkupfalse%
%
\endisatagproof
{\isafoldproof}%
%
\isadelimproof
\isanewline
%
\endisadelimproof
\isanewline
\isanewline
\isacommand{lemma}\isamarkupfalse%
\ IV{\isadigit{2}}{\isadigit{4}}{\isadigit{0}}b{\isacharunderscore}{\kern0pt}eq{\isacharcolon}{\kern0pt}\isanewline
\ \ \isakeyword{assumes}\isanewline
\ \ \ \ {\isachardoublequoteopen}M{\isacharunderscore}{\kern0pt}generic{\isacharparenleft}{\kern0pt}G{\isacharparenright}{\kern0pt}{\isachardoublequoteclose}\ {\isachardoublequoteopen}val{\isacharparenleft}{\kern0pt}G{\isacharcomma}{\kern0pt}{\isasymtau}{\isacharparenright}{\kern0pt}\ {\isacharequal}{\kern0pt}\ val{\isacharparenleft}{\kern0pt}G{\isacharcomma}{\kern0pt}{\isasymtheta}{\isacharparenright}{\kern0pt}{\isachardoublequoteclose}\ {\isachardoublequoteopen}{\isasymtau}{\isasymin}M{\isachardoublequoteclose}\ {\isachardoublequoteopen}{\isasymtheta}{\isasymin}M{\isachardoublequoteclose}\ \isanewline
\ \ \ \ \isakeyword{and}\isanewline
\ \ \ \ IH{\isacharcolon}{\kern0pt}{\isachardoublequoteopen}{\isasymAnd}{\isasymsigma}{\isachardot}{\kern0pt}\ {\isasymsigma}{\isasymin}domain{\isacharparenleft}{\kern0pt}{\isasymtau}{\isacharparenright}{\kern0pt}{\isasymunion}domain{\isacharparenleft}{\kern0pt}{\isasymtheta}{\isacharparenright}{\kern0pt}\ {\isasymLongrightarrow}\ \isanewline
\ \ \ \ \ \ {\isacharparenleft}{\kern0pt}val{\isacharparenleft}{\kern0pt}G{\isacharcomma}{\kern0pt}{\isasymsigma}{\isacharparenright}{\kern0pt}{\isasymin}val{\isacharparenleft}{\kern0pt}G{\isacharcomma}{\kern0pt}{\isasymtau}{\isacharparenright}{\kern0pt}\ {\isasymlongrightarrow}\ {\isacharparenleft}{\kern0pt}{\isasymexists}q{\isasymin}G{\isachardot}{\kern0pt}\ forces{\isacharunderscore}{\kern0pt}mem{\isacharparenleft}{\kern0pt}q{\isacharcomma}{\kern0pt}{\isasymsigma}{\isacharcomma}{\kern0pt}{\isasymtau}{\isacharparenright}{\kern0pt}{\isacharparenright}{\kern0pt}{\isacharparenright}{\kern0pt}\ {\isasymand}\ \isanewline
\ \ \ \ \ \ {\isacharparenleft}{\kern0pt}val{\isacharparenleft}{\kern0pt}G{\isacharcomma}{\kern0pt}{\isasymsigma}{\isacharparenright}{\kern0pt}{\isasymin}val{\isacharparenleft}{\kern0pt}G{\isacharcomma}{\kern0pt}{\isasymtheta}{\isacharparenright}{\kern0pt}\ {\isasymlongrightarrow}\ {\isacharparenleft}{\kern0pt}{\isasymexists}q{\isasymin}G{\isachardot}{\kern0pt}\ forces{\isacharunderscore}{\kern0pt}mem{\isacharparenleft}{\kern0pt}q{\isacharcomma}{\kern0pt}{\isasymsigma}{\isacharcomma}{\kern0pt}{\isasymtheta}{\isacharparenright}{\kern0pt}{\isacharparenright}{\kern0pt}{\isacharparenright}{\kern0pt}{\isachardoublequoteclose}\isanewline
\ \ \ \ \isanewline
\ \ \isakeyword{shows}\isanewline
\ \ \ \ {\isachardoublequoteopen}{\isasymexists}p{\isasymin}G{\isachardot}{\kern0pt}\ forces{\isacharunderscore}{\kern0pt}eq{\isacharparenleft}{\kern0pt}p{\isacharcomma}{\kern0pt}{\isasymtau}{\isacharcomma}{\kern0pt}{\isasymtheta}{\isacharparenright}{\kern0pt}{\isachardoublequoteclose}\isanewline
%
\isadelimproof
%
\endisadelimproof
%
\isatagproof
\isacommand{proof}\isamarkupfalse%
\ {\isacharminus}{\kern0pt}\isanewline
\ \ \isacommand{let}\isamarkupfalse%
\ {\isacharquery}{\kern0pt}D{\isadigit{1}}{\isacharequal}{\kern0pt}{\isachardoublequoteopen}{\isacharbraceleft}{\kern0pt}p{\isasymin}P{\isachardot}{\kern0pt}\ forces{\isacharunderscore}{\kern0pt}eq{\isacharparenleft}{\kern0pt}p{\isacharcomma}{\kern0pt}{\isasymtau}{\isacharcomma}{\kern0pt}{\isasymtheta}{\isacharparenright}{\kern0pt}{\isacharbraceright}{\kern0pt}{\isachardoublequoteclose}\isanewline
\ \ \isacommand{let}\isamarkupfalse%
\ {\isacharquery}{\kern0pt}D{\isadigit{2}}{\isacharequal}{\kern0pt}{\isachardoublequoteopen}{\isacharbraceleft}{\kern0pt}p{\isasymin}P{\isachardot}{\kern0pt}\ {\isasymexists}{\isasymsigma}{\isasymin}domain{\isacharparenleft}{\kern0pt}{\isasymtau}{\isacharparenright}{\kern0pt}\ {\isasymunion}\ domain{\isacharparenleft}{\kern0pt}{\isasymtheta}{\isacharparenright}{\kern0pt}{\isachardot}{\kern0pt}\ forces{\isacharunderscore}{\kern0pt}mem{\isacharparenleft}{\kern0pt}p{\isacharcomma}{\kern0pt}{\isasymsigma}{\isacharcomma}{\kern0pt}{\isasymtau}{\isacharparenright}{\kern0pt}\ {\isasymand}\ forces{\isacharunderscore}{\kern0pt}nmem{\isacharparenleft}{\kern0pt}p{\isacharcomma}{\kern0pt}{\isasymsigma}{\isacharcomma}{\kern0pt}{\isasymtheta}{\isacharparenright}{\kern0pt}{\isacharbraceright}{\kern0pt}{\isachardoublequoteclose}\isanewline
\ \ \isacommand{let}\isamarkupfalse%
\ {\isacharquery}{\kern0pt}D{\isadigit{3}}{\isacharequal}{\kern0pt}{\isachardoublequoteopen}{\isacharbraceleft}{\kern0pt}p{\isasymin}P{\isachardot}{\kern0pt}\ {\isasymexists}{\isasymsigma}{\isasymin}domain{\isacharparenleft}{\kern0pt}{\isasymtau}{\isacharparenright}{\kern0pt}\ {\isasymunion}\ domain{\isacharparenleft}{\kern0pt}{\isasymtheta}{\isacharparenright}{\kern0pt}{\isachardot}{\kern0pt}\ forces{\isacharunderscore}{\kern0pt}nmem{\isacharparenleft}{\kern0pt}p{\isacharcomma}{\kern0pt}{\isasymsigma}{\isacharcomma}{\kern0pt}{\isasymtau}{\isacharparenright}{\kern0pt}\ {\isasymand}\ forces{\isacharunderscore}{\kern0pt}mem{\isacharparenleft}{\kern0pt}p{\isacharcomma}{\kern0pt}{\isasymsigma}{\isacharcomma}{\kern0pt}{\isasymtheta}{\isacharparenright}{\kern0pt}{\isacharbraceright}{\kern0pt}{\isachardoublequoteclose}\isanewline
\ \ \isacommand{let}\isamarkupfalse%
\ {\isacharquery}{\kern0pt}D{\isacharequal}{\kern0pt}{\isachardoublequoteopen}{\isacharquery}{\kern0pt}D{\isadigit{1}}\ {\isasymunion}\ {\isacharquery}{\kern0pt}D{\isadigit{2}}\ {\isasymunion}\ {\isacharquery}{\kern0pt}D{\isadigit{3}}{\isachardoublequoteclose}\isanewline
\ \ \isacommand{note}\isamarkupfalse%
\ assms\isanewline
\ \ \isacommand{moreover}\isamarkupfalse%
\ \isacommand{from}\isamarkupfalse%
\ this\isanewline
\ \ \isacommand{have}\isamarkupfalse%
\ {\isachardoublequoteopen}domain{\isacharparenleft}{\kern0pt}{\isasymtau}{\isacharparenright}{\kern0pt}\ {\isasymunion}\ domain{\isacharparenleft}{\kern0pt}{\isasymtheta}{\isacharparenright}{\kern0pt}{\isasymin}M{\isachardoublequoteclose}\ {\isacharparenleft}{\kern0pt}\isakeyword{is}\ {\isachardoublequoteopen}{\isacharquery}{\kern0pt}B{\isasymin}M{\isachardoublequoteclose}{\isacharparenright}{\kern0pt}\ \isacommand{using}\isamarkupfalse%
\ domain{\isacharunderscore}{\kern0pt}closed\ Un{\isacharunderscore}{\kern0pt}closed\ \isacommand{by}\isamarkupfalse%
\ auto\isanewline
\ \ \isacommand{moreover}\isamarkupfalse%
\ \isacommand{from}\isamarkupfalse%
\ calculation\isanewline
\ \ \isacommand{have}\isamarkupfalse%
\ {\isachardoublequoteopen}{\isacharquery}{\kern0pt}D{\isadigit{2}}{\isasymin}M{\isachardoublequoteclose}\ \isakeyword{and}\ {\isachardoublequoteopen}{\isacharquery}{\kern0pt}D{\isadigit{3}}{\isasymin}M{\isachardoublequoteclose}\ \isacommand{using}\isamarkupfalse%
\ IV{\isadigit{2}}{\isadigit{4}}{\isadigit{0}}b{\isacharunderscore}{\kern0pt}eq{\isacharunderscore}{\kern0pt}Collects\ \isacommand{by}\isamarkupfalse%
\ simp{\isacharunderscore}{\kern0pt}all\isanewline
\ \ \isacommand{ultimately}\isamarkupfalse%
\isanewline
\ \ \isacommand{have}\isamarkupfalse%
\ {\isachardoublequoteopen}{\isacharquery}{\kern0pt}D{\isasymin}M{\isachardoublequoteclose}\ \isacommand{using}\isamarkupfalse%
\ Collect{\isacharunderscore}{\kern0pt}forces{\isacharunderscore}{\kern0pt}eq{\isacharunderscore}{\kern0pt}in{\isacharunderscore}{\kern0pt}M\ Un{\isacharunderscore}{\kern0pt}closed\ \isacommand{by}\isamarkupfalse%
\ auto\isanewline
\ \ \isacommand{moreover}\isamarkupfalse%
\isanewline
\ \ \isacommand{have}\isamarkupfalse%
\ {\isachardoublequoteopen}dense{\isacharparenleft}{\kern0pt}{\isacharquery}{\kern0pt}D{\isacharparenright}{\kern0pt}{\isachardoublequoteclose}\isanewline
\ \ \isacommand{proof}\isamarkupfalse%
\isanewline
\ \ \ \ \isacommand{fix}\isamarkupfalse%
\ p\isanewline
\ \ \ \ \isacommand{assume}\isamarkupfalse%
\ {\isachardoublequoteopen}p{\isasymin}P{\isachardoublequoteclose}\isanewline
\ \ \ \ \isacommand{have}\isamarkupfalse%
\ {\isachardoublequoteopen}{\isasymexists}d{\isasymin}P{\isachardot}{\kern0pt}\ {\isacharparenleft}{\kern0pt}forces{\isacharunderscore}{\kern0pt}eq{\isacharparenleft}{\kern0pt}d{\isacharcomma}{\kern0pt}\ {\isasymtau}{\isacharcomma}{\kern0pt}\ {\isasymtheta}{\isacharparenright}{\kern0pt}\ {\isasymor}\isanewline
\ \ \ \ \ \ \ \ \ \ \ \ {\isacharparenleft}{\kern0pt}{\isasymexists}{\isasymsigma}{\isasymin}domain{\isacharparenleft}{\kern0pt}{\isasymtau}{\isacharparenright}{\kern0pt}\ {\isasymunion}\ domain{\isacharparenleft}{\kern0pt}{\isasymtheta}{\isacharparenright}{\kern0pt}{\isachardot}{\kern0pt}\ forces{\isacharunderscore}{\kern0pt}mem{\isacharparenleft}{\kern0pt}d{\isacharcomma}{\kern0pt}\ {\isasymsigma}{\isacharcomma}{\kern0pt}\ {\isasymtau}{\isacharparenright}{\kern0pt}\ {\isasymand}\ forces{\isacharunderscore}{\kern0pt}nmem{\isacharparenleft}{\kern0pt}d{\isacharcomma}{\kern0pt}\ {\isasymsigma}{\isacharcomma}{\kern0pt}\ {\isasymtheta}{\isacharparenright}{\kern0pt}{\isacharparenright}{\kern0pt}\ {\isasymor}\isanewline
\ \ \ \ \ \ \ \ \ \ \ \ {\isacharparenleft}{\kern0pt}{\isasymexists}{\isasymsigma}{\isasymin}domain{\isacharparenleft}{\kern0pt}{\isasymtau}{\isacharparenright}{\kern0pt}\ {\isasymunion}\ domain{\isacharparenleft}{\kern0pt}{\isasymtheta}{\isacharparenright}{\kern0pt}{\isachardot}{\kern0pt}\ forces{\isacharunderscore}{\kern0pt}nmem{\isacharparenleft}{\kern0pt}d{\isacharcomma}{\kern0pt}\ {\isasymsigma}{\isacharcomma}{\kern0pt}\ {\isasymtau}{\isacharparenright}{\kern0pt}\ {\isasymand}\ forces{\isacharunderscore}{\kern0pt}mem{\isacharparenleft}{\kern0pt}d{\isacharcomma}{\kern0pt}\ {\isasymsigma}{\isacharcomma}{\kern0pt}\ {\isasymtheta}{\isacharparenright}{\kern0pt}{\isacharparenright}{\kern0pt}{\isacharparenright}{\kern0pt}\ {\isasymand}\isanewline
\ \ \ \ \ \ \ \ \ \ \ d\ {\isasympreceq}\ p{\isachardoublequoteclose}\ \isanewline
\ \ \ \ \isacommand{proof}\isamarkupfalse%
\ {\isacharparenleft}{\kern0pt}cases\ {\isachardoublequoteopen}forces{\isacharunderscore}{\kern0pt}eq{\isacharparenleft}{\kern0pt}p{\isacharcomma}{\kern0pt}\ {\isasymtau}{\isacharcomma}{\kern0pt}\ {\isasymtheta}{\isacharparenright}{\kern0pt}{\isachardoublequoteclose}{\isacharparenright}{\kern0pt}\isanewline
\ \ \ \ \ \ \isacommand{case}\isamarkupfalse%
\ True\isanewline
\ \ \ \ \ \ \isacommand{with}\isamarkupfalse%
\ {\isacartoucheopen}p{\isasymin}P{\isacartoucheclose}\ \isanewline
\ \ \ \ \ \ \isacommand{show}\isamarkupfalse%
\ {\isacharquery}{\kern0pt}thesis\ \isacommand{using}\isamarkupfalse%
\ leq{\isacharunderscore}{\kern0pt}reflI\ \isacommand{by}\isamarkupfalse%
\ blast\isanewline
\ \ \ \ \isacommand{next}\isamarkupfalse%
\isanewline
\ \ \ \ \ \ \isacommand{case}\isamarkupfalse%
\ False\isanewline
\ \ \ \ \ \ \isacommand{moreover}\isamarkupfalse%
\ \isacommand{note}\isamarkupfalse%
\ {\isacartoucheopen}p{\isasymin}P{\isacartoucheclose}\isanewline
\ \ \ \ \ \ \isacommand{moreover}\isamarkupfalse%
\ \isacommand{from}\isamarkupfalse%
\ calculation\isanewline
\ \ \ \ \ \ \isacommand{obtain}\isamarkupfalse%
\ {\isasymsigma}\ q\ \isakeyword{where}\ {\isachardoublequoteopen}{\isasymsigma}{\isasymin}domain{\isacharparenleft}{\kern0pt}{\isasymtau}{\isacharparenright}{\kern0pt}{\isasymunion}domain{\isacharparenleft}{\kern0pt}{\isasymtheta}{\isacharparenright}{\kern0pt}{\isachardoublequoteclose}\ {\isachardoublequoteopen}q{\isasymin}P{\isachardoublequoteclose}\ {\isachardoublequoteopen}q{\isasympreceq}p{\isachardoublequoteclose}\isanewline
\ \ \ \ \ \ \ \ {\isachardoublequoteopen}{\isacharparenleft}{\kern0pt}forces{\isacharunderscore}{\kern0pt}mem{\isacharparenleft}{\kern0pt}q{\isacharcomma}{\kern0pt}\ {\isasymsigma}{\isacharcomma}{\kern0pt}\ {\isasymtau}{\isacharparenright}{\kern0pt}\ {\isasymand}\ {\isasymnot}\ forces{\isacharunderscore}{\kern0pt}mem{\isacharparenleft}{\kern0pt}q{\isacharcomma}{\kern0pt}\ {\isasymsigma}{\isacharcomma}{\kern0pt}\ {\isasymtheta}{\isacharparenright}{\kern0pt}{\isacharparenright}{\kern0pt}\ {\isasymor}\isanewline
\ \ \ \ \ \ \ \ \ {\isacharparenleft}{\kern0pt}{\isasymnot}\ forces{\isacharunderscore}{\kern0pt}mem{\isacharparenleft}{\kern0pt}q{\isacharcomma}{\kern0pt}\ {\isasymsigma}{\isacharcomma}{\kern0pt}\ {\isasymtau}{\isacharparenright}{\kern0pt}\ {\isasymand}\ forces{\isacharunderscore}{\kern0pt}mem{\isacharparenleft}{\kern0pt}q{\isacharcomma}{\kern0pt}\ {\isasymsigma}{\isacharcomma}{\kern0pt}\ {\isasymtheta}{\isacharparenright}{\kern0pt}{\isacharparenright}{\kern0pt}{\isachardoublequoteclose}\isanewline
\ \ \ \ \ \ \ \ \isacommand{using}\isamarkupfalse%
\ def{\isacharunderscore}{\kern0pt}forces{\isacharunderscore}{\kern0pt}eq\ \isacommand{by}\isamarkupfalse%
\ blast\isanewline
\ \ \ \ \ \ \isacommand{moreover}\isamarkupfalse%
\ \isacommand{from}\isamarkupfalse%
\ this\isanewline
\ \ \ \ \ \ \isacommand{obtain}\isamarkupfalse%
\ r\ \isakeyword{where}\ {\isachardoublequoteopen}r{\isasympreceq}q{\isachardoublequoteclose}\ {\isachardoublequoteopen}r{\isasymin}P{\isachardoublequoteclose}\isanewline
\ \ \ \ \ \ \ \ {\isachardoublequoteopen}{\isacharparenleft}{\kern0pt}forces{\isacharunderscore}{\kern0pt}mem{\isacharparenleft}{\kern0pt}r{\isacharcomma}{\kern0pt}\ {\isasymsigma}{\isacharcomma}{\kern0pt}\ {\isasymtau}{\isacharparenright}{\kern0pt}\ {\isasymand}\ forces{\isacharunderscore}{\kern0pt}nmem{\isacharparenleft}{\kern0pt}r{\isacharcomma}{\kern0pt}\ {\isasymsigma}{\isacharcomma}{\kern0pt}\ {\isasymtheta}{\isacharparenright}{\kern0pt}{\isacharparenright}{\kern0pt}\ {\isasymor}\isanewline
\ \ \ \ \ \ \ \ \ {\isacharparenleft}{\kern0pt}forces{\isacharunderscore}{\kern0pt}nmem{\isacharparenleft}{\kern0pt}r{\isacharcomma}{\kern0pt}\ {\isasymsigma}{\isacharcomma}{\kern0pt}\ {\isasymtau}{\isacharparenright}{\kern0pt}\ {\isasymand}\ forces{\isacharunderscore}{\kern0pt}mem{\isacharparenleft}{\kern0pt}r{\isacharcomma}{\kern0pt}\ {\isasymsigma}{\isacharcomma}{\kern0pt}\ {\isasymtheta}{\isacharparenright}{\kern0pt}{\isacharparenright}{\kern0pt}{\isachardoublequoteclose}\isanewline
\ \ \ \ \ \ \ \ \isacommand{using}\isamarkupfalse%
\ not{\isacharunderscore}{\kern0pt}forces{\isacharunderscore}{\kern0pt}nmem\ strengthening{\isacharunderscore}{\kern0pt}mem\ \isacommand{by}\isamarkupfalse%
\ blast\isanewline
\ \ \ \ \ \ \isacommand{ultimately}\isamarkupfalse%
\isanewline
\ \ \ \ \ \ \isacommand{show}\isamarkupfalse%
\ {\isacharquery}{\kern0pt}thesis\ \isacommand{using}\isamarkupfalse%
\ leq{\isacharunderscore}{\kern0pt}transD\ \isacommand{by}\isamarkupfalse%
\ blast\isanewline
\ \ \ \ \isacommand{qed}\isamarkupfalse%
\isanewline
\ \ \ \ \isacommand{then}\isamarkupfalse%
\isanewline
\ \ \ \ \isacommand{show}\isamarkupfalse%
\ {\isachardoublequoteopen}{\isasymexists}d{\isasymin}{\isacharquery}{\kern0pt}D{\isadigit{1}}\ {\isasymunion}\ {\isacharquery}{\kern0pt}D{\isadigit{2}}\ {\isasymunion}\ {\isacharquery}{\kern0pt}D{\isadigit{3}}{\isachardot}{\kern0pt}\ d\ {\isasympreceq}\ p{\isachardoublequoteclose}\ \isacommand{by}\isamarkupfalse%
\ blast\isanewline
\ \ \isacommand{qed}\isamarkupfalse%
\isanewline
\ \ \isacommand{moreover}\isamarkupfalse%
\isanewline
\ \ \isacommand{have}\isamarkupfalse%
\ {\isachardoublequoteopen}{\isacharquery}{\kern0pt}D\ {\isasymsubseteq}\ P{\isachardoublequoteclose}\isanewline
\ \ \ \ \isacommand{by}\isamarkupfalse%
\ auto\isanewline
\ \ \isacommand{moreover}\isamarkupfalse%
\isanewline
\ \ \isacommand{note}\isamarkupfalse%
\ {\isacartoucheopen}M{\isacharunderscore}{\kern0pt}generic{\isacharparenleft}{\kern0pt}G{\isacharparenright}{\kern0pt}{\isacartoucheclose}\isanewline
\ \ \isacommand{ultimately}\isamarkupfalse%
\isanewline
\ \ \isacommand{obtain}\isamarkupfalse%
\ p\ \isakeyword{where}\ {\isachardoublequoteopen}p{\isasymin}G{\isachardoublequoteclose}\ {\isachardoublequoteopen}p{\isasymin}{\isacharquery}{\kern0pt}D{\isachardoublequoteclose}\isanewline
\ \ \ \ \isacommand{unfolding}\isamarkupfalse%
\ M{\isacharunderscore}{\kern0pt}generic{\isacharunderscore}{\kern0pt}def\ \isacommand{by}\isamarkupfalse%
\ blast\isanewline
\ \ \isacommand{then}\isamarkupfalse%
\ \isanewline
\ \ \isacommand{consider}\isamarkupfalse%
\ \isanewline
\ \ \ \ {\isacharparenleft}{\kern0pt}{\isadigit{1}}{\isacharparenright}{\kern0pt}\ {\isachardoublequoteopen}forces{\isacharunderscore}{\kern0pt}eq{\isacharparenleft}{\kern0pt}p{\isacharcomma}{\kern0pt}{\isasymtau}{\isacharcomma}{\kern0pt}{\isasymtheta}{\isacharparenright}{\kern0pt}{\isachardoublequoteclose}\ {\isacharbar}{\kern0pt}\ \isanewline
\ \ \ \ {\isacharparenleft}{\kern0pt}{\isadigit{2}}{\isacharparenright}{\kern0pt}\ {\isachardoublequoteopen}{\isasymexists}{\isasymsigma}{\isasymin}domain{\isacharparenleft}{\kern0pt}{\isasymtau}{\isacharparenright}{\kern0pt}\ {\isasymunion}\ domain{\isacharparenleft}{\kern0pt}{\isasymtheta}{\isacharparenright}{\kern0pt}{\isachardot}{\kern0pt}\ forces{\isacharunderscore}{\kern0pt}mem{\isacharparenleft}{\kern0pt}p{\isacharcomma}{\kern0pt}{\isasymsigma}{\isacharcomma}{\kern0pt}{\isasymtau}{\isacharparenright}{\kern0pt}\ {\isasymand}\ forces{\isacharunderscore}{\kern0pt}nmem{\isacharparenleft}{\kern0pt}p{\isacharcomma}{\kern0pt}{\isasymsigma}{\isacharcomma}{\kern0pt}{\isasymtheta}{\isacharparenright}{\kern0pt}{\isachardoublequoteclose}\ {\isacharbar}{\kern0pt}\ \isanewline
\ \ \ \ {\isacharparenleft}{\kern0pt}{\isadigit{3}}{\isacharparenright}{\kern0pt}\ {\isachardoublequoteopen}{\isasymexists}{\isasymsigma}{\isasymin}domain{\isacharparenleft}{\kern0pt}{\isasymtau}{\isacharparenright}{\kern0pt}\ {\isasymunion}\ domain{\isacharparenleft}{\kern0pt}{\isasymtheta}{\isacharparenright}{\kern0pt}{\isachardot}{\kern0pt}\ forces{\isacharunderscore}{\kern0pt}nmem{\isacharparenleft}{\kern0pt}p{\isacharcomma}{\kern0pt}{\isasymsigma}{\isacharcomma}{\kern0pt}{\isasymtau}{\isacharparenright}{\kern0pt}\ {\isasymand}\ forces{\isacharunderscore}{\kern0pt}mem{\isacharparenleft}{\kern0pt}p{\isacharcomma}{\kern0pt}{\isasymsigma}{\isacharcomma}{\kern0pt}{\isasymtheta}{\isacharparenright}{\kern0pt}{\isachardoublequoteclose}\isanewline
\ \ \ \ \isacommand{by}\isamarkupfalse%
\ blast\isanewline
\ \ \isacommand{then}\isamarkupfalse%
\isanewline
\ \ \isacommand{show}\isamarkupfalse%
\ {\isacharquery}{\kern0pt}thesis\isanewline
\ \ \isacommand{proof}\isamarkupfalse%
\ {\isacharparenleft}{\kern0pt}cases{\isacharparenright}{\kern0pt}\isanewline
\ \ \ \ \isacommand{case}\isamarkupfalse%
\ {\isadigit{1}}\isanewline
\ \ \ \ \isacommand{with}\isamarkupfalse%
\ {\isacartoucheopen}p{\isasymin}G{\isacartoucheclose}\ \isanewline
\ \ \ \ \isacommand{show}\isamarkupfalse%
\ {\isacharquery}{\kern0pt}thesis\ \isacommand{by}\isamarkupfalse%
\ blast\isanewline
\ \ \isacommand{next}\isamarkupfalse%
\isanewline
\ \ \ \ \isacommand{case}\isamarkupfalse%
\ {\isadigit{2}}\isanewline
\ \ \ \ \isacommand{then}\isamarkupfalse%
\ \isanewline
\ \ \ \ \isacommand{obtain}\isamarkupfalse%
\ {\isasymsigma}\ \isakeyword{where}\ {\isachardoublequoteopen}{\isasymsigma}{\isasymin}domain{\isacharparenleft}{\kern0pt}{\isasymtau}{\isacharparenright}{\kern0pt}\ {\isasymunion}\ domain{\isacharparenleft}{\kern0pt}{\isasymtheta}{\isacharparenright}{\kern0pt}{\isachardoublequoteclose}\ {\isachardoublequoteopen}forces{\isacharunderscore}{\kern0pt}mem{\isacharparenleft}{\kern0pt}p{\isacharcomma}{\kern0pt}{\isasymsigma}{\isacharcomma}{\kern0pt}{\isasymtau}{\isacharparenright}{\kern0pt}{\isachardoublequoteclose}\ {\isachardoublequoteopen}forces{\isacharunderscore}{\kern0pt}nmem{\isacharparenleft}{\kern0pt}p{\isacharcomma}{\kern0pt}{\isasymsigma}{\isacharcomma}{\kern0pt}{\isasymtheta}{\isacharparenright}{\kern0pt}{\isachardoublequoteclose}\ \isanewline
\ \ \ \ \ \ \isacommand{by}\isamarkupfalse%
\ blast\isanewline
\ \ \ \ \isacommand{moreover}\isamarkupfalse%
\ \isacommand{from}\isamarkupfalse%
\ this\ \isakeyword{and}\ {\isacartoucheopen}p{\isasymin}G{\isacartoucheclose}\ \isakeyword{and}\ assms\isanewline
\ \ \ \ \isacommand{have}\isamarkupfalse%
\ {\isachardoublequoteopen}val{\isacharparenleft}{\kern0pt}G{\isacharcomma}{\kern0pt}{\isasymsigma}{\isacharparenright}{\kern0pt}{\isasymin}val{\isacharparenleft}{\kern0pt}G{\isacharcomma}{\kern0pt}{\isasymtau}{\isacharparenright}{\kern0pt}{\isachardoublequoteclose}\isanewline
\ \ \ \ \ \ \isacommand{using}\isamarkupfalse%
\ IV{\isadigit{2}}{\isadigit{4}}{\isadigit{0}}a{\isacharbrackleft}{\kern0pt}of\ G\ {\isasymsigma}\ {\isasymtau}{\isacharbrackright}{\kern0pt}\ transitivity{\isacharbrackleft}{\kern0pt}OF\ {\isacharunderscore}{\kern0pt}\ domain{\isacharunderscore}{\kern0pt}closed{\isacharbrackleft}{\kern0pt}simplified{\isacharbrackright}{\kern0pt}{\isacharbrackright}{\kern0pt}\ \isacommand{by}\isamarkupfalse%
\ blast\isanewline
\ \ \ \ \isacommand{moreover}\isamarkupfalse%
\ \isacommand{note}\isamarkupfalse%
\ IH\ {\isacartoucheopen}val{\isacharparenleft}{\kern0pt}G{\isacharcomma}{\kern0pt}{\isasymtau}{\isacharparenright}{\kern0pt}\ {\isacharequal}{\kern0pt}\ {\isacharunderscore}{\kern0pt}{\isacartoucheclose}\isanewline
\ \ \ \ \isacommand{ultimately}\isamarkupfalse%
\isanewline
\ \ \ \ \isacommand{obtain}\isamarkupfalse%
\ q\ \isakeyword{where}\ {\isachardoublequoteopen}q{\isasymin}G{\isachardoublequoteclose}\ {\isachardoublequoteopen}forces{\isacharunderscore}{\kern0pt}mem{\isacharparenleft}{\kern0pt}q{\isacharcomma}{\kern0pt}\ {\isasymsigma}{\isacharcomma}{\kern0pt}\ {\isasymtheta}{\isacharparenright}{\kern0pt}{\isachardoublequoteclose}\ \isacommand{by}\isamarkupfalse%
\ auto\isanewline
\ \ \ \ \isacommand{moreover}\isamarkupfalse%
\ \isacommand{from}\isamarkupfalse%
\ this\ \isakeyword{and}\ {\isacartoucheopen}p{\isasymin}G{\isacartoucheclose}\ {\isacartoucheopen}M{\isacharunderscore}{\kern0pt}generic{\isacharparenleft}{\kern0pt}G{\isacharparenright}{\kern0pt}{\isacartoucheclose}\isanewline
\ \ \ \ \isacommand{obtain}\isamarkupfalse%
\ r\ \isakeyword{where}\ {\isachardoublequoteopen}r{\isasymin}P{\isachardoublequoteclose}\ {\isachardoublequoteopen}r{\isasympreceq}p{\isachardoublequoteclose}\ {\isachardoublequoteopen}r{\isasympreceq}q{\isachardoublequoteclose}\isanewline
\ \ \ \ \ \ \isacommand{by}\isamarkupfalse%
\ blast\isanewline
\ \ \ \ \isacommand{moreover}\isamarkupfalse%
\isanewline
\ \ \ \ \isacommand{note}\isamarkupfalse%
\ {\isacartoucheopen}M{\isacharunderscore}{\kern0pt}generic{\isacharparenleft}{\kern0pt}G{\isacharparenright}{\kern0pt}{\isacartoucheclose}\isanewline
\ \ \ \ \isacommand{ultimately}\isamarkupfalse%
\isanewline
\ \ \ \ \isacommand{have}\isamarkupfalse%
\ {\isachardoublequoteopen}forces{\isacharunderscore}{\kern0pt}mem{\isacharparenleft}{\kern0pt}r{\isacharcomma}{\kern0pt}\ {\isasymsigma}{\isacharcomma}{\kern0pt}\ {\isasymtheta}{\isacharparenright}{\kern0pt}{\isachardoublequoteclose}\isanewline
\ \ \ \ \ \ \isacommand{using}\isamarkupfalse%
\ strengthening{\isacharunderscore}{\kern0pt}mem\ \isacommand{by}\isamarkupfalse%
\ blast\isanewline
\ \ \ \ \isacommand{with}\isamarkupfalse%
\ {\isacartoucheopen}r{\isasympreceq}p{\isacartoucheclose}\ {\isacartoucheopen}forces{\isacharunderscore}{\kern0pt}nmem{\isacharparenleft}{\kern0pt}p{\isacharcomma}{\kern0pt}{\isasymsigma}{\isacharcomma}{\kern0pt}{\isasymtheta}{\isacharparenright}{\kern0pt}{\isacartoucheclose}\ {\isacartoucheopen}r{\isasymin}P{\isacartoucheclose}\isanewline
\ \ \ \ \isacommand{have}\isamarkupfalse%
\ {\isachardoublequoteopen}False{\isachardoublequoteclose}\isanewline
\ \ \ \ \ \ \isacommand{unfolding}\isamarkupfalse%
\ forces{\isacharunderscore}{\kern0pt}nmem{\isacharunderscore}{\kern0pt}def\ \isacommand{by}\isamarkupfalse%
\ blast\isanewline
\ \ \ \ \isacommand{then}\isamarkupfalse%
\isanewline
\ \ \ \ \isacommand{show}\isamarkupfalse%
\ {\isacharquery}{\kern0pt}thesis\ \isacommand{by}\isamarkupfalse%
\ simp\isanewline
\ \ \isacommand{next}\isamarkupfalse%
\ \isanewline
\ \ \ \ \isacommand{case}\isamarkupfalse%
\ {\isadigit{3}}\isanewline
\ \ \ \ \isacommand{then}\isamarkupfalse%
\isanewline
\ \ \ \ \isacommand{obtain}\isamarkupfalse%
\ {\isasymsigma}\ \isakeyword{where}\ {\isachardoublequoteopen}{\isasymsigma}{\isasymin}domain{\isacharparenleft}{\kern0pt}{\isasymtau}{\isacharparenright}{\kern0pt}\ {\isasymunion}\ domain{\isacharparenleft}{\kern0pt}{\isasymtheta}{\isacharparenright}{\kern0pt}{\isachardoublequoteclose}\ {\isachardoublequoteopen}forces{\isacharunderscore}{\kern0pt}mem{\isacharparenleft}{\kern0pt}p{\isacharcomma}{\kern0pt}{\isasymsigma}{\isacharcomma}{\kern0pt}{\isasymtheta}{\isacharparenright}{\kern0pt}{\isachardoublequoteclose}\ {\isachardoublequoteopen}forces{\isacharunderscore}{\kern0pt}nmem{\isacharparenleft}{\kern0pt}p{\isacharcomma}{\kern0pt}{\isasymsigma}{\isacharcomma}{\kern0pt}{\isasymtau}{\isacharparenright}{\kern0pt}{\isachardoublequoteclose}\ \isanewline
\ \ \ \ \ \ \isacommand{by}\isamarkupfalse%
\ blast\isanewline
\ \ \ \ \isacommand{moreover}\isamarkupfalse%
\ \isacommand{from}\isamarkupfalse%
\ this\ \isakeyword{and}\ {\isacartoucheopen}p{\isasymin}G{\isacartoucheclose}\ \isakeyword{and}\ assms\isanewline
\ \ \ \ \isacommand{have}\isamarkupfalse%
\ {\isachardoublequoteopen}val{\isacharparenleft}{\kern0pt}G{\isacharcomma}{\kern0pt}{\isasymsigma}{\isacharparenright}{\kern0pt}{\isasymin}val{\isacharparenleft}{\kern0pt}G{\isacharcomma}{\kern0pt}{\isasymtheta}{\isacharparenright}{\kern0pt}{\isachardoublequoteclose}\isanewline
\ \ \ \ \ \ \isacommand{using}\isamarkupfalse%
\ IV{\isadigit{2}}{\isadigit{4}}{\isadigit{0}}a{\isacharbrackleft}{\kern0pt}of\ G\ {\isasymsigma}\ {\isasymtheta}{\isacharbrackright}{\kern0pt}\ transitivity{\isacharbrackleft}{\kern0pt}OF\ {\isacharunderscore}{\kern0pt}\ domain{\isacharunderscore}{\kern0pt}closed{\isacharbrackleft}{\kern0pt}simplified{\isacharbrackright}{\kern0pt}{\isacharbrackright}{\kern0pt}\ \isacommand{by}\isamarkupfalse%
\ blast\isanewline
\ \ \ \ \isacommand{moreover}\isamarkupfalse%
\ \isacommand{note}\isamarkupfalse%
\ IH\ {\isacartoucheopen}val{\isacharparenleft}{\kern0pt}G{\isacharcomma}{\kern0pt}{\isasymtau}{\isacharparenright}{\kern0pt}\ {\isacharequal}{\kern0pt}\ {\isacharunderscore}{\kern0pt}{\isacartoucheclose}\isanewline
\ \ \ \ \isacommand{ultimately}\isamarkupfalse%
\isanewline
\ \ \ \ \isacommand{obtain}\isamarkupfalse%
\ q\ \isakeyword{where}\ {\isachardoublequoteopen}q{\isasymin}G{\isachardoublequoteclose}\ {\isachardoublequoteopen}forces{\isacharunderscore}{\kern0pt}mem{\isacharparenleft}{\kern0pt}q{\isacharcomma}{\kern0pt}\ {\isasymsigma}{\isacharcomma}{\kern0pt}\ {\isasymtau}{\isacharparenright}{\kern0pt}{\isachardoublequoteclose}\ \isacommand{by}\isamarkupfalse%
\ auto\isanewline
\ \ \ \ \isacommand{moreover}\isamarkupfalse%
\ \isacommand{from}\isamarkupfalse%
\ this\ \isakeyword{and}\ {\isacartoucheopen}p{\isasymin}G{\isacartoucheclose}\ {\isacartoucheopen}M{\isacharunderscore}{\kern0pt}generic{\isacharparenleft}{\kern0pt}G{\isacharparenright}{\kern0pt}{\isacartoucheclose}\isanewline
\ \ \ \ \isacommand{obtain}\isamarkupfalse%
\ r\ \isakeyword{where}\ {\isachardoublequoteopen}r{\isasymin}P{\isachardoublequoteclose}\ {\isachardoublequoteopen}r{\isasympreceq}p{\isachardoublequoteclose}\ {\isachardoublequoteopen}r{\isasympreceq}q{\isachardoublequoteclose}\isanewline
\ \ \ \ \ \ \isacommand{by}\isamarkupfalse%
\ blast\isanewline
\ \ \ \ \isacommand{moreover}\isamarkupfalse%
\isanewline
\ \ \ \ \isacommand{note}\isamarkupfalse%
\ {\isacartoucheopen}M{\isacharunderscore}{\kern0pt}generic{\isacharparenleft}{\kern0pt}G{\isacharparenright}{\kern0pt}{\isacartoucheclose}\isanewline
\ \ \ \ \isacommand{ultimately}\isamarkupfalse%
\isanewline
\ \ \ \ \isacommand{have}\isamarkupfalse%
\ {\isachardoublequoteopen}forces{\isacharunderscore}{\kern0pt}mem{\isacharparenleft}{\kern0pt}r{\isacharcomma}{\kern0pt}\ {\isasymsigma}{\isacharcomma}{\kern0pt}\ {\isasymtau}{\isacharparenright}{\kern0pt}{\isachardoublequoteclose}\isanewline
\ \ \ \ \ \ \isacommand{using}\isamarkupfalse%
\ strengthening{\isacharunderscore}{\kern0pt}mem\ \isacommand{by}\isamarkupfalse%
\ blast\isanewline
\ \ \ \ \isacommand{with}\isamarkupfalse%
\ {\isacartoucheopen}r{\isasympreceq}p{\isacartoucheclose}\ {\isacartoucheopen}forces{\isacharunderscore}{\kern0pt}nmem{\isacharparenleft}{\kern0pt}p{\isacharcomma}{\kern0pt}{\isasymsigma}{\isacharcomma}{\kern0pt}{\isasymtau}{\isacharparenright}{\kern0pt}{\isacartoucheclose}\ {\isacartoucheopen}r{\isasymin}P{\isacartoucheclose}\isanewline
\ \ \ \ \isacommand{have}\isamarkupfalse%
\ {\isachardoublequoteopen}False{\isachardoublequoteclose}\isanewline
\ \ \ \ \ \ \isacommand{unfolding}\isamarkupfalse%
\ forces{\isacharunderscore}{\kern0pt}nmem{\isacharunderscore}{\kern0pt}def\ \isacommand{by}\isamarkupfalse%
\ blast\isanewline
\ \ \ \ \isacommand{then}\isamarkupfalse%
\isanewline
\ \ \ \ \isacommand{show}\isamarkupfalse%
\ {\isacharquery}{\kern0pt}thesis\ \isacommand{by}\isamarkupfalse%
\ simp\isanewline
\ \ \isacommand{qed}\isamarkupfalse%
\isanewline
\isacommand{qed}\isamarkupfalse%
%
\endisatagproof
{\isafoldproof}%
%
\isadelimproof
\isanewline
%
\endisadelimproof
\isanewline
\isanewline
\isacommand{lemma}\isamarkupfalse%
\ IV{\isadigit{2}}{\isadigit{4}}{\isadigit{0}}b{\isacharcolon}{\kern0pt}\isanewline
\ \ \isakeyword{assumes}\isanewline
\ \ \ \ {\isachardoublequoteopen}M{\isacharunderscore}{\kern0pt}generic{\isacharparenleft}{\kern0pt}G{\isacharparenright}{\kern0pt}{\isachardoublequoteclose}\isanewline
\ \ \isakeyword{shows}\ \isanewline
\ \ \ \ {\isachardoublequoteopen}{\isacharparenleft}{\kern0pt}{\isasymtau}{\isasymin}M{\isasymlongrightarrow}{\isasymtheta}{\isasymin}M{\isasymlongrightarrow}val{\isacharparenleft}{\kern0pt}G{\isacharcomma}{\kern0pt}{\isasymtau}{\isacharparenright}{\kern0pt}\ {\isacharequal}{\kern0pt}\ val{\isacharparenleft}{\kern0pt}G{\isacharcomma}{\kern0pt}{\isasymtheta}{\isacharparenright}{\kern0pt}\ {\isasymlongrightarrow}\ {\isacharparenleft}{\kern0pt}{\isasymexists}p{\isasymin}G{\isachardot}{\kern0pt}\ forces{\isacharunderscore}{\kern0pt}eq{\isacharparenleft}{\kern0pt}p{\isacharcomma}{\kern0pt}{\isasymtau}{\isacharcomma}{\kern0pt}{\isasymtheta}{\isacharparenright}{\kern0pt}{\isacharparenright}{\kern0pt}{\isacharparenright}{\kern0pt}\ {\isasymand}\isanewline
\ \ \ \ \ {\isacharparenleft}{\kern0pt}{\isasymtau}{\isasymin}M{\isasymlongrightarrow}{\isasymtheta}{\isasymin}M{\isasymlongrightarrow}val{\isacharparenleft}{\kern0pt}G{\isacharcomma}{\kern0pt}{\isasymtau}{\isacharparenright}{\kern0pt}\ {\isasymin}\ val{\isacharparenleft}{\kern0pt}G{\isacharcomma}{\kern0pt}{\isasymtheta}{\isacharparenright}{\kern0pt}\ {\isasymlongrightarrow}\ {\isacharparenleft}{\kern0pt}{\isasymexists}p{\isasymin}G{\isachardot}{\kern0pt}\ forces{\isacharunderscore}{\kern0pt}mem{\isacharparenleft}{\kern0pt}p{\isacharcomma}{\kern0pt}{\isasymtau}{\isacharcomma}{\kern0pt}{\isasymtheta}{\isacharparenright}{\kern0pt}{\isacharparenright}{\kern0pt}{\isacharparenright}{\kern0pt}{\isachardoublequoteclose}\ \isanewline
\ \ \ \ {\isacharparenleft}{\kern0pt}\isakeyword{is}\ {\isachardoublequoteopen}{\isacharquery}{\kern0pt}Q{\isacharparenleft}{\kern0pt}{\isasymtau}{\isacharcomma}{\kern0pt}{\isasymtheta}{\isacharparenright}{\kern0pt}\ {\isasymand}\ {\isacharquery}{\kern0pt}R{\isacharparenleft}{\kern0pt}{\isasymtau}{\isacharcomma}{\kern0pt}{\isasymtheta}{\isacharparenright}{\kern0pt}{\isachardoublequoteclose}{\isacharparenright}{\kern0pt}\isanewline
%
\isadelimproof
%
\endisadelimproof
%
\isatagproof
\isacommand{proof}\isamarkupfalse%
\ {\isacharparenleft}{\kern0pt}intro\ forces{\isacharunderscore}{\kern0pt}induction{\isacharparenright}{\kern0pt}\isanewline
\ \ \isacommand{fix}\isamarkupfalse%
\ {\isasymtau}\ {\isasymtheta}\ p\isanewline
\ \ \isacommand{assume}\isamarkupfalse%
\ {\isachardoublequoteopen}{\isasymsigma}{\isasymin}domain{\isacharparenleft}{\kern0pt}{\isasymtheta}{\isacharparenright}{\kern0pt}\ {\isasymLongrightarrow}\ {\isacharquery}{\kern0pt}Q{\isacharparenleft}{\kern0pt}{\isasymtau}{\isacharcomma}{\kern0pt}\ {\isasymsigma}{\isacharparenright}{\kern0pt}{\isachardoublequoteclose}\ \isakeyword{for}\ {\isasymsigma}\isanewline
\ \ \isacommand{with}\isamarkupfalse%
\ assms\isanewline
\ \ \isacommand{show}\isamarkupfalse%
\ {\isachardoublequoteopen}{\isacharquery}{\kern0pt}R{\isacharparenleft}{\kern0pt}{\isasymtau}{\isacharcomma}{\kern0pt}\ {\isasymtheta}{\isacharparenright}{\kern0pt}{\isachardoublequoteclose}\isanewline
\ \ \ \ \isacommand{using}\isamarkupfalse%
\ IV{\isadigit{2}}{\isadigit{4}}{\isadigit{0}}b{\isacharunderscore}{\kern0pt}mem\ domain{\isacharunderscore}{\kern0pt}closed\ transitivity\ \isacommand{by}\isamarkupfalse%
\ {\isacharparenleft}{\kern0pt}simp{\isacharparenright}{\kern0pt}\isanewline
\isacommand{next}\isamarkupfalse%
\isanewline
\ \ \isacommand{fix}\isamarkupfalse%
\ {\isasymtau}\ {\isasymtheta}\ p\isanewline
\ \ \isacommand{assume}\isamarkupfalse%
\ {\isachardoublequoteopen}{\isasymsigma}\ {\isasymin}\ domain{\isacharparenleft}{\kern0pt}{\isasymtau}{\isacharparenright}{\kern0pt}\ {\isasymunion}\ domain{\isacharparenleft}{\kern0pt}{\isasymtheta}{\isacharparenright}{\kern0pt}\ {\isasymLongrightarrow}\ {\isacharquery}{\kern0pt}R{\isacharparenleft}{\kern0pt}{\isasymsigma}{\isacharcomma}{\kern0pt}{\isasymtau}{\isacharparenright}{\kern0pt}\ {\isasymand}\ {\isacharquery}{\kern0pt}R{\isacharparenleft}{\kern0pt}{\isasymsigma}{\isacharcomma}{\kern0pt}{\isasymtheta}{\isacharparenright}{\kern0pt}{\isachardoublequoteclose}\ \isakeyword{for}\ {\isasymsigma}\isanewline
\ \ \isacommand{moreover}\isamarkupfalse%
\ \isacommand{from}\isamarkupfalse%
\ this\isanewline
\ \ \isacommand{have}\isamarkupfalse%
\ IH{\isacharprime}{\kern0pt}{\isacharcolon}{\kern0pt}{\isachardoublequoteopen}{\isasymtau}{\isasymin}M\ {\isasymLongrightarrow}\ {\isasymtheta}{\isasymin}M\ {\isasymLongrightarrow}\ {\isasymsigma}\ {\isasymin}\ domain{\isacharparenleft}{\kern0pt}{\isasymtau}{\isacharparenright}{\kern0pt}\ {\isasymunion}\ domain{\isacharparenleft}{\kern0pt}{\isasymtheta}{\isacharparenright}{\kern0pt}\ {\isasymLongrightarrow}\isanewline
\ \ \ \ \ \ \ \ \ \ {\isacharparenleft}{\kern0pt}val{\isacharparenleft}{\kern0pt}G{\isacharcomma}{\kern0pt}\ {\isasymsigma}{\isacharparenright}{\kern0pt}\ {\isasymin}\ val{\isacharparenleft}{\kern0pt}G{\isacharcomma}{\kern0pt}\ {\isasymtau}{\isacharparenright}{\kern0pt}\ {\isasymlongrightarrow}\ {\isacharparenleft}{\kern0pt}{\isasymexists}q{\isasymin}G{\isachardot}{\kern0pt}\ forces{\isacharunderscore}{\kern0pt}mem{\isacharparenleft}{\kern0pt}q{\isacharcomma}{\kern0pt}\ {\isasymsigma}{\isacharcomma}{\kern0pt}\ {\isasymtau}{\isacharparenright}{\kern0pt}{\isacharparenright}{\kern0pt}{\isacharparenright}{\kern0pt}\ {\isasymand}\isanewline
\ \ \ \ \ \ \ \ \ \ {\isacharparenleft}{\kern0pt}val{\isacharparenleft}{\kern0pt}G{\isacharcomma}{\kern0pt}\ {\isasymsigma}{\isacharparenright}{\kern0pt}\ {\isasymin}\ val{\isacharparenleft}{\kern0pt}G{\isacharcomma}{\kern0pt}\ {\isasymtheta}{\isacharparenright}{\kern0pt}\ {\isasymlongrightarrow}\ {\isacharparenleft}{\kern0pt}{\isasymexists}q{\isasymin}G{\isachardot}{\kern0pt}\ forces{\isacharunderscore}{\kern0pt}mem{\isacharparenleft}{\kern0pt}q{\isacharcomma}{\kern0pt}\ {\isasymsigma}{\isacharcomma}{\kern0pt}\ {\isasymtheta}{\isacharparenright}{\kern0pt}{\isacharparenright}{\kern0pt}{\isacharparenright}{\kern0pt}{\isachardoublequoteclose}\ \isakeyword{for}\ {\isasymsigma}\ \isanewline
\ \ \ \ \isacommand{by}\isamarkupfalse%
\ {\isacharparenleft}{\kern0pt}blast\ intro{\isacharcolon}{\kern0pt}left{\isacharunderscore}{\kern0pt}in{\isacharunderscore}{\kern0pt}M{\isacharparenright}{\kern0pt}\ \isanewline
\ \ \isacommand{ultimately}\isamarkupfalse%
\isanewline
\ \ \isacommand{show}\isamarkupfalse%
\ {\isachardoublequoteopen}{\isacharquery}{\kern0pt}Q{\isacharparenleft}{\kern0pt}{\isasymtau}{\isacharcomma}{\kern0pt}{\isasymtheta}{\isacharparenright}{\kern0pt}{\isachardoublequoteclose}\isanewline
\ \ \ \ \isacommand{using}\isamarkupfalse%
\ IV{\isadigit{2}}{\isadigit{4}}{\isadigit{0}}b{\isacharunderscore}{\kern0pt}eq{\isacharbrackleft}{\kern0pt}OF\ assms{\isacharparenleft}{\kern0pt}{\isadigit{1}}{\isacharparenright}{\kern0pt}{\isacharbrackright}{\kern0pt}\ \isacommand{by}\isamarkupfalse%
\ {\isacharparenleft}{\kern0pt}auto{\isacharparenright}{\kern0pt}\isanewline
\isacommand{qed}\isamarkupfalse%
%
\endisatagproof
{\isafoldproof}%
%
\isadelimproof
\isanewline
%
\endisadelimproof
\isanewline
\isacommand{lemma}\isamarkupfalse%
\ map{\isacharunderscore}{\kern0pt}val{\isacharunderscore}{\kern0pt}in{\isacharunderscore}{\kern0pt}MG{\isacharcolon}{\kern0pt}\isanewline
\ \ \isakeyword{assumes}\ \isanewline
\ \ \ \ {\isachardoublequoteopen}env{\isasymin}list{\isacharparenleft}{\kern0pt}M{\isacharparenright}{\kern0pt}{\isachardoublequoteclose}\isanewline
\ \ \isakeyword{shows}\ \isanewline
\ \ \ \ {\isachardoublequoteopen}map{\isacharparenleft}{\kern0pt}val{\isacharparenleft}{\kern0pt}G{\isacharparenright}{\kern0pt}{\isacharcomma}{\kern0pt}env{\isacharparenright}{\kern0pt}{\isasymin}list{\isacharparenleft}{\kern0pt}M{\isacharbrackleft}{\kern0pt}G{\isacharbrackright}{\kern0pt}{\isacharparenright}{\kern0pt}{\isachardoublequoteclose}\isanewline
%
\isadelimproof
\ \ %
\endisadelimproof
%
\isatagproof
\isacommand{unfolding}\isamarkupfalse%
\ GenExt{\isacharunderscore}{\kern0pt}def\ \isacommand{using}\isamarkupfalse%
\ assms\ map{\isacharunderscore}{\kern0pt}type{\isadigit{2}}\ \isacommand{by}\isamarkupfalse%
\ simp%
\endisatagproof
{\isafoldproof}%
%
\isadelimproof
\isanewline
%
\endisadelimproof
\isanewline
\isacommand{lemma}\isamarkupfalse%
\ truth{\isacharunderscore}{\kern0pt}lemma{\isacharunderscore}{\kern0pt}mem{\isacharcolon}{\kern0pt}\isanewline
\ \ \isakeyword{assumes}\ \isanewline
\ \ \ \ {\isachardoublequoteopen}env{\isasymin}list{\isacharparenleft}{\kern0pt}M{\isacharparenright}{\kern0pt}{\isachardoublequoteclose}\ {\isachardoublequoteopen}M{\isacharunderscore}{\kern0pt}generic{\isacharparenleft}{\kern0pt}G{\isacharparenright}{\kern0pt}{\isachardoublequoteclose}\isanewline
\ \ \ \ {\isachardoublequoteopen}n{\isasymin}nat{\isachardoublequoteclose}\ {\isachardoublequoteopen}m{\isasymin}nat{\isachardoublequoteclose}\ {\isachardoublequoteopen}n{\isacharless}{\kern0pt}length{\isacharparenleft}{\kern0pt}env{\isacharparenright}{\kern0pt}{\isachardoublequoteclose}\ {\isachardoublequoteopen}m{\isacharless}{\kern0pt}length{\isacharparenleft}{\kern0pt}env{\isacharparenright}{\kern0pt}{\isachardoublequoteclose}\isanewline
\ \ \isakeyword{shows}\ \isanewline
\ \ \ \ {\isachardoublequoteopen}{\isacharparenleft}{\kern0pt}{\isasymexists}p{\isasymin}G{\isachardot}{\kern0pt}\ p\ {\isasymtturnstile}\ Member{\isacharparenleft}{\kern0pt}n{\isacharcomma}{\kern0pt}m{\isacharparenright}{\kern0pt}\ env{\isacharparenright}{\kern0pt}\ \ {\isasymlongleftrightarrow}\ \ M{\isacharbrackleft}{\kern0pt}G{\isacharbrackright}{\kern0pt}{\isacharcomma}{\kern0pt}\ map{\isacharparenleft}{\kern0pt}val{\isacharparenleft}{\kern0pt}G{\isacharparenright}{\kern0pt}{\isacharcomma}{\kern0pt}env{\isacharparenright}{\kern0pt}\ {\isasymTurnstile}\ Member{\isacharparenleft}{\kern0pt}n{\isacharcomma}{\kern0pt}m{\isacharparenright}{\kern0pt}{\isachardoublequoteclose}\isanewline
%
\isadelimproof
\ \ %
\endisadelimproof
%
\isatagproof
\isacommand{using}\isamarkupfalse%
\ assms\ IV{\isadigit{2}}{\isadigit{4}}{\isadigit{0}}a{\isacharbrackleft}{\kern0pt}OF\ assms{\isacharparenleft}{\kern0pt}{\isadigit{2}}{\isacharparenright}{\kern0pt}{\isacharcomma}{\kern0pt}\ of\ {\isachardoublequoteopen}nth{\isacharparenleft}{\kern0pt}n{\isacharcomma}{\kern0pt}env{\isacharparenright}{\kern0pt}{\isachardoublequoteclose}\ {\isachardoublequoteopen}nth{\isacharparenleft}{\kern0pt}m{\isacharcomma}{\kern0pt}env{\isacharparenright}{\kern0pt}{\isachardoublequoteclose}{\isacharbrackright}{\kern0pt}\ \isanewline
\ \ \ \ IV{\isadigit{2}}{\isadigit{4}}{\isadigit{0}}b{\isacharbrackleft}{\kern0pt}OF\ assms{\isacharparenleft}{\kern0pt}{\isadigit{2}}{\isacharparenright}{\kern0pt}{\isacharcomma}{\kern0pt}\ of\ {\isachardoublequoteopen}nth{\isacharparenleft}{\kern0pt}n{\isacharcomma}{\kern0pt}env{\isacharparenright}{\kern0pt}{\isachardoublequoteclose}\ {\isachardoublequoteopen}nth{\isacharparenleft}{\kern0pt}m{\isacharcomma}{\kern0pt}env{\isacharparenright}{\kern0pt}{\isachardoublequoteclose}{\isacharbrackright}{\kern0pt}\ \isanewline
\ \ \ \ P{\isacharunderscore}{\kern0pt}in{\isacharunderscore}{\kern0pt}M\ leq{\isacharunderscore}{\kern0pt}in{\isacharunderscore}{\kern0pt}M\ one{\isacharunderscore}{\kern0pt}in{\isacharunderscore}{\kern0pt}M\ \isanewline
\ \ \ \ Forces{\isacharunderscore}{\kern0pt}Member{\isacharbrackleft}{\kern0pt}of\ {\isacharunderscore}{\kern0pt}\ \ {\isachardoublequoteopen}nth{\isacharparenleft}{\kern0pt}n{\isacharcomma}{\kern0pt}env{\isacharparenright}{\kern0pt}{\isachardoublequoteclose}\ {\isachardoublequoteopen}nth{\isacharparenleft}{\kern0pt}m{\isacharcomma}{\kern0pt}env{\isacharparenright}{\kern0pt}{\isachardoublequoteclose}\ env\ n\ m{\isacharbrackright}{\kern0pt}\ map{\isacharunderscore}{\kern0pt}val{\isacharunderscore}{\kern0pt}in{\isacharunderscore}{\kern0pt}MG\isanewline
\ \ \isacommand{by}\isamarkupfalse%
\ {\isacharparenleft}{\kern0pt}auto{\isacharparenright}{\kern0pt}%
\endisatagproof
{\isafoldproof}%
%
\isadelimproof
\isanewline
%
\endisadelimproof
\isanewline
\isacommand{lemma}\isamarkupfalse%
\ truth{\isacharunderscore}{\kern0pt}lemma{\isacharunderscore}{\kern0pt}eq{\isacharcolon}{\kern0pt}\isanewline
\ \ \isakeyword{assumes}\ \isanewline
\ \ \ \ {\isachardoublequoteopen}env{\isasymin}list{\isacharparenleft}{\kern0pt}M{\isacharparenright}{\kern0pt}{\isachardoublequoteclose}\ {\isachardoublequoteopen}M{\isacharunderscore}{\kern0pt}generic{\isacharparenleft}{\kern0pt}G{\isacharparenright}{\kern0pt}{\isachardoublequoteclose}\ \isanewline
\ \ \ \ {\isachardoublequoteopen}n{\isasymin}nat{\isachardoublequoteclose}\ {\isachardoublequoteopen}m{\isasymin}nat{\isachardoublequoteclose}\ {\isachardoublequoteopen}n{\isacharless}{\kern0pt}length{\isacharparenleft}{\kern0pt}env{\isacharparenright}{\kern0pt}{\isachardoublequoteclose}\ {\isachardoublequoteopen}m{\isacharless}{\kern0pt}length{\isacharparenleft}{\kern0pt}env{\isacharparenright}{\kern0pt}{\isachardoublequoteclose}\isanewline
\ \ \isakeyword{shows}\ \isanewline
\ \ \ \ {\isachardoublequoteopen}{\isacharparenleft}{\kern0pt}{\isasymexists}p{\isasymin}G{\isachardot}{\kern0pt}\ p\ {\isasymtturnstile}\ Equal{\isacharparenleft}{\kern0pt}n{\isacharcomma}{\kern0pt}m{\isacharparenright}{\kern0pt}\ env{\isacharparenright}{\kern0pt}\ \ {\isasymlongleftrightarrow}\ \ M{\isacharbrackleft}{\kern0pt}G{\isacharbrackright}{\kern0pt}{\isacharcomma}{\kern0pt}\ map{\isacharparenleft}{\kern0pt}val{\isacharparenleft}{\kern0pt}G{\isacharparenright}{\kern0pt}{\isacharcomma}{\kern0pt}env{\isacharparenright}{\kern0pt}\ {\isasymTurnstile}\ Equal{\isacharparenleft}{\kern0pt}n{\isacharcomma}{\kern0pt}m{\isacharparenright}{\kern0pt}{\isachardoublequoteclose}\isanewline
%
\isadelimproof
\ \ %
\endisadelimproof
%
\isatagproof
\isacommand{using}\isamarkupfalse%
\ assms\ IV{\isadigit{2}}{\isadigit{4}}{\isadigit{0}}a{\isacharparenleft}{\kern0pt}{\isadigit{1}}{\isacharparenright}{\kern0pt}{\isacharbrackleft}{\kern0pt}OF\ assms{\isacharparenleft}{\kern0pt}{\isadigit{2}}{\isacharparenright}{\kern0pt}{\isacharcomma}{\kern0pt}\ of\ {\isachardoublequoteopen}nth{\isacharparenleft}{\kern0pt}n{\isacharcomma}{\kern0pt}env{\isacharparenright}{\kern0pt}{\isachardoublequoteclose}\ {\isachardoublequoteopen}nth{\isacharparenleft}{\kern0pt}m{\isacharcomma}{\kern0pt}env{\isacharparenright}{\kern0pt}{\isachardoublequoteclose}{\isacharbrackright}{\kern0pt}\ \isanewline
\ \ \ \ IV{\isadigit{2}}{\isadigit{4}}{\isadigit{0}}b{\isacharparenleft}{\kern0pt}{\isadigit{1}}{\isacharparenright}{\kern0pt}{\isacharbrackleft}{\kern0pt}OF\ assms{\isacharparenleft}{\kern0pt}{\isadigit{2}}{\isacharparenright}{\kern0pt}{\isacharcomma}{\kern0pt}\ of\ {\isachardoublequoteopen}nth{\isacharparenleft}{\kern0pt}n{\isacharcomma}{\kern0pt}env{\isacharparenright}{\kern0pt}{\isachardoublequoteclose}\ {\isachardoublequoteopen}nth{\isacharparenleft}{\kern0pt}m{\isacharcomma}{\kern0pt}env{\isacharparenright}{\kern0pt}{\isachardoublequoteclose}{\isacharbrackright}{\kern0pt}\ \isanewline
\ \ \ \ P{\isacharunderscore}{\kern0pt}in{\isacharunderscore}{\kern0pt}M\ leq{\isacharunderscore}{\kern0pt}in{\isacharunderscore}{\kern0pt}M\ one{\isacharunderscore}{\kern0pt}in{\isacharunderscore}{\kern0pt}M\ \isanewline
\ \ \ \ Forces{\isacharunderscore}{\kern0pt}Equal{\isacharbrackleft}{\kern0pt}of\ {\isacharunderscore}{\kern0pt}\ \ {\isachardoublequoteopen}nth{\isacharparenleft}{\kern0pt}n{\isacharcomma}{\kern0pt}env{\isacharparenright}{\kern0pt}{\isachardoublequoteclose}\ {\isachardoublequoteopen}nth{\isacharparenleft}{\kern0pt}m{\isacharcomma}{\kern0pt}env{\isacharparenright}{\kern0pt}{\isachardoublequoteclose}\ env\ n\ m{\isacharbrackright}{\kern0pt}\ map{\isacharunderscore}{\kern0pt}val{\isacharunderscore}{\kern0pt}in{\isacharunderscore}{\kern0pt}MG\isanewline
\ \ \isacommand{by}\isamarkupfalse%
\ {\isacharparenleft}{\kern0pt}auto{\isacharparenright}{\kern0pt}%
\endisatagproof
{\isafoldproof}%
%
\isadelimproof
\isanewline
%
\endisadelimproof
\isanewline
\isacommand{lemma}\isamarkupfalse%
\ arities{\isacharunderscore}{\kern0pt}at{\isacharunderscore}{\kern0pt}aux{\isacharcolon}{\kern0pt}\isanewline
\ \ \isakeyword{assumes}\isanewline
\ \ \ \ {\isachardoublequoteopen}n\ {\isasymin}\ nat{\isachardoublequoteclose}\ {\isachardoublequoteopen}m\ {\isasymin}\ nat{\isachardoublequoteclose}\ {\isachardoublequoteopen}env\ {\isasymin}\ list{\isacharparenleft}{\kern0pt}M{\isacharparenright}{\kern0pt}{\isachardoublequoteclose}\ {\isachardoublequoteopen}succ{\isacharparenleft}{\kern0pt}n{\isacharparenright}{\kern0pt}\ {\isasymunion}\ succ{\isacharparenleft}{\kern0pt}m{\isacharparenright}{\kern0pt}\ {\isasymle}\ length{\isacharparenleft}{\kern0pt}env{\isacharparenright}{\kern0pt}{\isachardoublequoteclose}\isanewline
\ \ \isakeyword{shows}\isanewline
\ \ \ \ {\isachardoublequoteopen}n\ {\isacharless}{\kern0pt}\ length{\isacharparenleft}{\kern0pt}env{\isacharparenright}{\kern0pt}{\isachardoublequoteclose}\ {\isachardoublequoteopen}m\ {\isacharless}{\kern0pt}\ length{\isacharparenleft}{\kern0pt}env{\isacharparenright}{\kern0pt}{\isachardoublequoteclose}\isanewline
%
\isadelimproof
\ \ %
\endisadelimproof
%
\isatagproof
\isacommand{using}\isamarkupfalse%
\ assms\ succ{\isacharunderscore}{\kern0pt}leE{\isacharbrackleft}{\kern0pt}OF\ Un{\isacharunderscore}{\kern0pt}leD{\isadigit{1}}{\isacharcomma}{\kern0pt}\ of\ n\ {\isachardoublequoteopen}succ{\isacharparenleft}{\kern0pt}m{\isacharparenright}{\kern0pt}{\isachardoublequoteclose}\ {\isachardoublequoteopen}length{\isacharparenleft}{\kern0pt}env{\isacharparenright}{\kern0pt}{\isachardoublequoteclose}{\isacharbrackright}{\kern0pt}\ \isanewline
\ \ \ succ{\isacharunderscore}{\kern0pt}leE{\isacharbrackleft}{\kern0pt}OF\ Un{\isacharunderscore}{\kern0pt}leD{\isadigit{2}}{\isacharcomma}{\kern0pt}\ of\ {\isachardoublequoteopen}succ{\isacharparenleft}{\kern0pt}n{\isacharparenright}{\kern0pt}{\isachardoublequoteclose}\ m\ {\isachardoublequoteopen}length{\isacharparenleft}{\kern0pt}env{\isacharparenright}{\kern0pt}{\isachardoublequoteclose}{\isacharbrackright}{\kern0pt}\ \isacommand{by}\isamarkupfalse%
\ auto%
\endisatagproof
{\isafoldproof}%
%
\isadelimproof
%
\endisadelimproof
%
\isadelimdocument
%
\endisadelimdocument
%
\isatagdocument
%
\isamarkupsubsection{The Strenghtening Lemma%
}
\isamarkuptrue%
%
\endisatagdocument
{\isafolddocument}%
%
\isadelimdocument
%
\endisadelimdocument
\isacommand{lemma}\isamarkupfalse%
\ strengthening{\isacharunderscore}{\kern0pt}lemma{\isacharcolon}{\kern0pt}\isanewline
\ \ \isakeyword{assumes}\ \isanewline
\ \ \ \ {\isachardoublequoteopen}p{\isasymin}P{\isachardoublequoteclose}\ {\isachardoublequoteopen}{\isasymphi}{\isasymin}formula{\isachardoublequoteclose}\ {\isachardoublequoteopen}r{\isasymin}P{\isachardoublequoteclose}\ {\isachardoublequoteopen}r{\isasympreceq}p{\isachardoublequoteclose}\isanewline
\ \ \isakeyword{shows}\isanewline
\ \ \ \ {\isachardoublequoteopen}{\isasymAnd}env{\isachardot}{\kern0pt}\ env{\isasymin}list{\isacharparenleft}{\kern0pt}M{\isacharparenright}{\kern0pt}\ {\isasymLongrightarrow}\ arity{\isacharparenleft}{\kern0pt}{\isasymphi}{\isacharparenright}{\kern0pt}{\isasymle}length{\isacharparenleft}{\kern0pt}env{\isacharparenright}{\kern0pt}\ {\isasymLongrightarrow}\ p\ {\isasymtturnstile}\ {\isasymphi}\ env\ {\isasymLongrightarrow}\ r\ {\isasymtturnstile}\ {\isasymphi}\ env{\isachardoublequoteclose}\isanewline
%
\isadelimproof
\ \ %
\endisadelimproof
%
\isatagproof
\isacommand{using}\isamarkupfalse%
\ assms{\isacharparenleft}{\kern0pt}{\isadigit{2}}{\isacharparenright}{\kern0pt}\isanewline
\isacommand{proof}\isamarkupfalse%
\ {\isacharparenleft}{\kern0pt}induct{\isacharparenright}{\kern0pt}\isanewline
\ \ \isacommand{case}\isamarkupfalse%
\ {\isacharparenleft}{\kern0pt}Member\ n\ m{\isacharparenright}{\kern0pt}\isanewline
\ \ \isacommand{then}\isamarkupfalse%
\isanewline
\ \ \isacommand{have}\isamarkupfalse%
\ {\isachardoublequoteopen}n{\isacharless}{\kern0pt}length{\isacharparenleft}{\kern0pt}env{\isacharparenright}{\kern0pt}{\isachardoublequoteclose}\ {\isachardoublequoteopen}m{\isacharless}{\kern0pt}length{\isacharparenleft}{\kern0pt}env{\isacharparenright}{\kern0pt}{\isachardoublequoteclose}\isanewline
\ \ \ \ \isacommand{using}\isamarkupfalse%
\ arities{\isacharunderscore}{\kern0pt}at{\isacharunderscore}{\kern0pt}aux\ \isacommand{by}\isamarkupfalse%
\ simp{\isacharunderscore}{\kern0pt}all\isanewline
\ \ \isacommand{moreover}\isamarkupfalse%
\isanewline
\ \ \isacommand{assume}\isamarkupfalse%
\ {\isachardoublequoteopen}env{\isasymin}list{\isacharparenleft}{\kern0pt}M{\isacharparenright}{\kern0pt}{\isachardoublequoteclose}\isanewline
\ \ \isacommand{moreover}\isamarkupfalse%
\isanewline
\ \ \isacommand{note}\isamarkupfalse%
\ assms\ Member\isanewline
\ \ \isacommand{ultimately}\isamarkupfalse%
\isanewline
\ \ \isacommand{show}\isamarkupfalse%
\ {\isacharquery}{\kern0pt}case\ \isanewline
\ \ \ \ \isacommand{using}\isamarkupfalse%
\ Forces{\isacharunderscore}{\kern0pt}Member{\isacharbrackleft}{\kern0pt}of\ {\isacharunderscore}{\kern0pt}\ {\isachardoublequoteopen}nth{\isacharparenleft}{\kern0pt}n{\isacharcomma}{\kern0pt}env{\isacharparenright}{\kern0pt}{\isachardoublequoteclose}\ {\isachardoublequoteopen}nth{\isacharparenleft}{\kern0pt}m{\isacharcomma}{\kern0pt}env{\isacharparenright}{\kern0pt}{\isachardoublequoteclose}\ env\ n\ m{\isacharbrackright}{\kern0pt}\isanewline
\ \ \ \ \ \ strengthening{\isacharunderscore}{\kern0pt}mem{\isacharbrackleft}{\kern0pt}of\ p\ r\ {\isachardoublequoteopen}nth{\isacharparenleft}{\kern0pt}n{\isacharcomma}{\kern0pt}env{\isacharparenright}{\kern0pt}{\isachardoublequoteclose}\ {\isachardoublequoteopen}nth{\isacharparenleft}{\kern0pt}m{\isacharcomma}{\kern0pt}env{\isacharparenright}{\kern0pt}{\isachardoublequoteclose}{\isacharbrackright}{\kern0pt}\ \isacommand{by}\isamarkupfalse%
\ simp\isanewline
\isacommand{next}\isamarkupfalse%
\isanewline
\ \ \isacommand{case}\isamarkupfalse%
\ {\isacharparenleft}{\kern0pt}Equal\ n\ m{\isacharparenright}{\kern0pt}\isanewline
\ \ \isacommand{then}\isamarkupfalse%
\isanewline
\ \ \isacommand{have}\isamarkupfalse%
\ {\isachardoublequoteopen}n{\isacharless}{\kern0pt}length{\isacharparenleft}{\kern0pt}env{\isacharparenright}{\kern0pt}{\isachardoublequoteclose}\ {\isachardoublequoteopen}m{\isacharless}{\kern0pt}length{\isacharparenleft}{\kern0pt}env{\isacharparenright}{\kern0pt}{\isachardoublequoteclose}\isanewline
\ \ \ \ \isacommand{using}\isamarkupfalse%
\ arities{\isacharunderscore}{\kern0pt}at{\isacharunderscore}{\kern0pt}aux\ \isacommand{by}\isamarkupfalse%
\ simp{\isacharunderscore}{\kern0pt}all\isanewline
\ \ \isacommand{moreover}\isamarkupfalse%
\isanewline
\ \ \isacommand{assume}\isamarkupfalse%
\ {\isachardoublequoteopen}env{\isasymin}list{\isacharparenleft}{\kern0pt}M{\isacharparenright}{\kern0pt}{\isachardoublequoteclose}\isanewline
\ \ \isacommand{moreover}\isamarkupfalse%
\isanewline
\ \ \isacommand{note}\isamarkupfalse%
\ assms\ Equal\isanewline
\ \ \isacommand{ultimately}\isamarkupfalse%
\isanewline
\ \ \isacommand{show}\isamarkupfalse%
\ {\isacharquery}{\kern0pt}case\ \isanewline
\ \ \ \ \isacommand{using}\isamarkupfalse%
\ Forces{\isacharunderscore}{\kern0pt}Equal{\isacharbrackleft}{\kern0pt}of\ {\isacharunderscore}{\kern0pt}\ {\isachardoublequoteopen}nth{\isacharparenleft}{\kern0pt}n{\isacharcomma}{\kern0pt}env{\isacharparenright}{\kern0pt}{\isachardoublequoteclose}\ {\isachardoublequoteopen}nth{\isacharparenleft}{\kern0pt}m{\isacharcomma}{\kern0pt}env{\isacharparenright}{\kern0pt}{\isachardoublequoteclose}\ env\ n\ m{\isacharbrackright}{\kern0pt}\isanewline
\ \ \ \ \ \ strengthening{\isacharunderscore}{\kern0pt}eq{\isacharbrackleft}{\kern0pt}of\ p\ r\ {\isachardoublequoteopen}nth{\isacharparenleft}{\kern0pt}n{\isacharcomma}{\kern0pt}env{\isacharparenright}{\kern0pt}{\isachardoublequoteclose}\ {\isachardoublequoteopen}nth{\isacharparenleft}{\kern0pt}m{\isacharcomma}{\kern0pt}env{\isacharparenright}{\kern0pt}{\isachardoublequoteclose}{\isacharbrackright}{\kern0pt}\ \isacommand{by}\isamarkupfalse%
\ simp\isanewline
\isacommand{next}\isamarkupfalse%
\isanewline
\ \ \isacommand{case}\isamarkupfalse%
\ {\isacharparenleft}{\kern0pt}Nand\ {\isasymphi}\ {\isasympsi}{\isacharparenright}{\kern0pt}\isanewline
\ \ \isacommand{with}\isamarkupfalse%
\ assms\isanewline
\ \ \isacommand{show}\isamarkupfalse%
\ {\isacharquery}{\kern0pt}case\ \isanewline
\ \ \ \ \isacommand{using}\isamarkupfalse%
\ Forces{\isacharunderscore}{\kern0pt}Nand\ transitivity{\isacharbrackleft}{\kern0pt}OF\ {\isacharunderscore}{\kern0pt}\ P{\isacharunderscore}{\kern0pt}in{\isacharunderscore}{\kern0pt}M{\isacharbrackright}{\kern0pt}\ pair{\isacharunderscore}{\kern0pt}in{\isacharunderscore}{\kern0pt}M{\isacharunderscore}{\kern0pt}iff\ \isanewline
\ \ \ \ \ \ transitivity{\isacharbrackleft}{\kern0pt}OF\ {\isacharunderscore}{\kern0pt}\ leq{\isacharunderscore}{\kern0pt}in{\isacharunderscore}{\kern0pt}M{\isacharbrackright}{\kern0pt}\ leq{\isacharunderscore}{\kern0pt}transD\ \isacommand{by}\isamarkupfalse%
\ auto\isanewline
\isacommand{next}\isamarkupfalse%
\isanewline
\ \ \isacommand{case}\isamarkupfalse%
\ {\isacharparenleft}{\kern0pt}Forall\ {\isasymphi}{\isacharparenright}{\kern0pt}\isanewline
\ \ \isacommand{with}\isamarkupfalse%
\ assms\isanewline
\ \ \isacommand{have}\isamarkupfalse%
\ {\isachardoublequoteopen}p\ {\isasymtturnstile}\ {\isasymphi}\ {\isacharparenleft}{\kern0pt}{\isacharbrackleft}{\kern0pt}x{\isacharbrackright}{\kern0pt}\ {\isacharat}{\kern0pt}\ env{\isacharparenright}{\kern0pt}{\isachardoublequoteclose}\ \isakeyword{if}\ {\isachardoublequoteopen}x{\isasymin}M{\isachardoublequoteclose}\ \isakeyword{for}\ x\isanewline
\ \ \ \ \isacommand{using}\isamarkupfalse%
\ that\ Forces{\isacharunderscore}{\kern0pt}Forall\ \isacommand{by}\isamarkupfalse%
\ simp\isanewline
\ \ \isacommand{with}\isamarkupfalse%
\ Forall\ \isanewline
\ \ \isacommand{have}\isamarkupfalse%
\ {\isachardoublequoteopen}r\ {\isasymtturnstile}\ {\isasymphi}\ {\isacharparenleft}{\kern0pt}{\isacharbrackleft}{\kern0pt}x{\isacharbrackright}{\kern0pt}\ {\isacharat}{\kern0pt}\ env{\isacharparenright}{\kern0pt}{\isachardoublequoteclose}\ \isakeyword{if}\ {\isachardoublequoteopen}x{\isasymin}M{\isachardoublequoteclose}\ \isakeyword{for}\ x\isanewline
\ \ \ \ \isacommand{using}\isamarkupfalse%
\ that\ pred{\isacharunderscore}{\kern0pt}le{\isadigit{2}}\ \isacommand{by}\isamarkupfalse%
\ {\isacharparenleft}{\kern0pt}simp{\isacharparenright}{\kern0pt}\isanewline
\ \ \isacommand{with}\isamarkupfalse%
\ assms\ Forall\isanewline
\ \ \isacommand{show}\isamarkupfalse%
\ {\isacharquery}{\kern0pt}case\ \isanewline
\ \ \ \ \isacommand{using}\isamarkupfalse%
\ Forces{\isacharunderscore}{\kern0pt}Forall\ \isacommand{by}\isamarkupfalse%
\ simp\isanewline
\isacommand{qed}\isamarkupfalse%
%
\endisatagproof
{\isafoldproof}%
%
\isadelimproof
%
\endisadelimproof
%
\isadelimdocument
%
\endisadelimdocument
%
\isatagdocument
%
\isamarkupsubsection{The Density Lemma%
}
\isamarkuptrue%
%
\endisatagdocument
{\isafolddocument}%
%
\isadelimdocument
%
\endisadelimdocument
\isacommand{lemma}\isamarkupfalse%
\ arity{\isacharunderscore}{\kern0pt}Nand{\isacharunderscore}{\kern0pt}le{\isacharcolon}{\kern0pt}\ \isanewline
\ \ \isakeyword{assumes}\ {\isachardoublequoteopen}{\isasymphi}\ {\isasymin}\ formula{\isachardoublequoteclose}\ {\isachardoublequoteopen}{\isasympsi}\ {\isasymin}\ formula{\isachardoublequoteclose}\ {\isachardoublequoteopen}arity{\isacharparenleft}{\kern0pt}Nand{\isacharparenleft}{\kern0pt}{\isasymphi}{\isacharcomma}{\kern0pt}\ {\isasympsi}{\isacharparenright}{\kern0pt}{\isacharparenright}{\kern0pt}\ {\isasymle}\ length{\isacharparenleft}{\kern0pt}env{\isacharparenright}{\kern0pt}{\isachardoublequoteclose}\ {\isachardoublequoteopen}env{\isasymin}list{\isacharparenleft}{\kern0pt}A{\isacharparenright}{\kern0pt}{\isachardoublequoteclose}\isanewline
\ \ \isakeyword{shows}\ {\isachardoublequoteopen}arity{\isacharparenleft}{\kern0pt}{\isasymphi}{\isacharparenright}{\kern0pt}\ {\isasymle}\ length{\isacharparenleft}{\kern0pt}env{\isacharparenright}{\kern0pt}{\isachardoublequoteclose}\ {\isachardoublequoteopen}arity{\isacharparenleft}{\kern0pt}{\isasympsi}{\isacharparenright}{\kern0pt}\ {\isasymle}\ length{\isacharparenleft}{\kern0pt}env{\isacharparenright}{\kern0pt}{\isachardoublequoteclose}\isanewline
%
\isadelimproof
\ \ %
\endisadelimproof
%
\isatagproof
\isacommand{using}\isamarkupfalse%
\ assms\ \isanewline
\ \ \isacommand{by}\isamarkupfalse%
\ {\isacharparenleft}{\kern0pt}rule{\isacharunderscore}{\kern0pt}tac\ Un{\isacharunderscore}{\kern0pt}leD{\isadigit{1}}{\isacharcomma}{\kern0pt}\ rule{\isacharunderscore}{\kern0pt}tac\ {\isacharbrackleft}{\kern0pt}{\isadigit{5}}{\isacharbrackright}{\kern0pt}\ Un{\isacharunderscore}{\kern0pt}leD{\isadigit{2}}{\isacharcomma}{\kern0pt}\ auto{\isacharparenright}{\kern0pt}%
\endisatagproof
{\isafoldproof}%
%
\isadelimproof
\isanewline
%
\endisadelimproof
\isanewline
\isacommand{lemma}\isamarkupfalse%
\ dense{\isacharunderscore}{\kern0pt}below{\isacharunderscore}{\kern0pt}imp{\isacharunderscore}{\kern0pt}forces{\isacharcolon}{\kern0pt}\isanewline
\ \ \isakeyword{assumes}\ \isanewline
\ \ \ \ {\isachardoublequoteopen}p{\isasymin}P{\isachardoublequoteclose}\ {\isachardoublequoteopen}{\isasymphi}{\isasymin}formula{\isachardoublequoteclose}\isanewline
\ \ \isakeyword{shows}\isanewline
\ \ \ \ {\isachardoublequoteopen}{\isasymAnd}env{\isachardot}{\kern0pt}\ env{\isasymin}list{\isacharparenleft}{\kern0pt}M{\isacharparenright}{\kern0pt}\ {\isasymLongrightarrow}\ arity{\isacharparenleft}{\kern0pt}{\isasymphi}{\isacharparenright}{\kern0pt}{\isasymle}length{\isacharparenleft}{\kern0pt}env{\isacharparenright}{\kern0pt}\ {\isasymLongrightarrow}\isanewline
\ \ \ \ \ dense{\isacharunderscore}{\kern0pt}below{\isacharparenleft}{\kern0pt}{\isacharbraceleft}{\kern0pt}q{\isasymin}P{\isachardot}{\kern0pt}\ {\isacharparenleft}{\kern0pt}q\ {\isasymtturnstile}\ {\isasymphi}\ env{\isacharparenright}{\kern0pt}{\isacharbraceright}{\kern0pt}{\isacharcomma}{\kern0pt}p{\isacharparenright}{\kern0pt}\ {\isasymLongrightarrow}\ {\isacharparenleft}{\kern0pt}p\ {\isasymtturnstile}\ {\isasymphi}\ env{\isacharparenright}{\kern0pt}{\isachardoublequoteclose}\isanewline
%
\isadelimproof
\ \ %
\endisadelimproof
%
\isatagproof
\isacommand{using}\isamarkupfalse%
\ assms{\isacharparenleft}{\kern0pt}{\isadigit{2}}{\isacharparenright}{\kern0pt}\isanewline
\isacommand{proof}\isamarkupfalse%
\ {\isacharparenleft}{\kern0pt}induct{\isacharparenright}{\kern0pt}\isanewline
\ \ \isacommand{case}\isamarkupfalse%
\ {\isacharparenleft}{\kern0pt}Member\ n\ m{\isacharparenright}{\kern0pt}\isanewline
\ \ \isacommand{then}\isamarkupfalse%
\isanewline
\ \ \isacommand{have}\isamarkupfalse%
\ {\isachardoublequoteopen}n{\isacharless}{\kern0pt}length{\isacharparenleft}{\kern0pt}env{\isacharparenright}{\kern0pt}{\isachardoublequoteclose}\ {\isachardoublequoteopen}m{\isacharless}{\kern0pt}length{\isacharparenleft}{\kern0pt}env{\isacharparenright}{\kern0pt}{\isachardoublequoteclose}\isanewline
\ \ \ \ \isacommand{using}\isamarkupfalse%
\ arities{\isacharunderscore}{\kern0pt}at{\isacharunderscore}{\kern0pt}aux\ \isacommand{by}\isamarkupfalse%
\ simp{\isacharunderscore}{\kern0pt}all\isanewline
\ \ \isacommand{moreover}\isamarkupfalse%
\isanewline
\ \ \isacommand{assume}\isamarkupfalse%
\ {\isachardoublequoteopen}env{\isasymin}list{\isacharparenleft}{\kern0pt}M{\isacharparenright}{\kern0pt}{\isachardoublequoteclose}\isanewline
\ \ \isacommand{moreover}\isamarkupfalse%
\isanewline
\ \ \isacommand{note}\isamarkupfalse%
\ assms\ Member\isanewline
\ \ \isacommand{ultimately}\isamarkupfalse%
\isanewline
\ \ \isacommand{show}\isamarkupfalse%
\ {\isacharquery}{\kern0pt}case\ \isanewline
\ \ \ \ \isacommand{using}\isamarkupfalse%
\ Forces{\isacharunderscore}{\kern0pt}Member{\isacharbrackleft}{\kern0pt}of\ {\isacharunderscore}{\kern0pt}\ {\isachardoublequoteopen}nth{\isacharparenleft}{\kern0pt}n{\isacharcomma}{\kern0pt}env{\isacharparenright}{\kern0pt}{\isachardoublequoteclose}\ {\isachardoublequoteopen}nth{\isacharparenleft}{\kern0pt}m{\isacharcomma}{\kern0pt}env{\isacharparenright}{\kern0pt}{\isachardoublequoteclose}\ env\ n\ m{\isacharbrackright}{\kern0pt}\isanewline
\ \ \ \ \ \ density{\isacharunderscore}{\kern0pt}mem{\isacharbrackleft}{\kern0pt}of\ p\ {\isachardoublequoteopen}nth{\isacharparenleft}{\kern0pt}n{\isacharcomma}{\kern0pt}env{\isacharparenright}{\kern0pt}{\isachardoublequoteclose}\ {\isachardoublequoteopen}nth{\isacharparenleft}{\kern0pt}m{\isacharcomma}{\kern0pt}env{\isacharparenright}{\kern0pt}{\isachardoublequoteclose}{\isacharbrackright}{\kern0pt}\ \isacommand{by}\isamarkupfalse%
\ simp\isanewline
\isacommand{next}\isamarkupfalse%
\isanewline
\ \ \isacommand{case}\isamarkupfalse%
\ {\isacharparenleft}{\kern0pt}Equal\ n\ m{\isacharparenright}{\kern0pt}\isanewline
\ \ \isacommand{then}\isamarkupfalse%
\isanewline
\ \ \isacommand{have}\isamarkupfalse%
\ {\isachardoublequoteopen}n{\isacharless}{\kern0pt}length{\isacharparenleft}{\kern0pt}env{\isacharparenright}{\kern0pt}{\isachardoublequoteclose}\ {\isachardoublequoteopen}m{\isacharless}{\kern0pt}length{\isacharparenleft}{\kern0pt}env{\isacharparenright}{\kern0pt}{\isachardoublequoteclose}\isanewline
\ \ \ \ \isacommand{using}\isamarkupfalse%
\ arities{\isacharunderscore}{\kern0pt}at{\isacharunderscore}{\kern0pt}aux\ \isacommand{by}\isamarkupfalse%
\ simp{\isacharunderscore}{\kern0pt}all\isanewline
\ \ \isacommand{moreover}\isamarkupfalse%
\isanewline
\ \ \isacommand{assume}\isamarkupfalse%
\ {\isachardoublequoteopen}env{\isasymin}list{\isacharparenleft}{\kern0pt}M{\isacharparenright}{\kern0pt}{\isachardoublequoteclose}\isanewline
\ \ \isacommand{moreover}\isamarkupfalse%
\isanewline
\ \ \isacommand{note}\isamarkupfalse%
\ assms\ Equal\isanewline
\ \ \isacommand{ultimately}\isamarkupfalse%
\isanewline
\ \ \isacommand{show}\isamarkupfalse%
\ {\isacharquery}{\kern0pt}case\ \isanewline
\ \ \ \ \isacommand{using}\isamarkupfalse%
\ Forces{\isacharunderscore}{\kern0pt}Equal{\isacharbrackleft}{\kern0pt}of\ {\isacharunderscore}{\kern0pt}\ {\isachardoublequoteopen}nth{\isacharparenleft}{\kern0pt}n{\isacharcomma}{\kern0pt}env{\isacharparenright}{\kern0pt}{\isachardoublequoteclose}\ {\isachardoublequoteopen}nth{\isacharparenleft}{\kern0pt}m{\isacharcomma}{\kern0pt}env{\isacharparenright}{\kern0pt}{\isachardoublequoteclose}\ env\ n\ m{\isacharbrackright}{\kern0pt}\isanewline
\ \ \ \ \ \ density{\isacharunderscore}{\kern0pt}eq{\isacharbrackleft}{\kern0pt}of\ p\ {\isachardoublequoteopen}nth{\isacharparenleft}{\kern0pt}n{\isacharcomma}{\kern0pt}env{\isacharparenright}{\kern0pt}{\isachardoublequoteclose}\ {\isachardoublequoteopen}nth{\isacharparenleft}{\kern0pt}m{\isacharcomma}{\kern0pt}env{\isacharparenright}{\kern0pt}{\isachardoublequoteclose}{\isacharbrackright}{\kern0pt}\ \isacommand{by}\isamarkupfalse%
\ simp\isanewline
\isacommand{next}\isamarkupfalse%
\isanewline
\isacommand{case}\isamarkupfalse%
\ {\isacharparenleft}{\kern0pt}Nand\ {\isasymphi}\ {\isasympsi}{\isacharparenright}{\kern0pt}\isanewline
\ \ \isacommand{{\isacharbraceleft}{\kern0pt}}\isamarkupfalse%
\ \ \isanewline
\ \ \ \ \isacommand{fix}\isamarkupfalse%
\ q\isanewline
\ \ \ \ \isacommand{assume}\isamarkupfalse%
\ {\isachardoublequoteopen}q{\isasymin}M{\isachardoublequoteclose}\ {\isachardoublequoteopen}q{\isasymin}P{\isachardoublequoteclose}\ {\isachardoublequoteopen}q{\isasympreceq}\ p{\isachardoublequoteclose}\ {\isachardoublequoteopen}q\ {\isasymtturnstile}\ {\isasymphi}\ env{\isachardoublequoteclose}\isanewline
\ \ \ \ \isacommand{moreover}\isamarkupfalse%
\ \isanewline
\ \ \ \ \isacommand{note}\isamarkupfalse%
\ Nand\isanewline
\ \ \ \ \isacommand{moreover}\isamarkupfalse%
\ \isacommand{from}\isamarkupfalse%
\ calculation\isanewline
\ \ \ \ \isacommand{obtain}\isamarkupfalse%
\ d\ \isakeyword{where}\ {\isachardoublequoteopen}d{\isasymin}P{\isachardoublequoteclose}\ {\isachardoublequoteopen}d\ {\isasymtturnstile}\ Nand{\isacharparenleft}{\kern0pt}{\isasymphi}{\isacharcomma}{\kern0pt}\ {\isasympsi}{\isacharparenright}{\kern0pt}\ env{\isachardoublequoteclose}\ {\isachardoublequoteopen}d{\isasympreceq}\ q{\isachardoublequoteclose}\isanewline
\ \ \ \ \ \ \isacommand{using}\isamarkupfalse%
\ dense{\isacharunderscore}{\kern0pt}belowI\ \isacommand{by}\isamarkupfalse%
\ auto\isanewline
\ \ \ \ \isacommand{moreover}\isamarkupfalse%
\ \isacommand{from}\isamarkupfalse%
\ calculation\isanewline
\ \ \ \ \isacommand{have}\isamarkupfalse%
\ {\isachardoublequoteopen}{\isasymnot}{\isacharparenleft}{\kern0pt}d{\isasymtturnstile}\ {\isasympsi}\ env{\isacharparenright}{\kern0pt}{\isachardoublequoteclose}\ \isakeyword{if}\ {\isachardoublequoteopen}d\ {\isasymtturnstile}\ {\isasymphi}\ env{\isachardoublequoteclose}\isanewline
\ \ \ \ \ \ \isacommand{using}\isamarkupfalse%
\ that\ Forces{\isacharunderscore}{\kern0pt}Nand\ leq{\isacharunderscore}{\kern0pt}reflI\ transitivity{\isacharbrackleft}{\kern0pt}OF\ {\isacharunderscore}{\kern0pt}\ P{\isacharunderscore}{\kern0pt}in{\isacharunderscore}{\kern0pt}M{\isacharcomma}{\kern0pt}\ of\ d{\isacharbrackright}{\kern0pt}\ \isacommand{by}\isamarkupfalse%
\ auto\isanewline
\ \ \ \ \isacommand{moreover}\isamarkupfalse%
\ \isanewline
\ \ \ \ \isacommand{note}\isamarkupfalse%
\ arity{\isacharunderscore}{\kern0pt}Nand{\isacharunderscore}{\kern0pt}le{\isacharbrackleft}{\kern0pt}of\ {\isasymphi}\ {\isasympsi}{\isacharbrackright}{\kern0pt}\isanewline
\ \ \ \ \isacommand{moreover}\isamarkupfalse%
\ \isacommand{from}\isamarkupfalse%
\ calculation\isanewline
\ \ \ \ \isacommand{have}\isamarkupfalse%
\ {\isachardoublequoteopen}d\ {\isasymtturnstile}\ {\isasymphi}\ env{\isachardoublequoteclose}\ \isanewline
\ \ \ \ \ \ \ \isacommand{using}\isamarkupfalse%
\ strengthening{\isacharunderscore}{\kern0pt}lemma{\isacharbrackleft}{\kern0pt}of\ q\ {\isasymphi}\ d\ env{\isacharbrackright}{\kern0pt}\ Un{\isacharunderscore}{\kern0pt}leD{\isadigit{1}}\ \isacommand{by}\isamarkupfalse%
\ auto\isanewline
\ \ \ \ \isacommand{ultimately}\isamarkupfalse%
\isanewline
\ \ \ \ \isacommand{have}\isamarkupfalse%
\ {\isachardoublequoteopen}{\isasymnot}\ {\isacharparenleft}{\kern0pt}q\ {\isasymtturnstile}\ {\isasympsi}\ env{\isacharparenright}{\kern0pt}{\isachardoublequoteclose}\isanewline
\ \ \ \ \ \ \isacommand{using}\isamarkupfalse%
\ strengthening{\isacharunderscore}{\kern0pt}lemma{\isacharbrackleft}{\kern0pt}of\ q\ {\isasympsi}\ d\ env{\isacharbrackright}{\kern0pt}\ \isacommand{by}\isamarkupfalse%
\ auto\isanewline
\ \ \isacommand{{\isacharbraceright}{\kern0pt}}\isamarkupfalse%
\isanewline
\ \ \isacommand{with}\isamarkupfalse%
\ {\isacartoucheopen}p{\isasymin}P{\isacartoucheclose}\isanewline
\ \ \isacommand{show}\isamarkupfalse%
\ {\isacharquery}{\kern0pt}case\isanewline
\ \ \ \ \isacommand{using}\isamarkupfalse%
\ Forces{\isacharunderscore}{\kern0pt}Nand{\isacharbrackleft}{\kern0pt}symmetric{\isacharcomma}{\kern0pt}\ OF\ {\isacharunderscore}{\kern0pt}\ Nand{\isacharparenleft}{\kern0pt}{\isadigit{5}}{\isacharcomma}{\kern0pt}{\isadigit{1}}{\isacharcomma}{\kern0pt}{\isadigit{3}}{\isacharparenright}{\kern0pt}{\isacharbrackright}{\kern0pt}\ \isacommand{by}\isamarkupfalse%
\ blast\isanewline
\isacommand{next}\isamarkupfalse%
\isanewline
\ \ \isacommand{case}\isamarkupfalse%
\ {\isacharparenleft}{\kern0pt}Forall\ {\isasymphi}{\isacharparenright}{\kern0pt}\isanewline
\ \ \isacommand{have}\isamarkupfalse%
\ {\isachardoublequoteopen}dense{\isacharunderscore}{\kern0pt}below{\isacharparenleft}{\kern0pt}{\isacharbraceleft}{\kern0pt}q{\isasymin}P{\isachardot}{\kern0pt}\ q\ {\isasymtturnstile}\ {\isasymphi}\ {\isacharparenleft}{\kern0pt}{\isacharbrackleft}{\kern0pt}a{\isacharbrackright}{\kern0pt}{\isacharat}{\kern0pt}env{\isacharparenright}{\kern0pt}{\isacharbraceright}{\kern0pt}{\isacharcomma}{\kern0pt}p{\isacharparenright}{\kern0pt}{\isachardoublequoteclose}\ \isakeyword{if}\ {\isachardoublequoteopen}a{\isasymin}M{\isachardoublequoteclose}\ \isakeyword{for}\ a\isanewline
\ \ \isacommand{proof}\isamarkupfalse%
\isanewline
\ \ \ \ \isacommand{fix}\isamarkupfalse%
\ r\isanewline
\ \ \ \ \isacommand{assume}\isamarkupfalse%
\ {\isachardoublequoteopen}r{\isasymin}P{\isachardoublequoteclose}\ {\isachardoublequoteopen}r{\isasympreceq}p{\isachardoublequoteclose}\isanewline
\ \ \ \ \isacommand{with}\isamarkupfalse%
\ {\isacartoucheopen}dense{\isacharunderscore}{\kern0pt}below{\isacharparenleft}{\kern0pt}{\isacharunderscore}{\kern0pt}{\isacharcomma}{\kern0pt}p{\isacharparenright}{\kern0pt}{\isacartoucheclose}\isanewline
\ \ \ \ \isacommand{obtain}\isamarkupfalse%
\ q\ \isakeyword{where}\ {\isachardoublequoteopen}q{\isasymin}P{\isachardoublequoteclose}\ {\isachardoublequoteopen}q{\isasympreceq}r{\isachardoublequoteclose}\ {\isachardoublequoteopen}q\ {\isasymtturnstile}\ Forall{\isacharparenleft}{\kern0pt}{\isasymphi}{\isacharparenright}{\kern0pt}\ env{\isachardoublequoteclose}\isanewline
\ \ \ \ \ \ \isacommand{by}\isamarkupfalse%
\ blast\isanewline
\ \ \ \ \isacommand{moreover}\isamarkupfalse%
\isanewline
\ \ \ \ \isacommand{note}\isamarkupfalse%
\ Forall\ {\isacartoucheopen}a{\isasymin}M{\isacartoucheclose}\isanewline
\ \ \ \ \isacommand{moreover}\isamarkupfalse%
\ \isacommand{from}\isamarkupfalse%
\ calculation\isanewline
\ \ \ \ \isacommand{have}\isamarkupfalse%
\ {\isachardoublequoteopen}q\ {\isasymtturnstile}\ {\isasymphi}\ {\isacharparenleft}{\kern0pt}{\isacharbrackleft}{\kern0pt}a{\isacharbrackright}{\kern0pt}{\isacharat}{\kern0pt}env{\isacharparenright}{\kern0pt}{\isachardoublequoteclose}\isanewline
\ \ \ \ \ \ \isacommand{using}\isamarkupfalse%
\ Forces{\isacharunderscore}{\kern0pt}Forall\ \isacommand{by}\isamarkupfalse%
\ simp\isanewline
\ \ \ \ \isacommand{ultimately}\isamarkupfalse%
\isanewline
\ \ \ \ \isacommand{show}\isamarkupfalse%
\ {\isachardoublequoteopen}{\isasymexists}d\ {\isasymin}\ {\isacharbraceleft}{\kern0pt}q{\isasymin}P{\isachardot}{\kern0pt}\ q\ {\isasymtturnstile}\ {\isasymphi}\ {\isacharparenleft}{\kern0pt}{\isacharbrackleft}{\kern0pt}a{\isacharbrackright}{\kern0pt}{\isacharat}{\kern0pt}env{\isacharparenright}{\kern0pt}{\isacharbraceright}{\kern0pt}{\isachardot}{\kern0pt}\ d\ {\isasymin}\ P\ {\isasymand}\ d{\isasympreceq}r{\isachardoublequoteclose}\isanewline
\ \ \ \ \ \ \isacommand{by}\isamarkupfalse%
\ auto\isanewline
\ \ \isacommand{qed}\isamarkupfalse%
\isanewline
\ \ \isacommand{moreover}\isamarkupfalse%
\ \isanewline
\ \ \isacommand{note}\isamarkupfalse%
\ Forall{\isacharparenleft}{\kern0pt}{\isadigit{2}}{\isacharparenright}{\kern0pt}{\isacharbrackleft}{\kern0pt}of\ {\isachardoublequoteopen}Cons{\isacharparenleft}{\kern0pt}{\isacharunderscore}{\kern0pt}{\isacharcomma}{\kern0pt}env{\isacharparenright}{\kern0pt}{\isachardoublequoteclose}{\isacharbrackright}{\kern0pt}\ Forall{\isacharparenleft}{\kern0pt}{\isadigit{1}}{\isacharcomma}{\kern0pt}{\isadigit{3}}{\isacharminus}{\kern0pt}{\isadigit{5}}{\isacharparenright}{\kern0pt}\isanewline
\ \ \isacommand{ultimately}\isamarkupfalse%
\isanewline
\ \ \isacommand{have}\isamarkupfalse%
\ {\isachardoublequoteopen}p\ {\isasymtturnstile}\ {\isasymphi}\ {\isacharparenleft}{\kern0pt}{\isacharbrackleft}{\kern0pt}a{\isacharbrackright}{\kern0pt}{\isacharat}{\kern0pt}env{\isacharparenright}{\kern0pt}{\isachardoublequoteclose}\ \isakeyword{if}\ {\isachardoublequoteopen}a{\isasymin}M{\isachardoublequoteclose}\ \isakeyword{for}\ a\isanewline
\ \ \ \ \isacommand{using}\isamarkupfalse%
\ that\ pred{\isacharunderscore}{\kern0pt}le{\isadigit{2}}\ \isacommand{by}\isamarkupfalse%
\ simp\isanewline
\ \ \isacommand{with}\isamarkupfalse%
\ assms\ Forall\isanewline
\ \ \isacommand{show}\isamarkupfalse%
\ {\isacharquery}{\kern0pt}case\ \isacommand{using}\isamarkupfalse%
\ Forces{\isacharunderscore}{\kern0pt}Forall\ \isacommand{by}\isamarkupfalse%
\ simp\isanewline
\isacommand{qed}\isamarkupfalse%
%
\endisatagproof
{\isafoldproof}%
%
\isadelimproof
\isanewline
%
\endisadelimproof
\isanewline
\isacommand{lemma}\isamarkupfalse%
\ density{\isacharunderscore}{\kern0pt}lemma{\isacharcolon}{\kern0pt}\isanewline
\ \ \isakeyword{assumes}\isanewline
\ \ \ \ {\isachardoublequoteopen}p{\isasymin}P{\isachardoublequoteclose}\ {\isachardoublequoteopen}{\isasymphi}{\isasymin}formula{\isachardoublequoteclose}\ {\isachardoublequoteopen}env{\isasymin}list{\isacharparenleft}{\kern0pt}M{\isacharparenright}{\kern0pt}{\isachardoublequoteclose}\ {\isachardoublequoteopen}arity{\isacharparenleft}{\kern0pt}{\isasymphi}{\isacharparenright}{\kern0pt}{\isasymle}length{\isacharparenleft}{\kern0pt}env{\isacharparenright}{\kern0pt}{\isachardoublequoteclose}\isanewline
\ \ \isakeyword{shows}\isanewline
\ \ \ \ {\isachardoublequoteopen}p\ {\isasymtturnstile}\ {\isasymphi}\ env\ \ \ {\isasymlongleftrightarrow}\ \ \ dense{\isacharunderscore}{\kern0pt}below{\isacharparenleft}{\kern0pt}{\isacharbraceleft}{\kern0pt}q{\isasymin}P{\isachardot}{\kern0pt}\ {\isacharparenleft}{\kern0pt}q\ {\isasymtturnstile}\ {\isasymphi}\ env{\isacharparenright}{\kern0pt}{\isacharbraceright}{\kern0pt}{\isacharcomma}{\kern0pt}p{\isacharparenright}{\kern0pt}{\isachardoublequoteclose}\isanewline
%
\isadelimproof
%
\endisadelimproof
%
\isatagproof
\isacommand{proof}\isamarkupfalse%
\isanewline
\ \ \isacommand{assume}\isamarkupfalse%
\ {\isachardoublequoteopen}dense{\isacharunderscore}{\kern0pt}below{\isacharparenleft}{\kern0pt}{\isacharbraceleft}{\kern0pt}q{\isasymin}P{\isachardot}{\kern0pt}\ {\isacharparenleft}{\kern0pt}q\ {\isasymtturnstile}\ {\isasymphi}\ env{\isacharparenright}{\kern0pt}{\isacharbraceright}{\kern0pt}{\isacharcomma}{\kern0pt}p{\isacharparenright}{\kern0pt}{\isachardoublequoteclose}\isanewline
\ \ \isacommand{with}\isamarkupfalse%
\ assms\isanewline
\ \ \isacommand{show}\isamarkupfalse%
\ \ {\isachardoublequoteopen}{\isacharparenleft}{\kern0pt}p\ {\isasymtturnstile}\ {\isasymphi}\ env{\isacharparenright}{\kern0pt}{\isachardoublequoteclose}\isanewline
\ \ \ \ \isacommand{using}\isamarkupfalse%
\ dense{\isacharunderscore}{\kern0pt}below{\isacharunderscore}{\kern0pt}imp{\isacharunderscore}{\kern0pt}forces\ \isacommand{by}\isamarkupfalse%
\ simp\isanewline
\isacommand{next}\isamarkupfalse%
\isanewline
\ \ \isacommand{assume}\isamarkupfalse%
\ {\isachardoublequoteopen}p\ {\isasymtturnstile}\ {\isasymphi}\ env{\isachardoublequoteclose}\isanewline
\ \ \isacommand{with}\isamarkupfalse%
\ assms\isanewline
\ \ \isacommand{show}\isamarkupfalse%
\ {\isachardoublequoteopen}dense{\isacharunderscore}{\kern0pt}below{\isacharparenleft}{\kern0pt}{\isacharbraceleft}{\kern0pt}q{\isasymin}P{\isachardot}{\kern0pt}\ q\ {\isasymtturnstile}\ {\isasymphi}\ env{\isacharbraceright}{\kern0pt}{\isacharcomma}{\kern0pt}p{\isacharparenright}{\kern0pt}{\isachardoublequoteclose}\isanewline
\ \ \ \ \isacommand{using}\isamarkupfalse%
\ strengthening{\isacharunderscore}{\kern0pt}lemma\ leq{\isacharunderscore}{\kern0pt}reflI\ \isacommand{by}\isamarkupfalse%
\ auto\isanewline
\isacommand{qed}\isamarkupfalse%
%
\endisatagproof
{\isafoldproof}%
%
\isadelimproof
%
\endisadelimproof
%
\isadelimdocument
%
\endisadelimdocument
%
\isatagdocument
%
\isamarkupsubsection{The Truth Lemma%
}
\isamarkuptrue%
%
\endisatagdocument
{\isafolddocument}%
%
\isadelimdocument
%
\endisadelimdocument
\isacommand{lemma}\isamarkupfalse%
\ Forces{\isacharunderscore}{\kern0pt}And{\isacharcolon}{\kern0pt}\isanewline
\ \ \isakeyword{assumes}\isanewline
\ \ \ \ {\isachardoublequoteopen}p{\isasymin}P{\isachardoublequoteclose}\ {\isachardoublequoteopen}env\ {\isasymin}\ list{\isacharparenleft}{\kern0pt}M{\isacharparenright}{\kern0pt}{\isachardoublequoteclose}\ {\isachardoublequoteopen}{\isasymphi}{\isasymin}formula{\isachardoublequoteclose}\ {\isachardoublequoteopen}{\isasympsi}{\isasymin}formula{\isachardoublequoteclose}\ \isanewline
\ \ \ \ {\isachardoublequoteopen}arity{\isacharparenleft}{\kern0pt}{\isasymphi}{\isacharparenright}{\kern0pt}\ {\isasymle}\ length{\isacharparenleft}{\kern0pt}env{\isacharparenright}{\kern0pt}{\isachardoublequoteclose}\ {\isachardoublequoteopen}arity{\isacharparenleft}{\kern0pt}{\isasympsi}{\isacharparenright}{\kern0pt}\ {\isasymle}\ length{\isacharparenleft}{\kern0pt}env{\isacharparenright}{\kern0pt}{\isachardoublequoteclose}\isanewline
\ \ \isakeyword{shows}\isanewline
\ \ \ \ {\isachardoublequoteopen}p\ {\isasymtturnstile}\ And{\isacharparenleft}{\kern0pt}{\isasymphi}{\isacharcomma}{\kern0pt}{\isasympsi}{\isacharparenright}{\kern0pt}\ env\ \ \ {\isasymlongleftrightarrow}\ \ {\isacharparenleft}{\kern0pt}p\ {\isasymtturnstile}\ {\isasymphi}\ env{\isacharparenright}{\kern0pt}\ {\isasymand}\ {\isacharparenleft}{\kern0pt}p\ {\isasymtturnstile}\ {\isasympsi}\ env{\isacharparenright}{\kern0pt}{\isachardoublequoteclose}\isanewline
%
\isadelimproof
%
\endisadelimproof
%
\isatagproof
\isacommand{proof}\isamarkupfalse%
\isanewline
\ \ \isacommand{assume}\isamarkupfalse%
\ {\isachardoublequoteopen}p\ {\isasymtturnstile}\ And{\isacharparenleft}{\kern0pt}{\isasymphi}{\isacharcomma}{\kern0pt}\ {\isasympsi}{\isacharparenright}{\kern0pt}\ env{\isachardoublequoteclose}\isanewline
\ \ \isacommand{with}\isamarkupfalse%
\ assms\isanewline
\ \ \isacommand{have}\isamarkupfalse%
\ {\isachardoublequoteopen}dense{\isacharunderscore}{\kern0pt}below{\isacharparenleft}{\kern0pt}{\isacharbraceleft}{\kern0pt}r\ {\isasymin}\ P\ {\isachardot}{\kern0pt}\ {\isacharparenleft}{\kern0pt}r\ {\isasymtturnstile}\ {\isasymphi}\ env{\isacharparenright}{\kern0pt}\ {\isasymand}\ {\isacharparenleft}{\kern0pt}r\ {\isasymtturnstile}\ {\isasympsi}\ env{\isacharparenright}{\kern0pt}{\isacharbraceright}{\kern0pt}{\isacharcomma}{\kern0pt}\ p{\isacharparenright}{\kern0pt}{\isachardoublequoteclose}\isanewline
\ \ \ \ \isacommand{using}\isamarkupfalse%
\ Forces{\isacharunderscore}{\kern0pt}And{\isacharunderscore}{\kern0pt}iff{\isacharunderscore}{\kern0pt}dense{\isacharunderscore}{\kern0pt}below\ \isacommand{by}\isamarkupfalse%
\ simp\isanewline
\ \ \isacommand{then}\isamarkupfalse%
\isanewline
\ \ \isacommand{have}\isamarkupfalse%
\ {\isachardoublequoteopen}dense{\isacharunderscore}{\kern0pt}below{\isacharparenleft}{\kern0pt}{\isacharbraceleft}{\kern0pt}r\ {\isasymin}\ P\ {\isachardot}{\kern0pt}\ {\isacharparenleft}{\kern0pt}r\ {\isasymtturnstile}\ {\isasymphi}\ env{\isacharparenright}{\kern0pt}{\isacharbraceright}{\kern0pt}{\isacharcomma}{\kern0pt}\ p{\isacharparenright}{\kern0pt}{\isachardoublequoteclose}\ {\isachardoublequoteopen}dense{\isacharunderscore}{\kern0pt}below{\isacharparenleft}{\kern0pt}{\isacharbraceleft}{\kern0pt}r\ {\isasymin}\ P\ {\isachardot}{\kern0pt}\ {\isacharparenleft}{\kern0pt}r\ {\isasymtturnstile}\ {\isasympsi}\ env{\isacharparenright}{\kern0pt}{\isacharbraceright}{\kern0pt}{\isacharcomma}{\kern0pt}\ p{\isacharparenright}{\kern0pt}{\isachardoublequoteclose}\isanewline
\ \ \ \ \isacommand{by}\isamarkupfalse%
\ blast{\isacharplus}{\kern0pt}\isanewline
\ \ \isacommand{with}\isamarkupfalse%
\ assms\isanewline
\ \ \isacommand{show}\isamarkupfalse%
\ {\isachardoublequoteopen}{\isacharparenleft}{\kern0pt}p\ {\isasymtturnstile}\ {\isasymphi}\ env{\isacharparenright}{\kern0pt}\ {\isasymand}\ {\isacharparenleft}{\kern0pt}p\ {\isasymtturnstile}\ {\isasympsi}\ env{\isacharparenright}{\kern0pt}{\isachardoublequoteclose}\isanewline
\ \ \ \ \isacommand{using}\isamarkupfalse%
\ density{\isacharunderscore}{\kern0pt}lemma{\isacharbrackleft}{\kern0pt}symmetric{\isacharbrackright}{\kern0pt}\ \isacommand{by}\isamarkupfalse%
\ simp\isanewline
\isacommand{next}\isamarkupfalse%
\isanewline
\ \ \isacommand{assume}\isamarkupfalse%
\ {\isachardoublequoteopen}{\isacharparenleft}{\kern0pt}p\ {\isasymtturnstile}\ {\isasymphi}\ env{\isacharparenright}{\kern0pt}\ {\isasymand}\ {\isacharparenleft}{\kern0pt}p\ {\isasymtturnstile}\ {\isasympsi}\ env{\isacharparenright}{\kern0pt}{\isachardoublequoteclose}\isanewline
\ \ \isacommand{have}\isamarkupfalse%
\ {\isachardoublequoteopen}dense{\isacharunderscore}{\kern0pt}below{\isacharparenleft}{\kern0pt}{\isacharbraceleft}{\kern0pt}r\ {\isasymin}\ P\ {\isachardot}{\kern0pt}\ {\isacharparenleft}{\kern0pt}r\ {\isasymtturnstile}\ {\isasymphi}\ env{\isacharparenright}{\kern0pt}\ {\isasymand}\ {\isacharparenleft}{\kern0pt}r\ {\isasymtturnstile}\ {\isasympsi}\ env{\isacharparenright}{\kern0pt}{\isacharbraceright}{\kern0pt}{\isacharcomma}{\kern0pt}\ p{\isacharparenright}{\kern0pt}{\isachardoublequoteclose}\isanewline
\ \ \isacommand{proof}\isamarkupfalse%
\ {\isacharparenleft}{\kern0pt}intro\ dense{\isacharunderscore}{\kern0pt}belowI\ bexI\ conjI{\isacharcomma}{\kern0pt}\ assumption{\isacharparenright}{\kern0pt}\isanewline
\ \ \ \ \isacommand{fix}\isamarkupfalse%
\ q\isanewline
\ \ \ \ \isacommand{assume}\isamarkupfalse%
\ {\isachardoublequoteopen}q{\isasymin}P{\isachardoublequoteclose}\ {\isachardoublequoteopen}q{\isasympreceq}\ p{\isachardoublequoteclose}\isanewline
\ \ \ \ \isacommand{with}\isamarkupfalse%
\ assms\ {\isacartoucheopen}{\isacharparenleft}{\kern0pt}p\ {\isasymtturnstile}\ {\isasymphi}\ env{\isacharparenright}{\kern0pt}\ {\isasymand}\ {\isacharparenleft}{\kern0pt}p\ {\isasymtturnstile}\ {\isasympsi}\ env{\isacharparenright}{\kern0pt}{\isacartoucheclose}\isanewline
\ \ \ \ \isacommand{show}\isamarkupfalse%
\ {\isachardoublequoteopen}q{\isasymin}{\isacharbraceleft}{\kern0pt}r\ {\isasymin}\ P\ {\isachardot}{\kern0pt}\ {\isacharparenleft}{\kern0pt}r\ {\isasymtturnstile}\ {\isasymphi}\ env{\isacharparenright}{\kern0pt}\ {\isasymand}\ {\isacharparenleft}{\kern0pt}r\ {\isasymtturnstile}\ {\isasympsi}\ env{\isacharparenright}{\kern0pt}{\isacharbraceright}{\kern0pt}{\isachardoublequoteclose}\ {\isachardoublequoteopen}q{\isasympreceq}\ q{\isachardoublequoteclose}\isanewline
\ \ \ \ \ \ \isacommand{using}\isamarkupfalse%
\ strengthening{\isacharunderscore}{\kern0pt}lemma\ leq{\isacharunderscore}{\kern0pt}reflI\ \isacommand{by}\isamarkupfalse%
\ auto\isanewline
\ \ \isacommand{qed}\isamarkupfalse%
\isanewline
\ \ \isacommand{with}\isamarkupfalse%
\ assms\isanewline
\ \ \isacommand{show}\isamarkupfalse%
\ {\isachardoublequoteopen}p\ {\isasymtturnstile}\ And{\isacharparenleft}{\kern0pt}{\isasymphi}{\isacharcomma}{\kern0pt}{\isasympsi}{\isacharparenright}{\kern0pt}\ env{\isachardoublequoteclose}\isanewline
\ \ \ \ \isacommand{using}\isamarkupfalse%
\ Forces{\isacharunderscore}{\kern0pt}And{\isacharunderscore}{\kern0pt}iff{\isacharunderscore}{\kern0pt}dense{\isacharunderscore}{\kern0pt}below\ \isacommand{by}\isamarkupfalse%
\ simp\isanewline
\isacommand{qed}\isamarkupfalse%
%
\endisatagproof
{\isafoldproof}%
%
\isadelimproof
\isanewline
%
\endisadelimproof
\isanewline
\isacommand{lemma}\isamarkupfalse%
\ Forces{\isacharunderscore}{\kern0pt}Nand{\isacharunderscore}{\kern0pt}alt{\isacharcolon}{\kern0pt}\isanewline
\ \ \isakeyword{assumes}\isanewline
\ \ \ \ {\isachardoublequoteopen}p{\isasymin}P{\isachardoublequoteclose}\ {\isachardoublequoteopen}env\ {\isasymin}\ list{\isacharparenleft}{\kern0pt}M{\isacharparenright}{\kern0pt}{\isachardoublequoteclose}\ {\isachardoublequoteopen}{\isasymphi}{\isasymin}formula{\isachardoublequoteclose}\ {\isachardoublequoteopen}{\isasympsi}{\isasymin}formula{\isachardoublequoteclose}\ \isanewline
\ \ \ \ {\isachardoublequoteopen}arity{\isacharparenleft}{\kern0pt}{\isasymphi}{\isacharparenright}{\kern0pt}\ {\isasymle}\ length{\isacharparenleft}{\kern0pt}env{\isacharparenright}{\kern0pt}{\isachardoublequoteclose}\ {\isachardoublequoteopen}arity{\isacharparenleft}{\kern0pt}{\isasympsi}{\isacharparenright}{\kern0pt}\ {\isasymle}\ length{\isacharparenleft}{\kern0pt}env{\isacharparenright}{\kern0pt}{\isachardoublequoteclose}\isanewline
\ \ \isakeyword{shows}\isanewline
\ \ \ \ {\isachardoublequoteopen}{\isacharparenleft}{\kern0pt}p\ {\isasymtturnstile}\ Nand{\isacharparenleft}{\kern0pt}{\isasymphi}{\isacharcomma}{\kern0pt}{\isasympsi}{\isacharparenright}{\kern0pt}\ env{\isacharparenright}{\kern0pt}\ {\isasymlongleftrightarrow}\ {\isacharparenleft}{\kern0pt}p\ {\isasymtturnstile}\ Neg{\isacharparenleft}{\kern0pt}And{\isacharparenleft}{\kern0pt}{\isasymphi}{\isacharcomma}{\kern0pt}{\isasympsi}{\isacharparenright}{\kern0pt}{\isacharparenright}{\kern0pt}\ env{\isacharparenright}{\kern0pt}{\isachardoublequoteclose}\isanewline
%
\isadelimproof
\ \ %
\endisadelimproof
%
\isatagproof
\isacommand{using}\isamarkupfalse%
\ assms\ Forces{\isacharunderscore}{\kern0pt}Nand\ Forces{\isacharunderscore}{\kern0pt}And\ Forces{\isacharunderscore}{\kern0pt}Neg\ \isacommand{by}\isamarkupfalse%
\ auto%
\endisatagproof
{\isafoldproof}%
%
\isadelimproof
\isanewline
%
\endisadelimproof
\isanewline
\isacommand{lemma}\isamarkupfalse%
\ truth{\isacharunderscore}{\kern0pt}lemma{\isacharunderscore}{\kern0pt}Neg{\isacharcolon}{\kern0pt}\isanewline
\ \ \isakeyword{assumes}\ \isanewline
\ \ \ \ {\isachardoublequoteopen}{\isasymphi}{\isasymin}formula{\isachardoublequoteclose}\ {\isachardoublequoteopen}M{\isacharunderscore}{\kern0pt}generic{\isacharparenleft}{\kern0pt}G{\isacharparenright}{\kern0pt}{\isachardoublequoteclose}\ {\isachardoublequoteopen}env{\isasymin}list{\isacharparenleft}{\kern0pt}M{\isacharparenright}{\kern0pt}{\isachardoublequoteclose}\ {\isachardoublequoteopen}arity{\isacharparenleft}{\kern0pt}{\isasymphi}{\isacharparenright}{\kern0pt}{\isasymle}length{\isacharparenleft}{\kern0pt}env{\isacharparenright}{\kern0pt}{\isachardoublequoteclose}\ \isakeyword{and}\isanewline
\ \ \ \ IH{\isacharcolon}{\kern0pt}\ {\isachardoublequoteopen}{\isacharparenleft}{\kern0pt}{\isasymexists}p{\isasymin}G{\isachardot}{\kern0pt}\ p\ {\isasymtturnstile}\ {\isasymphi}\ env{\isacharparenright}{\kern0pt}\ {\isasymlongleftrightarrow}\ M{\isacharbrackleft}{\kern0pt}G{\isacharbrackright}{\kern0pt}{\isacharcomma}{\kern0pt}\ map{\isacharparenleft}{\kern0pt}val{\isacharparenleft}{\kern0pt}G{\isacharparenright}{\kern0pt}{\isacharcomma}{\kern0pt}env{\isacharparenright}{\kern0pt}\ {\isasymTurnstile}\ {\isasymphi}{\isachardoublequoteclose}\isanewline
\ \ \isakeyword{shows}\isanewline
\ \ \ \ {\isachardoublequoteopen}{\isacharparenleft}{\kern0pt}{\isasymexists}p{\isasymin}G{\isachardot}{\kern0pt}\ p\ {\isasymtturnstile}\ Neg{\isacharparenleft}{\kern0pt}{\isasymphi}{\isacharparenright}{\kern0pt}\ env{\isacharparenright}{\kern0pt}\ \ {\isasymlongleftrightarrow}\ \ M{\isacharbrackleft}{\kern0pt}G{\isacharbrackright}{\kern0pt}{\isacharcomma}{\kern0pt}\ map{\isacharparenleft}{\kern0pt}val{\isacharparenleft}{\kern0pt}G{\isacharparenright}{\kern0pt}{\isacharcomma}{\kern0pt}env{\isacharparenright}{\kern0pt}\ {\isasymTurnstile}\ Neg{\isacharparenleft}{\kern0pt}{\isasymphi}{\isacharparenright}{\kern0pt}{\isachardoublequoteclose}\isanewline
%
\isadelimproof
%
\endisadelimproof
%
\isatagproof
\isacommand{proof}\isamarkupfalse%
\ {\isacharparenleft}{\kern0pt}intro\ iffI{\isacharcomma}{\kern0pt}\ elim\ bexE{\isacharcomma}{\kern0pt}\ rule\ ccontr{\isacharparenright}{\kern0pt}\ \isanewline
\ \ \isanewline
\ \ \isacommand{fix}\isamarkupfalse%
\ p\ \isanewline
\ \ \isacommand{assume}\isamarkupfalse%
\ {\isachardoublequoteopen}p{\isasymin}G{\isachardoublequoteclose}\ {\isachardoublequoteopen}p\ {\isasymtturnstile}\ Neg{\isacharparenleft}{\kern0pt}{\isasymphi}{\isacharparenright}{\kern0pt}\ env{\isachardoublequoteclose}\ {\isachardoublequoteopen}{\isasymnot}{\isacharparenleft}{\kern0pt}M{\isacharbrackleft}{\kern0pt}G{\isacharbrackright}{\kern0pt}{\isacharcomma}{\kern0pt}map{\isacharparenleft}{\kern0pt}val{\isacharparenleft}{\kern0pt}G{\isacharparenright}{\kern0pt}{\isacharcomma}{\kern0pt}env{\isacharparenright}{\kern0pt}\ {\isasymTurnstile}\ Neg{\isacharparenleft}{\kern0pt}{\isasymphi}{\isacharparenright}{\kern0pt}{\isacharparenright}{\kern0pt}{\isachardoublequoteclose}\isanewline
\ \ \isacommand{moreover}\isamarkupfalse%
\ \isanewline
\ \ \isacommand{note}\isamarkupfalse%
\ assms\isanewline
\ \ \isacommand{moreover}\isamarkupfalse%
\ \isacommand{from}\isamarkupfalse%
\ calculation\isanewline
\ \ \isacommand{have}\isamarkupfalse%
\ {\isachardoublequoteopen}M{\isacharbrackleft}{\kern0pt}G{\isacharbrackright}{\kern0pt}{\isacharcomma}{\kern0pt}\ map{\isacharparenleft}{\kern0pt}val{\isacharparenleft}{\kern0pt}G{\isacharparenright}{\kern0pt}{\isacharcomma}{\kern0pt}env{\isacharparenright}{\kern0pt}\ {\isasymTurnstile}\ {\isasymphi}{\isachardoublequoteclose}\isanewline
\ \ \ \ \isacommand{using}\isamarkupfalse%
\ map{\isacharunderscore}{\kern0pt}val{\isacharunderscore}{\kern0pt}in{\isacharunderscore}{\kern0pt}MG\ \isacommand{by}\isamarkupfalse%
\ simp\isanewline
\ \ \isacommand{with}\isamarkupfalse%
\ IH\isanewline
\ \ \isacommand{obtain}\isamarkupfalse%
\ r\ \isakeyword{where}\ {\isachardoublequoteopen}r\ {\isasymtturnstile}\ {\isasymphi}\ env{\isachardoublequoteclose}\ {\isachardoublequoteopen}r{\isasymin}G{\isachardoublequoteclose}\ \isacommand{by}\isamarkupfalse%
\ blast\isanewline
\ \ \isacommand{moreover}\isamarkupfalse%
\ \isacommand{from}\isamarkupfalse%
\ this\ \isakeyword{and}\ {\isacartoucheopen}M{\isacharunderscore}{\kern0pt}generic{\isacharparenleft}{\kern0pt}G{\isacharparenright}{\kern0pt}{\isacartoucheclose}\ {\isacartoucheopen}p{\isasymin}G{\isacartoucheclose}\isanewline
\ \ \isacommand{obtain}\isamarkupfalse%
\ q\ \isakeyword{where}\ {\isachardoublequoteopen}q{\isasympreceq}p{\isachardoublequoteclose}\ {\isachardoublequoteopen}q{\isasympreceq}r{\isachardoublequoteclose}\ {\isachardoublequoteopen}q{\isasymin}G{\isachardoublequoteclose}\isanewline
\ \ \ \ \isacommand{by}\isamarkupfalse%
\ blast\isanewline
\ \ \isacommand{moreover}\isamarkupfalse%
\ \isacommand{from}\isamarkupfalse%
\ calculation\ \isanewline
\ \ \isacommand{have}\isamarkupfalse%
\ {\isachardoublequoteopen}q\ {\isasymtturnstile}\ {\isasymphi}\ env{\isachardoublequoteclose}\isanewline
\ \ \ \ \isacommand{using}\isamarkupfalse%
\ strengthening{\isacharunderscore}{\kern0pt}lemma{\isacharbrackleft}{\kern0pt}\isakeyword{where}\ {\isasymphi}{\isacharequal}{\kern0pt}{\isasymphi}{\isacharbrackright}{\kern0pt}\ \isacommand{by}\isamarkupfalse%
\ blast\isanewline
\ \ \isacommand{ultimately}\isamarkupfalse%
\isanewline
\ \ \isacommand{show}\isamarkupfalse%
\ {\isachardoublequoteopen}False{\isachardoublequoteclose}\isanewline
\ \ \ \ \isacommand{using}\isamarkupfalse%
\ Forces{\isacharunderscore}{\kern0pt}Neg{\isacharbrackleft}{\kern0pt}\isakeyword{where}\ {\isasymphi}{\isacharequal}{\kern0pt}{\isasymphi}{\isacharbrackright}{\kern0pt}\ transitivity{\isacharbrackleft}{\kern0pt}OF\ {\isacharunderscore}{\kern0pt}\ P{\isacharunderscore}{\kern0pt}in{\isacharunderscore}{\kern0pt}M{\isacharbrackright}{\kern0pt}\ \isacommand{by}\isamarkupfalse%
\ blast\isanewline
\isacommand{next}\isamarkupfalse%
\isanewline
\ \ \isacommand{assume}\isamarkupfalse%
\ {\isachardoublequoteopen}M{\isacharbrackleft}{\kern0pt}G{\isacharbrackright}{\kern0pt}{\isacharcomma}{\kern0pt}\ map{\isacharparenleft}{\kern0pt}val{\isacharparenleft}{\kern0pt}G{\isacharparenright}{\kern0pt}{\isacharcomma}{\kern0pt}env{\isacharparenright}{\kern0pt}\ {\isasymTurnstile}\ Neg{\isacharparenleft}{\kern0pt}{\isasymphi}{\isacharparenright}{\kern0pt}{\isachardoublequoteclose}\isanewline
\ \ \isacommand{with}\isamarkupfalse%
\ assms\ \isanewline
\ \ \isacommand{have}\isamarkupfalse%
\ {\isachardoublequoteopen}{\isasymnot}\ {\isacharparenleft}{\kern0pt}M{\isacharbrackleft}{\kern0pt}G{\isacharbrackright}{\kern0pt}{\isacharcomma}{\kern0pt}\ map{\isacharparenleft}{\kern0pt}val{\isacharparenleft}{\kern0pt}G{\isacharparenright}{\kern0pt}{\isacharcomma}{\kern0pt}env{\isacharparenright}{\kern0pt}\ {\isasymTurnstile}\ {\isasymphi}{\isacharparenright}{\kern0pt}{\isachardoublequoteclose}\isanewline
\ \ \ \ \isacommand{using}\isamarkupfalse%
\ map{\isacharunderscore}{\kern0pt}val{\isacharunderscore}{\kern0pt}in{\isacharunderscore}{\kern0pt}MG\ \isacommand{by}\isamarkupfalse%
\ simp\isanewline
\ \ \isacommand{let}\isamarkupfalse%
\ {\isacharquery}{\kern0pt}D{\isacharequal}{\kern0pt}{\isachardoublequoteopen}{\isacharbraceleft}{\kern0pt}p{\isasymin}P{\isachardot}{\kern0pt}\ {\isacharparenleft}{\kern0pt}p\ {\isasymtturnstile}\ {\isasymphi}\ env{\isacharparenright}{\kern0pt}\ {\isasymor}\ {\isacharparenleft}{\kern0pt}p\ {\isasymtturnstile}\ Neg{\isacharparenleft}{\kern0pt}{\isasymphi}{\isacharparenright}{\kern0pt}\ env{\isacharparenright}{\kern0pt}{\isacharbraceright}{\kern0pt}{\isachardoublequoteclose}\isanewline
\ \ \isacommand{have}\isamarkupfalse%
\ {\isachardoublequoteopen}separation{\isacharparenleft}{\kern0pt}{\isacharhash}{\kern0pt}{\isacharhash}{\kern0pt}M{\isacharcomma}{\kern0pt}{\isasymlambda}p{\isachardot}{\kern0pt}\ {\isacharparenleft}{\kern0pt}p\ {\isasymtturnstile}\ {\isasymphi}\ env{\isacharparenright}{\kern0pt}{\isacharparenright}{\kern0pt}{\isachardoublequoteclose}\ \isanewline
\ \ \ \ \ \ \isacommand{using}\isamarkupfalse%
\ separation{\isacharunderscore}{\kern0pt}ax\ arity{\isacharunderscore}{\kern0pt}forces\ assms\ P{\isacharunderscore}{\kern0pt}in{\isacharunderscore}{\kern0pt}M\ leq{\isacharunderscore}{\kern0pt}in{\isacharunderscore}{\kern0pt}M\ one{\isacharunderscore}{\kern0pt}in{\isacharunderscore}{\kern0pt}M\ arity{\isacharunderscore}{\kern0pt}forces{\isacharunderscore}{\kern0pt}le\isanewline
\ \ \ \ \isacommand{by}\isamarkupfalse%
\ simp\isanewline
\ \ \isacommand{moreover}\isamarkupfalse%
\isanewline
\ \ \isacommand{have}\isamarkupfalse%
\ {\isachardoublequoteopen}separation{\isacharparenleft}{\kern0pt}{\isacharhash}{\kern0pt}{\isacharhash}{\kern0pt}M{\isacharcomma}{\kern0pt}{\isasymlambda}p{\isachardot}{\kern0pt}\ {\isacharparenleft}{\kern0pt}p\ {\isasymtturnstile}\ Neg{\isacharparenleft}{\kern0pt}{\isasymphi}{\isacharparenright}{\kern0pt}\ env{\isacharparenright}{\kern0pt}{\isacharparenright}{\kern0pt}{\isachardoublequoteclose}\isanewline
\ \ \ \ \ \ \isacommand{using}\isamarkupfalse%
\ separation{\isacharunderscore}{\kern0pt}ax\ arity{\isacharunderscore}{\kern0pt}forces\ assms\ P{\isacharunderscore}{\kern0pt}in{\isacharunderscore}{\kern0pt}M\ leq{\isacharunderscore}{\kern0pt}in{\isacharunderscore}{\kern0pt}M\ one{\isacharunderscore}{\kern0pt}in{\isacharunderscore}{\kern0pt}M\ arity{\isacharunderscore}{\kern0pt}forces{\isacharunderscore}{\kern0pt}le\isanewline
\ \ \ \ \isacommand{by}\isamarkupfalse%
\ simp\isanewline
\ \ \isacommand{ultimately}\isamarkupfalse%
\isanewline
\ \ \isacommand{have}\isamarkupfalse%
\ {\isachardoublequoteopen}separation{\isacharparenleft}{\kern0pt}{\isacharhash}{\kern0pt}{\isacharhash}{\kern0pt}M{\isacharcomma}{\kern0pt}{\isasymlambda}p{\isachardot}{\kern0pt}\ {\isacharparenleft}{\kern0pt}p\ {\isasymtturnstile}\ {\isasymphi}\ env{\isacharparenright}{\kern0pt}\ {\isasymor}\ {\isacharparenleft}{\kern0pt}p\ {\isasymtturnstile}\ Neg{\isacharparenleft}{\kern0pt}{\isasymphi}{\isacharparenright}{\kern0pt}\ env{\isacharparenright}{\kern0pt}{\isacharparenright}{\kern0pt}{\isachardoublequoteclose}\ \isanewline
\ \ \ \ \isacommand{using}\isamarkupfalse%
\ separation{\isacharunderscore}{\kern0pt}disj\ \isacommand{by}\isamarkupfalse%
\ simp\isanewline
\ \ \isacommand{then}\isamarkupfalse%
\ \isanewline
\ \ \isacommand{have}\isamarkupfalse%
\ {\isachardoublequoteopen}{\isacharquery}{\kern0pt}D\ {\isasymin}\ M{\isachardoublequoteclose}\ \isanewline
\ \ \ \ \isacommand{using}\isamarkupfalse%
\ separation{\isacharunderscore}{\kern0pt}closed\ P{\isacharunderscore}{\kern0pt}in{\isacharunderscore}{\kern0pt}M\ \isacommand{by}\isamarkupfalse%
\ simp\isanewline
\ \ \isacommand{moreover}\isamarkupfalse%
\isanewline
\ \ \isacommand{have}\isamarkupfalse%
\ {\isachardoublequoteopen}{\isacharquery}{\kern0pt}D\ {\isasymsubseteq}\ P{\isachardoublequoteclose}\ \isacommand{by}\isamarkupfalse%
\ auto\isanewline
\ \ \isacommand{moreover}\isamarkupfalse%
\isanewline
\ \ \isacommand{have}\isamarkupfalse%
\ {\isachardoublequoteopen}dense{\isacharparenleft}{\kern0pt}{\isacharquery}{\kern0pt}D{\isacharparenright}{\kern0pt}{\isachardoublequoteclose}\isanewline
\ \ \isacommand{proof}\isamarkupfalse%
\isanewline
\ \ \ \ \isacommand{fix}\isamarkupfalse%
\ q\isanewline
\ \ \ \ \isacommand{assume}\isamarkupfalse%
\ {\isachardoublequoteopen}q{\isasymin}P{\isachardoublequoteclose}\isanewline
\ \ \ \ \isacommand{show}\isamarkupfalse%
\ {\isachardoublequoteopen}{\isasymexists}d{\isasymin}{\isacharbraceleft}{\kern0pt}p\ {\isasymin}\ P\ {\isachardot}{\kern0pt}\ {\isacharparenleft}{\kern0pt}p\ {\isasymtturnstile}\ {\isasymphi}\ env{\isacharparenright}{\kern0pt}\ {\isasymor}\ {\isacharparenleft}{\kern0pt}p\ {\isasymtturnstile}\ Neg{\isacharparenleft}{\kern0pt}{\isasymphi}{\isacharparenright}{\kern0pt}\ env{\isacharparenright}{\kern0pt}{\isacharbraceright}{\kern0pt}{\isachardot}{\kern0pt}\ d{\isasympreceq}\ q{\isachardoublequoteclose}\isanewline
\ \ \ \ \isacommand{proof}\isamarkupfalse%
\ {\isacharparenleft}{\kern0pt}cases\ {\isachardoublequoteopen}q\ {\isasymtturnstile}\ Neg{\isacharparenleft}{\kern0pt}{\isasymphi}{\isacharparenright}{\kern0pt}\ env{\isachardoublequoteclose}{\isacharparenright}{\kern0pt}\isanewline
\ \ \ \ \ \ \isacommand{case}\isamarkupfalse%
\ True\isanewline
\ \ \ \ \ \ \isacommand{with}\isamarkupfalse%
\ {\isacartoucheopen}q{\isasymin}P{\isacartoucheclose}\isanewline
\ \ \ \ \ \ \isacommand{show}\isamarkupfalse%
\ {\isacharquery}{\kern0pt}thesis\ \isacommand{using}\isamarkupfalse%
\ leq{\isacharunderscore}{\kern0pt}reflI\ \isacommand{by}\isamarkupfalse%
\ blast\isanewline
\ \ \ \ \isacommand{next}\isamarkupfalse%
\isanewline
\ \ \ \ \ \ \isacommand{case}\isamarkupfalse%
\ False\isanewline
\ \ \ \ \ \ \isacommand{with}\isamarkupfalse%
\ {\isacartoucheopen}q{\isasymin}P{\isacartoucheclose}\ \isakeyword{and}\ assms\isanewline
\ \ \ \ \ \ \isacommand{show}\isamarkupfalse%
\ {\isacharquery}{\kern0pt}thesis\ \isacommand{using}\isamarkupfalse%
\ Forces{\isacharunderscore}{\kern0pt}Neg\ \isacommand{by}\isamarkupfalse%
\ auto\isanewline
\ \ \ \ \isacommand{qed}\isamarkupfalse%
\isanewline
\ \ \isacommand{qed}\isamarkupfalse%
\isanewline
\ \ \isacommand{moreover}\isamarkupfalse%
\isanewline
\ \ \isacommand{note}\isamarkupfalse%
\ {\isacartoucheopen}M{\isacharunderscore}{\kern0pt}generic{\isacharparenleft}{\kern0pt}G{\isacharparenright}{\kern0pt}{\isacartoucheclose}\isanewline
\ \ \isacommand{ultimately}\isamarkupfalse%
\isanewline
\ \ \isacommand{obtain}\isamarkupfalse%
\ p\ \isakeyword{where}\ {\isachardoublequoteopen}p{\isasymin}G{\isachardoublequoteclose}\ {\isachardoublequoteopen}{\isacharparenleft}{\kern0pt}p\ {\isasymtturnstile}\ {\isasymphi}\ env{\isacharparenright}{\kern0pt}\ {\isasymor}\ {\isacharparenleft}{\kern0pt}p\ {\isasymtturnstile}\ Neg{\isacharparenleft}{\kern0pt}{\isasymphi}{\isacharparenright}{\kern0pt}\ env{\isacharparenright}{\kern0pt}{\isachardoublequoteclose}\isanewline
\ \ \ \ \isacommand{by}\isamarkupfalse%
\ blast\isanewline
\ \ \isacommand{then}\isamarkupfalse%
\isanewline
\ \ \isacommand{consider}\isamarkupfalse%
\ {\isacharparenleft}{\kern0pt}{\isadigit{1}}{\isacharparenright}{\kern0pt}\ {\isachardoublequoteopen}p\ {\isasymtturnstile}\ {\isasymphi}\ env{\isachardoublequoteclose}\ {\isacharbar}{\kern0pt}\ {\isacharparenleft}{\kern0pt}{\isadigit{2}}{\isacharparenright}{\kern0pt}\ {\isachardoublequoteopen}p\ {\isasymtturnstile}\ Neg{\isacharparenleft}{\kern0pt}{\isasymphi}{\isacharparenright}{\kern0pt}\ env{\isachardoublequoteclose}\ \isacommand{by}\isamarkupfalse%
\ blast\isanewline
\ \ \isacommand{then}\isamarkupfalse%
\isanewline
\ \ \isacommand{show}\isamarkupfalse%
\ {\isachardoublequoteopen}{\isasymexists}p{\isasymin}G{\isachardot}{\kern0pt}\ {\isacharparenleft}{\kern0pt}p\ {\isasymtturnstile}\ Neg{\isacharparenleft}{\kern0pt}{\isasymphi}{\isacharparenright}{\kern0pt}\ env{\isacharparenright}{\kern0pt}{\isachardoublequoteclose}\isanewline
\ \ \isacommand{proof}\isamarkupfalse%
\ {\isacharparenleft}{\kern0pt}cases{\isacharparenright}{\kern0pt}\isanewline
\ \ \ \ \isacommand{case}\isamarkupfalse%
\ {\isadigit{1}}\isanewline
\ \ \ \ \isacommand{with}\isamarkupfalse%
\ {\isacartoucheopen}{\isasymnot}\ {\isacharparenleft}{\kern0pt}M{\isacharbrackleft}{\kern0pt}G{\isacharbrackright}{\kern0pt}{\isacharcomma}{\kern0pt}map{\isacharparenleft}{\kern0pt}val{\isacharparenleft}{\kern0pt}G{\isacharparenright}{\kern0pt}{\isacharcomma}{\kern0pt}env{\isacharparenright}{\kern0pt}\ {\isasymTurnstile}\ {\isasymphi}{\isacharparenright}{\kern0pt}{\isacartoucheclose}\ {\isacartoucheopen}p{\isasymin}G{\isacartoucheclose}\ IH\isanewline
\ \ \ \ \isacommand{show}\isamarkupfalse%
\ {\isacharquery}{\kern0pt}thesis\isanewline
\ \ \ \ \ \ \isacommand{by}\isamarkupfalse%
\ blast\isanewline
\ \ \isacommand{next}\isamarkupfalse%
\isanewline
\ \ \ \ \isacommand{case}\isamarkupfalse%
\ {\isadigit{2}}\isanewline
\ \ \ \ \isacommand{with}\isamarkupfalse%
\ {\isacartoucheopen}p{\isasymin}G{\isacartoucheclose}\ \isanewline
\ \ \ \ \isacommand{show}\isamarkupfalse%
\ {\isacharquery}{\kern0pt}thesis\ \isacommand{by}\isamarkupfalse%
\ blast\isanewline
\ \ \isacommand{qed}\isamarkupfalse%
\isanewline
\isacommand{qed}\isamarkupfalse%
%
\endisatagproof
{\isafoldproof}%
%
\isadelimproof
\ \isanewline
%
\endisadelimproof
\isanewline
\isacommand{lemma}\isamarkupfalse%
\ truth{\isacharunderscore}{\kern0pt}lemma{\isacharunderscore}{\kern0pt}And{\isacharcolon}{\kern0pt}\isanewline
\ \ \isakeyword{assumes}\ \isanewline
\ \ \ \ {\isachardoublequoteopen}env{\isasymin}list{\isacharparenleft}{\kern0pt}M{\isacharparenright}{\kern0pt}{\isachardoublequoteclose}\ {\isachardoublequoteopen}{\isasymphi}{\isasymin}formula{\isachardoublequoteclose}\ {\isachardoublequoteopen}{\isasympsi}{\isasymin}formula{\isachardoublequoteclose}\isanewline
\ \ \ \ {\isachardoublequoteopen}arity{\isacharparenleft}{\kern0pt}{\isasymphi}{\isacharparenright}{\kern0pt}{\isasymle}length{\isacharparenleft}{\kern0pt}env{\isacharparenright}{\kern0pt}{\isachardoublequoteclose}\ {\isachardoublequoteopen}arity{\isacharparenleft}{\kern0pt}{\isasympsi}{\isacharparenright}{\kern0pt}\ {\isasymle}\ length{\isacharparenleft}{\kern0pt}env{\isacharparenright}{\kern0pt}{\isachardoublequoteclose}\ {\isachardoublequoteopen}M{\isacharunderscore}{\kern0pt}generic{\isacharparenleft}{\kern0pt}G{\isacharparenright}{\kern0pt}{\isachardoublequoteclose}\isanewline
\ \ \ \ \isakeyword{and}\isanewline
\ \ \ \ IH{\isacharcolon}{\kern0pt}\ {\isachardoublequoteopen}{\isacharparenleft}{\kern0pt}{\isasymexists}p{\isasymin}G{\isachardot}{\kern0pt}\ p\ {\isasymtturnstile}\ {\isasymphi}\ env{\isacharparenright}{\kern0pt}\ \ {\isasymlongleftrightarrow}\ \ \ M{\isacharbrackleft}{\kern0pt}G{\isacharbrackright}{\kern0pt}{\isacharcomma}{\kern0pt}\ map{\isacharparenleft}{\kern0pt}val{\isacharparenleft}{\kern0pt}G{\isacharparenright}{\kern0pt}{\isacharcomma}{\kern0pt}env{\isacharparenright}{\kern0pt}\ {\isasymTurnstile}\ {\isasymphi}{\isachardoublequoteclose}\isanewline
\ \ \ \ \ \ \ \ {\isachardoublequoteopen}{\isacharparenleft}{\kern0pt}{\isasymexists}p{\isasymin}G{\isachardot}{\kern0pt}\ p\ {\isasymtturnstile}\ {\isasympsi}\ env{\isacharparenright}{\kern0pt}\ \ {\isasymlongleftrightarrow}\ \ \ M{\isacharbrackleft}{\kern0pt}G{\isacharbrackright}{\kern0pt}{\isacharcomma}{\kern0pt}\ map{\isacharparenleft}{\kern0pt}val{\isacharparenleft}{\kern0pt}G{\isacharparenright}{\kern0pt}{\isacharcomma}{\kern0pt}env{\isacharparenright}{\kern0pt}\ {\isasymTurnstile}\ {\isasympsi}{\isachardoublequoteclose}\isanewline
\ \ \isakeyword{shows}\isanewline
\ \ \ \ {\isachardoublequoteopen}{\isacharparenleft}{\kern0pt}{\isasymexists}p{\isasymin}G{\isachardot}{\kern0pt}\ {\isacharparenleft}{\kern0pt}p\ {\isasymtturnstile}\ And{\isacharparenleft}{\kern0pt}{\isasymphi}{\isacharcomma}{\kern0pt}{\isasympsi}{\isacharparenright}{\kern0pt}\ env{\isacharparenright}{\kern0pt}{\isacharparenright}{\kern0pt}\ {\isasymlongleftrightarrow}\ M{\isacharbrackleft}{\kern0pt}G{\isacharbrackright}{\kern0pt}\ {\isacharcomma}{\kern0pt}\ map{\isacharparenleft}{\kern0pt}val{\isacharparenleft}{\kern0pt}G{\isacharparenright}{\kern0pt}{\isacharcomma}{\kern0pt}env{\isacharparenright}{\kern0pt}\ {\isasymTurnstile}\ And{\isacharparenleft}{\kern0pt}{\isasymphi}{\isacharcomma}{\kern0pt}{\isasympsi}{\isacharparenright}{\kern0pt}{\isachardoublequoteclose}\isanewline
%
\isadelimproof
\ \ %
\endisadelimproof
%
\isatagproof
\isacommand{using}\isamarkupfalse%
\ assms\ map{\isacharunderscore}{\kern0pt}val{\isacharunderscore}{\kern0pt}in{\isacharunderscore}{\kern0pt}MG\ Forces{\isacharunderscore}{\kern0pt}And{\isacharbrackleft}{\kern0pt}OF\ M{\isacharunderscore}{\kern0pt}genericD\ assms{\isacharparenleft}{\kern0pt}{\isadigit{1}}{\isacharminus}{\kern0pt}{\isadigit{5}}{\isacharparenright}{\kern0pt}{\isacharbrackright}{\kern0pt}\isanewline
\isacommand{proof}\isamarkupfalse%
\ {\isacharparenleft}{\kern0pt}intro\ iffI{\isacharcomma}{\kern0pt}\ elim\ bexE{\isacharparenright}{\kern0pt}\isanewline
\ \ \isacommand{fix}\isamarkupfalse%
\ p\isanewline
\ \ \isacommand{assume}\isamarkupfalse%
\ {\isachardoublequoteopen}p{\isasymin}G{\isachardoublequoteclose}\ {\isachardoublequoteopen}p\ {\isasymtturnstile}\ And{\isacharparenleft}{\kern0pt}{\isasymphi}{\isacharcomma}{\kern0pt}{\isasympsi}{\isacharparenright}{\kern0pt}\ env{\isachardoublequoteclose}\isanewline
\ \ \isacommand{with}\isamarkupfalse%
\ assms\isanewline
\ \ \isacommand{show}\isamarkupfalse%
\ {\isachardoublequoteopen}M{\isacharbrackleft}{\kern0pt}G{\isacharbrackright}{\kern0pt}{\isacharcomma}{\kern0pt}\ map{\isacharparenleft}{\kern0pt}val{\isacharparenleft}{\kern0pt}G{\isacharparenright}{\kern0pt}{\isacharcomma}{\kern0pt}env{\isacharparenright}{\kern0pt}\ {\isasymTurnstile}\ And{\isacharparenleft}{\kern0pt}{\isasymphi}{\isacharcomma}{\kern0pt}{\isasympsi}{\isacharparenright}{\kern0pt}{\isachardoublequoteclose}\ \isanewline
\ \ \ \ \isacommand{using}\isamarkupfalse%
\ Forces{\isacharunderscore}{\kern0pt}And{\isacharbrackleft}{\kern0pt}OF\ M{\isacharunderscore}{\kern0pt}genericD{\isacharcomma}{\kern0pt}\ of\ {\isacharunderscore}{\kern0pt}\ {\isacharunderscore}{\kern0pt}\ {\isacharunderscore}{\kern0pt}\ {\isasymphi}\ {\isasympsi}{\isacharbrackright}{\kern0pt}\ map{\isacharunderscore}{\kern0pt}val{\isacharunderscore}{\kern0pt}in{\isacharunderscore}{\kern0pt}MG\ \isacommand{by}\isamarkupfalse%
\ auto\isanewline
\isacommand{next}\isamarkupfalse%
\ \isanewline
\ \ \isacommand{assume}\isamarkupfalse%
\ {\isachardoublequoteopen}M{\isacharbrackleft}{\kern0pt}G{\isacharbrackright}{\kern0pt}{\isacharcomma}{\kern0pt}\ map{\isacharparenleft}{\kern0pt}val{\isacharparenleft}{\kern0pt}G{\isacharparenright}{\kern0pt}{\isacharcomma}{\kern0pt}env{\isacharparenright}{\kern0pt}\ {\isasymTurnstile}\ And{\isacharparenleft}{\kern0pt}{\isasymphi}{\isacharcomma}{\kern0pt}{\isasympsi}{\isacharparenright}{\kern0pt}{\isachardoublequoteclose}\isanewline
\ \ \isacommand{moreover}\isamarkupfalse%
\isanewline
\ \ \isacommand{note}\isamarkupfalse%
\ assms\isanewline
\ \ \isacommand{moreover}\isamarkupfalse%
\ \isacommand{from}\isamarkupfalse%
\ calculation\isanewline
\ \ \isacommand{obtain}\isamarkupfalse%
\ q\ r\ \isakeyword{where}\ {\isachardoublequoteopen}q\ {\isasymtturnstile}\ {\isasymphi}\ env{\isachardoublequoteclose}\ {\isachardoublequoteopen}r\ {\isasymtturnstile}\ {\isasympsi}\ env{\isachardoublequoteclose}\ {\isachardoublequoteopen}q{\isasymin}G{\isachardoublequoteclose}\ {\isachardoublequoteopen}r{\isasymin}G{\isachardoublequoteclose}\isanewline
\ \ \ \ \isacommand{using}\isamarkupfalse%
\ map{\isacharunderscore}{\kern0pt}val{\isacharunderscore}{\kern0pt}in{\isacharunderscore}{\kern0pt}MG\ Forces{\isacharunderscore}{\kern0pt}And{\isacharbrackleft}{\kern0pt}OF\ M{\isacharunderscore}{\kern0pt}genericD\ assms{\isacharparenleft}{\kern0pt}{\isadigit{1}}{\isacharminus}{\kern0pt}{\isadigit{5}}{\isacharparenright}{\kern0pt}{\isacharbrackright}{\kern0pt}\ \isacommand{by}\isamarkupfalse%
\ auto\isanewline
\ \ \isacommand{moreover}\isamarkupfalse%
\ \isacommand{from}\isamarkupfalse%
\ calculation\isanewline
\ \ \isacommand{obtain}\isamarkupfalse%
\ p\ \isakeyword{where}\ {\isachardoublequoteopen}p{\isasympreceq}q{\isachardoublequoteclose}\ {\isachardoublequoteopen}p{\isasympreceq}r{\isachardoublequoteclose}\ {\isachardoublequoteopen}p{\isasymin}G{\isachardoublequoteclose}\isanewline
\ \ \ \ \isacommand{by}\isamarkupfalse%
\ blast\isanewline
\ \ \isacommand{moreover}\isamarkupfalse%
\ \isacommand{from}\isamarkupfalse%
\ calculation\isanewline
\ \ \isacommand{have}\isamarkupfalse%
\ {\isachardoublequoteopen}{\isacharparenleft}{\kern0pt}p\ {\isasymtturnstile}\ {\isasymphi}\ env{\isacharparenright}{\kern0pt}\ {\isasymand}\ {\isacharparenleft}{\kern0pt}p\ {\isasymtturnstile}\ {\isasympsi}\ env{\isacharparenright}{\kern0pt}{\isachardoublequoteclose}\ \isanewline
\ \ \ \ \isacommand{using}\isamarkupfalse%
\ strengthening{\isacharunderscore}{\kern0pt}lemma\ \isacommand{by}\isamarkupfalse%
\ {\isacharparenleft}{\kern0pt}blast{\isacharparenright}{\kern0pt}\isanewline
\ \ \isacommand{ultimately}\isamarkupfalse%
\isanewline
\ \ \isacommand{show}\isamarkupfalse%
\ {\isachardoublequoteopen}{\isasymexists}p{\isasymin}G{\isachardot}{\kern0pt}\ {\isacharparenleft}{\kern0pt}p\ {\isasymtturnstile}\ And{\isacharparenleft}{\kern0pt}{\isasymphi}{\isacharcomma}{\kern0pt}{\isasympsi}{\isacharparenright}{\kern0pt}\ env{\isacharparenright}{\kern0pt}{\isachardoublequoteclose}\isanewline
\ \ \ \ \isacommand{using}\isamarkupfalse%
\ Forces{\isacharunderscore}{\kern0pt}And{\isacharbrackleft}{\kern0pt}OF\ M{\isacharunderscore}{\kern0pt}genericD\ assms{\isacharparenleft}{\kern0pt}{\isadigit{1}}{\isacharminus}{\kern0pt}{\isadigit{5}}{\isacharparenright}{\kern0pt}{\isacharbrackright}{\kern0pt}\ \isacommand{by}\isamarkupfalse%
\ auto\isanewline
\isacommand{qed}\isamarkupfalse%
%
\endisatagproof
{\isafoldproof}%
%
\isadelimproof
\ \isanewline
%
\endisadelimproof
\isanewline
\isacommand{definition}\isamarkupfalse%
\ \isanewline
\ \ ren{\isacharunderscore}{\kern0pt}truth{\isacharunderscore}{\kern0pt}lemma\ {\isacharcolon}{\kern0pt}{\isacharcolon}{\kern0pt}\ {\isachardoublequoteopen}i{\isasymRightarrow}i{\isachardoublequoteclose}\ \isakeyword{where}\isanewline
\ \ {\isachardoublequoteopen}ren{\isacharunderscore}{\kern0pt}truth{\isacharunderscore}{\kern0pt}lemma{\isacharparenleft}{\kern0pt}{\isasymphi}{\isacharparenright}{\kern0pt}\ {\isasymequiv}\ \isanewline
\ \ \ \ Exists{\isacharparenleft}{\kern0pt}Exists{\isacharparenleft}{\kern0pt}Exists{\isacharparenleft}{\kern0pt}Exists{\isacharparenleft}{\kern0pt}Exists{\isacharparenleft}{\kern0pt}\isanewline
\ \ \ \ And{\isacharparenleft}{\kern0pt}Equal{\isacharparenleft}{\kern0pt}{\isadigit{0}}{\isacharcomma}{\kern0pt}{\isadigit{5}}{\isacharparenright}{\kern0pt}{\isacharcomma}{\kern0pt}And{\isacharparenleft}{\kern0pt}Equal{\isacharparenleft}{\kern0pt}{\isadigit{1}}{\isacharcomma}{\kern0pt}{\isadigit{8}}{\isacharparenright}{\kern0pt}{\isacharcomma}{\kern0pt}And{\isacharparenleft}{\kern0pt}Equal{\isacharparenleft}{\kern0pt}{\isadigit{2}}{\isacharcomma}{\kern0pt}{\isadigit{9}}{\isacharparenright}{\kern0pt}{\isacharcomma}{\kern0pt}And{\isacharparenleft}{\kern0pt}Equal{\isacharparenleft}{\kern0pt}{\isadigit{3}}{\isacharcomma}{\kern0pt}{\isadigit{1}}{\isadigit{0}}{\isacharparenright}{\kern0pt}{\isacharcomma}{\kern0pt}And{\isacharparenleft}{\kern0pt}Equal{\isacharparenleft}{\kern0pt}{\isadigit{4}}{\isacharcomma}{\kern0pt}{\isadigit{6}}{\isacharparenright}{\kern0pt}{\isacharcomma}{\kern0pt}\isanewline
\ \ \ \ iterates{\isacharparenleft}{\kern0pt}{\isasymlambda}p{\isachardot}{\kern0pt}\ incr{\isacharunderscore}{\kern0pt}bv{\isacharparenleft}{\kern0pt}p{\isacharparenright}{\kern0pt}{\isacharbackquote}{\kern0pt}{\isadigit{5}}\ {\isacharcomma}{\kern0pt}\ {\isadigit{6}}{\isacharcomma}{\kern0pt}\ {\isasymphi}{\isacharparenright}{\kern0pt}{\isacharparenright}{\kern0pt}{\isacharparenright}{\kern0pt}{\isacharparenright}{\kern0pt}{\isacharparenright}{\kern0pt}{\isacharparenright}{\kern0pt}{\isacharparenright}{\kern0pt}{\isacharparenright}{\kern0pt}{\isacharparenright}{\kern0pt}{\isacharparenright}{\kern0pt}{\isacharparenright}{\kern0pt}{\isachardoublequoteclose}\isanewline
\isanewline
\isacommand{lemma}\isamarkupfalse%
\ ren{\isacharunderscore}{\kern0pt}truth{\isacharunderscore}{\kern0pt}lemma{\isacharunderscore}{\kern0pt}type{\isacharbrackleft}{\kern0pt}TC{\isacharbrackright}{\kern0pt}\ {\isacharcolon}{\kern0pt}\isanewline
\ \ {\isachardoublequoteopen}{\isasymphi}{\isasymin}formula\ {\isasymLongrightarrow}\ ren{\isacharunderscore}{\kern0pt}truth{\isacharunderscore}{\kern0pt}lemma{\isacharparenleft}{\kern0pt}{\isasymphi}{\isacharparenright}{\kern0pt}\ {\isasymin}formula{\isachardoublequoteclose}\ \isanewline
%
\isadelimproof
\ \ %
\endisadelimproof
%
\isatagproof
\isacommand{unfolding}\isamarkupfalse%
\ ren{\isacharunderscore}{\kern0pt}truth{\isacharunderscore}{\kern0pt}lemma{\isacharunderscore}{\kern0pt}def\isanewline
\ \ \isacommand{by}\isamarkupfalse%
\ simp%
\endisatagproof
{\isafoldproof}%
%
\isadelimproof
\isanewline
%
\endisadelimproof
\isanewline
\isacommand{lemma}\isamarkupfalse%
\ arity{\isacharunderscore}{\kern0pt}ren{\isacharunderscore}{\kern0pt}truth\ {\isacharcolon}{\kern0pt}\ \isanewline
\ \ \isakeyword{assumes}\ {\isachardoublequoteopen}{\isasymphi}{\isasymin}formula{\isachardoublequoteclose}\isanewline
\ \ \isakeyword{shows}\ {\isachardoublequoteopen}arity{\isacharparenleft}{\kern0pt}ren{\isacharunderscore}{\kern0pt}truth{\isacharunderscore}{\kern0pt}lemma{\isacharparenleft}{\kern0pt}{\isasymphi}{\isacharparenright}{\kern0pt}{\isacharparenright}{\kern0pt}\ {\isasymle}\ {\isadigit{6}}\ {\isasymunion}\ succ{\isacharparenleft}{\kern0pt}arity{\isacharparenleft}{\kern0pt}{\isasymphi}{\isacharparenright}{\kern0pt}{\isacharparenright}{\kern0pt}{\isachardoublequoteclose}\isanewline
%
\isadelimproof
%
\endisadelimproof
%
\isatagproof
\isacommand{proof}\isamarkupfalse%
\ {\isacharminus}{\kern0pt}\isanewline
\ \ \isacommand{consider}\isamarkupfalse%
\ {\isacharparenleft}{\kern0pt}lt{\isacharparenright}{\kern0pt}\ {\isachardoublequoteopen}{\isadigit{5}}\ {\isacharless}{\kern0pt}arity{\isacharparenleft}{\kern0pt}{\isasymphi}{\isacharparenright}{\kern0pt}{\isachardoublequoteclose}\ {\isacharbar}{\kern0pt}\ {\isacharparenleft}{\kern0pt}ge{\isacharparenright}{\kern0pt}\ {\isachardoublequoteopen}{\isasymnot}\ {\isadigit{5}}\ {\isacharless}{\kern0pt}\ arity{\isacharparenleft}{\kern0pt}{\isasymphi}{\isacharparenright}{\kern0pt}{\isachardoublequoteclose}\isanewline
\ \ \ \ \isacommand{by}\isamarkupfalse%
\ auto\isanewline
\ \ \isacommand{then}\isamarkupfalse%
\isanewline
\ \ \isacommand{show}\isamarkupfalse%
\ {\isacharquery}{\kern0pt}thesis\isanewline
\ \ \isacommand{proof}\isamarkupfalse%
\ cases\isanewline
\ \ \ \ \isacommand{case}\isamarkupfalse%
\ lt\isanewline
\ \ \ \ \isacommand{consider}\isamarkupfalse%
\ {\isacharparenleft}{\kern0pt}a{\isacharparenright}{\kern0pt}\ {\isachardoublequoteopen}{\isadigit{5}}{\isacharless}{\kern0pt}arity{\isacharparenleft}{\kern0pt}{\isasymphi}{\isacharparenright}{\kern0pt}{\isacharhash}{\kern0pt}{\isacharplus}{\kern0pt}{\isadigit{5}}{\isachardoublequoteclose}\ \ {\isacharbar}{\kern0pt}\ {\isacharparenleft}{\kern0pt}b{\isacharparenright}{\kern0pt}\ {\isachardoublequoteopen}arity{\isacharparenleft}{\kern0pt}{\isasymphi}{\isacharparenright}{\kern0pt}{\isacharhash}{\kern0pt}{\isacharplus}{\kern0pt}{\isadigit{5}}\ {\isasymle}\ {\isadigit{5}}{\isachardoublequoteclose}\isanewline
\ \ \ \ \ \ \isacommand{using}\isamarkupfalse%
\ not{\isacharunderscore}{\kern0pt}lt{\isacharunderscore}{\kern0pt}iff{\isacharunderscore}{\kern0pt}le\ {\isacartoucheopen}{\isasymphi}{\isasymin}{\isacharunderscore}{\kern0pt}{\isacartoucheclose}\ \isacommand{by}\isamarkupfalse%
\ force\isanewline
\ \ \ \ \isacommand{then}\isamarkupfalse%
\ \isanewline
\ \ \ \ \isacommand{show}\isamarkupfalse%
\ {\isacharquery}{\kern0pt}thesis\isanewline
\ \ \ \ \isacommand{proof}\isamarkupfalse%
\ cases\isanewline
\ \ \ \ \ \ \isacommand{case}\isamarkupfalse%
\ a\isanewline
\ \ \ \ \ \ \isacommand{with}\isamarkupfalse%
\ {\isacartoucheopen}{\isasymphi}{\isasymin}{\isacharunderscore}{\kern0pt}{\isacartoucheclose}\ lt\isanewline
\ \ \ \ \ \ \isacommand{have}\isamarkupfalse%
\ {\isachardoublequoteopen}{\isadigit{5}}\ {\isacharless}{\kern0pt}\ succ{\isacharparenleft}{\kern0pt}arity{\isacharparenleft}{\kern0pt}{\isasymphi}{\isacharparenright}{\kern0pt}{\isacharparenright}{\kern0pt}{\isachardoublequoteclose}\ {\isachardoublequoteopen}{\isadigit{5}}{\isacharless}{\kern0pt}arity{\isacharparenleft}{\kern0pt}{\isasymphi}{\isacharparenright}{\kern0pt}{\isacharhash}{\kern0pt}{\isacharplus}{\kern0pt}{\isadigit{2}}{\isachardoublequoteclose}\ \ {\isachardoublequoteopen}{\isadigit{5}}{\isacharless}{\kern0pt}arity{\isacharparenleft}{\kern0pt}{\isasymphi}{\isacharparenright}{\kern0pt}{\isacharhash}{\kern0pt}{\isacharplus}{\kern0pt}{\isadigit{3}}{\isachardoublequoteclose}\ \ {\isachardoublequoteopen}{\isadigit{5}}{\isacharless}{\kern0pt}arity{\isacharparenleft}{\kern0pt}{\isasymphi}{\isacharparenright}{\kern0pt}{\isacharhash}{\kern0pt}{\isacharplus}{\kern0pt}{\isadigit{4}}{\isachardoublequoteclose}\isanewline
\ \ \ \ \ \ \ \ \isacommand{using}\isamarkupfalse%
\ succ{\isacharunderscore}{\kern0pt}ltI\ \isacommand{by}\isamarkupfalse%
\ auto\isanewline
\ \ \ \ \ \ \ \isacommand{with}\isamarkupfalse%
\ {\isacartoucheopen}{\isasymphi}{\isasymin}{\isacharunderscore}{\kern0pt}{\isacartoucheclose}\ \isanewline
\ \ \ \ \ \ \isacommand{have}\isamarkupfalse%
\ c{\isacharcolon}{\kern0pt}{\isachardoublequoteopen}arity{\isacharparenleft}{\kern0pt}iterates{\isacharparenleft}{\kern0pt}{\isasymlambda}p{\isachardot}{\kern0pt}\ incr{\isacharunderscore}{\kern0pt}bv{\isacharparenleft}{\kern0pt}p{\isacharparenright}{\kern0pt}{\isacharbackquote}{\kern0pt}{\isadigit{5}}{\isacharcomma}{\kern0pt}{\isadigit{5}}{\isacharcomma}{\kern0pt}{\isasymphi}{\isacharparenright}{\kern0pt}{\isacharparenright}{\kern0pt}\ {\isacharequal}{\kern0pt}\ {\isadigit{5}}{\isacharhash}{\kern0pt}{\isacharplus}{\kern0pt}arity{\isacharparenleft}{\kern0pt}{\isasymphi}{\isacharparenright}{\kern0pt}{\isachardoublequoteclose}\ {\isacharparenleft}{\kern0pt}\isakeyword{is}\ {\isachardoublequoteopen}arity{\isacharparenleft}{\kern0pt}{\isacharquery}{\kern0pt}{\isasymphi}{\isacharprime}{\kern0pt}{\isacharparenright}{\kern0pt}\ {\isacharequal}{\kern0pt}\ {\isacharunderscore}{\kern0pt}{\isachardoublequoteclose}{\isacharparenright}{\kern0pt}\ \isanewline
\ \ \ \ \ \ \ \ \isacommand{using}\isamarkupfalse%
\ arity{\isacharunderscore}{\kern0pt}incr{\isacharunderscore}{\kern0pt}bv{\isacharunderscore}{\kern0pt}lemma\ lt\ a\isanewline
\ \ \ \ \ \ \ \ \isacommand{by}\isamarkupfalse%
\ simp\isanewline
\ \ \ \ \ \ \isacommand{with}\isamarkupfalse%
\ {\isacartoucheopen}{\isasymphi}{\isasymin}{\isacharunderscore}{\kern0pt}{\isacartoucheclose}\isanewline
\ \ \ \ \ \ \isacommand{have}\isamarkupfalse%
\ {\isachardoublequoteopen}arity{\isacharparenleft}{\kern0pt}incr{\isacharunderscore}{\kern0pt}bv{\isacharparenleft}{\kern0pt}{\isacharquery}{\kern0pt}{\isasymphi}{\isacharprime}{\kern0pt}{\isacharparenright}{\kern0pt}{\isacharbackquote}{\kern0pt}{\isadigit{5}}{\isacharparenright}{\kern0pt}\ {\isacharequal}{\kern0pt}\ {\isadigit{6}}{\isacharhash}{\kern0pt}{\isacharplus}{\kern0pt}arity{\isacharparenleft}{\kern0pt}{\isasymphi}{\isacharparenright}{\kern0pt}{\isachardoublequoteclose}\isanewline
\ \ \ \ \ \ \ \ \isacommand{using}\isamarkupfalse%
\ arity{\isacharunderscore}{\kern0pt}incr{\isacharunderscore}{\kern0pt}bv{\isacharunderscore}{\kern0pt}lemma{\isacharbrackleft}{\kern0pt}of\ {\isacharquery}{\kern0pt}{\isasymphi}{\isacharprime}{\kern0pt}\ {\isadigit{5}}{\isacharbrackright}{\kern0pt}\ a\ \isacommand{by}\isamarkupfalse%
\ auto\isanewline
\ \ \ \ \ \ \isacommand{with}\isamarkupfalse%
\ {\isacartoucheopen}{\isasymphi}{\isasymin}{\isacharunderscore}{\kern0pt}{\isacartoucheclose}\isanewline
\ \ \ \ \ \ \isacommand{show}\isamarkupfalse%
\ {\isacharquery}{\kern0pt}thesis\isanewline
\ \ \ \ \ \ \ \ \isacommand{unfolding}\isamarkupfalse%
\ ren{\isacharunderscore}{\kern0pt}truth{\isacharunderscore}{\kern0pt}lemma{\isacharunderscore}{\kern0pt}def\isanewline
\ \ \ \ \ \ \ \ \isacommand{using}\isamarkupfalse%
\ pred{\isacharunderscore}{\kern0pt}Un{\isacharunderscore}{\kern0pt}distrib\ nat{\isacharunderscore}{\kern0pt}union{\isacharunderscore}{\kern0pt}abs{\isadigit{1}}\ Un{\isacharunderscore}{\kern0pt}assoc{\isacharbrackleft}{\kern0pt}symmetric{\isacharbrackright}{\kern0pt}\ a\ c\ nat{\isacharunderscore}{\kern0pt}union{\isacharunderscore}{\kern0pt}abs{\isadigit{2}}\isanewline
\ \ \ \ \ \ \ \ \isacommand{by}\isamarkupfalse%
\ simp\isanewline
\ \ \ \ \isacommand{next}\isamarkupfalse%
\isanewline
\ \ \ \ \ \ \isacommand{case}\isamarkupfalse%
\ b\isanewline
\ \ \ \ \ \ \isacommand{with}\isamarkupfalse%
\ {\isacartoucheopen}{\isasymphi}{\isasymin}{\isacharunderscore}{\kern0pt}{\isacartoucheclose}\ lt\isanewline
\ \ \ \ \ \ \isacommand{have}\isamarkupfalse%
\ {\isachardoublequoteopen}{\isadigit{5}}\ {\isacharless}{\kern0pt}\ succ{\isacharparenleft}{\kern0pt}arity{\isacharparenleft}{\kern0pt}{\isasymphi}{\isacharparenright}{\kern0pt}{\isacharparenright}{\kern0pt}{\isachardoublequoteclose}\ {\isachardoublequoteopen}{\isadigit{5}}{\isacharless}{\kern0pt}arity{\isacharparenleft}{\kern0pt}{\isasymphi}{\isacharparenright}{\kern0pt}{\isacharhash}{\kern0pt}{\isacharplus}{\kern0pt}{\isadigit{2}}{\isachardoublequoteclose}\ \ {\isachardoublequoteopen}{\isadigit{5}}{\isacharless}{\kern0pt}arity{\isacharparenleft}{\kern0pt}{\isasymphi}{\isacharparenright}{\kern0pt}{\isacharhash}{\kern0pt}{\isacharplus}{\kern0pt}{\isadigit{3}}{\isachardoublequoteclose}\ \ {\isachardoublequoteopen}{\isadigit{5}}{\isacharless}{\kern0pt}arity{\isacharparenleft}{\kern0pt}{\isasymphi}{\isacharparenright}{\kern0pt}{\isacharhash}{\kern0pt}{\isacharplus}{\kern0pt}{\isadigit{4}}{\isachardoublequoteclose}\ {\isachardoublequoteopen}{\isadigit{5}}{\isacharless}{\kern0pt}arity{\isacharparenleft}{\kern0pt}{\isasymphi}{\isacharparenright}{\kern0pt}{\isacharhash}{\kern0pt}{\isacharplus}{\kern0pt}{\isadigit{5}}{\isachardoublequoteclose}\isanewline
\ \ \ \ \ \ \ \ \isacommand{using}\isamarkupfalse%
\ succ{\isacharunderscore}{\kern0pt}ltI\ \isacommand{by}\isamarkupfalse%
\ auto\isanewline
\ \ \ \ \ \ \isacommand{with}\isamarkupfalse%
\ {\isacartoucheopen}{\isasymphi}{\isasymin}{\isacharunderscore}{\kern0pt}{\isacartoucheclose}\ \isanewline
\ \ \ \ \ \ \isacommand{have}\isamarkupfalse%
\ {\isachardoublequoteopen}arity{\isacharparenleft}{\kern0pt}iterates{\isacharparenleft}{\kern0pt}{\isasymlambda}p{\isachardot}{\kern0pt}\ incr{\isacharunderscore}{\kern0pt}bv{\isacharparenleft}{\kern0pt}p{\isacharparenright}{\kern0pt}{\isacharbackquote}{\kern0pt}{\isadigit{5}}{\isacharcomma}{\kern0pt}{\isadigit{6}}{\isacharcomma}{\kern0pt}{\isasymphi}{\isacharparenright}{\kern0pt}{\isacharparenright}{\kern0pt}\ {\isacharequal}{\kern0pt}\ {\isadigit{6}}{\isacharhash}{\kern0pt}{\isacharplus}{\kern0pt}arity{\isacharparenleft}{\kern0pt}{\isasymphi}{\isacharparenright}{\kern0pt}{\isachardoublequoteclose}\ {\isacharparenleft}{\kern0pt}\isakeyword{is}\ {\isachardoublequoteopen}arity{\isacharparenleft}{\kern0pt}{\isacharquery}{\kern0pt}{\isasymphi}{\isacharprime}{\kern0pt}{\isacharparenright}{\kern0pt}\ {\isacharequal}{\kern0pt}\ {\isacharunderscore}{\kern0pt}{\isachardoublequoteclose}{\isacharparenright}{\kern0pt}\ \isanewline
\ \ \ \ \ \ \ \ \isacommand{using}\isamarkupfalse%
\ arity{\isacharunderscore}{\kern0pt}incr{\isacharunderscore}{\kern0pt}bv{\isacharunderscore}{\kern0pt}lemma\ lt\ \isanewline
\ \ \ \ \ \ \ \ \isacommand{by}\isamarkupfalse%
\ simp\isanewline
\ \ \ \ \ \ \isacommand{with}\isamarkupfalse%
\ {\isacartoucheopen}{\isasymphi}{\isasymin}{\isacharunderscore}{\kern0pt}{\isacartoucheclose}\isanewline
\ \ \ \ \ \ \isacommand{show}\isamarkupfalse%
\ {\isacharquery}{\kern0pt}thesis\isanewline
\ \ \ \ \ \ \ \ \isacommand{unfolding}\isamarkupfalse%
\ ren{\isacharunderscore}{\kern0pt}truth{\isacharunderscore}{\kern0pt}lemma{\isacharunderscore}{\kern0pt}def\isanewline
\ \ \ \ \ \ \ \ \isacommand{using}\isamarkupfalse%
\ pred{\isacharunderscore}{\kern0pt}Un{\isacharunderscore}{\kern0pt}distrib\ nat{\isacharunderscore}{\kern0pt}union{\isacharunderscore}{\kern0pt}abs{\isadigit{1}}\ Un{\isacharunderscore}{\kern0pt}assoc{\isacharbrackleft}{\kern0pt}symmetric{\isacharbrackright}{\kern0pt}\ \ nat{\isacharunderscore}{\kern0pt}union{\isacharunderscore}{\kern0pt}abs{\isadigit{2}}\isanewline
\ \ \ \ \ \ \ \ \isacommand{by}\isamarkupfalse%
\ simp\isanewline
\ \ \ \ \isacommand{qed}\isamarkupfalse%
\isanewline
\ \ \isacommand{next}\isamarkupfalse%
\isanewline
\ \ \ \ \isacommand{case}\isamarkupfalse%
\ ge\isanewline
\ \ \ \ \isacommand{with}\isamarkupfalse%
\ {\isacartoucheopen}{\isasymphi}{\isasymin}{\isacharunderscore}{\kern0pt}{\isacartoucheclose}\isanewline
\ \ \ \ \isacommand{have}\isamarkupfalse%
\ {\isachardoublequoteopen}arity{\isacharparenleft}{\kern0pt}{\isasymphi}{\isacharparenright}{\kern0pt}\ {\isasymle}\ {\isadigit{5}}{\isachardoublequoteclose}\ {\isachardoublequoteopen}pred{\isacharcircum}{\kern0pt}{\isadigit{5}}{\isacharparenleft}{\kern0pt}arity{\isacharparenleft}{\kern0pt}{\isasymphi}{\isacharparenright}{\kern0pt}{\isacharparenright}{\kern0pt}\ {\isasymle}\ {\isadigit{5}}{\isachardoublequoteclose}\isanewline
\ \ \ \ \ \ \isacommand{using}\isamarkupfalse%
\ not{\isacharunderscore}{\kern0pt}lt{\isacharunderscore}{\kern0pt}iff{\isacharunderscore}{\kern0pt}le\ le{\isacharunderscore}{\kern0pt}trans{\isacharbrackleft}{\kern0pt}OF\ le{\isacharunderscore}{\kern0pt}pred{\isacharbrackright}{\kern0pt}\isanewline
\ \ \ \ \ \ \isacommand{by}\isamarkupfalse%
\ auto\isanewline
\ \ \ \ \isacommand{with}\isamarkupfalse%
\ {\isacartoucheopen}{\isasymphi}{\isasymin}{\isacharunderscore}{\kern0pt}{\isacartoucheclose}\isanewline
\ \ \ \ \isacommand{have}\isamarkupfalse%
\ {\isachardoublequoteopen}arity{\isacharparenleft}{\kern0pt}iterates{\isacharparenleft}{\kern0pt}{\isasymlambda}p{\isachardot}{\kern0pt}\ incr{\isacharunderscore}{\kern0pt}bv{\isacharparenleft}{\kern0pt}p{\isacharparenright}{\kern0pt}{\isacharbackquote}{\kern0pt}{\isadigit{5}}{\isacharcomma}{\kern0pt}{\isadigit{6}}{\isacharcomma}{\kern0pt}{\isasymphi}{\isacharparenright}{\kern0pt}{\isacharparenright}{\kern0pt}\ {\isacharequal}{\kern0pt}\ arity{\isacharparenleft}{\kern0pt}{\isasymphi}{\isacharparenright}{\kern0pt}{\isachardoublequoteclose}\ {\isachardoublequoteopen}arity{\isacharparenleft}{\kern0pt}{\isasymphi}{\isacharparenright}{\kern0pt}{\isasymle}{\isadigit{6}}{\isachardoublequoteclose}\ {\isachardoublequoteopen}pred{\isacharcircum}{\kern0pt}{\isadigit{5}}{\isacharparenleft}{\kern0pt}arity{\isacharparenleft}{\kern0pt}{\isasymphi}{\isacharparenright}{\kern0pt}{\isacharparenright}{\kern0pt}\ {\isasymle}\ {\isadigit{6}}{\isachardoublequoteclose}\isanewline
\ \ \ \ \ \ \isacommand{using}\isamarkupfalse%
\ arity{\isacharunderscore}{\kern0pt}incr{\isacharunderscore}{\kern0pt}bv{\isacharunderscore}{\kern0pt}lemma\ ge\ le{\isacharunderscore}{\kern0pt}trans{\isacharbrackleft}{\kern0pt}OF\ {\isacartoucheopen}arity{\isacharparenleft}{\kern0pt}{\isasymphi}{\isacharparenright}{\kern0pt}{\isasymle}{\isadigit{5}}{\isacartoucheclose}{\isacharbrackright}{\kern0pt}\ le{\isacharunderscore}{\kern0pt}trans{\isacharbrackleft}{\kern0pt}OF\ {\isacartoucheopen}pred{\isacharcircum}{\kern0pt}{\isadigit{5}}{\isacharparenleft}{\kern0pt}arity{\isacharparenleft}{\kern0pt}{\isasymphi}{\isacharparenright}{\kern0pt}{\isacharparenright}{\kern0pt}{\isasymle}{\isadigit{5}}{\isacartoucheclose}{\isacharbrackright}{\kern0pt}\isanewline
\ \ \ \ \ \ \isacommand{by}\isamarkupfalse%
\ auto\isanewline
\ \ \ \ \isacommand{with}\isamarkupfalse%
\ {\isacartoucheopen}arity{\isacharparenleft}{\kern0pt}{\isasymphi}{\isacharparenright}{\kern0pt}\ {\isasymle}\ {\isadigit{5}}{\isacartoucheclose}\ {\isacartoucheopen}{\isasymphi}{\isasymin}{\isacharunderscore}{\kern0pt}{\isacartoucheclose}\ {\isacartoucheopen}pred{\isacharcircum}{\kern0pt}{\isadigit{5}}{\isacharparenleft}{\kern0pt}{\isacharunderscore}{\kern0pt}{\isacharparenright}{\kern0pt}\ {\isasymle}\ {\isadigit{5}}{\isacartoucheclose}\isanewline
\ \ \ \ \isacommand{show}\isamarkupfalse%
\ {\isacharquery}{\kern0pt}thesis\isanewline
\ \ \ \ \ \ \isacommand{unfolding}\isamarkupfalse%
\ ren{\isacharunderscore}{\kern0pt}truth{\isacharunderscore}{\kern0pt}lemma{\isacharunderscore}{\kern0pt}def\isanewline
\ \ \ \ \ \ \isacommand{using}\isamarkupfalse%
\ \ pred{\isacharunderscore}{\kern0pt}Un{\isacharunderscore}{\kern0pt}distrib\ nat{\isacharunderscore}{\kern0pt}union{\isacharunderscore}{\kern0pt}abs{\isadigit{1}}\ Un{\isacharunderscore}{\kern0pt}assoc{\isacharbrackleft}{\kern0pt}symmetric{\isacharbrackright}{\kern0pt}\ nat{\isacharunderscore}{\kern0pt}union{\isacharunderscore}{\kern0pt}abs{\isadigit{2}}\ \isanewline
\ \ \ \ \ \ \isacommand{by}\isamarkupfalse%
\ simp\isanewline
\ \ \isacommand{qed}\isamarkupfalse%
\isanewline
\isacommand{qed}\isamarkupfalse%
%
\endisatagproof
{\isafoldproof}%
%
\isadelimproof
\isanewline
%
\endisadelimproof
\isanewline
\isacommand{lemma}\isamarkupfalse%
\ sats{\isacharunderscore}{\kern0pt}ren{\isacharunderscore}{\kern0pt}truth{\isacharunderscore}{\kern0pt}lemma{\isacharcolon}{\kern0pt}\isanewline
\ \ {\isachardoublequoteopen}{\isacharbrackleft}{\kern0pt}q{\isacharcomma}{\kern0pt}b{\isacharcomma}{\kern0pt}d{\isacharcomma}{\kern0pt}a{\isadigit{1}}{\isacharcomma}{\kern0pt}a{\isadigit{2}}{\isacharcomma}{\kern0pt}a{\isadigit{3}}{\isacharbrackright}{\kern0pt}\ {\isacharat}{\kern0pt}\ env\ {\isasymin}\ list{\isacharparenleft}{\kern0pt}M{\isacharparenright}{\kern0pt}\ {\isasymLongrightarrow}\ {\isasymphi}\ {\isasymin}\ formula\ {\isasymLongrightarrow}\isanewline
\ \ \ {\isacharparenleft}{\kern0pt}M{\isacharcomma}{\kern0pt}\ {\isacharbrackleft}{\kern0pt}q{\isacharcomma}{\kern0pt}b{\isacharcomma}{\kern0pt}d{\isacharcomma}{\kern0pt}a{\isadigit{1}}{\isacharcomma}{\kern0pt}a{\isadigit{2}}{\isacharcomma}{\kern0pt}a{\isadigit{3}}{\isacharbrackright}{\kern0pt}\ {\isacharat}{\kern0pt}\ env\ {\isasymTurnstile}\ ren{\isacharunderscore}{\kern0pt}truth{\isacharunderscore}{\kern0pt}lemma{\isacharparenleft}{\kern0pt}{\isasymphi}{\isacharparenright}{\kern0pt}\ {\isacharparenright}{\kern0pt}\ {\isasymlongleftrightarrow}\isanewline
\ \ \ {\isacharparenleft}{\kern0pt}M{\isacharcomma}{\kern0pt}\ {\isacharbrackleft}{\kern0pt}q{\isacharcomma}{\kern0pt}a{\isadigit{1}}{\isacharcomma}{\kern0pt}a{\isadigit{2}}{\isacharcomma}{\kern0pt}a{\isadigit{3}}{\isacharcomma}{\kern0pt}b{\isacharbrackright}{\kern0pt}\ {\isacharat}{\kern0pt}\ env\ {\isasymTurnstile}\ {\isasymphi}{\isacharparenright}{\kern0pt}{\isachardoublequoteclose}\isanewline
%
\isadelimproof
\ \ %
\endisadelimproof
%
\isatagproof
\isacommand{unfolding}\isamarkupfalse%
\ ren{\isacharunderscore}{\kern0pt}truth{\isacharunderscore}{\kern0pt}lemma{\isacharunderscore}{\kern0pt}def\isanewline
\ \ \isacommand{by}\isamarkupfalse%
\ {\isacharparenleft}{\kern0pt}insert\ sats{\isacharunderscore}{\kern0pt}incr{\isacharunderscore}{\kern0pt}bv{\isacharunderscore}{\kern0pt}iff\ {\isacharbrackleft}{\kern0pt}of\ {\isacharunderscore}{\kern0pt}\ {\isacharunderscore}{\kern0pt}\ M\ {\isacharunderscore}{\kern0pt}\ {\isachardoublequoteopen}{\isacharbrackleft}{\kern0pt}q{\isacharcomma}{\kern0pt}a{\isadigit{1}}{\isacharcomma}{\kern0pt}a{\isadigit{2}}{\isacharcomma}{\kern0pt}a{\isadigit{3}}{\isacharcomma}{\kern0pt}b{\isacharbrackright}{\kern0pt}{\isachardoublequoteclose}{\isacharbrackright}{\kern0pt}{\isacharcomma}{\kern0pt}\ simp{\isacharparenright}{\kern0pt}%
\endisatagproof
{\isafoldproof}%
%
\isadelimproof
\isanewline
%
\endisadelimproof
\isanewline
\isacommand{lemma}\isamarkupfalse%
\ truth{\isacharunderscore}{\kern0pt}lemma{\isacharprime}{\kern0pt}\ {\isacharcolon}{\kern0pt}\isanewline
\ \ \isakeyword{assumes}\isanewline
\ \ \ \ {\isachardoublequoteopen}{\isasymphi}{\isasymin}formula{\isachardoublequoteclose}\ {\isachardoublequoteopen}env{\isasymin}list{\isacharparenleft}{\kern0pt}M{\isacharparenright}{\kern0pt}{\isachardoublequoteclose}\ {\isachardoublequoteopen}arity{\isacharparenleft}{\kern0pt}{\isasymphi}{\isacharparenright}{\kern0pt}\ {\isasymle}\ succ{\isacharparenleft}{\kern0pt}length{\isacharparenleft}{\kern0pt}env{\isacharparenright}{\kern0pt}{\isacharparenright}{\kern0pt}{\isachardoublequoteclose}\ \isanewline
\ \ \isakeyword{shows}\isanewline
\ \ \ \ {\isachardoublequoteopen}separation{\isacharparenleft}{\kern0pt}{\isacharhash}{\kern0pt}{\isacharhash}{\kern0pt}M{\isacharcomma}{\kern0pt}{\isasymlambda}d{\isachardot}{\kern0pt}\ {\isasymexists}b{\isasymin}M{\isachardot}{\kern0pt}\ {\isasymforall}q{\isasymin}P{\isachardot}{\kern0pt}\ q{\isasympreceq}d\ {\isasymlongrightarrow}\ {\isasymnot}{\isacharparenleft}{\kern0pt}q\ {\isasymtturnstile}\ {\isasymphi}\ {\isacharparenleft}{\kern0pt}{\isacharbrackleft}{\kern0pt}b{\isacharbrackright}{\kern0pt}{\isacharat}{\kern0pt}env{\isacharparenright}{\kern0pt}{\isacharparenright}{\kern0pt}{\isacharparenright}{\kern0pt}{\isachardoublequoteclose}\isanewline
%
\isadelimproof
%
\endisadelimproof
%
\isatagproof
\isacommand{proof}\isamarkupfalse%
\ {\isacharminus}{\kern0pt}\isanewline
\ \ \isacommand{let}\isamarkupfalse%
\ {\isacharquery}{\kern0pt}rel{\isacharunderscore}{\kern0pt}pred{\isacharequal}{\kern0pt}{\isachardoublequoteopen}{\isasymlambda}M\ x\ a{\isadigit{1}}\ a{\isadigit{2}}\ a{\isadigit{3}}{\isachardot}{\kern0pt}\ {\isasymexists}b{\isasymin}M{\isachardot}{\kern0pt}\ {\isasymforall}q{\isasymin}M{\isachardot}{\kern0pt}\ q{\isasymin}a{\isadigit{1}}\ {\isasymand}\ is{\isacharunderscore}{\kern0pt}leq{\isacharparenleft}{\kern0pt}{\isacharhash}{\kern0pt}{\isacharhash}{\kern0pt}M{\isacharcomma}{\kern0pt}a{\isadigit{2}}{\isacharcomma}{\kern0pt}q{\isacharcomma}{\kern0pt}x{\isacharparenright}{\kern0pt}\ {\isasymlongrightarrow}\ \isanewline
\ \ \ \ \ \ \ \ \ \ \ \ \ \ \ \ \ \ {\isasymnot}{\isacharparenleft}{\kern0pt}M{\isacharcomma}{\kern0pt}\ {\isacharbrackleft}{\kern0pt}q{\isacharcomma}{\kern0pt}a{\isadigit{1}}{\isacharcomma}{\kern0pt}a{\isadigit{2}}{\isacharcomma}{\kern0pt}a{\isadigit{3}}{\isacharcomma}{\kern0pt}b{\isacharbrackright}{\kern0pt}\ {\isacharat}{\kern0pt}\ env\ {\isasymTurnstile}\ forces{\isacharparenleft}{\kern0pt}{\isasymphi}{\isacharparenright}{\kern0pt}{\isacharparenright}{\kern0pt}{\isachardoublequoteclose}\ \isanewline
\ \ \isacommand{let}\isamarkupfalse%
\ {\isacharquery}{\kern0pt}{\isasympsi}{\isacharequal}{\kern0pt}{\isachardoublequoteopen}Exists{\isacharparenleft}{\kern0pt}Forall{\isacharparenleft}{\kern0pt}Implies{\isacharparenleft}{\kern0pt}And{\isacharparenleft}{\kern0pt}Member{\isacharparenleft}{\kern0pt}{\isadigit{0}}{\isacharcomma}{\kern0pt}{\isadigit{3}}{\isacharparenright}{\kern0pt}{\isacharcomma}{\kern0pt}leq{\isacharunderscore}{\kern0pt}fm{\isacharparenleft}{\kern0pt}{\isadigit{4}}{\isacharcomma}{\kern0pt}{\isadigit{0}}{\isacharcomma}{\kern0pt}{\isadigit{2}}{\isacharparenright}{\kern0pt}{\isacharparenright}{\kern0pt}{\isacharcomma}{\kern0pt}\isanewline
\ \ \ \ \ \ \ \ \ \ Neg{\isacharparenleft}{\kern0pt}ren{\isacharunderscore}{\kern0pt}truth{\isacharunderscore}{\kern0pt}lemma{\isacharparenleft}{\kern0pt}forces{\isacharparenleft}{\kern0pt}{\isasymphi}{\isacharparenright}{\kern0pt}{\isacharparenright}{\kern0pt}{\isacharparenright}{\kern0pt}{\isacharparenright}{\kern0pt}{\isacharparenright}{\kern0pt}{\isacharparenright}{\kern0pt}{\isachardoublequoteclose}\isanewline
\ \ \isacommand{have}\isamarkupfalse%
\ {\isachardoublequoteopen}q{\isasymin}M{\isachardoublequoteclose}\ \isakeyword{if}\ {\isachardoublequoteopen}q{\isasymin}P{\isachardoublequoteclose}\ \isakeyword{for}\ q\ \isacommand{using}\isamarkupfalse%
\ that\ transitivity{\isacharbrackleft}{\kern0pt}OF\ {\isacharunderscore}{\kern0pt}\ P{\isacharunderscore}{\kern0pt}in{\isacharunderscore}{\kern0pt}M{\isacharbrackright}{\kern0pt}\ \isacommand{by}\isamarkupfalse%
\ simp\isanewline
\ \ \isacommand{then}\isamarkupfalse%
\isanewline
\ \ \isacommand{have}\isamarkupfalse%
\ {\isadigit{1}}{\isacharcolon}{\kern0pt}{\isachardoublequoteopen}{\isasymforall}q{\isasymin}M{\isachardot}{\kern0pt}\ q{\isasymin}P\ {\isasymand}\ R{\isacharparenleft}{\kern0pt}q{\isacharparenright}{\kern0pt}\ {\isasymlongrightarrow}\ Q{\isacharparenleft}{\kern0pt}q{\isacharparenright}{\kern0pt}\ {\isasymLongrightarrow}\ {\isacharparenleft}{\kern0pt}{\isasymforall}q{\isasymin}P{\isachardot}{\kern0pt}\ R{\isacharparenleft}{\kern0pt}q{\isacharparenright}{\kern0pt}\ {\isasymlongrightarrow}\ Q{\isacharparenleft}{\kern0pt}q{\isacharparenright}{\kern0pt}{\isacharparenright}{\kern0pt}{\isachardoublequoteclose}\ \isakeyword{for}\ R\ Q\ \isanewline
\ \ \ \ \isacommand{by}\isamarkupfalse%
\ auto\isanewline
\ \ \isacommand{then}\isamarkupfalse%
\isanewline
\ \ \isacommand{have}\isamarkupfalse%
\ {\isachardoublequoteopen}{\isasymlbrakk}b\ {\isasymin}\ M{\isacharsemicolon}{\kern0pt}\ {\isasymforall}q{\isasymin}M{\isachardot}{\kern0pt}\ q\ {\isasymin}\ P\ {\isasymand}\ q\ {\isasympreceq}\ d\ {\isasymlongrightarrow}\ {\isasymnot}{\isacharparenleft}{\kern0pt}q\ {\isasymtturnstile}\ {\isasymphi}\ {\isacharparenleft}{\kern0pt}{\isacharbrackleft}{\kern0pt}b{\isacharbrackright}{\kern0pt}{\isacharat}{\kern0pt}env{\isacharparenright}{\kern0pt}{\isacharparenright}{\kern0pt}{\isasymrbrakk}\ {\isasymLongrightarrow}\isanewline
\ \ \ \ \ \ \ \ \ {\isasymexists}c{\isasymin}M{\isachardot}{\kern0pt}\ {\isasymforall}q{\isasymin}P{\isachardot}{\kern0pt}\ q\ {\isasympreceq}\ d\ {\isasymlongrightarrow}\ {\isasymnot}{\isacharparenleft}{\kern0pt}q\ {\isasymtturnstile}\ {\isasymphi}\ {\isacharparenleft}{\kern0pt}{\isacharbrackleft}{\kern0pt}c{\isacharbrackright}{\kern0pt}{\isacharat}{\kern0pt}env{\isacharparenright}{\kern0pt}{\isacharparenright}{\kern0pt}{\isachardoublequoteclose}\ \isakeyword{for}\ b\ d\isanewline
\ \ \ \ \isacommand{by}\isamarkupfalse%
\ {\isacharparenleft}{\kern0pt}rule\ bexI{\isacharcomma}{\kern0pt}simp{\isacharunderscore}{\kern0pt}all{\isacharparenright}{\kern0pt}\isanewline
\ \ \isacommand{then}\isamarkupfalse%
\isanewline
\ \ \isacommand{have}\isamarkupfalse%
\ {\isachardoublequoteopen}{\isacharquery}{\kern0pt}rel{\isacharunderscore}{\kern0pt}pred{\isacharparenleft}{\kern0pt}M{\isacharcomma}{\kern0pt}d{\isacharcomma}{\kern0pt}P{\isacharcomma}{\kern0pt}leq{\isacharcomma}{\kern0pt}one{\isacharparenright}{\kern0pt}\ {\isasymlongleftrightarrow}\ {\isacharparenleft}{\kern0pt}{\isasymexists}b{\isasymin}M{\isachardot}{\kern0pt}\ {\isasymforall}q{\isasymin}P{\isachardot}{\kern0pt}\ q{\isasympreceq}d\ {\isasymlongrightarrow}\ {\isasymnot}{\isacharparenleft}{\kern0pt}q\ {\isasymtturnstile}\ {\isasymphi}\ {\isacharparenleft}{\kern0pt}{\isacharbrackleft}{\kern0pt}b{\isacharbrackright}{\kern0pt}{\isacharat}{\kern0pt}env{\isacharparenright}{\kern0pt}{\isacharparenright}{\kern0pt}{\isacharparenright}{\kern0pt}{\isachardoublequoteclose}\ \isakeyword{if}\ {\isachardoublequoteopen}d{\isasymin}M{\isachardoublequoteclose}\ \isakeyword{for}\ d\isanewline
\ \ \ \ \isacommand{using}\isamarkupfalse%
\ that\ leq{\isacharunderscore}{\kern0pt}abs\ leq{\isacharunderscore}{\kern0pt}in{\isacharunderscore}{\kern0pt}M\ P{\isacharunderscore}{\kern0pt}in{\isacharunderscore}{\kern0pt}M\ one{\isacharunderscore}{\kern0pt}in{\isacharunderscore}{\kern0pt}M\ assms\isanewline
\ \ \ \ \isacommand{by}\isamarkupfalse%
\ auto\isanewline
\ \ \isacommand{moreover}\isamarkupfalse%
\isanewline
\ \ \isacommand{have}\isamarkupfalse%
\ {\isachardoublequoteopen}{\isacharquery}{\kern0pt}{\isasympsi}{\isasymin}formula{\isachardoublequoteclose}\ \isacommand{using}\isamarkupfalse%
\ assms\ \isacommand{by}\isamarkupfalse%
\ simp\isanewline
\ \ \isacommand{moreover}\isamarkupfalse%
\isanewline
\ \ \isacommand{have}\isamarkupfalse%
\ {\isachardoublequoteopen}{\isacharparenleft}{\kern0pt}M{\isacharcomma}{\kern0pt}\ {\isacharbrackleft}{\kern0pt}d{\isacharcomma}{\kern0pt}P{\isacharcomma}{\kern0pt}leq{\isacharcomma}{\kern0pt}one{\isacharbrackright}{\kern0pt}{\isacharat}{\kern0pt}env\ {\isasymTurnstile}\ {\isacharquery}{\kern0pt}{\isasympsi}{\isacharparenright}{\kern0pt}\ {\isasymlongleftrightarrow}\ {\isacharquery}{\kern0pt}rel{\isacharunderscore}{\kern0pt}pred{\isacharparenleft}{\kern0pt}M{\isacharcomma}{\kern0pt}d{\isacharcomma}{\kern0pt}P{\isacharcomma}{\kern0pt}leq{\isacharcomma}{\kern0pt}one{\isacharparenright}{\kern0pt}{\isachardoublequoteclose}\ \isakeyword{if}\ {\isachardoublequoteopen}d{\isasymin}M{\isachardoublequoteclose}\ \isakeyword{for}\ d\isanewline
\ \ \ \ \isacommand{using}\isamarkupfalse%
\ assms\ that\ P{\isacharunderscore}{\kern0pt}in{\isacharunderscore}{\kern0pt}M\ leq{\isacharunderscore}{\kern0pt}in{\isacharunderscore}{\kern0pt}M\ one{\isacharunderscore}{\kern0pt}in{\isacharunderscore}{\kern0pt}M\ sats{\isacharunderscore}{\kern0pt}leq{\isacharunderscore}{\kern0pt}fm\ sats{\isacharunderscore}{\kern0pt}ren{\isacharunderscore}{\kern0pt}truth{\isacharunderscore}{\kern0pt}lemma\isanewline
\ \ \ \ \isacommand{by}\isamarkupfalse%
\ simp\isanewline
\ \ \isacommand{moreover}\isamarkupfalse%
\isanewline
\ \ \isacommand{have}\isamarkupfalse%
\ {\isachardoublequoteopen}arity{\isacharparenleft}{\kern0pt}{\isacharquery}{\kern0pt}{\isasympsi}{\isacharparenright}{\kern0pt}\ {\isasymle}\ {\isadigit{4}}{\isacharhash}{\kern0pt}{\isacharplus}{\kern0pt}length{\isacharparenleft}{\kern0pt}env{\isacharparenright}{\kern0pt}{\isachardoublequoteclose}\ \isanewline
\ \ \isacommand{proof}\isamarkupfalse%
\ {\isacharminus}{\kern0pt}\isanewline
\ \ \ \ \isacommand{have}\isamarkupfalse%
\ eq{\isacharcolon}{\kern0pt}{\isachardoublequoteopen}arity{\isacharparenleft}{\kern0pt}leq{\isacharunderscore}{\kern0pt}fm{\isacharparenleft}{\kern0pt}{\isadigit{4}}{\isacharcomma}{\kern0pt}\ {\isadigit{0}}{\isacharcomma}{\kern0pt}\ {\isadigit{2}}{\isacharparenright}{\kern0pt}{\isacharparenright}{\kern0pt}\ {\isacharequal}{\kern0pt}\ {\isadigit{5}}{\isachardoublequoteclose}\isanewline
\ \ \ \ \ \ \isacommand{using}\isamarkupfalse%
\ arity{\isacharunderscore}{\kern0pt}leq{\isacharunderscore}{\kern0pt}fm\ succ{\isacharunderscore}{\kern0pt}Un{\isacharunderscore}{\kern0pt}distrib\ nat{\isacharunderscore}{\kern0pt}simp{\isacharunderscore}{\kern0pt}union\isanewline
\ \ \ \ \ \ \isacommand{by}\isamarkupfalse%
\ simp\isanewline
\ \ \ \ \isacommand{with}\isamarkupfalse%
\ {\isacartoucheopen}{\isasymphi}{\isasymin}{\isacharunderscore}{\kern0pt}{\isacartoucheclose}\isanewline
\ \ \ \ \isacommand{have}\isamarkupfalse%
\ {\isachardoublequoteopen}arity{\isacharparenleft}{\kern0pt}{\isacharquery}{\kern0pt}{\isasympsi}{\isacharparenright}{\kern0pt}\ {\isacharequal}{\kern0pt}\ {\isadigit{3}}\ {\isasymunion}\ {\isacharparenleft}{\kern0pt}pred{\isacharcircum}{\kern0pt}{\isadigit{2}}{\isacharparenleft}{\kern0pt}arity{\isacharparenleft}{\kern0pt}ren{\isacharunderscore}{\kern0pt}truth{\isacharunderscore}{\kern0pt}lemma{\isacharparenleft}{\kern0pt}forces{\isacharparenleft}{\kern0pt}{\isasymphi}{\isacharparenright}{\kern0pt}{\isacharparenright}{\kern0pt}{\isacharparenright}{\kern0pt}{\isacharparenright}{\kern0pt}{\isacharparenright}{\kern0pt}{\isachardoublequoteclose}\isanewline
\ \ \ \ \ \ \isacommand{using}\isamarkupfalse%
\ nat{\isacharunderscore}{\kern0pt}union{\isacharunderscore}{\kern0pt}abs{\isadigit{1}}\ pred{\isacharunderscore}{\kern0pt}Un{\isacharunderscore}{\kern0pt}distrib\ \isacommand{by}\isamarkupfalse%
\ simp\isanewline
\ \ \ \ \isacommand{moreover}\isamarkupfalse%
\isanewline
\ \ \ \ \isacommand{have}\isamarkupfalse%
\ {\isachardoublequoteopen}{\isachardot}{\kern0pt}{\isachardot}{\kern0pt}{\isachardot}{\kern0pt}\ {\isasymle}\ {\isadigit{3}}\ {\isasymunion}\ {\isacharparenleft}{\kern0pt}pred{\isacharparenleft}{\kern0pt}pred{\isacharparenleft}{\kern0pt}{\isadigit{6}}\ {\isasymunion}\ succ{\isacharparenleft}{\kern0pt}arity{\isacharparenleft}{\kern0pt}forces{\isacharparenleft}{\kern0pt}{\isasymphi}{\isacharparenright}{\kern0pt}{\isacharparenright}{\kern0pt}{\isacharparenright}{\kern0pt}{\isacharparenright}{\kern0pt}{\isacharparenright}{\kern0pt}{\isacharparenright}{\kern0pt}{\isachardoublequoteclose}\ {\isacharparenleft}{\kern0pt}\isakeyword{is}\ {\isachardoublequoteopen}{\isacharunderscore}{\kern0pt}\ {\isasymle}\ {\isacharquery}{\kern0pt}r{\isachardoublequoteclose}{\isacharparenright}{\kern0pt}\isanewline
\ \ \ \ \ \ \isacommand{using}\isamarkupfalse%
\ \ {\isacartoucheopen}{\isasymphi}{\isasymin}{\isacharunderscore}{\kern0pt}{\isacartoucheclose}\ Un{\isacharunderscore}{\kern0pt}le{\isacharunderscore}{\kern0pt}compat{\isacharbrackleft}{\kern0pt}OF\ le{\isacharunderscore}{\kern0pt}refl{\isacharbrackleft}{\kern0pt}of\ {\isadigit{3}}{\isacharbrackright}{\kern0pt}{\isacharbrackright}{\kern0pt}\ \isanewline
\ \ \ \ \ \ \ \ \ \ \ \ \ \ \ \ \ \ le{\isacharunderscore}{\kern0pt}imp{\isacharunderscore}{\kern0pt}subset\ arity{\isacharunderscore}{\kern0pt}ren{\isacharunderscore}{\kern0pt}truth{\isacharbrackleft}{\kern0pt}of\ {\isachardoublequoteopen}forces{\isacharparenleft}{\kern0pt}{\isasymphi}{\isacharparenright}{\kern0pt}{\isachardoublequoteclose}{\isacharbrackright}{\kern0pt}\isanewline
\ \ \ \ \ \ \ \ \ \ \ \ \ \ \ \ \ \ pred{\isacharunderscore}{\kern0pt}mono\isanewline
\ \ \ \ \ \ \isacommand{by}\isamarkupfalse%
\ auto\isanewline
\ \ \ \ \isacommand{finally}\isamarkupfalse%
\isanewline
\ \ \ \ \isacommand{have}\isamarkupfalse%
\ {\isachardoublequoteopen}arity{\isacharparenleft}{\kern0pt}{\isacharquery}{\kern0pt}{\isasympsi}{\isacharparenright}{\kern0pt}\ {\isasymle}\ {\isacharquery}{\kern0pt}r{\isachardoublequoteclose}\ \isacommand{by}\isamarkupfalse%
\ simp\isanewline
\ \ \ \ \isacommand{have}\isamarkupfalse%
\ i{\isacharcolon}{\kern0pt}{\isachardoublequoteopen}{\isacharquery}{\kern0pt}r\ {\isasymle}\ {\isadigit{4}}\ {\isasymunion}\ pred{\isacharparenleft}{\kern0pt}arity{\isacharparenleft}{\kern0pt}forces{\isacharparenleft}{\kern0pt}{\isasymphi}{\isacharparenright}{\kern0pt}{\isacharparenright}{\kern0pt}{\isacharparenright}{\kern0pt}{\isachardoublequoteclose}\ \isanewline
\ \ \ \ \ \ \isacommand{using}\isamarkupfalse%
\ pred{\isacharunderscore}{\kern0pt}Un{\isacharunderscore}{\kern0pt}distrib\ pred{\isacharunderscore}{\kern0pt}succ{\isacharunderscore}{\kern0pt}eq\ {\isacartoucheopen}{\isasymphi}{\isasymin}{\isacharunderscore}{\kern0pt}{\isacartoucheclose}\ Un{\isacharunderscore}{\kern0pt}assoc{\isacharbrackleft}{\kern0pt}symmetric{\isacharbrackright}{\kern0pt}\ nat{\isacharunderscore}{\kern0pt}union{\isacharunderscore}{\kern0pt}abs{\isadigit{1}}\ \isacommand{by}\isamarkupfalse%
\ simp\isanewline
\ \ \ \ \isacommand{have}\isamarkupfalse%
\ h{\isacharcolon}{\kern0pt}{\isachardoublequoteopen}{\isadigit{4}}\ {\isasymunion}\ pred{\isacharparenleft}{\kern0pt}arity{\isacharparenleft}{\kern0pt}forces{\isacharparenleft}{\kern0pt}{\isasymphi}{\isacharparenright}{\kern0pt}{\isacharparenright}{\kern0pt}{\isacharparenright}{\kern0pt}\ {\isasymle}\ {\isadigit{4}}\ {\isasymunion}\ {\isacharparenleft}{\kern0pt}{\isadigit{4}}{\isacharhash}{\kern0pt}{\isacharplus}{\kern0pt}length{\isacharparenleft}{\kern0pt}env{\isacharparenright}{\kern0pt}{\isacharparenright}{\kern0pt}{\isachardoublequoteclose}\isanewline
\ \ \ \ \ \ \isacommand{using}\isamarkupfalse%
\ \ {\isacartoucheopen}env{\isasymin}{\isacharunderscore}{\kern0pt}{\isacartoucheclose}\ add{\isacharunderscore}{\kern0pt}commute\ {\isacartoucheopen}{\isasymphi}{\isasymin}{\isacharunderscore}{\kern0pt}{\isacartoucheclose}\isanewline
\ \ \ \ \ \ \ \ \ \ \ \ Un{\isacharunderscore}{\kern0pt}le{\isacharunderscore}{\kern0pt}compat{\isacharbrackleft}{\kern0pt}of\ {\isadigit{4}}\ {\isadigit{4}}{\isacharcomma}{\kern0pt}OF\ {\isacharunderscore}{\kern0pt}\ pred{\isacharunderscore}{\kern0pt}mono{\isacharbrackleft}{\kern0pt}OF\ {\isacharunderscore}{\kern0pt}\ arity{\isacharunderscore}{\kern0pt}forces{\isacharunderscore}{\kern0pt}le{\isacharbrackleft}{\kern0pt}OF\ {\isacharunderscore}{\kern0pt}\ {\isacharunderscore}{\kern0pt}\ {\isacartoucheopen}arity{\isacharparenleft}{\kern0pt}{\isasymphi}{\isacharparenright}{\kern0pt}{\isasymle}{\isacharunderscore}{\kern0pt}{\isacartoucheclose}{\isacharbrackright}{\kern0pt}{\isacharbrackright}{\kern0pt}\ {\isacharbrackright}{\kern0pt}\isanewline
\ \ \ \ \ \ \ \ \ \ \ \ {\isacartoucheopen}env{\isasymin}{\isacharunderscore}{\kern0pt}{\isacartoucheclose}\ \isacommand{by}\isamarkupfalse%
\ auto\isanewline
\ \ \ \ \isacommand{with}\isamarkupfalse%
\ {\isacartoucheopen}{\isasymphi}{\isasymin}{\isacharunderscore}{\kern0pt}{\isacartoucheclose}\ {\isacartoucheopen}env{\isasymin}{\isacharunderscore}{\kern0pt}{\isacartoucheclose}\isanewline
\ \ \ \ \isacommand{show}\isamarkupfalse%
\ {\isacharquery}{\kern0pt}thesis\isanewline
\ \ \ \ \ \ \isacommand{using}\isamarkupfalse%
\ le{\isacharunderscore}{\kern0pt}trans{\isacharbrackleft}{\kern0pt}OF\ \ {\isacartoucheopen}arity{\isacharparenleft}{\kern0pt}{\isacharquery}{\kern0pt}{\isasympsi}{\isacharparenright}{\kern0pt}\ {\isasymle}\ {\isacharquery}{\kern0pt}r{\isacartoucheclose}\ \ le{\isacharunderscore}{\kern0pt}trans{\isacharbrackleft}{\kern0pt}OF\ i\ h{\isacharbrackright}{\kern0pt}{\isacharbrackright}{\kern0pt}\ nat{\isacharunderscore}{\kern0pt}simp{\isacharunderscore}{\kern0pt}union\ \isacommand{by}\isamarkupfalse%
\ simp\isanewline
\ \ \isacommand{qed}\isamarkupfalse%
\isanewline
\ \ \isacommand{ultimately}\isamarkupfalse%
\isanewline
\ \ \isacommand{show}\isamarkupfalse%
\ {\isacharquery}{\kern0pt}thesis\ \isacommand{using}\isamarkupfalse%
\ assms\ P{\isacharunderscore}{\kern0pt}in{\isacharunderscore}{\kern0pt}M\ leq{\isacharunderscore}{\kern0pt}in{\isacharunderscore}{\kern0pt}M\ one{\isacharunderscore}{\kern0pt}in{\isacharunderscore}{\kern0pt}M\ \isanewline
\ \ \ \ \ \ \ separation{\isacharunderscore}{\kern0pt}ax{\isacharbrackleft}{\kern0pt}of\ {\isachardoublequoteopen}{\isacharquery}{\kern0pt}{\isasympsi}{\isachardoublequoteclose}\ {\isachardoublequoteopen}{\isacharbrackleft}{\kern0pt}P{\isacharcomma}{\kern0pt}leq{\isacharcomma}{\kern0pt}one{\isacharbrackright}{\kern0pt}{\isacharat}{\kern0pt}env{\isachardoublequoteclose}{\isacharbrackright}{\kern0pt}\ \isanewline
\ \ \ \ \ \ \ separation{\isacharunderscore}{\kern0pt}cong{\isacharbrackleft}{\kern0pt}of\ {\isachardoublequoteopen}{\isacharhash}{\kern0pt}{\isacharhash}{\kern0pt}M{\isachardoublequoteclose}\ {\isachardoublequoteopen}{\isasymlambda}y{\isachardot}{\kern0pt}\ {\isacharparenleft}{\kern0pt}M{\isacharcomma}{\kern0pt}\ {\isacharbrackleft}{\kern0pt}y{\isacharcomma}{\kern0pt}P{\isacharcomma}{\kern0pt}leq{\isacharcomma}{\kern0pt}one{\isacharbrackright}{\kern0pt}{\isacharat}{\kern0pt}env\ {\isasymTurnstile}{\isacharquery}{\kern0pt}{\isasympsi}{\isacharparenright}{\kern0pt}{\isachardoublequoteclose}{\isacharbrackright}{\kern0pt}\isanewline
\ \ \ \ \isacommand{by}\isamarkupfalse%
\ simp\isanewline
\isacommand{qed}\isamarkupfalse%
%
\endisatagproof
{\isafoldproof}%
%
\isadelimproof
\isanewline
%
\endisadelimproof
\isanewline
\isanewline
\isacommand{lemma}\isamarkupfalse%
\ truth{\isacharunderscore}{\kern0pt}lemma{\isacharcolon}{\kern0pt}\isanewline
\ \ \isakeyword{assumes}\ \isanewline
\ \ \ \ {\isachardoublequoteopen}{\isasymphi}{\isasymin}formula{\isachardoublequoteclose}\ {\isachardoublequoteopen}M{\isacharunderscore}{\kern0pt}generic{\isacharparenleft}{\kern0pt}G{\isacharparenright}{\kern0pt}{\isachardoublequoteclose}\isanewline
\ \ \isakeyword{shows}\ \isanewline
\ \ \ \ \ {\isachardoublequoteopen}{\isasymAnd}env{\isachardot}{\kern0pt}\ env{\isasymin}list{\isacharparenleft}{\kern0pt}M{\isacharparenright}{\kern0pt}\ {\isasymLongrightarrow}\ arity{\isacharparenleft}{\kern0pt}{\isasymphi}{\isacharparenright}{\kern0pt}{\isasymle}length{\isacharparenleft}{\kern0pt}env{\isacharparenright}{\kern0pt}\ {\isasymLongrightarrow}\ \isanewline
\ \ \ \ \ \ {\isacharparenleft}{\kern0pt}{\isasymexists}p{\isasymin}G{\isachardot}{\kern0pt}\ p\ {\isasymtturnstile}\ {\isasymphi}\ env{\isacharparenright}{\kern0pt}\ \ \ {\isasymlongleftrightarrow}\ \ \ M{\isacharbrackleft}{\kern0pt}G{\isacharbrackright}{\kern0pt}{\isacharcomma}{\kern0pt}\ map{\isacharparenleft}{\kern0pt}val{\isacharparenleft}{\kern0pt}G{\isacharparenright}{\kern0pt}{\isacharcomma}{\kern0pt}env{\isacharparenright}{\kern0pt}\ {\isasymTurnstile}\ {\isasymphi}{\isachardoublequoteclose}\isanewline
%
\isadelimproof
\ \ %
\endisadelimproof
%
\isatagproof
\isacommand{using}\isamarkupfalse%
\ assms{\isacharparenleft}{\kern0pt}{\isadigit{1}}{\isacharparenright}{\kern0pt}\isanewline
\isacommand{proof}\isamarkupfalse%
\ {\isacharparenleft}{\kern0pt}induct{\isacharparenright}{\kern0pt}\isanewline
\ \ \isacommand{case}\isamarkupfalse%
\ {\isacharparenleft}{\kern0pt}Member\ x\ y{\isacharparenright}{\kern0pt}\isanewline
\ \ \isacommand{then}\isamarkupfalse%
\isanewline
\ \ \isacommand{show}\isamarkupfalse%
\ {\isacharquery}{\kern0pt}case\isanewline
\ \ \ \ \isacommand{using}\isamarkupfalse%
\ assms\ truth{\isacharunderscore}{\kern0pt}lemma{\isacharunderscore}{\kern0pt}mem{\isacharbrackleft}{\kern0pt}OF\ {\isacartoucheopen}env{\isasymin}list{\isacharparenleft}{\kern0pt}M{\isacharparenright}{\kern0pt}{\isacartoucheclose}\ assms{\isacharparenleft}{\kern0pt}{\isadigit{2}}{\isacharparenright}{\kern0pt}\ {\isacartoucheopen}x{\isasymin}nat{\isacartoucheclose}\ {\isacartoucheopen}y{\isasymin}nat{\isacartoucheclose}{\isacharbrackright}{\kern0pt}\ \isanewline
\ \ \ \ \ \ arities{\isacharunderscore}{\kern0pt}at{\isacharunderscore}{\kern0pt}aux\ \isacommand{by}\isamarkupfalse%
\ simp\isanewline
\isacommand{next}\isamarkupfalse%
\isanewline
\ \ \isacommand{case}\isamarkupfalse%
\ {\isacharparenleft}{\kern0pt}Equal\ x\ y{\isacharparenright}{\kern0pt}\isanewline
\ \ \isacommand{then}\isamarkupfalse%
\isanewline
\ \ \isacommand{show}\isamarkupfalse%
\ {\isacharquery}{\kern0pt}case\isanewline
\ \ \ \ \isacommand{using}\isamarkupfalse%
\ assms\ truth{\isacharunderscore}{\kern0pt}lemma{\isacharunderscore}{\kern0pt}eq{\isacharbrackleft}{\kern0pt}OF\ {\isacartoucheopen}env{\isasymin}list{\isacharparenleft}{\kern0pt}M{\isacharparenright}{\kern0pt}{\isacartoucheclose}\ assms{\isacharparenleft}{\kern0pt}{\isadigit{2}}{\isacharparenright}{\kern0pt}\ {\isacartoucheopen}x{\isasymin}nat{\isacartoucheclose}\ {\isacartoucheopen}y{\isasymin}nat{\isacartoucheclose}{\isacharbrackright}{\kern0pt}\ \isanewline
\ \ \ \ \ \ arities{\isacharunderscore}{\kern0pt}at{\isacharunderscore}{\kern0pt}aux\ \isacommand{by}\isamarkupfalse%
\ simp\isanewline
\isacommand{next}\isamarkupfalse%
\isanewline
\ \ \isacommand{case}\isamarkupfalse%
\ {\isacharparenleft}{\kern0pt}Nand\ {\isasymphi}\ {\isasympsi}{\isacharparenright}{\kern0pt}\isanewline
\ \ \isacommand{moreover}\isamarkupfalse%
\ \isanewline
\ \ \isacommand{note}\isamarkupfalse%
\ {\isacartoucheopen}M{\isacharunderscore}{\kern0pt}generic{\isacharparenleft}{\kern0pt}G{\isacharparenright}{\kern0pt}{\isacartoucheclose}\isanewline
\ \ \isacommand{ultimately}\isamarkupfalse%
\isanewline
\ \ \isacommand{show}\isamarkupfalse%
\ {\isacharquery}{\kern0pt}case\ \isanewline
\ \ \ \ \isacommand{using}\isamarkupfalse%
\ truth{\isacharunderscore}{\kern0pt}lemma{\isacharunderscore}{\kern0pt}And\ truth{\isacharunderscore}{\kern0pt}lemma{\isacharunderscore}{\kern0pt}Neg\ Forces{\isacharunderscore}{\kern0pt}Nand{\isacharunderscore}{\kern0pt}alt\ \isanewline
\ \ \ \ \ \ M{\isacharunderscore}{\kern0pt}genericD\ map{\isacharunderscore}{\kern0pt}val{\isacharunderscore}{\kern0pt}in{\isacharunderscore}{\kern0pt}MG\ arity{\isacharunderscore}{\kern0pt}Nand{\isacharunderscore}{\kern0pt}le{\isacharbrackleft}{\kern0pt}of\ {\isasymphi}\ {\isasympsi}{\isacharbrackright}{\kern0pt}\ \isacommand{by}\isamarkupfalse%
\ auto\isanewline
\isacommand{next}\isamarkupfalse%
\isanewline
\ \ \isacommand{case}\isamarkupfalse%
\ {\isacharparenleft}{\kern0pt}Forall\ {\isasymphi}{\isacharparenright}{\kern0pt}\isanewline
\ \ \isacommand{with}\isamarkupfalse%
\ {\isacartoucheopen}M{\isacharunderscore}{\kern0pt}generic{\isacharparenleft}{\kern0pt}G{\isacharparenright}{\kern0pt}{\isacartoucheclose}\isanewline
\ \ \isacommand{show}\isamarkupfalse%
\ {\isacharquery}{\kern0pt}case\isanewline
\ \ \isacommand{proof}\isamarkupfalse%
\ {\isacharparenleft}{\kern0pt}intro\ iffI{\isacharparenright}{\kern0pt}\isanewline
\ \ \ \ \isacommand{assume}\isamarkupfalse%
\ {\isachardoublequoteopen}{\isasymexists}p{\isasymin}G{\isachardot}{\kern0pt}\ {\isacharparenleft}{\kern0pt}p\ {\isasymtturnstile}\ Forall{\isacharparenleft}{\kern0pt}{\isasymphi}{\isacharparenright}{\kern0pt}\ env{\isacharparenright}{\kern0pt}{\isachardoublequoteclose}\isanewline
\ \ \ \ \isacommand{with}\isamarkupfalse%
\ {\isacartoucheopen}M{\isacharunderscore}{\kern0pt}generic{\isacharparenleft}{\kern0pt}G{\isacharparenright}{\kern0pt}{\isacartoucheclose}\isanewline
\ \ \ \ \isacommand{obtain}\isamarkupfalse%
\ p\ \isakeyword{where}\ {\isachardoublequoteopen}p{\isasymin}G{\isachardoublequoteclose}\ {\isachardoublequoteopen}p{\isasymin}M{\isachardoublequoteclose}\ {\isachardoublequoteopen}p{\isasymin}P{\isachardoublequoteclose}\ {\isachardoublequoteopen}p\ {\isasymtturnstile}\ Forall{\isacharparenleft}{\kern0pt}{\isasymphi}{\isacharparenright}{\kern0pt}\ env{\isachardoublequoteclose}\isanewline
\ \ \ \ \ \ \isacommand{using}\isamarkupfalse%
\ transitivity{\isacharbrackleft}{\kern0pt}OF\ {\isacharunderscore}{\kern0pt}\ P{\isacharunderscore}{\kern0pt}in{\isacharunderscore}{\kern0pt}M{\isacharbrackright}{\kern0pt}\ \isacommand{by}\isamarkupfalse%
\ auto\isanewline
\ \ \ \ \isacommand{with}\isamarkupfalse%
\ {\isacartoucheopen}env{\isasymin}list{\isacharparenleft}{\kern0pt}M{\isacharparenright}{\kern0pt}{\isacartoucheclose}\ {\isacartoucheopen}{\isasymphi}{\isasymin}formula{\isacartoucheclose}\isanewline
\ \ \ \ \isacommand{have}\isamarkupfalse%
\ {\isachardoublequoteopen}p\ {\isasymtturnstile}\ {\isasymphi}\ {\isacharparenleft}{\kern0pt}{\isacharbrackleft}{\kern0pt}x{\isacharbrackright}{\kern0pt}{\isacharat}{\kern0pt}env{\isacharparenright}{\kern0pt}{\isachardoublequoteclose}\ \isakeyword{if}\ {\isachardoublequoteopen}x{\isasymin}M{\isachardoublequoteclose}\ \isakeyword{for}\ x\isanewline
\ \ \ \ \ \ \isacommand{using}\isamarkupfalse%
\ that\ Forces{\isacharunderscore}{\kern0pt}Forall\ \isacommand{by}\isamarkupfalse%
\ simp\isanewline
\ \ \ \ \isacommand{with}\isamarkupfalse%
\ {\isacartoucheopen}p{\isasymin}G{\isacartoucheclose}\ {\isacartoucheopen}{\isasymphi}{\isasymin}formula{\isacartoucheclose}\ {\isacartoucheopen}env{\isasymin}{\isacharunderscore}{\kern0pt}{\isacartoucheclose}\ {\isacartoucheopen}arity{\isacharparenleft}{\kern0pt}Forall{\isacharparenleft}{\kern0pt}{\isasymphi}{\isacharparenright}{\kern0pt}{\isacharparenright}{\kern0pt}\ {\isasymle}\ length{\isacharparenleft}{\kern0pt}env{\isacharparenright}{\kern0pt}{\isacartoucheclose}\isanewline
\ \ \ \ \ \ Forall{\isacharparenleft}{\kern0pt}{\isadigit{2}}{\isacharparenright}{\kern0pt}{\isacharbrackleft}{\kern0pt}of\ {\isachardoublequoteopen}Cons{\isacharparenleft}{\kern0pt}{\isacharunderscore}{\kern0pt}{\isacharcomma}{\kern0pt}env{\isacharparenright}{\kern0pt}{\isachardoublequoteclose}{\isacharbrackright}{\kern0pt}\ \isanewline
\ \ \ \ \isacommand{show}\isamarkupfalse%
\ {\isachardoublequoteopen}M{\isacharbrackleft}{\kern0pt}G{\isacharbrackright}{\kern0pt}{\isacharcomma}{\kern0pt}\ map{\isacharparenleft}{\kern0pt}val{\isacharparenleft}{\kern0pt}G{\isacharparenright}{\kern0pt}{\isacharcomma}{\kern0pt}env{\isacharparenright}{\kern0pt}\ {\isasymTurnstile}\ \ Forall{\isacharparenleft}{\kern0pt}{\isasymphi}{\isacharparenright}{\kern0pt}{\isachardoublequoteclose}\isanewline
\ \ \ \ \ \ \isacommand{using}\isamarkupfalse%
\ pred{\isacharunderscore}{\kern0pt}le{\isadigit{2}}\ map{\isacharunderscore}{\kern0pt}val{\isacharunderscore}{\kern0pt}in{\isacharunderscore}{\kern0pt}MG\isanewline
\ \ \ \ \ \ \isacommand{by}\isamarkupfalse%
\ {\isacharparenleft}{\kern0pt}auto\ iff{\isacharcolon}{\kern0pt}GenExtD{\isacharparenright}{\kern0pt}\isanewline
\ \ \isacommand{next}\isamarkupfalse%
\isanewline
\ \ \ \ \isacommand{assume}\isamarkupfalse%
\ {\isachardoublequoteopen}M{\isacharbrackleft}{\kern0pt}G{\isacharbrackright}{\kern0pt}{\isacharcomma}{\kern0pt}\ map{\isacharparenleft}{\kern0pt}val{\isacharparenleft}{\kern0pt}G{\isacharparenright}{\kern0pt}{\isacharcomma}{\kern0pt}env{\isacharparenright}{\kern0pt}\ {\isasymTurnstile}\ Forall{\isacharparenleft}{\kern0pt}{\isasymphi}{\isacharparenright}{\kern0pt}{\isachardoublequoteclose}\isanewline
\ \ \ \ \isacommand{let}\isamarkupfalse%
\ {\isacharquery}{\kern0pt}D{\isadigit{1}}{\isacharequal}{\kern0pt}{\isachardoublequoteopen}{\isacharbraceleft}{\kern0pt}d{\isasymin}P{\isachardot}{\kern0pt}\ {\isacharparenleft}{\kern0pt}d\ {\isasymtturnstile}\ Forall{\isacharparenleft}{\kern0pt}{\isasymphi}{\isacharparenright}{\kern0pt}\ env{\isacharparenright}{\kern0pt}{\isacharbraceright}{\kern0pt}{\isachardoublequoteclose}\isanewline
\ \ \ \ \isacommand{let}\isamarkupfalse%
\ {\isacharquery}{\kern0pt}D{\isadigit{2}}{\isacharequal}{\kern0pt}{\isachardoublequoteopen}{\isacharbraceleft}{\kern0pt}d{\isasymin}P{\isachardot}{\kern0pt}\ {\isasymexists}b{\isasymin}M{\isachardot}{\kern0pt}\ {\isasymforall}q{\isasymin}P{\isachardot}{\kern0pt}\ q{\isasympreceq}d\ {\isasymlongrightarrow}\ {\isasymnot}{\isacharparenleft}{\kern0pt}q\ {\isasymtturnstile}\ {\isasymphi}\ {\isacharparenleft}{\kern0pt}{\isacharbrackleft}{\kern0pt}b{\isacharbrackright}{\kern0pt}{\isacharat}{\kern0pt}env{\isacharparenright}{\kern0pt}{\isacharparenright}{\kern0pt}{\isacharbraceright}{\kern0pt}{\isachardoublequoteclose}\isanewline
\ \ \ \ \isacommand{define}\isamarkupfalse%
\ D\ \isakeyword{where}\ {\isachardoublequoteopen}D\ {\isasymequiv}\ {\isacharquery}{\kern0pt}D{\isadigit{1}}\ {\isasymunion}\ {\isacharquery}{\kern0pt}D{\isadigit{2}}{\isachardoublequoteclose}\isanewline
\ \ \ \ \isacommand{have}\isamarkupfalse%
\ ar{\isasymphi}{\isacharcolon}{\kern0pt}{\isachardoublequoteopen}arity{\isacharparenleft}{\kern0pt}{\isasymphi}{\isacharparenright}{\kern0pt}{\isasymle}succ{\isacharparenleft}{\kern0pt}length{\isacharparenleft}{\kern0pt}env{\isacharparenright}{\kern0pt}{\isacharparenright}{\kern0pt}{\isachardoublequoteclose}\ \isanewline
\ \ \ \ \ \ \isacommand{using}\isamarkupfalse%
\ assms\ {\isacartoucheopen}arity{\isacharparenleft}{\kern0pt}Forall{\isacharparenleft}{\kern0pt}{\isasymphi}{\isacharparenright}{\kern0pt}{\isacharparenright}{\kern0pt}\ {\isasymle}\ length{\isacharparenleft}{\kern0pt}env{\isacharparenright}{\kern0pt}{\isacartoucheclose}\ {\isacartoucheopen}{\isasymphi}{\isasymin}formula{\isacartoucheclose}\ {\isacartoucheopen}env{\isasymin}list{\isacharparenleft}{\kern0pt}M{\isacharparenright}{\kern0pt}{\isacartoucheclose}\ pred{\isacharunderscore}{\kern0pt}le{\isadigit{2}}\ \isanewline
\ \ \ \ \ \ \isacommand{by}\isamarkupfalse%
\ simp\isanewline
\ \ \ \ \isacommand{then}\isamarkupfalse%
\isanewline
\ \ \ \ \isacommand{have}\isamarkupfalse%
\ {\isachardoublequoteopen}arity{\isacharparenleft}{\kern0pt}Forall{\isacharparenleft}{\kern0pt}{\isasymphi}{\isacharparenright}{\kern0pt}{\isacharparenright}{\kern0pt}\ {\isasymle}\ length{\isacharparenleft}{\kern0pt}env{\isacharparenright}{\kern0pt}{\isachardoublequoteclose}\ \isanewline
\ \ \ \ \ \ \isacommand{using}\isamarkupfalse%
\ pred{\isacharunderscore}{\kern0pt}le\ {\isacartoucheopen}{\isasymphi}{\isasymin}formula{\isacartoucheclose}\ {\isacartoucheopen}env{\isasymin}list{\isacharparenleft}{\kern0pt}M{\isacharparenright}{\kern0pt}{\isacartoucheclose}\ \isacommand{by}\isamarkupfalse%
\ simp\isanewline
\ \ \ \ \isacommand{then}\isamarkupfalse%
\isanewline
\ \ \ \ \isacommand{have}\isamarkupfalse%
\ {\isachardoublequoteopen}{\isacharquery}{\kern0pt}D{\isadigit{1}}{\isasymin}M{\isachardoublequoteclose}\ \isacommand{using}\isamarkupfalse%
\ Collect{\isacharunderscore}{\kern0pt}forces\ ar{\isasymphi}\ {\isacartoucheopen}{\isasymphi}{\isasymin}formula{\isacartoucheclose}\ {\isacartoucheopen}env{\isasymin}list{\isacharparenleft}{\kern0pt}M{\isacharparenright}{\kern0pt}{\isacartoucheclose}\ \isacommand{by}\isamarkupfalse%
\ simp\isanewline
\ \ \ \ \isacommand{moreover}\isamarkupfalse%
\isanewline
\ \ \ \ \isacommand{have}\isamarkupfalse%
\ {\isachardoublequoteopen}{\isacharquery}{\kern0pt}D{\isadigit{2}}{\isasymin}M{\isachardoublequoteclose}\ \isacommand{using}\isamarkupfalse%
\ {\isacartoucheopen}env{\isasymin}list{\isacharparenleft}{\kern0pt}M{\isacharparenright}{\kern0pt}{\isacartoucheclose}\ {\isacartoucheopen}{\isasymphi}{\isasymin}formula{\isacartoucheclose}\ \ truth{\isacharunderscore}{\kern0pt}lemma{\isacharprime}{\kern0pt}\ separation{\isacharunderscore}{\kern0pt}closed\ ar{\isasymphi}\isanewline
\ \ \ \ \ \ \ \ \ \ \ \ \ \ \ \ \ \ \ \ \ \ \ \ P{\isacharunderscore}{\kern0pt}in{\isacharunderscore}{\kern0pt}M\isanewline
\ \ \ \ \ \ \isacommand{by}\isamarkupfalse%
\ simp\isanewline
\ \ \ \ \isacommand{ultimately}\isamarkupfalse%
\isanewline
\ \ \ \ \isacommand{have}\isamarkupfalse%
\ {\isachardoublequoteopen}D{\isasymin}M{\isachardoublequoteclose}\ \isacommand{unfolding}\isamarkupfalse%
\ D{\isacharunderscore}{\kern0pt}def\ \isacommand{using}\isamarkupfalse%
\ Un{\isacharunderscore}{\kern0pt}closed\ \isacommand{by}\isamarkupfalse%
\ simp\isanewline
\ \ \ \ \isacommand{moreover}\isamarkupfalse%
\isanewline
\ \ \ \ \isacommand{have}\isamarkupfalse%
\ {\isachardoublequoteopen}D\ {\isasymsubseteq}\ P{\isachardoublequoteclose}\ \isacommand{unfolding}\isamarkupfalse%
\ D{\isacharunderscore}{\kern0pt}def\ \isacommand{by}\isamarkupfalse%
\ auto\isanewline
\ \ \ \ \isacommand{moreover}\isamarkupfalse%
\isanewline
\ \ \ \ \isacommand{have}\isamarkupfalse%
\ {\isachardoublequoteopen}dense{\isacharparenleft}{\kern0pt}D{\isacharparenright}{\kern0pt}{\isachardoublequoteclose}\ \isanewline
\ \ \ \ \isacommand{proof}\isamarkupfalse%
\isanewline
\ \ \ \ \ \ \isacommand{fix}\isamarkupfalse%
\ p\isanewline
\ \ \ \ \ \ \isacommand{assume}\isamarkupfalse%
\ {\isachardoublequoteopen}p{\isasymin}P{\isachardoublequoteclose}\isanewline
\ \ \ \ \ \ \isacommand{show}\isamarkupfalse%
\ {\isachardoublequoteopen}{\isasymexists}d{\isasymin}D{\isachardot}{\kern0pt}\ d{\isasympreceq}\ p{\isachardoublequoteclose}\isanewline
\ \ \ \ \ \ \isacommand{proof}\isamarkupfalse%
\ {\isacharparenleft}{\kern0pt}cases\ {\isachardoublequoteopen}p\ {\isasymtturnstile}\ Forall{\isacharparenleft}{\kern0pt}{\isasymphi}{\isacharparenright}{\kern0pt}\ env{\isachardoublequoteclose}{\isacharparenright}{\kern0pt}\isanewline
\ \ \ \ \ \ \ \ \isacommand{case}\isamarkupfalse%
\ True\isanewline
\ \ \ \ \ \ \ \ \isacommand{with}\isamarkupfalse%
\ {\isacartoucheopen}p{\isasymin}P{\isacartoucheclose}\ \isanewline
\ \ \ \ \ \ \ \ \isacommand{show}\isamarkupfalse%
\ {\isacharquery}{\kern0pt}thesis\ \isacommand{unfolding}\isamarkupfalse%
\ D{\isacharunderscore}{\kern0pt}def\ \isacommand{using}\isamarkupfalse%
\ leq{\isacharunderscore}{\kern0pt}reflI\ \isacommand{by}\isamarkupfalse%
\ blast\isanewline
\ \ \ \ \ \ \isacommand{next}\isamarkupfalse%
\isanewline
\ \ \ \ \ \ \ \ \isacommand{case}\isamarkupfalse%
\ False\isanewline
\ \ \ \ \ \ \ \ \isacommand{with}\isamarkupfalse%
\ Forall\ {\isacartoucheopen}p{\isasymin}P{\isacartoucheclose}\isanewline
\ \ \ \ \ \ \ \ \isacommand{obtain}\isamarkupfalse%
\ b\ \isakeyword{where}\ {\isachardoublequoteopen}b{\isasymin}M{\isachardoublequoteclose}\ {\isachardoublequoteopen}{\isasymnot}{\isacharparenleft}{\kern0pt}p\ {\isasymtturnstile}\ {\isasymphi}\ {\isacharparenleft}{\kern0pt}{\isacharbrackleft}{\kern0pt}b{\isacharbrackright}{\kern0pt}{\isacharat}{\kern0pt}env{\isacharparenright}{\kern0pt}{\isacharparenright}{\kern0pt}{\isachardoublequoteclose}\isanewline
\ \ \ \ \ \ \ \ \ \ \isacommand{using}\isamarkupfalse%
\ Forces{\isacharunderscore}{\kern0pt}Forall\ \isacommand{by}\isamarkupfalse%
\ blast\isanewline
\ \ \ \ \ \ \ \ \isacommand{moreover}\isamarkupfalse%
\ \isacommand{from}\isamarkupfalse%
\ this\ {\isacartoucheopen}p{\isasymin}P{\isacartoucheclose}\ Forall\isanewline
\ \ \ \ \ \ \ \ \isacommand{have}\isamarkupfalse%
\ {\isachardoublequoteopen}{\isasymnot}dense{\isacharunderscore}{\kern0pt}below{\isacharparenleft}{\kern0pt}{\isacharbraceleft}{\kern0pt}q{\isasymin}P{\isachardot}{\kern0pt}\ q\ {\isasymtturnstile}\ {\isasymphi}\ {\isacharparenleft}{\kern0pt}{\isacharbrackleft}{\kern0pt}b{\isacharbrackright}{\kern0pt}{\isacharat}{\kern0pt}env{\isacharparenright}{\kern0pt}{\isacharbraceright}{\kern0pt}{\isacharcomma}{\kern0pt}p{\isacharparenright}{\kern0pt}{\isachardoublequoteclose}\isanewline
\ \ \ \ \ \ \ \ \ \ \isacommand{using}\isamarkupfalse%
\ density{\isacharunderscore}{\kern0pt}lemma\ pred{\isacharunderscore}{\kern0pt}le{\isadigit{2}}\ \ \isacommand{by}\isamarkupfalse%
\ auto\isanewline
\ \ \ \ \ \ \ \ \isacommand{moreover}\isamarkupfalse%
\ \isacommand{from}\isamarkupfalse%
\ this\isanewline
\ \ \ \ \ \ \ \ \isacommand{obtain}\isamarkupfalse%
\ d\ \isakeyword{where}\ {\isachardoublequoteopen}d{\isasympreceq}p{\isachardoublequoteclose}\ {\isachardoublequoteopen}{\isasymforall}q{\isasymin}P{\isachardot}{\kern0pt}\ q{\isasympreceq}d\ {\isasymlongrightarrow}\ {\isasymnot}{\isacharparenleft}{\kern0pt}q\ {\isasymtturnstile}\ {\isasymphi}\ {\isacharparenleft}{\kern0pt}{\isacharbrackleft}{\kern0pt}b{\isacharbrackright}{\kern0pt}\ {\isacharat}{\kern0pt}\ env{\isacharparenright}{\kern0pt}{\isacharparenright}{\kern0pt}{\isachardoublequoteclose}\isanewline
\ \ \ \ \ \ \ \ \ \ {\isachardoublequoteopen}d{\isasymin}P{\isachardoublequoteclose}\ \isacommand{by}\isamarkupfalse%
\ blast\isanewline
\ \ \ \ \ \ \ \ \isacommand{ultimately}\isamarkupfalse%
\isanewline
\ \ \ \ \ \ \ \ \isacommand{show}\isamarkupfalse%
\ {\isacharquery}{\kern0pt}thesis\ \isacommand{unfolding}\isamarkupfalse%
\ D{\isacharunderscore}{\kern0pt}def\ \isacommand{by}\isamarkupfalse%
\ auto\isanewline
\ \ \ \ \ \ \isacommand{qed}\isamarkupfalse%
\isanewline
\ \ \ \ \isacommand{qed}\isamarkupfalse%
\isanewline
\ \ \ \ \isacommand{moreover}\isamarkupfalse%
\isanewline
\ \ \ \ \isacommand{note}\isamarkupfalse%
\ {\isacartoucheopen}M{\isacharunderscore}{\kern0pt}generic{\isacharparenleft}{\kern0pt}G{\isacharparenright}{\kern0pt}{\isacartoucheclose}\isanewline
\ \ \ \ \isacommand{ultimately}\isamarkupfalse%
\isanewline
\ \ \ \ \isacommand{obtain}\isamarkupfalse%
\ d\ \isakeyword{where}\ {\isachardoublequoteopen}d\ {\isasymin}\ D{\isachardoublequoteclose}\ \ {\isachardoublequoteopen}d\ {\isasymin}\ G{\isachardoublequoteclose}\ \isacommand{by}\isamarkupfalse%
\ blast\isanewline
\ \ \ \ \isacommand{then}\isamarkupfalse%
\isanewline
\ \ \ \ \isacommand{consider}\isamarkupfalse%
\ {\isacharparenleft}{\kern0pt}{\isadigit{1}}{\isacharparenright}{\kern0pt}\ {\isachardoublequoteopen}d{\isasymin}{\isacharquery}{\kern0pt}D{\isadigit{1}}{\isachardoublequoteclose}\ {\isacharbar}{\kern0pt}\ {\isacharparenleft}{\kern0pt}{\isadigit{2}}{\isacharparenright}{\kern0pt}\ {\isachardoublequoteopen}d{\isasymin}{\isacharquery}{\kern0pt}D{\isadigit{2}}{\isachardoublequoteclose}\ \isacommand{unfolding}\isamarkupfalse%
\ D{\isacharunderscore}{\kern0pt}def\ \isacommand{by}\isamarkupfalse%
\ blast\isanewline
\ \ \ \ \isacommand{then}\isamarkupfalse%
\isanewline
\ \ \ \ \isacommand{show}\isamarkupfalse%
\ {\isachardoublequoteopen}{\isasymexists}p{\isasymin}G{\isachardot}{\kern0pt}\ {\isacharparenleft}{\kern0pt}p\ {\isasymtturnstile}\ Forall{\isacharparenleft}{\kern0pt}{\isasymphi}{\isacharparenright}{\kern0pt}\ env{\isacharparenright}{\kern0pt}{\isachardoublequoteclose}\isanewline
\ \ \ \ \isacommand{proof}\isamarkupfalse%
\ {\isacharparenleft}{\kern0pt}cases{\isacharparenright}{\kern0pt}\isanewline
\ \ \ \ \ \ \isacommand{case}\isamarkupfalse%
\ {\isadigit{1}}\isanewline
\ \ \ \ \ \ \isacommand{with}\isamarkupfalse%
\ {\isacartoucheopen}d{\isasymin}G{\isacartoucheclose}\isanewline
\ \ \ \ \ \ \isacommand{show}\isamarkupfalse%
\ {\isacharquery}{\kern0pt}thesis\ \isacommand{by}\isamarkupfalse%
\ blast\isanewline
\ \ \ \ \isacommand{next}\isamarkupfalse%
\isanewline
\ \ \ \ \ \ \isacommand{case}\isamarkupfalse%
\ {\isadigit{2}}\isanewline
\ \ \ \ \ \ \isacommand{then}\isamarkupfalse%
\isanewline
\ \ \ \ \ \ \isacommand{obtain}\isamarkupfalse%
\ b\ \isakeyword{where}\ {\isachardoublequoteopen}b{\isasymin}M{\isachardoublequoteclose}\ {\isachardoublequoteopen}{\isasymforall}q{\isasymin}P{\isachardot}{\kern0pt}\ q{\isasympreceq}d\ {\isasymlongrightarrow}{\isasymnot}{\isacharparenleft}{\kern0pt}q\ {\isasymtturnstile}\ {\isasymphi}\ {\isacharparenleft}{\kern0pt}{\isacharbrackleft}{\kern0pt}b{\isacharbrackright}{\kern0pt}\ {\isacharat}{\kern0pt}\ env{\isacharparenright}{\kern0pt}{\isacharparenright}{\kern0pt}{\isachardoublequoteclose}\isanewline
\ \ \ \ \ \ \ \ \isacommand{by}\isamarkupfalse%
\ blast\isanewline
\ \ \ \ \ \ \isacommand{moreover}\isamarkupfalse%
\ \isacommand{from}\isamarkupfalse%
\ this{\isacharparenleft}{\kern0pt}{\isadigit{1}}{\isacharparenright}{\kern0pt}\ \isakeyword{and}\ \ {\isacartoucheopen}M{\isacharbrackleft}{\kern0pt}G{\isacharbrackright}{\kern0pt}{\isacharcomma}{\kern0pt}\ {\isacharunderscore}{\kern0pt}\ {\isasymTurnstile}\ \ Forall{\isacharparenleft}{\kern0pt}{\isasymphi}{\isacharparenright}{\kern0pt}{\isacartoucheclose}\ \isakeyword{and}\ \isanewline
\ \ \ \ \ \ \ \ Forall{\isacharparenleft}{\kern0pt}{\isadigit{2}}{\isacharparenright}{\kern0pt}{\isacharbrackleft}{\kern0pt}of\ {\isachardoublequoteopen}Cons{\isacharparenleft}{\kern0pt}b{\isacharcomma}{\kern0pt}env{\isacharparenright}{\kern0pt}{\isachardoublequoteclose}{\isacharbrackright}{\kern0pt}\ Forall{\isacharparenleft}{\kern0pt}{\isadigit{1}}{\isacharcomma}{\kern0pt}{\isadigit{3}}{\isacharminus}{\kern0pt}{\isadigit{4}}{\isacharparenright}{\kern0pt}\ {\isacartoucheopen}M{\isacharunderscore}{\kern0pt}generic{\isacharparenleft}{\kern0pt}G{\isacharparenright}{\kern0pt}{\isacartoucheclose}\isanewline
\ \ \ \ \ \ \isacommand{obtain}\isamarkupfalse%
\ p\ \isakeyword{where}\ {\isachardoublequoteopen}p{\isasymin}G{\isachardoublequoteclose}\ {\isachardoublequoteopen}p{\isasymin}P{\isachardoublequoteclose}\ {\isachardoublequoteopen}p\ {\isasymtturnstile}\ {\isasymphi}\ {\isacharparenleft}{\kern0pt}{\isacharbrackleft}{\kern0pt}b{\isacharbrackright}{\kern0pt}\ {\isacharat}{\kern0pt}\ env{\isacharparenright}{\kern0pt}{\isachardoublequoteclose}\ \isanewline
\ \ \ \ \ \ \ \ \isacommand{using}\isamarkupfalse%
\ pred{\isacharunderscore}{\kern0pt}le{\isadigit{2}}\ \isacommand{using}\isamarkupfalse%
\ map{\isacharunderscore}{\kern0pt}val{\isacharunderscore}{\kern0pt}in{\isacharunderscore}{\kern0pt}MG\ \isacommand{by}\isamarkupfalse%
\ {\isacharparenleft}{\kern0pt}auto\ iff{\isacharcolon}{\kern0pt}GenExtD{\isacharparenright}{\kern0pt}\isanewline
\ \ \ \ \ \ \isacommand{moreover}\isamarkupfalse%
\isanewline
\ \ \ \ \ \ \isacommand{note}\isamarkupfalse%
\ {\isacartoucheopen}d{\isasymin}G{\isacartoucheclose}\ {\isacartoucheopen}M{\isacharunderscore}{\kern0pt}generic{\isacharparenleft}{\kern0pt}G{\isacharparenright}{\kern0pt}{\isacartoucheclose}\isanewline
\ \ \ \ \ \ \isacommand{ultimately}\isamarkupfalse%
\isanewline
\ \ \ \ \ \ \isacommand{obtain}\isamarkupfalse%
\ q\ \isakeyword{where}\ {\isachardoublequoteopen}q{\isasymin}G{\isachardoublequoteclose}\ {\isachardoublequoteopen}q{\isasymin}P{\isachardoublequoteclose}\ {\isachardoublequoteopen}q{\isasympreceq}d{\isachardoublequoteclose}\ {\isachardoublequoteopen}q{\isasympreceq}p{\isachardoublequoteclose}\ \isacommand{by}\isamarkupfalse%
\ blast\isanewline
\ \ \ \ \ \ \isacommand{moreover}\isamarkupfalse%
\ \isacommand{from}\isamarkupfalse%
\ this\ \isakeyword{and}\ \ {\isacartoucheopen}p\ {\isasymtturnstile}\ {\isasymphi}\ {\isacharparenleft}{\kern0pt}{\isacharbrackleft}{\kern0pt}b{\isacharbrackright}{\kern0pt}\ {\isacharat}{\kern0pt}\ env{\isacharparenright}{\kern0pt}{\isacartoucheclose}\ \isanewline
\ \ \ \ \ \ \ \ Forall\ \ {\isacartoucheopen}b{\isasymin}M{\isacartoucheclose}\ {\isacartoucheopen}p{\isasymin}P{\isacartoucheclose}\isanewline
\ \ \ \ \ \ \isacommand{have}\isamarkupfalse%
\ {\isachardoublequoteopen}q\ {\isasymtturnstile}\ {\isasymphi}\ {\isacharparenleft}{\kern0pt}{\isacharbrackleft}{\kern0pt}b{\isacharbrackright}{\kern0pt}\ {\isacharat}{\kern0pt}\ env{\isacharparenright}{\kern0pt}{\isachardoublequoteclose}\isanewline
\ \ \ \ \ \ \ \ \isacommand{using}\isamarkupfalse%
\ pred{\isacharunderscore}{\kern0pt}le{\isadigit{2}}\ strengthening{\isacharunderscore}{\kern0pt}lemma\ \isacommand{by}\isamarkupfalse%
\ simp\isanewline
\ \ \ \ \ \ \isacommand{moreover}\isamarkupfalse%
\ \isanewline
\ \ \ \ \ \ \isacommand{note}\isamarkupfalse%
\ {\isacartoucheopen}{\isasymforall}q{\isasymin}P{\isachardot}{\kern0pt}\ q{\isasympreceq}d\ {\isasymlongrightarrow}{\isasymnot}{\isacharparenleft}{\kern0pt}q\ {\isasymtturnstile}\ {\isasymphi}\ {\isacharparenleft}{\kern0pt}{\isacharbrackleft}{\kern0pt}b{\isacharbrackright}{\kern0pt}\ {\isacharat}{\kern0pt}\ env{\isacharparenright}{\kern0pt}{\isacharparenright}{\kern0pt}{\isacartoucheclose}\isanewline
\ \ \ \ \ \ \isacommand{ultimately}\isamarkupfalse%
\isanewline
\ \ \ \ \ \ \isacommand{show}\isamarkupfalse%
\ {\isacharquery}{\kern0pt}thesis\ \isacommand{by}\isamarkupfalse%
\ simp\isanewline
\ \ \ \ \isacommand{qed}\isamarkupfalse%
\isanewline
\ \ \isacommand{qed}\isamarkupfalse%
\isanewline
\isacommand{qed}\isamarkupfalse%
%
\endisatagproof
{\isafoldproof}%
%
\isadelimproof
%
\endisadelimproof
%
\isadelimdocument
%
\endisadelimdocument
%
\isatagdocument
%
\isamarkupsubsection{The ``Definition of forcing''%
}
\isamarkuptrue%
%
\endisatagdocument
{\isafolddocument}%
%
\isadelimdocument
%
\endisadelimdocument
\isacommand{lemma}\isamarkupfalse%
\ definition{\isacharunderscore}{\kern0pt}of{\isacharunderscore}{\kern0pt}forcing{\isacharcolon}{\kern0pt}\isanewline
\ \ \isakeyword{assumes}\isanewline
\ \ \ \ {\isachardoublequoteopen}p{\isasymin}P{\isachardoublequoteclose}\ {\isachardoublequoteopen}{\isasymphi}{\isasymin}formula{\isachardoublequoteclose}\ {\isachardoublequoteopen}env{\isasymin}list{\isacharparenleft}{\kern0pt}M{\isacharparenright}{\kern0pt}{\isachardoublequoteclose}\ {\isachardoublequoteopen}arity{\isacharparenleft}{\kern0pt}{\isasymphi}{\isacharparenright}{\kern0pt}{\isasymle}length{\isacharparenleft}{\kern0pt}env{\isacharparenright}{\kern0pt}{\isachardoublequoteclose}\isanewline
\ \ \isakeyword{shows}\isanewline
\ \ \ \ {\isachardoublequoteopen}{\isacharparenleft}{\kern0pt}p\ {\isasymtturnstile}\ {\isasymphi}\ env{\isacharparenright}{\kern0pt}\ {\isasymlongleftrightarrow}\isanewline
\ \ \ \ \ {\isacharparenleft}{\kern0pt}{\isasymforall}G{\isachardot}{\kern0pt}\ M{\isacharunderscore}{\kern0pt}generic{\isacharparenleft}{\kern0pt}G{\isacharparenright}{\kern0pt}\ {\isasymand}\ p{\isasymin}G\ \ {\isasymlongrightarrow}\ \ M{\isacharbrackleft}{\kern0pt}G{\isacharbrackright}{\kern0pt}{\isacharcomma}{\kern0pt}\ map{\isacharparenleft}{\kern0pt}val{\isacharparenleft}{\kern0pt}G{\isacharparenright}{\kern0pt}{\isacharcomma}{\kern0pt}env{\isacharparenright}{\kern0pt}\ {\isasymTurnstile}\ {\isasymphi}{\isacharparenright}{\kern0pt}{\isachardoublequoteclose}\isanewline
%
\isadelimproof
%
\endisadelimproof
%
\isatagproof
\isacommand{proof}\isamarkupfalse%
\ {\isacharparenleft}{\kern0pt}intro\ iffI\ allI\ impI{\isacharcomma}{\kern0pt}\ elim\ conjE{\isacharparenright}{\kern0pt}\isanewline
\ \ \isacommand{fix}\isamarkupfalse%
\ G\isanewline
\ \ \isacommand{assume}\isamarkupfalse%
\ {\isachardoublequoteopen}{\isacharparenleft}{\kern0pt}p\ {\isasymtturnstile}\ {\isasymphi}\ env{\isacharparenright}{\kern0pt}{\isachardoublequoteclose}\ {\isachardoublequoteopen}M{\isacharunderscore}{\kern0pt}generic{\isacharparenleft}{\kern0pt}G{\isacharparenright}{\kern0pt}{\isachardoublequoteclose}\ {\isachardoublequoteopen}p\ {\isasymin}\ G{\isachardoublequoteclose}\isanewline
\ \ \isacommand{with}\isamarkupfalse%
\ assms\ \isanewline
\ \ \isacommand{show}\isamarkupfalse%
\ {\isachardoublequoteopen}M{\isacharbrackleft}{\kern0pt}G{\isacharbrackright}{\kern0pt}{\isacharcomma}{\kern0pt}\ map{\isacharparenleft}{\kern0pt}val{\isacharparenleft}{\kern0pt}G{\isacharparenright}{\kern0pt}{\isacharcomma}{\kern0pt}env{\isacharparenright}{\kern0pt}\ {\isasymTurnstile}\ {\isasymphi}{\isachardoublequoteclose}\isanewline
\ \ \ \ \isacommand{using}\isamarkupfalse%
\ truth{\isacharunderscore}{\kern0pt}lemma\ \isacommand{by}\isamarkupfalse%
\ blast\isanewline
\isacommand{next}\isamarkupfalse%
\isanewline
\ \ \isacommand{assume}\isamarkupfalse%
\ {\isadigit{1}}{\isacharcolon}{\kern0pt}\ {\isachardoublequoteopen}{\isasymforall}G{\isachardot}{\kern0pt}{\isacharparenleft}{\kern0pt}M{\isacharunderscore}{\kern0pt}generic{\isacharparenleft}{\kern0pt}G{\isacharparenright}{\kern0pt}{\isasymand}\ p{\isasymin}G{\isacharparenright}{\kern0pt}{\isasymlongrightarrow}\ M{\isacharbrackleft}{\kern0pt}G{\isacharbrackright}{\kern0pt}\ {\isacharcomma}{\kern0pt}\ map{\isacharparenleft}{\kern0pt}val{\isacharparenleft}{\kern0pt}G{\isacharparenright}{\kern0pt}{\isacharcomma}{\kern0pt}env{\isacharparenright}{\kern0pt}\ {\isasymTurnstile}\ {\isasymphi}{\isachardoublequoteclose}\isanewline
\ \ \isacommand{{\isacharbraceleft}{\kern0pt}}\isamarkupfalse%
\isanewline
\ \ \ \ \isacommand{fix}\isamarkupfalse%
\ r\ \isanewline
\ \ \ \ \isacommand{assume}\isamarkupfalse%
\ {\isadigit{2}}{\isacharcolon}{\kern0pt}\ {\isachardoublequoteopen}r{\isasymin}P{\isachardoublequoteclose}\ {\isachardoublequoteopen}r{\isasympreceq}p{\isachardoublequoteclose}\isanewline
\ \ \ \ \isacommand{then}\isamarkupfalse%
\ \isanewline
\ \ \ \ \isacommand{obtain}\isamarkupfalse%
\ G\ \isakeyword{where}\ {\isachardoublequoteopen}r{\isasymin}G{\isachardoublequoteclose}\ {\isachardoublequoteopen}M{\isacharunderscore}{\kern0pt}generic{\isacharparenleft}{\kern0pt}G{\isacharparenright}{\kern0pt}{\isachardoublequoteclose}\isanewline
\ \ \ \ \ \ \isacommand{using}\isamarkupfalse%
\ generic{\isacharunderscore}{\kern0pt}filter{\isacharunderscore}{\kern0pt}existence\ \isacommand{by}\isamarkupfalse%
\ auto\isanewline
\ \ \ \ \isacommand{moreover}\isamarkupfalse%
\ \isacommand{from}\isamarkupfalse%
\ calculation\ {\isadigit{2}}\ {\isacartoucheopen}p{\isasymin}P{\isacartoucheclose}\ \isanewline
\ \ \ \ \isacommand{have}\isamarkupfalse%
\ {\isachardoublequoteopen}p{\isasymin}G{\isachardoublequoteclose}\ \isanewline
\ \ \ \ \ \ \isacommand{unfolding}\isamarkupfalse%
\ M{\isacharunderscore}{\kern0pt}generic{\isacharunderscore}{\kern0pt}def\ \isacommand{using}\isamarkupfalse%
\ filter{\isacharunderscore}{\kern0pt}leqD\ \isacommand{by}\isamarkupfalse%
\ simp\isanewline
\ \ \ \ \isacommand{moreover}\isamarkupfalse%
\ \isacommand{note}\isamarkupfalse%
\ {\isadigit{1}}\isanewline
\ \ \ \ \isacommand{ultimately}\isamarkupfalse%
\isanewline
\ \ \ \ \isacommand{have}\isamarkupfalse%
\ {\isachardoublequoteopen}M{\isacharbrackleft}{\kern0pt}G{\isacharbrackright}{\kern0pt}{\isacharcomma}{\kern0pt}\ map{\isacharparenleft}{\kern0pt}val{\isacharparenleft}{\kern0pt}G{\isacharparenright}{\kern0pt}{\isacharcomma}{\kern0pt}env{\isacharparenright}{\kern0pt}\ {\isasymTurnstile}\ {\isasymphi}{\isachardoublequoteclose}\isanewline
\ \ \ \ \ \ \isacommand{by}\isamarkupfalse%
\ simp\isanewline
\ \ \ \ \isacommand{with}\isamarkupfalse%
\ assms\ {\isacartoucheopen}M{\isacharunderscore}{\kern0pt}generic{\isacharparenleft}{\kern0pt}G{\isacharparenright}{\kern0pt}{\isacartoucheclose}\ \isanewline
\ \ \ \ \isacommand{obtain}\isamarkupfalse%
\ s\ \isakeyword{where}\ {\isachardoublequoteopen}s{\isasymin}G{\isachardoublequoteclose}\ {\isachardoublequoteopen}{\isacharparenleft}{\kern0pt}s\ {\isasymtturnstile}\ {\isasymphi}\ env{\isacharparenright}{\kern0pt}{\isachardoublequoteclose}\isanewline
\ \ \ \ \ \ \isacommand{using}\isamarkupfalse%
\ truth{\isacharunderscore}{\kern0pt}lemma\ \isacommand{by}\isamarkupfalse%
\ blast\isanewline
\ \ \ \ \isacommand{moreover}\isamarkupfalse%
\ \isacommand{from}\isamarkupfalse%
\ this\ \isakeyword{and}\ \ {\isacartoucheopen}M{\isacharunderscore}{\kern0pt}generic{\isacharparenleft}{\kern0pt}G{\isacharparenright}{\kern0pt}{\isacartoucheclose}\ {\isacartoucheopen}r{\isasymin}G{\isacartoucheclose}\ \isanewline
\ \ \ \ \isacommand{obtain}\isamarkupfalse%
\ q\ \isakeyword{where}\ {\isachardoublequoteopen}q{\isasymin}G{\isachardoublequoteclose}\ {\isachardoublequoteopen}q{\isasympreceq}s{\isachardoublequoteclose}\ {\isachardoublequoteopen}q{\isasympreceq}r{\isachardoublequoteclose}\isanewline
\ \ \ \ \ \ \isacommand{by}\isamarkupfalse%
\ blast\isanewline
\ \ \ \ \isacommand{moreover}\isamarkupfalse%
\ \isacommand{from}\isamarkupfalse%
\ calculation\ {\isacartoucheopen}s{\isasymin}G{\isacartoucheclose}\ {\isacartoucheopen}M{\isacharunderscore}{\kern0pt}generic{\isacharparenleft}{\kern0pt}G{\isacharparenright}{\kern0pt}{\isacartoucheclose}\ \isanewline
\ \ \ \ \isacommand{have}\isamarkupfalse%
\ {\isachardoublequoteopen}s{\isasymin}P{\isachardoublequoteclose}\ {\isachardoublequoteopen}q{\isasymin}P{\isachardoublequoteclose}\ \isanewline
\ \ \ \ \ \ \isacommand{unfolding}\isamarkupfalse%
\ M{\isacharunderscore}{\kern0pt}generic{\isacharunderscore}{\kern0pt}def\ filter{\isacharunderscore}{\kern0pt}def\ \isacommand{by}\isamarkupfalse%
\ auto\isanewline
\ \ \ \ \isacommand{moreover}\isamarkupfalse%
\ \isanewline
\ \ \ \ \isacommand{note}\isamarkupfalse%
\ assms\isanewline
\ \ \ \ \isacommand{ultimately}\isamarkupfalse%
\ \isanewline
\ \ \ \ \isacommand{have}\isamarkupfalse%
\ {\isachardoublequoteopen}{\isasymexists}q{\isasymin}P{\isachardot}{\kern0pt}\ q{\isasympreceq}r\ {\isasymand}\ {\isacharparenleft}{\kern0pt}q\ {\isasymtturnstile}\ {\isasymphi}\ env{\isacharparenright}{\kern0pt}{\isachardoublequoteclose}\isanewline
\ \ \ \ \ \ \isacommand{using}\isamarkupfalse%
\ strengthening{\isacharunderscore}{\kern0pt}lemma\ \isacommand{by}\isamarkupfalse%
\ blast\isanewline
\ \ \isacommand{{\isacharbraceright}{\kern0pt}}\isamarkupfalse%
\isanewline
\ \ \isacommand{then}\isamarkupfalse%
\isanewline
\ \ \isacommand{have}\isamarkupfalse%
\ {\isachardoublequoteopen}dense{\isacharunderscore}{\kern0pt}below{\isacharparenleft}{\kern0pt}{\isacharbraceleft}{\kern0pt}q{\isasymin}P{\isachardot}{\kern0pt}\ {\isacharparenleft}{\kern0pt}q\ {\isasymtturnstile}\ {\isasymphi}\ env{\isacharparenright}{\kern0pt}{\isacharbraceright}{\kern0pt}{\isacharcomma}{\kern0pt}p{\isacharparenright}{\kern0pt}{\isachardoublequoteclose}\isanewline
\ \ \ \ \isacommand{unfolding}\isamarkupfalse%
\ dense{\isacharunderscore}{\kern0pt}below{\isacharunderscore}{\kern0pt}def\ \isacommand{by}\isamarkupfalse%
\ blast\isanewline
\ \ \isacommand{with}\isamarkupfalse%
\ assms\isanewline
\ \ \isacommand{show}\isamarkupfalse%
\ {\isachardoublequoteopen}{\isacharparenleft}{\kern0pt}p\ {\isasymtturnstile}\ {\isasymphi}\ env{\isacharparenright}{\kern0pt}{\isachardoublequoteclose}\isanewline
\ \ \ \ \isacommand{using}\isamarkupfalse%
\ density{\isacharunderscore}{\kern0pt}lemma\ \isacommand{by}\isamarkupfalse%
\ blast\isanewline
\isacommand{qed}\isamarkupfalse%
%
\endisatagproof
{\isafoldproof}%
%
\isadelimproof
\isanewline
%
\endisadelimproof
\isanewline
\isacommand{lemmas}\isamarkupfalse%
\ definability\ {\isacharequal}{\kern0pt}\ forces{\isacharunderscore}{\kern0pt}type\ \isanewline
\isacommand{end}\isamarkupfalse%
\ \isanewline
%
\isadelimtheory
\isanewline
%
\endisadelimtheory
%
\isatagtheory
\isacommand{end}\isamarkupfalse%
%
\endisatagtheory
{\isafoldtheory}%
%
\isadelimtheory
%
\endisadelimtheory
%
\end{isabellebody}%
\endinput
%:%file=~/source/repos/ZF-notAC/code/Forcing/Forcing_Theorems.thy%:%
%:%11=1%:%
%:%27=3%:%
%:%28=3%:%
%:%29=4%:%
%:%30=5%:%
%:%31=6%:%
%:%32=7%:%
%:%37=7%:%
%:%40=8%:%
%:%41=9%:%
%:%42=9%:%
%:%43=10%:%
%:%50=12%:%
%:%60=14%:%
%:%61=14%:%
%:%62=15%:%
%:%63=16%:%
%:%64=17%:%
%:%65=17%:%
%:%66=18%:%
%:%67=19%:%
%:%68=20%:%
%:%69=21%:%
%:%70=22%:%
%:%71=23%:%
%:%78=24%:%
%:%79=24%:%
%:%80=25%:%
%:%81=25%:%
%:%82=26%:%
%:%83=26%:%
%:%84=26%:%
%:%85=27%:%
%:%86=27%:%
%:%87=28%:%
%:%88=28%:%
%:%89=29%:%
%:%90=29%:%
%:%91=30%:%
%:%92=30%:%
%:%93=31%:%
%:%94=31%:%
%:%95=32%:%
%:%96=32%:%
%:%97=33%:%
%:%98=33%:%
%:%99=33%:%
%:%100=34%:%
%:%101=34%:%
%:%102=34%:%
%:%103=34%:%
%:%104=35%:%
%:%110=35%:%
%:%113=36%:%
%:%114=37%:%
%:%115=37%:%
%:%117=39%:%
%:%120=40%:%
%:%124=40%:%
%:%125=40%:%
%:%126=40%:%
%:%140=42%:%
%:%150=44%:%
%:%151=44%:%
%:%152=45%:%
%:%153=46%:%
%:%156=47%:%
%:%160=47%:%
%:%161=47%:%
%:%162=47%:%
%:%176=70%:%
%:%186=71%:%
%:%187=71%:%
%:%188=72%:%
%:%189=73%:%
%:%192=74%:%
%:%196=74%:%
%:%197=74%:%
%:%198=74%:%
%:%212=76%:%
%:%222=77%:%
%:%223=77%:%
%:%224=78%:%
%:%225=79%:%
%:%232=80%:%
%:%233=80%:%
%:%234=81%:%
%:%235=81%:%
%:%236=82%:%
%:%237=82%:%
%:%238=83%:%
%:%239=83%:%
%:%240=84%:%
%:%241=84%:%
%:%242=84%:%
%:%243=85%:%
%:%244=85%:%
%:%245=86%:%
%:%246=86%:%
%:%247=87%:%
%:%248=87%:%
%:%249=88%:%
%:%250=88%:%
%:%252=90%:%
%:%253=91%:%
%:%254=91%:%
%:%255=91%:%
%:%256=92%:%
%:%257=92%:%
%:%258=93%:%
%:%259=93%:%
%:%260=94%:%
%:%261=94%:%
%:%262=94%:%
%:%263=95%:%
%:%269=95%:%
%:%272=96%:%
%:%273=97%:%
%:%274=97%:%
%:%275=98%:%
%:%276=99%:%
%:%278=101%:%
%:%279=102%:%
%:%280=103%:%
%:%281=104%:%
%:%288=105%:%
%:%289=105%:%
%:%290=106%:%
%:%291=106%:%
%:%292=107%:%
%:%293=107%:%
%:%294=108%:%
%:%295=108%:%
%:%296=108%:%
%:%297=109%:%
%:%298=109%:%
%:%299=110%:%
%:%300=110%:%
%:%301=110%:%
%:%302=111%:%
%:%303=111%:%
%:%304=112%:%
%:%305=112%:%
%:%306=113%:%
%:%307=113%:%
%:%308=114%:%
%:%309=114%:%
%:%310=115%:%
%:%311=115%:%
%:%312=115%:%
%:%313=116%:%
%:%314=116%:%
%:%315=117%:%
%:%316=117%:%
%:%317=118%:%
%:%318=118%:%
%:%319=119%:%
%:%325=119%:%
%:%328=120%:%
%:%329=121%:%
%:%330=122%:%
%:%331=122%:%
%:%332=123%:%
%:%333=124%:%
%:%340=125%:%
%:%341=125%:%
%:%342=126%:%
%:%343=126%:%
%:%344=127%:%
%:%345=127%:%
%:%346=128%:%
%:%347=128%:%
%:%348=129%:%
%:%349=129%:%
%:%350=129%:%
%:%351=130%:%
%:%352=130%:%
%:%353=131%:%
%:%354=131%:%
%:%355=132%:%
%:%356=132%:%
%:%357=133%:%
%:%358=133%:%
%:%359=134%:%
%:%360=134%:%
%:%361=135%:%
%:%362=136%:%
%:%363=136%:%
%:%364=137%:%
%:%365=137%:%
%:%366=138%:%
%:%367=138%:%
%:%368=139%:%
%:%369=139%:%
%:%370=139%:%
%:%371=140%:%
%:%372=140%:%
%:%373=141%:%
%:%374=141%:%
%:%375=142%:%
%:%376=143%:%
%:%377=143%:%
%:%378=143%:%
%:%379=144%:%
%:%380=144%:%
%:%381=145%:%
%:%382=145%:%
%:%383=146%:%
%:%384=147%:%
%:%385=147%:%
%:%386=147%:%
%:%387=148%:%
%:%388=148%:%
%:%389=148%:%
%:%389=149%:%
%:%390=150%:%
%:%391=150%:%
%:%392=151%:%
%:%393=152%:%
%:%394=152%:%
%:%395=153%:%
%:%396=153%:%
%:%397=154%:%
%:%398=154%:%
%:%399=155%:%
%:%400=155%:%
%:%401=156%:%
%:%402=156%:%
%:%403=157%:%
%:%404=157%:%
%:%405=158%:%
%:%406=158%:%
%:%407=159%:%
%:%408=160%:%
%:%409=160%:%
%:%410=160%:%
%:%411=161%:%
%:%412=161%:%
%:%413=162%:%
%:%414=162%:%
%:%415=163%:%
%:%416=163%:%
%:%417=163%:%
%:%418=164%:%
%:%419=164%:%
%:%420=165%:%
%:%421=165%:%
%:%422=166%:%
%:%423=166%:%
%:%424=166%:%
%:%425=166%:%
%:%426=167%:%
%:%441=169%:%
%:%451=170%:%
%:%452=170%:%
%:%453=171%:%
%:%454=172%:%
%:%457=173%:%
%:%461=173%:%
%:%462=173%:%
%:%463=173%:%
%:%464=173%:%
%:%469=173%:%
%:%472=174%:%
%:%473=175%:%
%:%474=176%:%
%:%475=176%:%
%:%476=177%:%
%:%477=178%:%
%:%480=179%:%
%:%484=179%:%
%:%485=179%:%
%:%486=179%:%
%:%487=179%:%
%:%492=179%:%
%:%495=180%:%
%:%496=181%:%
%:%497=182%:%
%:%498=183%:%
%:%499=184%:%
%:%500=185%:%
%:%501=185%:%
%:%502=186%:%
%:%503=187%:%
%:%504=188%:%
%:%505=189%:%
%:%508=192%:%
%:%511=193%:%
%:%515=193%:%
%:%516=193%:%
%:%517=194%:%
%:%518=194%:%
%:%519=195%:%
%:%520=195%:%
%:%525=195%:%
%:%528=196%:%
%:%529=197%:%
%:%530=197%:%
%:%531=198%:%
%:%532=199%:%
%:%533=200%:%
%:%534=201%:%
%:%536=203%:%
%:%539=204%:%
%:%543=204%:%
%:%544=204%:%
%:%545=205%:%
%:%546=205%:%
%:%547=206%:%
%:%548=206%:%
%:%553=206%:%
%:%556=207%:%
%:%557=208%:%
%:%558=208%:%
%:%559=209%:%
%:%560=210%:%
%:%561=211%:%
%:%562=212%:%
%:%563=213%:%
%:%566=214%:%
%:%570=214%:%
%:%571=214%:%
%:%572=215%:%
%:%573=215%:%
%:%574=216%:%
%:%575=216%:%
%:%589=218%:%
%:%599=219%:%
%:%600=219%:%
%:%601=220%:%
%:%602=221%:%
%:%603=222%:%
%:%604=223%:%
%:%607=224%:%
%:%611=224%:%
%:%612=224%:%
%:%613=225%:%
%:%614=225%:%
%:%619=225%:%
%:%622=226%:%
%:%623=227%:%
%:%624=227%:%
%:%625=228%:%
%:%626=229%:%
%:%627=230%:%
%:%628=231%:%
%:%631=232%:%
%:%635=232%:%
%:%636=232%:%
%:%637=233%:%
%:%638=233%:%
%:%643=233%:%
%:%646=234%:%
%:%647=235%:%
%:%648=235%:%
%:%649=236%:%
%:%650=237%:%
%:%651=238%:%
%:%652=239%:%
%:%655=240%:%
%:%659=240%:%
%:%660=240%:%
%:%661=241%:%
%:%662=241%:%
%:%676=243%:%
%:%686=245%:%
%:%687=245%:%
%:%688=246%:%
%:%689=247%:%
%:%690=248%:%
%:%691=249%:%
%:%694=250%:%
%:%698=250%:%
%:%699=250%:%
%:%700=251%:%
%:%701=251%:%
%:%706=251%:%
%:%709=252%:%
%:%710=253%:%
%:%711=253%:%
%:%712=254%:%
%:%713=255%:%
%:%714=256%:%
%:%715=257%:%
%:%716=258%:%
%:%719=259%:%
%:%723=259%:%
%:%724=259%:%
%:%725=259%:%
%:%726=259%:%
%:%731=259%:%
%:%734=260%:%
%:%735=261%:%
%:%736=261%:%
%:%737=262%:%
%:%738=263%:%
%:%739=264%:%
%:%740=265%:%
%:%743=266%:%
%:%747=266%:%
%:%748=266%:%
%:%749=266%:%
%:%750=267%:%
%:%751=267%:%
%:%756=267%:%
%:%759=268%:%
%:%760=269%:%
%:%761=269%:%
%:%762=270%:%
%:%763=271%:%
%:%764=272%:%
%:%765=273%:%
%:%768=274%:%
%:%772=274%:%
%:%773=274%:%
%:%774=274%:%
%:%779=274%:%
%:%782=275%:%
%:%783=276%:%
%:%784=277%:%
%:%785=277%:%
%:%786=278%:%
%:%787=279%:%
%:%788=279%:%
%:%789=280%:%
%:%790=281%:%
%:%791=282%:%
%:%792=283%:%
%:%793=283%:%
%:%796=284%:%
%:%800=284%:%
%:%801=284%:%
%:%806=284%:%
%:%809=285%:%
%:%810=286%:%
%:%811=286%:%
%:%814=287%:%
%:%818=287%:%
%:%819=287%:%
%:%820=287%:%
%:%825=287%:%
%:%828=288%:%
%:%829=289%:%
%:%830=289%:%
%:%833=290%:%
%:%837=290%:%
%:%838=290%:%
%:%839=290%:%
%:%853=293%:%
%:%863=294%:%
%:%864=294%:%
%:%865=295%:%
%:%866=296%:%
%:%873=297%:%
%:%874=297%:%
%:%875=298%:%
%:%876=298%:%
%:%877=299%:%
%:%878=299%:%
%:%879=300%:%
%:%880=300%:%
%:%881=301%:%
%:%882=301%:%
%:%883=302%:%
%:%884=302%:%
%:%885=303%:%
%:%886=303%:%
%:%887=304%:%
%:%888=304%:%
%:%889=305%:%
%:%890=305%:%
%:%891=306%:%
%:%892=306%:%
%:%893=307%:%
%:%894=308%:%
%:%895=308%:%
%:%896=308%:%
%:%897=308%:%
%:%898=309%:%
%:%899=309%:%
%:%900=310%:%
%:%901=310%:%
%:%902=311%:%
%:%903=311%:%
%:%904=312%:%
%:%905=312%:%
%:%906=312%:%
%:%907=313%:%
%:%908=313%:%
%:%909=314%:%
%:%910=314%:%
%:%911=315%:%
%:%912=315%:%
%:%913=316%:%
%:%914=316%:%
%:%915=317%:%
%:%916=317%:%
%:%917=318%:%
%:%918=318%:%
%:%919=319%:%
%:%920=319%:%
%:%921=319%:%
%:%922=320%:%
%:%923=321%:%
%:%924=321%:%
%:%925=322%:%
%:%926=322%:%
%:%927=323%:%
%:%928=323%:%
%:%929=324%:%
%:%930=324%:%
%:%931=325%:%
%:%932=325%:%
%:%933=326%:%
%:%934=326%:%
%:%935=326%:%
%:%936=327%:%
%:%937=327%:%
%:%938=328%:%
%:%939=328%:%
%:%940=328%:%
%:%941=329%:%
%:%942=329%:%
%:%943=330%:%
%:%944=330%:%
%:%945=331%:%
%:%946=331%:%
%:%947=332%:%
%:%948=333%:%
%:%949=333%:%
%:%950=333%:%
%:%951=334%:%
%:%952=334%:%
%:%953=335%:%
%:%954=335%:%
%:%955=336%:%
%:%956=336%:%
%:%957=336%:%
%:%958=336%:%
%:%959=337%:%
%:%960=337%:%
%:%961=338%:%
%:%962=338%:%
%:%963=338%:%
%:%964=339%:%
%:%965=339%:%
%:%966=340%:%
%:%967=340%:%
%:%968=341%:%
%:%969=341%:%
%:%970=341%:%
%:%971=342%:%
%:%972=342%:%
%:%973=343%:%
%:%974=343%:%
%:%975=343%:%
%:%976=344%:%
%:%977=344%:%
%:%978=345%:%
%:%979=345%:%
%:%980=346%:%
%:%981=346%:%
%:%982=347%:%
%:%983=347%:%
%:%984=348%:%
%:%985=348%:%
%:%986=348%:%
%:%987=349%:%
%:%1002=351%:%
%:%1012=352%:%
%:%1013=352%:%
%:%1014=353%:%
%:%1015=354%:%
%:%1016=355%:%
%:%1017=356%:%
%:%1024=357%:%
%:%1025=357%:%
%:%1026=358%:%
%:%1027=358%:%
%:%1028=359%:%
%:%1029=360%:%
%:%1030=360%:%
%:%1031=361%:%
%:%1032=362%:%
%:%1033=362%:%
%:%1034=363%:%
%:%1035=363%:%
%:%1036=363%:%
%:%1037=364%:%
%:%1038=364%:%
%:%1039=365%:%
%:%1040=365%:%
%:%1041=366%:%
%:%1042=367%:%
%:%1043=367%:%
%:%1044=367%:%
%:%1045=368%:%
%:%1046=368%:%
%:%1047=369%:%
%:%1048=369%:%
%:%1049=370%:%
%:%1050=370%:%
%:%1051=371%:%
%:%1052=371%:%
%:%1053=371%:%
%:%1054=372%:%
%:%1055=372%:%
%:%1056=373%:%
%:%1057=373%:%
%:%1058=373%:%
%:%1059=374%:%
%:%1060=374%:%
%:%1061=375%:%
%:%1062=375%:%
%:%1063=376%:%
%:%1064=376%:%
%:%1065=377%:%
%:%1066=377%:%
%:%1067=377%:%
%:%1068=378%:%
%:%1069=378%:%
%:%1070=379%:%
%:%1071=379%:%
%:%1072=380%:%
%:%1073=380%:%
%:%1074=380%:%
%:%1075=381%:%
%:%1076=382%:%
%:%1077=382%:%
%:%1078=383%:%
%:%1084=383%:%
%:%1087=384%:%
%:%1088=385%:%
%:%1089=386%:%
%:%1090=386%:%
%:%1091=387%:%
%:%1092=388%:%
%:%1093=389%:%
%:%1094=390%:%
%:%1095=391%:%
%:%1096=392%:%
%:%1103=393%:%
%:%1104=393%:%
%:%1105=394%:%
%:%1106=394%:%
%:%1107=395%:%
%:%1108=395%:%
%:%1109=396%:%
%:%1110=396%:%
%:%1111=396%:%
%:%1112=397%:%
%:%1113=397%:%
%:%1114=398%:%
%:%1115=398%:%
%:%1116=399%:%
%:%1117=399%:%
%:%1118=400%:%
%:%1119=400%:%
%:%1120=400%:%
%:%1121=400%:%
%:%1122=401%:%
%:%1123=401%:%
%:%1124=401%:%
%:%1125=402%:%
%:%1126=402%:%
%:%1127=403%:%
%:%1128=403%:%
%:%1129=403%:%
%:%1130=404%:%
%:%1131=404%:%
%:%1132=405%:%
%:%1133=405%:%
%:%1134=406%:%
%:%1135=406%:%
%:%1136=407%:%
%:%1137=407%:%
%:%1138=407%:%
%:%1139=407%:%
%:%1140=408%:%
%:%1141=408%:%
%:%1142=409%:%
%:%1143=409%:%
%:%1144=409%:%
%:%1145=410%:%
%:%1146=410%:%
%:%1147=410%:%
%:%1148=411%:%
%:%1149=411%:%
%:%1150=411%:%
%:%1151=412%:%
%:%1152=412%:%
%:%1153=413%:%
%:%1154=413%:%
%:%1155=413%:%
%:%1156=414%:%
%:%1162=414%:%
%:%1165=415%:%
%:%1166=416%:%
%:%1167=417%:%
%:%1168=417%:%
%:%1171=418%:%
%:%1175=418%:%
%:%1176=418%:%
%:%1177=418%:%
%:%1182=418%:%
%:%1185=419%:%
%:%1186=420%:%
%:%1187=420%:%
%:%1190=421%:%
%:%1194=421%:%
%:%1195=421%:%
%:%1196=422%:%
%:%1197=422%:%
%:%1202=422%:%
%:%1205=423%:%
%:%1206=424%:%
%:%1207=425%:%
%:%1208=425%:%
%:%1209=426%:%
%:%1210=427%:%
%:%1211=428%:%
%:%1212=429%:%
%:%1214=431%:%
%:%1215=432%:%
%:%1216=434%:%
%:%1217=435%:%
%:%1218=436%:%
%:%1225=437%:%
%:%1226=437%:%
%:%1227=438%:%
%:%1228=438%:%
%:%1229=439%:%
%:%1230=439%:%
%:%1231=440%:%
%:%1232=440%:%
%:%1233=441%:%
%:%1234=441%:%
%:%1235=441%:%
%:%1236=442%:%
%:%1237=442%:%
%:%1238=442%:%
%:%1239=443%:%
%:%1240=443%:%
%:%1241=443%:%
%:%1242=444%:%
%:%1243=444%:%
%:%1244=444%:%
%:%1245=445%:%
%:%1246=445%:%
%:%1247=445%:%
%:%1248=446%:%
%:%1249=446%:%
%:%1250=446%:%
%:%1251=447%:%
%:%1252=447%:%
%:%1253=448%:%
%:%1254=448%:%
%:%1255=448%:%
%:%1256=449%:%
%:%1257=449%:%
%:%1258=450%:%
%:%1259=450%:%
%:%1260=451%:%
%:%1261=451%:%
%:%1262=452%:%
%:%1263=452%:%
%:%1264=453%:%
%:%1265=453%:%
%:%1266=453%:%
%:%1267=454%:%
%:%1268=454%:%
%:%1269=455%:%
%:%1270=455%:%
%:%1271=455%:%
%:%1272=456%:%
%:%1278=456%:%
%:%1281=457%:%
%:%1282=458%:%
%:%1283=459%:%
%:%1284=459%:%
%:%1285=460%:%
%:%1286=461%:%
%:%1287=462%:%
%:%1288=463%:%
%:%1290=465%:%
%:%1291=466%:%
%:%1292=467%:%
%:%1299=468%:%
%:%1300=468%:%
%:%1301=469%:%
%:%1302=469%:%
%:%1303=470%:%
%:%1304=470%:%
%:%1305=471%:%
%:%1306=471%:%
%:%1307=472%:%
%:%1308=472%:%
%:%1309=472%:%
%:%1310=473%:%
%:%1311=473%:%
%:%1312=473%:%
%:%1313=474%:%
%:%1314=474%:%
%:%1315=474%:%
%:%1316=475%:%
%:%1317=475%:%
%:%1318=475%:%
%:%1319=476%:%
%:%1320=476%:%
%:%1321=476%:%
%:%1322=477%:%
%:%1323=477%:%
%:%1324=477%:%
%:%1325=478%:%
%:%1326=478%:%
%:%1327=479%:%
%:%1328=479%:%
%:%1329=479%:%
%:%1330=480%:%
%:%1331=480%:%
%:%1332=481%:%
%:%1333=481%:%
%:%1334=482%:%
%:%1335=482%:%
%:%1336=483%:%
%:%1337=483%:%
%:%1338=484%:%
%:%1339=484%:%
%:%1340=484%:%
%:%1341=485%:%
%:%1342=485%:%
%:%1343=486%:%
%:%1344=486%:%
%:%1345=486%:%
%:%1346=487%:%
%:%1352=487%:%
%:%1355=488%:%
%:%1356=489%:%
%:%1357=490%:%
%:%1358=490%:%
%:%1359=491%:%
%:%1360=492%:%
%:%1361=493%:%
%:%1362=494%:%
%:%1364=496%:%
%:%1365=497%:%
%:%1366=498%:%
%:%1369=499%:%
%:%1373=499%:%
%:%1374=499%:%
%:%1375=499%:%
%:%1389=501%:%
%:%1399=503%:%
%:%1400=503%:%
%:%1401=504%:%
%:%1402=505%:%
%:%1403=506%:%
%:%1404=507%:%
%:%1405=508%:%
%:%1406=509%:%
%:%1413=510%:%
%:%1414=510%:%
%:%1415=511%:%
%:%1416=511%:%
%:%1417=512%:%
%:%1418=512%:%
%:%1419=513%:%
%:%1420=513%:%
%:%1421=514%:%
%:%1422=514%:%
%:%1423=514%:%
%:%1424=515%:%
%:%1425=515%:%
%:%1426=516%:%
%:%1427=516%:%
%:%1428=516%:%
%:%1429=517%:%
%:%1430=517%:%
%:%1431=518%:%
%:%1432=518%:%
%:%1433=519%:%
%:%1434=519%:%
%:%1435=520%:%
%:%1436=520%:%
%:%1437=520%:%
%:%1438=521%:%
%:%1439=521%:%
%:%1440=522%:%
%:%1441=522%:%
%:%1442=523%:%
%:%1443=523%:%
%:%1444=524%:%
%:%1445=524%:%
%:%1446=525%:%
%:%1447=525%:%
%:%1448=525%:%
%:%1449=525%:%
%:%1450=526%:%
%:%1451=526%:%
%:%1452=527%:%
%:%1453=527%:%
%:%1454=527%:%
%:%1455=527%:%
%:%1456=527%:%
%:%1457=528%:%
%:%1463=528%:%
%:%1466=529%:%
%:%1467=530%:%
%:%1468=530%:%
%:%1469=531%:%
%:%1470=532%:%
%:%1471=533%:%
%:%1472=534%:%
%:%1473=535%:%
%:%1474=536%:%
%:%1481=537%:%
%:%1482=537%:%
%:%1483=538%:%
%:%1484=538%:%
%:%1485=539%:%
%:%1486=539%:%
%:%1487=540%:%
%:%1488=540%:%
%:%1489=541%:%
%:%1490=541%:%
%:%1491=542%:%
%:%1492=542%:%
%:%1493=542%:%
%:%1494=543%:%
%:%1495=543%:%
%:%1496=544%:%
%:%1497=544%:%
%:%1498=545%:%
%:%1499=545%:%
%:%1500=546%:%
%:%1501=546%:%
%:%1502=547%:%
%:%1503=547%:%
%:%1504=548%:%
%:%1505=548%:%
%:%1506=549%:%
%:%1507=549%:%
%:%1508=550%:%
%:%1509=550%:%
%:%1510=551%:%
%:%1511=551%:%
%:%1512=551%:%
%:%1513=552%:%
%:%1519=552%:%
%:%1522=553%:%
%:%1523=554%:%
%:%1524=554%:%
%:%1525=555%:%
%:%1526=556%:%
%:%1527=557%:%
%:%1528=558%:%
%:%1529=559%:%
%:%1536=560%:%
%:%1537=560%:%
%:%1538=561%:%
%:%1539=561%:%
%:%1540=562%:%
%:%1541=562%:%
%:%1542=563%:%
%:%1543=563%:%
%:%1544=564%:%
%:%1545=564%:%
%:%1546=565%:%
%:%1547=565%:%
%:%1548=565%:%
%:%1549=566%:%
%:%1550=566%:%
%:%1551=567%:%
%:%1552=567%:%
%:%1553=568%:%
%:%1554=568%:%
%:%1555=569%:%
%:%1556=569%:%
%:%1557=570%:%
%:%1558=570%:%
%:%1559=571%:%
%:%1560=571%:%
%:%1561=571%:%
%:%1562=572%:%
%:%1577=574%:%
%:%1587=575%:%
%:%1588=575%:%
%:%1589=576%:%
%:%1590=577%:%
%:%1591=578%:%
%:%1592=579%:%
%:%1593=580%:%
%:%1594=581%:%
%:%1601=582%:%
%:%1602=582%:%
%:%1603=583%:%
%:%1604=583%:%
%:%1605=584%:%
%:%1606=584%:%
%:%1607=585%:%
%:%1608=585%:%
%:%1609=585%:%
%:%1610=586%:%
%:%1611=586%:%
%:%1612=587%:%
%:%1613=588%:%
%:%1614=588%:%
%:%1615=588%:%
%:%1616=589%:%
%:%1617=589%:%
%:%1618=590%:%
%:%1619=590%:%
%:%1620=591%:%
%:%1621=591%:%
%:%1622=592%:%
%:%1623=592%:%
%:%1624=593%:%
%:%1625=593%:%
%:%1626=593%:%
%:%1627=594%:%
%:%1628=594%:%
%:%1629=595%:%
%:%1630=595%:%
%:%1631=596%:%
%:%1632=596%:%
%:%1633=597%:%
%:%1634=597%:%
%:%1635=597%:%
%:%1636=598%:%
%:%1637=598%:%
%:%1639=600%:%
%:%1640=601%:%
%:%1641=601%:%
%:%1642=602%:%
%:%1643=602%:%
%:%1644=603%:%
%:%1645=603%:%
%:%1646=604%:%
%:%1647=604%:%
%:%1648=604%:%
%:%1649=605%:%
%:%1664=607%:%
%:%1674=609%:%
%:%1675=609%:%
%:%1676=610%:%
%:%1677=611%:%
%:%1678=612%:%
%:%1679=613%:%
%:%1680=614%:%
%:%1681=615%:%
%:%1682=616%:%
%:%1689=617%:%
%:%1690=617%:%
%:%1691=618%:%
%:%1692=618%:%
%:%1693=619%:%
%:%1694=619%:%
%:%1695=619%:%
%:%1696=620%:%
%:%1697=620%:%
%:%1698=620%:%
%:%1699=621%:%
%:%1700=621%:%
%:%1701=621%:%
%:%1702=622%:%
%:%1703=622%:%
%:%1704=623%:%
%:%1705=623%:%
%:%1706=624%:%
%:%1707=624%:%
%:%1708=625%:%
%:%1709=625%:%
%:%1710=626%:%
%:%1711=626%:%
%:%1712=626%:%
%:%1713=627%:%
%:%1714=627%:%
%:%1715=628%:%
%:%1716=628%:%
%:%1717=629%:%
%:%1718=629%:%
%:%1719=629%:%
%:%1720=630%:%
%:%1721=630%:%
%:%1722=631%:%
%:%1723=631%:%
%:%1724=631%:%
%:%1725=632%:%
%:%1726=632%:%
%:%1727=633%:%
%:%1728=633%:%
%:%1729=634%:%
%:%1730=634%:%
%:%1731=635%:%
%:%1732=635%:%
%:%1733=636%:%
%:%1734=636%:%
%:%1735=636%:%
%:%1736=637%:%
%:%1737=637%:%
%:%1738=638%:%
%:%1739=638%:%
%:%1740=639%:%
%:%1741=639%:%
%:%1742=639%:%
%:%1743=640%:%
%:%1744=640%:%
%:%1745=641%:%
%:%1746=641%:%
%:%1747=642%:%
%:%1748=642%:%
%:%1749=642%:%
%:%1750=643%:%
%:%1751=643%:%
%:%1752=644%:%
%:%1753=644%:%
%:%1754=644%:%
%:%1755=645%:%
%:%1756=645%:%
%:%1757=646%:%
%:%1758=646%:%
%:%1759=646%:%
%:%1760=647%:%
%:%1766=647%:%
%:%1769=648%:%
%:%1770=649%:%
%:%1771=649%:%
%:%1772=650%:%
%:%1773=651%:%
%:%1774=651%:%
%:%1775=652%:%
%:%1776=653%:%
%:%1779=654%:%
%:%1783=654%:%
%:%1784=654%:%
%:%1785=655%:%
%:%1786=656%:%
%:%1787=657%:%
%:%1788=658%:%
%:%1789=658%:%
%:%1794=658%:%
%:%1797=659%:%
%:%1798=660%:%
%:%1799=660%:%
%:%1800=661%:%
%:%1801=662%:%
%:%1802=663%:%
%:%1809=664%:%
%:%1810=664%:%
%:%1811=665%:%
%:%1812=665%:%
%:%1815=668%:%
%:%1816=669%:%
%:%1817=669%:%
%:%1819=671%:%
%:%1820=672%:%
%:%1821=672%:%
%:%1822=673%:%
%:%1823=673%:%
%:%1824=673%:%
%:%1825=674%:%
%:%1826=674%:%
%:%1827=675%:%
%:%1828=676%:%
%:%1829=677%:%
%:%1830=677%:%
%:%1831=678%:%
%:%1832=678%:%
%:%1833=679%:%
%:%1834=680%:%
%:%1835=680%:%
%:%1836=681%:%
%:%1837=681%:%
%:%1838=682%:%
%:%1839=683%:%
%:%1840=683%:%
%:%1841=684%:%
%:%1842=684%:%
%:%1843=685%:%
%:%1844=686%:%
%:%1845=686%:%
%:%1846=687%:%
%:%1847=687%:%
%:%1848=688%:%
%:%1849=688%:%
%:%1850=689%:%
%:%1851=689%:%
%:%1852=690%:%
%:%1853=690%:%
%:%1854=691%:%
%:%1855=691%:%
%:%1856=692%:%
%:%1857=692%:%
%:%1858=693%:%
%:%1859=693%:%
%:%1860=693%:%
%:%1861=694%:%
%:%1862=694%:%
%:%1863=695%:%
%:%1864=695%:%
%:%1865=696%:%
%:%1866=696%:%
%:%1867=696%:%
%:%1868=697%:%
%:%1869=697%:%
%:%1870=698%:%
%:%1871=699%:%
%:%1872=700%:%
%:%1873=700%:%
%:%1874=701%:%
%:%1875=701%:%
%:%1876=702%:%
%:%1877=703%:%
%:%1878=704%:%
%:%1879=705%:%
%:%1880=705%:%
%:%1881=706%:%
%:%1882=707%:%
%:%1883=708%:%
%:%1884=709%:%
%:%1885=709%:%
%:%1886=710%:%
%:%1892=710%:%
%:%1895=711%:%
%:%1896=712%:%
%:%1897=713%:%
%:%1898=713%:%
%:%1899=714%:%
%:%1900=715%:%
%:%1901=716%:%
%:%1902=717%:%
%:%1904=719%:%
%:%1905=720%:%
%:%1906=721%:%
%:%1907=722%:%
%:%1914=723%:%
%:%1915=723%:%
%:%1916=724%:%
%:%1917=724%:%
%:%1918=725%:%
%:%1919=725%:%
%:%1920=726%:%
%:%1921=726%:%
%:%1922=727%:%
%:%1923=727%:%
%:%1924=728%:%
%:%1925=728%:%
%:%1926=729%:%
%:%1927=729%:%
%:%1928=729%:%
%:%1929=730%:%
%:%1930=730%:%
%:%1931=730%:%
%:%1932=730%:%
%:%1933=731%:%
%:%1934=731%:%
%:%1935=731%:%
%:%1936=732%:%
%:%1937=732%:%
%:%1938=732%:%
%:%1939=732%:%
%:%1940=733%:%
%:%1941=733%:%
%:%1942=734%:%
%:%1943=734%:%
%:%1944=734%:%
%:%1945=734%:%
%:%1946=735%:%
%:%1947=735%:%
%:%1948=736%:%
%:%1949=736%:%
%:%1950=737%:%
%:%1951=737%:%
%:%1952=738%:%
%:%1953=738%:%
%:%1954=739%:%
%:%1955=739%:%
%:%1956=740%:%
%:%1957=740%:%
%:%1960=743%:%
%:%1961=744%:%
%:%1962=744%:%
%:%1963=745%:%
%:%1964=745%:%
%:%1965=746%:%
%:%1966=746%:%
%:%1967=747%:%
%:%1968=747%:%
%:%1969=747%:%
%:%1970=747%:%
%:%1971=748%:%
%:%1972=748%:%
%:%1973=749%:%
%:%1974=749%:%
%:%1975=750%:%
%:%1976=750%:%
%:%1977=750%:%
%:%1978=751%:%
%:%1979=751%:%
%:%1980=751%:%
%:%1981=752%:%
%:%1982=752%:%
%:%1983=753%:%
%:%1984=754%:%
%:%1985=755%:%
%:%1986=755%:%
%:%1987=755%:%
%:%1988=756%:%
%:%1989=756%:%
%:%1990=756%:%
%:%1991=757%:%
%:%1992=757%:%
%:%1993=758%:%
%:%1994=759%:%
%:%1995=760%:%
%:%1996=760%:%
%:%1997=760%:%
%:%1998=761%:%
%:%1999=761%:%
%:%2000=762%:%
%:%2001=762%:%
%:%2002=762%:%
%:%2003=762%:%
%:%2004=763%:%
%:%2005=763%:%
%:%2006=764%:%
%:%2007=764%:%
%:%2008=765%:%
%:%2009=765%:%
%:%2010=765%:%
%:%2011=766%:%
%:%2012=766%:%
%:%2013=767%:%
%:%2014=767%:%
%:%2015=768%:%
%:%2016=768%:%
%:%2017=769%:%
%:%2018=769%:%
%:%2019=770%:%
%:%2020=770%:%
%:%2021=771%:%
%:%2022=771%:%
%:%2023=772%:%
%:%2024=772%:%
%:%2025=773%:%
%:%2026=773%:%
%:%2027=774%:%
%:%2028=774%:%
%:%2029=774%:%
%:%2030=775%:%
%:%2031=775%:%
%:%2032=776%:%
%:%2033=776%:%
%:%2034=777%:%
%:%2035=778%:%
%:%2036=779%:%
%:%2037=780%:%
%:%2038=780%:%
%:%2039=781%:%
%:%2040=781%:%
%:%2041=782%:%
%:%2042=782%:%
%:%2043=783%:%
%:%2044=783%:%
%:%2045=784%:%
%:%2046=784%:%
%:%2047=785%:%
%:%2048=785%:%
%:%2049=786%:%
%:%2050=786%:%
%:%2051=786%:%
%:%2052=787%:%
%:%2053=787%:%
%:%2054=788%:%
%:%2055=788%:%
%:%2056=789%:%
%:%2057=789%:%
%:%2058=790%:%
%:%2059=790%:%
%:%2060=791%:%
%:%2061=791%:%
%:%2062=792%:%
%:%2063=792%:%
%:%2064=792%:%
%:%2065=793%:%
%:%2066=793%:%
%:%2067=794%:%
%:%2068=794%:%
%:%2069=794%:%
%:%2070=795%:%
%:%2071=795%:%
%:%2072=795%:%
%:%2073=796%:%
%:%2074=796%:%
%:%2075=797%:%
%:%2076=797%:%
%:%2077=797%:%
%:%2078=798%:%
%:%2079=798%:%
%:%2080=798%:%
%:%2081=799%:%
%:%2082=799%:%
%:%2083=800%:%
%:%2084=800%:%
%:%2085=801%:%
%:%2086=801%:%
%:%2087=802%:%
%:%2088=802%:%
%:%2089=803%:%
%:%2090=803%:%
%:%2091=804%:%
%:%2092=804%:%
%:%2093=805%:%
%:%2094=805%:%
%:%2095=805%:%
%:%2096=806%:%
%:%2097=806%:%
%:%2098=807%:%
%:%2099=807%:%
%:%2100=808%:%
%:%2101=808%:%
%:%2102=808%:%
%:%2103=809%:%
%:%2104=809%:%
%:%2105=810%:%
%:%2106=810%:%
%:%2107=810%:%
%:%2108=811%:%
%:%2109=811%:%
%:%2110=812%:%
%:%2111=812%:%
%:%2112=813%:%
%:%2113=813%:%
%:%2114=814%:%
%:%2115=814%:%
%:%2116=815%:%
%:%2117=815%:%
%:%2118=816%:%
%:%2119=816%:%
%:%2120=816%:%
%:%2121=817%:%
%:%2122=817%:%
%:%2123=818%:%
%:%2124=818%:%
%:%2125=818%:%
%:%2126=819%:%
%:%2127=819%:%
%:%2128=819%:%
%:%2129=820%:%
%:%2130=820%:%
%:%2131=821%:%
%:%2132=821%:%
%:%2133=821%:%
%:%2134=822%:%
%:%2135=822%:%
%:%2136=822%:%
%:%2137=823%:%
%:%2138=823%:%
%:%2139=824%:%
%:%2140=824%:%
%:%2141=825%:%
%:%2142=825%:%
%:%2143=826%:%
%:%2144=826%:%
%:%2145=827%:%
%:%2146=827%:%
%:%2147=828%:%
%:%2148=828%:%
%:%2149=829%:%
%:%2150=829%:%
%:%2151=829%:%
%:%2152=830%:%
%:%2153=830%:%
%:%2154=831%:%
%:%2155=831%:%
%:%2156=832%:%
%:%2157=832%:%
%:%2158=832%:%
%:%2159=833%:%
%:%2160=833%:%
%:%2161=834%:%
%:%2162=834%:%
%:%2163=834%:%
%:%2164=835%:%
%:%2165=835%:%
%:%2166=836%:%
%:%2172=836%:%
%:%2175=837%:%
%:%2176=838%:%
%:%2177=839%:%
%:%2178=839%:%
%:%2179=840%:%
%:%2180=841%:%
%:%2181=842%:%
%:%2182=843%:%
%:%2183=844%:%
%:%2184=845%:%
%:%2191=846%:%
%:%2192=846%:%
%:%2193=847%:%
%:%2194=847%:%
%:%2195=848%:%
%:%2196=848%:%
%:%2197=849%:%
%:%2198=849%:%
%:%2199=850%:%
%:%2200=850%:%
%:%2201=851%:%
%:%2202=851%:%
%:%2203=851%:%
%:%2204=852%:%
%:%2205=852%:%
%:%2206=853%:%
%:%2207=853%:%
%:%2208=854%:%
%:%2209=854%:%
%:%2210=855%:%
%:%2211=855%:%
%:%2212=855%:%
%:%2213=856%:%
%:%2214=856%:%
%:%2216=858%:%
%:%2217=859%:%
%:%2218=859%:%
%:%2219=860%:%
%:%2220=860%:%
%:%2221=861%:%
%:%2222=861%:%
%:%2223=862%:%
%:%2224=862%:%
%:%2225=862%:%
%:%2226=863%:%
%:%2232=863%:%
%:%2235=864%:%
%:%2236=865%:%
%:%2237=865%:%
%:%2238=866%:%
%:%2239=867%:%
%:%2240=868%:%
%:%2241=869%:%
%:%2244=870%:%
%:%2248=870%:%
%:%2249=870%:%
%:%2250=870%:%
%:%2251=870%:%
%:%2256=870%:%
%:%2259=871%:%
%:%2260=872%:%
%:%2261=872%:%
%:%2262=873%:%
%:%2263=874%:%
%:%2264=875%:%
%:%2265=876%:%
%:%2266=877%:%
%:%2269=878%:%
%:%2273=878%:%
%:%2274=878%:%
%:%2275=879%:%
%:%2276=880%:%
%:%2277=881%:%
%:%2278=882%:%
%:%2279=882%:%
%:%2284=882%:%
%:%2287=883%:%
%:%2288=884%:%
%:%2289=884%:%
%:%2290=885%:%
%:%2291=886%:%
%:%2292=887%:%
%:%2293=888%:%
%:%2294=889%:%
%:%2297=890%:%
%:%2301=890%:%
%:%2302=890%:%
%:%2303=891%:%
%:%2304=892%:%
%:%2305=893%:%
%:%2306=894%:%
%:%2307=894%:%
%:%2312=894%:%
%:%2315=895%:%
%:%2316=896%:%
%:%2317=896%:%
%:%2318=897%:%
%:%2319=898%:%
%:%2320=899%:%
%:%2321=900%:%
%:%2324=901%:%
%:%2328=901%:%
%:%2329=901%:%
%:%2330=902%:%
%:%2331=902%:%
%:%2345=904%:%
%:%2355=906%:%
%:%2356=906%:%
%:%2357=907%:%
%:%2358=908%:%
%:%2359=909%:%
%:%2360=910%:%
%:%2363=911%:%
%:%2367=911%:%
%:%2368=911%:%
%:%2369=912%:%
%:%2370=912%:%
%:%2371=913%:%
%:%2372=913%:%
%:%2373=914%:%
%:%2374=914%:%
%:%2375=915%:%
%:%2376=915%:%
%:%2377=916%:%
%:%2378=916%:%
%:%2379=916%:%
%:%2380=917%:%
%:%2381=917%:%
%:%2382=918%:%
%:%2383=918%:%
%:%2384=919%:%
%:%2385=919%:%
%:%2386=920%:%
%:%2387=920%:%
%:%2388=921%:%
%:%2389=921%:%
%:%2390=922%:%
%:%2391=922%:%
%:%2392=923%:%
%:%2393=923%:%
%:%2394=924%:%
%:%2395=924%:%
%:%2396=925%:%
%:%2397=925%:%
%:%2398=926%:%
%:%2399=926%:%
%:%2400=927%:%
%:%2401=927%:%
%:%2402=928%:%
%:%2403=928%:%
%:%2404=929%:%
%:%2405=929%:%
%:%2406=929%:%
%:%2407=930%:%
%:%2408=930%:%
%:%2409=931%:%
%:%2410=931%:%
%:%2411=932%:%
%:%2412=932%:%
%:%2413=933%:%
%:%2414=933%:%
%:%2415=934%:%
%:%2416=934%:%
%:%2417=935%:%
%:%2418=935%:%
%:%2419=936%:%
%:%2420=936%:%
%:%2421=937%:%
%:%2422=937%:%
%:%2423=938%:%
%:%2424=938%:%
%:%2425=939%:%
%:%2426=939%:%
%:%2427=940%:%
%:%2428=940%:%
%:%2429=941%:%
%:%2430=941%:%
%:%2431=942%:%
%:%2432=942%:%
%:%2433=943%:%
%:%2434=943%:%
%:%2435=944%:%
%:%2436=944%:%
%:%2437=945%:%
%:%2438=945%:%
%:%2439=946%:%
%:%2440=946%:%
%:%2441=947%:%
%:%2442=947%:%
%:%2443=948%:%
%:%2444=948%:%
%:%2445=948%:%
%:%2446=949%:%
%:%2447=949%:%
%:%2448=950%:%
%:%2449=950%:%
%:%2450=951%:%
%:%2451=951%:%
%:%2452=951%:%
%:%2453=952%:%
%:%2454=952%:%
%:%2455=953%:%
%:%2456=953%:%
%:%2457=954%:%
%:%2458=954%:%
%:%2459=954%:%
%:%2460=955%:%
%:%2475=957%:%
%:%2485=958%:%
%:%2486=958%:%
%:%2487=959%:%
%:%2488=960%:%
%:%2491=961%:%
%:%2495=961%:%
%:%2496=961%:%
%:%2497=962%:%
%:%2498=962%:%
%:%2503=962%:%
%:%2506=963%:%
%:%2507=964%:%
%:%2508=964%:%
%:%2509=965%:%
%:%2510=966%:%
%:%2511=967%:%
%:%2512=968%:%
%:%2513=969%:%
%:%2516=970%:%
%:%2520=970%:%
%:%2521=970%:%
%:%2522=971%:%
%:%2523=971%:%
%:%2524=972%:%
%:%2525=972%:%
%:%2526=973%:%
%:%2527=973%:%
%:%2528=974%:%
%:%2529=974%:%
%:%2530=975%:%
%:%2531=975%:%
%:%2532=975%:%
%:%2533=976%:%
%:%2534=976%:%
%:%2535=977%:%
%:%2536=977%:%
%:%2537=978%:%
%:%2538=978%:%
%:%2539=979%:%
%:%2540=979%:%
%:%2541=980%:%
%:%2542=980%:%
%:%2543=981%:%
%:%2544=981%:%
%:%2545=982%:%
%:%2546=982%:%
%:%2547=983%:%
%:%2548=983%:%
%:%2549=984%:%
%:%2550=984%:%
%:%2551=985%:%
%:%2552=985%:%
%:%2553=986%:%
%:%2554=986%:%
%:%2555=987%:%
%:%2556=987%:%
%:%2557=988%:%
%:%2558=988%:%
%:%2559=988%:%
%:%2560=989%:%
%:%2561=989%:%
%:%2562=990%:%
%:%2563=990%:%
%:%2564=991%:%
%:%2565=991%:%
%:%2566=992%:%
%:%2567=992%:%
%:%2568=993%:%
%:%2569=993%:%
%:%2570=994%:%
%:%2571=994%:%
%:%2572=995%:%
%:%2573=995%:%
%:%2574=996%:%
%:%2575=996%:%
%:%2576=997%:%
%:%2577=997%:%
%:%2578=998%:%
%:%2579=998%:%
%:%2580=999%:%
%:%2581=999%:%
%:%2582=1000%:%
%:%2583=1000%:%
%:%2584=1001%:%
%:%2585=1001%:%
%:%2586=1002%:%
%:%2587=1002%:%
%:%2588=1003%:%
%:%2589=1003%:%
%:%2590=1004%:%
%:%2591=1004%:%
%:%2592=1004%:%
%:%2593=1005%:%
%:%2594=1005%:%
%:%2595=1006%:%
%:%2596=1006%:%
%:%2597=1006%:%
%:%2598=1007%:%
%:%2599=1007%:%
%:%2600=1007%:%
%:%2601=1008%:%
%:%2602=1008%:%
%:%2603=1009%:%
%:%2604=1009%:%
%:%2605=1009%:%
%:%2606=1010%:%
%:%2607=1010%:%
%:%2608=1011%:%
%:%2609=1011%:%
%:%2610=1012%:%
%:%2611=1012%:%
%:%2612=1012%:%
%:%2613=1013%:%
%:%2614=1013%:%
%:%2615=1014%:%
%:%2616=1014%:%
%:%2617=1014%:%
%:%2618=1015%:%
%:%2619=1015%:%
%:%2620=1016%:%
%:%2621=1016%:%
%:%2622=1017%:%
%:%2623=1017%:%
%:%2624=1017%:%
%:%2625=1018%:%
%:%2626=1018%:%
%:%2627=1019%:%
%:%2628=1019%:%
%:%2629=1020%:%
%:%2630=1020%:%
%:%2631=1021%:%
%:%2632=1021%:%
%:%2633=1021%:%
%:%2634=1022%:%
%:%2635=1022%:%
%:%2636=1023%:%
%:%2637=1023%:%
%:%2638=1024%:%
%:%2639=1024%:%
%:%2640=1025%:%
%:%2641=1025%:%
%:%2642=1026%:%
%:%2643=1026%:%
%:%2644=1027%:%
%:%2645=1027%:%
%:%2646=1028%:%
%:%2647=1028%:%
%:%2648=1029%:%
%:%2649=1029%:%
%:%2650=1030%:%
%:%2651=1030%:%
%:%2652=1031%:%
%:%2653=1031%:%
%:%2654=1032%:%
%:%2655=1032%:%
%:%2656=1033%:%
%:%2657=1033%:%
%:%2658=1033%:%
%:%2659=1034%:%
%:%2660=1034%:%
%:%2661=1035%:%
%:%2662=1035%:%
%:%2663=1035%:%
%:%2664=1036%:%
%:%2665=1036%:%
%:%2666=1037%:%
%:%2667=1037%:%
%:%2668=1038%:%
%:%2669=1038%:%
%:%2670=1039%:%
%:%2671=1039%:%
%:%2672=1040%:%
%:%2673=1040%:%
%:%2674=1041%:%
%:%2675=1041%:%
%:%2676=1042%:%
%:%2677=1042%:%
%:%2678=1043%:%
%:%2679=1043%:%
%:%2680=1044%:%
%:%2681=1044%:%
%:%2682=1044%:%
%:%2683=1045%:%
%:%2684=1045%:%
%:%2685=1046%:%
%:%2686=1046%:%
%:%2687=1046%:%
%:%2688=1046%:%
%:%2689=1047%:%
%:%2695=1047%:%
%:%2698=1048%:%
%:%2699=1049%:%
%:%2700=1049%:%
%:%2701=1050%:%
%:%2702=1051%:%
%:%2703=1052%:%
%:%2704=1053%:%
%:%2711=1054%:%
%:%2712=1054%:%
%:%2713=1055%:%
%:%2714=1055%:%
%:%2715=1056%:%
%:%2716=1056%:%
%:%2717=1057%:%
%:%2718=1057%:%
%:%2719=1058%:%
%:%2720=1058%:%
%:%2721=1058%:%
%:%2722=1059%:%
%:%2723=1059%:%
%:%2724=1060%:%
%:%2725=1060%:%
%:%2726=1061%:%
%:%2727=1061%:%
%:%2728=1062%:%
%:%2729=1062%:%
%:%2730=1063%:%
%:%2731=1063%:%
%:%2732=1063%:%
%:%2733=1064%:%
%:%2748=1066%:%
%:%2758=1067%:%
%:%2759=1067%:%
%:%2760=1068%:%
%:%2761=1069%:%
%:%2762=1070%:%
%:%2763=1071%:%
%:%2764=1072%:%
%:%2771=1073%:%
%:%2772=1073%:%
%:%2773=1074%:%
%:%2774=1074%:%
%:%2775=1075%:%
%:%2776=1075%:%
%:%2777=1076%:%
%:%2778=1076%:%
%:%2779=1077%:%
%:%2780=1077%:%
%:%2781=1077%:%
%:%2782=1078%:%
%:%2783=1078%:%
%:%2784=1079%:%
%:%2785=1079%:%
%:%2786=1080%:%
%:%2787=1080%:%
%:%2788=1081%:%
%:%2789=1081%:%
%:%2790=1082%:%
%:%2791=1082%:%
%:%2792=1083%:%
%:%2793=1083%:%
%:%2794=1083%:%
%:%2795=1084%:%
%:%2796=1084%:%
%:%2797=1085%:%
%:%2798=1085%:%
%:%2799=1086%:%
%:%2800=1086%:%
%:%2801=1087%:%
%:%2802=1087%:%
%:%2803=1088%:%
%:%2804=1088%:%
%:%2805=1089%:%
%:%2806=1089%:%
%:%2807=1090%:%
%:%2808=1090%:%
%:%2809=1091%:%
%:%2810=1091%:%
%:%2811=1092%:%
%:%2812=1092%:%
%:%2813=1092%:%
%:%2814=1093%:%
%:%2815=1093%:%
%:%2816=1094%:%
%:%2817=1094%:%
%:%2818=1095%:%
%:%2819=1095%:%
%:%2820=1096%:%
%:%2821=1096%:%
%:%2822=1096%:%
%:%2823=1097%:%
%:%2829=1097%:%
%:%2832=1098%:%
%:%2833=1099%:%
%:%2834=1099%:%
%:%2835=1100%:%
%:%2836=1101%:%
%:%2837=1102%:%
%:%2838=1103%:%
%:%2839=1104%:%
%:%2842=1105%:%
%:%2846=1105%:%
%:%2847=1105%:%
%:%2848=1105%:%
%:%2853=1105%:%
%:%2856=1106%:%
%:%2857=1107%:%
%:%2858=1107%:%
%:%2859=1108%:%
%:%2860=1109%:%
%:%2861=1110%:%
%:%2862=1111%:%
%:%2863=1112%:%
%:%2870=1113%:%
%:%2871=1113%:%
%:%2872=1114%:%
%:%2873=1115%:%
%:%2874=1115%:%
%:%2875=1116%:%
%:%2876=1116%:%
%:%2877=1117%:%
%:%2878=1117%:%
%:%2879=1118%:%
%:%2880=1118%:%
%:%2881=1119%:%
%:%2882=1119%:%
%:%2883=1119%:%
%:%2884=1120%:%
%:%2885=1120%:%
%:%2886=1121%:%
%:%2887=1121%:%
%:%2888=1121%:%
%:%2889=1122%:%
%:%2890=1122%:%
%:%2891=1123%:%
%:%2892=1123%:%
%:%2893=1123%:%
%:%2894=1124%:%
%:%2895=1124%:%
%:%2896=1124%:%
%:%2897=1125%:%
%:%2898=1125%:%
%:%2899=1126%:%
%:%2900=1126%:%
%:%2901=1127%:%
%:%2902=1127%:%
%:%2903=1127%:%
%:%2904=1128%:%
%:%2905=1128%:%
%:%2906=1129%:%
%:%2907=1129%:%
%:%2908=1129%:%
%:%2909=1130%:%
%:%2910=1130%:%
%:%2911=1131%:%
%:%2912=1131%:%
%:%2913=1132%:%
%:%2914=1132%:%
%:%2915=1132%:%
%:%2916=1133%:%
%:%2917=1133%:%
%:%2918=1134%:%
%:%2919=1134%:%
%:%2920=1135%:%
%:%2921=1135%:%
%:%2922=1136%:%
%:%2923=1136%:%
%:%2924=1137%:%
%:%2925=1137%:%
%:%2926=1137%:%
%:%2927=1138%:%
%:%2928=1138%:%
%:%2929=1139%:%
%:%2930=1139%:%
%:%2931=1140%:%
%:%2932=1140%:%
%:%2933=1141%:%
%:%2934=1141%:%
%:%2935=1142%:%
%:%2936=1142%:%
%:%2937=1143%:%
%:%2938=1143%:%
%:%2939=1144%:%
%:%2940=1144%:%
%:%2941=1145%:%
%:%2942=1145%:%
%:%2943=1146%:%
%:%2944=1146%:%
%:%2945=1147%:%
%:%2946=1147%:%
%:%2947=1148%:%
%:%2948=1148%:%
%:%2949=1148%:%
%:%2950=1149%:%
%:%2951=1149%:%
%:%2952=1150%:%
%:%2953=1150%:%
%:%2954=1151%:%
%:%2955=1151%:%
%:%2956=1151%:%
%:%2957=1152%:%
%:%2958=1152%:%
%:%2959=1153%:%
%:%2960=1153%:%
%:%2961=1153%:%
%:%2962=1154%:%
%:%2963=1154%:%
%:%2964=1155%:%
%:%2965=1155%:%
%:%2966=1156%:%
%:%2967=1156%:%
%:%2968=1157%:%
%:%2969=1157%:%
%:%2970=1158%:%
%:%2971=1158%:%
%:%2972=1159%:%
%:%2973=1159%:%
%:%2974=1160%:%
%:%2975=1160%:%
%:%2976=1161%:%
%:%2977=1161%:%
%:%2978=1162%:%
%:%2979=1162%:%
%:%2980=1163%:%
%:%2981=1163%:%
%:%2982=1163%:%
%:%2983=1163%:%
%:%2984=1164%:%
%:%2985=1164%:%
%:%2986=1165%:%
%:%2987=1165%:%
%:%2988=1166%:%
%:%2989=1166%:%
%:%2990=1167%:%
%:%2991=1167%:%
%:%2992=1167%:%
%:%2993=1167%:%
%:%2994=1168%:%
%:%2995=1168%:%
%:%2996=1169%:%
%:%2997=1169%:%
%:%2998=1170%:%
%:%2999=1170%:%
%:%3000=1171%:%
%:%3001=1171%:%
%:%3002=1172%:%
%:%3003=1172%:%
%:%3004=1173%:%
%:%3005=1173%:%
%:%3006=1174%:%
%:%3007=1174%:%
%:%3008=1175%:%
%:%3009=1175%:%
%:%3010=1176%:%
%:%3011=1176%:%
%:%3012=1176%:%
%:%3013=1177%:%
%:%3014=1177%:%
%:%3015=1178%:%
%:%3016=1178%:%
%:%3017=1179%:%
%:%3018=1179%:%
%:%3019=1180%:%
%:%3020=1180%:%
%:%3021=1181%:%
%:%3022=1181%:%
%:%3023=1182%:%
%:%3024=1182%:%
%:%3025=1183%:%
%:%3026=1183%:%
%:%3027=1184%:%
%:%3028=1184%:%
%:%3029=1185%:%
%:%3030=1185%:%
%:%3031=1186%:%
%:%3032=1186%:%
%:%3033=1187%:%
%:%3034=1187%:%
%:%3035=1187%:%
%:%3036=1188%:%
%:%3037=1188%:%
%:%3038=1189%:%
%:%3044=1189%:%
%:%3047=1190%:%
%:%3048=1191%:%
%:%3049=1191%:%
%:%3050=1192%:%
%:%3051=1193%:%
%:%3052=1194%:%
%:%3053=1195%:%
%:%3054=1196%:%
%:%3055=1197%:%
%:%3056=1198%:%
%:%3057=1199%:%
%:%3060=1200%:%
%:%3064=1200%:%
%:%3065=1200%:%
%:%3066=1201%:%
%:%3067=1201%:%
%:%3068=1202%:%
%:%3069=1202%:%
%:%3070=1203%:%
%:%3071=1203%:%
%:%3072=1204%:%
%:%3073=1204%:%
%:%3074=1205%:%
%:%3075=1205%:%
%:%3076=1206%:%
%:%3077=1206%:%
%:%3078=1206%:%
%:%3079=1207%:%
%:%3080=1207%:%
%:%3081=1208%:%
%:%3082=1208%:%
%:%3083=1209%:%
%:%3084=1209%:%
%:%3085=1210%:%
%:%3086=1210%:%
%:%3087=1211%:%
%:%3088=1211%:%
%:%3089=1211%:%
%:%3090=1212%:%
%:%3091=1212%:%
%:%3092=1213%:%
%:%3093=1213%:%
%:%3094=1213%:%
%:%3095=1214%:%
%:%3096=1214%:%
%:%3097=1214%:%
%:%3098=1215%:%
%:%3099=1215%:%
%:%3100=1216%:%
%:%3101=1216%:%
%:%3102=1217%:%
%:%3103=1217%:%
%:%3104=1217%:%
%:%3105=1218%:%
%:%3106=1218%:%
%:%3107=1219%:%
%:%3108=1219%:%
%:%3109=1219%:%
%:%3110=1220%:%
%:%3111=1220%:%
%:%3112=1221%:%
%:%3113=1221%:%
%:%3114=1222%:%
%:%3115=1222%:%
%:%3116=1222%:%
%:%3117=1223%:%
%:%3123=1223%:%
%:%3126=1224%:%
%:%3127=1225%:%
%:%3128=1225%:%
%:%3129=1226%:%
%:%3130=1227%:%
%:%3133=1230%:%
%:%3134=1231%:%
%:%3135=1232%:%
%:%3136=1232%:%
%:%3137=1233%:%
%:%3140=1234%:%
%:%3144=1234%:%
%:%3145=1234%:%
%:%3146=1235%:%
%:%3147=1235%:%
%:%3152=1235%:%
%:%3155=1236%:%
%:%3156=1237%:%
%:%3157=1237%:%
%:%3158=1238%:%
%:%3159=1239%:%
%:%3166=1240%:%
%:%3167=1240%:%
%:%3168=1241%:%
%:%3169=1241%:%
%:%3170=1242%:%
%:%3171=1242%:%
%:%3172=1243%:%
%:%3173=1243%:%
%:%3174=1244%:%
%:%3175=1244%:%
%:%3176=1245%:%
%:%3177=1245%:%
%:%3178=1246%:%
%:%3179=1246%:%
%:%3180=1247%:%
%:%3181=1247%:%
%:%3182=1248%:%
%:%3183=1248%:%
%:%3184=1248%:%
%:%3185=1249%:%
%:%3186=1249%:%
%:%3187=1250%:%
%:%3188=1250%:%
%:%3189=1251%:%
%:%3190=1251%:%
%:%3191=1252%:%
%:%3192=1252%:%
%:%3193=1253%:%
%:%3194=1253%:%
%:%3195=1254%:%
%:%3196=1254%:%
%:%3197=1255%:%
%:%3198=1255%:%
%:%3199=1255%:%
%:%3200=1256%:%
%:%3201=1256%:%
%:%3202=1257%:%
%:%3203=1257%:%
%:%3204=1258%:%
%:%3205=1258%:%
%:%3206=1259%:%
%:%3207=1259%:%
%:%3208=1260%:%
%:%3209=1260%:%
%:%3210=1261%:%
%:%3211=1261%:%
%:%3212=1262%:%
%:%3213=1262%:%
%:%3214=1262%:%
%:%3215=1263%:%
%:%3216=1263%:%
%:%3217=1264%:%
%:%3218=1264%:%
%:%3219=1265%:%
%:%3220=1265%:%
%:%3221=1266%:%
%:%3222=1266%:%
%:%3223=1267%:%
%:%3224=1267%:%
%:%3225=1268%:%
%:%3226=1268%:%
%:%3227=1269%:%
%:%3228=1269%:%
%:%3229=1270%:%
%:%3230=1270%:%
%:%3231=1271%:%
%:%3232=1271%:%
%:%3233=1272%:%
%:%3234=1272%:%
%:%3235=1272%:%
%:%3236=1273%:%
%:%3237=1273%:%
%:%3238=1274%:%
%:%3239=1274%:%
%:%3240=1275%:%
%:%3241=1275%:%
%:%3242=1276%:%
%:%3243=1276%:%
%:%3244=1277%:%
%:%3245=1277%:%
%:%3246=1278%:%
%:%3247=1278%:%
%:%3248=1279%:%
%:%3249=1279%:%
%:%3250=1280%:%
%:%3251=1280%:%
%:%3252=1281%:%
%:%3253=1281%:%
%:%3254=1282%:%
%:%3255=1282%:%
%:%3256=1283%:%
%:%3257=1283%:%
%:%3258=1284%:%
%:%3259=1284%:%
%:%3260=1285%:%
%:%3261=1285%:%
%:%3262=1286%:%
%:%3263=1286%:%
%:%3264=1287%:%
%:%3265=1287%:%
%:%3266=1288%:%
%:%3267=1288%:%
%:%3268=1289%:%
%:%3269=1289%:%
%:%3270=1290%:%
%:%3271=1290%:%
%:%3272=1291%:%
%:%3273=1291%:%
%:%3274=1292%:%
%:%3275=1292%:%
%:%3276=1293%:%
%:%3277=1293%:%
%:%3278=1294%:%
%:%3279=1294%:%
%:%3280=1295%:%
%:%3281=1295%:%
%:%3282=1296%:%
%:%3283=1296%:%
%:%3284=1297%:%
%:%3285=1297%:%
%:%3286=1298%:%
%:%3287=1298%:%
%:%3288=1299%:%
%:%3294=1299%:%
%:%3297=1300%:%
%:%3298=1301%:%
%:%3299=1301%:%
%:%3300=1302%:%
%:%3302=1304%:%
%:%3305=1305%:%
%:%3309=1305%:%
%:%3310=1305%:%
%:%3311=1306%:%
%:%3312=1306%:%
%:%3317=1306%:%
%:%3320=1307%:%
%:%3321=1308%:%
%:%3322=1308%:%
%:%3323=1309%:%
%:%3324=1310%:%
%:%3325=1311%:%
%:%3326=1312%:%
%:%3333=1313%:%
%:%3334=1313%:%
%:%3335=1314%:%
%:%3336=1314%:%
%:%3337=1315%:%
%:%3338=1316%:%
%:%3339=1316%:%
%:%3340=1317%:%
%:%3341=1318%:%
%:%3342=1318%:%
%:%3343=1318%:%
%:%3344=1318%:%
%:%3345=1319%:%
%:%3346=1319%:%
%:%3347=1320%:%
%:%3348=1320%:%
%:%3349=1321%:%
%:%3350=1321%:%
%:%3351=1322%:%
%:%3352=1322%:%
%:%3353=1323%:%
%:%3354=1323%:%
%:%3355=1324%:%
%:%3356=1325%:%
%:%3357=1325%:%
%:%3358=1326%:%
%:%3359=1326%:%
%:%3360=1327%:%
%:%3361=1327%:%
%:%3362=1328%:%
%:%3363=1328%:%
%:%3364=1329%:%
%:%3365=1329%:%
%:%3366=1330%:%
%:%3367=1330%:%
%:%3368=1331%:%
%:%3369=1331%:%
%:%3370=1331%:%
%:%3371=1331%:%
%:%3372=1332%:%
%:%3373=1332%:%
%:%3374=1333%:%
%:%3375=1333%:%
%:%3376=1334%:%
%:%3377=1334%:%
%:%3378=1335%:%
%:%3379=1335%:%
%:%3380=1336%:%
%:%3381=1336%:%
%:%3382=1337%:%
%:%3383=1337%:%
%:%3384=1338%:%
%:%3385=1338%:%
%:%3386=1339%:%
%:%3387=1339%:%
%:%3388=1340%:%
%:%3389=1340%:%
%:%3390=1341%:%
%:%3391=1341%:%
%:%3392=1342%:%
%:%3393=1342%:%
%:%3394=1343%:%
%:%3395=1343%:%
%:%3396=1344%:%
%:%3397=1344%:%
%:%3398=1344%:%
%:%3399=1345%:%
%:%3400=1345%:%
%:%3401=1346%:%
%:%3402=1346%:%
%:%3403=1347%:%
%:%3404=1347%:%
%:%3405=1348%:%
%:%3406=1349%:%
%:%3407=1350%:%
%:%3408=1350%:%
%:%3409=1351%:%
%:%3410=1351%:%
%:%3411=1352%:%
%:%3412=1352%:%
%:%3413=1352%:%
%:%3414=1353%:%
%:%3415=1353%:%
%:%3416=1354%:%
%:%3417=1354%:%
%:%3418=1354%:%
%:%3419=1355%:%
%:%3420=1355%:%
%:%3421=1356%:%
%:%3422=1356%:%
%:%3423=1357%:%
%:%3424=1358%:%
%:%3425=1358%:%
%:%3426=1359%:%
%:%3427=1359%:%
%:%3428=1360%:%
%:%3429=1360%:%
%:%3430=1361%:%
%:%3431=1361%:%
%:%3432=1361%:%
%:%3433=1362%:%
%:%3434=1362%:%
%:%3435=1363%:%
%:%3436=1363%:%
%:%3437=1364%:%
%:%3438=1364%:%
%:%3439=1364%:%
%:%3440=1365%:%
%:%3441=1366%:%
%:%3442=1367%:%
%:%3443=1367%:%
%:%3444=1368%:%
%:%3450=1368%:%
%:%3453=1369%:%
%:%3454=1370%:%
%:%3455=1371%:%
%:%3456=1371%:%
%:%3457=1372%:%
%:%3458=1373%:%
%:%3459=1374%:%
%:%3460=1375%:%
%:%3461=1376%:%
%:%3464=1377%:%
%:%3468=1377%:%
%:%3469=1377%:%
%:%3470=1378%:%
%:%3471=1378%:%
%:%3472=1379%:%
%:%3473=1379%:%
%:%3474=1380%:%
%:%3475=1380%:%
%:%3476=1381%:%
%:%3477=1381%:%
%:%3478=1382%:%
%:%3479=1382%:%
%:%3480=1383%:%
%:%3481=1383%:%
%:%3482=1384%:%
%:%3483=1384%:%
%:%3484=1385%:%
%:%3485=1385%:%
%:%3486=1386%:%
%:%3487=1386%:%
%:%3488=1387%:%
%:%3489=1387%:%
%:%3490=1388%:%
%:%3491=1388%:%
%:%3492=1389%:%
%:%3493=1389%:%
%:%3494=1390%:%
%:%3495=1390%:%
%:%3496=1391%:%
%:%3497=1391%:%
%:%3498=1392%:%
%:%3499=1392%:%
%:%3500=1393%:%
%:%3501=1393%:%
%:%3502=1394%:%
%:%3503=1394%:%
%:%3504=1395%:%
%:%3505=1395%:%
%:%3506=1396%:%
%:%3507=1396%:%
%:%3508=1397%:%
%:%3509=1397%:%
%:%3510=1398%:%
%:%3511=1398%:%
%:%3512=1399%:%
%:%3513=1399%:%
%:%3514=1400%:%
%:%3515=1400%:%
%:%3516=1401%:%
%:%3517=1401%:%
%:%3518=1402%:%
%:%3519=1402%:%
%:%3520=1403%:%
%:%3521=1403%:%
%:%3522=1404%:%
%:%3523=1404%:%
%:%3524=1405%:%
%:%3525=1405%:%
%:%3526=1406%:%
%:%3527=1406%:%
%:%3528=1406%:%
%:%3529=1407%:%
%:%3530=1407%:%
%:%3531=1408%:%
%:%3532=1408%:%
%:%3533=1409%:%
%:%3534=1409%:%
%:%3535=1409%:%
%:%3536=1410%:%
%:%3537=1410%:%
%:%3538=1411%:%
%:%3539=1412%:%
%:%3540=1412%:%
%:%3541=1413%:%
%:%3542=1413%:%
%:%3543=1414%:%
%:%3544=1414%:%
%:%3545=1415%:%
%:%3546=1415%:%
%:%3547=1416%:%
%:%3548=1416%:%
%:%3549=1417%:%
%:%3550=1417%:%
%:%3551=1418%:%
%:%3552=1418%:%
%:%3553=1419%:%
%:%3554=1419%:%
%:%3555=1420%:%
%:%3556=1420%:%
%:%3557=1421%:%
%:%3558=1421%:%
%:%3559=1422%:%
%:%3560=1422%:%
%:%3561=1423%:%
%:%3562=1423%:%
%:%3563=1424%:%
%:%3564=1424%:%
%:%3565=1425%:%
%:%3566=1425%:%
%:%3567=1425%:%
%:%3568=1426%:%
%:%3569=1426%:%
%:%3570=1427%:%
%:%3571=1427%:%
%:%3572=1427%:%
%:%3573=1427%:%
%:%3574=1428%:%
%:%3575=1428%:%
%:%3576=1429%:%
%:%3577=1429%:%
%:%3578=1429%:%
%:%3579=1430%:%
%:%3580=1431%:%
%:%3581=1431%:%
%:%3582=1432%:%
%:%3583=1432%:%
%:%3584=1433%:%
%:%3585=1433%:%
%:%3586=1433%:%
%:%3587=1433%:%
%:%3588=1433%:%
%:%3589=1434%:%
%:%3590=1434%:%
%:%3591=1435%:%
%:%3592=1435%:%
%:%3593=1435%:%
%:%3594=1435%:%
%:%3595=1436%:%
%:%3596=1436%:%
%:%3597=1437%:%
%:%3598=1437%:%
%:%3599=1438%:%
%:%3600=1438%:%
%:%3601=1439%:%
%:%3602=1439%:%
%:%3603=1440%:%
%:%3604=1440%:%
%:%3605=1441%:%
%:%3606=1441%:%
%:%3607=1442%:%
%:%3608=1442%:%
%:%3609=1443%:%
%:%3610=1443%:%
%:%3611=1444%:%
%:%3612=1444%:%
%:%3613=1445%:%
%:%3614=1445%:%
%:%3615=1445%:%
%:%3616=1445%:%
%:%3617=1445%:%
%:%3618=1446%:%
%:%3619=1446%:%
%:%3620=1447%:%
%:%3621=1447%:%
%:%3622=1448%:%
%:%3623=1448%:%
%:%3624=1449%:%
%:%3625=1449%:%
%:%3626=1450%:%
%:%3627=1450%:%
%:%3628=1450%:%
%:%3629=1451%:%
%:%3630=1451%:%
%:%3631=1451%:%
%:%3632=1452%:%
%:%3633=1452%:%
%:%3634=1453%:%
%:%3635=1453%:%
%:%3636=1453%:%
%:%3637=1454%:%
%:%3638=1454%:%
%:%3639=1454%:%
%:%3640=1455%:%
%:%3641=1455%:%
%:%3642=1456%:%
%:%3643=1456%:%
%:%3644=1457%:%
%:%3645=1457%:%
%:%3646=1458%:%
%:%3647=1458%:%
%:%3648=1458%:%
%:%3649=1458%:%
%:%3650=1459%:%
%:%3651=1459%:%
%:%3652=1460%:%
%:%3653=1460%:%
%:%3654=1461%:%
%:%3655=1461%:%
%:%3656=1462%:%
%:%3657=1462%:%
%:%3658=1463%:%
%:%3659=1463%:%
%:%3660=1464%:%
%:%3661=1464%:%
%:%3662=1464%:%
%:%3663=1465%:%
%:%3664=1465%:%
%:%3665=1466%:%
%:%3666=1466%:%
%:%3667=1466%:%
%:%3668=1466%:%
%:%3669=1467%:%
%:%3670=1467%:%
%:%3671=1468%:%
%:%3672=1468%:%
%:%3673=1469%:%
%:%3674=1469%:%
%:%3675=1470%:%
%:%3676=1470%:%
%:%3677=1471%:%
%:%3678=1471%:%
%:%3679=1472%:%
%:%3680=1472%:%
%:%3681=1472%:%
%:%3682=1473%:%
%:%3683=1473%:%
%:%3684=1474%:%
%:%3685=1474%:%
%:%3686=1475%:%
%:%3687=1475%:%
%:%3688=1476%:%
%:%3689=1476%:%
%:%3690=1477%:%
%:%3691=1477%:%
%:%3692=1478%:%
%:%3693=1478%:%
%:%3694=1478%:%
%:%3695=1479%:%
%:%3696=1480%:%
%:%3697=1480%:%
%:%3698=1481%:%
%:%3699=1481%:%
%:%3700=1481%:%
%:%3701=1481%:%
%:%3702=1482%:%
%:%3703=1482%:%
%:%3704=1483%:%
%:%3705=1483%:%
%:%3706=1484%:%
%:%3707=1484%:%
%:%3708=1485%:%
%:%3709=1485%:%
%:%3710=1485%:%
%:%3711=1486%:%
%:%3712=1486%:%
%:%3713=1486%:%
%:%3714=1487%:%
%:%3715=1488%:%
%:%3716=1488%:%
%:%3717=1489%:%
%:%3718=1489%:%
%:%3719=1489%:%
%:%3720=1490%:%
%:%3721=1490%:%
%:%3722=1491%:%
%:%3723=1491%:%
%:%3724=1492%:%
%:%3725=1492%:%
%:%3726=1493%:%
%:%3727=1493%:%
%:%3728=1493%:%
%:%3729=1494%:%
%:%3730=1494%:%
%:%3731=1495%:%
%:%3732=1495%:%
%:%3733=1496%:%
%:%3748=1497%:%
%:%3758=1498%:%
%:%3759=1498%:%
%:%3760=1499%:%
%:%3761=1500%:%
%:%3762=1501%:%
%:%3763=1502%:%
%:%3764=1503%:%
%:%3771=1504%:%
%:%3772=1504%:%
%:%3773=1505%:%
%:%3774=1505%:%
%:%3775=1506%:%
%:%3776=1506%:%
%:%3777=1507%:%
%:%3778=1507%:%
%:%3779=1508%:%
%:%3780=1508%:%
%:%3781=1509%:%
%:%3782=1509%:%
%:%3783=1509%:%
%:%3784=1510%:%
%:%3785=1510%:%
%:%3786=1511%:%
%:%3787=1511%:%
%:%3788=1512%:%
%:%3789=1512%:%
%:%3790=1513%:%
%:%3791=1513%:%
%:%3792=1514%:%
%:%3793=1514%:%
%:%3794=1515%:%
%:%3795=1515%:%
%:%3796=1516%:%
%:%3797=1516%:%
%:%3798=1517%:%
%:%3799=1517%:%
%:%3800=1517%:%
%:%3801=1518%:%
%:%3802=1518%:%
%:%3803=1518%:%
%:%3804=1519%:%
%:%3805=1519%:%
%:%3806=1520%:%
%:%3807=1520%:%
%:%3808=1520%:%
%:%3809=1520%:%
%:%3810=1521%:%
%:%3811=1521%:%
%:%3812=1521%:%
%:%3813=1522%:%
%:%3814=1522%:%
%:%3815=1523%:%
%:%3816=1523%:%
%:%3817=1524%:%
%:%3818=1524%:%
%:%3819=1525%:%
%:%3820=1525%:%
%:%3821=1526%:%
%:%3822=1526%:%
%:%3823=1527%:%
%:%3824=1527%:%
%:%3825=1527%:%
%:%3826=1528%:%
%:%3827=1528%:%
%:%3828=1528%:%
%:%3829=1529%:%
%:%3830=1529%:%
%:%3831=1530%:%
%:%3832=1530%:%
%:%3833=1531%:%
%:%3834=1531%:%
%:%3835=1531%:%
%:%3836=1532%:%
%:%3837=1532%:%
%:%3838=1533%:%
%:%3839=1533%:%
%:%3840=1533%:%
%:%3841=1534%:%
%:%3842=1534%:%
%:%3843=1535%:%
%:%3844=1535%:%
%:%3845=1536%:%
%:%3846=1536%:%
%:%3847=1537%:%
%:%3848=1537%:%
%:%3849=1538%:%
%:%3850=1538%:%
%:%3851=1538%:%
%:%3852=1539%:%
%:%3853=1539%:%
%:%3854=1540%:%
%:%3855=1540%:%
%:%3856=1541%:%
%:%3857=1541%:%
%:%3858=1542%:%
%:%3859=1542%:%
%:%3860=1542%:%
%:%3861=1543%:%
%:%3862=1543%:%
%:%3863=1544%:%
%:%3864=1544%:%
%:%3865=1545%:%
%:%3866=1545%:%
%:%3867=1545%:%
%:%3868=1546%:%
%:%3874=1546%:%
%:%3877=1547%:%
%:%3878=1548%:%
%:%3879=1548%:%
%:%3880=1549%:%
%:%3881=1549%:%
%:%3884=1550%:%
%:%3889=1551%:%

%
\begin{isabellebody}%
\setisabellecontext{Separation{\isacharunderscore}{\kern0pt}Rename}%
%
\isadelimdocument
%
\endisadelimdocument
%
\isatagdocument
%
\isamarkupsection{Auxiliary renamings for Separation%
}
\isamarkuptrue%
%
\endisatagdocument
{\isafolddocument}%
%
\isadelimdocument
%
\endisadelimdocument
%
\isadelimtheory
%
\endisadelimtheory
%
\isatagtheory
\isacommand{theory}\isamarkupfalse%
\ Separation{\isacharunderscore}{\kern0pt}Rename\isanewline
\ \ \isakeyword{imports}\ Interface\ Renaming\isanewline
\isakeyword{begin}%
\endisatagtheory
{\isafoldtheory}%
%
\isadelimtheory
\isanewline
%
\endisadelimtheory
\isanewline
\isacommand{lemmas}\isamarkupfalse%
\ apply{\isacharunderscore}{\kern0pt}fun\ {\isacharequal}{\kern0pt}\ apply{\isacharunderscore}{\kern0pt}iff{\isacharbrackleft}{\kern0pt}THEN\ iffD{\isadigit{1}}{\isacharbrackright}{\kern0pt}\isanewline
\isanewline
\isacommand{lemma}\isamarkupfalse%
\ nth{\isacharunderscore}{\kern0pt}concat\ {\isacharcolon}{\kern0pt}\ {\isachardoublequoteopen}{\isacharbrackleft}{\kern0pt}p{\isacharcomma}{\kern0pt}t{\isacharbrackright}{\kern0pt}\ {\isasymin}\ list{\isacharparenleft}{\kern0pt}A{\isacharparenright}{\kern0pt}\ {\isasymLongrightarrow}\ env{\isasymin}\ list{\isacharparenleft}{\kern0pt}A{\isacharparenright}{\kern0pt}\ {\isasymLongrightarrow}\ nth{\isacharparenleft}{\kern0pt}{\isadigit{1}}\ {\isacharhash}{\kern0pt}{\isacharplus}{\kern0pt}\ length{\isacharparenleft}{\kern0pt}env{\isacharparenright}{\kern0pt}{\isacharcomma}{\kern0pt}{\isacharbrackleft}{\kern0pt}p{\isacharbrackright}{\kern0pt}{\isacharat}{\kern0pt}\ env\ {\isacharat}{\kern0pt}\ {\isacharbrackleft}{\kern0pt}t{\isacharbrackright}{\kern0pt}{\isacharparenright}{\kern0pt}\ {\isacharequal}{\kern0pt}\ t{\isachardoublequoteclose}\isanewline
%
\isadelimproof
\ \ %
\endisadelimproof
%
\isatagproof
\isacommand{by}\isamarkupfalse%
{\isacharparenleft}{\kern0pt}auto\ simp\ add{\isacharcolon}{\kern0pt}nth{\isacharunderscore}{\kern0pt}append{\isacharparenright}{\kern0pt}%
\endisatagproof
{\isafoldproof}%
%
\isadelimproof
\isanewline
%
\endisadelimproof
\isanewline
\isacommand{lemma}\isamarkupfalse%
\ nth{\isacharunderscore}{\kern0pt}concat{\isadigit{2}}\ {\isacharcolon}{\kern0pt}\ {\isachardoublequoteopen}env{\isasymin}\ list{\isacharparenleft}{\kern0pt}A{\isacharparenright}{\kern0pt}\ {\isasymLongrightarrow}\ nth{\isacharparenleft}{\kern0pt}length{\isacharparenleft}{\kern0pt}env{\isacharparenright}{\kern0pt}{\isacharcomma}{\kern0pt}env\ {\isacharat}{\kern0pt}\ {\isacharbrackleft}{\kern0pt}p{\isacharcomma}{\kern0pt}t{\isacharbrackright}{\kern0pt}{\isacharparenright}{\kern0pt}\ {\isacharequal}{\kern0pt}\ p{\isachardoublequoteclose}\isanewline
%
\isadelimproof
\ \ %
\endisadelimproof
%
\isatagproof
\isacommand{by}\isamarkupfalse%
{\isacharparenleft}{\kern0pt}auto\ simp\ add{\isacharcolon}{\kern0pt}nth{\isacharunderscore}{\kern0pt}append{\isacharparenright}{\kern0pt}%
\endisatagproof
{\isafoldproof}%
%
\isadelimproof
\isanewline
%
\endisadelimproof
\isanewline
\isacommand{lemma}\isamarkupfalse%
\ nth{\isacharunderscore}{\kern0pt}concat{\isadigit{3}}\ {\isacharcolon}{\kern0pt}\ {\isachardoublequoteopen}env{\isasymin}\ list{\isacharparenleft}{\kern0pt}A{\isacharparenright}{\kern0pt}\ {\isasymLongrightarrow}\ u\ {\isacharequal}{\kern0pt}\ nth{\isacharparenleft}{\kern0pt}succ{\isacharparenleft}{\kern0pt}length{\isacharparenleft}{\kern0pt}env{\isacharparenright}{\kern0pt}{\isacharparenright}{\kern0pt}{\isacharcomma}{\kern0pt}\ env\ {\isacharat}{\kern0pt}\ {\isacharbrackleft}{\kern0pt}pi{\isacharcomma}{\kern0pt}\ u{\isacharbrackright}{\kern0pt}{\isacharparenright}{\kern0pt}{\isachardoublequoteclose}\isanewline
%
\isadelimproof
\ \ %
\endisadelimproof
%
\isatagproof
\isacommand{by}\isamarkupfalse%
{\isacharparenleft}{\kern0pt}auto\ simp\ add{\isacharcolon}{\kern0pt}nth{\isacharunderscore}{\kern0pt}append{\isacharparenright}{\kern0pt}%
\endisatagproof
{\isafoldproof}%
%
\isadelimproof
\isanewline
%
\endisadelimproof
\isanewline
\isacommand{definition}\isamarkupfalse%
\ \isanewline
\ \ sep{\isacharunderscore}{\kern0pt}var\ {\isacharcolon}{\kern0pt}{\isacharcolon}{\kern0pt}\ {\isachardoublequoteopen}i\ {\isasymRightarrow}\ i{\isachardoublequoteclose}\ \isakeyword{where}\isanewline
\ \ {\isachardoublequoteopen}sep{\isacharunderscore}{\kern0pt}var{\isacharparenleft}{\kern0pt}n{\isacharparenright}{\kern0pt}\ {\isasymequiv}\ {\isacharbraceleft}{\kern0pt}{\isasymlangle}{\isadigit{0}}{\isacharcomma}{\kern0pt}{\isadigit{1}}{\isasymrangle}{\isacharcomma}{\kern0pt}{\isasymlangle}{\isadigit{1}}{\isacharcomma}{\kern0pt}{\isadigit{3}}{\isasymrangle}{\isacharcomma}{\kern0pt}{\isasymlangle}{\isadigit{2}}{\isacharcomma}{\kern0pt}{\isadigit{4}}{\isasymrangle}{\isacharcomma}{\kern0pt}{\isasymlangle}{\isadigit{3}}{\isacharcomma}{\kern0pt}{\isadigit{5}}{\isasymrangle}{\isacharcomma}{\kern0pt}{\isasymlangle}{\isadigit{4}}{\isacharcomma}{\kern0pt}{\isadigit{0}}{\isasymrangle}{\isacharcomma}{\kern0pt}{\isasymlangle}{\isadigit{5}}{\isacharhash}{\kern0pt}{\isacharplus}{\kern0pt}n{\isacharcomma}{\kern0pt}{\isadigit{6}}{\isasymrangle}{\isacharcomma}{\kern0pt}{\isasymlangle}{\isadigit{6}}{\isacharhash}{\kern0pt}{\isacharplus}{\kern0pt}n{\isacharcomma}{\kern0pt}{\isadigit{2}}{\isasymrangle}{\isacharbraceright}{\kern0pt}{\isachardoublequoteclose}\isanewline
\isanewline
\isacommand{definition}\isamarkupfalse%
\isanewline
\ \ sep{\isacharunderscore}{\kern0pt}env\ {\isacharcolon}{\kern0pt}{\isacharcolon}{\kern0pt}\ {\isachardoublequoteopen}i\ {\isasymRightarrow}\ i{\isachardoublequoteclose}\ \isakeyword{where}\isanewline
\ \ {\isachardoublequoteopen}sep{\isacharunderscore}{\kern0pt}env{\isacharparenleft}{\kern0pt}n{\isacharparenright}{\kern0pt}\ {\isasymequiv}\ {\isasymlambda}\ i\ {\isasymin}\ {\isacharparenleft}{\kern0pt}{\isadigit{5}}{\isacharhash}{\kern0pt}{\isacharplus}{\kern0pt}n{\isacharparenright}{\kern0pt}{\isacharminus}{\kern0pt}{\isadigit{5}}\ {\isachardot}{\kern0pt}\ i{\isacharhash}{\kern0pt}{\isacharplus}{\kern0pt}{\isadigit{2}}{\isachardoublequoteclose}\isanewline
\isanewline
\isacommand{definition}\isamarkupfalse%
\ weak\ {\isacharcolon}{\kern0pt}{\isacharcolon}{\kern0pt}\ {\isachardoublequoteopen}{\isacharbrackleft}{\kern0pt}i{\isacharcomma}{\kern0pt}\ i{\isacharbrackright}{\kern0pt}\ {\isasymRightarrow}\ i{\isachardoublequoteclose}\ \isakeyword{where}\isanewline
\ \ {\isachardoublequoteopen}weak{\isacharparenleft}{\kern0pt}n{\isacharcomma}{\kern0pt}m{\isacharparenright}{\kern0pt}\ {\isasymequiv}\ {\isacharbraceleft}{\kern0pt}i{\isacharhash}{\kern0pt}{\isacharplus}{\kern0pt}m\ {\isachardot}{\kern0pt}\ i\ {\isasymin}\ n{\isacharbraceright}{\kern0pt}{\isachardoublequoteclose}\isanewline
\isanewline
\isacommand{lemma}\isamarkupfalse%
\ weakD\ {\isacharcolon}{\kern0pt}\ \isanewline
\ \ \isakeyword{assumes}\ {\isachardoublequoteopen}n\ {\isasymin}\ nat{\isachardoublequoteclose}\ {\isachardoublequoteopen}k{\isasymin}nat{\isachardoublequoteclose}\ {\isachardoublequoteopen}x\ {\isasymin}\ weak{\isacharparenleft}{\kern0pt}n{\isacharcomma}{\kern0pt}k{\isacharparenright}{\kern0pt}{\isachardoublequoteclose}\isanewline
\ \ \isakeyword{shows}\ {\isachardoublequoteopen}{\isasymexists}\ i\ {\isasymin}\ n\ {\isachardot}{\kern0pt}\ x\ {\isacharequal}{\kern0pt}\ i{\isacharhash}{\kern0pt}{\isacharplus}{\kern0pt}k{\isachardoublequoteclose}\isanewline
%
\isadelimproof
\ \ %
\endisadelimproof
%
\isatagproof
\isacommand{using}\isamarkupfalse%
\ assms\ \isacommand{unfolding}\isamarkupfalse%
\ weak{\isacharunderscore}{\kern0pt}def\ \isacommand{by}\isamarkupfalse%
\ blast%
\endisatagproof
{\isafoldproof}%
%
\isadelimproof
\isanewline
%
\endisadelimproof
\isanewline
\isacommand{lemma}\isamarkupfalse%
\ weak{\isacharunderscore}{\kern0pt}equal\ {\isacharcolon}{\kern0pt}\isanewline
\ \ \isakeyword{assumes}\ {\isachardoublequoteopen}n{\isasymin}nat{\isachardoublequoteclose}\ {\isachardoublequoteopen}m{\isasymin}nat{\isachardoublequoteclose}\isanewline
\ \ \isakeyword{shows}\ {\isachardoublequoteopen}weak{\isacharparenleft}{\kern0pt}n{\isacharcomma}{\kern0pt}m{\isacharparenright}{\kern0pt}\ {\isacharequal}{\kern0pt}\ {\isacharparenleft}{\kern0pt}m{\isacharhash}{\kern0pt}{\isacharplus}{\kern0pt}n{\isacharparenright}{\kern0pt}\ {\isacharminus}{\kern0pt}\ m{\isachardoublequoteclose}\isanewline
%
\isadelimproof
%
\endisadelimproof
%
\isatagproof
\isacommand{proof}\isamarkupfalse%
\ {\isacharminus}{\kern0pt}\isanewline
\ \ \isacommand{have}\isamarkupfalse%
\ {\isachardoublequoteopen}weak{\isacharparenleft}{\kern0pt}n{\isacharcomma}{\kern0pt}m{\isacharparenright}{\kern0pt}{\isasymsubseteq}{\isacharparenleft}{\kern0pt}m{\isacharhash}{\kern0pt}{\isacharplus}{\kern0pt}n{\isacharparenright}{\kern0pt}{\isacharminus}{\kern0pt}m{\isachardoublequoteclose}\ \isanewline
\ \ \isacommand{proof}\isamarkupfalse%
{\isacharparenleft}{\kern0pt}intro\ subsetI{\isacharparenright}{\kern0pt}\isanewline
\ \ \ \ \isacommand{fix}\isamarkupfalse%
\ x\isanewline
\ \ \ \ \isacommand{assume}\isamarkupfalse%
\ {\isachardoublequoteopen}x{\isasymin}weak{\isacharparenleft}{\kern0pt}n{\isacharcomma}{\kern0pt}m{\isacharparenright}{\kern0pt}{\isachardoublequoteclose}\isanewline
\ \ \ \ \isacommand{with}\isamarkupfalse%
\ assms\ \isanewline
\ \ \ \ \isacommand{obtain}\isamarkupfalse%
\ i\ \isakeyword{where}\isanewline
\ \ \ \ \ \ {\isachardoublequoteopen}i{\isasymin}n{\isachardoublequoteclose}\ {\isachardoublequoteopen}x{\isacharequal}{\kern0pt}i{\isacharhash}{\kern0pt}{\isacharplus}{\kern0pt}m{\isachardoublequoteclose}\isanewline
\ \ \ \ \ \ \isacommand{using}\isamarkupfalse%
\ weakD\ \isacommand{by}\isamarkupfalse%
\ blast\isanewline
\ \ \ \ \isacommand{then}\isamarkupfalse%
\isanewline
\ \ \ \ \isacommand{have}\isamarkupfalse%
\ {\isachardoublequoteopen}m{\isasymle}i{\isacharhash}{\kern0pt}{\isacharplus}{\kern0pt}m{\isachardoublequoteclose}\ {\isachardoublequoteopen}i{\isacharless}{\kern0pt}n{\isachardoublequoteclose}\isanewline
\ \ \ \ \ \ \isacommand{using}\isamarkupfalse%
\ add{\isacharunderscore}{\kern0pt}le{\isacharunderscore}{\kern0pt}self{\isadigit{2}}{\isacharbrackleft}{\kern0pt}of\ m\ i{\isacharbrackright}{\kern0pt}\ {\isacartoucheopen}m{\isasymin}nat{\isacartoucheclose}\ {\isacartoucheopen}n{\isasymin}nat{\isacartoucheclose}\ ltI{\isacharbrackleft}{\kern0pt}OF\ {\isacartoucheopen}i{\isasymin}n{\isacartoucheclose}{\isacharbrackright}{\kern0pt}\ \isacommand{by}\isamarkupfalse%
\ simp{\isacharunderscore}{\kern0pt}all\isanewline
\ \ \ \ \isacommand{then}\isamarkupfalse%
\isanewline
\ \ \ \ \isacommand{have}\isamarkupfalse%
\ {\isachardoublequoteopen}{\isasymnot}i{\isacharhash}{\kern0pt}{\isacharplus}{\kern0pt}m{\isacharless}{\kern0pt}m{\isachardoublequoteclose}\isanewline
\ \ \ \ \ \ \isacommand{using}\isamarkupfalse%
\ not{\isacharunderscore}{\kern0pt}lt{\isacharunderscore}{\kern0pt}iff{\isacharunderscore}{\kern0pt}le\ in{\isacharunderscore}{\kern0pt}n{\isacharunderscore}{\kern0pt}in{\isacharunderscore}{\kern0pt}nat{\isacharbrackleft}{\kern0pt}OF\ {\isacartoucheopen}n{\isasymin}nat{\isacartoucheclose}\ {\isacartoucheopen}i{\isasymin}n{\isacartoucheclose}{\isacharbrackright}{\kern0pt}\ {\isacartoucheopen}m{\isasymin}nat{\isacartoucheclose}\ \isacommand{by}\isamarkupfalse%
\ simp\isanewline
\ \ \ \ \isacommand{with}\isamarkupfalse%
\ {\isacartoucheopen}x{\isacharequal}{\kern0pt}i{\isacharhash}{\kern0pt}{\isacharplus}{\kern0pt}m{\isacartoucheclose}\isanewline
\ \ \ \ \isacommand{have}\isamarkupfalse%
\ {\isachardoublequoteopen}x{\isasymnotin}m{\isachardoublequoteclose}\ \isanewline
\ \ \ \ \ \ \isacommand{using}\isamarkupfalse%
\ ltI\ {\isacartoucheopen}m{\isasymin}nat{\isacartoucheclose}\ \isacommand{by}\isamarkupfalse%
\ auto\isanewline
\ \ \ \ \isacommand{moreover}\isamarkupfalse%
\isanewline
\ \ \ \ \isacommand{from}\isamarkupfalse%
\ assms\ {\isacartoucheopen}x{\isacharequal}{\kern0pt}i{\isacharhash}{\kern0pt}{\isacharplus}{\kern0pt}m{\isacartoucheclose}\ {\isacartoucheopen}i{\isacharless}{\kern0pt}n{\isacartoucheclose}\isanewline
\ \ \ \ \isacommand{have}\isamarkupfalse%
\ {\isachardoublequoteopen}x{\isacharless}{\kern0pt}m{\isacharhash}{\kern0pt}{\isacharplus}{\kern0pt}n{\isachardoublequoteclose}\isanewline
\ \ \ \ \ \ \isacommand{using}\isamarkupfalse%
\ add{\isacharunderscore}{\kern0pt}lt{\isacharunderscore}{\kern0pt}mono{\isadigit{1}}{\isacharbrackleft}{\kern0pt}OF\ {\isacartoucheopen}i{\isacharless}{\kern0pt}n{\isacartoucheclose}\ {\isacartoucheopen}n{\isasymin}nat{\isacartoucheclose}{\isacharbrackright}{\kern0pt}\ \isacommand{by}\isamarkupfalse%
\ simp\isanewline
\ \ \ \ \isacommand{ultimately}\isamarkupfalse%
\isanewline
\ \ \ \ \isacommand{show}\isamarkupfalse%
\ {\isachardoublequoteopen}x{\isasymin}{\isacharparenleft}{\kern0pt}m{\isacharhash}{\kern0pt}{\isacharplus}{\kern0pt}n{\isacharparenright}{\kern0pt}{\isacharminus}{\kern0pt}m{\isachardoublequoteclose}\ \isanewline
\ \ \ \ \ \ \isacommand{using}\isamarkupfalse%
\ ltD\ DiffI\ \isacommand{by}\isamarkupfalse%
\ simp\isanewline
\ \ \isacommand{qed}\isamarkupfalse%
\isanewline
\ \ \isacommand{moreover}\isamarkupfalse%
\isanewline
\ \ \isacommand{have}\isamarkupfalse%
\ {\isachardoublequoteopen}{\isacharparenleft}{\kern0pt}m{\isacharhash}{\kern0pt}{\isacharplus}{\kern0pt}n{\isacharparenright}{\kern0pt}{\isacharminus}{\kern0pt}m{\isasymsubseteq}weak{\isacharparenleft}{\kern0pt}n{\isacharcomma}{\kern0pt}m{\isacharparenright}{\kern0pt}{\isachardoublequoteclose}\ \isanewline
\ \ \isacommand{proof}\isamarkupfalse%
\ {\isacharparenleft}{\kern0pt}intro\ subsetI{\isacharparenright}{\kern0pt}\isanewline
\ \ \ \ \isacommand{fix}\isamarkupfalse%
\ x\ \isanewline
\ \ \ \ \isacommand{assume}\isamarkupfalse%
\ {\isachardoublequoteopen}x{\isasymin}{\isacharparenleft}{\kern0pt}m{\isacharhash}{\kern0pt}{\isacharplus}{\kern0pt}n{\isacharparenright}{\kern0pt}{\isacharminus}{\kern0pt}m{\isachardoublequoteclose}\isanewline
\ \ \ \ \isacommand{then}\isamarkupfalse%
\ \isanewline
\ \ \ \ \isacommand{have}\isamarkupfalse%
\ {\isachardoublequoteopen}x{\isasymin}m{\isacharhash}{\kern0pt}{\isacharplus}{\kern0pt}n{\isachardoublequoteclose}\ {\isachardoublequoteopen}x{\isasymnotin}m{\isachardoublequoteclose}\isanewline
\ \ \ \ \ \ \isacommand{using}\isamarkupfalse%
\ DiffD{\isadigit{1}}{\isacharbrackleft}{\kern0pt}of\ x\ {\isachardoublequoteopen}n{\isacharhash}{\kern0pt}{\isacharplus}{\kern0pt}m{\isachardoublequoteclose}\ m{\isacharbrackright}{\kern0pt}\ DiffD{\isadigit{2}}{\isacharbrackleft}{\kern0pt}of\ x\ {\isachardoublequoteopen}n{\isacharhash}{\kern0pt}{\isacharplus}{\kern0pt}m{\isachardoublequoteclose}\ m{\isacharbrackright}{\kern0pt}\ \isacommand{by}\isamarkupfalse%
\ simp{\isacharunderscore}{\kern0pt}all\isanewline
\ \ \ \ \isacommand{then}\isamarkupfalse%
\isanewline
\ \ \ \ \isacommand{have}\isamarkupfalse%
\ {\isachardoublequoteopen}x{\isacharless}{\kern0pt}m{\isacharhash}{\kern0pt}{\isacharplus}{\kern0pt}n{\isachardoublequoteclose}\ {\isachardoublequoteopen}x{\isasymin}nat{\isachardoublequoteclose}\ \isanewline
\ \ \ \ \ \ \isacommand{using}\isamarkupfalse%
\ ltI\ in{\isacharunderscore}{\kern0pt}n{\isacharunderscore}{\kern0pt}in{\isacharunderscore}{\kern0pt}nat{\isacharbrackleft}{\kern0pt}OF\ add{\isacharunderscore}{\kern0pt}type{\isacharbrackleft}{\kern0pt}of\ m\ n{\isacharbrackright}{\kern0pt}{\isacharbrackright}{\kern0pt}\ \isacommand{by}\isamarkupfalse%
\ simp{\isacharunderscore}{\kern0pt}all\isanewline
\ \ \ \ \isacommand{then}\isamarkupfalse%
\isanewline
\ \ \ \ \isacommand{obtain}\isamarkupfalse%
\ i\ \isakeyword{where}\isanewline
\ \ \ \ \ \ {\isachardoublequoteopen}m{\isacharhash}{\kern0pt}{\isacharplus}{\kern0pt}n\ {\isacharequal}{\kern0pt}\ succ{\isacharparenleft}{\kern0pt}x{\isacharhash}{\kern0pt}{\isacharplus}{\kern0pt}i{\isacharparenright}{\kern0pt}{\isachardoublequoteclose}\ \isanewline
\ \ \ \ \ \ \isacommand{using}\isamarkupfalse%
\ less{\isacharunderscore}{\kern0pt}iff{\isacharunderscore}{\kern0pt}succ{\isacharunderscore}{\kern0pt}add{\isacharbrackleft}{\kern0pt}OF\ {\isacartoucheopen}x{\isasymin}nat{\isacartoucheclose}{\isacharcomma}{\kern0pt}of\ {\isachardoublequoteopen}m{\isacharhash}{\kern0pt}{\isacharplus}{\kern0pt}n{\isachardoublequoteclose}{\isacharbrackright}{\kern0pt}\ add{\isacharunderscore}{\kern0pt}type\ \isacommand{by}\isamarkupfalse%
\ auto\isanewline
\ \ \ \ \isacommand{then}\isamarkupfalse%
\ \isanewline
\ \ \ \ \isacommand{have}\isamarkupfalse%
\ {\isachardoublequoteopen}x{\isacharhash}{\kern0pt}{\isacharplus}{\kern0pt}i{\isacharless}{\kern0pt}m{\isacharhash}{\kern0pt}{\isacharplus}{\kern0pt}n{\isachardoublequoteclose}\ \isacommand{using}\isamarkupfalse%
\ succ{\isacharunderscore}{\kern0pt}le{\isacharunderscore}{\kern0pt}iff\ \isacommand{by}\isamarkupfalse%
\ simp\isanewline
\ \ \ \ \isacommand{with}\isamarkupfalse%
\ {\isacartoucheopen}x{\isasymnotin}m{\isacartoucheclose}\ \isanewline
\ \ \ \ \isacommand{have}\isamarkupfalse%
\ {\isachardoublequoteopen}{\isasymnot}x{\isacharless}{\kern0pt}m{\isachardoublequoteclose}\ \isacommand{using}\isamarkupfalse%
\ ltD\ \isacommand{by}\isamarkupfalse%
\ blast\isanewline
\ \ \ \ \isacommand{with}\isamarkupfalse%
\ {\isacartoucheopen}m{\isasymin}nat{\isacartoucheclose}\ {\isacartoucheopen}x{\isasymin}nat{\isacartoucheclose}\isanewline
\ \ \ \ \isacommand{have}\isamarkupfalse%
\ {\isachardoublequoteopen}m{\isasymle}x{\isachardoublequoteclose}\ \isacommand{using}\isamarkupfalse%
\ not{\isacharunderscore}{\kern0pt}lt{\isacharunderscore}{\kern0pt}iff{\isacharunderscore}{\kern0pt}le\ \isacommand{by}\isamarkupfalse%
\ simp\ \isanewline
\ \ \ \ \isacommand{with}\isamarkupfalse%
\ {\isacartoucheopen}x{\isacharless}{\kern0pt}m{\isacharhash}{\kern0pt}{\isacharplus}{\kern0pt}n{\isacartoucheclose}\ \ {\isacartoucheopen}n{\isasymin}nat{\isacartoucheclose}\isanewline
\ \ \ \ \isacommand{have}\isamarkupfalse%
\ {\isachardoublequoteopen}x{\isacharhash}{\kern0pt}{\isacharminus}{\kern0pt}m{\isacharless}{\kern0pt}m{\isacharhash}{\kern0pt}{\isacharplus}{\kern0pt}n{\isacharhash}{\kern0pt}{\isacharminus}{\kern0pt}m{\isachardoublequoteclose}\ \isanewline
\ \ \ \ \ \ \isacommand{using}\isamarkupfalse%
\ diff{\isacharunderscore}{\kern0pt}mono{\isacharbrackleft}{\kern0pt}OF\ {\isacartoucheopen}x{\isasymin}nat{\isacartoucheclose}\ {\isacharunderscore}{\kern0pt}\ {\isacartoucheopen}m{\isasymin}nat{\isacartoucheclose}{\isacharbrackright}{\kern0pt}\ \isacommand{by}\isamarkupfalse%
\ simp\isanewline
\ \ \ \ \isacommand{have}\isamarkupfalse%
\ {\isachardoublequoteopen}m{\isacharhash}{\kern0pt}{\isacharplus}{\kern0pt}n{\isacharhash}{\kern0pt}{\isacharminus}{\kern0pt}m\ {\isacharequal}{\kern0pt}\ \ n{\isachardoublequoteclose}\ \isacommand{using}\isamarkupfalse%
\ diff{\isacharunderscore}{\kern0pt}cancel{\isadigit{2}}\ {\isacartoucheopen}m{\isasymin}nat{\isacartoucheclose}\ {\isacartoucheopen}n{\isasymin}nat{\isacartoucheclose}\ \isacommand{by}\isamarkupfalse%
\ simp\ \ \ \isanewline
\ \ \ \ \isacommand{with}\isamarkupfalse%
\ {\isacartoucheopen}x{\isacharhash}{\kern0pt}{\isacharminus}{\kern0pt}m{\isacharless}{\kern0pt}m{\isacharhash}{\kern0pt}{\isacharplus}{\kern0pt}n{\isacharhash}{\kern0pt}{\isacharminus}{\kern0pt}m{\isacartoucheclose}\ {\isacartoucheopen}x{\isasymin}nat{\isacartoucheclose}\isanewline
\ \ \ \ \isacommand{have}\isamarkupfalse%
\ {\isachardoublequoteopen}x{\isacharhash}{\kern0pt}{\isacharminus}{\kern0pt}m\ {\isasymin}\ n{\isachardoublequoteclose}\ {\isachardoublequoteopen}x{\isacharequal}{\kern0pt}x{\isacharhash}{\kern0pt}{\isacharminus}{\kern0pt}m{\isacharhash}{\kern0pt}{\isacharplus}{\kern0pt}m{\isachardoublequoteclose}\ \isanewline
\ \ \ \ \ \ \isacommand{using}\isamarkupfalse%
\ ltD\ add{\isacharunderscore}{\kern0pt}diff{\isacharunderscore}{\kern0pt}inverse{\isadigit{2}}{\isacharbrackleft}{\kern0pt}OF\ {\isacartoucheopen}m{\isasymle}x{\isacartoucheclose}{\isacharbrackright}{\kern0pt}\ \isacommand{by}\isamarkupfalse%
\ simp{\isacharunderscore}{\kern0pt}all\isanewline
\ \ \ \ \isacommand{then}\isamarkupfalse%
\ \isanewline
\ \ \ \ \isacommand{show}\isamarkupfalse%
\ {\isachardoublequoteopen}x{\isasymin}weak{\isacharparenleft}{\kern0pt}n{\isacharcomma}{\kern0pt}m{\isacharparenright}{\kern0pt}{\isachardoublequoteclose}\ \isanewline
\ \ \ \ \ \ \isacommand{unfolding}\isamarkupfalse%
\ weak{\isacharunderscore}{\kern0pt}def\ \isacommand{by}\isamarkupfalse%
\ auto\isanewline
\ \ \isacommand{qed}\isamarkupfalse%
\isanewline
\ \ \isacommand{ultimately}\isamarkupfalse%
\isanewline
\ \ \isacommand{show}\isamarkupfalse%
\ {\isacharquery}{\kern0pt}thesis\ \isacommand{by}\isamarkupfalse%
\ auto\isanewline
\isacommand{qed}\isamarkupfalse%
%
\endisatagproof
{\isafoldproof}%
%
\isadelimproof
\isanewline
%
\endisadelimproof
\isanewline
\isacommand{lemma}\isamarkupfalse%
\ weak{\isacharunderscore}{\kern0pt}zero{\isacharcolon}{\kern0pt}\isanewline
\ \ \isakeyword{shows}\ {\isachardoublequoteopen}weak{\isacharparenleft}{\kern0pt}{\isadigit{0}}{\isacharcomma}{\kern0pt}n{\isacharparenright}{\kern0pt}\ {\isacharequal}{\kern0pt}\ {\isadigit{0}}{\isachardoublequoteclose}\isanewline
%
\isadelimproof
\ \ %
\endisadelimproof
%
\isatagproof
\isacommand{unfolding}\isamarkupfalse%
\ weak{\isacharunderscore}{\kern0pt}def\ \isacommand{by}\isamarkupfalse%
\ simp%
\endisatagproof
{\isafoldproof}%
%
\isadelimproof
\isanewline
%
\endisadelimproof
\isanewline
\isacommand{lemma}\isamarkupfalse%
\ weakening{\isacharunderscore}{\kern0pt}diff\ {\isacharcolon}{\kern0pt}\isanewline
\ \ \isakeyword{assumes}\ {\isachardoublequoteopen}n\ {\isasymin}\ nat{\isachardoublequoteclose}\isanewline
\ \ \isakeyword{shows}\ {\isachardoublequoteopen}weak{\isacharparenleft}{\kern0pt}n{\isacharcomma}{\kern0pt}{\isadigit{7}}{\isacharparenright}{\kern0pt}\ {\isacharminus}{\kern0pt}\ weak{\isacharparenleft}{\kern0pt}n{\isacharcomma}{\kern0pt}{\isadigit{5}}{\isacharparenright}{\kern0pt}\ {\isasymsubseteq}\ {\isacharbraceleft}{\kern0pt}{\isadigit{5}}{\isacharhash}{\kern0pt}{\isacharplus}{\kern0pt}n{\isacharcomma}{\kern0pt}\ {\isadigit{6}}{\isacharhash}{\kern0pt}{\isacharplus}{\kern0pt}n{\isacharbraceright}{\kern0pt}{\isachardoublequoteclose}\isanewline
%
\isadelimproof
\ \ %
\endisadelimproof
%
\isatagproof
\isacommand{unfolding}\isamarkupfalse%
\ weak{\isacharunderscore}{\kern0pt}def\ \isacommand{using}\isamarkupfalse%
\ assms\isanewline
\isacommand{proof}\isamarkupfalse%
{\isacharparenleft}{\kern0pt}auto{\isacharparenright}{\kern0pt}\isanewline
\ \ \isacommand{{\isacharbraceleft}{\kern0pt}}\isamarkupfalse%
\isanewline
\ \ \ \ \isacommand{fix}\isamarkupfalse%
\ i\isanewline
\ \ \ \ \isacommand{assume}\isamarkupfalse%
\ {\isachardoublequoteopen}i{\isasymin}n{\isachardoublequoteclose}\ {\isachardoublequoteopen}succ{\isacharparenleft}{\kern0pt}succ{\isacharparenleft}{\kern0pt}natify{\isacharparenleft}{\kern0pt}i{\isacharparenright}{\kern0pt}{\isacharparenright}{\kern0pt}{\isacharparenright}{\kern0pt}{\isasymnoteq}n{\isachardoublequoteclose}\ {\isachardoublequoteopen}{\isasymforall}w{\isasymin}n{\isachardot}{\kern0pt}\ succ{\isacharparenleft}{\kern0pt}succ{\isacharparenleft}{\kern0pt}natify{\isacharparenleft}{\kern0pt}i{\isacharparenright}{\kern0pt}{\isacharparenright}{\kern0pt}{\isacharparenright}{\kern0pt}\ {\isasymnoteq}\ natify{\isacharparenleft}{\kern0pt}w{\isacharparenright}{\kern0pt}{\isachardoublequoteclose}\isanewline
\ \ \ \ \isacommand{then}\isamarkupfalse%
\ \isanewline
\ \ \ \ \isacommand{have}\isamarkupfalse%
\ {\isachardoublequoteopen}i{\isacharless}{\kern0pt}n{\isachardoublequoteclose}\ \isanewline
\ \ \ \ \ \ \isacommand{using}\isamarkupfalse%
\ ltI\ {\isacartoucheopen}n{\isasymin}nat{\isacartoucheclose}\ \isacommand{by}\isamarkupfalse%
\ simp\isanewline
\ \ \ \ \isacommand{from}\isamarkupfalse%
\ {\isacartoucheopen}n{\isasymin}nat{\isacartoucheclose}\ {\isacartoucheopen}i{\isasymin}n{\isacartoucheclose}\ {\isacartoucheopen}succ{\isacharparenleft}{\kern0pt}succ{\isacharparenleft}{\kern0pt}natify{\isacharparenleft}{\kern0pt}i{\isacharparenright}{\kern0pt}{\isacharparenright}{\kern0pt}{\isacharparenright}{\kern0pt}{\isasymnoteq}n{\isacartoucheclose}\isanewline
\ \ \ \ \isacommand{have}\isamarkupfalse%
\ {\isachardoublequoteopen}i{\isasymin}nat{\isachardoublequoteclose}\ {\isachardoublequoteopen}succ{\isacharparenleft}{\kern0pt}succ{\isacharparenleft}{\kern0pt}i{\isacharparenright}{\kern0pt}{\isacharparenright}{\kern0pt}{\isasymnoteq}n{\isachardoublequoteclose}\ \isacommand{using}\isamarkupfalse%
\ in{\isacharunderscore}{\kern0pt}n{\isacharunderscore}{\kern0pt}in{\isacharunderscore}{\kern0pt}nat\ \isacommand{by}\isamarkupfalse%
\ simp{\isacharunderscore}{\kern0pt}all\isanewline
\ \ \ \ \isacommand{from}\isamarkupfalse%
\ {\isacartoucheopen}i{\isacharless}{\kern0pt}n{\isacartoucheclose}\isanewline
\ \ \ \ \isacommand{have}\isamarkupfalse%
\ {\isachardoublequoteopen}succ{\isacharparenleft}{\kern0pt}i{\isacharparenright}{\kern0pt}{\isasymle}n{\isachardoublequoteclose}\ \isacommand{using}\isamarkupfalse%
\ succ{\isacharunderscore}{\kern0pt}leI\ \isacommand{by}\isamarkupfalse%
\ simp\isanewline
\ \ \ \ \isacommand{with}\isamarkupfalse%
\ {\isacartoucheopen}n{\isasymin}nat{\isacartoucheclose}\isanewline
\ \ \ \ \isacommand{consider}\isamarkupfalse%
\ {\isacharparenleft}{\kern0pt}a{\isacharparenright}{\kern0pt}\ {\isachardoublequoteopen}succ{\isacharparenleft}{\kern0pt}i{\isacharparenright}{\kern0pt}\ {\isacharequal}{\kern0pt}\ n{\isachardoublequoteclose}\ {\isacharbar}{\kern0pt}\ {\isacharparenleft}{\kern0pt}b{\isacharparenright}{\kern0pt}\ {\isachardoublequoteopen}succ{\isacharparenleft}{\kern0pt}i{\isacharparenright}{\kern0pt}\ {\isacharless}{\kern0pt}\ n{\isachardoublequoteclose}\isanewline
\ \ \ \ \ \ \isacommand{using}\isamarkupfalse%
\ leD\ \isacommand{by}\isamarkupfalse%
\ auto\isanewline
\ \ \ \ \isacommand{then}\isamarkupfalse%
\ \isacommand{have}\isamarkupfalse%
\ {\isachardoublequoteopen}succ{\isacharparenleft}{\kern0pt}i{\isacharparenright}{\kern0pt}\ {\isacharequal}{\kern0pt}\ n{\isachardoublequoteclose}\ \isanewline
\ \ \ \ \isacommand{proof}\isamarkupfalse%
\ cases\isanewline
\ \ \ \ \ \ \isacommand{case}\isamarkupfalse%
\ a\isanewline
\ \ \ \ \ \ \isacommand{then}\isamarkupfalse%
\ \isacommand{show}\isamarkupfalse%
\ {\isacharquery}{\kern0pt}thesis\ \isacommand{{\isachardot}{\kern0pt}}\isamarkupfalse%
\isanewline
\ \ \ \ \isacommand{next}\isamarkupfalse%
\isanewline
\ \ \ \ \ \ \isacommand{case}\isamarkupfalse%
\ b\isanewline
\ \ \ \ \ \ \isacommand{then}\isamarkupfalse%
\ \isanewline
\ \ \ \ \ \ \isacommand{have}\isamarkupfalse%
\ {\isachardoublequoteopen}succ{\isacharparenleft}{\kern0pt}succ{\isacharparenleft}{\kern0pt}i{\isacharparenright}{\kern0pt}{\isacharparenright}{\kern0pt}{\isasymle}n{\isachardoublequoteclose}\ \isacommand{using}\isamarkupfalse%
\ succ{\isacharunderscore}{\kern0pt}leI\ \isacommand{by}\isamarkupfalse%
\ simp\isanewline
\ \ \ \ \ \ \isacommand{with}\isamarkupfalse%
\ {\isacartoucheopen}n{\isasymin}nat{\isacartoucheclose}\isanewline
\ \ \ \ \ \ \isacommand{consider}\isamarkupfalse%
\ {\isacharparenleft}{\kern0pt}a{\isacharparenright}{\kern0pt}\ {\isachardoublequoteopen}succ{\isacharparenleft}{\kern0pt}succ{\isacharparenleft}{\kern0pt}i{\isacharparenright}{\kern0pt}{\isacharparenright}{\kern0pt}\ {\isacharequal}{\kern0pt}\ n{\isachardoublequoteclose}\ {\isacharbar}{\kern0pt}\ {\isacharparenleft}{\kern0pt}b{\isacharparenright}{\kern0pt}\ {\isachardoublequoteopen}succ{\isacharparenleft}{\kern0pt}succ{\isacharparenleft}{\kern0pt}i{\isacharparenright}{\kern0pt}{\isacharparenright}{\kern0pt}\ {\isacharless}{\kern0pt}\ n{\isachardoublequoteclose}\isanewline
\ \ \ \ \ \ \ \ \isacommand{using}\isamarkupfalse%
\ leD\ \isacommand{by}\isamarkupfalse%
\ auto\isanewline
\ \ \ \ \ \ \isacommand{then}\isamarkupfalse%
\ \isacommand{have}\isamarkupfalse%
\ {\isachardoublequoteopen}succ{\isacharparenleft}{\kern0pt}i{\isacharparenright}{\kern0pt}\ {\isacharequal}{\kern0pt}\ n{\isachardoublequoteclose}\ \isanewline
\ \ \ \ \ \ \isacommand{proof}\isamarkupfalse%
\ cases\isanewline
\ \ \ \ \ \ \ \ \isacommand{case}\isamarkupfalse%
\ a\isanewline
\ \ \ \ \ \ \ \ \isacommand{with}\isamarkupfalse%
\ {\isacartoucheopen}succ{\isacharparenleft}{\kern0pt}succ{\isacharparenleft}{\kern0pt}i{\isacharparenright}{\kern0pt}{\isacharparenright}{\kern0pt}{\isasymnoteq}n{\isacartoucheclose}\ \isacommand{show}\isamarkupfalse%
\ {\isacharquery}{\kern0pt}thesis\ \isacommand{by}\isamarkupfalse%
\ blast\isanewline
\ \ \ \ \ \ \isacommand{next}\isamarkupfalse%
\isanewline
\ \ \ \ \ \ \ \ \isacommand{case}\isamarkupfalse%
\ b\isanewline
\ \ \ \ \ \ \ \ \isacommand{then}\isamarkupfalse%
\ \isanewline
\ \ \ \ \ \ \ \ \isacommand{have}\isamarkupfalse%
\ {\isachardoublequoteopen}succ{\isacharparenleft}{\kern0pt}succ{\isacharparenleft}{\kern0pt}i{\isacharparenright}{\kern0pt}{\isacharparenright}{\kern0pt}{\isasymin}n{\isachardoublequoteclose}\ \isacommand{using}\isamarkupfalse%
\ ltD\ \isacommand{by}\isamarkupfalse%
\ simp\isanewline
\ \ \ \ \ \ \ \ \isacommand{with}\isamarkupfalse%
\ {\isacartoucheopen}i{\isasymin}nat{\isacartoucheclose}\isanewline
\ \ \ \ \ \ \ \ \isacommand{have}\isamarkupfalse%
\ {\isachardoublequoteopen}succ{\isacharparenleft}{\kern0pt}succ{\isacharparenleft}{\kern0pt}natify{\isacharparenleft}{\kern0pt}i{\isacharparenright}{\kern0pt}{\isacharparenright}{\kern0pt}{\isacharparenright}{\kern0pt}\ {\isasymnoteq}\ natify{\isacharparenleft}{\kern0pt}succ{\isacharparenleft}{\kern0pt}succ{\isacharparenleft}{\kern0pt}i{\isacharparenright}{\kern0pt}{\isacharparenright}{\kern0pt}{\isacharparenright}{\kern0pt}{\isachardoublequoteclose}\isanewline
\ \ \ \ \ \ \ \ \ \ \isacommand{using}\isamarkupfalse%
\ \ {\isacartoucheopen}{\isasymforall}w{\isasymin}n{\isachardot}{\kern0pt}\ succ{\isacharparenleft}{\kern0pt}succ{\isacharparenleft}{\kern0pt}natify{\isacharparenleft}{\kern0pt}i{\isacharparenright}{\kern0pt}{\isacharparenright}{\kern0pt}{\isacharparenright}{\kern0pt}\ {\isasymnoteq}\ natify{\isacharparenleft}{\kern0pt}w{\isacharparenright}{\kern0pt}{\isacartoucheclose}\ \isacommand{by}\isamarkupfalse%
\ auto\isanewline
\ \ \ \ \ \ \ \ \isacommand{then}\isamarkupfalse%
\ \isanewline
\ \ \ \ \ \ \ \ \isacommand{have}\isamarkupfalse%
\ {\isachardoublequoteopen}False{\isachardoublequoteclose}\ \isacommand{using}\isamarkupfalse%
\ {\isacartoucheopen}i{\isasymin}nat{\isacartoucheclose}\ \isacommand{by}\isamarkupfalse%
\ auto\isanewline
\ \ \ \ \ \ \ \ \isacommand{then}\isamarkupfalse%
\ \isacommand{show}\isamarkupfalse%
\ {\isacharquery}{\kern0pt}thesis\ \isacommand{by}\isamarkupfalse%
\ blast\isanewline
\ \ \ \ \ \ \isacommand{qed}\isamarkupfalse%
\isanewline
\ \ \ \ \ \ \isacommand{then}\isamarkupfalse%
\ \isacommand{show}\isamarkupfalse%
\ {\isacharquery}{\kern0pt}thesis\ \isacommand{{\isachardot}{\kern0pt}}\isamarkupfalse%
\isanewline
\ \ \ \ \isacommand{qed}\isamarkupfalse%
\isanewline
\ \ \ \ \isacommand{with}\isamarkupfalse%
\ {\isacartoucheopen}i{\isasymin}nat{\isacartoucheclose}\ \isacommand{have}\isamarkupfalse%
\ {\isachardoublequoteopen}succ{\isacharparenleft}{\kern0pt}natify{\isacharparenleft}{\kern0pt}i{\isacharparenright}{\kern0pt}{\isacharparenright}{\kern0pt}\ {\isacharequal}{\kern0pt}\ n{\isachardoublequoteclose}\ \isacommand{by}\isamarkupfalse%
\ simp\isanewline
\ \ \isacommand{{\isacharbraceright}{\kern0pt}}\isamarkupfalse%
\isanewline
\ \ \isacommand{then}\isamarkupfalse%
\ \isanewline
\ \ \isacommand{show}\isamarkupfalse%
\ {\isachardoublequoteopen}n\ {\isasymin}\ nat\ {\isasymLongrightarrow}\ \isanewline
\ \ \ \ succ{\isacharparenleft}{\kern0pt}succ{\isacharparenleft}{\kern0pt}natify{\isacharparenleft}{\kern0pt}y{\isacharparenright}{\kern0pt}{\isacharparenright}{\kern0pt}{\isacharparenright}{\kern0pt}\ {\isasymnoteq}\ n\ {\isasymLongrightarrow}\ \isanewline
\ \ \ \ {\isasymforall}x{\isasymin}n{\isachardot}{\kern0pt}\ succ{\isacharparenleft}{\kern0pt}succ{\isacharparenleft}{\kern0pt}natify{\isacharparenleft}{\kern0pt}y{\isacharparenright}{\kern0pt}{\isacharparenright}{\kern0pt}{\isacharparenright}{\kern0pt}\ {\isasymnoteq}\ natify{\isacharparenleft}{\kern0pt}x{\isacharparenright}{\kern0pt}\ {\isasymLongrightarrow}\isanewline
\ \ \ \ y\ {\isasymin}\ n\ {\isasymLongrightarrow}\ succ{\isacharparenleft}{\kern0pt}natify{\isacharparenleft}{\kern0pt}y{\isacharparenright}{\kern0pt}{\isacharparenright}{\kern0pt}\ {\isacharequal}{\kern0pt}\ n{\isachardoublequoteclose}\ \isakeyword{for}\ y\isanewline
\ \ \ \ \isacommand{by}\isamarkupfalse%
\ blast\isanewline
\isacommand{qed}\isamarkupfalse%
%
\endisatagproof
{\isafoldproof}%
%
\isadelimproof
\isanewline
%
\endisadelimproof
\isanewline
\isacommand{lemma}\isamarkupfalse%
\ in{\isacharunderscore}{\kern0pt}add{\isacharunderscore}{\kern0pt}del\ {\isacharcolon}{\kern0pt}\isanewline
\ \ \isakeyword{assumes}\ {\isachardoublequoteopen}x{\isasymin}j{\isacharhash}{\kern0pt}{\isacharplus}{\kern0pt}n{\isachardoublequoteclose}\ {\isachardoublequoteopen}n{\isasymin}nat{\isachardoublequoteclose}\ {\isachardoublequoteopen}j{\isasymin}nat{\isachardoublequoteclose}\isanewline
\ \ \isakeyword{shows}\ {\isachardoublequoteopen}x\ {\isacharless}{\kern0pt}\ j\ {\isasymor}\ x\ {\isasymin}\ weak{\isacharparenleft}{\kern0pt}n{\isacharcomma}{\kern0pt}j{\isacharparenright}{\kern0pt}{\isachardoublequoteclose}\ \isanewline
%
\isadelimproof
%
\endisadelimproof
%
\isatagproof
\isacommand{proof}\isamarkupfalse%
\ {\isacharparenleft}{\kern0pt}cases\ {\isachardoublequoteopen}x{\isacharless}{\kern0pt}j{\isachardoublequoteclose}{\isacharparenright}{\kern0pt}\isanewline
\ \ \isacommand{case}\isamarkupfalse%
\ True\isanewline
\ \ \isacommand{then}\isamarkupfalse%
\ \isacommand{show}\isamarkupfalse%
\ {\isacharquery}{\kern0pt}thesis\ \isacommand{{\isachardot}{\kern0pt}{\isachardot}{\kern0pt}}\isamarkupfalse%
\isanewline
\isacommand{next}\isamarkupfalse%
\isanewline
\ \ \isacommand{case}\isamarkupfalse%
\ False\isanewline
\ \ \isacommand{have}\isamarkupfalse%
\ {\isachardoublequoteopen}x{\isasymin}nat{\isachardoublequoteclose}\ {\isachardoublequoteopen}j{\isacharhash}{\kern0pt}{\isacharplus}{\kern0pt}n{\isasymin}nat{\isachardoublequoteclose}\isanewline
\ \ \ \ \isacommand{using}\isamarkupfalse%
\ in{\isacharunderscore}{\kern0pt}n{\isacharunderscore}{\kern0pt}in{\isacharunderscore}{\kern0pt}nat{\isacharbrackleft}{\kern0pt}OF\ {\isacharunderscore}{\kern0pt}\ {\isacartoucheopen}x{\isasymin}j{\isacharhash}{\kern0pt}{\isacharplus}{\kern0pt}n{\isacartoucheclose}{\isacharbrackright}{\kern0pt}\ assms\ \isacommand{by}\isamarkupfalse%
\ simp{\isacharunderscore}{\kern0pt}all\isanewline
\ \ \isacommand{then}\isamarkupfalse%
\isanewline
\ \ \isacommand{have}\isamarkupfalse%
\ {\isachardoublequoteopen}j\ {\isasymle}\ x{\isachardoublequoteclose}\ {\isachardoublequoteopen}x\ {\isacharless}{\kern0pt}\ j{\isacharhash}{\kern0pt}{\isacharplus}{\kern0pt}n{\isachardoublequoteclose}\ \isanewline
\ \ \ \ \isacommand{using}\isamarkupfalse%
\ not{\isacharunderscore}{\kern0pt}lt{\isacharunderscore}{\kern0pt}iff{\isacharunderscore}{\kern0pt}le\ False\ {\isacartoucheopen}j{\isasymin}nat{\isacartoucheclose}\ {\isacartoucheopen}n{\isasymin}nat{\isacartoucheclose}\ ltI{\isacharbrackleft}{\kern0pt}OF\ {\isacartoucheopen}x{\isasymin}j{\isacharhash}{\kern0pt}{\isacharplus}{\kern0pt}n{\isacartoucheclose}{\isacharbrackright}{\kern0pt}\ \isacommand{by}\isamarkupfalse%
\ auto\isanewline
\ \ \isacommand{then}\isamarkupfalse%
\ \isanewline
\ \ \isacommand{have}\isamarkupfalse%
\ {\isachardoublequoteopen}x{\isacharhash}{\kern0pt}{\isacharminus}{\kern0pt}j\ {\isacharless}{\kern0pt}\ {\isacharparenleft}{\kern0pt}j\ {\isacharhash}{\kern0pt}{\isacharplus}{\kern0pt}\ n{\isacharparenright}{\kern0pt}\ {\isacharhash}{\kern0pt}{\isacharminus}{\kern0pt}\ j{\isachardoublequoteclose}\ {\isachardoublequoteopen}x\ {\isacharequal}{\kern0pt}\ j\ {\isacharhash}{\kern0pt}{\isacharplus}{\kern0pt}\ {\isacharparenleft}{\kern0pt}x\ {\isacharhash}{\kern0pt}{\isacharminus}{\kern0pt}j{\isacharparenright}{\kern0pt}{\isachardoublequoteclose}\isanewline
\ \ \ \ \isacommand{using}\isamarkupfalse%
\ diff{\isacharunderscore}{\kern0pt}mono\ {\isacartoucheopen}x{\isasymin}nat{\isacartoucheclose}\ {\isacartoucheopen}j{\isacharhash}{\kern0pt}{\isacharplus}{\kern0pt}n{\isasymin}nat{\isacartoucheclose}\ {\isacartoucheopen}j{\isasymin}nat{\isacartoucheclose}\ {\isacartoucheopen}n{\isasymin}nat{\isacartoucheclose}\ \isanewline
\ \ \ \ \ \ add{\isacharunderscore}{\kern0pt}diff{\isacharunderscore}{\kern0pt}inverse{\isacharbrackleft}{\kern0pt}OF\ {\isacartoucheopen}j{\isasymle}x{\isacartoucheclose}{\isacharbrackright}{\kern0pt}\ \isacommand{by}\isamarkupfalse%
\ simp{\isacharunderscore}{\kern0pt}all\isanewline
\ \ \isacommand{then}\isamarkupfalse%
\ \isanewline
\ \ \isacommand{have}\isamarkupfalse%
\ {\isachardoublequoteopen}x{\isacharhash}{\kern0pt}{\isacharminus}{\kern0pt}j\ {\isacharless}{\kern0pt}\ n{\isachardoublequoteclose}\ {\isachardoublequoteopen}x\ {\isacharequal}{\kern0pt}\ {\isacharparenleft}{\kern0pt}x\ {\isacharhash}{\kern0pt}{\isacharminus}{\kern0pt}j\ {\isacharparenright}{\kern0pt}\ {\isacharhash}{\kern0pt}{\isacharplus}{\kern0pt}\ j{\isachardoublequoteclose}\isanewline
\ \ \ \ \isacommand{using}\isamarkupfalse%
\ diff{\isacharunderscore}{\kern0pt}add{\isacharunderscore}{\kern0pt}inverse\ {\isacartoucheopen}n{\isasymin}nat{\isacartoucheclose}\ add{\isacharunderscore}{\kern0pt}commute\ \isacommand{by}\isamarkupfalse%
\ simp{\isacharunderscore}{\kern0pt}all\isanewline
\ \ \isacommand{then}\isamarkupfalse%
\ \isanewline
\ \ \isacommand{have}\isamarkupfalse%
\ {\isachardoublequoteopen}x{\isacharhash}{\kern0pt}{\isacharminus}{\kern0pt}j\ {\isasymin}n{\isachardoublequoteclose}\ \isacommand{using}\isamarkupfalse%
\ ltD\ \isacommand{by}\isamarkupfalse%
\ simp\isanewline
\ \ \isacommand{then}\isamarkupfalse%
\ \isanewline
\ \ \isacommand{have}\isamarkupfalse%
\ {\isachardoublequoteopen}x\ {\isasymin}\ weak{\isacharparenleft}{\kern0pt}n{\isacharcomma}{\kern0pt}j{\isacharparenright}{\kern0pt}{\isachardoublequoteclose}\ \isanewline
\ \ \ \ \isacommand{unfolding}\isamarkupfalse%
\ weak{\isacharunderscore}{\kern0pt}def\isanewline
\ \ \ \ \isacommand{using}\isamarkupfalse%
\ {\isacartoucheopen}x{\isacharequal}{\kern0pt}\ {\isacharparenleft}{\kern0pt}x{\isacharhash}{\kern0pt}{\isacharminus}{\kern0pt}j{\isacharparenright}{\kern0pt}\ {\isacharhash}{\kern0pt}{\isacharplus}{\kern0pt}j{\isacartoucheclose}\ RepFunI{\isacharbrackleft}{\kern0pt}OF\ {\isacartoucheopen}x{\isacharhash}{\kern0pt}{\isacharminus}{\kern0pt}j{\isasymin}n{\isacartoucheclose}{\isacharbrackright}{\kern0pt}\ add{\isacharunderscore}{\kern0pt}commute\ \isacommand{by}\isamarkupfalse%
\ force\isanewline
\ \ \isacommand{then}\isamarkupfalse%
\ \isacommand{show}\isamarkupfalse%
\ {\isacharquery}{\kern0pt}thesis\ \ \isacommand{{\isachardot}{\kern0pt}{\isachardot}{\kern0pt}}\isamarkupfalse%
\isanewline
\isacommand{qed}\isamarkupfalse%
%
\endisatagproof
{\isafoldproof}%
%
\isadelimproof
\isanewline
%
\endisadelimproof
\isanewline
\isanewline
\isacommand{lemma}\isamarkupfalse%
\ sep{\isacharunderscore}{\kern0pt}env{\isacharunderscore}{\kern0pt}action{\isacharcolon}{\kern0pt}\isanewline
\ \ \isakeyword{assumes}\isanewline
\ \ \ \ {\isachardoublequoteopen}{\isacharbrackleft}{\kern0pt}t{\isacharcomma}{\kern0pt}p{\isacharcomma}{\kern0pt}u{\isacharcomma}{\kern0pt}P{\isacharcomma}{\kern0pt}leq{\isacharcomma}{\kern0pt}o{\isacharcomma}{\kern0pt}pi{\isacharbrackright}{\kern0pt}\ {\isasymin}\ list{\isacharparenleft}{\kern0pt}M{\isacharparenright}{\kern0pt}{\isachardoublequoteclose}\isanewline
\ \ \ \ {\isachardoublequoteopen}env\ {\isasymin}\ list{\isacharparenleft}{\kern0pt}M{\isacharparenright}{\kern0pt}{\isachardoublequoteclose}\isanewline
\ \ \isakeyword{shows}\ {\isachardoublequoteopen}{\isasymforall}\ i\ {\isachardot}{\kern0pt}\ i\ {\isasymin}\ weak{\isacharparenleft}{\kern0pt}length{\isacharparenleft}{\kern0pt}env{\isacharparenright}{\kern0pt}{\isacharcomma}{\kern0pt}{\isadigit{5}}{\isacharparenright}{\kern0pt}\ {\isasymlongrightarrow}\ \isanewline
\ \ \ \ \ \ nth{\isacharparenleft}{\kern0pt}sep{\isacharunderscore}{\kern0pt}env{\isacharparenleft}{\kern0pt}length{\isacharparenleft}{\kern0pt}env{\isacharparenright}{\kern0pt}{\isacharparenright}{\kern0pt}{\isacharbackquote}{\kern0pt}i{\isacharcomma}{\kern0pt}{\isacharbrackleft}{\kern0pt}t{\isacharcomma}{\kern0pt}p{\isacharcomma}{\kern0pt}u{\isacharcomma}{\kern0pt}P{\isacharcomma}{\kern0pt}leq{\isacharcomma}{\kern0pt}o{\isacharcomma}{\kern0pt}pi{\isacharbrackright}{\kern0pt}{\isacharat}{\kern0pt}env{\isacharparenright}{\kern0pt}\ {\isacharequal}{\kern0pt}\ nth{\isacharparenleft}{\kern0pt}i{\isacharcomma}{\kern0pt}{\isacharbrackleft}{\kern0pt}p{\isacharcomma}{\kern0pt}P{\isacharcomma}{\kern0pt}leq{\isacharcomma}{\kern0pt}o{\isacharcomma}{\kern0pt}t{\isacharbrackright}{\kern0pt}\ {\isacharat}{\kern0pt}\ env\ {\isacharat}{\kern0pt}\ {\isacharbrackleft}{\kern0pt}pi{\isacharcomma}{\kern0pt}u{\isacharbrackright}{\kern0pt}{\isacharparenright}{\kern0pt}{\isachardoublequoteclose}\isanewline
%
\isadelimproof
%
\endisadelimproof
%
\isatagproof
\isacommand{proof}\isamarkupfalse%
\ {\isacharminus}{\kern0pt}\isanewline
\ \ \isacommand{from}\isamarkupfalse%
\ assms\isanewline
\ \ \isacommand{have}\isamarkupfalse%
\ A{\isacharcolon}{\kern0pt}\ {\isachardoublequoteopen}{\isadigit{5}}{\isacharhash}{\kern0pt}{\isacharplus}{\kern0pt}length{\isacharparenleft}{\kern0pt}env{\isacharparenright}{\kern0pt}{\isasymin}nat{\isachardoublequoteclose}\ {\isachardoublequoteopen}{\isacharbrackleft}{\kern0pt}p{\isacharcomma}{\kern0pt}\ P{\isacharcomma}{\kern0pt}\ leq{\isacharcomma}{\kern0pt}\ o{\isacharcomma}{\kern0pt}\ t{\isacharbrackright}{\kern0pt}\ {\isasymin}list{\isacharparenleft}{\kern0pt}M{\isacharparenright}{\kern0pt}{\isachardoublequoteclose}\isanewline
\ \ \ \ \isacommand{by}\isamarkupfalse%
\ simp{\isacharunderscore}{\kern0pt}all\isanewline
\ \ \isacommand{let}\isamarkupfalse%
\ {\isacharquery}{\kern0pt}f{\isacharequal}{\kern0pt}{\isachardoublequoteopen}sep{\isacharunderscore}{\kern0pt}env{\isacharparenleft}{\kern0pt}length{\isacharparenleft}{\kern0pt}env{\isacharparenright}{\kern0pt}{\isacharparenright}{\kern0pt}{\isachardoublequoteclose}\isanewline
\ \ \isacommand{have}\isamarkupfalse%
\ EQ{\isacharcolon}{\kern0pt}\ {\isachardoublequoteopen}weak{\isacharparenleft}{\kern0pt}length{\isacharparenleft}{\kern0pt}env{\isacharparenright}{\kern0pt}{\isacharcomma}{\kern0pt}{\isadigit{5}}{\isacharparenright}{\kern0pt}\ {\isacharequal}{\kern0pt}\ {\isadigit{5}}{\isacharhash}{\kern0pt}{\isacharplus}{\kern0pt}length{\isacharparenleft}{\kern0pt}env{\isacharparenright}{\kern0pt}\ {\isacharminus}{\kern0pt}\ {\isadigit{5}}{\isachardoublequoteclose}\isanewline
\ \ \ \ \isacommand{using}\isamarkupfalse%
\ weak{\isacharunderscore}{\kern0pt}equal\ length{\isacharunderscore}{\kern0pt}type{\isacharbrackleft}{\kern0pt}OF\ {\isacartoucheopen}env{\isasymin}list{\isacharparenleft}{\kern0pt}M{\isacharparenright}{\kern0pt}{\isacartoucheclose}{\isacharbrackright}{\kern0pt}\ \isacommand{by}\isamarkupfalse%
\ simp\isanewline
\ \ \isacommand{let}\isamarkupfalse%
\ {\isacharquery}{\kern0pt}tgt{\isacharequal}{\kern0pt}{\isachardoublequoteopen}{\isacharbrackleft}{\kern0pt}t{\isacharcomma}{\kern0pt}p{\isacharcomma}{\kern0pt}u{\isacharcomma}{\kern0pt}P{\isacharcomma}{\kern0pt}leq{\isacharcomma}{\kern0pt}o{\isacharcomma}{\kern0pt}pi{\isacharbrackright}{\kern0pt}{\isacharat}{\kern0pt}env{\isachardoublequoteclose}\isanewline
\ \ \isacommand{let}\isamarkupfalse%
\ {\isacharquery}{\kern0pt}src{\isacharequal}{\kern0pt}{\isachardoublequoteopen}{\isacharbrackleft}{\kern0pt}p{\isacharcomma}{\kern0pt}P{\isacharcomma}{\kern0pt}leq{\isacharcomma}{\kern0pt}o{\isacharcomma}{\kern0pt}t{\isacharbrackright}{\kern0pt}\ {\isacharat}{\kern0pt}\ env\ {\isacharat}{\kern0pt}\ {\isacharbrackleft}{\kern0pt}pi{\isacharcomma}{\kern0pt}u{\isacharbrackright}{\kern0pt}{\isachardoublequoteclose}\isanewline
\ \ \isacommand{have}\isamarkupfalse%
\ {\isachardoublequoteopen}nth{\isacharparenleft}{\kern0pt}{\isacharquery}{\kern0pt}f{\isacharbackquote}{\kern0pt}i{\isacharcomma}{\kern0pt}{\isacharbrackleft}{\kern0pt}t{\isacharcomma}{\kern0pt}p{\isacharcomma}{\kern0pt}u{\isacharcomma}{\kern0pt}P{\isacharcomma}{\kern0pt}leq{\isacharcomma}{\kern0pt}o{\isacharcomma}{\kern0pt}pi{\isacharbrackright}{\kern0pt}{\isacharat}{\kern0pt}env{\isacharparenright}{\kern0pt}\ {\isacharequal}{\kern0pt}\ nth{\isacharparenleft}{\kern0pt}i{\isacharcomma}{\kern0pt}{\isacharbrackleft}{\kern0pt}p{\isacharcomma}{\kern0pt}P{\isacharcomma}{\kern0pt}leq{\isacharcomma}{\kern0pt}o{\isacharcomma}{\kern0pt}t{\isacharbrackright}{\kern0pt}\ {\isacharat}{\kern0pt}\ env\ {\isacharat}{\kern0pt}\ {\isacharbrackleft}{\kern0pt}pi{\isacharcomma}{\kern0pt}u{\isacharbrackright}{\kern0pt}{\isacharparenright}{\kern0pt}{\isachardoublequoteclose}\ \isanewline
\ \ \ \ \isakeyword{if}\ {\isachardoublequoteopen}i\ {\isasymin}\ {\isacharparenleft}{\kern0pt}{\isadigit{5}}{\isacharhash}{\kern0pt}{\isacharplus}{\kern0pt}length{\isacharparenleft}{\kern0pt}env{\isacharparenright}{\kern0pt}{\isacharminus}{\kern0pt}{\isadigit{5}}{\isacharparenright}{\kern0pt}{\isachardoublequoteclose}\ \isakeyword{for}\ i\ \isanewline
\ \ \isacommand{proof}\isamarkupfalse%
\ {\isacharminus}{\kern0pt}\isanewline
\ \ \ \ \isacommand{from}\isamarkupfalse%
\ that\ \isanewline
\ \ \ \ \isacommand{have}\isamarkupfalse%
\ {\isadigit{2}}{\isacharcolon}{\kern0pt}\ {\isachardoublequoteopen}i\ {\isasymin}\ {\isadigit{5}}{\isacharhash}{\kern0pt}{\isacharplus}{\kern0pt}length{\isacharparenleft}{\kern0pt}env{\isacharparenright}{\kern0pt}{\isachardoublequoteclose}\ \ {\isachardoublequoteopen}i\ {\isasymnotin}\ {\isadigit{5}}{\isachardoublequoteclose}\ {\isachardoublequoteopen}i\ {\isasymin}\ nat{\isachardoublequoteclose}\ {\isachardoublequoteopen}i{\isacharhash}{\kern0pt}{\isacharminus}{\kern0pt}{\isadigit{5}}{\isasymin}nat{\isachardoublequoteclose}\ {\isachardoublequoteopen}i{\isacharhash}{\kern0pt}{\isacharplus}{\kern0pt}{\isadigit{2}}{\isasymin}nat{\isachardoublequoteclose}\isanewline
\ \ \ \ \ \ \isacommand{using}\isamarkupfalse%
\ in{\isacharunderscore}{\kern0pt}n{\isacharunderscore}{\kern0pt}in{\isacharunderscore}{\kern0pt}nat{\isacharbrackleft}{\kern0pt}OF\ {\isacartoucheopen}{\isadigit{5}}{\isacharhash}{\kern0pt}{\isacharplus}{\kern0pt}length{\isacharparenleft}{\kern0pt}env{\isacharparenright}{\kern0pt}{\isasymin}nat{\isacartoucheclose}{\isacharbrackright}{\kern0pt}\ \isacommand{by}\isamarkupfalse%
\ simp{\isacharunderscore}{\kern0pt}all\isanewline
\ \ \ \ \isacommand{then}\isamarkupfalse%
\ \isanewline
\ \ \ \ \isacommand{have}\isamarkupfalse%
\ {\isadigit{3}}{\isacharcolon}{\kern0pt}\ {\isachardoublequoteopen}{\isasymnot}\ i\ {\isacharless}{\kern0pt}\ {\isadigit{5}}{\isachardoublequoteclose}\ \isacommand{using}\isamarkupfalse%
\ ltD\ \isacommand{by}\isamarkupfalse%
\ force\isanewline
\ \ \ \ \isacommand{then}\isamarkupfalse%
\isanewline
\ \ \ \ \isacommand{have}\isamarkupfalse%
\ {\isachardoublequoteopen}{\isadigit{5}}\ {\isasymle}\ i{\isachardoublequoteclose}\ {\isachardoublequoteopen}{\isadigit{2}}\ {\isasymle}\ {\isadigit{5}}{\isachardoublequoteclose}\ \isanewline
\ \ \ \ \ \ \isacommand{using}\isamarkupfalse%
\ \ not{\isacharunderscore}{\kern0pt}lt{\isacharunderscore}{\kern0pt}iff{\isacharunderscore}{\kern0pt}le\ {\isacartoucheopen}i{\isasymin}nat{\isacartoucheclose}\ \isacommand{by}\isamarkupfalse%
\ simp{\isacharunderscore}{\kern0pt}all\isanewline
\ \ \ \ \isacommand{then}\isamarkupfalse%
\ \isacommand{have}\isamarkupfalse%
\ {\isachardoublequoteopen}{\isadigit{2}}\ {\isasymle}\ i{\isachardoublequoteclose}\ \isacommand{using}\isamarkupfalse%
\ le{\isacharunderscore}{\kern0pt}trans{\isacharbrackleft}{\kern0pt}OF\ {\isacartoucheopen}{\isadigit{2}}{\isasymle}{\isadigit{5}}{\isacartoucheclose}{\isacharbrackright}{\kern0pt}\ \isacommand{by}\isamarkupfalse%
\ simp\isanewline
\ \ \ \ \isacommand{from}\isamarkupfalse%
\ A\ {\isacartoucheopen}i\ {\isasymin}\ {\isadigit{5}}{\isacharhash}{\kern0pt}{\isacharplus}{\kern0pt}length{\isacharparenleft}{\kern0pt}env{\isacharparenright}{\kern0pt}{\isacartoucheclose}\ \isanewline
\ \ \ \ \isacommand{have}\isamarkupfalse%
\ {\isachardoublequoteopen}i\ {\isacharless}{\kern0pt}\ {\isadigit{5}}{\isacharhash}{\kern0pt}{\isacharplus}{\kern0pt}length{\isacharparenleft}{\kern0pt}env{\isacharparenright}{\kern0pt}{\isachardoublequoteclose}\ \isacommand{using}\isamarkupfalse%
\ ltI\ \isacommand{by}\isamarkupfalse%
\ simp\isanewline
\ \ \ \ \isacommand{with}\isamarkupfalse%
\ {\isacartoucheopen}i{\isasymin}nat{\isacartoucheclose}\ {\isacartoucheopen}{\isadigit{2}}{\isasymle}i{\isacartoucheclose}\ A\isanewline
\ \ \ \ \isacommand{have}\isamarkupfalse%
\ C{\isacharcolon}{\kern0pt}{\isachardoublequoteopen}i{\isacharhash}{\kern0pt}{\isacharplus}{\kern0pt}{\isadigit{2}}\ {\isacharless}{\kern0pt}\ {\isadigit{7}}{\isacharhash}{\kern0pt}{\isacharplus}{\kern0pt}length{\isacharparenleft}{\kern0pt}env{\isacharparenright}{\kern0pt}{\isachardoublequoteclose}\ \ \isacommand{by}\isamarkupfalse%
\ simp\isanewline
\ \ \ \ \isacommand{with}\isamarkupfalse%
\ that\ \isanewline
\ \ \ \ \isacommand{have}\isamarkupfalse%
\ B{\isacharcolon}{\kern0pt}\ {\isachardoublequoteopen}{\isacharquery}{\kern0pt}f{\isacharbackquote}{\kern0pt}i\ {\isacharequal}{\kern0pt}\ i{\isacharhash}{\kern0pt}{\isacharplus}{\kern0pt}{\isadigit{2}}{\isachardoublequoteclose}\ \isacommand{unfolding}\isamarkupfalse%
\ sep{\isacharunderscore}{\kern0pt}env{\isacharunderscore}{\kern0pt}def\ \isacommand{by}\isamarkupfalse%
\ simp\isanewline
\ \ \ \ \isacommand{from}\isamarkupfalse%
\ {\isadigit{3}}\ assms{\isacharparenleft}{\kern0pt}{\isadigit{1}}{\isacharparenright}{\kern0pt}\ {\isacartoucheopen}i{\isasymin}nat{\isacartoucheclose}\isanewline
\ \ \ \ \isacommand{have}\isamarkupfalse%
\ {\isachardoublequoteopen}{\isasymnot}\ i{\isacharhash}{\kern0pt}{\isacharplus}{\kern0pt}{\isadigit{2}}\ {\isacharless}{\kern0pt}\ {\isadigit{7}}{\isachardoublequoteclose}\ \isacommand{using}\isamarkupfalse%
\ not{\isacharunderscore}{\kern0pt}lt{\isacharunderscore}{\kern0pt}iff{\isacharunderscore}{\kern0pt}le\ add{\isacharunderscore}{\kern0pt}le{\isacharunderscore}{\kern0pt}mono\ \isacommand{by}\isamarkupfalse%
\ simp\isanewline
\ \ \ \ \isacommand{from}\isamarkupfalse%
\ {\isacartoucheopen}i\ {\isacharless}{\kern0pt}\ {\isadigit{5}}{\isacharhash}{\kern0pt}{\isacharplus}{\kern0pt}length{\isacharparenleft}{\kern0pt}env{\isacharparenright}{\kern0pt}{\isacartoucheclose}\ {\isadigit{3}}\ {\isacartoucheopen}i{\isasymin}nat{\isacartoucheclose}\isanewline
\ \ \ \ \isacommand{have}\isamarkupfalse%
\ {\isachardoublequoteopen}i{\isacharhash}{\kern0pt}{\isacharminus}{\kern0pt}{\isadigit{5}}\ {\isacharless}{\kern0pt}\ {\isadigit{5}}{\isacharhash}{\kern0pt}{\isacharplus}{\kern0pt}length{\isacharparenleft}{\kern0pt}env{\isacharparenright}{\kern0pt}\ {\isacharhash}{\kern0pt}{\isacharminus}{\kern0pt}\ {\isadigit{5}}{\isachardoublequoteclose}\ \isanewline
\ \ \ \ \ \ \isacommand{using}\isamarkupfalse%
\ diff{\isacharunderscore}{\kern0pt}mono{\isacharbrackleft}{\kern0pt}of\ i\ {\isachardoublequoteopen}{\isadigit{5}}{\isacharhash}{\kern0pt}{\isacharplus}{\kern0pt}length{\isacharparenleft}{\kern0pt}env{\isacharparenright}{\kern0pt}{\isachardoublequoteclose}\ {\isadigit{5}}{\isacharcomma}{\kern0pt}OF\ {\isacharunderscore}{\kern0pt}\ {\isacharunderscore}{\kern0pt}\ {\isacharunderscore}{\kern0pt}\ {\isacartoucheopen}i\ {\isacharless}{\kern0pt}\ {\isadigit{5}}{\isacharhash}{\kern0pt}{\isacharplus}{\kern0pt}length{\isacharparenleft}{\kern0pt}env{\isacharparenright}{\kern0pt}{\isacartoucheclose}{\isacharbrackright}{\kern0pt}\ \isanewline
\ \ \ \ \ \ \ \ not{\isacharunderscore}{\kern0pt}lt{\isacharunderscore}{\kern0pt}iff{\isacharunderscore}{\kern0pt}le{\isacharbrackleft}{\kern0pt}THEN\ iffD{\isadigit{1}}{\isacharbrackright}{\kern0pt}\ \isacommand{by}\isamarkupfalse%
\ force\isanewline
\ \ \ \ \isacommand{with}\isamarkupfalse%
\ assms{\isacharparenleft}{\kern0pt}{\isadigit{2}}{\isacharparenright}{\kern0pt}\isanewline
\ \ \ \ \isacommand{have}\isamarkupfalse%
\ {\isachardoublequoteopen}i{\isacharhash}{\kern0pt}{\isacharminus}{\kern0pt}{\isadigit{5}}\ {\isacharless}{\kern0pt}\ length{\isacharparenleft}{\kern0pt}env{\isacharparenright}{\kern0pt}{\isachardoublequoteclose}\ \isacommand{using}\isamarkupfalse%
\ diff{\isacharunderscore}{\kern0pt}add{\isacharunderscore}{\kern0pt}inverse\ length{\isacharunderscore}{\kern0pt}type\ \isacommand{by}\isamarkupfalse%
\ simp\isanewline
\ \ \ \ \isacommand{have}\isamarkupfalse%
\ {\isachardoublequoteopen}nth{\isacharparenleft}{\kern0pt}i{\isacharcomma}{\kern0pt}{\isacharquery}{\kern0pt}src{\isacharparenright}{\kern0pt}\ {\isacharequal}{\kern0pt}nth{\isacharparenleft}{\kern0pt}i{\isacharhash}{\kern0pt}{\isacharminus}{\kern0pt}{\isadigit{5}}{\isacharcomma}{\kern0pt}env{\isacharat}{\kern0pt}{\isacharbrackleft}{\kern0pt}pi{\isacharcomma}{\kern0pt}u{\isacharbrackright}{\kern0pt}{\isacharparenright}{\kern0pt}{\isachardoublequoteclose}\isanewline
\ \ \ \ \ \ \isacommand{using}\isamarkupfalse%
\ nth{\isacharunderscore}{\kern0pt}append{\isacharbrackleft}{\kern0pt}OF\ A{\isacharparenleft}{\kern0pt}{\isadigit{2}}{\isacharparenright}{\kern0pt}\ {\isacartoucheopen}i{\isasymin}nat{\isacartoucheclose}{\isacharbrackright}{\kern0pt}\ {\isadigit{3}}\ \isacommand{by}\isamarkupfalse%
\ simp\isanewline
\ \ \ \ \isacommand{also}\isamarkupfalse%
\ \isanewline
\ \ \ \ \isacommand{have}\isamarkupfalse%
\ {\isachardoublequoteopen}{\isachardot}{\kern0pt}{\isachardot}{\kern0pt}{\isachardot}{\kern0pt}\ {\isacharequal}{\kern0pt}\ nth{\isacharparenleft}{\kern0pt}i{\isacharhash}{\kern0pt}{\isacharminus}{\kern0pt}{\isadigit{5}}{\isacharcomma}{\kern0pt}\ env{\isacharparenright}{\kern0pt}{\isachardoublequoteclose}\ \isanewline
\ \ \ \ \ \ \isacommand{using}\isamarkupfalse%
\ nth{\isacharunderscore}{\kern0pt}append{\isacharbrackleft}{\kern0pt}OF\ {\isacartoucheopen}env\ {\isasymin}list{\isacharparenleft}{\kern0pt}M{\isacharparenright}{\kern0pt}{\isacartoucheclose}\ {\isacartoucheopen}i{\isacharhash}{\kern0pt}{\isacharminus}{\kern0pt}{\isadigit{5}}{\isasymin}nat{\isacartoucheclose}{\isacharbrackright}{\kern0pt}\ {\isacartoucheopen}i{\isacharhash}{\kern0pt}{\isacharminus}{\kern0pt}{\isadigit{5}}\ {\isacharless}{\kern0pt}\ length{\isacharparenleft}{\kern0pt}env{\isacharparenright}{\kern0pt}{\isacartoucheclose}\ \isacommand{by}\isamarkupfalse%
\ simp\isanewline
\ \ \ \ \isacommand{also}\isamarkupfalse%
\ \isanewline
\ \ \ \ \isacommand{have}\isamarkupfalse%
\ {\isachardoublequoteopen}{\isachardot}{\kern0pt}{\isachardot}{\kern0pt}{\isachardot}{\kern0pt}\ {\isacharequal}{\kern0pt}\ nth{\isacharparenleft}{\kern0pt}i{\isacharhash}{\kern0pt}{\isacharplus}{\kern0pt}{\isadigit{2}}{\isacharcomma}{\kern0pt}\ {\isacharquery}{\kern0pt}tgt{\isacharparenright}{\kern0pt}{\isachardoublequoteclose}\isanewline
\ \ \ \ \ \ \isacommand{using}\isamarkupfalse%
\ nth{\isacharunderscore}{\kern0pt}append{\isacharbrackleft}{\kern0pt}OF\ assms{\isacharparenleft}{\kern0pt}{\isadigit{1}}{\isacharparenright}{\kern0pt}\ {\isacartoucheopen}i{\isacharhash}{\kern0pt}{\isacharplus}{\kern0pt}{\isadigit{2}}{\isasymin}nat{\isacartoucheclose}{\isacharbrackright}{\kern0pt}\ {\isacartoucheopen}{\isasymnot}\ i{\isacharhash}{\kern0pt}{\isacharplus}{\kern0pt}{\isadigit{2}}\ {\isacharless}{\kern0pt}{\isadigit{7}}{\isacartoucheclose}\ \isacommand{by}\isamarkupfalse%
\ simp\isanewline
\ \ \ \ \isacommand{ultimately}\isamarkupfalse%
\ \isanewline
\ \ \ \ \isacommand{have}\isamarkupfalse%
\ {\isachardoublequoteopen}nth{\isacharparenleft}{\kern0pt}i{\isacharcomma}{\kern0pt}{\isacharquery}{\kern0pt}src{\isacharparenright}{\kern0pt}\ {\isacharequal}{\kern0pt}\ nth{\isacharparenleft}{\kern0pt}{\isacharquery}{\kern0pt}f{\isacharbackquote}{\kern0pt}i{\isacharcomma}{\kern0pt}{\isacharquery}{\kern0pt}tgt{\isacharparenright}{\kern0pt}{\isachardoublequoteclose}\isanewline
\ \ \ \ \ \ \isacommand{using}\isamarkupfalse%
\ B\ \isacommand{by}\isamarkupfalse%
\ simp\ \isanewline
\ \ \ \ \isacommand{then}\isamarkupfalse%
\ \isacommand{show}\isamarkupfalse%
\ {\isacharquery}{\kern0pt}thesis\ \isacommand{using}\isamarkupfalse%
\ that\ \isacommand{by}\isamarkupfalse%
\ simp\isanewline
\ \ \isacommand{qed}\isamarkupfalse%
\isanewline
\ \ \isacommand{then}\isamarkupfalse%
\ \isacommand{show}\isamarkupfalse%
\ {\isacharquery}{\kern0pt}thesis\ \isacommand{using}\isamarkupfalse%
\ EQ\ \isacommand{by}\isamarkupfalse%
\ force\isanewline
\isacommand{qed}\isamarkupfalse%
%
\endisatagproof
{\isafoldproof}%
%
\isadelimproof
\isanewline
%
\endisadelimproof
\isanewline
\isacommand{lemma}\isamarkupfalse%
\ sep{\isacharunderscore}{\kern0pt}env{\isacharunderscore}{\kern0pt}type\ {\isacharcolon}{\kern0pt}\isanewline
\ \ \isakeyword{assumes}\ {\isachardoublequoteopen}n\ {\isasymin}\ nat{\isachardoublequoteclose}\isanewline
\ \ \isakeyword{shows}\ {\isachardoublequoteopen}sep{\isacharunderscore}{\kern0pt}env{\isacharparenleft}{\kern0pt}n{\isacharparenright}{\kern0pt}\ {\isacharcolon}{\kern0pt}\ {\isacharparenleft}{\kern0pt}{\isadigit{5}}{\isacharhash}{\kern0pt}{\isacharplus}{\kern0pt}n{\isacharparenright}{\kern0pt}{\isacharminus}{\kern0pt}{\isadigit{5}}\ {\isasymrightarrow}\ {\isacharparenleft}{\kern0pt}{\isadigit{7}}{\isacharhash}{\kern0pt}{\isacharplus}{\kern0pt}n{\isacharparenright}{\kern0pt}{\isacharminus}{\kern0pt}{\isadigit{7}}{\isachardoublequoteclose}\isanewline
%
\isadelimproof
%
\endisadelimproof
%
\isatagproof
\isacommand{proof}\isamarkupfalse%
\ {\isacharminus}{\kern0pt}\isanewline
\ \ \isacommand{let}\isamarkupfalse%
\ {\isacharquery}{\kern0pt}h{\isacharequal}{\kern0pt}{\isachardoublequoteopen}sep{\isacharunderscore}{\kern0pt}env{\isacharparenleft}{\kern0pt}n{\isacharparenright}{\kern0pt}{\isachardoublequoteclose}\isanewline
\ \ \isacommand{from}\isamarkupfalse%
\ {\isacartoucheopen}n{\isasymin}nat{\isacartoucheclose}\ \isanewline
\ \ \isacommand{have}\isamarkupfalse%
\ {\isachardoublequoteopen}{\isacharparenleft}{\kern0pt}{\isadigit{5}}{\isacharhash}{\kern0pt}{\isacharplus}{\kern0pt}n{\isacharparenright}{\kern0pt}{\isacharhash}{\kern0pt}{\isacharplus}{\kern0pt}{\isadigit{2}}\ {\isacharequal}{\kern0pt}\ {\isadigit{7}}{\isacharhash}{\kern0pt}{\isacharplus}{\kern0pt}n{\isachardoublequoteclose}\ {\isachardoublequoteopen}{\isadigit{7}}{\isacharhash}{\kern0pt}{\isacharplus}{\kern0pt}n{\isasymin}nat{\isachardoublequoteclose}\ {\isachardoublequoteopen}{\isadigit{5}}{\isacharhash}{\kern0pt}{\isacharplus}{\kern0pt}n{\isasymin}nat{\isachardoublequoteclose}\ \isacommand{by}\isamarkupfalse%
\ simp{\isacharunderscore}{\kern0pt}all\isanewline
\ \ \isacommand{have}\isamarkupfalse%
\isanewline
\ \ \ \ D{\isacharcolon}{\kern0pt}\ {\isachardoublequoteopen}sep{\isacharunderscore}{\kern0pt}env{\isacharparenleft}{\kern0pt}n{\isacharparenright}{\kern0pt}{\isacharbackquote}{\kern0pt}x\ {\isasymin}\ {\isacharparenleft}{\kern0pt}{\isadigit{7}}{\isacharhash}{\kern0pt}{\isacharplus}{\kern0pt}n{\isacharparenright}{\kern0pt}{\isacharminus}{\kern0pt}{\isadigit{7}}{\isachardoublequoteclose}\ \isakeyword{if}\ {\isachardoublequoteopen}x\ {\isasymin}\ {\isacharparenleft}{\kern0pt}{\isadigit{5}}{\isacharhash}{\kern0pt}{\isacharplus}{\kern0pt}n{\isacharparenright}{\kern0pt}{\isacharminus}{\kern0pt}{\isadigit{5}}{\isachardoublequoteclose}\ \isakeyword{for}\ x\isanewline
\ \ \isacommand{proof}\isamarkupfalse%
\ {\isacharminus}{\kern0pt}\isanewline
\ \ \ \ \isacommand{from}\isamarkupfalse%
\ {\isacartoucheopen}x{\isasymin}{\isadigit{5}}{\isacharhash}{\kern0pt}{\isacharplus}{\kern0pt}n{\isacharminus}{\kern0pt}{\isadigit{5}}{\isacartoucheclose}\isanewline
\ \ \ \ \isacommand{have}\isamarkupfalse%
\ {\isachardoublequoteopen}{\isacharquery}{\kern0pt}h{\isacharbackquote}{\kern0pt}x\ {\isacharequal}{\kern0pt}\ x{\isacharhash}{\kern0pt}{\isacharplus}{\kern0pt}{\isadigit{2}}{\isachardoublequoteclose}\ {\isachardoublequoteopen}x{\isacharless}{\kern0pt}{\isadigit{5}}{\isacharhash}{\kern0pt}{\isacharplus}{\kern0pt}n{\isachardoublequoteclose}\ {\isachardoublequoteopen}x{\isasymin}nat{\isachardoublequoteclose}\isanewline
\ \ \ \ \ \ \isacommand{unfolding}\isamarkupfalse%
\ sep{\isacharunderscore}{\kern0pt}env{\isacharunderscore}{\kern0pt}def\ \isacommand{using}\isamarkupfalse%
\ ltI\ in{\isacharunderscore}{\kern0pt}n{\isacharunderscore}{\kern0pt}in{\isacharunderscore}{\kern0pt}nat{\isacharbrackleft}{\kern0pt}OF\ {\isacartoucheopen}{\isadigit{5}}{\isacharhash}{\kern0pt}{\isacharplus}{\kern0pt}n{\isasymin}nat{\isacartoucheclose}{\isacharbrackright}{\kern0pt}\ \isacommand{by}\isamarkupfalse%
\ simp{\isacharunderscore}{\kern0pt}all\isanewline
\ \ \ \ \isacommand{then}\isamarkupfalse%
\ \isanewline
\ \ \ \ \isacommand{have}\isamarkupfalse%
\ {\isachardoublequoteopen}x{\isacharhash}{\kern0pt}{\isacharplus}{\kern0pt}{\isadigit{2}}\ {\isacharless}{\kern0pt}\ {\isadigit{7}}{\isacharhash}{\kern0pt}{\isacharplus}{\kern0pt}n{\isachardoublequoteclose}\ \isacommand{by}\isamarkupfalse%
\ simp\isanewline
\ \ \ \ \isacommand{then}\isamarkupfalse%
\ \isanewline
\ \ \ \ \isacommand{have}\isamarkupfalse%
\ {\isachardoublequoteopen}x{\isacharhash}{\kern0pt}{\isacharplus}{\kern0pt}{\isadigit{2}}\ {\isasymin}\ {\isadigit{7}}{\isacharhash}{\kern0pt}{\isacharplus}{\kern0pt}n{\isachardoublequoteclose}\ \isacommand{using}\isamarkupfalse%
\ ltD\ \isacommand{by}\isamarkupfalse%
\ simp\isanewline
\ \ \ \ \isacommand{from}\isamarkupfalse%
\ {\isacartoucheopen}x{\isasymin}{\isadigit{5}}{\isacharhash}{\kern0pt}{\isacharplus}{\kern0pt}n{\isacharminus}{\kern0pt}{\isadigit{5}}{\isacartoucheclose}\isanewline
\ \ \ \ \isacommand{have}\isamarkupfalse%
\ {\isachardoublequoteopen}x{\isasymnotin}{\isadigit{5}}{\isachardoublequoteclose}\ \isacommand{by}\isamarkupfalse%
\ simp\ \isanewline
\ \ \ \ \isacommand{then}\isamarkupfalse%
\ \isacommand{have}\isamarkupfalse%
\ {\isachardoublequoteopen}{\isasymnot}x{\isacharless}{\kern0pt}{\isadigit{5}}{\isachardoublequoteclose}\ \isacommand{using}\isamarkupfalse%
\ ltD\ \isacommand{by}\isamarkupfalse%
\ blast\isanewline
\ \ \ \ \isacommand{then}\isamarkupfalse%
\ \isacommand{have}\isamarkupfalse%
\ {\isachardoublequoteopen}{\isadigit{5}}{\isasymle}x{\isachardoublequoteclose}\ \isacommand{using}\isamarkupfalse%
\ not{\isacharunderscore}{\kern0pt}lt{\isacharunderscore}{\kern0pt}iff{\isacharunderscore}{\kern0pt}le\ {\isacartoucheopen}x{\isasymin}nat{\isacartoucheclose}\ \isacommand{by}\isamarkupfalse%
\ simp\isanewline
\ \ \ \ \isacommand{then}\isamarkupfalse%
\ \isacommand{have}\isamarkupfalse%
\ {\isachardoublequoteopen}{\isadigit{7}}{\isasymle}x{\isacharhash}{\kern0pt}{\isacharplus}{\kern0pt}{\isadigit{2}}{\isachardoublequoteclose}\ \isacommand{using}\isamarkupfalse%
\ add{\isacharunderscore}{\kern0pt}le{\isacharunderscore}{\kern0pt}mono\ {\isacartoucheopen}x{\isasymin}nat{\isacartoucheclose}\ \isacommand{by}\isamarkupfalse%
\ simp\isanewline
\ \ \ \ \isacommand{then}\isamarkupfalse%
\ \isacommand{have}\isamarkupfalse%
\ {\isachardoublequoteopen}{\isasymnot}x{\isacharhash}{\kern0pt}{\isacharplus}{\kern0pt}{\isadigit{2}}{\isacharless}{\kern0pt}{\isadigit{7}}{\isachardoublequoteclose}\ \isacommand{using}\isamarkupfalse%
\ not{\isacharunderscore}{\kern0pt}lt{\isacharunderscore}{\kern0pt}iff{\isacharunderscore}{\kern0pt}le\ {\isacartoucheopen}x{\isasymin}nat{\isacartoucheclose}\ \isacommand{by}\isamarkupfalse%
\ simp\isanewline
\ \ \ \ \isacommand{then}\isamarkupfalse%
\ \isacommand{have}\isamarkupfalse%
\ {\isachardoublequoteopen}x{\isacharhash}{\kern0pt}{\isacharplus}{\kern0pt}{\isadigit{2}}\ {\isasymnotin}\ {\isadigit{7}}{\isachardoublequoteclose}\ \isacommand{using}\isamarkupfalse%
\ ltI\ {\isacartoucheopen}x{\isasymin}nat{\isacartoucheclose}\ \isacommand{by}\isamarkupfalse%
\ force\isanewline
\ \ \ \ \isacommand{with}\isamarkupfalse%
\ {\isacartoucheopen}x{\isacharhash}{\kern0pt}{\isacharplus}{\kern0pt}{\isadigit{2}}\ {\isasymin}\ {\isadigit{7}}{\isacharhash}{\kern0pt}{\isacharplus}{\kern0pt}n{\isacartoucheclose}\ \isacommand{show}\isamarkupfalse%
\ {\isacharquery}{\kern0pt}thesis\ \isacommand{using}\isamarkupfalse%
\ \ {\isacartoucheopen}{\isacharquery}{\kern0pt}h{\isacharbackquote}{\kern0pt}x\ {\isacharequal}{\kern0pt}\ x{\isacharhash}{\kern0pt}{\isacharplus}{\kern0pt}{\isadigit{2}}{\isacartoucheclose}\ DiffI\ \isacommand{by}\isamarkupfalse%
\ simp\isanewline
\ \ \isacommand{qed}\isamarkupfalse%
\isanewline
\ \ \isacommand{then}\isamarkupfalse%
\ \isacommand{show}\isamarkupfalse%
\ {\isacharquery}{\kern0pt}thesis\ \isacommand{unfolding}\isamarkupfalse%
\ sep{\isacharunderscore}{\kern0pt}env{\isacharunderscore}{\kern0pt}def\ \isacommand{using}\isamarkupfalse%
\ lam{\isacharunderscore}{\kern0pt}type\ \isacommand{by}\isamarkupfalse%
\ simp\isanewline
\isacommand{qed}\isamarkupfalse%
%
\endisatagproof
{\isafoldproof}%
%
\isadelimproof
\isanewline
%
\endisadelimproof
\isanewline
\isacommand{lemma}\isamarkupfalse%
\ sep{\isacharunderscore}{\kern0pt}var{\isacharunderscore}{\kern0pt}fin{\isacharunderscore}{\kern0pt}type\ {\isacharcolon}{\kern0pt}\isanewline
\ \ \isakeyword{assumes}\ {\isachardoublequoteopen}n\ {\isasymin}\ nat{\isachardoublequoteclose}\isanewline
\ \ \isakeyword{shows}\ {\isachardoublequoteopen}sep{\isacharunderscore}{\kern0pt}var{\isacharparenleft}{\kern0pt}n{\isacharparenright}{\kern0pt}\ {\isacharcolon}{\kern0pt}\ {\isadigit{7}}{\isacharhash}{\kern0pt}{\isacharplus}{\kern0pt}n\ \ {\isacharminus}{\kern0pt}{\isacharbar}{\kern0pt}{\isacharbar}{\kern0pt}{\isachargreater}{\kern0pt}\ {\isadigit{7}}{\isacharhash}{\kern0pt}{\isacharplus}{\kern0pt}n{\isachardoublequoteclose}\isanewline
%
\isadelimproof
\ \ %
\endisadelimproof
%
\isatagproof
\isacommand{unfolding}\isamarkupfalse%
\ sep{\isacharunderscore}{\kern0pt}var{\isacharunderscore}{\kern0pt}def\ \isanewline
\ \ \isacommand{using}\isamarkupfalse%
\ consI\ ltD\ emptyI\ \isacommand{by}\isamarkupfalse%
\ force%
\endisatagproof
{\isafoldproof}%
%
\isadelimproof
\isanewline
%
\endisadelimproof
\isanewline
\isacommand{lemma}\isamarkupfalse%
\ sep{\isacharunderscore}{\kern0pt}var{\isacharunderscore}{\kern0pt}domain\ {\isacharcolon}{\kern0pt}\isanewline
\ \ \isakeyword{assumes}\ {\isachardoublequoteopen}n\ {\isasymin}\ nat{\isachardoublequoteclose}\isanewline
\ \ \isakeyword{shows}\ {\isachardoublequoteopen}domain{\isacharparenleft}{\kern0pt}sep{\isacharunderscore}{\kern0pt}var{\isacharparenleft}{\kern0pt}n{\isacharparenright}{\kern0pt}{\isacharparenright}{\kern0pt}\ {\isacharequal}{\kern0pt}\ \ {\isadigit{7}}{\isacharhash}{\kern0pt}{\isacharplus}{\kern0pt}n\ {\isacharminus}{\kern0pt}\ weak{\isacharparenleft}{\kern0pt}n{\isacharcomma}{\kern0pt}{\isadigit{5}}{\isacharparenright}{\kern0pt}{\isachardoublequoteclose}\isanewline
%
\isadelimproof
%
\endisadelimproof
%
\isatagproof
\isacommand{proof}\isamarkupfalse%
\ {\isacharminus}{\kern0pt}\ \isanewline
\ \ \isacommand{let}\isamarkupfalse%
\ {\isacharquery}{\kern0pt}A{\isacharequal}{\kern0pt}{\isachardoublequoteopen}weak{\isacharparenleft}{\kern0pt}n{\isacharcomma}{\kern0pt}{\isadigit{5}}{\isacharparenright}{\kern0pt}{\isachardoublequoteclose}\isanewline
\ \ \isacommand{have}\isamarkupfalse%
\ A{\isacharcolon}{\kern0pt}{\isachardoublequoteopen}domain{\isacharparenleft}{\kern0pt}sep{\isacharunderscore}{\kern0pt}var{\isacharparenleft}{\kern0pt}n{\isacharparenright}{\kern0pt}{\isacharparenright}{\kern0pt}\ {\isasymsubseteq}\ {\isacharparenleft}{\kern0pt}{\isadigit{7}}{\isacharhash}{\kern0pt}{\isacharplus}{\kern0pt}n{\isacharparenright}{\kern0pt}{\isachardoublequoteclose}\ \isanewline
\ \ \ \ \isacommand{unfolding}\isamarkupfalse%
\ sep{\isacharunderscore}{\kern0pt}var{\isacharunderscore}{\kern0pt}def\ \isanewline
\ \ \ \ \isacommand{by}\isamarkupfalse%
{\isacharparenleft}{\kern0pt}auto\ simp\ add{\isacharcolon}{\kern0pt}\ le{\isacharunderscore}{\kern0pt}natE{\isacharparenright}{\kern0pt}\isanewline
\ \ \isacommand{have}\isamarkupfalse%
\ C{\isacharcolon}{\kern0pt}\ {\isachardoublequoteopen}x{\isacharequal}{\kern0pt}{\isadigit{5}}{\isacharhash}{\kern0pt}{\isacharplus}{\kern0pt}n\ {\isasymor}\ x{\isacharequal}{\kern0pt}{\isadigit{6}}{\isacharhash}{\kern0pt}{\isacharplus}{\kern0pt}n\ {\isasymor}\ x\ {\isasymle}\ {\isadigit{4}}{\isachardoublequoteclose}\ \isakeyword{if}\ {\isachardoublequoteopen}x{\isasymin}domain{\isacharparenleft}{\kern0pt}sep{\isacharunderscore}{\kern0pt}var{\isacharparenleft}{\kern0pt}n{\isacharparenright}{\kern0pt}{\isacharparenright}{\kern0pt}{\isachardoublequoteclose}\ \isakeyword{for}\ x\isanewline
\ \ \ \ \isacommand{using}\isamarkupfalse%
\ that\ \isacommand{unfolding}\isamarkupfalse%
\ sep{\isacharunderscore}{\kern0pt}var{\isacharunderscore}{\kern0pt}def\ \isacommand{by}\isamarkupfalse%
\ auto\isanewline
\ \ \isacommand{have}\isamarkupfalse%
\ D\ {\isacharcolon}{\kern0pt}\ {\isachardoublequoteopen}x{\isacharless}{\kern0pt}n{\isacharhash}{\kern0pt}{\isacharplus}{\kern0pt}{\isadigit{7}}{\isachardoublequoteclose}\ \isakeyword{if}\ {\isachardoublequoteopen}x{\isasymin}{\isadigit{7}}{\isacharhash}{\kern0pt}{\isacharplus}{\kern0pt}n{\isachardoublequoteclose}\ \isakeyword{for}\ x\isanewline
\ \ \ \ \isacommand{using}\isamarkupfalse%
\ that\ {\isacartoucheopen}n{\isasymin}nat{\isacartoucheclose}\ ltI\ \isacommand{by}\isamarkupfalse%
\ simp\isanewline
\ \ \isacommand{have}\isamarkupfalse%
\ {\isachardoublequoteopen}{\isasymnot}\ {\isadigit{5}}{\isacharhash}{\kern0pt}{\isacharplus}{\kern0pt}n\ {\isacharless}{\kern0pt}\ {\isadigit{5}}{\isacharhash}{\kern0pt}{\isacharplus}{\kern0pt}n{\isachardoublequoteclose}\ \isacommand{using}\isamarkupfalse%
\ {\isacartoucheopen}n{\isasymin}nat{\isacartoucheclose}\ \ lt{\isacharunderscore}{\kern0pt}irrefl{\isacharbrackleft}{\kern0pt}of\ {\isacharunderscore}{\kern0pt}\ False{\isacharbrackright}{\kern0pt}\ \isacommand{by}\isamarkupfalse%
\ force\isanewline
\ \ \isacommand{have}\isamarkupfalse%
\ {\isachardoublequoteopen}{\isasymnot}\ {\isadigit{6}}{\isacharhash}{\kern0pt}{\isacharplus}{\kern0pt}n\ {\isacharless}{\kern0pt}\ {\isadigit{5}}{\isacharhash}{\kern0pt}{\isacharplus}{\kern0pt}n{\isachardoublequoteclose}\ \isacommand{using}\isamarkupfalse%
\ {\isacartoucheopen}n{\isasymin}nat{\isacartoucheclose}\ \isacommand{by}\isamarkupfalse%
\ force\isanewline
\ \ \isacommand{have}\isamarkupfalse%
\ R{\isacharcolon}{\kern0pt}\ {\isachardoublequoteopen}x\ {\isacharless}{\kern0pt}\ {\isadigit{5}}{\isacharhash}{\kern0pt}{\isacharplus}{\kern0pt}n{\isachardoublequoteclose}\ \isakeyword{if}\ {\isachardoublequoteopen}x{\isasymin}{\isacharquery}{\kern0pt}A{\isachardoublequoteclose}\ \isakeyword{for}\ x\isanewline
\ \ \isacommand{proof}\isamarkupfalse%
\ {\isacharminus}{\kern0pt}\isanewline
\ \ \ \ \isacommand{from}\isamarkupfalse%
\ that\isanewline
\ \ \ \ \isacommand{obtain}\isamarkupfalse%
\ i\ \isakeyword{where}\isanewline
\ \ \ \ \ \ {\isachardoublequoteopen}i{\isacharless}{\kern0pt}n{\isachardoublequoteclose}\ {\isachardoublequoteopen}x{\isacharequal}{\kern0pt}{\isadigit{5}}{\isacharhash}{\kern0pt}{\isacharplus}{\kern0pt}i{\isachardoublequoteclose}\ \isanewline
\ \ \ \ \ \ \isacommand{unfolding}\isamarkupfalse%
\ weak{\isacharunderscore}{\kern0pt}def\isanewline
\ \ \ \ \ \ \isacommand{using}\isamarkupfalse%
\ ltI\ {\isacartoucheopen}n{\isasymin}nat{\isacartoucheclose}\ RepFun{\isacharunderscore}{\kern0pt}iff\ \isacommand{by}\isamarkupfalse%
\ force\isanewline
\ \ \ \ \isacommand{with}\isamarkupfalse%
\ {\isacartoucheopen}n{\isasymin}nat{\isacartoucheclose}\isanewline
\ \ \ \ \isacommand{have}\isamarkupfalse%
\ {\isachardoublequoteopen}{\isadigit{5}}{\isacharhash}{\kern0pt}{\isacharplus}{\kern0pt}i\ {\isacharless}{\kern0pt}\ {\isadigit{5}}{\isacharhash}{\kern0pt}{\isacharplus}{\kern0pt}n{\isachardoublequoteclose}\ \isacommand{using}\isamarkupfalse%
\ add{\isacharunderscore}{\kern0pt}lt{\isacharunderscore}{\kern0pt}mono{\isadigit{2}}\ \isacommand{by}\isamarkupfalse%
\ simp\isanewline
\ \ \ \ \isacommand{with}\isamarkupfalse%
\ {\isacartoucheopen}x{\isacharequal}{\kern0pt}{\isadigit{5}}{\isacharhash}{\kern0pt}{\isacharplus}{\kern0pt}i{\isacartoucheclose}\ \isanewline
\ \ \ \ \isacommand{show}\isamarkupfalse%
\ {\isachardoublequoteopen}x\ {\isacharless}{\kern0pt}\ {\isadigit{5}}{\isacharhash}{\kern0pt}{\isacharplus}{\kern0pt}n{\isachardoublequoteclose}\ \isacommand{by}\isamarkupfalse%
\ simp\isanewline
\ \ \isacommand{qed}\isamarkupfalse%
\isanewline
\ \ \isacommand{then}\isamarkupfalse%
\ \isanewline
\ \ \isacommand{have}\isamarkupfalse%
\ {\isadigit{1}}{\isacharcolon}{\kern0pt}{\isachardoublequoteopen}x{\isasymnotin}{\isacharquery}{\kern0pt}A{\isachardoublequoteclose}\ \isakeyword{if}\ {\isachardoublequoteopen}{\isasymnot}x\ {\isacharless}{\kern0pt}{\isadigit{5}}{\isacharhash}{\kern0pt}{\isacharplus}{\kern0pt}n{\isachardoublequoteclose}\ \isakeyword{for}\ x\ \isacommand{using}\isamarkupfalse%
\ that\ \isacommand{by}\isamarkupfalse%
\ blast\isanewline
\ \ \isacommand{have}\isamarkupfalse%
\ {\isachardoublequoteopen}{\isadigit{5}}{\isacharhash}{\kern0pt}{\isacharplus}{\kern0pt}n\ {\isasymnotin}\ {\isacharquery}{\kern0pt}A{\isachardoublequoteclose}\ {\isachardoublequoteopen}{\isadigit{6}}{\isacharhash}{\kern0pt}{\isacharplus}{\kern0pt}n{\isasymnotin}{\isacharquery}{\kern0pt}A{\isachardoublequoteclose}\isanewline
\ \ \isacommand{proof}\isamarkupfalse%
\ {\isacharminus}{\kern0pt}\isanewline
\ \ \ \ \isacommand{show}\isamarkupfalse%
\ {\isachardoublequoteopen}{\isadigit{5}}{\isacharhash}{\kern0pt}{\isacharplus}{\kern0pt}n\ {\isasymnotin}\ {\isacharquery}{\kern0pt}A{\isachardoublequoteclose}\ \isacommand{using}\isamarkupfalse%
\ {\isadigit{1}}\ {\isacartoucheopen}{\isasymnot}{\isadigit{5}}{\isacharhash}{\kern0pt}{\isacharplus}{\kern0pt}n{\isacharless}{\kern0pt}{\isadigit{5}}{\isacharhash}{\kern0pt}{\isacharplus}{\kern0pt}n{\isacartoucheclose}\ \isacommand{by}\isamarkupfalse%
\ blast\ \ \ \ \isanewline
\ \ \ \ \isacommand{with}\isamarkupfalse%
\ {\isadigit{1}}\ \isacommand{show}\isamarkupfalse%
\ {\isachardoublequoteopen}{\isadigit{6}}{\isacharhash}{\kern0pt}{\isacharplus}{\kern0pt}n\ {\isasymnotin}\ {\isacharquery}{\kern0pt}A{\isachardoublequoteclose}\ \isacommand{using}\isamarkupfalse%
\ \ {\isacartoucheopen}{\isasymnot}{\isadigit{6}}{\isacharhash}{\kern0pt}{\isacharplus}{\kern0pt}n{\isacharless}{\kern0pt}{\isadigit{5}}{\isacharhash}{\kern0pt}{\isacharplus}{\kern0pt}n{\isacartoucheclose}\ \isacommand{by}\isamarkupfalse%
\ blast\isanewline
\ \ \isacommand{qed}\isamarkupfalse%
\isanewline
\ \ \isacommand{then}\isamarkupfalse%
\ \isanewline
\ \ \isacommand{have}\isamarkupfalse%
\ E{\isacharcolon}{\kern0pt}{\isachardoublequoteopen}x{\isasymnotin}{\isacharquery}{\kern0pt}A{\isachardoublequoteclose}\ \isakeyword{if}\ {\isachardoublequoteopen}x{\isasymin}domain{\isacharparenleft}{\kern0pt}sep{\isacharunderscore}{\kern0pt}var{\isacharparenleft}{\kern0pt}n{\isacharparenright}{\kern0pt}{\isacharparenright}{\kern0pt}{\isachardoublequoteclose}\ \isakeyword{for}\ x\ \isanewline
\ \ \ \ \isacommand{unfolding}\isamarkupfalse%
\ weak{\isacharunderscore}{\kern0pt}def\isanewline
\ \ \ \ \isacommand{using}\isamarkupfalse%
\ C\ that\ \isacommand{by}\isamarkupfalse%
\ force\isanewline
\ \ \isacommand{then}\isamarkupfalse%
\ \isanewline
\ \ \isacommand{have}\isamarkupfalse%
\ F{\isacharcolon}{\kern0pt}\ {\isachardoublequoteopen}domain{\isacharparenleft}{\kern0pt}sep{\isacharunderscore}{\kern0pt}var{\isacharparenleft}{\kern0pt}n{\isacharparenright}{\kern0pt}{\isacharparenright}{\kern0pt}\ {\isasymsubseteq}\ {\isadigit{7}}{\isacharhash}{\kern0pt}{\isacharplus}{\kern0pt}n\ {\isacharminus}{\kern0pt}\ {\isacharquery}{\kern0pt}A{\isachardoublequoteclose}\ \isacommand{using}\isamarkupfalse%
\ A\ \isacommand{by}\isamarkupfalse%
\ auto\isanewline
\ \ \isacommand{from}\isamarkupfalse%
\ assms\isanewline
\ \ \isacommand{have}\isamarkupfalse%
\ {\isachardoublequoteopen}x{\isacharless}{\kern0pt}{\isadigit{7}}\ {\isasymor}\ x{\isasymin}weak{\isacharparenleft}{\kern0pt}n{\isacharcomma}{\kern0pt}{\isadigit{7}}{\isacharparenright}{\kern0pt}{\isachardoublequoteclose}\ \isakeyword{if}\ {\isachardoublequoteopen}x{\isasymin}{\isadigit{7}}{\isacharhash}{\kern0pt}{\isacharplus}{\kern0pt}n{\isachardoublequoteclose}\ \isakeyword{for}\ x\isanewline
\ \ \ \ \isacommand{using}\isamarkupfalse%
\ in{\isacharunderscore}{\kern0pt}add{\isacharunderscore}{\kern0pt}del{\isacharbrackleft}{\kern0pt}OF\ {\isacartoucheopen}x{\isasymin}{\isadigit{7}}{\isacharhash}{\kern0pt}{\isacharplus}{\kern0pt}n{\isacartoucheclose}{\isacharbrackright}{\kern0pt}\ \isacommand{by}\isamarkupfalse%
\ simp\isanewline
\ \ \isacommand{moreover}\isamarkupfalse%
\isanewline
\ \ \isacommand{{\isacharbraceleft}{\kern0pt}}\isamarkupfalse%
\isanewline
\ \ \ \ \isacommand{fix}\isamarkupfalse%
\ x\isanewline
\ \ \ \ \isacommand{assume}\isamarkupfalse%
\ asm{\isacharcolon}{\kern0pt}{\isachardoublequoteopen}x{\isasymin}{\isadigit{7}}{\isacharhash}{\kern0pt}{\isacharplus}{\kern0pt}n{\isachardoublequoteclose}\ \ {\isachardoublequoteopen}x{\isasymnotin}{\isacharquery}{\kern0pt}A{\isachardoublequoteclose}\ \ {\isachardoublequoteopen}x{\isasymin}weak{\isacharparenleft}{\kern0pt}n{\isacharcomma}{\kern0pt}{\isadigit{7}}{\isacharparenright}{\kern0pt}{\isachardoublequoteclose}\isanewline
\ \ \ \ \isacommand{then}\isamarkupfalse%
\ \isanewline
\ \ \ \ \isacommand{have}\isamarkupfalse%
\ {\isachardoublequoteopen}x{\isasymin}domain{\isacharparenleft}{\kern0pt}sep{\isacharunderscore}{\kern0pt}var{\isacharparenleft}{\kern0pt}n{\isacharparenright}{\kern0pt}{\isacharparenright}{\kern0pt}{\isachardoublequoteclose}\ \isanewline
\ \ \ \ \isacommand{proof}\isamarkupfalse%
\ {\isacharminus}{\kern0pt}\isanewline
\ \ \ \ \ \ \isacommand{from}\isamarkupfalse%
\ {\isacartoucheopen}n{\isasymin}nat{\isacartoucheclose}\isanewline
\ \ \ \ \ \ \isacommand{have}\isamarkupfalse%
\ {\isachardoublequoteopen}weak{\isacharparenleft}{\kern0pt}n{\isacharcomma}{\kern0pt}{\isadigit{7}}{\isacharparenright}{\kern0pt}{\isacharminus}{\kern0pt}weak{\isacharparenleft}{\kern0pt}n{\isacharcomma}{\kern0pt}{\isadigit{5}}{\isacharparenright}{\kern0pt}{\isasymsubseteq}{\isacharbraceleft}{\kern0pt}n{\isacharhash}{\kern0pt}{\isacharplus}{\kern0pt}{\isadigit{5}}{\isacharcomma}{\kern0pt}n{\isacharhash}{\kern0pt}{\isacharplus}{\kern0pt}{\isadigit{6}}{\isacharbraceright}{\kern0pt}{\isachardoublequoteclose}\ \isanewline
\ \ \ \ \ \ \ \ \isacommand{using}\isamarkupfalse%
\ weakening{\isacharunderscore}{\kern0pt}diff\ \isacommand{by}\isamarkupfalse%
\ simp\isanewline
\ \ \ \ \ \ \isacommand{with}\isamarkupfalse%
\ \ {\isacartoucheopen}x{\isasymnotin}{\isacharquery}{\kern0pt}A{\isacartoucheclose}\ asm\isanewline
\ \ \ \ \ \ \isacommand{have}\isamarkupfalse%
\ {\isachardoublequoteopen}x{\isasymin}{\isacharbraceleft}{\kern0pt}n{\isacharhash}{\kern0pt}{\isacharplus}{\kern0pt}{\isadigit{5}}{\isacharcomma}{\kern0pt}n{\isacharhash}{\kern0pt}{\isacharplus}{\kern0pt}{\isadigit{6}}{\isacharbraceright}{\kern0pt}{\isachardoublequoteclose}\ \isacommand{using}\isamarkupfalse%
\ \ subsetD\ DiffI\ \isacommand{by}\isamarkupfalse%
\ blast\isanewline
\ \ \ \ \ \ \isacommand{then}\isamarkupfalse%
\ \isanewline
\ \ \ \ \ \ \isacommand{show}\isamarkupfalse%
\ {\isacharquery}{\kern0pt}thesis\ \isacommand{unfolding}\isamarkupfalse%
\ sep{\isacharunderscore}{\kern0pt}var{\isacharunderscore}{\kern0pt}def\ \isacommand{by}\isamarkupfalse%
\ simp\isanewline
\ \ \ \ \isacommand{qed}\isamarkupfalse%
\isanewline
\ \ \isacommand{{\isacharbraceright}{\kern0pt}}\isamarkupfalse%
\isanewline
\ \ \isacommand{moreover}\isamarkupfalse%
\isanewline
\ \ \isacommand{{\isacharbraceleft}{\kern0pt}}\isamarkupfalse%
\isanewline
\ \ \ \ \isacommand{fix}\isamarkupfalse%
\ x\isanewline
\ \ \ \ \isacommand{assume}\isamarkupfalse%
\ asm{\isacharcolon}{\kern0pt}{\isachardoublequoteopen}x{\isasymin}{\isadigit{7}}{\isacharhash}{\kern0pt}{\isacharplus}{\kern0pt}n{\isachardoublequoteclose}\ \ {\isachardoublequoteopen}x{\isasymnotin}{\isacharquery}{\kern0pt}A{\isachardoublequoteclose}\ {\isachardoublequoteopen}x{\isacharless}{\kern0pt}{\isadigit{7}}{\isachardoublequoteclose}\isanewline
\ \ \ \ \isacommand{then}\isamarkupfalse%
\ \isacommand{have}\isamarkupfalse%
\ {\isachardoublequoteopen}x{\isasymin}domain{\isacharparenleft}{\kern0pt}sep{\isacharunderscore}{\kern0pt}var{\isacharparenleft}{\kern0pt}n{\isacharparenright}{\kern0pt}{\isacharparenright}{\kern0pt}{\isachardoublequoteclose}\isanewline
\ \ \ \ \isacommand{proof}\isamarkupfalse%
\ {\isacharparenleft}{\kern0pt}cases\ {\isachardoublequoteopen}{\isadigit{2}}\ {\isasymle}\ n{\isachardoublequoteclose}{\isacharparenright}{\kern0pt}\isanewline
\ \ \ \ \ \ \isacommand{case}\isamarkupfalse%
\ True\isanewline
\ \ \ \ \ \ \isacommand{moreover}\isamarkupfalse%
\isanewline
\ \ \ \ \ \ \isacommand{have}\isamarkupfalse%
\ {\isachardoublequoteopen}{\isadigit{0}}{\isacharless}{\kern0pt}n{\isachardoublequoteclose}\ \isacommand{using}\isamarkupfalse%
\ \ leD{\isacharbrackleft}{\kern0pt}OF\ {\isacartoucheopen}n{\isasymin}nat{\isacartoucheclose}\ {\isacartoucheopen}{\isadigit{2}}{\isasymle}n{\isacartoucheclose}{\isacharbrackright}{\kern0pt}\ lt{\isacharunderscore}{\kern0pt}imp{\isacharunderscore}{\kern0pt}{\isadigit{0}}{\isacharunderscore}{\kern0pt}lt\ \isacommand{by}\isamarkupfalse%
\ auto\isanewline
\ \ \ \ \ \ \isacommand{ultimately}\isamarkupfalse%
\isanewline
\ \ \ \ \ \ \isacommand{have}\isamarkupfalse%
\ {\isachardoublequoteopen}x{\isacharless}{\kern0pt}{\isadigit{5}}{\isachardoublequoteclose}\isanewline
\ \ \ \ \ \ \ \ \isacommand{using}\isamarkupfalse%
\ {\isacartoucheopen}x{\isacharless}{\kern0pt}{\isadigit{7}}{\isacartoucheclose}\ {\isacartoucheopen}x{\isasymnotin}{\isacharquery}{\kern0pt}A{\isacartoucheclose}\ {\isacartoucheopen}n{\isasymin}nat{\isacartoucheclose}\ in{\isacharunderscore}{\kern0pt}n{\isacharunderscore}{\kern0pt}in{\isacharunderscore}{\kern0pt}nat\isanewline
\ \ \ \ \ \ \ \ \isacommand{unfolding}\isamarkupfalse%
\ weak{\isacharunderscore}{\kern0pt}def\isanewline
\ \ \ \ \ \ \ \ \isacommand{by}\isamarkupfalse%
\ {\isacharparenleft}{\kern0pt}clarsimp\ simp\ add{\isacharcolon}{\kern0pt}not{\isacharunderscore}{\kern0pt}lt{\isacharunderscore}{\kern0pt}iff{\isacharunderscore}{\kern0pt}le{\isacharcomma}{\kern0pt}\ auto\ simp\ add{\isacharcolon}{\kern0pt}lt{\isacharunderscore}{\kern0pt}def{\isacharparenright}{\kern0pt}\isanewline
\ \ \ \ \ \ \isacommand{then}\isamarkupfalse%
\isanewline
\ \ \ \ \ \ \isacommand{show}\isamarkupfalse%
\ {\isacharquery}{\kern0pt}thesis\ \isacommand{unfolding}\isamarkupfalse%
\ sep{\isacharunderscore}{\kern0pt}var{\isacharunderscore}{\kern0pt}def\ \isanewline
\ \ \ \ \ \ \ \ \isacommand{by}\isamarkupfalse%
\ {\isacharparenleft}{\kern0pt}clarsimp\ simp\ add{\isacharcolon}{\kern0pt}not{\isacharunderscore}{\kern0pt}lt{\isacharunderscore}{\kern0pt}iff{\isacharunderscore}{\kern0pt}le{\isacharcomma}{\kern0pt}\ auto\ simp\ add{\isacharcolon}{\kern0pt}lt{\isacharunderscore}{\kern0pt}def{\isacharparenright}{\kern0pt}\isanewline
\ \ \ \ \isacommand{next}\isamarkupfalse%
\isanewline
\ \ \ \ \ \ \isacommand{case}\isamarkupfalse%
\ False\ \isanewline
\ \ \ \ \ \ \isacommand{then}\isamarkupfalse%
\ \isanewline
\ \ \ \ \ \ \isacommand{show}\isamarkupfalse%
\ {\isacharquery}{\kern0pt}thesis\ \isanewline
\ \ \ \ \ \ \isacommand{proof}\isamarkupfalse%
\ {\isacharparenleft}{\kern0pt}cases\ {\isachardoublequoteopen}n{\isacharequal}{\kern0pt}{\isadigit{0}}{\isachardoublequoteclose}{\isacharparenright}{\kern0pt}\isanewline
\ \ \ \ \ \ \ \ \isacommand{case}\isamarkupfalse%
\ True\isanewline
\ \ \ \ \ \ \ \ \isacommand{then}\isamarkupfalse%
\ \isacommand{show}\isamarkupfalse%
\ {\isacharquery}{\kern0pt}thesis\ \isanewline
\ \ \ \ \ \ \ \ \ \ \isacommand{unfolding}\isamarkupfalse%
\ sep{\isacharunderscore}{\kern0pt}var{\isacharunderscore}{\kern0pt}def\ \isacommand{using}\isamarkupfalse%
\ ltD\ asm\ {\isacartoucheopen}n{\isasymin}nat{\isacartoucheclose}\ \isacommand{by}\isamarkupfalse%
\ auto\isanewline
\ \ \ \ \ \ \isacommand{next}\isamarkupfalse%
\isanewline
\ \ \ \ \ \ \ \ \isacommand{case}\isamarkupfalse%
\ False\isanewline
\ \ \ \ \ \ \ \ \isacommand{then}\isamarkupfalse%
\ \isanewline
\ \ \ \ \ \ \ \ \isacommand{have}\isamarkupfalse%
\ {\isachardoublequoteopen}n\ {\isacharless}{\kern0pt}\ {\isadigit{2}}{\isachardoublequoteclose}\ \isacommand{using}\isamarkupfalse%
\ \ {\isacartoucheopen}n{\isasymin}nat{\isacartoucheclose}\ not{\isacharunderscore}{\kern0pt}lt{\isacharunderscore}{\kern0pt}iff{\isacharunderscore}{\kern0pt}le\ {\isacartoucheopen}{\isasymnot}\ {\isadigit{2}}\ {\isasymle}\ n{\isacartoucheclose}\ \ \isacommand{by}\isamarkupfalse%
\ force\isanewline
\ \ \ \ \ \ \ \ \isacommand{then}\isamarkupfalse%
\ \isanewline
\ \ \ \ \ \ \ \ \isacommand{have}\isamarkupfalse%
\ {\isachardoublequoteopen}{\isasymnot}\ n\ {\isacharless}{\kern0pt}{\isadigit{1}}{\isachardoublequoteclose}\ \isacommand{using}\isamarkupfalse%
\ {\isacartoucheopen}n{\isasymnoteq}{\isadigit{0}}{\isacartoucheclose}\ \isacommand{by}\isamarkupfalse%
\ simp\isanewline
\ \ \ \ \ \ \ \ \isacommand{then}\isamarkupfalse%
\ \isanewline
\ \ \ \ \ \ \ \ \isacommand{have}\isamarkupfalse%
\ {\isachardoublequoteopen}n{\isacharequal}{\kern0pt}{\isadigit{1}}{\isachardoublequoteclose}\ \isacommand{using}\isamarkupfalse%
\ not{\isacharunderscore}{\kern0pt}lt{\isacharunderscore}{\kern0pt}iff{\isacharunderscore}{\kern0pt}le\ {\isacartoucheopen}n{\isacharless}{\kern0pt}{\isadigit{2}}{\isacartoucheclose}\ le{\isacharunderscore}{\kern0pt}iff\ \isacommand{by}\isamarkupfalse%
\ auto\isanewline
\ \ \ \ \ \ \ \ \isacommand{then}\isamarkupfalse%
\ \isacommand{show}\isamarkupfalse%
\ {\isacharquery}{\kern0pt}thesis\ \isanewline
\ \ \ \ \ \ \ \ \ \ \isacommand{using}\isamarkupfalse%
\ {\isacartoucheopen}x{\isasymnotin}{\isacharquery}{\kern0pt}A{\isacartoucheclose}\ \isanewline
\ \ \ \ \ \ \ \ \ \ \isacommand{unfolding}\isamarkupfalse%
\ weak{\isacharunderscore}{\kern0pt}def\ sep{\isacharunderscore}{\kern0pt}var{\isacharunderscore}{\kern0pt}def\ \isanewline
\ \ \ \ \ \ \ \ \ \ \isacommand{using}\isamarkupfalse%
\ ltD\ asm\ {\isacartoucheopen}n{\isasymin}nat{\isacartoucheclose}\ \isacommand{by}\isamarkupfalse%
\ force\isanewline
\ \ \ \ \ \ \isacommand{qed}\isamarkupfalse%
\isanewline
\ \ \ \ \isacommand{qed}\isamarkupfalse%
\isanewline
\ \ \isacommand{{\isacharbraceright}{\kern0pt}}\isamarkupfalse%
\isanewline
\ \ \isacommand{ultimately}\isamarkupfalse%
\isanewline
\ \ \isacommand{have}\isamarkupfalse%
\ {\isachardoublequoteopen}w{\isasymin}domain{\isacharparenleft}{\kern0pt}sep{\isacharunderscore}{\kern0pt}var{\isacharparenleft}{\kern0pt}n{\isacharparenright}{\kern0pt}{\isacharparenright}{\kern0pt}{\isachardoublequoteclose}\ \isakeyword{if}\ {\isachardoublequoteopen}w{\isasymin}\ {\isadigit{7}}{\isacharhash}{\kern0pt}{\isacharplus}{\kern0pt}n\ {\isacharminus}{\kern0pt}\ {\isacharquery}{\kern0pt}A{\isachardoublequoteclose}\ \isakeyword{for}\ w\isanewline
\ \ \ \ \isacommand{using}\isamarkupfalse%
\ that\ \isacommand{by}\isamarkupfalse%
\ blast\isanewline
\ \ \isacommand{then}\isamarkupfalse%
\isanewline
\ \ \isacommand{have}\isamarkupfalse%
\ {\isachardoublequoteopen}{\isadigit{7}}{\isacharhash}{\kern0pt}{\isacharplus}{\kern0pt}n\ {\isacharminus}{\kern0pt}\ {\isacharquery}{\kern0pt}A\ {\isasymsubseteq}\ domain{\isacharparenleft}{\kern0pt}sep{\isacharunderscore}{\kern0pt}var{\isacharparenleft}{\kern0pt}n{\isacharparenright}{\kern0pt}{\isacharparenright}{\kern0pt}{\isachardoublequoteclose}\ \isacommand{by}\isamarkupfalse%
\ blast\isanewline
\ \ \isacommand{with}\isamarkupfalse%
\ F\ \isanewline
\ \ \isacommand{show}\isamarkupfalse%
\ {\isacharquery}{\kern0pt}thesis\ \isacommand{by}\isamarkupfalse%
\ auto\isanewline
\isacommand{qed}\isamarkupfalse%
%
\endisatagproof
{\isafoldproof}%
%
\isadelimproof
\isanewline
%
\endisadelimproof
\isanewline
\isacommand{lemma}\isamarkupfalse%
\ sep{\isacharunderscore}{\kern0pt}var{\isacharunderscore}{\kern0pt}type\ {\isacharcolon}{\kern0pt}\isanewline
\ \ \isakeyword{assumes}\ {\isachardoublequoteopen}n\ {\isasymin}\ nat{\isachardoublequoteclose}\isanewline
\ \ \isakeyword{shows}\ {\isachardoublequoteopen}sep{\isacharunderscore}{\kern0pt}var{\isacharparenleft}{\kern0pt}n{\isacharparenright}{\kern0pt}\ {\isacharcolon}{\kern0pt}\ {\isacharparenleft}{\kern0pt}{\isadigit{7}}{\isacharhash}{\kern0pt}{\isacharplus}{\kern0pt}n{\isacharparenright}{\kern0pt}{\isacharminus}{\kern0pt}weak{\isacharparenleft}{\kern0pt}n{\isacharcomma}{\kern0pt}{\isadigit{5}}{\isacharparenright}{\kern0pt}\ {\isasymrightarrow}\ {\isadigit{7}}{\isacharhash}{\kern0pt}{\isacharplus}{\kern0pt}n{\isachardoublequoteclose}\isanewline
%
\isadelimproof
\ \ %
\endisadelimproof
%
\isatagproof
\isacommand{using}\isamarkupfalse%
\ FiniteFun{\isacharunderscore}{\kern0pt}is{\isacharunderscore}{\kern0pt}fun{\isacharbrackleft}{\kern0pt}OF\ sep{\isacharunderscore}{\kern0pt}var{\isacharunderscore}{\kern0pt}fin{\isacharunderscore}{\kern0pt}type{\isacharbrackleft}{\kern0pt}OF\ {\isacartoucheopen}n{\isasymin}nat{\isacartoucheclose}{\isacharbrackright}{\kern0pt}{\isacharbrackright}{\kern0pt}\isanewline
\ \ \ \ sep{\isacharunderscore}{\kern0pt}var{\isacharunderscore}{\kern0pt}domain{\isacharbrackleft}{\kern0pt}OF\ {\isacartoucheopen}n{\isasymin}nat{\isacartoucheclose}{\isacharbrackright}{\kern0pt}\ \isacommand{by}\isamarkupfalse%
\ simp%
\endisatagproof
{\isafoldproof}%
%
\isadelimproof
\isanewline
%
\endisadelimproof
\isanewline
\isacommand{lemma}\isamarkupfalse%
\ sep{\isacharunderscore}{\kern0pt}var{\isacharunderscore}{\kern0pt}action\ {\isacharcolon}{\kern0pt}\isanewline
\ \ \isakeyword{assumes}\ \isanewline
\ \ \ \ {\isachardoublequoteopen}{\isacharbrackleft}{\kern0pt}t{\isacharcomma}{\kern0pt}p{\isacharcomma}{\kern0pt}u{\isacharcomma}{\kern0pt}P{\isacharcomma}{\kern0pt}leq{\isacharcomma}{\kern0pt}o{\isacharcomma}{\kern0pt}pi{\isacharbrackright}{\kern0pt}\ {\isasymin}\ list{\isacharparenleft}{\kern0pt}M{\isacharparenright}{\kern0pt}{\isachardoublequoteclose}\isanewline
\ \ \ \ {\isachardoublequoteopen}env\ {\isasymin}\ list{\isacharparenleft}{\kern0pt}M{\isacharparenright}{\kern0pt}{\isachardoublequoteclose}\isanewline
\ \ \isakeyword{shows}\ {\isachardoublequoteopen}{\isasymforall}\ i\ {\isachardot}{\kern0pt}\ i\ {\isasymin}\ {\isacharparenleft}{\kern0pt}{\isadigit{7}}{\isacharhash}{\kern0pt}{\isacharplus}{\kern0pt}length{\isacharparenleft}{\kern0pt}env{\isacharparenright}{\kern0pt}{\isacharparenright}{\kern0pt}\ {\isacharminus}{\kern0pt}\ weak{\isacharparenleft}{\kern0pt}length{\isacharparenleft}{\kern0pt}env{\isacharparenright}{\kern0pt}{\isacharcomma}{\kern0pt}{\isadigit{5}}{\isacharparenright}{\kern0pt}\ {\isasymlongrightarrow}\ \isanewline
\ \ \ \ nth{\isacharparenleft}{\kern0pt}sep{\isacharunderscore}{\kern0pt}var{\isacharparenleft}{\kern0pt}length{\isacharparenleft}{\kern0pt}env{\isacharparenright}{\kern0pt}{\isacharparenright}{\kern0pt}{\isacharbackquote}{\kern0pt}i{\isacharcomma}{\kern0pt}{\isacharbrackleft}{\kern0pt}t{\isacharcomma}{\kern0pt}p{\isacharcomma}{\kern0pt}u{\isacharcomma}{\kern0pt}P{\isacharcomma}{\kern0pt}leq{\isacharcomma}{\kern0pt}o{\isacharcomma}{\kern0pt}pi{\isacharbrackright}{\kern0pt}{\isacharat}{\kern0pt}env{\isacharparenright}{\kern0pt}\ {\isacharequal}{\kern0pt}\ nth{\isacharparenleft}{\kern0pt}i{\isacharcomma}{\kern0pt}{\isacharbrackleft}{\kern0pt}p{\isacharcomma}{\kern0pt}P{\isacharcomma}{\kern0pt}leq{\isacharcomma}{\kern0pt}o{\isacharcomma}{\kern0pt}t{\isacharbrackright}{\kern0pt}\ {\isacharat}{\kern0pt}\ env\ {\isacharat}{\kern0pt}\ {\isacharbrackleft}{\kern0pt}pi{\isacharcomma}{\kern0pt}u{\isacharbrackright}{\kern0pt}{\isacharparenright}{\kern0pt}{\isachardoublequoteclose}\isanewline
%
\isadelimproof
\ \ %
\endisadelimproof
%
\isatagproof
\isacommand{using}\isamarkupfalse%
\ assms\isanewline
\isacommand{proof}\isamarkupfalse%
\ {\isacharparenleft}{\kern0pt}subst\ sep{\isacharunderscore}{\kern0pt}var{\isacharunderscore}{\kern0pt}domain{\isacharbrackleft}{\kern0pt}OF\ length{\isacharunderscore}{\kern0pt}type{\isacharbrackleft}{\kern0pt}OF\ {\isacartoucheopen}env{\isasymin}list{\isacharparenleft}{\kern0pt}M{\isacharparenright}{\kern0pt}{\isacartoucheclose}{\isacharbrackright}{\kern0pt}{\isacharcomma}{\kern0pt}symmetric{\isacharbrackright}{\kern0pt}{\isacharcomma}{\kern0pt}auto{\isacharparenright}{\kern0pt}\isanewline
\ \ \isacommand{fix}\isamarkupfalse%
\ i\ y\isanewline
\ \ \isacommand{assume}\isamarkupfalse%
\ {\isachardoublequoteopen}{\isasymlangle}i{\isacharcomma}{\kern0pt}\ y{\isasymrangle}\ {\isasymin}\ sep{\isacharunderscore}{\kern0pt}var{\isacharparenleft}{\kern0pt}length{\isacharparenleft}{\kern0pt}env{\isacharparenright}{\kern0pt}{\isacharparenright}{\kern0pt}{\isachardoublequoteclose}\isanewline
\ \ \isacommand{with}\isamarkupfalse%
\ assms\isanewline
\ \ \isacommand{show}\isamarkupfalse%
\ {\isachardoublequoteopen}nth{\isacharparenleft}{\kern0pt}sep{\isacharunderscore}{\kern0pt}var{\isacharparenleft}{\kern0pt}length{\isacharparenleft}{\kern0pt}env{\isacharparenright}{\kern0pt}{\isacharparenright}{\kern0pt}\ {\isacharbackquote}{\kern0pt}\ i{\isacharcomma}{\kern0pt}\isanewline
\ \ \ \ \ \ \ \ \ \ \ \ \ \ \ Cons{\isacharparenleft}{\kern0pt}t{\isacharcomma}{\kern0pt}\ Cons{\isacharparenleft}{\kern0pt}p{\isacharcomma}{\kern0pt}\ Cons{\isacharparenleft}{\kern0pt}u{\isacharcomma}{\kern0pt}\ Cons{\isacharparenleft}{\kern0pt}P{\isacharcomma}{\kern0pt}\ Cons{\isacharparenleft}{\kern0pt}leq{\isacharcomma}{\kern0pt}\ Cons{\isacharparenleft}{\kern0pt}o{\isacharcomma}{\kern0pt}\ Cons{\isacharparenleft}{\kern0pt}pi{\isacharcomma}{\kern0pt}\ env{\isacharparenright}{\kern0pt}{\isacharparenright}{\kern0pt}{\isacharparenright}{\kern0pt}{\isacharparenright}{\kern0pt}{\isacharparenright}{\kern0pt}{\isacharparenright}{\kern0pt}{\isacharparenright}{\kern0pt}{\isacharparenright}{\kern0pt}\ {\isacharequal}{\kern0pt}\isanewline
\ \ \ \ \ \ \ \ \ \ \ nth{\isacharparenleft}{\kern0pt}i{\isacharcomma}{\kern0pt}\ Cons{\isacharparenleft}{\kern0pt}p{\isacharcomma}{\kern0pt}\ Cons{\isacharparenleft}{\kern0pt}P{\isacharcomma}{\kern0pt}\ Cons{\isacharparenleft}{\kern0pt}leq{\isacharcomma}{\kern0pt}\ Cons{\isacharparenleft}{\kern0pt}o{\isacharcomma}{\kern0pt}\ Cons{\isacharparenleft}{\kern0pt}t{\isacharcomma}{\kern0pt}\ env\ {\isacharat}{\kern0pt}\ {\isacharbrackleft}{\kern0pt}pi{\isacharcomma}{\kern0pt}\ u{\isacharbrackright}{\kern0pt}{\isacharparenright}{\kern0pt}{\isacharparenright}{\kern0pt}{\isacharparenright}{\kern0pt}{\isacharparenright}{\kern0pt}{\isacharparenright}{\kern0pt}{\isacharparenright}{\kern0pt}{\isachardoublequoteclose}\ \ \isanewline
\ \ \ \ \isacommand{using}\isamarkupfalse%
\ apply{\isacharunderscore}{\kern0pt}fun{\isacharbrackleft}{\kern0pt}OF\ sep{\isacharunderscore}{\kern0pt}var{\isacharunderscore}{\kern0pt}type{\isacharbrackright}{\kern0pt}\ assms\isanewline
\ \ \ \ \ \ \isacommand{unfolding}\isamarkupfalse%
\ sep{\isacharunderscore}{\kern0pt}var{\isacharunderscore}{\kern0pt}def\isanewline
\ \ \ \ \ \ \isacommand{using}\isamarkupfalse%
\ nth{\isacharunderscore}{\kern0pt}concat{\isadigit{2}}{\isacharbrackleft}{\kern0pt}OF\ {\isacartoucheopen}env{\isasymin}list{\isacharparenleft}{\kern0pt}M{\isacharparenright}{\kern0pt}{\isacartoucheclose}{\isacharbrackright}{\kern0pt}\ \ nth{\isacharunderscore}{\kern0pt}concat{\isadigit{3}}{\isacharbrackleft}{\kern0pt}OF\ {\isacartoucheopen}env{\isasymin}list{\isacharparenleft}{\kern0pt}M{\isacharparenright}{\kern0pt}{\isacartoucheclose}{\isacharcomma}{\kern0pt}symmetric{\isacharbrackright}{\kern0pt}\isanewline
\ \ \ \ \ \ \isacommand{by}\isamarkupfalse%
\ force\isanewline
\ \ \isacommand{qed}\isamarkupfalse%
%
\endisatagproof
{\isafoldproof}%
%
\isadelimproof
\isanewline
%
\endisadelimproof
\isanewline
\isacommand{definition}\isamarkupfalse%
\isanewline
\ \ rensep\ {\isacharcolon}{\kern0pt}{\isacharcolon}{\kern0pt}\ {\isachardoublequoteopen}i\ {\isasymRightarrow}\ i{\isachardoublequoteclose}\ \isakeyword{where}\isanewline
\ \ {\isachardoublequoteopen}rensep{\isacharparenleft}{\kern0pt}n{\isacharparenright}{\kern0pt}\ {\isasymequiv}\ union{\isacharunderscore}{\kern0pt}fun{\isacharparenleft}{\kern0pt}sep{\isacharunderscore}{\kern0pt}var{\isacharparenleft}{\kern0pt}n{\isacharparenright}{\kern0pt}{\isacharcomma}{\kern0pt}sep{\isacharunderscore}{\kern0pt}env{\isacharparenleft}{\kern0pt}n{\isacharparenright}{\kern0pt}{\isacharcomma}{\kern0pt}{\isadigit{7}}{\isacharhash}{\kern0pt}{\isacharplus}{\kern0pt}n{\isacharminus}{\kern0pt}weak{\isacharparenleft}{\kern0pt}n{\isacharcomma}{\kern0pt}{\isadigit{5}}{\isacharparenright}{\kern0pt}{\isacharcomma}{\kern0pt}weak{\isacharparenleft}{\kern0pt}n{\isacharcomma}{\kern0pt}{\isadigit{5}}{\isacharparenright}{\kern0pt}{\isacharparenright}{\kern0pt}{\isachardoublequoteclose}\isanewline
\isanewline
\isacommand{lemma}\isamarkupfalse%
\ rensep{\isacharunderscore}{\kern0pt}aux\ {\isacharcolon}{\kern0pt}\isanewline
\ \ \isakeyword{assumes}\ {\isachardoublequoteopen}n{\isasymin}nat{\isachardoublequoteclose}\isanewline
\ \ \isakeyword{shows}\ {\isachardoublequoteopen}{\isacharparenleft}{\kern0pt}{\isadigit{7}}{\isacharhash}{\kern0pt}{\isacharplus}{\kern0pt}n{\isacharminus}{\kern0pt}weak{\isacharparenleft}{\kern0pt}n{\isacharcomma}{\kern0pt}{\isadigit{5}}{\isacharparenright}{\kern0pt}{\isacharparenright}{\kern0pt}\ {\isasymunion}\ weak{\isacharparenleft}{\kern0pt}n{\isacharcomma}{\kern0pt}{\isadigit{5}}{\isacharparenright}{\kern0pt}\ {\isacharequal}{\kern0pt}\ {\isadigit{7}}{\isacharhash}{\kern0pt}{\isacharplus}{\kern0pt}n{\isachardoublequoteclose}\ {\isachardoublequoteopen}{\isadigit{7}}{\isacharhash}{\kern0pt}{\isacharplus}{\kern0pt}n\ {\isasymunion}\ {\isacharparenleft}{\kern0pt}\ {\isadigit{7}}\ {\isacharhash}{\kern0pt}{\isacharplus}{\kern0pt}\ n\ {\isacharminus}{\kern0pt}\ {\isadigit{7}}{\isacharparenright}{\kern0pt}\ {\isacharequal}{\kern0pt}\ {\isadigit{7}}{\isacharhash}{\kern0pt}{\isacharplus}{\kern0pt}n{\isachardoublequoteclose}\isanewline
%
\isadelimproof
%
\endisadelimproof
%
\isatagproof
\isacommand{proof}\isamarkupfalse%
\ {\isacharminus}{\kern0pt}\isanewline
\ \ \isacommand{from}\isamarkupfalse%
\ {\isacartoucheopen}n{\isasymin}nat{\isacartoucheclose}\isanewline
\ \ \isacommand{have}\isamarkupfalse%
\ {\isachardoublequoteopen}weak{\isacharparenleft}{\kern0pt}n{\isacharcomma}{\kern0pt}{\isadigit{5}}{\isacharparenright}{\kern0pt}\ {\isacharequal}{\kern0pt}\ n{\isacharhash}{\kern0pt}{\isacharplus}{\kern0pt}{\isadigit{5}}{\isacharminus}{\kern0pt}{\isadigit{5}}{\isachardoublequoteclose}\isanewline
\ \ \ \ \isacommand{using}\isamarkupfalse%
\ weak{\isacharunderscore}{\kern0pt}equal\ \isacommand{by}\isamarkupfalse%
\ simp\isanewline
\ \ \isacommand{with}\isamarkupfalse%
\ \ {\isacartoucheopen}n{\isasymin}nat{\isacartoucheclose}\isanewline
\ \ \isacommand{show}\isamarkupfalse%
\ {\isachardoublequoteopen}{\isacharparenleft}{\kern0pt}{\isadigit{7}}{\isacharhash}{\kern0pt}{\isacharplus}{\kern0pt}n{\isacharminus}{\kern0pt}weak{\isacharparenleft}{\kern0pt}n{\isacharcomma}{\kern0pt}{\isadigit{5}}{\isacharparenright}{\kern0pt}{\isacharparenright}{\kern0pt}\ {\isasymunion}\ weak{\isacharparenleft}{\kern0pt}n{\isacharcomma}{\kern0pt}{\isadigit{5}}{\isacharparenright}{\kern0pt}\ {\isacharequal}{\kern0pt}\ {\isadigit{7}}{\isacharhash}{\kern0pt}{\isacharplus}{\kern0pt}n{\isachardoublequoteclose}\ {\isachardoublequoteopen}{\isadigit{7}}{\isacharhash}{\kern0pt}{\isacharplus}{\kern0pt}n\ {\isasymunion}\ {\isacharparenleft}{\kern0pt}\ {\isadigit{7}}\ {\isacharhash}{\kern0pt}{\isacharplus}{\kern0pt}\ n\ {\isacharminus}{\kern0pt}\ {\isadigit{7}}{\isacharparenright}{\kern0pt}\ {\isacharequal}{\kern0pt}\ {\isadigit{7}}{\isacharhash}{\kern0pt}{\isacharplus}{\kern0pt}n{\isachardoublequoteclose}\isanewline
\ \ \ \ \isacommand{using}\isamarkupfalse%
\ Diff{\isacharunderscore}{\kern0pt}partition\ le{\isacharunderscore}{\kern0pt}imp{\isacharunderscore}{\kern0pt}subset\ \isacommand{by}\isamarkupfalse%
\ auto\isanewline
\isacommand{qed}\isamarkupfalse%
%
\endisatagproof
{\isafoldproof}%
%
\isadelimproof
\isanewline
%
\endisadelimproof
\isanewline
\isacommand{lemma}\isamarkupfalse%
\ rensep{\isacharunderscore}{\kern0pt}type\ {\isacharcolon}{\kern0pt}\isanewline
\ \ \isakeyword{assumes}\ {\isachardoublequoteopen}n{\isasymin}nat{\isachardoublequoteclose}\isanewline
\ \ \isakeyword{shows}\ {\isachardoublequoteopen}rensep{\isacharparenleft}{\kern0pt}n{\isacharparenright}{\kern0pt}\ {\isasymin}\ {\isadigit{7}}{\isacharhash}{\kern0pt}{\isacharplus}{\kern0pt}n\ {\isasymrightarrow}\ {\isadigit{7}}{\isacharhash}{\kern0pt}{\isacharplus}{\kern0pt}n{\isachardoublequoteclose}\isanewline
%
\isadelimproof
%
\endisadelimproof
%
\isatagproof
\isacommand{proof}\isamarkupfalse%
\ {\isacharminus}{\kern0pt}\isanewline
\ \ \isacommand{from}\isamarkupfalse%
\ {\isacartoucheopen}n{\isasymin}nat{\isacartoucheclose}\isanewline
\ \ \isacommand{have}\isamarkupfalse%
\ {\isachardoublequoteopen}rensep{\isacharparenleft}{\kern0pt}n{\isacharparenright}{\kern0pt}\ {\isasymin}\ {\isacharparenleft}{\kern0pt}{\isadigit{7}}{\isacharhash}{\kern0pt}{\isacharplus}{\kern0pt}n{\isacharminus}{\kern0pt}weak{\isacharparenleft}{\kern0pt}n{\isacharcomma}{\kern0pt}{\isadigit{5}}{\isacharparenright}{\kern0pt}{\isacharparenright}{\kern0pt}\ {\isasymunion}\ weak{\isacharparenleft}{\kern0pt}n{\isacharcomma}{\kern0pt}{\isadigit{5}}{\isacharparenright}{\kern0pt}\ {\isasymrightarrow}\ {\isadigit{7}}{\isacharhash}{\kern0pt}{\isacharplus}{\kern0pt}n\ {\isasymunion}\ {\isacharparenleft}{\kern0pt}{\isadigit{7}}{\isacharhash}{\kern0pt}{\isacharplus}{\kern0pt}n\ {\isacharminus}{\kern0pt}\ {\isadigit{7}}{\isacharparenright}{\kern0pt}{\isachardoublequoteclose}\isanewline
\ \ \ \ \isacommand{unfolding}\isamarkupfalse%
\ rensep{\isacharunderscore}{\kern0pt}def\ \isanewline
\ \ \ \ \isacommand{using}\isamarkupfalse%
\ union{\isacharunderscore}{\kern0pt}fun{\isacharunderscore}{\kern0pt}type\ \ sep{\isacharunderscore}{\kern0pt}var{\isacharunderscore}{\kern0pt}type\ {\isacartoucheopen}n{\isasymin}nat{\isacartoucheclose}\ sep{\isacharunderscore}{\kern0pt}env{\isacharunderscore}{\kern0pt}type\ weak{\isacharunderscore}{\kern0pt}equal\isanewline
\ \ \ \ \isacommand{by}\isamarkupfalse%
\ force\isanewline
\ \ \isacommand{then}\isamarkupfalse%
\isanewline
\ \ \isacommand{show}\isamarkupfalse%
\ {\isacharquery}{\kern0pt}thesis\ \isacommand{using}\isamarkupfalse%
\ rensep{\isacharunderscore}{\kern0pt}aux\ {\isacartoucheopen}n{\isasymin}nat{\isacartoucheclose}\ \isacommand{by}\isamarkupfalse%
\ auto\ \isanewline
\isacommand{qed}\isamarkupfalse%
%
\endisatagproof
{\isafoldproof}%
%
\isadelimproof
\isanewline
%
\endisadelimproof
\isanewline
\isacommand{lemma}\isamarkupfalse%
\ rensep{\isacharunderscore}{\kern0pt}action\ {\isacharcolon}{\kern0pt}\isanewline
\ \ \isakeyword{assumes}\ {\isachardoublequoteopen}{\isacharbrackleft}{\kern0pt}t{\isacharcomma}{\kern0pt}p{\isacharcomma}{\kern0pt}u{\isacharcomma}{\kern0pt}P{\isacharcomma}{\kern0pt}leq{\isacharcomma}{\kern0pt}o{\isacharcomma}{\kern0pt}pi{\isacharbrackright}{\kern0pt}\ {\isacharat}{\kern0pt}\ env\ {\isasymin}\ list{\isacharparenleft}{\kern0pt}M{\isacharparenright}{\kern0pt}{\isachardoublequoteclose}\isanewline
\ \ \isakeyword{shows}\ {\isachardoublequoteopen}{\isasymforall}\ i\ {\isachardot}{\kern0pt}\ i\ {\isacharless}{\kern0pt}\ {\isadigit{7}}{\isacharhash}{\kern0pt}{\isacharplus}{\kern0pt}length{\isacharparenleft}{\kern0pt}env{\isacharparenright}{\kern0pt}\ {\isasymlongrightarrow}\ nth{\isacharparenleft}{\kern0pt}rensep{\isacharparenleft}{\kern0pt}length{\isacharparenleft}{\kern0pt}env{\isacharparenright}{\kern0pt}{\isacharparenright}{\kern0pt}{\isacharbackquote}{\kern0pt}i{\isacharcomma}{\kern0pt}{\isacharbrackleft}{\kern0pt}t{\isacharcomma}{\kern0pt}p{\isacharcomma}{\kern0pt}u{\isacharcomma}{\kern0pt}P{\isacharcomma}{\kern0pt}leq{\isacharcomma}{\kern0pt}o{\isacharcomma}{\kern0pt}pi{\isacharbrackright}{\kern0pt}{\isacharat}{\kern0pt}env{\isacharparenright}{\kern0pt}\ {\isacharequal}{\kern0pt}\ nth{\isacharparenleft}{\kern0pt}i{\isacharcomma}{\kern0pt}{\isacharbrackleft}{\kern0pt}p{\isacharcomma}{\kern0pt}P{\isacharcomma}{\kern0pt}leq{\isacharcomma}{\kern0pt}o{\isacharcomma}{\kern0pt}t{\isacharbrackright}{\kern0pt}\ {\isacharat}{\kern0pt}\ env\ {\isacharat}{\kern0pt}\ {\isacharbrackleft}{\kern0pt}pi{\isacharcomma}{\kern0pt}u{\isacharbrackright}{\kern0pt}{\isacharparenright}{\kern0pt}{\isachardoublequoteclose}\isanewline
%
\isadelimproof
%
\endisadelimproof
%
\isatagproof
\isacommand{proof}\isamarkupfalse%
\ {\isacharminus}{\kern0pt}\ \isanewline
\ \ \isacommand{let}\isamarkupfalse%
\ {\isacharquery}{\kern0pt}tgt{\isacharequal}{\kern0pt}{\isachardoublequoteopen}{\isacharbrackleft}{\kern0pt}t{\isacharcomma}{\kern0pt}p{\isacharcomma}{\kern0pt}u{\isacharcomma}{\kern0pt}P{\isacharcomma}{\kern0pt}leq{\isacharcomma}{\kern0pt}o{\isacharcomma}{\kern0pt}pi{\isacharbrackright}{\kern0pt}{\isacharat}{\kern0pt}env{\isachardoublequoteclose}\isanewline
\ \ \isacommand{let}\isamarkupfalse%
\ {\isacharquery}{\kern0pt}src{\isacharequal}{\kern0pt}{\isachardoublequoteopen}{\isacharbrackleft}{\kern0pt}p{\isacharcomma}{\kern0pt}P{\isacharcomma}{\kern0pt}leq{\isacharcomma}{\kern0pt}o{\isacharcomma}{\kern0pt}t{\isacharbrackright}{\kern0pt}\ {\isacharat}{\kern0pt}\ env\ {\isacharat}{\kern0pt}\ {\isacharbrackleft}{\kern0pt}pi{\isacharcomma}{\kern0pt}u{\isacharbrackright}{\kern0pt}{\isachardoublequoteclose}\isanewline
\ \ \isacommand{let}\isamarkupfalse%
\ {\isacharquery}{\kern0pt}m{\isacharequal}{\kern0pt}{\isachardoublequoteopen}{\isadigit{7}}\ {\isacharhash}{\kern0pt}{\isacharplus}{\kern0pt}\ length{\isacharparenleft}{\kern0pt}env{\isacharparenright}{\kern0pt}\ {\isacharminus}{\kern0pt}\ weak{\isacharparenleft}{\kern0pt}length{\isacharparenleft}{\kern0pt}env{\isacharparenright}{\kern0pt}{\isacharcomma}{\kern0pt}{\isadigit{5}}{\isacharparenright}{\kern0pt}{\isachardoublequoteclose}\isanewline
\ \ \isacommand{let}\isamarkupfalse%
\ {\isacharquery}{\kern0pt}p{\isacharequal}{\kern0pt}{\isachardoublequoteopen}weak{\isacharparenleft}{\kern0pt}length{\isacharparenleft}{\kern0pt}env{\isacharparenright}{\kern0pt}{\isacharcomma}{\kern0pt}{\isadigit{5}}{\isacharparenright}{\kern0pt}{\isachardoublequoteclose}\isanewline
\ \ \isacommand{let}\isamarkupfalse%
\ {\isacharquery}{\kern0pt}f{\isacharequal}{\kern0pt}{\isachardoublequoteopen}sep{\isacharunderscore}{\kern0pt}var{\isacharparenleft}{\kern0pt}length{\isacharparenleft}{\kern0pt}env{\isacharparenright}{\kern0pt}{\isacharparenright}{\kern0pt}{\isachardoublequoteclose}\isanewline
\ \ \isacommand{let}\isamarkupfalse%
\ {\isacharquery}{\kern0pt}g{\isacharequal}{\kern0pt}{\isachardoublequoteopen}sep{\isacharunderscore}{\kern0pt}env{\isacharparenleft}{\kern0pt}length{\isacharparenleft}{\kern0pt}env{\isacharparenright}{\kern0pt}{\isacharparenright}{\kern0pt}{\isachardoublequoteclose}\isanewline
\ \ \isacommand{let}\isamarkupfalse%
\ {\isacharquery}{\kern0pt}n{\isacharequal}{\kern0pt}{\isachardoublequoteopen}length{\isacharparenleft}{\kern0pt}env{\isacharparenright}{\kern0pt}{\isachardoublequoteclose}\isanewline
\ \ \isacommand{from}\isamarkupfalse%
\ assms\isanewline
\ \ \isacommand{have}\isamarkupfalse%
\ {\isadigit{1}}\ {\isacharcolon}{\kern0pt}\ {\isachardoublequoteopen}{\isacharbrackleft}{\kern0pt}t{\isacharcomma}{\kern0pt}p{\isacharcomma}{\kern0pt}u{\isacharcomma}{\kern0pt}P{\isacharcomma}{\kern0pt}leq{\isacharcomma}{\kern0pt}o{\isacharcomma}{\kern0pt}pi{\isacharbrackright}{\kern0pt}\ {\isasymin}\ list{\isacharparenleft}{\kern0pt}M{\isacharparenright}{\kern0pt}{\isachardoublequoteclose}\ {\isachardoublequoteopen}\ env\ {\isasymin}\ list{\isacharparenleft}{\kern0pt}M{\isacharparenright}{\kern0pt}{\isachardoublequoteclose}\isanewline
\ \ \ \ {\isachardoublequoteopen}{\isacharquery}{\kern0pt}src\ {\isasymin}\ list{\isacharparenleft}{\kern0pt}M{\isacharparenright}{\kern0pt}{\isachardoublequoteclose}\ {\isachardoublequoteopen}{\isacharquery}{\kern0pt}tgt\ {\isasymin}\ list{\isacharparenleft}{\kern0pt}M{\isacharparenright}{\kern0pt}{\isachardoublequoteclose}\ \ \isanewline
\ \ \ \ {\isachardoublequoteopen}{\isadigit{7}}{\isacharhash}{\kern0pt}{\isacharplus}{\kern0pt}{\isacharquery}{\kern0pt}n\ {\isacharequal}{\kern0pt}\ {\isacharparenleft}{\kern0pt}{\isadigit{7}}{\isacharhash}{\kern0pt}{\isacharplus}{\kern0pt}{\isacharquery}{\kern0pt}n{\isacharminus}{\kern0pt}weak{\isacharparenleft}{\kern0pt}{\isacharquery}{\kern0pt}n{\isacharcomma}{\kern0pt}{\isadigit{5}}{\isacharparenright}{\kern0pt}{\isacharparenright}{\kern0pt}\ {\isasymunion}\ weak{\isacharparenleft}{\kern0pt}{\isacharquery}{\kern0pt}n{\isacharcomma}{\kern0pt}{\isadigit{5}}{\isacharparenright}{\kern0pt}{\isachardoublequoteclose}\isanewline
\ \ \ \ {\isachardoublequoteopen}\ length{\isacharparenleft}{\kern0pt}{\isacharquery}{\kern0pt}src{\isacharparenright}{\kern0pt}\ {\isacharequal}{\kern0pt}\ {\isacharparenleft}{\kern0pt}{\isadigit{7}}{\isacharhash}{\kern0pt}{\isacharplus}{\kern0pt}{\isacharquery}{\kern0pt}n{\isacharminus}{\kern0pt}weak{\isacharparenleft}{\kern0pt}{\isacharquery}{\kern0pt}n{\isacharcomma}{\kern0pt}{\isadigit{5}}{\isacharparenright}{\kern0pt}{\isacharparenright}{\kern0pt}\ {\isasymunion}\ weak{\isacharparenleft}{\kern0pt}{\isacharquery}{\kern0pt}n{\isacharcomma}{\kern0pt}{\isadigit{5}}{\isacharparenright}{\kern0pt}{\isachardoublequoteclose}\isanewline
\ \ \ \ \isacommand{using}\isamarkupfalse%
\ Diff{\isacharunderscore}{\kern0pt}partition\ le{\isacharunderscore}{\kern0pt}imp{\isacharunderscore}{\kern0pt}subset\ rensep{\isacharunderscore}{\kern0pt}aux\ \isacommand{by}\isamarkupfalse%
\ auto\isanewline
\ \ \isacommand{then}\isamarkupfalse%
\isanewline
\ \ \isacommand{have}\isamarkupfalse%
\ {\isachardoublequoteopen}nth{\isacharparenleft}{\kern0pt}i{\isacharcomma}{\kern0pt}\ {\isacharquery}{\kern0pt}src{\isacharparenright}{\kern0pt}\ {\isacharequal}{\kern0pt}\ nth{\isacharparenleft}{\kern0pt}union{\isacharunderscore}{\kern0pt}fun{\isacharparenleft}{\kern0pt}{\isacharquery}{\kern0pt}f{\isacharcomma}{\kern0pt}\ {\isacharquery}{\kern0pt}g{\isacharcomma}{\kern0pt}\ {\isacharquery}{\kern0pt}m{\isacharcomma}{\kern0pt}\ {\isacharquery}{\kern0pt}p{\isacharparenright}{\kern0pt}\ {\isacharbackquote}{\kern0pt}\ i{\isacharcomma}{\kern0pt}\ {\isacharquery}{\kern0pt}tgt{\isacharparenright}{\kern0pt}{\isachardoublequoteclose}\ \isakeyword{if}\ {\isachardoublequoteopen}i\ {\isacharless}{\kern0pt}\ {\isadigit{7}}{\isacharhash}{\kern0pt}{\isacharplus}{\kern0pt}length{\isacharparenleft}{\kern0pt}env{\isacharparenright}{\kern0pt}{\isachardoublequoteclose}\ \isakeyword{for}\ i\isanewline
\ \ \isacommand{proof}\isamarkupfalse%
\ {\isacharminus}{\kern0pt}\isanewline
\ \ \ \ \isacommand{from}\isamarkupfalse%
\ {\isacartoucheopen}i{\isacharless}{\kern0pt}{\isadigit{7}}{\isacharhash}{\kern0pt}{\isacharplus}{\kern0pt}{\isacharquery}{\kern0pt}n{\isacartoucheclose}\isanewline
\ \ \ \ \isacommand{have}\isamarkupfalse%
\ {\isachardoublequoteopen}i\ {\isasymin}\ {\isacharparenleft}{\kern0pt}{\isadigit{7}}{\isacharhash}{\kern0pt}{\isacharplus}{\kern0pt}{\isacharquery}{\kern0pt}n{\isacharminus}{\kern0pt}weak{\isacharparenleft}{\kern0pt}{\isacharquery}{\kern0pt}n{\isacharcomma}{\kern0pt}{\isadigit{5}}{\isacharparenright}{\kern0pt}{\isacharparenright}{\kern0pt}\ {\isasymunion}\ weak{\isacharparenleft}{\kern0pt}{\isacharquery}{\kern0pt}n{\isacharcomma}{\kern0pt}{\isadigit{5}}{\isacharparenright}{\kern0pt}{\isachardoublequoteclose}\ \isanewline
\ \ \ \ \ \ \isacommand{using}\isamarkupfalse%
\ ltD\ \isacommand{by}\isamarkupfalse%
\ simp\ \isanewline
\ \ \ \ \isacommand{then}\isamarkupfalse%
\ \isacommand{show}\isamarkupfalse%
\ {\isacharquery}{\kern0pt}thesis\isanewline
\ \ \ \ \ \ \isacommand{unfolding}\isamarkupfalse%
\ rensep{\isacharunderscore}{\kern0pt}def\ \isacommand{using}\isamarkupfalse%
\ \ \isanewline
\ \ \ \ \ \ \ \ union{\isacharunderscore}{\kern0pt}fun{\isacharunderscore}{\kern0pt}action{\isacharbrackleft}{\kern0pt}OF\ {\isacartoucheopen}{\isacharquery}{\kern0pt}src{\isasymin}list{\isacharparenleft}{\kern0pt}M{\isacharparenright}{\kern0pt}{\isacartoucheclose}\ {\isacartoucheopen}{\isacharquery}{\kern0pt}tgt{\isasymin}list{\isacharparenleft}{\kern0pt}M{\isacharparenright}{\kern0pt}{\isacartoucheclose}\ {\isacartoucheopen}length{\isacharparenleft}{\kern0pt}{\isacharquery}{\kern0pt}src{\isacharparenright}{\kern0pt}\ {\isacharequal}{\kern0pt}\ {\isacharparenleft}{\kern0pt}{\isadigit{7}}{\isacharhash}{\kern0pt}{\isacharplus}{\kern0pt}{\isacharquery}{\kern0pt}n{\isacharminus}{\kern0pt}weak{\isacharparenleft}{\kern0pt}{\isacharquery}{\kern0pt}n{\isacharcomma}{\kern0pt}{\isadigit{5}}{\isacharparenright}{\kern0pt}{\isacharparenright}{\kern0pt}\ {\isasymunion}\ weak{\isacharparenleft}{\kern0pt}{\isacharquery}{\kern0pt}n{\isacharcomma}{\kern0pt}{\isadigit{5}}{\isacharparenright}{\kern0pt}{\isacartoucheclose}\isanewline
\ \ \ \ \ \ \ \ \ \ sep{\isacharunderscore}{\kern0pt}var{\isacharunderscore}{\kern0pt}action{\isacharbrackleft}{\kern0pt}OF\ {\isacartoucheopen}{\isacharbrackleft}{\kern0pt}t{\isacharcomma}{\kern0pt}p{\isacharcomma}{\kern0pt}u{\isacharcomma}{\kern0pt}P{\isacharcomma}{\kern0pt}leq{\isacharcomma}{\kern0pt}o{\isacharcomma}{\kern0pt}pi{\isacharbrackright}{\kern0pt}\ {\isasymin}\ list{\isacharparenleft}{\kern0pt}M{\isacharparenright}{\kern0pt}{\isacartoucheclose}\ {\isacartoucheopen}env{\isasymin}list{\isacharparenleft}{\kern0pt}M{\isacharparenright}{\kern0pt}{\isacartoucheclose}{\isacharbrackright}{\kern0pt}\ \ \ \ \ \ \isanewline
\ \ \ \ \ \ \ \ \ \ sep{\isacharunderscore}{\kern0pt}env{\isacharunderscore}{\kern0pt}action{\isacharbrackleft}{\kern0pt}OF\ {\isacartoucheopen}{\isacharbrackleft}{\kern0pt}t{\isacharcomma}{\kern0pt}p{\isacharcomma}{\kern0pt}u{\isacharcomma}{\kern0pt}P{\isacharcomma}{\kern0pt}leq{\isacharcomma}{\kern0pt}o{\isacharcomma}{\kern0pt}pi{\isacharbrackright}{\kern0pt}\ {\isasymin}\ list{\isacharparenleft}{\kern0pt}M{\isacharparenright}{\kern0pt}{\isacartoucheclose}\ {\isacartoucheopen}env{\isasymin}list{\isacharparenleft}{\kern0pt}M{\isacharparenright}{\kern0pt}{\isacartoucheclose}{\isacharbrackright}{\kern0pt}\isanewline
\ \ \ \ \ \ \ \ \ \ {\isacharbrackright}{\kern0pt}\ that\ \isanewline
\ \ \ \ \ \ \isacommand{by}\isamarkupfalse%
\ simp\isanewline
\ \ \isacommand{qed}\isamarkupfalse%
\isanewline
\ \ \isacommand{then}\isamarkupfalse%
\ \isacommand{show}\isamarkupfalse%
\ {\isacharquery}{\kern0pt}thesis\ \isacommand{unfolding}\isamarkupfalse%
\ rensep{\isacharunderscore}{\kern0pt}def\ \isacommand{by}\isamarkupfalse%
\ simp\isanewline
\isacommand{qed}\isamarkupfalse%
%
\endisatagproof
{\isafoldproof}%
%
\isadelimproof
\isanewline
%
\endisadelimproof
\isanewline
\isacommand{definition}\isamarkupfalse%
\ sep{\isacharunderscore}{\kern0pt}ren\ {\isacharcolon}{\kern0pt}{\isacharcolon}{\kern0pt}\ {\isachardoublequoteopen}{\isacharbrackleft}{\kern0pt}i{\isacharcomma}{\kern0pt}i{\isacharbrackright}{\kern0pt}\ {\isasymRightarrow}\ i{\isachardoublequoteclose}\ \isakeyword{where}\isanewline
\ \ {\isachardoublequoteopen}sep{\isacharunderscore}{\kern0pt}ren{\isacharparenleft}{\kern0pt}n{\isacharcomma}{\kern0pt}{\isasymphi}{\isacharparenright}{\kern0pt}\ {\isasymequiv}\ ren{\isacharparenleft}{\kern0pt}{\isasymphi}{\isacharparenright}{\kern0pt}{\isacharbackquote}{\kern0pt}{\isacharparenleft}{\kern0pt}{\isadigit{7}}{\isacharhash}{\kern0pt}{\isacharplus}{\kern0pt}n{\isacharparenright}{\kern0pt}{\isacharbackquote}{\kern0pt}{\isacharparenleft}{\kern0pt}{\isadigit{7}}{\isacharhash}{\kern0pt}{\isacharplus}{\kern0pt}n{\isacharparenright}{\kern0pt}{\isacharbackquote}{\kern0pt}rensep{\isacharparenleft}{\kern0pt}n{\isacharparenright}{\kern0pt}{\isachardoublequoteclose}\isanewline
\isanewline
\isacommand{lemma}\isamarkupfalse%
\ arity{\isacharunderscore}{\kern0pt}rensep{\isacharcolon}{\kern0pt}\ \isakeyword{assumes}\ {\isachardoublequoteopen}{\isasymphi}{\isasymin}formula{\isachardoublequoteclose}\ {\isachardoublequoteopen}env\ {\isasymin}\ list{\isacharparenleft}{\kern0pt}M{\isacharparenright}{\kern0pt}{\isachardoublequoteclose}\isanewline
\ \ {\isachardoublequoteopen}arity{\isacharparenleft}{\kern0pt}{\isasymphi}{\isacharparenright}{\kern0pt}\ {\isasymle}\ {\isadigit{7}}{\isacharhash}{\kern0pt}{\isacharplus}{\kern0pt}length{\isacharparenleft}{\kern0pt}env{\isacharparenright}{\kern0pt}{\isachardoublequoteclose}\isanewline
\isakeyword{shows}\ {\isachardoublequoteopen}arity{\isacharparenleft}{\kern0pt}sep{\isacharunderscore}{\kern0pt}ren{\isacharparenleft}{\kern0pt}length{\isacharparenleft}{\kern0pt}env{\isacharparenright}{\kern0pt}{\isacharcomma}{\kern0pt}{\isasymphi}{\isacharparenright}{\kern0pt}{\isacharparenright}{\kern0pt}\ {\isasymle}\ {\isadigit{7}}{\isacharhash}{\kern0pt}{\isacharplus}{\kern0pt}length{\isacharparenleft}{\kern0pt}env{\isacharparenright}{\kern0pt}{\isachardoublequoteclose}\isanewline
%
\isadelimproof
\ \ %
\endisadelimproof
%
\isatagproof
\isacommand{unfolding}\isamarkupfalse%
\ sep{\isacharunderscore}{\kern0pt}ren{\isacharunderscore}{\kern0pt}def\isanewline
\ \ \isacommand{using}\isamarkupfalse%
\ arity{\isacharunderscore}{\kern0pt}ren\ rensep{\isacharunderscore}{\kern0pt}type\ assms\isanewline
\ \ \isacommand{by}\isamarkupfalse%
\ simp%
\endisatagproof
{\isafoldproof}%
%
\isadelimproof
\isanewline
%
\endisadelimproof
\isanewline
\isacommand{lemma}\isamarkupfalse%
\ type{\isacharunderscore}{\kern0pt}rensep\ {\isacharbrackleft}{\kern0pt}TC{\isacharbrackright}{\kern0pt}{\isacharcolon}{\kern0pt}\ \isanewline
\ \ \isakeyword{assumes}\ {\isachardoublequoteopen}{\isasymphi}{\isasymin}formula{\isachardoublequoteclose}\ {\isachardoublequoteopen}env{\isasymin}list{\isacharparenleft}{\kern0pt}M{\isacharparenright}{\kern0pt}{\isachardoublequoteclose}\ \isanewline
\ \ \isakeyword{shows}\ {\isachardoublequoteopen}sep{\isacharunderscore}{\kern0pt}ren{\isacharparenleft}{\kern0pt}length{\isacharparenleft}{\kern0pt}env{\isacharparenright}{\kern0pt}{\isacharcomma}{\kern0pt}{\isasymphi}{\isacharparenright}{\kern0pt}\ {\isasymin}\ formula{\isachardoublequoteclose}\isanewline
%
\isadelimproof
\ \ %
\endisadelimproof
%
\isatagproof
\isacommand{unfolding}\isamarkupfalse%
\ sep{\isacharunderscore}{\kern0pt}ren{\isacharunderscore}{\kern0pt}def\isanewline
\ \ \isacommand{using}\isamarkupfalse%
\ ren{\isacharunderscore}{\kern0pt}tc\ rensep{\isacharunderscore}{\kern0pt}type\ assms\isanewline
\ \ \isacommand{by}\isamarkupfalse%
\ simp%
\endisatagproof
{\isafoldproof}%
%
\isadelimproof
\isanewline
%
\endisadelimproof
\isanewline
\isacommand{lemma}\isamarkupfalse%
\ sepren{\isacharunderscore}{\kern0pt}action{\isacharcolon}{\kern0pt}\ \isanewline
\ \ \isakeyword{assumes}\ {\isachardoublequoteopen}arity{\isacharparenleft}{\kern0pt}{\isasymphi}{\isacharparenright}{\kern0pt}\ {\isasymle}\ {\isadigit{7}}\ {\isacharhash}{\kern0pt}{\isacharplus}{\kern0pt}\ length{\isacharparenleft}{\kern0pt}env{\isacharparenright}{\kern0pt}{\isachardoublequoteclose}\isanewline
\ \ \ \ {\isachardoublequoteopen}{\isacharbrackleft}{\kern0pt}t{\isacharcomma}{\kern0pt}p{\isacharcomma}{\kern0pt}u{\isacharcomma}{\kern0pt}P{\isacharcomma}{\kern0pt}leq{\isacharcomma}{\kern0pt}o{\isacharcomma}{\kern0pt}pi{\isacharbrackright}{\kern0pt}\ {\isasymin}\ list{\isacharparenleft}{\kern0pt}M{\isacharparenright}{\kern0pt}{\isachardoublequoteclose}\isanewline
\ \ \ \ {\isachardoublequoteopen}env{\isasymin}list{\isacharparenleft}{\kern0pt}M{\isacharparenright}{\kern0pt}{\isachardoublequoteclose}\isanewline
\ \ \ \ {\isachardoublequoteopen}{\isasymphi}{\isasymin}formula{\isachardoublequoteclose}\isanewline
\ \ \isakeyword{shows}\ {\isachardoublequoteopen}sats{\isacharparenleft}{\kern0pt}M{\isacharcomma}{\kern0pt}\ sep{\isacharunderscore}{\kern0pt}ren{\isacharparenleft}{\kern0pt}length{\isacharparenleft}{\kern0pt}env{\isacharparenright}{\kern0pt}{\isacharcomma}{\kern0pt}{\isasymphi}{\isacharparenright}{\kern0pt}{\isacharcomma}{\kern0pt}{\isacharbrackleft}{\kern0pt}t{\isacharcomma}{\kern0pt}p{\isacharcomma}{\kern0pt}u{\isacharcomma}{\kern0pt}P{\isacharcomma}{\kern0pt}leq{\isacharcomma}{\kern0pt}o{\isacharcomma}{\kern0pt}pi{\isacharbrackright}{\kern0pt}\ {\isacharat}{\kern0pt}\ env{\isacharparenright}{\kern0pt}\ {\isasymlongleftrightarrow}\ sats{\isacharparenleft}{\kern0pt}M{\isacharcomma}{\kern0pt}\ {\isasymphi}{\isacharcomma}{\kern0pt}{\isacharbrackleft}{\kern0pt}p{\isacharcomma}{\kern0pt}P{\isacharcomma}{\kern0pt}leq{\isacharcomma}{\kern0pt}o{\isacharcomma}{\kern0pt}t{\isacharbrackright}{\kern0pt}\ {\isacharat}{\kern0pt}\ env\ {\isacharat}{\kern0pt}\ {\isacharbrackleft}{\kern0pt}pi{\isacharcomma}{\kern0pt}u{\isacharbrackright}{\kern0pt}{\isacharparenright}{\kern0pt}{\isachardoublequoteclose}\isanewline
%
\isadelimproof
%
\endisadelimproof
%
\isatagproof
\isacommand{proof}\isamarkupfalse%
\ {\isacharminus}{\kern0pt}\isanewline
\ \ \isacommand{from}\isamarkupfalse%
\ assms\isanewline
\ \ \isacommand{have}\isamarkupfalse%
\ {\isadigit{1}}{\isacharcolon}{\kern0pt}\ {\isachardoublequoteopen}\ {\isacharbrackleft}{\kern0pt}t{\isacharcomma}{\kern0pt}\ p{\isacharcomma}{\kern0pt}\ u{\isacharcomma}{\kern0pt}\ P{\isacharcomma}{\kern0pt}\ leq{\isacharcomma}{\kern0pt}\ o{\isacharcomma}{\kern0pt}\ pi{\isacharbrackright}{\kern0pt}\ {\isacharat}{\kern0pt}\ env\ {\isasymin}\ list{\isacharparenleft}{\kern0pt}M{\isacharparenright}{\kern0pt}{\isachardoublequoteclose}\ \isanewline
\ \ \ \ {\isachardoublequoteopen}{\isacharbrackleft}{\kern0pt}P{\isacharcomma}{\kern0pt}leq{\isacharcomma}{\kern0pt}o{\isacharcomma}{\kern0pt}p{\isacharcomma}{\kern0pt}t{\isacharbrackright}{\kern0pt}\ {\isasymin}\ list{\isacharparenleft}{\kern0pt}M{\isacharparenright}{\kern0pt}{\isachardoublequoteclose}\isanewline
\ \ \ \ {\isachardoublequoteopen}{\isacharbrackleft}{\kern0pt}pi{\isacharcomma}{\kern0pt}u{\isacharbrackright}{\kern0pt}\ {\isasymin}\ list{\isacharparenleft}{\kern0pt}M{\isacharparenright}{\kern0pt}{\isachardoublequoteclose}\ \ \ \ \isanewline
\ \ \ \ \isacommand{by}\isamarkupfalse%
\ simp{\isacharunderscore}{\kern0pt}all\isanewline
\ \ \isacommand{then}\isamarkupfalse%
\ \isanewline
\ \ \isacommand{have}\isamarkupfalse%
\ {\isadigit{2}}{\isacharcolon}{\kern0pt}\ {\isachardoublequoteopen}{\isacharbrackleft}{\kern0pt}p{\isacharcomma}{\kern0pt}P{\isacharcomma}{\kern0pt}leq{\isacharcomma}{\kern0pt}o{\isacharcomma}{\kern0pt}t{\isacharbrackright}{\kern0pt}\ {\isacharat}{\kern0pt}\ env\ {\isacharat}{\kern0pt}\ {\isacharbrackleft}{\kern0pt}pi{\isacharcomma}{\kern0pt}u{\isacharbrackright}{\kern0pt}\ {\isasymin}\ list{\isacharparenleft}{\kern0pt}M{\isacharparenright}{\kern0pt}{\isachardoublequoteclose}\ \isacommand{using}\isamarkupfalse%
\ app{\isacharunderscore}{\kern0pt}type\ \isacommand{by}\isamarkupfalse%
\ simp\isanewline
\ \ \isacommand{show}\isamarkupfalse%
\ {\isacharquery}{\kern0pt}thesis\ \isanewline
\ \ \ \ \isacommand{unfolding}\isamarkupfalse%
\ sep{\isacharunderscore}{\kern0pt}ren{\isacharunderscore}{\kern0pt}def\isanewline
\ \ \ \ \isacommand{using}\isamarkupfalse%
\ sats{\isacharunderscore}{\kern0pt}iff{\isacharunderscore}{\kern0pt}sats{\isacharunderscore}{\kern0pt}ren{\isacharbrackleft}{\kern0pt}OF\ {\isacartoucheopen}{\isasymphi}{\isasymin}formula{\isacartoucheclose}\ \ \ \ \ \ \ \isanewline
\ \ \ \ \ \ \ \ add{\isacharunderscore}{\kern0pt}type{\isacharbrackleft}{\kern0pt}of\ {\isadigit{7}}\ {\isachardoublequoteopen}length{\isacharparenleft}{\kern0pt}env{\isacharparenright}{\kern0pt}{\isachardoublequoteclose}{\isacharbrackright}{\kern0pt}\isanewline
\ \ \ \ \ \ \ \ add{\isacharunderscore}{\kern0pt}type{\isacharbrackleft}{\kern0pt}of\ {\isadigit{7}}\ {\isachardoublequoteopen}length{\isacharparenleft}{\kern0pt}env{\isacharparenright}{\kern0pt}{\isachardoublequoteclose}{\isacharbrackright}{\kern0pt}\isanewline
\ \ \ \ \ \ \ \ {\isadigit{2}}\ {\isadigit{1}}{\isacharparenleft}{\kern0pt}{\isadigit{1}}{\isacharparenright}{\kern0pt}\ \isanewline
\ \ \ \ \ \ \ \ rensep{\isacharunderscore}{\kern0pt}type{\isacharbrackleft}{\kern0pt}OF\ length{\isacharunderscore}{\kern0pt}type{\isacharbrackleft}{\kern0pt}OF\ {\isacartoucheopen}env{\isasymin}list{\isacharparenleft}{\kern0pt}M{\isacharparenright}{\kern0pt}{\isacartoucheclose}{\isacharbrackright}{\kern0pt}{\isacharbrackright}{\kern0pt}\ \isanewline
\ \ \ \ \ \ \ \ {\isacartoucheopen}arity{\isacharparenleft}{\kern0pt}{\isasymphi}{\isacharparenright}{\kern0pt}\ {\isasymle}\ {\isadigit{7}}\ {\isacharhash}{\kern0pt}{\isacharplus}{\kern0pt}\ length{\isacharparenleft}{\kern0pt}env{\isacharparenright}{\kern0pt}{\isacartoucheclose}{\isacharbrackright}{\kern0pt}\isanewline
\ \ \ \ \ \ rensep{\isacharunderscore}{\kern0pt}action{\isacharbrackleft}{\kern0pt}OF\ {\isadigit{1}}{\isacharparenleft}{\kern0pt}{\isadigit{1}}{\isacharparenright}{\kern0pt}{\isacharcomma}{\kern0pt}rule{\isacharunderscore}{\kern0pt}format{\isacharcomma}{\kern0pt}symmetric{\isacharbrackright}{\kern0pt}\isanewline
\ \ \ \ \isacommand{by}\isamarkupfalse%
\ simp\isanewline
\isacommand{qed}\isamarkupfalse%
%
\endisatagproof
{\isafoldproof}%
%
\isadelimproof
\isanewline
%
\endisadelimproof
%
\isadelimtheory
\isanewline
%
\endisadelimtheory
%
\isatagtheory
\isacommand{end}\isamarkupfalse%
%
\endisatagtheory
{\isafoldtheory}%
%
\isadelimtheory
%
\endisadelimtheory
%
\end{isabellebody}%
\endinput
%:%file=~/source/repos/ZF-notAC/code/Forcing/Separation_Rename.thy%:%
%:%11=1%:%
%:%27=2%:%
%:%28=2%:%
%:%29=3%:%
%:%30=4%:%
%:%35=4%:%
%:%38=5%:%
%:%39=6%:%
%:%40=6%:%
%:%41=7%:%
%:%42=8%:%
%:%43=8%:%
%:%46=9%:%
%:%50=9%:%
%:%51=9%:%
%:%56=9%:%
%:%59=10%:%
%:%60=11%:%
%:%61=11%:%
%:%64=12%:%
%:%68=12%:%
%:%69=12%:%
%:%74=12%:%
%:%77=13%:%
%:%78=14%:%
%:%79=14%:%
%:%82=15%:%
%:%86=15%:%
%:%87=15%:%
%:%92=15%:%
%:%95=16%:%
%:%96=17%:%
%:%97=17%:%
%:%98=18%:%
%:%99=19%:%
%:%100=20%:%
%:%101=21%:%
%:%102=21%:%
%:%103=22%:%
%:%104=23%:%
%:%105=24%:%
%:%106=25%:%
%:%107=25%:%
%:%108=26%:%
%:%109=27%:%
%:%110=28%:%
%:%111=28%:%
%:%112=29%:%
%:%113=30%:%
%:%116=31%:%
%:%120=31%:%
%:%121=31%:%
%:%122=31%:%
%:%123=31%:%
%:%128=31%:%
%:%131=32%:%
%:%132=33%:%
%:%133=33%:%
%:%134=34%:%
%:%135=35%:%
%:%142=36%:%
%:%143=36%:%
%:%144=37%:%
%:%145=37%:%
%:%146=38%:%
%:%147=38%:%
%:%148=39%:%
%:%149=39%:%
%:%150=40%:%
%:%151=40%:%
%:%152=41%:%
%:%153=41%:%
%:%154=42%:%
%:%155=42%:%
%:%156=43%:%
%:%157=44%:%
%:%158=44%:%
%:%159=44%:%
%:%160=45%:%
%:%161=45%:%
%:%162=46%:%
%:%163=46%:%
%:%164=47%:%
%:%165=47%:%
%:%166=47%:%
%:%167=48%:%
%:%168=48%:%
%:%169=49%:%
%:%170=49%:%
%:%171=50%:%
%:%172=50%:%
%:%173=50%:%
%:%174=51%:%
%:%175=51%:%
%:%176=52%:%
%:%177=52%:%
%:%178=53%:%
%:%179=53%:%
%:%180=53%:%
%:%181=54%:%
%:%182=54%:%
%:%183=55%:%
%:%184=55%:%
%:%185=56%:%
%:%186=56%:%
%:%187=57%:%
%:%188=57%:%
%:%189=57%:%
%:%190=58%:%
%:%191=58%:%
%:%192=59%:%
%:%193=59%:%
%:%194=60%:%
%:%195=60%:%
%:%196=60%:%
%:%197=61%:%
%:%198=61%:%
%:%199=62%:%
%:%200=62%:%
%:%201=63%:%
%:%202=63%:%
%:%203=64%:%
%:%204=64%:%
%:%205=65%:%
%:%206=65%:%
%:%207=66%:%
%:%208=66%:%
%:%209=67%:%
%:%210=67%:%
%:%211=68%:%
%:%212=68%:%
%:%213=69%:%
%:%214=69%:%
%:%215=69%:%
%:%216=70%:%
%:%217=70%:%
%:%218=71%:%
%:%219=71%:%
%:%220=72%:%
%:%221=72%:%
%:%222=72%:%
%:%223=73%:%
%:%224=73%:%
%:%225=74%:%
%:%226=74%:%
%:%227=75%:%
%:%228=76%:%
%:%229=76%:%
%:%230=76%:%
%:%231=77%:%
%:%232=77%:%
%:%233=78%:%
%:%234=78%:%
%:%235=78%:%
%:%236=78%:%
%:%237=79%:%
%:%238=79%:%
%:%239=80%:%
%:%240=80%:%
%:%241=80%:%
%:%242=80%:%
%:%243=81%:%
%:%244=81%:%
%:%245=82%:%
%:%246=82%:%
%:%247=82%:%
%:%248=82%:%
%:%249=83%:%
%:%250=83%:%
%:%251=84%:%
%:%252=84%:%
%:%253=85%:%
%:%254=85%:%
%:%255=85%:%
%:%256=86%:%
%:%257=86%:%
%:%258=86%:%
%:%259=86%:%
%:%260=87%:%
%:%261=87%:%
%:%262=88%:%
%:%263=88%:%
%:%264=89%:%
%:%265=89%:%
%:%266=89%:%
%:%267=90%:%
%:%268=90%:%
%:%269=91%:%
%:%270=91%:%
%:%271=92%:%
%:%272=92%:%
%:%273=92%:%
%:%274=93%:%
%:%275=93%:%
%:%276=94%:%
%:%277=94%:%
%:%278=95%:%
%:%279=95%:%
%:%280=95%:%
%:%281=96%:%
%:%287=96%:%
%:%290=97%:%
%:%291=98%:%
%:%292=98%:%
%:%293=99%:%
%:%296=100%:%
%:%300=100%:%
%:%301=100%:%
%:%302=100%:%
%:%307=100%:%
%:%310=101%:%
%:%311=102%:%
%:%312=102%:%
%:%313=103%:%
%:%314=104%:%
%:%317=105%:%
%:%321=105%:%
%:%322=105%:%
%:%323=105%:%
%:%324=106%:%
%:%325=106%:%
%:%326=107%:%
%:%327=107%:%
%:%328=108%:%
%:%329=108%:%
%:%330=109%:%
%:%331=109%:%
%:%332=110%:%
%:%333=110%:%
%:%334=111%:%
%:%335=111%:%
%:%336=112%:%
%:%337=112%:%
%:%338=112%:%
%:%339=113%:%
%:%340=113%:%
%:%341=114%:%
%:%342=114%:%
%:%343=114%:%
%:%344=114%:%
%:%345=115%:%
%:%346=115%:%
%:%347=116%:%
%:%348=116%:%
%:%349=116%:%
%:%350=116%:%
%:%351=117%:%
%:%352=117%:%
%:%353=118%:%
%:%354=118%:%
%:%355=119%:%
%:%356=119%:%
%:%357=119%:%
%:%358=120%:%
%:%359=120%:%
%:%360=120%:%
%:%361=121%:%
%:%362=121%:%
%:%363=122%:%
%:%364=122%:%
%:%365=123%:%
%:%366=123%:%
%:%367=123%:%
%:%368=123%:%
%:%369=124%:%
%:%370=124%:%
%:%371=125%:%
%:%372=125%:%
%:%373=126%:%
%:%374=126%:%
%:%375=127%:%
%:%376=127%:%
%:%377=127%:%
%:%378=127%:%
%:%379=128%:%
%:%380=128%:%
%:%381=129%:%
%:%382=129%:%
%:%383=130%:%
%:%384=130%:%
%:%385=130%:%
%:%386=131%:%
%:%387=131%:%
%:%388=131%:%
%:%389=132%:%
%:%390=132%:%
%:%391=133%:%
%:%392=133%:%
%:%393=134%:%
%:%394=134%:%
%:%395=134%:%
%:%396=134%:%
%:%397=135%:%
%:%398=135%:%
%:%399=136%:%
%:%400=136%:%
%:%401=137%:%
%:%402=137%:%
%:%403=138%:%
%:%404=138%:%
%:%405=138%:%
%:%406=138%:%
%:%407=139%:%
%:%408=139%:%
%:%409=140%:%
%:%410=140%:%
%:%411=141%:%
%:%412=141%:%
%:%413=141%:%
%:%414=142%:%
%:%415=142%:%
%:%416=143%:%
%:%417=143%:%
%:%418=143%:%
%:%419=143%:%
%:%420=144%:%
%:%421=144%:%
%:%422=144%:%
%:%423=144%:%
%:%424=145%:%
%:%425=145%:%
%:%426=146%:%
%:%427=146%:%
%:%428=146%:%
%:%429=146%:%
%:%430=147%:%
%:%431=147%:%
%:%432=148%:%
%:%433=148%:%
%:%434=148%:%
%:%435=148%:%
%:%436=149%:%
%:%437=149%:%
%:%438=150%:%
%:%439=150%:%
%:%440=151%:%
%:%441=151%:%
%:%444=154%:%
%:%445=155%:%
%:%446=155%:%
%:%447=156%:%
%:%453=156%:%
%:%456=157%:%
%:%457=158%:%
%:%458=158%:%
%:%459=159%:%
%:%460=160%:%
%:%467=161%:%
%:%468=161%:%
%:%469=162%:%
%:%470=162%:%
%:%471=163%:%
%:%472=163%:%
%:%473=163%:%
%:%474=163%:%
%:%475=164%:%
%:%476=164%:%
%:%477=165%:%
%:%478=165%:%
%:%479=166%:%
%:%480=166%:%
%:%481=167%:%
%:%482=167%:%
%:%483=167%:%
%:%484=168%:%
%:%485=168%:%
%:%486=169%:%
%:%487=169%:%
%:%488=170%:%
%:%489=170%:%
%:%490=170%:%
%:%491=171%:%
%:%492=171%:%
%:%493=172%:%
%:%494=172%:%
%:%495=173%:%
%:%496=173%:%
%:%497=174%:%
%:%498=174%:%
%:%499=175%:%
%:%500=175%:%
%:%501=176%:%
%:%502=176%:%
%:%503=177%:%
%:%504=177%:%
%:%505=177%:%
%:%506=178%:%
%:%507=178%:%
%:%508=179%:%
%:%509=179%:%
%:%510=179%:%
%:%511=179%:%
%:%512=180%:%
%:%513=180%:%
%:%514=181%:%
%:%515=181%:%
%:%516=182%:%
%:%517=182%:%
%:%518=183%:%
%:%519=183%:%
%:%520=183%:%
%:%521=184%:%
%:%522=184%:%
%:%523=184%:%
%:%524=184%:%
%:%525=185%:%
%:%531=185%:%
%:%534=186%:%
%:%535=187%:%
%:%536=188%:%
%:%537=188%:%
%:%538=189%:%
%:%539=190%:%
%:%540=191%:%
%:%541=192%:%
%:%542=193%:%
%:%549=194%:%
%:%550=194%:%
%:%551=195%:%
%:%552=195%:%
%:%553=196%:%
%:%554=196%:%
%:%555=197%:%
%:%556=197%:%
%:%557=198%:%
%:%558=198%:%
%:%559=199%:%
%:%560=199%:%
%:%561=200%:%
%:%562=200%:%
%:%563=200%:%
%:%564=201%:%
%:%565=201%:%
%:%566=202%:%
%:%567=202%:%
%:%568=203%:%
%:%569=203%:%
%:%570=204%:%
%:%571=205%:%
%:%572=205%:%
%:%573=206%:%
%:%574=206%:%
%:%575=207%:%
%:%576=207%:%
%:%577=208%:%
%:%578=208%:%
%:%579=208%:%
%:%580=209%:%
%:%581=209%:%
%:%582=210%:%
%:%583=210%:%
%:%584=210%:%
%:%585=210%:%
%:%586=211%:%
%:%587=211%:%
%:%588=212%:%
%:%589=212%:%
%:%590=213%:%
%:%591=213%:%
%:%592=213%:%
%:%593=214%:%
%:%594=214%:%
%:%595=214%:%
%:%596=214%:%
%:%597=214%:%
%:%598=215%:%
%:%599=215%:%
%:%600=216%:%
%:%601=216%:%
%:%602=216%:%
%:%603=216%:%
%:%604=217%:%
%:%605=217%:%
%:%606=218%:%
%:%607=218%:%
%:%608=218%:%
%:%609=219%:%
%:%610=219%:%
%:%611=220%:%
%:%612=220%:%
%:%613=220%:%
%:%614=220%:%
%:%615=221%:%
%:%616=221%:%
%:%617=222%:%
%:%618=222%:%
%:%619=222%:%
%:%620=222%:%
%:%621=223%:%
%:%622=223%:%
%:%623=224%:%
%:%624=224%:%
%:%625=225%:%
%:%626=225%:%
%:%627=226%:%
%:%628=226%:%
%:%629=227%:%
%:%630=227%:%
%:%631=228%:%
%:%632=228%:%
%:%633=228%:%
%:%634=228%:%
%:%635=229%:%
%:%636=229%:%
%:%637=230%:%
%:%638=230%:%
%:%639=230%:%
%:%640=231%:%
%:%641=231%:%
%:%642=232%:%
%:%643=232%:%
%:%644=233%:%
%:%645=233%:%
%:%646=233%:%
%:%647=234%:%
%:%648=234%:%
%:%649=235%:%
%:%650=235%:%
%:%651=236%:%
%:%652=236%:%
%:%653=236%:%
%:%654=237%:%
%:%655=237%:%
%:%656=238%:%
%:%657=238%:%
%:%658=239%:%
%:%659=239%:%
%:%660=239%:%
%:%661=240%:%
%:%662=240%:%
%:%663=240%:%
%:%664=240%:%
%:%665=240%:%
%:%666=241%:%
%:%667=241%:%
%:%668=242%:%
%:%669=242%:%
%:%670=242%:%
%:%671=242%:%
%:%672=242%:%
%:%673=243%:%
%:%679=243%:%
%:%682=244%:%
%:%683=245%:%
%:%684=245%:%
%:%685=246%:%
%:%686=247%:%
%:%693=248%:%
%:%694=248%:%
%:%695=249%:%
%:%696=249%:%
%:%697=250%:%
%:%698=250%:%
%:%699=251%:%
%:%700=251%:%
%:%701=251%:%
%:%702=252%:%
%:%703=252%:%
%:%704=253%:%
%:%705=254%:%
%:%706=254%:%
%:%707=255%:%
%:%708=255%:%
%:%709=256%:%
%:%710=256%:%
%:%711=257%:%
%:%712=257%:%
%:%713=257%:%
%:%714=257%:%
%:%715=258%:%
%:%716=258%:%
%:%717=259%:%
%:%718=259%:%
%:%719=259%:%
%:%720=260%:%
%:%721=260%:%
%:%722=261%:%
%:%723=261%:%
%:%724=261%:%
%:%725=261%:%
%:%726=262%:%
%:%727=262%:%
%:%728=263%:%
%:%729=263%:%
%:%730=263%:%
%:%731=264%:%
%:%732=264%:%
%:%733=264%:%
%:%734=264%:%
%:%735=264%:%
%:%736=265%:%
%:%737=265%:%
%:%738=265%:%
%:%739=265%:%
%:%740=265%:%
%:%741=266%:%
%:%742=266%:%
%:%743=266%:%
%:%744=266%:%
%:%745=266%:%
%:%746=267%:%
%:%747=267%:%
%:%748=267%:%
%:%749=267%:%
%:%750=267%:%
%:%751=268%:%
%:%752=268%:%
%:%753=268%:%
%:%754=268%:%
%:%755=268%:%
%:%756=269%:%
%:%757=269%:%
%:%758=269%:%
%:%759=269%:%
%:%760=269%:%
%:%761=270%:%
%:%762=270%:%
%:%763=271%:%
%:%764=271%:%
%:%765=271%:%
%:%766=271%:%
%:%767=271%:%
%:%768=271%:%
%:%769=272%:%
%:%775=272%:%
%:%778=273%:%
%:%779=274%:%
%:%780=274%:%
%:%781=275%:%
%:%782=276%:%
%:%785=277%:%
%:%789=277%:%
%:%790=277%:%
%:%791=278%:%
%:%792=278%:%
%:%793=278%:%
%:%798=278%:%
%:%801=279%:%
%:%802=280%:%
%:%803=280%:%
%:%804=281%:%
%:%805=282%:%
%:%812=283%:%
%:%813=283%:%
%:%814=284%:%
%:%815=284%:%
%:%816=285%:%
%:%817=285%:%
%:%818=286%:%
%:%819=286%:%
%:%820=287%:%
%:%821=287%:%
%:%822=288%:%
%:%823=288%:%
%:%824=289%:%
%:%825=289%:%
%:%826=289%:%
%:%827=289%:%
%:%828=290%:%
%:%829=290%:%
%:%830=291%:%
%:%831=291%:%
%:%832=291%:%
%:%833=292%:%
%:%834=292%:%
%:%835=292%:%
%:%836=292%:%
%:%837=293%:%
%:%838=293%:%
%:%839=293%:%
%:%840=293%:%
%:%841=294%:%
%:%842=294%:%
%:%843=295%:%
%:%844=295%:%
%:%845=296%:%
%:%846=296%:%
%:%847=297%:%
%:%848=297%:%
%:%849=298%:%
%:%850=299%:%
%:%851=299%:%
%:%852=300%:%
%:%853=300%:%
%:%854=300%:%
%:%855=301%:%
%:%856=301%:%
%:%857=302%:%
%:%858=302%:%
%:%859=302%:%
%:%860=302%:%
%:%861=303%:%
%:%862=303%:%
%:%863=304%:%
%:%864=304%:%
%:%865=304%:%
%:%866=305%:%
%:%867=305%:%
%:%868=306%:%
%:%869=306%:%
%:%870=307%:%
%:%871=307%:%
%:%872=307%:%
%:%873=307%:%
%:%874=308%:%
%:%875=308%:%
%:%876=309%:%
%:%877=309%:%
%:%878=310%:%
%:%879=310%:%
%:%880=310%:%
%:%881=310%:%
%:%882=311%:%
%:%883=311%:%
%:%884=311%:%
%:%885=311%:%
%:%886=311%:%
%:%887=312%:%
%:%888=312%:%
%:%889=313%:%
%:%890=313%:%
%:%891=314%:%
%:%892=314%:%
%:%893=315%:%
%:%894=315%:%
%:%895=316%:%
%:%896=316%:%
%:%897=316%:%
%:%898=317%:%
%:%899=317%:%
%:%900=318%:%
%:%901=318%:%
%:%902=318%:%
%:%903=318%:%
%:%904=319%:%
%:%905=319%:%
%:%906=320%:%
%:%907=320%:%
%:%908=321%:%
%:%909=321%:%
%:%910=321%:%
%:%911=322%:%
%:%912=322%:%
%:%913=323%:%
%:%914=323%:%
%:%915=324%:%
%:%916=324%:%
%:%917=325%:%
%:%918=325%:%
%:%919=326%:%
%:%920=326%:%
%:%921=327%:%
%:%922=327%:%
%:%923=328%:%
%:%924=328%:%
%:%925=329%:%
%:%926=329%:%
%:%927=330%:%
%:%928=330%:%
%:%929=331%:%
%:%930=331%:%
%:%931=331%:%
%:%932=332%:%
%:%933=332%:%
%:%934=333%:%
%:%935=333%:%
%:%936=333%:%
%:%937=333%:%
%:%938=334%:%
%:%939=334%:%
%:%940=335%:%
%:%941=335%:%
%:%942=335%:%
%:%943=335%:%
%:%944=336%:%
%:%945=336%:%
%:%946=337%:%
%:%947=337%:%
%:%948=338%:%
%:%949=338%:%
%:%950=339%:%
%:%951=339%:%
%:%952=340%:%
%:%953=340%:%
%:%954=341%:%
%:%955=341%:%
%:%956=342%:%
%:%957=342%:%
%:%958=342%:%
%:%959=343%:%
%:%960=343%:%
%:%961=344%:%
%:%962=344%:%
%:%963=345%:%
%:%964=345%:%
%:%965=346%:%
%:%966=346%:%
%:%967=346%:%
%:%968=346%:%
%:%969=347%:%
%:%970=347%:%
%:%971=348%:%
%:%972=348%:%
%:%973=349%:%
%:%974=349%:%
%:%975=350%:%
%:%976=350%:%
%:%977=351%:%
%:%978=351%:%
%:%979=352%:%
%:%980=352%:%
%:%981=353%:%
%:%982=353%:%
%:%983=353%:%
%:%984=354%:%
%:%985=354%:%
%:%986=355%:%
%:%987=355%:%
%:%988=356%:%
%:%989=356%:%
%:%990=357%:%
%:%991=357%:%
%:%992=358%:%
%:%993=358%:%
%:%994=359%:%
%:%995=359%:%
%:%996=360%:%
%:%997=360%:%
%:%998=361%:%
%:%999=361%:%
%:%1000=361%:%
%:%1001=362%:%
%:%1002=362%:%
%:%1003=362%:%
%:%1004=362%:%
%:%1005=363%:%
%:%1006=363%:%
%:%1007=364%:%
%:%1008=364%:%
%:%1009=365%:%
%:%1010=365%:%
%:%1011=366%:%
%:%1012=366%:%
%:%1013=366%:%
%:%1014=366%:%
%:%1015=367%:%
%:%1016=367%:%
%:%1017=368%:%
%:%1018=368%:%
%:%1019=368%:%
%:%1020=368%:%
%:%1021=369%:%
%:%1022=369%:%
%:%1023=370%:%
%:%1024=370%:%
%:%1025=370%:%
%:%1026=370%:%
%:%1027=371%:%
%:%1028=371%:%
%:%1029=371%:%
%:%1030=372%:%
%:%1031=372%:%
%:%1032=373%:%
%:%1033=373%:%
%:%1034=374%:%
%:%1035=374%:%
%:%1036=374%:%
%:%1037=375%:%
%:%1038=375%:%
%:%1039=376%:%
%:%1040=376%:%
%:%1041=377%:%
%:%1042=377%:%
%:%1043=378%:%
%:%1044=378%:%
%:%1045=379%:%
%:%1046=379%:%
%:%1047=380%:%
%:%1048=380%:%
%:%1049=380%:%
%:%1050=381%:%
%:%1051=381%:%
%:%1052=382%:%
%:%1053=382%:%
%:%1054=382%:%
%:%1055=383%:%
%:%1056=383%:%
%:%1057=384%:%
%:%1058=384%:%
%:%1059=384%:%
%:%1060=385%:%
%:%1066=385%:%
%:%1069=386%:%
%:%1070=387%:%
%:%1071=387%:%
%:%1072=388%:%
%:%1073=389%:%
%:%1076=390%:%
%:%1080=390%:%
%:%1081=390%:%
%:%1082=391%:%
%:%1083=391%:%
%:%1088=391%:%
%:%1091=392%:%
%:%1092=393%:%
%:%1093=393%:%
%:%1094=394%:%
%:%1095=395%:%
%:%1096=396%:%
%:%1097=397%:%
%:%1098=398%:%
%:%1101=399%:%
%:%1105=399%:%
%:%1106=399%:%
%:%1107=400%:%
%:%1108=400%:%
%:%1109=401%:%
%:%1110=401%:%
%:%1111=402%:%
%:%1112=402%:%
%:%1113=403%:%
%:%1114=403%:%
%:%1115=404%:%
%:%1116=404%:%
%:%1118=406%:%
%:%1119=407%:%
%:%1120=407%:%
%:%1121=408%:%
%:%1122=408%:%
%:%1123=409%:%
%:%1124=409%:%
%:%1125=410%:%
%:%1126=410%:%
%:%1127=411%:%
%:%1133=411%:%
%:%1136=412%:%
%:%1137=413%:%
%:%1138=413%:%
%:%1139=414%:%
%:%1140=415%:%
%:%1141=416%:%
%:%1142=417%:%
%:%1143=417%:%
%:%1144=418%:%
%:%1145=419%:%
%:%1152=420%:%
%:%1153=420%:%
%:%1154=421%:%
%:%1155=421%:%
%:%1156=422%:%
%:%1157=422%:%
%:%1158=423%:%
%:%1159=423%:%
%:%1160=423%:%
%:%1161=424%:%
%:%1162=424%:%
%:%1163=425%:%
%:%1164=425%:%
%:%1165=426%:%
%:%1166=426%:%
%:%1167=426%:%
%:%1168=427%:%
%:%1174=427%:%
%:%1177=428%:%
%:%1178=429%:%
%:%1179=429%:%
%:%1180=430%:%
%:%1181=431%:%
%:%1188=432%:%
%:%1189=432%:%
%:%1190=433%:%
%:%1191=433%:%
%:%1192=434%:%
%:%1193=434%:%
%:%1194=435%:%
%:%1195=435%:%
%:%1196=436%:%
%:%1197=436%:%
%:%1198=437%:%
%:%1199=437%:%
%:%1200=438%:%
%:%1201=438%:%
%:%1202=439%:%
%:%1203=439%:%
%:%1204=439%:%
%:%1205=439%:%
%:%1206=440%:%
%:%1212=440%:%
%:%1215=441%:%
%:%1216=442%:%
%:%1217=442%:%
%:%1218=443%:%
%:%1219=444%:%
%:%1226=445%:%
%:%1227=445%:%
%:%1228=446%:%
%:%1229=446%:%
%:%1230=447%:%
%:%1231=447%:%
%:%1232=448%:%
%:%1233=448%:%
%:%1234=449%:%
%:%1235=449%:%
%:%1236=450%:%
%:%1237=450%:%
%:%1238=451%:%
%:%1239=451%:%
%:%1240=452%:%
%:%1241=452%:%
%:%1242=453%:%
%:%1243=453%:%
%:%1244=454%:%
%:%1245=454%:%
%:%1246=455%:%
%:%1247=456%:%
%:%1248=457%:%
%:%1249=458%:%
%:%1250=458%:%
%:%1251=458%:%
%:%1252=459%:%
%:%1253=459%:%
%:%1254=460%:%
%:%1255=460%:%
%:%1256=461%:%
%:%1257=461%:%
%:%1258=462%:%
%:%1259=462%:%
%:%1260=463%:%
%:%1261=463%:%
%:%1262=464%:%
%:%1263=464%:%
%:%1264=464%:%
%:%1265=465%:%
%:%1266=465%:%
%:%1267=465%:%
%:%1268=466%:%
%:%1269=466%:%
%:%1270=466%:%
%:%1271=467%:%
%:%1272=468%:%
%:%1273=469%:%
%:%1274=470%:%
%:%1275=471%:%
%:%1276=471%:%
%:%1277=472%:%
%:%1278=472%:%
%:%1279=473%:%
%:%1280=473%:%
%:%1281=473%:%
%:%1282=473%:%
%:%1283=473%:%
%:%1284=474%:%
%:%1290=474%:%
%:%1293=475%:%
%:%1294=476%:%
%:%1295=476%:%
%:%1296=477%:%
%:%1297=478%:%
%:%1298=479%:%
%:%1299=479%:%
%:%1300=480%:%
%:%1301=481%:%
%:%1304=482%:%
%:%1308=482%:%
%:%1309=482%:%
%:%1310=483%:%
%:%1311=483%:%
%:%1312=484%:%
%:%1313=484%:%
%:%1318=484%:%
%:%1321=485%:%
%:%1322=486%:%
%:%1323=486%:%
%:%1324=487%:%
%:%1325=488%:%
%:%1328=489%:%
%:%1332=489%:%
%:%1333=489%:%
%:%1334=490%:%
%:%1335=490%:%
%:%1336=491%:%
%:%1337=491%:%
%:%1342=491%:%
%:%1345=492%:%
%:%1346=493%:%
%:%1347=493%:%
%:%1348=494%:%
%:%1349=495%:%
%:%1350=496%:%
%:%1351=497%:%
%:%1352=498%:%
%:%1359=499%:%
%:%1360=499%:%
%:%1361=500%:%
%:%1362=500%:%
%:%1363=501%:%
%:%1364=501%:%
%:%1365=502%:%
%:%1366=503%:%
%:%1367=504%:%
%:%1368=504%:%
%:%1369=505%:%
%:%1370=505%:%
%:%1371=506%:%
%:%1372=506%:%
%:%1373=506%:%
%:%1374=506%:%
%:%1375=507%:%
%:%1376=507%:%
%:%1377=508%:%
%:%1378=508%:%
%:%1379=509%:%
%:%1380=509%:%
%:%1381=510%:%
%:%1382=511%:%
%:%1383=512%:%
%:%1384=513%:%
%:%1385=514%:%
%:%1386=515%:%
%:%1387=516%:%
%:%1388=516%:%
%:%1389=517%:%
%:%1395=517%:%
%:%1400=518%:%
%:%1405=519%:%

%
\begin{isabellebody}%
\setisabellecontext{Separation{\isacharunderscore}{\kern0pt}Axiom}%
%
\isadelimdocument
%
\endisadelimdocument
%
\isatagdocument
%
\isamarkupsection{The Axiom of Separation in $M[G]$%
}
\isamarkuptrue%
%
\endisatagdocument
{\isafolddocument}%
%
\isadelimdocument
%
\endisadelimdocument
%
\isadelimtheory
%
\endisadelimtheory
%
\isatagtheory
\isacommand{theory}\isamarkupfalse%
\ Separation{\isacharunderscore}{\kern0pt}Axiom\isanewline
\ \ \isakeyword{imports}\ Forcing{\isacharunderscore}{\kern0pt}Theorems\ Separation{\isacharunderscore}{\kern0pt}Rename\isanewline
\isakeyword{begin}%
\endisatagtheory
{\isafoldtheory}%
%
\isadelimtheory
\isanewline
%
\endisadelimtheory
\isanewline
\isacommand{context}\isamarkupfalse%
\ G{\isacharunderscore}{\kern0pt}generic\isanewline
\isakeyword{begin}\isanewline
\isanewline
\isacommand{lemma}\isamarkupfalse%
\ map{\isacharunderscore}{\kern0pt}val\ {\isacharcolon}{\kern0pt}\isanewline
\ \ \isakeyword{assumes}\ {\isachardoublequoteopen}env{\isasymin}list{\isacharparenleft}{\kern0pt}M{\isacharbrackleft}{\kern0pt}G{\isacharbrackright}{\kern0pt}{\isacharparenright}{\kern0pt}{\isachardoublequoteclose}\isanewline
\ \ \isakeyword{shows}\ {\isachardoublequoteopen}{\isasymexists}nenv{\isasymin}list{\isacharparenleft}{\kern0pt}M{\isacharparenright}{\kern0pt}{\isachardot}{\kern0pt}\ env\ {\isacharequal}{\kern0pt}\ map{\isacharparenleft}{\kern0pt}val{\isacharparenleft}{\kern0pt}G{\isacharparenright}{\kern0pt}{\isacharcomma}{\kern0pt}nenv{\isacharparenright}{\kern0pt}{\isachardoublequoteclose}\isanewline
%
\isadelimproof
\ \ %
\endisadelimproof
%
\isatagproof
\isacommand{using}\isamarkupfalse%
\ assms\isanewline
\ \ \isacommand{proof}\isamarkupfalse%
{\isacharparenleft}{\kern0pt}induct\ env{\isacharparenright}{\kern0pt}\isanewline
\ \ \ \ \isacommand{case}\isamarkupfalse%
\ Nil\isanewline
\ \ \ \ \isacommand{have}\isamarkupfalse%
\ {\isachardoublequoteopen}map{\isacharparenleft}{\kern0pt}val{\isacharparenleft}{\kern0pt}G{\isacharparenright}{\kern0pt}{\isacharcomma}{\kern0pt}Nil{\isacharparenright}{\kern0pt}\ {\isacharequal}{\kern0pt}\ Nil{\isachardoublequoteclose}\ \isacommand{by}\isamarkupfalse%
\ simp\isanewline
\ \ \ \ \isacommand{then}\isamarkupfalse%
\ \isacommand{show}\isamarkupfalse%
\ {\isacharquery}{\kern0pt}case\ \isacommand{by}\isamarkupfalse%
\ force\isanewline
\ \ \isacommand{next}\isamarkupfalse%
\isanewline
\ \ \ \ \isacommand{case}\isamarkupfalse%
\ {\isacharparenleft}{\kern0pt}Cons\ a\ l{\isacharparenright}{\kern0pt}\isanewline
\ \ \ \ \isacommand{then}\isamarkupfalse%
\ \isacommand{obtain}\isamarkupfalse%
\ a{\isacharprime}{\kern0pt}\ l{\isacharprime}{\kern0pt}\ \isakeyword{where}\isanewline
\ \ \ \ \ \ {\isachardoublequoteopen}l{\isacharprime}{\kern0pt}\ {\isasymin}\ list{\isacharparenleft}{\kern0pt}M{\isacharparenright}{\kern0pt}{\isachardoublequoteclose}\ {\isachardoublequoteopen}l{\isacharequal}{\kern0pt}map{\isacharparenleft}{\kern0pt}val{\isacharparenleft}{\kern0pt}G{\isacharparenright}{\kern0pt}{\isacharcomma}{\kern0pt}l{\isacharprime}{\kern0pt}{\isacharparenright}{\kern0pt}{\isachardoublequoteclose}\ {\isachardoublequoteopen}a\ {\isacharequal}{\kern0pt}\ val{\isacharparenleft}{\kern0pt}G{\isacharcomma}{\kern0pt}a{\isacharprime}{\kern0pt}{\isacharparenright}{\kern0pt}{\isachardoublequoteclose}\isanewline
\ \ \ \ \ \ {\isachardoublequoteopen}Cons{\isacharparenleft}{\kern0pt}a{\isacharcomma}{\kern0pt}l{\isacharparenright}{\kern0pt}\ {\isacharequal}{\kern0pt}\ map{\isacharparenleft}{\kern0pt}val{\isacharparenleft}{\kern0pt}G{\isacharparenright}{\kern0pt}{\isacharcomma}{\kern0pt}Cons{\isacharparenleft}{\kern0pt}a{\isacharprime}{\kern0pt}{\isacharcomma}{\kern0pt}l{\isacharprime}{\kern0pt}{\isacharparenright}{\kern0pt}{\isacharparenright}{\kern0pt}{\isachardoublequoteclose}\ {\isachardoublequoteopen}Cons{\isacharparenleft}{\kern0pt}a{\isacharprime}{\kern0pt}{\isacharcomma}{\kern0pt}l{\isacharprime}{\kern0pt}{\isacharparenright}{\kern0pt}\ {\isasymin}\ list{\isacharparenleft}{\kern0pt}M{\isacharparenright}{\kern0pt}{\isachardoublequoteclose}\isanewline
\ \ \ \ \ \ \isacommand{using}\isamarkupfalse%
\ {\isacartoucheopen}a{\isasymin}M{\isacharbrackleft}{\kern0pt}G{\isacharbrackright}{\kern0pt}{\isacartoucheclose}\ GenExtD\isanewline
\ \ \ \ \ \ \isacommand{by}\isamarkupfalse%
\ force\isanewline
\ \ \ \ \isacommand{then}\isamarkupfalse%
\ \isacommand{show}\isamarkupfalse%
\ {\isacharquery}{\kern0pt}case\ \isacommand{by}\isamarkupfalse%
\ force\isanewline
\isacommand{qed}\isamarkupfalse%
%
\endisatagproof
{\isafoldproof}%
%
\isadelimproof
\isanewline
%
\endisadelimproof
\isanewline
\isanewline
\isacommand{lemma}\isamarkupfalse%
\ Collect{\isacharunderscore}{\kern0pt}sats{\isacharunderscore}{\kern0pt}in{\isacharunderscore}{\kern0pt}MG\ {\isacharcolon}{\kern0pt}\isanewline
\ \ \isakeyword{assumes}\isanewline
\ \ \ \ {\isachardoublequoteopen}c{\isasymin}M{\isacharbrackleft}{\kern0pt}G{\isacharbrackright}{\kern0pt}{\isachardoublequoteclose}\isanewline
\ \ \ \ {\isachardoublequoteopen}{\isasymphi}\ {\isasymin}\ formula{\isachardoublequoteclose}\ {\isachardoublequoteopen}env{\isasymin}list{\isacharparenleft}{\kern0pt}M{\isacharbrackleft}{\kern0pt}G{\isacharbrackright}{\kern0pt}{\isacharparenright}{\kern0pt}{\isachardoublequoteclose}\ {\isachardoublequoteopen}arity{\isacharparenleft}{\kern0pt}{\isasymphi}{\isacharparenright}{\kern0pt}\ {\isasymle}\ {\isadigit{1}}\ {\isacharhash}{\kern0pt}{\isacharplus}{\kern0pt}\ length{\isacharparenleft}{\kern0pt}env{\isacharparenright}{\kern0pt}{\isachardoublequoteclose}\isanewline
\ \ \isakeyword{shows}\ \ \ \ \isanewline
\ \ \ \ {\isachardoublequoteopen}{\isacharbraceleft}{\kern0pt}x{\isasymin}c{\isachardot}{\kern0pt}\ {\isacharparenleft}{\kern0pt}M{\isacharbrackleft}{\kern0pt}G{\isacharbrackright}{\kern0pt}{\isacharcomma}{\kern0pt}\ {\isacharbrackleft}{\kern0pt}x{\isacharbrackright}{\kern0pt}\ {\isacharat}{\kern0pt}\ env\ {\isasymTurnstile}\ {\isasymphi}{\isacharparenright}{\kern0pt}{\isacharbraceright}{\kern0pt}{\isasymin}\ M{\isacharbrackleft}{\kern0pt}G{\isacharbrackright}{\kern0pt}{\isachardoublequoteclose}\isanewline
%
\isadelimproof
%
\endisadelimproof
%
\isatagproof
\isacommand{proof}\isamarkupfalse%
\ {\isacharminus}{\kern0pt}\ \ \isanewline
\ \ \isacommand{from}\isamarkupfalse%
\ {\isacartoucheopen}c{\isasymin}M{\isacharbrackleft}{\kern0pt}G{\isacharbrackright}{\kern0pt}{\isacartoucheclose}\isanewline
\ \ \isacommand{obtain}\isamarkupfalse%
\ {\isasympi}\ \isakeyword{where}\ {\isachardoublequoteopen}{\isasympi}\ {\isasymin}\ M{\isachardoublequoteclose}\ {\isachardoublequoteopen}val{\isacharparenleft}{\kern0pt}G{\isacharcomma}{\kern0pt}\ {\isasympi}{\isacharparenright}{\kern0pt}\ {\isacharequal}{\kern0pt}\ c{\isachardoublequoteclose}\isanewline
\ \ \ \ \isacommand{using}\isamarkupfalse%
\ GenExt{\isacharunderscore}{\kern0pt}def\ \isacommand{by}\isamarkupfalse%
\ auto\isanewline
\ \ \isacommand{let}\isamarkupfalse%
\ {\isacharquery}{\kern0pt}{\isasymchi}{\isacharequal}{\kern0pt}{\isachardoublequoteopen}And{\isacharparenleft}{\kern0pt}Member{\isacharparenleft}{\kern0pt}{\isadigit{0}}{\isacharcomma}{\kern0pt}{\isadigit{1}}\ {\isacharhash}{\kern0pt}{\isacharplus}{\kern0pt}\ length{\isacharparenleft}{\kern0pt}env{\isacharparenright}{\kern0pt}{\isacharparenright}{\kern0pt}{\isacharcomma}{\kern0pt}{\isasymphi}{\isacharparenright}{\kern0pt}{\isachardoublequoteclose}\ \isakeyword{and}\ {\isacharquery}{\kern0pt}Pl{\isadigit{1}}{\isacharequal}{\kern0pt}{\isachardoublequoteopen}{\isacharbrackleft}{\kern0pt}P{\isacharcomma}{\kern0pt}leq{\isacharcomma}{\kern0pt}one{\isacharbrackright}{\kern0pt}{\isachardoublequoteclose}\isanewline
\ \ \isacommand{let}\isamarkupfalse%
\ {\isacharquery}{\kern0pt}new{\isacharunderscore}{\kern0pt}form{\isacharequal}{\kern0pt}{\isachardoublequoteopen}sep{\isacharunderscore}{\kern0pt}ren{\isacharparenleft}{\kern0pt}length{\isacharparenleft}{\kern0pt}env{\isacharparenright}{\kern0pt}{\isacharcomma}{\kern0pt}forces{\isacharparenleft}{\kern0pt}{\isacharquery}{\kern0pt}{\isasymchi}{\isacharparenright}{\kern0pt}{\isacharparenright}{\kern0pt}{\isachardoublequoteclose}\isanewline
\ \ \isacommand{let}\isamarkupfalse%
\ {\isacharquery}{\kern0pt}{\isasympsi}{\isacharequal}{\kern0pt}{\isachardoublequoteopen}Exists{\isacharparenleft}{\kern0pt}Exists{\isacharparenleft}{\kern0pt}And{\isacharparenleft}{\kern0pt}pair{\isacharunderscore}{\kern0pt}fm{\isacharparenleft}{\kern0pt}{\isadigit{0}}{\isacharcomma}{\kern0pt}{\isadigit{1}}{\isacharcomma}{\kern0pt}{\isadigit{2}}{\isacharparenright}{\kern0pt}{\isacharcomma}{\kern0pt}{\isacharquery}{\kern0pt}new{\isacharunderscore}{\kern0pt}form{\isacharparenright}{\kern0pt}{\isacharparenright}{\kern0pt}{\isacharparenright}{\kern0pt}{\isachardoublequoteclose}\isanewline
\ \ \isacommand{note}\isamarkupfalse%
\ phi\ {\isacharequal}{\kern0pt}\ {\isacartoucheopen}{\isasymphi}{\isasymin}formula{\isacartoucheclose}\ {\isacartoucheopen}arity{\isacharparenleft}{\kern0pt}{\isasymphi}{\isacharparenright}{\kern0pt}\ {\isasymle}\ {\isadigit{1}}\ {\isacharhash}{\kern0pt}{\isacharplus}{\kern0pt}\ length{\isacharparenleft}{\kern0pt}env{\isacharparenright}{\kern0pt}{\isacartoucheclose}\ \isanewline
\ \ \isacommand{then}\isamarkupfalse%
\isanewline
\ \ \isacommand{have}\isamarkupfalse%
\ {\isachardoublequoteopen}{\isacharquery}{\kern0pt}{\isasymchi}{\isasymin}formula{\isachardoublequoteclose}\ \isacommand{by}\isamarkupfalse%
\ simp\isanewline
\ \ \isacommand{with}\isamarkupfalse%
\ {\isacartoucheopen}env{\isasymin}{\isacharunderscore}{\kern0pt}{\isacartoucheclose}\ phi\isanewline
\ \ \isacommand{have}\isamarkupfalse%
\ {\isachardoublequoteopen}arity{\isacharparenleft}{\kern0pt}{\isacharquery}{\kern0pt}{\isasymchi}{\isacharparenright}{\kern0pt}\ {\isasymle}\ {\isadigit{2}}{\isacharhash}{\kern0pt}{\isacharplus}{\kern0pt}length{\isacharparenleft}{\kern0pt}env{\isacharparenright}{\kern0pt}\ {\isachardoublequoteclose}\ \isanewline
\ \ \ \ \isacommand{using}\isamarkupfalse%
\ nat{\isacharunderscore}{\kern0pt}simp{\isacharunderscore}{\kern0pt}union\ leI\ \isacommand{by}\isamarkupfalse%
\ simp\isanewline
\ \ \isacommand{with}\isamarkupfalse%
\ {\isacartoucheopen}env{\isasymin}list{\isacharparenleft}{\kern0pt}{\isacharunderscore}{\kern0pt}{\isacharparenright}{\kern0pt}{\isacartoucheclose}\ phi\isanewline
\ \ \isacommand{have}\isamarkupfalse%
\ {\isachardoublequoteopen}arity{\isacharparenleft}{\kern0pt}forces{\isacharparenleft}{\kern0pt}{\isacharquery}{\kern0pt}{\isasymchi}{\isacharparenright}{\kern0pt}{\isacharparenright}{\kern0pt}\ {\isasymle}\ {\isadigit{6}}\ {\isacharhash}{\kern0pt}{\isacharplus}{\kern0pt}\ length{\isacharparenleft}{\kern0pt}env{\isacharparenright}{\kern0pt}{\isachardoublequoteclose}\isanewline
\ \ \ \ \isacommand{using}\isamarkupfalse%
\ \ arity{\isacharunderscore}{\kern0pt}forces{\isacharunderscore}{\kern0pt}le\ \isacommand{by}\isamarkupfalse%
\ simp\isanewline
\ \ \isacommand{then}\isamarkupfalse%
\isanewline
\ \ \isacommand{have}\isamarkupfalse%
\ {\isachardoublequoteopen}arity{\isacharparenleft}{\kern0pt}forces{\isacharparenleft}{\kern0pt}{\isacharquery}{\kern0pt}{\isasymchi}{\isacharparenright}{\kern0pt}{\isacharparenright}{\kern0pt}\ {\isasymle}\ {\isadigit{7}}\ {\isacharhash}{\kern0pt}{\isacharplus}{\kern0pt}\ length{\isacharparenleft}{\kern0pt}env{\isacharparenright}{\kern0pt}{\isachardoublequoteclose}\isanewline
\ \ \ \ \isacommand{using}\isamarkupfalse%
\ nat{\isacharunderscore}{\kern0pt}simp{\isacharunderscore}{\kern0pt}union\ arity{\isacharunderscore}{\kern0pt}forces\ leI\ \isacommand{by}\isamarkupfalse%
\ simp\isanewline
\ \ \isacommand{with}\isamarkupfalse%
\ {\isacartoucheopen}arity{\isacharparenleft}{\kern0pt}forces{\isacharparenleft}{\kern0pt}{\isacharquery}{\kern0pt}{\isasymchi}{\isacharparenright}{\kern0pt}{\isacharparenright}{\kern0pt}\ {\isasymle}{\isadigit{7}}\ {\isacharhash}{\kern0pt}{\isacharplus}{\kern0pt}\ {\isacharunderscore}{\kern0pt}{\isacartoucheclose}\ {\isacartoucheopen}env\ {\isasymin}\ {\isacharunderscore}{\kern0pt}{\isacartoucheclose}\ {\isacartoucheopen}{\isasymphi}\ {\isasymin}\ formula{\isacartoucheclose}\isanewline
\ \ \isacommand{have}\isamarkupfalse%
\ {\isachardoublequoteopen}arity{\isacharparenleft}{\kern0pt}{\isacharquery}{\kern0pt}new{\isacharunderscore}{\kern0pt}form{\isacharparenright}{\kern0pt}\ {\isasymle}\ {\isadigit{7}}\ {\isacharhash}{\kern0pt}{\isacharplus}{\kern0pt}\ length{\isacharparenleft}{\kern0pt}env{\isacharparenright}{\kern0pt}{\isachardoublequoteclose}\ {\isachardoublequoteopen}{\isacharquery}{\kern0pt}new{\isacharunderscore}{\kern0pt}form\ {\isasymin}\ formula{\isachardoublequoteclose}\isanewline
\ \ \ \ \isacommand{using}\isamarkupfalse%
\ arity{\isacharunderscore}{\kern0pt}rensep{\isacharbrackleft}{\kern0pt}OF\ definability{\isacharbrackleft}{\kern0pt}of\ {\isachardoublequoteopen}{\isacharquery}{\kern0pt}{\isasymchi}{\isachardoublequoteclose}{\isacharbrackright}{\kern0pt}{\isacharbrackright}{\kern0pt}\ \ definability{\isacharbrackleft}{\kern0pt}of\ {\isachardoublequoteopen}{\isacharquery}{\kern0pt}{\isasymchi}{\isachardoublequoteclose}{\isacharbrackright}{\kern0pt}\ type{\isacharunderscore}{\kern0pt}rensep\ \isanewline
\ \ \ \ \isacommand{by}\isamarkupfalse%
\ auto\isanewline
\ \ \isacommand{then}\isamarkupfalse%
\isanewline
\ \ \isacommand{have}\isamarkupfalse%
\ {\isachardoublequoteopen}pred{\isacharparenleft}{\kern0pt}pred{\isacharparenleft}{\kern0pt}arity{\isacharparenleft}{\kern0pt}{\isacharquery}{\kern0pt}new{\isacharunderscore}{\kern0pt}form{\isacharparenright}{\kern0pt}{\isacharparenright}{\kern0pt}{\isacharparenright}{\kern0pt}\ {\isasymle}\ {\isadigit{5}}\ {\isacharhash}{\kern0pt}{\isacharplus}{\kern0pt}\ length{\isacharparenleft}{\kern0pt}env{\isacharparenright}{\kern0pt}{\isachardoublequoteclose}\ {\isachardoublequoteopen}{\isacharquery}{\kern0pt}{\isasympsi}{\isasymin}formula{\isachardoublequoteclose}\isanewline
\ \ \ \ \isacommand{unfolding}\isamarkupfalse%
\ pair{\isacharunderscore}{\kern0pt}fm{\isacharunderscore}{\kern0pt}def\ upair{\isacharunderscore}{\kern0pt}fm{\isacharunderscore}{\kern0pt}def\ \isanewline
\ \ \ \ \isacommand{using}\isamarkupfalse%
\ nat{\isacharunderscore}{\kern0pt}simp{\isacharunderscore}{\kern0pt}union\ length{\isacharunderscore}{\kern0pt}type{\isacharbrackleft}{\kern0pt}OF\ {\isacartoucheopen}env{\isasymin}list{\isacharparenleft}{\kern0pt}M{\isacharbrackleft}{\kern0pt}G{\isacharbrackright}{\kern0pt}{\isacharparenright}{\kern0pt}{\isacartoucheclose}{\isacharbrackright}{\kern0pt}\ \isanewline
\ \ \ \ \ \ \ \ pred{\isacharunderscore}{\kern0pt}mono{\isacharbrackleft}{\kern0pt}OF\ {\isacharunderscore}{\kern0pt}\ pred{\isacharunderscore}{\kern0pt}mono{\isacharbrackleft}{\kern0pt}OF\ {\isacharunderscore}{\kern0pt}\ {\isacartoucheopen}arity{\isacharparenleft}{\kern0pt}{\isacharquery}{\kern0pt}new{\isacharunderscore}{\kern0pt}form{\isacharparenright}{\kern0pt}\ {\isasymle}\ {\isacharunderscore}{\kern0pt}{\isacartoucheclose}{\isacharbrackright}{\kern0pt}{\isacharbrackright}{\kern0pt}\isanewline
\ \ \ \ \isacommand{by}\isamarkupfalse%
\ auto\isanewline
\ \ \isacommand{with}\isamarkupfalse%
\ {\isacartoucheopen}arity{\isacharparenleft}{\kern0pt}{\isacharquery}{\kern0pt}new{\isacharunderscore}{\kern0pt}form{\isacharparenright}{\kern0pt}\ {\isasymle}\ {\isacharunderscore}{\kern0pt}{\isacartoucheclose}\ {\isacartoucheopen}{\isacharquery}{\kern0pt}new{\isacharunderscore}{\kern0pt}form\ {\isasymin}\ formula{\isacartoucheclose}\isanewline
\ \ \isacommand{have}\isamarkupfalse%
\ {\isachardoublequoteopen}arity{\isacharparenleft}{\kern0pt}{\isacharquery}{\kern0pt}{\isasympsi}{\isacharparenright}{\kern0pt}\ {\isasymle}\ {\isadigit{5}}\ {\isacharhash}{\kern0pt}{\isacharplus}{\kern0pt}\ length{\isacharparenleft}{\kern0pt}env{\isacharparenright}{\kern0pt}{\isachardoublequoteclose}\isanewline
\ \ \ \ \isacommand{unfolding}\isamarkupfalse%
\ pair{\isacharunderscore}{\kern0pt}fm{\isacharunderscore}{\kern0pt}def\ upair{\isacharunderscore}{\kern0pt}fm{\isacharunderscore}{\kern0pt}def\ \isanewline
\ \ \ \ \isacommand{using}\isamarkupfalse%
\ nat{\isacharunderscore}{\kern0pt}simp{\isacharunderscore}{\kern0pt}union\ arity{\isacharunderscore}{\kern0pt}forces\isanewline
\ \ \ \ \isacommand{by}\isamarkupfalse%
\ auto\isanewline
\ \ \isacommand{from}\isamarkupfalse%
\ {\isacartoucheopen}{\isasymphi}{\isasymin}formula{\isacartoucheclose}\isanewline
\ \ \isacommand{have}\isamarkupfalse%
\ {\isachardoublequoteopen}forces{\isacharparenleft}{\kern0pt}{\isacharquery}{\kern0pt}{\isasymchi}{\isacharparenright}{\kern0pt}\ {\isasymin}\ formula{\isachardoublequoteclose}\isanewline
\ \ \ \ \isacommand{using}\isamarkupfalse%
\ definability\ \isacommand{by}\isamarkupfalse%
\ simp\isanewline
\ \ \isacommand{from}\isamarkupfalse%
\ {\isacartoucheopen}{\isasympi}{\isasymin}M{\isacartoucheclose}\ P{\isacharunderscore}{\kern0pt}in{\isacharunderscore}{\kern0pt}M\ \isanewline
\ \ \isacommand{have}\isamarkupfalse%
\ {\isachardoublequoteopen}domain{\isacharparenleft}{\kern0pt}{\isasympi}{\isacharparenright}{\kern0pt}{\isasymin}M{\isachardoublequoteclose}\ {\isachardoublequoteopen}domain{\isacharparenleft}{\kern0pt}{\isasympi}{\isacharparenright}{\kern0pt}\ {\isasymtimes}\ P\ {\isasymin}\ M{\isachardoublequoteclose}\isanewline
\ \ \ \ \isacommand{by}\isamarkupfalse%
\ {\isacharparenleft}{\kern0pt}simp{\isacharunderscore}{\kern0pt}all\ flip{\isacharcolon}{\kern0pt}setclass{\isacharunderscore}{\kern0pt}iff{\isacharparenright}{\kern0pt}\isanewline
\ \ \isacommand{from}\isamarkupfalse%
\ {\isacartoucheopen}env\ {\isasymin}\ {\isacharunderscore}{\kern0pt}{\isacartoucheclose}\isanewline
\ \ \isacommand{obtain}\isamarkupfalse%
\ nenv\ \isakeyword{where}\ {\isachardoublequoteopen}nenv{\isasymin}list{\isacharparenleft}{\kern0pt}M{\isacharparenright}{\kern0pt}{\isachardoublequoteclose}\ {\isachardoublequoteopen}env\ {\isacharequal}{\kern0pt}\ map{\isacharparenleft}{\kern0pt}val{\isacharparenleft}{\kern0pt}G{\isacharparenright}{\kern0pt}{\isacharcomma}{\kern0pt}nenv{\isacharparenright}{\kern0pt}{\isachardoublequoteclose}\ {\isachardoublequoteopen}length{\isacharparenleft}{\kern0pt}nenv{\isacharparenright}{\kern0pt}\ {\isacharequal}{\kern0pt}\ length{\isacharparenleft}{\kern0pt}env{\isacharparenright}{\kern0pt}{\isachardoublequoteclose}\isanewline
\ \ \ \ \isacommand{using}\isamarkupfalse%
\ map{\isacharunderscore}{\kern0pt}val\ \isacommand{by}\isamarkupfalse%
\ auto\isanewline
\ \ \isacommand{from}\isamarkupfalse%
\ {\isacartoucheopen}arity{\isacharparenleft}{\kern0pt}{\isasymphi}{\isacharparenright}{\kern0pt}\ {\isasymle}\ {\isacharunderscore}{\kern0pt}{\isacartoucheclose}\ {\isacartoucheopen}env{\isasymin}{\isacharunderscore}{\kern0pt}{\isacartoucheclose}\ {\isacartoucheopen}{\isasymphi}{\isasymin}{\isacharunderscore}{\kern0pt}{\isacartoucheclose}\isanewline
\ \ \isacommand{have}\isamarkupfalse%
\ {\isachardoublequoteopen}arity{\isacharparenleft}{\kern0pt}{\isasymphi}{\isacharparenright}{\kern0pt}\ {\isasymle}\ {\isadigit{2}}{\isacharhash}{\kern0pt}{\isacharplus}{\kern0pt}\ length{\isacharparenleft}{\kern0pt}env{\isacharparenright}{\kern0pt}{\isachardoublequoteclose}\ \isanewline
\ \ \ \ \isacommand{using}\isamarkupfalse%
\ le{\isacharunderscore}{\kern0pt}trans{\isacharbrackleft}{\kern0pt}OF\ {\isacartoucheopen}arity{\isacharparenleft}{\kern0pt}{\isasymphi}{\isacharparenright}{\kern0pt}{\isasymle}{\isacharunderscore}{\kern0pt}{\isacartoucheclose}{\isacharbrackright}{\kern0pt}\ add{\isacharunderscore}{\kern0pt}le{\isacharunderscore}{\kern0pt}mono{\isacharbrackleft}{\kern0pt}of\ {\isadigit{1}}\ {\isadigit{2}}{\isacharcomma}{\kern0pt}OF\ {\isacharunderscore}{\kern0pt}\ le{\isacharunderscore}{\kern0pt}refl{\isacharbrackright}{\kern0pt}\ \isanewline
\ \ \ \ \isacommand{by}\isamarkupfalse%
\ auto\isanewline
\ \ \isacommand{with}\isamarkupfalse%
\ {\isacartoucheopen}nenv{\isasymin}{\isacharunderscore}{\kern0pt}{\isacartoucheclose}\ {\isacartoucheopen}env{\isasymin}{\isacharunderscore}{\kern0pt}{\isacartoucheclose}\ {\isacartoucheopen}{\isasympi}{\isasymin}M{\isacartoucheclose}\ {\isacartoucheopen}{\isasymphi}{\isasymin}{\isacharunderscore}{\kern0pt}{\isacartoucheclose}\ {\isacartoucheopen}length{\isacharparenleft}{\kern0pt}nenv{\isacharparenright}{\kern0pt}\ {\isacharequal}{\kern0pt}\ length{\isacharparenleft}{\kern0pt}env{\isacharparenright}{\kern0pt}{\isacartoucheclose}\isanewline
\ \ \isacommand{have}\isamarkupfalse%
\ {\isachardoublequoteopen}arity{\isacharparenleft}{\kern0pt}{\isacharquery}{\kern0pt}{\isasymchi}{\isacharparenright}{\kern0pt}\ {\isasymle}\ length{\isacharparenleft}{\kern0pt}{\isacharbrackleft}{\kern0pt}{\isasymtheta}{\isacharbrackright}{\kern0pt}\ {\isacharat}{\kern0pt}\ nenv\ {\isacharat}{\kern0pt}\ {\isacharbrackleft}{\kern0pt}{\isasympi}{\isacharbrackright}{\kern0pt}{\isacharparenright}{\kern0pt}{\isachardoublequoteclose}\ \isakeyword{for}\ {\isasymtheta}\ \isanewline
\ \ \ \ \isacommand{using}\isamarkupfalse%
\ nat{\isacharunderscore}{\kern0pt}union{\isacharunderscore}{\kern0pt}abs{\isadigit{2}}{\isacharbrackleft}{\kern0pt}OF\ {\isacharunderscore}{\kern0pt}\ {\isacharunderscore}{\kern0pt}\ {\isacartoucheopen}arity{\isacharparenleft}{\kern0pt}{\isasymphi}{\isacharparenright}{\kern0pt}\ {\isasymle}\ {\isadigit{2}}{\isacharhash}{\kern0pt}{\isacharplus}{\kern0pt}\ {\isacharunderscore}{\kern0pt}{\isacartoucheclose}{\isacharbrackright}{\kern0pt}\ nat{\isacharunderscore}{\kern0pt}simp{\isacharunderscore}{\kern0pt}union\ \isanewline
\ \ \ \ \isacommand{by}\isamarkupfalse%
\ simp\ \ \ \ \isanewline
\ \ \isacommand{note}\isamarkupfalse%
\ in{\isacharunderscore}{\kern0pt}M\ {\isacharequal}{\kern0pt}\ {\isacartoucheopen}{\isasympi}{\isasymin}M{\isacartoucheclose}\ {\isacartoucheopen}domain{\isacharparenleft}{\kern0pt}{\isasympi}{\isacharparenright}{\kern0pt}\ {\isasymtimes}\ P\ {\isasymin}\ M{\isacartoucheclose}\ \ P{\isacharunderscore}{\kern0pt}in{\isacharunderscore}{\kern0pt}M\ one{\isacharunderscore}{\kern0pt}in{\isacharunderscore}{\kern0pt}M\ leq{\isacharunderscore}{\kern0pt}in{\isacharunderscore}{\kern0pt}M\isanewline
\ \ \isacommand{{\isacharbraceleft}{\kern0pt}}\isamarkupfalse%
\isanewline
\ \ \ \ \isacommand{fix}\isamarkupfalse%
\ u\isanewline
\ \ \ \ \isacommand{assume}\isamarkupfalse%
\ {\isachardoublequoteopen}u\ {\isasymin}\ domain{\isacharparenleft}{\kern0pt}{\isasympi}{\isacharparenright}{\kern0pt}\ {\isasymtimes}\ P{\isachardoublequoteclose}\ {\isachardoublequoteopen}u\ {\isasymin}\ M{\isachardoublequoteclose}\isanewline
\ \ \ \ \isacommand{with}\isamarkupfalse%
\ in{\isacharunderscore}{\kern0pt}M\ {\isacartoucheopen}{\isacharquery}{\kern0pt}new{\isacharunderscore}{\kern0pt}form\ {\isasymin}\ formula{\isacartoucheclose}\ {\isacartoucheopen}{\isacharquery}{\kern0pt}{\isasympsi}{\isasymin}formula{\isacartoucheclose}\ {\isacartoucheopen}nenv\ {\isasymin}\ {\isacharunderscore}{\kern0pt}{\isacartoucheclose}\isanewline
\ \ \ \ \isacommand{have}\isamarkupfalse%
\ Eq{\isadigit{1}}{\isacharcolon}{\kern0pt}\ {\isachardoublequoteopen}{\isacharparenleft}{\kern0pt}M{\isacharcomma}{\kern0pt}\ {\isacharbrackleft}{\kern0pt}u{\isacharbrackright}{\kern0pt}\ {\isacharat}{\kern0pt}\ {\isacharquery}{\kern0pt}Pl{\isadigit{1}}\ {\isacharat}{\kern0pt}\ {\isacharbrackleft}{\kern0pt}{\isasympi}{\isacharbrackright}{\kern0pt}\ {\isacharat}{\kern0pt}\ nenv\ {\isasymTurnstile}\ {\isacharquery}{\kern0pt}{\isasympsi}{\isacharparenright}{\kern0pt}\ {\isasymlongleftrightarrow}\ \isanewline
\ \ \ \ \ \ \ \ \ \ \ \ \ \ \ \ \ \ \ \ \ \ \ \ {\isacharparenleft}{\kern0pt}{\isasymexists}{\isasymtheta}{\isasymin}M{\isachardot}{\kern0pt}\ {\isasymexists}p{\isasymin}P{\isachardot}{\kern0pt}\ u\ {\isacharequal}{\kern0pt}{\isasymlangle}{\isasymtheta}{\isacharcomma}{\kern0pt}p{\isasymrangle}\ {\isasymand}\ \isanewline
\ \ \ \ \ \ \ \ \ \ \ \ \ \ \ \ \ \ \ \ \ \ \ \ \ \ M{\isacharcomma}{\kern0pt}\ {\isacharbrackleft}{\kern0pt}{\isasymtheta}{\isacharcomma}{\kern0pt}p{\isacharcomma}{\kern0pt}u{\isacharbrackright}{\kern0pt}{\isacharat}{\kern0pt}{\isacharquery}{\kern0pt}Pl{\isadigit{1}}{\isacharat}{\kern0pt}{\isacharbrackleft}{\kern0pt}{\isasympi}{\isacharbrackright}{\kern0pt}\ {\isacharat}{\kern0pt}\ nenv\ {\isasymTurnstile}\ {\isacharquery}{\kern0pt}new{\isacharunderscore}{\kern0pt}form{\isacharparenright}{\kern0pt}{\isachardoublequoteclose}\isanewline
\ \ \ \ \ \ \isacommand{by}\isamarkupfalse%
\ {\isacharparenleft}{\kern0pt}auto\ simp\ add{\isacharcolon}{\kern0pt}\ transitivity{\isacharparenright}{\kern0pt}\isanewline
\ \ \ \ \isacommand{have}\isamarkupfalse%
\ Eq{\isadigit{3}}{\isacharcolon}{\kern0pt}\ {\isachardoublequoteopen}{\isasymtheta}{\isasymin}M\ {\isasymLongrightarrow}\ p{\isasymin}P\ {\isasymLongrightarrow}\isanewline
\ \ \ \ \ \ \ {\isacharparenleft}{\kern0pt}M{\isacharcomma}{\kern0pt}\ {\isacharbrackleft}{\kern0pt}{\isasymtheta}{\isacharcomma}{\kern0pt}p{\isacharcomma}{\kern0pt}u{\isacharbrackright}{\kern0pt}{\isacharat}{\kern0pt}{\isacharquery}{\kern0pt}Pl{\isadigit{1}}{\isacharat}{\kern0pt}{\isacharbrackleft}{\kern0pt}{\isasympi}{\isacharbrackright}{\kern0pt}{\isacharat}{\kern0pt}nenv\ {\isasymTurnstile}\ {\isacharquery}{\kern0pt}new{\isacharunderscore}{\kern0pt}form{\isacharparenright}{\kern0pt}\ {\isasymlongleftrightarrow}\isanewline
\ \ \ \ \ \ \ \ \ \ {\isacharparenleft}{\kern0pt}{\isasymforall}F{\isachardot}{\kern0pt}\ M{\isacharunderscore}{\kern0pt}generic{\isacharparenleft}{\kern0pt}F{\isacharparenright}{\kern0pt}\ {\isasymand}\ p\ {\isasymin}\ F\ {\isasymlongrightarrow}\ {\isacharparenleft}{\kern0pt}M{\isacharbrackleft}{\kern0pt}F{\isacharbrackright}{\kern0pt}{\isacharcomma}{\kern0pt}\ \ map{\isacharparenleft}{\kern0pt}val{\isacharparenleft}{\kern0pt}F{\isacharparenright}{\kern0pt}{\isacharcomma}{\kern0pt}\ {\isacharbrackleft}{\kern0pt}{\isasymtheta}{\isacharbrackright}{\kern0pt}\ {\isacharat}{\kern0pt}\ nenv{\isacharat}{\kern0pt}{\isacharbrackleft}{\kern0pt}{\isasympi}{\isacharbrackright}{\kern0pt}{\isacharparenright}{\kern0pt}\ {\isasymTurnstile}\ \ {\isacharquery}{\kern0pt}{\isasymchi}{\isacharparenright}{\kern0pt}{\isacharparenright}{\kern0pt}{\isachardoublequoteclose}\ \isanewline
\ \ \ \ \ \ \isakeyword{for}\ {\isasymtheta}\ p\ \isanewline
\ \ \ \ \isacommand{proof}\isamarkupfalse%
\ {\isacharminus}{\kern0pt}\isanewline
\ \ \ \ \ \ \isacommand{fix}\isamarkupfalse%
\ p\ {\isasymtheta}\ \isanewline
\ \ \ \ \ \ \isacommand{assume}\isamarkupfalse%
\ {\isachardoublequoteopen}{\isasymtheta}\ {\isasymin}\ M{\isachardoublequoteclose}\ {\isachardoublequoteopen}p{\isasymin}P{\isachardoublequoteclose}\isanewline
\ \ \ \ \ \ \isacommand{then}\isamarkupfalse%
\ \isanewline
\ \ \ \ \ \ \isacommand{have}\isamarkupfalse%
\ {\isachardoublequoteopen}p{\isasymin}M{\isachardoublequoteclose}\ \isacommand{using}\isamarkupfalse%
\ P{\isacharunderscore}{\kern0pt}in{\isacharunderscore}{\kern0pt}M\ \isacommand{by}\isamarkupfalse%
\ {\isacharparenleft}{\kern0pt}simp\ add{\isacharcolon}{\kern0pt}\ transitivity{\isacharparenright}{\kern0pt}\isanewline
\ \ \ \ \ \ \isacommand{note}\isamarkupfalse%
\ in{\isacharunderscore}{\kern0pt}M{\isacharprime}{\kern0pt}\ {\isacharequal}{\kern0pt}\ in{\isacharunderscore}{\kern0pt}M\ {\isacartoucheopen}{\isasymtheta}\ {\isasymin}\ M{\isacartoucheclose}\ {\isacartoucheopen}p{\isasymin}M{\isacartoucheclose}\ {\isacartoucheopen}u\ {\isasymin}\ domain{\isacharparenleft}{\kern0pt}{\isasympi}{\isacharparenright}{\kern0pt}\ {\isasymtimes}\ P{\isacartoucheclose}\ {\isacartoucheopen}u\ {\isasymin}\ M{\isacartoucheclose}\ {\isacartoucheopen}nenv{\isasymin}{\isacharunderscore}{\kern0pt}{\isacartoucheclose}\isanewline
\ \ \ \ \ \ \isacommand{then}\isamarkupfalse%
\ \isanewline
\ \ \ \ \ \ \isacommand{have}\isamarkupfalse%
\ {\isachardoublequoteopen}{\isacharbrackleft}{\kern0pt}{\isasymtheta}{\isacharcomma}{\kern0pt}u{\isacharbrackright}{\kern0pt}\ {\isasymin}\ list{\isacharparenleft}{\kern0pt}M{\isacharparenright}{\kern0pt}{\isachardoublequoteclose}\ \isacommand{by}\isamarkupfalse%
\ simp\isanewline
\ \ \ \ \ \ \isacommand{let}\isamarkupfalse%
\ {\isacharquery}{\kern0pt}env{\isacharequal}{\kern0pt}{\isachardoublequoteopen}{\isacharbrackleft}{\kern0pt}p{\isacharbrackright}{\kern0pt}{\isacharat}{\kern0pt}{\isacharquery}{\kern0pt}Pl{\isadigit{1}}{\isacharat}{\kern0pt}{\isacharbrackleft}{\kern0pt}{\isasymtheta}{\isacharbrackright}{\kern0pt}\ {\isacharat}{\kern0pt}\ nenv\ {\isacharat}{\kern0pt}\ {\isacharbrackleft}{\kern0pt}{\isasympi}{\isacharcomma}{\kern0pt}u{\isacharbrackright}{\kern0pt}{\isachardoublequoteclose}\isanewline
\ \ \ \ \ \ \isacommand{let}\isamarkupfalse%
\ {\isacharquery}{\kern0pt}new{\isacharunderscore}{\kern0pt}env{\isacharequal}{\kern0pt}{\isachardoublequoteopen}\ {\isacharbrackleft}{\kern0pt}{\isasymtheta}{\isacharcomma}{\kern0pt}p{\isacharcomma}{\kern0pt}u{\isacharcomma}{\kern0pt}P{\isacharcomma}{\kern0pt}leq{\isacharcomma}{\kern0pt}one{\isacharcomma}{\kern0pt}{\isasympi}{\isacharbrackright}{\kern0pt}\ {\isacharat}{\kern0pt}\ nenv{\isachardoublequoteclose}\isanewline
\ \ \ \ \ \ \isacommand{let}\isamarkupfalse%
\ {\isacharquery}{\kern0pt}{\isasympsi}{\isacharequal}{\kern0pt}{\isachardoublequoteopen}Exists{\isacharparenleft}{\kern0pt}Exists{\isacharparenleft}{\kern0pt}And{\isacharparenleft}{\kern0pt}pair{\isacharunderscore}{\kern0pt}fm{\isacharparenleft}{\kern0pt}{\isadigit{0}}{\isacharcomma}{\kern0pt}{\isadigit{1}}{\isacharcomma}{\kern0pt}{\isadigit{2}}{\isacharparenright}{\kern0pt}{\isacharcomma}{\kern0pt}{\isacharquery}{\kern0pt}new{\isacharunderscore}{\kern0pt}form{\isacharparenright}{\kern0pt}{\isacharparenright}{\kern0pt}{\isacharparenright}{\kern0pt}{\isachardoublequoteclose}\isanewline
\ \ \ \ \ \ \isacommand{have}\isamarkupfalse%
\ {\isachardoublequoteopen}{\isacharbrackleft}{\kern0pt}{\isasymtheta}{\isacharcomma}{\kern0pt}\ p{\isacharcomma}{\kern0pt}\ u{\isacharcomma}{\kern0pt}\ {\isasympi}{\isacharcomma}{\kern0pt}\ leq{\isacharcomma}{\kern0pt}\ one{\isacharcomma}{\kern0pt}\ {\isasympi}{\isacharbrackright}{\kern0pt}\ {\isasymin}\ list{\isacharparenleft}{\kern0pt}M{\isacharparenright}{\kern0pt}{\isachardoublequoteclose}\ \isanewline
\ \ \ \ \ \ \ \ \isacommand{using}\isamarkupfalse%
\ in{\isacharunderscore}{\kern0pt}M{\isacharprime}{\kern0pt}\ \isacommand{by}\isamarkupfalse%
\ simp\isanewline
\ \ \ \ \ \ \isacommand{have}\isamarkupfalse%
\ {\isachardoublequoteopen}{\isacharquery}{\kern0pt}{\isasymchi}\ {\isasymin}\ formula{\isachardoublequoteclose}\ {\isachardoublequoteopen}forces{\isacharparenleft}{\kern0pt}{\isacharquery}{\kern0pt}{\isasymchi}{\isacharparenright}{\kern0pt}{\isasymin}\ formula{\isachardoublequoteclose}\ \ \isanewline
\ \ \ \ \ \ \ \ \isacommand{using}\isamarkupfalse%
\ phi\ \isacommand{by}\isamarkupfalse%
\ simp{\isacharunderscore}{\kern0pt}all\isanewline
\ \ \ \ \ \ \isacommand{from}\isamarkupfalse%
\ in{\isacharunderscore}{\kern0pt}M{\isacharprime}{\kern0pt}\ \isanewline
\ \ \ \ \ \ \isacommand{have}\isamarkupfalse%
\ {\isachardoublequoteopen}{\isacharquery}{\kern0pt}Pl{\isadigit{1}}\ {\isasymin}\ list{\isacharparenleft}{\kern0pt}M{\isacharparenright}{\kern0pt}{\isachardoublequoteclose}\ \isacommand{by}\isamarkupfalse%
\ simp\isanewline
\ \ \ \ \ \ \isacommand{from}\isamarkupfalse%
\ in{\isacharunderscore}{\kern0pt}M{\isacharprime}{\kern0pt}\ \isacommand{have}\isamarkupfalse%
\ {\isachardoublequoteopen}{\isacharquery}{\kern0pt}env\ {\isasymin}\ list{\isacharparenleft}{\kern0pt}M{\isacharparenright}{\kern0pt}{\isachardoublequoteclose}\ \isacommand{by}\isamarkupfalse%
\ simp\isanewline
\ \ \ \ \ \ \isacommand{have}\isamarkupfalse%
\ Eq{\isadigit{1}}{\isacharprime}{\kern0pt}{\isacharcolon}{\kern0pt}\ {\isachardoublequoteopen}{\isacharquery}{\kern0pt}new{\isacharunderscore}{\kern0pt}env\ {\isasymin}\ list{\isacharparenleft}{\kern0pt}M{\isacharparenright}{\kern0pt}{\isachardoublequoteclose}\ \isacommand{using}\isamarkupfalse%
\ in{\isacharunderscore}{\kern0pt}M{\isacharprime}{\kern0pt}\ \ \isacommand{by}\isamarkupfalse%
\ simp\ \isanewline
\ \ \ \ \ \ \isacommand{then}\isamarkupfalse%
\ \isanewline
\ \ \ \ \ \ \isacommand{have}\isamarkupfalse%
\ {\isachardoublequoteopen}{\isacharparenleft}{\kern0pt}M{\isacharcomma}{\kern0pt}\ {\isacharbrackleft}{\kern0pt}{\isasymtheta}{\isacharcomma}{\kern0pt}p{\isacharcomma}{\kern0pt}u{\isacharbrackright}{\kern0pt}{\isacharat}{\kern0pt}{\isacharquery}{\kern0pt}Pl{\isadigit{1}}{\isacharat}{\kern0pt}{\isacharbrackleft}{\kern0pt}{\isasympi}{\isacharbrackright}{\kern0pt}\ {\isacharat}{\kern0pt}\ nenv\ {\isasymTurnstile}\ {\isacharquery}{\kern0pt}new{\isacharunderscore}{\kern0pt}form{\isacharparenright}{\kern0pt}\ {\isasymlongleftrightarrow}\ {\isacharparenleft}{\kern0pt}M{\isacharcomma}{\kern0pt}\ {\isacharquery}{\kern0pt}new{\isacharunderscore}{\kern0pt}env\ {\isasymTurnstile}\ {\isacharquery}{\kern0pt}new{\isacharunderscore}{\kern0pt}form{\isacharparenright}{\kern0pt}{\isachardoublequoteclose}\isanewline
\ \ \ \ \ \ \ \ \isacommand{by}\isamarkupfalse%
\ simp\isanewline
\ \ \ \ \ \ \isacommand{from}\isamarkupfalse%
\ in{\isacharunderscore}{\kern0pt}M{\isacharprime}{\kern0pt}\ {\isacartoucheopen}env\ {\isasymin}\ {\isacharunderscore}{\kern0pt}{\isacartoucheclose}\ Eq{\isadigit{1}}{\isacharprime}{\kern0pt}\ {\isacartoucheopen}length{\isacharparenleft}{\kern0pt}nenv{\isacharparenright}{\kern0pt}\ {\isacharequal}{\kern0pt}\ length{\isacharparenleft}{\kern0pt}env{\isacharparenright}{\kern0pt}{\isacartoucheclose}\ \isanewline
\ \ \ \ \ \ \ \ {\isacartoucheopen}arity{\isacharparenleft}{\kern0pt}forces{\isacharparenleft}{\kern0pt}{\isacharquery}{\kern0pt}{\isasymchi}{\isacharparenright}{\kern0pt}{\isacharparenright}{\kern0pt}\ {\isasymle}\ {\isadigit{7}}\ {\isacharhash}{\kern0pt}{\isacharplus}{\kern0pt}\ length{\isacharparenleft}{\kern0pt}env{\isacharparenright}{\kern0pt}{\isacartoucheclose}\ {\isacartoucheopen}forces{\isacharparenleft}{\kern0pt}{\isacharquery}{\kern0pt}{\isasymchi}{\isacharparenright}{\kern0pt}{\isasymin}\ formula{\isacartoucheclose}\isanewline
\ \ \ \ \ \ \ \ {\isacartoucheopen}{\isacharbrackleft}{\kern0pt}{\isasymtheta}{\isacharcomma}{\kern0pt}\ p{\isacharcomma}{\kern0pt}\ u{\isacharcomma}{\kern0pt}\ {\isasympi}{\isacharcomma}{\kern0pt}\ leq{\isacharcomma}{\kern0pt}\ one{\isacharcomma}{\kern0pt}\ {\isasympi}{\isacharbrackright}{\kern0pt}\ {\isasymin}\ list{\isacharparenleft}{\kern0pt}M{\isacharparenright}{\kern0pt}{\isacartoucheclose}\ \isanewline
\ \ \ \ \ \ \isacommand{have}\isamarkupfalse%
\ {\isachardoublequoteopen}{\isachardot}{\kern0pt}{\isachardot}{\kern0pt}{\isachardot}{\kern0pt}\ {\isasymlongleftrightarrow}\ M{\isacharcomma}{\kern0pt}\ {\isacharquery}{\kern0pt}env\ {\isasymTurnstile}\ forces{\isacharparenleft}{\kern0pt}{\isacharquery}{\kern0pt}{\isasymchi}{\isacharparenright}{\kern0pt}{\isachardoublequoteclose}\isanewline
\ \ \ \ \ \ \ \ \isacommand{using}\isamarkupfalse%
\ sepren{\isacharunderscore}{\kern0pt}action{\isacharbrackleft}{\kern0pt}of\ {\isachardoublequoteopen}forces{\isacharparenleft}{\kern0pt}{\isacharquery}{\kern0pt}{\isasymchi}{\isacharparenright}{\kern0pt}{\isachardoublequoteclose}\ \ {\isachardoublequoteopen}nenv{\isachardoublequoteclose}{\isacharcomma}{\kern0pt}OF\ {\isacharunderscore}{\kern0pt}\ {\isacharunderscore}{\kern0pt}\ {\isacartoucheopen}nenv{\isasymin}list{\isacharparenleft}{\kern0pt}M{\isacharparenright}{\kern0pt}{\isacartoucheclose}{\isacharbrackright}{\kern0pt}\ \isanewline
\ \ \ \ \ \ \ \ \isacommand{by}\isamarkupfalse%
\ simp\isanewline
\ \ \ \ \ \ \isacommand{also}\isamarkupfalse%
\ \isacommand{from}\isamarkupfalse%
\ in{\isacharunderscore}{\kern0pt}M{\isacharprime}{\kern0pt}\isanewline
\ \ \ \ \ \ \isacommand{have}\isamarkupfalse%
\ {\isachardoublequoteopen}{\isachardot}{\kern0pt}{\isachardot}{\kern0pt}{\isachardot}{\kern0pt}\ {\isasymlongleftrightarrow}\ M{\isacharcomma}{\kern0pt}\ \ {\isacharparenleft}{\kern0pt}{\isacharbrackleft}{\kern0pt}p{\isacharcomma}{\kern0pt}P{\isacharcomma}{\kern0pt}\ leq{\isacharcomma}{\kern0pt}\ one{\isacharcomma}{\kern0pt}{\isasymtheta}{\isacharbrackright}{\kern0pt}{\isacharat}{\kern0pt}nenv{\isacharat}{\kern0pt}\ {\isacharbrackleft}{\kern0pt}{\isasympi}{\isacharbrackright}{\kern0pt}{\isacharparenright}{\kern0pt}{\isacharat}{\kern0pt}{\isacharbrackleft}{\kern0pt}u{\isacharbrackright}{\kern0pt}\ {\isasymTurnstile}\ forces{\isacharparenleft}{\kern0pt}{\isacharquery}{\kern0pt}{\isasymchi}{\isacharparenright}{\kern0pt}{\isachardoublequoteclose}\ \isanewline
\ \ \ \ \ \ \ \ \isacommand{using}\isamarkupfalse%
\ app{\isacharunderscore}{\kern0pt}assoc\ \isacommand{by}\isamarkupfalse%
\ simp\isanewline
\ \ \ \ \ \ \isacommand{also}\isamarkupfalse%
\ \isanewline
\ \ \ \ \ \ \isacommand{from}\isamarkupfalse%
\ in{\isacharunderscore}{\kern0pt}M{\isacharprime}{\kern0pt}\ {\isacartoucheopen}env{\isasymin}{\isacharunderscore}{\kern0pt}{\isacartoucheclose}\ phi\ {\isacartoucheopen}length{\isacharparenleft}{\kern0pt}nenv{\isacharparenright}{\kern0pt}\ {\isacharequal}{\kern0pt}\ length{\isacharparenleft}{\kern0pt}env{\isacharparenright}{\kern0pt}{\isacartoucheclose}\isanewline
\ \ \ \ \ \ \ \ {\isacartoucheopen}arity{\isacharparenleft}{\kern0pt}forces{\isacharparenleft}{\kern0pt}{\isacharquery}{\kern0pt}{\isasymchi}{\isacharparenright}{\kern0pt}{\isacharparenright}{\kern0pt}\ {\isasymle}\ {\isadigit{6}}\ {\isacharhash}{\kern0pt}{\isacharplus}{\kern0pt}\ length{\isacharparenleft}{\kern0pt}env{\isacharparenright}{\kern0pt}{\isacartoucheclose}\ {\isacartoucheopen}forces{\isacharparenleft}{\kern0pt}{\isacharquery}{\kern0pt}{\isasymchi}{\isacharparenright}{\kern0pt}{\isasymin}formula{\isacartoucheclose}\isanewline
\ \ \ \ \ \ \isacommand{have}\isamarkupfalse%
\ {\isachardoublequoteopen}{\isachardot}{\kern0pt}{\isachardot}{\kern0pt}{\isachardot}{\kern0pt}\ {\isasymlongleftrightarrow}\ M{\isacharcomma}{\kern0pt}\ \ {\isacharbrackleft}{\kern0pt}p{\isacharcomma}{\kern0pt}P{\isacharcomma}{\kern0pt}\ leq{\isacharcomma}{\kern0pt}\ one{\isacharcomma}{\kern0pt}{\isasymtheta}{\isacharbrackright}{\kern0pt}{\isacharat}{\kern0pt}\ nenv\ {\isacharat}{\kern0pt}\ {\isacharbrackleft}{\kern0pt}{\isasympi}{\isacharbrackright}{\kern0pt}\ {\isasymTurnstile}\ forces{\isacharparenleft}{\kern0pt}{\isacharquery}{\kern0pt}{\isasymchi}{\isacharparenright}{\kern0pt}{\isachardoublequoteclose}\ \ \ \ \ \ \ \ \isanewline
\ \ \ \ \ \ \ \ \isacommand{by}\isamarkupfalse%
\ {\isacharparenleft}{\kern0pt}rule{\isacharunderscore}{\kern0pt}tac\ arity{\isacharunderscore}{\kern0pt}sats{\isacharunderscore}{\kern0pt}iff{\isacharcomma}{\kern0pt}auto{\isacharparenright}{\kern0pt}\isanewline
\ \ \ \ \ \ \isacommand{also}\isamarkupfalse%
\ \isanewline
\ \ \ \ \ \ \isacommand{from}\isamarkupfalse%
\ {\isacartoucheopen}arity{\isacharparenleft}{\kern0pt}forces{\isacharparenleft}{\kern0pt}{\isacharquery}{\kern0pt}{\isasymchi}{\isacharparenright}{\kern0pt}{\isacharparenright}{\kern0pt}\ {\isasymle}\ {\isadigit{6}}\ {\isacharhash}{\kern0pt}{\isacharplus}{\kern0pt}\ length{\isacharparenleft}{\kern0pt}env{\isacharparenright}{\kern0pt}{\isacartoucheclose}\ {\isacartoucheopen}forces{\isacharparenleft}{\kern0pt}{\isacharquery}{\kern0pt}{\isasymchi}{\isacharparenright}{\kern0pt}{\isasymin}formula{\isacartoucheclose}\ in{\isacharunderscore}{\kern0pt}M{\isacharprime}{\kern0pt}\ phi\ \isanewline
\ \ \ \ \ \ \isacommand{have}\isamarkupfalse%
\ {\isachardoublequoteopen}\ {\isachardot}{\kern0pt}{\isachardot}{\kern0pt}{\isachardot}{\kern0pt}\ {\isasymlongleftrightarrow}\ {\isacharparenleft}{\kern0pt}{\isasymforall}F{\isachardot}{\kern0pt}\ M{\isacharunderscore}{\kern0pt}generic{\isacharparenleft}{\kern0pt}F{\isacharparenright}{\kern0pt}\ {\isasymand}\ p\ {\isasymin}\ F\ {\isasymlongrightarrow}\ \isanewline
\ \ \ \ \ \ \ \ \ \ \ \ \ \ \ \ \ \ \ \ \ \ \ \ \ \ \ M{\isacharbrackleft}{\kern0pt}F{\isacharbrackright}{\kern0pt}{\isacharcomma}{\kern0pt}\ \ map{\isacharparenleft}{\kern0pt}val{\isacharparenleft}{\kern0pt}F{\isacharparenright}{\kern0pt}{\isacharcomma}{\kern0pt}\ {\isacharbrackleft}{\kern0pt}{\isasymtheta}{\isacharbrackright}{\kern0pt}\ {\isacharat}{\kern0pt}\ nenv\ {\isacharat}{\kern0pt}\ {\isacharbrackleft}{\kern0pt}{\isasympi}{\isacharbrackright}{\kern0pt}{\isacharparenright}{\kern0pt}\ {\isasymTurnstile}\ \ {\isacharquery}{\kern0pt}{\isasymchi}{\isacharparenright}{\kern0pt}{\isachardoublequoteclose}\isanewline
\ \ \ \ \ \ \ \ \isacommand{using}\isamarkupfalse%
\ \ definition{\isacharunderscore}{\kern0pt}of{\isacharunderscore}{\kern0pt}forcing\ \isanewline
\ \ \ \ \ \ \isacommand{proof}\isamarkupfalse%
\ {\isacharparenleft}{\kern0pt}intro\ iffI{\isacharparenright}{\kern0pt}\isanewline
\ \ \ \ \ \ \ \ \isacommand{assume}\isamarkupfalse%
\ a{\isadigit{1}}{\isacharcolon}{\kern0pt}\ {\isachardoublequoteopen}M{\isacharcomma}{\kern0pt}\ \ {\isacharbrackleft}{\kern0pt}p{\isacharcomma}{\kern0pt}P{\isacharcomma}{\kern0pt}\ leq{\isacharcomma}{\kern0pt}\ one{\isacharcomma}{\kern0pt}{\isasymtheta}{\isacharbrackright}{\kern0pt}\ {\isacharat}{\kern0pt}\ nenv\ {\isacharat}{\kern0pt}\ {\isacharbrackleft}{\kern0pt}{\isasympi}{\isacharbrackright}{\kern0pt}\ {\isasymTurnstile}\ \ forces{\isacharparenleft}{\kern0pt}{\isacharquery}{\kern0pt}{\isasymchi}{\isacharparenright}{\kern0pt}{\isachardoublequoteclose}\isanewline
\ \ \ \ \ \ \ \ \isacommand{note}\isamarkupfalse%
\ definition{\isacharunderscore}{\kern0pt}of{\isacharunderscore}{\kern0pt}forcing\ {\isacartoucheopen}arity{\isacharparenleft}{\kern0pt}{\isasymphi}{\isacharparenright}{\kern0pt}{\isasymle}\ {\isadigit{1}}{\isacharhash}{\kern0pt}{\isacharplus}{\kern0pt}{\isacharunderscore}{\kern0pt}{\isacartoucheclose}\isanewline
\ \ \ \ \ \ \ \ \isacommand{with}\isamarkupfalse%
\ {\isacartoucheopen}nenv{\isasymin}{\isacharunderscore}{\kern0pt}{\isacartoucheclose}\ {\isacartoucheopen}arity{\isacharparenleft}{\kern0pt}{\isacharquery}{\kern0pt}{\isasymchi}{\isacharparenright}{\kern0pt}\ {\isasymle}\ length{\isacharparenleft}{\kern0pt}{\isacharbrackleft}{\kern0pt}{\isasymtheta}{\isacharbrackright}{\kern0pt}\ {\isacharat}{\kern0pt}\ nenv\ {\isacharat}{\kern0pt}\ {\isacharbrackleft}{\kern0pt}{\isasympi}{\isacharbrackright}{\kern0pt}{\isacharparenright}{\kern0pt}{\isacartoucheclose}\ {\isacartoucheopen}env{\isasymin}{\isacharunderscore}{\kern0pt}{\isacartoucheclose}\isanewline
\ \ \ \ \ \ \ \ \isacommand{have}\isamarkupfalse%
\ {\isachardoublequoteopen}p\ {\isasymin}\ P\ {\isasymLongrightarrow}\ {\isacharquery}{\kern0pt}{\isasymchi}{\isasymin}formula\ {\isasymLongrightarrow}\ {\isacharbrackleft}{\kern0pt}{\isasymtheta}{\isacharcomma}{\kern0pt}{\isasympi}{\isacharbrackright}{\kern0pt}\ {\isasymin}\ list{\isacharparenleft}{\kern0pt}M{\isacharparenright}{\kern0pt}\ {\isasymLongrightarrow}\isanewline
\ \ \ \ \ \ \ \ \ \ \ \ \ \ \ \ \ \ M{\isacharcomma}{\kern0pt}\ {\isacharbrackleft}{\kern0pt}p{\isacharcomma}{\kern0pt}P{\isacharcomma}{\kern0pt}\ leq{\isacharcomma}{\kern0pt}\ one{\isacharbrackright}{\kern0pt}\ {\isacharat}{\kern0pt}\ {\isacharbrackleft}{\kern0pt}{\isasymtheta}{\isacharbrackright}{\kern0pt}{\isacharat}{\kern0pt}\ nenv{\isacharat}{\kern0pt}{\isacharbrackleft}{\kern0pt}{\isasympi}{\isacharbrackright}{\kern0pt}\ {\isasymTurnstile}\ forces{\isacharparenleft}{\kern0pt}{\isacharquery}{\kern0pt}{\isasymchi}{\isacharparenright}{\kern0pt}\ {\isasymLongrightarrow}\ \isanewline
\ \ \ \ \ \ \ \ \ \ \ \ \ \ {\isasymforall}G{\isachardot}{\kern0pt}\ M{\isacharunderscore}{\kern0pt}generic{\isacharparenleft}{\kern0pt}G{\isacharparenright}{\kern0pt}\ {\isasymand}\ p\ {\isasymin}\ G\ {\isasymlongrightarrow}\ M{\isacharbrackleft}{\kern0pt}G{\isacharbrackright}{\kern0pt}{\isacharcomma}{\kern0pt}\ \ map{\isacharparenleft}{\kern0pt}val{\isacharparenleft}{\kern0pt}G{\isacharparenright}{\kern0pt}{\isacharcomma}{\kern0pt}\ {\isacharbrackleft}{\kern0pt}{\isasymtheta}{\isacharbrackright}{\kern0pt}\ {\isacharat}{\kern0pt}\ nenv\ {\isacharat}{\kern0pt}{\isacharbrackleft}{\kern0pt}{\isasympi}{\isacharbrackright}{\kern0pt}{\isacharparenright}{\kern0pt}\ {\isasymTurnstile}\ \ {\isacharquery}{\kern0pt}{\isasymchi}{\isachardoublequoteclose}\isanewline
\ \ \ \ \ \ \ \ \ \ \isacommand{by}\isamarkupfalse%
\ auto\isanewline
\ \ \ \ \ \ \ \ \isacommand{then}\isamarkupfalse%
\isanewline
\ \ \ \ \ \ \ \ \isacommand{show}\isamarkupfalse%
\ {\isachardoublequoteopen}{\isasymforall}F{\isachardot}{\kern0pt}\ M{\isacharunderscore}{\kern0pt}generic{\isacharparenleft}{\kern0pt}F{\isacharparenright}{\kern0pt}\ {\isasymand}\ p\ {\isasymin}\ F\ {\isasymlongrightarrow}\ \isanewline
\ \ \ \ \ \ \ \ \ \ \ \ \ \ \ \ \ \ M{\isacharbrackleft}{\kern0pt}F{\isacharbrackright}{\kern0pt}{\isacharcomma}{\kern0pt}\ \ map{\isacharparenleft}{\kern0pt}val{\isacharparenleft}{\kern0pt}F{\isacharparenright}{\kern0pt}{\isacharcomma}{\kern0pt}\ {\isacharbrackleft}{\kern0pt}{\isasymtheta}{\isacharbrackright}{\kern0pt}\ {\isacharat}{\kern0pt}\ nenv\ {\isacharat}{\kern0pt}\ {\isacharbrackleft}{\kern0pt}{\isasympi}{\isacharbrackright}{\kern0pt}{\isacharparenright}{\kern0pt}\ {\isasymTurnstile}\ \ {\isacharquery}{\kern0pt}{\isasymchi}{\isachardoublequoteclose}\isanewline
\ \ \ \ \ \ \ \ \ \ \isacommand{using}\isamarkupfalse%
\ \ {\isacartoucheopen}{\isacharquery}{\kern0pt}{\isasymchi}{\isasymin}formula{\isacartoucheclose}\ {\isacartoucheopen}p{\isasymin}P{\isacartoucheclose}\ a{\isadigit{1}}\ {\isacartoucheopen}{\isasymtheta}{\isasymin}M{\isacartoucheclose}\ {\isacartoucheopen}{\isasympi}{\isasymin}M{\isacartoucheclose}\ \isacommand{by}\isamarkupfalse%
\ simp\isanewline
\ \ \ \ \ \ \isacommand{next}\isamarkupfalse%
\isanewline
\ \ \ \ \ \ \ \ \isacommand{assume}\isamarkupfalse%
\ {\isachardoublequoteopen}{\isasymforall}F{\isachardot}{\kern0pt}\ M{\isacharunderscore}{\kern0pt}generic{\isacharparenleft}{\kern0pt}F{\isacharparenright}{\kern0pt}\ {\isasymand}\ p\ {\isasymin}\ F\ {\isasymlongrightarrow}\ \isanewline
\ \ \ \ \ \ \ \ \ \ \ \ \ \ \ \ \ \ \ M{\isacharbrackleft}{\kern0pt}F{\isacharbrackright}{\kern0pt}{\isacharcomma}{\kern0pt}\ \ map{\isacharparenleft}{\kern0pt}val{\isacharparenleft}{\kern0pt}F{\isacharparenright}{\kern0pt}{\isacharcomma}{\kern0pt}\ {\isacharbrackleft}{\kern0pt}{\isasymtheta}{\isacharbrackright}{\kern0pt}\ {\isacharat}{\kern0pt}\ nenv\ {\isacharat}{\kern0pt}{\isacharbrackleft}{\kern0pt}{\isasympi}{\isacharbrackright}{\kern0pt}{\isacharparenright}{\kern0pt}\ {\isasymTurnstile}\ \ {\isacharquery}{\kern0pt}{\isasymchi}{\isachardoublequoteclose}\isanewline
\ \ \ \ \ \ \ \ \isacommand{with}\isamarkupfalse%
\ definition{\isacharunderscore}{\kern0pt}of{\isacharunderscore}{\kern0pt}forcing\ {\isacharbrackleft}{\kern0pt}THEN\ iffD{\isadigit{2}}{\isacharbrackright}{\kern0pt}\ {\isacartoucheopen}arity{\isacharparenleft}{\kern0pt}{\isacharquery}{\kern0pt}{\isasymchi}{\isacharparenright}{\kern0pt}\ {\isasymle}\ length{\isacharparenleft}{\kern0pt}{\isacharbrackleft}{\kern0pt}{\isasymtheta}{\isacharbrackright}{\kern0pt}\ {\isacharat}{\kern0pt}\ nenv\ {\isacharat}{\kern0pt}\ {\isacharbrackleft}{\kern0pt}{\isasympi}{\isacharbrackright}{\kern0pt}{\isacharparenright}{\kern0pt}{\isacartoucheclose}\isanewline
\ \ \ \ \ \ \ \ \isacommand{show}\isamarkupfalse%
\ {\isachardoublequoteopen}M{\isacharcomma}{\kern0pt}\ \ {\isacharbrackleft}{\kern0pt}p{\isacharcomma}{\kern0pt}\ P{\isacharcomma}{\kern0pt}\ leq{\isacharcomma}{\kern0pt}\ one{\isacharcomma}{\kern0pt}{\isasymtheta}{\isacharbrackright}{\kern0pt}\ {\isacharat}{\kern0pt}\ nenv\ {\isacharat}{\kern0pt}\ {\isacharbrackleft}{\kern0pt}{\isasympi}{\isacharbrackright}{\kern0pt}\ {\isasymTurnstile}\ \ forces{\isacharparenleft}{\kern0pt}{\isacharquery}{\kern0pt}{\isasymchi}{\isacharparenright}{\kern0pt}{\isachardoublequoteclose}\isanewline
\ \ \ \ \ \ \ \ \ \ \isacommand{using}\isamarkupfalse%
\ \ {\isacartoucheopen}{\isacharquery}{\kern0pt}{\isasymchi}{\isasymin}formula{\isacartoucheclose}\ {\isacartoucheopen}p{\isasymin}P{\isacartoucheclose}\ in{\isacharunderscore}{\kern0pt}M{\isacharprime}{\kern0pt}\ \isanewline
\ \ \ \ \ \ \ \ \ \ \isacommand{by}\isamarkupfalse%
\ auto\isanewline
\ \ \ \ \ \ \isacommand{qed}\isamarkupfalse%
\isanewline
\ \ \ \ \ \ \isacommand{finally}\isamarkupfalse%
\ \isanewline
\ \ \ \ \ \ \isacommand{show}\isamarkupfalse%
\ {\isachardoublequoteopen}{\isacharparenleft}{\kern0pt}M{\isacharcomma}{\kern0pt}\ {\isacharbrackleft}{\kern0pt}{\isasymtheta}{\isacharcomma}{\kern0pt}p{\isacharcomma}{\kern0pt}u{\isacharbrackright}{\kern0pt}{\isacharat}{\kern0pt}{\isacharquery}{\kern0pt}Pl{\isadigit{1}}{\isacharat}{\kern0pt}{\isacharbrackleft}{\kern0pt}{\isasympi}{\isacharbrackright}{\kern0pt}{\isacharat}{\kern0pt}nenv\ {\isasymTurnstile}\ {\isacharquery}{\kern0pt}new{\isacharunderscore}{\kern0pt}form{\isacharparenright}{\kern0pt}\ {\isasymlongleftrightarrow}\ {\isacharparenleft}{\kern0pt}{\isasymforall}F{\isachardot}{\kern0pt}\ M{\isacharunderscore}{\kern0pt}generic{\isacharparenleft}{\kern0pt}F{\isacharparenright}{\kern0pt}\ {\isasymand}\ p\ {\isasymin}\ F\ {\isasymlongrightarrow}\ \isanewline
\ \ \ \ \ \ \ \ \ \ \ \ \ \ \ \ \ \ \ \ \ \ \ \ \ \ \ M{\isacharbrackleft}{\kern0pt}F{\isacharbrackright}{\kern0pt}{\isacharcomma}{\kern0pt}\ \ map{\isacharparenleft}{\kern0pt}val{\isacharparenleft}{\kern0pt}F{\isacharparenright}{\kern0pt}{\isacharcomma}{\kern0pt}\ {\isacharbrackleft}{\kern0pt}{\isasymtheta}{\isacharbrackright}{\kern0pt}\ {\isacharat}{\kern0pt}\ nenv\ {\isacharat}{\kern0pt}\ {\isacharbrackleft}{\kern0pt}{\isasympi}{\isacharbrackright}{\kern0pt}{\isacharparenright}{\kern0pt}\ {\isasymTurnstile}\ \ {\isacharquery}{\kern0pt}{\isasymchi}{\isacharparenright}{\kern0pt}{\isachardoublequoteclose}\ \isanewline
\ \ \ \ \ \ \ \ \isacommand{by}\isamarkupfalse%
\ simp\isanewline
\ \ \ \ \isacommand{qed}\isamarkupfalse%
\isanewline
\ \ \ \ \isacommand{with}\isamarkupfalse%
\ Eq{\isadigit{1}}\ \isanewline
\ \ \ \ \isacommand{have}\isamarkupfalse%
\ {\isachardoublequoteopen}{\isacharparenleft}{\kern0pt}M{\isacharcomma}{\kern0pt}\ {\isacharbrackleft}{\kern0pt}u{\isacharbrackright}{\kern0pt}\ {\isacharat}{\kern0pt}\ {\isacharquery}{\kern0pt}Pl{\isadigit{1}}\ {\isacharat}{\kern0pt}\ {\isacharbrackleft}{\kern0pt}{\isasympi}{\isacharbrackright}{\kern0pt}\ {\isacharat}{\kern0pt}\ nenv\ {\isasymTurnstile}\ {\isacharquery}{\kern0pt}{\isasympsi}{\isacharparenright}{\kern0pt}\ {\isasymlongleftrightarrow}\ \isanewline
\ \ \ \ \ \ \ \ \ {\isacharparenleft}{\kern0pt}{\isasymexists}{\isasymtheta}{\isasymin}M{\isachardot}{\kern0pt}\ {\isasymexists}p{\isasymin}P{\isachardot}{\kern0pt}\ u\ {\isacharequal}{\kern0pt}{\isasymlangle}{\isasymtheta}{\isacharcomma}{\kern0pt}p{\isasymrangle}\ {\isasymand}\ \isanewline
\ \ \ \ \ \ \ \ \ \ {\isacharparenleft}{\kern0pt}{\isasymforall}F{\isachardot}{\kern0pt}\ M{\isacharunderscore}{\kern0pt}generic{\isacharparenleft}{\kern0pt}F{\isacharparenright}{\kern0pt}\ {\isasymand}\ p\ {\isasymin}\ F\ {\isasymlongrightarrow}\ M{\isacharbrackleft}{\kern0pt}F{\isacharbrackright}{\kern0pt}{\isacharcomma}{\kern0pt}\ \ map{\isacharparenleft}{\kern0pt}val{\isacharparenleft}{\kern0pt}F{\isacharparenright}{\kern0pt}{\isacharcomma}{\kern0pt}\ {\isacharbrackleft}{\kern0pt}{\isasymtheta}{\isacharbrackright}{\kern0pt}\ {\isacharat}{\kern0pt}\ nenv\ {\isacharat}{\kern0pt}\ {\isacharbrackleft}{\kern0pt}{\isasympi}{\isacharbrackright}{\kern0pt}{\isacharparenright}{\kern0pt}\ {\isasymTurnstile}\ \ {\isacharquery}{\kern0pt}{\isasymchi}{\isacharparenright}{\kern0pt}{\isacharparenright}{\kern0pt}{\isachardoublequoteclose}\isanewline
\ \ \ \ \ \ \isacommand{by}\isamarkupfalse%
\ auto\ \isanewline
\ \ \isacommand{{\isacharbraceright}{\kern0pt}}\isamarkupfalse%
\isanewline
\ \ \isacommand{then}\isamarkupfalse%
\ \isanewline
\ \ \isacommand{have}\isamarkupfalse%
\ Equivalence{\isacharcolon}{\kern0pt}\ {\isachardoublequoteopen}u{\isasymin}\ domain{\isacharparenleft}{\kern0pt}{\isasympi}{\isacharparenright}{\kern0pt}\ {\isasymtimes}\ P\ {\isasymLongrightarrow}\ u\ {\isasymin}\ M\ {\isasymLongrightarrow}\ \isanewline
\ \ \ \ \ \ \ {\isacharparenleft}{\kern0pt}M{\isacharcomma}{\kern0pt}\ {\isacharbrackleft}{\kern0pt}u{\isacharbrackright}{\kern0pt}\ {\isacharat}{\kern0pt}\ {\isacharquery}{\kern0pt}Pl{\isadigit{1}}\ {\isacharat}{\kern0pt}\ {\isacharbrackleft}{\kern0pt}{\isasympi}{\isacharbrackright}{\kern0pt}\ {\isacharat}{\kern0pt}\ nenv\ {\isasymTurnstile}\ {\isacharquery}{\kern0pt}{\isasympsi}{\isacharparenright}{\kern0pt}\ {\isasymlongleftrightarrow}\ \isanewline
\ \ \ \ \ \ \ \ \ {\isacharparenleft}{\kern0pt}{\isasymexists}{\isasymtheta}{\isasymin}M{\isachardot}{\kern0pt}\ {\isasymexists}p{\isasymin}P{\isachardot}{\kern0pt}\ u\ {\isacharequal}{\kern0pt}{\isasymlangle}{\isasymtheta}{\isacharcomma}{\kern0pt}p{\isasymrangle}\ {\isasymand}\ \isanewline
\ \ \ \ \ \ \ \ \ \ {\isacharparenleft}{\kern0pt}{\isasymforall}F{\isachardot}{\kern0pt}\ M{\isacharunderscore}{\kern0pt}generic{\isacharparenleft}{\kern0pt}F{\isacharparenright}{\kern0pt}\ {\isasymand}\ p\ {\isasymin}\ F\ {\isasymlongrightarrow}\ M{\isacharbrackleft}{\kern0pt}F{\isacharbrackright}{\kern0pt}{\isacharcomma}{\kern0pt}\ \ \ map{\isacharparenleft}{\kern0pt}val{\isacharparenleft}{\kern0pt}F{\isacharparenright}{\kern0pt}{\isacharcomma}{\kern0pt}\ {\isacharbrackleft}{\kern0pt}{\isasymtheta}{\isacharbrackright}{\kern0pt}\ {\isacharat}{\kern0pt}\ nenv\ {\isacharat}{\kern0pt}{\isacharbrackleft}{\kern0pt}{\isasympi}{\isacharbrackright}{\kern0pt}{\isacharparenright}{\kern0pt}\ {\isasymTurnstile}\ \ {\isacharquery}{\kern0pt}{\isasymchi}{\isacharparenright}{\kern0pt}{\isacharparenright}{\kern0pt}{\isachardoublequoteclose}\ \isanewline
\ \ \ \ \isakeyword{for}\ u\ \isanewline
\ \ \ \ \isacommand{by}\isamarkupfalse%
\ simp\isanewline
\ \ \isacommand{moreover}\isamarkupfalse%
\ \isacommand{from}\isamarkupfalse%
\ {\isacartoucheopen}env\ {\isacharequal}{\kern0pt}\ {\isacharunderscore}{\kern0pt}{\isacartoucheclose}\ {\isacartoucheopen}{\isasympi}{\isasymin}M{\isacartoucheclose}\ {\isacartoucheopen}nenv{\isasymin}list{\isacharparenleft}{\kern0pt}M{\isacharparenright}{\kern0pt}{\isacartoucheclose}\isanewline
\ \ \isacommand{have}\isamarkupfalse%
\ map{\isacharunderscore}{\kern0pt}nenv{\isacharcolon}{\kern0pt}{\isachardoublequoteopen}map{\isacharparenleft}{\kern0pt}val{\isacharparenleft}{\kern0pt}G{\isacharparenright}{\kern0pt}{\isacharcomma}{\kern0pt}\ nenv{\isacharat}{\kern0pt}{\isacharbrackleft}{\kern0pt}{\isasympi}{\isacharbrackright}{\kern0pt}{\isacharparenright}{\kern0pt}\ {\isacharequal}{\kern0pt}\ env\ {\isacharat}{\kern0pt}\ {\isacharbrackleft}{\kern0pt}val{\isacharparenleft}{\kern0pt}G{\isacharcomma}{\kern0pt}{\isasympi}{\isacharparenright}{\kern0pt}{\isacharbrackright}{\kern0pt}{\isachardoublequoteclose}\isanewline
\ \ \ \ \isacommand{using}\isamarkupfalse%
\ map{\isacharunderscore}{\kern0pt}app{\isacharunderscore}{\kern0pt}distrib\ append{\isadigit{1}}{\isacharunderscore}{\kern0pt}eq{\isacharunderscore}{\kern0pt}iff\ \isacommand{by}\isamarkupfalse%
\ auto\isanewline
\ \ \isacommand{ultimately}\isamarkupfalse%
\isanewline
\ \ \isacommand{have}\isamarkupfalse%
\ aux{\isacharcolon}{\kern0pt}{\isachardoublequoteopen}{\isacharparenleft}{\kern0pt}{\isasymexists}{\isasymtheta}{\isasymin}M{\isachardot}{\kern0pt}\ {\isasymexists}p{\isasymin}P{\isachardot}{\kern0pt}\ u\ {\isacharequal}{\kern0pt}{\isasymlangle}{\isasymtheta}{\isacharcomma}{\kern0pt}p{\isasymrangle}\ {\isasymand}\ {\isacharparenleft}{\kern0pt}p{\isasymin}G\ {\isasymlongrightarrow}\ M{\isacharbrackleft}{\kern0pt}G{\isacharbrackright}{\kern0pt}{\isacharcomma}{\kern0pt}\ {\isacharbrackleft}{\kern0pt}val{\isacharparenleft}{\kern0pt}G{\isacharcomma}{\kern0pt}{\isasymtheta}{\isacharparenright}{\kern0pt}{\isacharbrackright}{\kern0pt}\ {\isacharat}{\kern0pt}\ env\ {\isacharat}{\kern0pt}\ {\isacharbrackleft}{\kern0pt}val{\isacharparenleft}{\kern0pt}G{\isacharcomma}{\kern0pt}{\isasympi}{\isacharparenright}{\kern0pt}{\isacharbrackright}{\kern0pt}\ {\isasymTurnstile}\ {\isacharquery}{\kern0pt}{\isasymchi}{\isacharparenright}{\kern0pt}{\isacharparenright}{\kern0pt}{\isachardoublequoteclose}\ \isanewline
\ \ \ {\isacharparenleft}{\kern0pt}\isakeyword{is}\ {\isachardoublequoteopen}{\isacharparenleft}{\kern0pt}{\isasymexists}{\isasymtheta}{\isasymin}M{\isachardot}{\kern0pt}\ {\isasymexists}p{\isasymin}P{\isachardot}{\kern0pt}\ {\isacharunderscore}{\kern0pt}\ {\isacharparenleft}{\kern0pt}\ {\isacharunderscore}{\kern0pt}\ {\isasymlongrightarrow}\ {\isacharunderscore}{\kern0pt}{\isacharcomma}{\kern0pt}\ {\isacharquery}{\kern0pt}vals{\isacharparenleft}{\kern0pt}{\isasymtheta}{\isacharparenright}{\kern0pt}\ {\isasymTurnstile}\ {\isacharunderscore}{\kern0pt}{\isacharparenright}{\kern0pt}{\isacharparenright}{\kern0pt}{\isachardoublequoteclose}{\isacharparenright}{\kern0pt}\isanewline
\ \ \ \isakeyword{if}\ {\isachardoublequoteopen}u\ {\isasymin}\ domain{\isacharparenleft}{\kern0pt}{\isasympi}{\isacharparenright}{\kern0pt}\ {\isasymtimes}\ P{\isachardoublequoteclose}\ {\isachardoublequoteopen}u\ {\isasymin}\ M{\isachardoublequoteclose}\ \ {\isachardoublequoteopen}M{\isacharcomma}{\kern0pt}\ {\isacharbrackleft}{\kern0pt}u{\isacharbrackright}{\kern0pt}{\isacharat}{\kern0pt}\ {\isacharquery}{\kern0pt}Pl{\isadigit{1}}\ {\isacharat}{\kern0pt}{\isacharbrackleft}{\kern0pt}{\isasympi}{\isacharbrackright}{\kern0pt}\ {\isacharat}{\kern0pt}\ nenv\ {\isasymTurnstile}\ {\isacharquery}{\kern0pt}{\isasympsi}{\isachardoublequoteclose}\ \isakeyword{for}\ u\isanewline
\ \ \ \ \isacommand{using}\isamarkupfalse%
\ Equivalence{\isacharbrackleft}{\kern0pt}THEN\ iffD{\isadigit{1}}{\isacharcomma}{\kern0pt}\ OF\ that{\isacharbrackright}{\kern0pt}\ generic\ \isacommand{by}\isamarkupfalse%
\ force\isanewline
\ \ \isacommand{moreover}\isamarkupfalse%
\ \isanewline
\ \ \isacommand{have}\isamarkupfalse%
\ {\isachardoublequoteopen}{\isasymtheta}{\isasymin}M\ {\isasymLongrightarrow}\ val{\isacharparenleft}{\kern0pt}G{\isacharcomma}{\kern0pt}{\isasymtheta}{\isacharparenright}{\kern0pt}{\isasymin}M{\isacharbrackleft}{\kern0pt}G{\isacharbrackright}{\kern0pt}{\isachardoublequoteclose}\ \isakeyword{for}\ {\isasymtheta}\isanewline
\ \ \ \ \isacommand{using}\isamarkupfalse%
\ GenExt{\isacharunderscore}{\kern0pt}def\ \isacommand{by}\isamarkupfalse%
\ auto\isanewline
\ \ \isacommand{moreover}\isamarkupfalse%
\isanewline
\ \ \isacommand{have}\isamarkupfalse%
\ {\isachardoublequoteopen}{\isasymtheta}{\isasymin}\ M\ {\isasymLongrightarrow}\ {\isacharbrackleft}{\kern0pt}val{\isacharparenleft}{\kern0pt}G{\isacharcomma}{\kern0pt}\ {\isasymtheta}{\isacharparenright}{\kern0pt}{\isacharbrackright}{\kern0pt}\ {\isacharat}{\kern0pt}\ env\ {\isacharat}{\kern0pt}\ {\isacharbrackleft}{\kern0pt}val{\isacharparenleft}{\kern0pt}G{\isacharcomma}{\kern0pt}\ {\isasympi}{\isacharparenright}{\kern0pt}{\isacharbrackright}{\kern0pt}\ {\isasymin}\ list{\isacharparenleft}{\kern0pt}M{\isacharbrackleft}{\kern0pt}G{\isacharbrackright}{\kern0pt}{\isacharparenright}{\kern0pt}{\isachardoublequoteclose}\ \isakeyword{for}\ {\isasymtheta}\isanewline
\ \ \isacommand{proof}\isamarkupfalse%
\ {\isacharminus}{\kern0pt}\isanewline
\ \ \ \ \isacommand{from}\isamarkupfalse%
\ {\isacartoucheopen}{\isasympi}{\isasymin}M{\isacartoucheclose}\isanewline
\ \ \ \ \isacommand{have}\isamarkupfalse%
\ {\isachardoublequoteopen}val{\isacharparenleft}{\kern0pt}G{\isacharcomma}{\kern0pt}{\isasympi}{\isacharparenright}{\kern0pt}{\isasymin}\ M{\isacharbrackleft}{\kern0pt}G{\isacharbrackright}{\kern0pt}{\isachardoublequoteclose}\ \isacommand{using}\isamarkupfalse%
\ GenExtI\ \isacommand{by}\isamarkupfalse%
\ simp\isanewline
\ \ \ \ \isacommand{moreover}\isamarkupfalse%
\isanewline
\ \ \ \ \isacommand{assume}\isamarkupfalse%
\ {\isachardoublequoteopen}{\isasymtheta}\ {\isasymin}\ M{\isachardoublequoteclose}\isanewline
\ \ \ \ \isacommand{moreover}\isamarkupfalse%
\isanewline
\ \ \ \ \isacommand{note}\isamarkupfalse%
\ {\isacartoucheopen}env\ {\isasymin}\ list{\isacharparenleft}{\kern0pt}M{\isacharbrackleft}{\kern0pt}G{\isacharbrackright}{\kern0pt}{\isacharparenright}{\kern0pt}{\isacartoucheclose}\isanewline
\ \ \ \ \isacommand{ultimately}\isamarkupfalse%
\isanewline
\ \ \ \ \isacommand{show}\isamarkupfalse%
\ {\isacharquery}{\kern0pt}thesis\ \isanewline
\ \ \ \ \ \ \isacommand{using}\isamarkupfalse%
\ GenExtI\ \isacommand{by}\isamarkupfalse%
\ simp\isanewline
\ \ \isacommand{qed}\isamarkupfalse%
\isanewline
\ \ \isacommand{ultimately}\isamarkupfalse%
\ \isanewline
\ \ \isacommand{have}\isamarkupfalse%
\ {\isachardoublequoteopen}{\isacharparenleft}{\kern0pt}{\isasymexists}{\isasymtheta}{\isasymin}M{\isachardot}{\kern0pt}\ {\isasymexists}p{\isasymin}P{\isachardot}{\kern0pt}\ u{\isacharequal}{\kern0pt}{\isasymlangle}{\isasymtheta}{\isacharcomma}{\kern0pt}p{\isasymrangle}\ {\isasymand}\ {\isacharparenleft}{\kern0pt}p{\isasymin}G\ {\isasymlongrightarrow}\ val{\isacharparenleft}{\kern0pt}G{\isacharcomma}{\kern0pt}{\isasymtheta}{\isacharparenright}{\kern0pt}{\isasymin}nth{\isacharparenleft}{\kern0pt}{\isadigit{1}}\ {\isacharhash}{\kern0pt}{\isacharplus}{\kern0pt}\ length{\isacharparenleft}{\kern0pt}env{\isacharparenright}{\kern0pt}{\isacharcomma}{\kern0pt}{\isacharbrackleft}{\kern0pt}val{\isacharparenleft}{\kern0pt}G{\isacharcomma}{\kern0pt}\ {\isasymtheta}{\isacharparenright}{\kern0pt}{\isacharbrackright}{\kern0pt}\ {\isacharat}{\kern0pt}\ env\ {\isacharat}{\kern0pt}\ {\isacharbrackleft}{\kern0pt}val{\isacharparenleft}{\kern0pt}G{\isacharcomma}{\kern0pt}\ {\isasympi}{\isacharparenright}{\kern0pt}{\isacharbrackright}{\kern0pt}{\isacharparenright}{\kern0pt}\ \isanewline
\ \ \ \ \ \ \ \ {\isasymand}\ M{\isacharbrackleft}{\kern0pt}G{\isacharbrackright}{\kern0pt}{\isacharcomma}{\kern0pt}\ \ {\isacharquery}{\kern0pt}vals{\isacharparenleft}{\kern0pt}{\isasymtheta}{\isacharparenright}{\kern0pt}\ {\isasymTurnstile}\ \ {\isasymphi}{\isacharparenright}{\kern0pt}{\isacharparenright}{\kern0pt}{\isachardoublequoteclose}\isanewline
\ \ \ \ \isakeyword{if}\ {\isachardoublequoteopen}u\ {\isasymin}\ domain{\isacharparenleft}{\kern0pt}{\isasympi}{\isacharparenright}{\kern0pt}\ {\isasymtimes}\ P{\isachardoublequoteclose}\ {\isachardoublequoteopen}u\ {\isasymin}\ M{\isachardoublequoteclose}\ \ {\isachardoublequoteopen}M{\isacharcomma}{\kern0pt}\ {\isacharbrackleft}{\kern0pt}u{\isacharbrackright}{\kern0pt}\ {\isacharat}{\kern0pt}\ {\isacharquery}{\kern0pt}Pl{\isadigit{1}}\ {\isacharat}{\kern0pt}{\isacharbrackleft}{\kern0pt}{\isasympi}{\isacharbrackright}{\kern0pt}\ {\isacharat}{\kern0pt}\ nenv\ {\isasymTurnstile}\ {\isacharquery}{\kern0pt}{\isasympsi}{\isachardoublequoteclose}\ \isakeyword{for}\ u\isanewline
\ \ \ \ \isacommand{using}\isamarkupfalse%
\ aux{\isacharbrackleft}{\kern0pt}OF\ that{\isacharbrackright}{\kern0pt}\ \isacommand{by}\isamarkupfalse%
\ simp\isanewline
\ \ \isacommand{moreover}\isamarkupfalse%
\ \isacommand{from}\isamarkupfalse%
\ {\isacartoucheopen}env\ {\isasymin}\ {\isacharunderscore}{\kern0pt}{\isacartoucheclose}\ {\isacartoucheopen}{\isasympi}{\isasymin}M{\isacartoucheclose}\isanewline
\ \ \isacommand{have}\isamarkupfalse%
\ nth{\isacharcolon}{\kern0pt}{\isachardoublequoteopen}nth{\isacharparenleft}{\kern0pt}{\isadigit{1}}\ {\isacharhash}{\kern0pt}{\isacharplus}{\kern0pt}\ length{\isacharparenleft}{\kern0pt}env{\isacharparenright}{\kern0pt}{\isacharcomma}{\kern0pt}{\isacharbrackleft}{\kern0pt}val{\isacharparenleft}{\kern0pt}G{\isacharcomma}{\kern0pt}\ {\isasymtheta}{\isacharparenright}{\kern0pt}{\isacharbrackright}{\kern0pt}\ {\isacharat}{\kern0pt}\ env\ {\isacharat}{\kern0pt}\ {\isacharbrackleft}{\kern0pt}val{\isacharparenleft}{\kern0pt}G{\isacharcomma}{\kern0pt}\ {\isasympi}{\isacharparenright}{\kern0pt}{\isacharbrackright}{\kern0pt}{\isacharparenright}{\kern0pt}\ {\isacharequal}{\kern0pt}\ val{\isacharparenleft}{\kern0pt}G{\isacharcomma}{\kern0pt}{\isasympi}{\isacharparenright}{\kern0pt}{\isachardoublequoteclose}\ \isanewline
\ \ \ \ \isakeyword{if}\ {\isachardoublequoteopen}{\isasymtheta}{\isasymin}M{\isachardoublequoteclose}\ \isakeyword{for}\ {\isasymtheta}\isanewline
\ \ \ \ \isacommand{using}\isamarkupfalse%
\ nth{\isacharunderscore}{\kern0pt}concat{\isacharbrackleft}{\kern0pt}of\ {\isachardoublequoteopen}val{\isacharparenleft}{\kern0pt}G{\isacharcomma}{\kern0pt}{\isasymtheta}{\isacharparenright}{\kern0pt}{\isachardoublequoteclose}\ {\isachardoublequoteopen}val{\isacharparenleft}{\kern0pt}G{\isacharcomma}{\kern0pt}{\isasympi}{\isacharparenright}{\kern0pt}{\isachardoublequoteclose}\ {\isachardoublequoteopen}M{\isacharbrackleft}{\kern0pt}G{\isacharbrackright}{\kern0pt}{\isachardoublequoteclose}{\isacharbrackright}{\kern0pt}\ \isacommand{using}\isamarkupfalse%
\ that\ GenExtI\ \isacommand{by}\isamarkupfalse%
\ simp\isanewline
\ \ \isacommand{ultimately}\isamarkupfalse%
\isanewline
\ \ \isacommand{have}\isamarkupfalse%
\ {\isachardoublequoteopen}{\isacharparenleft}{\kern0pt}{\isasymexists}{\isasymtheta}{\isasymin}M{\isachardot}{\kern0pt}\ {\isasymexists}p{\isasymin}P{\isachardot}{\kern0pt}\ u{\isacharequal}{\kern0pt}{\isasymlangle}{\isasymtheta}{\isacharcomma}{\kern0pt}p{\isasymrangle}\ {\isasymand}\ {\isacharparenleft}{\kern0pt}p{\isasymin}G\ {\isasymlongrightarrow}\ val{\isacharparenleft}{\kern0pt}G{\isacharcomma}{\kern0pt}{\isasymtheta}{\isacharparenright}{\kern0pt}{\isasymin}val{\isacharparenleft}{\kern0pt}G{\isacharcomma}{\kern0pt}{\isasympi}{\isacharparenright}{\kern0pt}\ {\isasymand}\ M{\isacharbrackleft}{\kern0pt}G{\isacharbrackright}{\kern0pt}{\isacharcomma}{\kern0pt}\ \ {\isacharquery}{\kern0pt}vals{\isacharparenleft}{\kern0pt}{\isasymtheta}{\isacharparenright}{\kern0pt}\ {\isasymTurnstile}\ \ {\isasymphi}{\isacharparenright}{\kern0pt}{\isacharparenright}{\kern0pt}{\isachardoublequoteclose}\isanewline
\ \ \ \ \isakeyword{if}\ {\isachardoublequoteopen}u\ {\isasymin}\ domain{\isacharparenleft}{\kern0pt}{\isasympi}{\isacharparenright}{\kern0pt}\ {\isasymtimes}\ P{\isachardoublequoteclose}\ {\isachardoublequoteopen}u\ {\isasymin}\ M{\isachardoublequoteclose}\ \ {\isachardoublequoteopen}M{\isacharcomma}{\kern0pt}\ {\isacharbrackleft}{\kern0pt}u{\isacharbrackright}{\kern0pt}\ {\isacharat}{\kern0pt}\ {\isacharquery}{\kern0pt}Pl{\isadigit{1}}\ {\isacharat}{\kern0pt}{\isacharbrackleft}{\kern0pt}{\isasympi}{\isacharbrackright}{\kern0pt}\ {\isacharat}{\kern0pt}\ nenv\ {\isasymTurnstile}\ {\isacharquery}{\kern0pt}{\isasympsi}{\isachardoublequoteclose}\ \isakeyword{for}\ u\isanewline
\ \ \ \ \isacommand{using}\isamarkupfalse%
\ that\ {\isacartoucheopen}{\isasympi}{\isasymin}M{\isacartoucheclose}\ {\isacartoucheopen}env\ {\isasymin}\ {\isacharunderscore}{\kern0pt}{\isacartoucheclose}\ \isacommand{by}\isamarkupfalse%
\ simp\isanewline
\ \ \isacommand{with}\isamarkupfalse%
\ {\isacartoucheopen}domain{\isacharparenleft}{\kern0pt}{\isasympi}{\isacharparenright}{\kern0pt}{\isasymtimes}P{\isasymin}M{\isacartoucheclose}\isanewline
\ \ \isacommand{have}\isamarkupfalse%
\ {\isachardoublequoteopen}{\isasymforall}u{\isasymin}domain{\isacharparenleft}{\kern0pt}{\isasympi}{\isacharparenright}{\kern0pt}{\isasymtimes}P\ {\isachardot}{\kern0pt}\ {\isacharparenleft}{\kern0pt}M{\isacharcomma}{\kern0pt}\ {\isacharbrackleft}{\kern0pt}u{\isacharbrackright}{\kern0pt}\ {\isacharat}{\kern0pt}\ {\isacharquery}{\kern0pt}Pl{\isadigit{1}}\ {\isacharat}{\kern0pt}{\isacharbrackleft}{\kern0pt}{\isasympi}{\isacharbrackright}{\kern0pt}\ {\isacharat}{\kern0pt}\ nenv\ {\isasymTurnstile}\ {\isacharquery}{\kern0pt}{\isasympsi}{\isacharparenright}{\kern0pt}\ {\isasymlongrightarrow}\ {\isacharparenleft}{\kern0pt}{\isasymexists}{\isasymtheta}{\isasymin}M{\isachardot}{\kern0pt}\ {\isasymexists}p{\isasymin}P{\isachardot}{\kern0pt}\ u\ {\isacharequal}{\kern0pt}{\isasymlangle}{\isasymtheta}{\isacharcomma}{\kern0pt}p{\isasymrangle}\ {\isasymand}\isanewline
\ \ \ \ \ \ \ \ {\isacharparenleft}{\kern0pt}p\ {\isasymin}\ G\ {\isasymlongrightarrow}\ val{\isacharparenleft}{\kern0pt}G{\isacharcomma}{\kern0pt}\ {\isasymtheta}{\isacharparenright}{\kern0pt}{\isasymin}val{\isacharparenleft}{\kern0pt}G{\isacharcomma}{\kern0pt}\ {\isasympi}{\isacharparenright}{\kern0pt}\ {\isasymand}\ M{\isacharbrackleft}{\kern0pt}G{\isacharbrackright}{\kern0pt}{\isacharcomma}{\kern0pt}\ \ {\isacharquery}{\kern0pt}vals{\isacharparenleft}{\kern0pt}{\isasymtheta}{\isacharparenright}{\kern0pt}\ {\isasymTurnstile}\ \ {\isasymphi}{\isacharparenright}{\kern0pt}{\isacharparenright}{\kern0pt}{\isachardoublequoteclose}\isanewline
\ \ \ \ \isacommand{by}\isamarkupfalse%
\ {\isacharparenleft}{\kern0pt}simp\ add{\isacharcolon}{\kern0pt}transitivity{\isacharparenright}{\kern0pt}\isanewline
\ \ \isacommand{then}\isamarkupfalse%
\ \isanewline
\ \ \isacommand{have}\isamarkupfalse%
\ {\isachardoublequoteopen}{\isacharbraceleft}{\kern0pt}u{\isasymin}domain{\isacharparenleft}{\kern0pt}{\isasympi}{\isacharparenright}{\kern0pt}{\isasymtimes}P\ {\isachardot}{\kern0pt}\ {\isacharparenleft}{\kern0pt}M{\isacharcomma}{\kern0pt}{\isacharbrackleft}{\kern0pt}u{\isacharbrackright}{\kern0pt}\ {\isacharat}{\kern0pt}\ {\isacharquery}{\kern0pt}Pl{\isadigit{1}}\ {\isacharat}{\kern0pt}{\isacharbrackleft}{\kern0pt}{\isasympi}{\isacharbrackright}{\kern0pt}\ {\isacharat}{\kern0pt}\ nenv\ {\isasymTurnstile}\ {\isacharquery}{\kern0pt}{\isasympsi}{\isacharparenright}{\kern0pt}\ {\isacharbraceright}{\kern0pt}\ {\isasymsubseteq}\isanewline
\ \ \ \ \ {\isacharbraceleft}{\kern0pt}u{\isasymin}domain{\isacharparenleft}{\kern0pt}{\isasympi}{\isacharparenright}{\kern0pt}{\isasymtimes}P\ {\isachardot}{\kern0pt}\ {\isasymexists}{\isasymtheta}{\isasymin}M{\isachardot}{\kern0pt}\ {\isasymexists}p{\isasymin}P{\isachardot}{\kern0pt}\ u\ {\isacharequal}{\kern0pt}{\isasymlangle}{\isasymtheta}{\isacharcomma}{\kern0pt}p{\isasymrangle}\ {\isasymand}\ \isanewline
\ \ \ \ \ \ \ {\isacharparenleft}{\kern0pt}p\ {\isasymin}\ G\ {\isasymlongrightarrow}\ val{\isacharparenleft}{\kern0pt}G{\isacharcomma}{\kern0pt}\ {\isasymtheta}{\isacharparenright}{\kern0pt}{\isasymin}val{\isacharparenleft}{\kern0pt}G{\isacharcomma}{\kern0pt}\ {\isasympi}{\isacharparenright}{\kern0pt}\ {\isasymand}\ {\isacharparenleft}{\kern0pt}M{\isacharbrackleft}{\kern0pt}G{\isacharbrackright}{\kern0pt}{\isacharcomma}{\kern0pt}\ {\isacharquery}{\kern0pt}vals{\isacharparenleft}{\kern0pt}{\isasymtheta}{\isacharparenright}{\kern0pt}\ {\isasymTurnstile}\ {\isasymphi}{\isacharparenright}{\kern0pt}{\isacharparenright}{\kern0pt}{\isacharbraceright}{\kern0pt}{\isachardoublequoteclose}\isanewline
\ \ \ \ {\isacharparenleft}{\kern0pt}\isakeyword{is}\ {\isachardoublequoteopen}{\isacharquery}{\kern0pt}n{\isasymsubseteq}{\isacharquery}{\kern0pt}m{\isachardoublequoteclose}{\isacharparenright}{\kern0pt}\ \isanewline
\ \ \ \ \isacommand{by}\isamarkupfalse%
\ auto\isanewline
\ \ \isacommand{with}\isamarkupfalse%
\ val{\isacharunderscore}{\kern0pt}mono\ \isanewline
\ \ \isacommand{have}\isamarkupfalse%
\ first{\isacharunderscore}{\kern0pt}incl{\isacharcolon}{\kern0pt}\ {\isachardoublequoteopen}val{\isacharparenleft}{\kern0pt}G{\isacharcomma}{\kern0pt}{\isacharquery}{\kern0pt}n{\isacharparenright}{\kern0pt}\ {\isasymsubseteq}\ val{\isacharparenleft}{\kern0pt}G{\isacharcomma}{\kern0pt}{\isacharquery}{\kern0pt}m{\isacharparenright}{\kern0pt}{\isachardoublequoteclose}\ \isanewline
\ \ \ \ \isacommand{by}\isamarkupfalse%
\ simp\isanewline
\ \ \isacommand{note}\isamarkupfalse%
\ \ {\isacartoucheopen}val{\isacharparenleft}{\kern0pt}G{\isacharcomma}{\kern0pt}{\isasympi}{\isacharparenright}{\kern0pt}\ {\isacharequal}{\kern0pt}\ c{\isacartoucheclose}\ \isanewline
\ \ \isacommand{with}\isamarkupfalse%
\ {\isacartoucheopen}{\isacharquery}{\kern0pt}{\isasympsi}{\isasymin}formula{\isacartoucheclose}\ \ {\isacartoucheopen}arity{\isacharparenleft}{\kern0pt}{\isacharquery}{\kern0pt}{\isasympsi}{\isacharparenright}{\kern0pt}\ {\isasymle}\ {\isacharunderscore}{\kern0pt}{\isacartoucheclose}\ in{\isacharunderscore}{\kern0pt}M\ {\isacartoucheopen}nenv\ {\isasymin}\ {\isacharunderscore}{\kern0pt}{\isacartoucheclose}\ {\isacartoucheopen}env\ {\isasymin}\ {\isacharunderscore}{\kern0pt}{\isacartoucheclose}\ {\isacartoucheopen}length{\isacharparenleft}{\kern0pt}nenv{\isacharparenright}{\kern0pt}\ {\isacharequal}{\kern0pt}\ {\isacharunderscore}{\kern0pt}{\isacartoucheclose}\ \isanewline
\ \ \isacommand{have}\isamarkupfalse%
\ {\isachardoublequoteopen}{\isacharquery}{\kern0pt}n{\isasymin}M{\isachardoublequoteclose}\ \isanewline
\ \ \ \ \isacommand{using}\isamarkupfalse%
\ separation{\isacharunderscore}{\kern0pt}ax\ leI\ separation{\isacharunderscore}{\kern0pt}iff\ \isacommand{by}\isamarkupfalse%
\ auto\ \isanewline
\ \ \isacommand{from}\isamarkupfalse%
\ generic\ \isanewline
\ \ \isacommand{have}\isamarkupfalse%
\ {\isachardoublequoteopen}filter{\isacharparenleft}{\kern0pt}G{\isacharparenright}{\kern0pt}{\isachardoublequoteclose}\ {\isachardoublequoteopen}G{\isasymsubseteq}P{\isachardoublequoteclose}\ \isanewline
\ \ \ \ \isacommand{unfolding}\isamarkupfalse%
\ M{\isacharunderscore}{\kern0pt}generic{\isacharunderscore}{\kern0pt}def\ filter{\isacharunderscore}{\kern0pt}def\ \isacommand{by}\isamarkupfalse%
\ simp{\isacharunderscore}{\kern0pt}all\isanewline
\ \ \isacommand{from}\isamarkupfalse%
\ {\isacartoucheopen}val{\isacharparenleft}{\kern0pt}G{\isacharcomma}{\kern0pt}{\isasympi}{\isacharparenright}{\kern0pt}\ {\isacharequal}{\kern0pt}\ c{\isacartoucheclose}\ \isanewline
\ \ \isacommand{have}\isamarkupfalse%
\ {\isachardoublequoteopen}val{\isacharparenleft}{\kern0pt}G{\isacharcomma}{\kern0pt}{\isacharquery}{\kern0pt}m{\isacharparenright}{\kern0pt}\ {\isacharequal}{\kern0pt}\isanewline
\ \ \ \ \ \ \ \ \ \ \ \ \ \ \ {\isacharbraceleft}{\kern0pt}val{\isacharparenleft}{\kern0pt}G{\isacharcomma}{\kern0pt}t{\isacharparenright}{\kern0pt}\ {\isachardot}{\kern0pt}{\isachardot}{\kern0pt}\ t{\isasymin}domain{\isacharparenleft}{\kern0pt}{\isasympi}{\isacharparenright}{\kern0pt}\ {\isacharcomma}{\kern0pt}\ {\isasymexists}q{\isasymin}P\ {\isachardot}{\kern0pt}\ \ \isanewline
\ \ \ \ \ \ \ \ \ \ \ \ \ \ \ \ \ \ \ \ {\isacharparenleft}{\kern0pt}{\isasymexists}{\isasymtheta}{\isasymin}M{\isachardot}{\kern0pt}\ {\isasymexists}p{\isasymin}P{\isachardot}{\kern0pt}\ {\isasymlangle}t{\isacharcomma}{\kern0pt}q{\isasymrangle}\ {\isacharequal}{\kern0pt}\ {\isasymlangle}{\isasymtheta}{\isacharcomma}{\kern0pt}\ p{\isasymrangle}\ {\isasymand}\ \isanewline
\ \ \ \ \ \ \ \ \ \ \ \ {\isacharparenleft}{\kern0pt}p\ {\isasymin}\ G\ {\isasymlongrightarrow}\ val{\isacharparenleft}{\kern0pt}G{\isacharcomma}{\kern0pt}\ {\isasymtheta}{\isacharparenright}{\kern0pt}\ {\isasymin}\ c\ {\isasymand}\ {\isacharparenleft}{\kern0pt}M{\isacharbrackleft}{\kern0pt}G{\isacharbrackright}{\kern0pt}{\isacharcomma}{\kern0pt}\ \ {\isacharbrackleft}{\kern0pt}val{\isacharparenleft}{\kern0pt}G{\isacharcomma}{\kern0pt}\ {\isasymtheta}{\isacharparenright}{\kern0pt}{\isacharbrackright}{\kern0pt}\ {\isacharat}{\kern0pt}\ env\ {\isacharat}{\kern0pt}\ {\isacharbrackleft}{\kern0pt}c{\isacharbrackright}{\kern0pt}\ {\isasymTurnstile}\ \ {\isasymphi}{\isacharparenright}{\kern0pt}{\isacharparenright}{\kern0pt}\ {\isasymand}\ q\ {\isasymin}\ G{\isacharparenright}{\kern0pt}{\isacharbraceright}{\kern0pt}{\isachardoublequoteclose}\isanewline
\ \ \ \ \isacommand{using}\isamarkupfalse%
\ val{\isacharunderscore}{\kern0pt}of{\isacharunderscore}{\kern0pt}name\ \isacommand{by}\isamarkupfalse%
\ auto\isanewline
\ \ \isacommand{also}\isamarkupfalse%
\ \isanewline
\ \ \isacommand{have}\isamarkupfalse%
\ {\isachardoublequoteopen}{\isachardot}{\kern0pt}{\isachardot}{\kern0pt}{\isachardot}{\kern0pt}\ {\isacharequal}{\kern0pt}\ \ {\isacharbraceleft}{\kern0pt}val{\isacharparenleft}{\kern0pt}G{\isacharcomma}{\kern0pt}t{\isacharparenright}{\kern0pt}\ {\isachardot}{\kern0pt}{\isachardot}{\kern0pt}\ t{\isasymin}domain{\isacharparenleft}{\kern0pt}{\isasympi}{\isacharparenright}{\kern0pt}\ {\isacharcomma}{\kern0pt}\ {\isasymexists}q{\isasymin}P{\isachardot}{\kern0pt}\ \isanewline
\ \ \ \ \ \ \ \ \ \ \ \ \ \ \ \ \ \ \ val{\isacharparenleft}{\kern0pt}G{\isacharcomma}{\kern0pt}\ t{\isacharparenright}{\kern0pt}\ {\isasymin}\ c\ {\isasymand}\ {\isacharparenleft}{\kern0pt}M{\isacharbrackleft}{\kern0pt}G{\isacharbrackright}{\kern0pt}{\isacharcomma}{\kern0pt}\ \ {\isacharbrackleft}{\kern0pt}val{\isacharparenleft}{\kern0pt}G{\isacharcomma}{\kern0pt}\ t{\isacharparenright}{\kern0pt}{\isacharbrackright}{\kern0pt}\ {\isacharat}{\kern0pt}\ env\ {\isacharat}{\kern0pt}\ {\isacharbrackleft}{\kern0pt}c{\isacharbrackright}{\kern0pt}\ {\isasymTurnstile}\ \ {\isasymphi}{\isacharparenright}{\kern0pt}\ {\isasymand}\ q\ {\isasymin}\ G{\isacharbraceright}{\kern0pt}{\isachardoublequoteclose}\ \isanewline
\ \ \isacommand{proof}\isamarkupfalse%
\ {\isacharminus}{\kern0pt}\isanewline
\isanewline
\ \ \ \ \isacommand{have}\isamarkupfalse%
\ {\isachardoublequoteopen}t{\isasymin}M\ {\isasymLongrightarrow}\isanewline
\ \ \ \ \ \ {\isacharparenleft}{\kern0pt}{\isasymexists}q{\isasymin}P{\isachardot}{\kern0pt}\ {\isacharparenleft}{\kern0pt}{\isasymexists}{\isasymtheta}{\isasymin}M{\isachardot}{\kern0pt}\ {\isasymexists}p{\isasymin}P{\isachardot}{\kern0pt}\ {\isasymlangle}t{\isacharcomma}{\kern0pt}q{\isasymrangle}\ {\isacharequal}{\kern0pt}\ {\isasymlangle}{\isasymtheta}{\isacharcomma}{\kern0pt}\ p{\isasymrangle}\ {\isasymand}\ \isanewline
\ \ \ \ \ \ \ \ \ \ \ \ \ \ {\isacharparenleft}{\kern0pt}p\ {\isasymin}\ G\ {\isasymlongrightarrow}\ val{\isacharparenleft}{\kern0pt}G{\isacharcomma}{\kern0pt}\ {\isasymtheta}{\isacharparenright}{\kern0pt}\ {\isasymin}\ c\ {\isasymand}\ {\isacharparenleft}{\kern0pt}M{\isacharbrackleft}{\kern0pt}G{\isacharbrackright}{\kern0pt}{\isacharcomma}{\kern0pt}\ \ {\isacharbrackleft}{\kern0pt}val{\isacharparenleft}{\kern0pt}G{\isacharcomma}{\kern0pt}\ {\isasymtheta}{\isacharparenright}{\kern0pt}{\isacharbrackright}{\kern0pt}\ {\isacharat}{\kern0pt}\ env\ {\isacharat}{\kern0pt}\ {\isacharbrackleft}{\kern0pt}c{\isacharbrackright}{\kern0pt}\ {\isasymTurnstile}\ \ {\isasymphi}{\isacharparenright}{\kern0pt}{\isacharparenright}{\kern0pt}\ {\isasymand}\ q\ {\isasymin}\ G{\isacharparenright}{\kern0pt}{\isacharparenright}{\kern0pt}\ \isanewline
\ \ \ \ \ \ {\isasymlongleftrightarrow}\ \isanewline
\ \ \ \ \ \ {\isacharparenleft}{\kern0pt}{\isasymexists}q{\isasymin}P{\isachardot}{\kern0pt}\ val{\isacharparenleft}{\kern0pt}G{\isacharcomma}{\kern0pt}\ t{\isacharparenright}{\kern0pt}\ {\isasymin}\ c\ {\isasymand}\ {\isacharparenleft}{\kern0pt}\ M{\isacharbrackleft}{\kern0pt}G{\isacharbrackright}{\kern0pt}{\isacharcomma}{\kern0pt}\ {\isacharbrackleft}{\kern0pt}val{\isacharparenleft}{\kern0pt}G{\isacharcomma}{\kern0pt}\ t{\isacharparenright}{\kern0pt}{\isacharbrackright}{\kern0pt}{\isacharat}{\kern0pt}env{\isacharat}{\kern0pt}{\isacharbrackleft}{\kern0pt}c{\isacharbrackright}{\kern0pt}{\isasymTurnstile}\ {\isasymphi}\ {\isacharparenright}{\kern0pt}\ {\isasymand}\ q\ {\isasymin}\ G{\isacharparenright}{\kern0pt}{\isachardoublequoteclose}\ \isakeyword{for}\ t\isanewline
\ \ \ \ \ \ \isacommand{by}\isamarkupfalse%
\ auto\isanewline
\ \ \ \ \isacommand{then}\isamarkupfalse%
\ \isacommand{show}\isamarkupfalse%
\ {\isacharquery}{\kern0pt}thesis\ \isacommand{using}\isamarkupfalse%
\ {\isacartoucheopen}domain{\isacharparenleft}{\kern0pt}{\isasympi}{\isacharparenright}{\kern0pt}{\isasymin}M{\isacartoucheclose}\ \isacommand{by}\isamarkupfalse%
\ {\isacharparenleft}{\kern0pt}auto\ simp\ add{\isacharcolon}{\kern0pt}transitivity{\isacharparenright}{\kern0pt}\isanewline
\ \ \isacommand{qed}\isamarkupfalse%
\isanewline
\ \ \isacommand{also}\isamarkupfalse%
\ \isanewline
\ \ \isacommand{have}\isamarkupfalse%
\ {\isachardoublequoteopen}{\isachardot}{\kern0pt}{\isachardot}{\kern0pt}{\isachardot}{\kern0pt}\ {\isacharequal}{\kern0pt}\ \ {\isacharbraceleft}{\kern0pt}x\ {\isachardot}{\kern0pt}{\isachardot}{\kern0pt}\ x{\isasymin}c\ {\isacharcomma}{\kern0pt}\ {\isasymexists}q{\isasymin}P{\isachardot}{\kern0pt}\ x\ {\isasymin}\ c\ {\isasymand}\ {\isacharparenleft}{\kern0pt}M{\isacharbrackleft}{\kern0pt}G{\isacharbrackright}{\kern0pt}{\isacharcomma}{\kern0pt}\ \ {\isacharbrackleft}{\kern0pt}x{\isacharbrackright}{\kern0pt}\ {\isacharat}{\kern0pt}\ env\ {\isacharat}{\kern0pt}\ {\isacharbrackleft}{\kern0pt}c{\isacharbrackright}{\kern0pt}\ {\isasymTurnstile}\ \ {\isasymphi}{\isacharparenright}{\kern0pt}\ {\isasymand}\ q\ {\isasymin}\ G{\isacharbraceright}{\kern0pt}{\isachardoublequoteclose}\isanewline
\ \ \isacommand{proof}\isamarkupfalse%
\isanewline
\isanewline
\ \ \ \ \isacommand{show}\isamarkupfalse%
\ {\isachardoublequoteopen}{\isachardot}{\kern0pt}{\isachardot}{\kern0pt}{\isachardot}{\kern0pt}\ {\isasymsubseteq}\ {\isacharbraceleft}{\kern0pt}x\ {\isachardot}{\kern0pt}{\isachardot}{\kern0pt}\ x{\isasymin}c\ {\isacharcomma}{\kern0pt}\ {\isasymexists}q{\isasymin}P{\isachardot}{\kern0pt}\ x\ {\isasymin}\ c\ {\isasymand}\ {\isacharparenleft}{\kern0pt}M{\isacharbrackleft}{\kern0pt}G{\isacharbrackright}{\kern0pt}{\isacharcomma}{\kern0pt}\ \ {\isacharbrackleft}{\kern0pt}x{\isacharbrackright}{\kern0pt}\ {\isacharat}{\kern0pt}\ env\ {\isacharat}{\kern0pt}\ {\isacharbrackleft}{\kern0pt}c{\isacharbrackright}{\kern0pt}\ {\isasymTurnstile}\ \ {\isasymphi}{\isacharparenright}{\kern0pt}\ {\isasymand}\ q\ {\isasymin}\ G{\isacharbraceright}{\kern0pt}{\isachardoublequoteclose}\isanewline
\ \ \ \ \ \ \isacommand{by}\isamarkupfalse%
\ auto\isanewline
\ \ \isacommand{next}\isamarkupfalse%
\ \isanewline
\ \ \ \ \isanewline
\ \ \ \ \isacommand{{\isacharbraceleft}{\kern0pt}}\isamarkupfalse%
\isanewline
\ \ \ \ \ \ \isacommand{fix}\isamarkupfalse%
\ x\isanewline
\ \ \ \ \ \ \isacommand{assume}\isamarkupfalse%
\ {\isachardoublequoteopen}x{\isasymin}{\isacharbraceleft}{\kern0pt}x\ {\isachardot}{\kern0pt}{\isachardot}{\kern0pt}\ x{\isasymin}c\ {\isacharcomma}{\kern0pt}\ {\isasymexists}q{\isasymin}P{\isachardot}{\kern0pt}\ x\ {\isasymin}\ c\ {\isasymand}\ {\isacharparenleft}{\kern0pt}M{\isacharbrackleft}{\kern0pt}G{\isacharbrackright}{\kern0pt}{\isacharcomma}{\kern0pt}\ \ {\isacharbrackleft}{\kern0pt}x{\isacharbrackright}{\kern0pt}\ {\isacharat}{\kern0pt}\ env\ {\isacharat}{\kern0pt}\ {\isacharbrackleft}{\kern0pt}c{\isacharbrackright}{\kern0pt}\ {\isasymTurnstile}\ \ {\isasymphi}{\isacharparenright}{\kern0pt}\ {\isasymand}\ q\ {\isasymin}\ G{\isacharbraceright}{\kern0pt}{\isachardoublequoteclose}\isanewline
\ \ \ \ \ \ \isacommand{then}\isamarkupfalse%
\ \isanewline
\ \ \ \ \ \ \isacommand{have}\isamarkupfalse%
\ {\isachardoublequoteopen}{\isasymexists}q{\isasymin}P{\isachardot}{\kern0pt}\ x\ {\isasymin}\ c\ {\isasymand}\ {\isacharparenleft}{\kern0pt}M{\isacharbrackleft}{\kern0pt}G{\isacharbrackright}{\kern0pt}{\isacharcomma}{\kern0pt}\ \ {\isacharbrackleft}{\kern0pt}x{\isacharbrackright}{\kern0pt}\ {\isacharat}{\kern0pt}\ env\ {\isacharat}{\kern0pt}\ {\isacharbrackleft}{\kern0pt}c{\isacharbrackright}{\kern0pt}\ {\isasymTurnstile}\ \ {\isasymphi}{\isacharparenright}{\kern0pt}\ {\isasymand}\ q\ {\isasymin}\ G{\isachardoublequoteclose}\isanewline
\ \ \ \ \ \ \ \ \isacommand{by}\isamarkupfalse%
\ simp\isanewline
\ \ \ \ \ \ \isacommand{with}\isamarkupfalse%
\ {\isacartoucheopen}val{\isacharparenleft}{\kern0pt}G{\isacharcomma}{\kern0pt}{\isasympi}{\isacharparenright}{\kern0pt}\ {\isacharequal}{\kern0pt}\ c{\isacartoucheclose}\ \ \isanewline
\ \ \ \ \ \ \isacommand{have}\isamarkupfalse%
\ {\isachardoublequoteopen}{\isasymexists}q{\isasymin}P{\isachardot}{\kern0pt}\ {\isasymexists}t{\isasymin}domain{\isacharparenleft}{\kern0pt}{\isasympi}{\isacharparenright}{\kern0pt}{\isachardot}{\kern0pt}\ val{\isacharparenleft}{\kern0pt}G{\isacharcomma}{\kern0pt}t{\isacharparenright}{\kern0pt}\ {\isacharequal}{\kern0pt}x\ {\isasymand}\ {\isacharparenleft}{\kern0pt}M{\isacharbrackleft}{\kern0pt}G{\isacharbrackright}{\kern0pt}{\isacharcomma}{\kern0pt}\ \ {\isacharbrackleft}{\kern0pt}val{\isacharparenleft}{\kern0pt}G{\isacharcomma}{\kern0pt}t{\isacharparenright}{\kern0pt}{\isacharbrackright}{\kern0pt}\ {\isacharat}{\kern0pt}\ env\ {\isacharat}{\kern0pt}\ {\isacharbrackleft}{\kern0pt}c{\isacharbrackright}{\kern0pt}\ {\isasymTurnstile}\ \ {\isasymphi}{\isacharparenright}{\kern0pt}\ {\isasymand}\ q\ {\isasymin}\ G{\isachardoublequoteclose}\ \isanewline
\ \ \ \ \ \ \ \ \isacommand{using}\isamarkupfalse%
\ Sep{\isacharunderscore}{\kern0pt}and{\isacharunderscore}{\kern0pt}Replace\ elem{\isacharunderscore}{\kern0pt}of{\isacharunderscore}{\kern0pt}val\ \isacommand{by}\isamarkupfalse%
\ auto\isanewline
\ \ \ \ \isacommand{{\isacharbraceright}{\kern0pt}}\isamarkupfalse%
\isanewline
\ \ \ \ \isacommand{then}\isamarkupfalse%
\ \isanewline
\ \ \ \ \isacommand{show}\isamarkupfalse%
\ {\isachardoublequoteopen}\ {\isacharbraceleft}{\kern0pt}x\ {\isachardot}{\kern0pt}{\isachardot}{\kern0pt}\ x{\isasymin}c\ {\isacharcomma}{\kern0pt}\ {\isasymexists}q{\isasymin}P{\isachardot}{\kern0pt}\ x\ {\isasymin}\ c\ {\isasymand}\ {\isacharparenleft}{\kern0pt}M{\isacharbrackleft}{\kern0pt}G{\isacharbrackright}{\kern0pt}{\isacharcomma}{\kern0pt}\ \ {\isacharbrackleft}{\kern0pt}x{\isacharbrackright}{\kern0pt}\ {\isacharat}{\kern0pt}\ env\ {\isacharat}{\kern0pt}\ {\isacharbrackleft}{\kern0pt}c{\isacharbrackright}{\kern0pt}\ {\isasymTurnstile}\ \ {\isasymphi}{\isacharparenright}{\kern0pt}\ {\isasymand}\ q\ {\isasymin}\ G{\isacharbraceright}{\kern0pt}\ {\isasymsubseteq}\ {\isachardot}{\kern0pt}{\isachardot}{\kern0pt}{\isachardot}{\kern0pt}{\isachardoublequoteclose}\isanewline
\ \ \ \ \ \ \isacommand{using}\isamarkupfalse%
\ SepReplace{\isacharunderscore}{\kern0pt}iff\ \isacommand{by}\isamarkupfalse%
\ force\isanewline
\ \ \isacommand{qed}\isamarkupfalse%
\isanewline
\ \ \isacommand{also}\isamarkupfalse%
\ \isanewline
\ \ \isacommand{have}\isamarkupfalse%
\ {\isachardoublequoteopen}\ {\isachardot}{\kern0pt}{\isachardot}{\kern0pt}{\isachardot}{\kern0pt}\ {\isacharequal}{\kern0pt}\ {\isacharbraceleft}{\kern0pt}x{\isasymin}c{\isachardot}{\kern0pt}\ {\isacharparenleft}{\kern0pt}M{\isacharbrackleft}{\kern0pt}G{\isacharbrackright}{\kern0pt}{\isacharcomma}{\kern0pt}\ {\isacharbrackleft}{\kern0pt}x{\isacharbrackright}{\kern0pt}\ {\isacharat}{\kern0pt}\ env\ {\isacharat}{\kern0pt}\ {\isacharbrackleft}{\kern0pt}c{\isacharbrackright}{\kern0pt}\ {\isasymTurnstile}\ {\isasymphi}{\isacharparenright}{\kern0pt}{\isacharbraceright}{\kern0pt}{\isachardoublequoteclose}\isanewline
\ \ \ \ \isacommand{using}\isamarkupfalse%
\ {\isacartoucheopen}G{\isasymsubseteq}P{\isacartoucheclose}\ G{\isacharunderscore}{\kern0pt}nonempty\ \isacommand{by}\isamarkupfalse%
\ force\isanewline
\ \ \isacommand{finally}\isamarkupfalse%
\ \isanewline
\ \ \isacommand{have}\isamarkupfalse%
\ val{\isacharunderscore}{\kern0pt}m{\isacharcolon}{\kern0pt}\ {\isachardoublequoteopen}val{\isacharparenleft}{\kern0pt}G{\isacharcomma}{\kern0pt}{\isacharquery}{\kern0pt}m{\isacharparenright}{\kern0pt}\ {\isacharequal}{\kern0pt}\ {\isacharbraceleft}{\kern0pt}x{\isasymin}c{\isachardot}{\kern0pt}\ {\isacharparenleft}{\kern0pt}M{\isacharbrackleft}{\kern0pt}G{\isacharbrackright}{\kern0pt}{\isacharcomma}{\kern0pt}\ {\isacharbrackleft}{\kern0pt}x{\isacharbrackright}{\kern0pt}\ {\isacharat}{\kern0pt}\ env\ {\isacharat}{\kern0pt}\ {\isacharbrackleft}{\kern0pt}c{\isacharbrackright}{\kern0pt}\ {\isasymTurnstile}\ {\isasymphi}{\isacharparenright}{\kern0pt}{\isacharbraceright}{\kern0pt}{\isachardoublequoteclose}\ \isacommand{by}\isamarkupfalse%
\ simp\isanewline
\ \ \isacommand{have}\isamarkupfalse%
\ {\isachardoublequoteopen}val{\isacharparenleft}{\kern0pt}G{\isacharcomma}{\kern0pt}{\isacharquery}{\kern0pt}m{\isacharparenright}{\kern0pt}\ {\isasymsubseteq}\ val{\isacharparenleft}{\kern0pt}G{\isacharcomma}{\kern0pt}{\isacharquery}{\kern0pt}n{\isacharparenright}{\kern0pt}{\isachardoublequoteclose}\ \isanewline
\ \ \isacommand{proof}\isamarkupfalse%
\isanewline
\ \ \ \ \isacommand{fix}\isamarkupfalse%
\ x\isanewline
\ \ \ \ \isacommand{assume}\isamarkupfalse%
\ {\isachardoublequoteopen}x\ {\isasymin}\ val{\isacharparenleft}{\kern0pt}G{\isacharcomma}{\kern0pt}{\isacharquery}{\kern0pt}m{\isacharparenright}{\kern0pt}{\isachardoublequoteclose}\isanewline
\ \ \ \ \isacommand{with}\isamarkupfalse%
\ val{\isacharunderscore}{\kern0pt}m\ \isanewline
\ \ \ \ \isacommand{have}\isamarkupfalse%
\ Eq{\isadigit{4}}{\isacharcolon}{\kern0pt}\ {\isachardoublequoteopen}x\ {\isasymin}\ {\isacharbraceleft}{\kern0pt}x{\isasymin}c{\isachardot}{\kern0pt}\ {\isacharparenleft}{\kern0pt}M{\isacharbrackleft}{\kern0pt}G{\isacharbrackright}{\kern0pt}{\isacharcomma}{\kern0pt}\ {\isacharbrackleft}{\kern0pt}x{\isacharbrackright}{\kern0pt}\ {\isacharat}{\kern0pt}\ env\ {\isacharat}{\kern0pt}\ {\isacharbrackleft}{\kern0pt}c{\isacharbrackright}{\kern0pt}\ {\isasymTurnstile}\ {\isasymphi}{\isacharparenright}{\kern0pt}{\isacharbraceright}{\kern0pt}{\isachardoublequoteclose}\ \isacommand{by}\isamarkupfalse%
\ simp\isanewline
\ \ \ \ \isacommand{with}\isamarkupfalse%
\ {\isacartoucheopen}val{\isacharparenleft}{\kern0pt}G{\isacharcomma}{\kern0pt}{\isasympi}{\isacharparenright}{\kern0pt}\ {\isacharequal}{\kern0pt}\ c{\isacartoucheclose}\isanewline
\ \ \ \ \isacommand{have}\isamarkupfalse%
\ {\isachardoublequoteopen}x\ {\isasymin}\ val{\isacharparenleft}{\kern0pt}G{\isacharcomma}{\kern0pt}{\isasympi}{\isacharparenright}{\kern0pt}{\isachardoublequoteclose}\ \isacommand{by}\isamarkupfalse%
\ simp\isanewline
\ \ \ \ \isacommand{then}\isamarkupfalse%
\ \isanewline
\ \ \ \ \isacommand{have}\isamarkupfalse%
\ {\isachardoublequoteopen}{\isasymexists}{\isasymtheta}{\isachardot}{\kern0pt}\ {\isasymexists}q{\isasymin}G{\isachardot}{\kern0pt}\ {\isasymlangle}{\isasymtheta}{\isacharcomma}{\kern0pt}q{\isasymrangle}{\isasymin}{\isasympi}\ {\isasymand}\ val{\isacharparenleft}{\kern0pt}G{\isacharcomma}{\kern0pt}{\isasymtheta}{\isacharparenright}{\kern0pt}\ {\isacharequal}{\kern0pt}x{\isachardoublequoteclose}\ \isanewline
\ \ \ \ \ \ \isacommand{using}\isamarkupfalse%
\ elem{\isacharunderscore}{\kern0pt}of{\isacharunderscore}{\kern0pt}val{\isacharunderscore}{\kern0pt}pair\ \isacommand{by}\isamarkupfalse%
\ auto\isanewline
\ \ \ \ \isacommand{then}\isamarkupfalse%
\ \isacommand{obtain}\isamarkupfalse%
\ {\isasymtheta}\ q\ \isakeyword{where}\isanewline
\ \ \ \ \ \ {\isachardoublequoteopen}{\isasymlangle}{\isasymtheta}{\isacharcomma}{\kern0pt}q{\isasymrangle}{\isasymin}{\isasympi}{\isachardoublequoteclose}\ {\isachardoublequoteopen}q{\isasymin}G{\isachardoublequoteclose}\ {\isachardoublequoteopen}val{\isacharparenleft}{\kern0pt}G{\isacharcomma}{\kern0pt}{\isasymtheta}{\isacharparenright}{\kern0pt}{\isacharequal}{\kern0pt}x{\isachardoublequoteclose}\ \isacommand{by}\isamarkupfalse%
\ auto\isanewline
\ \ \ \ \isacommand{from}\isamarkupfalse%
\ {\isacartoucheopen}{\isasymlangle}{\isasymtheta}{\isacharcomma}{\kern0pt}q{\isasymrangle}{\isasymin}{\isasympi}{\isacartoucheclose}\isanewline
\ \ \ \ \isacommand{have}\isamarkupfalse%
\ {\isachardoublequoteopen}{\isasymtheta}{\isasymin}M{\isachardoublequoteclose}\isanewline
\ \ \ \ \ \ \isacommand{using}\isamarkupfalse%
\ domain{\isacharunderscore}{\kern0pt}trans{\isacharbrackleft}{\kern0pt}OF\ trans{\isacharunderscore}{\kern0pt}M\ {\isacartoucheopen}{\isasympi}{\isasymin}{\isacharunderscore}{\kern0pt}{\isacartoucheclose}{\isacharbrackright}{\kern0pt}\ \isacommand{by}\isamarkupfalse%
\ auto\isanewline
\ \ \ \ \isacommand{with}\isamarkupfalse%
\ {\isacartoucheopen}{\isasympi}{\isasymin}M{\isacartoucheclose}\ {\isacartoucheopen}nenv\ {\isasymin}\ {\isacharunderscore}{\kern0pt}{\isacartoucheclose}\ {\isacartoucheopen}env\ {\isacharequal}{\kern0pt}\ {\isacharunderscore}{\kern0pt}{\isacartoucheclose}\isanewline
\ \ \ \ \isacommand{have}\isamarkupfalse%
\ {\isachardoublequoteopen}{\isacharbrackleft}{\kern0pt}val{\isacharparenleft}{\kern0pt}G{\isacharcomma}{\kern0pt}{\isasymtheta}{\isacharparenright}{\kern0pt}{\isacharcomma}{\kern0pt}\ val{\isacharparenleft}{\kern0pt}G{\isacharcomma}{\kern0pt}{\isasympi}{\isacharparenright}{\kern0pt}{\isacharbrackright}{\kern0pt}\ {\isacharat}{\kern0pt}\ env\ {\isasymin}list{\isacharparenleft}{\kern0pt}M{\isacharbrackleft}{\kern0pt}G{\isacharbrackright}{\kern0pt}{\isacharparenright}{\kern0pt}{\isachardoublequoteclose}\ \isanewline
\ \ \ \ \ \ \isacommand{using}\isamarkupfalse%
\ GenExt{\isacharunderscore}{\kern0pt}def\ \isacommand{by}\isamarkupfalse%
\ auto\isanewline
\ \ \ \ \isacommand{with}\isamarkupfalse%
\ \ Eq{\isadigit{4}}\ {\isacartoucheopen}val{\isacharparenleft}{\kern0pt}G{\isacharcomma}{\kern0pt}{\isasymtheta}{\isacharparenright}{\kern0pt}{\isacharequal}{\kern0pt}x{\isacartoucheclose}\ {\isacartoucheopen}val{\isacharparenleft}{\kern0pt}G{\isacharcomma}{\kern0pt}{\isasympi}{\isacharparenright}{\kern0pt}\ {\isacharequal}{\kern0pt}\ c{\isacartoucheclose}\ {\isacartoucheopen}x\ {\isasymin}\ val{\isacharparenleft}{\kern0pt}G{\isacharcomma}{\kern0pt}{\isasympi}{\isacharparenright}{\kern0pt}{\isacartoucheclose}\ nth\ {\isacartoucheopen}{\isasymtheta}{\isasymin}M{\isacartoucheclose}\isanewline
\ \ \ \ \isacommand{have}\isamarkupfalse%
\ Eq{\isadigit{5}}{\isacharcolon}{\kern0pt}\ {\isachardoublequoteopen}M{\isacharbrackleft}{\kern0pt}G{\isacharbrackright}{\kern0pt}{\isacharcomma}{\kern0pt}\ \ {\isacharbrackleft}{\kern0pt}val{\isacharparenleft}{\kern0pt}G{\isacharcomma}{\kern0pt}{\isasymtheta}{\isacharparenright}{\kern0pt}{\isacharbrackright}{\kern0pt}\ {\isacharat}{\kern0pt}\ env\ {\isacharat}{\kern0pt}{\isacharbrackleft}{\kern0pt}val{\isacharparenleft}{\kern0pt}G{\isacharcomma}{\kern0pt}{\isasympi}{\isacharparenright}{\kern0pt}{\isacharbrackright}{\kern0pt}\ {\isasymTurnstile}\ And{\isacharparenleft}{\kern0pt}Member{\isacharparenleft}{\kern0pt}{\isadigit{0}}{\isacharcomma}{\kern0pt}{\isadigit{1}}\ {\isacharhash}{\kern0pt}{\isacharplus}{\kern0pt}\ length{\isacharparenleft}{\kern0pt}env{\isacharparenright}{\kern0pt}{\isacharparenright}{\kern0pt}{\isacharcomma}{\kern0pt}{\isasymphi}{\isacharparenright}{\kern0pt}{\isachardoublequoteclose}\ \isanewline
\ \ \ \ \ \ \isacommand{by}\isamarkupfalse%
\ auto\isanewline
\ \ \ \ \ \ \ \ \isanewline
\ \ \ \ \isacommand{with}\isamarkupfalse%
\ {\isacartoucheopen}{\isasymtheta}{\isasymin}M{\isacartoucheclose}\ {\isacartoucheopen}{\isasympi}{\isasymin}M{\isacartoucheclose}\ \ Eq{\isadigit{5}}\ {\isacartoucheopen}M{\isacharunderscore}{\kern0pt}generic{\isacharparenleft}{\kern0pt}G{\isacharparenright}{\kern0pt}{\isacartoucheclose}\ {\isacartoucheopen}{\isasymphi}{\isasymin}formula{\isacartoucheclose}\ {\isacartoucheopen}nenv\ {\isasymin}\ {\isacharunderscore}{\kern0pt}\ {\isacartoucheclose}\ {\isacartoucheopen}env\ {\isacharequal}{\kern0pt}\ {\isacharunderscore}{\kern0pt}\ {\isacartoucheclose}\ map{\isacharunderscore}{\kern0pt}nenv\ \isanewline
\ \ \ \ \ \ {\isacartoucheopen}arity{\isacharparenleft}{\kern0pt}{\isacharquery}{\kern0pt}{\isasymchi}{\isacharparenright}{\kern0pt}\ {\isasymle}\ length{\isacharparenleft}{\kern0pt}{\isacharbrackleft}{\kern0pt}{\isasymtheta}{\isacharbrackright}{\kern0pt}\ {\isacharat}{\kern0pt}\ nenv\ {\isacharat}{\kern0pt}\ {\isacharbrackleft}{\kern0pt}{\isasympi}{\isacharbrackright}{\kern0pt}{\isacharparenright}{\kern0pt}{\isacartoucheclose}\isanewline
\ \ \ \ \isacommand{have}\isamarkupfalse%
\ {\isachardoublequoteopen}{\isacharparenleft}{\kern0pt}{\isasymexists}r{\isasymin}G{\isachardot}{\kern0pt}\ M{\isacharcomma}{\kern0pt}\ \ {\isacharbrackleft}{\kern0pt}r{\isacharcomma}{\kern0pt}P{\isacharcomma}{\kern0pt}leq{\isacharcomma}{\kern0pt}one{\isacharcomma}{\kern0pt}{\isasymtheta}{\isacharbrackright}{\kern0pt}\ {\isacharat}{\kern0pt}\ nenv\ {\isacharat}{\kern0pt}{\isacharbrackleft}{\kern0pt}{\isasympi}{\isacharbrackright}{\kern0pt}\ {\isasymTurnstile}\ forces{\isacharparenleft}{\kern0pt}{\isacharquery}{\kern0pt}{\isasymchi}{\isacharparenright}{\kern0pt}{\isacharparenright}{\kern0pt}{\isachardoublequoteclose}\isanewline
\ \ \ \ \ \ \isacommand{using}\isamarkupfalse%
\ truth{\isacharunderscore}{\kern0pt}lemma\ \ \isanewline
\ \ \ \ \ \ \isacommand{by}\isamarkupfalse%
\ auto\isanewline
\ \ \ \ \isacommand{then}\isamarkupfalse%
\ \isacommand{obtain}\isamarkupfalse%
\ r\ \isakeyword{where}\ \ \ \ \ \ \isanewline
\ \ \ \ \ \ {\isachardoublequoteopen}r{\isasymin}G{\isachardoublequoteclose}\ {\isachardoublequoteopen}M{\isacharcomma}{\kern0pt}\ \ {\isacharbrackleft}{\kern0pt}r{\isacharcomma}{\kern0pt}P{\isacharcomma}{\kern0pt}leq{\isacharcomma}{\kern0pt}one{\isacharcomma}{\kern0pt}{\isasymtheta}{\isacharbrackright}{\kern0pt}\ {\isacharat}{\kern0pt}\ nenv\ {\isacharat}{\kern0pt}\ {\isacharbrackleft}{\kern0pt}{\isasympi}{\isacharbrackright}{\kern0pt}\ {\isasymTurnstile}\ forces{\isacharparenleft}{\kern0pt}{\isacharquery}{\kern0pt}{\isasymchi}{\isacharparenright}{\kern0pt}{\isachardoublequoteclose}\ \isacommand{by}\isamarkupfalse%
\ auto\isanewline
\ \ \ \ \isacommand{with}\isamarkupfalse%
\ {\isacartoucheopen}filter{\isacharparenleft}{\kern0pt}G{\isacharparenright}{\kern0pt}{\isacartoucheclose}\ \isakeyword{and}\ {\isacartoucheopen}q{\isasymin}G{\isacartoucheclose}\ \isacommand{obtain}\isamarkupfalse%
\ p\ \isakeyword{where}\isanewline
\ \ \ \ \ \ {\isachardoublequoteopen}p{\isasymin}G{\isachardoublequoteclose}\ {\isachardoublequoteopen}p{\isasympreceq}q{\isachardoublequoteclose}\ {\isachardoublequoteopen}p{\isasympreceq}r{\isachardoublequoteclose}\ \isanewline
\ \ \ \ \ \ \isacommand{unfolding}\isamarkupfalse%
\ filter{\isacharunderscore}{\kern0pt}def\ compat{\isacharunderscore}{\kern0pt}in{\isacharunderscore}{\kern0pt}def\ \isacommand{by}\isamarkupfalse%
\ force\isanewline
\ \ \ \ \isacommand{with}\isamarkupfalse%
\ {\isacartoucheopen}r{\isasymin}G{\isacartoucheclose}\ \ {\isacartoucheopen}q{\isasymin}G{\isacartoucheclose}\ {\isacartoucheopen}G{\isasymsubseteq}P{\isacartoucheclose}\ \isanewline
\ \ \ \ \isacommand{have}\isamarkupfalse%
\ {\isachardoublequoteopen}p{\isasymin}P{\isachardoublequoteclose}\ {\isachardoublequoteopen}r{\isasymin}P{\isachardoublequoteclose}\ {\isachardoublequoteopen}q{\isasymin}P{\isachardoublequoteclose}\ {\isachardoublequoteopen}p{\isasymin}M{\isachardoublequoteclose}\isanewline
\ \ \ \ \ \ \isacommand{using}\isamarkupfalse%
\ \ P{\isacharunderscore}{\kern0pt}in{\isacharunderscore}{\kern0pt}M\ \ \isacommand{by}\isamarkupfalse%
\ {\isacharparenleft}{\kern0pt}auto\ simp\ add{\isacharcolon}{\kern0pt}transitivity{\isacharparenright}{\kern0pt}\isanewline
\ \ \ \ \isacommand{with}\isamarkupfalse%
\ {\isacartoucheopen}{\isasymphi}{\isasymin}formula{\isacartoucheclose}\ {\isacartoucheopen}{\isasymtheta}{\isasymin}M{\isacartoucheclose}\ {\isacartoucheopen}{\isasympi}{\isasymin}M{\isacartoucheclose}\ \ {\isacartoucheopen}p{\isasympreceq}r{\isacartoucheclose}\ {\isacartoucheopen}nenv\ {\isasymin}\ {\isacharunderscore}{\kern0pt}{\isacartoucheclose}\ {\isacartoucheopen}arity{\isacharparenleft}{\kern0pt}{\isacharquery}{\kern0pt}{\isasymchi}{\isacharparenright}{\kern0pt}\ {\isasymle}\ length{\isacharparenleft}{\kern0pt}{\isacharbrackleft}{\kern0pt}{\isasymtheta}{\isacharbrackright}{\kern0pt}\ {\isacharat}{\kern0pt}\ nenv\ {\isacharat}{\kern0pt}\ {\isacharbrackleft}{\kern0pt}{\isasympi}{\isacharbrackright}{\kern0pt}{\isacharparenright}{\kern0pt}{\isacartoucheclose}\isanewline
\ \ \ \ \ \ {\isacartoucheopen}M{\isacharcomma}{\kern0pt}\ {\isacharbrackleft}{\kern0pt}r{\isacharcomma}{\kern0pt}P{\isacharcomma}{\kern0pt}leq{\isacharcomma}{\kern0pt}one{\isacharcomma}{\kern0pt}{\isasymtheta}{\isacharbrackright}{\kern0pt}\ {\isacharat}{\kern0pt}\ nenv\ {\isacharat}{\kern0pt}\ {\isacharbrackleft}{\kern0pt}{\isasympi}{\isacharbrackright}{\kern0pt}\ {\isasymTurnstile}\ forces{\isacharparenleft}{\kern0pt}{\isacharquery}{\kern0pt}{\isasymchi}{\isacharparenright}{\kern0pt}{\isacartoucheclose}\ {\isacartoucheopen}env{\isasymin}{\isacharunderscore}{\kern0pt}{\isacartoucheclose}\isanewline
\ \ \ \ \isacommand{have}\isamarkupfalse%
\ {\isachardoublequoteopen}M{\isacharcomma}{\kern0pt}\ \ {\isacharbrackleft}{\kern0pt}p{\isacharcomma}{\kern0pt}P{\isacharcomma}{\kern0pt}leq{\isacharcomma}{\kern0pt}one{\isacharcomma}{\kern0pt}{\isasymtheta}{\isacharbrackright}{\kern0pt}\ {\isacharat}{\kern0pt}\ nenv\ {\isacharat}{\kern0pt}\ {\isacharbrackleft}{\kern0pt}{\isasympi}{\isacharbrackright}{\kern0pt}\ {\isasymTurnstile}\ forces{\isacharparenleft}{\kern0pt}{\isacharquery}{\kern0pt}{\isasymchi}{\isacharparenright}{\kern0pt}{\isachardoublequoteclose}\isanewline
\ \ \ \ \ \ \isacommand{using}\isamarkupfalse%
\ strengthening{\isacharunderscore}{\kern0pt}lemma\ \isanewline
\ \ \ \ \ \ \isacommand{by}\isamarkupfalse%
\ simp\isanewline
\ \ \ \ \isacommand{with}\isamarkupfalse%
\ {\isacartoucheopen}p{\isasymin}P{\isacartoucheclose}\ {\isacartoucheopen}{\isasymphi}{\isasymin}formula{\isacartoucheclose}\ {\isacartoucheopen}{\isasymtheta}{\isasymin}M{\isacartoucheclose}\ {\isacartoucheopen}{\isasympi}{\isasymin}M{\isacartoucheclose}\ {\isacartoucheopen}nenv\ {\isasymin}\ {\isacharunderscore}{\kern0pt}{\isacartoucheclose}\ {\isacartoucheopen}arity{\isacharparenleft}{\kern0pt}{\isacharquery}{\kern0pt}{\isasymchi}{\isacharparenright}{\kern0pt}\ {\isasymle}\ length{\isacharparenleft}{\kern0pt}{\isacharbrackleft}{\kern0pt}{\isasymtheta}{\isacharbrackright}{\kern0pt}\ {\isacharat}{\kern0pt}\ nenv\ {\isacharat}{\kern0pt}\ {\isacharbrackleft}{\kern0pt}{\isasympi}{\isacharbrackright}{\kern0pt}{\isacharparenright}{\kern0pt}{\isacartoucheclose}\isanewline
\ \ \ \ \isacommand{have}\isamarkupfalse%
\ {\isachardoublequoteopen}{\isasymforall}F{\isachardot}{\kern0pt}\ M{\isacharunderscore}{\kern0pt}generic{\isacharparenleft}{\kern0pt}F{\isacharparenright}{\kern0pt}\ {\isasymand}\ p\ {\isasymin}\ F\ {\isasymlongrightarrow}\ \isanewline
\ \ \ \ \ \ \ \ \ \ \ \ \ \ \ \ \ M{\isacharbrackleft}{\kern0pt}F{\isacharbrackright}{\kern0pt}{\isacharcomma}{\kern0pt}\ \ \ map{\isacharparenleft}{\kern0pt}val{\isacharparenleft}{\kern0pt}F{\isacharparenright}{\kern0pt}{\isacharcomma}{\kern0pt}\ {\isacharbrackleft}{\kern0pt}{\isasymtheta}{\isacharbrackright}{\kern0pt}\ {\isacharat}{\kern0pt}\ nenv\ {\isacharat}{\kern0pt}{\isacharbrackleft}{\kern0pt}{\isasympi}{\isacharbrackright}{\kern0pt}{\isacharparenright}{\kern0pt}\ {\isasymTurnstile}\ \ {\isacharquery}{\kern0pt}{\isasymchi}{\isachardoublequoteclose}\isanewline
\ \ \ \ \ \ \isacommand{using}\isamarkupfalse%
\ definition{\isacharunderscore}{\kern0pt}of{\isacharunderscore}{\kern0pt}forcing\isanewline
\ \ \ \ \ \ \isacommand{by}\isamarkupfalse%
\ simp\isanewline
\ \ \ \ \isacommand{with}\isamarkupfalse%
\ {\isacartoucheopen}p{\isasymin}P{\isacartoucheclose}\ {\isacartoucheopen}{\isasymtheta}{\isasymin}M{\isacartoucheclose}\ \ \isanewline
\ \ \ \ \isacommand{have}\isamarkupfalse%
\ Eq{\isadigit{6}}{\isacharcolon}{\kern0pt}\ {\isachardoublequoteopen}{\isasymexists}{\isasymtheta}{\isacharprime}{\kern0pt}{\isasymin}M{\isachardot}{\kern0pt}\ {\isasymexists}p{\isacharprime}{\kern0pt}{\isasymin}P{\isachardot}{\kern0pt}\ \ {\isasymlangle}{\isasymtheta}{\isacharcomma}{\kern0pt}p{\isasymrangle}\ {\isacharequal}{\kern0pt}\ {\isacharless}{\kern0pt}{\isasymtheta}{\isacharprime}{\kern0pt}{\isacharcomma}{\kern0pt}p{\isacharprime}{\kern0pt}{\isachargreater}{\kern0pt}\ {\isasymand}\ {\isacharparenleft}{\kern0pt}{\isasymforall}F{\isachardot}{\kern0pt}\ M{\isacharunderscore}{\kern0pt}generic{\isacharparenleft}{\kern0pt}F{\isacharparenright}{\kern0pt}\ {\isasymand}\ p{\isacharprime}{\kern0pt}\ {\isasymin}\ F\ {\isasymlongrightarrow}\ \isanewline
\ \ \ \ \ \ \ \ \ \ \ \ \ \ \ \ \ M{\isacharbrackleft}{\kern0pt}F{\isacharbrackright}{\kern0pt}{\isacharcomma}{\kern0pt}\ \ \ map{\isacharparenleft}{\kern0pt}val{\isacharparenleft}{\kern0pt}F{\isacharparenright}{\kern0pt}{\isacharcomma}{\kern0pt}\ {\isacharbrackleft}{\kern0pt}{\isasymtheta}{\isacharprime}{\kern0pt}{\isacharbrackright}{\kern0pt}\ {\isacharat}{\kern0pt}\ nenv\ {\isacharat}{\kern0pt}\ {\isacharbrackleft}{\kern0pt}{\isasympi}{\isacharbrackright}{\kern0pt}{\isacharparenright}{\kern0pt}\ {\isasymTurnstile}\ \ {\isacharquery}{\kern0pt}{\isasymchi}{\isacharparenright}{\kern0pt}{\isachardoublequoteclose}\ \isacommand{by}\isamarkupfalse%
\ auto\isanewline
\ \ \ \ \isacommand{from}\isamarkupfalse%
\ {\isacartoucheopen}{\isasympi}{\isasymin}M{\isacartoucheclose}\ {\isacartoucheopen}{\isasymlangle}{\isasymtheta}{\isacharcomma}{\kern0pt}q{\isasymrangle}{\isasymin}{\isasympi}{\isacartoucheclose}\ \isanewline
\ \ \ \ \isacommand{have}\isamarkupfalse%
\ {\isachardoublequoteopen}{\isasymlangle}{\isasymtheta}{\isacharcomma}{\kern0pt}q{\isasymrangle}\ {\isasymin}\ M{\isachardoublequoteclose}\ \isacommand{by}\isamarkupfalse%
\ {\isacharparenleft}{\kern0pt}simp\ add{\isacharcolon}{\kern0pt}transitivity{\isacharparenright}{\kern0pt}\isanewline
\ \ \ \ \isacommand{from}\isamarkupfalse%
\ {\isacartoucheopen}{\isasymlangle}{\isasymtheta}{\isacharcomma}{\kern0pt}q{\isasymrangle}{\isasymin}{\isasympi}{\isacartoucheclose}\ {\isacartoucheopen}{\isasymtheta}{\isasymin}M{\isacartoucheclose}\ {\isacartoucheopen}p{\isasymin}P{\isacartoucheclose}\ \ {\isacartoucheopen}p{\isasymin}M{\isacartoucheclose}\ \isanewline
\ \ \ \ \isacommand{have}\isamarkupfalse%
\ {\isachardoublequoteopen}{\isasymlangle}{\isasymtheta}{\isacharcomma}{\kern0pt}p{\isasymrangle}{\isasymin}M{\isachardoublequoteclose}\ {\isachardoublequoteopen}{\isasymlangle}{\isasymtheta}{\isacharcomma}{\kern0pt}p{\isasymrangle}{\isasymin}domain{\isacharparenleft}{\kern0pt}{\isasympi}{\isacharparenright}{\kern0pt}{\isasymtimes}P{\isachardoublequoteclose}\ \isanewline
\ \ \ \ \ \ \isacommand{using}\isamarkupfalse%
\ tuples{\isacharunderscore}{\kern0pt}in{\isacharunderscore}{\kern0pt}M\ \isacommand{by}\isamarkupfalse%
\ auto\isanewline
\ \ \ \ \isacommand{with}\isamarkupfalse%
\ {\isacartoucheopen}{\isasymtheta}{\isasymin}M{\isacartoucheclose}\ Eq{\isadigit{6}}\ {\isacartoucheopen}p{\isasymin}P{\isacartoucheclose}\isanewline
\ \ \ \ \isacommand{have}\isamarkupfalse%
\ {\isachardoublequoteopen}M{\isacharcomma}{\kern0pt}\ {\isacharbrackleft}{\kern0pt}{\isasymlangle}{\isasymtheta}{\isacharcomma}{\kern0pt}p{\isasymrangle}{\isacharbrackright}{\kern0pt}\ {\isacharat}{\kern0pt}\ {\isacharquery}{\kern0pt}Pl{\isadigit{1}}\ {\isacharat}{\kern0pt}\ {\isacharbrackleft}{\kern0pt}{\isasympi}{\isacharbrackright}{\kern0pt}\ {\isacharat}{\kern0pt}\ nenv\ {\isasymTurnstile}\ {\isacharquery}{\kern0pt}{\isasympsi}{\isachardoublequoteclose}\isanewline
\ \ \ \ \ \ \isacommand{using}\isamarkupfalse%
\ Equivalence\ \ \isacommand{by}\isamarkupfalse%
\ auto\isanewline
\ \ \ \ \isacommand{with}\isamarkupfalse%
\ {\isacartoucheopen}{\isasymlangle}{\isasymtheta}{\isacharcomma}{\kern0pt}p{\isasymrangle}{\isasymin}domain{\isacharparenleft}{\kern0pt}{\isasympi}{\isacharparenright}{\kern0pt}{\isasymtimes}P{\isacartoucheclose}\ \isanewline
\ \ \ \ \isacommand{have}\isamarkupfalse%
\ {\isachardoublequoteopen}{\isasymlangle}{\isasymtheta}{\isacharcomma}{\kern0pt}p{\isasymrangle}{\isasymin}{\isacharquery}{\kern0pt}n{\isachardoublequoteclose}\ \isacommand{by}\isamarkupfalse%
\ simp\isanewline
\ \ \ \ \isacommand{with}\isamarkupfalse%
\ {\isacartoucheopen}p{\isasymin}G{\isacartoucheclose}\ {\isacartoucheopen}p{\isasymin}P{\isacartoucheclose}\ \isanewline
\ \ \ \ \isacommand{have}\isamarkupfalse%
\ {\isachardoublequoteopen}val{\isacharparenleft}{\kern0pt}G{\isacharcomma}{\kern0pt}{\isasymtheta}{\isacharparenright}{\kern0pt}{\isasymin}val{\isacharparenleft}{\kern0pt}G{\isacharcomma}{\kern0pt}{\isacharquery}{\kern0pt}n{\isacharparenright}{\kern0pt}{\isachardoublequoteclose}\ \isanewline
\ \ \ \ \ \ \isacommand{using}\isamarkupfalse%
\ \ val{\isacharunderscore}{\kern0pt}of{\isacharunderscore}{\kern0pt}elem{\isacharbrackleft}{\kern0pt}of\ {\isasymtheta}\ p{\isacharbrackright}{\kern0pt}\ \isacommand{by}\isamarkupfalse%
\ simp\isanewline
\ \ \ \ \isacommand{with}\isamarkupfalse%
\ {\isacartoucheopen}val{\isacharparenleft}{\kern0pt}G{\isacharcomma}{\kern0pt}{\isasymtheta}{\isacharparenright}{\kern0pt}{\isacharequal}{\kern0pt}x{\isacartoucheclose}\ \isanewline
\ \ \ \ \isacommand{show}\isamarkupfalse%
\ {\isachardoublequoteopen}x{\isasymin}val{\isacharparenleft}{\kern0pt}G{\isacharcomma}{\kern0pt}{\isacharquery}{\kern0pt}n{\isacharparenright}{\kern0pt}{\isachardoublequoteclose}\ \isacommand{by}\isamarkupfalse%
\ simp\isanewline
\ \ \isacommand{qed}\isamarkupfalse%
\ \isanewline
\ \ \isacommand{with}\isamarkupfalse%
\ val{\isacharunderscore}{\kern0pt}m\ first{\isacharunderscore}{\kern0pt}incl\ \isanewline
\ \ \isacommand{have}\isamarkupfalse%
\ {\isachardoublequoteopen}val{\isacharparenleft}{\kern0pt}G{\isacharcomma}{\kern0pt}{\isacharquery}{\kern0pt}n{\isacharparenright}{\kern0pt}\ {\isacharequal}{\kern0pt}\ {\isacharbraceleft}{\kern0pt}x{\isasymin}c{\isachardot}{\kern0pt}\ {\isacharparenleft}{\kern0pt}M{\isacharbrackleft}{\kern0pt}G{\isacharbrackright}{\kern0pt}{\isacharcomma}{\kern0pt}\ {\isacharbrackleft}{\kern0pt}x{\isacharbrackright}{\kern0pt}\ {\isacharat}{\kern0pt}\ env\ {\isacharat}{\kern0pt}\ {\isacharbrackleft}{\kern0pt}c{\isacharbrackright}{\kern0pt}\ {\isasymTurnstile}\ {\isasymphi}{\isacharparenright}{\kern0pt}{\isacharbraceright}{\kern0pt}{\isachardoublequoteclose}\ \isacommand{by}\isamarkupfalse%
\ auto\isanewline
\ \ \isacommand{also}\isamarkupfalse%
\ \isanewline
\ \ \isacommand{have}\isamarkupfalse%
\ {\isachardoublequoteopen}\ {\isachardot}{\kern0pt}{\isachardot}{\kern0pt}{\isachardot}{\kern0pt}\ {\isacharequal}{\kern0pt}\ {\isacharbraceleft}{\kern0pt}x{\isasymin}c{\isachardot}{\kern0pt}\ {\isacharparenleft}{\kern0pt}M{\isacharbrackleft}{\kern0pt}G{\isacharbrackright}{\kern0pt}{\isacharcomma}{\kern0pt}\ {\isacharbrackleft}{\kern0pt}x{\isacharbrackright}{\kern0pt}\ {\isacharat}{\kern0pt}\ env\ {\isasymTurnstile}\ {\isasymphi}{\isacharparenright}{\kern0pt}{\isacharbraceright}{\kern0pt}{\isachardoublequoteclose}\ \isanewline
\ \ \isacommand{proof}\isamarkupfalse%
\ {\isacharminus}{\kern0pt}\isanewline
\ \ \ \ \isacommand{{\isacharbraceleft}{\kern0pt}}\isamarkupfalse%
\isanewline
\ \ \ \ \ \ \isacommand{fix}\isamarkupfalse%
\ x\isanewline
\ \ \ \ \ \ \isacommand{assume}\isamarkupfalse%
\ {\isachardoublequoteopen}x{\isasymin}c{\isachardoublequoteclose}\isanewline
\ \ \ \ \ \ \isacommand{moreover}\isamarkupfalse%
\ \isacommand{from}\isamarkupfalse%
\ assms\ \isanewline
\ \ \ \ \ \ \isacommand{have}\isamarkupfalse%
\ {\isachardoublequoteopen}c{\isasymin}M{\isacharbrackleft}{\kern0pt}G{\isacharbrackright}{\kern0pt}{\isachardoublequoteclose}\isanewline
\ \ \ \ \ \ \ \ \isacommand{unfolding}\isamarkupfalse%
\ GenExt{\isacharunderscore}{\kern0pt}def\ \isacommand{by}\isamarkupfalse%
\ auto\isanewline
\ \ \ \ \ \ \isacommand{moreover}\isamarkupfalse%
\ \isacommand{from}\isamarkupfalse%
\ this\ \isakeyword{and}\ {\isacartoucheopen}x{\isasymin}c{\isacartoucheclose}\ \isanewline
\ \ \ \ \ \ \isacommand{have}\isamarkupfalse%
\ {\isachardoublequoteopen}x{\isasymin}M{\isacharbrackleft}{\kern0pt}G{\isacharbrackright}{\kern0pt}{\isachardoublequoteclose}\isanewline
\ \ \ \ \ \ \ \ \isacommand{using}\isamarkupfalse%
\ transitivity{\isacharunderscore}{\kern0pt}MG\isanewline
\ \ \ \ \ \ \ \ \isacommand{by}\isamarkupfalse%
\ simp\isanewline
\ \ \ \ \ \ \isacommand{ultimately}\isamarkupfalse%
\ \isanewline
\ \ \ \ \ \ \isacommand{have}\isamarkupfalse%
\ {\isachardoublequoteopen}{\isacharparenleft}{\kern0pt}M{\isacharbrackleft}{\kern0pt}G{\isacharbrackright}{\kern0pt}{\isacharcomma}{\kern0pt}\ \ {\isacharparenleft}{\kern0pt}{\isacharbrackleft}{\kern0pt}x{\isacharbrackright}{\kern0pt}\ {\isacharat}{\kern0pt}\ env{\isacharparenright}{\kern0pt}\ {\isacharat}{\kern0pt}{\isacharbrackleft}{\kern0pt}c{\isacharbrackright}{\kern0pt}\ {\isasymTurnstile}\ \ {\isasymphi}{\isacharparenright}{\kern0pt}\ {\isasymlongleftrightarrow}\ {\isacharparenleft}{\kern0pt}M{\isacharbrackleft}{\kern0pt}G{\isacharbrackright}{\kern0pt}{\isacharcomma}{\kern0pt}\ \ {\isacharbrackleft}{\kern0pt}x{\isacharbrackright}{\kern0pt}\ {\isacharat}{\kern0pt}\ env\ {\isasymTurnstile}\ \ {\isasymphi}{\isacharparenright}{\kern0pt}{\isachardoublequoteclose}\ \isanewline
\ \ \ \ \ \ \ \ \isacommand{using}\isamarkupfalse%
\ phi\ {\isacartoucheopen}env\ {\isasymin}\ {\isacharunderscore}{\kern0pt}{\isacartoucheclose}\ \isacommand{by}\isamarkupfalse%
\ {\isacharparenleft}{\kern0pt}rule{\isacharunderscore}{\kern0pt}tac\ arity{\isacharunderscore}{\kern0pt}sats{\isacharunderscore}{\kern0pt}iff{\isacharcomma}{\kern0pt}\ simp{\isacharunderscore}{\kern0pt}all{\isacharparenright}{\kern0pt}\ \ \ \isanewline
\ \ \ \ \isacommand{{\isacharbraceright}{\kern0pt}}\isamarkupfalse%
\isanewline
\ \ \ \ \isacommand{then}\isamarkupfalse%
\ \isacommand{show}\isamarkupfalse%
\ {\isacharquery}{\kern0pt}thesis\ \isacommand{by}\isamarkupfalse%
\ auto\isanewline
\ \ \isacommand{qed}\isamarkupfalse%
\ \ \ \ \ \ \isanewline
\ \ \isacommand{finally}\isamarkupfalse%
\ \isanewline
\ \ \isacommand{show}\isamarkupfalse%
\ {\isachardoublequoteopen}{\isacharbraceleft}{\kern0pt}x{\isasymin}c{\isachardot}{\kern0pt}\ {\isacharparenleft}{\kern0pt}M{\isacharbrackleft}{\kern0pt}G{\isacharbrackright}{\kern0pt}{\isacharcomma}{\kern0pt}\ {\isacharbrackleft}{\kern0pt}x{\isacharbrackright}{\kern0pt}\ {\isacharat}{\kern0pt}\ env\ {\isasymTurnstile}\ {\isasymphi}{\isacharparenright}{\kern0pt}{\isacharbraceright}{\kern0pt}{\isasymin}\ M{\isacharbrackleft}{\kern0pt}G{\isacharbrackright}{\kern0pt}{\isachardoublequoteclose}\ \isanewline
\ \ \ \ \isacommand{using}\isamarkupfalse%
\ {\isacartoucheopen}{\isacharquery}{\kern0pt}n{\isasymin}M{\isacartoucheclose}\ GenExt{\isacharunderscore}{\kern0pt}def\ \isacommand{by}\isamarkupfalse%
\ force\isanewline
\isacommand{qed}\isamarkupfalse%
%
\endisatagproof
{\isafoldproof}%
%
\isadelimproof
\isanewline
%
\endisadelimproof
\isanewline
\isacommand{theorem}\isamarkupfalse%
\ separation{\isacharunderscore}{\kern0pt}in{\isacharunderscore}{\kern0pt}MG{\isacharcolon}{\kern0pt}\isanewline
\ \ \isakeyword{assumes}\ \isanewline
\ \ \ \ {\isachardoublequoteopen}{\isasymphi}{\isasymin}formula{\isachardoublequoteclose}\ \isakeyword{and}\ {\isachardoublequoteopen}arity{\isacharparenleft}{\kern0pt}{\isasymphi}{\isacharparenright}{\kern0pt}\ {\isasymle}\ {\isadigit{1}}\ {\isacharhash}{\kern0pt}{\isacharplus}{\kern0pt}\ length{\isacharparenleft}{\kern0pt}env{\isacharparenright}{\kern0pt}{\isachardoublequoteclose}\ \isakeyword{and}\ {\isachardoublequoteopen}env{\isasymin}list{\isacharparenleft}{\kern0pt}M{\isacharbrackleft}{\kern0pt}G{\isacharbrackright}{\kern0pt}{\isacharparenright}{\kern0pt}{\isachardoublequoteclose}\isanewline
\ \ \isakeyword{shows}\ \ \isanewline
\ \ \ \ {\isachardoublequoteopen}separation{\isacharparenleft}{\kern0pt}{\isacharhash}{\kern0pt}{\isacharhash}{\kern0pt}M{\isacharbrackleft}{\kern0pt}G{\isacharbrackright}{\kern0pt}{\isacharcomma}{\kern0pt}{\isasymlambda}x{\isachardot}{\kern0pt}\ {\isacharparenleft}{\kern0pt}M{\isacharbrackleft}{\kern0pt}G{\isacharbrackright}{\kern0pt}{\isacharcomma}{\kern0pt}\ {\isacharbrackleft}{\kern0pt}x{\isacharbrackright}{\kern0pt}\ {\isacharat}{\kern0pt}\ env\ {\isasymTurnstile}\ {\isasymphi}{\isacharparenright}{\kern0pt}{\isacharparenright}{\kern0pt}{\isachardoublequoteclose}\isanewline
%
\isadelimproof
%
\endisadelimproof
%
\isatagproof
\isacommand{proof}\isamarkupfalse%
\ {\isacharminus}{\kern0pt}\isanewline
\ \ \isacommand{{\isacharbraceleft}{\kern0pt}}\isamarkupfalse%
\ \isanewline
\ \ \ \ \isacommand{fix}\isamarkupfalse%
\ c\isanewline
\ \ \ \ \isacommand{assume}\isamarkupfalse%
\ {\isachardoublequoteopen}c{\isasymin}M{\isacharbrackleft}{\kern0pt}G{\isacharbrackright}{\kern0pt}{\isachardoublequoteclose}\ \isanewline
\ \ \ \ \isacommand{moreover}\isamarkupfalse%
\ \isacommand{from}\isamarkupfalse%
\ {\isacartoucheopen}env\ {\isasymin}\ {\isacharunderscore}{\kern0pt}{\isacartoucheclose}\isanewline
\ \ \ \ \isacommand{obtain}\isamarkupfalse%
\ nenv\ \isakeyword{where}\ \ {\isachardoublequoteopen}nenv{\isasymin}list{\isacharparenleft}{\kern0pt}M{\isacharparenright}{\kern0pt}{\isachardoublequoteclose}\ \isanewline
\ \ \ \ \ \ {\isachardoublequoteopen}env\ {\isacharequal}{\kern0pt}\ map{\isacharparenleft}{\kern0pt}val{\isacharparenleft}{\kern0pt}G{\isacharparenright}{\kern0pt}{\isacharcomma}{\kern0pt}nenv{\isacharparenright}{\kern0pt}{\isachardoublequoteclose}\ {\isachardoublequoteopen}length{\isacharparenleft}{\kern0pt}env{\isacharparenright}{\kern0pt}\ {\isacharequal}{\kern0pt}\ length{\isacharparenleft}{\kern0pt}nenv{\isacharparenright}{\kern0pt}{\isachardoublequoteclose}\isanewline
\ \ \ \ \ \ \isacommand{using}\isamarkupfalse%
\ GenExt{\isacharunderscore}{\kern0pt}def\ map{\isacharunderscore}{\kern0pt}val{\isacharbrackleft}{\kern0pt}of\ env{\isacharbrackright}{\kern0pt}\ \isacommand{by}\isamarkupfalse%
\ auto\isanewline
\ \ \ \ \isacommand{moreover}\isamarkupfalse%
\ \isacommand{note}\isamarkupfalse%
\ {\isacartoucheopen}{\isasymphi}\ {\isasymin}\ {\isacharunderscore}{\kern0pt}{\isacartoucheclose}\ {\isacartoucheopen}arity{\isacharparenleft}{\kern0pt}{\isasymphi}{\isacharparenright}{\kern0pt}\ {\isasymle}\ {\isacharunderscore}{\kern0pt}{\isacartoucheclose}\ {\isacartoucheopen}env\ {\isasymin}\ {\isacharunderscore}{\kern0pt}{\isacartoucheclose}\isanewline
\ \ \ \ \isacommand{ultimately}\isamarkupfalse%
\isanewline
\ \ \ \ \isacommand{have}\isamarkupfalse%
\ Eq{\isadigit{1}}{\isacharcolon}{\kern0pt}\ {\isachardoublequoteopen}{\isacharbraceleft}{\kern0pt}x{\isasymin}c{\isachardot}{\kern0pt}\ {\isacharparenleft}{\kern0pt}M{\isacharbrackleft}{\kern0pt}G{\isacharbrackright}{\kern0pt}{\isacharcomma}{\kern0pt}\ {\isacharbrackleft}{\kern0pt}x{\isacharbrackright}{\kern0pt}\ {\isacharat}{\kern0pt}\ env\ {\isasymTurnstile}\ {\isasymphi}{\isacharparenright}{\kern0pt}{\isacharbraceright}{\kern0pt}\ {\isasymin}\ M{\isacharbrackleft}{\kern0pt}G{\isacharbrackright}{\kern0pt}{\isachardoublequoteclose}\isanewline
\ \ \ \ \ \ \isacommand{using}\isamarkupfalse%
\ Collect{\isacharunderscore}{\kern0pt}sats{\isacharunderscore}{\kern0pt}in{\isacharunderscore}{\kern0pt}MG\ \ \isacommand{by}\isamarkupfalse%
\ auto\isanewline
\ \ \isacommand{{\isacharbraceright}{\kern0pt}}\isamarkupfalse%
\isanewline
\ \ \isacommand{then}\isamarkupfalse%
\ \isanewline
\ \ \isacommand{show}\isamarkupfalse%
\ {\isacharquery}{\kern0pt}thesis\ \isanewline
\ \ \ \ \isacommand{using}\isamarkupfalse%
\ separation{\isacharunderscore}{\kern0pt}iff\ rev{\isacharunderscore}{\kern0pt}bexI\ \isacommand{unfolding}\isamarkupfalse%
\ is{\isacharunderscore}{\kern0pt}Collect{\isacharunderscore}{\kern0pt}def\ \isacommand{by}\isamarkupfalse%
\ force\isanewline
\isacommand{qed}\isamarkupfalse%
%
\endisatagproof
{\isafoldproof}%
%
\isadelimproof
\isanewline
%
\endisadelimproof
\isanewline
\isacommand{end}\isamarkupfalse%
\ \isanewline
%
\isadelimtheory
\isanewline
%
\endisadelimtheory
%
\isatagtheory
\isacommand{end}\isamarkupfalse%
%
\endisatagtheory
{\isafoldtheory}%
%
\isadelimtheory
%
\endisadelimtheory
%
\end{isabellebody}%
\endinput
%:%file=~/source/repos/ZF-notAC/code/Forcing/Separation_Axiom.thy%:%
%:%11=1%:%
%:%27=2%:%
%:%28=2%:%
%:%29=3%:%
%:%30=4%:%
%:%35=4%:%
%:%38=5%:%
%:%39=6%:%
%:%40=6%:%
%:%41=7%:%
%:%42=8%:%
%:%43=9%:%
%:%44=9%:%
%:%45=10%:%
%:%46=11%:%
%:%49=12%:%
%:%53=12%:%
%:%54=12%:%
%:%55=13%:%
%:%56=13%:%
%:%57=14%:%
%:%58=14%:%
%:%59=15%:%
%:%60=15%:%
%:%61=15%:%
%:%62=16%:%
%:%63=16%:%
%:%64=16%:%
%:%65=16%:%
%:%66=17%:%
%:%67=17%:%
%:%68=18%:%
%:%69=18%:%
%:%70=19%:%
%:%71=19%:%
%:%72=19%:%
%:%73=20%:%
%:%74=21%:%
%:%75=22%:%
%:%76=22%:%
%:%77=23%:%
%:%78=23%:%
%:%79=24%:%
%:%80=24%:%
%:%81=24%:%
%:%82=24%:%
%:%83=25%:%
%:%89=25%:%
%:%92=26%:%
%:%93=27%:%
%:%94=28%:%
%:%95=28%:%
%:%96=29%:%
%:%97=30%:%
%:%98=31%:%
%:%99=32%:%
%:%100=33%:%
%:%107=34%:%
%:%108=34%:%
%:%109=35%:%
%:%110=35%:%
%:%111=36%:%
%:%112=36%:%
%:%113=37%:%
%:%114=37%:%
%:%115=37%:%
%:%116=38%:%
%:%117=38%:%
%:%118=39%:%
%:%119=39%:%
%:%120=40%:%
%:%121=40%:%
%:%122=41%:%
%:%123=41%:%
%:%124=42%:%
%:%125=42%:%
%:%126=43%:%
%:%127=43%:%
%:%128=43%:%
%:%129=44%:%
%:%130=44%:%
%:%131=45%:%
%:%132=45%:%
%:%133=46%:%
%:%134=46%:%
%:%135=46%:%
%:%136=47%:%
%:%137=47%:%
%:%138=48%:%
%:%139=48%:%
%:%140=49%:%
%:%141=49%:%
%:%142=49%:%
%:%143=50%:%
%:%144=50%:%
%:%145=51%:%
%:%146=51%:%
%:%147=52%:%
%:%148=52%:%
%:%149=52%:%
%:%150=53%:%
%:%151=53%:%
%:%152=54%:%
%:%153=54%:%
%:%154=55%:%
%:%155=55%:%
%:%156=56%:%
%:%157=56%:%
%:%158=57%:%
%:%159=57%:%
%:%160=58%:%
%:%161=58%:%
%:%162=59%:%
%:%163=59%:%
%:%164=60%:%
%:%165=60%:%
%:%166=61%:%
%:%167=62%:%
%:%168=62%:%
%:%169=63%:%
%:%170=63%:%
%:%171=64%:%
%:%172=64%:%
%:%173=65%:%
%:%174=65%:%
%:%175=66%:%
%:%176=66%:%
%:%177=67%:%
%:%178=67%:%
%:%179=68%:%
%:%180=68%:%
%:%181=69%:%
%:%182=69%:%
%:%183=70%:%
%:%184=70%:%
%:%185=70%:%
%:%186=71%:%
%:%187=71%:%
%:%188=72%:%
%:%189=72%:%
%:%190=73%:%
%:%191=73%:%
%:%192=74%:%
%:%193=74%:%
%:%194=75%:%
%:%195=75%:%
%:%196=76%:%
%:%197=76%:%
%:%198=76%:%
%:%199=77%:%
%:%200=77%:%
%:%201=78%:%
%:%202=78%:%
%:%203=79%:%
%:%204=79%:%
%:%205=80%:%
%:%206=80%:%
%:%207=81%:%
%:%208=81%:%
%:%209=82%:%
%:%210=82%:%
%:%211=83%:%
%:%212=83%:%
%:%213=84%:%
%:%214=84%:%
%:%215=85%:%
%:%216=85%:%
%:%217=86%:%
%:%218=86%:%
%:%219=87%:%
%:%220=87%:%
%:%221=88%:%
%:%222=88%:%
%:%223=89%:%
%:%224=89%:%
%:%225=90%:%
%:%226=90%:%
%:%228=92%:%
%:%229=93%:%
%:%230=93%:%
%:%231=94%:%
%:%232=94%:%
%:%234=96%:%
%:%235=97%:%
%:%236=98%:%
%:%237=98%:%
%:%238=99%:%
%:%239=99%:%
%:%240=100%:%
%:%241=100%:%
%:%242=101%:%
%:%243=101%:%
%:%244=102%:%
%:%245=102%:%
%:%246=102%:%
%:%247=102%:%
%:%248=103%:%
%:%249=103%:%
%:%250=104%:%
%:%251=104%:%
%:%252=105%:%
%:%253=105%:%
%:%254=105%:%
%:%255=106%:%
%:%256=106%:%
%:%257=107%:%
%:%258=107%:%
%:%259=108%:%
%:%260=108%:%
%:%261=109%:%
%:%262=109%:%
%:%263=110%:%
%:%264=110%:%
%:%265=110%:%
%:%266=111%:%
%:%267=111%:%
%:%268=112%:%
%:%269=112%:%
%:%270=112%:%
%:%271=113%:%
%:%272=113%:%
%:%273=114%:%
%:%274=114%:%
%:%275=114%:%
%:%276=115%:%
%:%277=115%:%
%:%278=115%:%
%:%279=115%:%
%:%280=116%:%
%:%281=116%:%
%:%282=116%:%
%:%283=116%:%
%:%284=117%:%
%:%285=117%:%
%:%286=118%:%
%:%287=118%:%
%:%288=119%:%
%:%289=119%:%
%:%290=120%:%
%:%291=120%:%
%:%292=121%:%
%:%293=122%:%
%:%294=123%:%
%:%295=123%:%
%:%296=124%:%
%:%297=124%:%
%:%298=125%:%
%:%299=125%:%
%:%300=126%:%
%:%301=126%:%
%:%302=126%:%
%:%303=127%:%
%:%304=127%:%
%:%305=128%:%
%:%306=128%:%
%:%307=128%:%
%:%308=129%:%
%:%309=129%:%
%:%310=130%:%
%:%311=130%:%
%:%312=131%:%
%:%313=132%:%
%:%314=132%:%
%:%315=133%:%
%:%316=133%:%
%:%317=134%:%
%:%318=134%:%
%:%319=135%:%
%:%320=135%:%
%:%321=136%:%
%:%322=136%:%
%:%323=137%:%
%:%324=138%:%
%:%325=138%:%
%:%326=139%:%
%:%327=139%:%
%:%328=140%:%
%:%329=140%:%
%:%330=141%:%
%:%331=141%:%
%:%332=142%:%
%:%333=142%:%
%:%334=143%:%
%:%335=143%:%
%:%337=145%:%
%:%338=146%:%
%:%339=146%:%
%:%340=147%:%
%:%341=147%:%
%:%342=148%:%
%:%343=148%:%
%:%344=149%:%
%:%345=150%:%
%:%346=150%:%
%:%347=150%:%
%:%348=151%:%
%:%349=151%:%
%:%350=152%:%
%:%351=152%:%
%:%352=153%:%
%:%353=154%:%
%:%354=154%:%
%:%355=155%:%
%:%356=155%:%
%:%357=156%:%
%:%358=156%:%
%:%359=157%:%
%:%360=157%:%
%:%361=158%:%
%:%362=158%:%
%:%363=159%:%
%:%364=159%:%
%:%365=160%:%
%:%366=160%:%
%:%367=161%:%
%:%368=162%:%
%:%369=162%:%
%:%370=163%:%
%:%371=163%:%
%:%372=164%:%
%:%373=164%:%
%:%374=165%:%
%:%375=165%:%
%:%377=167%:%
%:%378=168%:%
%:%379=168%:%
%:%380=169%:%
%:%381=169%:%
%:%382=170%:%
%:%383=170%:%
%:%384=171%:%
%:%385=171%:%
%:%388=174%:%
%:%389=175%:%
%:%390=176%:%
%:%391=176%:%
%:%392=177%:%
%:%393=177%:%
%:%394=177%:%
%:%395=178%:%
%:%396=178%:%
%:%397=179%:%
%:%398=179%:%
%:%399=179%:%
%:%400=180%:%
%:%401=180%:%
%:%402=181%:%
%:%403=181%:%
%:%404=182%:%
%:%405=183%:%
%:%406=184%:%
%:%407=184%:%
%:%408=184%:%
%:%409=185%:%
%:%410=185%:%
%:%411=186%:%
%:%412=186%:%
%:%413=187%:%
%:%414=187%:%
%:%415=187%:%
%:%416=188%:%
%:%417=188%:%
%:%418=189%:%
%:%419=189%:%
%:%420=190%:%
%:%421=190%:%
%:%422=191%:%
%:%423=191%:%
%:%424=192%:%
%:%425=192%:%
%:%426=192%:%
%:%427=192%:%
%:%428=193%:%
%:%429=193%:%
%:%430=194%:%
%:%431=194%:%
%:%432=195%:%
%:%433=195%:%
%:%434=196%:%
%:%435=196%:%
%:%436=197%:%
%:%437=197%:%
%:%438=198%:%
%:%439=198%:%
%:%440=199%:%
%:%441=199%:%
%:%442=199%:%
%:%443=200%:%
%:%444=200%:%
%:%445=201%:%
%:%446=201%:%
%:%447=202%:%
%:%448=202%:%
%:%449=203%:%
%:%450=204%:%
%:%451=205%:%
%:%452=205%:%
%:%453=205%:%
%:%454=206%:%
%:%455=206%:%
%:%456=206%:%
%:%457=207%:%
%:%458=207%:%
%:%459=208%:%
%:%460=209%:%
%:%461=209%:%
%:%462=209%:%
%:%463=209%:%
%:%464=210%:%
%:%465=210%:%
%:%466=211%:%
%:%467=211%:%
%:%468=212%:%
%:%469=213%:%
%:%470=213%:%
%:%471=213%:%
%:%472=214%:%
%:%473=214%:%
%:%474=215%:%
%:%475=215%:%
%:%476=216%:%
%:%477=217%:%
%:%478=217%:%
%:%479=218%:%
%:%480=218%:%
%:%481=219%:%
%:%482=219%:%
%:%484=221%:%
%:%485=222%:%
%:%486=223%:%
%:%487=223%:%
%:%488=224%:%
%:%489=224%:%
%:%490=225%:%
%:%491=225%:%
%:%492=226%:%
%:%493=226%:%
%:%494=227%:%
%:%495=227%:%
%:%496=228%:%
%:%497=228%:%
%:%498=229%:%
%:%499=229%:%
%:%500=230%:%
%:%501=230%:%
%:%502=230%:%
%:%503=231%:%
%:%504=231%:%
%:%505=232%:%
%:%506=232%:%
%:%507=233%:%
%:%508=233%:%
%:%509=233%:%
%:%510=234%:%
%:%511=234%:%
%:%512=235%:%
%:%513=235%:%
%:%516=238%:%
%:%517=239%:%
%:%518=239%:%
%:%519=239%:%
%:%520=240%:%
%:%521=240%:%
%:%522=241%:%
%:%523=241%:%
%:%524=242%:%
%:%525=243%:%
%:%526=243%:%
%:%527=244%:%
%:%528=245%:%
%:%529=245%:%
%:%533=249%:%
%:%534=250%:%
%:%535=250%:%
%:%536=251%:%
%:%537=251%:%
%:%538=251%:%
%:%539=251%:%
%:%540=251%:%
%:%541=252%:%
%:%542=252%:%
%:%543=253%:%
%:%544=253%:%
%:%545=254%:%
%:%546=254%:%
%:%547=255%:%
%:%548=255%:%
%:%549=256%:%
%:%550=257%:%
%:%551=257%:%
%:%552=258%:%
%:%553=258%:%
%:%554=259%:%
%:%555=259%:%
%:%556=260%:%
%:%556=264%:%
%:%557=265%:%
%:%558=265%:%
%:%559=266%:%
%:%560=266%:%
%:%561=267%:%
%:%562=267%:%
%:%563=268%:%
%:%564=268%:%
%:%565=269%:%
%:%566=269%:%
%:%567=270%:%
%:%568=270%:%
%:%569=271%:%
%:%570=271%:%
%:%571=272%:%
%:%572=272%:%
%:%573=273%:%
%:%574=273%:%
%:%575=273%:%
%:%576=274%:%
%:%577=274%:%
%:%578=275%:%
%:%579=275%:%
%:%580=276%:%
%:%581=276%:%
%:%582=277%:%
%:%583=277%:%
%:%584=277%:%
%:%585=278%:%
%:%586=278%:%
%:%587=279%:%
%:%588=279%:%
%:%589=280%:%
%:%590=280%:%
%:%591=281%:%
%:%592=281%:%
%:%593=281%:%
%:%594=282%:%
%:%595=282%:%
%:%596=283%:%
%:%597=283%:%
%:%598=283%:%
%:%599=284%:%
%:%600=284%:%
%:%601=285%:%
%:%602=285%:%
%:%603=286%:%
%:%604=286%:%
%:%605=287%:%
%:%606=287%:%
%:%607=288%:%
%:%608=288%:%
%:%609=289%:%
%:%610=289%:%
%:%611=289%:%
%:%612=290%:%
%:%613=290%:%
%:%614=291%:%
%:%615=291%:%
%:%616=291%:%
%:%617=292%:%
%:%618=292%:%
%:%619=293%:%
%:%620=293%:%
%:%621=294%:%
%:%622=294%:%
%:%623=294%:%
%:%624=295%:%
%:%625=295%:%
%:%626=295%:%
%:%627=296%:%
%:%628=296%:%
%:%629=297%:%
%:%630=297%:%
%:%631=298%:%
%:%632=298%:%
%:%633=299%:%
%:%634=299%:%
%:%635=299%:%
%:%636=300%:%
%:%637=300%:%
%:%638=301%:%
%:%639=301%:%
%:%640=302%:%
%:%641=302%:%
%:%642=302%:%
%:%643=303%:%
%:%644=303%:%
%:%645=304%:%
%:%646=304%:%
%:%647=305%:%
%:%648=305%:%
%:%649=306%:%
%:%650=307%:%
%:%651=307%:%
%:%652=308%:%
%:%653=309%:%
%:%654=309%:%
%:%655=310%:%
%:%656=310%:%
%:%657=311%:%
%:%658=311%:%
%:%659=312%:%
%:%660=312%:%
%:%661=312%:%
%:%662=313%:%
%:%663=313%:%
%:%664=314%:%
%:%665=314%:%
%:%666=314%:%
%:%667=315%:%
%:%668=316%:%
%:%669=316%:%
%:%670=316%:%
%:%671=317%:%
%:%672=317%:%
%:%673=318%:%
%:%674=318%:%
%:%675=319%:%
%:%676=319%:%
%:%677=319%:%
%:%678=320%:%
%:%679=320%:%
%:%680=321%:%
%:%681=322%:%
%:%682=322%:%
%:%683=323%:%
%:%684=323%:%
%:%685=324%:%
%:%686=324%:%
%:%687=325%:%
%:%688=325%:%
%:%689=326%:%
%:%690=326%:%
%:%691=327%:%
%:%692=328%:%
%:%693=328%:%
%:%694=329%:%
%:%695=329%:%
%:%696=330%:%
%:%697=330%:%
%:%698=331%:%
%:%699=331%:%
%:%700=332%:%
%:%701=332%:%
%:%702=333%:%
%:%703=333%:%
%:%704=334%:%
%:%705=334%:%
%:%706=334%:%
%:%707=335%:%
%:%708=335%:%
%:%709=336%:%
%:%710=336%:%
%:%711=337%:%
%:%712=337%:%
%:%713=337%:%
%:%714=338%:%
%:%715=338%:%
%:%716=339%:%
%:%717=339%:%
%:%718=340%:%
%:%719=340%:%
%:%720=340%:%
%:%721=341%:%
%:%722=341%:%
%:%723=342%:%
%:%724=342%:%
%:%725=342%:%
%:%726=343%:%
%:%727=343%:%
%:%728=344%:%
%:%729=344%:%
%:%730=345%:%
%:%731=345%:%
%:%732=345%:%
%:%733=346%:%
%:%734=346%:%
%:%735=347%:%
%:%736=347%:%
%:%737=347%:%
%:%738=348%:%
%:%739=348%:%
%:%740=349%:%
%:%741=349%:%
%:%742=350%:%
%:%743=350%:%
%:%744=350%:%
%:%745=351%:%
%:%746=351%:%
%:%747=352%:%
%:%748=352%:%
%:%749=353%:%
%:%750=353%:%
%:%751=354%:%
%:%752=354%:%
%:%753=355%:%
%:%754=355%:%
%:%755=356%:%
%:%756=356%:%
%:%757=357%:%
%:%758=357%:%
%:%759=357%:%
%:%760=358%:%
%:%761=358%:%
%:%762=359%:%
%:%763=359%:%
%:%764=359%:%
%:%765=360%:%
%:%766=360%:%
%:%767=360%:%
%:%768=361%:%
%:%769=361%:%
%:%770=362%:%
%:%771=362%:%
%:%772=363%:%
%:%773=363%:%
%:%774=364%:%
%:%775=364%:%
%:%776=365%:%
%:%777=365%:%
%:%778=366%:%
%:%779=366%:%
%:%780=366%:%
%:%781=367%:%
%:%782=367%:%
%:%783=368%:%
%:%784=368%:%
%:%785=368%:%
%:%786=368%:%
%:%787=369%:%
%:%788=369%:%
%:%789=370%:%
%:%790=370%:%
%:%791=371%:%
%:%792=371%:%
%:%793=372%:%
%:%794=372%:%
%:%795=372%:%
%:%796=373%:%
%:%802=373%:%
%:%805=374%:%
%:%806=375%:%
%:%807=375%:%
%:%808=376%:%
%:%809=377%:%
%:%810=378%:%
%:%811=379%:%
%:%818=380%:%
%:%819=380%:%
%:%820=381%:%
%:%821=381%:%
%:%822=382%:%
%:%823=382%:%
%:%824=383%:%
%:%825=383%:%
%:%826=384%:%
%:%827=384%:%
%:%828=384%:%
%:%829=385%:%
%:%830=385%:%
%:%831=386%:%
%:%832=387%:%
%:%833=387%:%
%:%834=387%:%
%:%835=388%:%
%:%836=388%:%
%:%837=388%:%
%:%838=389%:%
%:%839=389%:%
%:%840=390%:%
%:%841=390%:%
%:%842=391%:%
%:%843=391%:%
%:%844=391%:%
%:%845=392%:%
%:%846=392%:%
%:%847=393%:%
%:%848=393%:%
%:%849=394%:%
%:%850=394%:%
%:%851=395%:%
%:%852=395%:%
%:%853=395%:%
%:%854=395%:%
%:%855=396%:%
%:%861=396%:%
%:%864=397%:%
%:%865=398%:%
%:%866=398%:%
%:%869=399%:%
%:%874=400%:%

%
\begin{isabellebody}%
\setisabellecontext{Pairing{\isacharunderscore}{\kern0pt}Axiom}%
%
\isadelimdocument
%
\endisadelimdocument
%
\isatagdocument
%
\isamarkupsection{The Axiom of Pairing in $M[G]$%
}
\isamarkuptrue%
%
\endisatagdocument
{\isafolddocument}%
%
\isadelimdocument
%
\endisadelimdocument
%
\isadelimtheory
%
\endisadelimtheory
%
\isatagtheory
\isacommand{theory}\isamarkupfalse%
\ Pairing{\isacharunderscore}{\kern0pt}Axiom\ \isakeyword{imports}\ Names\ \isakeyword{begin}%
\endisatagtheory
{\isafoldtheory}%
%
\isadelimtheory
\isanewline
%
\endisadelimtheory
\isanewline
\isacommand{context}\isamarkupfalse%
\ forcing{\isacharunderscore}{\kern0pt}data\isanewline
\isakeyword{begin}\isanewline
\isanewline
\isacommand{lemma}\isamarkupfalse%
\ val{\isacharunderscore}{\kern0pt}Upair\ {\isacharcolon}{\kern0pt}\isanewline
\ \ {\isachardoublequoteopen}one\ {\isasymin}\ G\ {\isasymLongrightarrow}\ val{\isacharparenleft}{\kern0pt}G{\isacharcomma}{\kern0pt}{\isacharbraceleft}{\kern0pt}{\isasymlangle}{\isasymtau}{\isacharcomma}{\kern0pt}one{\isasymrangle}{\isacharcomma}{\kern0pt}{\isasymlangle}{\isasymrho}{\isacharcomma}{\kern0pt}one{\isasymrangle}{\isacharbraceright}{\kern0pt}{\isacharparenright}{\kern0pt}\ {\isacharequal}{\kern0pt}\ {\isacharbraceleft}{\kern0pt}val{\isacharparenleft}{\kern0pt}G{\isacharcomma}{\kern0pt}{\isasymtau}{\isacharparenright}{\kern0pt}{\isacharcomma}{\kern0pt}val{\isacharparenleft}{\kern0pt}G{\isacharcomma}{\kern0pt}{\isasymrho}{\isacharparenright}{\kern0pt}{\isacharbraceright}{\kern0pt}{\isachardoublequoteclose}\isanewline
%
\isadelimproof
\ \ %
\endisadelimproof
%
\isatagproof
\isacommand{by}\isamarkupfalse%
\ {\isacharparenleft}{\kern0pt}insert\ one{\isacharunderscore}{\kern0pt}in{\isacharunderscore}{\kern0pt}P{\isacharcomma}{\kern0pt}\ rule\ trans{\isacharcomma}{\kern0pt}\ subst\ def{\isacharunderscore}{\kern0pt}val{\isacharcomma}{\kern0pt}auto\ simp\ add{\isacharcolon}{\kern0pt}\ Sep{\isacharunderscore}{\kern0pt}and{\isacharunderscore}{\kern0pt}Replace{\isacharparenright}{\kern0pt}%
\endisatagproof
{\isafoldproof}%
%
\isadelimproof
\isanewline
%
\endisadelimproof
\isanewline
\isacommand{lemma}\isamarkupfalse%
\ pairing{\isacharunderscore}{\kern0pt}in{\isacharunderscore}{\kern0pt}MG\ {\isacharcolon}{\kern0pt}\ \isanewline
\ \ \isakeyword{assumes}\ {\isachardoublequoteopen}M{\isacharunderscore}{\kern0pt}generic{\isacharparenleft}{\kern0pt}G{\isacharparenright}{\kern0pt}{\isachardoublequoteclose}\isanewline
\ \ \isakeyword{shows}\ {\isachardoublequoteopen}upair{\isacharunderscore}{\kern0pt}ax{\isacharparenleft}{\kern0pt}{\isacharhash}{\kern0pt}{\isacharhash}{\kern0pt}M{\isacharbrackleft}{\kern0pt}G{\isacharbrackright}{\kern0pt}{\isacharparenright}{\kern0pt}{\isachardoublequoteclose}\isanewline
%
\isadelimproof
%
\endisadelimproof
%
\isatagproof
\isacommand{proof}\isamarkupfalse%
\ {\isacharminus}{\kern0pt}\ \isanewline
\ \ \isacommand{{\isacharbraceleft}{\kern0pt}}\isamarkupfalse%
\isanewline
\ \ \ \ \isacommand{fix}\isamarkupfalse%
\ x\ y\isanewline
\ \ \ \ \isacommand{have}\isamarkupfalse%
\ {\isachardoublequoteopen}one{\isasymin}G{\isachardoublequoteclose}\ \isacommand{using}\isamarkupfalse%
\ assms\ one{\isacharunderscore}{\kern0pt}in{\isacharunderscore}{\kern0pt}G\ \isacommand{by}\isamarkupfalse%
\ simp\isanewline
\ \ \ \ \isacommand{from}\isamarkupfalse%
\ assms\ \isanewline
\ \ \ \ \isacommand{have}\isamarkupfalse%
\ {\isachardoublequoteopen}G{\isasymsubseteq}P{\isachardoublequoteclose}\ \isacommand{unfolding}\isamarkupfalse%
\ M{\isacharunderscore}{\kern0pt}generic{\isacharunderscore}{\kern0pt}def\ \isakeyword{and}\ filter{\isacharunderscore}{\kern0pt}def\ \isacommand{by}\isamarkupfalse%
\ simp\isanewline
\ \ \ \ \isacommand{with}\isamarkupfalse%
\ {\isacartoucheopen}one{\isasymin}G{\isacartoucheclose}\isanewline
\ \ \ \ \isacommand{have}\isamarkupfalse%
\ {\isachardoublequoteopen}one{\isasymin}P{\isachardoublequoteclose}\ \isacommand{using}\isamarkupfalse%
\ subsetD\ \isacommand{by}\isamarkupfalse%
\ simp\isanewline
\ \ \ \ \isacommand{then}\isamarkupfalse%
\ \isanewline
\ \ \ \ \isacommand{have}\isamarkupfalse%
\ {\isachardoublequoteopen}one{\isasymin}M{\isachardoublequoteclose}\ \isacommand{using}\isamarkupfalse%
\ transitivity{\isacharbrackleft}{\kern0pt}OF\ {\isacharunderscore}{\kern0pt}\ P{\isacharunderscore}{\kern0pt}in{\isacharunderscore}{\kern0pt}M{\isacharbrackright}{\kern0pt}\ \isacommand{by}\isamarkupfalse%
\ simp\isanewline
\ \ \ \ \isacommand{assume}\isamarkupfalse%
\ {\isachardoublequoteopen}x\ {\isasymin}\ M{\isacharbrackleft}{\kern0pt}G{\isacharbrackright}{\kern0pt}{\isachardoublequoteclose}\ {\isachardoublequoteopen}y\ {\isasymin}\ M{\isacharbrackleft}{\kern0pt}G{\isacharbrackright}{\kern0pt}{\isachardoublequoteclose}\isanewline
\ \ \ \ \isacommand{then}\isamarkupfalse%
\ \isanewline
\ \ \ \ \isacommand{obtain}\isamarkupfalse%
\ {\isasymtau}\ {\isasymrho}\ \isakeyword{where}\isanewline
\ \ \ \ \ \ {\isadigit{0}}\ {\isacharcolon}{\kern0pt}\ {\isachardoublequoteopen}val{\isacharparenleft}{\kern0pt}G{\isacharcomma}{\kern0pt}{\isasymtau}{\isacharparenright}{\kern0pt}\ {\isacharequal}{\kern0pt}\ x{\isachardoublequoteclose}\ {\isachardoublequoteopen}val{\isacharparenleft}{\kern0pt}G{\isacharcomma}{\kern0pt}{\isasymrho}{\isacharparenright}{\kern0pt}\ {\isacharequal}{\kern0pt}\ y{\isachardoublequoteclose}\ {\isachardoublequoteopen}{\isasymrho}\ {\isasymin}\ M{\isachardoublequoteclose}\ \ {\isachardoublequoteopen}{\isasymtau}\ {\isasymin}\ M{\isachardoublequoteclose}\isanewline
\ \ \ \ \ \ \isacommand{using}\isamarkupfalse%
\ GenExtD\ \isacommand{by}\isamarkupfalse%
\ blast\isanewline
\ \ \ \ \isacommand{with}\isamarkupfalse%
\ {\isacartoucheopen}one{\isasymin}M{\isacartoucheclose}\ \isanewline
\ \ \ \ \isacommand{have}\isamarkupfalse%
\ {\isachardoublequoteopen}{\isasymlangle}{\isasymtau}{\isacharcomma}{\kern0pt}one{\isasymrangle}\ {\isasymin}\ M{\isachardoublequoteclose}\ {\isachardoublequoteopen}{\isasymlangle}{\isasymrho}{\isacharcomma}{\kern0pt}one{\isasymrangle}{\isasymin}M{\isachardoublequoteclose}\ \isacommand{using}\isamarkupfalse%
\ pair{\isacharunderscore}{\kern0pt}in{\isacharunderscore}{\kern0pt}M{\isacharunderscore}{\kern0pt}iff\ \isacommand{by}\isamarkupfalse%
\ auto\isanewline
\ \ \ \ \isacommand{then}\isamarkupfalse%
\ \isanewline
\ \ \ \ \isacommand{have}\isamarkupfalse%
\ {\isadigit{1}}{\isacharcolon}{\kern0pt}\ {\isachardoublequoteopen}{\isacharbraceleft}{\kern0pt}{\isasymlangle}{\isasymtau}{\isacharcomma}{\kern0pt}one{\isasymrangle}{\isacharcomma}{\kern0pt}{\isasymlangle}{\isasymrho}{\isacharcomma}{\kern0pt}one{\isasymrangle}{\isacharbraceright}{\kern0pt}\ {\isasymin}\ M{\isachardoublequoteclose}\ {\isacharparenleft}{\kern0pt}\isakeyword{is}\ {\isachardoublequoteopen}{\isacharquery}{\kern0pt}{\isasymsigma}\ {\isasymin}\ {\isacharunderscore}{\kern0pt}{\isachardoublequoteclose}{\isacharparenright}{\kern0pt}\ \isacommand{using}\isamarkupfalse%
\ upair{\isacharunderscore}{\kern0pt}in{\isacharunderscore}{\kern0pt}M{\isacharunderscore}{\kern0pt}iff\ \isacommand{by}\isamarkupfalse%
\ simp\isanewline
\ \ \ \ \isacommand{then}\isamarkupfalse%
\ \isanewline
\ \ \ \ \isacommand{have}\isamarkupfalse%
\ {\isachardoublequoteopen}val{\isacharparenleft}{\kern0pt}G{\isacharcomma}{\kern0pt}{\isacharquery}{\kern0pt}{\isasymsigma}{\isacharparenright}{\kern0pt}\ {\isasymin}\ M{\isacharbrackleft}{\kern0pt}G{\isacharbrackright}{\kern0pt}{\isachardoublequoteclose}\ \isacommand{using}\isamarkupfalse%
\ GenExtI\ \isacommand{by}\isamarkupfalse%
\ simp\isanewline
\ \ \ \ \isacommand{with}\isamarkupfalse%
\ {\isadigit{1}}\ \isanewline
\ \ \ \ \isacommand{have}\isamarkupfalse%
\ {\isachardoublequoteopen}{\isacharbraceleft}{\kern0pt}val{\isacharparenleft}{\kern0pt}G{\isacharcomma}{\kern0pt}{\isasymtau}{\isacharparenright}{\kern0pt}{\isacharcomma}{\kern0pt}val{\isacharparenleft}{\kern0pt}G{\isacharcomma}{\kern0pt}{\isasymrho}{\isacharparenright}{\kern0pt}{\isacharbraceright}{\kern0pt}\ {\isasymin}\ M{\isacharbrackleft}{\kern0pt}G{\isacharbrackright}{\kern0pt}{\isachardoublequoteclose}\ \isacommand{using}\isamarkupfalse%
\ val{\isacharunderscore}{\kern0pt}Upair\ assms\ one{\isacharunderscore}{\kern0pt}in{\isacharunderscore}{\kern0pt}G\ \isacommand{by}\isamarkupfalse%
\ simp\isanewline
\ \ \ \ \isacommand{with}\isamarkupfalse%
\ {\isadigit{0}}\ \isanewline
\ \ \ \ \isacommand{have}\isamarkupfalse%
\ {\isachardoublequoteopen}{\isacharbraceleft}{\kern0pt}x{\isacharcomma}{\kern0pt}y{\isacharbraceright}{\kern0pt}\ {\isasymin}\ M{\isacharbrackleft}{\kern0pt}G{\isacharbrackright}{\kern0pt}{\isachardoublequoteclose}\ \isacommand{by}\isamarkupfalse%
\ simp\isanewline
\ \ \isacommand{{\isacharbraceright}{\kern0pt}}\isamarkupfalse%
\isanewline
\ \ \isacommand{then}\isamarkupfalse%
\ \isacommand{show}\isamarkupfalse%
\ {\isacharquery}{\kern0pt}thesis\ \isacommand{unfolding}\isamarkupfalse%
\ upair{\isacharunderscore}{\kern0pt}ax{\isacharunderscore}{\kern0pt}def\ upair{\isacharunderscore}{\kern0pt}def\ \isacommand{by}\isamarkupfalse%
\ auto\isanewline
\isacommand{qed}\isamarkupfalse%
%
\endisatagproof
{\isafoldproof}%
%
\isadelimproof
\isanewline
%
\endisadelimproof
\isanewline
\isacommand{end}\isamarkupfalse%
\ \ \isanewline
%
\isadelimtheory
%
\endisadelimtheory
%
\isatagtheory
\isacommand{end}\isamarkupfalse%
%
\endisatagtheory
{\isafoldtheory}%
%
\isadelimtheory
%
\endisadelimtheory
%
\end{isabellebody}%
\endinput
%:%file=~/source/repos/ZF-notAC/code/Forcing/Pairing_Axiom.thy%:%
%:%11=1%:%
%:%27=2%:%
%:%28=2%:%
%:%33=2%:%
%:%36=3%:%
%:%37=4%:%
%:%38=4%:%
%:%39=5%:%
%:%40=6%:%
%:%41=7%:%
%:%42=7%:%
%:%43=8%:%
%:%46=9%:%
%:%50=9%:%
%:%51=9%:%
%:%56=9%:%
%:%59=10%:%
%:%60=11%:%
%:%61=11%:%
%:%62=12%:%
%:%63=13%:%
%:%70=14%:%
%:%71=14%:%
%:%72=15%:%
%:%73=15%:%
%:%74=16%:%
%:%75=16%:%
%:%76=17%:%
%:%77=17%:%
%:%78=17%:%
%:%79=17%:%
%:%80=18%:%
%:%81=18%:%
%:%82=19%:%
%:%83=19%:%
%:%84=19%:%
%:%85=19%:%
%:%86=20%:%
%:%87=20%:%
%:%88=21%:%
%:%89=21%:%
%:%90=21%:%
%:%91=21%:%
%:%92=22%:%
%:%93=22%:%
%:%94=23%:%
%:%95=23%:%
%:%96=23%:%
%:%97=23%:%
%:%98=24%:%
%:%99=24%:%
%:%100=25%:%
%:%101=25%:%
%:%102=26%:%
%:%103=26%:%
%:%104=27%:%
%:%105=28%:%
%:%106=28%:%
%:%107=28%:%
%:%108=29%:%
%:%109=29%:%
%:%110=30%:%
%:%111=30%:%
%:%112=30%:%
%:%113=30%:%
%:%114=31%:%
%:%115=31%:%
%:%116=32%:%
%:%117=32%:%
%:%118=32%:%
%:%119=32%:%
%:%120=33%:%
%:%121=33%:%
%:%122=34%:%
%:%123=34%:%
%:%124=34%:%
%:%125=34%:%
%:%126=35%:%
%:%127=35%:%
%:%128=36%:%
%:%129=36%:%
%:%130=36%:%
%:%131=36%:%
%:%132=37%:%
%:%133=37%:%
%:%134=38%:%
%:%135=38%:%
%:%136=38%:%
%:%137=39%:%
%:%138=39%:%
%:%139=40%:%
%:%140=40%:%
%:%141=40%:%
%:%142=40%:%
%:%143=40%:%
%:%144=41%:%
%:%150=41%:%
%:%153=42%:%
%:%154=43%:%
%:%155=43%:%
%:%162=44%:%

%
\begin{isabellebody}%
\setisabellecontext{Union{\isacharunderscore}{\kern0pt}Axiom}%
%
\isadelimdocument
%
\endisadelimdocument
%
\isatagdocument
%
\isamarkupsection{The Axiom of Unions in $M[G]$%
}
\isamarkuptrue%
%
\endisatagdocument
{\isafolddocument}%
%
\isadelimdocument
%
\endisadelimdocument
%
\isadelimtheory
%
\endisadelimtheory
%
\isatagtheory
\isacommand{theory}\isamarkupfalse%
\ Union{\isacharunderscore}{\kern0pt}Axiom\isanewline
\ \ \isakeyword{imports}\ Names\isanewline
\isakeyword{begin}%
\endisatagtheory
{\isafoldtheory}%
%
\isadelimtheory
\isanewline
%
\endisadelimtheory
\isanewline
\isacommand{context}\isamarkupfalse%
\ forcing{\isacharunderscore}{\kern0pt}data\isanewline
\isakeyword{begin}\isanewline
\isanewline
\isanewline
\isacommand{definition}\isamarkupfalse%
\ Union{\isacharunderscore}{\kern0pt}name{\isacharunderscore}{\kern0pt}body\ {\isacharcolon}{\kern0pt}{\isacharcolon}{\kern0pt}\ {\isachardoublequoteopen}{\isacharbrackleft}{\kern0pt}i{\isacharcomma}{\kern0pt}i{\isacharcomma}{\kern0pt}i{\isacharcomma}{\kern0pt}i{\isacharbrackright}{\kern0pt}\ {\isasymRightarrow}\ o{\isachardoublequoteclose}\ \isakeyword{where}\isanewline
\ \ {\isachardoublequoteopen}Union{\isacharunderscore}{\kern0pt}name{\isacharunderscore}{\kern0pt}body{\isacharparenleft}{\kern0pt}P{\isacharprime}{\kern0pt}{\isacharcomma}{\kern0pt}leq{\isacharprime}{\kern0pt}{\isacharcomma}{\kern0pt}{\isasymtau}{\isacharcomma}{\kern0pt}{\isasymtheta}p{\isacharparenright}{\kern0pt}\ {\isasymequiv}\ {\isacharparenleft}{\kern0pt}{\isasymexists}\ {\isasymsigma}{\isacharbrackleft}{\kern0pt}{\isacharhash}{\kern0pt}{\isacharhash}{\kern0pt}M{\isacharbrackright}{\kern0pt}{\isachardot}{\kern0pt}\isanewline
\ \ \ \ \ \ \ \ \ \ \ {\isasymexists}\ q{\isacharbrackleft}{\kern0pt}{\isacharhash}{\kern0pt}{\isacharhash}{\kern0pt}M{\isacharbrackright}{\kern0pt}{\isachardot}{\kern0pt}\ {\isacharparenleft}{\kern0pt}q{\isasymin}\ P{\isacharprime}{\kern0pt}\ {\isasymand}\ {\isacharparenleft}{\kern0pt}{\isasymlangle}{\isasymsigma}{\isacharcomma}{\kern0pt}q{\isasymrangle}\ {\isasymin}\ {\isasymtau}\ {\isasymand}\isanewline
\ \ \ \ \ \ \ \ \ \ \ \ {\isacharparenleft}{\kern0pt}{\isasymexists}\ r{\isacharbrackleft}{\kern0pt}{\isacharhash}{\kern0pt}{\isacharhash}{\kern0pt}M{\isacharbrackright}{\kern0pt}{\isachardot}{\kern0pt}r{\isasymin}P{\isacharprime}{\kern0pt}\ {\isasymand}\ {\isacharparenleft}{\kern0pt}{\isasymlangle}fst{\isacharparenleft}{\kern0pt}{\isasymtheta}p{\isacharparenright}{\kern0pt}{\isacharcomma}{\kern0pt}r{\isasymrangle}\ {\isasymin}\ {\isasymsigma}\ {\isasymand}\ {\isasymlangle}snd{\isacharparenleft}{\kern0pt}{\isasymtheta}p{\isacharparenright}{\kern0pt}{\isacharcomma}{\kern0pt}r{\isasymrangle}\ {\isasymin}\ leq{\isacharprime}{\kern0pt}\ {\isasymand}\ {\isasymlangle}snd{\isacharparenleft}{\kern0pt}{\isasymtheta}p{\isacharparenright}{\kern0pt}{\isacharcomma}{\kern0pt}q{\isasymrangle}\ {\isasymin}\ leq{\isacharprime}{\kern0pt}{\isacharparenright}{\kern0pt}{\isacharparenright}{\kern0pt}{\isacharparenright}{\kern0pt}{\isacharparenright}{\kern0pt}{\isacharparenright}{\kern0pt}{\isachardoublequoteclose}\isanewline
\isanewline
\isacommand{definition}\isamarkupfalse%
\ Union{\isacharunderscore}{\kern0pt}name{\isacharunderscore}{\kern0pt}fm\ {\isacharcolon}{\kern0pt}{\isacharcolon}{\kern0pt}\ {\isachardoublequoteopen}i{\isachardoublequoteclose}\ \isakeyword{where}\isanewline
\ \ {\isachardoublequoteopen}Union{\isacharunderscore}{\kern0pt}name{\isacharunderscore}{\kern0pt}fm\ {\isasymequiv}\isanewline
\ \ \ \ Exists{\isacharparenleft}{\kern0pt}\isanewline
\ \ \ \ Exists{\isacharparenleft}{\kern0pt}And{\isacharparenleft}{\kern0pt}pair{\isacharunderscore}{\kern0pt}fm{\isacharparenleft}{\kern0pt}{\isadigit{1}}{\isacharcomma}{\kern0pt}{\isadigit{0}}{\isacharcomma}{\kern0pt}{\isadigit{2}}{\isacharparenright}{\kern0pt}{\isacharcomma}{\kern0pt}\isanewline
\ \ \ \ Exists\ {\isacharparenleft}{\kern0pt}\isanewline
\ \ \ \ Exists\ {\isacharparenleft}{\kern0pt}And{\isacharparenleft}{\kern0pt}Member{\isacharparenleft}{\kern0pt}{\isadigit{0}}{\isacharcomma}{\kern0pt}{\isadigit{7}}{\isacharparenright}{\kern0pt}{\isacharcomma}{\kern0pt}\isanewline
\ \ \ \ \ \ Exists\ {\isacharparenleft}{\kern0pt}And{\isacharparenleft}{\kern0pt}And{\isacharparenleft}{\kern0pt}pair{\isacharunderscore}{\kern0pt}fm{\isacharparenleft}{\kern0pt}{\isadigit{2}}{\isacharcomma}{\kern0pt}{\isadigit{1}}{\isacharcomma}{\kern0pt}{\isadigit{0}}{\isacharparenright}{\kern0pt}{\isacharcomma}{\kern0pt}Member{\isacharparenleft}{\kern0pt}{\isadigit{0}}{\isacharcomma}{\kern0pt}{\isadigit{6}}{\isacharparenright}{\kern0pt}{\isacharparenright}{\kern0pt}{\isacharcomma}{\kern0pt}\isanewline
\ \ \ \ \ \ \ \ Exists\ {\isacharparenleft}{\kern0pt}And{\isacharparenleft}{\kern0pt}Member{\isacharparenleft}{\kern0pt}{\isadigit{0}}{\isacharcomma}{\kern0pt}{\isadigit{9}}{\isacharparenright}{\kern0pt}{\isacharcomma}{\kern0pt}\isanewline
\ \ \ \ \ \ \ \ \ Exists\ {\isacharparenleft}{\kern0pt}And{\isacharparenleft}{\kern0pt}And{\isacharparenleft}{\kern0pt}pair{\isacharunderscore}{\kern0pt}fm{\isacharparenleft}{\kern0pt}{\isadigit{6}}{\isacharcomma}{\kern0pt}{\isadigit{1}}{\isacharcomma}{\kern0pt}{\isadigit{0}}{\isacharparenright}{\kern0pt}{\isacharcomma}{\kern0pt}Member{\isacharparenleft}{\kern0pt}{\isadigit{0}}{\isacharcomma}{\kern0pt}{\isadigit{4}}{\isacharparenright}{\kern0pt}{\isacharparenright}{\kern0pt}{\isacharcomma}{\kern0pt}\isanewline
\ \ \ \ \ \ \ \ \ \ Exists\ {\isacharparenleft}{\kern0pt}And{\isacharparenleft}{\kern0pt}And{\isacharparenleft}{\kern0pt}pair{\isacharunderscore}{\kern0pt}fm{\isacharparenleft}{\kern0pt}{\isadigit{6}}{\isacharcomma}{\kern0pt}{\isadigit{2}}{\isacharcomma}{\kern0pt}{\isadigit{0}}{\isacharparenright}{\kern0pt}{\isacharcomma}{\kern0pt}Member{\isacharparenleft}{\kern0pt}{\isadigit{0}}{\isacharcomma}{\kern0pt}{\isadigit{1}}{\isadigit{0}}{\isacharparenright}{\kern0pt}{\isacharparenright}{\kern0pt}{\isacharcomma}{\kern0pt}\isanewline
\ \ \ \ \ \ \ \ \ \ Exists\ {\isacharparenleft}{\kern0pt}And{\isacharparenleft}{\kern0pt}pair{\isacharunderscore}{\kern0pt}fm{\isacharparenleft}{\kern0pt}{\isadigit{7}}{\isacharcomma}{\kern0pt}{\isadigit{5}}{\isacharcomma}{\kern0pt}{\isadigit{0}}{\isacharparenright}{\kern0pt}{\isacharcomma}{\kern0pt}Member{\isacharparenleft}{\kern0pt}{\isadigit{0}}{\isacharcomma}{\kern0pt}{\isadigit{1}}{\isadigit{1}}{\isacharparenright}{\kern0pt}{\isacharparenright}{\kern0pt}{\isacharparenright}{\kern0pt}{\isacharparenright}{\kern0pt}{\isacharparenright}{\kern0pt}{\isacharparenright}{\kern0pt}{\isacharparenright}{\kern0pt}{\isacharparenright}{\kern0pt}{\isacharparenright}{\kern0pt}{\isacharparenright}{\kern0pt}{\isacharparenright}{\kern0pt}{\isacharparenright}{\kern0pt}{\isacharparenright}{\kern0pt}{\isacharparenright}{\kern0pt}{\isacharparenright}{\kern0pt}{\isacharparenright}{\kern0pt}{\isacharparenright}{\kern0pt}{\isachardoublequoteclose}\isanewline
\isanewline
\isacommand{lemma}\isamarkupfalse%
\ Union{\isacharunderscore}{\kern0pt}name{\isacharunderscore}{\kern0pt}fm{\isacharunderscore}{\kern0pt}type\ {\isacharbrackleft}{\kern0pt}TC{\isacharbrackright}{\kern0pt}{\isacharcolon}{\kern0pt}\isanewline
\ \ {\isachardoublequoteopen}Union{\isacharunderscore}{\kern0pt}name{\isacharunderscore}{\kern0pt}fm\ {\isasymin}formula{\isachardoublequoteclose}\isanewline
%
\isadelimproof
\ \ %
\endisadelimproof
%
\isatagproof
\isacommand{unfolding}\isamarkupfalse%
\ Union{\isacharunderscore}{\kern0pt}name{\isacharunderscore}{\kern0pt}fm{\isacharunderscore}{\kern0pt}def\ \isacommand{by}\isamarkupfalse%
\ simp%
\endisatagproof
{\isafoldproof}%
%
\isadelimproof
\isanewline
%
\endisadelimproof
\isanewline
\isanewline
\isacommand{lemma}\isamarkupfalse%
\ arity{\isacharunderscore}{\kern0pt}Union{\isacharunderscore}{\kern0pt}name{\isacharunderscore}{\kern0pt}fm\ {\isacharcolon}{\kern0pt}\isanewline
\ \ {\isachardoublequoteopen}arity{\isacharparenleft}{\kern0pt}Union{\isacharunderscore}{\kern0pt}name{\isacharunderscore}{\kern0pt}fm{\isacharparenright}{\kern0pt}\ {\isacharequal}{\kern0pt}\ {\isadigit{4}}{\isachardoublequoteclose}\isanewline
%
\isadelimproof
\ \ %
\endisadelimproof
%
\isatagproof
\isacommand{unfolding}\isamarkupfalse%
\ Union{\isacharunderscore}{\kern0pt}name{\isacharunderscore}{\kern0pt}fm{\isacharunderscore}{\kern0pt}def\ upair{\isacharunderscore}{\kern0pt}fm{\isacharunderscore}{\kern0pt}def\ pair{\isacharunderscore}{\kern0pt}fm{\isacharunderscore}{\kern0pt}def\isanewline
\ \ \isacommand{by}\isamarkupfalse%
{\isacharparenleft}{\kern0pt}auto\ simp\ add{\isacharcolon}{\kern0pt}\ nat{\isacharunderscore}{\kern0pt}simp{\isacharunderscore}{\kern0pt}union{\isacharparenright}{\kern0pt}%
\endisatagproof
{\isafoldproof}%
%
\isadelimproof
\isanewline
%
\endisadelimproof
\isanewline
\isacommand{lemma}\isamarkupfalse%
\ sats{\isacharunderscore}{\kern0pt}Union{\isacharunderscore}{\kern0pt}name{\isacharunderscore}{\kern0pt}fm\ {\isacharcolon}{\kern0pt}\isanewline
\ \ {\isachardoublequoteopen}{\isasymlbrakk}\ a\ {\isasymin}\ M{\isacharsemicolon}{\kern0pt}\ b\ {\isasymin}\ M\ {\isacharsemicolon}{\kern0pt}\ P{\isacharprime}{\kern0pt}\ {\isasymin}\ M\ {\isacharsemicolon}{\kern0pt}\ p\ {\isasymin}\ M\ {\isacharsemicolon}{\kern0pt}\ {\isasymtheta}\ {\isasymin}\ M\ {\isacharsemicolon}{\kern0pt}\ {\isasymtau}\ {\isasymin}\ M\ {\isacharsemicolon}{\kern0pt}\ leq{\isacharprime}{\kern0pt}\ {\isasymin}\ M\ {\isasymrbrakk}\ {\isasymLongrightarrow}\isanewline
\ \ \ \ \ sats{\isacharparenleft}{\kern0pt}M{\isacharcomma}{\kern0pt}Union{\isacharunderscore}{\kern0pt}name{\isacharunderscore}{\kern0pt}fm{\isacharcomma}{\kern0pt}{\isacharbrackleft}{\kern0pt}{\isasymlangle}{\isasymtheta}{\isacharcomma}{\kern0pt}p{\isasymrangle}{\isacharcomma}{\kern0pt}{\isasymtau}{\isacharcomma}{\kern0pt}leq{\isacharprime}{\kern0pt}{\isacharcomma}{\kern0pt}P{\isacharprime}{\kern0pt}{\isacharbrackright}{\kern0pt}{\isacharat}{\kern0pt}{\isacharbrackleft}{\kern0pt}a{\isacharcomma}{\kern0pt}b{\isacharbrackright}{\kern0pt}{\isacharparenright}{\kern0pt}\ {\isasymlongleftrightarrow}\isanewline
\ \ \ \ \ Union{\isacharunderscore}{\kern0pt}name{\isacharunderscore}{\kern0pt}body{\isacharparenleft}{\kern0pt}P{\isacharprime}{\kern0pt}{\isacharcomma}{\kern0pt}leq{\isacharprime}{\kern0pt}{\isacharcomma}{\kern0pt}{\isasymtau}{\isacharcomma}{\kern0pt}{\isasymlangle}{\isasymtheta}{\isacharcomma}{\kern0pt}p{\isasymrangle}{\isacharparenright}{\kern0pt}{\isachardoublequoteclose}\isanewline
%
\isadelimproof
\ \ %
\endisadelimproof
%
\isatagproof
\isacommand{unfolding}\isamarkupfalse%
\ Union{\isacharunderscore}{\kern0pt}name{\isacharunderscore}{\kern0pt}fm{\isacharunderscore}{\kern0pt}def\ Union{\isacharunderscore}{\kern0pt}name{\isacharunderscore}{\kern0pt}body{\isacharunderscore}{\kern0pt}def\ tuples{\isacharunderscore}{\kern0pt}in{\isacharunderscore}{\kern0pt}M\isanewline
\ \ \isacommand{by}\isamarkupfalse%
\ {\isacharparenleft}{\kern0pt}subgoal{\isacharunderscore}{\kern0pt}tac\ {\isachardoublequoteopen}{\isasymlangle}{\isasymtheta}{\isacharcomma}{\kern0pt}p{\isasymrangle}\ {\isasymin}\ M{\isachardoublequoteclose}{\isacharcomma}{\kern0pt}\ auto\ simp\ add\ {\isacharcolon}{\kern0pt}\ tuples{\isacharunderscore}{\kern0pt}in{\isacharunderscore}{\kern0pt}M{\isacharparenright}{\kern0pt}%
\endisatagproof
{\isafoldproof}%
%
\isadelimproof
\isanewline
%
\endisadelimproof
\isanewline
\isanewline
\isacommand{lemma}\isamarkupfalse%
\ domD\ {\isacharcolon}{\kern0pt}\isanewline
\ \ \isakeyword{assumes}\ {\isachardoublequoteopen}{\isasymtau}\ {\isasymin}\ M{\isachardoublequoteclose}\ {\isachardoublequoteopen}{\isasymsigma}\ {\isasymin}\ domain{\isacharparenleft}{\kern0pt}{\isasymtau}{\isacharparenright}{\kern0pt}{\isachardoublequoteclose}\isanewline
\ \ \isakeyword{shows}\ {\isachardoublequoteopen}{\isasymsigma}\ {\isasymin}\ M{\isachardoublequoteclose}\isanewline
%
\isadelimproof
\ \ %
\endisadelimproof
%
\isatagproof
\isacommand{using}\isamarkupfalse%
\ assms\ Transset{\isacharunderscore}{\kern0pt}M\ trans{\isacharunderscore}{\kern0pt}M\isanewline
\ \ \isacommand{by}\isamarkupfalse%
\ {\isacharparenleft}{\kern0pt}simp\ flip{\isacharcolon}{\kern0pt}\ setclass{\isacharunderscore}{\kern0pt}iff{\isacharparenright}{\kern0pt}%
\endisatagproof
{\isafoldproof}%
%
\isadelimproof
\isanewline
%
\endisadelimproof
\isanewline
\isanewline
\isacommand{definition}\isamarkupfalse%
\ Union{\isacharunderscore}{\kern0pt}name\ {\isacharcolon}{\kern0pt}{\isacharcolon}{\kern0pt}\ {\isachardoublequoteopen}i\ {\isasymRightarrow}\ i{\isachardoublequoteclose}\ \isakeyword{where}\isanewline
\ \ {\isachardoublequoteopen}Union{\isacharunderscore}{\kern0pt}name{\isacharparenleft}{\kern0pt}{\isasymtau}{\isacharparenright}{\kern0pt}\ {\isasymequiv}\isanewline
\ \ \ \ {\isacharbraceleft}{\kern0pt}u\ {\isasymin}\ domain{\isacharparenleft}{\kern0pt}{\isasymUnion}{\isacharparenleft}{\kern0pt}domain{\isacharparenleft}{\kern0pt}{\isasymtau}{\isacharparenright}{\kern0pt}{\isacharparenright}{\kern0pt}{\isacharparenright}{\kern0pt}\ {\isasymtimes}\ P\ {\isachardot}{\kern0pt}\ Union{\isacharunderscore}{\kern0pt}name{\isacharunderscore}{\kern0pt}body{\isacharparenleft}{\kern0pt}P{\isacharcomma}{\kern0pt}leq{\isacharcomma}{\kern0pt}{\isasymtau}{\isacharcomma}{\kern0pt}u{\isacharparenright}{\kern0pt}{\isacharbraceright}{\kern0pt}{\isachardoublequoteclose}\isanewline
\isanewline
\isacommand{lemma}\isamarkupfalse%
\ Union{\isacharunderscore}{\kern0pt}name{\isacharunderscore}{\kern0pt}M\ {\isacharcolon}{\kern0pt}\ \isakeyword{assumes}\ {\isachardoublequoteopen}{\isasymtau}\ {\isasymin}\ M{\isachardoublequoteclose}\isanewline
\ \ \isakeyword{shows}\ {\isachardoublequoteopen}{\isacharbraceleft}{\kern0pt}u\ {\isasymin}\ domain{\isacharparenleft}{\kern0pt}{\isasymUnion}{\isacharparenleft}{\kern0pt}domain{\isacharparenleft}{\kern0pt}{\isasymtau}{\isacharparenright}{\kern0pt}{\isacharparenright}{\kern0pt}{\isacharparenright}{\kern0pt}\ {\isasymtimes}\ P\ {\isachardot}{\kern0pt}\ Union{\isacharunderscore}{\kern0pt}name{\isacharunderscore}{\kern0pt}body{\isacharparenleft}{\kern0pt}P{\isacharcomma}{\kern0pt}leq{\isacharcomma}{\kern0pt}{\isasymtau}{\isacharcomma}{\kern0pt}u{\isacharparenright}{\kern0pt}{\isacharbraceright}{\kern0pt}\ {\isasymin}\ M{\isachardoublequoteclose}\isanewline
%
\isadelimproof
\ \ %
\endisadelimproof
%
\isatagproof
\isacommand{unfolding}\isamarkupfalse%
\ Union{\isacharunderscore}{\kern0pt}name{\isacharunderscore}{\kern0pt}def\isanewline
\isacommand{proof}\isamarkupfalse%
\ {\isacharminus}{\kern0pt}\isanewline
\ \ \isacommand{let}\isamarkupfalse%
\ {\isacharquery}{\kern0pt}P{\isacharequal}{\kern0pt}{\isachardoublequoteopen}{\isasymlambda}\ x\ {\isachardot}{\kern0pt}\ sats{\isacharparenleft}{\kern0pt}M{\isacharcomma}{\kern0pt}Union{\isacharunderscore}{\kern0pt}name{\isacharunderscore}{\kern0pt}fm{\isacharcomma}{\kern0pt}{\isacharbrackleft}{\kern0pt}x{\isacharcomma}{\kern0pt}{\isasymtau}{\isacharcomma}{\kern0pt}leq{\isacharbrackright}{\kern0pt}{\isacharat}{\kern0pt}{\isacharbrackleft}{\kern0pt}P{\isacharcomma}{\kern0pt}{\isasymtau}{\isacharcomma}{\kern0pt}leq{\isacharbrackright}{\kern0pt}{\isacharparenright}{\kern0pt}{\isachardoublequoteclose}\isanewline
\ \ \isacommand{let}\isamarkupfalse%
\ {\isacharquery}{\kern0pt}Q{\isacharequal}{\kern0pt}{\isachardoublequoteopen}{\isasymlambda}\ x\ {\isachardot}{\kern0pt}\ Union{\isacharunderscore}{\kern0pt}name{\isacharunderscore}{\kern0pt}body{\isacharparenleft}{\kern0pt}P{\isacharcomma}{\kern0pt}leq{\isacharcomma}{\kern0pt}{\isasymtau}{\isacharcomma}{\kern0pt}x{\isacharparenright}{\kern0pt}{\isachardoublequoteclose}\isanewline
\ \ \isacommand{from}\isamarkupfalse%
\ {\isacartoucheopen}{\isasymtau}{\isasymin}M{\isacartoucheclose}\isanewline
\ \ \isacommand{have}\isamarkupfalse%
\ {\isachardoublequoteopen}domain{\isacharparenleft}{\kern0pt}{\isasymUnion}{\isacharparenleft}{\kern0pt}domain{\isacharparenleft}{\kern0pt}{\isasymtau}{\isacharparenright}{\kern0pt}{\isacharparenright}{\kern0pt}{\isacharparenright}{\kern0pt}{\isasymin}M{\isachardoublequoteclose}\ {\isacharparenleft}{\kern0pt}\isakeyword{is}\ {\isachardoublequoteopen}{\isacharquery}{\kern0pt}d\ {\isasymin}\ {\isacharunderscore}{\kern0pt}{\isachardoublequoteclose}{\isacharparenright}{\kern0pt}\ \isacommand{using}\isamarkupfalse%
\ domain{\isacharunderscore}{\kern0pt}closed\ Union{\isacharunderscore}{\kern0pt}closed\ \isacommand{by}\isamarkupfalse%
\ simp\isanewline
\ \ \isacommand{then}\isamarkupfalse%
\isanewline
\ \ \isacommand{have}\isamarkupfalse%
\ {\isachardoublequoteopen}{\isacharquery}{\kern0pt}d\ {\isasymtimes}\ P\ {\isasymin}\ M{\isachardoublequoteclose}\ \isacommand{using}\isamarkupfalse%
\ cartprod{\isacharunderscore}{\kern0pt}closed\ P{\isacharunderscore}{\kern0pt}in{\isacharunderscore}{\kern0pt}M\ \isacommand{by}\isamarkupfalse%
\ simp\isanewline
\ \ \isacommand{have}\isamarkupfalse%
\ {\isachardoublequoteopen}arity{\isacharparenleft}{\kern0pt}Union{\isacharunderscore}{\kern0pt}name{\isacharunderscore}{\kern0pt}fm{\isacharparenright}{\kern0pt}{\isasymle}{\isadigit{6}}{\isachardoublequoteclose}\ \isacommand{using}\isamarkupfalse%
\ arity{\isacharunderscore}{\kern0pt}Union{\isacharunderscore}{\kern0pt}name{\isacharunderscore}{\kern0pt}fm\ \isacommand{by}\isamarkupfalse%
\ simp\isanewline
\ \ \isacommand{from}\isamarkupfalse%
\ assms\ P{\isacharunderscore}{\kern0pt}in{\isacharunderscore}{\kern0pt}M\ leq{\isacharunderscore}{\kern0pt}in{\isacharunderscore}{\kern0pt}M\ \ arity{\isacharunderscore}{\kern0pt}Union{\isacharunderscore}{\kern0pt}name{\isacharunderscore}{\kern0pt}fm\isanewline
\ \ \isacommand{have}\isamarkupfalse%
\ {\isachardoublequoteopen}{\isacharbrackleft}{\kern0pt}{\isasymtau}{\isacharcomma}{\kern0pt}leq{\isacharbrackright}{\kern0pt}\ {\isasymin}\ list{\isacharparenleft}{\kern0pt}M{\isacharparenright}{\kern0pt}{\isachardoublequoteclose}\ {\isachardoublequoteopen}{\isacharbrackleft}{\kern0pt}P{\isacharcomma}{\kern0pt}{\isasymtau}{\isacharcomma}{\kern0pt}leq{\isacharbrackright}{\kern0pt}\ {\isasymin}\ list{\isacharparenleft}{\kern0pt}M{\isacharparenright}{\kern0pt}{\isachardoublequoteclose}\ \isacommand{by}\isamarkupfalse%
\ auto\isanewline
\ \ \isacommand{with}\isamarkupfalse%
\ assms\ assms\ P{\isacharunderscore}{\kern0pt}in{\isacharunderscore}{\kern0pt}M\ leq{\isacharunderscore}{\kern0pt}in{\isacharunderscore}{\kern0pt}M\ \ {\isacartoucheopen}arity{\isacharparenleft}{\kern0pt}Union{\isacharunderscore}{\kern0pt}name{\isacharunderscore}{\kern0pt}fm{\isacharparenright}{\kern0pt}{\isasymle}{\isadigit{6}}{\isacartoucheclose}\isanewline
\ \ \isacommand{have}\isamarkupfalse%
\ {\isachardoublequoteopen}separation{\isacharparenleft}{\kern0pt}{\isacharhash}{\kern0pt}{\isacharhash}{\kern0pt}M{\isacharcomma}{\kern0pt}{\isacharquery}{\kern0pt}P{\isacharparenright}{\kern0pt}{\isachardoublequoteclose}\isanewline
\ \ \ \ \isacommand{using}\isamarkupfalse%
\ separation{\isacharunderscore}{\kern0pt}ax\ \isacommand{by}\isamarkupfalse%
\ simp\isanewline
\ \ \isacommand{with}\isamarkupfalse%
\ {\isacartoucheopen}{\isacharquery}{\kern0pt}d\ {\isasymtimes}\ P\ {\isasymin}\ M{\isacartoucheclose}\isanewline
\ \ \isacommand{have}\isamarkupfalse%
\ A{\isacharcolon}{\kern0pt}{\isachardoublequoteopen}{\isacharbraceleft}{\kern0pt}\ u\ {\isasymin}\ {\isacharquery}{\kern0pt}d\ {\isasymtimes}\ P\ {\isachardot}{\kern0pt}\ {\isacharquery}{\kern0pt}P{\isacharparenleft}{\kern0pt}u{\isacharparenright}{\kern0pt}\ {\isacharbraceright}{\kern0pt}\ {\isasymin}\ M{\isachardoublequoteclose}\isanewline
\ \ \ \ \isacommand{using}\isamarkupfalse%
\ \ separation{\isacharunderscore}{\kern0pt}iff\ \isacommand{by}\isamarkupfalse%
\ force\isanewline
\ \ \isacommand{have}\isamarkupfalse%
\ {\isachardoublequoteopen}{\isacharquery}{\kern0pt}P{\isacharparenleft}{\kern0pt}x{\isacharparenright}{\kern0pt}{\isasymlongleftrightarrow}\ {\isacharquery}{\kern0pt}Q{\isacharparenleft}{\kern0pt}x{\isacharparenright}{\kern0pt}{\isachardoublequoteclose}\ \isakeyword{if}\ {\isachardoublequoteopen}x{\isasymin}\ {\isacharquery}{\kern0pt}d{\isasymtimes}P{\isachardoublequoteclose}\ \isakeyword{for}\ x\isanewline
\ \ \isacommand{proof}\isamarkupfalse%
\ {\isacharminus}{\kern0pt}\isanewline
\ \ \ \ \isacommand{from}\isamarkupfalse%
\ {\isacartoucheopen}x{\isasymin}\ {\isacharquery}{\kern0pt}d{\isasymtimes}P{\isacartoucheclose}\isanewline
\ \ \ \ \isacommand{have}\isamarkupfalse%
\ {\isachardoublequoteopen}x\ {\isacharequal}{\kern0pt}\ {\isasymlangle}fst{\isacharparenleft}{\kern0pt}x{\isacharparenright}{\kern0pt}{\isacharcomma}{\kern0pt}snd{\isacharparenleft}{\kern0pt}x{\isacharparenright}{\kern0pt}{\isasymrangle}{\isachardoublequoteclose}\ \isacommand{using}\isamarkupfalse%
\ Pair{\isacharunderscore}{\kern0pt}fst{\isacharunderscore}{\kern0pt}snd{\isacharunderscore}{\kern0pt}eq\ \isacommand{by}\isamarkupfalse%
\ simp\isanewline
\ \ \ \ \isacommand{with}\isamarkupfalse%
\ {\isacartoucheopen}x{\isasymin}{\isacharquery}{\kern0pt}d{\isasymtimes}P{\isacartoucheclose}\ {\isacartoucheopen}{\isacharquery}{\kern0pt}d{\isasymin}M{\isacartoucheclose}\isanewline
\ \ \ \ \isacommand{have}\isamarkupfalse%
\ {\isachardoublequoteopen}fst{\isacharparenleft}{\kern0pt}x{\isacharparenright}{\kern0pt}\ {\isasymin}\ M{\isachardoublequoteclose}\ {\isachardoublequoteopen}snd{\isacharparenleft}{\kern0pt}x{\isacharparenright}{\kern0pt}\ {\isasymin}\ M{\isachardoublequoteclose}\isanewline
\ \ \ \ \ \ \isacommand{using}\isamarkupfalse%
\ mtrans\ fst{\isacharunderscore}{\kern0pt}type\ snd{\isacharunderscore}{\kern0pt}type\ P{\isacharunderscore}{\kern0pt}in{\isacharunderscore}{\kern0pt}M\ \isacommand{unfolding}\isamarkupfalse%
\ M{\isacharunderscore}{\kern0pt}trans{\isacharunderscore}{\kern0pt}def\ \isacommand{by}\isamarkupfalse%
\ auto\isanewline
\ \ \ \ \isacommand{then}\isamarkupfalse%
\isanewline
\ \ \ \ \isacommand{have}\isamarkupfalse%
\ {\isachardoublequoteopen}{\isacharquery}{\kern0pt}P{\isacharparenleft}{\kern0pt}{\isasymlangle}fst{\isacharparenleft}{\kern0pt}x{\isacharparenright}{\kern0pt}{\isacharcomma}{\kern0pt}snd{\isacharparenleft}{\kern0pt}x{\isacharparenright}{\kern0pt}{\isasymrangle}{\isacharparenright}{\kern0pt}\ {\isasymlongleftrightarrow}\ \ {\isacharquery}{\kern0pt}Q{\isacharparenleft}{\kern0pt}{\isasymlangle}fst{\isacharparenleft}{\kern0pt}x{\isacharparenright}{\kern0pt}{\isacharcomma}{\kern0pt}snd{\isacharparenleft}{\kern0pt}x{\isacharparenright}{\kern0pt}{\isasymrangle}{\isacharparenright}{\kern0pt}{\isachardoublequoteclose}\isanewline
\ \ \ \ \ \ \isacommand{using}\isamarkupfalse%
\ P{\isacharunderscore}{\kern0pt}in{\isacharunderscore}{\kern0pt}M\ sats{\isacharunderscore}{\kern0pt}Union{\isacharunderscore}{\kern0pt}name{\isacharunderscore}{\kern0pt}fm\ P{\isacharunderscore}{\kern0pt}in{\isacharunderscore}{\kern0pt}M\ {\isacartoucheopen}{\isasymtau}{\isasymin}M{\isacartoucheclose}\ leq{\isacharunderscore}{\kern0pt}in{\isacharunderscore}{\kern0pt}M\ \isacommand{by}\isamarkupfalse%
\ simp\isanewline
\ \ \ \ \isacommand{with}\isamarkupfalse%
\ {\isacartoucheopen}x\ {\isacharequal}{\kern0pt}\ {\isasymlangle}fst{\isacharparenleft}{\kern0pt}x{\isacharparenright}{\kern0pt}{\isacharcomma}{\kern0pt}snd{\isacharparenleft}{\kern0pt}x{\isacharparenright}{\kern0pt}{\isasymrangle}{\isacartoucheclose}\isanewline
\ \ \ \ \isacommand{show}\isamarkupfalse%
\ {\isachardoublequoteopen}{\isacharquery}{\kern0pt}P{\isacharparenleft}{\kern0pt}x{\isacharparenright}{\kern0pt}\ {\isasymlongleftrightarrow}\ {\isacharquery}{\kern0pt}Q{\isacharparenleft}{\kern0pt}x{\isacharparenright}{\kern0pt}{\isachardoublequoteclose}\ \isacommand{using}\isamarkupfalse%
\ that\ \isacommand{by}\isamarkupfalse%
\ simp\isanewline
\ \ \isacommand{qed}\isamarkupfalse%
\isanewline
\ \ \isacommand{then}\isamarkupfalse%
\ \isacommand{show}\isamarkupfalse%
\ {\isacharquery}{\kern0pt}thesis\ \isacommand{using}\isamarkupfalse%
\ Collect{\isacharunderscore}{\kern0pt}cong\ A\ \isacommand{by}\isamarkupfalse%
\ simp\isanewline
\isacommand{qed}\isamarkupfalse%
%
\endisatagproof
{\isafoldproof}%
%
\isadelimproof
\isanewline
%
\endisadelimproof
\isanewline
\isanewline
\isanewline
\isacommand{lemma}\isamarkupfalse%
\ Union{\isacharunderscore}{\kern0pt}MG{\isacharunderscore}{\kern0pt}Eq\ {\isacharcolon}{\kern0pt}\isanewline
\ \ \isakeyword{assumes}\ {\isachardoublequoteopen}a\ {\isasymin}\ M{\isacharbrackleft}{\kern0pt}G{\isacharbrackright}{\kern0pt}{\isachardoublequoteclose}\ \isakeyword{and}\ {\isachardoublequoteopen}a\ {\isacharequal}{\kern0pt}\ val{\isacharparenleft}{\kern0pt}G{\isacharcomma}{\kern0pt}{\isasymtau}{\isacharparenright}{\kern0pt}{\isachardoublequoteclose}\ \isakeyword{and}\ {\isachardoublequoteopen}filter{\isacharparenleft}{\kern0pt}G{\isacharparenright}{\kern0pt}{\isachardoublequoteclose}\ \isakeyword{and}\ {\isachardoublequoteopen}{\isasymtau}\ {\isasymin}\ M{\isachardoublequoteclose}\isanewline
\ \ \isakeyword{shows}\ {\isachardoublequoteopen}{\isasymUnion}\ a\ {\isacharequal}{\kern0pt}\ val{\isacharparenleft}{\kern0pt}G{\isacharcomma}{\kern0pt}Union{\isacharunderscore}{\kern0pt}name{\isacharparenleft}{\kern0pt}{\isasymtau}{\isacharparenright}{\kern0pt}{\isacharparenright}{\kern0pt}{\isachardoublequoteclose}\isanewline
%
\isadelimproof
%
\endisadelimproof
%
\isatagproof
\isacommand{proof}\isamarkupfalse%
\ {\isacharminus}{\kern0pt}\isanewline
\ \ \isacommand{{\isacharbraceleft}{\kern0pt}}\isamarkupfalse%
\isanewline
\ \ \ \ \isacommand{fix}\isamarkupfalse%
\ x\isanewline
\ \ \ \ \isacommand{assume}\isamarkupfalse%
\ {\isachardoublequoteopen}x\ {\isasymin}\ {\isasymUnion}\ {\isacharparenleft}{\kern0pt}val{\isacharparenleft}{\kern0pt}G{\isacharcomma}{\kern0pt}{\isasymtau}{\isacharparenright}{\kern0pt}{\isacharparenright}{\kern0pt}{\isachardoublequoteclose}\isanewline
\ \ \ \ \isacommand{then}\isamarkupfalse%
\ \isacommand{obtain}\isamarkupfalse%
\ i\ \isakeyword{where}\ {\isachardoublequoteopen}i\ {\isasymin}\ val{\isacharparenleft}{\kern0pt}G{\isacharcomma}{\kern0pt}{\isasymtau}{\isacharparenright}{\kern0pt}{\isachardoublequoteclose}\ {\isachardoublequoteopen}x\ {\isasymin}\ i{\isachardoublequoteclose}\ \isacommand{by}\isamarkupfalse%
\ blast\isanewline
\ \ \ \ \isacommand{with}\isamarkupfalse%
\ {\isacartoucheopen}{\isasymtau}\ {\isasymin}\ M{\isacartoucheclose}\ \isacommand{obtain}\isamarkupfalse%
\ {\isasymsigma}\ q\ \isakeyword{where}\isanewline
\ \ \ \ \ \ {\isachardoublequoteopen}q\ {\isasymin}\ G{\isachardoublequoteclose}\ {\isachardoublequoteopen}{\isasymlangle}{\isasymsigma}{\isacharcomma}{\kern0pt}q{\isasymrangle}\ {\isasymin}\ {\isasymtau}{\isachardoublequoteclose}\ {\isachardoublequoteopen}val{\isacharparenleft}{\kern0pt}G{\isacharcomma}{\kern0pt}{\isasymsigma}{\isacharparenright}{\kern0pt}\ {\isacharequal}{\kern0pt}\ i{\isachardoublequoteclose}\ {\isachardoublequoteopen}{\isasymsigma}\ {\isasymin}\ M{\isachardoublequoteclose}\isanewline
\ \ \ \ \ \ \isacommand{using}\isamarkupfalse%
\ elem{\isacharunderscore}{\kern0pt}of{\isacharunderscore}{\kern0pt}val{\isacharunderscore}{\kern0pt}pair\ domD\ \isacommand{by}\isamarkupfalse%
\ blast\isanewline
\ \ \ \ \isacommand{with}\isamarkupfalse%
\ {\isacartoucheopen}x\ {\isasymin}\ i{\isacartoucheclose}\ \isacommand{obtain}\isamarkupfalse%
\ {\isasymtheta}\ r\ \isakeyword{where}\isanewline
\ \ \ \ \ \ {\isachardoublequoteopen}r\ {\isasymin}\ G{\isachardoublequoteclose}\ {\isachardoublequoteopen}{\isasymlangle}{\isasymtheta}{\isacharcomma}{\kern0pt}r{\isasymrangle}\ {\isasymin}\ {\isasymsigma}{\isachardoublequoteclose}\ {\isachardoublequoteopen}val{\isacharparenleft}{\kern0pt}G{\isacharcomma}{\kern0pt}{\isasymtheta}{\isacharparenright}{\kern0pt}\ {\isacharequal}{\kern0pt}\ x{\isachardoublequoteclose}\ {\isachardoublequoteopen}{\isasymtheta}\ {\isasymin}\ M{\isachardoublequoteclose}\isanewline
\ \ \ \ \ \ \isacommand{using}\isamarkupfalse%
\ elem{\isacharunderscore}{\kern0pt}of{\isacharunderscore}{\kern0pt}val{\isacharunderscore}{\kern0pt}pair\ domD\ \isacommand{by}\isamarkupfalse%
\ blast\isanewline
\ \ \ \ \isacommand{with}\isamarkupfalse%
\ {\isacartoucheopen}{\isasymlangle}{\isasymsigma}{\isacharcomma}{\kern0pt}q{\isasymrangle}{\isasymin}{\isasymtau}{\isacartoucheclose}\ \isacommand{have}\isamarkupfalse%
\ {\isachardoublequoteopen}{\isasymtheta}\ {\isasymin}\ domain{\isacharparenleft}{\kern0pt}{\isasymUnion}{\isacharparenleft}{\kern0pt}domain{\isacharparenleft}{\kern0pt}{\isasymtau}{\isacharparenright}{\kern0pt}{\isacharparenright}{\kern0pt}{\isacharparenright}{\kern0pt}{\isachardoublequoteclose}\ \isacommand{by}\isamarkupfalse%
\ auto\isanewline
\ \ \ \ \isacommand{with}\isamarkupfalse%
\ {\isacartoucheopen}filter{\isacharparenleft}{\kern0pt}G{\isacharparenright}{\kern0pt}{\isacartoucheclose}\ {\isacartoucheopen}q{\isasymin}G{\isacartoucheclose}\ {\isacartoucheopen}r{\isasymin}G{\isacartoucheclose}\ \isacommand{obtain}\isamarkupfalse%
\ p\ \isakeyword{where}\isanewline
\ \ \ \ \ \ A{\isacharcolon}{\kern0pt}\ {\isachardoublequoteopen}p\ {\isasymin}\ G{\isachardoublequoteclose}\ {\isachardoublequoteopen}{\isasymlangle}p{\isacharcomma}{\kern0pt}r{\isasymrangle}\ {\isasymin}\ leq{\isachardoublequoteclose}\ {\isachardoublequoteopen}{\isasymlangle}p{\isacharcomma}{\kern0pt}q{\isasymrangle}\ {\isasymin}\ leq{\isachardoublequoteclose}\ {\isachardoublequoteopen}p\ {\isasymin}\ P{\isachardoublequoteclose}\ {\isachardoublequoteopen}r\ {\isasymin}\ P{\isachardoublequoteclose}\ {\isachardoublequoteopen}q\ {\isasymin}\ P{\isachardoublequoteclose}\isanewline
\ \ \ \ \ \ \isacommand{using}\isamarkupfalse%
\ low{\isacharunderscore}{\kern0pt}bound{\isacharunderscore}{\kern0pt}filter\ filterD\ \ \isacommand{by}\isamarkupfalse%
\ blast\isanewline
\ \ \ \ \isacommand{then}\isamarkupfalse%
\ \isacommand{have}\isamarkupfalse%
\ {\isachardoublequoteopen}p\ {\isasymin}\ M{\isachardoublequoteclose}\ {\isachardoublequoteopen}q{\isasymin}M{\isachardoublequoteclose}\ {\isachardoublequoteopen}r{\isasymin}M{\isachardoublequoteclose}\isanewline
\ \ \ \ \ \ \isacommand{using}\isamarkupfalse%
\ mtrans\ P{\isacharunderscore}{\kern0pt}in{\isacharunderscore}{\kern0pt}M\ \isacommand{unfolding}\isamarkupfalse%
\ M{\isacharunderscore}{\kern0pt}trans{\isacharunderscore}{\kern0pt}def\ \isacommand{by}\isamarkupfalse%
\ auto\isanewline
\ \ \ \ \isacommand{with}\isamarkupfalse%
\ A\ {\isacartoucheopen}{\isasymlangle}{\isasymtheta}{\isacharcomma}{\kern0pt}r{\isasymrangle}\ {\isasymin}\ {\isasymsigma}{\isacartoucheclose}\ {\isacartoucheopen}{\isasymlangle}{\isasymsigma}{\isacharcomma}{\kern0pt}q{\isasymrangle}\ {\isasymin}\ {\isasymtau}{\isacartoucheclose}\ {\isacartoucheopen}{\isasymtheta}\ {\isasymin}\ M{\isacartoucheclose}\ {\isacartoucheopen}{\isasymtheta}\ {\isasymin}\ domain{\isacharparenleft}{\kern0pt}{\isasymUnion}{\isacharparenleft}{\kern0pt}domain{\isacharparenleft}{\kern0pt}{\isasymtau}{\isacharparenright}{\kern0pt}{\isacharparenright}{\kern0pt}{\isacharparenright}{\kern0pt}{\isacartoucheclose}\ \ {\isacartoucheopen}{\isasymsigma}{\isasymin}M{\isacartoucheclose}\ \isacommand{have}\isamarkupfalse%
\isanewline
\ \ \ \ \ \ {\isachardoublequoteopen}{\isasymlangle}{\isasymtheta}{\isacharcomma}{\kern0pt}p{\isasymrangle}\ {\isasymin}\ Union{\isacharunderscore}{\kern0pt}name{\isacharparenleft}{\kern0pt}{\isasymtau}{\isacharparenright}{\kern0pt}{\isachardoublequoteclose}\ \isacommand{unfolding}\isamarkupfalse%
\ Union{\isacharunderscore}{\kern0pt}name{\isacharunderscore}{\kern0pt}def\ Union{\isacharunderscore}{\kern0pt}name{\isacharunderscore}{\kern0pt}body{\isacharunderscore}{\kern0pt}def\isanewline
\ \ \ \ \ \ \isacommand{by}\isamarkupfalse%
\ auto\isanewline
\ \ \ \ \isacommand{with}\isamarkupfalse%
\ {\isacartoucheopen}p{\isasymin}P{\isacartoucheclose}\ {\isacartoucheopen}p{\isasymin}G{\isacartoucheclose}\ \isacommand{have}\isamarkupfalse%
\ {\isachardoublequoteopen}val{\isacharparenleft}{\kern0pt}G{\isacharcomma}{\kern0pt}{\isasymtheta}{\isacharparenright}{\kern0pt}\ {\isasymin}\ val{\isacharparenleft}{\kern0pt}G{\isacharcomma}{\kern0pt}Union{\isacharunderscore}{\kern0pt}name{\isacharparenleft}{\kern0pt}{\isasymtau}{\isacharparenright}{\kern0pt}{\isacharparenright}{\kern0pt}{\isachardoublequoteclose}\isanewline
\ \ \ \ \ \ \isacommand{using}\isamarkupfalse%
\ val{\isacharunderscore}{\kern0pt}of{\isacharunderscore}{\kern0pt}elem\ \isacommand{by}\isamarkupfalse%
\ simp\isanewline
\ \ \ \ \isacommand{with}\isamarkupfalse%
\ {\isacartoucheopen}val{\isacharparenleft}{\kern0pt}G{\isacharcomma}{\kern0pt}{\isasymtheta}{\isacharparenright}{\kern0pt}{\isacharequal}{\kern0pt}x{\isacartoucheclose}\ \isacommand{have}\isamarkupfalse%
\ {\isachardoublequoteopen}x\ {\isasymin}\ val{\isacharparenleft}{\kern0pt}G{\isacharcomma}{\kern0pt}Union{\isacharunderscore}{\kern0pt}name{\isacharparenleft}{\kern0pt}{\isasymtau}{\isacharparenright}{\kern0pt}{\isacharparenright}{\kern0pt}{\isachardoublequoteclose}\ \isacommand{by}\isamarkupfalse%
\ simp\isanewline
\ \ \isacommand{{\isacharbraceright}{\kern0pt}}\isamarkupfalse%
\isanewline
\ \ \isacommand{with}\isamarkupfalse%
\ {\isacartoucheopen}a{\isacharequal}{\kern0pt}val{\isacharparenleft}{\kern0pt}G{\isacharcomma}{\kern0pt}{\isasymtau}{\isacharparenright}{\kern0pt}{\isacartoucheclose}\ \isacommand{have}\isamarkupfalse%
\ {\isadigit{1}}{\isacharcolon}{\kern0pt}\ {\isachardoublequoteopen}x\ {\isasymin}\ {\isasymUnion}\ a\ {\isasymLongrightarrow}\ x\ {\isasymin}\ val{\isacharparenleft}{\kern0pt}G{\isacharcomma}{\kern0pt}Union{\isacharunderscore}{\kern0pt}name{\isacharparenleft}{\kern0pt}{\isasymtau}{\isacharparenright}{\kern0pt}{\isacharparenright}{\kern0pt}{\isachardoublequoteclose}\ \isakeyword{for}\ x\ \isacommand{by}\isamarkupfalse%
\ simp\isanewline
\ \ \isacommand{{\isacharbraceleft}{\kern0pt}}\isamarkupfalse%
\isanewline
\ \ \ \ \isacommand{fix}\isamarkupfalse%
\ x\isanewline
\ \ \ \ \isacommand{assume}\isamarkupfalse%
\ {\isachardoublequoteopen}x\ {\isasymin}\ {\isacharparenleft}{\kern0pt}val{\isacharparenleft}{\kern0pt}G{\isacharcomma}{\kern0pt}Union{\isacharunderscore}{\kern0pt}name{\isacharparenleft}{\kern0pt}{\isasymtau}{\isacharparenright}{\kern0pt}{\isacharparenright}{\kern0pt}{\isacharparenright}{\kern0pt}{\isachardoublequoteclose}\isanewline
\ \ \ \ \isacommand{then}\isamarkupfalse%
\ \isacommand{obtain}\isamarkupfalse%
\ {\isasymtheta}\ p\ \isakeyword{where}\isanewline
\ \ \ \ \ \ {\isachardoublequoteopen}p\ {\isasymin}\ G{\isachardoublequoteclose}\ {\isachardoublequoteopen}{\isasymlangle}{\isasymtheta}{\isacharcomma}{\kern0pt}p{\isasymrangle}\ {\isasymin}\ Union{\isacharunderscore}{\kern0pt}name{\isacharparenleft}{\kern0pt}{\isasymtau}{\isacharparenright}{\kern0pt}{\isachardoublequoteclose}\ {\isachardoublequoteopen}val{\isacharparenleft}{\kern0pt}G{\isacharcomma}{\kern0pt}{\isasymtheta}{\isacharparenright}{\kern0pt}\ {\isacharequal}{\kern0pt}\ x{\isachardoublequoteclose}\isanewline
\ \ \ \ \ \ \isacommand{using}\isamarkupfalse%
\ elem{\isacharunderscore}{\kern0pt}of{\isacharunderscore}{\kern0pt}val{\isacharunderscore}{\kern0pt}pair\ \isacommand{by}\isamarkupfalse%
\ blast\isanewline
\ \ \ \ \isacommand{with}\isamarkupfalse%
\ {\isacartoucheopen}filter{\isacharparenleft}{\kern0pt}G{\isacharparenright}{\kern0pt}{\isacartoucheclose}\ \isacommand{have}\isamarkupfalse%
\ {\isachardoublequoteopen}p{\isasymin}P{\isachardoublequoteclose}\ \isacommand{using}\isamarkupfalse%
\ filterD\ \isacommand{by}\isamarkupfalse%
\ simp\isanewline
\ \ \ \ \isacommand{from}\isamarkupfalse%
\ {\isacartoucheopen}{\isasymlangle}{\isasymtheta}{\isacharcomma}{\kern0pt}p{\isasymrangle}\ {\isasymin}\ Union{\isacharunderscore}{\kern0pt}name{\isacharparenleft}{\kern0pt}{\isasymtau}{\isacharparenright}{\kern0pt}{\isacartoucheclose}\ \isacommand{obtain}\isamarkupfalse%
\ {\isasymsigma}\ q\ r\ \isakeyword{where}\isanewline
\ \ \ \ \ \ {\isachardoublequoteopen}{\isasymsigma}\ {\isasymin}\ domain{\isacharparenleft}{\kern0pt}{\isasymtau}{\isacharparenright}{\kern0pt}{\isachardoublequoteclose}\ \ {\isachardoublequoteopen}{\isasymlangle}{\isasymsigma}{\isacharcomma}{\kern0pt}q{\isasymrangle}\ {\isasymin}\ {\isasymtau}\ {\isachardoublequoteclose}\ {\isachardoublequoteopen}{\isasymlangle}{\isasymtheta}{\isacharcomma}{\kern0pt}r{\isasymrangle}\ {\isasymin}\ {\isasymsigma}{\isachardoublequoteclose}\ {\isachardoublequoteopen}r{\isasymin}P{\isachardoublequoteclose}\ {\isachardoublequoteopen}q{\isasymin}P{\isachardoublequoteclose}\ {\isachardoublequoteopen}{\isasymlangle}p{\isacharcomma}{\kern0pt}r{\isasymrangle}\ {\isasymin}\ leq{\isachardoublequoteclose}\ {\isachardoublequoteopen}{\isasymlangle}p{\isacharcomma}{\kern0pt}q{\isasymrangle}\ {\isasymin}\ leq{\isachardoublequoteclose}\isanewline
\ \ \ \ \ \ \isacommand{unfolding}\isamarkupfalse%
\ Union{\isacharunderscore}{\kern0pt}name{\isacharunderscore}{\kern0pt}def\ Union{\isacharunderscore}{\kern0pt}name{\isacharunderscore}{\kern0pt}body{\isacharunderscore}{\kern0pt}def\ \isacommand{by}\isamarkupfalse%
\ force\isanewline
\ \ \ \ \isacommand{with}\isamarkupfalse%
\ {\isacartoucheopen}p{\isasymin}G{\isacartoucheclose}\ {\isacartoucheopen}filter{\isacharparenleft}{\kern0pt}G{\isacharparenright}{\kern0pt}{\isacartoucheclose}\ \isacommand{have}\isamarkupfalse%
\ {\isachardoublequoteopen}r\ {\isasymin}\ G{\isachardoublequoteclose}\ {\isachardoublequoteopen}q\ {\isasymin}\ G{\isachardoublequoteclose}\isanewline
\ \ \ \ \ \ \isacommand{using}\isamarkupfalse%
\ filter{\isacharunderscore}{\kern0pt}leqD\ \isacommand{by}\isamarkupfalse%
\ auto\isanewline
\ \ \ \ \isacommand{with}\isamarkupfalse%
\ {\isacartoucheopen}{\isasymlangle}{\isasymtheta}{\isacharcomma}{\kern0pt}r{\isasymrangle}\ {\isasymin}\ {\isasymsigma}{\isacartoucheclose}\ {\isacartoucheopen}{\isasymlangle}{\isasymsigma}{\isacharcomma}{\kern0pt}q{\isasymrangle}{\isasymin}{\isasymtau}{\isacartoucheclose}\ {\isacartoucheopen}q{\isasymin}P{\isacartoucheclose}\ {\isacartoucheopen}r{\isasymin}P{\isacartoucheclose}\ \isacommand{have}\isamarkupfalse%
\isanewline
\ \ \ \ \ \ {\isachardoublequoteopen}val{\isacharparenleft}{\kern0pt}G{\isacharcomma}{\kern0pt}{\isasymsigma}{\isacharparenright}{\kern0pt}\ {\isasymin}\ val{\isacharparenleft}{\kern0pt}G{\isacharcomma}{\kern0pt}{\isasymtau}{\isacharparenright}{\kern0pt}{\isachardoublequoteclose}\ {\isachardoublequoteopen}val{\isacharparenleft}{\kern0pt}G{\isacharcomma}{\kern0pt}{\isasymtheta}{\isacharparenright}{\kern0pt}\ {\isasymin}\ val{\isacharparenleft}{\kern0pt}G{\isacharcomma}{\kern0pt}{\isasymsigma}{\isacharparenright}{\kern0pt}{\isachardoublequoteclose}\isanewline
\ \ \ \ \ \ \isacommand{using}\isamarkupfalse%
\ val{\isacharunderscore}{\kern0pt}of{\isacharunderscore}{\kern0pt}elem\ \isacommand{by}\isamarkupfalse%
\ simp{\isacharplus}{\kern0pt}\isanewline
\ \ \ \ \isacommand{then}\isamarkupfalse%
\ \isacommand{have}\isamarkupfalse%
\ {\isachardoublequoteopen}val{\isacharparenleft}{\kern0pt}G{\isacharcomma}{\kern0pt}{\isasymtheta}{\isacharparenright}{\kern0pt}\ {\isasymin}\ {\isasymUnion}\ val{\isacharparenleft}{\kern0pt}G{\isacharcomma}{\kern0pt}{\isasymtau}{\isacharparenright}{\kern0pt}{\isachardoublequoteclose}\ \isacommand{by}\isamarkupfalse%
\ blast\isanewline
\ \ \ \ \isacommand{with}\isamarkupfalse%
\ {\isacartoucheopen}val{\isacharparenleft}{\kern0pt}G{\isacharcomma}{\kern0pt}{\isasymtheta}{\isacharparenright}{\kern0pt}{\isacharequal}{\kern0pt}x{\isacartoucheclose}\ {\isacartoucheopen}a{\isacharequal}{\kern0pt}val{\isacharparenleft}{\kern0pt}G{\isacharcomma}{\kern0pt}{\isasymtau}{\isacharparenright}{\kern0pt}{\isacartoucheclose}\ \isacommand{have}\isamarkupfalse%
\isanewline
\ \ \ \ \ \ {\isachardoublequoteopen}x\ {\isasymin}\ {\isasymUnion}\ a{\isachardoublequoteclose}\ \isacommand{by}\isamarkupfalse%
\ simp\isanewline
\ \ \isacommand{{\isacharbraceright}{\kern0pt}}\isamarkupfalse%
\isanewline
\ \ \isacommand{with}\isamarkupfalse%
\ {\isacartoucheopen}a{\isacharequal}{\kern0pt}val{\isacharparenleft}{\kern0pt}G{\isacharcomma}{\kern0pt}{\isasymtau}{\isacharparenright}{\kern0pt}{\isacartoucheclose}\isanewline
\ \ \isacommand{have}\isamarkupfalse%
\ {\isachardoublequoteopen}x\ {\isasymin}\ val{\isacharparenleft}{\kern0pt}G{\isacharcomma}{\kern0pt}Union{\isacharunderscore}{\kern0pt}name{\isacharparenleft}{\kern0pt}{\isasymtau}{\isacharparenright}{\kern0pt}{\isacharparenright}{\kern0pt}\ {\isasymLongrightarrow}\ x\ {\isasymin}\ {\isasymUnion}\ a{\isachardoublequoteclose}\ \isakeyword{for}\ x\ \isacommand{by}\isamarkupfalse%
\ blast\isanewline
\ \ \isacommand{then}\isamarkupfalse%
\isanewline
\ \ \isacommand{show}\isamarkupfalse%
\ {\isacharquery}{\kern0pt}thesis\ \isacommand{using}\isamarkupfalse%
\ {\isadigit{1}}\ \isacommand{by}\isamarkupfalse%
\ blast\isanewline
\isacommand{qed}\isamarkupfalse%
%
\endisatagproof
{\isafoldproof}%
%
\isadelimproof
\isanewline
%
\endisadelimproof
\isanewline
\isacommand{lemma}\isamarkupfalse%
\ union{\isacharunderscore}{\kern0pt}in{\isacharunderscore}{\kern0pt}MG\ {\isacharcolon}{\kern0pt}\ \isakeyword{assumes}\ {\isachardoublequoteopen}filter{\isacharparenleft}{\kern0pt}G{\isacharparenright}{\kern0pt}{\isachardoublequoteclose}\isanewline
\ \ \isakeyword{shows}\ {\isachardoublequoteopen}Union{\isacharunderscore}{\kern0pt}ax{\isacharparenleft}{\kern0pt}{\isacharhash}{\kern0pt}{\isacharhash}{\kern0pt}M{\isacharbrackleft}{\kern0pt}G{\isacharbrackright}{\kern0pt}{\isacharparenright}{\kern0pt}{\isachardoublequoteclose}\isanewline
%
\isadelimproof
%
\endisadelimproof
%
\isatagproof
\isacommand{proof}\isamarkupfalse%
\ {\isacharminus}{\kern0pt}\isanewline
\ \ \isacommand{{\isacharbraceleft}{\kern0pt}}\isamarkupfalse%
\ \isacommand{fix}\isamarkupfalse%
\ a\isanewline
\ \ \ \ \isacommand{assume}\isamarkupfalse%
\ {\isachardoublequoteopen}a\ {\isasymin}\ M{\isacharbrackleft}{\kern0pt}G{\isacharbrackright}{\kern0pt}{\isachardoublequoteclose}\isanewline
\ \ \ \ \isacommand{then}\isamarkupfalse%
\isanewline
\ \ \ \ \isacommand{interpret}\isamarkupfalse%
\ mgtrans\ {\isacharcolon}{\kern0pt}\ M{\isacharunderscore}{\kern0pt}trans\ {\isachardoublequoteopen}{\isacharhash}{\kern0pt}{\isacharhash}{\kern0pt}M{\isacharbrackleft}{\kern0pt}G{\isacharbrackright}{\kern0pt}{\isachardoublequoteclose}\isanewline
\ \ \ \ \ \ \isacommand{using}\isamarkupfalse%
\ transitivity{\isacharunderscore}{\kern0pt}MG\ \isacommand{by}\isamarkupfalse%
\ {\isacharparenleft}{\kern0pt}unfold{\isacharunderscore}{\kern0pt}locales{\isacharsemicolon}{\kern0pt}\ auto{\isacharparenright}{\kern0pt}\isanewline
\ \ \ \ \isacommand{from}\isamarkupfalse%
\ {\isacartoucheopen}a{\isasymin}{\isacharunderscore}{\kern0pt}{\isacartoucheclose}\ \isacommand{obtain}\isamarkupfalse%
\ {\isasymtau}\ \isakeyword{where}\ {\isachardoublequoteopen}{\isasymtau}\ {\isasymin}\ M{\isachardoublequoteclose}\ {\isachardoublequoteopen}a{\isacharequal}{\kern0pt}val{\isacharparenleft}{\kern0pt}G{\isacharcomma}{\kern0pt}{\isasymtau}{\isacharparenright}{\kern0pt}{\isachardoublequoteclose}\ \isacommand{using}\isamarkupfalse%
\ GenExtD\ \isacommand{by}\isamarkupfalse%
\ blast\isanewline
\ \ \ \ \isacommand{then}\isamarkupfalse%
\isanewline
\ \ \ \ \isacommand{have}\isamarkupfalse%
\ {\isachardoublequoteopen}Union{\isacharunderscore}{\kern0pt}name{\isacharparenleft}{\kern0pt}{\isasymtau}{\isacharparenright}{\kern0pt}\ {\isasymin}\ M{\isachardoublequoteclose}\ {\isacharparenleft}{\kern0pt}\isakeyword{is}\ {\isachardoublequoteopen}{\isacharquery}{\kern0pt}{\isasympi}\ {\isasymin}\ {\isacharunderscore}{\kern0pt}{\isachardoublequoteclose}{\isacharparenright}{\kern0pt}\ \isacommand{using}\isamarkupfalse%
\ Union{\isacharunderscore}{\kern0pt}name{\isacharunderscore}{\kern0pt}M\ \isacommand{unfolding}\isamarkupfalse%
\ Union{\isacharunderscore}{\kern0pt}name{\isacharunderscore}{\kern0pt}def\ \isacommand{by}\isamarkupfalse%
\ simp\isanewline
\ \ \ \ \isacommand{then}\isamarkupfalse%
\isanewline
\ \ \ \ \isacommand{have}\isamarkupfalse%
\ {\isachardoublequoteopen}val{\isacharparenleft}{\kern0pt}G{\isacharcomma}{\kern0pt}{\isacharquery}{\kern0pt}{\isasympi}{\isacharparenright}{\kern0pt}\ {\isasymin}\ M{\isacharbrackleft}{\kern0pt}G{\isacharbrackright}{\kern0pt}{\isachardoublequoteclose}\ {\isacharparenleft}{\kern0pt}\isakeyword{is}\ {\isachardoublequoteopen}{\isacharquery}{\kern0pt}U\ {\isasymin}\ {\isacharunderscore}{\kern0pt}{\isachardoublequoteclose}{\isacharparenright}{\kern0pt}\ \isacommand{using}\isamarkupfalse%
\ GenExtI\ \isacommand{by}\isamarkupfalse%
\ simp\isanewline
\ \ \ \ \isacommand{with}\isamarkupfalse%
\ {\isacartoucheopen}a{\isasymin}{\isacharunderscore}{\kern0pt}{\isacartoucheclose}\isanewline
\ \ \ \ \isacommand{have}\isamarkupfalse%
\ {\isachardoublequoteopen}{\isacharparenleft}{\kern0pt}{\isacharhash}{\kern0pt}{\isacharhash}{\kern0pt}M{\isacharbrackleft}{\kern0pt}G{\isacharbrackright}{\kern0pt}{\isacharparenright}{\kern0pt}{\isacharparenleft}{\kern0pt}a{\isacharparenright}{\kern0pt}{\isachardoublequoteclose}\ {\isachardoublequoteopen}{\isacharparenleft}{\kern0pt}{\isacharhash}{\kern0pt}{\isacharhash}{\kern0pt}M{\isacharbrackleft}{\kern0pt}G{\isacharbrackright}{\kern0pt}{\isacharparenright}{\kern0pt}{\isacharparenleft}{\kern0pt}{\isacharquery}{\kern0pt}U{\isacharparenright}{\kern0pt}{\isachardoublequoteclose}\ \isacommand{by}\isamarkupfalse%
\ auto\isanewline
\ \ \ \ \isacommand{with}\isamarkupfalse%
\ {\isacartoucheopen}{\isasymtau}\ {\isasymin}\ M{\isacartoucheclose}\ {\isacartoucheopen}filter{\isacharparenleft}{\kern0pt}G{\isacharparenright}{\kern0pt}{\isacartoucheclose}\ {\isacartoucheopen}{\isacharquery}{\kern0pt}U\ {\isasymin}\ M{\isacharbrackleft}{\kern0pt}G{\isacharbrackright}{\kern0pt}{\isacartoucheclose}\ {\isacartoucheopen}a{\isacharequal}{\kern0pt}val{\isacharparenleft}{\kern0pt}G{\isacharcomma}{\kern0pt}{\isasymtau}{\isacharparenright}{\kern0pt}{\isacartoucheclose}\isanewline
\ \ \ \ \isacommand{have}\isamarkupfalse%
\ {\isachardoublequoteopen}big{\isacharunderscore}{\kern0pt}union{\isacharparenleft}{\kern0pt}{\isacharhash}{\kern0pt}{\isacharhash}{\kern0pt}M{\isacharbrackleft}{\kern0pt}G{\isacharbrackright}{\kern0pt}{\isacharcomma}{\kern0pt}a{\isacharcomma}{\kern0pt}{\isacharquery}{\kern0pt}U{\isacharparenright}{\kern0pt}{\isachardoublequoteclose}\isanewline
\ \ \ \ \ \ \isacommand{using}\isamarkupfalse%
\ Union{\isacharunderscore}{\kern0pt}MG{\isacharunderscore}{\kern0pt}Eq\ Union{\isacharunderscore}{\kern0pt}abs\ \ \isacommand{by}\isamarkupfalse%
\ simp\isanewline
\ \ \ \ \isacommand{with}\isamarkupfalse%
\ {\isacartoucheopen}{\isacharquery}{\kern0pt}U\ {\isasymin}\ M{\isacharbrackleft}{\kern0pt}G{\isacharbrackright}{\kern0pt}{\isacartoucheclose}\isanewline
\ \ \ \ \isacommand{have}\isamarkupfalse%
\ {\isachardoublequoteopen}{\isasymexists}z{\isacharbrackleft}{\kern0pt}{\isacharhash}{\kern0pt}{\isacharhash}{\kern0pt}M{\isacharbrackleft}{\kern0pt}G{\isacharbrackright}{\kern0pt}{\isacharbrackright}{\kern0pt}{\isachardot}{\kern0pt}\ big{\isacharunderscore}{\kern0pt}union{\isacharparenleft}{\kern0pt}{\isacharhash}{\kern0pt}{\isacharhash}{\kern0pt}M{\isacharbrackleft}{\kern0pt}G{\isacharbrackright}{\kern0pt}{\isacharcomma}{\kern0pt}a{\isacharcomma}{\kern0pt}z{\isacharparenright}{\kern0pt}{\isachardoublequoteclose}\ \isacommand{by}\isamarkupfalse%
\ force\isanewline
\ \ \isacommand{{\isacharbraceright}{\kern0pt}}\isamarkupfalse%
\isanewline
\ \ \isacommand{then}\isamarkupfalse%
\isanewline
\ \ \isacommand{have}\isamarkupfalse%
\ {\isachardoublequoteopen}Union{\isacharunderscore}{\kern0pt}ax{\isacharparenleft}{\kern0pt}{\isacharhash}{\kern0pt}{\isacharhash}{\kern0pt}M{\isacharbrackleft}{\kern0pt}G{\isacharbrackright}{\kern0pt}{\isacharparenright}{\kern0pt}{\isachardoublequoteclose}\ \isacommand{unfolding}\isamarkupfalse%
\ Union{\isacharunderscore}{\kern0pt}ax{\isacharunderscore}{\kern0pt}def\ \isacommand{by}\isamarkupfalse%
\ force\isanewline
\ \ \isacommand{then}\isamarkupfalse%
\isanewline
\ \ \isacommand{show}\isamarkupfalse%
\ {\isacharquery}{\kern0pt}thesis\ \isacommand{by}\isamarkupfalse%
\ simp\isanewline
\isacommand{qed}\isamarkupfalse%
%
\endisatagproof
{\isafoldproof}%
%
\isadelimproof
\isanewline
%
\endisadelimproof
\isanewline
\isacommand{theorem}\isamarkupfalse%
\ Union{\isacharunderscore}{\kern0pt}MG\ {\isacharcolon}{\kern0pt}\ {\isachardoublequoteopen}M{\isacharunderscore}{\kern0pt}generic{\isacharparenleft}{\kern0pt}G{\isacharparenright}{\kern0pt}\ {\isasymLongrightarrow}\ Union{\isacharunderscore}{\kern0pt}ax{\isacharparenleft}{\kern0pt}{\isacharhash}{\kern0pt}{\isacharhash}{\kern0pt}M{\isacharbrackleft}{\kern0pt}G{\isacharbrackright}{\kern0pt}{\isacharparenright}{\kern0pt}{\isachardoublequoteclose}\isanewline
%
\isadelimproof
\ \ %
\endisadelimproof
%
\isatagproof
\isacommand{by}\isamarkupfalse%
\ {\isacharparenleft}{\kern0pt}simp\ add{\isacharcolon}{\kern0pt}M{\isacharunderscore}{\kern0pt}generic{\isacharunderscore}{\kern0pt}def\ union{\isacharunderscore}{\kern0pt}in{\isacharunderscore}{\kern0pt}MG{\isacharparenright}{\kern0pt}%
\endisatagproof
{\isafoldproof}%
%
\isadelimproof
\isanewline
%
\endisadelimproof
\isanewline
\isacommand{end}\isamarkupfalse%
\ \isanewline
%
\isadelimtheory
%
\endisadelimtheory
%
\isatagtheory
\isacommand{end}\isamarkupfalse%
%
\endisatagtheory
{\isafoldtheory}%
%
\isadelimtheory
%
\endisadelimtheory
%
\end{isabellebody}%
\endinput
%:%file=~/source/repos/ZF-notAC/code/Forcing/Union_Axiom.thy%:%
%:%11=1%:%
%:%27=2%:%
%:%28=2%:%
%:%29=3%:%
%:%30=4%:%
%:%35=4%:%
%:%38=5%:%
%:%39=6%:%
%:%40=6%:%
%:%41=7%:%
%:%42=8%:%
%:%43=9%:%
%:%44=10%:%
%:%45=10%:%
%:%46=11%:%
%:%48=13%:%
%:%49=14%:%
%:%50=15%:%
%:%51=15%:%
%:%52=16%:%
%:%61=25%:%
%:%62=26%:%
%:%63=27%:%
%:%64=27%:%
%:%65=28%:%
%:%68=29%:%
%:%72=29%:%
%:%73=29%:%
%:%74=29%:%
%:%79=29%:%
%:%82=30%:%
%:%83=31%:%
%:%84=32%:%
%:%85=32%:%
%:%86=33%:%
%:%89=34%:%
%:%93=34%:%
%:%94=34%:%
%:%95=35%:%
%:%96=35%:%
%:%101=35%:%
%:%104=36%:%
%:%105=37%:%
%:%106=37%:%
%:%107=38%:%
%:%109=40%:%
%:%112=41%:%
%:%116=41%:%
%:%117=41%:%
%:%118=42%:%
%:%119=42%:%
%:%124=42%:%
%:%127=43%:%
%:%128=44%:%
%:%129=45%:%
%:%130=45%:%
%:%131=46%:%
%:%132=47%:%
%:%135=48%:%
%:%139=48%:%
%:%140=48%:%
%:%141=49%:%
%:%142=49%:%
%:%147=49%:%
%:%150=50%:%
%:%151=51%:%
%:%152=52%:%
%:%153=52%:%
%:%154=53%:%
%:%155=54%:%
%:%156=55%:%
%:%157=56%:%
%:%158=56%:%
%:%159=57%:%
%:%162=58%:%
%:%166=58%:%
%:%167=58%:%
%:%168=59%:%
%:%169=59%:%
%:%170=60%:%
%:%171=60%:%
%:%172=61%:%
%:%173=61%:%
%:%174=62%:%
%:%175=62%:%
%:%176=63%:%
%:%177=63%:%
%:%178=63%:%
%:%179=63%:%
%:%180=64%:%
%:%181=64%:%
%:%182=65%:%
%:%183=65%:%
%:%184=65%:%
%:%185=65%:%
%:%186=66%:%
%:%187=66%:%
%:%188=66%:%
%:%189=66%:%
%:%190=67%:%
%:%191=67%:%
%:%192=68%:%
%:%193=68%:%
%:%194=68%:%
%:%195=69%:%
%:%196=69%:%
%:%197=70%:%
%:%198=70%:%
%:%199=71%:%
%:%200=71%:%
%:%201=71%:%
%:%202=72%:%
%:%203=72%:%
%:%204=73%:%
%:%205=73%:%
%:%206=74%:%
%:%207=74%:%
%:%208=74%:%
%:%209=75%:%
%:%210=75%:%
%:%211=76%:%
%:%212=76%:%
%:%213=77%:%
%:%214=77%:%
%:%215=78%:%
%:%216=78%:%
%:%217=78%:%
%:%218=78%:%
%:%219=79%:%
%:%220=79%:%
%:%221=80%:%
%:%222=80%:%
%:%223=81%:%
%:%224=81%:%
%:%225=81%:%
%:%226=81%:%
%:%227=82%:%
%:%228=82%:%
%:%229=83%:%
%:%230=83%:%
%:%231=84%:%
%:%232=84%:%
%:%233=84%:%
%:%234=85%:%
%:%235=85%:%
%:%236=86%:%
%:%237=86%:%
%:%238=86%:%
%:%239=86%:%
%:%240=87%:%
%:%241=87%:%
%:%242=88%:%
%:%243=88%:%
%:%244=88%:%
%:%245=88%:%
%:%246=88%:%
%:%247=89%:%
%:%253=89%:%
%:%256=90%:%
%:%257=91%:%
%:%258=92%:%
%:%259=93%:%
%:%260=93%:%
%:%261=94%:%
%:%262=95%:%
%:%269=96%:%
%:%270=96%:%
%:%271=97%:%
%:%272=97%:%
%:%273=98%:%
%:%274=98%:%
%:%275=99%:%
%:%276=99%:%
%:%277=100%:%
%:%278=100%:%
%:%279=100%:%
%:%280=100%:%
%:%281=101%:%
%:%282=101%:%
%:%283=101%:%
%:%284=102%:%
%:%285=103%:%
%:%286=103%:%
%:%287=103%:%
%:%288=104%:%
%:%289=104%:%
%:%290=104%:%
%:%291=105%:%
%:%292=106%:%
%:%293=106%:%
%:%294=106%:%
%:%295=107%:%
%:%296=107%:%
%:%297=107%:%
%:%298=107%:%
%:%299=108%:%
%:%300=108%:%
%:%301=108%:%
%:%302=109%:%
%:%303=110%:%
%:%304=110%:%
%:%305=110%:%
%:%306=111%:%
%:%307=111%:%
%:%308=111%:%
%:%309=112%:%
%:%310=112%:%
%:%311=112%:%
%:%312=112%:%
%:%313=113%:%
%:%314=113%:%
%:%315=113%:%
%:%316=114%:%
%:%317=114%:%
%:%318=115%:%
%:%319=115%:%
%:%320=116%:%
%:%321=116%:%
%:%322=116%:%
%:%323=117%:%
%:%324=117%:%
%:%325=117%:%
%:%326=118%:%
%:%327=118%:%
%:%328=118%:%
%:%329=118%:%
%:%330=119%:%
%:%331=119%:%
%:%332=120%:%
%:%333=120%:%
%:%334=120%:%
%:%335=120%:%
%:%336=121%:%
%:%337=121%:%
%:%338=122%:%
%:%339=122%:%
%:%340=123%:%
%:%341=123%:%
%:%342=124%:%
%:%343=124%:%
%:%344=124%:%
%:%345=125%:%
%:%346=126%:%
%:%347=126%:%
%:%348=126%:%
%:%349=127%:%
%:%350=127%:%
%:%351=127%:%
%:%352=127%:%
%:%353=127%:%
%:%354=128%:%
%:%355=128%:%
%:%356=128%:%
%:%357=129%:%
%:%358=130%:%
%:%359=130%:%
%:%360=130%:%
%:%361=131%:%
%:%362=131%:%
%:%363=131%:%
%:%364=132%:%
%:%365=132%:%
%:%366=132%:%
%:%367=133%:%
%:%368=133%:%
%:%369=133%:%
%:%370=134%:%
%:%371=135%:%
%:%372=135%:%
%:%373=135%:%
%:%374=136%:%
%:%375=136%:%
%:%376=136%:%
%:%377=136%:%
%:%378=137%:%
%:%379=137%:%
%:%380=137%:%
%:%381=138%:%
%:%382=138%:%
%:%383=139%:%
%:%384=139%:%
%:%385=140%:%
%:%386=140%:%
%:%387=141%:%
%:%388=141%:%
%:%389=141%:%
%:%390=142%:%
%:%391=142%:%
%:%392=143%:%
%:%393=143%:%
%:%394=143%:%
%:%395=143%:%
%:%396=144%:%
%:%402=144%:%
%:%405=145%:%
%:%406=146%:%
%:%407=146%:%
%:%408=147%:%
%:%415=148%:%
%:%416=148%:%
%:%417=149%:%
%:%418=149%:%
%:%419=149%:%
%:%420=150%:%
%:%421=150%:%
%:%422=151%:%
%:%423=151%:%
%:%424=152%:%
%:%425=152%:%
%:%426=153%:%
%:%427=153%:%
%:%428=153%:%
%:%429=154%:%
%:%430=154%:%
%:%431=154%:%
%:%432=154%:%
%:%433=154%:%
%:%434=155%:%
%:%435=155%:%
%:%436=156%:%
%:%437=156%:%
%:%438=156%:%
%:%439=156%:%
%:%440=156%:%
%:%441=157%:%
%:%442=157%:%
%:%443=158%:%
%:%444=158%:%
%:%445=158%:%
%:%446=158%:%
%:%447=159%:%
%:%448=159%:%
%:%449=160%:%
%:%450=160%:%
%:%451=160%:%
%:%452=161%:%
%:%453=161%:%
%:%454=162%:%
%:%455=162%:%
%:%456=163%:%
%:%457=163%:%
%:%458=163%:%
%:%459=164%:%
%:%460=164%:%
%:%461=165%:%
%:%462=165%:%
%:%463=165%:%
%:%464=166%:%
%:%465=166%:%
%:%466=167%:%
%:%467=167%:%
%:%468=168%:%
%:%469=168%:%
%:%470=168%:%
%:%471=168%:%
%:%472=169%:%
%:%473=169%:%
%:%474=170%:%
%:%475=170%:%
%:%476=170%:%
%:%477=171%:%
%:%483=171%:%
%:%486=172%:%
%:%487=173%:%
%:%488=173%:%
%:%491=174%:%
%:%495=174%:%
%:%496=174%:%
%:%501=174%:%
%:%504=175%:%
%:%505=176%:%
%:%506=176%:%
%:%513=177%:%

%
\begin{isabellebody}%
\setisabellecontext{Powerset{\isacharunderscore}{\kern0pt}Axiom}%
%
\isadelimdocument
%
\endisadelimdocument
%
\isatagdocument
%
\isamarkupsection{The Powerset Axiom in $M[G]$%
}
\isamarkuptrue%
%
\endisatagdocument
{\isafolddocument}%
%
\isadelimdocument
%
\endisadelimdocument
%
\isadelimtheory
%
\endisadelimtheory
%
\isatagtheory
\isacommand{theory}\isamarkupfalse%
\ Powerset{\isacharunderscore}{\kern0pt}Axiom\isanewline
\ \ \isakeyword{imports}\ Renaming{\isacharunderscore}{\kern0pt}Auto\ Separation{\isacharunderscore}{\kern0pt}Axiom\ Pairing{\isacharunderscore}{\kern0pt}Axiom\ Union{\isacharunderscore}{\kern0pt}Axiom\isanewline
\isakeyword{begin}%
\endisatagtheory
{\isafoldtheory}%
%
\isadelimtheory
\isanewline
%
\endisadelimtheory
%
\isadelimML
\isanewline
%
\endisadelimML
%
\isatagML
\isacommand{simple{\isacharunderscore}{\kern0pt}rename}\isamarkupfalse%
\ {\isachardoublequoteopen}perm{\isacharunderscore}{\kern0pt}pow{\isachardoublequoteclose}\ \isakeyword{src}\ {\isachardoublequoteopen}{\isacharbrackleft}{\kern0pt}ss{\isacharcomma}{\kern0pt}p{\isacharcomma}{\kern0pt}l{\isacharcomma}{\kern0pt}o{\isacharcomma}{\kern0pt}fs{\isacharcomma}{\kern0pt}{\isasymchi}{\isacharbrackright}{\kern0pt}{\isachardoublequoteclose}\ \isakeyword{tgt}\ {\isachardoublequoteopen}{\isacharbrackleft}{\kern0pt}fs{\isacharcomma}{\kern0pt}ss{\isacharcomma}{\kern0pt}sp{\isacharcomma}{\kern0pt}p{\isacharcomma}{\kern0pt}l{\isacharcomma}{\kern0pt}o{\isacharcomma}{\kern0pt}{\isasymchi}{\isacharbrackright}{\kern0pt}{\isachardoublequoteclose}%
\endisatagML
{\isafoldML}%
%
\isadelimML
\isanewline
%
\endisadelimML
\isanewline
\isacommand{lemma}\isamarkupfalse%
\ Collect{\isacharunderscore}{\kern0pt}inter{\isacharunderscore}{\kern0pt}Transset{\isacharcolon}{\kern0pt}\isanewline
\ \ \isakeyword{assumes}\isanewline
\ \ \ \ {\isachardoublequoteopen}Transset{\isacharparenleft}{\kern0pt}M{\isacharparenright}{\kern0pt}{\isachardoublequoteclose}\ {\isachardoublequoteopen}b\ {\isasymin}\ M{\isachardoublequoteclose}\isanewline
\ \ \isakeyword{shows}\isanewline
\ \ \ \ {\isachardoublequoteopen}{\isacharbraceleft}{\kern0pt}x{\isasymin}b\ {\isachardot}{\kern0pt}\ P{\isacharparenleft}{\kern0pt}x{\isacharparenright}{\kern0pt}{\isacharbraceright}{\kern0pt}\ {\isacharequal}{\kern0pt}\ {\isacharbraceleft}{\kern0pt}x{\isasymin}b\ {\isachardot}{\kern0pt}\ P{\isacharparenleft}{\kern0pt}x{\isacharparenright}{\kern0pt}{\isacharbraceright}{\kern0pt}\ {\isasyminter}\ M{\isachardoublequoteclose}\isanewline
%
\isadelimproof
\ \ %
\endisadelimproof
%
\isatagproof
\isacommand{using}\isamarkupfalse%
\ assms\ \isacommand{unfolding}\isamarkupfalse%
\ Transset{\isacharunderscore}{\kern0pt}def\isanewline
\ \ \isacommand{by}\isamarkupfalse%
\ {\isacharparenleft}{\kern0pt}auto{\isacharparenright}{\kern0pt}%
\endisatagproof
{\isafoldproof}%
%
\isadelimproof
\isanewline
%
\endisadelimproof
\isanewline
\isacommand{context}\isamarkupfalse%
\ G{\isacharunderscore}{\kern0pt}generic\ \ \isakeyword{begin}\isanewline
\isanewline
\isacommand{lemma}\isamarkupfalse%
\ name{\isacharunderscore}{\kern0pt}components{\isacharunderscore}{\kern0pt}in{\isacharunderscore}{\kern0pt}M{\isacharcolon}{\kern0pt}\isanewline
\ \ \isakeyword{assumes}\ {\isachardoublequoteopen}{\isacharless}{\kern0pt}{\isasymsigma}{\isacharcomma}{\kern0pt}p{\isachargreater}{\kern0pt}{\isasymin}{\isasymtheta}{\isachardoublequoteclose}\ {\isachardoublequoteopen}{\isasymtheta}\ {\isasymin}\ M{\isachardoublequoteclose}\isanewline
\ \ \isakeyword{shows}\ \ \ {\isachardoublequoteopen}{\isasymsigma}{\isasymin}M{\isachardoublequoteclose}\ {\isachardoublequoteopen}p{\isasymin}M{\isachardoublequoteclose}\isanewline
%
\isadelimproof
%
\endisadelimproof
%
\isatagproof
\isacommand{proof}\isamarkupfalse%
\ {\isacharminus}{\kern0pt}\isanewline
\ \ \isacommand{from}\isamarkupfalse%
\ assms\ \isacommand{obtain}\isamarkupfalse%
\ a\ \isakeyword{where}\isanewline
\ \ \ \ {\isachardoublequoteopen}{\isasymsigma}\ {\isasymin}\ a{\isachardoublequoteclose}\ {\isachardoublequoteopen}p\ {\isasymin}\ a{\isachardoublequoteclose}\ {\isachardoublequoteopen}a{\isasymin}{\isacharless}{\kern0pt}{\isasymsigma}{\isacharcomma}{\kern0pt}p{\isachargreater}{\kern0pt}{\isachardoublequoteclose}\isanewline
\ \ \ \ \isacommand{unfolding}\isamarkupfalse%
\ Pair{\isacharunderscore}{\kern0pt}def\ \isacommand{by}\isamarkupfalse%
\ auto\isanewline
\ \ \isacommand{moreover}\isamarkupfalse%
\ \isacommand{from}\isamarkupfalse%
\ assms\isanewline
\ \ \isacommand{have}\isamarkupfalse%
\ {\isachardoublequoteopen}{\isacharless}{\kern0pt}{\isasymsigma}{\isacharcomma}{\kern0pt}p{\isachargreater}{\kern0pt}{\isasymin}M{\isachardoublequoteclose}\isanewline
\ \ \ \ \isacommand{using}\isamarkupfalse%
\ transitivity\ \isacommand{by}\isamarkupfalse%
\ simp\isanewline
\ \ \isacommand{moreover}\isamarkupfalse%
\ \isacommand{from}\isamarkupfalse%
\ calculation\isanewline
\ \ \isacommand{have}\isamarkupfalse%
\ {\isachardoublequoteopen}a{\isasymin}M{\isachardoublequoteclose}\isanewline
\ \ \ \ \isacommand{using}\isamarkupfalse%
\ transitivity\ \isacommand{by}\isamarkupfalse%
\ simp\isanewline
\ \ \isacommand{ultimately}\isamarkupfalse%
\isanewline
\ \ \isacommand{show}\isamarkupfalse%
\ {\isachardoublequoteopen}{\isasymsigma}{\isasymin}M{\isachardoublequoteclose}\ {\isachardoublequoteopen}p{\isasymin}M{\isachardoublequoteclose}\isanewline
\ \ \ \ \isacommand{using}\isamarkupfalse%
\ transitivity\ \isacommand{by}\isamarkupfalse%
\ simp{\isacharunderscore}{\kern0pt}all\isanewline
\isacommand{qed}\isamarkupfalse%
%
\endisatagproof
{\isafoldproof}%
%
\isadelimproof
\isanewline
%
\endisadelimproof
\isanewline
\isacommand{lemma}\isamarkupfalse%
\ sats{\isacharunderscore}{\kern0pt}fst{\isacharunderscore}{\kern0pt}snd{\isacharunderscore}{\kern0pt}in{\isacharunderscore}{\kern0pt}M{\isacharcolon}{\kern0pt}\isanewline
\ \ \isakeyword{assumes}\isanewline
\ \ \ \ {\isachardoublequoteopen}A{\isasymin}M{\isachardoublequoteclose}\ {\isachardoublequoteopen}B{\isasymin}M{\isachardoublequoteclose}\ {\isachardoublequoteopen}{\isasymphi}\ {\isasymin}\ formula{\isachardoublequoteclose}\ {\isachardoublequoteopen}p{\isasymin}M{\isachardoublequoteclose}\ {\isachardoublequoteopen}l{\isasymin}M{\isachardoublequoteclose}\ {\isachardoublequoteopen}o{\isasymin}M{\isachardoublequoteclose}\ {\isachardoublequoteopen}{\isasymchi}{\isasymin}M{\isachardoublequoteclose}\isanewline
\ \ \ \ {\isachardoublequoteopen}arity{\isacharparenleft}{\kern0pt}{\isasymphi}{\isacharparenright}{\kern0pt}\ {\isasymle}\ {\isadigit{6}}{\isachardoublequoteclose}\isanewline
\ \ \isakeyword{shows}\isanewline
\ \ \ \ {\isachardoublequoteopen}{\isacharbraceleft}{\kern0pt}sq\ {\isasymin}A{\isasymtimes}B\ {\isachardot}{\kern0pt}\ sats{\isacharparenleft}{\kern0pt}M{\isacharcomma}{\kern0pt}{\isasymphi}{\isacharcomma}{\kern0pt}{\isacharbrackleft}{\kern0pt}snd{\isacharparenleft}{\kern0pt}sq{\isacharparenright}{\kern0pt}{\isacharcomma}{\kern0pt}p{\isacharcomma}{\kern0pt}l{\isacharcomma}{\kern0pt}o{\isacharcomma}{\kern0pt}fst{\isacharparenleft}{\kern0pt}sq{\isacharparenright}{\kern0pt}{\isacharcomma}{\kern0pt}{\isasymchi}{\isacharbrackright}{\kern0pt}{\isacharparenright}{\kern0pt}{\isacharbraceright}{\kern0pt}\ {\isasymin}\ M{\isachardoublequoteclose}\isanewline
\ \ \ \ {\isacharparenleft}{\kern0pt}\isakeyword{is}\ {\isachardoublequoteopen}{\isacharquery}{\kern0pt}{\isasymtheta}\ {\isasymin}\ M{\isachardoublequoteclose}{\isacharparenright}{\kern0pt}\isanewline
%
\isadelimproof
%
\endisadelimproof
%
\isatagproof
\isacommand{proof}\isamarkupfalse%
\ {\isacharminus}{\kern0pt}\isanewline
\ \ \isacommand{have}\isamarkupfalse%
\ {\isachardoublequoteopen}{\isadigit{6}}{\isasymin}nat{\isachardoublequoteclose}\ {\isachardoublequoteopen}{\isadigit{7}}{\isasymin}nat{\isachardoublequoteclose}\ \isacommand{by}\isamarkupfalse%
\ simp{\isacharunderscore}{\kern0pt}all\isanewline
\ \ \isacommand{let}\isamarkupfalse%
\ {\isacharquery}{\kern0pt}{\isasymphi}{\isacharprime}{\kern0pt}\ {\isacharequal}{\kern0pt}\ {\isachardoublequoteopen}ren{\isacharparenleft}{\kern0pt}{\isasymphi}{\isacharparenright}{\kern0pt}{\isacharbackquote}{\kern0pt}{\isadigit{6}}{\isacharbackquote}{\kern0pt}{\isadigit{7}}{\isacharbackquote}{\kern0pt}perm{\isacharunderscore}{\kern0pt}pow{\isacharunderscore}{\kern0pt}fn{\isachardoublequoteclose}\isanewline
\ \ \isacommand{from}\isamarkupfalse%
\ {\isacartoucheopen}A{\isasymin}M{\isacartoucheclose}\ {\isacartoucheopen}B{\isasymin}M{\isacartoucheclose}\ \isacommand{have}\isamarkupfalse%
\isanewline
\ \ \ \ {\isachardoublequoteopen}A{\isasymtimes}B\ {\isasymin}\ M{\isachardoublequoteclose}\isanewline
\ \ \ \ \isacommand{using}\isamarkupfalse%
\ cartprod{\isacharunderscore}{\kern0pt}closed\ \isacommand{by}\isamarkupfalse%
\ simp\isanewline
\ \ \isacommand{from}\isamarkupfalse%
\ {\isacartoucheopen}arity{\isacharparenleft}{\kern0pt}{\isasymphi}{\isacharparenright}{\kern0pt}\ {\isasymle}\ {\isadigit{6}}{\isacartoucheclose}\ {\isacartoucheopen}{\isasymphi}{\isasymin}\ formula{\isacartoucheclose}\ {\isacartoucheopen}{\isadigit{6}}{\isasymin}{\isacharunderscore}{\kern0pt}{\isacartoucheclose}\ {\isacartoucheopen}{\isadigit{7}}{\isasymin}{\isacharunderscore}{\kern0pt}{\isacartoucheclose}\isanewline
\ \ \isacommand{have}\isamarkupfalse%
\ {\isachardoublequoteopen}{\isacharquery}{\kern0pt}{\isasymphi}{\isacharprime}{\kern0pt}\ {\isasymin}\ formula{\isachardoublequoteclose}\ {\isachardoublequoteopen}arity{\isacharparenleft}{\kern0pt}{\isacharquery}{\kern0pt}{\isasymphi}{\isacharprime}{\kern0pt}{\isacharparenright}{\kern0pt}{\isasymle}{\isadigit{7}}{\isachardoublequoteclose}\isanewline
\ \ \ \ \isacommand{unfolding}\isamarkupfalse%
\ perm{\isacharunderscore}{\kern0pt}pow{\isacharunderscore}{\kern0pt}fn{\isacharunderscore}{\kern0pt}def\isanewline
\ \ \ \ \isacommand{using}\isamarkupfalse%
\ \ perm{\isacharunderscore}{\kern0pt}pow{\isacharunderscore}{\kern0pt}thm\ \ arity{\isacharunderscore}{\kern0pt}ren\ ren{\isacharunderscore}{\kern0pt}tc\ Nil{\isacharunderscore}{\kern0pt}type\isanewline
\ \ \ \ \isacommand{by}\isamarkupfalse%
\ auto\isanewline
\ \ \isacommand{with}\isamarkupfalse%
\ {\isacartoucheopen}{\isacharquery}{\kern0pt}{\isasymphi}{\isacharprime}{\kern0pt}\ {\isasymin}\ formula{\isacartoucheclose}\isanewline
\ \ \isacommand{have}\isamarkupfalse%
\ {\isadigit{1}}{\isacharcolon}{\kern0pt}\ {\isachardoublequoteopen}arity{\isacharparenleft}{\kern0pt}Exists{\isacharparenleft}{\kern0pt}Exists{\isacharparenleft}{\kern0pt}And{\isacharparenleft}{\kern0pt}pair{\isacharunderscore}{\kern0pt}fm{\isacharparenleft}{\kern0pt}{\isadigit{0}}{\isacharcomma}{\kern0pt}{\isadigit{1}}{\isacharcomma}{\kern0pt}{\isadigit{2}}{\isacharparenright}{\kern0pt}{\isacharcomma}{\kern0pt}{\isacharquery}{\kern0pt}{\isasymphi}{\isacharprime}{\kern0pt}{\isacharparenright}{\kern0pt}{\isacharparenright}{\kern0pt}{\isacharparenright}{\kern0pt}{\isacharparenright}{\kern0pt}{\isasymle}{\isadigit{5}}{\isachardoublequoteclose}\ \ \ \ \ {\isacharparenleft}{\kern0pt}\isakeyword{is}\ {\isachardoublequoteopen}arity{\isacharparenleft}{\kern0pt}{\isacharquery}{\kern0pt}{\isasympsi}{\isacharparenright}{\kern0pt}{\isasymle}{\isadigit{5}}{\isachardoublequoteclose}{\isacharparenright}{\kern0pt}\isanewline
\ \ \ \ \isacommand{unfolding}\isamarkupfalse%
\ pair{\isacharunderscore}{\kern0pt}fm{\isacharunderscore}{\kern0pt}def\ upair{\isacharunderscore}{\kern0pt}fm{\isacharunderscore}{\kern0pt}def\isanewline
\ \ \ \ \isacommand{using}\isamarkupfalse%
\ nat{\isacharunderscore}{\kern0pt}simp{\isacharunderscore}{\kern0pt}union\ pred{\isacharunderscore}{\kern0pt}le\ arity{\isacharunderscore}{\kern0pt}type\ \isacommand{by}\isamarkupfalse%
\ auto\isanewline
\ \ \isacommand{{\isacharbraceleft}{\kern0pt}}\isamarkupfalse%
\isanewline
\ \ \ \ \isacommand{fix}\isamarkupfalse%
\ sp\isanewline
\ \ \ \ \isacommand{note}\isamarkupfalse%
\ {\isacartoucheopen}A{\isasymtimes}B\ {\isasymin}\ M{\isacartoucheclose}\isanewline
\ \ \ \ \isacommand{moreover}\isamarkupfalse%
\isanewline
\ \ \ \ \isacommand{assume}\isamarkupfalse%
\ {\isachardoublequoteopen}sp\ {\isasymin}\ A{\isasymtimes}B{\isachardoublequoteclose}\isanewline
\ \ \ \ \isacommand{moreover}\isamarkupfalse%
\ \isacommand{from}\isamarkupfalse%
\ calculation\isanewline
\ \ \ \ \isacommand{have}\isamarkupfalse%
\ {\isachardoublequoteopen}fst{\isacharparenleft}{\kern0pt}sp{\isacharparenright}{\kern0pt}\ {\isasymin}\ A{\isachardoublequoteclose}\ {\isachardoublequoteopen}snd{\isacharparenleft}{\kern0pt}sp{\isacharparenright}{\kern0pt}\ {\isasymin}\ B{\isachardoublequoteclose}\isanewline
\ \ \ \ \ \ \isacommand{using}\isamarkupfalse%
\ fst{\isacharunderscore}{\kern0pt}type\ snd{\isacharunderscore}{\kern0pt}type\ \isacommand{by}\isamarkupfalse%
\ simp{\isacharunderscore}{\kern0pt}all\isanewline
\ \ \ \ \isacommand{ultimately}\isamarkupfalse%
\isanewline
\ \ \ \ \isacommand{have}\isamarkupfalse%
\ {\isachardoublequoteopen}sp\ {\isasymin}\ M{\isachardoublequoteclose}\ {\isachardoublequoteopen}fst{\isacharparenleft}{\kern0pt}sp{\isacharparenright}{\kern0pt}\ {\isasymin}\ M{\isachardoublequoteclose}\ {\isachardoublequoteopen}snd{\isacharparenleft}{\kern0pt}sp{\isacharparenright}{\kern0pt}\ {\isasymin}\ M{\isachardoublequoteclose}\isanewline
\ \ \ \ \ \ \isacommand{using}\isamarkupfalse%
\ \ {\isacartoucheopen}A{\isasymin}M{\isacartoucheclose}\ {\isacartoucheopen}B{\isasymin}M{\isacartoucheclose}\ transitivity\isanewline
\ \ \ \ \ \ \isacommand{by}\isamarkupfalse%
\ simp{\isacharunderscore}{\kern0pt}all\isanewline
\ \ \ \ \isacommand{note}\isamarkupfalse%
\ inM\ {\isacharequal}{\kern0pt}\ {\isacartoucheopen}A{\isasymin}M{\isacartoucheclose}\ {\isacartoucheopen}B{\isasymin}M{\isacartoucheclose}\ {\isacartoucheopen}p{\isasymin}M{\isacartoucheclose}\ {\isacartoucheopen}l{\isasymin}M{\isacartoucheclose}\ {\isacartoucheopen}o{\isasymin}M{\isacartoucheclose}\ {\isacartoucheopen}{\isasymchi}{\isasymin}M{\isacartoucheclose}\isanewline
\ \ \ \ \ \ {\isacartoucheopen}sp{\isasymin}M{\isacartoucheclose}\ {\isacartoucheopen}fst{\isacharparenleft}{\kern0pt}sp{\isacharparenright}{\kern0pt}{\isasymin}M{\isacartoucheclose}\ {\isacartoucheopen}snd{\isacharparenleft}{\kern0pt}sp{\isacharparenright}{\kern0pt}{\isasymin}M{\isacartoucheclose}\isanewline
\ \ \ \ \isacommand{with}\isamarkupfalse%
\ {\isadigit{1}}\ {\isacartoucheopen}sp\ {\isasymin}\ M{\isacartoucheclose}\ {\isacartoucheopen}{\isacharquery}{\kern0pt}{\isasymphi}{\isacharprime}{\kern0pt}\ {\isasymin}\ formula{\isacartoucheclose}\isanewline
\ \ \ \ \isacommand{have}\isamarkupfalse%
\ {\isachardoublequoteopen}M{\isacharcomma}{\kern0pt}\ {\isacharbrackleft}{\kern0pt}sp{\isacharcomma}{\kern0pt}p{\isacharcomma}{\kern0pt}l{\isacharcomma}{\kern0pt}o{\isacharcomma}{\kern0pt}{\isasymchi}{\isacharbrackright}{\kern0pt}{\isacharat}{\kern0pt}{\isacharbrackleft}{\kern0pt}p{\isacharbrackright}{\kern0pt}\ {\isasymTurnstile}\ {\isacharquery}{\kern0pt}{\isasympsi}\ {\isasymlongleftrightarrow}\ M{\isacharcomma}{\kern0pt}{\isacharbrackleft}{\kern0pt}sp{\isacharcomma}{\kern0pt}p{\isacharcomma}{\kern0pt}l{\isacharcomma}{\kern0pt}o{\isacharcomma}{\kern0pt}{\isasymchi}{\isacharbrackright}{\kern0pt}\ {\isasymTurnstile}\ {\isacharquery}{\kern0pt}{\isasympsi}{\isachardoublequoteclose}\ {\isacharparenleft}{\kern0pt}\isakeyword{is}\ {\isachardoublequoteopen}M{\isacharcomma}{\kern0pt}{\isacharquery}{\kern0pt}env{\isadigit{0}}{\isacharat}{\kern0pt}\ {\isacharunderscore}{\kern0pt}{\isasymTurnstile}{\isacharunderscore}{\kern0pt}\ {\isasymlongleftrightarrow}\ {\isacharunderscore}{\kern0pt}{\isachardoublequoteclose}{\isacharparenright}{\kern0pt}\isanewline
\ \ \ \ \ \ \isacommand{using}\isamarkupfalse%
\ arity{\isacharunderscore}{\kern0pt}sats{\isacharunderscore}{\kern0pt}iff{\isacharbrackleft}{\kern0pt}of\ {\isacharquery}{\kern0pt}{\isasympsi}\ {\isachardoublequoteopen}{\isacharbrackleft}{\kern0pt}p{\isacharbrackright}{\kern0pt}{\isachardoublequoteclose}\ M\ {\isacharquery}{\kern0pt}env{\isadigit{0}}{\isacharbrackright}{\kern0pt}\ \isacommand{by}\isamarkupfalse%
\ auto\isanewline
\ \ \ \ \isacommand{also}\isamarkupfalse%
\ \isacommand{from}\isamarkupfalse%
\ inM\ {\isacartoucheopen}sp\ {\isasymin}\ A{\isasymtimes}B{\isacartoucheclose}\isanewline
\ \ \ \ \isacommand{have}\isamarkupfalse%
\ {\isachardoublequoteopen}{\isachardot}{\kern0pt}{\isachardot}{\kern0pt}{\isachardot}{\kern0pt}\ {\isasymlongleftrightarrow}\ sats{\isacharparenleft}{\kern0pt}M{\isacharcomma}{\kern0pt}{\isacharquery}{\kern0pt}{\isasymphi}{\isacharprime}{\kern0pt}{\isacharcomma}{\kern0pt}{\isacharbrackleft}{\kern0pt}fst{\isacharparenleft}{\kern0pt}sp{\isacharparenright}{\kern0pt}{\isacharcomma}{\kern0pt}snd{\isacharparenleft}{\kern0pt}sp{\isacharparenright}{\kern0pt}{\isacharcomma}{\kern0pt}sp{\isacharcomma}{\kern0pt}p{\isacharcomma}{\kern0pt}l{\isacharcomma}{\kern0pt}o{\isacharcomma}{\kern0pt}{\isasymchi}{\isacharbrackright}{\kern0pt}{\isacharparenright}{\kern0pt}{\isachardoublequoteclose}\isanewline
\ \ \ \ \ \ \isacommand{by}\isamarkupfalse%
\ auto\isanewline
\ \ \ \ \isacommand{also}\isamarkupfalse%
\ \isacommand{from}\isamarkupfalse%
\ inM\ {\isacartoucheopen}{\isasymphi}\ {\isasymin}\ formula{\isacartoucheclose}\ {\isacartoucheopen}arity{\isacharparenleft}{\kern0pt}{\isasymphi}{\isacharparenright}{\kern0pt}\ {\isasymle}\ {\isadigit{6}}{\isacartoucheclose}\isanewline
\ \ \ \ \isacommand{have}\isamarkupfalse%
\ {\isachardoublequoteopen}{\isachardot}{\kern0pt}{\isachardot}{\kern0pt}{\isachardot}{\kern0pt}\ {\isasymlongleftrightarrow}\ sats{\isacharparenleft}{\kern0pt}M{\isacharcomma}{\kern0pt}{\isasymphi}{\isacharcomma}{\kern0pt}{\isacharbrackleft}{\kern0pt}snd{\isacharparenleft}{\kern0pt}sp{\isacharparenright}{\kern0pt}{\isacharcomma}{\kern0pt}p{\isacharcomma}{\kern0pt}l{\isacharcomma}{\kern0pt}o{\isacharcomma}{\kern0pt}fst{\isacharparenleft}{\kern0pt}sp{\isacharparenright}{\kern0pt}{\isacharcomma}{\kern0pt}{\isasymchi}{\isacharbrackright}{\kern0pt}{\isacharparenright}{\kern0pt}{\isachardoublequoteclose}\isanewline
\ \ \ \ \ \ {\isacharparenleft}{\kern0pt}\isakeyword{is}\ {\isachardoublequoteopen}sats{\isacharparenleft}{\kern0pt}{\isacharunderscore}{\kern0pt}{\isacharcomma}{\kern0pt}{\isacharunderscore}{\kern0pt}{\isacharcomma}{\kern0pt}{\isacharquery}{\kern0pt}env{\isadigit{1}}{\isacharparenright}{\kern0pt}\ {\isasymlongleftrightarrow}\ sats{\isacharparenleft}{\kern0pt}{\isacharunderscore}{\kern0pt}{\isacharcomma}{\kern0pt}{\isacharunderscore}{\kern0pt}{\isacharcomma}{\kern0pt}{\isacharquery}{\kern0pt}env{\isadigit{2}}{\isacharparenright}{\kern0pt}{\isachardoublequoteclose}{\isacharparenright}{\kern0pt}\isanewline
\ \ \ \ \ \ \isacommand{using}\isamarkupfalse%
\ sats{\isacharunderscore}{\kern0pt}iff{\isacharunderscore}{\kern0pt}sats{\isacharunderscore}{\kern0pt}ren{\isacharbrackleft}{\kern0pt}of\ {\isasymphi}\ {\isadigit{6}}\ {\isadigit{7}}\ {\isacharquery}{\kern0pt}env{\isadigit{2}}\ M\ {\isacharquery}{\kern0pt}env{\isadigit{1}}\ perm{\isacharunderscore}{\kern0pt}pow{\isacharunderscore}{\kern0pt}fn{\isacharbrackright}{\kern0pt}\ perm{\isacharunderscore}{\kern0pt}pow{\isacharunderscore}{\kern0pt}thm\isanewline
\ \ \ \ \ \ \isacommand{unfolding}\isamarkupfalse%
\ perm{\isacharunderscore}{\kern0pt}pow{\isacharunderscore}{\kern0pt}fn{\isacharunderscore}{\kern0pt}def\ \isacommand{by}\isamarkupfalse%
\ simp\isanewline
\ \ \ \ \isacommand{finally}\isamarkupfalse%
\isanewline
\ \ \ \ \isacommand{have}\isamarkupfalse%
\ {\isachardoublequoteopen}sats{\isacharparenleft}{\kern0pt}M{\isacharcomma}{\kern0pt}{\isacharquery}{\kern0pt}{\isasympsi}{\isacharcomma}{\kern0pt}{\isacharbrackleft}{\kern0pt}sp{\isacharcomma}{\kern0pt}p{\isacharcomma}{\kern0pt}l{\isacharcomma}{\kern0pt}o{\isacharcomma}{\kern0pt}{\isasymchi}{\isacharcomma}{\kern0pt}p{\isacharbrackright}{\kern0pt}{\isacharparenright}{\kern0pt}\ {\isasymlongleftrightarrow}\ sats{\isacharparenleft}{\kern0pt}M{\isacharcomma}{\kern0pt}{\isasymphi}{\isacharcomma}{\kern0pt}{\isacharbrackleft}{\kern0pt}snd{\isacharparenleft}{\kern0pt}sp{\isacharparenright}{\kern0pt}{\isacharcomma}{\kern0pt}p{\isacharcomma}{\kern0pt}l{\isacharcomma}{\kern0pt}o{\isacharcomma}{\kern0pt}fst{\isacharparenleft}{\kern0pt}sp{\isacharparenright}{\kern0pt}{\isacharcomma}{\kern0pt}{\isasymchi}{\isacharbrackright}{\kern0pt}{\isacharparenright}{\kern0pt}{\isachardoublequoteclose}\isanewline
\ \ \ \ \ \ \isacommand{by}\isamarkupfalse%
\ simp\isanewline
\ \ \isacommand{{\isacharbraceright}{\kern0pt}}\isamarkupfalse%
\isanewline
\ \ \isacommand{then}\isamarkupfalse%
\ \isacommand{have}\isamarkupfalse%
\isanewline
\ \ \ \ {\isachardoublequoteopen}{\isacharquery}{\kern0pt}{\isasymtheta}\ {\isacharequal}{\kern0pt}\ {\isacharbraceleft}{\kern0pt}sp{\isasymin}A{\isasymtimes}B\ {\isachardot}{\kern0pt}\ sats{\isacharparenleft}{\kern0pt}M{\isacharcomma}{\kern0pt}{\isacharquery}{\kern0pt}{\isasympsi}{\isacharcomma}{\kern0pt}{\isacharbrackleft}{\kern0pt}sp{\isacharcomma}{\kern0pt}p{\isacharcomma}{\kern0pt}l{\isacharcomma}{\kern0pt}o{\isacharcomma}{\kern0pt}{\isasymchi}{\isacharcomma}{\kern0pt}p{\isacharbrackright}{\kern0pt}{\isacharparenright}{\kern0pt}{\isacharbraceright}{\kern0pt}{\isachardoublequoteclose}\isanewline
\ \ \ \ \isacommand{by}\isamarkupfalse%
\ auto\isanewline
\ \ \isacommand{also}\isamarkupfalse%
\ \isacommand{from}\isamarkupfalse%
\ assms\ {\isacartoucheopen}A{\isasymtimes}B{\isasymin}M{\isacartoucheclose}\ \isacommand{have}\isamarkupfalse%
\isanewline
\ \ \ \ {\isachardoublequoteopen}\ {\isachardot}{\kern0pt}{\isachardot}{\kern0pt}{\isachardot}{\kern0pt}\ {\isasymin}\ M{\isachardoublequoteclose}\isanewline
\ \ \isacommand{proof}\isamarkupfalse%
\ {\isacharminus}{\kern0pt}\isanewline
\ \ \ \ \isacommand{from}\isamarkupfalse%
\ {\isadigit{1}}\isanewline
\ \ \ \ \isacommand{have}\isamarkupfalse%
\ {\isachardoublequoteopen}arity{\isacharparenleft}{\kern0pt}{\isacharquery}{\kern0pt}{\isasympsi}{\isacharparenright}{\kern0pt}\ {\isasymle}\ {\isadigit{6}}{\isachardoublequoteclose}\isanewline
\ \ \ \ \ \ \isacommand{using}\isamarkupfalse%
\ leI\ \isacommand{by}\isamarkupfalse%
\ simp\isanewline
\ \ \ \ \isacommand{moreover}\isamarkupfalse%
\ \isacommand{from}\isamarkupfalse%
\ {\isacartoucheopen}{\isacharquery}{\kern0pt}{\isasymphi}{\isacharprime}{\kern0pt}\ {\isasymin}\ formula{\isacartoucheclose}\isanewline
\ \ \ \ \isacommand{have}\isamarkupfalse%
\ {\isachardoublequoteopen}{\isacharquery}{\kern0pt}{\isasympsi}\ {\isasymin}\ formula{\isachardoublequoteclose}\isanewline
\ \ \ \ \ \ \isacommand{by}\isamarkupfalse%
\ simp\isanewline
\ \ \ \ \isacommand{moreover}\isamarkupfalse%
\ \isacommand{note}\isamarkupfalse%
\ assms\ {\isacartoucheopen}A{\isasymtimes}B{\isasymin}M{\isacartoucheclose}\isanewline
\ \ \ \ \isacommand{ultimately}\isamarkupfalse%
\ \isanewline
\ \ \ \ \isacommand{show}\isamarkupfalse%
\ {\isachardoublequoteopen}{\isacharbraceleft}{\kern0pt}x\ {\isasymin}\ A{\isasymtimes}B\ {\isachardot}{\kern0pt}\ sats{\isacharparenleft}{\kern0pt}M{\isacharcomma}{\kern0pt}\ {\isacharquery}{\kern0pt}{\isasympsi}{\isacharcomma}{\kern0pt}\ {\isacharbrackleft}{\kern0pt}x{\isacharcomma}{\kern0pt}\ p{\isacharcomma}{\kern0pt}\ l{\isacharcomma}{\kern0pt}\ o{\isacharcomma}{\kern0pt}\ {\isasymchi}{\isacharcomma}{\kern0pt}\ p{\isacharbrackright}{\kern0pt}{\isacharparenright}{\kern0pt}{\isacharbraceright}{\kern0pt}\ {\isasymin}\ M{\isachardoublequoteclose}\isanewline
\ \ \ \ \ \ \isacommand{using}\isamarkupfalse%
\ separation{\isacharunderscore}{\kern0pt}ax\ separation{\isacharunderscore}{\kern0pt}iff\isanewline
\ \ \ \ \ \ \isacommand{by}\isamarkupfalse%
\ simp\isanewline
\ \ \isacommand{qed}\isamarkupfalse%
\isanewline
\ \ \isacommand{finally}\isamarkupfalse%
\ \isacommand{show}\isamarkupfalse%
\ {\isacharquery}{\kern0pt}thesis\ \isacommand{{\isachardot}{\kern0pt}}\isamarkupfalse%
\isanewline
\isacommand{qed}\isamarkupfalse%
%
\endisatagproof
{\isafoldproof}%
%
\isadelimproof
\isanewline
%
\endisadelimproof
\isanewline
\isacommand{lemma}\isamarkupfalse%
\ Pow{\isacharunderscore}{\kern0pt}inter{\isacharunderscore}{\kern0pt}MG{\isacharcolon}{\kern0pt}\isanewline
\ \ \isakeyword{assumes}\isanewline
\ \ \ \ {\isachardoublequoteopen}a{\isasymin}M{\isacharbrackleft}{\kern0pt}G{\isacharbrackright}{\kern0pt}{\isachardoublequoteclose}\isanewline
\ \ \isakeyword{shows}\isanewline
\ \ \ \ {\isachardoublequoteopen}Pow{\isacharparenleft}{\kern0pt}a{\isacharparenright}{\kern0pt}\ {\isasyminter}\ M{\isacharbrackleft}{\kern0pt}G{\isacharbrackright}{\kern0pt}\ {\isasymin}\ M{\isacharbrackleft}{\kern0pt}G{\isacharbrackright}{\kern0pt}{\isachardoublequoteclose}\isanewline
%
\isadelimproof
%
\endisadelimproof
%
\isatagproof
\isacommand{proof}\isamarkupfalse%
\ {\isacharminus}{\kern0pt}\isanewline
\ \ \isacommand{from}\isamarkupfalse%
\ assms\ \isacommand{obtain}\isamarkupfalse%
\ {\isasymtau}\ \isakeyword{where}\isanewline
\ \ \ \ {\isachardoublequoteopen}{\isasymtau}\ {\isasymin}\ M{\isachardoublequoteclose}\ {\isachardoublequoteopen}val{\isacharparenleft}{\kern0pt}G{\isacharcomma}{\kern0pt}\ {\isasymtau}{\isacharparenright}{\kern0pt}\ {\isacharequal}{\kern0pt}\ a{\isachardoublequoteclose}\isanewline
\ \ \ \ \isacommand{using}\isamarkupfalse%
\ GenExtD\ \isacommand{by}\isamarkupfalse%
\ auto\isanewline
\ \ \isacommand{let}\isamarkupfalse%
\ {\isacharquery}{\kern0pt}Q{\isacharequal}{\kern0pt}{\isachardoublequoteopen}Pow{\isacharparenleft}{\kern0pt}domain{\isacharparenleft}{\kern0pt}{\isasymtau}{\isacharparenright}{\kern0pt}{\isasymtimes}P{\isacharparenright}{\kern0pt}\ {\isasyminter}\ M{\isachardoublequoteclose}\isanewline
\ \ \isacommand{from}\isamarkupfalse%
\ {\isacartoucheopen}{\isasymtau}{\isasymin}M{\isacartoucheclose}\ \isanewline
\ \ \isacommand{have}\isamarkupfalse%
\ {\isachardoublequoteopen}domain{\isacharparenleft}{\kern0pt}{\isasymtau}{\isacharparenright}{\kern0pt}{\isasymtimes}P\ {\isasymin}\ M{\isachardoublequoteclose}\ {\isachardoublequoteopen}domain{\isacharparenleft}{\kern0pt}{\isasymtau}{\isacharparenright}{\kern0pt}\ {\isasymin}\ M{\isachardoublequoteclose}\isanewline
\ \ \ \ \isacommand{using}\isamarkupfalse%
\ domain{\isacharunderscore}{\kern0pt}closed\ cartprod{\isacharunderscore}{\kern0pt}closed\ P{\isacharunderscore}{\kern0pt}in{\isacharunderscore}{\kern0pt}M\isanewline
\ \ \ \ \isacommand{by}\isamarkupfalse%
\ simp{\isacharunderscore}{\kern0pt}all\isanewline
\ \ \isacommand{then}\isamarkupfalse%
\ \isanewline
\ \ \isacommand{have}\isamarkupfalse%
\ {\isachardoublequoteopen}{\isacharquery}{\kern0pt}Q\ {\isasymin}\ M{\isachardoublequoteclose}\isanewline
\ \ \isacommand{proof}\isamarkupfalse%
\ {\isacharminus}{\kern0pt}\isanewline
\ \ \ \ \isacommand{from}\isamarkupfalse%
\ power{\isacharunderscore}{\kern0pt}ax\ {\isacartoucheopen}domain{\isacharparenleft}{\kern0pt}{\isasymtau}{\isacharparenright}{\kern0pt}{\isasymtimes}P\ {\isasymin}\ M{\isacartoucheclose}\ \isacommand{obtain}\isamarkupfalse%
\ Q\ \isakeyword{where}\isanewline
\ \ \ \ \ \ {\isachardoublequoteopen}powerset{\isacharparenleft}{\kern0pt}{\isacharhash}{\kern0pt}{\isacharhash}{\kern0pt}M{\isacharcomma}{\kern0pt}domain{\isacharparenleft}{\kern0pt}{\isasymtau}{\isacharparenright}{\kern0pt}{\isasymtimes}P{\isacharcomma}{\kern0pt}Q{\isacharparenright}{\kern0pt}{\isachardoublequoteclose}\ {\isachardoublequoteopen}Q\ {\isasymin}\ M{\isachardoublequoteclose}\isanewline
\ \ \ \ \ \ \isacommand{unfolding}\isamarkupfalse%
\ power{\isacharunderscore}{\kern0pt}ax{\isacharunderscore}{\kern0pt}def\ \isacommand{by}\isamarkupfalse%
\ auto\isanewline
\ \ \ \ \isacommand{moreover}\isamarkupfalse%
\ \isacommand{from}\isamarkupfalse%
\ calculation\ \isanewline
\ \ \ \ \isacommand{have}\isamarkupfalse%
\ {\isachardoublequoteopen}z{\isasymin}Q\ {\isasymLongrightarrow}\ z{\isasymin}M{\isachardoublequoteclose}\ \isakeyword{for}\ z\isanewline
\ \ \ \ \ \ \isacommand{using}\isamarkupfalse%
\ transitivity\ \isacommand{by}\isamarkupfalse%
\ blast\isanewline
\ \ \ \ \isacommand{ultimately}\isamarkupfalse%
\isanewline
\ \ \ \ \isacommand{have}\isamarkupfalse%
\ {\isachardoublequoteopen}Q\ {\isacharequal}{\kern0pt}\ {\isacharbraceleft}{\kern0pt}a{\isasymin}Pow{\isacharparenleft}{\kern0pt}domain{\isacharparenleft}{\kern0pt}{\isasymtau}{\isacharparenright}{\kern0pt}{\isasymtimes}P{\isacharparenright}{\kern0pt}\ {\isachardot}{\kern0pt}\ a{\isasymin}M{\isacharbraceright}{\kern0pt}{\isachardoublequoteclose}\isanewline
\ \ \ \ \ \ \isacommand{using}\isamarkupfalse%
\ {\isacartoucheopen}domain{\isacharparenleft}{\kern0pt}{\isasymtau}{\isacharparenright}{\kern0pt}{\isasymtimes}P\ {\isasymin}\ M{\isacartoucheclose}\ powerset{\isacharunderscore}{\kern0pt}abs{\isacharbrackleft}{\kern0pt}of\ {\isachardoublequoteopen}domain{\isacharparenleft}{\kern0pt}{\isasymtau}{\isacharparenright}{\kern0pt}{\isasymtimes}P{\isachardoublequoteclose}\ Q{\isacharbrackright}{\kern0pt}\isanewline
\ \ \ \ \ \ \isacommand{by}\isamarkupfalse%
\ {\isacharparenleft}{\kern0pt}simp\ flip{\isacharcolon}{\kern0pt}\ setclass{\isacharunderscore}{\kern0pt}iff{\isacharparenright}{\kern0pt}\isanewline
\ \ \ \ \isacommand{also}\isamarkupfalse%
\ \isanewline
\ \ \ \ \isacommand{have}\isamarkupfalse%
\ {\isachardoublequoteopen}\ {\isachardot}{\kern0pt}{\isachardot}{\kern0pt}{\isachardot}{\kern0pt}\ {\isacharequal}{\kern0pt}\ {\isacharquery}{\kern0pt}Q{\isachardoublequoteclose}\isanewline
\ \ \ \ \ \ \isacommand{by}\isamarkupfalse%
\ auto\isanewline
\ \ \ \ \isacommand{finally}\isamarkupfalse%
\ \isanewline
\ \ \ \ \isacommand{show}\isamarkupfalse%
\ {\isacharquery}{\kern0pt}thesis\ \isacommand{using}\isamarkupfalse%
\ {\isacartoucheopen}Q{\isasymin}M{\isacartoucheclose}\ \isacommand{by}\isamarkupfalse%
\ simp\isanewline
\ \ \isacommand{qed}\isamarkupfalse%
\isanewline
\ \ \isacommand{let}\isamarkupfalse%
\isanewline
\ \ \ \ {\isacharquery}{\kern0pt}{\isasympi}{\isacharequal}{\kern0pt}{\isachardoublequoteopen}{\isacharquery}{\kern0pt}Q{\isasymtimes}{\isacharbraceleft}{\kern0pt}one{\isacharbraceright}{\kern0pt}{\isachardoublequoteclose}\isanewline
\ \ \isacommand{let}\isamarkupfalse%
\isanewline
\ \ \ \ {\isacharquery}{\kern0pt}b{\isacharequal}{\kern0pt}{\isachardoublequoteopen}val{\isacharparenleft}{\kern0pt}G{\isacharcomma}{\kern0pt}{\isacharquery}{\kern0pt}{\isasympi}{\isacharparenright}{\kern0pt}{\isachardoublequoteclose}\isanewline
\ \ \isacommand{from}\isamarkupfalse%
\ {\isacartoucheopen}{\isacharquery}{\kern0pt}Q{\isasymin}M{\isacartoucheclose}\ \isanewline
\ \ \isacommand{have}\isamarkupfalse%
\ {\isachardoublequoteopen}{\isacharquery}{\kern0pt}{\isasympi}{\isasymin}M{\isachardoublequoteclose}\isanewline
\ \ \ \ \isacommand{using}\isamarkupfalse%
\ one{\isacharunderscore}{\kern0pt}in{\isacharunderscore}{\kern0pt}P\ P{\isacharunderscore}{\kern0pt}in{\isacharunderscore}{\kern0pt}M\ transitivity\isanewline
\ \ \ \ \isacommand{by}\isamarkupfalse%
\ {\isacharparenleft}{\kern0pt}simp\ flip{\isacharcolon}{\kern0pt}\ setclass{\isacharunderscore}{\kern0pt}iff{\isacharparenright}{\kern0pt}\isanewline
\ \ \isacommand{from}\isamarkupfalse%
\ {\isacartoucheopen}{\isacharquery}{\kern0pt}{\isasympi}{\isasymin}M{\isacartoucheclose}\ \isanewline
\ \ \isacommand{have}\isamarkupfalse%
\ {\isachardoublequoteopen}{\isacharquery}{\kern0pt}b\ {\isasymin}\ M{\isacharbrackleft}{\kern0pt}G{\isacharbrackright}{\kern0pt}{\isachardoublequoteclose}\isanewline
\ \ \ \ \isacommand{using}\isamarkupfalse%
\ GenExtI\ \isacommand{by}\isamarkupfalse%
\ simp\isanewline
\ \ \isacommand{have}\isamarkupfalse%
\ {\isachardoublequoteopen}Pow{\isacharparenleft}{\kern0pt}a{\isacharparenright}{\kern0pt}\ {\isasyminter}\ M{\isacharbrackleft}{\kern0pt}G{\isacharbrackright}{\kern0pt}\ {\isasymsubseteq}\ {\isacharquery}{\kern0pt}b{\isachardoublequoteclose}\isanewline
\ \ \isacommand{proof}\isamarkupfalse%
\isanewline
\ \ \ \ \isacommand{fix}\isamarkupfalse%
\ c\isanewline
\ \ \ \ \isacommand{assume}\isamarkupfalse%
\ {\isachardoublequoteopen}c\ {\isasymin}\ Pow{\isacharparenleft}{\kern0pt}a{\isacharparenright}{\kern0pt}\ {\isasyminter}\ M{\isacharbrackleft}{\kern0pt}G{\isacharbrackright}{\kern0pt}{\isachardoublequoteclose}\isanewline
\ \ \ \ \isacommand{then}\isamarkupfalse%
\ \isacommand{obtain}\isamarkupfalse%
\ {\isasymchi}\ \isakeyword{where}\isanewline
\ \ \ \ \ \ {\isachardoublequoteopen}c{\isasymin}M{\isacharbrackleft}{\kern0pt}G{\isacharbrackright}{\kern0pt}{\isachardoublequoteclose}\ {\isachardoublequoteopen}{\isasymchi}\ {\isasymin}\ M{\isachardoublequoteclose}\ {\isachardoublequoteopen}val{\isacharparenleft}{\kern0pt}G{\isacharcomma}{\kern0pt}{\isasymchi}{\isacharparenright}{\kern0pt}\ {\isacharequal}{\kern0pt}\ c{\isachardoublequoteclose}\isanewline
\ \ \ \ \ \ \isacommand{using}\isamarkupfalse%
\ GenExtD\ \isacommand{by}\isamarkupfalse%
\ auto\isanewline
\ \ \ \ \isacommand{let}\isamarkupfalse%
\ {\isacharquery}{\kern0pt}{\isasymtheta}{\isacharequal}{\kern0pt}{\isachardoublequoteopen}{\isacharbraceleft}{\kern0pt}sp\ {\isasymin}domain{\isacharparenleft}{\kern0pt}{\isasymtau}{\isacharparenright}{\kern0pt}{\isasymtimes}P\ {\isachardot}{\kern0pt}\ snd{\isacharparenleft}{\kern0pt}sp{\isacharparenright}{\kern0pt}\ {\isasymtturnstile}\ {\isacharparenleft}{\kern0pt}Member{\isacharparenleft}{\kern0pt}{\isadigit{0}}{\isacharcomma}{\kern0pt}{\isadigit{1}}{\isacharparenright}{\kern0pt}{\isacharparenright}{\kern0pt}\ {\isacharbrackleft}{\kern0pt}fst{\isacharparenleft}{\kern0pt}sp{\isacharparenright}{\kern0pt}{\isacharcomma}{\kern0pt}{\isasymchi}{\isacharbrackright}{\kern0pt}\ {\isacharbraceright}{\kern0pt}{\isachardoublequoteclose}\isanewline
\ \ \ \ \isacommand{have}\isamarkupfalse%
\ {\isachardoublequoteopen}arity{\isacharparenleft}{\kern0pt}forces{\isacharparenleft}{\kern0pt}Member{\isacharparenleft}{\kern0pt}{\isadigit{0}}{\isacharcomma}{\kern0pt}{\isadigit{1}}{\isacharparenright}{\kern0pt}{\isacharparenright}{\kern0pt}{\isacharparenright}{\kern0pt}\ {\isacharequal}{\kern0pt}\ {\isadigit{6}}{\isachardoublequoteclose}\isanewline
\ \ \ \ \ \ \isacommand{using}\isamarkupfalse%
\ arity{\isacharunderscore}{\kern0pt}forces{\isacharunderscore}{\kern0pt}at\ \isacommand{by}\isamarkupfalse%
\ auto\isanewline
\ \ \ \ \isacommand{with}\isamarkupfalse%
\ {\isacartoucheopen}domain{\isacharparenleft}{\kern0pt}{\isasymtau}{\isacharparenright}{\kern0pt}\ {\isasymin}\ M{\isacartoucheclose}\ {\isacartoucheopen}{\isasymchi}\ {\isasymin}\ M{\isacartoucheclose}\ \isanewline
\ \ \ \ \isacommand{have}\isamarkupfalse%
\ {\isachardoublequoteopen}{\isacharquery}{\kern0pt}{\isasymtheta}\ {\isasymin}\ M{\isachardoublequoteclose}\isanewline
\ \ \ \ \ \ \isacommand{using}\isamarkupfalse%
\ P{\isacharunderscore}{\kern0pt}in{\isacharunderscore}{\kern0pt}M\ one{\isacharunderscore}{\kern0pt}in{\isacharunderscore}{\kern0pt}M\ leq{\isacharunderscore}{\kern0pt}in{\isacharunderscore}{\kern0pt}M\ sats{\isacharunderscore}{\kern0pt}fst{\isacharunderscore}{\kern0pt}snd{\isacharunderscore}{\kern0pt}in{\isacharunderscore}{\kern0pt}M\isanewline
\ \ \ \ \ \ \isacommand{by}\isamarkupfalse%
\ simp\isanewline
\ \ \ \ \isacommand{then}\isamarkupfalse%
\ \isanewline
\ \ \ \ \isacommand{have}\isamarkupfalse%
\ {\isachardoublequoteopen}{\isacharquery}{\kern0pt}{\isasymtheta}\ {\isasymin}\ {\isacharquery}{\kern0pt}Q{\isachardoublequoteclose}\isanewline
\ \ \ \ \ \ \isacommand{by}\isamarkupfalse%
\ auto\isanewline
\ \ \ \ \isacommand{then}\isamarkupfalse%
\ \isanewline
\ \ \ \ \isacommand{have}\isamarkupfalse%
\ {\isachardoublequoteopen}val{\isacharparenleft}{\kern0pt}G{\isacharcomma}{\kern0pt}{\isacharquery}{\kern0pt}{\isasymtheta}{\isacharparenright}{\kern0pt}\ {\isasymin}\ {\isacharquery}{\kern0pt}b{\isachardoublequoteclose}\isanewline
\ \ \ \ \ \ \isacommand{using}\isamarkupfalse%
\ one{\isacharunderscore}{\kern0pt}in{\isacharunderscore}{\kern0pt}G\ one{\isacharunderscore}{\kern0pt}in{\isacharunderscore}{\kern0pt}P\ generic\ val{\isacharunderscore}{\kern0pt}of{\isacharunderscore}{\kern0pt}elem\ {\isacharbrackleft}{\kern0pt}of\ {\isacharquery}{\kern0pt}{\isasymtheta}\ one\ {\isacharquery}{\kern0pt}{\isasympi}\ G{\isacharbrackright}{\kern0pt}\isanewline
\ \ \ \ \ \ \isacommand{by}\isamarkupfalse%
\ auto\isanewline
\ \ \ \ \isacommand{have}\isamarkupfalse%
\ {\isachardoublequoteopen}val{\isacharparenleft}{\kern0pt}G{\isacharcomma}{\kern0pt}{\isacharquery}{\kern0pt}{\isasymtheta}{\isacharparenright}{\kern0pt}\ {\isacharequal}{\kern0pt}\ c{\isachardoublequoteclose}\isanewline
\ \ \ \ \isacommand{proof}\isamarkupfalse%
{\isacharparenleft}{\kern0pt}intro\ equalityI\ subsetI{\isacharparenright}{\kern0pt}\isanewline
\ \ \ \ \ \ \isacommand{fix}\isamarkupfalse%
\ x\isanewline
\ \ \ \ \ \ \isacommand{assume}\isamarkupfalse%
\ {\isachardoublequoteopen}x\ {\isasymin}\ val{\isacharparenleft}{\kern0pt}G{\isacharcomma}{\kern0pt}{\isacharquery}{\kern0pt}{\isasymtheta}{\isacharparenright}{\kern0pt}{\isachardoublequoteclose}\isanewline
\ \ \ \ \ \ \isacommand{then}\isamarkupfalse%
\ \isacommand{obtain}\isamarkupfalse%
\ {\isasymsigma}\ p\ \isakeyword{where}\isanewline
\ \ \ \ \ \ \ \ {\isadigit{1}}{\isacharcolon}{\kern0pt}\ {\isachardoublequoteopen}{\isacharless}{\kern0pt}{\isasymsigma}{\isacharcomma}{\kern0pt}p{\isachargreater}{\kern0pt}{\isasymin}{\isacharquery}{\kern0pt}{\isasymtheta}{\isachardoublequoteclose}\ {\isachardoublequoteopen}p{\isasymin}G{\isachardoublequoteclose}\ {\isachardoublequoteopen}val{\isacharparenleft}{\kern0pt}G{\isacharcomma}{\kern0pt}{\isasymsigma}{\isacharparenright}{\kern0pt}\ {\isacharequal}{\kern0pt}\ \ x{\isachardoublequoteclose}\isanewline
\ \ \ \ \ \ \ \ \isacommand{using}\isamarkupfalse%
\ elem{\isacharunderscore}{\kern0pt}of{\isacharunderscore}{\kern0pt}val{\isacharunderscore}{\kern0pt}pair\isanewline
\ \ \ \ \ \ \ \ \isacommand{by}\isamarkupfalse%
\ blast\isanewline
\ \ \ \ \ \ \isacommand{moreover}\isamarkupfalse%
\ \isacommand{from}\isamarkupfalse%
\ {\isacartoucheopen}{\isacharless}{\kern0pt}{\isasymsigma}{\isacharcomma}{\kern0pt}p{\isachargreater}{\kern0pt}{\isasymin}{\isacharquery}{\kern0pt}{\isasymtheta}{\isacartoucheclose}\ {\isacartoucheopen}{\isacharquery}{\kern0pt}{\isasymtheta}\ {\isasymin}\ M{\isacartoucheclose}\isanewline
\ \ \ \ \ \ \isacommand{have}\isamarkupfalse%
\ {\isachardoublequoteopen}{\isasymsigma}{\isasymin}M{\isachardoublequoteclose}\isanewline
\ \ \ \ \ \ \ \ \isacommand{using}\isamarkupfalse%
\ name{\isacharunderscore}{\kern0pt}components{\isacharunderscore}{\kern0pt}in{\isacharunderscore}{\kern0pt}M{\isacharbrackleft}{\kern0pt}of\ {\isacharunderscore}{\kern0pt}\ {\isacharunderscore}{\kern0pt}\ {\isacharquery}{\kern0pt}{\isasymtheta}{\isacharbrackright}{\kern0pt}\ \isacommand{by}\isamarkupfalse%
\ auto\isanewline
\ \ \ \ \ \ \isacommand{moreover}\isamarkupfalse%
\ \isacommand{from}\isamarkupfalse%
\ {\isadigit{1}}\ \isanewline
\ \ \ \ \ \ \isacommand{have}\isamarkupfalse%
\ {\isachardoublequoteopen}{\isacharparenleft}{\kern0pt}p\ {\isasymtturnstile}\ {\isacharparenleft}{\kern0pt}Member{\isacharparenleft}{\kern0pt}{\isadigit{0}}{\isacharcomma}{\kern0pt}{\isadigit{1}}{\isacharparenright}{\kern0pt}{\isacharparenright}{\kern0pt}\ {\isacharbrackleft}{\kern0pt}{\isasymsigma}{\isacharcomma}{\kern0pt}{\isasymchi}{\isacharbrackright}{\kern0pt}{\isacharparenright}{\kern0pt}{\isachardoublequoteclose}\ {\isachardoublequoteopen}p{\isasymin}P{\isachardoublequoteclose}\isanewline
\ \ \ \ \ \ \ \ \isacommand{by}\isamarkupfalse%
\ simp{\isacharunderscore}{\kern0pt}all\isanewline
\ \ \ \ \ \ \isacommand{moreover}\isamarkupfalse%
\ \isanewline
\ \ \ \ \ \ \isacommand{note}\isamarkupfalse%
\ {\isacartoucheopen}val{\isacharparenleft}{\kern0pt}G{\isacharcomma}{\kern0pt}{\isasymchi}{\isacharparenright}{\kern0pt}\ {\isacharequal}{\kern0pt}\ c{\isacartoucheclose}\isanewline
\ \ \ \ \ \ \isacommand{ultimately}\isamarkupfalse%
\ \isanewline
\ \ \ \ \ \ \isacommand{have}\isamarkupfalse%
\ {\isachardoublequoteopen}sats{\isacharparenleft}{\kern0pt}M{\isacharbrackleft}{\kern0pt}G{\isacharbrackright}{\kern0pt}{\isacharcomma}{\kern0pt}Member{\isacharparenleft}{\kern0pt}{\isadigit{0}}{\isacharcomma}{\kern0pt}{\isadigit{1}}{\isacharparenright}{\kern0pt}{\isacharcomma}{\kern0pt}{\isacharbrackleft}{\kern0pt}x{\isacharcomma}{\kern0pt}c{\isacharbrackright}{\kern0pt}{\isacharparenright}{\kern0pt}{\isachardoublequoteclose}\isanewline
\ \ \ \ \ \ \ \ \isacommand{using}\isamarkupfalse%
\ {\isacartoucheopen}{\isasymchi}\ {\isasymin}\ M{\isacartoucheclose}\ generic\ definition{\isacharunderscore}{\kern0pt}of{\isacharunderscore}{\kern0pt}forcing\ nat{\isacharunderscore}{\kern0pt}simp{\isacharunderscore}{\kern0pt}union\isanewline
\ \ \ \ \ \ \ \ \isacommand{by}\isamarkupfalse%
\ auto\isanewline
\ \ \ \ \ \ \isacommand{moreover}\isamarkupfalse%
\ \isanewline
\ \ \ \ \ \ \isacommand{have}\isamarkupfalse%
\ {\isachardoublequoteopen}x{\isasymin}M{\isacharbrackleft}{\kern0pt}G{\isacharbrackright}{\kern0pt}{\isachardoublequoteclose}\isanewline
\ \ \ \ \ \ \ \ \isacommand{using}\isamarkupfalse%
\ {\isacartoucheopen}val{\isacharparenleft}{\kern0pt}G{\isacharcomma}{\kern0pt}{\isasymsigma}{\isacharparenright}{\kern0pt}\ {\isacharequal}{\kern0pt}\ \ x{\isacartoucheclose}\ {\isacartoucheopen}{\isasymsigma}{\isasymin}M{\isacartoucheclose}\ \ {\isacartoucheopen}{\isasymchi}{\isasymin}M{\isacartoucheclose}\ GenExtI\ \isacommand{by}\isamarkupfalse%
\ blast\isanewline
\ \ \ \ \ \ \isacommand{ultimately}\isamarkupfalse%
\ \isanewline
\ \ \ \ \ \ \isacommand{show}\isamarkupfalse%
\ {\isachardoublequoteopen}x{\isasymin}c{\isachardoublequoteclose}\isanewline
\ \ \ \ \ \ \ \ \isacommand{using}\isamarkupfalse%
\ {\isacartoucheopen}c{\isasymin}M{\isacharbrackleft}{\kern0pt}G{\isacharbrackright}{\kern0pt}{\isacartoucheclose}\ \isacommand{by}\isamarkupfalse%
\ simp\isanewline
\ \ \ \ \isacommand{next}\isamarkupfalse%
\isanewline
\ \ \ \ \ \ \isacommand{fix}\isamarkupfalse%
\ x\isanewline
\ \ \ \ \ \ \isacommand{assume}\isamarkupfalse%
\ {\isachardoublequoteopen}x\ {\isasymin}\ c{\isachardoublequoteclose}\isanewline
\ \ \ \ \ \ \isacommand{with}\isamarkupfalse%
\ {\isacartoucheopen}c\ {\isasymin}\ Pow{\isacharparenleft}{\kern0pt}a{\isacharparenright}{\kern0pt}\ {\isasyminter}\ M{\isacharbrackleft}{\kern0pt}G{\isacharbrackright}{\kern0pt}{\isacartoucheclose}\ \isanewline
\ \ \ \ \ \ \isacommand{have}\isamarkupfalse%
\ {\isachardoublequoteopen}x\ {\isasymin}\ a{\isachardoublequoteclose}\ {\isachardoublequoteopen}c{\isasymin}M{\isacharbrackleft}{\kern0pt}G{\isacharbrackright}{\kern0pt}{\isachardoublequoteclose}\ {\isachardoublequoteopen}x{\isasymin}M{\isacharbrackleft}{\kern0pt}G{\isacharbrackright}{\kern0pt}{\isachardoublequoteclose}\isanewline
\ \ \ \ \ \ \ \ \isacommand{using}\isamarkupfalse%
\ transitivity{\isacharunderscore}{\kern0pt}MG\isanewline
\ \ \ \ \ \ \ \ \isacommand{by}\isamarkupfalse%
\ auto\isanewline
\ \ \ \ \ \ \isacommand{with}\isamarkupfalse%
\ {\isacartoucheopen}val{\isacharparenleft}{\kern0pt}G{\isacharcomma}{\kern0pt}\ {\isasymtau}{\isacharparenright}{\kern0pt}\ {\isacharequal}{\kern0pt}\ a{\isacartoucheclose}\ \isanewline
\ \ \ \ \ \ \isacommand{obtain}\isamarkupfalse%
\ {\isasymsigma}\ \isakeyword{where}\isanewline
\ \ \ \ \ \ \ \ {\isachardoublequoteopen}{\isasymsigma}{\isasymin}domain{\isacharparenleft}{\kern0pt}{\isasymtau}{\isacharparenright}{\kern0pt}{\isachardoublequoteclose}\ {\isachardoublequoteopen}val{\isacharparenleft}{\kern0pt}G{\isacharcomma}{\kern0pt}{\isasymsigma}{\isacharparenright}{\kern0pt}\ {\isacharequal}{\kern0pt}\ \ x{\isachardoublequoteclose}\isanewline
\ \ \ \ \ \ \ \ \isacommand{using}\isamarkupfalse%
\ elem{\isacharunderscore}{\kern0pt}of{\isacharunderscore}{\kern0pt}val\isanewline
\ \ \ \ \ \ \ \ \isacommand{by}\isamarkupfalse%
\ blast\isanewline
\ \ \ \ \ \ \isacommand{moreover}\isamarkupfalse%
\ \isacommand{note}\isamarkupfalse%
\ {\isacartoucheopen}x{\isasymin}c{\isacartoucheclose}\ {\isacartoucheopen}val{\isacharparenleft}{\kern0pt}G{\isacharcomma}{\kern0pt}{\isasymchi}{\isacharparenright}{\kern0pt}\ {\isacharequal}{\kern0pt}\ c{\isacartoucheclose}\isanewline
\ \ \ \ \ \ \isacommand{moreover}\isamarkupfalse%
\ \isacommand{from}\isamarkupfalse%
\ calculation\ \isanewline
\ \ \ \ \ \ \isacommand{have}\isamarkupfalse%
\ {\isachardoublequoteopen}val{\isacharparenleft}{\kern0pt}G{\isacharcomma}{\kern0pt}{\isasymsigma}{\isacharparenright}{\kern0pt}\ {\isasymin}\ val{\isacharparenleft}{\kern0pt}G{\isacharcomma}{\kern0pt}{\isasymchi}{\isacharparenright}{\kern0pt}{\isachardoublequoteclose}\isanewline
\ \ \ \ \ \ \ \ \isacommand{by}\isamarkupfalse%
\ simp\isanewline
\ \ \ \ \ \ \isacommand{moreover}\isamarkupfalse%
\ \isacommand{note}\isamarkupfalse%
\ {\isacartoucheopen}c{\isasymin}M{\isacharbrackleft}{\kern0pt}G{\isacharbrackright}{\kern0pt}{\isacartoucheclose}\ {\isacartoucheopen}x{\isasymin}M{\isacharbrackleft}{\kern0pt}G{\isacharbrackright}{\kern0pt}{\isacartoucheclose}\isanewline
\ \ \ \ \ \ \isacommand{moreover}\isamarkupfalse%
\ \isacommand{from}\isamarkupfalse%
\ calculation\ \isanewline
\ \ \ \ \ \ \isacommand{have}\isamarkupfalse%
\ {\isachardoublequoteopen}sats{\isacharparenleft}{\kern0pt}M{\isacharbrackleft}{\kern0pt}G{\isacharbrackright}{\kern0pt}{\isacharcomma}{\kern0pt}Member{\isacharparenleft}{\kern0pt}{\isadigit{0}}{\isacharcomma}{\kern0pt}{\isadigit{1}}{\isacharparenright}{\kern0pt}{\isacharcomma}{\kern0pt}{\isacharbrackleft}{\kern0pt}x{\isacharcomma}{\kern0pt}c{\isacharbrackright}{\kern0pt}{\isacharparenright}{\kern0pt}{\isachardoublequoteclose}\isanewline
\ \ \ \ \ \ \ \ \isacommand{by}\isamarkupfalse%
\ simp\isanewline
\ \ \ \ \ \ \isacommand{moreover}\isamarkupfalse%
\ \isanewline
\ \ \ \ \ \ \isacommand{have}\isamarkupfalse%
\ {\isachardoublequoteopen}Member{\isacharparenleft}{\kern0pt}{\isadigit{0}}{\isacharcomma}{\kern0pt}{\isadigit{1}}{\isacharparenright}{\kern0pt}{\isasymin}formula{\isachardoublequoteclose}\ \isacommand{by}\isamarkupfalse%
\ simp\isanewline
\ \ \ \ \ \ \isacommand{moreover}\isamarkupfalse%
\ \isanewline
\ \ \ \ \ \ \isacommand{have}\isamarkupfalse%
\ {\isachardoublequoteopen}{\isasymsigma}{\isasymin}M{\isachardoublequoteclose}\isanewline
\ \ \ \ \ \ \isacommand{proof}\isamarkupfalse%
\ {\isacharminus}{\kern0pt}\isanewline
\ \ \ \ \ \ \ \ \isacommand{from}\isamarkupfalse%
\ {\isacartoucheopen}{\isasymsigma}{\isasymin}domain{\isacharparenleft}{\kern0pt}{\isasymtau}{\isacharparenright}{\kern0pt}{\isacartoucheclose}\ \isanewline
\ \ \ \ \ \ \ \ \isacommand{obtain}\isamarkupfalse%
\ p\ \isakeyword{where}\ {\isachardoublequoteopen}{\isacharless}{\kern0pt}{\isasymsigma}{\isacharcomma}{\kern0pt}p{\isachargreater}{\kern0pt}\ {\isasymin}\ {\isasymtau}{\isachardoublequoteclose}\isanewline
\ \ \ \ \ \ \ \ \ \ \isacommand{by}\isamarkupfalse%
\ auto\isanewline
\ \ \ \ \ \ \ \ \isacommand{with}\isamarkupfalse%
\ {\isacartoucheopen}{\isasymtau}{\isasymin}M{\isacartoucheclose}\ \isanewline
\ \ \ \ \ \ \ \ \isacommand{show}\isamarkupfalse%
\ {\isacharquery}{\kern0pt}thesis\isanewline
\ \ \ \ \ \ \ \ \ \ \isacommand{using}\isamarkupfalse%
\ name{\isacharunderscore}{\kern0pt}components{\isacharunderscore}{\kern0pt}in{\isacharunderscore}{\kern0pt}M\ \isacommand{by}\isamarkupfalse%
\ blast\isanewline
\ \ \ \ \ \ \isacommand{qed}\isamarkupfalse%
\isanewline
\ \ \ \ \ \ \isacommand{moreover}\isamarkupfalse%
\ \isacommand{note}\isamarkupfalse%
\ {\isacartoucheopen}{\isasymchi}\ {\isasymin}\ M{\isacartoucheclose}\isanewline
\ \ \ \ \ \ \isacommand{ultimately}\isamarkupfalse%
\ \isanewline
\ \ \ \ \ \ \isacommand{obtain}\isamarkupfalse%
\ p\ \isakeyword{where}\ {\isachardoublequoteopen}p{\isasymin}G{\isachardoublequoteclose}\ {\isachardoublequoteopen}{\isacharparenleft}{\kern0pt}p\ {\isasymtturnstile}\ Member{\isacharparenleft}{\kern0pt}{\isadigit{0}}{\isacharcomma}{\kern0pt}{\isadigit{1}}{\isacharparenright}{\kern0pt}\ {\isacharbrackleft}{\kern0pt}{\isasymsigma}{\isacharcomma}{\kern0pt}{\isasymchi}{\isacharbrackright}{\kern0pt}{\isacharparenright}{\kern0pt}{\isachardoublequoteclose}\isanewline
\ \ \ \ \ \ \ \ \isacommand{using}\isamarkupfalse%
\ generic\ truth{\isacharunderscore}{\kern0pt}lemma{\isacharbrackleft}{\kern0pt}of\ {\isachardoublequoteopen}Member{\isacharparenleft}{\kern0pt}{\isadigit{0}}{\isacharcomma}{\kern0pt}{\isadigit{1}}{\isacharparenright}{\kern0pt}{\isachardoublequoteclose}\ {\isachardoublequoteopen}G{\isachardoublequoteclose}\ {\isachardoublequoteopen}{\isacharbrackleft}{\kern0pt}{\isasymsigma}{\isacharcomma}{\kern0pt}{\isasymchi}{\isacharbrackright}{\kern0pt}{\isachardoublequoteclose}\ {\isacharbrackright}{\kern0pt}\ nat{\isacharunderscore}{\kern0pt}simp{\isacharunderscore}{\kern0pt}union\isanewline
\ \ \ \ \ \ \ \ \isacommand{by}\isamarkupfalse%
\ auto\isanewline
\ \ \ \ \ \ \isacommand{moreover}\isamarkupfalse%
\ \isacommand{from}\isamarkupfalse%
\ {\isacartoucheopen}p{\isasymin}G{\isacartoucheclose}\ \isanewline
\ \ \ \ \ \ \isacommand{have}\isamarkupfalse%
\ {\isachardoublequoteopen}p{\isasymin}P{\isachardoublequoteclose}\isanewline
\ \ \ \ \ \ \ \ \isacommand{using}\isamarkupfalse%
\ generic\ \isacommand{unfolding}\isamarkupfalse%
\ M{\isacharunderscore}{\kern0pt}generic{\isacharunderscore}{\kern0pt}def\ filter{\isacharunderscore}{\kern0pt}def\ \isacommand{by}\isamarkupfalse%
\ blast\isanewline
\ \ \ \ \ \ \isacommand{ultimately}\isamarkupfalse%
\isanewline
\ \ \ \ \ \ \isacommand{have}\isamarkupfalse%
\ {\isachardoublequoteopen}{\isacharless}{\kern0pt}{\isasymsigma}{\isacharcomma}{\kern0pt}p{\isachargreater}{\kern0pt}{\isasymin}{\isacharquery}{\kern0pt}{\isasymtheta}{\isachardoublequoteclose}\isanewline
\ \ \ \ \ \ \ \ \isacommand{using}\isamarkupfalse%
\ {\isacartoucheopen}{\isasymsigma}{\isasymin}domain{\isacharparenleft}{\kern0pt}{\isasymtau}{\isacharparenright}{\kern0pt}{\isacartoucheclose}\ \isacommand{by}\isamarkupfalse%
\ simp\isanewline
\ \ \ \ \ \ \isacommand{with}\isamarkupfalse%
\ {\isacartoucheopen}val{\isacharparenleft}{\kern0pt}G{\isacharcomma}{\kern0pt}{\isasymsigma}{\isacharparenright}{\kern0pt}\ {\isacharequal}{\kern0pt}\ \ x{\isacartoucheclose}\ {\isacartoucheopen}p{\isasymin}G{\isacartoucheclose}\ \isanewline
\ \ \ \ \ \ \isacommand{show}\isamarkupfalse%
\ {\isachardoublequoteopen}x{\isasymin}val{\isacharparenleft}{\kern0pt}G{\isacharcomma}{\kern0pt}{\isacharquery}{\kern0pt}{\isasymtheta}{\isacharparenright}{\kern0pt}{\isachardoublequoteclose}\isanewline
\ \ \ \ \ \ \ \ \isacommand{using}\isamarkupfalse%
\ val{\isacharunderscore}{\kern0pt}of{\isacharunderscore}{\kern0pt}elem\ {\isacharbrackleft}{\kern0pt}of\ {\isacharunderscore}{\kern0pt}\ {\isacharunderscore}{\kern0pt}\ {\isachardoublequoteopen}{\isacharquery}{\kern0pt}{\isasymtheta}{\isachardoublequoteclose}{\isacharbrackright}{\kern0pt}\ \isacommand{by}\isamarkupfalse%
\ auto\isanewline
\ \ \ \ \isacommand{qed}\isamarkupfalse%
\isanewline
\ \ \ \ \isacommand{with}\isamarkupfalse%
\ {\isacartoucheopen}val{\isacharparenleft}{\kern0pt}G{\isacharcomma}{\kern0pt}{\isacharquery}{\kern0pt}{\isasymtheta}{\isacharparenright}{\kern0pt}\ {\isasymin}\ {\isacharquery}{\kern0pt}b{\isacartoucheclose}\ \isanewline
\ \ \ \ \isacommand{show}\isamarkupfalse%
\ {\isachardoublequoteopen}c{\isasymin}{\isacharquery}{\kern0pt}b{\isachardoublequoteclose}\ \isacommand{by}\isamarkupfalse%
\ simp\isanewline
\ \ \isacommand{qed}\isamarkupfalse%
\isanewline
\ \ \isacommand{then}\isamarkupfalse%
\ \isanewline
\ \ \isacommand{have}\isamarkupfalse%
\ {\isachardoublequoteopen}Pow{\isacharparenleft}{\kern0pt}a{\isacharparenright}{\kern0pt}\ {\isasyminter}\ M{\isacharbrackleft}{\kern0pt}G{\isacharbrackright}{\kern0pt}\ {\isacharequal}{\kern0pt}\ {\isacharbraceleft}{\kern0pt}x{\isasymin}{\isacharquery}{\kern0pt}b\ {\isachardot}{\kern0pt}\ x{\isasymsubseteq}a\ {\isacharampersand}{\kern0pt}\ x{\isasymin}M{\isacharbrackleft}{\kern0pt}G{\isacharbrackright}{\kern0pt}{\isacharbraceright}{\kern0pt}{\isachardoublequoteclose}\isanewline
\ \ \ \ \isacommand{by}\isamarkupfalse%
\ auto\isanewline
\ \ \isacommand{also}\isamarkupfalse%
\ \isacommand{from}\isamarkupfalse%
\ {\isacartoucheopen}a{\isasymin}M{\isacharbrackleft}{\kern0pt}G{\isacharbrackright}{\kern0pt}{\isacartoucheclose}\ \isanewline
\ \ \isacommand{have}\isamarkupfalse%
\ {\isachardoublequoteopen}\ {\isachardot}{\kern0pt}{\isachardot}{\kern0pt}{\isachardot}{\kern0pt}\ {\isacharequal}{\kern0pt}\ {\isacharbraceleft}{\kern0pt}x{\isasymin}{\isacharquery}{\kern0pt}b\ {\isachardot}{\kern0pt}\ sats{\isacharparenleft}{\kern0pt}M{\isacharbrackleft}{\kern0pt}G{\isacharbrackright}{\kern0pt}{\isacharcomma}{\kern0pt}subset{\isacharunderscore}{\kern0pt}fm{\isacharparenleft}{\kern0pt}{\isadigit{0}}{\isacharcomma}{\kern0pt}{\isadigit{1}}{\isacharparenright}{\kern0pt}{\isacharcomma}{\kern0pt}{\isacharbrackleft}{\kern0pt}x{\isacharcomma}{\kern0pt}a{\isacharbrackright}{\kern0pt}{\isacharparenright}{\kern0pt}\ {\isacharampersand}{\kern0pt}\ x{\isasymin}M{\isacharbrackleft}{\kern0pt}G{\isacharbrackright}{\kern0pt}{\isacharbraceright}{\kern0pt}{\isachardoublequoteclose}\isanewline
\ \ \ \ \isacommand{using}\isamarkupfalse%
\ Transset{\isacharunderscore}{\kern0pt}MG\ \isacommand{by}\isamarkupfalse%
\ force\isanewline
\ \ \isacommand{also}\isamarkupfalse%
\ \isanewline
\ \ \isacommand{have}\isamarkupfalse%
\ {\isachardoublequoteopen}\ {\isachardot}{\kern0pt}{\isachardot}{\kern0pt}{\isachardot}{\kern0pt}\ {\isacharequal}{\kern0pt}\ {\isacharbraceleft}{\kern0pt}x{\isasymin}{\isacharquery}{\kern0pt}b\ {\isachardot}{\kern0pt}\ sats{\isacharparenleft}{\kern0pt}M{\isacharbrackleft}{\kern0pt}G{\isacharbrackright}{\kern0pt}{\isacharcomma}{\kern0pt}subset{\isacharunderscore}{\kern0pt}fm{\isacharparenleft}{\kern0pt}{\isadigit{0}}{\isacharcomma}{\kern0pt}{\isadigit{1}}{\isacharparenright}{\kern0pt}{\isacharcomma}{\kern0pt}{\isacharbrackleft}{\kern0pt}x{\isacharcomma}{\kern0pt}a{\isacharbrackright}{\kern0pt}{\isacharparenright}{\kern0pt}{\isacharbraceright}{\kern0pt}\ {\isasyminter}\ M{\isacharbrackleft}{\kern0pt}G{\isacharbrackright}{\kern0pt}{\isachardoublequoteclose}\isanewline
\ \ \ \ \isacommand{by}\isamarkupfalse%
\ auto\isanewline
\ \ \isacommand{also}\isamarkupfalse%
\ \isacommand{from}\isamarkupfalse%
\ {\isacartoucheopen}{\isacharquery}{\kern0pt}b{\isasymin}M{\isacharbrackleft}{\kern0pt}G{\isacharbrackright}{\kern0pt}{\isacartoucheclose}\ \isanewline
\ \ \isacommand{have}\isamarkupfalse%
\ {\isachardoublequoteopen}\ {\isachardot}{\kern0pt}{\isachardot}{\kern0pt}{\isachardot}{\kern0pt}\ {\isacharequal}{\kern0pt}\ {\isacharbraceleft}{\kern0pt}x{\isasymin}{\isacharquery}{\kern0pt}b\ {\isachardot}{\kern0pt}\ sats{\isacharparenleft}{\kern0pt}M{\isacharbrackleft}{\kern0pt}G{\isacharbrackright}{\kern0pt}{\isacharcomma}{\kern0pt}subset{\isacharunderscore}{\kern0pt}fm{\isacharparenleft}{\kern0pt}{\isadigit{0}}{\isacharcomma}{\kern0pt}{\isadigit{1}}{\isacharparenright}{\kern0pt}{\isacharcomma}{\kern0pt}{\isacharbrackleft}{\kern0pt}x{\isacharcomma}{\kern0pt}a{\isacharbrackright}{\kern0pt}{\isacharparenright}{\kern0pt}{\isacharbraceright}{\kern0pt}{\isachardoublequoteclose}\isanewline
\ \ \ \ \isacommand{using}\isamarkupfalse%
\ Collect{\isacharunderscore}{\kern0pt}inter{\isacharunderscore}{\kern0pt}Transset\ Transset{\isacharunderscore}{\kern0pt}MG\isanewline
\ \ \ \ \isacommand{by}\isamarkupfalse%
\ simp\isanewline
\ \ \isacommand{also}\isamarkupfalse%
\ \isacommand{from}\isamarkupfalse%
\ {\isacartoucheopen}{\isacharquery}{\kern0pt}b{\isasymin}M{\isacharbrackleft}{\kern0pt}G{\isacharbrackright}{\kern0pt}{\isacartoucheclose}\ {\isacartoucheopen}a{\isasymin}M{\isacharbrackleft}{\kern0pt}G{\isacharbrackright}{\kern0pt}{\isacartoucheclose}\isanewline
\ \ \isacommand{have}\isamarkupfalse%
\ {\isachardoublequoteopen}\ {\isachardot}{\kern0pt}{\isachardot}{\kern0pt}{\isachardot}{\kern0pt}\ {\isasymin}\ M{\isacharbrackleft}{\kern0pt}G{\isacharbrackright}{\kern0pt}{\isachardoublequoteclose}\isanewline
\ \ \ \ \isacommand{using}\isamarkupfalse%
\ Collect{\isacharunderscore}{\kern0pt}sats{\isacharunderscore}{\kern0pt}in{\isacharunderscore}{\kern0pt}MG\ GenExtI\ nat{\isacharunderscore}{\kern0pt}simp{\isacharunderscore}{\kern0pt}union\ \isacommand{by}\isamarkupfalse%
\ simp\isanewline
\ \ \isacommand{finally}\isamarkupfalse%
\ \isacommand{show}\isamarkupfalse%
\ {\isacharquery}{\kern0pt}thesis\ \isacommand{{\isachardot}{\kern0pt}}\isamarkupfalse%
\isanewline
\isacommand{qed}\isamarkupfalse%
%
\endisatagproof
{\isafoldproof}%
%
\isadelimproof
\isanewline
%
\endisadelimproof
\isacommand{end}\isamarkupfalse%
\ \isanewline
\isanewline
\isanewline
\isacommand{context}\isamarkupfalse%
\ G{\isacharunderscore}{\kern0pt}generic\ \isakeyword{begin}\isanewline
\isanewline
\isacommand{interpretation}\isamarkupfalse%
\ mgtriv{\isacharcolon}{\kern0pt}\ M{\isacharunderscore}{\kern0pt}trivial\ {\isachardoublequoteopen}{\isacharhash}{\kern0pt}{\isacharhash}{\kern0pt}M{\isacharbrackleft}{\kern0pt}G{\isacharbrackright}{\kern0pt}{\isachardoublequoteclose}\isanewline
%
\isadelimproof
\ \ %
\endisadelimproof
%
\isatagproof
\isacommand{using}\isamarkupfalse%
\ generic\ Union{\isacharunderscore}{\kern0pt}MG\ pairing{\isacharunderscore}{\kern0pt}in{\isacharunderscore}{\kern0pt}MG\ zero{\isacharunderscore}{\kern0pt}in{\isacharunderscore}{\kern0pt}MG\ transitivity{\isacharunderscore}{\kern0pt}MG\isanewline
\ \ \isacommand{unfolding}\isamarkupfalse%
\ M{\isacharunderscore}{\kern0pt}trivial{\isacharunderscore}{\kern0pt}def\ M{\isacharunderscore}{\kern0pt}trans{\isacharunderscore}{\kern0pt}def\ M{\isacharunderscore}{\kern0pt}trivial{\isacharunderscore}{\kern0pt}axioms{\isacharunderscore}{\kern0pt}def\ \isacommand{by}\isamarkupfalse%
\ {\isacharparenleft}{\kern0pt}simp{\isacharsemicolon}{\kern0pt}\ blast{\isacharparenright}{\kern0pt}%
\endisatagproof
{\isafoldproof}%
%
\isadelimproof
\isanewline
%
\endisadelimproof
\isanewline
\isanewline
\isacommand{theorem}\isamarkupfalse%
\ power{\isacharunderscore}{\kern0pt}in{\isacharunderscore}{\kern0pt}MG\ {\isacharcolon}{\kern0pt}\ {\isachardoublequoteopen}power{\isacharunderscore}{\kern0pt}ax{\isacharparenleft}{\kern0pt}{\isacharhash}{\kern0pt}{\isacharhash}{\kern0pt}{\isacharparenleft}{\kern0pt}M{\isacharbrackleft}{\kern0pt}G{\isacharbrackright}{\kern0pt}{\isacharparenright}{\kern0pt}{\isacharparenright}{\kern0pt}{\isachardoublequoteclose}\isanewline
%
\isadelimproof
\ \ %
\endisadelimproof
%
\isatagproof
\isacommand{unfolding}\isamarkupfalse%
\ power{\isacharunderscore}{\kern0pt}ax{\isacharunderscore}{\kern0pt}def\isanewline
\isacommand{proof}\isamarkupfalse%
\ {\isacharparenleft}{\kern0pt}intro\ rallI{\isacharcomma}{\kern0pt}\ simp\ only{\isacharcolon}{\kern0pt}setclass{\isacharunderscore}{\kern0pt}iff\ rex{\isacharunderscore}{\kern0pt}setclass{\isacharunderscore}{\kern0pt}is{\isacharunderscore}{\kern0pt}bex{\isacharparenright}{\kern0pt}\isanewline
\ \ \isanewline
\ \ \isacommand{fix}\isamarkupfalse%
\ a\isanewline
\ \ \isacommand{assume}\isamarkupfalse%
\ {\isachardoublequoteopen}a\ {\isasymin}\ M{\isacharbrackleft}{\kern0pt}G{\isacharbrackright}{\kern0pt}{\isachardoublequoteclose}\isanewline
\ \ \isacommand{then}\isamarkupfalse%
\isanewline
\ \ \isacommand{have}\isamarkupfalse%
\ {\isachardoublequoteopen}{\isacharparenleft}{\kern0pt}{\isacharhash}{\kern0pt}{\isacharhash}{\kern0pt}M{\isacharbrackleft}{\kern0pt}G{\isacharbrackright}{\kern0pt}{\isacharparenright}{\kern0pt}{\isacharparenleft}{\kern0pt}a{\isacharparenright}{\kern0pt}{\isachardoublequoteclose}\ \isacommand{by}\isamarkupfalse%
\ simp\isanewline
\ \ \isacommand{have}\isamarkupfalse%
\ {\isachardoublequoteopen}{\isacharbraceleft}{\kern0pt}x{\isasymin}Pow{\isacharparenleft}{\kern0pt}a{\isacharparenright}{\kern0pt}\ {\isachardot}{\kern0pt}\ x\ {\isasymin}\ M{\isacharbrackleft}{\kern0pt}G{\isacharbrackright}{\kern0pt}{\isacharbraceright}{\kern0pt}\ {\isacharequal}{\kern0pt}\ Pow{\isacharparenleft}{\kern0pt}a{\isacharparenright}{\kern0pt}\ {\isasyminter}\ M{\isacharbrackleft}{\kern0pt}G{\isacharbrackright}{\kern0pt}{\isachardoublequoteclose}\isanewline
\ \ \ \ \isacommand{by}\isamarkupfalse%
\ auto\isanewline
\ \ \isacommand{also}\isamarkupfalse%
\ \isacommand{from}\isamarkupfalse%
\ {\isacartoucheopen}a{\isasymin}M{\isacharbrackleft}{\kern0pt}G{\isacharbrackright}{\kern0pt}{\isacartoucheclose}\ \isanewline
\ \ \isacommand{have}\isamarkupfalse%
\ {\isachardoublequoteopen}\ {\isachardot}{\kern0pt}{\isachardot}{\kern0pt}{\isachardot}{\kern0pt}\ {\isasymin}\ M{\isacharbrackleft}{\kern0pt}G{\isacharbrackright}{\kern0pt}{\isachardoublequoteclose}\isanewline
\ \ \ \ \isacommand{using}\isamarkupfalse%
\ Pow{\isacharunderscore}{\kern0pt}inter{\isacharunderscore}{\kern0pt}MG\ \isacommand{by}\isamarkupfalse%
\ simp\isanewline
\ \ \isacommand{finally}\isamarkupfalse%
\ \isanewline
\ \ \isacommand{have}\isamarkupfalse%
\ {\isachardoublequoteopen}{\isacharbraceleft}{\kern0pt}x{\isasymin}Pow{\isacharparenleft}{\kern0pt}a{\isacharparenright}{\kern0pt}\ {\isachardot}{\kern0pt}\ x\ {\isasymin}\ M{\isacharbrackleft}{\kern0pt}G{\isacharbrackright}{\kern0pt}{\isacharbraceright}{\kern0pt}\ {\isasymin}\ M{\isacharbrackleft}{\kern0pt}G{\isacharbrackright}{\kern0pt}{\isachardoublequoteclose}\ \isacommand{{\isachardot}{\kern0pt}}\isamarkupfalse%
\isanewline
\ \ \isacommand{moreover}\isamarkupfalse%
\ \isacommand{from}\isamarkupfalse%
\ {\isacartoucheopen}a{\isasymin}M{\isacharbrackleft}{\kern0pt}G{\isacharbrackright}{\kern0pt}{\isacartoucheclose}\ {\isacartoucheopen}{\isacharbraceleft}{\kern0pt}x{\isasymin}Pow{\isacharparenleft}{\kern0pt}a{\isacharparenright}{\kern0pt}\ {\isachardot}{\kern0pt}\ x\ {\isasymin}\ M{\isacharbrackleft}{\kern0pt}G{\isacharbrackright}{\kern0pt}{\isacharbraceright}{\kern0pt}\ {\isasymin}\ {\isacharunderscore}{\kern0pt}{\isacartoucheclose}\ \isanewline
\ \ \isacommand{have}\isamarkupfalse%
\ {\isachardoublequoteopen}powerset{\isacharparenleft}{\kern0pt}{\isacharhash}{\kern0pt}{\isacharhash}{\kern0pt}M{\isacharbrackleft}{\kern0pt}G{\isacharbrackright}{\kern0pt}{\isacharcomma}{\kern0pt}\ a{\isacharcomma}{\kern0pt}\ {\isacharbraceleft}{\kern0pt}x{\isasymin}Pow{\isacharparenleft}{\kern0pt}a{\isacharparenright}{\kern0pt}\ {\isachardot}{\kern0pt}\ x\ {\isasymin}\ M{\isacharbrackleft}{\kern0pt}G{\isacharbrackright}{\kern0pt}{\isacharbraceright}{\kern0pt}{\isacharparenright}{\kern0pt}{\isachardoublequoteclose}\isanewline
\ \ \ \ \isacommand{using}\isamarkupfalse%
\ mgtriv{\isachardot}{\kern0pt}powerset{\isacharunderscore}{\kern0pt}abs{\isacharbrackleft}{\kern0pt}OF\ {\isacartoucheopen}{\isacharparenleft}{\kern0pt}{\isacharhash}{\kern0pt}{\isacharhash}{\kern0pt}M{\isacharbrackleft}{\kern0pt}G{\isacharbrackright}{\kern0pt}{\isacharparenright}{\kern0pt}{\isacharparenleft}{\kern0pt}a{\isacharparenright}{\kern0pt}{\isacartoucheclose}{\isacharbrackright}{\kern0pt}\isanewline
\ \ \ \ \isacommand{by}\isamarkupfalse%
\ simp\isanewline
\ \ \isacommand{ultimately}\isamarkupfalse%
\ \isanewline
\ \ \isacommand{show}\isamarkupfalse%
\ {\isachardoublequoteopen}{\isasymexists}x{\isasymin}M{\isacharbrackleft}{\kern0pt}G{\isacharbrackright}{\kern0pt}\ {\isachardot}{\kern0pt}\ powerset{\isacharparenleft}{\kern0pt}{\isacharhash}{\kern0pt}{\isacharhash}{\kern0pt}M{\isacharbrackleft}{\kern0pt}G{\isacharbrackright}{\kern0pt}{\isacharcomma}{\kern0pt}\ a{\isacharcomma}{\kern0pt}\ x{\isacharparenright}{\kern0pt}{\isachardoublequoteclose}\isanewline
\ \ \ \ \isacommand{by}\isamarkupfalse%
\ auto\isanewline
\isacommand{qed}\isamarkupfalse%
%
\endisatagproof
{\isafoldproof}%
%
\isadelimproof
\isanewline
%
\endisadelimproof
\isacommand{end}\isamarkupfalse%
\ \isanewline
%
\isadelimtheory
%
\endisadelimtheory
%
\isatagtheory
\isacommand{end}\isamarkupfalse%
%
\endisatagtheory
{\isafoldtheory}%
%
\isadelimtheory
%
\endisadelimtheory
%
\end{isabellebody}%
\endinput
%:%file=~/source/repos/ZF-notAC/code/Forcing/Powerset_Axiom.thy%:%
%:%11=1%:%
%:%27=2%:%
%:%28=2%:%
%:%29=3%:%
%:%30=4%:%
%:%35=4%:%
%:%40=5%:%
%:%45=6%:%
%:%46=6%:%
%:%51=6%:%
%:%54=7%:%
%:%55=8%:%
%:%56=8%:%
%:%57=9%:%
%:%58=10%:%
%:%59=11%:%
%:%60=12%:%
%:%63=13%:%
%:%67=13%:%
%:%68=13%:%
%:%69=13%:%
%:%70=14%:%
%:%71=14%:%
%:%76=14%:%
%:%79=15%:%
%:%80=16%:%
%:%81=16%:%
%:%82=17%:%
%:%83=18%:%
%:%84=18%:%
%:%85=19%:%
%:%86=20%:%
%:%93=21%:%
%:%94=21%:%
%:%95=22%:%
%:%96=22%:%
%:%97=22%:%
%:%98=23%:%
%:%99=24%:%
%:%100=24%:%
%:%101=24%:%
%:%102=25%:%
%:%103=25%:%
%:%104=25%:%
%:%105=26%:%
%:%106=26%:%
%:%107=27%:%
%:%108=27%:%
%:%109=27%:%
%:%110=28%:%
%:%111=28%:%
%:%112=28%:%
%:%113=29%:%
%:%114=29%:%
%:%115=30%:%
%:%116=30%:%
%:%117=30%:%
%:%118=31%:%
%:%119=31%:%
%:%120=32%:%
%:%121=32%:%
%:%122=33%:%
%:%123=33%:%
%:%124=33%:%
%:%125=34%:%
%:%131=34%:%
%:%134=35%:%
%:%135=36%:%
%:%136=36%:%
%:%137=37%:%
%:%138=38%:%
%:%139=39%:%
%:%140=40%:%
%:%141=41%:%
%:%142=42%:%
%:%149=43%:%
%:%150=43%:%
%:%151=44%:%
%:%152=44%:%
%:%153=44%:%
%:%154=45%:%
%:%155=45%:%
%:%156=46%:%
%:%157=46%:%
%:%158=46%:%
%:%159=47%:%
%:%160=48%:%
%:%161=48%:%
%:%162=48%:%
%:%163=49%:%
%:%164=49%:%
%:%165=50%:%
%:%166=50%:%
%:%167=51%:%
%:%168=51%:%
%:%169=52%:%
%:%170=52%:%
%:%171=53%:%
%:%172=53%:%
%:%173=54%:%
%:%174=54%:%
%:%175=55%:%
%:%176=55%:%
%:%177=56%:%
%:%178=56%:%
%:%179=57%:%
%:%180=57%:%
%:%181=57%:%
%:%182=58%:%
%:%183=58%:%
%:%184=59%:%
%:%185=59%:%
%:%186=60%:%
%:%187=60%:%
%:%188=61%:%
%:%189=61%:%
%:%190=62%:%
%:%191=62%:%
%:%192=63%:%
%:%193=63%:%
%:%194=63%:%
%:%195=64%:%
%:%196=64%:%
%:%197=65%:%
%:%198=65%:%
%:%199=65%:%
%:%200=66%:%
%:%201=66%:%
%:%202=67%:%
%:%203=67%:%
%:%204=68%:%
%:%205=68%:%
%:%206=69%:%
%:%207=69%:%
%:%208=70%:%
%:%209=70%:%
%:%210=71%:%
%:%211=72%:%
%:%212=72%:%
%:%213=73%:%
%:%214=73%:%
%:%215=74%:%
%:%216=74%:%
%:%217=74%:%
%:%218=75%:%
%:%219=75%:%
%:%220=75%:%
%:%221=76%:%
%:%222=76%:%
%:%223=77%:%
%:%224=77%:%
%:%225=78%:%
%:%226=78%:%
%:%227=78%:%
%:%228=79%:%
%:%229=79%:%
%:%230=80%:%
%:%231=81%:%
%:%232=81%:%
%:%233=82%:%
%:%234=82%:%
%:%235=82%:%
%:%236=83%:%
%:%237=83%:%
%:%238=84%:%
%:%239=84%:%
%:%240=85%:%
%:%241=85%:%
%:%242=86%:%
%:%243=86%:%
%:%244=87%:%
%:%245=87%:%
%:%246=87%:%
%:%247=88%:%
%:%248=89%:%
%:%249=89%:%
%:%250=90%:%
%:%251=90%:%
%:%252=90%:%
%:%253=90%:%
%:%254=91%:%
%:%255=92%:%
%:%256=92%:%
%:%257=93%:%
%:%258=93%:%
%:%259=94%:%
%:%260=94%:%
%:%261=95%:%
%:%262=95%:%
%:%263=95%:%
%:%264=96%:%
%:%265=96%:%
%:%266=96%:%
%:%267=97%:%
%:%268=97%:%
%:%269=98%:%
%:%270=98%:%
%:%271=99%:%
%:%272=99%:%
%:%273=99%:%
%:%274=100%:%
%:%275=100%:%
%:%276=101%:%
%:%277=101%:%
%:%278=102%:%
%:%279=102%:%
%:%280=103%:%
%:%281=103%:%
%:%282=104%:%
%:%283=104%:%
%:%284=105%:%
%:%285=105%:%
%:%286=105%:%
%:%287=105%:%
%:%288=106%:%
%:%294=106%:%
%:%297=107%:%
%:%298=108%:%
%:%299=108%:%
%:%300=109%:%
%:%301=110%:%
%:%302=111%:%
%:%303=112%:%
%:%310=113%:%
%:%311=113%:%
%:%312=114%:%
%:%313=114%:%
%:%314=114%:%
%:%315=115%:%
%:%316=116%:%
%:%317=116%:%
%:%318=116%:%
%:%319=117%:%
%:%320=117%:%
%:%321=118%:%
%:%322=118%:%
%:%323=119%:%
%:%324=119%:%
%:%325=120%:%
%:%326=120%:%
%:%327=121%:%
%:%328=121%:%
%:%329=122%:%
%:%330=122%:%
%:%331=123%:%
%:%332=123%:%
%:%333=124%:%
%:%334=124%:%
%:%335=125%:%
%:%336=125%:%
%:%337=125%:%
%:%338=126%:%
%:%339=127%:%
%:%340=127%:%
%:%341=127%:%
%:%342=128%:%
%:%343=128%:%
%:%344=128%:%
%:%345=129%:%
%:%346=129%:%
%:%347=130%:%
%:%348=130%:%
%:%349=130%:%
%:%350=131%:%
%:%351=131%:%
%:%352=132%:%
%:%353=132%:%
%:%354=133%:%
%:%355=133%:%
%:%356=134%:%
%:%357=134%:%
%:%358=135%:%
%:%359=135%:%
%:%360=136%:%
%:%361=136%:%
%:%362=137%:%
%:%363=137%:%
%:%364=138%:%
%:%365=138%:%
%:%366=139%:%
%:%367=139%:%
%:%368=139%:%
%:%369=139%:%
%:%370=140%:%
%:%371=140%:%
%:%372=141%:%
%:%373=141%:%
%:%374=142%:%
%:%375=143%:%
%:%376=143%:%
%:%377=144%:%
%:%378=145%:%
%:%379=145%:%
%:%380=146%:%
%:%381=146%:%
%:%382=147%:%
%:%383=147%:%
%:%384=148%:%
%:%385=148%:%
%:%386=149%:%
%:%387=149%:%
%:%388=150%:%
%:%389=150%:%
%:%390=151%:%
%:%391=151%:%
%:%392=151%:%
%:%393=152%:%
%:%394=152%:%
%:%395=153%:%
%:%396=153%:%
%:%397=154%:%
%:%398=154%:%
%:%399=155%:%
%:%400=155%:%
%:%401=156%:%
%:%402=156%:%
%:%403=156%:%
%:%404=157%:%
%:%405=158%:%
%:%406=158%:%
%:%407=158%:%
%:%408=159%:%
%:%409=159%:%
%:%410=160%:%
%:%411=160%:%
%:%412=161%:%
%:%413=161%:%
%:%414=161%:%
%:%415=162%:%
%:%416=162%:%
%:%417=163%:%
%:%418=163%:%
%:%419=164%:%
%:%420=164%:%
%:%421=165%:%
%:%422=165%:%
%:%423=166%:%
%:%424=166%:%
%:%425=167%:%
%:%426=167%:%
%:%427=168%:%
%:%428=168%:%
%:%429=169%:%
%:%430=169%:%
%:%431=170%:%
%:%432=170%:%
%:%433=171%:%
%:%434=171%:%
%:%435=172%:%
%:%436=172%:%
%:%437=173%:%
%:%438=173%:%
%:%439=174%:%
%:%440=174%:%
%:%441=175%:%
%:%442=175%:%
%:%443=176%:%
%:%444=176%:%
%:%445=177%:%
%:%446=177%:%
%:%447=177%:%
%:%448=178%:%
%:%449=179%:%
%:%450=179%:%
%:%451=180%:%
%:%452=180%:%
%:%453=181%:%
%:%454=181%:%
%:%455=181%:%
%:%456=182%:%
%:%457=182%:%
%:%458=183%:%
%:%459=183%:%
%:%460=183%:%
%:%461=184%:%
%:%462=184%:%
%:%463=184%:%
%:%464=185%:%
%:%465=185%:%
%:%466=186%:%
%:%467=186%:%
%:%468=187%:%
%:%469=187%:%
%:%470=188%:%
%:%471=188%:%
%:%472=189%:%
%:%473=189%:%
%:%474=190%:%
%:%475=190%:%
%:%476=191%:%
%:%477=191%:%
%:%478=192%:%
%:%479=192%:%
%:%480=193%:%
%:%481=193%:%
%:%482=194%:%
%:%483=194%:%
%:%484=195%:%
%:%485=195%:%
%:%486=195%:%
%:%487=196%:%
%:%488=196%:%
%:%489=197%:%
%:%490=197%:%
%:%491=198%:%
%:%492=198%:%
%:%493=198%:%
%:%494=199%:%
%:%495=199%:%
%:%496=200%:%
%:%497=200%:%
%:%498=201%:%
%:%499=201%:%
%:%500=202%:%
%:%501=202%:%
%:%502=203%:%
%:%503=203%:%
%:%504=204%:%
%:%505=204%:%
%:%506=205%:%
%:%507=205%:%
%:%508=206%:%
%:%509=206%:%
%:%510=207%:%
%:%511=207%:%
%:%512=208%:%
%:%513=209%:%
%:%514=209%:%
%:%515=210%:%
%:%516=210%:%
%:%517=211%:%
%:%518=211%:%
%:%519=211%:%
%:%520=212%:%
%:%521=212%:%
%:%522=212%:%
%:%523=213%:%
%:%524=213%:%
%:%525=214%:%
%:%526=214%:%
%:%527=215%:%
%:%528=215%:%
%:%529=215%:%
%:%530=216%:%
%:%531=216%:%
%:%532=216%:%
%:%533=217%:%
%:%534=217%:%
%:%535=218%:%
%:%536=218%:%
%:%537=219%:%
%:%538=219%:%
%:%539=220%:%
%:%540=220%:%
%:%541=220%:%
%:%542=221%:%
%:%543=221%:%
%:%544=222%:%
%:%545=222%:%
%:%546=223%:%
%:%547=223%:%
%:%548=224%:%
%:%549=224%:%
%:%550=225%:%
%:%551=225%:%
%:%552=226%:%
%:%553=226%:%
%:%554=227%:%
%:%555=227%:%
%:%556=228%:%
%:%557=228%:%
%:%558=229%:%
%:%559=229%:%
%:%560=229%:%
%:%561=230%:%
%:%562=230%:%
%:%563=231%:%
%:%564=231%:%
%:%565=231%:%
%:%566=232%:%
%:%567=232%:%
%:%568=233%:%
%:%569=233%:%
%:%570=234%:%
%:%571=234%:%
%:%572=235%:%
%:%573=235%:%
%:%574=236%:%
%:%575=236%:%
%:%576=236%:%
%:%577=237%:%
%:%578=237%:%
%:%579=238%:%
%:%580=238%:%
%:%581=238%:%
%:%582=238%:%
%:%583=239%:%
%:%584=239%:%
%:%585=240%:%
%:%586=240%:%
%:%587=241%:%
%:%588=241%:%
%:%589=241%:%
%:%590=242%:%
%:%591=242%:%
%:%592=243%:%
%:%593=243%:%
%:%594=244%:%
%:%595=244%:%
%:%596=244%:%
%:%597=245%:%
%:%598=245%:%
%:%599=246%:%
%:%600=246%:%
%:%601=247%:%
%:%602=247%:%
%:%603=247%:%
%:%604=248%:%
%:%605=248%:%
%:%606=249%:%
%:%607=249%:%
%:%608=250%:%
%:%609=250%:%
%:%610=251%:%
%:%611=251%:%
%:%612=252%:%
%:%613=252%:%
%:%614=252%:%
%:%615=253%:%
%:%616=253%:%
%:%617=254%:%
%:%618=254%:%
%:%619=254%:%
%:%620=255%:%
%:%621=255%:%
%:%622=256%:%
%:%623=256%:%
%:%624=257%:%
%:%625=257%:%
%:%626=258%:%
%:%627=258%:%
%:%628=258%:%
%:%629=259%:%
%:%630=259%:%
%:%631=260%:%
%:%632=260%:%
%:%633=261%:%
%:%634=261%:%
%:%635=262%:%
%:%636=262%:%
%:%637=262%:%
%:%638=263%:%
%:%639=263%:%
%:%640=264%:%
%:%641=264%:%
%:%642=264%:%
%:%643=265%:%
%:%644=265%:%
%:%645=265%:%
%:%646=265%:%
%:%647=266%:%
%:%653=266%:%
%:%656=267%:%
%:%657=267%:%
%:%658=268%:%
%:%659=269%:%
%:%660=270%:%
%:%661=270%:%
%:%662=271%:%
%:%663=272%:%
%:%664=272%:%
%:%667=273%:%
%:%671=273%:%
%:%672=273%:%
%:%673=274%:%
%:%674=274%:%
%:%675=274%:%
%:%680=274%:%
%:%683=275%:%
%:%684=276%:%
%:%685=277%:%
%:%686=277%:%
%:%689=278%:%
%:%693=278%:%
%:%694=278%:%
%:%695=279%:%
%:%696=279%:%
%:%697=280%:%
%:%697=282%:%
%:%698=283%:%
%:%699=283%:%
%:%700=284%:%
%:%701=284%:%
%:%702=285%:%
%:%703=285%:%
%:%704=286%:%
%:%705=286%:%
%:%706=286%:%
%:%707=287%:%
%:%708=287%:%
%:%709=288%:%
%:%710=288%:%
%:%711=289%:%
%:%712=289%:%
%:%713=289%:%
%:%714=290%:%
%:%715=290%:%
%:%716=291%:%
%:%717=291%:%
%:%718=291%:%
%:%719=292%:%
%:%720=292%:%
%:%721=293%:%
%:%722=293%:%
%:%723=293%:%
%:%724=294%:%
%:%725=294%:%
%:%726=294%:%
%:%727=295%:%
%:%728=295%:%
%:%729=296%:%
%:%730=296%:%
%:%731=297%:%
%:%732=297%:%
%:%733=298%:%
%:%734=298%:%
%:%735=299%:%
%:%736=299%:%
%:%737=300%:%
%:%738=300%:%
%:%739=301%:%
%:%745=301%:%
%:%748=302%:%
%:%749=302%:%
%:%756=303%:%

%
\begin{isabellebody}%
\setisabellecontext{Extensionality{\isacharunderscore}{\kern0pt}Axiom}%
%
\isadelimdocument
%
\endisadelimdocument
%
\isatagdocument
%
\isamarkupsection{The Axiom of Extensionality in $M[G]$%
}
\isamarkuptrue%
%
\endisatagdocument
{\isafolddocument}%
%
\isadelimdocument
%
\endisadelimdocument
%
\isadelimtheory
%
\endisadelimtheory
%
\isatagtheory
\isacommand{theory}\isamarkupfalse%
\ Extensionality{\isacharunderscore}{\kern0pt}Axiom\isanewline
\isakeyword{imports}\isanewline
\ \ Names\isanewline
\isakeyword{begin}%
\endisatagtheory
{\isafoldtheory}%
%
\isadelimtheory
\isanewline
%
\endisadelimtheory
\isanewline
\isacommand{context}\isamarkupfalse%
\ forcing{\isacharunderscore}{\kern0pt}data\isanewline
\isakeyword{begin}\isanewline
\ \ \isanewline
\isacommand{lemma}\isamarkupfalse%
\ extensionality{\isacharunderscore}{\kern0pt}in{\isacharunderscore}{\kern0pt}MG\ {\isacharcolon}{\kern0pt}\ {\isachardoublequoteopen}extensionality{\isacharparenleft}{\kern0pt}{\isacharhash}{\kern0pt}{\isacharhash}{\kern0pt}{\isacharparenleft}{\kern0pt}M{\isacharbrackleft}{\kern0pt}G{\isacharbrackright}{\kern0pt}{\isacharparenright}{\kern0pt}{\isacharparenright}{\kern0pt}{\isachardoublequoteclose}\isanewline
%
\isadelimproof
%
\endisadelimproof
%
\isatagproof
\isacommand{proof}\isamarkupfalse%
\ {\isacharminus}{\kern0pt}\isanewline
\ \ \isacommand{{\isacharbraceleft}{\kern0pt}}\isamarkupfalse%
\isanewline
\ \ \ \ \isacommand{fix}\isamarkupfalse%
\ x\ y\ z\isanewline
\ \ \ \ \isacommand{assume}\isamarkupfalse%
\ \isanewline
\ \ \ \ \ \ asms{\isacharcolon}{\kern0pt}\ {\isachardoublequoteopen}x{\isasymin}M{\isacharbrackleft}{\kern0pt}G{\isacharbrackright}{\kern0pt}{\isachardoublequoteclose}\ {\isachardoublequoteopen}y{\isasymin}M{\isacharbrackleft}{\kern0pt}G{\isacharbrackright}{\kern0pt}{\isachardoublequoteclose}\ {\isachardoublequoteopen}{\isacharparenleft}{\kern0pt}{\isasymforall}w{\isasymin}M{\isacharbrackleft}{\kern0pt}G{\isacharbrackright}{\kern0pt}\ {\isachardot}{\kern0pt}\ w\ {\isasymin}\ x\ {\isasymlongleftrightarrow}\ w\ {\isasymin}\ y{\isacharparenright}{\kern0pt}{\isachardoublequoteclose}\isanewline
\ \ \ \ \isacommand{from}\isamarkupfalse%
\ {\isacartoucheopen}x{\isasymin}M{\isacharbrackleft}{\kern0pt}G{\isacharbrackright}{\kern0pt}{\isacartoucheclose}\ \isacommand{have}\isamarkupfalse%
\isanewline
\ \ \ \ \ \ {\isachardoublequoteopen}z{\isasymin}x\ {\isasymlongleftrightarrow}\ z{\isasymin}M{\isacharbrackleft}{\kern0pt}G{\isacharbrackright}{\kern0pt}\ {\isasymand}\ z{\isasymin}x{\isachardoublequoteclose}\isanewline
\ \ \ \ \ \ \isacommand{using}\isamarkupfalse%
\ transitivity{\isacharunderscore}{\kern0pt}MG\ \isacommand{by}\isamarkupfalse%
\ auto\isanewline
\ \ \ \ \isacommand{also}\isamarkupfalse%
\ \isacommand{have}\isamarkupfalse%
\isanewline
\ \ \ \ \ \ {\isachardoublequoteopen}{\isachardot}{\kern0pt}{\isachardot}{\kern0pt}{\isachardot}{\kern0pt}\ {\isasymlongleftrightarrow}\ z{\isasymin}y{\isachardoublequoteclose}\isanewline
\ \ \ \ \ \ \isacommand{using}\isamarkupfalse%
\ asms\ transitivity{\isacharunderscore}{\kern0pt}MG\ \isacommand{by}\isamarkupfalse%
\ auto\isanewline
\ \ \ \ \isacommand{finally}\isamarkupfalse%
\ \isacommand{have}\isamarkupfalse%
\isanewline
\ \ \ \ \ \ {\isachardoublequoteopen}z{\isasymin}x\ {\isasymlongleftrightarrow}\ z{\isasymin}y{\isachardoublequoteclose}\ \isacommand{{\isachardot}{\kern0pt}}\isamarkupfalse%
\isanewline
\ \ \isacommand{{\isacharbraceright}{\kern0pt}}\isamarkupfalse%
\isanewline
\ \ \isacommand{then}\isamarkupfalse%
\ \isacommand{have}\isamarkupfalse%
\isanewline
\ \ \ \ {\isachardoublequoteopen}{\isasymforall}x{\isasymin}M{\isacharbrackleft}{\kern0pt}G{\isacharbrackright}{\kern0pt}\ {\isachardot}{\kern0pt}\ {\isasymforall}y{\isasymin}M{\isacharbrackleft}{\kern0pt}G{\isacharbrackright}{\kern0pt}\ {\isachardot}{\kern0pt}\ {\isacharparenleft}{\kern0pt}{\isasymforall}z{\isasymin}M{\isacharbrackleft}{\kern0pt}G{\isacharbrackright}{\kern0pt}\ {\isachardot}{\kern0pt}\ z\ {\isasymin}\ x\ {\isasymlongleftrightarrow}\ z\ {\isasymin}\ y{\isacharparenright}{\kern0pt}\ {\isasymlongrightarrow}\ x\ {\isacharequal}{\kern0pt}\ y{\isachardoublequoteclose}\isanewline
\ \ \ \ \isacommand{by}\isamarkupfalse%
\ blast\isanewline
\ \ \isacommand{then}\isamarkupfalse%
\ \isacommand{show}\isamarkupfalse%
\ {\isacharquery}{\kern0pt}thesis\ \isacommand{unfolding}\isamarkupfalse%
\ extensionality{\isacharunderscore}{\kern0pt}def\ \isacommand{by}\isamarkupfalse%
\ simp\isanewline
\isacommand{qed}\isamarkupfalse%
%
\endisatagproof
{\isafoldproof}%
%
\isadelimproof
\isanewline
%
\endisadelimproof
\ \isanewline
\isacommand{end}\isamarkupfalse%
\ \ \isanewline
%
\isadelimtheory
%
\endisadelimtheory
%
\isatagtheory
\isacommand{end}\isamarkupfalse%
%
\endisatagtheory
{\isafoldtheory}%
%
\isadelimtheory
%
\endisadelimtheory
%
\end{isabellebody}%
\endinput
%:%file=~/source/repos/ZF-notAC/code/Forcing/Extensionality_Axiom.thy%:%
%:%11=1%:%
%:%27=2%:%
%:%28=2%:%
%:%29=3%:%
%:%30=4%:%
%:%31=5%:%
%:%36=5%:%
%:%39=6%:%
%:%40=7%:%
%:%41=7%:%
%:%42=8%:%
%:%43=9%:%
%:%44=10%:%
%:%45=10%:%
%:%52=11%:%
%:%53=11%:%
%:%54=12%:%
%:%55=12%:%
%:%56=13%:%
%:%57=13%:%
%:%58=14%:%
%:%59=14%:%
%:%60=15%:%
%:%61=16%:%
%:%62=16%:%
%:%63=16%:%
%:%64=17%:%
%:%65=18%:%
%:%66=18%:%
%:%67=18%:%
%:%68=19%:%
%:%69=19%:%
%:%70=19%:%
%:%71=20%:%
%:%72=21%:%
%:%73=21%:%
%:%74=21%:%
%:%75=22%:%
%:%76=22%:%
%:%77=22%:%
%:%78=23%:%
%:%79=23%:%
%:%80=24%:%
%:%81=24%:%
%:%82=25%:%
%:%83=25%:%
%:%84=25%:%
%:%85=26%:%
%:%86=27%:%
%:%87=27%:%
%:%88=28%:%
%:%89=28%:%
%:%90=28%:%
%:%91=28%:%
%:%92=28%:%
%:%93=29%:%
%:%99=29%:%
%:%102=30%:%
%:%103=31%:%
%:%104=31%:%
%:%111=32%:%

%
\begin{isabellebody}%
\setisabellecontext{Foundation{\isacharunderscore}{\kern0pt}Axiom}%
%
\isadelimdocument
%
\endisadelimdocument
%
\isatagdocument
%
\isamarkupsection{The Axiom of Foundation in $M[G]$%
}
\isamarkuptrue%
%
\endisatagdocument
{\isafolddocument}%
%
\isadelimdocument
%
\endisadelimdocument
%
\isadelimtheory
%
\endisadelimtheory
%
\isatagtheory
\isacommand{theory}\isamarkupfalse%
\ Foundation{\isacharunderscore}{\kern0pt}Axiom\isanewline
\isakeyword{imports}\isanewline
\ \ Names\isanewline
\isakeyword{begin}%
\endisatagtheory
{\isafoldtheory}%
%
\isadelimtheory
\isanewline
%
\endisadelimtheory
\isanewline
\isacommand{context}\isamarkupfalse%
\ forcing{\isacharunderscore}{\kern0pt}data\isanewline
\isakeyword{begin}\isanewline
\ \ \isanewline
\ \ \isanewline
\isacommand{lemma}\isamarkupfalse%
\ foundation{\isacharunderscore}{\kern0pt}in{\isacharunderscore}{\kern0pt}MG\ {\isacharcolon}{\kern0pt}\ {\isachardoublequoteopen}foundation{\isacharunderscore}{\kern0pt}ax{\isacharparenleft}{\kern0pt}{\isacharhash}{\kern0pt}{\isacharhash}{\kern0pt}{\isacharparenleft}{\kern0pt}M{\isacharbrackleft}{\kern0pt}G{\isacharbrackright}{\kern0pt}{\isacharparenright}{\kern0pt}{\isacharparenright}{\kern0pt}{\isachardoublequoteclose}\isanewline
%
\isadelimproof
\ \ %
\endisadelimproof
%
\isatagproof
\isacommand{unfolding}\isamarkupfalse%
\ foundation{\isacharunderscore}{\kern0pt}ax{\isacharunderscore}{\kern0pt}def\isanewline
\ \ \isacommand{by}\isamarkupfalse%
\ {\isacharparenleft}{\kern0pt}rule\ rallI{\isacharcomma}{\kern0pt}\ cut{\isacharunderscore}{\kern0pt}tac\ A{\isacharequal}{\kern0pt}x\ \isakeyword{in}\ foundation{\isacharcomma}{\kern0pt}\ auto\ intro{\isacharcolon}{\kern0pt}\ transitivity{\isacharunderscore}{\kern0pt}MG{\isacharparenright}{\kern0pt}%
\endisatagproof
{\isafoldproof}%
%
\isadelimproof
\isanewline
%
\endisadelimproof
\isanewline
\isanewline
\isacommand{lemma}\isamarkupfalse%
\ {\isachardoublequoteopen}foundation{\isacharunderscore}{\kern0pt}ax{\isacharparenleft}{\kern0pt}{\isacharhash}{\kern0pt}{\isacharhash}{\kern0pt}{\isacharparenleft}{\kern0pt}M{\isacharbrackleft}{\kern0pt}G{\isacharbrackright}{\kern0pt}{\isacharparenright}{\kern0pt}{\isacharparenright}{\kern0pt}{\isachardoublequoteclose}\isanewline
%
\isadelimproof
%
\endisadelimproof
%
\isatagproof
\isacommand{proof}\isamarkupfalse%
\ {\isacharminus}{\kern0pt}\isanewline
\ \ \isacommand{{\isacharbraceleft}{\kern0pt}}\isamarkupfalse%
\ \ \ \isanewline
\ \ \ \ \isacommand{fix}\isamarkupfalse%
\ x\ \isanewline
\ \ \ \ \isacommand{assume}\isamarkupfalse%
\ {\isachardoublequoteopen}x{\isasymin}M{\isacharbrackleft}{\kern0pt}G{\isacharbrackright}{\kern0pt}{\isachardoublequoteclose}\ {\isachardoublequoteopen}{\isasymexists}y{\isasymin}M{\isacharbrackleft}{\kern0pt}G{\isacharbrackright}{\kern0pt}\ {\isachardot}{\kern0pt}\ y{\isasymin}x{\isachardoublequoteclose}\isanewline
\ \ \ \ \isacommand{then}\isamarkupfalse%
\ \isanewline
\ \ \ \ \isacommand{have}\isamarkupfalse%
\ {\isachardoublequoteopen}{\isasymexists}y{\isasymin}M{\isacharbrackleft}{\kern0pt}G{\isacharbrackright}{\kern0pt}\ {\isachardot}{\kern0pt}\ y{\isasymin}x{\isasyminter}M{\isacharbrackleft}{\kern0pt}G{\isacharbrackright}{\kern0pt}{\isachardoublequoteclose}\ \isacommand{by}\isamarkupfalse%
\ simp\isanewline
\ \ \ \ \isacommand{then}\isamarkupfalse%
\ \isanewline
\ \ \ \ \isacommand{obtain}\isamarkupfalse%
\ y\ \isakeyword{where}\ {\isachardoublequoteopen}y{\isasymin}x{\isasyminter}M{\isacharbrackleft}{\kern0pt}G{\isacharbrackright}{\kern0pt}{\isachardoublequoteclose}\ {\isachardoublequoteopen}{\isasymforall}z{\isasymin}y{\isachardot}{\kern0pt}\ z\ {\isasymnotin}\ x{\isasyminter}M{\isacharbrackleft}{\kern0pt}G{\isacharbrackright}{\kern0pt}{\isachardoublequoteclose}\ \isanewline
\ \ \ \ \ \ \isacommand{using}\isamarkupfalse%
\ foundation{\isacharbrackleft}{\kern0pt}of\ {\isachardoublequoteopen}x{\isasyminter}M{\isacharbrackleft}{\kern0pt}G{\isacharbrackright}{\kern0pt}{\isachardoublequoteclose}{\isacharbrackright}{\kern0pt}\ \ \isacommand{by}\isamarkupfalse%
\ blast\isanewline
\ \ \ \ \isacommand{then}\isamarkupfalse%
\ \isanewline
\ \ \ \ \isacommand{have}\isamarkupfalse%
\ {\isachardoublequoteopen}{\isasymexists}y{\isasymin}M{\isacharbrackleft}{\kern0pt}G{\isacharbrackright}{\kern0pt}\ {\isachardot}{\kern0pt}\ y\ {\isasymin}\ x\ {\isasymand}\ {\isacharparenleft}{\kern0pt}{\isasymforall}z{\isasymin}M{\isacharbrackleft}{\kern0pt}G{\isacharbrackright}{\kern0pt}\ {\isachardot}{\kern0pt}\ z\ {\isasymnotin}\ x\ {\isasymor}\ z\ {\isasymnotin}\ y{\isacharparenright}{\kern0pt}{\isachardoublequoteclose}\isacommand{by}\isamarkupfalse%
\ auto\isanewline
\ \ \isacommand{{\isacharbraceright}{\kern0pt}}\isamarkupfalse%
\isanewline
\ \ \isacommand{then}\isamarkupfalse%
\ \isacommand{show}\isamarkupfalse%
\ {\isacharquery}{\kern0pt}thesis\isanewline
\ \ \ \ \isacommand{unfolding}\isamarkupfalse%
\ foundation{\isacharunderscore}{\kern0pt}ax{\isacharunderscore}{\kern0pt}def\ \isacommand{by}\isamarkupfalse%
\ auto\isanewline
\isacommand{qed}\isamarkupfalse%
%
\endisatagproof
{\isafoldproof}%
%
\isadelimproof
\isanewline
%
\endisadelimproof
\ \ \ \ \isanewline
\isacommand{end}\isamarkupfalse%
\ \ \isanewline
%
\isadelimtheory
%
\endisadelimtheory
%
\isatagtheory
\isacommand{end}\isamarkupfalse%
%
\endisatagtheory
{\isafoldtheory}%
%
\isadelimtheory
%
\endisadelimtheory
%
\end{isabellebody}%
\endinput
%:%file=~/source/repos/ZF-notAC/code/Forcing/Foundation_Axiom.thy%:%
%:%11=1%:%
%:%27=2%:%
%:%28=2%:%
%:%29=3%:%
%:%30=4%:%
%:%31=5%:%
%:%36=5%:%
%:%39=6%:%
%:%40=7%:%
%:%41=7%:%
%:%42=8%:%
%:%43=9%:%
%:%44=10%:%
%:%45=11%:%
%:%46=11%:%
%:%49=12%:%
%:%53=12%:%
%:%54=12%:%
%:%55=13%:%
%:%56=13%:%
%:%61=13%:%
%:%64=14%:%
%:%65=16%:%
%:%66=17%:%
%:%67=17%:%
%:%74=18%:%
%:%75=18%:%
%:%76=19%:%
%:%77=19%:%
%:%78=20%:%
%:%79=20%:%
%:%80=21%:%
%:%81=21%:%
%:%82=22%:%
%:%83=22%:%
%:%84=23%:%
%:%85=23%:%
%:%86=23%:%
%:%87=24%:%
%:%88=24%:%
%:%89=25%:%
%:%90=25%:%
%:%91=26%:%
%:%92=26%:%
%:%93=26%:%
%:%94=27%:%
%:%95=27%:%
%:%96=28%:%
%:%97=28%:%
%:%98=28%:%
%:%99=29%:%
%:%100=29%:%
%:%101=30%:%
%:%102=30%:%
%:%103=30%:%
%:%104=31%:%
%:%105=31%:%
%:%106=31%:%
%:%107=32%:%
%:%113=32%:%
%:%116=33%:%
%:%117=34%:%
%:%118=34%:%
%:%125=35%:%

%
\begin{isabellebody}%
\setisabellecontext{Least}%
%
\isadelimdocument
%
\endisadelimdocument
%
\isatagdocument
%
\isamarkupsection{The binder \isa{Least}%
}
\isamarkuptrue%
%
\endisatagdocument
{\isafolddocument}%
%
\isadelimdocument
%
\endisadelimdocument
%
\isadelimtheory
%
\endisadelimtheory
%
\isatagtheory
\isacommand{theory}\isamarkupfalse%
\ Least\isanewline
\ \ \isakeyword{imports}\isanewline
\ \ \ \ Names\isanewline
\isanewline
\isakeyword{begin}%
\endisatagtheory
{\isafoldtheory}%
%
\isadelimtheory
%
\endisadelimtheory
%
\begin{isamarkuptext}%
We have some basic results on the least ordinal satisfying
a predicate.%
\end{isamarkuptext}\isamarkuptrue%
\isacommand{lemma}\isamarkupfalse%
\ Least{\isacharunderscore}{\kern0pt}Ord{\isacharcolon}{\kern0pt}\ {\isachardoublequoteopen}{\isacharparenleft}{\kern0pt}{\isasymmu}\ {\isasymalpha}{\isachardot}{\kern0pt}\ R{\isacharparenleft}{\kern0pt}{\isasymalpha}{\isacharparenright}{\kern0pt}{\isacharparenright}{\kern0pt}\ {\isacharequal}{\kern0pt}\ {\isacharparenleft}{\kern0pt}{\isasymmu}\ {\isasymalpha}{\isachardot}{\kern0pt}\ Ord{\isacharparenleft}{\kern0pt}{\isasymalpha}{\isacharparenright}{\kern0pt}\ {\isasymand}\ R{\isacharparenleft}{\kern0pt}{\isasymalpha}{\isacharparenright}{\kern0pt}{\isacharparenright}{\kern0pt}{\isachardoublequoteclose}\isanewline
%
\isadelimproof
\ \ %
\endisadelimproof
%
\isatagproof
\isacommand{unfolding}\isamarkupfalse%
\ Least{\isacharunderscore}{\kern0pt}def\ \isacommand{by}\isamarkupfalse%
\ {\isacharparenleft}{\kern0pt}simp\ add{\isacharcolon}{\kern0pt}lt{\isacharunderscore}{\kern0pt}Ord{\isacharparenright}{\kern0pt}%
\endisatagproof
{\isafoldproof}%
%
\isadelimproof
\isanewline
%
\endisadelimproof
\isanewline
\isacommand{lemma}\isamarkupfalse%
\ Ord{\isacharunderscore}{\kern0pt}Least{\isacharunderscore}{\kern0pt}cong{\isacharcolon}{\kern0pt}\ \isanewline
\ \ \isakeyword{assumes}\ {\isachardoublequoteopen}{\isasymAnd}y{\isachardot}{\kern0pt}\ Ord{\isacharparenleft}{\kern0pt}y{\isacharparenright}{\kern0pt}\ {\isasymLongrightarrow}\ R{\isacharparenleft}{\kern0pt}y{\isacharparenright}{\kern0pt}\ {\isasymlongleftrightarrow}\ Q{\isacharparenleft}{\kern0pt}y{\isacharparenright}{\kern0pt}{\isachardoublequoteclose}\isanewline
\ \ \isakeyword{shows}\ {\isachardoublequoteopen}{\isacharparenleft}{\kern0pt}{\isasymmu}\ {\isasymalpha}{\isachardot}{\kern0pt}\ R{\isacharparenleft}{\kern0pt}{\isasymalpha}{\isacharparenright}{\kern0pt}{\isacharparenright}{\kern0pt}\ {\isacharequal}{\kern0pt}\ {\isacharparenleft}{\kern0pt}{\isasymmu}\ {\isasymalpha}{\isachardot}{\kern0pt}\ Q{\isacharparenleft}{\kern0pt}{\isasymalpha}{\isacharparenright}{\kern0pt}{\isacharparenright}{\kern0pt}{\isachardoublequoteclose}\isanewline
%
\isadelimproof
%
\endisadelimproof
%
\isatagproof
\isacommand{proof}\isamarkupfalse%
\ {\isacharminus}{\kern0pt}\isanewline
\ \ \isacommand{from}\isamarkupfalse%
\ assms\isanewline
\ \ \isacommand{have}\isamarkupfalse%
\ {\isachardoublequoteopen}{\isacharparenleft}{\kern0pt}{\isasymmu}\ {\isasymalpha}{\isachardot}{\kern0pt}\ Ord{\isacharparenleft}{\kern0pt}{\isasymalpha}{\isacharparenright}{\kern0pt}\ {\isasymand}\ R{\isacharparenleft}{\kern0pt}{\isasymalpha}{\isacharparenright}{\kern0pt}{\isacharparenright}{\kern0pt}\ {\isacharequal}{\kern0pt}\ {\isacharparenleft}{\kern0pt}{\isasymmu}\ {\isasymalpha}{\isachardot}{\kern0pt}\ Ord{\isacharparenleft}{\kern0pt}{\isasymalpha}{\isacharparenright}{\kern0pt}\ {\isasymand}\ Q{\isacharparenleft}{\kern0pt}{\isasymalpha}{\isacharparenright}{\kern0pt}{\isacharparenright}{\kern0pt}{\isachardoublequoteclose}\isanewline
\ \ \ \ \isacommand{by}\isamarkupfalse%
\ simp\ \isanewline
\ \ \isacommand{then}\isamarkupfalse%
\isanewline
\ \ \isacommand{show}\isamarkupfalse%
\ {\isacharquery}{\kern0pt}thesis\ \isacommand{using}\isamarkupfalse%
\ Least{\isacharunderscore}{\kern0pt}Ord\ \isacommand{by}\isamarkupfalse%
\ simp\isanewline
\isacommand{qed}\isamarkupfalse%
%
\endisatagproof
{\isafoldproof}%
%
\isadelimproof
\isanewline
%
\endisadelimproof
\isanewline
\isacommand{definition}\isamarkupfalse%
\isanewline
\ \ least\ {\isacharcolon}{\kern0pt}{\isacharcolon}{\kern0pt}\ {\isachardoublequoteopen}{\isacharbrackleft}{\kern0pt}i{\isasymRightarrow}o{\isacharcomma}{\kern0pt}i{\isasymRightarrow}o{\isacharcomma}{\kern0pt}i{\isacharbrackright}{\kern0pt}\ {\isasymRightarrow}\ o{\isachardoublequoteclose}\ \isakeyword{where}\isanewline
\ \ {\isachardoublequoteopen}least{\isacharparenleft}{\kern0pt}M{\isacharcomma}{\kern0pt}Q{\isacharcomma}{\kern0pt}i{\isacharparenright}{\kern0pt}\ {\isasymequiv}\ ordinal{\isacharparenleft}{\kern0pt}M{\isacharcomma}{\kern0pt}i{\isacharparenright}{\kern0pt}\ {\isasymand}\ {\isacharparenleft}{\kern0pt}\isanewline
\ \ \ \ \ \ \ \ \ {\isacharparenleft}{\kern0pt}empty{\isacharparenleft}{\kern0pt}M{\isacharcomma}{\kern0pt}i{\isacharparenright}{\kern0pt}\ {\isasymand}\ {\isacharparenleft}{\kern0pt}{\isasymforall}b{\isacharbrackleft}{\kern0pt}M{\isacharbrackright}{\kern0pt}{\isachardot}{\kern0pt}\ ordinal{\isacharparenleft}{\kern0pt}M{\isacharcomma}{\kern0pt}b{\isacharparenright}{\kern0pt}\ {\isasymlongrightarrow}\ {\isasymnot}Q{\isacharparenleft}{\kern0pt}b{\isacharparenright}{\kern0pt}{\isacharparenright}{\kern0pt}{\isacharparenright}{\kern0pt}\isanewline
\ \ \ \ \ \ \ {\isasymor}\ {\isacharparenleft}{\kern0pt}Q{\isacharparenleft}{\kern0pt}i{\isacharparenright}{\kern0pt}\ {\isasymand}\ {\isacharparenleft}{\kern0pt}{\isasymforall}b{\isacharbrackleft}{\kern0pt}M{\isacharbrackright}{\kern0pt}{\isachardot}{\kern0pt}\ ordinal{\isacharparenleft}{\kern0pt}M{\isacharcomma}{\kern0pt}b{\isacharparenright}{\kern0pt}\ {\isasymand}\ b{\isasymin}i{\isasymlongrightarrow}\ {\isasymnot}Q{\isacharparenleft}{\kern0pt}b{\isacharparenright}{\kern0pt}{\isacharparenright}{\kern0pt}{\isacharparenright}{\kern0pt}{\isacharparenright}{\kern0pt}{\isachardoublequoteclose}\isanewline
\isanewline
\isacommand{definition}\isamarkupfalse%
\isanewline
\ \ least{\isacharunderscore}{\kern0pt}fm\ {\isacharcolon}{\kern0pt}{\isacharcolon}{\kern0pt}\ {\isachardoublequoteopen}{\isacharbrackleft}{\kern0pt}i{\isacharcomma}{\kern0pt}i{\isacharbrackright}{\kern0pt}\ {\isasymRightarrow}\ i{\isachardoublequoteclose}\ \isakeyword{where}\isanewline
\ \ {\isachardoublequoteopen}least{\isacharunderscore}{\kern0pt}fm{\isacharparenleft}{\kern0pt}q{\isacharcomma}{\kern0pt}i{\isacharparenright}{\kern0pt}\ {\isasymequiv}\ And{\isacharparenleft}{\kern0pt}ordinal{\isacharunderscore}{\kern0pt}fm{\isacharparenleft}{\kern0pt}i{\isacharparenright}{\kern0pt}{\isacharcomma}{\kern0pt}\isanewline
\ \ \ Or{\isacharparenleft}{\kern0pt}And{\isacharparenleft}{\kern0pt}empty{\isacharunderscore}{\kern0pt}fm{\isacharparenleft}{\kern0pt}i{\isacharparenright}{\kern0pt}{\isacharcomma}{\kern0pt}Forall{\isacharparenleft}{\kern0pt}Implies{\isacharparenleft}{\kern0pt}ordinal{\isacharunderscore}{\kern0pt}fm{\isacharparenleft}{\kern0pt}{\isadigit{0}}{\isacharparenright}{\kern0pt}{\isacharcomma}{\kern0pt}Neg{\isacharparenleft}{\kern0pt}q{\isacharparenright}{\kern0pt}{\isacharparenright}{\kern0pt}{\isacharparenright}{\kern0pt}{\isacharparenright}{\kern0pt}{\isacharcomma}{\kern0pt}\ \isanewline
\ \ \ \ \ \ And{\isacharparenleft}{\kern0pt}Exists{\isacharparenleft}{\kern0pt}And{\isacharparenleft}{\kern0pt}q{\isacharcomma}{\kern0pt}Equal{\isacharparenleft}{\kern0pt}{\isadigit{0}}{\isacharcomma}{\kern0pt}succ{\isacharparenleft}{\kern0pt}i{\isacharparenright}{\kern0pt}{\isacharparenright}{\kern0pt}{\isacharparenright}{\kern0pt}{\isacharparenright}{\kern0pt}{\isacharcomma}{\kern0pt}\isanewline
\ \ \ \ \ \ \ \ \ \ Forall{\isacharparenleft}{\kern0pt}Implies{\isacharparenleft}{\kern0pt}And{\isacharparenleft}{\kern0pt}ordinal{\isacharunderscore}{\kern0pt}fm{\isacharparenleft}{\kern0pt}{\isadigit{0}}{\isacharparenright}{\kern0pt}{\isacharcomma}{\kern0pt}Member{\isacharparenleft}{\kern0pt}{\isadigit{0}}{\isacharcomma}{\kern0pt}succ{\isacharparenleft}{\kern0pt}i{\isacharparenright}{\kern0pt}{\isacharparenright}{\kern0pt}{\isacharparenright}{\kern0pt}{\isacharcomma}{\kern0pt}Neg{\isacharparenleft}{\kern0pt}q{\isacharparenright}{\kern0pt}{\isacharparenright}{\kern0pt}{\isacharparenright}{\kern0pt}{\isacharparenright}{\kern0pt}{\isacharparenright}{\kern0pt}{\isacharparenright}{\kern0pt}{\isachardoublequoteclose}\isanewline
\isanewline
\isacommand{lemma}\isamarkupfalse%
\ least{\isacharunderscore}{\kern0pt}fm{\isacharunderscore}{\kern0pt}type{\isacharbrackleft}{\kern0pt}TC{\isacharbrackright}{\kern0pt}\ {\isacharcolon}{\kern0pt}{\isachardoublequoteopen}i\ {\isasymin}\ nat\ {\isasymLongrightarrow}\ q{\isasymin}formula\ {\isasymLongrightarrow}\ least{\isacharunderscore}{\kern0pt}fm{\isacharparenleft}{\kern0pt}q{\isacharcomma}{\kern0pt}i{\isacharparenright}{\kern0pt}\ {\isasymin}\ formula{\isachardoublequoteclose}\isanewline
%
\isadelimproof
\ \ %
\endisadelimproof
%
\isatagproof
\isacommand{unfolding}\isamarkupfalse%
\ least{\isacharunderscore}{\kern0pt}fm{\isacharunderscore}{\kern0pt}def\isanewline
\ \ \isacommand{by}\isamarkupfalse%
\ simp%
\endisatagproof
{\isafoldproof}%
%
\isadelimproof
\isanewline
%
\endisadelimproof
\isanewline
\isanewline
\isacommand{lemmas}\isamarkupfalse%
\ basic{\isacharunderscore}{\kern0pt}fm{\isacharunderscore}{\kern0pt}simps\ {\isacharequal}{\kern0pt}\ sats{\isacharunderscore}{\kern0pt}subset{\isacharunderscore}{\kern0pt}fm{\isacharprime}{\kern0pt}\ sats{\isacharunderscore}{\kern0pt}transset{\isacharunderscore}{\kern0pt}fm{\isacharprime}{\kern0pt}\ sats{\isacharunderscore}{\kern0pt}ordinal{\isacharunderscore}{\kern0pt}fm{\isacharprime}{\kern0pt}\isanewline
\isanewline
\isacommand{lemma}\isamarkupfalse%
\ sats{\isacharunderscore}{\kern0pt}least{\isacharunderscore}{\kern0pt}fm\ {\isacharcolon}{\kern0pt}\isanewline
\ \ \isakeyword{assumes}\ p{\isacharunderscore}{\kern0pt}iff{\isacharunderscore}{\kern0pt}sats{\isacharcolon}{\kern0pt}\ \isanewline
\ \ \ \ {\isachardoublequoteopen}{\isasymAnd}a{\isachardot}{\kern0pt}\ a\ {\isasymin}\ A\ {\isasymLongrightarrow}\ P{\isacharparenleft}{\kern0pt}a{\isacharparenright}{\kern0pt}\ {\isasymlongleftrightarrow}\ sats{\isacharparenleft}{\kern0pt}A{\isacharcomma}{\kern0pt}\ p{\isacharcomma}{\kern0pt}\ Cons{\isacharparenleft}{\kern0pt}a{\isacharcomma}{\kern0pt}\ env{\isacharparenright}{\kern0pt}{\isacharparenright}{\kern0pt}{\isachardoublequoteclose}\isanewline
\ \ \isakeyword{shows}\isanewline
\ \ \ \ {\isachardoublequoteopen}{\isasymlbrakk}y\ {\isasymin}\ nat{\isacharsemicolon}{\kern0pt}\ env\ {\isasymin}\ list{\isacharparenleft}{\kern0pt}A{\isacharparenright}{\kern0pt}\ {\isacharsemicolon}{\kern0pt}\ {\isadigit{0}}{\isasymin}A{\isasymrbrakk}\isanewline
\ \ \ \ {\isasymLongrightarrow}\ sats{\isacharparenleft}{\kern0pt}A{\isacharcomma}{\kern0pt}\ least{\isacharunderscore}{\kern0pt}fm{\isacharparenleft}{\kern0pt}p{\isacharcomma}{\kern0pt}y{\isacharparenright}{\kern0pt}{\isacharcomma}{\kern0pt}\ env{\isacharparenright}{\kern0pt}\ {\isasymlongleftrightarrow}\isanewline
\ \ \ \ \ \ \ \ least{\isacharparenleft}{\kern0pt}{\isacharhash}{\kern0pt}{\isacharhash}{\kern0pt}A{\isacharcomma}{\kern0pt}\ P{\isacharcomma}{\kern0pt}\ nth{\isacharparenleft}{\kern0pt}y{\isacharcomma}{\kern0pt}env{\isacharparenright}{\kern0pt}{\isacharparenright}{\kern0pt}{\isachardoublequoteclose}\isanewline
%
\isadelimproof
\ \ %
\endisadelimproof
%
\isatagproof
\isacommand{using}\isamarkupfalse%
\ nth{\isacharunderscore}{\kern0pt}closed\ p{\isacharunderscore}{\kern0pt}iff{\isacharunderscore}{\kern0pt}sats\ \isacommand{unfolding}\isamarkupfalse%
\ least{\isacharunderscore}{\kern0pt}def\ least{\isacharunderscore}{\kern0pt}fm{\isacharunderscore}{\kern0pt}def\isanewline
\ \ \isacommand{by}\isamarkupfalse%
\ {\isacharparenleft}{\kern0pt}simp\ add{\isacharcolon}{\kern0pt}basic{\isacharunderscore}{\kern0pt}fm{\isacharunderscore}{\kern0pt}simps{\isacharparenright}{\kern0pt}%
\endisatagproof
{\isafoldproof}%
%
\isadelimproof
\isanewline
%
\endisadelimproof
\isanewline
\isacommand{lemma}\isamarkupfalse%
\ least{\isacharunderscore}{\kern0pt}iff{\isacharunderscore}{\kern0pt}sats{\isacharcolon}{\kern0pt}\isanewline
\ \ \isakeyword{assumes}\ is{\isacharunderscore}{\kern0pt}Q{\isacharunderscore}{\kern0pt}iff{\isacharunderscore}{\kern0pt}sats{\isacharcolon}{\kern0pt}\ \isanewline
\ \ \ \ \ \ {\isachardoublequoteopen}{\isasymAnd}a{\isachardot}{\kern0pt}\ a\ {\isasymin}\ A\ {\isasymLongrightarrow}\ is{\isacharunderscore}{\kern0pt}Q{\isacharparenleft}{\kern0pt}a{\isacharparenright}{\kern0pt}\ {\isasymlongleftrightarrow}\ sats{\isacharparenleft}{\kern0pt}A{\isacharcomma}{\kern0pt}\ q{\isacharcomma}{\kern0pt}\ Cons{\isacharparenleft}{\kern0pt}a{\isacharcomma}{\kern0pt}env{\isacharparenright}{\kern0pt}{\isacharparenright}{\kern0pt}{\isachardoublequoteclose}\isanewline
\ \ \isakeyword{shows}\ \isanewline
\ \ {\isachardoublequoteopen}{\isasymlbrakk}nth{\isacharparenleft}{\kern0pt}j{\isacharcomma}{\kern0pt}env{\isacharparenright}{\kern0pt}\ {\isacharequal}{\kern0pt}\ y{\isacharsemicolon}{\kern0pt}\ j\ {\isasymin}\ nat{\isacharsemicolon}{\kern0pt}\ env\ {\isasymin}\ list{\isacharparenleft}{\kern0pt}A{\isacharparenright}{\kern0pt}{\isacharsemicolon}{\kern0pt}\ {\isadigit{0}}{\isasymin}A{\isasymrbrakk}\isanewline
\ \ \ {\isasymLongrightarrow}\ least{\isacharparenleft}{\kern0pt}{\isacharhash}{\kern0pt}{\isacharhash}{\kern0pt}A{\isacharcomma}{\kern0pt}\ is{\isacharunderscore}{\kern0pt}Q{\isacharcomma}{\kern0pt}\ y{\isacharparenright}{\kern0pt}\ {\isasymlongleftrightarrow}\ sats{\isacharparenleft}{\kern0pt}A{\isacharcomma}{\kern0pt}\ least{\isacharunderscore}{\kern0pt}fm{\isacharparenleft}{\kern0pt}q{\isacharcomma}{\kern0pt}j{\isacharparenright}{\kern0pt}{\isacharcomma}{\kern0pt}\ env{\isacharparenright}{\kern0pt}{\isachardoublequoteclose}\isanewline
%
\isadelimproof
\ \ %
\endisadelimproof
%
\isatagproof
\isacommand{using}\isamarkupfalse%
\ sats{\isacharunderscore}{\kern0pt}least{\isacharunderscore}{\kern0pt}fm\ {\isacharbrackleft}{\kern0pt}OF\ is{\isacharunderscore}{\kern0pt}Q{\isacharunderscore}{\kern0pt}iff{\isacharunderscore}{\kern0pt}sats{\isacharcomma}{\kern0pt}\ of\ j\ {\isacharcomma}{\kern0pt}\ symmetric{\isacharbrackright}{\kern0pt}\isanewline
\ \ \isacommand{by}\isamarkupfalse%
\ simp%
\endisatagproof
{\isafoldproof}%
%
\isadelimproof
\isanewline
%
\endisadelimproof
\isanewline
\isacommand{lemma}\isamarkupfalse%
\ least{\isacharunderscore}{\kern0pt}conj{\isacharcolon}{\kern0pt}\ {\isachardoublequoteopen}a{\isasymin}M\ {\isasymLongrightarrow}\ least{\isacharparenleft}{\kern0pt}{\isacharhash}{\kern0pt}{\isacharhash}{\kern0pt}M{\isacharcomma}{\kern0pt}\ {\isasymlambda}x{\isachardot}{\kern0pt}\ x{\isasymin}M\ {\isasymand}\ Q{\isacharparenleft}{\kern0pt}x{\isacharparenright}{\kern0pt}{\isacharcomma}{\kern0pt}a{\isacharparenright}{\kern0pt}\ {\isasymlongleftrightarrow}\ least{\isacharparenleft}{\kern0pt}{\isacharhash}{\kern0pt}{\isacharhash}{\kern0pt}M{\isacharcomma}{\kern0pt}Q{\isacharcomma}{\kern0pt}a{\isacharparenright}{\kern0pt}{\isachardoublequoteclose}\isanewline
%
\isadelimproof
\ \ %
\endisadelimproof
%
\isatagproof
\isacommand{unfolding}\isamarkupfalse%
\ least{\isacharunderscore}{\kern0pt}def\ \isacommand{by}\isamarkupfalse%
\ simp%
\endisatagproof
{\isafoldproof}%
%
\isadelimproof
\isanewline
%
\endisadelimproof
\isanewline
\isanewline
\isacommand{lemma}\isamarkupfalse%
\ {\isacharparenleft}{\kern0pt}\isakeyword{in}\ M{\isacharunderscore}{\kern0pt}ctm{\isacharparenright}{\kern0pt}\ unique{\isacharunderscore}{\kern0pt}least{\isacharcolon}{\kern0pt}\ {\isachardoublequoteopen}a{\isasymin}M\ {\isasymLongrightarrow}\ b{\isasymin}M\ {\isasymLongrightarrow}\ least{\isacharparenleft}{\kern0pt}{\isacharhash}{\kern0pt}{\isacharhash}{\kern0pt}M{\isacharcomma}{\kern0pt}Q{\isacharcomma}{\kern0pt}a{\isacharparenright}{\kern0pt}\ {\isasymLongrightarrow}\ least{\isacharparenleft}{\kern0pt}{\isacharhash}{\kern0pt}{\isacharhash}{\kern0pt}M{\isacharcomma}{\kern0pt}Q{\isacharcomma}{\kern0pt}b{\isacharparenright}{\kern0pt}\ {\isasymLongrightarrow}\ a{\isacharequal}{\kern0pt}b{\isachardoublequoteclose}\isanewline
%
\isadelimproof
\ \ %
\endisadelimproof
%
\isatagproof
\isacommand{unfolding}\isamarkupfalse%
\ least{\isacharunderscore}{\kern0pt}def\isanewline
\ \ \isacommand{by}\isamarkupfalse%
\ {\isacharparenleft}{\kern0pt}auto{\isacharcomma}{\kern0pt}\ erule{\isacharunderscore}{\kern0pt}tac\ i{\isacharequal}{\kern0pt}a\ \isakeyword{and}\ j{\isacharequal}{\kern0pt}b\ \isakeyword{in}\ Ord{\isacharunderscore}{\kern0pt}linear{\isacharunderscore}{\kern0pt}lt{\isacharsemicolon}{\kern0pt}\ {\isacharparenleft}{\kern0pt}drule\ ltD\ {\isacharbar}{\kern0pt}\ simp{\isacharparenright}{\kern0pt}{\isacharsemicolon}{\kern0pt}\ auto\ intro{\isacharcolon}{\kern0pt}Ord{\isacharunderscore}{\kern0pt}in{\isacharunderscore}{\kern0pt}Ord{\isacharparenright}{\kern0pt}%
\endisatagproof
{\isafoldproof}%
%
\isadelimproof
\isanewline
%
\endisadelimproof
\isanewline
\isacommand{context}\isamarkupfalse%
\ M{\isacharunderscore}{\kern0pt}trivial\isanewline
\isakeyword{begin}%
\isadelimdocument
%
\endisadelimdocument
%
\isatagdocument
%
\isamarkupsubsection{Absoluteness and closure under \isa{Least}%
}
\isamarkuptrue%
%
\endisatagdocument
{\isafolddocument}%
%
\isadelimdocument
%
\endisadelimdocument
\isacommand{lemma}\isamarkupfalse%
\ least{\isacharunderscore}{\kern0pt}abs{\isacharcolon}{\kern0pt}\isanewline
\ \ \isakeyword{assumes}\ {\isachardoublequoteopen}{\isasymAnd}x{\isachardot}{\kern0pt}\ Q{\isacharparenleft}{\kern0pt}x{\isacharparenright}{\kern0pt}\ {\isasymLongrightarrow}\ M{\isacharparenleft}{\kern0pt}x{\isacharparenright}{\kern0pt}{\isachardoublequoteclose}\ {\isachardoublequoteopen}M{\isacharparenleft}{\kern0pt}a{\isacharparenright}{\kern0pt}{\isachardoublequoteclose}\ \isanewline
\ \ \isakeyword{shows}\ {\isachardoublequoteopen}least{\isacharparenleft}{\kern0pt}M{\isacharcomma}{\kern0pt}Q{\isacharcomma}{\kern0pt}a{\isacharparenright}{\kern0pt}\ {\isasymlongleftrightarrow}\ a\ {\isacharequal}{\kern0pt}\ {\isacharparenleft}{\kern0pt}{\isasymmu}\ x{\isachardot}{\kern0pt}\ Q{\isacharparenleft}{\kern0pt}x{\isacharparenright}{\kern0pt}{\isacharparenright}{\kern0pt}{\isachardoublequoteclose}\isanewline
%
\isadelimproof
\ \ %
\endisadelimproof
%
\isatagproof
\isacommand{unfolding}\isamarkupfalse%
\ least{\isacharunderscore}{\kern0pt}def\isanewline
\isacommand{proof}\isamarkupfalse%
\ {\isacharparenleft}{\kern0pt}cases\ {\isachardoublequoteopen}{\isasymforall}b{\isacharbrackleft}{\kern0pt}M{\isacharbrackright}{\kern0pt}{\isachardot}{\kern0pt}\ Ord{\isacharparenleft}{\kern0pt}b{\isacharparenright}{\kern0pt}\ {\isasymlongrightarrow}\ {\isasymnot}\ Q{\isacharparenleft}{\kern0pt}b{\isacharparenright}{\kern0pt}{\isachardoublequoteclose}{\isacharsemicolon}{\kern0pt}\ intro\ iffI{\isacharsemicolon}{\kern0pt}\ simp\ add{\isacharcolon}{\kern0pt}assms{\isacharparenright}{\kern0pt}\isanewline
\ \ \isacommand{case}\isamarkupfalse%
\ True\isanewline
\ \ \isacommand{with}\isamarkupfalse%
\ {\isacartoucheopen}{\isasymAnd}x{\isachardot}{\kern0pt}\ Q{\isacharparenleft}{\kern0pt}x{\isacharparenright}{\kern0pt}\ {\isasymLongrightarrow}\ M{\isacharparenleft}{\kern0pt}x{\isacharparenright}{\kern0pt}{\isacartoucheclose}\isanewline
\ \ \isacommand{have}\isamarkupfalse%
\ {\isachardoublequoteopen}{\isasymnot}\ {\isacharparenleft}{\kern0pt}{\isasymexists}i{\isachardot}{\kern0pt}\ Ord{\isacharparenleft}{\kern0pt}i{\isacharparenright}{\kern0pt}\ {\isasymand}\ Q{\isacharparenleft}{\kern0pt}i{\isacharparenright}{\kern0pt}{\isacharparenright}{\kern0pt}\ {\isachardoublequoteclose}\ \isacommand{by}\isamarkupfalse%
\ blast\isanewline
\ \ \isacommand{then}\isamarkupfalse%
\isanewline
\ \ \isacommand{show}\isamarkupfalse%
\ {\isachardoublequoteopen}{\isadigit{0}}\ {\isacharequal}{\kern0pt}{\isacharparenleft}{\kern0pt}{\isasymmu}\ x{\isachardot}{\kern0pt}\ Q{\isacharparenleft}{\kern0pt}x{\isacharparenright}{\kern0pt}{\isacharparenright}{\kern0pt}{\isachardoublequoteclose}\ \isacommand{using}\isamarkupfalse%
\ Least{\isacharunderscore}{\kern0pt}{\isadigit{0}}\ \isacommand{by}\isamarkupfalse%
\ simp\isanewline
\ \ \isacommand{then}\isamarkupfalse%
\isanewline
\ \ \isacommand{show}\isamarkupfalse%
\ {\isachardoublequoteopen}ordinal{\isacharparenleft}{\kern0pt}M{\isacharcomma}{\kern0pt}\ {\isasymmu}\ x{\isachardot}{\kern0pt}\ Q{\isacharparenleft}{\kern0pt}x{\isacharparenright}{\kern0pt}{\isacharparenright}{\kern0pt}\ {\isasymand}\ {\isacharparenleft}{\kern0pt}empty{\isacharparenleft}{\kern0pt}M{\isacharcomma}{\kern0pt}\ Least{\isacharparenleft}{\kern0pt}Q{\isacharparenright}{\kern0pt}{\isacharparenright}{\kern0pt}\ {\isasymor}\ Q{\isacharparenleft}{\kern0pt}Least{\isacharparenleft}{\kern0pt}Q{\isacharparenright}{\kern0pt}{\isacharparenright}{\kern0pt}{\isacharparenright}{\kern0pt}{\isachardoublequoteclose}\isanewline
\ \ \ \ \isacommand{by}\isamarkupfalse%
\ simp\ \isanewline
\isacommand{next}\isamarkupfalse%
\isanewline
\ \ \isacommand{assume}\isamarkupfalse%
\ {\isachardoublequoteopen}{\isasymexists}b{\isacharbrackleft}{\kern0pt}M{\isacharbrackright}{\kern0pt}{\isachardot}{\kern0pt}\ Ord{\isacharparenleft}{\kern0pt}b{\isacharparenright}{\kern0pt}\ {\isasymand}\ Q{\isacharparenleft}{\kern0pt}b{\isacharparenright}{\kern0pt}{\isachardoublequoteclose}\isanewline
\ \ \isacommand{then}\isamarkupfalse%
\ \isanewline
\ \ \isacommand{obtain}\isamarkupfalse%
\ i\ \isakeyword{where}\ {\isachardoublequoteopen}M{\isacharparenleft}{\kern0pt}i{\isacharparenright}{\kern0pt}{\isachardoublequoteclose}\ {\isachardoublequoteopen}Ord{\isacharparenleft}{\kern0pt}i{\isacharparenright}{\kern0pt}{\isachardoublequoteclose}\ {\isachardoublequoteopen}Q{\isacharparenleft}{\kern0pt}i{\isacharparenright}{\kern0pt}{\isachardoublequoteclose}\ \isacommand{by}\isamarkupfalse%
\ blast\isanewline
\ \ \isacommand{assume}\isamarkupfalse%
\ {\isachardoublequoteopen}a\ {\isacharequal}{\kern0pt}\ {\isacharparenleft}{\kern0pt}{\isasymmu}\ x{\isachardot}{\kern0pt}\ Q{\isacharparenleft}{\kern0pt}x{\isacharparenright}{\kern0pt}{\isacharparenright}{\kern0pt}{\isachardoublequoteclose}\isanewline
\ \ \isacommand{moreover}\isamarkupfalse%
\isanewline
\ \ \isacommand{note}\isamarkupfalse%
\ {\isacartoucheopen}M{\isacharparenleft}{\kern0pt}a{\isacharparenright}{\kern0pt}{\isacartoucheclose}\isanewline
\ \ \isacommand{moreover}\isamarkupfalse%
\ \isacommand{from}\isamarkupfalse%
\ \ {\isacartoucheopen}Q{\isacharparenleft}{\kern0pt}i{\isacharparenright}{\kern0pt}{\isacartoucheclose}\ {\isacartoucheopen}Ord{\isacharparenleft}{\kern0pt}i{\isacharparenright}{\kern0pt}{\isacartoucheclose}\isanewline
\ \ \isacommand{have}\isamarkupfalse%
\ {\isachardoublequoteopen}Q{\isacharparenleft}{\kern0pt}{\isasymmu}\ x{\isachardot}{\kern0pt}\ Q{\isacharparenleft}{\kern0pt}x{\isacharparenright}{\kern0pt}{\isacharparenright}{\kern0pt}{\isachardoublequoteclose}\ {\isacharparenleft}{\kern0pt}\isakeyword{is}\ {\isacharquery}{\kern0pt}G{\isacharparenright}{\kern0pt}\isanewline
\ \ \ \ \isacommand{by}\isamarkupfalse%
\ {\isacharparenleft}{\kern0pt}blast\ intro{\isacharcolon}{\kern0pt}LeastI{\isacharparenright}{\kern0pt}\isanewline
\ \ \isacommand{moreover}\isamarkupfalse%
\isanewline
\ \ \isacommand{have}\isamarkupfalse%
\ {\isachardoublequoteopen}{\isacharparenleft}{\kern0pt}{\isasymforall}b{\isacharbrackleft}{\kern0pt}M{\isacharbrackright}{\kern0pt}{\isachardot}{\kern0pt}\ Ord{\isacharparenleft}{\kern0pt}b{\isacharparenright}{\kern0pt}\ {\isasymand}\ b\ {\isasymin}\ {\isacharparenleft}{\kern0pt}{\isasymmu}\ x{\isachardot}{\kern0pt}\ Q{\isacharparenleft}{\kern0pt}x{\isacharparenright}{\kern0pt}{\isacharparenright}{\kern0pt}\ {\isasymlongrightarrow}\ {\isasymnot}\ Q{\isacharparenleft}{\kern0pt}b{\isacharparenright}{\kern0pt}{\isacharparenright}{\kern0pt}{\isachardoublequoteclose}\ {\isacharparenleft}{\kern0pt}\isakeyword{is}\ {\isachardoublequoteopen}{\isacharquery}{\kern0pt}H{\isachardoublequoteclose}{\isacharparenright}{\kern0pt}\isanewline
\ \ \ \ \isacommand{using}\isamarkupfalse%
\ less{\isacharunderscore}{\kern0pt}LeastE{\isacharbrackleft}{\kern0pt}of\ Q\ {\isacharunderscore}{\kern0pt}\ False{\isacharbrackright}{\kern0pt}\isanewline
\ \ \ \ \isacommand{by}\isamarkupfalse%
\ {\isacharparenleft}{\kern0pt}auto{\isacharcomma}{\kern0pt}\ drule{\isacharunderscore}{\kern0pt}tac\ ltI{\isacharcomma}{\kern0pt}\ simp{\isacharcomma}{\kern0pt}\ blast{\isacharparenright}{\kern0pt}\isanewline
\ \ \isacommand{ultimately}\isamarkupfalse%
\isanewline
\ \ \isacommand{show}\isamarkupfalse%
\ {\isachardoublequoteopen}ordinal{\isacharparenleft}{\kern0pt}M{\isacharcomma}{\kern0pt}\ {\isasymmu}\ x{\isachardot}{\kern0pt}\ Q{\isacharparenleft}{\kern0pt}x{\isacharparenright}{\kern0pt}{\isacharparenright}{\kern0pt}\ {\isasymand}\ {\isacharparenleft}{\kern0pt}empty{\isacharparenleft}{\kern0pt}M{\isacharcomma}{\kern0pt}\ {\isasymmu}\ x{\isachardot}{\kern0pt}\ Q{\isacharparenleft}{\kern0pt}x{\isacharparenright}{\kern0pt}{\isacharparenright}{\kern0pt}\ {\isasymand}\ {\isacharparenleft}{\kern0pt}{\isasymforall}b{\isacharbrackleft}{\kern0pt}M{\isacharbrackright}{\kern0pt}{\isachardot}{\kern0pt}\ Ord{\isacharparenleft}{\kern0pt}b{\isacharparenright}{\kern0pt}\ {\isasymlongrightarrow}\ {\isasymnot}\ Q{\isacharparenleft}{\kern0pt}b{\isacharparenright}{\kern0pt}{\isacharparenright}{\kern0pt}\ {\isasymor}\ {\isacharquery}{\kern0pt}G\ {\isasymand}\ {\isacharquery}{\kern0pt}H{\isacharparenright}{\kern0pt}{\isachardoublequoteclose}\isanewline
\ \ \ \ \isacommand{by}\isamarkupfalse%
\ simp\isanewline
\isacommand{next}\isamarkupfalse%
\isanewline
\ \ \isacommand{assume}\isamarkupfalse%
\ {\isadigit{1}}{\isacharcolon}{\kern0pt}{\isachardoublequoteopen}{\isasymexists}b{\isacharbrackleft}{\kern0pt}M{\isacharbrackright}{\kern0pt}{\isachardot}{\kern0pt}\ Ord{\isacharparenleft}{\kern0pt}b{\isacharparenright}{\kern0pt}\ {\isasymand}\ Q{\isacharparenleft}{\kern0pt}b{\isacharparenright}{\kern0pt}{\isachardoublequoteclose}\isanewline
\ \ \isacommand{then}\isamarkupfalse%
\ \isanewline
\ \ \isacommand{obtain}\isamarkupfalse%
\ i\ \isakeyword{where}\ {\isachardoublequoteopen}M{\isacharparenleft}{\kern0pt}i{\isacharparenright}{\kern0pt}{\isachardoublequoteclose}\ {\isachardoublequoteopen}Ord{\isacharparenleft}{\kern0pt}i{\isacharparenright}{\kern0pt}{\isachardoublequoteclose}\ {\isachardoublequoteopen}Q{\isacharparenleft}{\kern0pt}i{\isacharparenright}{\kern0pt}{\isachardoublequoteclose}\ \isacommand{by}\isamarkupfalse%
\ blast\isanewline
\ \ \isacommand{assume}\isamarkupfalse%
\ {\isachardoublequoteopen}Ord{\isacharparenleft}{\kern0pt}a{\isacharparenright}{\kern0pt}\ {\isasymand}\ {\isacharparenleft}{\kern0pt}a\ {\isacharequal}{\kern0pt}\ {\isadigit{0}}\ {\isasymand}\ {\isacharparenleft}{\kern0pt}{\isasymforall}b{\isacharbrackleft}{\kern0pt}M{\isacharbrackright}{\kern0pt}{\isachardot}{\kern0pt}\ Ord{\isacharparenleft}{\kern0pt}b{\isacharparenright}{\kern0pt}\ {\isasymlongrightarrow}\ {\isasymnot}\ Q{\isacharparenleft}{\kern0pt}b{\isacharparenright}{\kern0pt}{\isacharparenright}{\kern0pt}\ {\isasymor}\ Q{\isacharparenleft}{\kern0pt}a{\isacharparenright}{\kern0pt}\ {\isasymand}\ {\isacharparenleft}{\kern0pt}{\isasymforall}b{\isacharbrackleft}{\kern0pt}M{\isacharbrackright}{\kern0pt}{\isachardot}{\kern0pt}\ Ord{\isacharparenleft}{\kern0pt}b{\isacharparenright}{\kern0pt}\ {\isasymand}\ b\ {\isasymin}\ a\ {\isasymlongrightarrow}\ {\isasymnot}\ Q{\isacharparenleft}{\kern0pt}b{\isacharparenright}{\kern0pt}{\isacharparenright}{\kern0pt}{\isacharparenright}{\kern0pt}{\isachardoublequoteclose}\isanewline
\ \ \isacommand{with}\isamarkupfalse%
\ {\isadigit{1}}\isanewline
\ \ \isacommand{have}\isamarkupfalse%
\ {\isachardoublequoteopen}Ord{\isacharparenleft}{\kern0pt}a{\isacharparenright}{\kern0pt}{\isachardoublequoteclose}\ {\isachardoublequoteopen}Q{\isacharparenleft}{\kern0pt}a{\isacharparenright}{\kern0pt}{\isachardoublequoteclose}\ {\isachardoublequoteopen}{\isasymforall}b{\isacharbrackleft}{\kern0pt}M{\isacharbrackright}{\kern0pt}{\isachardot}{\kern0pt}\ Ord{\isacharparenleft}{\kern0pt}b{\isacharparenright}{\kern0pt}\ {\isasymand}\ b\ {\isasymin}\ a\ {\isasymlongrightarrow}\ {\isasymnot}\ Q{\isacharparenleft}{\kern0pt}b{\isacharparenright}{\kern0pt}{\isachardoublequoteclose}\isanewline
\ \ \ \ \isacommand{by}\isamarkupfalse%
\ blast{\isacharplus}{\kern0pt}\isanewline
\ \ \isacommand{moreover}\isamarkupfalse%
\ \isacommand{from}\isamarkupfalse%
\ this\ \isakeyword{and}\ {\isacartoucheopen}{\isasymAnd}x{\isachardot}{\kern0pt}\ Q{\isacharparenleft}{\kern0pt}x{\isacharparenright}{\kern0pt}\ {\isasymLongrightarrow}\ M{\isacharparenleft}{\kern0pt}x{\isacharparenright}{\kern0pt}{\isacartoucheclose}\isanewline
\ \ \isacommand{have}\isamarkupfalse%
\ {\isachardoublequoteopen}Ord{\isacharparenleft}{\kern0pt}b{\isacharparenright}{\kern0pt}\ {\isasymLongrightarrow}\ b\ {\isasymin}\ a\ {\isasymLongrightarrow}\ {\isasymnot}\ Q{\isacharparenleft}{\kern0pt}b{\isacharparenright}{\kern0pt}{\isachardoublequoteclose}\ \isakeyword{for}\ b\isanewline
\ \ \ \ \isacommand{by}\isamarkupfalse%
\ blast\isanewline
\ \ \isacommand{moreover}\isamarkupfalse%
\ \isacommand{from}\isamarkupfalse%
\ this\ \isakeyword{and}\ {\isacartoucheopen}Ord{\isacharparenleft}{\kern0pt}a{\isacharparenright}{\kern0pt}{\isacartoucheclose}\isanewline
\ \ \isacommand{have}\isamarkupfalse%
\ {\isachardoublequoteopen}b\ {\isacharless}{\kern0pt}\ a\ {\isasymLongrightarrow}\ {\isasymnot}\ Q{\isacharparenleft}{\kern0pt}b{\isacharparenright}{\kern0pt}{\isachardoublequoteclose}\ \isakeyword{for}\ b\isanewline
\ \ \ \ \isacommand{unfolding}\isamarkupfalse%
\ lt{\isacharunderscore}{\kern0pt}def\ \isacommand{using}\isamarkupfalse%
\ Ord{\isacharunderscore}{\kern0pt}in{\isacharunderscore}{\kern0pt}Ord\ \isacommand{by}\isamarkupfalse%
\ blast\isanewline
\ \ \isacommand{ultimately}\isamarkupfalse%
\isanewline
\ \ \isacommand{show}\isamarkupfalse%
\ {\isachardoublequoteopen}a\ {\isacharequal}{\kern0pt}\ {\isacharparenleft}{\kern0pt}{\isasymmu}\ x{\isachardot}{\kern0pt}\ Q{\isacharparenleft}{\kern0pt}x{\isacharparenright}{\kern0pt}{\isacharparenright}{\kern0pt}{\isachardoublequoteclose}\isanewline
\ \ \ \ \isacommand{using}\isamarkupfalse%
\ Least{\isacharunderscore}{\kern0pt}equality\ \isacommand{by}\isamarkupfalse%
\ simp\isanewline
\isacommand{qed}\isamarkupfalse%
%
\endisatagproof
{\isafoldproof}%
%
\isadelimproof
\isanewline
%
\endisadelimproof
\isanewline
\isacommand{lemma}\isamarkupfalse%
\ Least{\isacharunderscore}{\kern0pt}closed{\isacharcolon}{\kern0pt}\isanewline
\ \ \isakeyword{assumes}\ {\isachardoublequoteopen}{\isasymAnd}x{\isachardot}{\kern0pt}\ Q{\isacharparenleft}{\kern0pt}x{\isacharparenright}{\kern0pt}\ {\isasymLongrightarrow}\ M{\isacharparenleft}{\kern0pt}x{\isacharparenright}{\kern0pt}{\isachardoublequoteclose}\isanewline
\ \ \isakeyword{shows}\ {\isachardoublequoteopen}M{\isacharparenleft}{\kern0pt}{\isasymmu}\ x{\isachardot}{\kern0pt}\ Q{\isacharparenleft}{\kern0pt}x{\isacharparenright}{\kern0pt}{\isacharparenright}{\kern0pt}{\isachardoublequoteclose}\isanewline
%
\isadelimproof
\ \ %
\endisadelimproof
%
\isatagproof
\isacommand{using}\isamarkupfalse%
\ assms\ LeastI{\isacharbrackleft}{\kern0pt}of\ Q{\isacharbrackright}{\kern0pt}\ Least{\isacharunderscore}{\kern0pt}{\isadigit{0}}\ \isacommand{by}\isamarkupfalse%
\ {\isacharparenleft}{\kern0pt}cases\ {\isachardoublequoteopen}{\isacharparenleft}{\kern0pt}{\isasymexists}i{\isachardot}{\kern0pt}\ Ord{\isacharparenleft}{\kern0pt}i{\isacharparenright}{\kern0pt}\ {\isasymand}\ Q{\isacharparenleft}{\kern0pt}i{\isacharparenright}{\kern0pt}{\isacharparenright}{\kern0pt}{\isachardoublequoteclose}{\isacharcomma}{\kern0pt}\ auto{\isacharparenright}{\kern0pt}%
\endisatagproof
{\isafoldproof}%
%
\isadelimproof
\isanewline
%
\endisadelimproof
\isanewline
\isacommand{end}\isamarkupfalse%
\ \isanewline
%
\isadelimtheory
\isanewline
%
\endisadelimtheory
%
\isatagtheory
\isacommand{end}\isamarkupfalse%
%
\endisatagtheory
{\isafoldtheory}%
%
\isadelimtheory
%
\endisadelimtheory
%
\end{isabellebody}%
\endinput
%:%file=~/source/repos/ZF-notAC/code/Forcing/Least.thy%:%
%:%11=1%:%
%:%27=2%:%
%:%28=2%:%
%:%29=3%:%
%:%30=4%:%
%:%31=5%:%
%:%32=6%:%
%:%41=8%:%
%:%42=9%:%
%:%44=11%:%
%:%45=11%:%
%:%48=12%:%
%:%52=12%:%
%:%53=12%:%
%:%54=12%:%
%:%59=12%:%
%:%62=13%:%
%:%63=14%:%
%:%64=14%:%
%:%65=15%:%
%:%66=16%:%
%:%73=17%:%
%:%74=17%:%
%:%75=18%:%
%:%76=18%:%
%:%77=19%:%
%:%78=19%:%
%:%79=20%:%
%:%80=20%:%
%:%81=21%:%
%:%82=21%:%
%:%83=22%:%
%:%84=22%:%
%:%85=22%:%
%:%86=22%:%
%:%87=23%:%
%:%93=23%:%
%:%96=24%:%
%:%97=25%:%
%:%98=25%:%
%:%99=26%:%
%:%100=27%:%
%:%102=29%:%
%:%103=30%:%
%:%104=31%:%
%:%105=31%:%
%:%106=32%:%
%:%107=33%:%
%:%110=36%:%
%:%111=37%:%
%:%112=38%:%
%:%113=38%:%
%:%116=39%:%
%:%120=39%:%
%:%121=39%:%
%:%122=40%:%
%:%123=40%:%
%:%128=40%:%
%:%131=41%:%
%:%132=42%:%
%:%133=43%:%
%:%134=43%:%
%:%135=44%:%
%:%136=45%:%
%:%137=45%:%
%:%138=46%:%
%:%139=47%:%
%:%140=48%:%
%:%141=49%:%
%:%143=51%:%
%:%146=52%:%
%:%150=52%:%
%:%151=52%:%
%:%152=52%:%
%:%153=53%:%
%:%154=53%:%
%:%159=53%:%
%:%162=54%:%
%:%163=55%:%
%:%164=55%:%
%:%165=56%:%
%:%166=57%:%
%:%167=58%:%
%:%168=59%:%
%:%169=60%:%
%:%172=61%:%
%:%176=61%:%
%:%177=61%:%
%:%178=62%:%
%:%179=62%:%
%:%184=62%:%
%:%187=63%:%
%:%188=64%:%
%:%189=64%:%
%:%192=65%:%
%:%196=65%:%
%:%197=65%:%
%:%198=65%:%
%:%203=65%:%
%:%206=66%:%
%:%207=67%:%
%:%208=68%:%
%:%209=68%:%
%:%212=69%:%
%:%216=69%:%
%:%217=69%:%
%:%218=70%:%
%:%219=70%:%
%:%224=70%:%
%:%227=71%:%
%:%228=72%:%
%:%229=72%:%
%:%230=73%:%
%:%237=75%:%
%:%247=77%:%
%:%248=77%:%
%:%249=78%:%
%:%250=79%:%
%:%253=80%:%
%:%257=80%:%
%:%258=80%:%
%:%259=81%:%
%:%260=81%:%
%:%261=82%:%
%:%262=82%:%
%:%263=83%:%
%:%264=83%:%
%:%265=84%:%
%:%266=84%:%
%:%267=84%:%
%:%268=85%:%
%:%269=85%:%
%:%270=86%:%
%:%271=86%:%
%:%272=86%:%
%:%273=86%:%
%:%274=87%:%
%:%275=87%:%
%:%276=88%:%
%:%277=88%:%
%:%278=89%:%
%:%279=89%:%
%:%280=90%:%
%:%281=90%:%
%:%282=91%:%
%:%283=91%:%
%:%284=92%:%
%:%285=92%:%
%:%286=93%:%
%:%287=93%:%
%:%288=93%:%
%:%289=94%:%
%:%290=94%:%
%:%291=95%:%
%:%292=95%:%
%:%293=96%:%
%:%294=96%:%
%:%295=97%:%
%:%296=97%:%
%:%297=97%:%
%:%298=98%:%
%:%299=98%:%
%:%300=99%:%
%:%301=99%:%
%:%302=100%:%
%:%303=100%:%
%:%304=101%:%
%:%305=101%:%
%:%306=102%:%
%:%307=102%:%
%:%308=103%:%
%:%309=103%:%
%:%310=104%:%
%:%311=104%:%
%:%312=105%:%
%:%313=105%:%
%:%314=106%:%
%:%315=106%:%
%:%316=107%:%
%:%317=107%:%
%:%318=108%:%
%:%319=108%:%
%:%320=109%:%
%:%321=109%:%
%:%322=110%:%
%:%323=110%:%
%:%324=110%:%
%:%325=111%:%
%:%326=111%:%
%:%327=112%:%
%:%328=112%:%
%:%329=113%:%
%:%330=113%:%
%:%331=114%:%
%:%332=114%:%
%:%333=115%:%
%:%334=115%:%
%:%335=115%:%
%:%336=116%:%
%:%337=116%:%
%:%338=117%:%
%:%339=117%:%
%:%340=118%:%
%:%341=118%:%
%:%342=118%:%
%:%343=119%:%
%:%344=119%:%
%:%345=120%:%
%:%346=120%:%
%:%347=120%:%
%:%348=120%:%
%:%349=121%:%
%:%350=121%:%
%:%351=122%:%
%:%352=122%:%
%:%353=123%:%
%:%354=123%:%
%:%355=123%:%
%:%356=124%:%
%:%362=124%:%
%:%365=125%:%
%:%366=126%:%
%:%367=126%:%
%:%368=127%:%
%:%369=128%:%
%:%372=129%:%
%:%376=129%:%
%:%377=129%:%
%:%378=129%:%
%:%383=129%:%
%:%386=130%:%
%:%387=131%:%
%:%388=131%:%
%:%391=132%:%
%:%396=133%:%

%
\begin{isabellebody}%
\setisabellecontext{Replacement{\isacharunderscore}{\kern0pt}Axiom}%
%
\isadelimdocument
%
\endisadelimdocument
%
\isatagdocument
%
\isamarkupsection{The Axiom of Replacement in $M[G]$%
}
\isamarkuptrue%
%
\endisatagdocument
{\isafolddocument}%
%
\isadelimdocument
%
\endisadelimdocument
%
\isadelimtheory
%
\endisadelimtheory
%
\isatagtheory
\isacommand{theory}\isamarkupfalse%
\ Replacement{\isacharunderscore}{\kern0pt}Axiom\isanewline
\ \ \isakeyword{imports}\isanewline
\ \ \ \ Least\ Relative{\isacharunderscore}{\kern0pt}Univ\ Separation{\isacharunderscore}{\kern0pt}Axiom\ Renaming{\isacharunderscore}{\kern0pt}Auto\isanewline
\isakeyword{begin}%
\endisatagtheory
{\isafoldtheory}%
%
\isadelimtheory
\isanewline
%
\endisadelimtheory
%
\isadelimML
\isanewline
%
\endisadelimML
%
\isatagML
\isacommand{rename}\isamarkupfalse%
\ {\isachardoublequoteopen}renrep{\isadigit{1}}{\isachardoublequoteclose}\ \isakeyword{src}\ {\isachardoublequoteopen}{\isacharbrackleft}{\kern0pt}p{\isacharcomma}{\kern0pt}P{\isacharcomma}{\kern0pt}leq{\isacharcomma}{\kern0pt}o{\isacharcomma}{\kern0pt}{\isasymrho}{\isacharcomma}{\kern0pt}{\isasymtau}{\isacharbrackright}{\kern0pt}{\isachardoublequoteclose}\ \isakeyword{tgt}\ {\isachardoublequoteopen}{\isacharbrackleft}{\kern0pt}V{\isacharcomma}{\kern0pt}{\isasymtau}{\isacharcomma}{\kern0pt}{\isasymrho}{\isacharcomma}{\kern0pt}p{\isacharcomma}{\kern0pt}{\isasymalpha}{\isacharcomma}{\kern0pt}P{\isacharcomma}{\kern0pt}leq{\isacharcomma}{\kern0pt}o{\isacharbrackright}{\kern0pt}{\isachardoublequoteclose}%
\endisatagML
{\isafoldML}%
%
\isadelimML
\isanewline
%
\endisadelimML
\isanewline
\isacommand{definition}\isamarkupfalse%
\ renrep{\isacharunderscore}{\kern0pt}fn\ {\isacharcolon}{\kern0pt}{\isacharcolon}{\kern0pt}\ {\isachardoublequoteopen}i\ {\isasymRightarrow}\ i{\isachardoublequoteclose}\ \isakeyword{where}\isanewline
\ \ {\isachardoublequoteopen}renrep{\isacharunderscore}{\kern0pt}fn{\isacharparenleft}{\kern0pt}env{\isacharparenright}{\kern0pt}\ {\isasymequiv}\ sum{\isacharparenleft}{\kern0pt}renrep{\isadigit{1}}{\isacharunderscore}{\kern0pt}fn{\isacharcomma}{\kern0pt}id{\isacharparenleft}{\kern0pt}length{\isacharparenleft}{\kern0pt}env{\isacharparenright}{\kern0pt}{\isacharparenright}{\kern0pt}{\isacharcomma}{\kern0pt}{\isadigit{6}}{\isacharcomma}{\kern0pt}{\isadigit{8}}{\isacharcomma}{\kern0pt}length{\isacharparenleft}{\kern0pt}env{\isacharparenright}{\kern0pt}{\isacharparenright}{\kern0pt}{\isachardoublequoteclose}\isanewline
\isanewline
\isacommand{definition}\isamarkupfalse%
\isanewline
\ \ renrep\ {\isacharcolon}{\kern0pt}{\isacharcolon}{\kern0pt}\ {\isachardoublequoteopen}{\isacharbrackleft}{\kern0pt}i{\isacharcomma}{\kern0pt}i{\isacharbrackright}{\kern0pt}\ {\isasymRightarrow}\ i{\isachardoublequoteclose}\ \isakeyword{where}\isanewline
\ \ {\isachardoublequoteopen}renrep{\isacharparenleft}{\kern0pt}{\isasymphi}{\isacharcomma}{\kern0pt}env{\isacharparenright}{\kern0pt}\ {\isacharequal}{\kern0pt}\ ren{\isacharparenleft}{\kern0pt}{\isasymphi}{\isacharparenright}{\kern0pt}{\isacharbackquote}{\kern0pt}{\isacharparenleft}{\kern0pt}{\isadigit{6}}{\isacharhash}{\kern0pt}{\isacharplus}{\kern0pt}length{\isacharparenleft}{\kern0pt}env{\isacharparenright}{\kern0pt}{\isacharparenright}{\kern0pt}{\isacharbackquote}{\kern0pt}{\isacharparenleft}{\kern0pt}{\isadigit{8}}{\isacharhash}{\kern0pt}{\isacharplus}{\kern0pt}length{\isacharparenleft}{\kern0pt}env{\isacharparenright}{\kern0pt}{\isacharparenright}{\kern0pt}{\isacharbackquote}{\kern0pt}renrep{\isacharunderscore}{\kern0pt}fn{\isacharparenleft}{\kern0pt}env{\isacharparenright}{\kern0pt}{\isachardoublequoteclose}\isanewline
\isanewline
\isacommand{lemma}\isamarkupfalse%
\ renrep{\isacharunderscore}{\kern0pt}type\ {\isacharbrackleft}{\kern0pt}TC{\isacharbrackright}{\kern0pt}{\isacharcolon}{\kern0pt}\isanewline
\ \ \isakeyword{assumes}\ {\isachardoublequoteopen}{\isasymphi}{\isasymin}formula{\isachardoublequoteclose}\ {\isachardoublequoteopen}env\ {\isasymin}\ list{\isacharparenleft}{\kern0pt}M{\isacharparenright}{\kern0pt}{\isachardoublequoteclose}\isanewline
\ \ \isakeyword{shows}\ {\isachardoublequoteopen}renrep{\isacharparenleft}{\kern0pt}{\isasymphi}{\isacharcomma}{\kern0pt}env{\isacharparenright}{\kern0pt}\ {\isasymin}\ formula{\isachardoublequoteclose}\isanewline
%
\isadelimproof
\ \ %
\endisadelimproof
%
\isatagproof
\isacommand{unfolding}\isamarkupfalse%
\ renrep{\isacharunderscore}{\kern0pt}def\ renrep{\isacharunderscore}{\kern0pt}fn{\isacharunderscore}{\kern0pt}def\ renrep{\isadigit{1}}{\isacharunderscore}{\kern0pt}fn{\isacharunderscore}{\kern0pt}def\isanewline
\ \ \isacommand{using}\isamarkupfalse%
\ assms\ renrep{\isadigit{1}}{\isacharunderscore}{\kern0pt}thm{\isacharparenleft}{\kern0pt}{\isadigit{1}}{\isacharparenright}{\kern0pt}\ ren{\isacharunderscore}{\kern0pt}tc\isanewline
\ \ \isacommand{by}\isamarkupfalse%
\ simp%
\endisatagproof
{\isafoldproof}%
%
\isadelimproof
\isanewline
%
\endisadelimproof
\isanewline
\isacommand{lemma}\isamarkupfalse%
\ arity{\isacharunderscore}{\kern0pt}renrep{\isacharcolon}{\kern0pt}\isanewline
\ \ \isakeyword{assumes}\ \ {\isachardoublequoteopen}{\isasymphi}{\isasymin}formula{\isachardoublequoteclose}\ {\isachardoublequoteopen}arity{\isacharparenleft}{\kern0pt}{\isasymphi}{\isacharparenright}{\kern0pt}{\isasymle}\ {\isadigit{6}}{\isacharhash}{\kern0pt}{\isacharplus}{\kern0pt}length{\isacharparenleft}{\kern0pt}env{\isacharparenright}{\kern0pt}{\isachardoublequoteclose}\ {\isachardoublequoteopen}env\ {\isasymin}\ list{\isacharparenleft}{\kern0pt}M{\isacharparenright}{\kern0pt}{\isachardoublequoteclose}\isanewline
\ \ \isakeyword{shows}\ {\isachardoublequoteopen}arity{\isacharparenleft}{\kern0pt}renrep{\isacharparenleft}{\kern0pt}{\isasymphi}{\isacharcomma}{\kern0pt}env{\isacharparenright}{\kern0pt}{\isacharparenright}{\kern0pt}\ {\isasymle}\ {\isadigit{8}}{\isacharhash}{\kern0pt}{\isacharplus}{\kern0pt}length{\isacharparenleft}{\kern0pt}env{\isacharparenright}{\kern0pt}{\isachardoublequoteclose}\isanewline
%
\isadelimproof
\ \ %
\endisadelimproof
%
\isatagproof
\isacommand{unfolding}\isamarkupfalse%
\ \ renrep{\isacharunderscore}{\kern0pt}def\ renrep{\isacharunderscore}{\kern0pt}fn{\isacharunderscore}{\kern0pt}def\ renrep{\isadigit{1}}{\isacharunderscore}{\kern0pt}fn{\isacharunderscore}{\kern0pt}def\isanewline
\ \ \isacommand{using}\isamarkupfalse%
\ assms\ renrep{\isadigit{1}}{\isacharunderscore}{\kern0pt}thm{\isacharparenleft}{\kern0pt}{\isadigit{1}}{\isacharparenright}{\kern0pt}\ arity{\isacharunderscore}{\kern0pt}ren\isanewline
\ \ \isacommand{by}\isamarkupfalse%
\ simp%
\endisatagproof
{\isafoldproof}%
%
\isadelimproof
\isanewline
%
\endisadelimproof
\isanewline
\isacommand{lemma}\isamarkupfalse%
\ renrep{\isacharunderscore}{\kern0pt}sats\ {\isacharcolon}{\kern0pt}\isanewline
\ \ \isakeyword{assumes}\ \ {\isachardoublequoteopen}arity{\isacharparenleft}{\kern0pt}{\isasymphi}{\isacharparenright}{\kern0pt}\ {\isasymle}\ {\isadigit{6}}\ {\isacharhash}{\kern0pt}{\isacharplus}{\kern0pt}\ length{\isacharparenleft}{\kern0pt}env{\isacharparenright}{\kern0pt}{\isachardoublequoteclose}\isanewline
\ \ \ \ \ \ \ \ \ \ {\isachardoublequoteopen}{\isacharbrackleft}{\kern0pt}P{\isacharcomma}{\kern0pt}leq{\isacharcomma}{\kern0pt}o{\isacharcomma}{\kern0pt}p{\isacharcomma}{\kern0pt}{\isasymrho}{\isacharcomma}{\kern0pt}{\isasymtau}{\isacharbrackright}{\kern0pt}\ {\isacharat}{\kern0pt}\ env\ {\isasymin}\ list{\isacharparenleft}{\kern0pt}M{\isacharparenright}{\kern0pt}{\isachardoublequoteclose}\isanewline
\ \ \ \ {\isachardoublequoteopen}V\ {\isasymin}\ M{\isachardoublequoteclose}\ {\isachardoublequoteopen}{\isasymalpha}\ {\isasymin}\ M{\isachardoublequoteclose}\isanewline
\ \ \ \ {\isachardoublequoteopen}{\isasymphi}{\isasymin}formula{\isachardoublequoteclose}\isanewline
\ \ \isakeyword{shows}\ {\isachardoublequoteopen}sats{\isacharparenleft}{\kern0pt}M{\isacharcomma}{\kern0pt}\ {\isasymphi}{\isacharcomma}{\kern0pt}\ {\isacharbrackleft}{\kern0pt}p{\isacharcomma}{\kern0pt}P{\isacharcomma}{\kern0pt}leq{\isacharcomma}{\kern0pt}o{\isacharcomma}{\kern0pt}{\isasymrho}{\isacharcomma}{\kern0pt}{\isasymtau}{\isacharbrackright}{\kern0pt}\ {\isacharat}{\kern0pt}\ env{\isacharparenright}{\kern0pt}\ {\isasymlongleftrightarrow}\ sats{\isacharparenleft}{\kern0pt}M{\isacharcomma}{\kern0pt}\ renrep{\isacharparenleft}{\kern0pt}{\isasymphi}{\isacharcomma}{\kern0pt}env{\isacharparenright}{\kern0pt}{\isacharcomma}{\kern0pt}\ {\isacharbrackleft}{\kern0pt}V{\isacharcomma}{\kern0pt}{\isasymtau}{\isacharcomma}{\kern0pt}{\isasymrho}{\isacharcomma}{\kern0pt}p{\isacharcomma}{\kern0pt}{\isasymalpha}{\isacharcomma}{\kern0pt}P{\isacharcomma}{\kern0pt}leq{\isacharcomma}{\kern0pt}o{\isacharbrackright}{\kern0pt}\ {\isacharat}{\kern0pt}\ env{\isacharparenright}{\kern0pt}{\isachardoublequoteclose}\isanewline
%
\isadelimproof
\ \ %
\endisadelimproof
%
\isatagproof
\isacommand{unfolding}\isamarkupfalse%
\ \ renrep{\isacharunderscore}{\kern0pt}def\ renrep{\isacharunderscore}{\kern0pt}fn{\isacharunderscore}{\kern0pt}def\ renrep{\isadigit{1}}{\isacharunderscore}{\kern0pt}fn{\isacharunderscore}{\kern0pt}def\isanewline
\ \ \isacommand{by}\isamarkupfalse%
\ {\isacharparenleft}{\kern0pt}rule\ sats{\isacharunderscore}{\kern0pt}iff{\isacharunderscore}{\kern0pt}sats{\isacharunderscore}{\kern0pt}ren{\isacharcomma}{\kern0pt}insert\ assms{\isacharcomma}{\kern0pt}\ auto\ simp\ add{\isacharcolon}{\kern0pt}renrep{\isadigit{1}}{\isacharunderscore}{\kern0pt}thm{\isacharparenleft}{\kern0pt}{\isadigit{1}}{\isacharparenright}{\kern0pt}{\isacharbrackleft}{\kern0pt}of\ {\isacharunderscore}{\kern0pt}\ M{\isacharcomma}{\kern0pt}simplified{\isacharbrackright}{\kern0pt}\isanewline
\ \ \ \ \ \ \ \ renrep{\isadigit{1}}{\isacharunderscore}{\kern0pt}thm{\isacharparenleft}{\kern0pt}{\isadigit{2}}{\isacharparenright}{\kern0pt}{\isacharbrackleft}{\kern0pt}simplified{\isacharcomma}{\kern0pt}\isakeyword{where}\ p{\isacharequal}{\kern0pt}p\ \isakeyword{and}\ {\isasymalpha}{\isacharequal}{\kern0pt}{\isasymalpha}{\isacharbrackright}{\kern0pt}{\isacharparenright}{\kern0pt}%
\endisatagproof
{\isafoldproof}%
%
\isadelimproof
\isanewline
%
\endisadelimproof
%
\isadelimML
\isanewline
%
\endisadelimML
%
\isatagML
\isacommand{rename}\isamarkupfalse%
\ {\isachardoublequoteopen}renpbdy{\isadigit{1}}{\isachardoublequoteclose}\ \isakeyword{src}\ {\isachardoublequoteopen}{\isacharbrackleft}{\kern0pt}{\isasymrho}{\isacharcomma}{\kern0pt}p{\isacharcomma}{\kern0pt}{\isasymalpha}{\isacharcomma}{\kern0pt}P{\isacharcomma}{\kern0pt}leq{\isacharcomma}{\kern0pt}o{\isacharbrackright}{\kern0pt}{\isachardoublequoteclose}\ \isakeyword{tgt}\ {\isachardoublequoteopen}{\isacharbrackleft}{\kern0pt}{\isasymrho}{\isacharcomma}{\kern0pt}p{\isacharcomma}{\kern0pt}x{\isacharcomma}{\kern0pt}{\isasymalpha}{\isacharcomma}{\kern0pt}P{\isacharcomma}{\kern0pt}leq{\isacharcomma}{\kern0pt}o{\isacharbrackright}{\kern0pt}{\isachardoublequoteclose}%
\endisatagML
{\isafoldML}%
%
\isadelimML
\isanewline
%
\endisadelimML
\isanewline
\isacommand{definition}\isamarkupfalse%
\ renpbdy{\isacharunderscore}{\kern0pt}fn\ {\isacharcolon}{\kern0pt}{\isacharcolon}{\kern0pt}\ {\isachardoublequoteopen}i\ {\isasymRightarrow}\ i{\isachardoublequoteclose}\ \isakeyword{where}\isanewline
\ \ {\isachardoublequoteopen}renpbdy{\isacharunderscore}{\kern0pt}fn{\isacharparenleft}{\kern0pt}env{\isacharparenright}{\kern0pt}\ {\isasymequiv}\ sum{\isacharparenleft}{\kern0pt}renpbdy{\isadigit{1}}{\isacharunderscore}{\kern0pt}fn{\isacharcomma}{\kern0pt}id{\isacharparenleft}{\kern0pt}length{\isacharparenleft}{\kern0pt}env{\isacharparenright}{\kern0pt}{\isacharparenright}{\kern0pt}{\isacharcomma}{\kern0pt}{\isadigit{6}}{\isacharcomma}{\kern0pt}{\isadigit{7}}{\isacharcomma}{\kern0pt}length{\isacharparenleft}{\kern0pt}env{\isacharparenright}{\kern0pt}{\isacharparenright}{\kern0pt}{\isachardoublequoteclose}\isanewline
\isanewline
\isacommand{definition}\isamarkupfalse%
\isanewline
\ \ renpbdy\ {\isacharcolon}{\kern0pt}{\isacharcolon}{\kern0pt}\ {\isachardoublequoteopen}{\isacharbrackleft}{\kern0pt}i{\isacharcomma}{\kern0pt}i{\isacharbrackright}{\kern0pt}\ {\isasymRightarrow}\ i{\isachardoublequoteclose}\ \isakeyword{where}\isanewline
\ \ {\isachardoublequoteopen}renpbdy{\isacharparenleft}{\kern0pt}{\isasymphi}{\isacharcomma}{\kern0pt}env{\isacharparenright}{\kern0pt}\ {\isacharequal}{\kern0pt}\ ren{\isacharparenleft}{\kern0pt}{\isasymphi}{\isacharparenright}{\kern0pt}{\isacharbackquote}{\kern0pt}{\isacharparenleft}{\kern0pt}{\isadigit{6}}{\isacharhash}{\kern0pt}{\isacharplus}{\kern0pt}length{\isacharparenleft}{\kern0pt}env{\isacharparenright}{\kern0pt}{\isacharparenright}{\kern0pt}{\isacharbackquote}{\kern0pt}{\isacharparenleft}{\kern0pt}{\isadigit{7}}{\isacharhash}{\kern0pt}{\isacharplus}{\kern0pt}length{\isacharparenleft}{\kern0pt}env{\isacharparenright}{\kern0pt}{\isacharparenright}{\kern0pt}{\isacharbackquote}{\kern0pt}renpbdy{\isacharunderscore}{\kern0pt}fn{\isacharparenleft}{\kern0pt}env{\isacharparenright}{\kern0pt}{\isachardoublequoteclose}\isanewline
\isanewline
\isanewline
\isacommand{lemma}\isamarkupfalse%
\isanewline
\ \ renpbdy{\isacharunderscore}{\kern0pt}type\ {\isacharbrackleft}{\kern0pt}TC{\isacharbrackright}{\kern0pt}{\isacharcolon}{\kern0pt}\ {\isachardoublequoteopen}{\isasymphi}{\isasymin}formula\ {\isasymLongrightarrow}\ env{\isasymin}list{\isacharparenleft}{\kern0pt}M{\isacharparenright}{\kern0pt}\ {\isasymLongrightarrow}\ renpbdy{\isacharparenleft}{\kern0pt}{\isasymphi}{\isacharcomma}{\kern0pt}env{\isacharparenright}{\kern0pt}\ {\isasymin}\ formula{\isachardoublequoteclose}\isanewline
%
\isadelimproof
\ \ %
\endisadelimproof
%
\isatagproof
\isacommand{unfolding}\isamarkupfalse%
\ renpbdy{\isacharunderscore}{\kern0pt}def\ renpbdy{\isacharunderscore}{\kern0pt}fn{\isacharunderscore}{\kern0pt}def\ renpbdy{\isadigit{1}}{\isacharunderscore}{\kern0pt}fn{\isacharunderscore}{\kern0pt}def\isanewline
\ \ \isacommand{using}\isamarkupfalse%
\ \ renpbdy{\isadigit{1}}{\isacharunderscore}{\kern0pt}thm{\isacharparenleft}{\kern0pt}{\isadigit{1}}{\isacharparenright}{\kern0pt}\ ren{\isacharunderscore}{\kern0pt}tc\isanewline
\ \ \isacommand{by}\isamarkupfalse%
\ simp%
\endisatagproof
{\isafoldproof}%
%
\isadelimproof
\isanewline
%
\endisadelimproof
\isanewline
\isacommand{lemma}\isamarkupfalse%
\ \ arity{\isacharunderscore}{\kern0pt}renpbdy{\isacharcolon}{\kern0pt}\ {\isachardoublequoteopen}{\isasymphi}{\isasymin}formula\ {\isasymLongrightarrow}\ arity{\isacharparenleft}{\kern0pt}{\isasymphi}{\isacharparenright}{\kern0pt}\ {\isasymle}\ {\isadigit{6}}\ {\isacharhash}{\kern0pt}{\isacharplus}{\kern0pt}\ length{\isacharparenleft}{\kern0pt}env{\isacharparenright}{\kern0pt}\ {\isasymLongrightarrow}\ env{\isasymin}list{\isacharparenleft}{\kern0pt}M{\isacharparenright}{\kern0pt}\ {\isasymLongrightarrow}\ arity{\isacharparenleft}{\kern0pt}renpbdy{\isacharparenleft}{\kern0pt}{\isasymphi}{\isacharcomma}{\kern0pt}env{\isacharparenright}{\kern0pt}{\isacharparenright}{\kern0pt}\ {\isasymle}\ {\isadigit{7}}\ {\isacharhash}{\kern0pt}{\isacharplus}{\kern0pt}\ length{\isacharparenleft}{\kern0pt}env{\isacharparenright}{\kern0pt}{\isachardoublequoteclose}\isanewline
%
\isadelimproof
\ \ %
\endisadelimproof
%
\isatagproof
\isacommand{unfolding}\isamarkupfalse%
\ renpbdy{\isacharunderscore}{\kern0pt}def\ renpbdy{\isacharunderscore}{\kern0pt}fn{\isacharunderscore}{\kern0pt}def\ renpbdy{\isadigit{1}}{\isacharunderscore}{\kern0pt}fn{\isacharunderscore}{\kern0pt}def\isanewline
\ \ \isacommand{using}\isamarkupfalse%
\ \ renpbdy{\isadigit{1}}{\isacharunderscore}{\kern0pt}thm{\isacharparenleft}{\kern0pt}{\isadigit{1}}{\isacharparenright}{\kern0pt}\ arity{\isacharunderscore}{\kern0pt}ren\isanewline
\ \ \isacommand{by}\isamarkupfalse%
\ simp%
\endisatagproof
{\isafoldproof}%
%
\isadelimproof
\isanewline
%
\endisadelimproof
\isanewline
\isacommand{lemma}\isamarkupfalse%
\isanewline
\ \ sats{\isacharunderscore}{\kern0pt}renpbdy{\isacharcolon}{\kern0pt}\ {\isachardoublequoteopen}arity{\isacharparenleft}{\kern0pt}{\isasymphi}{\isacharparenright}{\kern0pt}\ {\isasymle}\ {\isadigit{6}}\ {\isacharhash}{\kern0pt}{\isacharplus}{\kern0pt}\ length{\isacharparenleft}{\kern0pt}nenv{\isacharparenright}{\kern0pt}\ {\isasymLongrightarrow}\ {\isacharbrackleft}{\kern0pt}{\isasymrho}{\isacharcomma}{\kern0pt}p{\isacharcomma}{\kern0pt}x{\isacharcomma}{\kern0pt}{\isasymalpha}{\isacharcomma}{\kern0pt}P{\isacharcomma}{\kern0pt}leq{\isacharcomma}{\kern0pt}o{\isacharcomma}{\kern0pt}{\isasympi}{\isacharbrackright}{\kern0pt}\ {\isacharat}{\kern0pt}\ nenv\ {\isasymin}\ list{\isacharparenleft}{\kern0pt}M{\isacharparenright}{\kern0pt}\ {\isasymLongrightarrow}\ {\isasymphi}{\isasymin}formula\ {\isasymLongrightarrow}\isanewline
\ \ \ \ \ \ \ sats{\isacharparenleft}{\kern0pt}M{\isacharcomma}{\kern0pt}\ {\isasymphi}{\isacharcomma}{\kern0pt}\ {\isacharbrackleft}{\kern0pt}{\isasymrho}{\isacharcomma}{\kern0pt}p{\isacharcomma}{\kern0pt}{\isasymalpha}{\isacharcomma}{\kern0pt}P{\isacharcomma}{\kern0pt}leq{\isacharcomma}{\kern0pt}o{\isacharbrackright}{\kern0pt}\ {\isacharat}{\kern0pt}\ nenv{\isacharparenright}{\kern0pt}\ {\isasymlongleftrightarrow}\ sats{\isacharparenleft}{\kern0pt}M{\isacharcomma}{\kern0pt}\ renpbdy{\isacharparenleft}{\kern0pt}{\isasymphi}{\isacharcomma}{\kern0pt}nenv{\isacharparenright}{\kern0pt}{\isacharcomma}{\kern0pt}\ {\isacharbrackleft}{\kern0pt}{\isasymrho}{\isacharcomma}{\kern0pt}p{\isacharcomma}{\kern0pt}x{\isacharcomma}{\kern0pt}{\isasymalpha}{\isacharcomma}{\kern0pt}P{\isacharcomma}{\kern0pt}leq{\isacharcomma}{\kern0pt}o{\isacharbrackright}{\kern0pt}\ {\isacharat}{\kern0pt}\ nenv{\isacharparenright}{\kern0pt}{\isachardoublequoteclose}\isanewline
%
\isadelimproof
\ \ %
\endisadelimproof
%
\isatagproof
\isacommand{unfolding}\isamarkupfalse%
\ renpbdy{\isacharunderscore}{\kern0pt}def\ renpbdy{\isacharunderscore}{\kern0pt}fn{\isacharunderscore}{\kern0pt}def\ renpbdy{\isadigit{1}}{\isacharunderscore}{\kern0pt}fn{\isacharunderscore}{\kern0pt}def\isanewline
\ \ \isacommand{by}\isamarkupfalse%
\ {\isacharparenleft}{\kern0pt}rule\ sats{\isacharunderscore}{\kern0pt}iff{\isacharunderscore}{\kern0pt}sats{\isacharunderscore}{\kern0pt}ren{\isacharcomma}{\kern0pt}auto\ simp\ add{\isacharcolon}{\kern0pt}\ renpbdy{\isadigit{1}}{\isacharunderscore}{\kern0pt}thm{\isacharparenleft}{\kern0pt}{\isadigit{1}}{\isacharparenright}{\kern0pt}{\isacharbrackleft}{\kern0pt}of\ {\isacharunderscore}{\kern0pt}\ M{\isacharcomma}{\kern0pt}simplified{\isacharbrackright}{\kern0pt}\isanewline
\ \ \ \ \ \ \ \ \ \ \ \ \ \ \ \ \ \ \ \ \ \ \ \ \ \ \ \ \ \ \ \ \ \ \ \ \ \ \ \ \ \ \ \ renpbdy{\isadigit{1}}{\isacharunderscore}{\kern0pt}thm{\isacharparenleft}{\kern0pt}{\isadigit{2}}{\isacharparenright}{\kern0pt}{\isacharbrackleft}{\kern0pt}simplified{\isacharcomma}{\kern0pt}\isakeyword{where}\ {\isasymalpha}{\isacharequal}{\kern0pt}{\isasymalpha}\ \isakeyword{and}\ x{\isacharequal}{\kern0pt}x{\isacharbrackright}{\kern0pt}{\isacharparenright}{\kern0pt}%
\endisatagproof
{\isafoldproof}%
%
\isadelimproof
\isanewline
%
\endisadelimproof
\isanewline
%
\isadelimML
\isanewline
%
\endisadelimML
%
\isatagML
\isacommand{rename}\isamarkupfalse%
\ {\isachardoublequoteopen}renbody{\isadigit{1}}{\isachardoublequoteclose}\ \isakeyword{src}\ {\isachardoublequoteopen}{\isacharbrackleft}{\kern0pt}x{\isacharcomma}{\kern0pt}{\isasymalpha}{\isacharcomma}{\kern0pt}P{\isacharcomma}{\kern0pt}leq{\isacharcomma}{\kern0pt}o{\isacharbrackright}{\kern0pt}{\isachardoublequoteclose}\ \isakeyword{tgt}\ {\isachardoublequoteopen}{\isacharbrackleft}{\kern0pt}{\isasymalpha}{\isacharcomma}{\kern0pt}x{\isacharcomma}{\kern0pt}m{\isacharcomma}{\kern0pt}P{\isacharcomma}{\kern0pt}leq{\isacharcomma}{\kern0pt}o{\isacharbrackright}{\kern0pt}{\isachardoublequoteclose}%
\endisatagML
{\isafoldML}%
%
\isadelimML
\isanewline
%
\endisadelimML
\isanewline
\isacommand{definition}\isamarkupfalse%
\ renbody{\isacharunderscore}{\kern0pt}fn\ {\isacharcolon}{\kern0pt}{\isacharcolon}{\kern0pt}\ {\isachardoublequoteopen}i\ {\isasymRightarrow}\ i{\isachardoublequoteclose}\ \isakeyword{where}\isanewline
\ \ {\isachardoublequoteopen}renbody{\isacharunderscore}{\kern0pt}fn{\isacharparenleft}{\kern0pt}env{\isacharparenright}{\kern0pt}\ {\isasymequiv}\ sum{\isacharparenleft}{\kern0pt}renbody{\isadigit{1}}{\isacharunderscore}{\kern0pt}fn{\isacharcomma}{\kern0pt}id{\isacharparenleft}{\kern0pt}length{\isacharparenleft}{\kern0pt}env{\isacharparenright}{\kern0pt}{\isacharparenright}{\kern0pt}{\isacharcomma}{\kern0pt}{\isadigit{5}}{\isacharcomma}{\kern0pt}{\isadigit{6}}{\isacharcomma}{\kern0pt}length{\isacharparenleft}{\kern0pt}env{\isacharparenright}{\kern0pt}{\isacharparenright}{\kern0pt}{\isachardoublequoteclose}\isanewline
\isanewline
\isacommand{definition}\isamarkupfalse%
\isanewline
\ \ renbody\ {\isacharcolon}{\kern0pt}{\isacharcolon}{\kern0pt}\ {\isachardoublequoteopen}{\isacharbrackleft}{\kern0pt}i{\isacharcomma}{\kern0pt}i{\isacharbrackright}{\kern0pt}\ {\isasymRightarrow}\ i{\isachardoublequoteclose}\ \isakeyword{where}\isanewline
\ \ {\isachardoublequoteopen}renbody{\isacharparenleft}{\kern0pt}{\isasymphi}{\isacharcomma}{\kern0pt}env{\isacharparenright}{\kern0pt}\ {\isacharequal}{\kern0pt}\ ren{\isacharparenleft}{\kern0pt}{\isasymphi}{\isacharparenright}{\kern0pt}{\isacharbackquote}{\kern0pt}{\isacharparenleft}{\kern0pt}{\isadigit{5}}{\isacharhash}{\kern0pt}{\isacharplus}{\kern0pt}length{\isacharparenleft}{\kern0pt}env{\isacharparenright}{\kern0pt}{\isacharparenright}{\kern0pt}{\isacharbackquote}{\kern0pt}{\isacharparenleft}{\kern0pt}{\isadigit{6}}{\isacharhash}{\kern0pt}{\isacharplus}{\kern0pt}length{\isacharparenleft}{\kern0pt}env{\isacharparenright}{\kern0pt}{\isacharparenright}{\kern0pt}{\isacharbackquote}{\kern0pt}renbody{\isacharunderscore}{\kern0pt}fn{\isacharparenleft}{\kern0pt}env{\isacharparenright}{\kern0pt}{\isachardoublequoteclose}\isanewline
\isanewline
\isacommand{lemma}\isamarkupfalse%
\isanewline
\ \ renbody{\isacharunderscore}{\kern0pt}type\ {\isacharbrackleft}{\kern0pt}TC{\isacharbrackright}{\kern0pt}{\isacharcolon}{\kern0pt}\ {\isachardoublequoteopen}{\isasymphi}{\isasymin}formula\ {\isasymLongrightarrow}\ env{\isasymin}list{\isacharparenleft}{\kern0pt}M{\isacharparenright}{\kern0pt}\ {\isasymLongrightarrow}\ renbody{\isacharparenleft}{\kern0pt}{\isasymphi}{\isacharcomma}{\kern0pt}env{\isacharparenright}{\kern0pt}\ {\isasymin}\ formula{\isachardoublequoteclose}\isanewline
%
\isadelimproof
\ \ %
\endisadelimproof
%
\isatagproof
\isacommand{unfolding}\isamarkupfalse%
\ renbody{\isacharunderscore}{\kern0pt}def\ renbody{\isacharunderscore}{\kern0pt}fn{\isacharunderscore}{\kern0pt}def\ renbody{\isadigit{1}}{\isacharunderscore}{\kern0pt}fn{\isacharunderscore}{\kern0pt}def\isanewline
\ \ \isacommand{using}\isamarkupfalse%
\ \ renbody{\isadigit{1}}{\isacharunderscore}{\kern0pt}thm{\isacharparenleft}{\kern0pt}{\isadigit{1}}{\isacharparenright}{\kern0pt}\ ren{\isacharunderscore}{\kern0pt}tc\isanewline
\ \ \isacommand{by}\isamarkupfalse%
\ simp%
\endisatagproof
{\isafoldproof}%
%
\isadelimproof
\isanewline
%
\endisadelimproof
\isanewline
\isacommand{lemma}\isamarkupfalse%
\ \ arity{\isacharunderscore}{\kern0pt}renbody{\isacharcolon}{\kern0pt}\ {\isachardoublequoteopen}{\isasymphi}{\isasymin}formula\ {\isasymLongrightarrow}\ arity{\isacharparenleft}{\kern0pt}{\isasymphi}{\isacharparenright}{\kern0pt}\ {\isasymle}\ {\isadigit{5}}\ {\isacharhash}{\kern0pt}{\isacharplus}{\kern0pt}\ length{\isacharparenleft}{\kern0pt}env{\isacharparenright}{\kern0pt}\ {\isasymLongrightarrow}\ env{\isasymin}list{\isacharparenleft}{\kern0pt}M{\isacharparenright}{\kern0pt}\ {\isasymLongrightarrow}\isanewline
\ \ arity{\isacharparenleft}{\kern0pt}renbody{\isacharparenleft}{\kern0pt}{\isasymphi}{\isacharcomma}{\kern0pt}env{\isacharparenright}{\kern0pt}{\isacharparenright}{\kern0pt}\ {\isasymle}\ {\isadigit{6}}\ {\isacharhash}{\kern0pt}{\isacharplus}{\kern0pt}\ length{\isacharparenleft}{\kern0pt}env{\isacharparenright}{\kern0pt}{\isachardoublequoteclose}\isanewline
%
\isadelimproof
\ \ %
\endisadelimproof
%
\isatagproof
\isacommand{unfolding}\isamarkupfalse%
\ renbody{\isacharunderscore}{\kern0pt}def\ renbody{\isacharunderscore}{\kern0pt}fn{\isacharunderscore}{\kern0pt}def\ renbody{\isadigit{1}}{\isacharunderscore}{\kern0pt}fn{\isacharunderscore}{\kern0pt}def\isanewline
\ \ \isacommand{using}\isamarkupfalse%
\ \ renbody{\isadigit{1}}{\isacharunderscore}{\kern0pt}thm{\isacharparenleft}{\kern0pt}{\isadigit{1}}{\isacharparenright}{\kern0pt}\ arity{\isacharunderscore}{\kern0pt}ren\isanewline
\ \ \isacommand{by}\isamarkupfalse%
\ simp%
\endisatagproof
{\isafoldproof}%
%
\isadelimproof
\isanewline
%
\endisadelimproof
\isanewline
\isacommand{lemma}\isamarkupfalse%
\isanewline
\ \ sats{\isacharunderscore}{\kern0pt}renbody{\isacharcolon}{\kern0pt}\ {\isachardoublequoteopen}arity{\isacharparenleft}{\kern0pt}{\isasymphi}{\isacharparenright}{\kern0pt}\ {\isasymle}\ {\isadigit{5}}\ {\isacharhash}{\kern0pt}{\isacharplus}{\kern0pt}\ length{\isacharparenleft}{\kern0pt}nenv{\isacharparenright}{\kern0pt}\ {\isasymLongrightarrow}\ {\isacharbrackleft}{\kern0pt}{\isasymalpha}{\isacharcomma}{\kern0pt}x{\isacharcomma}{\kern0pt}m{\isacharcomma}{\kern0pt}P{\isacharcomma}{\kern0pt}leq{\isacharcomma}{\kern0pt}o{\isacharbrackright}{\kern0pt}\ {\isacharat}{\kern0pt}\ nenv\ {\isasymin}\ list{\isacharparenleft}{\kern0pt}M{\isacharparenright}{\kern0pt}\ {\isasymLongrightarrow}\ {\isasymphi}{\isasymin}formula\ {\isasymLongrightarrow}\isanewline
\ \ \ \ \ \ \ sats{\isacharparenleft}{\kern0pt}M{\isacharcomma}{\kern0pt}\ {\isasymphi}{\isacharcomma}{\kern0pt}\ {\isacharbrackleft}{\kern0pt}x{\isacharcomma}{\kern0pt}{\isasymalpha}{\isacharcomma}{\kern0pt}P{\isacharcomma}{\kern0pt}leq{\isacharcomma}{\kern0pt}o{\isacharbrackright}{\kern0pt}\ {\isacharat}{\kern0pt}\ nenv{\isacharparenright}{\kern0pt}\ {\isasymlongleftrightarrow}\ sats{\isacharparenleft}{\kern0pt}M{\isacharcomma}{\kern0pt}\ renbody{\isacharparenleft}{\kern0pt}{\isasymphi}{\isacharcomma}{\kern0pt}nenv{\isacharparenright}{\kern0pt}{\isacharcomma}{\kern0pt}\ {\isacharbrackleft}{\kern0pt}{\isasymalpha}{\isacharcomma}{\kern0pt}x{\isacharcomma}{\kern0pt}m{\isacharcomma}{\kern0pt}P{\isacharcomma}{\kern0pt}leq{\isacharcomma}{\kern0pt}o{\isacharbrackright}{\kern0pt}\ {\isacharat}{\kern0pt}\ nenv{\isacharparenright}{\kern0pt}{\isachardoublequoteclose}\isanewline
%
\isadelimproof
\ \ %
\endisadelimproof
%
\isatagproof
\isacommand{unfolding}\isamarkupfalse%
\ renbody{\isacharunderscore}{\kern0pt}def\ renbody{\isacharunderscore}{\kern0pt}fn{\isacharunderscore}{\kern0pt}def\ renbody{\isadigit{1}}{\isacharunderscore}{\kern0pt}fn{\isacharunderscore}{\kern0pt}def\isanewline
\ \ \isacommand{by}\isamarkupfalse%
\ {\isacharparenleft}{\kern0pt}rule\ sats{\isacharunderscore}{\kern0pt}iff{\isacharunderscore}{\kern0pt}sats{\isacharunderscore}{\kern0pt}ren{\isacharcomma}{\kern0pt}\ auto\ simp\ add{\isacharcolon}{\kern0pt}renbody{\isadigit{1}}{\isacharunderscore}{\kern0pt}thm{\isacharparenleft}{\kern0pt}{\isadigit{1}}{\isacharparenright}{\kern0pt}{\isacharbrackleft}{\kern0pt}of\ {\isacharunderscore}{\kern0pt}\ M{\isacharcomma}{\kern0pt}simplified{\isacharbrackright}{\kern0pt}\isanewline
\ \ \ \ \ \ \ \ \ \ \ \ \ \ \ \ \ \ \ \ \ \ \ \ \ \ \ \ \ \ \ \ \ \ \ \ \ \ \ \ \ \ \ \ renbody{\isadigit{1}}{\isacharunderscore}{\kern0pt}thm{\isacharparenleft}{\kern0pt}{\isadigit{2}}{\isacharparenright}{\kern0pt}{\isacharbrackleft}{\kern0pt}\isakeyword{where}\ {\isasymalpha}{\isacharequal}{\kern0pt}{\isasymalpha}\ \isakeyword{and}\ m{\isacharequal}{\kern0pt}m{\isacharcomma}{\kern0pt}simplified{\isacharbrackright}{\kern0pt}{\isacharparenright}{\kern0pt}%
\endisatagproof
{\isafoldproof}%
%
\isadelimproof
\isanewline
%
\endisadelimproof
\isanewline
\isacommand{context}\isamarkupfalse%
\ G{\isacharunderscore}{\kern0pt}generic\isanewline
\isakeyword{begin}\isanewline
\isanewline
\isacommand{lemma}\isamarkupfalse%
\ pow{\isacharunderscore}{\kern0pt}inter{\isacharunderscore}{\kern0pt}M{\isacharcolon}{\kern0pt}\isanewline
\ \ \isakeyword{assumes}\isanewline
\ \ \ \ {\isachardoublequoteopen}x{\isasymin}M{\isachardoublequoteclose}\ {\isachardoublequoteopen}y{\isasymin}M{\isachardoublequoteclose}\isanewline
\ \ \isakeyword{shows}\isanewline
\ \ \ \ {\isachardoublequoteopen}powerset{\isacharparenleft}{\kern0pt}{\isacharhash}{\kern0pt}{\isacharhash}{\kern0pt}M{\isacharcomma}{\kern0pt}x{\isacharcomma}{\kern0pt}y{\isacharparenright}{\kern0pt}\ {\isasymlongleftrightarrow}\ y\ {\isacharequal}{\kern0pt}\ Pow{\isacharparenleft}{\kern0pt}x{\isacharparenright}{\kern0pt}\ {\isasyminter}\ M{\isachardoublequoteclose}\isanewline
%
\isadelimproof
\ \ %
\endisadelimproof
%
\isatagproof
\isacommand{using}\isamarkupfalse%
\ assms\ \isacommand{by}\isamarkupfalse%
\ auto%
\endisatagproof
{\isafoldproof}%
%
\isadelimproof
\isanewline
%
\endisadelimproof
\isanewline
\isanewline
\isacommand{schematic{\isacharunderscore}{\kern0pt}goal}\isamarkupfalse%
\ sats{\isacharunderscore}{\kern0pt}prebody{\isacharunderscore}{\kern0pt}fm{\isacharunderscore}{\kern0pt}auto{\isacharcolon}{\kern0pt}\isanewline
\ \ \isakeyword{assumes}\isanewline
\ \ \ \ {\isachardoublequoteopen}{\isasymphi}{\isasymin}formula{\isachardoublequoteclose}\ {\isachardoublequoteopen}{\isacharbrackleft}{\kern0pt}P{\isacharcomma}{\kern0pt}leq{\isacharcomma}{\kern0pt}one{\isacharcomma}{\kern0pt}p{\isacharcomma}{\kern0pt}{\isasymrho}{\isacharcomma}{\kern0pt}{\isasympi}{\isacharbrackright}{\kern0pt}\ {\isacharat}{\kern0pt}\ nenv\ {\isasymin}list{\isacharparenleft}{\kern0pt}M{\isacharparenright}{\kern0pt}{\isachardoublequoteclose}\ \ {\isachardoublequoteopen}{\isasymalpha}{\isasymin}M{\isachardoublequoteclose}\ {\isachardoublequoteopen}arity{\isacharparenleft}{\kern0pt}{\isasymphi}{\isacharparenright}{\kern0pt}\ {\isasymle}\ {\isadigit{2}}\ {\isacharhash}{\kern0pt}{\isacharplus}{\kern0pt}\ length{\isacharparenleft}{\kern0pt}nenv{\isacharparenright}{\kern0pt}{\isachardoublequoteclose}\isanewline
\ \ \isakeyword{shows}\isanewline
\ \ \ \ {\isachardoublequoteopen}{\isacharparenleft}{\kern0pt}{\isasymexists}{\isasymtau}{\isasymin}M{\isachardot}{\kern0pt}\ {\isasymexists}V{\isasymin}M{\isachardot}{\kern0pt}\ is{\isacharunderscore}{\kern0pt}Vset{\isacharparenleft}{\kern0pt}{\isacharhash}{\kern0pt}{\isacharhash}{\kern0pt}M{\isacharcomma}{\kern0pt}{\isasymalpha}{\isacharcomma}{\kern0pt}V{\isacharparenright}{\kern0pt}\ {\isasymand}\ {\isasymtau}{\isasymin}V\ {\isasymand}\ sats{\isacharparenleft}{\kern0pt}M{\isacharcomma}{\kern0pt}forces{\isacharparenleft}{\kern0pt}{\isasymphi}{\isacharparenright}{\kern0pt}{\isacharcomma}{\kern0pt}{\isacharbrackleft}{\kern0pt}p{\isacharcomma}{\kern0pt}P{\isacharcomma}{\kern0pt}leq{\isacharcomma}{\kern0pt}one{\isacharcomma}{\kern0pt}{\isasymrho}{\isacharcomma}{\kern0pt}{\isasymtau}{\isacharbrackright}{\kern0pt}\ {\isacharat}{\kern0pt}\ nenv{\isacharparenright}{\kern0pt}{\isacharparenright}{\kern0pt}\isanewline
\ \ \ {\isasymlongleftrightarrow}\ sats{\isacharparenleft}{\kern0pt}M{\isacharcomma}{\kern0pt}{\isacharquery}{\kern0pt}prebody{\isacharunderscore}{\kern0pt}fm{\isacharcomma}{\kern0pt}{\isacharbrackleft}{\kern0pt}{\isasymrho}{\isacharcomma}{\kern0pt}p{\isacharcomma}{\kern0pt}{\isasymalpha}{\isacharcomma}{\kern0pt}P{\isacharcomma}{\kern0pt}leq{\isacharcomma}{\kern0pt}one{\isacharbrackright}{\kern0pt}\ {\isacharat}{\kern0pt}\ nenv{\isacharparenright}{\kern0pt}{\isachardoublequoteclose}\isanewline
%
\isadelimproof
\ \ %
\endisadelimproof
%
\isatagproof
\isacommand{apply}\isamarkupfalse%
\ {\isacharparenleft}{\kern0pt}insert\ assms{\isacharsemicolon}{\kern0pt}\ {\isacharparenleft}{\kern0pt}rule\ sep{\isacharunderscore}{\kern0pt}rules\ is{\isacharunderscore}{\kern0pt}Vset{\isacharunderscore}{\kern0pt}iff{\isacharunderscore}{\kern0pt}sats{\isacharbrackleft}{\kern0pt}OF\ {\isacharunderscore}{\kern0pt}\ {\isacharunderscore}{\kern0pt}\ {\isacharunderscore}{\kern0pt}\ {\isacharunderscore}{\kern0pt}\ {\isacharunderscore}{\kern0pt}\ nonempty{\isacharbrackleft}{\kern0pt}simplified{\isacharbrackright}{\kern0pt}{\isacharbrackright}{\kern0pt}\ {\isacharbar}{\kern0pt}\ simp{\isacharparenright}{\kern0pt}{\isacharparenright}{\kern0pt}\isanewline
\ \ \ \isacommand{apply}\isamarkupfalse%
\ {\isacharparenleft}{\kern0pt}rule\ sep{\isacharunderscore}{\kern0pt}rules\ is{\isacharunderscore}{\kern0pt}Vset{\isacharunderscore}{\kern0pt}iff{\isacharunderscore}{\kern0pt}sats\ is{\isacharunderscore}{\kern0pt}Vset{\isacharunderscore}{\kern0pt}iff{\isacharunderscore}{\kern0pt}sats{\isacharbrackleft}{\kern0pt}OF\ {\isacharunderscore}{\kern0pt}\ {\isacharunderscore}{\kern0pt}\ {\isacharunderscore}{\kern0pt}\ {\isacharunderscore}{\kern0pt}\ {\isacharunderscore}{\kern0pt}\ nonempty{\isacharbrackleft}{\kern0pt}simplified{\isacharbrackright}{\kern0pt}{\isacharbrackright}{\kern0pt}\ {\isacharbar}{\kern0pt}\ simp{\isacharparenright}{\kern0pt}{\isacharplus}{\kern0pt}\isanewline
\ \ \ \ \ \ \ \ \isacommand{apply}\isamarkupfalse%
\ {\isacharparenleft}{\kern0pt}rule\ nonempty{\isacharbrackleft}{\kern0pt}simplified{\isacharbrackright}{\kern0pt}{\isacharparenright}{\kern0pt}\isanewline
\ \ \ \ \ \ \ \isacommand{apply}\isamarkupfalse%
\ {\isacharparenleft}{\kern0pt}simp{\isacharunderscore}{\kern0pt}all{\isacharparenright}{\kern0pt}\isanewline
\ \ \ \ \isacommand{apply}\isamarkupfalse%
\ {\isacharparenleft}{\kern0pt}rule\ length{\isacharunderscore}{\kern0pt}type{\isacharbrackleft}{\kern0pt}THEN\ nat{\isacharunderscore}{\kern0pt}into{\isacharunderscore}{\kern0pt}Ord{\isacharbrackright}{\kern0pt}{\isacharcomma}{\kern0pt}\ blast{\isacharparenright}{\kern0pt}{\isacharplus}{\kern0pt}\isanewline
\ \ \isacommand{apply}\isamarkupfalse%
\ {\isacharparenleft}{\kern0pt}{\isacharparenleft}{\kern0pt}rule\ sep{\isacharunderscore}{\kern0pt}rules\ {\isacharbar}{\kern0pt}\ simp{\isacharparenright}{\kern0pt}{\isacharparenright}{\kern0pt}\isanewline
\ \ \ \ \isacommand{apply}\isamarkupfalse%
\ {\isacharparenleft}{\kern0pt}{\isacharparenleft}{\kern0pt}rule\ sep{\isacharunderscore}{\kern0pt}rules\ {\isacharbar}{\kern0pt}\ simp{\isacharparenright}{\kern0pt}{\isacharparenright}{\kern0pt}\isanewline
\ \ \ \ \ \ \isacommand{apply}\isamarkupfalse%
\ {\isacharparenleft}{\kern0pt}{\isacharparenleft}{\kern0pt}rule\ sep{\isacharunderscore}{\kern0pt}rules\ {\isacharbar}{\kern0pt}\ simp{\isacharparenright}{\kern0pt}{\isacharparenright}{\kern0pt}\isanewline
\ \ \ \ \ \ \ \isacommand{apply}\isamarkupfalse%
\ {\isacharparenleft}{\kern0pt}{\isacharparenleft}{\kern0pt}rule\ sep{\isacharunderscore}{\kern0pt}rules\ {\isacharbar}{\kern0pt}\ simp{\isacharparenright}{\kern0pt}{\isacharparenright}{\kern0pt}\isanewline
\ \ \ \ \ \ \isacommand{apply}\isamarkupfalse%
\ {\isacharparenleft}{\kern0pt}{\isacharparenleft}{\kern0pt}rule\ sep{\isacharunderscore}{\kern0pt}rules\ {\isacharbar}{\kern0pt}\ simp{\isacharparenright}{\kern0pt}{\isacharparenright}{\kern0pt}\isanewline
\ \ \ \ \ \isacommand{apply}\isamarkupfalse%
\ {\isacharparenleft}{\kern0pt}{\isacharparenleft}{\kern0pt}rule\ sep{\isacharunderscore}{\kern0pt}rules\ {\isacharbar}{\kern0pt}\ simp{\isacharparenright}{\kern0pt}{\isacharparenright}{\kern0pt}\isanewline
\ \ \ \ \isacommand{apply}\isamarkupfalse%
\ {\isacharparenleft}{\kern0pt}{\isacharparenleft}{\kern0pt}rule\ sep{\isacharunderscore}{\kern0pt}rules\ {\isacharbar}{\kern0pt}\ simp{\isacharparenright}{\kern0pt}{\isacharparenright}{\kern0pt}\isanewline
\ \ \ \isacommand{apply}\isamarkupfalse%
\ {\isacharparenleft}{\kern0pt}rule\ renrep{\isacharunderscore}{\kern0pt}sats{\isacharbrackleft}{\kern0pt}simplified{\isacharbrackright}{\kern0pt}{\isacharparenright}{\kern0pt}\isanewline
\ \ \ \ \ \ \ \isacommand{apply}\isamarkupfalse%
\ {\isacharparenleft}{\kern0pt}insert\ assms{\isacharparenright}{\kern0pt}\isanewline
\ \ \ \ \ \ \ \isacommand{apply}\isamarkupfalse%
{\isacharparenleft}{\kern0pt}auto\ simp\ add{\isacharcolon}{\kern0pt}\ renrep{\isacharunderscore}{\kern0pt}type\ definability{\isacharparenright}{\kern0pt}\isanewline
\isacommand{proof}\isamarkupfalse%
\ {\isacharminus}{\kern0pt}\isanewline
\ \ \isacommand{from}\isamarkupfalse%
\ assms\isanewline
\ \ \isacommand{have}\isamarkupfalse%
\ {\isachardoublequoteopen}nenv{\isasymin}list{\isacharparenleft}{\kern0pt}M{\isacharparenright}{\kern0pt}{\isachardoublequoteclose}\ \isacommand{by}\isamarkupfalse%
\ simp\isanewline
\ \ \isacommand{with}\isamarkupfalse%
\ {\isacartoucheopen}arity{\isacharparenleft}{\kern0pt}{\isasymphi}{\isacharparenright}{\kern0pt}{\isasymle}{\isacharunderscore}{\kern0pt}{\isacartoucheclose}\ {\isacartoucheopen}{\isasymphi}{\isasymin}{\isacharunderscore}{\kern0pt}{\isacartoucheclose}\isanewline
\ \ \isacommand{show}\isamarkupfalse%
\ {\isachardoublequoteopen}arity{\isacharparenleft}{\kern0pt}forces{\isacharparenleft}{\kern0pt}{\isasymphi}{\isacharparenright}{\kern0pt}{\isacharparenright}{\kern0pt}\ {\isasymle}\ succ{\isacharparenleft}{\kern0pt}succ{\isacharparenleft}{\kern0pt}succ{\isacharparenleft}{\kern0pt}succ{\isacharparenleft}{\kern0pt}succ{\isacharparenleft}{\kern0pt}succ{\isacharparenleft}{\kern0pt}length{\isacharparenleft}{\kern0pt}nenv{\isacharparenright}{\kern0pt}{\isacharparenright}{\kern0pt}{\isacharparenright}{\kern0pt}{\isacharparenright}{\kern0pt}{\isacharparenright}{\kern0pt}{\isacharparenright}{\kern0pt}{\isacharparenright}{\kern0pt}{\isachardoublequoteclose}\isanewline
\ \ \ \ \isacommand{using}\isamarkupfalse%
\ arity{\isacharunderscore}{\kern0pt}forces{\isacharunderscore}{\kern0pt}le\ \isacommand{by}\isamarkupfalse%
\ simp\isanewline
\isacommand{qed}\isamarkupfalse%
%
\endisatagproof
{\isafoldproof}%
%
\isadelimproof
\isanewline
%
\endisadelimproof
\isanewline
%
\isadelimML
\isanewline
%
\endisadelimML
%
\isatagML
\isacommand{synthesize{\isacharunderscore}{\kern0pt}notc}\isamarkupfalse%
\ {\isachardoublequoteopen}prebody{\isacharunderscore}{\kern0pt}fm{\isachardoublequoteclose}\ \isakeyword{from{\isacharunderscore}{\kern0pt}schematic}\ sats{\isacharunderscore}{\kern0pt}prebody{\isacharunderscore}{\kern0pt}fm{\isacharunderscore}{\kern0pt}auto%
\endisatagML
{\isafoldML}%
%
\isadelimML
\isanewline
%
\endisadelimML
\isanewline
\isacommand{lemma}\isamarkupfalse%
\ prebody{\isacharunderscore}{\kern0pt}fm{\isacharunderscore}{\kern0pt}type\ {\isacharbrackleft}{\kern0pt}TC{\isacharbrackright}{\kern0pt}{\isacharcolon}{\kern0pt}\isanewline
\ \ \isakeyword{assumes}\ {\isachardoublequoteopen}{\isasymphi}{\isasymin}formula{\isachardoublequoteclose}\isanewline
\ \ \ \ {\isachardoublequoteopen}env\ {\isasymin}\ list{\isacharparenleft}{\kern0pt}M{\isacharparenright}{\kern0pt}{\isachardoublequoteclose}\isanewline
\ \ \isakeyword{shows}\ {\isachardoublequoteopen}prebody{\isacharunderscore}{\kern0pt}fm{\isacharparenleft}{\kern0pt}{\isasymphi}{\isacharcomma}{\kern0pt}env{\isacharparenright}{\kern0pt}{\isasymin}formula{\isachardoublequoteclose}\isanewline
%
\isadelimproof
%
\endisadelimproof
%
\isatagproof
\isacommand{proof}\isamarkupfalse%
\ {\isacharminus}{\kern0pt}\isanewline
\ \ \isacommand{from}\isamarkupfalse%
\ {\isacartoucheopen}{\isasymphi}{\isasymin}formula{\isacartoucheclose}\isanewline
\ \ \isacommand{have}\isamarkupfalse%
\ {\isachardoublequoteopen}forces{\isacharparenleft}{\kern0pt}{\isasymphi}{\isacharparenright}{\kern0pt}{\isasymin}formula{\isachardoublequoteclose}\ \isacommand{by}\isamarkupfalse%
\ simp\isanewline
\ \ \isacommand{then}\isamarkupfalse%
\isanewline
\ \ \isacommand{have}\isamarkupfalse%
\ {\isachardoublequoteopen}renrep{\isacharparenleft}{\kern0pt}forces{\isacharparenleft}{\kern0pt}{\isasymphi}{\isacharparenright}{\kern0pt}{\isacharcomma}{\kern0pt}env{\isacharparenright}{\kern0pt}{\isasymin}formula{\isachardoublequoteclose}\isanewline
\ \ \ \ \isacommand{using}\isamarkupfalse%
\ {\isacartoucheopen}env{\isasymin}list{\isacharparenleft}{\kern0pt}M{\isacharparenright}{\kern0pt}{\isacartoucheclose}\ \isacommand{by}\isamarkupfalse%
\ simp\isanewline
\ \ \isacommand{then}\isamarkupfalse%
\ \isacommand{show}\isamarkupfalse%
\ {\isacharquery}{\kern0pt}thesis\ \isacommand{unfolding}\isamarkupfalse%
\ prebody{\isacharunderscore}{\kern0pt}fm{\isacharunderscore}{\kern0pt}def\ \isacommand{by}\isamarkupfalse%
\ simp\isanewline
\isacommand{qed}\isamarkupfalse%
%
\endisatagproof
{\isafoldproof}%
%
\isadelimproof
\isanewline
%
\endisadelimproof
\isanewline
\isacommand{lemmas}\isamarkupfalse%
\ new{\isacharunderscore}{\kern0pt}fm{\isacharunderscore}{\kern0pt}defs\ {\isacharequal}{\kern0pt}\ fm{\isacharunderscore}{\kern0pt}defs\ is{\isacharunderscore}{\kern0pt}transrec{\isacharunderscore}{\kern0pt}fm{\isacharunderscore}{\kern0pt}def\ is{\isacharunderscore}{\kern0pt}eclose{\isacharunderscore}{\kern0pt}fm{\isacharunderscore}{\kern0pt}def\ mem{\isacharunderscore}{\kern0pt}eclose{\isacharunderscore}{\kern0pt}fm{\isacharunderscore}{\kern0pt}def\isanewline
\ \ finite{\isacharunderscore}{\kern0pt}ordinal{\isacharunderscore}{\kern0pt}fm{\isacharunderscore}{\kern0pt}def\ is{\isacharunderscore}{\kern0pt}wfrec{\isacharunderscore}{\kern0pt}fm{\isacharunderscore}{\kern0pt}def\ \ Memrel{\isacharunderscore}{\kern0pt}fm{\isacharunderscore}{\kern0pt}def\ eclose{\isacharunderscore}{\kern0pt}n{\isacharunderscore}{\kern0pt}fm{\isacharunderscore}{\kern0pt}def\ is{\isacharunderscore}{\kern0pt}recfun{\isacharunderscore}{\kern0pt}fm{\isacharunderscore}{\kern0pt}def\ is{\isacharunderscore}{\kern0pt}iterates{\isacharunderscore}{\kern0pt}fm{\isacharunderscore}{\kern0pt}def\isanewline
\ \ iterates{\isacharunderscore}{\kern0pt}MH{\isacharunderscore}{\kern0pt}fm{\isacharunderscore}{\kern0pt}def\ is{\isacharunderscore}{\kern0pt}nat{\isacharunderscore}{\kern0pt}case{\isacharunderscore}{\kern0pt}fm{\isacharunderscore}{\kern0pt}def\ quasinat{\isacharunderscore}{\kern0pt}fm{\isacharunderscore}{\kern0pt}def\ pre{\isacharunderscore}{\kern0pt}image{\isacharunderscore}{\kern0pt}fm{\isacharunderscore}{\kern0pt}def\ restriction{\isacharunderscore}{\kern0pt}fm{\isacharunderscore}{\kern0pt}def\isanewline
\isanewline
\isacommand{lemma}\isamarkupfalse%
\ sats{\isacharunderscore}{\kern0pt}prebody{\isacharunderscore}{\kern0pt}fm{\isacharcolon}{\kern0pt}\isanewline
\ \ \isakeyword{assumes}\isanewline
\ \ \ \ {\isachardoublequoteopen}{\isacharbrackleft}{\kern0pt}P{\isacharcomma}{\kern0pt}leq{\isacharcomma}{\kern0pt}one{\isacharcomma}{\kern0pt}p{\isacharcomma}{\kern0pt}{\isasymrho}{\isacharbrackright}{\kern0pt}\ {\isacharat}{\kern0pt}\ nenv\ {\isasymin}list{\isacharparenleft}{\kern0pt}M{\isacharparenright}{\kern0pt}{\isachardoublequoteclose}\ {\isachardoublequoteopen}{\isasymphi}{\isasymin}formula{\isachardoublequoteclose}\ {\isachardoublequoteopen}{\isasymalpha}{\isasymin}M{\isachardoublequoteclose}\ {\isachardoublequoteopen}arity{\isacharparenleft}{\kern0pt}{\isasymphi}{\isacharparenright}{\kern0pt}\ {\isasymle}\ {\isadigit{2}}\ {\isacharhash}{\kern0pt}{\isacharplus}{\kern0pt}\ length{\isacharparenleft}{\kern0pt}nenv{\isacharparenright}{\kern0pt}{\isachardoublequoteclose}\isanewline
\ \ \isakeyword{shows}\isanewline
\ \ \ \ {\isachardoublequoteopen}sats{\isacharparenleft}{\kern0pt}M{\isacharcomma}{\kern0pt}prebody{\isacharunderscore}{\kern0pt}fm{\isacharparenleft}{\kern0pt}{\isasymphi}{\isacharcomma}{\kern0pt}nenv{\isacharparenright}{\kern0pt}{\isacharcomma}{\kern0pt}{\isacharbrackleft}{\kern0pt}{\isasymrho}{\isacharcomma}{\kern0pt}p{\isacharcomma}{\kern0pt}{\isasymalpha}{\isacharcomma}{\kern0pt}P{\isacharcomma}{\kern0pt}leq{\isacharcomma}{\kern0pt}one{\isacharbrackright}{\kern0pt}\ {\isacharat}{\kern0pt}\ nenv{\isacharparenright}{\kern0pt}\ {\isasymlongleftrightarrow}\isanewline
\ \ \ \ \ {\isacharparenleft}{\kern0pt}{\isasymexists}{\isasymtau}{\isasymin}M{\isachardot}{\kern0pt}\ {\isasymexists}V{\isasymin}M{\isachardot}{\kern0pt}\ is{\isacharunderscore}{\kern0pt}Vset{\isacharparenleft}{\kern0pt}{\isacharhash}{\kern0pt}{\isacharhash}{\kern0pt}M{\isacharcomma}{\kern0pt}{\isasymalpha}{\isacharcomma}{\kern0pt}V{\isacharparenright}{\kern0pt}\ {\isasymand}\ {\isasymtau}{\isasymin}V\ {\isasymand}\ sats{\isacharparenleft}{\kern0pt}M{\isacharcomma}{\kern0pt}forces{\isacharparenleft}{\kern0pt}{\isasymphi}{\isacharparenright}{\kern0pt}{\isacharcomma}{\kern0pt}{\isacharbrackleft}{\kern0pt}p{\isacharcomma}{\kern0pt}P{\isacharcomma}{\kern0pt}leq{\isacharcomma}{\kern0pt}one{\isacharcomma}{\kern0pt}{\isasymrho}{\isacharcomma}{\kern0pt}{\isasymtau}{\isacharbrackright}{\kern0pt}\ {\isacharat}{\kern0pt}\ nenv{\isacharparenright}{\kern0pt}{\isacharparenright}{\kern0pt}{\isachardoublequoteclose}\isanewline
%
\isadelimproof
\ \ %
\endisadelimproof
%
\isatagproof
\isacommand{unfolding}\isamarkupfalse%
\ prebody{\isacharunderscore}{\kern0pt}fm{\isacharunderscore}{\kern0pt}def\ \isacommand{using}\isamarkupfalse%
\ assms\ sats{\isacharunderscore}{\kern0pt}prebody{\isacharunderscore}{\kern0pt}fm{\isacharunderscore}{\kern0pt}auto\ \isacommand{by}\isamarkupfalse%
\ force%
\endisatagproof
{\isafoldproof}%
%
\isadelimproof
\isanewline
%
\endisadelimproof
\isanewline
\isanewline
\isacommand{lemma}\isamarkupfalse%
\ arity{\isacharunderscore}{\kern0pt}prebody{\isacharunderscore}{\kern0pt}fm{\isacharcolon}{\kern0pt}\isanewline
\ \ \isakeyword{assumes}\isanewline
\ \ \ \ {\isachardoublequoteopen}{\isasymphi}{\isasymin}formula{\isachardoublequoteclose}\ {\isachardoublequoteopen}{\isasymalpha}{\isasymin}M{\isachardoublequoteclose}\ {\isachardoublequoteopen}env\ {\isasymin}\ list{\isacharparenleft}{\kern0pt}M{\isacharparenright}{\kern0pt}{\isachardoublequoteclose}\ {\isachardoublequoteopen}arity{\isacharparenleft}{\kern0pt}{\isasymphi}{\isacharparenright}{\kern0pt}\ {\isasymle}\ {\isadigit{2}}\ {\isacharhash}{\kern0pt}{\isacharplus}{\kern0pt}\ length{\isacharparenleft}{\kern0pt}env{\isacharparenright}{\kern0pt}{\isachardoublequoteclose}\isanewline
\ \ \isakeyword{shows}\isanewline
\ \ \ \ {\isachardoublequoteopen}arity{\isacharparenleft}{\kern0pt}prebody{\isacharunderscore}{\kern0pt}fm{\isacharparenleft}{\kern0pt}{\isasymphi}{\isacharcomma}{\kern0pt}env{\isacharparenright}{\kern0pt}{\isacharparenright}{\kern0pt}{\isasymle}{\isadigit{6}}\ {\isacharhash}{\kern0pt}{\isacharplus}{\kern0pt}\ length{\isacharparenleft}{\kern0pt}env{\isacharparenright}{\kern0pt}{\isachardoublequoteclose}\isanewline
%
\isadelimproof
\ \ %
\endisadelimproof
%
\isatagproof
\isacommand{unfolding}\isamarkupfalse%
\ prebody{\isacharunderscore}{\kern0pt}fm{\isacharunderscore}{\kern0pt}def\ is{\isacharunderscore}{\kern0pt}HVfrom{\isacharunderscore}{\kern0pt}fm{\isacharunderscore}{\kern0pt}def\ is{\isacharunderscore}{\kern0pt}powapply{\isacharunderscore}{\kern0pt}fm{\isacharunderscore}{\kern0pt}def\isanewline
\ \ \isacommand{using}\isamarkupfalse%
\ assms\ new{\isacharunderscore}{\kern0pt}fm{\isacharunderscore}{\kern0pt}defs\ nat{\isacharunderscore}{\kern0pt}simp{\isacharunderscore}{\kern0pt}union\isanewline
\ \ \ \ arity{\isacharunderscore}{\kern0pt}renrep{\isacharbrackleft}{\kern0pt}of\ {\isachardoublequoteopen}forces{\isacharparenleft}{\kern0pt}{\isasymphi}{\isacharparenright}{\kern0pt}{\isachardoublequoteclose}{\isacharbrackright}{\kern0pt}\ arity{\isacharunderscore}{\kern0pt}forces{\isacharunderscore}{\kern0pt}le{\isacharbrackleft}{\kern0pt}simplified{\isacharbrackright}{\kern0pt}\ pred{\isacharunderscore}{\kern0pt}le\ \isacommand{by}\isamarkupfalse%
\ auto%
\endisatagproof
{\isafoldproof}%
%
\isadelimproof
\isanewline
%
\endisadelimproof
\isanewline
\isanewline
\isacommand{definition}\isamarkupfalse%
\isanewline
\ \ body{\isacharunderscore}{\kern0pt}fm{\isacharprime}{\kern0pt}\ {\isacharcolon}{\kern0pt}{\isacharcolon}{\kern0pt}\ {\isachardoublequoteopen}{\isacharbrackleft}{\kern0pt}i{\isacharcomma}{\kern0pt}i{\isacharbrackright}{\kern0pt}{\isasymRightarrow}i{\isachardoublequoteclose}\ \isakeyword{where}\isanewline
\ \ {\isachardoublequoteopen}body{\isacharunderscore}{\kern0pt}fm{\isacharprime}{\kern0pt}{\isacharparenleft}{\kern0pt}{\isasymphi}{\isacharcomma}{\kern0pt}env{\isacharparenright}{\kern0pt}\ {\isasymequiv}\ Exists{\isacharparenleft}{\kern0pt}Exists{\isacharparenleft}{\kern0pt}And{\isacharparenleft}{\kern0pt}pair{\isacharunderscore}{\kern0pt}fm{\isacharparenleft}{\kern0pt}{\isadigit{0}}{\isacharcomma}{\kern0pt}{\isadigit{1}}{\isacharcomma}{\kern0pt}{\isadigit{2}}{\isacharparenright}{\kern0pt}{\isacharcomma}{\kern0pt}renpbdy{\isacharparenleft}{\kern0pt}prebody{\isacharunderscore}{\kern0pt}fm{\isacharparenleft}{\kern0pt}{\isasymphi}{\isacharcomma}{\kern0pt}env{\isacharparenright}{\kern0pt}{\isacharcomma}{\kern0pt}env{\isacharparenright}{\kern0pt}{\isacharparenright}{\kern0pt}{\isacharparenright}{\kern0pt}{\isacharparenright}{\kern0pt}{\isachardoublequoteclose}\isanewline
\isanewline
\isacommand{lemma}\isamarkupfalse%
\ body{\isacharunderscore}{\kern0pt}fm{\isacharprime}{\kern0pt}{\isacharunderscore}{\kern0pt}type{\isacharbrackleft}{\kern0pt}TC{\isacharbrackright}{\kern0pt}{\isacharcolon}{\kern0pt}\ {\isachardoublequoteopen}{\isasymphi}{\isasymin}formula\ {\isasymLongrightarrow}\ env{\isasymin}list{\isacharparenleft}{\kern0pt}M{\isacharparenright}{\kern0pt}\ {\isasymLongrightarrow}\ body{\isacharunderscore}{\kern0pt}fm{\isacharprime}{\kern0pt}{\isacharparenleft}{\kern0pt}{\isasymphi}{\isacharcomma}{\kern0pt}env{\isacharparenright}{\kern0pt}{\isasymin}formula{\isachardoublequoteclose}\isanewline
%
\isadelimproof
\ \ %
\endisadelimproof
%
\isatagproof
\isacommand{unfolding}\isamarkupfalse%
\ body{\isacharunderscore}{\kern0pt}fm{\isacharprime}{\kern0pt}{\isacharunderscore}{\kern0pt}def\ \isacommand{using}\isamarkupfalse%
\ prebody{\isacharunderscore}{\kern0pt}fm{\isacharunderscore}{\kern0pt}type\isanewline
\ \ \isacommand{by}\isamarkupfalse%
\ simp%
\endisatagproof
{\isafoldproof}%
%
\isadelimproof
\isanewline
%
\endisadelimproof
\isanewline
\isacommand{lemma}\isamarkupfalse%
\ arity{\isacharunderscore}{\kern0pt}body{\isacharunderscore}{\kern0pt}fm{\isacharprime}{\kern0pt}{\isacharcolon}{\kern0pt}\isanewline
\ \ \isakeyword{assumes}\isanewline
\ \ \ \ {\isachardoublequoteopen}{\isasymphi}{\isasymin}formula{\isachardoublequoteclose}\ {\isachardoublequoteopen}{\isasymalpha}{\isasymin}M{\isachardoublequoteclose}\ {\isachardoublequoteopen}env{\isasymin}list{\isacharparenleft}{\kern0pt}M{\isacharparenright}{\kern0pt}{\isachardoublequoteclose}\ {\isachardoublequoteopen}arity{\isacharparenleft}{\kern0pt}{\isasymphi}{\isacharparenright}{\kern0pt}\ {\isasymle}\ {\isadigit{2}}\ {\isacharhash}{\kern0pt}{\isacharplus}{\kern0pt}\ length{\isacharparenleft}{\kern0pt}env{\isacharparenright}{\kern0pt}{\isachardoublequoteclose}\isanewline
\ \ \isakeyword{shows}\isanewline
\ \ \ \ {\isachardoublequoteopen}arity{\isacharparenleft}{\kern0pt}body{\isacharunderscore}{\kern0pt}fm{\isacharprime}{\kern0pt}{\isacharparenleft}{\kern0pt}{\isasymphi}{\isacharcomma}{\kern0pt}env{\isacharparenright}{\kern0pt}{\isacharparenright}{\kern0pt}{\isasymle}{\isadigit{5}}\ \ {\isacharhash}{\kern0pt}{\isacharplus}{\kern0pt}\ length{\isacharparenleft}{\kern0pt}env{\isacharparenright}{\kern0pt}{\isachardoublequoteclose}\isanewline
%
\isadelimproof
\ \ %
\endisadelimproof
%
\isatagproof
\isacommand{unfolding}\isamarkupfalse%
\ body{\isacharunderscore}{\kern0pt}fm{\isacharprime}{\kern0pt}{\isacharunderscore}{\kern0pt}def\isanewline
\ \ \isacommand{using}\isamarkupfalse%
\ assms\ new{\isacharunderscore}{\kern0pt}fm{\isacharunderscore}{\kern0pt}defs\ nat{\isacharunderscore}{\kern0pt}simp{\isacharunderscore}{\kern0pt}union\ arity{\isacharunderscore}{\kern0pt}prebody{\isacharunderscore}{\kern0pt}fm\ pred{\isacharunderscore}{\kern0pt}le\ \ arity{\isacharunderscore}{\kern0pt}renpbdy{\isacharbrackleft}{\kern0pt}of\ {\isachardoublequoteopen}prebody{\isacharunderscore}{\kern0pt}fm{\isacharparenleft}{\kern0pt}{\isasymphi}{\isacharcomma}{\kern0pt}env{\isacharparenright}{\kern0pt}{\isachardoublequoteclose}{\isacharbrackright}{\kern0pt}\isanewline
\ \ \isacommand{by}\isamarkupfalse%
\ auto%
\endisatagproof
{\isafoldproof}%
%
\isadelimproof
\isanewline
%
\endisadelimproof
\isanewline
\isacommand{lemma}\isamarkupfalse%
\ sats{\isacharunderscore}{\kern0pt}body{\isacharunderscore}{\kern0pt}fm{\isacharprime}{\kern0pt}{\isacharcolon}{\kern0pt}\isanewline
\ \ \isakeyword{assumes}\isanewline
\ \ \ \ {\isachardoublequoteopen}{\isasymexists}t\ p{\isachardot}{\kern0pt}\ x{\isacharequal}{\kern0pt}{\isasymlangle}t{\isacharcomma}{\kern0pt}p{\isasymrangle}{\isachardoublequoteclose}\ {\isachardoublequoteopen}x{\isasymin}M{\isachardoublequoteclose}\ {\isachardoublequoteopen}{\isacharbrackleft}{\kern0pt}{\isasymalpha}{\isacharcomma}{\kern0pt}P{\isacharcomma}{\kern0pt}leq{\isacharcomma}{\kern0pt}one{\isacharcomma}{\kern0pt}p{\isacharcomma}{\kern0pt}{\isasymrho}{\isacharbrackright}{\kern0pt}\ {\isacharat}{\kern0pt}\ nenv\ {\isasymin}list{\isacharparenleft}{\kern0pt}M{\isacharparenright}{\kern0pt}{\isachardoublequoteclose}\ {\isachardoublequoteopen}{\isasymphi}{\isasymin}formula{\isachardoublequoteclose}\ {\isachardoublequoteopen}arity{\isacharparenleft}{\kern0pt}{\isasymphi}{\isacharparenright}{\kern0pt}\ {\isasymle}\ {\isadigit{2}}\ {\isacharhash}{\kern0pt}{\isacharplus}{\kern0pt}\ length{\isacharparenleft}{\kern0pt}nenv{\isacharparenright}{\kern0pt}{\isachardoublequoteclose}\isanewline
\ \ \isakeyword{shows}\isanewline
\ \ \ \ {\isachardoublequoteopen}sats{\isacharparenleft}{\kern0pt}M{\isacharcomma}{\kern0pt}body{\isacharunderscore}{\kern0pt}fm{\isacharprime}{\kern0pt}{\isacharparenleft}{\kern0pt}{\isasymphi}{\isacharcomma}{\kern0pt}nenv{\isacharparenright}{\kern0pt}{\isacharcomma}{\kern0pt}{\isacharbrackleft}{\kern0pt}x{\isacharcomma}{\kern0pt}{\isasymalpha}{\isacharcomma}{\kern0pt}P{\isacharcomma}{\kern0pt}leq{\isacharcomma}{\kern0pt}one{\isacharbrackright}{\kern0pt}\ {\isacharat}{\kern0pt}\ nenv{\isacharparenright}{\kern0pt}\ {\isasymlongleftrightarrow}\isanewline
\ \ \ \ \ sats{\isacharparenleft}{\kern0pt}M{\isacharcomma}{\kern0pt}renpbdy{\isacharparenleft}{\kern0pt}prebody{\isacharunderscore}{\kern0pt}fm{\isacharparenleft}{\kern0pt}{\isasymphi}{\isacharcomma}{\kern0pt}nenv{\isacharparenright}{\kern0pt}{\isacharcomma}{\kern0pt}nenv{\isacharparenright}{\kern0pt}{\isacharcomma}{\kern0pt}{\isacharbrackleft}{\kern0pt}fst{\isacharparenleft}{\kern0pt}x{\isacharparenright}{\kern0pt}{\isacharcomma}{\kern0pt}snd{\isacharparenleft}{\kern0pt}x{\isacharparenright}{\kern0pt}{\isacharcomma}{\kern0pt}x{\isacharcomma}{\kern0pt}{\isasymalpha}{\isacharcomma}{\kern0pt}P{\isacharcomma}{\kern0pt}leq{\isacharcomma}{\kern0pt}one{\isacharbrackright}{\kern0pt}\ {\isacharat}{\kern0pt}\ nenv{\isacharparenright}{\kern0pt}{\isachardoublequoteclose}\isanewline
%
\isadelimproof
\ \ %
\endisadelimproof
%
\isatagproof
\isacommand{using}\isamarkupfalse%
\ assms\ fst{\isacharunderscore}{\kern0pt}snd{\isacharunderscore}{\kern0pt}closed{\isacharbrackleft}{\kern0pt}OF\ {\isacartoucheopen}x{\isasymin}M{\isacartoucheclose}{\isacharbrackright}{\kern0pt}\ \isacommand{unfolding}\isamarkupfalse%
\ body{\isacharunderscore}{\kern0pt}fm{\isacharprime}{\kern0pt}{\isacharunderscore}{\kern0pt}def\isanewline
\ \ \isacommand{by}\isamarkupfalse%
\ {\isacharparenleft}{\kern0pt}auto{\isacharparenright}{\kern0pt}%
\endisatagproof
{\isafoldproof}%
%
\isadelimproof
\isanewline
%
\endisadelimproof
\isanewline
\isacommand{definition}\isamarkupfalse%
\isanewline
\ \ body{\isacharunderscore}{\kern0pt}fm\ {\isacharcolon}{\kern0pt}{\isacharcolon}{\kern0pt}\ {\isachardoublequoteopen}{\isacharbrackleft}{\kern0pt}i{\isacharcomma}{\kern0pt}i{\isacharbrackright}{\kern0pt}{\isasymRightarrow}i{\isachardoublequoteclose}\ \isakeyword{where}\isanewline
\ \ {\isachardoublequoteopen}body{\isacharunderscore}{\kern0pt}fm{\isacharparenleft}{\kern0pt}{\isasymphi}{\isacharcomma}{\kern0pt}env{\isacharparenright}{\kern0pt}\ {\isasymequiv}\ renbody{\isacharparenleft}{\kern0pt}body{\isacharunderscore}{\kern0pt}fm{\isacharprime}{\kern0pt}{\isacharparenleft}{\kern0pt}{\isasymphi}{\isacharcomma}{\kern0pt}env{\isacharparenright}{\kern0pt}{\isacharcomma}{\kern0pt}env{\isacharparenright}{\kern0pt}{\isachardoublequoteclose}\isanewline
\isanewline
\isacommand{lemma}\isamarkupfalse%
\ body{\isacharunderscore}{\kern0pt}fm{\isacharunderscore}{\kern0pt}type\ {\isacharbrackleft}{\kern0pt}TC{\isacharbrackright}{\kern0pt}{\isacharcolon}{\kern0pt}\ {\isachardoublequoteopen}env{\isasymin}list{\isacharparenleft}{\kern0pt}M{\isacharparenright}{\kern0pt}\ {\isasymLongrightarrow}\ {\isasymphi}{\isasymin}formula\ {\isasymLongrightarrow}\ \ body{\isacharunderscore}{\kern0pt}fm{\isacharparenleft}{\kern0pt}{\isasymphi}{\isacharcomma}{\kern0pt}env{\isacharparenright}{\kern0pt}{\isasymin}formula{\isachardoublequoteclose}\isanewline
%
\isadelimproof
\ \ %
\endisadelimproof
%
\isatagproof
\isacommand{unfolding}\isamarkupfalse%
\ body{\isacharunderscore}{\kern0pt}fm{\isacharunderscore}{\kern0pt}def\ \isacommand{by}\isamarkupfalse%
\ simp%
\endisatagproof
{\isafoldproof}%
%
\isadelimproof
\isanewline
%
\endisadelimproof
\isanewline
\isacommand{lemma}\isamarkupfalse%
\ sats{\isacharunderscore}{\kern0pt}body{\isacharunderscore}{\kern0pt}fm{\isacharcolon}{\kern0pt}\isanewline
\ \ \isakeyword{assumes}\isanewline
\ \ \ \ {\isachardoublequoteopen}{\isasymexists}t\ p{\isachardot}{\kern0pt}\ x{\isacharequal}{\kern0pt}{\isasymlangle}t{\isacharcomma}{\kern0pt}p{\isasymrangle}{\isachardoublequoteclose}\ {\isachardoublequoteopen}{\isacharbrackleft}{\kern0pt}{\isasymalpha}{\isacharcomma}{\kern0pt}x{\isacharcomma}{\kern0pt}m{\isacharcomma}{\kern0pt}P{\isacharcomma}{\kern0pt}leq{\isacharcomma}{\kern0pt}one{\isacharbrackright}{\kern0pt}\ {\isacharat}{\kern0pt}\ nenv\ {\isasymin}list{\isacharparenleft}{\kern0pt}M{\isacharparenright}{\kern0pt}{\isachardoublequoteclose}\isanewline
\ \ \ \ {\isachardoublequoteopen}{\isasymphi}{\isasymin}formula{\isachardoublequoteclose}\ {\isachardoublequoteopen}arity{\isacharparenleft}{\kern0pt}{\isasymphi}{\isacharparenright}{\kern0pt}\ {\isasymle}\ {\isadigit{2}}\ {\isacharhash}{\kern0pt}{\isacharplus}{\kern0pt}\ length{\isacharparenleft}{\kern0pt}nenv{\isacharparenright}{\kern0pt}{\isachardoublequoteclose}\isanewline
\ \ \isakeyword{shows}\isanewline
\ \ \ \ {\isachardoublequoteopen}sats{\isacharparenleft}{\kern0pt}M{\isacharcomma}{\kern0pt}body{\isacharunderscore}{\kern0pt}fm{\isacharparenleft}{\kern0pt}{\isasymphi}{\isacharcomma}{\kern0pt}nenv{\isacharparenright}{\kern0pt}{\isacharcomma}{\kern0pt}{\isacharbrackleft}{\kern0pt}{\isasymalpha}{\isacharcomma}{\kern0pt}x{\isacharcomma}{\kern0pt}m{\isacharcomma}{\kern0pt}P{\isacharcomma}{\kern0pt}leq{\isacharcomma}{\kern0pt}one{\isacharbrackright}{\kern0pt}\ {\isacharat}{\kern0pt}\ nenv{\isacharparenright}{\kern0pt}\ {\isasymlongleftrightarrow}\isanewline
\ \ \ \ \ sats{\isacharparenleft}{\kern0pt}M{\isacharcomma}{\kern0pt}renpbdy{\isacharparenleft}{\kern0pt}prebody{\isacharunderscore}{\kern0pt}fm{\isacharparenleft}{\kern0pt}{\isasymphi}{\isacharcomma}{\kern0pt}nenv{\isacharparenright}{\kern0pt}{\isacharcomma}{\kern0pt}nenv{\isacharparenright}{\kern0pt}{\isacharcomma}{\kern0pt}{\isacharbrackleft}{\kern0pt}fst{\isacharparenleft}{\kern0pt}x{\isacharparenright}{\kern0pt}{\isacharcomma}{\kern0pt}snd{\isacharparenleft}{\kern0pt}x{\isacharparenright}{\kern0pt}{\isacharcomma}{\kern0pt}x{\isacharcomma}{\kern0pt}{\isasymalpha}{\isacharcomma}{\kern0pt}P{\isacharcomma}{\kern0pt}leq{\isacharcomma}{\kern0pt}one{\isacharbrackright}{\kern0pt}\ {\isacharat}{\kern0pt}\ nenv{\isacharparenright}{\kern0pt}{\isachardoublequoteclose}\isanewline
%
\isadelimproof
\ \ %
\endisadelimproof
%
\isatagproof
\isacommand{using}\isamarkupfalse%
\ assms\ sats{\isacharunderscore}{\kern0pt}body{\isacharunderscore}{\kern0pt}fm{\isacharprime}{\kern0pt}\ sats{\isacharunderscore}{\kern0pt}renbody{\isacharbrackleft}{\kern0pt}OF\ {\isacharunderscore}{\kern0pt}\ assms{\isacharparenleft}{\kern0pt}{\isadigit{2}}{\isacharparenright}{\kern0pt}{\isacharcomma}{\kern0pt}\ symmetric{\isacharbrackright}{\kern0pt}\ arity{\isacharunderscore}{\kern0pt}body{\isacharunderscore}{\kern0pt}fm{\isacharprime}{\kern0pt}\isanewline
\ \ \isacommand{unfolding}\isamarkupfalse%
\ body{\isacharunderscore}{\kern0pt}fm{\isacharunderscore}{\kern0pt}def\isanewline
\ \ \isacommand{by}\isamarkupfalse%
\ auto%
\endisatagproof
{\isafoldproof}%
%
\isadelimproof
\isanewline
%
\endisadelimproof
\isanewline
\isacommand{lemma}\isamarkupfalse%
\ sats{\isacharunderscore}{\kern0pt}renpbdy{\isacharunderscore}{\kern0pt}prebody{\isacharunderscore}{\kern0pt}fm{\isacharcolon}{\kern0pt}\isanewline
\ \ \isakeyword{assumes}\isanewline
\ \ \ \ {\isachardoublequoteopen}{\isasymexists}t\ p{\isachardot}{\kern0pt}\ x{\isacharequal}{\kern0pt}{\isasymlangle}t{\isacharcomma}{\kern0pt}p{\isasymrangle}{\isachardoublequoteclose}\ {\isachardoublequoteopen}x{\isasymin}M{\isachardoublequoteclose}\ {\isachardoublequoteopen}{\isacharbrackleft}{\kern0pt}{\isasymalpha}{\isacharcomma}{\kern0pt}m{\isacharcomma}{\kern0pt}P{\isacharcomma}{\kern0pt}leq{\isacharcomma}{\kern0pt}one{\isacharbrackright}{\kern0pt}\ {\isacharat}{\kern0pt}\ nenv\ {\isasymin}list{\isacharparenleft}{\kern0pt}M{\isacharparenright}{\kern0pt}{\isachardoublequoteclose}\isanewline
\ \ \ \ {\isachardoublequoteopen}{\isasymphi}{\isasymin}formula{\isachardoublequoteclose}\ {\isachardoublequoteopen}arity{\isacharparenleft}{\kern0pt}{\isasymphi}{\isacharparenright}{\kern0pt}\ {\isasymle}\ {\isadigit{2}}\ {\isacharhash}{\kern0pt}{\isacharplus}{\kern0pt}\ length{\isacharparenleft}{\kern0pt}nenv{\isacharparenright}{\kern0pt}{\isachardoublequoteclose}\isanewline
\ \ \isakeyword{shows}\isanewline
\ \ \ \ {\isachardoublequoteopen}sats{\isacharparenleft}{\kern0pt}M{\isacharcomma}{\kern0pt}renpbdy{\isacharparenleft}{\kern0pt}prebody{\isacharunderscore}{\kern0pt}fm{\isacharparenleft}{\kern0pt}{\isasymphi}{\isacharcomma}{\kern0pt}nenv{\isacharparenright}{\kern0pt}{\isacharcomma}{\kern0pt}nenv{\isacharparenright}{\kern0pt}{\isacharcomma}{\kern0pt}{\isacharbrackleft}{\kern0pt}fst{\isacharparenleft}{\kern0pt}x{\isacharparenright}{\kern0pt}{\isacharcomma}{\kern0pt}snd{\isacharparenleft}{\kern0pt}x{\isacharparenright}{\kern0pt}{\isacharcomma}{\kern0pt}x{\isacharcomma}{\kern0pt}{\isasymalpha}{\isacharcomma}{\kern0pt}P{\isacharcomma}{\kern0pt}leq{\isacharcomma}{\kern0pt}one{\isacharbrackright}{\kern0pt}\ {\isacharat}{\kern0pt}\ nenv{\isacharparenright}{\kern0pt}\ {\isasymlongleftrightarrow}\isanewline
\ \ \ \ \ sats{\isacharparenleft}{\kern0pt}M{\isacharcomma}{\kern0pt}prebody{\isacharunderscore}{\kern0pt}fm{\isacharparenleft}{\kern0pt}{\isasymphi}{\isacharcomma}{\kern0pt}nenv{\isacharparenright}{\kern0pt}{\isacharcomma}{\kern0pt}{\isacharbrackleft}{\kern0pt}fst{\isacharparenleft}{\kern0pt}x{\isacharparenright}{\kern0pt}{\isacharcomma}{\kern0pt}snd{\isacharparenleft}{\kern0pt}x{\isacharparenright}{\kern0pt}{\isacharcomma}{\kern0pt}{\isasymalpha}{\isacharcomma}{\kern0pt}P{\isacharcomma}{\kern0pt}leq{\isacharcomma}{\kern0pt}one{\isacharbrackright}{\kern0pt}\ {\isacharat}{\kern0pt}\ nenv{\isacharparenright}{\kern0pt}{\isachardoublequoteclose}\isanewline
%
\isadelimproof
\ \ %
\endisadelimproof
%
\isatagproof
\isacommand{using}\isamarkupfalse%
\ assms\ fst{\isacharunderscore}{\kern0pt}snd{\isacharunderscore}{\kern0pt}closed{\isacharbrackleft}{\kern0pt}OF\ {\isacartoucheopen}x{\isasymin}M{\isacartoucheclose}{\isacharbrackright}{\kern0pt}\isanewline
\ \ \ \ sats{\isacharunderscore}{\kern0pt}renpbdy{\isacharbrackleft}{\kern0pt}OF\ arity{\isacharunderscore}{\kern0pt}prebody{\isacharunderscore}{\kern0pt}fm\ {\isacharunderscore}{\kern0pt}\ prebody{\isacharunderscore}{\kern0pt}fm{\isacharunderscore}{\kern0pt}type{\isacharcomma}{\kern0pt}\ of\ concl{\isacharcolon}{\kern0pt}M{\isacharcomma}{\kern0pt}\ symmetric{\isacharbrackright}{\kern0pt}\isanewline
\ \ \isacommand{by}\isamarkupfalse%
\ force%
\endisatagproof
{\isafoldproof}%
%
\isadelimproof
\isanewline
%
\endisadelimproof
\isanewline
\isacommand{lemma}\isamarkupfalse%
\ body{\isacharunderscore}{\kern0pt}lemma{\isacharcolon}{\kern0pt}\isanewline
\ \ \isakeyword{assumes}\isanewline
\ \ \ \ {\isachardoublequoteopen}{\isasymexists}t\ p{\isachardot}{\kern0pt}\ x{\isacharequal}{\kern0pt}{\isasymlangle}t{\isacharcomma}{\kern0pt}p{\isasymrangle}{\isachardoublequoteclose}\ {\isachardoublequoteopen}x{\isasymin}M{\isachardoublequoteclose}\ {\isachardoublequoteopen}{\isacharbrackleft}{\kern0pt}x{\isacharcomma}{\kern0pt}{\isasymalpha}{\isacharcomma}{\kern0pt}m{\isacharcomma}{\kern0pt}P{\isacharcomma}{\kern0pt}leq{\isacharcomma}{\kern0pt}one{\isacharbrackright}{\kern0pt}\ {\isacharat}{\kern0pt}\ nenv\ {\isasymin}list{\isacharparenleft}{\kern0pt}M{\isacharparenright}{\kern0pt}{\isachardoublequoteclose}\isanewline
\ \ \ \ {\isachardoublequoteopen}{\isasymphi}{\isasymin}formula{\isachardoublequoteclose}\ {\isachardoublequoteopen}arity{\isacharparenleft}{\kern0pt}{\isasymphi}{\isacharparenright}{\kern0pt}\ {\isasymle}\ {\isadigit{2}}\ {\isacharhash}{\kern0pt}{\isacharplus}{\kern0pt}\ length{\isacharparenleft}{\kern0pt}nenv{\isacharparenright}{\kern0pt}{\isachardoublequoteclose}\isanewline
\ \ \isakeyword{shows}\isanewline
\ \ \ \ {\isachardoublequoteopen}sats{\isacharparenleft}{\kern0pt}M{\isacharcomma}{\kern0pt}body{\isacharunderscore}{\kern0pt}fm{\isacharparenleft}{\kern0pt}{\isasymphi}{\isacharcomma}{\kern0pt}nenv{\isacharparenright}{\kern0pt}{\isacharcomma}{\kern0pt}{\isacharbrackleft}{\kern0pt}{\isasymalpha}{\isacharcomma}{\kern0pt}x{\isacharcomma}{\kern0pt}m{\isacharcomma}{\kern0pt}P{\isacharcomma}{\kern0pt}leq{\isacharcomma}{\kern0pt}one{\isacharbrackright}{\kern0pt}\ {\isacharat}{\kern0pt}\ nenv{\isacharparenright}{\kern0pt}\ {\isasymlongleftrightarrow}\isanewline
\ \ {\isacharparenleft}{\kern0pt}{\isasymexists}{\isasymtau}{\isasymin}M{\isachardot}{\kern0pt}\ {\isasymexists}V{\isasymin}M{\isachardot}{\kern0pt}\ is{\isacharunderscore}{\kern0pt}Vset{\isacharparenleft}{\kern0pt}{\isasymlambda}a{\isachardot}{\kern0pt}\ {\isacharparenleft}{\kern0pt}{\isacharhash}{\kern0pt}{\isacharhash}{\kern0pt}M{\isacharparenright}{\kern0pt}{\isacharparenleft}{\kern0pt}a{\isacharparenright}{\kern0pt}{\isacharcomma}{\kern0pt}{\isasymalpha}{\isacharcomma}{\kern0pt}V{\isacharparenright}{\kern0pt}\ {\isasymand}\ {\isasymtau}\ {\isasymin}\ V\ {\isasymand}\ {\isacharparenleft}{\kern0pt}snd{\isacharparenleft}{\kern0pt}x{\isacharparenright}{\kern0pt}\ {\isasymtturnstile}\ {\isasymphi}\ {\isacharparenleft}{\kern0pt}{\isacharbrackleft}{\kern0pt}fst{\isacharparenleft}{\kern0pt}x{\isacharparenright}{\kern0pt}{\isacharcomma}{\kern0pt}{\isasymtau}{\isacharbrackright}{\kern0pt}{\isacharat}{\kern0pt}nenv{\isacharparenright}{\kern0pt}{\isacharparenright}{\kern0pt}{\isacharparenright}{\kern0pt}{\isachardoublequoteclose}\isanewline
%
\isadelimproof
\ \ %
\endisadelimproof
%
\isatagproof
\isacommand{using}\isamarkupfalse%
\ assms\ sats{\isacharunderscore}{\kern0pt}body{\isacharunderscore}{\kern0pt}fm{\isacharbrackleft}{\kern0pt}of\ x\ {\isasymalpha}\ m\ nenv{\isacharbrackright}{\kern0pt}\ sats{\isacharunderscore}{\kern0pt}renpbdy{\isacharunderscore}{\kern0pt}prebody{\isacharunderscore}{\kern0pt}fm{\isacharbrackleft}{\kern0pt}of\ x\ {\isasymalpha}{\isacharbrackright}{\kern0pt}\isanewline
\ \ \ \ sats{\isacharunderscore}{\kern0pt}prebody{\isacharunderscore}{\kern0pt}fm{\isacharbrackleft}{\kern0pt}of\ {\isachardoublequoteopen}snd{\isacharparenleft}{\kern0pt}x{\isacharparenright}{\kern0pt}{\isachardoublequoteclose}\ {\isachardoublequoteopen}fst{\isacharparenleft}{\kern0pt}x{\isacharparenright}{\kern0pt}{\isachardoublequoteclose}{\isacharbrackright}{\kern0pt}\ fst{\isacharunderscore}{\kern0pt}snd{\isacharunderscore}{\kern0pt}closed{\isacharbrackleft}{\kern0pt}OF\ {\isacartoucheopen}x{\isasymin}M{\isacartoucheclose}{\isacharbrackright}{\kern0pt}\isanewline
\ \ \isacommand{by}\isamarkupfalse%
\ {\isacharparenleft}{\kern0pt}simp{\isacharcomma}{\kern0pt}\ simp\ flip{\isacharcolon}{\kern0pt}\ setclass{\isacharunderscore}{\kern0pt}iff{\isacharcomma}{\kern0pt}simp{\isacharparenright}{\kern0pt}%
\endisatagproof
{\isafoldproof}%
%
\isadelimproof
\isanewline
%
\endisadelimproof
\isanewline
\isacommand{lemma}\isamarkupfalse%
\ Replace{\isacharunderscore}{\kern0pt}sats{\isacharunderscore}{\kern0pt}in{\isacharunderscore}{\kern0pt}MG{\isacharcolon}{\kern0pt}\isanewline
\ \ \isakeyword{assumes}\isanewline
\ \ \ \ {\isachardoublequoteopen}c{\isasymin}M{\isacharbrackleft}{\kern0pt}G{\isacharbrackright}{\kern0pt}{\isachardoublequoteclose}\ {\isachardoublequoteopen}env\ {\isasymin}\ list{\isacharparenleft}{\kern0pt}M{\isacharbrackleft}{\kern0pt}G{\isacharbrackright}{\kern0pt}{\isacharparenright}{\kern0pt}{\isachardoublequoteclose}\isanewline
\ \ \ \ {\isachardoublequoteopen}{\isasymphi}\ {\isasymin}\ formula{\isachardoublequoteclose}\ {\isachardoublequoteopen}arity{\isacharparenleft}{\kern0pt}{\isasymphi}{\isacharparenright}{\kern0pt}\ {\isasymle}\ {\isadigit{2}}\ {\isacharhash}{\kern0pt}{\isacharplus}{\kern0pt}\ length{\isacharparenleft}{\kern0pt}env{\isacharparenright}{\kern0pt}{\isachardoublequoteclose}\isanewline
\ \ \ \ {\isachardoublequoteopen}univalent{\isacharparenleft}{\kern0pt}{\isacharhash}{\kern0pt}{\isacharhash}{\kern0pt}M{\isacharbrackleft}{\kern0pt}G{\isacharbrackright}{\kern0pt}{\isacharcomma}{\kern0pt}\ c{\isacharcomma}{\kern0pt}\ {\isasymlambda}x\ v{\isachardot}{\kern0pt}\ {\isacharparenleft}{\kern0pt}M{\isacharbrackleft}{\kern0pt}G{\isacharbrackright}{\kern0pt}\ {\isacharcomma}{\kern0pt}\ {\isacharbrackleft}{\kern0pt}x{\isacharcomma}{\kern0pt}v{\isacharbrackright}{\kern0pt}{\isacharat}{\kern0pt}env\ {\isasymTurnstile}\ {\isasymphi}{\isacharparenright}{\kern0pt}\ {\isacharparenright}{\kern0pt}{\isachardoublequoteclose}\isanewline
\ \ \isakeyword{shows}\isanewline
\ \ \ \ {\isachardoublequoteopen}{\isacharbraceleft}{\kern0pt}v{\isachardot}{\kern0pt}\ x{\isasymin}c{\isacharcomma}{\kern0pt}\ v{\isasymin}M{\isacharbrackleft}{\kern0pt}G{\isacharbrackright}{\kern0pt}\ {\isasymand}\ {\isacharparenleft}{\kern0pt}M{\isacharbrackleft}{\kern0pt}G{\isacharbrackright}{\kern0pt}\ {\isacharcomma}{\kern0pt}\ {\isacharbrackleft}{\kern0pt}x{\isacharcomma}{\kern0pt}v{\isacharbrackright}{\kern0pt}{\isacharat}{\kern0pt}env\ {\isasymTurnstile}\ {\isasymphi}{\isacharparenright}{\kern0pt}{\isacharbraceright}{\kern0pt}\ {\isasymin}\ M{\isacharbrackleft}{\kern0pt}G{\isacharbrackright}{\kern0pt}{\isachardoublequoteclose}\isanewline
%
\isadelimproof
%
\endisadelimproof
%
\isatagproof
\isacommand{proof}\isamarkupfalse%
\ {\isacharminus}{\kern0pt}\isanewline
\ \ \isacommand{let}\isamarkupfalse%
\ {\isacharquery}{\kern0pt}R\ {\isacharequal}{\kern0pt}\ {\isachardoublequoteopen}{\isasymlambda}\ x\ v\ {\isachardot}{\kern0pt}\ v{\isasymin}M{\isacharbrackleft}{\kern0pt}G{\isacharbrackright}{\kern0pt}\ {\isasymand}\ {\isacharparenleft}{\kern0pt}M{\isacharbrackleft}{\kern0pt}G{\isacharbrackright}{\kern0pt}\ {\isacharcomma}{\kern0pt}\ {\isacharbrackleft}{\kern0pt}x{\isacharcomma}{\kern0pt}v{\isacharbrackright}{\kern0pt}{\isacharat}{\kern0pt}env\ {\isasymTurnstile}\ {\isasymphi}{\isacharparenright}{\kern0pt}{\isachardoublequoteclose}\isanewline
\ \ \isacommand{from}\isamarkupfalse%
\ {\isacartoucheopen}c{\isasymin}M{\isacharbrackleft}{\kern0pt}G{\isacharbrackright}{\kern0pt}{\isacartoucheclose}\isanewline
\ \ \isacommand{obtain}\isamarkupfalse%
\ {\isasympi}{\isacharprime}{\kern0pt}\ \isakeyword{where}\ {\isachardoublequoteopen}val{\isacharparenleft}{\kern0pt}G{\isacharcomma}{\kern0pt}\ {\isasympi}{\isacharprime}{\kern0pt}{\isacharparenright}{\kern0pt}\ {\isacharequal}{\kern0pt}\ c{\isachardoublequoteclose}\ {\isachardoublequoteopen}{\isasympi}{\isacharprime}{\kern0pt}\ {\isasymin}\ M{\isachardoublequoteclose}\isanewline
\ \ \ \ \isacommand{using}\isamarkupfalse%
\ GenExt{\isacharunderscore}{\kern0pt}def\ \isacommand{by}\isamarkupfalse%
\ auto\isanewline
\ \ \isacommand{then}\isamarkupfalse%
\isanewline
\ \ \isacommand{have}\isamarkupfalse%
\ {\isachardoublequoteopen}domain{\isacharparenleft}{\kern0pt}{\isasympi}{\isacharprime}{\kern0pt}{\isacharparenright}{\kern0pt}{\isasymtimes}P{\isasymin}M{\isachardoublequoteclose}\ {\isacharparenleft}{\kern0pt}\isakeyword{is}\ {\isachardoublequoteopen}{\isacharquery}{\kern0pt}{\isasympi}{\isasymin}M{\isachardoublequoteclose}{\isacharparenright}{\kern0pt}\isanewline
\ \ \ \ \isacommand{using}\isamarkupfalse%
\ cartprod{\isacharunderscore}{\kern0pt}closed\ P{\isacharunderscore}{\kern0pt}in{\isacharunderscore}{\kern0pt}M\ domain{\isacharunderscore}{\kern0pt}closed\ \isacommand{by}\isamarkupfalse%
\ simp\isanewline
\ \ \isacommand{from}\isamarkupfalse%
\ {\isacartoucheopen}val{\isacharparenleft}{\kern0pt}G{\isacharcomma}{\kern0pt}\ {\isasympi}{\isacharprime}{\kern0pt}{\isacharparenright}{\kern0pt}\ {\isacharequal}{\kern0pt}\ c{\isacartoucheclose}\isanewline
\ \ \isacommand{have}\isamarkupfalse%
\ {\isachardoublequoteopen}c\ {\isasymsubseteq}\ val{\isacharparenleft}{\kern0pt}G{\isacharcomma}{\kern0pt}{\isacharquery}{\kern0pt}{\isasympi}{\isacharparenright}{\kern0pt}{\isachardoublequoteclose}\isanewline
\ \ \ \ \isacommand{using}\isamarkupfalse%
\ def{\isacharunderscore}{\kern0pt}val{\isacharbrackleft}{\kern0pt}of\ G\ {\isacharquery}{\kern0pt}{\isasympi}{\isacharbrackright}{\kern0pt}\ one{\isacharunderscore}{\kern0pt}in{\isacharunderscore}{\kern0pt}P\ one{\isacharunderscore}{\kern0pt}in{\isacharunderscore}{\kern0pt}G{\isacharbrackleft}{\kern0pt}OF\ generic{\isacharbrackright}{\kern0pt}\ elem{\isacharunderscore}{\kern0pt}of{\isacharunderscore}{\kern0pt}val\isanewline
\ \ \ \ \ \ domain{\isacharunderscore}{\kern0pt}of{\isacharunderscore}{\kern0pt}prod{\isacharbrackleft}{\kern0pt}OF\ one{\isacharunderscore}{\kern0pt}in{\isacharunderscore}{\kern0pt}P{\isacharcomma}{\kern0pt}\ of\ {\isachardoublequoteopen}domain{\isacharparenleft}{\kern0pt}{\isasympi}{\isacharprime}{\kern0pt}{\isacharparenright}{\kern0pt}{\isachardoublequoteclose}{\isacharbrackright}{\kern0pt}\ \isacommand{by}\isamarkupfalse%
\ force\isanewline
\ \ \isacommand{from}\isamarkupfalse%
\ {\isacartoucheopen}env\ {\isasymin}\ {\isacharunderscore}{\kern0pt}{\isacartoucheclose}\isanewline
\ \ \isacommand{obtain}\isamarkupfalse%
\ nenv\ \isakeyword{where}\ {\isachardoublequoteopen}nenv{\isasymin}list{\isacharparenleft}{\kern0pt}M{\isacharparenright}{\kern0pt}{\isachardoublequoteclose}\ {\isachardoublequoteopen}env\ {\isacharequal}{\kern0pt}\ map{\isacharparenleft}{\kern0pt}val{\isacharparenleft}{\kern0pt}G{\isacharparenright}{\kern0pt}{\isacharcomma}{\kern0pt}nenv{\isacharparenright}{\kern0pt}{\isachardoublequoteclose}\isanewline
\ \ \ \ \isacommand{using}\isamarkupfalse%
\ map{\isacharunderscore}{\kern0pt}val\ \isacommand{by}\isamarkupfalse%
\ auto\isanewline
\ \ \isacommand{then}\isamarkupfalse%
\isanewline
\ \ \isacommand{have}\isamarkupfalse%
\ {\isachardoublequoteopen}length{\isacharparenleft}{\kern0pt}nenv{\isacharparenright}{\kern0pt}\ {\isacharequal}{\kern0pt}\ length{\isacharparenleft}{\kern0pt}env{\isacharparenright}{\kern0pt}{\isachardoublequoteclose}\ \isacommand{by}\isamarkupfalse%
\ simp\isanewline
\ \ \isacommand{define}\isamarkupfalse%
\ f\ \isakeyword{where}\ {\isachardoublequoteopen}f{\isacharparenleft}{\kern0pt}{\isasymrho}p{\isacharparenright}{\kern0pt}\ {\isasymequiv}\ {\isasymmu}\ {\isasymalpha}{\isachardot}{\kern0pt}\ {\isasymalpha}{\isasymin}M\ {\isasymand}\ {\isacharparenleft}{\kern0pt}{\isasymexists}{\isasymtau}{\isasymin}M{\isachardot}{\kern0pt}\ {\isasymtau}\ {\isasymin}\ Vset{\isacharparenleft}{\kern0pt}{\isasymalpha}{\isacharparenright}{\kern0pt}\ {\isasymand}\isanewline
\ \ \ \ \ \ \ \ {\isacharparenleft}{\kern0pt}snd{\isacharparenleft}{\kern0pt}{\isasymrho}p{\isacharparenright}{\kern0pt}\ {\isasymtturnstile}\ {\isasymphi}\ {\isacharparenleft}{\kern0pt}{\isacharbrackleft}{\kern0pt}fst{\isacharparenleft}{\kern0pt}{\isasymrho}p{\isacharparenright}{\kern0pt}{\isacharcomma}{\kern0pt}{\isasymtau}{\isacharbrackright}{\kern0pt}\ {\isacharat}{\kern0pt}\ nenv{\isacharparenright}{\kern0pt}{\isacharparenright}{\kern0pt}{\isacharparenright}{\kern0pt}{\isachardoublequoteclose}\ {\isacharparenleft}{\kern0pt}\isakeyword{is}\ {\isachardoublequoteopen}{\isacharunderscore}{\kern0pt}\ {\isasymequiv}\ {\isasymmu}\ {\isasymalpha}{\isachardot}{\kern0pt}\ {\isacharquery}{\kern0pt}P{\isacharparenleft}{\kern0pt}{\isasymrho}p{\isacharcomma}{\kern0pt}{\isasymalpha}{\isacharparenright}{\kern0pt}{\isachardoublequoteclose}{\isacharparenright}{\kern0pt}\ \isakeyword{for}\ {\isasymrho}p\isanewline
\ \ \isacommand{have}\isamarkupfalse%
\ {\isachardoublequoteopen}f{\isacharparenleft}{\kern0pt}{\isasymrho}p{\isacharparenright}{\kern0pt}\ {\isacharequal}{\kern0pt}\ {\isacharparenleft}{\kern0pt}{\isasymmu}\ {\isasymalpha}{\isachardot}{\kern0pt}\ {\isasymalpha}{\isasymin}M\ {\isasymand}\ {\isacharparenleft}{\kern0pt}{\isasymexists}{\isasymtau}{\isasymin}M{\isachardot}{\kern0pt}\ {\isasymexists}V{\isasymin}M{\isachardot}{\kern0pt}\ is{\isacharunderscore}{\kern0pt}Vset{\isacharparenleft}{\kern0pt}{\isacharhash}{\kern0pt}{\isacharhash}{\kern0pt}M{\isacharcomma}{\kern0pt}{\isasymalpha}{\isacharcomma}{\kern0pt}V{\isacharparenright}{\kern0pt}\ {\isasymand}\ {\isasymtau}{\isasymin}V\ {\isasymand}\isanewline
\ \ \ \ \ \ \ \ {\isacharparenleft}{\kern0pt}snd{\isacharparenleft}{\kern0pt}{\isasymrho}p{\isacharparenright}{\kern0pt}\ {\isasymtturnstile}\ {\isasymphi}\ {\isacharparenleft}{\kern0pt}{\isacharbrackleft}{\kern0pt}fst{\isacharparenleft}{\kern0pt}{\isasymrho}p{\isacharparenright}{\kern0pt}{\isacharcomma}{\kern0pt}{\isasymtau}{\isacharbrackright}{\kern0pt}\ {\isacharat}{\kern0pt}\ nenv{\isacharparenright}{\kern0pt}{\isacharparenright}{\kern0pt}{\isacharparenright}{\kern0pt}{\isacharparenright}{\kern0pt}{\isachardoublequoteclose}\ {\isacharparenleft}{\kern0pt}\isakeyword{is}\ {\isachardoublequoteopen}{\isacharunderscore}{\kern0pt}\ {\isacharequal}{\kern0pt}\ {\isacharparenleft}{\kern0pt}{\isasymmu}\ {\isasymalpha}{\isachardot}{\kern0pt}\ {\isasymalpha}{\isasymin}M\ {\isasymand}\ {\isacharquery}{\kern0pt}Q{\isacharparenleft}{\kern0pt}{\isasymrho}p{\isacharcomma}{\kern0pt}{\isasymalpha}{\isacharparenright}{\kern0pt}{\isacharparenright}{\kern0pt}{\isachardoublequoteclose}{\isacharparenright}{\kern0pt}\ \isakeyword{for}\ {\isasymrho}p\isanewline
\ \ \ \ \isacommand{unfolding}\isamarkupfalse%
\ f{\isacharunderscore}{\kern0pt}def\ \isacommand{using}\isamarkupfalse%
\ Vset{\isacharunderscore}{\kern0pt}abs\ Vset{\isacharunderscore}{\kern0pt}closed\ Ord{\isacharunderscore}{\kern0pt}Least{\isacharunderscore}{\kern0pt}cong{\isacharbrackleft}{\kern0pt}of\ {\isachardoublequoteopen}{\isacharquery}{\kern0pt}P{\isacharparenleft}{\kern0pt}{\isasymrho}p{\isacharparenright}{\kern0pt}{\isachardoublequoteclose}\ {\isachardoublequoteopen}{\isasymlambda}\ {\isasymalpha}{\isachardot}{\kern0pt}\ {\isasymalpha}{\isasymin}M\ {\isasymand}\ {\isacharquery}{\kern0pt}Q{\isacharparenleft}{\kern0pt}{\isasymrho}p{\isacharcomma}{\kern0pt}{\isasymalpha}{\isacharparenright}{\kern0pt}{\isachardoublequoteclose}{\isacharbrackright}{\kern0pt}\isanewline
\ \ \ \ \isacommand{by}\isamarkupfalse%
\ {\isacharparenleft}{\kern0pt}simp{\isacharcomma}{\kern0pt}\ simp\ del{\isacharcolon}{\kern0pt}setclass{\isacharunderscore}{\kern0pt}iff{\isacharparenright}{\kern0pt}\isanewline
\ \ \isacommand{moreover}\isamarkupfalse%
\isanewline
\ \ \isacommand{have}\isamarkupfalse%
\ {\isachardoublequoteopen}f{\isacharparenleft}{\kern0pt}{\isasymrho}p{\isacharparenright}{\kern0pt}\ {\isasymin}\ M{\isachardoublequoteclose}\ \isakeyword{for}\ {\isasymrho}p\isanewline
\ \ \ \ \isacommand{unfolding}\isamarkupfalse%
\ f{\isacharunderscore}{\kern0pt}def\ \isacommand{using}\isamarkupfalse%
\ Least{\isacharunderscore}{\kern0pt}closed{\isacharbrackleft}{\kern0pt}of\ {\isachardoublequoteopen}{\isacharquery}{\kern0pt}P{\isacharparenleft}{\kern0pt}{\isasymrho}p{\isacharparenright}{\kern0pt}{\isachardoublequoteclose}{\isacharbrackright}{\kern0pt}\ \isacommand{by}\isamarkupfalse%
\ simp\isanewline
\ \ \isacommand{ultimately}\isamarkupfalse%
\isanewline
\ \ \isacommand{have}\isamarkupfalse%
\ {\isadigit{1}}{\isacharcolon}{\kern0pt}{\isachardoublequoteopen}least{\isacharparenleft}{\kern0pt}{\isacharhash}{\kern0pt}{\isacharhash}{\kern0pt}M{\isacharcomma}{\kern0pt}{\isasymlambda}{\isasymalpha}{\isachardot}{\kern0pt}\ {\isacharquery}{\kern0pt}Q{\isacharparenleft}{\kern0pt}{\isasymrho}p{\isacharcomma}{\kern0pt}{\isasymalpha}{\isacharparenright}{\kern0pt}{\isacharcomma}{\kern0pt}f{\isacharparenleft}{\kern0pt}{\isasymrho}p{\isacharparenright}{\kern0pt}{\isacharparenright}{\kern0pt}{\isachardoublequoteclose}\ \isakeyword{for}\ {\isasymrho}p\isanewline
\ \ \ \ \isacommand{using}\isamarkupfalse%
\ least{\isacharunderscore}{\kern0pt}abs{\isacharbrackleft}{\kern0pt}of\ {\isachardoublequoteopen}{\isasymlambda}{\isasymalpha}{\isachardot}{\kern0pt}\ {\isasymalpha}{\isasymin}M\ {\isasymand}\ {\isacharquery}{\kern0pt}Q{\isacharparenleft}{\kern0pt}{\isasymrho}p{\isacharcomma}{\kern0pt}{\isasymalpha}{\isacharparenright}{\kern0pt}{\isachardoublequoteclose}\ {\isachardoublequoteopen}f{\isacharparenleft}{\kern0pt}{\isasymrho}p{\isacharparenright}{\kern0pt}{\isachardoublequoteclose}{\isacharbrackright}{\kern0pt}\ least{\isacharunderscore}{\kern0pt}conj\isanewline
\ \ \ \ \isacommand{by}\isamarkupfalse%
\ {\isacharparenleft}{\kern0pt}simp\ flip{\isacharcolon}{\kern0pt}\ setclass{\isacharunderscore}{\kern0pt}iff{\isacharparenright}{\kern0pt}\isanewline
\ \ \isacommand{have}\isamarkupfalse%
\ {\isachardoublequoteopen}Ord{\isacharparenleft}{\kern0pt}f{\isacharparenleft}{\kern0pt}{\isasymrho}p{\isacharparenright}{\kern0pt}{\isacharparenright}{\kern0pt}{\isachardoublequoteclose}\ \isakeyword{for}\ {\isasymrho}p\ \isacommand{unfolding}\isamarkupfalse%
\ f{\isacharunderscore}{\kern0pt}def\ \isacommand{by}\isamarkupfalse%
\ simp\isanewline
\ \ \isacommand{define}\isamarkupfalse%
\ QQ\ \isakeyword{where}\ {\isachardoublequoteopen}QQ{\isasymequiv}{\isacharquery}{\kern0pt}Q{\isachardoublequoteclose}\isanewline
\ \ \isacommand{from}\isamarkupfalse%
\ {\isadigit{1}}\isanewline
\ \ \isacommand{have}\isamarkupfalse%
\ {\isachardoublequoteopen}least{\isacharparenleft}{\kern0pt}{\isacharhash}{\kern0pt}{\isacharhash}{\kern0pt}M{\isacharcomma}{\kern0pt}{\isasymlambda}{\isasymalpha}{\isachardot}{\kern0pt}\ QQ{\isacharparenleft}{\kern0pt}{\isasymrho}p{\isacharcomma}{\kern0pt}{\isasymalpha}{\isacharparenright}{\kern0pt}{\isacharcomma}{\kern0pt}f{\isacharparenleft}{\kern0pt}{\isasymrho}p{\isacharparenright}{\kern0pt}{\isacharparenright}{\kern0pt}{\isachardoublequoteclose}\ \isakeyword{for}\ {\isasymrho}p\isanewline
\ \ \ \ \isacommand{unfolding}\isamarkupfalse%
\ QQ{\isacharunderscore}{\kern0pt}def\ \isacommand{{\isachardot}{\kern0pt}}\isamarkupfalse%
\isanewline
\ \ \isacommand{from}\isamarkupfalse%
\ {\isacartoucheopen}arity{\isacharparenleft}{\kern0pt}{\isasymphi}{\isacharparenright}{\kern0pt}\ {\isasymle}\ {\isacharunderscore}{\kern0pt}{\isacartoucheclose}\ {\isacartoucheopen}length{\isacharparenleft}{\kern0pt}nenv{\isacharparenright}{\kern0pt}\ {\isacharequal}{\kern0pt}\ {\isacharunderscore}{\kern0pt}{\isacartoucheclose}\isanewline
\ \ \isacommand{have}\isamarkupfalse%
\ {\isachardoublequoteopen}arity{\isacharparenleft}{\kern0pt}{\isasymphi}{\isacharparenright}{\kern0pt}\ {\isasymle}\ {\isadigit{2}}\ {\isacharhash}{\kern0pt}{\isacharplus}{\kern0pt}\ length{\isacharparenleft}{\kern0pt}nenv{\isacharparenright}{\kern0pt}{\isachardoublequoteclose}\isanewline
\ \ \ \ \isacommand{by}\isamarkupfalse%
\ simp\isanewline
\ \ \isacommand{moreover}\isamarkupfalse%
\isanewline
\ \ \isacommand{note}\isamarkupfalse%
\ assms\ {\isacartoucheopen}nenv{\isasymin}list{\isacharparenleft}{\kern0pt}M{\isacharparenright}{\kern0pt}{\isacartoucheclose}\ {\isacartoucheopen}{\isacharquery}{\kern0pt}{\isasympi}{\isasymin}M{\isacartoucheclose}\isanewline
\ \ \isacommand{moreover}\isamarkupfalse%
\isanewline
\ \ \isacommand{have}\isamarkupfalse%
\ {\isachardoublequoteopen}{\isasymrho}p{\isasymin}{\isacharquery}{\kern0pt}{\isasympi}\ {\isasymLongrightarrow}\ {\isasymexists}t\ p{\isachardot}{\kern0pt}\ {\isasymrho}p{\isacharequal}{\kern0pt}{\isasymlangle}t{\isacharcomma}{\kern0pt}p{\isasymrangle}{\isachardoublequoteclose}\ \isakeyword{for}\ {\isasymrho}p\isanewline
\ \ \ \ \isacommand{by}\isamarkupfalse%
\ auto\isanewline
\ \ \isacommand{ultimately}\isamarkupfalse%
\isanewline
\ \ \isacommand{have}\isamarkupfalse%
\ body{\isacharcolon}{\kern0pt}{\isachardoublequoteopen}M\ {\isacharcomma}{\kern0pt}\ {\isacharbrackleft}{\kern0pt}{\isasymalpha}{\isacharcomma}{\kern0pt}{\isasymrho}p{\isacharcomma}{\kern0pt}m{\isacharcomma}{\kern0pt}P{\isacharcomma}{\kern0pt}leq{\isacharcomma}{\kern0pt}one{\isacharbrackright}{\kern0pt}\ {\isacharat}{\kern0pt}\ nenv\ {\isasymTurnstile}\ body{\isacharunderscore}{\kern0pt}fm{\isacharparenleft}{\kern0pt}{\isasymphi}{\isacharcomma}{\kern0pt}nenv{\isacharparenright}{\kern0pt}\ {\isasymlongleftrightarrow}\ {\isacharquery}{\kern0pt}Q{\isacharparenleft}{\kern0pt}{\isasymrho}p{\isacharcomma}{\kern0pt}{\isasymalpha}{\isacharparenright}{\kern0pt}{\isachardoublequoteclose}\isanewline
\ \ \ \ \isakeyword{if}\ {\isachardoublequoteopen}{\isasymrho}p{\isasymin}{\isacharquery}{\kern0pt}{\isasympi}{\isachardoublequoteclose}\ {\isachardoublequoteopen}{\isasymrho}p{\isasymin}M{\isachardoublequoteclose}\ {\isachardoublequoteopen}m{\isasymin}M{\isachardoublequoteclose}\ {\isachardoublequoteopen}{\isasymalpha}{\isasymin}M{\isachardoublequoteclose}\ \isakeyword{for}\ {\isasymalpha}\ {\isasymrho}p\ m\isanewline
\ \ \ \ \isacommand{using}\isamarkupfalse%
\ that\ P{\isacharunderscore}{\kern0pt}in{\isacharunderscore}{\kern0pt}M\ leq{\isacharunderscore}{\kern0pt}in{\isacharunderscore}{\kern0pt}M\ one{\isacharunderscore}{\kern0pt}in{\isacharunderscore}{\kern0pt}M\ body{\isacharunderscore}{\kern0pt}lemma{\isacharbrackleft}{\kern0pt}of\ {\isasymrho}p\ {\isasymalpha}\ m\ nenv\ {\isasymphi}{\isacharbrackright}{\kern0pt}\ \isacommand{by}\isamarkupfalse%
\ simp\isanewline
\ \ \isacommand{let}\isamarkupfalse%
\ {\isacharquery}{\kern0pt}f{\isacharunderscore}{\kern0pt}fm{\isacharequal}{\kern0pt}{\isachardoublequoteopen}least{\isacharunderscore}{\kern0pt}fm{\isacharparenleft}{\kern0pt}body{\isacharunderscore}{\kern0pt}fm{\isacharparenleft}{\kern0pt}{\isasymphi}{\isacharcomma}{\kern0pt}nenv{\isacharparenright}{\kern0pt}{\isacharcomma}{\kern0pt}{\isadigit{1}}{\isacharparenright}{\kern0pt}{\isachardoublequoteclose}\isanewline
\ \ \isacommand{{\isacharbraceleft}{\kern0pt}}\isamarkupfalse%
\isanewline
\ \ \ \ \isacommand{fix}\isamarkupfalse%
\ {\isasymrho}p\ m\isanewline
\ \ \ \ \isacommand{assume}\isamarkupfalse%
\ asm{\isacharcolon}{\kern0pt}\ {\isachardoublequoteopen}{\isasymrho}p{\isasymin}M{\isachardoublequoteclose}\ {\isachardoublequoteopen}{\isasymrho}p{\isasymin}{\isacharquery}{\kern0pt}{\isasympi}{\isachardoublequoteclose}\ {\isachardoublequoteopen}m{\isasymin}M{\isachardoublequoteclose}\isanewline
\ \ \ \ \isacommand{note}\isamarkupfalse%
\ inM\ {\isacharequal}{\kern0pt}\ this\ P{\isacharunderscore}{\kern0pt}in{\isacharunderscore}{\kern0pt}M\ leq{\isacharunderscore}{\kern0pt}in{\isacharunderscore}{\kern0pt}M\ one{\isacharunderscore}{\kern0pt}in{\isacharunderscore}{\kern0pt}M\ {\isacartoucheopen}nenv{\isasymin}list{\isacharparenleft}{\kern0pt}M{\isacharparenright}{\kern0pt}{\isacartoucheclose}\isanewline
\ \ \ \ \isacommand{with}\isamarkupfalse%
\ body\isanewline
\ \ \ \ \isacommand{have}\isamarkupfalse%
\ body{\isacharprime}{\kern0pt}{\isacharcolon}{\kern0pt}{\isachardoublequoteopen}{\isasymAnd}{\isasymalpha}{\isachardot}{\kern0pt}\ {\isasymalpha}\ {\isasymin}\ M\ {\isasymLongrightarrow}\ {\isacharparenleft}{\kern0pt}{\isasymexists}{\isasymtau}{\isasymin}M{\isachardot}{\kern0pt}\ {\isasymexists}V{\isasymin}M{\isachardot}{\kern0pt}\ is{\isacharunderscore}{\kern0pt}Vset{\isacharparenleft}{\kern0pt}{\isasymlambda}a{\isachardot}{\kern0pt}\ {\isacharparenleft}{\kern0pt}{\isacharhash}{\kern0pt}{\isacharhash}{\kern0pt}M{\isacharparenright}{\kern0pt}{\isacharparenleft}{\kern0pt}a{\isacharparenright}{\kern0pt}{\isacharcomma}{\kern0pt}\ {\isasymalpha}{\isacharcomma}{\kern0pt}\ V{\isacharparenright}{\kern0pt}\ {\isasymand}\ {\isasymtau}\ {\isasymin}\ V\ {\isasymand}\isanewline
\ \ \ \ \ \ \ \ \ \ {\isacharparenleft}{\kern0pt}snd{\isacharparenleft}{\kern0pt}{\isasymrho}p{\isacharparenright}{\kern0pt}\ {\isasymtturnstile}\ {\isasymphi}\ {\isacharparenleft}{\kern0pt}{\isacharbrackleft}{\kern0pt}fst{\isacharparenleft}{\kern0pt}{\isasymrho}p{\isacharparenright}{\kern0pt}{\isacharcomma}{\kern0pt}{\isasymtau}{\isacharbrackright}{\kern0pt}\ {\isacharat}{\kern0pt}\ nenv{\isacharparenright}{\kern0pt}{\isacharparenright}{\kern0pt}{\isacharparenright}{\kern0pt}\ {\isasymlongleftrightarrow}\isanewline
\ \ \ \ \ \ \ \ \ \ M{\isacharcomma}{\kern0pt}\ Cons{\isacharparenleft}{\kern0pt}{\isasymalpha}{\isacharcomma}{\kern0pt}\ {\isacharbrackleft}{\kern0pt}{\isasymrho}p{\isacharcomma}{\kern0pt}\ m{\isacharcomma}{\kern0pt}\ P{\isacharcomma}{\kern0pt}\ leq{\isacharcomma}{\kern0pt}\ one{\isacharbrackright}{\kern0pt}\ {\isacharat}{\kern0pt}\ nenv{\isacharparenright}{\kern0pt}\ {\isasymTurnstile}\ body{\isacharunderscore}{\kern0pt}fm{\isacharparenleft}{\kern0pt}{\isasymphi}{\isacharcomma}{\kern0pt}nenv{\isacharparenright}{\kern0pt}{\isachardoublequoteclose}\ \isacommand{by}\isamarkupfalse%
\ simp\isanewline
\ \ \ \ \isacommand{from}\isamarkupfalse%
\ inM\isanewline
\ \ \ \ \isacommand{have}\isamarkupfalse%
\ {\isachardoublequoteopen}M\ {\isacharcomma}{\kern0pt}\ {\isacharbrackleft}{\kern0pt}{\isasymrho}p{\isacharcomma}{\kern0pt}m{\isacharcomma}{\kern0pt}P{\isacharcomma}{\kern0pt}leq{\isacharcomma}{\kern0pt}one{\isacharbrackright}{\kern0pt}\ {\isacharat}{\kern0pt}\ nenv\ {\isasymTurnstile}\ {\isacharquery}{\kern0pt}f{\isacharunderscore}{\kern0pt}fm\ {\isasymlongleftrightarrow}\ least{\isacharparenleft}{\kern0pt}{\isacharhash}{\kern0pt}{\isacharhash}{\kern0pt}M{\isacharcomma}{\kern0pt}\ QQ{\isacharparenleft}{\kern0pt}{\isasymrho}p{\isacharparenright}{\kern0pt}{\isacharcomma}{\kern0pt}\ m{\isacharparenright}{\kern0pt}{\isachardoublequoteclose}\isanewline
\ \ \ \ \ \ \isacommand{using}\isamarkupfalse%
\ sats{\isacharunderscore}{\kern0pt}least{\isacharunderscore}{\kern0pt}fm{\isacharbrackleft}{\kern0pt}OF\ body{\isacharprime}{\kern0pt}{\isacharcomma}{\kern0pt}\ of\ {\isadigit{1}}{\isacharbrackright}{\kern0pt}\ \isacommand{unfolding}\isamarkupfalse%
\ QQ{\isacharunderscore}{\kern0pt}def\isanewline
\ \ \ \ \ \ \isacommand{by}\isamarkupfalse%
\ {\isacharparenleft}{\kern0pt}simp{\isacharcomma}{\kern0pt}\ simp\ flip{\isacharcolon}{\kern0pt}\ setclass{\isacharunderscore}{\kern0pt}iff{\isacharparenright}{\kern0pt}\isanewline
\ \ \isacommand{{\isacharbraceright}{\kern0pt}}\isamarkupfalse%
\isanewline
\ \ \isacommand{then}\isamarkupfalse%
\isanewline
\ \ \isacommand{have}\isamarkupfalse%
\ {\isachardoublequoteopen}M{\isacharcomma}{\kern0pt}\ {\isacharbrackleft}{\kern0pt}{\isasymrho}p{\isacharcomma}{\kern0pt}m{\isacharcomma}{\kern0pt}P{\isacharcomma}{\kern0pt}leq{\isacharcomma}{\kern0pt}one{\isacharbrackright}{\kern0pt}\ {\isacharat}{\kern0pt}\ nenv\ {\isasymTurnstile}\ {\isacharquery}{\kern0pt}f{\isacharunderscore}{\kern0pt}fm\ {\isasymlongleftrightarrow}\ least{\isacharparenleft}{\kern0pt}{\isacharhash}{\kern0pt}{\isacharhash}{\kern0pt}M{\isacharcomma}{\kern0pt}\ QQ{\isacharparenleft}{\kern0pt}{\isasymrho}p{\isacharparenright}{\kern0pt}{\isacharcomma}{\kern0pt}\ m{\isacharparenright}{\kern0pt}{\isachardoublequoteclose}\isanewline
\ \ \ \ \isakeyword{if}\ {\isachardoublequoteopen}{\isasymrho}p{\isasymin}M{\isachardoublequoteclose}\ {\isachardoublequoteopen}{\isasymrho}p{\isasymin}{\isacharquery}{\kern0pt}{\isasympi}{\isachardoublequoteclose}\ {\isachardoublequoteopen}m{\isasymin}M{\isachardoublequoteclose}\ \isakeyword{for}\ {\isasymrho}p\ m\ \isacommand{using}\isamarkupfalse%
\ that\ \isacommand{by}\isamarkupfalse%
\ simp\isanewline
\ \ \isacommand{then}\isamarkupfalse%
\isanewline
\ \ \isacommand{have}\isamarkupfalse%
\ {\isachardoublequoteopen}univalent{\isacharparenleft}{\kern0pt}{\isacharhash}{\kern0pt}{\isacharhash}{\kern0pt}M{\isacharcomma}{\kern0pt}\ {\isacharquery}{\kern0pt}{\isasympi}{\isacharcomma}{\kern0pt}\ {\isasymlambda}{\isasymrho}p\ m{\isachardot}{\kern0pt}\ M\ {\isacharcomma}{\kern0pt}\ {\isacharbrackleft}{\kern0pt}{\isasymrho}p{\isacharcomma}{\kern0pt}m{\isacharbrackright}{\kern0pt}\ {\isacharat}{\kern0pt}\ {\isacharparenleft}{\kern0pt}{\isacharbrackleft}{\kern0pt}P{\isacharcomma}{\kern0pt}leq{\isacharcomma}{\kern0pt}one{\isacharbrackright}{\kern0pt}\ {\isacharat}{\kern0pt}\ nenv{\isacharparenright}{\kern0pt}\ {\isasymTurnstile}\ {\isacharquery}{\kern0pt}f{\isacharunderscore}{\kern0pt}fm{\isacharparenright}{\kern0pt}{\isachardoublequoteclose}\isanewline
\ \ \ \ \isacommand{unfolding}\isamarkupfalse%
\ univalent{\isacharunderscore}{\kern0pt}def\ \isacommand{by}\isamarkupfalse%
\ {\isacharparenleft}{\kern0pt}auto\ intro{\isacharcolon}{\kern0pt}unique{\isacharunderscore}{\kern0pt}least{\isacharparenright}{\kern0pt}\isanewline
\ \ \isacommand{moreover}\isamarkupfalse%
\ \isacommand{from}\isamarkupfalse%
\ {\isacartoucheopen}length{\isacharparenleft}{\kern0pt}{\isacharunderscore}{\kern0pt}{\isacharparenright}{\kern0pt}\ {\isacharequal}{\kern0pt}\ {\isacharunderscore}{\kern0pt}{\isacartoucheclose}\ {\isacartoucheopen}env\ {\isasymin}\ {\isacharunderscore}{\kern0pt}{\isacartoucheclose}\isanewline
\ \ \isacommand{have}\isamarkupfalse%
\ {\isachardoublequoteopen}length{\isacharparenleft}{\kern0pt}{\isacharbrackleft}{\kern0pt}P{\isacharcomma}{\kern0pt}leq{\isacharcomma}{\kern0pt}one{\isacharbrackright}{\kern0pt}\ {\isacharat}{\kern0pt}\ nenv{\isacharparenright}{\kern0pt}\ {\isacharequal}{\kern0pt}\ {\isadigit{3}}\ {\isacharhash}{\kern0pt}{\isacharplus}{\kern0pt}\ length{\isacharparenleft}{\kern0pt}env{\isacharparenright}{\kern0pt}{\isachardoublequoteclose}\ \isacommand{by}\isamarkupfalse%
\ simp\isanewline
\ \ \isacommand{moreover}\isamarkupfalse%
\ \isacommand{from}\isamarkupfalse%
\ {\isacartoucheopen}arity{\isacharparenleft}{\kern0pt}{\isacharunderscore}{\kern0pt}{\isacharparenright}{\kern0pt}\ {\isasymle}\ {\isadigit{2}}\ {\isacharhash}{\kern0pt}{\isacharplus}{\kern0pt}\ length{\isacharparenleft}{\kern0pt}nenv{\isacharparenright}{\kern0pt}{\isacartoucheclose}\isanewline
\ \ \ \ {\isacartoucheopen}length{\isacharparenleft}{\kern0pt}{\isacharunderscore}{\kern0pt}{\isacharparenright}{\kern0pt}\ {\isacharequal}{\kern0pt}\ length{\isacharparenleft}{\kern0pt}{\isacharunderscore}{\kern0pt}{\isacharparenright}{\kern0pt}{\isacartoucheclose}{\isacharbrackleft}{\kern0pt}symmetric{\isacharbrackright}{\kern0pt}\ {\isacartoucheopen}nenv{\isasymin}{\isacharunderscore}{\kern0pt}{\isacartoucheclose}\ {\isacartoucheopen}{\isasymphi}{\isasymin}{\isacharunderscore}{\kern0pt}{\isacartoucheclose}\isanewline
\ \ \isacommand{have}\isamarkupfalse%
\ {\isachardoublequoteopen}arity{\isacharparenleft}{\kern0pt}{\isacharquery}{\kern0pt}f{\isacharunderscore}{\kern0pt}fm{\isacharparenright}{\kern0pt}\ {\isasymle}\ {\isadigit{5}}\ {\isacharhash}{\kern0pt}{\isacharplus}{\kern0pt}\ length{\isacharparenleft}{\kern0pt}env{\isacharparenright}{\kern0pt}{\isachardoublequoteclose}\isanewline
\ \ \ \ \isacommand{unfolding}\isamarkupfalse%
\ body{\isacharunderscore}{\kern0pt}fm{\isacharunderscore}{\kern0pt}def\ \ new{\isacharunderscore}{\kern0pt}fm{\isacharunderscore}{\kern0pt}defs\ least{\isacharunderscore}{\kern0pt}fm{\isacharunderscore}{\kern0pt}def\isanewline
\ \ \ \ \isacommand{using}\isamarkupfalse%
\ arity{\isacharunderscore}{\kern0pt}forces\ arity{\isacharunderscore}{\kern0pt}renrep\ arity{\isacharunderscore}{\kern0pt}renbody\ arity{\isacharunderscore}{\kern0pt}body{\isacharunderscore}{\kern0pt}fm{\isacharprime}{\kern0pt}\ nonempty\isanewline
\ \ \ \ \isacommand{by}\isamarkupfalse%
\ {\isacharparenleft}{\kern0pt}simp\ add{\isacharcolon}{\kern0pt}\ pred{\isacharunderscore}{\kern0pt}Un\ Un{\isacharunderscore}{\kern0pt}assoc{\isacharcomma}{\kern0pt}\ simp\ add{\isacharcolon}{\kern0pt}\ Un{\isacharunderscore}{\kern0pt}assoc{\isacharbrackleft}{\kern0pt}symmetric{\isacharbrackright}{\kern0pt}\ nat{\isacharunderscore}{\kern0pt}union{\isacharunderscore}{\kern0pt}abs{\isadigit{1}}\ pred{\isacharunderscore}{\kern0pt}Un{\isacharparenright}{\kern0pt}\isanewline
\ \ \ \ \ \ {\isacharparenleft}{\kern0pt}auto\ simp\ add{\isacharcolon}{\kern0pt}\ nat{\isacharunderscore}{\kern0pt}simp{\isacharunderscore}{\kern0pt}union{\isacharcomma}{\kern0pt}\ rule\ pred{\isacharunderscore}{\kern0pt}le{\isacharcomma}{\kern0pt}\ auto\ intro{\isacharcolon}{\kern0pt}leI{\isacharparenright}{\kern0pt}\isanewline
\ \ \isacommand{moreover}\isamarkupfalse%
\ \isacommand{from}\isamarkupfalse%
\ {\isacartoucheopen}{\isasymphi}{\isasymin}formula{\isacartoucheclose}\ {\isacartoucheopen}nenv{\isasymin}list{\isacharparenleft}{\kern0pt}M{\isacharparenright}{\kern0pt}{\isacartoucheclose}\isanewline
\ \ \isacommand{have}\isamarkupfalse%
\ {\isachardoublequoteopen}{\isacharquery}{\kern0pt}f{\isacharunderscore}{\kern0pt}fm{\isasymin}formula{\isachardoublequoteclose}\ \isacommand{by}\isamarkupfalse%
\ simp\isanewline
\ \ \isacommand{moreover}\isamarkupfalse%
\isanewline
\ \ \isacommand{note}\isamarkupfalse%
\ inM\ {\isacharequal}{\kern0pt}\ P{\isacharunderscore}{\kern0pt}in{\isacharunderscore}{\kern0pt}M\ leq{\isacharunderscore}{\kern0pt}in{\isacharunderscore}{\kern0pt}M\ one{\isacharunderscore}{\kern0pt}in{\isacharunderscore}{\kern0pt}M\ {\isacartoucheopen}nenv{\isasymin}list{\isacharparenleft}{\kern0pt}M{\isacharparenright}{\kern0pt}{\isacartoucheclose}\ {\isacartoucheopen}{\isacharquery}{\kern0pt}{\isasympi}{\isasymin}M{\isacartoucheclose}\isanewline
\ \ \isacommand{ultimately}\isamarkupfalse%
\isanewline
\ \ \isacommand{obtain}\isamarkupfalse%
\ Y\ \isakeyword{where}\ {\isachardoublequoteopen}Y{\isasymin}M{\isachardoublequoteclose}\isanewline
\ \ \ \ {\isachardoublequoteopen}{\isasymforall}m{\isasymin}M{\isachardot}{\kern0pt}\ m\ {\isasymin}\ Y\ {\isasymlongleftrightarrow}\ {\isacharparenleft}{\kern0pt}{\isasymexists}{\isasymrho}p{\isasymin}M{\isachardot}{\kern0pt}\ {\isasymrho}p\ {\isasymin}\ {\isacharquery}{\kern0pt}{\isasympi}\ {\isasymand}\ M{\isacharcomma}{\kern0pt}\ {\isacharbrackleft}{\kern0pt}{\isasymrho}p{\isacharcomma}{\kern0pt}m{\isacharbrackright}{\kern0pt}\ {\isacharat}{\kern0pt}\ {\isacharparenleft}{\kern0pt}{\isacharbrackleft}{\kern0pt}P{\isacharcomma}{\kern0pt}leq{\isacharcomma}{\kern0pt}one{\isacharbrackright}{\kern0pt}\ {\isacharat}{\kern0pt}\ nenv{\isacharparenright}{\kern0pt}\ {\isasymTurnstile}\ {\isacharquery}{\kern0pt}f{\isacharunderscore}{\kern0pt}fm{\isacharparenright}{\kern0pt}{\isachardoublequoteclose}\isanewline
\ \ \ \ \isacommand{using}\isamarkupfalse%
\ replacement{\isacharunderscore}{\kern0pt}ax{\isacharbrackleft}{\kern0pt}of\ {\isacharquery}{\kern0pt}f{\isacharunderscore}{\kern0pt}fm\ {\isachardoublequoteopen}{\isacharbrackleft}{\kern0pt}P{\isacharcomma}{\kern0pt}leq{\isacharcomma}{\kern0pt}one{\isacharbrackright}{\kern0pt}\ {\isacharat}{\kern0pt}\ nenv{\isachardoublequoteclose}{\isacharbrackright}{\kern0pt}\isanewline
\ \ \ \ \isacommand{unfolding}\isamarkupfalse%
\ strong{\isacharunderscore}{\kern0pt}replacement{\isacharunderscore}{\kern0pt}def\ \isacommand{by}\isamarkupfalse%
\ auto\isanewline
\ \ \isacommand{with}\isamarkupfalse%
\ {\isacartoucheopen}least{\isacharparenleft}{\kern0pt}{\isacharunderscore}{\kern0pt}{\isacharcomma}{\kern0pt}QQ{\isacharparenleft}{\kern0pt}{\isacharunderscore}{\kern0pt}{\isacharparenright}{\kern0pt}{\isacharcomma}{\kern0pt}f{\isacharparenleft}{\kern0pt}{\isacharunderscore}{\kern0pt}{\isacharparenright}{\kern0pt}{\isacharparenright}{\kern0pt}{\isacartoucheclose}\ {\isacartoucheopen}f{\isacharparenleft}{\kern0pt}{\isacharunderscore}{\kern0pt}{\isacharparenright}{\kern0pt}\ {\isasymin}\ M{\isacartoucheclose}\ {\isacartoucheopen}{\isacharquery}{\kern0pt}{\isasympi}{\isasymin}M{\isacartoucheclose}\isanewline
\ \ \ \ {\isacartoucheopen}{\isacharunderscore}{\kern0pt}\ {\isasymLongrightarrow}\ {\isacharunderscore}{\kern0pt}\ {\isasymLongrightarrow}\ {\isacharunderscore}{\kern0pt}\ {\isasymLongrightarrow}\ M{\isacharcomma}{\kern0pt}{\isacharunderscore}{\kern0pt}\ {\isasymTurnstile}\ {\isacharquery}{\kern0pt}f{\isacharunderscore}{\kern0pt}fm\ {\isasymlongleftrightarrow}\ least{\isacharparenleft}{\kern0pt}{\isacharunderscore}{\kern0pt}{\isacharcomma}{\kern0pt}{\isacharunderscore}{\kern0pt}{\isacharcomma}{\kern0pt}{\isacharunderscore}{\kern0pt}{\isacharparenright}{\kern0pt}{\isacartoucheclose}\isanewline
\ \ \isacommand{have}\isamarkupfalse%
\ {\isachardoublequoteopen}f{\isacharparenleft}{\kern0pt}{\isasymrho}p{\isacharparenright}{\kern0pt}{\isasymin}Y{\isachardoublequoteclose}\ \isakeyword{if}\ {\isachardoublequoteopen}{\isasymrho}p{\isasymin}{\isacharquery}{\kern0pt}{\isasympi}{\isachardoublequoteclose}\ \isakeyword{for}\ {\isasymrho}p\isanewline
\ \ \ \ \isacommand{using}\isamarkupfalse%
\ that\ transitivity{\isacharbrackleft}{\kern0pt}OF\ {\isacharunderscore}{\kern0pt}\ {\isacartoucheopen}{\isacharquery}{\kern0pt}{\isasympi}{\isasymin}M{\isacartoucheclose}{\isacharbrackright}{\kern0pt}\isanewline
\ \ \ \ \isacommand{by}\isamarkupfalse%
\ {\isacharparenleft}{\kern0pt}clarsimp{\isacharcomma}{\kern0pt}\ rule{\isacharunderscore}{\kern0pt}tac\ x{\isacharequal}{\kern0pt}{\isachardoublequoteopen}{\isasymlangle}x{\isacharcomma}{\kern0pt}y{\isasymrangle}{\isachardoublequoteclose}\ \isakeyword{in}\ bexI{\isacharcomma}{\kern0pt}\ auto{\isacharparenright}{\kern0pt}\isanewline
\ \ \isacommand{moreover}\isamarkupfalse%
\isanewline
\ \ \isacommand{have}\isamarkupfalse%
\ {\isachardoublequoteopen}{\isacharbraceleft}{\kern0pt}y{\isasymin}Y{\isachardot}{\kern0pt}\ Ord{\isacharparenleft}{\kern0pt}y{\isacharparenright}{\kern0pt}{\isacharbraceright}{\kern0pt}\ {\isasymin}\ M{\isachardoublequoteclose}\isanewline
\ \ \ \ \isacommand{using}\isamarkupfalse%
\ {\isacartoucheopen}Y{\isasymin}M{\isacartoucheclose}\ separation{\isacharunderscore}{\kern0pt}ax\ sats{\isacharunderscore}{\kern0pt}ordinal{\isacharunderscore}{\kern0pt}fm\ trans{\isacharunderscore}{\kern0pt}M\isanewline
\ \ \ \ \ \ separation{\isacharunderscore}{\kern0pt}cong{\isacharbrackleft}{\kern0pt}of\ {\isachardoublequoteopen}{\isacharhash}{\kern0pt}{\isacharhash}{\kern0pt}M{\isachardoublequoteclose}\ {\isachardoublequoteopen}{\isasymlambda}y{\isachardot}{\kern0pt}\ sats{\isacharparenleft}{\kern0pt}M{\isacharcomma}{\kern0pt}ordinal{\isacharunderscore}{\kern0pt}fm{\isacharparenleft}{\kern0pt}{\isadigit{0}}{\isacharparenright}{\kern0pt}{\isacharcomma}{\kern0pt}{\isacharbrackleft}{\kern0pt}y{\isacharbrackright}{\kern0pt}{\isacharparenright}{\kern0pt}{\isachardoublequoteclose}\ {\isachardoublequoteopen}Ord{\isachardoublequoteclose}{\isacharbrackright}{\kern0pt}\isanewline
\ \ \ \ \ \ separation{\isacharunderscore}{\kern0pt}closed\ \isacommand{by}\isamarkupfalse%
\ simp\isanewline
\ \ \isacommand{then}\isamarkupfalse%
\isanewline
\ \ \isacommand{have}\isamarkupfalse%
\ {\isachardoublequoteopen}{\isasymUnion}\ {\isacharbraceleft}{\kern0pt}y{\isasymin}Y{\isachardot}{\kern0pt}\ Ord{\isacharparenleft}{\kern0pt}y{\isacharparenright}{\kern0pt}{\isacharbraceright}{\kern0pt}\ {\isasymin}\ M{\isachardoublequoteclose}\ {\isacharparenleft}{\kern0pt}\isakeyword{is}\ {\isachardoublequoteopen}{\isacharquery}{\kern0pt}sup\ {\isasymin}\ M{\isachardoublequoteclose}{\isacharparenright}{\kern0pt}\isanewline
\ \ \ \ \isacommand{using}\isamarkupfalse%
\ Union{\isacharunderscore}{\kern0pt}closed\ \isacommand{by}\isamarkupfalse%
\ simp\isanewline
\ \ \isacommand{then}\isamarkupfalse%
\isanewline
\ \ \isacommand{have}\isamarkupfalse%
\ {\isachardoublequoteopen}{\isacharbraceleft}{\kern0pt}x{\isasymin}Vset{\isacharparenleft}{\kern0pt}{\isacharquery}{\kern0pt}sup{\isacharparenright}{\kern0pt}{\isachardot}{\kern0pt}\ x\ {\isasymin}\ M{\isacharbraceright}{\kern0pt}\ {\isasymin}\ M{\isachardoublequoteclose}\isanewline
\ \ \ \ \isacommand{using}\isamarkupfalse%
\ Vset{\isacharunderscore}{\kern0pt}closed\ \isacommand{by}\isamarkupfalse%
\ simp\isanewline
\ \ \isacommand{moreover}\isamarkupfalse%
\isanewline
\ \ \isacommand{have}\isamarkupfalse%
\ {\isachardoublequoteopen}{\isacharbraceleft}{\kern0pt}one{\isacharbraceright}{\kern0pt}\ {\isasymin}\ M{\isachardoublequoteclose}\isanewline
\ \ \ \ \isacommand{using}\isamarkupfalse%
\ one{\isacharunderscore}{\kern0pt}in{\isacharunderscore}{\kern0pt}M\ singletonM\ \isacommand{by}\isamarkupfalse%
\ simp\isanewline
\ \ \isacommand{ultimately}\isamarkupfalse%
\isanewline
\ \ \isacommand{have}\isamarkupfalse%
\ {\isachardoublequoteopen}{\isacharbraceleft}{\kern0pt}x{\isasymin}Vset{\isacharparenleft}{\kern0pt}{\isacharquery}{\kern0pt}sup{\isacharparenright}{\kern0pt}{\isachardot}{\kern0pt}\ x\ {\isasymin}\ M{\isacharbraceright}{\kern0pt}\ {\isasymtimes}\ {\isacharbraceleft}{\kern0pt}one{\isacharbraceright}{\kern0pt}\ {\isasymin}\ M{\isachardoublequoteclose}\ {\isacharparenleft}{\kern0pt}\isakeyword{is}\ {\isachardoublequoteopen}{\isacharquery}{\kern0pt}big{\isacharunderscore}{\kern0pt}name\ {\isasymin}\ M{\isachardoublequoteclose}{\isacharparenright}{\kern0pt}\isanewline
\ \ \ \ \isacommand{using}\isamarkupfalse%
\ cartprod{\isacharunderscore}{\kern0pt}closed\ \isacommand{by}\isamarkupfalse%
\ simp\isanewline
\ \ \isacommand{then}\isamarkupfalse%
\isanewline
\ \ \isacommand{have}\isamarkupfalse%
\ {\isachardoublequoteopen}val{\isacharparenleft}{\kern0pt}G{\isacharcomma}{\kern0pt}{\isacharquery}{\kern0pt}big{\isacharunderscore}{\kern0pt}name{\isacharparenright}{\kern0pt}\ {\isasymin}\ M{\isacharbrackleft}{\kern0pt}G{\isacharbrackright}{\kern0pt}{\isachardoublequoteclose}\isanewline
\ \ \ \ \isacommand{by}\isamarkupfalse%
\ {\isacharparenleft}{\kern0pt}blast\ intro{\isacharcolon}{\kern0pt}GenExtI{\isacharparenright}{\kern0pt}\isanewline
\ \ \isacommand{{\isacharbraceleft}{\kern0pt}}\isamarkupfalse%
\isanewline
\ \ \ \ \isacommand{fix}\isamarkupfalse%
\ v\ x\isanewline
\ \ \ \ \isacommand{assume}\isamarkupfalse%
\ {\isachardoublequoteopen}x{\isasymin}c{\isachardoublequoteclose}\isanewline
\ \ \ \ \isacommand{moreover}\isamarkupfalse%
\isanewline
\ \ \ \ \isacommand{note}\isamarkupfalse%
\ {\isacartoucheopen}val{\isacharparenleft}{\kern0pt}G{\isacharcomma}{\kern0pt}{\isasympi}{\isacharprime}{\kern0pt}{\isacharparenright}{\kern0pt}{\isacharequal}{\kern0pt}c{\isacartoucheclose}\ {\isacartoucheopen}{\isasympi}{\isacharprime}{\kern0pt}{\isasymin}M{\isacartoucheclose}\isanewline
\ \ \ \ \isacommand{moreover}\isamarkupfalse%
\isanewline
\ \ \ \ \isacommand{from}\isamarkupfalse%
\ calculation\isanewline
\ \ \ \ \isacommand{obtain}\isamarkupfalse%
\ {\isasymrho}\ p\ \isakeyword{where}\ {\isachardoublequoteopen}{\isasymlangle}{\isasymrho}{\isacharcomma}{\kern0pt}p{\isasymrangle}{\isasymin}{\isasympi}{\isacharprime}{\kern0pt}{\isachardoublequoteclose}\ \ {\isachardoublequoteopen}val{\isacharparenleft}{\kern0pt}G{\isacharcomma}{\kern0pt}{\isasymrho}{\isacharparenright}{\kern0pt}\ {\isacharequal}{\kern0pt}\ x{\isachardoublequoteclose}\ {\isachardoublequoteopen}p{\isasymin}G{\isachardoublequoteclose}\ {\isachardoublequoteopen}{\isasymrho}{\isasymin}M{\isachardoublequoteclose}\isanewline
\ \ \ \ \ \ \isacommand{using}\isamarkupfalse%
\ elem{\isacharunderscore}{\kern0pt}of{\isacharunderscore}{\kern0pt}val{\isacharunderscore}{\kern0pt}pair{\isacharprime}{\kern0pt}{\isacharbrackleft}{\kern0pt}of\ {\isasympi}{\isacharprime}{\kern0pt}\ x\ G{\isacharbrackright}{\kern0pt}\ \isacommand{by}\isamarkupfalse%
\ blast\isanewline
\ \ \ \ \isacommand{moreover}\isamarkupfalse%
\isanewline
\ \ \ \ \isacommand{assume}\isamarkupfalse%
\ {\isachardoublequoteopen}v{\isasymin}M{\isacharbrackleft}{\kern0pt}G{\isacharbrackright}{\kern0pt}{\isachardoublequoteclose}\isanewline
\ \ \ \ \isacommand{then}\isamarkupfalse%
\isanewline
\ \ \ \ \isacommand{obtain}\isamarkupfalse%
\ {\isasymsigma}\ \isakeyword{where}\ {\isachardoublequoteopen}val{\isacharparenleft}{\kern0pt}G{\isacharcomma}{\kern0pt}{\isasymsigma}{\isacharparenright}{\kern0pt}\ {\isacharequal}{\kern0pt}\ v{\isachardoublequoteclose}\ {\isachardoublequoteopen}{\isasymsigma}{\isasymin}M{\isachardoublequoteclose}\isanewline
\ \ \ \ \ \ \isacommand{using}\isamarkupfalse%
\ GenExtD\ \isacommand{by}\isamarkupfalse%
\ auto\isanewline
\ \ \ \ \isacommand{moreover}\isamarkupfalse%
\isanewline
\ \ \ \ \isacommand{assume}\isamarkupfalse%
\ {\isachardoublequoteopen}sats{\isacharparenleft}{\kern0pt}M{\isacharbrackleft}{\kern0pt}G{\isacharbrackright}{\kern0pt}{\isacharcomma}{\kern0pt}\ {\isasymphi}{\isacharcomma}{\kern0pt}\ {\isacharbrackleft}{\kern0pt}x{\isacharcomma}{\kern0pt}v{\isacharbrackright}{\kern0pt}\ {\isacharat}{\kern0pt}\ env{\isacharparenright}{\kern0pt}{\isachardoublequoteclose}\isanewline
\ \ \ \ \isacommand{moreover}\isamarkupfalse%
\isanewline
\ \ \ \ \isacommand{note}\isamarkupfalse%
\ {\isacartoucheopen}{\isasymphi}{\isasymin}{\isacharunderscore}{\kern0pt}{\isacartoucheclose}\ {\isacartoucheopen}nenv{\isasymin}{\isacharunderscore}{\kern0pt}{\isacartoucheclose}\ {\isacartoucheopen}env\ {\isacharequal}{\kern0pt}\ {\isacharunderscore}{\kern0pt}{\isacartoucheclose}\ {\isacartoucheopen}arity{\isacharparenleft}{\kern0pt}{\isasymphi}{\isacharparenright}{\kern0pt}{\isasymle}\ {\isadigit{2}}\ {\isacharhash}{\kern0pt}{\isacharplus}{\kern0pt}\ length{\isacharparenleft}{\kern0pt}env{\isacharparenright}{\kern0pt}{\isacartoucheclose}\isanewline
\ \ \ \ \isacommand{ultimately}\isamarkupfalse%
\isanewline
\ \ \ \ \isacommand{obtain}\isamarkupfalse%
\ q\ \isakeyword{where}\ {\isachardoublequoteopen}q{\isasymin}G{\isachardoublequoteclose}\ {\isachardoublequoteopen}q\ {\isasymtturnstile}\ {\isasymphi}\ {\isacharparenleft}{\kern0pt}{\isacharbrackleft}{\kern0pt}{\isasymrho}{\isacharcomma}{\kern0pt}{\isasymsigma}{\isacharbrackright}{\kern0pt}{\isacharat}{\kern0pt}nenv{\isacharparenright}{\kern0pt}{\isachardoublequoteclose}\isanewline
\ \ \ \ \ \ \isacommand{using}\isamarkupfalse%
\ truth{\isacharunderscore}{\kern0pt}lemma{\isacharbrackleft}{\kern0pt}OF\ {\isacartoucheopen}{\isasymphi}{\isasymin}{\isacharunderscore}{\kern0pt}{\isacartoucheclose}\ generic{\isacharcomma}{\kern0pt}\ symmetric{\isacharcomma}{\kern0pt}\ of\ {\isachardoublequoteopen}{\isacharbrackleft}{\kern0pt}{\isasymrho}{\isacharcomma}{\kern0pt}{\isasymsigma}{\isacharbrackright}{\kern0pt}\ {\isacharat}{\kern0pt}\ nenv{\isachardoublequoteclose}{\isacharbrackright}{\kern0pt}\isanewline
\ \ \ \ \ \ \isacommand{by}\isamarkupfalse%
\ auto\isanewline
\ \ \ \ \isacommand{with}\isamarkupfalse%
\ {\isacartoucheopen}{\isasymlangle}{\isasymrho}{\isacharcomma}{\kern0pt}p{\isasymrangle}{\isasymin}{\isasympi}{\isacharprime}{\kern0pt}{\isacartoucheclose}\ {\isacartoucheopen}{\isasymlangle}{\isasymrho}{\isacharcomma}{\kern0pt}q{\isasymrangle}{\isasymin}{\isacharquery}{\kern0pt}{\isasympi}\ {\isasymLongrightarrow}\ f{\isacharparenleft}{\kern0pt}{\isasymlangle}{\isasymrho}{\isacharcomma}{\kern0pt}q{\isasymrangle}{\isacharparenright}{\kern0pt}{\isasymin}Y{\isacartoucheclose}\isanewline
\ \ \ \ \isacommand{have}\isamarkupfalse%
\ {\isachardoublequoteopen}f{\isacharparenleft}{\kern0pt}{\isasymlangle}{\isasymrho}{\isacharcomma}{\kern0pt}q{\isasymrangle}{\isacharparenright}{\kern0pt}{\isasymin}Y{\isachardoublequoteclose}\isanewline
\ \ \ \ \ \ \isacommand{using}\isamarkupfalse%
\ generic\ \isacommand{unfolding}\isamarkupfalse%
\ M{\isacharunderscore}{\kern0pt}generic{\isacharunderscore}{\kern0pt}def\ filter{\isacharunderscore}{\kern0pt}def\ \isacommand{by}\isamarkupfalse%
\ blast\isanewline
\ \ \ \ \isacommand{let}\isamarkupfalse%
\ {\isacharquery}{\kern0pt}{\isasymalpha}{\isacharequal}{\kern0pt}{\isachardoublequoteopen}succ{\isacharparenleft}{\kern0pt}rank{\isacharparenleft}{\kern0pt}{\isasymsigma}{\isacharparenright}{\kern0pt}{\isacharparenright}{\kern0pt}{\isachardoublequoteclose}\isanewline
\ \ \ \ \isacommand{note}\isamarkupfalse%
\ {\isacartoucheopen}{\isasymsigma}{\isasymin}M{\isacartoucheclose}\isanewline
\ \ \ \ \isacommand{moreover}\isamarkupfalse%
\ \isacommand{from}\isamarkupfalse%
\ this\isanewline
\ \ \ \ \isacommand{have}\isamarkupfalse%
\ {\isachardoublequoteopen}{\isacharquery}{\kern0pt}{\isasymalpha}\ {\isasymin}\ M{\isachardoublequoteclose}\isanewline
\ \ \ \ \ \ \isacommand{using}\isamarkupfalse%
\ rank{\isacharunderscore}{\kern0pt}closed\ cons{\isacharunderscore}{\kern0pt}closed\ \isacommand{by}\isamarkupfalse%
\ {\isacharparenleft}{\kern0pt}simp\ flip{\isacharcolon}{\kern0pt}\ setclass{\isacharunderscore}{\kern0pt}iff{\isacharparenright}{\kern0pt}\isanewline
\ \ \ \ \isacommand{moreover}\isamarkupfalse%
\isanewline
\ \ \ \ \isacommand{have}\isamarkupfalse%
\ {\isachardoublequoteopen}{\isasymsigma}\ {\isasymin}\ Vset{\isacharparenleft}{\kern0pt}{\isacharquery}{\kern0pt}{\isasymalpha}{\isacharparenright}{\kern0pt}{\isachardoublequoteclose}\isanewline
\ \ \ \ \ \ \isacommand{using}\isamarkupfalse%
\ Vset{\isacharunderscore}{\kern0pt}Ord{\isacharunderscore}{\kern0pt}rank{\isacharunderscore}{\kern0pt}iff\ \isacommand{by}\isamarkupfalse%
\ auto\isanewline
\ \ \ \ \isacommand{moreover}\isamarkupfalse%
\isanewline
\ \ \ \ \isacommand{note}\isamarkupfalse%
\ {\isacartoucheopen}q\ {\isasymtturnstile}\ {\isasymphi}\ {\isacharparenleft}{\kern0pt}{\isacharbrackleft}{\kern0pt}{\isasymrho}{\isacharcomma}{\kern0pt}{\isasymsigma}{\isacharbrackright}{\kern0pt}\ {\isacharat}{\kern0pt}\ nenv{\isacharparenright}{\kern0pt}{\isacartoucheclose}\isanewline
\ \ \ \ \isacommand{ultimately}\isamarkupfalse%
\isanewline
\ \ \ \ \isacommand{have}\isamarkupfalse%
\ {\isachardoublequoteopen}{\isacharquery}{\kern0pt}P{\isacharparenleft}{\kern0pt}{\isasymlangle}{\isasymrho}{\isacharcomma}{\kern0pt}q{\isasymrangle}{\isacharcomma}{\kern0pt}{\isacharquery}{\kern0pt}{\isasymalpha}{\isacharparenright}{\kern0pt}{\isachardoublequoteclose}\ \isacommand{by}\isamarkupfalse%
\ {\isacharparenleft}{\kern0pt}auto\ simp\ del{\isacharcolon}{\kern0pt}\ Vset{\isacharunderscore}{\kern0pt}rank{\isacharunderscore}{\kern0pt}iff{\isacharparenright}{\kern0pt}\isanewline
\ \ \ \ \isacommand{moreover}\isamarkupfalse%
\isanewline
\ \ \ \ \isacommand{have}\isamarkupfalse%
\ {\isachardoublequoteopen}{\isacharparenleft}{\kern0pt}{\isasymmu}\ {\isasymalpha}{\isachardot}{\kern0pt}\ {\isacharquery}{\kern0pt}P{\isacharparenleft}{\kern0pt}{\isasymlangle}{\isasymrho}{\isacharcomma}{\kern0pt}q{\isasymrangle}{\isacharcomma}{\kern0pt}{\isasymalpha}{\isacharparenright}{\kern0pt}{\isacharparenright}{\kern0pt}\ {\isacharequal}{\kern0pt}\ f{\isacharparenleft}{\kern0pt}{\isasymlangle}{\isasymrho}{\isacharcomma}{\kern0pt}q{\isasymrangle}{\isacharparenright}{\kern0pt}{\isachardoublequoteclose}\isanewline
\ \ \ \ \ \ \isacommand{unfolding}\isamarkupfalse%
\ f{\isacharunderscore}{\kern0pt}def\ \isacommand{by}\isamarkupfalse%
\ simp\isanewline
\ \ \ \ \isacommand{ultimately}\isamarkupfalse%
\isanewline
\ \ \ \ \isacommand{obtain}\isamarkupfalse%
\ {\isasymtau}\ \isakeyword{where}\ {\isachardoublequoteopen}{\isasymtau}{\isasymin}M{\isachardoublequoteclose}\ {\isachardoublequoteopen}{\isasymtau}\ {\isasymin}\ Vset{\isacharparenleft}{\kern0pt}f{\isacharparenleft}{\kern0pt}{\isasymlangle}{\isasymrho}{\isacharcomma}{\kern0pt}q{\isasymrangle}{\isacharparenright}{\kern0pt}{\isacharparenright}{\kern0pt}{\isachardoublequoteclose}\ {\isachardoublequoteopen}q\ {\isasymtturnstile}\ {\isasymphi}\ {\isacharparenleft}{\kern0pt}{\isacharbrackleft}{\kern0pt}{\isasymrho}{\isacharcomma}{\kern0pt}{\isasymtau}{\isacharbrackright}{\kern0pt}\ {\isacharat}{\kern0pt}\ nenv{\isacharparenright}{\kern0pt}{\isachardoublequoteclose}\isanewline
\ \ \ \ \ \ \isacommand{using}\isamarkupfalse%
\ LeastI{\isacharbrackleft}{\kern0pt}of\ {\isachardoublequoteopen}{\isasymlambda}\ {\isasymalpha}{\isachardot}{\kern0pt}\ {\isacharquery}{\kern0pt}P{\isacharparenleft}{\kern0pt}{\isasymlangle}{\isasymrho}{\isacharcomma}{\kern0pt}q{\isasymrangle}{\isacharcomma}{\kern0pt}{\isasymalpha}{\isacharparenright}{\kern0pt}{\isachardoublequoteclose}\ {\isacharquery}{\kern0pt}{\isasymalpha}{\isacharbrackright}{\kern0pt}\ \isacommand{by}\isamarkupfalse%
\ auto\isanewline
\ \ \ \ \isacommand{with}\isamarkupfalse%
\ {\isacartoucheopen}q{\isasymin}G{\isacartoucheclose}\ {\isacartoucheopen}{\isasymrho}{\isasymin}M{\isacartoucheclose}\ {\isacartoucheopen}nenv{\isasymin}{\isacharunderscore}{\kern0pt}{\isacartoucheclose}\ {\isacartoucheopen}arity{\isacharparenleft}{\kern0pt}{\isasymphi}{\isacharparenright}{\kern0pt}{\isasymle}\ {\isadigit{2}}\ {\isacharhash}{\kern0pt}{\isacharplus}{\kern0pt}\ length{\isacharparenleft}{\kern0pt}nenv{\isacharparenright}{\kern0pt}{\isacartoucheclose}\isanewline
\ \ \ \ \isacommand{have}\isamarkupfalse%
\ {\isachardoublequoteopen}M{\isacharbrackleft}{\kern0pt}G{\isacharbrackright}{\kern0pt}{\isacharcomma}{\kern0pt}\ map{\isacharparenleft}{\kern0pt}val{\isacharparenleft}{\kern0pt}G{\isacharparenright}{\kern0pt}{\isacharcomma}{\kern0pt}{\isacharbrackleft}{\kern0pt}{\isasymrho}{\isacharcomma}{\kern0pt}{\isasymtau}{\isacharbrackright}{\kern0pt}\ {\isacharat}{\kern0pt}\ nenv{\isacharparenright}{\kern0pt}\ {\isasymTurnstile}\ {\isasymphi}{\isachardoublequoteclose}\isanewline
\ \ \ \ \ \ \isacommand{using}\isamarkupfalse%
\ truth{\isacharunderscore}{\kern0pt}lemma{\isacharbrackleft}{\kern0pt}OF\ {\isacartoucheopen}{\isasymphi}{\isasymin}{\isacharunderscore}{\kern0pt}{\isacartoucheclose}\ generic{\isacharcomma}{\kern0pt}\ of\ {\isachardoublequoteopen}{\isacharbrackleft}{\kern0pt}{\isasymrho}{\isacharcomma}{\kern0pt}{\isasymtau}{\isacharbrackright}{\kern0pt}\ {\isacharat}{\kern0pt}\ nenv{\isachardoublequoteclose}{\isacharbrackright}{\kern0pt}\ \isacommand{by}\isamarkupfalse%
\ auto\isanewline
\ \ \ \ \isacommand{moreover}\isamarkupfalse%
\ \isacommand{from}\isamarkupfalse%
\ {\isacartoucheopen}x{\isasymin}c{\isacartoucheclose}\ {\isacartoucheopen}c{\isasymin}M{\isacharbrackleft}{\kern0pt}G{\isacharbrackright}{\kern0pt}{\isacartoucheclose}\isanewline
\ \ \ \ \isacommand{have}\isamarkupfalse%
\ {\isachardoublequoteopen}x{\isasymin}M{\isacharbrackleft}{\kern0pt}G{\isacharbrackright}{\kern0pt}{\isachardoublequoteclose}\ \isacommand{using}\isamarkupfalse%
\ transitivity{\isacharunderscore}{\kern0pt}MG\ \isacommand{by}\isamarkupfalse%
\ simp\isanewline
\ \ \ \ \isacommand{moreover}\isamarkupfalse%
\isanewline
\ \ \ \ \isacommand{note}\isamarkupfalse%
\ {\isacartoucheopen}M{\isacharbrackleft}{\kern0pt}G{\isacharbrackright}{\kern0pt}{\isacharcomma}{\kern0pt}{\isacharbrackleft}{\kern0pt}x{\isacharcomma}{\kern0pt}v{\isacharbrackright}{\kern0pt}\ {\isacharat}{\kern0pt}\ env{\isasymTurnstile}\ {\isasymphi}{\isacartoucheclose}\ {\isacartoucheopen}env\ {\isacharequal}{\kern0pt}\ map{\isacharparenleft}{\kern0pt}val{\isacharparenleft}{\kern0pt}G{\isacharparenright}{\kern0pt}{\isacharcomma}{\kern0pt}nenv{\isacharparenright}{\kern0pt}{\isacartoucheclose}\ {\isacartoucheopen}{\isasymtau}{\isasymin}M{\isacartoucheclose}\ {\isacartoucheopen}val{\isacharparenleft}{\kern0pt}G{\isacharcomma}{\kern0pt}{\isasymrho}{\isacharparenright}{\kern0pt}{\isacharequal}{\kern0pt}x{\isacartoucheclose}\isanewline
\ \ \ \ \ \ {\isacartoucheopen}univalent{\isacharparenleft}{\kern0pt}{\isacharhash}{\kern0pt}{\isacharhash}{\kern0pt}M{\isacharbrackleft}{\kern0pt}G{\isacharbrackright}{\kern0pt}{\isacharcomma}{\kern0pt}{\isacharunderscore}{\kern0pt}{\isacharcomma}{\kern0pt}{\isacharunderscore}{\kern0pt}{\isacharparenright}{\kern0pt}{\isacartoucheclose}\ {\isacartoucheopen}x{\isasymin}c{\isacartoucheclose}\ {\isacartoucheopen}v{\isasymin}M{\isacharbrackleft}{\kern0pt}G{\isacharbrackright}{\kern0pt}{\isacartoucheclose}\isanewline
\ \ \ \ \isacommand{ultimately}\isamarkupfalse%
\isanewline
\ \ \ \ \isacommand{have}\isamarkupfalse%
\ {\isachardoublequoteopen}v{\isacharequal}{\kern0pt}val{\isacharparenleft}{\kern0pt}G{\isacharcomma}{\kern0pt}{\isasymtau}{\isacharparenright}{\kern0pt}{\isachardoublequoteclose}\isanewline
\ \ \ \ \ \ \isacommand{using}\isamarkupfalse%
\ GenExtI{\isacharbrackleft}{\kern0pt}of\ {\isasymtau}\ G{\isacharbrackright}{\kern0pt}\ \isacommand{unfolding}\isamarkupfalse%
\ univalent{\isacharunderscore}{\kern0pt}def\ \isacommand{by}\isamarkupfalse%
\ {\isacharparenleft}{\kern0pt}auto{\isacharparenright}{\kern0pt}\isanewline
\ \ \ \ \isacommand{from}\isamarkupfalse%
\ {\isacartoucheopen}{\isasymtau}\ {\isasymin}\ Vset{\isacharparenleft}{\kern0pt}f{\isacharparenleft}{\kern0pt}{\isasymlangle}{\isasymrho}{\isacharcomma}{\kern0pt}q{\isasymrangle}{\isacharparenright}{\kern0pt}{\isacharparenright}{\kern0pt}{\isacartoucheclose}\ {\isacartoucheopen}Ord{\isacharparenleft}{\kern0pt}f{\isacharparenleft}{\kern0pt}{\isacharunderscore}{\kern0pt}{\isacharparenright}{\kern0pt}{\isacharparenright}{\kern0pt}{\isacartoucheclose}\ \ {\isacartoucheopen}f{\isacharparenleft}{\kern0pt}{\isasymlangle}{\isasymrho}{\isacharcomma}{\kern0pt}q{\isasymrangle}{\isacharparenright}{\kern0pt}{\isasymin}Y{\isacartoucheclose}\isanewline
\ \ \ \ \isacommand{have}\isamarkupfalse%
\ {\isachardoublequoteopen}{\isasymtau}\ {\isasymin}\ Vset{\isacharparenleft}{\kern0pt}{\isacharquery}{\kern0pt}sup{\isacharparenright}{\kern0pt}{\isachardoublequoteclose}\isanewline
\ \ \ \ \ \ \isacommand{using}\isamarkupfalse%
\ Vset{\isacharunderscore}{\kern0pt}Ord{\isacharunderscore}{\kern0pt}rank{\isacharunderscore}{\kern0pt}iff\ lt{\isacharunderscore}{\kern0pt}Union{\isacharunderscore}{\kern0pt}iff{\isacharbrackleft}{\kern0pt}of\ {\isacharunderscore}{\kern0pt}\ {\isachardoublequoteopen}rank{\isacharparenleft}{\kern0pt}{\isasymtau}{\isacharparenright}{\kern0pt}{\isachardoublequoteclose}{\isacharbrackright}{\kern0pt}\ \isacommand{by}\isamarkupfalse%
\ auto\isanewline
\ \ \ \ \isacommand{with}\isamarkupfalse%
\ {\isacartoucheopen}{\isasymtau}{\isasymin}M{\isacartoucheclose}\isanewline
\ \ \ \ \isacommand{have}\isamarkupfalse%
\ {\isachardoublequoteopen}val{\isacharparenleft}{\kern0pt}G{\isacharcomma}{\kern0pt}{\isasymtau}{\isacharparenright}{\kern0pt}\ {\isasymin}\ val{\isacharparenleft}{\kern0pt}G{\isacharcomma}{\kern0pt}{\isacharquery}{\kern0pt}big{\isacharunderscore}{\kern0pt}name{\isacharparenright}{\kern0pt}{\isachardoublequoteclose}\isanewline
\ \ \ \ \ \ \isacommand{using}\isamarkupfalse%
\ domain{\isacharunderscore}{\kern0pt}of{\isacharunderscore}{\kern0pt}prod{\isacharbrackleft}{\kern0pt}of\ one\ {\isachardoublequoteopen}{\isacharbraceleft}{\kern0pt}one{\isacharbraceright}{\kern0pt}{\isachardoublequoteclose}\ {\isachardoublequoteopen}{\isacharbraceleft}{\kern0pt}x{\isasymin}Vset{\isacharparenleft}{\kern0pt}{\isacharquery}{\kern0pt}sup{\isacharparenright}{\kern0pt}{\isachardot}{\kern0pt}\ x\ {\isasymin}\ M{\isacharbraceright}{\kern0pt}{\isachardoublequoteclose}\ {\isacharbrackright}{\kern0pt}\ def{\isacharunderscore}{\kern0pt}val{\isacharbrackleft}{\kern0pt}of\ G\ {\isacharquery}{\kern0pt}big{\isacharunderscore}{\kern0pt}name{\isacharbrackright}{\kern0pt}\isanewline
\ \ \ \ \ \ \ \ one{\isacharunderscore}{\kern0pt}in{\isacharunderscore}{\kern0pt}G{\isacharbrackleft}{\kern0pt}OF\ generic{\isacharbrackright}{\kern0pt}\ one{\isacharunderscore}{\kern0pt}in{\isacharunderscore}{\kern0pt}P\ \ \isacommand{by}\isamarkupfalse%
\ {\isacharparenleft}{\kern0pt}auto\ simp\ del{\isacharcolon}{\kern0pt}\ Vset{\isacharunderscore}{\kern0pt}rank{\isacharunderscore}{\kern0pt}iff{\isacharparenright}{\kern0pt}\isanewline
\ \ \ \ \isacommand{with}\isamarkupfalse%
\ {\isacartoucheopen}v{\isacharequal}{\kern0pt}val{\isacharparenleft}{\kern0pt}G{\isacharcomma}{\kern0pt}{\isasymtau}{\isacharparenright}{\kern0pt}{\isacartoucheclose}\isanewline
\ \ \ \ \isacommand{have}\isamarkupfalse%
\ {\isachardoublequoteopen}v\ {\isasymin}\ val{\isacharparenleft}{\kern0pt}G{\isacharcomma}{\kern0pt}{\isacharbraceleft}{\kern0pt}x{\isasymin}Vset{\isacharparenleft}{\kern0pt}{\isacharquery}{\kern0pt}sup{\isacharparenright}{\kern0pt}{\isachardot}{\kern0pt}\ x\ {\isasymin}\ M{\isacharbraceright}{\kern0pt}\ {\isasymtimes}\ {\isacharbraceleft}{\kern0pt}one{\isacharbraceright}{\kern0pt}{\isacharparenright}{\kern0pt}{\isachardoublequoteclose}\isanewline
\ \ \ \ \ \ \isacommand{by}\isamarkupfalse%
\ simp\isanewline
\ \ \isacommand{{\isacharbraceright}{\kern0pt}}\isamarkupfalse%
\isanewline
\ \ \isacommand{then}\isamarkupfalse%
\isanewline
\ \ \isacommand{have}\isamarkupfalse%
\ {\isachardoublequoteopen}{\isacharbraceleft}{\kern0pt}v{\isachardot}{\kern0pt}\ x{\isasymin}c{\isacharcomma}{\kern0pt}\ {\isacharquery}{\kern0pt}R{\isacharparenleft}{\kern0pt}x{\isacharcomma}{\kern0pt}v{\isacharparenright}{\kern0pt}{\isacharbraceright}{\kern0pt}\ {\isasymsubseteq}\ val{\isacharparenleft}{\kern0pt}G{\isacharcomma}{\kern0pt}{\isacharquery}{\kern0pt}big{\isacharunderscore}{\kern0pt}name{\isacharparenright}{\kern0pt}{\isachardoublequoteclose}\ {\isacharparenleft}{\kern0pt}\isakeyword{is}\ {\isachardoublequoteopen}{\isacharquery}{\kern0pt}repl{\isasymsubseteq}{\isacharquery}{\kern0pt}big{\isachardoublequoteclose}{\isacharparenright}{\kern0pt}\isanewline
\ \ \ \ \isacommand{by}\isamarkupfalse%
\ blast\isanewline
\ \ \isacommand{with}\isamarkupfalse%
\ {\isacartoucheopen}{\isacharquery}{\kern0pt}big{\isacharunderscore}{\kern0pt}name{\isasymin}M{\isacartoucheclose}\isanewline
\ \ \isacommand{have}\isamarkupfalse%
\ {\isachardoublequoteopen}{\isacharquery}{\kern0pt}repl\ {\isacharequal}{\kern0pt}\ {\isacharbraceleft}{\kern0pt}v{\isasymin}{\isacharquery}{\kern0pt}big{\isachardot}{\kern0pt}\ {\isasymexists}x{\isasymin}c{\isachardot}{\kern0pt}\ sats{\isacharparenleft}{\kern0pt}M{\isacharbrackleft}{\kern0pt}G{\isacharbrackright}{\kern0pt}{\isacharcomma}{\kern0pt}\ {\isasymphi}{\isacharcomma}{\kern0pt}\ {\isacharbrackleft}{\kern0pt}x{\isacharcomma}{\kern0pt}v{\isacharbrackright}{\kern0pt}\ {\isacharat}{\kern0pt}\ env\ {\isacharparenright}{\kern0pt}{\isacharbraceright}{\kern0pt}{\isachardoublequoteclose}\ {\isacharparenleft}{\kern0pt}\isakeyword{is}\ {\isachardoublequoteopen}{\isacharunderscore}{\kern0pt}\ {\isacharequal}{\kern0pt}\ {\isacharquery}{\kern0pt}rhs{\isachardoublequoteclose}{\isacharparenright}{\kern0pt}\isanewline
\ \ \isacommand{proof}\isamarkupfalse%
{\isacharparenleft}{\kern0pt}intro\ equalityI\ subsetI{\isacharparenright}{\kern0pt}\isanewline
\ \ \ \ \isacommand{fix}\isamarkupfalse%
\ v\isanewline
\ \ \ \ \isacommand{assume}\isamarkupfalse%
\ {\isachardoublequoteopen}v{\isasymin}{\isacharquery}{\kern0pt}repl{\isachardoublequoteclose}\isanewline
\ \ \ \ \isacommand{with}\isamarkupfalse%
\ {\isacartoucheopen}{\isacharquery}{\kern0pt}repl{\isasymsubseteq}{\isacharquery}{\kern0pt}big{\isacartoucheclose}\isanewline
\ \ \ \ \isacommand{obtain}\isamarkupfalse%
\ x\ \isakeyword{where}\ {\isachardoublequoteopen}x{\isasymin}c{\isachardoublequoteclose}\ {\isachardoublequoteopen}M{\isacharbrackleft}{\kern0pt}G{\isacharbrackright}{\kern0pt}{\isacharcomma}{\kern0pt}\ {\isacharbrackleft}{\kern0pt}x{\isacharcomma}{\kern0pt}\ v{\isacharbrackright}{\kern0pt}\ {\isacharat}{\kern0pt}\ env\ {\isasymTurnstile}\ {\isasymphi}{\isachardoublequoteclose}\ {\isachardoublequoteopen}v{\isasymin}{\isacharquery}{\kern0pt}big{\isachardoublequoteclose}\isanewline
\ \ \ \ \ \ \isacommand{using}\isamarkupfalse%
\ subsetD\ \isacommand{by}\isamarkupfalse%
\ auto\isanewline
\ \ \ \ \isacommand{with}\isamarkupfalse%
\ {\isacartoucheopen}univalent{\isacharparenleft}{\kern0pt}{\isacharhash}{\kern0pt}{\isacharhash}{\kern0pt}M{\isacharbrackleft}{\kern0pt}G{\isacharbrackright}{\kern0pt}{\isacharcomma}{\kern0pt}{\isacharunderscore}{\kern0pt}{\isacharcomma}{\kern0pt}{\isacharunderscore}{\kern0pt}{\isacharparenright}{\kern0pt}{\isacartoucheclose}\ {\isacartoucheopen}c{\isasymin}M{\isacharbrackleft}{\kern0pt}G{\isacharbrackright}{\kern0pt}{\isacartoucheclose}\isanewline
\ \ \ \ \isacommand{show}\isamarkupfalse%
\ {\isachardoublequoteopen}v\ {\isasymin}\ {\isacharquery}{\kern0pt}rhs{\isachardoublequoteclose}\isanewline
\ \ \ \ \ \ \isacommand{unfolding}\isamarkupfalse%
\ univalent{\isacharunderscore}{\kern0pt}def\isanewline
\ \ \ \ \ \ \isacommand{using}\isamarkupfalse%
\ transitivity{\isacharunderscore}{\kern0pt}MG\ ReplaceI{\isacharbrackleft}{\kern0pt}of\ {\isachardoublequoteopen}{\isasymlambda}\ x\ v{\isachardot}{\kern0pt}\ {\isasymexists}x{\isasymin}c{\isachardot}{\kern0pt}\ M{\isacharbrackleft}{\kern0pt}G{\isacharbrackright}{\kern0pt}{\isacharcomma}{\kern0pt}\ {\isacharbrackleft}{\kern0pt}x{\isacharcomma}{\kern0pt}\ v{\isacharbrackright}{\kern0pt}\ {\isacharat}{\kern0pt}\ env\ {\isasymTurnstile}\ {\isasymphi}{\isachardoublequoteclose}{\isacharbrackright}{\kern0pt}\ \isacommand{by}\isamarkupfalse%
\ blast\isanewline
\ \ \isacommand{next}\isamarkupfalse%
\isanewline
\ \ \ \ \isacommand{fix}\isamarkupfalse%
\ v\isanewline
\ \ \ \ \isacommand{assume}\isamarkupfalse%
\ {\isachardoublequoteopen}v{\isasymin}{\isacharquery}{\kern0pt}rhs{\isachardoublequoteclose}\isanewline
\ \ \ \ \isacommand{then}\isamarkupfalse%
\isanewline
\ \ \ \ \isacommand{obtain}\isamarkupfalse%
\ x\ \isakeyword{where}\isanewline
\ \ \ \ \ \ {\isachardoublequoteopen}v{\isasymin}val{\isacharparenleft}{\kern0pt}G{\isacharcomma}{\kern0pt}\ {\isacharquery}{\kern0pt}big{\isacharunderscore}{\kern0pt}name{\isacharparenright}{\kern0pt}{\isachardoublequoteclose}\ {\isachardoublequoteopen}M{\isacharbrackleft}{\kern0pt}G{\isacharbrackright}{\kern0pt}{\isacharcomma}{\kern0pt}\ {\isacharbrackleft}{\kern0pt}x{\isacharcomma}{\kern0pt}\ v{\isacharbrackright}{\kern0pt}\ {\isacharat}{\kern0pt}\ env\ {\isasymTurnstile}\ {\isasymphi}{\isachardoublequoteclose}\ {\isachardoublequoteopen}x{\isasymin}c{\isachardoublequoteclose}\isanewline
\ \ \ \ \ \ \isacommand{by}\isamarkupfalse%
\ blast\isanewline
\ \ \ \ \isacommand{moreover}\isamarkupfalse%
\ \isacommand{from}\isamarkupfalse%
\ this\ {\isacartoucheopen}c{\isasymin}M{\isacharbrackleft}{\kern0pt}G{\isacharbrackright}{\kern0pt}{\isacartoucheclose}\isanewline
\ \ \ \ \isacommand{have}\isamarkupfalse%
\ {\isachardoublequoteopen}v{\isasymin}M{\isacharbrackleft}{\kern0pt}G{\isacharbrackright}{\kern0pt}{\isachardoublequoteclose}\ {\isachardoublequoteopen}x{\isasymin}M{\isacharbrackleft}{\kern0pt}G{\isacharbrackright}{\kern0pt}{\isachardoublequoteclose}\isanewline
\ \ \ \ \ \ \isacommand{using}\isamarkupfalse%
\ transitivity{\isacharunderscore}{\kern0pt}MG\ GenExtI{\isacharbrackleft}{\kern0pt}OF\ {\isacartoucheopen}{\isacharquery}{\kern0pt}big{\isacharunderscore}{\kern0pt}name{\isasymin}{\isacharunderscore}{\kern0pt}{\isacartoucheclose}{\isacharcomma}{\kern0pt}of\ G{\isacharbrackright}{\kern0pt}\ \isacommand{by}\isamarkupfalse%
\ auto\isanewline
\ \ \ \ \isacommand{moreover}\isamarkupfalse%
\ \isacommand{from}\isamarkupfalse%
\ calculation\ {\isacartoucheopen}univalent{\isacharparenleft}{\kern0pt}{\isacharhash}{\kern0pt}{\isacharhash}{\kern0pt}M{\isacharbrackleft}{\kern0pt}G{\isacharbrackright}{\kern0pt}{\isacharcomma}{\kern0pt}{\isacharunderscore}{\kern0pt}{\isacharcomma}{\kern0pt}{\isacharunderscore}{\kern0pt}{\isacharparenright}{\kern0pt}{\isacartoucheclose}\isanewline
\ \ \ \ \isacommand{have}\isamarkupfalse%
\ {\isachardoublequoteopen}{\isacharquery}{\kern0pt}R{\isacharparenleft}{\kern0pt}x{\isacharcomma}{\kern0pt}y{\isacharparenright}{\kern0pt}\ {\isasymLongrightarrow}\ y\ {\isacharequal}{\kern0pt}\ v{\isachardoublequoteclose}\ \isakeyword{for}\ y\isanewline
\ \ \ \ \ \ \isacommand{unfolding}\isamarkupfalse%
\ univalent{\isacharunderscore}{\kern0pt}def\ \isacommand{by}\isamarkupfalse%
\ auto\isanewline
\ \ \ \ \isacommand{ultimately}\isamarkupfalse%
\isanewline
\ \ \ \ \isacommand{show}\isamarkupfalse%
\ {\isachardoublequoteopen}v{\isasymin}{\isacharquery}{\kern0pt}repl{\isachardoublequoteclose}\isanewline
\ \ \ \ \ \ \isacommand{using}\isamarkupfalse%
\ ReplaceI{\isacharbrackleft}{\kern0pt}of\ {\isacharquery}{\kern0pt}R\ x\ v\ c{\isacharbrackright}{\kern0pt}\isanewline
\ \ \ \ \ \ \isacommand{by}\isamarkupfalse%
\ blast\isanewline
\ \ \isacommand{qed}\isamarkupfalse%
\isanewline
\ \ \isacommand{moreover}\isamarkupfalse%
\isanewline
\ \ \isacommand{let}\isamarkupfalse%
\ {\isacharquery}{\kern0pt}{\isasympsi}\ {\isacharequal}{\kern0pt}\ {\isachardoublequoteopen}Exists{\isacharparenleft}{\kern0pt}And{\isacharparenleft}{\kern0pt}Member{\isacharparenleft}{\kern0pt}{\isadigit{0}}{\isacharcomma}{\kern0pt}{\isadigit{2}}{\isacharhash}{\kern0pt}{\isacharplus}{\kern0pt}length{\isacharparenleft}{\kern0pt}env{\isacharparenright}{\kern0pt}{\isacharparenright}{\kern0pt}{\isacharcomma}{\kern0pt}{\isasymphi}{\isacharparenright}{\kern0pt}{\isacharparenright}{\kern0pt}{\isachardoublequoteclose}\isanewline
\ \ \isacommand{have}\isamarkupfalse%
\ {\isachardoublequoteopen}v{\isasymin}M{\isacharbrackleft}{\kern0pt}G{\isacharbrackright}{\kern0pt}\ {\isasymLongrightarrow}\ {\isacharparenleft}{\kern0pt}{\isasymexists}x{\isasymin}c{\isachardot}{\kern0pt}\ M{\isacharbrackleft}{\kern0pt}G{\isacharbrackright}{\kern0pt}{\isacharcomma}{\kern0pt}\ {\isacharbrackleft}{\kern0pt}x{\isacharcomma}{\kern0pt}v{\isacharbrackright}{\kern0pt}\ {\isacharat}{\kern0pt}\ env\ {\isasymTurnstile}\ {\isasymphi}{\isacharparenright}{\kern0pt}\ {\isasymlongleftrightarrow}\ M{\isacharbrackleft}{\kern0pt}G{\isacharbrackright}{\kern0pt}{\isacharcomma}{\kern0pt}\ {\isacharbrackleft}{\kern0pt}v{\isacharbrackright}{\kern0pt}\ {\isacharat}{\kern0pt}\ env\ {\isacharat}{\kern0pt}\ {\isacharbrackleft}{\kern0pt}c{\isacharbrackright}{\kern0pt}\ {\isasymTurnstile}\ {\isacharquery}{\kern0pt}{\isasympsi}{\isachardoublequoteclose}\isanewline
\ \ \ \ {\isachardoublequoteopen}arity{\isacharparenleft}{\kern0pt}{\isacharquery}{\kern0pt}{\isasympsi}{\isacharparenright}{\kern0pt}\ {\isasymle}\ {\isadigit{2}}\ {\isacharhash}{\kern0pt}{\isacharplus}{\kern0pt}\ length{\isacharparenleft}{\kern0pt}env{\isacharparenright}{\kern0pt}{\isachardoublequoteclose}\ {\isachardoublequoteopen}{\isacharquery}{\kern0pt}{\isasympsi}{\isasymin}formula{\isachardoublequoteclose}\isanewline
\ \ \ \ \isakeyword{for}\ v\isanewline
\ \ \isacommand{proof}\isamarkupfalse%
\ {\isacharminus}{\kern0pt}\isanewline
\ \ \ \ \isacommand{fix}\isamarkupfalse%
\ v\isanewline
\ \ \ \ \isacommand{assume}\isamarkupfalse%
\ {\isachardoublequoteopen}v{\isasymin}M{\isacharbrackleft}{\kern0pt}G{\isacharbrackright}{\kern0pt}{\isachardoublequoteclose}\isanewline
\ \ \ \ \isacommand{with}\isamarkupfalse%
\ {\isacartoucheopen}c{\isasymin}M{\isacharbrackleft}{\kern0pt}G{\isacharbrackright}{\kern0pt}{\isacartoucheclose}\isanewline
\ \ \ \ \isacommand{have}\isamarkupfalse%
\ {\isachardoublequoteopen}nth{\isacharparenleft}{\kern0pt}length{\isacharparenleft}{\kern0pt}env{\isacharparenright}{\kern0pt}{\isacharhash}{\kern0pt}{\isacharplus}{\kern0pt}{\isadigit{1}}{\isacharcomma}{\kern0pt}{\isacharbrackleft}{\kern0pt}v{\isacharbrackright}{\kern0pt}{\isacharat}{\kern0pt}env{\isacharat}{\kern0pt}{\isacharbrackleft}{\kern0pt}c{\isacharbrackright}{\kern0pt}{\isacharparenright}{\kern0pt}\ {\isacharequal}{\kern0pt}\ c{\isachardoublequoteclose}\isanewline
\ \ \ \ \ \ \isacommand{using}\isamarkupfalse%
\ \ {\isacartoucheopen}env{\isasymin}{\isacharunderscore}{\kern0pt}{\isacartoucheclose}nth{\isacharunderscore}{\kern0pt}concat{\isacharbrackleft}{\kern0pt}of\ v\ c\ {\isachardoublequoteopen}M{\isacharbrackleft}{\kern0pt}G{\isacharbrackright}{\kern0pt}{\isachardoublequoteclose}\ env{\isacharbrackright}{\kern0pt}\isanewline
\ \ \ \ \ \ \isacommand{by}\isamarkupfalse%
\ auto\isanewline
\ \ \ \ \isacommand{note}\isamarkupfalse%
\ inMG{\isacharequal}{\kern0pt}\ {\isacartoucheopen}nth{\isacharparenleft}{\kern0pt}length{\isacharparenleft}{\kern0pt}env{\isacharparenright}{\kern0pt}{\isacharhash}{\kern0pt}{\isacharplus}{\kern0pt}{\isadigit{1}}{\isacharcomma}{\kern0pt}{\isacharbrackleft}{\kern0pt}v{\isacharbrackright}{\kern0pt}{\isacharat}{\kern0pt}env{\isacharat}{\kern0pt}{\isacharbrackleft}{\kern0pt}c{\isacharbrackright}{\kern0pt}{\isacharparenright}{\kern0pt}\ {\isacharequal}{\kern0pt}\ c{\isacartoucheclose}\ {\isacartoucheopen}c{\isasymin}M{\isacharbrackleft}{\kern0pt}G{\isacharbrackright}{\kern0pt}{\isacartoucheclose}\ {\isacartoucheopen}v{\isasymin}M{\isacharbrackleft}{\kern0pt}G{\isacharbrackright}{\kern0pt}{\isacartoucheclose}\ {\isacartoucheopen}env{\isasymin}{\isacharunderscore}{\kern0pt}{\isacartoucheclose}\isanewline
\ \ \ \ \isacommand{show}\isamarkupfalse%
\ {\isachardoublequoteopen}{\isacharparenleft}{\kern0pt}{\isasymexists}x{\isasymin}c{\isachardot}{\kern0pt}\ M{\isacharbrackleft}{\kern0pt}G{\isacharbrackright}{\kern0pt}{\isacharcomma}{\kern0pt}\ {\isacharbrackleft}{\kern0pt}x{\isacharcomma}{\kern0pt}v{\isacharbrackright}{\kern0pt}\ {\isacharat}{\kern0pt}\ env\ {\isasymTurnstile}\ {\isasymphi}{\isacharparenright}{\kern0pt}\ {\isasymlongleftrightarrow}\ M{\isacharbrackleft}{\kern0pt}G{\isacharbrackright}{\kern0pt}{\isacharcomma}{\kern0pt}\ {\isacharbrackleft}{\kern0pt}v{\isacharbrackright}{\kern0pt}\ {\isacharat}{\kern0pt}\ env\ {\isacharat}{\kern0pt}\ {\isacharbrackleft}{\kern0pt}c{\isacharbrackright}{\kern0pt}\ {\isasymTurnstile}\ {\isacharquery}{\kern0pt}{\isasympsi}{\isachardoublequoteclose}\isanewline
\ \ \ \ \isacommand{proof}\isamarkupfalse%
\isanewline
\ \ \ \ \ \ \isacommand{assume}\isamarkupfalse%
\ {\isachardoublequoteopen}{\isasymexists}x{\isasymin}c{\isachardot}{\kern0pt}\ M{\isacharbrackleft}{\kern0pt}G{\isacharbrackright}{\kern0pt}{\isacharcomma}{\kern0pt}\ {\isacharbrackleft}{\kern0pt}x{\isacharcomma}{\kern0pt}\ v{\isacharbrackright}{\kern0pt}\ {\isacharat}{\kern0pt}\ env\ {\isasymTurnstile}\ {\isasymphi}{\isachardoublequoteclose}\isanewline
\ \ \ \ \ \ \isacommand{then}\isamarkupfalse%
\ \isacommand{obtain}\isamarkupfalse%
\ x\ \isakeyword{where}\isanewline
\ \ \ \ \ \ \ \ {\isachardoublequoteopen}x{\isasymin}c{\isachardoublequoteclose}\ {\isachardoublequoteopen}M{\isacharbrackleft}{\kern0pt}G{\isacharbrackright}{\kern0pt}{\isacharcomma}{\kern0pt}\ {\isacharbrackleft}{\kern0pt}x{\isacharcomma}{\kern0pt}\ v{\isacharbrackright}{\kern0pt}\ {\isacharat}{\kern0pt}\ env\ {\isasymTurnstile}\ {\isasymphi}{\isachardoublequoteclose}\ {\isachardoublequoteopen}x{\isasymin}M{\isacharbrackleft}{\kern0pt}G{\isacharbrackright}{\kern0pt}{\isachardoublequoteclose}\isanewline
\ \ \ \ \ \ \ \ \isacommand{using}\isamarkupfalse%
\ transitivity{\isacharunderscore}{\kern0pt}MG{\isacharbrackleft}{\kern0pt}OF\ {\isacharunderscore}{\kern0pt}\ {\isacartoucheopen}c{\isasymin}M{\isacharbrackleft}{\kern0pt}G{\isacharbrackright}{\kern0pt}{\isacartoucheclose}{\isacharbrackright}{\kern0pt}\isanewline
\ \ \ \ \ \ \ \ \isacommand{by}\isamarkupfalse%
\ auto\isanewline
\ \ \ \ \ \ \isacommand{with}\isamarkupfalse%
\ {\isacartoucheopen}{\isasymphi}{\isasymin}{\isacharunderscore}{\kern0pt}{\isacartoucheclose}\ {\isacartoucheopen}arity{\isacharparenleft}{\kern0pt}{\isasymphi}{\isacharparenright}{\kern0pt}{\isasymle}{\isadigit{2}}{\isacharhash}{\kern0pt}{\isacharplus}{\kern0pt}length{\isacharparenleft}{\kern0pt}env{\isacharparenright}{\kern0pt}{\isacartoucheclose}\ inMG\isanewline
\ \ \ \ \ \ \isacommand{show}\isamarkupfalse%
\ {\isachardoublequoteopen}M{\isacharbrackleft}{\kern0pt}G{\isacharbrackright}{\kern0pt}{\isacharcomma}{\kern0pt}\ {\isacharbrackleft}{\kern0pt}v{\isacharbrackright}{\kern0pt}\ {\isacharat}{\kern0pt}\ env\ {\isacharat}{\kern0pt}\ {\isacharbrackleft}{\kern0pt}c{\isacharbrackright}{\kern0pt}\ {\isasymTurnstile}\ Exists{\isacharparenleft}{\kern0pt}And{\isacharparenleft}{\kern0pt}Member{\isacharparenleft}{\kern0pt}{\isadigit{0}}{\isacharcomma}{\kern0pt}\ {\isadigit{2}}\ {\isacharhash}{\kern0pt}{\isacharplus}{\kern0pt}\ length{\isacharparenleft}{\kern0pt}env{\isacharparenright}{\kern0pt}{\isacharparenright}{\kern0pt}{\isacharcomma}{\kern0pt}\ {\isasymphi}{\isacharparenright}{\kern0pt}{\isacharparenright}{\kern0pt}{\isachardoublequoteclose}\isanewline
\ \ \ \ \ \ \ \ \isacommand{using}\isamarkupfalse%
\ arity{\isacharunderscore}{\kern0pt}sats{\isacharunderscore}{\kern0pt}iff{\isacharbrackleft}{\kern0pt}of\ {\isasymphi}\ {\isachardoublequoteopen}{\isacharbrackleft}{\kern0pt}c{\isacharbrackright}{\kern0pt}{\isachardoublequoteclose}\ {\isacharunderscore}{\kern0pt}\ {\isachardoublequoteopen}{\isacharbrackleft}{\kern0pt}x{\isacharcomma}{\kern0pt}v{\isacharbrackright}{\kern0pt}{\isacharat}{\kern0pt}env{\isachardoublequoteclose}{\isacharbrackright}{\kern0pt}\isanewline
\ \ \ \ \ \ \ \ \isacommand{by}\isamarkupfalse%
\ auto\isanewline
\ \ \ \ \isacommand{next}\isamarkupfalse%
\isanewline
\ \ \ \ \ \ \isacommand{assume}\isamarkupfalse%
\ {\isachardoublequoteopen}M{\isacharbrackleft}{\kern0pt}G{\isacharbrackright}{\kern0pt}{\isacharcomma}{\kern0pt}\ {\isacharbrackleft}{\kern0pt}v{\isacharbrackright}{\kern0pt}\ {\isacharat}{\kern0pt}\ env\ {\isacharat}{\kern0pt}\ {\isacharbrackleft}{\kern0pt}c{\isacharbrackright}{\kern0pt}\ {\isasymTurnstile}\ Exists{\isacharparenleft}{\kern0pt}And{\isacharparenleft}{\kern0pt}Member{\isacharparenleft}{\kern0pt}{\isadigit{0}}{\isacharcomma}{\kern0pt}\ {\isadigit{2}}\ {\isacharhash}{\kern0pt}{\isacharplus}{\kern0pt}\ length{\isacharparenleft}{\kern0pt}env{\isacharparenright}{\kern0pt}{\isacharparenright}{\kern0pt}{\isacharcomma}{\kern0pt}\ {\isasymphi}{\isacharparenright}{\kern0pt}{\isacharparenright}{\kern0pt}{\isachardoublequoteclose}\isanewline
\ \ \ \ \ \ \isacommand{with}\isamarkupfalse%
\ inMG\isanewline
\ \ \ \ \ \ \isacommand{obtain}\isamarkupfalse%
\ x\ \isakeyword{where}\isanewline
\ \ \ \ \ \ \ \ {\isachardoublequoteopen}x{\isasymin}M{\isacharbrackleft}{\kern0pt}G{\isacharbrackright}{\kern0pt}{\isachardoublequoteclose}\ {\isachardoublequoteopen}x{\isasymin}c{\isachardoublequoteclose}\ {\isachardoublequoteopen}M{\isacharbrackleft}{\kern0pt}G{\isacharbrackright}{\kern0pt}{\isacharcomma}{\kern0pt}\ {\isacharbrackleft}{\kern0pt}x{\isacharcomma}{\kern0pt}v{\isacharbrackright}{\kern0pt}{\isacharat}{\kern0pt}env{\isacharat}{\kern0pt}{\isacharbrackleft}{\kern0pt}c{\isacharbrackright}{\kern0pt}\ {\isasymTurnstile}\ {\isasymphi}{\isachardoublequoteclose}\isanewline
\ \ \ \ \ \ \ \ \isacommand{by}\isamarkupfalse%
\ auto\isanewline
\ \ \ \ \ \ \isacommand{with}\isamarkupfalse%
\ {\isacartoucheopen}{\isasymphi}{\isasymin}{\isacharunderscore}{\kern0pt}{\isacartoucheclose}\ {\isacartoucheopen}arity{\isacharparenleft}{\kern0pt}{\isasymphi}{\isacharparenright}{\kern0pt}{\isasymle}{\isadigit{2}}{\isacharhash}{\kern0pt}{\isacharplus}{\kern0pt}length{\isacharparenleft}{\kern0pt}env{\isacharparenright}{\kern0pt}{\isacartoucheclose}\ inMG\isanewline
\ \ \ \ \ \ \isacommand{show}\isamarkupfalse%
\ {\isachardoublequoteopen}{\isasymexists}x{\isasymin}c{\isachardot}{\kern0pt}\ M{\isacharbrackleft}{\kern0pt}G{\isacharbrackright}{\kern0pt}{\isacharcomma}{\kern0pt}\ {\isacharbrackleft}{\kern0pt}x{\isacharcomma}{\kern0pt}\ v{\isacharbrackright}{\kern0pt}\ {\isacharat}{\kern0pt}\ env{\isasymTurnstile}\ {\isasymphi}{\isachardoublequoteclose}\isanewline
\ \ \ \ \ \ \ \ \isacommand{using}\isamarkupfalse%
\ arity{\isacharunderscore}{\kern0pt}sats{\isacharunderscore}{\kern0pt}iff{\isacharbrackleft}{\kern0pt}of\ {\isasymphi}\ {\isachardoublequoteopen}{\isacharbrackleft}{\kern0pt}c{\isacharbrackright}{\kern0pt}{\isachardoublequoteclose}\ {\isacharunderscore}{\kern0pt}\ {\isachardoublequoteopen}{\isacharbrackleft}{\kern0pt}x{\isacharcomma}{\kern0pt}v{\isacharbrackright}{\kern0pt}{\isacharat}{\kern0pt}env{\isachardoublequoteclose}{\isacharbrackright}{\kern0pt}\isanewline
\ \ \ \ \ \ \ \ \isacommand{by}\isamarkupfalse%
\ auto\isanewline
\ \ \ \ \isacommand{qed}\isamarkupfalse%
\isanewline
\ \ \isacommand{next}\isamarkupfalse%
\isanewline
\ \ \ \ \isacommand{from}\isamarkupfalse%
\ {\isacartoucheopen}env{\isasymin}{\isacharunderscore}{\kern0pt}{\isacartoucheclose}\ {\isacartoucheopen}{\isasymphi}{\isasymin}{\isacharunderscore}{\kern0pt}{\isacartoucheclose}\isanewline
\ \ \ \ \isacommand{show}\isamarkupfalse%
\ {\isachardoublequoteopen}arity{\isacharparenleft}{\kern0pt}{\isacharquery}{\kern0pt}{\isasympsi}{\isacharparenright}{\kern0pt}{\isasymle}{\isadigit{2}}{\isacharhash}{\kern0pt}{\isacharplus}{\kern0pt}length{\isacharparenleft}{\kern0pt}env{\isacharparenright}{\kern0pt}{\isachardoublequoteclose}\isanewline
\ \ \ \ \ \ \isacommand{using}\isamarkupfalse%
\ pred{\isacharunderscore}{\kern0pt}mono{\isacharbrackleft}{\kern0pt}OF\ {\isacharunderscore}{\kern0pt}\ {\isacartoucheopen}arity{\isacharparenleft}{\kern0pt}{\isasymphi}{\isacharparenright}{\kern0pt}{\isasymle}{\isadigit{2}}{\isacharhash}{\kern0pt}{\isacharplus}{\kern0pt}length{\isacharparenleft}{\kern0pt}env{\isacharparenright}{\kern0pt}{\isacartoucheclose}{\isacharbrackright}{\kern0pt}\ lt{\isacharunderscore}{\kern0pt}trans{\isacharbrackleft}{\kern0pt}OF\ {\isacharunderscore}{\kern0pt}\ le{\isacharunderscore}{\kern0pt}refl{\isacharbrackright}{\kern0pt}\isanewline
\ \ \ \ \ \ \isacommand{by}\isamarkupfalse%
\ {\isacharparenleft}{\kern0pt}auto\ simp\ add{\isacharcolon}{\kern0pt}nat{\isacharunderscore}{\kern0pt}simp{\isacharunderscore}{\kern0pt}union{\isacharparenright}{\kern0pt}\isanewline
\ \ \isacommand{next}\isamarkupfalse%
\isanewline
\ \ \ \ \isacommand{from}\isamarkupfalse%
\ {\isacartoucheopen}{\isasymphi}{\isasymin}{\isacharunderscore}{\kern0pt}{\isacartoucheclose}\isanewline
\ \ \ \ \isacommand{show}\isamarkupfalse%
\ {\isachardoublequoteopen}{\isacharquery}{\kern0pt}{\isasympsi}{\isasymin}formula{\isachardoublequoteclose}\ \isacommand{by}\isamarkupfalse%
\ simp\isanewline
\ \ \isacommand{qed}\isamarkupfalse%
\isanewline
\ \ \isacommand{moreover}\isamarkupfalse%
\ \isacommand{from}\isamarkupfalse%
\ this\isanewline
\ \ \isacommand{have}\isamarkupfalse%
\ {\isachardoublequoteopen}{\isacharbraceleft}{\kern0pt}v{\isasymin}{\isacharquery}{\kern0pt}big{\isachardot}{\kern0pt}\ {\isasymexists}x{\isasymin}c{\isachardot}{\kern0pt}\ M{\isacharbrackleft}{\kern0pt}G{\isacharbrackright}{\kern0pt}{\isacharcomma}{\kern0pt}\ {\isacharbrackleft}{\kern0pt}x{\isacharcomma}{\kern0pt}v{\isacharbrackright}{\kern0pt}\ {\isacharat}{\kern0pt}\ env\ {\isasymTurnstile}\ {\isasymphi}{\isacharbraceright}{\kern0pt}\ {\isacharequal}{\kern0pt}\ {\isacharbraceleft}{\kern0pt}v{\isasymin}{\isacharquery}{\kern0pt}big{\isachardot}{\kern0pt}\ M{\isacharbrackleft}{\kern0pt}G{\isacharbrackright}{\kern0pt}{\isacharcomma}{\kern0pt}\ {\isacharbrackleft}{\kern0pt}v{\isacharbrackright}{\kern0pt}\ {\isacharat}{\kern0pt}\ env\ {\isacharat}{\kern0pt}\ {\isacharbrackleft}{\kern0pt}c{\isacharbrackright}{\kern0pt}\ {\isasymTurnstile}\ \ {\isacharquery}{\kern0pt}{\isasympsi}{\isacharbraceright}{\kern0pt}{\isachardoublequoteclose}\isanewline
\ \ \ \ \isacommand{using}\isamarkupfalse%
\ transitivity{\isacharunderscore}{\kern0pt}MG{\isacharbrackleft}{\kern0pt}OF\ {\isacharunderscore}{\kern0pt}\ GenExtI{\isacharcomma}{\kern0pt}\ OF\ {\isacharunderscore}{\kern0pt}\ {\isacartoucheopen}{\isacharquery}{\kern0pt}big{\isacharunderscore}{\kern0pt}name{\isasymin}M{\isacartoucheclose}{\isacharbrackright}{\kern0pt}\isanewline
\ \ \ \ \isacommand{by}\isamarkupfalse%
\ simp\isanewline
\ \ \isacommand{moreover}\isamarkupfalse%
\ \isacommand{from}\isamarkupfalse%
\ calculation\ \isakeyword{and}\ {\isacartoucheopen}env{\isasymin}{\isacharunderscore}{\kern0pt}{\isacartoucheclose}\ {\isacartoucheopen}c{\isasymin}{\isacharunderscore}{\kern0pt}{\isacartoucheclose}\ {\isacartoucheopen}{\isacharquery}{\kern0pt}big{\isasymin}M{\isacharbrackleft}{\kern0pt}G{\isacharbrackright}{\kern0pt}{\isacartoucheclose}\isanewline
\ \ \isacommand{have}\isamarkupfalse%
\ {\isachardoublequoteopen}{\isacharbraceleft}{\kern0pt}v{\isasymin}{\isacharquery}{\kern0pt}big{\isachardot}{\kern0pt}\ M{\isacharbrackleft}{\kern0pt}G{\isacharbrackright}{\kern0pt}\ {\isacharcomma}{\kern0pt}\ {\isacharbrackleft}{\kern0pt}v{\isacharbrackright}{\kern0pt}\ {\isacharat}{\kern0pt}\ env\ {\isacharat}{\kern0pt}\ {\isacharbrackleft}{\kern0pt}c{\isacharbrackright}{\kern0pt}\ {\isasymTurnstile}\ {\isacharquery}{\kern0pt}{\isasympsi}{\isacharbraceright}{\kern0pt}\ {\isasymin}\ M{\isacharbrackleft}{\kern0pt}G{\isacharbrackright}{\kern0pt}{\isachardoublequoteclose}\isanewline
\ \ \ \ \isacommand{using}\isamarkupfalse%
\ Collect{\isacharunderscore}{\kern0pt}sats{\isacharunderscore}{\kern0pt}in{\isacharunderscore}{\kern0pt}MG\ \isacommand{by}\isamarkupfalse%
\ auto\isanewline
\ \ \isacommand{ultimately}\isamarkupfalse%
\isanewline
\ \ \isacommand{show}\isamarkupfalse%
\ {\isacharquery}{\kern0pt}thesis\ \isacommand{by}\isamarkupfalse%
\ simp\isanewline
\isacommand{qed}\isamarkupfalse%
%
\endisatagproof
{\isafoldproof}%
%
\isadelimproof
\isanewline
%
\endisadelimproof
\isanewline
\isacommand{theorem}\isamarkupfalse%
\ strong{\isacharunderscore}{\kern0pt}replacement{\isacharunderscore}{\kern0pt}in{\isacharunderscore}{\kern0pt}MG{\isacharcolon}{\kern0pt}\isanewline
\ \ \isakeyword{assumes}\isanewline
\ \ \ \ {\isachardoublequoteopen}{\isasymphi}{\isasymin}formula{\isachardoublequoteclose}\ \isakeyword{and}\ {\isachardoublequoteopen}arity{\isacharparenleft}{\kern0pt}{\isasymphi}{\isacharparenright}{\kern0pt}\ {\isasymle}\ {\isadigit{2}}\ {\isacharhash}{\kern0pt}{\isacharplus}{\kern0pt}\ length{\isacharparenleft}{\kern0pt}env{\isacharparenright}{\kern0pt}{\isachardoublequoteclose}\ {\isachardoublequoteopen}env\ {\isasymin}\ list{\isacharparenleft}{\kern0pt}M{\isacharbrackleft}{\kern0pt}G{\isacharbrackright}{\kern0pt}{\isacharparenright}{\kern0pt}{\isachardoublequoteclose}\isanewline
\ \ \isakeyword{shows}\isanewline
\ \ \ \ {\isachardoublequoteopen}strong{\isacharunderscore}{\kern0pt}replacement{\isacharparenleft}{\kern0pt}{\isacharhash}{\kern0pt}{\isacharhash}{\kern0pt}M{\isacharbrackleft}{\kern0pt}G{\isacharbrackright}{\kern0pt}{\isacharcomma}{\kern0pt}{\isasymlambda}x\ v{\isachardot}{\kern0pt}\ sats{\isacharparenleft}{\kern0pt}M{\isacharbrackleft}{\kern0pt}G{\isacharbrackright}{\kern0pt}{\isacharcomma}{\kern0pt}{\isasymphi}{\isacharcomma}{\kern0pt}{\isacharbrackleft}{\kern0pt}x{\isacharcomma}{\kern0pt}v{\isacharbrackright}{\kern0pt}\ {\isacharat}{\kern0pt}\ env{\isacharparenright}{\kern0pt}{\isacharparenright}{\kern0pt}{\isachardoublequoteclose}\isanewline
%
\isadelimproof
%
\endisadelimproof
%
\isatagproof
\isacommand{proof}\isamarkupfalse%
\ {\isacharminus}{\kern0pt}\isanewline
\ \ \isacommand{let}\isamarkupfalse%
\ {\isacharquery}{\kern0pt}R{\isacharequal}{\kern0pt}{\isachardoublequoteopen}{\isasymlambda}x\ y\ {\isachardot}{\kern0pt}\ M{\isacharbrackleft}{\kern0pt}G{\isacharbrackright}{\kern0pt}{\isacharcomma}{\kern0pt}\ {\isacharbrackleft}{\kern0pt}x{\isacharcomma}{\kern0pt}\ y{\isacharbrackright}{\kern0pt}\ {\isacharat}{\kern0pt}\ env\ {\isasymTurnstile}\ {\isasymphi}{\isachardoublequoteclose}\isanewline
\ \ \isacommand{{\isacharbraceleft}{\kern0pt}}\isamarkupfalse%
\isanewline
\ \ \ \ \isacommand{fix}\isamarkupfalse%
\ A\isanewline
\ \ \ \ \isacommand{let}\isamarkupfalse%
\ {\isacharquery}{\kern0pt}Y{\isacharequal}{\kern0pt}{\isachardoublequoteopen}{\isacharbraceleft}{\kern0pt}v\ {\isachardot}{\kern0pt}\ x\ {\isasymin}\ A{\isacharcomma}{\kern0pt}\ v{\isasymin}M{\isacharbrackleft}{\kern0pt}G{\isacharbrackright}{\kern0pt}\ {\isasymand}\ {\isacharquery}{\kern0pt}R{\isacharparenleft}{\kern0pt}x{\isacharcomma}{\kern0pt}v{\isacharparenright}{\kern0pt}{\isacharbraceright}{\kern0pt}{\isachardoublequoteclose}\isanewline
\ \ \ \ \isacommand{assume}\isamarkupfalse%
\ {\isadigit{1}}{\isacharcolon}{\kern0pt}\ {\isachardoublequoteopen}{\isacharparenleft}{\kern0pt}{\isacharhash}{\kern0pt}{\isacharhash}{\kern0pt}M{\isacharbrackleft}{\kern0pt}G{\isacharbrackright}{\kern0pt}{\isacharparenright}{\kern0pt}{\isacharparenleft}{\kern0pt}A{\isacharparenright}{\kern0pt}{\isachardoublequoteclose}\isanewline
\ \ \ \ \ \ {\isachardoublequoteopen}{\isasymforall}x{\isacharbrackleft}{\kern0pt}{\isacharhash}{\kern0pt}{\isacharhash}{\kern0pt}M{\isacharbrackleft}{\kern0pt}G{\isacharbrackright}{\kern0pt}{\isacharbrackright}{\kern0pt}{\isachardot}{\kern0pt}\ x\ {\isasymin}\ A\ {\isasymlongrightarrow}\ {\isacharparenleft}{\kern0pt}{\isasymforall}y{\isacharbrackleft}{\kern0pt}{\isacharhash}{\kern0pt}{\isacharhash}{\kern0pt}M{\isacharbrackleft}{\kern0pt}G{\isacharbrackright}{\kern0pt}{\isacharbrackright}{\kern0pt}{\isachardot}{\kern0pt}\ {\isasymforall}z{\isacharbrackleft}{\kern0pt}{\isacharhash}{\kern0pt}{\isacharhash}{\kern0pt}M{\isacharbrackleft}{\kern0pt}G{\isacharbrackright}{\kern0pt}{\isacharbrackright}{\kern0pt}{\isachardot}{\kern0pt}\ {\isacharquery}{\kern0pt}R{\isacharparenleft}{\kern0pt}x{\isacharcomma}{\kern0pt}y{\isacharparenright}{\kern0pt}\ {\isasymand}\ {\isacharquery}{\kern0pt}R{\isacharparenleft}{\kern0pt}x{\isacharcomma}{\kern0pt}z{\isacharparenright}{\kern0pt}\ {\isasymlongrightarrow}\ y\ {\isacharequal}{\kern0pt}\ z{\isacharparenright}{\kern0pt}{\isachardoublequoteclose}\isanewline
\ \ \ \ \isacommand{then}\isamarkupfalse%
\isanewline
\ \ \ \ \isacommand{have}\isamarkupfalse%
\ {\isachardoublequoteopen}univalent{\isacharparenleft}{\kern0pt}{\isacharhash}{\kern0pt}{\isacharhash}{\kern0pt}M{\isacharbrackleft}{\kern0pt}G{\isacharbrackright}{\kern0pt}{\isacharcomma}{\kern0pt}\ A{\isacharcomma}{\kern0pt}\ {\isacharquery}{\kern0pt}R{\isacharparenright}{\kern0pt}{\isachardoublequoteclose}\ {\isachardoublequoteopen}A{\isasymin}M{\isacharbrackleft}{\kern0pt}G{\isacharbrackright}{\kern0pt}{\isachardoublequoteclose}\isanewline
\ \ \ \ \ \ \isacommand{unfolding}\isamarkupfalse%
\ univalent{\isacharunderscore}{\kern0pt}def\ \isacommand{by}\isamarkupfalse%
\ simp{\isacharunderscore}{\kern0pt}all\isanewline
\ \ \ \ \isacommand{with}\isamarkupfalse%
\ assms\ {\isacartoucheopen}A{\isasymin}{\isacharunderscore}{\kern0pt}{\isacartoucheclose}\isanewline
\ \ \ \ \isacommand{have}\isamarkupfalse%
\ {\isachardoublequoteopen}{\isacharparenleft}{\kern0pt}{\isacharhash}{\kern0pt}{\isacharhash}{\kern0pt}M{\isacharbrackleft}{\kern0pt}G{\isacharbrackright}{\kern0pt}{\isacharparenright}{\kern0pt}{\isacharparenleft}{\kern0pt}{\isacharquery}{\kern0pt}Y{\isacharparenright}{\kern0pt}{\isachardoublequoteclose}\isanewline
\ \ \ \ \ \ \isacommand{using}\isamarkupfalse%
\ Replace{\isacharunderscore}{\kern0pt}sats{\isacharunderscore}{\kern0pt}in{\isacharunderscore}{\kern0pt}MG\ \isacommand{by}\isamarkupfalse%
\ auto\isanewline
\ \ \ \ \isacommand{have}\isamarkupfalse%
\ {\isachardoublequoteopen}b\ {\isasymin}\ {\isacharquery}{\kern0pt}Y\ {\isasymlongleftrightarrow}\ {\isacharparenleft}{\kern0pt}{\isasymexists}x{\isacharbrackleft}{\kern0pt}{\isacharhash}{\kern0pt}{\isacharhash}{\kern0pt}M{\isacharbrackleft}{\kern0pt}G{\isacharbrackright}{\kern0pt}{\isacharbrackright}{\kern0pt}{\isachardot}{\kern0pt}\ x\ {\isasymin}\ A\ {\isasymand}\ {\isacharquery}{\kern0pt}R{\isacharparenleft}{\kern0pt}x{\isacharcomma}{\kern0pt}b{\isacharparenright}{\kern0pt}{\isacharparenright}{\kern0pt}{\isachardoublequoteclose}\ \isakeyword{if}\ {\isachardoublequoteopen}{\isacharparenleft}{\kern0pt}{\isacharhash}{\kern0pt}{\isacharhash}{\kern0pt}M{\isacharbrackleft}{\kern0pt}G{\isacharbrackright}{\kern0pt}{\isacharparenright}{\kern0pt}{\isacharparenleft}{\kern0pt}b{\isacharparenright}{\kern0pt}{\isachardoublequoteclose}\ \isakeyword{for}\ b\isanewline
\ \ \ \ \isacommand{proof}\isamarkupfalse%
{\isacharparenleft}{\kern0pt}rule{\isacharparenright}{\kern0pt}\isanewline
\ \ \ \ \ \ \isacommand{from}\isamarkupfalse%
\ {\isacartoucheopen}A{\isasymin}{\isacharunderscore}{\kern0pt}{\isacartoucheclose}\isanewline
\ \ \ \ \ \ \isacommand{show}\isamarkupfalse%
\ {\isachardoublequoteopen}{\isasymexists}x{\isacharbrackleft}{\kern0pt}{\isacharhash}{\kern0pt}{\isacharhash}{\kern0pt}M{\isacharbrackleft}{\kern0pt}G{\isacharbrackright}{\kern0pt}{\isacharbrackright}{\kern0pt}{\isachardot}{\kern0pt}\ x\ {\isasymin}\ A\ {\isasymand}\ {\isacharquery}{\kern0pt}R{\isacharparenleft}{\kern0pt}x{\isacharcomma}{\kern0pt}b{\isacharparenright}{\kern0pt}{\isachardoublequoteclose}\ \isakeyword{if}\ {\isachardoublequoteopen}b\ {\isasymin}\ {\isacharquery}{\kern0pt}Y{\isachardoublequoteclose}\isanewline
\ \ \ \ \ \ \ \ \isacommand{using}\isamarkupfalse%
\ that\ transitivity{\isacharunderscore}{\kern0pt}MG\ \isacommand{by}\isamarkupfalse%
\ auto\isanewline
\ \ \ \ \isacommand{next}\isamarkupfalse%
\isanewline
\ \ \ \ \ \ \isacommand{show}\isamarkupfalse%
\ {\isachardoublequoteopen}b\ {\isasymin}\ {\isacharquery}{\kern0pt}Y{\isachardoublequoteclose}\ \isakeyword{if}\ {\isachardoublequoteopen}{\isasymexists}x{\isacharbrackleft}{\kern0pt}{\isacharhash}{\kern0pt}{\isacharhash}{\kern0pt}M{\isacharbrackleft}{\kern0pt}G{\isacharbrackright}{\kern0pt}{\isacharbrackright}{\kern0pt}{\isachardot}{\kern0pt}\ x\ {\isasymin}\ A\ {\isasymand}\ {\isacharquery}{\kern0pt}R{\isacharparenleft}{\kern0pt}x{\isacharcomma}{\kern0pt}b{\isacharparenright}{\kern0pt}{\isachardoublequoteclose}\isanewline
\ \ \ \ \ \ \isacommand{proof}\isamarkupfalse%
\ {\isacharminus}{\kern0pt}\isanewline
\ \ \ \ \ \ \ \ \isacommand{from}\isamarkupfalse%
\ {\isacartoucheopen}{\isacharparenleft}{\kern0pt}{\isacharhash}{\kern0pt}{\isacharhash}{\kern0pt}M{\isacharbrackleft}{\kern0pt}G{\isacharbrackright}{\kern0pt}{\isacharparenright}{\kern0pt}{\isacharparenleft}{\kern0pt}b{\isacharparenright}{\kern0pt}{\isacartoucheclose}\isanewline
\ \ \ \ \ \ \ \ \isacommand{have}\isamarkupfalse%
\ {\isachardoublequoteopen}b{\isasymin}M{\isacharbrackleft}{\kern0pt}G{\isacharbrackright}{\kern0pt}{\isachardoublequoteclose}\ \isacommand{by}\isamarkupfalse%
\ simp\isanewline
\ \ \ \ \ \ \ \ \isacommand{with}\isamarkupfalse%
\ that\isanewline
\ \ \ \ \ \ \ \ \isacommand{obtain}\isamarkupfalse%
\ x\ \isakeyword{where}\ {\isachardoublequoteopen}{\isacharparenleft}{\kern0pt}{\isacharhash}{\kern0pt}{\isacharhash}{\kern0pt}M{\isacharbrackleft}{\kern0pt}G{\isacharbrackright}{\kern0pt}{\isacharparenright}{\kern0pt}{\isacharparenleft}{\kern0pt}x{\isacharparenright}{\kern0pt}{\isachardoublequoteclose}\ {\isachardoublequoteopen}x{\isasymin}A{\isachardoublequoteclose}\ {\isachardoublequoteopen}b{\isasymin}M{\isacharbrackleft}{\kern0pt}G{\isacharbrackright}{\kern0pt}\ {\isasymand}\ {\isacharquery}{\kern0pt}R{\isacharparenleft}{\kern0pt}x{\isacharcomma}{\kern0pt}b{\isacharparenright}{\kern0pt}{\isachardoublequoteclose}\isanewline
\ \ \ \ \ \ \ \ \ \ \isacommand{by}\isamarkupfalse%
\ blast\isanewline
\ \ \ \ \ \ \ \ \isacommand{moreover}\isamarkupfalse%
\ \isacommand{from}\isamarkupfalse%
\ this\ {\isadigit{1}}\ {\isacartoucheopen}{\isacharparenleft}{\kern0pt}{\isacharhash}{\kern0pt}{\isacharhash}{\kern0pt}M{\isacharbrackleft}{\kern0pt}G{\isacharbrackright}{\kern0pt}{\isacharparenright}{\kern0pt}{\isacharparenleft}{\kern0pt}b{\isacharparenright}{\kern0pt}{\isacartoucheclose}\isanewline
\ \ \ \ \ \ \ \ \isacommand{have}\isamarkupfalse%
\ {\isachardoublequoteopen}x{\isasymin}M{\isacharbrackleft}{\kern0pt}G{\isacharbrackright}{\kern0pt}{\isachardoublequoteclose}\ {\isachardoublequoteopen}z{\isasymin}M{\isacharbrackleft}{\kern0pt}G{\isacharbrackright}{\kern0pt}\ {\isasymand}\ {\isacharquery}{\kern0pt}R{\isacharparenleft}{\kern0pt}x{\isacharcomma}{\kern0pt}z{\isacharparenright}{\kern0pt}\ {\isasymLongrightarrow}\ b\ {\isacharequal}{\kern0pt}\ z{\isachardoublequoteclose}\ \isakeyword{for}\ z\isanewline
\ \ \ \ \ \ \ \ \ \ \isacommand{by}\isamarkupfalse%
\ auto\isanewline
\ \ \ \ \ \ \ \ \isacommand{ultimately}\isamarkupfalse%
\isanewline
\ \ \ \ \ \ \ \ \isacommand{show}\isamarkupfalse%
\ {\isacharquery}{\kern0pt}thesis\isanewline
\ \ \ \ \ \ \ \ \ \ \isacommand{using}\isamarkupfalse%
\ ReplaceI{\isacharbrackleft}{\kern0pt}of\ {\isachardoublequoteopen}{\isasymlambda}\ x\ y{\isachardot}{\kern0pt}\ y{\isasymin}M{\isacharbrackleft}{\kern0pt}G{\isacharbrackright}{\kern0pt}\ {\isasymand}\ {\isacharquery}{\kern0pt}R{\isacharparenleft}{\kern0pt}x{\isacharcomma}{\kern0pt}y{\isacharparenright}{\kern0pt}{\isachardoublequoteclose}{\isacharbrackright}{\kern0pt}\ \isacommand{by}\isamarkupfalse%
\ auto\isanewline
\ \ \ \ \ \ \isacommand{qed}\isamarkupfalse%
\isanewline
\ \ \ \ \isacommand{qed}\isamarkupfalse%
\isanewline
\ \ \ \ \isacommand{then}\isamarkupfalse%
\isanewline
\ \ \ \ \isacommand{have}\isamarkupfalse%
\ {\isachardoublequoteopen}{\isasymforall}b{\isacharbrackleft}{\kern0pt}{\isacharhash}{\kern0pt}{\isacharhash}{\kern0pt}M{\isacharbrackleft}{\kern0pt}G{\isacharbrackright}{\kern0pt}{\isacharbrackright}{\kern0pt}{\isachardot}{\kern0pt}\ b\ {\isasymin}\ {\isacharquery}{\kern0pt}Y\ {\isasymlongleftrightarrow}\ {\isacharparenleft}{\kern0pt}{\isasymexists}x{\isacharbrackleft}{\kern0pt}{\isacharhash}{\kern0pt}{\isacharhash}{\kern0pt}M{\isacharbrackleft}{\kern0pt}G{\isacharbrackright}{\kern0pt}{\isacharbrackright}{\kern0pt}{\isachardot}{\kern0pt}\ x\ {\isasymin}\ A\ {\isasymand}\ {\isacharquery}{\kern0pt}R{\isacharparenleft}{\kern0pt}x{\isacharcomma}{\kern0pt}b{\isacharparenright}{\kern0pt}{\isacharparenright}{\kern0pt}{\isachardoublequoteclose}\isanewline
\ \ \ \ \ \ \isacommand{by}\isamarkupfalse%
\ simp\isanewline
\ \ \ \ \isacommand{with}\isamarkupfalse%
\ {\isacartoucheopen}{\isacharparenleft}{\kern0pt}{\isacharhash}{\kern0pt}{\isacharhash}{\kern0pt}M{\isacharbrackleft}{\kern0pt}G{\isacharbrackright}{\kern0pt}{\isacharparenright}{\kern0pt}{\isacharparenleft}{\kern0pt}{\isacharquery}{\kern0pt}Y{\isacharparenright}{\kern0pt}{\isacartoucheclose}\isanewline
\ \ \ \ \isacommand{have}\isamarkupfalse%
\ {\isachardoublequoteopen}\ {\isacharparenleft}{\kern0pt}{\isasymexists}Y{\isacharbrackleft}{\kern0pt}{\isacharhash}{\kern0pt}{\isacharhash}{\kern0pt}M{\isacharbrackleft}{\kern0pt}G{\isacharbrackright}{\kern0pt}{\isacharbrackright}{\kern0pt}{\isachardot}{\kern0pt}\ {\isasymforall}b{\isacharbrackleft}{\kern0pt}{\isacharhash}{\kern0pt}{\isacharhash}{\kern0pt}M{\isacharbrackleft}{\kern0pt}G{\isacharbrackright}{\kern0pt}{\isacharbrackright}{\kern0pt}{\isachardot}{\kern0pt}\ b\ {\isasymin}\ Y\ {\isasymlongleftrightarrow}\ {\isacharparenleft}{\kern0pt}{\isasymexists}x{\isacharbrackleft}{\kern0pt}{\isacharhash}{\kern0pt}{\isacharhash}{\kern0pt}M{\isacharbrackleft}{\kern0pt}G{\isacharbrackright}{\kern0pt}{\isacharbrackright}{\kern0pt}{\isachardot}{\kern0pt}\ x\ {\isasymin}\ A\ {\isasymand}\ {\isacharquery}{\kern0pt}R{\isacharparenleft}{\kern0pt}x{\isacharcomma}{\kern0pt}b{\isacharparenright}{\kern0pt}{\isacharparenright}{\kern0pt}{\isacharparenright}{\kern0pt}{\isachardoublequoteclose}\isanewline
\ \ \ \ \ \ \isacommand{by}\isamarkupfalse%
\ auto\isanewline
\ \ \isacommand{{\isacharbraceright}{\kern0pt}}\isamarkupfalse%
\isanewline
\ \ \isacommand{then}\isamarkupfalse%
\ \isacommand{show}\isamarkupfalse%
\ {\isacharquery}{\kern0pt}thesis\ \isacommand{unfolding}\isamarkupfalse%
\ strong{\isacharunderscore}{\kern0pt}replacement{\isacharunderscore}{\kern0pt}def\ univalent{\isacharunderscore}{\kern0pt}def\isanewline
\ \ \ \ \isacommand{by}\isamarkupfalse%
\ auto\isanewline
\isacommand{qed}\isamarkupfalse%
%
\endisatagproof
{\isafoldproof}%
%
\isadelimproof
\isanewline
%
\endisadelimproof
\isanewline
\isacommand{end}\isamarkupfalse%
\ \isanewline
%
\isadelimtheory
\isanewline
%
\endisadelimtheory
%
\isatagtheory
\isacommand{end}\isamarkupfalse%
%
\endisatagtheory
{\isafoldtheory}%
%
\isadelimtheory
%
\endisadelimtheory
%
\end{isabellebody}%
\endinput
%:%file=~/source/repos/ZF-notAC/code/Forcing/Replacement_Axiom.thy%:%
%:%11=1%:%
%:%27=2%:%
%:%28=2%:%
%:%29=3%:%
%:%30=4%:%
%:%31=5%:%
%:%36=5%:%
%:%41=6%:%
%:%46=7%:%
%:%47=7%:%
%:%52=7%:%
%:%55=8%:%
%:%56=9%:%
%:%57=9%:%
%:%58=10%:%
%:%59=11%:%
%:%60=12%:%
%:%61=12%:%
%:%62=13%:%
%:%63=14%:%
%:%64=15%:%
%:%65=16%:%
%:%66=16%:%
%:%67=17%:%
%:%68=18%:%
%:%71=19%:%
%:%75=19%:%
%:%76=19%:%
%:%77=20%:%
%:%78=20%:%
%:%79=21%:%
%:%80=21%:%
%:%85=21%:%
%:%88=22%:%
%:%89=23%:%
%:%90=23%:%
%:%91=24%:%
%:%92=25%:%
%:%95=26%:%
%:%99=26%:%
%:%100=26%:%
%:%101=27%:%
%:%102=27%:%
%:%103=28%:%
%:%104=28%:%
%:%109=28%:%
%:%112=29%:%
%:%113=30%:%
%:%114=30%:%
%:%115=31%:%
%:%116=32%:%
%:%117=33%:%
%:%118=34%:%
%:%119=35%:%
%:%122=36%:%
%:%126=36%:%
%:%127=36%:%
%:%128=37%:%
%:%129=37%:%
%:%130=38%:%
%:%135=38%:%
%:%140=39%:%
%:%145=40%:%
%:%146=40%:%
%:%151=40%:%
%:%154=41%:%
%:%155=42%:%
%:%156=42%:%
%:%157=43%:%
%:%158=44%:%
%:%159=45%:%
%:%160=45%:%
%:%161=46%:%
%:%162=47%:%
%:%163=48%:%
%:%164=49%:%
%:%165=50%:%
%:%166=50%:%
%:%167=51%:%
%:%170=52%:%
%:%174=52%:%
%:%175=52%:%
%:%176=53%:%
%:%177=53%:%
%:%178=54%:%
%:%179=54%:%
%:%184=54%:%
%:%187=55%:%
%:%188=56%:%
%:%189=56%:%
%:%192=57%:%
%:%196=57%:%
%:%197=57%:%
%:%198=58%:%
%:%199=58%:%
%:%200=59%:%
%:%201=59%:%
%:%206=59%:%
%:%209=60%:%
%:%210=61%:%
%:%211=61%:%
%:%212=62%:%
%:%213=63%:%
%:%216=64%:%
%:%220=64%:%
%:%221=64%:%
%:%222=65%:%
%:%223=65%:%
%:%224=66%:%
%:%229=66%:%
%:%232=67%:%
%:%235=68%:%
%:%240=69%:%
%:%241=69%:%
%:%246=69%:%
%:%249=70%:%
%:%250=71%:%
%:%251=71%:%
%:%252=72%:%
%:%253=73%:%
%:%254=74%:%
%:%255=74%:%
%:%256=75%:%
%:%257=76%:%
%:%258=77%:%
%:%259=78%:%
%:%260=78%:%
%:%261=79%:%
%:%264=80%:%
%:%268=80%:%
%:%269=80%:%
%:%270=81%:%
%:%271=81%:%
%:%272=82%:%
%:%273=82%:%
%:%278=82%:%
%:%281=83%:%
%:%282=84%:%
%:%283=84%:%
%:%284=85%:%
%:%287=86%:%
%:%291=86%:%
%:%292=86%:%
%:%293=87%:%
%:%294=87%:%
%:%295=88%:%
%:%296=88%:%
%:%301=88%:%
%:%304=89%:%
%:%305=90%:%
%:%306=90%:%
%:%307=91%:%
%:%308=92%:%
%:%311=93%:%
%:%315=93%:%
%:%316=93%:%
%:%317=94%:%
%:%318=94%:%
%:%319=95%:%
%:%324=95%:%
%:%327=96%:%
%:%328=97%:%
%:%329=97%:%
%:%330=98%:%
%:%331=99%:%
%:%332=100%:%
%:%333=100%:%
%:%334=101%:%
%:%335=102%:%
%:%336=103%:%
%:%337=104%:%
%:%340=105%:%
%:%344=105%:%
%:%345=105%:%
%:%346=105%:%
%:%351=105%:%
%:%354=106%:%
%:%355=107%:%
%:%356=108%:%
%:%357=108%:%
%:%358=109%:%
%:%359=110%:%
%:%360=111%:%
%:%361=112%:%
%:%362=113%:%
%:%365=114%:%
%:%369=114%:%
%:%370=114%:%
%:%371=115%:%
%:%372=115%:%
%:%373=116%:%
%:%374=116%:%
%:%375=117%:%
%:%376=117%:%
%:%377=118%:%
%:%378=118%:%
%:%379=119%:%
%:%380=119%:%
%:%381=120%:%
%:%382=120%:%
%:%383=121%:%
%:%384=121%:%
%:%385=122%:%
%:%386=122%:%
%:%387=123%:%
%:%388=123%:%
%:%389=124%:%
%:%390=124%:%
%:%391=125%:%
%:%392=125%:%
%:%393=126%:%
%:%394=126%:%
%:%395=127%:%
%:%396=127%:%
%:%397=128%:%
%:%398=128%:%
%:%399=129%:%
%:%400=129%:%
%:%401=130%:%
%:%402=130%:%
%:%403=131%:%
%:%404=131%:%
%:%405=131%:%
%:%406=132%:%
%:%407=132%:%
%:%408=133%:%
%:%409=133%:%
%:%410=134%:%
%:%411=134%:%
%:%412=134%:%
%:%413=135%:%
%:%419=135%:%
%:%422=136%:%
%:%425=137%:%
%:%430=138%:%
%:%431=138%:%
%:%436=138%:%
%:%439=139%:%
%:%440=140%:%
%:%441=140%:%
%:%442=141%:%
%:%443=142%:%
%:%444=143%:%
%:%451=144%:%
%:%452=144%:%
%:%453=145%:%
%:%454=145%:%
%:%455=146%:%
%:%456=146%:%
%:%457=146%:%
%:%458=147%:%
%:%459=147%:%
%:%460=148%:%
%:%461=148%:%
%:%462=149%:%
%:%463=149%:%
%:%464=149%:%
%:%465=150%:%
%:%466=150%:%
%:%467=150%:%
%:%468=150%:%
%:%469=150%:%
%:%470=151%:%
%:%476=151%:%
%:%479=152%:%
%:%480=153%:%
%:%481=153%:%
%:%482=154%:%
%:%483=155%:%
%:%484=156%:%
%:%485=157%:%
%:%486=157%:%
%:%487=158%:%
%:%488=159%:%
%:%489=160%:%
%:%490=161%:%
%:%491=162%:%
%:%494=163%:%
%:%498=163%:%
%:%499=163%:%
%:%500=163%:%
%:%501=163%:%
%:%506=163%:%
%:%509=164%:%
%:%510=165%:%
%:%511=166%:%
%:%512=166%:%
%:%513=167%:%
%:%514=168%:%
%:%515=169%:%
%:%516=170%:%
%:%519=171%:%
%:%523=171%:%
%:%524=171%:%
%:%525=172%:%
%:%526=172%:%
%:%527=173%:%
%:%528=173%:%
%:%533=173%:%
%:%536=174%:%
%:%537=175%:%
%:%538=176%:%
%:%539=176%:%
%:%540=177%:%
%:%541=178%:%
%:%542=179%:%
%:%543=180%:%
%:%544=180%:%
%:%547=181%:%
%:%551=181%:%
%:%552=181%:%
%:%553=181%:%
%:%554=182%:%
%:%555=182%:%
%:%560=182%:%
%:%563=183%:%
%:%564=184%:%
%:%565=184%:%
%:%566=185%:%
%:%567=186%:%
%:%568=187%:%
%:%569=188%:%
%:%572=189%:%
%:%576=189%:%
%:%577=189%:%
%:%578=190%:%
%:%579=190%:%
%:%580=191%:%
%:%581=191%:%
%:%586=191%:%
%:%589=192%:%
%:%590=193%:%
%:%591=193%:%
%:%592=194%:%
%:%593=195%:%
%:%594=196%:%
%:%595=197%:%
%:%596=198%:%
%:%599=199%:%
%:%603=199%:%
%:%604=199%:%
%:%605=199%:%
%:%606=200%:%
%:%607=200%:%
%:%612=200%:%
%:%615=201%:%
%:%616=202%:%
%:%617=202%:%
%:%618=203%:%
%:%619=204%:%
%:%620=205%:%
%:%621=206%:%
%:%622=206%:%
%:%625=207%:%
%:%629=207%:%
%:%630=207%:%
%:%631=207%:%
%:%636=207%:%
%:%639=208%:%
%:%640=209%:%
%:%641=209%:%
%:%642=210%:%
%:%643=211%:%
%:%644=212%:%
%:%645=213%:%
%:%646=214%:%
%:%647=215%:%
%:%650=216%:%
%:%654=216%:%
%:%655=216%:%
%:%656=217%:%
%:%657=217%:%
%:%658=218%:%
%:%659=218%:%
%:%664=218%:%
%:%667=219%:%
%:%668=220%:%
%:%669=220%:%
%:%670=221%:%
%:%671=222%:%
%:%672=223%:%
%:%673=224%:%
%:%674=225%:%
%:%675=226%:%
%:%678=227%:%
%:%682=227%:%
%:%683=227%:%
%:%684=228%:%
%:%685=229%:%
%:%686=229%:%
%:%691=229%:%
%:%694=230%:%
%:%695=231%:%
%:%696=231%:%
%:%697=232%:%
%:%698=233%:%
%:%699=234%:%
%:%700=235%:%
%:%701=236%:%
%:%702=237%:%
%:%705=238%:%
%:%709=238%:%
%:%710=238%:%
%:%711=239%:%
%:%712=240%:%
%:%713=240%:%
%:%718=240%:%
%:%721=241%:%
%:%722=242%:%
%:%723=242%:%
%:%724=243%:%
%:%725=244%:%
%:%726=245%:%
%:%727=246%:%
%:%728=247%:%
%:%729=248%:%
%:%736=249%:%
%:%737=249%:%
%:%738=250%:%
%:%739=250%:%
%:%740=251%:%
%:%741=251%:%
%:%742=252%:%
%:%743=252%:%
%:%744=253%:%
%:%745=253%:%
%:%746=253%:%
%:%747=254%:%
%:%748=254%:%
%:%749=255%:%
%:%750=255%:%
%:%751=256%:%
%:%752=256%:%
%:%753=256%:%
%:%754=257%:%
%:%755=257%:%
%:%756=258%:%
%:%757=258%:%
%:%758=259%:%
%:%759=259%:%
%:%760=260%:%
%:%761=260%:%
%:%762=261%:%
%:%763=261%:%
%:%764=262%:%
%:%765=262%:%
%:%766=263%:%
%:%767=263%:%
%:%768=263%:%
%:%769=264%:%
%:%770=264%:%
%:%771=265%:%
%:%772=265%:%
%:%773=265%:%
%:%774=266%:%
%:%775=266%:%
%:%776=267%:%
%:%777=268%:%
%:%778=268%:%
%:%779=269%:%
%:%780=270%:%
%:%781=270%:%
%:%782=270%:%
%:%783=271%:%
%:%784=271%:%
%:%785=272%:%
%:%786=272%:%
%:%787=273%:%
%:%788=273%:%
%:%789=274%:%
%:%790=274%:%
%:%791=274%:%
%:%792=274%:%
%:%793=275%:%
%:%794=275%:%
%:%795=276%:%
%:%796=276%:%
%:%797=277%:%
%:%798=277%:%
%:%799=278%:%
%:%800=278%:%
%:%801=279%:%
%:%802=279%:%
%:%803=279%:%
%:%804=279%:%
%:%805=280%:%
%:%806=280%:%
%:%807=281%:%
%:%808=281%:%
%:%809=282%:%
%:%810=282%:%
%:%811=283%:%
%:%812=283%:%
%:%813=283%:%
%:%814=284%:%
%:%815=284%:%
%:%816=285%:%
%:%817=285%:%
%:%818=286%:%
%:%819=286%:%
%:%820=287%:%
%:%821=287%:%
%:%822=288%:%
%:%823=288%:%
%:%824=289%:%
%:%825=289%:%
%:%826=290%:%
%:%827=290%:%
%:%828=291%:%
%:%829=291%:%
%:%830=292%:%
%:%831=292%:%
%:%832=293%:%
%:%833=293%:%
%:%834=294%:%
%:%835=295%:%
%:%836=295%:%
%:%837=295%:%
%:%838=296%:%
%:%839=296%:%
%:%840=297%:%
%:%841=297%:%
%:%842=298%:%
%:%843=298%:%
%:%844=299%:%
%:%845=299%:%
%:%846=300%:%
%:%847=300%:%
%:%848=301%:%
%:%849=301%:%
%:%850=302%:%
%:%851=302%:%
%:%853=304%:%
%:%854=304%:%
%:%855=305%:%
%:%856=305%:%
%:%857=306%:%
%:%858=306%:%
%:%859=307%:%
%:%860=307%:%
%:%861=307%:%
%:%862=308%:%
%:%863=308%:%
%:%864=309%:%
%:%865=309%:%
%:%866=310%:%
%:%867=310%:%
%:%868=311%:%
%:%869=311%:%
%:%870=312%:%
%:%871=312%:%
%:%872=312%:%
%:%873=313%:%
%:%874=313%:%
%:%875=314%:%
%:%876=314%:%
%:%877=315%:%
%:%878=315%:%
%:%879=315%:%
%:%880=316%:%
%:%881=316%:%
%:%882=316%:%
%:%883=317%:%
%:%884=317%:%
%:%885=317%:%
%:%886=318%:%
%:%887=318%:%
%:%888=318%:%
%:%889=319%:%
%:%890=320%:%
%:%891=320%:%
%:%892=321%:%
%:%893=321%:%
%:%894=322%:%
%:%895=322%:%
%:%896=323%:%
%:%897=323%:%
%:%898=324%:%
%:%899=325%:%
%:%900=325%:%
%:%901=325%:%
%:%902=326%:%
%:%903=326%:%
%:%904=326%:%
%:%905=327%:%
%:%906=327%:%
%:%907=328%:%
%:%908=328%:%
%:%909=329%:%
%:%910=329%:%
%:%911=330%:%
%:%912=330%:%
%:%913=331%:%
%:%914=332%:%
%:%915=332%:%
%:%916=333%:%
%:%917=333%:%
%:%918=333%:%
%:%919=334%:%
%:%920=334%:%
%:%921=335%:%
%:%922=336%:%
%:%923=336%:%
%:%924=337%:%
%:%925=337%:%
%:%926=338%:%
%:%927=338%:%
%:%928=339%:%
%:%929=339%:%
%:%930=340%:%
%:%931=340%:%
%:%932=341%:%
%:%933=341%:%
%:%934=342%:%
%:%935=343%:%
%:%936=343%:%
%:%937=344%:%
%:%938=344%:%
%:%939=345%:%
%:%940=345%:%
%:%941=346%:%
%:%942=346%:%
%:%943=346%:%
%:%944=347%:%
%:%945=347%:%
%:%946=348%:%
%:%947=348%:%
%:%948=349%:%
%:%949=349%:%
%:%950=349%:%
%:%951=350%:%
%:%952=350%:%
%:%953=351%:%
%:%954=351%:%
%:%955=352%:%
%:%956=352%:%
%:%957=352%:%
%:%958=353%:%
%:%959=353%:%
%:%960=354%:%
%:%961=354%:%
%:%962=355%:%
%:%963=355%:%
%:%964=355%:%
%:%965=356%:%
%:%966=356%:%
%:%967=357%:%
%:%968=357%:%
%:%969=358%:%
%:%970=358%:%
%:%971=359%:%
%:%972=359%:%
%:%973=360%:%
%:%974=360%:%
%:%975=361%:%
%:%976=361%:%
%:%977=362%:%
%:%978=362%:%
%:%979=363%:%
%:%980=363%:%
%:%981=364%:%
%:%982=364%:%
%:%983=365%:%
%:%984=365%:%
%:%985=366%:%
%:%986=366%:%
%:%987=367%:%
%:%988=367%:%
%:%989=367%:%
%:%990=368%:%
%:%991=368%:%
%:%992=369%:%
%:%993=369%:%
%:%994=370%:%
%:%995=370%:%
%:%996=371%:%
%:%997=371%:%
%:%998=372%:%
%:%999=372%:%
%:%1000=372%:%
%:%1001=373%:%
%:%1002=373%:%
%:%1003=374%:%
%:%1004=374%:%
%:%1005=375%:%
%:%1006=375%:%
%:%1007=376%:%
%:%1008=376%:%
%:%1009=377%:%
%:%1010=377%:%
%:%1011=378%:%
%:%1012=378%:%
%:%1013=379%:%
%:%1014=379%:%
%:%1015=380%:%
%:%1016=380%:%
%:%1017=381%:%
%:%1018=381%:%
%:%1019=382%:%
%:%1020=382%:%
%:%1021=383%:%
%:%1022=383%:%
%:%1023=383%:%
%:%1024=383%:%
%:%1025=384%:%
%:%1026=384%:%
%:%1027=385%:%
%:%1028=385%:%
%:%1029=386%:%
%:%1030=386%:%
%:%1031=386%:%
%:%1032=387%:%
%:%1033=387%:%
%:%1034=388%:%
%:%1035=388%:%
%:%1036=388%:%
%:%1037=389%:%
%:%1038=389%:%
%:%1039=390%:%
%:%1040=390%:%
%:%1041=391%:%
%:%1042=391%:%
%:%1043=391%:%
%:%1044=392%:%
%:%1045=392%:%
%:%1046=393%:%
%:%1047=393%:%
%:%1048=394%:%
%:%1049=394%:%
%:%1050=395%:%
%:%1051=395%:%
%:%1052=395%:%
%:%1053=396%:%
%:%1054=396%:%
%:%1055=397%:%
%:%1056=397%:%
%:%1057=398%:%
%:%1058=398%:%
%:%1059=398%:%
%:%1060=399%:%
%:%1061=399%:%
%:%1062=400%:%
%:%1063=400%:%
%:%1064=401%:%
%:%1065=401%:%
%:%1066=401%:%
%:%1067=402%:%
%:%1068=402%:%
%:%1069=403%:%
%:%1070=403%:%
%:%1071=404%:%
%:%1072=404%:%
%:%1073=404%:%
%:%1074=405%:%
%:%1075=405%:%
%:%1076=405%:%
%:%1077=406%:%
%:%1078=406%:%
%:%1079=406%:%
%:%1080=406%:%
%:%1081=407%:%
%:%1082=407%:%
%:%1083=408%:%
%:%1084=408%:%
%:%1085=409%:%
%:%1086=410%:%
%:%1087=410%:%
%:%1088=411%:%
%:%1089=411%:%
%:%1090=412%:%
%:%1091=412%:%
%:%1092=412%:%
%:%1093=412%:%
%:%1094=413%:%
%:%1095=413%:%
%:%1096=414%:%
%:%1097=414%:%
%:%1098=415%:%
%:%1099=415%:%
%:%1100=415%:%
%:%1101=416%:%
%:%1102=416%:%
%:%1103=417%:%
%:%1104=417%:%
%:%1105=418%:%
%:%1106=418%:%
%:%1107=419%:%
%:%1108=419%:%
%:%1109=420%:%
%:%1110=420%:%
%:%1111=421%:%
%:%1112=421%:%
%:%1113=422%:%
%:%1114=422%:%
%:%1115=423%:%
%:%1116=423%:%
%:%1117=424%:%
%:%1118=424%:%
%:%1119=425%:%
%:%1120=425%:%
%:%1121=426%:%
%:%1122=426%:%
%:%1123=427%:%
%:%1124=427%:%
%:%1125=428%:%
%:%1126=428%:%
%:%1127=429%:%
%:%1128=429%:%
%:%1129=430%:%
%:%1130=430%:%
%:%1131=431%:%
%:%1132=431%:%
%:%1133=432%:%
%:%1134=432%:%
%:%1135=433%:%
%:%1136=433%:%
%:%1137=434%:%
%:%1138=434%:%
%:%1139=434%:%
%:%1140=435%:%
%:%1141=435%:%
%:%1142=436%:%
%:%1143=436%:%
%:%1144=437%:%
%:%1145=437%:%
%:%1146=438%:%
%:%1147=438%:%
%:%1148=438%:%
%:%1149=439%:%
%:%1150=439%:%
%:%1151=440%:%
%:%1152=440%:%
%:%1153=441%:%
%:%1154=441%:%
%:%1155=442%:%
%:%1156=442%:%
%:%1157=443%:%
%:%1158=443%:%
%:%1159=444%:%
%:%1160=445%:%
%:%1161=445%:%
%:%1162=446%:%
%:%1163=446%:%
%:%1164=446%:%
%:%1165=447%:%
%:%1166=447%:%
%:%1167=448%:%
%:%1168=448%:%
%:%1169=448%:%
%:%1170=449%:%
%:%1171=449%:%
%:%1172=449%:%
%:%1173=450%:%
%:%1174=450%:%
%:%1175=451%:%
%:%1176=451%:%
%:%1177=451%:%
%:%1178=452%:%
%:%1179=452%:%
%:%1180=453%:%
%:%1181=453%:%
%:%1182=454%:%
%:%1183=454%:%
%:%1184=455%:%
%:%1185=455%:%
%:%1186=456%:%
%:%1187=456%:%
%:%1188=457%:%
%:%1189=457%:%
%:%1190=458%:%
%:%1191=458%:%
%:%1192=459%:%
%:%1193=459%:%
%:%1194=460%:%
%:%1195=461%:%
%:%1196=462%:%
%:%1197=462%:%
%:%1198=463%:%
%:%1199=463%:%
%:%1200=464%:%
%:%1201=464%:%
%:%1202=465%:%
%:%1203=465%:%
%:%1204=466%:%
%:%1205=466%:%
%:%1206=467%:%
%:%1207=467%:%
%:%1208=468%:%
%:%1209=468%:%
%:%1210=469%:%
%:%1211=469%:%
%:%1212=470%:%
%:%1213=470%:%
%:%1214=471%:%
%:%1215=471%:%
%:%1216=472%:%
%:%1217=472%:%
%:%1218=473%:%
%:%1219=473%:%
%:%1220=473%:%
%:%1221=474%:%
%:%1222=475%:%
%:%1223=475%:%
%:%1224=476%:%
%:%1225=476%:%
%:%1226=477%:%
%:%1227=477%:%
%:%1228=478%:%
%:%1229=478%:%
%:%1230=479%:%
%:%1231=479%:%
%:%1232=480%:%
%:%1233=480%:%
%:%1234=481%:%
%:%1235=481%:%
%:%1236=482%:%
%:%1237=482%:%
%:%1238=483%:%
%:%1239=483%:%
%:%1240=484%:%
%:%1241=484%:%
%:%1242=485%:%
%:%1243=486%:%
%:%1244=486%:%
%:%1245=487%:%
%:%1246=487%:%
%:%1247=488%:%
%:%1248=488%:%
%:%1249=489%:%
%:%1250=489%:%
%:%1251=490%:%
%:%1252=490%:%
%:%1253=491%:%
%:%1254=491%:%
%:%1255=492%:%
%:%1256=492%:%
%:%1257=493%:%
%:%1258=493%:%
%:%1259=494%:%
%:%1260=494%:%
%:%1261=495%:%
%:%1262=495%:%
%:%1263=496%:%
%:%1264=496%:%
%:%1265=497%:%
%:%1266=497%:%
%:%1267=498%:%
%:%1268=498%:%
%:%1269=499%:%
%:%1270=499%:%
%:%1271=499%:%
%:%1272=500%:%
%:%1273=500%:%
%:%1274=501%:%
%:%1275=501%:%
%:%1276=501%:%
%:%1277=502%:%
%:%1278=502%:%
%:%1279=503%:%
%:%1280=503%:%
%:%1281=504%:%
%:%1282=504%:%
%:%1283=505%:%
%:%1284=505%:%
%:%1285=505%:%
%:%1286=506%:%
%:%1287=506%:%
%:%1288=507%:%
%:%1289=507%:%
%:%1290=507%:%
%:%1291=508%:%
%:%1292=508%:%
%:%1293=509%:%
%:%1294=509%:%
%:%1295=509%:%
%:%1296=510%:%
%:%1302=510%:%
%:%1305=511%:%
%:%1306=512%:%
%:%1307=512%:%
%:%1308=513%:%
%:%1309=514%:%
%:%1310=515%:%
%:%1311=516%:%
%:%1318=517%:%
%:%1319=517%:%
%:%1320=518%:%
%:%1321=518%:%
%:%1322=519%:%
%:%1323=519%:%
%:%1324=520%:%
%:%1325=520%:%
%:%1326=521%:%
%:%1327=521%:%
%:%1328=522%:%
%:%1329=522%:%
%:%1330=523%:%
%:%1331=524%:%
%:%1332=524%:%
%:%1333=525%:%
%:%1334=525%:%
%:%1335=526%:%
%:%1336=526%:%
%:%1337=526%:%
%:%1338=527%:%
%:%1339=527%:%
%:%1340=528%:%
%:%1341=528%:%
%:%1342=529%:%
%:%1343=529%:%
%:%1344=529%:%
%:%1345=530%:%
%:%1346=530%:%
%:%1347=531%:%
%:%1348=531%:%
%:%1349=532%:%
%:%1350=532%:%
%:%1351=533%:%
%:%1352=533%:%
%:%1353=534%:%
%:%1354=534%:%
%:%1355=534%:%
%:%1356=535%:%
%:%1357=535%:%
%:%1358=536%:%
%:%1359=536%:%
%:%1360=537%:%
%:%1361=537%:%
%:%1362=538%:%
%:%1363=538%:%
%:%1364=539%:%
%:%1365=539%:%
%:%1366=539%:%
%:%1367=540%:%
%:%1368=540%:%
%:%1369=541%:%
%:%1370=541%:%
%:%1371=542%:%
%:%1372=542%:%
%:%1373=543%:%
%:%1374=543%:%
%:%1375=543%:%
%:%1376=544%:%
%:%1377=544%:%
%:%1378=545%:%
%:%1379=545%:%
%:%1380=546%:%
%:%1381=546%:%
%:%1382=547%:%
%:%1383=547%:%
%:%1384=548%:%
%:%1385=548%:%
%:%1386=548%:%
%:%1387=549%:%
%:%1388=549%:%
%:%1389=550%:%
%:%1390=550%:%
%:%1391=551%:%
%:%1392=551%:%
%:%1393=552%:%
%:%1394=552%:%
%:%1395=553%:%
%:%1396=553%:%
%:%1397=554%:%
%:%1398=554%:%
%:%1399=555%:%
%:%1400=555%:%
%:%1401=556%:%
%:%1402=556%:%
%:%1403=557%:%
%:%1404=557%:%
%:%1405=558%:%
%:%1406=558%:%
%:%1407=558%:%
%:%1408=558%:%
%:%1409=559%:%
%:%1410=559%:%
%:%1411=560%:%
%:%1417=560%:%
%:%1420=561%:%
%:%1421=562%:%
%:%1422=562%:%
%:%1425=563%:%
%:%1430=564%:%

%
\begin{isabellebody}%
\setisabellecontext{Infinity{\isacharunderscore}{\kern0pt}Axiom}%
%
\isadelimdocument
%
\endisadelimdocument
%
\isatagdocument
%
\isamarkupsection{The Axiom of Infinity in $M[G]$%
}
\isamarkuptrue%
%
\endisatagdocument
{\isafolddocument}%
%
\isadelimdocument
%
\endisadelimdocument
%
\isadelimtheory
%
\endisadelimtheory
%
\isatagtheory
\isacommand{theory}\isamarkupfalse%
\ Infinity{\isacharunderscore}{\kern0pt}Axiom\isanewline
\ \ \isakeyword{imports}\ Pairing{\isacharunderscore}{\kern0pt}Axiom\ Union{\isacharunderscore}{\kern0pt}Axiom\ Separation{\isacharunderscore}{\kern0pt}Axiom\isanewline
\isakeyword{begin}%
\endisatagtheory
{\isafoldtheory}%
%
\isadelimtheory
\isanewline
%
\endisadelimtheory
\isanewline
\isacommand{context}\isamarkupfalse%
\ G{\isacharunderscore}{\kern0pt}generic\ \isakeyword{begin}\isanewline
\isanewline
\isacommand{interpretation}\isamarkupfalse%
\ mg{\isacharunderscore}{\kern0pt}triv{\isacharcolon}{\kern0pt}\ M{\isacharunderscore}{\kern0pt}trivial{\isachardoublequoteopen}{\isacharhash}{\kern0pt}{\isacharhash}{\kern0pt}M{\isacharbrackleft}{\kern0pt}G{\isacharbrackright}{\kern0pt}{\isachardoublequoteclose}\isanewline
%
\isadelimproof
\ \ %
\endisadelimproof
%
\isatagproof
\isacommand{using}\isamarkupfalse%
\ transitivity{\isacharunderscore}{\kern0pt}MG\ zero{\isacharunderscore}{\kern0pt}in{\isacharunderscore}{\kern0pt}MG\ generic\ Union{\isacharunderscore}{\kern0pt}MG\ pairing{\isacharunderscore}{\kern0pt}in{\isacharunderscore}{\kern0pt}MG\isanewline
\ \ \isacommand{by}\isamarkupfalse%
\ unfold{\isacharunderscore}{\kern0pt}locales\ auto%
\endisatagproof
{\isafoldproof}%
%
\isadelimproof
\isanewline
%
\endisadelimproof
\isanewline
\isacommand{lemma}\isamarkupfalse%
\ infinity{\isacharunderscore}{\kern0pt}in{\isacharunderscore}{\kern0pt}MG\ {\isacharcolon}{\kern0pt}\ {\isachardoublequoteopen}infinity{\isacharunderscore}{\kern0pt}ax{\isacharparenleft}{\kern0pt}{\isacharhash}{\kern0pt}{\isacharhash}{\kern0pt}M{\isacharbrackleft}{\kern0pt}G{\isacharbrackright}{\kern0pt}{\isacharparenright}{\kern0pt}{\isachardoublequoteclose}\isanewline
%
\isadelimproof
%
\endisadelimproof
%
\isatagproof
\isacommand{proof}\isamarkupfalse%
\ {\isacharminus}{\kern0pt}\isanewline
\ \ \isacommand{from}\isamarkupfalse%
\ infinity{\isacharunderscore}{\kern0pt}ax\ \isacommand{obtain}\isamarkupfalse%
\ I\ \isakeyword{where}\isanewline
\ \ \ \ Eq{\isadigit{1}}{\isacharcolon}{\kern0pt}\ {\isachardoublequoteopen}I{\isasymin}M{\isachardoublequoteclose}\ {\isachardoublequoteopen}{\isadigit{0}}\ {\isasymin}\ I{\isachardoublequoteclose}\ {\isachardoublequoteopen}{\isasymforall}y{\isasymin}M{\isachardot}{\kern0pt}\ y\ {\isasymin}\ I\ {\isasymlongrightarrow}\ succ{\isacharparenleft}{\kern0pt}y{\isacharparenright}{\kern0pt}\ {\isasymin}\ I{\isachardoublequoteclose}\isanewline
\ \ \ \ \isacommand{unfolding}\isamarkupfalse%
\ infinity{\isacharunderscore}{\kern0pt}ax{\isacharunderscore}{\kern0pt}def\ \ \isacommand{by}\isamarkupfalse%
\ auto\isanewline
\ \ \isacommand{then}\isamarkupfalse%
\isanewline
\ \ \isacommand{have}\isamarkupfalse%
\ {\isachardoublequoteopen}check{\isacharparenleft}{\kern0pt}I{\isacharparenright}{\kern0pt}\ {\isasymin}\ M{\isachardoublequoteclose}\isanewline
\ \ \ \ \isacommand{using}\isamarkupfalse%
\ check{\isacharunderscore}{\kern0pt}in{\isacharunderscore}{\kern0pt}M\ \isacommand{by}\isamarkupfalse%
\ simp\isanewline
\ \ \isacommand{then}\isamarkupfalse%
\isanewline
\ \ \isacommand{have}\isamarkupfalse%
\ {\isachardoublequoteopen}I{\isasymin}\ M{\isacharbrackleft}{\kern0pt}G{\isacharbrackright}{\kern0pt}{\isachardoublequoteclose}\isanewline
\ \ \ \ \isacommand{using}\isamarkupfalse%
\ valcheck\ generic\ one{\isacharunderscore}{\kern0pt}in{\isacharunderscore}{\kern0pt}G\ one{\isacharunderscore}{\kern0pt}in{\isacharunderscore}{\kern0pt}P\ GenExtI{\isacharbrackleft}{\kern0pt}of\ {\isachardoublequoteopen}check{\isacharparenleft}{\kern0pt}I{\isacharparenright}{\kern0pt}{\isachardoublequoteclose}\ G{\isacharbrackright}{\kern0pt}\ \isacommand{by}\isamarkupfalse%
\ simp\isanewline
\ \ \isacommand{with}\isamarkupfalse%
\ {\isacartoucheopen}{\isadigit{0}}{\isasymin}I{\isacartoucheclose}\isanewline
\ \ \isacommand{have}\isamarkupfalse%
\ {\isachardoublequoteopen}{\isadigit{0}}{\isasymin}M{\isacharbrackleft}{\kern0pt}G{\isacharbrackright}{\kern0pt}{\isachardoublequoteclose}\ \isacommand{using}\isamarkupfalse%
\ transitivity{\isacharunderscore}{\kern0pt}MG\ \isacommand{by}\isamarkupfalse%
\ simp\isanewline
\ \ \isacommand{with}\isamarkupfalse%
\ {\isacartoucheopen}I{\isasymin}M{\isacartoucheclose}\isanewline
\ \ \isacommand{have}\isamarkupfalse%
\ {\isachardoublequoteopen}y\ {\isasymin}\ M{\isachardoublequoteclose}\ \isakeyword{if}\ {\isachardoublequoteopen}y\ {\isasymin}\ I{\isachardoublequoteclose}\ \isakeyword{for}\ y\isanewline
\ \ \ \ \isacommand{using}\isamarkupfalse%
\ \ transitivity{\isacharbrackleft}{\kern0pt}OF\ {\isacharunderscore}{\kern0pt}\ {\isacartoucheopen}I{\isasymin}M{\isacartoucheclose}{\isacharbrackright}{\kern0pt}\ that\ \isacommand{by}\isamarkupfalse%
\ simp\isanewline
\ \ \isacommand{with}\isamarkupfalse%
\ {\isacartoucheopen}I{\isasymin}M{\isacharbrackleft}{\kern0pt}G{\isacharbrackright}{\kern0pt}{\isacartoucheclose}\isanewline
\ \ \isacommand{have}\isamarkupfalse%
\ {\isachardoublequoteopen}succ{\isacharparenleft}{\kern0pt}y{\isacharparenright}{\kern0pt}\ {\isasymin}\ I\ {\isasyminter}\ M{\isacharbrackleft}{\kern0pt}G{\isacharbrackright}{\kern0pt}{\isachardoublequoteclose}\ \isakeyword{if}\ {\isachardoublequoteopen}y\ {\isasymin}\ I{\isachardoublequoteclose}\ \isakeyword{for}\ y\isanewline
\ \ \ \ \isacommand{using}\isamarkupfalse%
\ that\ Eq{\isadigit{1}}\ transitivity{\isacharunderscore}{\kern0pt}MG\ \isacommand{by}\isamarkupfalse%
\ blast\isanewline
\ \ \isacommand{with}\isamarkupfalse%
\ Eq{\isadigit{1}}\ {\isacartoucheopen}I{\isasymin}M{\isacharbrackleft}{\kern0pt}G{\isacharbrackright}{\kern0pt}{\isacartoucheclose}\ {\isacartoucheopen}{\isadigit{0}}{\isasymin}M{\isacharbrackleft}{\kern0pt}G{\isacharbrackright}{\kern0pt}{\isacartoucheclose}\isanewline
\ \ \isacommand{show}\isamarkupfalse%
\ {\isacharquery}{\kern0pt}thesis\isanewline
\ \ \ \ \isacommand{unfolding}\isamarkupfalse%
\ infinity{\isacharunderscore}{\kern0pt}ax{\isacharunderscore}{\kern0pt}def\ \isacommand{by}\isamarkupfalse%
\ auto\isanewline
\isacommand{qed}\isamarkupfalse%
%
\endisatagproof
{\isafoldproof}%
%
\isadelimproof
\isanewline
%
\endisadelimproof
\isanewline
\isacommand{end}\isamarkupfalse%
\ \isanewline
%
\isadelimtheory
%
\endisadelimtheory
%
\isatagtheory
\isacommand{end}\isamarkupfalse%
%
\endisatagtheory
{\isafoldtheory}%
%
\isadelimtheory
%
\endisadelimtheory
%
\end{isabellebody}%
\endinput
%:%file=~/source/repos/ZF-notAC/code/Forcing/Infinity_Axiom.thy%:%
%:%11=1%:%
%:%27=2%:%
%:%28=2%:%
%:%29=3%:%
%:%30=4%:%
%:%35=4%:%
%:%38=5%:%
%:%39=6%:%
%:%40=6%:%
%:%41=7%:%
%:%42=8%:%
%:%43=8%:%
%:%46=9%:%
%:%50=9%:%
%:%51=9%:%
%:%52=10%:%
%:%53=10%:%
%:%58=10%:%
%:%61=11%:%
%:%62=12%:%
%:%63=12%:%
%:%70=13%:%
%:%71=13%:%
%:%72=14%:%
%:%73=14%:%
%:%74=14%:%
%:%75=15%:%
%:%76=16%:%
%:%77=16%:%
%:%78=16%:%
%:%79=17%:%
%:%80=17%:%
%:%81=18%:%
%:%82=18%:%
%:%83=19%:%
%:%84=19%:%
%:%85=19%:%
%:%86=20%:%
%:%87=20%:%
%:%88=21%:%
%:%89=21%:%
%:%90=22%:%
%:%91=22%:%
%:%92=22%:%
%:%93=23%:%
%:%94=23%:%
%:%95=24%:%
%:%96=24%:%
%:%97=24%:%
%:%98=24%:%
%:%99=25%:%
%:%100=25%:%
%:%101=26%:%
%:%102=26%:%
%:%103=27%:%
%:%104=27%:%
%:%105=27%:%
%:%106=28%:%
%:%107=28%:%
%:%108=29%:%
%:%109=29%:%
%:%110=30%:%
%:%111=30%:%
%:%112=30%:%
%:%113=31%:%
%:%114=31%:%
%:%115=32%:%
%:%116=32%:%
%:%117=33%:%
%:%118=33%:%
%:%119=33%:%
%:%120=34%:%
%:%126=34%:%
%:%129=35%:%
%:%130=36%:%
%:%131=36%:%
%:%138=37%:%

%
\begin{isabellebody}%
\setisabellecontext{Choice{\isacharunderscore}{\kern0pt}Axiom}%
%
\isadelimdocument
%
\endisadelimdocument
%
\isatagdocument
%
\isamarkupsection{The Axiom of Choice in $M[G]$%
}
\isamarkuptrue%
%
\endisatagdocument
{\isafolddocument}%
%
\isadelimdocument
%
\endisadelimdocument
%
\isadelimtheory
%
\endisadelimtheory
%
\isatagtheory
\isacommand{theory}\isamarkupfalse%
\ Choice{\isacharunderscore}{\kern0pt}Axiom\isanewline
\ \ \isakeyword{imports}\ Powerset{\isacharunderscore}{\kern0pt}Axiom\ Pairing{\isacharunderscore}{\kern0pt}Axiom\ Union{\isacharunderscore}{\kern0pt}Axiom\ Extensionality{\isacharunderscore}{\kern0pt}Axiom\ \isanewline
\ \ \ \ \ \ \ \ \ \ Foundation{\isacharunderscore}{\kern0pt}Axiom\ Powerset{\isacharunderscore}{\kern0pt}Axiom\ Separation{\isacharunderscore}{\kern0pt}Axiom\ \isanewline
\ \ \ \ \ \ \ \ \ \ Replacement{\isacharunderscore}{\kern0pt}Axiom\ Interface\ Infinity{\isacharunderscore}{\kern0pt}Axiom\isanewline
\isakeyword{begin}%
\endisatagtheory
{\isafoldtheory}%
%
\isadelimtheory
\isanewline
%
\endisadelimtheory
\isanewline
\isacommand{definition}\isamarkupfalse%
\ \isanewline
\ \ induced{\isacharunderscore}{\kern0pt}surj\ {\isacharcolon}{\kern0pt}{\isacharcolon}{\kern0pt}\ {\isachardoublequoteopen}i{\isasymRightarrow}i{\isasymRightarrow}i{\isasymRightarrow}i{\isachardoublequoteclose}\ \isakeyword{where}\isanewline
\ \ {\isachardoublequoteopen}induced{\isacharunderscore}{\kern0pt}surj{\isacharparenleft}{\kern0pt}f{\isacharcomma}{\kern0pt}a{\isacharcomma}{\kern0pt}e{\isacharparenright}{\kern0pt}\ {\isasymequiv}\ f{\isacharminus}{\kern0pt}{\isacharbackquote}{\kern0pt}{\isacharbackquote}{\kern0pt}{\isacharparenleft}{\kern0pt}range{\isacharparenleft}{\kern0pt}f{\isacharparenright}{\kern0pt}{\isacharminus}{\kern0pt}a{\isacharparenright}{\kern0pt}{\isasymtimes}{\isacharbraceleft}{\kern0pt}e{\isacharbraceright}{\kern0pt}\ {\isasymunion}\ restrict{\isacharparenleft}{\kern0pt}f{\isacharcomma}{\kern0pt}f{\isacharminus}{\kern0pt}{\isacharbackquote}{\kern0pt}{\isacharbackquote}{\kern0pt}a{\isacharparenright}{\kern0pt}{\isachardoublequoteclose}\isanewline
\ \ \isanewline
\isacommand{lemma}\isamarkupfalse%
\ domain{\isacharunderscore}{\kern0pt}induced{\isacharunderscore}{\kern0pt}surj{\isacharcolon}{\kern0pt}\ {\isachardoublequoteopen}domain{\isacharparenleft}{\kern0pt}induced{\isacharunderscore}{\kern0pt}surj{\isacharparenleft}{\kern0pt}f{\isacharcomma}{\kern0pt}a{\isacharcomma}{\kern0pt}e{\isacharparenright}{\kern0pt}{\isacharparenright}{\kern0pt}\ {\isacharequal}{\kern0pt}\ domain{\isacharparenleft}{\kern0pt}f{\isacharparenright}{\kern0pt}{\isachardoublequoteclose}\isanewline
%
\isadelimproof
\ \ %
\endisadelimproof
%
\isatagproof
\isacommand{unfolding}\isamarkupfalse%
\ induced{\isacharunderscore}{\kern0pt}surj{\isacharunderscore}{\kern0pt}def\ \isacommand{using}\isamarkupfalse%
\ domain{\isacharunderscore}{\kern0pt}restrict\ domain{\isacharunderscore}{\kern0pt}of{\isacharunderscore}{\kern0pt}prod\ \isacommand{by}\isamarkupfalse%
\ auto%
\endisatagproof
{\isafoldproof}%
%
\isadelimproof
\isanewline
%
\endisadelimproof
\ \ \ \ \isanewline
\isacommand{lemma}\isamarkupfalse%
\ range{\isacharunderscore}{\kern0pt}restrict{\isacharunderscore}{\kern0pt}vimage{\isacharcolon}{\kern0pt}\ \isanewline
\ \ \isakeyword{assumes}\ {\isachardoublequoteopen}function{\isacharparenleft}{\kern0pt}f{\isacharparenright}{\kern0pt}{\isachardoublequoteclose}\isanewline
\ \ \isakeyword{shows}\ {\isachardoublequoteopen}range{\isacharparenleft}{\kern0pt}restrict{\isacharparenleft}{\kern0pt}f{\isacharcomma}{\kern0pt}f{\isacharminus}{\kern0pt}{\isacharbackquote}{\kern0pt}{\isacharbackquote}{\kern0pt}a{\isacharparenright}{\kern0pt}{\isacharparenright}{\kern0pt}\ {\isasymsubseteq}\ a{\isachardoublequoteclose}\isanewline
%
\isadelimproof
%
\endisadelimproof
%
\isatagproof
\isacommand{proof}\isamarkupfalse%
\isanewline
\ \ \isacommand{from}\isamarkupfalse%
\ assms\ \isanewline
\ \ \isacommand{have}\isamarkupfalse%
\ {\isachardoublequoteopen}function{\isacharparenleft}{\kern0pt}restrict{\isacharparenleft}{\kern0pt}f{\isacharcomma}{\kern0pt}f{\isacharminus}{\kern0pt}{\isacharbackquote}{\kern0pt}{\isacharbackquote}{\kern0pt}a{\isacharparenright}{\kern0pt}{\isacharparenright}{\kern0pt}{\isachardoublequoteclose}\ \isanewline
\ \ \ \ \isacommand{using}\isamarkupfalse%
\ function{\isacharunderscore}{\kern0pt}restrictI\ \isacommand{by}\isamarkupfalse%
\ simp\isanewline
\ \ \isacommand{fix}\isamarkupfalse%
\ y\isanewline
\ \ \isacommand{assume}\isamarkupfalse%
\ {\isachardoublequoteopen}y\ {\isasymin}\ range{\isacharparenleft}{\kern0pt}restrict{\isacharparenleft}{\kern0pt}f{\isacharcomma}{\kern0pt}f{\isacharminus}{\kern0pt}{\isacharbackquote}{\kern0pt}{\isacharbackquote}{\kern0pt}a{\isacharparenright}{\kern0pt}{\isacharparenright}{\kern0pt}{\isachardoublequoteclose}\isanewline
\ \ \isacommand{then}\isamarkupfalse%
\ \isanewline
\ \ \isacommand{obtain}\isamarkupfalse%
\ x\ \isakeyword{where}\ {\isachardoublequoteopen}{\isasymlangle}x{\isacharcomma}{\kern0pt}y{\isasymrangle}\ {\isasymin}\ restrict{\isacharparenleft}{\kern0pt}f{\isacharcomma}{\kern0pt}f{\isacharminus}{\kern0pt}{\isacharbackquote}{\kern0pt}{\isacharbackquote}{\kern0pt}a{\isacharparenright}{\kern0pt}{\isachardoublequoteclose}\ \ {\isachardoublequoteopen}x\ {\isasymin}\ f{\isacharminus}{\kern0pt}{\isacharbackquote}{\kern0pt}{\isacharbackquote}{\kern0pt}a{\isachardoublequoteclose}\ {\isachardoublequoteopen}x{\isasymin}domain{\isacharparenleft}{\kern0pt}f{\isacharparenright}{\kern0pt}{\isachardoublequoteclose}\isanewline
\ \ \ \ \isacommand{using}\isamarkupfalse%
\ domain{\isacharunderscore}{\kern0pt}restrict\ domainI{\isacharbrackleft}{\kern0pt}of\ {\isacharunderscore}{\kern0pt}\ {\isacharunderscore}{\kern0pt}\ {\isachardoublequoteopen}restrict{\isacharparenleft}{\kern0pt}f{\isacharcomma}{\kern0pt}f{\isacharminus}{\kern0pt}{\isacharbackquote}{\kern0pt}{\isacharbackquote}{\kern0pt}a{\isacharparenright}{\kern0pt}{\isachardoublequoteclose}{\isacharbrackright}{\kern0pt}\ \isacommand{by}\isamarkupfalse%
\ auto\isanewline
\ \ \isacommand{moreover}\isamarkupfalse%
\ \isanewline
\ \ \isacommand{note}\isamarkupfalse%
\ {\isacartoucheopen}function{\isacharparenleft}{\kern0pt}restrict{\isacharparenleft}{\kern0pt}f{\isacharcomma}{\kern0pt}f{\isacharminus}{\kern0pt}{\isacharbackquote}{\kern0pt}{\isacharbackquote}{\kern0pt}a{\isacharparenright}{\kern0pt}{\isacharparenright}{\kern0pt}{\isacartoucheclose}\ \isanewline
\ \ \isacommand{ultimately}\isamarkupfalse%
\ \isanewline
\ \ \isacommand{have}\isamarkupfalse%
\ {\isachardoublequoteopen}y\ {\isacharequal}{\kern0pt}\ restrict{\isacharparenleft}{\kern0pt}f{\isacharcomma}{\kern0pt}f{\isacharminus}{\kern0pt}{\isacharbackquote}{\kern0pt}{\isacharbackquote}{\kern0pt}a{\isacharparenright}{\kern0pt}{\isacharbackquote}{\kern0pt}x{\isachardoublequoteclose}\ \isanewline
\ \ \ \ \isacommand{using}\isamarkupfalse%
\ function{\isacharunderscore}{\kern0pt}apply{\isacharunderscore}{\kern0pt}equality\ \isacommand{by}\isamarkupfalse%
\ blast\isanewline
\ \ \isacommand{also}\isamarkupfalse%
\ \isacommand{from}\isamarkupfalse%
\ {\isacartoucheopen}x\ {\isasymin}\ f{\isacharminus}{\kern0pt}{\isacharbackquote}{\kern0pt}{\isacharbackquote}{\kern0pt}a{\isacartoucheclose}\ \isanewline
\ \ \isacommand{have}\isamarkupfalse%
\ {\isachardoublequoteopen}restrict{\isacharparenleft}{\kern0pt}f{\isacharcomma}{\kern0pt}f{\isacharminus}{\kern0pt}{\isacharbackquote}{\kern0pt}{\isacharbackquote}{\kern0pt}a{\isacharparenright}{\kern0pt}{\isacharbackquote}{\kern0pt}x\ {\isacharequal}{\kern0pt}\ f{\isacharbackquote}{\kern0pt}x{\isachardoublequoteclose}\ \isanewline
\ \ \ \ \isacommand{by}\isamarkupfalse%
\ simp\isanewline
\ \ \isacommand{finally}\isamarkupfalse%
\ \isanewline
\ \ \isacommand{have}\isamarkupfalse%
\ {\isachardoublequoteopen}y{\isacharequal}{\kern0pt}f{\isacharbackquote}{\kern0pt}x{\isachardoublequoteclose}\ \isacommand{{\isachardot}{\kern0pt}}\isamarkupfalse%
\isanewline
\ \ \isacommand{moreover}\isamarkupfalse%
\ \isacommand{from}\isamarkupfalse%
\ assms\ {\isacartoucheopen}x{\isasymin}domain{\isacharparenleft}{\kern0pt}f{\isacharparenright}{\kern0pt}{\isacartoucheclose}\ \isanewline
\ \ \isacommand{have}\isamarkupfalse%
\ {\isachardoublequoteopen}{\isasymlangle}x{\isacharcomma}{\kern0pt}f{\isacharbackquote}{\kern0pt}x{\isasymrangle}\ {\isasymin}\ f{\isachardoublequoteclose}\ \isanewline
\ \ \ \ \isacommand{using}\isamarkupfalse%
\ function{\isacharunderscore}{\kern0pt}apply{\isacharunderscore}{\kern0pt}Pair\ \isacommand{by}\isamarkupfalse%
\ auto\ \isanewline
\ \ \isacommand{moreover}\isamarkupfalse%
\ \isanewline
\ \ \isacommand{note}\isamarkupfalse%
\ assms\ {\isacartoucheopen}x\ {\isasymin}\ f{\isacharminus}{\kern0pt}{\isacharbackquote}{\kern0pt}{\isacharbackquote}{\kern0pt}a{\isacartoucheclose}\ \isanewline
\ \ \isacommand{ultimately}\isamarkupfalse%
\ \isanewline
\ \ \isacommand{show}\isamarkupfalse%
\ {\isachardoublequoteopen}y{\isasymin}a{\isachardoublequoteclose}\isanewline
\ \ \ \ \isacommand{using}\isamarkupfalse%
\ function{\isacharunderscore}{\kern0pt}image{\isacharunderscore}{\kern0pt}vimage{\isacharbrackleft}{\kern0pt}of\ f\ a{\isacharbrackright}{\kern0pt}\ \isacommand{by}\isamarkupfalse%
\ auto\isanewline
\isacommand{qed}\isamarkupfalse%
%
\endisatagproof
{\isafoldproof}%
%
\isadelimproof
\isanewline
%
\endisadelimproof
\ \ \isanewline
\isacommand{lemma}\isamarkupfalse%
\ induced{\isacharunderscore}{\kern0pt}surj{\isacharunderscore}{\kern0pt}type{\isacharcolon}{\kern0pt}\ \isanewline
\ \ \isakeyword{assumes}\isanewline
\ \ \ \ {\isachardoublequoteopen}function{\isacharparenleft}{\kern0pt}f{\isacharparenright}{\kern0pt}{\isachardoublequoteclose}\ \isanewline
\ \ \isakeyword{shows}\ \isanewline
\ \ \ \ {\isachardoublequoteopen}induced{\isacharunderscore}{\kern0pt}surj{\isacharparenleft}{\kern0pt}f{\isacharcomma}{\kern0pt}a{\isacharcomma}{\kern0pt}e{\isacharparenright}{\kern0pt}{\isacharcolon}{\kern0pt}\ domain{\isacharparenleft}{\kern0pt}f{\isacharparenright}{\kern0pt}\ {\isasymrightarrow}\ {\isacharbraceleft}{\kern0pt}e{\isacharbraceright}{\kern0pt}\ {\isasymunion}\ a{\isachardoublequoteclose}\isanewline
\ \ \ \ \isakeyword{and}\isanewline
\ \ \ \ {\isachardoublequoteopen}x\ {\isasymin}\ f{\isacharminus}{\kern0pt}{\isacharbackquote}{\kern0pt}{\isacharbackquote}{\kern0pt}a\ {\isasymLongrightarrow}\ induced{\isacharunderscore}{\kern0pt}surj{\isacharparenleft}{\kern0pt}f{\isacharcomma}{\kern0pt}a{\isacharcomma}{\kern0pt}e{\isacharparenright}{\kern0pt}{\isacharbackquote}{\kern0pt}x\ {\isacharequal}{\kern0pt}\ f{\isacharbackquote}{\kern0pt}x{\isachardoublequoteclose}\ \isanewline
%
\isadelimproof
%
\endisadelimproof
%
\isatagproof
\isacommand{proof}\isamarkupfalse%
\ {\isacharminus}{\kern0pt}\isanewline
\ \ \isacommand{let}\isamarkupfalse%
\ {\isacharquery}{\kern0pt}f{\isadigit{1}}{\isacharequal}{\kern0pt}{\isachardoublequoteopen}f{\isacharminus}{\kern0pt}{\isacharbackquote}{\kern0pt}{\isacharbackquote}{\kern0pt}{\isacharparenleft}{\kern0pt}range{\isacharparenleft}{\kern0pt}f{\isacharparenright}{\kern0pt}{\isacharminus}{\kern0pt}a{\isacharparenright}{\kern0pt}\ {\isasymtimes}\ {\isacharbraceleft}{\kern0pt}e{\isacharbraceright}{\kern0pt}{\isachardoublequoteclose}\ \isakeyword{and}\ {\isacharquery}{\kern0pt}f{\isadigit{2}}{\isacharequal}{\kern0pt}{\isachardoublequoteopen}restrict{\isacharparenleft}{\kern0pt}f{\isacharcomma}{\kern0pt}\ f{\isacharminus}{\kern0pt}{\isacharbackquote}{\kern0pt}{\isacharbackquote}{\kern0pt}a{\isacharparenright}{\kern0pt}{\isachardoublequoteclose}\isanewline
\ \ \isacommand{have}\isamarkupfalse%
\ {\isachardoublequoteopen}domain{\isacharparenleft}{\kern0pt}{\isacharquery}{\kern0pt}f{\isadigit{2}}{\isacharparenright}{\kern0pt}\ {\isacharequal}{\kern0pt}\ domain{\isacharparenleft}{\kern0pt}f{\isacharparenright}{\kern0pt}\ {\isasyminter}\ f{\isacharminus}{\kern0pt}{\isacharbackquote}{\kern0pt}{\isacharbackquote}{\kern0pt}a{\isachardoublequoteclose}\isanewline
\ \ \ \ \isacommand{using}\isamarkupfalse%
\ domain{\isacharunderscore}{\kern0pt}restrict\ \isacommand{by}\isamarkupfalse%
\ simp\isanewline
\ \ \isacommand{moreover}\isamarkupfalse%
\ \isacommand{from}\isamarkupfalse%
\ assms\ \isanewline
\ \ \isacommand{have}\isamarkupfalse%
\ {\isadigit{1}}{\isacharcolon}{\kern0pt}\ {\isachardoublequoteopen}domain{\isacharparenleft}{\kern0pt}{\isacharquery}{\kern0pt}f{\isadigit{1}}{\isacharparenright}{\kern0pt}\ {\isacharequal}{\kern0pt}\ f{\isacharminus}{\kern0pt}{\isacharbackquote}{\kern0pt}{\isacharbackquote}{\kern0pt}{\isacharparenleft}{\kern0pt}range{\isacharparenleft}{\kern0pt}f{\isacharparenright}{\kern0pt}{\isacharparenright}{\kern0pt}{\isacharminus}{\kern0pt}f{\isacharminus}{\kern0pt}{\isacharbackquote}{\kern0pt}{\isacharbackquote}{\kern0pt}a{\isachardoublequoteclose}\isanewline
\ \ \ \ \isacommand{using}\isamarkupfalse%
\ domain{\isacharunderscore}{\kern0pt}of{\isacharunderscore}{\kern0pt}prod\ function{\isacharunderscore}{\kern0pt}vimage{\isacharunderscore}{\kern0pt}Diff\ \isacommand{by}\isamarkupfalse%
\ simp\isanewline
\ \ \isacommand{ultimately}\isamarkupfalse%
\ \isanewline
\ \ \isacommand{have}\isamarkupfalse%
\ {\isachardoublequoteopen}domain{\isacharparenleft}{\kern0pt}{\isacharquery}{\kern0pt}f{\isadigit{1}}{\isacharparenright}{\kern0pt}\ {\isasyminter}\ domain{\isacharparenleft}{\kern0pt}{\isacharquery}{\kern0pt}f{\isadigit{2}}{\isacharparenright}{\kern0pt}\ {\isacharequal}{\kern0pt}\ {\isadigit{0}}{\isachardoublequoteclose}\isanewline
\ \ \ \ \isacommand{by}\isamarkupfalse%
\ auto\isanewline
\ \ \isacommand{moreover}\isamarkupfalse%
\ \isanewline
\ \ \isacommand{have}\isamarkupfalse%
\ {\isachardoublequoteopen}function{\isacharparenleft}{\kern0pt}{\isacharquery}{\kern0pt}f{\isadigit{1}}{\isacharparenright}{\kern0pt}{\isachardoublequoteclose}\ {\isachardoublequoteopen}relation{\isacharparenleft}{\kern0pt}{\isacharquery}{\kern0pt}f{\isadigit{1}}{\isacharparenright}{\kern0pt}{\isachardoublequoteclose}\ {\isachardoublequoteopen}range{\isacharparenleft}{\kern0pt}{\isacharquery}{\kern0pt}f{\isadigit{1}}{\isacharparenright}{\kern0pt}\ {\isasymsubseteq}\ {\isacharbraceleft}{\kern0pt}e{\isacharbraceright}{\kern0pt}{\isachardoublequoteclose}\isanewline
\ \ \ \ \isacommand{unfolding}\isamarkupfalse%
\ function{\isacharunderscore}{\kern0pt}def\ relation{\isacharunderscore}{\kern0pt}def\ range{\isacharunderscore}{\kern0pt}def\ \isacommand{by}\isamarkupfalse%
\ auto\isanewline
\ \ \isacommand{moreover}\isamarkupfalse%
\ \isacommand{from}\isamarkupfalse%
\ this\ \isakeyword{and}\ assms\ \isanewline
\ \ \isacommand{have}\isamarkupfalse%
\ {\isachardoublequoteopen}{\isacharquery}{\kern0pt}f{\isadigit{1}}{\isacharcolon}{\kern0pt}\ domain{\isacharparenleft}{\kern0pt}{\isacharquery}{\kern0pt}f{\isadigit{1}}{\isacharparenright}{\kern0pt}\ {\isasymrightarrow}\ range{\isacharparenleft}{\kern0pt}{\isacharquery}{\kern0pt}f{\isadigit{1}}{\isacharparenright}{\kern0pt}{\isachardoublequoteclose}\isanewline
\ \ \ \ \isacommand{using}\isamarkupfalse%
\ function{\isacharunderscore}{\kern0pt}imp{\isacharunderscore}{\kern0pt}Pi\ \isacommand{by}\isamarkupfalse%
\ simp\isanewline
\ \ \isacommand{moreover}\isamarkupfalse%
\ \isacommand{from}\isamarkupfalse%
\ assms\ \isanewline
\ \ \isacommand{have}\isamarkupfalse%
\ {\isachardoublequoteopen}{\isacharquery}{\kern0pt}f{\isadigit{2}}{\isacharcolon}{\kern0pt}\ domain{\isacharparenleft}{\kern0pt}{\isacharquery}{\kern0pt}f{\isadigit{2}}{\isacharparenright}{\kern0pt}\ {\isasymrightarrow}\ range{\isacharparenleft}{\kern0pt}{\isacharquery}{\kern0pt}f{\isadigit{2}}{\isacharparenright}{\kern0pt}{\isachardoublequoteclose}\isanewline
\ \ \ \ \isacommand{using}\isamarkupfalse%
\ function{\isacharunderscore}{\kern0pt}imp{\isacharunderscore}{\kern0pt}Pi{\isacharbrackleft}{\kern0pt}of\ {\isachardoublequoteopen}restrict{\isacharparenleft}{\kern0pt}f{\isacharcomma}{\kern0pt}\ f\ {\isacharminus}{\kern0pt}{\isacharbackquote}{\kern0pt}{\isacharbackquote}{\kern0pt}\ a{\isacharparenright}{\kern0pt}{\isachardoublequoteclose}{\isacharbrackright}{\kern0pt}\ function{\isacharunderscore}{\kern0pt}restrictI\ \isacommand{by}\isamarkupfalse%
\ simp\isanewline
\ \ \isacommand{moreover}\isamarkupfalse%
\ \isacommand{from}\isamarkupfalse%
\ assms\ \isanewline
\ \ \isacommand{have}\isamarkupfalse%
\ {\isachardoublequoteopen}range{\isacharparenleft}{\kern0pt}{\isacharquery}{\kern0pt}f{\isadigit{2}}{\isacharparenright}{\kern0pt}\ {\isasymsubseteq}\ a{\isachardoublequoteclose}\ \isanewline
\ \ \ \ \isacommand{using}\isamarkupfalse%
\ range{\isacharunderscore}{\kern0pt}restrict{\isacharunderscore}{\kern0pt}vimage\ \isacommand{by}\isamarkupfalse%
\ simp\isanewline
\ \ \isacommand{ultimately}\isamarkupfalse%
\ \isanewline
\ \ \isacommand{have}\isamarkupfalse%
\ {\isachardoublequoteopen}induced{\isacharunderscore}{\kern0pt}surj{\isacharparenleft}{\kern0pt}f{\isacharcomma}{\kern0pt}a{\isacharcomma}{\kern0pt}e{\isacharparenright}{\kern0pt}{\isacharcolon}{\kern0pt}\ domain{\isacharparenleft}{\kern0pt}{\isacharquery}{\kern0pt}f{\isadigit{1}}{\isacharparenright}{\kern0pt}\ {\isasymunion}\ domain{\isacharparenleft}{\kern0pt}{\isacharquery}{\kern0pt}f{\isadigit{2}}{\isacharparenright}{\kern0pt}\ {\isasymrightarrow}\ {\isacharbraceleft}{\kern0pt}e{\isacharbraceright}{\kern0pt}\ {\isasymunion}\ a{\isachardoublequoteclose}\isanewline
\ \ \ \ \isacommand{unfolding}\isamarkupfalse%
\ induced{\isacharunderscore}{\kern0pt}surj{\isacharunderscore}{\kern0pt}def\ \isacommand{using}\isamarkupfalse%
\ fun{\isacharunderscore}{\kern0pt}is{\isacharunderscore}{\kern0pt}function\ fun{\isacharunderscore}{\kern0pt}disjoint{\isacharunderscore}{\kern0pt}Un\ fun{\isacharunderscore}{\kern0pt}weaken{\isacharunderscore}{\kern0pt}type\ \isacommand{by}\isamarkupfalse%
\ simp\isanewline
\ \ \isacommand{moreover}\isamarkupfalse%
\ \isanewline
\ \ \isacommand{have}\isamarkupfalse%
\ {\isachardoublequoteopen}domain{\isacharparenleft}{\kern0pt}{\isacharquery}{\kern0pt}f{\isadigit{1}}{\isacharparenright}{\kern0pt}\ {\isasymunion}\ domain{\isacharparenleft}{\kern0pt}{\isacharquery}{\kern0pt}f{\isadigit{2}}{\isacharparenright}{\kern0pt}\ {\isacharequal}{\kern0pt}\ domain{\isacharparenleft}{\kern0pt}f{\isacharparenright}{\kern0pt}{\isachardoublequoteclose}\isanewline
\ \ \ \ \isacommand{using}\isamarkupfalse%
\ domain{\isacharunderscore}{\kern0pt}restrict\ domain{\isacharunderscore}{\kern0pt}of{\isacharunderscore}{\kern0pt}prod\ \isacommand{by}\isamarkupfalse%
\ auto\ \isanewline
\ \ \isacommand{ultimately}\isamarkupfalse%
\isanewline
\ \ \isacommand{show}\isamarkupfalse%
\ {\isachardoublequoteopen}induced{\isacharunderscore}{\kern0pt}surj{\isacharparenleft}{\kern0pt}f{\isacharcomma}{\kern0pt}a{\isacharcomma}{\kern0pt}e{\isacharparenright}{\kern0pt}{\isacharcolon}{\kern0pt}\ domain{\isacharparenleft}{\kern0pt}f{\isacharparenright}{\kern0pt}\ {\isasymrightarrow}\ {\isacharbraceleft}{\kern0pt}e{\isacharbraceright}{\kern0pt}\ {\isasymunion}\ a{\isachardoublequoteclose}\isanewline
\ \ \ \ \isacommand{by}\isamarkupfalse%
\ simp\isanewline
\ \ \isacommand{assume}\isamarkupfalse%
\ {\isachardoublequoteopen}x\ {\isasymin}\ f{\isacharminus}{\kern0pt}{\isacharbackquote}{\kern0pt}{\isacharbackquote}{\kern0pt}a{\isachardoublequoteclose}\isanewline
\ \ \isacommand{then}\isamarkupfalse%
\ \isanewline
\ \ \isacommand{have}\isamarkupfalse%
\ {\isachardoublequoteopen}{\isacharquery}{\kern0pt}f{\isadigit{2}}{\isacharbackquote}{\kern0pt}x\ {\isacharequal}{\kern0pt}\ f{\isacharbackquote}{\kern0pt}x{\isachardoublequoteclose}\isanewline
\ \ \ \ \isacommand{using}\isamarkupfalse%
\ restrict\ \isacommand{by}\isamarkupfalse%
\ simp\isanewline
\ \ \isacommand{moreover}\isamarkupfalse%
\ \isacommand{from}\isamarkupfalse%
\ {\isacartoucheopen}x\ {\isasymin}\ f{\isacharminus}{\kern0pt}{\isacharbackquote}{\kern0pt}{\isacharbackquote}{\kern0pt}a{\isacartoucheclose}\ \isakeyword{and}\ {\isadigit{1}}\ \isanewline
\ \ \isacommand{have}\isamarkupfalse%
\ {\isachardoublequoteopen}x\ {\isasymnotin}\ domain{\isacharparenleft}{\kern0pt}{\isacharquery}{\kern0pt}f{\isadigit{1}}{\isacharparenright}{\kern0pt}{\isachardoublequoteclose}\isanewline
\ \ \ \ \isacommand{by}\isamarkupfalse%
\ simp\isanewline
\ \ \isacommand{ultimately}\isamarkupfalse%
\ \isanewline
\ \ \isacommand{show}\isamarkupfalse%
\ {\isachardoublequoteopen}induced{\isacharunderscore}{\kern0pt}surj{\isacharparenleft}{\kern0pt}f{\isacharcomma}{\kern0pt}a{\isacharcomma}{\kern0pt}e{\isacharparenright}{\kern0pt}{\isacharbackquote}{\kern0pt}x\ {\isacharequal}{\kern0pt}\ f{\isacharbackquote}{\kern0pt}x{\isachardoublequoteclose}\ \isanewline
\ \ \ \ \isacommand{unfolding}\isamarkupfalse%
\ induced{\isacharunderscore}{\kern0pt}surj{\isacharunderscore}{\kern0pt}def\ \isacommand{using}\isamarkupfalse%
\ fun{\isacharunderscore}{\kern0pt}disjoint{\isacharunderscore}{\kern0pt}apply{\isadigit{2}}{\isacharbrackleft}{\kern0pt}of\ x\ {\isacharquery}{\kern0pt}f{\isadigit{1}}\ {\isacharquery}{\kern0pt}f{\isadigit{2}}{\isacharbrackright}{\kern0pt}\ \isacommand{by}\isamarkupfalse%
\ simp\isanewline
\isacommand{qed}\isamarkupfalse%
%
\endisatagproof
{\isafoldproof}%
%
\isadelimproof
\isanewline
%
\endisadelimproof
\ \ \isanewline
\isacommand{lemma}\isamarkupfalse%
\ induced{\isacharunderscore}{\kern0pt}surj{\isacharunderscore}{\kern0pt}is{\isacharunderscore}{\kern0pt}surj\ {\isacharcolon}{\kern0pt}\ \isanewline
\ \ \isakeyword{assumes}\ \isanewline
\ \ \ \ {\isachardoublequoteopen}e{\isasymin}a{\isachardoublequoteclose}\ {\isachardoublequoteopen}function{\isacharparenleft}{\kern0pt}f{\isacharparenright}{\kern0pt}{\isachardoublequoteclose}\ {\isachardoublequoteopen}domain{\isacharparenleft}{\kern0pt}f{\isacharparenright}{\kern0pt}\ {\isacharequal}{\kern0pt}\ {\isasymalpha}{\isachardoublequoteclose}\ {\isachardoublequoteopen}{\isasymAnd}y{\isachardot}{\kern0pt}\ y\ {\isasymin}\ a\ {\isasymLongrightarrow}\ {\isasymexists}x{\isasymin}{\isasymalpha}{\isachardot}{\kern0pt}\ f\ {\isacharbackquote}{\kern0pt}\ x\ {\isacharequal}{\kern0pt}\ y{\isachardoublequoteclose}\ \isanewline
\ \ \isakeyword{shows}\isanewline
\ \ \ \ {\isachardoublequoteopen}induced{\isacharunderscore}{\kern0pt}surj{\isacharparenleft}{\kern0pt}f{\isacharcomma}{\kern0pt}a{\isacharcomma}{\kern0pt}e{\isacharparenright}{\kern0pt}\ {\isasymin}\ surj{\isacharparenleft}{\kern0pt}{\isasymalpha}{\isacharcomma}{\kern0pt}a{\isacharparenright}{\kern0pt}{\isachardoublequoteclose}\isanewline
%
\isadelimproof
\ \ %
\endisadelimproof
%
\isatagproof
\isacommand{unfolding}\isamarkupfalse%
\ surj{\isacharunderscore}{\kern0pt}def\isanewline
\isacommand{proof}\isamarkupfalse%
\ {\isacharparenleft}{\kern0pt}intro\ CollectI\ ballI{\isacharparenright}{\kern0pt}\isanewline
\ \ \isacommand{from}\isamarkupfalse%
\ assms\ \isanewline
\ \ \isacommand{show}\isamarkupfalse%
\ {\isachardoublequoteopen}induced{\isacharunderscore}{\kern0pt}surj{\isacharparenleft}{\kern0pt}f{\isacharcomma}{\kern0pt}a{\isacharcomma}{\kern0pt}e{\isacharparenright}{\kern0pt}{\isacharcolon}{\kern0pt}\ {\isasymalpha}\ {\isasymrightarrow}\ a{\isachardoublequoteclose}\isanewline
\ \ \ \ \isacommand{using}\isamarkupfalse%
\ induced{\isacharunderscore}{\kern0pt}surj{\isacharunderscore}{\kern0pt}type{\isacharbrackleft}{\kern0pt}of\ f\ a\ e{\isacharbrackright}{\kern0pt}\ cons{\isacharunderscore}{\kern0pt}eq\ cons{\isacharunderscore}{\kern0pt}absorb\ \isacommand{by}\isamarkupfalse%
\ simp\isanewline
\ \ \isacommand{fix}\isamarkupfalse%
\ y\isanewline
\ \ \isacommand{assume}\isamarkupfalse%
\ {\isachardoublequoteopen}y\ {\isasymin}\ a{\isachardoublequoteclose}\isanewline
\ \ \isacommand{with}\isamarkupfalse%
\ assms\ \isanewline
\ \ \isacommand{have}\isamarkupfalse%
\ {\isachardoublequoteopen}{\isasymexists}x{\isasymin}{\isasymalpha}{\isachardot}{\kern0pt}\ f\ {\isacharbackquote}{\kern0pt}\ x\ {\isacharequal}{\kern0pt}\ y{\isachardoublequoteclose}\ \isanewline
\ \ \ \ \isacommand{by}\isamarkupfalse%
\ simp\isanewline
\ \ \isacommand{then}\isamarkupfalse%
\isanewline
\ \ \isacommand{obtain}\isamarkupfalse%
\ x\ \isakeyword{where}\ {\isachardoublequoteopen}x{\isasymin}{\isasymalpha}{\isachardoublequoteclose}\ {\isachardoublequoteopen}f\ {\isacharbackquote}{\kern0pt}\ x\ {\isacharequal}{\kern0pt}\ y{\isachardoublequoteclose}\ \isacommand{by}\isamarkupfalse%
\ auto\isanewline
\ \ \isacommand{with}\isamarkupfalse%
\ {\isacartoucheopen}y{\isasymin}a{\isacartoucheclose}\ assms\isanewline
\ \ \isacommand{have}\isamarkupfalse%
\ {\isachardoublequoteopen}x{\isasymin}f{\isacharminus}{\kern0pt}{\isacharbackquote}{\kern0pt}{\isacharbackquote}{\kern0pt}a{\isachardoublequoteclose}\ \isanewline
\ \ \ \ \isacommand{using}\isamarkupfalse%
\ vimage{\isacharunderscore}{\kern0pt}iff\ function{\isacharunderscore}{\kern0pt}apply{\isacharunderscore}{\kern0pt}Pair{\isacharbrackleft}{\kern0pt}of\ f\ x{\isacharbrackright}{\kern0pt}\ \isacommand{by}\isamarkupfalse%
\ auto\isanewline
\ \ \isacommand{with}\isamarkupfalse%
\ {\isacartoucheopen}f\ {\isacharbackquote}{\kern0pt}\ x\ {\isacharequal}{\kern0pt}\ y{\isacartoucheclose}\ assms\isanewline
\ \ \isacommand{have}\isamarkupfalse%
\ {\isachardoublequoteopen}induced{\isacharunderscore}{\kern0pt}surj{\isacharparenleft}{\kern0pt}f{\isacharcomma}{\kern0pt}\ a{\isacharcomma}{\kern0pt}\ e{\isacharparenright}{\kern0pt}\ {\isacharbackquote}{\kern0pt}\ x\ {\isacharequal}{\kern0pt}\ y{\isachardoublequoteclose}\isanewline
\ \ \ \ \isacommand{using}\isamarkupfalse%
\ induced{\isacharunderscore}{\kern0pt}surj{\isacharunderscore}{\kern0pt}type\ \isacommand{by}\isamarkupfalse%
\ simp\isanewline
\ \ \isacommand{with}\isamarkupfalse%
\ {\isacartoucheopen}x{\isasymin}{\isasymalpha}{\isacartoucheclose}\ \isacommand{show}\isamarkupfalse%
\isanewline
\ \ \ \ {\isachardoublequoteopen}{\isasymexists}x{\isasymin}{\isasymalpha}{\isachardot}{\kern0pt}\ induced{\isacharunderscore}{\kern0pt}surj{\isacharparenleft}{\kern0pt}f{\isacharcomma}{\kern0pt}\ a{\isacharcomma}{\kern0pt}\ e{\isacharparenright}{\kern0pt}\ {\isacharbackquote}{\kern0pt}\ x\ {\isacharequal}{\kern0pt}\ y{\isachardoublequoteclose}\ \isacommand{by}\isamarkupfalse%
\ auto\isanewline
\isacommand{qed}\isamarkupfalse%
%
\endisatagproof
{\isafoldproof}%
%
\isadelimproof
\isanewline
%
\endisadelimproof
\ \ \isanewline
\isacommand{context}\isamarkupfalse%
\ G{\isacharunderscore}{\kern0pt}generic\ \isanewline
\isakeyword{begin}\isanewline
\isanewline
\isacommand{definition}\isamarkupfalse%
\isanewline
\ \ upair{\isacharunderscore}{\kern0pt}name\ {\isacharcolon}{\kern0pt}{\isacharcolon}{\kern0pt}\ {\isachardoublequoteopen}i\ {\isasymRightarrow}\ i\ {\isasymRightarrow}\ i{\isachardoublequoteclose}\ \isakeyword{where}\isanewline
\ \ {\isachardoublequoteopen}upair{\isacharunderscore}{\kern0pt}name{\isacharparenleft}{\kern0pt}{\isasymtau}{\isacharcomma}{\kern0pt}{\isasymrho}{\isacharparenright}{\kern0pt}\ {\isasymequiv}\ {\isacharbraceleft}{\kern0pt}{\isasymlangle}{\isasymtau}{\isacharcomma}{\kern0pt}one{\isasymrangle}{\isacharcomma}{\kern0pt}{\isasymlangle}{\isasymrho}{\isacharcomma}{\kern0pt}one{\isasymrangle}{\isacharbraceright}{\kern0pt}{\isachardoublequoteclose}\isanewline
\isanewline
\isacommand{definition}\isamarkupfalse%
\isanewline
\ \ is{\isacharunderscore}{\kern0pt}upair{\isacharunderscore}{\kern0pt}name\ {\isacharcolon}{\kern0pt}{\isacharcolon}{\kern0pt}\ {\isachardoublequoteopen}{\isacharbrackleft}{\kern0pt}i{\isacharcomma}{\kern0pt}i{\isacharcomma}{\kern0pt}i{\isacharbrackright}{\kern0pt}\ {\isasymRightarrow}\ o{\isachardoublequoteclose}\ \isakeyword{where}\isanewline
\ \ {\isachardoublequoteopen}is{\isacharunderscore}{\kern0pt}upair{\isacharunderscore}{\kern0pt}name{\isacharparenleft}{\kern0pt}x{\isacharcomma}{\kern0pt}y{\isacharcomma}{\kern0pt}z{\isacharparenright}{\kern0pt}\ {\isasymequiv}\ {\isasymexists}xo{\isasymin}M{\isachardot}{\kern0pt}\ {\isasymexists}yo{\isasymin}M{\isachardot}{\kern0pt}\ pair{\isacharparenleft}{\kern0pt}{\isacharhash}{\kern0pt}{\isacharhash}{\kern0pt}M{\isacharcomma}{\kern0pt}x{\isacharcomma}{\kern0pt}one{\isacharcomma}{\kern0pt}xo{\isacharparenright}{\kern0pt}\ {\isasymand}\ pair{\isacharparenleft}{\kern0pt}{\isacharhash}{\kern0pt}{\isacharhash}{\kern0pt}M{\isacharcomma}{\kern0pt}y{\isacharcomma}{\kern0pt}one{\isacharcomma}{\kern0pt}yo{\isacharparenright}{\kern0pt}\ {\isasymand}\ \isanewline
\ \ \ \ \ \ \ \ \ \ \ \ \ \ \ \ \ \ \ \ \ \ \ \ \ \ \ \ \ \ \ \ \ \ \ \ \ \ \ upair{\isacharparenleft}{\kern0pt}{\isacharhash}{\kern0pt}{\isacharhash}{\kern0pt}M{\isacharcomma}{\kern0pt}xo{\isacharcomma}{\kern0pt}yo{\isacharcomma}{\kern0pt}z{\isacharparenright}{\kern0pt}{\isachardoublequoteclose}\isanewline
\isanewline
\isacommand{lemma}\isamarkupfalse%
\ upair{\isacharunderscore}{\kern0pt}name{\isacharunderscore}{\kern0pt}abs\ {\isacharcolon}{\kern0pt}\ \isanewline
\ \ \isakeyword{assumes}\ {\isachardoublequoteopen}x{\isasymin}M{\isachardoublequoteclose}\ {\isachardoublequoteopen}y{\isasymin}M{\isachardoublequoteclose}\ {\isachardoublequoteopen}z{\isasymin}M{\isachardoublequoteclose}\ \isanewline
\ \ \isakeyword{shows}\ {\isachardoublequoteopen}is{\isacharunderscore}{\kern0pt}upair{\isacharunderscore}{\kern0pt}name{\isacharparenleft}{\kern0pt}x{\isacharcomma}{\kern0pt}y{\isacharcomma}{\kern0pt}z{\isacharparenright}{\kern0pt}\ {\isasymlongleftrightarrow}\ z\ {\isacharequal}{\kern0pt}\ upair{\isacharunderscore}{\kern0pt}name{\isacharparenleft}{\kern0pt}x{\isacharcomma}{\kern0pt}y{\isacharparenright}{\kern0pt}{\isachardoublequoteclose}\ \isanewline
%
\isadelimproof
\ \ %
\endisadelimproof
%
\isatagproof
\isacommand{unfolding}\isamarkupfalse%
\ is{\isacharunderscore}{\kern0pt}upair{\isacharunderscore}{\kern0pt}name{\isacharunderscore}{\kern0pt}def\ upair{\isacharunderscore}{\kern0pt}name{\isacharunderscore}{\kern0pt}def\ \isacommand{using}\isamarkupfalse%
\ assms\ one{\isacharunderscore}{\kern0pt}in{\isacharunderscore}{\kern0pt}M\ pair{\isacharunderscore}{\kern0pt}in{\isacharunderscore}{\kern0pt}M{\isacharunderscore}{\kern0pt}iff\ \isacommand{by}\isamarkupfalse%
\ simp%
\endisatagproof
{\isafoldproof}%
%
\isadelimproof
\isanewline
%
\endisadelimproof
\isanewline
\isacommand{lemma}\isamarkupfalse%
\ upair{\isacharunderscore}{\kern0pt}name{\isacharunderscore}{\kern0pt}closed\ {\isacharcolon}{\kern0pt}\isanewline
\ \ {\isachardoublequoteopen}{\isasymlbrakk}\ x{\isasymin}M{\isacharsemicolon}{\kern0pt}\ y{\isasymin}M\ {\isasymrbrakk}\ {\isasymLongrightarrow}\ upair{\isacharunderscore}{\kern0pt}name{\isacharparenleft}{\kern0pt}x{\isacharcomma}{\kern0pt}y{\isacharparenright}{\kern0pt}{\isasymin}M{\isachardoublequoteclose}\ \isanewline
%
\isadelimproof
\ \ %
\endisadelimproof
%
\isatagproof
\isacommand{unfolding}\isamarkupfalse%
\ upair{\isacharunderscore}{\kern0pt}name{\isacharunderscore}{\kern0pt}def\ \isacommand{using}\isamarkupfalse%
\ upair{\isacharunderscore}{\kern0pt}in{\isacharunderscore}{\kern0pt}M{\isacharunderscore}{\kern0pt}iff\ pair{\isacharunderscore}{\kern0pt}in{\isacharunderscore}{\kern0pt}M{\isacharunderscore}{\kern0pt}iff\ one{\isacharunderscore}{\kern0pt}in{\isacharunderscore}{\kern0pt}M\ \isacommand{by}\isamarkupfalse%
\ simp%
\endisatagproof
{\isafoldproof}%
%
\isadelimproof
\isanewline
%
\endisadelimproof
\isanewline
\isacommand{definition}\isamarkupfalse%
\isanewline
\ \ upair{\isacharunderscore}{\kern0pt}name{\isacharunderscore}{\kern0pt}fm\ {\isacharcolon}{\kern0pt}{\isacharcolon}{\kern0pt}\ {\isachardoublequoteopen}{\isacharbrackleft}{\kern0pt}i{\isacharcomma}{\kern0pt}i{\isacharcomma}{\kern0pt}i{\isacharcomma}{\kern0pt}i{\isacharbrackright}{\kern0pt}\ {\isasymRightarrow}\ i{\isachardoublequoteclose}\ \isakeyword{where}\isanewline
\ \ {\isachardoublequoteopen}upair{\isacharunderscore}{\kern0pt}name{\isacharunderscore}{\kern0pt}fm{\isacharparenleft}{\kern0pt}x{\isacharcomma}{\kern0pt}y{\isacharcomma}{\kern0pt}o{\isacharcomma}{\kern0pt}z{\isacharparenright}{\kern0pt}\ {\isasymequiv}\ Exists{\isacharparenleft}{\kern0pt}Exists{\isacharparenleft}{\kern0pt}And{\isacharparenleft}{\kern0pt}pair{\isacharunderscore}{\kern0pt}fm{\isacharparenleft}{\kern0pt}x{\isacharhash}{\kern0pt}{\isacharplus}{\kern0pt}{\isadigit{2}}{\isacharcomma}{\kern0pt}o{\isacharhash}{\kern0pt}{\isacharplus}{\kern0pt}{\isadigit{2}}{\isacharcomma}{\kern0pt}{\isadigit{1}}{\isacharparenright}{\kern0pt}{\isacharcomma}{\kern0pt}\isanewline
\ \ \ \ \ \ \ \ \ \ \ \ \ \ \ \ \ \ \ \ \ \ \ \ \ \ \ \ \ \ \ \ \ \ \ \ \ \ \ \ \ \ And{\isacharparenleft}{\kern0pt}pair{\isacharunderscore}{\kern0pt}fm{\isacharparenleft}{\kern0pt}y{\isacharhash}{\kern0pt}{\isacharplus}{\kern0pt}{\isadigit{2}}{\isacharcomma}{\kern0pt}o{\isacharhash}{\kern0pt}{\isacharplus}{\kern0pt}{\isadigit{2}}{\isacharcomma}{\kern0pt}{\isadigit{0}}{\isacharparenright}{\kern0pt}{\isacharcomma}{\kern0pt}upair{\isacharunderscore}{\kern0pt}fm{\isacharparenleft}{\kern0pt}{\isadigit{1}}{\isacharcomma}{\kern0pt}{\isadigit{0}}{\isacharcomma}{\kern0pt}z{\isacharhash}{\kern0pt}{\isacharplus}{\kern0pt}{\isadigit{2}}{\isacharparenright}{\kern0pt}{\isacharparenright}{\kern0pt}{\isacharparenright}{\kern0pt}{\isacharparenright}{\kern0pt}{\isacharparenright}{\kern0pt}{\isachardoublequoteclose}\ \isanewline
\isanewline
\isacommand{lemma}\isamarkupfalse%
\ upair{\isacharunderscore}{\kern0pt}name{\isacharunderscore}{\kern0pt}fm{\isacharunderscore}{\kern0pt}type{\isacharbrackleft}{\kern0pt}TC{\isacharbrackright}{\kern0pt}\ {\isacharcolon}{\kern0pt}\isanewline
\ \ \ \ {\isachardoublequoteopen}{\isasymlbrakk}\ s{\isasymin}nat{\isacharsemicolon}{\kern0pt}x{\isasymin}nat{\isacharsemicolon}{\kern0pt}y{\isasymin}nat{\isacharsemicolon}{\kern0pt}o{\isasymin}nat{\isasymrbrakk}\ {\isasymLongrightarrow}\ upair{\isacharunderscore}{\kern0pt}name{\isacharunderscore}{\kern0pt}fm{\isacharparenleft}{\kern0pt}s{\isacharcomma}{\kern0pt}x{\isacharcomma}{\kern0pt}y{\isacharcomma}{\kern0pt}o{\isacharparenright}{\kern0pt}{\isasymin}formula{\isachardoublequoteclose}\isanewline
%
\isadelimproof
\ \ %
\endisadelimproof
%
\isatagproof
\isacommand{unfolding}\isamarkupfalse%
\ upair{\isacharunderscore}{\kern0pt}name{\isacharunderscore}{\kern0pt}fm{\isacharunderscore}{\kern0pt}def\ \isacommand{by}\isamarkupfalse%
\ simp%
\endisatagproof
{\isafoldproof}%
%
\isadelimproof
\isanewline
%
\endisadelimproof
\isanewline
\isacommand{lemma}\isamarkupfalse%
\ sats{\isacharunderscore}{\kern0pt}upair{\isacharunderscore}{\kern0pt}name{\isacharunderscore}{\kern0pt}fm\ {\isacharcolon}{\kern0pt}\isanewline
\ \ \isakeyword{assumes}\ {\isachardoublequoteopen}x{\isasymin}nat{\isachardoublequoteclose}\ {\isachardoublequoteopen}y{\isasymin}nat{\isachardoublequoteclose}\ {\isachardoublequoteopen}z{\isasymin}nat{\isachardoublequoteclose}\ {\isachardoublequoteopen}o{\isasymin}nat{\isachardoublequoteclose}\ {\isachardoublequoteopen}env{\isasymin}list{\isacharparenleft}{\kern0pt}M{\isacharparenright}{\kern0pt}{\isachardoublequoteclose}{\isachardoublequoteopen}nth{\isacharparenleft}{\kern0pt}o{\isacharcomma}{\kern0pt}env{\isacharparenright}{\kern0pt}{\isacharequal}{\kern0pt}one{\isachardoublequoteclose}\ \isanewline
\ \ \isakeyword{shows}\ \isanewline
\ \ \ \ {\isachardoublequoteopen}sats{\isacharparenleft}{\kern0pt}M{\isacharcomma}{\kern0pt}upair{\isacharunderscore}{\kern0pt}name{\isacharunderscore}{\kern0pt}fm{\isacharparenleft}{\kern0pt}x{\isacharcomma}{\kern0pt}y{\isacharcomma}{\kern0pt}o{\isacharcomma}{\kern0pt}z{\isacharparenright}{\kern0pt}{\isacharcomma}{\kern0pt}env{\isacharparenright}{\kern0pt}\ {\isasymlongleftrightarrow}\ is{\isacharunderscore}{\kern0pt}upair{\isacharunderscore}{\kern0pt}name{\isacharparenleft}{\kern0pt}nth{\isacharparenleft}{\kern0pt}x{\isacharcomma}{\kern0pt}env{\isacharparenright}{\kern0pt}{\isacharcomma}{\kern0pt}nth{\isacharparenleft}{\kern0pt}y{\isacharcomma}{\kern0pt}env{\isacharparenright}{\kern0pt}{\isacharcomma}{\kern0pt}nth{\isacharparenleft}{\kern0pt}z{\isacharcomma}{\kern0pt}env{\isacharparenright}{\kern0pt}{\isacharparenright}{\kern0pt}{\isachardoublequoteclose}\isanewline
%
\isadelimproof
\ \ %
\endisadelimproof
%
\isatagproof
\isacommand{unfolding}\isamarkupfalse%
\ upair{\isacharunderscore}{\kern0pt}name{\isacharunderscore}{\kern0pt}fm{\isacharunderscore}{\kern0pt}def\ is{\isacharunderscore}{\kern0pt}upair{\isacharunderscore}{\kern0pt}name{\isacharunderscore}{\kern0pt}def\ \isacommand{using}\isamarkupfalse%
\ assms\ \isacommand{by}\isamarkupfalse%
\ simp%
\endisatagproof
{\isafoldproof}%
%
\isadelimproof
\isanewline
%
\endisadelimproof
\isanewline
\isacommand{definition}\isamarkupfalse%
\isanewline
\ \ opair{\isacharunderscore}{\kern0pt}name\ {\isacharcolon}{\kern0pt}{\isacharcolon}{\kern0pt}\ {\isachardoublequoteopen}i\ {\isasymRightarrow}\ i\ {\isasymRightarrow}\ i{\isachardoublequoteclose}\ \isakeyword{where}\isanewline
\ \ {\isachardoublequoteopen}opair{\isacharunderscore}{\kern0pt}name{\isacharparenleft}{\kern0pt}{\isasymtau}{\isacharcomma}{\kern0pt}{\isasymrho}{\isacharparenright}{\kern0pt}\ {\isasymequiv}\ upair{\isacharunderscore}{\kern0pt}name{\isacharparenleft}{\kern0pt}upair{\isacharunderscore}{\kern0pt}name{\isacharparenleft}{\kern0pt}{\isasymtau}{\isacharcomma}{\kern0pt}{\isasymtau}{\isacharparenright}{\kern0pt}{\isacharcomma}{\kern0pt}upair{\isacharunderscore}{\kern0pt}name{\isacharparenleft}{\kern0pt}{\isasymtau}{\isacharcomma}{\kern0pt}{\isasymrho}{\isacharparenright}{\kern0pt}{\isacharparenright}{\kern0pt}{\isachardoublequoteclose}\isanewline
\isanewline
\isacommand{definition}\isamarkupfalse%
\isanewline
\ \ is{\isacharunderscore}{\kern0pt}opair{\isacharunderscore}{\kern0pt}name\ {\isacharcolon}{\kern0pt}{\isacharcolon}{\kern0pt}\ {\isachardoublequoteopen}{\isacharbrackleft}{\kern0pt}i{\isacharcomma}{\kern0pt}i{\isacharcomma}{\kern0pt}i{\isacharbrackright}{\kern0pt}\ {\isasymRightarrow}\ o{\isachardoublequoteclose}\ \isakeyword{where}\isanewline
\ \ {\isachardoublequoteopen}is{\isacharunderscore}{\kern0pt}opair{\isacharunderscore}{\kern0pt}name{\isacharparenleft}{\kern0pt}x{\isacharcomma}{\kern0pt}y{\isacharcomma}{\kern0pt}z{\isacharparenright}{\kern0pt}\ {\isasymequiv}\ {\isasymexists}upxx{\isasymin}M{\isachardot}{\kern0pt}\ {\isasymexists}upxy{\isasymin}M{\isachardot}{\kern0pt}\ is{\isacharunderscore}{\kern0pt}upair{\isacharunderscore}{\kern0pt}name{\isacharparenleft}{\kern0pt}x{\isacharcomma}{\kern0pt}x{\isacharcomma}{\kern0pt}upxx{\isacharparenright}{\kern0pt}\ {\isasymand}\ is{\isacharunderscore}{\kern0pt}upair{\isacharunderscore}{\kern0pt}name{\isacharparenleft}{\kern0pt}x{\isacharcomma}{\kern0pt}y{\isacharcomma}{\kern0pt}upxy{\isacharparenright}{\kern0pt}\isanewline
\ \ \ \ \ \ \ \ \ \ \ \ \ \ \ \ \ \ \ \ \ \ \ \ \ \ \ \ \ \ \ \ \ \ \ \ \ \ \ \ \ \ {\isasymand}\ is{\isacharunderscore}{\kern0pt}upair{\isacharunderscore}{\kern0pt}name{\isacharparenleft}{\kern0pt}upxx{\isacharcomma}{\kern0pt}upxy{\isacharcomma}{\kern0pt}z{\isacharparenright}{\kern0pt}{\isachardoublequoteclose}\ \isanewline
\isanewline
\isacommand{lemma}\isamarkupfalse%
\ opair{\isacharunderscore}{\kern0pt}name{\isacharunderscore}{\kern0pt}abs\ {\isacharcolon}{\kern0pt}\ \isanewline
\ \ \isakeyword{assumes}\ {\isachardoublequoteopen}x{\isasymin}M{\isachardoublequoteclose}\ {\isachardoublequoteopen}y{\isasymin}M{\isachardoublequoteclose}\ {\isachardoublequoteopen}z{\isasymin}M{\isachardoublequoteclose}\ \isanewline
\ \ \isakeyword{shows}\ {\isachardoublequoteopen}is{\isacharunderscore}{\kern0pt}opair{\isacharunderscore}{\kern0pt}name{\isacharparenleft}{\kern0pt}x{\isacharcomma}{\kern0pt}y{\isacharcomma}{\kern0pt}z{\isacharparenright}{\kern0pt}\ {\isasymlongleftrightarrow}\ z\ {\isacharequal}{\kern0pt}\ opair{\isacharunderscore}{\kern0pt}name{\isacharparenleft}{\kern0pt}x{\isacharcomma}{\kern0pt}y{\isacharparenright}{\kern0pt}{\isachardoublequoteclose}\ \isanewline
%
\isadelimproof
\ \ %
\endisadelimproof
%
\isatagproof
\isacommand{unfolding}\isamarkupfalse%
\ is{\isacharunderscore}{\kern0pt}opair{\isacharunderscore}{\kern0pt}name{\isacharunderscore}{\kern0pt}def\ opair{\isacharunderscore}{\kern0pt}name{\isacharunderscore}{\kern0pt}def\ \isacommand{using}\isamarkupfalse%
\ assms\ upair{\isacharunderscore}{\kern0pt}name{\isacharunderscore}{\kern0pt}abs\ upair{\isacharunderscore}{\kern0pt}name{\isacharunderscore}{\kern0pt}closed\ \isacommand{by}\isamarkupfalse%
\ simp%
\endisatagproof
{\isafoldproof}%
%
\isadelimproof
\isanewline
%
\endisadelimproof
\isanewline
\isacommand{lemma}\isamarkupfalse%
\ opair{\isacharunderscore}{\kern0pt}name{\isacharunderscore}{\kern0pt}closed\ {\isacharcolon}{\kern0pt}\isanewline
\ \ {\isachardoublequoteopen}{\isasymlbrakk}\ x{\isasymin}M{\isacharsemicolon}{\kern0pt}\ y{\isasymin}M\ {\isasymrbrakk}\ {\isasymLongrightarrow}\ opair{\isacharunderscore}{\kern0pt}name{\isacharparenleft}{\kern0pt}x{\isacharcomma}{\kern0pt}y{\isacharparenright}{\kern0pt}{\isasymin}M{\isachardoublequoteclose}\ \isanewline
%
\isadelimproof
\ \ %
\endisadelimproof
%
\isatagproof
\isacommand{unfolding}\isamarkupfalse%
\ opair{\isacharunderscore}{\kern0pt}name{\isacharunderscore}{\kern0pt}def\ \isacommand{using}\isamarkupfalse%
\ upair{\isacharunderscore}{\kern0pt}name{\isacharunderscore}{\kern0pt}closed\ \isacommand{by}\isamarkupfalse%
\ simp%
\endisatagproof
{\isafoldproof}%
%
\isadelimproof
\isanewline
%
\endisadelimproof
\isanewline
\isacommand{definition}\isamarkupfalse%
\isanewline
\ \ opair{\isacharunderscore}{\kern0pt}name{\isacharunderscore}{\kern0pt}fm\ {\isacharcolon}{\kern0pt}{\isacharcolon}{\kern0pt}\ {\isachardoublequoteopen}{\isacharbrackleft}{\kern0pt}i{\isacharcomma}{\kern0pt}i{\isacharcomma}{\kern0pt}i{\isacharcomma}{\kern0pt}i{\isacharbrackright}{\kern0pt}\ {\isasymRightarrow}\ i{\isachardoublequoteclose}\ \isakeyword{where}\isanewline
\ \ {\isachardoublequoteopen}opair{\isacharunderscore}{\kern0pt}name{\isacharunderscore}{\kern0pt}fm{\isacharparenleft}{\kern0pt}x{\isacharcomma}{\kern0pt}y{\isacharcomma}{\kern0pt}o{\isacharcomma}{\kern0pt}z{\isacharparenright}{\kern0pt}\ {\isasymequiv}\ Exists{\isacharparenleft}{\kern0pt}Exists{\isacharparenleft}{\kern0pt}And{\isacharparenleft}{\kern0pt}upair{\isacharunderscore}{\kern0pt}name{\isacharunderscore}{\kern0pt}fm{\isacharparenleft}{\kern0pt}x{\isacharhash}{\kern0pt}{\isacharplus}{\kern0pt}{\isadigit{2}}{\isacharcomma}{\kern0pt}x{\isacharhash}{\kern0pt}{\isacharplus}{\kern0pt}{\isadigit{2}}{\isacharcomma}{\kern0pt}o{\isacharhash}{\kern0pt}{\isacharplus}{\kern0pt}{\isadigit{2}}{\isacharcomma}{\kern0pt}{\isadigit{1}}{\isacharparenright}{\kern0pt}{\isacharcomma}{\kern0pt}\isanewline
\ \ \ \ \ \ \ \ \ \ \ \ \ \ \ \ \ \ \ \ And{\isacharparenleft}{\kern0pt}upair{\isacharunderscore}{\kern0pt}name{\isacharunderscore}{\kern0pt}fm{\isacharparenleft}{\kern0pt}x{\isacharhash}{\kern0pt}{\isacharplus}{\kern0pt}{\isadigit{2}}{\isacharcomma}{\kern0pt}y{\isacharhash}{\kern0pt}{\isacharplus}{\kern0pt}{\isadigit{2}}{\isacharcomma}{\kern0pt}o{\isacharhash}{\kern0pt}{\isacharplus}{\kern0pt}{\isadigit{2}}{\isacharcomma}{\kern0pt}{\isadigit{0}}{\isacharparenright}{\kern0pt}{\isacharcomma}{\kern0pt}upair{\isacharunderscore}{\kern0pt}name{\isacharunderscore}{\kern0pt}fm{\isacharparenleft}{\kern0pt}{\isadigit{1}}{\isacharcomma}{\kern0pt}{\isadigit{0}}{\isacharcomma}{\kern0pt}o{\isacharhash}{\kern0pt}{\isacharplus}{\kern0pt}{\isadigit{2}}{\isacharcomma}{\kern0pt}z{\isacharhash}{\kern0pt}{\isacharplus}{\kern0pt}{\isadigit{2}}{\isacharparenright}{\kern0pt}{\isacharparenright}{\kern0pt}{\isacharparenright}{\kern0pt}{\isacharparenright}{\kern0pt}{\isacharparenright}{\kern0pt}{\isachardoublequoteclose}\ \isanewline
\isanewline
\isacommand{lemma}\isamarkupfalse%
\ opair{\isacharunderscore}{\kern0pt}name{\isacharunderscore}{\kern0pt}fm{\isacharunderscore}{\kern0pt}type{\isacharbrackleft}{\kern0pt}TC{\isacharbrackright}{\kern0pt}\ {\isacharcolon}{\kern0pt}\isanewline
\ \ \ \ {\isachardoublequoteopen}{\isasymlbrakk}\ s{\isasymin}nat{\isacharsemicolon}{\kern0pt}x{\isasymin}nat{\isacharsemicolon}{\kern0pt}y{\isasymin}nat{\isacharsemicolon}{\kern0pt}o{\isasymin}nat{\isasymrbrakk}\ {\isasymLongrightarrow}\ opair{\isacharunderscore}{\kern0pt}name{\isacharunderscore}{\kern0pt}fm{\isacharparenleft}{\kern0pt}s{\isacharcomma}{\kern0pt}x{\isacharcomma}{\kern0pt}y{\isacharcomma}{\kern0pt}o{\isacharparenright}{\kern0pt}{\isasymin}formula{\isachardoublequoteclose}\isanewline
%
\isadelimproof
\ \ %
\endisadelimproof
%
\isatagproof
\isacommand{unfolding}\isamarkupfalse%
\ opair{\isacharunderscore}{\kern0pt}name{\isacharunderscore}{\kern0pt}fm{\isacharunderscore}{\kern0pt}def\ \isacommand{by}\isamarkupfalse%
\ simp%
\endisatagproof
{\isafoldproof}%
%
\isadelimproof
\isanewline
%
\endisadelimproof
\isanewline
\isacommand{lemma}\isamarkupfalse%
\ sats{\isacharunderscore}{\kern0pt}opair{\isacharunderscore}{\kern0pt}name{\isacharunderscore}{\kern0pt}fm\ {\isacharcolon}{\kern0pt}\isanewline
\ \ \isakeyword{assumes}\ {\isachardoublequoteopen}x{\isasymin}nat{\isachardoublequoteclose}\ {\isachardoublequoteopen}y{\isasymin}nat{\isachardoublequoteclose}\ {\isachardoublequoteopen}z{\isasymin}nat{\isachardoublequoteclose}\ {\isachardoublequoteopen}o{\isasymin}nat{\isachardoublequoteclose}\ {\isachardoublequoteopen}env{\isasymin}list{\isacharparenleft}{\kern0pt}M{\isacharparenright}{\kern0pt}{\isachardoublequoteclose}{\isachardoublequoteopen}nth{\isacharparenleft}{\kern0pt}o{\isacharcomma}{\kern0pt}env{\isacharparenright}{\kern0pt}{\isacharequal}{\kern0pt}one{\isachardoublequoteclose}\ \isanewline
\ \ \isakeyword{shows}\ \isanewline
\ \ \ \ {\isachardoublequoteopen}sats{\isacharparenleft}{\kern0pt}M{\isacharcomma}{\kern0pt}opair{\isacharunderscore}{\kern0pt}name{\isacharunderscore}{\kern0pt}fm{\isacharparenleft}{\kern0pt}x{\isacharcomma}{\kern0pt}y{\isacharcomma}{\kern0pt}o{\isacharcomma}{\kern0pt}z{\isacharparenright}{\kern0pt}{\isacharcomma}{\kern0pt}env{\isacharparenright}{\kern0pt}\ {\isasymlongleftrightarrow}\ is{\isacharunderscore}{\kern0pt}opair{\isacharunderscore}{\kern0pt}name{\isacharparenleft}{\kern0pt}nth{\isacharparenleft}{\kern0pt}x{\isacharcomma}{\kern0pt}env{\isacharparenright}{\kern0pt}{\isacharcomma}{\kern0pt}nth{\isacharparenleft}{\kern0pt}y{\isacharcomma}{\kern0pt}env{\isacharparenright}{\kern0pt}{\isacharcomma}{\kern0pt}nth{\isacharparenleft}{\kern0pt}z{\isacharcomma}{\kern0pt}env{\isacharparenright}{\kern0pt}{\isacharparenright}{\kern0pt}{\isachardoublequoteclose}\isanewline
%
\isadelimproof
\ \ %
\endisadelimproof
%
\isatagproof
\isacommand{unfolding}\isamarkupfalse%
\ opair{\isacharunderscore}{\kern0pt}name{\isacharunderscore}{\kern0pt}fm{\isacharunderscore}{\kern0pt}def\ is{\isacharunderscore}{\kern0pt}opair{\isacharunderscore}{\kern0pt}name{\isacharunderscore}{\kern0pt}def\ \isacommand{using}\isamarkupfalse%
\ assms\ sats{\isacharunderscore}{\kern0pt}upair{\isacharunderscore}{\kern0pt}name{\isacharunderscore}{\kern0pt}fm\ \isacommand{by}\isamarkupfalse%
\ simp%
\endisatagproof
{\isafoldproof}%
%
\isadelimproof
\isanewline
%
\endisadelimproof
\isanewline
\isacommand{lemma}\isamarkupfalse%
\ val{\isacharunderscore}{\kern0pt}upair{\isacharunderscore}{\kern0pt}name\ {\isacharcolon}{\kern0pt}\ {\isachardoublequoteopen}val{\isacharparenleft}{\kern0pt}G{\isacharcomma}{\kern0pt}upair{\isacharunderscore}{\kern0pt}name{\isacharparenleft}{\kern0pt}{\isasymtau}{\isacharcomma}{\kern0pt}{\isasymrho}{\isacharparenright}{\kern0pt}{\isacharparenright}{\kern0pt}\ {\isacharequal}{\kern0pt}\ {\isacharbraceleft}{\kern0pt}val{\isacharparenleft}{\kern0pt}G{\isacharcomma}{\kern0pt}{\isasymtau}{\isacharparenright}{\kern0pt}{\isacharcomma}{\kern0pt}val{\isacharparenleft}{\kern0pt}G{\isacharcomma}{\kern0pt}{\isasymrho}{\isacharparenright}{\kern0pt}{\isacharbraceright}{\kern0pt}{\isachardoublequoteclose}\isanewline
%
\isadelimproof
\ \ %
\endisadelimproof
%
\isatagproof
\isacommand{unfolding}\isamarkupfalse%
\ upair{\isacharunderscore}{\kern0pt}name{\isacharunderscore}{\kern0pt}def\ \isacommand{using}\isamarkupfalse%
\ val{\isacharunderscore}{\kern0pt}Upair\ \ generic\ one{\isacharunderscore}{\kern0pt}in{\isacharunderscore}{\kern0pt}G\ one{\isacharunderscore}{\kern0pt}in{\isacharunderscore}{\kern0pt}P\ \isacommand{by}\isamarkupfalse%
\ simp%
\endisatagproof
{\isafoldproof}%
%
\isadelimproof
\isanewline
%
\endisadelimproof
\ \ \ \ \isanewline
\isacommand{lemma}\isamarkupfalse%
\ val{\isacharunderscore}{\kern0pt}opair{\isacharunderscore}{\kern0pt}name\ {\isacharcolon}{\kern0pt}\ {\isachardoublequoteopen}val{\isacharparenleft}{\kern0pt}G{\isacharcomma}{\kern0pt}opair{\isacharunderscore}{\kern0pt}name{\isacharparenleft}{\kern0pt}{\isasymtau}{\isacharcomma}{\kern0pt}{\isasymrho}{\isacharparenright}{\kern0pt}{\isacharparenright}{\kern0pt}\ {\isacharequal}{\kern0pt}\ {\isasymlangle}val{\isacharparenleft}{\kern0pt}G{\isacharcomma}{\kern0pt}{\isasymtau}{\isacharparenright}{\kern0pt}{\isacharcomma}{\kern0pt}val{\isacharparenleft}{\kern0pt}G{\isacharcomma}{\kern0pt}{\isasymrho}{\isacharparenright}{\kern0pt}{\isasymrangle}{\isachardoublequoteclose}\isanewline
%
\isadelimproof
\ \ %
\endisadelimproof
%
\isatagproof
\isacommand{unfolding}\isamarkupfalse%
\ opair{\isacharunderscore}{\kern0pt}name{\isacharunderscore}{\kern0pt}def\ Pair{\isacharunderscore}{\kern0pt}def\ \isacommand{using}\isamarkupfalse%
\ val{\isacharunderscore}{\kern0pt}upair{\isacharunderscore}{\kern0pt}name\ \ \isacommand{by}\isamarkupfalse%
\ simp%
\endisatagproof
{\isafoldproof}%
%
\isadelimproof
\isanewline
%
\endisadelimproof
\ \ \ \ \isanewline
\isacommand{lemma}\isamarkupfalse%
\ val{\isacharunderscore}{\kern0pt}RepFun{\isacharunderscore}{\kern0pt}one{\isacharcolon}{\kern0pt}\ {\isachardoublequoteopen}val{\isacharparenleft}{\kern0pt}G{\isacharcomma}{\kern0pt}{\isacharbraceleft}{\kern0pt}{\isasymlangle}f{\isacharparenleft}{\kern0pt}x{\isacharparenright}{\kern0pt}{\isacharcomma}{\kern0pt}one{\isasymrangle}\ {\isachardot}{\kern0pt}\ x{\isasymin}a{\isacharbraceright}{\kern0pt}{\isacharparenright}{\kern0pt}\ {\isacharequal}{\kern0pt}\ {\isacharbraceleft}{\kern0pt}val{\isacharparenleft}{\kern0pt}G{\isacharcomma}{\kern0pt}f{\isacharparenleft}{\kern0pt}x{\isacharparenright}{\kern0pt}{\isacharparenright}{\kern0pt}\ {\isachardot}{\kern0pt}\ x{\isasymin}a{\isacharbraceright}{\kern0pt}{\isachardoublequoteclose}\isanewline
%
\isadelimproof
%
\endisadelimproof
%
\isatagproof
\isacommand{proof}\isamarkupfalse%
\ {\isacharminus}{\kern0pt}\isanewline
\ \ \isacommand{let}\isamarkupfalse%
\ {\isacharquery}{\kern0pt}A\ {\isacharequal}{\kern0pt}\ {\isachardoublequoteopen}{\isacharbraceleft}{\kern0pt}f{\isacharparenleft}{\kern0pt}x{\isacharparenright}{\kern0pt}\ {\isachardot}{\kern0pt}\ x\ {\isasymin}\ a{\isacharbraceright}{\kern0pt}{\isachardoublequoteclose}\isanewline
\ \ \isacommand{let}\isamarkupfalse%
\ {\isacharquery}{\kern0pt}Q\ {\isacharequal}{\kern0pt}\ {\isachardoublequoteopen}{\isasymlambda}{\isasymlangle}x{\isacharcomma}{\kern0pt}p{\isasymrangle}\ {\isachardot}{\kern0pt}\ p\ {\isacharequal}{\kern0pt}\ one{\isachardoublequoteclose}\isanewline
\ \ \isacommand{have}\isamarkupfalse%
\ {\isachardoublequoteopen}one\ {\isasymin}\ P{\isasyminter}G{\isachardoublequoteclose}\ \isacommand{using}\isamarkupfalse%
\ generic\ one{\isacharunderscore}{\kern0pt}in{\isacharunderscore}{\kern0pt}G\ one{\isacharunderscore}{\kern0pt}in{\isacharunderscore}{\kern0pt}P\ \isacommand{by}\isamarkupfalse%
\ simp\isanewline
\ \ \isacommand{have}\isamarkupfalse%
\ {\isachardoublequoteopen}{\isacharbraceleft}{\kern0pt}{\isasymlangle}f{\isacharparenleft}{\kern0pt}x{\isacharparenright}{\kern0pt}{\isacharcomma}{\kern0pt}one{\isasymrangle}\ {\isachardot}{\kern0pt}\ x\ {\isasymin}\ a{\isacharbraceright}{\kern0pt}\ {\isacharequal}{\kern0pt}\ {\isacharbraceleft}{\kern0pt}t\ {\isasymin}\ {\isacharquery}{\kern0pt}A\ {\isasymtimes}\ P\ {\isachardot}{\kern0pt}\ {\isacharquery}{\kern0pt}Q{\isacharparenleft}{\kern0pt}t{\isacharparenright}{\kern0pt}{\isacharbraceright}{\kern0pt}{\isachardoublequoteclose}\ \isanewline
\ \ \ \ \isacommand{using}\isamarkupfalse%
\ one{\isacharunderscore}{\kern0pt}in{\isacharunderscore}{\kern0pt}P\ \isacommand{by}\isamarkupfalse%
\ force\isanewline
\ \ \isacommand{then}\isamarkupfalse%
\isanewline
\ \ \isacommand{have}\isamarkupfalse%
\ {\isachardoublequoteopen}val{\isacharparenleft}{\kern0pt}G{\isacharcomma}{\kern0pt}{\isacharbraceleft}{\kern0pt}{\isasymlangle}f{\isacharparenleft}{\kern0pt}x{\isacharparenright}{\kern0pt}{\isacharcomma}{\kern0pt}one{\isasymrangle}\ \ {\isachardot}{\kern0pt}\ x\ {\isasymin}\ a{\isacharbraceright}{\kern0pt}{\isacharparenright}{\kern0pt}\ {\isacharequal}{\kern0pt}\ val{\isacharparenleft}{\kern0pt}G{\isacharcomma}{\kern0pt}{\isacharbraceleft}{\kern0pt}t\ {\isasymin}\ {\isacharquery}{\kern0pt}A\ {\isasymtimes}\ P\ {\isachardot}{\kern0pt}\ {\isacharquery}{\kern0pt}Q{\isacharparenleft}{\kern0pt}t{\isacharparenright}{\kern0pt}{\isacharbraceright}{\kern0pt}{\isacharparenright}{\kern0pt}{\isachardoublequoteclose}\isanewline
\ \ \ \ \isacommand{by}\isamarkupfalse%
\ simp\isanewline
\ \ \isacommand{also}\isamarkupfalse%
\isanewline
\ \ \isacommand{have}\isamarkupfalse%
\ {\isachardoublequoteopen}{\isachardot}{\kern0pt}{\isachardot}{\kern0pt}{\isachardot}{\kern0pt}\ {\isacharequal}{\kern0pt}\ {\isacharbraceleft}{\kern0pt}val{\isacharparenleft}{\kern0pt}G{\isacharcomma}{\kern0pt}t{\isacharparenright}{\kern0pt}\ {\isachardot}{\kern0pt}{\isachardot}{\kern0pt}\ t\ {\isasymin}\ {\isacharquery}{\kern0pt}A\ {\isacharcomma}{\kern0pt}\ {\isasymexists}p{\isasymin}P{\isasyminter}G\ {\isachardot}{\kern0pt}\ {\isacharquery}{\kern0pt}Q{\isacharparenleft}{\kern0pt}{\isasymlangle}t{\isacharcomma}{\kern0pt}p{\isasymrangle}{\isacharparenright}{\kern0pt}{\isacharbraceright}{\kern0pt}{\isachardoublequoteclose}\isanewline
\ \ \ \ \isacommand{using}\isamarkupfalse%
\ val{\isacharunderscore}{\kern0pt}of{\isacharunderscore}{\kern0pt}name{\isacharunderscore}{\kern0pt}alt\ \isacommand{by}\isamarkupfalse%
\ simp\isanewline
\ \ \isacommand{also}\isamarkupfalse%
\isanewline
\ \ \isacommand{have}\isamarkupfalse%
\ {\isachardoublequoteopen}{\isachardot}{\kern0pt}{\isachardot}{\kern0pt}{\isachardot}{\kern0pt}\ {\isacharequal}{\kern0pt}\ {\isacharbraceleft}{\kern0pt}val{\isacharparenleft}{\kern0pt}G{\isacharcomma}{\kern0pt}t{\isacharparenright}{\kern0pt}\ {\isachardot}{\kern0pt}\ t\ {\isasymin}\ {\isacharquery}{\kern0pt}A\ {\isacharbraceright}{\kern0pt}{\isachardoublequoteclose}\isanewline
\ \ \ \ \isacommand{using}\isamarkupfalse%
\ {\isacartoucheopen}one{\isasymin}P{\isasyminter}G{\isacartoucheclose}\ \isacommand{by}\isamarkupfalse%
\ force\isanewline
\ \ \isacommand{also}\isamarkupfalse%
\isanewline
\ \ \isacommand{have}\isamarkupfalse%
\ {\isachardoublequoteopen}{\isachardot}{\kern0pt}{\isachardot}{\kern0pt}{\isachardot}{\kern0pt}\ {\isacharequal}{\kern0pt}\ {\isacharbraceleft}{\kern0pt}val{\isacharparenleft}{\kern0pt}G{\isacharcomma}{\kern0pt}f{\isacharparenleft}{\kern0pt}x{\isacharparenright}{\kern0pt}{\isacharparenright}{\kern0pt}\ {\isachardot}{\kern0pt}\ x\ {\isasymin}\ a{\isacharbraceright}{\kern0pt}{\isachardoublequoteclose}\isanewline
\ \ \ \ \isacommand{by}\isamarkupfalse%
\ auto\isanewline
\ \ \isacommand{finally}\isamarkupfalse%
\ \isacommand{show}\isamarkupfalse%
\ {\isacharquery}{\kern0pt}thesis\ \isacommand{by}\isamarkupfalse%
\ simp\isanewline
\isacommand{qed}\isamarkupfalse%
%
\endisatagproof
{\isafoldproof}%
%
\isadelimproof
%
\endisadelimproof
%
\isadelimdocument
%
\endisadelimdocument
%
\isatagdocument
%
\isamarkupsubsection{$M[G]$ is a transitive model of ZF%
}
\isamarkuptrue%
%
\endisatagdocument
{\isafolddocument}%
%
\isadelimdocument
%
\endisadelimdocument
\isacommand{interpretation}\isamarkupfalse%
\ mgzf{\isacharcolon}{\kern0pt}\ M{\isacharunderscore}{\kern0pt}ZF{\isacharunderscore}{\kern0pt}trans\ {\isachardoublequoteopen}M{\isacharbrackleft}{\kern0pt}G{\isacharbrackright}{\kern0pt}{\isachardoublequoteclose}\isanewline
%
\isadelimproof
\ \ %
\endisadelimproof
%
\isatagproof
\isacommand{using}\isamarkupfalse%
\ Transset{\isacharunderscore}{\kern0pt}MG\ generic\ pairing{\isacharunderscore}{\kern0pt}in{\isacharunderscore}{\kern0pt}MG\ Union{\isacharunderscore}{\kern0pt}MG\ \isanewline
\ \ \ \ extensionality{\isacharunderscore}{\kern0pt}in{\isacharunderscore}{\kern0pt}MG\ power{\isacharunderscore}{\kern0pt}in{\isacharunderscore}{\kern0pt}MG\ foundation{\isacharunderscore}{\kern0pt}in{\isacharunderscore}{\kern0pt}MG\ \ \isanewline
\ \ \ \ strong{\isacharunderscore}{\kern0pt}replacement{\isacharunderscore}{\kern0pt}in{\isacharunderscore}{\kern0pt}MG\ separation{\isacharunderscore}{\kern0pt}in{\isacharunderscore}{\kern0pt}MG\ infinity{\isacharunderscore}{\kern0pt}in{\isacharunderscore}{\kern0pt}MG\isanewline
\ \ \isacommand{by}\isamarkupfalse%
\ unfold{\isacharunderscore}{\kern0pt}locales\ simp{\isacharunderscore}{\kern0pt}all%
\endisatagproof
{\isafoldproof}%
%
\isadelimproof
\isanewline
%
\endisadelimproof
\isanewline
\isanewline
\isacommand{definition}\isamarkupfalse%
\isanewline
\ \ is{\isacharunderscore}{\kern0pt}opname{\isacharunderscore}{\kern0pt}check\ {\isacharcolon}{\kern0pt}{\isacharcolon}{\kern0pt}\ {\isachardoublequoteopen}{\isacharbrackleft}{\kern0pt}i{\isacharcomma}{\kern0pt}i{\isacharcomma}{\kern0pt}i{\isacharbrackright}{\kern0pt}\ {\isasymRightarrow}\ o{\isachardoublequoteclose}\ \isakeyword{where}\isanewline
\ \ {\isachardoublequoteopen}is{\isacharunderscore}{\kern0pt}opname{\isacharunderscore}{\kern0pt}check{\isacharparenleft}{\kern0pt}s{\isacharcomma}{\kern0pt}x{\isacharcomma}{\kern0pt}y{\isacharparenright}{\kern0pt}\ {\isasymequiv}\ {\isasymexists}chx{\isasymin}M{\isachardot}{\kern0pt}\ {\isasymexists}sx{\isasymin}M{\isachardot}{\kern0pt}\ is{\isacharunderscore}{\kern0pt}check{\isacharparenleft}{\kern0pt}x{\isacharcomma}{\kern0pt}chx{\isacharparenright}{\kern0pt}\ {\isasymand}\ fun{\isacharunderscore}{\kern0pt}apply{\isacharparenleft}{\kern0pt}{\isacharhash}{\kern0pt}{\isacharhash}{\kern0pt}M{\isacharcomma}{\kern0pt}s{\isacharcomma}{\kern0pt}x{\isacharcomma}{\kern0pt}sx{\isacharparenright}{\kern0pt}\ {\isasymand}\ \isanewline
\ \ \ \ \ \ \ \ \ \ \ \ \ \ \ \ \ \ \ \ \ \ \ \ \ \ \ \ \ is{\isacharunderscore}{\kern0pt}opair{\isacharunderscore}{\kern0pt}name{\isacharparenleft}{\kern0pt}chx{\isacharcomma}{\kern0pt}sx{\isacharcomma}{\kern0pt}y{\isacharparenright}{\kern0pt}{\isachardoublequoteclose}\ \isanewline
\isanewline
\isacommand{definition}\isamarkupfalse%
\isanewline
\ \ opname{\isacharunderscore}{\kern0pt}check{\isacharunderscore}{\kern0pt}fm\ {\isacharcolon}{\kern0pt}{\isacharcolon}{\kern0pt}\ {\isachardoublequoteopen}{\isacharbrackleft}{\kern0pt}i{\isacharcomma}{\kern0pt}i{\isacharcomma}{\kern0pt}i{\isacharcomma}{\kern0pt}i{\isacharbrackright}{\kern0pt}\ {\isasymRightarrow}\ i{\isachardoublequoteclose}\ \isakeyword{where}\isanewline
\ \ {\isachardoublequoteopen}opname{\isacharunderscore}{\kern0pt}check{\isacharunderscore}{\kern0pt}fm{\isacharparenleft}{\kern0pt}s{\isacharcomma}{\kern0pt}x{\isacharcomma}{\kern0pt}y{\isacharcomma}{\kern0pt}o{\isacharparenright}{\kern0pt}\ {\isasymequiv}\ Exists{\isacharparenleft}{\kern0pt}Exists{\isacharparenleft}{\kern0pt}And{\isacharparenleft}{\kern0pt}check{\isacharunderscore}{\kern0pt}fm{\isacharparenleft}{\kern0pt}{\isadigit{2}}{\isacharhash}{\kern0pt}{\isacharplus}{\kern0pt}x{\isacharcomma}{\kern0pt}{\isadigit{2}}{\isacharhash}{\kern0pt}{\isacharplus}{\kern0pt}o{\isacharcomma}{\kern0pt}{\isadigit{1}}{\isacharparenright}{\kern0pt}{\isacharcomma}{\kern0pt}\isanewline
\ \ \ \ \ \ \ \ \ \ \ \ \ \ \ \ \ \ \ \ \ \ \ \ \ \ \ \ \ \ And{\isacharparenleft}{\kern0pt}fun{\isacharunderscore}{\kern0pt}apply{\isacharunderscore}{\kern0pt}fm{\isacharparenleft}{\kern0pt}{\isadigit{2}}{\isacharhash}{\kern0pt}{\isacharplus}{\kern0pt}s{\isacharcomma}{\kern0pt}{\isadigit{2}}{\isacharhash}{\kern0pt}{\isacharplus}{\kern0pt}x{\isacharcomma}{\kern0pt}{\isadigit{0}}{\isacharparenright}{\kern0pt}{\isacharcomma}{\kern0pt}opair{\isacharunderscore}{\kern0pt}name{\isacharunderscore}{\kern0pt}fm{\isacharparenleft}{\kern0pt}{\isadigit{1}}{\isacharcomma}{\kern0pt}{\isadigit{0}}{\isacharcomma}{\kern0pt}{\isadigit{2}}{\isacharhash}{\kern0pt}{\isacharplus}{\kern0pt}o{\isacharcomma}{\kern0pt}{\isadigit{2}}{\isacharhash}{\kern0pt}{\isacharplus}{\kern0pt}y{\isacharparenright}{\kern0pt}{\isacharparenright}{\kern0pt}{\isacharparenright}{\kern0pt}{\isacharparenright}{\kern0pt}{\isacharparenright}{\kern0pt}{\isachardoublequoteclose}\isanewline
\isanewline
\isacommand{lemma}\isamarkupfalse%
\ opname{\isacharunderscore}{\kern0pt}check{\isacharunderscore}{\kern0pt}fm{\isacharunderscore}{\kern0pt}type{\isacharbrackleft}{\kern0pt}TC{\isacharbrackright}{\kern0pt}\ {\isacharcolon}{\kern0pt}\isanewline
\ \ {\isachardoublequoteopen}{\isasymlbrakk}\ s{\isasymin}nat{\isacharsemicolon}{\kern0pt}x{\isasymin}nat{\isacharsemicolon}{\kern0pt}y{\isasymin}nat{\isacharsemicolon}{\kern0pt}o{\isasymin}nat{\isasymrbrakk}\ {\isasymLongrightarrow}\ opname{\isacharunderscore}{\kern0pt}check{\isacharunderscore}{\kern0pt}fm{\isacharparenleft}{\kern0pt}s{\isacharcomma}{\kern0pt}x{\isacharcomma}{\kern0pt}y{\isacharcomma}{\kern0pt}o{\isacharparenright}{\kern0pt}{\isasymin}formula{\isachardoublequoteclose}\isanewline
%
\isadelimproof
\ \ %
\endisadelimproof
%
\isatagproof
\isacommand{unfolding}\isamarkupfalse%
\ opname{\isacharunderscore}{\kern0pt}check{\isacharunderscore}{\kern0pt}fm{\isacharunderscore}{\kern0pt}def\ \isacommand{by}\isamarkupfalse%
\ simp%
\endisatagproof
{\isafoldproof}%
%
\isadelimproof
\isanewline
%
\endisadelimproof
\isanewline
\isacommand{lemma}\isamarkupfalse%
\ sats{\isacharunderscore}{\kern0pt}opname{\isacharunderscore}{\kern0pt}check{\isacharunderscore}{\kern0pt}fm{\isacharcolon}{\kern0pt}\isanewline
\ \ \isakeyword{assumes}\ {\isachardoublequoteopen}x{\isasymin}nat{\isachardoublequoteclose}\ {\isachardoublequoteopen}y{\isasymin}nat{\isachardoublequoteclose}\ {\isachardoublequoteopen}z{\isasymin}nat{\isachardoublequoteclose}\ {\isachardoublequoteopen}o{\isasymin}nat{\isachardoublequoteclose}\ {\isachardoublequoteopen}env{\isasymin}list{\isacharparenleft}{\kern0pt}M{\isacharparenright}{\kern0pt}{\isachardoublequoteclose}\ {\isachardoublequoteopen}nth{\isacharparenleft}{\kern0pt}o{\isacharcomma}{\kern0pt}env{\isacharparenright}{\kern0pt}{\isacharequal}{\kern0pt}one{\isachardoublequoteclose}\ \isanewline
\ \ \ \ \ \ \ \ \ \ {\isachardoublequoteopen}y{\isacharless}{\kern0pt}length{\isacharparenleft}{\kern0pt}env{\isacharparenright}{\kern0pt}{\isachardoublequoteclose}\isanewline
\ \ \isakeyword{shows}\ \isanewline
\ \ \ \ {\isachardoublequoteopen}sats{\isacharparenleft}{\kern0pt}M{\isacharcomma}{\kern0pt}opname{\isacharunderscore}{\kern0pt}check{\isacharunderscore}{\kern0pt}fm{\isacharparenleft}{\kern0pt}x{\isacharcomma}{\kern0pt}y{\isacharcomma}{\kern0pt}z{\isacharcomma}{\kern0pt}o{\isacharparenright}{\kern0pt}{\isacharcomma}{\kern0pt}env{\isacharparenright}{\kern0pt}\ {\isasymlongleftrightarrow}\ is{\isacharunderscore}{\kern0pt}opname{\isacharunderscore}{\kern0pt}check{\isacharparenleft}{\kern0pt}nth{\isacharparenleft}{\kern0pt}x{\isacharcomma}{\kern0pt}env{\isacharparenright}{\kern0pt}{\isacharcomma}{\kern0pt}nth{\isacharparenleft}{\kern0pt}y{\isacharcomma}{\kern0pt}env{\isacharparenright}{\kern0pt}{\isacharcomma}{\kern0pt}nth{\isacharparenleft}{\kern0pt}z{\isacharcomma}{\kern0pt}env{\isacharparenright}{\kern0pt}{\isacharparenright}{\kern0pt}{\isachardoublequoteclose}\isanewline
%
\isadelimproof
\ \ %
\endisadelimproof
%
\isatagproof
\isacommand{unfolding}\isamarkupfalse%
\ opname{\isacharunderscore}{\kern0pt}check{\isacharunderscore}{\kern0pt}fm{\isacharunderscore}{\kern0pt}def\ is{\isacharunderscore}{\kern0pt}opname{\isacharunderscore}{\kern0pt}check{\isacharunderscore}{\kern0pt}def\ \isanewline
\ \ \isacommand{using}\isamarkupfalse%
\ assms\ sats{\isacharunderscore}{\kern0pt}check{\isacharunderscore}{\kern0pt}fm\ sats{\isacharunderscore}{\kern0pt}opair{\isacharunderscore}{\kern0pt}name{\isacharunderscore}{\kern0pt}fm\ one{\isacharunderscore}{\kern0pt}in{\isacharunderscore}{\kern0pt}M\ \isacommand{by}\isamarkupfalse%
\ simp%
\endisatagproof
{\isafoldproof}%
%
\isadelimproof
\isanewline
%
\endisadelimproof
\isanewline
\isanewline
\isacommand{lemma}\isamarkupfalse%
\ opname{\isacharunderscore}{\kern0pt}check{\isacharunderscore}{\kern0pt}abs\ {\isacharcolon}{\kern0pt}\isanewline
\ \ \isakeyword{assumes}\ {\isachardoublequoteopen}s{\isasymin}M{\isachardoublequoteclose}\ {\isachardoublequoteopen}x{\isasymin}M{\isachardoublequoteclose}\ {\isachardoublequoteopen}y{\isasymin}M{\isachardoublequoteclose}\ \isanewline
\ \ \isakeyword{shows}\ {\isachardoublequoteopen}is{\isacharunderscore}{\kern0pt}opname{\isacharunderscore}{\kern0pt}check{\isacharparenleft}{\kern0pt}s{\isacharcomma}{\kern0pt}x{\isacharcomma}{\kern0pt}y{\isacharparenright}{\kern0pt}\ {\isasymlongleftrightarrow}\ y\ {\isacharequal}{\kern0pt}\ opair{\isacharunderscore}{\kern0pt}name{\isacharparenleft}{\kern0pt}check{\isacharparenleft}{\kern0pt}x{\isacharparenright}{\kern0pt}{\isacharcomma}{\kern0pt}s{\isacharbackquote}{\kern0pt}x{\isacharparenright}{\kern0pt}{\isachardoublequoteclose}\ \isanewline
%
\isadelimproof
\ \ %
\endisadelimproof
%
\isatagproof
\isacommand{unfolding}\isamarkupfalse%
\ is{\isacharunderscore}{\kern0pt}opname{\isacharunderscore}{\kern0pt}check{\isacharunderscore}{\kern0pt}def\ \ \isanewline
\ \ \isacommand{using}\isamarkupfalse%
\ assms\ check{\isacharunderscore}{\kern0pt}abs\ check{\isacharunderscore}{\kern0pt}in{\isacharunderscore}{\kern0pt}M\ opair{\isacharunderscore}{\kern0pt}name{\isacharunderscore}{\kern0pt}abs\ apply{\isacharunderscore}{\kern0pt}abs\ apply{\isacharunderscore}{\kern0pt}closed\ \isacommand{by}\isamarkupfalse%
\ simp%
\endisatagproof
{\isafoldproof}%
%
\isadelimproof
\isanewline
%
\endisadelimproof
\isanewline
\isacommand{lemma}\isamarkupfalse%
\ repl{\isacharunderscore}{\kern0pt}opname{\isacharunderscore}{\kern0pt}check\ {\isacharcolon}{\kern0pt}\isanewline
\ \ \isakeyword{assumes}\isanewline
\ \ \ \ {\isachardoublequoteopen}A{\isasymin}M{\isachardoublequoteclose}\ {\isachardoublequoteopen}f{\isasymin}M{\isachardoublequoteclose}\ \isanewline
\ \ \isakeyword{shows}\isanewline
\ \ \ {\isachardoublequoteopen}{\isacharbraceleft}{\kern0pt}opair{\isacharunderscore}{\kern0pt}name{\isacharparenleft}{\kern0pt}check{\isacharparenleft}{\kern0pt}x{\isacharparenright}{\kern0pt}{\isacharcomma}{\kern0pt}f{\isacharbackquote}{\kern0pt}x{\isacharparenright}{\kern0pt}{\isachardot}{\kern0pt}\ x{\isasymin}A{\isacharbraceright}{\kern0pt}{\isasymin}M{\isachardoublequoteclose}\isanewline
%
\isadelimproof
%
\endisadelimproof
%
\isatagproof
\isacommand{proof}\isamarkupfalse%
\ {\isacharminus}{\kern0pt}\isanewline
\ \ \isacommand{have}\isamarkupfalse%
\ {\isachardoublequoteopen}arity{\isacharparenleft}{\kern0pt}opname{\isacharunderscore}{\kern0pt}check{\isacharunderscore}{\kern0pt}fm{\isacharparenleft}{\kern0pt}{\isadigit{3}}{\isacharcomma}{\kern0pt}{\isadigit{0}}{\isacharcomma}{\kern0pt}{\isadigit{1}}{\isacharcomma}{\kern0pt}{\isadigit{2}}{\isacharparenright}{\kern0pt}{\isacharparenright}{\kern0pt}{\isacharequal}{\kern0pt}\ {\isadigit{4}}{\isachardoublequoteclose}\ \isanewline
\ \ \ \ \isacommand{unfolding}\isamarkupfalse%
\ opname{\isacharunderscore}{\kern0pt}check{\isacharunderscore}{\kern0pt}fm{\isacharunderscore}{\kern0pt}def\ opair{\isacharunderscore}{\kern0pt}name{\isacharunderscore}{\kern0pt}fm{\isacharunderscore}{\kern0pt}def\ upair{\isacharunderscore}{\kern0pt}name{\isacharunderscore}{\kern0pt}fm{\isacharunderscore}{\kern0pt}def\isanewline
\ \ \ \ \ \ \ \ \ \ check{\isacharunderscore}{\kern0pt}fm{\isacharunderscore}{\kern0pt}def\ rcheck{\isacharunderscore}{\kern0pt}fm{\isacharunderscore}{\kern0pt}def\ trans{\isacharunderscore}{\kern0pt}closure{\isacharunderscore}{\kern0pt}fm{\isacharunderscore}{\kern0pt}def\ is{\isacharunderscore}{\kern0pt}eclose{\isacharunderscore}{\kern0pt}fm{\isacharunderscore}{\kern0pt}def\ mem{\isacharunderscore}{\kern0pt}eclose{\isacharunderscore}{\kern0pt}fm{\isacharunderscore}{\kern0pt}def\isanewline
\ \ \ \ \ \ \ \ \ is{\isacharunderscore}{\kern0pt}Hcheck{\isacharunderscore}{\kern0pt}fm{\isacharunderscore}{\kern0pt}def\ Replace{\isacharunderscore}{\kern0pt}fm{\isacharunderscore}{\kern0pt}def\ PHcheck{\isacharunderscore}{\kern0pt}fm{\isacharunderscore}{\kern0pt}def\ finite{\isacharunderscore}{\kern0pt}ordinal{\isacharunderscore}{\kern0pt}fm{\isacharunderscore}{\kern0pt}def\ is{\isacharunderscore}{\kern0pt}iterates{\isacharunderscore}{\kern0pt}fm{\isacharunderscore}{\kern0pt}def\isanewline
\ \ \ \ \ \ \ \ \ \ \ \ \ is{\isacharunderscore}{\kern0pt}wfrec{\isacharunderscore}{\kern0pt}fm{\isacharunderscore}{\kern0pt}def\ is{\isacharunderscore}{\kern0pt}recfun{\isacharunderscore}{\kern0pt}fm{\isacharunderscore}{\kern0pt}def\ restriction{\isacharunderscore}{\kern0pt}fm{\isacharunderscore}{\kern0pt}def\ pre{\isacharunderscore}{\kern0pt}image{\isacharunderscore}{\kern0pt}fm{\isacharunderscore}{\kern0pt}def\ eclose{\isacharunderscore}{\kern0pt}n{\isacharunderscore}{\kern0pt}fm{\isacharunderscore}{\kern0pt}def\isanewline
\ \ \ \ \ \ \ \ is{\isacharunderscore}{\kern0pt}nat{\isacharunderscore}{\kern0pt}case{\isacharunderscore}{\kern0pt}fm{\isacharunderscore}{\kern0pt}def\ quasinat{\isacharunderscore}{\kern0pt}fm{\isacharunderscore}{\kern0pt}def\ Memrel{\isacharunderscore}{\kern0pt}fm{\isacharunderscore}{\kern0pt}def\ singleton{\isacharunderscore}{\kern0pt}fm{\isacharunderscore}{\kern0pt}def\ fm{\isacharunderscore}{\kern0pt}defs\ iterates{\isacharunderscore}{\kern0pt}MH{\isacharunderscore}{\kern0pt}fm{\isacharunderscore}{\kern0pt}def\isanewline
\ \ \ \ \isacommand{by}\isamarkupfalse%
\ {\isacharparenleft}{\kern0pt}simp\ add{\isacharcolon}{\kern0pt}nat{\isacharunderscore}{\kern0pt}simp{\isacharunderscore}{\kern0pt}union{\isacharparenright}{\kern0pt}\isanewline
\ \ \isacommand{moreover}\isamarkupfalse%
\isanewline
\ \ \isacommand{have}\isamarkupfalse%
\ {\isachardoublequoteopen}x{\isasymin}A\ {\isasymLongrightarrow}\ opair{\isacharunderscore}{\kern0pt}name{\isacharparenleft}{\kern0pt}check{\isacharparenleft}{\kern0pt}x{\isacharparenright}{\kern0pt}{\isacharcomma}{\kern0pt}\ f\ {\isacharbackquote}{\kern0pt}\ x{\isacharparenright}{\kern0pt}{\isasymin}M{\isachardoublequoteclose}\ \isakeyword{for}\ x\isanewline
\ \ \ \ \isacommand{using}\isamarkupfalse%
\ assms\ opair{\isacharunderscore}{\kern0pt}name{\isacharunderscore}{\kern0pt}closed\ apply{\isacharunderscore}{\kern0pt}closed\ transitivity\ check{\isacharunderscore}{\kern0pt}in{\isacharunderscore}{\kern0pt}M\isanewline
\ \ \ \ \isacommand{by}\isamarkupfalse%
\ simp\isanewline
\ \ \isacommand{ultimately}\isamarkupfalse%
\isanewline
\ \ \isacommand{show}\isamarkupfalse%
\ {\isacharquery}{\kern0pt}thesis\ \isacommand{using}\isamarkupfalse%
\ assms\ opname{\isacharunderscore}{\kern0pt}check{\isacharunderscore}{\kern0pt}abs{\isacharbrackleft}{\kern0pt}of\ f{\isacharbrackright}{\kern0pt}\ sats{\isacharunderscore}{\kern0pt}opname{\isacharunderscore}{\kern0pt}check{\isacharunderscore}{\kern0pt}fm\isanewline
\ \ \ \ \ \ \ \ one{\isacharunderscore}{\kern0pt}in{\isacharunderscore}{\kern0pt}M\isanewline
\ \ \ \ \ \ \ \ Repl{\isacharunderscore}{\kern0pt}in{\isacharunderscore}{\kern0pt}M{\isacharbrackleft}{\kern0pt}of\ {\isachardoublequoteopen}opname{\isacharunderscore}{\kern0pt}check{\isacharunderscore}{\kern0pt}fm{\isacharparenleft}{\kern0pt}{\isadigit{3}}{\isacharcomma}{\kern0pt}{\isadigit{0}}{\isacharcomma}{\kern0pt}{\isadigit{1}}{\isacharcomma}{\kern0pt}{\isadigit{2}}{\isacharparenright}{\kern0pt}{\isachardoublequoteclose}\ {\isachardoublequoteopen}{\isacharbrackleft}{\kern0pt}one{\isacharcomma}{\kern0pt}f{\isacharbrackright}{\kern0pt}{\isachardoublequoteclose}\ {\isachardoublequoteopen}is{\isacharunderscore}{\kern0pt}opname{\isacharunderscore}{\kern0pt}check{\isacharparenleft}{\kern0pt}f{\isacharparenright}{\kern0pt}{\isachardoublequoteclose}\ \isanewline
\ \ \ \ \ \ \ \ \ \ \ \ \ \ \ \ \ \ \ \ {\isachardoublequoteopen}{\isasymlambda}x{\isachardot}{\kern0pt}\ opair{\isacharunderscore}{\kern0pt}name{\isacharparenleft}{\kern0pt}check{\isacharparenleft}{\kern0pt}x{\isacharparenright}{\kern0pt}{\isacharcomma}{\kern0pt}f{\isacharbackquote}{\kern0pt}x{\isacharparenright}{\kern0pt}{\isachardoublequoteclose}{\isacharbrackright}{\kern0pt}\ \isanewline
\ \ \ \ \isacommand{by}\isamarkupfalse%
\ simp\isanewline
\isacommand{qed}\isamarkupfalse%
%
\endisatagproof
{\isafoldproof}%
%
\isadelimproof
\isanewline
%
\endisadelimproof
\isanewline
\isanewline
\isanewline
\isacommand{theorem}\isamarkupfalse%
\ choice{\isacharunderscore}{\kern0pt}in{\isacharunderscore}{\kern0pt}MG{\isacharcolon}{\kern0pt}\ \isanewline
\ \ \isakeyword{assumes}\ {\isachardoublequoteopen}choice{\isacharunderscore}{\kern0pt}ax{\isacharparenleft}{\kern0pt}{\isacharhash}{\kern0pt}{\isacharhash}{\kern0pt}M{\isacharparenright}{\kern0pt}{\isachardoublequoteclose}\isanewline
\ \ \isakeyword{shows}\ {\isachardoublequoteopen}choice{\isacharunderscore}{\kern0pt}ax{\isacharparenleft}{\kern0pt}{\isacharhash}{\kern0pt}{\isacharhash}{\kern0pt}M{\isacharbrackleft}{\kern0pt}G{\isacharbrackright}{\kern0pt}{\isacharparenright}{\kern0pt}{\isachardoublequoteclose}\isanewline
%
\isadelimproof
%
\endisadelimproof
%
\isatagproof
\isacommand{proof}\isamarkupfalse%
\ {\isacharminus}{\kern0pt}\isanewline
\ \ \isacommand{{\isacharbraceleft}{\kern0pt}}\isamarkupfalse%
\isanewline
\ \ \ \ \isacommand{fix}\isamarkupfalse%
\ a\isanewline
\ \ \ \ \isacommand{assume}\isamarkupfalse%
\ {\isachardoublequoteopen}a{\isasymin}M{\isacharbrackleft}{\kern0pt}G{\isacharbrackright}{\kern0pt}{\isachardoublequoteclose}\isanewline
\ \ \ \ \isacommand{then}\isamarkupfalse%
\isanewline
\ \ \ \ \isacommand{obtain}\isamarkupfalse%
\ {\isasymtau}\ \isakeyword{where}\ {\isachardoublequoteopen}{\isasymtau}{\isasymin}M{\isachardoublequoteclose}\ {\isachardoublequoteopen}val{\isacharparenleft}{\kern0pt}G{\isacharcomma}{\kern0pt}{\isasymtau}{\isacharparenright}{\kern0pt}\ {\isacharequal}{\kern0pt}\ a{\isachardoublequoteclose}\ \isanewline
\ \ \ \ \ \ \isacommand{using}\isamarkupfalse%
\ GenExt{\isacharunderscore}{\kern0pt}def\ \isacommand{by}\isamarkupfalse%
\ auto\isanewline
\ \ \ \ \isacommand{with}\isamarkupfalse%
\ {\isacartoucheopen}{\isasymtau}{\isasymin}M{\isacartoucheclose}\isanewline
\ \ \ \ \isacommand{have}\isamarkupfalse%
\ {\isachardoublequoteopen}domain{\isacharparenleft}{\kern0pt}{\isasymtau}{\isacharparenright}{\kern0pt}{\isasymin}M{\isachardoublequoteclose}\isanewline
\ \ \ \ \ \ \isacommand{using}\isamarkupfalse%
\ domain{\isacharunderscore}{\kern0pt}closed\ \isacommand{by}\isamarkupfalse%
\ simp\isanewline
\ \ \ \ \isacommand{then}\isamarkupfalse%
\isanewline
\ \ \ \ \isacommand{obtain}\isamarkupfalse%
\ s\ {\isasymalpha}\ \isakeyword{where}\ {\isachardoublequoteopen}s{\isasymin}surj{\isacharparenleft}{\kern0pt}{\isasymalpha}{\isacharcomma}{\kern0pt}domain{\isacharparenleft}{\kern0pt}{\isasymtau}{\isacharparenright}{\kern0pt}{\isacharparenright}{\kern0pt}{\isachardoublequoteclose}\ {\isachardoublequoteopen}Ord{\isacharparenleft}{\kern0pt}{\isasymalpha}{\isacharparenright}{\kern0pt}{\isachardoublequoteclose}\ {\isachardoublequoteopen}s{\isasymin}M{\isachardoublequoteclose}\ {\isachardoublequoteopen}{\isasymalpha}{\isasymin}M{\isachardoublequoteclose}\isanewline
\ \ \ \ \ \ \isacommand{using}\isamarkupfalse%
\ assms\ choice{\isacharunderscore}{\kern0pt}ax{\isacharunderscore}{\kern0pt}abs\ \isacommand{by}\isamarkupfalse%
\ auto\isanewline
\ \ \ \ \isacommand{then}\isamarkupfalse%
\isanewline
\ \ \ \ \isacommand{have}\isamarkupfalse%
\ {\isachardoublequoteopen}{\isasymalpha}{\isasymin}M{\isacharbrackleft}{\kern0pt}G{\isacharbrackright}{\kern0pt}{\isachardoublequoteclose}\ \ \ \ \ \ \ \ \ \isanewline
\ \ \ \ \ \ \isacommand{using}\isamarkupfalse%
\ M{\isacharunderscore}{\kern0pt}subset{\isacharunderscore}{\kern0pt}MG\ generic\ one{\isacharunderscore}{\kern0pt}in{\isacharunderscore}{\kern0pt}G\ subsetD\ \isacommand{by}\isamarkupfalse%
\ blast\isanewline
\ \ \ \ \isacommand{let}\isamarkupfalse%
\ {\isacharquery}{\kern0pt}A{\isacharequal}{\kern0pt}{\isachardoublequoteopen}domain{\isacharparenleft}{\kern0pt}{\isasymtau}{\isacharparenright}{\kern0pt}{\isasymtimes}P{\isachardoublequoteclose}\isanewline
\ \ \ \ \isacommand{let}\isamarkupfalse%
\ {\isacharquery}{\kern0pt}g\ {\isacharequal}{\kern0pt}\ {\isachardoublequoteopen}{\isacharbraceleft}{\kern0pt}opair{\isacharunderscore}{\kern0pt}name{\isacharparenleft}{\kern0pt}check{\isacharparenleft}{\kern0pt}{\isasymbeta}{\isacharparenright}{\kern0pt}{\isacharcomma}{\kern0pt}s{\isacharbackquote}{\kern0pt}{\isasymbeta}{\isacharparenright}{\kern0pt}{\isachardot}{\kern0pt}\ {\isasymbeta}{\isasymin}{\isasymalpha}{\isacharbraceright}{\kern0pt}{\isachardoublequoteclose}\isanewline
\ \ \ \ \isacommand{have}\isamarkupfalse%
\ {\isachardoublequoteopen}{\isacharquery}{\kern0pt}g\ {\isasymin}\ M{\isachardoublequoteclose}\ \isacommand{using}\isamarkupfalse%
\ {\isacartoucheopen}s{\isasymin}M{\isacartoucheclose}\ {\isacartoucheopen}{\isasymalpha}{\isasymin}M{\isacartoucheclose}\ repl{\isacharunderscore}{\kern0pt}opname{\isacharunderscore}{\kern0pt}check\ \isacommand{by}\isamarkupfalse%
\ simp\isanewline
\ \ \ \ \isacommand{let}\isamarkupfalse%
\ {\isacharquery}{\kern0pt}f{\isacharunderscore}{\kern0pt}dot{\isacharequal}{\kern0pt}{\isachardoublequoteopen}{\isacharbraceleft}{\kern0pt}{\isasymlangle}opair{\isacharunderscore}{\kern0pt}name{\isacharparenleft}{\kern0pt}check{\isacharparenleft}{\kern0pt}{\isasymbeta}{\isacharparenright}{\kern0pt}{\isacharcomma}{\kern0pt}s{\isacharbackquote}{\kern0pt}{\isasymbeta}{\isacharparenright}{\kern0pt}{\isacharcomma}{\kern0pt}one{\isasymrangle}{\isachardot}{\kern0pt}\ {\isasymbeta}{\isasymin}{\isasymalpha}{\isacharbraceright}{\kern0pt}{\isachardoublequoteclose}\isanewline
\ \ \ \ \isacommand{have}\isamarkupfalse%
\ {\isachardoublequoteopen}{\isacharquery}{\kern0pt}f{\isacharunderscore}{\kern0pt}dot\ {\isacharequal}{\kern0pt}\ {\isacharquery}{\kern0pt}g\ {\isasymtimes}\ {\isacharbraceleft}{\kern0pt}one{\isacharbraceright}{\kern0pt}{\isachardoublequoteclose}\ \isacommand{by}\isamarkupfalse%
\ blast\isanewline
\ \ \ \ \isacommand{from}\isamarkupfalse%
\ one{\isacharunderscore}{\kern0pt}in{\isacharunderscore}{\kern0pt}M\ \isacommand{have}\isamarkupfalse%
\ {\isachardoublequoteopen}{\isacharbraceleft}{\kern0pt}one{\isacharbraceright}{\kern0pt}\ {\isasymin}\ M{\isachardoublequoteclose}\ \isacommand{using}\isamarkupfalse%
\ singletonM\ \isacommand{by}\isamarkupfalse%
\ simp\isanewline
\ \ \ \ \isacommand{define}\isamarkupfalse%
\ f\ \isakeyword{where}\isanewline
\ \ \ \ \ \ {\isachardoublequoteopen}f\ {\isasymequiv}\ val{\isacharparenleft}{\kern0pt}G{\isacharcomma}{\kern0pt}{\isacharquery}{\kern0pt}f{\isacharunderscore}{\kern0pt}dot{\isacharparenright}{\kern0pt}{\isachardoublequoteclose}\ \isanewline
\ \ \ \ \isacommand{from}\isamarkupfalse%
\ {\isacartoucheopen}{\isacharbraceleft}{\kern0pt}one{\isacharbraceright}{\kern0pt}{\isasymin}M{\isacartoucheclose}\ {\isacartoucheopen}{\isacharquery}{\kern0pt}g{\isasymin}M{\isacartoucheclose}\ {\isacartoucheopen}{\isacharquery}{\kern0pt}f{\isacharunderscore}{\kern0pt}dot\ {\isacharequal}{\kern0pt}\ {\isacharquery}{\kern0pt}g{\isasymtimes}{\isacharbraceleft}{\kern0pt}one{\isacharbraceright}{\kern0pt}{\isacartoucheclose}\ \isanewline
\ \ \ \ \isacommand{have}\isamarkupfalse%
\ {\isachardoublequoteopen}{\isacharquery}{\kern0pt}f{\isacharunderscore}{\kern0pt}dot{\isasymin}M{\isachardoublequoteclose}\ \isanewline
\ \ \ \ \ \ \isacommand{using}\isamarkupfalse%
\ cartprod{\isacharunderscore}{\kern0pt}closed\ \isacommand{by}\isamarkupfalse%
\ simp\isanewline
\ \ \ \ \isacommand{then}\isamarkupfalse%
\isanewline
\ \ \ \ \isacommand{have}\isamarkupfalse%
\ {\isachardoublequoteopen}f\ {\isasymin}\ M{\isacharbrackleft}{\kern0pt}G{\isacharbrackright}{\kern0pt}{\isachardoublequoteclose}\isanewline
\ \ \ \ \ \ \isacommand{unfolding}\isamarkupfalse%
\ f{\isacharunderscore}{\kern0pt}def\ \isacommand{by}\isamarkupfalse%
\ {\isacharparenleft}{\kern0pt}blast\ intro{\isacharcolon}{\kern0pt}GenExtI{\isacharparenright}{\kern0pt}\isanewline
\ \ \ \ \isacommand{have}\isamarkupfalse%
\ {\isachardoublequoteopen}f\ {\isacharequal}{\kern0pt}\ {\isacharbraceleft}{\kern0pt}val{\isacharparenleft}{\kern0pt}G{\isacharcomma}{\kern0pt}opair{\isacharunderscore}{\kern0pt}name{\isacharparenleft}{\kern0pt}check{\isacharparenleft}{\kern0pt}{\isasymbeta}{\isacharparenright}{\kern0pt}{\isacharcomma}{\kern0pt}s{\isacharbackquote}{\kern0pt}{\isasymbeta}{\isacharparenright}{\kern0pt}{\isacharparenright}{\kern0pt}\ {\isachardot}{\kern0pt}\ {\isasymbeta}{\isasymin}{\isasymalpha}{\isacharbraceright}{\kern0pt}{\isachardoublequoteclose}\isanewline
\ \ \ \ \ \ \isacommand{unfolding}\isamarkupfalse%
\ f{\isacharunderscore}{\kern0pt}def\ \isacommand{using}\isamarkupfalse%
\ val{\isacharunderscore}{\kern0pt}RepFun{\isacharunderscore}{\kern0pt}one\ \isacommand{by}\isamarkupfalse%
\ simp\isanewline
\ \ \ \ \isacommand{also}\isamarkupfalse%
\isanewline
\ \ \ \ \isacommand{have}\isamarkupfalse%
\ {\isachardoublequoteopen}{\isachardot}{\kern0pt}{\isachardot}{\kern0pt}{\isachardot}{\kern0pt}\ {\isacharequal}{\kern0pt}\ {\isacharbraceleft}{\kern0pt}{\isasymlangle}{\isasymbeta}{\isacharcomma}{\kern0pt}val{\isacharparenleft}{\kern0pt}G{\isacharcomma}{\kern0pt}s{\isacharbackquote}{\kern0pt}{\isasymbeta}{\isacharparenright}{\kern0pt}{\isasymrangle}\ {\isachardot}{\kern0pt}\ {\isasymbeta}{\isasymin}{\isasymalpha}{\isacharbraceright}{\kern0pt}{\isachardoublequoteclose}\isanewline
\ \ \ \ \ \ \isacommand{using}\isamarkupfalse%
\ val{\isacharunderscore}{\kern0pt}opair{\isacharunderscore}{\kern0pt}name\ valcheck\ generic\ one{\isacharunderscore}{\kern0pt}in{\isacharunderscore}{\kern0pt}G\ one{\isacharunderscore}{\kern0pt}in{\isacharunderscore}{\kern0pt}P\ \isacommand{by}\isamarkupfalse%
\ simp\isanewline
\ \ \ \ \isacommand{finally}\isamarkupfalse%
\isanewline
\ \ \ \ \isacommand{have}\isamarkupfalse%
\ {\isachardoublequoteopen}f\ {\isacharequal}{\kern0pt}\ {\isacharbraceleft}{\kern0pt}{\isasymlangle}{\isasymbeta}{\isacharcomma}{\kern0pt}val{\isacharparenleft}{\kern0pt}G{\isacharcomma}{\kern0pt}s{\isacharbackquote}{\kern0pt}{\isasymbeta}{\isacharparenright}{\kern0pt}{\isasymrangle}\ {\isachardot}{\kern0pt}\ {\isasymbeta}{\isasymin}{\isasymalpha}{\isacharbraceright}{\kern0pt}{\isachardoublequoteclose}\ \isacommand{{\isachardot}{\kern0pt}}\isamarkupfalse%
\isanewline
\ \ \ \ \isacommand{then}\isamarkupfalse%
\isanewline
\ \ \ \ \isacommand{have}\isamarkupfalse%
\ {\isadigit{1}}{\isacharcolon}{\kern0pt}\ {\isachardoublequoteopen}domain{\isacharparenleft}{\kern0pt}f{\isacharparenright}{\kern0pt}\ {\isacharequal}{\kern0pt}\ {\isasymalpha}{\isachardoublequoteclose}\ {\isachardoublequoteopen}function{\isacharparenleft}{\kern0pt}f{\isacharparenright}{\kern0pt}{\isachardoublequoteclose}\isanewline
\ \ \ \ \ \ \isacommand{unfolding}\isamarkupfalse%
\ function{\isacharunderscore}{\kern0pt}def\ \isacommand{by}\isamarkupfalse%
\ auto\isanewline
\ \ \ \ \isacommand{have}\isamarkupfalse%
\ {\isadigit{2}}{\isacharcolon}{\kern0pt}\ {\isachardoublequoteopen}y\ {\isasymin}\ a\ {\isasymLongrightarrow}\ {\isasymexists}x{\isasymin}{\isasymalpha}{\isachardot}{\kern0pt}\ f\ {\isacharbackquote}{\kern0pt}\ x\ {\isacharequal}{\kern0pt}\ y{\isachardoublequoteclose}\ \isakeyword{for}\ y\isanewline
\ \ \ \ \isacommand{proof}\isamarkupfalse%
\ {\isacharminus}{\kern0pt}\isanewline
\ \ \ \ \ \ \isacommand{fix}\isamarkupfalse%
\ y\isanewline
\ \ \ \ \ \ \isacommand{assume}\isamarkupfalse%
\isanewline
\ \ \ \ \ \ \ \ {\isachardoublequoteopen}y\ {\isasymin}\ a{\isachardoublequoteclose}\isanewline
\ \ \ \ \ \ \isacommand{with}\isamarkupfalse%
\ {\isacartoucheopen}val{\isacharparenleft}{\kern0pt}G{\isacharcomma}{\kern0pt}{\isasymtau}{\isacharparenright}{\kern0pt}\ {\isacharequal}{\kern0pt}\ a{\isacartoucheclose}\ \isanewline
\ \ \ \ \ \ \isacommand{obtain}\isamarkupfalse%
\ {\isasymsigma}\ \isakeyword{where}\ \ {\isachardoublequoteopen}{\isasymsigma}{\isasymin}domain{\isacharparenleft}{\kern0pt}{\isasymtau}{\isacharparenright}{\kern0pt}{\isachardoublequoteclose}\ {\isachardoublequoteopen}val{\isacharparenleft}{\kern0pt}G{\isacharcomma}{\kern0pt}{\isasymsigma}{\isacharparenright}{\kern0pt}\ {\isacharequal}{\kern0pt}\ y{\isachardoublequoteclose}\isanewline
\ \ \ \ \ \ \ \ \isacommand{using}\isamarkupfalse%
\ elem{\isacharunderscore}{\kern0pt}of{\isacharunderscore}{\kern0pt}val{\isacharbrackleft}{\kern0pt}of\ y\ {\isacharunderscore}{\kern0pt}\ {\isasymtau}{\isacharbrackright}{\kern0pt}\ \isacommand{by}\isamarkupfalse%
\ blast\isanewline
\ \ \ \ \ \ \isacommand{with}\isamarkupfalse%
\ {\isacartoucheopen}s{\isasymin}surj{\isacharparenleft}{\kern0pt}{\isasymalpha}{\isacharcomma}{\kern0pt}domain{\isacharparenleft}{\kern0pt}{\isasymtau}{\isacharparenright}{\kern0pt}{\isacharparenright}{\kern0pt}{\isacartoucheclose}\ \isanewline
\ \ \ \ \ \ \isacommand{obtain}\isamarkupfalse%
\ {\isasymbeta}\ \isakeyword{where}\ {\isachardoublequoteopen}{\isasymbeta}{\isasymin}{\isasymalpha}{\isachardoublequoteclose}\ {\isachardoublequoteopen}s{\isacharbackquote}{\kern0pt}{\isasymbeta}\ {\isacharequal}{\kern0pt}\ {\isasymsigma}{\isachardoublequoteclose}\ \isanewline
\ \ \ \ \ \ \ \ \isacommand{unfolding}\isamarkupfalse%
\ surj{\isacharunderscore}{\kern0pt}def\ \isacommand{by}\isamarkupfalse%
\ auto\isanewline
\ \ \ \ \ \ \isacommand{with}\isamarkupfalse%
\ {\isacartoucheopen}val{\isacharparenleft}{\kern0pt}G{\isacharcomma}{\kern0pt}{\isasymsigma}{\isacharparenright}{\kern0pt}\ {\isacharequal}{\kern0pt}\ y{\isacartoucheclose}\isanewline
\ \ \ \ \ \ \isacommand{have}\isamarkupfalse%
\ {\isachardoublequoteopen}val{\isacharparenleft}{\kern0pt}G{\isacharcomma}{\kern0pt}s{\isacharbackquote}{\kern0pt}{\isasymbeta}{\isacharparenright}{\kern0pt}\ {\isacharequal}{\kern0pt}\ y{\isachardoublequoteclose}\ \isanewline
\ \ \ \ \ \ \ \ \isacommand{by}\isamarkupfalse%
\ simp\isanewline
\ \ \ \ \ \ \isacommand{with}\isamarkupfalse%
\ {\isacartoucheopen}f\ {\isacharequal}{\kern0pt}\ {\isacharbraceleft}{\kern0pt}{\isasymlangle}{\isasymbeta}{\isacharcomma}{\kern0pt}val{\isacharparenleft}{\kern0pt}G{\isacharcomma}{\kern0pt}s{\isacharbackquote}{\kern0pt}{\isasymbeta}{\isacharparenright}{\kern0pt}{\isasymrangle}\ {\isachardot}{\kern0pt}\ {\isasymbeta}{\isasymin}{\isasymalpha}{\isacharbraceright}{\kern0pt}{\isacartoucheclose}\ {\isacartoucheopen}{\isasymbeta}{\isasymin}{\isasymalpha}{\isacartoucheclose}\isanewline
\ \ \ \ \ \ \isacommand{have}\isamarkupfalse%
\ {\isachardoublequoteopen}{\isasymlangle}{\isasymbeta}{\isacharcomma}{\kern0pt}y{\isasymrangle}{\isasymin}f{\isachardoublequoteclose}\ \isanewline
\ \ \ \ \ \ \ \ \isacommand{by}\isamarkupfalse%
\ auto\isanewline
\ \ \ \ \ \ \isacommand{with}\isamarkupfalse%
\ {\isacartoucheopen}function{\isacharparenleft}{\kern0pt}f{\isacharparenright}{\kern0pt}{\isacartoucheclose}\isanewline
\ \ \ \ \ \ \isacommand{have}\isamarkupfalse%
\ {\isachardoublequoteopen}f{\isacharbackquote}{\kern0pt}{\isasymbeta}\ {\isacharequal}{\kern0pt}\ y{\isachardoublequoteclose}\isanewline
\ \ \ \ \ \ \ \ \isacommand{using}\isamarkupfalse%
\ function{\isacharunderscore}{\kern0pt}apply{\isacharunderscore}{\kern0pt}equality\ \isacommand{by}\isamarkupfalse%
\ simp\isanewline
\ \ \ \ \ \ \isacommand{with}\isamarkupfalse%
\ {\isacartoucheopen}{\isasymbeta}{\isasymin}{\isasymalpha}{\isacartoucheclose}\ \isacommand{show}\isamarkupfalse%
\isanewline
\ \ \ \ \ \ \ \ {\isachardoublequoteopen}{\isasymexists}{\isasymbeta}{\isasymin}{\isasymalpha}{\isachardot}{\kern0pt}\ f\ {\isacharbackquote}{\kern0pt}\ {\isasymbeta}\ {\isacharequal}{\kern0pt}\ y{\isachardoublequoteclose}\ \isanewline
\ \ \ \ \ \ \ \ \isacommand{by}\isamarkupfalse%
\ auto\isanewline
\ \ \ \ \isacommand{qed}\isamarkupfalse%
\isanewline
\ \ \ \ \isacommand{then}\isamarkupfalse%
\isanewline
\ \ \ \ \isacommand{have}\isamarkupfalse%
\ {\isachardoublequoteopen}{\isasymexists}{\isasymalpha}{\isasymin}{\isacharparenleft}{\kern0pt}M{\isacharbrackleft}{\kern0pt}G{\isacharbrackright}{\kern0pt}{\isacharparenright}{\kern0pt}{\isachardot}{\kern0pt}\ {\isasymexists}f{\isacharprime}{\kern0pt}{\isasymin}{\isacharparenleft}{\kern0pt}M{\isacharbrackleft}{\kern0pt}G{\isacharbrackright}{\kern0pt}{\isacharparenright}{\kern0pt}{\isachardot}{\kern0pt}\ Ord{\isacharparenleft}{\kern0pt}{\isasymalpha}{\isacharparenright}{\kern0pt}\ {\isasymand}\ f{\isacharprime}{\kern0pt}\ {\isasymin}\ surj{\isacharparenleft}{\kern0pt}{\isasymalpha}{\isacharcomma}{\kern0pt}a{\isacharparenright}{\kern0pt}{\isachardoublequoteclose}\isanewline
\ \ \ \ \isacommand{proof}\isamarkupfalse%
\ {\isacharparenleft}{\kern0pt}cases\ {\isachardoublequoteopen}a{\isacharequal}{\kern0pt}{\isadigit{0}}{\isachardoublequoteclose}{\isacharparenright}{\kern0pt}\isanewline
\ \ \ \ \ \ \isacommand{case}\isamarkupfalse%
\ True\isanewline
\ \ \ \ \ \ \isacommand{then}\isamarkupfalse%
\isanewline
\ \ \ \ \ \ \isacommand{have}\isamarkupfalse%
\ {\isachardoublequoteopen}{\isadigit{0}}{\isasymin}surj{\isacharparenleft}{\kern0pt}{\isadigit{0}}{\isacharcomma}{\kern0pt}a{\isacharparenright}{\kern0pt}{\isachardoublequoteclose}\ \isanewline
\ \ \ \ \ \ \ \ \isacommand{unfolding}\isamarkupfalse%
\ surj{\isacharunderscore}{\kern0pt}def\ \isacommand{by}\isamarkupfalse%
\ simp\isanewline
\ \ \ \ \ \ \isacommand{then}\isamarkupfalse%
\isanewline
\ \ \ \ \ \ \isacommand{show}\isamarkupfalse%
\ {\isacharquery}{\kern0pt}thesis\ \isacommand{using}\isamarkupfalse%
\ zero{\isacharunderscore}{\kern0pt}in{\isacharunderscore}{\kern0pt}MG\ \isacommand{by}\isamarkupfalse%
\ auto\isanewline
\ \ \ \ \isacommand{next}\isamarkupfalse%
\isanewline
\ \ \ \ \ \ \isacommand{case}\isamarkupfalse%
\ False\isanewline
\ \ \ \ \ \ \isacommand{with}\isamarkupfalse%
\ {\isacartoucheopen}a{\isasymin}M{\isacharbrackleft}{\kern0pt}G{\isacharbrackright}{\kern0pt}{\isacartoucheclose}\ \isanewline
\ \ \ \ \ \ \isacommand{obtain}\isamarkupfalse%
\ e\ \isakeyword{where}\ {\isachardoublequoteopen}e{\isasymin}a{\isachardoublequoteclose}\ {\isachardoublequoteopen}e{\isasymin}M{\isacharbrackleft}{\kern0pt}G{\isacharbrackright}{\kern0pt}{\isachardoublequoteclose}\ \isanewline
\ \ \ \ \ \ \ \ \isacommand{using}\isamarkupfalse%
\ transitivity{\isacharunderscore}{\kern0pt}MG\ \isacommand{by}\isamarkupfalse%
\ blast\isanewline
\ \ \ \ \ \ \isacommand{with}\isamarkupfalse%
\ {\isadigit{1}}\ \isakeyword{and}\ {\isadigit{2}}\isanewline
\ \ \ \ \ \ \isacommand{have}\isamarkupfalse%
\ {\isachardoublequoteopen}induced{\isacharunderscore}{\kern0pt}surj{\isacharparenleft}{\kern0pt}f{\isacharcomma}{\kern0pt}a{\isacharcomma}{\kern0pt}e{\isacharparenright}{\kern0pt}\ {\isasymin}\ surj{\isacharparenleft}{\kern0pt}{\isasymalpha}{\isacharcomma}{\kern0pt}a{\isacharparenright}{\kern0pt}{\isachardoublequoteclose}\isanewline
\ \ \ \ \ \ \ \ \isacommand{using}\isamarkupfalse%
\ induced{\isacharunderscore}{\kern0pt}surj{\isacharunderscore}{\kern0pt}is{\isacharunderscore}{\kern0pt}surj\ \isacommand{by}\isamarkupfalse%
\ simp\isanewline
\ \ \ \ \ \ \isacommand{moreover}\isamarkupfalse%
\ \isacommand{from}\isamarkupfalse%
\ {\isacartoucheopen}f{\isasymin}M{\isacharbrackleft}{\kern0pt}G{\isacharbrackright}{\kern0pt}{\isacartoucheclose}\ {\isacartoucheopen}a{\isasymin}M{\isacharbrackleft}{\kern0pt}G{\isacharbrackright}{\kern0pt}{\isacartoucheclose}\ {\isacartoucheopen}e{\isasymin}M{\isacharbrackleft}{\kern0pt}G{\isacharbrackright}{\kern0pt}{\isacartoucheclose}\isanewline
\ \ \ \ \ \ \isacommand{have}\isamarkupfalse%
\ {\isachardoublequoteopen}induced{\isacharunderscore}{\kern0pt}surj{\isacharparenleft}{\kern0pt}f{\isacharcomma}{\kern0pt}a{\isacharcomma}{\kern0pt}e{\isacharparenright}{\kern0pt}\ {\isasymin}\ M{\isacharbrackleft}{\kern0pt}G{\isacharbrackright}{\kern0pt}{\isachardoublequoteclose}\isanewline
\ \ \ \ \ \ \ \ \isacommand{unfolding}\isamarkupfalse%
\ induced{\isacharunderscore}{\kern0pt}surj{\isacharunderscore}{\kern0pt}def\ \isanewline
\ \ \ \ \ \ \ \ \isacommand{by}\isamarkupfalse%
\ {\isacharparenleft}{\kern0pt}simp\ flip{\isacharcolon}{\kern0pt}\ setclass{\isacharunderscore}{\kern0pt}iff{\isacharparenright}{\kern0pt}\isanewline
\ \ \ \ \ \ \isacommand{moreover}\isamarkupfalse%
\ \isacommand{note}\isamarkupfalse%
\isanewline
\ \ \ \ \ \ \ \ {\isacartoucheopen}{\isasymalpha}{\isasymin}M{\isacharbrackleft}{\kern0pt}G{\isacharbrackright}{\kern0pt}{\isacartoucheclose}\ {\isacartoucheopen}Ord{\isacharparenleft}{\kern0pt}{\isasymalpha}{\isacharparenright}{\kern0pt}{\isacartoucheclose}\isanewline
\ \ \ \ \ \ \isacommand{ultimately}\isamarkupfalse%
\ \isacommand{show}\isamarkupfalse%
\ {\isacharquery}{\kern0pt}thesis\ \isacommand{by}\isamarkupfalse%
\ auto\isanewline
\ \ \ \ \isacommand{qed}\isamarkupfalse%
\isanewline
\ \ \isacommand{{\isacharbraceright}{\kern0pt}}\isamarkupfalse%
\isanewline
\ \ \isacommand{then}\isamarkupfalse%
\isanewline
\ \ \isacommand{show}\isamarkupfalse%
\ {\isacharquery}{\kern0pt}thesis\ \isacommand{using}\isamarkupfalse%
\ mgzf{\isachardot}{\kern0pt}choice{\isacharunderscore}{\kern0pt}ax{\isacharunderscore}{\kern0pt}abs\ \isacommand{by}\isamarkupfalse%
\ simp\isanewline
\isacommand{qed}\isamarkupfalse%
%
\endisatagproof
{\isafoldproof}%
%
\isadelimproof
\isanewline
%
\endisadelimproof
\ \ \isanewline
\isacommand{end}\isamarkupfalse%
\ \isanewline
%
\isadelimtheory
\ \ \isanewline
%
\endisadelimtheory
%
\isatagtheory
\isacommand{end}\isamarkupfalse%
%
\endisatagtheory
{\isafoldtheory}%
%
\isadelimtheory
%
\endisadelimtheory
%
\end{isabellebody}%
\endinput
%:%file=~/source/repos/ZF-notAC/code/Forcing/Choice_Axiom.thy%:%
%:%11=1%:%
%:%27=2%:%
%:%28=2%:%
%:%29=3%:%
%:%30=4%:%
%:%31=5%:%
%:%32=6%:%
%:%37=6%:%
%:%40=7%:%
%:%41=8%:%
%:%42=8%:%
%:%43=9%:%
%:%44=10%:%
%:%45=11%:%
%:%46=12%:%
%:%47=12%:%
%:%50=13%:%
%:%54=13%:%
%:%55=13%:%
%:%56=13%:%
%:%57=13%:%
%:%62=13%:%
%:%65=14%:%
%:%66=15%:%
%:%67=15%:%
%:%68=16%:%
%:%69=17%:%
%:%76=18%:%
%:%77=18%:%
%:%78=19%:%
%:%79=19%:%
%:%80=20%:%
%:%81=20%:%
%:%82=21%:%
%:%83=21%:%
%:%84=21%:%
%:%85=22%:%
%:%86=22%:%
%:%87=23%:%
%:%88=23%:%
%:%89=24%:%
%:%90=24%:%
%:%91=25%:%
%:%92=25%:%
%:%93=26%:%
%:%94=26%:%
%:%95=26%:%
%:%96=27%:%
%:%97=27%:%
%:%98=28%:%
%:%99=28%:%
%:%100=29%:%
%:%101=29%:%
%:%102=30%:%
%:%103=30%:%
%:%104=31%:%
%:%105=31%:%
%:%106=31%:%
%:%107=32%:%
%:%108=32%:%
%:%109=32%:%
%:%110=33%:%
%:%111=33%:%
%:%112=34%:%
%:%113=34%:%
%:%114=35%:%
%:%115=35%:%
%:%116=36%:%
%:%117=36%:%
%:%118=36%:%
%:%119=37%:%
%:%120=37%:%
%:%121=37%:%
%:%122=38%:%
%:%123=38%:%
%:%124=39%:%
%:%125=39%:%
%:%126=39%:%
%:%127=40%:%
%:%128=40%:%
%:%129=41%:%
%:%130=41%:%
%:%131=42%:%
%:%132=42%:%
%:%133=43%:%
%:%134=43%:%
%:%135=44%:%
%:%136=44%:%
%:%137=44%:%
%:%138=45%:%
%:%144=45%:%
%:%147=46%:%
%:%148=47%:%
%:%149=47%:%
%:%150=48%:%
%:%151=49%:%
%:%152=50%:%
%:%153=51%:%
%:%154=52%:%
%:%155=53%:%
%:%162=54%:%
%:%163=54%:%
%:%164=55%:%
%:%165=55%:%
%:%166=56%:%
%:%167=56%:%
%:%168=57%:%
%:%169=57%:%
%:%170=57%:%
%:%171=58%:%
%:%172=58%:%
%:%173=58%:%
%:%174=59%:%
%:%175=59%:%
%:%176=60%:%
%:%177=60%:%
%:%178=60%:%
%:%179=61%:%
%:%180=61%:%
%:%181=62%:%
%:%182=62%:%
%:%183=63%:%
%:%184=63%:%
%:%185=64%:%
%:%186=64%:%
%:%187=65%:%
%:%188=65%:%
%:%189=66%:%
%:%190=66%:%
%:%191=66%:%
%:%192=67%:%
%:%193=67%:%
%:%194=67%:%
%:%195=68%:%
%:%196=68%:%
%:%197=69%:%
%:%198=69%:%
%:%199=69%:%
%:%200=70%:%
%:%201=70%:%
%:%202=70%:%
%:%203=71%:%
%:%204=71%:%
%:%205=72%:%
%:%206=72%:%
%:%207=72%:%
%:%208=73%:%
%:%209=73%:%
%:%210=73%:%
%:%211=74%:%
%:%212=74%:%
%:%213=75%:%
%:%214=75%:%
%:%215=75%:%
%:%216=76%:%
%:%217=76%:%
%:%218=77%:%
%:%219=77%:%
%:%220=78%:%
%:%221=78%:%
%:%222=78%:%
%:%223=78%:%
%:%224=79%:%
%:%225=79%:%
%:%226=80%:%
%:%227=80%:%
%:%228=81%:%
%:%229=81%:%
%:%230=81%:%
%:%231=82%:%
%:%232=82%:%
%:%233=83%:%
%:%234=83%:%
%:%235=84%:%
%:%236=84%:%
%:%237=85%:%
%:%238=85%:%
%:%239=86%:%
%:%240=86%:%
%:%241=87%:%
%:%242=87%:%
%:%243=88%:%
%:%244=88%:%
%:%245=88%:%
%:%246=89%:%
%:%247=89%:%
%:%248=89%:%
%:%249=90%:%
%:%250=90%:%
%:%251=91%:%
%:%252=91%:%
%:%253=92%:%
%:%254=92%:%
%:%255=93%:%
%:%256=93%:%
%:%257=94%:%
%:%258=94%:%
%:%259=94%:%
%:%260=94%:%
%:%261=95%:%
%:%267=95%:%
%:%270=96%:%
%:%271=97%:%
%:%272=97%:%
%:%273=98%:%
%:%274=99%:%
%:%275=100%:%
%:%276=101%:%
%:%279=102%:%
%:%283=102%:%
%:%284=102%:%
%:%285=103%:%
%:%286=103%:%
%:%287=104%:%
%:%288=104%:%
%:%289=105%:%
%:%290=105%:%
%:%291=106%:%
%:%292=106%:%
%:%293=106%:%
%:%294=107%:%
%:%295=107%:%
%:%296=108%:%
%:%297=108%:%
%:%298=109%:%
%:%299=109%:%
%:%300=110%:%
%:%301=110%:%
%:%302=111%:%
%:%303=111%:%
%:%304=112%:%
%:%305=112%:%
%:%306=113%:%
%:%307=113%:%
%:%308=113%:%
%:%309=114%:%
%:%310=114%:%
%:%311=115%:%
%:%312=115%:%
%:%313=116%:%
%:%314=116%:%
%:%315=116%:%
%:%316=117%:%
%:%317=117%:%
%:%318=118%:%
%:%319=118%:%
%:%320=119%:%
%:%321=119%:%
%:%322=119%:%
%:%323=120%:%
%:%324=120%:%
%:%325=120%:%
%:%326=121%:%
%:%327=121%:%
%:%328=122%:%
%:%334=122%:%
%:%337=123%:%
%:%338=124%:%
%:%339=124%:%
%:%340=125%:%
%:%341=126%:%
%:%342=127%:%
%:%343=127%:%
%:%344=128%:%
%:%345=129%:%
%:%346=130%:%
%:%347=131%:%
%:%348=131%:%
%:%349=132%:%
%:%350=133%:%
%:%351=134%:%
%:%352=135%:%
%:%353=136%:%
%:%354=136%:%
%:%355=137%:%
%:%356=138%:%
%:%359=139%:%
%:%363=139%:%
%:%364=139%:%
%:%365=139%:%
%:%366=139%:%
%:%371=139%:%
%:%374=140%:%
%:%375=141%:%
%:%376=141%:%
%:%377=142%:%
%:%380=143%:%
%:%384=143%:%
%:%385=143%:%
%:%386=143%:%
%:%387=143%:%
%:%392=143%:%
%:%395=144%:%
%:%396=145%:%
%:%397=145%:%
%:%398=146%:%
%:%399=147%:%
%:%400=148%:%
%:%401=149%:%
%:%402=150%:%
%:%403=150%:%
%:%404=151%:%
%:%407=152%:%
%:%411=152%:%
%:%412=152%:%
%:%413=152%:%
%:%418=152%:%
%:%421=153%:%
%:%422=154%:%
%:%423=154%:%
%:%424=155%:%
%:%425=156%:%
%:%426=157%:%
%:%429=158%:%
%:%433=158%:%
%:%434=158%:%
%:%435=158%:%
%:%436=158%:%
%:%441=158%:%
%:%444=159%:%
%:%445=160%:%
%:%446=160%:%
%:%447=161%:%
%:%448=162%:%
%:%449=163%:%
%:%450=164%:%
%:%451=164%:%
%:%452=165%:%
%:%453=166%:%
%:%454=167%:%
%:%455=168%:%
%:%456=169%:%
%:%457=169%:%
%:%458=170%:%
%:%459=171%:%
%:%462=172%:%
%:%466=172%:%
%:%467=172%:%
%:%468=172%:%
%:%469=172%:%
%:%474=172%:%
%:%477=173%:%
%:%478=174%:%
%:%479=174%:%
%:%480=175%:%
%:%483=176%:%
%:%487=176%:%
%:%488=176%:%
%:%489=176%:%
%:%490=176%:%
%:%495=176%:%
%:%498=177%:%
%:%499=178%:%
%:%500=178%:%
%:%501=179%:%
%:%502=180%:%
%:%503=181%:%
%:%504=182%:%
%:%505=183%:%
%:%506=183%:%
%:%507=184%:%
%:%510=185%:%
%:%514=185%:%
%:%515=185%:%
%:%516=185%:%
%:%521=185%:%
%:%524=186%:%
%:%525=187%:%
%:%526=187%:%
%:%527=188%:%
%:%528=189%:%
%:%529=190%:%
%:%532=191%:%
%:%536=191%:%
%:%537=191%:%
%:%538=191%:%
%:%539=191%:%
%:%544=191%:%
%:%547=192%:%
%:%548=193%:%
%:%549=193%:%
%:%552=194%:%
%:%556=194%:%
%:%557=194%:%
%:%558=194%:%
%:%559=194%:%
%:%564=194%:%
%:%567=195%:%
%:%568=196%:%
%:%569=196%:%
%:%572=197%:%
%:%576=197%:%
%:%577=197%:%
%:%578=197%:%
%:%579=197%:%
%:%584=197%:%
%:%587=198%:%
%:%588=199%:%
%:%589=199%:%
%:%596=200%:%
%:%597=200%:%
%:%598=201%:%
%:%599=201%:%
%:%600=202%:%
%:%601=202%:%
%:%602=203%:%
%:%603=203%:%
%:%604=203%:%
%:%605=203%:%
%:%606=204%:%
%:%607=204%:%
%:%608=205%:%
%:%609=205%:%
%:%610=205%:%
%:%611=206%:%
%:%612=206%:%
%:%613=207%:%
%:%614=207%:%
%:%615=208%:%
%:%616=208%:%
%:%617=209%:%
%:%618=209%:%
%:%619=210%:%
%:%620=210%:%
%:%621=211%:%
%:%622=211%:%
%:%623=211%:%
%:%624=212%:%
%:%625=212%:%
%:%626=213%:%
%:%627=213%:%
%:%628=214%:%
%:%629=214%:%
%:%630=214%:%
%:%631=215%:%
%:%632=215%:%
%:%633=216%:%
%:%634=216%:%
%:%635=217%:%
%:%636=217%:%
%:%637=218%:%
%:%638=218%:%
%:%639=218%:%
%:%640=218%:%
%:%641=219%:%
%:%656=221%:%
%:%666=223%:%
%:%667=223%:%
%:%670=224%:%
%:%674=224%:%
%:%675=224%:%
%:%676=225%:%
%:%677=226%:%
%:%678=227%:%
%:%679=227%:%
%:%684=227%:%
%:%687=228%:%
%:%688=229%:%
%:%689=230%:%
%:%690=230%:%
%:%691=231%:%
%:%692=232%:%
%:%693=233%:%
%:%694=234%:%
%:%695=235%:%
%:%696=235%:%
%:%697=236%:%
%:%698=237%:%
%:%699=238%:%
%:%700=239%:%
%:%701=240%:%
%:%702=240%:%
%:%703=241%:%
%:%706=242%:%
%:%710=242%:%
%:%711=242%:%
%:%712=242%:%
%:%717=242%:%
%:%720=243%:%
%:%721=244%:%
%:%722=244%:%
%:%723=245%:%
%:%724=246%:%
%:%725=247%:%
%:%726=248%:%
%:%729=249%:%
%:%733=249%:%
%:%734=249%:%
%:%735=250%:%
%:%736=250%:%
%:%737=250%:%
%:%742=250%:%
%:%745=251%:%
%:%746=252%:%
%:%747=253%:%
%:%748=253%:%
%:%749=254%:%
%:%750=255%:%
%:%753=256%:%
%:%757=256%:%
%:%758=256%:%
%:%759=257%:%
%:%760=257%:%
%:%761=257%:%
%:%766=257%:%
%:%769=258%:%
%:%770=259%:%
%:%771=259%:%
%:%772=260%:%
%:%773=261%:%
%:%774=262%:%
%:%775=263%:%
%:%782=264%:%
%:%783=264%:%
%:%784=265%:%
%:%785=265%:%
%:%786=266%:%
%:%787=266%:%
%:%788=267%:%
%:%789=268%:%
%:%790=269%:%
%:%791=270%:%
%:%792=271%:%
%:%793=271%:%
%:%794=272%:%
%:%795=272%:%
%:%796=273%:%
%:%797=273%:%
%:%798=274%:%
%:%799=274%:%
%:%800=275%:%
%:%801=275%:%
%:%802=276%:%
%:%803=276%:%
%:%804=277%:%
%:%805=277%:%
%:%806=277%:%
%:%807=278%:%
%:%808=279%:%
%:%809=280%:%
%:%810=281%:%
%:%811=281%:%
%:%812=282%:%
%:%818=282%:%
%:%821=283%:%
%:%822=284%:%
%:%823=285%:%
%:%824=286%:%
%:%825=286%:%
%:%826=287%:%
%:%827=288%:%
%:%834=289%:%
%:%835=289%:%
%:%836=290%:%
%:%837=290%:%
%:%838=291%:%
%:%839=291%:%
%:%840=292%:%
%:%841=292%:%
%:%842=293%:%
%:%843=293%:%
%:%844=294%:%
%:%845=294%:%
%:%846=295%:%
%:%847=295%:%
%:%848=295%:%
%:%849=296%:%
%:%850=296%:%
%:%851=297%:%
%:%852=297%:%
%:%853=298%:%
%:%854=298%:%
%:%855=298%:%
%:%856=299%:%
%:%857=299%:%
%:%858=300%:%
%:%859=300%:%
%:%860=301%:%
%:%861=301%:%
%:%862=301%:%
%:%863=302%:%
%:%864=302%:%
%:%865=303%:%
%:%866=303%:%
%:%867=304%:%
%:%868=304%:%
%:%869=304%:%
%:%870=305%:%
%:%871=305%:%
%:%872=306%:%
%:%873=306%:%
%:%874=307%:%
%:%875=307%:%
%:%876=307%:%
%:%877=307%:%
%:%878=308%:%
%:%879=308%:%
%:%880=309%:%
%:%881=309%:%
%:%882=309%:%
%:%883=310%:%
%:%884=310%:%
%:%885=310%:%
%:%886=310%:%
%:%887=310%:%
%:%888=311%:%
%:%889=311%:%
%:%890=312%:%
%:%891=313%:%
%:%892=313%:%
%:%893=314%:%
%:%894=314%:%
%:%895=315%:%
%:%896=315%:%
%:%897=315%:%
%:%898=316%:%
%:%899=316%:%
%:%900=317%:%
%:%901=317%:%
%:%902=318%:%
%:%903=318%:%
%:%904=318%:%
%:%905=319%:%
%:%906=319%:%
%:%907=320%:%
%:%908=320%:%
%:%909=320%:%
%:%910=320%:%
%:%911=321%:%
%:%912=321%:%
%:%913=322%:%
%:%914=322%:%
%:%915=323%:%
%:%916=323%:%
%:%917=323%:%
%:%918=324%:%
%:%919=324%:%
%:%920=325%:%
%:%921=325%:%
%:%922=325%:%
%:%923=326%:%
%:%924=326%:%
%:%925=327%:%
%:%926=327%:%
%:%927=328%:%
%:%928=328%:%
%:%929=328%:%
%:%930=329%:%
%:%931=329%:%
%:%932=330%:%
%:%933=330%:%
%:%934=331%:%
%:%935=331%:%
%:%936=332%:%
%:%937=332%:%
%:%938=333%:%
%:%939=334%:%
%:%940=334%:%
%:%941=335%:%
%:%942=335%:%
%:%943=336%:%
%:%944=336%:%
%:%945=336%:%
%:%946=337%:%
%:%947=337%:%
%:%948=338%:%
%:%949=338%:%
%:%950=339%:%
%:%951=339%:%
%:%952=339%:%
%:%953=340%:%
%:%954=340%:%
%:%955=341%:%
%:%956=341%:%
%:%957=342%:%
%:%958=342%:%
%:%959=343%:%
%:%960=343%:%
%:%961=344%:%
%:%962=344%:%
%:%963=345%:%
%:%964=345%:%
%:%965=346%:%
%:%966=346%:%
%:%967=347%:%
%:%968=347%:%
%:%969=348%:%
%:%970=348%:%
%:%971=348%:%
%:%972=349%:%
%:%973=349%:%
%:%974=349%:%
%:%975=350%:%
%:%976=351%:%
%:%977=351%:%
%:%978=352%:%
%:%979=352%:%
%:%980=353%:%
%:%981=353%:%
%:%982=354%:%
%:%983=354%:%
%:%984=355%:%
%:%985=355%:%
%:%986=356%:%
%:%987=356%:%
%:%988=357%:%
%:%989=357%:%
%:%990=358%:%
%:%991=358%:%
%:%992=359%:%
%:%993=359%:%
%:%994=359%:%
%:%995=360%:%
%:%996=360%:%
%:%997=361%:%
%:%998=361%:%
%:%999=361%:%
%:%1000=361%:%
%:%1001=362%:%
%:%1002=362%:%
%:%1003=363%:%
%:%1004=363%:%
%:%1005=364%:%
%:%1006=364%:%
%:%1007=365%:%
%:%1008=365%:%
%:%1009=366%:%
%:%1010=366%:%
%:%1011=366%:%
%:%1012=367%:%
%:%1013=367%:%
%:%1014=368%:%
%:%1015=368%:%
%:%1016=369%:%
%:%1017=369%:%
%:%1018=369%:%
%:%1019=370%:%
%:%1020=370%:%
%:%1021=370%:%
%:%1022=371%:%
%:%1023=371%:%
%:%1024=372%:%
%:%1025=372%:%
%:%1026=373%:%
%:%1027=373%:%
%:%1028=374%:%
%:%1029=374%:%
%:%1030=374%:%
%:%1031=375%:%
%:%1032=376%:%
%:%1033=376%:%
%:%1034=376%:%
%:%1035=376%:%
%:%1036=377%:%
%:%1037=377%:%
%:%1038=378%:%
%:%1039=378%:%
%:%1040=379%:%
%:%1041=379%:%
%:%1042=380%:%
%:%1043=380%:%
%:%1044=380%:%
%:%1045=380%:%
%:%1046=381%:%
%:%1052=381%:%
%:%1055=382%:%
%:%1056=383%:%
%:%1057=383%:%
%:%1060=384%:%
%:%1065=385%:%

%
\begin{isabellebody}%
\setisabellecontext{Ordinals{\isacharunderscore}{\kern0pt}In{\isacharunderscore}{\kern0pt}MG}%
%
\isadelimdocument
%
\endisadelimdocument
%
\isatagdocument
%
\isamarkupsection{Ordinals in generic extensions%
}
\isamarkuptrue%
%
\endisatagdocument
{\isafolddocument}%
%
\isadelimdocument
%
\endisadelimdocument
%
\isadelimtheory
%
\endisadelimtheory
%
\isatagtheory
\isacommand{theory}\isamarkupfalse%
\ Ordinals{\isacharunderscore}{\kern0pt}In{\isacharunderscore}{\kern0pt}MG\isanewline
\ \ \isakeyword{imports}\isanewline
\ \ \ \ Forcing{\isacharunderscore}{\kern0pt}Theorems\ Relative{\isacharunderscore}{\kern0pt}Univ\isanewline
\isakeyword{begin}%
\endisatagtheory
{\isafoldtheory}%
%
\isadelimtheory
\isanewline
%
\endisadelimtheory
\isanewline
\isacommand{context}\isamarkupfalse%
\ G{\isacharunderscore}{\kern0pt}generic\isanewline
\isakeyword{begin}\isanewline
\isanewline
\isacommand{lemma}\isamarkupfalse%
\ rank{\isacharunderscore}{\kern0pt}val{\isacharcolon}{\kern0pt}\ {\isachardoublequoteopen}rank{\isacharparenleft}{\kern0pt}val{\isacharparenleft}{\kern0pt}G{\isacharcomma}{\kern0pt}x{\isacharparenright}{\kern0pt}{\isacharparenright}{\kern0pt}\ {\isasymle}\ rank{\isacharparenleft}{\kern0pt}x{\isacharparenright}{\kern0pt}{\isachardoublequoteclose}\ {\isacharparenleft}{\kern0pt}\isakeyword{is}\ {\isachardoublequoteopen}{\isacharquery}{\kern0pt}Q{\isacharparenleft}{\kern0pt}x{\isacharparenright}{\kern0pt}{\isachardoublequoteclose}{\isacharparenright}{\kern0pt}\isanewline
%
\isadelimproof
%
\endisadelimproof
%
\isatagproof
\isacommand{proof}\isamarkupfalse%
\ {\isacharparenleft}{\kern0pt}induct\ rule{\isacharcolon}{\kern0pt}ed{\isacharunderscore}{\kern0pt}induction{\isacharbrackleft}{\kern0pt}of\ {\isacharquery}{\kern0pt}Q{\isacharbrackright}{\kern0pt}{\isacharparenright}{\kern0pt}\isanewline
\ \ \isacommand{case}\isamarkupfalse%
\ {\isacharparenleft}{\kern0pt}{\isadigit{1}}\ x{\isacharparenright}{\kern0pt}\isanewline
\ \ \isacommand{have}\isamarkupfalse%
\ {\isachardoublequoteopen}val{\isacharparenleft}{\kern0pt}G{\isacharcomma}{\kern0pt}x{\isacharparenright}{\kern0pt}\ {\isacharequal}{\kern0pt}\ {\isacharbraceleft}{\kern0pt}val{\isacharparenleft}{\kern0pt}G{\isacharcomma}{\kern0pt}u{\isacharparenright}{\kern0pt}{\isachardot}{\kern0pt}\ u{\isasymin}{\isacharbraceleft}{\kern0pt}t{\isasymin}domain{\isacharparenleft}{\kern0pt}x{\isacharparenright}{\kern0pt}{\isachardot}{\kern0pt}\ {\isasymexists}p{\isasymin}P\ {\isachardot}{\kern0pt}\ \ {\isasymlangle}t{\isacharcomma}{\kern0pt}p{\isasymrangle}{\isasymin}x\ {\isasymand}\ p\ {\isasymin}\ G\ {\isacharbraceright}{\kern0pt}{\isacharbraceright}{\kern0pt}{\isachardoublequoteclose}\isanewline
\ \ \ \ \isacommand{using}\isamarkupfalse%
\ def{\isacharunderscore}{\kern0pt}val\ \isacommand{unfolding}\isamarkupfalse%
\ Sep{\isacharunderscore}{\kern0pt}and{\isacharunderscore}{\kern0pt}Replace\ \isacommand{by}\isamarkupfalse%
\ blast\isanewline
\ \ \isacommand{then}\isamarkupfalse%
\isanewline
\ \ \isacommand{have}\isamarkupfalse%
\ {\isachardoublequoteopen}rank{\isacharparenleft}{\kern0pt}val{\isacharparenleft}{\kern0pt}G{\isacharcomma}{\kern0pt}x{\isacharparenright}{\kern0pt}{\isacharparenright}{\kern0pt}\ {\isacharequal}{\kern0pt}\ {\isacharparenleft}{\kern0pt}{\isasymUnion}u{\isasymin}{\isacharbraceleft}{\kern0pt}t{\isasymin}domain{\isacharparenleft}{\kern0pt}x{\isacharparenright}{\kern0pt}{\isachardot}{\kern0pt}\ {\isasymexists}p{\isasymin}P\ {\isachardot}{\kern0pt}\ \ {\isasymlangle}t{\isacharcomma}{\kern0pt}p{\isasymrangle}{\isasymin}x\ {\isasymand}\ p\ {\isasymin}\ G\ {\isacharbraceright}{\kern0pt}{\isachardot}{\kern0pt}\ succ{\isacharparenleft}{\kern0pt}rank{\isacharparenleft}{\kern0pt}val{\isacharparenleft}{\kern0pt}G{\isacharcomma}{\kern0pt}u{\isacharparenright}{\kern0pt}{\isacharparenright}{\kern0pt}{\isacharparenright}{\kern0pt}{\isacharparenright}{\kern0pt}{\isachardoublequoteclose}\isanewline
\ \ \ \ \isacommand{using}\isamarkupfalse%
\ rank{\isacharbrackleft}{\kern0pt}of\ {\isachardoublequoteopen}val{\isacharparenleft}{\kern0pt}G{\isacharcomma}{\kern0pt}x{\isacharparenright}{\kern0pt}{\isachardoublequoteclose}{\isacharbrackright}{\kern0pt}\ \isacommand{by}\isamarkupfalse%
\ simp\isanewline
\ \ \isacommand{moreover}\isamarkupfalse%
\isanewline
\ \ \isacommand{have}\isamarkupfalse%
\ {\isachardoublequoteopen}succ{\isacharparenleft}{\kern0pt}rank{\isacharparenleft}{\kern0pt}val{\isacharparenleft}{\kern0pt}G{\isacharcomma}{\kern0pt}\ y{\isacharparenright}{\kern0pt}{\isacharparenright}{\kern0pt}{\isacharparenright}{\kern0pt}\ {\isasymle}\ rank{\isacharparenleft}{\kern0pt}x{\isacharparenright}{\kern0pt}{\isachardoublequoteclose}\ \isakeyword{if}\ {\isachardoublequoteopen}ed{\isacharparenleft}{\kern0pt}y{\isacharcomma}{\kern0pt}\ x{\isacharparenright}{\kern0pt}{\isachardoublequoteclose}\ \isakeyword{for}\ y\ \isanewline
\ \ \ \ \isacommand{using}\isamarkupfalse%
\ {\isadigit{1}}{\isacharbrackleft}{\kern0pt}OF\ that{\isacharbrackright}{\kern0pt}\ rank{\isacharunderscore}{\kern0pt}ed{\isacharbrackleft}{\kern0pt}OF\ that{\isacharbrackright}{\kern0pt}\ \isacommand{by}\isamarkupfalse%
\ {\isacharparenleft}{\kern0pt}auto\ intro{\isacharcolon}{\kern0pt}lt{\isacharunderscore}{\kern0pt}trans{\isadigit{1}}{\isacharparenright}{\kern0pt}\isanewline
\ \ \isacommand{moreover}\isamarkupfalse%
\ \isacommand{from}\isamarkupfalse%
\ this\isanewline
\ \ \isacommand{have}\isamarkupfalse%
\ {\isachardoublequoteopen}{\isacharparenleft}{\kern0pt}{\isasymUnion}u{\isasymin}{\isacharbraceleft}{\kern0pt}t{\isasymin}domain{\isacharparenleft}{\kern0pt}x{\isacharparenright}{\kern0pt}{\isachardot}{\kern0pt}\ {\isasymexists}p{\isasymin}P\ {\isachardot}{\kern0pt}\ \ {\isasymlangle}t{\isacharcomma}{\kern0pt}p{\isasymrangle}{\isasymin}x\ {\isasymand}\ p\ {\isasymin}\ G\ {\isacharbraceright}{\kern0pt}{\isachardot}{\kern0pt}\ succ{\isacharparenleft}{\kern0pt}rank{\isacharparenleft}{\kern0pt}val{\isacharparenleft}{\kern0pt}G{\isacharcomma}{\kern0pt}u{\isacharparenright}{\kern0pt}{\isacharparenright}{\kern0pt}{\isacharparenright}{\kern0pt}{\isacharparenright}{\kern0pt}\ {\isasymle}\ rank{\isacharparenleft}{\kern0pt}x{\isacharparenright}{\kern0pt}{\isachardoublequoteclose}\ \isanewline
\ \ \ \ \isacommand{by}\isamarkupfalse%
\ {\isacharparenleft}{\kern0pt}rule{\isacharunderscore}{\kern0pt}tac\ UN{\isacharunderscore}{\kern0pt}least{\isacharunderscore}{\kern0pt}le{\isacharparenright}{\kern0pt}\ {\isacharparenleft}{\kern0pt}auto{\isacharparenright}{\kern0pt}\isanewline
\ \ \isacommand{ultimately}\isamarkupfalse%
\isanewline
\ \ \isacommand{show}\isamarkupfalse%
\ {\isacharquery}{\kern0pt}case\ \isacommand{by}\isamarkupfalse%
\ simp\isanewline
\isacommand{qed}\isamarkupfalse%
%
\endisatagproof
{\isafoldproof}%
%
\isadelimproof
\isanewline
%
\endisadelimproof
\isanewline
\isacommand{lemma}\isamarkupfalse%
\ Ord{\isacharunderscore}{\kern0pt}MG{\isacharunderscore}{\kern0pt}iff{\isacharcolon}{\kern0pt}\isanewline
\ \ \isakeyword{assumes}\ {\isachardoublequoteopen}Ord{\isacharparenleft}{\kern0pt}{\isasymalpha}{\isacharparenright}{\kern0pt}{\isachardoublequoteclose}\ \isanewline
\ \ \isakeyword{shows}\ {\isachardoublequoteopen}{\isasymalpha}\ {\isasymin}\ M\ {\isasymlongleftrightarrow}\ {\isasymalpha}\ {\isasymin}\ M{\isacharbrackleft}{\kern0pt}G{\isacharbrackright}{\kern0pt}{\isachardoublequoteclose}\isanewline
%
\isadelimproof
%
\endisadelimproof
%
\isatagproof
\isacommand{proof}\isamarkupfalse%
\isanewline
\ \ \isacommand{show}\isamarkupfalse%
\ {\isachardoublequoteopen}{\isasymalpha}\ {\isasymin}\ M\ {\isasymLongrightarrow}\ {\isasymalpha}\ {\isasymin}\ M{\isacharbrackleft}{\kern0pt}G{\isacharbrackright}{\kern0pt}{\isachardoublequoteclose}\ \isanewline
\ \ \ \ \isacommand{using}\isamarkupfalse%
\ generic{\isacharbrackleft}{\kern0pt}THEN\ one{\isacharunderscore}{\kern0pt}in{\isacharunderscore}{\kern0pt}G{\isacharcomma}{\kern0pt}\ THEN\ M{\isacharunderscore}{\kern0pt}subset{\isacharunderscore}{\kern0pt}MG{\isacharbrackright}{\kern0pt}\ \isacommand{{\isachardot}{\kern0pt}{\isachardot}{\kern0pt}}\isamarkupfalse%
\isanewline
\isacommand{next}\isamarkupfalse%
\isanewline
\ \ \isacommand{assume}\isamarkupfalse%
\ {\isachardoublequoteopen}{\isasymalpha}\ {\isasymin}\ M{\isacharbrackleft}{\kern0pt}G{\isacharbrackright}{\kern0pt}{\isachardoublequoteclose}\isanewline
\ \ \isacommand{then}\isamarkupfalse%
\isanewline
\ \ \isacommand{obtain}\isamarkupfalse%
\ x\ \isakeyword{where}\ {\isachardoublequoteopen}x{\isasymin}M{\isachardoublequoteclose}\ {\isachardoublequoteopen}val{\isacharparenleft}{\kern0pt}G{\isacharcomma}{\kern0pt}x{\isacharparenright}{\kern0pt}\ {\isacharequal}{\kern0pt}\ {\isasymalpha}{\isachardoublequoteclose}\isanewline
\ \ \ \ \isacommand{using}\isamarkupfalse%
\ GenExtD\ \isacommand{by}\isamarkupfalse%
\ auto\isanewline
\ \ \isacommand{then}\isamarkupfalse%
\isanewline
\ \ \isacommand{have}\isamarkupfalse%
\ {\isachardoublequoteopen}rank{\isacharparenleft}{\kern0pt}{\isasymalpha}{\isacharparenright}{\kern0pt}\ {\isasymle}\ rank{\isacharparenleft}{\kern0pt}x{\isacharparenright}{\kern0pt}{\isachardoublequoteclose}\ \isanewline
\ \ \ \ \isacommand{using}\isamarkupfalse%
\ rank{\isacharunderscore}{\kern0pt}val\ \isacommand{by}\isamarkupfalse%
\ blast\isanewline
\ \ \isacommand{with}\isamarkupfalse%
\ assms\isanewline
\ \ \isacommand{have}\isamarkupfalse%
\ {\isachardoublequoteopen}{\isasymalpha}\ {\isasymle}\ rank{\isacharparenleft}{\kern0pt}x{\isacharparenright}{\kern0pt}{\isachardoublequoteclose}\isanewline
\ \ \ \ \isacommand{using}\isamarkupfalse%
\ rank{\isacharunderscore}{\kern0pt}of{\isacharunderscore}{\kern0pt}Ord\ \isacommand{by}\isamarkupfalse%
\ simp\isanewline
\ \ \isacommand{then}\isamarkupfalse%
\ \isanewline
\ \ \isacommand{have}\isamarkupfalse%
\ {\isachardoublequoteopen}{\isasymalpha}\ {\isasymin}\ succ{\isacharparenleft}{\kern0pt}rank{\isacharparenleft}{\kern0pt}x{\isacharparenright}{\kern0pt}{\isacharparenright}{\kern0pt}{\isachardoublequoteclose}\ \isacommand{using}\isamarkupfalse%
\ ltD\ \isacommand{by}\isamarkupfalse%
\ simp\isanewline
\ \ \isacommand{with}\isamarkupfalse%
\ {\isacartoucheopen}x{\isasymin}M{\isacartoucheclose}\isanewline
\ \ \isacommand{show}\isamarkupfalse%
\ {\isachardoublequoteopen}{\isasymalpha}\ {\isasymin}\ M{\isachardoublequoteclose}\isanewline
\ \ \ \ \isacommand{using}\isamarkupfalse%
\ cons{\isacharunderscore}{\kern0pt}closed\ transitivity{\isacharbrackleft}{\kern0pt}of\ {\isasymalpha}\ {\isachardoublequoteopen}succ{\isacharparenleft}{\kern0pt}rank{\isacharparenleft}{\kern0pt}x{\isacharparenright}{\kern0pt}{\isacharparenright}{\kern0pt}{\isachardoublequoteclose}{\isacharbrackright}{\kern0pt}\ \isanewline
\ \ \ \ \ \ rank{\isacharunderscore}{\kern0pt}closed\ \isacommand{unfolding}\isamarkupfalse%
\ succ{\isacharunderscore}{\kern0pt}def\ \isacommand{by}\isamarkupfalse%
\ simp\ \ \isanewline
\isacommand{qed}\isamarkupfalse%
%
\endisatagproof
{\isafoldproof}%
%
\isadelimproof
\isanewline
%
\endisadelimproof
\ \ \isanewline
\isacommand{end}\isamarkupfalse%
\ \isanewline
%
\isadelimtheory
\isanewline
%
\endisadelimtheory
%
\isatagtheory
\isacommand{end}\isamarkupfalse%
%
\endisatagtheory
{\isafoldtheory}%
%
\isadelimtheory
%
\endisadelimtheory
%
\end{isabellebody}%
\endinput
%:%file=~/source/repos/ZF-notAC/code/Forcing/Ordinals_In_MG.thy%:%
%:%11=1%:%
%:%27=2%:%
%:%28=2%:%
%:%29=3%:%
%:%30=4%:%
%:%31=5%:%
%:%36=5%:%
%:%39=6%:%
%:%40=7%:%
%:%41=7%:%
%:%42=8%:%
%:%43=9%:%
%:%44=10%:%
%:%45=10%:%
%:%52=11%:%
%:%53=11%:%
%:%54=12%:%
%:%55=12%:%
%:%56=13%:%
%:%57=13%:%
%:%58=14%:%
%:%59=14%:%
%:%60=14%:%
%:%61=14%:%
%:%62=15%:%
%:%63=15%:%
%:%64=16%:%
%:%65=16%:%
%:%66=17%:%
%:%67=17%:%
%:%68=17%:%
%:%69=18%:%
%:%70=18%:%
%:%71=19%:%
%:%72=19%:%
%:%73=20%:%
%:%74=20%:%
%:%75=20%:%
%:%76=21%:%
%:%77=21%:%
%:%78=21%:%
%:%79=22%:%
%:%80=22%:%
%:%81=23%:%
%:%82=23%:%
%:%83=24%:%
%:%84=24%:%
%:%85=25%:%
%:%86=25%:%
%:%87=25%:%
%:%88=26%:%
%:%94=26%:%
%:%97=27%:%
%:%98=28%:%
%:%99=28%:%
%:%100=29%:%
%:%101=30%:%
%:%108=31%:%
%:%109=31%:%
%:%110=32%:%
%:%111=32%:%
%:%112=33%:%
%:%113=33%:%
%:%114=33%:%
%:%115=34%:%
%:%116=34%:%
%:%117=35%:%
%:%118=35%:%
%:%119=36%:%
%:%120=36%:%
%:%121=37%:%
%:%122=37%:%
%:%123=38%:%
%:%124=38%:%
%:%125=38%:%
%:%126=39%:%
%:%127=39%:%
%:%128=40%:%
%:%129=40%:%
%:%130=41%:%
%:%131=41%:%
%:%132=41%:%
%:%133=42%:%
%:%134=42%:%
%:%135=43%:%
%:%136=43%:%
%:%137=44%:%
%:%138=44%:%
%:%139=44%:%
%:%140=45%:%
%:%141=45%:%
%:%142=46%:%
%:%143=46%:%
%:%144=46%:%
%:%145=46%:%
%:%146=47%:%
%:%147=47%:%
%:%148=48%:%
%:%149=48%:%
%:%150=49%:%
%:%151=49%:%
%:%152=50%:%
%:%153=50%:%
%:%154=50%:%
%:%155=51%:%
%:%161=51%:%
%:%164=52%:%
%:%165=53%:%
%:%166=53%:%
%:%169=54%:%
%:%174=55%:%

%
\begin{isabellebody}%
\setisabellecontext{Proper{\isacharunderscore}{\kern0pt}Extension}%
%
\isadelimdocument
%
\endisadelimdocument
%
\isatagdocument
%
\isamarkupsection{Separative notions and proper extensions%
}
\isamarkuptrue%
%
\endisatagdocument
{\isafolddocument}%
%
\isadelimdocument
%
\endisadelimdocument
%
\isadelimtheory
%
\endisadelimtheory
%
\isatagtheory
\isacommand{theory}\isamarkupfalse%
\ Proper{\isacharunderscore}{\kern0pt}Extension\isanewline
\ \ \isakeyword{imports}\isanewline
\ \ \ \ Names\isanewline
\isanewline
\isakeyword{begin}%
\endisatagtheory
{\isafoldtheory}%
%
\isadelimtheory
%
\endisadelimtheory
%
\begin{isamarkuptext}%
The key ingredient to obtain a proper extension is to have
a \emph{separative preorder}:%
\end{isamarkuptext}\isamarkuptrue%
\isacommand{locale}\isamarkupfalse%
\ separative{\isacharunderscore}{\kern0pt}notion\ {\isacharequal}{\kern0pt}\ forcing{\isacharunderscore}{\kern0pt}notion\ {\isacharplus}{\kern0pt}\isanewline
\ \ \isakeyword{assumes}\ separative{\isacharcolon}{\kern0pt}\ {\isachardoublequoteopen}p{\isasymin}P\ {\isasymLongrightarrow}\ {\isasymexists}q{\isasymin}P{\isachardot}{\kern0pt}\ {\isasymexists}r{\isasymin}P{\isachardot}{\kern0pt}\ q\ {\isasympreceq}\ p\ {\isasymand}\ r\ {\isasympreceq}\ p\ {\isasymand}\ q\ {\isasymbottom}\ r{\isachardoublequoteclose}\isanewline
\isakeyword{begin}%
\begin{isamarkuptext}%
For separative preorders, the complement of every filter is
dense. Hence an $M$-generic filter can't belong to the ground model.%
\end{isamarkuptext}\isamarkuptrue%
\isacommand{lemma}\isamarkupfalse%
\ filter{\isacharunderscore}{\kern0pt}complement{\isacharunderscore}{\kern0pt}dense{\isacharcolon}{\kern0pt}\isanewline
\ \ \isakeyword{assumes}\ {\isachardoublequoteopen}filter{\isacharparenleft}{\kern0pt}G{\isacharparenright}{\kern0pt}{\isachardoublequoteclose}\ \isakeyword{shows}\ {\isachardoublequoteopen}dense{\isacharparenleft}{\kern0pt}P\ {\isacharminus}{\kern0pt}\ G{\isacharparenright}{\kern0pt}{\isachardoublequoteclose}\isanewline
%
\isadelimproof
%
\endisadelimproof
%
\isatagproof
\isacommand{proof}\isamarkupfalse%
\isanewline
\ \ \isacommand{fix}\isamarkupfalse%
\ p\isanewline
\ \ \isacommand{assume}\isamarkupfalse%
\ {\isachardoublequoteopen}p{\isasymin}P{\isachardoublequoteclose}\isanewline
\ \ \isacommand{show}\isamarkupfalse%
\ {\isachardoublequoteopen}{\isasymexists}d{\isasymin}P\ {\isacharminus}{\kern0pt}\ G{\isachardot}{\kern0pt}\ d\ {\isasympreceq}\ p{\isachardoublequoteclose}\isanewline
\ \ \isacommand{proof}\isamarkupfalse%
\ {\isacharparenleft}{\kern0pt}cases\ {\isachardoublequoteopen}p{\isasymin}G{\isachardoublequoteclose}{\isacharparenright}{\kern0pt}\isanewline
\ \ \ \ \isacommand{case}\isamarkupfalse%
\ True\isanewline
\ \ \ \ \isacommand{note}\isamarkupfalse%
\ {\isacartoucheopen}p{\isasymin}P{\isacartoucheclose}\ assms\isanewline
\ \ \ \ \isacommand{moreover}\isamarkupfalse%
\isanewline
\ \ \ \ \isacommand{obtain}\isamarkupfalse%
\ q\ r\ \isakeyword{where}\ {\isachardoublequoteopen}q\ {\isasympreceq}\ p{\isachardoublequoteclose}\ {\isachardoublequoteopen}r\ {\isasympreceq}\ p{\isachardoublequoteclose}\ {\isachardoublequoteopen}q\ {\isasymbottom}\ r{\isachardoublequoteclose}\ {\isachardoublequoteopen}q{\isasymin}P{\isachardoublequoteclose}\ {\isachardoublequoteopen}r{\isasymin}P{\isachardoublequoteclose}\ \isanewline
\ \ \ \ \ \ \isacommand{using}\isamarkupfalse%
\ separative{\isacharbrackleft}{\kern0pt}OF\ {\isacartoucheopen}p{\isasymin}P{\isacartoucheclose}{\isacharbrackright}{\kern0pt}\isanewline
\ \ \ \ \ \ \isacommand{by}\isamarkupfalse%
\ force\isanewline
\ \ \ \ \isacommand{with}\isamarkupfalse%
\ {\isacartoucheopen}filter{\isacharparenleft}{\kern0pt}G{\isacharparenright}{\kern0pt}{\isacartoucheclose}\isanewline
\ \ \ \ \isacommand{obtain}\isamarkupfalse%
\ s\ \isakeyword{where}\ {\isachardoublequoteopen}s\ {\isasympreceq}\ p{\isachardoublequoteclose}\ {\isachardoublequoteopen}s\ {\isasymnotin}\ G{\isachardoublequoteclose}\ {\isachardoublequoteopen}s\ {\isasymin}\ P{\isachardoublequoteclose}\isanewline
\ \ \ \ \ \ \isacommand{using}\isamarkupfalse%
\ filter{\isacharunderscore}{\kern0pt}imp{\isacharunderscore}{\kern0pt}compat{\isacharbrackleft}{\kern0pt}of\ G\ q\ r{\isacharbrackright}{\kern0pt}\isanewline
\ \ \ \ \ \ \isacommand{by}\isamarkupfalse%
\ auto\isanewline
\ \ \ \ \isacommand{then}\isamarkupfalse%
\isanewline
\ \ \ \ \isacommand{show}\isamarkupfalse%
\ {\isacharquery}{\kern0pt}thesis\ \isacommand{by}\isamarkupfalse%
\ blast\isanewline
\ \ \isacommand{next}\isamarkupfalse%
\isanewline
\ \ \ \ \isacommand{case}\isamarkupfalse%
\ False\isanewline
\ \ \ \ \isacommand{with}\isamarkupfalse%
\ {\isacartoucheopen}p{\isasymin}P{\isacartoucheclose}\ \isanewline
\ \ \ \ \isacommand{show}\isamarkupfalse%
\ {\isacharquery}{\kern0pt}thesis\ \isacommand{using}\isamarkupfalse%
\ leq{\isacharunderscore}{\kern0pt}reflI\ \isacommand{unfolding}\isamarkupfalse%
\ Diff{\isacharunderscore}{\kern0pt}def\ \isacommand{by}\isamarkupfalse%
\ auto\isanewline
\ \ \isacommand{qed}\isamarkupfalse%
\isanewline
\isacommand{qed}\isamarkupfalse%
%
\endisatagproof
{\isafoldproof}%
%
\isadelimproof
\isanewline
%
\endisadelimproof
\isanewline
\isacommand{end}\isamarkupfalse%
\ \isanewline
\isanewline
\isacommand{locale}\isamarkupfalse%
\ ctm{\isacharunderscore}{\kern0pt}separative\ {\isacharequal}{\kern0pt}\ forcing{\isacharunderscore}{\kern0pt}data\ {\isacharplus}{\kern0pt}\ separative{\isacharunderscore}{\kern0pt}notion\isanewline
\isakeyword{begin}\isanewline
\isanewline
\isacommand{lemma}\isamarkupfalse%
\ generic{\isacharunderscore}{\kern0pt}not{\isacharunderscore}{\kern0pt}in{\isacharunderscore}{\kern0pt}M{\isacharcolon}{\kern0pt}\ \isakeyword{assumes}\ {\isachardoublequoteopen}M{\isacharunderscore}{\kern0pt}generic{\isacharparenleft}{\kern0pt}G{\isacharparenright}{\kern0pt}{\isachardoublequoteclose}\ \ \isakeyword{shows}\ {\isachardoublequoteopen}G\ {\isasymnotin}\ M{\isachardoublequoteclose}\isanewline
%
\isadelimproof
%
\endisadelimproof
%
\isatagproof
\isacommand{proof}\isamarkupfalse%
\isanewline
\ \ \isacommand{assume}\isamarkupfalse%
\ {\isachardoublequoteopen}G{\isasymin}M{\isachardoublequoteclose}\isanewline
\ \ \isacommand{then}\isamarkupfalse%
\isanewline
\ \ \isacommand{have}\isamarkupfalse%
\ {\isachardoublequoteopen}P\ {\isacharminus}{\kern0pt}\ G\ {\isasymin}\ M{\isachardoublequoteclose}\ \isanewline
\ \ \ \ \isacommand{using}\isamarkupfalse%
\ P{\isacharunderscore}{\kern0pt}in{\isacharunderscore}{\kern0pt}M\ Diff{\isacharunderscore}{\kern0pt}closed\ \isacommand{by}\isamarkupfalse%
\ simp\isanewline
\ \ \isacommand{moreover}\isamarkupfalse%
\isanewline
\ \ \isacommand{have}\isamarkupfalse%
\ {\isachardoublequoteopen}{\isasymnot}{\isacharparenleft}{\kern0pt}{\isasymexists}q{\isasymin}G{\isachardot}{\kern0pt}\ q\ {\isasymin}\ P\ {\isacharminus}{\kern0pt}\ G{\isacharparenright}{\kern0pt}{\isachardoublequoteclose}\ {\isachardoublequoteopen}{\isacharparenleft}{\kern0pt}P\ {\isacharminus}{\kern0pt}\ G{\isacharparenright}{\kern0pt}\ {\isasymsubseteq}\ P{\isachardoublequoteclose}\isanewline
\ \ \ \ \isacommand{unfolding}\isamarkupfalse%
\ Diff{\isacharunderscore}{\kern0pt}def\ \isacommand{by}\isamarkupfalse%
\ auto\isanewline
\ \ \isacommand{moreover}\isamarkupfalse%
\isanewline
\ \ \isacommand{note}\isamarkupfalse%
\ assms\isanewline
\ \ \isacommand{ultimately}\isamarkupfalse%
\isanewline
\ \ \isacommand{show}\isamarkupfalse%
\ {\isachardoublequoteopen}False{\isachardoublequoteclose}\isanewline
\ \ \ \ \isacommand{using}\isamarkupfalse%
\ filter{\isacharunderscore}{\kern0pt}complement{\isacharunderscore}{\kern0pt}dense{\isacharbrackleft}{\kern0pt}of\ G{\isacharbrackright}{\kern0pt}\ M{\isacharunderscore}{\kern0pt}generic{\isacharunderscore}{\kern0pt}denseD{\isacharbrackleft}{\kern0pt}of\ G\ {\isachardoublequoteopen}P{\isacharminus}{\kern0pt}G{\isachardoublequoteclose}{\isacharbrackright}{\kern0pt}\ \isanewline
\ \ \ \ \ \ M{\isacharunderscore}{\kern0pt}generic{\isacharunderscore}{\kern0pt}def\ \isacommand{by}\isamarkupfalse%
\ simp\ %
\isamarkupcmt{need to put generic ==> filter in claset%
}\isanewline
\isacommand{qed}\isamarkupfalse%
%
\endisatagproof
{\isafoldproof}%
%
\isadelimproof
\isanewline
%
\endisadelimproof
\isanewline
\isacommand{theorem}\isamarkupfalse%
\ proper{\isacharunderscore}{\kern0pt}extension{\isacharcolon}{\kern0pt}\ \isakeyword{assumes}\ {\isachardoublequoteopen}M{\isacharunderscore}{\kern0pt}generic{\isacharparenleft}{\kern0pt}G{\isacharparenright}{\kern0pt}{\isachardoublequoteclose}\ \isakeyword{shows}\ {\isachardoublequoteopen}M\ {\isasymnoteq}\ M{\isacharbrackleft}{\kern0pt}G{\isacharbrackright}{\kern0pt}{\isachardoublequoteclose}\isanewline
%
\isadelimproof
\ \ %
\endisadelimproof
%
\isatagproof
\isacommand{using}\isamarkupfalse%
\ assms\ G{\isacharunderscore}{\kern0pt}in{\isacharunderscore}{\kern0pt}Gen{\isacharunderscore}{\kern0pt}Ext{\isacharbrackleft}{\kern0pt}of\ G{\isacharbrackright}{\kern0pt}\ one{\isacharunderscore}{\kern0pt}in{\isacharunderscore}{\kern0pt}G{\isacharbrackleft}{\kern0pt}of\ G{\isacharbrackright}{\kern0pt}\ generic{\isacharunderscore}{\kern0pt}not{\isacharunderscore}{\kern0pt}in{\isacharunderscore}{\kern0pt}M\isanewline
\ \ \isacommand{by}\isamarkupfalse%
\ force%
\endisatagproof
{\isafoldproof}%
%
\isadelimproof
\isanewline
%
\endisadelimproof
\isanewline
\isacommand{end}\isamarkupfalse%
\ \isanewline
%
\isadelimtheory
\isanewline
%
\endisadelimtheory
%
\isatagtheory
\isacommand{end}\isamarkupfalse%
%
\endisatagtheory
{\isafoldtheory}%
%
\isadelimtheory
%
\endisadelimtheory
%
\end{isabellebody}%
\endinput
%:%file=~/source/repos/ZF-notAC/code/Forcing/Proper_Extension.thy%:%
%:%11=1%:%
%:%27=2%:%
%:%28=2%:%
%:%29=3%:%
%:%30=4%:%
%:%31=5%:%
%:%32=6%:%
%:%41=8%:%
%:%42=9%:%
%:%44=11%:%
%:%45=11%:%
%:%46=12%:%
%:%47=13%:%
%:%49=15%:%
%:%50=16%:%
%:%52=18%:%
%:%53=18%:%
%:%54=19%:%
%:%61=20%:%
%:%62=20%:%
%:%63=21%:%
%:%64=21%:%
%:%65=22%:%
%:%66=22%:%
%:%67=23%:%
%:%68=23%:%
%:%69=24%:%
%:%70=24%:%
%:%71=25%:%
%:%72=25%:%
%:%73=26%:%
%:%74=26%:%
%:%75=27%:%
%:%76=27%:%
%:%77=28%:%
%:%78=28%:%
%:%79=29%:%
%:%80=29%:%
%:%81=30%:%
%:%82=30%:%
%:%83=31%:%
%:%84=31%:%
%:%85=32%:%
%:%86=32%:%
%:%87=33%:%
%:%88=33%:%
%:%89=34%:%
%:%90=34%:%
%:%91=35%:%
%:%92=35%:%
%:%93=36%:%
%:%94=36%:%
%:%95=36%:%
%:%96=37%:%
%:%97=37%:%
%:%98=38%:%
%:%99=38%:%
%:%100=39%:%
%:%101=39%:%
%:%102=40%:%
%:%103=40%:%
%:%104=40%:%
%:%105=40%:%
%:%106=40%:%
%:%107=41%:%
%:%108=41%:%
%:%109=42%:%
%:%115=42%:%
%:%118=43%:%
%:%119=44%:%
%:%120=44%:%
%:%121=45%:%
%:%122=46%:%
%:%123=46%:%
%:%124=47%:%
%:%125=48%:%
%:%126=49%:%
%:%127=49%:%
%:%134=50%:%
%:%135=50%:%
%:%136=51%:%
%:%137=51%:%
%:%138=52%:%
%:%139=52%:%
%:%140=53%:%
%:%141=53%:%
%:%142=54%:%
%:%143=54%:%
%:%144=54%:%
%:%145=55%:%
%:%146=55%:%
%:%147=56%:%
%:%148=56%:%
%:%149=57%:%
%:%150=57%:%
%:%151=57%:%
%:%152=58%:%
%:%153=58%:%
%:%154=59%:%
%:%155=59%:%
%:%156=60%:%
%:%157=60%:%
%:%158=61%:%
%:%159=61%:%
%:%160=62%:%
%:%161=62%:%
%:%162=63%:%
%:%163=63%:%
%:%164=63%:%
%:%165=63%:%
%:%166=64%:%
%:%172=64%:%
%:%175=65%:%
%:%176=66%:%
%:%177=66%:%
%:%180=67%:%
%:%184=67%:%
%:%185=67%:%
%:%186=68%:%
%:%187=68%:%
%:%192=68%:%
%:%195=69%:%
%:%196=70%:%
%:%197=70%:%
%:%200=71%:%
%:%205=72%:%

%
\begin{isabellebody}%
\setisabellecontext{Succession{\isacharunderscore}{\kern0pt}Poset}%
%
\isadelimdocument
%
\endisadelimdocument
%
\isatagdocument
%
\isamarkupsection{A poset of successions%
}
\isamarkuptrue%
%
\endisatagdocument
{\isafolddocument}%
%
\isadelimdocument
%
\endisadelimdocument
%
\isadelimtheory
%
\endisadelimtheory
%
\isatagtheory
\isacommand{theory}\isamarkupfalse%
\ Succession{\isacharunderscore}{\kern0pt}Poset\isanewline
\ \ \isakeyword{imports}\isanewline
\ \ \ \ Arities\ Proper{\isacharunderscore}{\kern0pt}Extension\ Synthetic{\isacharunderscore}{\kern0pt}Definition\isanewline
\ \ \ \ Names\isanewline
\isakeyword{begin}%
\endisatagtheory
{\isafoldtheory}%
%
\isadelimtheory
%
\endisadelimtheory
%
\isadelimdocument
%
\endisadelimdocument
%
\isatagdocument
%
\isamarkupsubsection{The set of finite binary sequences%
}
\isamarkuptrue%
%
\endisatagdocument
{\isafolddocument}%
%
\isadelimdocument
%
\endisadelimdocument
%
\begin{isamarkuptext}%
We implement the poset for adding one Cohen real, the set 
$2^{<\omega}$ of of finite binary sequences.%
\end{isamarkuptext}\isamarkuptrue%
\isacommand{definition}\isamarkupfalse%
\isanewline
\ \ seqspace\ {\isacharcolon}{\kern0pt}{\isacharcolon}{\kern0pt}\ {\isachardoublequoteopen}i\ {\isasymRightarrow}\ i{\isachardoublequoteclose}\ {\isacharparenleft}{\kern0pt}{\isachardoublequoteopen}{\isacharunderscore}{\kern0pt}{\isacharcircum}{\kern0pt}{\isacharless}{\kern0pt}{\isasymomega}{\isachardoublequoteclose}\ {\isacharbrackleft}{\kern0pt}{\isadigit{1}}{\isadigit{0}}{\isadigit{0}}{\isacharbrackright}{\kern0pt}{\isadigit{1}}{\isadigit{0}}{\isadigit{0}}{\isacharparenright}{\kern0pt}\ \isakeyword{where}\isanewline
\ \ {\isachardoublequoteopen}seqspace{\isacharparenleft}{\kern0pt}B{\isacharparenright}{\kern0pt}\ {\isasymequiv}\ {\isasymUnion}n{\isasymin}nat{\isachardot}{\kern0pt}\ {\isacharparenleft}{\kern0pt}n{\isasymrightarrow}B{\isacharparenright}{\kern0pt}{\isachardoublequoteclose}\isanewline
\isanewline
\isacommand{lemma}\isamarkupfalse%
\ seqspaceI{\isacharbrackleft}{\kern0pt}intro{\isacharbrackright}{\kern0pt}{\isacharcolon}{\kern0pt}\ {\isachardoublequoteopen}n{\isasymin}nat\ {\isasymLongrightarrow}\ f{\isacharcolon}{\kern0pt}n{\isasymrightarrow}B\ {\isasymLongrightarrow}\ f{\isasymin}seqspace{\isacharparenleft}{\kern0pt}B{\isacharparenright}{\kern0pt}{\isachardoublequoteclose}\isanewline
%
\isadelimproof
\ \ %
\endisadelimproof
%
\isatagproof
\isacommand{unfolding}\isamarkupfalse%
\ seqspace{\isacharunderscore}{\kern0pt}def\ \isacommand{by}\isamarkupfalse%
\ blast%
\endisatagproof
{\isafoldproof}%
%
\isadelimproof
\isanewline
%
\endisadelimproof
\isanewline
\isacommand{lemma}\isamarkupfalse%
\ seqspaceD{\isacharbrackleft}{\kern0pt}dest{\isacharbrackright}{\kern0pt}{\isacharcolon}{\kern0pt}\ {\isachardoublequoteopen}f{\isasymin}seqspace{\isacharparenleft}{\kern0pt}B{\isacharparenright}{\kern0pt}\ {\isasymLongrightarrow}\ {\isasymexists}n{\isasymin}nat{\isachardot}{\kern0pt}\ f{\isacharcolon}{\kern0pt}n{\isasymrightarrow}B{\isachardoublequoteclose}\isanewline
%
\isadelimproof
\ \ %
\endisadelimproof
%
\isatagproof
\isacommand{unfolding}\isamarkupfalse%
\ seqspace{\isacharunderscore}{\kern0pt}def\ \isacommand{by}\isamarkupfalse%
\ blast%
\endisatagproof
{\isafoldproof}%
%
\isadelimproof
\isanewline
%
\endisadelimproof
\isanewline
\isacommand{lemma}\isamarkupfalse%
\ seqspace{\isacharunderscore}{\kern0pt}type{\isacharcolon}{\kern0pt}\ \isanewline
\ \ {\isachardoublequoteopen}f\ {\isasymin}\ B{\isacharcircum}{\kern0pt}{\isacharless}{\kern0pt}{\isasymomega}\ {\isasymLongrightarrow}\ {\isasymexists}n{\isasymin}nat{\isachardot}{\kern0pt}\ f{\isacharcolon}{\kern0pt}n{\isasymrightarrow}B{\isachardoublequoteclose}\ \isanewline
%
\isadelimproof
\ \ %
\endisadelimproof
%
\isatagproof
\isacommand{unfolding}\isamarkupfalse%
\ seqspace{\isacharunderscore}{\kern0pt}def\ \isacommand{by}\isamarkupfalse%
\ auto%
\endisatagproof
{\isafoldproof}%
%
\isadelimproof
\isanewline
%
\endisadelimproof
\isanewline
\isacommand{schematic{\isacharunderscore}{\kern0pt}goal}\isamarkupfalse%
\ seqspace{\isacharunderscore}{\kern0pt}fm{\isacharunderscore}{\kern0pt}auto{\isacharcolon}{\kern0pt}\isanewline
\ \ \isakeyword{assumes}\ \isanewline
\ \ \ \ {\isachardoublequoteopen}nth{\isacharparenleft}{\kern0pt}i{\isacharcomma}{\kern0pt}env{\isacharparenright}{\kern0pt}\ {\isacharequal}{\kern0pt}\ n{\isachardoublequoteclose}\ {\isachardoublequoteopen}nth{\isacharparenleft}{\kern0pt}j{\isacharcomma}{\kern0pt}env{\isacharparenright}{\kern0pt}\ {\isacharequal}{\kern0pt}\ z{\isachardoublequoteclose}\ \ {\isachardoublequoteopen}nth{\isacharparenleft}{\kern0pt}h{\isacharcomma}{\kern0pt}env{\isacharparenright}{\kern0pt}\ {\isacharequal}{\kern0pt}\ B{\isachardoublequoteclose}\ \isanewline
\ \ \ \ {\isachardoublequoteopen}i\ {\isasymin}\ nat{\isachardoublequoteclose}\ {\isachardoublequoteopen}j\ {\isasymin}\ nat{\isachardoublequoteclose}\ {\isachardoublequoteopen}h{\isasymin}nat{\isachardoublequoteclose}\ {\isachardoublequoteopen}env\ {\isasymin}\ list{\isacharparenleft}{\kern0pt}A{\isacharparenright}{\kern0pt}{\isachardoublequoteclose}\isanewline
\ \ \isakeyword{shows}\ \isanewline
\ \ \ \ {\isachardoublequoteopen}{\isacharparenleft}{\kern0pt}{\isasymexists}om{\isasymin}A{\isachardot}{\kern0pt}\ omega{\isacharparenleft}{\kern0pt}{\isacharhash}{\kern0pt}{\isacharhash}{\kern0pt}A{\isacharcomma}{\kern0pt}om{\isacharparenright}{\kern0pt}\ {\isasymand}\ n\ {\isasymin}\ om\ {\isasymand}\ is{\isacharunderscore}{\kern0pt}funspace{\isacharparenleft}{\kern0pt}{\isacharhash}{\kern0pt}{\isacharhash}{\kern0pt}A{\isacharcomma}{\kern0pt}\ n{\isacharcomma}{\kern0pt}\ B{\isacharcomma}{\kern0pt}\ z{\isacharparenright}{\kern0pt}{\isacharparenright}{\kern0pt}\ {\isasymlongleftrightarrow}\ {\isacharparenleft}{\kern0pt}A{\isacharcomma}{\kern0pt}\ env\ {\isasymTurnstile}\ {\isacharparenleft}{\kern0pt}{\isacharquery}{\kern0pt}sqsprp{\isacharparenleft}{\kern0pt}i{\isacharcomma}{\kern0pt}j{\isacharcomma}{\kern0pt}h{\isacharparenright}{\kern0pt}{\isacharparenright}{\kern0pt}{\isacharparenright}{\kern0pt}{\isachardoublequoteclose}\isanewline
%
\isadelimproof
\ \ %
\endisadelimproof
%
\isatagproof
\isacommand{unfolding}\isamarkupfalse%
\ is{\isacharunderscore}{\kern0pt}funspace{\isacharunderscore}{\kern0pt}def\ \isanewline
\ \ \isacommand{by}\isamarkupfalse%
\ {\isacharparenleft}{\kern0pt}insert\ assms\ {\isacharsemicolon}{\kern0pt}\ {\isacharparenleft}{\kern0pt}rule\ sep{\isacharunderscore}{\kern0pt}rules\ {\isacharbar}{\kern0pt}\ simp{\isacharparenright}{\kern0pt}{\isacharplus}{\kern0pt}{\isacharparenright}{\kern0pt}%
\endisatagproof
{\isafoldproof}%
%
\isadelimproof
\isanewline
%
\endisadelimproof
%
\isadelimML
\isanewline
%
\endisadelimML
%
\isatagML
\isacommand{synthesize}\isamarkupfalse%
\ {\isachardoublequoteopen}seqspace{\isacharunderscore}{\kern0pt}rep{\isacharunderscore}{\kern0pt}fm{\isachardoublequoteclose}\ \isakeyword{from{\isacharunderscore}{\kern0pt}schematic}\ seqspace{\isacharunderscore}{\kern0pt}fm{\isacharunderscore}{\kern0pt}auto%
\endisatagML
{\isafoldML}%
%
\isadelimML
\isanewline
%
\endisadelimML
\ \isanewline
\isacommand{locale}\isamarkupfalse%
\ M{\isacharunderscore}{\kern0pt}seqspace\ {\isacharequal}{\kern0pt}\ \ M{\isacharunderscore}{\kern0pt}trancl\ {\isacharplus}{\kern0pt}\isanewline
\ \ \isakeyword{assumes}\ \isanewline
\ \ \ \ seqspace{\isacharunderscore}{\kern0pt}replacement{\isacharcolon}{\kern0pt}\ {\isachardoublequoteopen}M{\isacharparenleft}{\kern0pt}B{\isacharparenright}{\kern0pt}\ {\isasymLongrightarrow}\ strong{\isacharunderscore}{\kern0pt}replacement{\isacharparenleft}{\kern0pt}M{\isacharcomma}{\kern0pt}{\isasymlambda}n\ z{\isachardot}{\kern0pt}\ n{\isasymin}nat\ {\isasymand}\ is{\isacharunderscore}{\kern0pt}funspace{\isacharparenleft}{\kern0pt}M{\isacharcomma}{\kern0pt}n{\isacharcomma}{\kern0pt}B{\isacharcomma}{\kern0pt}z{\isacharparenright}{\kern0pt}{\isacharparenright}{\kern0pt}{\isachardoublequoteclose}\isanewline
\isakeyword{begin}\isanewline
\isanewline
\isacommand{lemma}\isamarkupfalse%
\ seqspace{\isacharunderscore}{\kern0pt}closed{\isacharcolon}{\kern0pt}\isanewline
\ \ {\isachardoublequoteopen}M{\isacharparenleft}{\kern0pt}B{\isacharparenright}{\kern0pt}\ {\isasymLongrightarrow}\ M{\isacharparenleft}{\kern0pt}B{\isacharcircum}{\kern0pt}{\isacharless}{\kern0pt}{\isasymomega}{\isacharparenright}{\kern0pt}{\isachardoublequoteclose}\isanewline
%
\isadelimproof
\ \ %
\endisadelimproof
%
\isatagproof
\isacommand{unfolding}\isamarkupfalse%
\ seqspace{\isacharunderscore}{\kern0pt}def\ \isacommand{using}\isamarkupfalse%
\ seqspace{\isacharunderscore}{\kern0pt}replacement{\isacharbrackleft}{\kern0pt}of\ B{\isacharbrackright}{\kern0pt}\ RepFun{\isacharunderscore}{\kern0pt}closed{\isadigit{2}}\ \isanewline
\ \ \isacommand{by}\isamarkupfalse%
\ simp%
\endisatagproof
{\isafoldproof}%
%
\isadelimproof
\isanewline
%
\endisadelimproof
\isanewline
\isacommand{end}\isamarkupfalse%
\ \isanewline
\isanewline
\isanewline
\isacommand{sublocale}\isamarkupfalse%
\ M{\isacharunderscore}{\kern0pt}ctm\ {\isasymsubseteq}\ M{\isacharunderscore}{\kern0pt}seqspace\ {\isachardoublequoteopen}{\isacharhash}{\kern0pt}{\isacharhash}{\kern0pt}M{\isachardoublequoteclose}\isanewline
%
\isadelimproof
%
\endisadelimproof
%
\isatagproof
\isacommand{proof}\isamarkupfalse%
\ {\isacharparenleft}{\kern0pt}unfold{\isacharunderscore}{\kern0pt}locales{\isacharcomma}{\kern0pt}\ simp{\isacharparenright}{\kern0pt}\isanewline
\ \ \isacommand{fix}\isamarkupfalse%
\ B\isanewline
\ \ \isacommand{have}\isamarkupfalse%
\ {\isachardoublequoteopen}arity{\isacharparenleft}{\kern0pt}seqspace{\isacharunderscore}{\kern0pt}rep{\isacharunderscore}{\kern0pt}fm{\isacharparenleft}{\kern0pt}{\isadigit{0}}{\isacharcomma}{\kern0pt}{\isadigit{1}}{\isacharcomma}{\kern0pt}{\isadigit{2}}{\isacharparenright}{\kern0pt}{\isacharparenright}{\kern0pt}\ {\isasymle}\ {\isadigit{3}}{\isachardoublequoteclose}\ {\isachardoublequoteopen}seqspace{\isacharunderscore}{\kern0pt}rep{\isacharunderscore}{\kern0pt}fm{\isacharparenleft}{\kern0pt}{\isadigit{0}}{\isacharcomma}{\kern0pt}{\isadigit{1}}{\isacharcomma}{\kern0pt}{\isadigit{2}}{\isacharparenright}{\kern0pt}{\isasymin}formula{\isachardoublequoteclose}\ \isanewline
\ \ \ \ \isacommand{unfolding}\isamarkupfalse%
\ seqspace{\isacharunderscore}{\kern0pt}rep{\isacharunderscore}{\kern0pt}fm{\isacharunderscore}{\kern0pt}def\ \isanewline
\ \ \ \ \isacommand{using}\isamarkupfalse%
\ arity{\isacharunderscore}{\kern0pt}pair{\isacharunderscore}{\kern0pt}fm\ arity{\isacharunderscore}{\kern0pt}omega{\isacharunderscore}{\kern0pt}fm\ arity{\isacharunderscore}{\kern0pt}typed{\isacharunderscore}{\kern0pt}function{\isacharunderscore}{\kern0pt}fm\ nat{\isacharunderscore}{\kern0pt}simp{\isacharunderscore}{\kern0pt}union\ \isanewline
\ \ \ \ \isacommand{by}\isamarkupfalse%
\ auto\isanewline
\ \ \isacommand{moreover}\isamarkupfalse%
\isanewline
\ \ \isacommand{assume}\isamarkupfalse%
\ {\isachardoublequoteopen}B{\isasymin}M{\isachardoublequoteclose}\isanewline
\ \ \isacommand{ultimately}\isamarkupfalse%
\isanewline
\ \ \isacommand{have}\isamarkupfalse%
\ {\isachardoublequoteopen}strong{\isacharunderscore}{\kern0pt}replacement{\isacharparenleft}{\kern0pt}{\isacharhash}{\kern0pt}{\isacharhash}{\kern0pt}M{\isacharcomma}{\kern0pt}\ {\isasymlambda}x\ y{\isachardot}{\kern0pt}\ M{\isacharcomma}{\kern0pt}\ {\isacharbrackleft}{\kern0pt}x{\isacharcomma}{\kern0pt}\ y{\isacharcomma}{\kern0pt}\ B{\isacharbrackright}{\kern0pt}\ {\isasymTurnstile}\ seqspace{\isacharunderscore}{\kern0pt}rep{\isacharunderscore}{\kern0pt}fm{\isacharparenleft}{\kern0pt}{\isadigit{0}}{\isacharcomma}{\kern0pt}\ {\isadigit{1}}{\isacharcomma}{\kern0pt}\ {\isadigit{2}}{\isacharparenright}{\kern0pt}{\isacharparenright}{\kern0pt}{\isachardoublequoteclose}\isanewline
\ \ \ \ \isacommand{using}\isamarkupfalse%
\ replacement{\isacharunderscore}{\kern0pt}ax{\isacharbrackleft}{\kern0pt}of\ {\isachardoublequoteopen}seqspace{\isacharunderscore}{\kern0pt}rep{\isacharunderscore}{\kern0pt}fm{\isacharparenleft}{\kern0pt}{\isadigit{0}}{\isacharcomma}{\kern0pt}{\isadigit{1}}{\isacharcomma}{\kern0pt}{\isadigit{2}}{\isacharparenright}{\kern0pt}{\isachardoublequoteclose}{\isacharbrackright}{\kern0pt}\isanewline
\ \ \ \ \isacommand{by}\isamarkupfalse%
\ simp\isanewline
\ \ \isacommand{moreover}\isamarkupfalse%
\ \isanewline
\ \ \isacommand{note}\isamarkupfalse%
\ {\isacartoucheopen}B{\isasymin}M{\isacartoucheclose}\isanewline
\ \ \isacommand{moreover}\isamarkupfalse%
\ \isacommand{from}\isamarkupfalse%
\ this\isanewline
\ \ \isacommand{have}\isamarkupfalse%
\ {\isachardoublequoteopen}univalent{\isacharparenleft}{\kern0pt}{\isacharhash}{\kern0pt}{\isacharhash}{\kern0pt}M{\isacharcomma}{\kern0pt}\ A{\isacharcomma}{\kern0pt}\ {\isasymlambda}x\ y{\isachardot}{\kern0pt}\ M{\isacharcomma}{\kern0pt}\ {\isacharbrackleft}{\kern0pt}x{\isacharcomma}{\kern0pt}\ y{\isacharcomma}{\kern0pt}\ B{\isacharbrackright}{\kern0pt}\ {\isasymTurnstile}\ seqspace{\isacharunderscore}{\kern0pt}rep{\isacharunderscore}{\kern0pt}fm{\isacharparenleft}{\kern0pt}{\isadigit{0}}{\isacharcomma}{\kern0pt}\ {\isadigit{1}}{\isacharcomma}{\kern0pt}\ {\isadigit{2}}{\isacharparenright}{\kern0pt}{\isacharparenright}{\kern0pt}{\isachardoublequoteclose}\ \isanewline
\ \ \ \ \isakeyword{if}\ {\isachardoublequoteopen}A{\isasymin}M{\isachardoublequoteclose}\ \isakeyword{for}\ A\ \isanewline
\ \ \ \ \isacommand{using}\isamarkupfalse%
\ that\ \isacommand{unfolding}\isamarkupfalse%
\ univalent{\isacharunderscore}{\kern0pt}def\ seqspace{\isacharunderscore}{\kern0pt}rep{\isacharunderscore}{\kern0pt}fm{\isacharunderscore}{\kern0pt}def\ \ \isanewline
\ \ \ \ \isacommand{by}\isamarkupfalse%
\ {\isacharparenleft}{\kern0pt}auto{\isacharcomma}{\kern0pt}\ blast\ dest{\isacharcolon}{\kern0pt}transitivity{\isacharparenright}{\kern0pt}\isanewline
\ \ \isacommand{ultimately}\isamarkupfalse%
\isanewline
\ \ \isacommand{have}\isamarkupfalse%
\ {\isachardoublequoteopen}strong{\isacharunderscore}{\kern0pt}replacement{\isacharparenleft}{\kern0pt}{\isacharhash}{\kern0pt}{\isacharhash}{\kern0pt}M{\isacharcomma}{\kern0pt}\ {\isasymlambda}n\ z{\isachardot}{\kern0pt}\ {\isasymexists}om{\isacharbrackleft}{\kern0pt}{\isacharhash}{\kern0pt}{\isacharhash}{\kern0pt}M{\isacharbrackright}{\kern0pt}{\isachardot}{\kern0pt}\ omega{\isacharparenleft}{\kern0pt}{\isacharhash}{\kern0pt}{\isacharhash}{\kern0pt}M{\isacharcomma}{\kern0pt}om{\isacharparenright}{\kern0pt}\ {\isasymand}\ n\ {\isasymin}\ om\ {\isasymand}\ is{\isacharunderscore}{\kern0pt}funspace{\isacharparenleft}{\kern0pt}{\isacharhash}{\kern0pt}{\isacharhash}{\kern0pt}M{\isacharcomma}{\kern0pt}\ n{\isacharcomma}{\kern0pt}\ B{\isacharcomma}{\kern0pt}\ z{\isacharparenright}{\kern0pt}{\isacharparenright}{\kern0pt}{\isachardoublequoteclose}\isanewline
\ \ \ \ \isacommand{using}\isamarkupfalse%
\ seqspace{\isacharunderscore}{\kern0pt}fm{\isacharunderscore}{\kern0pt}auto{\isacharbrackleft}{\kern0pt}of\ {\isadigit{0}}\ {\isachardoublequoteopen}{\isacharbrackleft}{\kern0pt}{\isacharunderscore}{\kern0pt}{\isacharcomma}{\kern0pt}{\isacharunderscore}{\kern0pt}{\isacharcomma}{\kern0pt}B{\isacharbrackright}{\kern0pt}{\isachardoublequoteclose}\ {\isacharunderscore}{\kern0pt}\ {\isadigit{1}}\ {\isacharunderscore}{\kern0pt}\ {\isadigit{2}}\ B\ M{\isacharbrackright}{\kern0pt}\ \isacommand{unfolding}\isamarkupfalse%
\ seqspace{\isacharunderscore}{\kern0pt}rep{\isacharunderscore}{\kern0pt}fm{\isacharunderscore}{\kern0pt}def\ strong{\isacharunderscore}{\kern0pt}replacement{\isacharunderscore}{\kern0pt}def\isanewline
\ \ \ \ \isacommand{by}\isamarkupfalse%
\ simp\isanewline
\ \ \isacommand{with}\isamarkupfalse%
\ {\isacartoucheopen}B{\isasymin}M{\isacartoucheclose}\ \isanewline
\ \ \isacommand{show}\isamarkupfalse%
\ {\isachardoublequoteopen}strong{\isacharunderscore}{\kern0pt}replacement{\isacharparenleft}{\kern0pt}{\isacharhash}{\kern0pt}{\isacharhash}{\kern0pt}M{\isacharcomma}{\kern0pt}\ {\isasymlambda}n\ z{\isachardot}{\kern0pt}\ n\ {\isasymin}\ nat\ {\isasymand}\ is{\isacharunderscore}{\kern0pt}funspace{\isacharparenleft}{\kern0pt}{\isacharhash}{\kern0pt}{\isacharhash}{\kern0pt}M{\isacharcomma}{\kern0pt}\ n{\isacharcomma}{\kern0pt}\ B{\isacharcomma}{\kern0pt}\ z{\isacharparenright}{\kern0pt}{\isacharparenright}{\kern0pt}{\isachardoublequoteclose}\isanewline
\ \ \ \ \isacommand{using}\isamarkupfalse%
\ M{\isacharunderscore}{\kern0pt}nat\ \isacommand{by}\isamarkupfalse%
\ simp\isanewline
\isacommand{qed}\isamarkupfalse%
%
\endisatagproof
{\isafoldproof}%
%
\isadelimproof
\isanewline
%
\endisadelimproof
\isanewline
\isacommand{definition}\isamarkupfalse%
\ seq{\isacharunderscore}{\kern0pt}upd\ {\isacharcolon}{\kern0pt}{\isacharcolon}{\kern0pt}\ {\isachardoublequoteopen}i\ {\isasymRightarrow}\ i\ {\isasymRightarrow}\ i{\isachardoublequoteclose}\ \isakeyword{where}\isanewline
\ \ {\isachardoublequoteopen}seq{\isacharunderscore}{\kern0pt}upd{\isacharparenleft}{\kern0pt}f{\isacharcomma}{\kern0pt}a{\isacharparenright}{\kern0pt}\ {\isasymequiv}\ {\isasymlambda}\ j\ {\isasymin}\ succ{\isacharparenleft}{\kern0pt}domain{\isacharparenleft}{\kern0pt}f{\isacharparenright}{\kern0pt}{\isacharparenright}{\kern0pt}\ {\isachardot}{\kern0pt}\ if\ j\ {\isacharless}{\kern0pt}\ domain{\isacharparenleft}{\kern0pt}f{\isacharparenright}{\kern0pt}\ then\ f{\isacharbackquote}{\kern0pt}j\ else\ a{\isachardoublequoteclose}\isanewline
\isanewline
\isacommand{lemma}\isamarkupfalse%
\ seq{\isacharunderscore}{\kern0pt}upd{\isacharunderscore}{\kern0pt}succ{\isacharunderscore}{\kern0pt}type\ {\isacharcolon}{\kern0pt}\ \isanewline
\ \ \isakeyword{assumes}\ {\isachardoublequoteopen}n{\isasymin}nat{\isachardoublequoteclose}\ {\isachardoublequoteopen}f{\isasymin}n{\isasymrightarrow}A{\isachardoublequoteclose}\ {\isachardoublequoteopen}a{\isasymin}A{\isachardoublequoteclose}\isanewline
\ \ \isakeyword{shows}\ {\isachardoublequoteopen}seq{\isacharunderscore}{\kern0pt}upd{\isacharparenleft}{\kern0pt}f{\isacharcomma}{\kern0pt}a{\isacharparenright}{\kern0pt}{\isasymin}\ succ{\isacharparenleft}{\kern0pt}n{\isacharparenright}{\kern0pt}\ {\isasymrightarrow}\ A{\isachardoublequoteclose}\isanewline
%
\isadelimproof
%
\endisadelimproof
%
\isatagproof
\isacommand{proof}\isamarkupfalse%
\ {\isacharminus}{\kern0pt}\isanewline
\ \ \isacommand{from}\isamarkupfalse%
\ assms\isanewline
\ \ \isacommand{have}\isamarkupfalse%
\ equ{\isacharcolon}{\kern0pt}\ {\isachardoublequoteopen}domain{\isacharparenleft}{\kern0pt}f{\isacharparenright}{\kern0pt}\ {\isacharequal}{\kern0pt}\ n{\isachardoublequoteclose}\ \isacommand{using}\isamarkupfalse%
\ domain{\isacharunderscore}{\kern0pt}of{\isacharunderscore}{\kern0pt}fun\ \isacommand{by}\isamarkupfalse%
\ simp\isanewline
\ \ \isacommand{{\isacharbraceleft}{\kern0pt}}\isamarkupfalse%
\isanewline
\ \ \ \ \isacommand{fix}\isamarkupfalse%
\ j\isanewline
\ \ \ \ \isacommand{assume}\isamarkupfalse%
\ {\isachardoublequoteopen}j{\isasymin}succ{\isacharparenleft}{\kern0pt}domain{\isacharparenleft}{\kern0pt}f{\isacharparenright}{\kern0pt}{\isacharparenright}{\kern0pt}{\isachardoublequoteclose}\isanewline
\ \ \ \ \isacommand{with}\isamarkupfalse%
\ equ\ {\isacartoucheopen}n{\isasymin}{\isacharunderscore}{\kern0pt}{\isacartoucheclose}\isanewline
\ \ \ \ \isacommand{have}\isamarkupfalse%
\ {\isachardoublequoteopen}j{\isasymle}n{\isachardoublequoteclose}\ \isacommand{using}\isamarkupfalse%
\ ltI\ \isacommand{by}\isamarkupfalse%
\ auto\isanewline
\ \ \ \ \isacommand{with}\isamarkupfalse%
\ {\isacartoucheopen}n{\isasymin}{\isacharunderscore}{\kern0pt}{\isacartoucheclose}\isanewline
\ \ \ \ \isacommand{consider}\isamarkupfalse%
\ {\isacharparenleft}{\kern0pt}lt{\isacharparenright}{\kern0pt}\ {\isachardoublequoteopen}j{\isacharless}{\kern0pt}n{\isachardoublequoteclose}\ {\isacharbar}{\kern0pt}\ {\isacharparenleft}{\kern0pt}eq{\isacharparenright}{\kern0pt}\ {\isachardoublequoteopen}j{\isacharequal}{\kern0pt}n{\isachardoublequoteclose}\ \isacommand{using}\isamarkupfalse%
\ leD\ \isacommand{by}\isamarkupfalse%
\ auto\isanewline
\ \ \ \ \isacommand{then}\isamarkupfalse%
\ \isanewline
\ \ \ \ \isacommand{have}\isamarkupfalse%
\ {\isachardoublequoteopen}{\isacharparenleft}{\kern0pt}if\ j\ {\isacharless}{\kern0pt}\ n\ then\ f{\isacharbackquote}{\kern0pt}j\ else\ a{\isacharparenright}{\kern0pt}\ {\isasymin}\ A{\isachardoublequoteclose}\isanewline
\ \ \ \ \isacommand{proof}\isamarkupfalse%
\ cases\isanewline
\ \ \ \ \ \ \isacommand{case}\isamarkupfalse%
\ lt\isanewline
\ \ \ \ \ \ \isacommand{with}\isamarkupfalse%
\ {\isacartoucheopen}f{\isasymin}{\isacharunderscore}{\kern0pt}{\isacartoucheclose}\ \isanewline
\ \ \ \ \ \ \isacommand{show}\isamarkupfalse%
\ {\isacharquery}{\kern0pt}thesis\ \isacommand{using}\isamarkupfalse%
\ apply{\isacharunderscore}{\kern0pt}type\ ltD{\isacharbrackleft}{\kern0pt}OF\ lt{\isacharbrackright}{\kern0pt}\ \isacommand{by}\isamarkupfalse%
\ simp\isanewline
\ \ \ \ \isacommand{next}\isamarkupfalse%
\isanewline
\ \ \ \ \ \ \isacommand{case}\isamarkupfalse%
\ eq\isanewline
\ \ \ \ \ \ \isacommand{with}\isamarkupfalse%
\ {\isacartoucheopen}a{\isasymin}{\isacharunderscore}{\kern0pt}{\isacartoucheclose}\isanewline
\ \ \ \ \ \ \isacommand{show}\isamarkupfalse%
\ {\isacharquery}{\kern0pt}thesis\ \isacommand{by}\isamarkupfalse%
\ auto\isanewline
\ \ \ \ \isacommand{qed}\isamarkupfalse%
\isanewline
\ \ \isacommand{{\isacharbraceright}{\kern0pt}}\isamarkupfalse%
\isanewline
\ \ \isacommand{with}\isamarkupfalse%
\ equ\isanewline
\ \ \isacommand{show}\isamarkupfalse%
\ {\isacharquery}{\kern0pt}thesis\isanewline
\ \ \ \ \isacommand{unfolding}\isamarkupfalse%
\ seq{\isacharunderscore}{\kern0pt}upd{\isacharunderscore}{\kern0pt}def\isanewline
\ \ \ \ \isacommand{using}\isamarkupfalse%
\ lam{\isacharunderscore}{\kern0pt}type{\isacharbrackleft}{\kern0pt}of\ {\isachardoublequoteopen}succ{\isacharparenleft}{\kern0pt}domain{\isacharparenleft}{\kern0pt}f{\isacharparenright}{\kern0pt}{\isacharparenright}{\kern0pt}{\isachardoublequoteclose}{\isacharbrackright}{\kern0pt}\isanewline
\ \ \ \ \isacommand{by}\isamarkupfalse%
\ auto\isanewline
\isacommand{qed}\isamarkupfalse%
%
\endisatagproof
{\isafoldproof}%
%
\isadelimproof
\isanewline
%
\endisadelimproof
\isanewline
\isacommand{lemma}\isamarkupfalse%
\ seq{\isacharunderscore}{\kern0pt}upd{\isacharunderscore}{\kern0pt}type\ {\isacharcolon}{\kern0pt}\ \isanewline
\ \ \isakeyword{assumes}\ {\isachardoublequoteopen}f{\isasymin}A{\isacharcircum}{\kern0pt}{\isacharless}{\kern0pt}{\isasymomega}{\isachardoublequoteclose}\ {\isachardoublequoteopen}a{\isasymin}A{\isachardoublequoteclose}\isanewline
\ \ \isakeyword{shows}\ {\isachardoublequoteopen}seq{\isacharunderscore}{\kern0pt}upd{\isacharparenleft}{\kern0pt}f{\isacharcomma}{\kern0pt}a{\isacharparenright}{\kern0pt}\ {\isasymin}\ A{\isacharcircum}{\kern0pt}{\isacharless}{\kern0pt}{\isasymomega}{\isachardoublequoteclose}\isanewline
%
\isadelimproof
%
\endisadelimproof
%
\isatagproof
\isacommand{proof}\isamarkupfalse%
\ {\isacharminus}{\kern0pt}\isanewline
\ \ \isacommand{from}\isamarkupfalse%
\ {\isacartoucheopen}f{\isasymin}{\isacharunderscore}{\kern0pt}{\isacartoucheclose}\isanewline
\ \ \isacommand{obtain}\isamarkupfalse%
\ y\ \isakeyword{where}\ {\isachardoublequoteopen}y{\isasymin}nat{\isachardoublequoteclose}\ {\isachardoublequoteopen}f{\isasymin}y{\isasymrightarrow}A{\isachardoublequoteclose}\isanewline
\ \ \ \ \isacommand{unfolding}\isamarkupfalse%
\ seqspace{\isacharunderscore}{\kern0pt}def\ \isacommand{by}\isamarkupfalse%
\ blast\isanewline
\ \ \isacommand{with}\isamarkupfalse%
\ {\isacartoucheopen}a{\isasymin}A{\isacartoucheclose}\isanewline
\ \ \isacommand{have}\isamarkupfalse%
\ {\isachardoublequoteopen}seq{\isacharunderscore}{\kern0pt}upd{\isacharparenleft}{\kern0pt}f{\isacharcomma}{\kern0pt}a{\isacharparenright}{\kern0pt}{\isasymin}succ{\isacharparenleft}{\kern0pt}y{\isacharparenright}{\kern0pt}{\isasymrightarrow}A{\isachardoublequoteclose}\ \isanewline
\ \ \ \ \isacommand{using}\isamarkupfalse%
\ seq{\isacharunderscore}{\kern0pt}upd{\isacharunderscore}{\kern0pt}succ{\isacharunderscore}{\kern0pt}type\ \isacommand{by}\isamarkupfalse%
\ simp\isanewline
\ \ \isacommand{with}\isamarkupfalse%
\ {\isacartoucheopen}y{\isasymin}{\isacharunderscore}{\kern0pt}{\isacartoucheclose}\isanewline
\ \ \isacommand{show}\isamarkupfalse%
\ {\isacharquery}{\kern0pt}thesis\isanewline
\ \ \ \ \isacommand{unfolding}\isamarkupfalse%
\ seqspace{\isacharunderscore}{\kern0pt}def\ \isacommand{by}\isamarkupfalse%
\ auto\isanewline
\isacommand{qed}\isamarkupfalse%
%
\endisatagproof
{\isafoldproof}%
%
\isadelimproof
\isanewline
%
\endisadelimproof
\isanewline
\isacommand{lemma}\isamarkupfalse%
\ seq{\isacharunderscore}{\kern0pt}upd{\isacharunderscore}{\kern0pt}apply{\isacharunderscore}{\kern0pt}domain\ {\isacharbrackleft}{\kern0pt}simp{\isacharbrackright}{\kern0pt}{\isacharcolon}{\kern0pt}\ \isanewline
\ \ \isakeyword{assumes}\ {\isachardoublequoteopen}f{\isacharcolon}{\kern0pt}n{\isasymrightarrow}A{\isachardoublequoteclose}\ {\isachardoublequoteopen}n{\isasymin}nat{\isachardoublequoteclose}\isanewline
\ \ \isakeyword{shows}\ {\isachardoublequoteopen}seq{\isacharunderscore}{\kern0pt}upd{\isacharparenleft}{\kern0pt}f{\isacharcomma}{\kern0pt}a{\isacharparenright}{\kern0pt}{\isacharbackquote}{\kern0pt}n\ {\isacharequal}{\kern0pt}\ a{\isachardoublequoteclose}\isanewline
%
\isadelimproof
\ \ %
\endisadelimproof
%
\isatagproof
\isacommand{unfolding}\isamarkupfalse%
\ seq{\isacharunderscore}{\kern0pt}upd{\isacharunderscore}{\kern0pt}def\ \isacommand{using}\isamarkupfalse%
\ assms\ domain{\isacharunderscore}{\kern0pt}of{\isacharunderscore}{\kern0pt}fun\ \isacommand{by}\isamarkupfalse%
\ auto%
\endisatagproof
{\isafoldproof}%
%
\isadelimproof
\isanewline
%
\endisadelimproof
\isanewline
\isacommand{lemma}\isamarkupfalse%
\ zero{\isacharunderscore}{\kern0pt}in{\isacharunderscore}{\kern0pt}seqspace\ {\isacharcolon}{\kern0pt}\ \isanewline
\ \ \isakeyword{shows}\ {\isachardoublequoteopen}{\isadigit{0}}\ {\isasymin}\ A{\isacharcircum}{\kern0pt}{\isacharless}{\kern0pt}{\isasymomega}{\isachardoublequoteclose}\isanewline
%
\isadelimproof
\ \ %
\endisadelimproof
%
\isatagproof
\isacommand{unfolding}\isamarkupfalse%
\ seqspace{\isacharunderscore}{\kern0pt}def\isanewline
\ \ \isacommand{by}\isamarkupfalse%
\ force%
\endisatagproof
{\isafoldproof}%
%
\isadelimproof
\isanewline
%
\endisadelimproof
\isanewline
\isacommand{definition}\isamarkupfalse%
\isanewline
\ \ seqleR\ {\isacharcolon}{\kern0pt}{\isacharcolon}{\kern0pt}\ {\isachardoublequoteopen}i\ {\isasymRightarrow}\ i\ {\isasymRightarrow}\ o{\isachardoublequoteclose}\ \isakeyword{where}\isanewline
\ \ {\isachardoublequoteopen}seqleR{\isacharparenleft}{\kern0pt}f{\isacharcomma}{\kern0pt}g{\isacharparenright}{\kern0pt}\ {\isasymequiv}\ g\ {\isasymsubseteq}\ f{\isachardoublequoteclose}\isanewline
\isanewline
\isacommand{definition}\isamarkupfalse%
\isanewline
\ \ seqlerel\ {\isacharcolon}{\kern0pt}{\isacharcolon}{\kern0pt}\ {\isachardoublequoteopen}i\ {\isasymRightarrow}\ i{\isachardoublequoteclose}\ \isakeyword{where}\isanewline
\ \ {\isachardoublequoteopen}seqlerel{\isacharparenleft}{\kern0pt}A{\isacharparenright}{\kern0pt}\ {\isasymequiv}\ Rrel{\isacharparenleft}{\kern0pt}{\isasymlambda}x\ y{\isachardot}{\kern0pt}\ y\ {\isasymsubseteq}\ x{\isacharcomma}{\kern0pt}A{\isacharcircum}{\kern0pt}{\isacharless}{\kern0pt}{\isasymomega}{\isacharparenright}{\kern0pt}{\isachardoublequoteclose}\isanewline
\isanewline
\isacommand{definition}\isamarkupfalse%
\isanewline
\ \ seqle\ {\isacharcolon}{\kern0pt}{\isacharcolon}{\kern0pt}\ {\isachardoublequoteopen}i{\isachardoublequoteclose}\ \isakeyword{where}\isanewline
\ \ {\isachardoublequoteopen}seqle\ {\isasymequiv}\ seqlerel{\isacharparenleft}{\kern0pt}{\isadigit{2}}{\isacharparenright}{\kern0pt}{\isachardoublequoteclose}\isanewline
\isanewline
\isacommand{lemma}\isamarkupfalse%
\ seqleI{\isacharbrackleft}{\kern0pt}intro{\isacharbang}{\kern0pt}{\isacharbrackright}{\kern0pt}{\isacharcolon}{\kern0pt}\ \isanewline
\ \ {\isachardoublequoteopen}{\isasymlangle}f{\isacharcomma}{\kern0pt}g{\isasymrangle}\ {\isasymin}\ {\isadigit{2}}{\isacharcircum}{\kern0pt}{\isacharless}{\kern0pt}{\isasymomega}{\isasymtimes}{\isadigit{2}}{\isacharcircum}{\kern0pt}{\isacharless}{\kern0pt}{\isasymomega}\ {\isasymLongrightarrow}\ g\ {\isasymsubseteq}\ f\ \ {\isasymLongrightarrow}\ {\isasymlangle}f{\isacharcomma}{\kern0pt}g{\isasymrangle}\ {\isasymin}\ seqle{\isachardoublequoteclose}\isanewline
%
\isadelimproof
\ \ %
\endisadelimproof
%
\isatagproof
\isacommand{unfolding}\isamarkupfalse%
\ \ seqspace{\isacharunderscore}{\kern0pt}def\ seqle{\isacharunderscore}{\kern0pt}def\ seqlerel{\isacharunderscore}{\kern0pt}def\ Rrel{\isacharunderscore}{\kern0pt}def\ \isanewline
\ \ \isacommand{by}\isamarkupfalse%
\ blast%
\endisatagproof
{\isafoldproof}%
%
\isadelimproof
\isanewline
%
\endisadelimproof
\isanewline
\isacommand{lemma}\isamarkupfalse%
\ seqleD{\isacharbrackleft}{\kern0pt}dest{\isacharbang}{\kern0pt}{\isacharbrackright}{\kern0pt}{\isacharcolon}{\kern0pt}\ \isanewline
\ \ {\isachardoublequoteopen}z\ {\isasymin}\ seqle\ {\isasymLongrightarrow}\ {\isasymexists}x\ y{\isachardot}{\kern0pt}\ {\isasymlangle}x{\isacharcomma}{\kern0pt}y{\isasymrangle}\ {\isasymin}\ {\isadigit{2}}{\isacharcircum}{\kern0pt}{\isacharless}{\kern0pt}{\isasymomega}{\isasymtimes}{\isadigit{2}}{\isacharcircum}{\kern0pt}{\isacharless}{\kern0pt}{\isasymomega}\ {\isasymand}\ y\ {\isasymsubseteq}\ x\ {\isasymand}\ z\ {\isacharequal}{\kern0pt}\ {\isasymlangle}x{\isacharcomma}{\kern0pt}y{\isasymrangle}{\isachardoublequoteclose}\isanewline
%
\isadelimproof
\ \ %
\endisadelimproof
%
\isatagproof
\isacommand{unfolding}\isamarkupfalse%
\ seqle{\isacharunderscore}{\kern0pt}def\ seqlerel{\isacharunderscore}{\kern0pt}def\ Rrel{\isacharunderscore}{\kern0pt}def\ \isanewline
\ \ \isacommand{by}\isamarkupfalse%
\ blast%
\endisatagproof
{\isafoldproof}%
%
\isadelimproof
\isanewline
%
\endisadelimproof
\isanewline
\isacommand{lemma}\isamarkupfalse%
\ upd{\isacharunderscore}{\kern0pt}leI\ {\isacharcolon}{\kern0pt}\ \isanewline
\ \ \isakeyword{assumes}\ {\isachardoublequoteopen}f{\isasymin}{\isadigit{2}}{\isacharcircum}{\kern0pt}{\isacharless}{\kern0pt}{\isasymomega}{\isachardoublequoteclose}\ {\isachardoublequoteopen}a{\isasymin}{\isadigit{2}}{\isachardoublequoteclose}\isanewline
\ \ \isakeyword{shows}\ {\isachardoublequoteopen}{\isasymlangle}seq{\isacharunderscore}{\kern0pt}upd{\isacharparenleft}{\kern0pt}f{\isacharcomma}{\kern0pt}a{\isacharparenright}{\kern0pt}{\isacharcomma}{\kern0pt}f{\isasymrangle}{\isasymin}seqle{\isachardoublequoteclose}\ \ {\isacharparenleft}{\kern0pt}\isakeyword{is}\ {\isachardoublequoteopen}{\isasymlangle}{\isacharquery}{\kern0pt}f{\isacharcomma}{\kern0pt}{\isacharunderscore}{\kern0pt}{\isasymrangle}{\isasymin}{\isacharunderscore}{\kern0pt}{\isachardoublequoteclose}{\isacharparenright}{\kern0pt}\isanewline
%
\isadelimproof
%
\endisadelimproof
%
\isatagproof
\isacommand{proof}\isamarkupfalse%
\isanewline
\ \ \isacommand{show}\isamarkupfalse%
\ {\isachardoublequoteopen}\ {\isasymlangle}{\isacharquery}{\kern0pt}f{\isacharcomma}{\kern0pt}\ f{\isasymrangle}\ {\isasymin}\ {\isadigit{2}}{\isacharcircum}{\kern0pt}{\isacharless}{\kern0pt}{\isasymomega}\ {\isasymtimes}\ {\isadigit{2}}{\isacharcircum}{\kern0pt}{\isacharless}{\kern0pt}{\isasymomega}{\isachardoublequoteclose}\ \isanewline
\ \ \ \ \isacommand{using}\isamarkupfalse%
\ assms\ seq{\isacharunderscore}{\kern0pt}upd{\isacharunderscore}{\kern0pt}type\ \isacommand{by}\isamarkupfalse%
\ auto\isanewline
\isacommand{next}\isamarkupfalse%
\isanewline
\ \ \isacommand{show}\isamarkupfalse%
\ \ {\isachardoublequoteopen}f\ {\isasymsubseteq}\ seq{\isacharunderscore}{\kern0pt}upd{\isacharparenleft}{\kern0pt}f{\isacharcomma}{\kern0pt}a{\isacharparenright}{\kern0pt}{\isachardoublequoteclose}\ \isanewline
\ \ \isacommand{proof}\isamarkupfalse%
\ \isanewline
\ \ \ \ \isacommand{fix}\isamarkupfalse%
\ x\isanewline
\ \ \ \ \isacommand{assume}\isamarkupfalse%
\ {\isachardoublequoteopen}x\ {\isasymin}\ f{\isachardoublequoteclose}\isanewline
\ \ \ \ \isacommand{moreover}\isamarkupfalse%
\ \isacommand{from}\isamarkupfalse%
\ {\isacartoucheopen}f\ {\isasymin}\ {\isadigit{2}}{\isacharcircum}{\kern0pt}{\isacharless}{\kern0pt}{\isasymomega}{\isacartoucheclose}\isanewline
\ \ \ \ \isacommand{obtain}\isamarkupfalse%
\ n\ \isakeyword{where}\ \ {\isachardoublequoteopen}n{\isasymin}nat{\isachardoublequoteclose}\ {\isachardoublequoteopen}f\ {\isacharcolon}{\kern0pt}\ n\ {\isasymrightarrow}\ {\isadigit{2}}{\isachardoublequoteclose}\isanewline
\ \ \ \ \ \ \isacommand{using}\isamarkupfalse%
\ seqspace{\isacharunderscore}{\kern0pt}type\ \isacommand{by}\isamarkupfalse%
\ blast\isanewline
\ \ \ \ \isacommand{moreover}\isamarkupfalse%
\ \isacommand{from}\isamarkupfalse%
\ calculation\isanewline
\ \ \ \ \isacommand{obtain}\isamarkupfalse%
\ y\ \isakeyword{where}\ {\isachardoublequoteopen}y{\isasymin}n{\isachardoublequoteclose}\ {\isachardoublequoteopen}x{\isacharequal}{\kern0pt}{\isasymlangle}y{\isacharcomma}{\kern0pt}f{\isacharbackquote}{\kern0pt}y{\isasymrangle}{\isachardoublequoteclose}\ \isacommand{using}\isamarkupfalse%
\ Pi{\isacharunderscore}{\kern0pt}memberD{\isacharbrackleft}{\kern0pt}of\ f\ n\ {\isachardoublequoteopen}{\isasymlambda}{\isacharunderscore}{\kern0pt}\ {\isachardot}{\kern0pt}\ {\isadigit{2}}{\isachardoublequoteclose}{\isacharbrackright}{\kern0pt}\ \isanewline
\ \ \ \ \ \ \isacommand{by}\isamarkupfalse%
\ blast\isanewline
\ \ \ \ \isacommand{moreover}\isamarkupfalse%
\ \isacommand{from}\isamarkupfalse%
\ {\isacartoucheopen}f{\isacharcolon}{\kern0pt}n{\isasymrightarrow}{\isadigit{2}}{\isacartoucheclose}\isanewline
\ \ \ \ \isacommand{have}\isamarkupfalse%
\ {\isachardoublequoteopen}domain{\isacharparenleft}{\kern0pt}f{\isacharparenright}{\kern0pt}\ {\isacharequal}{\kern0pt}\ n{\isachardoublequoteclose}\ \isacommand{using}\isamarkupfalse%
\ domain{\isacharunderscore}{\kern0pt}of{\isacharunderscore}{\kern0pt}fun\ \isacommand{by}\isamarkupfalse%
\ simp\isanewline
\ \ \ \ \isacommand{ultimately}\isamarkupfalse%
\isanewline
\ \ \ \ \isacommand{show}\isamarkupfalse%
\ {\isachardoublequoteopen}x\ {\isasymin}\ seq{\isacharunderscore}{\kern0pt}upd{\isacharparenleft}{\kern0pt}f{\isacharcomma}{\kern0pt}a{\isacharparenright}{\kern0pt}{\isachardoublequoteclose}\isanewline
\ \ \ \ \ \ \isacommand{unfolding}\isamarkupfalse%
\ seq{\isacharunderscore}{\kern0pt}upd{\isacharunderscore}{\kern0pt}def\ lam{\isacharunderscore}{\kern0pt}def\ \ \isanewline
\ \ \ \ \ \ \isacommand{by}\isamarkupfalse%
\ {\isacharparenleft}{\kern0pt}auto\ intro{\isacharcolon}{\kern0pt}ltI{\isacharparenright}{\kern0pt}\isanewline
\ \ \isacommand{qed}\isamarkupfalse%
\isanewline
\isacommand{qed}\isamarkupfalse%
%
\endisatagproof
{\isafoldproof}%
%
\isadelimproof
\isanewline
%
\endisadelimproof
\isanewline
\isacommand{lemma}\isamarkupfalse%
\ preorder{\isacharunderscore}{\kern0pt}on{\isacharunderscore}{\kern0pt}seqle{\isacharcolon}{\kern0pt}\ {\isachardoublequoteopen}preorder{\isacharunderscore}{\kern0pt}on{\isacharparenleft}{\kern0pt}{\isadigit{2}}{\isacharcircum}{\kern0pt}{\isacharless}{\kern0pt}{\isasymomega}{\isacharcomma}{\kern0pt}seqle{\isacharparenright}{\kern0pt}{\isachardoublequoteclose}\isanewline
%
\isadelimproof
\ \ %
\endisadelimproof
%
\isatagproof
\isacommand{unfolding}\isamarkupfalse%
\ preorder{\isacharunderscore}{\kern0pt}on{\isacharunderscore}{\kern0pt}def\ refl{\isacharunderscore}{\kern0pt}def\ trans{\isacharunderscore}{\kern0pt}on{\isacharunderscore}{\kern0pt}def\ \isacommand{by}\isamarkupfalse%
\ blast%
\endisatagproof
{\isafoldproof}%
%
\isadelimproof
\isanewline
%
\endisadelimproof
\isanewline
\isacommand{lemma}\isamarkupfalse%
\ zero{\isacharunderscore}{\kern0pt}seqle{\isacharunderscore}{\kern0pt}max{\isacharcolon}{\kern0pt}\ {\isachardoublequoteopen}x{\isasymin}{\isadigit{2}}{\isacharcircum}{\kern0pt}{\isacharless}{\kern0pt}{\isasymomega}\ {\isasymLongrightarrow}\ {\isasymlangle}x{\isacharcomma}{\kern0pt}{\isadigit{0}}{\isasymrangle}\ {\isasymin}\ seqle{\isachardoublequoteclose}\isanewline
%
\isadelimproof
\ \ %
\endisadelimproof
%
\isatagproof
\isacommand{using}\isamarkupfalse%
\ zero{\isacharunderscore}{\kern0pt}in{\isacharunderscore}{\kern0pt}seqspace\ \isanewline
\ \ \isacommand{by}\isamarkupfalse%
\ auto%
\endisatagproof
{\isafoldproof}%
%
\isadelimproof
\isanewline
%
\endisadelimproof
\isanewline
\isacommand{interpretation}\isamarkupfalse%
\ forcing{\isacharunderscore}{\kern0pt}notion\ {\isachardoublequoteopen}{\isadigit{2}}{\isacharcircum}{\kern0pt}{\isacharless}{\kern0pt}{\isasymomega}{\isachardoublequoteclose}\ {\isachardoublequoteopen}seqle{\isachardoublequoteclose}\ {\isachardoublequoteopen}{\isadigit{0}}{\isachardoublequoteclose}\isanewline
%
\isadelimproof
\ \ %
\endisadelimproof
%
\isatagproof
\isacommand{using}\isamarkupfalse%
\ preorder{\isacharunderscore}{\kern0pt}on{\isacharunderscore}{\kern0pt}seqle\ zero{\isacharunderscore}{\kern0pt}seqle{\isacharunderscore}{\kern0pt}max\ zero{\isacharunderscore}{\kern0pt}in{\isacharunderscore}{\kern0pt}seqspace\ \isanewline
\ \ \isacommand{by}\isamarkupfalse%
\ unfold{\isacharunderscore}{\kern0pt}locales\ simp{\isacharunderscore}{\kern0pt}all%
\endisatagproof
{\isafoldproof}%
%
\isadelimproof
\isanewline
%
\endisadelimproof
\isanewline
\isacommand{abbreviation}\isamarkupfalse%
\ SEQle\ {\isacharcolon}{\kern0pt}{\isacharcolon}{\kern0pt}\ {\isachardoublequoteopen}{\isacharbrackleft}{\kern0pt}i{\isacharcomma}{\kern0pt}\ i{\isacharbrackright}{\kern0pt}\ {\isasymRightarrow}\ o{\isachardoublequoteclose}\ \ {\isacharparenleft}{\kern0pt}\isakeyword{infixl}\ {\isachardoublequoteopen}{\isasympreceq}s{\isachardoublequoteclose}\ {\isadigit{5}}{\isadigit{0}}{\isacharparenright}{\kern0pt}\isanewline
\ \ \isakeyword{where}\ {\isachardoublequoteopen}x\ {\isasympreceq}s\ y\ {\isasymequiv}\ Leq{\isacharparenleft}{\kern0pt}x{\isacharcomma}{\kern0pt}y{\isacharparenright}{\kern0pt}{\isachardoublequoteclose}\isanewline
\isanewline
\isacommand{abbreviation}\isamarkupfalse%
\ SEQIncompatible\ {\isacharcolon}{\kern0pt}{\isacharcolon}{\kern0pt}\ {\isachardoublequoteopen}{\isacharbrackleft}{\kern0pt}i{\isacharcomma}{\kern0pt}\ i{\isacharbrackright}{\kern0pt}\ {\isasymRightarrow}\ o{\isachardoublequoteclose}\ \ {\isacharparenleft}{\kern0pt}\isakeyword{infixl}\ {\isachardoublequoteopen}{\isasymbottom}s{\isachardoublequoteclose}\ {\isadigit{5}}{\isadigit{0}}{\isacharparenright}{\kern0pt}\isanewline
\ \ \isakeyword{where}\ {\isachardoublequoteopen}x\ {\isasymbottom}s\ y\ {\isasymequiv}\ Incompatible{\isacharparenleft}{\kern0pt}x{\isacharcomma}{\kern0pt}y{\isacharparenright}{\kern0pt}{\isachardoublequoteclose}\isanewline
\isanewline
\isacommand{lemma}\isamarkupfalse%
\ seqspace{\isacharunderscore}{\kern0pt}separative{\isacharcolon}{\kern0pt}\isanewline
\ \ \isakeyword{assumes}\ {\isachardoublequoteopen}f{\isasymin}{\isadigit{2}}{\isacharcircum}{\kern0pt}{\isacharless}{\kern0pt}{\isasymomega}{\isachardoublequoteclose}\isanewline
\ \ \isakeyword{shows}\ {\isachardoublequoteopen}seq{\isacharunderscore}{\kern0pt}upd{\isacharparenleft}{\kern0pt}f{\isacharcomma}{\kern0pt}{\isadigit{0}}{\isacharparenright}{\kern0pt}\ {\isasymbottom}s\ seq{\isacharunderscore}{\kern0pt}upd{\isacharparenleft}{\kern0pt}f{\isacharcomma}{\kern0pt}{\isadigit{1}}{\isacharparenright}{\kern0pt}{\isachardoublequoteclose}\ {\isacharparenleft}{\kern0pt}\isakeyword{is}\ {\isachardoublequoteopen}{\isacharquery}{\kern0pt}f\ {\isasymbottom}s\ {\isacharquery}{\kern0pt}g{\isachardoublequoteclose}{\isacharparenright}{\kern0pt}\isanewline
%
\isadelimproof
%
\endisadelimproof
%
\isatagproof
\isacommand{proof}\isamarkupfalse%
\ \isanewline
\ \ \isacommand{assume}\isamarkupfalse%
\ {\isachardoublequoteopen}compat{\isacharparenleft}{\kern0pt}{\isacharquery}{\kern0pt}f{\isacharcomma}{\kern0pt}\ {\isacharquery}{\kern0pt}g{\isacharparenright}{\kern0pt}{\isachardoublequoteclose}\isanewline
\ \ \isacommand{then}\isamarkupfalse%
\ \isanewline
\ \ \isacommand{obtain}\isamarkupfalse%
\ h\ \isakeyword{where}\ {\isachardoublequoteopen}h\ {\isasymin}\ {\isadigit{2}}{\isacharcircum}{\kern0pt}{\isacharless}{\kern0pt}{\isasymomega}{\isachardoublequoteclose}\ {\isachardoublequoteopen}{\isacharquery}{\kern0pt}f\ {\isasymsubseteq}\ h{\isachardoublequoteclose}\ {\isachardoublequoteopen}{\isacharquery}{\kern0pt}g\ {\isasymsubseteq}\ h{\isachardoublequoteclose}\isanewline
\ \ \ \ \isacommand{by}\isamarkupfalse%
\ blast\isanewline
\ \ \isacommand{moreover}\isamarkupfalse%
\ \isacommand{from}\isamarkupfalse%
\ {\isacartoucheopen}f{\isasymin}{\isacharunderscore}{\kern0pt}{\isacartoucheclose}\isanewline
\ \ \isacommand{obtain}\isamarkupfalse%
\ y\ \isakeyword{where}\ {\isachardoublequoteopen}y{\isasymin}nat{\isachardoublequoteclose}\ {\isachardoublequoteopen}f{\isacharcolon}{\kern0pt}y{\isasymrightarrow}{\isadigit{2}}{\isachardoublequoteclose}\ \isacommand{by}\isamarkupfalse%
\ blast\isanewline
\ \ \isacommand{moreover}\isamarkupfalse%
\ \isacommand{from}\isamarkupfalse%
\ this\isanewline
\ \ \isacommand{have}\isamarkupfalse%
\ {\isachardoublequoteopen}{\isacharquery}{\kern0pt}f{\isacharcolon}{\kern0pt}\ succ{\isacharparenleft}{\kern0pt}y{\isacharparenright}{\kern0pt}\ {\isasymrightarrow}\ {\isadigit{2}}{\isachardoublequoteclose}\ {\isachardoublequoteopen}{\isacharquery}{\kern0pt}g{\isacharcolon}{\kern0pt}\ succ{\isacharparenleft}{\kern0pt}y{\isacharparenright}{\kern0pt}\ {\isasymrightarrow}\ {\isadigit{2}}{\isachardoublequoteclose}\ \isanewline
\ \ \ \ \isacommand{using}\isamarkupfalse%
\ seq{\isacharunderscore}{\kern0pt}upd{\isacharunderscore}{\kern0pt}succ{\isacharunderscore}{\kern0pt}type\ \isacommand{by}\isamarkupfalse%
\ blast{\isacharplus}{\kern0pt}\isanewline
\ \ \isacommand{moreover}\isamarkupfalse%
\ \isacommand{from}\isamarkupfalse%
\ this\isanewline
\ \ \isacommand{have}\isamarkupfalse%
\ {\isachardoublequoteopen}{\isasymlangle}y{\isacharcomma}{\kern0pt}{\isacharquery}{\kern0pt}f{\isacharbackquote}{\kern0pt}y{\isasymrangle}\ {\isasymin}\ {\isacharquery}{\kern0pt}f{\isachardoublequoteclose}\ {\isachardoublequoteopen}{\isasymlangle}y{\isacharcomma}{\kern0pt}{\isacharquery}{\kern0pt}g{\isacharbackquote}{\kern0pt}y{\isasymrangle}\ {\isasymin}\ {\isacharquery}{\kern0pt}g{\isachardoublequoteclose}\ \isacommand{using}\isamarkupfalse%
\ apply{\isacharunderscore}{\kern0pt}Pair\ \isacommand{by}\isamarkupfalse%
\ auto\isanewline
\ \ \isacommand{ultimately}\isamarkupfalse%
\isanewline
\ \ \isacommand{have}\isamarkupfalse%
\ {\isachardoublequoteopen}{\isasymlangle}y{\isacharcomma}{\kern0pt}{\isadigit{0}}{\isasymrangle}\ {\isasymin}\ h{\isachardoublequoteclose}\ {\isachardoublequoteopen}{\isasymlangle}y{\isacharcomma}{\kern0pt}{\isadigit{1}}{\isasymrangle}\ {\isasymin}\ h{\isachardoublequoteclose}\ \isacommand{by}\isamarkupfalse%
\ auto\isanewline
\ \ \isacommand{moreover}\isamarkupfalse%
\ \isacommand{from}\isamarkupfalse%
\ {\isacartoucheopen}h\ {\isasymin}\ {\isadigit{2}}{\isacharcircum}{\kern0pt}{\isacharless}{\kern0pt}{\isasymomega}{\isacartoucheclose}\isanewline
\ \ \isacommand{obtain}\isamarkupfalse%
\ n\ \isakeyword{where}\ {\isachardoublequoteopen}n{\isasymin}nat{\isachardoublequoteclose}\ {\isachardoublequoteopen}h{\isacharcolon}{\kern0pt}n{\isasymrightarrow}{\isadigit{2}}{\isachardoublequoteclose}\ \isacommand{by}\isamarkupfalse%
\ blast\isanewline
\ \ \isacommand{ultimately}\isamarkupfalse%
\isanewline
\ \ \isacommand{show}\isamarkupfalse%
\ {\isachardoublequoteopen}False{\isachardoublequoteclose}\isanewline
\ \ \ \ \isacommand{using}\isamarkupfalse%
\ fun{\isacharunderscore}{\kern0pt}is{\isacharunderscore}{\kern0pt}function{\isacharbrackleft}{\kern0pt}of\ h\ n\ {\isachardoublequoteopen}{\isasymlambda}{\isacharunderscore}{\kern0pt}{\isachardot}{\kern0pt}\ {\isadigit{2}}{\isachardoublequoteclose}{\isacharbrackright}{\kern0pt}\ \isanewline
\ \ \ \ \isacommand{unfolding}\isamarkupfalse%
\ seqspace{\isacharunderscore}{\kern0pt}def\ function{\isacharunderscore}{\kern0pt}def\ \isacommand{by}\isamarkupfalse%
\ auto\isanewline
\isacommand{qed}\isamarkupfalse%
%
\endisatagproof
{\isafoldproof}%
%
\isadelimproof
\isanewline
%
\endisadelimproof
\isanewline
\isacommand{definition}\isamarkupfalse%
\ is{\isacharunderscore}{\kern0pt}seqleR\ {\isacharcolon}{\kern0pt}{\isacharcolon}{\kern0pt}\ {\isachardoublequoteopen}{\isacharbrackleft}{\kern0pt}i{\isasymRightarrow}o{\isacharcomma}{\kern0pt}i{\isacharcomma}{\kern0pt}i{\isacharbrackright}{\kern0pt}\ {\isasymRightarrow}\ o{\isachardoublequoteclose}\ \isakeyword{where}\isanewline
\ \ {\isachardoublequoteopen}is{\isacharunderscore}{\kern0pt}seqleR{\isacharparenleft}{\kern0pt}Q{\isacharcomma}{\kern0pt}f{\isacharcomma}{\kern0pt}g{\isacharparenright}{\kern0pt}\ {\isasymequiv}\ g\ {\isasymsubseteq}\ f{\isachardoublequoteclose}\isanewline
\isanewline
\isacommand{definition}\isamarkupfalse%
\ seqleR{\isacharunderscore}{\kern0pt}fm\ {\isacharcolon}{\kern0pt}{\isacharcolon}{\kern0pt}\ {\isachardoublequoteopen}i\ {\isasymRightarrow}\ i{\isachardoublequoteclose}\ \isakeyword{where}\isanewline
\ \ {\isachardoublequoteopen}seqleR{\isacharunderscore}{\kern0pt}fm{\isacharparenleft}{\kern0pt}fg{\isacharparenright}{\kern0pt}\ {\isasymequiv}\ Exists{\isacharparenleft}{\kern0pt}Exists{\isacharparenleft}{\kern0pt}And{\isacharparenleft}{\kern0pt}pair{\isacharunderscore}{\kern0pt}fm{\isacharparenleft}{\kern0pt}{\isadigit{0}}{\isacharcomma}{\kern0pt}{\isadigit{1}}{\isacharcomma}{\kern0pt}fg{\isacharhash}{\kern0pt}{\isacharplus}{\kern0pt}{\isadigit{2}}{\isacharparenright}{\kern0pt}{\isacharcomma}{\kern0pt}subset{\isacharunderscore}{\kern0pt}fm{\isacharparenleft}{\kern0pt}{\isadigit{1}}{\isacharcomma}{\kern0pt}{\isadigit{0}}{\isacharparenright}{\kern0pt}{\isacharparenright}{\kern0pt}{\isacharparenright}{\kern0pt}{\isacharparenright}{\kern0pt}{\isachardoublequoteclose}\isanewline
\isanewline
\isacommand{lemma}\isamarkupfalse%
\ type{\isacharunderscore}{\kern0pt}seqleR{\isacharunderscore}{\kern0pt}fm\ {\isacharcolon}{\kern0pt}\isanewline
\ \ {\isachardoublequoteopen}fg\ {\isasymin}\ nat\ {\isasymLongrightarrow}\ seqleR{\isacharunderscore}{\kern0pt}fm{\isacharparenleft}{\kern0pt}fg{\isacharparenright}{\kern0pt}\ {\isasymin}\ formula{\isachardoublequoteclose}\isanewline
%
\isadelimproof
\ \ %
\endisadelimproof
%
\isatagproof
\isacommand{unfolding}\isamarkupfalse%
\ seqleR{\isacharunderscore}{\kern0pt}fm{\isacharunderscore}{\kern0pt}def\ \isanewline
\ \ \isacommand{by}\isamarkupfalse%
\ simp%
\endisatagproof
{\isafoldproof}%
%
\isadelimproof
\isanewline
%
\endisadelimproof
\isanewline
\isacommand{lemma}\isamarkupfalse%
\ arity{\isacharunderscore}{\kern0pt}seqleR{\isacharunderscore}{\kern0pt}fm\ {\isacharcolon}{\kern0pt}\isanewline
\ \ {\isachardoublequoteopen}fg\ {\isasymin}\ nat\ {\isasymLongrightarrow}\ arity{\isacharparenleft}{\kern0pt}seqleR{\isacharunderscore}{\kern0pt}fm{\isacharparenleft}{\kern0pt}fg{\isacharparenright}{\kern0pt}{\isacharparenright}{\kern0pt}\ {\isacharequal}{\kern0pt}\ succ{\isacharparenleft}{\kern0pt}fg{\isacharparenright}{\kern0pt}{\isachardoublequoteclose}\isanewline
%
\isadelimproof
\ \ %
\endisadelimproof
%
\isatagproof
\isacommand{unfolding}\isamarkupfalse%
\ seqleR{\isacharunderscore}{\kern0pt}fm{\isacharunderscore}{\kern0pt}def\ \isanewline
\ \ \isacommand{using}\isamarkupfalse%
\ arity{\isacharunderscore}{\kern0pt}pair{\isacharunderscore}{\kern0pt}fm\ arity{\isacharunderscore}{\kern0pt}subset{\isacharunderscore}{\kern0pt}fm\ nat{\isacharunderscore}{\kern0pt}simp{\isacharunderscore}{\kern0pt}union\ \isacommand{by}\isamarkupfalse%
\ simp%
\endisatagproof
{\isafoldproof}%
%
\isadelimproof
\isanewline
%
\endisadelimproof
\isanewline
\isacommand{lemma}\isamarkupfalse%
\ {\isacharparenleft}{\kern0pt}\isakeyword{in}\ M{\isacharunderscore}{\kern0pt}basic{\isacharparenright}{\kern0pt}\ seqleR{\isacharunderscore}{\kern0pt}abs{\isacharcolon}{\kern0pt}\ \isanewline
\ \ \isakeyword{assumes}\ {\isachardoublequoteopen}M{\isacharparenleft}{\kern0pt}f{\isacharparenright}{\kern0pt}{\isachardoublequoteclose}\ {\isachardoublequoteopen}M{\isacharparenleft}{\kern0pt}g{\isacharparenright}{\kern0pt}{\isachardoublequoteclose}\isanewline
\ \ \isakeyword{shows}\ {\isachardoublequoteopen}seqleR{\isacharparenleft}{\kern0pt}f{\isacharcomma}{\kern0pt}g{\isacharparenright}{\kern0pt}\ {\isasymlongleftrightarrow}\ is{\isacharunderscore}{\kern0pt}seqleR{\isacharparenleft}{\kern0pt}M{\isacharcomma}{\kern0pt}f{\isacharcomma}{\kern0pt}g{\isacharparenright}{\kern0pt}{\isachardoublequoteclose}\isanewline
%
\isadelimproof
\ \ %
\endisadelimproof
%
\isatagproof
\isacommand{unfolding}\isamarkupfalse%
\ seqleR{\isacharunderscore}{\kern0pt}def\ is{\isacharunderscore}{\kern0pt}seqleR{\isacharunderscore}{\kern0pt}def\ \isanewline
\ \ \isacommand{using}\isamarkupfalse%
\ assms\ apply{\isacharunderscore}{\kern0pt}abs\ domain{\isacharunderscore}{\kern0pt}abs\ domain{\isacharunderscore}{\kern0pt}closed{\isacharbrackleft}{\kern0pt}OF\ {\isacartoucheopen}M{\isacharparenleft}{\kern0pt}f{\isacharparenright}{\kern0pt}{\isacartoucheclose}{\isacharbrackright}{\kern0pt}\ \ domain{\isacharunderscore}{\kern0pt}closed{\isacharbrackleft}{\kern0pt}OF\ {\isacartoucheopen}M{\isacharparenleft}{\kern0pt}g{\isacharparenright}{\kern0pt}{\isacartoucheclose}{\isacharbrackright}{\kern0pt}\isanewline
\ \ \isacommand{by}\isamarkupfalse%
\ auto%
\endisatagproof
{\isafoldproof}%
%
\isadelimproof
\isanewline
%
\endisadelimproof
\isanewline
\isacommand{definition}\isamarkupfalse%
\isanewline
\ \ relP\ {\isacharcolon}{\kern0pt}{\isacharcolon}{\kern0pt}\ {\isachardoublequoteopen}{\isacharbrackleft}{\kern0pt}i{\isasymRightarrow}o{\isacharcomma}{\kern0pt}{\isacharbrackleft}{\kern0pt}i{\isasymRightarrow}o{\isacharcomma}{\kern0pt}i{\isacharcomma}{\kern0pt}i{\isacharbrackright}{\kern0pt}{\isasymRightarrow}o{\isacharcomma}{\kern0pt}i{\isacharbrackright}{\kern0pt}\ {\isasymRightarrow}\ o{\isachardoublequoteclose}\ \isakeyword{where}\isanewline
\ \ {\isachardoublequoteopen}relP{\isacharparenleft}{\kern0pt}M{\isacharcomma}{\kern0pt}r{\isacharcomma}{\kern0pt}xy{\isacharparenright}{\kern0pt}\ {\isasymequiv}\ {\isacharparenleft}{\kern0pt}{\isasymexists}x{\isacharbrackleft}{\kern0pt}M{\isacharbrackright}{\kern0pt}{\isachardot}{\kern0pt}\ {\isasymexists}y{\isacharbrackleft}{\kern0pt}M{\isacharbrackright}{\kern0pt}{\isachardot}{\kern0pt}\ pair{\isacharparenleft}{\kern0pt}M{\isacharcomma}{\kern0pt}x{\isacharcomma}{\kern0pt}y{\isacharcomma}{\kern0pt}xy{\isacharparenright}{\kern0pt}\ {\isasymand}\ r{\isacharparenleft}{\kern0pt}M{\isacharcomma}{\kern0pt}x{\isacharcomma}{\kern0pt}y{\isacharparenright}{\kern0pt}{\isacharparenright}{\kern0pt}{\isachardoublequoteclose}\isanewline
\isanewline
\isacommand{lemma}\isamarkupfalse%
\ {\isacharparenleft}{\kern0pt}\isakeyword{in}\ M{\isacharunderscore}{\kern0pt}ctm{\isacharparenright}{\kern0pt}\ seqleR{\isacharunderscore}{\kern0pt}fm{\isacharunderscore}{\kern0pt}sats\ {\isacharcolon}{\kern0pt}\ \isanewline
\ \ \isakeyword{assumes}\ {\isachardoublequoteopen}fg{\isasymin}nat{\isachardoublequoteclose}\ {\isachardoublequoteopen}env{\isasymin}list{\isacharparenleft}{\kern0pt}M{\isacharparenright}{\kern0pt}{\isachardoublequoteclose}\ \isanewline
\ \ \isakeyword{shows}\ {\isachardoublequoteopen}sats{\isacharparenleft}{\kern0pt}M{\isacharcomma}{\kern0pt}seqleR{\isacharunderscore}{\kern0pt}fm{\isacharparenleft}{\kern0pt}fg{\isacharparenright}{\kern0pt}{\isacharcomma}{\kern0pt}env{\isacharparenright}{\kern0pt}\ {\isasymlongleftrightarrow}\ relP{\isacharparenleft}{\kern0pt}{\isacharhash}{\kern0pt}{\isacharhash}{\kern0pt}M{\isacharcomma}{\kern0pt}is{\isacharunderscore}{\kern0pt}seqleR{\isacharcomma}{\kern0pt}nth{\isacharparenleft}{\kern0pt}fg{\isacharcomma}{\kern0pt}\ env{\isacharparenright}{\kern0pt}{\isacharparenright}{\kern0pt}{\isachardoublequoteclose}\isanewline
%
\isadelimproof
\ \ %
\endisadelimproof
%
\isatagproof
\isacommand{unfolding}\isamarkupfalse%
\ seqleR{\isacharunderscore}{\kern0pt}fm{\isacharunderscore}{\kern0pt}def\ is{\isacharunderscore}{\kern0pt}seqleR{\isacharunderscore}{\kern0pt}def\ relP{\isacharunderscore}{\kern0pt}def\isanewline
\ \ \isacommand{using}\isamarkupfalse%
\ assms\ trans{\isacharunderscore}{\kern0pt}M\ sats{\isacharunderscore}{\kern0pt}subset{\isacharunderscore}{\kern0pt}fm\ pair{\isacharunderscore}{\kern0pt}iff{\isacharunderscore}{\kern0pt}sats\isanewline
\ \ \isacommand{by}\isamarkupfalse%
\ auto%
\endisatagproof
{\isafoldproof}%
%
\isadelimproof
\isanewline
%
\endisadelimproof
\isanewline
\isanewline
\isacommand{lemma}\isamarkupfalse%
\ {\isacharparenleft}{\kern0pt}\isakeyword{in}\ M{\isacharunderscore}{\kern0pt}basic{\isacharparenright}{\kern0pt}\ is{\isacharunderscore}{\kern0pt}related{\isacharunderscore}{\kern0pt}abs\ {\isacharcolon}{\kern0pt}\isanewline
\ \ \isakeyword{assumes}\ {\isachardoublequoteopen}{\isasymAnd}\ f\ g\ {\isachardot}{\kern0pt}\ M{\isacharparenleft}{\kern0pt}f{\isacharparenright}{\kern0pt}\ {\isasymLongrightarrow}\ M{\isacharparenleft}{\kern0pt}g{\isacharparenright}{\kern0pt}\ {\isasymLongrightarrow}\ rel{\isacharparenleft}{\kern0pt}f{\isacharcomma}{\kern0pt}g{\isacharparenright}{\kern0pt}\ {\isasymlongleftrightarrow}\ is{\isacharunderscore}{\kern0pt}rel{\isacharparenleft}{\kern0pt}M{\isacharcomma}{\kern0pt}f{\isacharcomma}{\kern0pt}g{\isacharparenright}{\kern0pt}{\isachardoublequoteclose}\isanewline
\ \ \isakeyword{shows}\ {\isachardoublequoteopen}{\isasymAnd}z\ {\isachardot}{\kern0pt}\ M{\isacharparenleft}{\kern0pt}z{\isacharparenright}{\kern0pt}\ {\isasymLongrightarrow}\ relP{\isacharparenleft}{\kern0pt}M{\isacharcomma}{\kern0pt}is{\isacharunderscore}{\kern0pt}rel{\isacharcomma}{\kern0pt}z{\isacharparenright}{\kern0pt}\ {\isasymlongleftrightarrow}\ {\isacharparenleft}{\kern0pt}{\isasymexists}x\ y{\isachardot}{\kern0pt}\ z\ {\isacharequal}{\kern0pt}\ {\isasymlangle}x{\isacharcomma}{\kern0pt}y{\isasymrangle}\ {\isasymand}\ rel{\isacharparenleft}{\kern0pt}x{\isacharcomma}{\kern0pt}y{\isacharparenright}{\kern0pt}{\isacharparenright}{\kern0pt}{\isachardoublequoteclose}\isanewline
%
\isadelimproof
\ \ %
\endisadelimproof
%
\isatagproof
\isacommand{unfolding}\isamarkupfalse%
\ relP{\isacharunderscore}{\kern0pt}def\ \isacommand{using}\isamarkupfalse%
\ pair{\isacharunderscore}{\kern0pt}in{\isacharunderscore}{\kern0pt}M{\isacharunderscore}{\kern0pt}iff\ assms\ \isacommand{by}\isamarkupfalse%
\ auto%
\endisatagproof
{\isafoldproof}%
%
\isadelimproof
\isanewline
%
\endisadelimproof
\isanewline
\isacommand{definition}\isamarkupfalse%
\isanewline
\ \ is{\isacharunderscore}{\kern0pt}RRel\ {\isacharcolon}{\kern0pt}{\isacharcolon}{\kern0pt}\ {\isachardoublequoteopen}{\isacharbrackleft}{\kern0pt}i{\isasymRightarrow}o{\isacharcomma}{\kern0pt}{\isacharbrackleft}{\kern0pt}i{\isasymRightarrow}o{\isacharcomma}{\kern0pt}i{\isacharcomma}{\kern0pt}i{\isacharbrackright}{\kern0pt}{\isasymRightarrow}o{\isacharcomma}{\kern0pt}i{\isacharcomma}{\kern0pt}i{\isacharbrackright}{\kern0pt}\ {\isasymRightarrow}\ o{\isachardoublequoteclose}\ \isakeyword{where}\isanewline
\ \ {\isachardoublequoteopen}is{\isacharunderscore}{\kern0pt}RRel{\isacharparenleft}{\kern0pt}M{\isacharcomma}{\kern0pt}is{\isacharunderscore}{\kern0pt}r{\isacharcomma}{\kern0pt}A{\isacharcomma}{\kern0pt}r{\isacharparenright}{\kern0pt}\ {\isasymequiv}\ {\isasymexists}A{\isadigit{2}}{\isacharbrackleft}{\kern0pt}M{\isacharbrackright}{\kern0pt}{\isachardot}{\kern0pt}\ cartprod{\isacharparenleft}{\kern0pt}M{\isacharcomma}{\kern0pt}A{\isacharcomma}{\kern0pt}A{\isacharcomma}{\kern0pt}A{\isadigit{2}}{\isacharparenright}{\kern0pt}\ {\isasymand}\ is{\isacharunderscore}{\kern0pt}Collect{\isacharparenleft}{\kern0pt}M{\isacharcomma}{\kern0pt}A{\isadigit{2}}{\isacharcomma}{\kern0pt}\ relP{\isacharparenleft}{\kern0pt}M{\isacharcomma}{\kern0pt}is{\isacharunderscore}{\kern0pt}r{\isacharparenright}{\kern0pt}{\isacharcomma}{\kern0pt}r{\isacharparenright}{\kern0pt}{\isachardoublequoteclose}\isanewline
\isanewline
\isacommand{lemma}\isamarkupfalse%
\ {\isacharparenleft}{\kern0pt}\isakeyword{in}\ M{\isacharunderscore}{\kern0pt}basic{\isacharparenright}{\kern0pt}\ is{\isacharunderscore}{\kern0pt}Rrel{\isacharunderscore}{\kern0pt}abs\ {\isacharcolon}{\kern0pt}\isanewline
\ \ \isakeyword{assumes}\ {\isachardoublequoteopen}M{\isacharparenleft}{\kern0pt}A{\isacharparenright}{\kern0pt}{\isachardoublequoteclose}\ \ {\isachardoublequoteopen}M{\isacharparenleft}{\kern0pt}r{\isacharparenright}{\kern0pt}{\isachardoublequoteclose}\isanewline
\ \ \ \ {\isachardoublequoteopen}{\isasymAnd}\ f\ g\ {\isachardot}{\kern0pt}\ M{\isacharparenleft}{\kern0pt}f{\isacharparenright}{\kern0pt}\ {\isasymLongrightarrow}\ M{\isacharparenleft}{\kern0pt}g{\isacharparenright}{\kern0pt}\ {\isasymLongrightarrow}\ rel{\isacharparenleft}{\kern0pt}f{\isacharcomma}{\kern0pt}g{\isacharparenright}{\kern0pt}\ {\isasymlongleftrightarrow}\ is{\isacharunderscore}{\kern0pt}rel{\isacharparenleft}{\kern0pt}M{\isacharcomma}{\kern0pt}f{\isacharcomma}{\kern0pt}g{\isacharparenright}{\kern0pt}{\isachardoublequoteclose}\isanewline
\ \ \isakeyword{shows}\ {\isachardoublequoteopen}is{\isacharunderscore}{\kern0pt}RRel{\isacharparenleft}{\kern0pt}M{\isacharcomma}{\kern0pt}is{\isacharunderscore}{\kern0pt}rel{\isacharcomma}{\kern0pt}A{\isacharcomma}{\kern0pt}r{\isacharparenright}{\kern0pt}\ {\isasymlongleftrightarrow}\ \ r\ {\isacharequal}{\kern0pt}\ Rrel{\isacharparenleft}{\kern0pt}rel{\isacharcomma}{\kern0pt}A{\isacharparenright}{\kern0pt}{\isachardoublequoteclose}\isanewline
%
\isadelimproof
%
\endisadelimproof
%
\isatagproof
\isacommand{proof}\isamarkupfalse%
\ {\isacharminus}{\kern0pt}\isanewline
\ \ \isacommand{from}\isamarkupfalse%
\ {\isacartoucheopen}M{\isacharparenleft}{\kern0pt}A{\isacharparenright}{\kern0pt}{\isacartoucheclose}\ \isanewline
\ \ \isacommand{have}\isamarkupfalse%
\ {\isachardoublequoteopen}M{\isacharparenleft}{\kern0pt}z{\isacharparenright}{\kern0pt}{\isachardoublequoteclose}\ \isakeyword{if}\ {\isachardoublequoteopen}z{\isasymin}A{\isasymtimes}A{\isachardoublequoteclose}\ \isakeyword{for}\ z\isanewline
\ \ \ \ \isacommand{using}\isamarkupfalse%
\ cartprod{\isacharunderscore}{\kern0pt}closed\ transM{\isacharbrackleft}{\kern0pt}of\ z\ {\isachardoublequoteopen}A{\isasymtimes}A{\isachardoublequoteclose}{\isacharbrackright}{\kern0pt}\ that\ \isacommand{by}\isamarkupfalse%
\ simp\isanewline
\ \ \isacommand{then}\isamarkupfalse%
\isanewline
\ \ \isacommand{have}\isamarkupfalse%
\ A{\isacharcolon}{\kern0pt}{\isachardoublequoteopen}relP{\isacharparenleft}{\kern0pt}M{\isacharcomma}{\kern0pt}\ is{\isacharunderscore}{\kern0pt}rel{\isacharcomma}{\kern0pt}\ z{\isacharparenright}{\kern0pt}\ {\isasymlongleftrightarrow}\ {\isacharparenleft}{\kern0pt}{\isasymexists}x\ y{\isachardot}{\kern0pt}\ z\ {\isacharequal}{\kern0pt}\ {\isasymlangle}x{\isacharcomma}{\kern0pt}\ y{\isasymrangle}\ {\isasymand}\ rel{\isacharparenleft}{\kern0pt}x{\isacharcomma}{\kern0pt}\ y{\isacharparenright}{\kern0pt}{\isacharparenright}{\kern0pt}{\isachardoublequoteclose}\ {\isachardoublequoteopen}M{\isacharparenleft}{\kern0pt}z{\isacharparenright}{\kern0pt}{\isachardoublequoteclose}\ \isakeyword{if}\ {\isachardoublequoteopen}z{\isasymin}A{\isasymtimes}A{\isachardoublequoteclose}\ \isakeyword{for}\ z\isanewline
\ \ \ \ \isacommand{using}\isamarkupfalse%
\ that\ is{\isacharunderscore}{\kern0pt}related{\isacharunderscore}{\kern0pt}abs{\isacharbrackleft}{\kern0pt}of\ rel\ is{\isacharunderscore}{\kern0pt}rel{\isacharcomma}{\kern0pt}OF\ assms{\isacharparenleft}{\kern0pt}{\isadigit{3}}{\isacharparenright}{\kern0pt}{\isacharbrackright}{\kern0pt}\ \isacommand{by}\isamarkupfalse%
\ auto\isanewline
\ \ \isacommand{then}\isamarkupfalse%
\isanewline
\ \ \isacommand{have}\isamarkupfalse%
\ {\isachardoublequoteopen}Collect{\isacharparenleft}{\kern0pt}A{\isasymtimes}A{\isacharcomma}{\kern0pt}relP{\isacharparenleft}{\kern0pt}M{\isacharcomma}{\kern0pt}is{\isacharunderscore}{\kern0pt}rel{\isacharparenright}{\kern0pt}{\isacharparenright}{\kern0pt}\ {\isacharequal}{\kern0pt}\ Collect{\isacharparenleft}{\kern0pt}A{\isasymtimes}A{\isacharcomma}{\kern0pt}{\isasymlambda}z{\isachardot}{\kern0pt}\ {\isacharparenleft}{\kern0pt}{\isasymexists}x\ y{\isachardot}{\kern0pt}\ z\ {\isacharequal}{\kern0pt}\ {\isasymlangle}x{\isacharcomma}{\kern0pt}y{\isasymrangle}\ {\isasymand}\ rel{\isacharparenleft}{\kern0pt}x{\isacharcomma}{\kern0pt}y{\isacharparenright}{\kern0pt}{\isacharparenright}{\kern0pt}{\isacharparenright}{\kern0pt}{\isachardoublequoteclose}\isanewline
\ \ \ \ \isacommand{using}\isamarkupfalse%
\ Collect{\isacharunderscore}{\kern0pt}cong{\isacharbrackleft}{\kern0pt}of\ {\isachardoublequoteopen}A{\isasymtimes}A{\isachardoublequoteclose}\ {\isachardoublequoteopen}A{\isasymtimes}A{\isachardoublequoteclose}\ {\isachardoublequoteopen}relP{\isacharparenleft}{\kern0pt}M{\isacharcomma}{\kern0pt}is{\isacharunderscore}{\kern0pt}rel{\isacharparenright}{\kern0pt}{\isachardoublequoteclose}{\isacharcomma}{\kern0pt}OF\ {\isacharunderscore}{\kern0pt}\ A{\isacharparenleft}{\kern0pt}{\isadigit{1}}{\isacharparenright}{\kern0pt}{\isacharbrackright}{\kern0pt}\ assms{\isacharparenleft}{\kern0pt}{\isadigit{1}}{\isacharparenright}{\kern0pt}\ assms{\isacharparenleft}{\kern0pt}{\isadigit{2}}{\isacharparenright}{\kern0pt}\isanewline
\ \ \ \ \isacommand{by}\isamarkupfalse%
\ auto\isanewline
\ \ \isacommand{with}\isamarkupfalse%
\ assms\isanewline
\ \ \isacommand{show}\isamarkupfalse%
\ {\isacharquery}{\kern0pt}thesis\ \isacommand{unfolding}\isamarkupfalse%
\ is{\isacharunderscore}{\kern0pt}RRel{\isacharunderscore}{\kern0pt}def\ Rrel{\isacharunderscore}{\kern0pt}def\ \isacommand{using}\isamarkupfalse%
\ cartprod{\isacharunderscore}{\kern0pt}closed\isanewline
\ \ \ \ \isacommand{by}\isamarkupfalse%
\ auto\isanewline
\isacommand{qed}\isamarkupfalse%
%
\endisatagproof
{\isafoldproof}%
%
\isadelimproof
\isanewline
%
\endisadelimproof
\isanewline
\isacommand{definition}\isamarkupfalse%
\isanewline
\ \ is{\isacharunderscore}{\kern0pt}seqlerel\ {\isacharcolon}{\kern0pt}{\isacharcolon}{\kern0pt}\ {\isachardoublequoteopen}{\isacharbrackleft}{\kern0pt}i{\isasymRightarrow}o{\isacharcomma}{\kern0pt}i{\isacharcomma}{\kern0pt}i{\isacharbrackright}{\kern0pt}\ {\isasymRightarrow}\ o{\isachardoublequoteclose}\ \isakeyword{where}\isanewline
\ \ {\isachardoublequoteopen}is{\isacharunderscore}{\kern0pt}seqlerel{\isacharparenleft}{\kern0pt}M{\isacharcomma}{\kern0pt}A{\isacharcomma}{\kern0pt}r{\isacharparenright}{\kern0pt}\ {\isasymequiv}\ is{\isacharunderscore}{\kern0pt}RRel{\isacharparenleft}{\kern0pt}M{\isacharcomma}{\kern0pt}is{\isacharunderscore}{\kern0pt}seqleR{\isacharcomma}{\kern0pt}A{\isacharcomma}{\kern0pt}r{\isacharparenright}{\kern0pt}{\isachardoublequoteclose}\isanewline
\isanewline
\isacommand{lemma}\isamarkupfalse%
\ {\isacharparenleft}{\kern0pt}\isakeyword{in}\ M{\isacharunderscore}{\kern0pt}basic{\isacharparenright}{\kern0pt}\ seqlerel{\isacharunderscore}{\kern0pt}abs\ {\isacharcolon}{\kern0pt}\isanewline
\ \ \isakeyword{assumes}\ {\isachardoublequoteopen}M{\isacharparenleft}{\kern0pt}A{\isacharparenright}{\kern0pt}{\isachardoublequoteclose}\ \ {\isachardoublequoteopen}M{\isacharparenleft}{\kern0pt}r{\isacharparenright}{\kern0pt}{\isachardoublequoteclose}\isanewline
\ \ \isakeyword{shows}\ {\isachardoublequoteopen}is{\isacharunderscore}{\kern0pt}seqlerel{\isacharparenleft}{\kern0pt}M{\isacharcomma}{\kern0pt}A{\isacharcomma}{\kern0pt}r{\isacharparenright}{\kern0pt}\ {\isasymlongleftrightarrow}\ r\ {\isacharequal}{\kern0pt}\ Rrel{\isacharparenleft}{\kern0pt}seqleR{\isacharcomma}{\kern0pt}A{\isacharparenright}{\kern0pt}{\isachardoublequoteclose}\isanewline
%
\isadelimproof
\ \ %
\endisadelimproof
%
\isatagproof
\isacommand{unfolding}\isamarkupfalse%
\ is{\isacharunderscore}{\kern0pt}seqlerel{\isacharunderscore}{\kern0pt}def\isanewline
\ \ \isacommand{using}\isamarkupfalse%
\ is{\isacharunderscore}{\kern0pt}Rrel{\isacharunderscore}{\kern0pt}abs{\isacharbrackleft}{\kern0pt}OF\ {\isacartoucheopen}M{\isacharparenleft}{\kern0pt}A{\isacharparenright}{\kern0pt}{\isacartoucheclose}\ {\isacartoucheopen}M{\isacharparenleft}{\kern0pt}r{\isacharparenright}{\kern0pt}{\isacartoucheclose}{\isacharcomma}{\kern0pt}of\ seqleR\ is{\isacharunderscore}{\kern0pt}seqleR{\isacharbrackright}{\kern0pt}\ seqleR{\isacharunderscore}{\kern0pt}abs\isanewline
\ \ \isacommand{by}\isamarkupfalse%
\ auto%
\endisatagproof
{\isafoldproof}%
%
\isadelimproof
\isanewline
%
\endisadelimproof
\isanewline
\isacommand{definition}\isamarkupfalse%
\ RrelP\ {\isacharcolon}{\kern0pt}{\isacharcolon}{\kern0pt}\ {\isachardoublequoteopen}{\isacharbrackleft}{\kern0pt}i{\isasymRightarrow}i{\isasymRightarrow}o{\isacharcomma}{\kern0pt}i{\isacharbrackright}{\kern0pt}\ {\isasymRightarrow}\ i{\isachardoublequoteclose}\ \isakeyword{where}\isanewline
\ \ {\isachardoublequoteopen}RrelP{\isacharparenleft}{\kern0pt}R{\isacharcomma}{\kern0pt}A{\isacharparenright}{\kern0pt}\ {\isasymequiv}\ {\isacharbraceleft}{\kern0pt}z{\isasymin}A{\isasymtimes}A{\isachardot}{\kern0pt}\ {\isasymexists}x\ y{\isachardot}{\kern0pt}\ z\ {\isacharequal}{\kern0pt}\ {\isasymlangle}x{\isacharcomma}{\kern0pt}\ y{\isasymrangle}\ {\isasymand}\ R{\isacharparenleft}{\kern0pt}x{\isacharcomma}{\kern0pt}y{\isacharparenright}{\kern0pt}{\isacharbraceright}{\kern0pt}{\isachardoublequoteclose}\isanewline
\ \ \isanewline
\isacommand{lemma}\isamarkupfalse%
\ Rrel{\isacharunderscore}{\kern0pt}eq\ {\isacharcolon}{\kern0pt}\ {\isachardoublequoteopen}RrelP{\isacharparenleft}{\kern0pt}R{\isacharcomma}{\kern0pt}A{\isacharparenright}{\kern0pt}\ {\isacharequal}{\kern0pt}\ Rrel{\isacharparenleft}{\kern0pt}R{\isacharcomma}{\kern0pt}A{\isacharparenright}{\kern0pt}{\isachardoublequoteclose}\isanewline
%
\isadelimproof
\ \ %
\endisadelimproof
%
\isatagproof
\isacommand{unfolding}\isamarkupfalse%
\ Rrel{\isacharunderscore}{\kern0pt}def\ RrelP{\isacharunderscore}{\kern0pt}def\ \isacommand{by}\isamarkupfalse%
\ auto%
\endisatagproof
{\isafoldproof}%
%
\isadelimproof
\isanewline
%
\endisadelimproof
\isanewline
\isacommand{context}\isamarkupfalse%
\ M{\isacharunderscore}{\kern0pt}ctm\isanewline
\isakeyword{begin}\isanewline
\isanewline
\isacommand{lemma}\isamarkupfalse%
\ Rrel{\isacharunderscore}{\kern0pt}closed{\isacharcolon}{\kern0pt}\isanewline
\ \ \isakeyword{assumes}\ {\isachardoublequoteopen}A{\isasymin}M{\isachardoublequoteclose}\ \isanewline
\ \ \ \ {\isachardoublequoteopen}{\isasymAnd}\ a{\isachardot}{\kern0pt}\ a\ {\isasymin}\ nat\ {\isasymLongrightarrow}\ rel{\isacharunderscore}{\kern0pt}fm{\isacharparenleft}{\kern0pt}a{\isacharparenright}{\kern0pt}{\isasymin}formula{\isachardoublequoteclose}\isanewline
\ \ \ \ {\isachardoublequoteopen}{\isasymAnd}\ f\ g\ {\isachardot}{\kern0pt}\ {\isacharparenleft}{\kern0pt}{\isacharhash}{\kern0pt}{\isacharhash}{\kern0pt}M{\isacharparenright}{\kern0pt}{\isacharparenleft}{\kern0pt}f{\isacharparenright}{\kern0pt}\ {\isasymLongrightarrow}\ {\isacharparenleft}{\kern0pt}{\isacharhash}{\kern0pt}{\isacharhash}{\kern0pt}M{\isacharparenright}{\kern0pt}{\isacharparenleft}{\kern0pt}g{\isacharparenright}{\kern0pt}\ {\isasymLongrightarrow}\ rel{\isacharparenleft}{\kern0pt}f{\isacharcomma}{\kern0pt}g{\isacharparenright}{\kern0pt}\ {\isasymlongleftrightarrow}\ is{\isacharunderscore}{\kern0pt}rel{\isacharparenleft}{\kern0pt}{\isacharhash}{\kern0pt}{\isacharhash}{\kern0pt}M{\isacharcomma}{\kern0pt}f{\isacharcomma}{\kern0pt}g{\isacharparenright}{\kern0pt}{\isachardoublequoteclose}\isanewline
\ \ \ \ {\isachardoublequoteopen}arity{\isacharparenleft}{\kern0pt}rel{\isacharunderscore}{\kern0pt}fm{\isacharparenleft}{\kern0pt}{\isadigit{0}}{\isacharparenright}{\kern0pt}{\isacharparenright}{\kern0pt}\ {\isacharequal}{\kern0pt}\ {\isadigit{1}}{\isachardoublequoteclose}\ \isanewline
\ \ \ \ {\isachardoublequoteopen}{\isasymAnd}\ a\ {\isachardot}{\kern0pt}\ a\ {\isasymin}\ M\ {\isasymLongrightarrow}\ sats{\isacharparenleft}{\kern0pt}M{\isacharcomma}{\kern0pt}rel{\isacharunderscore}{\kern0pt}fm{\isacharparenleft}{\kern0pt}{\isadigit{0}}{\isacharparenright}{\kern0pt}{\isacharcomma}{\kern0pt}{\isacharbrackleft}{\kern0pt}a{\isacharbrackright}{\kern0pt}{\isacharparenright}{\kern0pt}\ {\isasymlongleftrightarrow}\ relP{\isacharparenleft}{\kern0pt}{\isacharhash}{\kern0pt}{\isacharhash}{\kern0pt}M{\isacharcomma}{\kern0pt}is{\isacharunderscore}{\kern0pt}rel{\isacharcomma}{\kern0pt}a{\isacharparenright}{\kern0pt}{\isachardoublequoteclose}\isanewline
\ \ \isakeyword{shows}\ {\isachardoublequoteopen}{\isacharparenleft}{\kern0pt}{\isacharhash}{\kern0pt}{\isacharhash}{\kern0pt}M{\isacharparenright}{\kern0pt}{\isacharparenleft}{\kern0pt}Rrel{\isacharparenleft}{\kern0pt}rel{\isacharcomma}{\kern0pt}A{\isacharparenright}{\kern0pt}{\isacharparenright}{\kern0pt}{\isachardoublequoteclose}\ \isanewline
%
\isadelimproof
%
\endisadelimproof
%
\isatagproof
\isacommand{proof}\isamarkupfalse%
\ {\isacharminus}{\kern0pt}\isanewline
\ \ \isacommand{have}\isamarkupfalse%
\ {\isachardoublequoteopen}z{\isasymin}\ M\ {\isasymLongrightarrow}\ relP{\isacharparenleft}{\kern0pt}{\isacharhash}{\kern0pt}{\isacharhash}{\kern0pt}M{\isacharcomma}{\kern0pt}\ is{\isacharunderscore}{\kern0pt}rel{\isacharcomma}{\kern0pt}\ z{\isacharparenright}{\kern0pt}\ {\isasymlongleftrightarrow}\ {\isacharparenleft}{\kern0pt}{\isasymexists}x\ y{\isachardot}{\kern0pt}\ z\ {\isacharequal}{\kern0pt}\ {\isasymlangle}x{\isacharcomma}{\kern0pt}\ y{\isasymrangle}\ {\isasymand}\ rel{\isacharparenleft}{\kern0pt}x{\isacharcomma}{\kern0pt}\ y{\isacharparenright}{\kern0pt}{\isacharparenright}{\kern0pt}{\isachardoublequoteclose}\ \isakeyword{for}\ z\isanewline
\ \ \ \ \isacommand{using}\isamarkupfalse%
\ assms{\isacharparenleft}{\kern0pt}{\isadigit{3}}{\isacharparenright}{\kern0pt}\ is{\isacharunderscore}{\kern0pt}related{\isacharunderscore}{\kern0pt}abs{\isacharbrackleft}{\kern0pt}of\ rel\ is{\isacharunderscore}{\kern0pt}rel{\isacharbrackright}{\kern0pt}\isanewline
\ \ \ \ \isacommand{by}\isamarkupfalse%
\ auto\isanewline
\ \ \isacommand{with}\isamarkupfalse%
\ assms\isanewline
\ \ \isacommand{have}\isamarkupfalse%
\ {\isachardoublequoteopen}Collect{\isacharparenleft}{\kern0pt}A{\isasymtimes}A{\isacharcomma}{\kern0pt}{\isasymlambda}z{\isachardot}{\kern0pt}\ {\isacharparenleft}{\kern0pt}{\isasymexists}x\ y{\isachardot}{\kern0pt}\ z\ {\isacharequal}{\kern0pt}\ {\isasymlangle}x{\isacharcomma}{\kern0pt}y{\isasymrangle}\ {\isasymand}\ rel{\isacharparenleft}{\kern0pt}x{\isacharcomma}{\kern0pt}y{\isacharparenright}{\kern0pt}{\isacharparenright}{\kern0pt}{\isacharparenright}{\kern0pt}\ {\isasymin}\ M{\isachardoublequoteclose}\isanewline
\ \ \ \ \isacommand{using}\isamarkupfalse%
\ Collect{\isacharunderscore}{\kern0pt}in{\isacharunderscore}{\kern0pt}M{\isacharunderscore}{\kern0pt}{\isadigit{0}}p{\isacharbrackleft}{\kern0pt}of\ {\isachardoublequoteopen}rel{\isacharunderscore}{\kern0pt}fm{\isacharparenleft}{\kern0pt}{\isadigit{0}}{\isacharparenright}{\kern0pt}{\isachardoublequoteclose}\ {\isachardoublequoteopen}{\isasymlambda}\ A\ z\ {\isachardot}{\kern0pt}\ relP{\isacharparenleft}{\kern0pt}A{\isacharcomma}{\kern0pt}is{\isacharunderscore}{\kern0pt}rel{\isacharcomma}{\kern0pt}z{\isacharparenright}{\kern0pt}{\isachardoublequoteclose}\ {\isachardoublequoteopen}{\isasymlambda}\ z{\isachardot}{\kern0pt}{\isasymexists}x\ y{\isachardot}{\kern0pt}\ z\ {\isacharequal}{\kern0pt}\ {\isasymlangle}x{\isacharcomma}{\kern0pt}\ y{\isasymrangle}\ {\isasymand}\ rel{\isacharparenleft}{\kern0pt}x{\isacharcomma}{\kern0pt}\ y{\isacharparenright}{\kern0pt}{\isachardoublequoteclose}\ {\isacharbrackright}{\kern0pt}\isanewline
\ \ \ \ \ \ \ \ cartprod{\isacharunderscore}{\kern0pt}closed\isanewline
\ \ \ \ \isacommand{by}\isamarkupfalse%
\ simp\isanewline
\ \ \isacommand{then}\isamarkupfalse%
\ \isacommand{show}\isamarkupfalse%
\ {\isacharquery}{\kern0pt}thesis\isanewline
\ \ \isacommand{unfolding}\isamarkupfalse%
\ Rrel{\isacharunderscore}{\kern0pt}def\ \isacommand{by}\isamarkupfalse%
\ simp\isanewline
\isacommand{qed}\isamarkupfalse%
%
\endisatagproof
{\isafoldproof}%
%
\isadelimproof
\isanewline
%
\endisadelimproof
\isanewline
\isacommand{lemma}\isamarkupfalse%
\ seqle{\isacharunderscore}{\kern0pt}in{\isacharunderscore}{\kern0pt}M{\isacharcolon}{\kern0pt}\ {\isachardoublequoteopen}seqle\ {\isasymin}\ M{\isachardoublequoteclose}\isanewline
%
\isadelimproof
\ \ %
\endisadelimproof
%
\isatagproof
\isacommand{using}\isamarkupfalse%
\ Rrel{\isacharunderscore}{\kern0pt}closed\ seqspace{\isacharunderscore}{\kern0pt}closed\ \isanewline
\ \ \ \ transitivity{\isacharbrackleft}{\kern0pt}OF\ {\isacharunderscore}{\kern0pt}\ nat{\isacharunderscore}{\kern0pt}in{\isacharunderscore}{\kern0pt}M{\isacharbrackright}{\kern0pt}\ type{\isacharunderscore}{\kern0pt}seqleR{\isacharunderscore}{\kern0pt}fm{\isacharbrackleft}{\kern0pt}of\ {\isadigit{0}}{\isacharbrackright}{\kern0pt}\ arity{\isacharunderscore}{\kern0pt}seqleR{\isacharunderscore}{\kern0pt}fm{\isacharbrackleft}{\kern0pt}of\ {\isadigit{0}}{\isacharbrackright}{\kern0pt}\isanewline
\ \ \ \ seqleR{\isacharunderscore}{\kern0pt}fm{\isacharunderscore}{\kern0pt}sats{\isacharbrackleft}{\kern0pt}of\ {\isadigit{0}}{\isacharbrackright}{\kern0pt}\ seqleR{\isacharunderscore}{\kern0pt}abs\ seqlerel{\isacharunderscore}{\kern0pt}abs\ \isanewline
\ \ \isacommand{unfolding}\isamarkupfalse%
\ seqle{\isacharunderscore}{\kern0pt}def\ seqlerel{\isacharunderscore}{\kern0pt}def\ seqleR{\isacharunderscore}{\kern0pt}def\isanewline
\ \ \isacommand{by}\isamarkupfalse%
\ auto%
\endisatagproof
{\isafoldproof}%
%
\isadelimproof
%
\endisadelimproof
%
\isadelimdocument
%
\endisadelimdocument
%
\isatagdocument
%
\isamarkupsubsection{Cohen extension is proper%
}
\isamarkuptrue%
%
\endisatagdocument
{\isafolddocument}%
%
\isadelimdocument
%
\endisadelimdocument
\isacommand{interpretation}\isamarkupfalse%
\ ctm{\isacharunderscore}{\kern0pt}separative\ {\isachardoublequoteopen}{\isadigit{2}}{\isacharcircum}{\kern0pt}{\isacharless}{\kern0pt}{\isasymomega}{\isachardoublequoteclose}\ seqle\ {\isadigit{0}}\isanewline
%
\isadelimproof
%
\endisadelimproof
%
\isatagproof
\isacommand{proof}\isamarkupfalse%
\ {\isacharparenleft}{\kern0pt}unfold{\isacharunderscore}{\kern0pt}locales{\isacharparenright}{\kern0pt}\isanewline
\ \ \isacommand{fix}\isamarkupfalse%
\ f\isanewline
\ \ \isacommand{let}\isamarkupfalse%
\ {\isacharquery}{\kern0pt}q{\isacharequal}{\kern0pt}{\isachardoublequoteopen}seq{\isacharunderscore}{\kern0pt}upd{\isacharparenleft}{\kern0pt}f{\isacharcomma}{\kern0pt}{\isadigit{0}}{\isacharparenright}{\kern0pt}{\isachardoublequoteclose}\ \isakeyword{and}\ {\isacharquery}{\kern0pt}r{\isacharequal}{\kern0pt}{\isachardoublequoteopen}seq{\isacharunderscore}{\kern0pt}upd{\isacharparenleft}{\kern0pt}f{\isacharcomma}{\kern0pt}{\isadigit{1}}{\isacharparenright}{\kern0pt}{\isachardoublequoteclose}\isanewline
\ \ \isacommand{assume}\isamarkupfalse%
\ {\isachardoublequoteopen}f\ {\isasymin}\ {\isadigit{2}}{\isacharcircum}{\kern0pt}{\isacharless}{\kern0pt}{\isasymomega}{\isachardoublequoteclose}\isanewline
\ \ \isacommand{then}\isamarkupfalse%
\isanewline
\ \ \isacommand{have}\isamarkupfalse%
\ {\isachardoublequoteopen}{\isacharquery}{\kern0pt}q\ {\isasympreceq}s\ f\ {\isasymand}\ {\isacharquery}{\kern0pt}r\ {\isasympreceq}s\ f\ {\isasymand}\ {\isacharquery}{\kern0pt}q\ {\isasymbottom}s\ {\isacharquery}{\kern0pt}r{\isachardoublequoteclose}\ \isanewline
\ \ \ \ \isacommand{using}\isamarkupfalse%
\ upd{\isacharunderscore}{\kern0pt}leI\ seqspace{\isacharunderscore}{\kern0pt}separative\ \isacommand{by}\isamarkupfalse%
\ auto\isanewline
\ \ \isacommand{moreover}\isamarkupfalse%
\ \isacommand{from}\isamarkupfalse%
\ calculation\isanewline
\ \ \isacommand{have}\isamarkupfalse%
\ {\isachardoublequoteopen}{\isacharquery}{\kern0pt}q\ {\isasymin}\ {\isadigit{2}}{\isacharcircum}{\kern0pt}{\isacharless}{\kern0pt}{\isasymomega}{\isachardoublequoteclose}\ \ {\isachardoublequoteopen}{\isacharquery}{\kern0pt}r\ {\isasymin}\ {\isadigit{2}}{\isacharcircum}{\kern0pt}{\isacharless}{\kern0pt}{\isasymomega}{\isachardoublequoteclose}\isanewline
\ \ \ \ \isacommand{using}\isamarkupfalse%
\ seq{\isacharunderscore}{\kern0pt}upd{\isacharunderscore}{\kern0pt}type{\isacharbrackleft}{\kern0pt}of\ f\ {\isadigit{2}}{\isacharbrackright}{\kern0pt}\ \isacommand{by}\isamarkupfalse%
\ auto\isanewline
\ \ \isacommand{ultimately}\isamarkupfalse%
\isanewline
\ \ \isacommand{show}\isamarkupfalse%
\ {\isachardoublequoteopen}{\isasymexists}q{\isasymin}{\isadigit{2}}{\isacharcircum}{\kern0pt}{\isacharless}{\kern0pt}{\isasymomega}{\isachardot}{\kern0pt}\ {\isasymexists}r{\isasymin}{\isadigit{2}}{\isacharcircum}{\kern0pt}{\isacharless}{\kern0pt}{\isasymomega}{\isachardot}{\kern0pt}\ q\ {\isasympreceq}s\ f\ {\isasymand}\ r\ {\isasympreceq}s\ f\ {\isasymand}\ q\ {\isasymbottom}s\ r{\isachardoublequoteclose}\isanewline
\ \ \ \ \isacommand{by}\isamarkupfalse%
\ {\isacharparenleft}{\kern0pt}rule{\isacharunderscore}{\kern0pt}tac\ bexI{\isacharparenright}{\kern0pt}{\isacharplus}{\kern0pt}\ %
\isamarkupcmt{why the heck auto-tools don't solve this?%
}\isanewline
\isacommand{next}\isamarkupfalse%
\isanewline
\ \ \isacommand{show}\isamarkupfalse%
\ {\isachardoublequoteopen}{\isadigit{2}}{\isacharcircum}{\kern0pt}{\isacharless}{\kern0pt}{\isasymomega}\ {\isasymin}\ M{\isachardoublequoteclose}\ \isacommand{using}\isamarkupfalse%
\ nat{\isacharunderscore}{\kern0pt}into{\isacharunderscore}{\kern0pt}M\ seqspace{\isacharunderscore}{\kern0pt}closed\ \isacommand{by}\isamarkupfalse%
\ simp\isanewline
\isacommand{next}\isamarkupfalse%
\isanewline
\ \ \isacommand{show}\isamarkupfalse%
\ {\isachardoublequoteopen}seqle\ {\isasymin}\ M{\isachardoublequoteclose}\ \isacommand{using}\isamarkupfalse%
\ seqle{\isacharunderscore}{\kern0pt}in{\isacharunderscore}{\kern0pt}M\ \isacommand{{\isachardot}{\kern0pt}}\isamarkupfalse%
\isanewline
\isacommand{qed}\isamarkupfalse%
%
\endisatagproof
{\isafoldproof}%
%
\isadelimproof
\isanewline
%
\endisadelimproof
\isanewline
\isacommand{lemma}\isamarkupfalse%
\ cohen{\isacharunderscore}{\kern0pt}extension{\isacharunderscore}{\kern0pt}is{\isacharunderscore}{\kern0pt}proper{\isacharcolon}{\kern0pt}\ {\isachardoublequoteopen}{\isasymexists}G{\isachardot}{\kern0pt}\ M{\isacharunderscore}{\kern0pt}generic{\isacharparenleft}{\kern0pt}G{\isacharparenright}{\kern0pt}\ {\isasymand}\ M\ {\isasymnoteq}\ GenExt{\isacharparenleft}{\kern0pt}G{\isacharparenright}{\kern0pt}{\isachardoublequoteclose}\isanewline
%
\isadelimproof
\ \ %
\endisadelimproof
%
\isatagproof
\isacommand{using}\isamarkupfalse%
\ proper{\isacharunderscore}{\kern0pt}extension\ generic{\isacharunderscore}{\kern0pt}filter{\isacharunderscore}{\kern0pt}existence\ zero{\isacharunderscore}{\kern0pt}in{\isacharunderscore}{\kern0pt}seqspace\isanewline
\ \ \isacommand{by}\isamarkupfalse%
\ force%
\endisatagproof
{\isafoldproof}%
%
\isadelimproof
\isanewline
%
\endisadelimproof
\isanewline
\isacommand{end}\isamarkupfalse%
\ \isanewline
%
\isadelimtheory
\isanewline
%
\endisadelimtheory
%
\isatagtheory
\isacommand{end}\isamarkupfalse%
%
\endisatagtheory
{\isafoldtheory}%
%
\isadelimtheory
%
\endisadelimtheory
%
\end{isabellebody}%
\endinput
%:%file=~/source/repos/ZF-notAC/code/Forcing/Succession_Poset.thy%:%
%:%11=1%:%
%:%27=2%:%
%:%28=2%:%
%:%29=3%:%
%:%30=4%:%
%:%31=5%:%
%:%32=6%:%
%:%46=8%:%
%:%58=10%:%
%:%59=11%:%
%:%61=13%:%
%:%62=13%:%
%:%63=14%:%
%:%64=15%:%
%:%65=16%:%
%:%66=17%:%
%:%67=17%:%
%:%70=18%:%
%:%74=18%:%
%:%75=18%:%
%:%76=18%:%
%:%81=18%:%
%:%84=19%:%
%:%85=20%:%
%:%86=20%:%
%:%89=21%:%
%:%93=21%:%
%:%94=21%:%
%:%95=21%:%
%:%100=21%:%
%:%103=22%:%
%:%104=23%:%
%:%105=23%:%
%:%106=24%:%
%:%109=25%:%
%:%113=25%:%
%:%114=25%:%
%:%115=25%:%
%:%120=25%:%
%:%123=26%:%
%:%124=27%:%
%:%125=27%:%
%:%126=28%:%
%:%127=29%:%
%:%128=30%:%
%:%129=31%:%
%:%130=32%:%
%:%133=33%:%
%:%137=33%:%
%:%138=33%:%
%:%139=34%:%
%:%140=34%:%
%:%145=34%:%
%:%150=35%:%
%:%155=36%:%
%:%156=36%:%
%:%161=36%:%
%:%164=37%:%
%:%165=38%:%
%:%166=38%:%
%:%167=39%:%
%:%168=40%:%
%:%169=41%:%
%:%170=42%:%
%:%171=43%:%
%:%172=43%:%
%:%173=44%:%
%:%176=45%:%
%:%180=45%:%
%:%181=45%:%
%:%182=45%:%
%:%183=46%:%
%:%184=46%:%
%:%189=46%:%
%:%192=47%:%
%:%193=48%:%
%:%194=48%:%
%:%195=49%:%
%:%196=50%:%
%:%197=51%:%
%:%198=51%:%
%:%205=52%:%
%:%206=52%:%
%:%207=53%:%
%:%208=53%:%
%:%209=54%:%
%:%210=54%:%
%:%211=55%:%
%:%212=55%:%
%:%213=56%:%
%:%214=56%:%
%:%215=57%:%
%:%216=57%:%
%:%217=58%:%
%:%218=58%:%
%:%219=59%:%
%:%220=59%:%
%:%221=60%:%
%:%222=60%:%
%:%223=61%:%
%:%224=61%:%
%:%225=62%:%
%:%226=62%:%
%:%227=63%:%
%:%228=63%:%
%:%229=64%:%
%:%230=64%:%
%:%231=65%:%
%:%232=65%:%
%:%233=66%:%
%:%234=66%:%
%:%235=66%:%
%:%236=67%:%
%:%237=67%:%
%:%238=68%:%
%:%239=69%:%
%:%240=69%:%
%:%241=69%:%
%:%242=70%:%
%:%243=70%:%
%:%244=71%:%
%:%245=71%:%
%:%246=72%:%
%:%247=72%:%
%:%248=73%:%
%:%249=73%:%
%:%250=73%:%
%:%251=74%:%
%:%252=74%:%
%:%253=75%:%
%:%254=75%:%
%:%255=76%:%
%:%256=76%:%
%:%257=77%:%
%:%258=77%:%
%:%259=77%:%
%:%260=78%:%
%:%266=78%:%
%:%269=79%:%
%:%270=80%:%
%:%271=80%:%
%:%272=81%:%
%:%273=82%:%
%:%274=83%:%
%:%275=83%:%
%:%276=84%:%
%:%277=85%:%
%:%284=86%:%
%:%285=86%:%
%:%286=87%:%
%:%287=87%:%
%:%288=88%:%
%:%289=88%:%
%:%290=88%:%
%:%291=88%:%
%:%292=89%:%
%:%293=89%:%
%:%294=90%:%
%:%295=90%:%
%:%296=91%:%
%:%297=91%:%
%:%298=92%:%
%:%299=92%:%
%:%300=93%:%
%:%301=93%:%
%:%302=93%:%
%:%303=93%:%
%:%304=94%:%
%:%305=94%:%
%:%306=95%:%
%:%307=95%:%
%:%308=95%:%
%:%309=95%:%
%:%310=96%:%
%:%311=96%:%
%:%312=97%:%
%:%313=97%:%
%:%314=98%:%
%:%315=98%:%
%:%316=99%:%
%:%317=99%:%
%:%318=100%:%
%:%319=100%:%
%:%320=101%:%
%:%321=101%:%
%:%322=101%:%
%:%323=101%:%
%:%324=102%:%
%:%325=102%:%
%:%326=103%:%
%:%327=103%:%
%:%328=104%:%
%:%329=104%:%
%:%330=105%:%
%:%331=105%:%
%:%332=105%:%
%:%333=106%:%
%:%334=106%:%
%:%335=107%:%
%:%336=107%:%
%:%337=108%:%
%:%338=108%:%
%:%339=109%:%
%:%340=109%:%
%:%341=110%:%
%:%342=110%:%
%:%343=111%:%
%:%344=111%:%
%:%345=112%:%
%:%346=112%:%
%:%347=113%:%
%:%353=113%:%
%:%356=114%:%
%:%357=115%:%
%:%358=115%:%
%:%359=116%:%
%:%360=117%:%
%:%367=118%:%
%:%368=118%:%
%:%369=119%:%
%:%370=119%:%
%:%371=120%:%
%:%372=120%:%
%:%373=121%:%
%:%374=121%:%
%:%375=121%:%
%:%376=122%:%
%:%377=122%:%
%:%378=123%:%
%:%379=123%:%
%:%380=124%:%
%:%381=124%:%
%:%382=124%:%
%:%383=125%:%
%:%384=125%:%
%:%385=126%:%
%:%386=126%:%
%:%387=127%:%
%:%388=127%:%
%:%389=127%:%
%:%390=128%:%
%:%396=128%:%
%:%399=129%:%
%:%400=130%:%
%:%401=130%:%
%:%402=131%:%
%:%403=132%:%
%:%406=133%:%
%:%410=133%:%
%:%411=133%:%
%:%412=133%:%
%:%413=133%:%
%:%418=133%:%
%:%421=134%:%
%:%422=135%:%
%:%423=135%:%
%:%424=136%:%
%:%427=137%:%
%:%431=137%:%
%:%432=137%:%
%:%433=138%:%
%:%434=138%:%
%:%439=138%:%
%:%442=139%:%
%:%443=140%:%
%:%444=140%:%
%:%445=141%:%
%:%446=142%:%
%:%447=143%:%
%:%448=144%:%
%:%449=144%:%
%:%450=145%:%
%:%451=146%:%
%:%452=147%:%
%:%453=148%:%
%:%454=148%:%
%:%455=149%:%
%:%456=150%:%
%:%457=151%:%
%:%458=152%:%
%:%459=152%:%
%:%460=153%:%
%:%463=154%:%
%:%467=154%:%
%:%468=154%:%
%:%469=155%:%
%:%470=155%:%
%:%475=155%:%
%:%478=156%:%
%:%479=157%:%
%:%480=157%:%
%:%481=158%:%
%:%484=159%:%
%:%488=159%:%
%:%489=159%:%
%:%490=160%:%
%:%491=160%:%
%:%496=160%:%
%:%499=161%:%
%:%500=162%:%
%:%501=162%:%
%:%502=163%:%
%:%503=164%:%
%:%510=165%:%
%:%511=165%:%
%:%512=166%:%
%:%513=166%:%
%:%514=167%:%
%:%515=167%:%
%:%516=167%:%
%:%517=168%:%
%:%518=168%:%
%:%519=169%:%
%:%520=169%:%
%:%521=170%:%
%:%522=170%:%
%:%523=171%:%
%:%524=171%:%
%:%525=172%:%
%:%526=172%:%
%:%527=173%:%
%:%528=173%:%
%:%529=173%:%
%:%530=174%:%
%:%531=174%:%
%:%532=175%:%
%:%533=175%:%
%:%534=175%:%
%:%535=176%:%
%:%536=176%:%
%:%537=176%:%
%:%538=177%:%
%:%539=177%:%
%:%540=177%:%
%:%541=178%:%
%:%542=178%:%
%:%543=179%:%
%:%544=179%:%
%:%545=179%:%
%:%546=180%:%
%:%547=180%:%
%:%548=180%:%
%:%549=180%:%
%:%550=181%:%
%:%551=181%:%
%:%552=182%:%
%:%553=182%:%
%:%554=183%:%
%:%555=183%:%
%:%556=184%:%
%:%557=184%:%
%:%558=185%:%
%:%559=185%:%
%:%560=186%:%
%:%566=186%:%
%:%569=187%:%
%:%570=188%:%
%:%571=188%:%
%:%574=189%:%
%:%578=189%:%
%:%579=189%:%
%:%580=189%:%
%:%585=189%:%
%:%588=190%:%
%:%589=191%:%
%:%590=191%:%
%:%593=192%:%
%:%597=192%:%
%:%598=192%:%
%:%599=193%:%
%:%600=193%:%
%:%605=193%:%
%:%608=194%:%
%:%609=195%:%
%:%610=195%:%
%:%613=196%:%
%:%617=196%:%
%:%618=196%:%
%:%619=197%:%
%:%620=197%:%
%:%625=197%:%
%:%628=198%:%
%:%629=199%:%
%:%630=199%:%
%:%631=200%:%
%:%632=201%:%
%:%633=202%:%
%:%634=202%:%
%:%635=203%:%
%:%636=204%:%
%:%637=205%:%
%:%638=205%:%
%:%639=206%:%
%:%640=207%:%
%:%647=208%:%
%:%648=208%:%
%:%649=209%:%
%:%650=209%:%
%:%651=210%:%
%:%652=210%:%
%:%653=211%:%
%:%654=211%:%
%:%655=212%:%
%:%656=212%:%
%:%657=213%:%
%:%658=213%:%
%:%659=213%:%
%:%660=214%:%
%:%661=214%:%
%:%662=214%:%
%:%663=215%:%
%:%664=215%:%
%:%665=215%:%
%:%666=216%:%
%:%667=216%:%
%:%668=217%:%
%:%669=217%:%
%:%670=217%:%
%:%671=218%:%
%:%672=218%:%
%:%673=218%:%
%:%674=219%:%
%:%675=219%:%
%:%676=219%:%
%:%677=219%:%
%:%678=220%:%
%:%679=220%:%
%:%680=221%:%
%:%681=221%:%
%:%682=221%:%
%:%683=222%:%
%:%684=222%:%
%:%685=222%:%
%:%686=223%:%
%:%687=223%:%
%:%688=223%:%
%:%689=224%:%
%:%690=224%:%
%:%691=225%:%
%:%692=225%:%
%:%693=226%:%
%:%694=226%:%
%:%695=227%:%
%:%696=227%:%
%:%697=227%:%
%:%698=228%:%
%:%704=228%:%
%:%707=229%:%
%:%708=230%:%
%:%709=230%:%
%:%710=231%:%
%:%711=232%:%
%:%712=233%:%
%:%713=233%:%
%:%714=234%:%
%:%715=235%:%
%:%716=236%:%
%:%717=236%:%
%:%718=237%:%
%:%721=238%:%
%:%725=238%:%
%:%726=238%:%
%:%727=239%:%
%:%728=239%:%
%:%733=239%:%
%:%736=240%:%
%:%737=241%:%
%:%738=241%:%
%:%739=242%:%
%:%742=243%:%
%:%746=243%:%
%:%747=243%:%
%:%748=244%:%
%:%749=244%:%
%:%750=244%:%
%:%755=244%:%
%:%758=245%:%
%:%759=246%:%
%:%760=246%:%
%:%761=247%:%
%:%762=248%:%
%:%765=249%:%
%:%769=249%:%
%:%770=249%:%
%:%771=250%:%
%:%772=250%:%
%:%773=251%:%
%:%774=251%:%
%:%779=251%:%
%:%782=252%:%
%:%783=253%:%
%:%784=253%:%
%:%785=254%:%
%:%786=255%:%
%:%787=256%:%
%:%788=257%:%
%:%789=257%:%
%:%790=258%:%
%:%791=259%:%
%:%794=260%:%
%:%798=260%:%
%:%799=260%:%
%:%800=261%:%
%:%801=261%:%
%:%802=262%:%
%:%803=262%:%
%:%808=262%:%
%:%811=263%:%
%:%812=264%:%
%:%813=265%:%
%:%814=265%:%
%:%815=266%:%
%:%816=267%:%
%:%819=268%:%
%:%823=268%:%
%:%824=268%:%
%:%825=268%:%
%:%826=268%:%
%:%831=268%:%
%:%834=269%:%
%:%835=270%:%
%:%836=270%:%
%:%837=271%:%
%:%838=272%:%
%:%839=273%:%
%:%840=274%:%
%:%841=274%:%
%:%842=275%:%
%:%843=276%:%
%:%844=277%:%
%:%851=278%:%
%:%852=278%:%
%:%853=279%:%
%:%854=279%:%
%:%855=280%:%
%:%856=280%:%
%:%857=281%:%
%:%858=281%:%
%:%859=281%:%
%:%860=282%:%
%:%861=282%:%
%:%862=283%:%
%:%863=283%:%
%:%864=284%:%
%:%865=284%:%
%:%866=284%:%
%:%867=285%:%
%:%868=285%:%
%:%869=286%:%
%:%870=286%:%
%:%871=287%:%
%:%872=287%:%
%:%873=288%:%
%:%874=288%:%
%:%875=289%:%
%:%876=289%:%
%:%877=290%:%
%:%878=290%:%
%:%879=290%:%
%:%880=290%:%
%:%881=291%:%
%:%882=291%:%
%:%883=292%:%
%:%889=292%:%
%:%892=293%:%
%:%893=294%:%
%:%894=294%:%
%:%895=295%:%
%:%896=296%:%
%:%897=297%:%
%:%898=298%:%
%:%899=298%:%
%:%900=299%:%
%:%901=300%:%
%:%904=301%:%
%:%908=301%:%
%:%909=301%:%
%:%910=302%:%
%:%911=302%:%
%:%912=303%:%
%:%913=303%:%
%:%918=303%:%
%:%921=304%:%
%:%922=305%:%
%:%923=305%:%
%:%924=306%:%
%:%925=307%:%
%:%926=308%:%
%:%927=308%:%
%:%930=309%:%
%:%934=309%:%
%:%935=309%:%
%:%936=309%:%
%:%941=309%:%
%:%944=310%:%
%:%945=311%:%
%:%946=311%:%
%:%947=312%:%
%:%948=313%:%
%:%949=314%:%
%:%950=314%:%
%:%951=315%:%
%:%952=316%:%
%:%953=317%:%
%:%954=318%:%
%:%955=319%:%
%:%956=320%:%
%:%963=321%:%
%:%964=321%:%
%:%965=322%:%
%:%966=322%:%
%:%967=323%:%
%:%968=323%:%
%:%969=324%:%
%:%970=324%:%
%:%971=325%:%
%:%972=325%:%
%:%973=326%:%
%:%974=326%:%
%:%975=327%:%
%:%976=327%:%
%:%977=328%:%
%:%978=329%:%
%:%979=329%:%
%:%980=330%:%
%:%981=330%:%
%:%982=330%:%
%:%983=331%:%
%:%984=331%:%
%:%985=331%:%
%:%986=332%:%
%:%992=332%:%
%:%995=333%:%
%:%996=334%:%
%:%997=334%:%
%:%1000=335%:%
%:%1004=335%:%
%:%1005=335%:%
%:%1006=336%:%
%:%1007=337%:%
%:%1008=338%:%
%:%1009=338%:%
%:%1010=339%:%
%:%1011=339%:%
%:%1025=341%:%
%:%1035=343%:%
%:%1036=343%:%
%:%1043=344%:%
%:%1044=344%:%
%:%1045=345%:%
%:%1046=345%:%
%:%1047=346%:%
%:%1048=346%:%
%:%1049=347%:%
%:%1050=347%:%
%:%1051=348%:%
%:%1052=348%:%
%:%1053=349%:%
%:%1054=349%:%
%:%1055=350%:%
%:%1056=350%:%
%:%1057=350%:%
%:%1058=351%:%
%:%1059=351%:%
%:%1060=351%:%
%:%1061=352%:%
%:%1062=352%:%
%:%1063=353%:%
%:%1064=353%:%
%:%1065=353%:%
%:%1066=354%:%
%:%1067=354%:%
%:%1068=355%:%
%:%1069=355%:%
%:%1070=356%:%
%:%1071=356%:%
%:%1072=356%:%
%:%1073=356%:%
%:%1074=357%:%
%:%1075=357%:%
%:%1076=358%:%
%:%1077=358%:%
%:%1078=358%:%
%:%1079=358%:%
%:%1080=359%:%
%:%1081=359%:%
%:%1082=360%:%
%:%1083=360%:%
%:%1084=360%:%
%:%1085=360%:%
%:%1086=361%:%
%:%1092=361%:%
%:%1095=362%:%
%:%1096=363%:%
%:%1097=363%:%
%:%1100=364%:%
%:%1104=364%:%
%:%1105=364%:%
%:%1106=365%:%
%:%1107=365%:%
%:%1112=365%:%
%:%1115=366%:%
%:%1116=367%:%
%:%1117=367%:%
%:%1120=368%:%
%:%1125=369%:%

%
\begin{isabellebody}%
\setisabellecontext{Forcing{\isacharunderscore}{\kern0pt}Main}%
%
\isadelimdocument
%
\endisadelimdocument
%
\isatagdocument
%
\isamarkupsection{The main theorem%
}
\isamarkuptrue%
%
\endisatagdocument
{\isafolddocument}%
%
\isadelimdocument
%
\endisadelimdocument
%
\isadelimtheory
%
\endisadelimtheory
%
\isatagtheory
\isacommand{theory}\isamarkupfalse%
\ Forcing{\isacharunderscore}{\kern0pt}Main\isanewline
\ \ \isakeyword{imports}\ \isanewline
\ \ Internal{\isacharunderscore}{\kern0pt}ZFC{\isacharunderscore}{\kern0pt}Axioms\isanewline
\ \ Choice{\isacharunderscore}{\kern0pt}Axiom\isanewline
\ \ Ordinals{\isacharunderscore}{\kern0pt}In{\isacharunderscore}{\kern0pt}MG\isanewline
\ \ Succession{\isacharunderscore}{\kern0pt}Poset\isanewline
\isanewline
\isakeyword{begin}%
\endisatagtheory
{\isafoldtheory}%
%
\isadelimtheory
%
\endisadelimtheory
%
\isadelimdocument
%
\endisadelimdocument
%
\isatagdocument
%
\isamarkupsubsection{The generic extension is countable%
}
\isamarkuptrue%
%
\endisatagdocument
{\isafolddocument}%
%
\isadelimdocument
%
\endisadelimdocument
\isacommand{definition}\isamarkupfalse%
\isanewline
\ \ minimum\ {\isacharcolon}{\kern0pt}{\isacharcolon}{\kern0pt}\ {\isachardoublequoteopen}i\ {\isasymRightarrow}\ i\ {\isasymRightarrow}\ i{\isachardoublequoteclose}\ \isakeyword{where}\isanewline
\ \ {\isachardoublequoteopen}minimum{\isacharparenleft}{\kern0pt}r{\isacharcomma}{\kern0pt}B{\isacharparenright}{\kern0pt}\ {\isasymequiv}\ THE\ b{\isachardot}{\kern0pt}\ b{\isasymin}B\ {\isasymand}\ {\isacharparenleft}{\kern0pt}{\isasymforall}y{\isasymin}B{\isachardot}{\kern0pt}\ y\ {\isasymnoteq}\ b\ {\isasymlongrightarrow}\ {\isasymlangle}b{\isacharcomma}{\kern0pt}\ y{\isasymrangle}\ {\isasymin}\ r{\isacharparenright}{\kern0pt}{\isachardoublequoteclose}\isanewline
\isanewline
\isacommand{lemma}\isamarkupfalse%
\ well{\isacharunderscore}{\kern0pt}ord{\isacharunderscore}{\kern0pt}imp{\isacharunderscore}{\kern0pt}min{\isacharcolon}{\kern0pt}\isanewline
\ \ \isakeyword{assumes}\ \isanewline
\ \ \ \ {\isachardoublequoteopen}well{\isacharunderscore}{\kern0pt}ord{\isacharparenleft}{\kern0pt}A{\isacharcomma}{\kern0pt}r{\isacharparenright}{\kern0pt}{\isachardoublequoteclose}\ {\isachardoublequoteopen}B\ {\isasymsubseteq}\ A{\isachardoublequoteclose}\ {\isachardoublequoteopen}B\ {\isasymnoteq}\ {\isadigit{0}}{\isachardoublequoteclose}\isanewline
\ \ \isakeyword{shows}\ \isanewline
\ \ \ \ {\isachardoublequoteopen}minimum{\isacharparenleft}{\kern0pt}r{\isacharcomma}{\kern0pt}B{\isacharparenright}{\kern0pt}\ {\isasymin}\ B{\isachardoublequoteclose}\ \isanewline
%
\isadelimproof
%
\endisadelimproof
%
\isatagproof
\isacommand{proof}\isamarkupfalse%
\ {\isacharminus}{\kern0pt}\isanewline
\ \ \isacommand{from}\isamarkupfalse%
\ {\isacartoucheopen}well{\isacharunderscore}{\kern0pt}ord{\isacharparenleft}{\kern0pt}A{\isacharcomma}{\kern0pt}r{\isacharparenright}{\kern0pt}{\isacartoucheclose}\isanewline
\ \ \isacommand{have}\isamarkupfalse%
\ {\isachardoublequoteopen}wf{\isacharbrackleft}{\kern0pt}A{\isacharbrackright}{\kern0pt}{\isacharparenleft}{\kern0pt}r{\isacharparenright}{\kern0pt}{\isachardoublequoteclose}\isanewline
\ \ \ \ \isacommand{using}\isamarkupfalse%
\ well{\isacharunderscore}{\kern0pt}ord{\isacharunderscore}{\kern0pt}is{\isacharunderscore}{\kern0pt}wf{\isacharbrackleft}{\kern0pt}OF\ {\isacartoucheopen}well{\isacharunderscore}{\kern0pt}ord{\isacharparenleft}{\kern0pt}A{\isacharcomma}{\kern0pt}r{\isacharparenright}{\kern0pt}{\isacartoucheclose}{\isacharbrackright}{\kern0pt}\ \isacommand{by}\isamarkupfalse%
\ simp\isanewline
\ \ \isacommand{with}\isamarkupfalse%
\ {\isacartoucheopen}B{\isasymsubseteq}A{\isacartoucheclose}\isanewline
\ \ \isacommand{have}\isamarkupfalse%
\ {\isachardoublequoteopen}wf{\isacharbrackleft}{\kern0pt}B{\isacharbrackright}{\kern0pt}{\isacharparenleft}{\kern0pt}r{\isacharparenright}{\kern0pt}{\isachardoublequoteclose}\isanewline
\ \ \ \ \isacommand{using}\isamarkupfalse%
\ Sigma{\isacharunderscore}{\kern0pt}mono\ Int{\isacharunderscore}{\kern0pt}mono\ wf{\isacharunderscore}{\kern0pt}subset\ \isacommand{unfolding}\isamarkupfalse%
\ wf{\isacharunderscore}{\kern0pt}on{\isacharunderscore}{\kern0pt}def\ \isacommand{by}\isamarkupfalse%
\ simp\isanewline
\ \ \isacommand{then}\isamarkupfalse%
\isanewline
\ \ \isacommand{have}\isamarkupfalse%
\ {\isachardoublequoteopen}{\isasymforall}\ x{\isachardot}{\kern0pt}\ x\ {\isasymin}\ B\ {\isasymlongrightarrow}\ {\isacharparenleft}{\kern0pt}{\isasymexists}z{\isasymin}B{\isachardot}{\kern0pt}\ {\isasymforall}y{\isachardot}{\kern0pt}\ {\isasymlangle}y{\isacharcomma}{\kern0pt}\ z{\isasymrangle}\ {\isasymin}\ r{\isasyminter}B{\isasymtimes}B\ {\isasymlongrightarrow}\ y\ {\isasymnotin}\ B{\isacharparenright}{\kern0pt}{\isachardoublequoteclose}\isanewline
\ \ \ \ \isacommand{unfolding}\isamarkupfalse%
\ wf{\isacharunderscore}{\kern0pt}on{\isacharunderscore}{\kern0pt}def\ \isacommand{using}\isamarkupfalse%
\ wf{\isacharunderscore}{\kern0pt}eq{\isacharunderscore}{\kern0pt}minimal\isanewline
\ \ \ \ \isacommand{by}\isamarkupfalse%
\ blast\isanewline
\ \ \isacommand{with}\isamarkupfalse%
\ {\isacartoucheopen}B{\isasymnoteq}{\isadigit{0}}{\isacartoucheclose}\isanewline
\ \ \isacommand{obtain}\isamarkupfalse%
\ z\ \isakeyword{where}\isanewline
\ \ \ \ B{\isacharcolon}{\kern0pt}\ {\isachardoublequoteopen}z{\isasymin}B\ {\isasymand}\ {\isacharparenleft}{\kern0pt}{\isasymforall}y{\isachardot}{\kern0pt}\ {\isasymlangle}y{\isacharcomma}{\kern0pt}z{\isasymrangle}{\isasymin}r{\isasyminter}B{\isasymtimes}B\ {\isasymlongrightarrow}\ y{\isasymnotin}B{\isacharparenright}{\kern0pt}{\isachardoublequoteclose}\isanewline
\ \ \ \ \isacommand{by}\isamarkupfalse%
\ blast\isanewline
\ \ \isacommand{then}\isamarkupfalse%
\isanewline
\ \ \isacommand{have}\isamarkupfalse%
\ {\isachardoublequoteopen}z{\isasymin}B\ {\isasymand}\ {\isacharparenleft}{\kern0pt}{\isasymforall}y{\isasymin}B{\isachardot}{\kern0pt}\ y\ {\isasymnoteq}\ z\ {\isasymlongrightarrow}\ {\isasymlangle}z{\isacharcomma}{\kern0pt}\ y{\isasymrangle}\ {\isasymin}\ r{\isacharparenright}{\kern0pt}{\isachardoublequoteclose}\isanewline
\ \ \isacommand{proof}\isamarkupfalse%
\ {\isacharminus}{\kern0pt}\isanewline
\ \ \ \ \isacommand{{\isacharbraceleft}{\kern0pt}}\isamarkupfalse%
\isanewline
\ \ \ \ \ \ \isacommand{fix}\isamarkupfalse%
\ y\isanewline
\ \ \ \ \ \ \isacommand{assume}\isamarkupfalse%
\ {\isachardoublequoteopen}y{\isasymin}B{\isachardoublequoteclose}\ {\isachardoublequoteopen}y{\isasymnoteq}z{\isachardoublequoteclose}\isanewline
\ \ \ \ \ \ \isacommand{with}\isamarkupfalse%
\ {\isacartoucheopen}well{\isacharunderscore}{\kern0pt}ord{\isacharparenleft}{\kern0pt}A{\isacharcomma}{\kern0pt}r{\isacharparenright}{\kern0pt}{\isacartoucheclose}\ B\ {\isacartoucheopen}B{\isasymsubseteq}A{\isacartoucheclose}\isanewline
\ \ \ \ \ \ \isacommand{have}\isamarkupfalse%
\ {\isachardoublequoteopen}{\isasymlangle}z{\isacharcomma}{\kern0pt}y{\isasymrangle}{\isasymin}r{\isacharbar}{\kern0pt}{\isasymlangle}y{\isacharcomma}{\kern0pt}z{\isasymrangle}{\isasymin}r{\isacharbar}{\kern0pt}y{\isacharequal}{\kern0pt}z{\isachardoublequoteclose}\isanewline
\ \ \ \ \ \ \ \ \isacommand{unfolding}\isamarkupfalse%
\ well{\isacharunderscore}{\kern0pt}ord{\isacharunderscore}{\kern0pt}def\ tot{\isacharunderscore}{\kern0pt}ord{\isacharunderscore}{\kern0pt}def\ linear{\isacharunderscore}{\kern0pt}def\ \isacommand{by}\isamarkupfalse%
\ auto\isanewline
\ \ \ \ \ \ \isacommand{with}\isamarkupfalse%
\ B\ {\isacartoucheopen}y{\isasymin}B{\isacartoucheclose}\ {\isacartoucheopen}y{\isasymnoteq}z{\isacartoucheclose}\isanewline
\ \ \ \ \ \ \isacommand{have}\isamarkupfalse%
\ {\isachardoublequoteopen}{\isasymlangle}z{\isacharcomma}{\kern0pt}y{\isasymrangle}{\isasymin}r{\isachardoublequoteclose}\isanewline
\ \ \ \ \ \ \ \ \isacommand{by}\isamarkupfalse%
\ {\isacharparenleft}{\kern0pt}cases{\isacharsemicolon}{\kern0pt}auto{\isacharparenright}{\kern0pt}\isanewline
\ \ \ \ \isacommand{{\isacharbraceright}{\kern0pt}}\isamarkupfalse%
\isanewline
\ \ \ \ \isacommand{with}\isamarkupfalse%
\ B\isanewline
\ \ \ \ \isacommand{show}\isamarkupfalse%
\ {\isacharquery}{\kern0pt}thesis\ \isacommand{by}\isamarkupfalse%
\ blast\isanewline
\ \ \isacommand{qed}\isamarkupfalse%
\isanewline
\ \ \isacommand{have}\isamarkupfalse%
\ {\isachardoublequoteopen}v\ {\isacharequal}{\kern0pt}\ z{\isachardoublequoteclose}\ \isakeyword{if}\ {\isachardoublequoteopen}v{\isasymin}B\ {\isasymand}\ {\isacharparenleft}{\kern0pt}{\isasymforall}y{\isasymin}B{\isachardot}{\kern0pt}\ y\ {\isasymnoteq}\ v\ {\isasymlongrightarrow}\ {\isasymlangle}v{\isacharcomma}{\kern0pt}\ y{\isasymrangle}\ {\isasymin}\ r{\isacharparenright}{\kern0pt}{\isachardoublequoteclose}\ \isakeyword{for}\ v\isanewline
\ \ \ \ \isacommand{using}\isamarkupfalse%
\ that\ B\ \isacommand{by}\isamarkupfalse%
\ auto\isanewline
\ \ \isacommand{with}\isamarkupfalse%
\ {\isacartoucheopen}z{\isasymin}B\ {\isasymand}\ {\isacharparenleft}{\kern0pt}{\isasymforall}y{\isasymin}B{\isachardot}{\kern0pt}\ y\ {\isasymnoteq}\ z\ {\isasymlongrightarrow}\ {\isasymlangle}z{\isacharcomma}{\kern0pt}\ y{\isasymrangle}\ {\isasymin}\ r{\isacharparenright}{\kern0pt}{\isacartoucheclose}\isanewline
\ \ \isacommand{show}\isamarkupfalse%
\ {\isacharquery}{\kern0pt}thesis\isanewline
\ \ \ \ \isacommand{unfolding}\isamarkupfalse%
\ minimum{\isacharunderscore}{\kern0pt}def\ \isanewline
\ \ \ \ \isacommand{using}\isamarkupfalse%
\ the{\isacharunderscore}{\kern0pt}equality{\isadigit{2}}{\isacharbrackleft}{\kern0pt}OF\ ex{\isadigit{1}}I{\isacharbrackleft}{\kern0pt}of\ {\isachardoublequoteopen}{\isasymlambda}x\ {\isachardot}{\kern0pt}x{\isasymin}B\ {\isasymand}\ {\isacharparenleft}{\kern0pt}{\isasymforall}y{\isasymin}B{\isachardot}{\kern0pt}\ y\ {\isasymnoteq}\ x\ {\isasymlongrightarrow}\ {\isasymlangle}x{\isacharcomma}{\kern0pt}\ y{\isasymrangle}\ {\isasymin}\ r{\isacharparenright}{\kern0pt}{\isachardoublequoteclose}\ z{\isacharbrackright}{\kern0pt}{\isacharbrackright}{\kern0pt}\isanewline
\ \ \ \ \isacommand{by}\isamarkupfalse%
\ auto\isanewline
\isacommand{qed}\isamarkupfalse%
%
\endisatagproof
{\isafoldproof}%
%
\isadelimproof
\isanewline
%
\endisadelimproof
\isanewline
\isacommand{lemma}\isamarkupfalse%
\ well{\isacharunderscore}{\kern0pt}ord{\isacharunderscore}{\kern0pt}surj{\isacharunderscore}{\kern0pt}imp{\isacharunderscore}{\kern0pt}lepoll{\isacharcolon}{\kern0pt}\isanewline
\ \ \isakeyword{assumes}\ {\isachardoublequoteopen}well{\isacharunderscore}{\kern0pt}ord{\isacharparenleft}{\kern0pt}A{\isacharcomma}{\kern0pt}r{\isacharparenright}{\kern0pt}{\isachardoublequoteclose}\ {\isachardoublequoteopen}h\ {\isasymin}\ surj{\isacharparenleft}{\kern0pt}A{\isacharcomma}{\kern0pt}B{\isacharparenright}{\kern0pt}{\isachardoublequoteclose}\isanewline
\ \ \isakeyword{shows}\ {\isachardoublequoteopen}B\ {\isasymlesssim}\ A{\isachardoublequoteclose}\isanewline
%
\isadelimproof
%
\endisadelimproof
%
\isatagproof
\isacommand{proof}\isamarkupfalse%
\ {\isacharminus}{\kern0pt}\isanewline
\ \ \isacommand{let}\isamarkupfalse%
\ {\isacharquery}{\kern0pt}f{\isacharequal}{\kern0pt}{\isachardoublequoteopen}{\isasymlambda}b{\isasymin}B{\isachardot}{\kern0pt}\ minimum{\isacharparenleft}{\kern0pt}r{\isacharcomma}{\kern0pt}\ {\isacharbraceleft}{\kern0pt}a{\isasymin}A{\isachardot}{\kern0pt}\ h{\isacharbackquote}{\kern0pt}a{\isacharequal}{\kern0pt}b{\isacharbraceright}{\kern0pt}{\isacharparenright}{\kern0pt}{\isachardoublequoteclose}\isanewline
\ \ \isacommand{have}\isamarkupfalse%
\ {\isachardoublequoteopen}b\ {\isasymin}\ B\ {\isasymLongrightarrow}\ minimum{\isacharparenleft}{\kern0pt}r{\isacharcomma}{\kern0pt}\ {\isacharbraceleft}{\kern0pt}a\ {\isasymin}\ A\ {\isachardot}{\kern0pt}\ h\ {\isacharbackquote}{\kern0pt}\ a\ {\isacharequal}{\kern0pt}\ b{\isacharbraceright}{\kern0pt}{\isacharparenright}{\kern0pt}\ {\isasymin}\ {\isacharbraceleft}{\kern0pt}a{\isasymin}A{\isachardot}{\kern0pt}\ h{\isacharbackquote}{\kern0pt}a{\isacharequal}{\kern0pt}b{\isacharbraceright}{\kern0pt}{\isachardoublequoteclose}\ \isakeyword{for}\ b\isanewline
\ \ \isacommand{proof}\isamarkupfalse%
\ {\isacharminus}{\kern0pt}\isanewline
\ \ \ \ \isacommand{fix}\isamarkupfalse%
\ b\isanewline
\ \ \ \ \isacommand{assume}\isamarkupfalse%
\ {\isachardoublequoteopen}b{\isasymin}B{\isachardoublequoteclose}\isanewline
\ \ \ \ \isacommand{with}\isamarkupfalse%
\ {\isacartoucheopen}h\ {\isasymin}\ surj{\isacharparenleft}{\kern0pt}A{\isacharcomma}{\kern0pt}B{\isacharparenright}{\kern0pt}{\isacartoucheclose}\isanewline
\ \ \ \ \isacommand{have}\isamarkupfalse%
\ {\isachardoublequoteopen}{\isasymexists}a{\isasymin}A{\isachardot}{\kern0pt}\ h{\isacharbackquote}{\kern0pt}a{\isacharequal}{\kern0pt}b{\isachardoublequoteclose}\ \isanewline
\ \ \ \ \ \ \isacommand{unfolding}\isamarkupfalse%
\ surj{\isacharunderscore}{\kern0pt}def\ \isacommand{by}\isamarkupfalse%
\ blast\isanewline
\ \ \ \ \isacommand{then}\isamarkupfalse%
\isanewline
\ \ \ \ \isacommand{have}\isamarkupfalse%
\ {\isachardoublequoteopen}{\isacharbraceleft}{\kern0pt}a{\isasymin}A{\isachardot}{\kern0pt}\ h{\isacharbackquote}{\kern0pt}a{\isacharequal}{\kern0pt}b{\isacharbraceright}{\kern0pt}\ {\isasymnoteq}\ {\isadigit{0}}{\isachardoublequoteclose}\isanewline
\ \ \ \ \ \ \isacommand{by}\isamarkupfalse%
\ auto\isanewline
\ \ \ \ \isacommand{with}\isamarkupfalse%
\ assms\isanewline
\ \ \ \ \isacommand{show}\isamarkupfalse%
\ {\isachardoublequoteopen}minimum{\isacharparenleft}{\kern0pt}r{\isacharcomma}{\kern0pt}{\isacharbraceleft}{\kern0pt}a{\isasymin}A{\isachardot}{\kern0pt}\ h{\isacharbackquote}{\kern0pt}a{\isacharequal}{\kern0pt}b{\isacharbraceright}{\kern0pt}{\isacharparenright}{\kern0pt}\ {\isasymin}\ {\isacharbraceleft}{\kern0pt}a{\isasymin}A{\isachardot}{\kern0pt}\ h{\isacharbackquote}{\kern0pt}a{\isacharequal}{\kern0pt}b{\isacharbraceright}{\kern0pt}{\isachardoublequoteclose}\isanewline
\ \ \ \ \ \ \isacommand{using}\isamarkupfalse%
\ well{\isacharunderscore}{\kern0pt}ord{\isacharunderscore}{\kern0pt}imp{\isacharunderscore}{\kern0pt}min\ \isacommand{by}\isamarkupfalse%
\ blast\isanewline
\ \ \isacommand{qed}\isamarkupfalse%
\isanewline
\ \ \isacommand{moreover}\isamarkupfalse%
\ \isacommand{from}\isamarkupfalse%
\ this\isanewline
\ \ \isacommand{have}\isamarkupfalse%
\ {\isachardoublequoteopen}{\isacharquery}{\kern0pt}f\ {\isacharcolon}{\kern0pt}\ B\ {\isasymrightarrow}\ A{\isachardoublequoteclose}\isanewline
\ \ \ \ \ \ \isacommand{using}\isamarkupfalse%
\ lam{\isacharunderscore}{\kern0pt}type{\isacharbrackleft}{\kern0pt}of\ B\ {\isacharunderscore}{\kern0pt}\ {\isachardoublequoteopen}{\isasymlambda}{\isacharunderscore}{\kern0pt}{\isachardot}{\kern0pt}A{\isachardoublequoteclose}{\isacharbrackright}{\kern0pt}\ \isacommand{by}\isamarkupfalse%
\ simp\isanewline
\ \ \isacommand{moreover}\isamarkupfalse%
\ \isanewline
\ \ \isacommand{have}\isamarkupfalse%
\ {\isachardoublequoteopen}{\isacharquery}{\kern0pt}f\ {\isacharbackquote}{\kern0pt}\ w\ {\isacharequal}{\kern0pt}\ {\isacharquery}{\kern0pt}f\ {\isacharbackquote}{\kern0pt}\ x\ {\isasymLongrightarrow}\ w\ {\isacharequal}{\kern0pt}\ x{\isachardoublequoteclose}\ \isakeyword{if}\ {\isachardoublequoteopen}w{\isasymin}B{\isachardoublequoteclose}\ {\isachardoublequoteopen}x{\isasymin}B{\isachardoublequoteclose}\ \isakeyword{for}\ w\ x\isanewline
\ \ \isacommand{proof}\isamarkupfalse%
\ {\isacharminus}{\kern0pt}\isanewline
\ \ \ \ \isacommand{from}\isamarkupfalse%
\ calculation{\isacharparenleft}{\kern0pt}{\isadigit{1}}{\isacharparenright}{\kern0pt}{\isacharbrackleft}{\kern0pt}OF\ that{\isacharparenleft}{\kern0pt}{\isadigit{1}}{\isacharparenright}{\kern0pt}{\isacharbrackright}{\kern0pt}\ calculation{\isacharparenleft}{\kern0pt}{\isadigit{1}}{\isacharparenright}{\kern0pt}{\isacharbrackleft}{\kern0pt}OF\ that{\isacharparenleft}{\kern0pt}{\isadigit{2}}{\isacharparenright}{\kern0pt}{\isacharbrackright}{\kern0pt}\isanewline
\ \ \ \ \isacommand{have}\isamarkupfalse%
\ {\isachardoublequoteopen}w\ {\isacharequal}{\kern0pt}\ h\ {\isacharbackquote}{\kern0pt}\ minimum{\isacharparenleft}{\kern0pt}r{\isacharcomma}{\kern0pt}\ {\isacharbraceleft}{\kern0pt}a\ {\isasymin}\ A\ {\isachardot}{\kern0pt}\ h\ {\isacharbackquote}{\kern0pt}\ a\ {\isacharequal}{\kern0pt}\ w{\isacharbraceright}{\kern0pt}{\isacharparenright}{\kern0pt}{\isachardoublequoteclose}\isanewline
\ \ \ \ \ \ \ \ \ {\isachardoublequoteopen}x\ {\isacharequal}{\kern0pt}\ h\ {\isacharbackquote}{\kern0pt}\ minimum{\isacharparenleft}{\kern0pt}r{\isacharcomma}{\kern0pt}\ {\isacharbraceleft}{\kern0pt}a\ {\isasymin}\ A\ {\isachardot}{\kern0pt}\ h\ {\isacharbackquote}{\kern0pt}\ a\ {\isacharequal}{\kern0pt}\ x{\isacharbraceright}{\kern0pt}{\isacharparenright}{\kern0pt}{\isachardoublequoteclose}\isanewline
\ \ \ \ \ \ \isacommand{by}\isamarkupfalse%
\ simp{\isacharunderscore}{\kern0pt}all\ \ \isanewline
\ \ \ \ \isacommand{moreover}\isamarkupfalse%
\isanewline
\ \ \ \ \isacommand{assume}\isamarkupfalse%
\ {\isachardoublequoteopen}{\isacharquery}{\kern0pt}f\ {\isacharbackquote}{\kern0pt}\ w\ {\isacharequal}{\kern0pt}\ {\isacharquery}{\kern0pt}f\ {\isacharbackquote}{\kern0pt}\ x{\isachardoublequoteclose}\isanewline
\ \ \ \ \isacommand{moreover}\isamarkupfalse%
\ \isacommand{from}\isamarkupfalse%
\ this\ \isakeyword{and}\ that\isanewline
\ \ \ \ \isacommand{have}\isamarkupfalse%
\ {\isachardoublequoteopen}minimum{\isacharparenleft}{\kern0pt}r{\isacharcomma}{\kern0pt}\ {\isacharbraceleft}{\kern0pt}a\ {\isasymin}\ A\ {\isachardot}{\kern0pt}\ h\ {\isacharbackquote}{\kern0pt}\ a\ {\isacharequal}{\kern0pt}\ w{\isacharbraceright}{\kern0pt}{\isacharparenright}{\kern0pt}\ {\isacharequal}{\kern0pt}\ minimum{\isacharparenleft}{\kern0pt}r{\isacharcomma}{\kern0pt}\ {\isacharbraceleft}{\kern0pt}a\ {\isasymin}\ A\ {\isachardot}{\kern0pt}\ h\ {\isacharbackquote}{\kern0pt}\ a\ {\isacharequal}{\kern0pt}\ x{\isacharbraceright}{\kern0pt}{\isacharparenright}{\kern0pt}{\isachardoublequoteclose}\isanewline
\ \ \ \ \ \ \isacommand{by}\isamarkupfalse%
\ simp{\isacharunderscore}{\kern0pt}all\isanewline
\ \ \ \ \isacommand{moreover}\isamarkupfalse%
\ \isacommand{from}\isamarkupfalse%
\ calculation{\isacharparenleft}{\kern0pt}{\isadigit{1}}{\isacharcomma}{\kern0pt}{\isadigit{2}}{\isacharcomma}{\kern0pt}{\isadigit{4}}{\isacharparenright}{\kern0pt}\isanewline
\ \ \ \ \isacommand{show}\isamarkupfalse%
\ {\isachardoublequoteopen}w{\isacharequal}{\kern0pt}x{\isachardoublequoteclose}\ \isacommand{by}\isamarkupfalse%
\ simp\isanewline
\ \ \ \ \isacommand{qed}\isamarkupfalse%
\isanewline
\ \ \isacommand{ultimately}\isamarkupfalse%
\isanewline
\ \ \isacommand{show}\isamarkupfalse%
\ {\isacharquery}{\kern0pt}thesis\isanewline
\ \ \isacommand{unfolding}\isamarkupfalse%
\ lepoll{\isacharunderscore}{\kern0pt}def\ inj{\isacharunderscore}{\kern0pt}def\ \isacommand{by}\isamarkupfalse%
\ blast\isanewline
\isacommand{qed}\isamarkupfalse%
%
\endisatagproof
{\isafoldproof}%
%
\isadelimproof
\isanewline
%
\endisadelimproof
\isanewline
\isacommand{lemma}\isamarkupfalse%
\ {\isacharparenleft}{\kern0pt}\isakeyword{in}\ forcing{\isacharunderscore}{\kern0pt}data{\isacharparenright}{\kern0pt}\ surj{\isacharunderscore}{\kern0pt}nat{\isacharunderscore}{\kern0pt}MG\ {\isacharcolon}{\kern0pt}\isanewline
\ \ {\isachardoublequoteopen}{\isasymexists}f{\isachardot}{\kern0pt}\ f\ {\isasymin}\ surj{\isacharparenleft}{\kern0pt}nat{\isacharcomma}{\kern0pt}M{\isacharbrackleft}{\kern0pt}G{\isacharbrackright}{\kern0pt}{\isacharparenright}{\kern0pt}{\isachardoublequoteclose}\isanewline
%
\isadelimproof
%
\endisadelimproof
%
\isatagproof
\isacommand{proof}\isamarkupfalse%
\ {\isacharminus}{\kern0pt}\isanewline
\ \ \isacommand{let}\isamarkupfalse%
\ {\isacharquery}{\kern0pt}f{\isacharequal}{\kern0pt}{\isachardoublequoteopen}{\isasymlambda}n{\isasymin}nat{\isachardot}{\kern0pt}\ val{\isacharparenleft}{\kern0pt}G{\isacharcomma}{\kern0pt}enum{\isacharbackquote}{\kern0pt}n{\isacharparenright}{\kern0pt}{\isachardoublequoteclose}\isanewline
\ \ \isacommand{have}\isamarkupfalse%
\ {\isachardoublequoteopen}x\ {\isasymin}\ nat\ {\isasymLongrightarrow}\ val{\isacharparenleft}{\kern0pt}G{\isacharcomma}{\kern0pt}\ enum\ {\isacharbackquote}{\kern0pt}\ x{\isacharparenright}{\kern0pt}{\isasymin}\ M{\isacharbrackleft}{\kern0pt}G{\isacharbrackright}{\kern0pt}{\isachardoublequoteclose}\ \isakeyword{for}\ x\isanewline
\ \ \ \ \isacommand{using}\isamarkupfalse%
\ GenExtD{\isacharbrackleft}{\kern0pt}THEN\ iffD{\isadigit{2}}{\isacharcomma}{\kern0pt}\ of\ {\isacharunderscore}{\kern0pt}\ G{\isacharbrackright}{\kern0pt}\ bij{\isacharunderscore}{\kern0pt}is{\isacharunderscore}{\kern0pt}fun{\isacharbrackleft}{\kern0pt}OF\ M{\isacharunderscore}{\kern0pt}countable{\isacharbrackright}{\kern0pt}\ \isacommand{by}\isamarkupfalse%
\ force\isanewline
\ \ \isacommand{then}\isamarkupfalse%
\isanewline
\ \ \isacommand{have}\isamarkupfalse%
\ {\isachardoublequoteopen}{\isacharquery}{\kern0pt}f{\isacharcolon}{\kern0pt}\ nat\ {\isasymrightarrow}\ M{\isacharbrackleft}{\kern0pt}G{\isacharbrackright}{\kern0pt}{\isachardoublequoteclose}\isanewline
\ \ \ \ \isacommand{using}\isamarkupfalse%
\ lam{\isacharunderscore}{\kern0pt}type{\isacharbrackleft}{\kern0pt}of\ nat\ {\isachardoublequoteopen}{\isasymlambda}n{\isachardot}{\kern0pt}\ val{\isacharparenleft}{\kern0pt}G{\isacharcomma}{\kern0pt}enum{\isacharbackquote}{\kern0pt}n{\isacharparenright}{\kern0pt}{\isachardoublequoteclose}\ {\isachardoublequoteopen}{\isasymlambda}{\isacharunderscore}{\kern0pt}{\isachardot}{\kern0pt}M{\isacharbrackleft}{\kern0pt}G{\isacharbrackright}{\kern0pt}{\isachardoublequoteclose}{\isacharbrackright}{\kern0pt}\ \isacommand{by}\isamarkupfalse%
\ simp\isanewline
\ \ \isacommand{moreover}\isamarkupfalse%
\isanewline
\ \ \isacommand{have}\isamarkupfalse%
\ {\isachardoublequoteopen}{\isasymexists}n{\isasymin}nat{\isachardot}{\kern0pt}\ {\isacharquery}{\kern0pt}f{\isacharbackquote}{\kern0pt}n\ {\isacharequal}{\kern0pt}\ x{\isachardoublequoteclose}\ \isakeyword{if}\ {\isachardoublequoteopen}x{\isasymin}M{\isacharbrackleft}{\kern0pt}G{\isacharbrackright}{\kern0pt}{\isachardoublequoteclose}\ \isakeyword{for}\ x\isanewline
\ \ \ \ \isacommand{using}\isamarkupfalse%
\ that\ GenExtD{\isacharbrackleft}{\kern0pt}of\ {\isacharunderscore}{\kern0pt}\ G{\isacharbrackright}{\kern0pt}\ bij{\isacharunderscore}{\kern0pt}is{\isacharunderscore}{\kern0pt}surj{\isacharbrackleft}{\kern0pt}OF\ M{\isacharunderscore}{\kern0pt}countable{\isacharbrackright}{\kern0pt}\ \isanewline
\ \ \ \ \isacommand{unfolding}\isamarkupfalse%
\ surj{\isacharunderscore}{\kern0pt}def\ \isacommand{by}\isamarkupfalse%
\ auto\isanewline
\ \ \isacommand{ultimately}\isamarkupfalse%
\isanewline
\ \ \isacommand{show}\isamarkupfalse%
\ {\isacharquery}{\kern0pt}thesis\isanewline
\ \ \ \ \isacommand{unfolding}\isamarkupfalse%
\ surj{\isacharunderscore}{\kern0pt}def\ \isacommand{by}\isamarkupfalse%
\ blast\isanewline
\isacommand{qed}\isamarkupfalse%
%
\endisatagproof
{\isafoldproof}%
%
\isadelimproof
\isanewline
%
\endisadelimproof
\isanewline
\isacommand{lemma}\isamarkupfalse%
\ {\isacharparenleft}{\kern0pt}\isakeyword{in}\ G{\isacharunderscore}{\kern0pt}generic{\isacharparenright}{\kern0pt}\ MG{\isacharunderscore}{\kern0pt}eqpoll{\isacharunderscore}{\kern0pt}nat{\isacharcolon}{\kern0pt}\ {\isachardoublequoteopen}M{\isacharbrackleft}{\kern0pt}G{\isacharbrackright}{\kern0pt}\ {\isasymapprox}\ nat{\isachardoublequoteclose}\isanewline
%
\isadelimproof
%
\endisadelimproof
%
\isatagproof
\isacommand{proof}\isamarkupfalse%
\ {\isacharminus}{\kern0pt}\isanewline
\ \ \isacommand{interpret}\isamarkupfalse%
\ MG{\isacharcolon}{\kern0pt}\ M{\isacharunderscore}{\kern0pt}ZF{\isacharunderscore}{\kern0pt}trans\ {\isachardoublequoteopen}M{\isacharbrackleft}{\kern0pt}G{\isacharbrackright}{\kern0pt}{\isachardoublequoteclose}\isanewline
\ \ \ \ \isacommand{using}\isamarkupfalse%
\ Transset{\isacharunderscore}{\kern0pt}MG\ generic\ pairing{\isacharunderscore}{\kern0pt}in{\isacharunderscore}{\kern0pt}MG\ \isanewline
\ \ \ \ \ \ Union{\isacharunderscore}{\kern0pt}MG\ \ extensionality{\isacharunderscore}{\kern0pt}in{\isacharunderscore}{\kern0pt}MG\ power{\isacharunderscore}{\kern0pt}in{\isacharunderscore}{\kern0pt}MG\isanewline
\ \ \ \ \ \ foundation{\isacharunderscore}{\kern0pt}in{\isacharunderscore}{\kern0pt}MG\ \ strong{\isacharunderscore}{\kern0pt}replacement{\isacharunderscore}{\kern0pt}in{\isacharunderscore}{\kern0pt}MG{\isacharbrackleft}{\kern0pt}simplified{\isacharbrackright}{\kern0pt}\isanewline
\ \ \ \ \ \ separation{\isacharunderscore}{\kern0pt}in{\isacharunderscore}{\kern0pt}MG{\isacharbrackleft}{\kern0pt}simplified{\isacharbrackright}{\kern0pt}\ infinity{\isacharunderscore}{\kern0pt}in{\isacharunderscore}{\kern0pt}MG\isanewline
\ \ \ \ \isacommand{by}\isamarkupfalse%
\ unfold{\isacharunderscore}{\kern0pt}locales\ simp{\isacharunderscore}{\kern0pt}all\isanewline
\ \ \isacommand{obtain}\isamarkupfalse%
\ f\ \isakeyword{where}\ {\isachardoublequoteopen}f\ {\isasymin}\ surj{\isacharparenleft}{\kern0pt}nat{\isacharcomma}{\kern0pt}M{\isacharbrackleft}{\kern0pt}G{\isacharbrackright}{\kern0pt}{\isacharparenright}{\kern0pt}{\isachardoublequoteclose}\isanewline
\ \ \ \ \isacommand{using}\isamarkupfalse%
\ surj{\isacharunderscore}{\kern0pt}nat{\isacharunderscore}{\kern0pt}MG\ \isacommand{by}\isamarkupfalse%
\ blast\isanewline
\ \ \isacommand{then}\isamarkupfalse%
\isanewline
\ \ \isacommand{have}\isamarkupfalse%
\ {\isachardoublequoteopen}M{\isacharbrackleft}{\kern0pt}G{\isacharbrackright}{\kern0pt}\ {\isasymlesssim}\ nat{\isachardoublequoteclose}\ \isanewline
\ \ \ \ \isacommand{using}\isamarkupfalse%
\ well{\isacharunderscore}{\kern0pt}ord{\isacharunderscore}{\kern0pt}surj{\isacharunderscore}{\kern0pt}imp{\isacharunderscore}{\kern0pt}lepoll\ well{\isacharunderscore}{\kern0pt}ord{\isacharunderscore}{\kern0pt}Memrel{\isacharbrackleft}{\kern0pt}of\ nat{\isacharbrackright}{\kern0pt}\isanewline
\ \ \ \ \isacommand{by}\isamarkupfalse%
\ simp\isanewline
\ \ \isacommand{moreover}\isamarkupfalse%
\isanewline
\ \ \isacommand{have}\isamarkupfalse%
\ {\isachardoublequoteopen}nat\ {\isasymlesssim}\ M{\isacharbrackleft}{\kern0pt}G{\isacharbrackright}{\kern0pt}{\isachardoublequoteclose}\isanewline
\ \ \ \ \isacommand{using}\isamarkupfalse%
\ MG{\isachardot}{\kern0pt}nat{\isacharunderscore}{\kern0pt}into{\isacharunderscore}{\kern0pt}M\ subset{\isacharunderscore}{\kern0pt}imp{\isacharunderscore}{\kern0pt}lepoll\ \isacommand{by}\isamarkupfalse%
\ auto\isanewline
\ \ \isacommand{ultimately}\isamarkupfalse%
\isanewline
\ \ \isacommand{show}\isamarkupfalse%
\ {\isacharquery}{\kern0pt}thesis\ \isacommand{using}\isamarkupfalse%
\ eqpollI\ \isanewline
\ \ \ \ \isacommand{by}\isamarkupfalse%
\ simp\isanewline
\isacommand{qed}\isamarkupfalse%
%
\endisatagproof
{\isafoldproof}%
%
\isadelimproof
%
\endisadelimproof
%
\isadelimdocument
%
\endisadelimdocument
%
\isatagdocument
%
\isamarkupsubsection{The main result%
}
\isamarkuptrue%
%
\endisatagdocument
{\isafolddocument}%
%
\isadelimdocument
%
\endisadelimdocument
\isacommand{theorem}\isamarkupfalse%
\ extensions{\isacharunderscore}{\kern0pt}of{\isacharunderscore}{\kern0pt}ctms{\isacharcolon}{\kern0pt}\isanewline
\ \ \isakeyword{assumes}\ \isanewline
\ \ \ \ {\isachardoublequoteopen}M\ {\isasymapprox}\ nat{\isachardoublequoteclose}\ {\isachardoublequoteopen}Transset{\isacharparenleft}{\kern0pt}M{\isacharparenright}{\kern0pt}{\isachardoublequoteclose}\ {\isachardoublequoteopen}M\ {\isasymTurnstile}\ ZF{\isachardoublequoteclose}\isanewline
\ \ \isakeyword{shows}\ \isanewline
\ \ \ \ {\isachardoublequoteopen}{\isasymexists}N{\isachardot}{\kern0pt}\ \isanewline
\ \ \ \ \ \ M\ {\isasymsubseteq}\ N\ {\isasymand}\ N\ {\isasymapprox}\ nat\ {\isasymand}\ Transset{\isacharparenleft}{\kern0pt}N{\isacharparenright}{\kern0pt}\ {\isasymand}\ N\ {\isasymTurnstile}\ ZF\ {\isasymand}\ M{\isasymnoteq}N\ {\isasymand}\isanewline
\ \ \ \ \ \ {\isacharparenleft}{\kern0pt}{\isasymforall}{\isasymalpha}{\isachardot}{\kern0pt}\ Ord{\isacharparenleft}{\kern0pt}{\isasymalpha}{\isacharparenright}{\kern0pt}\ {\isasymlongrightarrow}\ {\isacharparenleft}{\kern0pt}{\isasymalpha}\ {\isasymin}\ M\ {\isasymlongleftrightarrow}\ {\isasymalpha}\ {\isasymin}\ N{\isacharparenright}{\kern0pt}{\isacharparenright}{\kern0pt}\ {\isasymand}\isanewline
\ \ \ \ \ \ {\isacharparenleft}{\kern0pt}M{\isacharcomma}{\kern0pt}\ {\isacharbrackleft}{\kern0pt}{\isacharbrackright}{\kern0pt}{\isasymTurnstile}\ AC\ {\isasymlongrightarrow}\ N\ {\isasymTurnstile}\ ZFC{\isacharparenright}{\kern0pt}{\isachardoublequoteclose}\isanewline
%
\isadelimproof
%
\endisadelimproof
%
\isatagproof
\isacommand{proof}\isamarkupfalse%
\ {\isacharminus}{\kern0pt}\isanewline
\ \ \isacommand{from}\isamarkupfalse%
\ {\isacartoucheopen}M\ {\isasymapprox}\ nat{\isacartoucheclose}\isanewline
\ \ \isacommand{obtain}\isamarkupfalse%
\ enum\ \isakeyword{where}\ {\isachardoublequoteopen}enum\ {\isasymin}\ bij{\isacharparenleft}{\kern0pt}nat{\isacharcomma}{\kern0pt}M{\isacharparenright}{\kern0pt}{\isachardoublequoteclose}\isanewline
\ \ \ \ \isacommand{using}\isamarkupfalse%
\ eqpoll{\isacharunderscore}{\kern0pt}sym\ \isacommand{unfolding}\isamarkupfalse%
\ eqpoll{\isacharunderscore}{\kern0pt}def\ \isacommand{by}\isamarkupfalse%
\ blast\isanewline
\ \ \isacommand{with}\isamarkupfalse%
\ assms\isanewline
\ \ \isacommand{interpret}\isamarkupfalse%
\ M{\isacharunderscore}{\kern0pt}ctm\ M\ enum\isanewline
\ \ \ \ \isacommand{using}\isamarkupfalse%
\ M{\isacharunderscore}{\kern0pt}ZF{\isacharunderscore}{\kern0pt}iff{\isacharunderscore}{\kern0pt}M{\isacharunderscore}{\kern0pt}satT\isanewline
\ \ \ \ \isacommand{by}\isamarkupfalse%
\ intro{\isacharunderscore}{\kern0pt}locales\ {\isacharparenleft}{\kern0pt}simp{\isacharunderscore}{\kern0pt}all\ add{\isacharcolon}{\kern0pt}M{\isacharunderscore}{\kern0pt}ctm{\isacharunderscore}{\kern0pt}axioms{\isacharunderscore}{\kern0pt}def{\isacharparenright}{\kern0pt}\isanewline
\ \ \isacommand{interpret}\isamarkupfalse%
\ ctm{\isacharunderscore}{\kern0pt}separative\ {\isachardoublequoteopen}{\isadigit{2}}{\isacharcircum}{\kern0pt}{\isacharless}{\kern0pt}{\isasymomega}{\isachardoublequoteclose}\ seqle\ {\isadigit{0}}\ M\ enum\isanewline
\ \ \isacommand{proof}\isamarkupfalse%
\ {\isacharparenleft}{\kern0pt}unfold{\isacharunderscore}{\kern0pt}locales{\isacharparenright}{\kern0pt}\isanewline
\ \ \ \ \isacommand{fix}\isamarkupfalse%
\ f\isanewline
\ \ \ \ \isacommand{let}\isamarkupfalse%
\ {\isacharquery}{\kern0pt}q{\isacharequal}{\kern0pt}{\isachardoublequoteopen}seq{\isacharunderscore}{\kern0pt}upd{\isacharparenleft}{\kern0pt}f{\isacharcomma}{\kern0pt}{\isadigit{0}}{\isacharparenright}{\kern0pt}{\isachardoublequoteclose}\ \isakeyword{and}\ {\isacharquery}{\kern0pt}r{\isacharequal}{\kern0pt}{\isachardoublequoteopen}seq{\isacharunderscore}{\kern0pt}upd{\isacharparenleft}{\kern0pt}f{\isacharcomma}{\kern0pt}{\isadigit{1}}{\isacharparenright}{\kern0pt}{\isachardoublequoteclose}\isanewline
\ \ \ \ \isacommand{assume}\isamarkupfalse%
\ {\isachardoublequoteopen}f\ {\isasymin}\ {\isadigit{2}}{\isacharcircum}{\kern0pt}{\isacharless}{\kern0pt}{\isasymomega}{\isachardoublequoteclose}\isanewline
\ \ \ \ \isacommand{then}\isamarkupfalse%
\isanewline
\ \ \ \ \isacommand{have}\isamarkupfalse%
\ {\isachardoublequoteopen}{\isacharquery}{\kern0pt}q\ {\isasympreceq}s\ f\ {\isasymand}\ {\isacharquery}{\kern0pt}r\ {\isasympreceq}s\ f\ {\isasymand}\ {\isacharquery}{\kern0pt}q\ {\isasymbottom}s\ {\isacharquery}{\kern0pt}r{\isachardoublequoteclose}\ \isanewline
\ \ \ \ \ \ \isacommand{using}\isamarkupfalse%
\ upd{\isacharunderscore}{\kern0pt}leI\ seqspace{\isacharunderscore}{\kern0pt}separative\ \isacommand{by}\isamarkupfalse%
\ auto\isanewline
\ \ \ \ \isacommand{moreover}\isamarkupfalse%
\ \isacommand{from}\isamarkupfalse%
\ calculation\isanewline
\ \ \ \ \isacommand{have}\isamarkupfalse%
\ {\isachardoublequoteopen}{\isacharquery}{\kern0pt}q\ {\isasymin}\ {\isadigit{2}}{\isacharcircum}{\kern0pt}{\isacharless}{\kern0pt}{\isasymomega}{\isachardoublequoteclose}\ \ {\isachardoublequoteopen}{\isacharquery}{\kern0pt}r\ {\isasymin}\ {\isadigit{2}}{\isacharcircum}{\kern0pt}{\isacharless}{\kern0pt}{\isasymomega}{\isachardoublequoteclose}\isanewline
\ \ \ \ \ \ \isacommand{using}\isamarkupfalse%
\ seq{\isacharunderscore}{\kern0pt}upd{\isacharunderscore}{\kern0pt}type{\isacharbrackleft}{\kern0pt}of\ f\ {\isadigit{2}}{\isacharbrackright}{\kern0pt}\ \isacommand{by}\isamarkupfalse%
\ auto\isanewline
\ \ \ \ \isacommand{ultimately}\isamarkupfalse%
\isanewline
\ \ \ \ \isacommand{show}\isamarkupfalse%
\ {\isachardoublequoteopen}{\isasymexists}q{\isasymin}{\isadigit{2}}{\isacharcircum}{\kern0pt}{\isacharless}{\kern0pt}{\isasymomega}{\isachardot}{\kern0pt}\ \ {\isasymexists}r{\isasymin}{\isadigit{2}}{\isacharcircum}{\kern0pt}{\isacharless}{\kern0pt}{\isasymomega}{\isachardot}{\kern0pt}\ q\ {\isasympreceq}s\ f\ {\isasymand}\ r\ {\isasympreceq}s\ f\ {\isasymand}\ q\ {\isasymbottom}s\ r{\isachardoublequoteclose}\isanewline
\ \ \ \ \ \ \isacommand{by}\isamarkupfalse%
\ {\isacharparenleft}{\kern0pt}rule{\isacharunderscore}{\kern0pt}tac\ bexI{\isacharparenright}{\kern0pt}{\isacharplus}{\kern0pt}\ %
\isamarkupcmt{why the heck auto-tools don't solve this?%
}\isanewline
\ \ \isacommand{next}\isamarkupfalse%
\isanewline
\ \ \ \ \isacommand{show}\isamarkupfalse%
\ {\isachardoublequoteopen}{\isadigit{2}}{\isacharcircum}{\kern0pt}{\isacharless}{\kern0pt}{\isasymomega}\ {\isasymin}\ M{\isachardoublequoteclose}\ \isacommand{using}\isamarkupfalse%
\ nat{\isacharunderscore}{\kern0pt}into{\isacharunderscore}{\kern0pt}M\ seqspace{\isacharunderscore}{\kern0pt}closed\ \isacommand{by}\isamarkupfalse%
\ simp\isanewline
\ \ \isacommand{next}\isamarkupfalse%
\isanewline
\ \ \ \ \isacommand{show}\isamarkupfalse%
\ {\isachardoublequoteopen}seqle\ {\isasymin}\ M{\isachardoublequoteclose}\ \isacommand{using}\isamarkupfalse%
\ seqle{\isacharunderscore}{\kern0pt}in{\isacharunderscore}{\kern0pt}M\ \isacommand{{\isachardot}{\kern0pt}}\isamarkupfalse%
\isanewline
\ \ \isacommand{qed}\isamarkupfalse%
\isanewline
\ \ \isacommand{from}\isamarkupfalse%
\ cohen{\isacharunderscore}{\kern0pt}extension{\isacharunderscore}{\kern0pt}is{\isacharunderscore}{\kern0pt}proper\isanewline
\ \ \isacommand{obtain}\isamarkupfalse%
\ G\ \isakeyword{where}\ {\isachardoublequoteopen}M{\isacharunderscore}{\kern0pt}generic{\isacharparenleft}{\kern0pt}G{\isacharparenright}{\kern0pt}{\isachardoublequoteclose}\ \isanewline
\ \ \ \ {\isachardoublequoteopen}M\ {\isasymnoteq}\ GenExt{\isacharparenleft}{\kern0pt}G{\isacharparenright}{\kern0pt}{\isachardoublequoteclose}\ {\isacharparenleft}{\kern0pt}\isakeyword{is}\ {\isachardoublequoteopen}M{\isasymnoteq}{\isacharquery}{\kern0pt}N{\isachardoublequoteclose}{\isacharparenright}{\kern0pt}\ \isanewline
\ \ \ \ \isacommand{by}\isamarkupfalse%
\ blast\isanewline
\ \ \isacommand{then}\isamarkupfalse%
\ \isanewline
\ \ \isacommand{interpret}\isamarkupfalse%
\ G{\isacharunderscore}{\kern0pt}generic\ {\isachardoublequoteopen}{\isadigit{2}}{\isacharcircum}{\kern0pt}{\isacharless}{\kern0pt}{\isasymomega}{\isachardoublequoteclose}\ seqle\ {\isadigit{0}}\ {\isacharunderscore}{\kern0pt}\ enum\ G\ \isacommand{by}\isamarkupfalse%
\ unfold{\isacharunderscore}{\kern0pt}locales\isanewline
\ \ \isacommand{interpret}\isamarkupfalse%
\ MG{\isacharcolon}{\kern0pt}\ M{\isacharunderscore}{\kern0pt}ZF\ {\isachardoublequoteopen}{\isacharquery}{\kern0pt}N{\isachardoublequoteclose}\isanewline
\ \ \ \ \isacommand{using}\isamarkupfalse%
\ generic\ pairing{\isacharunderscore}{\kern0pt}in{\isacharunderscore}{\kern0pt}MG\ \isanewline
\ \ \ \ \ \ Union{\isacharunderscore}{\kern0pt}MG\ \ extensionality{\isacharunderscore}{\kern0pt}in{\isacharunderscore}{\kern0pt}MG\ power{\isacharunderscore}{\kern0pt}in{\isacharunderscore}{\kern0pt}MG\isanewline
\ \ \ \ \ \ foundation{\isacharunderscore}{\kern0pt}in{\isacharunderscore}{\kern0pt}MG\ \ strong{\isacharunderscore}{\kern0pt}replacement{\isacharunderscore}{\kern0pt}in{\isacharunderscore}{\kern0pt}MG{\isacharbrackleft}{\kern0pt}simplified{\isacharbrackright}{\kern0pt}\isanewline
\ \ \ \ \ \ separation{\isacharunderscore}{\kern0pt}in{\isacharunderscore}{\kern0pt}MG{\isacharbrackleft}{\kern0pt}simplified{\isacharbrackright}{\kern0pt}\ infinity{\isacharunderscore}{\kern0pt}in{\isacharunderscore}{\kern0pt}MG\isanewline
\ \ \ \ \isacommand{by}\isamarkupfalse%
\ unfold{\isacharunderscore}{\kern0pt}locales\ simp{\isacharunderscore}{\kern0pt}all\isanewline
\ \ \isacommand{have}\isamarkupfalse%
\ {\isachardoublequoteopen}{\isacharquery}{\kern0pt}N\ {\isasymTurnstile}\ ZF{\isachardoublequoteclose}\ \isanewline
\ \ \ \ \isacommand{using}\isamarkupfalse%
\ M{\isacharunderscore}{\kern0pt}ZF{\isacharunderscore}{\kern0pt}iff{\isacharunderscore}{\kern0pt}M{\isacharunderscore}{\kern0pt}satT{\isacharbrackleft}{\kern0pt}of\ {\isacharquery}{\kern0pt}N{\isacharbrackright}{\kern0pt}\ MG{\isachardot}{\kern0pt}M{\isacharunderscore}{\kern0pt}ZF{\isacharunderscore}{\kern0pt}axioms\ \isacommand{by}\isamarkupfalse%
\ simp\isanewline
\ \ \isacommand{moreover}\isamarkupfalse%
\ \isanewline
\ \ \isacommand{have}\isamarkupfalse%
\ {\isachardoublequoteopen}M{\isacharcomma}{\kern0pt}\ {\isacharbrackleft}{\kern0pt}{\isacharbrackright}{\kern0pt}{\isasymTurnstile}\ AC\ {\isasymLongrightarrow}\ {\isacharquery}{\kern0pt}N\ {\isasymTurnstile}\ ZFC{\isachardoublequoteclose}\isanewline
\ \ \isacommand{proof}\isamarkupfalse%
\ {\isacharminus}{\kern0pt}\isanewline
\ \ \ \ \isacommand{assume}\isamarkupfalse%
\ {\isachardoublequoteopen}M{\isacharcomma}{\kern0pt}\ {\isacharbrackleft}{\kern0pt}{\isacharbrackright}{\kern0pt}\ {\isasymTurnstile}\ AC{\isachardoublequoteclose}\isanewline
\ \ \ \ \isacommand{then}\isamarkupfalse%
\isanewline
\ \ \ \ \isacommand{have}\isamarkupfalse%
\ {\isachardoublequoteopen}choice{\isacharunderscore}{\kern0pt}ax{\isacharparenleft}{\kern0pt}{\isacharhash}{\kern0pt}{\isacharhash}{\kern0pt}M{\isacharparenright}{\kern0pt}{\isachardoublequoteclose}\isanewline
\ \ \ \ \ \ \isacommand{unfolding}\isamarkupfalse%
\ ZF{\isacharunderscore}{\kern0pt}choice{\isacharunderscore}{\kern0pt}fm{\isacharunderscore}{\kern0pt}def\ \isacommand{using}\isamarkupfalse%
\ ZF{\isacharunderscore}{\kern0pt}choice{\isacharunderscore}{\kern0pt}auto\ \isacommand{by}\isamarkupfalse%
\ simp\isanewline
\ \ \ \ \isacommand{then}\isamarkupfalse%
\isanewline
\ \ \ \ \isacommand{have}\isamarkupfalse%
\ {\isachardoublequoteopen}choice{\isacharunderscore}{\kern0pt}ax{\isacharparenleft}{\kern0pt}{\isacharhash}{\kern0pt}{\isacharhash}{\kern0pt}{\isacharquery}{\kern0pt}N{\isacharparenright}{\kern0pt}{\isachardoublequoteclose}\ \isacommand{using}\isamarkupfalse%
\ choice{\isacharunderscore}{\kern0pt}in{\isacharunderscore}{\kern0pt}MG\ \isacommand{by}\isamarkupfalse%
\ simp\isanewline
\ \ \ \ \isacommand{with}\isamarkupfalse%
\ {\isacartoucheopen}{\isacharquery}{\kern0pt}N\ {\isasymTurnstile}\ ZF{\isacartoucheclose}\isanewline
\ \ \ \ \isacommand{show}\isamarkupfalse%
\ {\isachardoublequoteopen}{\isacharquery}{\kern0pt}N\ {\isasymTurnstile}\ ZFC{\isachardoublequoteclose}\isanewline
\ \ \ \ \ \ \isacommand{using}\isamarkupfalse%
\ ZF{\isacharunderscore}{\kern0pt}choice{\isacharunderscore}{\kern0pt}auto\ sats{\isacharunderscore}{\kern0pt}ZFC{\isacharunderscore}{\kern0pt}iff{\isacharunderscore}{\kern0pt}sats{\isacharunderscore}{\kern0pt}ZF{\isacharunderscore}{\kern0pt}AC\ \isanewline
\ \ \ \ \ \ \isacommand{unfolding}\isamarkupfalse%
\ ZF{\isacharunderscore}{\kern0pt}choice{\isacharunderscore}{\kern0pt}fm{\isacharunderscore}{\kern0pt}def\ \isacommand{by}\isamarkupfalse%
\ simp\isanewline
\ \ \isacommand{qed}\isamarkupfalse%
\isanewline
\ \ \isacommand{moreover}\isamarkupfalse%
\isanewline
\ \ \isacommand{note}\isamarkupfalse%
\ {\isacartoucheopen}M\ {\isasymnoteq}\ {\isacharquery}{\kern0pt}N{\isacartoucheclose}\isanewline
\ \ \isacommand{moreover}\isamarkupfalse%
\isanewline
\ \ \isacommand{have}\isamarkupfalse%
\ {\isachardoublequoteopen}Transset{\isacharparenleft}{\kern0pt}{\isacharquery}{\kern0pt}N{\isacharparenright}{\kern0pt}{\isachardoublequoteclose}\ \isacommand{using}\isamarkupfalse%
\ Transset{\isacharunderscore}{\kern0pt}MG\ \isacommand{{\isachardot}{\kern0pt}}\isamarkupfalse%
\isanewline
\ \ \isacommand{moreover}\isamarkupfalse%
\isanewline
\ \ \isacommand{have}\isamarkupfalse%
\ {\isachardoublequoteopen}M\ {\isasymsubseteq}\ {\isacharquery}{\kern0pt}N{\isachardoublequoteclose}\ \isacommand{using}\isamarkupfalse%
\ M{\isacharunderscore}{\kern0pt}subset{\isacharunderscore}{\kern0pt}MG{\isacharbrackleft}{\kern0pt}OF\ one{\isacharunderscore}{\kern0pt}in{\isacharunderscore}{\kern0pt}G{\isacharbrackright}{\kern0pt}\ generic\ \isacommand{by}\isamarkupfalse%
\ simp\isanewline
\ \ \isacommand{ultimately}\isamarkupfalse%
\isanewline
\ \ \isacommand{show}\isamarkupfalse%
\ {\isacharquery}{\kern0pt}thesis\isanewline
\ \ \ \ \isacommand{using}\isamarkupfalse%
\ Ord{\isacharunderscore}{\kern0pt}MG{\isacharunderscore}{\kern0pt}iff\ MG{\isacharunderscore}{\kern0pt}eqpoll{\isacharunderscore}{\kern0pt}nat\isanewline
\ \ \ \ \isacommand{by}\isamarkupfalse%
\ {\isacharparenleft}{\kern0pt}rule{\isacharunderscore}{\kern0pt}tac\ x{\isacharequal}{\kern0pt}{\isachardoublequoteopen}{\isacharquery}{\kern0pt}N{\isachardoublequoteclose}\ \isakeyword{in}\ exI{\isacharcomma}{\kern0pt}\ simp{\isacharparenright}{\kern0pt}\isanewline
\isacommand{qed}\isamarkupfalse%
%
\endisatagproof
{\isafoldproof}%
%
\isadelimproof
\isanewline
%
\endisadelimproof
%
\isadelimtheory
\isanewline
%
\endisadelimtheory
%
\isatagtheory
\isacommand{end}\isamarkupfalse%
%
\endisatagtheory
{\isafoldtheory}%
%
\isadelimtheory
%
\endisadelimtheory
%
\end{isabellebody}%
\endinput
%:%file=~/source/repos/ZF-notAC/code/Forcing/Forcing_Main.thy%:%
%:%11=1%:%
%:%27=2%:%
%:%28=2%:%
%:%29=3%:%
%:%30=4%:%
%:%31=5%:%
%:%32=6%:%
%:%33=7%:%
%:%34=8%:%
%:%35=9%:%
%:%49=11%:%
%:%59=19%:%
%:%60=19%:%
%:%61=20%:%
%:%62=21%:%
%:%63=22%:%
%:%64=23%:%
%:%65=23%:%
%:%66=24%:%
%:%67=25%:%
%:%68=26%:%
%:%69=27%:%
%:%76=28%:%
%:%77=28%:%
%:%78=29%:%
%:%79=29%:%
%:%80=30%:%
%:%81=30%:%
%:%82=31%:%
%:%83=31%:%
%:%84=31%:%
%:%85=32%:%
%:%86=32%:%
%:%87=33%:%
%:%88=33%:%
%:%89=34%:%
%:%90=34%:%
%:%91=34%:%
%:%92=34%:%
%:%93=35%:%
%:%94=35%:%
%:%95=36%:%
%:%96=36%:%
%:%97=37%:%
%:%98=37%:%
%:%99=37%:%
%:%100=38%:%
%:%101=38%:%
%:%102=39%:%
%:%103=39%:%
%:%104=40%:%
%:%105=40%:%
%:%106=41%:%
%:%107=42%:%
%:%108=42%:%
%:%109=43%:%
%:%110=43%:%
%:%111=44%:%
%:%112=44%:%
%:%113=45%:%
%:%114=45%:%
%:%115=46%:%
%:%116=46%:%
%:%117=47%:%
%:%118=47%:%
%:%119=48%:%
%:%120=48%:%
%:%121=49%:%
%:%122=49%:%
%:%123=50%:%
%:%124=50%:%
%:%125=51%:%
%:%126=51%:%
%:%127=51%:%
%:%128=52%:%
%:%129=52%:%
%:%130=53%:%
%:%131=53%:%
%:%132=54%:%
%:%133=54%:%
%:%134=55%:%
%:%135=55%:%
%:%136=56%:%
%:%137=56%:%
%:%138=57%:%
%:%139=57%:%
%:%140=57%:%
%:%141=58%:%
%:%142=58%:%
%:%143=59%:%
%:%144=59%:%
%:%145=60%:%
%:%146=60%:%
%:%147=60%:%
%:%148=61%:%
%:%149=61%:%
%:%150=62%:%
%:%151=62%:%
%:%152=63%:%
%:%153=63%:%
%:%154=64%:%
%:%155=64%:%
%:%156=65%:%
%:%157=65%:%
%:%158=66%:%
%:%164=66%:%
%:%167=67%:%
%:%168=68%:%
%:%169=68%:%
%:%170=69%:%
%:%171=70%:%
%:%178=71%:%
%:%179=71%:%
%:%180=72%:%
%:%181=72%:%
%:%182=73%:%
%:%183=73%:%
%:%184=74%:%
%:%185=74%:%
%:%186=75%:%
%:%187=75%:%
%:%188=76%:%
%:%189=76%:%
%:%190=77%:%
%:%191=77%:%
%:%192=78%:%
%:%193=78%:%
%:%194=79%:%
%:%195=79%:%
%:%196=79%:%
%:%197=80%:%
%:%198=80%:%
%:%199=81%:%
%:%200=81%:%
%:%201=82%:%
%:%202=82%:%
%:%203=83%:%
%:%204=83%:%
%:%205=84%:%
%:%206=84%:%
%:%207=85%:%
%:%208=85%:%
%:%209=85%:%
%:%210=86%:%
%:%211=86%:%
%:%212=87%:%
%:%213=87%:%
%:%214=87%:%
%:%215=88%:%
%:%216=88%:%
%:%217=89%:%
%:%218=89%:%
%:%219=89%:%
%:%220=90%:%
%:%221=90%:%
%:%222=91%:%
%:%223=91%:%
%:%224=92%:%
%:%225=92%:%
%:%226=93%:%
%:%227=93%:%
%:%228=94%:%
%:%229=94%:%
%:%230=95%:%
%:%231=96%:%
%:%232=96%:%
%:%233=97%:%
%:%234=97%:%
%:%235=98%:%
%:%236=98%:%
%:%237=99%:%
%:%238=99%:%
%:%239=99%:%
%:%240=100%:%
%:%241=100%:%
%:%242=101%:%
%:%243=101%:%
%:%244=102%:%
%:%245=102%:%
%:%246=102%:%
%:%247=103%:%
%:%248=103%:%
%:%249=103%:%
%:%250=104%:%
%:%251=104%:%
%:%252=105%:%
%:%253=105%:%
%:%254=106%:%
%:%255=106%:%
%:%256=107%:%
%:%257=107%:%
%:%258=107%:%
%:%259=108%:%
%:%265=108%:%
%:%268=109%:%
%:%269=110%:%
%:%270=110%:%
%:%271=111%:%
%:%278=112%:%
%:%279=112%:%
%:%280=113%:%
%:%281=113%:%
%:%282=114%:%
%:%283=114%:%
%:%284=115%:%
%:%285=115%:%
%:%286=115%:%
%:%287=116%:%
%:%288=116%:%
%:%289=117%:%
%:%290=117%:%
%:%291=118%:%
%:%292=118%:%
%:%293=118%:%
%:%294=119%:%
%:%295=119%:%
%:%296=120%:%
%:%297=120%:%
%:%298=121%:%
%:%299=121%:%
%:%300=122%:%
%:%301=122%:%
%:%302=122%:%
%:%303=123%:%
%:%304=123%:%
%:%305=124%:%
%:%306=124%:%
%:%307=125%:%
%:%308=125%:%
%:%309=125%:%
%:%310=126%:%
%:%316=126%:%
%:%319=127%:%
%:%320=128%:%
%:%321=128%:%
%:%328=129%:%
%:%329=129%:%
%:%330=130%:%
%:%331=130%:%
%:%332=131%:%
%:%333=131%:%
%:%334=132%:%
%:%335=133%:%
%:%336=134%:%
%:%337=135%:%
%:%338=135%:%
%:%339=136%:%
%:%340=136%:%
%:%341=137%:%
%:%342=137%:%
%:%343=137%:%
%:%344=138%:%
%:%345=138%:%
%:%346=139%:%
%:%347=139%:%
%:%348=140%:%
%:%349=140%:%
%:%350=141%:%
%:%351=141%:%
%:%352=142%:%
%:%353=142%:%
%:%354=143%:%
%:%355=143%:%
%:%356=144%:%
%:%357=144%:%
%:%358=144%:%
%:%359=145%:%
%:%360=145%:%
%:%361=146%:%
%:%362=146%:%
%:%363=146%:%
%:%364=147%:%
%:%365=147%:%
%:%366=148%:%
%:%381=150%:%
%:%391=152%:%
%:%392=152%:%
%:%393=153%:%
%:%394=154%:%
%:%395=155%:%
%:%396=156%:%
%:%399=159%:%
%:%406=160%:%
%:%407=160%:%
%:%408=161%:%
%:%409=161%:%
%:%410=162%:%
%:%411=162%:%
%:%412=163%:%
%:%413=163%:%
%:%414=163%:%
%:%415=163%:%
%:%416=164%:%
%:%417=164%:%
%:%418=165%:%
%:%419=165%:%
%:%420=166%:%
%:%421=166%:%
%:%422=167%:%
%:%423=167%:%
%:%424=168%:%
%:%425=168%:%
%:%426=169%:%
%:%427=169%:%
%:%428=170%:%
%:%429=170%:%
%:%430=171%:%
%:%431=171%:%
%:%432=172%:%
%:%433=172%:%
%:%434=173%:%
%:%435=173%:%
%:%436=174%:%
%:%437=174%:%
%:%438=175%:%
%:%439=175%:%
%:%440=175%:%
%:%441=176%:%
%:%442=176%:%
%:%443=176%:%
%:%444=177%:%
%:%445=177%:%
%:%446=178%:%
%:%447=178%:%
%:%448=178%:%
%:%449=179%:%
%:%450=179%:%
%:%451=180%:%
%:%452=180%:%
%:%453=181%:%
%:%454=181%:%
%:%455=181%:%
%:%456=181%:%
%:%457=182%:%
%:%458=182%:%
%:%459=183%:%
%:%460=183%:%
%:%461=183%:%
%:%462=183%:%
%:%463=184%:%
%:%464=184%:%
%:%465=185%:%
%:%466=185%:%
%:%467=185%:%
%:%468=185%:%
%:%469=186%:%
%:%470=186%:%
%:%471=187%:%
%:%472=187%:%
%:%473=188%:%
%:%474=188%:%
%:%475=189%:%
%:%476=190%:%
%:%477=190%:%
%:%478=191%:%
%:%479=191%:%
%:%480=192%:%
%:%481=192%:%
%:%482=192%:%
%:%483=193%:%
%:%484=193%:%
%:%485=194%:%
%:%486=194%:%
%:%487=195%:%
%:%488=196%:%
%:%489=197%:%
%:%490=198%:%
%:%491=198%:%
%:%492=199%:%
%:%493=199%:%
%:%494=200%:%
%:%495=200%:%
%:%496=200%:%
%:%497=201%:%
%:%498=201%:%
%:%499=202%:%
%:%500=202%:%
%:%501=203%:%
%:%502=203%:%
%:%503=204%:%
%:%504=204%:%
%:%505=205%:%
%:%506=205%:%
%:%507=206%:%
%:%508=206%:%
%:%509=207%:%
%:%510=207%:%
%:%511=207%:%
%:%512=207%:%
%:%513=208%:%
%:%514=208%:%
%:%515=209%:%
%:%516=209%:%
%:%517=209%:%
%:%518=209%:%
%:%519=210%:%
%:%520=210%:%
%:%521=211%:%
%:%522=211%:%
%:%523=212%:%
%:%524=212%:%
%:%525=213%:%
%:%526=213%:%
%:%527=213%:%
%:%528=214%:%
%:%529=214%:%
%:%530=215%:%
%:%531=215%:%
%:%532=216%:%
%:%533=216%:%
%:%534=217%:%
%:%535=217%:%
%:%536=218%:%
%:%537=218%:%
%:%538=218%:%
%:%539=218%:%
%:%540=219%:%
%:%541=219%:%
%:%542=220%:%
%:%543=220%:%
%:%544=220%:%
%:%545=220%:%
%:%546=221%:%
%:%547=221%:%
%:%548=222%:%
%:%549=222%:%
%:%550=223%:%
%:%551=223%:%
%:%552=224%:%
%:%553=224%:%
%:%554=225%:%
%:%560=225%:%
%:%565=226%:%
%:%570=227%:%

%
\begin{isabellebody}%
\setisabellecontext{Utilities}%
%
\isadelimtheory
%
\endisadelimtheory
%
\isatagtheory
\isacommand{theory}\isamarkupfalse%
\ Utilities\ \isanewline
\ \ \isakeyword{imports}\ \isanewline
\ \ \ \ ZF\ \isanewline
\ \ \ \ {\isachardoublequoteopen}Forcing{\isacharslash}{\kern0pt}Forcing{\isacharunderscore}{\kern0pt}Main{\isachardoublequoteclose}\isanewline
\isakeyword{begin}%
\endisatagtheory
{\isafoldtheory}%
%
\isadelimtheory
\ \isanewline
%
\endisadelimtheory
\isanewline
\isacommand{lemma}\isamarkupfalse%
\ function{\isacharunderscore}{\kern0pt}value\ {\isacharcolon}{\kern0pt}\ {\isachardoublequoteopen}x\ {\isasymin}\ D\ {\isasymLongrightarrow}\ {\isacharbraceleft}{\kern0pt}\ {\isacharless}{\kern0pt}y{\isacharcomma}{\kern0pt}\ f{\isacharparenleft}{\kern0pt}y{\isacharparenright}{\kern0pt}{\isachargreater}{\kern0pt}{\isachardot}{\kern0pt}\ y\ {\isasymin}\ D\ {\isacharbraceright}{\kern0pt}{\isacharbackquote}{\kern0pt}x\ {\isacharequal}{\kern0pt}\ f{\isacharparenleft}{\kern0pt}x{\isacharparenright}{\kern0pt}{\isachardoublequoteclose}\ \isanewline
%
\isadelimproof
%
\endisadelimproof
%
\isatagproof
\isacommand{proof}\isamarkupfalse%
\ {\isacharminus}{\kern0pt}\ \isanewline
\ \ \isacommand{fix}\isamarkupfalse%
\ x\ \isacommand{assume}\isamarkupfalse%
\ p{\isadigit{1}}\ {\isacharcolon}{\kern0pt}\ {\isachardoublequoteopen}x\ {\isasymin}\ D{\isachardoublequoteclose}\isanewline
\ \ \isacommand{have}\isamarkupfalse%
\ {\isachardoublequoteopen}f{\isacharparenleft}{\kern0pt}x{\isacharparenright}{\kern0pt}\ {\isasymin}\ {\isacharparenleft}{\kern0pt}{\isacharbraceleft}{\kern0pt}\ {\isacharless}{\kern0pt}y{\isacharcomma}{\kern0pt}\ f{\isacharparenleft}{\kern0pt}y{\isacharparenright}{\kern0pt}{\isachargreater}{\kern0pt}{\isachardot}{\kern0pt}\ y\ {\isasymin}\ D\ {\isacharbraceright}{\kern0pt}{\isacharbackquote}{\kern0pt}{\isacharbackquote}{\kern0pt}{\isacharbraceleft}{\kern0pt}x{\isacharbraceright}{\kern0pt}{\isacharparenright}{\kern0pt}{\isachardoublequoteclose}\ \isacommand{using}\isamarkupfalse%
\ p{\isadigit{1}}\ \isacommand{by}\isamarkupfalse%
\ auto\isanewline
\ \ \isacommand{then}\isamarkupfalse%
\ \isacommand{have}\isamarkupfalse%
\ {\isachardoublequoteopen}{\isacharparenleft}{\kern0pt}{\isacharbraceleft}{\kern0pt}\ {\isacharless}{\kern0pt}y{\isacharcomma}{\kern0pt}\ f{\isacharparenleft}{\kern0pt}y{\isacharparenright}{\kern0pt}{\isachargreater}{\kern0pt}{\isachardot}{\kern0pt}\ y\ {\isasymin}\ D\ {\isacharbraceright}{\kern0pt}{\isacharbackquote}{\kern0pt}{\isacharbackquote}{\kern0pt}{\isacharbraceleft}{\kern0pt}x{\isacharbraceright}{\kern0pt}{\isacharparenright}{\kern0pt}\ {\isacharequal}{\kern0pt}\ {\isacharbraceleft}{\kern0pt}\ f{\isacharparenleft}{\kern0pt}x{\isacharparenright}{\kern0pt}\ {\isacharbraceright}{\kern0pt}{\isachardoublequoteclose}\ \isanewline
\ \ \ \ \isacommand{apply}\isamarkupfalse%
\ {\isacharparenleft}{\kern0pt}rule{\isacharunderscore}{\kern0pt}tac\ b{\isacharequal}{\kern0pt}{\isachardoublequoteopen}f{\isacharparenleft}{\kern0pt}x{\isacharparenright}{\kern0pt}{\isachardoublequoteclose}\ \isakeyword{and}\ a\ {\isacharequal}{\kern0pt}\ {\isachardoublequoteopen}f{\isacharparenleft}{\kern0pt}x{\isacharparenright}{\kern0pt}{\isachardoublequoteclose}\ \isakeyword{in}\ equal{\isacharunderscore}{\kern0pt}singleton{\isacharparenright}{\kern0pt}\isanewline
\ \ \ \ \isacommand{by}\isamarkupfalse%
\ auto\ \isanewline
\ \ \isacommand{then}\isamarkupfalse%
\ \isacommand{have}\isamarkupfalse%
\ p{\isadigit{2}}\ {\isacharcolon}{\kern0pt}\ {\isachardoublequoteopen}{\isacharparenleft}{\kern0pt}{\isasymUnion}{\isacharparenleft}{\kern0pt}{\isacharbraceleft}{\kern0pt}\ {\isacharless}{\kern0pt}y{\isacharcomma}{\kern0pt}\ f{\isacharparenleft}{\kern0pt}y{\isacharparenright}{\kern0pt}{\isachargreater}{\kern0pt}{\isachardot}{\kern0pt}\ y\ {\isasymin}\ D\ {\isacharbraceright}{\kern0pt}{\isacharbackquote}{\kern0pt}{\isacharbackquote}{\kern0pt}{\isacharbraceleft}{\kern0pt}x{\isacharbraceright}{\kern0pt}{\isacharparenright}{\kern0pt}{\isacharparenright}{\kern0pt}\ {\isacharequal}{\kern0pt}\ f{\isacharparenleft}{\kern0pt}x{\isacharparenright}{\kern0pt}{\isachardoublequoteclose}\ \isacommand{by}\isamarkupfalse%
\ auto\ \isanewline
\ \ \isacommand{have}\isamarkupfalse%
\ {\isachardoublequoteopen}{\isacharparenleft}{\kern0pt}{\isacharbraceleft}{\kern0pt}\ {\isacharless}{\kern0pt}y{\isacharcomma}{\kern0pt}\ f{\isacharparenleft}{\kern0pt}y{\isacharparenright}{\kern0pt}{\isachargreater}{\kern0pt}{\isachardot}{\kern0pt}\ y\ {\isasymin}\ D\ {\isacharbraceright}{\kern0pt}{\isacharparenright}{\kern0pt}{\isacharbackquote}{\kern0pt}x\ {\isacharequal}{\kern0pt}\ {\isacharparenleft}{\kern0pt}{\isasymUnion}{\isacharparenleft}{\kern0pt}{\isacharbraceleft}{\kern0pt}\ {\isacharless}{\kern0pt}y{\isacharcomma}{\kern0pt}\ f{\isacharparenleft}{\kern0pt}y{\isacharparenright}{\kern0pt}{\isachargreater}{\kern0pt}{\isachardot}{\kern0pt}\ y\ {\isasymin}\ D\ {\isacharbraceright}{\kern0pt}{\isacharbackquote}{\kern0pt}{\isacharbackquote}{\kern0pt}{\isacharbraceleft}{\kern0pt}x{\isacharbraceright}{\kern0pt}{\isacharparenright}{\kern0pt}{\isacharparenright}{\kern0pt}{\isachardoublequoteclose}\ \isanewline
\ \ \ \ \isacommand{unfolding}\isamarkupfalse%
\ ZF{\isacharunderscore}{\kern0pt}Base{\isachardot}{\kern0pt}apply{\isacharunderscore}{\kern0pt}def\ \isacommand{by}\isamarkupfalse%
\ auto\ \isanewline
\ \ \isacommand{then}\isamarkupfalse%
\ \isacommand{show}\isamarkupfalse%
\ {\isachardoublequoteopen}{\isacharparenleft}{\kern0pt}{\isacharbraceleft}{\kern0pt}\ {\isacharless}{\kern0pt}y{\isacharcomma}{\kern0pt}\ f{\isacharparenleft}{\kern0pt}y{\isacharparenright}{\kern0pt}{\isachargreater}{\kern0pt}{\isachardot}{\kern0pt}\ y\ {\isasymin}\ D\ {\isacharbraceright}{\kern0pt}{\isacharparenright}{\kern0pt}{\isacharbackquote}{\kern0pt}x\ {\isacharequal}{\kern0pt}\ f{\isacharparenleft}{\kern0pt}x{\isacharparenright}{\kern0pt}{\isachardoublequoteclose}\ \isacommand{using}\isamarkupfalse%
\ p{\isadigit{2}}\ \isacommand{by}\isamarkupfalse%
\ auto\ \isanewline
\isacommand{qed}\isamarkupfalse%
%
\endisatagproof
{\isafoldproof}%
%
\isadelimproof
\isanewline
%
\endisadelimproof
\isanewline
\isacommand{lemma}\isamarkupfalse%
\ function{\isacharunderscore}{\kern0pt}value{\isacharunderscore}{\kern0pt}in\ {\isacharcolon}{\kern0pt}\ {\isachardoublequoteopen}f\ {\isasymin}\ A\ {\isasymrightarrow}\ B\ {\isasymLongrightarrow}\ a\ {\isasymin}\ A\ {\isasymLongrightarrow}\ f{\isacharbackquote}{\kern0pt}a\ {\isasymin}\ B{\isachardoublequoteclose}\ \isanewline
%
\isadelimproof
\ \ %
\endisadelimproof
%
\isatagproof
\isacommand{by}\isamarkupfalse%
\ auto%
\endisatagproof
{\isafoldproof}%
%
\isadelimproof
\ \isanewline
%
\endisadelimproof
\isanewline
\isacommand{lemma}\isamarkupfalse%
\ relation{\isacharunderscore}{\kern0pt}subset{\isacharunderscore}{\kern0pt}domran\ {\isacharcolon}{\kern0pt}\ \isanewline
\ \ {\isachardoublequoteopen}relation{\isacharparenleft}{\kern0pt}C{\isacharparenright}{\kern0pt}\ {\isasymLongrightarrow}\ x\ {\isasymin}\ C\ {\isasymLongrightarrow}\ x\ {\isasymin}\ domain{\isacharparenleft}{\kern0pt}C{\isacharparenright}{\kern0pt}\ {\isasymtimes}\ range{\isacharparenleft}{\kern0pt}C{\isacharparenright}{\kern0pt}{\isachardoublequoteclose}\ \isanewline
%
\isadelimproof
%
\endisadelimproof
%
\isatagproof
\isacommand{proof}\isamarkupfalse%
\ {\isacharminus}{\kern0pt}\ \isanewline
\ \ \isacommand{assume}\isamarkupfalse%
\ assms{\isacharcolon}{\kern0pt}\ {\isachardoublequoteopen}relation{\isacharparenleft}{\kern0pt}C{\isacharparenright}{\kern0pt}{\isachardoublequoteclose}\ {\isachardoublequoteopen}x\ {\isasymin}\ C{\isachardoublequoteclose}\isanewline
\ \ \isacommand{then}\isamarkupfalse%
\ \isacommand{obtain}\isamarkupfalse%
\ a\ b\ \isakeyword{where}\ abh{\isacharcolon}{\kern0pt}\ {\isachardoublequoteopen}x\ {\isacharequal}{\kern0pt}\ {\isacharless}{\kern0pt}a{\isacharcomma}{\kern0pt}\ b{\isachargreater}{\kern0pt}{\isachardoublequoteclose}\ \isacommand{unfolding}\isamarkupfalse%
\ relation{\isacharunderscore}{\kern0pt}def\ \isacommand{by}\isamarkupfalse%
\ auto\ \isanewline
\ \ \isacommand{then}\isamarkupfalse%
\ \isacommand{have}\isamarkupfalse%
\ adom\ {\isacharcolon}{\kern0pt}\ {\isachardoublequoteopen}a\ {\isasymin}\ domain{\isacharparenleft}{\kern0pt}C{\isacharparenright}{\kern0pt}{\isachardoublequoteclose}\ \isacommand{unfolding}\isamarkupfalse%
\ domain{\isacharunderscore}{\kern0pt}def\ \isacommand{using}\isamarkupfalse%
\ assms\ \isacommand{by}\isamarkupfalse%
\ auto\isanewline
\ \ \isacommand{have}\isamarkupfalse%
\ {\isachardoublequoteopen}b\ {\isasymin}\ range{\isacharparenleft}{\kern0pt}C{\isacharparenright}{\kern0pt}{\isachardoublequoteclose}\ \isacommand{using}\isamarkupfalse%
\ rangeI\ abh\ assms\ \isacommand{by}\isamarkupfalse%
\ auto\ \isanewline
\ \ \isacommand{then}\isamarkupfalse%
\ \isacommand{have}\isamarkupfalse%
\ {\isachardoublequoteopen}{\isacharless}{\kern0pt}a{\isacharcomma}{\kern0pt}\ b{\isachargreater}{\kern0pt}\ {\isasymin}\ domain{\isacharparenleft}{\kern0pt}C{\isacharparenright}{\kern0pt}\ {\isasymtimes}\ range{\isacharparenleft}{\kern0pt}C{\isacharparenright}{\kern0pt}{\isachardoublequoteclose}\ \isacommand{using}\isamarkupfalse%
\ adom\ \isacommand{by}\isamarkupfalse%
\ auto\ \isanewline
\ \ \isacommand{then}\isamarkupfalse%
\ \isacommand{show}\isamarkupfalse%
\ {\isachardoublequoteopen}x\ {\isasymin}\ \ domain{\isacharparenleft}{\kern0pt}C{\isacharparenright}{\kern0pt}\ {\isasymtimes}\ range{\isacharparenleft}{\kern0pt}C{\isacharparenright}{\kern0pt}{\isachardoublequoteclose}\ \isacommand{using}\isamarkupfalse%
\ abh\ \isacommand{by}\isamarkupfalse%
\ auto\ \isanewline
\isacommand{qed}\isamarkupfalse%
%
\endisatagproof
{\isafoldproof}%
%
\isadelimproof
\isanewline
%
\endisadelimproof
\isanewline
\isacommand{lemma}\isamarkupfalse%
\ pair{\isacharunderscore}{\kern0pt}rel{\isacharunderscore}{\kern0pt}arg\ {\isacharcolon}{\kern0pt}\ \isanewline
\ \ {\isachardoublequoteopen}relation{\isacharparenleft}{\kern0pt}C{\isacharparenright}{\kern0pt}\ {\isasymLongrightarrow}\ v\ {\isasymin}\ {\isacharbraceleft}{\kern0pt}\ F{\isacharparenleft}{\kern0pt}x{\isacharcomma}{\kern0pt}\ y{\isacharparenright}{\kern0pt}\ {\isachardot}{\kern0pt}\ {\isacharless}{\kern0pt}x{\isacharcomma}{\kern0pt}\ y{\isachargreater}{\kern0pt}\ {\isasymin}\ C\ {\isacharbraceright}{\kern0pt}\ {\isasymLongrightarrow}\ {\isasymexists}x{\isachardot}{\kern0pt}\ {\isasymexists}y{\isachardot}{\kern0pt}\ {\isacharless}{\kern0pt}x{\isacharcomma}{\kern0pt}\ y{\isachargreater}{\kern0pt}\ {\isasymin}\ C\ {\isasymand}\ v\ {\isacharequal}{\kern0pt}\ F{\isacharparenleft}{\kern0pt}x{\isacharcomma}{\kern0pt}\ y{\isacharparenright}{\kern0pt}{\isachardoublequoteclose}\isanewline
%
\isadelimproof
%
\endisadelimproof
%
\isatagproof
\isacommand{proof}\isamarkupfalse%
\ {\isacharminus}{\kern0pt}\ \isanewline
\ \ \isacommand{assume}\isamarkupfalse%
\ assms{\isacharcolon}{\kern0pt}\ {\isachardoublequoteopen}relation{\isacharparenleft}{\kern0pt}C{\isacharparenright}{\kern0pt}{\isachardoublequoteclose}\ {\isachardoublequoteopen}v\ {\isasymin}\ {\isacharbraceleft}{\kern0pt}\ F{\isacharparenleft}{\kern0pt}x{\isacharcomma}{\kern0pt}\ y{\isacharparenright}{\kern0pt}\ {\isachardot}{\kern0pt}\ {\isacharless}{\kern0pt}x{\isacharcomma}{\kern0pt}\ y{\isachargreater}{\kern0pt}\ {\isasymin}\ C\ {\isacharbraceright}{\kern0pt}{\isachardoublequoteclose}\ \isanewline
\ \ \isacommand{then}\isamarkupfalse%
\ \isacommand{obtain}\isamarkupfalse%
\ u\ \isakeyword{where}\ i{\isacharcolon}{\kern0pt}\ {\isachardoublequoteopen}u\ {\isasymin}\ C{\isachardoublequoteclose}\ {\isachardoublequoteopen}v\ {\isacharequal}{\kern0pt}\ split{\isacharparenleft}{\kern0pt}F{\isacharcomma}{\kern0pt}\ u{\isacharparenright}{\kern0pt}{\isachardoublequoteclose}\ \isacommand{by}\isamarkupfalse%
\ auto\ \isanewline
\ \ \isacommand{then}\isamarkupfalse%
\ \isacommand{have}\isamarkupfalse%
\ h\ {\isacharcolon}{\kern0pt}\ {\isachardoublequoteopen}v\ {\isacharequal}{\kern0pt}\ F{\isacharparenleft}{\kern0pt}fst{\isacharparenleft}{\kern0pt}u{\isacharparenright}{\kern0pt}{\isacharcomma}{\kern0pt}\ snd{\isacharparenleft}{\kern0pt}u{\isacharparenright}{\kern0pt}{\isacharparenright}{\kern0pt}{\isachardoublequoteclose}\ \isacommand{unfolding}\isamarkupfalse%
\ split{\isacharunderscore}{\kern0pt}def\ i\ \isacommand{by}\isamarkupfalse%
\ auto\ \isanewline
\ \ \isacommand{have}\isamarkupfalse%
\ {\isachardoublequoteopen}{\isacharless}{\kern0pt}fst{\isacharparenleft}{\kern0pt}u{\isacharparenright}{\kern0pt}{\isacharcomma}{\kern0pt}\ snd{\isacharparenleft}{\kern0pt}u{\isacharparenright}{\kern0pt}{\isachargreater}{\kern0pt}\ {\isacharequal}{\kern0pt}\ u{\isachardoublequoteclose}\ \isanewline
\ \ \ \ \isacommand{apply}\isamarkupfalse%
\ {\isacharparenleft}{\kern0pt}rule{\isacharunderscore}{\kern0pt}tac\ a{\isacharequal}{\kern0pt}u\ \isakeyword{and}\ A{\isacharequal}{\kern0pt}{\isachardoublequoteopen}domain{\isacharparenleft}{\kern0pt}C{\isacharparenright}{\kern0pt}{\isachardoublequoteclose}\ \isakeyword{and}\ B{\isacharequal}{\kern0pt}{\isachardoublequoteopen}{\isasymlambda}x{\isachardot}{\kern0pt}\ range{\isacharparenleft}{\kern0pt}C{\isacharparenright}{\kern0pt}{\isachardoublequoteclose}\ \isakeyword{in}\ pair{\isachardot}{\kern0pt}Pair{\isacharunderscore}{\kern0pt}fst{\isacharunderscore}{\kern0pt}snd{\isacharunderscore}{\kern0pt}eq{\isacharparenright}{\kern0pt}\isanewline
\ \ \ \ \isacommand{apply}\isamarkupfalse%
\ {\isacharparenleft}{\kern0pt}rule{\isacharunderscore}{\kern0pt}tac\ relation{\isacharunderscore}{\kern0pt}subset{\isacharunderscore}{\kern0pt}domran{\isacharparenright}{\kern0pt}\ \isanewline
\ \ \ \ \isacommand{using}\isamarkupfalse%
\ assms\ i\ \isacommand{by}\isamarkupfalse%
\ auto\isanewline
\ \ \isacommand{then}\isamarkupfalse%
\ \isacommand{have}\isamarkupfalse%
\ {\isachardoublequoteopen}{\isacharless}{\kern0pt}fst{\isacharparenleft}{\kern0pt}u{\isacharparenright}{\kern0pt}{\isacharcomma}{\kern0pt}\ snd{\isacharparenleft}{\kern0pt}u{\isacharparenright}{\kern0pt}{\isachargreater}{\kern0pt}\ {\isasymin}\ C{\isachardoublequoteclose}\ \isacommand{using}\isamarkupfalse%
\ i\ \isacommand{by}\isamarkupfalse%
\ auto\ \isanewline
\ \ \isacommand{then}\isamarkupfalse%
\ \isacommand{show}\isamarkupfalse%
\ {\isacharquery}{\kern0pt}thesis\ \isacommand{using}\isamarkupfalse%
\ h\ \isacommand{by}\isamarkupfalse%
\ auto\ \isanewline
\isacommand{qed}\isamarkupfalse%
%
\endisatagproof
{\isafoldproof}%
%
\isadelimproof
\isanewline
%
\endisadelimproof
\isanewline
\isacommand{lemma}\isamarkupfalse%
\ pair{\isacharunderscore}{\kern0pt}relI\ {\isacharcolon}{\kern0pt}\ {\isachardoublequoteopen}{\isacharless}{\kern0pt}a{\isacharcomma}{\kern0pt}\ b{\isachargreater}{\kern0pt}\ {\isasymin}\ x\ {\isasymLongrightarrow}\ F{\isacharparenleft}{\kern0pt}a{\isacharcomma}{\kern0pt}\ b{\isacharparenright}{\kern0pt}\ {\isasymin}\ {\isacharbraceleft}{\kern0pt}\ F{\isacharparenleft}{\kern0pt}a{\isacharcomma}{\kern0pt}\ b{\isacharparenright}{\kern0pt}{\isachardot}{\kern0pt}\ {\isacharless}{\kern0pt}a{\isacharcomma}{\kern0pt}\ b{\isachargreater}{\kern0pt}\ {\isasymin}\ x\ {\isacharbraceright}{\kern0pt}{\isachardoublequoteclose}\isanewline
%
\isadelimproof
\ \ %
\endisadelimproof
%
\isatagproof
\isacommand{apply}\isamarkupfalse%
\ auto\ \isacommand{apply}\isamarkupfalse%
\ {\isacharparenleft}{\kern0pt}rule{\isacharunderscore}{\kern0pt}tac\ x{\isacharequal}{\kern0pt}{\isachardoublequoteopen}{\isacharless}{\kern0pt}a{\isacharcomma}{\kern0pt}\ b{\isachargreater}{\kern0pt}{\isachardoublequoteclose}\ \isakeyword{in}\ bexI{\isacharparenright}{\kern0pt}\ \isacommand{apply}\isamarkupfalse%
\ auto\ \isacommand{done}\isamarkupfalse%
%
\endisatagproof
{\isafoldproof}%
%
\isadelimproof
\ \isanewline
%
\endisadelimproof
\isanewline
\isacommand{lemma}\isamarkupfalse%
\ eq{\isacharunderscore}{\kern0pt}flip\ {\isacharcolon}{\kern0pt}\ {\isachardoublequoteopen}A\ {\isacharequal}{\kern0pt}\ B\ {\isasymLongrightarrow}\ B\ {\isacharequal}{\kern0pt}\ A{\isachardoublequoteclose}%
\isadelimproof
\ %
\endisadelimproof
%
\isatagproof
\isacommand{by}\isamarkupfalse%
\ auto%
\endisatagproof
{\isafoldproof}%
%
\isadelimproof
%
\endisadelimproof
\ \isanewline
\isanewline
\isacommand{lemma}\isamarkupfalse%
\ pair{\isacharunderscore}{\kern0pt}rel{\isacharunderscore}{\kern0pt}eq\ {\isacharcolon}{\kern0pt}\ \isanewline
\ \ {\isachardoublequoteopen}relation{\isacharparenleft}{\kern0pt}C{\isacharparenright}{\kern0pt}\isanewline
\ \ \ {\isasymLongrightarrow}\ {\isacharparenleft}{\kern0pt}{\isasymforall}x{\isachardot}{\kern0pt}\ {\isasymforall}y{\isachardot}{\kern0pt}\ {\isacharless}{\kern0pt}x{\isacharcomma}{\kern0pt}\ y{\isachargreater}{\kern0pt}\ {\isasymin}\ C\ {\isasymlongrightarrow}\ F{\isacharparenleft}{\kern0pt}x{\isacharcomma}{\kern0pt}\ y{\isacharparenright}{\kern0pt}\ {\isacharequal}{\kern0pt}\ G{\isacharparenleft}{\kern0pt}x{\isacharcomma}{\kern0pt}\ y{\isacharparenright}{\kern0pt}{\isacharparenright}{\kern0pt}\ \isanewline
\ \ \ {\isasymLongrightarrow}\ {\isacharbraceleft}{\kern0pt}\ F{\isacharparenleft}{\kern0pt}x{\isacharcomma}{\kern0pt}\ y{\isacharparenright}{\kern0pt}{\isachardot}{\kern0pt}\ {\isacharless}{\kern0pt}x{\isacharcomma}{\kern0pt}\ y{\isachargreater}{\kern0pt}\ {\isasymin}\ C\ {\isacharbraceright}{\kern0pt}\ {\isacharequal}{\kern0pt}\ {\isacharbraceleft}{\kern0pt}\ G{\isacharparenleft}{\kern0pt}x{\isacharcomma}{\kern0pt}\ y{\isacharparenright}{\kern0pt}{\isachardot}{\kern0pt}\ {\isacharless}{\kern0pt}x{\isacharcomma}{\kern0pt}\ y{\isachargreater}{\kern0pt}\ {\isasymin}\ C\ {\isacharbraceright}{\kern0pt}{\isachardoublequoteclose}\ \isanewline
%
\isadelimproof
\ \ %
\endisadelimproof
%
\isatagproof
\isacommand{apply}\isamarkupfalse%
\ {\isacharparenleft}{\kern0pt}rule\ equality{\isacharunderscore}{\kern0pt}iffI{\isacharsemicolon}{\kern0pt}\ rule\ iffI{\isacharparenright}{\kern0pt}\isanewline
\isacommand{proof}\isamarkupfalse%
\ {\isacharminus}{\kern0pt}\ \isanewline
\ \ \isacommand{fix}\isamarkupfalse%
\ v\ \isacommand{assume}\isamarkupfalse%
\ assms\ {\isacharcolon}{\kern0pt}\isanewline
\ \ \ \ {\isachardoublequoteopen}relation{\isacharparenleft}{\kern0pt}C{\isacharparenright}{\kern0pt}{\isachardoublequoteclose}\isanewline
\ \ \ \ {\isachardoublequoteopen}{\isasymforall}x{\isachardot}{\kern0pt}\ {\isasymforall}y{\isachardot}{\kern0pt}\ {\isacharless}{\kern0pt}x{\isacharcomma}{\kern0pt}\ y{\isachargreater}{\kern0pt}\ {\isasymin}\ C\ {\isasymlongrightarrow}\ F{\isacharparenleft}{\kern0pt}x{\isacharcomma}{\kern0pt}\ y{\isacharparenright}{\kern0pt}\ {\isacharequal}{\kern0pt}\ G{\isacharparenleft}{\kern0pt}x{\isacharcomma}{\kern0pt}\ y{\isacharparenright}{\kern0pt}{\isachardoublequoteclose}\ \isanewline
\ \ \ \ {\isachardoublequoteopen}v\ {\isasymin}\ {\isacharbraceleft}{\kern0pt}F{\isacharparenleft}{\kern0pt}x{\isacharcomma}{\kern0pt}\ y{\isacharparenright}{\kern0pt}\ {\isachardot}{\kern0pt}\ {\isasymlangle}x{\isacharcomma}{\kern0pt}y{\isasymrangle}\ {\isasymin}\ C{\isacharbraceright}{\kern0pt}{\isachardoublequoteclose}\isanewline
\ \ \isacommand{then}\isamarkupfalse%
\ \isacommand{have}\isamarkupfalse%
\ {\isachardoublequoteopen}{\isasymexists}x\ y{\isachardot}{\kern0pt}\ {\isasymlangle}x{\isacharcomma}{\kern0pt}\ y{\isasymrangle}\ {\isasymin}\ C\ {\isasymand}\ v\ {\isacharequal}{\kern0pt}\ F{\isacharparenleft}{\kern0pt}x{\isacharcomma}{\kern0pt}\ y{\isacharparenright}{\kern0pt}{\isachardoublequoteclose}\ \isanewline
\ \ \ \ \isacommand{by}\isamarkupfalse%
\ {\isacharparenleft}{\kern0pt}rule{\isacharunderscore}{\kern0pt}tac\ pair{\isacharunderscore}{\kern0pt}rel{\isacharunderscore}{\kern0pt}arg{\isacharsemicolon}{\kern0pt}\ auto{\isacharparenright}{\kern0pt}\isanewline
\ \ \isacommand{then}\isamarkupfalse%
\ \isacommand{obtain}\isamarkupfalse%
\ x\ y\ \isakeyword{where}\ p{\isadigit{1}}\ {\isacharcolon}{\kern0pt}\ {\isachardoublequoteopen}{\isacharless}{\kern0pt}x{\isacharcomma}{\kern0pt}\ y{\isachargreater}{\kern0pt}\ {\isasymin}\ C{\isachardoublequoteclose}\ {\isachardoublequoteopen}v{\isacharequal}{\kern0pt}F{\isacharparenleft}{\kern0pt}x{\isacharcomma}{\kern0pt}\ y{\isacharparenright}{\kern0pt}{\isachardoublequoteclose}\ \isacommand{by}\isamarkupfalse%
\ auto\isanewline
\ \ \isacommand{then}\isamarkupfalse%
\ \isacommand{have}\isamarkupfalse%
\ p{\isadigit{2}}\ {\isacharcolon}{\kern0pt}\ {\isachardoublequoteopen}v\ {\isacharequal}{\kern0pt}\ G{\isacharparenleft}{\kern0pt}x{\isacharcomma}{\kern0pt}\ y{\isacharparenright}{\kern0pt}{\isachardoublequoteclose}\ \isacommand{using}\isamarkupfalse%
\ assms\ \isacommand{by}\isamarkupfalse%
\ auto\ \isanewline
\ \ \isacommand{then}\isamarkupfalse%
\ \isacommand{have}\isamarkupfalse%
\ p{\isadigit{3}}\ {\isacharcolon}{\kern0pt}\ {\isachardoublequoteopen}v\ {\isacharequal}{\kern0pt}\ split{\isacharparenleft}{\kern0pt}G{\isacharcomma}{\kern0pt}\ {\isacharless}{\kern0pt}x{\isacharcomma}{\kern0pt}\ y{\isachargreater}{\kern0pt}{\isacharparenright}{\kern0pt}{\isachardoublequoteclose}\ \isacommand{using}\isamarkupfalse%
\ p{\isadigit{2}}\ \isacommand{by}\isamarkupfalse%
\ auto\ \isanewline
\ \ \isacommand{then}\isamarkupfalse%
\ \isacommand{show}\isamarkupfalse%
\ {\isachardoublequoteopen}v\ {\isasymin}\ {\isacharbraceleft}{\kern0pt}G{\isacharparenleft}{\kern0pt}x{\isacharcomma}{\kern0pt}\ y{\isacharparenright}{\kern0pt}\ {\isachardot}{\kern0pt}\ {\isasymlangle}x{\isacharcomma}{\kern0pt}y{\isasymrangle}\ {\isasymin}\ C{\isacharbraceright}{\kern0pt}{\isachardoublequoteclose}\ \isanewline
\ \ \ \ \isacommand{apply}\isamarkupfalse%
\ simp\isanewline
\ \ \ \ \isacommand{apply}\isamarkupfalse%
\ {\isacharparenleft}{\kern0pt}rule{\isacharunderscore}{\kern0pt}tac\ x{\isacharequal}{\kern0pt}{\isachardoublequoteopen}{\isacharless}{\kern0pt}x{\isacharcomma}{\kern0pt}y{\isachargreater}{\kern0pt}{\isachardoublequoteclose}\ \isakeyword{in}\ bexI{\isacharparenright}{\kern0pt}\isanewline
\ \ \ \ \isacommand{using}\isamarkupfalse%
\ p{\isadigit{1}}\ p{\isadigit{2}}\ p{\isadigit{3}}\ \isacommand{by}\isamarkupfalse%
\ auto\ \isanewline
\isacommand{next}\isamarkupfalse%
\ \isanewline
\ \ \isacommand{fix}\isamarkupfalse%
\ v\ \isacommand{assume}\isamarkupfalse%
\ assms\ {\isacharcolon}{\kern0pt}\isanewline
\ \ \ \ {\isachardoublequoteopen}relation{\isacharparenleft}{\kern0pt}C{\isacharparenright}{\kern0pt}{\isachardoublequoteclose}\ \isanewline
\ \ \ \ {\isachardoublequoteopen}{\isasymforall}x{\isachardot}{\kern0pt}\ {\isasymforall}y{\isachardot}{\kern0pt}\ {\isacharless}{\kern0pt}x{\isacharcomma}{\kern0pt}\ y{\isachargreater}{\kern0pt}\ {\isasymin}\ C\ {\isasymlongrightarrow}\ F{\isacharparenleft}{\kern0pt}x{\isacharcomma}{\kern0pt}\ y{\isacharparenright}{\kern0pt}\ {\isacharequal}{\kern0pt}\ G{\isacharparenleft}{\kern0pt}x{\isacharcomma}{\kern0pt}\ y{\isacharparenright}{\kern0pt}{\isachardoublequoteclose}\ \isanewline
\ \ \ \ {\isachardoublequoteopen}v\ {\isasymin}\ {\isacharbraceleft}{\kern0pt}G{\isacharparenleft}{\kern0pt}x{\isacharcomma}{\kern0pt}\ y{\isacharparenright}{\kern0pt}\ {\isachardot}{\kern0pt}\ {\isasymlangle}x{\isacharcomma}{\kern0pt}y{\isasymrangle}\ {\isasymin}\ C{\isacharbraceright}{\kern0pt}{\isachardoublequoteclose}\isanewline
\ \ \isacommand{then}\isamarkupfalse%
\ \isacommand{have}\isamarkupfalse%
\ {\isachardoublequoteopen}{\isasymexists}x\ y{\isachardot}{\kern0pt}\ {\isasymlangle}x{\isacharcomma}{\kern0pt}\ y{\isasymrangle}\ {\isasymin}\ C\ {\isasymand}\ v\ {\isacharequal}{\kern0pt}\ G{\isacharparenleft}{\kern0pt}x{\isacharcomma}{\kern0pt}\ y{\isacharparenright}{\kern0pt}{\isachardoublequoteclose}\ \isanewline
\ \ \ \ \isacommand{by}\isamarkupfalse%
\ {\isacharparenleft}{\kern0pt}rule{\isacharunderscore}{\kern0pt}tac\ pair{\isacharunderscore}{\kern0pt}rel{\isacharunderscore}{\kern0pt}arg{\isacharsemicolon}{\kern0pt}\ auto{\isacharparenright}{\kern0pt}\isanewline
\ \ \isacommand{then}\isamarkupfalse%
\ \isacommand{obtain}\isamarkupfalse%
\ x\ y\ \isakeyword{where}\ p{\isadigit{1}}\ {\isacharcolon}{\kern0pt}\ {\isachardoublequoteopen}{\isacharless}{\kern0pt}x{\isacharcomma}{\kern0pt}\ y{\isachargreater}{\kern0pt}\ {\isasymin}\ C{\isachardoublequoteclose}\ {\isachardoublequoteopen}v{\isacharequal}{\kern0pt}G{\isacharparenleft}{\kern0pt}x{\isacharcomma}{\kern0pt}\ y{\isacharparenright}{\kern0pt}{\isachardoublequoteclose}\ \isacommand{using}\isamarkupfalse%
\ pair{\isacharunderscore}{\kern0pt}rel{\isacharunderscore}{\kern0pt}arg\ \isacommand{by}\isamarkupfalse%
\ auto\ \isanewline
\ \ \isacommand{then}\isamarkupfalse%
\ \isacommand{have}\isamarkupfalse%
\ p{\isadigit{2}}\ {\isacharcolon}{\kern0pt}\ {\isachardoublequoteopen}v\ {\isacharequal}{\kern0pt}\ F{\isacharparenleft}{\kern0pt}x{\isacharcomma}{\kern0pt}\ y{\isacharparenright}{\kern0pt}{\isachardoublequoteclose}\ \isacommand{using}\isamarkupfalse%
\ assms\ \isacommand{by}\isamarkupfalse%
\ auto\ \isanewline
\ \ \isacommand{then}\isamarkupfalse%
\ \isacommand{have}\isamarkupfalse%
\ p{\isadigit{3}}\ {\isacharcolon}{\kern0pt}\ {\isachardoublequoteopen}v\ {\isacharequal}{\kern0pt}\ split{\isacharparenleft}{\kern0pt}F{\isacharcomma}{\kern0pt}\ {\isacharless}{\kern0pt}x{\isacharcomma}{\kern0pt}\ y{\isachargreater}{\kern0pt}{\isacharparenright}{\kern0pt}{\isachardoublequoteclose}\ \isacommand{using}\isamarkupfalse%
\ p{\isadigit{2}}\ \isacommand{by}\isamarkupfalse%
\ auto\ \isanewline
\ \ \isacommand{then}\isamarkupfalse%
\ \isacommand{show}\isamarkupfalse%
\ {\isachardoublequoteopen}v\ {\isasymin}\ {\isacharbraceleft}{\kern0pt}F{\isacharparenleft}{\kern0pt}x{\isacharcomma}{\kern0pt}\ y{\isacharparenright}{\kern0pt}\ {\isachardot}{\kern0pt}\ {\isasymlangle}x{\isacharcomma}{\kern0pt}y{\isasymrangle}\ {\isasymin}\ C{\isacharbraceright}{\kern0pt}{\isachardoublequoteclose}\ \isanewline
\ \ \ \ \isacommand{apply}\isamarkupfalse%
\ simp\isanewline
\ \ \ \ \isacommand{apply}\isamarkupfalse%
\ {\isacharparenleft}{\kern0pt}rule{\isacharunderscore}{\kern0pt}tac\ x{\isacharequal}{\kern0pt}{\isachardoublequoteopen}{\isacharless}{\kern0pt}x{\isacharcomma}{\kern0pt}y{\isachargreater}{\kern0pt}{\isachardoublequoteclose}\ \isakeyword{in}\ bexI{\isacharparenright}{\kern0pt}\isanewline
\ \ \ \ \isacommand{using}\isamarkupfalse%
\ p{\isadigit{1}}\ p{\isadigit{2}}\ p{\isadigit{3}}\ \isacommand{by}\isamarkupfalse%
\ auto\ \isanewline
\isacommand{qed}\isamarkupfalse%
%
\endisatagproof
{\isafoldproof}%
%
\isadelimproof
\isanewline
%
\endisadelimproof
\isanewline
\isanewline
\isacommand{lemma}\isamarkupfalse%
\ OUN{\isacharunderscore}{\kern0pt}UN{\isacharunderscore}{\kern0pt}equals\ {\isacharcolon}{\kern0pt}\ {\isachardoublequoteopen}Ord{\isacharparenleft}{\kern0pt}a{\isacharparenright}{\kern0pt}\ {\isasymLongrightarrow}\ {\isacharparenleft}{\kern0pt}{\isasymUnion}\ b\ {\isacharless}{\kern0pt}\ a{\isachardot}{\kern0pt}\ F{\isacharparenleft}{\kern0pt}b{\isacharparenright}{\kern0pt}{\isacharparenright}{\kern0pt}\ {\isacharequal}{\kern0pt}\ {\isacharparenleft}{\kern0pt}{\isasymUnion}\ b\ {\isasymin}\ a\ {\isachardot}{\kern0pt}\ F{\isacharparenleft}{\kern0pt}b{\isacharparenright}{\kern0pt}{\isacharparenright}{\kern0pt}{\isachardoublequoteclose}\ \isanewline
%
\isadelimproof
\ \ %
\endisadelimproof
%
\isatagproof
\isacommand{apply}\isamarkupfalse%
\ {\isacharparenleft}{\kern0pt}rule\ equality{\isacharunderscore}{\kern0pt}iffI{\isacharsemicolon}{\kern0pt}\ rule\ iffI{\isacharparenright}{\kern0pt}\isanewline
\isacommand{proof}\isamarkupfalse%
\ {\isacharminus}{\kern0pt}\ \isanewline
\ \ \isacommand{fix}\isamarkupfalse%
\ x\ \isacommand{assume}\isamarkupfalse%
\ {\isachardoublequoteopen}x\ {\isasymin}\ {\isacharparenleft}{\kern0pt}{\isasymUnion}\ b\ {\isacharless}{\kern0pt}\ a{\isachardot}{\kern0pt}\ F{\isacharparenleft}{\kern0pt}b{\isacharparenright}{\kern0pt}{\isacharparenright}{\kern0pt}{\isachardoublequoteclose}\ \isanewline
\ \ \isacommand{then}\isamarkupfalse%
\ \isacommand{obtain}\isamarkupfalse%
\ b\ \isakeyword{where}\ bp\ {\isacharcolon}{\kern0pt}\ \ {\isachardoublequoteopen}b\ {\isacharless}{\kern0pt}\ a{\isachardoublequoteclose}\ {\isachardoublequoteopen}x\ {\isasymin}\ F{\isacharparenleft}{\kern0pt}b{\isacharparenright}{\kern0pt}{\isachardoublequoteclose}\ \isacommand{using}\isamarkupfalse%
\ OUN{\isacharunderscore}{\kern0pt}iff\ \isacommand{by}\isamarkupfalse%
\ auto\ \isanewline
\ \ \isacommand{then}\isamarkupfalse%
\ \isacommand{have}\isamarkupfalse%
\ {\isachardoublequoteopen}b\ {\isasymin}\ a{\isachardoublequoteclose}\ \isacommand{using}\isamarkupfalse%
\ ltD\ \isacommand{by}\isamarkupfalse%
\ auto\ \isanewline
\ \ \isacommand{then}\isamarkupfalse%
\ \isacommand{have}\isamarkupfalse%
\ {\isachardoublequoteopen}{\isasymexists}\ b\ {\isasymin}\ a{\isachardot}{\kern0pt}\ x\ {\isasymin}\ F{\isacharparenleft}{\kern0pt}b{\isacharparenright}{\kern0pt}{\isachardoublequoteclose}\ \isacommand{using}\isamarkupfalse%
\ bp\ \isacommand{by}\isamarkupfalse%
\ auto\ \isanewline
\ \ \isacommand{then}\isamarkupfalse%
\ \isacommand{show}\isamarkupfalse%
\ {\isachardoublequoteopen}x\ {\isasymin}\ {\isacharparenleft}{\kern0pt}{\isasymUnion}\ b\ {\isasymin}\ a\ {\isachardot}{\kern0pt}\ F{\isacharparenleft}{\kern0pt}b{\isacharparenright}{\kern0pt}{\isacharparenright}{\kern0pt}{\isachardoublequoteclose}\ \isacommand{using}\isamarkupfalse%
\ UN{\isacharunderscore}{\kern0pt}iff\ \isacommand{by}\isamarkupfalse%
\ auto\ \isanewline
\isacommand{next}\isamarkupfalse%
\ \isanewline
\ \ \isacommand{fix}\isamarkupfalse%
\ x\ \isacommand{assume}\isamarkupfalse%
\ assms\ {\isacharcolon}{\kern0pt}\ {\isachardoublequoteopen}x\ {\isasymin}\ {\isacharparenleft}{\kern0pt}{\isasymUnion}\ b\ {\isasymin}\ a\ {\isachardot}{\kern0pt}\ F{\isacharparenleft}{\kern0pt}b{\isacharparenright}{\kern0pt}{\isacharparenright}{\kern0pt}{\isachardoublequoteclose}\ {\isachardoublequoteopen}Ord{\isacharparenleft}{\kern0pt}a{\isacharparenright}{\kern0pt}{\isachardoublequoteclose}\isanewline
\ \ \isacommand{then}\isamarkupfalse%
\ \isacommand{obtain}\isamarkupfalse%
\ b\ \isakeyword{where}\ bp\ {\isacharcolon}{\kern0pt}\ \ {\isachardoublequoteopen}b\ {\isasymin}\ a{\isachardoublequoteclose}\ {\isachardoublequoteopen}x\ {\isasymin}\ F{\isacharparenleft}{\kern0pt}b{\isacharparenright}{\kern0pt}{\isachardoublequoteclose}\ \isacommand{using}\isamarkupfalse%
\ UN{\isacharunderscore}{\kern0pt}iff\ \isacommand{by}\isamarkupfalse%
\ auto\isanewline
\ \ \isacommand{then}\isamarkupfalse%
\ \isacommand{have}\isamarkupfalse%
\ {\isachardoublequoteopen}b\ {\isacharless}{\kern0pt}\ a{\isachardoublequoteclose}\ \isacommand{using}\isamarkupfalse%
\ {\isacartoucheopen}b\ {\isasymin}\ a{\isacartoucheclose}\ {\isacartoucheopen}Ord{\isacharparenleft}{\kern0pt}a{\isacharparenright}{\kern0pt}{\isacartoucheclose}\ ltI\ \isacommand{by}\isamarkupfalse%
\ auto\ \isanewline
\ \ \isacommand{then}\isamarkupfalse%
\ \isacommand{have}\isamarkupfalse%
\ {\isachardoublequoteopen}{\isasymexists}b\ {\isacharless}{\kern0pt}\ a{\isachardot}{\kern0pt}\ x\ {\isasymin}\ F{\isacharparenleft}{\kern0pt}b{\isacharparenright}{\kern0pt}{\isachardoublequoteclose}\ \isacommand{using}\isamarkupfalse%
\ bp\ \isacommand{by}\isamarkupfalse%
\ auto\ \isanewline
\ \ \isacommand{then}\isamarkupfalse%
\ \isacommand{show}\isamarkupfalse%
\ {\isachardoublequoteopen}x\ {\isasymin}\ {\isacharparenleft}{\kern0pt}{\isasymUnion}\ b\ {\isacharless}{\kern0pt}\ a{\isachardot}{\kern0pt}\ F{\isacharparenleft}{\kern0pt}b{\isacharparenright}{\kern0pt}{\isacharparenright}{\kern0pt}{\isachardoublequoteclose}\ \isacommand{by}\isamarkupfalse%
\ auto\ \isanewline
\isacommand{qed}\isamarkupfalse%
%
\endisatagproof
{\isafoldproof}%
%
\isadelimproof
\isanewline
%
\endisadelimproof
\isanewline
\isacommand{lemma}\isamarkupfalse%
\ lt{\isacharunderscore}{\kern0pt}le{\isacharunderscore}{\kern0pt}lt\ {\isacharcolon}{\kern0pt}\ {\isachardoublequoteopen}a\ {\isacharless}{\kern0pt}\ b\ {\isasymLongrightarrow}\ b\ {\isasymle}\ c\ {\isasymLongrightarrow}\ a\ {\isacharless}{\kern0pt}\ c{\isachardoublequoteclose}\ \isanewline
%
\isadelimproof
\ \ %
\endisadelimproof
%
\isatagproof
\isacommand{apply}\isamarkupfalse%
\ {\isacharparenleft}{\kern0pt}rule{\isacharunderscore}{\kern0pt}tac\ i{\isacharequal}{\kern0pt}b\ \isakeyword{and}\ j{\isacharequal}{\kern0pt}c\ \isakeyword{in}\ leE{\isacharparenright}{\kern0pt}\isanewline
\ \ \isacommand{apply}\isamarkupfalse%
\ assumption\ \isanewline
\ \ \isacommand{using}\isamarkupfalse%
\ lt{\isacharunderscore}{\kern0pt}trans\ \isacommand{by}\isamarkupfalse%
\ auto%
\endisatagproof
{\isafoldproof}%
%
\isadelimproof
\isanewline
%
\endisadelimproof
\isanewline
\isacommand{lemma}\isamarkupfalse%
\ le{\isacharunderscore}{\kern0pt}lt{\isacharunderscore}{\kern0pt}lt\ {\isacharcolon}{\kern0pt}\ {\isachardoublequoteopen}a\ {\isasymle}\ b\ {\isasymLongrightarrow}\ b\ {\isacharless}{\kern0pt}\ c\ {\isasymLongrightarrow}\ a\ {\isacharless}{\kern0pt}\ c{\isachardoublequoteclose}\ \ \isanewline
%
\isadelimproof
\ \ %
\endisadelimproof
%
\isatagproof
\isacommand{apply}\isamarkupfalse%
\ {\isacharparenleft}{\kern0pt}rule{\isacharunderscore}{\kern0pt}tac\ i{\isacharequal}{\kern0pt}a\ \isakeyword{and}\ j{\isacharequal}{\kern0pt}b\ \isakeyword{in}\ leE{\isacharparenright}{\kern0pt}\isanewline
\ \ \isacommand{apply}\isamarkupfalse%
\ assumption\ \isanewline
\ \ \isacommand{using}\isamarkupfalse%
\ lt{\isacharunderscore}{\kern0pt}trans\ \isacommand{by}\isamarkupfalse%
\ auto%
\endisatagproof
{\isafoldproof}%
%
\isadelimproof
\isanewline
%
\endisadelimproof
\isanewline
\isacommand{lemma}\isamarkupfalse%
\ le{\isacharunderscore}{\kern0pt}succ{\isacharunderscore}{\kern0pt}le\ {\isacharcolon}{\kern0pt}\ {\isachardoublequoteopen}Ord{\isacharparenleft}{\kern0pt}b{\isacharparenright}{\kern0pt}\ {\isasymLongrightarrow}\ a\ {\isasymle}\ b\ {\isasymLongrightarrow}\ a\ {\isasymle}\ succ{\isacharparenleft}{\kern0pt}b{\isacharparenright}{\kern0pt}{\isachardoublequoteclose}\ \isanewline
%
\isadelimproof
\ \ %
\endisadelimproof
%
\isatagproof
\isacommand{apply}\isamarkupfalse%
\ {\isacharparenleft}{\kern0pt}rule{\isacharunderscore}{\kern0pt}tac\ j{\isacharequal}{\kern0pt}b\ \isakeyword{in}\ le{\isacharunderscore}{\kern0pt}trans{\isacharparenright}{\kern0pt}\ \isanewline
\ \ \isacommand{apply}\isamarkupfalse%
\ simp\isanewline
\ \ \isacommand{apply}\isamarkupfalse%
\ {\isacharparenleft}{\kern0pt}rule{\isacharunderscore}{\kern0pt}tac\ j{\isacharequal}{\kern0pt}{\isachardoublequoteopen}succ{\isacharparenleft}{\kern0pt}b{\isacharparenright}{\kern0pt}{\isachardoublequoteclose}\ \isakeyword{in}\ lt{\isacharunderscore}{\kern0pt}trans{\isacharparenright}{\kern0pt}\ \isanewline
\ \ \isacommand{using}\isamarkupfalse%
\ le{\isacharunderscore}{\kern0pt}refl\ \isacommand{by}\isamarkupfalse%
\ auto%
\endisatagproof
{\isafoldproof}%
%
\isadelimproof
\isanewline
%
\endisadelimproof
\isanewline
\isacommand{lemma}\isamarkupfalse%
\ lt{\isacharunderscore}{\kern0pt}succ{\isacharunderscore}{\kern0pt}lt\ {\isacharcolon}{\kern0pt}\ {\isachardoublequoteopen}Ord{\isacharparenleft}{\kern0pt}b{\isacharparenright}{\kern0pt}\ {\isasymLongrightarrow}\ a\ {\isacharless}{\kern0pt}\ b\ {\isasymLongrightarrow}\ a\ {\isacharless}{\kern0pt}\ succ{\isacharparenleft}{\kern0pt}b{\isacharparenright}{\kern0pt}{\isachardoublequoteclose}\ \isanewline
%
\isadelimproof
\ \ %
\endisadelimproof
%
\isatagproof
\isacommand{apply}\isamarkupfalse%
\ {\isacharparenleft}{\kern0pt}rule{\isacharunderscore}{\kern0pt}tac\ j{\isacharequal}{\kern0pt}b\ \isakeyword{in}\ lt{\isacharunderscore}{\kern0pt}trans{\isacharparenright}{\kern0pt}\ \isanewline
\ \ \isacommand{using}\isamarkupfalse%
\ le{\isacharunderscore}{\kern0pt}refl\ \isacommand{by}\isamarkupfalse%
\ auto%
\endisatagproof
{\isafoldproof}%
%
\isadelimproof
\isanewline
%
\endisadelimproof
\ \ \ \ \isanewline
\isanewline
\isacommand{lemma}\isamarkupfalse%
\ ex{\isacharunderscore}{\kern0pt}larger{\isacharunderscore}{\kern0pt}limit\ {\isacharcolon}{\kern0pt}\ {\isachardoublequoteopen}Ord{\isacharparenleft}{\kern0pt}a{\isacharparenright}{\kern0pt}\ {\isasymLongrightarrow}\ {\isasymexists}b{\isachardot}{\kern0pt}\ a\ {\isacharless}{\kern0pt}\ b\ {\isasymand}\ Limit{\isacharparenleft}{\kern0pt}b{\isacharparenright}{\kern0pt}{\isachardoublequoteclose}\isanewline
%
\isadelimproof
%
\endisadelimproof
%
\isatagproof
\isacommand{proof}\isamarkupfalse%
\ {\isacharminus}{\kern0pt}\isanewline
\ \ \isacommand{assume}\isamarkupfalse%
\ assm\ {\isacharcolon}{\kern0pt}\ {\isachardoublequoteopen}Ord{\isacharparenleft}{\kern0pt}a{\isacharparenright}{\kern0pt}{\isachardoublequoteclose}\isanewline
\ \ \isacommand{have}\isamarkupfalse%
\ p{\isadigit{1}}\ {\isacharcolon}{\kern0pt}\ {\isachardoublequoteopen}Card{\isacharparenleft}{\kern0pt}csucc{\isacharparenleft}{\kern0pt}a{\isacharparenright}{\kern0pt}{\isacharparenright}{\kern0pt}{\isachardoublequoteclose}\ {\isachardoublequoteopen}a\ {\isacharless}{\kern0pt}\ csucc{\isacharparenleft}{\kern0pt}a{\isacharparenright}{\kern0pt}{\isachardoublequoteclose}\ \isacommand{using}\isamarkupfalse%
\ assm\ csucc{\isacharunderscore}{\kern0pt}basic\ \isacommand{by}\isamarkupfalse%
\ auto\isanewline
\ \ \isacommand{then}\isamarkupfalse%
\ \isacommand{have}\isamarkupfalse%
\ p{\isadigit{2}}\ {\isacharcolon}{\kern0pt}\ {\isachardoublequoteopen}csucc{\isacharparenleft}{\kern0pt}a{\isacharparenright}{\kern0pt}\ {\isasymle}\ {\isacharparenleft}{\kern0pt}csucc{\isacharparenleft}{\kern0pt}a{\isacharparenright}{\kern0pt}\ {\isasymoplus}\ nat{\isacharparenright}{\kern0pt}{\isachardoublequoteclose}\ \isacommand{using}\isamarkupfalse%
\ cadd{\isacharunderscore}{\kern0pt}le{\isacharunderscore}{\kern0pt}self\ \isacommand{by}\isamarkupfalse%
\ auto\isanewline
\ \ \isacommand{have}\isamarkupfalse%
\ p{\isadigit{3}}\ {\isacharcolon}{\kern0pt}\ {\isachardoublequoteopen}Ord{\isacharparenleft}{\kern0pt}csucc{\isacharparenleft}{\kern0pt}a{\isacharparenright}{\kern0pt}{\isacharparenright}{\kern0pt}{\isachardoublequoteclose}\ \isacommand{using}\isamarkupfalse%
\ p{\isadigit{1}}\ Card{\isacharunderscore}{\kern0pt}is{\isacharunderscore}{\kern0pt}Ord\ \isacommand{by}\isamarkupfalse%
\ auto\isanewline
\ \ \isacommand{have}\isamarkupfalse%
\ p{\isadigit{4}}\ {\isacharcolon}{\kern0pt}\ {\isachardoublequoteopen}Ord{\isacharparenleft}{\kern0pt}nat{\isacharparenright}{\kern0pt}{\isachardoublequoteclose}\ \isacommand{by}\isamarkupfalse%
\ auto\ \isanewline
\ \ \isacommand{have}\isamarkupfalse%
\ p{\isadigit{5}}\ {\isacharcolon}{\kern0pt}\ {\isachardoublequoteopen}Card{\isacharparenleft}{\kern0pt}nat{\isacharparenright}{\kern0pt}{\isachardoublequoteclose}\ \isacommand{using}\isamarkupfalse%
\ Ord{\isacharunderscore}{\kern0pt}nat{\isacharunderscore}{\kern0pt}subset{\isacharunderscore}{\kern0pt}into{\isacharunderscore}{\kern0pt}Card\ \isacommand{by}\isamarkupfalse%
\ auto\ \isanewline
\ \ \isacommand{then}\isamarkupfalse%
\ \isacommand{have}\isamarkupfalse%
\ p{\isadigit{6}}\ {\isacharcolon}{\kern0pt}\ {\isachardoublequoteopen}nat\ {\isasymle}\ {\isacharparenleft}{\kern0pt}nat\ {\isasymoplus}\ csucc{\isacharparenleft}{\kern0pt}a{\isacharparenright}{\kern0pt}{\isacharparenright}{\kern0pt}{\isachardoublequoteclose}\ \isacommand{using}\isamarkupfalse%
\ p{\isadigit{3}}\ cadd{\isacharunderscore}{\kern0pt}le{\isacharunderscore}{\kern0pt}self\ \isacommand{by}\isamarkupfalse%
\ auto\isanewline
\ \ \isacommand{then}\isamarkupfalse%
\ \isacommand{have}\isamarkupfalse%
\ p{\isadigit{7}}\ {\isacharcolon}{\kern0pt}\ {\isachardoublequoteopen}nat\ {\isasymle}\ {\isacharparenleft}{\kern0pt}csucc{\isacharparenleft}{\kern0pt}a{\isacharparenright}{\kern0pt}\ {\isasymoplus}\ nat{\isacharparenright}{\kern0pt}{\isachardoublequoteclose}\ \isacommand{using}\isamarkupfalse%
\ cadd{\isacharunderscore}{\kern0pt}commute\ \isacommand{by}\isamarkupfalse%
\ auto\ \isanewline
\ \ \isacommand{have}\isamarkupfalse%
\ p{\isadigit{8}}\ {\isacharcolon}{\kern0pt}\ {\isachardoublequoteopen}Card{\isacharparenleft}{\kern0pt}csucc{\isacharparenleft}{\kern0pt}a{\isacharparenright}{\kern0pt}\ {\isasymoplus}\ nat{\isacharparenright}{\kern0pt}{\isachardoublequoteclose}\ {\isachardoublequoteopen}Ord{\isacharparenleft}{\kern0pt}csucc{\isacharparenleft}{\kern0pt}a{\isacharparenright}{\kern0pt}\ {\isasymoplus}\ nat{\isacharparenright}{\kern0pt}{\isachardoublequoteclose}\ \isacommand{unfolding}\isamarkupfalse%
\ cadd{\isacharunderscore}{\kern0pt}def\ \isacommand{by}\isamarkupfalse%
\ auto\ \isanewline
\ \ \isacommand{then}\isamarkupfalse%
\ \isacommand{have}\isamarkupfalse%
\ p{\isadigit{9}}\ {\isacharcolon}{\kern0pt}\ {\isachardoublequoteopen}InfCard{\isacharparenleft}{\kern0pt}csucc{\isacharparenleft}{\kern0pt}a{\isacharparenright}{\kern0pt}\ {\isasymoplus}\ nat{\isacharparenright}{\kern0pt}{\isachardoublequoteclose}\ \isacommand{unfolding}\isamarkupfalse%
\ InfCard{\isacharunderscore}{\kern0pt}def\ \isanewline
\ \ \ \ \isacommand{using}\isamarkupfalse%
\ p{\isadigit{7}}\ \isacommand{by}\isamarkupfalse%
\ auto\ \isanewline
\ \ \isacommand{then}\isamarkupfalse%
\ \isacommand{have}\isamarkupfalse%
\ p{\isadigit{1}}{\isadigit{0}}\ {\isacharcolon}{\kern0pt}\ {\isachardoublequoteopen}Limit{\isacharparenleft}{\kern0pt}csucc{\isacharparenleft}{\kern0pt}a{\isacharparenright}{\kern0pt}\ {\isasymoplus}\ nat{\isacharparenright}{\kern0pt}{\isachardoublequoteclose}\ \isacommand{using}\isamarkupfalse%
\ InfCard{\isacharunderscore}{\kern0pt}is{\isacharunderscore}{\kern0pt}Limit\ \isacommand{by}\isamarkupfalse%
\ auto\ \isanewline
\ \ \isacommand{have}\isamarkupfalse%
\ p{\isadigit{1}}{\isadigit{1}}\ {\isacharcolon}{\kern0pt}\ {\isachardoublequoteopen}a\ {\isacharless}{\kern0pt}\ csucc{\isacharparenleft}{\kern0pt}a{\isacharparenright}{\kern0pt}\ {\isasymoplus}\ nat{\isachardoublequoteclose}\ \isanewline
\ \ \ \ \isacommand{apply}\isamarkupfalse%
\ {\isacharparenleft}{\kern0pt}rule{\isacharunderscore}{\kern0pt}tac\ P{\isacharequal}{\kern0pt}{\isachardoublequoteopen}csucc{\isacharparenleft}{\kern0pt}a{\isacharparenright}{\kern0pt}\ {\isacharless}{\kern0pt}\ {\isacharparenleft}{\kern0pt}csucc{\isacharparenleft}{\kern0pt}a{\isacharparenright}{\kern0pt}\ {\isasymoplus}\ nat{\isacharparenright}{\kern0pt}{\isachardoublequoteclose}\ \isakeyword{and}\ Q\ {\isacharequal}{\kern0pt}\ {\isachardoublequoteopen}csucc{\isacharparenleft}{\kern0pt}a{\isacharparenright}{\kern0pt}\ {\isacharequal}{\kern0pt}\ {\isacharparenleft}{\kern0pt}csucc{\isacharparenleft}{\kern0pt}a{\isacharparenright}{\kern0pt}\ {\isasymoplus}\ nat{\isacharparenright}{\kern0pt}{\isachardoublequoteclose}\ \isakeyword{in}\ disjE{\isacharparenright}{\kern0pt}\ \isanewline
\ \ \isacommand{proof}\isamarkupfalse%
\ {\isacharminus}{\kern0pt}\ \isanewline
\ \ \ \ \isacommand{show}\isamarkupfalse%
\ {\isachardoublequoteopen}csucc{\isacharparenleft}{\kern0pt}a{\isacharparenright}{\kern0pt}\ {\isacharless}{\kern0pt}\ csucc{\isacharparenleft}{\kern0pt}a{\isacharparenright}{\kern0pt}\ {\isasymoplus}\ nat\ {\isasymor}\ csucc{\isacharparenleft}{\kern0pt}a{\isacharparenright}{\kern0pt}\ {\isacharequal}{\kern0pt}\ csucc{\isacharparenleft}{\kern0pt}a{\isacharparenright}{\kern0pt}\ {\isasymoplus}\ nat{\isachardoublequoteclose}\ \isanewline
\ \ \ \ \ \ \isacommand{using}\isamarkupfalse%
\ le{\isacharunderscore}{\kern0pt}iff\ p{\isadigit{2}}\ \isacommand{by}\isamarkupfalse%
\ auto\ \isanewline
\ \ \isacommand{next}\isamarkupfalse%
\ \isanewline
\ \ \ \ \isacommand{assume}\isamarkupfalse%
\ {\isachardoublequoteopen}csucc{\isacharparenleft}{\kern0pt}a{\isacharparenright}{\kern0pt}\ {\isacharless}{\kern0pt}\ csucc{\isacharparenleft}{\kern0pt}a{\isacharparenright}{\kern0pt}\ {\isasymoplus}\ nat{\isachardoublequoteclose}\ \isanewline
\ \ \ \ \isacommand{then}\isamarkupfalse%
\ \isacommand{show}\isamarkupfalse%
\ {\isachardoublequoteopen}a\ {\isacharless}{\kern0pt}\ csucc{\isacharparenleft}{\kern0pt}a{\isacharparenright}{\kern0pt}\ {\isasymoplus}\ nat{\isachardoublequoteclose}\ \isanewline
\ \ \ \ \ \ \isacommand{using}\isamarkupfalse%
\ lt{\isacharunderscore}{\kern0pt}trans\ p{\isadigit{1}}\ \isacommand{by}\isamarkupfalse%
\ auto\ \isanewline
\ \ \isacommand{next}\isamarkupfalse%
\ \isanewline
\ \ \ \ \isacommand{assume}\isamarkupfalse%
\ {\isachardoublequoteopen}csucc{\isacharparenleft}{\kern0pt}a{\isacharparenright}{\kern0pt}\ {\isacharequal}{\kern0pt}\ csucc{\isacharparenleft}{\kern0pt}a{\isacharparenright}{\kern0pt}\ {\isasymoplus}\ nat{\isachardoublequoteclose}\ \isanewline
\ \ \ \ \isacommand{then}\isamarkupfalse%
\ \isacommand{show}\isamarkupfalse%
\ {\isachardoublequoteopen}a\ {\isacharless}{\kern0pt}\ csucc{\isacharparenleft}{\kern0pt}a{\isacharparenright}{\kern0pt}\ {\isasymoplus}\ nat{\isachardoublequoteclose}\ \isacommand{using}\isamarkupfalse%
\ p{\isadigit{1}}\ \isacommand{by}\isamarkupfalse%
\ auto\ \isanewline
\ \ \isacommand{qed}\isamarkupfalse%
\isanewline
\ \ \isacommand{then}\isamarkupfalse%
\ \isacommand{show}\isamarkupfalse%
\ {\isacharquery}{\kern0pt}thesis\ \isacommand{using}\isamarkupfalse%
\ p{\isadigit{1}}{\isadigit{0}}\ \isacommand{by}\isamarkupfalse%
\ auto\ \isanewline
\isacommand{qed}\isamarkupfalse%
%
\endisatagproof
{\isafoldproof}%
%
\isadelimproof
\isanewline
%
\endisadelimproof
\isanewline
\isanewline
\isacommand{lemma}\isamarkupfalse%
\ UN{\isacharunderscore}{\kern0pt}RepFunI\ {\isacharcolon}{\kern0pt}\ {\isachardoublequoteopen}a\ {\isasymin}\ b\ {\isasymLongrightarrow}\ x\ {\isasymin}\ F{\isacharparenleft}{\kern0pt}a{\isacharparenright}{\kern0pt}\ {\isasymLongrightarrow}\ x\ {\isasymin}\ {\isacharparenleft}{\kern0pt}{\isasymUnion}a\ {\isasymin}\ b{\isachardot}{\kern0pt}\ F{\isacharparenleft}{\kern0pt}a{\isacharparenright}{\kern0pt}{\isacharparenright}{\kern0pt}{\isachardoublequoteclose}\ \isanewline
%
\isadelimproof
\ \ %
\endisadelimproof
%
\isatagproof
\isacommand{by}\isamarkupfalse%
\ auto%
\endisatagproof
{\isafoldproof}%
%
\isadelimproof
\isanewline
%
\endisadelimproof
\isacommand{lemma}\isamarkupfalse%
\ OUN{\isacharunderscore}{\kern0pt}RepFunI\ {\isacharcolon}{\kern0pt}\ {\isachardoublequoteopen}a\ {\isacharless}{\kern0pt}\ b\ {\isasymLongrightarrow}\ x\ {\isasymin}\ F{\isacharparenleft}{\kern0pt}a{\isacharparenright}{\kern0pt}\ {\isasymLongrightarrow}\ x\ {\isasymin}\ {\isacharparenleft}{\kern0pt}{\isasymUnion}a\ {\isacharless}{\kern0pt}\ b{\isachardot}{\kern0pt}\ F{\isacharparenleft}{\kern0pt}a{\isacharparenright}{\kern0pt}{\isacharparenright}{\kern0pt}{\isachardoublequoteclose}\isanewline
%
\isadelimproof
\ \ %
\endisadelimproof
%
\isatagproof
\isacommand{by}\isamarkupfalse%
\ auto%
\endisatagproof
{\isafoldproof}%
%
\isadelimproof
\ \isanewline
%
\endisadelimproof
\isanewline
\isacommand{lemma}\isamarkupfalse%
\ RepFun{\isacharunderscore}{\kern0pt}eq\ {\isacharcolon}{\kern0pt}\ {\isachardoublequoteopen}{\isasymforall}x\ {\isasymin}\ A{\isachardot}{\kern0pt}\ {\isacharparenleft}{\kern0pt}{\isasymexists}y\ {\isasymin}\ B{\isachardot}{\kern0pt}\ F{\isacharparenleft}{\kern0pt}x{\isacharparenright}{\kern0pt}\ {\isacharequal}{\kern0pt}\ G{\isacharparenleft}{\kern0pt}y{\isacharparenright}{\kern0pt}{\isacharparenright}{\kern0pt}\isanewline
\ \ \ \ \ \ \ \ \ \ \ \ \ \ \ \ \ \ \ {\isasymLongrightarrow}\ {\isasymforall}y\ {\isasymin}\ B{\isachardot}{\kern0pt}\ {\isacharparenleft}{\kern0pt}{\isasymexists}x\ {\isasymin}\ A{\isachardot}{\kern0pt}\ G{\isacharparenleft}{\kern0pt}y{\isacharparenright}{\kern0pt}\ {\isacharequal}{\kern0pt}\ F{\isacharparenleft}{\kern0pt}x{\isacharparenright}{\kern0pt}{\isacharparenright}{\kern0pt}\isanewline
\ \ \ \ \ \ \ \ \ \ \ \ \ \ \ \ \ \ \ {\isasymLongrightarrow}\ {\isacharbraceleft}{\kern0pt}\ F{\isacharparenleft}{\kern0pt}x{\isacharparenright}{\kern0pt}\ {\isachardot}{\kern0pt}\ x\ {\isasymin}\ A\ {\isacharbraceright}{\kern0pt}\ {\isacharequal}{\kern0pt}\ {\isacharbraceleft}{\kern0pt}\ G{\isacharparenleft}{\kern0pt}y{\isacharparenright}{\kern0pt}\ {\isachardot}{\kern0pt}\ y\ {\isasymin}\ B{\isacharbraceright}{\kern0pt}\ {\isachardoublequoteclose}\ \isanewline
%
\isadelimproof
\ \ %
\endisadelimproof
%
\isatagproof
\isacommand{apply}\isamarkupfalse%
\ {\isacharparenleft}{\kern0pt}rule{\isacharunderscore}{\kern0pt}tac\ equality{\isacharunderscore}{\kern0pt}iffI{\isacharsemicolon}{\kern0pt}\ rule\ iffI{\isacharparenright}{\kern0pt}\isanewline
\isacommand{proof}\isamarkupfalse%
\ {\isacharminus}{\kern0pt}\ \isanewline
\ \ \isacommand{fix}\isamarkupfalse%
\ v\ \isacommand{assume}\isamarkupfalse%
\ assms\ {\isacharcolon}{\kern0pt}\ {\isachardoublequoteopen}v\ {\isasymin}\ {\isacharbraceleft}{\kern0pt}F{\isacharparenleft}{\kern0pt}x{\isacharparenright}{\kern0pt}\ {\isachardot}{\kern0pt}\ x\ {\isasymin}\ A{\isacharbraceright}{\kern0pt}{\isachardoublequoteclose}\ {\isachardoublequoteopen}{\isasymforall}x{\isasymin}A{\isachardot}{\kern0pt}\ {\isasymexists}y{\isasymin}B{\isachardot}{\kern0pt}\ F{\isacharparenleft}{\kern0pt}x{\isacharparenright}{\kern0pt}\ {\isacharequal}{\kern0pt}\ G{\isacharparenleft}{\kern0pt}y{\isacharparenright}{\kern0pt}{\isachardoublequoteclose}\ \isanewline
\ \ \isacommand{then}\isamarkupfalse%
\ \isacommand{obtain}\isamarkupfalse%
\ x\ \isakeyword{where}\ vp{\isacharcolon}{\kern0pt}\ {\isachardoublequoteopen}v\ {\isacharequal}{\kern0pt}\ F{\isacharparenleft}{\kern0pt}x{\isacharparenright}{\kern0pt}{\isachardoublequoteclose}\ {\isachardoublequoteopen}x\ {\isasymin}\ A{\isachardoublequoteclose}\ \isacommand{by}\isamarkupfalse%
\ auto\ \isanewline
\ \ \isacommand{then}\isamarkupfalse%
\ \isacommand{obtain}\isamarkupfalse%
\ y\ \isakeyword{where}\ yp{\isacharcolon}{\kern0pt}\ {\isachardoublequoteopen}F{\isacharparenleft}{\kern0pt}x{\isacharparenright}{\kern0pt}\ {\isacharequal}{\kern0pt}\ G{\isacharparenleft}{\kern0pt}y{\isacharparenright}{\kern0pt}{\isachardoublequoteclose}\ {\isachardoublequoteopen}y\ {\isasymin}\ B{\isachardoublequoteclose}\ \isacommand{using}\isamarkupfalse%
\ assms\ \isacommand{by}\isamarkupfalse%
\ auto\isanewline
\ \ \isacommand{then}\isamarkupfalse%
\ \isacommand{have}\isamarkupfalse%
\ {\isachardoublequoteopen}v\ {\isacharequal}{\kern0pt}\ G{\isacharparenleft}{\kern0pt}y{\isacharparenright}{\kern0pt}{\isachardoublequoteclose}\ \isacommand{using}\isamarkupfalse%
\ vp\ \isacommand{by}\isamarkupfalse%
\ auto\ \isanewline
\ \ \isacommand{then}\isamarkupfalse%
\ \isacommand{show}\isamarkupfalse%
\ {\isachardoublequoteopen}v\ {\isasymin}\ {\isacharbraceleft}{\kern0pt}G{\isacharparenleft}{\kern0pt}y{\isacharparenright}{\kern0pt}\ {\isachardot}{\kern0pt}\ y\ {\isasymin}\ B{\isacharbraceright}{\kern0pt}{\isachardoublequoteclose}\ \isacommand{using}\isamarkupfalse%
\ yp\ \isacommand{by}\isamarkupfalse%
\ auto\ \isanewline
\isacommand{next}\isamarkupfalse%
\ \ \ \ \ \ \ \ \ \ \ \ \ \ \ \ \ \ \isanewline
\ \ \isacommand{fix}\isamarkupfalse%
\ v\ \isacommand{assume}\isamarkupfalse%
\ assms\ {\isacharcolon}{\kern0pt}\ {\isachardoublequoteopen}v\ {\isasymin}\ {\isacharbraceleft}{\kern0pt}G{\isacharparenleft}{\kern0pt}x{\isacharparenright}{\kern0pt}\ {\isachardot}{\kern0pt}\ x\ {\isasymin}\ B{\isacharbraceright}{\kern0pt}{\isachardoublequoteclose}\ {\isachardoublequoteopen}{\isasymforall}x{\isasymin}B{\isachardot}{\kern0pt}\ {\isasymexists}y{\isasymin}A{\isachardot}{\kern0pt}\ G{\isacharparenleft}{\kern0pt}x{\isacharparenright}{\kern0pt}\ {\isacharequal}{\kern0pt}\ F{\isacharparenleft}{\kern0pt}y{\isacharparenright}{\kern0pt}{\isachardoublequoteclose}\ \isanewline
\ \ \isacommand{then}\isamarkupfalse%
\ \isacommand{obtain}\isamarkupfalse%
\ x\ \isakeyword{where}\ vp{\isacharcolon}{\kern0pt}\ {\isachardoublequoteopen}v\ {\isacharequal}{\kern0pt}\ G{\isacharparenleft}{\kern0pt}x{\isacharparenright}{\kern0pt}{\isachardoublequoteclose}\ {\isachardoublequoteopen}x\ {\isasymin}\ B{\isachardoublequoteclose}\ \isacommand{by}\isamarkupfalse%
\ auto\ \isanewline
\ \ \isacommand{then}\isamarkupfalse%
\ \isacommand{obtain}\isamarkupfalse%
\ y\ \isakeyword{where}\ yp{\isacharcolon}{\kern0pt}\ {\isachardoublequoteopen}G{\isacharparenleft}{\kern0pt}x{\isacharparenright}{\kern0pt}\ {\isacharequal}{\kern0pt}\ F{\isacharparenleft}{\kern0pt}y{\isacharparenright}{\kern0pt}{\isachardoublequoteclose}\ {\isachardoublequoteopen}y\ {\isasymin}\ A{\isachardoublequoteclose}\ \isacommand{using}\isamarkupfalse%
\ assms\ \isacommand{by}\isamarkupfalse%
\ auto\isanewline
\ \ \isacommand{then}\isamarkupfalse%
\ \isacommand{have}\isamarkupfalse%
\ {\isachardoublequoteopen}v\ {\isacharequal}{\kern0pt}\ F{\isacharparenleft}{\kern0pt}y{\isacharparenright}{\kern0pt}{\isachardoublequoteclose}\ \isacommand{using}\isamarkupfalse%
\ vp\ \isacommand{by}\isamarkupfalse%
\ auto\ \isanewline
\ \ \isacommand{then}\isamarkupfalse%
\ \isacommand{show}\isamarkupfalse%
\ {\isachardoublequoteopen}v\ {\isasymin}\ {\isacharbraceleft}{\kern0pt}F{\isacharparenleft}{\kern0pt}y{\isacharparenright}{\kern0pt}\ {\isachardot}{\kern0pt}\ y\ {\isasymin}\ A{\isacharbraceright}{\kern0pt}{\isachardoublequoteclose}\ \isacommand{using}\isamarkupfalse%
\ yp\ \isacommand{by}\isamarkupfalse%
\ auto\ \isanewline
\isacommand{qed}\isamarkupfalse%
%
\endisatagproof
{\isafoldproof}%
%
\isadelimproof
\isanewline
%
\endisadelimproof
\isanewline
\isanewline
\isacommand{lemma}\isamarkupfalse%
\ pair{\isacharunderscore}{\kern0pt}ball{\isacharunderscore}{\kern0pt}mp\ {\isacharcolon}{\kern0pt}\isanewline
\ \ {\isachardoublequoteopen}C\ {\isasymsubseteq}\ D\ {\isasymtimes}\ E\ {\isasymLongrightarrow}\ {\isasymforall}{\isacharless}{\kern0pt}d{\isacharcomma}{\kern0pt}\ e{\isachargreater}{\kern0pt}\ {\isasymin}\ C{\isachardot}{\kern0pt}\ P{\isacharparenleft}{\kern0pt}d{\isacharcomma}{\kern0pt}\ e{\isacharparenright}{\kern0pt}\isanewline
\ \ \ {\isasymLongrightarrow}\ {\isasymforall}d\ {\isasymin}\ D{\isachardot}{\kern0pt}\ {\isasymforall}e\ {\isasymin}\ E{\isachardot}{\kern0pt}\ P{\isacharparenleft}{\kern0pt}d{\isacharcomma}{\kern0pt}\ e{\isacharparenright}{\kern0pt}\ {\isasymlongrightarrow}\ Q{\isacharparenleft}{\kern0pt}d{\isacharcomma}{\kern0pt}\ e{\isacharparenright}{\kern0pt}\ {\isasymLongrightarrow}\ {\isasymforall}{\isacharless}{\kern0pt}d{\isacharcomma}{\kern0pt}\ e{\isachargreater}{\kern0pt}\ {\isasymin}\ C{\isachardot}{\kern0pt}\ Q{\isacharparenleft}{\kern0pt}d{\isacharcomma}{\kern0pt}\ e{\isacharparenright}{\kern0pt}{\isachardoublequoteclose}\ \isanewline
%
\isadelimproof
%
\endisadelimproof
%
\isatagproof
\isacommand{proof}\isamarkupfalse%
\ {\isacharparenleft}{\kern0pt}clarify{\isacharparenright}{\kern0pt}\isanewline
\ \ \isacommand{fix}\isamarkupfalse%
\ x\isanewline
\ \ \isacommand{assume}\isamarkupfalse%
\ assms{\isacharcolon}{\kern0pt}\ {\isachardoublequoteopen}{\isasymforall}{\isasymlangle}d{\isacharcomma}{\kern0pt}e{\isasymrangle}{\isasymin}C{\isachardot}{\kern0pt}\ P{\isacharparenleft}{\kern0pt}d{\isacharcomma}{\kern0pt}\ e{\isacharparenright}{\kern0pt}{\isachardoublequoteclose}\ {\isachardoublequoteopen}\ {\isasymforall}d{\isasymin}D{\isachardot}{\kern0pt}\ {\isasymforall}e{\isasymin}E{\isachardot}{\kern0pt}\ P{\isacharparenleft}{\kern0pt}d{\isacharcomma}{\kern0pt}\ e{\isacharparenright}{\kern0pt}\ {\isasymlongrightarrow}\ Q{\isacharparenleft}{\kern0pt}d{\isacharcomma}{\kern0pt}\ e{\isacharparenright}{\kern0pt}{\isachardoublequoteclose}\ {\isachardoublequoteopen}x\ {\isasymin}\ C{\isachardoublequoteclose}\ {\isachardoublequoteopen}C\ {\isasymsubseteq}\ D\ {\isasymtimes}\ E{\isachardoublequoteclose}\ \isanewline
\ \ \isacommand{then}\isamarkupfalse%
\ \isacommand{obtain}\isamarkupfalse%
\ d\ e\ \isakeyword{where}\ de\ {\isacharcolon}{\kern0pt}\ {\isachardoublequoteopen}x\ {\isacharequal}{\kern0pt}\ {\isacharless}{\kern0pt}d{\isacharcomma}{\kern0pt}\ e{\isachargreater}{\kern0pt}{\isachardoublequoteclose}\ {\isachardoublequoteopen}d\ {\isasymin}\ D{\isachardoublequoteclose}\ {\isachardoublequoteopen}e\ {\isasymin}\ E{\isachardoublequoteclose}\ \isacommand{by}\isamarkupfalse%
\ auto\isanewline
\ \ \isacommand{then}\isamarkupfalse%
\ \isacommand{have}\isamarkupfalse%
\ {\isachardoublequoteopen}P{\isacharparenleft}{\kern0pt}d{\isacharcomma}{\kern0pt}\ e{\isacharparenright}{\kern0pt}{\isachardoublequoteclose}\ \isacommand{using}\isamarkupfalse%
\ assms\ \isacommand{by}\isamarkupfalse%
\ auto\ \isanewline
\ \ \isacommand{then}\isamarkupfalse%
\ \isacommand{have}\isamarkupfalse%
\ {\isachardoublequoteopen}Q{\isacharparenleft}{\kern0pt}d{\isacharcomma}{\kern0pt}\ e{\isacharparenright}{\kern0pt}{\isachardoublequoteclose}\ \isacommand{using}\isamarkupfalse%
\ de\ assms\ \isacommand{by}\isamarkupfalse%
\ auto\ \isanewline
\ \ \isacommand{then}\isamarkupfalse%
\ \isacommand{show}\isamarkupfalse%
\ {\isachardoublequoteopen}{\isacharparenleft}{\kern0pt}{\isasymlambda}{\isasymlangle}d{\isacharcomma}{\kern0pt}e{\isasymrangle}{\isachardot}{\kern0pt}\ Q{\isacharparenleft}{\kern0pt}d{\isacharcomma}{\kern0pt}\ e{\isacharparenright}{\kern0pt}{\isacharparenright}{\kern0pt}{\isacharparenleft}{\kern0pt}x{\isacharparenright}{\kern0pt}{\isachardoublequoteclose}\ \isacommand{using}\isamarkupfalse%
\ de\ \isacommand{by}\isamarkupfalse%
\ auto\isanewline
\isacommand{qed}\isamarkupfalse%
%
\endisatagproof
{\isafoldproof}%
%
\isadelimproof
\isanewline
%
\endisadelimproof
\isanewline
\isacommand{lemma}\isamarkupfalse%
\ pair{\isacharunderscore}{\kern0pt}ball{\isacharunderscore}{\kern0pt}conj\ {\isacharcolon}{\kern0pt}\ \isanewline
\ \ {\isachardoublequoteopen}C\ {\isasymsubseteq}\ D\ {\isasymtimes}\ E\ {\isasymLongrightarrow}\ {\isasymforall}{\isacharless}{\kern0pt}d{\isacharcomma}{\kern0pt}\ e{\isachargreater}{\kern0pt}\ {\isasymin}\ C{\isachardot}{\kern0pt}\ P{\isacharparenleft}{\kern0pt}d{\isacharcomma}{\kern0pt}\ e{\isacharparenright}{\kern0pt}\ {\isasymLongrightarrow}\ {\isasymforall}{\isacharless}{\kern0pt}d{\isacharcomma}{\kern0pt}\ e{\isachargreater}{\kern0pt}\ {\isasymin}\ C{\isachardot}{\kern0pt}\ Q{\isacharparenleft}{\kern0pt}d{\isacharcomma}{\kern0pt}\ e{\isacharparenright}{\kern0pt}\ {\isasymLongrightarrow}\ {\isasymforall}{\isacharless}{\kern0pt}d{\isacharcomma}{\kern0pt}\ e{\isachargreater}{\kern0pt}\ {\isasymin}\ C{\isachardot}{\kern0pt}\ P{\isacharparenleft}{\kern0pt}d{\isacharcomma}{\kern0pt}\ e{\isacharparenright}{\kern0pt}\ {\isasymand}\ Q{\isacharparenleft}{\kern0pt}d{\isacharcomma}{\kern0pt}e{\isacharparenright}{\kern0pt}{\isachardoublequoteclose}\ \isanewline
%
\isadelimproof
%
\endisadelimproof
%
\isatagproof
\isacommand{proof}\isamarkupfalse%
\ {\isacharparenleft}{\kern0pt}clarify{\isacharparenright}{\kern0pt}\isanewline
\ \ \isacommand{fix}\isamarkupfalse%
\ x\isanewline
\ \ \isacommand{assume}\isamarkupfalse%
\ assms{\isacharcolon}{\kern0pt}\ {\isachardoublequoteopen}{\isasymforall}{\isasymlangle}d{\isacharcomma}{\kern0pt}e{\isasymrangle}{\isasymin}C{\isachardot}{\kern0pt}\ P{\isacharparenleft}{\kern0pt}d{\isacharcomma}{\kern0pt}\ e{\isacharparenright}{\kern0pt}{\isachardoublequoteclose}\ {\isachardoublequoteopen}{\isasymforall}{\isasymlangle}d{\isacharcomma}{\kern0pt}e{\isasymrangle}{\isasymin}C{\isachardot}{\kern0pt}\ Q{\isacharparenleft}{\kern0pt}d{\isacharcomma}{\kern0pt}\ e{\isacharparenright}{\kern0pt}{\isachardoublequoteclose}\ {\isachardoublequoteopen}x\ {\isasymin}\ C{\isachardoublequoteclose}\ {\isachardoublequoteopen}C\ {\isasymsubseteq}\ D\ {\isasymtimes}\ E{\isachardoublequoteclose}\ \isanewline
\ \ \isacommand{then}\isamarkupfalse%
\ \isacommand{obtain}\isamarkupfalse%
\ d\ e\ \isakeyword{where}\ de\ {\isacharcolon}{\kern0pt}\ {\isachardoublequoteopen}x\ {\isacharequal}{\kern0pt}\ {\isacharless}{\kern0pt}d{\isacharcomma}{\kern0pt}\ e{\isachargreater}{\kern0pt}{\isachardoublequoteclose}\ {\isachardoublequoteopen}d\ {\isasymin}\ D{\isachardoublequoteclose}\ {\isachardoublequoteopen}e\ {\isasymin}\ E{\isachardoublequoteclose}\ \isacommand{by}\isamarkupfalse%
\ auto\isanewline
\ \ \isacommand{then}\isamarkupfalse%
\ \isacommand{have}\isamarkupfalse%
\ p{\isadigit{1}}{\isacharcolon}{\kern0pt}{\isachardoublequoteopen}P{\isacharparenleft}{\kern0pt}d{\isacharcomma}{\kern0pt}\ e{\isacharparenright}{\kern0pt}{\isachardoublequoteclose}\ \isacommand{using}\isamarkupfalse%
\ assms\ \isacommand{by}\isamarkupfalse%
\ auto\ \isanewline
\ \ \isacommand{then}\isamarkupfalse%
\ \isacommand{have}\isamarkupfalse%
\ p{\isadigit{2}}{\isacharcolon}{\kern0pt}{\isachardoublequoteopen}Q{\isacharparenleft}{\kern0pt}d{\isacharcomma}{\kern0pt}\ e{\isacharparenright}{\kern0pt}{\isachardoublequoteclose}\ \isacommand{using}\isamarkupfalse%
\ de\ assms\ \isacommand{by}\isamarkupfalse%
\ auto\ \isanewline
\ \ \isacommand{then}\isamarkupfalse%
\ \isacommand{show}\isamarkupfalse%
\ {\isachardoublequoteopen}\ {\isacharparenleft}{\kern0pt}{\isasymlambda}{\isasymlangle}d{\isacharcomma}{\kern0pt}e{\isasymrangle}{\isachardot}{\kern0pt}\ P{\isacharparenleft}{\kern0pt}d{\isacharcomma}{\kern0pt}\ e{\isacharparenright}{\kern0pt}\ {\isasymand}\ Q{\isacharparenleft}{\kern0pt}d{\isacharcomma}{\kern0pt}\ e{\isacharparenright}{\kern0pt}{\isacharparenright}{\kern0pt}{\isacharparenleft}{\kern0pt}x{\isacharparenright}{\kern0pt}{\isachardoublequoteclose}\ \isacommand{using}\isamarkupfalse%
\ de\ p{\isadigit{1}}\ p{\isadigit{2}}\ \isacommand{by}\isamarkupfalse%
\ auto\isanewline
\isacommand{qed}\isamarkupfalse%
%
\endisatagproof
{\isafoldproof}%
%
\isadelimproof
\isanewline
%
\endisadelimproof
\isanewline
\isacommand{lemma}\isamarkupfalse%
\ inter{\isacharunderscore}{\kern0pt}eq\ {\isacharcolon}{\kern0pt}\ {\isachardoublequoteopen}A\ {\isacharequal}{\kern0pt}\ B\ {\isasymLongrightarrow}\ A\ {\isasyminter}\ C\ {\isacharequal}{\kern0pt}\ B\ {\isasyminter}\ C{\isachardoublequoteclose}%
\isadelimproof
\ %
\endisadelimproof
%
\isatagproof
\isacommand{by}\isamarkupfalse%
\ auto%
\endisatagproof
{\isafoldproof}%
%
\isadelimproof
%
\endisadelimproof
\ \isanewline
\isanewline
\isacommand{lemma}\isamarkupfalse%
\ iff{\isacharunderscore}{\kern0pt}flip\ {\isacharcolon}{\kern0pt}\ {\isachardoublequoteopen}A\ {\isasymlongleftrightarrow}\ B\ {\isasymLongrightarrow}\ B\ {\isasymlongleftrightarrow}\ A{\isachardoublequoteclose}%
\isadelimproof
\ %
\endisadelimproof
%
\isatagproof
\isacommand{by}\isamarkupfalse%
\ auto%
\endisatagproof
{\isafoldproof}%
%
\isadelimproof
%
\endisadelimproof
\ \isanewline
\isanewline
\isacommand{lemma}\isamarkupfalse%
\ empty{\isacharunderscore}{\kern0pt}or{\isacharunderscore}{\kern0pt}not\ {\isacharcolon}{\kern0pt}\ {\isachardoublequoteopen}{\isasymAnd}x{\isachardot}{\kern0pt}\ {\isacharparenleft}{\kern0pt}x\ {\isacharequal}{\kern0pt}\ {\isadigit{0}}\ {\isasymlongrightarrow}\ P{\isacharparenleft}{\kern0pt}x{\isacharparenright}{\kern0pt}{\isacharparenright}{\kern0pt}\ {\isasymLongrightarrow}\ {\isacharparenleft}{\kern0pt}x\ {\isasymnoteq}\ {\isadigit{0}}\ {\isasymlongrightarrow}\ P{\isacharparenleft}{\kern0pt}x{\isacharparenright}{\kern0pt}{\isacharparenright}{\kern0pt}\ {\isasymLongrightarrow}\ P{\isacharparenleft}{\kern0pt}x{\isacharparenright}{\kern0pt}{\isachardoublequoteclose}\isanewline
%
\isadelimproof
\ \ %
\endisadelimproof
%
\isatagproof
\isacommand{by}\isamarkupfalse%
\ auto%
\endisatagproof
{\isafoldproof}%
%
\isadelimproof
\isanewline
%
\endisadelimproof
\isanewline
\isacommand{lemma}\isamarkupfalse%
\ Ord{\isacharunderscore}{\kern0pt}nat{\isacharprime}{\kern0pt}\ {\isacharcolon}{\kern0pt}\ {\isachardoublequoteopen}m\ {\isasymin}\ nat\ {\isasymLongrightarrow}\ Ord{\isacharparenleft}{\kern0pt}m{\isacharparenright}{\kern0pt}{\isachardoublequoteclose}\ \isanewline
%
\isadelimproof
\ \ %
\endisadelimproof
%
\isatagproof
\isacommand{using}\isamarkupfalse%
\ lt{\isacharunderscore}{\kern0pt}Ord\ ltI\ Ord{\isacharunderscore}{\kern0pt}nat\ \isacommand{by}\isamarkupfalse%
\ auto%
\endisatagproof
{\isafoldproof}%
%
\isadelimproof
\isanewline
%
\endisadelimproof
\isanewline
\isacommand{lemma}\isamarkupfalse%
\ in{\isacharunderscore}{\kern0pt}dom{\isacharunderscore}{\kern0pt}or{\isacharunderscore}{\kern0pt}not\ {\isacharcolon}{\kern0pt}\ {\isachardoublequoteopen}{\isasymAnd}P\ F\ x{\isachardot}{\kern0pt}\ function{\isacharparenleft}{\kern0pt}F{\isacharparenright}{\kern0pt}\ {\isasymLongrightarrow}\ P{\isacharparenleft}{\kern0pt}{\isadigit{0}}{\isacharparenright}{\kern0pt}\ {\isasymLongrightarrow}\ {\isacharparenleft}{\kern0pt}x\ {\isasymin}\ domain{\isacharparenleft}{\kern0pt}F{\isacharparenright}{\kern0pt}\ {\isasymLongrightarrow}\ P{\isacharparenleft}{\kern0pt}F{\isacharbackquote}{\kern0pt}x{\isacharparenright}{\kern0pt}{\isacharparenright}{\kern0pt}\ {\isasymLongrightarrow}\ P{\isacharparenleft}{\kern0pt}F{\isacharbackquote}{\kern0pt}x{\isacharparenright}{\kern0pt}{\isachardoublequoteclose}\ \isanewline
%
\isadelimproof
%
\endisadelimproof
%
\isatagproof
\isacommand{proof}\isamarkupfalse%
\ {\isacharminus}{\kern0pt}\ \isanewline
\ \isacommand{fix}\isamarkupfalse%
\ P\ F\ x\ \isanewline
\ \isacommand{assume}\isamarkupfalse%
\ assm\ {\isacharcolon}{\kern0pt}\ {\isachardoublequoteopen}function{\isacharparenleft}{\kern0pt}F{\isacharparenright}{\kern0pt}{\isachardoublequoteclose}\ {\isachardoublequoteopen}P{\isacharparenleft}{\kern0pt}{\isadigit{0}}{\isacharparenright}{\kern0pt}{\isachardoublequoteclose}\ {\isachardoublequoteopen}{\isacharparenleft}{\kern0pt}x\ {\isasymin}\ domain{\isacharparenleft}{\kern0pt}F{\isacharparenright}{\kern0pt}\ {\isasymLongrightarrow}\ P{\isacharparenleft}{\kern0pt}F{\isacharbackquote}{\kern0pt}x{\isacharparenright}{\kern0pt}{\isacharparenright}{\kern0pt}{\isachardoublequoteclose}\isanewline
\ \isacommand{show}\isamarkupfalse%
\ {\isachardoublequoteopen}P{\isacharparenleft}{\kern0pt}F{\isacharbackquote}{\kern0pt}x{\isacharparenright}{\kern0pt}{\isachardoublequoteclose}\ \isanewline
\ \isacommand{proof}\isamarkupfalse%
\ {\isacharparenleft}{\kern0pt}cases\ {\isachardoublequoteopen}x\ {\isasymin}\ domain{\isacharparenleft}{\kern0pt}F{\isacharparenright}{\kern0pt}{\isachardoublequoteclose}{\isacharparenright}{\kern0pt}\isanewline
\ \ \ \isacommand{case}\isamarkupfalse%
\ True\isanewline
\ \ \ \isacommand{then}\isamarkupfalse%
\ \isacommand{show}\isamarkupfalse%
\ {\isacharquery}{\kern0pt}thesis\ \isacommand{using}\isamarkupfalse%
\ assm\ \isacommand{by}\isamarkupfalse%
\ auto\isanewline
\ \isacommand{next}\isamarkupfalse%
\isanewline
\ \ \ \isacommand{case}\isamarkupfalse%
\ False\isanewline
\ \ \ \isacommand{then}\isamarkupfalse%
\ \isacommand{have}\isamarkupfalse%
\ {\isachardoublequoteopen}F{\isacharbackquote}{\kern0pt}x\ {\isacharequal}{\kern0pt}\ {\isadigit{0}}{\isachardoublequoteclose}\ \isacommand{unfolding}\isamarkupfalse%
\ apply{\isacharunderscore}{\kern0pt}def\ ZF{\isacharunderscore}{\kern0pt}Base{\isachardot}{\kern0pt}image{\isacharunderscore}{\kern0pt}def\ domain{\isacharunderscore}{\kern0pt}def\ \isacommand{by}\isamarkupfalse%
\ auto\ \isanewline
\ \ \ \isacommand{then}\isamarkupfalse%
\ \isacommand{show}\isamarkupfalse%
\ {\isachardoublequoteopen}P{\isacharparenleft}{\kern0pt}F{\isacharbackquote}{\kern0pt}x{\isacharparenright}{\kern0pt}{\isachardoublequoteclose}\ \isacommand{using}\isamarkupfalse%
\ assm\ \isacommand{by}\isamarkupfalse%
\ auto\ \isanewline
\ \isacommand{qed}\isamarkupfalse%
\isanewline
\isacommand{qed}\isamarkupfalse%
%
\endisatagproof
{\isafoldproof}%
%
\isadelimproof
\isanewline
%
\endisadelimproof
\isanewline
\isacommand{lemma}\isamarkupfalse%
\ pair{\isacharunderscore}{\kern0pt}forallI\ {\isacharcolon}{\kern0pt}\ \isanewline
\ \ {\isachardoublequoteopen}{\isasymAnd}A\ x{\isachardot}{\kern0pt}\ relation{\isacharparenleft}{\kern0pt}x{\isacharparenright}{\kern0pt}\ {\isasymLongrightarrow}\ {\isacharparenleft}{\kern0pt}{\isasymAnd}y\ p{\isachardot}{\kern0pt}\ {\isacharless}{\kern0pt}y{\isacharcomma}{\kern0pt}\ p{\isachargreater}{\kern0pt}\ {\isasymin}\ x\ {\isasymLongrightarrow}\ A{\isacharparenleft}{\kern0pt}y{\isacharcomma}{\kern0pt}\ p{\isacharparenright}{\kern0pt}{\isacharparenright}{\kern0pt}\ {\isasymLongrightarrow}\ {\isasymforall}{\isacharless}{\kern0pt}y{\isacharcomma}{\kern0pt}\ p{\isachargreater}{\kern0pt}\ {\isasymin}\ x{\isachardot}{\kern0pt}\ A{\isacharparenleft}{\kern0pt}y{\isacharcomma}{\kern0pt}\ p{\isacharparenright}{\kern0pt}{\isachardoublequoteclose}\isanewline
%
\isadelimproof
%
\endisadelimproof
%
\isatagproof
\isacommand{proof}\isamarkupfalse%
\ {\isacharparenleft}{\kern0pt}clarify{\isacharparenright}{\kern0pt}\ \isanewline
\ \ \isacommand{fix}\isamarkupfalse%
\ A\ x\ v\ \isacommand{assume}\isamarkupfalse%
\ assms\ {\isacharcolon}{\kern0pt}\ {\isachardoublequoteopen}relation{\isacharparenleft}{\kern0pt}x{\isacharparenright}{\kern0pt}{\isachardoublequoteclose}{\isachardoublequoteopen}{\isacharparenleft}{\kern0pt}{\isasymAnd}y\ p{\isachardot}{\kern0pt}\ {\isasymlangle}y{\isacharcomma}{\kern0pt}\ p{\isasymrangle}\ {\isasymin}\ x\ {\isasymLongrightarrow}\ A{\isacharparenleft}{\kern0pt}y{\isacharcomma}{\kern0pt}\ p{\isacharparenright}{\kern0pt}{\isacharparenright}{\kern0pt}{\isachardoublequoteclose}{\isachardoublequoteopen}v\ {\isasymin}\ x{\isachardoublequoteclose}\ \isanewline
\ \ \isacommand{then}\isamarkupfalse%
\ \isacommand{obtain}\isamarkupfalse%
\ y\ p\ \isakeyword{where}\ yph\ {\isacharcolon}{\kern0pt}\ {\isachardoublequoteopen}v\ {\isacharequal}{\kern0pt}\ {\isacharless}{\kern0pt}y{\isacharcomma}{\kern0pt}\ p{\isachargreater}{\kern0pt}{\isachardoublequoteclose}\ \isacommand{unfolding}\isamarkupfalse%
\ relation{\isacharunderscore}{\kern0pt}def\ \isacommand{by}\isamarkupfalse%
\ auto\ \isanewline
\ \ \isacommand{then}\isamarkupfalse%
\ \isacommand{have}\isamarkupfalse%
\ {\isachardoublequoteopen}A{\isacharparenleft}{\kern0pt}y{\isacharcomma}{\kern0pt}\ p{\isacharparenright}{\kern0pt}{\isachardoublequoteclose}\ \isacommand{using}\isamarkupfalse%
\ assms\ \isacommand{by}\isamarkupfalse%
\ auto\ \isanewline
\ \ \isacommand{then}\isamarkupfalse%
\ \isacommand{show}\isamarkupfalse%
\ {\isachardoublequoteopen}{\isacharparenleft}{\kern0pt}{\isasymlambda}{\isasymlangle}y{\isacharcomma}{\kern0pt}p{\isasymrangle}{\isachardot}{\kern0pt}\ A{\isacharparenleft}{\kern0pt}y{\isacharcomma}{\kern0pt}\ p{\isacharparenright}{\kern0pt}{\isacharparenright}{\kern0pt}{\isacharparenleft}{\kern0pt}v{\isacharparenright}{\kern0pt}{\isachardoublequoteclose}\ \isacommand{using}\isamarkupfalse%
\ assms\ yph\ \isacommand{by}\isamarkupfalse%
\ auto\ \isanewline
\isacommand{qed}\isamarkupfalse%
%
\endisatagproof
{\isafoldproof}%
%
\isadelimproof
\isanewline
%
\endisadelimproof
\isanewline
\isacommand{lemma}\isamarkupfalse%
\ pair{\isacharunderscore}{\kern0pt}forallD\ {\isacharcolon}{\kern0pt}\ \isanewline
\ \ {\isachardoublequoteopen}{\isasymAnd}A\ x{\isachardot}{\kern0pt}\ relation{\isacharparenleft}{\kern0pt}x{\isacharparenright}{\kern0pt}\ {\isasymLongrightarrow}\ {\isasymforall}{\isacharless}{\kern0pt}y{\isacharcomma}{\kern0pt}\ p{\isachargreater}{\kern0pt}\ {\isasymin}\ x{\isachardot}{\kern0pt}\ A{\isacharparenleft}{\kern0pt}y{\isacharcomma}{\kern0pt}\ p{\isacharparenright}{\kern0pt}\ {\isasymLongrightarrow}\ {\isacharparenleft}{\kern0pt}{\isasymAnd}y\ p{\isachardot}{\kern0pt}\ {\isacharless}{\kern0pt}y{\isacharcomma}{\kern0pt}\ p{\isachargreater}{\kern0pt}\ {\isasymin}\ x\ {\isasymLongrightarrow}\ A{\isacharparenleft}{\kern0pt}y{\isacharcomma}{\kern0pt}\ p{\isacharparenright}{\kern0pt}{\isacharparenright}{\kern0pt}{\isachardoublequoteclose}\isanewline
%
\isadelimproof
%
\endisadelimproof
%
\isatagproof
\isacommand{proof}\isamarkupfalse%
\ {\isacharparenleft}{\kern0pt}simp{\isacharparenright}{\kern0pt}\isanewline
\ \ \isacommand{fix}\isamarkupfalse%
\ A\ x\ \isacommand{assume}\isamarkupfalse%
\ assms\ {\isacharcolon}{\kern0pt}\ {\isachardoublequoteopen}relation{\isacharparenleft}{\kern0pt}x{\isacharparenright}{\kern0pt}{\isachardoublequoteclose}\ {\isachardoublequoteopen}{\isasymforall}v{\isasymin}x{\isachardot}{\kern0pt}\ {\isacharparenleft}{\kern0pt}{\isasymlambda}{\isasymlangle}y{\isacharcomma}{\kern0pt}p{\isasymrangle}{\isachardot}{\kern0pt}\ A{\isacharparenleft}{\kern0pt}y{\isacharcomma}{\kern0pt}\ p{\isacharparenright}{\kern0pt}{\isacharparenright}{\kern0pt}{\isacharparenleft}{\kern0pt}v{\isacharparenright}{\kern0pt}{\isachardoublequoteclose}\isanewline
\ \ \isacommand{show}\isamarkupfalse%
\ {\isachardoublequoteopen}{\isasymAnd}y\ p{\isachardot}{\kern0pt}\ {\isasymlangle}y{\isacharcomma}{\kern0pt}\ p{\isasymrangle}\ {\isasymin}\ x\ {\isasymLongrightarrow}\ A{\isacharparenleft}{\kern0pt}y{\isacharcomma}{\kern0pt}\ p{\isacharparenright}{\kern0pt}{\isachardoublequoteclose}\isanewline
\ \ \isacommand{proof}\isamarkupfalse%
\ {\isacharminus}{\kern0pt}\ \isanewline
\ \ \ \ \isacommand{fix}\isamarkupfalse%
\ y\ p\ \isacommand{assume}\isamarkupfalse%
\ {\isachardoublequoteopen}{\isacharless}{\kern0pt}y{\isacharcomma}{\kern0pt}\ p{\isachargreater}{\kern0pt}\ {\isasymin}\ x{\isachardoublequoteclose}\ \isanewline
\ \ \ \ \isacommand{then}\isamarkupfalse%
\ \isacommand{show}\isamarkupfalse%
\ {\isachardoublequoteopen}A{\isacharparenleft}{\kern0pt}y{\isacharcomma}{\kern0pt}\ p{\isacharparenright}{\kern0pt}{\isachardoublequoteclose}\ \isacommand{using}\isamarkupfalse%
\ assms\ \isacommand{by}\isamarkupfalse%
\ auto\isanewline
\ \ \isacommand{qed}\isamarkupfalse%
\isanewline
\isacommand{qed}\isamarkupfalse%
%
\endisatagproof
{\isafoldproof}%
%
\isadelimproof
\isanewline
%
\endisadelimproof
\isanewline
\isacommand{lemma}\isamarkupfalse%
\ pair{\isacharunderscore}{\kern0pt}forallE\ {\isacharcolon}{\kern0pt}\ \isanewline
\ \ {\isachardoublequoteopen}{\isasymAnd}A\ Q\ x{\isachardot}{\kern0pt}\ relation{\isacharparenleft}{\kern0pt}x{\isacharparenright}{\kern0pt}\ {\isasymLongrightarrow}\ {\isasymforall}{\isacharless}{\kern0pt}y{\isacharcomma}{\kern0pt}\ p{\isachargreater}{\kern0pt}\ {\isasymin}\ x{\isachardot}{\kern0pt}\ A{\isacharparenleft}{\kern0pt}y{\isacharcomma}{\kern0pt}\ p{\isacharparenright}{\kern0pt}\ {\isasymLongrightarrow}\ {\isacharparenleft}{\kern0pt}{\isacharparenleft}{\kern0pt}{\isasymAnd}y\ p{\isachardot}{\kern0pt}\ {\isacharless}{\kern0pt}y{\isacharcomma}{\kern0pt}\ p{\isachargreater}{\kern0pt}\ {\isasymin}\ x\ {\isasymLongrightarrow}\ A{\isacharparenleft}{\kern0pt}y{\isacharcomma}{\kern0pt}\ p{\isacharparenright}{\kern0pt}{\isacharparenright}{\kern0pt}\ {\isasymLongrightarrow}\ Q{\isacharparenright}{\kern0pt}\ {\isasymLongrightarrow}\ Q{\isachardoublequoteclose}\ \isanewline
%
\isadelimproof
%
\endisadelimproof
%
\isatagproof
\isacommand{proof}\isamarkupfalse%
\ {\isacharminus}{\kern0pt}\ \isanewline
\ \ \isacommand{fix}\isamarkupfalse%
\ A\ Q\ x\ \isacommand{assume}\isamarkupfalse%
\ {\isachardoublequoteopen}relation{\isacharparenleft}{\kern0pt}x{\isacharparenright}{\kern0pt}{\isachardoublequoteclose}\ {\isachardoublequoteopen}{\isasymforall}{\isacharless}{\kern0pt}y{\isacharcomma}{\kern0pt}\ p{\isachargreater}{\kern0pt}\ {\isasymin}\ x{\isachardot}{\kern0pt}\ A{\isacharparenleft}{\kern0pt}y{\isacharcomma}{\kern0pt}\ p{\isacharparenright}{\kern0pt}{\isachardoublequoteclose}\ \isakeyword{and}\ r{\isacharcolon}{\kern0pt}\ {\isachardoublequoteopen}{\isacharparenleft}{\kern0pt}{\isacharparenleft}{\kern0pt}{\isasymAnd}y\ p{\isachardot}{\kern0pt}\ {\isacharless}{\kern0pt}y{\isacharcomma}{\kern0pt}\ p{\isachargreater}{\kern0pt}\ {\isasymin}\ x\ {\isasymLongrightarrow}\ A{\isacharparenleft}{\kern0pt}y{\isacharcomma}{\kern0pt}\ p{\isacharparenright}{\kern0pt}{\isacharparenright}{\kern0pt}\ {\isasymLongrightarrow}\ Q{\isacharparenright}{\kern0pt}{\isachardoublequoteclose}\ \isanewline
\ \ \isacommand{then}\isamarkupfalse%
\ \isacommand{show}\isamarkupfalse%
\ {\isachardoublequoteopen}Q{\isachardoublequoteclose}\ \isanewline
\ \ \ \ \isacommand{apply}\isamarkupfalse%
\ {\isacharparenleft}{\kern0pt}rule{\isacharunderscore}{\kern0pt}tac\ r{\isacharparenright}{\kern0pt}\ \isanewline
\ \ \ \ \isacommand{apply}\isamarkupfalse%
\ {\isacharparenleft}{\kern0pt}rule{\isacharunderscore}{\kern0pt}tac\ x{\isacharequal}{\kern0pt}x\ \isakeyword{in}\ pair{\isacharunderscore}{\kern0pt}forallD{\isacharparenright}{\kern0pt}\ \isanewline
\ \ \ \ \isacommand{by}\isamarkupfalse%
\ auto\ \isanewline
\isacommand{qed}\isamarkupfalse%
%
\endisatagproof
{\isafoldproof}%
%
\isadelimproof
\isanewline
%
\endisadelimproof
\isanewline
\isacommand{lemma}\isamarkupfalse%
\ converse{\isacharunderscore}{\kern0pt}apply\ {\isacharcolon}{\kern0pt}\ {\isachardoublequoteopen}f\ {\isasymin}\ bij{\isacharparenleft}{\kern0pt}A{\isacharcomma}{\kern0pt}\ B{\isacharparenright}{\kern0pt}\ {\isasymLongrightarrow}\ x\ {\isasymin}\ A\ {\isasymLongrightarrow}\ converse{\isacharparenleft}{\kern0pt}f{\isacharparenright}{\kern0pt}\ {\isacharbackquote}{\kern0pt}\ {\isacharparenleft}{\kern0pt}f\ {\isacharbackquote}{\kern0pt}\ x{\isacharparenright}{\kern0pt}\ {\isacharequal}{\kern0pt}\ x{\isachardoublequoteclose}\ \isanewline
%
\isadelimproof
%
\endisadelimproof
%
\isatagproof
\isacommand{proof}\isamarkupfalse%
\ {\isacharminus}{\kern0pt}\ \isanewline
\ \ \isacommand{assume}\isamarkupfalse%
\ assms\ {\isacharcolon}{\kern0pt}\ {\isachardoublequoteopen}f\ {\isasymin}\ bij{\isacharparenleft}{\kern0pt}A{\isacharcomma}{\kern0pt}\ B{\isacharparenright}{\kern0pt}{\isachardoublequoteclose}\ {\isachardoublequoteopen}x\ {\isasymin}\ A{\isachardoublequoteclose}\ \ \ \ \ \ \ \ \ \ \ \ \ \ \ \ \ \ \ \isanewline
\ \ \isacommand{then}\isamarkupfalse%
\ \isacommand{have}\isamarkupfalse%
\ {\isachardoublequoteopen}A\ {\isasymsubseteq}\ domain{\isacharparenleft}{\kern0pt}f{\isacharparenright}{\kern0pt}{\isachardoublequoteclose}\ \isacommand{using}\isamarkupfalse%
\ bij{\isacharunderscore}{\kern0pt}is{\isacharunderscore}{\kern0pt}fun\ \isacommand{unfolding}\isamarkupfalse%
\ Pi{\isacharunderscore}{\kern0pt}def\ \isacommand{by}\isamarkupfalse%
\ auto\ \isanewline
\ \ \isacommand{then}\isamarkupfalse%
\ \isacommand{have}\isamarkupfalse%
\ {\isachardoublequoteopen}{\isacharless}{\kern0pt}x{\isacharcomma}{\kern0pt}\ f{\isacharbackquote}{\kern0pt}x{\isachargreater}{\kern0pt}\ {\isasymin}\ f{\isachardoublequoteclose}\ \ \isanewline
\ \ \ \ \isacommand{apply}\isamarkupfalse%
\ {\isacharparenleft}{\kern0pt}rule{\isacharunderscore}{\kern0pt}tac\ function{\isacharunderscore}{\kern0pt}apply{\isacharunderscore}{\kern0pt}Pair{\isacharparenright}{\kern0pt}\ \isacommand{using}\isamarkupfalse%
\ bij{\isacharunderscore}{\kern0pt}is{\isacharunderscore}{\kern0pt}fun\ assms\ \isacommand{unfolding}\isamarkupfalse%
\ Pi{\isacharunderscore}{\kern0pt}def\ \isacommand{apply}\isamarkupfalse%
\ simp\isanewline
\ \ \ \ \isacommand{using}\isamarkupfalse%
\ bij{\isacharunderscore}{\kern0pt}is{\isacharunderscore}{\kern0pt}fun\ assms\ \isacommand{by}\isamarkupfalse%
\ auto\isanewline
\ \ \isacommand{then}\isamarkupfalse%
\ \isacommand{have}\isamarkupfalse%
\ H{\isacharcolon}{\kern0pt}\ {\isachardoublequoteopen}{\isacharless}{\kern0pt}f{\isacharbackquote}{\kern0pt}x{\isacharcomma}{\kern0pt}\ x{\isachargreater}{\kern0pt}\ {\isasymin}\ converse{\isacharparenleft}{\kern0pt}f{\isacharparenright}{\kern0pt}{\isachardoublequoteclose}\ \isacommand{apply}\isamarkupfalse%
\ {\isacharparenleft}{\kern0pt}rule{\isacharunderscore}{\kern0pt}tac\ converseI{\isacharparenright}{\kern0pt}\ \isacommand{apply}\isamarkupfalse%
\ simp\ \isacommand{done}\isamarkupfalse%
\ \isanewline
\ \ \isacommand{have}\isamarkupfalse%
\ {\isachardoublequoteopen}converse{\isacharparenleft}{\kern0pt}f{\isacharparenright}{\kern0pt}\ {\isasymin}\ bij{\isacharparenleft}{\kern0pt}B{\isacharcomma}{\kern0pt}\ A{\isacharparenright}{\kern0pt}{\isachardoublequoteclose}\ \isacommand{using}\isamarkupfalse%
\ bij{\isacharunderscore}{\kern0pt}converse{\isacharunderscore}{\kern0pt}bij\ assms\ \isacommand{by}\isamarkupfalse%
\ auto\ \isanewline
\ \ \isacommand{then}\isamarkupfalse%
\ \isacommand{have}\isamarkupfalse%
\ {\isachardoublequoteopen}converse{\isacharparenleft}{\kern0pt}f{\isacharparenright}{\kern0pt}\ {\isasymin}\ B\ {\isasymrightarrow}\ A{\isachardoublequoteclose}\ \isacommand{using}\isamarkupfalse%
\ bij{\isacharunderscore}{\kern0pt}is{\isacharunderscore}{\kern0pt}fun\ \ \isacommand{by}\isamarkupfalse%
\ auto\ \isanewline
\ \ \isacommand{then}\isamarkupfalse%
\ \isacommand{show}\isamarkupfalse%
\ {\isachardoublequoteopen}converse{\isacharparenleft}{\kern0pt}f{\isacharparenright}{\kern0pt}\ {\isacharbackquote}{\kern0pt}\ {\isacharparenleft}{\kern0pt}f\ {\isacharbackquote}{\kern0pt}\ x{\isacharparenright}{\kern0pt}\ {\isacharequal}{\kern0pt}\ x{\isachardoublequoteclose}\ \isacommand{using}\isamarkupfalse%
\ apply{\isacharunderscore}{\kern0pt}fun\ H\ \isacommand{by}\isamarkupfalse%
\ auto\ \isanewline
\isacommand{qed}\isamarkupfalse%
%
\endisatagproof
{\isafoldproof}%
%
\isadelimproof
\isanewline
%
\endisadelimproof
\isanewline
\isacommand{lemma}\isamarkupfalse%
\ image{\isacharunderscore}{\kern0pt}converse{\isacharunderscore}{\kern0pt}image\ {\isacharcolon}{\kern0pt}\ {\isachardoublequoteopen}f\ {\isasymin}\ bij{\isacharparenleft}{\kern0pt}A{\isacharcomma}{\kern0pt}\ B{\isacharparenright}{\kern0pt}\ {\isasymLongrightarrow}\ C\ {\isasymsubseteq}\ A\ {\isasymLongrightarrow}\ converse{\isacharparenleft}{\kern0pt}f{\isacharparenright}{\kern0pt}\ {\isacharbackquote}{\kern0pt}{\isacharbackquote}{\kern0pt}\ {\isacharparenleft}{\kern0pt}f\ {\isacharbackquote}{\kern0pt}{\isacharbackquote}{\kern0pt}\ C{\isacharparenright}{\kern0pt}\ {\isacharequal}{\kern0pt}\ C{\isachardoublequoteclose}\ \isanewline
%
\isadelimproof
\ \ %
\endisadelimproof
%
\isatagproof
\isacommand{apply}\isamarkupfalse%
\ {\isacharparenleft}{\kern0pt}rule\ equality{\isacharunderscore}{\kern0pt}iffI{\isacharparenright}{\kern0pt}\ \isanewline
\ \ \isacommand{apply}\isamarkupfalse%
\ {\isacharparenleft}{\kern0pt}rule\ iffI{\isacharparenright}{\kern0pt}\ \isanewline
\isacommand{proof}\isamarkupfalse%
\ {\isacharminus}{\kern0pt}\ \isanewline
\ \ \isacommand{fix}\isamarkupfalse%
\ x\ \isacommand{assume}\isamarkupfalse%
\ assms\ {\isacharcolon}{\kern0pt}\ {\isachardoublequoteopen}f\ {\isasymin}\ bij{\isacharparenleft}{\kern0pt}A{\isacharcomma}{\kern0pt}\ B{\isacharparenright}{\kern0pt}\ {\isachardoublequoteclose}\ {\isachardoublequoteopen}C{\isasymsubseteq}A{\isachardoublequoteclose}\ {\isachardoublequoteopen}x\ {\isasymin}\ converse{\isacharparenleft}{\kern0pt}f{\isacharparenright}{\kern0pt}\ {\isacharbackquote}{\kern0pt}{\isacharbackquote}{\kern0pt}\ {\isacharparenleft}{\kern0pt}f\ {\isacharbackquote}{\kern0pt}{\isacharbackquote}{\kern0pt}\ C{\isacharparenright}{\kern0pt}{\isachardoublequoteclose}\ \isanewline
\ \ \isacommand{then}\isamarkupfalse%
\ \isacommand{obtain}\isamarkupfalse%
\ y\ \isakeyword{where}\ yh\ {\isacharcolon}{\kern0pt}\ {\isachardoublequoteopen}y\ {\isasymin}\ f{\isacharbackquote}{\kern0pt}{\isacharbackquote}{\kern0pt}C{\isachardoublequoteclose}\ {\isachardoublequoteopen}{\isacharless}{\kern0pt}y{\isacharcomma}{\kern0pt}\ x{\isachargreater}{\kern0pt}\ {\isasymin}\ converse{\isacharparenleft}{\kern0pt}f{\isacharparenright}{\kern0pt}{\isachardoublequoteclose}\ \isacommand{using}\isamarkupfalse%
\ image{\isacharunderscore}{\kern0pt}iff\ \isacommand{by}\isamarkupfalse%
\ auto\ \isanewline
\ \ \isacommand{then}\isamarkupfalse%
\ \isacommand{have}\isamarkupfalse%
\ {\isachardoublequoteopen}{\isacharless}{\kern0pt}x{\isacharcomma}{\kern0pt}\ y{\isachargreater}{\kern0pt}\ {\isasymin}\ f{\isachardoublequoteclose}\ \isacommand{using}\isamarkupfalse%
\ converseE\ \isacommand{by}\isamarkupfalse%
\ auto\ \isanewline
\ \ \isacommand{then}\isamarkupfalse%
\ \isacommand{have}\isamarkupfalse%
\ H\ {\isacharcolon}{\kern0pt}\ {\isachardoublequoteopen}x\ {\isasymin}\ A\ {\isasymand}\ f{\isacharbackquote}{\kern0pt}x\ {\isacharequal}{\kern0pt}\ y{\isachardoublequoteclose}\ \isanewline
\ \ \ \ \isacommand{apply}\isamarkupfalse%
\ {\isacharparenleft}{\kern0pt}rule{\isacharunderscore}{\kern0pt}tac\ B{\isacharequal}{\kern0pt}{\isachardoublequoteopen}{\isacharparenleft}{\kern0pt}{\isasymlambda}{\isacharunderscore}{\kern0pt}{\isachardot}{\kern0pt}\ B{\isacharparenright}{\kern0pt}{\isachardoublequoteclose}\ \isakeyword{in}\ apply{\isacharunderscore}{\kern0pt}fun{\isacharparenright}{\kern0pt}\ \isanewline
\ \ \ \ \isacommand{using}\isamarkupfalse%
\ bij{\isacharunderscore}{\kern0pt}is{\isacharunderscore}{\kern0pt}fun\ assms\ \isacommand{by}\isamarkupfalse%
\ auto\ \isanewline
\ \ \isacommand{then}\isamarkupfalse%
\ \isacommand{obtain}\isamarkupfalse%
\ z\ \isakeyword{where}\ zh{\isacharcolon}{\kern0pt}\ {\isachardoublequoteopen}z\ {\isasymin}\ C{\isachardoublequoteclose}\ {\isachardoublequoteopen}{\isacharless}{\kern0pt}z{\isacharcomma}{\kern0pt}\ y{\isachargreater}{\kern0pt}\ {\isasymin}\ f{\isachardoublequoteclose}\ \isacommand{using}\isamarkupfalse%
\ yh\ image{\isacharunderscore}{\kern0pt}iff\ \isacommand{by}\isamarkupfalse%
\ auto\ \isanewline
\ \ \isacommand{then}\isamarkupfalse%
\ \isacommand{have}\isamarkupfalse%
\ H{\isadigit{2}}\ {\isacharcolon}{\kern0pt}\ {\isachardoublequoteopen}z\ {\isasymin}\ A\ {\isasymand}\ f{\isacharbackquote}{\kern0pt}z\ {\isacharequal}{\kern0pt}\ y{\isachardoublequoteclose}\ \isanewline
\ \ \ \ \isacommand{apply}\isamarkupfalse%
\ {\isacharparenleft}{\kern0pt}rule{\isacharunderscore}{\kern0pt}tac\ B{\isacharequal}{\kern0pt}{\isachardoublequoteopen}{\isacharparenleft}{\kern0pt}{\isasymlambda}{\isacharunderscore}{\kern0pt}{\isachardot}{\kern0pt}\ B{\isacharparenright}{\kern0pt}{\isachardoublequoteclose}\ \isakeyword{in}\ apply{\isacharunderscore}{\kern0pt}fun{\isacharparenright}{\kern0pt}\ \isanewline
\ \ \ \ \isacommand{using}\isamarkupfalse%
\ bij{\isacharunderscore}{\kern0pt}is{\isacharunderscore}{\kern0pt}fun\ assms\ \isacommand{by}\isamarkupfalse%
\ auto\ \isanewline
\ \ \isacommand{have}\isamarkupfalse%
\ {\isachardoublequoteopen}f\ {\isasymin}\ inj{\isacharparenleft}{\kern0pt}A{\isacharcomma}{\kern0pt}\ B{\isacharparenright}{\kern0pt}{\isachardoublequoteclose}\ \isacommand{using}\isamarkupfalse%
\ bij{\isacharunderscore}{\kern0pt}is{\isacharunderscore}{\kern0pt}inj\ assms\ \isacommand{by}\isamarkupfalse%
\ auto\ \isanewline
\ \ \isacommand{then}\isamarkupfalse%
\ \isacommand{have}\isamarkupfalse%
\ {\isachardoublequoteopen}x{\isacharequal}{\kern0pt}z{\isachardoublequoteclose}\ \isacommand{unfolding}\isamarkupfalse%
\ inj{\isacharunderscore}{\kern0pt}def\ \isacommand{using}\isamarkupfalse%
\ H\ H{\isadigit{2}}\ \isacommand{by}\isamarkupfalse%
\ auto\ \isanewline
\ \ \isacommand{then}\isamarkupfalse%
\ \isacommand{show}\isamarkupfalse%
\ {\isachardoublequoteopen}x\ {\isasymin}\ C{\isachardoublequoteclose}\ \isacommand{using}\isamarkupfalse%
\ zh\ \isacommand{by}\isamarkupfalse%
\ auto\ \isanewline
\isacommand{next}\isamarkupfalse%
\ \isanewline
\ \ \isacommand{fix}\isamarkupfalse%
\ x\ \isacommand{assume}\isamarkupfalse%
\ assms\ {\isacharcolon}{\kern0pt}\ {\isachardoublequoteopen}f\ {\isasymin}\ bij{\isacharparenleft}{\kern0pt}A{\isacharcomma}{\kern0pt}\ B{\isacharparenright}{\kern0pt}\ {\isachardoublequoteclose}\ {\isachardoublequoteopen}C{\isasymsubseteq}A{\isachardoublequoteclose}\ {\isachardoublequoteopen}x\ {\isasymin}\ C{\isachardoublequoteclose}\isanewline
\ \ \isacommand{then}\isamarkupfalse%
\ \isacommand{have}\isamarkupfalse%
\ {\isachardoublequoteopen}f{\isacharbackquote}{\kern0pt}x\ {\isasymin}\ f{\isacharbackquote}{\kern0pt}{\isacharbackquote}{\kern0pt}C{\isachardoublequoteclose}\ \isanewline
\ \ \ \ \isacommand{apply}\isamarkupfalse%
\ {\isacharparenleft}{\kern0pt}rule{\isacharunderscore}{\kern0pt}tac\ a{\isacharequal}{\kern0pt}x\ \isakeyword{in}\ imageI{\isacharparenright}{\kern0pt}\ \isanewline
\ \ \ \ \isacommand{apply}\isamarkupfalse%
\ {\isacharparenleft}{\kern0pt}rule{\isacharunderscore}{\kern0pt}tac\ function{\isacharunderscore}{\kern0pt}apply{\isacharunderscore}{\kern0pt}Pair{\isacharparenright}{\kern0pt}\ \isanewline
\ \ \ \ \isacommand{using}\isamarkupfalse%
\ bij{\isacharunderscore}{\kern0pt}is{\isacharunderscore}{\kern0pt}fun\ \isacommand{unfolding}\isamarkupfalse%
\ Pi{\isacharunderscore}{\kern0pt}def\ \isacommand{apply}\isamarkupfalse%
\ simp{\isacharunderscore}{\kern0pt}all\ \isacommand{apply}\isamarkupfalse%
\ blast\ \isacommand{done}\isamarkupfalse%
\isanewline
\ \ \isacommand{then}\isamarkupfalse%
\ \isacommand{show}\isamarkupfalse%
\ {\isachardoublequoteopen}x\ {\isasymin}\ converse{\isacharparenleft}{\kern0pt}f{\isacharparenright}{\kern0pt}\ {\isacharbackquote}{\kern0pt}{\isacharbackquote}{\kern0pt}\ {\isacharparenleft}{\kern0pt}f{\isacharbackquote}{\kern0pt}{\isacharbackquote}{\kern0pt}C{\isacharparenright}{\kern0pt}{\isachardoublequoteclose}\ \isanewline
\ \ \ \ \isacommand{apply}\isamarkupfalse%
\ {\isacharparenleft}{\kern0pt}rule{\isacharunderscore}{\kern0pt}tac\ a{\isacharequal}{\kern0pt}{\isachardoublequoteopen}f{\isacharbackquote}{\kern0pt}x{\isachardoublequoteclose}\ \isakeyword{in}\ imageI{\isacharparenright}{\kern0pt}\ \isanewline
\ \ \ \ \isacommand{apply}\isamarkupfalse%
\ {\isacharparenleft}{\kern0pt}rule{\isacharunderscore}{\kern0pt}tac\ converseI{\isacharparenright}{\kern0pt}\isanewline
\ \ \ \ \isacommand{apply}\isamarkupfalse%
\ {\isacharparenleft}{\kern0pt}rule{\isacharunderscore}{\kern0pt}tac\ function{\isacharunderscore}{\kern0pt}apply{\isacharunderscore}{\kern0pt}Pair{\isacharparenright}{\kern0pt}\isanewline
\ \ \ \ \isacommand{using}\isamarkupfalse%
\ bij{\isacharunderscore}{\kern0pt}is{\isacharunderscore}{\kern0pt}fun\ assms\ \isacommand{unfolding}\isamarkupfalse%
\ Pi{\isacharunderscore}{\kern0pt}def\ \isacommand{apply}\isamarkupfalse%
\ simp{\isacharunderscore}{\kern0pt}all\ \isacommand{apply}\isamarkupfalse%
\ blast\ \isacommand{done}\isamarkupfalse%
\isanewline
\isacommand{qed}\isamarkupfalse%
%
\endisatagproof
{\isafoldproof}%
%
\isadelimproof
\ \ \isanewline
%
\endisadelimproof
\isanewline
\isacommand{lemma}\isamarkupfalse%
\ ifT{\isacharunderscore}{\kern0pt}eq\ {\isacharcolon}{\kern0pt}\ {\isachardoublequoteopen}a\ {\isasymnoteq}\ b\ {\isasymLongrightarrow}\ {\isacharparenleft}{\kern0pt}if\ P\ then\ a\ else\ b{\isacharparenright}{\kern0pt}\ {\isacharequal}{\kern0pt}\ a\ {\isasymLongrightarrow}\ P{\isachardoublequoteclose}\ \isanewline
%
\isadelimproof
\ \ %
\endisadelimproof
%
\isatagproof
\isacommand{apply}\isamarkupfalse%
\ {\isacharparenleft}{\kern0pt}rule{\isacharunderscore}{\kern0pt}tac\ P{\isacharequal}{\kern0pt}P\ \isakeyword{and}\ Q{\isacharequal}{\kern0pt}{\isachardoublequoteopen}{\isasymnot}P{\isachardoublequoteclose}\ \isakeyword{in}\ disjE{\isacharparenright}{\kern0pt}\ \isacommand{apply}\isamarkupfalse%
\ simp{\isacharunderscore}{\kern0pt}all\ \isacommand{done}\isamarkupfalse%
%
\endisatagproof
{\isafoldproof}%
%
\isadelimproof
\ \isanewline
%
\endisadelimproof
\isanewline
\isacommand{lemma}\isamarkupfalse%
\ ifF{\isacharunderscore}{\kern0pt}eq\ {\isacharcolon}{\kern0pt}\ {\isachardoublequoteopen}a\ {\isasymnoteq}\ b\ {\isasymLongrightarrow}\ {\isacharparenleft}{\kern0pt}if\ P\ then\ a\ else\ b{\isacharparenright}{\kern0pt}\ {\isacharequal}{\kern0pt}\ b\ {\isasymLongrightarrow}\ {\isasymnot}P{\isachardoublequoteclose}\ \isanewline
%
\isadelimproof
\ \ %
\endisadelimproof
%
\isatagproof
\isacommand{apply}\isamarkupfalse%
\ {\isacharparenleft}{\kern0pt}rule{\isacharunderscore}{\kern0pt}tac\ P{\isacharequal}{\kern0pt}P\ \isakeyword{and}\ Q{\isacharequal}{\kern0pt}{\isachardoublequoteopen}{\isasymnot}P{\isachardoublequoteclose}\ \isakeyword{in}\ disjE{\isacharparenright}{\kern0pt}\ \isacommand{apply}\isamarkupfalse%
\ simp{\isacharunderscore}{\kern0pt}all\ \isacommand{done}\isamarkupfalse%
%
\endisatagproof
{\isafoldproof}%
%
\isadelimproof
\ \isanewline
%
\endisadelimproof
\isanewline
\isacommand{lemma}\isamarkupfalse%
\ neq{\isacharunderscore}{\kern0pt}flip\ {\isacharcolon}{\kern0pt}\ {\isachardoublequoteopen}a\ {\isasymnoteq}\ b\ {\isasymLongrightarrow}\ b\ {\isasymnoteq}\ a{\isachardoublequoteclose}%
\isadelimproof
\ %
\endisadelimproof
%
\isatagproof
\isacommand{by}\isamarkupfalse%
\ auto%
\endisatagproof
{\isafoldproof}%
%
\isadelimproof
%
\endisadelimproof
\isanewline
\isanewline
\isacommand{lemma}\isamarkupfalse%
\ iff{\isacharunderscore}{\kern0pt}eq\ {\isacharcolon}{\kern0pt}\ {\isachardoublequoteopen}{\isasymAnd}A\ P\ Q{\isachardot}{\kern0pt}\ {\isacharparenleft}{\kern0pt}{\isasymAnd}x{\isachardot}{\kern0pt}\ x\ {\isasymin}\ A\ {\isasymLongrightarrow}\ P{\isacharparenleft}{\kern0pt}x{\isacharparenright}{\kern0pt}\ {\isasymlongleftrightarrow}\ Q{\isacharparenleft}{\kern0pt}x{\isacharparenright}{\kern0pt}{\isacharparenright}{\kern0pt}\ {\isasymLongrightarrow}\ {\isacharbraceleft}{\kern0pt}\ x\ {\isasymin}\ A{\isachardot}{\kern0pt}\ P{\isacharparenleft}{\kern0pt}x{\isacharparenright}{\kern0pt}\ {\isacharbraceright}{\kern0pt}\ {\isacharequal}{\kern0pt}\ {\isacharbraceleft}{\kern0pt}\ x\ {\isasymin}\ A{\isachardot}{\kern0pt}\ Q{\isacharparenleft}{\kern0pt}x{\isacharparenright}{\kern0pt}\ {\isacharbraceright}{\kern0pt}{\isachardoublequoteclose}%
\isadelimproof
\ %
\endisadelimproof
%
\isatagproof
\isacommand{by}\isamarkupfalse%
\ auto%
\endisatagproof
{\isafoldproof}%
%
\isadelimproof
%
\endisadelimproof
\isanewline
\isanewline
\isacommand{lemma}\isamarkupfalse%
\ ex{\isacharunderscore}{\kern0pt}iff\ {\isacharcolon}{\kern0pt}\ {\isachardoublequoteopen}{\isasymAnd}A\ P\ Q{\isachardot}{\kern0pt}\ {\isacharparenleft}{\kern0pt}{\isasymAnd}x{\isachardot}{\kern0pt}\ P{\isacharparenleft}{\kern0pt}x{\isacharparenright}{\kern0pt}\ {\isasymlongleftrightarrow}\ Q{\isacharparenleft}{\kern0pt}x{\isacharparenright}{\kern0pt}{\isacharparenright}{\kern0pt}\ {\isasymLongrightarrow}\ {\isacharparenleft}{\kern0pt}{\isasymexists}x{\isachardot}{\kern0pt}\ P{\isacharparenleft}{\kern0pt}x{\isacharparenright}{\kern0pt}{\isacharparenright}{\kern0pt}\ {\isasymlongleftrightarrow}\ {\isacharparenleft}{\kern0pt}{\isasymexists}x{\isachardot}{\kern0pt}\ Q{\isacharparenleft}{\kern0pt}x{\isacharparenright}{\kern0pt}{\isacharparenright}{\kern0pt}{\isachardoublequoteclose}%
\isadelimproof
\ %
\endisadelimproof
%
\isatagproof
\isacommand{by}\isamarkupfalse%
\ auto%
\endisatagproof
{\isafoldproof}%
%
\isadelimproof
%
\endisadelimproof
\ \isanewline
\isacommand{lemma}\isamarkupfalse%
\ bex{\isacharunderscore}{\kern0pt}iff\ {\isacharcolon}{\kern0pt}\ {\isachardoublequoteopen}{\isasymAnd}A\ P\ Q{\isachardot}{\kern0pt}\ {\isacharparenleft}{\kern0pt}{\isasymAnd}x{\isachardot}{\kern0pt}\ x\ {\isasymin}\ A\ {\isasymLongrightarrow}\ P{\isacharparenleft}{\kern0pt}x{\isacharparenright}{\kern0pt}\ {\isasymlongleftrightarrow}\ Q{\isacharparenleft}{\kern0pt}x{\isacharparenright}{\kern0pt}{\isacharparenright}{\kern0pt}\ {\isasymLongrightarrow}\ {\isacharparenleft}{\kern0pt}{\isasymexists}x\ {\isasymin}\ A{\isachardot}{\kern0pt}\ P{\isacharparenleft}{\kern0pt}x{\isacharparenright}{\kern0pt}{\isacharparenright}{\kern0pt}\ {\isasymlongleftrightarrow}\ {\isacharparenleft}{\kern0pt}{\isasymexists}x\ {\isasymin}\ A{\isachardot}{\kern0pt}\ Q{\isacharparenleft}{\kern0pt}x{\isacharparenright}{\kern0pt}{\isacharparenright}{\kern0pt}{\isachardoublequoteclose}%
\isadelimproof
\ %
\endisadelimproof
%
\isatagproof
\isacommand{by}\isamarkupfalse%
\ auto%
\endisatagproof
{\isafoldproof}%
%
\isadelimproof
%
\endisadelimproof
\ \isanewline
\isacommand{lemma}\isamarkupfalse%
\ all{\isacharunderscore}{\kern0pt}iff\ {\isacharcolon}{\kern0pt}\ {\isachardoublequoteopen}{\isasymAnd}A\ P\ Q{\isachardot}{\kern0pt}\ {\isacharparenleft}{\kern0pt}{\isasymAnd}x{\isachardot}{\kern0pt}\ P{\isacharparenleft}{\kern0pt}x{\isacharparenright}{\kern0pt}\ {\isasymlongleftrightarrow}\ Q{\isacharparenleft}{\kern0pt}x{\isacharparenright}{\kern0pt}{\isacharparenright}{\kern0pt}\ {\isasymLongrightarrow}\ {\isacharparenleft}{\kern0pt}{\isasymforall}x{\isachardot}{\kern0pt}\ P{\isacharparenleft}{\kern0pt}x{\isacharparenright}{\kern0pt}{\isacharparenright}{\kern0pt}\ {\isasymlongleftrightarrow}\ {\isacharparenleft}{\kern0pt}{\isasymforall}x{\isachardot}{\kern0pt}\ Q{\isacharparenleft}{\kern0pt}x{\isacharparenright}{\kern0pt}{\isacharparenright}{\kern0pt}{\isachardoublequoteclose}%
\isadelimproof
\ %
\endisadelimproof
%
\isatagproof
\isacommand{by}\isamarkupfalse%
\ auto%
\endisatagproof
{\isafoldproof}%
%
\isadelimproof
%
\endisadelimproof
\ \isanewline
\isacommand{lemma}\isamarkupfalse%
\ ball{\isacharunderscore}{\kern0pt}iff\ {\isacharcolon}{\kern0pt}\ {\isachardoublequoteopen}{\isasymAnd}A\ P\ Q{\isachardot}{\kern0pt}\ {\isacharparenleft}{\kern0pt}{\isasymAnd}x{\isachardot}{\kern0pt}\ x\ {\isasymin}\ A\ {\isasymLongrightarrow}\ P{\isacharparenleft}{\kern0pt}x{\isacharparenright}{\kern0pt}\ {\isasymlongleftrightarrow}\ Q{\isacharparenleft}{\kern0pt}x{\isacharparenright}{\kern0pt}{\isacharparenright}{\kern0pt}\ {\isasymLongrightarrow}\ {\isacharparenleft}{\kern0pt}{\isasymforall}x\ {\isasymin}\ A{\isachardot}{\kern0pt}\ P{\isacharparenleft}{\kern0pt}x{\isacharparenright}{\kern0pt}{\isacharparenright}{\kern0pt}\ {\isasymlongleftrightarrow}\ {\isacharparenleft}{\kern0pt}{\isasymforall}x\ {\isasymin}\ A{\isachardot}{\kern0pt}\ Q{\isacharparenleft}{\kern0pt}x{\isacharparenright}{\kern0pt}{\isacharparenright}{\kern0pt}{\isachardoublequoteclose}%
\isadelimproof
\ %
\endisadelimproof
%
\isatagproof
\isacommand{by}\isamarkupfalse%
\ auto%
\endisatagproof
{\isafoldproof}%
%
\isadelimproof
%
\endisadelimproof
\ \isanewline
\isacommand{lemma}\isamarkupfalse%
\ iff{\isacharunderscore}{\kern0pt}disjI\ {\isacharcolon}{\kern0pt}\ {\isachardoublequoteopen}{\isasymAnd}P\ Q\ R\ S{\isachardot}{\kern0pt}\ P\ {\isasymlongleftrightarrow}\ Q\ {\isasymLongrightarrow}\ R\ {\isasymlongleftrightarrow}\ S\ {\isasymLongrightarrow}\ P\ {\isasymor}\ R\ {\isasymlongleftrightarrow}\ Q\ {\isasymor}\ S{\isachardoublequoteclose}%
\isadelimproof
\ %
\endisadelimproof
%
\isatagproof
\isacommand{by}\isamarkupfalse%
\ auto%
\endisatagproof
{\isafoldproof}%
%
\isadelimproof
%
\endisadelimproof
\isanewline
\isacommand{lemma}\isamarkupfalse%
\ iff{\isacharunderscore}{\kern0pt}conjI\ {\isacharcolon}{\kern0pt}\ {\isachardoublequoteopen}{\isasymAnd}P\ Q\ R\ S{\isachardot}{\kern0pt}\ P\ {\isasymlongleftrightarrow}\ Q\ {\isasymLongrightarrow}\ R\ {\isasymlongleftrightarrow}\ S\ {\isasymLongrightarrow}\ P\ {\isasymand}\ R\ {\isasymlongleftrightarrow}\ Q\ {\isasymand}\ S{\isachardoublequoteclose}%
\isadelimproof
\ %
\endisadelimproof
%
\isatagproof
\isacommand{by}\isamarkupfalse%
\ auto%
\endisatagproof
{\isafoldproof}%
%
\isadelimproof
%
\endisadelimproof
\isanewline
\isacommand{lemma}\isamarkupfalse%
\ iff{\isacharunderscore}{\kern0pt}conjI{\isadigit{2}}\ {\isacharcolon}{\kern0pt}\ {\isachardoublequoteopen}{\isasymAnd}P\ Q\ R\ S{\isachardot}{\kern0pt}\ P\ {\isasymlongleftrightarrow}\ Q\ {\isasymLongrightarrow}\ {\isacharparenleft}{\kern0pt}Q\ {\isasymLongrightarrow}\ R\ {\isasymlongleftrightarrow}\ S{\isacharparenright}{\kern0pt}\ {\isasymLongrightarrow}\ {\isacharparenleft}{\kern0pt}P\ {\isasymand}\ R\ {\isasymlongleftrightarrow}\ Q\ {\isasymand}\ S{\isacharparenright}{\kern0pt}{\isachardoublequoteclose}%
\isadelimproof
\ %
\endisadelimproof
%
\isatagproof
\isacommand{by}\isamarkupfalse%
\ auto%
\endisatagproof
{\isafoldproof}%
%
\isadelimproof
%
\endisadelimproof
\ \isanewline
\isacommand{lemma}\isamarkupfalse%
\ iff{\isacharunderscore}{\kern0pt}iff\ {\isacharcolon}{\kern0pt}\ {\isachardoublequoteopen}{\isasymAnd}P\ Q\ R\ S{\isachardot}{\kern0pt}\ P\ {\isasymlongleftrightarrow}\ Q\ {\isasymLongrightarrow}\ R\ {\isasymlongleftrightarrow}\ S\ {\isasymLongrightarrow}\ {\isacharparenleft}{\kern0pt}P\ {\isasymlongleftrightarrow}\ R{\isacharparenright}{\kern0pt}\ {\isasymlongleftrightarrow}\ {\isacharparenleft}{\kern0pt}Q\ {\isasymlongleftrightarrow}\ S{\isacharparenright}{\kern0pt}{\isachardoublequoteclose}%
\isadelimproof
\ %
\endisadelimproof
%
\isatagproof
\isacommand{by}\isamarkupfalse%
\ auto%
\endisatagproof
{\isafoldproof}%
%
\isadelimproof
%
\endisadelimproof
\ \ \isanewline
\isacommand{lemma}\isamarkupfalse%
\ imp{\isacharunderscore}{\kern0pt}iff\ {\isacharcolon}{\kern0pt}\ {\isachardoublequoteopen}{\isasymAnd}P\ Q\ R\ S{\isachardot}{\kern0pt}\ P\ {\isasymlongleftrightarrow}\ Q\ {\isasymLongrightarrow}\ {\isacharparenleft}{\kern0pt}R\ {\isasymlongleftrightarrow}\ S{\isacharparenright}{\kern0pt}\ {\isasymLongrightarrow}\ {\isacharparenleft}{\kern0pt}P\ {\isasymlongrightarrow}\ R{\isacharparenright}{\kern0pt}\ {\isasymlongleftrightarrow}\ {\isacharparenleft}{\kern0pt}Q\ {\isasymlongrightarrow}\ S{\isacharparenright}{\kern0pt}{\isachardoublequoteclose}%
\isadelimproof
\ %
\endisadelimproof
%
\isatagproof
\isacommand{by}\isamarkupfalse%
\ auto%
\endisatagproof
{\isafoldproof}%
%
\isadelimproof
%
\endisadelimproof
\isanewline
\isacommand{lemma}\isamarkupfalse%
\ imp{\isacharunderscore}{\kern0pt}iff{\isadigit{2}}\ {\isacharcolon}{\kern0pt}\ {\isachardoublequoteopen}{\isasymAnd}P\ Q\ R\ S{\isachardot}{\kern0pt}\ P\ {\isasymlongleftrightarrow}\ Q\ {\isasymLongrightarrow}\ {\isacharparenleft}{\kern0pt}Q\ {\isasymLongrightarrow}\ R\ {\isasymlongleftrightarrow}\ S{\isacharparenright}{\kern0pt}\ {\isasymLongrightarrow}\ {\isacharparenleft}{\kern0pt}P\ {\isasymlongrightarrow}\ R{\isacharparenright}{\kern0pt}\ {\isasymlongleftrightarrow}\ {\isacharparenleft}{\kern0pt}Q\ {\isasymlongrightarrow}\ S{\isacharparenright}{\kern0pt}{\isachardoublequoteclose}%
\isadelimproof
\ %
\endisadelimproof
%
\isatagproof
\isacommand{by}\isamarkupfalse%
\ auto%
\endisatagproof
{\isafoldproof}%
%
\isadelimproof
%
\endisadelimproof
\ \ \isanewline
\isacommand{lemma}\isamarkupfalse%
\ notnot{\isacharunderscore}{\kern0pt}iff\ {\isacharcolon}{\kern0pt}\ {\isachardoublequoteopen}{\isasymAnd}P\ Q{\isachardot}{\kern0pt}\ P\ {\isasymlongleftrightarrow}\ Q\ {\isasymLongrightarrow}\ {\isasymnot}P\ {\isasymlongleftrightarrow}\ {\isasymnot}Q{\isachardoublequoteclose}%
\isadelimproof
\ %
\endisadelimproof
%
\isatagproof
\isacommand{by}\isamarkupfalse%
\ auto%
\endisatagproof
{\isafoldproof}%
%
\isadelimproof
%
\endisadelimproof
\isanewline
\isanewline
\isacommand{lemma}\isamarkupfalse%
\ max{\isacharunderscore}{\kern0pt}le{\isadigit{1}}\ {\isacharcolon}{\kern0pt}\ {\isachardoublequoteopen}Ord{\isacharparenleft}{\kern0pt}a{\isacharparenright}{\kern0pt}\ {\isasymLongrightarrow}\ Ord{\isacharparenleft}{\kern0pt}b{\isacharparenright}{\kern0pt}\ {\isasymLongrightarrow}\ a\ {\isasymle}\ a\ {\isasymunion}\ b{\isachardoublequoteclose}\ \isanewline
%
\isadelimproof
\ \ %
\endisadelimproof
%
\isatagproof
\isacommand{using}\isamarkupfalse%
\ le{\isacharunderscore}{\kern0pt}Un{\isacharunderscore}{\kern0pt}iff\ le{\isacharunderscore}{\kern0pt}refl\ \isacommand{by}\isamarkupfalse%
\ auto%
\endisatagproof
{\isafoldproof}%
%
\isadelimproof
%
\endisadelimproof
\isanewline
\isanewline
\isacommand{lemma}\isamarkupfalse%
\ max{\isacharunderscore}{\kern0pt}le{\isadigit{2}}\ {\isacharcolon}{\kern0pt}\ {\isachardoublequoteopen}Ord{\isacharparenleft}{\kern0pt}a{\isacharparenright}{\kern0pt}\ {\isasymLongrightarrow}\ Ord{\isacharparenleft}{\kern0pt}b{\isacharparenright}{\kern0pt}\ {\isasymLongrightarrow}\ b\ {\isasymle}\ a\ {\isasymunion}\ b{\isachardoublequoteclose}\ \isanewline
%
\isadelimproof
\ \ %
\endisadelimproof
%
\isatagproof
\isacommand{using}\isamarkupfalse%
\ le{\isacharunderscore}{\kern0pt}Un{\isacharunderscore}{\kern0pt}iff\ le{\isacharunderscore}{\kern0pt}refl\ \isacommand{by}\isamarkupfalse%
\ auto%
\endisatagproof
{\isafoldproof}%
%
\isadelimproof
%
\endisadelimproof
\ \ \isanewline
\isanewline
\isacommand{lemma}\isamarkupfalse%
\ zero{\isacharunderscore}{\kern0pt}le\ {\isacharcolon}{\kern0pt}\ {\isachardoublequoteopen}{\isasymAnd}i{\isachardot}{\kern0pt}\ Ord{\isacharparenleft}{\kern0pt}i{\isacharparenright}{\kern0pt}\ {\isasymLongrightarrow}\ {\isadigit{0}}\ {\isasymle}\ i{\isachardoublequoteclose}\ \isanewline
%
\isadelimproof
\ \ %
\endisadelimproof
%
\isatagproof
\isacommand{apply}\isamarkupfalse%
{\isacharparenleft}{\kern0pt}rule\ mp{\isacharparenright}{\kern0pt}\isanewline
\ \ \isacommand{apply}\isamarkupfalse%
{\isacharparenleft}{\kern0pt}rule{\isacharunderscore}{\kern0pt}tac\ P{\isacharequal}{\kern0pt}{\isachardoublequoteopen}{\isasymlambda}i{\isachardot}{\kern0pt}\ Ord{\isacharparenleft}{\kern0pt}i{\isacharparenright}{\kern0pt}\ {\isasymlongrightarrow}\ {\isadigit{0}}\ {\isasymle}\ i{\isachardoublequoteclose}\ \isakeyword{in}\ eps{\isacharunderscore}{\kern0pt}induct{\isacharcomma}{\kern0pt}\ rule\ impI{\isacharparenright}{\kern0pt}\isanewline
\ \ \isacommand{apply}\isamarkupfalse%
{\isacharparenleft}{\kern0pt}rename{\isacharunderscore}{\kern0pt}tac\ i\ x{\isacharcomma}{\kern0pt}\ case{\isacharunderscore}{\kern0pt}tac\ {\isachardoublequoteopen}{\isasymexists}y{\isachardot}{\kern0pt}\ y\ {\isasymin}\ x{\isachardoublequoteclose}{\isacharparenright}{\kern0pt}\isanewline
\ \ \ \isacommand{apply}\isamarkupfalse%
\ clarify\ \isanewline
\ \ \ \ \isacommand{apply}\isamarkupfalse%
{\isacharparenleft}{\kern0pt}rule\ lt{\isacharunderscore}{\kern0pt}succ{\isacharunderscore}{\kern0pt}lt{\isacharcomma}{\kern0pt}\ simp{\isacharparenright}{\kern0pt}\isanewline
\ \ \ \ \isacommand{apply}\isamarkupfalse%
{\isacharparenleft}{\kern0pt}rename{\isacharunderscore}{\kern0pt}tac\ i\ x\ y{\isacharcomma}{\kern0pt}\ rule{\isacharunderscore}{\kern0pt}tac\ b{\isacharequal}{\kern0pt}y\ \isakeyword{in}\ le{\isacharunderscore}{\kern0pt}lt{\isacharunderscore}{\kern0pt}lt{\isacharparenright}{\kern0pt}\isanewline
\ \ \isacommand{using}\isamarkupfalse%
\ Ord{\isacharunderscore}{\kern0pt}in{\isacharunderscore}{\kern0pt}Ord\ ltI\ \isanewline
\ \ \ \ \ \isacommand{apply}\isamarkupfalse%
\ auto{\isacharbrackleft}{\kern0pt}{\isadigit{2}}{\isacharbrackright}{\kern0pt}\isanewline
\ \ \ \isacommand{apply}\isamarkupfalse%
{\isacharparenleft}{\kern0pt}rename{\isacharunderscore}{\kern0pt}tac\ i\ x{\isacharcomma}{\kern0pt}\ rule{\isacharunderscore}{\kern0pt}tac\ b{\isacharequal}{\kern0pt}x\ \isakeyword{and}\ a{\isacharequal}{\kern0pt}{\isadigit{0}}\ \isakeyword{in}\ ssubst{\isacharparenright}{\kern0pt}\isanewline
\ \ \isacommand{by}\isamarkupfalse%
\ auto%
\endisatagproof
{\isafoldproof}%
%
\isadelimproof
\isanewline
%
\endisadelimproof
\isanewline
\isacommand{lemma}\isamarkupfalse%
\ Ord{\isacharunderscore}{\kern0pt}un{\isacharunderscore}{\kern0pt}eq{\isadigit{1}}\ {\isacharcolon}{\kern0pt}\ {\isachardoublequoteopen}Ord{\isacharparenleft}{\kern0pt}a{\isacharparenright}{\kern0pt}\ {\isasymLongrightarrow}\ Ord{\isacharparenleft}{\kern0pt}b{\isacharparenright}{\kern0pt}\ {\isasymLongrightarrow}\ b\ {\isasymle}\ a\ {\isasymLongrightarrow}\ a\ {\isasymunion}\ b\ {\isacharequal}{\kern0pt}\ a{\isachardoublequoteclose}\isanewline
%
\isadelimproof
\ \ %
\endisadelimproof
%
\isatagproof
\isacommand{apply}\isamarkupfalse%
{\isacharparenleft}{\kern0pt}rule\ leE{\isacharparenright}{\kern0pt}\ \isanewline
\ \ \ \ \isacommand{apply}\isamarkupfalse%
\ simp\isanewline
\ \ \ \isacommand{apply}\isamarkupfalse%
{\isacharparenleft}{\kern0pt}subst\ Ord{\isacharunderscore}{\kern0pt}Un{\isacharunderscore}{\kern0pt}if{\isacharparenright}{\kern0pt}\ \isanewline
\ \ \isacommand{by}\isamarkupfalse%
\ simp{\isacharunderscore}{\kern0pt}all%
\endisatagproof
{\isafoldproof}%
%
\isadelimproof
\isanewline
%
\endisadelimproof
\isanewline
\isacommand{lemma}\isamarkupfalse%
\ Ord{\isacharunderscore}{\kern0pt}un{\isacharunderscore}{\kern0pt}eq{\isadigit{2}}\ {\isacharcolon}{\kern0pt}\ {\isachardoublequoteopen}Ord{\isacharparenleft}{\kern0pt}a{\isacharparenright}{\kern0pt}\ {\isasymLongrightarrow}\ Ord{\isacharparenleft}{\kern0pt}b{\isacharparenright}{\kern0pt}\ {\isasymLongrightarrow}\ a\ {\isasymle}\ b\ {\isasymLongrightarrow}\ a\ {\isasymunion}\ b\ {\isacharequal}{\kern0pt}\ b{\isachardoublequoteclose}\isanewline
%
\isadelimproof
\ \ %
\endisadelimproof
%
\isatagproof
\isacommand{apply}\isamarkupfalse%
{\isacharparenleft}{\kern0pt}subst\ Un{\isacharunderscore}{\kern0pt}commute{\isacharparenright}{\kern0pt}\isanewline
\ \ \isacommand{apply}\isamarkupfalse%
{\isacharparenleft}{\kern0pt}rule\ Ord{\isacharunderscore}{\kern0pt}un{\isacharunderscore}{\kern0pt}eq{\isadigit{1}}{\isacharparenright}{\kern0pt}\ \isanewline
\ \ \isacommand{by}\isamarkupfalse%
\ auto%
\endisatagproof
{\isafoldproof}%
%
\isadelimproof
\isanewline
%
\endisadelimproof
\isanewline
\isacommand{lemma}\isamarkupfalse%
\ pred{\isacharunderscore}{\kern0pt}le{\isacharprime}{\kern0pt}\ {\isacharcolon}{\kern0pt}\ {\isachardoublequoteopen}a\ {\isasymin}\ nat\ {\isasymLongrightarrow}\ b\ {\isasymin}\ nat\ {\isasymLongrightarrow}\ a\ {\isasymle}\ b\ {\isasymLongrightarrow}\ pred{\isacharparenleft}{\kern0pt}a{\isacharparenright}{\kern0pt}\ {\isasymle}\ pred{\isacharparenleft}{\kern0pt}b{\isacharparenright}{\kern0pt}{\isachardoublequoteclose}\ \isanewline
%
\isadelimproof
\ \ %
\endisadelimproof
%
\isatagproof
\isacommand{apply}\isamarkupfalse%
{\isacharparenleft}{\kern0pt}rule{\isacharunderscore}{\kern0pt}tac\ n{\isacharequal}{\kern0pt}b\ \isakeyword{in}\ natE{\isacharparenright}{\kern0pt}\isanewline
\ \ \ \ \isacommand{apply}\isamarkupfalse%
\ simp\isanewline
\ \ \ \isacommand{apply}\isamarkupfalse%
{\isacharparenleft}{\kern0pt}subgoal{\isacharunderscore}{\kern0pt}tac\ {\isachardoublequoteopen}a\ {\isacharequal}{\kern0pt}\ {\isadigit{0}}{\isachardoublequoteclose}{\isacharcomma}{\kern0pt}\ simp{\isacharcomma}{\kern0pt}\ simp{\isacharcomma}{\kern0pt}\ simp{\isacharparenright}{\kern0pt}\isanewline
\ \ \isacommand{using}\isamarkupfalse%
\ pred{\isacharunderscore}{\kern0pt}le\ \isanewline
\ \ \isacommand{by}\isamarkupfalse%
\ auto%
\endisatagproof
{\isafoldproof}%
%
\isadelimproof
\isanewline
%
\endisadelimproof
\isanewline
\isacommand{lemma}\isamarkupfalse%
\ domain{\isacharunderscore}{\kern0pt}elem{\isacharunderscore}{\kern0pt}rank{\isacharunderscore}{\kern0pt}lt\ {\isacharcolon}{\kern0pt}\ {\isachardoublequoteopen}y\ {\isasymin}\ domain{\isacharparenleft}{\kern0pt}x{\isacharparenright}{\kern0pt}\ {\isasymLongrightarrow}\ rank{\isacharparenleft}{\kern0pt}y{\isacharparenright}{\kern0pt}\ {\isacharless}{\kern0pt}\ rank{\isacharparenleft}{\kern0pt}x{\isacharparenright}{\kern0pt}{\isachardoublequoteclose}\ \isanewline
%
\isadelimproof
%
\endisadelimproof
%
\isatagproof
\isacommand{proof}\isamarkupfalse%
\ {\isacharminus}{\kern0pt}\ \isanewline
\ \ \isacommand{assume}\isamarkupfalse%
\ {\isachardoublequoteopen}y\ {\isasymin}\ domain{\isacharparenleft}{\kern0pt}x{\isacharparenright}{\kern0pt}{\isachardoublequoteclose}\ \isanewline
\ \ \isacommand{then}\isamarkupfalse%
\ \isacommand{obtain}\isamarkupfalse%
\ p\ \isakeyword{where}\ {\isachardoublequoteopen}{\isacharless}{\kern0pt}y{\isacharcomma}{\kern0pt}\ p{\isachargreater}{\kern0pt}\ {\isasymin}\ x{\isachardoublequoteclose}\ \isacommand{by}\isamarkupfalse%
\ auto\ \isanewline
\ \ \isacommand{then}\isamarkupfalse%
\ \isacommand{show}\isamarkupfalse%
\ {\isachardoublequoteopen}rank{\isacharparenleft}{\kern0pt}y{\isacharparenright}{\kern0pt}\ {\isacharless}{\kern0pt}\ rank{\isacharparenleft}{\kern0pt}x{\isacharparenright}{\kern0pt}{\isachardoublequoteclose}\ \isanewline
\ \ \ \ \isacommand{apply}\isamarkupfalse%
{\isacharparenleft}{\kern0pt}rule{\isacharunderscore}{\kern0pt}tac\ j{\isacharequal}{\kern0pt}{\isachardoublequoteopen}rank{\isacharparenleft}{\kern0pt}{\isacharless}{\kern0pt}y{\isacharcomma}{\kern0pt}\ p{\isachargreater}{\kern0pt}{\isacharparenright}{\kern0pt}{\isachardoublequoteclose}\ \isakeyword{in}\ lt{\isacharunderscore}{\kern0pt}trans{\isacharparenright}{\kern0pt}\isanewline
\ \ \ \ \ \isacommand{apply}\isamarkupfalse%
{\isacharparenleft}{\kern0pt}rule\ rank{\isacharunderscore}{\kern0pt}pair{\isadigit{1}}{\isacharcomma}{\kern0pt}\ rule\ rank{\isacharunderscore}{\kern0pt}lt{\isacharcomma}{\kern0pt}\ simp{\isacharparenright}{\kern0pt}\isanewline
\ \ \ \ \isacommand{done}\isamarkupfalse%
\isanewline
\isacommand{qed}\isamarkupfalse%
%
\endisatagproof
{\isafoldproof}%
%
\isadelimproof
\isanewline
%
\endisadelimproof
\isanewline
\isacommand{lemma}\isamarkupfalse%
\ final{\isacharunderscore}{\kern0pt}app{\isacharunderscore}{\kern0pt}notation\ {\isacharcolon}{\kern0pt}\ \isanewline
\ \ \isakeyword{fixes}\ l\ \isakeyword{assumes}\ lin\ {\isacharcolon}{\kern0pt}\ {\isachardoublequoteopen}l\ {\isasymin}\ list{\isacharparenleft}{\kern0pt}A{\isacharparenright}{\kern0pt}{\isachardoublequoteclose}\ \isakeyword{and}\ lnotnil\ {\isacharcolon}{\kern0pt}\ {\isachardoublequoteopen}l\ {\isasymnoteq}\ {\isacharbrackleft}{\kern0pt}{\isacharbrackright}{\kern0pt}{\isachardoublequoteclose}\isanewline
\ \ \isakeyword{shows}\ {\isachardoublequoteopen}{\isasymexists}l{\isacharprime}{\kern0pt}\ {\isasymin}\ list{\isacharparenleft}{\kern0pt}A{\isacharparenright}{\kern0pt}{\isachardot}{\kern0pt}\ {\isasymexists}\ a\ {\isasymin}\ A{\isachardot}{\kern0pt}\ l\ {\isacharequal}{\kern0pt}\ l{\isacharprime}{\kern0pt}\ {\isacharat}{\kern0pt}\ {\isacharbrackleft}{\kern0pt}a{\isacharbrackright}{\kern0pt}{\isachardoublequoteclose}\ \isanewline
%
\isadelimproof
\isanewline
\ \ %
\endisadelimproof
%
\isatagproof
\isacommand{apply}\isamarkupfalse%
{\isacharparenleft}{\kern0pt}rule{\isacharunderscore}{\kern0pt}tac\ P{\isacharequal}{\kern0pt}{\isachardoublequoteopen}l\ {\isasymnoteq}\ {\isacharbrackleft}{\kern0pt}{\isacharbrackright}{\kern0pt}\ {\isasymlongrightarrow}\ {\isacharparenleft}{\kern0pt}{\isasymexists}l{\isacharprime}{\kern0pt}\ {\isasymin}\ list{\isacharparenleft}{\kern0pt}A{\isacharparenright}{\kern0pt}{\isachardot}{\kern0pt}\ {\isasymexists}\ a\ {\isasymin}\ A{\isachardot}{\kern0pt}\ l\ {\isacharequal}{\kern0pt}\ l{\isacharprime}{\kern0pt}\ {\isacharat}{\kern0pt}\ {\isacharbrackleft}{\kern0pt}a{\isacharbrackright}{\kern0pt}{\isacharparenright}{\kern0pt}{\isachardoublequoteclose}\ \isakeyword{in}\ mp{\isacharparenright}{\kern0pt}\ \ \isanewline
\ \ \isacommand{using}\isamarkupfalse%
\ lnotnil\ \isacommand{apply}\isamarkupfalse%
\ simp\isanewline
\ \ \isacommand{using}\isamarkupfalse%
\ lin\isanewline
\isacommand{proof}\isamarkupfalse%
\ {\isacharparenleft}{\kern0pt}induct{\isacharparenright}{\kern0pt}\isanewline
\ \ \isacommand{case}\isamarkupfalse%
\ Nil\ \isanewline
\ \ \isacommand{then}\isamarkupfalse%
\ \isacommand{show}\isamarkupfalse%
\ {\isacharquery}{\kern0pt}case\ \isacommand{by}\isamarkupfalse%
\ auto\isanewline
\isacommand{next}\isamarkupfalse%
\isanewline
\ \ \isacommand{case}\isamarkupfalse%
\ {\isacharparenleft}{\kern0pt}Cons\ a\ l{\isacharprime}{\kern0pt}{\isacharparenright}{\kern0pt}\isanewline
\ \ \isacommand{then}\isamarkupfalse%
\ \isacommand{show}\isamarkupfalse%
\ {\isacharquery}{\kern0pt}case\isanewline
\ \ \ \ \isacommand{apply}\isamarkupfalse%
{\isacharparenleft}{\kern0pt}cases\ {\isachardoublequoteopen}l{\isacharprime}{\kern0pt}\ {\isacharequal}{\kern0pt}\ {\isacharbrackleft}{\kern0pt}{\isacharbrackright}{\kern0pt}{\isachardoublequoteclose}{\isacharparenright}{\kern0pt}\ \isacommand{apply}\isamarkupfalse%
\ clarify\ \isacommand{apply}\isamarkupfalse%
{\isacharparenleft}{\kern0pt}rule{\isacharunderscore}{\kern0pt}tac\ x{\isacharequal}{\kern0pt}{\isachardoublequoteopen}{\isacharbrackleft}{\kern0pt}{\isacharbrackright}{\kern0pt}{\isachardoublequoteclose}\ \isakeyword{in}\ bexI{\isacharparenright}{\kern0pt}\ \isacommand{apply}\isamarkupfalse%
{\isacharparenleft}{\kern0pt}rule{\isacharunderscore}{\kern0pt}tac\ x{\isacharequal}{\kern0pt}a\ \isakeyword{in}\ bexI{\isacharparenright}{\kern0pt}\ \isacommand{apply}\isamarkupfalse%
\ simp{\isacharunderscore}{\kern0pt}all\ \isanewline
\ \ \isacommand{proof}\isamarkupfalse%
\ {\isacharminus}{\kern0pt}\ \isanewline
\ \ \ \ \isacommand{assume}\isamarkupfalse%
\ {\isachardoublequoteopen}{\isasymexists}l{\isacharprime}{\kern0pt}a{\isasymin}list{\isacharparenleft}{\kern0pt}A{\isacharparenright}{\kern0pt}{\isachardot}{\kern0pt}\ {\isasymexists}a{\isasymin}A{\isachardot}{\kern0pt}\ l{\isacharprime}{\kern0pt}\ {\isacharequal}{\kern0pt}\ l{\isacharprime}{\kern0pt}a\ {\isacharat}{\kern0pt}\ {\isacharbrackleft}{\kern0pt}a{\isacharbrackright}{\kern0pt}{\isachardoublequoteclose}\ \isanewline
\ \ \ \ \isacommand{then}\isamarkupfalse%
\ \isacommand{obtain}\isamarkupfalse%
\ l{\isacharprime}{\kern0pt}{\isacharprime}{\kern0pt}\ t\ \isakeyword{where}\ {\isachardoublequoteopen}t\ {\isasymin}\ A{\isachardoublequoteclose}\ {\isachardoublequoteopen}l{\isacharprime}{\kern0pt}{\isacharprime}{\kern0pt}\ {\isasymin}\ list{\isacharparenleft}{\kern0pt}A{\isacharparenright}{\kern0pt}{\isachardoublequoteclose}\ {\isachardoublequoteopen}l{\isacharprime}{\kern0pt}\ {\isacharequal}{\kern0pt}\ l{\isacharprime}{\kern0pt}{\isacharprime}{\kern0pt}\ {\isacharat}{\kern0pt}\ {\isacharbrackleft}{\kern0pt}t{\isacharbrackright}{\kern0pt}{\isachardoublequoteclose}\ \isacommand{by}\isamarkupfalse%
\ auto\ \isanewline
\ \ \ \ \isacommand{then}\isamarkupfalse%
\ \isacommand{show}\isamarkupfalse%
\ {\isachardoublequoteopen}{\isasymexists}l{\isacharprime}{\kern0pt}a{\isasymin}list{\isacharparenleft}{\kern0pt}A{\isacharparenright}{\kern0pt}{\isachardot}{\kern0pt}\ {\isasymexists}aa{\isasymin}A{\isachardot}{\kern0pt}\ Cons{\isacharparenleft}{\kern0pt}a{\isacharcomma}{\kern0pt}\ l{\isacharprime}{\kern0pt}{\isacharparenright}{\kern0pt}\ {\isacharequal}{\kern0pt}\ l{\isacharprime}{\kern0pt}a\ {\isacharat}{\kern0pt}\ {\isacharbrackleft}{\kern0pt}aa{\isacharbrackright}{\kern0pt}{\isachardoublequoteclose}\ \isanewline
\ \ \ \ \ \ \isacommand{apply}\isamarkupfalse%
{\isacharparenleft}{\kern0pt}rule{\isacharunderscore}{\kern0pt}tac\ x{\isacharequal}{\kern0pt}{\isachardoublequoteopen}Cons{\isacharparenleft}{\kern0pt}a{\isacharcomma}{\kern0pt}\ l{\isacharprime}{\kern0pt}{\isacharprime}{\kern0pt}{\isacharparenright}{\kern0pt}{\isachardoublequoteclose}\ \isakeyword{in}\ bexI{\isacharparenright}{\kern0pt}\ \isanewline
\ \ \ \ \ \ \isacommand{apply}\isamarkupfalse%
{\isacharparenleft}{\kern0pt}rule{\isacharunderscore}{\kern0pt}tac\ x{\isacharequal}{\kern0pt}t\ \isakeyword{in}\ bexI{\isacharparenright}{\kern0pt}\ \isacommand{using}\isamarkupfalse%
\ {\isacartoucheopen}a\ {\isasymin}\ A{\isacartoucheclose}\ \isacommand{apply}\isamarkupfalse%
\ simp{\isacharunderscore}{\kern0pt}all\ \isacommand{done}\isamarkupfalse%
\ \isanewline
\ \ \isacommand{qed}\isamarkupfalse%
\isanewline
\isacommand{qed}\isamarkupfalse%
%
\endisatagproof
{\isafoldproof}%
%
\isadelimproof
\isanewline
%
\endisadelimproof
\isanewline
\isacommand{lemma}\isamarkupfalse%
\ length{\isadigit{1}}{\isacharunderscore}{\kern0pt}notation\ {\isacharcolon}{\kern0pt}\ \isanewline
\ \ \isakeyword{fixes}\ l\ \isakeyword{assumes}\ {\isachardoublequoteopen}l\ {\isasymin}\ list{\isacharparenleft}{\kern0pt}A{\isacharparenright}{\kern0pt}{\isachardoublequoteclose}\ \ {\isachardoublequoteopen}length{\isacharparenleft}{\kern0pt}l{\isacharparenright}{\kern0pt}\ {\isacharequal}{\kern0pt}\ {\isadigit{1}}{\isachardoublequoteclose}\ \isanewline
\ \ \isakeyword{shows}\ {\isachardoublequoteopen}{\isasymexists}a\ {\isasymin}\ A{\isachardot}{\kern0pt}\ l\ {\isacharequal}{\kern0pt}\ {\isacharbrackleft}{\kern0pt}a{\isacharbrackright}{\kern0pt}{\isachardoublequoteclose}\ \isanewline
%
\isadelimproof
%
\endisadelimproof
%
\isatagproof
\isacommand{proof}\isamarkupfalse%
\ {\isacharminus}{\kern0pt}\ \isanewline
\ \ \isacommand{have}\isamarkupfalse%
\ {\isachardoublequoteopen}l\ {\isasymnoteq}\ Nil{\isachardoublequoteclose}\ \isacommand{using}\isamarkupfalse%
\ assms\ length{\isacharunderscore}{\kern0pt}is{\isacharunderscore}{\kern0pt}{\isadigit{0}}{\isacharunderscore}{\kern0pt}iff\ \isacommand{by}\isamarkupfalse%
\ auto\ \isanewline
\ \ \isacommand{then}\isamarkupfalse%
\ \isacommand{have}\isamarkupfalse%
\ {\isachardoublequoteopen}{\isasymexists}l{\isacharprime}{\kern0pt}\ {\isasymin}\ list{\isacharparenleft}{\kern0pt}A{\isacharparenright}{\kern0pt}{\isachardot}{\kern0pt}\ {\isasymexists}\ a\ {\isasymin}\ A{\isachardot}{\kern0pt}\ l\ {\isacharequal}{\kern0pt}\ l{\isacharprime}{\kern0pt}\ {\isacharat}{\kern0pt}\ {\isacharbrackleft}{\kern0pt}a{\isacharbrackright}{\kern0pt}{\isachardoublequoteclose}\ \isacommand{using}\isamarkupfalse%
\ final{\isacharunderscore}{\kern0pt}app{\isacharunderscore}{\kern0pt}notation\ assms\ \isacommand{by}\isamarkupfalse%
\ auto\ \isanewline
\ \ \isacommand{then}\isamarkupfalse%
\ \isacommand{obtain}\isamarkupfalse%
\ l{\isacharprime}{\kern0pt}\ a\ \isakeyword{where}\ H{\isacharcolon}{\kern0pt}\ {\isachardoublequoteopen}l{\isacharprime}{\kern0pt}\ {\isasymin}\ list{\isacharparenleft}{\kern0pt}A{\isacharparenright}{\kern0pt}{\isachardoublequoteclose}\ {\isachardoublequoteopen}a\ {\isasymin}\ A{\isachardoublequoteclose}\ {\isachardoublequoteopen}l\ {\isacharequal}{\kern0pt}\ l{\isacharprime}{\kern0pt}\ {\isacharat}{\kern0pt}\ {\isacharbrackleft}{\kern0pt}a{\isacharbrackright}{\kern0pt}{\isachardoublequoteclose}\ \isacommand{by}\isamarkupfalse%
\ auto\ \isanewline
\ \ \isacommand{then}\isamarkupfalse%
\ \isacommand{have}\isamarkupfalse%
\ {\isachardoublequoteopen}l\ {\isacharequal}{\kern0pt}\ {\isacharbrackleft}{\kern0pt}a{\isacharbrackright}{\kern0pt}{\isachardoublequoteclose}\ \isacommand{using}\isamarkupfalse%
\ assms\ \isacommand{by}\isamarkupfalse%
\ auto\isanewline
\ \ \isacommand{then}\isamarkupfalse%
\ \isacommand{show}\isamarkupfalse%
\ {\isacharquery}{\kern0pt}thesis\ \isacommand{using}\isamarkupfalse%
\ H\ \isacommand{by}\isamarkupfalse%
\ auto\isanewline
\isacommand{qed}\isamarkupfalse%
%
\endisatagproof
{\isafoldproof}%
%
\isadelimproof
\ \isanewline
%
\endisadelimproof
\isanewline
\isacommand{lemma}\isamarkupfalse%
\ not{\isacharunderscore}{\kern0pt}nil{\isacharunderscore}{\kern0pt}obtain{\isacharunderscore}{\kern0pt}hd{\isacharunderscore}{\kern0pt}tl\ {\isacharcolon}{\kern0pt}\ \isanewline
\ \ \isakeyword{fixes}\ A\ l\ \isakeyword{assumes}\ {\isachardoublequoteopen}l\ {\isasymin}\ list{\isacharparenleft}{\kern0pt}A{\isacharparenright}{\kern0pt}{\isachardoublequoteclose}\ {\isachardoublequoteopen}l\ {\isasymnoteq}\ Nil{\isachardoublequoteclose}\ \isanewline
\ \ \isakeyword{shows}\ {\isachardoublequoteopen}{\isasymexists}hd\ {\isasymin}\ A{\isachardot}{\kern0pt}\ {\isasymexists}tl\ {\isasymin}\ list{\isacharparenleft}{\kern0pt}A{\isacharparenright}{\kern0pt}{\isachardot}{\kern0pt}\ l\ {\isacharequal}{\kern0pt}\ Cons{\isacharparenleft}{\kern0pt}hd{\isacharcomma}{\kern0pt}\ tl{\isacharparenright}{\kern0pt}{\isachardoublequoteclose}\ \isanewline
%
\isadelimproof
\isanewline
\ \ %
\endisadelimproof
%
\isatagproof
\isacommand{using}\isamarkupfalse%
\ {\isacartoucheopen}l\ {\isasymin}\ list{\isacharparenleft}{\kern0pt}A{\isacharparenright}{\kern0pt}{\isacartoucheclose}\ \isacommand{apply}\isamarkupfalse%
\ cases\ \isacommand{using}\isamarkupfalse%
\ assms\ \isacommand{apply}\isamarkupfalse%
\ simp\ \isanewline
\ \ \isacommand{by}\isamarkupfalse%
\ auto%
\endisatagproof
{\isafoldproof}%
%
\isadelimproof
\ \isanewline
%
\endisadelimproof
\isanewline
\isacommand{lemma}\isamarkupfalse%
\ append{\isacharunderscore}{\kern0pt}notation\ {\isacharcolon}{\kern0pt}\ \isanewline
\ \ \isakeyword{fixes}\ l\ n\ \isakeyword{assumes}\ {\isachardoublequoteopen}l\ {\isasymin}\ list{\isacharparenleft}{\kern0pt}A{\isacharparenright}{\kern0pt}{\isachardoublequoteclose}\ {\isachardoublequoteopen}n\ {\isasymin}\ nat{\isachardoublequoteclose}\ {\isachardoublequoteopen}n\ {\isasymle}\ length{\isacharparenleft}{\kern0pt}l{\isacharparenright}{\kern0pt}{\isachardoublequoteclose}\ \isanewline
\ \ \isakeyword{shows}\ {\isachardoublequoteopen}{\isasymexists}l{\isadigit{1}}\ {\isasymin}\ list{\isacharparenleft}{\kern0pt}A{\isacharparenright}{\kern0pt}{\isachardot}{\kern0pt}\ {\isasymexists}l{\isadigit{2}}\ {\isasymin}\ list{\isacharparenleft}{\kern0pt}A{\isacharparenright}{\kern0pt}{\isachardot}{\kern0pt}\ length{\isacharparenleft}{\kern0pt}l{\isadigit{1}}{\isacharparenright}{\kern0pt}\ {\isacharequal}{\kern0pt}\ n\ {\isasymand}\ l\ {\isacharequal}{\kern0pt}\ l{\isadigit{1}}\ {\isacharat}{\kern0pt}\ l{\isadigit{2}}{\isachardoublequoteclose}\ \isanewline
%
\isadelimproof
\ \ %
\endisadelimproof
%
\isatagproof
\isacommand{apply}\isamarkupfalse%
{\isacharparenleft}{\kern0pt}rule{\isacharunderscore}{\kern0pt}tac\ P{\isacharequal}{\kern0pt}{\isachardoublequoteopen}{\isasymforall}n\ {\isasymin}\ nat{\isachardot}{\kern0pt}\ n\ {\isasymle}\ length{\isacharparenleft}{\kern0pt}l{\isacharparenright}{\kern0pt}\ {\isasymlongrightarrow}\ {\isacharparenleft}{\kern0pt}\ {\isasymexists}l{\isadigit{1}}\ {\isasymin}\ list{\isacharparenleft}{\kern0pt}A{\isacharparenright}{\kern0pt}{\isachardot}{\kern0pt}\ {\isasymexists}l{\isadigit{2}}\ {\isasymin}\ list{\isacharparenleft}{\kern0pt}A{\isacharparenright}{\kern0pt}{\isachardot}{\kern0pt}\ length{\isacharparenleft}{\kern0pt}l{\isadigit{1}}{\isacharparenright}{\kern0pt}\ {\isacharequal}{\kern0pt}\ n\ {\isasymand}\ l\ {\isacharequal}{\kern0pt}\ l{\isadigit{1}}\ {\isacharat}{\kern0pt}\ l{\isadigit{2}}\ {\isacharparenright}{\kern0pt}{\isachardoublequoteclose}\ \isakeyword{in}\ mp{\isacharparenright}{\kern0pt}\ \isanewline
\ \ \isacommand{using}\isamarkupfalse%
\ assms\ \isacommand{apply}\isamarkupfalse%
\ simp\ \isacommand{apply}\isamarkupfalse%
\ clarify\ \isanewline
\ \ \isacommand{using}\isamarkupfalse%
\ {\isacartoucheopen}l\ {\isasymin}\ list{\isacharparenleft}{\kern0pt}A{\isacharparenright}{\kern0pt}{\isacartoucheclose}\isanewline
\isacommand{proof}\isamarkupfalse%
\ {\isacharparenleft}{\kern0pt}induct{\isacharparenright}{\kern0pt}\isanewline
\ \ \isacommand{case}\isamarkupfalse%
\ Nil\isanewline
\ \ \isacommand{then}\isamarkupfalse%
\ \isacommand{have}\isamarkupfalse%
\ {\isachardoublequoteopen}n\ {\isacharequal}{\kern0pt}\ {\isadigit{0}}{\isachardoublequoteclose}\ \isacommand{by}\isamarkupfalse%
\ auto\ \ \isanewline
\ \ \isacommand{then}\isamarkupfalse%
\ \isacommand{show}\isamarkupfalse%
\ {\isacharquery}{\kern0pt}case\ \isanewline
\ \ \ \ \isacommand{apply}\isamarkupfalse%
{\isacharparenleft}{\kern0pt}rule{\isacharunderscore}{\kern0pt}tac\ x{\isacharequal}{\kern0pt}Nil\ \isakeyword{in}\ bexI{\isacharparenright}{\kern0pt}\isanewline
\ \ \ \ \isacommand{apply}\isamarkupfalse%
{\isacharparenleft}{\kern0pt}rule{\isacharunderscore}{\kern0pt}tac\ x{\isacharequal}{\kern0pt}Nil\ \isakeyword{in}\ bexI{\isacharparenright}{\kern0pt}\isanewline
\ \ \ \ \isacommand{by}\isamarkupfalse%
\ auto\ \isanewline
\isacommand{next}\isamarkupfalse%
\isanewline
\ \ \isacommand{case}\isamarkupfalse%
\ {\isacharparenleft}{\kern0pt}Cons\ hd\ tl{\isacharparenright}{\kern0pt}\isanewline
\ \ \isacommand{then}\isamarkupfalse%
\ \isacommand{show}\isamarkupfalse%
\ {\isacharquery}{\kern0pt}case\ \isanewline
\ \ \ \ \isacommand{apply}\isamarkupfalse%
{\isacharparenleft}{\kern0pt}rule{\isacharunderscore}{\kern0pt}tac\ n{\isacharequal}{\kern0pt}n\ \isakeyword{in}\ natE{\isacharparenright}{\kern0pt}\ \isacommand{apply}\isamarkupfalse%
\ simp\ \isacommand{apply}\isamarkupfalse%
\ simp\isanewline
\ \ \isacommand{proof}\isamarkupfalse%
\ {\isacharminus}{\kern0pt}\ \isanewline
\ \ \ \ \isacommand{fix}\isamarkupfalse%
\ n{\isacharprime}{\kern0pt}\ \isacommand{assume}\isamarkupfalse%
\ assms{\isadigit{1}}{\isacharcolon}{\kern0pt}\ {\isachardoublequoteopen}n\ {\isacharequal}{\kern0pt}\ succ{\isacharparenleft}{\kern0pt}n{\isacharprime}{\kern0pt}{\isacharparenright}{\kern0pt}{\isachardoublequoteclose}\ {\isachardoublequoteopen}n{\isacharprime}{\kern0pt}\ {\isasymin}\ nat{\isachardoublequoteclose}\ {\isachardoublequoteopen}n\ {\isasymle}\ length{\isacharparenleft}{\kern0pt}Cons{\isacharparenleft}{\kern0pt}hd{\isacharcomma}{\kern0pt}\ tl{\isacharparenright}{\kern0pt}{\isacharparenright}{\kern0pt}{\isachardoublequoteclose}\ {\isachardoublequoteopen}hd\ {\isasymin}\ A\ {\isachardoublequoteclose}\ {\isachardoublequoteopen}tl\ {\isasymin}\ list{\isacharparenleft}{\kern0pt}A{\isacharparenright}{\kern0pt}{\isachardoublequoteclose}\ \isanewline
\ \ \ \ \ \ {\isachardoublequoteopen}{\isacharparenleft}{\kern0pt}{\isasymAnd}n{\isachardot}{\kern0pt}\ n\ {\isasymin}\ nat\ {\isasymLongrightarrow}\ n\ {\isasymle}\ length{\isacharparenleft}{\kern0pt}tl{\isacharparenright}{\kern0pt}\ {\isasymLongrightarrow}\ {\isasymexists}l{\isadigit{1}}{\isasymin}list{\isacharparenleft}{\kern0pt}A{\isacharparenright}{\kern0pt}{\isachardot}{\kern0pt}\ {\isasymexists}l{\isadigit{2}}{\isasymin}list{\isacharparenleft}{\kern0pt}A{\isacharparenright}{\kern0pt}{\isachardot}{\kern0pt}\ length{\isacharparenleft}{\kern0pt}l{\isadigit{1}}{\isacharparenright}{\kern0pt}\ {\isacharequal}{\kern0pt}\ n\ {\isasymand}\ tl\ {\isacharequal}{\kern0pt}\ l{\isadigit{1}}\ {\isacharat}{\kern0pt}\ l{\isadigit{2}}{\isacharparenright}{\kern0pt}{\isachardoublequoteclose}\ \isanewline
\isanewline
\ \ \ \ \isacommand{then}\isamarkupfalse%
\ \isacommand{have}\isamarkupfalse%
\ {\isachardoublequoteopen}n{\isacharprime}{\kern0pt}\ {\isasymle}\ length{\isacharparenleft}{\kern0pt}tl{\isacharparenright}{\kern0pt}{\isachardoublequoteclose}\ \isacommand{by}\isamarkupfalse%
\ auto\ \isanewline
\ \ \ \ \isacommand{then}\isamarkupfalse%
\ \isacommand{have}\isamarkupfalse%
\ {\isachardoublequoteopen}{\isasymexists}l{\isadigit{1}}{\isasymin}list{\isacharparenleft}{\kern0pt}A{\isacharparenright}{\kern0pt}{\isachardot}{\kern0pt}\ {\isasymexists}l{\isadigit{2}}{\isasymin}list{\isacharparenleft}{\kern0pt}A{\isacharparenright}{\kern0pt}{\isachardot}{\kern0pt}\ length{\isacharparenleft}{\kern0pt}l{\isadigit{1}}{\isacharparenright}{\kern0pt}\ {\isacharequal}{\kern0pt}\ n{\isacharprime}{\kern0pt}\ {\isasymand}\ tl\ {\isacharequal}{\kern0pt}\ l{\isadigit{1}}\ {\isacharat}{\kern0pt}\ l{\isadigit{2}}{\isachardoublequoteclose}\ \isacommand{using}\isamarkupfalse%
\ assms{\isadigit{1}}\ \isacommand{by}\isamarkupfalse%
\ auto\ \isanewline
\ \ \ \ \isacommand{then}\isamarkupfalse%
\ \isacommand{obtain}\isamarkupfalse%
\ l{\isadigit{1}}\ l{\isadigit{2}}\ \isakeyword{where}\ H\ {\isacharcolon}{\kern0pt}\ {\isachardoublequoteopen}l{\isadigit{1}}\ {\isasymin}\ list{\isacharparenleft}{\kern0pt}A{\isacharparenright}{\kern0pt}{\isachardoublequoteclose}\ {\isachardoublequoteopen}l{\isadigit{2}}\ {\isasymin}\ list{\isacharparenleft}{\kern0pt}A{\isacharparenright}{\kern0pt}{\isachardoublequoteclose}\ {\isachardoublequoteopen}length{\isacharparenleft}{\kern0pt}l{\isadigit{1}}{\isacharparenright}{\kern0pt}\ {\isacharequal}{\kern0pt}\ n{\isacharprime}{\kern0pt}{\isachardoublequoteclose}\ {\isachardoublequoteopen}tl\ {\isacharequal}{\kern0pt}\ l{\isadigit{1}}\ {\isacharat}{\kern0pt}\ l{\isadigit{2}}{\isachardoublequoteclose}\ \isacommand{by}\isamarkupfalse%
\ auto\ \isanewline
\ \ \ \ \isacommand{then}\isamarkupfalse%
\ \isacommand{show}\isamarkupfalse%
\ {\isachardoublequoteopen}{\isasymexists}l{\isadigit{1}}{\isasymin}list{\isacharparenleft}{\kern0pt}A{\isacharparenright}{\kern0pt}{\isachardot}{\kern0pt}\ {\isasymexists}l{\isadigit{2}}{\isasymin}list{\isacharparenleft}{\kern0pt}A{\isacharparenright}{\kern0pt}{\isachardot}{\kern0pt}\ length{\isacharparenleft}{\kern0pt}l{\isadigit{1}}{\isacharparenright}{\kern0pt}\ {\isacharequal}{\kern0pt}\ n\ {\isasymand}\ Cons{\isacharparenleft}{\kern0pt}hd{\isacharcomma}{\kern0pt}\ tl{\isacharparenright}{\kern0pt}\ {\isacharequal}{\kern0pt}\ l{\isadigit{1}}\ {\isacharat}{\kern0pt}\ l{\isadigit{2}}{\isachardoublequoteclose}\ \isanewline
\ \ \ \ \ \ \isacommand{apply}\isamarkupfalse%
{\isacharparenleft}{\kern0pt}rule{\isacharunderscore}{\kern0pt}tac\ x{\isacharequal}{\kern0pt}{\isachardoublequoteopen}Cons{\isacharparenleft}{\kern0pt}hd{\isacharcomma}{\kern0pt}\ l{\isadigit{1}}{\isacharparenright}{\kern0pt}{\isachardoublequoteclose}\ \isakeyword{in}\ bexI{\isacharparenright}{\kern0pt}\isanewline
\ \ \ \ \ \ \isacommand{apply}\isamarkupfalse%
{\isacharparenleft}{\kern0pt}rule{\isacharunderscore}{\kern0pt}tac\ x{\isacharequal}{\kern0pt}l{\isadigit{2}}\ \isakeyword{in}\ bexI{\isacharparenright}{\kern0pt}\ \isanewline
\ \ \ \ \ \ \isacommand{using}\isamarkupfalse%
\ assms{\isadigit{1}}\ H\ \isacommand{by}\isamarkupfalse%
\ auto\ \isanewline
\ \ \isacommand{qed}\isamarkupfalse%
\isanewline
\isacommand{qed}\isamarkupfalse%
%
\endisatagproof
{\isafoldproof}%
%
\isadelimproof
\isanewline
%
\endisadelimproof
\isanewline
\isacommand{lemma}\isamarkupfalse%
\ Pi{\isacharunderscore}{\kern0pt}memberI\ {\isacharcolon}{\kern0pt}\ {\isachardoublequoteopen}relation{\isacharparenleft}{\kern0pt}f{\isacharparenright}{\kern0pt}\ {\isasymLongrightarrow}\ function{\isacharparenleft}{\kern0pt}f{\isacharparenright}{\kern0pt}\ {\isasymLongrightarrow}\ A\ {\isacharequal}{\kern0pt}\ domain{\isacharparenleft}{\kern0pt}f{\isacharparenright}{\kern0pt}\ {\isasymLongrightarrow}\ range{\isacharparenleft}{\kern0pt}f{\isacharparenright}{\kern0pt}\ {\isasymsubseteq}\ B\ {\isasymLongrightarrow}\ f\ {\isasymin}\ Pi{\isacharparenleft}{\kern0pt}A{\isacharcomma}{\kern0pt}\ {\isasymlambda}{\isacharunderscore}{\kern0pt}{\isachardot}{\kern0pt}\ B{\isacharparenright}{\kern0pt}{\isachardoublequoteclose}\isanewline
%
\isadelimproof
\ \ %
\endisadelimproof
%
\isatagproof
\isacommand{apply}\isamarkupfalse%
{\isacharparenleft}{\kern0pt}rule{\isacharunderscore}{\kern0pt}tac\ iffD{\isadigit{2}}{\isacharparenright}{\kern0pt}\ \isacommand{apply}\isamarkupfalse%
{\isacharparenleft}{\kern0pt}rule{\isacharunderscore}{\kern0pt}tac\ Pi{\isacharunderscore}{\kern0pt}iff{\isacharparenright}{\kern0pt}\ \isacommand{apply}\isamarkupfalse%
\ simp\ \isacommand{apply}\isamarkupfalse%
{\isacharparenleft}{\kern0pt}rule\ subsetI{\isacharparenright}{\kern0pt}\ \ \ \isanewline
\isacommand{proof}\isamarkupfalse%
\ {\isacharminus}{\kern0pt}\ \isanewline
\ \ \isacommand{fix}\isamarkupfalse%
\ x\ \isacommand{assume}\isamarkupfalse%
\ assms\ {\isacharcolon}{\kern0pt}\ {\isachardoublequoteopen}x\ {\isasymin}\ f{\isachardoublequoteclose}\ {\isachardoublequoteopen}relation{\isacharparenleft}{\kern0pt}f{\isacharparenright}{\kern0pt}{\isachardoublequoteclose}\ {\isachardoublequoteopen}function{\isacharparenleft}{\kern0pt}f{\isacharparenright}{\kern0pt}{\isachardoublequoteclose}\ {\isachardoublequoteopen}A\ {\isacharequal}{\kern0pt}\ domain{\isacharparenleft}{\kern0pt}f{\isacharparenright}{\kern0pt}{\isachardoublequoteclose}\ {\isachardoublequoteopen}range{\isacharparenleft}{\kern0pt}f{\isacharparenright}{\kern0pt}\ {\isasymsubseteq}\ B{\isachardoublequoteclose}\isanewline
\ \ \isacommand{then}\isamarkupfalse%
\ \isacommand{obtain}\isamarkupfalse%
\ a\ b\ \isakeyword{where}\ abH\ {\isacharcolon}{\kern0pt}\ {\isachardoublequoteopen}x\ {\isacharequal}{\kern0pt}\ {\isacharless}{\kern0pt}a{\isacharcomma}{\kern0pt}\ b{\isachargreater}{\kern0pt}{\isachardoublequoteclose}\ \isacommand{unfolding}\isamarkupfalse%
\ relation{\isacharunderscore}{\kern0pt}def\ \isacommand{by}\isamarkupfalse%
\ auto\ \isanewline
\ \ \isacommand{then}\isamarkupfalse%
\ \isacommand{have}\isamarkupfalse%
\ {\isachardoublequoteopen}a\ {\isasymin}\ A\ {\isasymand}\ b\ {\isasymin}\ B{\isachardoublequoteclose}\ \isacommand{using}\isamarkupfalse%
\ assms\ \isacommand{apply}\isamarkupfalse%
\ auto\ \isacommand{done}\isamarkupfalse%
\ \isanewline
\ \ \isacommand{then}\isamarkupfalse%
\ \isacommand{show}\isamarkupfalse%
\ {\isachardoublequoteopen}x\ {\isasymin}\ domain{\isacharparenleft}{\kern0pt}f{\isacharparenright}{\kern0pt}\ {\isasymtimes}\ B{\isachardoublequoteclose}\ \isacommand{using}\isamarkupfalse%
\ assms\ abH\ \isacommand{by}\isamarkupfalse%
\ auto\isanewline
\isacommand{qed}\isamarkupfalse%
%
\endisatagproof
{\isafoldproof}%
%
\isadelimproof
\isanewline
%
\endisadelimproof
\isanewline
\isacommand{lemma}\isamarkupfalse%
\ domain{\isacharunderscore}{\kern0pt}Pi\ {\isacharcolon}{\kern0pt}\ {\isachardoublequoteopen}f\ {\isasymin}\ A\ {\isasymrightarrow}\ B\ {\isasymLongrightarrow}\ domain{\isacharparenleft}{\kern0pt}f{\isacharparenright}{\kern0pt}\ {\isacharequal}{\kern0pt}\ A{\isachardoublequoteclose}\ \isanewline
%
\isadelimproof
\ \ %
\endisadelimproof
%
\isatagproof
\isacommand{unfolding}\isamarkupfalse%
\ Pi{\isacharunderscore}{\kern0pt}def\ \isanewline
\ \ \isacommand{by}\isamarkupfalse%
\ auto%
\endisatagproof
{\isafoldproof}%
%
\isadelimproof
\isanewline
%
\endisadelimproof
\isanewline
\isacommand{lemma}\isamarkupfalse%
\ function{\isacharunderscore}{\kern0pt}eq{\isacharunderscore}{\kern0pt}helper\ {\isacharcolon}{\kern0pt}\ {\isachardoublequoteopen}{\isasymAnd}A\ B\ C\ f\ g{\isachardot}{\kern0pt}\ f\ {\isasymin}\ A\ {\isasymrightarrow}\ B\ {\isasymLongrightarrow}\ g\ {\isasymin}\ A\ {\isasymrightarrow}\ C\ {\isasymLongrightarrow}\ {\isasymforall}x\ {\isasymin}\ A{\isachardot}{\kern0pt}\ f{\isacharbackquote}{\kern0pt}x\ {\isacharequal}{\kern0pt}\ g{\isacharbackquote}{\kern0pt}x\ {\isasymLongrightarrow}\ f\ {\isacharequal}{\kern0pt}\ g{\isachardoublequoteclose}\ \isanewline
%
\isadelimproof
\ \ %
\endisadelimproof
%
\isatagproof
\isacommand{apply}\isamarkupfalse%
\ {\isacharparenleft}{\kern0pt}rule{\isacharunderscore}{\kern0pt}tac\ equality{\isacharunderscore}{\kern0pt}iffI{\isacharparenright}{\kern0pt}\isanewline
\isacommand{proof}\isamarkupfalse%
\ {\isacharminus}{\kern0pt}\isanewline
\ \ \isanewline
\ \ \isacommand{have}\isamarkupfalse%
\ helper\ {\isacharcolon}{\kern0pt}\ \ {\isachardoublequoteopen}{\isasymAnd}A\ B\ f\ x{\isachardot}{\kern0pt}\ f\ {\isasymin}\ A\ {\isasymrightarrow}\ B\ {\isasymLongrightarrow}\ x\ {\isasymin}\ f\ {\isasymlongleftrightarrow}\ {\isacharparenleft}{\kern0pt}{\isasymexists}a\ {\isasymin}\ A{\isachardot}{\kern0pt}\ x\ {\isacharequal}{\kern0pt}\ {\isacharless}{\kern0pt}a{\isacharcomma}{\kern0pt}\ f{\isacharbackquote}{\kern0pt}a{\isachargreater}{\kern0pt}{\isacharparenright}{\kern0pt}{\isachardoublequoteclose}\ \isanewline
\ \ \isacommand{proof}\isamarkupfalse%
\ {\isacharparenleft}{\kern0pt}rule\ iffI{\isacharparenright}{\kern0pt}\isanewline
\ \ \ \ \isacommand{fix}\isamarkupfalse%
\ A\ B\ f\ x\ \isacommand{assume}\isamarkupfalse%
\ assms\ {\isacharcolon}{\kern0pt}\ {\isachardoublequoteopen}f\ {\isasymin}\ A\ {\isasymrightarrow}\ B{\isachardoublequoteclose}\ {\isachardoublequoteopen}x\ {\isasymin}\ f{\isachardoublequoteclose}\ \isanewline
\ \ \ \ \isacommand{then}\isamarkupfalse%
\ \isacommand{obtain}\isamarkupfalse%
\ a\ b\ \isakeyword{where}\ abH\ {\isacharcolon}{\kern0pt}\ {\isachardoublequoteopen}a\ {\isasymin}\ A{\isachardoublequoteclose}\ {\isachardoublequoteopen}{\isacharless}{\kern0pt}a{\isacharcomma}{\kern0pt}\ b{\isachargreater}{\kern0pt}\ {\isacharequal}{\kern0pt}\ x{\isachardoublequoteclose}\ \isacommand{unfolding}\isamarkupfalse%
\ Pi{\isacharunderscore}{\kern0pt}def\ \isacommand{by}\isamarkupfalse%
\ auto\ \isanewline
\ \ \ \ \isacommand{then}\isamarkupfalse%
\ \isacommand{have}\isamarkupfalse%
\ {\isachardoublequoteopen}b\ {\isacharequal}{\kern0pt}\ f{\isacharbackquote}{\kern0pt}a{\isachardoublequoteclose}\ \isacommand{using}\isamarkupfalse%
\ apply{\isacharunderscore}{\kern0pt}equality\ assms\ \isacommand{by}\isamarkupfalse%
\ auto\ \isanewline
\ \ \ \ \isacommand{then}\isamarkupfalse%
\ \isacommand{show}\isamarkupfalse%
\ {\isachardoublequoteopen}{\isasymexists}a{\isasymin}A{\isachardot}{\kern0pt}\ x\ {\isacharequal}{\kern0pt}\ {\isacharless}{\kern0pt}a{\isacharcomma}{\kern0pt}\ f{\isacharbackquote}{\kern0pt}a{\isachargreater}{\kern0pt}{\isachardoublequoteclose}\ \isacommand{using}\isamarkupfalse%
\ abH\ \isacommand{by}\isamarkupfalse%
\ auto\ \isanewline
\ \ \isacommand{next}\isamarkupfalse%
\ \isanewline
\ \ \ \ \isacommand{fix}\isamarkupfalse%
\ A\ B\ f\ x\ \isacommand{assume}\isamarkupfalse%
\ assms\ {\isacharcolon}{\kern0pt}\ {\isachardoublequoteopen}f\ {\isasymin}\ A\ {\isasymrightarrow}\ B{\isachardoublequoteclose}\ {\isachardoublequoteopen}{\isasymexists}a{\isasymin}A{\isachardot}{\kern0pt}\ x\ {\isacharequal}{\kern0pt}\ {\isasymlangle}a{\isacharcomma}{\kern0pt}\ f\ {\isacharbackquote}{\kern0pt}\ a{\isasymrangle}{\isachardoublequoteclose}\ \isanewline
\ \ \ \ \isacommand{then}\isamarkupfalse%
\ \isacommand{obtain}\isamarkupfalse%
\ a\ \isakeyword{where}\ aH\ {\isacharcolon}{\kern0pt}\ {\isachardoublequoteopen}a\ {\isasymin}\ A{\isachardoublequoteclose}\ {\isachardoublequoteopen}x\ {\isacharequal}{\kern0pt}\ {\isacharless}{\kern0pt}a{\isacharcomma}{\kern0pt}\ f{\isacharbackquote}{\kern0pt}a{\isachargreater}{\kern0pt}{\isachardoublequoteclose}\ \isacommand{by}\isamarkupfalse%
\ auto\ \isanewline
\ \ \ \ \isacommand{then}\isamarkupfalse%
\ \isacommand{have}\isamarkupfalse%
\ {\isachardoublequoteopen}{\isacharless}{\kern0pt}a{\isacharcomma}{\kern0pt}\ f{\isacharbackquote}{\kern0pt}a{\isachargreater}{\kern0pt}\ {\isasymin}\ f{\isachardoublequoteclose}\ \isanewline
\ \ \ \ \ \ \isacommand{apply}\isamarkupfalse%
\ {\isacharparenleft}{\kern0pt}rule{\isacharunderscore}{\kern0pt}tac\ function{\isacharunderscore}{\kern0pt}apply{\isacharunderscore}{\kern0pt}Pair{\isacharparenright}{\kern0pt}\ \isacommand{using}\isamarkupfalse%
\ assms\ Pi{\isacharunderscore}{\kern0pt}iff\ aH\ \isacommand{by}\isamarkupfalse%
\ auto\ \isanewline
\ \ \ \ \isacommand{then}\isamarkupfalse%
\ \isacommand{show}\isamarkupfalse%
\ {\isachardoublequoteopen}x\ {\isasymin}\ f{\isachardoublequoteclose}\ \isacommand{using}\isamarkupfalse%
\ aH\ \isacommand{by}\isamarkupfalse%
\ auto\isanewline
\ \ \isacommand{qed}\isamarkupfalse%
\isanewline
\isanewline
\ \ \isacommand{fix}\isamarkupfalse%
\ A\ B\ C\ f\ g\ x\ \isacommand{assume}\isamarkupfalse%
\ assms\ {\isacharcolon}{\kern0pt}\ {\isachardoublequoteopen}f\ {\isasymin}\ A\ {\isasymrightarrow}\ B{\isachardoublequoteclose}\ {\isachardoublequoteopen}g\ {\isasymin}\ A\ {\isasymrightarrow}\ C{\isachardoublequoteclose}\ {\isachardoublequoteopen}{\isasymforall}x\ {\isasymin}\ A{\isachardot}{\kern0pt}\ f{\isacharbackquote}{\kern0pt}x\ {\isacharequal}{\kern0pt}\ g{\isacharbackquote}{\kern0pt}x{\isachardoublequoteclose}\ \ \isanewline
\ \ \isacommand{have}\isamarkupfalse%
\ H{\isadigit{1}}{\isacharcolon}{\kern0pt}{\isachardoublequoteopen}x\ {\isasymin}\ f\ {\isasymlongleftrightarrow}\ {\isacharparenleft}{\kern0pt}{\isasymexists}a\ {\isasymin}\ A{\isachardot}{\kern0pt}\ {\isacharless}{\kern0pt}a{\isacharcomma}{\kern0pt}\ f{\isacharbackquote}{\kern0pt}a{\isachargreater}{\kern0pt}\ {\isacharequal}{\kern0pt}\ x{\isacharparenright}{\kern0pt}{\isachardoublequoteclose}\ \isacommand{apply}\isamarkupfalse%
\ {\isacharparenleft}{\kern0pt}rule{\isacharunderscore}{\kern0pt}tac\ iffI{\isacharparenright}{\kern0pt}\ \isacommand{using}\isamarkupfalse%
\ helper\ assms\ \isacommand{by}\isamarkupfalse%
\ auto\ \isanewline
\ \ \isacommand{have}\isamarkupfalse%
\ H{\isadigit{2}}{\isacharcolon}{\kern0pt}{\isachardoublequoteopen}{\isachardot}{\kern0pt}{\isachardot}{\kern0pt}{\isachardot}{\kern0pt}\ {\isasymlongleftrightarrow}\ {\isacharparenleft}{\kern0pt}{\isasymexists}a\ {\isasymin}\ A{\isachardot}{\kern0pt}\ {\isacharless}{\kern0pt}a{\isacharcomma}{\kern0pt}\ g{\isacharbackquote}{\kern0pt}a{\isachargreater}{\kern0pt}\ {\isacharequal}{\kern0pt}\ x{\isacharparenright}{\kern0pt}{\isachardoublequoteclose}\ \isacommand{using}\isamarkupfalse%
\ assms\ \isacommand{by}\isamarkupfalse%
\ auto\isanewline
\ \ \isacommand{have}\isamarkupfalse%
\ H{\isadigit{3}}{\isacharcolon}{\kern0pt}{\isachardoublequoteopen}{\isachardot}{\kern0pt}{\isachardot}{\kern0pt}{\isachardot}{\kern0pt}\ {\isasymlongleftrightarrow}\ x\ {\isasymin}\ g{\isachardoublequoteclose}\ \isacommand{using}\isamarkupfalse%
\ helper\ assms\ \isacommand{by}\isamarkupfalse%
\ auto\ \isanewline
\ \ \isacommand{show}\isamarkupfalse%
\ {\isachardoublequoteopen}\ x\ {\isasymin}\ f\ {\isasymlongleftrightarrow}\ x\ {\isasymin}\ g\ {\isachardoublequoteclose}\ \isacommand{using}\isamarkupfalse%
\ H{\isadigit{1}}\ H{\isadigit{2}}\ H{\isadigit{3}}\ \isacommand{by}\isamarkupfalse%
\ auto\ \isanewline
\isacommand{qed}\isamarkupfalse%
%
\endisatagproof
{\isafoldproof}%
%
\isadelimproof
\isanewline
%
\endisadelimproof
\isanewline
\isacommand{lemma}\isamarkupfalse%
\ function{\isacharunderscore}{\kern0pt}eq\ {\isacharcolon}{\kern0pt}\ {\isachardoublequoteopen}{\isasymAnd}f\ g{\isachardot}{\kern0pt}\ relation{\isacharparenleft}{\kern0pt}f{\isacharparenright}{\kern0pt}\ {\isasymLongrightarrow}\ relation{\isacharparenleft}{\kern0pt}g{\isacharparenright}{\kern0pt}\ {\isasymLongrightarrow}\ function{\isacharparenleft}{\kern0pt}f{\isacharparenright}{\kern0pt}\ {\isasymLongrightarrow}\ function{\isacharparenleft}{\kern0pt}g{\isacharparenright}{\kern0pt}\ {\isasymLongrightarrow}\ domain{\isacharparenleft}{\kern0pt}f{\isacharparenright}{\kern0pt}\ {\isacharequal}{\kern0pt}\ domain{\isacharparenleft}{\kern0pt}g{\isacharparenright}{\kern0pt}\ {\isasymLongrightarrow}\ {\isacharparenleft}{\kern0pt}{\isasymAnd}x{\isachardot}{\kern0pt}\ x\ {\isasymin}\ domain{\isacharparenleft}{\kern0pt}f{\isacharparenright}{\kern0pt}\ {\isasymLongrightarrow}\ f{\isacharbackquote}{\kern0pt}x\ {\isacharequal}{\kern0pt}\ g{\isacharbackquote}{\kern0pt}x{\isacharparenright}{\kern0pt}\ {\isasymLongrightarrow}\ f\ {\isacharequal}{\kern0pt}\ g{\isachardoublequoteclose}\ \isanewline
%
\isadelimproof
\ \ %
\endisadelimproof
%
\isatagproof
\isacommand{apply}\isamarkupfalse%
{\isacharparenleft}{\kern0pt}rule{\isacharunderscore}{\kern0pt}tac\ A{\isacharequal}{\kern0pt}{\isachardoublequoteopen}domain{\isacharparenleft}{\kern0pt}f{\isacharparenright}{\kern0pt}{\isachardoublequoteclose}\ \isakeyword{and}\ B{\isacharequal}{\kern0pt}{\isachardoublequoteopen}range{\isacharparenleft}{\kern0pt}f{\isacharparenright}{\kern0pt}{\isachardoublequoteclose}\ \isakeyword{and}\ C{\isacharequal}{\kern0pt}{\isachardoublequoteopen}range{\isacharparenleft}{\kern0pt}g{\isacharparenright}{\kern0pt}{\isachardoublequoteclose}\ \isakeyword{in}\ function{\isacharunderscore}{\kern0pt}eq{\isacharunderscore}{\kern0pt}helper{\isacharparenright}{\kern0pt}\isanewline
\ \ \ \ \isacommand{apply}\isamarkupfalse%
{\isacharparenleft}{\kern0pt}rule\ Pi{\isacharunderscore}{\kern0pt}memberI{\isacharparenright}{\kern0pt}\isanewline
\ \ \isacommand{apply}\isamarkupfalse%
\ simp{\isacharunderscore}{\kern0pt}all\isanewline
\ \ \ \ \isacommand{apply}\isamarkupfalse%
{\isacharparenleft}{\kern0pt}rule\ Pi{\isacharunderscore}{\kern0pt}memberI{\isacharparenright}{\kern0pt}\isanewline
\ \ \ \ \ \isacommand{apply}\isamarkupfalse%
\ simp{\isacharunderscore}{\kern0pt}all\isanewline
\ \ \isacommand{done}\isamarkupfalse%
%
\endisatagproof
{\isafoldproof}%
%
\isadelimproof
\isanewline
%
\endisadelimproof
\isanewline
\isacommand{lemma}\isamarkupfalse%
\ nat{\isacharunderscore}{\kern0pt}in{\isacharunderscore}{\kern0pt}nat\ {\isacharcolon}{\kern0pt}\ {\isachardoublequoteopen}n\ {\isasymin}\ nat\ {\isasymLongrightarrow}\ m\ {\isasymin}\ n\ {\isasymLongrightarrow}\ m\ {\isasymin}\ nat{\isachardoublequoteclose}\isanewline
%
\isadelimproof
\ \ %
\endisadelimproof
%
\isatagproof
\isacommand{by}\isamarkupfalse%
{\isacharparenleft}{\kern0pt}rule\ lt{\isacharunderscore}{\kern0pt}nat{\isacharunderscore}{\kern0pt}in{\isacharunderscore}{\kern0pt}nat{\isacharcomma}{\kern0pt}\ rule\ ltI{\isacharcomma}{\kern0pt}\ simp{\isacharunderscore}{\kern0pt}all{\isacharparenright}{\kern0pt}%
\endisatagproof
{\isafoldproof}%
%
\isadelimproof
\isanewline
%
\endisadelimproof
\isanewline
\isacommand{lemma}\isamarkupfalse%
\ union{\isacharunderscore}{\kern0pt}in{\isacharunderscore}{\kern0pt}nat\ {\isacharcolon}{\kern0pt}\ {\isachardoublequoteopen}n\ {\isasymin}\ nat\ {\isasymLongrightarrow}\ m\ {\isasymin}\ nat\ {\isasymLongrightarrow}\ n\ {\isasymunion}\ m\ {\isasymin}\ nat{\isachardoublequoteclose}\ \isanewline
%
\isadelimproof
\ \ %
\endisadelimproof
%
\isatagproof
\isacommand{by}\isamarkupfalse%
\ auto%
\endisatagproof
{\isafoldproof}%
%
\isadelimproof
\isanewline
%
\endisadelimproof
\isanewline
\isacommand{lemma}\isamarkupfalse%
\ union{\isacharunderscore}{\kern0pt}lt{\isadigit{1}}\ {\isacharcolon}{\kern0pt}\ {\isachardoublequoteopen}Ord{\isacharparenleft}{\kern0pt}n{\isacharparenright}{\kern0pt}\ {\isasymLongrightarrow}\ Ord{\isacharparenleft}{\kern0pt}m{\isacharparenright}{\kern0pt}\ {\isasymLongrightarrow}\ Ord{\isacharparenleft}{\kern0pt}l{\isacharparenright}{\kern0pt}\ {\isasymLongrightarrow}\ n\ {\isacharless}{\kern0pt}\ m\ {\isasymLongrightarrow}\ n\ {\isacharless}{\kern0pt}\ m\ {\isasymunion}\ l{\isachardoublequoteclose}\ \isanewline
%
\isadelimproof
\ \ %
\endisadelimproof
%
\isatagproof
\isacommand{apply}\isamarkupfalse%
{\isacharparenleft}{\kern0pt}rule{\isacharunderscore}{\kern0pt}tac\ b{\isacharequal}{\kern0pt}m\ \isakeyword{in}\ lt{\isacharunderscore}{\kern0pt}le{\isacharunderscore}{\kern0pt}lt{\isacharparenright}{\kern0pt}\isanewline
\ \ \isacommand{using}\isamarkupfalse%
\ max{\isacharunderscore}{\kern0pt}le{\isadigit{1}}\ \isanewline
\ \ \isacommand{by}\isamarkupfalse%
\ auto%
\endisatagproof
{\isafoldproof}%
%
\isadelimproof
\isanewline
%
\endisadelimproof
\isacommand{lemma}\isamarkupfalse%
\ union{\isacharunderscore}{\kern0pt}lt{\isadigit{2}}\ {\isacharcolon}{\kern0pt}\ {\isachardoublequoteopen}Ord{\isacharparenleft}{\kern0pt}n{\isacharparenright}{\kern0pt}\ {\isasymLongrightarrow}\ Ord{\isacharparenleft}{\kern0pt}m{\isacharparenright}{\kern0pt}\ {\isasymLongrightarrow}\ Ord{\isacharparenleft}{\kern0pt}l{\isacharparenright}{\kern0pt}\ {\isasymLongrightarrow}\ n\ {\isacharless}{\kern0pt}\ l\ {\isasymLongrightarrow}\ n\ {\isacharless}{\kern0pt}\ m\ {\isasymunion}\ l{\isachardoublequoteclose}\ \isanewline
%
\isadelimproof
\ \ %
\endisadelimproof
%
\isatagproof
\isacommand{apply}\isamarkupfalse%
{\isacharparenleft}{\kern0pt}rule{\isacharunderscore}{\kern0pt}tac\ b{\isacharequal}{\kern0pt}l\ \isakeyword{in}\ lt{\isacharunderscore}{\kern0pt}le{\isacharunderscore}{\kern0pt}lt{\isacharparenright}{\kern0pt}\isanewline
\ \ \isacommand{using}\isamarkupfalse%
\ max{\isacharunderscore}{\kern0pt}le{\isadigit{2}}\ \isanewline
\ \ \isacommand{by}\isamarkupfalse%
\ auto%
\endisatagproof
{\isafoldproof}%
%
\isadelimproof
\isanewline
%
\endisadelimproof
\isanewline
\isanewline
\isacommand{lemma}\isamarkupfalse%
\ add{\isacharunderscore}{\kern0pt}diff{\isacharunderscore}{\kern0pt}swap\ {\isacharcolon}{\kern0pt}\ \isanewline
\ \ \isakeyword{fixes}\ a\ b\ c\ \isanewline
\ \ \isakeyword{assumes}\ {\isachardoublequoteopen}a\ {\isasymin}\ nat{\isachardoublequoteclose}\ {\isachardoublequoteopen}b\ {\isasymin}\ nat{\isachardoublequoteclose}\ {\isachardoublequoteopen}c\ {\isasymin}\ nat{\isachardoublequoteclose}\ {\isachardoublequoteopen}c\ {\isasymle}\ b{\isachardoublequoteclose}\isanewline
\ \ \isakeyword{shows}\ {\isachardoublequoteopen}a\ {\isacharhash}{\kern0pt}{\isacharplus}{\kern0pt}\ b\ {\isacharhash}{\kern0pt}{\isacharminus}{\kern0pt}\ c\ {\isacharequal}{\kern0pt}\ a\ {\isacharhash}{\kern0pt}{\isacharplus}{\kern0pt}\ {\isacharparenleft}{\kern0pt}b\ {\isacharhash}{\kern0pt}{\isacharminus}{\kern0pt}\ c{\isacharparenright}{\kern0pt}{\isachardoublequoteclose}\isanewline
%
\isadelimproof
%
\endisadelimproof
%
\isatagproof
\isacommand{proof}\isamarkupfalse%
\ {\isacharminus}{\kern0pt}\ \isanewline
\ \ \isacommand{have}\isamarkupfalse%
\ {\isachardoublequoteopen}c\ {\isacharhash}{\kern0pt}{\isacharplus}{\kern0pt}\ {\isacharparenleft}{\kern0pt}a\ {\isacharhash}{\kern0pt}{\isacharplus}{\kern0pt}\ b\ {\isacharhash}{\kern0pt}{\isacharminus}{\kern0pt}\ c{\isacharparenright}{\kern0pt}\ {\isacharequal}{\kern0pt}\ {\isacharparenleft}{\kern0pt}a\ {\isacharhash}{\kern0pt}{\isacharplus}{\kern0pt}\ b\ {\isacharhash}{\kern0pt}{\isacharminus}{\kern0pt}\ c{\isacharparenright}{\kern0pt}\ {\isacharhash}{\kern0pt}{\isacharplus}{\kern0pt}\ c{\isachardoublequoteclose}\ \isacommand{by}\isamarkupfalse%
\ auto\ \isanewline
\ \ \isacommand{also}\isamarkupfalse%
\ \isacommand{have}\isamarkupfalse%
\ {\isachardoublequoteopen}{\isachardot}{\kern0pt}{\isachardot}{\kern0pt}{\isachardot}{\kern0pt}\ {\isacharequal}{\kern0pt}\ a\ {\isacharhash}{\kern0pt}{\isacharplus}{\kern0pt}\ b{\isachardoublequoteclose}\ \isanewline
\ \ \ \ \isacommand{apply}\isamarkupfalse%
{\isacharparenleft}{\kern0pt}rule\ add{\isacharunderscore}{\kern0pt}diff{\isacharunderscore}{\kern0pt}inverse{\isadigit{2}}{\isacharparenright}{\kern0pt}\isanewline
\ \ \ \ \ \isacommand{apply}\isamarkupfalse%
{\isacharparenleft}{\kern0pt}rule{\isacharunderscore}{\kern0pt}tac\ j{\isacharequal}{\kern0pt}b\ \isakeyword{in}\ le{\isacharunderscore}{\kern0pt}trans{\isacharparenright}{\kern0pt}\isanewline
\ \ \ \ \isacommand{using}\isamarkupfalse%
\ assms\ \isanewline
\ \ \ \ \isacommand{by}\isamarkupfalse%
\ auto\isanewline
\ \ \isacommand{also}\isamarkupfalse%
\ \isacommand{have}\isamarkupfalse%
\ {\isachardoublequoteopen}{\isachardot}{\kern0pt}{\isachardot}{\kern0pt}{\isachardot}{\kern0pt}\ {\isacharequal}{\kern0pt}\ a\ {\isacharhash}{\kern0pt}{\isacharplus}{\kern0pt}\ {\isacharparenleft}{\kern0pt}{\isacharparenleft}{\kern0pt}b\ {\isacharhash}{\kern0pt}{\isacharminus}{\kern0pt}\ c{\isacharparenright}{\kern0pt}\ {\isacharhash}{\kern0pt}{\isacharplus}{\kern0pt}\ c{\isacharparenright}{\kern0pt}{\isachardoublequoteclose}\ \isanewline
\ \ \ \ \isacommand{apply}\isamarkupfalse%
{\isacharparenleft}{\kern0pt}rule{\isacharunderscore}{\kern0pt}tac\ i{\isacharequal}{\kern0pt}a\ \isakeyword{and}\ j\ {\isacharequal}{\kern0pt}\ a\ \isakeyword{in}\ add{\isacharunderscore}{\kern0pt}left{\isacharunderscore}{\kern0pt}cancel{\isacharcomma}{\kern0pt}\ simp{\isacharcomma}{\kern0pt}\ simp\ add{\isacharcolon}{\kern0pt}assms{\isacharparenright}{\kern0pt}\isanewline
\ \ \ \ \ \ \isacommand{apply}\isamarkupfalse%
{\isacharparenleft}{\kern0pt}subst\ add{\isacharunderscore}{\kern0pt}diff{\isacharunderscore}{\kern0pt}inverse{\isadigit{2}}{\isacharparenright}{\kern0pt}\isanewline
\ \ \ \ \isacommand{using}\isamarkupfalse%
\ assms\ \isanewline
\ \ \ \ \isacommand{by}\isamarkupfalse%
\ auto\isanewline
\ \ \isacommand{also}\isamarkupfalse%
\ \isacommand{have}\isamarkupfalse%
\ {\isachardoublequoteopen}{\isachardot}{\kern0pt}{\isachardot}{\kern0pt}{\isachardot}{\kern0pt}\ {\isacharequal}{\kern0pt}\ c\ {\isacharhash}{\kern0pt}{\isacharplus}{\kern0pt}\ {\isacharparenleft}{\kern0pt}a\ {\isacharhash}{\kern0pt}{\isacharplus}{\kern0pt}\ {\isacharparenleft}{\kern0pt}b\ {\isacharhash}{\kern0pt}{\isacharminus}{\kern0pt}\ c{\isacharparenright}{\kern0pt}{\isacharparenright}{\kern0pt}{\isachardoublequoteclose}\ \isacommand{by}\isamarkupfalse%
\ auto\ \isanewline
\isanewline
\ \ \isacommand{finally}\isamarkupfalse%
\ \isacommand{show}\isamarkupfalse%
\ {\isacharquery}{\kern0pt}thesis\ \isanewline
\ \ \ \ \isacommand{apply}\isamarkupfalse%
{\isacharparenleft}{\kern0pt}rule{\isacharunderscore}{\kern0pt}tac\ i{\isacharequal}{\kern0pt}c\ \isakeyword{and}\ j{\isacharequal}{\kern0pt}c\ \isakeyword{in}\ add{\isacharunderscore}{\kern0pt}left{\isacharunderscore}{\kern0pt}cancel{\isacharparenright}{\kern0pt}\isanewline
\ \ \ \ \isacommand{by}\isamarkupfalse%
\ auto\isanewline
\isacommand{qed}\isamarkupfalse%
%
\endisatagproof
{\isafoldproof}%
%
\isadelimproof
\isanewline
%
\endisadelimproof
\isanewline
\isacommand{lemma}\isamarkupfalse%
\ le{\isacharunderscore}{\kern0pt}succ{\isacharunderscore}{\kern0pt}pred\ {\isacharcolon}{\kern0pt}\ \isanewline
\ \ {\isachardoublequoteopen}{\isasymAnd}n{\isachardot}{\kern0pt}\ n\ {\isasymin}\ nat\ {\isasymLongrightarrow}\ n\ {\isasymle}\ succ{\isacharparenleft}{\kern0pt}pred{\isacharparenleft}{\kern0pt}n{\isacharparenright}{\kern0pt}{\isacharparenright}{\kern0pt}{\isachardoublequoteclose}\ \isanewline
%
\isadelimproof
\ \ %
\endisadelimproof
%
\isatagproof
\isacommand{apply}\isamarkupfalse%
{\isacharparenleft}{\kern0pt}rule{\isacharunderscore}{\kern0pt}tac\ n{\isacharequal}{\kern0pt}n\ \isakeyword{in}\ natE{\isacharparenright}{\kern0pt}\isanewline
\ \ \isacommand{by}\isamarkupfalse%
\ auto%
\endisatagproof
{\isafoldproof}%
%
\isadelimproof
\isanewline
%
\endisadelimproof
\isanewline
\isacommand{lemma}\isamarkupfalse%
\ relation{\isacharunderscore}{\kern0pt}recfun\ {\isacharcolon}{\kern0pt}\ \isanewline
\ \ \isakeyword{assumes}\ {\isachardoublequoteopen}is{\isacharunderscore}{\kern0pt}recfun{\isacharparenleft}{\kern0pt}r{\isacharcomma}{\kern0pt}\ x{\isacharcomma}{\kern0pt}\ H{\isacharcomma}{\kern0pt}\ f{\isacharparenright}{\kern0pt}{\isachardoublequoteclose}\ \isanewline
\ \ \isakeyword{shows}\ {\isachardoublequoteopen}relation{\isacharparenleft}{\kern0pt}f{\isacharparenright}{\kern0pt}{\isachardoublequoteclose}\isanewline
%
\isadelimproof
\isanewline
\ \ %
\endisadelimproof
%
\isatagproof
\isacommand{using}\isamarkupfalse%
\ assms\ \isanewline
\ \ \isacommand{unfolding}\isamarkupfalse%
\ is{\isacharunderscore}{\kern0pt}recfun{\isacharunderscore}{\kern0pt}def\ \isanewline
\ \ \isacommand{apply}\isamarkupfalse%
{\isacharparenleft}{\kern0pt}rule{\isacharunderscore}{\kern0pt}tac\ a{\isacharequal}{\kern0pt}{\isachardoublequoteopen}{\isacharparenleft}{\kern0pt}{\isasymlambda}x{\isasymin}r\ {\isacharminus}{\kern0pt}{\isacharbackquote}{\kern0pt}{\isacharbackquote}{\kern0pt}\ {\isacharbraceleft}{\kern0pt}x{\isacharbraceright}{\kern0pt}{\isachardot}{\kern0pt}\ H{\isacharparenleft}{\kern0pt}x{\isacharcomma}{\kern0pt}\ restrict{\isacharparenleft}{\kern0pt}f{\isacharcomma}{\kern0pt}\ r\ {\isacharminus}{\kern0pt}{\isacharbackquote}{\kern0pt}{\isacharbackquote}{\kern0pt}\ {\isacharbraceleft}{\kern0pt}x{\isacharbraceright}{\kern0pt}{\isacharparenright}{\kern0pt}{\isacharparenright}{\kern0pt}{\isacharparenright}{\kern0pt}{\isachardoublequoteclose}\ \isakeyword{and}\ b{\isacharequal}{\kern0pt}f\ \isakeyword{in}\ ssubst{\isacharcomma}{\kern0pt}\ simp{\isacharparenright}{\kern0pt}\isanewline
\ \ \isacommand{apply}\isamarkupfalse%
{\isacharparenleft}{\kern0pt}rule\ relation{\isacharunderscore}{\kern0pt}lam{\isacharparenright}{\kern0pt}\isanewline
\ \ \isacommand{done}\isamarkupfalse%
%
\endisatagproof
{\isafoldproof}%
%
\isadelimproof
\isanewline
%
\endisadelimproof
\isanewline
%
\isadelimtheory
\isanewline
%
\endisadelimtheory
%
\isatagtheory
\isacommand{end}\isamarkupfalse%
%
\endisatagtheory
{\isafoldtheory}%
%
\isadelimtheory
%
\endisadelimtheory
%
\end{isabellebody}%
\endinput
%:%file=~/source/repos/ZF-notAC/code/Utilities.thy%:%
%:%10=1%:%
%:%11=1%:%
%:%12=2%:%
%:%13=3%:%
%:%14=4%:%
%:%15=5%:%
%:%20=5%:%
%:%23=6%:%
%:%24=7%:%
%:%25=7%:%
%:%32=8%:%
%:%33=8%:%
%:%34=9%:%
%:%35=9%:%
%:%36=9%:%
%:%37=10%:%
%:%38=10%:%
%:%39=10%:%
%:%40=10%:%
%:%41=11%:%
%:%42=11%:%
%:%43=11%:%
%:%44=12%:%
%:%45=12%:%
%:%46=13%:%
%:%47=13%:%
%:%48=14%:%
%:%49=14%:%
%:%50=14%:%
%:%51=14%:%
%:%52=15%:%
%:%53=15%:%
%:%54=16%:%
%:%55=16%:%
%:%56=16%:%
%:%57=17%:%
%:%58=17%:%
%:%59=17%:%
%:%60=17%:%
%:%61=17%:%
%:%62=18%:%
%:%68=18%:%
%:%71=19%:%
%:%72=20%:%
%:%73=20%:%
%:%76=21%:%
%:%80=21%:%
%:%81=21%:%
%:%86=21%:%
%:%89=22%:%
%:%90=23%:%
%:%91=23%:%
%:%92=24%:%
%:%99=25%:%
%:%100=25%:%
%:%101=26%:%
%:%102=26%:%
%:%103=27%:%
%:%104=27%:%
%:%105=27%:%
%:%106=27%:%
%:%107=27%:%
%:%108=28%:%
%:%109=28%:%
%:%110=28%:%
%:%111=28%:%
%:%112=28%:%
%:%113=28%:%
%:%114=29%:%
%:%115=29%:%
%:%116=29%:%
%:%117=29%:%
%:%118=30%:%
%:%119=30%:%
%:%120=30%:%
%:%121=30%:%
%:%122=30%:%
%:%123=31%:%
%:%124=31%:%
%:%125=31%:%
%:%126=31%:%
%:%127=31%:%
%:%128=32%:%
%:%134=32%:%
%:%137=33%:%
%:%138=34%:%
%:%139=34%:%
%:%140=35%:%
%:%147=36%:%
%:%148=36%:%
%:%149=37%:%
%:%150=37%:%
%:%151=38%:%
%:%152=38%:%
%:%153=38%:%
%:%154=38%:%
%:%155=39%:%
%:%156=39%:%
%:%157=39%:%
%:%158=39%:%
%:%159=39%:%
%:%160=40%:%
%:%161=40%:%
%:%162=41%:%
%:%163=41%:%
%:%164=42%:%
%:%165=42%:%
%:%166=43%:%
%:%167=43%:%
%:%168=43%:%
%:%169=44%:%
%:%170=44%:%
%:%171=44%:%
%:%172=44%:%
%:%173=44%:%
%:%174=45%:%
%:%175=45%:%
%:%176=45%:%
%:%177=45%:%
%:%178=45%:%
%:%179=46%:%
%:%185=46%:%
%:%188=47%:%
%:%189=48%:%
%:%190=48%:%
%:%193=49%:%
%:%197=49%:%
%:%198=49%:%
%:%199=49%:%
%:%200=49%:%
%:%206=49%:%
%:%209=50%:%
%:%210=51%:%
%:%211=51%:%
%:%213=51%:%
%:%217=51%:%
%:%218=51%:%
%:%225=51%:%
%:%226=52%:%
%:%227=53%:%
%:%228=53%:%
%:%229=54%:%
%:%231=56%:%
%:%234=57%:%
%:%238=57%:%
%:%239=57%:%
%:%240=58%:%
%:%241=58%:%
%:%242=59%:%
%:%243=59%:%
%:%244=59%:%
%:%245=60%:%
%:%246=61%:%
%:%247=62%:%
%:%248=63%:%
%:%249=63%:%
%:%250=63%:%
%:%251=64%:%
%:%252=64%:%
%:%253=65%:%
%:%254=65%:%
%:%255=65%:%
%:%256=65%:%
%:%257=66%:%
%:%258=66%:%
%:%259=66%:%
%:%260=66%:%
%:%261=66%:%
%:%262=67%:%
%:%263=67%:%
%:%264=67%:%
%:%265=67%:%
%:%266=67%:%
%:%267=68%:%
%:%268=68%:%
%:%269=68%:%
%:%270=69%:%
%:%271=69%:%
%:%272=70%:%
%:%273=70%:%
%:%274=71%:%
%:%275=71%:%
%:%276=71%:%
%:%277=72%:%
%:%278=72%:%
%:%279=73%:%
%:%280=73%:%
%:%281=73%:%
%:%282=74%:%
%:%283=75%:%
%:%284=76%:%
%:%285=77%:%
%:%286=77%:%
%:%287=77%:%
%:%288=78%:%
%:%289=78%:%
%:%290=79%:%
%:%291=79%:%
%:%292=79%:%
%:%293=79%:%
%:%294=79%:%
%:%295=80%:%
%:%296=80%:%
%:%297=80%:%
%:%298=80%:%
%:%299=80%:%
%:%300=81%:%
%:%301=81%:%
%:%302=81%:%
%:%303=81%:%
%:%304=81%:%
%:%305=82%:%
%:%306=82%:%
%:%307=82%:%
%:%308=83%:%
%:%309=83%:%
%:%310=84%:%
%:%311=84%:%
%:%312=85%:%
%:%313=85%:%
%:%314=85%:%
%:%315=86%:%
%:%321=86%:%
%:%324=87%:%
%:%325=88%:%
%:%326=89%:%
%:%327=89%:%
%:%330=90%:%
%:%334=90%:%
%:%335=90%:%
%:%336=91%:%
%:%337=91%:%
%:%338=92%:%
%:%339=92%:%
%:%340=92%:%
%:%341=93%:%
%:%342=93%:%
%:%343=93%:%
%:%344=93%:%
%:%345=93%:%
%:%346=94%:%
%:%347=94%:%
%:%348=94%:%
%:%349=94%:%
%:%350=94%:%
%:%351=95%:%
%:%352=95%:%
%:%353=95%:%
%:%354=95%:%
%:%355=95%:%
%:%356=96%:%
%:%357=96%:%
%:%358=96%:%
%:%359=96%:%
%:%360=96%:%
%:%361=97%:%
%:%362=97%:%
%:%363=98%:%
%:%364=98%:%
%:%365=98%:%
%:%366=99%:%
%:%367=99%:%
%:%368=99%:%
%:%369=99%:%
%:%370=99%:%
%:%371=100%:%
%:%372=100%:%
%:%373=100%:%
%:%374=100%:%
%:%375=100%:%
%:%376=101%:%
%:%377=101%:%
%:%378=101%:%
%:%379=101%:%
%:%380=101%:%
%:%381=102%:%
%:%382=102%:%
%:%383=102%:%
%:%384=102%:%
%:%385=103%:%
%:%391=103%:%
%:%394=104%:%
%:%395=105%:%
%:%396=105%:%
%:%399=106%:%
%:%403=106%:%
%:%404=106%:%
%:%405=107%:%
%:%406=107%:%
%:%407=108%:%
%:%408=108%:%
%:%409=108%:%
%:%414=108%:%
%:%417=109%:%
%:%418=110%:%
%:%419=110%:%
%:%422=111%:%
%:%426=111%:%
%:%427=111%:%
%:%428=112%:%
%:%429=112%:%
%:%430=113%:%
%:%431=113%:%
%:%432=113%:%
%:%437=113%:%
%:%440=114%:%
%:%441=115%:%
%:%442=115%:%
%:%445=116%:%
%:%449=116%:%
%:%450=116%:%
%:%451=117%:%
%:%452=117%:%
%:%453=118%:%
%:%454=118%:%
%:%455=119%:%
%:%456=119%:%
%:%457=119%:%
%:%462=119%:%
%:%465=120%:%
%:%466=121%:%
%:%467=121%:%
%:%470=122%:%
%:%474=122%:%
%:%475=122%:%
%:%476=123%:%
%:%477=123%:%
%:%478=123%:%
%:%483=123%:%
%:%486=124%:%
%:%487=125%:%
%:%488=126%:%
%:%489=126%:%
%:%496=127%:%
%:%497=127%:%
%:%498=128%:%
%:%499=128%:%
%:%500=129%:%
%:%501=129%:%
%:%502=129%:%
%:%503=129%:%
%:%504=130%:%
%:%505=130%:%
%:%506=130%:%
%:%507=130%:%
%:%508=130%:%
%:%509=131%:%
%:%510=131%:%
%:%511=131%:%
%:%512=131%:%
%:%513=132%:%
%:%514=132%:%
%:%515=132%:%
%:%516=133%:%
%:%517=133%:%
%:%518=133%:%
%:%519=133%:%
%:%520=134%:%
%:%521=134%:%
%:%522=134%:%
%:%523=134%:%
%:%524=134%:%
%:%525=135%:%
%:%526=135%:%
%:%527=135%:%
%:%528=135%:%
%:%529=135%:%
%:%530=136%:%
%:%531=136%:%
%:%532=136%:%
%:%533=136%:%
%:%534=137%:%
%:%535=137%:%
%:%536=137%:%
%:%537=137%:%
%:%538=138%:%
%:%539=138%:%
%:%540=138%:%
%:%541=139%:%
%:%542=139%:%
%:%543=139%:%
%:%544=139%:%
%:%545=139%:%
%:%546=140%:%
%:%547=140%:%
%:%548=141%:%
%:%549=141%:%
%:%550=142%:%
%:%551=142%:%
%:%552=143%:%
%:%553=143%:%
%:%554=144%:%
%:%555=144%:%
%:%556=144%:%
%:%557=145%:%
%:%558=145%:%
%:%559=146%:%
%:%560=146%:%
%:%561=147%:%
%:%562=147%:%
%:%563=147%:%
%:%564=148%:%
%:%565=148%:%
%:%566=148%:%
%:%567=149%:%
%:%568=149%:%
%:%569=150%:%
%:%570=150%:%
%:%571=151%:%
%:%572=151%:%
%:%573=151%:%
%:%574=151%:%
%:%575=151%:%
%:%576=152%:%
%:%577=152%:%
%:%578=153%:%
%:%579=153%:%
%:%580=153%:%
%:%581=153%:%
%:%582=153%:%
%:%583=154%:%
%:%589=154%:%
%:%592=155%:%
%:%593=156%:%
%:%594=157%:%
%:%595=157%:%
%:%598=158%:%
%:%602=158%:%
%:%603=158%:%
%:%608=158%:%
%:%611=159%:%
%:%612=159%:%
%:%615=160%:%
%:%619=160%:%
%:%620=160%:%
%:%625=160%:%
%:%628=161%:%
%:%629=162%:%
%:%630=162%:%
%:%632=164%:%
%:%635=165%:%
%:%639=165%:%
%:%640=165%:%
%:%641=166%:%
%:%642=166%:%
%:%643=167%:%
%:%644=167%:%
%:%645=167%:%
%:%646=168%:%
%:%647=168%:%
%:%648=168%:%
%:%649=168%:%
%:%650=169%:%
%:%651=169%:%
%:%652=169%:%
%:%653=169%:%
%:%654=169%:%
%:%655=170%:%
%:%656=170%:%
%:%657=170%:%
%:%658=170%:%
%:%659=170%:%
%:%660=171%:%
%:%661=171%:%
%:%662=171%:%
%:%663=171%:%
%:%664=171%:%
%:%665=172%:%
%:%666=172%:%
%:%667=173%:%
%:%668=173%:%
%:%669=173%:%
%:%670=174%:%
%:%671=174%:%
%:%672=174%:%
%:%673=174%:%
%:%674=175%:%
%:%675=175%:%
%:%676=175%:%
%:%677=175%:%
%:%678=175%:%
%:%679=176%:%
%:%680=176%:%
%:%681=176%:%
%:%682=176%:%
%:%683=176%:%
%:%684=177%:%
%:%685=177%:%
%:%686=177%:%
%:%687=177%:%
%:%688=177%:%
%:%689=178%:%
%:%695=178%:%
%:%698=179%:%
%:%699=180%:%
%:%700=181%:%
%:%701=181%:%
%:%702=182%:%
%:%703=183%:%
%:%710=184%:%
%:%711=184%:%
%:%712=185%:%
%:%713=185%:%
%:%714=186%:%
%:%715=186%:%
%:%716=187%:%
%:%717=187%:%
%:%718=187%:%
%:%719=187%:%
%:%720=188%:%
%:%721=188%:%
%:%722=188%:%
%:%723=188%:%
%:%724=188%:%
%:%725=189%:%
%:%726=189%:%
%:%727=189%:%
%:%728=189%:%
%:%729=189%:%
%:%730=190%:%
%:%731=190%:%
%:%732=190%:%
%:%733=190%:%
%:%734=190%:%
%:%735=191%:%
%:%741=191%:%
%:%744=192%:%
%:%745=193%:%
%:%746=193%:%
%:%747=194%:%
%:%754=195%:%
%:%755=195%:%
%:%756=196%:%
%:%757=196%:%
%:%758=197%:%
%:%759=197%:%
%:%760=198%:%
%:%761=198%:%
%:%762=198%:%
%:%763=198%:%
%:%764=199%:%
%:%765=199%:%
%:%766=199%:%
%:%767=199%:%
%:%768=199%:%
%:%769=200%:%
%:%770=200%:%
%:%771=200%:%
%:%772=200%:%
%:%773=200%:%
%:%774=201%:%
%:%775=201%:%
%:%776=201%:%
%:%777=201%:%
%:%778=201%:%
%:%779=202%:%
%:%785=202%:%
%:%788=203%:%
%:%789=204%:%
%:%790=204%:%
%:%792=204%:%
%:%796=204%:%
%:%797=204%:%
%:%804=204%:%
%:%805=205%:%
%:%806=206%:%
%:%807=206%:%
%:%809=206%:%
%:%813=206%:%
%:%814=206%:%
%:%821=206%:%
%:%822=207%:%
%:%823=208%:%
%:%824=208%:%
%:%827=209%:%
%:%831=209%:%
%:%832=209%:%
%:%837=209%:%
%:%840=210%:%
%:%841=211%:%
%:%842=211%:%
%:%845=212%:%
%:%849=212%:%
%:%850=212%:%
%:%851=212%:%
%:%856=212%:%
%:%859=213%:%
%:%860=214%:%
%:%861=214%:%
%:%868=215%:%
%:%869=215%:%
%:%870=216%:%
%:%871=216%:%
%:%872=217%:%
%:%873=217%:%
%:%874=218%:%
%:%875=218%:%
%:%876=219%:%
%:%877=219%:%
%:%878=220%:%
%:%879=220%:%
%:%880=221%:%
%:%881=221%:%
%:%882=221%:%
%:%883=221%:%
%:%884=221%:%
%:%885=222%:%
%:%886=222%:%
%:%887=223%:%
%:%888=223%:%
%:%889=224%:%
%:%890=224%:%
%:%891=224%:%
%:%892=224%:%
%:%893=224%:%
%:%894=225%:%
%:%895=225%:%
%:%896=225%:%
%:%897=225%:%
%:%898=225%:%
%:%899=226%:%
%:%900=226%:%
%:%901=227%:%
%:%907=227%:%
%:%910=228%:%
%:%911=229%:%
%:%912=229%:%
%:%913=230%:%
%:%920=231%:%
%:%921=231%:%
%:%922=232%:%
%:%923=232%:%
%:%924=232%:%
%:%925=233%:%
%:%926=233%:%
%:%927=233%:%
%:%928=233%:%
%:%929=233%:%
%:%930=234%:%
%:%931=234%:%
%:%932=234%:%
%:%933=234%:%
%:%934=234%:%
%:%935=235%:%
%:%936=235%:%
%:%937=235%:%
%:%938=235%:%
%:%939=235%:%
%:%940=236%:%
%:%946=236%:%
%:%949=237%:%
%:%950=238%:%
%:%951=238%:%
%:%952=239%:%
%:%959=240%:%
%:%960=240%:%
%:%961=241%:%
%:%962=241%:%
%:%963=241%:%
%:%964=242%:%
%:%965=242%:%
%:%966=243%:%
%:%967=243%:%
%:%968=244%:%
%:%969=244%:%
%:%970=244%:%
%:%971=245%:%
%:%972=245%:%
%:%973=245%:%
%:%974=245%:%
%:%975=245%:%
%:%976=246%:%
%:%977=246%:%
%:%978=247%:%
%:%984=247%:%
%:%987=248%:%
%:%988=249%:%
%:%989=249%:%
%:%990=250%:%
%:%997=251%:%
%:%998=251%:%
%:%999=252%:%
%:%1000=252%:%
%:%1001=252%:%
%:%1002=253%:%
%:%1003=253%:%
%:%1004=253%:%
%:%1005=254%:%
%:%1006=254%:%
%:%1007=255%:%
%:%1008=255%:%
%:%1009=256%:%
%:%1010=256%:%
%:%1011=257%:%
%:%1017=257%:%
%:%1020=258%:%
%:%1021=259%:%
%:%1022=259%:%
%:%1029=260%:%
%:%1030=260%:%
%:%1031=261%:%
%:%1032=261%:%
%:%1033=262%:%
%:%1034=262%:%
%:%1035=262%:%
%:%1036=262%:%
%:%1037=262%:%
%:%1038=262%:%
%:%1039=263%:%
%:%1040=263%:%
%:%1041=263%:%
%:%1042=264%:%
%:%1043=264%:%
%:%1044=264%:%
%:%1045=264%:%
%:%1046=264%:%
%:%1047=265%:%
%:%1048=265%:%
%:%1049=265%:%
%:%1050=266%:%
%:%1051=266%:%
%:%1052=266%:%
%:%1053=266%:%
%:%1054=266%:%
%:%1055=266%:%
%:%1056=267%:%
%:%1057=267%:%
%:%1058=267%:%
%:%1059=267%:%
%:%1060=268%:%
%:%1061=268%:%
%:%1062=268%:%
%:%1063=268%:%
%:%1064=268%:%
%:%1065=269%:%
%:%1066=269%:%
%:%1067=269%:%
%:%1068=269%:%
%:%1069=269%:%
%:%1070=270%:%
%:%1076=270%:%
%:%1079=271%:%
%:%1080=272%:%
%:%1081=272%:%
%:%1084=273%:%
%:%1088=273%:%
%:%1089=273%:%
%:%1090=274%:%
%:%1091=274%:%
%:%1092=275%:%
%:%1093=275%:%
%:%1094=276%:%
%:%1095=276%:%
%:%1096=276%:%
%:%1097=277%:%
%:%1098=277%:%
%:%1099=277%:%
%:%1100=277%:%
%:%1101=277%:%
%:%1102=278%:%
%:%1103=278%:%
%:%1104=278%:%
%:%1105=278%:%
%:%1106=278%:%
%:%1107=279%:%
%:%1108=279%:%
%:%1109=279%:%
%:%1110=280%:%
%:%1111=280%:%
%:%1112=281%:%
%:%1113=281%:%
%:%1114=281%:%
%:%1115=282%:%
%:%1116=282%:%
%:%1117=282%:%
%:%1118=282%:%
%:%1119=282%:%
%:%1120=283%:%
%:%1121=283%:%
%:%1122=283%:%
%:%1123=284%:%
%:%1124=284%:%
%:%1125=285%:%
%:%1126=285%:%
%:%1127=285%:%
%:%1128=286%:%
%:%1129=286%:%
%:%1130=286%:%
%:%1131=286%:%
%:%1132=287%:%
%:%1133=287%:%
%:%1134=287%:%
%:%1135=287%:%
%:%1136=287%:%
%:%1137=287%:%
%:%1138=288%:%
%:%1139=288%:%
%:%1140=288%:%
%:%1141=288%:%
%:%1142=288%:%
%:%1143=289%:%
%:%1144=289%:%
%:%1145=290%:%
%:%1146=290%:%
%:%1147=290%:%
%:%1148=291%:%
%:%1149=291%:%
%:%1150=291%:%
%:%1151=292%:%
%:%1152=292%:%
%:%1153=293%:%
%:%1154=293%:%
%:%1155=294%:%
%:%1156=294%:%
%:%1157=294%:%
%:%1158=294%:%
%:%1159=294%:%
%:%1160=294%:%
%:%1161=295%:%
%:%1162=295%:%
%:%1163=295%:%
%:%1164=296%:%
%:%1165=296%:%
%:%1166=297%:%
%:%1167=297%:%
%:%1168=298%:%
%:%1169=298%:%
%:%1170=299%:%
%:%1171=299%:%
%:%1172=299%:%
%:%1173=299%:%
%:%1174=299%:%
%:%1175=299%:%
%:%1176=300%:%
%:%1182=300%:%
%:%1185=301%:%
%:%1186=302%:%
%:%1187=302%:%
%:%1190=303%:%
%:%1194=303%:%
%:%1195=303%:%
%:%1196=303%:%
%:%1202=303%:%
%:%1205=304%:%
%:%1206=305%:%
%:%1207=305%:%
%:%1210=306%:%
%:%1214=306%:%
%:%1215=306%:%
%:%1216=306%:%
%:%1222=306%:%
%:%1225=307%:%
%:%1226=308%:%
%:%1227=308%:%
%:%1229=308%:%
%:%1233=308%:%
%:%1234=308%:%
%:%1241=308%:%
%:%1242=309%:%
%:%1243=310%:%
%:%1244=310%:%
%:%1246=310%:%
%:%1250=310%:%
%:%1251=310%:%
%:%1258=310%:%
%:%1259=311%:%
%:%1260=312%:%
%:%1261=312%:%
%:%1263=312%:%
%:%1267=312%:%
%:%1268=312%:%
%:%1275=312%:%
%:%1276=313%:%
%:%1277=313%:%
%:%1279=313%:%
%:%1283=313%:%
%:%1284=313%:%
%:%1291=313%:%
%:%1292=314%:%
%:%1293=314%:%
%:%1295=314%:%
%:%1299=314%:%
%:%1300=314%:%
%:%1307=314%:%
%:%1308=315%:%
%:%1309=315%:%
%:%1311=315%:%
%:%1315=315%:%
%:%1316=315%:%
%:%1323=315%:%
%:%1324=316%:%
%:%1325=316%:%
%:%1327=316%:%
%:%1331=316%:%
%:%1332=316%:%
%:%1339=316%:%
%:%1340=317%:%
%:%1341=317%:%
%:%1343=317%:%
%:%1347=317%:%
%:%1348=317%:%
%:%1355=317%:%
%:%1356=318%:%
%:%1357=318%:%
%:%1359=318%:%
%:%1363=318%:%
%:%1364=318%:%
%:%1371=318%:%
%:%1372=319%:%
%:%1373=319%:%
%:%1375=319%:%
%:%1379=319%:%
%:%1380=319%:%
%:%1387=319%:%
%:%1388=320%:%
%:%1389=320%:%
%:%1391=320%:%
%:%1395=320%:%
%:%1396=320%:%
%:%1403=320%:%
%:%1404=321%:%
%:%1405=321%:%
%:%1407=321%:%
%:%1411=321%:%
%:%1412=321%:%
%:%1419=321%:%
%:%1420=322%:%
%:%1421=322%:%
%:%1423=322%:%
%:%1427=322%:%
%:%1428=322%:%
%:%1435=322%:%
%:%1436=323%:%
%:%1437=324%:%
%:%1438=324%:%
%:%1441=325%:%
%:%1445=325%:%
%:%1446=325%:%
%:%1447=325%:%
%:%1454=325%:%
%:%1455=326%:%
%:%1456=327%:%
%:%1457=327%:%
%:%1460=328%:%
%:%1464=328%:%
%:%1465=328%:%
%:%1466=328%:%
%:%1473=328%:%
%:%1474=329%:%
%:%1475=330%:%
%:%1476=330%:%
%:%1479=331%:%
%:%1483=331%:%
%:%1484=331%:%
%:%1485=332%:%
%:%1486=332%:%
%:%1487=333%:%
%:%1488=333%:%
%:%1489=334%:%
%:%1490=334%:%
%:%1491=335%:%
%:%1492=335%:%
%:%1493=336%:%
%:%1494=336%:%
%:%1495=337%:%
%:%1496=337%:%
%:%1497=338%:%
%:%1498=338%:%
%:%1499=339%:%
%:%1500=339%:%
%:%1501=340%:%
%:%1502=340%:%
%:%1507=340%:%
%:%1510=341%:%
%:%1511=342%:%
%:%1512=342%:%
%:%1515=343%:%
%:%1519=343%:%
%:%1520=343%:%
%:%1521=344%:%
%:%1522=344%:%
%:%1523=345%:%
%:%1524=345%:%
%:%1525=346%:%
%:%1526=346%:%
%:%1531=346%:%
%:%1534=347%:%
%:%1535=348%:%
%:%1536=348%:%
%:%1539=349%:%
%:%1543=349%:%
%:%1544=349%:%
%:%1545=350%:%
%:%1546=350%:%
%:%1547=351%:%
%:%1548=351%:%
%:%1553=351%:%
%:%1556=352%:%
%:%1557=353%:%
%:%1558=353%:%
%:%1561=354%:%
%:%1565=354%:%
%:%1566=354%:%
%:%1567=355%:%
%:%1568=355%:%
%:%1569=356%:%
%:%1570=356%:%
%:%1571=357%:%
%:%1572=357%:%
%:%1573=358%:%
%:%1574=358%:%
%:%1579=358%:%
%:%1582=359%:%
%:%1583=360%:%
%:%1584=360%:%
%:%1591=361%:%
%:%1592=361%:%
%:%1593=362%:%
%:%1594=362%:%
%:%1595=363%:%
%:%1596=363%:%
%:%1597=363%:%
%:%1598=363%:%
%:%1599=364%:%
%:%1600=364%:%
%:%1601=364%:%
%:%1602=365%:%
%:%1603=365%:%
%:%1604=366%:%
%:%1605=366%:%
%:%1606=367%:%
%:%1607=367%:%
%:%1608=368%:%
%:%1614=368%:%
%:%1617=369%:%
%:%1618=370%:%
%:%1619=370%:%
%:%1620=371%:%
%:%1621=372%:%
%:%1624=373%:%
%:%1625=374%:%
%:%1629=374%:%
%:%1630=374%:%
%:%1631=375%:%
%:%1632=375%:%
%:%1633=375%:%
%:%1634=376%:%
%:%1635=376%:%
%:%1636=377%:%
%:%1637=377%:%
%:%1638=378%:%
%:%1639=378%:%
%:%1640=379%:%
%:%1641=379%:%
%:%1642=379%:%
%:%1643=379%:%
%:%1644=380%:%
%:%1645=380%:%
%:%1646=381%:%
%:%1647=381%:%
%:%1648=382%:%
%:%1649=382%:%
%:%1650=382%:%
%:%1651=383%:%
%:%1652=383%:%
%:%1653=383%:%
%:%1654=383%:%
%:%1655=383%:%
%:%1656=383%:%
%:%1657=384%:%
%:%1658=384%:%
%:%1659=385%:%
%:%1660=385%:%
%:%1661=386%:%
%:%1662=386%:%
%:%1663=386%:%
%:%1664=386%:%
%:%1665=387%:%
%:%1666=387%:%
%:%1667=387%:%
%:%1668=388%:%
%:%1669=388%:%
%:%1670=389%:%
%:%1671=389%:%
%:%1672=389%:%
%:%1673=389%:%
%:%1674=389%:%
%:%1675=390%:%
%:%1676=390%:%
%:%1677=391%:%
%:%1683=391%:%
%:%1686=392%:%
%:%1687=393%:%
%:%1688=393%:%
%:%1689=394%:%
%:%1690=395%:%
%:%1697=396%:%
%:%1698=396%:%
%:%1699=397%:%
%:%1700=397%:%
%:%1701=397%:%
%:%1702=397%:%
%:%1703=398%:%
%:%1704=398%:%
%:%1705=398%:%
%:%1706=398%:%
%:%1707=398%:%
%:%1708=399%:%
%:%1709=399%:%
%:%1710=399%:%
%:%1711=399%:%
%:%1712=400%:%
%:%1713=400%:%
%:%1714=400%:%
%:%1715=400%:%
%:%1716=400%:%
%:%1717=401%:%
%:%1718=401%:%
%:%1719=401%:%
%:%1720=401%:%
%:%1721=401%:%
%:%1722=402%:%
%:%1728=402%:%
%:%1731=403%:%
%:%1732=404%:%
%:%1733=404%:%
%:%1734=405%:%
%:%1735=406%:%
%:%1738=407%:%
%:%1739=408%:%
%:%1743=408%:%
%:%1744=408%:%
%:%1745=408%:%
%:%1746=408%:%
%:%1747=408%:%
%:%1748=409%:%
%:%1749=409%:%
%:%1754=409%:%
%:%1757=410%:%
%:%1758=411%:%
%:%1759=411%:%
%:%1760=412%:%
%:%1761=413%:%
%:%1764=414%:%
%:%1768=414%:%
%:%1769=414%:%
%:%1770=415%:%
%:%1771=415%:%
%:%1772=415%:%
%:%1773=415%:%
%:%1774=416%:%
%:%1775=416%:%
%:%1776=417%:%
%:%1777=417%:%
%:%1778=418%:%
%:%1779=418%:%
%:%1780=419%:%
%:%1781=419%:%
%:%1782=419%:%
%:%1783=419%:%
%:%1784=420%:%
%:%1785=420%:%
%:%1786=420%:%
%:%1787=421%:%
%:%1788=421%:%
%:%1789=422%:%
%:%1790=422%:%
%:%1791=423%:%
%:%1792=423%:%
%:%1793=424%:%
%:%1794=424%:%
%:%1795=425%:%
%:%1796=425%:%
%:%1797=426%:%
%:%1798=426%:%
%:%1799=426%:%
%:%1800=427%:%
%:%1801=427%:%
%:%1802=427%:%
%:%1803=427%:%
%:%1804=428%:%
%:%1805=428%:%
%:%1806=429%:%
%:%1807=429%:%
%:%1808=429%:%
%:%1809=430%:%
%:%1810=431%:%
%:%1811=432%:%
%:%1812=432%:%
%:%1813=432%:%
%:%1814=432%:%
%:%1815=433%:%
%:%1816=433%:%
%:%1817=433%:%
%:%1818=433%:%
%:%1819=433%:%
%:%1820=434%:%
%:%1821=434%:%
%:%1822=434%:%
%:%1823=434%:%
%:%1824=435%:%
%:%1825=435%:%
%:%1826=435%:%
%:%1827=436%:%
%:%1828=436%:%
%:%1829=437%:%
%:%1830=437%:%
%:%1831=438%:%
%:%1832=438%:%
%:%1833=438%:%
%:%1834=439%:%
%:%1835=439%:%
%:%1836=440%:%
%:%1842=440%:%
%:%1845=441%:%
%:%1846=442%:%
%:%1847=442%:%
%:%1850=443%:%
%:%1854=443%:%
%:%1855=443%:%
%:%1856=443%:%
%:%1857=443%:%
%:%1858=443%:%
%:%1859=444%:%
%:%1860=444%:%
%:%1861=445%:%
%:%1862=445%:%
%:%1863=445%:%
%:%1864=446%:%
%:%1865=446%:%
%:%1866=446%:%
%:%1867=446%:%
%:%1868=446%:%
%:%1869=447%:%
%:%1870=447%:%
%:%1871=447%:%
%:%1872=447%:%
%:%1873=447%:%
%:%1874=447%:%
%:%1875=448%:%
%:%1876=448%:%
%:%1877=448%:%
%:%1878=448%:%
%:%1879=448%:%
%:%1880=449%:%
%:%1886=449%:%
%:%1889=450%:%
%:%1890=451%:%
%:%1891=451%:%
%:%1894=452%:%
%:%1898=452%:%
%:%1899=452%:%
%:%1900=453%:%
%:%1901=453%:%
%:%1906=453%:%
%:%1909=454%:%
%:%1910=455%:%
%:%1911=455%:%
%:%1914=456%:%
%:%1918=456%:%
%:%1919=456%:%
%:%1920=457%:%
%:%1921=457%:%
%:%1922=458%:%
%:%1923=459%:%
%:%1924=459%:%
%:%1925=460%:%
%:%1926=460%:%
%:%1927=461%:%
%:%1928=461%:%
%:%1929=461%:%
%:%1930=462%:%
%:%1931=462%:%
%:%1932=462%:%
%:%1933=462%:%
%:%1934=462%:%
%:%1935=463%:%
%:%1936=463%:%
%:%1937=463%:%
%:%1938=463%:%
%:%1939=463%:%
%:%1940=464%:%
%:%1941=464%:%
%:%1942=464%:%
%:%1943=464%:%
%:%1944=464%:%
%:%1945=465%:%
%:%1946=465%:%
%:%1947=466%:%
%:%1948=466%:%
%:%1949=466%:%
%:%1950=467%:%
%:%1951=467%:%
%:%1952=467%:%
%:%1953=467%:%
%:%1954=468%:%
%:%1955=468%:%
%:%1956=468%:%
%:%1957=469%:%
%:%1958=469%:%
%:%1959=469%:%
%:%1960=469%:%
%:%1961=470%:%
%:%1962=470%:%
%:%1963=470%:%
%:%1964=470%:%
%:%1965=470%:%
%:%1966=471%:%
%:%1967=471%:%
%:%1968=472%:%
%:%1969=473%:%
%:%1970=473%:%
%:%1971=473%:%
%:%1972=474%:%
%:%1973=474%:%
%:%1974=474%:%
%:%1975=474%:%
%:%1976=474%:%
%:%1977=475%:%
%:%1978=475%:%
%:%1979=475%:%
%:%1980=475%:%
%:%1981=476%:%
%:%1982=476%:%
%:%1983=476%:%
%:%1984=476%:%
%:%1985=477%:%
%:%1986=477%:%
%:%1987=477%:%
%:%1988=477%:%
%:%1989=478%:%
%:%1995=478%:%
%:%1998=479%:%
%:%1999=480%:%
%:%2000=480%:%
%:%2003=481%:%
%:%2007=481%:%
%:%2008=481%:%
%:%2009=482%:%
%:%2010=482%:%
%:%2011=483%:%
%:%2012=483%:%
%:%2013=484%:%
%:%2014=484%:%
%:%2015=485%:%
%:%2016=485%:%
%:%2017=486%:%
%:%2023=486%:%
%:%2026=487%:%
%:%2027=488%:%
%:%2028=488%:%
%:%2031=489%:%
%:%2035=489%:%
%:%2036=489%:%
%:%2041=489%:%
%:%2044=490%:%
%:%2045=491%:%
%:%2046=491%:%
%:%2049=492%:%
%:%2053=492%:%
%:%2054=492%:%
%:%2059=492%:%
%:%2062=493%:%
%:%2063=494%:%
%:%2064=494%:%
%:%2067=495%:%
%:%2071=495%:%
%:%2072=495%:%
%:%2073=496%:%
%:%2074=496%:%
%:%2075=497%:%
%:%2076=497%:%
%:%2081=497%:%
%:%2084=498%:%
%:%2085=498%:%
%:%2088=499%:%
%:%2092=499%:%
%:%2093=499%:%
%:%2094=500%:%
%:%2095=500%:%
%:%2096=501%:%
%:%2097=501%:%
%:%2102=501%:%
%:%2105=502%:%
%:%2106=503%:%
%:%2107=504%:%
%:%2108=504%:%
%:%2109=505%:%
%:%2110=506%:%
%:%2111=507%:%
%:%2118=508%:%
%:%2119=508%:%
%:%2120=509%:%
%:%2121=509%:%
%:%2122=509%:%
%:%2123=510%:%
%:%2124=510%:%
%:%2125=510%:%
%:%2126=511%:%
%:%2127=511%:%
%:%2128=512%:%
%:%2129=512%:%
%:%2130=513%:%
%:%2131=513%:%
%:%2132=514%:%
%:%2133=514%:%
%:%2134=515%:%
%:%2135=515%:%
%:%2136=515%:%
%:%2137=516%:%
%:%2138=516%:%
%:%2139=517%:%
%:%2140=517%:%
%:%2141=518%:%
%:%2142=518%:%
%:%2143=519%:%
%:%2144=519%:%
%:%2145=520%:%
%:%2146=520%:%
%:%2147=520%:%
%:%2148=520%:%
%:%2149=521%:%
%:%2150=522%:%
%:%2151=522%:%
%:%2152=522%:%
%:%2153=523%:%
%:%2154=523%:%
%:%2155=524%:%
%:%2156=524%:%
%:%2157=525%:%
%:%2163=525%:%
%:%2166=526%:%
%:%2167=527%:%
%:%2168=527%:%
%:%2169=528%:%
%:%2172=529%:%
%:%2176=529%:%
%:%2177=529%:%
%:%2178=530%:%
%:%2179=530%:%
%:%2184=530%:%
%:%2187=531%:%
%:%2188=532%:%
%:%2189=532%:%
%:%2190=533%:%
%:%2191=534%:%
%:%2194=535%:%
%:%2195=536%:%
%:%2199=536%:%
%:%2200=536%:%
%:%2201=537%:%
%:%2202=537%:%
%:%2203=538%:%
%:%2204=538%:%
%:%2205=539%:%
%:%2206=539%:%
%:%2207=540%:%
%:%2213=540%:%
%:%2216=541%:%
%:%2219=542%:%
%:%2224=543%:%

%
\begin{isabellebody}%
\setisabellecontext{Utilities{\isacharunderscore}{\kern0pt}M}%
%
\isadelimtheory
%
\endisadelimtheory
%
\isatagtheory
\isacommand{theory}\isamarkupfalse%
\ Utilities{\isacharunderscore}{\kern0pt}M\ \isanewline
\ \ \isakeyword{imports}\ \isanewline
\ \ \ \ ZF\ \isanewline
\ \ \ \ Utilities\isanewline
\isakeyword{begin}%
\endisatagtheory
{\isafoldtheory}%
%
\isadelimtheory
\ \isanewline
%
\endisadelimtheory
\isanewline
\isacommand{context}\isamarkupfalse%
\ M{\isacharunderscore}{\kern0pt}ctm\isanewline
\isakeyword{begin}\ \isanewline
\isanewline
\isacommand{lemma}\isamarkupfalse%
\ to{\isacharunderscore}{\kern0pt}rex\ {\isacharcolon}{\kern0pt}\ {\isachardoublequoteopen}{\isasymAnd}P{\isachardot}{\kern0pt}\ {\isasymexists}x{\isacharbrackleft}{\kern0pt}{\isacharhash}{\kern0pt}{\isacharhash}{\kern0pt}M{\isacharbrackright}{\kern0pt}{\isachardot}{\kern0pt}\ P{\isacharparenleft}{\kern0pt}x{\isacharparenright}{\kern0pt}\ {\isasymLongrightarrow}\ {\isasymexists}x\ {\isasymin}\ M{\isachardot}{\kern0pt}\ P{\isacharparenleft}{\kern0pt}x{\isacharparenright}{\kern0pt}{\isachardoublequoteclose}%
\isadelimproof
\ %
\endisadelimproof
%
\isatagproof
\isacommand{by}\isamarkupfalse%
\ auto%
\endisatagproof
{\isafoldproof}%
%
\isadelimproof
%
\endisadelimproof
\isanewline
\isacommand{lemma}\isamarkupfalse%
\ to{\isacharunderscore}{\kern0pt}rin\ {\isacharcolon}{\kern0pt}\ {\isachardoublequoteopen}{\isacharparenleft}{\kern0pt}{\isacharhash}{\kern0pt}{\isacharhash}{\kern0pt}M{\isacharparenright}{\kern0pt}{\isacharparenleft}{\kern0pt}x{\isacharparenright}{\kern0pt}\ {\isasymLongrightarrow}\ x\ {\isasymin}\ M{\isachardoublequoteclose}%
\isadelimproof
\ %
\endisadelimproof
%
\isatagproof
\isacommand{by}\isamarkupfalse%
\ auto%
\endisatagproof
{\isafoldproof}%
%
\isadelimproof
%
\endisadelimproof
\isanewline
\isanewline
\isacommand{lemma}\isamarkupfalse%
\ powerset{\isacharunderscore}{\kern0pt}powerA{\isacharunderscore}{\kern0pt}inter{\isacharunderscore}{\kern0pt}M{\isacharunderscore}{\kern0pt}helper\isanewline
\ \ {\isacharcolon}{\kern0pt}\ {\isachardoublequoteopen}{\isasymforall}\ A\ {\isasymin}\ M{\isachardot}{\kern0pt}\ {\isacharparenleft}{\kern0pt}Pow{\isacharparenleft}{\kern0pt}A{\isacharparenright}{\kern0pt}\ {\isasyminter}\ M{\isacharparenright}{\kern0pt}\ {\isasymin}\ M\ {\isasymand}\ {\isacharparenleft}{\kern0pt}{\isasymforall}\ X\ {\isasymin}\ M{\isachardot}{\kern0pt}\ powerset{\isacharparenleft}{\kern0pt}{\isacharhash}{\kern0pt}{\isacharhash}{\kern0pt}M{\isacharcomma}{\kern0pt}\ A{\isacharcomma}{\kern0pt}\ X{\isacharparenright}{\kern0pt}\ {\isasymlongrightarrow}\ X\ {\isacharequal}{\kern0pt}\ {\isacharparenleft}{\kern0pt}Pow{\isacharparenleft}{\kern0pt}A{\isacharparenright}{\kern0pt}\ {\isasyminter}\ M{\isacharparenright}{\kern0pt}{\isacharparenright}{\kern0pt}{\isachardoublequoteclose}\ \isanewline
%
\isadelimproof
\ \ %
\endisadelimproof
%
\isatagproof
\isacommand{apply}\isamarkupfalse%
\ {\isacharparenleft}{\kern0pt}clarify{\isacharparenright}{\kern0pt}\isanewline
\ \ \isacommand{apply}\isamarkupfalse%
\ {\isacharparenleft}{\kern0pt}auto{\isacharparenright}{\kern0pt}\isanewline
\isacommand{proof}\isamarkupfalse%
\ {\isacharminus}{\kern0pt}\ \isanewline
\ \ \isacommand{fix}\isamarkupfalse%
\ A\ \isacommand{assume}\isamarkupfalse%
\ {\isachardoublequoteopen}A\ {\isasymin}\ M{\isachardoublequoteclose}\ \isanewline
\ \ \isacommand{then}\isamarkupfalse%
\ \isacommand{have}\isamarkupfalse%
\ {\isachardoublequoteopen}powerset{\isacharparenleft}{\kern0pt}{\isacharhash}{\kern0pt}{\isacharhash}{\kern0pt}M{\isacharcomma}{\kern0pt}\ A{\isacharcomma}{\kern0pt}\ Pow{\isacharparenleft}{\kern0pt}A{\isacharparenright}{\kern0pt}\ {\isasyminter}\ M{\isacharparenright}{\kern0pt}{\isachardoublequoteclose}\isacommand{unfolding}\isamarkupfalse%
\ powerset{\isacharunderscore}{\kern0pt}def\ \isacommand{by}\isamarkupfalse%
\ auto\isanewline
\ \ \isacommand{obtain}\isamarkupfalse%
\ AP\ \isakeyword{where}\ apassm{\isacharcolon}{\kern0pt}\ \ {\isachardoublequoteopen}AP\ {\isasymin}\ M{\isachardoublequoteclose}\ {\isachardoublequoteopen}powerset{\isacharparenleft}{\kern0pt}{\isacharhash}{\kern0pt}{\isacharhash}{\kern0pt}M{\isacharcomma}{\kern0pt}\ A{\isacharcomma}{\kern0pt}\ AP{\isacharparenright}{\kern0pt}{\isachardoublequoteclose}\ \isanewline
\ \ \ \ \isacommand{using}\isamarkupfalse%
\ power{\isacharunderscore}{\kern0pt}ax\ {\isacartoucheopen}A\ {\isasymin}\ M{\isacartoucheclose}\ \isacommand{unfolding}\isamarkupfalse%
\ power{\isacharunderscore}{\kern0pt}ax{\isacharunderscore}{\kern0pt}def\ \isacommand{by}\isamarkupfalse%
\ auto\isanewline
\ \ \isacommand{then}\isamarkupfalse%
\ \isacommand{have}\isamarkupfalse%
\ apeq\ {\isacharcolon}{\kern0pt}\ \ {\isachardoublequoteopen}AP\ {\isacharequal}{\kern0pt}\ {\isacharbraceleft}{\kern0pt}\ x\ {\isasymin}\ Pow{\isacharparenleft}{\kern0pt}A{\isacharparenright}{\kern0pt}{\isachardot}{\kern0pt}\ {\isacharparenleft}{\kern0pt}{\isacharhash}{\kern0pt}{\isacharhash}{\kern0pt}M{\isacharparenright}{\kern0pt}{\isacharparenleft}{\kern0pt}x{\isacharparenright}{\kern0pt}\ {\isacharbraceright}{\kern0pt}{\isachardoublequoteclose}\ \isacommand{using}\isamarkupfalse%
\ powerset{\isacharunderscore}{\kern0pt}abs\ \isacommand{by}\isamarkupfalse%
\ auto\ \isanewline
\ \ \isacommand{have}\isamarkupfalse%
\ p{\isadigit{1}}\ {\isacharcolon}{\kern0pt}\ {\isachardoublequoteopen}powerset{\isacharparenleft}{\kern0pt}{\isacharhash}{\kern0pt}{\isacharhash}{\kern0pt}M{\isacharcomma}{\kern0pt}\ A{\isacharcomma}{\kern0pt}\ Pow{\isacharparenleft}{\kern0pt}A{\isacharparenright}{\kern0pt}\ {\isasyminter}\ M{\isacharparenright}{\kern0pt}{\isachardoublequoteclose}\ \isacommand{unfolding}\isamarkupfalse%
\ powerset{\isacharunderscore}{\kern0pt}def\ \isacommand{by}\isamarkupfalse%
\ auto\ \isanewline
\ \ \isacommand{then}\isamarkupfalse%
\ \isacommand{have}\isamarkupfalse%
\ {\isachardoublequoteopen}Pow{\isacharparenleft}{\kern0pt}A{\isacharparenright}{\kern0pt}\ {\isasyminter}\ M\ {\isacharequal}{\kern0pt}\ {\isacharbraceleft}{\kern0pt}\ x\ {\isasymin}\ Pow{\isacharparenleft}{\kern0pt}A{\isacharparenright}{\kern0pt}{\isachardot}{\kern0pt}\ {\isacharparenleft}{\kern0pt}{\isacharhash}{\kern0pt}{\isacharhash}{\kern0pt}M{\isacharparenright}{\kern0pt}{\isacharparenleft}{\kern0pt}x{\isacharparenright}{\kern0pt}\ {\isacharbraceright}{\kern0pt}{\isachardoublequoteclose}\ \isacommand{using}\isamarkupfalse%
\ powerset{\isacharunderscore}{\kern0pt}abs\ \isacommand{by}\isamarkupfalse%
\ auto\ \isanewline
\ \ \isacommand{then}\isamarkupfalse%
\ \isacommand{have}\isamarkupfalse%
\ {\isachardoublequoteopen}AP\ {\isacharequal}{\kern0pt}\ Pow{\isacharparenleft}{\kern0pt}A{\isacharparenright}{\kern0pt}\ {\isasyminter}\ M{\isachardoublequoteclose}\ \isacommand{using}\isamarkupfalse%
\ apeq\ \isacommand{by}\isamarkupfalse%
\ auto\ \isanewline
\ \ \isacommand{then}\isamarkupfalse%
\ \isacommand{show}\isamarkupfalse%
\ {\isachardoublequoteopen}{\isacharparenleft}{\kern0pt}Pow{\isacharparenleft}{\kern0pt}A{\isacharparenright}{\kern0pt}\ {\isasyminter}\ M{\isacharparenright}{\kern0pt}\ {\isasymin}\ M{\isachardoublequoteclose}\ \isacommand{using}\isamarkupfalse%
\ apassm\ \isacommand{by}\isamarkupfalse%
\ auto\ \isanewline
\isacommand{qed}\isamarkupfalse%
%
\endisatagproof
{\isafoldproof}%
%
\isadelimproof
\isanewline
%
\endisadelimproof
\isanewline
\isacommand{lemma}\isamarkupfalse%
\ powerset{\isacharunderscore}{\kern0pt}powerA{\isacharunderscore}{\kern0pt}inter{\isacharunderscore}{\kern0pt}M\isanewline
\ \ {\isacharcolon}{\kern0pt}\ {\isachardoublequoteopen}A\ {\isasymin}\ M\ {\isasymLongrightarrow}\ X\ {\isasymin}\ M\ {\isasymLongrightarrow}\ powerset{\isacharparenleft}{\kern0pt}{\isacharhash}{\kern0pt}{\isacharhash}{\kern0pt}M{\isacharcomma}{\kern0pt}\ A{\isacharcomma}{\kern0pt}\ X{\isacharparenright}{\kern0pt}\ {\isasymLongrightarrow}\ X\ {\isacharequal}{\kern0pt}\ {\isacharparenleft}{\kern0pt}Pow{\isacharparenleft}{\kern0pt}A{\isacharparenright}{\kern0pt}\ {\isasyminter}\ M{\isacharparenright}{\kern0pt}{\isachardoublequoteclose}\ \isanewline
%
\isadelimproof
\ \ %
\endisadelimproof
%
\isatagproof
\isacommand{using}\isamarkupfalse%
\ powerset{\isacharunderscore}{\kern0pt}powerA{\isacharunderscore}{\kern0pt}inter{\isacharunderscore}{\kern0pt}M{\isacharunderscore}{\kern0pt}helper\ \isacommand{by}\isamarkupfalse%
\ auto%
\endisatagproof
{\isafoldproof}%
%
\isadelimproof
\ \isanewline
%
\endisadelimproof
\isanewline
\isacommand{lemma}\isamarkupfalse%
\ M{\isacharunderscore}{\kern0pt}powerset\ {\isacharcolon}{\kern0pt}\ \isanewline
\ \ {\isachardoublequoteopen}A\ {\isasymin}\ M\ {\isasymLongrightarrow}\ {\isacharparenleft}{\kern0pt}Pow{\isacharparenleft}{\kern0pt}A{\isacharparenright}{\kern0pt}\ {\isasyminter}\ M{\isacharparenright}{\kern0pt}\ {\isasymin}\ M{\isachardoublequoteclose}\ \isanewline
%
\isadelimproof
\ \ %
\endisadelimproof
%
\isatagproof
\isacommand{using}\isamarkupfalse%
\ powerset{\isacharunderscore}{\kern0pt}powerA{\isacharunderscore}{\kern0pt}inter{\isacharunderscore}{\kern0pt}M{\isacharunderscore}{\kern0pt}helper\ \isacommand{by}\isamarkupfalse%
\ auto%
\endisatagproof
{\isafoldproof}%
%
\isadelimproof
\isanewline
%
\endisadelimproof
\isanewline
\isacommand{lemma}\isamarkupfalse%
\ separation{\isacharunderscore}{\kern0pt}notation{\isacharunderscore}{\kern0pt}helper\ {\isacharcolon}{\kern0pt}\isanewline
\ \ {\isachardoublequoteopen}S\ {\isasymin}\ M\ {\isasymLongrightarrow}\ T\ {\isasymin}\ M\ {\isasymLongrightarrow}\ {\isasymforall}x{\isacharbrackleft}{\kern0pt}{\isacharhash}{\kern0pt}{\isacharhash}{\kern0pt}M{\isacharbrackright}{\kern0pt}{\isachardot}{\kern0pt}{\isacharparenleft}{\kern0pt}x\ {\isasymin}\ S\ {\isasymlongleftrightarrow}\ x\ {\isasymin}\ T\ {\isasymand}\ P{\isacharparenleft}{\kern0pt}x{\isacharparenright}{\kern0pt}{\isacharparenright}{\kern0pt}\ {\isasymLongrightarrow}\ S\ {\isacharequal}{\kern0pt}\ {\isacharbraceleft}{\kern0pt}\ x\ {\isasymin}\ T{\isachardot}{\kern0pt}\ P{\isacharparenleft}{\kern0pt}x{\isacharparenright}{\kern0pt}\ {\isacharbraceright}{\kern0pt}{\isachardoublequoteclose}\ \isanewline
%
\isadelimproof
\ \ %
\endisadelimproof
%
\isatagproof
\isacommand{apply}\isamarkupfalse%
\ {\isacharparenleft}{\kern0pt}rule\ equality{\isacharunderscore}{\kern0pt}iffI{\isacharsemicolon}{\kern0pt}\ rule\ iffI{\isacharparenright}{\kern0pt}\ \isanewline
\isacommand{proof}\isamarkupfalse%
\ {\isacharminus}{\kern0pt}\ \isanewline
\ \ \isacommand{fix}\isamarkupfalse%
\ x\ \isacommand{assume}\isamarkupfalse%
\ assms\ {\isacharcolon}{\kern0pt}\ {\isachardoublequoteopen}{\isasymforall}x{\isacharbrackleft}{\kern0pt}{\isacharhash}{\kern0pt}{\isacharhash}{\kern0pt}M{\isacharbrackright}{\kern0pt}{\isachardot}{\kern0pt}\ x\ {\isasymin}\ S\ {\isasymlongleftrightarrow}\ x\ {\isasymin}\ T\ {\isasymand}\ P{\isacharparenleft}{\kern0pt}x{\isacharparenright}{\kern0pt}{\isachardoublequoteclose}\ {\isachardoublequoteopen}x\ {\isasymin}\ S{\isachardoublequoteclose}\ {\isachardoublequoteopen}S\ {\isasymin}\ M{\isachardoublequoteclose}\isanewline
\ \ \isacommand{then}\isamarkupfalse%
\ \isacommand{have}\isamarkupfalse%
\ {\isachardoublequoteopen}x\ {\isasymin}\ M{\isachardoublequoteclose}\ \isacommand{using}\isamarkupfalse%
\ transM\ \isacommand{by}\isamarkupfalse%
\ auto\isanewline
\ \ \isacommand{then}\isamarkupfalse%
\ \isacommand{have}\isamarkupfalse%
\ {\isachardoublequoteopen}x\ {\isasymin}\ T{\isachardoublequoteclose}\ {\isachardoublequoteopen}P{\isacharparenleft}{\kern0pt}x{\isacharparenright}{\kern0pt}{\isachardoublequoteclose}\ \isacommand{using}\isamarkupfalse%
\ assms\ \isacommand{by}\isamarkupfalse%
\ auto\ \isanewline
\ \ \isacommand{then}\isamarkupfalse%
\ \isacommand{show}\isamarkupfalse%
\ {\isachardoublequoteopen}x\ {\isasymin}\ {\isacharbraceleft}{\kern0pt}\ x\ {\isasymin}\ T{\isachardot}{\kern0pt}\ P{\isacharparenleft}{\kern0pt}x{\isacharparenright}{\kern0pt}\ {\isacharbraceright}{\kern0pt}{\isachardoublequoteclose}\ \isacommand{by}\isamarkupfalse%
\ auto\ \isanewline
\isacommand{next}\isamarkupfalse%
\ \isanewline
\ \ \isacommand{fix}\isamarkupfalse%
\ x\ \isacommand{assume}\isamarkupfalse%
\ assms\ {\isacharcolon}{\kern0pt}\ {\isachardoublequoteopen}{\isasymforall}x{\isacharbrackleft}{\kern0pt}{\isacharhash}{\kern0pt}{\isacharhash}{\kern0pt}M{\isacharbrackright}{\kern0pt}{\isachardot}{\kern0pt}\ x\ {\isasymin}\ S\ {\isasymlongleftrightarrow}\ x\ {\isasymin}\ T\ {\isasymand}\ P{\isacharparenleft}{\kern0pt}x{\isacharparenright}{\kern0pt}{\isachardoublequoteclose}\ {\isachardoublequoteopen}x\ {\isasymin}\ {\isacharbraceleft}{\kern0pt}\ y\ {\isasymin}\ T{\isachardot}{\kern0pt}\ P{\isacharparenleft}{\kern0pt}y{\isacharparenright}{\kern0pt}\ {\isacharbraceright}{\kern0pt}{\isachardoublequoteclose}\ {\isachardoublequoteopen}T\ {\isasymin}\ M{\isachardoublequoteclose}\isanewline
\ \ \isacommand{then}\isamarkupfalse%
\ \isacommand{have}\isamarkupfalse%
\ p{\isadigit{1}}\ {\isacharcolon}{\kern0pt}{\isachardoublequoteopen}x\ {\isasymin}\ T\ {\isasymand}\ P{\isacharparenleft}{\kern0pt}x{\isacharparenright}{\kern0pt}{\isachardoublequoteclose}\ \isacommand{by}\isamarkupfalse%
\ auto\ \isanewline
\ \ \isacommand{then}\isamarkupfalse%
\ \isacommand{have}\isamarkupfalse%
\ {\isachardoublequoteopen}x\ {\isasymin}\ T{\isachardoublequoteclose}\ \isacommand{by}\isamarkupfalse%
\ auto\ \isanewline
\ \ \isacommand{then}\isamarkupfalse%
\ \isacommand{have}\isamarkupfalse%
\ {\isachardoublequoteopen}x\ {\isasymin}\ M{\isachardoublequoteclose}\ \isacommand{using}\isamarkupfalse%
\ transM\ assms\ \isacommand{by}\isamarkupfalse%
\ auto\ \isanewline
\ \ \isacommand{then}\isamarkupfalse%
\ \isacommand{show}\isamarkupfalse%
\ {\isachardoublequoteopen}x\ {\isasymin}\ S{\isachardoublequoteclose}\ \isacommand{using}\isamarkupfalse%
\ assms\ \isacommand{by}\isamarkupfalse%
\ auto\ \isanewline
\isacommand{qed}\isamarkupfalse%
%
\endisatagproof
{\isafoldproof}%
%
\isadelimproof
\isanewline
%
\endisadelimproof
\isanewline
\isacommand{lemma}\isamarkupfalse%
\ separation{\isacharunderscore}{\kern0pt}notation\ {\isacharcolon}{\kern0pt}\ \isanewline
\ \ {\isachardoublequoteopen}separation{\isacharparenleft}{\kern0pt}{\isacharhash}{\kern0pt}{\isacharhash}{\kern0pt}M{\isacharcomma}{\kern0pt}\ P{\isacharparenright}{\kern0pt}\ {\isasymLongrightarrow}\ A\ {\isasymin}\ M\ {\isasymLongrightarrow}\ {\isacharbraceleft}{\kern0pt}\ x\ {\isasymin}\ A{\isachardot}{\kern0pt}\ P{\isacharparenleft}{\kern0pt}x{\isacharparenright}{\kern0pt}\ {\isacharbraceright}{\kern0pt}\ {\isasymin}\ M{\isachardoublequoteclose}\ \isanewline
%
\isadelimproof
%
\endisadelimproof
%
\isatagproof
\isacommand{proof}\isamarkupfalse%
\ {\isacharminus}{\kern0pt}\ \isanewline
\ \ \isacommand{assume}\isamarkupfalse%
\ assms\ {\isacharcolon}{\kern0pt}\ {\isachardoublequoteopen}separation{\isacharparenleft}{\kern0pt}{\isacharhash}{\kern0pt}{\isacharhash}{\kern0pt}M{\isacharcomma}{\kern0pt}\ P{\isacharparenright}{\kern0pt}{\isachardoublequoteclose}\ {\isachardoublequoteopen}A\ {\isasymin}\ M{\isachardoublequoteclose}\ \ \isanewline
\ \ \isacommand{then}\isamarkupfalse%
\ \isacommand{have}\isamarkupfalse%
\ {\isachardoublequoteopen}{\isasymexists}y{\isacharbrackleft}{\kern0pt}{\isacharhash}{\kern0pt}{\isacharhash}{\kern0pt}M{\isacharbrackright}{\kern0pt}{\isachardot}{\kern0pt}\ {\isasymforall}x{\isacharbrackleft}{\kern0pt}{\isacharhash}{\kern0pt}{\isacharhash}{\kern0pt}M{\isacharbrackright}{\kern0pt}{\isachardot}{\kern0pt}\ {\isacharparenleft}{\kern0pt}x\ {\isasymin}\ y\ {\isasymlongleftrightarrow}\ x\ {\isasymin}\ A\ {\isasymand}\ P{\isacharparenleft}{\kern0pt}x{\isacharparenright}{\kern0pt}{\isacharparenright}{\kern0pt}{\isachardoublequoteclose}\ \isacommand{unfolding}\isamarkupfalse%
\ separation{\isacharunderscore}{\kern0pt}def\ \isacommand{by}\isamarkupfalse%
\ auto\ \isanewline
\ \ \isacommand{then}\isamarkupfalse%
\ \isacommand{obtain}\isamarkupfalse%
\ S\ \isakeyword{where}\ sh{\isacharcolon}{\kern0pt}\ {\isachardoublequoteopen}{\isasymforall}x{\isacharbrackleft}{\kern0pt}{\isacharhash}{\kern0pt}{\isacharhash}{\kern0pt}M{\isacharbrackright}{\kern0pt}{\isachardot}{\kern0pt}\ {\isacharparenleft}{\kern0pt}x\ {\isasymin}\ S\ {\isasymlongleftrightarrow}\ x\ {\isasymin}\ A\ {\isasymand}\ P{\isacharparenleft}{\kern0pt}x{\isacharparenright}{\kern0pt}{\isacharparenright}{\kern0pt}{\isachardoublequoteclose}\ {\isachardoublequoteopen}S\ {\isasymin}\ M{\isachardoublequoteclose}\ \isacommand{by}\isamarkupfalse%
\ auto\ \isanewline
\ \ \isacommand{then}\isamarkupfalse%
\ \isacommand{have}\isamarkupfalse%
\ {\isachardoublequoteopen}S\ {\isacharequal}{\kern0pt}\ {\isacharbraceleft}{\kern0pt}\ x\ {\isasymin}\ A{\isachardot}{\kern0pt}\ P{\isacharparenleft}{\kern0pt}x{\isacharparenright}{\kern0pt}\ {\isacharbraceright}{\kern0pt}{\isachardoublequoteclose}\isanewline
\ \ \ \ \isacommand{apply}\isamarkupfalse%
\ {\isacharparenleft}{\kern0pt}rule{\isacharunderscore}{\kern0pt}tac\ separation{\isacharunderscore}{\kern0pt}notation{\isacharunderscore}{\kern0pt}helper{\isacharparenright}{\kern0pt}\ \isacommand{using}\isamarkupfalse%
\ assms\ \isacommand{by}\isamarkupfalse%
\ auto\ \isanewline
\ \ \isacommand{then}\isamarkupfalse%
\ \isacommand{show}\isamarkupfalse%
\ {\isachardoublequoteopen}{\isacharbraceleft}{\kern0pt}\ x\ {\isasymin}\ A{\isachardot}{\kern0pt}\ P{\isacharparenleft}{\kern0pt}x{\isacharparenright}{\kern0pt}\ {\isacharbraceright}{\kern0pt}\ {\isasymin}\ M{\isachardoublequoteclose}\ \isacommand{using}\isamarkupfalse%
\ sh\ \isacommand{by}\isamarkupfalse%
\ auto\ \isanewline
\isacommand{qed}\isamarkupfalse%
%
\endisatagproof
{\isafoldproof}%
%
\isadelimproof
\ \ \isanewline
%
\endisadelimproof
\isanewline
\isanewline
\isacommand{lemma}\isamarkupfalse%
\ separation{\isacharunderscore}{\kern0pt}fst\ {\isacharcolon}{\kern0pt}\isanewline
\ \ {\isachardoublequoteopen}separation{\isacharparenleft}{\kern0pt}{\isacharhash}{\kern0pt}{\isacharhash}{\kern0pt}M{\isacharcomma}{\kern0pt}\ P{\isacharparenright}{\kern0pt}\ {\isasymLongrightarrow}\ X\ {\isasymin}\ M\ {\isasymLongrightarrow}\ Y\ {\isasymin}\ M\ {\isasymLongrightarrow}\ {\isacharbraceleft}{\kern0pt}\ {\isacharless}{\kern0pt}x{\isacharcomma}{\kern0pt}\ y{\isachargreater}{\kern0pt}\ {\isasymin}\ X\ {\isasymtimes}\ Y{\isachardot}{\kern0pt}\ P{\isacharparenleft}{\kern0pt}x{\isacharparenright}{\kern0pt}\ {\isacharbraceright}{\kern0pt}\ {\isasymin}\ M{\isachardoublequoteclose}\ \ \ \isanewline
%
\isadelimproof
%
\endisadelimproof
%
\isatagproof
\isacommand{proof}\isamarkupfalse%
\ {\isacharminus}{\kern0pt}\ \isanewline
\ \ \isacommand{assume}\isamarkupfalse%
\ assms\ {\isacharcolon}{\kern0pt}\ {\isachardoublequoteopen}separation{\isacharparenleft}{\kern0pt}{\isacharhash}{\kern0pt}{\isacharhash}{\kern0pt}M{\isacharcomma}{\kern0pt}\ P{\isacharparenright}{\kern0pt}{\isachardoublequoteclose}\ {\isachardoublequoteopen}X\ {\isasymin}\ M{\isachardoublequoteclose}\ {\isachardoublequoteopen}Y\ {\isasymin}\ M{\isachardoublequoteclose}\ \ \isanewline
\ \ \isacommand{then}\isamarkupfalse%
\ \isacommand{have}\isamarkupfalse%
\ {\isachardoublequoteopen}{\isacharbraceleft}{\kern0pt}\ x\ {\isasymin}\ X{\isachardot}{\kern0pt}\ P{\isacharparenleft}{\kern0pt}x{\isacharparenright}{\kern0pt}\ {\isacharbraceright}{\kern0pt}\ {\isasymin}\ M{\isachardoublequoteclose}\ \isacommand{using}\isamarkupfalse%
\ separation{\isacharunderscore}{\kern0pt}notation\ \isacommand{by}\isamarkupfalse%
\ auto\ \isanewline
\ \ \isacommand{then}\isamarkupfalse%
\ \isacommand{have}\isamarkupfalse%
\ p{\isadigit{1}}{\isacharcolon}{\kern0pt}\ {\isachardoublequoteopen}{\isacharbraceleft}{\kern0pt}\ x\ {\isasymin}\ X{\isachardot}{\kern0pt}\ P{\isacharparenleft}{\kern0pt}x{\isacharparenright}{\kern0pt}\ {\isacharbraceright}{\kern0pt}\ {\isasymtimes}\ Y\ {\isasymin}\ M{\isachardoublequoteclose}\ \isacommand{using}\isamarkupfalse%
\ cartprod{\isacharunderscore}{\kern0pt}closed\ assms\ \isacommand{by}\isamarkupfalse%
\ auto\ \isanewline
\ \ \isacommand{have}\isamarkupfalse%
\ {\isachardoublequoteopen}{\isacharbraceleft}{\kern0pt}\ x\ {\isasymin}\ X{\isachardot}{\kern0pt}\ P{\isacharparenleft}{\kern0pt}x{\isacharparenright}{\kern0pt}\ {\isacharbraceright}{\kern0pt}\ {\isasymtimes}\ Y\ {\isacharequal}{\kern0pt}\ {\isacharbraceleft}{\kern0pt}\ {\isacharless}{\kern0pt}x{\isacharcomma}{\kern0pt}\ y{\isachargreater}{\kern0pt}\ {\isasymin}\ X\ {\isasymtimes}\ Y{\isachardot}{\kern0pt}\ P{\isacharparenleft}{\kern0pt}x{\isacharparenright}{\kern0pt}\ {\isacharbraceright}{\kern0pt}{\isachardoublequoteclose}\ \isacommand{by}\isamarkupfalse%
\ auto\isanewline
\ \ \isacommand{then}\isamarkupfalse%
\ \isacommand{show}\isamarkupfalse%
\ {\isachardoublequoteopen}{\isacharbraceleft}{\kern0pt}\ {\isacharless}{\kern0pt}x{\isacharcomma}{\kern0pt}\ y{\isachargreater}{\kern0pt}\ {\isasymin}\ X\ {\isasymtimes}\ Y{\isachardot}{\kern0pt}\ P{\isacharparenleft}{\kern0pt}x{\isacharparenright}{\kern0pt}\ {\isacharbraceright}{\kern0pt}\ {\isasymin}\ M{\isachardoublequoteclose}\ \isacommand{using}\isamarkupfalse%
\ p{\isadigit{1}}\ \isacommand{by}\isamarkupfalse%
\ auto\ \isanewline
\isacommand{qed}\isamarkupfalse%
%
\endisatagproof
{\isafoldproof}%
%
\isadelimproof
\ \isanewline
%
\endisadelimproof
\isanewline
\isacommand{lemma}\isamarkupfalse%
\ singleton{\isacharunderscore}{\kern0pt}iff{\isadigit{2}}\ {\isacharcolon}{\kern0pt}\ {\isachardoublequoteopen}x\ {\isasymin}\ M\ {\isasymLongrightarrow}\ y\ {\isasymin}\ M\ {\isasymLongrightarrow}\ x\ {\isacharequal}{\kern0pt}\ {\isacharbraceleft}{\kern0pt}y{\isacharbraceright}{\kern0pt}\ {\isasymlongleftrightarrow}\ {\isacharparenleft}{\kern0pt}{\isasymforall}z{\isasymin}M{\isachardot}{\kern0pt}\ z\ {\isasymin}\ x\ {\isasymlongleftrightarrow}\ z\ {\isacharequal}{\kern0pt}\ y{\isacharparenright}{\kern0pt}{\isachardoublequoteclose}\isanewline
%
\isadelimproof
%
\endisadelimproof
%
\isatagproof
\isacommand{proof}\isamarkupfalse%
\ {\isacharparenleft}{\kern0pt}auto{\isacharparenright}{\kern0pt}\ \ \isanewline
\ \ \isacommand{assume}\isamarkupfalse%
\ assms\ {\isacharcolon}{\kern0pt}\ {\isachardoublequoteopen}x\ {\isasymin}\ M{\isachardoublequoteclose}\ {\isachardoublequoteopen}y\ {\isasymin}\ M{\isachardoublequoteclose}\ {\isachardoublequoteopen}{\isasymforall}z{\isasymin}M{\isachardot}{\kern0pt}\ z\ {\isasymin}\ x\ {\isasymlongleftrightarrow}\ z\ {\isacharequal}{\kern0pt}\ y{\isachardoublequoteclose}\isanewline
\ \ \isacommand{have}\isamarkupfalse%
\ p\ {\isacharcolon}{\kern0pt}\ {\isachardoublequoteopen}{\isasymforall}z{\isasymin}M{\isachardot}{\kern0pt}\ z\ {\isasymin}\ x\ {\isasymlongleftrightarrow}\ z\ {\isasymin}\ {\isacharbraceleft}{\kern0pt}y{\isacharbraceright}{\kern0pt}{\isachardoublequoteclose}\ \isacommand{using}\isamarkupfalse%
\ singleton{\isacharunderscore}{\kern0pt}iff\ assms\ \isacommand{by}\isamarkupfalse%
\ auto\ \isanewline
\ \ \isacommand{have}\isamarkupfalse%
\ {\isachardoublequoteopen}{\isasymforall}z{\isachardot}{\kern0pt}\ z\ {\isasymin}\ x\ {\isasymlongleftrightarrow}\ z\ {\isasymin}\ {\isacharbraceleft}{\kern0pt}y{\isacharbraceright}{\kern0pt}{\isachardoublequoteclose}\isanewline
\ \ \ \ \isacommand{apply}\isamarkupfalse%
\ {\isacharparenleft}{\kern0pt}rule{\isacharunderscore}{\kern0pt}tac\ allI{\isacharsemicolon}{\kern0pt}\ rule{\isacharunderscore}{\kern0pt}tac\ iffI{\isacharparenright}{\kern0pt}\ \isanewline
\ \ \isacommand{proof}\isamarkupfalse%
\ {\isacharminus}{\kern0pt}\ \isanewline
\ \ \ \ \isacommand{fix}\isamarkupfalse%
\ z\ \isacommand{assume}\isamarkupfalse%
\ zin\ {\isacharcolon}{\kern0pt}\ {\isachardoublequoteopen}z\ {\isasymin}\ x{\isachardoublequoteclose}\ \isanewline
\ \ \ \ \isacommand{then}\isamarkupfalse%
\ \isacommand{have}\isamarkupfalse%
\ {\isachardoublequoteopen}z\ {\isasymin}\ M{\isachardoublequoteclose}\ \isacommand{using}\isamarkupfalse%
\ assms\ transM\ \isacommand{by}\isamarkupfalse%
\ auto\ \isanewline
\ \ \ \ \isacommand{then}\isamarkupfalse%
\ \isacommand{show}\isamarkupfalse%
\ {\isachardoublequoteopen}z\ {\isasymin}\ {\isacharbraceleft}{\kern0pt}y{\isacharbraceright}{\kern0pt}{\isachardoublequoteclose}\ \isacommand{using}\isamarkupfalse%
\ p\ zin\ \isacommand{by}\isamarkupfalse%
\ auto\isanewline
\ \ \isacommand{next}\isamarkupfalse%
\ \isanewline
\ \ \ \ \isacommand{fix}\isamarkupfalse%
\ z\ \isacommand{assume}\isamarkupfalse%
\ zin\ {\isacharcolon}{\kern0pt}\ {\isachardoublequoteopen}z\ {\isasymin}\ {\isacharbraceleft}{\kern0pt}y{\isacharbraceright}{\kern0pt}{\isachardoublequoteclose}\ \isanewline
\ \ \ \ \isacommand{have}\isamarkupfalse%
\ ysin\ {\isacharcolon}{\kern0pt}\ {\isachardoublequoteopen}{\isacharbraceleft}{\kern0pt}y{\isacharbraceright}{\kern0pt}\ {\isasymin}\ M{\isachardoublequoteclose}\ \isacommand{using}\isamarkupfalse%
\ assms\ singleton{\isacharunderscore}{\kern0pt}in{\isacharunderscore}{\kern0pt}M{\isacharunderscore}{\kern0pt}iff\ \isacommand{by}\isamarkupfalse%
\ auto\ \isanewline
\ \ \ \ \isacommand{then}\isamarkupfalse%
\ \isacommand{have}\isamarkupfalse%
\ {\isachardoublequoteopen}z\ {\isasymin}\ M{\isachardoublequoteclose}\ \isacommand{using}\isamarkupfalse%
\ zin\ transM\ \isacommand{by}\isamarkupfalse%
\ auto\ \isanewline
\ \ \ \ \isacommand{then}\isamarkupfalse%
\ \isacommand{show}\isamarkupfalse%
\ {\isachardoublequoteopen}z\ {\isasymin}\ x{\isachardoublequoteclose}\ \isacommand{using}\isamarkupfalse%
\ p\ zin\ \isacommand{by}\isamarkupfalse%
\ auto\ \isanewline
\ \ \isacommand{qed}\isamarkupfalse%
\isanewline
\ \ \isacommand{then}\isamarkupfalse%
\ \isacommand{show}\isamarkupfalse%
\ {\isachardoublequoteopen}x\ {\isacharequal}{\kern0pt}\ {\isacharbraceleft}{\kern0pt}y{\isacharbraceright}{\kern0pt}{\isachardoublequoteclose}\ \isacommand{by}\isamarkupfalse%
\ auto\ \isanewline
\isacommand{qed}\isamarkupfalse%
%
\endisatagproof
{\isafoldproof}%
%
\isadelimproof
\isanewline
%
\endisadelimproof
\isanewline
\isacommand{definition}\isamarkupfalse%
\ MVset\ \isakeyword{where}\ {\isachardoublequoteopen}MVset{\isacharparenleft}{\kern0pt}a{\isacharparenright}{\kern0pt}\ {\isasymequiv}\ {\isacharbraceleft}{\kern0pt}\ x\ {\isasymin}\ Vset{\isacharparenleft}{\kern0pt}a{\isacharparenright}{\kern0pt}{\isachardot}{\kern0pt}\ x\ {\isasymin}\ M\ {\isacharbraceright}{\kern0pt}{\isachardoublequoteclose}\ \isanewline
\isanewline
\isacommand{lemma}\isamarkupfalse%
\ MVsetI\ {\isacharcolon}{\kern0pt}\ {\isachardoublequoteopen}x\ {\isasymin}\ M\ {\isasymLongrightarrow}\ rank{\isacharparenleft}{\kern0pt}x{\isacharparenright}{\kern0pt}\ {\isacharless}{\kern0pt}\ a\ {\isasymLongrightarrow}\ x\ {\isasymin}\ MVset{\isacharparenleft}{\kern0pt}a{\isacharparenright}{\kern0pt}{\isachardoublequoteclose}\ \isanewline
%
\isadelimproof
\ \ %
\endisadelimproof
%
\isatagproof
\isacommand{unfolding}\isamarkupfalse%
\ MVset{\isacharunderscore}{\kern0pt}def\ \isacommand{using}\isamarkupfalse%
\ lt{\isacharunderscore}{\kern0pt}Ord{\isadigit{2}}\ Ord{\isacharunderscore}{\kern0pt}rank\ VsetI\ \isacommand{by}\isamarkupfalse%
\ auto%
\endisatagproof
{\isafoldproof}%
%
\isadelimproof
\isanewline
%
\endisadelimproof
\isanewline
\isacommand{lemma}\isamarkupfalse%
\ MVsetD\ {\isacharcolon}{\kern0pt}\ {\isachardoublequoteopen}Ord{\isacharparenleft}{\kern0pt}a{\isacharparenright}{\kern0pt}\ {\isasymLongrightarrow}\ x\ {\isasymin}\ MVset{\isacharparenleft}{\kern0pt}a{\isacharparenright}{\kern0pt}\ {\isasymLongrightarrow}\ x\ {\isasymin}\ M\ {\isasymand}\ rank{\isacharparenleft}{\kern0pt}x{\isacharparenright}{\kern0pt}\ {\isacharless}{\kern0pt}\ a{\isachardoublequoteclose}\ \isanewline
%
\isadelimproof
\ \ %
\endisadelimproof
%
\isatagproof
\isacommand{unfolding}\isamarkupfalse%
\ MVset{\isacharunderscore}{\kern0pt}def\ \isacommand{using}\isamarkupfalse%
\ VsetD\ \isacommand{by}\isamarkupfalse%
\ auto%
\endisatagproof
{\isafoldproof}%
%
\isadelimproof
\ \isanewline
%
\endisadelimproof
\isanewline
\isacommand{lemma}\isamarkupfalse%
\ MVset{\isacharunderscore}{\kern0pt}in{\isacharunderscore}{\kern0pt}M\ {\isacharcolon}{\kern0pt}\ {\isachardoublequoteopen}{\isasymAnd}a{\isachardot}{\kern0pt}\ a\ {\isasymin}\ M\ {\isasymLongrightarrow}\ Ord{\isacharparenleft}{\kern0pt}a{\isacharparenright}{\kern0pt}\ {\isasymLongrightarrow}\ MVset{\isacharparenleft}{\kern0pt}a{\isacharparenright}{\kern0pt}\ {\isasymin}\ M{\isachardoublequoteclose}\ \isanewline
%
\isadelimproof
\ \ %
\endisadelimproof
%
\isatagproof
\isacommand{using}\isamarkupfalse%
\ Vset{\isacharunderscore}{\kern0pt}closed\ \isacommand{unfolding}\isamarkupfalse%
\ MVset{\isacharunderscore}{\kern0pt}def\ \isacommand{by}\isamarkupfalse%
\ auto%
\endisatagproof
{\isafoldproof}%
%
\isadelimproof
\ \isanewline
%
\endisadelimproof
\isanewline
\isacommand{lemma}\isamarkupfalse%
\ MVset{\isacharunderscore}{\kern0pt}succ{\isacharunderscore}{\kern0pt}elem{\isacharunderscore}{\kern0pt}elem\ {\isacharcolon}{\kern0pt}\ {\isachardoublequoteopen}\ x\ {\isasymin}\ MVset{\isacharparenleft}{\kern0pt}succ{\isacharparenleft}{\kern0pt}a{\isacharparenright}{\kern0pt}{\isacharparenright}{\kern0pt}\ {\isasymLongrightarrow}\ y\ {\isasymin}\ x\ {\isasymLongrightarrow}\ y\ {\isasymin}\ MVset{\isacharparenleft}{\kern0pt}a{\isacharparenright}{\kern0pt}{\isachardoublequoteclose}\ \isanewline
%
\isadelimproof
%
\endisadelimproof
%
\isatagproof
\isacommand{proof}\isamarkupfalse%
\ {\isacharminus}{\kern0pt}\ \isanewline
\ \ \isacommand{assume}\isamarkupfalse%
\ assms\ {\isacharcolon}{\kern0pt}\ {\isachardoublequoteopen}x\ {\isasymin}\ MVset{\isacharparenleft}{\kern0pt}succ{\isacharparenleft}{\kern0pt}a{\isacharparenright}{\kern0pt}{\isacharparenright}{\kern0pt}{\isachardoublequoteclose}\ {\isachardoublequoteopen}y\ {\isasymin}\ x{\isachardoublequoteclose}\isanewline
\ \ \isacommand{then}\isamarkupfalse%
\ \isacommand{have}\isamarkupfalse%
\ {\isachardoublequoteopen}x\ {\isasymin}\ Vset{\isacharparenleft}{\kern0pt}succ{\isacharparenleft}{\kern0pt}a{\isacharparenright}{\kern0pt}{\isacharparenright}{\kern0pt}{\isachardoublequoteclose}\ \isacommand{unfolding}\isamarkupfalse%
\ MVset{\isacharunderscore}{\kern0pt}def\ \isacommand{by}\isamarkupfalse%
\ auto\ \isanewline
\ \ \isacommand{then}\isamarkupfalse%
\ \isacommand{have}\isamarkupfalse%
\ {\isachardoublequoteopen}x\ {\isasymin}\ Pow{\isacharparenleft}{\kern0pt}Vset{\isacharparenleft}{\kern0pt}a{\isacharparenright}{\kern0pt}{\isacharparenright}{\kern0pt}{\isachardoublequoteclose}\ \isacommand{using}\isamarkupfalse%
\ Vset{\isacharunderscore}{\kern0pt}succ\ \isacommand{by}\isamarkupfalse%
\ auto\ \isanewline
\ \ \isacommand{then}\isamarkupfalse%
\ \isacommand{have}\isamarkupfalse%
\ {\isachardoublequoteopen}x\ {\isasymsubseteq}\ Vset{\isacharparenleft}{\kern0pt}a{\isacharparenright}{\kern0pt}{\isachardoublequoteclose}\ \isacommand{by}\isamarkupfalse%
\ auto\ \isanewline
\ \ \isacommand{then}\isamarkupfalse%
\ \isacommand{have}\isamarkupfalse%
\ yin\ {\isacharcolon}{\kern0pt}\ {\isachardoublequoteopen}y\ {\isasymin}\ Vset{\isacharparenleft}{\kern0pt}a{\isacharparenright}{\kern0pt}{\isachardoublequoteclose}\ \isacommand{using}\isamarkupfalse%
\ assms\ \isacommand{by}\isamarkupfalse%
\ auto\ \isanewline
\ \ \isacommand{have}\isamarkupfalse%
\ {\isachardoublequoteopen}x\ {\isasymin}\ M{\isachardoublequoteclose}\ \isacommand{using}\isamarkupfalse%
\ assms\ \isacommand{unfolding}\isamarkupfalse%
\ MVset{\isacharunderscore}{\kern0pt}def\ \isacommand{by}\isamarkupfalse%
\ auto\ \isanewline
\ \ \isacommand{then}\isamarkupfalse%
\ \isacommand{have}\isamarkupfalse%
\ {\isachardoublequoteopen}y\ {\isasymin}\ M{\isachardoublequoteclose}\ \isacommand{using}\isamarkupfalse%
\ transM\ assms\ \isacommand{by}\isamarkupfalse%
\ auto\ \isanewline
\ \ \isacommand{then}\isamarkupfalse%
\ \isacommand{show}\isamarkupfalse%
\ {\isachardoublequoteopen}\ y\ {\isasymin}\ MVset{\isacharparenleft}{\kern0pt}a{\isacharparenright}{\kern0pt}{\isachardoublequoteclose}\ \isacommand{using}\isamarkupfalse%
\ yin\ \isacommand{unfolding}\isamarkupfalse%
\ MVset{\isacharunderscore}{\kern0pt}def\ \isacommand{by}\isamarkupfalse%
\ auto\ \isanewline
\isacommand{qed}\isamarkupfalse%
%
\endisatagproof
{\isafoldproof}%
%
\isadelimproof
\ \isanewline
%
\endisadelimproof
\isanewline
\isacommand{lemma}\isamarkupfalse%
\ MVset{\isacharunderscore}{\kern0pt}trans\ {\isacharcolon}{\kern0pt}\ {\isachardoublequoteopen}Ord{\isacharparenleft}{\kern0pt}a{\isacharparenright}{\kern0pt}\ {\isasymLongrightarrow}\ x\ {\isasymin}\ MVset{\isacharparenleft}{\kern0pt}a{\isacharparenright}{\kern0pt}\ {\isasymLongrightarrow}\ y\ {\isasymin}\ x\ {\isasymLongrightarrow}\ y\ {\isasymin}\ MVset{\isacharparenleft}{\kern0pt}a{\isacharparenright}{\kern0pt}{\isachardoublequoteclose}\ \isanewline
%
\isadelimproof
%
\endisadelimproof
%
\isatagproof
\isacommand{proof}\isamarkupfalse%
\ {\isacharminus}{\kern0pt}\ \isanewline
\ \ \isacommand{assume}\isamarkupfalse%
\ assms{\isacharcolon}{\kern0pt}\ {\isachardoublequoteopen}Ord{\isacharparenleft}{\kern0pt}a{\isacharparenright}{\kern0pt}{\isachardoublequoteclose}\ {\isachardoublequoteopen}x\ {\isasymin}\ MVset{\isacharparenleft}{\kern0pt}a{\isacharparenright}{\kern0pt}{\isachardoublequoteclose}\ {\isachardoublequoteopen}y\ {\isasymin}\ x{\isachardoublequoteclose}\isanewline
\ \ \isacommand{then}\isamarkupfalse%
\ \isacommand{have}\isamarkupfalse%
\ yinM\ {\isacharcolon}{\kern0pt}\ {\isachardoublequoteopen}y\ {\isasymin}\ M{\isachardoublequoteclose}\ \isacommand{unfolding}\isamarkupfalse%
\ MVset{\isacharunderscore}{\kern0pt}def\ \isacommand{using}\isamarkupfalse%
\ transM\ \isacommand{by}\isamarkupfalse%
\ auto\ \isanewline
\ \ \isacommand{have}\isamarkupfalse%
\ {\isachardoublequoteopen}y\ {\isasymin}\ Vset{\isacharparenleft}{\kern0pt}a{\isacharparenright}{\kern0pt}{\isachardoublequoteclose}\ \isanewline
\ \ \ \ \isacommand{apply}\isamarkupfalse%
\ {\isacharparenleft}{\kern0pt}rule\ VsetI{\isacharsemicolon}{\kern0pt}\ rule{\isacharunderscore}{\kern0pt}tac\ j{\isacharequal}{\kern0pt}{\isachardoublequoteopen}rank{\isacharparenleft}{\kern0pt}x{\isacharparenright}{\kern0pt}{\isachardoublequoteclose}\ \isakeyword{in}\ lt{\isacharunderscore}{\kern0pt}trans{\isacharparenright}{\kern0pt}\ \isanewline
\ \ \ \ \isacommand{using}\isamarkupfalse%
\ assms\ rank{\isacharunderscore}{\kern0pt}lt\ \isacommand{apply}\isamarkupfalse%
\ simp\ \isanewline
\ \ \ \ \isacommand{using}\isamarkupfalse%
\ assms\ \isacommand{unfolding}\isamarkupfalse%
\ MVset{\isacharunderscore}{\kern0pt}def\ \isacommand{using}\isamarkupfalse%
\ VsetD\ \isacommand{by}\isamarkupfalse%
\ auto\isanewline
\ \ \isacommand{then}\isamarkupfalse%
\ \isacommand{show}\isamarkupfalse%
\ {\isachardoublequoteopen}y\ {\isasymin}\ MVset{\isacharparenleft}{\kern0pt}a{\isacharparenright}{\kern0pt}{\isachardoublequoteclose}\ \isacommand{using}\isamarkupfalse%
\ yinM\ \isacommand{unfolding}\isamarkupfalse%
\ MVset{\isacharunderscore}{\kern0pt}def\ \isacommand{by}\isamarkupfalse%
\ auto\ \isanewline
\isacommand{qed}\isamarkupfalse%
%
\endisatagproof
{\isafoldproof}%
%
\isadelimproof
\isanewline
%
\endisadelimproof
\isanewline
\isacommand{lemma}\isamarkupfalse%
\ MVset{\isacharunderscore}{\kern0pt}domain\ {\isacharcolon}{\kern0pt}\ {\isachardoublequoteopen}Ord{\isacharparenleft}{\kern0pt}a{\isacharparenright}{\kern0pt}\ {\isasymLongrightarrow}\ x\ {\isasymin}\ MVset{\isacharparenleft}{\kern0pt}a{\isacharparenright}{\kern0pt}\ {\isasymLongrightarrow}\ y\ {\isasymin}\ domain{\isacharparenleft}{\kern0pt}x{\isacharparenright}{\kern0pt}\ {\isasymLongrightarrow}\ y\ {\isasymin}\ MVset{\isacharparenleft}{\kern0pt}a{\isacharparenright}{\kern0pt}{\isachardoublequoteclose}\ \isanewline
%
\isadelimproof
%
\endisadelimproof
%
\isatagproof
\isacommand{proof}\isamarkupfalse%
\ {\isacharminus}{\kern0pt}\ \isanewline
\ \ \isacommand{assume}\isamarkupfalse%
\ assms{\isacharcolon}{\kern0pt}\ {\isachardoublequoteopen}Ord{\isacharparenleft}{\kern0pt}a{\isacharparenright}{\kern0pt}{\isachardoublequoteclose}\ {\isachardoublequoteopen}x\ {\isasymin}\ MVset{\isacharparenleft}{\kern0pt}a{\isacharparenright}{\kern0pt}{\isachardoublequoteclose}\ {\isachardoublequoteopen}y\ {\isasymin}\ domain{\isacharparenleft}{\kern0pt}x{\isacharparenright}{\kern0pt}{\isachardoublequoteclose}\isanewline
\ \ \isacommand{then}\isamarkupfalse%
\ \isacommand{obtain}\isamarkupfalse%
\ z\ \isakeyword{where}\ zh\ {\isacharcolon}{\kern0pt}\ {\isachardoublequoteopen}{\isacharless}{\kern0pt}y{\isacharcomma}{\kern0pt}\ z{\isachargreater}{\kern0pt}\ {\isasymin}\ x{\isachardoublequoteclose}\ \isacommand{unfolding}\isamarkupfalse%
\ domain{\isacharunderscore}{\kern0pt}def\ \isacommand{by}\isamarkupfalse%
\ auto\isanewline
\ \ \isacommand{have}\isamarkupfalse%
\ {\isachardoublequoteopen}domain{\isacharparenleft}{\kern0pt}x{\isacharparenright}{\kern0pt}\ {\isasymin}\ M{\isachardoublequoteclose}\ \isanewline
\ \ \ \ \isacommand{using}\isamarkupfalse%
\ domain{\isacharunderscore}{\kern0pt}closed\ assms\ \isacommand{unfolding}\isamarkupfalse%
\ MVset{\isacharunderscore}{\kern0pt}def\ \isacommand{by}\isamarkupfalse%
\ auto\ \isanewline
\ \ \isacommand{then}\isamarkupfalse%
\ \isacommand{have}\isamarkupfalse%
\ yinM\ {\isacharcolon}{\kern0pt}\ {\isachardoublequoteopen}y\ {\isasymin}\ M{\isachardoublequoteclose}\ \isacommand{using}\isamarkupfalse%
\ transM\ assms\ \isacommand{by}\isamarkupfalse%
\ auto\isanewline
\ \ \isacommand{have}\isamarkupfalse%
\ {\isachardoublequoteopen}rank{\isacharparenleft}{\kern0pt}y{\isacharparenright}{\kern0pt}\ {\isacharless}{\kern0pt}\ a{\isachardoublequoteclose}\ \isanewline
\ \ \ \ \isacommand{apply}\isamarkupfalse%
\ {\isacharparenleft}{\kern0pt}rule{\isacharunderscore}{\kern0pt}tac\ j{\isacharequal}{\kern0pt}{\isachardoublequoteopen}rank{\isacharparenleft}{\kern0pt}x{\isacharparenright}{\kern0pt}{\isachardoublequoteclose}\ \isakeyword{in}\ lt{\isacharunderscore}{\kern0pt}trans{\isacharparenright}{\kern0pt}\ \isanewline
\ \ \ \ \isacommand{apply}\isamarkupfalse%
\ {\isacharparenleft}{\kern0pt}rule{\isacharunderscore}{\kern0pt}tac\ j{\isacharequal}{\kern0pt}{\isachardoublequoteopen}rank{\isacharparenleft}{\kern0pt}{\isacharless}{\kern0pt}y{\isacharcomma}{\kern0pt}\ z{\isachargreater}{\kern0pt}{\isacharparenright}{\kern0pt}{\isachardoublequoteclose}\ \isakeyword{in}\ lt{\isacharunderscore}{\kern0pt}trans{\isacharparenright}{\kern0pt}\ \isanewline
\ \ \ \ \isacommand{using}\isamarkupfalse%
\ rank{\isacharunderscore}{\kern0pt}pair{\isadigit{1}}\ \isacommand{apply}\isamarkupfalse%
\ simp\ \isanewline
\ \ \ \ \isacommand{using}\isamarkupfalse%
\ rank{\isacharunderscore}{\kern0pt}lt\ zh\ \isacommand{apply}\isamarkupfalse%
\ simp\ \isanewline
\ \ \ \ \isacommand{using}\isamarkupfalse%
\ VsetD\ assms\ \isacommand{unfolding}\isamarkupfalse%
\ MVset{\isacharunderscore}{\kern0pt}def\ \isacommand{by}\isamarkupfalse%
\ auto\ \isanewline
\ \ \isacommand{then}\isamarkupfalse%
\ \isacommand{show}\isamarkupfalse%
\ {\isachardoublequoteopen}y\ {\isasymin}\ MVset{\isacharparenleft}{\kern0pt}a{\isacharparenright}{\kern0pt}{\isachardoublequoteclose}\ \isacommand{unfolding}\isamarkupfalse%
\ MVset{\isacharunderscore}{\kern0pt}def\ \isacommand{using}\isamarkupfalse%
\ VsetI\ yinM\ \isacommand{by}\isamarkupfalse%
\ auto\ \isanewline
\isacommand{qed}\isamarkupfalse%
%
\endisatagproof
{\isafoldproof}%
%
\isadelimproof
\isanewline
%
\endisadelimproof
\isanewline
\isacommand{lemma}\isamarkupfalse%
\ MVset{\isacharunderscore}{\kern0pt}succ{\isacharunderscore}{\kern0pt}rank{\isacharunderscore}{\kern0pt}in\ {\isacharcolon}{\kern0pt}\ {\isachardoublequoteopen}x\ {\isasymin}\ M\ {\isasymLongrightarrow}\ x\ {\isasymin}\ MVset{\isacharparenleft}{\kern0pt}succ{\isacharparenleft}{\kern0pt}rank{\isacharparenleft}{\kern0pt}x{\isacharparenright}{\kern0pt}{\isacharparenright}{\kern0pt}{\isacharparenright}{\kern0pt}{\isachardoublequoteclose}\ \isanewline
%
\isadelimproof
%
\endisadelimproof
%
\isatagproof
\isacommand{proof}\isamarkupfalse%
\ {\isacharminus}{\kern0pt}\ \isanewline
\ \ \isacommand{assume}\isamarkupfalse%
\ xinM\ {\isacharcolon}{\kern0pt}\ {\isachardoublequoteopen}x\ {\isasymin}\ M{\isachardoublequoteclose}\ \isanewline
\ \ \isacommand{have}\isamarkupfalse%
\ {\isachardoublequoteopen}x\ {\isasymin}\ Vset{\isacharparenleft}{\kern0pt}succ{\isacharparenleft}{\kern0pt}rank{\isacharparenleft}{\kern0pt}x{\isacharparenright}{\kern0pt}{\isacharparenright}{\kern0pt}{\isacharparenright}{\kern0pt}{\isachardoublequoteclose}\ \isanewline
\ \ \ \ \isacommand{using}\isamarkupfalse%
\ Univ{\isachardot}{\kern0pt}VsetI\ \isacommand{by}\isamarkupfalse%
\ auto\ \isanewline
\ \ \isacommand{then}\isamarkupfalse%
\ \isacommand{show}\isamarkupfalse%
\ {\isacharquery}{\kern0pt}thesis\ \isacommand{unfolding}\isamarkupfalse%
\ MVset{\isacharunderscore}{\kern0pt}def\ \isacommand{using}\isamarkupfalse%
\ xinM\ \isacommand{by}\isamarkupfalse%
\ auto\ \isanewline
\isacommand{qed}\isamarkupfalse%
%
\endisatagproof
{\isafoldproof}%
%
\isadelimproof
\ \isanewline
%
\endisadelimproof
\isanewline
\isacommand{lemma}\isamarkupfalse%
\ zero{\isacharunderscore}{\kern0pt}in{\isacharunderscore}{\kern0pt}MVset\ {\isacharcolon}{\kern0pt}\ {\isachardoublequoteopen}Ord{\isacharparenleft}{\kern0pt}a{\isacharparenright}{\kern0pt}\ {\isasymLongrightarrow}\ {\isadigit{0}}\ {\isasymin}\ MVset{\isacharparenleft}{\kern0pt}succ{\isacharparenleft}{\kern0pt}a{\isacharparenright}{\kern0pt}{\isacharparenright}{\kern0pt}{\isachardoublequoteclose}\isanewline
%
\isadelimproof
%
\endisadelimproof
%
\isatagproof
\isacommand{proof}\isamarkupfalse%
\ {\isacharminus}{\kern0pt}\ \isanewline
\ \ \isacommand{assume}\isamarkupfalse%
\ assm\ {\isacharcolon}{\kern0pt}\ {\isachardoublequoteopen}Ord{\isacharparenleft}{\kern0pt}a{\isacharparenright}{\kern0pt}{\isachardoublequoteclose}\isanewline
\ \ \isacommand{then}\isamarkupfalse%
\ \isacommand{have}\isamarkupfalse%
\ {\isachardoublequoteopen}{\isadigit{0}}\ {\isasymin}\ Vset{\isacharparenleft}{\kern0pt}succ{\isacharparenleft}{\kern0pt}a{\isacharparenright}{\kern0pt}{\isacharparenright}{\kern0pt}{\isachardoublequoteclose}\ \isanewline
\ \ \ \ \isacommand{apply}\isamarkupfalse%
\ {\isacharparenleft}{\kern0pt}rule{\isacharunderscore}{\kern0pt}tac\ VsetI{\isacharparenright}{\kern0pt}\ \isanewline
\ \ \ \ \isacommand{using}\isamarkupfalse%
\ Ord{\isacharunderscore}{\kern0pt}rank\ Ord{\isacharunderscore}{\kern0pt}{\isadigit{0}}{\isacharunderscore}{\kern0pt}le\ \isacommand{by}\isamarkupfalse%
\ auto\isanewline
\ \ \isacommand{then}\isamarkupfalse%
\ \isacommand{show}\isamarkupfalse%
\ {\isacharquery}{\kern0pt}thesis\ \isacommand{unfolding}\isamarkupfalse%
\ MVset{\isacharunderscore}{\kern0pt}def\ \isacommand{using}\isamarkupfalse%
\ zero{\isacharunderscore}{\kern0pt}in{\isacharunderscore}{\kern0pt}M\ \isacommand{by}\isamarkupfalse%
\ auto\isanewline
\isacommand{qed}\isamarkupfalse%
%
\endisatagproof
{\isafoldproof}%
%
\isadelimproof
\isanewline
%
\endisadelimproof
\isanewline
\isacommand{lemma}\isamarkupfalse%
\ MVset{\isacharunderscore}{\kern0pt}pair\ {\isacharcolon}{\kern0pt}\ {\isachardoublequoteopen}Ord{\isacharparenleft}{\kern0pt}a{\isacharparenright}{\kern0pt}\ {\isasymLongrightarrow}\ {\isacharless}{\kern0pt}x{\isacharcomma}{\kern0pt}\ y{\isachargreater}{\kern0pt}\ {\isasymin}\ MVset{\isacharparenleft}{\kern0pt}a{\isacharparenright}{\kern0pt}\ {\isasymLongrightarrow}\ x\ {\isasymin}\ MVset{\isacharparenleft}{\kern0pt}a{\isacharparenright}{\kern0pt}\ {\isasymand}\ y\ {\isasymin}\ MVset{\isacharparenleft}{\kern0pt}a{\isacharparenright}{\kern0pt}{\isachardoublequoteclose}\ \isanewline
%
\isadelimproof
\ \ %
\endisadelimproof
%
\isatagproof
\isacommand{unfolding}\isamarkupfalse%
\ MVset{\isacharunderscore}{\kern0pt}def\isanewline
\ \ \isacommand{using}\isamarkupfalse%
\ pair{\isacharunderscore}{\kern0pt}in{\isacharunderscore}{\kern0pt}M{\isacharunderscore}{\kern0pt}iff\isanewline
\ \ \isacommand{apply}\isamarkupfalse%
\ simp\isanewline
\ \ \isacommand{apply}\isamarkupfalse%
{\isacharparenleft}{\kern0pt}rule\ conjI{\isacharcomma}{\kern0pt}\ rule{\isacharunderscore}{\kern0pt}tac\ j{\isacharequal}{\kern0pt}{\isachardoublequoteopen}rank{\isacharparenleft}{\kern0pt}{\isacharless}{\kern0pt}x{\isacharcomma}{\kern0pt}\ y{\isachargreater}{\kern0pt}{\isacharparenright}{\kern0pt}{\isachardoublequoteclose}\ \isakeyword{in}\ lt{\isacharunderscore}{\kern0pt}trans{\isacharcomma}{\kern0pt}\ simp\ add{\isacharcolon}{\kern0pt}rank{\isacharunderscore}{\kern0pt}pair{\isadigit{1}}{\isacharcomma}{\kern0pt}\ simp{\isacharparenright}{\kern0pt}\isanewline
\ \ \isacommand{apply}\isamarkupfalse%
{\isacharparenleft}{\kern0pt}rule{\isacharunderscore}{\kern0pt}tac\ j{\isacharequal}{\kern0pt}{\isachardoublequoteopen}rank{\isacharparenleft}{\kern0pt}{\isacharless}{\kern0pt}x{\isacharcomma}{\kern0pt}\ y{\isachargreater}{\kern0pt}{\isacharparenright}{\kern0pt}{\isachardoublequoteclose}\ \isakeyword{in}\ lt{\isacharunderscore}{\kern0pt}trans{\isacharcomma}{\kern0pt}\ simp\ add{\isacharcolon}{\kern0pt}rank{\isacharunderscore}{\kern0pt}pair{\isadigit{2}}{\isacharcomma}{\kern0pt}\ simp{\isacharparenright}{\kern0pt}\isanewline
\ \ \isacommand{done}\isamarkupfalse%
%
\endisatagproof
{\isafoldproof}%
%
\isadelimproof
\isanewline
%
\endisadelimproof
\ \isanewline
\isacommand{lemma}\isamarkupfalse%
\ domain{\isacharunderscore}{\kern0pt}elem{\isacharunderscore}{\kern0pt}in{\isacharunderscore}{\kern0pt}M\ {\isacharcolon}{\kern0pt}\ {\isachardoublequoteopen}x\ {\isasymin}\ M\ {\isasymLongrightarrow}\ y\ {\isasymin}\ domain{\isacharparenleft}{\kern0pt}x{\isacharparenright}{\kern0pt}\ {\isasymLongrightarrow}\ y\ {\isasymin}\ M{\isachardoublequoteclose}\ \isanewline
%
\isadelimproof
%
\endisadelimproof
%
\isatagproof
\isacommand{proof}\isamarkupfalse%
\ {\isacharminus}{\kern0pt}\ \isanewline
\ \ \isacommand{assume}\isamarkupfalse%
\ assms\ {\isacharcolon}{\kern0pt}\ {\isachardoublequoteopen}x\ {\isasymin}\ M{\isachardoublequoteclose}\ {\isachardoublequoteopen}y\ {\isasymin}\ domain{\isacharparenleft}{\kern0pt}x{\isacharparenright}{\kern0pt}{\isachardoublequoteclose}\isanewline
\ \ \isacommand{then}\isamarkupfalse%
\ \isacommand{obtain}\isamarkupfalse%
\ p\ \isakeyword{where}\ {\isachardoublequoteopen}{\isacharless}{\kern0pt}y{\isacharcomma}{\kern0pt}\ p{\isachargreater}{\kern0pt}\ {\isasymin}\ x{\isachardoublequoteclose}\ \isacommand{by}\isamarkupfalse%
\ auto\ \isanewline
\ \ \isacommand{then}\isamarkupfalse%
\ \isacommand{have}\isamarkupfalse%
\ {\isachardoublequoteopen}{\isacharless}{\kern0pt}y{\isacharcomma}{\kern0pt}\ p{\isachargreater}{\kern0pt}\ {\isasymin}\ M{\isachardoublequoteclose}\ \isacommand{using}\isamarkupfalse%
\ assms\ transM\ \isacommand{by}\isamarkupfalse%
\ auto\ \isanewline
\ \ \isacommand{then}\isamarkupfalse%
\ \isacommand{show}\isamarkupfalse%
\ {\isachardoublequoteopen}y\ {\isasymin}\ M{\isachardoublequoteclose}\ \isacommand{using}\isamarkupfalse%
\ pair{\isacharunderscore}{\kern0pt}in{\isacharunderscore}{\kern0pt}M{\isacharunderscore}{\kern0pt}iff\ \isacommand{by}\isamarkupfalse%
\ auto\ \isanewline
\isacommand{qed}\isamarkupfalse%
%
\endisatagproof
{\isafoldproof}%
%
\isadelimproof
\isanewline
%
\endisadelimproof
\isanewline
\isacommand{schematic{\isacharunderscore}{\kern0pt}goal}\isamarkupfalse%
\ refl{\isacharunderscore}{\kern0pt}pair{\isacharunderscore}{\kern0pt}fm{\isacharunderscore}{\kern0pt}auto{\isacharcolon}{\kern0pt}\isanewline
\ \ \isakeyword{assumes}\isanewline
\ \ \ \ {\isachardoublequoteopen}nth{\isacharparenleft}{\kern0pt}{\isadigit{0}}{\isacharcomma}{\kern0pt}env{\isacharparenright}{\kern0pt}\ {\isacharequal}{\kern0pt}\ v{\isachardoublequoteclose}\ \isanewline
\ \ \ \ {\isachardoublequoteopen}env\ {\isasymin}\ list{\isacharparenleft}{\kern0pt}M{\isacharparenright}{\kern0pt}{\isachardoublequoteclose}\isanewline
\ \isakeyword{shows}\isanewline
\ \ \ \ {\isachardoublequoteopen}{\isacharparenleft}{\kern0pt}{\isasymexists}x\ {\isasymin}\ M{\isachardot}{\kern0pt}\ pair{\isacharparenleft}{\kern0pt}{\isacharhash}{\kern0pt}{\isacharhash}{\kern0pt}M{\isacharcomma}{\kern0pt}\ x{\isacharcomma}{\kern0pt}\ x{\isacharcomma}{\kern0pt}\ v{\isacharparenright}{\kern0pt}{\isacharparenright}{\kern0pt}\ \isanewline
\ \ \ \ \ {\isasymlongleftrightarrow}\ sats{\isacharparenleft}{\kern0pt}M{\isacharcomma}{\kern0pt}{\isacharquery}{\kern0pt}fm{\isacharparenleft}{\kern0pt}{\isadigit{0}}{\isacharparenright}{\kern0pt}{\isacharcomma}{\kern0pt}env{\isacharparenright}{\kern0pt}{\isachardoublequoteclose}\isanewline
%
\isadelimproof
\ \ %
\endisadelimproof
%
\isatagproof
\isacommand{by}\isamarkupfalse%
\ {\isacharparenleft}{\kern0pt}insert\ assms\ {\isacharsemicolon}{\kern0pt}\ {\isacharparenleft}{\kern0pt}rule\ sep{\isacharunderscore}{\kern0pt}rules\ {\isacharbar}{\kern0pt}\ simp{\isacharparenright}{\kern0pt}{\isacharplus}{\kern0pt}{\isacharparenright}{\kern0pt}%
\endisatagproof
{\isafoldproof}%
%
\isadelimproof
\isanewline
%
\endisadelimproof
\isanewline
\isacommand{lemma}\isamarkupfalse%
\ refl{\isacharunderscore}{\kern0pt}rel{\isacharunderscore}{\kern0pt}closed\ {\isacharcolon}{\kern0pt}\ \ {\isachardoublequoteopen}{\isasymAnd}\ A{\isachardot}{\kern0pt}\ A\ {\isasymin}\ M\ {\isasymLongrightarrow}\ {\isacharbraceleft}{\kern0pt}\ {\isacharless}{\kern0pt}x{\isacharcomma}{\kern0pt}\ x{\isachargreater}{\kern0pt}\ {\isachardot}{\kern0pt}\ x\ {\isasymin}\ A\ {\isacharbraceright}{\kern0pt}\ {\isasymin}\ M{\isachardoublequoteclose}\ \isanewline
%
\isadelimproof
%
\endisadelimproof
%
\isatagproof
\isacommand{proof}\isamarkupfalse%
\ {\isacharminus}{\kern0pt}\ \isanewline
\ \ \isacommand{fix}\isamarkupfalse%
\ A\ \isacommand{assume}\isamarkupfalse%
\ assm\ {\isacharcolon}{\kern0pt}\ {\isachardoublequoteopen}A\ {\isasymin}\ M{\isachardoublequoteclose}\ \isanewline
\ \ \isacommand{define}\isamarkupfalse%
\ fm\ \isakeyword{where}\ {\isachardoublequoteopen}fm\ {\isasymequiv}\ Exists{\isacharparenleft}{\kern0pt}pair{\isacharunderscore}{\kern0pt}fm{\isacharparenleft}{\kern0pt}{\isadigit{0}}{\isacharcomma}{\kern0pt}\ {\isadigit{0}}{\isacharcomma}{\kern0pt}\ {\isadigit{1}}{\isacharparenright}{\kern0pt}{\isacharparenright}{\kern0pt}\ {\isachardoublequoteclose}\ \isanewline
\ \ \isacommand{have}\isamarkupfalse%
\ sep\ {\isacharcolon}{\kern0pt}\ {\isachardoublequoteopen}separation{\isacharparenleft}{\kern0pt}{\isacharhash}{\kern0pt}{\isacharhash}{\kern0pt}M{\isacharcomma}{\kern0pt}\ {\isasymlambda}x{\isachardot}{\kern0pt}\ sats{\isacharparenleft}{\kern0pt}M{\isacharcomma}{\kern0pt}\ fm{\isacharcomma}{\kern0pt}\ {\isacharbrackleft}{\kern0pt}x{\isacharbrackright}{\kern0pt}\ {\isacharat}{\kern0pt}\ {\isacharbrackleft}{\kern0pt}{\isacharbrackright}{\kern0pt}{\isacharparenright}{\kern0pt}{\isacharparenright}{\kern0pt}{\isachardoublequoteclose}\ \isanewline
\ \ \ \ \isacommand{apply}\isamarkupfalse%
\ {\isacharparenleft}{\kern0pt}rule{\isacharunderscore}{\kern0pt}tac\ separation{\isacharunderscore}{\kern0pt}ax{\isacharparenright}{\kern0pt}\ \isacommand{unfolding}\isamarkupfalse%
\ fm{\isacharunderscore}{\kern0pt}def\ \isacommand{apply}\isamarkupfalse%
\ simp{\isacharunderscore}{\kern0pt}all\ \ \isanewline
\ \ \ \ \isacommand{apply}\isamarkupfalse%
\ {\isacharparenleft}{\kern0pt}simp\ del{\isacharcolon}{\kern0pt}FOL{\isacharunderscore}{\kern0pt}sats{\isacharunderscore}{\kern0pt}iff\ pair{\isacharunderscore}{\kern0pt}abs\ add{\isacharcolon}{\kern0pt}\ fm{\isacharunderscore}{\kern0pt}defs\ nat{\isacharunderscore}{\kern0pt}simp{\isacharunderscore}{\kern0pt}union{\isacharparenright}{\kern0pt}\ \isacommand{done}\isamarkupfalse%
\ \isanewline
\ \ \isacommand{then}\isamarkupfalse%
\ \isacommand{have}\isamarkupfalse%
\ H{\isacharcolon}{\kern0pt}\ {\isachardoublequoteopen}{\isacharbraceleft}{\kern0pt}\ v\ {\isasymin}\ A{\isasymtimes}A{\isachardot}{\kern0pt}\ sats{\isacharparenleft}{\kern0pt}M{\isacharcomma}{\kern0pt}\ fm{\isacharcomma}{\kern0pt}\ {\isacharbrackleft}{\kern0pt}v{\isacharbrackright}{\kern0pt}{\isacharat}{\kern0pt}{\isacharbrackleft}{\kern0pt}{\isacharbrackright}{\kern0pt}{\isacharparenright}{\kern0pt}\ {\isacharbraceright}{\kern0pt}\ {\isasymin}\ M{\isachardoublequoteclose}\ \ \isanewline
\ \ \ \ \isacommand{apply}\isamarkupfalse%
\ {\isacharparenleft}{\kern0pt}rule{\isacharunderscore}{\kern0pt}tac\ separation{\isacharunderscore}{\kern0pt}notation{\isacharparenright}{\kern0pt}\ \isacommand{using}\isamarkupfalse%
\ assm\ cartprod{\isacharunderscore}{\kern0pt}closed\ \isacommand{apply}\isamarkupfalse%
\ auto\ \isacommand{done}\isamarkupfalse%
\ \isanewline
\ \ \isanewline
\ \ \isacommand{have}\isamarkupfalse%
\ {\isachardoublequoteopen}{\isasymAnd}v{\isachardot}{\kern0pt}\ v\ {\isasymin}\ A{\isasymtimes}A\ {\isasymLongrightarrow}\ sats{\isacharparenleft}{\kern0pt}M{\isacharcomma}{\kern0pt}\ fm{\isacharcomma}{\kern0pt}\ {\isacharbrackleft}{\kern0pt}v{\isacharbrackright}{\kern0pt}{\isacharat}{\kern0pt}{\isacharbrackleft}{\kern0pt}{\isacharbrackright}{\kern0pt}{\isacharparenright}{\kern0pt}\ {\isasymlongleftrightarrow}\ {\isacharparenleft}{\kern0pt}{\isasymexists}x{\isachardot}{\kern0pt}\ v\ {\isacharequal}{\kern0pt}\ {\isacharless}{\kern0pt}x{\isacharcomma}{\kern0pt}\ x{\isachargreater}{\kern0pt}{\isacharparenright}{\kern0pt}{\isachardoublequoteclose}\isanewline
\ \ \ \ \isacommand{apply}\isamarkupfalse%
\ {\isacharparenleft}{\kern0pt}rule{\isacharunderscore}{\kern0pt}tac\ Q{\isacharequal}{\kern0pt}{\isachardoublequoteopen}{\isacharparenleft}{\kern0pt}{\isasymexists}x\ {\isasymin}\ M{\isachardot}{\kern0pt}\ pair{\isacharparenleft}{\kern0pt}{\isacharhash}{\kern0pt}{\isacharhash}{\kern0pt}M{\isacharcomma}{\kern0pt}\ x{\isacharcomma}{\kern0pt}\ x{\isacharcomma}{\kern0pt}\ v{\isacharparenright}{\kern0pt}{\isacharparenright}{\kern0pt}\ {\isachardoublequoteclose}\ \isakeyword{in}\ iff{\isacharunderscore}{\kern0pt}trans{\isacharparenright}{\kern0pt}\ \isanewline
\ \ \ \ \isacommand{apply}\isamarkupfalse%
\ {\isacharparenleft}{\kern0pt}rule\ iff{\isacharunderscore}{\kern0pt}flip{\isacharparenright}{\kern0pt}\ \isacommand{unfolding}\isamarkupfalse%
\ fm{\isacharunderscore}{\kern0pt}def\isanewline
\ \ \ \ \isacommand{apply}\isamarkupfalse%
\ {\isacharparenleft}{\kern0pt}rule{\isacharunderscore}{\kern0pt}tac\ refl{\isacharunderscore}{\kern0pt}pair{\isacharunderscore}{\kern0pt}fm{\isacharunderscore}{\kern0pt}auto{\isacharparenright}{\kern0pt}\ \isacommand{apply}\isamarkupfalse%
\ simp\ \isacommand{using}\isamarkupfalse%
\ assm\ transM\ cartprod{\isacharunderscore}{\kern0pt}closed\ \isacommand{apply}\isamarkupfalse%
\ simp\ \isanewline
\ \ \ \ \isacommand{apply}\isamarkupfalse%
\ {\isacharparenleft}{\kern0pt}rule{\isacharunderscore}{\kern0pt}tac\ P{\isacharequal}{\kern0pt}{\isachardoublequoteopen}v\ {\isasymin}\ M{\isachardoublequoteclose}\ \isakeyword{in}\ mp{\isacharparenright}{\kern0pt}\ \isacommand{using}\isamarkupfalse%
\ pair{\isacharunderscore}{\kern0pt}abs\ pair{\isacharunderscore}{\kern0pt}in{\isacharunderscore}{\kern0pt}M{\isacharunderscore}{\kern0pt}iff\ assm\ transM\ \isacommand{apply}\isamarkupfalse%
\ auto\ \isacommand{done}\isamarkupfalse%
\ \isanewline
\ \ \isacommand{then}\isamarkupfalse%
\ \isacommand{have}\isamarkupfalse%
\ {\isachardoublequoteopen}{\isacharbraceleft}{\kern0pt}\ v\ {\isasymin}\ A{\isasymtimes}A{\isachardot}{\kern0pt}\ sats{\isacharparenleft}{\kern0pt}M{\isacharcomma}{\kern0pt}\ fm{\isacharcomma}{\kern0pt}\ {\isacharbrackleft}{\kern0pt}v{\isacharbrackright}{\kern0pt}{\isacharat}{\kern0pt}{\isacharbrackleft}{\kern0pt}{\isacharbrackright}{\kern0pt}{\isacharparenright}{\kern0pt}\ {\isacharbraceright}{\kern0pt}\ {\isacharequal}{\kern0pt}\ {\isacharbraceleft}{\kern0pt}\ v\ {\isasymin}\ A\ {\isasymtimes}\ A{\isachardot}{\kern0pt}\ {\isasymexists}x{\isachardot}{\kern0pt}\ v\ {\isacharequal}{\kern0pt}\ {\isacharless}{\kern0pt}x{\isacharcomma}{\kern0pt}\ x{\isachargreater}{\kern0pt}\ {\isacharbraceright}{\kern0pt}{\isachardoublequoteclose}\ \isacommand{by}\isamarkupfalse%
\ auto\isanewline
\ \ \isacommand{also}\isamarkupfalse%
\ \isacommand{have}\isamarkupfalse%
\ {\isachardoublequoteopen}{\isachardot}{\kern0pt}{\isachardot}{\kern0pt}{\isachardot}{\kern0pt}\ {\isacharequal}{\kern0pt}\ {\isacharbraceleft}{\kern0pt}\ {\isacharless}{\kern0pt}x{\isacharcomma}{\kern0pt}\ x{\isachargreater}{\kern0pt}{\isachardot}{\kern0pt}\ x\ {\isasymin}\ A\ {\isacharbraceright}{\kern0pt}{\isachardoublequoteclose}\ \isacommand{by}\isamarkupfalse%
\ auto\isanewline
\ \ \isacommand{finally}\isamarkupfalse%
\ \isacommand{have}\isamarkupfalse%
\ {\isachardoublequoteopen}\ {\isacharbraceleft}{\kern0pt}v\ {\isasymin}\ A\ {\isasymtimes}\ A\ {\isachardot}{\kern0pt}\ M{\isacharcomma}{\kern0pt}\ {\isacharbrackleft}{\kern0pt}v{\isacharbrackright}{\kern0pt}\ {\isacharat}{\kern0pt}\ {\isacharbrackleft}{\kern0pt}{\isacharbrackright}{\kern0pt}\ {\isasymTurnstile}\ fm{\isacharbraceright}{\kern0pt}\ {\isacharequal}{\kern0pt}\ {\isacharbraceleft}{\kern0pt}{\isasymlangle}x{\isacharcomma}{\kern0pt}\ x{\isasymrangle}\ {\isachardot}{\kern0pt}\ x\ {\isasymin}\ A{\isacharbraceright}{\kern0pt}\ {\isachardoublequoteclose}\ \isacommand{by}\isamarkupfalse%
\ simp\ \isanewline
\ \ \isacommand{then}\isamarkupfalse%
\ \isacommand{show}\isamarkupfalse%
\ {\isachardoublequoteopen}{\isacharbraceleft}{\kern0pt}{\isasymlangle}x{\isacharcomma}{\kern0pt}\ x{\isasymrangle}\ {\isachardot}{\kern0pt}\ x\ {\isasymin}\ A{\isacharbraceright}{\kern0pt}\ {\isasymin}\ M{\isachardoublequoteclose}\ \isacommand{using}\isamarkupfalse%
\ H\ \isacommand{by}\isamarkupfalse%
\ auto\ \isanewline
\isacommand{qed}\isamarkupfalse%
%
\endisatagproof
{\isafoldproof}%
%
\isadelimproof
\ \isanewline
%
\endisadelimproof
\isanewline
\isacommand{lemma}\isamarkupfalse%
\ id{\isacharunderscore}{\kern0pt}closed\ {\isacharcolon}{\kern0pt}\ {\isachardoublequoteopen}{\isasymAnd}A{\isachardot}{\kern0pt}\ A\ {\isasymin}\ M\ {\isasymLongrightarrow}\ id{\isacharparenleft}{\kern0pt}A{\isacharparenright}{\kern0pt}\ {\isasymin}\ M{\isachardoublequoteclose}\ \isanewline
%
\isadelimproof
\ \ %
\endisadelimproof
%
\isatagproof
\isacommand{unfolding}\isamarkupfalse%
\ id{\isacharunderscore}{\kern0pt}def\ lam{\isacharunderscore}{\kern0pt}def\ \isacommand{using}\isamarkupfalse%
\ refl{\isacharunderscore}{\kern0pt}rel{\isacharunderscore}{\kern0pt}closed\ \isacommand{by}\isamarkupfalse%
\ auto%
\endisatagproof
{\isafoldproof}%
%
\isadelimproof
\ \isanewline
%
\endisadelimproof
\isanewline
\isacommand{lemma}\isamarkupfalse%
\ int{\isacharunderscore}{\kern0pt}closed\ {\isacharcolon}{\kern0pt}\ \isanewline
\ \ {\isachardoublequoteopen}A\ {\isasymin}\ M\ {\isasymLongrightarrow}\ B\ {\isasymin}\ M\ {\isasymLongrightarrow}\ A\ {\isasyminter}\ B\ {\isasymin}\ M{\isachardoublequoteclose}\ \isanewline
%
\isadelimproof
%
\endisadelimproof
%
\isatagproof
\isacommand{proof}\isamarkupfalse%
\ {\isacharminus}{\kern0pt}\ \isanewline
\ \ \isacommand{assume}\isamarkupfalse%
\ assms\ {\isacharcolon}{\kern0pt}\ {\isachardoublequoteopen}A\ {\isasymin}\ M{\isachardoublequoteclose}\ {\isachardoublequoteopen}B\ {\isasymin}\ M{\isachardoublequoteclose}\ \isanewline
\ \ \isacommand{have}\isamarkupfalse%
\ {\isachardoublequoteopen}separation{\isacharparenleft}{\kern0pt}{\isacharhash}{\kern0pt}{\isacharhash}{\kern0pt}M{\isacharcomma}{\kern0pt}\ {\isasymlambda}x{\isachardot}{\kern0pt}\ sats{\isacharparenleft}{\kern0pt}M{\isacharcomma}{\kern0pt}\ Member{\isacharparenleft}{\kern0pt}{\isadigit{0}}{\isacharcomma}{\kern0pt}\ {\isadigit{1}}{\isacharparenright}{\kern0pt}{\isacharcomma}{\kern0pt}\ {\isacharbrackleft}{\kern0pt}x{\isacharbrackright}{\kern0pt}\ {\isacharat}{\kern0pt}\ {\isacharbrackleft}{\kern0pt}B{\isacharbrackright}{\kern0pt}{\isacharparenright}{\kern0pt}{\isacharparenright}{\kern0pt}{\isachardoublequoteclose}\ \isanewline
\ \ \ \ \isacommand{apply}\isamarkupfalse%
{\isacharparenleft}{\kern0pt}rule\ separation{\isacharunderscore}{\kern0pt}ax{\isacharparenright}{\kern0pt}\ \isanewline
\ \ \ \ \isacommand{using}\isamarkupfalse%
\ assms\ \isanewline
\ \ \ \ \isacommand{apply}\isamarkupfalse%
\ simp{\isacharunderscore}{\kern0pt}all\isanewline
\ \ \ \ \isacommand{by}\isamarkupfalse%
\ {\isacharparenleft}{\kern0pt}simp\ del{\isacharcolon}{\kern0pt}FOL{\isacharunderscore}{\kern0pt}sats{\isacharunderscore}{\kern0pt}iff\ pair{\isacharunderscore}{\kern0pt}abs\ add{\isacharcolon}{\kern0pt}\ fm{\isacharunderscore}{\kern0pt}defs\ nat{\isacharunderscore}{\kern0pt}simp{\isacharunderscore}{\kern0pt}union{\isacharparenright}{\kern0pt}\ \ \isanewline
\ \ \isacommand{then}\isamarkupfalse%
\ \isacommand{have}\isamarkupfalse%
\ H\ {\isacharcolon}{\kern0pt}\ {\isachardoublequoteopen}{\isacharbraceleft}{\kern0pt}\ x\ {\isasymin}\ A{\isachardot}{\kern0pt}\ sats{\isacharparenleft}{\kern0pt}M{\isacharcomma}{\kern0pt}\ Member{\isacharparenleft}{\kern0pt}{\isadigit{0}}{\isacharcomma}{\kern0pt}\ {\isadigit{1}}{\isacharparenright}{\kern0pt}{\isacharcomma}{\kern0pt}\ {\isacharbrackleft}{\kern0pt}x{\isacharbrackright}{\kern0pt}\ {\isacharat}{\kern0pt}\ {\isacharbrackleft}{\kern0pt}B{\isacharbrackright}{\kern0pt}{\isacharparenright}{\kern0pt}\ {\isacharbraceright}{\kern0pt}\ {\isasymin}\ M{\isachardoublequoteclose}\ \isanewline
\ \ \ \ \isacommand{apply}\isamarkupfalse%
{\isacharparenleft}{\kern0pt}rule\ separation{\isacharunderscore}{\kern0pt}notation{\isacharparenright}{\kern0pt}\ \isanewline
\ \ \ \ \isacommand{using}\isamarkupfalse%
\ assms\ \isanewline
\ \ \ \ \isacommand{by}\isamarkupfalse%
\ auto\ \isanewline
\ \ \isacommand{have}\isamarkupfalse%
\ {\isachardoublequoteopen}{\isacharbraceleft}{\kern0pt}\ x\ {\isasymin}\ A{\isachardot}{\kern0pt}\ sats{\isacharparenleft}{\kern0pt}M{\isacharcomma}{\kern0pt}\ Member{\isacharparenleft}{\kern0pt}{\isadigit{0}}{\isacharcomma}{\kern0pt}\ {\isadigit{1}}{\isacharparenright}{\kern0pt}{\isacharcomma}{\kern0pt}\ {\isacharbrackleft}{\kern0pt}x{\isacharbrackright}{\kern0pt}\ {\isacharat}{\kern0pt}\ {\isacharbrackleft}{\kern0pt}B{\isacharbrackright}{\kern0pt}{\isacharparenright}{\kern0pt}\ {\isacharbraceright}{\kern0pt}\ {\isacharequal}{\kern0pt}\ A\ {\isasyminter}\ B{\isachardoublequoteclose}\ \isanewline
\ \ \ \ \isacommand{using}\isamarkupfalse%
\ assms\ transM\ \isacommand{by}\isamarkupfalse%
\ auto\ \isanewline
\ \ \isacommand{then}\isamarkupfalse%
\ \isacommand{show}\isamarkupfalse%
\ {\isachardoublequoteopen}A\ {\isasyminter}\ B\ {\isasymin}\ M{\isachardoublequoteclose}\ \isacommand{using}\isamarkupfalse%
\ H\ \isacommand{by}\isamarkupfalse%
\ auto\isanewline
\isacommand{qed}\isamarkupfalse%
%
\endisatagproof
{\isafoldproof}%
%
\isadelimproof
\isanewline
%
\endisadelimproof
\isanewline
\isacommand{lemma}\isamarkupfalse%
\ fst{\isacharunderscore}{\kern0pt}closed\ {\isacharcolon}{\kern0pt}\ \isanewline
\ \ {\isachardoublequoteopen}x\ {\isasymin}\ M\ {\isasymLongrightarrow}\ fst{\isacharparenleft}{\kern0pt}x{\isacharparenright}{\kern0pt}\ {\isasymin}\ M{\isachardoublequoteclose}\ \isanewline
%
\isadelimproof
\ \ %
\endisadelimproof
%
\isatagproof
\isacommand{unfolding}\isamarkupfalse%
\ fst{\isacharunderscore}{\kern0pt}def\isanewline
\ \ \isacommand{apply}\isamarkupfalse%
{\isacharparenleft}{\kern0pt}cases\ {\isachardoublequoteopen}{\isasymexists}a\ b{\isachardot}{\kern0pt}\ x\ {\isacharequal}{\kern0pt}\ {\isacharless}{\kern0pt}a{\isacharcomma}{\kern0pt}\ b{\isachargreater}{\kern0pt}{\isachardoublequoteclose}{\isacharparenright}{\kern0pt}\ \isanewline
\ \ \isacommand{apply}\isamarkupfalse%
\ clarify\isanewline
\ \ \ \isacommand{apply}\isamarkupfalse%
{\isacharparenleft}{\kern0pt}rename{\isacharunderscore}{\kern0pt}tac\ a\ b{\isacharcomma}{\kern0pt}\ subst\ the{\isacharunderscore}{\kern0pt}equality{\isacharparenright}{\kern0pt}\isanewline
\ \ \isacommand{using}\isamarkupfalse%
\ pair{\isacharunderscore}{\kern0pt}in{\isacharunderscore}{\kern0pt}M{\isacharunderscore}{\kern0pt}iff\isanewline
\ \ \ \ \ \isacommand{apply}\isamarkupfalse%
\ auto{\isacharbrackleft}{\kern0pt}{\isadigit{3}}{\isacharbrackright}{\kern0pt}\isanewline
\ \ \isacommand{apply}\isamarkupfalse%
{\isacharparenleft}{\kern0pt}subst\ the{\isacharunderscore}{\kern0pt}{\isadigit{0}}{\isacharparenright}{\kern0pt}\isanewline
\ \ \isacommand{using}\isamarkupfalse%
\ zero{\isacharunderscore}{\kern0pt}in{\isacharunderscore}{\kern0pt}M\isanewline
\ \ \isacommand{by}\isamarkupfalse%
\ auto%
\endisatagproof
{\isafoldproof}%
%
\isadelimproof
\isanewline
%
\endisadelimproof
\isanewline
\isacommand{lemma}\isamarkupfalse%
\ snd{\isacharunderscore}{\kern0pt}closed\ {\isacharcolon}{\kern0pt}\ \isanewline
\ \ {\isachardoublequoteopen}x\ {\isasymin}\ M\ {\isasymLongrightarrow}\ snd{\isacharparenleft}{\kern0pt}x{\isacharparenright}{\kern0pt}\ {\isasymin}\ M{\isachardoublequoteclose}\ \isanewline
%
\isadelimproof
\ \ %
\endisadelimproof
%
\isatagproof
\isacommand{unfolding}\isamarkupfalse%
\ snd{\isacharunderscore}{\kern0pt}def\isanewline
\ \ \isacommand{apply}\isamarkupfalse%
{\isacharparenleft}{\kern0pt}cases\ {\isachardoublequoteopen}{\isasymexists}a\ b{\isachardot}{\kern0pt}\ x\ {\isacharequal}{\kern0pt}\ {\isacharless}{\kern0pt}a{\isacharcomma}{\kern0pt}\ b{\isachargreater}{\kern0pt}{\isachardoublequoteclose}{\isacharparenright}{\kern0pt}\ \isanewline
\ \ \isacommand{apply}\isamarkupfalse%
\ clarify\isanewline
\ \ \ \isacommand{apply}\isamarkupfalse%
{\isacharparenleft}{\kern0pt}rename{\isacharunderscore}{\kern0pt}tac\ a\ b{\isacharcomma}{\kern0pt}\ subst\ the{\isacharunderscore}{\kern0pt}equality{\isacharparenright}{\kern0pt}\isanewline
\ \ \isacommand{using}\isamarkupfalse%
\ pair{\isacharunderscore}{\kern0pt}in{\isacharunderscore}{\kern0pt}M{\isacharunderscore}{\kern0pt}iff\isanewline
\ \ \ \ \ \isacommand{apply}\isamarkupfalse%
\ auto{\isacharbrackleft}{\kern0pt}{\isadigit{3}}{\isacharbrackright}{\kern0pt}\isanewline
\ \ \isacommand{apply}\isamarkupfalse%
{\isacharparenleft}{\kern0pt}subst\ the{\isacharunderscore}{\kern0pt}{\isadigit{0}}{\isacharparenright}{\kern0pt}\isanewline
\ \ \isacommand{using}\isamarkupfalse%
\ zero{\isacharunderscore}{\kern0pt}in{\isacharunderscore}{\kern0pt}M\isanewline
\ \ \isacommand{by}\isamarkupfalse%
\ auto%
\endisatagproof
{\isafoldproof}%
%
\isadelimproof
\isanewline
%
\endisadelimproof
\isanewline
\isacommand{lemma}\isamarkupfalse%
\ least{\isacharunderscore}{\kern0pt}cong\ {\isacharcolon}{\kern0pt}\ \isanewline
\ \ {\isachardoublequoteopen}{\isasymAnd}P\ Q\ a{\isachardot}{\kern0pt}\ a\ {\isasymin}\ M\ {\isasymLongrightarrow}\ {\isacharparenleft}{\kern0pt}{\isasymAnd}a{\isachardot}{\kern0pt}\ a\ {\isasymin}\ M\ {\isasymLongrightarrow}\ Ord{\isacharparenleft}{\kern0pt}a{\isacharparenright}{\kern0pt}\ {\isasymLongrightarrow}\ P{\isacharparenleft}{\kern0pt}a{\isacharparenright}{\kern0pt}\ {\isasymlongleftrightarrow}\ Q{\isacharparenleft}{\kern0pt}a{\isacharparenright}{\kern0pt}{\isacharparenright}{\kern0pt}\ {\isasymLongrightarrow}\ least{\isacharparenleft}{\kern0pt}{\isacharhash}{\kern0pt}{\isacharhash}{\kern0pt}M{\isacharcomma}{\kern0pt}\ P{\isacharcomma}{\kern0pt}\ a{\isacharparenright}{\kern0pt}\ {\isasymlongleftrightarrow}\ least{\isacharparenleft}{\kern0pt}{\isacharhash}{\kern0pt}{\isacharhash}{\kern0pt}M{\isacharcomma}{\kern0pt}\ Q{\isacharcomma}{\kern0pt}\ a{\isacharparenright}{\kern0pt}{\isachardoublequoteclose}\ \isanewline
%
\isadelimproof
\ \ %
\endisadelimproof
%
\isatagproof
\isacommand{unfolding}\isamarkupfalse%
\ least{\isacharunderscore}{\kern0pt}def\ \isanewline
\ \ \isacommand{by}\isamarkupfalse%
\ auto%
\endisatagproof
{\isafoldproof}%
%
\isadelimproof
\isanewline
%
\endisadelimproof
\isanewline
\isanewline
\isanewline
\isacommand{end}\isamarkupfalse%
\isanewline
%
\isadelimtheory
%
\endisadelimtheory
%
\isatagtheory
\isacommand{end}\isamarkupfalse%
%
\endisatagtheory
{\isafoldtheory}%
%
\isadelimtheory
%
\endisadelimtheory
%
\end{isabellebody}%
\endinput
%:%file=~/source/repos/ZF-notAC/code/Utilities_M.thy%:%
%:%10=1%:%
%:%11=1%:%
%:%12=2%:%
%:%13=3%:%
%:%14=4%:%
%:%15=5%:%
%:%20=5%:%
%:%23=6%:%
%:%24=7%:%
%:%25=7%:%
%:%26=8%:%
%:%27=9%:%
%:%28=10%:%
%:%29=10%:%
%:%31=10%:%
%:%35=10%:%
%:%36=10%:%
%:%43=10%:%
%:%44=11%:%
%:%45=11%:%
%:%47=11%:%
%:%51=11%:%
%:%52=11%:%
%:%59=11%:%
%:%60=12%:%
%:%61=13%:%
%:%62=13%:%
%:%63=14%:%
%:%66=15%:%
%:%70=15%:%
%:%71=15%:%
%:%72=16%:%
%:%73=16%:%
%:%74=17%:%
%:%75=17%:%
%:%76=18%:%
%:%77=18%:%
%:%78=18%:%
%:%79=19%:%
%:%80=19%:%
%:%81=19%:%
%:%82=19%:%
%:%83=19%:%
%:%84=20%:%
%:%85=20%:%
%:%86=21%:%
%:%87=21%:%
%:%88=21%:%
%:%89=21%:%
%:%90=22%:%
%:%91=22%:%
%:%92=22%:%
%:%93=22%:%
%:%94=22%:%
%:%95=23%:%
%:%96=23%:%
%:%97=23%:%
%:%98=23%:%
%:%99=24%:%
%:%100=24%:%
%:%101=24%:%
%:%102=24%:%
%:%103=24%:%
%:%104=25%:%
%:%105=25%:%
%:%106=25%:%
%:%107=25%:%
%:%108=25%:%
%:%109=26%:%
%:%110=26%:%
%:%111=26%:%
%:%112=26%:%
%:%113=26%:%
%:%114=27%:%
%:%120=27%:%
%:%123=28%:%
%:%124=29%:%
%:%125=29%:%
%:%126=30%:%
%:%129=31%:%
%:%133=31%:%
%:%134=31%:%
%:%135=31%:%
%:%140=31%:%
%:%143=32%:%
%:%144=33%:%
%:%145=33%:%
%:%146=34%:%
%:%149=35%:%
%:%153=35%:%
%:%154=35%:%
%:%155=35%:%
%:%160=35%:%
%:%163=36%:%
%:%164=37%:%
%:%165=37%:%
%:%166=38%:%
%:%169=39%:%
%:%173=39%:%
%:%174=39%:%
%:%175=40%:%
%:%176=40%:%
%:%177=41%:%
%:%178=41%:%
%:%179=41%:%
%:%180=42%:%
%:%181=42%:%
%:%182=42%:%
%:%183=42%:%
%:%184=42%:%
%:%185=43%:%
%:%186=43%:%
%:%187=43%:%
%:%188=43%:%
%:%189=43%:%
%:%190=44%:%
%:%191=44%:%
%:%192=44%:%
%:%193=44%:%
%:%194=45%:%
%:%195=45%:%
%:%196=46%:%
%:%197=46%:%
%:%198=46%:%
%:%199=47%:%
%:%200=47%:%
%:%201=47%:%
%:%202=47%:%
%:%203=48%:%
%:%204=48%:%
%:%205=48%:%
%:%206=48%:%
%:%207=49%:%
%:%208=49%:%
%:%209=49%:%
%:%210=49%:%
%:%211=49%:%
%:%212=50%:%
%:%213=50%:%
%:%214=50%:%
%:%215=50%:%
%:%216=50%:%
%:%217=51%:%
%:%223=51%:%
%:%226=52%:%
%:%227=53%:%
%:%228=53%:%
%:%229=54%:%
%:%236=55%:%
%:%237=55%:%
%:%238=56%:%
%:%239=56%:%
%:%240=57%:%
%:%241=57%:%
%:%242=57%:%
%:%243=57%:%
%:%244=57%:%
%:%245=58%:%
%:%246=58%:%
%:%247=58%:%
%:%248=58%:%
%:%249=59%:%
%:%250=59%:%
%:%251=59%:%
%:%252=60%:%
%:%253=60%:%
%:%254=60%:%
%:%255=60%:%
%:%256=61%:%
%:%257=61%:%
%:%258=61%:%
%:%259=61%:%
%:%260=61%:%
%:%261=62%:%
%:%267=62%:%
%:%270=63%:%
%:%271=64%:%
%:%272=65%:%
%:%273=65%:%
%:%274=66%:%
%:%281=67%:%
%:%282=67%:%
%:%283=68%:%
%:%284=68%:%
%:%285=69%:%
%:%286=69%:%
%:%287=69%:%
%:%288=69%:%
%:%289=69%:%
%:%290=70%:%
%:%291=70%:%
%:%292=70%:%
%:%293=70%:%
%:%294=70%:%
%:%295=71%:%
%:%296=71%:%
%:%297=71%:%
%:%298=72%:%
%:%299=72%:%
%:%300=72%:%
%:%301=72%:%
%:%302=72%:%
%:%303=73%:%
%:%309=73%:%
%:%312=74%:%
%:%313=75%:%
%:%314=75%:%
%:%321=76%:%
%:%322=76%:%
%:%323=77%:%
%:%324=77%:%
%:%325=78%:%
%:%326=78%:%
%:%327=78%:%
%:%328=78%:%
%:%329=79%:%
%:%330=79%:%
%:%331=80%:%
%:%332=80%:%
%:%333=81%:%
%:%334=81%:%
%:%335=82%:%
%:%336=82%:%
%:%337=82%:%
%:%338=83%:%
%:%339=83%:%
%:%340=83%:%
%:%341=83%:%
%:%342=83%:%
%:%343=84%:%
%:%344=84%:%
%:%345=84%:%
%:%346=84%:%
%:%347=84%:%
%:%348=85%:%
%:%349=85%:%
%:%350=86%:%
%:%351=86%:%
%:%352=86%:%
%:%353=87%:%
%:%354=87%:%
%:%355=87%:%
%:%356=87%:%
%:%357=88%:%
%:%358=88%:%
%:%359=88%:%
%:%360=88%:%
%:%361=88%:%
%:%362=89%:%
%:%363=89%:%
%:%364=89%:%
%:%365=89%:%
%:%366=89%:%
%:%367=90%:%
%:%368=90%:%
%:%369=91%:%
%:%370=91%:%
%:%371=91%:%
%:%372=91%:%
%:%373=92%:%
%:%379=92%:%
%:%382=93%:%
%:%383=94%:%
%:%384=94%:%
%:%385=95%:%
%:%386=96%:%
%:%387=96%:%
%:%390=97%:%
%:%394=97%:%
%:%395=97%:%
%:%396=97%:%
%:%397=97%:%
%:%402=97%:%
%:%405=98%:%
%:%406=99%:%
%:%407=99%:%
%:%410=100%:%
%:%414=100%:%
%:%415=100%:%
%:%416=100%:%
%:%417=100%:%
%:%422=100%:%
%:%425=101%:%
%:%426=102%:%
%:%427=102%:%
%:%430=103%:%
%:%434=103%:%
%:%435=103%:%
%:%436=103%:%
%:%437=103%:%
%:%442=103%:%
%:%445=104%:%
%:%446=105%:%
%:%447=105%:%
%:%454=106%:%
%:%455=106%:%
%:%456=107%:%
%:%457=107%:%
%:%458=108%:%
%:%459=108%:%
%:%460=108%:%
%:%461=108%:%
%:%462=108%:%
%:%463=109%:%
%:%464=109%:%
%:%465=109%:%
%:%466=109%:%
%:%467=109%:%
%:%468=110%:%
%:%469=110%:%
%:%470=110%:%
%:%471=110%:%
%:%472=111%:%
%:%473=111%:%
%:%474=111%:%
%:%475=111%:%
%:%476=111%:%
%:%477=112%:%
%:%478=112%:%
%:%479=112%:%
%:%480=112%:%
%:%481=112%:%
%:%482=113%:%
%:%483=113%:%
%:%484=113%:%
%:%485=113%:%
%:%486=113%:%
%:%487=114%:%
%:%488=114%:%
%:%489=114%:%
%:%490=114%:%
%:%491=114%:%
%:%492=114%:%
%:%493=115%:%
%:%499=115%:%
%:%502=116%:%
%:%503=117%:%
%:%504=117%:%
%:%511=118%:%
%:%512=118%:%
%:%513=119%:%
%:%514=119%:%
%:%515=120%:%
%:%516=120%:%
%:%517=120%:%
%:%518=120%:%
%:%519=120%:%
%:%520=120%:%
%:%521=121%:%
%:%522=121%:%
%:%523=122%:%
%:%524=122%:%
%:%525=123%:%
%:%526=123%:%
%:%527=123%:%
%:%528=124%:%
%:%529=124%:%
%:%530=124%:%
%:%531=124%:%
%:%532=124%:%
%:%533=125%:%
%:%534=125%:%
%:%535=125%:%
%:%536=125%:%
%:%537=125%:%
%:%538=125%:%
%:%539=126%:%
%:%545=126%:%
%:%548=127%:%
%:%549=128%:%
%:%550=128%:%
%:%557=129%:%
%:%558=129%:%
%:%559=130%:%
%:%560=130%:%
%:%561=131%:%
%:%562=131%:%
%:%563=131%:%
%:%564=131%:%
%:%565=131%:%
%:%566=132%:%
%:%567=132%:%
%:%568=133%:%
%:%569=133%:%
%:%570=133%:%
%:%571=133%:%
%:%572=134%:%
%:%573=134%:%
%:%574=134%:%
%:%575=134%:%
%:%576=134%:%
%:%577=135%:%
%:%578=135%:%
%:%579=136%:%
%:%580=136%:%
%:%581=137%:%
%:%582=137%:%
%:%583=138%:%
%:%584=138%:%
%:%585=138%:%
%:%586=139%:%
%:%587=139%:%
%:%588=139%:%
%:%589=140%:%
%:%590=140%:%
%:%591=140%:%
%:%592=140%:%
%:%593=141%:%
%:%594=141%:%
%:%595=141%:%
%:%596=141%:%
%:%597=141%:%
%:%598=141%:%
%:%599=142%:%
%:%605=142%:%
%:%608=143%:%
%:%609=144%:%
%:%610=144%:%
%:%617=145%:%
%:%618=145%:%
%:%619=146%:%
%:%620=146%:%
%:%621=147%:%
%:%622=147%:%
%:%623=148%:%
%:%624=148%:%
%:%625=148%:%
%:%626=149%:%
%:%627=149%:%
%:%628=149%:%
%:%629=149%:%
%:%630=149%:%
%:%631=149%:%
%:%632=150%:%
%:%638=150%:%
%:%641=151%:%
%:%642=152%:%
%:%643=152%:%
%:%650=153%:%
%:%651=153%:%
%:%652=154%:%
%:%653=154%:%
%:%654=155%:%
%:%655=155%:%
%:%656=155%:%
%:%657=156%:%
%:%658=156%:%
%:%659=157%:%
%:%660=157%:%
%:%661=157%:%
%:%662=158%:%
%:%663=158%:%
%:%664=158%:%
%:%665=158%:%
%:%666=158%:%
%:%667=158%:%
%:%668=159%:%
%:%674=159%:%
%:%677=160%:%
%:%678=161%:%
%:%679=161%:%
%:%682=162%:%
%:%686=162%:%
%:%687=162%:%
%:%688=163%:%
%:%689=163%:%
%:%690=164%:%
%:%691=164%:%
%:%692=165%:%
%:%693=165%:%
%:%694=166%:%
%:%695=166%:%
%:%696=167%:%
%:%702=167%:%
%:%705=168%:%
%:%706=169%:%
%:%707=169%:%
%:%714=170%:%
%:%715=170%:%
%:%716=171%:%
%:%717=171%:%
%:%718=172%:%
%:%719=172%:%
%:%720=172%:%
%:%721=172%:%
%:%722=173%:%
%:%723=173%:%
%:%724=173%:%
%:%725=173%:%
%:%726=173%:%
%:%727=174%:%
%:%728=174%:%
%:%729=174%:%
%:%730=174%:%
%:%731=174%:%
%:%732=175%:%
%:%738=175%:%
%:%741=176%:%
%:%742=177%:%
%:%743=177%:%
%:%744=178%:%
%:%745=179%:%
%:%746=180%:%
%:%747=181%:%
%:%748=182%:%
%:%749=183%:%
%:%752=184%:%
%:%756=184%:%
%:%757=184%:%
%:%762=184%:%
%:%765=185%:%
%:%766=186%:%
%:%767=186%:%
%:%774=187%:%
%:%775=187%:%
%:%776=188%:%
%:%777=188%:%
%:%778=188%:%
%:%779=189%:%
%:%780=189%:%
%:%781=190%:%
%:%782=190%:%
%:%783=191%:%
%:%784=191%:%
%:%785=191%:%
%:%786=191%:%
%:%787=192%:%
%:%788=192%:%
%:%789=192%:%
%:%790=193%:%
%:%791=193%:%
%:%792=193%:%
%:%793=194%:%
%:%794=194%:%
%:%795=194%:%
%:%796=194%:%
%:%797=194%:%
%:%798=195%:%
%:%799=196%:%
%:%800=196%:%
%:%801=197%:%
%:%802=197%:%
%:%803=198%:%
%:%804=198%:%
%:%805=198%:%
%:%806=199%:%
%:%807=199%:%
%:%808=199%:%
%:%809=199%:%
%:%810=199%:%
%:%811=200%:%
%:%812=200%:%
%:%813=200%:%
%:%814=200%:%
%:%815=200%:%
%:%816=201%:%
%:%817=201%:%
%:%818=201%:%
%:%819=201%:%
%:%820=202%:%
%:%821=202%:%
%:%822=202%:%
%:%823=202%:%
%:%824=203%:%
%:%825=203%:%
%:%826=203%:%
%:%827=203%:%
%:%828=204%:%
%:%829=204%:%
%:%830=204%:%
%:%831=204%:%
%:%832=204%:%
%:%833=205%:%
%:%839=205%:%
%:%842=206%:%
%:%843=207%:%
%:%844=207%:%
%:%847=208%:%
%:%851=208%:%
%:%852=208%:%
%:%853=208%:%
%:%854=208%:%
%:%859=208%:%
%:%862=209%:%
%:%863=210%:%
%:%864=210%:%
%:%865=211%:%
%:%872=212%:%
%:%873=212%:%
%:%874=213%:%
%:%875=213%:%
%:%876=214%:%
%:%877=214%:%
%:%878=215%:%
%:%879=215%:%
%:%880=216%:%
%:%881=216%:%
%:%882=217%:%
%:%883=217%:%
%:%884=218%:%
%:%885=218%:%
%:%886=219%:%
%:%887=219%:%
%:%888=219%:%
%:%889=220%:%
%:%890=220%:%
%:%891=221%:%
%:%892=221%:%
%:%893=222%:%
%:%894=222%:%
%:%895=223%:%
%:%896=223%:%
%:%897=224%:%
%:%898=224%:%
%:%899=224%:%
%:%900=225%:%
%:%901=225%:%
%:%902=225%:%
%:%903=225%:%
%:%904=225%:%
%:%905=226%:%
%:%911=226%:%
%:%914=227%:%
%:%915=228%:%
%:%916=228%:%
%:%917=229%:%
%:%920=230%:%
%:%924=230%:%
%:%925=230%:%
%:%926=231%:%
%:%927=231%:%
%:%928=232%:%
%:%929=232%:%
%:%930=233%:%
%:%931=233%:%
%:%932=234%:%
%:%933=234%:%
%:%934=235%:%
%:%935=235%:%
%:%936=236%:%
%:%937=236%:%
%:%938=237%:%
%:%939=237%:%
%:%940=238%:%
%:%941=238%:%
%:%946=238%:%
%:%949=239%:%
%:%950=240%:%
%:%951=240%:%
%:%952=241%:%
%:%955=242%:%
%:%959=242%:%
%:%960=242%:%
%:%961=243%:%
%:%962=243%:%
%:%963=244%:%
%:%964=244%:%
%:%965=245%:%
%:%966=245%:%
%:%967=246%:%
%:%968=246%:%
%:%969=247%:%
%:%970=247%:%
%:%971=248%:%
%:%972=248%:%
%:%973=249%:%
%:%974=249%:%
%:%975=250%:%
%:%976=250%:%
%:%981=250%:%
%:%984=251%:%
%:%985=252%:%
%:%986=252%:%
%:%987=253%:%
%:%990=254%:%
%:%994=254%:%
%:%995=254%:%
%:%996=255%:%
%:%997=255%:%
%:%1002=255%:%
%:%1005=256%:%
%:%1006=257%:%
%:%1007=258%:%
%:%1008=259%:%
%:%1009=259%:%
%:%1016=260%:%

%
\begin{isabellebody}%
\setisabellecontext{RecFun{\isacharunderscore}{\kern0pt}M}%
%
\isadelimtheory
%
\endisadelimtheory
%
\isatagtheory
\isacommand{theory}\isamarkupfalse%
\ RecFun{\isacharunderscore}{\kern0pt}M\ \isanewline
\ \ \isakeyword{imports}\ \isanewline
\ \ \ \ ZF\ \isanewline
\ \ \ \ Utilities{\isacharunderscore}{\kern0pt}M\isanewline
\isakeyword{begin}%
\endisatagtheory
{\isafoldtheory}%
%
\isadelimtheory
\ \isanewline
%
\endisadelimtheory
\isanewline
\isacommand{definition}\isamarkupfalse%
\ rep{\isacharunderscore}{\kern0pt}for{\isacharunderscore}{\kern0pt}recfun{\isacharunderscore}{\kern0pt}fm\ \isakeyword{where}\ \isanewline
\ \ {\isachardoublequoteopen}rep{\isacharunderscore}{\kern0pt}for{\isacharunderscore}{\kern0pt}recfun{\isacharunderscore}{\kern0pt}fm{\isacharparenleft}{\kern0pt}p{\isacharcomma}{\kern0pt}\ x{\isacharcomma}{\kern0pt}\ z{\isacharcomma}{\kern0pt}\ r{\isacharparenright}{\kern0pt}\ {\isasymequiv}\ \isanewline
\ \ \ \ Exists{\isacharparenleft}{\kern0pt}Exists{\isacharparenleft}{\kern0pt}Exists{\isacharparenleft}{\kern0pt}And{\isacharparenleft}{\kern0pt}Equal{\isacharparenleft}{\kern0pt}x\ {\isacharhash}{\kern0pt}{\isacharplus}{\kern0pt}\ {\isadigit{3}}{\isacharcomma}{\kern0pt}\ {\isadigit{2}}{\isacharparenright}{\kern0pt}{\isacharcomma}{\kern0pt}\ And{\isacharparenleft}{\kern0pt}pair{\isacharunderscore}{\kern0pt}fm{\isacharparenleft}{\kern0pt}x{\isacharhash}{\kern0pt}{\isacharplus}{\kern0pt}{\isadigit{3}}{\isacharcomma}{\kern0pt}\ {\isadigit{0}}{\isacharcomma}{\kern0pt}\ z{\isacharhash}{\kern0pt}{\isacharplus}{\kern0pt}{\isadigit{3}}{\isacharparenright}{\kern0pt}{\isacharcomma}{\kern0pt}\ And{\isacharparenleft}{\kern0pt}is{\isacharunderscore}{\kern0pt}recfun{\isacharunderscore}{\kern0pt}fm{\isacharparenleft}{\kern0pt}p{\isacharcomma}{\kern0pt}\ r{\isacharhash}{\kern0pt}{\isacharplus}{\kern0pt}{\isadigit{3}}{\isacharcomma}{\kern0pt}\ x{\isacharhash}{\kern0pt}{\isacharplus}{\kern0pt}{\isadigit{3}}{\isacharcomma}{\kern0pt}\ {\isadigit{1}}{\isacharparenright}{\kern0pt}{\isacharcomma}{\kern0pt}\ p{\isacharparenright}{\kern0pt}{\isacharparenright}{\kern0pt}{\isacharparenright}{\kern0pt}{\isacharparenright}{\kern0pt}{\isacharparenright}{\kern0pt}{\isacharparenright}{\kern0pt}{\isachardoublequoteclose}\ \isanewline
\isanewline
\isacommand{context}\isamarkupfalse%
\ M{\isacharunderscore}{\kern0pt}ctm\ \isanewline
\isakeyword{begin}\ \isanewline
\isanewline
\isacommand{lemma}\isamarkupfalse%
\ rep{\isacharunderscore}{\kern0pt}for{\isacharunderscore}{\kern0pt}recfun{\isacharunderscore}{\kern0pt}fm{\isacharunderscore}{\kern0pt}sats{\isacharunderscore}{\kern0pt}iff\ {\isacharcolon}{\kern0pt}\isanewline
\ \ {\isachardoublequoteopen}{\isasymAnd}x\ z\ r\ p\ H\ e\ i\ j\ k{\isachardot}{\kern0pt}\ x\ {\isasymin}\ M\ {\isasymLongrightarrow}\ z\ {\isasymin}\ M\ {\isasymLongrightarrow}\ r\ {\isasymin}\ M\ {\isasymLongrightarrow}\ p\ {\isasymin}\ formula\ {\isasymLongrightarrow}\isanewline
\ \ \ \ i\ {\isasymin}\ nat\ {\isasymLongrightarrow}\ j\ {\isasymin}\ nat\ {\isasymLongrightarrow}\ k\ {\isasymin}\ nat\ {\isasymLongrightarrow}\ e\ {\isasymin}\ list{\isacharparenleft}{\kern0pt}M{\isacharparenright}{\kern0pt}\ {\isasymLongrightarrow}\ e\ {\isasymnoteq}\ {\isacharbrackleft}{\kern0pt}{\isacharbrackright}{\kern0pt}\ {\isasymLongrightarrow}\ \isanewline
\ \ \ \ nth{\isacharparenleft}{\kern0pt}i{\isacharcomma}{\kern0pt}\ e{\isacharparenright}{\kern0pt}\ {\isacharequal}{\kern0pt}\ x\ {\isasymLongrightarrow}\ nth{\isacharparenleft}{\kern0pt}j{\isacharcomma}{\kern0pt}\ e{\isacharparenright}{\kern0pt}\ {\isacharequal}{\kern0pt}\ z\ {\isasymLongrightarrow}\ nth{\isacharparenleft}{\kern0pt}k{\isacharcomma}{\kern0pt}\ e{\isacharparenright}{\kern0pt}\ {\isacharequal}{\kern0pt}\ r\ {\isasymLongrightarrow}\ \isanewline
\ \ {\isacharparenleft}{\kern0pt}{\isasymAnd}a{\isadigit{0}}\ a{\isadigit{1}}\ a{\isadigit{2}}\ a{\isadigit{3}}\ env{\isachardot}{\kern0pt}\ a{\isadigit{0}}\ {\isasymin}\ M\ {\isasymLongrightarrow}\ a{\isadigit{1}}\ {\isasymin}\ M\ {\isasymLongrightarrow}\ a{\isadigit{2}}\ {\isasymin}\ M\ {\isasymLongrightarrow}\ a{\isadigit{3}}\ {\isasymin}\ M\ {\isasymLongrightarrow}\ env\ {\isasymin}\ list{\isacharparenleft}{\kern0pt}M{\isacharparenright}{\kern0pt}\ \isanewline
\ \ \ \ {\isasymLongrightarrow}\ a{\isadigit{0}}\ {\isacharequal}{\kern0pt}\ H{\isacharparenleft}{\kern0pt}a{\isadigit{2}}{\isacharcomma}{\kern0pt}\ a{\isadigit{1}}{\isacharparenright}{\kern0pt}\ {\isasymlongleftrightarrow}\ sats{\isacharparenleft}{\kern0pt}M{\isacharcomma}{\kern0pt}\ p{\isacharcomma}{\kern0pt}\ {\isacharbrackleft}{\kern0pt}a{\isadigit{0}}{\isacharcomma}{\kern0pt}\ a{\isadigit{1}}{\isacharcomma}{\kern0pt}\ a{\isadigit{2}}{\isacharcomma}{\kern0pt}\ a{\isadigit{3}}{\isacharbrackright}{\kern0pt}\ {\isacharat}{\kern0pt}\ env{\isacharparenright}{\kern0pt}{\isacharparenright}{\kern0pt}\ {\isasymLongrightarrow}\isanewline
\ \ {\isacharparenleft}{\kern0pt}{\isasymforall}x{\isacharbrackleft}{\kern0pt}{\isacharhash}{\kern0pt}{\isacharhash}{\kern0pt}M{\isacharbrackright}{\kern0pt}{\isachardot}{\kern0pt}\ {\isasymforall}g{\isacharbrackleft}{\kern0pt}{\isacharhash}{\kern0pt}{\isacharhash}{\kern0pt}M{\isacharbrackright}{\kern0pt}{\isachardot}{\kern0pt}\ function{\isacharparenleft}{\kern0pt}g{\isacharparenright}{\kern0pt}\ {\isasymlongrightarrow}\ {\isacharparenleft}{\kern0pt}{\isacharhash}{\kern0pt}{\isacharhash}{\kern0pt}M{\isacharparenright}{\kern0pt}{\isacharparenleft}{\kern0pt}H{\isacharparenleft}{\kern0pt}x{\isacharcomma}{\kern0pt}\ g{\isacharparenright}{\kern0pt}{\isacharparenright}{\kern0pt}{\isacharparenright}{\kern0pt}\ {\isasymLongrightarrow}\isanewline
\isanewline
\ \ {\isacharparenleft}{\kern0pt}{\isasymexists}y{\isacharbrackleft}{\kern0pt}{\isacharhash}{\kern0pt}{\isacharhash}{\kern0pt}M{\isacharbrackright}{\kern0pt}{\isachardot}{\kern0pt}\ {\isasymexists}g{\isacharbrackleft}{\kern0pt}{\isacharhash}{\kern0pt}{\isacharhash}{\kern0pt}M{\isacharbrackright}{\kern0pt}{\isachardot}{\kern0pt}\ pair{\isacharparenleft}{\kern0pt}{\isacharhash}{\kern0pt}{\isacharhash}{\kern0pt}M{\isacharcomma}{\kern0pt}\ x{\isacharcomma}{\kern0pt}\ y{\isacharcomma}{\kern0pt}\ z{\isacharparenright}{\kern0pt}\ {\isasymand}\ is{\isacharunderscore}{\kern0pt}recfun{\isacharparenleft}{\kern0pt}r{\isacharcomma}{\kern0pt}\ x{\isacharcomma}{\kern0pt}\ H{\isacharcomma}{\kern0pt}\ g{\isacharparenright}{\kern0pt}\ {\isasymand}\ y\ {\isacharequal}{\kern0pt}\ H{\isacharparenleft}{\kern0pt}x{\isacharcomma}{\kern0pt}\ g{\isacharparenright}{\kern0pt}{\isacharparenright}{\kern0pt}\isanewline
\ \ \ {\isasymlongleftrightarrow}\ sats{\isacharparenleft}{\kern0pt}M{\isacharcomma}{\kern0pt}\ rep{\isacharunderscore}{\kern0pt}for{\isacharunderscore}{\kern0pt}recfun{\isacharunderscore}{\kern0pt}fm{\isacharparenleft}{\kern0pt}p{\isacharcomma}{\kern0pt}\ i{\isacharcomma}{\kern0pt}\ j{\isacharcomma}{\kern0pt}\ k{\isacharparenright}{\kern0pt}{\isacharcomma}{\kern0pt}\ e{\isacharparenright}{\kern0pt}{\isachardoublequoteclose}\ \isanewline
%
\isadelimproof
%
\endisadelimproof
%
\isatagproof
\isacommand{proof}\isamarkupfalse%
\ {\isacharminus}{\kern0pt}\ \isanewline
\ \ \isacommand{fix}\isamarkupfalse%
\ x\ z\ r\ p\ H\ e\ i\ j\ k\isanewline
\ \ \isacommand{assume}\isamarkupfalse%
\ inM\ {\isacharcolon}{\kern0pt}\ {\isachardoublequoteopen}x\ {\isasymin}\ M{\isachardoublequoteclose}\ {\isachardoublequoteopen}z\ {\isasymin}\ M{\isachardoublequoteclose}\ {\isachardoublequoteopen}r\ {\isasymin}\ M{\isachardoublequoteclose}\ \isanewline
\ \ \isakeyword{and}\ pformula\ {\isacharcolon}{\kern0pt}\ {\isachardoublequoteopen}p\ {\isasymin}\ formula{\isachardoublequoteclose}\isanewline
\ \ \isakeyword{and}\ ph\ {\isacharcolon}{\kern0pt}\ \isanewline
\ \ \ \ {\isachardoublequoteopen}{\isacharparenleft}{\kern0pt}{\isasymAnd}a{\isadigit{0}}\ a{\isadigit{1}}\ a{\isadigit{2}}\ a{\isadigit{3}}\ env{\isachardot}{\kern0pt}\ a{\isadigit{0}}\ {\isasymin}\ M\ {\isasymLongrightarrow}\ a{\isadigit{1}}\ {\isasymin}\ M\ {\isasymLongrightarrow}\ a{\isadigit{2}}\ {\isasymin}\ M\ {\isasymLongrightarrow}\ a{\isadigit{3}}\ {\isasymin}\ M\ {\isasymLongrightarrow}\ env\ {\isasymin}\ list{\isacharparenleft}{\kern0pt}M{\isacharparenright}{\kern0pt}\ \isanewline
\ \ \ \ \ \ {\isasymLongrightarrow}\ a{\isadigit{0}}\ {\isacharequal}{\kern0pt}\ H{\isacharparenleft}{\kern0pt}a{\isadigit{2}}{\isacharcomma}{\kern0pt}\ a{\isadigit{1}}{\isacharparenright}{\kern0pt}\ {\isasymlongleftrightarrow}\ sats{\isacharparenleft}{\kern0pt}M{\isacharcomma}{\kern0pt}\ p{\isacharcomma}{\kern0pt}\ {\isacharbrackleft}{\kern0pt}a{\isadigit{0}}{\isacharcomma}{\kern0pt}\ a{\isadigit{1}}{\isacharcomma}{\kern0pt}\ a{\isadigit{2}}{\isacharcomma}{\kern0pt}\ a{\isadigit{3}}{\isacharbrackright}{\kern0pt}\ {\isacharat}{\kern0pt}\ env{\isacharparenright}{\kern0pt}{\isacharparenright}{\kern0pt}{\isachardoublequoteclose}\isanewline
\ \ \isakeyword{and}\ Hh\ {\isacharcolon}{\kern0pt}\ {\isachardoublequoteopen}{\isacharparenleft}{\kern0pt}{\isasymforall}x{\isacharbrackleft}{\kern0pt}{\isacharhash}{\kern0pt}{\isacharhash}{\kern0pt}M{\isacharbrackright}{\kern0pt}{\isachardot}{\kern0pt}\ {\isasymforall}g{\isacharbrackleft}{\kern0pt}{\isacharhash}{\kern0pt}{\isacharhash}{\kern0pt}M{\isacharbrackright}{\kern0pt}{\isachardot}{\kern0pt}\ function{\isacharparenleft}{\kern0pt}g{\isacharparenright}{\kern0pt}\ {\isasymlongrightarrow}\ {\isacharparenleft}{\kern0pt}{\isacharhash}{\kern0pt}{\isacharhash}{\kern0pt}M{\isacharparenright}{\kern0pt}{\isacharparenleft}{\kern0pt}H{\isacharparenleft}{\kern0pt}x{\isacharcomma}{\kern0pt}\ g{\isacharparenright}{\kern0pt}{\isacharparenright}{\kern0pt}{\isacharparenright}{\kern0pt}{\isachardoublequoteclose}\ \isanewline
\ \ \isakeyword{and}\ envH\ {\isacharcolon}{\kern0pt}\ {\isachardoublequoteopen}i\ {\isasymin}\ nat{\isachardoublequoteclose}\ {\isachardoublequoteopen}j\ {\isasymin}\ nat{\isachardoublequoteclose}\ {\isachardoublequoteopen}k\ {\isasymin}\ nat{\isachardoublequoteclose}\ {\isachardoublequoteopen}e\ {\isasymin}\ list{\isacharparenleft}{\kern0pt}M{\isacharparenright}{\kern0pt}{\isachardoublequoteclose}\ {\isachardoublequoteopen}nth{\isacharparenleft}{\kern0pt}i{\isacharcomma}{\kern0pt}\ e{\isacharparenright}{\kern0pt}\ {\isacharequal}{\kern0pt}\ x{\isachardoublequoteclose}\ {\isachardoublequoteopen}nth{\isacharparenleft}{\kern0pt}j{\isacharcomma}{\kern0pt}\ e{\isacharparenright}{\kern0pt}\ {\isacharequal}{\kern0pt}\ z{\isachardoublequoteclose}\ {\isachardoublequoteopen}nth{\isacharparenleft}{\kern0pt}k{\isacharcomma}{\kern0pt}\ e{\isacharparenright}{\kern0pt}\ {\isacharequal}{\kern0pt}\ r{\isachardoublequoteclose}\ {\isachardoublequoteopen}e\ {\isasymnoteq}\ {\isacharbrackleft}{\kern0pt}{\isacharbrackright}{\kern0pt}{\isachardoublequoteclose}\isanewline
\isanewline
\ \ \isacommand{have}\isamarkupfalse%
\ iff{\isacharunderscore}{\kern0pt}conj{\isacharunderscore}{\kern0pt}lemma\ {\isacharcolon}{\kern0pt}\ {\isachardoublequoteopen}{\isasymAnd}P\ Q\ R\ S{\isachardot}{\kern0pt}\ P\ {\isasymlongleftrightarrow}\ Q\ {\isasymLongrightarrow}\ {\isacharparenleft}{\kern0pt}P\ {\isasymLongrightarrow}\ {\isacharparenleft}{\kern0pt}R\ {\isasymlongleftrightarrow}\ S{\isacharparenright}{\kern0pt}{\isacharparenright}{\kern0pt}\ {\isasymLongrightarrow}\ {\isacharparenleft}{\kern0pt}P\ {\isasymand}\ R{\isacharparenright}{\kern0pt}\ {\isasymlongleftrightarrow}\ {\isacharparenleft}{\kern0pt}Q\ {\isasymand}\ S{\isacharparenright}{\kern0pt}{\isachardoublequoteclose}\ \isacommand{by}\isamarkupfalse%
\ auto\isanewline
\isanewline
\ \ \isacommand{have}\isamarkupfalse%
\ t{\isadigit{1}}{\isacharcolon}{\kern0pt}\isanewline
\ \ \ \ {\isachardoublequoteopen}{\isacharparenleft}{\kern0pt}{\isasymexists}y{\isacharbrackleft}{\kern0pt}{\isacharhash}{\kern0pt}{\isacharhash}{\kern0pt}M{\isacharbrackright}{\kern0pt}{\isachardot}{\kern0pt}\ {\isasymexists}g{\isacharbrackleft}{\kern0pt}{\isacharhash}{\kern0pt}{\isacharhash}{\kern0pt}M{\isacharbrackright}{\kern0pt}{\isachardot}{\kern0pt}\ pair{\isacharparenleft}{\kern0pt}{\isacharhash}{\kern0pt}{\isacharhash}{\kern0pt}M{\isacharcomma}{\kern0pt}\ x{\isacharcomma}{\kern0pt}\ y{\isacharcomma}{\kern0pt}\ z{\isacharparenright}{\kern0pt}\ {\isasymand}\ is{\isacharunderscore}{\kern0pt}recfun{\isacharparenleft}{\kern0pt}r{\isacharcomma}{\kern0pt}\ x{\isacharcomma}{\kern0pt}\ H{\isacharcomma}{\kern0pt}\ g{\isacharparenright}{\kern0pt}\ {\isasymand}\ y\ {\isacharequal}{\kern0pt}\ H{\isacharparenleft}{\kern0pt}x{\isacharcomma}{\kern0pt}\ g{\isacharparenright}{\kern0pt}{\isacharparenright}{\kern0pt}\isanewline
\ \ \ \ \ \ {\isasymlongleftrightarrow}\ {\isacharparenleft}{\kern0pt}{\isasymexists}y\ {\isasymin}\ M{\isachardot}{\kern0pt}\ {\isasymexists}g\ {\isasymin}\ M{\isachardot}{\kern0pt}\ pair{\isacharparenleft}{\kern0pt}{\isacharhash}{\kern0pt}{\isacharhash}{\kern0pt}M{\isacharcomma}{\kern0pt}\ x{\isacharcomma}{\kern0pt}\ y{\isacharcomma}{\kern0pt}\ z{\isacharparenright}{\kern0pt}\ {\isasymand}\ is{\isacharunderscore}{\kern0pt}recfun{\isacharparenleft}{\kern0pt}r{\isacharcomma}{\kern0pt}\ x{\isacharcomma}{\kern0pt}\ H{\isacharcomma}{\kern0pt}\ g{\isacharparenright}{\kern0pt}\ {\isasymand}\ y\ {\isacharequal}{\kern0pt}\ H{\isacharparenleft}{\kern0pt}x{\isacharcomma}{\kern0pt}\ g{\isacharparenright}{\kern0pt}{\isacharparenright}{\kern0pt}{\isachardoublequoteclose}\ \isacommand{by}\isamarkupfalse%
\ simp\ \isanewline
\ \ \isacommand{have}\isamarkupfalse%
\ t{\isadigit{2}}\ {\isacharcolon}{\kern0pt}\ \isanewline
\ \ \ \ {\isachardoublequoteopen}{\isachardot}{\kern0pt}{\isachardot}{\kern0pt}{\isachardot}{\kern0pt}\ {\isasymlongleftrightarrow}\ {\isacharparenleft}{\kern0pt}{\isasymexists}g\ {\isasymin}\ M{\isachardot}{\kern0pt}\ {\isasymexists}y\ {\isasymin}\ M{\isachardot}{\kern0pt}\ pair{\isacharparenleft}{\kern0pt}{\isacharhash}{\kern0pt}{\isacharhash}{\kern0pt}M{\isacharcomma}{\kern0pt}\ x{\isacharcomma}{\kern0pt}\ y{\isacharcomma}{\kern0pt}\ z{\isacharparenright}{\kern0pt}\ {\isasymand}\ is{\isacharunderscore}{\kern0pt}recfun{\isacharparenleft}{\kern0pt}r{\isacharcomma}{\kern0pt}\ x{\isacharcomma}{\kern0pt}\ H{\isacharcomma}{\kern0pt}\ g{\isacharparenright}{\kern0pt}\ {\isasymand}\ y\ {\isacharequal}{\kern0pt}\ H{\isacharparenleft}{\kern0pt}x{\isacharcomma}{\kern0pt}\ g{\isacharparenright}{\kern0pt}{\isacharparenright}{\kern0pt}{\isachardoublequoteclose}\ \isacommand{by}\isamarkupfalse%
\ auto\ \isanewline
\ \ \isacommand{have}\isamarkupfalse%
\ t{\isadigit{3}}\ {\isacharcolon}{\kern0pt}\ \ \ \isanewline
\ \ \ \ {\isachardoublequoteopen}{\isachardot}{\kern0pt}{\isachardot}{\kern0pt}{\isachardot}{\kern0pt}\ {\isasymlongleftrightarrow}{\isacharparenleft}{\kern0pt}{\isasymexists}g\ {\isasymin}\ M{\isachardot}{\kern0pt}\ {\isasymexists}y\ {\isasymin}\ M{\isachardot}{\kern0pt}\ pair{\isacharparenleft}{\kern0pt}{\isacharhash}{\kern0pt}{\isacharhash}{\kern0pt}M{\isacharcomma}{\kern0pt}\ x{\isacharcomma}{\kern0pt}\ y{\isacharcomma}{\kern0pt}\ z{\isacharparenright}{\kern0pt}\ {\isasymand}\ M{\isacharunderscore}{\kern0pt}is{\isacharunderscore}{\kern0pt}recfun{\isacharparenleft}{\kern0pt}{\isacharhash}{\kern0pt}{\isacharhash}{\kern0pt}M{\isacharcomma}{\kern0pt}\ {\isasymlambda}a\ b\ c{\isachardot}{\kern0pt}\ c\ {\isacharequal}{\kern0pt}\ H{\isacharparenleft}{\kern0pt}a{\isacharcomma}{\kern0pt}\ b{\isacharparenright}{\kern0pt}{\isacharcomma}{\kern0pt}\ r{\isacharcomma}{\kern0pt}\ x{\isacharcomma}{\kern0pt}\ g{\isacharparenright}{\kern0pt}\ {\isasymand}\ y\ {\isacharequal}{\kern0pt}\ H{\isacharparenleft}{\kern0pt}x{\isacharcomma}{\kern0pt}\ g{\isacharparenright}{\kern0pt}{\isacharparenright}{\kern0pt}{\isachardoublequoteclose}\isanewline
\ \ \ \ \isacommand{apply}\isamarkupfalse%
\ {\isacharparenleft}{\kern0pt}auto{\isacharparenright}{\kern0pt}\isanewline
\ \ \ \ \isacommand{apply}\isamarkupfalse%
\ {\isacharparenleft}{\kern0pt}rename{\isacharunderscore}{\kern0pt}tac\ g{\isacharcomma}{\kern0pt}\ rule{\isacharunderscore}{\kern0pt}tac\ x{\isacharequal}{\kern0pt}{\isachardoublequoteopen}g{\isachardoublequoteclose}\ \isakeyword{in}\ bexI{\isacharsemicolon}{\kern0pt}\ auto{\isacharparenright}{\kern0pt}\isanewline
\ \ \ \ \isacommand{apply}\isamarkupfalse%
\ {\isacharparenleft}{\kern0pt}rename{\isacharunderscore}{\kern0pt}tac\ g{\isacharcomma}{\kern0pt}\ rule{\isacharunderscore}{\kern0pt}tac\ Q{\isacharequal}{\kern0pt}{\isachardoublequoteopen}is{\isacharunderscore}{\kern0pt}recfun{\isacharparenleft}{\kern0pt}r{\isacharcomma}{\kern0pt}\ x{\isacharcomma}{\kern0pt}\ H{\isacharcomma}{\kern0pt}\ g{\isacharparenright}{\kern0pt}{\isachardoublequoteclose}\ \isakeyword{in}\ iffD{\isadigit{2}}{\isacharparenright}{\kern0pt}\isanewline
\ \ \ \ \isacommand{apply}\isamarkupfalse%
\ {\isacharparenleft}{\kern0pt}rule{\isacharunderscore}{\kern0pt}tac\ is{\isacharunderscore}{\kern0pt}recfun{\isacharunderscore}{\kern0pt}abs{\isacharparenright}{\kern0pt}\isanewline
\ \ \ \ \isacommand{using}\isamarkupfalse%
\ Hh\ inM\ \isanewline
\ \ \ \ \ \ \ \ \ \ \isacommand{apply}\isamarkupfalse%
\ simp{\isacharunderscore}{\kern0pt}all\isanewline
\ \ \ \ \isacommand{unfolding}\isamarkupfalse%
\ relation{\isadigit{2}}{\isacharunderscore}{\kern0pt}def\ \isanewline
\ \ \ \ \ \isacommand{apply}\isamarkupfalse%
\ simp\isanewline
\ \ \ \ \isacommand{apply}\isamarkupfalse%
\ {\isacharparenleft}{\kern0pt}rename{\isacharunderscore}{\kern0pt}tac\ g{\isacharcomma}{\kern0pt}\ rule{\isacharunderscore}{\kern0pt}tac\ x{\isacharequal}{\kern0pt}{\isachardoublequoteopen}g{\isachardoublequoteclose}\ \isakeyword{in}\ bexI{\isacharsemicolon}{\kern0pt}\ auto{\isacharparenright}{\kern0pt}\isanewline
\ \ \ \ \isacommand{apply}\isamarkupfalse%
\ {\isacharparenleft}{\kern0pt}rename{\isacharunderscore}{\kern0pt}tac\ g{\isacharcomma}{\kern0pt}\ rule{\isacharunderscore}{\kern0pt}tac\ P{\isacharequal}{\kern0pt}{\isachardoublequoteopen}M{\isacharunderscore}{\kern0pt}is{\isacharunderscore}{\kern0pt}recfun{\isacharparenleft}{\kern0pt}{\isacharhash}{\kern0pt}{\isacharhash}{\kern0pt}M{\isacharcomma}{\kern0pt}\ {\isasymlambda}a\ b\ c{\isachardot}{\kern0pt}\ c\ {\isacharequal}{\kern0pt}\ H{\isacharparenleft}{\kern0pt}a{\isacharcomma}{\kern0pt}\ b{\isacharparenright}{\kern0pt}{\isacharcomma}{\kern0pt}\ r{\isacharcomma}{\kern0pt}\ x{\isacharcomma}{\kern0pt}\ g{\isacharparenright}{\kern0pt}{\isachardoublequoteclose}\ \isakeyword{in}\ iffD{\isadigit{1}}{\isacharparenright}{\kern0pt}\isanewline
\ \ \ \ \ \isacommand{apply}\isamarkupfalse%
\ {\isacharparenleft}{\kern0pt}rule{\isacharunderscore}{\kern0pt}tac\ is{\isacharunderscore}{\kern0pt}recfun{\isacharunderscore}{\kern0pt}abs{\isacharparenright}{\kern0pt}\isanewline
\ \ \ \ \isacommand{using}\isamarkupfalse%
\ Hh\ inM\ \isanewline
\ \ \ \ \ \ \ \ \ \isacommand{apply}\isamarkupfalse%
\ simp{\isacharunderscore}{\kern0pt}all\isanewline
\ \ \ \ \isacommand{unfolding}\isamarkupfalse%
\ relation{\isadigit{2}}{\isacharunderscore}{\kern0pt}def\ \isanewline
\ \ \ \ \isacommand{apply}\isamarkupfalse%
\ simp\ \isanewline
\ \ \ \ \isacommand{done}\isamarkupfalse%
\ \isanewline
\ \ \isacommand{have}\isamarkupfalse%
\ t{\isadigit{4}}\ {\isacharcolon}{\kern0pt}\ \ \ \isanewline
\ \ \ \ {\isachardoublequoteopen}{\isachardot}{\kern0pt}{\isachardot}{\kern0pt}{\isachardot}{\kern0pt}\ {\isasymlongleftrightarrow}{\isacharparenleft}{\kern0pt}{\isasymexists}x{\isacharprime}{\kern0pt}\ {\isasymin}\ M{\isachardot}{\kern0pt}\ {\isasymexists}g\ {\isasymin}\ M{\isachardot}{\kern0pt}\ {\isasymexists}y\ {\isasymin}\ M{\isachardot}{\kern0pt}\ x\ {\isacharequal}{\kern0pt}\ x{\isacharprime}{\kern0pt}\ {\isasymand}\ pair{\isacharparenleft}{\kern0pt}{\isacharhash}{\kern0pt}{\isacharhash}{\kern0pt}M{\isacharcomma}{\kern0pt}\ x{\isacharcomma}{\kern0pt}\ y{\isacharcomma}{\kern0pt}\ z{\isacharparenright}{\kern0pt}\ {\isasymand}\ M{\isacharunderscore}{\kern0pt}is{\isacharunderscore}{\kern0pt}recfun{\isacharparenleft}{\kern0pt}{\isacharhash}{\kern0pt}{\isacharhash}{\kern0pt}M{\isacharcomma}{\kern0pt}\ {\isasymlambda}a\ b\ c{\isachardot}{\kern0pt}\ c\ {\isacharequal}{\kern0pt}\ H{\isacharparenleft}{\kern0pt}a{\isacharcomma}{\kern0pt}\ b{\isacharparenright}{\kern0pt}{\isacharcomma}{\kern0pt}\ r{\isacharcomma}{\kern0pt}\ x{\isacharcomma}{\kern0pt}\ g{\isacharparenright}{\kern0pt}\ {\isasymand}\ y\ {\isacharequal}{\kern0pt}\ H{\isacharparenleft}{\kern0pt}x{\isacharcomma}{\kern0pt}\ g{\isacharparenright}{\kern0pt}{\isacharparenright}{\kern0pt}{\isachardoublequoteclose}\isanewline
\ \ \ \ \isacommand{using}\isamarkupfalse%
\ inM\ \isanewline
\ \ \ \ \isacommand{by}\isamarkupfalse%
\ auto\isanewline
\ \ \isacommand{also}\isamarkupfalse%
\ \isacommand{have}\isamarkupfalse%
\ t{\isadigit{5}}{\isacharcolon}{\kern0pt}\ \isanewline
\ \ \ \ {\isachardoublequoteopen}{\isachardot}{\kern0pt}{\isachardot}{\kern0pt}{\isachardot}{\kern0pt}\ {\isasymlongleftrightarrow}\ sats{\isacharparenleft}{\kern0pt}M{\isacharcomma}{\kern0pt}\ rep{\isacharunderscore}{\kern0pt}for{\isacharunderscore}{\kern0pt}recfun{\isacharunderscore}{\kern0pt}fm{\isacharparenleft}{\kern0pt}p{\isacharcomma}{\kern0pt}\ i{\isacharcomma}{\kern0pt}\ j{\isacharcomma}{\kern0pt}\ k{\isacharparenright}{\kern0pt}{\isacharcomma}{\kern0pt}\ e{\isacharparenright}{\kern0pt}{\isachardoublequoteclose}\ \isanewline
\ \ \ \ \isacommand{unfolding}\isamarkupfalse%
\ rep{\isacharunderscore}{\kern0pt}for{\isacharunderscore}{\kern0pt}recfun{\isacharunderscore}{\kern0pt}fm{\isacharunderscore}{\kern0pt}def\ \isanewline
\ \ \ \ \isacommand{apply}\isamarkupfalse%
\ {\isacharparenleft}{\kern0pt}rule\ bex{\isacharunderscore}{\kern0pt}iff{\isacharunderscore}{\kern0pt}sats{\isacharparenright}{\kern0pt}\isanewline
\ \ \ \ \ \isacommand{apply}\isamarkupfalse%
\ {\isacharparenleft}{\kern0pt}rule\ bex{\isacharunderscore}{\kern0pt}iff{\isacharunderscore}{\kern0pt}sats{\isacharparenright}{\kern0pt}\isanewline
\ \ \ \ \ \ \isacommand{apply}\isamarkupfalse%
\ {\isacharparenleft}{\kern0pt}rule\ bex{\isacharunderscore}{\kern0pt}iff{\isacharunderscore}{\kern0pt}sats{\isacharparenright}{\kern0pt}\isanewline
\ \ \ \ \ \ \ \isacommand{apply}\isamarkupfalse%
{\isacharparenleft}{\kern0pt}rule\ iff{\isacharunderscore}{\kern0pt}flip{\isacharparenright}{\kern0pt}\isanewline
\ \ \ \ \ \ \ \isacommand{apply}\isamarkupfalse%
{\isacharparenleft}{\kern0pt}rename{\isacharunderscore}{\kern0pt}tac\ x{\isacharprime}{\kern0pt}\ g\ y{\isacharcomma}{\kern0pt}\ rule{\isacharunderscore}{\kern0pt}tac\ Q{\isacharequal}{\kern0pt}{\isachardoublequoteopen}{\isacharparenleft}{\kern0pt}M{\isacharcomma}{\kern0pt}\ Cons{\isacharparenleft}{\kern0pt}y{\isacharcomma}{\kern0pt}\ Cons{\isacharparenleft}{\kern0pt}g{\isacharcomma}{\kern0pt}\ Cons{\isacharparenleft}{\kern0pt}x{\isacharprime}{\kern0pt}{\isacharcomma}{\kern0pt}\ e{\isacharparenright}{\kern0pt}{\isacharparenright}{\kern0pt}{\isacharparenright}{\kern0pt}\ {\isasymTurnstile}\ Equal{\isacharparenleft}{\kern0pt}i\ {\isacharhash}{\kern0pt}{\isacharplus}{\kern0pt}\ {\isadigit{3}}{\isacharcomma}{\kern0pt}\ {\isadigit{2}}{\isacharparenright}{\kern0pt}{\isacharparenright}{\kern0pt}\ {\isasymand}\ {\isacharparenleft}{\kern0pt}M{\isacharcomma}{\kern0pt}\ Cons{\isacharparenleft}{\kern0pt}y{\isacharcomma}{\kern0pt}\ Cons{\isacharparenleft}{\kern0pt}g{\isacharcomma}{\kern0pt}\ Cons{\isacharparenleft}{\kern0pt}x{\isacharprime}{\kern0pt}{\isacharcomma}{\kern0pt}\ e{\isacharparenright}{\kern0pt}{\isacharparenright}{\kern0pt}{\isacharparenright}{\kern0pt}\ {\isasymTurnstile}\ And{\isacharparenleft}{\kern0pt}pair{\isacharunderscore}{\kern0pt}fm{\isacharparenleft}{\kern0pt}i\ {\isacharhash}{\kern0pt}{\isacharplus}{\kern0pt}\ {\isadigit{3}}{\isacharcomma}{\kern0pt}\ {\isadigit{0}}{\isacharcomma}{\kern0pt}\ j\ {\isacharhash}{\kern0pt}{\isacharplus}{\kern0pt}\ {\isadigit{3}}{\isacharparenright}{\kern0pt}{\isacharcomma}{\kern0pt}\ And{\isacharparenleft}{\kern0pt}is{\isacharunderscore}{\kern0pt}recfun{\isacharunderscore}{\kern0pt}fm{\isacharparenleft}{\kern0pt}p{\isacharcomma}{\kern0pt}\ k\ {\isacharhash}{\kern0pt}{\isacharplus}{\kern0pt}\ {\isadigit{3}}{\isacharcomma}{\kern0pt}\ i\ {\isacharhash}{\kern0pt}{\isacharplus}{\kern0pt}\ {\isadigit{3}}{\isacharcomma}{\kern0pt}\ {\isadigit{1}}{\isacharparenright}{\kern0pt}{\isacharcomma}{\kern0pt}\ p{\isacharparenright}{\kern0pt}{\isacharparenright}{\kern0pt}{\isacharparenright}{\kern0pt}{\isachardoublequoteclose}\ \isakeyword{in}\ iff{\isacharunderscore}{\kern0pt}trans{\isacharparenright}{\kern0pt}\isanewline
\ \ \ \ \isacommand{using}\isamarkupfalse%
\ inM\ envH\ \isanewline
\ \ \ \ \ \ \ \ \isacommand{apply}\isamarkupfalse%
\ simp\ \isanewline
\ \ \ \ \ \ \ \isacommand{apply}\isamarkupfalse%
{\isacharparenleft}{\kern0pt}rule\ iff{\isacharunderscore}{\kern0pt}flip{\isacharcomma}{\kern0pt}\ rule\ iff{\isacharunderscore}{\kern0pt}conj{\isacharunderscore}{\kern0pt}lemma{\isacharparenright}{\kern0pt}\isanewline
\ \ \ \ \ \ \ \ \ \isacommand{apply}\isamarkupfalse%
{\isacharparenleft}{\kern0pt}rule\ equal{\isacharunderscore}{\kern0pt}iff{\isacharunderscore}{\kern0pt}sats{\isacharparenright}{\kern0pt}\isanewline
\ \ \ \ \isacommand{using}\isamarkupfalse%
\ envH\ \isanewline
\ \ \ \ \ \ \ \ \ \ \ \isacommand{apply}\isamarkupfalse%
\ {\isacharparenleft}{\kern0pt}simp{\isacharcomma}{\kern0pt}\ simp{\isacharcomma}{\kern0pt}\ simp{\isacharparenright}{\kern0pt}\isanewline
\ \ \ \ \ \ \ \isacommand{apply}\isamarkupfalse%
{\isacharparenleft}{\kern0pt}rule\ conj{\isacharunderscore}{\kern0pt}iff{\isacharunderscore}{\kern0pt}sats{\isacharparenright}{\kern0pt}\ \isanewline
\ \ \ \ \ \ \ \ \ \ \isacommand{apply}\isamarkupfalse%
\ {\isacharparenleft}{\kern0pt}rule\ pair{\isacharunderscore}{\kern0pt}iff{\isacharunderscore}{\kern0pt}sats{\isacharparenright}{\kern0pt}\isanewline
\ \ \ \ \isacommand{using}\isamarkupfalse%
\ inM\ envH\ \isanewline
\ \ \ \ \ \ \ \ \ \ \ \ \ \ \isacommand{apply}\isamarkupfalse%
\ {\isacharparenleft}{\kern0pt}simp{\isacharunderscore}{\kern0pt}all{\isacharparenright}{\kern0pt}\isanewline
\ \ \ \ \isacommand{apply}\isamarkupfalse%
\ {\isacharparenleft}{\kern0pt}rule\ conj{\isacharunderscore}{\kern0pt}cong{\isacharparenright}{\kern0pt}\isanewline
\ \ \ \ \ \isacommand{apply}\isamarkupfalse%
\ {\isacharparenleft}{\kern0pt}rule\ is{\isacharunderscore}{\kern0pt}recfun{\isacharunderscore}{\kern0pt}iff{\isacharunderscore}{\kern0pt}sats{\isacharparenright}{\kern0pt}\isanewline
\ \ \ \ \isacommand{using}\isamarkupfalse%
\ inM\ \isanewline
\ \ \ \ \ \ \ \ \ \ \ \ \isacommand{apply}\isamarkupfalse%
\ {\isacharparenleft}{\kern0pt}simp{\isacharunderscore}{\kern0pt}all{\isacharparenright}{\kern0pt}\isanewline
\ \ \ \ \ \isacommand{apply}\isamarkupfalse%
\ {\isacharparenleft}{\kern0pt}rename{\isacharunderscore}{\kern0pt}tac\ x{\isacharprime}{\kern0pt}\ g\ y\ a{\isadigit{0}}\ a{\isadigit{1}}\ a{\isadigit{2}}\ a{\isadigit{3}}{\isacharcomma}{\kern0pt}\ rule{\isacharunderscore}{\kern0pt}tac\ Q{\isacharequal}{\kern0pt}{\isachardoublequoteopen}sats{\isacharparenleft}{\kern0pt}M{\isacharcomma}{\kern0pt}\ p{\isacharcomma}{\kern0pt}\ {\isacharbrackleft}{\kern0pt}a{\isadigit{0}}{\isacharcomma}{\kern0pt}\ a{\isadigit{1}}{\isacharcomma}{\kern0pt}\ a{\isadigit{2}}{\isacharcomma}{\kern0pt}\ a{\isadigit{3}}{\isacharbrackright}{\kern0pt}\ {\isacharat}{\kern0pt}\ {\isacharparenleft}{\kern0pt}{\isacharbrackleft}{\kern0pt}y{\isacharcomma}{\kern0pt}\ g{\isacharcomma}{\kern0pt}\ x{\isacharprime}{\kern0pt}{\isacharbrackright}{\kern0pt}\ {\isacharat}{\kern0pt}\ e{\isacharparenright}{\kern0pt}{\isacharparenright}{\kern0pt}{\isachardoublequoteclose}\ \isakeyword{in}\ iff{\isacharunderscore}{\kern0pt}trans{\isacharparenright}{\kern0pt}\isanewline
\ \ \ \ \ \ \isacommand{apply}\isamarkupfalse%
\ {\isacharparenleft}{\kern0pt}rule{\isacharunderscore}{\kern0pt}tac\ ph{\isacharparenright}{\kern0pt}\isanewline
\ \ \ \ \isacommand{using}\isamarkupfalse%
\ inM\ \isanewline
\ \ \ \ \ \ \ \ \ \ \isacommand{apply}\isamarkupfalse%
\ {\isacharparenleft}{\kern0pt}simp{\isacharunderscore}{\kern0pt}all{\isacharparenright}{\kern0pt}\isanewline
\ \ \ \ \isacommand{using}\isamarkupfalse%
\ {\isacartoucheopen}e\ {\isasymin}\ list{\isacharparenleft}{\kern0pt}M{\isacharparenright}{\kern0pt}{\isacartoucheclose}\isanewline
\ \ \ \ \isacommand{apply}\isamarkupfalse%
{\isacharparenleft}{\kern0pt}cases\ e{\isacharcomma}{\kern0pt}\ force{\isacharparenright}{\kern0pt}\isanewline
\ \ \ \ \ \isacommand{apply}\isamarkupfalse%
\ {\isacharparenleft}{\kern0pt}rename{\isacharunderscore}{\kern0pt}tac\ x{\isacharprime}{\kern0pt}\ g\ y\ a\ l{\isacharcomma}{\kern0pt}\ rule{\isacharunderscore}{\kern0pt}tac\ Q{\isacharequal}{\kern0pt}{\isachardoublequoteopen}M{\isacharcomma}{\kern0pt}\ {\isacharbrackleft}{\kern0pt}y{\isacharcomma}{\kern0pt}\ g{\isacharcomma}{\kern0pt}\ x{\isacharprime}{\kern0pt}{\isacharcomma}{\kern0pt}\ a{\isacharbrackright}{\kern0pt}\ {\isacharat}{\kern0pt}\ l\ {\isasymTurnstile}\ p{\isachardoublequoteclose}\ \isakeyword{in}\ iff{\isacharunderscore}{\kern0pt}trans{\isacharparenright}{\kern0pt}\isanewline
\ \ \ \ \ \isacommand{apply}\isamarkupfalse%
\ {\isacharparenleft}{\kern0pt}rule{\isacharunderscore}{\kern0pt}tac\ ph{\isacharparenright}{\kern0pt}\ \isanewline
\ \ \ \ \isacommand{using}\isamarkupfalse%
\ inM\ \isanewline
\ \ \ \ \isacommand{by}\isamarkupfalse%
\ auto\ \ \ \ \isanewline
\ \ \isacommand{show}\isamarkupfalse%
\ {\isachardoublequoteopen}{\isacharparenleft}{\kern0pt}{\isasymexists}y{\isacharbrackleft}{\kern0pt}{\isacharhash}{\kern0pt}{\isacharhash}{\kern0pt}M{\isacharbrackright}{\kern0pt}{\isachardot}{\kern0pt}\ {\isasymexists}g{\isacharbrackleft}{\kern0pt}{\isacharhash}{\kern0pt}{\isacharhash}{\kern0pt}M{\isacharbrackright}{\kern0pt}{\isachardot}{\kern0pt}\ pair{\isacharparenleft}{\kern0pt}{\isacharhash}{\kern0pt}{\isacharhash}{\kern0pt}M{\isacharcomma}{\kern0pt}\ x{\isacharcomma}{\kern0pt}\ y{\isacharcomma}{\kern0pt}\ z{\isacharparenright}{\kern0pt}\ {\isasymand}\ is{\isacharunderscore}{\kern0pt}recfun{\isacharparenleft}{\kern0pt}r{\isacharcomma}{\kern0pt}\ x{\isacharcomma}{\kern0pt}\ H{\isacharcomma}{\kern0pt}\ g{\isacharparenright}{\kern0pt}\ {\isasymand}\ y\ {\isacharequal}{\kern0pt}\ H{\isacharparenleft}{\kern0pt}x{\isacharcomma}{\kern0pt}\ g{\isacharparenright}{\kern0pt}{\isacharparenright}{\kern0pt}\isanewline
\ \ \ {\isasymlongleftrightarrow}\ sats{\isacharparenleft}{\kern0pt}M{\isacharcomma}{\kern0pt}\ rep{\isacharunderscore}{\kern0pt}for{\isacharunderscore}{\kern0pt}recfun{\isacharunderscore}{\kern0pt}fm{\isacharparenleft}{\kern0pt}p{\isacharcomma}{\kern0pt}\ i{\isacharcomma}{\kern0pt}\ j{\isacharcomma}{\kern0pt}\ k{\isacharparenright}{\kern0pt}{\isacharcomma}{\kern0pt}\ e{\isacharparenright}{\kern0pt}{\isachardoublequoteclose}\ \isanewline
\ \ \ \ \isacommand{using}\isamarkupfalse%
\ t{\isadigit{1}}\ t{\isadigit{2}}\ t{\isadigit{3}}\ t{\isadigit{4}}\ t{\isadigit{5}}\ \isacommand{by}\isamarkupfalse%
\ auto\ \isanewline
\isacommand{qed}\isamarkupfalse%
%
\endisatagproof
{\isafoldproof}%
%
\isadelimproof
\isanewline
%
\endisadelimproof
\isanewline
\isacommand{lemma}\isamarkupfalse%
\ strep{\isacharunderscore}{\kern0pt}iff\ {\isacharcolon}{\kern0pt}\ \isanewline
\ \ {\isachardoublequoteopen}{\isasymAnd}P\ Q{\isachardot}{\kern0pt}\ strong{\isacharunderscore}{\kern0pt}replacement{\isacharparenleft}{\kern0pt}{\isacharhash}{\kern0pt}{\isacharhash}{\kern0pt}M{\isacharcomma}{\kern0pt}\ P{\isacharparenright}{\kern0pt}\ {\isasymLongrightarrow}\ {\isasymforall}x\ {\isasymin}\ M{\isachardot}{\kern0pt}\ {\isasymforall}z\ {\isasymin}\ M{\isachardot}{\kern0pt}\ P{\isacharparenleft}{\kern0pt}x{\isacharcomma}{\kern0pt}\ z{\isacharparenright}{\kern0pt}\ {\isasymlongleftrightarrow}\ Q{\isacharparenleft}{\kern0pt}x{\isacharcomma}{\kern0pt}\ z{\isacharparenright}{\kern0pt}\ {\isasymLongrightarrow}\ strong{\isacharunderscore}{\kern0pt}replacement{\isacharparenleft}{\kern0pt}{\isacharhash}{\kern0pt}{\isacharhash}{\kern0pt}M{\isacharcomma}{\kern0pt}\ Q{\isacharparenright}{\kern0pt}{\isachardoublequoteclose}\isanewline
%
\isadelimproof
\ \ %
\endisadelimproof
%
\isatagproof
\isacommand{unfolding}\isamarkupfalse%
\ strong{\isacharunderscore}{\kern0pt}replacement{\isacharunderscore}{\kern0pt}def\ univalent{\isacharunderscore}{\kern0pt}def\ \isacommand{by}\isamarkupfalse%
\ auto%
\endisatagproof
{\isafoldproof}%
%
\isadelimproof
\ \isanewline
%
\endisadelimproof
\isanewline
\isacommand{lemma}\isamarkupfalse%
\ recfun{\isacharunderscore}{\kern0pt}strong{\isacharunderscore}{\kern0pt}replacement{\isacharunderscore}{\kern0pt}lemma\ {\isacharcolon}{\kern0pt}\isanewline
\ \ \isakeyword{fixes}\ r\ H\ p\ \isanewline
\ \ \isakeyword{assumes}\ \isanewline
\ \ \ \ {\isachardoublequoteopen}wf{\isacharparenleft}{\kern0pt}r{\isacharparenright}{\kern0pt}{\isachardoublequoteclose}\ \isanewline
\ \ \ \ {\isachardoublequoteopen}trans{\isacharparenleft}{\kern0pt}r{\isacharparenright}{\kern0pt}{\isachardoublequoteclose}\ \isanewline
\ \ \ \ {\isachardoublequoteopen}r\ {\isasymin}\ M{\isachardoublequoteclose}\ \isanewline
\ \ \ \ {\isachardoublequoteopen}p\ {\isasymin}\ formula{\isachardoublequoteclose}\isanewline
\ \ \ \ {\isachardoublequoteopen}arity{\isacharparenleft}{\kern0pt}p{\isacharparenright}{\kern0pt}\ {\isasymle}\ {\isadigit{3}}{\isachardoublequoteclose}\isanewline
\ \ \ \ {\isachardoublequoteopen}{\isacharparenleft}{\kern0pt}{\isasymAnd}x\ g{\isachardot}{\kern0pt}\ x\ {\isasymin}\ M\ {\isasymLongrightarrow}\ g\ {\isasymin}\ M\ {\isasymLongrightarrow}\ function{\isacharparenleft}{\kern0pt}g{\isacharparenright}{\kern0pt}\ {\isasymLongrightarrow}\ H{\isacharparenleft}{\kern0pt}x{\isacharcomma}{\kern0pt}\ g{\isacharparenright}{\kern0pt}\ {\isasymin}\ M{\isacharparenright}{\kern0pt}{\isachardoublequoteclose}\ \isanewline
\ \ \ \ {\isachardoublequoteopen}\ {\isacharparenleft}{\kern0pt}{\isasymAnd}a{\isadigit{0}}\ a{\isadigit{1}}\ a{\isadigit{2}}\ env{\isachardot}{\kern0pt}\ a{\isadigit{0}}\ {\isasymin}\ M\ {\isasymLongrightarrow}\ a{\isadigit{1}}\ {\isasymin}\ M\ {\isasymLongrightarrow}\ a{\isadigit{2}}\ {\isasymin}\ M\ {\isasymLongrightarrow}\ env\ {\isasymin}\ list{\isacharparenleft}{\kern0pt}M{\isacharparenright}{\kern0pt}\ {\isasymLongrightarrow}\ a{\isadigit{0}}\ {\isacharequal}{\kern0pt}\ H{\isacharparenleft}{\kern0pt}a{\isadigit{2}}{\isacharcomma}{\kern0pt}\ a{\isadigit{1}}{\isacharparenright}{\kern0pt}\ {\isasymlongleftrightarrow}\ sats{\isacharparenleft}{\kern0pt}M{\isacharcomma}{\kern0pt}\ p{\isacharcomma}{\kern0pt}\ {\isacharbrackleft}{\kern0pt}a{\isadigit{0}}{\isacharcomma}{\kern0pt}\ a{\isadigit{1}}{\isacharcomma}{\kern0pt}\ a{\isadigit{2}}{\isacharbrackright}{\kern0pt}\ {\isacharat}{\kern0pt}\ env{\isacharparenright}{\kern0pt}{\isacharparenright}{\kern0pt}{\isachardoublequoteclose}\ \ \isanewline
\ \ \ \isanewline
\ \ \isakeyword{shows}\ {\isachardoublequoteopen}strong{\isacharunderscore}{\kern0pt}replacement{\isacharparenleft}{\kern0pt}{\isacharhash}{\kern0pt}{\isacharhash}{\kern0pt}M{\isacharcomma}{\kern0pt}\ {\isasymlambda}x\ z{\isachardot}{\kern0pt}\ {\isasymexists}y{\isacharbrackleft}{\kern0pt}{\isacharhash}{\kern0pt}{\isacharhash}{\kern0pt}M{\isacharbrackright}{\kern0pt}{\isachardot}{\kern0pt}\ {\isasymexists}g{\isacharbrackleft}{\kern0pt}{\isacharhash}{\kern0pt}{\isacharhash}{\kern0pt}M{\isacharbrackright}{\kern0pt}{\isachardot}{\kern0pt}\ pair{\isacharparenleft}{\kern0pt}{\isacharhash}{\kern0pt}{\isacharhash}{\kern0pt}M{\isacharcomma}{\kern0pt}\ x{\isacharcomma}{\kern0pt}\ y{\isacharcomma}{\kern0pt}\ z{\isacharparenright}{\kern0pt}\ {\isasymand}\ is{\isacharunderscore}{\kern0pt}recfun{\isacharparenleft}{\kern0pt}r{\isacharcomma}{\kern0pt}\ x{\isacharcomma}{\kern0pt}\ H{\isacharcomma}{\kern0pt}\ g{\isacharparenright}{\kern0pt}\ {\isasymand}\ y\ {\isacharequal}{\kern0pt}\ H{\isacharparenleft}{\kern0pt}x{\isacharcomma}{\kern0pt}\ g{\isacharparenright}{\kern0pt}{\isacharparenright}{\kern0pt}{\isachardoublequoteclose}\isanewline
%
\isadelimproof
%
\endisadelimproof
%
\isatagproof
\isacommand{proof}\isamarkupfalse%
\ {\isacharminus}{\kern0pt}\ \isanewline
\isanewline
\ \ \isacommand{have}\isamarkupfalse%
\ ph\ {\isacharcolon}{\kern0pt}\ \isanewline
\ \ \ \ \ \ {\isachardoublequoteopen}{\isacharparenleft}{\kern0pt}{\isasymAnd}a{\isadigit{0}}\ a{\isadigit{1}}\ a{\isadigit{2}}\ env{\isachardot}{\kern0pt}\ a{\isadigit{0}}\ {\isasymin}\ M\ {\isasymLongrightarrow}\ a{\isadigit{1}}\ {\isasymin}\ M\ {\isasymLongrightarrow}\ a{\isadigit{2}}\ {\isasymin}\ M\ {\isasymLongrightarrow}\ env\ {\isasymin}\ list{\isacharparenleft}{\kern0pt}M{\isacharparenright}{\kern0pt}\ \isanewline
\ \ \ \ \ \ \ \ {\isasymLongrightarrow}\ a{\isadigit{0}}\ {\isacharequal}{\kern0pt}\ H{\isacharparenleft}{\kern0pt}a{\isadigit{2}}{\isacharcomma}{\kern0pt}\ a{\isadigit{1}}{\isacharparenright}{\kern0pt}\ {\isasymlongleftrightarrow}\ sats{\isacharparenleft}{\kern0pt}M{\isacharcomma}{\kern0pt}\ p{\isacharcomma}{\kern0pt}\ {\isacharbrackleft}{\kern0pt}a{\isadigit{0}}{\isacharcomma}{\kern0pt}\ a{\isadigit{1}}{\isacharcomma}{\kern0pt}\ a{\isadigit{2}}{\isacharbrackright}{\kern0pt}\ {\isacharat}{\kern0pt}\ env{\isacharparenright}{\kern0pt}{\isacharparenright}{\kern0pt}{\isachardoublequoteclose}\isanewline
\ \ \ \ \isacommand{using}\isamarkupfalse%
\ assms\ \isacommand{by}\isamarkupfalse%
\ auto\isanewline
\isanewline
\ \ \isacommand{have}\isamarkupfalse%
\ strep{\isacharunderscore}{\kern0pt}sats\ {\isacharcolon}{\kern0pt}\ \isanewline
\ \ \ \ {\isachardoublequoteopen}strong{\isacharunderscore}{\kern0pt}replacement{\isacharparenleft}{\kern0pt}{\isacharhash}{\kern0pt}{\isacharhash}{\kern0pt}M{\isacharcomma}{\kern0pt}\ {\isasymlambda}x{\isachardot}{\kern0pt}\ {\isasymlambda}z{\isachardot}{\kern0pt}\ sats{\isacharparenleft}{\kern0pt}M{\isacharcomma}{\kern0pt}\ rep{\isacharunderscore}{\kern0pt}for{\isacharunderscore}{\kern0pt}recfun{\isacharunderscore}{\kern0pt}fm{\isacharparenleft}{\kern0pt}p{\isacharcomma}{\kern0pt}\ {\isadigit{0}}{\isacharcomma}{\kern0pt}\ {\isadigit{1}}{\isacharcomma}{\kern0pt}\ {\isadigit{2}}{\isacharparenright}{\kern0pt}{\isacharcomma}{\kern0pt}\ {\isacharbrackleft}{\kern0pt}x{\isacharcomma}{\kern0pt}\ z{\isacharbrackright}{\kern0pt}\ {\isacharat}{\kern0pt}\ {\isacharbrackleft}{\kern0pt}r{\isacharbrackright}{\kern0pt}{\isacharparenright}{\kern0pt}{\isacharparenright}{\kern0pt}{\isachardoublequoteclose}\isanewline
\ \ \ \ \isacommand{apply}\isamarkupfalse%
\ {\isacharparenleft}{\kern0pt}rule{\isacharunderscore}{\kern0pt}tac\ replacement{\isacharunderscore}{\kern0pt}ax{\isacharparenright}{\kern0pt}\ \isanewline
\ \ \ \ \isacommand{unfolding}\isamarkupfalse%
\ rep{\isacharunderscore}{\kern0pt}for{\isacharunderscore}{\kern0pt}recfun{\isacharunderscore}{\kern0pt}fm{\isacharunderscore}{\kern0pt}def\ \isanewline
\ \ \ \ \isacommand{using}\isamarkupfalse%
\ assms\ \isanewline
\ \ \ \ \ \ \isacommand{apply}\isamarkupfalse%
\ auto{\isacharbrackleft}{\kern0pt}{\isadigit{2}}{\isacharbrackright}{\kern0pt}\isanewline
\ \ \ \ \isacommand{apply}\isamarkupfalse%
\ simp\isanewline
\ \ \ \ \isacommand{using}\isamarkupfalse%
\ assms\isanewline
\ \ \ \ \isacommand{apply}\isamarkupfalse%
{\isacharparenleft}{\kern0pt}rule{\isacharunderscore}{\kern0pt}tac\ pred{\isacharunderscore}{\kern0pt}le{\isacharcomma}{\kern0pt}\ simp{\isacharunderscore}{\kern0pt}all{\isacharparenright}{\kern0pt}{\isacharplus}{\kern0pt}\isanewline
\ \ \ \ \isacommand{apply}\isamarkupfalse%
{\isacharparenleft}{\kern0pt}rule\ Un{\isacharunderscore}{\kern0pt}least{\isacharunderscore}{\kern0pt}lt{\isacharparenright}{\kern0pt}{\isacharplus}{\kern0pt}\isanewline
\ \ \ \ \ \ \isacommand{apply}\isamarkupfalse%
\ {\isacharparenleft}{\kern0pt}simp{\isacharcomma}{\kern0pt}\ simp{\isacharparenright}{\kern0pt}\isanewline
\ \ \ \ \isacommand{apply}\isamarkupfalse%
{\isacharparenleft}{\kern0pt}rule\ Un{\isacharunderscore}{\kern0pt}least{\isacharunderscore}{\kern0pt}lt{\isacharcomma}{\kern0pt}\ subst\ arity{\isacharunderscore}{\kern0pt}pair{\isacharunderscore}{\kern0pt}fm{\isacharparenright}{\kern0pt}\isanewline
\ \ \ \ \ \ \ \ \isacommand{apply}\isamarkupfalse%
\ auto{\isacharbrackleft}{\kern0pt}{\isadigit{4}}{\isacharbrackright}{\kern0pt}\isanewline
\ \ \ \ \ \isacommand{apply}\isamarkupfalse%
{\isacharparenleft}{\kern0pt}rule\ Un{\isacharunderscore}{\kern0pt}least{\isacharunderscore}{\kern0pt}lt{\isacharcomma}{\kern0pt}\ simp{\isacharparenright}{\kern0pt}{\isacharplus}{\kern0pt}\isanewline
\ \ \ \ \ \isacommand{apply}\isamarkupfalse%
\ simp\isanewline
\ \ \ \ \isacommand{apply}\isamarkupfalse%
{\isacharparenleft}{\kern0pt}rule\ Un{\isacharunderscore}{\kern0pt}least{\isacharunderscore}{\kern0pt}lt{\isacharcomma}{\kern0pt}\ subst\ arity{\isacharunderscore}{\kern0pt}is{\isacharunderscore}{\kern0pt}recfun{\isacharunderscore}{\kern0pt}fm{\isacharbrackleft}{\kern0pt}\isakeyword{where}\ i{\isacharequal}{\kern0pt}{\isachardoublequoteopen}arity{\isacharparenleft}{\kern0pt}p{\isacharparenright}{\kern0pt}{\isachardoublequoteclose}{\isacharbrackright}{\kern0pt}{\isacharparenright}{\kern0pt}\ \isanewline
\ \ \ \ \isacommand{using}\isamarkupfalse%
\ assms\isanewline
\ \ \ \ \ \ \ \ \ \ \ \isacommand{apply}\isamarkupfalse%
\ auto{\isacharbrackleft}{\kern0pt}{\isadigit{6}}{\isacharbrackright}{\kern0pt}\isanewline
\ \ \ \ \ \isacommand{apply}\isamarkupfalse%
{\isacharparenleft}{\kern0pt}rule\ Un{\isacharunderscore}{\kern0pt}least{\isacharunderscore}{\kern0pt}lt{\isacharparenright}{\kern0pt}{\isacharplus}{\kern0pt}\isanewline
\ \ \ \ \isacommand{apply}\isamarkupfalse%
\ simp{\isacharunderscore}{\kern0pt}all\isanewline
\ \ \ \ \ \isacommand{apply}\isamarkupfalse%
{\isacharparenleft}{\kern0pt}rule\ pred{\isacharunderscore}{\kern0pt}le{\isacharcomma}{\kern0pt}\ simp{\isacharunderscore}{\kern0pt}all{\isacharparenright}{\kern0pt}{\isacharplus}{\kern0pt}\isanewline
\ \ \ \ \ \isacommand{apply}\isamarkupfalse%
{\isacharparenleft}{\kern0pt}rule{\isacharunderscore}{\kern0pt}tac\ j{\isacharequal}{\kern0pt}{\isadigit{3}}\ \isakeyword{in}\ le{\isacharunderscore}{\kern0pt}trans{\isacharparenright}{\kern0pt}\isanewline
\ \ \ \ \isacommand{apply}\isamarkupfalse%
\ simp{\isacharunderscore}{\kern0pt}all\isanewline
\ \ \ \ \ \isacommand{apply}\isamarkupfalse%
{\isacharparenleft}{\kern0pt}rule{\isacharunderscore}{\kern0pt}tac\ j{\isacharequal}{\kern0pt}{\isadigit{3}}\ \isakeyword{in}\ le{\isacharunderscore}{\kern0pt}trans{\isacharparenright}{\kern0pt}\isanewline
\ \ \ \ \ \isacommand{apply}\isamarkupfalse%
\ simp{\isacharunderscore}{\kern0pt}all\isanewline
\ \ \ \ \isacommand{done}\isamarkupfalse%
\isanewline
\ \ \ \ \isanewline
\ \ \isacommand{show}\isamarkupfalse%
\ {\isachardoublequoteopen}strong{\isacharunderscore}{\kern0pt}replacement{\isacharparenleft}{\kern0pt}{\isacharhash}{\kern0pt}{\isacharhash}{\kern0pt}M{\isacharcomma}{\kern0pt}\ \isanewline
\ \ \ \ {\isasymlambda}x\ z{\isachardot}{\kern0pt}\ {\isasymexists}y{\isacharbrackleft}{\kern0pt}{\isacharhash}{\kern0pt}{\isacharhash}{\kern0pt}M{\isacharbrackright}{\kern0pt}{\isachardot}{\kern0pt}\ {\isasymexists}g{\isacharbrackleft}{\kern0pt}{\isacharhash}{\kern0pt}{\isacharhash}{\kern0pt}M{\isacharbrackright}{\kern0pt}{\isachardot}{\kern0pt}\ pair{\isacharparenleft}{\kern0pt}{\isacharhash}{\kern0pt}{\isacharhash}{\kern0pt}M{\isacharcomma}{\kern0pt}\ x{\isacharcomma}{\kern0pt}\ y{\isacharcomma}{\kern0pt}\ z{\isacharparenright}{\kern0pt}\ {\isasymand}\ is{\isacharunderscore}{\kern0pt}recfun{\isacharparenleft}{\kern0pt}r{\isacharcomma}{\kern0pt}\ x{\isacharcomma}{\kern0pt}\ H{\isacharcomma}{\kern0pt}\ g{\isacharparenright}{\kern0pt}\ {\isasymand}\ y\ {\isacharequal}{\kern0pt}\ H{\isacharparenleft}{\kern0pt}x{\isacharcomma}{\kern0pt}\ g{\isacharparenright}{\kern0pt}{\isacharparenright}{\kern0pt}{\isachardoublequoteclose}\isanewline
\ \ \ \ \isacommand{apply}\isamarkupfalse%
\ {\isacharparenleft}{\kern0pt}rule{\isacharunderscore}{\kern0pt}tac\ P{\isacharequal}{\kern0pt}{\isachardoublequoteopen}\ {\isasymlambda}x{\isachardot}{\kern0pt}\ {\isasymlambda}z{\isachardot}{\kern0pt}\ sats{\isacharparenleft}{\kern0pt}M{\isacharcomma}{\kern0pt}\ rep{\isacharunderscore}{\kern0pt}for{\isacharunderscore}{\kern0pt}recfun{\isacharunderscore}{\kern0pt}fm{\isacharparenleft}{\kern0pt}p{\isacharcomma}{\kern0pt}\ {\isadigit{0}}{\isacharcomma}{\kern0pt}\ {\isadigit{1}}{\isacharcomma}{\kern0pt}\ {\isadigit{2}}{\isacharparenright}{\kern0pt}{\isacharcomma}{\kern0pt}\ {\isacharbrackleft}{\kern0pt}x{\isacharcomma}{\kern0pt}\ z{\isacharbrackright}{\kern0pt}\ {\isacharat}{\kern0pt}\ {\isacharbrackleft}{\kern0pt}r{\isacharbrackright}{\kern0pt}{\isacharparenright}{\kern0pt}{\isachardoublequoteclose}\ \isakeyword{in}\ strep{\isacharunderscore}{\kern0pt}iff{\isacharparenright}{\kern0pt}\isanewline
\ \ \ \ \isacommand{apply}\isamarkupfalse%
\ {\isacharparenleft}{\kern0pt}rule\ strep{\isacharunderscore}{\kern0pt}sats{\isacharparenright}{\kern0pt}\isanewline
\ \ \ \ \isacommand{apply}\isamarkupfalse%
\ {\isacharparenleft}{\kern0pt}rule\ ballI{\isacharsemicolon}{\kern0pt}\ rule\ ballI{\isacharsemicolon}{\kern0pt}\ rule\ iff{\isacharunderscore}{\kern0pt}flip{\isacharparenright}{\kern0pt}\isanewline
\ \ \ \ \isacommand{apply}\isamarkupfalse%
\ {\isacharparenleft}{\kern0pt}rule{\isacharunderscore}{\kern0pt}tac\ Q{\isacharequal}{\kern0pt}{\isachardoublequoteopen}M{\isacharcomma}{\kern0pt}\ {\isacharbrackleft}{\kern0pt}x{\isacharcomma}{\kern0pt}\ z{\isacharcomma}{\kern0pt}\ r{\isacharbrackright}{\kern0pt}\ {\isasymTurnstile}\ rep{\isacharunderscore}{\kern0pt}for{\isacharunderscore}{\kern0pt}recfun{\isacharunderscore}{\kern0pt}fm{\isacharparenleft}{\kern0pt}p{\isacharcomma}{\kern0pt}\ {\isadigit{0}}{\isacharcomma}{\kern0pt}\ {\isadigit{1}}{\isacharcomma}{\kern0pt}\ {\isadigit{2}}{\isacharparenright}{\kern0pt}{\isachardoublequoteclose}\ \isakeyword{in}\ iff{\isacharunderscore}{\kern0pt}trans{\isacharparenright}{\kern0pt}\ \isanewline
\ \ \ \ \isacommand{apply}\isamarkupfalse%
\ {\isacharparenleft}{\kern0pt}rule{\isacharunderscore}{\kern0pt}tac\ rep{\isacharunderscore}{\kern0pt}for{\isacharunderscore}{\kern0pt}recfun{\isacharunderscore}{\kern0pt}fm{\isacharunderscore}{\kern0pt}sats{\isacharunderscore}{\kern0pt}iff{\isacharparenright}{\kern0pt}\isanewline
\ \ \ \ \isacommand{using}\isamarkupfalse%
\ assms\ \isacommand{apply}\isamarkupfalse%
\ simp{\isacharunderscore}{\kern0pt}all\isanewline
\ \ \ \ \isacommand{apply}\isamarkupfalse%
\ {\isacharparenleft}{\kern0pt}rule{\isacharunderscore}{\kern0pt}tac\ Q{\isacharequal}{\kern0pt}{\isachardoublequoteopen}a{\isadigit{0}}\ {\isacharequal}{\kern0pt}\ H{\isacharparenleft}{\kern0pt}a{\isadigit{2}}{\isacharcomma}{\kern0pt}\ a{\isadigit{1}}{\isacharparenright}{\kern0pt}{\isachardoublequoteclose}\ \isakeyword{in}\ iff{\isacharunderscore}{\kern0pt}trans{\isacharparenright}{\kern0pt}\isanewline
\ \ \ \ \isacommand{apply}\isamarkupfalse%
\ simp\isanewline
\ \ \ \ \isacommand{apply}\isamarkupfalse%
\ {\isacharparenleft}{\kern0pt}rule{\isacharunderscore}{\kern0pt}tac\ Q{\isacharequal}{\kern0pt}{\isachardoublequoteopen}M{\isacharcomma}{\kern0pt}\ {\isacharbrackleft}{\kern0pt}a{\isadigit{0}}{\isacharcomma}{\kern0pt}\ a{\isadigit{1}}{\isacharcomma}{\kern0pt}\ a{\isadigit{2}}{\isacharbrackright}{\kern0pt}\ {\isacharat}{\kern0pt}\ Cons{\isacharparenleft}{\kern0pt}a{\isadigit{3}}{\isacharcomma}{\kern0pt}\ env{\isacharparenright}{\kern0pt}\ {\isasymTurnstile}\ p{\isachardoublequoteclose}\ \isakeyword{in}\ iff{\isacharunderscore}{\kern0pt}trans{\isacharparenright}{\kern0pt}\isanewline
\ \ \ \ \isacommand{apply}\isamarkupfalse%
\ {\isacharparenleft}{\kern0pt}rule{\isacharunderscore}{\kern0pt}tac\ ph{\isacharparenright}{\kern0pt}\isanewline
\ \ \ \ \isacommand{by}\isamarkupfalse%
\ simp{\isacharunderscore}{\kern0pt}all\isanewline
\isacommand{qed}\isamarkupfalse%
%
\endisatagproof
{\isafoldproof}%
%
\isadelimproof
\isanewline
%
\endisadelimproof
\isanewline
\isacommand{lemma}\isamarkupfalse%
\ recfun{\isacharunderscore}{\kern0pt}wfrec{\isacharunderscore}{\kern0pt}replacement{\isacharunderscore}{\kern0pt}lemma\ {\isacharcolon}{\kern0pt}\ \isanewline
\ \ \isakeyword{fixes}\ r\ H\ p\ x\ \isanewline
\ \ \isakeyword{assumes}\ \isanewline
\ \ \ \ {\isachardoublequoteopen}wf{\isacharparenleft}{\kern0pt}r{\isacharparenright}{\kern0pt}{\isachardoublequoteclose}\ \isanewline
\ \ \ \ {\isachardoublequoteopen}trans{\isacharparenleft}{\kern0pt}r{\isacharparenright}{\kern0pt}{\isachardoublequoteclose}\ \isanewline
\ \ \ \ {\isachardoublequoteopen}r\ {\isasymin}\ M{\isachardoublequoteclose}\ \isanewline
\ \ \ \ {\isachardoublequoteopen}p\ {\isasymin}\ formula{\isachardoublequoteclose}\isanewline
\ \ \ \ {\isachardoublequoteopen}arity{\isacharparenleft}{\kern0pt}p{\isacharparenright}{\kern0pt}\ {\isasymle}\ {\isadigit{3}}{\isachardoublequoteclose}\isanewline
\ \ \ \ {\isachardoublequoteopen}x\ {\isasymin}\ M{\isachardoublequoteclose}\ \isanewline
\ \ \ \ {\isachardoublequoteopen}{\isacharparenleft}{\kern0pt}{\isasymAnd}x\ g{\isachardot}{\kern0pt}\ x\ {\isasymin}\ M\ {\isasymLongrightarrow}\ g\ {\isasymin}\ M\ {\isasymLongrightarrow}\ function{\isacharparenleft}{\kern0pt}g{\isacharparenright}{\kern0pt}\ {\isasymLongrightarrow}\ H{\isacharparenleft}{\kern0pt}x{\isacharcomma}{\kern0pt}\ g{\isacharparenright}{\kern0pt}\ {\isasymin}\ M{\isacharparenright}{\kern0pt}{\isachardoublequoteclose}\ \isanewline
\ \ \ \ {\isachardoublequoteopen}\ {\isacharparenleft}{\kern0pt}{\isasymAnd}a{\isadigit{0}}\ a{\isadigit{1}}\ a{\isadigit{2}}\ env{\isachardot}{\kern0pt}\ a{\isadigit{0}}\ {\isasymin}\ M\ {\isasymLongrightarrow}\ a{\isadigit{1}}\ {\isasymin}\ M\ {\isasymLongrightarrow}\ a{\isadigit{2}}\ {\isasymin}\ M\ {\isasymLongrightarrow}\ env\ {\isasymin}\ list{\isacharparenleft}{\kern0pt}M{\isacharparenright}{\kern0pt}\ {\isasymLongrightarrow}\ a{\isadigit{0}}\ {\isacharequal}{\kern0pt}\ H{\isacharparenleft}{\kern0pt}a{\isadigit{2}}{\isacharcomma}{\kern0pt}\ a{\isadigit{1}}{\isacharparenright}{\kern0pt}\ {\isasymlongleftrightarrow}\ sats{\isacharparenleft}{\kern0pt}M{\isacharcomma}{\kern0pt}\ p{\isacharcomma}{\kern0pt}\ {\isacharbrackleft}{\kern0pt}a{\isadigit{0}}{\isacharcomma}{\kern0pt}\ a{\isadigit{1}}{\isacharcomma}{\kern0pt}\ a{\isadigit{2}}{\isacharbrackright}{\kern0pt}\ {\isacharat}{\kern0pt}\ env{\isacharparenright}{\kern0pt}{\isacharparenright}{\kern0pt}{\isachardoublequoteclose}\ \ \isanewline
\ \ \isanewline
\ \ \isakeyword{shows}\ {\isachardoublequoteopen}wfrec{\isacharunderscore}{\kern0pt}replacement{\isacharparenleft}{\kern0pt}{\isacharhash}{\kern0pt}{\isacharhash}{\kern0pt}M{\isacharcomma}{\kern0pt}\ {\isasymlambda}a\ b\ c{\isachardot}{\kern0pt}\ c\ {\isacharequal}{\kern0pt}\ H{\isacharparenleft}{\kern0pt}a{\isacharcomma}{\kern0pt}\ b{\isacharparenright}{\kern0pt}{\isacharcomma}{\kern0pt}\ r{\isacharparenright}{\kern0pt}{\isachardoublequoteclose}\ \isanewline
%
\isadelimproof
\ \ \isanewline
\ \ %
\endisadelimproof
%
\isatagproof
\isacommand{using}\isamarkupfalse%
\ assms\isanewline
\ \ \isacommand{unfolding}\isamarkupfalse%
\ wfrec{\isacharunderscore}{\kern0pt}replacement{\isacharunderscore}{\kern0pt}def\isanewline
\ \ \isacommand{apply}\isamarkupfalse%
{\isacharparenleft}{\kern0pt}rule{\isacharunderscore}{\kern0pt}tac\ P{\isacharequal}{\kern0pt}{\isachardoublequoteopen}\ {\isasymlambda}x\ z{\isachardot}{\kern0pt}\ {\isasymexists}y{\isacharbrackleft}{\kern0pt}{\isacharhash}{\kern0pt}{\isacharhash}{\kern0pt}M{\isacharbrackright}{\kern0pt}{\isachardot}{\kern0pt}\ {\isasymexists}g{\isacharbrackleft}{\kern0pt}{\isacharhash}{\kern0pt}{\isacharhash}{\kern0pt}M{\isacharbrackright}{\kern0pt}{\isachardot}{\kern0pt}\ pair{\isacharparenleft}{\kern0pt}{\isacharhash}{\kern0pt}{\isacharhash}{\kern0pt}M{\isacharcomma}{\kern0pt}\ x{\isacharcomma}{\kern0pt}\ y{\isacharcomma}{\kern0pt}\ z{\isacharparenright}{\kern0pt}\ {\isasymand}\ is{\isacharunderscore}{\kern0pt}recfun{\isacharparenleft}{\kern0pt}r{\isacharcomma}{\kern0pt}\ x{\isacharcomma}{\kern0pt}\ H{\isacharcomma}{\kern0pt}\ g{\isacharparenright}{\kern0pt}\ {\isasymand}\ y\ {\isacharequal}{\kern0pt}\ H{\isacharparenleft}{\kern0pt}x{\isacharcomma}{\kern0pt}\ g{\isacharparenright}{\kern0pt}{\isachardoublequoteclose}\ \isakeyword{in}\ \ strep{\isacharunderscore}{\kern0pt}iff{\isacharparenright}{\kern0pt}\ \isanewline
\ \ \ \isacommand{apply}\isamarkupfalse%
{\isacharparenleft}{\kern0pt}rule{\isacharunderscore}{\kern0pt}tac\ recfun{\isacharunderscore}{\kern0pt}strong{\isacharunderscore}{\kern0pt}replacement{\isacharunderscore}{\kern0pt}lemma{\isacharparenright}{\kern0pt}\ \isanewline
\ \ \ \ \ \ \ \ \ \ \isacommand{apply}\isamarkupfalse%
\ simp{\isacharunderscore}{\kern0pt}all\isanewline
\ \ \isacommand{apply}\isamarkupfalse%
\ clarify\ \isanewline
\ \ \isacommand{unfolding}\isamarkupfalse%
\ is{\isacharunderscore}{\kern0pt}wfrec{\isacharunderscore}{\kern0pt}def\ \isanewline
\ \ \isacommand{apply}\isamarkupfalse%
{\isacharparenleft}{\kern0pt}rule{\isacharunderscore}{\kern0pt}tac\ P{\isacharequal}{\kern0pt}{\isachardoublequoteopen}{\isasymforall}f\ {\isasymin}\ M{\isachardot}{\kern0pt}\ M{\isacharunderscore}{\kern0pt}is{\isacharunderscore}{\kern0pt}recfun{\isacharparenleft}{\kern0pt}{\isacharhash}{\kern0pt}{\isacharhash}{\kern0pt}M{\isacharcomma}{\kern0pt}\ {\isasymlambda}a\ b\ c{\isachardot}{\kern0pt}\ c\ {\isacharequal}{\kern0pt}\ H{\isacharparenleft}{\kern0pt}a{\isacharcomma}{\kern0pt}\ b{\isacharparenright}{\kern0pt}{\isacharcomma}{\kern0pt}\ r{\isacharcomma}{\kern0pt}\ xa{\isacharcomma}{\kern0pt}\ f{\isacharparenright}{\kern0pt}\ {\isasymlongleftrightarrow}\ is{\isacharunderscore}{\kern0pt}recfun{\isacharparenleft}{\kern0pt}r{\isacharcomma}{\kern0pt}\ xa{\isacharcomma}{\kern0pt}\ H{\isacharcomma}{\kern0pt}\ f{\isacharparenright}{\kern0pt}{\isachardoublequoteclose}\ \isakeyword{in}\ mp{\isacharparenright}{\kern0pt}\ \isanewline
\ \ \ \isacommand{apply}\isamarkupfalse%
\ simp\ \isanewline
\ \ \isacommand{apply}\isamarkupfalse%
\ clarify\ \isanewline
\ \ \isacommand{apply}\isamarkupfalse%
{\isacharparenleft}{\kern0pt}rule{\isacharunderscore}{\kern0pt}tac\ is{\isacharunderscore}{\kern0pt}recfun{\isacharunderscore}{\kern0pt}abs{\isacharparenright}{\kern0pt}\ \isanewline
\ \ \isacommand{unfolding}\isamarkupfalse%
\ relation{\isadigit{2}}{\isacharunderscore}{\kern0pt}def\ \isanewline
\ \ \ \ \ \ \isacommand{apply}\isamarkupfalse%
\ simp{\isacharunderscore}{\kern0pt}all\ \isanewline
\ \ \isacommand{done}\isamarkupfalse%
%
\endisatagproof
{\isafoldproof}%
%
\isadelimproof
\ \isanewline
%
\endisadelimproof
\isanewline
\isacommand{lemma}\isamarkupfalse%
\ wf{\isacharunderscore}{\kern0pt}exists{\isacharunderscore}{\kern0pt}is{\isacharunderscore}{\kern0pt}recfun{\isacharprime}{\kern0pt}\ {\isacharcolon}{\kern0pt}\ \isanewline
\ \ \isakeyword{fixes}\ r\ H\ p\ x\ \isanewline
\ \ \isakeyword{assumes}\ \isanewline
\ \ \ \ {\isachardoublequoteopen}wf{\isacharparenleft}{\kern0pt}r{\isacharparenright}{\kern0pt}{\isachardoublequoteclose}\ \isanewline
\ \ \ \ {\isachardoublequoteopen}trans{\isacharparenleft}{\kern0pt}r{\isacharparenright}{\kern0pt}{\isachardoublequoteclose}\ \isanewline
\ \ \ \ {\isachardoublequoteopen}r\ {\isasymin}\ M{\isachardoublequoteclose}\ \isanewline
\ \ \ \ {\isachardoublequoteopen}p\ {\isasymin}\ formula{\isachardoublequoteclose}\isanewline
\ \ \ \ {\isachardoublequoteopen}arity{\isacharparenleft}{\kern0pt}p{\isacharparenright}{\kern0pt}\ {\isasymle}\ {\isadigit{3}}{\isachardoublequoteclose}\isanewline
\ \ \ \ {\isachardoublequoteopen}x\ {\isasymin}\ M{\isachardoublequoteclose}\ \isanewline
\ \ \ \ {\isachardoublequoteopen}{\isacharparenleft}{\kern0pt}{\isasymAnd}x\ g{\isachardot}{\kern0pt}\ x\ {\isasymin}\ M\ {\isasymLongrightarrow}\ g\ {\isasymin}\ M\ {\isasymLongrightarrow}\ function{\isacharparenleft}{\kern0pt}g{\isacharparenright}{\kern0pt}\ {\isasymLongrightarrow}\ H{\isacharparenleft}{\kern0pt}x{\isacharcomma}{\kern0pt}\ g{\isacharparenright}{\kern0pt}\ {\isasymin}\ M{\isacharparenright}{\kern0pt}{\isachardoublequoteclose}\ \isanewline
\ \ \ \ {\isachardoublequoteopen}\ {\isacharparenleft}{\kern0pt}{\isasymAnd}a{\isadigit{0}}\ a{\isadigit{1}}\ a{\isadigit{2}}\ env{\isachardot}{\kern0pt}\ a{\isadigit{0}}\ {\isasymin}\ M\ {\isasymLongrightarrow}\ a{\isadigit{1}}\ {\isasymin}\ M\ {\isasymLongrightarrow}\ a{\isadigit{2}}\ {\isasymin}\ M\ {\isasymLongrightarrow}\ env\ {\isasymin}\ list{\isacharparenleft}{\kern0pt}M{\isacharparenright}{\kern0pt}\ {\isasymLongrightarrow}\ a{\isadigit{0}}\ {\isacharequal}{\kern0pt}\ H{\isacharparenleft}{\kern0pt}a{\isadigit{2}}{\isacharcomma}{\kern0pt}\ a{\isadigit{1}}{\isacharparenright}{\kern0pt}\ {\isasymlongleftrightarrow}\ sats{\isacharparenleft}{\kern0pt}M{\isacharcomma}{\kern0pt}\ p{\isacharcomma}{\kern0pt}\ {\isacharbrackleft}{\kern0pt}a{\isadigit{0}}{\isacharcomma}{\kern0pt}\ a{\isadigit{1}}{\isacharcomma}{\kern0pt}\ a{\isadigit{2}}{\isacharbrackright}{\kern0pt}\ {\isacharat}{\kern0pt}\ env{\isacharparenright}{\kern0pt}{\isacharparenright}{\kern0pt}{\isachardoublequoteclose}\ \ \isanewline
\ \ \isakeyword{shows}\ {\isachardoublequoteopen}{\isasymexists}f\ {\isasymin}\ M{\isachardot}{\kern0pt}\ is{\isacharunderscore}{\kern0pt}recfun{\isacharparenleft}{\kern0pt}r{\isacharcomma}{\kern0pt}\ x{\isacharcomma}{\kern0pt}\ H{\isacharcomma}{\kern0pt}\ f{\isacharparenright}{\kern0pt}{\isachardoublequoteclose}\isanewline
%
\isadelimproof
\isanewline
\ \ %
\endisadelimproof
%
\isatagproof
\isacommand{using}\isamarkupfalse%
\ assms\ \isanewline
\ \ \isacommand{apply}\isamarkupfalse%
{\isacharparenleft}{\kern0pt}rule{\isacharunderscore}{\kern0pt}tac\ P{\isacharequal}{\kern0pt}{\isachardoublequoteopen}rex{\isacharparenleft}{\kern0pt}{\isacharhash}{\kern0pt}{\isacharhash}{\kern0pt}M{\isacharcomma}{\kern0pt}\ is{\isacharunderscore}{\kern0pt}recfun{\isacharparenleft}{\kern0pt}r{\isacharcomma}{\kern0pt}\ x{\isacharcomma}{\kern0pt}\ H{\isacharparenright}{\kern0pt}{\isacharparenright}{\kern0pt}{\isachardoublequoteclose}\ \isakeyword{in}\ mp{\isacharparenright}{\kern0pt}\ \isanewline
\ \ \ \isacommand{apply}\isamarkupfalse%
\ simp\ \isanewline
\ \ \isacommand{apply}\isamarkupfalse%
\ {\isacharparenleft}{\kern0pt}rule{\isacharunderscore}{\kern0pt}tac\ wf{\isacharunderscore}{\kern0pt}exists{\isacharunderscore}{\kern0pt}is{\isacharunderscore}{\kern0pt}recfun{\isacharparenright}{\kern0pt}\ \isanewline
\ \ \ \ \ \ \ \isacommand{apply}\isamarkupfalse%
\ simp\ \isanewline
\ \ \ \ \ \ \isacommand{apply}\isamarkupfalse%
\ simp\ \isanewline
\ \ \ \ \ \isacommand{apply}\isamarkupfalse%
\ simp\isanewline
\ \ \ \ \isacommand{apply}\isamarkupfalse%
{\isacharparenleft}{\kern0pt}rule{\isacharunderscore}{\kern0pt}tac\ recfun{\isacharunderscore}{\kern0pt}strong{\isacharunderscore}{\kern0pt}replacement{\isacharunderscore}{\kern0pt}lemma{\isacharparenright}{\kern0pt}\ \isanewline
\ \ \ \ \ \ \ \ \ \ \ \isacommand{apply}\isamarkupfalse%
\ simp{\isacharunderscore}{\kern0pt}all\ \isanewline
\ \ \isacommand{done}\isamarkupfalse%
%
\endisatagproof
{\isafoldproof}%
%
\isadelimproof
\ \isanewline
%
\endisadelimproof
\isanewline
\isacommand{lemma}\isamarkupfalse%
\ the{\isacharunderscore}{\kern0pt}recfun{\isacharunderscore}{\kern0pt}in{\isacharunderscore}{\kern0pt}M\ {\isacharcolon}{\kern0pt}\ \isanewline
\ \ \isakeyword{fixes}\ r\ H\ p\ x\ \isanewline
\ \ \isakeyword{assumes}\ \isanewline
\ \ \ \ {\isachardoublequoteopen}wf{\isacharparenleft}{\kern0pt}r{\isacharparenright}{\kern0pt}{\isachardoublequoteclose}\ \isanewline
\ \ \ \ {\isachardoublequoteopen}trans{\isacharparenleft}{\kern0pt}r{\isacharparenright}{\kern0pt}{\isachardoublequoteclose}\ \isanewline
\ \ \ \ {\isachardoublequoteopen}r\ {\isasymin}\ M{\isachardoublequoteclose}\ \isanewline
\ \ \ \ {\isachardoublequoteopen}p\ {\isasymin}\ formula{\isachardoublequoteclose}\isanewline
\ \ \ \ {\isachardoublequoteopen}arity{\isacharparenleft}{\kern0pt}p{\isacharparenright}{\kern0pt}\ {\isasymle}\ {\isadigit{3}}{\isachardoublequoteclose}\isanewline
\ \ \ \ {\isachardoublequoteopen}x\ {\isasymin}\ M{\isachardoublequoteclose}\ \isanewline
\ \ \ \ {\isachardoublequoteopen}{\isacharparenleft}{\kern0pt}{\isasymAnd}x\ g{\isachardot}{\kern0pt}\ x\ {\isasymin}\ M\ {\isasymLongrightarrow}\ g\ {\isasymin}\ M\ {\isasymLongrightarrow}\ function{\isacharparenleft}{\kern0pt}g{\isacharparenright}{\kern0pt}\ {\isasymLongrightarrow}\ H{\isacharparenleft}{\kern0pt}x{\isacharcomma}{\kern0pt}\ g{\isacharparenright}{\kern0pt}\ {\isasymin}\ M{\isacharparenright}{\kern0pt}{\isachardoublequoteclose}\ \isanewline
\ \ \ \ {\isachardoublequoteopen}\ {\isacharparenleft}{\kern0pt}{\isasymAnd}a{\isadigit{0}}\ a{\isadigit{1}}\ a{\isadigit{2}}\ env{\isachardot}{\kern0pt}\ a{\isadigit{0}}\ {\isasymin}\ M\ {\isasymLongrightarrow}\ a{\isadigit{1}}\ {\isasymin}\ M\ {\isasymLongrightarrow}\ a{\isadigit{2}}\ {\isasymin}\ M\ {\isasymLongrightarrow}\ env\ {\isasymin}\ list{\isacharparenleft}{\kern0pt}M{\isacharparenright}{\kern0pt}\ {\isasymLongrightarrow}\ a{\isadigit{0}}\ {\isacharequal}{\kern0pt}\ H{\isacharparenleft}{\kern0pt}a{\isadigit{2}}{\isacharcomma}{\kern0pt}\ a{\isadigit{1}}{\isacharparenright}{\kern0pt}\ {\isasymlongleftrightarrow}\ sats{\isacharparenleft}{\kern0pt}M{\isacharcomma}{\kern0pt}\ p{\isacharcomma}{\kern0pt}\ {\isacharbrackleft}{\kern0pt}a{\isadigit{0}}{\isacharcomma}{\kern0pt}\ a{\isadigit{1}}{\isacharcomma}{\kern0pt}\ a{\isadigit{2}}{\isacharbrackright}{\kern0pt}\ {\isacharat}{\kern0pt}\ env{\isacharparenright}{\kern0pt}{\isacharparenright}{\kern0pt}{\isachardoublequoteclose}\ \ \isanewline
\ \ \isakeyword{shows}\ {\isachardoublequoteopen}the{\isacharunderscore}{\kern0pt}recfun{\isacharparenleft}{\kern0pt}r{\isacharcomma}{\kern0pt}\ x{\isacharcomma}{\kern0pt}\ H{\isacharparenright}{\kern0pt}\ {\isasymin}\ M{\isachardoublequoteclose}\ \isanewline
%
\isadelimproof
%
\endisadelimproof
%
\isatagproof
\isacommand{proof}\isamarkupfalse%
\ {\isacharminus}{\kern0pt}\ \isanewline
\ \ \isacommand{have}\isamarkupfalse%
\ {\isachardoublequoteopen}{\isasymexists}f\ {\isasymin}\ M{\isachardot}{\kern0pt}\ is{\isacharunderscore}{\kern0pt}recfun{\isacharparenleft}{\kern0pt}r{\isacharcomma}{\kern0pt}\ x{\isacharcomma}{\kern0pt}\ H{\isacharcomma}{\kern0pt}\ f{\isacharparenright}{\kern0pt}{\isachardoublequoteclose}\ \isanewline
\ \ \ \ \isacommand{apply}\isamarkupfalse%
{\isacharparenleft}{\kern0pt}rule{\isacharunderscore}{\kern0pt}tac\ wf{\isacharunderscore}{\kern0pt}exists{\isacharunderscore}{\kern0pt}is{\isacharunderscore}{\kern0pt}recfun{\isacharprime}{\kern0pt}{\isacharparenright}{\kern0pt}\ \isanewline
\ \ \ \ \isacommand{using}\isamarkupfalse%
\ assms\ \isanewline
\ \ \ \ \isacommand{by}\isamarkupfalse%
\ auto\ \isanewline
\ \ \isacommand{then}\isamarkupfalse%
\ \isacommand{obtain}\isamarkupfalse%
\ f\ \isakeyword{where}\ fH\ {\isacharcolon}{\kern0pt}\ {\isachardoublequoteopen}f\ {\isasymin}\ M{\isachardoublequoteclose}\ {\isachardoublequoteopen}is{\isacharunderscore}{\kern0pt}recfun{\isacharparenleft}{\kern0pt}r{\isacharcomma}{\kern0pt}\ x{\isacharcomma}{\kern0pt}\ H{\isacharcomma}{\kern0pt}\ f{\isacharparenright}{\kern0pt}{\isachardoublequoteclose}\ \isacommand{by}\isamarkupfalse%
\ auto\ \isanewline
\ \ \isacommand{then}\isamarkupfalse%
\ \isacommand{have}\isamarkupfalse%
\ {\isachardoublequoteopen}the{\isacharunderscore}{\kern0pt}recfun{\isacharparenleft}{\kern0pt}r{\isacharcomma}{\kern0pt}\ x{\isacharcomma}{\kern0pt}\ H{\isacharparenright}{\kern0pt}\ {\isacharequal}{\kern0pt}\ f{\isachardoublequoteclose}\ \isanewline
\ \ \ \ \isacommand{apply}\isamarkupfalse%
{\isacharparenleft}{\kern0pt}rule{\isacharunderscore}{\kern0pt}tac\ the{\isacharunderscore}{\kern0pt}recfun{\isacharunderscore}{\kern0pt}eq{\isacharparenright}{\kern0pt}\ \isanewline
\ \ \ \ \isacommand{using}\isamarkupfalse%
\ assms\ \isanewline
\ \ \ \ \isacommand{by}\isamarkupfalse%
\ auto\ \isanewline
\ \ \isacommand{then}\isamarkupfalse%
\ \isacommand{show}\isamarkupfalse%
\ {\isachardoublequoteopen}the{\isacharunderscore}{\kern0pt}recfun{\isacharparenleft}{\kern0pt}r{\isacharcomma}{\kern0pt}\ x{\isacharcomma}{\kern0pt}\ H{\isacharparenright}{\kern0pt}\ {\isasymin}\ M{\isachardoublequoteclose}\ \isacommand{using}\isamarkupfalse%
\ fH\ \isacommand{by}\isamarkupfalse%
\ auto\ \isanewline
\isacommand{qed}\isamarkupfalse%
%
\endisatagproof
{\isafoldproof}%
%
\isadelimproof
\isanewline
%
\endisadelimproof
\isanewline
\isacommand{lemma}\isamarkupfalse%
\ wftrec{\isacharunderscore}{\kern0pt}in{\isacharunderscore}{\kern0pt}M\ {\isacharcolon}{\kern0pt}\ \isanewline
\ \ \isakeyword{fixes}\ r\ H\ p\ x\ \isanewline
\ \ \isakeyword{assumes}\ \isanewline
\ \ \ \ {\isachardoublequoteopen}wf{\isacharparenleft}{\kern0pt}r{\isacharparenright}{\kern0pt}{\isachardoublequoteclose}\ \isanewline
\ \ \ \ {\isachardoublequoteopen}trans{\isacharparenleft}{\kern0pt}r{\isacharparenright}{\kern0pt}{\isachardoublequoteclose}\ \isanewline
\ \ \ \ {\isachardoublequoteopen}r\ {\isasymin}\ M{\isachardoublequoteclose}\ \isanewline
\ \ \ \ {\isachardoublequoteopen}p\ {\isasymin}\ formula{\isachardoublequoteclose}\isanewline
\ \ \ \ {\isachardoublequoteopen}arity{\isacharparenleft}{\kern0pt}p{\isacharparenright}{\kern0pt}\ {\isasymle}\ {\isadigit{3}}{\isachardoublequoteclose}\isanewline
\ \ \ \ {\isachardoublequoteopen}x\ {\isasymin}\ M{\isachardoublequoteclose}\ \isanewline
\ \ \ \ {\isachardoublequoteopen}\ {\isacharparenleft}{\kern0pt}{\isasymAnd}a{\isadigit{0}}\ a{\isadigit{1}}\ a{\isadigit{2}}\ env{\isachardot}{\kern0pt}\ a{\isadigit{0}}\ {\isasymin}\ M\ {\isasymLongrightarrow}\ a{\isadigit{1}}\ {\isasymin}\ M\ {\isasymLongrightarrow}\ a{\isadigit{2}}\ {\isasymin}\ M\ {\isasymLongrightarrow}\ env\ {\isasymin}\ list{\isacharparenleft}{\kern0pt}M{\isacharparenright}{\kern0pt}\ {\isasymLongrightarrow}\ a{\isadigit{0}}\ {\isacharequal}{\kern0pt}\ H{\isacharparenleft}{\kern0pt}a{\isadigit{2}}{\isacharcomma}{\kern0pt}\ a{\isadigit{1}}{\isacharparenright}{\kern0pt}\ {\isasymlongleftrightarrow}\ sats{\isacharparenleft}{\kern0pt}M{\isacharcomma}{\kern0pt}\ p{\isacharcomma}{\kern0pt}\ {\isacharbrackleft}{\kern0pt}a{\isadigit{0}}{\isacharcomma}{\kern0pt}\ a{\isadigit{1}}{\isacharcomma}{\kern0pt}\ a{\isadigit{2}}{\isacharbrackright}{\kern0pt}\ {\isacharat}{\kern0pt}\ env{\isacharparenright}{\kern0pt}{\isacharparenright}{\kern0pt}{\isachardoublequoteclose}\ \ \isanewline
\ \ \isakeyword{and}\ HM\ {\isacharcolon}{\kern0pt}\ {\isachardoublequoteopen}{\isacharparenleft}{\kern0pt}{\isasymAnd}x\ g{\isachardot}{\kern0pt}\ x\ {\isasymin}\ M\ {\isasymLongrightarrow}\ g\ {\isasymin}\ M\ {\isasymLongrightarrow}\ function{\isacharparenleft}{\kern0pt}g{\isacharparenright}{\kern0pt}\ {\isasymLongrightarrow}\ H{\isacharparenleft}{\kern0pt}x{\isacharcomma}{\kern0pt}\ g{\isacharparenright}{\kern0pt}\ {\isasymin}\ M{\isacharparenright}{\kern0pt}{\isachardoublequoteclose}\ \isanewline
\ \ \isakeyword{shows}\ {\isachardoublequoteopen}wftrec{\isacharparenleft}{\kern0pt}r{\isacharcomma}{\kern0pt}\ x{\isacharcomma}{\kern0pt}\ H{\isacharparenright}{\kern0pt}\ {\isasymin}\ M{\isachardoublequoteclose}\ \isanewline
%
\isadelimproof
%
\endisadelimproof
%
\isatagproof
\isacommand{proof}\isamarkupfalse%
\ {\isacharminus}{\kern0pt}\ \isanewline
\ \ \isacommand{have}\isamarkupfalse%
\ recfuninM\ {\isacharcolon}{\kern0pt}\ {\isachardoublequoteopen}the{\isacharunderscore}{\kern0pt}recfun{\isacharparenleft}{\kern0pt}r{\isacharcomma}{\kern0pt}\ x{\isacharcomma}{\kern0pt}\ H{\isacharparenright}{\kern0pt}\ {\isasymin}\ M{\isachardoublequoteclose}\ \isanewline
\ \ \ \ \isacommand{apply}\isamarkupfalse%
{\isacharparenleft}{\kern0pt}rule\ the{\isacharunderscore}{\kern0pt}recfun{\isacharunderscore}{\kern0pt}in{\isacharunderscore}{\kern0pt}M{\isacharparenright}{\kern0pt}\isanewline
\ \ \ \ \isacommand{using}\isamarkupfalse%
\ assms\ \isanewline
\ \ \ \ \isacommand{by}\isamarkupfalse%
\ auto\isanewline
\isanewline
\ \ \isacommand{have}\isamarkupfalse%
\ {\isachardoublequoteopen}is{\isacharunderscore}{\kern0pt}recfun{\isacharparenleft}{\kern0pt}r{\isacharcomma}{\kern0pt}\ x{\isacharcomma}{\kern0pt}\ H{\isacharcomma}{\kern0pt}\ the{\isacharunderscore}{\kern0pt}recfun{\isacharparenleft}{\kern0pt}r{\isacharcomma}{\kern0pt}\ x{\isacharcomma}{\kern0pt}\ H{\isacharparenright}{\kern0pt}{\isacharparenright}{\kern0pt}{\isachardoublequoteclose}\ \isanewline
\ \ \ \ \isacommand{apply}\isamarkupfalse%
{\isacharparenleft}{\kern0pt}rule\ unfold{\isacharunderscore}{\kern0pt}the{\isacharunderscore}{\kern0pt}recfun{\isacharparenright}{\kern0pt}\isanewline
\ \ \ \ \isacommand{using}\isamarkupfalse%
\ assms\isanewline
\ \ \ \ \isacommand{by}\isamarkupfalse%
\ auto\isanewline
\ \ \isacommand{then}\isamarkupfalse%
\ \isacommand{have}\isamarkupfalse%
\ eq\ {\isacharcolon}{\kern0pt}\ {\isachardoublequoteopen}the{\isacharunderscore}{\kern0pt}recfun{\isacharparenleft}{\kern0pt}r{\isacharcomma}{\kern0pt}\ x{\isacharcomma}{\kern0pt}\ H{\isacharparenright}{\kern0pt}\ {\isacharequal}{\kern0pt}\ {\isacharparenleft}{\kern0pt}{\isasymlambda}y{\isasymin}r\ {\isacharminus}{\kern0pt}{\isacharbackquote}{\kern0pt}{\isacharbackquote}{\kern0pt}\ {\isacharbraceleft}{\kern0pt}x{\isacharbraceright}{\kern0pt}{\isachardot}{\kern0pt}\ H{\isacharparenleft}{\kern0pt}y{\isacharcomma}{\kern0pt}\ restrict{\isacharparenleft}{\kern0pt}the{\isacharunderscore}{\kern0pt}recfun{\isacharparenleft}{\kern0pt}r{\isacharcomma}{\kern0pt}\ x{\isacharcomma}{\kern0pt}\ H{\isacharparenright}{\kern0pt}{\isacharcomma}{\kern0pt}\ r\ {\isacharminus}{\kern0pt}{\isacharbackquote}{\kern0pt}{\isacharbackquote}{\kern0pt}\ {\isacharbraceleft}{\kern0pt}y{\isacharbraceright}{\kern0pt}{\isacharparenright}{\kern0pt}{\isacharparenright}{\kern0pt}{\isacharparenright}{\kern0pt}{\isachardoublequoteclose}\ \isanewline
\ \ \ \ \isacommand{unfolding}\isamarkupfalse%
\ is{\isacharunderscore}{\kern0pt}recfun{\isacharunderscore}{\kern0pt}def\ \isanewline
\ \ \ \ \isacommand{by}\isamarkupfalse%
\ simp\isanewline
\isanewline
\ \ \isacommand{have}\isamarkupfalse%
\ {\isachardoublequoteopen}H{\isacharparenleft}{\kern0pt}x{\isacharcomma}{\kern0pt}\ the{\isacharunderscore}{\kern0pt}recfun{\isacharparenleft}{\kern0pt}r{\isacharcomma}{\kern0pt}\ x{\isacharcomma}{\kern0pt}\ H{\isacharparenright}{\kern0pt}{\isacharparenright}{\kern0pt}\ {\isasymin}\ M{\isachardoublequoteclose}\ \isanewline
\ \ \ \ \isacommand{apply}\isamarkupfalse%
{\isacharparenleft}{\kern0pt}rule\ HM{\isacharparenright}{\kern0pt}\isanewline
\ \ \ \ \isacommand{using}\isamarkupfalse%
\ assms\ recfuninM\ \isanewline
\ \ \ \ \ \ \isacommand{apply}\isamarkupfalse%
\ auto{\isacharbrackleft}{\kern0pt}{\isadigit{2}}{\isacharbrackright}{\kern0pt}\isanewline
\ \ \ \ \isacommand{apply}\isamarkupfalse%
{\isacharparenleft}{\kern0pt}subst\ eq{\isacharcomma}{\kern0pt}\ rule\ function{\isacharunderscore}{\kern0pt}lam{\isacharparenright}{\kern0pt}\isanewline
\ \ \ \ \isacommand{done}\isamarkupfalse%
\isanewline
\ \ \isacommand{then}\isamarkupfalse%
\ \isacommand{show}\isamarkupfalse%
\ {\isacharquery}{\kern0pt}thesis\ \isanewline
\ \ \ \ \isacommand{unfolding}\isamarkupfalse%
\ wftrec{\isacharunderscore}{\kern0pt}def\ \isanewline
\ \ \ \ \isacommand{by}\isamarkupfalse%
\ auto\isanewline
\isacommand{qed}\isamarkupfalse%
%
\endisatagproof
{\isafoldproof}%
%
\isadelimproof
\isanewline
%
\endisadelimproof
\isanewline
\isanewline
\isacommand{lemma}\isamarkupfalse%
\ wftrec{\isacharunderscore}{\kern0pt}pair{\isacharunderscore}{\kern0pt}closed\ {\isacharcolon}{\kern0pt}\ \isanewline
\ \ \isakeyword{fixes}\ r\ H\ p\ A\ \ \isanewline
\ \ \isakeyword{assumes}\ \isanewline
\ \ \ \ {\isachardoublequoteopen}wf{\isacharparenleft}{\kern0pt}r{\isacharparenright}{\kern0pt}{\isachardoublequoteclose}\ \isanewline
\ \ \ \ {\isachardoublequoteopen}trans{\isacharparenleft}{\kern0pt}r{\isacharparenright}{\kern0pt}{\isachardoublequoteclose}\ \isanewline
\ \ \ \ {\isachardoublequoteopen}r\ {\isasymin}\ M{\isachardoublequoteclose}\ \isanewline
\ \ \ \ {\isachardoublequoteopen}p\ {\isasymin}\ formula{\isachardoublequoteclose}\isanewline
\ \ \ \ {\isachardoublequoteopen}arity{\isacharparenleft}{\kern0pt}p{\isacharparenright}{\kern0pt}\ {\isasymle}\ {\isadigit{3}}{\isachardoublequoteclose}\isanewline
\ \ \ \ {\isachardoublequoteopen}A\ {\isasymin}\ M{\isachardoublequoteclose}\ \isanewline
\ \ \ \ {\isachardoublequoteopen}{\isacharparenleft}{\kern0pt}{\isasymAnd}x\ g{\isachardot}{\kern0pt}\ x\ {\isasymin}\ M\ {\isasymLongrightarrow}\ g\ {\isasymin}\ M\ {\isasymLongrightarrow}\ function{\isacharparenleft}{\kern0pt}g{\isacharparenright}{\kern0pt}\ {\isasymLongrightarrow}\ H{\isacharparenleft}{\kern0pt}x{\isacharcomma}{\kern0pt}\ g{\isacharparenright}{\kern0pt}\ {\isasymin}\ M{\isacharparenright}{\kern0pt}{\isachardoublequoteclose}\ \isanewline
\ \ \ \ {\isachardoublequoteopen}\ {\isacharparenleft}{\kern0pt}{\isasymAnd}a{\isadigit{0}}\ a{\isadigit{1}}\ a{\isadigit{2}}\ env{\isachardot}{\kern0pt}\ a{\isadigit{0}}\ {\isasymin}\ M\ {\isasymLongrightarrow}\ a{\isadigit{1}}\ {\isasymin}\ M\ {\isasymLongrightarrow}\ a{\isadigit{2}}\ {\isasymin}\ M\ {\isasymLongrightarrow}\ env\ {\isasymin}\ list{\isacharparenleft}{\kern0pt}M{\isacharparenright}{\kern0pt}\ {\isasymLongrightarrow}\ a{\isadigit{0}}\ {\isacharequal}{\kern0pt}\ H{\isacharparenleft}{\kern0pt}a{\isadigit{2}}{\isacharcomma}{\kern0pt}\ a{\isadigit{1}}{\isacharparenright}{\kern0pt}\ {\isasymlongleftrightarrow}\ sats{\isacharparenleft}{\kern0pt}M{\isacharcomma}{\kern0pt}\ p{\isacharcomma}{\kern0pt}\ {\isacharbrackleft}{\kern0pt}a{\isadigit{0}}{\isacharcomma}{\kern0pt}\ a{\isadigit{1}}{\isacharcomma}{\kern0pt}\ a{\isadigit{2}}{\isacharbrackright}{\kern0pt}\ {\isacharat}{\kern0pt}\ env{\isacharparenright}{\kern0pt}{\isacharparenright}{\kern0pt}{\isachardoublequoteclose}\ \ \isanewline
\ \ \isakeyword{shows}\ {\isachardoublequoteopen}{\isacharbraceleft}{\kern0pt}\ {\isacharless}{\kern0pt}x{\isacharcomma}{\kern0pt}\ wftrec{\isacharparenleft}{\kern0pt}r{\isacharcomma}{\kern0pt}\ x{\isacharcomma}{\kern0pt}\ H{\isacharparenright}{\kern0pt}{\isachargreater}{\kern0pt}{\isachardot}{\kern0pt}\ x\ {\isasymin}\ A\ {\isacharbraceright}{\kern0pt}\ {\isasymin}\ M{\isachardoublequoteclose}\ \isanewline
%
\isadelimproof
%
\endisadelimproof
%
\isatagproof
\isacommand{proof}\isamarkupfalse%
\ {\isacharminus}{\kern0pt}\ \isanewline
\ \ \isacommand{have}\isamarkupfalse%
\ H{\isacharcolon}{\kern0pt}{\isachardoublequoteopen}strong{\isacharunderscore}{\kern0pt}replacement{\isacharparenleft}{\kern0pt}{\isacharhash}{\kern0pt}{\isacharhash}{\kern0pt}M{\isacharcomma}{\kern0pt}\ {\isasymlambda}x\ z{\isachardot}{\kern0pt}\ {\isasymexists}y{\isacharbrackleft}{\kern0pt}{\isacharhash}{\kern0pt}{\isacharhash}{\kern0pt}M{\isacharbrackright}{\kern0pt}{\isachardot}{\kern0pt}\ {\isasymexists}g{\isacharbrackleft}{\kern0pt}{\isacharhash}{\kern0pt}{\isacharhash}{\kern0pt}M{\isacharbrackright}{\kern0pt}{\isachardot}{\kern0pt}\ pair{\isacharparenleft}{\kern0pt}{\isacharhash}{\kern0pt}{\isacharhash}{\kern0pt}M{\isacharcomma}{\kern0pt}\ x{\isacharcomma}{\kern0pt}\ y{\isacharcomma}{\kern0pt}\ z{\isacharparenright}{\kern0pt}\ {\isasymand}\ is{\isacharunderscore}{\kern0pt}recfun{\isacharparenleft}{\kern0pt}r{\isacharcomma}{\kern0pt}\ x{\isacharcomma}{\kern0pt}\ H{\isacharcomma}{\kern0pt}\ g{\isacharparenright}{\kern0pt}\ {\isasymand}\ y\ {\isacharequal}{\kern0pt}\ H{\isacharparenleft}{\kern0pt}x{\isacharcomma}{\kern0pt}\ g{\isacharparenright}{\kern0pt}{\isacharparenright}{\kern0pt}{\isachardoublequoteclose}\isanewline
\ \ \ \ \isacommand{using}\isamarkupfalse%
\ assms\ recfun{\isacharunderscore}{\kern0pt}strong{\isacharunderscore}{\kern0pt}replacement{\isacharunderscore}{\kern0pt}lemma\ \isanewline
\ \ \ \ \isacommand{by}\isamarkupfalse%
\ auto\isanewline
\isanewline
\ \ \isacommand{have}\isamarkupfalse%
\ H{\isadigit{1}}{\isacharcolon}{\kern0pt}\ {\isachardoublequoteopen}strong{\isacharunderscore}{\kern0pt}replacement{\isacharparenleft}{\kern0pt}{\isacharhash}{\kern0pt}{\isacharhash}{\kern0pt}M{\isacharcomma}{\kern0pt}\ {\isasymlambda}x\ z{\isachardot}{\kern0pt}\ z\ {\isacharequal}{\kern0pt}\ {\isacharless}{\kern0pt}x{\isacharcomma}{\kern0pt}\ wftrec{\isacharparenleft}{\kern0pt}r{\isacharcomma}{\kern0pt}\ x{\isacharcomma}{\kern0pt}\ H{\isacharparenright}{\kern0pt}{\isachargreater}{\kern0pt}{\isacharparenright}{\kern0pt}{\isachardoublequoteclose}\ \isanewline
\ \ \ \ \isacommand{apply}\isamarkupfalse%
{\isacharparenleft}{\kern0pt}rule\ iffD{\isadigit{1}}{\isacharcomma}{\kern0pt}\ rule{\isacharunderscore}{\kern0pt}tac\ P{\isacharequal}{\kern0pt}{\isachardoublequoteopen}{\isasymlambda}x\ z{\isachardot}{\kern0pt}\ {\isasymexists}y{\isacharbrackleft}{\kern0pt}{\isacharhash}{\kern0pt}{\isacharhash}{\kern0pt}M{\isacharbrackright}{\kern0pt}{\isachardot}{\kern0pt}\ {\isasymexists}g{\isacharbrackleft}{\kern0pt}{\isacharhash}{\kern0pt}{\isacharhash}{\kern0pt}M{\isacharbrackright}{\kern0pt}{\isachardot}{\kern0pt}\ pair{\isacharparenleft}{\kern0pt}{\isacharhash}{\kern0pt}{\isacharhash}{\kern0pt}M{\isacharcomma}{\kern0pt}\ x{\isacharcomma}{\kern0pt}\ y{\isacharcomma}{\kern0pt}\ z{\isacharparenright}{\kern0pt}\ {\isasymand}\ is{\isacharunderscore}{\kern0pt}recfun{\isacharparenleft}{\kern0pt}r{\isacharcomma}{\kern0pt}\ x{\isacharcomma}{\kern0pt}\ H{\isacharcomma}{\kern0pt}\ g{\isacharparenright}{\kern0pt}\ {\isasymand}\ y\ {\isacharequal}{\kern0pt}\ H{\isacharparenleft}{\kern0pt}x{\isacharcomma}{\kern0pt}\ g{\isacharparenright}{\kern0pt}{\isachardoublequoteclose}\ \isakeyword{in}\ strong{\isacharunderscore}{\kern0pt}replacement{\isacharunderscore}{\kern0pt}cong{\isacharparenright}{\kern0pt}\isanewline
\ \ \ \ \ \isacommand{apply}\isamarkupfalse%
{\isacharparenleft}{\kern0pt}rule\ iffI{\isacharcomma}{\kern0pt}\ clarify{\isacharparenright}{\kern0pt}\isanewline
\ \ \ \ \ \ \isacommand{apply}\isamarkupfalse%
{\isacharparenleft}{\kern0pt}simp\ add{\isacharcolon}{\kern0pt}wftrec{\isacharunderscore}{\kern0pt}def{\isacharparenright}{\kern0pt}\isanewline
\ \ \ \ \ \ \isacommand{apply}\isamarkupfalse%
{\isacharparenleft}{\kern0pt}rename{\isacharunderscore}{\kern0pt}tac\ x\ y\ g{\isacharcomma}{\kern0pt}\ rule{\isacharunderscore}{\kern0pt}tac\ a{\isacharequal}{\kern0pt}g\ \isakeyword{and}\ b{\isacharequal}{\kern0pt}{\isachardoublequoteopen}the{\isacharunderscore}{\kern0pt}recfun{\isacharparenleft}{\kern0pt}r{\isacharcomma}{\kern0pt}\ x{\isacharcomma}{\kern0pt}\ H{\isacharparenright}{\kern0pt}{\isachardoublequoteclose}\ \isakeyword{in}\ ssubst{\isacharparenright}{\kern0pt}\isanewline
\ \ \ \ \ \ \ \isacommand{apply}\isamarkupfalse%
{\isacharparenleft}{\kern0pt}rule\ the{\isacharunderscore}{\kern0pt}recfun{\isacharunderscore}{\kern0pt}eq{\isacharcomma}{\kern0pt}\ simp{\isacharparenright}{\kern0pt}\isanewline
\ \ \ \ \isacommand{using}\isamarkupfalse%
\ assms\isanewline
\ \ \ \ \ \ \ \ \isacommand{apply}\isamarkupfalse%
\ auto{\isacharbrackleft}{\kern0pt}{\isadigit{3}}{\isacharbrackright}{\kern0pt}\isanewline
\ \ \ \ \ \isacommand{apply}\isamarkupfalse%
\ clarify\isanewline
\ \ \ \ \ \isacommand{apply}\isamarkupfalse%
{\isacharparenleft}{\kern0pt}rename{\isacharunderscore}{\kern0pt}tac\ x\ y{\isacharcomma}{\kern0pt}\ rule{\isacharunderscore}{\kern0pt}tac\ x{\isacharequal}{\kern0pt}{\isachardoublequoteopen}wftrec{\isacharparenleft}{\kern0pt}r{\isacharcomma}{\kern0pt}\ x{\isacharcomma}{\kern0pt}\ H{\isacharparenright}{\kern0pt}{\isachardoublequoteclose}\ \isakeyword{in}\ rexI{\isacharparenright}{\kern0pt}\isanewline
\ \ \ \ \ \ \isacommand{apply}\isamarkupfalse%
{\isacharparenleft}{\kern0pt}rename{\isacharunderscore}{\kern0pt}tac\ x\ y{\isacharcomma}{\kern0pt}\ rule{\isacharunderscore}{\kern0pt}tac\ x{\isacharequal}{\kern0pt}{\isachardoublequoteopen}the{\isacharunderscore}{\kern0pt}recfun{\isacharparenleft}{\kern0pt}r{\isacharcomma}{\kern0pt}\ x{\isacharcomma}{\kern0pt}\ H{\isacharparenright}{\kern0pt}{\isachardoublequoteclose}\ \isakeyword{in}\ rexI{\isacharparenright}{\kern0pt}\isanewline
\ \ \ \ \ \ \ \isacommand{apply}\isamarkupfalse%
{\isacharparenleft}{\kern0pt}rule\ conjI{\isacharparenright}{\kern0pt}\isanewline
\ \ \ \ \isacommand{using}\isamarkupfalse%
\ pair{\isacharunderscore}{\kern0pt}abs\ \isacommand{using}\isamarkupfalse%
\ pair{\isacharunderscore}{\kern0pt}in{\isacharunderscore}{\kern0pt}M{\isacharunderscore}{\kern0pt}iff\ \isanewline
\ \ \ \ \ \ \ \ \isacommand{apply}\isamarkupfalse%
\ force\isanewline
\ \ \ \ \ \ \ \isacommand{apply}\isamarkupfalse%
{\isacharparenleft}{\kern0pt}rule\ conjI{\isacharcomma}{\kern0pt}\ rule\ unfold{\isacharunderscore}{\kern0pt}the{\isacharunderscore}{\kern0pt}recfun{\isacharparenright}{\kern0pt}\isanewline
\ \ \ \ \isacommand{using}\isamarkupfalse%
\ assms\ \isanewline
\ \ \ \ \ \ \ \ \ \isacommand{apply}\isamarkupfalse%
\ auto{\isacharbrackleft}{\kern0pt}{\isadigit{2}}{\isacharbrackright}{\kern0pt}\isanewline
\ \ \ \ \ \ \ \isacommand{apply}\isamarkupfalse%
{\isacharparenleft}{\kern0pt}simp\ add{\isacharcolon}{\kern0pt}wftrec{\isacharunderscore}{\kern0pt}def{\isacharparenright}{\kern0pt}\isanewline
\ \ \ \ \ \ \isacommand{apply}\isamarkupfalse%
{\isacharparenleft}{\kern0pt}simp{\isacharcomma}{\kern0pt}\ rule\ the{\isacharunderscore}{\kern0pt}recfun{\isacharunderscore}{\kern0pt}in{\isacharunderscore}{\kern0pt}M{\isacharparenright}{\kern0pt}\isanewline
\ \ \ \ \isacommand{using}\isamarkupfalse%
\ assms\ \isanewline
\ \ \ \ \ \ \ \ \ \ \ \ \ \isacommand{apply}\isamarkupfalse%
\ auto{\isacharbrackleft}{\kern0pt}{\isadigit{8}}{\isacharbrackright}{\kern0pt}\isanewline
\ \ \ \ \ \isacommand{apply}\isamarkupfalse%
\ simp\isanewline
\ \ \ \ \isacommand{using}\isamarkupfalse%
\ H\ \isanewline
\ \ \ \ \isacommand{by}\isamarkupfalse%
\ auto\isanewline
\isanewline
\ \ \isacommand{show}\isamarkupfalse%
\ {\isacharquery}{\kern0pt}thesis\ \isanewline
\ \ \ \ \isacommand{apply}\isamarkupfalse%
{\isacharparenleft}{\kern0pt}rule\ to{\isacharunderscore}{\kern0pt}rin{\isacharcomma}{\kern0pt}\ rule\ RepFun{\isacharunderscore}{\kern0pt}closed{\isacharcomma}{\kern0pt}\ rule\ H{\isadigit{1}}{\isacharparenright}{\kern0pt}\isanewline
\ \ \ \ \isacommand{using}\isamarkupfalse%
\ assms\ \isanewline
\ \ \ \ \ \isacommand{apply}\isamarkupfalse%
\ simp\isanewline
\ \ \ \ \isacommand{apply}\isamarkupfalse%
{\isacharparenleft}{\kern0pt}rule\ ballI{\isacharcomma}{\kern0pt}\ rule\ iffD{\isadigit{2}}{\isacharcomma}{\kern0pt}\ rule\ pair{\isacharunderscore}{\kern0pt}in{\isacharunderscore}{\kern0pt}M{\isacharunderscore}{\kern0pt}iff{\isacharcomma}{\kern0pt}\ rule\ conjI{\isacharparenright}{\kern0pt}\isanewline
\ \ \ \ \isacommand{using}\isamarkupfalse%
\ transM\ assms\isanewline
\ \ \ \ \ \isacommand{apply}\isamarkupfalse%
\ force\isanewline
\ \ \ \ \isacommand{apply}\isamarkupfalse%
{\isacharparenleft}{\kern0pt}simp{\isacharcomma}{\kern0pt}\ rule\ wftrec{\isacharunderscore}{\kern0pt}in{\isacharunderscore}{\kern0pt}M{\isacharparenright}{\kern0pt}\isanewline
\ \ \ \ \isacommand{using}\isamarkupfalse%
\ assms\ transM\ \isanewline
\ \ \ \ \isacommand{by}\isamarkupfalse%
\ auto\isanewline
\isacommand{qed}\isamarkupfalse%
%
\endisatagproof
{\isafoldproof}%
%
\isadelimproof
\isanewline
%
\endisadelimproof
\isanewline
\isacommand{lemma}\isamarkupfalse%
\ sats{\isacharunderscore}{\kern0pt}is{\isacharunderscore}{\kern0pt}wfrec{\isacharunderscore}{\kern0pt}fm{\isacharunderscore}{\kern0pt}iff\ {\isacharcolon}{\kern0pt}\ \isanewline
\ \ \isakeyword{fixes}\ H\ Hfm\ r\ x\ v\ i\ j\ k\ env\ \isanewline
\ \ \isakeyword{assumes}\ {\isachardoublequoteopen}env\ {\isasymin}\ list{\isacharparenleft}{\kern0pt}M{\isacharparenright}{\kern0pt}{\isachardoublequoteclose}\ {\isachardoublequoteopen}nth{\isacharparenleft}{\kern0pt}i{\isacharcomma}{\kern0pt}\ env{\isacharparenright}{\kern0pt}\ {\isacharequal}{\kern0pt}\ r{\isachardoublequoteclose}\ {\isachardoublequoteopen}nth{\isacharparenleft}{\kern0pt}j{\isacharcomma}{\kern0pt}\ env{\isacharparenright}{\kern0pt}\ {\isacharequal}{\kern0pt}\ x{\isachardoublequoteclose}\ {\isachardoublequoteopen}nth{\isacharparenleft}{\kern0pt}k{\isacharcomma}{\kern0pt}\ env{\isacharparenright}{\kern0pt}\ {\isacharequal}{\kern0pt}\ v{\isachardoublequoteclose}\ \ \isanewline
\ \ \ \ \ \ \ \ \ \ {\isachardoublequoteopen}i\ {\isasymin}\ nat{\isachardoublequoteclose}\ {\isachardoublequoteopen}j\ {\isasymin}\ nat{\isachardoublequoteclose}\ {\isachardoublequoteopen}k\ {\isasymin}\ nat{\isachardoublequoteclose}\ \isanewline
\ \ \ \ \ \ \ \ \ \ {\isachardoublequoteopen}i\ {\isacharless}{\kern0pt}\ length{\isacharparenleft}{\kern0pt}env{\isacharparenright}{\kern0pt}{\isachardoublequoteclose}\ {\isachardoublequoteopen}j\ {\isacharless}{\kern0pt}\ length{\isacharparenleft}{\kern0pt}env{\isacharparenright}{\kern0pt}{\isachardoublequoteclose}\ {\isachardoublequoteopen}k\ {\isacharless}{\kern0pt}\ length{\isacharparenleft}{\kern0pt}env{\isacharparenright}{\kern0pt}{\isachardoublequoteclose}\ \isanewline
\ \ \ \ \ \ \ \ \ \ {\isachardoublequoteopen}x\ {\isasymin}\ M{\isachardoublequoteclose}\ {\isachardoublequoteopen}v\ {\isasymin}\ M{\isachardoublequoteclose}\ {\isachardoublequoteopen}r\ {\isasymin}\ M{\isachardoublequoteclose}\ \isanewline
\ \ \ \ \ \ \ \ \ \ {\isachardoublequoteopen}Hfm\ {\isasymin}\ formula{\isachardoublequoteclose}\ {\isachardoublequoteopen}arity{\isacharparenleft}{\kern0pt}Hfm{\isacharparenright}{\kern0pt}\ {\isasymle}\ {\isadigit{3}}{\isachardoublequoteclose}\isanewline
\ \ \ \ \ \ \ \ \ \ {\isachardoublequoteopen}{\isacharparenleft}{\kern0pt}{\isasymAnd}x\ g{\isachardot}{\kern0pt}\ x\ {\isasymin}\ M\ {\isasymLongrightarrow}\ g\ {\isasymin}\ M\ {\isasymLongrightarrow}\ function{\isacharparenleft}{\kern0pt}g{\isacharparenright}{\kern0pt}\ {\isasymLongrightarrow}\ H{\isacharparenleft}{\kern0pt}x{\isacharcomma}{\kern0pt}\ g{\isacharparenright}{\kern0pt}\ {\isasymin}\ M{\isacharparenright}{\kern0pt}{\isachardoublequoteclose}\isanewline
\ \ \ \ \ \ \ \ \ \ {\isachardoublequoteopen}wf{\isacharparenleft}{\kern0pt}r{\isacharparenright}{\kern0pt}{\isachardoublequoteclose}\ {\isachardoublequoteopen}trans{\isacharparenleft}{\kern0pt}r{\isacharparenright}{\kern0pt}{\isachardoublequoteclose}\ \isanewline
\ \ \isakeyword{and}\ HH{\isacharcolon}{\kern0pt}\ {\isachardoublequoteopen}{\isacharparenleft}{\kern0pt}{\isasymAnd}a{\isadigit{0}}\ a{\isadigit{1}}\ a{\isadigit{2}}\ env{\isachardot}{\kern0pt}\ a{\isadigit{0}}\ {\isasymin}\ M\ {\isasymLongrightarrow}\ a{\isadigit{1}}\ {\isasymin}\ M\ {\isasymLongrightarrow}\ a{\isadigit{2}}\ {\isasymin}\ M\ {\isasymLongrightarrow}\ env\ {\isasymin}\ list{\isacharparenleft}{\kern0pt}M{\isacharparenright}{\kern0pt}\ \isanewline
\ \ \ \ \ \ \ \ \ \ \ \ \ \ {\isasymLongrightarrow}\ a{\isadigit{0}}\ {\isacharequal}{\kern0pt}\ H{\isacharparenleft}{\kern0pt}a{\isadigit{2}}{\isacharcomma}{\kern0pt}\ a{\isadigit{1}}{\isacharparenright}{\kern0pt}\ {\isasymlongleftrightarrow}\ sats{\isacharparenleft}{\kern0pt}M{\isacharcomma}{\kern0pt}\ Hfm{\isacharcomma}{\kern0pt}\ {\isacharbrackleft}{\kern0pt}a{\isadigit{0}}{\isacharcomma}{\kern0pt}\ a{\isadigit{1}}{\isacharcomma}{\kern0pt}\ a{\isadigit{2}}{\isacharbrackright}{\kern0pt}\ {\isacharat}{\kern0pt}\ env{\isacharparenright}{\kern0pt}{\isacharparenright}{\kern0pt}{\isachardoublequoteclose}\isanewline
\ \ \isakeyword{shows}\ {\isachardoublequoteopen}sats{\isacharparenleft}{\kern0pt}M{\isacharcomma}{\kern0pt}\ is{\isacharunderscore}{\kern0pt}wfrec{\isacharunderscore}{\kern0pt}fm{\isacharparenleft}{\kern0pt}Hfm{\isacharcomma}{\kern0pt}\ i{\isacharcomma}{\kern0pt}\ j{\isacharcomma}{\kern0pt}\ k{\isacharparenright}{\kern0pt}{\isacharcomma}{\kern0pt}\ env{\isacharparenright}{\kern0pt}\ {\isasymlongleftrightarrow}\ v\ {\isacharequal}{\kern0pt}\ wftrec{\isacharparenleft}{\kern0pt}r{\isacharcomma}{\kern0pt}\ x{\isacharcomma}{\kern0pt}\ H{\isacharparenright}{\kern0pt}{\isachardoublequoteclose}\ \ \isanewline
%
\isadelimproof
%
\endisadelimproof
%
\isatagproof
\isacommand{proof}\isamarkupfalse%
\ {\isacharminus}{\kern0pt}\isanewline
\ \ \isacommand{have}\isamarkupfalse%
\ iff{\isacharunderscore}{\kern0pt}helper\ {\isacharcolon}{\kern0pt}\ {\isachardoublequoteopen}{\isacharparenleft}{\kern0pt}{\isasymexists}g{\isacharbrackleft}{\kern0pt}{\isacharhash}{\kern0pt}{\isacharhash}{\kern0pt}M{\isacharbrackright}{\kern0pt}{\isachardot}{\kern0pt}\ is{\isacharunderscore}{\kern0pt}recfun{\isacharparenleft}{\kern0pt}r{\isacharcomma}{\kern0pt}\ x{\isacharcomma}{\kern0pt}\ H{\isacharcomma}{\kern0pt}\ g{\isacharparenright}{\kern0pt}\ {\isasymand}\ v\ {\isacharequal}{\kern0pt}\ H{\isacharparenleft}{\kern0pt}x{\isacharcomma}{\kern0pt}\ g{\isacharparenright}{\kern0pt}{\isacharparenright}{\kern0pt}\ {\isasymlongleftrightarrow}\ v\ {\isacharequal}{\kern0pt}\ wftrec{\isacharparenleft}{\kern0pt}r{\isacharcomma}{\kern0pt}\ x{\isacharcomma}{\kern0pt}\ H{\isacharparenright}{\kern0pt}{\isachardoublequoteclose}\isanewline
\ \ \isacommand{proof}\isamarkupfalse%
\ {\isacharparenleft}{\kern0pt}rule\ iffI{\isacharparenright}{\kern0pt}\isanewline
\ \ \ \ \isacommand{assume}\isamarkupfalse%
\ {\isachardoublequoteopen}{\isasymexists}g{\isacharbrackleft}{\kern0pt}{\isacharhash}{\kern0pt}{\isacharhash}{\kern0pt}M{\isacharbrackright}{\kern0pt}{\isachardot}{\kern0pt}\ is{\isacharunderscore}{\kern0pt}recfun{\isacharparenleft}{\kern0pt}r{\isacharcomma}{\kern0pt}\ x{\isacharcomma}{\kern0pt}\ H{\isacharcomma}{\kern0pt}\ g{\isacharparenright}{\kern0pt}\ {\isasymand}\ v\ {\isacharequal}{\kern0pt}\ H{\isacharparenleft}{\kern0pt}x{\isacharcomma}{\kern0pt}\ g{\isacharparenright}{\kern0pt}{\isachardoublequoteclose}\ \isanewline
\ \ \ \ \isacommand{then}\isamarkupfalse%
\ \isacommand{obtain}\isamarkupfalse%
\ g\ \isakeyword{where}\ gH\ {\isacharcolon}{\kern0pt}\ {\isachardoublequoteopen}g\ {\isasymin}\ M{\isachardoublequoteclose}\ {\isachardoublequoteopen}is{\isacharunderscore}{\kern0pt}recfun{\isacharparenleft}{\kern0pt}r{\isacharcomma}{\kern0pt}\ x{\isacharcomma}{\kern0pt}\ H{\isacharcomma}{\kern0pt}\ g{\isacharparenright}{\kern0pt}{\isachardoublequoteclose}\ {\isachardoublequoteopen}v\ {\isacharequal}{\kern0pt}\ H{\isacharparenleft}{\kern0pt}x{\isacharcomma}{\kern0pt}\ g{\isacharparenright}{\kern0pt}{\isachardoublequoteclose}\ \isacommand{by}\isamarkupfalse%
\ auto\isanewline
\ \ \ \ \isacommand{have}\isamarkupfalse%
\ {\isachardoublequoteopen}the{\isacharunderscore}{\kern0pt}recfun{\isacharparenleft}{\kern0pt}r{\isacharcomma}{\kern0pt}\ x{\isacharcomma}{\kern0pt}\ H{\isacharparenright}{\kern0pt}\ {\isacharequal}{\kern0pt}\ g{\isachardoublequoteclose}\ \isanewline
\ \ \ \ \ \ \isacommand{apply}\isamarkupfalse%
{\isacharparenleft}{\kern0pt}rule\ the{\isacharunderscore}{\kern0pt}recfun{\isacharunderscore}{\kern0pt}eq{\isacharparenright}{\kern0pt}\isanewline
\ \ \ \ \ \ \isacommand{using}\isamarkupfalse%
\ assms\ gH\ \isanewline
\ \ \ \ \ \ \isacommand{by}\isamarkupfalse%
\ auto\isanewline
\ \ \ \ \isacommand{then}\isamarkupfalse%
\ \isacommand{show}\isamarkupfalse%
\ {\isachardoublequoteopen}v\ {\isacharequal}{\kern0pt}\ wftrec{\isacharparenleft}{\kern0pt}r{\isacharcomma}{\kern0pt}\ x{\isacharcomma}{\kern0pt}\ H{\isacharparenright}{\kern0pt}{\isachardoublequoteclose}\ \isanewline
\ \ \ \ \ \ \isacommand{unfolding}\isamarkupfalse%
\ wftrec{\isacharunderscore}{\kern0pt}def\ \isanewline
\ \ \ \ \ \ \isacommand{using}\isamarkupfalse%
\ gH\ \isanewline
\ \ \ \ \ \ \isacommand{by}\isamarkupfalse%
\ auto\ \isanewline
\ \ \isacommand{next}\isamarkupfalse%
\ \isanewline
\ \ \ \ \isacommand{assume}\isamarkupfalse%
\ assms{\isadigit{1}}{\isacharcolon}{\kern0pt}\ {\isachardoublequoteopen}v\ {\isacharequal}{\kern0pt}\ wftrec{\isacharparenleft}{\kern0pt}r{\isacharcomma}{\kern0pt}\ x{\isacharcomma}{\kern0pt}\ H{\isacharparenright}{\kern0pt}{\isachardoublequoteclose}\ \isanewline
\ \ \ \ \isacommand{have}\isamarkupfalse%
\ {\isachardoublequoteopen}is{\isacharunderscore}{\kern0pt}recfun{\isacharparenleft}{\kern0pt}r{\isacharcomma}{\kern0pt}\ x{\isacharcomma}{\kern0pt}\ H{\isacharcomma}{\kern0pt}\ the{\isacharunderscore}{\kern0pt}recfun{\isacharparenleft}{\kern0pt}r{\isacharcomma}{\kern0pt}\ x{\isacharcomma}{\kern0pt}\ H{\isacharparenright}{\kern0pt}{\isacharparenright}{\kern0pt}{\isachardoublequoteclose}\ \isanewline
\ \ \ \ \ \ \isacommand{apply}\isamarkupfalse%
{\isacharparenleft}{\kern0pt}rule\ unfold{\isacharunderscore}{\kern0pt}the{\isacharunderscore}{\kern0pt}recfun{\isacharparenright}{\kern0pt}\isanewline
\ \ \ \ \ \ \isacommand{using}\isamarkupfalse%
\ assms\ \isanewline
\ \ \ \ \ \ \isacommand{by}\isamarkupfalse%
\ auto\isanewline
\ \ \ \ \isacommand{then}\isamarkupfalse%
\ \isacommand{show}\isamarkupfalse%
\ {\isachardoublequoteopen}{\isacharparenleft}{\kern0pt}{\isasymexists}g{\isacharbrackleft}{\kern0pt}{\isacharhash}{\kern0pt}{\isacharhash}{\kern0pt}M{\isacharbrackright}{\kern0pt}{\isachardot}{\kern0pt}\ is{\isacharunderscore}{\kern0pt}recfun{\isacharparenleft}{\kern0pt}r{\isacharcomma}{\kern0pt}\ x{\isacharcomma}{\kern0pt}\ H{\isacharcomma}{\kern0pt}\ g{\isacharparenright}{\kern0pt}\ {\isasymand}\ v\ {\isacharequal}{\kern0pt}\ H{\isacharparenleft}{\kern0pt}x{\isacharcomma}{\kern0pt}\ g{\isacharparenright}{\kern0pt}{\isacharparenright}{\kern0pt}{\isachardoublequoteclose}\ \isanewline
\ \ \ \ \ \ \isacommand{apply}\isamarkupfalse%
\ simp\ \isanewline
\ \ \ \ \ \ \isacommand{apply}\isamarkupfalse%
{\isacharparenleft}{\kern0pt}rule{\isacharunderscore}{\kern0pt}tac\ x{\isacharequal}{\kern0pt}{\isachardoublequoteopen}the{\isacharunderscore}{\kern0pt}recfun{\isacharparenleft}{\kern0pt}r{\isacharcomma}{\kern0pt}\ x{\isacharcomma}{\kern0pt}\ H{\isacharparenright}{\kern0pt}{\isachardoublequoteclose}\ \isakeyword{in}\ bexI{\isacharcomma}{\kern0pt}\ simp{\isacharparenright}{\kern0pt}\isanewline
\ \ \ \ \ \ \isacommand{using}\isamarkupfalse%
\ assms{\isadigit{1}}\ \isanewline
\ \ \ \ \ \ \isacommand{unfolding}\isamarkupfalse%
\ wftrec{\isacharunderscore}{\kern0pt}def\ \isanewline
\ \ \ \ \ \ \ \isacommand{apply}\isamarkupfalse%
\ simp\isanewline
\ \ \ \ \ \ \isacommand{apply}\isamarkupfalse%
{\isacharparenleft}{\kern0pt}rule{\isacharunderscore}{\kern0pt}tac\ p{\isacharequal}{\kern0pt}Hfm\ \isakeyword{in}\ the{\isacharunderscore}{\kern0pt}recfun{\isacharunderscore}{\kern0pt}in{\isacharunderscore}{\kern0pt}M{\isacharparenright}{\kern0pt}\isanewline
\ \ \ \ \ \ \isacommand{using}\isamarkupfalse%
\ assms\ assms{\isadigit{1}}\isanewline
\ \ \ \ \ \ \isacommand{by}\isamarkupfalse%
\ simp{\isacharunderscore}{\kern0pt}all\isanewline
\ \ \isacommand{qed}\isamarkupfalse%
\isanewline
\isanewline
\ \ \isacommand{show}\isamarkupfalse%
\ {\isachardoublequoteopen}sats{\isacharparenleft}{\kern0pt}M{\isacharcomma}{\kern0pt}\ is{\isacharunderscore}{\kern0pt}wfrec{\isacharunderscore}{\kern0pt}fm{\isacharparenleft}{\kern0pt}Hfm{\isacharcomma}{\kern0pt}\ i{\isacharcomma}{\kern0pt}\ j{\isacharcomma}{\kern0pt}\ k{\isacharparenright}{\kern0pt}{\isacharcomma}{\kern0pt}\ env{\isacharparenright}{\kern0pt}\ {\isasymlongleftrightarrow}\ v\ {\isacharequal}{\kern0pt}\ wftrec{\isacharparenleft}{\kern0pt}r{\isacharcomma}{\kern0pt}\ x{\isacharcomma}{\kern0pt}\ H{\isacharparenright}{\kern0pt}{\isachardoublequoteclose}\ \isanewline
\ \ \ \ \isacommand{apply}\isamarkupfalse%
{\isacharparenleft}{\kern0pt}rule{\isacharunderscore}{\kern0pt}tac\ Q{\isacharequal}{\kern0pt}{\isachardoublequoteopen}is{\isacharunderscore}{\kern0pt}wfrec{\isacharparenleft}{\kern0pt}{\isacharhash}{\kern0pt}{\isacharhash}{\kern0pt}M{\isacharcomma}{\kern0pt}\ {\isasymlambda}a{\isadigit{0}}\ a{\isadigit{1}}\ a{\isadigit{2}}{\isachardot}{\kern0pt}\ a{\isadigit{2}}\ {\isacharequal}{\kern0pt}\ H{\isacharparenleft}{\kern0pt}a{\isadigit{0}}{\isacharcomma}{\kern0pt}\ a{\isadigit{1}}{\isacharparenright}{\kern0pt}{\isacharcomma}{\kern0pt}\ r{\isacharcomma}{\kern0pt}\ x{\isacharcomma}{\kern0pt}\ v{\isacharparenright}{\kern0pt}{\isachardoublequoteclose}\ \isakeyword{in}\ iff{\isacharunderscore}{\kern0pt}trans{\isacharcomma}{\kern0pt}\ rule\ iff{\isacharunderscore}{\kern0pt}flip{\isacharparenright}{\kern0pt}\isanewline
\ \ \ \ \ \isacommand{apply}\isamarkupfalse%
{\isacharparenleft}{\kern0pt}rule\ is{\isacharunderscore}{\kern0pt}wfrec{\isacharunderscore}{\kern0pt}iff{\isacharunderscore}{\kern0pt}sats{\isacharparenright}{\kern0pt}\isanewline
\ \ \ \ \ \ \ \ \ \ \ \ \isacommand{apply}\isamarkupfalse%
{\isacharparenleft}{\kern0pt}rename{\isacharunderscore}{\kern0pt}tac\ a{\isadigit{0}}\ a{\isadigit{1}}\ a{\isadigit{2}}\ a{\isadigit{3}}\ a{\isadigit{4}}{\isacharcomma}{\kern0pt}\ rule{\isacharunderscore}{\kern0pt}tac\ b{\isacharequal}{\kern0pt}{\isachardoublequoteopen}Cons{\isacharparenleft}{\kern0pt}a{\isadigit{0}}{\isacharcomma}{\kern0pt}\ Cons{\isacharparenleft}{\kern0pt}a{\isadigit{1}}{\isacharcomma}{\kern0pt}\ Cons{\isacharparenleft}{\kern0pt}a{\isadigit{2}}{\isacharcomma}{\kern0pt}\ Cons{\isacharparenleft}{\kern0pt}a{\isadigit{3}}{\isacharcomma}{\kern0pt}\ Cons{\isacharparenleft}{\kern0pt}a{\isadigit{4}}{\isacharcomma}{\kern0pt}\ env{\isacharparenright}{\kern0pt}{\isacharparenright}{\kern0pt}{\isacharparenright}{\kern0pt}{\isacharparenright}{\kern0pt}{\isacharparenright}{\kern0pt}{\isachardoublequoteclose}\ \isakeyword{and}\ a{\isacharequal}{\kern0pt}{\isachardoublequoteopen}{\isacharbrackleft}{\kern0pt}a{\isadigit{0}}{\isacharcomma}{\kern0pt}\ a{\isadigit{1}}{\isacharcomma}{\kern0pt}\ a{\isadigit{2}}{\isacharbrackright}{\kern0pt}\ {\isacharat}{\kern0pt}\ {\isacharparenleft}{\kern0pt}{\isacharbrackleft}{\kern0pt}a{\isadigit{3}}{\isacharcomma}{\kern0pt}\ a{\isadigit{4}}{\isacharbrackright}{\kern0pt}\ {\isacharat}{\kern0pt}\ env{\isacharparenright}{\kern0pt}{\isachardoublequoteclose}\ \isakeyword{in}\ ssubst{\isacharparenright}{\kern0pt}\isanewline
\ \ \ \ \ \ \ \ \ \ \ \ \ \isacommand{apply}\isamarkupfalse%
\ simp\isanewline
\ \ \ \ \ \ \ \ \ \ \ \ \isacommand{apply}\isamarkupfalse%
{\isacharparenleft}{\kern0pt}rule\ HH{\isacharparenright}{\kern0pt}\isanewline
\ \ \ \ \isacommand{using}\isamarkupfalse%
\ assms\isanewline
\ \ \ \ \ \ \ \ \ \ \ \ \ \ \ \isacommand{apply}\isamarkupfalse%
{\isacharparenleft}{\kern0pt}simp{\isacharunderscore}{\kern0pt}all{\isacharparenright}{\kern0pt}\isanewline
\ \ \ \ \isacommand{apply}\isamarkupfalse%
{\isacharparenleft}{\kern0pt}rule\ iff{\isacharunderscore}{\kern0pt}trans{\isacharcomma}{\kern0pt}\ rule{\isacharunderscore}{\kern0pt}tac\ H{\isacharequal}{\kern0pt}H\ \isakeyword{in}\ is{\isacharunderscore}{\kern0pt}wfrec{\isacharunderscore}{\kern0pt}abs{\isacharparenright}{\kern0pt}\isanewline
\ \ \ \ \isacommand{using}\isamarkupfalse%
\ assms\ \isanewline
\ \ \ \ \ \ \ \ \ \isacommand{apply}\isamarkupfalse%
\ force\isanewline
\ \ \ \ \isacommand{unfolding}\isamarkupfalse%
\ relation{\isadigit{2}}{\isacharunderscore}{\kern0pt}def\ \isanewline
\ \ \ \ \ \ \ \ \isacommand{apply}\isamarkupfalse%
\ {\isacharparenleft}{\kern0pt}simp{\isacharcomma}{\kern0pt}\ simp{\isacharcomma}{\kern0pt}\ simp{\isacharcomma}{\kern0pt}\ simp{\isacharparenright}{\kern0pt}\isanewline
\ \ \ \ \isacommand{by}\isamarkupfalse%
{\isacharparenleft}{\kern0pt}rule\ iff{\isacharunderscore}{\kern0pt}helper{\isacharparenright}{\kern0pt}\isanewline
\isacommand{qed}\isamarkupfalse%
%
\endisatagproof
{\isafoldproof}%
%
\isadelimproof
\ \isanewline
%
\endisadelimproof
\isanewline
\isacommand{definition}\isamarkupfalse%
\ preds\ \isakeyword{where}\ {\isachardoublequoteopen}preds{\isacharparenleft}{\kern0pt}R{\isacharcomma}{\kern0pt}\ x{\isacharparenright}{\kern0pt}\ {\isasymequiv}\ {\isacharbraceleft}{\kern0pt}\ y\ {\isasymin}\ M{\isachardot}{\kern0pt}\ R{\isacharparenleft}{\kern0pt}y{\isacharcomma}{\kern0pt}\ x{\isacharparenright}{\kern0pt}\ {\isacharbraceright}{\kern0pt}{\isachardoublequoteclose}\ \isanewline
\isacommand{definition}\isamarkupfalse%
\ preds{\isacharunderscore}{\kern0pt}rel\ \isakeyword{where}\ {\isachardoublequoteopen}preds{\isacharunderscore}{\kern0pt}rel{\isacharparenleft}{\kern0pt}R{\isacharcomma}{\kern0pt}\ x{\isacharparenright}{\kern0pt}\ {\isasymequiv}\ {\isacharbraceleft}{\kern0pt}\ {\isacharless}{\kern0pt}z{\isacharcomma}{\kern0pt}\ y{\isachargreater}{\kern0pt}\ {\isasymin}\ preds{\isacharparenleft}{\kern0pt}R{\isacharcomma}{\kern0pt}\ x{\isacharparenright}{\kern0pt}\ {\isasymtimes}\ {\isacharparenleft}{\kern0pt}preds{\isacharparenleft}{\kern0pt}R{\isacharcomma}{\kern0pt}\ x{\isacharparenright}{\kern0pt}\ {\isasymunion}\ {\isacharbraceleft}{\kern0pt}x{\isacharbraceright}{\kern0pt}{\isacharparenright}{\kern0pt}{\isachardot}{\kern0pt}\ R{\isacharparenleft}{\kern0pt}z{\isacharcomma}{\kern0pt}\ y{\isacharparenright}{\kern0pt}\ {\isacharbraceright}{\kern0pt}{\isachardoublequoteclose}\ \isanewline
\isanewline
\isacommand{lemma}\isamarkupfalse%
\ preds{\isacharunderscore}{\kern0pt}rel{\isacharunderscore}{\kern0pt}subset\ {\isacharcolon}{\kern0pt}\ {\isachardoublequoteopen}x\ {\isasymin}\ M\ {\isasymLongrightarrow}\ \ preds{\isacharunderscore}{\kern0pt}rel{\isacharparenleft}{\kern0pt}R{\isacharcomma}{\kern0pt}\ x{\isacharparenright}{\kern0pt}\ {\isasymsubseteq}\ Rrel{\isacharparenleft}{\kern0pt}R{\isacharcomma}{\kern0pt}\ M{\isacharparenright}{\kern0pt}{\isachardoublequoteclose}\ \isanewline
%
\isadelimproof
\ \ %
\endisadelimproof
%
\isatagproof
\isacommand{unfolding}\isamarkupfalse%
\ preds{\isacharunderscore}{\kern0pt}rel{\isacharunderscore}{\kern0pt}def\ Rrel{\isacharunderscore}{\kern0pt}def\ preds{\isacharunderscore}{\kern0pt}def\ \isacommand{by}\isamarkupfalse%
\ auto%
\endisatagproof
{\isafoldproof}%
%
\isadelimproof
\isanewline
%
\endisadelimproof
\isanewline
\isacommand{lemma}\isamarkupfalse%
\ preds{\isacharunderscore}{\kern0pt}rel{\isacharunderscore}{\kern0pt}subset{\isacharprime}{\kern0pt}\ {\isacharcolon}{\kern0pt}\ {\isachardoublequoteopen}x\ {\isasymin}\ M\ {\isasymLongrightarrow}\ {\isacharless}{\kern0pt}z{\isacharcomma}{\kern0pt}\ y{\isachargreater}{\kern0pt}\ {\isasymin}\ preds{\isacharunderscore}{\kern0pt}rel{\isacharparenleft}{\kern0pt}R{\isacharcomma}{\kern0pt}\ x{\isacharparenright}{\kern0pt}\ {\isasymLongrightarrow}\ {\isacharless}{\kern0pt}z{\isacharcomma}{\kern0pt}\ y{\isachargreater}{\kern0pt}\ {\isasymin}\ Rrel{\isacharparenleft}{\kern0pt}R{\isacharcomma}{\kern0pt}\ M{\isacharparenright}{\kern0pt}{\isachardoublequoteclose}\ \isanewline
%
\isadelimproof
\ \ %
\endisadelimproof
%
\isatagproof
\isacommand{unfolding}\isamarkupfalse%
\ preds{\isacharunderscore}{\kern0pt}rel{\isacharunderscore}{\kern0pt}def\ Rrel{\isacharunderscore}{\kern0pt}def\ preds{\isacharunderscore}{\kern0pt}def\ \isacommand{by}\isamarkupfalse%
\ auto%
\endisatagproof
{\isafoldproof}%
%
\isadelimproof
\isanewline
%
\endisadelimproof
\isanewline
\isacommand{definition}\isamarkupfalse%
\ Relation{\isacharunderscore}{\kern0pt}fm\ \isakeyword{where}\ {\isachardoublequoteopen}Relation{\isacharunderscore}{\kern0pt}fm{\isacharparenleft}{\kern0pt}R{\isacharcomma}{\kern0pt}\ fm{\isacharparenright}{\kern0pt}\ {\isasymequiv}\ fm\ {\isasymin}\ formula\ {\isasymand}\ arity{\isacharparenleft}{\kern0pt}fm{\isacharparenright}{\kern0pt}\ {\isacharequal}{\kern0pt}\ {\isadigit{2}}\ {\isasymand}\ {\isacharparenleft}{\kern0pt}{\isasymforall}env\ {\isasymin}\ list{\isacharparenleft}{\kern0pt}M{\isacharparenright}{\kern0pt}{\isachardot}{\kern0pt}\ {\isasymforall}x\ {\isasymin}\ M{\isachardot}{\kern0pt}\ {\isasymforall}y\ {\isasymin}\ M{\isachardot}{\kern0pt}\ R{\isacharparenleft}{\kern0pt}y{\isacharcomma}{\kern0pt}\ x{\isacharparenright}{\kern0pt}\ {\isasymlongleftrightarrow}\ sats{\isacharparenleft}{\kern0pt}M{\isacharcomma}{\kern0pt}\ fm{\isacharcomma}{\kern0pt}\ {\isacharbrackleft}{\kern0pt}y{\isacharcomma}{\kern0pt}\ x{\isacharbrackright}{\kern0pt}\ {\isacharat}{\kern0pt}\ env{\isacharparenright}{\kern0pt}{\isacharparenright}{\kern0pt}{\isachardoublequoteclose}\ \isanewline
\isanewline
\isacommand{lemma}\isamarkupfalse%
\ Relation{\isacharunderscore}{\kern0pt}fmD\ {\isacharcolon}{\kern0pt}\ {\isachardoublequoteopen}{\isasymAnd}R\ Rfm\ env\ a\ b{\isachardot}{\kern0pt}\ Relation{\isacharunderscore}{\kern0pt}fm{\isacharparenleft}{\kern0pt}R{\isacharcomma}{\kern0pt}\ Rfm{\isacharparenright}{\kern0pt}\ {\isasymLongrightarrow}\ a\ {\isasymin}\ M\ {\isasymLongrightarrow}\ b\ {\isasymin}\ M\ {\isasymLongrightarrow}\ env\ {\isasymin}\ list{\isacharparenleft}{\kern0pt}M{\isacharparenright}{\kern0pt}\ {\isasymLongrightarrow}\ \ R{\isacharparenleft}{\kern0pt}a{\isacharcomma}{\kern0pt}\ b{\isacharparenright}{\kern0pt}\ {\isasymlongleftrightarrow}\ M{\isacharcomma}{\kern0pt}\ {\isacharbrackleft}{\kern0pt}a{\isacharcomma}{\kern0pt}\ b{\isacharbrackright}{\kern0pt}\ {\isacharat}{\kern0pt}\ env\ {\isasymTurnstile}\ Rfm{\isachardoublequoteclose}\ \isanewline
%
\isadelimproof
\ \ %
\endisadelimproof
%
\isatagproof
\isacommand{using}\isamarkupfalse%
\ Relation{\isacharunderscore}{\kern0pt}fm{\isacharunderscore}{\kern0pt}def\ \isacommand{by}\isamarkupfalse%
\ auto%
\endisatagproof
{\isafoldproof}%
%
\isadelimproof
\isanewline
%
\endisadelimproof
\isanewline
\isacommand{end}\isamarkupfalse%
\isanewline
\isanewline
\isacommand{definition}\isamarkupfalse%
\ is{\isacharunderscore}{\kern0pt}preds{\isacharunderscore}{\kern0pt}fm{\isacharunderscore}{\kern0pt}ren\ \isakeyword{where}\ {\isachardoublequoteopen}is{\isacharunderscore}{\kern0pt}preds{\isacharunderscore}{\kern0pt}fm{\isacharunderscore}{\kern0pt}ren{\isacharparenleft}{\kern0pt}x{\isacharparenright}{\kern0pt}\ \ {\isasymequiv}\ {\isacharbraceleft}{\kern0pt}\ {\isacharless}{\kern0pt}{\isadigit{0}}{\isacharcomma}{\kern0pt}\ {\isadigit{0}}{\isachargreater}{\kern0pt}\ {\isacharcomma}{\kern0pt}\ {\isacharless}{\kern0pt}{\isadigit{1}}{\isacharcomma}{\kern0pt}\ x\ {\isacharhash}{\kern0pt}{\isacharplus}{\kern0pt}\ {\isadigit{1}}{\isachargreater}{\kern0pt}\ {\isacharbraceright}{\kern0pt}{\isachardoublequoteclose}\ \isanewline
\isacommand{definition}\isamarkupfalse%
\ is{\isacharunderscore}{\kern0pt}preds{\isacharunderscore}{\kern0pt}fm{\isacharunderscore}{\kern0pt}Rfm{\isacharunderscore}{\kern0pt}ren\ \isakeyword{where}\ {\isachardoublequoteopen}is{\isacharunderscore}{\kern0pt}preds{\isacharunderscore}{\kern0pt}fm{\isacharunderscore}{\kern0pt}Rfm{\isacharunderscore}{\kern0pt}ren{\isacharparenleft}{\kern0pt}Rfm{\isacharcomma}{\kern0pt}\ x{\isacharparenright}{\kern0pt}\ {\isasymequiv}\ ren{\isacharparenleft}{\kern0pt}Rfm{\isacharparenright}{\kern0pt}\ {\isacharbackquote}{\kern0pt}\ {\isadigit{2}}{\isacharbackquote}{\kern0pt}\ {\isacharparenleft}{\kern0pt}x\ {\isacharhash}{\kern0pt}{\isacharplus}{\kern0pt}\ {\isadigit{2}}{\isacharparenright}{\kern0pt}{\isacharbackquote}{\kern0pt}\ is{\isacharunderscore}{\kern0pt}preds{\isacharunderscore}{\kern0pt}fm{\isacharunderscore}{\kern0pt}ren{\isacharparenleft}{\kern0pt}x{\isacharparenright}{\kern0pt}{\isachardoublequoteclose}\ \isanewline
\isanewline
\isacommand{definition}\isamarkupfalse%
\ is{\isacharunderscore}{\kern0pt}preds{\isacharunderscore}{\kern0pt}fm\ \isakeyword{where}\ {\isachardoublequoteopen}is{\isacharunderscore}{\kern0pt}preds{\isacharunderscore}{\kern0pt}fm{\isacharparenleft}{\kern0pt}Rfm{\isacharcomma}{\kern0pt}\ x{\isacharcomma}{\kern0pt}\ S{\isacharparenright}{\kern0pt}\ {\isasymequiv}\ Forall{\isacharparenleft}{\kern0pt}Iff{\isacharparenleft}{\kern0pt}Member{\isacharparenleft}{\kern0pt}{\isadigit{0}}{\isacharcomma}{\kern0pt}\ S\ {\isacharhash}{\kern0pt}{\isacharplus}{\kern0pt}\ {\isadigit{1}}{\isacharparenright}{\kern0pt}{\isacharcomma}{\kern0pt}\ is{\isacharunderscore}{\kern0pt}preds{\isacharunderscore}{\kern0pt}fm{\isacharunderscore}{\kern0pt}Rfm{\isacharunderscore}{\kern0pt}ren{\isacharparenleft}{\kern0pt}Rfm{\isacharcomma}{\kern0pt}\ x{\isacharparenright}{\kern0pt}{\isacharparenright}{\kern0pt}{\isacharparenright}{\kern0pt}{\isachardoublequoteclose}\ \isanewline
\isanewline
\isacommand{context}\isamarkupfalse%
\ M{\isacharunderscore}{\kern0pt}ctm\ \isanewline
\isakeyword{begin}\ \isanewline
\isanewline
\isacommand{lemma}\isamarkupfalse%
\ is{\isacharunderscore}{\kern0pt}preds{\isacharunderscore}{\kern0pt}fm{\isacharunderscore}{\kern0pt}ren{\isacharunderscore}{\kern0pt}type\ {\isacharcolon}{\kern0pt}\ \isanewline
\ \ \isakeyword{fixes}\ x\ \isanewline
\ \ \isakeyword{assumes}\ {\isachardoublequoteopen}x\ {\isasymin}\ nat{\isachardoublequoteclose}\ \isanewline
\ \ \isakeyword{shows}\ {\isachardoublequoteopen}is{\isacharunderscore}{\kern0pt}preds{\isacharunderscore}{\kern0pt}fm{\isacharunderscore}{\kern0pt}ren{\isacharparenleft}{\kern0pt}x{\isacharparenright}{\kern0pt}\ {\isasymin}\ {\isadigit{2}}\ {\isasymrightarrow}\ {\isacharparenleft}{\kern0pt}x\ {\isacharhash}{\kern0pt}{\isacharplus}{\kern0pt}\ {\isadigit{2}}{\isacharparenright}{\kern0pt}{\isachardoublequoteclose}\ \isanewline
%
\isadelimproof
\ \ %
\endisadelimproof
%
\isatagproof
\isacommand{apply}\isamarkupfalse%
{\isacharparenleft}{\kern0pt}rule\ Pi{\isacharunderscore}{\kern0pt}memberI{\isacharparenright}{\kern0pt}\isanewline
\ \ \isacommand{unfolding}\isamarkupfalse%
\ is{\isacharunderscore}{\kern0pt}preds{\isacharunderscore}{\kern0pt}fm{\isacharunderscore}{\kern0pt}ren{\isacharunderscore}{\kern0pt}def\ relation{\isacharunderscore}{\kern0pt}def\ function{\isacharunderscore}{\kern0pt}def\ domain{\isacharunderscore}{\kern0pt}def\ range{\isacharunderscore}{\kern0pt}def\isanewline
\ \ \ \ \ \isacommand{apply}\isamarkupfalse%
{\isacharparenleft}{\kern0pt}force{\isacharcomma}{\kern0pt}\ force{\isacharcomma}{\kern0pt}\ force{\isacharparenright}{\kern0pt}\isanewline
\ \ \isacommand{using}\isamarkupfalse%
\ assms\ \isanewline
\ \ \isacommand{apply}\isamarkupfalse%
\ auto\isanewline
\ \ \ \isacommand{apply}\isamarkupfalse%
{\isacharparenleft}{\kern0pt}subgoal{\isacharunderscore}{\kern0pt}tac\ {\isachardoublequoteopen}x\ {\isasymle}\ {\isadigit{0}}{\isachardoublequoteclose}{\isacharparenright}{\kern0pt}\ \isanewline
\ \ \ \ \isacommand{apply}\isamarkupfalse%
\ force\isanewline
\ \ \isacommand{apply}\isamarkupfalse%
{\isacharparenleft}{\kern0pt}subgoal{\isacharunderscore}{\kern0pt}tac\ {\isachardoublequoteopen}{\isasymnot}\ {\isadigit{0}}\ {\isacharless}{\kern0pt}\ x{\isachardoublequoteclose}{\isacharcomma}{\kern0pt}\ rule\ not{\isacharunderscore}{\kern0pt}lt{\isacharunderscore}{\kern0pt}imp{\isacharunderscore}{\kern0pt}le{\isacharcomma}{\kern0pt}\ simp{\isacharcomma}{\kern0pt}\ simp{\isacharcomma}{\kern0pt}\ simp\ add{\isacharcolon}{\kern0pt}assms{\isacharparenright}{\kern0pt}\isanewline
\ \ \isacommand{apply}\isamarkupfalse%
\ clarify\ \isanewline
\ \ \isacommand{using}\isamarkupfalse%
\ ltD\ \isanewline
\ \ \isacommand{by}\isamarkupfalse%
\ auto%
\endisatagproof
{\isafoldproof}%
%
\isadelimproof
\isanewline
%
\endisadelimproof
\isanewline
\isacommand{lemma}\isamarkupfalse%
\ is{\isacharunderscore}{\kern0pt}preds{\isacharunderscore}{\kern0pt}fm{\isacharunderscore}{\kern0pt}type\ {\isacharcolon}{\kern0pt}\ \isanewline
\ \ \isakeyword{fixes}\ Rfm\ x\ S\ \isanewline
\ \ \isakeyword{assumes}\ {\isachardoublequoteopen}x\ {\isasymin}\ nat{\isachardoublequoteclose}\ {\isachardoublequoteopen}S\ {\isasymin}\ nat{\isachardoublequoteclose}\ {\isachardoublequoteopen}Rfm\ {\isasymin}\ formula{\isachardoublequoteclose}\ \isanewline
\ \ \isakeyword{shows}\ {\isachardoublequoteopen}is{\isacharunderscore}{\kern0pt}preds{\isacharunderscore}{\kern0pt}fm{\isacharparenleft}{\kern0pt}Rfm{\isacharcomma}{\kern0pt}\ x{\isacharcomma}{\kern0pt}\ S{\isacharparenright}{\kern0pt}\ {\isasymin}\ formula{\isachardoublequoteclose}\isanewline
%
\isadelimproof
%
\endisadelimproof
%
\isatagproof
\isacommand{proof}\isamarkupfalse%
\ {\isacharminus}{\kern0pt}\isanewline
\ \ \isacommand{have}\isamarkupfalse%
\ {\isachardoublequoteopen}is{\isacharunderscore}{\kern0pt}preds{\isacharunderscore}{\kern0pt}fm{\isacharunderscore}{\kern0pt}Rfm{\isacharunderscore}{\kern0pt}ren{\isacharparenleft}{\kern0pt}Rfm{\isacharcomma}{\kern0pt}\ x{\isacharparenright}{\kern0pt}\ {\isasymin}\ formula{\isachardoublequoteclose}\ \isanewline
\ \ \ \ \isacommand{unfolding}\isamarkupfalse%
\ is{\isacharunderscore}{\kern0pt}preds{\isacharunderscore}{\kern0pt}fm{\isacharunderscore}{\kern0pt}Rfm{\isacharunderscore}{\kern0pt}ren{\isacharunderscore}{\kern0pt}def\ \isanewline
\ \ \ \ \isacommand{apply}\isamarkupfalse%
{\isacharparenleft}{\kern0pt}rule\ ren{\isacharunderscore}{\kern0pt}tc{\isacharparenright}{\kern0pt}\isanewline
\ \ \ \ \isacommand{using}\isamarkupfalse%
\ assms\ \isanewline
\ \ \ \ \ \ \ \isacommand{apply}\isamarkupfalse%
{\isacharparenleft}{\kern0pt}simp{\isacharcomma}{\kern0pt}\ simp{\isacharcomma}{\kern0pt}\ simp{\isacharparenright}{\kern0pt}\isanewline
\ \ \ \ \isacommand{using}\isamarkupfalse%
\ is{\isacharunderscore}{\kern0pt}preds{\isacharunderscore}{\kern0pt}fm{\isacharunderscore}{\kern0pt}ren{\isacharunderscore}{\kern0pt}type\ assms\ \isanewline
\ \ \ \ \isacommand{by}\isamarkupfalse%
\ auto\isanewline
\ \ \isacommand{then}\isamarkupfalse%
\ \isacommand{show}\isamarkupfalse%
\ {\isachardoublequoteopen}is{\isacharunderscore}{\kern0pt}preds{\isacharunderscore}{\kern0pt}fm{\isacharparenleft}{\kern0pt}Rfm{\isacharcomma}{\kern0pt}\ x{\isacharcomma}{\kern0pt}\ S{\isacharparenright}{\kern0pt}\ {\isasymin}\ formula{\isachardoublequoteclose}\isanewline
\ \ \ \ \isacommand{unfolding}\isamarkupfalse%
\ is{\isacharunderscore}{\kern0pt}preds{\isacharunderscore}{\kern0pt}fm{\isacharunderscore}{\kern0pt}def\ \isanewline
\ \ \ \ \isacommand{using}\isamarkupfalse%
\ assms\ \isanewline
\ \ \ \ \isacommand{by}\isamarkupfalse%
\ auto\isanewline
\isacommand{qed}\isamarkupfalse%
%
\endisatagproof
{\isafoldproof}%
%
\isadelimproof
\isanewline
%
\endisadelimproof
\isanewline
\isacommand{lemma}\isamarkupfalse%
\ is{\isacharunderscore}{\kern0pt}preds{\isacharunderscore}{\kern0pt}fm{\isacharunderscore}{\kern0pt}arity\ {\isacharcolon}{\kern0pt}\ \isanewline
\ \ \isakeyword{fixes}\ Rfm\ x\ S\ \isanewline
\ \ \isakeyword{assumes}\ {\isachardoublequoteopen}x\ {\isasymin}\ nat{\isachardoublequoteclose}\ {\isachardoublequoteopen}S\ {\isasymin}\ nat{\isachardoublequoteclose}\ {\isachardoublequoteopen}Rfm\ {\isasymin}\ formula{\isachardoublequoteclose}\ {\isachardoublequoteopen}arity{\isacharparenleft}{\kern0pt}Rfm{\isacharparenright}{\kern0pt}\ {\isacharequal}{\kern0pt}\ {\isadigit{2}}{\isachardoublequoteclose}\isanewline
\ \ \isakeyword{shows}\ {\isachardoublequoteopen}arity{\isacharparenleft}{\kern0pt}is{\isacharunderscore}{\kern0pt}preds{\isacharunderscore}{\kern0pt}fm{\isacharparenleft}{\kern0pt}Rfm{\isacharcomma}{\kern0pt}\ x{\isacharcomma}{\kern0pt}\ S{\isacharparenright}{\kern0pt}{\isacharparenright}{\kern0pt}\ {\isasymle}\ succ{\isacharparenleft}{\kern0pt}x{\isacharparenright}{\kern0pt}\ {\isasymunion}\ succ{\isacharparenleft}{\kern0pt}S{\isacharparenright}{\kern0pt}{\isachardoublequoteclose}\isanewline
%
\isadelimproof
%
\endisadelimproof
%
\isatagproof
\isacommand{proof}\isamarkupfalse%
\ {\isacharminus}{\kern0pt}\ \isanewline
\ \ \isacommand{have}\isamarkupfalse%
\ H\ {\isacharcolon}{\kern0pt}\ {\isachardoublequoteopen}is{\isacharunderscore}{\kern0pt}preds{\isacharunderscore}{\kern0pt}fm{\isacharunderscore}{\kern0pt}Rfm{\isacharunderscore}{\kern0pt}ren{\isacharparenleft}{\kern0pt}Rfm{\isacharcomma}{\kern0pt}\ x{\isacharparenright}{\kern0pt}\ {\isasymin}\ formula{\isachardoublequoteclose}\ \isanewline
\ \ \ \ \isacommand{unfolding}\isamarkupfalse%
\ is{\isacharunderscore}{\kern0pt}preds{\isacharunderscore}{\kern0pt}fm{\isacharunderscore}{\kern0pt}Rfm{\isacharunderscore}{\kern0pt}ren{\isacharunderscore}{\kern0pt}def\isanewline
\ \ \ \ \isacommand{apply}\isamarkupfalse%
{\isacharparenleft}{\kern0pt}rule\ ren{\isacharunderscore}{\kern0pt}tc{\isacharparenright}{\kern0pt}\isanewline
\ \ \ \ \isacommand{using}\isamarkupfalse%
\ assms\ is{\isacharunderscore}{\kern0pt}preds{\isacharunderscore}{\kern0pt}fm{\isacharunderscore}{\kern0pt}ren{\isacharunderscore}{\kern0pt}type\ \isanewline
\ \ \ \ \isacommand{by}\isamarkupfalse%
\ auto\isanewline
\isanewline
\ \ \isacommand{have}\isamarkupfalse%
\ {\isachardoublequoteopen}arity{\isacharparenleft}{\kern0pt}is{\isacharunderscore}{\kern0pt}preds{\isacharunderscore}{\kern0pt}fm{\isacharunderscore}{\kern0pt}Rfm{\isacharunderscore}{\kern0pt}ren{\isacharparenleft}{\kern0pt}Rfm{\isacharcomma}{\kern0pt}\ x{\isacharparenright}{\kern0pt}{\isacharparenright}{\kern0pt}\ {\isasymle}\ x\ {\isacharhash}{\kern0pt}{\isacharplus}{\kern0pt}\ {\isadigit{2}}{\isachardoublequoteclose}\ \isanewline
\ \ \ \ \isacommand{unfolding}\isamarkupfalse%
\ is{\isacharunderscore}{\kern0pt}preds{\isacharunderscore}{\kern0pt}fm{\isacharunderscore}{\kern0pt}Rfm{\isacharunderscore}{\kern0pt}ren{\isacharunderscore}{\kern0pt}def\isanewline
\ \ \ \ \isacommand{apply}\isamarkupfalse%
{\isacharparenleft}{\kern0pt}rule\ arity{\isacharunderscore}{\kern0pt}ren{\isacharparenright}{\kern0pt}\isanewline
\ \ \ \ \isacommand{using}\isamarkupfalse%
\ assms\ is{\isacharunderscore}{\kern0pt}preds{\isacharunderscore}{\kern0pt}fm{\isacharunderscore}{\kern0pt}ren{\isacharunderscore}{\kern0pt}type\ Relation{\isacharunderscore}{\kern0pt}fm{\isacharunderscore}{\kern0pt}def\ le{\isacharunderscore}{\kern0pt}refl\isanewline
\ \ \ \ \isacommand{by}\isamarkupfalse%
\ auto\isanewline
\isanewline
\ \ \isacommand{then}\isamarkupfalse%
\ \isacommand{show}\isamarkupfalse%
\ {\isacharquery}{\kern0pt}thesis\ \isanewline
\ \ \ \ \isacommand{using}\isamarkupfalse%
\ assms\isanewline
\ \ \ \ \isacommand{unfolding}\isamarkupfalse%
\ is{\isacharunderscore}{\kern0pt}preds{\isacharunderscore}{\kern0pt}fm{\isacharunderscore}{\kern0pt}def\isanewline
\ \ \ \ \isacommand{apply}\isamarkupfalse%
\ simp\ \isanewline
\ \ \ \ \isacommand{apply}\isamarkupfalse%
{\isacharparenleft}{\kern0pt}subst\ pred{\isacharunderscore}{\kern0pt}Un{\isacharunderscore}{\kern0pt}distrib{\isacharcomma}{\kern0pt}\ simp{\isacharcomma}{\kern0pt}\ rule\ arity{\isacharunderscore}{\kern0pt}type{\isacharcomma}{\kern0pt}\ simp\ add{\isacharcolon}{\kern0pt}H{\isacharparenright}{\kern0pt}\isanewline
\ \ \ \ \isacommand{apply}\isamarkupfalse%
{\isacharparenleft}{\kern0pt}subst\ pred{\isacharunderscore}{\kern0pt}Un{\isacharunderscore}{\kern0pt}distrib{\isacharcomma}{\kern0pt}\ simp{\isacharcomma}{\kern0pt}\ simp\ add{\isacharcolon}{\kern0pt}assms{\isacharparenright}{\kern0pt}\isanewline
\ \ \ \ \isacommand{apply}\isamarkupfalse%
\ simp\isanewline
\ \ \ \ \isacommand{apply}\isamarkupfalse%
{\isacharparenleft}{\kern0pt}rule\ Un{\isacharunderscore}{\kern0pt}least{\isacharunderscore}{\kern0pt}lt{\isacharcomma}{\kern0pt}\ simp{\isacharcomma}{\kern0pt}\ rule\ ltI{\isacharcomma}{\kern0pt}\ simp{\isacharcomma}{\kern0pt}\ simp{\isacharparenright}{\kern0pt}\isanewline
\ \ \ \ \isacommand{apply}\isamarkupfalse%
{\isacharparenleft}{\kern0pt}rule{\isacharunderscore}{\kern0pt}tac\ j\ {\isacharequal}{\kern0pt}\ {\isachardoublequoteopen}pred{\isacharparenleft}{\kern0pt}x\ {\isacharhash}{\kern0pt}{\isacharplus}{\kern0pt}\ {\isadigit{2}}{\isacharparenright}{\kern0pt}{\isachardoublequoteclose}\ \isakeyword{in}\ le{\isacharunderscore}{\kern0pt}trans{\isacharcomma}{\kern0pt}\ rule\ pred{\isacharunderscore}{\kern0pt}le{\isacharcomma}{\kern0pt}\ rule\ arity{\isacharunderscore}{\kern0pt}type{\isacharcomma}{\kern0pt}\ simp\ add{\isacharcolon}{\kern0pt}H{\isacharparenright}{\kern0pt}\isanewline
\ \ \ \ \isacommand{using}\isamarkupfalse%
\ ltI\ \isanewline
\ \ \ \ \isacommand{by}\isamarkupfalse%
\ auto\isanewline
\isacommand{qed}\isamarkupfalse%
%
\endisatagproof
{\isafoldproof}%
%
\isadelimproof
\ \isanewline
%
\endisadelimproof
\isanewline
\isacommand{lemma}\isamarkupfalse%
\ sats{\isacharunderscore}{\kern0pt}is{\isacharunderscore}{\kern0pt}preds{\isacharunderscore}{\kern0pt}fm{\isacharunderscore}{\kern0pt}iff\ {\isacharcolon}{\kern0pt}\isanewline
\ \ \isakeyword{fixes}\ R\ Rfm\ i\ j\ x\ S\ env\isanewline
\ \ \isakeyword{assumes}\ {\isachardoublequoteopen}Relation{\isacharunderscore}{\kern0pt}fm{\isacharparenleft}{\kern0pt}R{\isacharcomma}{\kern0pt}\ Rfm{\isacharparenright}{\kern0pt}{\isachardoublequoteclose}\ {\isachardoublequoteopen}i\ {\isasymin}\ nat{\isachardoublequoteclose}\ {\isachardoublequoteopen}j\ {\isasymin}\ nat{\isachardoublequoteclose}\ {\isachardoublequoteopen}nth{\isacharparenleft}{\kern0pt}i{\isacharcomma}{\kern0pt}\ env{\isacharparenright}{\kern0pt}\ {\isacharequal}{\kern0pt}\ x{\isachardoublequoteclose}\ {\isachardoublequoteopen}nth{\isacharparenleft}{\kern0pt}j{\isacharcomma}{\kern0pt}\ env{\isacharparenright}{\kern0pt}\ {\isacharequal}{\kern0pt}\ S{\isachardoublequoteclose}\ {\isachardoublequoteopen}env\ {\isasymin}\ list{\isacharparenleft}{\kern0pt}M{\isacharparenright}{\kern0pt}{\isachardoublequoteclose}\ {\isachardoublequoteopen}x\ {\isasymin}\ M{\isachardoublequoteclose}\ {\isachardoublequoteopen}S\ {\isasymin}\ M{\isachardoublequoteclose}\ \isanewline
\ \ \isakeyword{shows}\ {\isachardoublequoteopen}sats{\isacharparenleft}{\kern0pt}M{\isacharcomma}{\kern0pt}\ is{\isacharunderscore}{\kern0pt}preds{\isacharunderscore}{\kern0pt}fm{\isacharparenleft}{\kern0pt}Rfm{\isacharcomma}{\kern0pt}\ i{\isacharcomma}{\kern0pt}\ j{\isacharparenright}{\kern0pt}{\isacharcomma}{\kern0pt}\ env{\isacharparenright}{\kern0pt}\ {\isasymlongleftrightarrow}\ S\ {\isacharequal}{\kern0pt}\ preds{\isacharparenleft}{\kern0pt}R{\isacharcomma}{\kern0pt}\ x{\isacharparenright}{\kern0pt}{\isachardoublequoteclose}\isanewline
%
\isadelimproof
%
\endisadelimproof
%
\isatagproof
\isacommand{proof}\isamarkupfalse%
\ {\isacharminus}{\kern0pt}\ \isanewline
\ \ \isacommand{have}\isamarkupfalse%
\ iff{\isacharunderscore}{\kern0pt}lemma\ {\isacharcolon}{\kern0pt}\ {\isachardoublequoteopen}{\isasymAnd}P\ Q\ R{\isachardot}{\kern0pt}\ {\isacharparenleft}{\kern0pt}P\ {\isasymlongleftrightarrow}\ Q{\isacharparenright}{\kern0pt}\ {\isasymLongrightarrow}\ {\isacharparenleft}{\kern0pt}R\ {\isasymlongleftrightarrow}\ P{\isacharparenright}{\kern0pt}\ {\isasymlongleftrightarrow}\ {\isacharparenleft}{\kern0pt}R\ {\isasymlongleftrightarrow}\ Q{\isacharparenright}{\kern0pt}{\isachardoublequoteclose}\ \isacommand{by}\isamarkupfalse%
\ auto\isanewline
\isanewline
\ \ \isacommand{have}\isamarkupfalse%
\ ren{\isacharunderscore}{\kern0pt}iff{\isacharcolon}{\kern0pt}\ \ {\isachardoublequoteopen}{\isasymAnd}v{\isachardot}{\kern0pt}\ v\ {\isasymin}\ M\ {\isasymLongrightarrow}\ sats{\isacharparenleft}{\kern0pt}M{\isacharcomma}{\kern0pt}\ is{\isacharunderscore}{\kern0pt}preds{\isacharunderscore}{\kern0pt}fm{\isacharunderscore}{\kern0pt}Rfm{\isacharunderscore}{\kern0pt}ren{\isacharparenleft}{\kern0pt}Rfm{\isacharcomma}{\kern0pt}\ i{\isacharparenright}{\kern0pt}{\isacharcomma}{\kern0pt}\ Cons{\isacharparenleft}{\kern0pt}v{\isacharcomma}{\kern0pt}\ env{\isacharparenright}{\kern0pt}{\isacharparenright}{\kern0pt}\ {\isasymlongleftrightarrow}\ sats{\isacharparenleft}{\kern0pt}M{\isacharcomma}{\kern0pt}\ Rfm{\isacharcomma}{\kern0pt}\ {\isacharbrackleft}{\kern0pt}v{\isacharcomma}{\kern0pt}\ nth{\isacharparenleft}{\kern0pt}i{\isacharcomma}{\kern0pt}\ env{\isacharparenright}{\kern0pt}\ {\isacharbrackright}{\kern0pt}{\isacharparenright}{\kern0pt}{\isachardoublequoteclose}\ \isanewline
\ \ \ \ \isacommand{unfolding}\isamarkupfalse%
\ is{\isacharunderscore}{\kern0pt}preds{\isacharunderscore}{\kern0pt}fm{\isacharunderscore}{\kern0pt}Rfm{\isacharunderscore}{\kern0pt}ren{\isacharunderscore}{\kern0pt}def\ \isanewline
\ \ \ \ \isacommand{apply}\isamarkupfalse%
{\isacharparenleft}{\kern0pt}rule\ iff{\isacharunderscore}{\kern0pt}flip{\isacharcomma}{\kern0pt}\ rule\ sats{\isacharunderscore}{\kern0pt}iff{\isacharunderscore}{\kern0pt}sats{\isacharunderscore}{\kern0pt}ren{\isacharparenright}{\kern0pt}\isanewline
\ \ \ \ \isacommand{using}\isamarkupfalse%
\ assms\ Relation{\isacharunderscore}{\kern0pt}fm{\isacharunderscore}{\kern0pt}def\ \isanewline
\ \ \ \ \ \ \ \ \ \ \ \isacommand{apply}\isamarkupfalse%
\ simp{\isacharunderscore}{\kern0pt}all\isanewline
\ \ \ \ \ \isacommand{apply}\isamarkupfalse%
{\isacharparenleft}{\kern0pt}rule\ Pi{\isacharunderscore}{\kern0pt}memberI{\isacharparenright}{\kern0pt}\isanewline
\ \ \ \ \isacommand{unfolding}\isamarkupfalse%
\ is{\isacharunderscore}{\kern0pt}preds{\isacharunderscore}{\kern0pt}fm{\isacharunderscore}{\kern0pt}ren{\isacharunderscore}{\kern0pt}def\ relation{\isacharunderscore}{\kern0pt}def\ function{\isacharunderscore}{\kern0pt}def\isanewline
\ \ \ \ \ \ \ \ \isacommand{apply}\isamarkupfalse%
{\isacharparenleft}{\kern0pt}simp{\isacharcomma}{\kern0pt}\ simp{\isacharcomma}{\kern0pt}\ force{\isacharcomma}{\kern0pt}\ force{\isacharcomma}{\kern0pt}\ simp{\isacharparenright}{\kern0pt}\isanewline
\ \ \ \ \ \isacommand{apply}\isamarkupfalse%
{\isacharparenleft}{\kern0pt}rule\ ltD{\isacharcomma}{\kern0pt}\ simp{\isacharparenright}{\kern0pt}\isanewline
\ \ \ \ \isacommand{apply}\isamarkupfalse%
{\isacharparenleft}{\kern0pt}rename{\isacharunderscore}{\kern0pt}tac\ v\ k{\isacharcomma}{\kern0pt}\ case{\isacharunderscore}{\kern0pt}tac\ {\isachardoublequoteopen}k{\isacharequal}{\kern0pt}{\isadigit{1}}{\isachardoublequoteclose}{\isacharparenright}{\kern0pt}\isanewline
\ \ \ \ \ \isacommand{apply}\isamarkupfalse%
\ simp\isanewline
\ \ \ \ \ \isacommand{apply}\isamarkupfalse%
{\isacharparenleft}{\kern0pt}subst\ function{\isacharunderscore}{\kern0pt}apply{\isacharunderscore}{\kern0pt}equality{\isacharcomma}{\kern0pt}\ simp{\isacharparenright}{\kern0pt}\isanewline
\ \ \ \ \isacommand{unfolding}\isamarkupfalse%
\ function{\isacharunderscore}{\kern0pt}def\ \ \isanewline
\ \ \ \ \ \ \isacommand{apply}\isamarkupfalse%
\ force\isanewline
\ \ \ \ \ \isacommand{apply}\isamarkupfalse%
\ simp\isanewline
\ \ \ \ \isacommand{apply}\isamarkupfalse%
{\isacharparenleft}{\kern0pt}rename{\isacharunderscore}{\kern0pt}tac\ v\ k{\isacharcomma}{\kern0pt}\ subgoal{\isacharunderscore}{\kern0pt}tac\ {\isachardoublequoteopen}k{\isacharequal}{\kern0pt}{\isadigit{0}}{\isachardoublequoteclose}{\isacharparenright}{\kern0pt}\isanewline
\ \ \ \ \ \isacommand{apply}\isamarkupfalse%
{\isacharparenleft}{\kern0pt}subst\ function{\isacharunderscore}{\kern0pt}apply{\isacharunderscore}{\kern0pt}equality{\isacharcomma}{\kern0pt}\ simp{\isacharparenright}{\kern0pt}\isanewline
\ \ \ \ \isacommand{unfolding}\isamarkupfalse%
\ function{\isacharunderscore}{\kern0pt}def\ \ \isanewline
\ \ \ \ \ \ \isacommand{apply}\isamarkupfalse%
\ force\isanewline
\ \ \ \ \ \isacommand{apply}\isamarkupfalse%
\ simp\isanewline
\ \ \ \ \isacommand{using}\isamarkupfalse%
\ le{\isacharunderscore}{\kern0pt}iff\ \isanewline
\ \ \ \ \isacommand{by}\isamarkupfalse%
\ auto\isanewline
\ \ \isacommand{have}\isamarkupfalse%
\ RH\ {\isacharcolon}{\kern0pt}\ {\isachardoublequoteopen}{\isasymAnd}v{\isachardot}{\kern0pt}\ v\ {\isasymin}\ M\ {\isasymLongrightarrow}\ R{\isacharparenleft}{\kern0pt}v{\isacharcomma}{\kern0pt}\ nth{\isacharparenleft}{\kern0pt}i{\isacharcomma}{\kern0pt}\ env{\isacharparenright}{\kern0pt}{\isacharparenright}{\kern0pt}\ {\isasymlongleftrightarrow}\ sats{\isacharparenleft}{\kern0pt}M{\isacharcomma}{\kern0pt}\ is{\isacharunderscore}{\kern0pt}preds{\isacharunderscore}{\kern0pt}fm{\isacharunderscore}{\kern0pt}Rfm{\isacharunderscore}{\kern0pt}ren{\isacharparenleft}{\kern0pt}Rfm{\isacharcomma}{\kern0pt}\ i{\isacharparenright}{\kern0pt}{\isacharcomma}{\kern0pt}\ Cons{\isacharparenleft}{\kern0pt}v{\isacharcomma}{\kern0pt}\ env{\isacharparenright}{\kern0pt}{\isacharparenright}{\kern0pt}{\isachardoublequoteclose}\isanewline
\ \ \ \ \isacommand{apply}\isamarkupfalse%
{\isacharparenleft}{\kern0pt}rule\ iff{\isacharunderscore}{\kern0pt}trans{\isacharcomma}{\kern0pt}\ rule{\isacharunderscore}{\kern0pt}tac\ R{\isacharequal}{\kern0pt}R\ \isakeyword{and}\ Rfm\ {\isacharequal}{\kern0pt}\ Rfm\ \isakeyword{and}\ env\ {\isacharequal}{\kern0pt}\ {\isachardoublequoteopen}{\isacharbrackleft}{\kern0pt}{\isacharbrackright}{\kern0pt}{\isachardoublequoteclose}\ \isakeyword{in}\ \ Relation{\isacharunderscore}{\kern0pt}fmD{\isacharparenright}{\kern0pt}\isanewline
\ \ \ \ \isacommand{using}\isamarkupfalse%
\ assms\ \isanewline
\ \ \ \ \ \ \ \ \isacommand{apply}\isamarkupfalse%
{\isacharparenleft}{\kern0pt}simp\ {\isacharcomma}{\kern0pt}simp\ {\isacharcomma}{\kern0pt}simp\ {\isacharcomma}{\kern0pt}simp{\isacharparenright}{\kern0pt}\isanewline
\ \ \ \ \isacommand{using}\isamarkupfalse%
\ ren{\isacharunderscore}{\kern0pt}iff\ \isanewline
\ \ \ \ \isacommand{by}\isamarkupfalse%
\ auto\isanewline
\isanewline
\ \ \isacommand{have}\isamarkupfalse%
\ I{\isadigit{1}}\ {\isacharcolon}{\kern0pt}\ {\isachardoublequoteopen}sats{\isacharparenleft}{\kern0pt}M{\isacharcomma}{\kern0pt}\ is{\isacharunderscore}{\kern0pt}preds{\isacharunderscore}{\kern0pt}fm{\isacharparenleft}{\kern0pt}Rfm{\isacharcomma}{\kern0pt}\ i{\isacharcomma}{\kern0pt}\ j{\isacharparenright}{\kern0pt}{\isacharcomma}{\kern0pt}\ env{\isacharparenright}{\kern0pt}\ {\isasymlongleftrightarrow}\ {\isacharparenleft}{\kern0pt}{\isasymforall}v\ {\isasymin}\ M{\isachardot}{\kern0pt}\ {\isacharparenleft}{\kern0pt}v\ {\isasymin}\ S\ {\isasymlongleftrightarrow}\ sats{\isacharparenleft}{\kern0pt}M{\isacharcomma}{\kern0pt}\ is{\isacharunderscore}{\kern0pt}preds{\isacharunderscore}{\kern0pt}fm{\isacharunderscore}{\kern0pt}Rfm{\isacharunderscore}{\kern0pt}ren{\isacharparenleft}{\kern0pt}Rfm{\isacharcomma}{\kern0pt}\ i{\isacharparenright}{\kern0pt}{\isacharcomma}{\kern0pt}\ Cons{\isacharparenleft}{\kern0pt}v{\isacharcomma}{\kern0pt}\ env{\isacharparenright}{\kern0pt}{\isacharparenright}{\kern0pt}{\isacharparenright}{\kern0pt}{\isacharparenright}{\kern0pt}{\isachardoublequoteclose}\ \isanewline
\ \ \ \ \isacommand{unfolding}\isamarkupfalse%
\ is{\isacharunderscore}{\kern0pt}preds{\isacharunderscore}{\kern0pt}fm{\isacharunderscore}{\kern0pt}def\ \isanewline
\ \ \ \ \isacommand{using}\isamarkupfalse%
\ assms\isanewline
\ \ \ \ \isacommand{by}\isamarkupfalse%
\ simp\isanewline
\ \ \isacommand{have}\isamarkupfalse%
\ I{\isadigit{2}}\ {\isacharcolon}{\kern0pt}\ {\isachardoublequoteopen}{\isachardot}{\kern0pt}{\isachardot}{\kern0pt}{\isachardot}{\kern0pt}\ {\isasymlongleftrightarrow}\ {\isacharparenleft}{\kern0pt}{\isasymforall}v\ {\isasymin}\ M{\isachardot}{\kern0pt}\ {\isacharparenleft}{\kern0pt}v\ {\isasymin}\ S\ {\isasymlongleftrightarrow}\ R{\isacharparenleft}{\kern0pt}v{\isacharcomma}{\kern0pt}\ x{\isacharparenright}{\kern0pt}{\isacharparenright}{\kern0pt}{\isacharparenright}{\kern0pt}{\isachardoublequoteclose}\isanewline
\ \ \ \ \isacommand{apply}\isamarkupfalse%
{\isacharparenleft}{\kern0pt}rule\ ball{\isacharunderscore}{\kern0pt}iff{\isacharcomma}{\kern0pt}\ rule\ iff{\isacharunderscore}{\kern0pt}lemma{\isacharparenright}{\kern0pt}\isanewline
\ \ \ \ \isacommand{using}\isamarkupfalse%
\ assms\ RH\isanewline
\ \ \ \ \isacommand{by}\isamarkupfalse%
\ auto\isanewline
\ \ \isacommand{have}\isamarkupfalse%
\ I{\isadigit{3}}\ {\isacharcolon}{\kern0pt}\ {\isachardoublequoteopen}{\isachardot}{\kern0pt}{\isachardot}{\kern0pt}{\isachardot}{\kern0pt}\ {\isasymlongleftrightarrow}\ S\ {\isacharequal}{\kern0pt}\ preds{\isacharparenleft}{\kern0pt}R{\isacharcomma}{\kern0pt}\ x{\isacharparenright}{\kern0pt}{\isachardoublequoteclose}\isanewline
\ \ \ \ \isacommand{unfolding}\isamarkupfalse%
\ preds{\isacharunderscore}{\kern0pt}def\isanewline
\ \ \ \ \isacommand{apply}\isamarkupfalse%
{\isacharparenleft}{\kern0pt}rule\ iffI{\isacharcomma}{\kern0pt}\ rule\ equality{\isacharunderscore}{\kern0pt}iffI{\isacharcomma}{\kern0pt}\ rule\ iffI{\isacharparenright}{\kern0pt}\isanewline
\ \ \ \ \ \ \isacommand{apply}\isamarkupfalse%
{\isacharparenleft}{\kern0pt}rename{\isacharunderscore}{\kern0pt}tac\ y{\isacharcomma}{\kern0pt}\ subgoal{\isacharunderscore}{\kern0pt}tac\ {\isachardoublequoteopen}y\ {\isasymin}\ M{\isachardoublequoteclose}{\isacharcomma}{\kern0pt}\ force{\isacharparenright}{\kern0pt}\isanewline
\ \ \ \ \isacommand{using}\isamarkupfalse%
\ assms\ transM\ \isanewline
\ \ \ \ \isacommand{by}\isamarkupfalse%
\ auto\ \ \ \isanewline
\isanewline
\ \ \isacommand{show}\isamarkupfalse%
\ {\isacharquery}{\kern0pt}thesis\ \isacommand{using}\isamarkupfalse%
\ I{\isadigit{1}}\ I{\isadigit{2}}\ I{\isadigit{3}}\ \isacommand{by}\isamarkupfalse%
\ auto\isanewline
\isacommand{qed}\isamarkupfalse%
%
\endisatagproof
{\isafoldproof}%
%
\isadelimproof
\isanewline
%
\endisadelimproof
\isanewline
\isacommand{end}\isamarkupfalse%
\isanewline
\isanewline
\isacommand{definition}\isamarkupfalse%
\ is{\isacharunderscore}{\kern0pt}preds{\isacharunderscore}{\kern0pt}rel{\isacharunderscore}{\kern0pt}fm\ \isakeyword{where}\ \isanewline
\ \ {\isachardoublequoteopen}is{\isacharunderscore}{\kern0pt}preds{\isacharunderscore}{\kern0pt}rel{\isacharunderscore}{\kern0pt}fm{\isacharparenleft}{\kern0pt}Rfm{\isacharcomma}{\kern0pt}\ x{\isacharcomma}{\kern0pt}\ S{\isacharparenright}{\kern0pt}\ {\isasymequiv}\ \isanewline
\ \ \ \ Exists{\isacharparenleft}{\kern0pt}And{\isacharparenleft}{\kern0pt}is{\isacharunderscore}{\kern0pt}preds{\isacharunderscore}{\kern0pt}fm{\isacharparenleft}{\kern0pt}Rfm{\isacharcomma}{\kern0pt}\ x\ {\isacharhash}{\kern0pt}{\isacharplus}{\kern0pt}\ {\isadigit{1}}{\isacharcomma}{\kern0pt}\ {\isadigit{0}}{\isacharparenright}{\kern0pt}{\isacharcomma}{\kern0pt}\ Forall{\isacharparenleft}{\kern0pt}Iff{\isacharparenleft}{\kern0pt}Member{\isacharparenleft}{\kern0pt}{\isadigit{0}}{\isacharcomma}{\kern0pt}\ S\ {\isacharhash}{\kern0pt}{\isacharplus}{\kern0pt}\ {\isadigit{2}}{\isacharparenright}{\kern0pt}{\isacharcomma}{\kern0pt}\ Exists{\isacharparenleft}{\kern0pt}Exists{\isacharparenleft}{\kern0pt}And{\isacharparenleft}{\kern0pt}pair{\isacharunderscore}{\kern0pt}fm{\isacharparenleft}{\kern0pt}{\isadigit{0}}{\isacharcomma}{\kern0pt}\ {\isadigit{1}}{\isacharcomma}{\kern0pt}\ {\isadigit{2}}{\isacharparenright}{\kern0pt}{\isacharcomma}{\kern0pt}\ And{\isacharparenleft}{\kern0pt}Member{\isacharparenleft}{\kern0pt}{\isadigit{0}}{\isacharcomma}{\kern0pt}\ {\isadigit{3}}{\isacharparenright}{\kern0pt}{\isacharcomma}{\kern0pt}\ And{\isacharparenleft}{\kern0pt}Or{\isacharparenleft}{\kern0pt}Member{\isacharparenleft}{\kern0pt}{\isadigit{1}}{\isacharcomma}{\kern0pt}\ {\isadigit{3}}{\isacharparenright}{\kern0pt}{\isacharcomma}{\kern0pt}\ Equal{\isacharparenleft}{\kern0pt}{\isadigit{1}}{\isacharcomma}{\kern0pt}\ x\ {\isacharhash}{\kern0pt}{\isacharplus}{\kern0pt}\ {\isadigit{4}}{\isacharparenright}{\kern0pt}{\isacharparenright}{\kern0pt}{\isacharcomma}{\kern0pt}\ Rfm{\isacharparenright}{\kern0pt}{\isacharparenright}{\kern0pt}{\isacharparenright}{\kern0pt}{\isacharparenright}{\kern0pt}{\isacharparenright}{\kern0pt}{\isacharparenright}{\kern0pt}{\isacharparenright}{\kern0pt}{\isacharparenright}{\kern0pt}{\isacharparenright}{\kern0pt}{\isachardoublequoteclose}\isanewline
\isanewline
\isacommand{context}\isamarkupfalse%
\ M{\isacharunderscore}{\kern0pt}ctm\ \isanewline
\isakeyword{begin}\ \isanewline
\isanewline
\isacommand{lemma}\isamarkupfalse%
\ is{\isacharunderscore}{\kern0pt}preds{\isacharunderscore}{\kern0pt}rel{\isacharunderscore}{\kern0pt}fm{\isacharunderscore}{\kern0pt}type\ {\isacharcolon}{\kern0pt}\ \isanewline
\ \ \isakeyword{fixes}\ Rfm\ x\ S\ \isanewline
\ \ \isakeyword{assumes}\ {\isachardoublequoteopen}x\ {\isasymin}\ nat{\isachardoublequoteclose}\ {\isachardoublequoteopen}S\ {\isasymin}\ nat{\isachardoublequoteclose}\ {\isachardoublequoteopen}Rfm\ {\isasymin}\ formula{\isachardoublequoteclose}\ \isanewline
\ \ \isakeyword{shows}\ {\isachardoublequoteopen}is{\isacharunderscore}{\kern0pt}preds{\isacharunderscore}{\kern0pt}rel{\isacharunderscore}{\kern0pt}fm{\isacharparenleft}{\kern0pt}Rfm{\isacharcomma}{\kern0pt}\ x{\isacharcomma}{\kern0pt}\ S{\isacharparenright}{\kern0pt}\ {\isasymin}\ formula{\isachardoublequoteclose}\isanewline
%
\isadelimproof
\isanewline
\ \ %
\endisadelimproof
%
\isatagproof
\isacommand{unfolding}\isamarkupfalse%
\ is{\isacharunderscore}{\kern0pt}preds{\isacharunderscore}{\kern0pt}rel{\isacharunderscore}{\kern0pt}fm{\isacharunderscore}{\kern0pt}def\isanewline
\ \ \isacommand{apply}\isamarkupfalse%
{\isacharparenleft}{\kern0pt}subgoal{\isacharunderscore}{\kern0pt}tac\ {\isachardoublequoteopen}is{\isacharunderscore}{\kern0pt}preds{\isacharunderscore}{\kern0pt}fm{\isacharparenleft}{\kern0pt}Rfm{\isacharcomma}{\kern0pt}\ x\ {\isacharhash}{\kern0pt}{\isacharplus}{\kern0pt}\ {\isadigit{1}}{\isacharcomma}{\kern0pt}\ {\isadigit{0}}{\isacharparenright}{\kern0pt}\ {\isasymin}\ formula{\isachardoublequoteclose}{\isacharparenright}{\kern0pt}\isanewline
\ \ \isacommand{using}\isamarkupfalse%
\ assms\ \isanewline
\ \ \ \isacommand{apply}\isamarkupfalse%
\ simp\isanewline
\ \ \isacommand{apply}\isamarkupfalse%
{\isacharparenleft}{\kern0pt}rule\ is{\isacharunderscore}{\kern0pt}preds{\isacharunderscore}{\kern0pt}fm{\isacharunderscore}{\kern0pt}type{\isacharparenright}{\kern0pt}\isanewline
\ \ \isacommand{using}\isamarkupfalse%
\ assms\isanewline
\ \ \isacommand{by}\isamarkupfalse%
\ auto%
\endisatagproof
{\isafoldproof}%
%
\isadelimproof
\isanewline
%
\endisadelimproof
\isanewline
\isacommand{lemma}\isamarkupfalse%
\ is{\isacharunderscore}{\kern0pt}preds{\isacharunderscore}{\kern0pt}rel{\isacharunderscore}{\kern0pt}fm{\isacharunderscore}{\kern0pt}arity\ {\isacharcolon}{\kern0pt}\ \isanewline
\ \ \isakeyword{fixes}\ Rfm\ x\ S\ \isanewline
\ \ \isakeyword{assumes}\ {\isachardoublequoteopen}x\ {\isasymin}\ nat{\isachardoublequoteclose}\ {\isachardoublequoteopen}S\ {\isasymin}\ nat{\isachardoublequoteclose}\ {\isachardoublequoteopen}Rfm\ {\isasymin}\ formula{\isachardoublequoteclose}\ {\isachardoublequoteopen}arity{\isacharparenleft}{\kern0pt}Rfm{\isacharparenright}{\kern0pt}\ {\isacharequal}{\kern0pt}\ {\isadigit{2}}{\isachardoublequoteclose}\isanewline
\ \ \isakeyword{shows}\ {\isachardoublequoteopen}arity{\isacharparenleft}{\kern0pt}is{\isacharunderscore}{\kern0pt}preds{\isacharunderscore}{\kern0pt}rel{\isacharunderscore}{\kern0pt}fm{\isacharparenleft}{\kern0pt}Rfm{\isacharcomma}{\kern0pt}\ x{\isacharcomma}{\kern0pt}\ S{\isacharparenright}{\kern0pt}{\isacharparenright}{\kern0pt}\ {\isasymle}\ succ{\isacharparenleft}{\kern0pt}x{\isacharparenright}{\kern0pt}\ {\isasymunion}\ succ{\isacharparenleft}{\kern0pt}S{\isacharparenright}{\kern0pt}{\isachardoublequoteclose}\isanewline
%
\isadelimproof
\isanewline
\ \ %
\endisadelimproof
%
\isatagproof
\isacommand{unfolding}\isamarkupfalse%
\ is{\isacharunderscore}{\kern0pt}preds{\isacharunderscore}{\kern0pt}rel{\isacharunderscore}{\kern0pt}fm{\isacharunderscore}{\kern0pt}def\ \isanewline
\ \ \isacommand{using}\isamarkupfalse%
\ assms\ \isanewline
\ \ \isacommand{apply}\isamarkupfalse%
\ simp\isanewline
\ \ \isacommand{apply}\isamarkupfalse%
{\isacharparenleft}{\kern0pt}subst\ pred{\isacharunderscore}{\kern0pt}Un{\isacharunderscore}{\kern0pt}distrib{\isacharcomma}{\kern0pt}\ rule\ arity{\isacharunderscore}{\kern0pt}type{\isacharcomma}{\kern0pt}\ rule\ is{\isacharunderscore}{\kern0pt}preds{\isacharunderscore}{\kern0pt}fm{\isacharunderscore}{\kern0pt}type{\isacharcomma}{\kern0pt}\ simp{\isacharunderscore}{\kern0pt}all{\isacharparenright}{\kern0pt}\isanewline
\ \ \isacommand{apply}\isamarkupfalse%
{\isacharparenleft}{\kern0pt}rule\ Un{\isacharunderscore}{\kern0pt}least{\isacharunderscore}{\kern0pt}lt{\isacharparenright}{\kern0pt}\isanewline
\ \ \ \isacommand{apply}\isamarkupfalse%
{\isacharparenleft}{\kern0pt}rule{\isacharunderscore}{\kern0pt}tac\ j{\isacharequal}{\kern0pt}{\isachardoublequoteopen}pred{\isacharparenleft}{\kern0pt}succ{\isacharparenleft}{\kern0pt}succ{\isacharparenleft}{\kern0pt}x{\isacharparenright}{\kern0pt}{\isacharparenright}{\kern0pt}\ {\isasymunion}\ succ{\isacharparenleft}{\kern0pt}{\isadigit{0}}{\isacharparenright}{\kern0pt}{\isacharparenright}{\kern0pt}{\isachardoublequoteclose}\ \isakeyword{in}\ le{\isacharunderscore}{\kern0pt}trans{\isacharcomma}{\kern0pt}\ rule\ pred{\isacharunderscore}{\kern0pt}le{\isacharprime}{\kern0pt}{\isacharparenright}{\kern0pt}\isanewline
\ \ \ \ \ \ \isacommand{apply}\isamarkupfalse%
{\isacharparenleft}{\kern0pt}rule\ arity{\isacharunderscore}{\kern0pt}type{\isacharcomma}{\kern0pt}\ rule\ is{\isacharunderscore}{\kern0pt}preds{\isacharunderscore}{\kern0pt}fm{\isacharunderscore}{\kern0pt}type{\isacharcomma}{\kern0pt}\ simp{\isacharunderscore}{\kern0pt}all{\isacharparenright}{\kern0pt}\isanewline
\ \ \ \ \isacommand{apply}\isamarkupfalse%
{\isacharparenleft}{\kern0pt}rule\ is{\isacharunderscore}{\kern0pt}preds{\isacharunderscore}{\kern0pt}fm{\isacharunderscore}{\kern0pt}arity{\isacharparenright}{\kern0pt}\isanewline
\ \ \isacommand{using}\isamarkupfalse%
\ assms\ \isanewline
\ \ \ \ \ \ \ \isacommand{apply}\isamarkupfalse%
\ simp{\isacharunderscore}{\kern0pt}all\isanewline
\ \ \ \isacommand{apply}\isamarkupfalse%
{\isacharparenleft}{\kern0pt}subst\ Ord{\isacharunderscore}{\kern0pt}un{\isacharunderscore}{\kern0pt}eq{\isadigit{1}}{\isacharcomma}{\kern0pt}\ simp{\isacharunderscore}{\kern0pt}all{\isacharcomma}{\kern0pt}\ rule\ ltI{\isacharcomma}{\kern0pt}\ simp{\isacharcomma}{\kern0pt}\ simp{\isacharparenright}{\kern0pt}\isanewline
\ \ \isacommand{apply}\isamarkupfalse%
{\isacharparenleft}{\kern0pt}rule{\isacharunderscore}{\kern0pt}tac\ a{\isacharequal}{\kern0pt}{\isadigit{2}}\ \isakeyword{and}\ \ b\ {\isacharequal}{\kern0pt}\ {\isachardoublequoteopen}arity{\isacharparenleft}{\kern0pt}Rfm{\isacharparenright}{\kern0pt}{\isachardoublequoteclose}\ \isakeyword{in}\ ssubst{\isacharparenright}{\kern0pt}\isanewline
\ \ \isacommand{using}\isamarkupfalse%
\ assms\ \isanewline
\ \ \isacommand{unfolding}\isamarkupfalse%
\ Relation{\isacharunderscore}{\kern0pt}fm{\isacharunderscore}{\kern0pt}def\ \isanewline
\ \ \ \isacommand{apply}\isamarkupfalse%
\ simp\isanewline
\ \ \isacommand{apply}\isamarkupfalse%
{\isacharparenleft}{\kern0pt}simp\ del{\isacharcolon}{\kern0pt}FOL{\isacharunderscore}{\kern0pt}sats{\isacharunderscore}{\kern0pt}iff\ pair{\isacharunderscore}{\kern0pt}abs\ add{\isacharcolon}{\kern0pt}\ fm{\isacharunderscore}{\kern0pt}defs\ nat{\isacharunderscore}{\kern0pt}simp{\isacharunderscore}{\kern0pt}union{\isacharparenright}{\kern0pt}\ \ \isanewline
\ \ \isacommand{apply}\isamarkupfalse%
{\isacharparenleft}{\kern0pt}clarify{\isacharcomma}{\kern0pt}\ rule\ lt{\isacharunderscore}{\kern0pt}succ{\isacharunderscore}{\kern0pt}lt{\isacharcomma}{\kern0pt}\ simp{\isacharparenright}{\kern0pt}\isanewline
\ \ \isacommand{using}\isamarkupfalse%
\ not{\isacharunderscore}{\kern0pt}le{\isacharunderscore}{\kern0pt}iff{\isacharunderscore}{\kern0pt}lt\ \isanewline
\ \ \isacommand{by}\isamarkupfalse%
\ auto%
\endisatagproof
{\isafoldproof}%
%
\isadelimproof
\isanewline
%
\endisadelimproof
\isanewline
\isacommand{lemma}\isamarkupfalse%
\ sats{\isacharunderscore}{\kern0pt}is{\isacharunderscore}{\kern0pt}preds{\isacharunderscore}{\kern0pt}rel{\isacharunderscore}{\kern0pt}fm{\isacharunderscore}{\kern0pt}iff\ {\isacharcolon}{\kern0pt}\isanewline
\ \ \isakeyword{fixes}\ R\ Rfm\ i\ j\ x\ S\ env\isanewline
\ \ \isakeyword{assumes}\ {\isachardoublequoteopen}Relation{\isacharunderscore}{\kern0pt}fm{\isacharparenleft}{\kern0pt}R{\isacharcomma}{\kern0pt}\ Rfm{\isacharparenright}{\kern0pt}{\isachardoublequoteclose}\ {\isachardoublequoteopen}i\ {\isasymin}\ nat{\isachardoublequoteclose}\ {\isachardoublequoteopen}j\ {\isasymin}\ nat{\isachardoublequoteclose}\ {\isachardoublequoteopen}nth{\isacharparenleft}{\kern0pt}i{\isacharcomma}{\kern0pt}\ env{\isacharparenright}{\kern0pt}\ {\isacharequal}{\kern0pt}\ x{\isachardoublequoteclose}\ {\isachardoublequoteopen}nth{\isacharparenleft}{\kern0pt}j{\isacharcomma}{\kern0pt}\ env{\isacharparenright}{\kern0pt}\ {\isacharequal}{\kern0pt}\ S{\isachardoublequoteclose}\ {\isachardoublequoteopen}env\ {\isasymin}\ list{\isacharparenleft}{\kern0pt}M{\isacharparenright}{\kern0pt}{\isachardoublequoteclose}\ {\isachardoublequoteopen}x\ {\isasymin}\ M{\isachardoublequoteclose}\ {\isachardoublequoteopen}S\ {\isasymin}\ M{\isachardoublequoteclose}\ {\isachardoublequoteopen}preds{\isacharparenleft}{\kern0pt}R{\isacharcomma}{\kern0pt}\ x{\isacharparenright}{\kern0pt}\ {\isasymin}\ M{\isachardoublequoteclose}\isanewline
\ \ \isakeyword{shows}\ {\isachardoublequoteopen}sats{\isacharparenleft}{\kern0pt}M{\isacharcomma}{\kern0pt}\ is{\isacharunderscore}{\kern0pt}preds{\isacharunderscore}{\kern0pt}rel{\isacharunderscore}{\kern0pt}fm{\isacharparenleft}{\kern0pt}Rfm{\isacharcomma}{\kern0pt}\ i{\isacharcomma}{\kern0pt}\ j{\isacharparenright}{\kern0pt}{\isacharcomma}{\kern0pt}\ env{\isacharparenright}{\kern0pt}\ {\isasymlongleftrightarrow}\ S\ {\isacharequal}{\kern0pt}\ preds{\isacharunderscore}{\kern0pt}rel{\isacharparenleft}{\kern0pt}R{\isacharcomma}{\kern0pt}\ x{\isacharparenright}{\kern0pt}{\isachardoublequoteclose}\ \isanewline
%
\isadelimproof
%
\endisadelimproof
%
\isatagproof
\isacommand{proof}\isamarkupfalse%
\ {\isacharminus}{\kern0pt}\ \isanewline
\ \ \isacommand{have}\isamarkupfalse%
\ iff{\isacharunderscore}{\kern0pt}lemma\ {\isacharcolon}{\kern0pt}\ {\isachardoublequoteopen}{\isasymAnd}P\ Q\ R\ S{\isachardot}{\kern0pt}\ {\isacharparenleft}{\kern0pt}P\ {\isasymlongleftrightarrow}\ Q{\isacharparenright}{\kern0pt}\ {\isasymLongrightarrow}\ {\isacharparenleft}{\kern0pt}Q\ {\isasymLongrightarrow}\ R\ {\isasymlongleftrightarrow}\ S{\isacharparenright}{\kern0pt}\ {\isasymLongrightarrow}\ {\isacharparenleft}{\kern0pt}P\ {\isasymand}\ R{\isacharparenright}{\kern0pt}\ {\isasymlongleftrightarrow}\ {\isacharparenleft}{\kern0pt}Q\ {\isasymand}\ S{\isacharparenright}{\kern0pt}{\isachardoublequoteclose}\ \isacommand{by}\isamarkupfalse%
\ auto\isanewline
\ \ \isacommand{have}\isamarkupfalse%
\ iff{\isacharunderscore}{\kern0pt}lemma{\isadigit{2}}\ {\isacharcolon}{\kern0pt}\ {\isachardoublequoteopen}{\isasymAnd}P\ Q\ R\ S{\isachardot}{\kern0pt}\ {\isacharparenleft}{\kern0pt}P\ {\isasymlongleftrightarrow}\ Q{\isacharparenright}{\kern0pt}\ {\isasymLongrightarrow}\ {\isacharparenleft}{\kern0pt}R\ {\isasymlongleftrightarrow}\ S{\isacharparenright}{\kern0pt}\ {\isasymLongrightarrow}\ {\isacharparenleft}{\kern0pt}P\ {\isasymlongleftrightarrow}\ R{\isacharparenright}{\kern0pt}\ {\isasymlongleftrightarrow}\ {\isacharparenleft}{\kern0pt}Q\ {\isasymlongleftrightarrow}\ S{\isacharparenright}{\kern0pt}{\isachardoublequoteclose}\ \isacommand{by}\isamarkupfalse%
\ auto\isanewline
\ \ \isacommand{have}\isamarkupfalse%
\ I{\isadigit{1}}\ {\isacharcolon}{\kern0pt}\ {\isachardoublequoteopen}sats{\isacharparenleft}{\kern0pt}M{\isacharcomma}{\kern0pt}\ is{\isacharunderscore}{\kern0pt}preds{\isacharunderscore}{\kern0pt}rel{\isacharunderscore}{\kern0pt}fm{\isacharparenleft}{\kern0pt}Rfm{\isacharcomma}{\kern0pt}\ i{\isacharcomma}{\kern0pt}\ j{\isacharparenright}{\kern0pt}{\isacharcomma}{\kern0pt}\ env{\isacharparenright}{\kern0pt}\ {\isasymlongleftrightarrow}\ {\isacharparenleft}{\kern0pt}{\isasymexists}P\ {\isasymin}\ M{\isachardot}{\kern0pt}\ P\ {\isacharequal}{\kern0pt}\ preds{\isacharparenleft}{\kern0pt}R{\isacharcomma}{\kern0pt}\ x{\isacharparenright}{\kern0pt}\ {\isasymand}\ {\isacharparenleft}{\kern0pt}{\isasymforall}v\ {\isasymin}\ M{\isachardot}{\kern0pt}\ v\ {\isasymin}\ S\ {\isasymlongleftrightarrow}\ {\isacharparenleft}{\kern0pt}{\isasymexists}y\ {\isasymin}\ M{\isachardot}{\kern0pt}\ {\isasymexists}z\ {\isasymin}\ M{\isachardot}{\kern0pt}\ v\ {\isacharequal}{\kern0pt}\ {\isacharless}{\kern0pt}z{\isacharcomma}{\kern0pt}\ y{\isachargreater}{\kern0pt}\ {\isasymand}\ z\ {\isasymin}\ P\ {\isasymand}\ {\isacharparenleft}{\kern0pt}y\ {\isasymin}\ P\ {\isasymor}\ y\ {\isacharequal}{\kern0pt}\ x{\isacharparenright}{\kern0pt}\ {\isasymand}\ R{\isacharparenleft}{\kern0pt}z{\isacharcomma}{\kern0pt}\ y{\isacharparenright}{\kern0pt}{\isacharparenright}{\kern0pt}{\isacharparenright}{\kern0pt}{\isacharparenright}{\kern0pt}{\isachardoublequoteclose}\isanewline
\ \ \ \ \isacommand{unfolding}\isamarkupfalse%
\ is{\isacharunderscore}{\kern0pt}preds{\isacharunderscore}{\kern0pt}rel{\isacharunderscore}{\kern0pt}fm{\isacharunderscore}{\kern0pt}def\ \isanewline
\ \ \ \ \isacommand{apply}\isamarkupfalse%
{\isacharparenleft}{\kern0pt}rule\ iff{\isacharunderscore}{\kern0pt}trans{\isacharcomma}{\kern0pt}\ rule\ sats{\isacharunderscore}{\kern0pt}Exists{\isacharunderscore}{\kern0pt}iff{\isacharcomma}{\kern0pt}\ simp\ add{\isacharcolon}{\kern0pt}assms{\isacharparenright}{\kern0pt}\isanewline
\ \ \ \ \isacommand{apply}\isamarkupfalse%
{\isacharparenleft}{\kern0pt}rule\ bex{\isacharunderscore}{\kern0pt}iff{\isacharcomma}{\kern0pt}\ rule\ iff{\isacharunderscore}{\kern0pt}trans{\isacharcomma}{\kern0pt}\ rule\ sats{\isacharunderscore}{\kern0pt}And{\isacharunderscore}{\kern0pt}iff{\isacharcomma}{\kern0pt}\ simp\ add{\isacharcolon}{\kern0pt}assms{\isacharcomma}{\kern0pt}\ rule\ iff{\isacharunderscore}{\kern0pt}lemma{\isacharparenright}{\kern0pt}\isanewline
\ \ \ \ \isacommand{apply}\isamarkupfalse%
{\isacharparenleft}{\kern0pt}rule\ sats{\isacharunderscore}{\kern0pt}is{\isacharunderscore}{\kern0pt}preds{\isacharunderscore}{\kern0pt}fm{\isacharunderscore}{\kern0pt}iff{\isacharparenright}{\kern0pt}\isanewline
\ \ \ \ \isacommand{using}\isamarkupfalse%
\ assms\ \isanewline
\ \ \ \ \ \ \ \ \ \ \ \isacommand{apply}\isamarkupfalse%
{\isacharparenleft}{\kern0pt}simp{\isacharcomma}{\kern0pt}\ simp{\isacharcomma}{\kern0pt}\ simp{\isacharcomma}{\kern0pt}\ simp{\isacharcomma}{\kern0pt}\ simp{\isacharcomma}{\kern0pt}\ simp{\isacharcomma}{\kern0pt}\ simp{\isacharcomma}{\kern0pt}\ simp{\isacharparenright}{\kern0pt}\isanewline
\ \ \ \ \isacommand{apply}\isamarkupfalse%
{\isacharparenleft}{\kern0pt}rule\ iff{\isacharunderscore}{\kern0pt}trans{\isacharcomma}{\kern0pt}\ rule\ sats{\isacharunderscore}{\kern0pt}Forall{\isacharunderscore}{\kern0pt}iff{\isacharcomma}{\kern0pt}\ simp\ add{\isacharcolon}{\kern0pt}assms{\isacharcomma}{\kern0pt}\ rule\ ball{\isacharunderscore}{\kern0pt}iff{\isacharcomma}{\kern0pt}\ rule\ iff{\isacharunderscore}{\kern0pt}trans{\isacharcomma}{\kern0pt}\ rule\ sats{\isacharunderscore}{\kern0pt}Iff{\isacharunderscore}{\kern0pt}iff{\isacharcomma}{\kern0pt}\ simp\ add{\isacharcolon}{\kern0pt}assms{\isacharparenright}{\kern0pt}\isanewline
\ \ \ \ \isacommand{apply}\isamarkupfalse%
{\isacharparenleft}{\kern0pt}rule\ iff{\isacharunderscore}{\kern0pt}lemma{\isadigit{2}}{\isacharcomma}{\kern0pt}\ simp\ add{\isacharcolon}{\kern0pt}assms{\isacharcomma}{\kern0pt}\ rule\ iff{\isacharunderscore}{\kern0pt}trans{\isacharcomma}{\kern0pt}\ rule\ sats{\isacharunderscore}{\kern0pt}Exists{\isacharunderscore}{\kern0pt}iff{\isacharcomma}{\kern0pt}\ simp\ add{\isacharcolon}{\kern0pt}assms{\isacharparenright}{\kern0pt}\isanewline
\ \ \ \ \isacommand{apply}\isamarkupfalse%
{\isacharparenleft}{\kern0pt}rule\ bex{\isacharunderscore}{\kern0pt}iff{\isacharcomma}{\kern0pt}\ rule\ iff{\isacharunderscore}{\kern0pt}trans{\isacharcomma}{\kern0pt}\ rule\ sats{\isacharunderscore}{\kern0pt}Exists{\isacharunderscore}{\kern0pt}iff{\isacharcomma}{\kern0pt}\ simp\ add{\isacharcolon}{\kern0pt}assms{\isacharcomma}{\kern0pt}\ rule\ bex{\isacharunderscore}{\kern0pt}iff{\isacharcomma}{\kern0pt}\ simp\ add{\isacharcolon}{\kern0pt}assms{\isacharcomma}{\kern0pt}\ rule\ iff{\isacharunderscore}{\kern0pt}conjI{\isacharcomma}{\kern0pt}\ simp{\isacharparenright}{\kern0pt}\isanewline
\ \ \ \ \isacommand{apply}\isamarkupfalse%
{\isacharparenleft}{\kern0pt}rule\ iff{\isacharunderscore}{\kern0pt}trans{\isacharcomma}{\kern0pt}\ rule\ sats{\isacharunderscore}{\kern0pt}And{\isacharunderscore}{\kern0pt}iff{\isacharcomma}{\kern0pt}\ simp\ add{\isacharcolon}{\kern0pt}assms{\isacharparenright}{\kern0pt}\isanewline
\ \ \ \ \isacommand{using}\isamarkupfalse%
\ pair{\isacharunderscore}{\kern0pt}in{\isacharunderscore}{\kern0pt}M{\isacharunderscore}{\kern0pt}iff\ \isanewline
\ \ \ \ \ \isacommand{apply}\isamarkupfalse%
\ force\isanewline
\ \ \ \ \isacommand{apply}\isamarkupfalse%
{\isacharparenleft}{\kern0pt}rule\ iff{\isacharunderscore}{\kern0pt}conjI{\isacharparenright}{\kern0pt}\isanewline
\ \ \ \ \isacommand{using}\isamarkupfalse%
\ pair{\isacharunderscore}{\kern0pt}in{\isacharunderscore}{\kern0pt}M{\isacharunderscore}{\kern0pt}iff\ assms\ \isanewline
\ \ \ \ \ \isacommand{apply}\isamarkupfalse%
\ simp\isanewline
\ \ \ \ \isacommand{apply}\isamarkupfalse%
{\isacharparenleft}{\kern0pt}rule\ iff{\isacharunderscore}{\kern0pt}trans{\isacharcomma}{\kern0pt}\ rule\ sats{\isacharunderscore}{\kern0pt}And{\isacharunderscore}{\kern0pt}iff{\isacharcomma}{\kern0pt}\ simp\ add{\isacharcolon}{\kern0pt}assms{\isacharparenright}{\kern0pt}\isanewline
\ \ \ \ \isacommand{using}\isamarkupfalse%
\ pair{\isacharunderscore}{\kern0pt}in{\isacharunderscore}{\kern0pt}M{\isacharunderscore}{\kern0pt}iff\ \isanewline
\ \ \ \ \ \isacommand{apply}\isamarkupfalse%
\ force\isanewline
\ \ \ \ \isacommand{apply}\isamarkupfalse%
{\isacharparenleft}{\kern0pt}rule\ iff{\isacharunderscore}{\kern0pt}conjI{\isacharcomma}{\kern0pt}\ rule\ iff{\isacharunderscore}{\kern0pt}trans{\isacharcomma}{\kern0pt}\ rule\ sats{\isacharunderscore}{\kern0pt}Or{\isacharunderscore}{\kern0pt}iff{\isacharcomma}{\kern0pt}\ simp\ add{\isacharcolon}{\kern0pt}assms{\isacharparenright}{\kern0pt}\isanewline
\ \ \ \ \isacommand{using}\isamarkupfalse%
\ pair{\isacharunderscore}{\kern0pt}in{\isacharunderscore}{\kern0pt}M{\isacharunderscore}{\kern0pt}iff\ \isanewline
\ \ \ \ \ \ \isacommand{apply}\isamarkupfalse%
\ force\isanewline
\ \ \ \ \isacommand{apply}\isamarkupfalse%
{\isacharparenleft}{\kern0pt}rule\ iff{\isacharunderscore}{\kern0pt}disjI{\isacharparenright}{\kern0pt}\isanewline
\ \ \ \ \isacommand{using}\isamarkupfalse%
\ pair{\isacharunderscore}{\kern0pt}in{\isacharunderscore}{\kern0pt}M{\isacharunderscore}{\kern0pt}iff\ assms\ \isanewline
\ \ \ \ \ \ \isacommand{apply}\isamarkupfalse%
\ {\isacharparenleft}{\kern0pt}simp{\isacharcomma}{\kern0pt}\ simp{\isacharparenright}{\kern0pt}\isanewline
\ \ \ \ \isacommand{apply}\isamarkupfalse%
{\isacharparenleft}{\kern0pt}rename{\isacharunderscore}{\kern0pt}tac\ a\ b\ c\ d\ {\isacharcomma}{\kern0pt}\ rule\ iff{\isacharunderscore}{\kern0pt}flip{\isacharcomma}{\kern0pt}\ rule\ iff{\isacharunderscore}{\kern0pt}trans{\isacharcomma}{\kern0pt}\ rule{\isacharunderscore}{\kern0pt}tac\ R{\isacharequal}{\kern0pt}R\ \isakeyword{and}\ Rfm\ {\isacharequal}{\kern0pt}\ Rfm\ \isakeyword{and}\ env\ {\isacharequal}{\kern0pt}\ {\isachardoublequoteopen}Cons{\isacharparenleft}{\kern0pt}{\isasymlangle}d{\isacharcomma}{\kern0pt}\ c{\isasymrangle}{\isacharcomma}{\kern0pt}\ Cons{\isacharparenleft}{\kern0pt}preds{\isacharparenleft}{\kern0pt}R{\isacharcomma}{\kern0pt}\ x{\isacharparenright}{\kern0pt}{\isacharcomma}{\kern0pt}\ env{\isacharparenright}{\kern0pt}{\isacharparenright}{\kern0pt}{\isachardoublequoteclose}\ \isakeyword{in}\ Relation{\isacharunderscore}{\kern0pt}fmD{\isacharparenright}{\kern0pt}\isanewline
\ \ \ \ \isacommand{using}\isamarkupfalse%
\ assms\ pair{\isacharunderscore}{\kern0pt}in{\isacharunderscore}{\kern0pt}M{\isacharunderscore}{\kern0pt}iff\ \isanewline
\ \ \ \ \isacommand{by}\isamarkupfalse%
\ auto\isanewline
\ \ \isacommand{have}\isamarkupfalse%
\ I{\isadigit{2}}\ {\isacharcolon}{\kern0pt}\ {\isachardoublequoteopen}{\isachardot}{\kern0pt}{\isachardot}{\kern0pt}{\isachardot}{\kern0pt}\ {\isasymlongleftrightarrow}\ {\isacharparenleft}{\kern0pt}{\isasymforall}v\ {\isasymin}\ M{\isachardot}{\kern0pt}\ v\ {\isasymin}\ S\ {\isasymlongleftrightarrow}\ {\isacharparenleft}{\kern0pt}{\isasymexists}y\ {\isasymin}\ M{\isachardot}{\kern0pt}\ {\isasymexists}z\ {\isasymin}\ M{\isachardot}{\kern0pt}\ v\ {\isacharequal}{\kern0pt}\ {\isacharless}{\kern0pt}z{\isacharcomma}{\kern0pt}\ y{\isachargreater}{\kern0pt}\ {\isasymand}\ z\ {\isasymin}\ preds{\isacharparenleft}{\kern0pt}R{\isacharcomma}{\kern0pt}\ x{\isacharparenright}{\kern0pt}\ {\isasymand}\ {\isacharparenleft}{\kern0pt}y\ {\isasymin}\ preds{\isacharparenleft}{\kern0pt}R{\isacharcomma}{\kern0pt}\ x{\isacharparenright}{\kern0pt}\ {\isasymor}\ y\ {\isacharequal}{\kern0pt}\ x{\isacharparenright}{\kern0pt}\ {\isasymand}\ R{\isacharparenleft}{\kern0pt}z{\isacharcomma}{\kern0pt}\ y{\isacharparenright}{\kern0pt}{\isacharparenright}{\kern0pt}{\isacharparenright}{\kern0pt}{\isachardoublequoteclose}\isanewline
\ \ \ \ \isacommand{using}\isamarkupfalse%
\ assms\ \isanewline
\ \ \ \ \isacommand{by}\isamarkupfalse%
\ auto\isanewline
\ \ \isacommand{have}\isamarkupfalse%
\ I{\isadigit{3}}\ {\isacharcolon}{\kern0pt}\ {\isachardoublequoteopen}{\isachardot}{\kern0pt}{\isachardot}{\kern0pt}{\isachardot}{\kern0pt}\ {\isasymlongleftrightarrow}\ S\ {\isacharequal}{\kern0pt}\ preds{\isacharunderscore}{\kern0pt}rel{\isacharparenleft}{\kern0pt}R{\isacharcomma}{\kern0pt}\ x{\isacharparenright}{\kern0pt}{\isachardoublequoteclose}\ \isanewline
\ \ \ \ \isacommand{unfolding}\isamarkupfalse%
\ preds{\isacharunderscore}{\kern0pt}rel{\isacharunderscore}{\kern0pt}def\isanewline
\ \ \isacommand{proof}\isamarkupfalse%
{\isacharparenleft}{\kern0pt}rule\ iffI{\isacharcomma}{\kern0pt}\ rule\ equality{\isacharunderscore}{\kern0pt}iffI{\isacharcomma}{\kern0pt}\ rule\ iffI{\isacharparenright}{\kern0pt}\isanewline
\ \ \ \ \isacommand{fix}\isamarkupfalse%
\ v\ \isacommand{assume}\isamarkupfalse%
\ assms{\isadigit{1}}\ {\isacharcolon}{\kern0pt}\ {\isachardoublequoteopen}{\isasymforall}v{\isasymin}M{\isachardot}{\kern0pt}\ v\ {\isasymin}\ S\ {\isasymlongleftrightarrow}\ {\isacharparenleft}{\kern0pt}{\isasymexists}y{\isasymin}M{\isachardot}{\kern0pt}\ {\isasymexists}z{\isasymin}M{\isachardot}{\kern0pt}\ v\ {\isacharequal}{\kern0pt}\ {\isasymlangle}z{\isacharcomma}{\kern0pt}\ y{\isasymrangle}\ {\isasymand}\ z\ {\isasymin}\ preds{\isacharparenleft}{\kern0pt}R{\isacharcomma}{\kern0pt}\ x{\isacharparenright}{\kern0pt}\ {\isasymand}\ {\isacharparenleft}{\kern0pt}y\ {\isasymin}\ preds{\isacharparenleft}{\kern0pt}R{\isacharcomma}{\kern0pt}\ x{\isacharparenright}{\kern0pt}\ {\isasymor}\ y\ {\isacharequal}{\kern0pt}\ x{\isacharparenright}{\kern0pt}\ {\isasymand}\ R{\isacharparenleft}{\kern0pt}z{\isacharcomma}{\kern0pt}\ y{\isacharparenright}{\kern0pt}{\isacharparenright}{\kern0pt}{\isachardoublequoteclose}\ {\isachardoublequoteopen}v\ {\isasymin}\ S{\isachardoublequoteclose}\ \isanewline
\ \ \ \ \isacommand{then}\isamarkupfalse%
\ \isacommand{have}\isamarkupfalse%
\ {\isachardoublequoteopen}v\ {\isasymin}\ M{\isachardoublequoteclose}\ \isacommand{using}\isamarkupfalse%
\ assms\ transM\ \isacommand{by}\isamarkupfalse%
\ auto\isanewline
\ \ \ \ \isacommand{then}\isamarkupfalse%
\ \isacommand{have}\isamarkupfalse%
\ {\isachardoublequoteopen}{\isacharparenleft}{\kern0pt}{\isasymexists}y{\isasymin}M{\isachardot}{\kern0pt}\ {\isasymexists}z{\isasymin}M{\isachardot}{\kern0pt}\ v\ {\isacharequal}{\kern0pt}\ {\isasymlangle}z{\isacharcomma}{\kern0pt}\ y{\isasymrangle}\ {\isasymand}\ z\ {\isasymin}\ preds{\isacharparenleft}{\kern0pt}R{\isacharcomma}{\kern0pt}\ x{\isacharparenright}{\kern0pt}\ {\isasymand}\ {\isacharparenleft}{\kern0pt}y\ {\isasymin}\ preds{\isacharparenleft}{\kern0pt}R{\isacharcomma}{\kern0pt}\ x{\isacharparenright}{\kern0pt}\ {\isasymor}\ y\ {\isacharequal}{\kern0pt}\ x{\isacharparenright}{\kern0pt}\ {\isasymand}\ R{\isacharparenleft}{\kern0pt}z{\isacharcomma}{\kern0pt}\ y{\isacharparenright}{\kern0pt}{\isacharparenright}{\kern0pt}{\isachardoublequoteclose}\ \isacommand{using}\isamarkupfalse%
\ assms{\isadigit{1}}\ \isacommand{by}\isamarkupfalse%
\ auto\ \isanewline
\ \ \ \ \isacommand{then}\isamarkupfalse%
\ \isacommand{obtain}\isamarkupfalse%
\ y\ z\ \isakeyword{where}\ yzH\ {\isacharcolon}{\kern0pt}{\isachardoublequoteopen}y\ {\isasymin}\ M{\isachardoublequoteclose}\ {\isachardoublequoteopen}z\ {\isasymin}\ M{\isachardoublequoteclose}\ {\isachardoublequoteopen}v\ {\isacharequal}{\kern0pt}\ {\isacharless}{\kern0pt}z{\isacharcomma}{\kern0pt}\ y{\isachargreater}{\kern0pt}{\isachardoublequoteclose}\ {\isachardoublequoteopen}z\ {\isasymin}\ preds{\isacharparenleft}{\kern0pt}R{\isacharcomma}{\kern0pt}\ x{\isacharparenright}{\kern0pt}{\isachardoublequoteclose}\ {\isachardoublequoteopen}y\ {\isasymin}\ preds{\isacharparenleft}{\kern0pt}R{\isacharcomma}{\kern0pt}\ x{\isacharparenright}{\kern0pt}\ {\isasymor}\ y\ {\isacharequal}{\kern0pt}\ x{\isachardoublequoteclose}\ {\isachardoublequoteopen}R{\isacharparenleft}{\kern0pt}z{\isacharcomma}{\kern0pt}\ y{\isacharparenright}{\kern0pt}{\isachardoublequoteclose}\ \isacommand{by}\isamarkupfalse%
\ auto\ \isanewline
\ \ \ \ \isacommand{then}\isamarkupfalse%
\ \isacommand{show}\isamarkupfalse%
\ {\isachardoublequoteopen}v\ {\isasymin}\ {\isacharbraceleft}{\kern0pt}\ {\isacharless}{\kern0pt}z{\isacharcomma}{\kern0pt}\ y{\isachargreater}{\kern0pt}\ {\isasymin}\ preds{\isacharparenleft}{\kern0pt}R{\isacharcomma}{\kern0pt}\ x{\isacharparenright}{\kern0pt}\ {\isasymtimes}\ {\isacharparenleft}{\kern0pt}preds{\isacharparenleft}{\kern0pt}R{\isacharcomma}{\kern0pt}\ x{\isacharparenright}{\kern0pt}\ {\isasymunion}\ {\isacharbraceleft}{\kern0pt}x{\isacharbraceright}{\kern0pt}{\isacharparenright}{\kern0pt}{\isachardot}{\kern0pt}\ R{\isacharparenleft}{\kern0pt}z{\isacharcomma}{\kern0pt}\ y{\isacharparenright}{\kern0pt}\ {\isacharbraceright}{\kern0pt}{\isachardoublequoteclose}\ \isacommand{by}\isamarkupfalse%
\ auto\ \isanewline
\ \ \isacommand{next}\isamarkupfalse%
\ \isanewline
\ \ \ \ \isacommand{fix}\isamarkupfalse%
\ v\ \isacommand{assume}\isamarkupfalse%
\ assms{\isadigit{1}}\ {\isacharcolon}{\kern0pt}\ \isanewline
\ \ \ \ \ \ {\isachardoublequoteopen}{\isasymforall}v{\isasymin}M{\isachardot}{\kern0pt}\ v\ {\isasymin}\ S\ {\isasymlongleftrightarrow}\ {\isacharparenleft}{\kern0pt}{\isasymexists}y{\isasymin}M{\isachardot}{\kern0pt}\ {\isasymexists}z{\isasymin}M{\isachardot}{\kern0pt}\ v\ {\isacharequal}{\kern0pt}\ {\isasymlangle}z{\isacharcomma}{\kern0pt}\ y{\isasymrangle}\ {\isasymand}\ z\ {\isasymin}\ preds{\isacharparenleft}{\kern0pt}R{\isacharcomma}{\kern0pt}\ x{\isacharparenright}{\kern0pt}\ {\isasymand}\ {\isacharparenleft}{\kern0pt}y\ {\isasymin}\ preds{\isacharparenleft}{\kern0pt}R{\isacharcomma}{\kern0pt}\ x{\isacharparenright}{\kern0pt}\ {\isasymor}\ y\ {\isacharequal}{\kern0pt}\ x{\isacharparenright}{\kern0pt}\ {\isasymand}\ R{\isacharparenleft}{\kern0pt}z{\isacharcomma}{\kern0pt}\ y{\isacharparenright}{\kern0pt}{\isacharparenright}{\kern0pt}{\isachardoublequoteclose}\ \isanewline
\ \ \ \ \ \ {\isachardoublequoteopen}v\ {\isasymin}\ {\isacharbraceleft}{\kern0pt}\ {\isacharless}{\kern0pt}z{\isacharcomma}{\kern0pt}\ y{\isachargreater}{\kern0pt}\ {\isasymin}\ preds{\isacharparenleft}{\kern0pt}R{\isacharcomma}{\kern0pt}\ x{\isacharparenright}{\kern0pt}\ {\isasymtimes}\ {\isacharparenleft}{\kern0pt}preds{\isacharparenleft}{\kern0pt}R{\isacharcomma}{\kern0pt}\ x{\isacharparenright}{\kern0pt}\ {\isasymunion}\ {\isacharbraceleft}{\kern0pt}x{\isacharbraceright}{\kern0pt}{\isacharparenright}{\kern0pt}{\isachardot}{\kern0pt}\ R{\isacharparenleft}{\kern0pt}z{\isacharcomma}{\kern0pt}\ y{\isacharparenright}{\kern0pt}\ {\isacharbraceright}{\kern0pt}{\isachardoublequoteclose}\isanewline
\ \ \ \ \isacommand{then}\isamarkupfalse%
\ \isacommand{obtain}\isamarkupfalse%
\ y\ z\ \isakeyword{where}\ yzH\ {\isacharcolon}{\kern0pt}\ {\isachardoublequoteopen}v\ {\isacharequal}{\kern0pt}\ {\isacharless}{\kern0pt}z{\isacharcomma}{\kern0pt}\ y{\isachargreater}{\kern0pt}{\isachardoublequoteclose}\ {\isachardoublequoteopen}z\ {\isasymin}\ preds{\isacharparenleft}{\kern0pt}R{\isacharcomma}{\kern0pt}\ x{\isacharparenright}{\kern0pt}{\isachardoublequoteclose}\ {\isachardoublequoteopen}y\ {\isasymin}\ preds{\isacharparenleft}{\kern0pt}R{\isacharcomma}{\kern0pt}\ x{\isacharparenright}{\kern0pt}\ {\isasymor}\ y\ {\isacharequal}{\kern0pt}\ x{\isachardoublequoteclose}\ {\isachardoublequoteopen}R{\isacharparenleft}{\kern0pt}z{\isacharcomma}{\kern0pt}\ y{\isacharparenright}{\kern0pt}{\isachardoublequoteclose}\ \isacommand{by}\isamarkupfalse%
\ auto\isanewline
\ \ \ \ \isacommand{then}\isamarkupfalse%
\ \isacommand{have}\isamarkupfalse%
\ zinM\ {\isacharcolon}{\kern0pt}\ {\isachardoublequoteopen}z\ {\isasymin}\ M{\isachardoublequoteclose}\ \isacommand{unfolding}\isamarkupfalse%
\ preds{\isacharunderscore}{\kern0pt}def\ \isacommand{by}\isamarkupfalse%
\ auto\ \isanewline
\ \ \ \ \isacommand{have}\isamarkupfalse%
\ yinM\ {\isacharcolon}{\kern0pt}\ {\isachardoublequoteopen}y\ {\isasymin}\ M{\isachardoublequoteclose}\ \isacommand{using}\isamarkupfalse%
\ yzH\ assms\ \isacommand{unfolding}\isamarkupfalse%
\ preds{\isacharunderscore}{\kern0pt}def\ \isacommand{by}\isamarkupfalse%
\ auto\ \isanewline
\ \ \ \ \isacommand{then}\isamarkupfalse%
\ \isacommand{have}\isamarkupfalse%
\ {\isachardoublequoteopen}{\isacharparenleft}{\kern0pt}{\isasymexists}y{\isasymin}M{\isachardot}{\kern0pt}\ {\isasymexists}z{\isasymin}M{\isachardot}{\kern0pt}\ v\ {\isacharequal}{\kern0pt}\ {\isasymlangle}z{\isacharcomma}{\kern0pt}\ y{\isasymrangle}\ {\isasymand}\ z\ {\isasymin}\ preds{\isacharparenleft}{\kern0pt}R{\isacharcomma}{\kern0pt}\ x{\isacharparenright}{\kern0pt}\ {\isasymand}\ {\isacharparenleft}{\kern0pt}y\ {\isasymin}\ preds{\isacharparenleft}{\kern0pt}R{\isacharcomma}{\kern0pt}\ x{\isacharparenright}{\kern0pt}\ {\isasymor}\ y\ {\isacharequal}{\kern0pt}\ x{\isacharparenright}{\kern0pt}\ {\isasymand}\ R{\isacharparenleft}{\kern0pt}z{\isacharcomma}{\kern0pt}\ y{\isacharparenright}{\kern0pt}{\isacharparenright}{\kern0pt}{\isachardoublequoteclose}\ \isacommand{using}\isamarkupfalse%
\ zinM\ yinM\ yzH\ \isacommand{by}\isamarkupfalse%
\ auto\ \isanewline
\ \ \ \ \isacommand{then}\isamarkupfalse%
\ \isacommand{show}\isamarkupfalse%
\ {\isachardoublequoteopen}v\ {\isasymin}\ S{\isachardoublequoteclose}\ \isacommand{using}\isamarkupfalse%
\ yzH\ pair{\isacharunderscore}{\kern0pt}in{\isacharunderscore}{\kern0pt}M{\isacharunderscore}{\kern0pt}iff\ zinM\ yinM\ assms{\isadigit{1}}\ \isacommand{by}\isamarkupfalse%
\ auto\isanewline
\ \ \isacommand{next}\isamarkupfalse%
\ \isanewline
\ \ \ \ \isacommand{assume}\isamarkupfalse%
\ assms{\isadigit{1}}{\isacharcolon}{\kern0pt}{\isachardoublequoteopen}S\ {\isacharequal}{\kern0pt}\ {\isacharbraceleft}{\kern0pt}{\isasymlangle}z{\isacharcomma}{\kern0pt}y{\isasymrangle}\ {\isasymin}\ preds{\isacharparenleft}{\kern0pt}R{\isacharcomma}{\kern0pt}\ x{\isacharparenright}{\kern0pt}\ {\isasymtimes}\ {\isacharparenleft}{\kern0pt}preds{\isacharparenleft}{\kern0pt}R{\isacharcomma}{\kern0pt}\ x{\isacharparenright}{\kern0pt}\ {\isasymunion}\ {\isacharbraceleft}{\kern0pt}x{\isacharbraceright}{\kern0pt}{\isacharparenright}{\kern0pt}\ {\isachardot}{\kern0pt}\ R{\isacharparenleft}{\kern0pt}z{\isacharcomma}{\kern0pt}\ y{\isacharparenright}{\kern0pt}{\isacharbraceright}{\kern0pt}{\isachardoublequoteclose}\ \isanewline
\ \ \ \ \isacommand{then}\isamarkupfalse%
\ \isacommand{show}\isamarkupfalse%
\ {\isachardoublequoteopen}{\isasymforall}v{\isasymin}M{\isachardot}{\kern0pt}\ v\ {\isasymin}\ S\ {\isasymlongleftrightarrow}\ {\isacharparenleft}{\kern0pt}{\isasymexists}y{\isasymin}M{\isachardot}{\kern0pt}\ {\isasymexists}z{\isasymin}M{\isachardot}{\kern0pt}\ v\ {\isacharequal}{\kern0pt}\ {\isasymlangle}z{\isacharcomma}{\kern0pt}\ y{\isasymrangle}\ {\isasymand}\ z\ {\isasymin}\ preds{\isacharparenleft}{\kern0pt}R{\isacharcomma}{\kern0pt}\ x{\isacharparenright}{\kern0pt}\ {\isasymand}\ {\isacharparenleft}{\kern0pt}y\ {\isasymin}\ preds{\isacharparenleft}{\kern0pt}R{\isacharcomma}{\kern0pt}\ x{\isacharparenright}{\kern0pt}\ {\isasymor}\ y\ {\isacharequal}{\kern0pt}\ x{\isacharparenright}{\kern0pt}\ {\isasymand}\ R{\isacharparenleft}{\kern0pt}z{\isacharcomma}{\kern0pt}\ y{\isacharparenright}{\kern0pt}{\isacharparenright}{\kern0pt}{\isachardoublequoteclose}\ \isanewline
\ \ \ \ \ \ \isacommand{unfolding}\isamarkupfalse%
\ preds{\isacharunderscore}{\kern0pt}def\isanewline
\ \ \ \ \ \ \isacommand{using}\isamarkupfalse%
\ assms\isanewline
\ \ \ \ \ \ \isacommand{by}\isamarkupfalse%
\ auto\isanewline
\ \ \isacommand{qed}\isamarkupfalse%
\isanewline
\isanewline
\ \ \isacommand{show}\isamarkupfalse%
\ {\isacharquery}{\kern0pt}thesis\ \isacommand{using}\isamarkupfalse%
\ I{\isadigit{1}}\ I{\isadigit{2}}\ I{\isadigit{3}}\ \isacommand{by}\isamarkupfalse%
\ auto\isanewline
\isacommand{qed}\isamarkupfalse%
%
\endisatagproof
{\isafoldproof}%
%
\isadelimproof
\isanewline
%
\endisadelimproof
\isanewline
\isacommand{lemma}\isamarkupfalse%
\ preds{\isacharunderscore}{\kern0pt}rel{\isacharunderscore}{\kern0pt}in{\isacharunderscore}{\kern0pt}M\ {\isacharcolon}{\kern0pt}\ \isanewline
\ \ \isakeyword{fixes}\ R\ Rfm\ x\isanewline
\ \ \isakeyword{assumes}\ {\isachardoublequoteopen}Relation{\isacharunderscore}{\kern0pt}fm{\isacharparenleft}{\kern0pt}R{\isacharcomma}{\kern0pt}\ Rfm{\isacharparenright}{\kern0pt}{\isachardoublequoteclose}\ {\isachardoublequoteopen}x\ {\isasymin}\ M{\isachardoublequoteclose}\ {\isachardoublequoteopen}preds{\isacharparenleft}{\kern0pt}R{\isacharcomma}{\kern0pt}\ x{\isacharparenright}{\kern0pt}\ {\isasymin}\ M{\isachardoublequoteclose}\isanewline
\ \ \isakeyword{shows}\ {\isachardoublequoteopen}preds{\isacharunderscore}{\kern0pt}rel{\isacharparenleft}{\kern0pt}R{\isacharcomma}{\kern0pt}\ x{\isacharparenright}{\kern0pt}\ {\isasymin}\ M{\isachardoublequoteclose}\isanewline
%
\isadelimproof
%
\endisadelimproof
%
\isatagproof
\isacommand{proof}\isamarkupfalse%
\ {\isacharminus}{\kern0pt}\ \isanewline
\ \ \isacommand{define}\isamarkupfalse%
\ fm\ \isakeyword{where}\ {\isachardoublequoteopen}fm\ {\isasymequiv}\ Exists{\isacharparenleft}{\kern0pt}Exists{\isacharparenleft}{\kern0pt}And{\isacharparenleft}{\kern0pt}pair{\isacharunderscore}{\kern0pt}fm{\isacharparenleft}{\kern0pt}{\isadigit{0}}{\isacharcomma}{\kern0pt}\ {\isadigit{1}}{\isacharcomma}{\kern0pt}\ {\isadigit{2}}{\isacharparenright}{\kern0pt}{\isacharcomma}{\kern0pt}\ Rfm{\isacharparenright}{\kern0pt}{\isacharparenright}{\kern0pt}{\isacharparenright}{\kern0pt}{\isachardoublequoteclose}\isanewline
\ \ \isacommand{have}\isamarkupfalse%
\ sats{\isacharunderscore}{\kern0pt}iff\ {\isacharcolon}{\kern0pt}\ \ {\isachardoublequoteopen}{\isasymAnd}v{\isachardot}{\kern0pt}\ v\ {\isasymin}\ M\ {\isasymLongrightarrow}\ sats{\isacharparenleft}{\kern0pt}M{\isacharcomma}{\kern0pt}\ fm{\isacharcomma}{\kern0pt}\ {\isacharbrackleft}{\kern0pt}v{\isacharbrackright}{\kern0pt}{\isacharparenright}{\kern0pt}\ {\isasymlongleftrightarrow}\ v\ {\isasymin}\ Rrel{\isacharparenleft}{\kern0pt}R{\isacharcomma}{\kern0pt}\ M{\isacharparenright}{\kern0pt}{\isachardoublequoteclose}\ \isanewline
\ \ \ \ \isacommand{unfolding}\isamarkupfalse%
\ fm{\isacharunderscore}{\kern0pt}def\ \isanewline
\ \ \ \ \isacommand{apply}\isamarkupfalse%
{\isacharparenleft}{\kern0pt}rule{\isacharunderscore}{\kern0pt}tac\ Q{\isacharequal}{\kern0pt}{\isachardoublequoteopen}{\isasymexists}y\ {\isasymin}\ M{\isachardot}{\kern0pt}\ {\isasymexists}z\ {\isasymin}\ M{\isachardot}{\kern0pt}\ v\ {\isacharequal}{\kern0pt}\ {\isacharless}{\kern0pt}z{\isacharcomma}{\kern0pt}\ y{\isachargreater}{\kern0pt}\ {\isasymand}\ R{\isacharparenleft}{\kern0pt}z{\isacharcomma}{\kern0pt}\ y{\isacharparenright}{\kern0pt}{\isachardoublequoteclose}\ \isakeyword{in}\ iff{\isacharunderscore}{\kern0pt}trans{\isacharparenright}{\kern0pt}\isanewline
\ \ \ \ \isacommand{using}\isamarkupfalse%
\ assms\ \isanewline
\ \ \ \ \isacommand{apply}\isamarkupfalse%
\ simp\isanewline
\ \ \ \ \isacommand{apply}\isamarkupfalse%
{\isacharparenleft}{\kern0pt}rule\ bex{\isacharunderscore}{\kern0pt}iff{\isacharcomma}{\kern0pt}\ rule\ bex{\isacharunderscore}{\kern0pt}iff{\isacharcomma}{\kern0pt}\ rule\ iff{\isacharunderscore}{\kern0pt}conjI{\isacharcomma}{\kern0pt}\ simp{\isacharparenright}{\kern0pt}\isanewline
\ \ \ \ \isacommand{apply}\isamarkupfalse%
{\isacharparenleft}{\kern0pt}rule\ iff{\isacharunderscore}{\kern0pt}flip{\isacharcomma}{\kern0pt}\ rule\ iff{\isacharunderscore}{\kern0pt}trans{\isacharcomma}{\kern0pt}\ rename{\isacharunderscore}{\kern0pt}tac\ v\ a\ b{\isacharcomma}{\kern0pt}\ rule{\isacharunderscore}{\kern0pt}tac\ Rfm{\isacharequal}{\kern0pt}Rfm\ \isakeyword{and}\ R{\isacharequal}{\kern0pt}R\ \isakeyword{in}\ Relation{\isacharunderscore}{\kern0pt}fmD{\isacharparenright}{\kern0pt}\isanewline
\ \ \ \ \isacommand{using}\isamarkupfalse%
\ assms\ pair{\isacharunderscore}{\kern0pt}in{\isacharunderscore}{\kern0pt}M{\isacharunderscore}{\kern0pt}iff\isanewline
\ \ \ \ \isacommand{unfolding}\isamarkupfalse%
\ Rrel{\isacharunderscore}{\kern0pt}def\ \isanewline
\ \ \ \ \isacommand{by}\isamarkupfalse%
\ auto\isanewline
\ \ \isanewline
\ \ \isacommand{have}\isamarkupfalse%
\ {\isachardoublequoteopen}separation{\isacharparenleft}{\kern0pt}{\isacharhash}{\kern0pt}{\isacharhash}{\kern0pt}M{\isacharcomma}{\kern0pt}\ {\isasymlambda}v{\isachardot}{\kern0pt}\ sats{\isacharparenleft}{\kern0pt}M{\isacharcomma}{\kern0pt}\ fm{\isacharcomma}{\kern0pt}\ {\isacharbrackleft}{\kern0pt}v{\isacharbrackright}{\kern0pt}\ {\isacharat}{\kern0pt}\ {\isacharbrackleft}{\kern0pt}{\isacharbrackright}{\kern0pt}{\isacharparenright}{\kern0pt}{\isacharparenright}{\kern0pt}{\isachardoublequoteclose}\ \isanewline
\ \ \ \ \isacommand{apply}\isamarkupfalse%
{\isacharparenleft}{\kern0pt}rule\ separation{\isacharunderscore}{\kern0pt}ax{\isacharparenright}{\kern0pt}\isanewline
\ \ \ \ \isacommand{unfolding}\isamarkupfalse%
\ fm{\isacharunderscore}{\kern0pt}def\isanewline
\ \ \ \ \isacommand{using}\isamarkupfalse%
\ assms\ Relation{\isacharunderscore}{\kern0pt}fm{\isacharunderscore}{\kern0pt}def\isanewline
\ \ \ \ \ \ \isacommand{apply}\isamarkupfalse%
\ {\isacharparenleft}{\kern0pt}simp{\isacharcomma}{\kern0pt}\ force{\isacharcomma}{\kern0pt}\ simp{\isacharcomma}{\kern0pt}\ simp{\isacharparenright}{\kern0pt}\isanewline
\ \ \ \ \isacommand{by}\isamarkupfalse%
{\isacharparenleft}{\kern0pt}simp\ del{\isacharcolon}{\kern0pt}FOL{\isacharunderscore}{\kern0pt}sats{\isacharunderscore}{\kern0pt}iff\ pair{\isacharunderscore}{\kern0pt}abs\ add{\isacharcolon}{\kern0pt}\ fm{\isacharunderscore}{\kern0pt}defs\ nat{\isacharunderscore}{\kern0pt}simp{\isacharunderscore}{\kern0pt}union{\isacharparenright}{\kern0pt}\ \ \isanewline
\isanewline
\ \ \isacommand{then}\isamarkupfalse%
\ \isacommand{have}\isamarkupfalse%
\ H\ {\isacharcolon}{\kern0pt}\ {\isachardoublequoteopen}{\isacharbraceleft}{\kern0pt}\ v\ {\isasymin}\ preds{\isacharparenleft}{\kern0pt}R{\isacharcomma}{\kern0pt}\ x{\isacharparenright}{\kern0pt}\ {\isasymtimes}\ {\isacharparenleft}{\kern0pt}preds{\isacharparenleft}{\kern0pt}R{\isacharcomma}{\kern0pt}\ x{\isacharparenright}{\kern0pt}\ {\isasymunion}\ {\isacharbraceleft}{\kern0pt}x{\isacharbraceright}{\kern0pt}{\isacharparenright}{\kern0pt}{\isachardot}{\kern0pt}\ sats{\isacharparenleft}{\kern0pt}M{\isacharcomma}{\kern0pt}\ fm{\isacharcomma}{\kern0pt}\ {\isacharbrackleft}{\kern0pt}v{\isacharbrackright}{\kern0pt}\ {\isacharat}{\kern0pt}\ {\isacharbrackleft}{\kern0pt}{\isacharbrackright}{\kern0pt}{\isacharparenright}{\kern0pt}\ {\isacharbraceright}{\kern0pt}\ {\isasymin}\ M{\isachardoublequoteclose}\isanewline
\ \ \ \ \isacommand{apply}\isamarkupfalse%
{\isacharparenleft}{\kern0pt}rule{\isacharunderscore}{\kern0pt}tac\ separation{\isacharunderscore}{\kern0pt}notation{\isacharcomma}{\kern0pt}\ simp{\isacharparenright}{\kern0pt}\isanewline
\ \ \ \ \isacommand{using}\isamarkupfalse%
\ assms\ cartprod{\isacharunderscore}{\kern0pt}closed\ Un{\isacharunderscore}{\kern0pt}closed\ singleton{\isacharunderscore}{\kern0pt}in{\isacharunderscore}{\kern0pt}M{\isacharunderscore}{\kern0pt}iff\ \isanewline
\ \ \ \ \isacommand{by}\isamarkupfalse%
\ auto\isanewline
\isanewline
\ \ \isacommand{have}\isamarkupfalse%
\ {\isachardoublequoteopen}{\isacharbraceleft}{\kern0pt}\ v\ {\isasymin}\ preds{\isacharparenleft}{\kern0pt}R{\isacharcomma}{\kern0pt}\ x{\isacharparenright}{\kern0pt}\ {\isasymtimes}\ {\isacharparenleft}{\kern0pt}preds{\isacharparenleft}{\kern0pt}R{\isacharcomma}{\kern0pt}\ x{\isacharparenright}{\kern0pt}\ {\isasymunion}\ {\isacharbraceleft}{\kern0pt}x{\isacharbraceright}{\kern0pt}{\isacharparenright}{\kern0pt}{\isachardot}{\kern0pt}\ sats{\isacharparenleft}{\kern0pt}M{\isacharcomma}{\kern0pt}\ fm{\isacharcomma}{\kern0pt}\ {\isacharbrackleft}{\kern0pt}v{\isacharbrackright}{\kern0pt}\ {\isacharat}{\kern0pt}\ {\isacharbrackleft}{\kern0pt}{\isacharbrackright}{\kern0pt}{\isacharparenright}{\kern0pt}\ {\isacharbraceright}{\kern0pt}\ {\isacharequal}{\kern0pt}\ {\isacharbraceleft}{\kern0pt}\ v\ {\isasymin}\ preds{\isacharparenleft}{\kern0pt}R{\isacharcomma}{\kern0pt}\ x{\isacharparenright}{\kern0pt}\ {\isasymtimes}\ {\isacharparenleft}{\kern0pt}preds{\isacharparenleft}{\kern0pt}R{\isacharcomma}{\kern0pt}\ x{\isacharparenright}{\kern0pt}\ {\isasymunion}\ {\isacharbraceleft}{\kern0pt}x{\isacharbraceright}{\kern0pt}{\isacharparenright}{\kern0pt}{\isachardot}{\kern0pt}\ v\ {\isasymin}\ Rrel{\isacharparenleft}{\kern0pt}R{\isacharcomma}{\kern0pt}\ M{\isacharparenright}{\kern0pt}\ {\isacharbraceright}{\kern0pt}{\isachardoublequoteclose}\ \isanewline
\ \ \ \ \isacommand{apply}\isamarkupfalse%
{\isacharparenleft}{\kern0pt}rule\ iff{\isacharunderscore}{\kern0pt}eq{\isacharcomma}{\kern0pt}\ simp{\isacharcomma}{\kern0pt}\ rule\ sats{\isacharunderscore}{\kern0pt}iff{\isacharparenright}{\kern0pt}\isanewline
\ \ \ \ \isacommand{using}\isamarkupfalse%
\ assms\ cartprod{\isacharunderscore}{\kern0pt}closed\ Un{\isacharunderscore}{\kern0pt}closed\ singleton{\isacharunderscore}{\kern0pt}in{\isacharunderscore}{\kern0pt}M{\isacharunderscore}{\kern0pt}iff\ transM\isanewline
\ \ \ \ \isacommand{by}\isamarkupfalse%
\ auto\isanewline
\isanewline
\ \ \isacommand{also}\isamarkupfalse%
\ \isacommand{have}\isamarkupfalse%
\ {\isachardoublequoteopen}{\isachardot}{\kern0pt}{\isachardot}{\kern0pt}{\isachardot}{\kern0pt}\ {\isacharequal}{\kern0pt}\ preds{\isacharunderscore}{\kern0pt}rel{\isacharparenleft}{\kern0pt}R{\isacharcomma}{\kern0pt}\ x{\isacharparenright}{\kern0pt}{\isachardoublequoteclose}\isanewline
\ \ \ \ \isacommand{unfolding}\isamarkupfalse%
\ preds{\isacharunderscore}{\kern0pt}rel{\isacharunderscore}{\kern0pt}def\ Rrel{\isacharunderscore}{\kern0pt}def\isanewline
\ \ \ \ \isacommand{using}\isamarkupfalse%
\ assms\ transM\ \isanewline
\ \ \ \ \isacommand{by}\isamarkupfalse%
\ auto\isanewline
\isanewline
\ \ \isacommand{finally}\isamarkupfalse%
\ \isacommand{show}\isamarkupfalse%
\ {\isacharquery}{\kern0pt}thesis\ \isacommand{using}\isamarkupfalse%
\ H\ \isacommand{by}\isamarkupfalse%
\ auto\isanewline
\isacommand{qed}\isamarkupfalse%
%
\endisatagproof
{\isafoldproof}%
%
\isadelimproof
\isanewline
%
\endisadelimproof
\ \ \ \isanewline
\isacommand{lemma}\isamarkupfalse%
\ trans{\isacharunderscore}{\kern0pt}preds{\isacharunderscore}{\kern0pt}rel\ {\isacharcolon}{\kern0pt}\ {\isachardoublequoteopen}t\ {\isasymin}\ M\ {\isasymLongrightarrow}\ trans{\isacharparenleft}{\kern0pt}Rrel{\isacharparenleft}{\kern0pt}R{\isacharcomma}{\kern0pt}\ M{\isacharparenright}{\kern0pt}{\isacharparenright}{\kern0pt}\ {\isasymLongrightarrow}\ trans{\isacharparenleft}{\kern0pt}preds{\isacharunderscore}{\kern0pt}rel{\isacharparenleft}{\kern0pt}R{\isacharcomma}{\kern0pt}\ t{\isacharparenright}{\kern0pt}{\isacharparenright}{\kern0pt}{\isachardoublequoteclose}\ \isanewline
%
\isadelimproof
\ \ %
\endisadelimproof
%
\isatagproof
\isacommand{unfolding}\isamarkupfalse%
\ trans{\isacharunderscore}{\kern0pt}def\ \isanewline
\isacommand{proof}\isamarkupfalse%
{\isacharparenleft}{\kern0pt}clarify{\isacharparenright}{\kern0pt}\isanewline
\ \ \isacommand{fix}\isamarkupfalse%
\ x\ y\ z\ \isacommand{assume}\isamarkupfalse%
\ assms\ {\isacharcolon}{\kern0pt}\ {\isachardoublequoteopen}t\ {\isasymin}\ M{\isachardoublequoteclose}\ {\isachardoublequoteopen}{\isasymforall}x\ y\ z{\isachardot}{\kern0pt}\ {\isasymlangle}x{\isacharcomma}{\kern0pt}\ y{\isasymrangle}\ {\isasymin}\ Rrel{\isacharparenleft}{\kern0pt}R{\isacharcomma}{\kern0pt}\ M{\isacharparenright}{\kern0pt}\ {\isasymlongrightarrow}\ {\isasymlangle}y{\isacharcomma}{\kern0pt}\ z{\isasymrangle}\ {\isasymin}\ Rrel{\isacharparenleft}{\kern0pt}R{\isacharcomma}{\kern0pt}\ M{\isacharparenright}{\kern0pt}\ {\isasymlongrightarrow}\ {\isasymlangle}x{\isacharcomma}{\kern0pt}\ z{\isasymrangle}\ {\isasymin}\ Rrel{\isacharparenleft}{\kern0pt}R{\isacharcomma}{\kern0pt}\ M{\isacharparenright}{\kern0pt}{\isachardoublequoteclose}\ {\isachardoublequoteopen}{\isacharless}{\kern0pt}x{\isacharcomma}{\kern0pt}\ y{\isachargreater}{\kern0pt}\ {\isasymin}\ preds{\isacharunderscore}{\kern0pt}rel{\isacharparenleft}{\kern0pt}R{\isacharcomma}{\kern0pt}\ t{\isacharparenright}{\kern0pt}{\isachardoublequoteclose}\ {\isachardoublequoteopen}{\isacharless}{\kern0pt}y{\isacharcomma}{\kern0pt}\ z{\isachargreater}{\kern0pt}\ {\isasymin}\ preds{\isacharunderscore}{\kern0pt}rel{\isacharparenleft}{\kern0pt}R{\isacharcomma}{\kern0pt}\ t{\isacharparenright}{\kern0pt}{\isachardoublequoteclose}\ \isanewline
\ \ \isacommand{then}\isamarkupfalse%
\ \isacommand{have}\isamarkupfalse%
\ xin\ {\isacharcolon}{\kern0pt}\ {\isachardoublequoteopen}x\ {\isasymin}\ preds{\isacharparenleft}{\kern0pt}R{\isacharcomma}{\kern0pt}\ t{\isacharparenright}{\kern0pt}{\isachardoublequoteclose}\ \isacommand{using}\isamarkupfalse%
\ assms\ \isacommand{unfolding}\isamarkupfalse%
\ preds{\isacharunderscore}{\kern0pt}rel{\isacharunderscore}{\kern0pt}def\ \isacommand{by}\isamarkupfalse%
\ auto\ \isanewline
\ \ \isacommand{have}\isamarkupfalse%
\ zin\ {\isacharcolon}{\kern0pt}\ {\isachardoublequoteopen}z\ {\isasymin}\ preds{\isacharparenleft}{\kern0pt}R{\isacharcomma}{\kern0pt}\ t{\isacharparenright}{\kern0pt}\ {\isasymunion}\ {\isacharbraceleft}{\kern0pt}t{\isacharbraceright}{\kern0pt}{\isachardoublequoteclose}\ \isacommand{using}\isamarkupfalse%
\ assms\ \isacommand{unfolding}\isamarkupfalse%
\ preds{\isacharunderscore}{\kern0pt}rel{\isacharunderscore}{\kern0pt}def\ \isacommand{by}\isamarkupfalse%
\ auto\isanewline
\ \ \isacommand{have}\isamarkupfalse%
\ H{\isadigit{1}}{\isacharcolon}{\kern0pt}{\isachardoublequoteopen}{\isacharless}{\kern0pt}x{\isacharcomma}{\kern0pt}\ y{\isachargreater}{\kern0pt}\ {\isasymin}\ Rrel{\isacharparenleft}{\kern0pt}R{\isacharcomma}{\kern0pt}\ M{\isacharparenright}{\kern0pt}{\isachardoublequoteclose}\ \isacommand{apply}\isamarkupfalse%
{\isacharparenleft}{\kern0pt}rule\ subsetD{\isacharparenright}{\kern0pt}\ \isacommand{using}\isamarkupfalse%
\ preds{\isacharunderscore}{\kern0pt}rel{\isacharunderscore}{\kern0pt}subset\ assms\ \isacommand{by}\isamarkupfalse%
\ auto\isanewline
\ \ \isacommand{have}\isamarkupfalse%
\ H{\isadigit{2}}{\isacharcolon}{\kern0pt}\ {\isachardoublequoteopen}{\isacharless}{\kern0pt}y{\isacharcomma}{\kern0pt}\ z{\isachargreater}{\kern0pt}\ {\isasymin}\ Rrel{\isacharparenleft}{\kern0pt}R{\isacharcomma}{\kern0pt}\ M{\isacharparenright}{\kern0pt}{\isachardoublequoteclose}\ \isacommand{apply}\isamarkupfalse%
{\isacharparenleft}{\kern0pt}rule\ subsetD{\isacharparenright}{\kern0pt}\ \isacommand{using}\isamarkupfalse%
\ preds{\isacharunderscore}{\kern0pt}rel{\isacharunderscore}{\kern0pt}subset\ assms\ \isacommand{by}\isamarkupfalse%
\ auto\isanewline
\ \ \isacommand{have}\isamarkupfalse%
\ {\isachardoublequoteopen}{\isacharless}{\kern0pt}x{\isacharcomma}{\kern0pt}\ z{\isachargreater}{\kern0pt}\ {\isasymin}\ Rrel{\isacharparenleft}{\kern0pt}R{\isacharcomma}{\kern0pt}\ M{\isacharparenright}{\kern0pt}{\isachardoublequoteclose}\ \isacommand{using}\isamarkupfalse%
\ H{\isadigit{1}}\ H{\isadigit{2}}\ assms\ \isacommand{by}\isamarkupfalse%
\ auto\ \isanewline
\ \ \isacommand{then}\isamarkupfalse%
\ \isacommand{have}\isamarkupfalse%
\ {\isachardoublequoteopen}R{\isacharparenleft}{\kern0pt}x{\isacharcomma}{\kern0pt}\ z{\isacharparenright}{\kern0pt}{\isachardoublequoteclose}\ \isacommand{unfolding}\isamarkupfalse%
\ Rrel{\isacharunderscore}{\kern0pt}def\ \isacommand{by}\isamarkupfalse%
\ auto\ \isanewline
\ \ \isacommand{then}\isamarkupfalse%
\ \isacommand{show}\isamarkupfalse%
\ {\isachardoublequoteopen}{\isasymlangle}x{\isacharcomma}{\kern0pt}\ z{\isasymrangle}\ {\isasymin}\ preds{\isacharunderscore}{\kern0pt}rel{\isacharparenleft}{\kern0pt}R{\isacharcomma}{\kern0pt}\ t{\isacharparenright}{\kern0pt}{\isachardoublequoteclose}\ \isanewline
\ \ \ \ \isacommand{unfolding}\isamarkupfalse%
\ preds{\isacharunderscore}{\kern0pt}rel{\isacharunderscore}{\kern0pt}def\ \isanewline
\ \ \ \ \isacommand{using}\isamarkupfalse%
\ xin\ zin\ \isacommand{by}\isamarkupfalse%
\ auto\isanewline
\isacommand{qed}\isamarkupfalse%
%
\endisatagproof
{\isafoldproof}%
%
\isadelimproof
\isanewline
%
\endisadelimproof
\isanewline
\isacommand{lemma}\isamarkupfalse%
\ wf{\isacharunderscore}{\kern0pt}preds{\isacharunderscore}{\kern0pt}rel\ {\isacharcolon}{\kern0pt}\ {\isachardoublequoteopen}x\ {\isasymin}\ M\ {\isasymLongrightarrow}\ wf{\isacharparenleft}{\kern0pt}Rrel{\isacharparenleft}{\kern0pt}R{\isacharcomma}{\kern0pt}\ M{\isacharparenright}{\kern0pt}{\isacharparenright}{\kern0pt}\ {\isasymLongrightarrow}\ wf{\isacharparenleft}{\kern0pt}preds{\isacharunderscore}{\kern0pt}rel{\isacharparenleft}{\kern0pt}R{\isacharcomma}{\kern0pt}\ x{\isacharparenright}{\kern0pt}{\isacharparenright}{\kern0pt}{\isachardoublequoteclose}\ \isanewline
%
\isadelimproof
\ \ %
\endisadelimproof
%
\isatagproof
\isacommand{apply}\isamarkupfalse%
{\isacharparenleft}{\kern0pt}rule{\isacharunderscore}{\kern0pt}tac\ s{\isacharequal}{\kern0pt}{\isachardoublequoteopen}Rrel{\isacharparenleft}{\kern0pt}R{\isacharcomma}{\kern0pt}\ M{\isacharparenright}{\kern0pt}{\isachardoublequoteclose}\ \isakeyword{in}\ \ wf{\isacharunderscore}{\kern0pt}subset{\isacharparenright}{\kern0pt}\isanewline
\ \ \isacommand{using}\isamarkupfalse%
\ preds{\isacharunderscore}{\kern0pt}rel{\isacharunderscore}{\kern0pt}subset\ \isanewline
\ \ \isacommand{by}\isamarkupfalse%
\ auto%
\endisatagproof
{\isafoldproof}%
%
\isadelimproof
\isanewline
%
\endisadelimproof
\isanewline
\isacommand{lemma}\isamarkupfalse%
\ preds{\isacharunderscore}{\kern0pt}rel{\isacharunderscore}{\kern0pt}vimage{\isacharunderscore}{\kern0pt}eq\ {\isacharcolon}{\kern0pt}\ \isanewline
\ \ {\isachardoublequoteopen}trans{\isacharparenleft}{\kern0pt}Rrel{\isacharparenleft}{\kern0pt}R{\isacharcomma}{\kern0pt}\ M{\isacharparenright}{\kern0pt}{\isacharparenright}{\kern0pt}\ {\isasymLongrightarrow}\ {\isacharless}{\kern0pt}y{\isacharcomma}{\kern0pt}\ x{\isachargreater}{\kern0pt}\ {\isasymin}\ Rrel{\isacharparenleft}{\kern0pt}R{\isacharcomma}{\kern0pt}\ M{\isacharparenright}{\kern0pt}\ {\isasymLongrightarrow}\ preds{\isacharunderscore}{\kern0pt}rel{\isacharparenleft}{\kern0pt}R{\isacharcomma}{\kern0pt}\ x{\isacharparenright}{\kern0pt}\ {\isacharminus}{\kern0pt}{\isacharbackquote}{\kern0pt}{\isacharbackquote}{\kern0pt}\ {\isacharbraceleft}{\kern0pt}y{\isacharbraceright}{\kern0pt}\ {\isacharequal}{\kern0pt}\ Rrel{\isacharparenleft}{\kern0pt}R{\isacharcomma}{\kern0pt}\ M{\isacharparenright}{\kern0pt}\ {\isacharminus}{\kern0pt}{\isacharbackquote}{\kern0pt}{\isacharbackquote}{\kern0pt}\ {\isacharbraceleft}{\kern0pt}y{\isacharbraceright}{\kern0pt}{\isachardoublequoteclose}\ \isanewline
%
\isadelimproof
%
\endisadelimproof
%
\isatagproof
\isacommand{proof}\isamarkupfalse%
{\isacharparenleft}{\kern0pt}rule\ equality{\isacharunderscore}{\kern0pt}iffI{\isacharcomma}{\kern0pt}\ rule\ iffI{\isacharparenright}{\kern0pt}\isanewline
\ \ \isacommand{fix}\isamarkupfalse%
\ z\ \isacommand{assume}\isamarkupfalse%
\ assms\ {\isacharcolon}{\kern0pt}\ {\isachardoublequoteopen}z\ {\isasymin}\ preds{\isacharunderscore}{\kern0pt}rel{\isacharparenleft}{\kern0pt}R{\isacharcomma}{\kern0pt}\ x{\isacharparenright}{\kern0pt}\ {\isacharminus}{\kern0pt}{\isacharbackquote}{\kern0pt}{\isacharbackquote}{\kern0pt}\ {\isacharbraceleft}{\kern0pt}y{\isacharbraceright}{\kern0pt}{\isachardoublequoteclose}\ {\isachardoublequoteopen}{\isacharless}{\kern0pt}y{\isacharcomma}{\kern0pt}\ x{\isachargreater}{\kern0pt}\ {\isasymin}\ Rrel{\isacharparenleft}{\kern0pt}R{\isacharcomma}{\kern0pt}\ M{\isacharparenright}{\kern0pt}{\isachardoublequoteclose}\isanewline
\ \ \isacommand{then}\isamarkupfalse%
\ \isacommand{have}\isamarkupfalse%
\ {\isachardoublequoteopen}{\isacharless}{\kern0pt}z{\isacharcomma}{\kern0pt}\ y{\isachargreater}{\kern0pt}\ {\isasymin}\ preds{\isacharunderscore}{\kern0pt}rel{\isacharparenleft}{\kern0pt}R{\isacharcomma}{\kern0pt}\ x{\isacharparenright}{\kern0pt}{\isachardoublequoteclose}\ \isacommand{by}\isamarkupfalse%
\ auto\ \isanewline
\ \ \isacommand{then}\isamarkupfalse%
\ \isacommand{have}\isamarkupfalse%
\ {\isachardoublequoteopen}{\isacharless}{\kern0pt}z{\isacharcomma}{\kern0pt}\ y{\isachargreater}{\kern0pt}\ {\isasymin}\ Rrel{\isacharparenleft}{\kern0pt}R{\isacharcomma}{\kern0pt}\ M{\isacharparenright}{\kern0pt}{\isachardoublequoteclose}\ \isanewline
\ \ \ \ \isacommand{apply}\isamarkupfalse%
{\isacharparenleft}{\kern0pt}rule{\isacharunderscore}{\kern0pt}tac\ x{\isacharequal}{\kern0pt}x\ \isakeyword{in}\ preds{\isacharunderscore}{\kern0pt}rel{\isacharunderscore}{\kern0pt}subset{\isacharprime}{\kern0pt}{\isacharparenright}{\kern0pt}\isanewline
\ \ \ \ \isacommand{using}\isamarkupfalse%
\ assms\ \isanewline
\ \ \ \ \isacommand{unfolding}\isamarkupfalse%
\ Rrel{\isacharunderscore}{\kern0pt}def\ \isanewline
\ \ \ \ \isacommand{by}\isamarkupfalse%
\ auto\isanewline
\ \ \isacommand{then}\isamarkupfalse%
\ \isacommand{show}\isamarkupfalse%
\ {\isachardoublequoteopen}z\ {\isasymin}\ Rrel{\isacharparenleft}{\kern0pt}R{\isacharcomma}{\kern0pt}\ M{\isacharparenright}{\kern0pt}\ {\isacharminus}{\kern0pt}{\isacharbackquote}{\kern0pt}{\isacharbackquote}{\kern0pt}\ {\isacharbraceleft}{\kern0pt}y{\isacharbraceright}{\kern0pt}{\isachardoublequoteclose}\ \isacommand{by}\isamarkupfalse%
\ auto\ \isanewline
\isacommand{next}\isamarkupfalse%
\ \isanewline
\ \ \isacommand{fix}\isamarkupfalse%
\ z\ \isacommand{assume}\isamarkupfalse%
\ assms\ {\isacharcolon}{\kern0pt}\ {\isachardoublequoteopen}{\isacharless}{\kern0pt}y{\isacharcomma}{\kern0pt}\ x{\isachargreater}{\kern0pt}\ {\isasymin}\ Rrel{\isacharparenleft}{\kern0pt}R{\isacharcomma}{\kern0pt}\ M{\isacharparenright}{\kern0pt}{\isachardoublequoteclose}\ \ {\isachardoublequoteopen}z\ {\isasymin}\ Rrel{\isacharparenleft}{\kern0pt}R{\isacharcomma}{\kern0pt}\ M{\isacharparenright}{\kern0pt}\ {\isacharminus}{\kern0pt}{\isacharbackquote}{\kern0pt}{\isacharbackquote}{\kern0pt}\ {\isacharbraceleft}{\kern0pt}y{\isacharbraceright}{\kern0pt}{\isachardoublequoteclose}\ {\isachardoublequoteopen}trans{\isacharparenleft}{\kern0pt}Rrel{\isacharparenleft}{\kern0pt}R{\isacharcomma}{\kern0pt}\ M{\isacharparenright}{\kern0pt}{\isacharparenright}{\kern0pt}{\isachardoublequoteclose}\ \isanewline
\ \ \isacommand{then}\isamarkupfalse%
\ \isacommand{have}\isamarkupfalse%
\ rel\ {\isacharcolon}{\kern0pt}\ {\isachardoublequoteopen}{\isacharless}{\kern0pt}z{\isacharcomma}{\kern0pt}\ y{\isachargreater}{\kern0pt}\ {\isasymin}\ Rrel{\isacharparenleft}{\kern0pt}R{\isacharcomma}{\kern0pt}\ M{\isacharparenright}{\kern0pt}{\isachardoublequoteclose}\ \isacommand{by}\isamarkupfalse%
\ auto\ \isanewline
\ \ \isacommand{then}\isamarkupfalse%
\ \isacommand{have}\isamarkupfalse%
\ rel{\isacharprime}{\kern0pt}\ {\isacharcolon}{\kern0pt}\ {\isachardoublequoteopen}{\isacharless}{\kern0pt}z{\isacharcomma}{\kern0pt}\ x{\isachargreater}{\kern0pt}\ {\isasymin}\ Rrel{\isacharparenleft}{\kern0pt}R{\isacharcomma}{\kern0pt}\ M{\isacharparenright}{\kern0pt}{\isachardoublequoteclose}\ \isacommand{using}\isamarkupfalse%
\ assms\ trans{\isacharunderscore}{\kern0pt}def\ \isacommand{by}\isamarkupfalse%
\ auto\ \isanewline
\isanewline
\ \ \isacommand{have}\isamarkupfalse%
\ yin\ {\isacharcolon}{\kern0pt}\ {\isachardoublequoteopen}y\ {\isasymin}\ preds{\isacharparenleft}{\kern0pt}R{\isacharcomma}{\kern0pt}\ x{\isacharparenright}{\kern0pt}{\isachardoublequoteclose}\ \isanewline
\ \ \ \ \isacommand{unfolding}\isamarkupfalse%
\ preds{\isacharunderscore}{\kern0pt}def\isanewline
\ \ \ \ \isacommand{using}\isamarkupfalse%
\ Rrel{\isacharunderscore}{\kern0pt}def\ assms\ \isanewline
\ \ \ \ \isacommand{by}\isamarkupfalse%
\ auto\isanewline
\ \ \isacommand{have}\isamarkupfalse%
\ zin{\isacharcolon}{\kern0pt}\ {\isachardoublequoteopen}z\ {\isasymin}\ preds{\isacharparenleft}{\kern0pt}R{\isacharcomma}{\kern0pt}\ x{\isacharparenright}{\kern0pt}{\isachardoublequoteclose}\isanewline
\ \ \ \ \isacommand{unfolding}\isamarkupfalse%
\ preds{\isacharunderscore}{\kern0pt}def\ \isanewline
\ \ \ \ \isacommand{using}\isamarkupfalse%
\ rel{\isacharprime}{\kern0pt}\ assms\ Rrel{\isacharunderscore}{\kern0pt}def\ \isanewline
\ \ \ \ \isacommand{by}\isamarkupfalse%
\ auto\isanewline
\isanewline
\ \ \isacommand{then}\isamarkupfalse%
\ \isacommand{have}\isamarkupfalse%
\ {\isachardoublequoteopen}{\isacharless}{\kern0pt}z{\isacharcomma}{\kern0pt}\ y{\isachargreater}{\kern0pt}\ {\isasymin}\ preds{\isacharunderscore}{\kern0pt}rel{\isacharparenleft}{\kern0pt}R{\isacharcomma}{\kern0pt}\ x{\isacharparenright}{\kern0pt}{\isachardoublequoteclose}\ \isanewline
\ \ \ \ \isacommand{unfolding}\isamarkupfalse%
\ preds{\isacharunderscore}{\kern0pt}rel{\isacharunderscore}{\kern0pt}def\ \isanewline
\ \ \ \ \isacommand{using}\isamarkupfalse%
\ yin\ zin\ rel\ Rrel{\isacharunderscore}{\kern0pt}def\ \isanewline
\ \ \ \ \isacommand{by}\isamarkupfalse%
\ auto\isanewline
\ \ \isacommand{then}\isamarkupfalse%
\ \isacommand{show}\isamarkupfalse%
\ {\isachardoublequoteopen}z\ {\isasymin}\ preds{\isacharunderscore}{\kern0pt}rel{\isacharparenleft}{\kern0pt}R{\isacharcomma}{\kern0pt}\ x{\isacharparenright}{\kern0pt}\ {\isacharminus}{\kern0pt}{\isacharbackquote}{\kern0pt}{\isacharbackquote}{\kern0pt}\ {\isacharbraceleft}{\kern0pt}y{\isacharbraceright}{\kern0pt}{\isachardoublequoteclose}\ \isacommand{by}\isamarkupfalse%
\ auto\isanewline
\isacommand{qed}\isamarkupfalse%
%
\endisatagproof
{\isafoldproof}%
%
\isadelimproof
\isanewline
%
\endisadelimproof
\isanewline
\isacommand{lemma}\isamarkupfalse%
\ preds{\isacharunderscore}{\kern0pt}rel{\isacharunderscore}{\kern0pt}vimage{\isacharunderscore}{\kern0pt}eq{\isacharprime}{\kern0pt}\ {\isacharcolon}{\kern0pt}\ \isanewline
\ \ {\isachardoublequoteopen}x\ {\isasymin}\ M\ {\isasymLongrightarrow}\ preds{\isacharunderscore}{\kern0pt}rel{\isacharparenleft}{\kern0pt}R{\isacharcomma}{\kern0pt}\ x{\isacharparenright}{\kern0pt}\ {\isacharminus}{\kern0pt}{\isacharbackquote}{\kern0pt}{\isacharbackquote}{\kern0pt}\ {\isacharbraceleft}{\kern0pt}x{\isacharbraceright}{\kern0pt}\ {\isacharequal}{\kern0pt}\ Rrel{\isacharparenleft}{\kern0pt}R{\isacharcomma}{\kern0pt}\ M{\isacharparenright}{\kern0pt}\ {\isacharminus}{\kern0pt}{\isacharbackquote}{\kern0pt}{\isacharbackquote}{\kern0pt}\ {\isacharbraceleft}{\kern0pt}x{\isacharbraceright}{\kern0pt}{\isachardoublequoteclose}\ \isanewline
%
\isadelimproof
%
\endisadelimproof
%
\isatagproof
\isacommand{proof}\isamarkupfalse%
\ {\isacharparenleft}{\kern0pt}rule\ equality{\isacharunderscore}{\kern0pt}iffI{\isacharcomma}{\kern0pt}\ rule\ iffI{\isacharparenright}{\kern0pt}\isanewline
\ \ \isacommand{fix}\isamarkupfalse%
\ y\ \isacommand{assume}\isamarkupfalse%
\ assms\ {\isacharcolon}{\kern0pt}\ {\isachardoublequoteopen}y\ {\isasymin}\ preds{\isacharunderscore}{\kern0pt}rel{\isacharparenleft}{\kern0pt}R{\isacharcomma}{\kern0pt}\ x{\isacharparenright}{\kern0pt}\ {\isacharminus}{\kern0pt}{\isacharbackquote}{\kern0pt}{\isacharbackquote}{\kern0pt}\ {\isacharbraceleft}{\kern0pt}x{\isacharbraceright}{\kern0pt}{\isachardoublequoteclose}\ {\isachardoublequoteopen}x\ {\isasymin}\ M{\isachardoublequoteclose}\ \isanewline
\ \ \isacommand{then}\isamarkupfalse%
\ \isacommand{have}\isamarkupfalse%
\ {\isachardoublequoteopen}{\isacharless}{\kern0pt}y{\isacharcomma}{\kern0pt}\ x{\isachargreater}{\kern0pt}\ {\isasymin}\ preds{\isacharunderscore}{\kern0pt}rel{\isacharparenleft}{\kern0pt}R{\isacharcomma}{\kern0pt}\ x{\isacharparenright}{\kern0pt}{\isachardoublequoteclose}\ \isacommand{by}\isamarkupfalse%
\ auto\isanewline
\ \ \isacommand{then}\isamarkupfalse%
\ \isacommand{show}\isamarkupfalse%
\ {\isachardoublequoteopen}y\ {\isasymin}\ Rrel{\isacharparenleft}{\kern0pt}R{\isacharcomma}{\kern0pt}\ M{\isacharparenright}{\kern0pt}\ {\isacharminus}{\kern0pt}{\isacharbackquote}{\kern0pt}{\isacharbackquote}{\kern0pt}\ {\isacharbraceleft}{\kern0pt}x{\isacharbraceright}{\kern0pt}{\isachardoublequoteclose}\ \isanewline
\ \ \ \ \isacommand{apply}\isamarkupfalse%
{\isacharparenleft}{\kern0pt}rule{\isacharunderscore}{\kern0pt}tac\ b{\isacharequal}{\kern0pt}x\ \isakeyword{in}\ vimageI{\isacharparenright}{\kern0pt}\isanewline
\ \ \ \ \ \isacommand{apply}\isamarkupfalse%
{\isacharparenleft}{\kern0pt}rule{\isacharunderscore}{\kern0pt}tac\ x{\isacharequal}{\kern0pt}x\ \isakeyword{in}\ preds{\isacharunderscore}{\kern0pt}rel{\isacharunderscore}{\kern0pt}subset{\isacharprime}{\kern0pt}{\isacharparenright}{\kern0pt}\isanewline
\ \ \ \ \isacommand{using}\isamarkupfalse%
\ assms\ \isanewline
\ \ \ \ \isacommand{by}\isamarkupfalse%
\ auto\isanewline
\isacommand{next}\isamarkupfalse%
\ \isanewline
\ \ \isacommand{fix}\isamarkupfalse%
\ y\ \isacommand{assume}\isamarkupfalse%
\ assms\ {\isacharcolon}{\kern0pt}\ {\isachardoublequoteopen}y\ {\isasymin}\ Rrel{\isacharparenleft}{\kern0pt}R{\isacharcomma}{\kern0pt}\ M{\isacharparenright}{\kern0pt}\ {\isacharminus}{\kern0pt}{\isacharbackquote}{\kern0pt}{\isacharbackquote}{\kern0pt}\ {\isacharbraceleft}{\kern0pt}x{\isacharbraceright}{\kern0pt}{\isachardoublequoteclose}\ \isanewline
\ \ \isacommand{then}\isamarkupfalse%
\ \isacommand{have}\isamarkupfalse%
\ {\isachardoublequoteopen}{\isacharless}{\kern0pt}y{\isacharcomma}{\kern0pt}\ x{\isachargreater}{\kern0pt}\ {\isasymin}\ Rrel{\isacharparenleft}{\kern0pt}R{\isacharcomma}{\kern0pt}\ M{\isacharparenright}{\kern0pt}{\isachardoublequoteclose}\ \isacommand{by}\isamarkupfalse%
\ auto\ \isanewline
\ \ \isacommand{then}\isamarkupfalse%
\ \isacommand{have}\isamarkupfalse%
\ {\isachardoublequoteopen}{\isacharless}{\kern0pt}y{\isacharcomma}{\kern0pt}\ x{\isachargreater}{\kern0pt}\ {\isasymin}\ preds{\isacharunderscore}{\kern0pt}rel{\isacharparenleft}{\kern0pt}R{\isacharcomma}{\kern0pt}\ x{\isacharparenright}{\kern0pt}{\isachardoublequoteclose}\ \isanewline
\ \ \ \ \isacommand{unfolding}\isamarkupfalse%
\ preds{\isacharunderscore}{\kern0pt}rel{\isacharunderscore}{\kern0pt}def\ \isanewline
\ \ \ \ \isacommand{apply}\isamarkupfalse%
\ simp\isanewline
\ \ \ \ \isacommand{apply}\isamarkupfalse%
{\isacharparenleft}{\kern0pt}rule\ conjI{\isacharparenright}{\kern0pt}\isanewline
\ \ \ \ \isacommand{unfolding}\isamarkupfalse%
\ preds{\isacharunderscore}{\kern0pt}def\ Rrel{\isacharunderscore}{\kern0pt}def\ \isanewline
\ \ \ \ \isacommand{by}\isamarkupfalse%
\ auto\isanewline
\ \ \isacommand{then}\isamarkupfalse%
\ \isacommand{show}\isamarkupfalse%
\ {\isachardoublequoteopen}y\ {\isasymin}\ preds{\isacharunderscore}{\kern0pt}rel{\isacharparenleft}{\kern0pt}R{\isacharcomma}{\kern0pt}\ x{\isacharparenright}{\kern0pt}\ {\isacharminus}{\kern0pt}{\isacharbackquote}{\kern0pt}{\isacharbackquote}{\kern0pt}\ {\isacharbraceleft}{\kern0pt}x{\isacharbraceright}{\kern0pt}{\isachardoublequoteclose}\ \isacommand{by}\isamarkupfalse%
\ auto\isanewline
\isacommand{qed}\isamarkupfalse%
%
\endisatagproof
{\isafoldproof}%
%
\isadelimproof
\isanewline
%
\endisadelimproof
\isanewline
\isacommand{lemma}\isamarkupfalse%
\ the{\isacharunderscore}{\kern0pt}recfun{\isacharunderscore}{\kern0pt}preds{\isacharunderscore}{\kern0pt}rel{\isacharunderscore}{\kern0pt}eq\ {\isacharcolon}{\kern0pt}\ \isanewline
\ \ \isakeyword{fixes}\ x\ y\ R\ H\isanewline
\ \ \isakeyword{assumes}\ {\isachardoublequoteopen}x\ {\isasymin}\ M{\isachardoublequoteclose}\ {\isachardoublequoteopen}trans{\isacharparenleft}{\kern0pt}Rrel{\isacharparenleft}{\kern0pt}R{\isacharcomma}{\kern0pt}\ M{\isacharparenright}{\kern0pt}{\isacharparenright}{\kern0pt}{\isachardoublequoteclose}\ {\isachardoublequoteopen}wf{\isacharparenleft}{\kern0pt}Rrel{\isacharparenleft}{\kern0pt}R{\isacharcomma}{\kern0pt}\ M{\isacharparenright}{\kern0pt}{\isacharparenright}{\kern0pt}{\isachardoublequoteclose}\ {\isachardoublequoteopen}y\ {\isasymin}\ preds{\isacharparenleft}{\kern0pt}R{\isacharcomma}{\kern0pt}\ x{\isacharparenright}{\kern0pt}\ {\isasymunion}\ {\isacharbraceleft}{\kern0pt}x{\isacharbraceright}{\kern0pt}{\isachardoublequoteclose}\ \isanewline
\ \ \isakeyword{shows}\ {\isachardoublequoteopen}the{\isacharunderscore}{\kern0pt}recfun{\isacharparenleft}{\kern0pt}preds{\isacharunderscore}{\kern0pt}rel{\isacharparenleft}{\kern0pt}R{\isacharcomma}{\kern0pt}\ x{\isacharparenright}{\kern0pt}{\isacharcomma}{\kern0pt}\ y{\isacharcomma}{\kern0pt}\ H{\isacharparenright}{\kern0pt}\ {\isacharequal}{\kern0pt}\ the{\isacharunderscore}{\kern0pt}recfun{\isacharparenleft}{\kern0pt}Rrel{\isacharparenleft}{\kern0pt}R{\isacharcomma}{\kern0pt}\ M{\isacharparenright}{\kern0pt}{\isacharcomma}{\kern0pt}\ y{\isacharcomma}{\kern0pt}\ H{\isacharparenright}{\kern0pt}{\isachardoublequoteclose}\ \isanewline
%
\isadelimproof
%
\endisadelimproof
%
\isatagproof
\isacommand{proof}\isamarkupfalse%
\ {\isacharminus}{\kern0pt}\ \isanewline
\ \ \isacommand{have}\isamarkupfalse%
\ eq\ {\isacharcolon}{\kern0pt}\ {\isachardoublequoteopen}{\isasymAnd}y{\isachardot}{\kern0pt}\ y\ {\isasymin}\ preds{\isacharparenleft}{\kern0pt}R{\isacharcomma}{\kern0pt}\ x{\isacharparenright}{\kern0pt}\ {\isasymunion}\ {\isacharbraceleft}{\kern0pt}x{\isacharbraceright}{\kern0pt}\ {\isasymlongrightarrow}\ the{\isacharunderscore}{\kern0pt}recfun{\isacharparenleft}{\kern0pt}preds{\isacharunderscore}{\kern0pt}rel{\isacharparenleft}{\kern0pt}R{\isacharcomma}{\kern0pt}\ x{\isacharparenright}{\kern0pt}{\isacharcomma}{\kern0pt}\ y{\isacharcomma}{\kern0pt}\ H{\isacharparenright}{\kern0pt}\ {\isacharequal}{\kern0pt}\ the{\isacharunderscore}{\kern0pt}recfun{\isacharparenleft}{\kern0pt}Rrel{\isacharparenleft}{\kern0pt}R{\isacharcomma}{\kern0pt}\ M{\isacharparenright}{\kern0pt}{\isacharcomma}{\kern0pt}\ y{\isacharcomma}{\kern0pt}\ H{\isacharparenright}{\kern0pt}{\isachardoublequoteclose}\isanewline
\ \ \isacommand{proof}\isamarkupfalse%
\ {\isacharparenleft}{\kern0pt}rule{\isacharunderscore}{\kern0pt}tac\ P{\isacharequal}{\kern0pt}{\isachardoublequoteopen}{\isasymlambda}y{\isachardot}{\kern0pt}\ y\ {\isasymin}\ preds{\isacharparenleft}{\kern0pt}R{\isacharcomma}{\kern0pt}\ x{\isacharparenright}{\kern0pt}\ {\isasymunion}\ {\isacharbraceleft}{\kern0pt}x{\isacharbraceright}{\kern0pt}\ {\isasymlongrightarrow}\ the{\isacharunderscore}{\kern0pt}recfun{\isacharparenleft}{\kern0pt}preds{\isacharunderscore}{\kern0pt}rel{\isacharparenleft}{\kern0pt}R{\isacharcomma}{\kern0pt}\ x{\isacharparenright}{\kern0pt}{\isacharcomma}{\kern0pt}\ y{\isacharcomma}{\kern0pt}\ H{\isacharparenright}{\kern0pt}\ {\isacharequal}{\kern0pt}\ the{\isacharunderscore}{\kern0pt}recfun{\isacharparenleft}{\kern0pt}Rrel{\isacharparenleft}{\kern0pt}R{\isacharcomma}{\kern0pt}\ M{\isacharparenright}{\kern0pt}{\isacharcomma}{\kern0pt}\ y{\isacharcomma}{\kern0pt}\ H{\isacharparenright}{\kern0pt}{\isachardoublequoteclose}\ \isakeyword{and}\ r{\isacharequal}{\kern0pt}{\isachardoublequoteopen}Rrel{\isacharparenleft}{\kern0pt}R{\isacharcomma}{\kern0pt}\ M{\isacharparenright}{\kern0pt}{\isachardoublequoteclose}\ \isakeyword{in}\ wf{\isacharunderscore}{\kern0pt}induct{\isacharcomma}{\kern0pt}\ simp\ add{\isacharcolon}{\kern0pt}assms{\isacharcomma}{\kern0pt}\ rule\ impI{\isacharparenright}{\kern0pt}\isanewline
\ \ \ \ \isacommand{fix}\isamarkupfalse%
\ y\ \ \isanewline
\ \ \ \ \isacommand{assume}\isamarkupfalse%
\ assms{\isadigit{1}}\ {\isacharcolon}{\kern0pt}\ {\isachardoublequoteopen}y\ {\isasymin}\ preds{\isacharparenleft}{\kern0pt}R{\isacharcomma}{\kern0pt}\ x{\isacharparenright}{\kern0pt}\ {\isasymunion}\ {\isacharbraceleft}{\kern0pt}x{\isacharbraceright}{\kern0pt}{\isachardoublequoteclose}\ {\isachardoublequoteopen}{\isacharparenleft}{\kern0pt}{\isasymAnd}z{\isachardot}{\kern0pt}\ {\isasymlangle}z{\isacharcomma}{\kern0pt}\ y{\isasymrangle}\ {\isasymin}\ Rrel{\isacharparenleft}{\kern0pt}R{\isacharcomma}{\kern0pt}\ M{\isacharparenright}{\kern0pt}\ {\isasymLongrightarrow}\ z\ {\isasymin}\ preds{\isacharparenleft}{\kern0pt}R{\isacharcomma}{\kern0pt}\ x{\isacharparenright}{\kern0pt}\ {\isasymunion}\ {\isacharbraceleft}{\kern0pt}x{\isacharbraceright}{\kern0pt}\ {\isasymlongrightarrow}\ the{\isacharunderscore}{\kern0pt}recfun{\isacharparenleft}{\kern0pt}preds{\isacharunderscore}{\kern0pt}rel{\isacharparenleft}{\kern0pt}R{\isacharcomma}{\kern0pt}\ x{\isacharparenright}{\kern0pt}{\isacharcomma}{\kern0pt}\ z{\isacharcomma}{\kern0pt}\ H{\isacharparenright}{\kern0pt}\ {\isacharequal}{\kern0pt}\ the{\isacharunderscore}{\kern0pt}recfun{\isacharparenleft}{\kern0pt}Rrel{\isacharparenleft}{\kern0pt}R{\isacharcomma}{\kern0pt}\ M{\isacharparenright}{\kern0pt}{\isacharcomma}{\kern0pt}\ z{\isacharcomma}{\kern0pt}\ H{\isacharparenright}{\kern0pt}{\isacharparenright}{\kern0pt}{\isachardoublequoteclose}\isanewline
\isanewline
\ \ \ \ \isacommand{have}\isamarkupfalse%
\ ih\ {\isacharcolon}{\kern0pt}\ {\isachardoublequoteopen}{\isasymAnd}z{\isachardot}{\kern0pt}\ {\isasymlangle}z{\isacharcomma}{\kern0pt}\ y{\isasymrangle}\ {\isasymin}\ Rrel{\isacharparenleft}{\kern0pt}R{\isacharcomma}{\kern0pt}\ M{\isacharparenright}{\kern0pt}\ {\isasymLongrightarrow}\ z\ {\isasymin}\ preds{\isacharparenleft}{\kern0pt}R{\isacharcomma}{\kern0pt}\ x{\isacharparenright}{\kern0pt}\ {\isasymunion}\ {\isacharbraceleft}{\kern0pt}x{\isacharbraceright}{\kern0pt}\ {\isasymLongrightarrow}\ the{\isacharunderscore}{\kern0pt}recfun{\isacharparenleft}{\kern0pt}preds{\isacharunderscore}{\kern0pt}rel{\isacharparenleft}{\kern0pt}R{\isacharcomma}{\kern0pt}\ x{\isacharparenright}{\kern0pt}{\isacharcomma}{\kern0pt}\ z{\isacharcomma}{\kern0pt}\ H{\isacharparenright}{\kern0pt}\ {\isacharequal}{\kern0pt}\ the{\isacharunderscore}{\kern0pt}recfun{\isacharparenleft}{\kern0pt}Rrel{\isacharparenleft}{\kern0pt}R{\isacharcomma}{\kern0pt}\ M{\isacharparenright}{\kern0pt}{\isacharcomma}{\kern0pt}\ z{\isacharcomma}{\kern0pt}\ H{\isacharparenright}{\kern0pt}{\isachardoublequoteclose}\isanewline
\ \ \ \ \ \ \isacommand{using}\isamarkupfalse%
\ assms{\isadigit{1}}\ \isacommand{by}\isamarkupfalse%
\ auto\isanewline
\isanewline
\ \ \ \ \isacommand{have}\isamarkupfalse%
\ vimage{\isacharunderscore}{\kern0pt}preds\ {\isacharcolon}{\kern0pt}\ {\isachardoublequoteopen}{\isasymAnd}z{\isachardot}{\kern0pt}\ z\ {\isasymin}\ Rrel{\isacharparenleft}{\kern0pt}R{\isacharcomma}{\kern0pt}\ M{\isacharparenright}{\kern0pt}\ {\isacharminus}{\kern0pt}{\isacharbackquote}{\kern0pt}{\isacharbackquote}{\kern0pt}\ {\isacharbraceleft}{\kern0pt}y{\isacharbraceright}{\kern0pt}\ {\isasymLongrightarrow}\ z\ {\isasymin}\ preds{\isacharparenleft}{\kern0pt}R{\isacharcomma}{\kern0pt}\ x{\isacharparenright}{\kern0pt}{\isachardoublequoteclose}\ \isanewline
\ \ \ \ \isacommand{proof}\isamarkupfalse%
\ {\isacharminus}{\kern0pt}\ \isanewline
\ \ \ \ \ \ \isacommand{fix}\isamarkupfalse%
\ z\ \isacommand{assume}\isamarkupfalse%
\ {\isachardoublequoteopen}z\ {\isasymin}\ Rrel{\isacharparenleft}{\kern0pt}R{\isacharcomma}{\kern0pt}\ M{\isacharparenright}{\kern0pt}\ {\isacharminus}{\kern0pt}{\isacharbackquote}{\kern0pt}{\isacharbackquote}{\kern0pt}\ {\isacharbraceleft}{\kern0pt}y{\isacharbraceright}{\kern0pt}{\isachardoublequoteclose}\ \isanewline
\ \ \ \ \ \ \isacommand{then}\isamarkupfalse%
\ \isacommand{have}\isamarkupfalse%
\ H{\isacharcolon}{\kern0pt}\ {\isachardoublequoteopen}{\isacharless}{\kern0pt}z{\isacharcomma}{\kern0pt}\ y{\isachargreater}{\kern0pt}\ {\isasymin}\ Rrel{\isacharparenleft}{\kern0pt}R{\isacharcomma}{\kern0pt}\ M{\isacharparenright}{\kern0pt}{\isachardoublequoteclose}\ \isacommand{by}\isamarkupfalse%
\ auto\ \isanewline
\ \ \ \ \ \ \isacommand{have}\isamarkupfalse%
\ {\isachardoublequoteopen}{\isacharless}{\kern0pt}z{\isacharcomma}{\kern0pt}\ x{\isachargreater}{\kern0pt}\ {\isasymin}\ Rrel{\isacharparenleft}{\kern0pt}R{\isacharcomma}{\kern0pt}\ M{\isacharparenright}{\kern0pt}{\isachardoublequoteclose}\ \isanewline
\ \ \ \ \ \ \isacommand{proof}\isamarkupfalse%
\ {\isacharparenleft}{\kern0pt}cases\ {\isachardoublequoteopen}y{\isacharequal}{\kern0pt}x{\isachardoublequoteclose}{\isacharparenright}{\kern0pt}\isanewline
\ \ \ \ \ \ \ \ \isacommand{case}\isamarkupfalse%
\ True\isanewline
\ \ \ \ \ \ \ \ \isacommand{then}\isamarkupfalse%
\ \isacommand{show}\isamarkupfalse%
\ {\isacharquery}{\kern0pt}thesis\ \isacommand{using}\isamarkupfalse%
\ H\ \isacommand{by}\isamarkupfalse%
\ auto\isanewline
\ \ \ \ \ \ \isacommand{next}\isamarkupfalse%
\isanewline
\ \ \ \ \ \ \ \ \isacommand{case}\isamarkupfalse%
\ False\isanewline
\ \ \ \ \ \ \ \ \isacommand{then}\isamarkupfalse%
\ \isacommand{have}\isamarkupfalse%
\ {\isachardoublequoteopen}y\ {\isasymin}\ preds{\isacharparenleft}{\kern0pt}R{\isacharcomma}{\kern0pt}\ x{\isacharparenright}{\kern0pt}{\isachardoublequoteclose}\ \isacommand{using}\isamarkupfalse%
\ assms{\isadigit{1}}\ \isacommand{by}\isamarkupfalse%
\ auto\ \isanewline
\ \ \ \ \ \ \ \ \isacommand{then}\isamarkupfalse%
\ \isacommand{have}\isamarkupfalse%
\ {\isachardoublequoteopen}{\isacharless}{\kern0pt}y{\isacharcomma}{\kern0pt}\ x{\isachargreater}{\kern0pt}\ {\isasymin}\ Rrel{\isacharparenleft}{\kern0pt}R{\isacharcomma}{\kern0pt}\ M{\isacharparenright}{\kern0pt}{\isachardoublequoteclose}\ \isacommand{unfolding}\isamarkupfalse%
\ preds{\isacharunderscore}{\kern0pt}def\ Rrel{\isacharunderscore}{\kern0pt}def\ \isacommand{using}\isamarkupfalse%
\ assms\ \isacommand{by}\isamarkupfalse%
\ auto\ \isanewline
\ \ \ \ \ \ \ \ \isacommand{then}\isamarkupfalse%
\ \isacommand{show}\isamarkupfalse%
\ {\isacharquery}{\kern0pt}thesis\ \isacommand{using}\isamarkupfalse%
\ assms\ H\ \isacommand{unfolding}\isamarkupfalse%
\ trans{\isacharunderscore}{\kern0pt}def\ \isacommand{by}\isamarkupfalse%
\ auto\isanewline
\ \ \ \ \ \ \isacommand{qed}\isamarkupfalse%
\isanewline
\ \ \ \ \ \ \isacommand{then}\isamarkupfalse%
\ \isacommand{show}\isamarkupfalse%
\ {\isachardoublequoteopen}z\ {\isasymin}\ preds{\isacharparenleft}{\kern0pt}R{\isacharcomma}{\kern0pt}\ x{\isacharparenright}{\kern0pt}{\isachardoublequoteclose}\ \isacommand{unfolding}\isamarkupfalse%
\ preds{\isacharunderscore}{\kern0pt}def\ Rrel{\isacharunderscore}{\kern0pt}def\ \isacommand{by}\isamarkupfalse%
\ auto\ \isanewline
\ \ \ \ \isacommand{qed}\isamarkupfalse%
\isanewline
\ \ \isanewline
\ \ \ \ \isacommand{have}\isamarkupfalse%
\ recfun\ {\isacharcolon}{\kern0pt}\ {\isachardoublequoteopen}is{\isacharunderscore}{\kern0pt}recfun{\isacharparenleft}{\kern0pt}preds{\isacharunderscore}{\kern0pt}rel{\isacharparenleft}{\kern0pt}R{\isacharcomma}{\kern0pt}\ x{\isacharparenright}{\kern0pt}{\isacharcomma}{\kern0pt}\ y{\isacharcomma}{\kern0pt}\ H{\isacharcomma}{\kern0pt}\ the{\isacharunderscore}{\kern0pt}recfun{\isacharparenleft}{\kern0pt}preds{\isacharunderscore}{\kern0pt}rel{\isacharparenleft}{\kern0pt}R{\isacharcomma}{\kern0pt}\ x{\isacharparenright}{\kern0pt}{\isacharcomma}{\kern0pt}\ y{\isacharcomma}{\kern0pt}\ H{\isacharparenright}{\kern0pt}{\isacharparenright}{\kern0pt}{\isachardoublequoteclose}\ \isanewline
\ \ \ \ \ \ \isacommand{apply}\isamarkupfalse%
{\isacharparenleft}{\kern0pt}rule\ unfold{\isacharunderscore}{\kern0pt}the{\isacharunderscore}{\kern0pt}recfun{\isacharparenright}{\kern0pt}\isanewline
\ \ \ \ \ \ \isacommand{using}\isamarkupfalse%
\ assms\ wf{\isacharunderscore}{\kern0pt}preds{\isacharunderscore}{\kern0pt}rel\ trans{\isacharunderscore}{\kern0pt}preds{\isacharunderscore}{\kern0pt}rel\ \isanewline
\ \ \ \ \ \ \isacommand{by}\isamarkupfalse%
\ auto\isanewline
\ \ \ \ \isacommand{have}\isamarkupfalse%
\ recfun{\isacharprime}{\kern0pt}\ {\isacharcolon}{\kern0pt}\ {\isachardoublequoteopen}is{\isacharunderscore}{\kern0pt}recfun{\isacharparenleft}{\kern0pt}Rrel{\isacharparenleft}{\kern0pt}R{\isacharcomma}{\kern0pt}\ M{\isacharparenright}{\kern0pt}{\isacharcomma}{\kern0pt}\ y{\isacharcomma}{\kern0pt}\ H{\isacharcomma}{\kern0pt}\ the{\isacharunderscore}{\kern0pt}recfun{\isacharparenleft}{\kern0pt}Rrel{\isacharparenleft}{\kern0pt}R{\isacharcomma}{\kern0pt}\ M{\isacharparenright}{\kern0pt}{\isacharcomma}{\kern0pt}\ y{\isacharcomma}{\kern0pt}\ H{\isacharparenright}{\kern0pt}{\isacharparenright}{\kern0pt}{\isachardoublequoteclose}\ \isanewline
\ \ \ \ \ \ \isacommand{apply}\isamarkupfalse%
{\isacharparenleft}{\kern0pt}rule\ unfold{\isacharunderscore}{\kern0pt}the{\isacharunderscore}{\kern0pt}recfun{\isacharparenright}{\kern0pt}\isanewline
\ \ \ \ \ \ \isacommand{using}\isamarkupfalse%
\ assms\ \isanewline
\ \ \ \ \ \ \isacommand{by}\isamarkupfalse%
\ auto\isanewline
\isanewline
\ \ \ \ \isacommand{have}\isamarkupfalse%
\ {\isachardoublequoteopen}the{\isacharunderscore}{\kern0pt}recfun{\isacharparenleft}{\kern0pt}preds{\isacharunderscore}{\kern0pt}rel{\isacharparenleft}{\kern0pt}R{\isacharcomma}{\kern0pt}\ x{\isacharparenright}{\kern0pt}{\isacharcomma}{\kern0pt}\ y{\isacharcomma}{\kern0pt}\ H{\isacharparenright}{\kern0pt}\ {\isacharequal}{\kern0pt}\ {\isacharparenleft}{\kern0pt}{\isasymlambda}z{\isasymin}preds{\isacharunderscore}{\kern0pt}rel{\isacharparenleft}{\kern0pt}R{\isacharcomma}{\kern0pt}\ x{\isacharparenright}{\kern0pt}\ {\isacharminus}{\kern0pt}{\isacharbackquote}{\kern0pt}{\isacharbackquote}{\kern0pt}\ {\isacharbraceleft}{\kern0pt}y{\isacharbraceright}{\kern0pt}{\isachardot}{\kern0pt}\ H{\isacharparenleft}{\kern0pt}z{\isacharcomma}{\kern0pt}\ restrict{\isacharparenleft}{\kern0pt}the{\isacharunderscore}{\kern0pt}recfun{\isacharparenleft}{\kern0pt}preds{\isacharunderscore}{\kern0pt}rel{\isacharparenleft}{\kern0pt}R{\isacharcomma}{\kern0pt}\ x{\isacharparenright}{\kern0pt}{\isacharcomma}{\kern0pt}\ y{\isacharcomma}{\kern0pt}\ H{\isacharparenright}{\kern0pt}{\isacharcomma}{\kern0pt}\ preds{\isacharunderscore}{\kern0pt}rel{\isacharparenleft}{\kern0pt}R{\isacharcomma}{\kern0pt}\ x{\isacharparenright}{\kern0pt}\ {\isacharminus}{\kern0pt}{\isacharbackquote}{\kern0pt}{\isacharbackquote}{\kern0pt}\ {\isacharbraceleft}{\kern0pt}z{\isacharbraceright}{\kern0pt}{\isacharparenright}{\kern0pt}{\isacharparenright}{\kern0pt}{\isacharparenright}{\kern0pt}{\isachardoublequoteclose}\isanewline
\ \ \ \ \ \ \isacommand{using}\isamarkupfalse%
\ recfun\isanewline
\ \ \ \ \ \ \isacommand{unfolding}\isamarkupfalse%
\ is{\isacharunderscore}{\kern0pt}recfun{\isacharunderscore}{\kern0pt}def\isanewline
\ \ \ \ \ \ \isacommand{by}\isamarkupfalse%
\ auto\isanewline
\ \ \ \ \isacommand{also}\isamarkupfalse%
\ \isacommand{have}\isamarkupfalse%
\ {\isachardoublequoteopen}{\isachardot}{\kern0pt}{\isachardot}{\kern0pt}{\isachardot}{\kern0pt}\ {\isacharequal}{\kern0pt}\ {\isacharparenleft}{\kern0pt}{\isasymlambda}z{\isasymin}preds{\isacharunderscore}{\kern0pt}rel{\isacharparenleft}{\kern0pt}R{\isacharcomma}{\kern0pt}\ x{\isacharparenright}{\kern0pt}\ {\isacharminus}{\kern0pt}{\isacharbackquote}{\kern0pt}{\isacharbackquote}{\kern0pt}\ {\isacharbraceleft}{\kern0pt}y{\isacharbraceright}{\kern0pt}{\isachardot}{\kern0pt}\ H{\isacharparenleft}{\kern0pt}z{\isacharcomma}{\kern0pt}\ the{\isacharunderscore}{\kern0pt}recfun{\isacharparenleft}{\kern0pt}preds{\isacharunderscore}{\kern0pt}rel{\isacharparenleft}{\kern0pt}R{\isacharcomma}{\kern0pt}\ x{\isacharparenright}{\kern0pt}{\isacharcomma}{\kern0pt}\ z{\isacharcomma}{\kern0pt}\ H{\isacharparenright}{\kern0pt}{\isacharparenright}{\kern0pt}{\isacharparenright}{\kern0pt}{\isachardoublequoteclose}\isanewline
\ \ \ \ \ \ \isacommand{apply}\isamarkupfalse%
{\isacharparenleft}{\kern0pt}rule\ lam{\isacharunderscore}{\kern0pt}cong{\isacharcomma}{\kern0pt}\ simp{\isacharparenright}{\kern0pt}\isanewline
\ \ \ \ \ \ \isacommand{apply}\isamarkupfalse%
{\isacharparenleft}{\kern0pt}subst\ the{\isacharunderscore}{\kern0pt}recfun{\isacharunderscore}{\kern0pt}cut{\isacharparenright}{\kern0pt}\isanewline
\ \ \ \ \ \ \isacommand{using}\isamarkupfalse%
\ assms\ wf{\isacharunderscore}{\kern0pt}preds{\isacharunderscore}{\kern0pt}rel\ trans{\isacharunderscore}{\kern0pt}preds{\isacharunderscore}{\kern0pt}rel\isanewline
\ \ \ \ \ \ \isacommand{by}\isamarkupfalse%
\ auto\isanewline
\ \ \ \ \isacommand{also}\isamarkupfalse%
\ \isacommand{have}\isamarkupfalse%
\ {\isachardoublequoteopen}{\isachardot}{\kern0pt}{\isachardot}{\kern0pt}{\isachardot}{\kern0pt}\ {\isacharequal}{\kern0pt}\ {\isacharparenleft}{\kern0pt}{\isasymlambda}z{\isasymin}Rrel{\isacharparenleft}{\kern0pt}R{\isacharcomma}{\kern0pt}\ M{\isacharparenright}{\kern0pt}\ {\isacharminus}{\kern0pt}{\isacharbackquote}{\kern0pt}{\isacharbackquote}{\kern0pt}\ {\isacharbraceleft}{\kern0pt}y{\isacharbraceright}{\kern0pt}{\isachardot}{\kern0pt}\ H{\isacharparenleft}{\kern0pt}z{\isacharcomma}{\kern0pt}\ the{\isacharunderscore}{\kern0pt}recfun{\isacharparenleft}{\kern0pt}preds{\isacharunderscore}{\kern0pt}rel{\isacharparenleft}{\kern0pt}R{\isacharcomma}{\kern0pt}\ x{\isacharparenright}{\kern0pt}{\isacharcomma}{\kern0pt}\ z{\isacharcomma}{\kern0pt}\ H{\isacharparenright}{\kern0pt}{\isacharparenright}{\kern0pt}{\isacharparenright}{\kern0pt}{\isachardoublequoteclose}\isanewline
\ \ \ \ \ \ \isacommand{apply}\isamarkupfalse%
{\isacharparenleft}{\kern0pt}rule\ lam{\isacharunderscore}{\kern0pt}cong{\isacharparenright}{\kern0pt}\isanewline
\ \ \ \ \ \ \ \isacommand{apply}\isamarkupfalse%
{\isacharparenleft}{\kern0pt}cases\ {\isachardoublequoteopen}y\ {\isacharequal}{\kern0pt}\ x{\isachardoublequoteclose}{\isacharcomma}{\kern0pt}\ simp{\isacharparenright}{\kern0pt}\isanewline
\ \ \ \ \ \ \ \ \isacommand{apply}\isamarkupfalse%
{\isacharparenleft}{\kern0pt}rule\ preds{\isacharunderscore}{\kern0pt}rel{\isacharunderscore}{\kern0pt}vimage{\isacharunderscore}{\kern0pt}eq{\isacharprime}{\kern0pt}{\isacharcomma}{\kern0pt}\ simp\ add{\isacharcolon}{\kern0pt}assms{\isacharparenright}{\kern0pt}\isanewline
\ \ \ \ \ \ \ \isacommand{apply}\isamarkupfalse%
{\isacharparenleft}{\kern0pt}subgoal{\isacharunderscore}{\kern0pt}tac\ {\isachardoublequoteopen}y\ {\isasymin}\ preds{\isacharparenleft}{\kern0pt}R{\isacharcomma}{\kern0pt}\ x{\isacharparenright}{\kern0pt}{\isachardoublequoteclose}{\isacharparenright}{\kern0pt}\isanewline
\ \ \ \ \ \ \ \ \isacommand{apply}\isamarkupfalse%
{\isacharparenleft}{\kern0pt}rule\ preds{\isacharunderscore}{\kern0pt}rel{\isacharunderscore}{\kern0pt}vimage{\isacharunderscore}{\kern0pt}eq{\isacharcomma}{\kern0pt}\ simp\ add{\isacharcolon}{\kern0pt}assms{\isacharparenright}{\kern0pt}\isanewline
\ \ \ \ \ \ \isacommand{using}\isamarkupfalse%
\ assms\ assms{\isadigit{1}}\isanewline
\ \ \ \ \ \ \isacommand{unfolding}\isamarkupfalse%
\ Rrel{\isacharunderscore}{\kern0pt}def\ preds{\isacharunderscore}{\kern0pt}def\ \isanewline
\ \ \ \ \ \ \isacommand{by}\isamarkupfalse%
\ auto\ \isanewline
\ \ \ \ \isacommand{also}\isamarkupfalse%
\ \isacommand{have}\isamarkupfalse%
\ {\isachardoublequoteopen}{\isachardot}{\kern0pt}{\isachardot}{\kern0pt}{\isachardot}{\kern0pt}\ {\isacharequal}{\kern0pt}\ {\isacharparenleft}{\kern0pt}{\isasymlambda}z{\isasymin}Rrel{\isacharparenleft}{\kern0pt}R{\isacharcomma}{\kern0pt}\ M{\isacharparenright}{\kern0pt}\ {\isacharminus}{\kern0pt}{\isacharbackquote}{\kern0pt}{\isacharbackquote}{\kern0pt}\ {\isacharbraceleft}{\kern0pt}y{\isacharbraceright}{\kern0pt}{\isachardot}{\kern0pt}\ H{\isacharparenleft}{\kern0pt}z{\isacharcomma}{\kern0pt}\ the{\isacharunderscore}{\kern0pt}recfun{\isacharparenleft}{\kern0pt}Rrel{\isacharparenleft}{\kern0pt}R{\isacharcomma}{\kern0pt}\ M{\isacharparenright}{\kern0pt}{\isacharcomma}{\kern0pt}\ z{\isacharcomma}{\kern0pt}\ H{\isacharparenright}{\kern0pt}{\isacharparenright}{\kern0pt}{\isacharparenright}{\kern0pt}{\isachardoublequoteclose}\isanewline
\ \ \ \ \ \ \isacommand{apply}\isamarkupfalse%
{\isacharparenleft}{\kern0pt}rule\ lam{\isacharunderscore}{\kern0pt}cong{\isacharcomma}{\kern0pt}\ simp{\isacharcomma}{\kern0pt}\ subst\ ih{\isacharparenright}{\kern0pt}\isanewline
\ \ \ \ \ \ \ \ \isacommand{apply}\isamarkupfalse%
\ force\ \isanewline
\ \ \ \ \ \ \ \isacommand{apply}\isamarkupfalse%
\ {\isacharparenleft}{\kern0pt}simp{\isacharcomma}{\kern0pt}\ rule\ disjI{\isadigit{1}}{\isacharcomma}{\kern0pt}\ rule\ vimage{\isacharunderscore}{\kern0pt}preds{\isacharcomma}{\kern0pt}\ simp{\isacharcomma}{\kern0pt}\ simp{\isacharparenright}{\kern0pt}\isanewline
\ \ \ \ \ \ \isacommand{done}\isamarkupfalse%
\isanewline
\ \ \ \ \isacommand{also}\isamarkupfalse%
\ \isacommand{have}\isamarkupfalse%
\ {\isachardoublequoteopen}{\isachardot}{\kern0pt}{\isachardot}{\kern0pt}{\isachardot}{\kern0pt}\ {\isacharequal}{\kern0pt}\ {\isacharparenleft}{\kern0pt}{\isasymlambda}z{\isasymin}Rrel{\isacharparenleft}{\kern0pt}R{\isacharcomma}{\kern0pt}\ M{\isacharparenright}{\kern0pt}\ {\isacharminus}{\kern0pt}{\isacharbackquote}{\kern0pt}{\isacharbackquote}{\kern0pt}\ {\isacharbraceleft}{\kern0pt}y{\isacharbraceright}{\kern0pt}{\isachardot}{\kern0pt}\ H{\isacharparenleft}{\kern0pt}z{\isacharcomma}{\kern0pt}\ restrict{\isacharparenleft}{\kern0pt}the{\isacharunderscore}{\kern0pt}recfun{\isacharparenleft}{\kern0pt}Rrel{\isacharparenleft}{\kern0pt}R{\isacharcomma}{\kern0pt}\ M{\isacharparenright}{\kern0pt}{\isacharcomma}{\kern0pt}\ y{\isacharcomma}{\kern0pt}\ H{\isacharparenright}{\kern0pt}{\isacharcomma}{\kern0pt}\ Rrel{\isacharparenleft}{\kern0pt}R{\isacharcomma}{\kern0pt}\ M{\isacharparenright}{\kern0pt}\ {\isacharminus}{\kern0pt}{\isacharbackquote}{\kern0pt}{\isacharbackquote}{\kern0pt}\ {\isacharbraceleft}{\kern0pt}z{\isacharbraceright}{\kern0pt}{\isacharparenright}{\kern0pt}{\isacharparenright}{\kern0pt}{\isacharparenright}{\kern0pt}{\isachardoublequoteclose}\isanewline
\ \ \ \ \ \ \isacommand{apply}\isamarkupfalse%
{\isacharparenleft}{\kern0pt}rule\ lam{\isacharunderscore}{\kern0pt}cong{\isacharcomma}{\kern0pt}\ simp{\isacharparenright}{\kern0pt}\isanewline
\ \ \ \ \ \ \isacommand{apply}\isamarkupfalse%
{\isacharparenleft}{\kern0pt}subst\ the{\isacharunderscore}{\kern0pt}recfun{\isacharunderscore}{\kern0pt}cut{\isacharparenright}{\kern0pt}\isanewline
\ \ \ \ \ \ \isacommand{using}\isamarkupfalse%
\ assms\ \isanewline
\ \ \ \ \ \ \isacommand{by}\isamarkupfalse%
\ auto\isanewline
\ \ \ \ \isacommand{also}\isamarkupfalse%
\ \isacommand{have}\isamarkupfalse%
\ {\isachardoublequoteopen}{\isachardot}{\kern0pt}{\isachardot}{\kern0pt}{\isachardot}{\kern0pt}\ {\isacharequal}{\kern0pt}\ the{\isacharunderscore}{\kern0pt}recfun{\isacharparenleft}{\kern0pt}Rrel{\isacharparenleft}{\kern0pt}R{\isacharcomma}{\kern0pt}\ M{\isacharparenright}{\kern0pt}{\isacharcomma}{\kern0pt}\ y{\isacharcomma}{\kern0pt}\ H{\isacharparenright}{\kern0pt}{\isachardoublequoteclose}\ \ \ \isanewline
\ \ \ \ \ \ \isacommand{using}\isamarkupfalse%
\ recfun{\isacharprime}{\kern0pt}\isanewline
\ \ \ \ \ \ \isacommand{unfolding}\isamarkupfalse%
\ is{\isacharunderscore}{\kern0pt}recfun{\isacharunderscore}{\kern0pt}def\isanewline
\ \ \ \ \ \ \isacommand{by}\isamarkupfalse%
\ auto\isanewline
\ \ \ \ \isacommand{finally}\isamarkupfalse%
\ \isacommand{show}\isamarkupfalse%
\ {\isachardoublequoteopen}the{\isacharunderscore}{\kern0pt}recfun{\isacharparenleft}{\kern0pt}preds{\isacharunderscore}{\kern0pt}rel{\isacharparenleft}{\kern0pt}R{\isacharcomma}{\kern0pt}\ x{\isacharparenright}{\kern0pt}{\isacharcomma}{\kern0pt}\ y{\isacharcomma}{\kern0pt}\ H{\isacharparenright}{\kern0pt}\ {\isacharequal}{\kern0pt}\ the{\isacharunderscore}{\kern0pt}recfun{\isacharparenleft}{\kern0pt}Rrel{\isacharparenleft}{\kern0pt}R{\isacharcomma}{\kern0pt}\ M{\isacharparenright}{\kern0pt}{\isacharcomma}{\kern0pt}\ y{\isacharcomma}{\kern0pt}\ H{\isacharparenright}{\kern0pt}{\isachardoublequoteclose}\ \isacommand{by}\isamarkupfalse%
\ simp\isanewline
\ \ \isacommand{qed}\isamarkupfalse%
\isanewline
\ \ \isacommand{then}\isamarkupfalse%
\ \isacommand{show}\isamarkupfalse%
\ {\isacharquery}{\kern0pt}thesis\ \isacommand{using}\isamarkupfalse%
\ assms\ \isacommand{by}\isamarkupfalse%
\ auto\isanewline
\isacommand{qed}\isamarkupfalse%
%
\endisatagproof
{\isafoldproof}%
%
\isadelimproof
\isanewline
%
\endisadelimproof
\isanewline
\isacommand{lemma}\isamarkupfalse%
\ wftrec{\isacharunderscore}{\kern0pt}preds{\isacharunderscore}{\kern0pt}rel{\isacharunderscore}{\kern0pt}eq\ {\isacharcolon}{\kern0pt}\ \isanewline
\ \ \isakeyword{fixes}\ x\ R\ H\isanewline
\ \ \isakeyword{assumes}\ {\isachardoublequoteopen}x\ {\isasymin}\ M{\isachardoublequoteclose}\ {\isachardoublequoteopen}trans{\isacharparenleft}{\kern0pt}Rrel{\isacharparenleft}{\kern0pt}R{\isacharcomma}{\kern0pt}\ M{\isacharparenright}{\kern0pt}{\isacharparenright}{\kern0pt}{\isachardoublequoteclose}\ {\isachardoublequoteopen}wf{\isacharparenleft}{\kern0pt}Rrel{\isacharparenleft}{\kern0pt}R{\isacharcomma}{\kern0pt}\ M{\isacharparenright}{\kern0pt}{\isacharparenright}{\kern0pt}{\isachardoublequoteclose}\isanewline
\ \ \isakeyword{shows}\ {\isachardoublequoteopen}wftrec{\isacharparenleft}{\kern0pt}Rrel{\isacharparenleft}{\kern0pt}R{\isacharcomma}{\kern0pt}\ M{\isacharparenright}{\kern0pt}{\isacharcomma}{\kern0pt}\ x{\isacharcomma}{\kern0pt}\ H{\isacharparenright}{\kern0pt}\ {\isacharequal}{\kern0pt}\ wftrec{\isacharparenleft}{\kern0pt}preds{\isacharunderscore}{\kern0pt}rel{\isacharparenleft}{\kern0pt}R{\isacharcomma}{\kern0pt}\ x{\isacharparenright}{\kern0pt}{\isacharcomma}{\kern0pt}\ x{\isacharcomma}{\kern0pt}\ H{\isacharparenright}{\kern0pt}{\isachardoublequoteclose}\ \isanewline
%
\isadelimproof
%
\endisadelimproof
%
\isatagproof
\isacommand{proof}\isamarkupfalse%
\ {\isacharminus}{\kern0pt}\isanewline
\ \ \isacommand{have}\isamarkupfalse%
\ {\isachardoublequoteopen}the{\isacharunderscore}{\kern0pt}recfun{\isacharparenleft}{\kern0pt}preds{\isacharunderscore}{\kern0pt}rel{\isacharparenleft}{\kern0pt}R{\isacharcomma}{\kern0pt}\ x{\isacharparenright}{\kern0pt}{\isacharcomma}{\kern0pt}\ x{\isacharcomma}{\kern0pt}\ H{\isacharparenright}{\kern0pt}\ {\isacharequal}{\kern0pt}\ the{\isacharunderscore}{\kern0pt}recfun{\isacharparenleft}{\kern0pt}Rrel{\isacharparenleft}{\kern0pt}R{\isacharcomma}{\kern0pt}\ M{\isacharparenright}{\kern0pt}{\isacharcomma}{\kern0pt}\ x{\isacharcomma}{\kern0pt}\ H{\isacharparenright}{\kern0pt}{\isachardoublequoteclose}\ \isanewline
\ \ \ \ \isacommand{apply}\isamarkupfalse%
{\isacharparenleft}{\kern0pt}rule\ the{\isacharunderscore}{\kern0pt}recfun{\isacharunderscore}{\kern0pt}preds{\isacharunderscore}{\kern0pt}rel{\isacharunderscore}{\kern0pt}eq{\isacharparenright}{\kern0pt}\isanewline
\ \ \ \ \isacommand{using}\isamarkupfalse%
\ assms\ \isanewline
\ \ \ \ \isacommand{by}\isamarkupfalse%
\ auto\isanewline
\ \ \isacommand{then}\isamarkupfalse%
\ \isacommand{show}\isamarkupfalse%
\ {\isachardoublequoteopen}wftrec{\isacharparenleft}{\kern0pt}Rrel{\isacharparenleft}{\kern0pt}R{\isacharcomma}{\kern0pt}\ M{\isacharparenright}{\kern0pt}{\isacharcomma}{\kern0pt}\ x{\isacharcomma}{\kern0pt}\ H{\isacharparenright}{\kern0pt}\ {\isacharequal}{\kern0pt}\ wftrec{\isacharparenleft}{\kern0pt}preds{\isacharunderscore}{\kern0pt}rel{\isacharparenleft}{\kern0pt}R{\isacharcomma}{\kern0pt}\ x{\isacharparenright}{\kern0pt}{\isacharcomma}{\kern0pt}\ x{\isacharcomma}{\kern0pt}\ H{\isacharparenright}{\kern0pt}{\isachardoublequoteclose}\ \isanewline
\ \ \ \ \isacommand{unfolding}\isamarkupfalse%
\ wftrec{\isacharunderscore}{\kern0pt}def\ \isacommand{by}\isamarkupfalse%
\ auto\isanewline
\isacommand{qed}\isamarkupfalse%
%
\endisatagproof
{\isafoldproof}%
%
\isadelimproof
\isanewline
%
\endisadelimproof
\isanewline
\isanewline
\ \isanewline
\isacommand{definition}\isamarkupfalse%
\ prel\ \isakeyword{where}\ {\isachardoublequoteopen}prel{\isacharparenleft}{\kern0pt}r{\isacharcomma}{\kern0pt}\ A{\isacharparenright}{\kern0pt}\ {\isasymequiv}\ {\isacharbraceleft}{\kern0pt}\ {\isacharless}{\kern0pt}x{\isacharcomma}{\kern0pt}\ y{\isachargreater}{\kern0pt}\ {\isasymin}\ {\isacharparenleft}{\kern0pt}{\isacharparenleft}{\kern0pt}field{\isacharparenleft}{\kern0pt}r{\isacharparenright}{\kern0pt}\ {\isasymtimes}\ A{\isacharparenright}{\kern0pt}\ {\isasymtimes}\ {\isacharparenleft}{\kern0pt}field{\isacharparenleft}{\kern0pt}r{\isacharparenright}{\kern0pt}\ {\isasymtimes}\ A{\isacharparenright}{\kern0pt}{\isacharparenright}{\kern0pt}{\isachardot}{\kern0pt}\ {\isasymexists}\ x{\isacharprime}{\kern0pt}\ y{\isacharprime}{\kern0pt}\ a{\isachardot}{\kern0pt}\ x\ {\isacharequal}{\kern0pt}\ {\isacharless}{\kern0pt}x{\isacharprime}{\kern0pt}{\isacharcomma}{\kern0pt}\ a{\isachargreater}{\kern0pt}\ {\isasymand}\ y\ {\isacharequal}{\kern0pt}\ {\isacharless}{\kern0pt}y{\isacharprime}{\kern0pt}{\isacharcomma}{\kern0pt}\ a{\isachargreater}{\kern0pt}\ {\isasymand}\ {\isacharless}{\kern0pt}x{\isacharprime}{\kern0pt}{\isacharcomma}{\kern0pt}\ y{\isacharprime}{\kern0pt}{\isachargreater}{\kern0pt}\ {\isasymin}\ r\ {\isacharbraceright}{\kern0pt}{\isachardoublequoteclose}\ \isanewline
\isanewline
\isacommand{lemma}\isamarkupfalse%
\ prelI\ {\isacharcolon}{\kern0pt}\ {\isachardoublequoteopen}{\isacharless}{\kern0pt}x{\isacharcomma}{\kern0pt}\ y{\isachargreater}{\kern0pt}\ {\isasymin}\ r\ {\isasymLongrightarrow}\ a\ {\isasymin}\ A\ {\isasymLongrightarrow}\ {\isacharless}{\kern0pt}{\isacharless}{\kern0pt}x{\isacharcomma}{\kern0pt}\ a{\isachargreater}{\kern0pt}{\isacharcomma}{\kern0pt}\ {\isacharless}{\kern0pt}y{\isacharcomma}{\kern0pt}\ a{\isachargreater}{\kern0pt}{\isachargreater}{\kern0pt}\ {\isasymin}\ prel{\isacharparenleft}{\kern0pt}r{\isacharcomma}{\kern0pt}\ A{\isacharparenright}{\kern0pt}{\isachardoublequoteclose}\ \isanewline
%
\isadelimproof
\ \ %
\endisadelimproof
%
\isatagproof
\isacommand{unfolding}\isamarkupfalse%
\ prel{\isacharunderscore}{\kern0pt}def\ \isacommand{by}\isamarkupfalse%
\ auto%
\endisatagproof
{\isafoldproof}%
%
\isadelimproof
\ \isanewline
%
\endisadelimproof
\isanewline
\isacommand{lemma}\isamarkupfalse%
\ prelD\ {\isacharcolon}{\kern0pt}\ {\isachardoublequoteopen}{\isacharless}{\kern0pt}x{\isacharcomma}{\kern0pt}\ y{\isachargreater}{\kern0pt}\ {\isasymin}\ prel{\isacharparenleft}{\kern0pt}r{\isacharcomma}{\kern0pt}\ A{\isacharparenright}{\kern0pt}\ {\isasymLongrightarrow}\ {\isasymexists}x{\isacharprime}{\kern0pt}\ y{\isacharprime}{\kern0pt}\ a{\isachardot}{\kern0pt}\ a\ {\isasymin}\ A\ {\isasymand}\ x\ {\isacharequal}{\kern0pt}\ {\isacharless}{\kern0pt}x{\isacharprime}{\kern0pt}{\isacharcomma}{\kern0pt}\ a{\isachargreater}{\kern0pt}\ {\isasymand}\ y\ {\isacharequal}{\kern0pt}\ {\isacharless}{\kern0pt}y{\isacharprime}{\kern0pt}{\isacharcomma}{\kern0pt}\ a{\isachargreater}{\kern0pt}\ {\isasymand}\ {\isacharless}{\kern0pt}x{\isacharprime}{\kern0pt}{\isacharcomma}{\kern0pt}\ y{\isacharprime}{\kern0pt}{\isachargreater}{\kern0pt}\ {\isasymin}\ r{\isachardoublequoteclose}\ \isanewline
%
\isadelimproof
\ \ %
\endisadelimproof
%
\isatagproof
\isacommand{unfolding}\isamarkupfalse%
\ prel{\isacharunderscore}{\kern0pt}def\ \isacommand{by}\isamarkupfalse%
\ auto%
\endisatagproof
{\isafoldproof}%
%
\isadelimproof
\ \isanewline
%
\endisadelimproof
\isanewline
\isacommand{schematic{\isacharunderscore}{\kern0pt}goal}\isamarkupfalse%
\ prel{\isacharunderscore}{\kern0pt}fm{\isacharunderscore}{\kern0pt}auto{\isacharcolon}{\kern0pt}\isanewline
\ \ \isakeyword{assumes}\isanewline
\ \ \ \ {\isachardoublequoteopen}nth{\isacharparenleft}{\kern0pt}{\isadigit{0}}{\isacharcomma}{\kern0pt}env{\isacharparenright}{\kern0pt}\ {\isacharequal}{\kern0pt}\ p{\isachardoublequoteclose}\ \isanewline
\ \ \ \ {\isachardoublequoteopen}nth{\isacharparenleft}{\kern0pt}{\isadigit{1}}{\isacharcomma}{\kern0pt}env{\isacharparenright}{\kern0pt}\ {\isacharequal}{\kern0pt}\ r{\isachardoublequoteclose}\ \isanewline
\ \ \ \ {\isachardoublequoteopen}nth{\isacharparenleft}{\kern0pt}{\isadigit{2}}{\isacharcomma}{\kern0pt}env{\isacharparenright}{\kern0pt}\ {\isacharequal}{\kern0pt}\ A{\isachardoublequoteclose}\ \ \isanewline
\ \ \ \ {\isachardoublequoteopen}env\ {\isasymin}\ list{\isacharparenleft}{\kern0pt}M{\isacharparenright}{\kern0pt}{\isachardoublequoteclose}\isanewline
\ \isakeyword{shows}\isanewline
\ \ \ \ {\isachardoublequoteopen}{\isacharparenleft}{\kern0pt}{\isasymexists}x\ {\isasymin}\ M{\isachardot}{\kern0pt}\ {\isasymexists}y\ {\isasymin}\ M{\isachardot}{\kern0pt}\ {\isasymexists}x{\isacharprime}{\kern0pt}\ {\isasymin}\ M{\isachardot}{\kern0pt}\ {\isasymexists}y{\isacharprime}{\kern0pt}\ {\isasymin}\ M{\isachardot}{\kern0pt}\ {\isasymexists}a\ {\isasymin}\ M{\isachardot}{\kern0pt}\ {\isasymexists}x{\isacharprime}{\kern0pt}{\isacharunderscore}{\kern0pt}y{\isacharprime}{\kern0pt}\ {\isasymin}\ M{\isachardot}{\kern0pt}\isanewline
\ \ \ \ \ \ pair{\isacharparenleft}{\kern0pt}{\isacharhash}{\kern0pt}{\isacharhash}{\kern0pt}M{\isacharcomma}{\kern0pt}\ x{\isacharcomma}{\kern0pt}\ y{\isacharcomma}{\kern0pt}\ p{\isacharparenright}{\kern0pt}\ {\isasymand}\ pair{\isacharparenleft}{\kern0pt}{\isacharhash}{\kern0pt}{\isacharhash}{\kern0pt}M{\isacharcomma}{\kern0pt}\ x{\isacharprime}{\kern0pt}{\isacharcomma}{\kern0pt}\ a{\isacharcomma}{\kern0pt}\ x{\isacharparenright}{\kern0pt}\ {\isasymand}\ pair{\isacharparenleft}{\kern0pt}{\isacharhash}{\kern0pt}{\isacharhash}{\kern0pt}M{\isacharcomma}{\kern0pt}\ y{\isacharprime}{\kern0pt}{\isacharcomma}{\kern0pt}\ a{\isacharcomma}{\kern0pt}\ y{\isacharparenright}{\kern0pt}\ {\isasymand}\ pair{\isacharparenleft}{\kern0pt}{\isacharhash}{\kern0pt}{\isacharhash}{\kern0pt}M{\isacharcomma}{\kern0pt}\ x{\isacharprime}{\kern0pt}{\isacharcomma}{\kern0pt}\ y{\isacharprime}{\kern0pt}{\isacharcomma}{\kern0pt}\ x{\isacharprime}{\kern0pt}{\isacharunderscore}{\kern0pt}y{\isacharprime}{\kern0pt}{\isacharparenright}{\kern0pt}\ {\isasymand}\ x{\isacharprime}{\kern0pt}{\isacharunderscore}{\kern0pt}y{\isacharprime}{\kern0pt}\ {\isasymin}\ r{\isacharparenright}{\kern0pt}\isanewline
\ \ \ \ \ {\isasymlongleftrightarrow}\ sats{\isacharparenleft}{\kern0pt}M{\isacharcomma}{\kern0pt}{\isacharquery}{\kern0pt}fm{\isacharparenleft}{\kern0pt}{\isadigit{0}}{\isacharcomma}{\kern0pt}{\isadigit{1}}{\isacharcomma}{\kern0pt}{\isadigit{2}}{\isacharparenright}{\kern0pt}{\isacharcomma}{\kern0pt}env{\isacharparenright}{\kern0pt}{\isachardoublequoteclose}\isanewline
%
\isadelimproof
\ \ %
\endisadelimproof
%
\isatagproof
\isacommand{by}\isamarkupfalse%
\ {\isacharparenleft}{\kern0pt}insert\ assms\ {\isacharsemicolon}{\kern0pt}\ {\isacharparenleft}{\kern0pt}rule\ sep{\isacharunderscore}{\kern0pt}rules\ {\isacharbar}{\kern0pt}\ simp{\isacharparenright}{\kern0pt}{\isacharplus}{\kern0pt}{\isacharparenright}{\kern0pt}%
\endisatagproof
{\isafoldproof}%
%
\isadelimproof
\isanewline
%
\endisadelimproof
\isanewline
\isacommand{end}\isamarkupfalse%
\isanewline
\isanewline
\isacommand{definition}\isamarkupfalse%
\ prel{\isacharunderscore}{\kern0pt}fm\ \isakeyword{where}\ \isanewline
\ \ {\isachardoublequoteopen}prel{\isacharunderscore}{\kern0pt}fm\ {\isasymequiv}\ Exists{\isacharparenleft}{\kern0pt}Exists{\isacharparenleft}{\kern0pt}Exists{\isacharparenleft}{\kern0pt}Exists{\isacharparenleft}{\kern0pt}Exists{\isacharparenleft}{\kern0pt}Exists{\isacharparenleft}{\kern0pt}And{\isacharparenleft}{\kern0pt}pair{\isacharunderscore}{\kern0pt}fm{\isacharparenleft}{\kern0pt}{\isadigit{5}}{\isacharcomma}{\kern0pt}\ {\isadigit{4}}{\isacharcomma}{\kern0pt}\ {\isadigit{6}}{\isacharparenright}{\kern0pt}{\isacharcomma}{\kern0pt}\ And{\isacharparenleft}{\kern0pt}pair{\isacharunderscore}{\kern0pt}fm{\isacharparenleft}{\kern0pt}{\isadigit{3}}{\isacharcomma}{\kern0pt}\ {\isadigit{1}}{\isacharcomma}{\kern0pt}\ {\isadigit{5}}{\isacharparenright}{\kern0pt}{\isacharcomma}{\kern0pt}\ And{\isacharparenleft}{\kern0pt}pair{\isacharunderscore}{\kern0pt}fm{\isacharparenleft}{\kern0pt}{\isadigit{2}}{\isacharcomma}{\kern0pt}\ {\isadigit{1}}{\isacharcomma}{\kern0pt}\ {\isadigit{4}}{\isacharparenright}{\kern0pt}{\isacharcomma}{\kern0pt}\ And{\isacharparenleft}{\kern0pt}pair{\isacharunderscore}{\kern0pt}fm{\isacharparenleft}{\kern0pt}{\isadigit{3}}{\isacharcomma}{\kern0pt}\ {\isadigit{2}}{\isacharcomma}{\kern0pt}\ {\isadigit{0}}{\isacharparenright}{\kern0pt}{\isacharcomma}{\kern0pt}\ Member{\isacharparenleft}{\kern0pt}{\isadigit{0}}{\isacharcomma}{\kern0pt}\ {\isadigit{7}}{\isacharparenright}{\kern0pt}{\isacharparenright}{\kern0pt}{\isacharparenright}{\kern0pt}{\isacharparenright}{\kern0pt}{\isacharparenright}{\kern0pt}{\isacharparenright}{\kern0pt}{\isacharparenright}{\kern0pt}{\isacharparenright}{\kern0pt}{\isacharparenright}{\kern0pt}{\isacharparenright}{\kern0pt}{\isacharparenright}{\kern0pt}\ \ {\isachardoublequoteclose}\ \isanewline
\isanewline
\isacommand{context}\isamarkupfalse%
\ M{\isacharunderscore}{\kern0pt}ctm\ \isanewline
\isakeyword{begin}\ \isanewline
\isanewline
\isacommand{lemma}\isamarkupfalse%
\ prel{\isacharunderscore}{\kern0pt}fm{\isacharunderscore}{\kern0pt}sats{\isacharunderscore}{\kern0pt}iff\ {\isacharcolon}{\kern0pt}\ \isanewline
\ \ {\isachardoublequoteopen}p\ {\isasymin}\ M\ {\isasymLongrightarrow}\ r\ {\isasymin}\ M\ {\isasymLongrightarrow}\ A\ {\isasymin}\ M\ {\isasymLongrightarrow}\ \isanewline
\ \ sats{\isacharparenleft}{\kern0pt}M{\isacharcomma}{\kern0pt}\ prel{\isacharunderscore}{\kern0pt}fm{\isacharcomma}{\kern0pt}\ {\isacharbrackleft}{\kern0pt}p{\isacharcomma}{\kern0pt}\ r{\isacharcomma}{\kern0pt}\ A{\isacharbrackright}{\kern0pt}{\isacharparenright}{\kern0pt}\ {\isasymlongleftrightarrow}\ {\isacharparenleft}{\kern0pt}{\isasymexists}\ x\ y\ x{\isacharprime}{\kern0pt}\ y{\isacharprime}{\kern0pt}\ a{\isachardot}{\kern0pt}\ p\ {\isacharequal}{\kern0pt}\ {\isacharless}{\kern0pt}x{\isacharcomma}{\kern0pt}\ y{\isachargreater}{\kern0pt}\ {\isasymand}\ x\ {\isacharequal}{\kern0pt}\ {\isacharless}{\kern0pt}x{\isacharprime}{\kern0pt}{\isacharcomma}{\kern0pt}\ a{\isachargreater}{\kern0pt}\ {\isasymand}\ y\ {\isacharequal}{\kern0pt}\ {\isacharless}{\kern0pt}y{\isacharprime}{\kern0pt}{\isacharcomma}{\kern0pt}\ a{\isachargreater}{\kern0pt}\ {\isasymand}\ {\isacharless}{\kern0pt}x{\isacharprime}{\kern0pt}{\isacharcomma}{\kern0pt}\ y{\isacharprime}{\kern0pt}{\isachargreater}{\kern0pt}\ {\isasymin}\ r{\isacharparenright}{\kern0pt}{\isachardoublequoteclose}\ \isanewline
%
\isadelimproof
\isanewline
\ \ %
\endisadelimproof
%
\isatagproof
\isacommand{apply}\isamarkupfalse%
{\isacharparenleft}{\kern0pt}rule{\isacharunderscore}{\kern0pt}tac\ Q{\isacharequal}{\kern0pt}{\isachardoublequoteopen}{\isacharparenleft}{\kern0pt}{\isasymexists}x\ {\isasymin}\ M{\isachardot}{\kern0pt}\ {\isasymexists}y\ {\isasymin}\ M{\isachardot}{\kern0pt}\ {\isasymexists}x{\isacharprime}{\kern0pt}\ {\isasymin}\ M{\isachardot}{\kern0pt}\ {\isasymexists}y{\isacharprime}{\kern0pt}\ {\isasymin}\ M{\isachardot}{\kern0pt}\ {\isasymexists}a\ {\isasymin}\ M{\isachardot}{\kern0pt}\ {\isasymexists}x{\isacharprime}{\kern0pt}{\isacharunderscore}{\kern0pt}y{\isacharprime}{\kern0pt}\ {\isasymin}\ M{\isachardot}{\kern0pt}\isanewline
\ \ \ \ \ \ \ \ \ \ \ \ \ \ \ \ \ \ \ \ pair{\isacharparenleft}{\kern0pt}{\isacharhash}{\kern0pt}{\isacharhash}{\kern0pt}M{\isacharcomma}{\kern0pt}\ x{\isacharcomma}{\kern0pt}\ y{\isacharcomma}{\kern0pt}\ p{\isacharparenright}{\kern0pt}\ {\isasymand}\ pair{\isacharparenleft}{\kern0pt}{\isacharhash}{\kern0pt}{\isacharhash}{\kern0pt}M{\isacharcomma}{\kern0pt}\ x{\isacharprime}{\kern0pt}{\isacharcomma}{\kern0pt}\ a{\isacharcomma}{\kern0pt}\ x{\isacharparenright}{\kern0pt}\ {\isasymand}\ pair{\isacharparenleft}{\kern0pt}{\isacharhash}{\kern0pt}{\isacharhash}{\kern0pt}M{\isacharcomma}{\kern0pt}\ y{\isacharprime}{\kern0pt}{\isacharcomma}{\kern0pt}\ a{\isacharcomma}{\kern0pt}\ y{\isacharparenright}{\kern0pt}\ {\isasymand}\ pair{\isacharparenleft}{\kern0pt}{\isacharhash}{\kern0pt}{\isacharhash}{\kern0pt}M{\isacharcomma}{\kern0pt}\ x{\isacharprime}{\kern0pt}{\isacharcomma}{\kern0pt}\ y{\isacharprime}{\kern0pt}{\isacharcomma}{\kern0pt}\ x{\isacharprime}{\kern0pt}{\isacharunderscore}{\kern0pt}y{\isacharprime}{\kern0pt}{\isacharparenright}{\kern0pt}\ {\isasymand}\ x{\isacharprime}{\kern0pt}{\isacharunderscore}{\kern0pt}y{\isacharprime}{\kern0pt}\ {\isasymin}\ r{\isacharparenright}{\kern0pt}\ {\isachardoublequoteclose}\ \isakeyword{in}\ iff{\isacharunderscore}{\kern0pt}trans{\isacharparenright}{\kern0pt}\isanewline
\ \ \isacommand{apply}\isamarkupfalse%
{\isacharparenleft}{\kern0pt}rule\ iff{\isacharunderscore}{\kern0pt}flip{\isacharparenright}{\kern0pt}\ \isacommand{unfolding}\isamarkupfalse%
\ prel{\isacharunderscore}{\kern0pt}fm{\isacharunderscore}{\kern0pt}def\ \isacommand{apply}\isamarkupfalse%
{\isacharparenleft}{\kern0pt}rule{\isacharunderscore}{\kern0pt}tac\ r{\isacharequal}{\kern0pt}r\ \isakeyword{and}\ A{\isacharequal}{\kern0pt}A\ \isakeyword{in}\ prel{\isacharunderscore}{\kern0pt}fm{\isacharunderscore}{\kern0pt}auto{\isacharparenright}{\kern0pt}\ \isacommand{apply}\isamarkupfalse%
\ simp{\isacharunderscore}{\kern0pt}all\ \isanewline
\ \ \isacommand{using}\isamarkupfalse%
\ pair{\isacharunderscore}{\kern0pt}in{\isacharunderscore}{\kern0pt}M{\isacharunderscore}{\kern0pt}iff\ transM\ \isacommand{apply}\isamarkupfalse%
\ auto\ \isacommand{done}\isamarkupfalse%
%
\endisatagproof
{\isafoldproof}%
%
\isadelimproof
\ \isanewline
%
\endisadelimproof
\isanewline
\isacommand{lemma}\isamarkupfalse%
\ prel{\isacharunderscore}{\kern0pt}closed\ {\isacharcolon}{\kern0pt}\ \isanewline
\ \ {\isachardoublequoteopen}r\ {\isasymin}\ M\ {\isasymLongrightarrow}\ A\ {\isasymin}\ M\ {\isasymLongrightarrow}\ prel{\isacharparenleft}{\kern0pt}r{\isacharcomma}{\kern0pt}\ A{\isacharparenright}{\kern0pt}\ {\isasymin}\ M{\isachardoublequoteclose}\ \isanewline
%
\isadelimproof
%
\endisadelimproof
%
\isatagproof
\isacommand{proof}\isamarkupfalse%
\ {\isacharminus}{\kern0pt}\ \isanewline
\ \ \isacommand{assume}\isamarkupfalse%
\ assms{\isacharcolon}{\kern0pt}\ {\isachardoublequoteopen}r\ {\isasymin}\ M{\isachardoublequoteclose}\ {\isachardoublequoteopen}A\ {\isasymin}\ M{\isachardoublequoteclose}\ \isanewline
\isanewline
\ \ \isacommand{have}\isamarkupfalse%
\ {\isachardoublequoteopen}field{\isacharparenleft}{\kern0pt}r{\isacharparenright}{\kern0pt}\ {\isasymin}\ M{\isachardoublequoteclose}\ \isacommand{using}\isamarkupfalse%
\ field{\isacharunderscore}{\kern0pt}closed\ assms\ \isacommand{by}\isamarkupfalse%
\ auto\ \isanewline
\ \ \isacommand{then}\isamarkupfalse%
\ \isacommand{have}\isamarkupfalse%
\ base{\isacharcolon}{\kern0pt}\ {\isachardoublequoteopen}{\isacharparenleft}{\kern0pt}{\isacharparenleft}{\kern0pt}field{\isacharparenleft}{\kern0pt}r{\isacharparenright}{\kern0pt}\ {\isasymtimes}\ A{\isacharparenright}{\kern0pt}\ {\isasymtimes}\ {\isacharparenleft}{\kern0pt}field{\isacharparenleft}{\kern0pt}r{\isacharparenright}{\kern0pt}\ {\isasymtimes}\ A{\isacharparenright}{\kern0pt}{\isacharparenright}{\kern0pt}\ {\isasymin}\ M{\isachardoublequoteclose}\ \isacommand{using}\isamarkupfalse%
\ assms\ cartprod{\isacharunderscore}{\kern0pt}closed\ \isacommand{by}\isamarkupfalse%
\ auto\ \isanewline
\isanewline
\ \ \isacommand{have}\isamarkupfalse%
\ {\isachardoublequoteopen}separation{\isacharparenleft}{\kern0pt}{\isacharhash}{\kern0pt}{\isacharhash}{\kern0pt}M{\isacharcomma}{\kern0pt}\ {\isasymlambda}x{\isachardot}{\kern0pt}\ sats{\isacharparenleft}{\kern0pt}M{\isacharcomma}{\kern0pt}\ prel{\isacharunderscore}{\kern0pt}fm{\isacharcomma}{\kern0pt}\ {\isacharbrackleft}{\kern0pt}x{\isacharbrackright}{\kern0pt}{\isacharat}{\kern0pt}{\isacharbrackleft}{\kern0pt}r{\isacharcomma}{\kern0pt}\ A{\isacharbrackright}{\kern0pt}{\isacharparenright}{\kern0pt}{\isacharparenright}{\kern0pt}{\isachardoublequoteclose}\ \isanewline
\ \ \ \ \isacommand{apply}\isamarkupfalse%
{\isacharparenleft}{\kern0pt}rule{\isacharunderscore}{\kern0pt}tac\ separation{\isacharunderscore}{\kern0pt}ax{\isacharparenright}{\kern0pt}\ \isacommand{unfolding}\isamarkupfalse%
\ prel{\isacharunderscore}{\kern0pt}fm{\isacharunderscore}{\kern0pt}def\ \isacommand{using}\isamarkupfalse%
\ assms\ \isacommand{apply}\isamarkupfalse%
\ auto\ \isanewline
\ \ \ \ \isacommand{by}\isamarkupfalse%
\ {\isacharparenleft}{\kern0pt}simp\ del{\isacharcolon}{\kern0pt}FOL{\isacharunderscore}{\kern0pt}sats{\isacharunderscore}{\kern0pt}iff\ pair{\isacharunderscore}{\kern0pt}abs\ add{\isacharcolon}{\kern0pt}\ fm{\isacharunderscore}{\kern0pt}defs\ nat{\isacharunderscore}{\kern0pt}simp{\isacharunderscore}{\kern0pt}union{\isacharparenright}{\kern0pt}\ \ \isanewline
\ \ \isacommand{then}\isamarkupfalse%
\ \isacommand{have}\isamarkupfalse%
\ H{\isacharcolon}{\kern0pt}\ {\isachardoublequoteopen}{\isacharbraceleft}{\kern0pt}\ p\ {\isasymin}\ {\isacharparenleft}{\kern0pt}{\isacharparenleft}{\kern0pt}field{\isacharparenleft}{\kern0pt}r{\isacharparenright}{\kern0pt}\ {\isasymtimes}\ A{\isacharparenright}{\kern0pt}\ {\isasymtimes}\ {\isacharparenleft}{\kern0pt}field{\isacharparenleft}{\kern0pt}r{\isacharparenright}{\kern0pt}\ {\isasymtimes}\ A{\isacharparenright}{\kern0pt}{\isacharparenright}{\kern0pt}{\isachardot}{\kern0pt}\ sats{\isacharparenleft}{\kern0pt}M{\isacharcomma}{\kern0pt}\ prel{\isacharunderscore}{\kern0pt}fm{\isacharcomma}{\kern0pt}\ {\isacharbrackleft}{\kern0pt}p{\isacharbrackright}{\kern0pt}{\isacharat}{\kern0pt}{\isacharbrackleft}{\kern0pt}r{\isacharcomma}{\kern0pt}\ A{\isacharbrackright}{\kern0pt}{\isacharparenright}{\kern0pt}\ {\isacharbraceright}{\kern0pt}\ {\isasymin}\ M{\isachardoublequoteclose}\isanewline
\ \ \ \ \isacommand{apply}\isamarkupfalse%
{\isacharparenleft}{\kern0pt}rule{\isacharunderscore}{\kern0pt}tac\ separation{\isacharunderscore}{\kern0pt}notation{\isacharparenright}{\kern0pt}\ \isacommand{using}\isamarkupfalse%
\ base\ \isacommand{by}\isamarkupfalse%
\ auto\ \isanewline
\isanewline
\ \ \isacommand{have}\isamarkupfalse%
\ {\isachardoublequoteopen}{\isacharbraceleft}{\kern0pt}\ p\ {\isasymin}\ {\isacharparenleft}{\kern0pt}{\isacharparenleft}{\kern0pt}field{\isacharparenleft}{\kern0pt}r{\isacharparenright}{\kern0pt}\ {\isasymtimes}\ A{\isacharparenright}{\kern0pt}\ {\isasymtimes}\ {\isacharparenleft}{\kern0pt}field{\isacharparenleft}{\kern0pt}r{\isacharparenright}{\kern0pt}\ {\isasymtimes}\ A{\isacharparenright}{\kern0pt}{\isacharparenright}{\kern0pt}{\isachardot}{\kern0pt}\ sats{\isacharparenleft}{\kern0pt}M{\isacharcomma}{\kern0pt}\ prel{\isacharunderscore}{\kern0pt}fm{\isacharcomma}{\kern0pt}\ {\isacharbrackleft}{\kern0pt}p{\isacharbrackright}{\kern0pt}{\isacharat}{\kern0pt}{\isacharbrackleft}{\kern0pt}r{\isacharcomma}{\kern0pt}\ A{\isacharbrackright}{\kern0pt}{\isacharparenright}{\kern0pt}\ {\isacharbraceright}{\kern0pt}\ {\isacharequal}{\kern0pt}\isanewline
\ \ \ \ \ \ \ \ {\isacharbraceleft}{\kern0pt}\ p\ {\isasymin}\ {\isacharparenleft}{\kern0pt}{\isacharparenleft}{\kern0pt}field{\isacharparenleft}{\kern0pt}r{\isacharparenright}{\kern0pt}\ {\isasymtimes}\ A{\isacharparenright}{\kern0pt}\ {\isasymtimes}\ {\isacharparenleft}{\kern0pt}field{\isacharparenleft}{\kern0pt}r{\isacharparenright}{\kern0pt}\ {\isasymtimes}\ A{\isacharparenright}{\kern0pt}{\isacharparenright}{\kern0pt}{\isachardot}{\kern0pt}\ {\isacharparenleft}{\kern0pt}{\isasymexists}\ x\ y\ x{\isacharprime}{\kern0pt}\ y{\isacharprime}{\kern0pt}\ a{\isachardot}{\kern0pt}\ p\ {\isacharequal}{\kern0pt}\ {\isacharless}{\kern0pt}x{\isacharcomma}{\kern0pt}\ y{\isachargreater}{\kern0pt}\ {\isasymand}\ x\ {\isacharequal}{\kern0pt}\ {\isacharless}{\kern0pt}x{\isacharprime}{\kern0pt}{\isacharcomma}{\kern0pt}\ a{\isachargreater}{\kern0pt}\ {\isasymand}\ y\ {\isacharequal}{\kern0pt}\ {\isacharless}{\kern0pt}y{\isacharprime}{\kern0pt}{\isacharcomma}{\kern0pt}\ a{\isachargreater}{\kern0pt}\ {\isasymand}\ {\isacharless}{\kern0pt}x{\isacharprime}{\kern0pt}{\isacharcomma}{\kern0pt}\ y{\isacharprime}{\kern0pt}{\isachargreater}{\kern0pt}\ {\isasymin}\ r{\isacharparenright}{\kern0pt}\ {\isacharbraceright}{\kern0pt}{\isachardoublequoteclose}\isanewline
\ \ \ \ \isacommand{apply}\isamarkupfalse%
{\isacharparenleft}{\kern0pt}rule{\isacharunderscore}{\kern0pt}tac\ iff{\isacharunderscore}{\kern0pt}eq{\isacharparenright}{\kern0pt}\ \isacommand{apply}\isamarkupfalse%
{\isacharparenleft}{\kern0pt}rule{\isacharunderscore}{\kern0pt}tac\ b{\isacharequal}{\kern0pt}{\isachardoublequoteopen}{\isacharbrackleft}{\kern0pt}p{\isacharbrackright}{\kern0pt}\ {\isacharat}{\kern0pt}\ {\isacharbrackleft}{\kern0pt}r{\isacharcomma}{\kern0pt}\ A{\isacharbrackright}{\kern0pt}{\isachardoublequoteclose}\ \isakeyword{and}\ a\ {\isacharequal}{\kern0pt}\ {\isachardoublequoteopen}{\isacharbrackleft}{\kern0pt}p{\isacharcomma}{\kern0pt}\ r{\isacharcomma}{\kern0pt}\ A{\isacharbrackright}{\kern0pt}{\isachardoublequoteclose}\ \isakeyword{in}\ ssubst{\isacharparenright}{\kern0pt}\ \isanewline
\ \ \ \ \isacommand{apply}\isamarkupfalse%
\ simp\ \isacommand{apply}\isamarkupfalse%
{\isacharparenleft}{\kern0pt}rule{\isacharunderscore}{\kern0pt}tac\ prel{\isacharunderscore}{\kern0pt}fm{\isacharunderscore}{\kern0pt}sats{\isacharunderscore}{\kern0pt}iff{\isacharparenright}{\kern0pt}\ \isacommand{apply}\isamarkupfalse%
{\isacharparenleft}{\kern0pt}rule\ to{\isacharunderscore}{\kern0pt}rin{\isacharparenright}{\kern0pt}\ \isanewline
\ \ \ \ \isacommand{apply}\isamarkupfalse%
{\isacharparenleft}{\kern0pt}rule{\isacharunderscore}{\kern0pt}tac\ x{\isacharequal}{\kern0pt}{\isachardoublequoteopen}{\isacharparenleft}{\kern0pt}field{\isacharparenleft}{\kern0pt}r{\isacharparenright}{\kern0pt}\ {\isasymtimes}\ A{\isacharparenright}{\kern0pt}\ {\isasymtimes}\ {\isacharparenleft}{\kern0pt}field{\isacharparenleft}{\kern0pt}r{\isacharparenright}{\kern0pt}\ {\isasymtimes}\ A{\isacharparenright}{\kern0pt}{\isachardoublequoteclose}\ \isakeyword{in}\ transM{\isacharparenright}{\kern0pt}\ \isacommand{apply}\isamarkupfalse%
\ simp\ \isanewline
\ \ \ \ \isacommand{using}\isamarkupfalse%
\ base\ transM\ assms\ \isacommand{by}\isamarkupfalse%
\ auto\isanewline
\ \ \isacommand{also}\isamarkupfalse%
\ \isacommand{have}\isamarkupfalse%
\ {\isachardoublequoteopen}{\isachardot}{\kern0pt}{\isachardot}{\kern0pt}{\isachardot}{\kern0pt}\ {\isacharequal}{\kern0pt}\ prel{\isacharparenleft}{\kern0pt}r{\isacharcomma}{\kern0pt}\ A{\isacharparenright}{\kern0pt}{\isachardoublequoteclose}\ \isanewline
\ \ \ \ \isacommand{unfolding}\isamarkupfalse%
\ prel{\isacharunderscore}{\kern0pt}def\ \isacommand{apply}\isamarkupfalse%
{\isacharparenleft}{\kern0pt}rule{\isacharunderscore}{\kern0pt}tac\ iff{\isacharunderscore}{\kern0pt}eq{\isacharparenright}{\kern0pt}\ \isacommand{by}\isamarkupfalse%
\ auto\ \isanewline
\ \ \isacommand{finally}\isamarkupfalse%
\ \isacommand{show}\isamarkupfalse%
\ {\isachardoublequoteopen}prel{\isacharparenleft}{\kern0pt}r{\isacharcomma}{\kern0pt}\ A{\isacharparenright}{\kern0pt}\ {\isasymin}\ M{\isachardoublequoteclose}\ \isacommand{using}\isamarkupfalse%
\ H\ \isacommand{by}\isamarkupfalse%
\ auto\ \isanewline
\isacommand{qed}\isamarkupfalse%
%
\endisatagproof
{\isafoldproof}%
%
\isadelimproof
\isanewline
%
\endisadelimproof
\isanewline
\isacommand{lemma}\isamarkupfalse%
\ wf{\isacharunderscore}{\kern0pt}prel\ {\isacharcolon}{\kern0pt}\ {\isachardoublequoteopen}wf{\isacharparenleft}{\kern0pt}r{\isacharparenright}{\kern0pt}\ {\isasymLongrightarrow}\ wf{\isacharparenleft}{\kern0pt}prel{\isacharparenleft}{\kern0pt}r{\isacharcomma}{\kern0pt}\ A{\isacharparenright}{\kern0pt}{\isacharparenright}{\kern0pt}{\isachardoublequoteclose}\ \isanewline
%
\isadelimproof
\ \ %
\endisadelimproof
%
\isatagproof
\isacommand{apply}\isamarkupfalse%
{\isacharparenleft}{\kern0pt}rule{\isacharunderscore}{\kern0pt}tac\ A{\isacharequal}{\kern0pt}{\isachardoublequoteopen}{\isacharparenleft}{\kern0pt}field{\isacharparenleft}{\kern0pt}r{\isacharparenright}{\kern0pt}\ {\isasymtimes}\ A{\isacharparenright}{\kern0pt}{\isachardoublequoteclose}\ \isakeyword{in}\ wfI{\isacharparenright}{\kern0pt}\ \isanewline
\ \ \isacommand{apply}\isamarkupfalse%
{\isacharparenleft}{\kern0pt}simp\ add{\isacharcolon}{\kern0pt}prel{\isacharunderscore}{\kern0pt}def\ field{\isacharunderscore}{\kern0pt}def{\isacharparenright}{\kern0pt}\ \isacommand{apply}\isamarkupfalse%
\ blast\ \isanewline
\isacommand{proof}\isamarkupfalse%
{\isacharparenleft}{\kern0pt}clarify{\isacharparenright}{\kern0pt}\ \isanewline
\ \ \isacommand{fix}\isamarkupfalse%
\ B\ a\ x\isanewline
\ \ \isacommand{assume}\isamarkupfalse%
\ assms{\isacharcolon}{\kern0pt}\ {\isachardoublequoteopen}wf{\isacharparenleft}{\kern0pt}r{\isacharparenright}{\kern0pt}{\isachardoublequoteclose}\ {\isachardoublequoteopen}x\ {\isasymin}\ field{\isacharparenleft}{\kern0pt}r{\isacharparenright}{\kern0pt}{\isachardoublequoteclose}\ {\isachardoublequoteopen}a\ {\isasymin}\ A{\isachardoublequoteclose}\ {\isachardoublequoteopen}{\isasymforall}x{\isasymin}field{\isacharparenleft}{\kern0pt}r{\isacharparenright}{\kern0pt}\ {\isasymtimes}\ A{\isachardot}{\kern0pt}\ {\isacharparenleft}{\kern0pt}{\isasymforall}y{\isasymin}field{\isacharparenleft}{\kern0pt}r{\isacharparenright}{\kern0pt}\ {\isasymtimes}\ A{\isachardot}{\kern0pt}\ {\isasymlangle}y{\isacharcomma}{\kern0pt}\ x{\isasymrangle}\ {\isasymin}\ prel{\isacharparenleft}{\kern0pt}r{\isacharcomma}{\kern0pt}\ A{\isacharparenright}{\kern0pt}\ {\isasymlongrightarrow}\ y\ {\isasymin}\ B{\isacharparenright}{\kern0pt}\ {\isasymlongrightarrow}\ x\ {\isasymin}\ B{\isachardoublequoteclose}\ \isanewline
\isanewline
\ \ \isacommand{then}\isamarkupfalse%
\ \isacommand{have}\isamarkupfalse%
\ ih\ {\isacharcolon}{\kern0pt}\ \isanewline
\ \ \ \ {\isachardoublequoteopen}{\isasymAnd}x\ a{\isachardot}{\kern0pt}\ {\isacharless}{\kern0pt}x{\isacharcomma}{\kern0pt}\ a{\isachargreater}{\kern0pt}\ {\isasymin}\ field{\isacharparenleft}{\kern0pt}r{\isacharparenright}{\kern0pt}\ {\isasymtimes}\ A\ {\isasymLongrightarrow}\ {\isacharparenleft}{\kern0pt}{\isasymAnd}\ y\ b{\isachardot}{\kern0pt}\ {\isacharless}{\kern0pt}y{\isacharcomma}{\kern0pt}\ b{\isachargreater}{\kern0pt}\ {\isasymin}\ field{\isacharparenleft}{\kern0pt}r{\isacharparenright}{\kern0pt}\ {\isasymtimes}\ A\ {\isasymLongrightarrow}\ {\isasymlangle}{\isacharless}{\kern0pt}y{\isacharcomma}{\kern0pt}\ b{\isachargreater}{\kern0pt}{\isacharcomma}{\kern0pt}\ {\isacharless}{\kern0pt}x{\isacharcomma}{\kern0pt}\ a{\isachargreater}{\kern0pt}{\isasymrangle}\ {\isasymin}\ prel{\isacharparenleft}{\kern0pt}r{\isacharcomma}{\kern0pt}\ A{\isacharparenright}{\kern0pt}\ {\isasymLongrightarrow}\ {\isacharless}{\kern0pt}y{\isacharcomma}{\kern0pt}\ b{\isachargreater}{\kern0pt}\ {\isasymin}\ B{\isacharparenright}{\kern0pt}\ {\isasymLongrightarrow}\ {\isacharless}{\kern0pt}x{\isacharcomma}{\kern0pt}\ a{\isachargreater}{\kern0pt}\ {\isasymin}\ B{\isachardoublequoteclose}\ \isacommand{by}\isamarkupfalse%
\ auto\ \isanewline
\isanewline
\ \ \isacommand{have}\isamarkupfalse%
\ {\isachardoublequoteopen}{\isacharless}{\kern0pt}x{\isacharcomma}{\kern0pt}\ a{\isachargreater}{\kern0pt}\ {\isasymin}\ field{\isacharparenleft}{\kern0pt}r{\isacharparenright}{\kern0pt}\ {\isasymtimes}\ A\ {\isasymlongrightarrow}\ {\isacharless}{\kern0pt}x{\isacharcomma}{\kern0pt}\ a{\isachargreater}{\kern0pt}\ {\isasymin}\ B{\isachardoublequoteclose}\ \isanewline
\ \ \ \ \isacommand{apply}\isamarkupfalse%
{\isacharparenleft}{\kern0pt}rule{\isacharunderscore}{\kern0pt}tac\ r{\isacharequal}{\kern0pt}r\ \isakeyword{and}\ P{\isacharequal}{\kern0pt}{\isachardoublequoteopen}{\isasymlambda}x{\isachardot}{\kern0pt}\ {\isacharless}{\kern0pt}x{\isacharcomma}{\kern0pt}\ a{\isachargreater}{\kern0pt}\ {\isasymin}\ field{\isacharparenleft}{\kern0pt}r{\isacharparenright}{\kern0pt}\ {\isasymtimes}\ A\ {\isasymlongrightarrow}\ {\isacharless}{\kern0pt}x{\isacharcomma}{\kern0pt}\ a{\isachargreater}{\kern0pt}\ {\isasymin}\ B{\isachardoublequoteclose}\ \isakeyword{in}\ wf{\isacharunderscore}{\kern0pt}induct{\isacharparenright}{\kern0pt}\ \isacommand{using}\isamarkupfalse%
\ assms\ \isacommand{apply}\isamarkupfalse%
\ simp\ \isanewline
\ \ \ \ \isacommand{apply}\isamarkupfalse%
\ clarify\ \isacommand{apply}\isamarkupfalse%
{\isacharparenleft}{\kern0pt}rule{\isacharunderscore}{\kern0pt}tac\ ih{\isacharparenright}{\kern0pt}\ \isacommand{using}\isamarkupfalse%
\ assms\ \isacommand{apply}\isamarkupfalse%
\ simp\ \isanewline
\ \ \isacommand{proof}\isamarkupfalse%
\ {\isacharminus}{\kern0pt}\isanewline
\ \ \ \ \isacommand{fix}\isamarkupfalse%
\ x\ y\ b\ \isacommand{assume}\isamarkupfalse%
\ assms{\isadigit{1}}{\isacharcolon}{\kern0pt}\ {\isachardoublequoteopen}{\isasymlangle}{\isasymlangle}y{\isacharcomma}{\kern0pt}\ b{\isasymrangle}{\isacharcomma}{\kern0pt}\ x{\isacharcomma}{\kern0pt}\ a{\isasymrangle}\ {\isasymin}\ prel{\isacharparenleft}{\kern0pt}r{\isacharcomma}{\kern0pt}\ A{\isacharparenright}{\kern0pt}{\isachardoublequoteclose}\ {\isachardoublequoteopen}{\isacharparenleft}{\kern0pt}{\isasymAnd}y{\isachardot}{\kern0pt}\ {\isasymlangle}y{\isacharcomma}{\kern0pt}\ x{\isasymrangle}\ {\isasymin}\ r\ {\isasymLongrightarrow}\ {\isasymlangle}y{\isacharcomma}{\kern0pt}\ a{\isasymrangle}\ {\isasymin}\ field{\isacharparenleft}{\kern0pt}r{\isacharparenright}{\kern0pt}\ {\isasymtimes}\ A\ {\isasymlongrightarrow}\ {\isasymlangle}y{\isacharcomma}{\kern0pt}\ a{\isasymrangle}\ {\isasymin}\ B{\isacharparenright}{\kern0pt}{\isachardoublequoteclose}\isanewline
\ \ \ \ \isacommand{then}\isamarkupfalse%
\ \isacommand{have}\isamarkupfalse%
\ H{\isadigit{1}}{\isacharcolon}{\kern0pt}{\isachardoublequoteopen}b\ {\isacharequal}{\kern0pt}\ a{\isachardoublequoteclose}\ \isacommand{unfolding}\isamarkupfalse%
\ prel{\isacharunderscore}{\kern0pt}def\ \isacommand{by}\isamarkupfalse%
\ auto\ \isanewline
\ \ \ \ \isacommand{then}\isamarkupfalse%
\ \isacommand{have}\isamarkupfalse%
\ H{\isadigit{2}}{\isacharcolon}{\kern0pt}\ {\isachardoublequoteopen}{\isacharless}{\kern0pt}y{\isacharcomma}{\kern0pt}\ x{\isachargreater}{\kern0pt}\ {\isasymin}\ r{\isachardoublequoteclose}\ \isacommand{using}\isamarkupfalse%
\ assms{\isadigit{1}}\ \isacommand{unfolding}\isamarkupfalse%
\ prel{\isacharunderscore}{\kern0pt}def\ \isacommand{by}\isamarkupfalse%
\ auto\isanewline
\ \ \ \ \isacommand{have}\isamarkupfalse%
\ {\isachardoublequoteopen}{\isacharless}{\kern0pt}y{\isacharcomma}{\kern0pt}\ a{\isachargreater}{\kern0pt}\ {\isasymin}\ field{\isacharparenleft}{\kern0pt}r{\isacharparenright}{\kern0pt}\ {\isasymtimes}\ A{\isachardoublequoteclose}\ \isacommand{using}\isamarkupfalse%
\ assms{\isadigit{1}}\ \isacommand{unfolding}\isamarkupfalse%
\ prel{\isacharunderscore}{\kern0pt}def\ \isacommand{by}\isamarkupfalse%
\ auto\isanewline
\ \ \ \ \isacommand{then}\isamarkupfalse%
\ \isacommand{show}\isamarkupfalse%
\ {\isachardoublequoteopen}{\isasymlangle}y{\isacharcomma}{\kern0pt}\ b{\isasymrangle}\ {\isasymin}\ B{\isachardoublequoteclose}\ \isacommand{using}\isamarkupfalse%
\ assms{\isadigit{1}}\ H{\isadigit{1}}\ H{\isadigit{2}}\ \isacommand{by}\isamarkupfalse%
\ auto\ \isanewline
\ \ \isacommand{qed}\isamarkupfalse%
\isanewline
\ \ \isacommand{then}\isamarkupfalse%
\ \isacommand{show}\isamarkupfalse%
\ {\isachardoublequoteopen}{\isacharless}{\kern0pt}x{\isacharcomma}{\kern0pt}\ a{\isachargreater}{\kern0pt}\ {\isasymin}\ B{\isachardoublequoteclose}\ \isacommand{using}\isamarkupfalse%
\ assms\ \isacommand{by}\isamarkupfalse%
\ auto\ \isanewline
\isacommand{qed}\isamarkupfalse%
%
\endisatagproof
{\isafoldproof}%
%
\isadelimproof
\ \isanewline
%
\endisadelimproof
\isanewline
\isacommand{lemma}\isamarkupfalse%
\ prel{\isacharunderscore}{\kern0pt}trans\ {\isacharcolon}{\kern0pt}\ {\isachardoublequoteopen}trans{\isacharparenleft}{\kern0pt}r{\isacharparenright}{\kern0pt}\ {\isasymLongrightarrow}\ trans{\isacharparenleft}{\kern0pt}prel{\isacharparenleft}{\kern0pt}r{\isacharcomma}{\kern0pt}\ p{\isacharparenright}{\kern0pt}{\isacharparenright}{\kern0pt}{\isachardoublequoteclose}\ \isanewline
%
\isadelimproof
\ \ %
\endisadelimproof
%
\isatagproof
\isacommand{unfolding}\isamarkupfalse%
\ trans{\isacharunderscore}{\kern0pt}def\ \isacommand{apply}\isamarkupfalse%
\ clarify\isanewline
\isacommand{proof}\isamarkupfalse%
\ {\isacharminus}{\kern0pt}\ \isanewline
\ \ \isacommand{fix}\isamarkupfalse%
\ x\ y\ z\ \isacommand{assume}\isamarkupfalse%
\ assms{\isacharcolon}{\kern0pt}\ {\isachardoublequoteopen}{\isasymforall}x\ y\ z{\isachardot}{\kern0pt}\ {\isasymlangle}x{\isacharcomma}{\kern0pt}\ y{\isasymrangle}\ {\isasymin}\ r\ {\isasymlongrightarrow}\ {\isasymlangle}y{\isacharcomma}{\kern0pt}\ z{\isasymrangle}\ {\isasymin}\ r\ {\isasymlongrightarrow}\ {\isasymlangle}x{\isacharcomma}{\kern0pt}\ z{\isasymrangle}\ {\isasymin}\ r{\isachardoublequoteclose}\ {\isachardoublequoteopen}{\isasymlangle}x{\isacharcomma}{\kern0pt}\ y{\isasymrangle}\ {\isasymin}\ prel{\isacharparenleft}{\kern0pt}r{\isacharcomma}{\kern0pt}\ p{\isacharparenright}{\kern0pt}{\isachardoublequoteclose}\ {\isachardoublequoteopen}{\isasymlangle}y{\isacharcomma}{\kern0pt}\ z{\isasymrangle}\ {\isasymin}\ prel{\isacharparenleft}{\kern0pt}r{\isacharcomma}{\kern0pt}\ p{\isacharparenright}{\kern0pt}{\isachardoublequoteclose}\isanewline
\ \ \isacommand{then}\isamarkupfalse%
\ \isacommand{have}\isamarkupfalse%
\ {\isachardoublequoteopen}\ {\isasymexists}x{\isacharprime}{\kern0pt}\ y{\isacharprime}{\kern0pt}\ a{\isachardot}{\kern0pt}\ a\ {\isasymin}\ p\ {\isasymand}\ x\ {\isacharequal}{\kern0pt}\ {\isacharless}{\kern0pt}x{\isacharprime}{\kern0pt}{\isacharcomma}{\kern0pt}\ a{\isachargreater}{\kern0pt}\ {\isasymand}\ y\ {\isacharequal}{\kern0pt}\ {\isacharless}{\kern0pt}y{\isacharprime}{\kern0pt}{\isacharcomma}{\kern0pt}\ a{\isachargreater}{\kern0pt}\ {\isasymand}\ {\isacharless}{\kern0pt}x{\isacharprime}{\kern0pt}{\isacharcomma}{\kern0pt}\ y{\isacharprime}{\kern0pt}{\isachargreater}{\kern0pt}\ {\isasymin}\ r{\isachardoublequoteclose}\ \ \isacommand{apply}\isamarkupfalse%
{\isacharparenleft}{\kern0pt}rule{\isacharunderscore}{\kern0pt}tac\ prelD{\isacharparenright}{\kern0pt}\ \isacommand{by}\isamarkupfalse%
\ simp\ \isanewline
\ \ \isacommand{then}\isamarkupfalse%
\ \isacommand{obtain}\isamarkupfalse%
\ x{\isacharprime}{\kern0pt}\ y{\isacharprime}{\kern0pt}\ a\ \isakeyword{where}\ H{\isadigit{1}}{\isacharcolon}{\kern0pt}\ {\isachardoublequoteopen}a\ {\isasymin}\ p{\isachardoublequoteclose}\ {\isachardoublequoteopen}x\ {\isacharequal}{\kern0pt}\ {\isacharless}{\kern0pt}x{\isacharprime}{\kern0pt}{\isacharcomma}{\kern0pt}\ a{\isachargreater}{\kern0pt}{\isachardoublequoteclose}\ {\isachardoublequoteopen}y\ {\isacharequal}{\kern0pt}\ {\isacharless}{\kern0pt}y{\isacharprime}{\kern0pt}{\isacharcomma}{\kern0pt}\ a{\isachargreater}{\kern0pt}{\isachardoublequoteclose}\ {\isachardoublequoteopen}{\isacharless}{\kern0pt}x{\isacharprime}{\kern0pt}{\isacharcomma}{\kern0pt}\ y{\isacharprime}{\kern0pt}{\isachargreater}{\kern0pt}\ {\isasymin}\ r{\isachardoublequoteclose}\ \isacommand{by}\isamarkupfalse%
\ auto\ \isanewline
\ \ \isacommand{then}\isamarkupfalse%
\ \isacommand{have}\isamarkupfalse%
\ {\isachardoublequoteopen}\ {\isasymexists}y{\isacharprime}{\kern0pt}\ z{\isacharprime}{\kern0pt}\ b{\isachardot}{\kern0pt}\ b\ {\isasymin}\ p\ {\isasymand}\ y\ {\isacharequal}{\kern0pt}\ {\isacharless}{\kern0pt}y{\isacharprime}{\kern0pt}{\isacharcomma}{\kern0pt}\ b{\isachargreater}{\kern0pt}\ {\isasymand}\ z\ {\isacharequal}{\kern0pt}\ {\isacharless}{\kern0pt}z{\isacharprime}{\kern0pt}{\isacharcomma}{\kern0pt}\ b{\isachargreater}{\kern0pt}\ {\isasymand}\ {\isacharless}{\kern0pt}y{\isacharprime}{\kern0pt}{\isacharcomma}{\kern0pt}\ z{\isacharprime}{\kern0pt}{\isachargreater}{\kern0pt}\ {\isasymin}\ r{\isachardoublequoteclose}\ \ \isacommand{apply}\isamarkupfalse%
{\isacharparenleft}{\kern0pt}rule{\isacharunderscore}{\kern0pt}tac\ prelD{\isacharparenright}{\kern0pt}\ \isacommand{using}\isamarkupfalse%
\ assms\ \isacommand{by}\isamarkupfalse%
\ simp\ \isanewline
\ \ \isacommand{then}\isamarkupfalse%
\ \isacommand{obtain}\isamarkupfalse%
\ y{\isacharprime}{\kern0pt}\ z{\isacharprime}{\kern0pt}\ b\ \isakeyword{where}\ H{\isadigit{2}}{\isacharcolon}{\kern0pt}\ {\isachardoublequoteopen}y\ {\isacharequal}{\kern0pt}\ {\isacharless}{\kern0pt}y{\isacharprime}{\kern0pt}{\isacharcomma}{\kern0pt}\ b{\isachargreater}{\kern0pt}{\isachardoublequoteclose}\ {\isachardoublequoteopen}z\ {\isacharequal}{\kern0pt}\ {\isacharless}{\kern0pt}z{\isacharprime}{\kern0pt}{\isacharcomma}{\kern0pt}\ b{\isachargreater}{\kern0pt}{\isachardoublequoteclose}\ {\isachardoublequoteopen}{\isacharless}{\kern0pt}y{\isacharprime}{\kern0pt}{\isacharcomma}{\kern0pt}\ z{\isacharprime}{\kern0pt}{\isachargreater}{\kern0pt}\ {\isasymin}\ r{\isachardoublequoteclose}\ \isacommand{by}\isamarkupfalse%
\ auto\ \isanewline
\ \ \isacommand{have}\isamarkupfalse%
\ {\isachardoublequoteopen}a\ {\isacharequal}{\kern0pt}\ b{\isachardoublequoteclose}\ \isacommand{using}\isamarkupfalse%
\ H{\isadigit{1}}\ H{\isadigit{2}}\ \isacommand{by}\isamarkupfalse%
\ auto\ \isanewline
\ \ \isacommand{have}\isamarkupfalse%
\ {\isachardoublequoteopen}{\isacharless}{\kern0pt}x{\isacharprime}{\kern0pt}{\isacharcomma}{\kern0pt}\ z{\isacharprime}{\kern0pt}{\isachargreater}{\kern0pt}\ {\isasymin}\ r{\isachardoublequoteclose}\ \isacommand{using}\isamarkupfalse%
\ H{\isadigit{1}}\ H{\isadigit{2}}\ assms\ \isacommand{by}\isamarkupfalse%
\ auto\ \isanewline
\ \ \isacommand{then}\isamarkupfalse%
\ \isacommand{have}\isamarkupfalse%
\ {\isachardoublequoteopen}{\isacharless}{\kern0pt}{\isacharless}{\kern0pt}x{\isacharprime}{\kern0pt}{\isacharcomma}{\kern0pt}\ a{\isachargreater}{\kern0pt}{\isacharcomma}{\kern0pt}\ {\isacharless}{\kern0pt}z{\isacharprime}{\kern0pt}{\isacharcomma}{\kern0pt}\ a{\isachargreater}{\kern0pt}{\isachargreater}{\kern0pt}\ {\isasymin}\ prel{\isacharparenleft}{\kern0pt}r{\isacharcomma}{\kern0pt}\ p{\isacharparenright}{\kern0pt}{\isachardoublequoteclose}\ \isanewline
\ \ \ \ \isacommand{apply}\isamarkupfalse%
{\isacharparenleft}{\kern0pt}rule{\isacharunderscore}{\kern0pt}tac\ prelI{\isacharparenright}{\kern0pt}\ \isacommand{using}\isamarkupfalse%
\ H{\isadigit{1}}\ \isacommand{by}\isamarkupfalse%
\ auto\ \isanewline
\ \ \isacommand{then}\isamarkupfalse%
\ \isacommand{show}\isamarkupfalse%
\ {\isachardoublequoteopen}{\isacharless}{\kern0pt}x{\isacharcomma}{\kern0pt}\ z{\isachargreater}{\kern0pt}\ {\isasymin}\ prel{\isacharparenleft}{\kern0pt}r{\isacharcomma}{\kern0pt}\ p{\isacharparenright}{\kern0pt}{\isachardoublequoteclose}\ \isacommand{using}\isamarkupfalse%
\ H{\isadigit{1}}\ H{\isadigit{2}}\ \isacommand{by}\isamarkupfalse%
\ auto\ \isanewline
\isacommand{qed}\isamarkupfalse%
%
\endisatagproof
{\isafoldproof}%
%
\isadelimproof
\isanewline
%
\endisadelimproof
\isanewline
\isacommand{end}\isamarkupfalse%
\isanewline
\isanewline
\isacommand{definition}\isamarkupfalse%
\ is{\isacharunderscore}{\kern0pt}preds{\isacharunderscore}{\kern0pt}prel{\isacharunderscore}{\kern0pt}elem{\isacharunderscore}{\kern0pt}fm\ \isakeyword{where}\ \isanewline
\ \ {\isachardoublequoteopen}is{\isacharunderscore}{\kern0pt}preds{\isacharunderscore}{\kern0pt}prel{\isacharunderscore}{\kern0pt}elem{\isacharunderscore}{\kern0pt}fm{\isacharparenleft}{\kern0pt}Rfm{\isacharcomma}{\kern0pt}\ x{\isacharcomma}{\kern0pt}\ a{\isacharcomma}{\kern0pt}\ v{\isacharparenright}{\kern0pt}\ {\isasymequiv}\ Exists{\isacharparenleft}{\kern0pt}And{\isacharparenleft}{\kern0pt}is{\isacharunderscore}{\kern0pt}preds{\isacharunderscore}{\kern0pt}rel{\isacharunderscore}{\kern0pt}fm{\isacharparenleft}{\kern0pt}Rfm{\isacharcomma}{\kern0pt}\ x\ {\isacharhash}{\kern0pt}{\isacharplus}{\kern0pt}\ {\isadigit{1}}{\isacharcomma}{\kern0pt}\ {\isadigit{0}}{\isacharparenright}{\kern0pt}{\isacharcomma}{\kern0pt}\ Exists{\isacharparenleft}{\kern0pt}Exists{\isacharparenleft}{\kern0pt}Exists{\isacharparenleft}{\kern0pt}Exists{\isacharparenleft}{\kern0pt}Exists{\isacharparenleft}{\kern0pt}And{\isacharparenleft}{\kern0pt}pair{\isacharunderscore}{\kern0pt}fm{\isacharparenleft}{\kern0pt}{\isadigit{2}}{\isacharcomma}{\kern0pt}\ {\isadigit{3}}{\isacharcomma}{\kern0pt}\ v\ {\isacharhash}{\kern0pt}{\isacharplus}{\kern0pt}\ {\isadigit{6}}{\isacharparenright}{\kern0pt}{\isacharcomma}{\kern0pt}\ And{\isacharparenleft}{\kern0pt}pair{\isacharunderscore}{\kern0pt}fm{\isacharparenleft}{\kern0pt}{\isadigit{0}}{\isacharcomma}{\kern0pt}\ a\ {\isacharhash}{\kern0pt}{\isacharplus}{\kern0pt}\ {\isadigit{6}}{\isacharcomma}{\kern0pt}\ {\isadigit{2}}{\isacharparenright}{\kern0pt}{\isacharcomma}{\kern0pt}\ And{\isacharparenleft}{\kern0pt}pair{\isacharunderscore}{\kern0pt}fm{\isacharparenleft}{\kern0pt}{\isadigit{1}}{\isacharcomma}{\kern0pt}\ a\ {\isacharhash}{\kern0pt}{\isacharplus}{\kern0pt}\ {\isadigit{6}}{\isacharcomma}{\kern0pt}\ {\isadigit{3}}{\isacharparenright}{\kern0pt}{\isacharcomma}{\kern0pt}\ And{\isacharparenleft}{\kern0pt}pair{\isacharunderscore}{\kern0pt}fm{\isacharparenleft}{\kern0pt}{\isadigit{0}}{\isacharcomma}{\kern0pt}\ {\isadigit{1}}{\isacharcomma}{\kern0pt}\ {\isadigit{4}}{\isacharparenright}{\kern0pt}{\isacharcomma}{\kern0pt}\ Member{\isacharparenleft}{\kern0pt}{\isadigit{4}}{\isacharcomma}{\kern0pt}\ {\isadigit{5}}{\isacharparenright}{\kern0pt}{\isacharparenright}{\kern0pt}{\isacharparenright}{\kern0pt}{\isacharparenright}{\kern0pt}{\isacharparenright}{\kern0pt}{\isacharparenright}{\kern0pt}{\isacharparenright}{\kern0pt}{\isacharparenright}{\kern0pt}{\isacharparenright}{\kern0pt}{\isacharparenright}{\kern0pt}{\isacharparenright}{\kern0pt}{\isacharparenright}{\kern0pt}{\isachardoublequoteclose}\ \isanewline
\isanewline
\isacommand{context}\isamarkupfalse%
\ M{\isacharunderscore}{\kern0pt}ctm\ \isanewline
\isakeyword{begin}\ \isanewline
\isanewline
\isacommand{lemma}\isamarkupfalse%
\ is{\isacharunderscore}{\kern0pt}preds{\isacharunderscore}{\kern0pt}prel{\isacharunderscore}{\kern0pt}elem{\isacharunderscore}{\kern0pt}fm{\isacharunderscore}{\kern0pt}type\ {\isacharcolon}{\kern0pt}\ \isanewline
\ \ \isakeyword{fixes}\ Rfm\ x\ a\ v\ \isanewline
\ \ \isakeyword{assumes}\ {\isachardoublequoteopen}x\ {\isasymin}\ nat{\isachardoublequoteclose}\ {\isachardoublequoteopen}a\ {\isasymin}\ nat{\isachardoublequoteclose}\ {\isachardoublequoteopen}v\ {\isasymin}\ nat{\isachardoublequoteclose}\ {\isachardoublequoteopen}Rfm\ {\isasymin}\ formula{\isachardoublequoteclose}\isanewline
\ \ \isakeyword{shows}\ {\isachardoublequoteopen}is{\isacharunderscore}{\kern0pt}preds{\isacharunderscore}{\kern0pt}prel{\isacharunderscore}{\kern0pt}elem{\isacharunderscore}{\kern0pt}fm{\isacharparenleft}{\kern0pt}Rfm{\isacharcomma}{\kern0pt}\ x{\isacharcomma}{\kern0pt}\ a{\isacharcomma}{\kern0pt}\ v{\isacharparenright}{\kern0pt}\ {\isasymin}\ formula{\isachardoublequoteclose}\ \isanewline
%
\isadelimproof
\isanewline
\ \ %
\endisadelimproof
%
\isatagproof
\isacommand{unfolding}\isamarkupfalse%
\ is{\isacharunderscore}{\kern0pt}preds{\isacharunderscore}{\kern0pt}prel{\isacharunderscore}{\kern0pt}elem{\isacharunderscore}{\kern0pt}fm{\isacharunderscore}{\kern0pt}def\isanewline
\ \ \isacommand{apply}\isamarkupfalse%
\ {\isacharparenleft}{\kern0pt}subgoal{\isacharunderscore}{\kern0pt}tac\ {\isachardoublequoteopen}is{\isacharunderscore}{\kern0pt}preds{\isacharunderscore}{\kern0pt}rel{\isacharunderscore}{\kern0pt}fm{\isacharparenleft}{\kern0pt}Rfm{\isacharcomma}{\kern0pt}\ x\ {\isacharhash}{\kern0pt}{\isacharplus}{\kern0pt}\ {\isadigit{1}}{\isacharcomma}{\kern0pt}\ {\isadigit{0}}{\isacharparenright}{\kern0pt}\ {\isasymin}\ formula{\isachardoublequoteclose}{\isacharparenright}{\kern0pt}\isanewline
\ \ \ \isacommand{apply}\isamarkupfalse%
\ simp\isanewline
\ \ \isacommand{apply}\isamarkupfalse%
{\isacharparenleft}{\kern0pt}rule\ is{\isacharunderscore}{\kern0pt}preds{\isacharunderscore}{\kern0pt}rel{\isacharunderscore}{\kern0pt}fm{\isacharunderscore}{\kern0pt}type{\isacharparenright}{\kern0pt}\isanewline
\ \ \isacommand{using}\isamarkupfalse%
\ assms\isanewline
\ \ \isacommand{by}\isamarkupfalse%
\ auto%
\endisatagproof
{\isafoldproof}%
%
\isadelimproof
\isanewline
%
\endisadelimproof
\isanewline
\isacommand{lemma}\isamarkupfalse%
\ is{\isacharunderscore}{\kern0pt}preds{\isacharunderscore}{\kern0pt}prel{\isacharunderscore}{\kern0pt}elem{\isacharunderscore}{\kern0pt}fm{\isacharunderscore}{\kern0pt}arity\ {\isacharcolon}{\kern0pt}\ \isanewline
\ \ \isakeyword{fixes}\ Rfm\ x\ a\ v\ \isanewline
\ \ \isakeyword{assumes}\ {\isachardoublequoteopen}x\ {\isasymin}\ nat{\isachardoublequoteclose}\ {\isachardoublequoteopen}a\ {\isasymin}\ nat{\isachardoublequoteclose}\ {\isachardoublequoteopen}v\ {\isasymin}\ nat{\isachardoublequoteclose}\ {\isachardoublequoteopen}Rfm\ {\isasymin}\ formula{\isachardoublequoteclose}\ {\isachardoublequoteopen}arity{\isacharparenleft}{\kern0pt}Rfm{\isacharparenright}{\kern0pt}\ {\isacharequal}{\kern0pt}\ {\isadigit{2}}{\isachardoublequoteclose}\isanewline
\ \ \isakeyword{shows}\ {\isachardoublequoteopen}arity{\isacharparenleft}{\kern0pt}is{\isacharunderscore}{\kern0pt}preds{\isacharunderscore}{\kern0pt}prel{\isacharunderscore}{\kern0pt}elem{\isacharunderscore}{\kern0pt}fm{\isacharparenleft}{\kern0pt}Rfm{\isacharcomma}{\kern0pt}\ x{\isacharcomma}{\kern0pt}\ a{\isacharcomma}{\kern0pt}\ v{\isacharparenright}{\kern0pt}{\isacharparenright}{\kern0pt}\ {\isasymle}\ succ{\isacharparenleft}{\kern0pt}x{\isacharparenright}{\kern0pt}\ {\isasymunion}\ succ{\isacharparenleft}{\kern0pt}a{\isacharparenright}{\kern0pt}\ {\isasymunion}\ succ{\isacharparenleft}{\kern0pt}v{\isacharparenright}{\kern0pt}{\isachardoublequoteclose}\isanewline
%
\isadelimproof
\ \ %
\endisadelimproof
%
\isatagproof
\isacommand{unfolding}\isamarkupfalse%
\ is{\isacharunderscore}{\kern0pt}preds{\isacharunderscore}{\kern0pt}prel{\isacharunderscore}{\kern0pt}elem{\isacharunderscore}{\kern0pt}fm{\isacharunderscore}{\kern0pt}def\ pair{\isacharunderscore}{\kern0pt}fm{\isacharunderscore}{\kern0pt}def\ upair{\isacharunderscore}{\kern0pt}fm{\isacharunderscore}{\kern0pt}def\isanewline
\ \ \isacommand{using}\isamarkupfalse%
\ assms\ \isanewline
\ \ \isacommand{apply}\isamarkupfalse%
\ simp\isanewline
\ \ \isacommand{apply}\isamarkupfalse%
{\isacharparenleft}{\kern0pt}subgoal{\isacharunderscore}{\kern0pt}tac\ {\isachardoublequoteopen}is{\isacharunderscore}{\kern0pt}preds{\isacharunderscore}{\kern0pt}rel{\isacharunderscore}{\kern0pt}fm{\isacharparenleft}{\kern0pt}Rfm{\isacharcomma}{\kern0pt}\ succ{\isacharparenleft}{\kern0pt}x{\isacharparenright}{\kern0pt}{\isacharcomma}{\kern0pt}\ {\isadigit{0}}{\isacharparenright}{\kern0pt}\ {\isasymin}\ formula{\isachardoublequoteclose}{\isacharparenright}{\kern0pt}\isanewline
\ \ \ \isacommand{apply}\isamarkupfalse%
{\isacharparenleft}{\kern0pt}subst\ pred{\isacharunderscore}{\kern0pt}Un{\isacharunderscore}{\kern0pt}distrib{\isacharcomma}{\kern0pt}\ simp{\isacharunderscore}{\kern0pt}all{\isacharparenright}{\kern0pt}{\isacharplus}{\kern0pt}\isanewline
\ \ \ \isacommand{apply}\isamarkupfalse%
{\isacharparenleft}{\kern0pt}rule\ Un{\isacharunderscore}{\kern0pt}least{\isacharunderscore}{\kern0pt}lt{\isacharparenright}{\kern0pt}\isanewline
\ \ \ \ \isacommand{apply}\isamarkupfalse%
{\isacharparenleft}{\kern0pt}rule{\isacharunderscore}{\kern0pt}tac\ j\ {\isacharequal}{\kern0pt}\ {\isachardoublequoteopen}pred{\isacharparenleft}{\kern0pt}succ{\isacharparenleft}{\kern0pt}succ{\isacharparenleft}{\kern0pt}x{\isacharparenright}{\kern0pt}{\isacharparenright}{\kern0pt}\ {\isasymunion}\ {\isadigit{1}}{\isacharparenright}{\kern0pt}{\isachardoublequoteclose}\ \isakeyword{in}\ le{\isacharunderscore}{\kern0pt}trans{\isacharcomma}{\kern0pt}\ rule\ pred{\isacharunderscore}{\kern0pt}le{\isacharprime}{\kern0pt}{\isacharcomma}{\kern0pt}\ simp{\isacharunderscore}{\kern0pt}all{\isacharparenright}{\kern0pt}\isanewline
\ \ \ \ \ \isacommand{apply}\isamarkupfalse%
{\isacharparenleft}{\kern0pt}rule\ is{\isacharunderscore}{\kern0pt}preds{\isacharunderscore}{\kern0pt}rel{\isacharunderscore}{\kern0pt}fm{\isacharunderscore}{\kern0pt}arity{\isacharcomma}{\kern0pt}\ simp{\isacharunderscore}{\kern0pt}all{\isacharparenright}{\kern0pt}\isanewline
\ \ \ \ \isacommand{apply}\isamarkupfalse%
{\isacharparenleft}{\kern0pt}subst\ Ord{\isacharunderscore}{\kern0pt}un{\isacharunderscore}{\kern0pt}eq{\isadigit{1}}{\isacharcomma}{\kern0pt}\ simp{\isacharunderscore}{\kern0pt}all{\isacharparenright}{\kern0pt}\isanewline
\ \ \ \ \isacommand{apply}\isamarkupfalse%
{\isacharparenleft}{\kern0pt}rule\ ltI{\isacharcomma}{\kern0pt}\ simp{\isacharunderscore}{\kern0pt}all{\isacharparenright}{\kern0pt}\isanewline
\ \ \isacommand{apply}\isamarkupfalse%
{\isacharparenleft}{\kern0pt}rule\ Un{\isacharunderscore}{\kern0pt}least{\isacharunderscore}{\kern0pt}lt{\isacharcomma}{\kern0pt}\ simp\ add{\isacharcolon}{\kern0pt}ltI{\isacharcomma}{\kern0pt}\ simp\ add{\isacharcolon}{\kern0pt}ltI{\isacharparenright}{\kern0pt}\isanewline
\ \ \isacommand{apply}\isamarkupfalse%
{\isacharparenleft}{\kern0pt}rule\ is{\isacharunderscore}{\kern0pt}preds{\isacharunderscore}{\kern0pt}rel{\isacharunderscore}{\kern0pt}fm{\isacharunderscore}{\kern0pt}type{\isacharparenright}{\kern0pt}\isanewline
\ \ \isacommand{using}\isamarkupfalse%
\ assms\ \isanewline
\ \ \isacommand{by}\isamarkupfalse%
\ auto%
\endisatagproof
{\isafoldproof}%
%
\isadelimproof
\isanewline
%
\endisadelimproof
\isanewline
\isacommand{lemma}\isamarkupfalse%
\ sats{\isacharunderscore}{\kern0pt}is{\isacharunderscore}{\kern0pt}preds{\isacharunderscore}{\kern0pt}prel{\isacharunderscore}{\kern0pt}elem{\isacharunderscore}{\kern0pt}fm{\isacharunderscore}{\kern0pt}iff\ {\isacharcolon}{\kern0pt}\ \isanewline
\ \ \isakeyword{fixes}\ R\ Rfm\ i\ j\ k\ x\ a\ v\ env\ p\isanewline
\ \ \isakeyword{assumes}\ {\isachardoublequoteopen}Relation{\isacharunderscore}{\kern0pt}fm{\isacharparenleft}{\kern0pt}R{\isacharcomma}{\kern0pt}\ Rfm{\isacharparenright}{\kern0pt}{\isachardoublequoteclose}\ {\isachardoublequoteopen}preds{\isacharparenleft}{\kern0pt}R{\isacharcomma}{\kern0pt}\ x{\isacharparenright}{\kern0pt}\ {\isasymin}\ M{\isachardoublequoteclose}\ \ {\isachardoublequoteopen}env\ {\isasymin}\ list{\isacharparenleft}{\kern0pt}M{\isacharparenright}{\kern0pt}{\isachardoublequoteclose}\ {\isachardoublequoteopen}i\ {\isasymin}\ nat{\isachardoublequoteclose}\ {\isachardoublequoteopen}j\ {\isasymin}\ nat{\isachardoublequoteclose}\ {\isachardoublequoteopen}k\ {\isasymin}\ nat{\isachardoublequoteclose}\ {\isachardoublequoteopen}nth{\isacharparenleft}{\kern0pt}i{\isacharcomma}{\kern0pt}\ env{\isacharparenright}{\kern0pt}\ {\isacharequal}{\kern0pt}\ x{\isachardoublequoteclose}\ {\isachardoublequoteopen}nth{\isacharparenleft}{\kern0pt}j{\isacharcomma}{\kern0pt}\ env{\isacharparenright}{\kern0pt}\ {\isacharequal}{\kern0pt}\ a{\isachardoublequoteclose}\ {\isachardoublequoteopen}nth{\isacharparenleft}{\kern0pt}k{\isacharcomma}{\kern0pt}\ env{\isacharparenright}{\kern0pt}\ {\isacharequal}{\kern0pt}\ v{\isachardoublequoteclose}\ \isanewline
\ \ \ \ \ \ \ \ \ \ {\isachardoublequoteopen}v\ {\isasymin}\ M{\isachardoublequoteclose}\ {\isachardoublequoteopen}x\ {\isasymin}\ M{\isachardoublequoteclose}\ {\isachardoublequoteopen}a\ {\isasymin}\ M{\isachardoublequoteclose}\ {\isachardoublequoteopen}p\ {\isasymin}\ M{\isachardoublequoteclose}\ {\isachardoublequoteopen}a\ {\isasymin}\ p{\isachardoublequoteclose}\ \isanewline
\ \ \isakeyword{shows}\ {\isachardoublequoteopen}sats{\isacharparenleft}{\kern0pt}M{\isacharcomma}{\kern0pt}\ is{\isacharunderscore}{\kern0pt}preds{\isacharunderscore}{\kern0pt}prel{\isacharunderscore}{\kern0pt}elem{\isacharunderscore}{\kern0pt}fm{\isacharparenleft}{\kern0pt}Rfm{\isacharcomma}{\kern0pt}\ i{\isacharcomma}{\kern0pt}\ j{\isacharcomma}{\kern0pt}\ k{\isacharparenright}{\kern0pt}{\isacharcomma}{\kern0pt}\ env{\isacharparenright}{\kern0pt}\ {\isasymlongleftrightarrow}\ v\ {\isasymin}\ preds{\isacharunderscore}{\kern0pt}rel{\isacharparenleft}{\kern0pt}{\isasymlambda}a\ b{\isachardot}{\kern0pt}\ {\isacharless}{\kern0pt}a{\isacharcomma}{\kern0pt}\ b{\isachargreater}{\kern0pt}\ {\isasymin}\ prel{\isacharparenleft}{\kern0pt}Rrel{\isacharparenleft}{\kern0pt}R{\isacharcomma}{\kern0pt}\ M{\isacharparenright}{\kern0pt}{\isacharcomma}{\kern0pt}\ p{\isacharparenright}{\kern0pt}{\isacharcomma}{\kern0pt}\ {\isacharless}{\kern0pt}x{\isacharcomma}{\kern0pt}\ a{\isachargreater}{\kern0pt}{\isacharparenright}{\kern0pt}{\isachardoublequoteclose}\ \isanewline
%
\isadelimproof
%
\endisadelimproof
%
\isatagproof
\isacommand{proof}\isamarkupfalse%
\ {\isacharminus}{\kern0pt}\ \isanewline
\ \ \isacommand{have}\isamarkupfalse%
\ iff{\isacharunderscore}{\kern0pt}lemma\ {\isacharcolon}{\kern0pt}\ {\isachardoublequoteopen}{\isasymAnd}P\ Q\ R\ S{\isachardot}{\kern0pt}\ {\isacharparenleft}{\kern0pt}P\ {\isasymlongleftrightarrow}\ Q{\isacharparenright}{\kern0pt}\ {\isasymLongrightarrow}\ {\isacharparenleft}{\kern0pt}Q\ {\isasymLongrightarrow}\ R\ {\isasymlongleftrightarrow}\ S{\isacharparenright}{\kern0pt}\ {\isasymLongrightarrow}\ {\isacharparenleft}{\kern0pt}P\ {\isasymand}\ R{\isacharparenright}{\kern0pt}\ {\isasymlongleftrightarrow}\ {\isacharparenleft}{\kern0pt}Q\ {\isasymand}\ S{\isacharparenright}{\kern0pt}{\isachardoublequoteclose}\ \isacommand{by}\isamarkupfalse%
\ auto\isanewline
\ \ \isacommand{have}\isamarkupfalse%
\ iff{\isacharunderscore}{\kern0pt}lemma{\isadigit{2}}\ {\isacharcolon}{\kern0pt}\ {\isachardoublequoteopen}{\isasymAnd}P\ Q\ R\ S{\isachardot}{\kern0pt}\ {\isacharparenleft}{\kern0pt}P\ {\isasymlongleftrightarrow}\ Q{\isacharparenright}{\kern0pt}\ {\isasymLongrightarrow}\ {\isacharparenleft}{\kern0pt}R\ {\isasymlongleftrightarrow}\ S{\isacharparenright}{\kern0pt}\ {\isasymLongrightarrow}\ {\isacharparenleft}{\kern0pt}P\ {\isasymlongleftrightarrow}\ R{\isacharparenright}{\kern0pt}\ {\isasymlongleftrightarrow}\ {\isacharparenleft}{\kern0pt}Q\ {\isasymlongleftrightarrow}\ S{\isacharparenright}{\kern0pt}{\isachardoublequoteclose}\ \isacommand{by}\isamarkupfalse%
\ auto\isanewline
\ \ \isacommand{have}\isamarkupfalse%
\ iff{\isacharunderscore}{\kern0pt}lemma{\isadigit{3}}\ {\isacharcolon}{\kern0pt}\ {\isachardoublequoteopen}{\isasymAnd}A\ B\ C{\isachardot}{\kern0pt}\ B\ {\isacharequal}{\kern0pt}\ C\ {\isasymLongrightarrow}\ {\isacharparenleft}{\kern0pt}A\ {\isacharequal}{\kern0pt}\ B{\isacharparenright}{\kern0pt}\ {\isasymlongleftrightarrow}\ {\isacharparenleft}{\kern0pt}A\ {\isacharequal}{\kern0pt}\ C{\isacharparenright}{\kern0pt}{\isachardoublequoteclose}\ \isacommand{by}\isamarkupfalse%
\ auto\isanewline
\isanewline
\ \ \isacommand{have}\isamarkupfalse%
\ I{\isadigit{1}}\ {\isacharcolon}{\kern0pt}\ {\isachardoublequoteopen}sats{\isacharparenleft}{\kern0pt}M{\isacharcomma}{\kern0pt}\ is{\isacharunderscore}{\kern0pt}preds{\isacharunderscore}{\kern0pt}prel{\isacharunderscore}{\kern0pt}elem{\isacharunderscore}{\kern0pt}fm{\isacharparenleft}{\kern0pt}Rfm{\isacharcomma}{\kern0pt}\ i{\isacharcomma}{\kern0pt}\ j{\isacharcomma}{\kern0pt}\ k{\isacharparenright}{\kern0pt}{\isacharcomma}{\kern0pt}\ env{\isacharparenright}{\kern0pt}\ {\isasymlongleftrightarrow}\ {\isacharparenleft}{\kern0pt}{\isasymexists}A\ {\isasymin}\ M{\isachardot}{\kern0pt}\ A\ {\isacharequal}{\kern0pt}\ preds{\isacharunderscore}{\kern0pt}rel{\isacharparenleft}{\kern0pt}R{\isacharcomma}{\kern0pt}\ x{\isacharparenright}{\kern0pt}\ {\isasymand}\ {\isacharparenleft}{\kern0pt}{\isasymexists}zy\ {\isasymin}\ M{\isachardot}{\kern0pt}\ {\isasymexists}ya\ {\isasymin}\ M{\isachardot}{\kern0pt}\ {\isasymexists}za\ {\isasymin}\ M{\isachardot}{\kern0pt}\ {\isasymexists}y\ {\isasymin}\ M{\isachardot}{\kern0pt}\ {\isasymexists}z\ {\isasymin}\ M{\isachardot}{\kern0pt}\ v\ {\isacharequal}{\kern0pt}\ {\isacharless}{\kern0pt}za{\isacharcomma}{\kern0pt}\ ya{\isachargreater}{\kern0pt}\ {\isasymand}\ za\ {\isacharequal}{\kern0pt}\ {\isacharless}{\kern0pt}z{\isacharcomma}{\kern0pt}\ a{\isachargreater}{\kern0pt}\ {\isasymand}\ ya\ {\isacharequal}{\kern0pt}\ {\isacharless}{\kern0pt}y{\isacharcomma}{\kern0pt}\ a{\isachargreater}{\kern0pt}\ {\isasymand}\ zy\ {\isacharequal}{\kern0pt}\ {\isacharless}{\kern0pt}z{\isacharcomma}{\kern0pt}\ y{\isachargreater}{\kern0pt}\ {\isasymand}\ zy\ {\isasymin}\ A{\isacharparenright}{\kern0pt}{\isacharparenright}{\kern0pt}{\isachardoublequoteclose}\ \isanewline
\ \ \ \ \isacommand{unfolding}\isamarkupfalse%
\ is{\isacharunderscore}{\kern0pt}preds{\isacharunderscore}{\kern0pt}prel{\isacharunderscore}{\kern0pt}elem{\isacharunderscore}{\kern0pt}fm{\isacharunderscore}{\kern0pt}def\isanewline
\ \ \ \ \isacommand{apply}\isamarkupfalse%
{\isacharparenleft}{\kern0pt}rule\ iff{\isacharunderscore}{\kern0pt}trans{\isacharcomma}{\kern0pt}\ rule\ sats{\isacharunderscore}{\kern0pt}Exists{\isacharunderscore}{\kern0pt}iff{\isacharcomma}{\kern0pt}\ simp\ add{\isacharcolon}{\kern0pt}assms{\isacharcomma}{\kern0pt}\ rule\ bex{\isacharunderscore}{\kern0pt}iff{\isacharcomma}{\kern0pt}\ rule\ iff{\isacharunderscore}{\kern0pt}trans{\isacharcomma}{\kern0pt}\ rule\ sats{\isacharunderscore}{\kern0pt}And{\isacharunderscore}{\kern0pt}iff{\isacharcomma}{\kern0pt}\ simp\ add{\isacharcolon}{\kern0pt}assms{\isacharparenright}{\kern0pt}\isanewline
\ \ \ \ \isacommand{apply}\isamarkupfalse%
{\isacharparenleft}{\kern0pt}rule\ iff{\isacharunderscore}{\kern0pt}lemma{\isacharcomma}{\kern0pt}\ rule\ sats{\isacharunderscore}{\kern0pt}is{\isacharunderscore}{\kern0pt}preds{\isacharunderscore}{\kern0pt}rel{\isacharunderscore}{\kern0pt}fm{\isacharunderscore}{\kern0pt}iff{\isacharparenright}{\kern0pt}\isanewline
\ \ \ \ \isacommand{using}\isamarkupfalse%
\ assms\ \isanewline
\ \ \ \ \ \ \ \ \ \ \ \ \ \isacommand{apply}\isamarkupfalse%
\ auto{\isacharbrackleft}{\kern0pt}{\isadigit{9}}{\isacharbrackright}{\kern0pt}\isanewline
\ \ \ \ \isacommand{apply}\isamarkupfalse%
{\isacharparenleft}{\kern0pt}rule\ iff{\isacharunderscore}{\kern0pt}trans{\isacharcomma}{\kern0pt}\ rule\ sats{\isacharunderscore}{\kern0pt}Exists{\isacharunderscore}{\kern0pt}iff{\isacharcomma}{\kern0pt}\ simp\ add{\isacharcolon}{\kern0pt}assms{\isacharcomma}{\kern0pt}\ rule\ bex{\isacharunderscore}{\kern0pt}iff{\isacharparenright}{\kern0pt}{\isacharplus}{\kern0pt}\isanewline
\ \ \ \ \isacommand{using}\isamarkupfalse%
\ assms\ \isanewline
\ \ \ \ \isacommand{apply}\isamarkupfalse%
\ {\isacharparenleft}{\kern0pt}simp{\isacharcomma}{\kern0pt}\ rule{\isacharunderscore}{\kern0pt}tac\ iff{\isacharunderscore}{\kern0pt}lemma{\isacharcomma}{\kern0pt}\ simp{\isacharparenright}{\kern0pt}{\isacharplus}{\kern0pt}\isanewline
\ \ \ \ \isacommand{using}\isamarkupfalse%
\ pair{\isacharunderscore}{\kern0pt}in{\isacharunderscore}{\kern0pt}M{\isacharunderscore}{\kern0pt}iff\ preds{\isacharunderscore}{\kern0pt}rel{\isacharunderscore}{\kern0pt}in{\isacharunderscore}{\kern0pt}M\ assms\isanewline
\ \ \ \ \isacommand{by}\isamarkupfalse%
\ auto\isanewline
\ \ \isacommand{have}\isamarkupfalse%
\ I{\isadigit{2}}\ {\isacharcolon}{\kern0pt}\ {\isachardoublequoteopen}{\isachardot}{\kern0pt}{\isachardot}{\kern0pt}{\isachardot}{\kern0pt}\ {\isasymlongleftrightarrow}{\isacharparenleft}{\kern0pt}{\isasymexists}y\ {\isasymin}\ M{\isachardot}{\kern0pt}\ {\isasymexists}z\ {\isasymin}\ M{\isachardot}{\kern0pt}\ v\ {\isacharequal}{\kern0pt}\ {\isacharless}{\kern0pt}{\isacharless}{\kern0pt}z{\isacharcomma}{\kern0pt}\ a{\isachargreater}{\kern0pt}{\isacharcomma}{\kern0pt}\ {\isacharless}{\kern0pt}y{\isacharcomma}{\kern0pt}\ a{\isachargreater}{\kern0pt}{\isachargreater}{\kern0pt}\ {\isasymand}\ {\isacharless}{\kern0pt}z{\isacharcomma}{\kern0pt}\ y{\isachargreater}{\kern0pt}\ {\isasymin}\ preds{\isacharunderscore}{\kern0pt}rel{\isacharparenleft}{\kern0pt}R{\isacharcomma}{\kern0pt}\ x{\isacharparenright}{\kern0pt}{\isacharparenright}{\kern0pt}{\isachardoublequoteclose}\ \isanewline
\ \ \ \ \isacommand{using}\isamarkupfalse%
\ assms\ preds{\isacharunderscore}{\kern0pt}rel{\isacharunderscore}{\kern0pt}in{\isacharunderscore}{\kern0pt}M\ pair{\isacharunderscore}{\kern0pt}in{\isacharunderscore}{\kern0pt}M{\isacharunderscore}{\kern0pt}iff\ \isacommand{by}\isamarkupfalse%
\ auto\isanewline
\ \ \isacommand{have}\isamarkupfalse%
\ I{\isadigit{3}}\ {\isacharcolon}{\kern0pt}\ {\isachardoublequoteopen}{\isachardot}{\kern0pt}{\isachardot}{\kern0pt}{\isachardot}{\kern0pt}\ {\isasymlongleftrightarrow}\ v\ {\isasymin}\ preds{\isacharunderscore}{\kern0pt}rel{\isacharparenleft}{\kern0pt}{\isasymlambda}a\ b{\isachardot}{\kern0pt}\ {\isacharless}{\kern0pt}a{\isacharcomma}{\kern0pt}\ b{\isachargreater}{\kern0pt}\ {\isasymin}\ prel{\isacharparenleft}{\kern0pt}Rrel{\isacharparenleft}{\kern0pt}R{\isacharcomma}{\kern0pt}\ M{\isacharparenright}{\kern0pt}{\isacharcomma}{\kern0pt}\ p{\isacharparenright}{\kern0pt}{\isacharcomma}{\kern0pt}\ {\isacharless}{\kern0pt}x{\isacharcomma}{\kern0pt}\ a{\isachargreater}{\kern0pt}{\isacharparenright}{\kern0pt}{\isachardoublequoteclose}\ \isanewline
\ \ \isacommand{proof}\isamarkupfalse%
{\isacharparenleft}{\kern0pt}rule\ iffI{\isacharparenright}{\kern0pt}\isanewline
\ \ \ \ \isacommand{assume}\isamarkupfalse%
\ {\isachardoublequoteopen}{\isasymexists}y{\isasymin}M{\isachardot}{\kern0pt}\ {\isasymexists}z{\isasymin}M{\isachardot}{\kern0pt}\ v\ {\isacharequal}{\kern0pt}\ {\isasymlangle}{\isasymlangle}z{\isacharcomma}{\kern0pt}\ a{\isasymrangle}{\isacharcomma}{\kern0pt}\ y{\isacharcomma}{\kern0pt}\ a{\isasymrangle}\ {\isasymand}\ {\isasymlangle}z{\isacharcomma}{\kern0pt}\ y{\isasymrangle}\ {\isasymin}\ preds{\isacharunderscore}{\kern0pt}rel{\isacharparenleft}{\kern0pt}R{\isacharcomma}{\kern0pt}\ x{\isacharparenright}{\kern0pt}{\isachardoublequoteclose}\ \isanewline
\ \ \ \ \isacommand{then}\isamarkupfalse%
\ \isacommand{obtain}\isamarkupfalse%
\ y\ z\ \isakeyword{where}\ yzH\ {\isacharcolon}{\kern0pt}\ {\isachardoublequoteopen}y\ {\isasymin}\ M{\isachardoublequoteclose}\ {\isachardoublequoteopen}z\ {\isasymin}\ M{\isachardoublequoteclose}\ {\isachardoublequoteopen}v\ {\isacharequal}{\kern0pt}\ {\isacharless}{\kern0pt}{\isacharless}{\kern0pt}z{\isacharcomma}{\kern0pt}\ a{\isachargreater}{\kern0pt}{\isacharcomma}{\kern0pt}\ {\isacharless}{\kern0pt}y{\isacharcomma}{\kern0pt}\ a{\isachargreater}{\kern0pt}{\isachargreater}{\kern0pt}{\isachardoublequoteclose}\ {\isachardoublequoteopen}{\isacharless}{\kern0pt}z{\isacharcomma}{\kern0pt}\ y{\isachargreater}{\kern0pt}\ {\isasymin}\ preds{\isacharunderscore}{\kern0pt}rel{\isacharparenleft}{\kern0pt}R{\isacharcomma}{\kern0pt}\ x{\isacharparenright}{\kern0pt}{\isachardoublequoteclose}\ \isacommand{by}\isamarkupfalse%
\ auto\ \isanewline
\isanewline
\ \ \ \ \isacommand{have}\isamarkupfalse%
\ {\isachardoublequoteopen}z\ {\isasymin}\ preds{\isacharparenleft}{\kern0pt}R{\isacharcomma}{\kern0pt}\ x{\isacharparenright}{\kern0pt}{\isachardoublequoteclose}\ \isacommand{using}\isamarkupfalse%
\ yzH\ \isacommand{unfolding}\isamarkupfalse%
\ preds{\isacharunderscore}{\kern0pt}rel{\isacharunderscore}{\kern0pt}def\ \isacommand{by}\isamarkupfalse%
\ auto\ \isanewline
\ \ \ \ \isacommand{then}\isamarkupfalse%
\ \isacommand{have}\isamarkupfalse%
\ {\isachardoublequoteopen}{\isacharless}{\kern0pt}{\isacharless}{\kern0pt}z{\isacharcomma}{\kern0pt}\ a{\isachargreater}{\kern0pt}{\isacharcomma}{\kern0pt}\ {\isacharless}{\kern0pt}x{\isacharcomma}{\kern0pt}\ a{\isachargreater}{\kern0pt}{\isachargreater}{\kern0pt}\ {\isasymin}\ prel{\isacharparenleft}{\kern0pt}Rrel{\isacharparenleft}{\kern0pt}R{\isacharcomma}{\kern0pt}\ M{\isacharparenright}{\kern0pt}{\isacharcomma}{\kern0pt}\ p{\isacharparenright}{\kern0pt}{\isachardoublequoteclose}\ \isanewline
\ \ \ \ \ \ \isacommand{unfolding}\isamarkupfalse%
\ prel{\isacharunderscore}{\kern0pt}def\ preds{\isacharunderscore}{\kern0pt}def\ Rrel{\isacharunderscore}{\kern0pt}def\ \isanewline
\ \ \ \ \ \ \isacommand{using}\isamarkupfalse%
\ assms\isanewline
\ \ \ \ \ \ \isacommand{by}\isamarkupfalse%
\ auto\ \isanewline
\ \ \ \ \isacommand{then}\isamarkupfalse%
\ \isacommand{have}\isamarkupfalse%
\ H{\isadigit{1}}{\isacharcolon}{\kern0pt}\ {\isachardoublequoteopen}{\isasymlangle}z{\isacharcomma}{\kern0pt}\ a{\isasymrangle}\ {\isasymin}\ preds{\isacharparenleft}{\kern0pt}{\isasymlambda}a\ b{\isachardot}{\kern0pt}\ {\isasymlangle}a{\isacharcomma}{\kern0pt}\ b{\isasymrangle}\ {\isasymin}\ prel{\isacharparenleft}{\kern0pt}Rrel{\isacharparenleft}{\kern0pt}R{\isacharcomma}{\kern0pt}\ M{\isacharparenright}{\kern0pt}{\isacharcomma}{\kern0pt}\ p{\isacharparenright}{\kern0pt}{\isacharcomma}{\kern0pt}\ {\isasymlangle}x{\isacharcomma}{\kern0pt}\ a{\isasymrangle}{\isacharparenright}{\kern0pt}{\isachardoublequoteclose}\ \isanewline
\ \ \ \ \ \ \isacommand{unfolding}\isamarkupfalse%
\ preds{\isacharunderscore}{\kern0pt}def\ \isacommand{using}\isamarkupfalse%
\ assms\ yzH\ pair{\isacharunderscore}{\kern0pt}in{\isacharunderscore}{\kern0pt}M{\isacharunderscore}{\kern0pt}iff\ \isacommand{by}\isamarkupfalse%
\ simp\isanewline
\isanewline
\ \ \ \ \isacommand{have}\isamarkupfalse%
\ {\isachardoublequoteopen}y\ {\isasymin}\ preds{\isacharparenleft}{\kern0pt}R{\isacharcomma}{\kern0pt}\ x{\isacharparenright}{\kern0pt}\ {\isasymunion}\ {\isacharbraceleft}{\kern0pt}x{\isacharbraceright}{\kern0pt}{\isachardoublequoteclose}\ \isacommand{using}\isamarkupfalse%
\ yzH\ preds{\isacharunderscore}{\kern0pt}rel{\isacharunderscore}{\kern0pt}def\ \isacommand{by}\isamarkupfalse%
\ auto\ \isanewline
\ \ \ \ \isacommand{then}\isamarkupfalse%
\ \isacommand{have}\isamarkupfalse%
\ {\isachardoublequoteopen}{\isacharless}{\kern0pt}{\isacharless}{\kern0pt}y{\isacharcomma}{\kern0pt}\ a{\isachargreater}{\kern0pt}{\isacharcomma}{\kern0pt}\ {\isacharless}{\kern0pt}x{\isacharcomma}{\kern0pt}\ a{\isachargreater}{\kern0pt}{\isachargreater}{\kern0pt}\ {\isasymin}\ prel{\isacharparenleft}{\kern0pt}Rrel{\isacharparenleft}{\kern0pt}R{\isacharcomma}{\kern0pt}\ M{\isacharparenright}{\kern0pt}{\isacharcomma}{\kern0pt}\ p{\isacharparenright}{\kern0pt}\ {\isasymor}\ y\ {\isacharequal}{\kern0pt}\ x{\isachardoublequoteclose}\ \isanewline
\ \ \ \ \ \ \isacommand{unfolding}\isamarkupfalse%
\ preds{\isacharunderscore}{\kern0pt}def\ prel{\isacharunderscore}{\kern0pt}def\ Rrel{\isacharunderscore}{\kern0pt}def\isanewline
\ \ \ \ \ \ \isacommand{using}\isamarkupfalse%
\ assms\ \isanewline
\ \ \ \ \ \ \isacommand{by}\isamarkupfalse%
\ auto\isanewline
\ \ \ \ \isacommand{then}\isamarkupfalse%
\ \isacommand{have}\isamarkupfalse%
\ H{\isadigit{2}}{\isacharcolon}{\kern0pt}\ {\isachardoublequoteopen}{\isacharless}{\kern0pt}y{\isacharcomma}{\kern0pt}\ a{\isachargreater}{\kern0pt}\ {\isasymin}\ preds{\isacharparenleft}{\kern0pt}{\isasymlambda}a\ b{\isachardot}{\kern0pt}\ {\isasymlangle}a{\isacharcomma}{\kern0pt}\ b{\isasymrangle}\ {\isasymin}\ prel{\isacharparenleft}{\kern0pt}Rrel{\isacharparenleft}{\kern0pt}R{\isacharcomma}{\kern0pt}\ M{\isacharparenright}{\kern0pt}{\isacharcomma}{\kern0pt}\ p{\isacharparenright}{\kern0pt}{\isacharcomma}{\kern0pt}\ {\isasymlangle}x{\isacharcomma}{\kern0pt}\ a{\isasymrangle}{\isacharparenright}{\kern0pt}\ {\isasymor}\ y\ {\isacharequal}{\kern0pt}\ x{\isachardoublequoteclose}\ \isanewline
\ \ \ \ \ \ \isacommand{unfolding}\isamarkupfalse%
\ preds{\isacharunderscore}{\kern0pt}def\ \isacommand{using}\isamarkupfalse%
\ assms\ yzH\ pair{\isacharunderscore}{\kern0pt}in{\isacharunderscore}{\kern0pt}M{\isacharunderscore}{\kern0pt}iff\ \isacommand{by}\isamarkupfalse%
\ simp\isanewline
\isanewline
\ \ \ \ \isacommand{have}\isamarkupfalse%
\ H{\isadigit{3}}\ {\isacharcolon}{\kern0pt}\ {\isachardoublequoteopen}{\isacharless}{\kern0pt}{\isacharless}{\kern0pt}z{\isacharcomma}{\kern0pt}\ a{\isachargreater}{\kern0pt}{\isacharcomma}{\kern0pt}\ {\isacharless}{\kern0pt}y{\isacharcomma}{\kern0pt}\ a{\isachargreater}{\kern0pt}{\isachargreater}{\kern0pt}\ {\isasymin}\ prel{\isacharparenleft}{\kern0pt}Rrel{\isacharparenleft}{\kern0pt}R{\isacharcomma}{\kern0pt}\ M{\isacharparenright}{\kern0pt}{\isacharcomma}{\kern0pt}\ p{\isacharparenright}{\kern0pt}{\isachardoublequoteclose}\isanewline
\ \ \ \ \ \ \isacommand{apply}\isamarkupfalse%
{\isacharparenleft}{\kern0pt}rule\ prelI{\isacharparenright}{\kern0pt}\isanewline
\ \ \ \ \ \ \isacommand{using}\isamarkupfalse%
\ yzH\ preds{\isacharunderscore}{\kern0pt}rel{\isacharunderscore}{\kern0pt}def\ Rrel{\isacharunderscore}{\kern0pt}def\ assms\ \isanewline
\ \ \ \ \ \ \isacommand{by}\isamarkupfalse%
\ auto\isanewline
\isanewline
\ \ \ \ \isacommand{show}\isamarkupfalse%
\ {\isachardoublequoteopen}v\ {\isasymin}\ preds{\isacharunderscore}{\kern0pt}rel{\isacharparenleft}{\kern0pt}{\isasymlambda}a\ b{\isachardot}{\kern0pt}\ {\isasymlangle}a{\isacharcomma}{\kern0pt}\ b{\isasymrangle}\ {\isasymin}\ prel{\isacharparenleft}{\kern0pt}Rrel{\isacharparenleft}{\kern0pt}R{\isacharcomma}{\kern0pt}\ M{\isacharparenright}{\kern0pt}{\isacharcomma}{\kern0pt}\ p{\isacharparenright}{\kern0pt}{\isacharcomma}{\kern0pt}\ {\isasymlangle}x{\isacharcomma}{\kern0pt}\ a{\isasymrangle}{\isacharparenright}{\kern0pt}{\isachardoublequoteclose}\ \isanewline
\ \ \ \ \ \ \isacommand{unfolding}\isamarkupfalse%
\ preds{\isacharunderscore}{\kern0pt}rel{\isacharunderscore}{\kern0pt}def\ \isanewline
\ \ \ \ \ \ \isacommand{apply}\isamarkupfalse%
\ simp\isanewline
\ \ \ \ \ \ \isacommand{apply}\isamarkupfalse%
{\isacharparenleft}{\kern0pt}rule\ conjI{\isacharparenright}{\kern0pt}\isanewline
\ \ \ \ \ \ \isacommand{using}\isamarkupfalse%
\ yzH\ H{\isadigit{1}}\ H{\isadigit{2}}\ H{\isadigit{3}}\isanewline
\ \ \ \ \ \ \isacommand{by}\isamarkupfalse%
\ auto\isanewline
\ \ \isacommand{next}\isamarkupfalse%
\ \isanewline
\ \ \ \ \isacommand{assume}\isamarkupfalse%
\ {\isachardoublequoteopen}v\ {\isasymin}\ preds{\isacharunderscore}{\kern0pt}rel{\isacharparenleft}{\kern0pt}{\isasymlambda}a\ b{\isachardot}{\kern0pt}\ {\isasymlangle}a{\isacharcomma}{\kern0pt}\ b{\isasymrangle}\ {\isasymin}\ prel{\isacharparenleft}{\kern0pt}Rrel{\isacharparenleft}{\kern0pt}R{\isacharcomma}{\kern0pt}\ M{\isacharparenright}{\kern0pt}{\isacharcomma}{\kern0pt}\ p{\isacharparenright}{\kern0pt}{\isacharcomma}{\kern0pt}\ {\isasymlangle}x{\isacharcomma}{\kern0pt}\ a{\isasymrangle}{\isacharparenright}{\kern0pt}{\isachardoublequoteclose}\ \isanewline
\isanewline
\ \ \ \ \isacommand{then}\isamarkupfalse%
\ \isacommand{obtain}\isamarkupfalse%
\ za\ ya\ \isakeyword{where}\ zyH\ {\isacharcolon}{\kern0pt}\ \isanewline
\ \ \ \ \ \ {\isachardoublequoteopen}v\ {\isacharequal}{\kern0pt}\ {\isacharless}{\kern0pt}za{\isacharcomma}{\kern0pt}\ ya{\isachargreater}{\kern0pt}{\isachardoublequoteclose}\ \isanewline
\ \ \ \ \ \ {\isachardoublequoteopen}{\isasymlangle}za{\isacharcomma}{\kern0pt}\ ya{\isasymrangle}\ {\isasymin}\ prel{\isacharparenleft}{\kern0pt}Rrel{\isacharparenleft}{\kern0pt}R{\isacharcomma}{\kern0pt}\ M{\isacharparenright}{\kern0pt}{\isacharcomma}{\kern0pt}\ p{\isacharparenright}{\kern0pt}{\isachardoublequoteclose}\ \isanewline
\ \ \ \ \ \ {\isachardoublequoteopen}za\ {\isasymin}\ preds{\isacharparenleft}{\kern0pt}{\isasymlambda}a\ b{\isachardot}{\kern0pt}\ {\isasymlangle}a{\isacharcomma}{\kern0pt}\ b{\isasymrangle}\ {\isasymin}\ prel{\isacharparenleft}{\kern0pt}Rrel{\isacharparenleft}{\kern0pt}R{\isacharcomma}{\kern0pt}\ M{\isacharparenright}{\kern0pt}{\isacharcomma}{\kern0pt}\ p{\isacharparenright}{\kern0pt}{\isacharcomma}{\kern0pt}\ {\isasymlangle}x{\isacharcomma}{\kern0pt}\ a{\isasymrangle}{\isacharparenright}{\kern0pt}{\isachardoublequoteclose}\isanewline
\ \ \ \ \ \ {\isachardoublequoteopen}ya\ {\isasymin}\ preds{\isacharparenleft}{\kern0pt}{\isasymlambda}a\ b{\isachardot}{\kern0pt}\ {\isasymlangle}a{\isacharcomma}{\kern0pt}\ b{\isasymrangle}\ {\isasymin}\ prel{\isacharparenleft}{\kern0pt}Rrel{\isacharparenleft}{\kern0pt}R{\isacharcomma}{\kern0pt}\ M{\isacharparenright}{\kern0pt}{\isacharcomma}{\kern0pt}\ p{\isacharparenright}{\kern0pt}{\isacharcomma}{\kern0pt}\ {\isasymlangle}x{\isacharcomma}{\kern0pt}\ a{\isasymrangle}{\isacharparenright}{\kern0pt}\ {\isasymunion}\ {\isacharbraceleft}{\kern0pt}{\isasymlangle}x{\isacharcomma}{\kern0pt}\ a{\isasymrangle}{\isacharbraceright}{\kern0pt}{\isachardoublequoteclose}\isanewline
\ \ \ \ \ \ \isacommand{unfolding}\isamarkupfalse%
\ preds{\isacharunderscore}{\kern0pt}rel{\isacharunderscore}{\kern0pt}def\ \isanewline
\ \ \ \ \ \ \isacommand{by}\isamarkupfalse%
\ auto\isanewline
\ \ \ \ \isanewline
\ \ \ \ \isacommand{obtain}\isamarkupfalse%
\ z\ \isakeyword{where}\ zaeq\ {\isacharcolon}{\kern0pt}\ {\isachardoublequoteopen}za\ {\isacharequal}{\kern0pt}\ {\isacharless}{\kern0pt}z{\isacharcomma}{\kern0pt}\ a{\isachargreater}{\kern0pt}{\isachardoublequoteclose}\ \isacommand{using}\isamarkupfalse%
\ zyH\ \isacommand{unfolding}\isamarkupfalse%
\ preds{\isacharunderscore}{\kern0pt}def\ prel{\isacharunderscore}{\kern0pt}def\ \isacommand{by}\isamarkupfalse%
\ auto\isanewline
\ \ \ \ \isacommand{obtain}\isamarkupfalse%
\ y\ \isakeyword{where}\ yaeq\ {\isacharcolon}{\kern0pt}\ {\isachardoublequoteopen}ya\ {\isacharequal}{\kern0pt}\ {\isacharless}{\kern0pt}y{\isacharcomma}{\kern0pt}\ a{\isachargreater}{\kern0pt}{\isachardoublequoteclose}\ \isacommand{using}\isamarkupfalse%
\ zyH\ \isacommand{unfolding}\isamarkupfalse%
\ preds{\isacharunderscore}{\kern0pt}def\ prel{\isacharunderscore}{\kern0pt}def\ \isacommand{by}\isamarkupfalse%
\ auto\isanewline
\isanewline
\ \ \ \ \isacommand{have}\isamarkupfalse%
\ rel\ {\isacharcolon}{\kern0pt}\ {\isachardoublequoteopen}{\isacharless}{\kern0pt}z{\isacharcomma}{\kern0pt}\ y{\isachargreater}{\kern0pt}\ {\isasymin}\ Rrel{\isacharparenleft}{\kern0pt}R{\isacharcomma}{\kern0pt}\ M{\isacharparenright}{\kern0pt}{\isachardoublequoteclose}\ \isacommand{using}\isamarkupfalse%
\ zyH\ zaeq\ yaeq\ \isacommand{unfolding}\isamarkupfalse%
\ prel{\isacharunderscore}{\kern0pt}def\ \isacommand{by}\isamarkupfalse%
\ auto\ \isanewline
\isanewline
\ \ \ \ \isacommand{have}\isamarkupfalse%
\ zinM\ {\isacharcolon}{\kern0pt}\ {\isachardoublequoteopen}z\ {\isasymin}\ M{\isachardoublequoteclose}\ \isacommand{using}\isamarkupfalse%
\ zaeq\ zyH\ \isacommand{unfolding}\isamarkupfalse%
\ prel{\isacharunderscore}{\kern0pt}def\ Rrel{\isacharunderscore}{\kern0pt}def\ \isacommand{by}\isamarkupfalse%
\ auto\ \isanewline
\ \ \ \ \isacommand{have}\isamarkupfalse%
\ yinM\ {\isacharcolon}{\kern0pt}\ {\isachardoublequoteopen}y\ {\isasymin}\ M{\isachardoublequoteclose}\ \isacommand{using}\isamarkupfalse%
\ yaeq\ zyH\ \isacommand{unfolding}\isamarkupfalse%
\ prel{\isacharunderscore}{\kern0pt}def\ Rrel{\isacharunderscore}{\kern0pt}def\ \isacommand{by}\isamarkupfalse%
\ auto\isanewline
\isanewline
\ \ \ \ \isacommand{have}\isamarkupfalse%
\ {\isachardoublequoteopen}{\isacharless}{\kern0pt}{\isacharless}{\kern0pt}z{\isacharcomma}{\kern0pt}\ a{\isachargreater}{\kern0pt}{\isacharcomma}{\kern0pt}\ {\isacharless}{\kern0pt}x{\isacharcomma}{\kern0pt}\ a{\isachargreater}{\kern0pt}{\isachargreater}{\kern0pt}\ {\isasymin}\ prel{\isacharparenleft}{\kern0pt}Rrel{\isacharparenleft}{\kern0pt}R{\isacharcomma}{\kern0pt}\ M{\isacharparenright}{\kern0pt}{\isacharcomma}{\kern0pt}\ p{\isacharparenright}{\kern0pt}{\isachardoublequoteclose}\ \isacommand{using}\isamarkupfalse%
\ zyH\ zaeq\ \isacommand{unfolding}\isamarkupfalse%
\ preds{\isacharunderscore}{\kern0pt}def\ \isacommand{by}\isamarkupfalse%
\ auto\isanewline
\ \ \ \ \isacommand{then}\isamarkupfalse%
\ \isacommand{have}\isamarkupfalse%
\ H{\isadigit{1}}{\isacharcolon}{\kern0pt}\ {\isachardoublequoteopen}z\ {\isasymin}\ preds{\isacharparenleft}{\kern0pt}R{\isacharcomma}{\kern0pt}\ x{\isacharparenright}{\kern0pt}{\isachardoublequoteclose}\ \isacommand{unfolding}\isamarkupfalse%
\ preds{\isacharunderscore}{\kern0pt}def\ prel{\isacharunderscore}{\kern0pt}def\ Rrel{\isacharunderscore}{\kern0pt}def\ \isacommand{by}\isamarkupfalse%
\ auto\ \isanewline
\isanewline
\ \ \ \ \isacommand{have}\isamarkupfalse%
\ {\isachardoublequoteopen}{\isacharless}{\kern0pt}{\isacharless}{\kern0pt}y{\isacharcomma}{\kern0pt}\ a{\isachargreater}{\kern0pt}{\isacharcomma}{\kern0pt}\ {\isacharless}{\kern0pt}x{\isacharcomma}{\kern0pt}\ a{\isachargreater}{\kern0pt}{\isachargreater}{\kern0pt}\ {\isasymin}\ prel{\isacharparenleft}{\kern0pt}Rrel{\isacharparenleft}{\kern0pt}R{\isacharcomma}{\kern0pt}\ M{\isacharparenright}{\kern0pt}{\isacharcomma}{\kern0pt}\ p{\isacharparenright}{\kern0pt}\ {\isasymor}\ y\ {\isacharequal}{\kern0pt}\ x{\isachardoublequoteclose}\ \isacommand{using}\isamarkupfalse%
\ zyH\ yaeq\ \isacommand{unfolding}\isamarkupfalse%
\ preds{\isacharunderscore}{\kern0pt}def\ \isacommand{by}\isamarkupfalse%
\ auto\isanewline
\ \ \ \ \isacommand{then}\isamarkupfalse%
\ \isacommand{have}\isamarkupfalse%
\ H{\isadigit{2}}{\isacharcolon}{\kern0pt}\ {\isachardoublequoteopen}y\ {\isasymin}\ preds{\isacharparenleft}{\kern0pt}R{\isacharcomma}{\kern0pt}\ x{\isacharparenright}{\kern0pt}\ {\isasymunion}\ {\isacharbraceleft}{\kern0pt}x{\isacharbraceright}{\kern0pt}{\isachardoublequoteclose}\ \isacommand{unfolding}\isamarkupfalse%
\ preds{\isacharunderscore}{\kern0pt}def\ prel{\isacharunderscore}{\kern0pt}def\ Rrel{\isacharunderscore}{\kern0pt}def\ \isacommand{by}\isamarkupfalse%
\ auto\ \isanewline
\isanewline
\ \ \ \ \isacommand{have}\isamarkupfalse%
\ rel{\isacharprime}{\kern0pt}\ {\isacharcolon}{\kern0pt}\ {\isachardoublequoteopen}{\isacharless}{\kern0pt}z{\isacharcomma}{\kern0pt}\ y{\isachargreater}{\kern0pt}\ {\isasymin}\ preds{\isacharunderscore}{\kern0pt}rel{\isacharparenleft}{\kern0pt}R{\isacharcomma}{\kern0pt}\ x{\isacharparenright}{\kern0pt}{\isachardoublequoteclose}\ \isanewline
\ \ \ \ \ \ \isacommand{unfolding}\isamarkupfalse%
\ preds{\isacharunderscore}{\kern0pt}rel{\isacharunderscore}{\kern0pt}def\ \isanewline
\ \ \ \ \ \ \isacommand{using}\isamarkupfalse%
\ H{\isadigit{1}}\ H{\isadigit{2}}\ rel\ Rrel{\isacharunderscore}{\kern0pt}def\ \isanewline
\ \ \ \ \ \ \isacommand{by}\isamarkupfalse%
\ auto\isanewline
\isanewline
\ \ \ \ \isacommand{show}\isamarkupfalse%
\ {\isachardoublequoteopen}{\isasymexists}y{\isasymin}M{\isachardot}{\kern0pt}\ {\isasymexists}z{\isasymin}M{\isachardot}{\kern0pt}\ v\ {\isacharequal}{\kern0pt}\ {\isasymlangle}{\isasymlangle}z{\isacharcomma}{\kern0pt}\ a{\isasymrangle}{\isacharcomma}{\kern0pt}\ y{\isacharcomma}{\kern0pt}\ a{\isasymrangle}\ {\isasymand}\ {\isasymlangle}z{\isacharcomma}{\kern0pt}\ y{\isasymrangle}\ {\isasymin}\ preds{\isacharunderscore}{\kern0pt}rel{\isacharparenleft}{\kern0pt}R{\isacharcomma}{\kern0pt}\ x{\isacharparenright}{\kern0pt}{\isachardoublequoteclose}\ \isanewline
\ \ \ \ \ \ \isacommand{apply}\isamarkupfalse%
{\isacharparenleft}{\kern0pt}rule{\isacharunderscore}{\kern0pt}tac\ x{\isacharequal}{\kern0pt}y\ \isakeyword{in}\ bexI{\isacharcomma}{\kern0pt}\ rule{\isacharunderscore}{\kern0pt}tac\ x{\isacharequal}{\kern0pt}z\ \isakeyword{in}\ bexI{\isacharparenright}{\kern0pt}\isanewline
\ \ \ \ \ \ \isacommand{using}\isamarkupfalse%
\ zyH\ zaeq\ yaeq\ rel{\isacharprime}{\kern0pt}\ yinM\ zinM\ \isanewline
\ \ \ \ \ \ \isacommand{by}\isamarkupfalse%
\ auto\isanewline
\ \ \isacommand{qed}\isamarkupfalse%
\isanewline
\isanewline
\ \ \isacommand{show}\isamarkupfalse%
\ {\isacharquery}{\kern0pt}thesis\ \isacommand{using}\isamarkupfalse%
\ I{\isadigit{1}}\ I{\isadigit{2}}\ I{\isadigit{3}}\ \isacommand{by}\isamarkupfalse%
\ auto\isanewline
\isacommand{qed}\isamarkupfalse%
%
\endisatagproof
{\isafoldproof}%
%
\isadelimproof
\isanewline
%
\endisadelimproof
\isanewline
\isacommand{lemma}\isamarkupfalse%
\ preds{\isacharunderscore}{\kern0pt}prel{\isacharunderscore}{\kern0pt}in{\isacharunderscore}{\kern0pt}M\ {\isacharcolon}{\kern0pt}\ \isanewline
\ \ \isakeyword{fixes}\ R\ Rfm\ x\ a\ p\isanewline
\ \ \isakeyword{assumes}\ {\isachardoublequoteopen}Relation{\isacharunderscore}{\kern0pt}fm{\isacharparenleft}{\kern0pt}R{\isacharcomma}{\kern0pt}\ Rfm{\isacharparenright}{\kern0pt}{\isachardoublequoteclose}\ {\isachardoublequoteopen}preds{\isacharparenleft}{\kern0pt}R{\isacharcomma}{\kern0pt}\ x{\isacharparenright}{\kern0pt}\ {\isasymin}\ M{\isachardoublequoteclose}\ \ \isanewline
\ \ \ \ \ \ \ \ \ \ {\isachardoublequoteopen}x\ {\isasymin}\ M{\isachardoublequoteclose}\ {\isachardoublequoteopen}a\ {\isasymin}\ M{\isachardoublequoteclose}\ {\isachardoublequoteopen}p\ {\isasymin}\ M{\isachardoublequoteclose}\ {\isachardoublequoteopen}a\ {\isasymin}\ p{\isachardoublequoteclose}\ \isanewline
\ \ \isakeyword{shows}\ {\isachardoublequoteopen}preds{\isacharunderscore}{\kern0pt}rel{\isacharparenleft}{\kern0pt}{\isasymlambda}a\ b{\isachardot}{\kern0pt}\ {\isacharless}{\kern0pt}a{\isacharcomma}{\kern0pt}\ b{\isachargreater}{\kern0pt}\ {\isasymin}\ prel{\isacharparenleft}{\kern0pt}Rrel{\isacharparenleft}{\kern0pt}R{\isacharcomma}{\kern0pt}\ M{\isacharparenright}{\kern0pt}{\isacharcomma}{\kern0pt}\ p{\isacharparenright}{\kern0pt}{\isacharcomma}{\kern0pt}\ {\isacharless}{\kern0pt}x{\isacharcomma}{\kern0pt}\ a{\isachargreater}{\kern0pt}{\isacharparenright}{\kern0pt}\ {\isasymin}\ M{\isachardoublequoteclose}\isanewline
%
\isadelimproof
%
\endisadelimproof
%
\isatagproof
\isacommand{proof}\isamarkupfalse%
\ {\isacharminus}{\kern0pt}\ \isanewline
\ \ \isacommand{define}\isamarkupfalse%
\ A\ \isakeyword{where}\ {\isachardoublequoteopen}A\ {\isasymequiv}\ {\isacharbraceleft}{\kern0pt}\ v\ {\isasymin}\ {\isacharparenleft}{\kern0pt}preds{\isacharparenleft}{\kern0pt}R{\isacharcomma}{\kern0pt}\ x{\isacharparenright}{\kern0pt}\ {\isasymtimes}\ {\isacharbraceleft}{\kern0pt}a{\isacharbraceright}{\kern0pt}{\isacharparenright}{\kern0pt}\ {\isasymtimes}\ {\isacharparenleft}{\kern0pt}{\isacharparenleft}{\kern0pt}preds{\isacharparenleft}{\kern0pt}R{\isacharcomma}{\kern0pt}\ x{\isacharparenright}{\kern0pt}\ {\isasymtimes}\ {\isacharbraceleft}{\kern0pt}a{\isacharbraceright}{\kern0pt}{\isacharparenright}{\kern0pt}\ {\isasymunion}\ {\isacharbraceleft}{\kern0pt}{\isacharless}{\kern0pt}x{\isacharcomma}{\kern0pt}\ a{\isachargreater}{\kern0pt}{\isacharbraceright}{\kern0pt}{\isacharparenright}{\kern0pt}{\isachardot}{\kern0pt}\ sats{\isacharparenleft}{\kern0pt}M{\isacharcomma}{\kern0pt}\ is{\isacharunderscore}{\kern0pt}preds{\isacharunderscore}{\kern0pt}prel{\isacharunderscore}{\kern0pt}elem{\isacharunderscore}{\kern0pt}fm{\isacharparenleft}{\kern0pt}Rfm{\isacharcomma}{\kern0pt}\ {\isadigit{1}}{\isacharcomma}{\kern0pt}\ {\isadigit{2}}{\isacharcomma}{\kern0pt}\ {\isadigit{0}}{\isacharparenright}{\kern0pt}{\isacharcomma}{\kern0pt}\ {\isacharbrackleft}{\kern0pt}v{\isacharbrackright}{\kern0pt}\ {\isacharat}{\kern0pt}\ {\isacharbrackleft}{\kern0pt}x{\isacharcomma}{\kern0pt}\ a{\isacharbrackright}{\kern0pt}{\isacharparenright}{\kern0pt}\ {\isacharbraceright}{\kern0pt}{\isachardoublequoteclose}\isanewline
\isanewline
\ \ \isacommand{have}\isamarkupfalse%
\ {\isachardoublequoteopen}separation{\isacharparenleft}{\kern0pt}{\isacharhash}{\kern0pt}{\isacharhash}{\kern0pt}M{\isacharcomma}{\kern0pt}\ {\isasymlambda}v{\isachardot}{\kern0pt}\ sats{\isacharparenleft}{\kern0pt}M{\isacharcomma}{\kern0pt}\ is{\isacharunderscore}{\kern0pt}preds{\isacharunderscore}{\kern0pt}prel{\isacharunderscore}{\kern0pt}elem{\isacharunderscore}{\kern0pt}fm{\isacharparenleft}{\kern0pt}Rfm{\isacharcomma}{\kern0pt}\ {\isadigit{1}}{\isacharcomma}{\kern0pt}\ {\isadigit{2}}{\isacharcomma}{\kern0pt}\ {\isadigit{0}}{\isacharparenright}{\kern0pt}{\isacharcomma}{\kern0pt}\ {\isacharbrackleft}{\kern0pt}v{\isacharbrackright}{\kern0pt}\ {\isacharat}{\kern0pt}\ {\isacharbrackleft}{\kern0pt}x{\isacharcomma}{\kern0pt}\ a{\isacharbrackright}{\kern0pt}{\isacharparenright}{\kern0pt}{\isacharparenright}{\kern0pt}{\isachardoublequoteclose}\ \isanewline
\ \ \ \ \isacommand{apply}\isamarkupfalse%
{\isacharparenleft}{\kern0pt}rule\ separation{\isacharunderscore}{\kern0pt}ax{\isacharparenright}{\kern0pt}\isanewline
\ \ \ \ \ \ \isacommand{apply}\isamarkupfalse%
{\isacharparenleft}{\kern0pt}rule\ is{\isacharunderscore}{\kern0pt}preds{\isacharunderscore}{\kern0pt}prel{\isacharunderscore}{\kern0pt}elem{\isacharunderscore}{\kern0pt}fm{\isacharunderscore}{\kern0pt}type{\isacharparenright}{\kern0pt}\isanewline
\ \ \ \ \isacommand{using}\isamarkupfalse%
\ assms\ \isanewline
\ \ \ \ \isacommand{unfolding}\isamarkupfalse%
\ Relation{\isacharunderscore}{\kern0pt}fm{\isacharunderscore}{\kern0pt}def\isanewline
\ \ \ \ \ \ \ \ \ \isacommand{apply}\isamarkupfalse%
\ simp{\isacharunderscore}{\kern0pt}all\isanewline
\ \ \ \ \isacommand{apply}\isamarkupfalse%
{\isacharparenleft}{\kern0pt}rule\ le{\isacharunderscore}{\kern0pt}trans{\isacharcomma}{\kern0pt}\ rule\ is{\isacharunderscore}{\kern0pt}preds{\isacharunderscore}{\kern0pt}prel{\isacharunderscore}{\kern0pt}elem{\isacharunderscore}{\kern0pt}fm{\isacharunderscore}{\kern0pt}arity{\isacharparenright}{\kern0pt}\isanewline
\ \ \ \ \isacommand{using}\isamarkupfalse%
\ assms\isanewline
\ \ \ \ \ \ \ \ \ \isacommand{apply}\isamarkupfalse%
\ simp{\isacharunderscore}{\kern0pt}all\isanewline
\ \ \ \ \isacommand{apply}\isamarkupfalse%
{\isacharparenleft}{\kern0pt}rule\ Un{\isacharunderscore}{\kern0pt}least{\isacharunderscore}{\kern0pt}lt{\isacharparenright}{\kern0pt}{\isacharplus}{\kern0pt}\isanewline
\ \ \ \ \isacommand{by}\isamarkupfalse%
\ auto\isanewline
\isanewline
\ \ \isacommand{then}\isamarkupfalse%
\ \isacommand{have}\isamarkupfalse%
\ AinM\ {\isacharcolon}{\kern0pt}\ \ {\isachardoublequoteopen}A\ {\isasymin}\ M{\isachardoublequoteclose}\ \isanewline
\ \ \ \ \isacommand{unfolding}\isamarkupfalse%
\ A{\isacharunderscore}{\kern0pt}def\ \isanewline
\ \ \ \ \isacommand{apply}\isamarkupfalse%
{\isacharparenleft}{\kern0pt}rule\ separation{\isacharunderscore}{\kern0pt}notation{\isacharparenright}{\kern0pt}\isanewline
\ \ \ \ \isacommand{using}\isamarkupfalse%
\ singleton{\isacharunderscore}{\kern0pt}in{\isacharunderscore}{\kern0pt}M{\isacharunderscore}{\kern0pt}iff\ assms\ cartprod{\isacharunderscore}{\kern0pt}closed\ pair{\isacharunderscore}{\kern0pt}in{\isacharunderscore}{\kern0pt}M{\isacharunderscore}{\kern0pt}iff\ Un{\isacharunderscore}{\kern0pt}closed\isanewline
\ \ \ \ \isacommand{by}\isamarkupfalse%
\ auto\ \isanewline
\isanewline
\ \ \isacommand{have}\isamarkupfalse%
\ {\isachardoublequoteopen}A\ {\isacharequal}{\kern0pt}\ {\isacharbraceleft}{\kern0pt}\ v\ {\isasymin}\ {\isacharparenleft}{\kern0pt}preds{\isacharparenleft}{\kern0pt}R{\isacharcomma}{\kern0pt}\ x{\isacharparenright}{\kern0pt}\ {\isasymtimes}\ {\isacharbraceleft}{\kern0pt}a{\isacharbraceright}{\kern0pt}{\isacharparenright}{\kern0pt}\ {\isasymtimes}\ {\isacharparenleft}{\kern0pt}{\isacharparenleft}{\kern0pt}preds{\isacharparenleft}{\kern0pt}R{\isacharcomma}{\kern0pt}\ x{\isacharparenright}{\kern0pt}\ {\isasymtimes}\ {\isacharbraceleft}{\kern0pt}a{\isacharbraceright}{\kern0pt}{\isacharparenright}{\kern0pt}\ {\isasymunion}\ {\isacharbraceleft}{\kern0pt}{\isacharless}{\kern0pt}x{\isacharcomma}{\kern0pt}\ a{\isachargreater}{\kern0pt}{\isacharbraceright}{\kern0pt}{\isacharparenright}{\kern0pt}{\isachardot}{\kern0pt}\ v\ {\isasymin}\ preds{\isacharunderscore}{\kern0pt}rel{\isacharparenleft}{\kern0pt}{\isasymlambda}a\ b{\isachardot}{\kern0pt}\ {\isacharless}{\kern0pt}a{\isacharcomma}{\kern0pt}\ b{\isachargreater}{\kern0pt}\ {\isasymin}\ prel{\isacharparenleft}{\kern0pt}Rrel{\isacharparenleft}{\kern0pt}R{\isacharcomma}{\kern0pt}\ M{\isacharparenright}{\kern0pt}{\isacharcomma}{\kern0pt}\ p{\isacharparenright}{\kern0pt}{\isacharcomma}{\kern0pt}\ {\isacharless}{\kern0pt}x{\isacharcomma}{\kern0pt}\ a{\isachargreater}{\kern0pt}{\isacharparenright}{\kern0pt}\ {\isacharbraceright}{\kern0pt}{\isachardoublequoteclose}\ \isanewline
\ \ \ \ \isacommand{unfolding}\isamarkupfalse%
\ A{\isacharunderscore}{\kern0pt}def\isanewline
\ \ \ \ \isacommand{apply}\isamarkupfalse%
{\isacharparenleft}{\kern0pt}rule\ iff{\isacharunderscore}{\kern0pt}eq{\isacharcomma}{\kern0pt}\ rule\ sats{\isacharunderscore}{\kern0pt}is{\isacharunderscore}{\kern0pt}preds{\isacharunderscore}{\kern0pt}prel{\isacharunderscore}{\kern0pt}elem{\isacharunderscore}{\kern0pt}fm{\isacharunderscore}{\kern0pt}iff{\isacharparenright}{\kern0pt}\isanewline
\ \ \ \ \isacommand{using}\isamarkupfalse%
\ assms\ pair{\isacharunderscore}{\kern0pt}in{\isacharunderscore}{\kern0pt}M{\isacharunderscore}{\kern0pt}iff\ transM\ \isanewline
\ \ \ \ \isacommand{by}\isamarkupfalse%
\ auto\isanewline
\isanewline
\ \ \isacommand{also}\isamarkupfalse%
\ \isacommand{have}\isamarkupfalse%
\ {\isachardoublequoteopen}{\isachardot}{\kern0pt}{\isachardot}{\kern0pt}{\isachardot}{\kern0pt}\ {\isacharequal}{\kern0pt}\ preds{\isacharunderscore}{\kern0pt}rel{\isacharparenleft}{\kern0pt}{\isasymlambda}a\ b{\isachardot}{\kern0pt}\ {\isacharless}{\kern0pt}a{\isacharcomma}{\kern0pt}\ b{\isachargreater}{\kern0pt}\ {\isasymin}\ prel{\isacharparenleft}{\kern0pt}Rrel{\isacharparenleft}{\kern0pt}R{\isacharcomma}{\kern0pt}\ M{\isacharparenright}{\kern0pt}{\isacharcomma}{\kern0pt}\ p{\isacharparenright}{\kern0pt}{\isacharcomma}{\kern0pt}\ {\isacharless}{\kern0pt}x{\isacharcomma}{\kern0pt}\ a{\isachargreater}{\kern0pt}{\isacharparenright}{\kern0pt}{\isachardoublequoteclose}\ \isanewline
\ \ \isacommand{proof}\isamarkupfalse%
\ {\isacharparenleft}{\kern0pt}rule\ equality{\isacharunderscore}{\kern0pt}iffI{\isacharcomma}{\kern0pt}\ rule\ iffI{\isacharcomma}{\kern0pt}\ simp{\isacharparenright}{\kern0pt}\isanewline
\ \ \ \ \isacommand{have}\isamarkupfalse%
\ H\ {\isacharcolon}{\kern0pt}\ {\isachardoublequoteopen}{\isasymAnd}X\ Y\ P\ x{\isachardot}{\kern0pt}\ x\ {\isasymin}\ {\isacharbraceleft}{\kern0pt}\ {\isacharless}{\kern0pt}a{\isacharcomma}{\kern0pt}\ b{\isachargreater}{\kern0pt}\ {\isasymin}\ X\ {\isasymtimes}\ Y{\isachardot}{\kern0pt}\ P{\isacharparenleft}{\kern0pt}a{\isacharcomma}{\kern0pt}\ b{\isacharparenright}{\kern0pt}\ {\isacharbraceright}{\kern0pt}\ {\isasymLongrightarrow}\ {\isasymexists}a\ b{\isachardot}{\kern0pt}\ a\ {\isasymin}\ X\ {\isasymand}\ b\ {\isasymin}\ Y\ {\isasymand}\ P{\isacharparenleft}{\kern0pt}a{\isacharcomma}{\kern0pt}\ b{\isacharparenright}{\kern0pt}\ {\isasymand}\ x\ {\isacharequal}{\kern0pt}\ {\isacharless}{\kern0pt}a{\isacharcomma}{\kern0pt}\ b{\isachargreater}{\kern0pt}{\isachardoublequoteclose}\ \isanewline
\ \ \ \ \ \ \isacommand{by}\isamarkupfalse%
\ auto\isanewline
\isanewline
\ \ \ \ \isacommand{fix}\isamarkupfalse%
\ s\ \isacommand{assume}\isamarkupfalse%
\ sin\ {\isacharcolon}{\kern0pt}\ {\isachardoublequoteopen}s\ {\isasymin}\ preds{\isacharunderscore}{\kern0pt}rel{\isacharparenleft}{\kern0pt}{\isasymlambda}a\ b{\isachardot}{\kern0pt}\ {\isasymlangle}a{\isacharcomma}{\kern0pt}\ b{\isasymrangle}\ {\isasymin}\ prel{\isacharparenleft}{\kern0pt}Rrel{\isacharparenleft}{\kern0pt}R{\isacharcomma}{\kern0pt}\ M{\isacharparenright}{\kern0pt}{\isacharcomma}{\kern0pt}\ p{\isacharparenright}{\kern0pt}{\isacharcomma}{\kern0pt}\ {\isasymlangle}x{\isacharcomma}{\kern0pt}\ a{\isasymrangle}{\isacharparenright}{\kern0pt}{\isachardoublequoteclose}\isanewline
\ \ \ \ \isacommand{then}\isamarkupfalse%
\ \isacommand{have}\isamarkupfalse%
\ {\isachardoublequoteopen}{\isasymexists}za\ ya{\isachardot}{\kern0pt}\ \isanewline
\ \ \ \ \ \ za\ {\isasymin}\ preds{\isacharparenleft}{\kern0pt}{\isasymlambda}a\ b{\isachardot}{\kern0pt}\ {\isasymlangle}a{\isacharcomma}{\kern0pt}\ b{\isasymrangle}\ {\isasymin}\ prel{\isacharparenleft}{\kern0pt}Rrel{\isacharparenleft}{\kern0pt}R{\isacharcomma}{\kern0pt}\ M{\isacharparenright}{\kern0pt}{\isacharcomma}{\kern0pt}\ p{\isacharparenright}{\kern0pt}{\isacharcomma}{\kern0pt}\ {\isasymlangle}x{\isacharcomma}{\kern0pt}\ a{\isasymrangle}{\isacharparenright}{\kern0pt}\isanewline
\ \ \ \ \ \ {\isasymand}\ ya\ {\isasymin}\ preds{\isacharparenleft}{\kern0pt}{\isasymlambda}a\ b{\isachardot}{\kern0pt}\ {\isasymlangle}a{\isacharcomma}{\kern0pt}\ b{\isasymrangle}\ {\isasymin}\ prel{\isacharparenleft}{\kern0pt}Rrel{\isacharparenleft}{\kern0pt}R{\isacharcomma}{\kern0pt}\ M{\isacharparenright}{\kern0pt}{\isacharcomma}{\kern0pt}\ p{\isacharparenright}{\kern0pt}{\isacharcomma}{\kern0pt}\ {\isasymlangle}x{\isacharcomma}{\kern0pt}\ a{\isasymrangle}{\isacharparenright}{\kern0pt}\ {\isasymunion}\ {\isacharbraceleft}{\kern0pt}\ {\isacharless}{\kern0pt}x{\isacharcomma}{\kern0pt}\ a{\isachargreater}{\kern0pt}\ {\isacharbraceright}{\kern0pt}\isanewline
\ \ \ \ \ \ {\isasymand}\ {\isasymlangle}za{\isacharcomma}{\kern0pt}\ ya{\isasymrangle}\ {\isasymin}\ prel{\isacharparenleft}{\kern0pt}Rrel{\isacharparenleft}{\kern0pt}R{\isacharcomma}{\kern0pt}\ M{\isacharparenright}{\kern0pt}{\isacharcomma}{\kern0pt}\ p{\isacharparenright}{\kern0pt}\isanewline
\ \ \ \ \ \ {\isasymand}\ s\ {\isacharequal}{\kern0pt}\ {\isacharless}{\kern0pt}za{\isacharcomma}{\kern0pt}\ ya{\isachargreater}{\kern0pt}{\isachardoublequoteclose}\ \isanewline
\ \ \ \ \ \ \isacommand{unfolding}\isamarkupfalse%
\ preds{\isacharunderscore}{\kern0pt}rel{\isacharunderscore}{\kern0pt}def\ \isanewline
\ \ \ \ \ \ \isacommand{by}\isamarkupfalse%
{\isacharparenleft}{\kern0pt}rule\ H{\isacharparenright}{\kern0pt}\isanewline
\ \ \ \ \isacommand{then}\isamarkupfalse%
\ \isacommand{obtain}\isamarkupfalse%
\ za\ ya\ \isakeyword{where}\ zayaH{\isacharcolon}{\kern0pt}\isanewline
\ \ \ \ \ \ {\isachardoublequoteopen}za\ {\isasymin}\ preds{\isacharparenleft}{\kern0pt}{\isasymlambda}a\ b{\isachardot}{\kern0pt}\ {\isasymlangle}a{\isacharcomma}{\kern0pt}\ b{\isasymrangle}\ {\isasymin}\ prel{\isacharparenleft}{\kern0pt}Rrel{\isacharparenleft}{\kern0pt}R{\isacharcomma}{\kern0pt}\ M{\isacharparenright}{\kern0pt}{\isacharcomma}{\kern0pt}\ p{\isacharparenright}{\kern0pt}{\isacharcomma}{\kern0pt}\ {\isasymlangle}x{\isacharcomma}{\kern0pt}\ a{\isasymrangle}{\isacharparenright}{\kern0pt}{\isachardoublequoteclose}\ \isanewline
\ \ \ \ \ \ {\isachardoublequoteopen}ya\ {\isasymin}\ preds{\isacharparenleft}{\kern0pt}{\isasymlambda}a\ b{\isachardot}{\kern0pt}\ {\isasymlangle}a{\isacharcomma}{\kern0pt}\ b{\isasymrangle}\ {\isasymin}\ prel{\isacharparenleft}{\kern0pt}Rrel{\isacharparenleft}{\kern0pt}R{\isacharcomma}{\kern0pt}\ M{\isacharparenright}{\kern0pt}{\isacharcomma}{\kern0pt}\ p{\isacharparenright}{\kern0pt}{\isacharcomma}{\kern0pt}\ {\isasymlangle}x{\isacharcomma}{\kern0pt}\ a{\isasymrangle}{\isacharparenright}{\kern0pt}\ {\isasymunion}\ {\isacharbraceleft}{\kern0pt}\ {\isacharless}{\kern0pt}x{\isacharcomma}{\kern0pt}\ a{\isachargreater}{\kern0pt}\ {\isacharbraceright}{\kern0pt}{\isachardoublequoteclose}\ \isanewline
\ \ \ \ \ \ {\isachardoublequoteopen}{\isasymlangle}za{\isacharcomma}{\kern0pt}\ ya{\isasymrangle}\ {\isasymin}\ prel{\isacharparenleft}{\kern0pt}Rrel{\isacharparenleft}{\kern0pt}R{\isacharcomma}{\kern0pt}\ M{\isacharparenright}{\kern0pt}{\isacharcomma}{\kern0pt}\ p{\isacharparenright}{\kern0pt}{\isachardoublequoteclose}\ {\isachardoublequoteopen}s\ {\isacharequal}{\kern0pt}\ {\isacharless}{\kern0pt}za{\isacharcomma}{\kern0pt}\ ya{\isachargreater}{\kern0pt}{\isachardoublequoteclose}\ \isacommand{by}\isamarkupfalse%
\ auto\isanewline
\ \ \ \ \isacommand{obtain}\isamarkupfalse%
\ z\ \isakeyword{where}\ zaeq\ {\isacharcolon}{\kern0pt}\ {\isachardoublequoteopen}za\ {\isacharequal}{\kern0pt}\ {\isacharless}{\kern0pt}z{\isacharcomma}{\kern0pt}\ a{\isachargreater}{\kern0pt}{\isachardoublequoteclose}\ \isacommand{using}\isamarkupfalse%
\ zayaH\ \isacommand{unfolding}\isamarkupfalse%
\ preds{\isacharunderscore}{\kern0pt}def\ prel{\isacharunderscore}{\kern0pt}def\ \isacommand{by}\isamarkupfalse%
\ auto\ \isanewline
\ \ \ \ \isacommand{obtain}\isamarkupfalse%
\ y\ \isakeyword{where}\ yaeq\ {\isacharcolon}{\kern0pt}\ {\isachardoublequoteopen}ya\ {\isacharequal}{\kern0pt}\ {\isacharless}{\kern0pt}y{\isacharcomma}{\kern0pt}\ a{\isachargreater}{\kern0pt}{\isachardoublequoteclose}\ \isacommand{using}\isamarkupfalse%
\ zayaH\ \isacommand{unfolding}\isamarkupfalse%
\ preds{\isacharunderscore}{\kern0pt}def\ prel{\isacharunderscore}{\kern0pt}def\ \isacommand{by}\isamarkupfalse%
\ auto\ \isanewline
\isanewline
\ \ \ \ \isacommand{have}\isamarkupfalse%
\ {\isachardoublequoteopen}{\isacharless}{\kern0pt}{\isacharless}{\kern0pt}z{\isacharcomma}{\kern0pt}\ a{\isachargreater}{\kern0pt}{\isacharcomma}{\kern0pt}\ {\isacharless}{\kern0pt}x{\isacharcomma}{\kern0pt}\ a{\isachargreater}{\kern0pt}{\isachargreater}{\kern0pt}\ {\isasymin}\ prel{\isacharparenleft}{\kern0pt}Rrel{\isacharparenleft}{\kern0pt}R{\isacharcomma}{\kern0pt}\ M{\isacharparenright}{\kern0pt}{\isacharcomma}{\kern0pt}\ p{\isacharparenright}{\kern0pt}{\isachardoublequoteclose}\ \isanewline
\ \ \ \ \ \ \isacommand{using}\isamarkupfalse%
\ zayaH\ zaeq\ \isacommand{unfolding}\isamarkupfalse%
\ preds{\isacharunderscore}{\kern0pt}def\ \isacommand{by}\isamarkupfalse%
\ auto\ \isanewline
\ \ \ \ \isacommand{then}\isamarkupfalse%
\ \isacommand{have}\isamarkupfalse%
\ {\isachardoublequoteopen}{\isacharless}{\kern0pt}z{\isacharcomma}{\kern0pt}\ x{\isachargreater}{\kern0pt}\ {\isasymin}\ Rrel{\isacharparenleft}{\kern0pt}R{\isacharcomma}{\kern0pt}\ M{\isacharparenright}{\kern0pt}{\isachardoublequoteclose}\ \isanewline
\ \ \ \ \ \ \isacommand{unfolding}\isamarkupfalse%
\ prel{\isacharunderscore}{\kern0pt}def\ \isacommand{by}\isamarkupfalse%
\ auto\isanewline
\ \ \ \ \isacommand{then}\isamarkupfalse%
\ \isacommand{have}\isamarkupfalse%
\ H{\isadigit{1}}\ {\isacharcolon}{\kern0pt}\ {\isachardoublequoteopen}z\ {\isasymin}\ preds{\isacharparenleft}{\kern0pt}R{\isacharcomma}{\kern0pt}\ x{\isacharparenright}{\kern0pt}{\isachardoublequoteclose}\ \isanewline
\ \ \ \ \ \ \isacommand{unfolding}\isamarkupfalse%
\ preds{\isacharunderscore}{\kern0pt}def\ Rrel{\isacharunderscore}{\kern0pt}def\ \isacommand{by}\isamarkupfalse%
\ auto\isanewline
\isanewline
\ \ \ \ \isacommand{have}\isamarkupfalse%
\ {\isachardoublequoteopen}{\isacharless}{\kern0pt}{\isacharless}{\kern0pt}y{\isacharcomma}{\kern0pt}\ a{\isachargreater}{\kern0pt}{\isacharcomma}{\kern0pt}\ {\isacharless}{\kern0pt}x{\isacharcomma}{\kern0pt}\ a{\isachargreater}{\kern0pt}{\isachargreater}{\kern0pt}\ {\isasymin}\ prel{\isacharparenleft}{\kern0pt}Rrel{\isacharparenleft}{\kern0pt}R{\isacharcomma}{\kern0pt}\ M{\isacharparenright}{\kern0pt}{\isacharcomma}{\kern0pt}\ p{\isacharparenright}{\kern0pt}\ {\isasymor}\ y\ {\isacharequal}{\kern0pt}\ x{\isachardoublequoteclose}\ \isanewline
\ \ \ \ \ \ \isacommand{using}\isamarkupfalse%
\ zayaH\ yaeq\ \isacommand{unfolding}\isamarkupfalse%
\ preds{\isacharunderscore}{\kern0pt}def\ \isacommand{by}\isamarkupfalse%
\ auto\isanewline
\ \ \ \ \isacommand{then}\isamarkupfalse%
\ \isacommand{have}\isamarkupfalse%
\ {\isachardoublequoteopen}{\isacharless}{\kern0pt}y{\isacharcomma}{\kern0pt}\ x{\isachargreater}{\kern0pt}\ {\isasymin}\ Rrel{\isacharparenleft}{\kern0pt}R{\isacharcomma}{\kern0pt}\ M{\isacharparenright}{\kern0pt}\ {\isasymor}\ y\ {\isacharequal}{\kern0pt}\ x{\isachardoublequoteclose}\ \isanewline
\ \ \ \ \ \ \isacommand{unfolding}\isamarkupfalse%
\ prel{\isacharunderscore}{\kern0pt}def\ \isacommand{by}\isamarkupfalse%
\ auto\ \isanewline
\ \ \ \ \isacommand{then}\isamarkupfalse%
\ \isacommand{have}\isamarkupfalse%
\ H{\isadigit{2}}{\isacharcolon}{\kern0pt}\ {\isachardoublequoteopen}y\ {\isasymin}\ preds{\isacharparenleft}{\kern0pt}R{\isacharcomma}{\kern0pt}\ x{\isacharparenright}{\kern0pt}\ {\isasymunion}\ {\isacharbraceleft}{\kern0pt}x{\isacharbraceright}{\kern0pt}{\isachardoublequoteclose}\ \isanewline
\ \ \ \ \ \ \isacommand{unfolding}\isamarkupfalse%
\ preds{\isacharunderscore}{\kern0pt}def\ Rrel{\isacharunderscore}{\kern0pt}def\ \isacommand{by}\isamarkupfalse%
\ auto\ \isanewline
\isanewline
\ \ \ \ \isacommand{show}\isamarkupfalse%
\ {\isachardoublequoteopen}s\ {\isasymin}\ {\isacharbraceleft}{\kern0pt}v\ {\isasymin}\ {\isacharparenleft}{\kern0pt}preds{\isacharparenleft}{\kern0pt}R{\isacharcomma}{\kern0pt}\ x{\isacharparenright}{\kern0pt}\ {\isasymtimes}\ {\isacharbraceleft}{\kern0pt}a{\isacharbraceright}{\kern0pt}{\isacharparenright}{\kern0pt}\ {\isasymtimes}\ {\isacharparenleft}{\kern0pt}preds{\isacharparenleft}{\kern0pt}R{\isacharcomma}{\kern0pt}\ x{\isacharparenright}{\kern0pt}\ {\isasymtimes}\ {\isacharbraceleft}{\kern0pt}a{\isacharbraceright}{\kern0pt}\ {\isasymunion}\ {\isacharbraceleft}{\kern0pt}{\isasymlangle}x{\isacharcomma}{\kern0pt}\ a{\isasymrangle}{\isacharbraceright}{\kern0pt}{\isacharparenright}{\kern0pt}\ {\isachardot}{\kern0pt}\ v\ {\isasymin}\ preds{\isacharunderscore}{\kern0pt}rel{\isacharparenleft}{\kern0pt}{\isasymlambda}a\ b{\isachardot}{\kern0pt}\ {\isasymlangle}a{\isacharcomma}{\kern0pt}\ b{\isasymrangle}\ {\isasymin}\ prel{\isacharparenleft}{\kern0pt}Rrel{\isacharparenleft}{\kern0pt}R{\isacharcomma}{\kern0pt}\ M{\isacharparenright}{\kern0pt}{\isacharcomma}{\kern0pt}\ p{\isacharparenright}{\kern0pt}{\isacharcomma}{\kern0pt}\ {\isasymlangle}x{\isacharcomma}{\kern0pt}\ a{\isasymrangle}{\isacharparenright}{\kern0pt}{\isacharbraceright}{\kern0pt}{\isachardoublequoteclose}\ \isanewline
\ \ \ \ \ \ \isacommand{using}\isamarkupfalse%
\ sin\ \isanewline
\ \ \ \ \ \ \isacommand{apply}\isamarkupfalse%
\ simp\isanewline
\ \ \ \ \ \ \isacommand{using}\isamarkupfalse%
\ zaeq\ yaeq\ zayaH\ H{\isadigit{1}}\ H{\isadigit{2}}\ \isanewline
\ \ \ \ \ \ \isacommand{by}\isamarkupfalse%
\ auto\isanewline
\ \ \isacommand{qed}\isamarkupfalse%
\isanewline
\isanewline
\ \ \isacommand{finally}\isamarkupfalse%
\ \isacommand{show}\isamarkupfalse%
\ {\isacharquery}{\kern0pt}thesis\ \isacommand{using}\isamarkupfalse%
\ AinM\ \isacommand{by}\isamarkupfalse%
\ auto\isanewline
\isacommand{qed}\isamarkupfalse%
%
\endisatagproof
{\isafoldproof}%
%
\isadelimproof
\isanewline
%
\endisadelimproof
\isanewline
\isacommand{end}\isamarkupfalse%
\isanewline
\isanewline
\isacommand{definition}\isamarkupfalse%
\ is{\isacharunderscore}{\kern0pt}preds{\isacharunderscore}{\kern0pt}prel{\isacharunderscore}{\kern0pt}fm\ \isakeyword{where}\ \isanewline
\ \ {\isachardoublequoteopen}is{\isacharunderscore}{\kern0pt}preds{\isacharunderscore}{\kern0pt}prel{\isacharunderscore}{\kern0pt}fm{\isacharparenleft}{\kern0pt}Rfm{\isacharcomma}{\kern0pt}\ x{\isacharcomma}{\kern0pt}\ a{\isacharcomma}{\kern0pt}\ S{\isacharparenright}{\kern0pt}\ {\isasymequiv}\ Forall{\isacharparenleft}{\kern0pt}Iff{\isacharparenleft}{\kern0pt}Member{\isacharparenleft}{\kern0pt}{\isadigit{0}}{\isacharcomma}{\kern0pt}\ S\ {\isacharhash}{\kern0pt}{\isacharplus}{\kern0pt}\ {\isadigit{1}}{\isacharparenright}{\kern0pt}{\isacharcomma}{\kern0pt}\ is{\isacharunderscore}{\kern0pt}preds{\isacharunderscore}{\kern0pt}prel{\isacharunderscore}{\kern0pt}elem{\isacharunderscore}{\kern0pt}fm{\isacharparenleft}{\kern0pt}Rfm{\isacharcomma}{\kern0pt}\ x\ {\isacharhash}{\kern0pt}{\isacharplus}{\kern0pt}\ {\isadigit{1}}{\isacharcomma}{\kern0pt}\ a\ {\isacharhash}{\kern0pt}{\isacharplus}{\kern0pt}\ {\isadigit{1}}{\isacharcomma}{\kern0pt}\ {\isadigit{0}}{\isacharparenright}{\kern0pt}{\isacharparenright}{\kern0pt}{\isacharparenright}{\kern0pt}{\isachardoublequoteclose}\ \isanewline
\isanewline
\isacommand{context}\isamarkupfalse%
\ M{\isacharunderscore}{\kern0pt}ctm\ \isanewline
\isakeyword{begin}\ \isanewline
\isanewline
\isacommand{lemma}\isamarkupfalse%
\ is{\isacharunderscore}{\kern0pt}preds{\isacharunderscore}{\kern0pt}prel{\isacharunderscore}{\kern0pt}fm{\isacharunderscore}{\kern0pt}type\ {\isacharcolon}{\kern0pt}\isanewline
\ \ \isakeyword{fixes}\ Rfm\ x\ a\ S\ \isanewline
\ \ \isakeyword{assumes}\ {\isachardoublequoteopen}Rfm\ {\isasymin}\ formula{\isachardoublequoteclose}\ {\isachardoublequoteopen}x\ {\isasymin}\ nat{\isachardoublequoteclose}\ {\isachardoublequoteopen}a\ {\isasymin}\ nat{\isachardoublequoteclose}\ {\isachardoublequoteopen}S\ {\isasymin}\ nat{\isachardoublequoteclose}\ \isanewline
\ \ \isakeyword{shows}\ {\isachardoublequoteopen}is{\isacharunderscore}{\kern0pt}preds{\isacharunderscore}{\kern0pt}prel{\isacharunderscore}{\kern0pt}fm{\isacharparenleft}{\kern0pt}Rfm{\isacharcomma}{\kern0pt}\ x{\isacharcomma}{\kern0pt}\ a{\isacharcomma}{\kern0pt}\ S{\isacharparenright}{\kern0pt}\ {\isasymin}\ formula{\isachardoublequoteclose}\ \isanewline
%
\isadelimproof
\ \ %
\endisadelimproof
%
\isatagproof
\isacommand{unfolding}\isamarkupfalse%
\ is{\isacharunderscore}{\kern0pt}preds{\isacharunderscore}{\kern0pt}prel{\isacharunderscore}{\kern0pt}fm{\isacharunderscore}{\kern0pt}def\ \isanewline
\ \ \isacommand{apply}\isamarkupfalse%
{\isacharparenleft}{\kern0pt}subgoal{\isacharunderscore}{\kern0pt}tac\ {\isachardoublequoteopen}is{\isacharunderscore}{\kern0pt}preds{\isacharunderscore}{\kern0pt}prel{\isacharunderscore}{\kern0pt}elem{\isacharunderscore}{\kern0pt}fm{\isacharparenleft}{\kern0pt}Rfm{\isacharcomma}{\kern0pt}\ x\ {\isacharhash}{\kern0pt}{\isacharplus}{\kern0pt}\ {\isadigit{1}}{\isacharcomma}{\kern0pt}\ a\ {\isacharhash}{\kern0pt}{\isacharplus}{\kern0pt}\ {\isadigit{1}}{\isacharcomma}{\kern0pt}\ {\isadigit{0}}{\isacharparenright}{\kern0pt}\ {\isasymin}\ formula{\isachardoublequoteclose}{\isacharcomma}{\kern0pt}\ simp{\isacharparenright}{\kern0pt}\isanewline
\ \ \isacommand{apply}\isamarkupfalse%
{\isacharparenleft}{\kern0pt}rule\ is{\isacharunderscore}{\kern0pt}preds{\isacharunderscore}{\kern0pt}prel{\isacharunderscore}{\kern0pt}elem{\isacharunderscore}{\kern0pt}fm{\isacharunderscore}{\kern0pt}type{\isacharparenright}{\kern0pt}\isanewline
\ \ \isacommand{using}\isamarkupfalse%
\ assms\ \isanewline
\ \ \isacommand{by}\isamarkupfalse%
\ auto%
\endisatagproof
{\isafoldproof}%
%
\isadelimproof
\ \isanewline
%
\endisadelimproof
\isanewline
\isacommand{lemma}\isamarkupfalse%
\ is{\isacharunderscore}{\kern0pt}preds{\isacharunderscore}{\kern0pt}prel{\isacharunderscore}{\kern0pt}fm{\isacharunderscore}{\kern0pt}arity\ {\isacharcolon}{\kern0pt}\ \isanewline
\ \ \isakeyword{fixes}\ R\ Rfm\ x\ a\ S\ \isanewline
\ \ \isakeyword{assumes}\ {\isachardoublequoteopen}Rfm\ {\isasymin}\ formula{\isachardoublequoteclose}\ {\isachardoublequoteopen}arity{\isacharparenleft}{\kern0pt}Rfm{\isacharparenright}{\kern0pt}\ {\isacharequal}{\kern0pt}\ {\isadigit{2}}{\isachardoublequoteclose}\ {\isachardoublequoteopen}x\ {\isasymin}\ nat{\isachardoublequoteclose}\ {\isachardoublequoteopen}a\ {\isasymin}\ nat{\isachardoublequoteclose}\ {\isachardoublequoteopen}S\ {\isasymin}\ nat{\isachardoublequoteclose}\ \isanewline
\ \ \isakeyword{shows}\ {\isachardoublequoteopen}arity{\isacharparenleft}{\kern0pt}is{\isacharunderscore}{\kern0pt}preds{\isacharunderscore}{\kern0pt}prel{\isacharunderscore}{\kern0pt}fm{\isacharparenleft}{\kern0pt}Rfm{\isacharcomma}{\kern0pt}\ x{\isacharcomma}{\kern0pt}\ a{\isacharcomma}{\kern0pt}\ S{\isacharparenright}{\kern0pt}{\isacharparenright}{\kern0pt}\ {\isasymle}\ succ{\isacharparenleft}{\kern0pt}x{\isacharparenright}{\kern0pt}\ {\isasymunion}\ succ{\isacharparenleft}{\kern0pt}a{\isacharparenright}{\kern0pt}\ {\isasymunion}\ succ{\isacharparenleft}{\kern0pt}S{\isacharparenright}{\kern0pt}{\isachardoublequoteclose}\isanewline
%
\isadelimproof
\ \ %
\endisadelimproof
%
\isatagproof
\isacommand{unfolding}\isamarkupfalse%
\ is{\isacharunderscore}{\kern0pt}preds{\isacharunderscore}{\kern0pt}prel{\isacharunderscore}{\kern0pt}fm{\isacharunderscore}{\kern0pt}def\ \isanewline
\ \ \isacommand{using}\isamarkupfalse%
\ assms\ \isanewline
\ \ \isacommand{apply}\isamarkupfalse%
{\isacharparenleft}{\kern0pt}subgoal{\isacharunderscore}{\kern0pt}tac\ {\isachardoublequoteopen}is{\isacharunderscore}{\kern0pt}preds{\isacharunderscore}{\kern0pt}prel{\isacharunderscore}{\kern0pt}elem{\isacharunderscore}{\kern0pt}fm{\isacharparenleft}{\kern0pt}Rfm{\isacharcomma}{\kern0pt}\ succ{\isacharparenleft}{\kern0pt}x{\isacharparenright}{\kern0pt}{\isacharcomma}{\kern0pt}\ succ{\isacharparenleft}{\kern0pt}a{\isacharparenright}{\kern0pt}{\isacharcomma}{\kern0pt}\ {\isadigit{0}}{\isacharparenright}{\kern0pt}\ {\isasymin}\ formula{\isachardoublequoteclose}{\isacharparenright}{\kern0pt}\isanewline
\ \ \isacommand{apply}\isamarkupfalse%
\ simp\ \isanewline
\ \ \ \isacommand{apply}\isamarkupfalse%
{\isacharparenleft}{\kern0pt}subst\ pred{\isacharunderscore}{\kern0pt}Un{\isacharunderscore}{\kern0pt}distrib{\isacharcomma}{\kern0pt}\ simp{\isacharunderscore}{\kern0pt}all{\isacharparenright}{\kern0pt}{\isacharplus}{\kern0pt}\isanewline
\ \ \ \isacommand{apply}\isamarkupfalse%
{\isacharparenleft}{\kern0pt}rule\ Un{\isacharunderscore}{\kern0pt}least{\isacharunderscore}{\kern0pt}lt{\isacharcomma}{\kern0pt}\ simp{\isacharcomma}{\kern0pt}\ rule\ ltI{\isacharcomma}{\kern0pt}\ simp{\isacharcomma}{\kern0pt}\ simp{\isacharcomma}{\kern0pt}\ rule\ pred{\isacharunderscore}{\kern0pt}le{\isacharcomma}{\kern0pt}\ simp{\isacharunderscore}{\kern0pt}all{\isacharparenright}{\kern0pt}\isanewline
\ \ \ \isacommand{apply}\isamarkupfalse%
{\isacharparenleft}{\kern0pt}rule{\isacharunderscore}{\kern0pt}tac\ b{\isacharequal}{\kern0pt}{\isachardoublequoteopen}succ{\isacharparenleft}{\kern0pt}succ{\isacharparenleft}{\kern0pt}x{\isacharparenright}{\kern0pt}\ {\isasymunion}\ succ{\isacharparenleft}{\kern0pt}a{\isacharparenright}{\kern0pt}\ {\isasymunion}\ succ{\isacharparenleft}{\kern0pt}S{\isacharparenright}{\kern0pt}{\isacharparenright}{\kern0pt}{\isachardoublequoteclose}\ \isakeyword{and}\ a{\isacharequal}{\kern0pt}{\isachardoublequoteopen}succ{\isacharparenleft}{\kern0pt}succ{\isacharparenleft}{\kern0pt}x{\isacharparenright}{\kern0pt}{\isacharparenright}{\kern0pt}\ {\isasymunion}\ succ{\isacharparenleft}{\kern0pt}succ{\isacharparenleft}{\kern0pt}a{\isacharparenright}{\kern0pt}{\isacharparenright}{\kern0pt}\ {\isasymunion}\ succ{\isacharparenleft}{\kern0pt}succ{\isacharparenleft}{\kern0pt}S{\isacharparenright}{\kern0pt}{\isacharparenright}{\kern0pt}{\isachardoublequoteclose}\ \isakeyword{in}\ ssubst{\isacharparenright}{\kern0pt}\isanewline
\ \ \ \ \isacommand{apply}\isamarkupfalse%
{\isacharparenleft}{\kern0pt}subst\ succ{\isacharunderscore}{\kern0pt}Un{\isacharunderscore}{\kern0pt}distrib{\isacharcomma}{\kern0pt}\ simp{\isacharunderscore}{\kern0pt}all{\isacharparenright}{\kern0pt}\isanewline
\ \ \ \ \isacommand{apply}\isamarkupfalse%
{\isacharparenleft}{\kern0pt}subst\ succ{\isacharunderscore}{\kern0pt}Un{\isacharunderscore}{\kern0pt}distrib{\isacharcomma}{\kern0pt}\ simp{\isacharunderscore}{\kern0pt}all{\isacharparenright}{\kern0pt}\isanewline
\ \ \ \isacommand{apply}\isamarkupfalse%
{\isacharparenleft}{\kern0pt}rule{\isacharunderscore}{\kern0pt}tac\ j{\isacharequal}{\kern0pt}{\isachardoublequoteopen}succ{\isacharparenleft}{\kern0pt}succ{\isacharparenleft}{\kern0pt}x{\isacharparenright}{\kern0pt}{\isacharparenright}{\kern0pt}\ {\isasymunion}\ succ{\isacharparenleft}{\kern0pt}succ{\isacharparenleft}{\kern0pt}a{\isacharparenright}{\kern0pt}{\isacharparenright}{\kern0pt}\ {\isasymunion}\ {\isadigit{1}}{\isachardoublequoteclose}\ \isakeyword{in}\ le{\isacharunderscore}{\kern0pt}trans{\isacharparenright}{\kern0pt}\isanewline
\ \ \ \ \isacommand{apply}\isamarkupfalse%
{\isacharparenleft}{\kern0pt}rule\ is{\isacharunderscore}{\kern0pt}preds{\isacharunderscore}{\kern0pt}prel{\isacharunderscore}{\kern0pt}elem{\isacharunderscore}{\kern0pt}fm{\isacharunderscore}{\kern0pt}arity{\isacharparenright}{\kern0pt}\isanewline
\ \ \isacommand{using}\isamarkupfalse%
\ assms\ \isanewline
\ \ \ \ \ \ \ \ \isacommand{apply}\isamarkupfalse%
\ simp{\isacharunderscore}{\kern0pt}all\isanewline
\ \ \ \isacommand{apply}\isamarkupfalse%
{\isacharparenleft}{\kern0pt}rule\ Un{\isacharunderscore}{\kern0pt}least{\isacharunderscore}{\kern0pt}lt{\isacharcomma}{\kern0pt}\ rule\ Un{\isacharunderscore}{\kern0pt}least{\isacharunderscore}{\kern0pt}lt{\isacharparenright}{\kern0pt}\isanewline
\ \ \ \ \ \isacommand{apply}\isamarkupfalse%
{\isacharparenleft}{\kern0pt}simp{\isacharcomma}{\kern0pt}\ rule\ ltI{\isacharcomma}{\kern0pt}\ simp{\isacharcomma}{\kern0pt}\ simp{\isacharparenright}{\kern0pt}{\isacharplus}{\kern0pt}\isanewline
\ \ \ \isacommand{apply}\isamarkupfalse%
\ {\isacharparenleft}{\kern0pt}rule{\isacharunderscore}{\kern0pt}tac\ j{\isacharequal}{\kern0pt}{\isachardoublequoteopen}succ{\isacharparenleft}{\kern0pt}succ{\isacharparenleft}{\kern0pt}S{\isacharparenright}{\kern0pt}{\isacharparenright}{\kern0pt}{\isachardoublequoteclose}\ \isakeyword{in}\ le{\isacharunderscore}{\kern0pt}trans{\isacharcomma}{\kern0pt}\ simp{\isacharcomma}{\kern0pt}\ rule\ Un{\isacharunderscore}{\kern0pt}upper{\isadigit{2}}{\isacharunderscore}{\kern0pt}le{\isacharcomma}{\kern0pt}\ simp{\isacharunderscore}{\kern0pt}all{\isacharparenright}{\kern0pt}\isanewline
\ \ \isacommand{apply}\isamarkupfalse%
{\isacharparenleft}{\kern0pt}rule\ is{\isacharunderscore}{\kern0pt}preds{\isacharunderscore}{\kern0pt}prel{\isacharunderscore}{\kern0pt}elem{\isacharunderscore}{\kern0pt}fm{\isacharunderscore}{\kern0pt}type{\isacharparenright}{\kern0pt}\isanewline
\ \ \isacommand{using}\isamarkupfalse%
\ assms\isanewline
\ \ \isacommand{by}\isamarkupfalse%
\ auto%
\endisatagproof
{\isafoldproof}%
%
\isadelimproof
\isanewline
%
\endisadelimproof
\isanewline
\isacommand{lemma}\isamarkupfalse%
\ sats{\isacharunderscore}{\kern0pt}is{\isacharunderscore}{\kern0pt}preds{\isacharunderscore}{\kern0pt}prel{\isacharunderscore}{\kern0pt}fm{\isacharunderscore}{\kern0pt}iff\ {\isacharcolon}{\kern0pt}\ \isanewline
\ \ \isakeyword{fixes}\ R\ Rfm\ i\ j\ k\ x\ a\ S\ env\ p\isanewline
\ \ \isakeyword{assumes}\ {\isachardoublequoteopen}Relation{\isacharunderscore}{\kern0pt}fm{\isacharparenleft}{\kern0pt}R{\isacharcomma}{\kern0pt}\ Rfm{\isacharparenright}{\kern0pt}{\isachardoublequoteclose}\ {\isachardoublequoteopen}preds{\isacharparenleft}{\kern0pt}R{\isacharcomma}{\kern0pt}\ x{\isacharparenright}{\kern0pt}\ {\isasymin}\ M{\isachardoublequoteclose}\ \ {\isachardoublequoteopen}env\ {\isasymin}\ list{\isacharparenleft}{\kern0pt}M{\isacharparenright}{\kern0pt}{\isachardoublequoteclose}\ {\isachardoublequoteopen}i\ {\isasymin}\ nat{\isachardoublequoteclose}\ {\isachardoublequoteopen}j\ {\isasymin}\ nat{\isachardoublequoteclose}\ {\isachardoublequoteopen}k\ {\isasymin}\ nat{\isachardoublequoteclose}\ {\isachardoublequoteopen}nth{\isacharparenleft}{\kern0pt}i{\isacharcomma}{\kern0pt}\ env{\isacharparenright}{\kern0pt}\ {\isacharequal}{\kern0pt}\ x{\isachardoublequoteclose}\ {\isachardoublequoteopen}nth{\isacharparenleft}{\kern0pt}j{\isacharcomma}{\kern0pt}\ env{\isacharparenright}{\kern0pt}\ {\isacharequal}{\kern0pt}\ a{\isachardoublequoteclose}\ {\isachardoublequoteopen}nth{\isacharparenleft}{\kern0pt}k{\isacharcomma}{\kern0pt}\ env{\isacharparenright}{\kern0pt}\ {\isacharequal}{\kern0pt}\ S{\isachardoublequoteclose}\ \isanewline
\ \ \ \ \ \ \ \ \ \ {\isachardoublequoteopen}S\ {\isasymin}\ M{\isachardoublequoteclose}\ {\isachardoublequoteopen}x\ {\isasymin}\ M{\isachardoublequoteclose}\ {\isachardoublequoteopen}a\ {\isasymin}\ M{\isachardoublequoteclose}\ {\isachardoublequoteopen}p\ {\isasymin}\ M{\isachardoublequoteclose}\ {\isachardoublequoteopen}a\ {\isasymin}\ p{\isachardoublequoteclose}\ \isanewline
\ \ \isakeyword{shows}\ {\isachardoublequoteopen}sats{\isacharparenleft}{\kern0pt}M{\isacharcomma}{\kern0pt}\ is{\isacharunderscore}{\kern0pt}preds{\isacharunderscore}{\kern0pt}prel{\isacharunderscore}{\kern0pt}fm{\isacharparenleft}{\kern0pt}Rfm{\isacharcomma}{\kern0pt}\ i{\isacharcomma}{\kern0pt}\ j{\isacharcomma}{\kern0pt}\ k{\isacharparenright}{\kern0pt}{\isacharcomma}{\kern0pt}\ env{\isacharparenright}{\kern0pt}\ {\isasymlongleftrightarrow}\ S\ {\isacharequal}{\kern0pt}\ preds{\isacharunderscore}{\kern0pt}rel{\isacharparenleft}{\kern0pt}{\isasymlambda}a\ b{\isachardot}{\kern0pt}\ {\isacharless}{\kern0pt}a{\isacharcomma}{\kern0pt}\ b{\isachargreater}{\kern0pt}\ {\isasymin}\ prel{\isacharparenleft}{\kern0pt}Rrel{\isacharparenleft}{\kern0pt}R{\isacharcomma}{\kern0pt}\ M{\isacharparenright}{\kern0pt}{\isacharcomma}{\kern0pt}\ p{\isacharparenright}{\kern0pt}{\isacharcomma}{\kern0pt}\ {\isacharless}{\kern0pt}x{\isacharcomma}{\kern0pt}\ a{\isachargreater}{\kern0pt}{\isacharparenright}{\kern0pt}{\isachardoublequoteclose}\ \isanewline
%
\isadelimproof
%
\endisadelimproof
%
\isatagproof
\isacommand{proof}\isamarkupfalse%
{\isacharminus}{\kern0pt}\ \isanewline
\ \ \isacommand{have}\isamarkupfalse%
\ iff{\isacharunderscore}{\kern0pt}lemma\ {\isacharcolon}{\kern0pt}\ {\isachardoublequoteopen}{\isasymAnd}P\ Q\ R\ S{\isachardot}{\kern0pt}\ {\isacharparenleft}{\kern0pt}P\ {\isasymlongleftrightarrow}\ Q{\isacharparenright}{\kern0pt}\ {\isasymLongrightarrow}\ {\isacharparenleft}{\kern0pt}R\ {\isasymlongleftrightarrow}\ S{\isacharparenright}{\kern0pt}\ {\isasymLongrightarrow}\ {\isacharparenleft}{\kern0pt}P\ {\isasymlongleftrightarrow}\ R{\isacharparenright}{\kern0pt}\ {\isasymlongleftrightarrow}\ {\isacharparenleft}{\kern0pt}Q\ {\isasymlongleftrightarrow}\ S{\isacharparenright}{\kern0pt}{\isachardoublequoteclose}\ \isacommand{by}\isamarkupfalse%
\ auto\isanewline
\ \ \isacommand{have}\isamarkupfalse%
\ iff{\isacharunderscore}{\kern0pt}lemma{\isadigit{2}}\ {\isacharcolon}{\kern0pt}\ {\isachardoublequoteopen}{\isasymAnd}A\ B\ C{\isachardot}{\kern0pt}\ B\ {\isacharequal}{\kern0pt}\ C\ {\isasymLongrightarrow}\ {\isacharparenleft}{\kern0pt}A\ {\isacharequal}{\kern0pt}\ B{\isacharparenright}{\kern0pt}\ {\isasymlongleftrightarrow}\ {\isacharparenleft}{\kern0pt}A\ {\isacharequal}{\kern0pt}\ C{\isacharparenright}{\kern0pt}{\isachardoublequoteclose}\ \isacommand{by}\isamarkupfalse%
\ auto\isanewline
\isanewline
\ \ \isacommand{have}\isamarkupfalse%
\ {\isachardoublequoteopen}preds{\isacharunderscore}{\kern0pt}rel{\isacharparenleft}{\kern0pt}{\isasymlambda}a\ b{\isachardot}{\kern0pt}\ {\isasymlangle}a{\isacharcomma}{\kern0pt}\ b{\isasymrangle}\ {\isasymin}\ prel{\isacharparenleft}{\kern0pt}Rrel{\isacharparenleft}{\kern0pt}R{\isacharcomma}{\kern0pt}\ M{\isacharparenright}{\kern0pt}{\isacharcomma}{\kern0pt}\ p{\isacharparenright}{\kern0pt}{\isacharcomma}{\kern0pt}\ {\isasymlangle}x{\isacharcomma}{\kern0pt}\ a{\isasymrangle}{\isacharparenright}{\kern0pt}\ {\isasymin}\ M{\isachardoublequoteclose}\isanewline
\ \ \ \ \isacommand{apply}\isamarkupfalse%
{\isacharparenleft}{\kern0pt}rule\ preds{\isacharunderscore}{\kern0pt}prel{\isacharunderscore}{\kern0pt}in{\isacharunderscore}{\kern0pt}M{\isacharparenright}{\kern0pt}\isanewline
\ \ \ \ \isacommand{using}\isamarkupfalse%
\ assms\ \isanewline
\ \ \ \ \isacommand{by}\isamarkupfalse%
\ auto\isanewline
\ \ \isacommand{then}\isamarkupfalse%
\ \isacommand{have}\isamarkupfalse%
\ H{\isacharcolon}{\kern0pt}\ {\isachardoublequoteopen}preds{\isacharunderscore}{\kern0pt}rel{\isacharparenleft}{\kern0pt}{\isasymlambda}a\ b{\isachardot}{\kern0pt}\ {\isasymlangle}a{\isacharcomma}{\kern0pt}\ b{\isasymrangle}\ {\isasymin}\ prel{\isacharparenleft}{\kern0pt}Rrel{\isacharparenleft}{\kern0pt}R{\isacharcomma}{\kern0pt}\ M{\isacharparenright}{\kern0pt}{\isacharcomma}{\kern0pt}\ p{\isacharparenright}{\kern0pt}{\isacharcomma}{\kern0pt}\ {\isasymlangle}x{\isacharcomma}{\kern0pt}\ a{\isasymrangle}{\isacharparenright}{\kern0pt}\ {\isasymsubseteq}\ M{\isachardoublequoteclose}\ \isanewline
\ \ \ \ \isacommand{using}\isamarkupfalse%
\ transM\ \isacommand{by}\isamarkupfalse%
\ auto\isanewline
\ \isanewline
\ \ \isacommand{have}\isamarkupfalse%
\ I{\isadigit{1}}\ {\isacharcolon}{\kern0pt}\ {\isachardoublequoteopen}sats{\isacharparenleft}{\kern0pt}M{\isacharcomma}{\kern0pt}\ is{\isacharunderscore}{\kern0pt}preds{\isacharunderscore}{\kern0pt}prel{\isacharunderscore}{\kern0pt}fm{\isacharparenleft}{\kern0pt}Rfm{\isacharcomma}{\kern0pt}\ i{\isacharcomma}{\kern0pt}\ j{\isacharcomma}{\kern0pt}\ k{\isacharparenright}{\kern0pt}{\isacharcomma}{\kern0pt}\ env{\isacharparenright}{\kern0pt}\ {\isasymlongleftrightarrow}\ {\isacharparenleft}{\kern0pt}{\isasymforall}v\ {\isasymin}\ M{\isachardot}{\kern0pt}\ v\ {\isasymin}\ S\ {\isasymlongleftrightarrow}\ v\ {\isasymin}\ preds{\isacharunderscore}{\kern0pt}rel{\isacharparenleft}{\kern0pt}{\isasymlambda}a\ b{\isachardot}{\kern0pt}\ {\isacharless}{\kern0pt}a{\isacharcomma}{\kern0pt}\ b{\isachargreater}{\kern0pt}\ {\isasymin}\ prel{\isacharparenleft}{\kern0pt}Rrel{\isacharparenleft}{\kern0pt}R{\isacharcomma}{\kern0pt}\ M{\isacharparenright}{\kern0pt}{\isacharcomma}{\kern0pt}\ p{\isacharparenright}{\kern0pt}{\isacharcomma}{\kern0pt}\ {\isacharless}{\kern0pt}x{\isacharcomma}{\kern0pt}\ a{\isachargreater}{\kern0pt}{\isacharparenright}{\kern0pt}{\isacharparenright}{\kern0pt}{\isachardoublequoteclose}\isanewline
\ \ \ \ \isacommand{unfolding}\isamarkupfalse%
\ is{\isacharunderscore}{\kern0pt}preds{\isacharunderscore}{\kern0pt}prel{\isacharunderscore}{\kern0pt}fm{\isacharunderscore}{\kern0pt}def\ \isanewline
\ \ \ \ \isacommand{apply}\isamarkupfalse%
{\isacharparenleft}{\kern0pt}rule\ iff{\isacharunderscore}{\kern0pt}trans{\isacharcomma}{\kern0pt}\ rule\ sats{\isacharunderscore}{\kern0pt}Forall{\isacharunderscore}{\kern0pt}iff{\isacharcomma}{\kern0pt}\ simp\ add{\isacharcolon}{\kern0pt}assms{\isacharcomma}{\kern0pt}\ rule\ ball{\isacharunderscore}{\kern0pt}iff{\isacharparenright}{\kern0pt}\isanewline
\ \ \ \ \isacommand{apply}\isamarkupfalse%
{\isacharparenleft}{\kern0pt}rule\ iff{\isacharunderscore}{\kern0pt}trans{\isacharcomma}{\kern0pt}\ rule\ sats{\isacharunderscore}{\kern0pt}Iff{\isacharunderscore}{\kern0pt}iff{\isacharcomma}{\kern0pt}\ simp\ add{\isacharcolon}{\kern0pt}assms{\isacharcomma}{\kern0pt}\ rule\ iff{\isacharunderscore}{\kern0pt}lemma{\isacharparenright}{\kern0pt}\isanewline
\ \ \ \ \isacommand{using}\isamarkupfalse%
\ assms\ \isanewline
\ \ \ \ \ \isacommand{apply}\isamarkupfalse%
\ simp\isanewline
\ \ \ \ \isacommand{apply}\isamarkupfalse%
{\isacharparenleft}{\kern0pt}rule\ sats{\isacharunderscore}{\kern0pt}is{\isacharunderscore}{\kern0pt}preds{\isacharunderscore}{\kern0pt}prel{\isacharunderscore}{\kern0pt}elem{\isacharunderscore}{\kern0pt}fm{\isacharunderscore}{\kern0pt}iff{\isacharparenright}{\kern0pt}\isanewline
\ \ \ \ \isacommand{using}\isamarkupfalse%
\ assms\ \isanewline
\ \ \ \ \isacommand{by}\isamarkupfalse%
\ simp{\isacharunderscore}{\kern0pt}all\isanewline
\ \ \isacommand{have}\isamarkupfalse%
\ I{\isadigit{2}}\ {\isacharcolon}{\kern0pt}\ {\isachardoublequoteopen}{\isachardot}{\kern0pt}{\isachardot}{\kern0pt}{\isachardot}{\kern0pt}\ {\isasymlongleftrightarrow}\ S\ {\isacharequal}{\kern0pt}\ preds{\isacharunderscore}{\kern0pt}rel{\isacharparenleft}{\kern0pt}{\isasymlambda}a\ b{\isachardot}{\kern0pt}\ {\isacharless}{\kern0pt}a{\isacharcomma}{\kern0pt}\ b{\isachargreater}{\kern0pt}\ {\isasymin}\ prel{\isacharparenleft}{\kern0pt}Rrel{\isacharparenleft}{\kern0pt}R{\isacharcomma}{\kern0pt}\ M{\isacharparenright}{\kern0pt}{\isacharcomma}{\kern0pt}\ p{\isacharparenright}{\kern0pt}{\isacharcomma}{\kern0pt}\ {\isacharless}{\kern0pt}x{\isacharcomma}{\kern0pt}\ a{\isachargreater}{\kern0pt}{\isacharparenright}{\kern0pt}{\isachardoublequoteclose}\ \isanewline
\ \ \ \ \isacommand{apply}\isamarkupfalse%
{\isacharparenleft}{\kern0pt}rule\ iffI{\isacharcomma}{\kern0pt}\ rule\ equality{\isacharunderscore}{\kern0pt}iffI{\isacharcomma}{\kern0pt}\ rule\ iffI{\isacharparenright}{\kern0pt}\isanewline
\ \ \ \ \ \ \isacommand{apply}\isamarkupfalse%
{\isacharparenleft}{\kern0pt}rename{\isacharunderscore}{\kern0pt}tac\ v{\isacharcomma}{\kern0pt}\ subgoal{\isacharunderscore}{\kern0pt}tac\ {\isachardoublequoteopen}v\ {\isasymin}\ M{\isachardoublequoteclose}{\isacharcomma}{\kern0pt}\ simp{\isacharparenright}{\kern0pt}\isanewline
\ \ \ \ \isacommand{using}\isamarkupfalse%
\ assms\ transM\ \isanewline
\ \ \ \ \ \ \isacommand{apply}\isamarkupfalse%
\ force\isanewline
\ \ \ \ \ \isacommand{apply}\isamarkupfalse%
{\isacharparenleft}{\kern0pt}rename{\isacharunderscore}{\kern0pt}tac\ v{\isacharcomma}{\kern0pt}\ subgoal{\isacharunderscore}{\kern0pt}tac\ {\isachardoublequoteopen}v\ {\isasymin}\ M{\isachardoublequoteclose}{\isacharcomma}{\kern0pt}\ simp{\isacharparenright}{\kern0pt}\isanewline
\ \ \ \ \isacommand{using}\isamarkupfalse%
\ H\ \isanewline
\ \ \ \ \ \isacommand{apply}\isamarkupfalse%
\ force\isanewline
\ \ \ \ \isacommand{by}\isamarkupfalse%
\ auto\isanewline
\isanewline
\ \ \isacommand{then}\isamarkupfalse%
\ \isacommand{show}\isamarkupfalse%
\ {\isacharquery}{\kern0pt}thesis\ \isacommand{using}\isamarkupfalse%
\ I{\isadigit{1}}\ I{\isadigit{2}}\ \isacommand{by}\isamarkupfalse%
\ auto\isanewline
\isacommand{qed}\isamarkupfalse%
%
\endisatagproof
{\isafoldproof}%
%
\isadelimproof
\isanewline
%
\endisadelimproof
\isanewline
\isacommand{end}\isamarkupfalse%
\isanewline
\isanewline
\isacommand{definition}\isamarkupfalse%
\ is{\isacharunderscore}{\kern0pt}wftrec{\isacharunderscore}{\kern0pt}fm\ \isakeyword{where}\ {\isachardoublequoteopen}is{\isacharunderscore}{\kern0pt}wftrec{\isacharunderscore}{\kern0pt}fm{\isacharparenleft}{\kern0pt}Gfm{\isacharcomma}{\kern0pt}\ Rfm{\isacharcomma}{\kern0pt}\ x{\isacharcomma}{\kern0pt}\ a{\isacharcomma}{\kern0pt}\ v{\isacharparenright}{\kern0pt}\ {\isasymequiv}\ Exists{\isacharparenleft}{\kern0pt}Exists{\isacharparenleft}{\kern0pt}And{\isacharparenleft}{\kern0pt}is{\isacharunderscore}{\kern0pt}preds{\isacharunderscore}{\kern0pt}prel{\isacharunderscore}{\kern0pt}fm{\isacharparenleft}{\kern0pt}Rfm{\isacharcomma}{\kern0pt}\ x{\isacharhash}{\kern0pt}{\isacharplus}{\kern0pt}{\isadigit{2}}{\isacharcomma}{\kern0pt}\ a{\isacharhash}{\kern0pt}{\isacharplus}{\kern0pt}{\isadigit{2}}{\isacharcomma}{\kern0pt}\ {\isadigit{0}}{\isacharparenright}{\kern0pt}{\isacharcomma}{\kern0pt}\ And{\isacharparenleft}{\kern0pt}pair{\isacharunderscore}{\kern0pt}fm{\isacharparenleft}{\kern0pt}x\ {\isacharhash}{\kern0pt}{\isacharplus}{\kern0pt}\ {\isadigit{2}}{\isacharcomma}{\kern0pt}\ a\ {\isacharhash}{\kern0pt}{\isacharplus}{\kern0pt}\ {\isadigit{2}}{\isacharcomma}{\kern0pt}\ {\isadigit{1}}{\isacharparenright}{\kern0pt}{\isacharcomma}{\kern0pt}\ is{\isacharunderscore}{\kern0pt}wfrec{\isacharunderscore}{\kern0pt}fm{\isacharparenleft}{\kern0pt}Gfm{\isacharcomma}{\kern0pt}\ {\isadigit{0}}{\isacharcomma}{\kern0pt}\ {\isadigit{1}}{\isacharcomma}{\kern0pt}\ v\ {\isacharhash}{\kern0pt}{\isacharplus}{\kern0pt}\ {\isadigit{2}}{\isacharparenright}{\kern0pt}{\isacharparenright}{\kern0pt}{\isacharparenright}{\kern0pt}{\isacharparenright}{\kern0pt}{\isacharparenright}{\kern0pt}{\isachardoublequoteclose}\ \isanewline
\isanewline
\isacommand{context}\isamarkupfalse%
\ M{\isacharunderscore}{\kern0pt}ctm\ \isanewline
\isakeyword{begin}\ \isanewline
\isanewline
\isacommand{lemma}\isamarkupfalse%
\ is{\isacharunderscore}{\kern0pt}wftrec{\isacharunderscore}{\kern0pt}fm{\isacharunderscore}{\kern0pt}type\ {\isacharcolon}{\kern0pt}\ \isanewline
\ \ \isakeyword{fixes}\ Gfm\ Rfm\ x\ a\ v\ \isanewline
\ \ \isakeyword{assumes}\ {\isachardoublequoteopen}Gfm\ {\isasymin}\ formula{\isachardoublequoteclose}\ {\isachardoublequoteopen}Rfm\ {\isasymin}\ formula{\isachardoublequoteclose}\ {\isachardoublequoteopen}x\ {\isasymin}\ nat{\isachardoublequoteclose}\ {\isachardoublequoteopen}a\ {\isasymin}\ nat{\isachardoublequoteclose}\ {\isachardoublequoteopen}v\ {\isasymin}\ nat{\isachardoublequoteclose}\ \isanewline
\ \ \isakeyword{shows}\ {\isachardoublequoteopen}is{\isacharunderscore}{\kern0pt}wftrec{\isacharunderscore}{\kern0pt}fm{\isacharparenleft}{\kern0pt}Gfm{\isacharcomma}{\kern0pt}\ Rfm{\isacharcomma}{\kern0pt}\ x{\isacharcomma}{\kern0pt}\ a{\isacharcomma}{\kern0pt}\ v{\isacharparenright}{\kern0pt}\ {\isasymin}\ formula{\isachardoublequoteclose}\ \isanewline
%
\isadelimproof
\isanewline
\ \ %
\endisadelimproof
%
\isatagproof
\isacommand{unfolding}\isamarkupfalse%
\ is{\isacharunderscore}{\kern0pt}wftrec{\isacharunderscore}{\kern0pt}fm{\isacharunderscore}{\kern0pt}def\ \isanewline
\ \ \isacommand{apply}\isamarkupfalse%
{\isacharparenleft}{\kern0pt}subgoal{\isacharunderscore}{\kern0pt}tac\ {\isachardoublequoteopen}is{\isacharunderscore}{\kern0pt}preds{\isacharunderscore}{\kern0pt}prel{\isacharunderscore}{\kern0pt}fm{\isacharparenleft}{\kern0pt}Rfm{\isacharcomma}{\kern0pt}\ x\ {\isacharhash}{\kern0pt}{\isacharplus}{\kern0pt}\ {\isadigit{2}}{\isacharcomma}{\kern0pt}\ a\ {\isacharhash}{\kern0pt}{\isacharplus}{\kern0pt}\ {\isadigit{2}}{\isacharcomma}{\kern0pt}\ {\isadigit{0}}{\isacharparenright}{\kern0pt}\ {\isasymin}\ formula{\isachardoublequoteclose}{\isacharparenright}{\kern0pt}\ \isanewline
\ \ \isacommand{using}\isamarkupfalse%
\ assms\isanewline
\ \ \ \isacommand{apply}\isamarkupfalse%
\ simp\isanewline
\ \ \isacommand{apply}\isamarkupfalse%
{\isacharparenleft}{\kern0pt}rule\ is{\isacharunderscore}{\kern0pt}preds{\isacharunderscore}{\kern0pt}prel{\isacharunderscore}{\kern0pt}fm{\isacharunderscore}{\kern0pt}type{\isacharparenright}{\kern0pt}\isanewline
\ \ \isacommand{using}\isamarkupfalse%
\ assms\isanewline
\ \ \isacommand{by}\isamarkupfalse%
\ auto%
\endisatagproof
{\isafoldproof}%
%
\isadelimproof
\isanewline
%
\endisadelimproof
\isanewline
\isacommand{lemma}\isamarkupfalse%
\ arity{\isacharunderscore}{\kern0pt}is{\isacharunderscore}{\kern0pt}wftrec{\isacharunderscore}{\kern0pt}fm\ {\isacharcolon}{\kern0pt}\ \isanewline
\ \ \isakeyword{fixes}\ Gfm\ Rfm\ x\ a\ v\ \isanewline
\ \ \isakeyword{assumes}\ {\isachardoublequoteopen}Gfm\ {\isasymin}\ formula{\isachardoublequoteclose}\ {\isachardoublequoteopen}Rfm\ {\isasymin}\ formula{\isachardoublequoteclose}\ {\isachardoublequoteopen}arity{\isacharparenleft}{\kern0pt}Gfm{\isacharparenright}{\kern0pt}\ {\isasymle}\ {\isadigit{3}}{\isachardoublequoteclose}\ {\isachardoublequoteopen}arity{\isacharparenleft}{\kern0pt}Rfm{\isacharparenright}{\kern0pt}\ {\isacharequal}{\kern0pt}\ {\isadigit{2}}{\isachardoublequoteclose}\ {\isachardoublequoteopen}x\ {\isasymin}\ nat{\isachardoublequoteclose}\ {\isachardoublequoteopen}a\ {\isasymin}\ nat{\isachardoublequoteclose}\ {\isachardoublequoteopen}v\ {\isasymin}\ nat{\isachardoublequoteclose}\ \isanewline
\ \ \isakeyword{shows}\ {\isachardoublequoteopen}arity{\isacharparenleft}{\kern0pt}is{\isacharunderscore}{\kern0pt}wftrec{\isacharunderscore}{\kern0pt}fm{\isacharparenleft}{\kern0pt}Gfm{\isacharcomma}{\kern0pt}\ Rfm{\isacharcomma}{\kern0pt}\ x{\isacharcomma}{\kern0pt}\ a{\isacharcomma}{\kern0pt}\ v{\isacharparenright}{\kern0pt}{\isacharparenright}{\kern0pt}\ {\isasymle}\ succ{\isacharparenleft}{\kern0pt}x{\isacharparenright}{\kern0pt}\ {\isasymunion}\ succ{\isacharparenleft}{\kern0pt}a{\isacharparenright}{\kern0pt}\ {\isasymunion}\ succ{\isacharparenleft}{\kern0pt}v{\isacharparenright}{\kern0pt}{\isachardoublequoteclose}\ \isanewline
%
\isadelimproof
\isanewline
\ \ %
\endisadelimproof
%
\isatagproof
\isacommand{unfolding}\isamarkupfalse%
\ is{\isacharunderscore}{\kern0pt}wftrec{\isacharunderscore}{\kern0pt}fm{\isacharunderscore}{\kern0pt}def\ \isanewline
\ \ \isacommand{apply}\isamarkupfalse%
{\isacharparenleft}{\kern0pt}subgoal{\isacharunderscore}{\kern0pt}tac\ {\isachardoublequoteopen}is{\isacharunderscore}{\kern0pt}preds{\isacharunderscore}{\kern0pt}prel{\isacharunderscore}{\kern0pt}fm{\isacharparenleft}{\kern0pt}Rfm{\isacharcomma}{\kern0pt}\ x\ {\isacharhash}{\kern0pt}{\isacharplus}{\kern0pt}\ {\isadigit{2}}{\isacharcomma}{\kern0pt}\ a\ {\isacharhash}{\kern0pt}{\isacharplus}{\kern0pt}\ {\isadigit{2}}{\isacharcomma}{\kern0pt}\ {\isadigit{0}}{\isacharparenright}{\kern0pt}\ {\isasymin}\ formula{\isachardoublequoteclose}{\isacharparenright}{\kern0pt}\ \ \isanewline
\ \ \isacommand{using}\isamarkupfalse%
\ assms\isanewline
\ \ \ \isacommand{apply}\isamarkupfalse%
\ simp\isanewline
\ \ \ \isacommand{apply}\isamarkupfalse%
{\isacharparenleft}{\kern0pt}subst\ pred{\isacharunderscore}{\kern0pt}Un{\isacharunderscore}{\kern0pt}distrib{\isacharcomma}{\kern0pt}\ simp{\isacharunderscore}{\kern0pt}all{\isacharparenright}{\kern0pt}{\isacharplus}{\kern0pt}\isanewline
\ \ \ \isacommand{apply}\isamarkupfalse%
{\isacharparenleft}{\kern0pt}rule\ Un{\isacharunderscore}{\kern0pt}least{\isacharunderscore}{\kern0pt}lt{\isacharcomma}{\kern0pt}\ rule\ pred{\isacharunderscore}{\kern0pt}le{\isacharcomma}{\kern0pt}\ simp{\isacharunderscore}{\kern0pt}all{\isacharcomma}{\kern0pt}\ rule\ pred{\isacharunderscore}{\kern0pt}le{\isacharcomma}{\kern0pt}\ simp{\isacharunderscore}{\kern0pt}all{\isacharparenright}{\kern0pt}\isanewline
\ \ \ \ \isacommand{apply}\isamarkupfalse%
{\isacharparenleft}{\kern0pt}rule{\isacharunderscore}{\kern0pt}tac\ j{\isacharequal}{\kern0pt}{\isachardoublequoteopen}succ{\isacharparenleft}{\kern0pt}succ{\isacharparenleft}{\kern0pt}succ{\isacharparenleft}{\kern0pt}x{\isacharparenright}{\kern0pt}{\isacharparenright}{\kern0pt}{\isacharparenright}{\kern0pt}\ {\isasymunion}\ succ{\isacharparenleft}{\kern0pt}succ{\isacharparenleft}{\kern0pt}succ{\isacharparenleft}{\kern0pt}a{\isacharparenright}{\kern0pt}{\isacharparenright}{\kern0pt}{\isacharparenright}{\kern0pt}\ {\isasymunion}\ {\isadigit{1}}{\isachardoublequoteclose}\ \isakeyword{in}\ le{\isacharunderscore}{\kern0pt}trans{\isacharparenright}{\kern0pt}\isanewline
\ \ \ \ \ \isacommand{apply}\isamarkupfalse%
{\isacharparenleft}{\kern0pt}rule\ is{\isacharunderscore}{\kern0pt}preds{\isacharunderscore}{\kern0pt}prel{\isacharunderscore}{\kern0pt}fm{\isacharunderscore}{\kern0pt}arity{\isacharcomma}{\kern0pt}\ simp{\isacharunderscore}{\kern0pt}all{\isacharparenright}{\kern0pt}\isanewline
\ \ \ \ \isacommand{apply}\isamarkupfalse%
{\isacharparenleft}{\kern0pt}subst\ succ{\isacharunderscore}{\kern0pt}Un{\isacharunderscore}{\kern0pt}distrib{\isacharcomma}{\kern0pt}\ simp{\isacharunderscore}{\kern0pt}all{\isacharparenright}{\kern0pt}{\isacharplus}{\kern0pt}\isanewline
\ \ \ \ \isacommand{apply}\isamarkupfalse%
{\isacharparenleft}{\kern0pt}rule\ Un{\isacharunderscore}{\kern0pt}least{\isacharunderscore}{\kern0pt}lt{\isacharparenright}{\kern0pt}{\isacharplus}{\kern0pt}\isanewline
\ \ \ \ \ \ \isacommand{apply}\isamarkupfalse%
{\isacharparenleft}{\kern0pt}rule\ ltI{\isacharcomma}{\kern0pt}\ simp{\isacharunderscore}{\kern0pt}all{\isacharparenright}{\kern0pt}{\isacharplus}{\kern0pt}\isanewline
\ \ \ \ \isacommand{apply}\isamarkupfalse%
{\isacharparenleft}{\kern0pt}rule\ disjI{\isadigit{1}}{\isacharcomma}{\kern0pt}\ rule\ ltD{\isacharcomma}{\kern0pt}\ simp{\isacharparenright}{\kern0pt}\isanewline
\ \ \ \isacommand{apply}\isamarkupfalse%
{\isacharparenleft}{\kern0pt}rule\ Un{\isacharunderscore}{\kern0pt}least{\isacharunderscore}{\kern0pt}lt{\isacharcomma}{\kern0pt}\ rule\ pred{\isacharunderscore}{\kern0pt}le{\isacharcomma}{\kern0pt}\ simp{\isacharunderscore}{\kern0pt}all{\isacharcomma}{\kern0pt}\ rule\ pred{\isacharunderscore}{\kern0pt}le{\isacharcomma}{\kern0pt}\ simp{\isacharunderscore}{\kern0pt}all{\isacharparenright}{\kern0pt}\isanewline
\ \ \isacommand{unfolding}\isamarkupfalse%
\ pair{\isacharunderscore}{\kern0pt}fm{\isacharunderscore}{\kern0pt}def\ upair{\isacharunderscore}{\kern0pt}fm{\isacharunderscore}{\kern0pt}def\isanewline
\ \ \ \ \isacommand{apply}\isamarkupfalse%
\ {\isacharparenleft}{\kern0pt}simp\ del{\isacharcolon}{\kern0pt}FOL{\isacharunderscore}{\kern0pt}sats{\isacharunderscore}{\kern0pt}iff\ pair{\isacharunderscore}{\kern0pt}abs\ add{\isacharcolon}{\kern0pt}\ fm{\isacharunderscore}{\kern0pt}defs\ nat{\isacharunderscore}{\kern0pt}simp{\isacharunderscore}{\kern0pt}union{\isacharparenright}{\kern0pt}\ \isanewline
\ \ \ \isacommand{apply}\isamarkupfalse%
{\isacharparenleft}{\kern0pt}rule\ pred{\isacharunderscore}{\kern0pt}le{\isacharcomma}{\kern0pt}\ simp{\isacharunderscore}{\kern0pt}all{\isacharparenright}{\kern0pt}{\isacharplus}{\kern0pt}\isanewline
\ \ \ \isacommand{apply}\isamarkupfalse%
{\isacharparenleft}{\kern0pt}subst\ arity{\isacharunderscore}{\kern0pt}is{\isacharunderscore}{\kern0pt}wfrec{\isacharunderscore}{\kern0pt}fm{\isacharcomma}{\kern0pt}\ simp{\isacharunderscore}{\kern0pt}all{\isacharparenright}{\kern0pt}\isanewline
\ \ \ \isacommand{apply}\isamarkupfalse%
{\isacharparenleft}{\kern0pt}rule\ Un{\isacharunderscore}{\kern0pt}least{\isacharunderscore}{\kern0pt}lt{\isacharparenright}{\kern0pt}{\isacharplus}{\kern0pt}\isanewline
\ \ \ \ \ \ \isacommand{apply}\isamarkupfalse%
\ {\isacharparenleft}{\kern0pt}simp{\isacharcomma}{\kern0pt}\ simp{\isacharcomma}{\kern0pt}\ simp{\isacharparenright}{\kern0pt}\isanewline
\ \ \ \ \isacommand{apply}\isamarkupfalse%
{\isacharparenleft}{\kern0pt}rule\ ltI{\isacharcomma}{\kern0pt}\ simp{\isacharcomma}{\kern0pt}\ simp\ add{\isacharcolon}{\kern0pt}assms{\isacharparenright}{\kern0pt}\isanewline
\ \ \ \isacommand{apply}\isamarkupfalse%
{\isacharparenleft}{\kern0pt}rule\ pred{\isacharunderscore}{\kern0pt}le{\isacharcomma}{\kern0pt}\ simp{\isacharunderscore}{\kern0pt}all{\isacharparenright}{\kern0pt}{\isacharplus}{\kern0pt}\isanewline
\ \ \isacommand{apply}\isamarkupfalse%
{\isacharparenleft}{\kern0pt}rule{\isacharunderscore}{\kern0pt}tac\ j{\isacharequal}{\kern0pt}{\isadigit{3}}\ \isakeyword{in}\ le{\isacharunderscore}{\kern0pt}trans{\isacharcomma}{\kern0pt}\ simp{\isacharcomma}{\kern0pt}\ simp{\isacharparenright}{\kern0pt}\isanewline
\ \ \isacommand{apply}\isamarkupfalse%
{\isacharparenleft}{\kern0pt}rule\ is{\isacharunderscore}{\kern0pt}preds{\isacharunderscore}{\kern0pt}prel{\isacharunderscore}{\kern0pt}fm{\isacharunderscore}{\kern0pt}type{\isacharparenright}{\kern0pt}\isanewline
\ \ \isacommand{using}\isamarkupfalse%
\ assms\ \isanewline
\ \ \isacommand{by}\isamarkupfalse%
\ auto%
\endisatagproof
{\isafoldproof}%
%
\isadelimproof
\isanewline
%
\endisadelimproof
\ \ \isanewline
\isanewline
\isacommand{lemma}\isamarkupfalse%
\ wftrec{\isacharunderscore}{\kern0pt}prel{\isacharunderscore}{\kern0pt}eq\ {\isacharcolon}{\kern0pt}\ \isanewline
\ \ \isakeyword{fixes}\ r\ x\ a\ G\ Gfm\ H\ p\isanewline
\ \ \isakeyword{assumes}\ {\isachardoublequoteopen}wf{\isacharparenleft}{\kern0pt}r{\isacharparenright}{\kern0pt}{\isachardoublequoteclose}\ {\isachardoublequoteopen}r\ {\isasymin}\ M{\isachardoublequoteclose}\ {\isachardoublequoteopen}trans{\isacharparenleft}{\kern0pt}r{\isacharparenright}{\kern0pt}{\isachardoublequoteclose}\ {\isachardoublequoteopen}a\ {\isasymin}\ p{\isachardoublequoteclose}\ {\isachardoublequoteopen}p\ {\isasymin}\ M{\isachardoublequoteclose}\ {\isachardoublequoteopen}x\ {\isasymin}\ field{\isacharparenleft}{\kern0pt}r{\isacharparenright}{\kern0pt}{\isachardoublequoteclose}\ {\isachardoublequoteopen}Gfm\ {\isasymin}\ formula{\isachardoublequoteclose}\ {\isachardoublequoteopen}arity{\isacharparenleft}{\kern0pt}Gfm{\isacharparenright}{\kern0pt}\ {\isasymle}\ {\isadigit{3}}{\isachardoublequoteclose}\ \isanewline
\ \ \isakeyword{and}\ satsGfm{\isacharcolon}{\kern0pt}\ {\isachardoublequoteopen}\ {\isacharparenleft}{\kern0pt}{\isasymAnd}a{\isadigit{0}}\ a{\isadigit{1}}\ a{\isadigit{2}}\ env{\isachardot}{\kern0pt}\ a{\isadigit{0}}\ {\isasymin}\ M\ {\isasymLongrightarrow}\ a{\isadigit{1}}\ {\isasymin}\ M\ {\isasymLongrightarrow}\ a{\isadigit{2}}\ {\isasymin}\ M\ {\isasymLongrightarrow}\ env\ {\isasymin}\ list{\isacharparenleft}{\kern0pt}M{\isacharparenright}{\kern0pt}\ {\isasymLongrightarrow}\ a{\isadigit{0}}\ {\isacharequal}{\kern0pt}\ G{\isacharparenleft}{\kern0pt}a{\isadigit{2}}{\isacharcomma}{\kern0pt}\ a{\isadigit{1}}{\isacharparenright}{\kern0pt}\ {\isasymlongleftrightarrow}\ sats{\isacharparenleft}{\kern0pt}M{\isacharcomma}{\kern0pt}\ Gfm{\isacharcomma}{\kern0pt}\ {\isacharbrackleft}{\kern0pt}a{\isadigit{0}}{\isacharcomma}{\kern0pt}\ a{\isadigit{1}}{\isacharcomma}{\kern0pt}\ a{\isadigit{2}}{\isacharbrackright}{\kern0pt}\ {\isacharat}{\kern0pt}\ env{\isacharparenright}{\kern0pt}{\isacharparenright}{\kern0pt}{\isachardoublequoteclose}\ \isanewline
\ \ \isakeyword{and}\ GM\ {\isacharcolon}{\kern0pt}\ {\isachardoublequoteopen}{\isasymAnd}x\ g{\isachardot}{\kern0pt}\ x\ {\isasymin}\ M\ {\isasymLongrightarrow}\ function{\isacharparenleft}{\kern0pt}g{\isacharparenright}{\kern0pt}\ {\isasymLongrightarrow}\ g\ {\isasymin}\ M\ {\isasymLongrightarrow}\ G{\isacharparenleft}{\kern0pt}x{\isacharcomma}{\kern0pt}\ g{\isacharparenright}{\kern0pt}\ {\isasymin}\ M{\isachardoublequoteclose}\ \ \isanewline
\ \ \isakeyword{and}\ HGeq{\isacharcolon}{\kern0pt}\ {\isachardoublequoteopen}{\isasymAnd}h\ g\ x{\isachardot}{\kern0pt}\ h\ {\isasymin}\ r\ {\isacharminus}{\kern0pt}{\isacharbackquote}{\kern0pt}{\isacharbackquote}{\kern0pt}\ {\isacharbraceleft}{\kern0pt}x{\isacharbraceright}{\kern0pt}\ {\isasymrightarrow}\ M\ {\isasymLongrightarrow}\ g\ {\isasymin}\ {\isacharparenleft}{\kern0pt}r\ {\isacharminus}{\kern0pt}{\isacharbackquote}{\kern0pt}{\isacharbackquote}{\kern0pt}\ {\isacharbraceleft}{\kern0pt}x{\isacharbraceright}{\kern0pt}\ {\isasymtimes}\ {\isacharbraceleft}{\kern0pt}a{\isacharbraceright}{\kern0pt}{\isacharparenright}{\kern0pt}\ {\isasymrightarrow}\ M\ {\isasymLongrightarrow}\ g\ {\isasymin}\ M\ \ \isanewline
\ \ \ \ \ \ \ \ \ \ \ \ \ \ \ {\isasymLongrightarrow}\ x\ {\isasymin}\ field{\isacharparenleft}{\kern0pt}r{\isacharparenright}{\kern0pt}\ {\isasymLongrightarrow}\ {\isacharparenleft}{\kern0pt}{\isasymAnd}y{\isachardot}{\kern0pt}\ y\ {\isasymin}\ r\ {\isacharminus}{\kern0pt}{\isacharbackquote}{\kern0pt}{\isacharbackquote}{\kern0pt}\ {\isacharbraceleft}{\kern0pt}x{\isacharbraceright}{\kern0pt}\ {\isasymLongrightarrow}\ h{\isacharbackquote}{\kern0pt}y\ {\isacharequal}{\kern0pt}\ g{\isacharbackquote}{\kern0pt}{\isacharless}{\kern0pt}y{\isacharcomma}{\kern0pt}\ a{\isachargreater}{\kern0pt}{\isacharparenright}{\kern0pt}\ {\isasymLongrightarrow}\ H{\isacharparenleft}{\kern0pt}x{\isacharcomma}{\kern0pt}\ h{\isacharparenright}{\kern0pt}\ {\isacharequal}{\kern0pt}\ G{\isacharparenleft}{\kern0pt}{\isacharless}{\kern0pt}x{\isacharcomma}{\kern0pt}\ a{\isachargreater}{\kern0pt}{\isacharcomma}{\kern0pt}\ g{\isacharparenright}{\kern0pt}{\isachardoublequoteclose}\ \ \isanewline
\isanewline
\ \ \isakeyword{shows}\ {\isachardoublequoteopen}wftrec{\isacharparenleft}{\kern0pt}r{\isacharcomma}{\kern0pt}\ x{\isacharcomma}{\kern0pt}\ H{\isacharparenright}{\kern0pt}\ {\isacharequal}{\kern0pt}\ wftrec{\isacharparenleft}{\kern0pt}prel{\isacharparenleft}{\kern0pt}r{\isacharcomma}{\kern0pt}\ p{\isacharparenright}{\kern0pt}{\isacharcomma}{\kern0pt}\ {\isacharless}{\kern0pt}x{\isacharcomma}{\kern0pt}\ a{\isachargreater}{\kern0pt}{\isacharcomma}{\kern0pt}\ G{\isacharparenright}{\kern0pt}{\isachardoublequoteclose}\ \isanewline
%
\isadelimproof
%
\endisadelimproof
%
\isatagproof
\isacommand{proof}\isamarkupfalse%
\ {\isacharminus}{\kern0pt}\ \isanewline
\ \ \isacommand{have}\isamarkupfalse%
\ {\isachardoublequoteopen}{\isasymAnd}x{\isachardot}{\kern0pt}\ x\ {\isasymin}\ field{\isacharparenleft}{\kern0pt}r{\isacharparenright}{\kern0pt}\ {\isasymlongrightarrow}\ wftrec{\isacharparenleft}{\kern0pt}r{\isacharcomma}{\kern0pt}\ x{\isacharcomma}{\kern0pt}\ H{\isacharparenright}{\kern0pt}\ {\isacharequal}{\kern0pt}\ wftrec{\isacharparenleft}{\kern0pt}prel{\isacharparenleft}{\kern0pt}r{\isacharcomma}{\kern0pt}\ p{\isacharparenright}{\kern0pt}{\isacharcomma}{\kern0pt}\ {\isacharless}{\kern0pt}x{\isacharcomma}{\kern0pt}\ a{\isachargreater}{\kern0pt}{\isacharcomma}{\kern0pt}\ G{\isacharparenright}{\kern0pt}{\isachardoublequoteclose}\isanewline
\ \ \isacommand{proof}\isamarkupfalse%
\ {\isacharparenleft}{\kern0pt}rule{\isacharunderscore}{\kern0pt}tac\ P{\isacharequal}{\kern0pt}{\isachardoublequoteopen}{\isasymlambda}x{\isachardot}{\kern0pt}\ x\ {\isasymin}\ field{\isacharparenleft}{\kern0pt}r{\isacharparenright}{\kern0pt}\ {\isasymlongrightarrow}\ wftrec{\isacharparenleft}{\kern0pt}r{\isacharcomma}{\kern0pt}\ x{\isacharcomma}{\kern0pt}\ H{\isacharparenright}{\kern0pt}\ {\isacharequal}{\kern0pt}\ wftrec{\isacharparenleft}{\kern0pt}prel{\isacharparenleft}{\kern0pt}r{\isacharcomma}{\kern0pt}\ p{\isacharparenright}{\kern0pt}{\isacharcomma}{\kern0pt}\ {\isacharless}{\kern0pt}x{\isacharcomma}{\kern0pt}\ a{\isachargreater}{\kern0pt}{\isacharcomma}{\kern0pt}\ G{\isacharparenright}{\kern0pt}{\isachardoublequoteclose}\ \isakeyword{and}\ r{\isacharequal}{\kern0pt}r\ \isakeyword{in}\ wf{\isacharunderscore}{\kern0pt}induct{\isacharcomma}{\kern0pt}\ simp\ add{\isacharcolon}{\kern0pt}assms{\isacharcomma}{\kern0pt}\ rule\ impI{\isacharparenright}{\kern0pt}\isanewline
\ \ \ \ \isacommand{fix}\isamarkupfalse%
\ x\ \isanewline
\ \ \ \ \isacommand{assume}\isamarkupfalse%
\ assms{\isadigit{1}}{\isacharcolon}{\kern0pt}\ {\isachardoublequoteopen}{\isacharparenleft}{\kern0pt}{\isasymAnd}y{\isachardot}{\kern0pt}\ {\isasymlangle}y{\isacharcomma}{\kern0pt}\ x{\isasymrangle}\ {\isasymin}\ r\ {\isasymLongrightarrow}\ y\ {\isasymin}\ field{\isacharparenleft}{\kern0pt}r{\isacharparenright}{\kern0pt}\ {\isasymlongrightarrow}\ wftrec{\isacharparenleft}{\kern0pt}r{\isacharcomma}{\kern0pt}\ y{\isacharcomma}{\kern0pt}\ H{\isacharparenright}{\kern0pt}\ {\isacharequal}{\kern0pt}\ wftrec{\isacharparenleft}{\kern0pt}prel{\isacharparenleft}{\kern0pt}r{\isacharcomma}{\kern0pt}\ p{\isacharparenright}{\kern0pt}{\isacharcomma}{\kern0pt}\ {\isasymlangle}y{\isacharcomma}{\kern0pt}\ a{\isasymrangle}{\isacharcomma}{\kern0pt}\ G{\isacharparenright}{\kern0pt}{\isacharparenright}{\kern0pt}{\isachardoublequoteclose}\ {\isachardoublequoteopen}x\ {\isasymin}\ field{\isacharparenleft}{\kern0pt}r{\isacharparenright}{\kern0pt}{\isachardoublequoteclose}\ \isanewline
\isanewline
\ \ \ \ \isacommand{have}\isamarkupfalse%
\ {\isachardoublequoteopen}field{\isacharparenleft}{\kern0pt}r{\isacharparenright}{\kern0pt}\ {\isasymin}\ M{\isachardoublequoteclose}\ \isacommand{using}\isamarkupfalse%
\ field{\isacharunderscore}{\kern0pt}closed\ assms\ \isacommand{by}\isamarkupfalse%
\ auto\isanewline
\ \ \ \ \isacommand{then}\isamarkupfalse%
\ \isacommand{have}\isamarkupfalse%
\ xinM\ {\isacharcolon}{\kern0pt}\ {\isachardoublequoteopen}x\ {\isasymin}\ M{\isachardoublequoteclose}\ \isacommand{using}\isamarkupfalse%
\ transM\ assms{\isadigit{1}}\ \isacommand{by}\isamarkupfalse%
\ auto\isanewline
\isanewline
\ \ \ \ \isacommand{have}\isamarkupfalse%
\ recfun\ {\isacharcolon}{\kern0pt}\ {\isachardoublequoteopen}is{\isacharunderscore}{\kern0pt}recfun{\isacharparenleft}{\kern0pt}r{\isacharcomma}{\kern0pt}\ x{\isacharcomma}{\kern0pt}\ H{\isacharcomma}{\kern0pt}\ the{\isacharunderscore}{\kern0pt}recfun{\isacharparenleft}{\kern0pt}r{\isacharcomma}{\kern0pt}\ x{\isacharcomma}{\kern0pt}\ H{\isacharparenright}{\kern0pt}{\isacharparenright}{\kern0pt}{\isachardoublequoteclose}\ \isanewline
\ \ \ \ \ \ \isacommand{apply}\isamarkupfalse%
{\isacharparenleft}{\kern0pt}rule\ unfold{\isacharunderscore}{\kern0pt}the{\isacharunderscore}{\kern0pt}recfun{\isacharparenright}{\kern0pt}\isanewline
\ \ \ \ \ \ \isacommand{using}\isamarkupfalse%
\ assms\ \isanewline
\ \ \ \ \ \ \isacommand{by}\isamarkupfalse%
\ auto\ \isanewline
\ \ \ \ \isacommand{then}\isamarkupfalse%
\ \isacommand{have}\isamarkupfalse%
\ eq{\isadigit{1}}{\isacharcolon}{\kern0pt}\ {\isachardoublequoteopen}the{\isacharunderscore}{\kern0pt}recfun{\isacharparenleft}{\kern0pt}r{\isacharcomma}{\kern0pt}\ x{\isacharcomma}{\kern0pt}\ H{\isacharparenright}{\kern0pt}\ {\isacharequal}{\kern0pt}\ {\isacharparenleft}{\kern0pt}{\isasymlambda}y{\isasymin}r\ {\isacharminus}{\kern0pt}{\isacharbackquote}{\kern0pt}{\isacharbackquote}{\kern0pt}\ {\isacharbraceleft}{\kern0pt}x{\isacharbraceright}{\kern0pt}{\isachardot}{\kern0pt}\ H{\isacharparenleft}{\kern0pt}y{\isacharcomma}{\kern0pt}\ restrict{\isacharparenleft}{\kern0pt}the{\isacharunderscore}{\kern0pt}recfun{\isacharparenleft}{\kern0pt}r{\isacharcomma}{\kern0pt}\ x{\isacharcomma}{\kern0pt}\ H{\isacharparenright}{\kern0pt}{\isacharcomma}{\kern0pt}\ r\ {\isacharminus}{\kern0pt}{\isacharbackquote}{\kern0pt}{\isacharbackquote}{\kern0pt}\ {\isacharbraceleft}{\kern0pt}y{\isacharbraceright}{\kern0pt}{\isacharparenright}{\kern0pt}{\isacharparenright}{\kern0pt}{\isacharparenright}{\kern0pt}{\isachardoublequoteclose}\ \isanewline
\ \ \ \ \ \ \isacommand{unfolding}\isamarkupfalse%
\ is{\isacharunderscore}{\kern0pt}recfun{\isacharunderscore}{\kern0pt}def\ \isanewline
\ \ \ \ \ \ \isacommand{by}\isamarkupfalse%
\ simp\isanewline
\isanewline
\ \ \ \ \isacommand{have}\isamarkupfalse%
\ recfun{\isadigit{2}}\ {\isacharcolon}{\kern0pt}\ {\isachardoublequoteopen}is{\isacharunderscore}{\kern0pt}recfun{\isacharparenleft}{\kern0pt}prel{\isacharparenleft}{\kern0pt}r{\isacharcomma}{\kern0pt}\ p{\isacharparenright}{\kern0pt}{\isacharcomma}{\kern0pt}\ {\isacharless}{\kern0pt}x{\isacharcomma}{\kern0pt}\ a{\isachargreater}{\kern0pt}{\isacharcomma}{\kern0pt}\ G{\isacharcomma}{\kern0pt}\ the{\isacharunderscore}{\kern0pt}recfun{\isacharparenleft}{\kern0pt}prel{\isacharparenleft}{\kern0pt}r{\isacharcomma}{\kern0pt}\ p{\isacharparenright}{\kern0pt}{\isacharcomma}{\kern0pt}\ {\isacharless}{\kern0pt}x{\isacharcomma}{\kern0pt}\ a{\isachargreater}{\kern0pt}{\isacharcomma}{\kern0pt}\ G{\isacharparenright}{\kern0pt}{\isacharparenright}{\kern0pt}{\isachardoublequoteclose}\ \isanewline
\ \ \ \ \ \ \isacommand{apply}\isamarkupfalse%
{\isacharparenleft}{\kern0pt}rule\ unfold{\isacharunderscore}{\kern0pt}the{\isacharunderscore}{\kern0pt}recfun{\isacharparenright}{\kern0pt}\isanewline
\ \ \ \ \ \ \isacommand{using}\isamarkupfalse%
\ assms\ wf{\isacharunderscore}{\kern0pt}prel\ prel{\isacharunderscore}{\kern0pt}trans\isanewline
\ \ \ \ \ \ \isacommand{by}\isamarkupfalse%
\ auto\ \isanewline
\ \ \ \ \isacommand{then}\isamarkupfalse%
\ \isacommand{have}\isamarkupfalse%
\ eq{\isadigit{2}}{\isacharcolon}{\kern0pt}\ {\isachardoublequoteopen}the{\isacharunderscore}{\kern0pt}recfun{\isacharparenleft}{\kern0pt}prel{\isacharparenleft}{\kern0pt}r{\isacharcomma}{\kern0pt}\ p{\isacharparenright}{\kern0pt}{\isacharcomma}{\kern0pt}\ {\isacharless}{\kern0pt}x{\isacharcomma}{\kern0pt}\ a{\isachargreater}{\kern0pt}{\isacharcomma}{\kern0pt}\ G{\isacharparenright}{\kern0pt}\ {\isacharequal}{\kern0pt}\ {\isacharparenleft}{\kern0pt}{\isasymlambda}y{\isasymin}prel{\isacharparenleft}{\kern0pt}r{\isacharcomma}{\kern0pt}\ p{\isacharparenright}{\kern0pt}\ {\isacharminus}{\kern0pt}{\isacharbackquote}{\kern0pt}{\isacharbackquote}{\kern0pt}\ {\isacharbraceleft}{\kern0pt}{\isacharless}{\kern0pt}x{\isacharcomma}{\kern0pt}\ a{\isachargreater}{\kern0pt}{\isacharbraceright}{\kern0pt}{\isachardot}{\kern0pt}\ G{\isacharparenleft}{\kern0pt}y{\isacharcomma}{\kern0pt}\ restrict{\isacharparenleft}{\kern0pt}the{\isacharunderscore}{\kern0pt}recfun{\isacharparenleft}{\kern0pt}prel{\isacharparenleft}{\kern0pt}r{\isacharcomma}{\kern0pt}\ p{\isacharparenright}{\kern0pt}{\isacharcomma}{\kern0pt}\ {\isacharless}{\kern0pt}x{\isacharcomma}{\kern0pt}\ a{\isachargreater}{\kern0pt}{\isacharcomma}{\kern0pt}\ G{\isacharparenright}{\kern0pt}{\isacharcomma}{\kern0pt}\ prel{\isacharparenleft}{\kern0pt}r{\isacharcomma}{\kern0pt}\ p{\isacharparenright}{\kern0pt}\ {\isacharminus}{\kern0pt}{\isacharbackquote}{\kern0pt}{\isacharbackquote}{\kern0pt}\ {\isacharbraceleft}{\kern0pt}y{\isacharbraceright}{\kern0pt}{\isacharparenright}{\kern0pt}{\isacharparenright}{\kern0pt}{\isacharparenright}{\kern0pt}{\isachardoublequoteclose}\ \isanewline
\ \ \ \ \ \ \isacommand{unfolding}\isamarkupfalse%
\ is{\isacharunderscore}{\kern0pt}recfun{\isacharunderscore}{\kern0pt}def\ \isanewline
\ \ \ \ \ \ \isacommand{by}\isamarkupfalse%
\ simp\isanewline
\isanewline
\ \ \ \ \isacommand{have}\isamarkupfalse%
\ app{\isacharunderscore}{\kern0pt}eq{\isacharcolon}{\kern0pt}\ {\isachardoublequoteopen}{\isasymAnd}y{\isachardot}{\kern0pt}\ y\ {\isasymin}\ r\ {\isacharminus}{\kern0pt}{\isacharbackquote}{\kern0pt}{\isacharbackquote}{\kern0pt}\ {\isacharbraceleft}{\kern0pt}x{\isacharbraceright}{\kern0pt}\ {\isasymLongrightarrow}\ the{\isacharunderscore}{\kern0pt}recfun{\isacharparenleft}{\kern0pt}r{\isacharcomma}{\kern0pt}\ x{\isacharcomma}{\kern0pt}\ H{\isacharparenright}{\kern0pt}{\isacharbackquote}{\kern0pt}y\ {\isacharequal}{\kern0pt}\ the{\isacharunderscore}{\kern0pt}recfun{\isacharparenleft}{\kern0pt}prel{\isacharparenleft}{\kern0pt}r{\isacharcomma}{\kern0pt}\ p{\isacharparenright}{\kern0pt}{\isacharcomma}{\kern0pt}\ {\isacharless}{\kern0pt}x{\isacharcomma}{\kern0pt}\ a{\isachargreater}{\kern0pt}{\isacharcomma}{\kern0pt}\ G{\isacharparenright}{\kern0pt}{\isacharbackquote}{\kern0pt}{\isacharless}{\kern0pt}y{\isacharcomma}{\kern0pt}\ a{\isachargreater}{\kern0pt}{\isachardoublequoteclose}\ \isanewline
\ \ \ \ \isacommand{proof}\isamarkupfalse%
\ {\isacharminus}{\kern0pt}\ \isanewline
\ \ \ \ \ \ \isacommand{fix}\isamarkupfalse%
\ y\ \isacommand{assume}\isamarkupfalse%
\ assms{\isadigit{2}}{\isacharcolon}{\kern0pt}\ {\isachardoublequoteopen}y\ {\isasymin}\ r\ {\isacharminus}{\kern0pt}{\isacharbackquote}{\kern0pt}{\isacharbackquote}{\kern0pt}\ {\isacharbraceleft}{\kern0pt}x{\isacharbraceright}{\kern0pt}{\isachardoublequoteclose}\ \isanewline
\ \ \ \ \ \ \isacommand{have}\isamarkupfalse%
\ {\isachardoublequoteopen}the{\isacharunderscore}{\kern0pt}recfun{\isacharparenleft}{\kern0pt}r{\isacharcomma}{\kern0pt}\ x{\isacharcomma}{\kern0pt}\ H{\isacharparenright}{\kern0pt}{\isacharbackquote}{\kern0pt}y\ {\isacharequal}{\kern0pt}\ H{\isacharparenleft}{\kern0pt}y{\isacharcomma}{\kern0pt}\ restrict{\isacharparenleft}{\kern0pt}the{\isacharunderscore}{\kern0pt}recfun{\isacharparenleft}{\kern0pt}r{\isacharcomma}{\kern0pt}\ x{\isacharcomma}{\kern0pt}\ H{\isacharparenright}{\kern0pt}{\isacharcomma}{\kern0pt}\ r\ {\isacharminus}{\kern0pt}{\isacharbackquote}{\kern0pt}{\isacharbackquote}{\kern0pt}\ {\isacharbraceleft}{\kern0pt}y{\isacharbraceright}{\kern0pt}{\isacharparenright}{\kern0pt}{\isacharparenright}{\kern0pt}{\isachardoublequoteclose}\isanewline
\ \ \ \ \ \ \ \ \isacommand{apply}\isamarkupfalse%
{\isacharparenleft}{\kern0pt}subst\ eq{\isadigit{1}}{\isacharparenright}{\kern0pt}\isanewline
\ \ \ \ \ \ \ \ \isacommand{using}\isamarkupfalse%
\ assms{\isadigit{2}}\ \isanewline
\ \ \ \ \ \ \ \ \isacommand{by}\isamarkupfalse%
\ simp\isanewline
\ \ \ \ \ \ \isacommand{also}\isamarkupfalse%
\ \isacommand{have}\isamarkupfalse%
\ {\isachardoublequoteopen}{\isachardot}{\kern0pt}{\isachardot}{\kern0pt}{\isachardot}{\kern0pt}\ {\isacharequal}{\kern0pt}\ H{\isacharparenleft}{\kern0pt}y{\isacharcomma}{\kern0pt}\ the{\isacharunderscore}{\kern0pt}recfun{\isacharparenleft}{\kern0pt}r{\isacharcomma}{\kern0pt}\ y{\isacharcomma}{\kern0pt}\ H{\isacharparenright}{\kern0pt}{\isacharparenright}{\kern0pt}{\isachardoublequoteclose}\ \isanewline
\ \ \ \ \ \ \ \ \isacommand{apply}\isamarkupfalse%
{\isacharparenleft}{\kern0pt}subst\ the{\isacharunderscore}{\kern0pt}recfun{\isacharunderscore}{\kern0pt}cut{\isacharparenright}{\kern0pt}\isanewline
\ \ \ \ \ \ \ \ \isacommand{using}\isamarkupfalse%
\ assms\ assms{\isadigit{2}}\ \isanewline
\ \ \ \ \ \ \ \ \isacommand{by}\isamarkupfalse%
\ auto\isanewline
\ \ \ \ \ \ \isacommand{also}\isamarkupfalse%
\ \isacommand{have}\isamarkupfalse%
\ {\isachardoublequoteopen}{\isachardot}{\kern0pt}{\isachardot}{\kern0pt}{\isachardot}{\kern0pt}\ {\isacharequal}{\kern0pt}\ wftrec{\isacharparenleft}{\kern0pt}r{\isacharcomma}{\kern0pt}\ y{\isacharcomma}{\kern0pt}\ H{\isacharparenright}{\kern0pt}{\isachardoublequoteclose}\ \isanewline
\ \ \ \ \ \ \ \ \isacommand{unfolding}\isamarkupfalse%
\ wftrec{\isacharunderscore}{\kern0pt}def\ \isanewline
\ \ \ \ \ \ \ \ \isacommand{by}\isamarkupfalse%
\ simp\isanewline
\ \ \ \ \ \ \isacommand{finally}\isamarkupfalse%
\ \isacommand{have}\isamarkupfalse%
\ H\ {\isacharcolon}{\kern0pt}\ {\isachardoublequoteopen}the{\isacharunderscore}{\kern0pt}recfun{\isacharparenleft}{\kern0pt}r{\isacharcomma}{\kern0pt}\ x{\isacharcomma}{\kern0pt}\ H{\isacharparenright}{\kern0pt}{\isacharbackquote}{\kern0pt}y\ {\isacharequal}{\kern0pt}\ wftrec{\isacharparenleft}{\kern0pt}r{\isacharcomma}{\kern0pt}\ y{\isacharcomma}{\kern0pt}\ H{\isacharparenright}{\kern0pt}{\isachardoublequoteclose}\ \isacommand{by}\isamarkupfalse%
\ simp\isanewline
\isanewline
\ \ \ \ \ \ \isacommand{have}\isamarkupfalse%
\ {\isachardoublequoteopen}{\isacharless}{\kern0pt}y{\isacharcomma}{\kern0pt}\ x{\isachargreater}{\kern0pt}\ {\isasymin}\ r{\isachardoublequoteclose}\ \isacommand{using}\isamarkupfalse%
\ assms{\isadigit{2}}\ \isacommand{by}\isamarkupfalse%
\ auto\isanewline
\ \ \ \ \ \ \isacommand{then}\isamarkupfalse%
\ \isacommand{have}\isamarkupfalse%
\ rel\ {\isacharcolon}{\kern0pt}\ {\isachardoublequoteopen}{\isacharless}{\kern0pt}{\isacharless}{\kern0pt}y{\isacharcomma}{\kern0pt}\ a{\isachargreater}{\kern0pt}{\isacharcomma}{\kern0pt}\ {\isacharless}{\kern0pt}x{\isacharcomma}{\kern0pt}\ a{\isachargreater}{\kern0pt}{\isachargreater}{\kern0pt}\ {\isasymin}\ prel{\isacharparenleft}{\kern0pt}r{\isacharcomma}{\kern0pt}\ p{\isacharparenright}{\kern0pt}{\isachardoublequoteclose}\isanewline
\ \ \ \ \ \ \ \ \isacommand{by}\isamarkupfalse%
{\isacharparenleft}{\kern0pt}rule\ prelI{\isacharcomma}{\kern0pt}\ simp\ add{\isacharcolon}{\kern0pt}assms{\isacharparenright}{\kern0pt}\isanewline
\ \ \ \ \ \ \isacommand{then}\isamarkupfalse%
\ \isacommand{have}\isamarkupfalse%
\ {\isachardoublequoteopen}{\isacharless}{\kern0pt}y{\isacharcomma}{\kern0pt}\ a{\isachargreater}{\kern0pt}\ {\isasymin}\ prel{\isacharparenleft}{\kern0pt}r{\isacharcomma}{\kern0pt}\ p{\isacharparenright}{\kern0pt}\ {\isacharminus}{\kern0pt}{\isacharbackquote}{\kern0pt}{\isacharbackquote}{\kern0pt}\ {\isacharbraceleft}{\kern0pt}{\isacharless}{\kern0pt}x{\isacharcomma}{\kern0pt}\ a{\isachargreater}{\kern0pt}{\isacharbraceright}{\kern0pt}{\isachardoublequoteclose}\ \isacommand{by}\isamarkupfalse%
\ auto\ \isanewline
\ \ \ \ \ \ \isacommand{then}\isamarkupfalse%
\ \isacommand{have}\isamarkupfalse%
\ {\isachardoublequoteopen}the{\isacharunderscore}{\kern0pt}recfun{\isacharparenleft}{\kern0pt}prel{\isacharparenleft}{\kern0pt}r{\isacharcomma}{\kern0pt}\ p{\isacharparenright}{\kern0pt}{\isacharcomma}{\kern0pt}\ {\isacharless}{\kern0pt}x{\isacharcomma}{\kern0pt}\ a{\isachargreater}{\kern0pt}{\isacharcomma}{\kern0pt}\ G{\isacharparenright}{\kern0pt}{\isacharbackquote}{\kern0pt}{\isacharless}{\kern0pt}y{\isacharcomma}{\kern0pt}\ a{\isachargreater}{\kern0pt}\ {\isacharequal}{\kern0pt}\ G{\isacharparenleft}{\kern0pt}{\isacharless}{\kern0pt}y{\isacharcomma}{\kern0pt}\ a{\isachargreater}{\kern0pt}{\isacharcomma}{\kern0pt}\ restrict{\isacharparenleft}{\kern0pt}the{\isacharunderscore}{\kern0pt}recfun{\isacharparenleft}{\kern0pt}prel{\isacharparenleft}{\kern0pt}r{\isacharcomma}{\kern0pt}\ p{\isacharparenright}{\kern0pt}{\isacharcomma}{\kern0pt}\ {\isacharless}{\kern0pt}x{\isacharcomma}{\kern0pt}\ a{\isachargreater}{\kern0pt}{\isacharcomma}{\kern0pt}\ G{\isacharparenright}{\kern0pt}{\isacharcomma}{\kern0pt}\ prel{\isacharparenleft}{\kern0pt}r{\isacharcomma}{\kern0pt}\ p{\isacharparenright}{\kern0pt}\ {\isacharminus}{\kern0pt}{\isacharbackquote}{\kern0pt}{\isacharbackquote}{\kern0pt}\ {\isacharbraceleft}{\kern0pt}{\isacharless}{\kern0pt}y{\isacharcomma}{\kern0pt}\ a{\isachargreater}{\kern0pt}{\isacharbraceright}{\kern0pt}{\isacharparenright}{\kern0pt}{\isacharparenright}{\kern0pt}{\isachardoublequoteclose}\isanewline
\ \ \ \ \ \ \ \ \isacommand{by}\isamarkupfalse%
{\isacharparenleft}{\kern0pt}subst\ eq{\isadigit{2}}{\isacharcomma}{\kern0pt}\ simp{\isacharparenright}{\kern0pt}\isanewline
\ \ \ \ \ \ \isacommand{also}\isamarkupfalse%
\ \isacommand{have}\isamarkupfalse%
\ {\isachardoublequoteopen}{\isachardot}{\kern0pt}{\isachardot}{\kern0pt}{\isachardot}{\kern0pt}\ {\isacharequal}{\kern0pt}\ G{\isacharparenleft}{\kern0pt}{\isacharless}{\kern0pt}y{\isacharcomma}{\kern0pt}\ a{\isachargreater}{\kern0pt}{\isacharcomma}{\kern0pt}\ the{\isacharunderscore}{\kern0pt}recfun{\isacharparenleft}{\kern0pt}prel{\isacharparenleft}{\kern0pt}r{\isacharcomma}{\kern0pt}\ p{\isacharparenright}{\kern0pt}{\isacharcomma}{\kern0pt}\ {\isacharless}{\kern0pt}y{\isacharcomma}{\kern0pt}\ a{\isachargreater}{\kern0pt}{\isacharcomma}{\kern0pt}\ G{\isacharparenright}{\kern0pt}{\isacharparenright}{\kern0pt}{\isachardoublequoteclose}\ \isanewline
\ \ \ \ \ \ \ \ \isacommand{apply}\isamarkupfalse%
{\isacharparenleft}{\kern0pt}subst\ the{\isacharunderscore}{\kern0pt}recfun{\isacharunderscore}{\kern0pt}cut{\isacharparenright}{\kern0pt}\isanewline
\ \ \ \ \ \ \ \ \isacommand{using}\isamarkupfalse%
\ assms\ rel\ prel{\isacharunderscore}{\kern0pt}trans\ wf{\isacharunderscore}{\kern0pt}prel\isanewline
\ \ \ \ \ \ \ \ \isacommand{by}\isamarkupfalse%
\ auto\ \isanewline
\ \ \ \ \ \ \isacommand{also}\isamarkupfalse%
\ \isacommand{have}\isamarkupfalse%
\ {\isachardoublequoteopen}{\isachardot}{\kern0pt}{\isachardot}{\kern0pt}{\isachardot}{\kern0pt}\ {\isacharequal}{\kern0pt}\ wftrec{\isacharparenleft}{\kern0pt}prel{\isacharparenleft}{\kern0pt}r{\isacharcomma}{\kern0pt}\ p{\isacharparenright}{\kern0pt}{\isacharcomma}{\kern0pt}\ {\isacharless}{\kern0pt}y{\isacharcomma}{\kern0pt}\ a{\isachargreater}{\kern0pt}{\isacharcomma}{\kern0pt}\ G{\isacharparenright}{\kern0pt}{\isachardoublequoteclose}\ \isanewline
\ \ \ \ \ \ \ \ \isacommand{unfolding}\isamarkupfalse%
\ wftrec{\isacharunderscore}{\kern0pt}def\isanewline
\ \ \ \ \ \ \ \ \isacommand{by}\isamarkupfalse%
\ simp\ \isanewline
\ \ \ \ \ \ \isacommand{finally}\isamarkupfalse%
\ \isacommand{have}\isamarkupfalse%
\ {\isachardoublequoteopen}the{\isacharunderscore}{\kern0pt}recfun{\isacharparenleft}{\kern0pt}prel{\isacharparenleft}{\kern0pt}r{\isacharcomma}{\kern0pt}\ p{\isacharparenright}{\kern0pt}{\isacharcomma}{\kern0pt}\ {\isacharless}{\kern0pt}x{\isacharcomma}{\kern0pt}\ a{\isachargreater}{\kern0pt}{\isacharcomma}{\kern0pt}\ G{\isacharparenright}{\kern0pt}{\isacharbackquote}{\kern0pt}{\isacharless}{\kern0pt}y{\isacharcomma}{\kern0pt}\ a{\isachargreater}{\kern0pt}\ {\isacharequal}{\kern0pt}\ wftrec{\isacharparenleft}{\kern0pt}prel{\isacharparenleft}{\kern0pt}r{\isacharcomma}{\kern0pt}\ p{\isacharparenright}{\kern0pt}{\isacharcomma}{\kern0pt}\ {\isacharless}{\kern0pt}y{\isacharcomma}{\kern0pt}\ a{\isachargreater}{\kern0pt}{\isacharcomma}{\kern0pt}\ G{\isacharparenright}{\kern0pt}{\isachardoublequoteclose}\ \isacommand{by}\isamarkupfalse%
\ simp\isanewline
\isanewline
\ \ \ \ \ \ \isacommand{then}\isamarkupfalse%
\ \isacommand{show}\isamarkupfalse%
\ {\isachardoublequoteopen}the{\isacharunderscore}{\kern0pt}recfun{\isacharparenleft}{\kern0pt}r{\isacharcomma}{\kern0pt}\ x{\isacharcomma}{\kern0pt}\ H{\isacharparenright}{\kern0pt}{\isacharbackquote}{\kern0pt}y\ {\isacharequal}{\kern0pt}\ the{\isacharunderscore}{\kern0pt}recfun{\isacharparenleft}{\kern0pt}prel{\isacharparenleft}{\kern0pt}r{\isacharcomma}{\kern0pt}\ p{\isacharparenright}{\kern0pt}{\isacharcomma}{\kern0pt}\ {\isacharless}{\kern0pt}x{\isacharcomma}{\kern0pt}\ a{\isachargreater}{\kern0pt}{\isacharcomma}{\kern0pt}\ G{\isacharparenright}{\kern0pt}{\isacharbackquote}{\kern0pt}{\isacharless}{\kern0pt}y{\isacharcomma}{\kern0pt}\ a{\isachargreater}{\kern0pt}{\isachardoublequoteclose}\ \isanewline
\ \ \ \ \ \ \ \ \isacommand{using}\isamarkupfalse%
\ H\ assms{\isadigit{1}}\ assms{\isadigit{2}}\ \isanewline
\ \ \ \ \ \ \ \ \isacommand{by}\isamarkupfalse%
\ auto\isanewline
\ \ \ \ \isacommand{qed}\isamarkupfalse%
\isanewline
\isanewline
\ \ \ \ \isacommand{have}\isamarkupfalse%
\ recfun{\isadigit{2}}inM\ {\isacharcolon}{\kern0pt}\ {\isachardoublequoteopen}the{\isacharunderscore}{\kern0pt}recfun{\isacharparenleft}{\kern0pt}prel{\isacharparenleft}{\kern0pt}r{\isacharcomma}{\kern0pt}\ p{\isacharparenright}{\kern0pt}{\isacharcomma}{\kern0pt}\ {\isasymlangle}x{\isacharcomma}{\kern0pt}\ a{\isasymrangle}{\isacharcomma}{\kern0pt}\ G{\isacharparenright}{\kern0pt}\ {\isasymin}\ M{\isachardoublequoteclose}\ \isanewline
\ \ \ \ \ \ \isacommand{apply}\isamarkupfalse%
{\isacharparenleft}{\kern0pt}rule{\isacharunderscore}{\kern0pt}tac\ p{\isacharequal}{\kern0pt}Gfm\ \isakeyword{in}\ the{\isacharunderscore}{\kern0pt}recfun{\isacharunderscore}{\kern0pt}in{\isacharunderscore}{\kern0pt}M{\isacharparenright}{\kern0pt}\isanewline
\ \ \ \ \ \ \ \ \ \ \ \ \ \isacommand{apply}\isamarkupfalse%
{\isacharparenleft}{\kern0pt}rule\ wf{\isacharunderscore}{\kern0pt}prel{\isacharcomma}{\kern0pt}\ simp\ add{\isacharcolon}{\kern0pt}assms{\isacharparenright}{\kern0pt}\isanewline
\ \ \ \ \ \ \ \ \ \ \ \ \isacommand{apply}\isamarkupfalse%
{\isacharparenleft}{\kern0pt}rule\ prel{\isacharunderscore}{\kern0pt}trans{\isacharcomma}{\kern0pt}\ simp\ add{\isacharcolon}{\kern0pt}assms{\isacharparenright}{\kern0pt}\isanewline
\ \ \ \ \ \ \ \ \ \ \ \isacommand{apply}\isamarkupfalse%
{\isacharparenleft}{\kern0pt}rule\ prel{\isacharunderscore}{\kern0pt}closed{\isacharparenright}{\kern0pt}\isanewline
\ \ \ \ \ \ \isacommand{using}\isamarkupfalse%
\ assms\ pair{\isacharunderscore}{\kern0pt}in{\isacharunderscore}{\kern0pt}M{\isacharunderscore}{\kern0pt}iff\ xinM\ transM\ \isanewline
\ \ \ \ \ \ \ \ \ \ \ \ \isacommand{apply}\isamarkupfalse%
\ auto{\isacharbrackleft}{\kern0pt}{\isadigit{5}}{\isacharbrackright}{\kern0pt}\isanewline
\ \ \ \ \ \ \ \isacommand{apply}\isamarkupfalse%
{\isacharparenleft}{\kern0pt}rule\ GM{\isacharparenright}{\kern0pt}\isanewline
\ \ \ \ \ \ \ \ \ \isacommand{apply}\isamarkupfalse%
\ auto{\isacharbrackleft}{\kern0pt}{\isadigit{3}}{\isacharbrackright}{\kern0pt}\isanewline
\ \ \ \ \ \ \isacommand{apply}\isamarkupfalse%
{\isacharparenleft}{\kern0pt}rule\ satsGfm{\isacharparenright}{\kern0pt}\isanewline
\ \ \ \ \ \ \isacommand{by}\isamarkupfalse%
\ auto\isanewline
\isanewline
\ \ \ \ \isacommand{then}\isamarkupfalse%
\ \isacommand{have}\isamarkupfalse%
\ {\isachardoublequoteopen}range{\isacharparenleft}{\kern0pt}the{\isacharunderscore}{\kern0pt}recfun{\isacharparenleft}{\kern0pt}prel{\isacharparenleft}{\kern0pt}r{\isacharcomma}{\kern0pt}\ p{\isacharparenright}{\kern0pt}{\isacharcomma}{\kern0pt}\ {\isasymlangle}x{\isacharcomma}{\kern0pt}\ a{\isasymrangle}{\isacharcomma}{\kern0pt}\ G{\isacharparenright}{\kern0pt}{\isacharparenright}{\kern0pt}\ {\isasymin}\ M{\isachardoublequoteclose}\ \isacommand{using}\isamarkupfalse%
\ range{\isacharunderscore}{\kern0pt}closed\ \isacommand{by}\isamarkupfalse%
\ auto\isanewline
\ \ \ \ \isacommand{then}\isamarkupfalse%
\ \isacommand{have}\isamarkupfalse%
\ rangesubset{\isadigit{2}}\ {\isacharcolon}{\kern0pt}\ {\isachardoublequoteopen}range{\isacharparenleft}{\kern0pt}the{\isacharunderscore}{\kern0pt}recfun{\isacharparenleft}{\kern0pt}prel{\isacharparenleft}{\kern0pt}r{\isacharcomma}{\kern0pt}\ p{\isacharparenright}{\kern0pt}{\isacharcomma}{\kern0pt}\ {\isasymlangle}x{\isacharcomma}{\kern0pt}\ a{\isasymrangle}{\isacharcomma}{\kern0pt}\ G{\isacharparenright}{\kern0pt}{\isacharparenright}{\kern0pt}\ {\isasymsubseteq}\ M{\isachardoublequoteclose}\ \isacommand{using}\isamarkupfalse%
\ transM\ \isacommand{by}\isamarkupfalse%
\ auto\isanewline
\isanewline
\ \ \ \ \isacommand{have}\isamarkupfalse%
\ rangesubset\ {\isacharcolon}{\kern0pt}\ {\isachardoublequoteopen}range{\isacharparenleft}{\kern0pt}the{\isacharunderscore}{\kern0pt}recfun{\isacharparenleft}{\kern0pt}r{\isacharcomma}{\kern0pt}\ x{\isacharcomma}{\kern0pt}\ H{\isacharparenright}{\kern0pt}{\isacharparenright}{\kern0pt}\ {\isasymsubseteq}\ M{\isachardoublequoteclose}\ \isanewline
\ \ \ \ \isacommand{proof}\isamarkupfalse%
\ {\isacharparenleft}{\kern0pt}rule\ subsetI{\isacharparenright}{\kern0pt}\ \isanewline
\ \ \ \ \ \ \isacommand{fix}\isamarkupfalse%
\ v\ \isacommand{assume}\isamarkupfalse%
\ {\isachardoublequoteopen}v\ {\isasymin}\ range{\isacharparenleft}{\kern0pt}the{\isacharunderscore}{\kern0pt}recfun{\isacharparenleft}{\kern0pt}r{\isacharcomma}{\kern0pt}\ x{\isacharcomma}{\kern0pt}\ H{\isacharparenright}{\kern0pt}{\isacharparenright}{\kern0pt}{\isachardoublequoteclose}\ \isanewline
\ \ \ \ \ \ \isacommand{then}\isamarkupfalse%
\ \isacommand{obtain}\isamarkupfalse%
\ u\ \isakeyword{where}\ uH{\isacharcolon}{\kern0pt}\ {\isachardoublequoteopen}{\isacharless}{\kern0pt}u{\isacharcomma}{\kern0pt}\ v{\isachargreater}{\kern0pt}\ {\isasymin}\ the{\isacharunderscore}{\kern0pt}recfun{\isacharparenleft}{\kern0pt}r{\isacharcomma}{\kern0pt}\ x{\isacharcomma}{\kern0pt}\ H{\isacharparenright}{\kern0pt}{\isachardoublequoteclose}\ \isacommand{by}\isamarkupfalse%
\ auto\ \isanewline
\isanewline
\ \ \ \ \ \ \isacommand{have}\isamarkupfalse%
\ eq\ {\isacharcolon}{\kern0pt}\ {\isachardoublequoteopen}domain{\isacharparenleft}{\kern0pt}the{\isacharunderscore}{\kern0pt}recfun{\isacharparenleft}{\kern0pt}r{\isacharcomma}{\kern0pt}\ x{\isacharcomma}{\kern0pt}\ H{\isacharparenright}{\kern0pt}{\isacharparenright}{\kern0pt}\ {\isacharequal}{\kern0pt}\ r\ {\isacharminus}{\kern0pt}{\isacharbackquote}{\kern0pt}{\isacharbackquote}{\kern0pt}\ {\isacharbraceleft}{\kern0pt}x{\isacharbraceright}{\kern0pt}{\isachardoublequoteclose}\ \isanewline
\ \ \ \ \ \ \ \ \isacommand{by}\isamarkupfalse%
{\isacharparenleft}{\kern0pt}subst\ eq{\isadigit{1}}{\isacharcomma}{\kern0pt}\ subst\ domain{\isacharunderscore}{\kern0pt}lam{\isacharcomma}{\kern0pt}\ simp{\isacharparenright}{\kern0pt}\isanewline
\ \ \ \ \ \ \isacommand{have}\isamarkupfalse%
\ {\isachardoublequoteopen}u\ {\isasymin}\ domain{\isacharparenleft}{\kern0pt}the{\isacharunderscore}{\kern0pt}recfun{\isacharparenleft}{\kern0pt}r{\isacharcomma}{\kern0pt}\ x{\isacharcomma}{\kern0pt}\ H{\isacharparenright}{\kern0pt}{\isacharparenright}{\kern0pt}{\isachardoublequoteclose}\ \isacommand{by}\isamarkupfalse%
{\isacharparenleft}{\kern0pt}rule{\isacharunderscore}{\kern0pt}tac\ b{\isacharequal}{\kern0pt}v\ \isakeyword{in}\ domainI{\isacharcomma}{\kern0pt}\ simp\ add{\isacharcolon}{\kern0pt}uH{\isacharparenright}{\kern0pt}\ \isanewline
\ \ \ \ \ \ \isacommand{then}\isamarkupfalse%
\ \isacommand{have}\isamarkupfalse%
\ uin\ {\isacharcolon}{\kern0pt}\ {\isachardoublequoteopen}u\ {\isasymin}\ r\ {\isacharminus}{\kern0pt}{\isacharbackquote}{\kern0pt}{\isacharbackquote}{\kern0pt}\ {\isacharbraceleft}{\kern0pt}x{\isacharbraceright}{\kern0pt}{\isachardoublequoteclose}\ \isacommand{using}\isamarkupfalse%
\ eq\ \isacommand{by}\isamarkupfalse%
\ auto\ \isanewline
\isanewline
\ \ \ \ \ \ \isacommand{have}\isamarkupfalse%
\ rxinM\ {\isacharcolon}{\kern0pt}\ {\isachardoublequoteopen}r\ {\isacharminus}{\kern0pt}{\isacharbackquote}{\kern0pt}{\isacharbackquote}{\kern0pt}\ {\isacharbraceleft}{\kern0pt}x{\isacharbraceright}{\kern0pt}\ {\isasymin}\ M{\isachardoublequoteclose}\ \isanewline
\ \ \ \ \ \ \ \ \isacommand{apply}\isamarkupfalse%
{\isacharparenleft}{\kern0pt}rule\ to{\isacharunderscore}{\kern0pt}rin{\isacharcomma}{\kern0pt}\ rule\ vimage{\isacharunderscore}{\kern0pt}closed{\isacharparenright}{\kern0pt}\isanewline
\ \ \ \ \ \ \ \ \isacommand{using}\isamarkupfalse%
\ xinM\ singleton{\isacharunderscore}{\kern0pt}in{\isacharunderscore}{\kern0pt}M{\isacharunderscore}{\kern0pt}iff\ assms\ \isanewline
\ \ \ \ \ \ \ \ \isacommand{by}\isamarkupfalse%
\ auto\isanewline
\ \ \ \ \ \ \isacommand{have}\isamarkupfalse%
\ uinM\ {\isacharcolon}{\kern0pt}\ {\isachardoublequoteopen}u\ {\isasymin}\ M{\isachardoublequoteclose}\ \isanewline
\ \ \ \ \ \ \ \ \isacommand{apply}\isamarkupfalse%
{\isacharparenleft}{\kern0pt}rule\ to{\isacharunderscore}{\kern0pt}rin{\isacharcomma}{\kern0pt}\ rule{\isacharunderscore}{\kern0pt}tac\ x{\isacharequal}{\kern0pt}{\isachardoublequoteopen}r\ {\isacharminus}{\kern0pt}{\isacharbackquote}{\kern0pt}{\isacharbackquote}{\kern0pt}\ {\isacharbraceleft}{\kern0pt}x{\isacharbraceright}{\kern0pt}{\isachardoublequoteclose}\ \isakeyword{in}\ transM{\isacharparenright}{\kern0pt}\isanewline
\ \ \ \ \ \ \ \ \isacommand{using}\isamarkupfalse%
\ uin\ rxinM\ \isanewline
\ \ \ \ \ \ \ \ \isacommand{by}\isamarkupfalse%
\ auto\isanewline
\isanewline
\ \ \ \ \ \ \isacommand{have}\isamarkupfalse%
\ {\isachardoublequoteopen}the{\isacharunderscore}{\kern0pt}recfun{\isacharparenleft}{\kern0pt}prel{\isacharparenleft}{\kern0pt}r{\isacharcomma}{\kern0pt}\ p{\isacharparenright}{\kern0pt}{\isacharcomma}{\kern0pt}\ {\isacharless}{\kern0pt}x{\isacharcomma}{\kern0pt}\ a{\isachargreater}{\kern0pt}{\isacharcomma}{\kern0pt}\ G{\isacharparenright}{\kern0pt}{\isacharbackquote}{\kern0pt}{\isacharless}{\kern0pt}u{\isacharcomma}{\kern0pt}\ a{\isachargreater}{\kern0pt}\ {\isacharequal}{\kern0pt}\ the{\isacharunderscore}{\kern0pt}recfun{\isacharparenleft}{\kern0pt}r{\isacharcomma}{\kern0pt}\ x{\isacharcomma}{\kern0pt}\ H{\isacharparenright}{\kern0pt}{\isacharbackquote}{\kern0pt}u{\isachardoublequoteclose}\ \isanewline
\ \ \ \ \ \ \ \ \isacommand{by}\isamarkupfalse%
{\isacharparenleft}{\kern0pt}rule\ eq{\isacharunderscore}{\kern0pt}flip{\isacharcomma}{\kern0pt}\ rule\ app{\isacharunderscore}{\kern0pt}eq{\isacharcomma}{\kern0pt}\ simp\ add{\isacharcolon}{\kern0pt}uin{\isacharparenright}{\kern0pt}\isanewline
\ \ \ \ \ \ \isacommand{also}\isamarkupfalse%
\ \isacommand{have}\isamarkupfalse%
\ {\isachardoublequoteopen}{\isachardot}{\kern0pt}{\isachardot}{\kern0pt}{\isachardot}{\kern0pt}\ {\isacharequal}{\kern0pt}\ v{\isachardoublequoteclose}\ \isanewline
\ \ \ \ \ \ \ \ \isacommand{apply}\isamarkupfalse%
{\isacharparenleft}{\kern0pt}rule\ function{\isacharunderscore}{\kern0pt}apply{\isacharunderscore}{\kern0pt}equality{\isacharcomma}{\kern0pt}\ simp\ add{\isacharcolon}{\kern0pt}uH{\isacharparenright}{\kern0pt}\isanewline
\ \ \ \ \ \ \ \ \isacommand{apply}\isamarkupfalse%
{\isacharparenleft}{\kern0pt}subst\ eq{\isadigit{1}}{\isacharcomma}{\kern0pt}\ rule\ function{\isacharunderscore}{\kern0pt}lam{\isacharparenright}{\kern0pt}\isanewline
\ \ \ \ \ \ \ \ \isacommand{done}\isamarkupfalse%
\isanewline
\ \ \ \ \ \ \isacommand{finally}\isamarkupfalse%
\ \isacommand{have}\isamarkupfalse%
\ veq{\isacharcolon}{\kern0pt}\ {\isachardoublequoteopen}v\ {\isacharequal}{\kern0pt}\ the{\isacharunderscore}{\kern0pt}recfun{\isacharparenleft}{\kern0pt}prel{\isacharparenleft}{\kern0pt}r{\isacharcomma}{\kern0pt}\ p{\isacharparenright}{\kern0pt}{\isacharcomma}{\kern0pt}\ {\isacharless}{\kern0pt}x{\isacharcomma}{\kern0pt}\ a{\isachargreater}{\kern0pt}{\isacharcomma}{\kern0pt}\ G{\isacharparenright}{\kern0pt}{\isacharbackquote}{\kern0pt}{\isacharless}{\kern0pt}u{\isacharcomma}{\kern0pt}\ a{\isachargreater}{\kern0pt}{\isachardoublequoteclose}\ \isacommand{by}\isamarkupfalse%
\ simp\isanewline
\ \ \ \ \isanewline
\ \ \ \ \ \ \isacommand{show}\isamarkupfalse%
\ {\isachardoublequoteopen}v\ {\isasymin}\ M{\isachardoublequoteclose}\ \isanewline
\ \ \ \ \ \ \ \ \isacommand{apply}\isamarkupfalse%
{\isacharparenleft}{\kern0pt}subst\ veq{\isacharcomma}{\kern0pt}\ rule\ to{\isacharunderscore}{\kern0pt}rin{\isacharcomma}{\kern0pt}\ rule\ apply{\isacharunderscore}{\kern0pt}closed{\isacharcomma}{\kern0pt}\ simp\ add{\isacharcolon}{\kern0pt}recfun{\isadigit{2}}inM{\isacharparenright}{\kern0pt}\isanewline
\ \ \ \ \ \ \ \ \isacommand{using}\isamarkupfalse%
\ uinM\ assms\ transM\ pair{\isacharunderscore}{\kern0pt}in{\isacharunderscore}{\kern0pt}M{\isacharunderscore}{\kern0pt}iff\ \isanewline
\ \ \ \ \ \ \ \ \isacommand{by}\isamarkupfalse%
\ auto\isanewline
\ \ \ \ \isacommand{qed}\isamarkupfalse%
\isanewline
\isanewline
\isanewline
\ \ \ \ \isacommand{have}\isamarkupfalse%
\ vimageeq\ {\isacharcolon}{\kern0pt}\ {\isachardoublequoteopen}r\ {\isacharminus}{\kern0pt}{\isacharbackquote}{\kern0pt}{\isacharbackquote}{\kern0pt}\ {\isacharbraceleft}{\kern0pt}x{\isacharbraceright}{\kern0pt}\ {\isasymtimes}\ {\isacharbraceleft}{\kern0pt}a{\isacharbraceright}{\kern0pt}\ {\isacharequal}{\kern0pt}\ prel{\isacharparenleft}{\kern0pt}r{\isacharcomma}{\kern0pt}\ p{\isacharparenright}{\kern0pt}\ {\isacharminus}{\kern0pt}{\isacharbackquote}{\kern0pt}{\isacharbackquote}{\kern0pt}\ {\isacharbraceleft}{\kern0pt}{\isasymlangle}x{\isacharcomma}{\kern0pt}\ a{\isasymrangle}{\isacharbraceright}{\kern0pt}{\isachardoublequoteclose}\ \isanewline
\ \ \ \ \isacommand{proof}\isamarkupfalse%
{\isacharparenleft}{\kern0pt}rule\ equality{\isacharunderscore}{\kern0pt}iffI{\isacharcomma}{\kern0pt}\ rule\ iffI{\isacharparenright}{\kern0pt}\isanewline
\ \ \ \ \ \ \isacommand{fix}\isamarkupfalse%
\ v\ \isacommand{assume}\isamarkupfalse%
\ {\isachardoublequoteopen}v\ {\isasymin}\ r\ {\isacharminus}{\kern0pt}{\isacharbackquote}{\kern0pt}{\isacharbackquote}{\kern0pt}\ {\isacharbraceleft}{\kern0pt}x{\isacharbraceright}{\kern0pt}\ {\isasymtimes}\ {\isacharbraceleft}{\kern0pt}a{\isacharbraceright}{\kern0pt}{\isachardoublequoteclose}\ \isanewline
\ \ \ \ \ \ \isacommand{then}\isamarkupfalse%
\ \isacommand{obtain}\isamarkupfalse%
\ u\ \isakeyword{where}\ uH\ {\isacharcolon}{\kern0pt}\ {\isachardoublequoteopen}v\ {\isacharequal}{\kern0pt}\ {\isacharless}{\kern0pt}u{\isacharcomma}{\kern0pt}\ a{\isachargreater}{\kern0pt}{\isachardoublequoteclose}\ {\isachardoublequoteopen}u\ {\isasymin}\ r\ {\isacharminus}{\kern0pt}{\isacharbackquote}{\kern0pt}{\isacharbackquote}{\kern0pt}\ {\isacharbraceleft}{\kern0pt}x{\isacharbraceright}{\kern0pt}{\isachardoublequoteclose}\ \isacommand{by}\isamarkupfalse%
\ auto\ \isanewline
\ \ \ \ \ \ \isacommand{then}\isamarkupfalse%
\ \isacommand{have}\isamarkupfalse%
\ {\isachardoublequoteopen}{\isacharless}{\kern0pt}u{\isacharcomma}{\kern0pt}\ x{\isachargreater}{\kern0pt}\ {\isasymin}\ r{\isachardoublequoteclose}\ \isacommand{by}\isamarkupfalse%
\ auto\ \isanewline
\ \ \ \ \ \ \isacommand{then}\isamarkupfalse%
\ \isacommand{have}\isamarkupfalse%
\ {\isachardoublequoteopen}{\isacharless}{\kern0pt}{\isacharless}{\kern0pt}u{\isacharcomma}{\kern0pt}\ a{\isachargreater}{\kern0pt}{\isacharcomma}{\kern0pt}\ {\isacharless}{\kern0pt}x{\isacharcomma}{\kern0pt}\ a{\isachargreater}{\kern0pt}{\isachargreater}{\kern0pt}\ {\isasymin}\ prel{\isacharparenleft}{\kern0pt}r{\isacharcomma}{\kern0pt}\ p{\isacharparenright}{\kern0pt}{\isachardoublequoteclose}\ \isacommand{by}\isamarkupfalse%
{\isacharparenleft}{\kern0pt}rule\ prelI{\isacharcomma}{\kern0pt}\ simp\ add{\isacharcolon}{\kern0pt}assms{\isacharparenright}{\kern0pt}\ \isanewline
\ \ \ \ \ \ \isacommand{then}\isamarkupfalse%
\ \isacommand{show}\isamarkupfalse%
\ {\isachardoublequoteopen}v\ {\isasymin}\ prel{\isacharparenleft}{\kern0pt}r{\isacharcomma}{\kern0pt}\ p{\isacharparenright}{\kern0pt}\ {\isacharminus}{\kern0pt}{\isacharbackquote}{\kern0pt}{\isacharbackquote}{\kern0pt}\ {\isacharbraceleft}{\kern0pt}{\isasymlangle}x{\isacharcomma}{\kern0pt}\ a{\isasymrangle}{\isacharbraceright}{\kern0pt}{\isachardoublequoteclose}\ \isacommand{using}\isamarkupfalse%
\ uH\ \isacommand{by}\isamarkupfalse%
\ auto\ \isanewline
\ \ \ \ \isacommand{next}\isamarkupfalse%
\ \isanewline
\ \ \ \ \ \ \isacommand{fix}\isamarkupfalse%
\ v\ \isacommand{assume}\isamarkupfalse%
\ {\isachardoublequoteopen}v\ {\isasymin}\ prel{\isacharparenleft}{\kern0pt}r{\isacharcomma}{\kern0pt}\ p{\isacharparenright}{\kern0pt}\ {\isacharminus}{\kern0pt}{\isacharbackquote}{\kern0pt}{\isacharbackquote}{\kern0pt}\ {\isacharbraceleft}{\kern0pt}{\isasymlangle}x{\isacharcomma}{\kern0pt}\ a{\isasymrangle}{\isacharbraceright}{\kern0pt}{\isachardoublequoteclose}\ \isanewline
\ \ \ \ \ \ \isacommand{then}\isamarkupfalse%
\ \isacommand{have}\isamarkupfalse%
\ {\isachardoublequoteopen}{\isacharless}{\kern0pt}v{\isacharcomma}{\kern0pt}\ {\isacharless}{\kern0pt}x{\isacharcomma}{\kern0pt}\ a{\isachargreater}{\kern0pt}{\isachargreater}{\kern0pt}\ {\isasymin}\ prel{\isacharparenleft}{\kern0pt}r{\isacharcomma}{\kern0pt}\ p{\isacharparenright}{\kern0pt}{\isachardoublequoteclose}\ \isacommand{by}\isamarkupfalse%
\ auto\ \isanewline
\ \ \ \ \ \ \isacommand{then}\isamarkupfalse%
\ \isacommand{obtain}\isamarkupfalse%
\ u\ \isakeyword{where}\ uH\ {\isacharcolon}{\kern0pt}\ {\isachardoublequoteopen}v\ {\isacharequal}{\kern0pt}\ {\isacharless}{\kern0pt}u{\isacharcomma}{\kern0pt}\ a{\isachargreater}{\kern0pt}{\isachardoublequoteclose}\ {\isachardoublequoteopen}{\isacharless}{\kern0pt}u{\isacharcomma}{\kern0pt}\ x{\isachargreater}{\kern0pt}\ {\isasymin}\ r{\isachardoublequoteclose}\ \isacommand{unfolding}\isamarkupfalse%
\ prel{\isacharunderscore}{\kern0pt}def\ \isacommand{by}\isamarkupfalse%
\ auto\ \isanewline
\ \ \ \ \ \ \isacommand{then}\isamarkupfalse%
\ \isacommand{have}\isamarkupfalse%
\ {\isachardoublequoteopen}u\ {\isasymin}\ r\ {\isacharminus}{\kern0pt}{\isacharbackquote}{\kern0pt}{\isacharbackquote}{\kern0pt}\ {\isacharbraceleft}{\kern0pt}x{\isacharbraceright}{\kern0pt}{\isachardoublequoteclose}\ \isacommand{by}\isamarkupfalse%
\ auto\ \isanewline
\ \ \ \ \ \ \isacommand{then}\isamarkupfalse%
\ \isacommand{show}\isamarkupfalse%
\ {\isachardoublequoteopen}v\ {\isasymin}\ r\ {\isacharminus}{\kern0pt}{\isacharbackquote}{\kern0pt}{\isacharbackquote}{\kern0pt}\ {\isacharbraceleft}{\kern0pt}x{\isacharbraceright}{\kern0pt}\ {\isasymtimes}\ {\isacharbraceleft}{\kern0pt}a{\isacharbraceright}{\kern0pt}{\isachardoublequoteclose}\ \isacommand{using}\isamarkupfalse%
\ uH\ \isacommand{by}\isamarkupfalse%
\ auto\isanewline
\ \ \ \ \isacommand{qed}\isamarkupfalse%
\isanewline
\isanewline
\ \ \ \ \isacommand{have}\isamarkupfalse%
\ {\isachardoublequoteopen}H{\isacharparenleft}{\kern0pt}x{\isacharcomma}{\kern0pt}\ the{\isacharunderscore}{\kern0pt}recfun{\isacharparenleft}{\kern0pt}r{\isacharcomma}{\kern0pt}\ x{\isacharcomma}{\kern0pt}\ H{\isacharparenright}{\kern0pt}{\isacharparenright}{\kern0pt}\ {\isacharequal}{\kern0pt}\ G{\isacharparenleft}{\kern0pt}{\isacharless}{\kern0pt}x{\isacharcomma}{\kern0pt}\ a{\isachargreater}{\kern0pt}{\isacharcomma}{\kern0pt}\ the{\isacharunderscore}{\kern0pt}recfun{\isacharparenleft}{\kern0pt}prel{\isacharparenleft}{\kern0pt}r{\isacharcomma}{\kern0pt}\ p{\isacharparenright}{\kern0pt}{\isacharcomma}{\kern0pt}\ {\isacharless}{\kern0pt}x{\isacharcomma}{\kern0pt}\ a{\isachargreater}{\kern0pt}{\isacharcomma}{\kern0pt}\ G{\isacharparenright}{\kern0pt}{\isacharparenright}{\kern0pt}{\isachardoublequoteclose}\ \isanewline
\ \ \ \ \ \ \isacommand{apply}\isamarkupfalse%
{\isacharparenleft}{\kern0pt}rule\ HGeq{\isacharparenright}{\kern0pt}\isanewline
\ \ \ \ \ \ \ \ \ \ \isacommand{apply}\isamarkupfalse%
{\isacharparenleft}{\kern0pt}rule\ Pi{\isacharunderscore}{\kern0pt}memberI{\isacharparenright}{\kern0pt}\isanewline
\ \ \ \ \ \ \ \ \ \ \ \ \ \isacommand{apply}\isamarkupfalse%
{\isacharparenleft}{\kern0pt}subst\ eq{\isadigit{1}}{\isacharcomma}{\kern0pt}\ rule\ relation{\isacharunderscore}{\kern0pt}lam{\isacharparenright}{\kern0pt}\isanewline
\ \ \ \ \ \ \ \ \ \ \ \ \isacommand{apply}\isamarkupfalse%
{\isacharparenleft}{\kern0pt}subst\ eq{\isadigit{1}}{\isacharcomma}{\kern0pt}\ rule\ function{\isacharunderscore}{\kern0pt}lam{\isacharparenright}{\kern0pt}\isanewline
\ \ \ \ \ \ \ \ \ \ \ \isacommand{apply}\isamarkupfalse%
{\isacharparenleft}{\kern0pt}subst\ eq{\isadigit{1}}{\isacharcomma}{\kern0pt}\ subst\ domain{\isacharunderscore}{\kern0pt}lam{\isacharcomma}{\kern0pt}\ simp{\isacharparenright}{\kern0pt}\isanewline
\ \ \ \ \ \ \ \ \ \ \isacommand{apply}\isamarkupfalse%
{\isacharparenleft}{\kern0pt}rule\ rangesubset{\isacharparenright}{\kern0pt}\isanewline
\ \ \ \ \ \ \ \ \ \isacommand{apply}\isamarkupfalse%
{\isacharparenleft}{\kern0pt}rule\ Pi{\isacharunderscore}{\kern0pt}memberI{\isacharparenright}{\kern0pt}\isanewline
\ \ \ \ \ \ \ \ \ \ \ \ \isacommand{apply}\isamarkupfalse%
{\isacharparenleft}{\kern0pt}subst\ eq{\isadigit{2}}{\isacharcomma}{\kern0pt}\ rule\ relation{\isacharunderscore}{\kern0pt}lam{\isacharparenright}{\kern0pt}\isanewline
\ \ \ \ \ \ \ \ \ \ \ \isacommand{apply}\isamarkupfalse%
{\isacharparenleft}{\kern0pt}subst\ eq{\isadigit{2}}{\isacharcomma}{\kern0pt}\ rule\ function{\isacharunderscore}{\kern0pt}lam{\isacharparenright}{\kern0pt}\isanewline
\ \ \ \ \ \ \ \ \ \ \isacommand{apply}\isamarkupfalse%
{\isacharparenleft}{\kern0pt}subst\ eq{\isadigit{2}}{\isacharcomma}{\kern0pt}\ subst\ domain{\isacharunderscore}{\kern0pt}lam{\isacharcomma}{\kern0pt}\ rule\ vimageeq{\isacharparenright}{\kern0pt}\isanewline
\ \ \ \ \ \ \ \ \ \isacommand{apply}\isamarkupfalse%
{\isacharparenleft}{\kern0pt}rule\ rangesubset{\isadigit{2}}{\isacharparenright}{\kern0pt}\isanewline
\ \ \ \ \ \ \ \ \isacommand{apply}\isamarkupfalse%
{\isacharparenleft}{\kern0pt}rule\ recfun{\isadigit{2}}inM{\isacharparenright}{\kern0pt}\isanewline
\ \ \ \ \ \ \isacommand{using}\isamarkupfalse%
\ assms\ assms{\isadigit{1}}\ app{\isacharunderscore}{\kern0pt}eq\ \isanewline
\ \ \ \ \ \ \isacommand{by}\isamarkupfalse%
\ auto\isanewline
\ \ \ \ \isacommand{then}\isamarkupfalse%
\ \isacommand{show}\isamarkupfalse%
\ {\isachardoublequoteopen}wftrec{\isacharparenleft}{\kern0pt}r{\isacharcomma}{\kern0pt}\ x{\isacharcomma}{\kern0pt}\ H{\isacharparenright}{\kern0pt}\ {\isacharequal}{\kern0pt}\ wftrec{\isacharparenleft}{\kern0pt}prel{\isacharparenleft}{\kern0pt}r{\isacharcomma}{\kern0pt}\ p{\isacharparenright}{\kern0pt}{\isacharcomma}{\kern0pt}\ {\isasymlangle}x{\isacharcomma}{\kern0pt}\ a{\isasymrangle}{\isacharcomma}{\kern0pt}\ G{\isacharparenright}{\kern0pt}{\isachardoublequoteclose}\isanewline
\ \ \ \ \ \ \isacommand{unfolding}\isamarkupfalse%
\ wftrec{\isacharunderscore}{\kern0pt}def\ \isanewline
\ \ \ \ \ \ \isacommand{by}\isamarkupfalse%
\ simp\isanewline
\ \ \isacommand{qed}\isamarkupfalse%
\isanewline
\isanewline
\ \ \isacommand{then}\isamarkupfalse%
\ \isacommand{show}\isamarkupfalse%
\ {\isacharquery}{\kern0pt}thesis\ \isacommand{using}\isamarkupfalse%
\ assms\ \isacommand{by}\isamarkupfalse%
\ auto\isanewline
\isacommand{qed}\isamarkupfalse%
%
\endisatagproof
{\isafoldproof}%
%
\isadelimproof
\isanewline
%
\endisadelimproof
\isanewline
\isacommand{lemma}\isamarkupfalse%
\ wftrec{\isacharunderscore}{\kern0pt}prel{\isacharunderscore}{\kern0pt}preds{\isacharunderscore}{\kern0pt}rel{\isacharunderscore}{\kern0pt}eq\ {\isacharcolon}{\kern0pt}\ \isanewline
\ \ \isakeyword{fixes}\ R\ Rfm\ G\ H\ x\ a\isanewline
\ \ \isakeyword{assumes}\ {\isachardoublequoteopen}x\ {\isasymin}\ M{\isachardoublequoteclose}\ {\isachardoublequoteopen}a\ {\isasymin}\ M{\isachardoublequoteclose}\ {\isachardoublequoteopen}x\ {\isasymin}\ field{\isacharparenleft}{\kern0pt}Rrel{\isacharparenleft}{\kern0pt}R{\isacharcomma}{\kern0pt}\ M{\isacharparenright}{\kern0pt}{\isacharparenright}{\kern0pt}{\isachardoublequoteclose}\ {\isachardoublequoteopen}wf{\isacharparenleft}{\kern0pt}Rrel{\isacharparenleft}{\kern0pt}R{\isacharcomma}{\kern0pt}\ M{\isacharparenright}{\kern0pt}{\isacharparenright}{\kern0pt}{\isachardoublequoteclose}\ {\isachardoublequoteopen}trans{\isacharparenleft}{\kern0pt}Rrel{\isacharparenleft}{\kern0pt}R{\isacharcomma}{\kern0pt}\ M{\isacharparenright}{\kern0pt}{\isacharparenright}{\kern0pt}{\isachardoublequoteclose}\isanewline
\ \ \ \ \ \ \ \ \ \ {\isachardoublequoteopen}Relation{\isacharunderscore}{\kern0pt}fm{\isacharparenleft}{\kern0pt}R{\isacharcomma}{\kern0pt}\ Rfm{\isacharparenright}{\kern0pt}{\isachardoublequoteclose}\ {\isachardoublequoteopen}preds{\isacharparenleft}{\kern0pt}R{\isacharcomma}{\kern0pt}\ x{\isacharparenright}{\kern0pt}\ {\isasymin}\ M{\isachardoublequoteclose}\ \ {\isachardoublequoteopen}Gfm\ {\isasymin}\ formula{\isachardoublequoteclose}\ {\isachardoublequoteopen}arity{\isacharparenleft}{\kern0pt}Gfm{\isacharparenright}{\kern0pt}\ {\isasymle}\ {\isadigit{3}}{\isachardoublequoteclose}\ \isanewline
\ \ \ \ \ \ \ \ \ \ {\isachardoublequoteopen}{\isasymAnd}x\ g{\isachardot}{\kern0pt}\ x\ {\isasymin}\ M\ {\isasymLongrightarrow}\ g\ {\isasymin}\ M\ {\isasymLongrightarrow}\ function{\isacharparenleft}{\kern0pt}g{\isacharparenright}{\kern0pt}\ {\isasymLongrightarrow}\ G{\isacharparenleft}{\kern0pt}x{\isacharcomma}{\kern0pt}\ g{\isacharparenright}{\kern0pt}\ {\isasymin}\ M{\isachardoublequoteclose}\ \isanewline
\ \ \ \ \ \ \ \ \ \ {\isachardoublequoteopen}{\isasymAnd}a{\isadigit{0}}\ a{\isadigit{1}}\ a{\isadigit{2}}\ env{\isachardot}{\kern0pt}\ a{\isadigit{0}}\ {\isasymin}\ M\ {\isasymLongrightarrow}\ a{\isadigit{1}}\ {\isasymin}\ M\ {\isasymLongrightarrow}\ a{\isadigit{2}}\ {\isasymin}\ M\ {\isasymLongrightarrow}\ env\ {\isasymin}\ list{\isacharparenleft}{\kern0pt}M{\isacharparenright}{\kern0pt}\ {\isasymLongrightarrow}\ a{\isadigit{0}}\ {\isacharequal}{\kern0pt}\ G{\isacharparenleft}{\kern0pt}a{\isadigit{2}}{\isacharcomma}{\kern0pt}\ a{\isadigit{1}}{\isacharparenright}{\kern0pt}\ {\isasymlongleftrightarrow}\ sats{\isacharparenleft}{\kern0pt}M{\isacharcomma}{\kern0pt}\ Gfm{\isacharcomma}{\kern0pt}\ {\isacharbrackleft}{\kern0pt}a{\isadigit{0}}{\isacharcomma}{\kern0pt}\ a{\isadigit{1}}{\isacharcomma}{\kern0pt}\ a{\isadigit{2}}{\isacharbrackright}{\kern0pt}\ {\isacharat}{\kern0pt}\ env{\isacharparenright}{\kern0pt}{\isachardoublequoteclose}\ \ \isanewline
\ \ \isakeyword{and}\ HGeq{\isacharcolon}{\kern0pt}\ {\isachardoublequoteopen}{\isasymAnd}h\ g\ x{\isachardot}{\kern0pt}\ h\ {\isasymin}\ Rrel{\isacharparenleft}{\kern0pt}R{\isacharcomma}{\kern0pt}\ M{\isacharparenright}{\kern0pt}\ {\isacharminus}{\kern0pt}{\isacharbackquote}{\kern0pt}{\isacharbackquote}{\kern0pt}\ {\isacharbraceleft}{\kern0pt}x{\isacharbraceright}{\kern0pt}\ {\isasymrightarrow}\ M\ {\isasymLongrightarrow}\ g\ {\isasymin}\ {\isacharparenleft}{\kern0pt}Rrel{\isacharparenleft}{\kern0pt}R{\isacharcomma}{\kern0pt}\ M{\isacharparenright}{\kern0pt}\ {\isacharminus}{\kern0pt}{\isacharbackquote}{\kern0pt}{\isacharbackquote}{\kern0pt}\ {\isacharbraceleft}{\kern0pt}x{\isacharbraceright}{\kern0pt}\ {\isasymtimes}\ {\isacharbraceleft}{\kern0pt}a{\isacharbraceright}{\kern0pt}{\isacharparenright}{\kern0pt}\ {\isasymrightarrow}\ M\ {\isasymLongrightarrow}\ g\ {\isasymin}\ M\ \ \isanewline
\ \ \ \ \ \ \ \ \ \ \ \ \ \ \ {\isasymLongrightarrow}\ x\ {\isasymin}\ field{\isacharparenleft}{\kern0pt}Rrel{\isacharparenleft}{\kern0pt}R{\isacharcomma}{\kern0pt}\ M{\isacharparenright}{\kern0pt}{\isacharparenright}{\kern0pt}\ {\isasymLongrightarrow}\ {\isacharparenleft}{\kern0pt}{\isasymAnd}y{\isachardot}{\kern0pt}\ y\ {\isasymin}\ Rrel{\isacharparenleft}{\kern0pt}R{\isacharcomma}{\kern0pt}\ M{\isacharparenright}{\kern0pt}\ {\isacharminus}{\kern0pt}{\isacharbackquote}{\kern0pt}{\isacharbackquote}{\kern0pt}\ {\isacharbraceleft}{\kern0pt}x{\isacharbraceright}{\kern0pt}\ {\isasymLongrightarrow}\ h{\isacharbackquote}{\kern0pt}y\ {\isacharequal}{\kern0pt}\ g{\isacharbackquote}{\kern0pt}{\isacharless}{\kern0pt}y{\isacharcomma}{\kern0pt}\ a{\isachargreater}{\kern0pt}{\isacharparenright}{\kern0pt}\ {\isasymLongrightarrow}\ H{\isacharparenleft}{\kern0pt}x{\isacharcomma}{\kern0pt}\ h{\isacharparenright}{\kern0pt}\ {\isacharequal}{\kern0pt}\ G{\isacharparenleft}{\kern0pt}{\isacharless}{\kern0pt}x{\isacharcomma}{\kern0pt}\ a{\isachargreater}{\kern0pt}{\isacharcomma}{\kern0pt}\ g{\isacharparenright}{\kern0pt}{\isachardoublequoteclose}\ \ \isanewline
\isanewline
\ \ \isakeyword{shows}\ {\isachardoublequoteopen}wftrec{\isacharparenleft}{\kern0pt}prel{\isacharparenleft}{\kern0pt}preds{\isacharunderscore}{\kern0pt}rel{\isacharparenleft}{\kern0pt}R{\isacharcomma}{\kern0pt}\ x{\isacharparenright}{\kern0pt}{\isacharcomma}{\kern0pt}\ {\isacharbraceleft}{\kern0pt}a{\isacharbraceright}{\kern0pt}{\isacharparenright}{\kern0pt}{\isacharcomma}{\kern0pt}\ {\isacharless}{\kern0pt}x{\isacharcomma}{\kern0pt}\ a{\isachargreater}{\kern0pt}{\isacharcomma}{\kern0pt}\ G{\isacharparenright}{\kern0pt}\ {\isacharequal}{\kern0pt}\ wftrec{\isacharparenleft}{\kern0pt}Rrel{\isacharparenleft}{\kern0pt}R{\isacharcomma}{\kern0pt}\ M{\isacharparenright}{\kern0pt}{\isacharcomma}{\kern0pt}\ x{\isacharcomma}{\kern0pt}\ H{\isacharparenright}{\kern0pt}{\isachardoublequoteclose}\ \isanewline
%
\isadelimproof
%
\endisadelimproof
%
\isatagproof
\isacommand{proof}\isamarkupfalse%
\ {\isacharminus}{\kern0pt}\ \isanewline
\isanewline
\ \ \isacommand{have}\isamarkupfalse%
\ E{\isadigit{1}}{\isacharcolon}{\kern0pt}\ {\isachardoublequoteopen}wftrec{\isacharparenleft}{\kern0pt}prel{\isacharparenleft}{\kern0pt}preds{\isacharunderscore}{\kern0pt}rel{\isacharparenleft}{\kern0pt}R{\isacharcomma}{\kern0pt}\ x{\isacharparenright}{\kern0pt}{\isacharcomma}{\kern0pt}\ {\isacharbraceleft}{\kern0pt}a{\isacharbraceright}{\kern0pt}{\isacharparenright}{\kern0pt}{\isacharcomma}{\kern0pt}\ {\isacharless}{\kern0pt}x{\isacharcomma}{\kern0pt}\ a{\isachargreater}{\kern0pt}{\isacharcomma}{\kern0pt}\ G{\isacharparenright}{\kern0pt}\ {\isacharequal}{\kern0pt}\ wftrec{\isacharparenleft}{\kern0pt}preds{\isacharunderscore}{\kern0pt}rel{\isacharparenleft}{\kern0pt}R{\isacharcomma}{\kern0pt}\ x{\isacharparenright}{\kern0pt}{\isacharcomma}{\kern0pt}\ x{\isacharcomma}{\kern0pt}\ H{\isacharparenright}{\kern0pt}{\isachardoublequoteclose}\ \isanewline
\ \ \isacommand{proof}\isamarkupfalse%
{\isacharparenleft}{\kern0pt}cases\ {\isachardoublequoteopen}preds{\isacharunderscore}{\kern0pt}rel{\isacharparenleft}{\kern0pt}R{\isacharcomma}{\kern0pt}\ x{\isacharparenright}{\kern0pt}\ {\isacharminus}{\kern0pt}{\isacharbackquote}{\kern0pt}{\isacharbackquote}{\kern0pt}\ {\isacharbraceleft}{\kern0pt}x{\isacharbraceright}{\kern0pt}\ {\isacharequal}{\kern0pt}\ {\isadigit{0}}{\isachardoublequoteclose}{\isacharparenright}{\kern0pt}\isanewline
\ \ \ \ \isacommand{case}\isamarkupfalse%
\ True\isanewline
\ \ \ \ \isacommand{then}\isamarkupfalse%
\ \isacommand{have}\isamarkupfalse%
\ eq{\isadigit{0}}\ {\isacharcolon}{\kern0pt}\ {\isachardoublequoteopen}preds{\isacharunderscore}{\kern0pt}rel{\isacharparenleft}{\kern0pt}R{\isacharcomma}{\kern0pt}\ x{\isacharparenright}{\kern0pt}\ {\isacharminus}{\kern0pt}{\isacharbackquote}{\kern0pt}{\isacharbackquote}{\kern0pt}\ {\isacharbraceleft}{\kern0pt}x{\isacharbraceright}{\kern0pt}\ {\isacharequal}{\kern0pt}\ {\isadigit{0}}{\isachardoublequoteclose}\ \isacommand{by}\isamarkupfalse%
\ auto\ \isanewline
\ \ \ \ \isacommand{then}\isamarkupfalse%
\ \isacommand{have}\isamarkupfalse%
\ eqH\ {\isacharcolon}{\kern0pt}\ {\isachardoublequoteopen}wftrec{\isacharparenleft}{\kern0pt}preds{\isacharunderscore}{\kern0pt}rel{\isacharparenleft}{\kern0pt}R{\isacharcomma}{\kern0pt}\ x{\isacharparenright}{\kern0pt}{\isacharcomma}{\kern0pt}\ x{\isacharcomma}{\kern0pt}\ H{\isacharparenright}{\kern0pt}\ {\isacharequal}{\kern0pt}\ H{\isacharparenleft}{\kern0pt}x{\isacharcomma}{\kern0pt}\ {\isadigit{0}}{\isacharparenright}{\kern0pt}{\isachardoublequoteclose}\ \isanewline
\ \ \ \ \ \ \isacommand{apply}\isamarkupfalse%
{\isacharparenleft}{\kern0pt}subst\ wftrec{\isacharparenright}{\kern0pt}\isanewline
\ \ \ \ \ \ \ \ \isacommand{apply}\isamarkupfalse%
{\isacharparenleft}{\kern0pt}rule\ wf{\isacharunderscore}{\kern0pt}preds{\isacharunderscore}{\kern0pt}rel{\isacharcomma}{\kern0pt}\ simp\ add{\isacharcolon}{\kern0pt}assms{\isacharcomma}{\kern0pt}\ simp\ add{\isacharcolon}{\kern0pt}assms{\isacharparenright}{\kern0pt}\isanewline
\ \ \ \ \ \ \ \isacommand{apply}\isamarkupfalse%
{\isacharparenleft}{\kern0pt}rule\ trans{\isacharunderscore}{\kern0pt}preds{\isacharunderscore}{\kern0pt}rel{\isacharcomma}{\kern0pt}\ simp\ add{\isacharcolon}{\kern0pt}assms{\isacharcomma}{\kern0pt}\ simp\ add{\isacharcolon}{\kern0pt}assms{\isacharparenright}{\kern0pt}\isanewline
\ \ \ \ \ \ \isacommand{by}\isamarkupfalse%
\ simp\isanewline
\ \ \ \ \isacommand{have}\isamarkupfalse%
\ eq{\isadigit{0}}{\isacharprime}{\kern0pt}{\isacharcolon}{\kern0pt}\ {\isachardoublequoteopen}prel{\isacharparenleft}{\kern0pt}preds{\isacharunderscore}{\kern0pt}rel{\isacharparenleft}{\kern0pt}R{\isacharcomma}{\kern0pt}\ x{\isacharparenright}{\kern0pt}{\isacharcomma}{\kern0pt}\ {\isacharbraceleft}{\kern0pt}a{\isacharbraceright}{\kern0pt}{\isacharparenright}{\kern0pt}\ {\isacharminus}{\kern0pt}{\isacharbackquote}{\kern0pt}{\isacharbackquote}{\kern0pt}\ {\isacharbraceleft}{\kern0pt}{\isacharless}{\kern0pt}x{\isacharcomma}{\kern0pt}\ a{\isachargreater}{\kern0pt}{\isacharbraceright}{\kern0pt}\ {\isacharequal}{\kern0pt}\ {\isadigit{0}}{\isachardoublequoteclose}\ \isanewline
\ \ \ \ \isacommand{proof}\isamarkupfalse%
{\isacharparenleft}{\kern0pt}rule\ ccontr{\isacharparenright}{\kern0pt}\isanewline
\ \ \ \ \ \ \isacommand{assume}\isamarkupfalse%
\ {\isachardoublequoteopen}prel{\isacharparenleft}{\kern0pt}preds{\isacharunderscore}{\kern0pt}rel{\isacharparenleft}{\kern0pt}R{\isacharcomma}{\kern0pt}\ x{\isacharparenright}{\kern0pt}{\isacharcomma}{\kern0pt}\ {\isacharbraceleft}{\kern0pt}a{\isacharbraceright}{\kern0pt}{\isacharparenright}{\kern0pt}\ {\isacharminus}{\kern0pt}{\isacharbackquote}{\kern0pt}{\isacharbackquote}{\kern0pt}\ {\isacharbraceleft}{\kern0pt}{\isasymlangle}x{\isacharcomma}{\kern0pt}\ a{\isasymrangle}{\isacharbraceright}{\kern0pt}\ {\isasymnoteq}\ {\isadigit{0}}{\isachardoublequoteclose}\ \isanewline
\ \ \ \ \ \ \isacommand{then}\isamarkupfalse%
\ \isacommand{obtain}\isamarkupfalse%
\ y\ \isakeyword{where}\ {\isachardoublequoteopen}{\isacharless}{\kern0pt}y{\isacharcomma}{\kern0pt}\ a{\isachargreater}{\kern0pt}\ {\isasymin}\ prel{\isacharparenleft}{\kern0pt}preds{\isacharunderscore}{\kern0pt}rel{\isacharparenleft}{\kern0pt}R{\isacharcomma}{\kern0pt}\ x{\isacharparenright}{\kern0pt}{\isacharcomma}{\kern0pt}\ {\isacharbraceleft}{\kern0pt}a{\isacharbraceright}{\kern0pt}{\isacharparenright}{\kern0pt}\ {\isacharminus}{\kern0pt}{\isacharbackquote}{\kern0pt}{\isacharbackquote}{\kern0pt}\ {\isacharbraceleft}{\kern0pt}{\isasymlangle}x{\isacharcomma}{\kern0pt}\ a{\isasymrangle}{\isacharbraceright}{\kern0pt}{\isachardoublequoteclose}\ \isacommand{unfolding}\isamarkupfalse%
\ prel{\isacharunderscore}{\kern0pt}def\ \isacommand{by}\isamarkupfalse%
\ blast\isanewline
\ \ \ \ \ \ \isacommand{then}\isamarkupfalse%
\ \isacommand{have}\isamarkupfalse%
\ {\isachardoublequoteopen}{\isacharless}{\kern0pt}y{\isacharcomma}{\kern0pt}\ x{\isachargreater}{\kern0pt}\ {\isasymin}\ preds{\isacharunderscore}{\kern0pt}rel{\isacharparenleft}{\kern0pt}R{\isacharcomma}{\kern0pt}\ x{\isacharparenright}{\kern0pt}{\isachardoublequoteclose}\ \isacommand{unfolding}\isamarkupfalse%
\ prel{\isacharunderscore}{\kern0pt}def\ \isacommand{by}\isamarkupfalse%
\ auto\ \isanewline
\ \ \ \ \ \ \isacommand{then}\isamarkupfalse%
\ \isacommand{have}\isamarkupfalse%
\ {\isachardoublequoteopen}y\ {\isasymin}\ preds{\isacharunderscore}{\kern0pt}rel{\isacharparenleft}{\kern0pt}R{\isacharcomma}{\kern0pt}\ x{\isacharparenright}{\kern0pt}\ {\isacharminus}{\kern0pt}{\isacharbackquote}{\kern0pt}{\isacharbackquote}{\kern0pt}\ {\isacharbraceleft}{\kern0pt}x{\isacharbraceright}{\kern0pt}{\isachardoublequoteclose}\ \isacommand{by}\isamarkupfalse%
\ auto\ \isanewline
\ \ \ \ \ \ \isacommand{then}\isamarkupfalse%
\ \isacommand{have}\isamarkupfalse%
\ {\isachardoublequoteopen}preds{\isacharunderscore}{\kern0pt}rel{\isacharparenleft}{\kern0pt}R{\isacharcomma}{\kern0pt}\ x{\isacharparenright}{\kern0pt}\ {\isacharminus}{\kern0pt}{\isacharbackquote}{\kern0pt}{\isacharbackquote}{\kern0pt}\ {\isacharbraceleft}{\kern0pt}x{\isacharbraceright}{\kern0pt}\ {\isasymnoteq}\ {\isadigit{0}}{\isachardoublequoteclose}\ \isacommand{by}\isamarkupfalse%
\ auto\ \isanewline
\ \ \ \ \ \ \isacommand{then}\isamarkupfalse%
\ \isacommand{show}\isamarkupfalse%
\ {\isachardoublequoteopen}False{\isachardoublequoteclose}\ \isacommand{using}\isamarkupfalse%
\ eq{\isadigit{0}}\ \isacommand{by}\isamarkupfalse%
\ auto\isanewline
\ \ \ \ \isacommand{qed}\isamarkupfalse%
\isanewline
\ \ \ \ \isacommand{then}\isamarkupfalse%
\ \isacommand{have}\isamarkupfalse%
\ eqG\ {\isacharcolon}{\kern0pt}\ {\isachardoublequoteopen}wftrec{\isacharparenleft}{\kern0pt}prel{\isacharparenleft}{\kern0pt}preds{\isacharunderscore}{\kern0pt}rel{\isacharparenleft}{\kern0pt}R{\isacharcomma}{\kern0pt}\ x{\isacharparenright}{\kern0pt}{\isacharcomma}{\kern0pt}\ {\isacharbraceleft}{\kern0pt}a{\isacharbraceright}{\kern0pt}{\isacharparenright}{\kern0pt}{\isacharcomma}{\kern0pt}\ {\isacharless}{\kern0pt}x{\isacharcomma}{\kern0pt}\ a{\isachargreater}{\kern0pt}{\isacharcomma}{\kern0pt}\ G{\isacharparenright}{\kern0pt}\ {\isacharequal}{\kern0pt}\ G{\isacharparenleft}{\kern0pt}{\isacharless}{\kern0pt}x{\isacharcomma}{\kern0pt}\ a{\isachargreater}{\kern0pt}{\isacharcomma}{\kern0pt}\ {\isadigit{0}}{\isacharparenright}{\kern0pt}{\isachardoublequoteclose}\ \isanewline
\ \ \ \ \ \ \isacommand{apply}\isamarkupfalse%
{\isacharparenleft}{\kern0pt}subst\ wftrec{\isacharparenright}{\kern0pt}\isanewline
\ \ \ \ \ \ \ \ \isacommand{apply}\isamarkupfalse%
{\isacharparenleft}{\kern0pt}rule\ wf{\isacharunderscore}{\kern0pt}prel{\isacharcomma}{\kern0pt}\ rule\ wf{\isacharunderscore}{\kern0pt}preds{\isacharunderscore}{\kern0pt}rel{\isacharcomma}{\kern0pt}\ simp\ add{\isacharcolon}{\kern0pt}assms{\isacharcomma}{\kern0pt}\ simp\ add{\isacharcolon}{\kern0pt}assms{\isacharparenright}{\kern0pt}\isanewline
\ \ \ \ \ \ \ \isacommand{apply}\isamarkupfalse%
{\isacharparenleft}{\kern0pt}rule\ prel{\isacharunderscore}{\kern0pt}trans{\isacharcomma}{\kern0pt}\ rule\ trans{\isacharunderscore}{\kern0pt}preds{\isacharunderscore}{\kern0pt}rel{\isacharcomma}{\kern0pt}\ simp\ add{\isacharcolon}{\kern0pt}assms{\isacharcomma}{\kern0pt}\ simp\ add{\isacharcolon}{\kern0pt}assms{\isacharparenright}{\kern0pt}\isanewline
\ \ \ \ \ \ \isacommand{by}\isamarkupfalse%
{\isacharparenleft}{\kern0pt}subst\ eq{\isadigit{0}}{\isacharprime}{\kern0pt}{\isacharcomma}{\kern0pt}\ simp{\isacharparenright}{\kern0pt}\isanewline
\isanewline
\ \ \ \ \isacommand{have}\isamarkupfalse%
\ eq{\isadigit{0}}{\isacharprime}{\kern0pt}{\isacharprime}{\kern0pt}\ {\isacharcolon}{\kern0pt}\ {\isachardoublequoteopen}Rrel{\isacharparenleft}{\kern0pt}R{\isacharcomma}{\kern0pt}\ M{\isacharparenright}{\kern0pt}\ {\isacharminus}{\kern0pt}{\isacharbackquote}{\kern0pt}{\isacharbackquote}{\kern0pt}\ {\isacharbraceleft}{\kern0pt}x{\isacharbraceright}{\kern0pt}\ {\isacharequal}{\kern0pt}\ preds{\isacharunderscore}{\kern0pt}rel{\isacharparenleft}{\kern0pt}R{\isacharcomma}{\kern0pt}\ x{\isacharparenright}{\kern0pt}\ {\isacharminus}{\kern0pt}{\isacharbackquote}{\kern0pt}{\isacharbackquote}{\kern0pt}\ {\isacharbraceleft}{\kern0pt}x{\isacharbraceright}{\kern0pt}{\isachardoublequoteclose}\ \isanewline
\ \ \ \ \ \ \isacommand{apply}\isamarkupfalse%
{\isacharparenleft}{\kern0pt}rule\ eq{\isacharunderscore}{\kern0pt}flip{\isacharcomma}{\kern0pt}\ rule\ preds{\isacharunderscore}{\kern0pt}rel{\isacharunderscore}{\kern0pt}vimage{\isacharunderscore}{\kern0pt}eq{\isacharprime}{\kern0pt}{\isacharparenright}{\kern0pt}\isanewline
\ \ \ \ \ \ \isacommand{using}\isamarkupfalse%
\ assms\isanewline
\ \ \ \ \ \ \isacommand{by}\isamarkupfalse%
\ auto\isanewline
\isanewline
\ \ \ \ \isacommand{show}\isamarkupfalse%
\ {\isachardoublequoteopen}wftrec{\isacharparenleft}{\kern0pt}prel{\isacharparenleft}{\kern0pt}preds{\isacharunderscore}{\kern0pt}rel{\isacharparenleft}{\kern0pt}R{\isacharcomma}{\kern0pt}\ x{\isacharparenright}{\kern0pt}{\isacharcomma}{\kern0pt}\ {\isacharbraceleft}{\kern0pt}a{\isacharbraceright}{\kern0pt}{\isacharparenright}{\kern0pt}{\isacharcomma}{\kern0pt}\ {\isasymlangle}x{\isacharcomma}{\kern0pt}\ a{\isasymrangle}{\isacharcomma}{\kern0pt}\ G{\isacharparenright}{\kern0pt}\ {\isacharequal}{\kern0pt}\ wftrec{\isacharparenleft}{\kern0pt}preds{\isacharunderscore}{\kern0pt}rel{\isacharparenleft}{\kern0pt}R{\isacharcomma}{\kern0pt}\ x{\isacharparenright}{\kern0pt}{\isacharcomma}{\kern0pt}\ x{\isacharcomma}{\kern0pt}\ H{\isacharparenright}{\kern0pt}{\isachardoublequoteclose}\ \isanewline
\ \ \ \ \ \ \isacommand{apply}\isamarkupfalse%
{\isacharparenleft}{\kern0pt}subst\ eqH{\isacharcomma}{\kern0pt}\ subst\ eqG{\isacharcomma}{\kern0pt}\ rule\ eq{\isacharunderscore}{\kern0pt}flip{\isacharparenright}{\kern0pt}\isanewline
\ \ \ \ \ \ \isacommand{apply}\isamarkupfalse%
{\isacharparenleft}{\kern0pt}rule\ HGeq{\isacharparenright}{\kern0pt}\isanewline
\ \ \ \ \ \ \isacommand{apply}\isamarkupfalse%
{\isacharparenleft}{\kern0pt}subst\ eq{\isadigit{0}}{\isacharprime}{\kern0pt}{\isacharprime}{\kern0pt}{\isacharcomma}{\kern0pt}\ subst\ eq{\isadigit{0}}{\isacharcomma}{\kern0pt}\ simp{\isacharparenright}{\kern0pt}\isanewline
\ \ \ \ \ \ \isacommand{apply}\isamarkupfalse%
{\isacharparenleft}{\kern0pt}subst\ eq{\isadigit{0}}{\isacharprime}{\kern0pt}{\isacharprime}{\kern0pt}{\isacharcomma}{\kern0pt}\ subst\ eq{\isadigit{0}}{\isacharcomma}{\kern0pt}\ simp{\isacharparenright}{\kern0pt}\isanewline
\ \ \ \ \ \ \isacommand{using}\isamarkupfalse%
\ assms\ zero{\isacharunderscore}{\kern0pt}in{\isacharunderscore}{\kern0pt}M\isanewline
\ \ \ \ \ \ \ \ \ \ \ \ \ \isacommand{apply}\isamarkupfalse%
\ auto{\isacharbrackleft}{\kern0pt}{\isadigit{2}}{\isacharbrackright}{\kern0pt}\isanewline
\ \ \ \ \ \ \isacommand{apply}\isamarkupfalse%
{\isacharparenleft}{\kern0pt}subst\ apply{\isacharunderscore}{\kern0pt}{\isadigit{0}}{\isacharcomma}{\kern0pt}\ simp{\isacharparenright}{\kern0pt}{\isacharplus}{\kern0pt}\isanewline
\ \ \ \ \ \ \isacommand{by}\isamarkupfalse%
\ simp\isanewline
\ \ \isacommand{next}\isamarkupfalse%
\isanewline
\ \ \ \ \isacommand{case}\isamarkupfalse%
\ False\isanewline
\ \ \ \ \isacommand{then}\isamarkupfalse%
\ \isacommand{obtain}\isamarkupfalse%
\ y\ \isakeyword{where}\ {\isachardoublequoteopen}{\isacharless}{\kern0pt}y{\isacharcomma}{\kern0pt}\ x{\isachargreater}{\kern0pt}\ {\isasymin}\ preds{\isacharunderscore}{\kern0pt}rel{\isacharparenleft}{\kern0pt}R{\isacharcomma}{\kern0pt}\ x{\isacharparenright}{\kern0pt}{\isachardoublequoteclose}\ \isacommand{by}\isamarkupfalse%
\ blast\ \isanewline
\ \ \ \ \isacommand{then}\isamarkupfalse%
\ \isacommand{have}\isamarkupfalse%
\ xinfield\ {\isacharcolon}{\kern0pt}\ {\isachardoublequoteopen}x\ {\isasymin}\ field{\isacharparenleft}{\kern0pt}preds{\isacharunderscore}{\kern0pt}rel{\isacharparenleft}{\kern0pt}R{\isacharcomma}{\kern0pt}\ x{\isacharparenright}{\kern0pt}{\isacharparenright}{\kern0pt}{\isachardoublequoteclose}\ \isacommand{by}\isamarkupfalse%
\ auto\ \isanewline
\isanewline
\ \ \ \ \isacommand{have}\isamarkupfalse%
\ subsetH{\isacharcolon}{\kern0pt}\ {\isachardoublequoteopen}preds{\isacharunderscore}{\kern0pt}rel{\isacharparenleft}{\kern0pt}R{\isacharcomma}{\kern0pt}\ x{\isacharparenright}{\kern0pt}\ {\isasymsubseteq}\ Rrel{\isacharparenleft}{\kern0pt}R{\isacharcomma}{\kern0pt}\ M{\isacharparenright}{\kern0pt}{\isachardoublequoteclose}\ \isanewline
\ \ \ \ \ \ \isacommand{unfolding}\isamarkupfalse%
\ preds{\isacharunderscore}{\kern0pt}rel{\isacharunderscore}{\kern0pt}def\ Rrel{\isacharunderscore}{\kern0pt}def\ preds{\isacharunderscore}{\kern0pt}def\isanewline
\ \ \ \ \ \ \isacommand{using}\isamarkupfalse%
\ assms\isanewline
\ \ \ \ \ \ \isacommand{by}\isamarkupfalse%
\ auto\isanewline
\isanewline
\ \ \ \ \isacommand{have}\isamarkupfalse%
\ fieldsubset\ {\isacharcolon}{\kern0pt}\ {\isachardoublequoteopen}field{\isacharparenleft}{\kern0pt}preds{\isacharunderscore}{\kern0pt}rel{\isacharparenleft}{\kern0pt}R{\isacharcomma}{\kern0pt}\ x{\isacharparenright}{\kern0pt}{\isacharparenright}{\kern0pt}\ {\isasymsubseteq}\ field{\isacharparenleft}{\kern0pt}Rrel{\isacharparenleft}{\kern0pt}R{\isacharcomma}{\kern0pt}\ M{\isacharparenright}{\kern0pt}{\isacharparenright}{\kern0pt}{\isachardoublequoteclose}\ \isanewline
\ \ \ \ \isacommand{proof}\isamarkupfalse%
{\isacharparenleft}{\kern0pt}rule\ subsetI{\isacharparenright}{\kern0pt}\isanewline
\ \ \ \ \ \ \isacommand{fix}\isamarkupfalse%
\ w\ \isacommand{assume}\isamarkupfalse%
\ {\isachardoublequoteopen}w\ {\isasymin}\ field{\isacharparenleft}{\kern0pt}preds{\isacharunderscore}{\kern0pt}rel{\isacharparenleft}{\kern0pt}R{\isacharcomma}{\kern0pt}\ x{\isacharparenright}{\kern0pt}{\isacharparenright}{\kern0pt}{\isachardoublequoteclose}\ \isanewline
\ \ \ \ \ \ \isacommand{then}\isamarkupfalse%
\ \isacommand{obtain}\isamarkupfalse%
\ u\ \isakeyword{where}\ {\isachardoublequoteopen}{\isacharless}{\kern0pt}w{\isacharcomma}{\kern0pt}\ u{\isachargreater}{\kern0pt}\ {\isasymin}\ preds{\isacharunderscore}{\kern0pt}rel{\isacharparenleft}{\kern0pt}R{\isacharcomma}{\kern0pt}\ x{\isacharparenright}{\kern0pt}\ {\isasymor}\ {\isacharless}{\kern0pt}u{\isacharcomma}{\kern0pt}\ w{\isachargreater}{\kern0pt}\ {\isasymin}\ preds{\isacharunderscore}{\kern0pt}rel{\isacharparenleft}{\kern0pt}R{\isacharcomma}{\kern0pt}\ x{\isacharparenright}{\kern0pt}{\isachardoublequoteclose}\ \isacommand{by}\isamarkupfalse%
\ auto\ \isanewline
\ \ \ \ \ \ \isacommand{then}\isamarkupfalse%
\ \isacommand{have}\isamarkupfalse%
\ {\isachardoublequoteopen}{\isacharless}{\kern0pt}w{\isacharcomma}{\kern0pt}\ u{\isachargreater}{\kern0pt}\ {\isasymin}\ Rrel{\isacharparenleft}{\kern0pt}R{\isacharcomma}{\kern0pt}\ M{\isacharparenright}{\kern0pt}\ {\isasymor}\ {\isacharless}{\kern0pt}u{\isacharcomma}{\kern0pt}\ w{\isachargreater}{\kern0pt}\ {\isasymin}\ Rrel{\isacharparenleft}{\kern0pt}R{\isacharcomma}{\kern0pt}\ M{\isacharparenright}{\kern0pt}{\isachardoublequoteclose}\ \isacommand{using}\isamarkupfalse%
\ subsetH\ \isacommand{by}\isamarkupfalse%
\ auto\ \isanewline
\ \ \ \ \ \ \isacommand{then}\isamarkupfalse%
\ \isacommand{show}\isamarkupfalse%
\ {\isachardoublequoteopen}w\ {\isasymin}\ field{\isacharparenleft}{\kern0pt}Rrel{\isacharparenleft}{\kern0pt}R{\isacharcomma}{\kern0pt}\ M{\isacharparenright}{\kern0pt}{\isacharparenright}{\kern0pt}{\isachardoublequoteclose}\ \isacommand{by}\isamarkupfalse%
\ auto\ \isanewline
\ \ \ \ \isacommand{qed}\isamarkupfalse%
\ \isanewline
\isanewline
\ \ \ \ \isacommand{have}\isamarkupfalse%
\ fieldcase\ {\isacharcolon}{\kern0pt}\ {\isachardoublequoteopen}{\isasymAnd}t{\isachardot}{\kern0pt}\ t\ {\isasymin}\ field{\isacharparenleft}{\kern0pt}preds{\isacharunderscore}{\kern0pt}rel{\isacharparenleft}{\kern0pt}R{\isacharcomma}{\kern0pt}\ x{\isacharparenright}{\kern0pt}{\isacharparenright}{\kern0pt}\ {\isasymLongrightarrow}\ {\isacharless}{\kern0pt}t{\isacharcomma}{\kern0pt}\ x{\isachargreater}{\kern0pt}\ {\isasymin}\ Rrel{\isacharparenleft}{\kern0pt}R{\isacharcomma}{\kern0pt}\ M{\isacharparenright}{\kern0pt}\ {\isasymor}\ t\ {\isacharequal}{\kern0pt}\ x{\isachardoublequoteclose}\ \isanewline
\ \ \ \ \isacommand{proof}\isamarkupfalse%
\ {\isacharminus}{\kern0pt}\ \isanewline
\ \ \ \ \ \ \isacommand{fix}\isamarkupfalse%
\ t\ \isacommand{assume}\isamarkupfalse%
\ {\isachardoublequoteopen}t\ {\isasymin}\ field{\isacharparenleft}{\kern0pt}preds{\isacharunderscore}{\kern0pt}rel{\isacharparenleft}{\kern0pt}R{\isacharcomma}{\kern0pt}\ x{\isacharparenright}{\kern0pt}{\isacharparenright}{\kern0pt}{\isachardoublequoteclose}\ \isanewline
\ \ \ \ \ \ \isacommand{then}\isamarkupfalse%
\ \isacommand{have}\isamarkupfalse%
\ {\isachardoublequoteopen}t\ {\isasymin}\ preds{\isacharparenleft}{\kern0pt}R{\isacharcomma}{\kern0pt}\ x{\isacharparenright}{\kern0pt}\ {\isasymor}\ t\ {\isacharequal}{\kern0pt}\ x{\isachardoublequoteclose}\ \isacommand{unfolding}\isamarkupfalse%
\ preds{\isacharunderscore}{\kern0pt}rel{\isacharunderscore}{\kern0pt}def\ \isacommand{by}\isamarkupfalse%
\ auto\ \isanewline
\ \ \ \ \ \ \isacommand{then}\isamarkupfalse%
\ \isacommand{show}\isamarkupfalse%
\ {\isachardoublequoteopen}{\isacharless}{\kern0pt}t{\isacharcomma}{\kern0pt}\ x{\isachargreater}{\kern0pt}\ {\isasymin}\ Rrel{\isacharparenleft}{\kern0pt}R{\isacharcomma}{\kern0pt}\ M{\isacharparenright}{\kern0pt}\ {\isasymor}\ t\ {\isacharequal}{\kern0pt}\ x{\isachardoublequoteclose}\ \isanewline
\ \ \ \ \ \ \ \ \isacommand{unfolding}\isamarkupfalse%
\ preds{\isacharunderscore}{\kern0pt}def\ Rrel{\isacharunderscore}{\kern0pt}def\ \isanewline
\ \ \ \ \ \ \ \ \isacommand{using}\isamarkupfalse%
\ assms\ \isanewline
\ \ \ \ \ \ \ \ \isacommand{by}\isamarkupfalse%
\ auto\isanewline
\ \ \ \ \isacommand{qed}\isamarkupfalse%
\isanewline
\isanewline
\ \ \ \ \isacommand{have}\isamarkupfalse%
\ H{\isacharcolon}{\kern0pt}\ {\isachardoublequoteopen}{\isasymAnd}t\ z{\isachardot}{\kern0pt}\ t\ {\isasymin}\ field{\isacharparenleft}{\kern0pt}preds{\isacharunderscore}{\kern0pt}rel{\isacharparenleft}{\kern0pt}R{\isacharcomma}{\kern0pt}\ x{\isacharparenright}{\kern0pt}{\isacharparenright}{\kern0pt}\ {\isasymLongrightarrow}\ z\ {\isasymin}\ Rrel{\isacharparenleft}{\kern0pt}R{\isacharcomma}{\kern0pt}\ M{\isacharparenright}{\kern0pt}\ {\isacharminus}{\kern0pt}{\isacharbackquote}{\kern0pt}{\isacharbackquote}{\kern0pt}\ {\isacharbraceleft}{\kern0pt}t{\isacharbraceright}{\kern0pt}\ {\isasymLongrightarrow}\ z\ {\isasymin}\ preds{\isacharunderscore}{\kern0pt}rel{\isacharparenleft}{\kern0pt}R{\isacharcomma}{\kern0pt}\ x{\isacharparenright}{\kern0pt}\ {\isacharminus}{\kern0pt}{\isacharbackquote}{\kern0pt}{\isacharbackquote}{\kern0pt}\ {\isacharbraceleft}{\kern0pt}t{\isacharbraceright}{\kern0pt}{\isachardoublequoteclose}\isanewline
\ \ \ \ \isacommand{proof}\isamarkupfalse%
\ {\isacharminus}{\kern0pt}\ \isanewline
\ \ \ \ \ \ \isacommand{fix}\isamarkupfalse%
\ t\ z\ \isanewline
\ \ \ \ \ \ \isacommand{assume}\isamarkupfalse%
\ assms{\isadigit{2}}\ {\isacharcolon}{\kern0pt}\ {\isachardoublequoteopen}t\ {\isasymin}\ field{\isacharparenleft}{\kern0pt}preds{\isacharunderscore}{\kern0pt}rel{\isacharparenleft}{\kern0pt}R{\isacharcomma}{\kern0pt}\ x{\isacharparenright}{\kern0pt}{\isacharparenright}{\kern0pt}{\isachardoublequoteclose}\ {\isachardoublequoteopen}z\ {\isasymin}\ Rrel{\isacharparenleft}{\kern0pt}R{\isacharcomma}{\kern0pt}\ M{\isacharparenright}{\kern0pt}\ {\isacharminus}{\kern0pt}{\isacharbackquote}{\kern0pt}{\isacharbackquote}{\kern0pt}\ {\isacharbraceleft}{\kern0pt}t{\isacharbraceright}{\kern0pt}{\isachardoublequoteclose}\ \isanewline
\ \ \ \ \ \ \isacommand{have}\isamarkupfalse%
\ tin\ {\isacharcolon}{\kern0pt}\ {\isachardoublequoteopen}t\ {\isasymin}\ preds{\isacharparenleft}{\kern0pt}R{\isacharcomma}{\kern0pt}\ x{\isacharparenright}{\kern0pt}\ {\isasymunion}\ {\isacharbraceleft}{\kern0pt}x{\isacharbraceright}{\kern0pt}{\isachardoublequoteclose}\ \isacommand{using}\isamarkupfalse%
\ assms{\isadigit{2}}\ \isacommand{unfolding}\isamarkupfalse%
\ preds{\isacharunderscore}{\kern0pt}rel{\isacharunderscore}{\kern0pt}def\ field{\isacharunderscore}{\kern0pt}def\ \isacommand{by}\isamarkupfalse%
\ auto\ \isanewline
\ \ \ \ \ \ \isacommand{have}\isamarkupfalse%
\ ztin\ {\isacharcolon}{\kern0pt}\ {\isachardoublequoteopen}{\isacharless}{\kern0pt}z{\isacharcomma}{\kern0pt}\ t{\isachargreater}{\kern0pt}\ {\isasymin}\ Rrel{\isacharparenleft}{\kern0pt}R{\isacharcomma}{\kern0pt}\ M{\isacharparenright}{\kern0pt}{\isachardoublequoteclose}\ \isacommand{using}\isamarkupfalse%
\ assms{\isadigit{2}}\ \isacommand{by}\isamarkupfalse%
\ auto\ \isanewline
\ \ \ \ \ \ \isacommand{then}\isamarkupfalse%
\ \isacommand{have}\isamarkupfalse%
\ {\isachardoublequoteopen}{\isacharless}{\kern0pt}z{\isacharcomma}{\kern0pt}\ x{\isachargreater}{\kern0pt}\ {\isasymin}\ Rrel{\isacharparenleft}{\kern0pt}R{\isacharcomma}{\kern0pt}\ M{\isacharparenright}{\kern0pt}{\isachardoublequoteclose}\ \isanewline
\ \ \ \ \ \ \ \ \isacommand{apply}\isamarkupfalse%
{\isacharparenleft}{\kern0pt}cases\ {\isachardoublequoteopen}t\ {\isacharequal}{\kern0pt}\ x{\isachardoublequoteclose}{\isacharcomma}{\kern0pt}\ simp{\isacharparenright}{\kern0pt}\isanewline
\ \ \ \ \ \ \ \ \isacommand{apply}\isamarkupfalse%
{\isacharparenleft}{\kern0pt}subgoal{\isacharunderscore}{\kern0pt}tac\ {\isachardoublequoteopen}{\isacharless}{\kern0pt}t{\isacharcomma}{\kern0pt}\ x{\isachargreater}{\kern0pt}\ {\isasymin}\ Rrel{\isacharparenleft}{\kern0pt}R{\isacharcomma}{\kern0pt}\ M{\isacharparenright}{\kern0pt}{\isachardoublequoteclose}{\isacharparenright}{\kern0pt}\isanewline
\ \ \ \ \ \ \ \ \ \isacommand{apply}\isamarkupfalse%
{\isacharparenleft}{\kern0pt}rule\ transD{\isacharparenright}{\kern0pt}\isanewline
\ \ \ \ \ \ \ \ \isacommand{using}\isamarkupfalse%
\ assms\ \isanewline
\ \ \ \ \ \ \ \ \ \ \ \isacommand{apply}\isamarkupfalse%
\ auto{\isacharbrackleft}{\kern0pt}{\isadigit{3}}{\isacharbrackright}{\kern0pt}\isanewline
\ \ \ \ \ \ \ \ \isacommand{using}\isamarkupfalse%
\ tin\ assms\isanewline
\ \ \ \ \ \ \ \ \isacommand{unfolding}\isamarkupfalse%
\ preds{\isacharunderscore}{\kern0pt}def\ Rrel{\isacharunderscore}{\kern0pt}def\ \isanewline
\ \ \ \ \ \ \ \ \isacommand{by}\isamarkupfalse%
\ auto\isanewline
\ \ \ \ \ \ \isacommand{then}\isamarkupfalse%
\ \isacommand{have}\isamarkupfalse%
\ {\isachardoublequoteopen}z\ {\isasymin}\ preds{\isacharparenleft}{\kern0pt}R{\isacharcomma}{\kern0pt}\ x{\isacharparenright}{\kern0pt}{\isachardoublequoteclose}\ \isacommand{unfolding}\isamarkupfalse%
\ Rrel{\isacharunderscore}{\kern0pt}def\ preds{\isacharunderscore}{\kern0pt}def\ \isacommand{by}\isamarkupfalse%
\ auto\isanewline
\ \ \ \ \ \ \isacommand{then}\isamarkupfalse%
\ \isacommand{show}\isamarkupfalse%
\ {\isachardoublequoteopen}z\ {\isasymin}\ preds{\isacharunderscore}{\kern0pt}rel{\isacharparenleft}{\kern0pt}R{\isacharcomma}{\kern0pt}\ x{\isacharparenright}{\kern0pt}\ {\isacharminus}{\kern0pt}{\isacharbackquote}{\kern0pt}{\isacharbackquote}{\kern0pt}\ {\isacharbraceleft}{\kern0pt}t{\isacharbraceright}{\kern0pt}{\isachardoublequoteclose}\ \isanewline
\ \ \ \ \ \ \ \ \isacommand{using}\isamarkupfalse%
\ tin\ ztin\isanewline
\ \ \ \ \ \ \ \ \isacommand{unfolding}\isamarkupfalse%
\ preds{\isacharunderscore}{\kern0pt}rel{\isacharunderscore}{\kern0pt}def\ Rrel{\isacharunderscore}{\kern0pt}def\ \isanewline
\ \ \ \ \ \ \ \ \isacommand{by}\isamarkupfalse%
\ auto\isanewline
\ \ \ \ \isacommand{qed}\isamarkupfalse%
\isanewline
\isanewline
\ \ \ \ \isacommand{show}\isamarkupfalse%
\ {\isachardoublequoteopen}wftrec{\isacharparenleft}{\kern0pt}prel{\isacharparenleft}{\kern0pt}preds{\isacharunderscore}{\kern0pt}rel{\isacharparenleft}{\kern0pt}R{\isacharcomma}{\kern0pt}\ x{\isacharparenright}{\kern0pt}{\isacharcomma}{\kern0pt}\ {\isacharbraceleft}{\kern0pt}a{\isacharbraceright}{\kern0pt}{\isacharparenright}{\kern0pt}{\isacharcomma}{\kern0pt}\ {\isasymlangle}x{\isacharcomma}{\kern0pt}\ a{\isasymrangle}{\isacharcomma}{\kern0pt}\ G{\isacharparenright}{\kern0pt}\ {\isacharequal}{\kern0pt}\ wftrec{\isacharparenleft}{\kern0pt}preds{\isacharunderscore}{\kern0pt}rel{\isacharparenleft}{\kern0pt}R{\isacharcomma}{\kern0pt}\ x{\isacharparenright}{\kern0pt}{\isacharcomma}{\kern0pt}\ x{\isacharcomma}{\kern0pt}\ H{\isacharparenright}{\kern0pt}{\isachardoublequoteclose}\ \ \isanewline
\ \ \ \ \ \ \isacommand{apply}\isamarkupfalse%
{\isacharparenleft}{\kern0pt}rule\ eq{\isacharunderscore}{\kern0pt}flip{\isacharcomma}{\kern0pt}\ rule{\isacharunderscore}{\kern0pt}tac\ Gfm\ {\isacharequal}{\kern0pt}\ Gfm\ \isakeyword{in}\ wftrec{\isacharunderscore}{\kern0pt}prel{\isacharunderscore}{\kern0pt}eq{\isacharparenright}{\kern0pt}\isanewline
\ \ \ \ \ \ \ \ \ \ \ \ \ \ \ \ \isacommand{apply}\isamarkupfalse%
{\isacharparenleft}{\kern0pt}rule\ wf{\isacharunderscore}{\kern0pt}preds{\isacharunderscore}{\kern0pt}rel{\isacharcomma}{\kern0pt}\ simp\ add{\isacharcolon}{\kern0pt}assms{\isacharparenright}{\kern0pt}\isanewline
\ \ \ \ \ \ \ \ \ \ \ \ \ \ \ \ \isacommand{apply}\isamarkupfalse%
{\isacharparenleft}{\kern0pt}simp\ add{\isacharcolon}{\kern0pt}assms{\isacharparenright}{\kern0pt}\isanewline
\ \ \ \ \ \ \ \ \ \ \ \ \ \ \ \isacommand{apply}\isamarkupfalse%
{\isacharparenleft}{\kern0pt}rule{\isacharunderscore}{\kern0pt}tac\ Rfm\ {\isacharequal}{\kern0pt}\ Rfm\ \isakeyword{in}\ preds{\isacharunderscore}{\kern0pt}rel{\isacharunderscore}{\kern0pt}in{\isacharunderscore}{\kern0pt}M{\isacharparenright}{\kern0pt}\isanewline
\ \ \ \ \ \ \ \ \ \ \ \ \ \ \ \ \ \isacommand{apply}\isamarkupfalse%
{\isacharparenleft}{\kern0pt}simp\ add{\isacharcolon}{\kern0pt}assms{\isacharparenright}{\kern0pt}{\isacharplus}{\kern0pt}\isanewline
\ \ \ \ \ \ \ \ \ \ \ \ \ \ \isacommand{apply}\isamarkupfalse%
{\isacharparenleft}{\kern0pt}rule\ trans{\isacharunderscore}{\kern0pt}preds{\isacharunderscore}{\kern0pt}rel{\isacharcomma}{\kern0pt}\ simp\ add{\isacharcolon}{\kern0pt}assms{\isacharcomma}{\kern0pt}\ simp\ add{\isacharcolon}{\kern0pt}assms{\isacharcomma}{\kern0pt}\ simp{\isacharparenright}{\kern0pt}\isanewline
\ \ \ \ \ \ \isacommand{using}\isamarkupfalse%
\ singleton{\isacharunderscore}{\kern0pt}in{\isacharunderscore}{\kern0pt}M{\isacharunderscore}{\kern0pt}iff\ xinfield\ assms\isanewline
\ \ \ \ \ \ \ \ \ \ \ \ \isacommand{apply}\isamarkupfalse%
\ auto{\isacharbrackleft}{\kern0pt}{\isadigit{6}}{\isacharbrackright}{\kern0pt}\isanewline
\ \ \ \ \ \ \isacommand{apply}\isamarkupfalse%
{\isacharparenleft}{\kern0pt}rule\ HGeq{\isacharparenright}{\kern0pt}\isanewline
\ \ \ \ \ \ \ \ \ \ \isacommand{apply}\isamarkupfalse%
{\isacharparenleft}{\kern0pt}rename{\isacharunderscore}{\kern0pt}tac\ h\ g\ u{\isacharcomma}{\kern0pt}\ case{\isacharunderscore}{\kern0pt}tac\ {\isachardoublequoteopen}u\ {\isacharequal}{\kern0pt}\ x{\isachardoublequoteclose}{\isacharcomma}{\kern0pt}\ simp{\isacharparenright}{\kern0pt}\isanewline
\ \ \ \ \ \ \isacommand{using}\isamarkupfalse%
\ preds{\isacharunderscore}{\kern0pt}rel{\isacharunderscore}{\kern0pt}vimage{\isacharunderscore}{\kern0pt}eq{\isacharprime}{\kern0pt}\ assms\ \isanewline
\ \ \ \ \ \ \ \ \ \ \ \isacommand{apply}\isamarkupfalse%
\ force\isanewline
\ \ \ \ \ \ \ \ \ \ \isacommand{apply}\isamarkupfalse%
{\isacharparenleft}{\kern0pt}rename{\isacharunderscore}{\kern0pt}tac\ h\ g\ u{\isacharcomma}{\kern0pt}\ rule{\isacharunderscore}{\kern0pt}tac\ b{\isacharequal}{\kern0pt}{\isachardoublequoteopen}Rrel{\isacharparenleft}{\kern0pt}R{\isacharcomma}{\kern0pt}\ M{\isacharparenright}{\kern0pt}\ {\isacharminus}{\kern0pt}{\isacharbackquote}{\kern0pt}{\isacharbackquote}{\kern0pt}\ {\isacharbraceleft}{\kern0pt}u{\isacharbraceright}{\kern0pt}{\isachardoublequoteclose}\ \isakeyword{and}\ a{\isacharequal}{\kern0pt}{\isachardoublequoteopen}preds{\isacharunderscore}{\kern0pt}rel{\isacharparenleft}{\kern0pt}R{\isacharcomma}{\kern0pt}\ x{\isacharparenright}{\kern0pt}\ {\isacharminus}{\kern0pt}{\isacharbackquote}{\kern0pt}{\isacharbackquote}{\kern0pt}\ {\isacharbraceleft}{\kern0pt}u{\isacharbraceright}{\kern0pt}{\isachardoublequoteclose}\ \isakeyword{in}\ ssubst{\isacharparenright}{\kern0pt}\isanewline
\ \ \ \ \ \ \ \ \ \ \ \isacommand{apply}\isamarkupfalse%
{\isacharparenleft}{\kern0pt}rule\ eq{\isacharunderscore}{\kern0pt}flip{\isacharcomma}{\kern0pt}\ rule\ preds{\isacharunderscore}{\kern0pt}rel{\isacharunderscore}{\kern0pt}vimage{\isacharunderscore}{\kern0pt}eq{\isacharcomma}{\kern0pt}\ simp\ add{\isacharcolon}{\kern0pt}assms{\isacharparenright}{\kern0pt}\isanewline
\ \ \ \ \ \ \isacommand{using}\isamarkupfalse%
\ fieldcase\ \isanewline
\ \ \ \ \ \ \ \ \ \ \ \isacommand{apply}\isamarkupfalse%
\ force\isanewline
\ \ \ \ \ \ \ \ \ \ \isacommand{apply}\isamarkupfalse%
\ simp\isanewline
\ \ \ \ \ \ \ \ \ \ \isacommand{apply}\isamarkupfalse%
{\isacharparenleft}{\kern0pt}rename{\isacharunderscore}{\kern0pt}tac\ h\ g\ u{\isacharcomma}{\kern0pt}\ case{\isacharunderscore}{\kern0pt}tac\ {\isachardoublequoteopen}u\ {\isacharequal}{\kern0pt}\ x{\isachardoublequoteclose}{\isacharcomma}{\kern0pt}\ simp{\isacharparenright}{\kern0pt}\isanewline
\ \ \ \ \ \ \isacommand{using}\isamarkupfalse%
\ preds{\isacharunderscore}{\kern0pt}rel{\isacharunderscore}{\kern0pt}vimage{\isacharunderscore}{\kern0pt}eq{\isacharprime}{\kern0pt}\ assms\ \isanewline
\ \ \ \ \ \ \ \ \ \ \ \isacommand{apply}\isamarkupfalse%
\ force\isanewline
\ \ \ \ \ \ \ \ \ \isacommand{apply}\isamarkupfalse%
{\isacharparenleft}{\kern0pt}rename{\isacharunderscore}{\kern0pt}tac\ h\ g\ u{\isacharcomma}{\kern0pt}\ rule{\isacharunderscore}{\kern0pt}tac\ b{\isacharequal}{\kern0pt}{\isachardoublequoteopen}Rrel{\isacharparenleft}{\kern0pt}R{\isacharcomma}{\kern0pt}\ M{\isacharparenright}{\kern0pt}\ {\isacharminus}{\kern0pt}{\isacharbackquote}{\kern0pt}{\isacharbackquote}{\kern0pt}\ {\isacharbraceleft}{\kern0pt}u{\isacharbraceright}{\kern0pt}{\isachardoublequoteclose}\ \isakeyword{and}\ a{\isacharequal}{\kern0pt}{\isachardoublequoteopen}preds{\isacharunderscore}{\kern0pt}rel{\isacharparenleft}{\kern0pt}R{\isacharcomma}{\kern0pt}\ x{\isacharparenright}{\kern0pt}\ {\isacharminus}{\kern0pt}{\isacharbackquote}{\kern0pt}{\isacharbackquote}{\kern0pt}\ {\isacharbraceleft}{\kern0pt}u{\isacharbraceright}{\kern0pt}{\isachardoublequoteclose}\ \isakeyword{in}\ ssubst{\isacharparenright}{\kern0pt}\isanewline
\ \ \ \ \ \ \ \ \ \ \ \isacommand{apply}\isamarkupfalse%
{\isacharparenleft}{\kern0pt}rule\ eq{\isacharunderscore}{\kern0pt}flip{\isacharcomma}{\kern0pt}\ rule\ preds{\isacharunderscore}{\kern0pt}rel{\isacharunderscore}{\kern0pt}vimage{\isacharunderscore}{\kern0pt}eq{\isacharcomma}{\kern0pt}\ simp\ add{\isacharcolon}{\kern0pt}assms{\isacharparenright}{\kern0pt}\isanewline
\ \ \ \ \ \ \isacommand{using}\isamarkupfalse%
\ fieldcase\ \isanewline
\ \ \ \ \ \ \ \ \ \ \ \isacommand{apply}\isamarkupfalse%
\ force\isanewline
\ \ \ \ \ \ \ \ \ \isacommand{apply}\isamarkupfalse%
\ simp\isanewline
\ \ \ \ \ \ \ \ \isacommand{apply}\isamarkupfalse%
\ simp\isanewline
\ \ \ \ \ \ \isacommand{using}\isamarkupfalse%
\ fieldsubset\ \isanewline
\ \ \ \ \ \ \ \isacommand{apply}\isamarkupfalse%
\ force\ \ \ \ \ \ \isanewline
\ \ \ \ \ \ \isacommand{using}\isamarkupfalse%
\ fieldsubset\ H\isanewline
\ \ \ \ \ \ \isacommand{by}\isamarkupfalse%
\ auto\isanewline
\ \ \isacommand{qed}\isamarkupfalse%
\ \isanewline
\ \ \isacommand{also}\isamarkupfalse%
\ \isacommand{have}\isamarkupfalse%
\ E{\isadigit{2}}{\isacharcolon}{\kern0pt}\ {\isachardoublequoteopen}{\isachardot}{\kern0pt}{\isachardot}{\kern0pt}{\isachardot}{\kern0pt}\ {\isacharequal}{\kern0pt}\ wftrec{\isacharparenleft}{\kern0pt}Rrel{\isacharparenleft}{\kern0pt}R{\isacharcomma}{\kern0pt}\ M{\isacharparenright}{\kern0pt}{\isacharcomma}{\kern0pt}\ x{\isacharcomma}{\kern0pt}\ H{\isacharparenright}{\kern0pt}{\isachardoublequoteclose}\ \isanewline
\ \ \ \ \isacommand{apply}\isamarkupfalse%
{\isacharparenleft}{\kern0pt}rule\ eq{\isacharunderscore}{\kern0pt}flip{\isacharparenright}{\kern0pt}\isanewline
\ \ \ \ \isacommand{apply}\isamarkupfalse%
{\isacharparenleft}{\kern0pt}rule\ wftrec{\isacharunderscore}{\kern0pt}preds{\isacharunderscore}{\kern0pt}rel{\isacharunderscore}{\kern0pt}eq{\isacharparenright}{\kern0pt}\isanewline
\ \ \ \ \isacommand{using}\isamarkupfalse%
\ assms\ \isanewline
\ \ \ \ \isacommand{by}\isamarkupfalse%
\ auto\isanewline
\ \ \isacommand{finally}\isamarkupfalse%
\ \isacommand{show}\isamarkupfalse%
\ {\isacharquery}{\kern0pt}thesis\ \isacommand{by}\isamarkupfalse%
\ simp\isanewline
\isacommand{qed}\isamarkupfalse%
%
\endisatagproof
{\isafoldproof}%
%
\isadelimproof
\isanewline
%
\endisadelimproof
\isanewline
\isacommand{lemma}\isamarkupfalse%
\ sats{\isacharunderscore}{\kern0pt}is{\isacharunderscore}{\kern0pt}Rrel{\isacharunderscore}{\kern0pt}wftrec{\isacharunderscore}{\kern0pt}fm{\isacharunderscore}{\kern0pt}iff\ {\isacharcolon}{\kern0pt}\isanewline
\ \ \isakeyword{fixes}\ H\ G\ Gfm\ R\ Rfm\ x\ v\ i\ j\ k\ env\ a\isanewline
\ \ \isakeyword{assumes}\ {\isachardoublequoteopen}env\ {\isasymin}\ list{\isacharparenleft}{\kern0pt}M{\isacharparenright}{\kern0pt}{\isachardoublequoteclose}\ {\isachardoublequoteopen}a\ {\isasymin}\ M{\isachardoublequoteclose}\ {\isachardoublequoteopen}x\ {\isasymin}\ field{\isacharparenleft}{\kern0pt}Rrel{\isacharparenleft}{\kern0pt}R{\isacharcomma}{\kern0pt}\ M{\isacharparenright}{\kern0pt}{\isacharparenright}{\kern0pt}{\isachardoublequoteclose}\ \isanewline
\ \ \ \ \ \ \ \ \ \ {\isachardoublequoteopen}i\ {\isacharless}{\kern0pt}\ length{\isacharparenleft}{\kern0pt}env{\isacharparenright}{\kern0pt}{\isachardoublequoteclose}\ {\isachardoublequoteopen}j\ {\isacharless}{\kern0pt}\ length{\isacharparenleft}{\kern0pt}env{\isacharparenright}{\kern0pt}{\isachardoublequoteclose}\ {\isachardoublequoteopen}k\ {\isacharless}{\kern0pt}\ length{\isacharparenleft}{\kern0pt}env{\isacharparenright}{\kern0pt}{\isachardoublequoteclose}\ \isanewline
\ \ \ \ \ \ \ \ \ \ {\isachardoublequoteopen}nth{\isacharparenleft}{\kern0pt}i{\isacharcomma}{\kern0pt}\ env{\isacharparenright}{\kern0pt}\ {\isacharequal}{\kern0pt}\ x{\isachardoublequoteclose}\ {\isachardoublequoteopen}nth{\isacharparenleft}{\kern0pt}j{\isacharcomma}{\kern0pt}\ env{\isacharparenright}{\kern0pt}\ {\isacharequal}{\kern0pt}\ a{\isachardoublequoteclose}\ {\isachardoublequoteopen}nth{\isacharparenleft}{\kern0pt}k{\isacharcomma}{\kern0pt}\ env{\isacharparenright}{\kern0pt}\ {\isacharequal}{\kern0pt}\ v{\isachardoublequoteclose}\ \isanewline
\ \ \ \ \ \ \ \ \ \ {\isachardoublequoteopen}wf{\isacharparenleft}{\kern0pt}Rrel{\isacharparenleft}{\kern0pt}R{\isacharcomma}{\kern0pt}\ M{\isacharparenright}{\kern0pt}{\isacharparenright}{\kern0pt}{\isachardoublequoteclose}\ {\isachardoublequoteopen}trans{\isacharparenleft}{\kern0pt}Rrel{\isacharparenleft}{\kern0pt}R{\isacharcomma}{\kern0pt}\ M{\isacharparenright}{\kern0pt}{\isacharparenright}{\kern0pt}{\isachardoublequoteclose}\isanewline
\ \ \ \ \ \ \ \ \ \ {\isachardoublequoteopen}preds{\isacharparenleft}{\kern0pt}R{\isacharcomma}{\kern0pt}\ x{\isacharparenright}{\kern0pt}\ {\isasymin}\ M{\isachardoublequoteclose}\ \isanewline
\ \ \ \ \ \ \ \ \ \ {\isachardoublequoteopen}Relation{\isacharunderscore}{\kern0pt}fm{\isacharparenleft}{\kern0pt}R{\isacharcomma}{\kern0pt}\ Rfm{\isacharparenright}{\kern0pt}{\isachardoublequoteclose}\isanewline
\ \ \ \ \ \ \ \ \ \ {\isachardoublequoteopen}Gfm\ {\isasymin}\ formula{\isachardoublequoteclose}\ \isanewline
\ \ \ \ \ \ \ \ \ \ {\isachardoublequoteopen}arity{\isacharparenleft}{\kern0pt}Gfm{\isacharparenright}{\kern0pt}\ {\isasymle}\ {\isadigit{3}}{\isachardoublequoteclose}\ \isanewline
\ \ \ \ \ \ \ \ \ \ {\isachardoublequoteopen}{\isasymAnd}x\ g{\isachardot}{\kern0pt}\ x\ {\isasymin}\ M\ {\isasymLongrightarrow}\ g\ {\isasymin}\ M\ {\isasymLongrightarrow}\ function{\isacharparenleft}{\kern0pt}g{\isacharparenright}{\kern0pt}\ {\isasymLongrightarrow}\ G{\isacharparenleft}{\kern0pt}x{\isacharcomma}{\kern0pt}\ g{\isacharparenright}{\kern0pt}\ {\isasymin}\ M{\isachardoublequoteclose}\ \isanewline
\ \ \ \ \ \ \ \ \ \ {\isachardoublequoteopen}\ {\isacharparenleft}{\kern0pt}{\isasymAnd}a{\isadigit{0}}\ a{\isadigit{1}}\ a{\isadigit{2}}\ env{\isachardot}{\kern0pt}\ a{\isadigit{0}}\ {\isasymin}\ M\ {\isasymLongrightarrow}\ a{\isadigit{1}}\ {\isasymin}\ M\ {\isasymLongrightarrow}\ a{\isadigit{2}}\ {\isasymin}\ M\ {\isasymLongrightarrow}\ env\ {\isasymin}\ list{\isacharparenleft}{\kern0pt}M{\isacharparenright}{\kern0pt}\ {\isasymLongrightarrow}\ a{\isadigit{0}}\ {\isacharequal}{\kern0pt}\ G{\isacharparenleft}{\kern0pt}a{\isadigit{2}}{\isacharcomma}{\kern0pt}\ a{\isadigit{1}}{\isacharparenright}{\kern0pt}\ {\isasymlongleftrightarrow}\ sats{\isacharparenleft}{\kern0pt}M{\isacharcomma}{\kern0pt}\ Gfm{\isacharcomma}{\kern0pt}\ {\isacharbrackleft}{\kern0pt}a{\isadigit{0}}{\isacharcomma}{\kern0pt}\ a{\isadigit{1}}{\isacharcomma}{\kern0pt}\ a{\isadigit{2}}{\isacharbrackright}{\kern0pt}\ {\isacharat}{\kern0pt}\ env{\isacharparenright}{\kern0pt}{\isacharparenright}{\kern0pt}{\isachardoublequoteclose}\ \ \isanewline
\ \ \isakeyword{and}\ HGeq{\isacharcolon}{\kern0pt}\ {\isachardoublequoteopen}{\isasymAnd}h\ g\ x{\isachardot}{\kern0pt}\ h\ {\isasymin}\ Rrel{\isacharparenleft}{\kern0pt}R{\isacharcomma}{\kern0pt}\ M{\isacharparenright}{\kern0pt}\ {\isacharminus}{\kern0pt}{\isacharbackquote}{\kern0pt}{\isacharbackquote}{\kern0pt}\ {\isacharbraceleft}{\kern0pt}x{\isacharbraceright}{\kern0pt}\ {\isasymrightarrow}\ M\ {\isasymLongrightarrow}\ g\ {\isasymin}\ {\isacharparenleft}{\kern0pt}Rrel{\isacharparenleft}{\kern0pt}R{\isacharcomma}{\kern0pt}\ M{\isacharparenright}{\kern0pt}\ {\isacharminus}{\kern0pt}{\isacharbackquote}{\kern0pt}{\isacharbackquote}{\kern0pt}\ {\isacharbraceleft}{\kern0pt}x{\isacharbraceright}{\kern0pt}\ {\isasymtimes}\ {\isacharbraceleft}{\kern0pt}a{\isacharbraceright}{\kern0pt}{\isacharparenright}{\kern0pt}\ {\isasymrightarrow}\ M\ {\isasymLongrightarrow}\ g\ {\isasymin}\ M\ \ \isanewline
\ \ \ \ \ \ \ \ \ \ \ \ \ \ \ {\isasymLongrightarrow}\ x\ {\isasymin}\ field{\isacharparenleft}{\kern0pt}Rrel{\isacharparenleft}{\kern0pt}R{\isacharcomma}{\kern0pt}\ M{\isacharparenright}{\kern0pt}{\isacharparenright}{\kern0pt}\ {\isasymLongrightarrow}\ {\isacharparenleft}{\kern0pt}{\isasymAnd}y{\isachardot}{\kern0pt}\ y\ {\isasymin}\ Rrel{\isacharparenleft}{\kern0pt}R{\isacharcomma}{\kern0pt}\ M{\isacharparenright}{\kern0pt}\ {\isacharminus}{\kern0pt}{\isacharbackquote}{\kern0pt}{\isacharbackquote}{\kern0pt}\ {\isacharbraceleft}{\kern0pt}x{\isacharbraceright}{\kern0pt}\ {\isasymLongrightarrow}\ h{\isacharbackquote}{\kern0pt}y\ {\isacharequal}{\kern0pt}\ g{\isacharbackquote}{\kern0pt}{\isacharless}{\kern0pt}y{\isacharcomma}{\kern0pt}\ a{\isachargreater}{\kern0pt}{\isacharparenright}{\kern0pt}\ {\isasymLongrightarrow}\ H{\isacharparenleft}{\kern0pt}x{\isacharcomma}{\kern0pt}\ h{\isacharparenright}{\kern0pt}\ {\isacharequal}{\kern0pt}\ G{\isacharparenleft}{\kern0pt}{\isacharless}{\kern0pt}x{\isacharcomma}{\kern0pt}\ a{\isachargreater}{\kern0pt}{\isacharcomma}{\kern0pt}\ g{\isacharparenright}{\kern0pt}{\isachardoublequoteclose}\ \ \isanewline
\isanewline
\ \ \isakeyword{shows}\ {\isachardoublequoteopen}sats{\isacharparenleft}{\kern0pt}M{\isacharcomma}{\kern0pt}\ is{\isacharunderscore}{\kern0pt}wftrec{\isacharunderscore}{\kern0pt}fm{\isacharparenleft}{\kern0pt}Gfm{\isacharcomma}{\kern0pt}\ Rfm{\isacharcomma}{\kern0pt}\ i{\isacharcomma}{\kern0pt}\ j{\isacharcomma}{\kern0pt}\ k{\isacharparenright}{\kern0pt}{\isacharcomma}{\kern0pt}\ env{\isacharparenright}{\kern0pt}\ {\isasymlongleftrightarrow}\ v\ {\isacharequal}{\kern0pt}\ wftrec{\isacharparenleft}{\kern0pt}Rrel{\isacharparenleft}{\kern0pt}R{\isacharcomma}{\kern0pt}\ M{\isacharparenright}{\kern0pt}{\isacharcomma}{\kern0pt}\ x{\isacharcomma}{\kern0pt}\ H{\isacharparenright}{\kern0pt}{\isachardoublequoteclose}\ \isanewline
%
\isadelimproof
%
\endisadelimproof
%
\isatagproof
\isacommand{proof}\isamarkupfalse%
\ {\isacharminus}{\kern0pt}\ \isanewline
\isanewline
\ \ \isacommand{have}\isamarkupfalse%
\ iff{\isacharunderscore}{\kern0pt}lemma{\isadigit{1}}\ {\isacharcolon}{\kern0pt}\ {\isachardoublequoteopen}{\isasymAnd}P\ Q\ R\ S{\isachardot}{\kern0pt}\ P\ {\isasymlongleftrightarrow}\ Q\ {\isasymLongrightarrow}\ {\isacharparenleft}{\kern0pt}Q\ {\isasymLongrightarrow}\ R\ {\isasymlongleftrightarrow}\ S{\isacharparenright}{\kern0pt}\ {\isasymLongrightarrow}\ {\isacharparenleft}{\kern0pt}P\ {\isasymand}\ R\ {\isasymlongleftrightarrow}\ Q\ {\isasymand}\ S{\isacharparenright}{\kern0pt}{\isachardoublequoteclose}\ \isacommand{by}\isamarkupfalse%
\ auto\ \isanewline
\ \ \isacommand{have}\isamarkupfalse%
\ iff{\isacharunderscore}{\kern0pt}lemma{\isadigit{2}}\ {\isacharcolon}{\kern0pt}\ {\isachardoublequoteopen}{\isasymAnd}a\ b\ c{\isachardot}{\kern0pt}\ b\ {\isacharequal}{\kern0pt}\ c\ {\isasymLongrightarrow}\ a\ {\isacharequal}{\kern0pt}\ b\ {\isasymlongleftrightarrow}\ a\ {\isacharequal}{\kern0pt}\ c{\isachardoublequoteclose}\ \isacommand{by}\isamarkupfalse%
\ auto\isanewline
\isanewline
\ \ \isacommand{define}\isamarkupfalse%
\ p\ \isakeyword{where}\ {\isachardoublequoteopen}p\ {\isasymequiv}\ {\isacharbraceleft}{\kern0pt}a{\isacharbraceright}{\kern0pt}{\isachardoublequoteclose}\ \isanewline
\ \ \isacommand{have}\isamarkupfalse%
\ pinM\ {\isacharcolon}{\kern0pt}\ {\isachardoublequoteopen}p\ {\isasymin}\ M{\isachardoublequoteclose}\ \isacommand{using}\isamarkupfalse%
\ singleton{\isacharunderscore}{\kern0pt}in{\isacharunderscore}{\kern0pt}M{\isacharunderscore}{\kern0pt}iff\ assms\ p{\isacharunderscore}{\kern0pt}def\ \isacommand{by}\isamarkupfalse%
\ auto\isanewline
\ \ \isacommand{have}\isamarkupfalse%
\ ainp\ {\isacharcolon}{\kern0pt}\ {\isachardoublequoteopen}a\ {\isasymin}\ p{\isachardoublequoteclose}\ \isacommand{using}\isamarkupfalse%
\ p{\isacharunderscore}{\kern0pt}def\ \isacommand{by}\isamarkupfalse%
\ auto\isanewline
\isanewline
\ \ \isacommand{have}\isamarkupfalse%
\ vinM\ {\isacharcolon}{\kern0pt}\ {\isachardoublequoteopen}v\ {\isasymin}\ M{\isachardoublequoteclose}\ \isacommand{using}\isamarkupfalse%
\ assms\ nth{\isacharunderscore}{\kern0pt}type\ \isacommand{by}\isamarkupfalse%
\ auto\isanewline
\ \ \isacommand{have}\isamarkupfalse%
\ xinM\ {\isacharcolon}{\kern0pt}\ {\isachardoublequoteopen}x\ {\isasymin}\ M{\isachardoublequoteclose}\ \isacommand{using}\isamarkupfalse%
\ assms\ nth{\isacharunderscore}{\kern0pt}type\ \isacommand{by}\isamarkupfalse%
\ auto\isanewline
\ \ \isacommand{have}\isamarkupfalse%
\ ainM\ {\isacharcolon}{\kern0pt}\ {\isachardoublequoteopen}a\ {\isasymin}\ M{\isachardoublequoteclose}\ \isacommand{using}\isamarkupfalse%
\ assms\ nth{\isacharunderscore}{\kern0pt}type\ \isacommand{by}\isamarkupfalse%
\ auto\isanewline
\ \ \isacommand{have}\isamarkupfalse%
\ inat\ {\isacharcolon}{\kern0pt}\ {\isachardoublequoteopen}i\ {\isasymin}\ nat{\isachardoublequoteclose}\ \isacommand{using}\isamarkupfalse%
\ lt{\isacharunderscore}{\kern0pt}nat{\isacharunderscore}{\kern0pt}in{\isacharunderscore}{\kern0pt}nat\ assms\ length{\isacharunderscore}{\kern0pt}type\ \isacommand{by}\isamarkupfalse%
\ auto\isanewline
\ \ \isacommand{have}\isamarkupfalse%
\ jnat\ {\isacharcolon}{\kern0pt}\ {\isachardoublequoteopen}j\ {\isasymin}\ nat{\isachardoublequoteclose}\ \isacommand{using}\isamarkupfalse%
\ lt{\isacharunderscore}{\kern0pt}nat{\isacharunderscore}{\kern0pt}in{\isacharunderscore}{\kern0pt}nat\ assms\ length{\isacharunderscore}{\kern0pt}type\ \isacommand{by}\isamarkupfalse%
\ auto\isanewline
\ \ \isacommand{have}\isamarkupfalse%
\ knat\ {\isacharcolon}{\kern0pt}\ {\isachardoublequoteopen}k\ {\isasymin}\ nat{\isachardoublequoteclose}\ \isacommand{using}\isamarkupfalse%
\ lt{\isacharunderscore}{\kern0pt}nat{\isacharunderscore}{\kern0pt}in{\isacharunderscore}{\kern0pt}nat\ assms\ length{\isacharunderscore}{\kern0pt}type\ \isacommand{by}\isamarkupfalse%
\ auto\isanewline
\isanewline
\ \ \isacommand{have}\isamarkupfalse%
\ eq{\isadigit{1}}\ {\isacharcolon}{\kern0pt}\ {\isachardoublequoteopen}Rrel{\isacharparenleft}{\kern0pt}{\isasymlambda}a\ b{\isachardot}{\kern0pt}\ {\isasymlangle}a{\isacharcomma}{\kern0pt}\ b{\isasymrangle}\ {\isasymin}\ prel{\isacharparenleft}{\kern0pt}Rrel{\isacharparenleft}{\kern0pt}R{\isacharcomma}{\kern0pt}\ M{\isacharparenright}{\kern0pt}{\isacharcomma}{\kern0pt}\ p{\isacharparenright}{\kern0pt}{\isacharcomma}{\kern0pt}\ M{\isacharparenright}{\kern0pt}\ {\isacharequal}{\kern0pt}\ prel{\isacharparenleft}{\kern0pt}Rrel{\isacharparenleft}{\kern0pt}R{\isacharcomma}{\kern0pt}\ M{\isacharparenright}{\kern0pt}{\isacharcomma}{\kern0pt}\ p{\isacharparenright}{\kern0pt}{\isachardoublequoteclose}\isanewline
\ \ \isacommand{proof}\isamarkupfalse%
\ {\isacharparenleft}{\kern0pt}rule\ equality{\isacharunderscore}{\kern0pt}iffI{\isacharcomma}{\kern0pt}\ rule\ iffI{\isacharparenright}{\kern0pt}\isanewline
\ \ \ \ \isacommand{fix}\isamarkupfalse%
\ v\ \isacommand{assume}\isamarkupfalse%
\ {\isachardoublequoteopen}v\ {\isasymin}\ Rrel{\isacharparenleft}{\kern0pt}{\isasymlambda}a\ b{\isachardot}{\kern0pt}\ {\isasymlangle}a{\isacharcomma}{\kern0pt}\ b{\isasymrangle}\ {\isasymin}\ prel{\isacharparenleft}{\kern0pt}Rrel{\isacharparenleft}{\kern0pt}R{\isacharcomma}{\kern0pt}\ M{\isacharparenright}{\kern0pt}{\isacharcomma}{\kern0pt}\ p{\isacharparenright}{\kern0pt}{\isacharcomma}{\kern0pt}\ M{\isacharparenright}{\kern0pt}{\isachardoublequoteclose}\isanewline
\ \ \ \ \isacommand{then}\isamarkupfalse%
\ \isacommand{show}\isamarkupfalse%
\ {\isachardoublequoteopen}v\ {\isasymin}\ prel{\isacharparenleft}{\kern0pt}Rrel{\isacharparenleft}{\kern0pt}R{\isacharcomma}{\kern0pt}\ M{\isacharparenright}{\kern0pt}{\isacharcomma}{\kern0pt}\ p{\isacharparenright}{\kern0pt}{\isachardoublequoteclose}\isanewline
\ \ \ \ \ \ \isacommand{unfolding}\isamarkupfalse%
\ Rrel{\isacharunderscore}{\kern0pt}def\ prel{\isacharunderscore}{\kern0pt}def\ \isanewline
\ \ \ \ \ \ \isacommand{using}\isamarkupfalse%
\ pair{\isacharunderscore}{\kern0pt}in{\isacharunderscore}{\kern0pt}M{\isacharunderscore}{\kern0pt}iff\ \isanewline
\ \ \ \ \ \ \isacommand{by}\isamarkupfalse%
\ auto\isanewline
\ \ \isacommand{next}\isamarkupfalse%
\ \isanewline
\ \ \ \ \isacommand{fix}\isamarkupfalse%
\ v\ \isacommand{assume}\isamarkupfalse%
\ vin{\isacharcolon}{\kern0pt}\ {\isachardoublequoteopen}v\ {\isasymin}\ prel{\isacharparenleft}{\kern0pt}Rrel{\isacharparenleft}{\kern0pt}R{\isacharcomma}{\kern0pt}\ M{\isacharparenright}{\kern0pt}{\isacharcomma}{\kern0pt}\ p{\isacharparenright}{\kern0pt}\ {\isachardoublequoteclose}\isanewline
\ \ \ \ \isacommand{then}\isamarkupfalse%
\ \isacommand{obtain}\isamarkupfalse%
\ a\ b\ q\ \isakeyword{where}\ abqH{\isacharcolon}{\kern0pt}\ {\isachardoublequoteopen}{\isacharless}{\kern0pt}a{\isacharcomma}{\kern0pt}\ b{\isachargreater}{\kern0pt}\ {\isasymin}\ Rrel{\isacharparenleft}{\kern0pt}R{\isacharcomma}{\kern0pt}\ M{\isacharparenright}{\kern0pt}{\isachardoublequoteclose}\ {\isachardoublequoteopen}q\ {\isasymin}\ p{\isachardoublequoteclose}\ {\isachardoublequoteopen}v\ {\isacharequal}{\kern0pt}\ {\isacharless}{\kern0pt}{\isacharless}{\kern0pt}a{\isacharcomma}{\kern0pt}\ q{\isachargreater}{\kern0pt}{\isacharcomma}{\kern0pt}\ {\isacharless}{\kern0pt}b{\isacharcomma}{\kern0pt}\ q{\isachargreater}{\kern0pt}{\isachargreater}{\kern0pt}{\isachardoublequoteclose}\ \isacommand{unfolding}\isamarkupfalse%
\ prel{\isacharunderscore}{\kern0pt}def\ \isacommand{by}\isamarkupfalse%
\ auto\isanewline
\ \ \ \ \isacommand{then}\isamarkupfalse%
\ \isacommand{have}\isamarkupfalse%
\ {\isachardoublequoteopen}q\ {\isasymin}\ M{\isachardoublequoteclose}\ \isacommand{using}\isamarkupfalse%
\ transM\ assms\ pinM\ \isacommand{by}\isamarkupfalse%
\ auto\isanewline
\ \ \ \ \isacommand{then}\isamarkupfalse%
\ \isacommand{have}\isamarkupfalse%
\ aqinM\ {\isacharcolon}{\kern0pt}\ {\isachardoublequoteopen}{\isacharless}{\kern0pt}a{\isacharcomma}{\kern0pt}\ q{\isachargreater}{\kern0pt}\ {\isasymin}\ M{\isachardoublequoteclose}\ \isacommand{using}\isamarkupfalse%
\ abqH\ pair{\isacharunderscore}{\kern0pt}in{\isacharunderscore}{\kern0pt}M{\isacharunderscore}{\kern0pt}iff\ \isacommand{unfolding}\isamarkupfalse%
\ Rrel{\isacharunderscore}{\kern0pt}def\ \isacommand{by}\isamarkupfalse%
\ auto\isanewline
\ \ \ \ \isacommand{then}\isamarkupfalse%
\ \isacommand{have}\isamarkupfalse%
\ bqinM\ {\isacharcolon}{\kern0pt}\ {\isachardoublequoteopen}{\isacharless}{\kern0pt}b{\isacharcomma}{\kern0pt}\ q{\isachargreater}{\kern0pt}\ {\isasymin}\ M{\isachardoublequoteclose}\ \isacommand{using}\isamarkupfalse%
\ abqH\ pair{\isacharunderscore}{\kern0pt}in{\isacharunderscore}{\kern0pt}M{\isacharunderscore}{\kern0pt}iff\ \isacommand{unfolding}\isamarkupfalse%
\ Rrel{\isacharunderscore}{\kern0pt}def\ \isacommand{by}\isamarkupfalse%
\ auto\isanewline
\isanewline
\ \ \ \ \isacommand{show}\isamarkupfalse%
\ {\isachardoublequoteopen}v\ {\isasymin}\ Rrel{\isacharparenleft}{\kern0pt}{\isasymlambda}a\ b{\isachardot}{\kern0pt}\ {\isasymlangle}a{\isacharcomma}{\kern0pt}\ b{\isasymrangle}\ {\isasymin}\ prel{\isacharparenleft}{\kern0pt}Rrel{\isacharparenleft}{\kern0pt}R{\isacharcomma}{\kern0pt}\ M{\isacharparenright}{\kern0pt}{\isacharcomma}{\kern0pt}\ p{\isacharparenright}{\kern0pt}{\isacharcomma}{\kern0pt}\ M{\isacharparenright}{\kern0pt}{\isachardoublequoteclose}\ \isanewline
\ \ \ \ \ \ \isacommand{unfolding}\isamarkupfalse%
\ Rrel{\isacharunderscore}{\kern0pt}def\ \isanewline
\ \ \ \ \ \ \isacommand{apply}\isamarkupfalse%
\ {\isacharparenleft}{\kern0pt}simp{\isacharcomma}{\kern0pt}\ rule\ conjI{\isacharparenright}{\kern0pt}\isanewline
\ \ \ \ \ \ \isacommand{using}\isamarkupfalse%
\ pair{\isacharunderscore}{\kern0pt}in{\isacharunderscore}{\kern0pt}M{\isacharunderscore}{\kern0pt}iff\ aqinM\ bqinM\ abqH\ \isanewline
\ \ \ \ \ \ \ \isacommand{apply}\isamarkupfalse%
\ force\isanewline
\ \ \ \ \ \ \isacommand{apply}\isamarkupfalse%
{\isacharparenleft}{\kern0pt}rule{\isacharunderscore}{\kern0pt}tac\ x{\isacharequal}{\kern0pt}{\isachardoublequoteopen}{\isacharless}{\kern0pt}a{\isacharcomma}{\kern0pt}\ q{\isachargreater}{\kern0pt}{\isachardoublequoteclose}\ \isakeyword{in}\ exI{\isacharcomma}{\kern0pt}\ rule{\isacharunderscore}{\kern0pt}tac\ x{\isacharequal}{\kern0pt}{\isachardoublequoteopen}{\isacharless}{\kern0pt}b{\isacharcomma}{\kern0pt}\ q{\isachargreater}{\kern0pt}{\isachardoublequoteclose}\ \isakeyword{in}\ exI{\isacharcomma}{\kern0pt}\ rule\ conjI{\isacharcomma}{\kern0pt}\ simp\ add{\isacharcolon}{\kern0pt}abqH{\isacharparenright}{\kern0pt}\isanewline
\ \ \ \ \ \ \isacommand{using}\isamarkupfalse%
\ abqH\ vin\ \isanewline
\ \ \ \ \ \ \isacommand{unfolding}\isamarkupfalse%
\ Rrel{\isacharunderscore}{\kern0pt}def\isanewline
\ \ \ \ \ \ \isacommand{by}\isamarkupfalse%
\ auto\isanewline
\ \ \isacommand{qed}\isamarkupfalse%
\isanewline
\isanewline
\ \ \isacommand{have}\isamarkupfalse%
\ eq{\isadigit{2}}\ {\isacharcolon}{\kern0pt}\ {\isachardoublequoteopen}prel{\isacharparenleft}{\kern0pt}preds{\isacharunderscore}{\kern0pt}rel{\isacharparenleft}{\kern0pt}R{\isacharcomma}{\kern0pt}\ x{\isacharparenright}{\kern0pt}{\isacharcomma}{\kern0pt}\ {\isacharbraceleft}{\kern0pt}a{\isacharbraceright}{\kern0pt}{\isacharparenright}{\kern0pt}\ {\isacharequal}{\kern0pt}\ preds{\isacharunderscore}{\kern0pt}rel{\isacharparenleft}{\kern0pt}{\isasymlambda}a\ b{\isachardot}{\kern0pt}\ {\isasymlangle}a{\isacharcomma}{\kern0pt}\ b{\isasymrangle}\ {\isasymin}\ prel{\isacharparenleft}{\kern0pt}Rrel{\isacharparenleft}{\kern0pt}R{\isacharcomma}{\kern0pt}\ M{\isacharparenright}{\kern0pt}{\isacharcomma}{\kern0pt}\ p{\isacharparenright}{\kern0pt}{\isacharcomma}{\kern0pt}\ {\isasymlangle}x{\isacharcomma}{\kern0pt}\ a{\isasymrangle}{\isacharparenright}{\kern0pt}{\isachardoublequoteclose}\ \isanewline
\ \ \isacommand{proof}\isamarkupfalse%
{\isacharparenleft}{\kern0pt}rule\ equality{\isacharunderscore}{\kern0pt}iffI{\isacharcomma}{\kern0pt}\ rule\ iffI{\isacharparenright}{\kern0pt}\isanewline
\ \ \ \ \isacommand{fix}\isamarkupfalse%
\ s\ \isacommand{assume}\isamarkupfalse%
\ assms{\isadigit{1}}\ {\isacharcolon}{\kern0pt}\ {\isachardoublequoteopen}s\ {\isasymin}\ prel{\isacharparenleft}{\kern0pt}preds{\isacharunderscore}{\kern0pt}rel{\isacharparenleft}{\kern0pt}R{\isacharcomma}{\kern0pt}\ x{\isacharparenright}{\kern0pt}{\isacharcomma}{\kern0pt}\ {\isacharbraceleft}{\kern0pt}a{\isacharbraceright}{\kern0pt}{\isacharparenright}{\kern0pt}{\isachardoublequoteclose}\ \isanewline
\ \ \ \ \isacommand{then}\isamarkupfalse%
\ \isacommand{obtain}\isamarkupfalse%
\ b\ c\ \isakeyword{where}\ bcH\ {\isacharcolon}{\kern0pt}\ {\isachardoublequoteopen}s\ {\isacharequal}{\kern0pt}\ {\isacharless}{\kern0pt}{\isacharless}{\kern0pt}b{\isacharcomma}{\kern0pt}\ a{\isachargreater}{\kern0pt}{\isacharcomma}{\kern0pt}\ {\isacharless}{\kern0pt}c{\isacharcomma}{\kern0pt}\ a{\isachargreater}{\kern0pt}{\isachargreater}{\kern0pt}{\isachardoublequoteclose}\ {\isachardoublequoteopen}{\isacharless}{\kern0pt}b{\isacharcomma}{\kern0pt}\ c{\isachargreater}{\kern0pt}\ {\isasymin}\ preds{\isacharunderscore}{\kern0pt}rel{\isacharparenleft}{\kern0pt}R{\isacharcomma}{\kern0pt}\ x{\isacharparenright}{\kern0pt}{\isachardoublequoteclose}\ \isacommand{unfolding}\isamarkupfalse%
\ prel{\isacharunderscore}{\kern0pt}def\ \isacommand{by}\isamarkupfalse%
\ auto\ \isanewline
\isanewline
\ \ \ \ \isacommand{then}\isamarkupfalse%
\ \isacommand{have}\isamarkupfalse%
\ {\isachardoublequoteopen}b\ {\isasymin}\ preds{\isacharparenleft}{\kern0pt}R{\isacharcomma}{\kern0pt}\ x{\isacharparenright}{\kern0pt}{\isachardoublequoteclose}\ \isacommand{unfolding}\isamarkupfalse%
\ preds{\isacharunderscore}{\kern0pt}rel{\isacharunderscore}{\kern0pt}def\ \isacommand{by}\isamarkupfalse%
\ auto\ \isanewline
\ \ \ \ \isacommand{then}\isamarkupfalse%
\ \isacommand{have}\isamarkupfalse%
\ H{\isadigit{1}}{\isacharcolon}{\kern0pt}{\isachardoublequoteopen}{\isasymlangle}b{\isacharcomma}{\kern0pt}\ a{\isasymrangle}\ {\isasymin}\ preds{\isacharparenleft}{\kern0pt}{\isasymlambda}a\ b{\isachardot}{\kern0pt}\ {\isasymlangle}a{\isacharcomma}{\kern0pt}\ b{\isasymrangle}\ {\isasymin}\ prel{\isacharparenleft}{\kern0pt}Rrel{\isacharparenleft}{\kern0pt}R{\isacharcomma}{\kern0pt}\ M{\isacharparenright}{\kern0pt}{\isacharcomma}{\kern0pt}\ p{\isacharparenright}{\kern0pt}{\isacharcomma}{\kern0pt}\ {\isasymlangle}x{\isacharcomma}{\kern0pt}\ a{\isasymrangle}{\isacharparenright}{\kern0pt}{\isachardoublequoteclose}\ \isanewline
\ \ \ \ \ \ \isacommand{unfolding}\isamarkupfalse%
\ preds{\isacharunderscore}{\kern0pt}def\ \isanewline
\ \ \ \ \ \ \isacommand{apply}\isamarkupfalse%
\ simp\isanewline
\ \ \ \ \ \ \isacommand{apply}\isamarkupfalse%
{\isacharparenleft}{\kern0pt}rule\ conjI{\isacharparenright}{\kern0pt}\isanewline
\ \ \ \ \ \ \isacommand{using}\isamarkupfalse%
\ pair{\isacharunderscore}{\kern0pt}in{\isacharunderscore}{\kern0pt}M{\isacharunderscore}{\kern0pt}iff\ assms\ transM\ \isanewline
\ \ \ \ \ \ \ \isacommand{apply}\isamarkupfalse%
\ force\isanewline
\ \ \ \ \ \ \isacommand{apply}\isamarkupfalse%
{\isacharparenleft}{\kern0pt}rule\ prelI{\isacharparenright}{\kern0pt}\isanewline
\ \ \ \ \ \ \isacommand{unfolding}\isamarkupfalse%
\ Rrel{\isacharunderscore}{\kern0pt}def\ p{\isacharunderscore}{\kern0pt}def\isanewline
\ \ \ \ \ \ \isacommand{using}\isamarkupfalse%
\ assms\ \isanewline
\ \ \ \ \ \ \isacommand{by}\isamarkupfalse%
\ auto\isanewline
\ \ \ \ \isacommand{then}\isamarkupfalse%
\ \isacommand{have}\isamarkupfalse%
\ {\isachardoublequoteopen}c\ {\isasymin}\ preds{\isacharparenleft}{\kern0pt}R{\isacharcomma}{\kern0pt}\ x{\isacharparenright}{\kern0pt}\ {\isasymunion}\ {\isacharbraceleft}{\kern0pt}x{\isacharbraceright}{\kern0pt}{\isachardoublequoteclose}\ \isacommand{using}\isamarkupfalse%
\ bcH\ \isacommand{unfolding}\isamarkupfalse%
\ preds{\isacharunderscore}{\kern0pt}rel{\isacharunderscore}{\kern0pt}def\ \isacommand{by}\isamarkupfalse%
\ auto\ \isanewline
\ \ \ \ \isacommand{then}\isamarkupfalse%
\ \isacommand{have}\isamarkupfalse%
\ H{\isadigit{2}}{\isacharcolon}{\kern0pt}{\isachardoublequoteopen}{\isasymlangle}c{\isacharcomma}{\kern0pt}\ a{\isasymrangle}\ {\isasymin}\ preds{\isacharparenleft}{\kern0pt}{\isasymlambda}a\ b{\isachardot}{\kern0pt}\ {\isasymlangle}a{\isacharcomma}{\kern0pt}\ b{\isasymrangle}\ {\isasymin}\ prel{\isacharparenleft}{\kern0pt}Rrel{\isacharparenleft}{\kern0pt}R{\isacharcomma}{\kern0pt}\ M{\isacharparenright}{\kern0pt}{\isacharcomma}{\kern0pt}\ p{\isacharparenright}{\kern0pt}{\isacharcomma}{\kern0pt}\ {\isasymlangle}x{\isacharcomma}{\kern0pt}\ a{\isasymrangle}{\isacharparenright}{\kern0pt}\ {\isasymor}\ c\ {\isacharequal}{\kern0pt}\ x{\isachardoublequoteclose}\isanewline
\ \ \ \ \ \ \isacommand{apply}\isamarkupfalse%
\ auto{\isacharbrackleft}{\kern0pt}{\isadigit{1}}{\isacharbrackright}{\kern0pt}\isanewline
\ \ \ \ \ \ \isacommand{unfolding}\isamarkupfalse%
\ preds{\isacharunderscore}{\kern0pt}def\isanewline
\ \ \ \ \ \ \isacommand{apply}\isamarkupfalse%
\ simp\isanewline
\ \ \ \ \ \ \isacommand{apply}\isamarkupfalse%
{\isacharparenleft}{\kern0pt}rule\ conjI{\isacharparenright}{\kern0pt}\isanewline
\ \ \ \ \ \ \isacommand{using}\isamarkupfalse%
\ pair{\isacharunderscore}{\kern0pt}in{\isacharunderscore}{\kern0pt}M{\isacharunderscore}{\kern0pt}iff\ assms\ transM\ \isanewline
\ \ \ \ \ \ \ \isacommand{apply}\isamarkupfalse%
\ force\isanewline
\ \ \ \ \ \ \isacommand{apply}\isamarkupfalse%
{\isacharparenleft}{\kern0pt}rule\ prelI{\isacharparenright}{\kern0pt}\isanewline
\ \ \ \ \ \ \isacommand{unfolding}\isamarkupfalse%
\ Rrel{\isacharunderscore}{\kern0pt}def\ \isanewline
\ \ \ \ \ \ \isacommand{using}\isamarkupfalse%
\ assms\ p{\isacharunderscore}{\kern0pt}def\isanewline
\ \ \ \ \ \ \isacommand{by}\isamarkupfalse%
\ auto\ \ \ \ \ \ \isanewline
\isanewline
\ \ \ \ \isacommand{have}\isamarkupfalse%
\ H{\isadigit{3}}\ {\isacharcolon}{\kern0pt}\ {\isachardoublequoteopen}{\isacharless}{\kern0pt}b{\isacharcomma}{\kern0pt}\ c{\isachargreater}{\kern0pt}\ {\isasymin}\ Rrel{\isacharparenleft}{\kern0pt}R{\isacharcomma}{\kern0pt}\ M{\isacharparenright}{\kern0pt}{\isachardoublequoteclose}\ \isanewline
\ \ \ \ \ \ \isacommand{unfolding}\isamarkupfalse%
\ Rrel{\isacharunderscore}{\kern0pt}def\ \isanewline
\ \ \ \ \ \ \isacommand{apply}\isamarkupfalse%
\ simp\isanewline
\ \ \ \ \ \ \isacommand{using}\isamarkupfalse%
\ bcH\ pair{\isacharunderscore}{\kern0pt}in{\isacharunderscore}{\kern0pt}M{\isacharunderscore}{\kern0pt}iff\ assms\isanewline
\ \ \ \ \ \ \isacommand{unfolding}\isamarkupfalse%
\ preds{\isacharunderscore}{\kern0pt}rel{\isacharunderscore}{\kern0pt}def\ preds{\isacharunderscore}{\kern0pt}def\ \isanewline
\ \ \ \ \ \ \isacommand{by}\isamarkupfalse%
\ auto\isanewline
\isanewline
\ \ \ \ \isacommand{have}\isamarkupfalse%
\ {\isachardoublequoteopen}{\isacharless}{\kern0pt}{\isacharless}{\kern0pt}b{\isacharcomma}{\kern0pt}\ a{\isachargreater}{\kern0pt}{\isacharcomma}{\kern0pt}\ {\isacharless}{\kern0pt}c{\isacharcomma}{\kern0pt}\ a{\isachargreater}{\kern0pt}{\isachargreater}{\kern0pt}\ {\isasymin}\ preds{\isacharunderscore}{\kern0pt}rel{\isacharparenleft}{\kern0pt}{\isasymlambda}a\ b{\isachardot}{\kern0pt}\ {\isasymlangle}a{\isacharcomma}{\kern0pt}\ b{\isasymrangle}\ {\isasymin}\ prel{\isacharparenleft}{\kern0pt}Rrel{\isacharparenleft}{\kern0pt}R{\isacharcomma}{\kern0pt}\ M{\isacharparenright}{\kern0pt}{\isacharcomma}{\kern0pt}\ p{\isacharparenright}{\kern0pt}{\isacharcomma}{\kern0pt}\ {\isasymlangle}x{\isacharcomma}{\kern0pt}\ a{\isasymrangle}{\isacharparenright}{\kern0pt}{\isachardoublequoteclose}\ \isanewline
\ \ \ \ \ \ \isacommand{unfolding}\isamarkupfalse%
\ preds{\isacharunderscore}{\kern0pt}rel{\isacharunderscore}{\kern0pt}def\ \isanewline
\ \ \ \ \ \ \isacommand{using}\isamarkupfalse%
\ H{\isadigit{1}}\ H{\isadigit{2}}\ \isanewline
\ \ \ \ \ \ \isacommand{apply}\isamarkupfalse%
\ simp\isanewline
\ \ \ \ \ \ \isacommand{apply}\isamarkupfalse%
{\isacharparenleft}{\kern0pt}rule\ prelI{\isacharparenright}{\kern0pt}\isanewline
\ \ \ \ \ \ \isacommand{using}\isamarkupfalse%
\ H{\isadigit{3}}\ assms\ p{\isacharunderscore}{\kern0pt}def\isanewline
\ \ \ \ \ \ \isacommand{by}\isamarkupfalse%
\ auto\isanewline
\isanewline
\ \ \ \ \isacommand{then}\isamarkupfalse%
\ \isacommand{show}\isamarkupfalse%
\ {\isachardoublequoteopen}s\ {\isasymin}\ preds{\isacharunderscore}{\kern0pt}rel{\isacharparenleft}{\kern0pt}{\isasymlambda}a\ b{\isachardot}{\kern0pt}\ {\isasymlangle}a{\isacharcomma}{\kern0pt}\ b{\isasymrangle}\ {\isasymin}\ prel{\isacharparenleft}{\kern0pt}Rrel{\isacharparenleft}{\kern0pt}R{\isacharcomma}{\kern0pt}\ M{\isacharparenright}{\kern0pt}{\isacharcomma}{\kern0pt}\ p{\isacharparenright}{\kern0pt}{\isacharcomma}{\kern0pt}\ {\isasymlangle}x{\isacharcomma}{\kern0pt}\ a{\isasymrangle}{\isacharparenright}{\kern0pt}{\isachardoublequoteclose}\ \isacommand{using}\isamarkupfalse%
\ bcH\ \isacommand{by}\isamarkupfalse%
\ auto\ \isanewline
\ \ \isacommand{next}\isamarkupfalse%
\ \isanewline
\ \ \ \ \isacommand{fix}\isamarkupfalse%
\ s\ \isacommand{assume}\isamarkupfalse%
\ {\isachardoublequoteopen}s\ {\isasymin}\ preds{\isacharunderscore}{\kern0pt}rel{\isacharparenleft}{\kern0pt}{\isasymlambda}a\ b{\isachardot}{\kern0pt}\ {\isasymlangle}a{\isacharcomma}{\kern0pt}\ b{\isasymrangle}\ {\isasymin}\ prel{\isacharparenleft}{\kern0pt}Rrel{\isacharparenleft}{\kern0pt}R{\isacharcomma}{\kern0pt}\ M{\isacharparenright}{\kern0pt}{\isacharcomma}{\kern0pt}\ p{\isacharparenright}{\kern0pt}{\isacharcomma}{\kern0pt}\ {\isasymlangle}x{\isacharcomma}{\kern0pt}\ a{\isasymrangle}{\isacharparenright}{\kern0pt}{\isachardoublequoteclose}\ \isanewline
\ \ \ \ \isacommand{then}\isamarkupfalse%
\ \isacommand{obtain}\isamarkupfalse%
\ ba\ ca\ \isakeyword{where}\ bacaH{\isacharcolon}{\kern0pt}\isanewline
\ \ \ \ \ \ {\isachardoublequoteopen}s\ {\isacharequal}{\kern0pt}\ {\isacharless}{\kern0pt}ba{\isacharcomma}{\kern0pt}\ ca{\isachargreater}{\kern0pt}{\isachardoublequoteclose}\ \isanewline
\ \ \ \ \ \ {\isachardoublequoteopen}{\isacharless}{\kern0pt}ba{\isacharcomma}{\kern0pt}\ ca{\isachargreater}{\kern0pt}\ {\isasymin}\ prel{\isacharparenleft}{\kern0pt}Rrel{\isacharparenleft}{\kern0pt}R{\isacharcomma}{\kern0pt}\ M{\isacharparenright}{\kern0pt}{\isacharcomma}{\kern0pt}\ p{\isacharparenright}{\kern0pt}{\isachardoublequoteclose}\ \isanewline
\ \ \ \ \ \ {\isachardoublequoteopen}ba\ {\isasymin}\ preds{\isacharparenleft}{\kern0pt}{\isasymlambda}a\ b{\isachardot}{\kern0pt}\ {\isasymlangle}a{\isacharcomma}{\kern0pt}\ b{\isasymrangle}\ {\isasymin}\ prel{\isacharparenleft}{\kern0pt}Rrel{\isacharparenleft}{\kern0pt}R{\isacharcomma}{\kern0pt}\ M{\isacharparenright}{\kern0pt}{\isacharcomma}{\kern0pt}\ p{\isacharparenright}{\kern0pt}{\isacharcomma}{\kern0pt}\ {\isasymlangle}x{\isacharcomma}{\kern0pt}\ a{\isasymrangle}{\isacharparenright}{\kern0pt}{\isachardoublequoteclose}\ \isanewline
\ \ \ \ \ \ {\isachardoublequoteopen}ca\ {\isasymin}\ {\isacharparenleft}{\kern0pt}preds{\isacharparenleft}{\kern0pt}{\isasymlambda}a\ b{\isachardot}{\kern0pt}\ {\isasymlangle}a{\isacharcomma}{\kern0pt}\ b{\isasymrangle}\ {\isasymin}\ prel{\isacharparenleft}{\kern0pt}Rrel{\isacharparenleft}{\kern0pt}R{\isacharcomma}{\kern0pt}\ M{\isacharparenright}{\kern0pt}{\isacharcomma}{\kern0pt}\ p{\isacharparenright}{\kern0pt}{\isacharcomma}{\kern0pt}\ {\isasymlangle}x{\isacharcomma}{\kern0pt}\ a{\isasymrangle}{\isacharparenright}{\kern0pt}\ {\isasymunion}\ {\isacharbraceleft}{\kern0pt}{\isasymlangle}x{\isacharcomma}{\kern0pt}\ a{\isasymrangle}{\isacharbraceright}{\kern0pt}{\isacharparenright}{\kern0pt}{\isachardoublequoteclose}\ \isanewline
\ \ \ \ \ \ \isacommand{unfolding}\isamarkupfalse%
\ preds{\isacharunderscore}{\kern0pt}rel{\isacharunderscore}{\kern0pt}def\ \isacommand{by}\isamarkupfalse%
\ auto\isanewline
\ \ \ \ \isanewline
\ \ \ \ \isacommand{obtain}\isamarkupfalse%
\ b\ \isakeyword{where}\ baeq{\isacharcolon}{\kern0pt}\ {\isachardoublequoteopen}ba\ {\isacharequal}{\kern0pt}\ {\isacharless}{\kern0pt}b{\isacharcomma}{\kern0pt}\ a{\isachargreater}{\kern0pt}{\isachardoublequoteclose}\ \isacommand{using}\isamarkupfalse%
\ bacaH\ \isacommand{unfolding}\isamarkupfalse%
\ preds{\isacharunderscore}{\kern0pt}def\ prel{\isacharunderscore}{\kern0pt}def\ \isacommand{by}\isamarkupfalse%
\ auto\isanewline
\ \ \ \ \isacommand{obtain}\isamarkupfalse%
\ c\ \isakeyword{where}\ caeq{\isacharcolon}{\kern0pt}\ {\isachardoublequoteopen}ca\ {\isacharequal}{\kern0pt}\ {\isacharless}{\kern0pt}c{\isacharcomma}{\kern0pt}\ a{\isachargreater}{\kern0pt}{\isachardoublequoteclose}\ \isacommand{using}\isamarkupfalse%
\ bacaH\ \isacommand{unfolding}\isamarkupfalse%
\ preds{\isacharunderscore}{\kern0pt}def\ prel{\isacharunderscore}{\kern0pt}def\ \isacommand{by}\isamarkupfalse%
\ auto\ \isanewline
\isanewline
\ \ \ \ \isacommand{have}\isamarkupfalse%
\ {\isachardoublequoteopen}ba\ {\isasymin}\ M{\isachardoublequoteclose}\ \isacommand{using}\isamarkupfalse%
\ bacaH\ \isacommand{unfolding}\isamarkupfalse%
\ preds{\isacharunderscore}{\kern0pt}def\ prel{\isacharunderscore}{\kern0pt}def\ \isacommand{by}\isamarkupfalse%
\ auto\ \isanewline
\ \ \ \ \isacommand{then}\isamarkupfalse%
\ \isacommand{have}\isamarkupfalse%
\ binM\ {\isacharcolon}{\kern0pt}\ {\isachardoublequoteopen}b\ {\isasymin}\ M{\isachardoublequoteclose}\ \isacommand{using}\isamarkupfalse%
\ baeq\ pair{\isacharunderscore}{\kern0pt}in{\isacharunderscore}{\kern0pt}M{\isacharunderscore}{\kern0pt}iff\ \isacommand{by}\isamarkupfalse%
\ auto\ \isanewline
\ \ \ \ \isacommand{have}\isamarkupfalse%
\ {\isachardoublequoteopen}ca\ {\isasymin}\ M{\isachardoublequoteclose}\ \isacommand{using}\isamarkupfalse%
\ bacaH\ assms\ transM\ pair{\isacharunderscore}{\kern0pt}in{\isacharunderscore}{\kern0pt}M{\isacharunderscore}{\kern0pt}iff\ \isacommand{unfolding}\isamarkupfalse%
\ preds{\isacharunderscore}{\kern0pt}def\ prel{\isacharunderscore}{\kern0pt}def\ \isacommand{by}\isamarkupfalse%
\ auto\ \isanewline
\ \ \ \ \isacommand{then}\isamarkupfalse%
\ \isacommand{have}\isamarkupfalse%
\ cinM\ {\isacharcolon}{\kern0pt}\ {\isachardoublequoteopen}c\ {\isasymin}\ M{\isachardoublequoteclose}\ \isacommand{using}\isamarkupfalse%
\ caeq\ pair{\isacharunderscore}{\kern0pt}in{\isacharunderscore}{\kern0pt}M{\isacharunderscore}{\kern0pt}iff\ \isacommand{by}\isamarkupfalse%
\ auto\ \isanewline
\isanewline
\ \ \ \ \isacommand{have}\isamarkupfalse%
\ bin\ {\isacharcolon}{\kern0pt}\ {\isachardoublequoteopen}b\ {\isasymin}\ preds{\isacharparenleft}{\kern0pt}R{\isacharcomma}{\kern0pt}\ x{\isacharparenright}{\kern0pt}{\isachardoublequoteclose}\ \isanewline
\ \ \ \ \ \ \isacommand{unfolding}\isamarkupfalse%
\ preds{\isacharunderscore}{\kern0pt}def\ \isanewline
\ \ \ \ \ \ \isacommand{apply}\isamarkupfalse%
\ {\isacharparenleft}{\kern0pt}simp\ add{\isacharcolon}{\kern0pt}binM{\isacharparenright}{\kern0pt}\isanewline
\ \ \ \ \ \ \isacommand{using}\isamarkupfalse%
\ bacaH\ baeq\ \isanewline
\ \ \ \ \ \ \isacommand{unfolding}\isamarkupfalse%
\ preds{\isacharunderscore}{\kern0pt}def\ prel{\isacharunderscore}{\kern0pt}def\ Rrel{\isacharunderscore}{\kern0pt}def\ \isanewline
\ \ \ \ \ \ \isacommand{by}\isamarkupfalse%
\ force\isanewline
\ \ \ \ \ \ \isanewline
\ \ \ \ \isacommand{have}\isamarkupfalse%
\ cin\ {\isacharcolon}{\kern0pt}\ {\isachardoublequoteopen}c\ {\isasymin}\ preds{\isacharparenleft}{\kern0pt}R{\isacharcomma}{\kern0pt}\ x{\isacharparenright}{\kern0pt}\ {\isasymor}\ c\ {\isacharequal}{\kern0pt}\ x{\isachardoublequoteclose}\ \isanewline
\ \ \ \ \ \ \isacommand{unfolding}\isamarkupfalse%
\ preds{\isacharunderscore}{\kern0pt}def\ \isanewline
\ \ \ \ \ \ \isacommand{apply}\isamarkupfalse%
\ {\isacharparenleft}{\kern0pt}simp\ add{\isacharcolon}{\kern0pt}cinM{\isacharparenright}{\kern0pt}\isanewline
\ \ \ \ \ \ \isacommand{using}\isamarkupfalse%
\ bacaH\ caeq\ \isanewline
\ \ \ \ \ \ \isacommand{unfolding}\isamarkupfalse%
\ preds{\isacharunderscore}{\kern0pt}def\ prel{\isacharunderscore}{\kern0pt}def\ Rrel{\isacharunderscore}{\kern0pt}def\ \isanewline
\ \ \ \ \ \ \isacommand{by}\isamarkupfalse%
\ force\isanewline
\isanewline
\ \ \ \ \isacommand{have}\isamarkupfalse%
\ {\isachardoublequoteopen}{\isacharless}{\kern0pt}{\isacharless}{\kern0pt}b{\isacharcomma}{\kern0pt}\ a{\isachargreater}{\kern0pt}{\isacharcomma}{\kern0pt}\ {\isacharless}{\kern0pt}c{\isacharcomma}{\kern0pt}\ a{\isachargreater}{\kern0pt}{\isachargreater}{\kern0pt}\ {\isasymin}\ prel{\isacharparenleft}{\kern0pt}preds{\isacharunderscore}{\kern0pt}rel{\isacharparenleft}{\kern0pt}R{\isacharcomma}{\kern0pt}\ x{\isacharparenright}{\kern0pt}{\isacharcomma}{\kern0pt}\ {\isacharbraceleft}{\kern0pt}a{\isacharbraceright}{\kern0pt}{\isacharparenright}{\kern0pt}{\isachardoublequoteclose}\ \isanewline
\ \ \ \ \ \ \isacommand{apply}\isamarkupfalse%
{\isacharparenleft}{\kern0pt}rule\ prelI{\isacharparenright}{\kern0pt}\isanewline
\ \ \ \ \ \ \isacommand{unfolding}\isamarkupfalse%
\ preds{\isacharunderscore}{\kern0pt}rel{\isacharunderscore}{\kern0pt}def\ \isanewline
\ \ \ \ \ \ \ \isacommand{apply}\isamarkupfalse%
\ {\isacharparenleft}{\kern0pt}simp\ add{\isacharcolon}{\kern0pt}bin\ cin{\isacharparenright}{\kern0pt}\isanewline
\ \ \ \ \ \ \isacommand{using}\isamarkupfalse%
\ bacaH\ baeq\ caeq\ \isanewline
\ \ \ \ \ \ \isacommand{unfolding}\isamarkupfalse%
\ prel{\isacharunderscore}{\kern0pt}def\ Rrel{\isacharunderscore}{\kern0pt}def\ \isanewline
\ \ \ \ \ \ \isacommand{by}\isamarkupfalse%
\ auto\ \isanewline
\ \ \ \ \isacommand{then}\isamarkupfalse%
\ \isacommand{show}\isamarkupfalse%
\ {\isachardoublequoteopen}s\ {\isasymin}\ prel{\isacharparenleft}{\kern0pt}preds{\isacharunderscore}{\kern0pt}rel{\isacharparenleft}{\kern0pt}R{\isacharcomma}{\kern0pt}\ x{\isacharparenright}{\kern0pt}{\isacharcomma}{\kern0pt}\ {\isacharbraceleft}{\kern0pt}a{\isacharbraceright}{\kern0pt}{\isacharparenright}{\kern0pt}{\isachardoublequoteclose}\ \isacommand{using}\isamarkupfalse%
\ bacaH\ baeq\ caeq\ \isacommand{by}\isamarkupfalse%
\ auto\ \isanewline
\ \ \isacommand{qed}\isamarkupfalse%
\isanewline
\isanewline
\ \ \isacommand{have}\isamarkupfalse%
\ I{\isadigit{1}}{\isacharcolon}{\kern0pt}\ {\isachardoublequoteopen}sats{\isacharparenleft}{\kern0pt}M{\isacharcomma}{\kern0pt}\ is{\isacharunderscore}{\kern0pt}wftrec{\isacharunderscore}{\kern0pt}fm{\isacharparenleft}{\kern0pt}Gfm{\isacharcomma}{\kern0pt}\ Rfm{\isacharcomma}{\kern0pt}\ i{\isacharcomma}{\kern0pt}\ j{\isacharcomma}{\kern0pt}\ k{\isacharparenright}{\kern0pt}{\isacharcomma}{\kern0pt}\ env{\isacharparenright}{\kern0pt}\ {\isasymlongleftrightarrow}\ {\isacharparenleft}{\kern0pt}{\isasymexists}xa\ {\isasymin}\ M{\isachardot}{\kern0pt}\ {\isasymexists}S\ {\isasymin}\ M{\isachardot}{\kern0pt}\ S\ {\isacharequal}{\kern0pt}\ preds{\isacharunderscore}{\kern0pt}rel{\isacharparenleft}{\kern0pt}{\isasymlambda}a\ b{\isachardot}{\kern0pt}\ {\isasymlangle}a{\isacharcomma}{\kern0pt}\ b{\isasymrangle}\ {\isasymin}\ prel{\isacharparenleft}{\kern0pt}Rrel{\isacharparenleft}{\kern0pt}R{\isacharcomma}{\kern0pt}\ M{\isacharparenright}{\kern0pt}{\isacharcomma}{\kern0pt}\ p{\isacharparenright}{\kern0pt}{\isacharcomma}{\kern0pt}\ {\isasymlangle}x{\isacharcomma}{\kern0pt}\ a{\isasymrangle}{\isacharparenright}{\kern0pt}\ {\isasymand}\ xa\ {\isacharequal}{\kern0pt}\ {\isacharless}{\kern0pt}x{\isacharcomma}{\kern0pt}\ a{\isachargreater}{\kern0pt}\ {\isasymand}\ v\ {\isacharequal}{\kern0pt}\ wftrec{\isacharparenleft}{\kern0pt}S{\isacharcomma}{\kern0pt}\ xa{\isacharcomma}{\kern0pt}\ G{\isacharparenright}{\kern0pt}{\isacharparenright}{\kern0pt}{\isachardoublequoteclose}\ \isanewline
\ \ \ \ \isacommand{unfolding}\isamarkupfalse%
\ is{\isacharunderscore}{\kern0pt}wftrec{\isacharunderscore}{\kern0pt}fm{\isacharunderscore}{\kern0pt}def\isanewline
\isanewline
\ \ \ \ \isacommand{apply}\isamarkupfalse%
{\isacharparenleft}{\kern0pt}rule\ iff{\isacharunderscore}{\kern0pt}trans{\isacharcomma}{\kern0pt}\ rule\ sats{\isacharunderscore}{\kern0pt}Exists{\isacharunderscore}{\kern0pt}iff{\isacharcomma}{\kern0pt}\ simp\ add{\isacharcolon}{\kern0pt}assms{\isacharcomma}{\kern0pt}\ rule\ bex{\isacharunderscore}{\kern0pt}iff{\isacharparenright}{\kern0pt}\isanewline
\ \ \ \ \isacommand{apply}\isamarkupfalse%
{\isacharparenleft}{\kern0pt}rule\ iff{\isacharunderscore}{\kern0pt}trans{\isacharcomma}{\kern0pt}\ rule\ sats{\isacharunderscore}{\kern0pt}Exists{\isacharunderscore}{\kern0pt}iff{\isacharcomma}{\kern0pt}\ simp\ add{\isacharcolon}{\kern0pt}assms{\isacharcomma}{\kern0pt}\ rule\ bex{\isacharunderscore}{\kern0pt}iff{\isacharparenright}{\kern0pt}\isanewline
\ \ \ \ \isacommand{apply}\isamarkupfalse%
{\isacharparenleft}{\kern0pt}rule\ iff{\isacharunderscore}{\kern0pt}trans{\isacharcomma}{\kern0pt}\ rule\ sats{\isacharunderscore}{\kern0pt}And{\isacharunderscore}{\kern0pt}iff{\isacharcomma}{\kern0pt}\ simp\ add{\isacharcolon}{\kern0pt}assms{\isacharcomma}{\kern0pt}\ rule\ iff{\isacharunderscore}{\kern0pt}lemma{\isadigit{1}}{\isacharparenright}{\kern0pt}\isanewline
\ \ \ \ \ \isacommand{apply}\isamarkupfalse%
{\isacharparenleft}{\kern0pt}rule\ sats{\isacharunderscore}{\kern0pt}is{\isacharunderscore}{\kern0pt}preds{\isacharunderscore}{\kern0pt}prel{\isacharunderscore}{\kern0pt}fm{\isacharunderscore}{\kern0pt}iff{\isacharparenright}{\kern0pt}\isanewline
\ \ \ \ \isacommand{using}\isamarkupfalse%
\ assms\ inat\ jnat\ knat\ xinM\ ainp\ pinM\isanewline
\ \ \ \ \ \ \ \ \ \ \ \ \ \ \ \ \ \ \isacommand{apply}\isamarkupfalse%
\ simp{\isacharunderscore}{\kern0pt}all\isanewline
\ \ \ \ \isacommand{apply}\isamarkupfalse%
{\isacharparenleft}{\kern0pt}rule\ iff{\isacharunderscore}{\kern0pt}lemma{\isadigit{1}}{\isacharcomma}{\kern0pt}\ simp{\isacharcomma}{\kern0pt}\ rule\ sats{\isacharunderscore}{\kern0pt}is{\isacharunderscore}{\kern0pt}wfrec{\isacharunderscore}{\kern0pt}fm{\isacharunderscore}{\kern0pt}iff{\isacharparenright}{\kern0pt}\isanewline
\ \ \ \ \isacommand{using}\isamarkupfalse%
\ assms\ inat\ jnat\ knat\ xinM\ ainM\ vinM\ \isanewline
\ \ \ \ \ \ \ \ \ \ \ \ \ \ \ \ \ \ \ \ \ \ \isacommand{apply}\isamarkupfalse%
\ simp{\isacharunderscore}{\kern0pt}all\isanewline
\ \ \ \ \ \isacommand{apply}\isamarkupfalse%
{\isacharparenleft}{\kern0pt}rule\ wf{\isacharunderscore}{\kern0pt}preds{\isacharunderscore}{\kern0pt}rel{\isacharparenright}{\kern0pt}\isanewline
\ \ \ \ \isacommand{using}\isamarkupfalse%
\ pair{\isacharunderscore}{\kern0pt}in{\isacharunderscore}{\kern0pt}M{\isacharunderscore}{\kern0pt}iff\ \isanewline
\ \ \ \ \ \ \isacommand{apply}\isamarkupfalse%
\ simp\isanewline
\ \ \ \ \ \isacommand{apply}\isamarkupfalse%
{\isacharparenleft}{\kern0pt}subst\ eq{\isadigit{1}}{\isacharcomma}{\kern0pt}\ rule\ wf{\isacharunderscore}{\kern0pt}prel{\isacharcomma}{\kern0pt}\ simp\ add{\isacharcolon}{\kern0pt}assms{\isacharparenright}{\kern0pt}\isanewline
\ \ \ \ \isacommand{apply}\isamarkupfalse%
{\isacharparenleft}{\kern0pt}rule\ trans{\isacharunderscore}{\kern0pt}preds{\isacharunderscore}{\kern0pt}rel{\isacharparenright}{\kern0pt}\isanewline
\ \ \ \ \isacommand{using}\isamarkupfalse%
\ pair{\isacharunderscore}{\kern0pt}in{\isacharunderscore}{\kern0pt}M{\isacharunderscore}{\kern0pt}iff\ \isanewline
\ \ \ \ \ \ \isacommand{apply}\isamarkupfalse%
\ simp\isanewline
\ \ \ \ \ \isacommand{apply}\isamarkupfalse%
{\isacharparenleft}{\kern0pt}subst\ eq{\isadigit{1}}{\isacharcomma}{\kern0pt}\ rule\ prel{\isacharunderscore}{\kern0pt}trans{\isacharcomma}{\kern0pt}\ simp\ add{\isacharcolon}{\kern0pt}assms{\isacharparenright}{\kern0pt}\isanewline
\ \ \ \ \isacommand{done}\isamarkupfalse%
\ \ \ \ \isanewline
\isanewline
\ \ \isacommand{have}\isamarkupfalse%
\ I{\isadigit{2}}{\isacharcolon}{\kern0pt}\ {\isachardoublequoteopen}{\isachardot}{\kern0pt}{\isachardot}{\kern0pt}{\isachardot}{\kern0pt}\ {\isasymlongleftrightarrow}\ v\ {\isacharequal}{\kern0pt}\ wftrec{\isacharparenleft}{\kern0pt}preds{\isacharunderscore}{\kern0pt}rel{\isacharparenleft}{\kern0pt}{\isasymlambda}a\ b{\isachardot}{\kern0pt}\ {\isacharless}{\kern0pt}a{\isacharcomma}{\kern0pt}\ b{\isachargreater}{\kern0pt}\ {\isasymin}\ prel{\isacharparenleft}{\kern0pt}Rrel{\isacharparenleft}{\kern0pt}R{\isacharcomma}{\kern0pt}\ M{\isacharparenright}{\kern0pt}{\isacharcomma}{\kern0pt}\ p{\isacharparenright}{\kern0pt}{\isacharcomma}{\kern0pt}\ {\isacharless}{\kern0pt}x{\isacharcomma}{\kern0pt}\ a{\isachargreater}{\kern0pt}{\isacharparenright}{\kern0pt}{\isacharcomma}{\kern0pt}\ {\isacharless}{\kern0pt}x{\isacharcomma}{\kern0pt}\ a{\isachargreater}{\kern0pt}{\isacharcomma}{\kern0pt}\ G{\isacharparenright}{\kern0pt}{\isachardoublequoteclose}\ \isanewline
\ \ \ \ \isacommand{apply}\isamarkupfalse%
{\isacharparenleft}{\kern0pt}rule\ iffI{\isacharcomma}{\kern0pt}\ simp{\isacharcomma}{\kern0pt}\ simp{\isacharparenright}{\kern0pt}\isanewline
\ \ \ \ \isacommand{apply}\isamarkupfalse%
{\isacharparenleft}{\kern0pt}rule\ conjI{\isacharcomma}{\kern0pt}\ rule\ preds{\isacharunderscore}{\kern0pt}prel{\isacharunderscore}{\kern0pt}in{\isacharunderscore}{\kern0pt}M{\isacharparenright}{\kern0pt}\isanewline
\ \ \ \ \isacommand{using}\isamarkupfalse%
\ assms\ pair{\isacharunderscore}{\kern0pt}in{\isacharunderscore}{\kern0pt}M{\isacharunderscore}{\kern0pt}iff\ xinM\ ainM\ pinM\ ainp\isanewline
\ \ \ \ \isacommand{by}\isamarkupfalse%
\ auto\isanewline
\isanewline
\ \ \isacommand{have}\isamarkupfalse%
\ I{\isadigit{3}}{\isacharcolon}{\kern0pt}\ {\isachardoublequoteopen}{\isachardot}{\kern0pt}{\isachardot}{\kern0pt}{\isachardot}{\kern0pt}\ {\isasymlongleftrightarrow}\ v\ {\isacharequal}{\kern0pt}\ wftrec{\isacharparenleft}{\kern0pt}prel{\isacharparenleft}{\kern0pt}preds{\isacharunderscore}{\kern0pt}rel{\isacharparenleft}{\kern0pt}R{\isacharcomma}{\kern0pt}\ x{\isacharparenright}{\kern0pt}{\isacharcomma}{\kern0pt}\ {\isacharbraceleft}{\kern0pt}a{\isacharbraceright}{\kern0pt}{\isacharparenright}{\kern0pt}{\isacharcomma}{\kern0pt}\ {\isacharless}{\kern0pt}x{\isacharcomma}{\kern0pt}\ a{\isachargreater}{\kern0pt}{\isacharcomma}{\kern0pt}\ G{\isacharparenright}{\kern0pt}{\isachardoublequoteclose}\ \isacommand{using}\isamarkupfalse%
\ eq{\isadigit{2}}\ \isacommand{by}\isamarkupfalse%
\ auto\ \isanewline
\isanewline
\ \ \isacommand{have}\isamarkupfalse%
\ I{\isadigit{4}}{\isacharcolon}{\kern0pt}\ {\isachardoublequoteopen}{\isachardot}{\kern0pt}{\isachardot}{\kern0pt}{\isachardot}{\kern0pt}\ {\isasymlongleftrightarrow}\ \ v\ {\isacharequal}{\kern0pt}\ wftrec{\isacharparenleft}{\kern0pt}Rrel{\isacharparenleft}{\kern0pt}R{\isacharcomma}{\kern0pt}\ M{\isacharparenright}{\kern0pt}{\isacharcomma}{\kern0pt}\ x{\isacharcomma}{\kern0pt}\ H{\isacharparenright}{\kern0pt}{\isachardoublequoteclose}\ \ \isanewline
\ \ \ \ \isacommand{apply}\isamarkupfalse%
{\isacharparenleft}{\kern0pt}rule\ iff{\isacharunderscore}{\kern0pt}lemma{\isadigit{2}}{\isacharcomma}{\kern0pt}\ rule{\isacharunderscore}{\kern0pt}tac\ Gfm{\isacharequal}{\kern0pt}Gfm\ \isakeyword{in}\ wftrec{\isacharunderscore}{\kern0pt}prel{\isacharunderscore}{\kern0pt}preds{\isacharunderscore}{\kern0pt}rel{\isacharunderscore}{\kern0pt}eq{\isacharparenright}{\kern0pt}\isanewline
\ \ \ \ \isacommand{using}\isamarkupfalse%
\ assms\ \isanewline
\ \ \ \ \isacommand{by}\isamarkupfalse%
\ auto\isanewline
\ \ \isacommand{show}\isamarkupfalse%
\ {\isacharquery}{\kern0pt}thesis\ \isacommand{using}\isamarkupfalse%
\ I{\isadigit{1}}\ I{\isadigit{2}}\ I{\isadigit{3}}\ I{\isadigit{4}}\ \isacommand{by}\isamarkupfalse%
\ auto\isanewline
\isacommand{qed}\isamarkupfalse%
%
\endisatagproof
{\isafoldproof}%
%
\isadelimproof
\isanewline
%
\endisadelimproof
\isanewline
\isacommand{lemma}\isamarkupfalse%
\ Rrel{\isacharunderscore}{\kern0pt}wftrec{\isacharunderscore}{\kern0pt}in{\isacharunderscore}{\kern0pt}M\ {\isacharcolon}{\kern0pt}\ \isanewline
\ \ \isakeyword{fixes}\ R\ Rfm\ G\ H\ x\ a\isanewline
\ \ \isakeyword{assumes}\ {\isachardoublequoteopen}x\ {\isasymin}\ M{\isachardoublequoteclose}\ {\isachardoublequoteopen}a\ {\isasymin}\ M{\isachardoublequoteclose}\ {\isachardoublequoteopen}x\ {\isasymin}\ field{\isacharparenleft}{\kern0pt}Rrel{\isacharparenleft}{\kern0pt}R{\isacharcomma}{\kern0pt}\ M{\isacharparenright}{\kern0pt}{\isacharparenright}{\kern0pt}{\isachardoublequoteclose}\ {\isachardoublequoteopen}wf{\isacharparenleft}{\kern0pt}Rrel{\isacharparenleft}{\kern0pt}R{\isacharcomma}{\kern0pt}\ M{\isacharparenright}{\kern0pt}{\isacharparenright}{\kern0pt}{\isachardoublequoteclose}\ {\isachardoublequoteopen}trans{\isacharparenleft}{\kern0pt}Rrel{\isacharparenleft}{\kern0pt}R{\isacharcomma}{\kern0pt}\ M{\isacharparenright}{\kern0pt}{\isacharparenright}{\kern0pt}{\isachardoublequoteclose}\isanewline
\ \ \ \ \ \ \ \ \ \ {\isachardoublequoteopen}Relation{\isacharunderscore}{\kern0pt}fm{\isacharparenleft}{\kern0pt}R{\isacharcomma}{\kern0pt}\ Rfm{\isacharparenright}{\kern0pt}{\isachardoublequoteclose}\ {\isachardoublequoteopen}preds{\isacharparenleft}{\kern0pt}R{\isacharcomma}{\kern0pt}\ x{\isacharparenright}{\kern0pt}\ {\isasymin}\ M{\isachardoublequoteclose}\ \ {\isachardoublequoteopen}Gfm\ {\isasymin}\ formula{\isachardoublequoteclose}\ {\isachardoublequoteopen}arity{\isacharparenleft}{\kern0pt}Gfm{\isacharparenright}{\kern0pt}\ {\isasymle}\ {\isadigit{3}}{\isachardoublequoteclose}\ \isanewline
\ \ \ \ \ \ \ \ \ \ {\isachardoublequoteopen}{\isasymAnd}a{\isadigit{0}}\ a{\isadigit{1}}\ a{\isadigit{2}}\ env{\isachardot}{\kern0pt}\ a{\isadigit{0}}\ {\isasymin}\ M\ {\isasymLongrightarrow}\ a{\isadigit{1}}\ {\isasymin}\ M\ {\isasymLongrightarrow}\ a{\isadigit{2}}\ {\isasymin}\ M\ {\isasymLongrightarrow}\ env\ {\isasymin}\ list{\isacharparenleft}{\kern0pt}M{\isacharparenright}{\kern0pt}\ {\isasymLongrightarrow}\ a{\isadigit{0}}\ {\isacharequal}{\kern0pt}\ G{\isacharparenleft}{\kern0pt}a{\isadigit{2}}{\isacharcomma}{\kern0pt}\ a{\isadigit{1}}{\isacharparenright}{\kern0pt}\ {\isasymlongleftrightarrow}\ sats{\isacharparenleft}{\kern0pt}M{\isacharcomma}{\kern0pt}\ Gfm{\isacharcomma}{\kern0pt}\ {\isacharbrackleft}{\kern0pt}a{\isadigit{0}}{\isacharcomma}{\kern0pt}\ a{\isadigit{1}}{\isacharcomma}{\kern0pt}\ a{\isadigit{2}}{\isacharbrackright}{\kern0pt}\ {\isacharat}{\kern0pt}\ env{\isacharparenright}{\kern0pt}{\isachardoublequoteclose}\ \isanewline
\ \ \isakeyword{and}\ GM{\isacharcolon}{\kern0pt}\ {\isachardoublequoteopen}{\isasymAnd}x\ g{\isachardot}{\kern0pt}\ x\ {\isasymin}\ M\ {\isasymLongrightarrow}\ g\ {\isasymin}\ M\ {\isasymLongrightarrow}\ function{\isacharparenleft}{\kern0pt}g{\isacharparenright}{\kern0pt}\ {\isasymLongrightarrow}\ G{\isacharparenleft}{\kern0pt}x{\isacharcomma}{\kern0pt}\ g{\isacharparenright}{\kern0pt}\ {\isasymin}\ M{\isachardoublequoteclose}\ \ \isanewline
\ \ \isakeyword{and}\ HGeq{\isacharcolon}{\kern0pt}\ {\isachardoublequoteopen}{\isasymAnd}h\ g\ x{\isachardot}{\kern0pt}\ h\ {\isasymin}\ Rrel{\isacharparenleft}{\kern0pt}R{\isacharcomma}{\kern0pt}\ M{\isacharparenright}{\kern0pt}\ {\isacharminus}{\kern0pt}{\isacharbackquote}{\kern0pt}{\isacharbackquote}{\kern0pt}\ {\isacharbraceleft}{\kern0pt}x{\isacharbraceright}{\kern0pt}\ {\isasymrightarrow}\ M\ {\isasymLongrightarrow}\ g\ {\isasymin}\ {\isacharparenleft}{\kern0pt}Rrel{\isacharparenleft}{\kern0pt}R{\isacharcomma}{\kern0pt}\ M{\isacharparenright}{\kern0pt}\ {\isacharminus}{\kern0pt}{\isacharbackquote}{\kern0pt}{\isacharbackquote}{\kern0pt}\ {\isacharbraceleft}{\kern0pt}x{\isacharbraceright}{\kern0pt}\ {\isasymtimes}\ {\isacharbraceleft}{\kern0pt}a{\isacharbraceright}{\kern0pt}{\isacharparenright}{\kern0pt}\ {\isasymrightarrow}\ M\ {\isasymLongrightarrow}\ g\ {\isasymin}\ M\ \ \isanewline
\ \ \ \ \ \ \ \ \ \ \ \ \ \ \ {\isasymLongrightarrow}\ x\ {\isasymin}\ field{\isacharparenleft}{\kern0pt}Rrel{\isacharparenleft}{\kern0pt}R{\isacharcomma}{\kern0pt}\ M{\isacharparenright}{\kern0pt}{\isacharparenright}{\kern0pt}\ {\isasymLongrightarrow}\ {\isacharparenleft}{\kern0pt}{\isasymAnd}y{\isachardot}{\kern0pt}\ y\ {\isasymin}\ Rrel{\isacharparenleft}{\kern0pt}R{\isacharcomma}{\kern0pt}\ M{\isacharparenright}{\kern0pt}\ {\isacharminus}{\kern0pt}{\isacharbackquote}{\kern0pt}{\isacharbackquote}{\kern0pt}\ {\isacharbraceleft}{\kern0pt}x{\isacharbraceright}{\kern0pt}\ {\isasymLongrightarrow}\ h{\isacharbackquote}{\kern0pt}y\ {\isacharequal}{\kern0pt}\ g{\isacharbackquote}{\kern0pt}{\isacharless}{\kern0pt}y{\isacharcomma}{\kern0pt}\ a{\isachargreater}{\kern0pt}{\isacharparenright}{\kern0pt}\ {\isasymLongrightarrow}\ H{\isacharparenleft}{\kern0pt}x{\isacharcomma}{\kern0pt}\ h{\isacharparenright}{\kern0pt}\ {\isacharequal}{\kern0pt}\ G{\isacharparenleft}{\kern0pt}{\isacharless}{\kern0pt}x{\isacharcomma}{\kern0pt}\ a{\isachargreater}{\kern0pt}{\isacharcomma}{\kern0pt}\ g{\isacharparenright}{\kern0pt}{\isachardoublequoteclose}\ \ \isanewline
\ \ \isanewline
\ \ \isakeyword{shows}\ {\isachardoublequoteopen}wftrec{\isacharparenleft}{\kern0pt}Rrel{\isacharparenleft}{\kern0pt}R{\isacharcomma}{\kern0pt}\ M{\isacharparenright}{\kern0pt}{\isacharcomma}{\kern0pt}\ x{\isacharcomma}{\kern0pt}\ H{\isacharparenright}{\kern0pt}\ {\isasymin}\ M{\isachardoublequoteclose}\isanewline
%
\isadelimproof
\isanewline
\ \ %
\endisadelimproof
%
\isatagproof
\isacommand{apply}\isamarkupfalse%
{\isacharparenleft}{\kern0pt}rule{\isacharunderscore}{\kern0pt}tac\ b{\isacharequal}{\kern0pt}{\isachardoublequoteopen}wftrec{\isacharparenleft}{\kern0pt}Rrel{\isacharparenleft}{\kern0pt}R{\isacharcomma}{\kern0pt}\ M{\isacharparenright}{\kern0pt}{\isacharcomma}{\kern0pt}\ x{\isacharcomma}{\kern0pt}\ H{\isacharparenright}{\kern0pt}{\isachardoublequoteclose}\ \isakeyword{in}\ ssubst{\isacharparenright}{\kern0pt}\ \isanewline
\ \ \ \isacommand{apply}\isamarkupfalse%
{\isacharparenleft}{\kern0pt}rule\ eq{\isacharunderscore}{\kern0pt}flip{\isacharcomma}{\kern0pt}\ rule{\isacharunderscore}{\kern0pt}tac\ G{\isacharequal}{\kern0pt}G\ \isakeyword{and}\ a{\isacharequal}{\kern0pt}a\ \isakeyword{in}\ wftrec{\isacharunderscore}{\kern0pt}prel{\isacharunderscore}{\kern0pt}preds{\isacharunderscore}{\kern0pt}rel{\isacharunderscore}{\kern0pt}eq{\isacharparenright}{\kern0pt}\ \isanewline
\ \ \isacommand{using}\isamarkupfalse%
\ assms\ \isanewline
\ \ \ \ \ \ \ \ \ \ \ \ \ \ \isacommand{apply}\isamarkupfalse%
\ auto{\isacharbrackleft}{\kern0pt}{\isadigit{1}}{\isadigit{2}}{\isacharbrackright}{\kern0pt}\isanewline
\ \ \isacommand{unfolding}\isamarkupfalse%
\ wftrec{\isacharunderscore}{\kern0pt}def\isanewline
\ \ \ \isacommand{apply}\isamarkupfalse%
{\isacharparenleft}{\kern0pt}rule\ GM{\isacharparenright}{\kern0pt}\isanewline
\ \ \isacommand{using}\isamarkupfalse%
\ pair{\isacharunderscore}{\kern0pt}in{\isacharunderscore}{\kern0pt}M{\isacharunderscore}{\kern0pt}iff\ assms\isanewline
\ \ \ \ \isacommand{apply}\isamarkupfalse%
\ force\isanewline
\ \ \ \isacommand{apply}\isamarkupfalse%
{\isacharparenleft}{\kern0pt}rule{\isacharunderscore}{\kern0pt}tac\ p{\isacharequal}{\kern0pt}Gfm\ \isakeyword{in}\ the{\isacharunderscore}{\kern0pt}recfun{\isacharunderscore}{\kern0pt}in{\isacharunderscore}{\kern0pt}M{\isacharparenright}{\kern0pt}\isanewline
\ \ \ \ \ \ \ \ \ \ \isacommand{apply}\isamarkupfalse%
{\isacharparenleft}{\kern0pt}rule\ wf{\isacharunderscore}{\kern0pt}prel{\isacharcomma}{\kern0pt}\ rule\ wf{\isacharunderscore}{\kern0pt}preds{\isacharunderscore}{\kern0pt}rel{\isacharcomma}{\kern0pt}\ simp\ add{\isacharcolon}{\kern0pt}assms{\isacharcomma}{\kern0pt}\ simp\ add{\isacharcolon}{\kern0pt}assms{\isacharparenright}{\kern0pt}\isanewline
\ \ \ \ \ \ \ \ \ \isacommand{apply}\isamarkupfalse%
{\isacharparenleft}{\kern0pt}rule\ prel{\isacharunderscore}{\kern0pt}trans{\isacharcomma}{\kern0pt}\ rule\ trans{\isacharunderscore}{\kern0pt}preds{\isacharunderscore}{\kern0pt}rel{\isacharcomma}{\kern0pt}\ simp\ add{\isacharcolon}{\kern0pt}assms{\isacharcomma}{\kern0pt}\ simp\ add{\isacharcolon}{\kern0pt}assms{\isacharparenright}{\kern0pt}\isanewline
\ \ \ \ \ \ \ \ \isacommand{apply}\isamarkupfalse%
{\isacharparenleft}{\kern0pt}rule\ prel{\isacharunderscore}{\kern0pt}closed{\isacharcomma}{\kern0pt}\ rule{\isacharunderscore}{\kern0pt}tac\ Rfm{\isacharequal}{\kern0pt}Rfm\ \isakeyword{in}\ preds{\isacharunderscore}{\kern0pt}rel{\isacharunderscore}{\kern0pt}in{\isacharunderscore}{\kern0pt}M{\isacharparenright}{\kern0pt}\isanewline
\ \ \isacommand{using}\isamarkupfalse%
\ assms\ singleton{\isacharunderscore}{\kern0pt}in{\isacharunderscore}{\kern0pt}M{\isacharunderscore}{\kern0pt}iff\ pair{\isacharunderscore}{\kern0pt}in{\isacharunderscore}{\kern0pt}M{\isacharunderscore}{\kern0pt}iff\isanewline
\ \ \ \ \ \ \ \ \ \ \ \isacommand{apply}\isamarkupfalse%
\ auto{\isacharbrackleft}{\kern0pt}{\isadigit{9}}{\isacharbrackright}{\kern0pt}\isanewline
\ \ \isacommand{apply}\isamarkupfalse%
{\isacharparenleft}{\kern0pt}rule{\isacharunderscore}{\kern0pt}tac\ a{\isacharequal}{\kern0pt}{\isachardoublequoteopen}{\isacharparenleft}{\kern0pt}{\isasymlambda}y{\isasymin}prel{\isacharparenleft}{\kern0pt}preds{\isacharunderscore}{\kern0pt}rel{\isacharparenleft}{\kern0pt}R{\isacharcomma}{\kern0pt}\ x{\isacharparenright}{\kern0pt}{\isacharcomma}{\kern0pt}\ {\isacharbraceleft}{\kern0pt}a{\isacharbraceright}{\kern0pt}{\isacharparenright}{\kern0pt}\ {\isacharminus}{\kern0pt}{\isacharbackquote}{\kern0pt}{\isacharbackquote}{\kern0pt}\ {\isacharbraceleft}{\kern0pt}{\isasymlangle}x{\isacharcomma}{\kern0pt}\ a{\isasymrangle}{\isacharbraceright}{\kern0pt}{\isachardot}{\kern0pt}\ G{\isacharparenleft}{\kern0pt}y{\isacharcomma}{\kern0pt}\ restrict{\isacharparenleft}{\kern0pt}the{\isacharunderscore}{\kern0pt}recfun{\isacharparenleft}{\kern0pt}prel{\isacharparenleft}{\kern0pt}preds{\isacharunderscore}{\kern0pt}rel{\isacharparenleft}{\kern0pt}R{\isacharcomma}{\kern0pt}\ x{\isacharparenright}{\kern0pt}{\isacharcomma}{\kern0pt}\ {\isacharbraceleft}{\kern0pt}a{\isacharbraceright}{\kern0pt}{\isacharparenright}{\kern0pt}{\isacharcomma}{\kern0pt}\ {\isasymlangle}x{\isacharcomma}{\kern0pt}\ a{\isasymrangle}{\isacharcomma}{\kern0pt}\ G{\isacharparenright}{\kern0pt}{\isacharcomma}{\kern0pt}\ prel{\isacharparenleft}{\kern0pt}preds{\isacharunderscore}{\kern0pt}rel{\isacharparenleft}{\kern0pt}R{\isacharcomma}{\kern0pt}\ x{\isacharparenright}{\kern0pt}{\isacharcomma}{\kern0pt}\ {\isacharbraceleft}{\kern0pt}a{\isacharbraceright}{\kern0pt}{\isacharparenright}{\kern0pt}\ {\isacharminus}{\kern0pt}{\isacharbackquote}{\kern0pt}{\isacharbackquote}{\kern0pt}\ {\isacharbraceleft}{\kern0pt}y{\isacharbraceright}{\kern0pt}{\isacharparenright}{\kern0pt}{\isacharparenright}{\kern0pt}{\isacharparenright}{\kern0pt}{\isachardoublequoteclose}\isanewline
\ \ \ \ \ \ \ \ \ \ \ \ \ \isakeyword{and}\ b\ {\isacharequal}{\kern0pt}\ {\isachardoublequoteopen}the{\isacharunderscore}{\kern0pt}recfun{\isacharparenleft}{\kern0pt}prel{\isacharparenleft}{\kern0pt}preds{\isacharunderscore}{\kern0pt}rel{\isacharparenleft}{\kern0pt}R{\isacharcomma}{\kern0pt}\ x{\isacharparenright}{\kern0pt}{\isacharcomma}{\kern0pt}\ {\isacharbraceleft}{\kern0pt}a{\isacharbraceright}{\kern0pt}{\isacharparenright}{\kern0pt}{\isacharcomma}{\kern0pt}\ {\isasymlangle}x{\isacharcomma}{\kern0pt}\ a{\isasymrangle}{\isacharcomma}{\kern0pt}\ G{\isacharparenright}{\kern0pt}{\isachardoublequoteclose}\ \isakeyword{in}\ ssubst{\isacharparenright}{\kern0pt}\ \isanewline
\ \ \ \isacommand{apply}\isamarkupfalse%
{\isacharparenleft}{\kern0pt}subgoal{\isacharunderscore}{\kern0pt}tac\ {\isachardoublequoteopen}is{\isacharunderscore}{\kern0pt}recfun{\isacharparenleft}{\kern0pt}prel{\isacharparenleft}{\kern0pt}preds{\isacharunderscore}{\kern0pt}rel{\isacharparenleft}{\kern0pt}R{\isacharcomma}{\kern0pt}\ x{\isacharparenright}{\kern0pt}{\isacharcomma}{\kern0pt}\ {\isacharbraceleft}{\kern0pt}a{\isacharbraceright}{\kern0pt}{\isacharparenright}{\kern0pt}{\isacharcomma}{\kern0pt}\ {\isacharless}{\kern0pt}x{\isacharcomma}{\kern0pt}\ a{\isachargreater}{\kern0pt}{\isacharcomma}{\kern0pt}\ G{\isacharcomma}{\kern0pt}\ the{\isacharunderscore}{\kern0pt}recfun{\isacharparenleft}{\kern0pt}prel{\isacharparenleft}{\kern0pt}preds{\isacharunderscore}{\kern0pt}rel{\isacharparenleft}{\kern0pt}R{\isacharcomma}{\kern0pt}\ x{\isacharparenright}{\kern0pt}{\isacharcomma}{\kern0pt}\ {\isacharbraceleft}{\kern0pt}a{\isacharbraceright}{\kern0pt}{\isacharparenright}{\kern0pt}{\isacharcomma}{\kern0pt}\ {\isasymlangle}x{\isacharcomma}{\kern0pt}\ a{\isasymrangle}{\isacharcomma}{\kern0pt}\ G{\isacharparenright}{\kern0pt}{\isacharparenright}{\kern0pt}{\isachardoublequoteclose}{\isacharparenright}{\kern0pt}\ \isanewline
\ \ \ \ \isacommand{apply}\isamarkupfalse%
\ {\isacharparenleft}{\kern0pt}simp\ add{\isacharcolon}{\kern0pt}is{\isacharunderscore}{\kern0pt}recfun{\isacharunderscore}{\kern0pt}def{\isacharparenright}{\kern0pt}\isanewline
\ \ \ \isacommand{apply}\isamarkupfalse%
{\isacharparenleft}{\kern0pt}rule\ unfold{\isacharunderscore}{\kern0pt}the{\isacharunderscore}{\kern0pt}recfun{\isacharparenright}{\kern0pt}\isanewline
\ \ \ \ \isacommand{apply}\isamarkupfalse%
{\isacharparenleft}{\kern0pt}rule\ wf{\isacharunderscore}{\kern0pt}prel{\isacharcomma}{\kern0pt}\ rule\ wf{\isacharunderscore}{\kern0pt}preds{\isacharunderscore}{\kern0pt}rel{\isacharcomma}{\kern0pt}\ simp\ add{\isacharcolon}{\kern0pt}assms{\isacharcomma}{\kern0pt}\ simp\ add{\isacharcolon}{\kern0pt}assms{\isacharparenright}{\kern0pt}\isanewline
\ \ \ \isacommand{apply}\isamarkupfalse%
{\isacharparenleft}{\kern0pt}rule\ prel{\isacharunderscore}{\kern0pt}trans{\isacharcomma}{\kern0pt}\ rule\ trans{\isacharunderscore}{\kern0pt}preds{\isacharunderscore}{\kern0pt}rel{\isacharcomma}{\kern0pt}\ simp\ add{\isacharcolon}{\kern0pt}assms{\isacharcomma}{\kern0pt}\ simp\ add{\isacharcolon}{\kern0pt}assms{\isacharparenright}{\kern0pt}\isanewline
\ \ \isacommand{apply}\isamarkupfalse%
{\isacharparenleft}{\kern0pt}rule\ function{\isacharunderscore}{\kern0pt}lam{\isacharparenright}{\kern0pt}\isanewline
\ \ \isacommand{done}\isamarkupfalse%
%
\endisatagproof
{\isafoldproof}%
%
\isadelimproof
\isanewline
%
\endisadelimproof
\isanewline
\isacommand{end}\isamarkupfalse%
\isanewline
%
\isadelimtheory
%
\endisadelimtheory
%
\isatagtheory
\isacommand{end}\isamarkupfalse%
%
\endisatagtheory
{\isafoldtheory}%
%
\isadelimtheory
%
\endisadelimtheory
%
\end{isabellebody}%
\endinput
%:%file=~/source/repos/ZF-notAC/code/RecFun_M.thy%:%
%:%10=1%:%
%:%11=1%:%
%:%12=2%:%
%:%13=3%:%
%:%14=4%:%
%:%15=5%:%
%:%20=5%:%
%:%23=6%:%
%:%24=7%:%
%:%25=7%:%
%:%26=8%:%
%:%27=9%:%
%:%28=10%:%
%:%29=11%:%
%:%30=11%:%
%:%31=12%:%
%:%32=13%:%
%:%33=14%:%
%:%34=14%:%
%:%35=15%:%
%:%43=23%:%
%:%50=24%:%
%:%51=24%:%
%:%52=25%:%
%:%53=25%:%
%:%54=26%:%
%:%55=26%:%
%:%56=27%:%
%:%57=28%:%
%:%58=29%:%
%:%59=30%:%
%:%60=31%:%
%:%61=32%:%
%:%62=33%:%
%:%63=34%:%
%:%64=34%:%
%:%65=34%:%
%:%66=35%:%
%:%67=36%:%
%:%68=36%:%
%:%69=37%:%
%:%70=38%:%
%:%71=38%:%
%:%72=39%:%
%:%73=39%:%
%:%74=40%:%
%:%75=40%:%
%:%76=41%:%
%:%77=41%:%
%:%78=42%:%
%:%79=43%:%
%:%80=43%:%
%:%81=44%:%
%:%82=44%:%
%:%83=45%:%
%:%84=45%:%
%:%85=46%:%
%:%86=46%:%
%:%87=47%:%
%:%88=47%:%
%:%89=48%:%
%:%90=48%:%
%:%91=49%:%
%:%92=49%:%
%:%93=50%:%
%:%94=50%:%
%:%95=51%:%
%:%96=51%:%
%:%97=52%:%
%:%98=52%:%
%:%99=53%:%
%:%100=53%:%
%:%101=54%:%
%:%102=54%:%
%:%103=55%:%
%:%104=55%:%
%:%105=56%:%
%:%106=56%:%
%:%107=57%:%
%:%108=57%:%
%:%109=58%:%
%:%110=58%:%
%:%111=59%:%
%:%112=59%:%
%:%113=60%:%
%:%114=61%:%
%:%115=61%:%
%:%116=62%:%
%:%117=62%:%
%:%118=63%:%
%:%119=63%:%
%:%120=63%:%
%:%121=64%:%
%:%122=65%:%
%:%123=65%:%
%:%124=66%:%
%:%125=66%:%
%:%126=67%:%
%:%127=67%:%
%:%128=68%:%
%:%129=68%:%
%:%130=69%:%
%:%131=69%:%
%:%132=70%:%
%:%133=70%:%
%:%134=71%:%
%:%135=71%:%
%:%136=72%:%
%:%137=72%:%
%:%138=73%:%
%:%139=73%:%
%:%140=74%:%
%:%141=74%:%
%:%142=75%:%
%:%143=75%:%
%:%144=76%:%
%:%145=76%:%
%:%146=77%:%
%:%147=77%:%
%:%148=78%:%
%:%149=78%:%
%:%150=79%:%
%:%151=79%:%
%:%152=80%:%
%:%153=80%:%
%:%154=81%:%
%:%155=81%:%
%:%156=82%:%
%:%157=82%:%
%:%158=83%:%
%:%159=83%:%
%:%160=84%:%
%:%161=84%:%
%:%162=85%:%
%:%163=85%:%
%:%164=86%:%
%:%165=86%:%
%:%166=87%:%
%:%167=87%:%
%:%168=88%:%
%:%169=88%:%
%:%170=89%:%
%:%171=89%:%
%:%172=90%:%
%:%173=90%:%
%:%174=91%:%
%:%175=91%:%
%:%176=92%:%
%:%177=92%:%
%:%178=93%:%
%:%179=93%:%
%:%180=94%:%
%:%181=94%:%
%:%182=95%:%
%:%183=95%:%
%:%184=96%:%
%:%185=97%:%
%:%186=97%:%
%:%187=97%:%
%:%188=98%:%
%:%194=98%:%
%:%197=99%:%
%:%198=100%:%
%:%199=100%:%
%:%200=101%:%
%:%203=102%:%
%:%207=102%:%
%:%208=102%:%
%:%209=102%:%
%:%214=102%:%
%:%217=103%:%
%:%218=104%:%
%:%219=104%:%
%:%220=105%:%
%:%221=106%:%
%:%222=107%:%
%:%223=108%:%
%:%224=109%:%
%:%225=110%:%
%:%226=111%:%
%:%227=112%:%
%:%228=113%:%
%:%229=114%:%
%:%230=115%:%
%:%237=116%:%
%:%238=116%:%
%:%239=117%:%
%:%240=118%:%
%:%241=118%:%
%:%242=119%:%
%:%243=120%:%
%:%244=121%:%
%:%245=121%:%
%:%246=121%:%
%:%247=122%:%
%:%248=123%:%
%:%249=123%:%
%:%250=124%:%
%:%251=125%:%
%:%252=125%:%
%:%253=126%:%
%:%254=126%:%
%:%255=127%:%
%:%256=127%:%
%:%257=128%:%
%:%258=128%:%
%:%259=129%:%
%:%260=129%:%
%:%261=130%:%
%:%262=130%:%
%:%263=131%:%
%:%264=131%:%
%:%265=132%:%
%:%266=132%:%
%:%267=133%:%
%:%268=133%:%
%:%269=134%:%
%:%270=134%:%
%:%271=135%:%
%:%272=135%:%
%:%273=136%:%
%:%274=136%:%
%:%275=137%:%
%:%276=137%:%
%:%277=138%:%
%:%278=138%:%
%:%279=139%:%
%:%280=139%:%
%:%281=140%:%
%:%282=140%:%
%:%283=141%:%
%:%284=141%:%
%:%285=142%:%
%:%286=142%:%
%:%287=143%:%
%:%288=143%:%
%:%289=144%:%
%:%290=144%:%
%:%291=145%:%
%:%292=145%:%
%:%293=146%:%
%:%294=146%:%
%:%295=147%:%
%:%296=147%:%
%:%297=148%:%
%:%298=148%:%
%:%299=149%:%
%:%300=150%:%
%:%301=150%:%
%:%302=151%:%
%:%303=152%:%
%:%304=152%:%
%:%305=153%:%
%:%306=153%:%
%:%307=154%:%
%:%308=154%:%
%:%309=155%:%
%:%310=155%:%
%:%311=156%:%
%:%312=156%:%
%:%313=157%:%
%:%314=157%:%
%:%315=157%:%
%:%316=158%:%
%:%317=158%:%
%:%318=159%:%
%:%319=159%:%
%:%320=160%:%
%:%321=160%:%
%:%322=161%:%
%:%323=161%:%
%:%324=162%:%
%:%325=162%:%
%:%326=163%:%
%:%332=163%:%
%:%335=164%:%
%:%336=165%:%
%:%337=165%:%
%:%338=166%:%
%:%339=167%:%
%:%340=168%:%
%:%341=169%:%
%:%342=170%:%
%:%343=171%:%
%:%344=172%:%
%:%345=173%:%
%:%346=174%:%
%:%347=175%:%
%:%348=176%:%
%:%349=177%:%
%:%352=178%:%
%:%353=179%:%
%:%357=179%:%
%:%358=179%:%
%:%359=180%:%
%:%360=180%:%
%:%361=181%:%
%:%362=181%:%
%:%363=182%:%
%:%364=182%:%
%:%365=183%:%
%:%366=183%:%
%:%367=184%:%
%:%368=184%:%
%:%369=185%:%
%:%370=185%:%
%:%371=186%:%
%:%372=186%:%
%:%373=187%:%
%:%374=187%:%
%:%375=188%:%
%:%376=188%:%
%:%377=189%:%
%:%378=189%:%
%:%379=190%:%
%:%380=190%:%
%:%381=191%:%
%:%382=191%:%
%:%383=192%:%
%:%389=192%:%
%:%392=193%:%
%:%393=194%:%
%:%394=194%:%
%:%395=195%:%
%:%396=196%:%
%:%397=197%:%
%:%398=198%:%
%:%399=199%:%
%:%400=200%:%
%:%401=201%:%
%:%402=202%:%
%:%403=203%:%
%:%404=204%:%
%:%405=205%:%
%:%408=206%:%
%:%409=207%:%
%:%413=207%:%
%:%414=207%:%
%:%415=208%:%
%:%416=208%:%
%:%417=209%:%
%:%418=209%:%
%:%419=210%:%
%:%420=210%:%
%:%421=211%:%
%:%422=211%:%
%:%423=212%:%
%:%424=212%:%
%:%425=213%:%
%:%426=213%:%
%:%427=214%:%
%:%428=214%:%
%:%429=215%:%
%:%430=215%:%
%:%431=216%:%
%:%437=216%:%
%:%440=217%:%
%:%441=218%:%
%:%442=218%:%
%:%443=219%:%
%:%444=220%:%
%:%445=221%:%
%:%446=222%:%
%:%447=223%:%
%:%448=224%:%
%:%449=225%:%
%:%450=226%:%
%:%451=227%:%
%:%452=228%:%
%:%453=229%:%
%:%460=230%:%
%:%461=230%:%
%:%462=231%:%
%:%463=231%:%
%:%464=232%:%
%:%465=232%:%
%:%466=233%:%
%:%467=233%:%
%:%468=234%:%
%:%469=234%:%
%:%470=235%:%
%:%471=235%:%
%:%472=235%:%
%:%473=235%:%
%:%474=236%:%
%:%475=236%:%
%:%476=236%:%
%:%477=237%:%
%:%478=237%:%
%:%479=238%:%
%:%480=238%:%
%:%481=239%:%
%:%482=239%:%
%:%483=240%:%
%:%484=240%:%
%:%485=240%:%
%:%486=240%:%
%:%487=240%:%
%:%488=241%:%
%:%494=241%:%
%:%497=242%:%
%:%498=243%:%
%:%499=243%:%
%:%500=244%:%
%:%501=245%:%
%:%502=246%:%
%:%503=247%:%
%:%504=248%:%
%:%505=249%:%
%:%506=250%:%
%:%507=251%:%
%:%508=252%:%
%:%509=253%:%
%:%510=254%:%
%:%517=255%:%
%:%518=255%:%
%:%519=256%:%
%:%520=256%:%
%:%521=257%:%
%:%522=257%:%
%:%523=258%:%
%:%524=258%:%
%:%525=259%:%
%:%526=259%:%
%:%527=260%:%
%:%528=261%:%
%:%529=261%:%
%:%530=262%:%
%:%531=262%:%
%:%532=263%:%
%:%533=263%:%
%:%534=264%:%
%:%535=264%:%
%:%536=265%:%
%:%537=265%:%
%:%538=265%:%
%:%539=266%:%
%:%540=266%:%
%:%541=267%:%
%:%542=267%:%
%:%543=268%:%
%:%544=269%:%
%:%545=269%:%
%:%546=270%:%
%:%547=270%:%
%:%548=271%:%
%:%549=271%:%
%:%550=272%:%
%:%551=272%:%
%:%552=273%:%
%:%553=273%:%
%:%554=274%:%
%:%555=274%:%
%:%556=275%:%
%:%557=275%:%
%:%558=275%:%
%:%559=276%:%
%:%560=276%:%
%:%561=277%:%
%:%562=277%:%
%:%563=278%:%
%:%569=278%:%
%:%572=279%:%
%:%573=280%:%
%:%574=281%:%
%:%575=281%:%
%:%576=282%:%
%:%577=283%:%
%:%578=284%:%
%:%579=285%:%
%:%580=286%:%
%:%581=287%:%
%:%582=288%:%
%:%583=289%:%
%:%584=290%:%
%:%585=291%:%
%:%586=292%:%
%:%593=293%:%
%:%594=293%:%
%:%595=294%:%
%:%596=294%:%
%:%597=295%:%
%:%598=295%:%
%:%599=296%:%
%:%600=296%:%
%:%601=297%:%
%:%602=298%:%
%:%603=298%:%
%:%604=299%:%
%:%605=299%:%
%:%606=300%:%
%:%607=300%:%
%:%608=301%:%
%:%609=301%:%
%:%610=302%:%
%:%611=302%:%
%:%612=303%:%
%:%613=303%:%
%:%614=304%:%
%:%615=304%:%
%:%616=305%:%
%:%617=305%:%
%:%618=306%:%
%:%619=306%:%
%:%620=307%:%
%:%621=307%:%
%:%622=308%:%
%:%623=308%:%
%:%624=309%:%
%:%625=309%:%
%:%626=310%:%
%:%627=310%:%
%:%628=310%:%
%:%629=311%:%
%:%630=311%:%
%:%631=312%:%
%:%632=312%:%
%:%633=313%:%
%:%634=313%:%
%:%635=314%:%
%:%636=314%:%
%:%637=315%:%
%:%638=315%:%
%:%639=316%:%
%:%640=316%:%
%:%641=317%:%
%:%642=317%:%
%:%643=318%:%
%:%644=318%:%
%:%645=319%:%
%:%646=319%:%
%:%647=320%:%
%:%648=320%:%
%:%649=321%:%
%:%650=321%:%
%:%651=322%:%
%:%652=323%:%
%:%653=323%:%
%:%654=324%:%
%:%655=324%:%
%:%656=325%:%
%:%657=325%:%
%:%658=326%:%
%:%659=326%:%
%:%660=327%:%
%:%661=327%:%
%:%662=328%:%
%:%663=328%:%
%:%664=329%:%
%:%665=329%:%
%:%666=330%:%
%:%667=330%:%
%:%668=331%:%
%:%669=331%:%
%:%670=332%:%
%:%671=332%:%
%:%672=333%:%
%:%678=333%:%
%:%681=334%:%
%:%682=335%:%
%:%683=335%:%
%:%684=336%:%
%:%685=337%:%
%:%686=338%:%
%:%687=339%:%
%:%688=340%:%
%:%689=341%:%
%:%690=342%:%
%:%691=343%:%
%:%692=344%:%
%:%693=345%:%
%:%694=346%:%
%:%701=347%:%
%:%702=347%:%
%:%703=348%:%
%:%704=348%:%
%:%705=349%:%
%:%706=349%:%
%:%707=350%:%
%:%708=350%:%
%:%709=351%:%
%:%710=351%:%
%:%711=351%:%
%:%712=351%:%
%:%713=352%:%
%:%714=352%:%
%:%715=353%:%
%:%716=353%:%
%:%717=354%:%
%:%718=354%:%
%:%719=355%:%
%:%720=355%:%
%:%721=356%:%
%:%722=356%:%
%:%723=356%:%
%:%724=357%:%
%:%725=357%:%
%:%726=358%:%
%:%727=358%:%
%:%728=359%:%
%:%729=359%:%
%:%730=360%:%
%:%731=360%:%
%:%732=361%:%
%:%733=361%:%
%:%734=362%:%
%:%735=362%:%
%:%736=363%:%
%:%737=363%:%
%:%738=364%:%
%:%739=364%:%
%:%740=365%:%
%:%741=365%:%
%:%742=366%:%
%:%743=366%:%
%:%744=366%:%
%:%745=367%:%
%:%746=367%:%
%:%747=368%:%
%:%748=368%:%
%:%749=369%:%
%:%750=369%:%
%:%751=370%:%
%:%752=370%:%
%:%753=371%:%
%:%754=371%:%
%:%755=372%:%
%:%756=372%:%
%:%757=373%:%
%:%758=373%:%
%:%759=374%:%
%:%760=374%:%
%:%761=375%:%
%:%762=375%:%
%:%763=376%:%
%:%764=377%:%
%:%765=377%:%
%:%766=378%:%
%:%767=378%:%
%:%768=379%:%
%:%769=379%:%
%:%770=380%:%
%:%771=380%:%
%:%772=381%:%
%:%773=381%:%
%:%774=382%:%
%:%775=382%:%
%:%776=383%:%
%:%777=383%:%
%:%778=384%:%
%:%779=384%:%
%:%780=385%:%
%:%781=385%:%
%:%782=386%:%
%:%783=386%:%
%:%784=387%:%
%:%785=387%:%
%:%786=388%:%
%:%787=388%:%
%:%788=389%:%
%:%789=389%:%
%:%790=390%:%
%:%791=390%:%
%:%792=391%:%
%:%798=391%:%
%:%801=392%:%
%:%802=393%:%
%:%803=393%:%
%:%804=394%:%
%:%805=394%:%
%:%806=395%:%
%:%807=396%:%
%:%808=396%:%
%:%811=397%:%
%:%815=397%:%
%:%816=397%:%
%:%817=397%:%
%:%822=397%:%
%:%825=398%:%
%:%826=399%:%
%:%827=399%:%
%:%830=400%:%
%:%834=400%:%
%:%835=400%:%
%:%836=400%:%
%:%841=400%:%
%:%844=401%:%
%:%845=402%:%
%:%846=402%:%
%:%847=403%:%
%:%848=404%:%
%:%849=404%:%
%:%852=405%:%
%:%856=405%:%
%:%857=405%:%
%:%858=405%:%
%:%863=405%:%
%:%866=406%:%
%:%867=407%:%
%:%868=407%:%
%:%869=408%:%
%:%870=409%:%
%:%871=409%:%
%:%872=410%:%
%:%873=410%:%
%:%874=411%:%
%:%875=412%:%
%:%876=412%:%
%:%877=413%:%
%:%878=414%:%
%:%879=414%:%
%:%880=415%:%
%:%881=416%:%
%:%882=417%:%
%:%883=417%:%
%:%884=418%:%
%:%885=419%:%
%:%886=420%:%
%:%889=421%:%
%:%893=421%:%
%:%894=421%:%
%:%895=422%:%
%:%896=422%:%
%:%897=423%:%
%:%898=423%:%
%:%899=424%:%
%:%900=424%:%
%:%901=425%:%
%:%902=425%:%
%:%903=426%:%
%:%904=426%:%
%:%905=427%:%
%:%906=427%:%
%:%907=428%:%
%:%908=428%:%
%:%909=429%:%
%:%910=429%:%
%:%911=430%:%
%:%912=430%:%
%:%913=431%:%
%:%914=431%:%
%:%919=431%:%
%:%922=432%:%
%:%923=433%:%
%:%924=433%:%
%:%925=434%:%
%:%926=435%:%
%:%927=436%:%
%:%934=437%:%
%:%935=437%:%
%:%936=438%:%
%:%937=438%:%
%:%938=439%:%
%:%939=439%:%
%:%940=440%:%
%:%941=440%:%
%:%942=441%:%
%:%943=441%:%
%:%944=442%:%
%:%945=442%:%
%:%946=443%:%
%:%947=443%:%
%:%948=444%:%
%:%949=444%:%
%:%950=445%:%
%:%951=445%:%
%:%952=445%:%
%:%953=446%:%
%:%954=446%:%
%:%955=447%:%
%:%956=447%:%
%:%957=448%:%
%:%958=448%:%
%:%959=449%:%
%:%965=449%:%
%:%968=450%:%
%:%969=451%:%
%:%970=451%:%
%:%971=452%:%
%:%972=453%:%
%:%973=454%:%
%:%980=455%:%
%:%981=455%:%
%:%982=456%:%
%:%983=456%:%
%:%984=457%:%
%:%985=457%:%
%:%986=458%:%
%:%987=458%:%
%:%988=459%:%
%:%989=459%:%
%:%990=460%:%
%:%991=460%:%
%:%992=461%:%
%:%993=462%:%
%:%994=462%:%
%:%995=463%:%
%:%996=463%:%
%:%997=464%:%
%:%998=464%:%
%:%999=465%:%
%:%1000=465%:%
%:%1001=466%:%
%:%1002=466%:%
%:%1003=467%:%
%:%1004=468%:%
%:%1005=468%:%
%:%1006=468%:%
%:%1007=469%:%
%:%1008=469%:%
%:%1009=470%:%
%:%1010=470%:%
%:%1011=471%:%
%:%1012=471%:%
%:%1013=472%:%
%:%1014=472%:%
%:%1015=473%:%
%:%1016=473%:%
%:%1017=474%:%
%:%1018=474%:%
%:%1019=475%:%
%:%1020=475%:%
%:%1021=476%:%
%:%1022=476%:%
%:%1023=477%:%
%:%1024=477%:%
%:%1025=478%:%
%:%1026=478%:%
%:%1027=479%:%
%:%1033=479%:%
%:%1036=480%:%
%:%1037=481%:%
%:%1038=481%:%
%:%1039=482%:%
%:%1040=483%:%
%:%1041=484%:%
%:%1048=485%:%
%:%1049=485%:%
%:%1050=486%:%
%:%1051=486%:%
%:%1052=486%:%
%:%1053=487%:%
%:%1054=488%:%
%:%1055=488%:%
%:%1056=489%:%
%:%1057=489%:%
%:%1058=490%:%
%:%1059=490%:%
%:%1060=491%:%
%:%1061=491%:%
%:%1062=492%:%
%:%1063=492%:%
%:%1064=493%:%
%:%1065=493%:%
%:%1066=494%:%
%:%1067=494%:%
%:%1068=495%:%
%:%1069=495%:%
%:%1070=496%:%
%:%1071=496%:%
%:%1072=497%:%
%:%1073=497%:%
%:%1074=498%:%
%:%1075=498%:%
%:%1076=499%:%
%:%1077=499%:%
%:%1078=500%:%
%:%1079=500%:%
%:%1080=501%:%
%:%1081=501%:%
%:%1082=502%:%
%:%1083=502%:%
%:%1084=503%:%
%:%1085=503%:%
%:%1086=504%:%
%:%1087=504%:%
%:%1088=505%:%
%:%1089=505%:%
%:%1090=506%:%
%:%1091=506%:%
%:%1092=507%:%
%:%1093=507%:%
%:%1094=508%:%
%:%1095=508%:%
%:%1096=509%:%
%:%1097=509%:%
%:%1098=510%:%
%:%1099=510%:%
%:%1100=511%:%
%:%1101=511%:%
%:%1102=512%:%
%:%1103=512%:%
%:%1104=513%:%
%:%1105=513%:%
%:%1106=514%:%
%:%1107=514%:%
%:%1108=515%:%
%:%1109=515%:%
%:%1110=516%:%
%:%1111=517%:%
%:%1112=517%:%
%:%1113=518%:%
%:%1114=518%:%
%:%1115=519%:%
%:%1116=519%:%
%:%1117=520%:%
%:%1118=520%:%
%:%1119=521%:%
%:%1120=521%:%
%:%1121=522%:%
%:%1122=522%:%
%:%1123=523%:%
%:%1124=523%:%
%:%1125=524%:%
%:%1126=524%:%
%:%1127=525%:%
%:%1128=525%:%
%:%1129=526%:%
%:%1130=526%:%
%:%1131=527%:%
%:%1132=527%:%
%:%1133=528%:%
%:%1134=528%:%
%:%1135=529%:%
%:%1136=529%:%
%:%1137=530%:%
%:%1138=530%:%
%:%1139=531%:%
%:%1140=532%:%
%:%1141=532%:%
%:%1142=532%:%
%:%1143=532%:%
%:%1144=533%:%
%:%1150=533%:%
%:%1153=534%:%
%:%1154=535%:%
%:%1155=535%:%
%:%1156=536%:%
%:%1157=537%:%
%:%1158=537%:%
%:%1159=538%:%
%:%1160=539%:%
%:%1161=540%:%
%:%1162=541%:%
%:%1163=541%:%
%:%1164=542%:%
%:%1165=543%:%
%:%1166=544%:%
%:%1167=544%:%
%:%1168=545%:%
%:%1169=546%:%
%:%1170=547%:%
%:%1173=548%:%
%:%1174=549%:%
%:%1178=549%:%
%:%1179=549%:%
%:%1180=550%:%
%:%1181=550%:%
%:%1182=551%:%
%:%1183=551%:%
%:%1184=552%:%
%:%1185=552%:%
%:%1186=553%:%
%:%1187=553%:%
%:%1188=554%:%
%:%1189=554%:%
%:%1190=555%:%
%:%1191=555%:%
%:%1196=555%:%
%:%1199=556%:%
%:%1200=557%:%
%:%1201=557%:%
%:%1202=558%:%
%:%1203=559%:%
%:%1204=560%:%
%:%1207=561%:%
%:%1208=562%:%
%:%1212=562%:%
%:%1213=562%:%
%:%1214=563%:%
%:%1215=563%:%
%:%1216=564%:%
%:%1217=564%:%
%:%1218=565%:%
%:%1219=565%:%
%:%1220=566%:%
%:%1221=566%:%
%:%1222=567%:%
%:%1223=567%:%
%:%1224=568%:%
%:%1225=568%:%
%:%1226=569%:%
%:%1227=569%:%
%:%1228=570%:%
%:%1229=570%:%
%:%1230=571%:%
%:%1231=571%:%
%:%1232=572%:%
%:%1233=572%:%
%:%1234=573%:%
%:%1235=573%:%
%:%1236=574%:%
%:%1237=574%:%
%:%1238=575%:%
%:%1239=575%:%
%:%1240=576%:%
%:%1241=576%:%
%:%1242=577%:%
%:%1243=577%:%
%:%1244=578%:%
%:%1245=578%:%
%:%1246=579%:%
%:%1247=579%:%
%:%1248=580%:%
%:%1249=580%:%
%:%1254=580%:%
%:%1257=581%:%
%:%1258=582%:%
%:%1259=582%:%
%:%1260=583%:%
%:%1261=584%:%
%:%1262=585%:%
%:%1269=586%:%
%:%1270=586%:%
%:%1271=587%:%
%:%1272=587%:%
%:%1273=587%:%
%:%1274=588%:%
%:%1275=588%:%
%:%1276=588%:%
%:%1277=589%:%
%:%1278=589%:%
%:%1279=590%:%
%:%1280=590%:%
%:%1281=591%:%
%:%1282=591%:%
%:%1283=592%:%
%:%1284=592%:%
%:%1285=593%:%
%:%1286=593%:%
%:%1287=594%:%
%:%1288=594%:%
%:%1289=595%:%
%:%1290=595%:%
%:%1291=596%:%
%:%1292=596%:%
%:%1293=597%:%
%:%1294=597%:%
%:%1295=598%:%
%:%1296=598%:%
%:%1297=599%:%
%:%1298=599%:%
%:%1299=600%:%
%:%1300=600%:%
%:%1301=601%:%
%:%1302=601%:%
%:%1303=602%:%
%:%1304=602%:%
%:%1305=603%:%
%:%1306=603%:%
%:%1307=604%:%
%:%1308=604%:%
%:%1309=605%:%
%:%1310=605%:%
%:%1311=606%:%
%:%1312=606%:%
%:%1313=607%:%
%:%1314=607%:%
%:%1315=608%:%
%:%1316=608%:%
%:%1317=609%:%
%:%1318=609%:%
%:%1319=610%:%
%:%1320=610%:%
%:%1321=611%:%
%:%1322=611%:%
%:%1323=612%:%
%:%1324=612%:%
%:%1325=613%:%
%:%1326=613%:%
%:%1327=614%:%
%:%1328=614%:%
%:%1329=615%:%
%:%1330=615%:%
%:%1331=616%:%
%:%1332=616%:%
%:%1333=617%:%
%:%1334=617%:%
%:%1335=618%:%
%:%1336=618%:%
%:%1337=619%:%
%:%1338=619%:%
%:%1339=620%:%
%:%1340=620%:%
%:%1341=621%:%
%:%1342=621%:%
%:%1343=622%:%
%:%1344=622%:%
%:%1345=623%:%
%:%1346=623%:%
%:%1347=623%:%
%:%1348=624%:%
%:%1349=624%:%
%:%1350=624%:%
%:%1351=624%:%
%:%1352=624%:%
%:%1353=625%:%
%:%1354=625%:%
%:%1355=625%:%
%:%1356=625%:%
%:%1357=625%:%
%:%1358=626%:%
%:%1359=626%:%
%:%1360=626%:%
%:%1361=626%:%
%:%1362=627%:%
%:%1363=627%:%
%:%1364=627%:%
%:%1365=627%:%
%:%1366=628%:%
%:%1367=628%:%
%:%1368=629%:%
%:%1369=629%:%
%:%1370=629%:%
%:%1371=630%:%
%:%1372=631%:%
%:%1373=632%:%
%:%1374=632%:%
%:%1375=632%:%
%:%1376=632%:%
%:%1377=633%:%
%:%1378=633%:%
%:%1379=633%:%
%:%1380=633%:%
%:%1381=633%:%
%:%1382=634%:%
%:%1383=634%:%
%:%1384=634%:%
%:%1385=634%:%
%:%1386=634%:%
%:%1387=635%:%
%:%1388=635%:%
%:%1389=635%:%
%:%1390=635%:%
%:%1391=635%:%
%:%1392=636%:%
%:%1393=636%:%
%:%1394=636%:%
%:%1395=636%:%
%:%1396=636%:%
%:%1397=637%:%
%:%1398=637%:%
%:%1399=638%:%
%:%1400=638%:%
%:%1401=639%:%
%:%1402=639%:%
%:%1403=639%:%
%:%1404=640%:%
%:%1405=640%:%
%:%1406=641%:%
%:%1407=641%:%
%:%1408=642%:%
%:%1409=642%:%
%:%1410=643%:%
%:%1411=643%:%
%:%1412=644%:%
%:%1413=645%:%
%:%1414=645%:%
%:%1415=645%:%
%:%1416=645%:%
%:%1417=646%:%
%:%1423=646%:%
%:%1426=647%:%
%:%1427=648%:%
%:%1428=648%:%
%:%1429=649%:%
%:%1430=650%:%
%:%1431=651%:%
%:%1438=652%:%
%:%1439=652%:%
%:%1440=653%:%
%:%1441=653%:%
%:%1442=654%:%
%:%1443=654%:%
%:%1444=655%:%
%:%1445=655%:%
%:%1446=656%:%
%:%1447=656%:%
%:%1448=657%:%
%:%1449=657%:%
%:%1450=658%:%
%:%1451=658%:%
%:%1452=659%:%
%:%1453=659%:%
%:%1454=660%:%
%:%1455=660%:%
%:%1456=661%:%
%:%1457=661%:%
%:%1458=662%:%
%:%1459=662%:%
%:%1460=663%:%
%:%1461=663%:%
%:%1462=664%:%
%:%1463=665%:%
%:%1464=665%:%
%:%1465=666%:%
%:%1466=666%:%
%:%1467=667%:%
%:%1468=667%:%
%:%1469=668%:%
%:%1470=668%:%
%:%1471=669%:%
%:%1472=669%:%
%:%1473=670%:%
%:%1474=670%:%
%:%1475=671%:%
%:%1476=672%:%
%:%1477=672%:%
%:%1478=672%:%
%:%1479=673%:%
%:%1480=673%:%
%:%1481=674%:%
%:%1482=674%:%
%:%1483=675%:%
%:%1484=675%:%
%:%1485=676%:%
%:%1486=677%:%
%:%1487=677%:%
%:%1488=678%:%
%:%1489=678%:%
%:%1490=679%:%
%:%1491=679%:%
%:%1492=680%:%
%:%1493=680%:%
%:%1494=681%:%
%:%1495=682%:%
%:%1496=682%:%
%:%1497=682%:%
%:%1498=683%:%
%:%1499=683%:%
%:%1500=684%:%
%:%1501=684%:%
%:%1502=685%:%
%:%1503=685%:%
%:%1504=686%:%
%:%1505=687%:%
%:%1506=687%:%
%:%1507=687%:%
%:%1508=687%:%
%:%1509=687%:%
%:%1510=688%:%
%:%1516=688%:%
%:%1519=689%:%
%:%1520=690%:%
%:%1521=690%:%
%:%1524=691%:%
%:%1528=691%:%
%:%1529=691%:%
%:%1530=692%:%
%:%1531=692%:%
%:%1532=693%:%
%:%1533=693%:%
%:%1534=693%:%
%:%1535=694%:%
%:%1536=694%:%
%:%1537=694%:%
%:%1538=694%:%
%:%1539=694%:%
%:%1540=694%:%
%:%1541=695%:%
%:%1542=695%:%
%:%1543=695%:%
%:%1544=695%:%
%:%1545=695%:%
%:%1546=696%:%
%:%1547=696%:%
%:%1548=696%:%
%:%1549=696%:%
%:%1550=696%:%
%:%1551=697%:%
%:%1552=697%:%
%:%1553=697%:%
%:%1554=697%:%
%:%1555=697%:%
%:%1556=698%:%
%:%1557=698%:%
%:%1558=698%:%
%:%1559=698%:%
%:%1560=699%:%
%:%1561=699%:%
%:%1562=699%:%
%:%1563=699%:%
%:%1564=699%:%
%:%1565=700%:%
%:%1566=700%:%
%:%1567=700%:%
%:%1568=701%:%
%:%1569=701%:%
%:%1570=702%:%
%:%1571=702%:%
%:%1572=702%:%
%:%1573=703%:%
%:%1579=703%:%
%:%1582=704%:%
%:%1583=705%:%
%:%1584=705%:%
%:%1587=706%:%
%:%1591=706%:%
%:%1592=706%:%
%:%1593=707%:%
%:%1594=707%:%
%:%1595=708%:%
%:%1596=708%:%
%:%1601=708%:%
%:%1604=709%:%
%:%1605=710%:%
%:%1606=710%:%
%:%1607=711%:%
%:%1614=712%:%
%:%1615=712%:%
%:%1616=713%:%
%:%1617=713%:%
%:%1618=713%:%
%:%1619=714%:%
%:%1620=714%:%
%:%1621=714%:%
%:%1622=714%:%
%:%1623=715%:%
%:%1624=715%:%
%:%1625=715%:%
%:%1626=716%:%
%:%1627=716%:%
%:%1628=717%:%
%:%1629=717%:%
%:%1630=718%:%
%:%1631=718%:%
%:%1632=719%:%
%:%1633=719%:%
%:%1634=720%:%
%:%1635=720%:%
%:%1636=720%:%
%:%1637=720%:%
%:%1638=721%:%
%:%1639=721%:%
%:%1640=722%:%
%:%1641=722%:%
%:%1642=722%:%
%:%1643=723%:%
%:%1644=723%:%
%:%1645=723%:%
%:%1646=723%:%
%:%1647=724%:%
%:%1648=724%:%
%:%1649=724%:%
%:%1650=724%:%
%:%1651=724%:%
%:%1652=725%:%
%:%1653=726%:%
%:%1654=726%:%
%:%1655=727%:%
%:%1656=727%:%
%:%1657=728%:%
%:%1658=728%:%
%:%1659=729%:%
%:%1660=729%:%
%:%1661=730%:%
%:%1662=730%:%
%:%1663=731%:%
%:%1664=731%:%
%:%1665=732%:%
%:%1666=732%:%
%:%1667=733%:%
%:%1668=733%:%
%:%1669=734%:%
%:%1670=735%:%
%:%1671=735%:%
%:%1672=735%:%
%:%1673=736%:%
%:%1674=736%:%
%:%1675=737%:%
%:%1676=737%:%
%:%1677=738%:%
%:%1678=738%:%
%:%1679=739%:%
%:%1680=739%:%
%:%1681=739%:%
%:%1682=739%:%
%:%1683=740%:%
%:%1689=740%:%
%:%1692=741%:%
%:%1693=742%:%
%:%1694=742%:%
%:%1695=743%:%
%:%1702=744%:%
%:%1703=744%:%
%:%1704=745%:%
%:%1705=745%:%
%:%1706=745%:%
%:%1707=746%:%
%:%1708=746%:%
%:%1709=746%:%
%:%1710=746%:%
%:%1711=747%:%
%:%1712=747%:%
%:%1713=747%:%
%:%1714=748%:%
%:%1715=748%:%
%:%1716=749%:%
%:%1717=749%:%
%:%1718=750%:%
%:%1719=750%:%
%:%1720=751%:%
%:%1721=751%:%
%:%1722=752%:%
%:%1723=752%:%
%:%1724=753%:%
%:%1725=753%:%
%:%1726=753%:%
%:%1727=754%:%
%:%1728=754%:%
%:%1729=754%:%
%:%1730=754%:%
%:%1731=755%:%
%:%1732=755%:%
%:%1733=755%:%
%:%1734=756%:%
%:%1735=756%:%
%:%1736=757%:%
%:%1737=757%:%
%:%1738=758%:%
%:%1739=758%:%
%:%1740=759%:%
%:%1741=759%:%
%:%1742=760%:%
%:%1743=760%:%
%:%1744=761%:%
%:%1745=761%:%
%:%1746=761%:%
%:%1747=761%:%
%:%1748=762%:%
%:%1754=762%:%
%:%1757=763%:%
%:%1758=764%:%
%:%1759=764%:%
%:%1760=765%:%
%:%1761=766%:%
%:%1762=767%:%
%:%1769=768%:%
%:%1770=768%:%
%:%1771=769%:%
%:%1772=769%:%
%:%1773=770%:%
%:%1774=770%:%
%:%1775=771%:%
%:%1776=771%:%
%:%1777=772%:%
%:%1778=772%:%
%:%1779=773%:%
%:%1780=774%:%
%:%1781=774%:%
%:%1782=775%:%
%:%1783=775%:%
%:%1784=775%:%
%:%1785=776%:%
%:%1786=777%:%
%:%1787=777%:%
%:%1788=778%:%
%:%1789=778%:%
%:%1790=779%:%
%:%1791=779%:%
%:%1792=779%:%
%:%1793=780%:%
%:%1794=780%:%
%:%1795=780%:%
%:%1796=780%:%
%:%1797=781%:%
%:%1798=781%:%
%:%1799=782%:%
%:%1800=782%:%
%:%1801=783%:%
%:%1802=783%:%
%:%1803=784%:%
%:%1804=784%:%
%:%1805=784%:%
%:%1806=784%:%
%:%1807=784%:%
%:%1808=785%:%
%:%1809=785%:%
%:%1810=786%:%
%:%1811=786%:%
%:%1812=787%:%
%:%1813=787%:%
%:%1814=787%:%
%:%1815=787%:%
%:%1816=787%:%
%:%1817=788%:%
%:%1818=788%:%
%:%1819=788%:%
%:%1820=788%:%
%:%1821=788%:%
%:%1822=788%:%
%:%1823=789%:%
%:%1824=789%:%
%:%1825=789%:%
%:%1826=789%:%
%:%1827=789%:%
%:%1828=789%:%
%:%1829=790%:%
%:%1830=790%:%
%:%1831=791%:%
%:%1832=791%:%
%:%1833=791%:%
%:%1834=791%:%
%:%1835=791%:%
%:%1836=792%:%
%:%1837=792%:%
%:%1838=793%:%
%:%1839=794%:%
%:%1840=794%:%
%:%1841=795%:%
%:%1842=795%:%
%:%1843=796%:%
%:%1844=796%:%
%:%1845=797%:%
%:%1846=797%:%
%:%1847=798%:%
%:%1848=798%:%
%:%1849=799%:%
%:%1850=799%:%
%:%1851=800%:%
%:%1852=800%:%
%:%1853=801%:%
%:%1854=801%:%
%:%1855=802%:%
%:%1856=803%:%
%:%1857=803%:%
%:%1858=804%:%
%:%1859=804%:%
%:%1860=805%:%
%:%1861=805%:%
%:%1862=806%:%
%:%1863=806%:%
%:%1864=807%:%
%:%1865=807%:%
%:%1866=807%:%
%:%1867=808%:%
%:%1868=808%:%
%:%1869=809%:%
%:%1870=809%:%
%:%1871=810%:%
%:%1872=810%:%
%:%1873=811%:%
%:%1874=811%:%
%:%1875=812%:%
%:%1876=812%:%
%:%1877=812%:%
%:%1878=813%:%
%:%1879=813%:%
%:%1880=814%:%
%:%1881=814%:%
%:%1882=815%:%
%:%1883=815%:%
%:%1884=816%:%
%:%1885=816%:%
%:%1886=817%:%
%:%1887=817%:%
%:%1888=818%:%
%:%1889=818%:%
%:%1890=819%:%
%:%1891=819%:%
%:%1892=820%:%
%:%1893=820%:%
%:%1894=821%:%
%:%1895=821%:%
%:%1896=821%:%
%:%1897=822%:%
%:%1898=822%:%
%:%1899=823%:%
%:%1900=823%:%
%:%1901=824%:%
%:%1902=824%:%
%:%1903=825%:%
%:%1904=825%:%
%:%1905=826%:%
%:%1906=826%:%
%:%1907=826%:%
%:%1908=827%:%
%:%1909=827%:%
%:%1910=828%:%
%:%1911=828%:%
%:%1912=829%:%
%:%1913=829%:%
%:%1914=830%:%
%:%1915=830%:%
%:%1916=831%:%
%:%1917=831%:%
%:%1918=831%:%
%:%1919=832%:%
%:%1920=832%:%
%:%1921=833%:%
%:%1922=833%:%
%:%1923=834%:%
%:%1924=834%:%
%:%1925=835%:%
%:%1926=835%:%
%:%1927=835%:%
%:%1928=835%:%
%:%1929=836%:%
%:%1930=836%:%
%:%1931=837%:%
%:%1932=837%:%
%:%1933=837%:%
%:%1934=837%:%
%:%1935=837%:%
%:%1936=838%:%
%:%1942=838%:%
%:%1945=839%:%
%:%1946=840%:%
%:%1947=840%:%
%:%1948=841%:%
%:%1949=842%:%
%:%1950=843%:%
%:%1957=844%:%
%:%1958=844%:%
%:%1959=845%:%
%:%1960=845%:%
%:%1961=846%:%
%:%1962=846%:%
%:%1963=847%:%
%:%1964=847%:%
%:%1965=848%:%
%:%1966=848%:%
%:%1967=849%:%
%:%1968=849%:%
%:%1969=849%:%
%:%1970=850%:%
%:%1971=850%:%
%:%1972=850%:%
%:%1973=851%:%
%:%1979=851%:%
%:%1982=852%:%
%:%1983=853%:%
%:%1984=854%:%
%:%1985=855%:%
%:%1986=855%:%
%:%1987=856%:%
%:%1988=857%:%
%:%1989=857%:%
%:%1992=858%:%
%:%1996=858%:%
%:%1997=858%:%
%:%1998=858%:%
%:%2003=858%:%
%:%2006=859%:%
%:%2007=860%:%
%:%2008=860%:%
%:%2011=861%:%
%:%2015=861%:%
%:%2016=861%:%
%:%2017=861%:%
%:%2022=861%:%
%:%2025=862%:%
%:%2026=863%:%
%:%2027=863%:%
%:%2028=864%:%
%:%2029=865%:%
%:%2030=866%:%
%:%2031=867%:%
%:%2032=868%:%
%:%2033=869%:%
%:%2034=870%:%
%:%2036=872%:%
%:%2039=873%:%
%:%2043=873%:%
%:%2044=873%:%
%:%2049=873%:%
%:%2052=874%:%
%:%2053=875%:%
%:%2054=875%:%
%:%2055=876%:%
%:%2056=877%:%
%:%2057=877%:%
%:%2058=878%:%
%:%2059=879%:%
%:%2060=880%:%
%:%2061=880%:%
%:%2062=881%:%
%:%2063=882%:%
%:%2064=883%:%
%:%2065=883%:%
%:%2066=884%:%
%:%2067=885%:%
%:%2070=886%:%
%:%2071=887%:%
%:%2075=887%:%
%:%2076=887%:%
%:%2077=888%:%
%:%2078=889%:%
%:%2079=889%:%
%:%2080=889%:%
%:%2081=889%:%
%:%2082=889%:%
%:%2083=890%:%
%:%2084=890%:%
%:%2085=890%:%
%:%2091=890%:%
%:%2094=891%:%
%:%2095=892%:%
%:%2096=892%:%
%:%2097=893%:%
%:%2104=894%:%
%:%2105=894%:%
%:%2106=895%:%
%:%2107=895%:%
%:%2108=896%:%
%:%2109=897%:%
%:%2110=897%:%
%:%2111=897%:%
%:%2112=897%:%
%:%2113=898%:%
%:%2114=898%:%
%:%2115=898%:%
%:%2116=898%:%
%:%2117=898%:%
%:%2118=899%:%
%:%2119=900%:%
%:%2120=900%:%
%:%2121=901%:%
%:%2122=901%:%
%:%2123=901%:%
%:%2124=901%:%
%:%2125=901%:%
%:%2126=902%:%
%:%2127=902%:%
%:%2128=903%:%
%:%2129=903%:%
%:%2130=903%:%
%:%2131=904%:%
%:%2132=904%:%
%:%2133=904%:%
%:%2134=904%:%
%:%2135=905%:%
%:%2136=906%:%
%:%2137=906%:%
%:%2138=907%:%
%:%2139=908%:%
%:%2140=908%:%
%:%2141=908%:%
%:%2142=909%:%
%:%2143=909%:%
%:%2144=909%:%
%:%2145=909%:%
%:%2146=910%:%
%:%2147=910%:%
%:%2148=910%:%
%:%2149=911%:%
%:%2150=911%:%
%:%2151=911%:%
%:%2152=912%:%
%:%2153=912%:%
%:%2154=912%:%
%:%2155=913%:%
%:%2156=913%:%
%:%2157=913%:%
%:%2158=913%:%
%:%2159=914%:%
%:%2160=914%:%
%:%2161=914%:%
%:%2162=914%:%
%:%2163=914%:%
%:%2164=915%:%
%:%2170=915%:%
%:%2173=916%:%
%:%2174=917%:%
%:%2175=917%:%
%:%2178=918%:%
%:%2182=918%:%
%:%2183=918%:%
%:%2184=919%:%
%:%2185=919%:%
%:%2186=919%:%
%:%2187=920%:%
%:%2188=920%:%
%:%2189=921%:%
%:%2190=921%:%
%:%2191=922%:%
%:%2192=922%:%
%:%2193=923%:%
%:%2194=924%:%
%:%2195=924%:%
%:%2196=924%:%
%:%2197=925%:%
%:%2198=925%:%
%:%2199=926%:%
%:%2200=927%:%
%:%2201=927%:%
%:%2202=928%:%
%:%2203=928%:%
%:%2204=928%:%
%:%2205=928%:%
%:%2206=929%:%
%:%2207=929%:%
%:%2208=929%:%
%:%2209=929%:%
%:%2210=929%:%
%:%2211=930%:%
%:%2212=930%:%
%:%2213=931%:%
%:%2214=931%:%
%:%2215=931%:%
%:%2216=932%:%
%:%2217=932%:%
%:%2218=932%:%
%:%2219=932%:%
%:%2220=932%:%
%:%2221=933%:%
%:%2222=933%:%
%:%2223=933%:%
%:%2224=933%:%
%:%2225=933%:%
%:%2226=933%:%
%:%2227=934%:%
%:%2228=934%:%
%:%2229=934%:%
%:%2230=934%:%
%:%2231=934%:%
%:%2232=935%:%
%:%2233=935%:%
%:%2234=935%:%
%:%2235=935%:%
%:%2236=935%:%
%:%2237=936%:%
%:%2238=936%:%
%:%2239=937%:%
%:%2240=937%:%
%:%2241=937%:%
%:%2242=937%:%
%:%2243=937%:%
%:%2244=938%:%
%:%2250=938%:%
%:%2253=939%:%
%:%2254=940%:%
%:%2255=940%:%
%:%2258=941%:%
%:%2262=941%:%
%:%2263=941%:%
%:%2264=941%:%
%:%2265=942%:%
%:%2266=942%:%
%:%2267=943%:%
%:%2268=943%:%
%:%2269=943%:%
%:%2270=944%:%
%:%2271=944%:%
%:%2272=944%:%
%:%2273=944%:%
%:%2274=944%:%
%:%2275=945%:%
%:%2276=945%:%
%:%2277=945%:%
%:%2278=945%:%
%:%2279=946%:%
%:%2280=946%:%
%:%2281=946%:%
%:%2282=946%:%
%:%2283=946%:%
%:%2284=946%:%
%:%2285=947%:%
%:%2286=947%:%
%:%2287=947%:%
%:%2288=947%:%
%:%2289=948%:%
%:%2290=948%:%
%:%2291=948%:%
%:%2292=948%:%
%:%2293=949%:%
%:%2294=949%:%
%:%2295=949%:%
%:%2296=949%:%
%:%2297=950%:%
%:%2298=950%:%
%:%2299=950%:%
%:%2300=951%:%
%:%2301=951%:%
%:%2302=951%:%
%:%2303=951%:%
%:%2304=952%:%
%:%2305=952%:%
%:%2306=952%:%
%:%2307=952%:%
%:%2308=952%:%
%:%2309=953%:%
%:%2315=953%:%
%:%2318=954%:%
%:%2319=955%:%
%:%2320=955%:%
%:%2321=956%:%
%:%2322=957%:%
%:%2323=957%:%
%:%2324=958%:%
%:%2325=959%:%
%:%2326=960%:%
%:%2327=960%:%
%:%2328=961%:%
%:%2329=962%:%
%:%2330=963%:%
%:%2331=963%:%
%:%2332=964%:%
%:%2333=965%:%
%:%2334=966%:%
%:%2337=967%:%
%:%2338=968%:%
%:%2342=968%:%
%:%2343=968%:%
%:%2344=969%:%
%:%2345=969%:%
%:%2346=970%:%
%:%2347=970%:%
%:%2348=971%:%
%:%2349=971%:%
%:%2350=972%:%
%:%2351=972%:%
%:%2352=973%:%
%:%2353=973%:%
%:%2358=973%:%
%:%2361=974%:%
%:%2362=975%:%
%:%2363=975%:%
%:%2364=976%:%
%:%2365=977%:%
%:%2366=978%:%
%:%2369=979%:%
%:%2373=979%:%
%:%2374=979%:%
%:%2375=980%:%
%:%2376=980%:%
%:%2377=981%:%
%:%2378=981%:%
%:%2379=982%:%
%:%2380=982%:%
%:%2381=983%:%
%:%2382=983%:%
%:%2383=984%:%
%:%2384=984%:%
%:%2385=985%:%
%:%2386=985%:%
%:%2387=986%:%
%:%2388=986%:%
%:%2389=987%:%
%:%2390=987%:%
%:%2391=988%:%
%:%2392=988%:%
%:%2393=989%:%
%:%2394=989%:%
%:%2395=990%:%
%:%2396=990%:%
%:%2397=991%:%
%:%2398=991%:%
%:%2399=992%:%
%:%2400=992%:%
%:%2405=992%:%
%:%2408=993%:%
%:%2409=994%:%
%:%2410=994%:%
%:%2411=995%:%
%:%2412=996%:%
%:%2413=997%:%
%:%2414=998%:%
%:%2421=999%:%
%:%2422=999%:%
%:%2423=1000%:%
%:%2424=1000%:%
%:%2425=1000%:%
%:%2426=1001%:%
%:%2427=1001%:%
%:%2428=1001%:%
%:%2429=1002%:%
%:%2430=1002%:%
%:%2431=1002%:%
%:%2432=1003%:%
%:%2433=1004%:%
%:%2434=1004%:%
%:%2435=1005%:%
%:%2436=1005%:%
%:%2437=1006%:%
%:%2438=1006%:%
%:%2439=1007%:%
%:%2440=1007%:%
%:%2441=1008%:%
%:%2442=1008%:%
%:%2443=1009%:%
%:%2444=1009%:%
%:%2445=1010%:%
%:%2446=1010%:%
%:%2447=1011%:%
%:%2448=1011%:%
%:%2449=1012%:%
%:%2450=1012%:%
%:%2451=1013%:%
%:%2452=1013%:%
%:%2453=1014%:%
%:%2454=1014%:%
%:%2455=1015%:%
%:%2456=1015%:%
%:%2457=1016%:%
%:%2458=1016%:%
%:%2459=1016%:%
%:%2460=1017%:%
%:%2461=1017%:%
%:%2462=1018%:%
%:%2463=1018%:%
%:%2464=1019%:%
%:%2465=1019%:%
%:%2466=1020%:%
%:%2467=1020%:%
%:%2468=1020%:%
%:%2469=1020%:%
%:%2470=1021%:%
%:%2471=1022%:%
%:%2472=1022%:%
%:%2473=1022%:%
%:%2474=1022%:%
%:%2475=1022%:%
%:%2476=1023%:%
%:%2477=1023%:%
%:%2478=1023%:%
%:%2479=1024%:%
%:%2480=1024%:%
%:%2481=1025%:%
%:%2482=1025%:%
%:%2483=1026%:%
%:%2484=1026%:%
%:%2485=1027%:%
%:%2486=1027%:%
%:%2487=1027%:%
%:%2488=1028%:%
%:%2489=1028%:%
%:%2490=1028%:%
%:%2491=1028%:%
%:%2492=1029%:%
%:%2493=1030%:%
%:%2494=1030%:%
%:%2495=1030%:%
%:%2496=1030%:%
%:%2497=1031%:%
%:%2498=1031%:%
%:%2499=1031%:%
%:%2500=1032%:%
%:%2501=1032%:%
%:%2502=1033%:%
%:%2503=1033%:%
%:%2504=1034%:%
%:%2505=1034%:%
%:%2506=1035%:%
%:%2507=1035%:%
%:%2508=1035%:%
%:%2509=1036%:%
%:%2510=1036%:%
%:%2511=1036%:%
%:%2512=1036%:%
%:%2513=1037%:%
%:%2514=1038%:%
%:%2515=1038%:%
%:%2516=1039%:%
%:%2517=1039%:%
%:%2518=1040%:%
%:%2519=1040%:%
%:%2520=1041%:%
%:%2521=1041%:%
%:%2522=1042%:%
%:%2523=1043%:%
%:%2524=1043%:%
%:%2525=1044%:%
%:%2526=1044%:%
%:%2527=1045%:%
%:%2528=1045%:%
%:%2529=1046%:%
%:%2530=1046%:%
%:%2531=1047%:%
%:%2532=1047%:%
%:%2533=1048%:%
%:%2534=1048%:%
%:%2535=1049%:%
%:%2536=1049%:%
%:%2537=1050%:%
%:%2538=1050%:%
%:%2539=1051%:%
%:%2540=1052%:%
%:%2541=1052%:%
%:%2542=1052%:%
%:%2543=1053%:%
%:%2544=1054%:%
%:%2545=1055%:%
%:%2546=1056%:%
%:%2547=1057%:%
%:%2548=1057%:%
%:%2549=1058%:%
%:%2550=1058%:%
%:%2551=1059%:%
%:%2552=1060%:%
%:%2553=1060%:%
%:%2554=1060%:%
%:%2555=1060%:%
%:%2556=1060%:%
%:%2557=1061%:%
%:%2558=1061%:%
%:%2559=1061%:%
%:%2560=1061%:%
%:%2561=1061%:%
%:%2562=1062%:%
%:%2563=1063%:%
%:%2564=1063%:%
%:%2565=1063%:%
%:%2566=1063%:%
%:%2567=1063%:%
%:%2568=1064%:%
%:%2569=1065%:%
%:%2570=1065%:%
%:%2571=1065%:%
%:%2572=1065%:%
%:%2573=1065%:%
%:%2574=1066%:%
%:%2575=1066%:%
%:%2576=1066%:%
%:%2577=1066%:%
%:%2578=1066%:%
%:%2579=1067%:%
%:%2580=1068%:%
%:%2581=1068%:%
%:%2582=1068%:%
%:%2583=1068%:%
%:%2584=1068%:%
%:%2585=1069%:%
%:%2586=1069%:%
%:%2587=1069%:%
%:%2588=1069%:%
%:%2589=1069%:%
%:%2590=1070%:%
%:%2591=1071%:%
%:%2592=1071%:%
%:%2593=1071%:%
%:%2594=1071%:%
%:%2595=1071%:%
%:%2596=1072%:%
%:%2597=1072%:%
%:%2598=1072%:%
%:%2599=1072%:%
%:%2600=1072%:%
%:%2601=1073%:%
%:%2602=1074%:%
%:%2603=1074%:%
%:%2604=1075%:%
%:%2605=1075%:%
%:%2606=1076%:%
%:%2607=1076%:%
%:%2608=1077%:%
%:%2609=1077%:%
%:%2610=1078%:%
%:%2611=1079%:%
%:%2612=1079%:%
%:%2613=1080%:%
%:%2614=1080%:%
%:%2615=1081%:%
%:%2616=1081%:%
%:%2617=1082%:%
%:%2618=1082%:%
%:%2619=1083%:%
%:%2620=1083%:%
%:%2621=1084%:%
%:%2622=1085%:%
%:%2623=1085%:%
%:%2624=1085%:%
%:%2625=1085%:%
%:%2626=1086%:%
%:%2632=1086%:%
%:%2635=1087%:%
%:%2636=1088%:%
%:%2637=1088%:%
%:%2638=1089%:%
%:%2639=1090%:%
%:%2640=1091%:%
%:%2641=1092%:%
%:%2648=1093%:%
%:%2649=1093%:%
%:%2650=1094%:%
%:%2651=1094%:%
%:%2652=1095%:%
%:%2653=1096%:%
%:%2654=1096%:%
%:%2655=1097%:%
%:%2656=1097%:%
%:%2657=1098%:%
%:%2658=1098%:%
%:%2659=1099%:%
%:%2660=1099%:%
%:%2661=1100%:%
%:%2662=1100%:%
%:%2663=1101%:%
%:%2664=1101%:%
%:%2665=1102%:%
%:%2666=1102%:%
%:%2667=1103%:%
%:%2668=1103%:%
%:%2669=1104%:%
%:%2670=1104%:%
%:%2671=1105%:%
%:%2672=1105%:%
%:%2673=1106%:%
%:%2674=1106%:%
%:%2675=1107%:%
%:%2676=1108%:%
%:%2677=1108%:%
%:%2678=1108%:%
%:%2679=1109%:%
%:%2680=1109%:%
%:%2681=1110%:%
%:%2682=1110%:%
%:%2683=1111%:%
%:%2684=1111%:%
%:%2685=1112%:%
%:%2686=1112%:%
%:%2687=1113%:%
%:%2688=1114%:%
%:%2689=1114%:%
%:%2690=1115%:%
%:%2691=1115%:%
%:%2692=1116%:%
%:%2693=1116%:%
%:%2694=1117%:%
%:%2695=1117%:%
%:%2696=1118%:%
%:%2697=1118%:%
%:%2698=1119%:%
%:%2699=1120%:%
%:%2700=1120%:%
%:%2701=1120%:%
%:%2702=1121%:%
%:%2703=1121%:%
%:%2704=1122%:%
%:%2705=1122%:%
%:%2706=1123%:%
%:%2707=1123%:%
%:%2708=1124%:%
%:%2709=1125%:%
%:%2710=1125%:%
%:%2711=1125%:%
%:%2712=1126%:%
%:%2713=1126%:%
%:%2714=1126%:%
%:%2718=1130%:%
%:%2719=1131%:%
%:%2720=1131%:%
%:%2721=1132%:%
%:%2722=1132%:%
%:%2723=1133%:%
%:%2724=1133%:%
%:%2725=1133%:%
%:%2726=1134%:%
%:%2727=1135%:%
%:%2728=1136%:%
%:%2729=1136%:%
%:%2730=1137%:%
%:%2731=1137%:%
%:%2732=1137%:%
%:%2733=1137%:%
%:%2734=1137%:%
%:%2735=1138%:%
%:%2736=1138%:%
%:%2737=1138%:%
%:%2738=1138%:%
%:%2739=1138%:%
%:%2740=1139%:%
%:%2741=1140%:%
%:%2742=1140%:%
%:%2743=1141%:%
%:%2744=1141%:%
%:%2745=1141%:%
%:%2746=1141%:%
%:%2747=1142%:%
%:%2748=1142%:%
%:%2749=1142%:%
%:%2750=1143%:%
%:%2751=1143%:%
%:%2752=1143%:%
%:%2753=1144%:%
%:%2754=1144%:%
%:%2755=1144%:%
%:%2756=1145%:%
%:%2757=1145%:%
%:%2758=1145%:%
%:%2759=1146%:%
%:%2760=1147%:%
%:%2761=1147%:%
%:%2762=1148%:%
%:%2763=1148%:%
%:%2764=1148%:%
%:%2765=1148%:%
%:%2766=1149%:%
%:%2767=1149%:%
%:%2768=1149%:%
%:%2769=1150%:%
%:%2770=1150%:%
%:%2771=1150%:%
%:%2772=1151%:%
%:%2773=1151%:%
%:%2774=1151%:%
%:%2775=1152%:%
%:%2776=1152%:%
%:%2777=1152%:%
%:%2778=1153%:%
%:%2779=1154%:%
%:%2780=1154%:%
%:%2781=1155%:%
%:%2782=1155%:%
%:%2783=1156%:%
%:%2784=1156%:%
%:%2785=1157%:%
%:%2786=1157%:%
%:%2787=1158%:%
%:%2788=1158%:%
%:%2789=1159%:%
%:%2790=1159%:%
%:%2791=1160%:%
%:%2792=1161%:%
%:%2793=1161%:%
%:%2794=1161%:%
%:%2795=1161%:%
%:%2796=1161%:%
%:%2797=1162%:%
%:%2803=1162%:%
%:%2806=1163%:%
%:%2807=1164%:%
%:%2808=1164%:%
%:%2809=1165%:%
%:%2810=1166%:%
%:%2811=1166%:%
%:%2812=1167%:%
%:%2813=1168%:%
%:%2814=1169%:%
%:%2815=1169%:%
%:%2816=1170%:%
%:%2817=1171%:%
%:%2818=1172%:%
%:%2819=1172%:%
%:%2820=1173%:%
%:%2821=1174%:%
%:%2822=1175%:%
%:%2825=1176%:%
%:%2829=1176%:%
%:%2830=1176%:%
%:%2831=1177%:%
%:%2832=1177%:%
%:%2833=1178%:%
%:%2834=1178%:%
%:%2835=1179%:%
%:%2836=1179%:%
%:%2837=1180%:%
%:%2838=1180%:%
%:%2843=1180%:%
%:%2846=1181%:%
%:%2847=1182%:%
%:%2848=1182%:%
%:%2849=1183%:%
%:%2850=1184%:%
%:%2851=1185%:%
%:%2854=1186%:%
%:%2858=1186%:%
%:%2859=1186%:%
%:%2860=1187%:%
%:%2861=1187%:%
%:%2862=1188%:%
%:%2863=1188%:%
%:%2864=1189%:%
%:%2865=1189%:%
%:%2866=1190%:%
%:%2867=1190%:%
%:%2868=1191%:%
%:%2869=1191%:%
%:%2870=1192%:%
%:%2871=1192%:%
%:%2872=1193%:%
%:%2873=1193%:%
%:%2874=1194%:%
%:%2875=1194%:%
%:%2876=1195%:%
%:%2877=1195%:%
%:%2878=1196%:%
%:%2879=1196%:%
%:%2880=1197%:%
%:%2881=1197%:%
%:%2882=1198%:%
%:%2883=1198%:%
%:%2884=1199%:%
%:%2885=1199%:%
%:%2886=1200%:%
%:%2887=1200%:%
%:%2888=1201%:%
%:%2889=1201%:%
%:%2890=1202%:%
%:%2891=1202%:%
%:%2892=1203%:%
%:%2893=1203%:%
%:%2894=1204%:%
%:%2895=1204%:%
%:%2900=1204%:%
%:%2903=1205%:%
%:%2904=1206%:%
%:%2905=1206%:%
%:%2906=1207%:%
%:%2907=1208%:%
%:%2908=1209%:%
%:%2909=1210%:%
%:%2916=1211%:%
%:%2917=1211%:%
%:%2918=1212%:%
%:%2919=1212%:%
%:%2920=1212%:%
%:%2921=1213%:%
%:%2922=1213%:%
%:%2923=1213%:%
%:%2924=1214%:%
%:%2925=1215%:%
%:%2926=1215%:%
%:%2927=1216%:%
%:%2928=1216%:%
%:%2929=1217%:%
%:%2930=1217%:%
%:%2931=1218%:%
%:%2932=1218%:%
%:%2933=1219%:%
%:%2934=1219%:%
%:%2935=1219%:%
%:%2936=1220%:%
%:%2937=1220%:%
%:%2938=1220%:%
%:%2939=1221%:%
%:%2940=1222%:%
%:%2941=1222%:%
%:%2942=1223%:%
%:%2943=1223%:%
%:%2944=1224%:%
%:%2945=1224%:%
%:%2946=1225%:%
%:%2947=1225%:%
%:%2948=1226%:%
%:%2949=1226%:%
%:%2950=1227%:%
%:%2951=1227%:%
%:%2952=1228%:%
%:%2953=1228%:%
%:%2954=1229%:%
%:%2955=1229%:%
%:%2956=1230%:%
%:%2957=1230%:%
%:%2958=1231%:%
%:%2959=1231%:%
%:%2960=1232%:%
%:%2961=1232%:%
%:%2962=1233%:%
%:%2963=1233%:%
%:%2964=1234%:%
%:%2965=1234%:%
%:%2966=1235%:%
%:%2967=1235%:%
%:%2968=1236%:%
%:%2969=1236%:%
%:%2970=1237%:%
%:%2971=1237%:%
%:%2972=1238%:%
%:%2973=1238%:%
%:%2974=1239%:%
%:%2975=1239%:%
%:%2976=1240%:%
%:%2977=1241%:%
%:%2978=1241%:%
%:%2979=1241%:%
%:%2980=1241%:%
%:%2981=1241%:%
%:%2982=1242%:%
%:%2988=1242%:%
%:%2991=1243%:%
%:%2992=1244%:%
%:%2993=1244%:%
%:%2994=1245%:%
%:%2995=1246%:%
%:%2996=1246%:%
%:%2997=1247%:%
%:%2998=1248%:%
%:%2999=1248%:%
%:%3000=1249%:%
%:%3001=1250%:%
%:%3002=1251%:%
%:%3003=1251%:%
%:%3004=1252%:%
%:%3005=1253%:%
%:%3006=1254%:%
%:%3009=1255%:%
%:%3010=1256%:%
%:%3014=1256%:%
%:%3015=1256%:%
%:%3016=1257%:%
%:%3017=1257%:%
%:%3018=1258%:%
%:%3019=1258%:%
%:%3020=1259%:%
%:%3021=1259%:%
%:%3022=1260%:%
%:%3023=1260%:%
%:%3024=1261%:%
%:%3025=1261%:%
%:%3026=1262%:%
%:%3027=1262%:%
%:%3032=1262%:%
%:%3035=1263%:%
%:%3036=1264%:%
%:%3037=1264%:%
%:%3038=1265%:%
%:%3039=1266%:%
%:%3040=1267%:%
%:%3043=1268%:%
%:%3044=1269%:%
%:%3048=1269%:%
%:%3049=1269%:%
%:%3050=1270%:%
%:%3051=1270%:%
%:%3052=1271%:%
%:%3053=1271%:%
%:%3054=1272%:%
%:%3055=1272%:%
%:%3056=1273%:%
%:%3057=1273%:%
%:%3058=1274%:%
%:%3059=1274%:%
%:%3060=1275%:%
%:%3061=1275%:%
%:%3062=1276%:%
%:%3063=1276%:%
%:%3064=1277%:%
%:%3065=1277%:%
%:%3066=1278%:%
%:%3067=1278%:%
%:%3068=1279%:%
%:%3069=1279%:%
%:%3070=1280%:%
%:%3071=1280%:%
%:%3072=1281%:%
%:%3073=1281%:%
%:%3074=1282%:%
%:%3075=1282%:%
%:%3076=1283%:%
%:%3077=1283%:%
%:%3078=1284%:%
%:%3079=1284%:%
%:%3080=1285%:%
%:%3081=1285%:%
%:%3082=1286%:%
%:%3083=1286%:%
%:%3084=1287%:%
%:%3085=1287%:%
%:%3086=1288%:%
%:%3087=1288%:%
%:%3088=1289%:%
%:%3089=1289%:%
%:%3090=1290%:%
%:%3091=1290%:%
%:%3092=1291%:%
%:%3093=1291%:%
%:%3094=1292%:%
%:%3095=1292%:%
%:%3096=1293%:%
%:%3097=1293%:%
%:%3102=1293%:%
%:%3105=1294%:%
%:%3106=1295%:%
%:%3107=1296%:%
%:%3108=1296%:%
%:%3109=1297%:%
%:%3110=1298%:%
%:%3111=1299%:%
%:%3112=1300%:%
%:%3113=1301%:%
%:%3114=1302%:%
%:%3115=1303%:%
%:%3116=1304%:%
%:%3123=1305%:%
%:%3124=1305%:%
%:%3125=1306%:%
%:%3126=1306%:%
%:%3127=1307%:%
%:%3128=1307%:%
%:%3129=1308%:%
%:%3130=1308%:%
%:%3131=1309%:%
%:%3132=1309%:%
%:%3133=1310%:%
%:%3134=1311%:%
%:%3135=1311%:%
%:%3136=1311%:%
%:%3137=1311%:%
%:%3138=1312%:%
%:%3139=1312%:%
%:%3140=1312%:%
%:%3141=1312%:%
%:%3142=1312%:%
%:%3143=1313%:%
%:%3144=1314%:%
%:%3145=1314%:%
%:%3146=1315%:%
%:%3147=1315%:%
%:%3148=1316%:%
%:%3149=1316%:%
%:%3150=1317%:%
%:%3151=1317%:%
%:%3152=1318%:%
%:%3153=1318%:%
%:%3154=1318%:%
%:%3155=1319%:%
%:%3156=1319%:%
%:%3157=1320%:%
%:%3158=1320%:%
%:%3159=1321%:%
%:%3160=1322%:%
%:%3161=1322%:%
%:%3162=1323%:%
%:%3163=1323%:%
%:%3164=1324%:%
%:%3165=1324%:%
%:%3166=1325%:%
%:%3167=1325%:%
%:%3168=1326%:%
%:%3169=1326%:%
%:%3170=1326%:%
%:%3171=1327%:%
%:%3172=1327%:%
%:%3173=1328%:%
%:%3174=1328%:%
%:%3175=1329%:%
%:%3176=1330%:%
%:%3177=1330%:%
%:%3178=1331%:%
%:%3179=1331%:%
%:%3180=1332%:%
%:%3181=1332%:%
%:%3182=1332%:%
%:%3183=1333%:%
%:%3184=1333%:%
%:%3185=1334%:%
%:%3186=1334%:%
%:%3187=1335%:%
%:%3188=1335%:%
%:%3189=1336%:%
%:%3190=1336%:%
%:%3191=1337%:%
%:%3192=1337%:%
%:%3193=1337%:%
%:%3194=1338%:%
%:%3195=1338%:%
%:%3196=1339%:%
%:%3197=1339%:%
%:%3198=1340%:%
%:%3199=1340%:%
%:%3200=1341%:%
%:%3201=1341%:%
%:%3202=1341%:%
%:%3203=1342%:%
%:%3204=1342%:%
%:%3205=1343%:%
%:%3206=1343%:%
%:%3207=1344%:%
%:%3208=1344%:%
%:%3209=1344%:%
%:%3210=1344%:%
%:%3211=1345%:%
%:%3212=1346%:%
%:%3213=1346%:%
%:%3214=1346%:%
%:%3215=1346%:%
%:%3216=1347%:%
%:%3217=1347%:%
%:%3218=1347%:%
%:%3219=1348%:%
%:%3220=1348%:%
%:%3221=1349%:%
%:%3222=1349%:%
%:%3223=1349%:%
%:%3224=1349%:%
%:%3225=1350%:%
%:%3226=1350%:%
%:%3227=1350%:%
%:%3228=1351%:%
%:%3229=1351%:%
%:%3230=1352%:%
%:%3231=1352%:%
%:%3232=1352%:%
%:%3233=1353%:%
%:%3234=1353%:%
%:%3235=1354%:%
%:%3236=1354%:%
%:%3237=1355%:%
%:%3238=1355%:%
%:%3239=1356%:%
%:%3240=1356%:%
%:%3241=1356%:%
%:%3242=1357%:%
%:%3243=1357%:%
%:%3244=1358%:%
%:%3245=1358%:%
%:%3246=1359%:%
%:%3247=1359%:%
%:%3248=1359%:%
%:%3249=1359%:%
%:%3250=1360%:%
%:%3251=1361%:%
%:%3252=1361%:%
%:%3253=1361%:%
%:%3254=1362%:%
%:%3255=1362%:%
%:%3256=1363%:%
%:%3257=1363%:%
%:%3258=1364%:%
%:%3259=1364%:%
%:%3260=1365%:%
%:%3261=1366%:%
%:%3262=1366%:%
%:%3263=1367%:%
%:%3264=1367%:%
%:%3265=1368%:%
%:%3266=1368%:%
%:%3267=1369%:%
%:%3268=1369%:%
%:%3269=1370%:%
%:%3270=1370%:%
%:%3271=1371%:%
%:%3272=1371%:%
%:%3273=1372%:%
%:%3274=1372%:%
%:%3275=1373%:%
%:%3276=1373%:%
%:%3277=1374%:%
%:%3278=1374%:%
%:%3279=1375%:%
%:%3280=1375%:%
%:%3281=1376%:%
%:%3282=1376%:%
%:%3283=1377%:%
%:%3284=1378%:%
%:%3285=1378%:%
%:%3286=1378%:%
%:%3287=1378%:%
%:%3288=1378%:%
%:%3289=1379%:%
%:%3290=1379%:%
%:%3291=1379%:%
%:%3292=1379%:%
%:%3293=1379%:%
%:%3294=1380%:%
%:%3295=1381%:%
%:%3296=1381%:%
%:%3297=1382%:%
%:%3298=1382%:%
%:%3299=1383%:%
%:%3300=1383%:%
%:%3301=1383%:%
%:%3302=1384%:%
%:%3303=1384%:%
%:%3304=1384%:%
%:%3305=1384%:%
%:%3306=1385%:%
%:%3307=1386%:%
%:%3308=1386%:%
%:%3309=1387%:%
%:%3310=1387%:%
%:%3311=1388%:%
%:%3312=1388%:%
%:%3313=1388%:%
%:%3314=1389%:%
%:%3315=1389%:%
%:%3316=1389%:%
%:%3317=1389%:%
%:%3318=1389%:%
%:%3319=1390%:%
%:%3320=1391%:%
%:%3321=1391%:%
%:%3322=1392%:%
%:%3323=1392%:%
%:%3324=1393%:%
%:%3325=1393%:%
%:%3326=1394%:%
%:%3327=1394%:%
%:%3328=1395%:%
%:%3329=1395%:%
%:%3330=1396%:%
%:%3331=1396%:%
%:%3332=1397%:%
%:%3333=1397%:%
%:%3334=1398%:%
%:%3335=1398%:%
%:%3336=1399%:%
%:%3337=1400%:%
%:%3338=1400%:%
%:%3339=1401%:%
%:%3340=1401%:%
%:%3341=1402%:%
%:%3342=1402%:%
%:%3343=1402%:%
%:%3344=1403%:%
%:%3345=1403%:%
%:%3346=1404%:%
%:%3347=1404%:%
%:%3348=1405%:%
%:%3349=1405%:%
%:%3350=1406%:%
%:%3351=1406%:%
%:%3352=1406%:%
%:%3353=1406%:%
%:%3354=1407%:%
%:%3355=1408%:%
%:%3356=1408%:%
%:%3357=1409%:%
%:%3358=1409%:%
%:%3359=1410%:%
%:%3360=1410%:%
%:%3361=1411%:%
%:%3362=1411%:%
%:%3363=1412%:%
%:%3364=1412%:%
%:%3365=1413%:%
%:%3366=1414%:%
%:%3367=1415%:%
%:%3368=1415%:%
%:%3369=1416%:%
%:%3370=1416%:%
%:%3371=1417%:%
%:%3372=1417%:%
%:%3373=1417%:%
%:%3374=1418%:%
%:%3375=1418%:%
%:%3376=1418%:%
%:%3377=1418%:%
%:%3378=1419%:%
%:%3379=1419%:%
%:%3380=1419%:%
%:%3381=1419%:%
%:%3382=1420%:%
%:%3383=1420%:%
%:%3384=1420%:%
%:%3385=1420%:%
%:%3386=1421%:%
%:%3387=1421%:%
%:%3388=1421%:%
%:%3389=1421%:%
%:%3390=1421%:%
%:%3391=1422%:%
%:%3392=1422%:%
%:%3393=1423%:%
%:%3394=1423%:%
%:%3395=1423%:%
%:%3396=1424%:%
%:%3397=1424%:%
%:%3398=1424%:%
%:%3399=1424%:%
%:%3400=1425%:%
%:%3401=1425%:%
%:%3402=1425%:%
%:%3403=1425%:%
%:%3404=1425%:%
%:%3405=1426%:%
%:%3406=1426%:%
%:%3407=1426%:%
%:%3408=1426%:%
%:%3409=1427%:%
%:%3410=1427%:%
%:%3411=1427%:%
%:%3412=1427%:%
%:%3413=1427%:%
%:%3414=1428%:%
%:%3415=1428%:%
%:%3416=1429%:%
%:%3417=1430%:%
%:%3418=1430%:%
%:%3419=1431%:%
%:%3420=1431%:%
%:%3421=1432%:%
%:%3422=1432%:%
%:%3423=1433%:%
%:%3424=1433%:%
%:%3425=1434%:%
%:%3426=1434%:%
%:%3427=1435%:%
%:%3428=1435%:%
%:%3429=1436%:%
%:%3430=1436%:%
%:%3431=1437%:%
%:%3432=1437%:%
%:%3433=1438%:%
%:%3434=1438%:%
%:%3435=1439%:%
%:%3436=1439%:%
%:%3437=1440%:%
%:%3438=1440%:%
%:%3439=1441%:%
%:%3440=1441%:%
%:%3441=1442%:%
%:%3442=1442%:%
%:%3443=1443%:%
%:%3444=1443%:%
%:%3445=1444%:%
%:%3446=1444%:%
%:%3447=1445%:%
%:%3448=1445%:%
%:%3449=1445%:%
%:%3450=1446%:%
%:%3451=1446%:%
%:%3452=1447%:%
%:%3453=1447%:%
%:%3454=1448%:%
%:%3455=1448%:%
%:%3456=1449%:%
%:%3457=1450%:%
%:%3458=1450%:%
%:%3459=1450%:%
%:%3460=1450%:%
%:%3461=1450%:%
%:%3462=1451%:%
%:%3468=1451%:%
%:%3471=1452%:%
%:%3472=1453%:%
%:%3473=1453%:%
%:%3474=1454%:%
%:%3475=1455%:%
%:%3476=1456%:%
%:%3477=1457%:%
%:%3478=1458%:%
%:%3479=1459%:%
%:%3480=1460%:%
%:%3481=1461%:%
%:%3482=1462%:%
%:%3489=1463%:%
%:%3490=1463%:%
%:%3491=1464%:%
%:%3492=1465%:%
%:%3493=1465%:%
%:%3494=1466%:%
%:%3495=1466%:%
%:%3496=1467%:%
%:%3497=1467%:%
%:%3498=1468%:%
%:%3499=1468%:%
%:%3500=1468%:%
%:%3501=1468%:%
%:%3502=1469%:%
%:%3503=1469%:%
%:%3504=1469%:%
%:%3505=1470%:%
%:%3506=1470%:%
%:%3507=1471%:%
%:%3508=1471%:%
%:%3509=1472%:%
%:%3510=1472%:%
%:%3511=1473%:%
%:%3512=1473%:%
%:%3513=1474%:%
%:%3514=1474%:%
%:%3515=1475%:%
%:%3516=1475%:%
%:%3517=1476%:%
%:%3518=1476%:%
%:%3519=1477%:%
%:%3520=1477%:%
%:%3521=1477%:%
%:%3522=1477%:%
%:%3523=1477%:%
%:%3524=1478%:%
%:%3525=1478%:%
%:%3526=1478%:%
%:%3527=1478%:%
%:%3528=1478%:%
%:%3529=1479%:%
%:%3530=1479%:%
%:%3531=1479%:%
%:%3532=1479%:%
%:%3533=1480%:%
%:%3534=1480%:%
%:%3535=1480%:%
%:%3536=1480%:%
%:%3537=1481%:%
%:%3538=1481%:%
%:%3539=1481%:%
%:%3540=1481%:%
%:%3541=1481%:%
%:%3542=1482%:%
%:%3543=1482%:%
%:%3544=1483%:%
%:%3545=1483%:%
%:%3546=1483%:%
%:%3547=1484%:%
%:%3548=1484%:%
%:%3549=1485%:%
%:%3550=1485%:%
%:%3551=1486%:%
%:%3552=1486%:%
%:%3553=1487%:%
%:%3554=1487%:%
%:%3555=1488%:%
%:%3556=1489%:%
%:%3557=1489%:%
%:%3558=1490%:%
%:%3559=1490%:%
%:%3560=1491%:%
%:%3561=1491%:%
%:%3562=1492%:%
%:%3563=1492%:%
%:%3564=1493%:%
%:%3565=1494%:%
%:%3566=1494%:%
%:%3567=1495%:%
%:%3568=1495%:%
%:%3569=1496%:%
%:%3570=1496%:%
%:%3571=1497%:%
%:%3572=1497%:%
%:%3573=1498%:%
%:%3574=1498%:%
%:%3575=1499%:%
%:%3576=1499%:%
%:%3577=1500%:%
%:%3578=1500%:%
%:%3579=1501%:%
%:%3580=1501%:%
%:%3581=1502%:%
%:%3582=1502%:%
%:%3583=1503%:%
%:%3584=1503%:%
%:%3585=1504%:%
%:%3586=1504%:%
%:%3587=1505%:%
%:%3588=1505%:%
%:%3589=1505%:%
%:%3590=1505%:%
%:%3591=1506%:%
%:%3592=1506%:%
%:%3593=1506%:%
%:%3594=1506%:%
%:%3595=1507%:%
%:%3596=1508%:%
%:%3597=1508%:%
%:%3598=1509%:%
%:%3599=1509%:%
%:%3600=1510%:%
%:%3601=1510%:%
%:%3602=1511%:%
%:%3603=1511%:%
%:%3604=1512%:%
%:%3605=1513%:%
%:%3606=1513%:%
%:%3607=1514%:%
%:%3608=1514%:%
%:%3609=1515%:%
%:%3610=1515%:%
%:%3611=1515%:%
%:%3612=1516%:%
%:%3613=1516%:%
%:%3614=1516%:%
%:%3615=1516%:%
%:%3616=1517%:%
%:%3617=1517%:%
%:%3618=1517%:%
%:%3619=1517%:%
%:%3620=1517%:%
%:%3621=1518%:%
%:%3622=1518%:%
%:%3623=1518%:%
%:%3624=1518%:%
%:%3625=1519%:%
%:%3626=1519%:%
%:%3627=1520%:%
%:%3628=1521%:%
%:%3629=1521%:%
%:%3630=1522%:%
%:%3631=1522%:%
%:%3632=1523%:%
%:%3633=1523%:%
%:%3634=1523%:%
%:%3635=1524%:%
%:%3636=1524%:%
%:%3637=1524%:%
%:%3638=1524%:%
%:%3639=1524%:%
%:%3640=1525%:%
%:%3641=1525%:%
%:%3642=1525%:%
%:%3643=1526%:%
%:%3644=1526%:%
%:%3645=1527%:%
%:%3646=1527%:%
%:%3647=1528%:%
%:%3648=1528%:%
%:%3649=1529%:%
%:%3650=1529%:%
%:%3651=1530%:%
%:%3652=1531%:%
%:%3653=1531%:%
%:%3654=1532%:%
%:%3655=1532%:%
%:%3656=1533%:%
%:%3657=1533%:%
%:%3658=1534%:%
%:%3659=1534%:%
%:%3660=1535%:%
%:%3661=1535%:%
%:%3662=1535%:%
%:%3663=1535%:%
%:%3664=1535%:%
%:%3665=1536%:%
%:%3666=1536%:%
%:%3667=1536%:%
%:%3668=1536%:%
%:%3669=1537%:%
%:%3670=1537%:%
%:%3671=1537%:%
%:%3672=1538%:%
%:%3673=1538%:%
%:%3674=1539%:%
%:%3675=1539%:%
%:%3676=1540%:%
%:%3677=1540%:%
%:%3678=1541%:%
%:%3679=1541%:%
%:%3680=1542%:%
%:%3681=1542%:%
%:%3682=1543%:%
%:%3683=1543%:%
%:%3684=1544%:%
%:%3685=1544%:%
%:%3686=1545%:%
%:%3687=1545%:%
%:%3688=1546%:%
%:%3689=1546%:%
%:%3690=1546%:%
%:%3691=1546%:%
%:%3692=1546%:%
%:%3693=1547%:%
%:%3694=1547%:%
%:%3695=1547%:%
%:%3696=1548%:%
%:%3697=1548%:%
%:%3698=1549%:%
%:%3699=1549%:%
%:%3700=1550%:%
%:%3701=1550%:%
%:%3702=1551%:%
%:%3703=1551%:%
%:%3704=1552%:%
%:%3705=1553%:%
%:%3706=1553%:%
%:%3707=1554%:%
%:%3708=1554%:%
%:%3709=1555%:%
%:%3710=1555%:%
%:%3711=1556%:%
%:%3712=1556%:%
%:%3713=1557%:%
%:%3714=1557%:%
%:%3715=1558%:%
%:%3716=1558%:%
%:%3717=1559%:%
%:%3718=1559%:%
%:%3719=1560%:%
%:%3720=1560%:%
%:%3721=1561%:%
%:%3722=1561%:%
%:%3723=1562%:%
%:%3724=1562%:%
%:%3725=1563%:%
%:%3726=1563%:%
%:%3727=1564%:%
%:%3728=1564%:%
%:%3729=1565%:%
%:%3730=1565%:%
%:%3731=1566%:%
%:%3732=1566%:%
%:%3733=1567%:%
%:%3734=1567%:%
%:%3735=1568%:%
%:%3736=1568%:%
%:%3737=1569%:%
%:%3738=1569%:%
%:%3739=1570%:%
%:%3740=1570%:%
%:%3741=1571%:%
%:%3742=1571%:%
%:%3743=1572%:%
%:%3744=1572%:%
%:%3745=1573%:%
%:%3746=1573%:%
%:%3747=1574%:%
%:%3748=1574%:%
%:%3749=1575%:%
%:%3750=1575%:%
%:%3751=1576%:%
%:%3752=1576%:%
%:%3753=1577%:%
%:%3754=1577%:%
%:%3755=1578%:%
%:%3756=1578%:%
%:%3757=1579%:%
%:%3758=1579%:%
%:%3759=1580%:%
%:%3760=1580%:%
%:%3761=1581%:%
%:%3762=1581%:%
%:%3763=1582%:%
%:%3764=1582%:%
%:%3765=1583%:%
%:%3766=1583%:%
%:%3767=1584%:%
%:%3768=1584%:%
%:%3769=1585%:%
%:%3770=1585%:%
%:%3771=1585%:%
%:%3772=1586%:%
%:%3773=1586%:%
%:%3774=1587%:%
%:%3775=1587%:%
%:%3776=1588%:%
%:%3777=1588%:%
%:%3778=1589%:%
%:%3779=1589%:%
%:%3780=1590%:%
%:%3781=1590%:%
%:%3782=1590%:%
%:%3783=1590%:%
%:%3784=1591%:%
%:%3790=1591%:%
%:%3793=1592%:%
%:%3794=1593%:%
%:%3795=1593%:%
%:%3796=1594%:%
%:%3797=1595%:%
%:%3798=1596%:%
%:%3799=1597%:%
%:%3800=1598%:%
%:%3801=1599%:%
%:%3802=1600%:%
%:%3803=1601%:%
%:%3804=1602%:%
%:%3805=1603%:%
%:%3806=1604%:%
%:%3807=1605%:%
%:%3808=1606%:%
%:%3809=1607%:%
%:%3810=1608%:%
%:%3817=1609%:%
%:%3818=1609%:%
%:%3819=1610%:%
%:%3820=1611%:%
%:%3821=1611%:%
%:%3822=1611%:%
%:%3823=1612%:%
%:%3824=1612%:%
%:%3825=1612%:%
%:%3826=1613%:%
%:%3827=1614%:%
%:%3828=1614%:%
%:%3829=1615%:%
%:%3830=1615%:%
%:%3831=1615%:%
%:%3832=1615%:%
%:%3833=1616%:%
%:%3834=1616%:%
%:%3835=1616%:%
%:%3836=1616%:%
%:%3837=1617%:%
%:%3838=1618%:%
%:%3839=1618%:%
%:%3840=1618%:%
%:%3841=1618%:%
%:%3842=1619%:%
%:%3843=1619%:%
%:%3844=1619%:%
%:%3845=1619%:%
%:%3846=1620%:%
%:%3847=1620%:%
%:%3848=1620%:%
%:%3849=1620%:%
%:%3850=1621%:%
%:%3851=1621%:%
%:%3852=1621%:%
%:%3853=1621%:%
%:%3854=1622%:%
%:%3855=1622%:%
%:%3856=1622%:%
%:%3857=1622%:%
%:%3858=1623%:%
%:%3859=1623%:%
%:%3860=1623%:%
%:%3861=1623%:%
%:%3862=1624%:%
%:%3863=1625%:%
%:%3864=1625%:%
%:%3865=1626%:%
%:%3866=1626%:%
%:%3867=1627%:%
%:%3868=1627%:%
%:%3869=1627%:%
%:%3870=1628%:%
%:%3871=1628%:%
%:%3872=1628%:%
%:%3873=1629%:%
%:%3874=1629%:%
%:%3875=1630%:%
%:%3876=1630%:%
%:%3877=1631%:%
%:%3878=1631%:%
%:%3879=1632%:%
%:%3880=1632%:%
%:%3881=1633%:%
%:%3882=1633%:%
%:%3883=1633%:%
%:%3884=1634%:%
%:%3885=1634%:%
%:%3886=1634%:%
%:%3887=1634%:%
%:%3888=1634%:%
%:%3889=1635%:%
%:%3890=1635%:%
%:%3891=1635%:%
%:%3892=1635%:%
%:%3893=1635%:%
%:%3894=1636%:%
%:%3895=1636%:%
%:%3896=1636%:%
%:%3897=1636%:%
%:%3898=1636%:%
%:%3899=1636%:%
%:%3900=1637%:%
%:%3901=1637%:%
%:%3902=1637%:%
%:%3903=1637%:%
%:%3904=1637%:%
%:%3905=1637%:%
%:%3906=1638%:%
%:%3907=1639%:%
%:%3908=1639%:%
%:%3909=1640%:%
%:%3910=1640%:%
%:%3911=1641%:%
%:%3912=1641%:%
%:%3913=1642%:%
%:%3914=1642%:%
%:%3915=1643%:%
%:%3916=1643%:%
%:%3917=1644%:%
%:%3918=1644%:%
%:%3919=1645%:%
%:%3920=1645%:%
%:%3921=1646%:%
%:%3922=1646%:%
%:%3923=1647%:%
%:%3924=1647%:%
%:%3925=1648%:%
%:%3926=1648%:%
%:%3927=1649%:%
%:%3928=1650%:%
%:%3929=1650%:%
%:%3930=1651%:%
%:%3931=1651%:%
%:%3932=1652%:%
%:%3933=1652%:%
%:%3934=1652%:%
%:%3935=1653%:%
%:%3936=1653%:%
%:%3937=1653%:%
%:%3938=1653%:%
%:%3939=1653%:%
%:%3940=1654%:%
%:%3941=1655%:%
%:%3942=1655%:%
%:%3943=1655%:%
%:%3944=1655%:%
%:%3945=1655%:%
%:%3946=1656%:%
%:%3947=1656%:%
%:%3948=1656%:%
%:%3949=1657%:%
%:%3950=1657%:%
%:%3951=1658%:%
%:%3952=1658%:%
%:%3953=1659%:%
%:%3954=1659%:%
%:%3955=1660%:%
%:%3956=1660%:%
%:%3957=1661%:%
%:%3958=1661%:%
%:%3959=1662%:%
%:%3960=1662%:%
%:%3961=1663%:%
%:%3962=1663%:%
%:%3963=1664%:%
%:%3964=1664%:%
%:%3965=1665%:%
%:%3966=1665%:%
%:%3967=1666%:%
%:%3968=1666%:%
%:%3969=1666%:%
%:%3970=1666%:%
%:%3971=1666%:%
%:%3972=1666%:%
%:%3973=1667%:%
%:%3974=1667%:%
%:%3975=1667%:%
%:%3976=1668%:%
%:%3977=1668%:%
%:%3978=1669%:%
%:%3979=1669%:%
%:%3980=1670%:%
%:%3981=1670%:%
%:%3982=1671%:%
%:%3983=1671%:%
%:%3984=1672%:%
%:%3985=1672%:%
%:%3986=1673%:%
%:%3987=1673%:%
%:%3988=1674%:%
%:%3989=1674%:%
%:%3990=1675%:%
%:%3991=1675%:%
%:%3992=1676%:%
%:%3993=1676%:%
%:%3994=1677%:%
%:%3995=1677%:%
%:%3996=1678%:%
%:%3997=1679%:%
%:%3998=1679%:%
%:%3999=1680%:%
%:%4000=1680%:%
%:%4001=1681%:%
%:%4002=1681%:%
%:%4003=1682%:%
%:%4004=1682%:%
%:%4005=1683%:%
%:%4006=1683%:%
%:%4007=1684%:%
%:%4008=1684%:%
%:%4009=1685%:%
%:%4010=1686%:%
%:%4011=1686%:%
%:%4012=1687%:%
%:%4013=1687%:%
%:%4014=1688%:%
%:%4015=1688%:%
%:%4016=1689%:%
%:%4017=1689%:%
%:%4018=1690%:%
%:%4019=1690%:%
%:%4020=1691%:%
%:%4021=1691%:%
%:%4022=1692%:%
%:%4023=1692%:%
%:%4024=1693%:%
%:%4025=1694%:%
%:%4026=1694%:%
%:%4027=1694%:%
%:%4028=1694%:%
%:%4029=1694%:%
%:%4030=1695%:%
%:%4031=1695%:%
%:%4032=1696%:%
%:%4033=1696%:%
%:%4034=1696%:%
%:%4035=1697%:%
%:%4036=1697%:%
%:%4037=1697%:%
%:%4038=1698%:%
%:%4039=1699%:%
%:%4040=1700%:%
%:%4041=1701%:%
%:%4042=1702%:%
%:%4043=1702%:%
%:%4044=1702%:%
%:%4045=1703%:%
%:%4046=1704%:%
%:%4047=1704%:%
%:%4048=1704%:%
%:%4049=1704%:%
%:%4050=1704%:%
%:%4051=1705%:%
%:%4052=1705%:%
%:%4053=1705%:%
%:%4054=1705%:%
%:%4055=1705%:%
%:%4056=1706%:%
%:%4057=1707%:%
%:%4058=1707%:%
%:%4059=1707%:%
%:%4060=1707%:%
%:%4061=1707%:%
%:%4062=1708%:%
%:%4063=1708%:%
%:%4064=1708%:%
%:%4065=1708%:%
%:%4066=1708%:%
%:%4067=1709%:%
%:%4068=1709%:%
%:%4069=1709%:%
%:%4070=1709%:%
%:%4071=1709%:%
%:%4072=1710%:%
%:%4073=1710%:%
%:%4074=1710%:%
%:%4075=1710%:%
%:%4076=1710%:%
%:%4077=1711%:%
%:%4078=1712%:%
%:%4079=1712%:%
%:%4080=1713%:%
%:%4081=1713%:%
%:%4082=1714%:%
%:%4083=1714%:%
%:%4084=1715%:%
%:%4085=1715%:%
%:%4086=1716%:%
%:%4087=1716%:%
%:%4088=1717%:%
%:%4089=1717%:%
%:%4090=1718%:%
%:%4091=1719%:%
%:%4092=1719%:%
%:%4093=1720%:%
%:%4094=1720%:%
%:%4095=1721%:%
%:%4096=1721%:%
%:%4097=1722%:%
%:%4098=1722%:%
%:%4099=1723%:%
%:%4100=1723%:%
%:%4101=1724%:%
%:%4102=1724%:%
%:%4103=1725%:%
%:%4104=1726%:%
%:%4105=1726%:%
%:%4106=1727%:%
%:%4107=1727%:%
%:%4108=1728%:%
%:%4109=1728%:%
%:%4110=1729%:%
%:%4111=1729%:%
%:%4112=1730%:%
%:%4113=1730%:%
%:%4114=1731%:%
%:%4115=1731%:%
%:%4116=1732%:%
%:%4117=1732%:%
%:%4118=1733%:%
%:%4119=1733%:%
%:%4120=1733%:%
%:%4121=1733%:%
%:%4122=1733%:%
%:%4123=1734%:%
%:%4124=1734%:%
%:%4125=1735%:%
%:%4126=1736%:%
%:%4127=1736%:%
%:%4128=1737%:%
%:%4129=1737%:%
%:%4130=1738%:%
%:%4131=1739%:%
%:%4132=1739%:%
%:%4133=1740%:%
%:%4134=1740%:%
%:%4135=1741%:%
%:%4136=1741%:%
%:%4137=1742%:%
%:%4138=1742%:%
%:%4139=1743%:%
%:%4140=1743%:%
%:%4141=1744%:%
%:%4142=1744%:%
%:%4143=1745%:%
%:%4144=1745%:%
%:%4145=1746%:%
%:%4146=1746%:%
%:%4147=1747%:%
%:%4148=1747%:%
%:%4149=1748%:%
%:%4150=1748%:%
%:%4151=1749%:%
%:%4152=1749%:%
%:%4153=1750%:%
%:%4154=1750%:%
%:%4155=1751%:%
%:%4156=1751%:%
%:%4157=1752%:%
%:%4158=1752%:%
%:%4159=1753%:%
%:%4160=1753%:%
%:%4161=1754%:%
%:%4162=1754%:%
%:%4163=1755%:%
%:%4164=1755%:%
%:%4165=1756%:%
%:%4166=1756%:%
%:%4167=1757%:%
%:%4168=1758%:%
%:%4169=1758%:%
%:%4170=1759%:%
%:%4171=1759%:%
%:%4172=1760%:%
%:%4173=1760%:%
%:%4174=1761%:%
%:%4175=1761%:%
%:%4176=1762%:%
%:%4177=1762%:%
%:%4178=1763%:%
%:%4179=1764%:%
%:%4180=1764%:%
%:%4181=1764%:%
%:%4182=1764%:%
%:%4183=1765%:%
%:%4184=1766%:%
%:%4185=1766%:%
%:%4186=1767%:%
%:%4187=1767%:%
%:%4188=1768%:%
%:%4189=1768%:%
%:%4190=1769%:%
%:%4191=1769%:%
%:%4192=1770%:%
%:%4193=1770%:%
%:%4194=1770%:%
%:%4195=1770%:%
%:%4196=1771%:%
%:%4202=1771%:%
%:%4205=1772%:%
%:%4206=1773%:%
%:%4207=1773%:%
%:%4208=1774%:%
%:%4209=1775%:%
%:%4210=1776%:%
%:%4211=1777%:%
%:%4212=1778%:%
%:%4213=1779%:%
%:%4214=1780%:%
%:%4215=1781%:%
%:%4216=1782%:%
%:%4219=1783%:%
%:%4220=1784%:%
%:%4224=1784%:%
%:%4225=1784%:%
%:%4226=1785%:%
%:%4227=1785%:%
%:%4228=1786%:%
%:%4229=1786%:%
%:%4230=1787%:%
%:%4231=1787%:%
%:%4232=1788%:%
%:%4233=1788%:%
%:%4234=1789%:%
%:%4235=1789%:%
%:%4236=1790%:%
%:%4237=1790%:%
%:%4238=1791%:%
%:%4239=1791%:%
%:%4240=1792%:%
%:%4241=1792%:%
%:%4242=1793%:%
%:%4243=1793%:%
%:%4244=1794%:%
%:%4245=1794%:%
%:%4246=1795%:%
%:%4247=1795%:%
%:%4248=1796%:%
%:%4249=1796%:%
%:%4250=1797%:%
%:%4251=1797%:%
%:%4252=1798%:%
%:%4253=1798%:%
%:%4254=1799%:%
%:%4255=1800%:%
%:%4256=1800%:%
%:%4257=1801%:%
%:%4258=1801%:%
%:%4259=1802%:%
%:%4260=1802%:%
%:%4261=1803%:%
%:%4262=1803%:%
%:%4263=1804%:%
%:%4264=1804%:%
%:%4265=1805%:%
%:%4266=1805%:%
%:%4267=1806%:%
%:%4273=1806%:%
%:%4276=1807%:%
%:%4277=1808%:%
%:%4278=1808%:%
%:%4285=1809%:%

%
\begin{isabellebody}%
\setisabellecontext{RecFun{\isacharunderscore}{\kern0pt}M{\isacharunderscore}{\kern0pt}Memrel}%
%
\isadelimtheory
%
\endisadelimtheory
%
\isatagtheory
\isacommand{theory}\isamarkupfalse%
\ RecFun{\isacharunderscore}{\kern0pt}M{\isacharunderscore}{\kern0pt}Memrel\isanewline
\ \ \isakeyword{imports}\ \isanewline
\ \ \ \ ZF\ \isanewline
\ \ \ \ RecFun{\isacharunderscore}{\kern0pt}M\isanewline
\isakeyword{begin}%
\endisatagtheory
{\isafoldtheory}%
%
\isadelimtheory
\ \isanewline
%
\endisadelimtheory
\isanewline
\isacommand{definition}\isamarkupfalse%
\ InEclose\ \isakeyword{where}\ {\isachardoublequoteopen}InEclose{\isacharparenleft}{\kern0pt}y{\isacharcomma}{\kern0pt}\ x{\isacharparenright}{\kern0pt}\ {\isasymequiv}\ y\ {\isasymin}\ eclose{\isacharparenleft}{\kern0pt}x{\isacharparenright}{\kern0pt}{\isachardoublequoteclose}\ \ \isanewline
\isacommand{definition}\isamarkupfalse%
\ InEclose{\isacharunderscore}{\kern0pt}fm\ \isakeyword{where}\ {\isachardoublequoteopen}InEclose{\isacharunderscore}{\kern0pt}fm\ {\isasymequiv}\ Exists{\isacharparenleft}{\kern0pt}And{\isacharparenleft}{\kern0pt}is{\isacharunderscore}{\kern0pt}eclose{\isacharunderscore}{\kern0pt}fm{\isacharparenleft}{\kern0pt}{\isadigit{2}}{\isacharcomma}{\kern0pt}\ {\isadigit{0}}{\isacharparenright}{\kern0pt}{\isacharcomma}{\kern0pt}\ Member{\isacharparenleft}{\kern0pt}{\isadigit{1}}{\isacharcomma}{\kern0pt}\ {\isadigit{0}}{\isacharparenright}{\kern0pt}{\isacharparenright}{\kern0pt}{\isacharparenright}{\kern0pt}{\isachardoublequoteclose}\isanewline
\isanewline
\isacommand{context}\isamarkupfalse%
\ M{\isacharunderscore}{\kern0pt}ctm\ \isanewline
\isakeyword{begin}\ \isanewline
\isanewline
\isacommand{lemma}\isamarkupfalse%
\ Relation{\isacharunderscore}{\kern0pt}fm{\isacharunderscore}{\kern0pt}InEclose\ {\isacharcolon}{\kern0pt}\ \isanewline
\ \ {\isachardoublequoteopen}Relation{\isacharunderscore}{\kern0pt}fm{\isacharparenleft}{\kern0pt}InEclose{\isacharcomma}{\kern0pt}\ InEclose{\isacharunderscore}{\kern0pt}fm{\isacharparenright}{\kern0pt}{\isachardoublequoteclose}\ \isanewline
%
\isadelimproof
\ \ %
\endisadelimproof
%
\isatagproof
\isacommand{unfolding}\isamarkupfalse%
\ Relation{\isacharunderscore}{\kern0pt}fm{\isacharunderscore}{\kern0pt}def\ \isanewline
\ \ \isacommand{apply}\isamarkupfalse%
{\isacharparenleft}{\kern0pt}rule\ conjI{\isacharcomma}{\kern0pt}\ simp\ add{\isacharcolon}{\kern0pt}InEclose{\isacharunderscore}{\kern0pt}fm{\isacharunderscore}{\kern0pt}def{\isacharcomma}{\kern0pt}\ rule\ conjI{\isacharcomma}{\kern0pt}\ simp\ add{\isacharcolon}{\kern0pt}InEclose{\isacharunderscore}{\kern0pt}fm{\isacharunderscore}{\kern0pt}def{\isacharparenright}{\kern0pt}\isanewline
\ \ \isacommand{apply}\isamarkupfalse%
{\isacharparenleft}{\kern0pt}subst\ arity{\isacharunderscore}{\kern0pt}is{\isacharunderscore}{\kern0pt}eclose{\isacharunderscore}{\kern0pt}fm{\isacharcomma}{\kern0pt}\ simp{\isacharcomma}{\kern0pt}\ simp{\isacharparenright}{\kern0pt}\isanewline
\ \ \ \isacommand{apply}\isamarkupfalse%
{\isacharparenleft}{\kern0pt}simp\ del{\isacharcolon}{\kern0pt}FOL{\isacharunderscore}{\kern0pt}sats{\isacharunderscore}{\kern0pt}iff\ pair{\isacharunderscore}{\kern0pt}abs\ add{\isacharcolon}{\kern0pt}\ fm{\isacharunderscore}{\kern0pt}defs\ nat{\isacharunderscore}{\kern0pt}simp{\isacharunderscore}{\kern0pt}union{\isacharparenright}{\kern0pt}\ \isanewline
\ \ \isacommand{apply}\isamarkupfalse%
\ clarify\isanewline
\ \ \isacommand{unfolding}\isamarkupfalse%
\ InEclose{\isacharunderscore}{\kern0pt}def\ InEclose{\isacharunderscore}{\kern0pt}fm{\isacharunderscore}{\kern0pt}def\isanewline
\ \ \isacommand{apply}\isamarkupfalse%
\ simp\isanewline
\ \ \isacommand{using}\isamarkupfalse%
\ eclose{\isacharunderscore}{\kern0pt}closed\ \isanewline
\ \ \isacommand{by}\isamarkupfalse%
\ auto%
\endisatagproof
{\isafoldproof}%
%
\isadelimproof
\isanewline
%
\endisadelimproof
\isanewline
\isacommand{lemma}\isamarkupfalse%
\ eclose{\isacharunderscore}{\kern0pt}M\ {\isacharcolon}{\kern0pt}\ {\isachardoublequoteopen}eclose{\isacharparenleft}{\kern0pt}M{\isacharparenright}{\kern0pt}\ {\isacharequal}{\kern0pt}\ M{\isachardoublequoteclose}\ \isanewline
%
\isadelimproof
%
\endisadelimproof
%
\isatagproof
\isacommand{proof}\isamarkupfalse%
{\isacharparenleft}{\kern0pt}rule\ equality{\isacharunderscore}{\kern0pt}iffI{\isacharcomma}{\kern0pt}\ rule\ iffI{\isacharparenright}{\kern0pt}\isanewline
\ \ \isacommand{have}\isamarkupfalse%
\ H{\isacharcolon}{\kern0pt}\ {\isachardoublequoteopen}{\isasymAnd}n{\isachardot}{\kern0pt}\ n\ {\isasymin}\ nat\ {\isasymLongrightarrow}\ {\isasymforall}x{\isachardot}{\kern0pt}\ x\ {\isasymin}\ Union{\isacharcircum}{\kern0pt}n{\isacharparenleft}{\kern0pt}M{\isacharparenright}{\kern0pt}\ {\isasymlongrightarrow}\ x\ {\isasymin}\ M{\isachardoublequoteclose}\isanewline
\ \ \isacommand{proof}\isamarkupfalse%
{\isacharparenleft}{\kern0pt}rule{\isacharunderscore}{\kern0pt}tac\ P{\isacharequal}{\kern0pt}{\isachardoublequoteopen}{\isasymlambda}n{\isachardot}{\kern0pt}\ {\isasymforall}x{\isachardot}{\kern0pt}\ x\ {\isasymin}\ Union{\isacharcircum}{\kern0pt}n{\isacharparenleft}{\kern0pt}M{\isacharparenright}{\kern0pt}\ {\isasymlongrightarrow}\ x\ {\isasymin}\ M{\isachardoublequoteclose}\ \isakeyword{in}\ nat{\isacharunderscore}{\kern0pt}induct{\isacharcomma}{\kern0pt}\ assumption{\isacharcomma}{\kern0pt}\ force{\isacharcomma}{\kern0pt}\ rule\ allI{\isacharcomma}{\kern0pt}\ rule\ impI{\isacharparenright}{\kern0pt}\isanewline
\ \ \ \ \isacommand{fix}\isamarkupfalse%
\ n\ x\ \isacommand{assume}\isamarkupfalse%
\ assms{\isadigit{1}}{\isacharcolon}{\kern0pt}\ {\isachardoublequoteopen}n\ {\isasymin}\ nat{\isachardoublequoteclose}\ {\isachardoublequoteopen}{\isasymforall}x{\isachardot}{\kern0pt}\ x\ {\isasymin}\ Union{\isacharcircum}{\kern0pt}n\ {\isacharparenleft}{\kern0pt}M{\isacharparenright}{\kern0pt}\ {\isasymlongrightarrow}\ x\ {\isasymin}\ M{\isachardoublequoteclose}\ {\isachardoublequoteopen}x\ {\isasymin}\ Union{\isacharcircum}{\kern0pt}succ{\isacharparenleft}{\kern0pt}n{\isacharparenright}{\kern0pt}\ {\isacharparenleft}{\kern0pt}M{\isacharparenright}{\kern0pt}{\isachardoublequoteclose}\ \isanewline
\ \ \ \ \isacommand{then}\isamarkupfalse%
\ \isacommand{have}\isamarkupfalse%
\ {\isachardoublequoteopen}x\ {\isasymin}\ Union{\isacharparenleft}{\kern0pt}Union{\isacharcircum}{\kern0pt}n{\isacharparenleft}{\kern0pt}M{\isacharparenright}{\kern0pt}{\isacharparenright}{\kern0pt}{\isachardoublequoteclose}\ \isacommand{by}\isamarkupfalse%
\ auto\ \isanewline
\ \ \ \ \isacommand{then}\isamarkupfalse%
\ \isacommand{obtain}\isamarkupfalse%
\ y\ \isakeyword{where}\ yH{\isacharcolon}{\kern0pt}\ {\isachardoublequoteopen}x\ {\isasymin}\ y{\isachardoublequoteclose}\ {\isachardoublequoteopen}y\ {\isasymin}\ Union{\isacharcircum}{\kern0pt}n{\isacharparenleft}{\kern0pt}M{\isacharparenright}{\kern0pt}{\isachardoublequoteclose}\ \isacommand{by}\isamarkupfalse%
\ auto\ \isanewline
\ \ \ \ \isacommand{then}\isamarkupfalse%
\ \isacommand{have}\isamarkupfalse%
\ {\isachardoublequoteopen}y\ {\isasymin}\ M{\isachardoublequoteclose}\ \isacommand{using}\isamarkupfalse%
\ assms{\isadigit{1}}\ \isacommand{by}\isamarkupfalse%
\ auto\ \isanewline
\ \ \ \ \isacommand{then}\isamarkupfalse%
\ \isacommand{show}\isamarkupfalse%
\ {\isachardoublequoteopen}x\ {\isasymin}\ M{\isachardoublequoteclose}\ \isacommand{using}\isamarkupfalse%
\ yH\ transM\ \isacommand{by}\isamarkupfalse%
\ auto\ \isanewline
\ \ \isacommand{qed}\isamarkupfalse%
\isanewline
\ \ \isacommand{fix}\isamarkupfalse%
\ x\ \isacommand{assume}\isamarkupfalse%
\ {\isachardoublequoteopen}x\ {\isasymin}\ eclose{\isacharparenleft}{\kern0pt}M{\isacharparenright}{\kern0pt}{\isachardoublequoteclose}\ \isanewline
\ \ \isacommand{then}\isamarkupfalse%
\ \isacommand{obtain}\isamarkupfalse%
\ n\ \isakeyword{where}\ {\isachardoublequoteopen}n\ {\isasymin}\ nat{\isachardoublequoteclose}\ {\isachardoublequoteopen}x\ {\isasymin}\ Union{\isacharcircum}{\kern0pt}n{\isacharparenleft}{\kern0pt}M{\isacharparenright}{\kern0pt}{\isachardoublequoteclose}\ \isacommand{using}\isamarkupfalse%
\ eclose{\isacharunderscore}{\kern0pt}eq{\isacharunderscore}{\kern0pt}Union\ \isacommand{by}\isamarkupfalse%
\ auto\ \isanewline
\ \ \isacommand{then}\isamarkupfalse%
\ \isacommand{show}\isamarkupfalse%
\ {\isachardoublequoteopen}x\ {\isasymin}\ M{\isachardoublequoteclose}\ \isacommand{using}\isamarkupfalse%
\ H\ \isacommand{by}\isamarkupfalse%
\ auto\ \isanewline
\isacommand{next}\isamarkupfalse%
\ \isanewline
\ \ \isacommand{fix}\isamarkupfalse%
\ x\ \isacommand{assume}\isamarkupfalse%
\ {\isachardoublequoteopen}x\ {\isasymin}\ M{\isachardoublequoteclose}\ \isanewline
\ \ \isacommand{then}\isamarkupfalse%
\ \isacommand{show}\isamarkupfalse%
\ {\isachardoublequoteopen}x\ {\isasymin}\ eclose{\isacharparenleft}{\kern0pt}M{\isacharparenright}{\kern0pt}{\isachardoublequoteclose}\ \isanewline
\ \ \ \ \isacommand{apply}\isamarkupfalse%
{\isacharparenleft}{\kern0pt}subst\ eclose{\isacharunderscore}{\kern0pt}eq{\isacharunderscore}{\kern0pt}Union{\isacharcomma}{\kern0pt}\ simp{\isacharparenright}{\kern0pt}\isanewline
\ \ \ \ \isacommand{apply}\isamarkupfalse%
{\isacharparenleft}{\kern0pt}rule{\isacharunderscore}{\kern0pt}tac\ x{\isacharequal}{\kern0pt}{\isadigit{0}}\ \isakeyword{in}\ bexI{\isacharcomma}{\kern0pt}\ auto{\isacharparenright}{\kern0pt}\isanewline
\ \ \ \ \isacommand{done}\isamarkupfalse%
\isanewline
\isacommand{qed}\isamarkupfalse%
%
\endisatagproof
{\isafoldproof}%
%
\isadelimproof
\isanewline
%
\endisadelimproof
\isanewline
\isacommand{lemma}\isamarkupfalse%
\ Rrel{\isacharunderscore}{\kern0pt}InEclose\ {\isacharcolon}{\kern0pt}\ \isanewline
\ \ {\isachardoublequoteopen}Rrel{\isacharparenleft}{\kern0pt}InEclose{\isacharcomma}{\kern0pt}\ M{\isacharparenright}{\kern0pt}\ {\isacharequal}{\kern0pt}\ Memrel{\isacharparenleft}{\kern0pt}M{\isacharparenright}{\kern0pt}{\isacharcircum}{\kern0pt}{\isacharplus}{\kern0pt}{\isachardoublequoteclose}\ \isanewline
%
\isadelimproof
%
\endisadelimproof
%
\isatagproof
\isacommand{proof}\isamarkupfalse%
\ {\isacharparenleft}{\kern0pt}rule\ equality{\isacharunderscore}{\kern0pt}iffI{\isacharcomma}{\kern0pt}\ rule\ iffI{\isacharparenright}{\kern0pt}\isanewline
\ \ \isacommand{fix}\isamarkupfalse%
\ v\ \isacommand{assume}\isamarkupfalse%
\ {\isachardoublequoteopen}v\ {\isasymin}\ Rrel{\isacharparenleft}{\kern0pt}InEclose{\isacharcomma}{\kern0pt}\ M{\isacharparenright}{\kern0pt}{\isachardoublequoteclose}\ \isanewline
\ \ \isacommand{then}\isamarkupfalse%
\ \isacommand{obtain}\isamarkupfalse%
\ y\ x\ \isakeyword{where}\ yxH\ {\isacharcolon}{\kern0pt}\ {\isachardoublequoteopen}v\ {\isacharequal}{\kern0pt}\ {\isacharless}{\kern0pt}y{\isacharcomma}{\kern0pt}\ x{\isachargreater}{\kern0pt}{\isachardoublequoteclose}\ {\isachardoublequoteopen}y\ {\isasymin}\ eclose{\isacharparenleft}{\kern0pt}x{\isacharparenright}{\kern0pt}{\isachardoublequoteclose}\ {\isachardoublequoteopen}y\ {\isasymin}\ M{\isachardoublequoteclose}\ {\isachardoublequoteopen}x\ {\isasymin}\ M{\isachardoublequoteclose}\ \isacommand{unfolding}\isamarkupfalse%
\ InEclose{\isacharunderscore}{\kern0pt}def\ Rrel{\isacharunderscore}{\kern0pt}def\ \isacommand{by}\isamarkupfalse%
\ auto\isanewline
\ \ \isacommand{then}\isamarkupfalse%
\ \isacommand{have}\isamarkupfalse%
\ {\isachardoublequoteopen}y\ {\isasymin}\ {\isacharparenleft}{\kern0pt}{\isasymUnion}n{\isasymin}nat{\isachardot}{\kern0pt}\ Union{\isacharcircum}{\kern0pt}n\ {\isacharparenleft}{\kern0pt}x{\isacharparenright}{\kern0pt}{\isacharparenright}{\kern0pt}{\isachardoublequoteclose}\ \isacommand{using}\isamarkupfalse%
\ eclose{\isacharunderscore}{\kern0pt}eq{\isacharunderscore}{\kern0pt}Union\ \isacommand{by}\isamarkupfalse%
\ auto\isanewline
\ \ \isacommand{then}\isamarkupfalse%
\ \isacommand{obtain}\isamarkupfalse%
\ n\ \isakeyword{where}\ nH{\isacharcolon}{\kern0pt}\ {\isachardoublequoteopen}n\ {\isasymin}\ nat{\isachardoublequoteclose}\ {\isachardoublequoteopen}y\ {\isasymin}\ Union{\isacharcircum}{\kern0pt}n{\isacharparenleft}{\kern0pt}x{\isacharparenright}{\kern0pt}{\isachardoublequoteclose}\ \isacommand{by}\isamarkupfalse%
\ auto\isanewline
\isanewline
\ \ \isacommand{have}\isamarkupfalse%
\ {\isachardoublequoteopen}{\isasymAnd}n{\isachardot}{\kern0pt}\ n\ {\isasymin}\ nat\ {\isasymLongrightarrow}\ {\isasymforall}y{\isachardot}{\kern0pt}\ y\ {\isasymin}\ Union{\isacharcircum}{\kern0pt}n{\isacharparenleft}{\kern0pt}x{\isacharparenright}{\kern0pt}\ {\isasymlongrightarrow}\ {\isacharless}{\kern0pt}y{\isacharcomma}{\kern0pt}\ x{\isachargreater}{\kern0pt}\ {\isasymin}\ Memrel{\isacharparenleft}{\kern0pt}M{\isacharparenright}{\kern0pt}{\isacharcircum}{\kern0pt}{\isacharplus}{\kern0pt}{\isachardoublequoteclose}\ \isanewline
\ \ \isacommand{proof}\isamarkupfalse%
{\isacharparenleft}{\kern0pt}rule{\isacharunderscore}{\kern0pt}tac\ P{\isacharequal}{\kern0pt}{\isachardoublequoteopen}{\isasymlambda}n{\isachardot}{\kern0pt}\ {\isasymforall}y{\isachardot}{\kern0pt}\ y\ {\isasymin}\ Union{\isacharcircum}{\kern0pt}n{\isacharparenleft}{\kern0pt}x{\isacharparenright}{\kern0pt}\ {\isasymlongrightarrow}\ {\isacharless}{\kern0pt}y{\isacharcomma}{\kern0pt}\ x{\isachargreater}{\kern0pt}\ {\isasymin}\ Memrel{\isacharparenleft}{\kern0pt}M{\isacharparenright}{\kern0pt}{\isacharcircum}{\kern0pt}{\isacharplus}{\kern0pt}{\isachardoublequoteclose}\ \isakeyword{in}\ nat{\isacharunderscore}{\kern0pt}induct{\isacharcomma}{\kern0pt}\ assumption{\isacharparenright}{\kern0pt}\isanewline
\ \ \ \ \isacommand{fix}\isamarkupfalse%
\ n\ \isacommand{assume}\isamarkupfalse%
\ nnat\ {\isacharcolon}{\kern0pt}\ {\isachardoublequoteopen}n\ {\isasymin}\ nat{\isachardoublequoteclose}\ \isanewline
\ \ \ \ \isacommand{show}\isamarkupfalse%
\ {\isachardoublequoteopen}{\isasymforall}y{\isachardot}{\kern0pt}\ y\ {\isasymin}\ Union{\isacharcircum}{\kern0pt}{\isadigit{0}}\ {\isacharparenleft}{\kern0pt}x{\isacharparenright}{\kern0pt}\ {\isasymlongrightarrow}\ {\isacharless}{\kern0pt}y{\isacharcomma}{\kern0pt}\ x{\isachargreater}{\kern0pt}\ {\isasymin}\ Memrel{\isacharparenleft}{\kern0pt}M{\isacharparenright}{\kern0pt}{\isacharcircum}{\kern0pt}{\isacharplus}{\kern0pt}{\isachardoublequoteclose}\isanewline
\ \ \ \ \isacommand{proof}\isamarkupfalse%
{\isacharparenleft}{\kern0pt}rule\ allI{\isacharcomma}{\kern0pt}\ rule\ impI{\isacharparenright}{\kern0pt}\isanewline
\ \ \ \ \ \ \isacommand{fix}\isamarkupfalse%
\ z\ \isacommand{assume}\isamarkupfalse%
\ {\isachardoublequoteopen}z\ {\isasymin}\ Union{\isacharcircum}{\kern0pt}{\isadigit{0}}{\isacharparenleft}{\kern0pt}x{\isacharparenright}{\kern0pt}{\isachardoublequoteclose}\ \isanewline
\ \ \ \ \ \ \isacommand{then}\isamarkupfalse%
\ \isacommand{have}\isamarkupfalse%
\ rel\ {\isacharcolon}{\kern0pt}\ {\isachardoublequoteopen}z\ {\isasymin}\ x{\isachardoublequoteclose}\ \isacommand{by}\isamarkupfalse%
\ auto\ \isanewline
\ \ \ \ \ \ \isacommand{then}\isamarkupfalse%
\ \isacommand{have}\isamarkupfalse%
\ zin\ {\isacharcolon}{\kern0pt}\ {\isachardoublequoteopen}z\ {\isasymin}\ M{\isachardoublequoteclose}\ \isacommand{using}\isamarkupfalse%
\ yxH\ transM\ \isacommand{by}\isamarkupfalse%
\ auto\ \isanewline
\isanewline
\ \ \ \ \ \ \isacommand{show}\isamarkupfalse%
\ {\isachardoublequoteopen}{\isacharless}{\kern0pt}z{\isacharcomma}{\kern0pt}\ x{\isachargreater}{\kern0pt}\ {\isasymin}\ Memrel{\isacharparenleft}{\kern0pt}M{\isacharparenright}{\kern0pt}{\isacharcircum}{\kern0pt}{\isacharplus}{\kern0pt}{\isachardoublequoteclose}\ \isanewline
\ \ \ \ \ \ \ \ \isacommand{apply}\isamarkupfalse%
{\isacharparenleft}{\kern0pt}rule\ r{\isacharunderscore}{\kern0pt}into{\isacharunderscore}{\kern0pt}trancl{\isacharparenright}{\kern0pt}\isanewline
\ \ \ \ \ \ \ \ \isacommand{unfolding}\isamarkupfalse%
\ Memrel{\isacharunderscore}{\kern0pt}def\ \isanewline
\ \ \ \ \ \ \ \ \isacommand{using}\isamarkupfalse%
\ rel\ zin\ yxH\ \isanewline
\ \ \ \ \ \ \ \ \isacommand{by}\isamarkupfalse%
\ auto\isanewline
\ \ \ \ \isacommand{qed}\isamarkupfalse%
\isanewline
\ \ \isacommand{next}\isamarkupfalse%
\ \isanewline
\ \ \ \ \isacommand{fix}\isamarkupfalse%
\ n\ \isanewline
\ \ \ \ \isacommand{assume}\isamarkupfalse%
\ assms{\isadigit{1}}\ {\isacharcolon}{\kern0pt}\ {\isachardoublequoteopen}n\ {\isasymin}\ nat{\isachardoublequoteclose}\ {\isachardoublequoteopen}{\isasymforall}y{\isachardot}{\kern0pt}\ y\ {\isasymin}\ Union{\isacharcircum}{\kern0pt}n\ {\isacharparenleft}{\kern0pt}x{\isacharparenright}{\kern0pt}\ {\isasymlongrightarrow}\ {\isasymlangle}y{\isacharcomma}{\kern0pt}\ x{\isasymrangle}\ {\isasymin}\ Memrel{\isacharparenleft}{\kern0pt}M{\isacharparenright}{\kern0pt}{\isacharcircum}{\kern0pt}{\isacharplus}{\kern0pt}{\isachardoublequoteclose}\isanewline
\ \ \ \ \isacommand{show}\isamarkupfalse%
\ {\isachardoublequoteopen}{\isasymforall}y{\isachardot}{\kern0pt}\ y\ {\isasymin}\ Union{\isacharcircum}{\kern0pt}succ{\isacharparenleft}{\kern0pt}n{\isacharparenright}{\kern0pt}\ {\isacharparenleft}{\kern0pt}x{\isacharparenright}{\kern0pt}\ {\isasymlongrightarrow}\ {\isasymlangle}y{\isacharcomma}{\kern0pt}\ x{\isasymrangle}\ {\isasymin}\ Memrel{\isacharparenleft}{\kern0pt}M{\isacharparenright}{\kern0pt}{\isacharcircum}{\kern0pt}{\isacharplus}{\kern0pt}{\isachardoublequoteclose}\ \isanewline
\ \ \ \ \isacommand{proof}\isamarkupfalse%
{\isacharparenleft}{\kern0pt}rule\ allI{\isacharcomma}{\kern0pt}\ rule\ impI{\isacharparenright}{\kern0pt}\isanewline
\ \ \ \ \ \ \isacommand{fix}\isamarkupfalse%
\ z\ \isacommand{assume}\isamarkupfalse%
\ {\isachardoublequoteopen}z\ {\isasymin}\ Union{\isacharcircum}{\kern0pt}succ{\isacharparenleft}{\kern0pt}n{\isacharparenright}{\kern0pt}\ {\isacharparenleft}{\kern0pt}x{\isacharparenright}{\kern0pt}{\isachardoublequoteclose}\ \isanewline
\ \ \ \ \ \ \isacommand{then}\isamarkupfalse%
\ \isacommand{have}\isamarkupfalse%
\ {\isachardoublequoteopen}z\ {\isasymin}\ Union{\isacharparenleft}{\kern0pt}Union{\isacharcircum}{\kern0pt}n{\isacharparenleft}{\kern0pt}x{\isacharparenright}{\kern0pt}{\isacharparenright}{\kern0pt}{\isachardoublequoteclose}\ \isacommand{by}\isamarkupfalse%
\ auto\ \isanewline
\ \ \ \ \ \ \isacommand{then}\isamarkupfalse%
\ \isacommand{obtain}\isamarkupfalse%
\ w\ \isakeyword{where}\ wH\ {\isacharcolon}{\kern0pt}\ {\isachardoublequoteopen}z\ {\isasymin}\ w{\isachardoublequoteclose}\ {\isachardoublequoteopen}w\ {\isasymin}\ Union{\isacharcircum}{\kern0pt}n{\isacharparenleft}{\kern0pt}x{\isacharparenright}{\kern0pt}{\isachardoublequoteclose}\ \isacommand{by}\isamarkupfalse%
\ auto\ \isanewline
\ \ \ \ \ \ \isacommand{then}\isamarkupfalse%
\ \isacommand{have}\isamarkupfalse%
\ rel\ {\isacharcolon}{\kern0pt}\ {\isachardoublequoteopen}{\isacharless}{\kern0pt}w{\isacharcomma}{\kern0pt}\ x{\isachargreater}{\kern0pt}\ {\isasymin}\ Memrel{\isacharparenleft}{\kern0pt}M{\isacharparenright}{\kern0pt}{\isacharcircum}{\kern0pt}{\isacharplus}{\kern0pt}{\isachardoublequoteclose}\ \isacommand{using}\isamarkupfalse%
\ assms{\isadigit{1}}\ \isacommand{by}\isamarkupfalse%
\ auto\ \isanewline
\ \ \ \ \ \ \isacommand{then}\isamarkupfalse%
\ \isacommand{have}\isamarkupfalse%
\ {\isachardoublequoteopen}w\ {\isasymin}\ field{\isacharparenleft}{\kern0pt}Memrel{\isacharparenleft}{\kern0pt}M{\isacharparenright}{\kern0pt}{\isacharcircum}{\kern0pt}{\isacharplus}{\kern0pt}{\isacharparenright}{\kern0pt}{\isachardoublequoteclose}\ \isacommand{by}\isamarkupfalse%
\ auto\ \isanewline
\ \ \ \ \ \ \isacommand{then}\isamarkupfalse%
\ \isacommand{have}\isamarkupfalse%
\ {\isachardoublequoteopen}w\ {\isasymin}\ field{\isacharparenleft}{\kern0pt}Memrel{\isacharparenleft}{\kern0pt}M{\isacharparenright}{\kern0pt}{\isacharparenright}{\kern0pt}{\isachardoublequoteclose}\ \isacommand{using}\isamarkupfalse%
\ field{\isacharunderscore}{\kern0pt}trancl\ \isacommand{by}\isamarkupfalse%
\ auto\ \isanewline
\ \ \ \ \ \ \isacommand{then}\isamarkupfalse%
\ \isacommand{have}\isamarkupfalse%
\ win\ {\isacharcolon}{\kern0pt}\ {\isachardoublequoteopen}w\ {\isasymin}\ M{\isachardoublequoteclose}\ \isacommand{by}\isamarkupfalse%
\ auto\ \isanewline
\ \ \ \ \ \ \isacommand{then}\isamarkupfalse%
\ \isacommand{have}\isamarkupfalse%
\ zin\ {\isacharcolon}{\kern0pt}\ {\isachardoublequoteopen}z\ {\isasymin}\ M{\isachardoublequoteclose}\ \isacommand{using}\isamarkupfalse%
\ transM\ wH\ \isacommand{by}\isamarkupfalse%
\ auto\ \isanewline
\ \ \ \ \ \ \isacommand{show}\isamarkupfalse%
\ {\isachardoublequoteopen}{\isacharless}{\kern0pt}z{\isacharcomma}{\kern0pt}\ x{\isachargreater}{\kern0pt}\ {\isasymin}\ Memrel{\isacharparenleft}{\kern0pt}M{\isacharparenright}{\kern0pt}{\isacharcircum}{\kern0pt}{\isacharplus}{\kern0pt}{\isachardoublequoteclose}\ \isanewline
\ \ \ \ \ \ \ \ \isacommand{apply}\isamarkupfalse%
{\isacharparenleft}{\kern0pt}rule{\isacharunderscore}{\kern0pt}tac\ b{\isacharequal}{\kern0pt}w\ \isakeyword{in}\ rtrancl{\isacharunderscore}{\kern0pt}into{\isacharunderscore}{\kern0pt}trancl{\isadigit{2}}{\isacharparenright}{\kern0pt}\isanewline
\ \ \ \ \ \ \ \ \isacommand{using}\isamarkupfalse%
\ win\ zin\ wH\ \isanewline
\ \ \ \ \ \ \ \ \ \isacommand{apply}\isamarkupfalse%
\ force\isanewline
\ \ \ \ \ \ \ \ \isacommand{apply}\isamarkupfalse%
{\isacharparenleft}{\kern0pt}rule\ trancl{\isacharunderscore}{\kern0pt}into{\isacharunderscore}{\kern0pt}rtrancl{\isacharcomma}{\kern0pt}\ simp\ add{\isacharcolon}{\kern0pt}rel{\isacharparenright}{\kern0pt}\isanewline
\ \ \ \ \ \ \ \ \isacommand{done}\isamarkupfalse%
\isanewline
\ \ \ \ \isacommand{qed}\isamarkupfalse%
\isanewline
\ \ \isacommand{qed}\isamarkupfalse%
\isanewline
\isanewline
\ \ \isacommand{then}\isamarkupfalse%
\ \isacommand{show}\isamarkupfalse%
\ {\isachardoublequoteopen}v\ {\isasymin}\ Memrel{\isacharparenleft}{\kern0pt}M{\isacharparenright}{\kern0pt}{\isacharcircum}{\kern0pt}{\isacharplus}{\kern0pt}{\isachardoublequoteclose}\ \isacommand{using}\isamarkupfalse%
\ yxH\ nH\ \isacommand{by}\isamarkupfalse%
\ auto\isanewline
\isacommand{next}\isamarkupfalse%
\ \isanewline
\ \ \isacommand{fix}\isamarkupfalse%
\ v\ \isacommand{assume}\isamarkupfalse%
\ vin{\isacharcolon}{\kern0pt}\ {\isachardoublequoteopen}v\ {\isasymin}\ Memrel{\isacharparenleft}{\kern0pt}M{\isacharparenright}{\kern0pt}{\isacharcircum}{\kern0pt}{\isacharplus}{\kern0pt}{\isachardoublequoteclose}\ \isanewline
\ \ \isacommand{then}\isamarkupfalse%
\ \isacommand{obtain}\isamarkupfalse%
\ y\ x\ \isakeyword{where}\ yxH{\isacharcolon}{\kern0pt}\ {\isachardoublequoteopen}v\ {\isacharequal}{\kern0pt}\ {\isacharless}{\kern0pt}y{\isacharcomma}{\kern0pt}\ x{\isachargreater}{\kern0pt}{\isachardoublequoteclose}\ \isacommand{unfolding}\isamarkupfalse%
\ trancl{\isacharunderscore}{\kern0pt}def\ comp{\isacharunderscore}{\kern0pt}def\ \isacommand{by}\isamarkupfalse%
\ auto\ \isanewline
\ \ \isacommand{then}\isamarkupfalse%
\ \isacommand{have}\isamarkupfalse%
\ {\isachardoublequoteopen}{\isacharless}{\kern0pt}y{\isacharcomma}{\kern0pt}\ x{\isachargreater}{\kern0pt}\ {\isasymin}\ Rrel{\isacharparenleft}{\kern0pt}InEclose{\isacharcomma}{\kern0pt}\ M{\isacharparenright}{\kern0pt}{\isachardoublequoteclose}\ \isanewline
\ \ \isacommand{proof}\isamarkupfalse%
{\isacharparenleft}{\kern0pt}rule{\isacharunderscore}{\kern0pt}tac\ P{\isacharequal}{\kern0pt}{\isachardoublequoteopen}{\isasymlambda}x{\isachardot}{\kern0pt}\ {\isacharless}{\kern0pt}y{\isacharcomma}{\kern0pt}\ x{\isachargreater}{\kern0pt}\ {\isasymin}\ Rrel{\isacharparenleft}{\kern0pt}InEclose{\isacharcomma}{\kern0pt}\ M{\isacharparenright}{\kern0pt}{\isachardoublequoteclose}\ \isakeyword{and}\ r{\isacharequal}{\kern0pt}{\isachardoublequoteopen}Memrel{\isacharparenleft}{\kern0pt}M{\isacharparenright}{\kern0pt}{\isachardoublequoteclose}\ \isakeyword{and}\ a{\isacharequal}{\kern0pt}y\ \isakeyword{in}\ trancl{\isacharunderscore}{\kern0pt}induct{\isacharparenright}{\kern0pt}\isanewline
\ \ \ \ \isacommand{show}\isamarkupfalse%
\ {\isachardoublequoteopen}{\isacharless}{\kern0pt}y{\isacharcomma}{\kern0pt}\ x{\isachargreater}{\kern0pt}\ {\isasymin}\ Memrel{\isacharparenleft}{\kern0pt}M{\isacharparenright}{\kern0pt}{\isacharcircum}{\kern0pt}{\isacharplus}{\kern0pt}{\isachardoublequoteclose}\ \isacommand{using}\isamarkupfalse%
\ yxH\ vin\ \isacommand{by}\isamarkupfalse%
\ auto\ \isanewline
\ \ \isacommand{next}\isamarkupfalse%
\ \isanewline
\ \ \ \ \isacommand{fix}\isamarkupfalse%
\ z\ \isacommand{assume}\isamarkupfalse%
\ {\isachardoublequoteopen}{\isacharless}{\kern0pt}y{\isacharcomma}{\kern0pt}\ z{\isachargreater}{\kern0pt}\ {\isasymin}\ Memrel{\isacharparenleft}{\kern0pt}M{\isacharparenright}{\kern0pt}{\isachardoublequoteclose}\isanewline
\ \ \ \ \isacommand{then}\isamarkupfalse%
\ \isacommand{have}\isamarkupfalse%
\ {\isachardoublequoteopen}y\ {\isasymin}\ M{\isachardoublequoteclose}\ {\isachardoublequoteopen}z\ {\isasymin}\ M{\isachardoublequoteclose}\ {\isachardoublequoteopen}y\ {\isasymin}\ z{\isachardoublequoteclose}\ \isacommand{by}\isamarkupfalse%
\ auto\isanewline
\ \ \ \ \isacommand{then}\isamarkupfalse%
\ \isacommand{show}\isamarkupfalse%
\ {\isachardoublequoteopen}{\isacharless}{\kern0pt}y{\isacharcomma}{\kern0pt}\ z{\isachargreater}{\kern0pt}\ {\isasymin}\ Rrel{\isacharparenleft}{\kern0pt}InEclose{\isacharcomma}{\kern0pt}\ M{\isacharparenright}{\kern0pt}{\isachardoublequoteclose}\ \isanewline
\ \ \ \ \ \ \isacommand{unfolding}\isamarkupfalse%
\ Rrel{\isacharunderscore}{\kern0pt}def\ InEclose{\isacharunderscore}{\kern0pt}def\ \isanewline
\ \ \ \ \ \ \isacommand{apply}\isamarkupfalse%
\ simp\isanewline
\ \ \ \ \ \ \isacommand{apply}\isamarkupfalse%
{\isacharparenleft}{\kern0pt}subst\ eclose{\isacharunderscore}{\kern0pt}eq{\isacharunderscore}{\kern0pt}Union{\isacharcomma}{\kern0pt}\ simp{\isacharcomma}{\kern0pt}\ rule{\isacharunderscore}{\kern0pt}tac\ x{\isacharequal}{\kern0pt}{\isadigit{0}}\ \isakeyword{in}\ bexI{\isacharparenright}{\kern0pt}\isanewline
\ \ \ \ \ \ \isacommand{by}\isamarkupfalse%
\ auto\ \isanewline
\ \ \isacommand{next}\isamarkupfalse%
\ \isanewline
\isanewline
\ \ \ \ \isacommand{have}\isamarkupfalse%
\ H\ {\isacharcolon}{\kern0pt}\ {\isachardoublequoteopen}{\isasymAnd}A\ B\ n{\isachardot}{\kern0pt}\ n\ {\isasymin}\ nat\ {\isasymLongrightarrow}\ A\ {\isasymsubseteq}\ B\ {\isasymLongrightarrow}\ Union{\isacharcircum}{\kern0pt}n{\isacharparenleft}{\kern0pt}A{\isacharparenright}{\kern0pt}\ {\isasymsubseteq}\ Union{\isacharcircum}{\kern0pt}n{\isacharparenleft}{\kern0pt}B{\isacharparenright}{\kern0pt}{\isachardoublequoteclose}\ \isanewline
\ \ \ \ \ \ \isacommand{apply}\isamarkupfalse%
{\isacharparenleft}{\kern0pt}rename{\isacharunderscore}{\kern0pt}tac\ A\ B\ n{\isacharcomma}{\kern0pt}\ rule{\isacharunderscore}{\kern0pt}tac\ P{\isacharequal}{\kern0pt}{\isachardoublequoteopen}{\isasymlambda}n{\isachardot}{\kern0pt}\ Union{\isacharcircum}{\kern0pt}n{\isacharparenleft}{\kern0pt}A{\isacharparenright}{\kern0pt}\ {\isasymsubseteq}\ Union{\isacharcircum}{\kern0pt}n{\isacharparenleft}{\kern0pt}B{\isacharparenright}{\kern0pt}{\isachardoublequoteclose}\ \isakeyword{in}\ nat{\isacharunderscore}{\kern0pt}induct{\isacharparenright}{\kern0pt}\isanewline
\ \ \ \ \ \ \ \ \isacommand{apply}\isamarkupfalse%
\ auto{\isacharbrackleft}{\kern0pt}{\isadigit{2}}{\isacharbrackright}{\kern0pt}\isanewline
\ \ \ \ \ \ \isacommand{apply}\isamarkupfalse%
\ simp\ \isanewline
\ \ \ \ \ \ \isacommand{apply}\isamarkupfalse%
{\isacharparenleft}{\kern0pt}rule\ Union{\isacharunderscore}{\kern0pt}mono{\isacharparenright}{\kern0pt}\isanewline
\ \ \ \ \ \ \isacommand{by}\isamarkupfalse%
\ auto\isanewline
\isanewline
\ \ \ \ \isacommand{fix}\isamarkupfalse%
\ z\ w\ \isanewline
\ \ \ \ \isacommand{assume}\isamarkupfalse%
\ assms{\isadigit{1}}{\isacharcolon}{\kern0pt}\ {\isachardoublequoteopen}{\isacharless}{\kern0pt}y{\isacharcomma}{\kern0pt}\ z{\isachargreater}{\kern0pt}\ {\isasymin}\ Memrel{\isacharparenleft}{\kern0pt}M{\isacharparenright}{\kern0pt}{\isacharcircum}{\kern0pt}{\isacharplus}{\kern0pt}{\isachardoublequoteclose}\ {\isachardoublequoteopen}{\isacharless}{\kern0pt}z{\isacharcomma}{\kern0pt}\ w{\isachargreater}{\kern0pt}\ {\isasymin}\ Memrel{\isacharparenleft}{\kern0pt}M{\isacharparenright}{\kern0pt}{\isachardoublequoteclose}\ {\isachardoublequoteopen}{\isacharless}{\kern0pt}y{\isacharcomma}{\kern0pt}\ z{\isachargreater}{\kern0pt}\ {\isasymin}\ Rrel{\isacharparenleft}{\kern0pt}InEclose{\isacharcomma}{\kern0pt}\ M{\isacharparenright}{\kern0pt}{\isachardoublequoteclose}\ \isanewline
\isanewline
\ \ \ \ \isacommand{have}\isamarkupfalse%
\ yinM\ {\isacharcolon}{\kern0pt}\ {\isachardoublequoteopen}y\ {\isasymin}\ M{\isachardoublequoteclose}\ \isacommand{using}\isamarkupfalse%
\ assms{\isadigit{1}}\ Rrel{\isacharunderscore}{\kern0pt}def\ \isacommand{by}\isamarkupfalse%
\ auto\ \isanewline
\ \ \ \ \isacommand{have}\isamarkupfalse%
\ winM\ {\isacharcolon}{\kern0pt}\ {\isachardoublequoteopen}w\ {\isasymin}\ M{\isachardoublequoteclose}\ \isacommand{using}\isamarkupfalse%
\ assms{\isadigit{1}}\ \isacommand{by}\isamarkupfalse%
\ auto\isanewline
\ \ \ \ \isacommand{have}\isamarkupfalse%
\ zinw\ {\isacharcolon}{\kern0pt}\ {\isachardoublequoteopen}z\ {\isasymin}\ w{\isachardoublequoteclose}\ \isacommand{using}\isamarkupfalse%
\ assms{\isadigit{1}}\ \isacommand{by}\isamarkupfalse%
\ auto\isanewline
\isanewline
\ \ \ \ \isacommand{have}\isamarkupfalse%
\ {\isachardoublequoteopen}y\ {\isasymin}\ eclose{\isacharparenleft}{\kern0pt}z{\isacharparenright}{\kern0pt}{\isachardoublequoteclose}\ \isacommand{using}\isamarkupfalse%
\ assms{\isadigit{1}}\ Rrel{\isacharunderscore}{\kern0pt}def\ InEclose{\isacharunderscore}{\kern0pt}def\ \isacommand{by}\isamarkupfalse%
\ auto\ \isanewline
\ \ \ \ \isacommand{then}\isamarkupfalse%
\ \isacommand{obtain}\isamarkupfalse%
\ n\ \isakeyword{where}\ nH{\isacharcolon}{\kern0pt}\ {\isachardoublequoteopen}n\ {\isasymin}\ nat{\isachardoublequoteclose}\ {\isachardoublequoteopen}y\ {\isasymin}\ Union{\isacharcircum}{\kern0pt}n{\isacharparenleft}{\kern0pt}z{\isacharparenright}{\kern0pt}{\isachardoublequoteclose}\ \isacommand{using}\isamarkupfalse%
\ eclose{\isacharunderscore}{\kern0pt}eq{\isacharunderscore}{\kern0pt}Union\ \isacommand{by}\isamarkupfalse%
\ auto\ \isanewline
\ \ \ \ \isacommand{then}\isamarkupfalse%
\ \isacommand{have}\isamarkupfalse%
\ {\isachardoublequoteopen}y\ {\isasymin}\ Union{\isacharcircum}{\kern0pt}n{\isacharparenleft}{\kern0pt}Union{\isacharparenleft}{\kern0pt}w{\isacharparenright}{\kern0pt}{\isacharparenright}{\kern0pt}{\isachardoublequoteclose}\ \isanewline
\ \ \ \ \ \ \isacommand{apply}\isamarkupfalse%
{\isacharparenleft}{\kern0pt}rule{\isacharunderscore}{\kern0pt}tac\ A{\isacharequal}{\kern0pt}{\isachardoublequoteopen}Union{\isacharcircum}{\kern0pt}n{\isacharparenleft}{\kern0pt}z{\isacharparenright}{\kern0pt}{\isachardoublequoteclose}\ \isakeyword{in}\ subsetD{\isacharparenright}{\kern0pt}\isanewline
\ \ \ \ \ \ \ \isacommand{apply}\isamarkupfalse%
{\isacharparenleft}{\kern0pt}rule\ H{\isacharcomma}{\kern0pt}\ simp{\isacharparenright}{\kern0pt}\isanewline
\ \ \ \ \ \ \isacommand{using}\isamarkupfalse%
\ zinw\ \isanewline
\ \ \ \ \ \ \isacommand{by}\isamarkupfalse%
\ auto\ \isanewline
\ \ \ \ \isacommand{then}\isamarkupfalse%
\ \isacommand{have}\isamarkupfalse%
\ {\isachardoublequoteopen}y\ {\isasymin}\ Union{\isacharparenleft}{\kern0pt}Union{\isacharcircum}{\kern0pt}n{\isacharparenleft}{\kern0pt}w{\isacharparenright}{\kern0pt}{\isacharparenright}{\kern0pt}{\isachardoublequoteclose}\ \isacommand{using}\isamarkupfalse%
\ iterates{\isacharunderscore}{\kern0pt}commute\ nH\ \isacommand{by}\isamarkupfalse%
\ auto\ \isanewline
\ \ \ \ \isacommand{then}\isamarkupfalse%
\ \isacommand{have}\isamarkupfalse%
\ {\isachardoublequoteopen}y\ {\isasymin}\ Union{\isacharcircum}{\kern0pt}{\isacharparenleft}{\kern0pt}succ{\isacharparenleft}{\kern0pt}n{\isacharparenright}{\kern0pt}{\isacharparenright}{\kern0pt}{\isacharparenleft}{\kern0pt}w{\isacharparenright}{\kern0pt}{\isachardoublequoteclose}\ \isacommand{by}\isamarkupfalse%
\ auto\ \isanewline
\ \ \ \ \isacommand{then}\isamarkupfalse%
\ \isacommand{have}\isamarkupfalse%
\ {\isachardoublequoteopen}y\ {\isasymin}\ eclose{\isacharparenleft}{\kern0pt}w{\isacharparenright}{\kern0pt}{\isachardoublequoteclose}\ \isanewline
\ \ \ \ \ \ \isacommand{apply}\isamarkupfalse%
{\isacharparenleft}{\kern0pt}subst\ eclose{\isacharunderscore}{\kern0pt}eq{\isacharunderscore}{\kern0pt}Union{\isacharcomma}{\kern0pt}\ simp{\isacharcomma}{\kern0pt}\ rule{\isacharunderscore}{\kern0pt}tac\ x{\isacharequal}{\kern0pt}{\isachardoublequoteopen}succ{\isacharparenleft}{\kern0pt}n{\isacharparenright}{\kern0pt}{\isachardoublequoteclose}\ \isakeyword{in}\ bexI{\isacharparenright}{\kern0pt}\isanewline
\ \ \ \ \ \ \isacommand{using}\isamarkupfalse%
\ nH\ \isanewline
\ \ \ \ \ \ \isacommand{by}\isamarkupfalse%
\ auto\isanewline
\ \ \ \ \isacommand{then}\isamarkupfalse%
\ \isacommand{show}\isamarkupfalse%
\ {\isachardoublequoteopen}{\isacharless}{\kern0pt}y{\isacharcomma}{\kern0pt}\ w{\isachargreater}{\kern0pt}\ {\isasymin}\ Rrel{\isacharparenleft}{\kern0pt}InEclose{\isacharcomma}{\kern0pt}\ M{\isacharparenright}{\kern0pt}{\isachardoublequoteclose}\ \isanewline
\ \ \ \ \ \ \isacommand{unfolding}\isamarkupfalse%
\ Rrel{\isacharunderscore}{\kern0pt}def\ InEclose{\isacharunderscore}{\kern0pt}def\isanewline
\ \ \ \ \ \ \isacommand{using}\isamarkupfalse%
\ yinM\ winM\ \isanewline
\ \ \ \ \ \ \isacommand{by}\isamarkupfalse%
\ auto\isanewline
\ \ \isacommand{qed}\isamarkupfalse%
\isanewline
\ \ \isacommand{then}\isamarkupfalse%
\ \isacommand{show}\isamarkupfalse%
\ {\isachardoublequoteopen}v\ {\isasymin}\ Rrel{\isacharparenleft}{\kern0pt}InEclose{\isacharcomma}{\kern0pt}M{\isacharparenright}{\kern0pt}{\isachardoublequoteclose}\ \isacommand{using}\isamarkupfalse%
\ yxH\ \isacommand{by}\isamarkupfalse%
\ auto\isanewline
\isacommand{qed}\isamarkupfalse%
%
\endisatagproof
{\isafoldproof}%
%
\isadelimproof
\isanewline
%
\endisadelimproof
\isanewline
\isacommand{lemma}\isamarkupfalse%
\ preds{\isacharunderscore}{\kern0pt}InEclose{\isacharunderscore}{\kern0pt}eq\ {\isacharcolon}{\kern0pt}\ {\isachardoublequoteopen}x\ {\isasymin}\ M\ {\isasymLongrightarrow}\ preds{\isacharparenleft}{\kern0pt}InEclose{\isacharcomma}{\kern0pt}\ x{\isacharparenright}{\kern0pt}\ {\isacharequal}{\kern0pt}\ eclose{\isacharparenleft}{\kern0pt}x{\isacharparenright}{\kern0pt}{\isachardoublequoteclose}\isanewline
%
\isadelimproof
\ \ %
\endisadelimproof
%
\isatagproof
\isacommand{unfolding}\isamarkupfalse%
\ preds{\isacharunderscore}{\kern0pt}def\ InEclose{\isacharunderscore}{\kern0pt}def\ \isanewline
\ \ \isacommand{apply}\isamarkupfalse%
{\isacharparenleft}{\kern0pt}rule\ equality{\isacharunderscore}{\kern0pt}iffI{\isacharcomma}{\kern0pt}\ rule\ iffI{\isacharcomma}{\kern0pt}\ simp{\isacharcomma}{\kern0pt}\ simp{\isacharparenright}{\kern0pt}\isanewline
\ \ \isacommand{apply}\isamarkupfalse%
{\isacharparenleft}{\kern0pt}subgoal{\isacharunderscore}{\kern0pt}tac\ {\isachardoublequoteopen}eclose{\isacharparenleft}{\kern0pt}x{\isacharparenright}{\kern0pt}\ {\isasymin}\ M{\isachardoublequoteclose}{\isacharparenright}{\kern0pt}\isanewline
\ \ \isacommand{using}\isamarkupfalse%
\ transM\ \isanewline
\ \ \ \isacommand{apply}\isamarkupfalse%
\ force\ \isanewline
\ \ \isacommand{using}\isamarkupfalse%
\ eclose{\isacharunderscore}{\kern0pt}closed\ \isanewline
\ \ \isacommand{by}\isamarkupfalse%
\ auto%
\endisatagproof
{\isafoldproof}%
%
\isadelimproof
\isanewline
%
\endisadelimproof
\isanewline
\isacommand{lemma}\isamarkupfalse%
\ field{\isacharunderscore}{\kern0pt}Rrel{\isacharunderscore}{\kern0pt}InEclose\ {\isacharcolon}{\kern0pt}\ {\isachardoublequoteopen}field{\isacharparenleft}{\kern0pt}Rrel{\isacharparenleft}{\kern0pt}InEclose{\isacharcomma}{\kern0pt}\ M{\isacharparenright}{\kern0pt}{\isacharparenright}{\kern0pt}\ {\isacharequal}{\kern0pt}\ M{\isachardoublequoteclose}\ \isanewline
%
\isadelimproof
\ \ %
\endisadelimproof
%
\isatagproof
\isacommand{apply}\isamarkupfalse%
{\isacharparenleft}{\kern0pt}rule\ equality{\isacharunderscore}{\kern0pt}iffI{\isacharcomma}{\kern0pt}\ rule\ iffI{\isacharparenright}{\kern0pt}\isanewline
\ \ \ \isacommand{apply}\isamarkupfalse%
{\isacharparenleft}{\kern0pt}simp\ add{\isacharcolon}{\kern0pt}field{\isacharunderscore}{\kern0pt}def\ Rrel{\isacharunderscore}{\kern0pt}def{\isacharcomma}{\kern0pt}\ force{\isacharparenright}{\kern0pt}\isanewline
\ \ \isacommand{apply}\isamarkupfalse%
{\isacharparenleft}{\kern0pt}simp\ add{\isacharcolon}{\kern0pt}field{\isacharunderscore}{\kern0pt}def{\isacharcomma}{\kern0pt}\ rule\ disjI{\isadigit{1}}{\isacharparenright}{\kern0pt}\isanewline
\ \ \isacommand{apply}\isamarkupfalse%
{\isacharparenleft}{\kern0pt}rename{\isacharunderscore}{\kern0pt}tac\ x{\isacharcomma}{\kern0pt}\ rule{\isacharunderscore}{\kern0pt}tac\ b{\isacharequal}{\kern0pt}{\isachardoublequoteopen}{\isacharbraceleft}{\kern0pt}x{\isacharbraceright}{\kern0pt}{\isachardoublequoteclose}\ \isakeyword{in}\ domainI{\isacharparenright}{\kern0pt}\isanewline
\ \ \isacommand{unfolding}\isamarkupfalse%
\ Rrel{\isacharunderscore}{\kern0pt}def\ InEclose{\isacharunderscore}{\kern0pt}def\ \isanewline
\ \ \isacommand{using}\isamarkupfalse%
\ singleton{\isacharunderscore}{\kern0pt}in{\isacharunderscore}{\kern0pt}M{\isacharunderscore}{\kern0pt}iff\isanewline
\ \ \isacommand{apply}\isamarkupfalse%
\ simp\isanewline
\ \ \isacommand{apply}\isamarkupfalse%
{\isacharparenleft}{\kern0pt}subst\ eclose{\isacharunderscore}{\kern0pt}eq{\isacharunderscore}{\kern0pt}Union{\isacharcomma}{\kern0pt}\ simp{\isacharparenright}{\kern0pt}\isanewline
\ \ \isacommand{apply}\isamarkupfalse%
{\isacharparenleft}{\kern0pt}rule{\isacharunderscore}{\kern0pt}tac\ x{\isacharequal}{\kern0pt}{\isadigit{0}}\ \isakeyword{in}\ bexI{\isacharparenright}{\kern0pt}\isanewline
\ \ \isacommand{by}\isamarkupfalse%
\ auto%
\endisatagproof
{\isafoldproof}%
%
\isadelimproof
\isanewline
%
\endisadelimproof
\isanewline
\isacommand{lemma}\isamarkupfalse%
\ Rrel{\isacharunderscore}{\kern0pt}InEclose{\isacharunderscore}{\kern0pt}vimage{\isacharunderscore}{\kern0pt}eq\ {\isacharcolon}{\kern0pt}\ {\isachardoublequoteopen}{\isasymAnd}y{\isachardot}{\kern0pt}\ y\ {\isasymin}\ M\ {\isasymLongrightarrow}\ Rrel{\isacharparenleft}{\kern0pt}InEclose{\isacharcomma}{\kern0pt}\ M{\isacharparenright}{\kern0pt}\ {\isacharminus}{\kern0pt}{\isacharbackquote}{\kern0pt}{\isacharbackquote}{\kern0pt}\ {\isacharbraceleft}{\kern0pt}y{\isacharbraceright}{\kern0pt}\ {\isacharequal}{\kern0pt}\ eclose{\isacharparenleft}{\kern0pt}y{\isacharparenright}{\kern0pt}{\isachardoublequoteclose}\isanewline
%
\isadelimproof
\ \ %
\endisadelimproof
%
\isatagproof
\isacommand{apply}\isamarkupfalse%
{\isacharparenleft}{\kern0pt}rule\ equality{\isacharunderscore}{\kern0pt}iffI{\isacharcomma}{\kern0pt}\ rule\ iffI{\isacharparenright}{\kern0pt}\isanewline
\ \ \ \isacommand{apply}\isamarkupfalse%
{\isacharparenleft}{\kern0pt}simp\ add{\isacharcolon}{\kern0pt}Rrel{\isacharunderscore}{\kern0pt}def\ InEclose{\isacharunderscore}{\kern0pt}def{\isacharcomma}{\kern0pt}\ force{\isacharparenright}{\kern0pt}\isanewline
\ \ \isacommand{apply}\isamarkupfalse%
{\isacharparenleft}{\kern0pt}rename{\isacharunderscore}{\kern0pt}tac\ y\ z{\isacharcomma}{\kern0pt}\ rule{\isacharunderscore}{\kern0pt}tac\ b{\isacharequal}{\kern0pt}y\ \isakeyword{in}\ vimageI{\isacharcomma}{\kern0pt}\ simp\ add{\isacharcolon}{\kern0pt}Rrel{\isacharunderscore}{\kern0pt}def\ InEclose{\isacharunderscore}{\kern0pt}def{\isacharparenright}{\kern0pt}\isanewline
\ \ \ \isacommand{apply}\isamarkupfalse%
{\isacharparenleft}{\kern0pt}rename{\isacharunderscore}{\kern0pt}tac\ y\ z{\isacharcomma}{\kern0pt}\ subgoal{\isacharunderscore}{\kern0pt}tac\ {\isachardoublequoteopen}eclose{\isacharparenleft}{\kern0pt}y{\isacharparenright}{\kern0pt}\ {\isasymin}\ M{\isachardoublequoteclose}{\isacharparenright}{\kern0pt}\isanewline
\ \ \isacommand{using}\isamarkupfalse%
\ transM\ eclose{\isacharunderscore}{\kern0pt}closed\isanewline
\ \ \isacommand{by}\isamarkupfalse%
\ auto%
\endisatagproof
{\isafoldproof}%
%
\isadelimproof
\isanewline
%
\endisadelimproof
\isanewline
\isacommand{end}\isamarkupfalse%
\isanewline
\isanewline
\isacommand{definition}\isamarkupfalse%
\ is{\isacharunderscore}{\kern0pt}memrel{\isacharunderscore}{\kern0pt}wftrec{\isacharunderscore}{\kern0pt}fm\ \isakeyword{where}\ {\isachardoublequoteopen}is{\isacharunderscore}{\kern0pt}memrel{\isacharunderscore}{\kern0pt}wftrec{\isacharunderscore}{\kern0pt}fm{\isacharparenleft}{\kern0pt}Gfm{\isacharcomma}{\kern0pt}\ i{\isacharcomma}{\kern0pt}\ j{\isacharcomma}{\kern0pt}\ k{\isacharparenright}{\kern0pt}\ {\isasymequiv}\ is{\isacharunderscore}{\kern0pt}wftrec{\isacharunderscore}{\kern0pt}fm{\isacharparenleft}{\kern0pt}Gfm{\isacharcomma}{\kern0pt}\ InEclose{\isacharunderscore}{\kern0pt}fm{\isacharcomma}{\kern0pt}\ i{\isacharcomma}{\kern0pt}\ j{\isacharcomma}{\kern0pt}\ k{\isacharparenright}{\kern0pt}{\isachardoublequoteclose}\ \isanewline
\isanewline
\isacommand{context}\isamarkupfalse%
\ M{\isacharunderscore}{\kern0pt}ctm\ \isanewline
\isakeyword{begin}\ \isanewline
\isanewline
\isacommand{lemma}\isamarkupfalse%
\ is{\isacharunderscore}{\kern0pt}memrel{\isacharunderscore}{\kern0pt}wftrec{\isacharunderscore}{\kern0pt}fm{\isacharunderscore}{\kern0pt}type\ {\isacharcolon}{\kern0pt}\ \isanewline
\ \ \isakeyword{fixes}\ Gfm\ i\ j\ k\ \isanewline
\ \ \isakeyword{assumes}\ {\isachardoublequoteopen}Gfm\ {\isasymin}\ formula{\isachardoublequoteclose}\ {\isachardoublequoteopen}i\ {\isasymin}\ nat{\isachardoublequoteclose}\ {\isachardoublequoteopen}j\ {\isasymin}\ nat{\isachardoublequoteclose}\ {\isachardoublequoteopen}k\ {\isasymin}\ nat{\isachardoublequoteclose}\ \isanewline
\ \ \isakeyword{shows}\ {\isachardoublequoteopen}is{\isacharunderscore}{\kern0pt}memrel{\isacharunderscore}{\kern0pt}wftrec{\isacharunderscore}{\kern0pt}fm{\isacharparenleft}{\kern0pt}Gfm{\isacharcomma}{\kern0pt}\ i{\isacharcomma}{\kern0pt}\ j{\isacharcomma}{\kern0pt}\ k{\isacharparenright}{\kern0pt}\ {\isasymin}\ formula{\isachardoublequoteclose}\ \isanewline
%
\isadelimproof
\ \isanewline
\ \ %
\endisadelimproof
%
\isatagproof
\isacommand{unfolding}\isamarkupfalse%
\ is{\isacharunderscore}{\kern0pt}memrel{\isacharunderscore}{\kern0pt}wftrec{\isacharunderscore}{\kern0pt}fm{\isacharunderscore}{\kern0pt}def\ InEclose{\isacharunderscore}{\kern0pt}fm{\isacharunderscore}{\kern0pt}def\ \isanewline
\ \ \isacommand{apply}\isamarkupfalse%
{\isacharparenleft}{\kern0pt}rule\ is{\isacharunderscore}{\kern0pt}wftrec{\isacharunderscore}{\kern0pt}fm{\isacharunderscore}{\kern0pt}type{\isacharparenright}{\kern0pt}\isanewline
\ \ \isacommand{using}\isamarkupfalse%
\ assms\isanewline
\ \ \isacommand{by}\isamarkupfalse%
\ auto%
\endisatagproof
{\isafoldproof}%
%
\isadelimproof
\isanewline
%
\endisadelimproof
\isanewline
\isacommand{lemma}\isamarkupfalse%
\ arity{\isacharunderscore}{\kern0pt}is{\isacharunderscore}{\kern0pt}memrel{\isacharunderscore}{\kern0pt}wftrec{\isacharunderscore}{\kern0pt}fm\ {\isacharcolon}{\kern0pt}\ \isanewline
\ \ \isakeyword{fixes}\ Gfm\ i\ j\ k\ \isanewline
\ \ \isakeyword{assumes}\ {\isachardoublequoteopen}Gfm\ {\isasymin}\ formula{\isachardoublequoteclose}\ {\isachardoublequoteopen}arity{\isacharparenleft}{\kern0pt}Gfm{\isacharparenright}{\kern0pt}\ {\isasymle}\ {\isadigit{3}}{\isachardoublequoteclose}\ {\isachardoublequoteopen}i\ {\isasymin}\ nat{\isachardoublequoteclose}\ {\isachardoublequoteopen}j\ {\isasymin}\ nat{\isachardoublequoteclose}\ {\isachardoublequoteopen}k\ {\isasymin}\ nat{\isachardoublequoteclose}\ \isanewline
\ \ \isakeyword{shows}\ {\isachardoublequoteopen}arity{\isacharparenleft}{\kern0pt}is{\isacharunderscore}{\kern0pt}memrel{\isacharunderscore}{\kern0pt}wftrec{\isacharunderscore}{\kern0pt}fm{\isacharparenleft}{\kern0pt}Gfm{\isacharcomma}{\kern0pt}\ i{\isacharcomma}{\kern0pt}\ j{\isacharcomma}{\kern0pt}\ k{\isacharparenright}{\kern0pt}{\isacharparenright}{\kern0pt}\ {\isasymle}\ succ{\isacharparenleft}{\kern0pt}i{\isacharparenright}{\kern0pt}\ {\isasymunion}\ succ{\isacharparenleft}{\kern0pt}j{\isacharparenright}{\kern0pt}\ {\isasymunion}\ succ{\isacharparenleft}{\kern0pt}k{\isacharparenright}{\kern0pt}{\isachardoublequoteclose}\isanewline
%
\isadelimproof
\isanewline
\ \ %
\endisadelimproof
%
\isatagproof
\isacommand{unfolding}\isamarkupfalse%
\ is{\isacharunderscore}{\kern0pt}memrel{\isacharunderscore}{\kern0pt}wftrec{\isacharunderscore}{\kern0pt}fm{\isacharunderscore}{\kern0pt}def\isanewline
\ \ \isacommand{apply}\isamarkupfalse%
{\isacharparenleft}{\kern0pt}rule\ le{\isacharunderscore}{\kern0pt}trans{\isacharcomma}{\kern0pt}\ rule\ arity{\isacharunderscore}{\kern0pt}is{\isacharunderscore}{\kern0pt}wftrec{\isacharunderscore}{\kern0pt}fm{\isacharparenright}{\kern0pt}\isanewline
\ \ \isacommand{unfolding}\isamarkupfalse%
\ InEclose{\isacharunderscore}{\kern0pt}fm{\isacharunderscore}{\kern0pt}def\ \isanewline
\ \ \isacommand{using}\isamarkupfalse%
\ assms\isanewline
\ \ \ \ \ \ \ \ \ \isacommand{apply}\isamarkupfalse%
\ auto{\isacharbrackleft}{\kern0pt}{\isadigit{3}}{\isacharbrackright}{\kern0pt}\isanewline
\ \ \ \ \ \ \isacommand{apply}\isamarkupfalse%
\ simp\isanewline
\ \ \ \ \ \ \isacommand{apply}\isamarkupfalse%
{\isacharparenleft}{\kern0pt}subst\ arity{\isacharunderscore}{\kern0pt}is{\isacharunderscore}{\kern0pt}eclose{\isacharunderscore}{\kern0pt}fm{\isacharcomma}{\kern0pt}\ simp{\isacharcomma}{\kern0pt}\ simp{\isacharparenright}{\kern0pt}\isanewline
\ \ \ \ \ \ \isacommand{apply}\isamarkupfalse%
\ {\isacharparenleft}{\kern0pt}simp\ del{\isacharcolon}{\kern0pt}FOL{\isacharunderscore}{\kern0pt}sats{\isacharunderscore}{\kern0pt}iff\ pair{\isacharunderscore}{\kern0pt}abs\ add{\isacharcolon}{\kern0pt}\ fm{\isacharunderscore}{\kern0pt}defs\ nat{\isacharunderscore}{\kern0pt}simp{\isacharunderscore}{\kern0pt}union{\isacharparenright}{\kern0pt}\ \isanewline
\ \ \isacommand{using}\isamarkupfalse%
\ assms\ le{\isacharunderscore}{\kern0pt}refl\ \isanewline
\ \ \isacommand{by}\isamarkupfalse%
\ auto%
\endisatagproof
{\isafoldproof}%
%
\isadelimproof
\ \ \isanewline
%
\endisadelimproof
\isanewline
\isacommand{lemma}\isamarkupfalse%
\ sats{\isacharunderscore}{\kern0pt}is{\isacharunderscore}{\kern0pt}memrel{\isacharunderscore}{\kern0pt}wftrec{\isacharunderscore}{\kern0pt}fm{\isacharunderscore}{\kern0pt}iff\ {\isacharcolon}{\kern0pt}\isanewline
\ \ \isakeyword{fixes}\ H\ G\ Gfm\ x\ v\ i\ j\ k\ env\ a\isanewline
\ \ \isakeyword{assumes}\ {\isachardoublequoteopen}env\ {\isasymin}\ list{\isacharparenleft}{\kern0pt}M{\isacharparenright}{\kern0pt}{\isachardoublequoteclose}\ {\isachardoublequoteopen}a\ {\isasymin}\ M{\isachardoublequoteclose}\ {\isachardoublequoteopen}x\ {\isasymin}\ M{\isachardoublequoteclose}\ \isanewline
\ \ \ \ \ \ \ \ \ \ {\isachardoublequoteopen}i\ {\isacharless}{\kern0pt}\ length{\isacharparenleft}{\kern0pt}env{\isacharparenright}{\kern0pt}{\isachardoublequoteclose}\ {\isachardoublequoteopen}j\ {\isacharless}{\kern0pt}\ length{\isacharparenleft}{\kern0pt}env{\isacharparenright}{\kern0pt}{\isachardoublequoteclose}\ {\isachardoublequoteopen}k\ {\isacharless}{\kern0pt}\ length{\isacharparenleft}{\kern0pt}env{\isacharparenright}{\kern0pt}{\isachardoublequoteclose}\ \isanewline
\ \ \ \ \ \ \ \ \ \ {\isachardoublequoteopen}nth{\isacharparenleft}{\kern0pt}i{\isacharcomma}{\kern0pt}\ env{\isacharparenright}{\kern0pt}\ {\isacharequal}{\kern0pt}\ x{\isachardoublequoteclose}\ {\isachardoublequoteopen}nth{\isacharparenleft}{\kern0pt}j{\isacharcomma}{\kern0pt}\ env{\isacharparenright}{\kern0pt}\ {\isacharequal}{\kern0pt}\ a{\isachardoublequoteclose}\ {\isachardoublequoteopen}nth{\isacharparenleft}{\kern0pt}k{\isacharcomma}{\kern0pt}\ env{\isacharparenright}{\kern0pt}\ {\isacharequal}{\kern0pt}\ v{\isachardoublequoteclose}\isanewline
\ \ \ \ \ \ \ \ \ \ {\isachardoublequoteopen}Gfm\ {\isasymin}\ formula{\isachardoublequoteclose}\ \isanewline
\ \ \ \ \ \ \ \ \ \ {\isachardoublequoteopen}arity{\isacharparenleft}{\kern0pt}Gfm{\isacharparenright}{\kern0pt}\ {\isasymle}\ {\isadigit{3}}{\isachardoublequoteclose}\ \isanewline
\ \ \ \ \ \ \ \ \ \ {\isachardoublequoteopen}{\isasymAnd}x\ g{\isachardot}{\kern0pt}\ x\ {\isasymin}\ M\ {\isasymLongrightarrow}\ g\ {\isasymin}\ M\ {\isasymLongrightarrow}\ function{\isacharparenleft}{\kern0pt}g{\isacharparenright}{\kern0pt}\ {\isasymLongrightarrow}\ G{\isacharparenleft}{\kern0pt}x{\isacharcomma}{\kern0pt}\ g{\isacharparenright}{\kern0pt}\ {\isasymin}\ M{\isachardoublequoteclose}\ \isanewline
\ \ \isakeyword{and}\ HGeq{\isacharcolon}{\kern0pt}\ {\isachardoublequoteopen}{\isasymAnd}h\ g\ x{\isachardot}{\kern0pt}\ h\ {\isasymin}\ eclose{\isacharparenleft}{\kern0pt}x{\isacharparenright}{\kern0pt}\ {\isasymrightarrow}\ M\ {\isasymLongrightarrow}\ g\ {\isasymin}\ {\isacharparenleft}{\kern0pt}eclose{\isacharparenleft}{\kern0pt}x{\isacharparenright}{\kern0pt}\ {\isasymtimes}\ {\isacharbraceleft}{\kern0pt}a{\isacharbraceright}{\kern0pt}{\isacharparenright}{\kern0pt}\ {\isasymrightarrow}\ M\ {\isasymLongrightarrow}\ g\ {\isasymin}\ M\ \ \isanewline
\ \ \ \ \ \ \ \ \ \ \ \ \ \ \ {\isasymLongrightarrow}\ x\ {\isasymin}\ M\ {\isasymLongrightarrow}\ {\isacharparenleft}{\kern0pt}{\isasymAnd}y{\isachardot}{\kern0pt}\ y\ {\isasymin}\ eclose{\isacharparenleft}{\kern0pt}x{\isacharparenright}{\kern0pt}\ {\isasymLongrightarrow}\ h{\isacharbackquote}{\kern0pt}y\ {\isacharequal}{\kern0pt}\ g{\isacharbackquote}{\kern0pt}{\isacharless}{\kern0pt}y{\isacharcomma}{\kern0pt}\ a{\isachargreater}{\kern0pt}{\isacharparenright}{\kern0pt}\ {\isasymLongrightarrow}\ H{\isacharparenleft}{\kern0pt}x{\isacharcomma}{\kern0pt}\ h{\isacharparenright}{\kern0pt}\ {\isacharequal}{\kern0pt}\ G{\isacharparenleft}{\kern0pt}{\isacharless}{\kern0pt}x{\isacharcomma}{\kern0pt}\ a{\isachargreater}{\kern0pt}{\isacharcomma}{\kern0pt}\ g{\isacharparenright}{\kern0pt}{\isachardoublequoteclose}\ \ \isanewline
\ \ \isakeyword{and}\ sats{\isacharunderscore}{\kern0pt}Gfm{\isacharunderscore}{\kern0pt}iff\ {\isacharcolon}{\kern0pt}\ {\isachardoublequoteopen}\ {\isacharparenleft}{\kern0pt}{\isasymAnd}a{\isadigit{0}}\ a{\isadigit{1}}\ a{\isadigit{2}}\ env{\isachardot}{\kern0pt}\ a{\isadigit{0}}\ {\isasymin}\ M\ {\isasymLongrightarrow}\ a{\isadigit{1}}\ {\isasymin}\ M\ {\isasymLongrightarrow}\ a{\isadigit{2}}\ {\isasymin}\ M\ {\isasymLongrightarrow}\ env\ {\isasymin}\ list{\isacharparenleft}{\kern0pt}M{\isacharparenright}{\kern0pt}\ {\isasymLongrightarrow}\ a{\isadigit{0}}\ {\isacharequal}{\kern0pt}\ G{\isacharparenleft}{\kern0pt}a{\isadigit{2}}{\isacharcomma}{\kern0pt}\ a{\isadigit{1}}{\isacharparenright}{\kern0pt}\ {\isasymlongleftrightarrow}\ sats{\isacharparenleft}{\kern0pt}M{\isacharcomma}{\kern0pt}\ Gfm{\isacharcomma}{\kern0pt}\ {\isacharbrackleft}{\kern0pt}a{\isadigit{0}}{\isacharcomma}{\kern0pt}\ a{\isadigit{1}}{\isacharcomma}{\kern0pt}\ a{\isadigit{2}}{\isacharbrackright}{\kern0pt}\ {\isacharat}{\kern0pt}\ env{\isacharparenright}{\kern0pt}{\isacharparenright}{\kern0pt}{\isachardoublequoteclose}\ \ \isanewline
\ \ \ \ \ \ \ \ \ \ \isanewline
\ \ \isakeyword{shows}\ {\isachardoublequoteopen}sats{\isacharparenleft}{\kern0pt}M{\isacharcomma}{\kern0pt}\ is{\isacharunderscore}{\kern0pt}memrel{\isacharunderscore}{\kern0pt}wftrec{\isacharunderscore}{\kern0pt}fm{\isacharparenleft}{\kern0pt}Gfm{\isacharcomma}{\kern0pt}\ i{\isacharcomma}{\kern0pt}\ j{\isacharcomma}{\kern0pt}\ k{\isacharparenright}{\kern0pt}{\isacharcomma}{\kern0pt}\ env{\isacharparenright}{\kern0pt}\ {\isasymlongleftrightarrow}\ v\ {\isacharequal}{\kern0pt}\ wftrec{\isacharparenleft}{\kern0pt}Memrel{\isacharparenleft}{\kern0pt}M{\isacharparenright}{\kern0pt}{\isacharcircum}{\kern0pt}{\isacharplus}{\kern0pt}{\isacharcomma}{\kern0pt}\ x{\isacharcomma}{\kern0pt}\ H{\isacharparenright}{\kern0pt}{\isachardoublequoteclose}\ \isanewline
%
\isadelimproof
%
\endisadelimproof
%
\isatagproof
\isacommand{proof}\isamarkupfalse%
\ {\isacharminus}{\kern0pt}\ \isanewline
\isanewline
\ \ \isacommand{have}\isamarkupfalse%
\ prelvimageeq\ {\isacharcolon}{\kern0pt}\ {\isachardoublequoteopen}{\isasymAnd}y{\isachardot}{\kern0pt}\ y\ {\isasymin}\ M\ {\isasymLongrightarrow}\ prel{\isacharparenleft}{\kern0pt}Rrel{\isacharparenleft}{\kern0pt}InEclose{\isacharcomma}{\kern0pt}\ M{\isacharparenright}{\kern0pt}{\isacharcomma}{\kern0pt}\ {\isacharbraceleft}{\kern0pt}a{\isacharbraceright}{\kern0pt}{\isacharparenright}{\kern0pt}\ {\isacharminus}{\kern0pt}{\isacharbackquote}{\kern0pt}{\isacharbackquote}{\kern0pt}\ {\isacharbraceleft}{\kern0pt}{\isasymlangle}y{\isacharcomma}{\kern0pt}\ a{\isasymrangle}{\isacharbraceright}{\kern0pt}\ {\isacharequal}{\kern0pt}\ eclose{\isacharparenleft}{\kern0pt}y{\isacharparenright}{\kern0pt}\ {\isasymtimes}\ {\isacharbraceleft}{\kern0pt}a{\isacharbraceright}{\kern0pt}{\isachardoublequoteclose}\ \isanewline
\ \ \isacommand{proof}\isamarkupfalse%
{\isacharparenleft}{\kern0pt}rule\ equality{\isacharunderscore}{\kern0pt}iffI{\isacharcomma}{\kern0pt}\ rule\ iffI{\isacharparenright}{\kern0pt}\isanewline
\ \ \ \ \isacommand{fix}\isamarkupfalse%
\ y\ v\ \isacommand{assume}\isamarkupfalse%
\ {\isachardoublequoteopen}y\ {\isasymin}\ M{\isachardoublequoteclose}\ {\isachardoublequoteopen}v\ {\isasymin}\ prel{\isacharparenleft}{\kern0pt}Rrel{\isacharparenleft}{\kern0pt}InEclose{\isacharcomma}{\kern0pt}\ M{\isacharparenright}{\kern0pt}{\isacharcomma}{\kern0pt}\ {\isacharbraceleft}{\kern0pt}a{\isacharbraceright}{\kern0pt}{\isacharparenright}{\kern0pt}\ {\isacharminus}{\kern0pt}{\isacharbackquote}{\kern0pt}{\isacharbackquote}{\kern0pt}\ {\isacharbraceleft}{\kern0pt}{\isasymlangle}y{\isacharcomma}{\kern0pt}\ a{\isasymrangle}{\isacharbraceright}{\kern0pt}{\isachardoublequoteclose}\isanewline
\ \ \ \ \isacommand{then}\isamarkupfalse%
\ \isacommand{have}\isamarkupfalse%
\ {\isachardoublequoteopen}{\isacharless}{\kern0pt}v{\isacharcomma}{\kern0pt}\ {\isacharless}{\kern0pt}y{\isacharcomma}{\kern0pt}\ a{\isachargreater}{\kern0pt}{\isachargreater}{\kern0pt}\ {\isasymin}\ prel{\isacharparenleft}{\kern0pt}Rrel{\isacharparenleft}{\kern0pt}InEclose{\isacharcomma}{\kern0pt}\ M{\isacharparenright}{\kern0pt}{\isacharcomma}{\kern0pt}\ {\isacharbraceleft}{\kern0pt}a{\isacharbraceright}{\kern0pt}{\isacharparenright}{\kern0pt}{\isachardoublequoteclose}\ \isacommand{by}\isamarkupfalse%
\ auto\ \isanewline
\ \ \ \ \isacommand{then}\isamarkupfalse%
\ \isacommand{obtain}\isamarkupfalse%
\ z\ \isakeyword{where}\ zH\ {\isacharcolon}{\kern0pt}\ {\isachardoublequoteopen}v\ {\isacharequal}{\kern0pt}\ {\isacharless}{\kern0pt}z{\isacharcomma}{\kern0pt}\ a{\isachargreater}{\kern0pt}{\isachardoublequoteclose}\ {\isachardoublequoteopen}{\isacharless}{\kern0pt}z{\isacharcomma}{\kern0pt}\ y{\isachargreater}{\kern0pt}\ {\isasymin}\ Rrel{\isacharparenleft}{\kern0pt}InEclose{\isacharcomma}{\kern0pt}\ M{\isacharparenright}{\kern0pt}{\isachardoublequoteclose}\ \isacommand{unfolding}\isamarkupfalse%
\ prel{\isacharunderscore}{\kern0pt}def\ \isacommand{by}\isamarkupfalse%
\ auto\ \isanewline
\ \ \ \ \isacommand{then}\isamarkupfalse%
\ \isacommand{have}\isamarkupfalse%
\ {\isachardoublequoteopen}z\ {\isasymin}\ eclose{\isacharparenleft}{\kern0pt}y{\isacharparenright}{\kern0pt}{\isachardoublequoteclose}\ \isacommand{unfolding}\isamarkupfalse%
\ Rrel{\isacharunderscore}{\kern0pt}def\ InEclose{\isacharunderscore}{\kern0pt}def\ \isacommand{by}\isamarkupfalse%
\ auto\ \isanewline
\ \ \ \ \isacommand{then}\isamarkupfalse%
\ \isacommand{show}\isamarkupfalse%
\ {\isachardoublequoteopen}v\ {\isasymin}\ eclose{\isacharparenleft}{\kern0pt}y{\isacharparenright}{\kern0pt}\ {\isasymtimes}\ {\isacharbraceleft}{\kern0pt}a{\isacharbraceright}{\kern0pt}{\isachardoublequoteclose}\ \isacommand{using}\isamarkupfalse%
\ zH\ \isacommand{by}\isamarkupfalse%
\ auto\ \isanewline
\ \ \isacommand{next}\isamarkupfalse%
\ \isanewline
\ \ \ \ \isacommand{fix}\isamarkupfalse%
\ y\ v\ \isacommand{assume}\isamarkupfalse%
\ assms{\isadigit{1}}{\isacharcolon}{\kern0pt}\ {\isachardoublequoteopen}y\ {\isasymin}\ M{\isachardoublequoteclose}\ {\isachardoublequoteopen}v\ {\isasymin}\ eclose{\isacharparenleft}{\kern0pt}y{\isacharparenright}{\kern0pt}\ {\isasymtimes}\ {\isacharbraceleft}{\kern0pt}a{\isacharbraceright}{\kern0pt}{\isachardoublequoteclose}\ \isanewline
\ \ \ \ \isacommand{then}\isamarkupfalse%
\ \isacommand{obtain}\isamarkupfalse%
\ z\ \isakeyword{where}\ zH\ {\isacharcolon}{\kern0pt}\ {\isachardoublequoteopen}v\ {\isacharequal}{\kern0pt}\ {\isacharless}{\kern0pt}z{\isacharcomma}{\kern0pt}\ a{\isachargreater}{\kern0pt}{\isachardoublequoteclose}\ {\isachardoublequoteopen}z\ {\isasymin}\ eclose{\isacharparenleft}{\kern0pt}y{\isacharparenright}{\kern0pt}{\isachardoublequoteclose}\ \isacommand{by}\isamarkupfalse%
\ auto\ \isanewline
\ \ \ \ \isacommand{then}\isamarkupfalse%
\ \isacommand{have}\isamarkupfalse%
\ {\isachardoublequoteopen}z\ {\isasymin}\ M{\isachardoublequoteclose}\ \isanewline
\ \ \ \ \ \ \isacommand{apply}\isamarkupfalse%
{\isacharparenleft}{\kern0pt}subgoal{\isacharunderscore}{\kern0pt}tac\ {\isachardoublequoteopen}eclose{\isacharparenleft}{\kern0pt}y{\isacharparenright}{\kern0pt}\ {\isasymin}\ M{\isachardoublequoteclose}{\isacharparenright}{\kern0pt}\isanewline
\ \ \ \ \ \ \isacommand{using}\isamarkupfalse%
\ transM\ zH\ assms{\isadigit{1}}\ eclose{\isacharunderscore}{\kern0pt}closed\ \isanewline
\ \ \ \ \ \ \isacommand{by}\isamarkupfalse%
\ auto\ \isanewline
\ \ \ \ \isacommand{then}\isamarkupfalse%
\ \isacommand{have}\isamarkupfalse%
\ H\ {\isacharcolon}{\kern0pt}\ {\isachardoublequoteopen}{\isacharless}{\kern0pt}z{\isacharcomma}{\kern0pt}\ y{\isachargreater}{\kern0pt}\ {\isasymin}\ Rrel{\isacharparenleft}{\kern0pt}InEclose{\isacharcomma}{\kern0pt}\ M{\isacharparenright}{\kern0pt}{\isachardoublequoteclose}\ \isanewline
\ \ \ \ \ \ \isacommand{unfolding}\isamarkupfalse%
\ Rrel{\isacharunderscore}{\kern0pt}def\ InEclose{\isacharunderscore}{\kern0pt}def\ \isanewline
\ \ \ \ \ \ \isacommand{using}\isamarkupfalse%
\ assms{\isadigit{1}}\ zH\ \isanewline
\ \ \ \ \ \ \isacommand{by}\isamarkupfalse%
\ auto\ \isanewline
\ \ \ \ \isacommand{have}\isamarkupfalse%
\ {\isachardoublequoteopen}{\isacharless}{\kern0pt}z{\isacharcomma}{\kern0pt}\ a{\isachargreater}{\kern0pt}\ {\isasymin}\ prel{\isacharparenleft}{\kern0pt}Rrel{\isacharparenleft}{\kern0pt}InEclose{\isacharcomma}{\kern0pt}\ M{\isacharparenright}{\kern0pt}{\isacharcomma}{\kern0pt}\ {\isacharbraceleft}{\kern0pt}a{\isacharbraceright}{\kern0pt}{\isacharparenright}{\kern0pt}\ {\isacharminus}{\kern0pt}{\isacharbackquote}{\kern0pt}{\isacharbackquote}{\kern0pt}\ {\isacharbraceleft}{\kern0pt}{\isasymlangle}y{\isacharcomma}{\kern0pt}\ a{\isasymrangle}{\isacharbraceright}{\kern0pt}{\isachardoublequoteclose}\ \isanewline
\ \ \ \ \ \ \isacommand{apply}\isamarkupfalse%
{\isacharparenleft}{\kern0pt}rule{\isacharunderscore}{\kern0pt}tac\ b{\isacharequal}{\kern0pt}{\isachardoublequoteopen}{\isacharless}{\kern0pt}y{\isacharcomma}{\kern0pt}\ a{\isachargreater}{\kern0pt}{\isachardoublequoteclose}\ \isakeyword{in}\ vimageI{\isacharparenright}{\kern0pt}\isanewline
\ \ \ \ \ \ \ \isacommand{apply}\isamarkupfalse%
{\isacharparenleft}{\kern0pt}rule\ prelI{\isacharcomma}{\kern0pt}\ simp\ add{\isacharcolon}{\kern0pt}H{\isacharparenright}{\kern0pt}\isanewline
\ \ \ \ \ \ \isacommand{using}\isamarkupfalse%
\ assms\isanewline
\ \ \ \ \ \ \isacommand{by}\isamarkupfalse%
\ auto\ \isanewline
\ \ \ \ \isacommand{then}\isamarkupfalse%
\ \isacommand{show}\isamarkupfalse%
\ {\isachardoublequoteopen}v\ {\isasymin}\ prel{\isacharparenleft}{\kern0pt}Rrel{\isacharparenleft}{\kern0pt}InEclose{\isacharcomma}{\kern0pt}\ M{\isacharparenright}{\kern0pt}{\isacharcomma}{\kern0pt}\ {\isacharbraceleft}{\kern0pt}a{\isacharbraceright}{\kern0pt}{\isacharparenright}{\kern0pt}\ {\isacharminus}{\kern0pt}{\isacharbackquote}{\kern0pt}{\isacharbackquote}{\kern0pt}\ {\isacharbraceleft}{\kern0pt}{\isasymlangle}y{\isacharcomma}{\kern0pt}\ a{\isasymrangle}{\isacharbraceright}{\kern0pt}{\isachardoublequoteclose}\ \isanewline
\ \ \ \ \ \ \isacommand{using}\isamarkupfalse%
\ zH\ \isacommand{by}\isamarkupfalse%
\ auto\isanewline
\ \ \isacommand{qed}\isamarkupfalse%
\isanewline
\isanewline
\ \ \isacommand{have}\isamarkupfalse%
\ {\isachardoublequoteopen}sats{\isacharparenleft}{\kern0pt}M{\isacharcomma}{\kern0pt}\ is{\isacharunderscore}{\kern0pt}wftrec{\isacharunderscore}{\kern0pt}fm{\isacharparenleft}{\kern0pt}Gfm{\isacharcomma}{\kern0pt}\ InEclose{\isacharunderscore}{\kern0pt}fm{\isacharcomma}{\kern0pt}\ i{\isacharcomma}{\kern0pt}\ j{\isacharcomma}{\kern0pt}\ k{\isacharparenright}{\kern0pt}{\isacharcomma}{\kern0pt}\ env{\isacharparenright}{\kern0pt}\ {\isasymlongleftrightarrow}\ v\ {\isacharequal}{\kern0pt}\ wftrec{\isacharparenleft}{\kern0pt}Rrel{\isacharparenleft}{\kern0pt}InEclose{\isacharcomma}{\kern0pt}\ M{\isacharparenright}{\kern0pt}{\isacharcomma}{\kern0pt}\ x{\isacharcomma}{\kern0pt}\ H{\isacharparenright}{\kern0pt}{\isachardoublequoteclose}\ \isanewline
\ \ \ \ \isacommand{apply}\isamarkupfalse%
{\isacharparenleft}{\kern0pt}rule{\isacharunderscore}{\kern0pt}tac\ \isanewline
\ \ \ \ \ \ a\ {\isacharequal}{\kern0pt}\ a\ \isakeyword{and}\ \isanewline
\ \ \ \ \ \ G\ {\isacharequal}{\kern0pt}\ G\isanewline
\ \ \ \ \ \ \isakeyword{in}\ sats{\isacharunderscore}{\kern0pt}is{\isacharunderscore}{\kern0pt}Rrel{\isacharunderscore}{\kern0pt}wftrec{\isacharunderscore}{\kern0pt}fm{\isacharunderscore}{\kern0pt}iff{\isacharparenright}{\kern0pt}\isanewline
\ \ \ \ \isacommand{using}\isamarkupfalse%
\ assms\ field{\isacharunderscore}{\kern0pt}Rrel{\isacharunderscore}{\kern0pt}InEclose\ \isanewline
\ \ \ \ \ \ \ \ \ \ \ \ \ \ \ \ \ \ \ \ \ \ \isacommand{apply}\isamarkupfalse%
\ auto{\isacharbrackleft}{\kern0pt}{\isadigit{1}}{\isadigit{0}}{\isacharbrackright}{\kern0pt}\isanewline
\ \ \ \ \ \ \ \ \ \ \ \ \isacommand{apply}\isamarkupfalse%
{\isacharparenleft}{\kern0pt}subst\ Rrel{\isacharunderscore}{\kern0pt}InEclose{\isacharcomma}{\kern0pt}\ rule\ wf{\isacharunderscore}{\kern0pt}trancl{\isacharcomma}{\kern0pt}\ rule\ wf{\isacharunderscore}{\kern0pt}Memrel{\isacharparenright}{\kern0pt}\isanewline
\ \ \ \ \ \ \ \ \ \ \ \isacommand{apply}\isamarkupfalse%
{\isacharparenleft}{\kern0pt}subst\ Rrel{\isacharunderscore}{\kern0pt}InEclose{\isacharcomma}{\kern0pt}\ rule\ trans{\isacharunderscore}{\kern0pt}trancl{\isacharparenright}{\kern0pt}\isanewline
\ \ \ \ \ \ \ \ \ \ \isacommand{apply}\isamarkupfalse%
{\isacharparenleft}{\kern0pt}subst\ preds{\isacharunderscore}{\kern0pt}InEclose{\isacharunderscore}{\kern0pt}eq{\isacharcomma}{\kern0pt}\ simp\ add{\isacharcolon}{\kern0pt}assms{\isacharcomma}{\kern0pt}\ rule\ to{\isacharunderscore}{\kern0pt}rin{\isacharcomma}{\kern0pt}\ rule\ eclose{\isacharunderscore}{\kern0pt}closed{\isacharcomma}{\kern0pt}\ simp\ add{\isacharcolon}{\kern0pt}assms{\isacharparenright}{\kern0pt}\isanewline
\ \ \ \ \isacommand{using}\isamarkupfalse%
\ assms\ Relation{\isacharunderscore}{\kern0pt}fm{\isacharunderscore}{\kern0pt}InEclose\isanewline
\ \ \ \ \ \ \ \ \isacommand{apply}\isamarkupfalse%
\ auto{\isacharbrackleft}{\kern0pt}{\isadigit{4}}{\isacharbrackright}{\kern0pt}\isanewline
\ \ \ \ \ \isacommand{apply}\isamarkupfalse%
{\isacharparenleft}{\kern0pt}rule\ sats{\isacharunderscore}{\kern0pt}Gfm{\isacharunderscore}{\kern0pt}iff{\isacharparenright}{\kern0pt}\isanewline
\ \ \ \ \ \ \ \ \isacommand{apply}\isamarkupfalse%
\ auto{\isacharbrackleft}{\kern0pt}{\isadigit{4}}{\isacharbrackright}{\kern0pt}\isanewline
\ \ \ \ \isacommand{apply}\isamarkupfalse%
{\isacharparenleft}{\kern0pt}rename{\isacharunderscore}{\kern0pt}tac\ h\ g\ x{\isacharcomma}{\kern0pt}\ subgoal{\isacharunderscore}{\kern0pt}tac\ {\isachardoublequoteopen}x\ {\isasymin}\ M{\isachardoublequoteclose}{\isacharparenright}{\kern0pt}\isanewline
\ \ \ \ \isacommand{apply}\isamarkupfalse%
{\isacharparenleft}{\kern0pt}rule\ HGeq{\isacharparenright}{\kern0pt}\isanewline
\ \ \ \ \ \ \ \ \isacommand{apply}\isamarkupfalse%
{\isacharparenleft}{\kern0pt}rename{\isacharunderscore}{\kern0pt}tac\ h\ g\ x{\isacharcomma}{\kern0pt}\ rule{\isacharunderscore}{\kern0pt}tac\ b{\isacharequal}{\kern0pt}{\isachardoublequoteopen}eclose{\isacharparenleft}{\kern0pt}x{\isacharparenright}{\kern0pt}{\isachardoublequoteclose}\ \isakeyword{and}\ a{\isacharequal}{\kern0pt}{\isachardoublequoteopen}Rrel{\isacharparenleft}{\kern0pt}InEclose{\isacharcomma}{\kern0pt}\ M{\isacharparenright}{\kern0pt}\ {\isacharminus}{\kern0pt}{\isacharbackquote}{\kern0pt}{\isacharbackquote}{\kern0pt}\ {\isacharbraceleft}{\kern0pt}x{\isacharbraceright}{\kern0pt}{\isachardoublequoteclose}\ \isakeyword{in}\ ssubst{\isacharparenright}{\kern0pt}\isanewline
\ \ \ \ \ \ \ \ \ \ \isacommand{apply}\isamarkupfalse%
{\isacharparenleft}{\kern0pt}rule\ eq{\isacharunderscore}{\kern0pt}flip{\isacharcomma}{\kern0pt}\ rule\ Rrel{\isacharunderscore}{\kern0pt}InEclose{\isacharunderscore}{\kern0pt}vimage{\isacharunderscore}{\kern0pt}eq{\isacharcomma}{\kern0pt}\ simp{\isacharcomma}{\kern0pt}\ simp{\isacharparenright}{\kern0pt}\isanewline
\ \ \ \ \ \ \ \ \isacommand{apply}\isamarkupfalse%
{\isacharparenleft}{\kern0pt}rename{\isacharunderscore}{\kern0pt}tac\ h\ g\ x{\isacharcomma}{\kern0pt}\ rule{\isacharunderscore}{\kern0pt}tac\ b{\isacharequal}{\kern0pt}{\isachardoublequoteopen}eclose{\isacharparenleft}{\kern0pt}x{\isacharparenright}{\kern0pt}{\isachardoublequoteclose}\ \isakeyword{and}\ a{\isacharequal}{\kern0pt}{\isachardoublequoteopen}Rrel{\isacharparenleft}{\kern0pt}InEclose{\isacharcomma}{\kern0pt}\ M{\isacharparenright}{\kern0pt}\ {\isacharminus}{\kern0pt}{\isacharbackquote}{\kern0pt}{\isacharbackquote}{\kern0pt}\ {\isacharbraceleft}{\kern0pt}x{\isacharbraceright}{\kern0pt}{\isachardoublequoteclose}\ \isakeyword{in}\ ssubst{\isacharparenright}{\kern0pt}\isanewline
\ \ \ \ \ \ \ \ \ \ \isacommand{apply}\isamarkupfalse%
{\isacharparenleft}{\kern0pt}rule\ eq{\isacharunderscore}{\kern0pt}flip{\isacharcomma}{\kern0pt}\ rule\ Rrel{\isacharunderscore}{\kern0pt}InEclose{\isacharunderscore}{\kern0pt}vimage{\isacharunderscore}{\kern0pt}eq{\isacharcomma}{\kern0pt}\ simp{\isacharcomma}{\kern0pt}\ simp{\isacharcomma}{\kern0pt}\ simp{\isacharcomma}{\kern0pt}\ simp{\isacharparenright}{\kern0pt}\isanewline
\ \ \ \ \isacommand{apply}\isamarkupfalse%
{\isacharparenleft}{\kern0pt}rename{\isacharunderscore}{\kern0pt}tac\ h\ g\ x\ y{\isacharcomma}{\kern0pt}\ subgoal{\isacharunderscore}{\kern0pt}tac\ {\isachardoublequoteopen}y\ {\isasymin}\ Rrel{\isacharparenleft}{\kern0pt}InEclose{\isacharcomma}{\kern0pt}\ M{\isacharparenright}{\kern0pt}\ {\isacharminus}{\kern0pt}{\isacharbackquote}{\kern0pt}{\isacharbackquote}{\kern0pt}\ {\isacharbraceleft}{\kern0pt}x{\isacharbraceright}{\kern0pt}{\isachardoublequoteclose}{\isacharcomma}{\kern0pt}\ force{\isacharparenright}{\kern0pt}\isanewline
\ \ \ \ \isacommand{using}\isamarkupfalse%
\ Rrel{\isacharunderscore}{\kern0pt}InEclose{\isacharunderscore}{\kern0pt}vimage{\isacharunderscore}{\kern0pt}eq\isanewline
\ \ \ \ \ \isacommand{apply}\isamarkupfalse%
\ force\isanewline
\ \ \ \ \isacommand{apply}\isamarkupfalse%
{\isacharparenleft}{\kern0pt}rule{\isacharunderscore}{\kern0pt}tac\ a{\isacharequal}{\kern0pt}{\isachardoublequoteopen}field{\isacharparenleft}{\kern0pt}Rrel{\isacharparenleft}{\kern0pt}InEclose{\isacharcomma}{\kern0pt}\ M{\isacharparenright}{\kern0pt}{\isacharparenright}{\kern0pt}{\isachardoublequoteclose}\ \isakeyword{and}\ b{\isacharequal}{\kern0pt}M\ \isakeyword{in}\ ssubst{\isacharcomma}{\kern0pt}\ rule\ eq{\isacharunderscore}{\kern0pt}flip{\isacharcomma}{\kern0pt}\ rule\ field{\isacharunderscore}{\kern0pt}Rrel{\isacharunderscore}{\kern0pt}InEclose{\isacharcomma}{\kern0pt}\ simp{\isacharparenright}{\kern0pt}\isanewline
\ \ \ \ \isacommand{done}\isamarkupfalse%
\isanewline
\isanewline
\ \ \isacommand{then}\isamarkupfalse%
\ \isacommand{show}\isamarkupfalse%
\ {\isacharquery}{\kern0pt}thesis\ \isacommand{unfolding}\isamarkupfalse%
\ is{\isacharunderscore}{\kern0pt}memrel{\isacharunderscore}{\kern0pt}wftrec{\isacharunderscore}{\kern0pt}fm{\isacharunderscore}{\kern0pt}def\ \ \isacommand{using}\isamarkupfalse%
\ Rrel{\isacharunderscore}{\kern0pt}InEclose\ \isacommand{by}\isamarkupfalse%
\ auto\isanewline
\isacommand{qed}\isamarkupfalse%
%
\endisatagproof
{\isafoldproof}%
%
\isadelimproof
\isanewline
%
\endisadelimproof
\isanewline
\isacommand{lemma}\isamarkupfalse%
\ memrel{\isacharunderscore}{\kern0pt}wftrec{\isacharunderscore}{\kern0pt}in{\isacharunderscore}{\kern0pt}M\ {\isacharcolon}{\kern0pt}\ \isanewline
\ \ \isakeyword{fixes}\ H\ G\ Gfm\ x\ a\isanewline
\ \ \isakeyword{assumes}\ {\isachardoublequoteopen}a\ {\isasymin}\ M{\isachardoublequoteclose}\ {\isachardoublequoteopen}x\ {\isasymin}\ M{\isachardoublequoteclose}\ \isanewline
\ \ \ \ \ \ \ \ \ \ {\isachardoublequoteopen}Gfm\ {\isasymin}\ formula{\isachardoublequoteclose}\ \isanewline
\ \ \ \ \ \ \ \ \ \ {\isachardoublequoteopen}arity{\isacharparenleft}{\kern0pt}Gfm{\isacharparenright}{\kern0pt}\ {\isasymle}\ {\isadigit{3}}{\isachardoublequoteclose}\ \isanewline
\ \ \ \ \ \ \ \ \ \ {\isachardoublequoteopen}{\isasymAnd}x\ g{\isachardot}{\kern0pt}\ x\ {\isasymin}\ M\ {\isasymLongrightarrow}\ g\ {\isasymin}\ M\ {\isasymLongrightarrow}\ function{\isacharparenleft}{\kern0pt}g{\isacharparenright}{\kern0pt}\ {\isasymLongrightarrow}\ G{\isacharparenleft}{\kern0pt}x{\isacharcomma}{\kern0pt}\ g{\isacharparenright}{\kern0pt}\ {\isasymin}\ M{\isachardoublequoteclose}\ \isanewline
\ \ \isakeyword{and}\ HGeq{\isacharcolon}{\kern0pt}\ {\isachardoublequoteopen}{\isasymAnd}h\ g\ x{\isachardot}{\kern0pt}\ h\ {\isasymin}\ eclose{\isacharparenleft}{\kern0pt}x{\isacharparenright}{\kern0pt}\ {\isasymrightarrow}\ M\ {\isasymLongrightarrow}\ g\ {\isasymin}\ {\isacharparenleft}{\kern0pt}eclose{\isacharparenleft}{\kern0pt}x{\isacharparenright}{\kern0pt}\ {\isasymtimes}\ {\isacharbraceleft}{\kern0pt}a{\isacharbraceright}{\kern0pt}{\isacharparenright}{\kern0pt}\ {\isasymrightarrow}\ M\ {\isasymLongrightarrow}\ g\ {\isasymin}\ M\ \ \isanewline
\ \ \ \ \ \ \ \ \ \ \ \ \ \ \ {\isasymLongrightarrow}\ x\ {\isasymin}\ M\ {\isasymLongrightarrow}\ {\isacharparenleft}{\kern0pt}{\isasymAnd}y{\isachardot}{\kern0pt}\ y\ {\isasymin}\ eclose{\isacharparenleft}{\kern0pt}x{\isacharparenright}{\kern0pt}\ {\isasymLongrightarrow}\ h{\isacharbackquote}{\kern0pt}y\ {\isacharequal}{\kern0pt}\ g{\isacharbackquote}{\kern0pt}{\isacharless}{\kern0pt}y{\isacharcomma}{\kern0pt}\ a{\isachargreater}{\kern0pt}{\isacharparenright}{\kern0pt}\ {\isasymLongrightarrow}\ H{\isacharparenleft}{\kern0pt}x{\isacharcomma}{\kern0pt}\ h{\isacharparenright}{\kern0pt}\ {\isacharequal}{\kern0pt}\ G{\isacharparenleft}{\kern0pt}{\isacharless}{\kern0pt}x{\isacharcomma}{\kern0pt}\ a{\isachargreater}{\kern0pt}{\isacharcomma}{\kern0pt}\ g{\isacharparenright}{\kern0pt}{\isachardoublequoteclose}\ \ \isanewline
\ \ \isakeyword{and}\ sats{\isacharunderscore}{\kern0pt}Gfm{\isacharunderscore}{\kern0pt}iff\ {\isacharcolon}{\kern0pt}\ {\isachardoublequoteopen}\ {\isacharparenleft}{\kern0pt}{\isasymAnd}a{\isadigit{0}}\ a{\isadigit{1}}\ a{\isadigit{2}}\ env{\isachardot}{\kern0pt}\ a{\isadigit{0}}\ {\isasymin}\ M\ {\isasymLongrightarrow}\ a{\isadigit{1}}\ {\isasymin}\ M\ {\isasymLongrightarrow}\ a{\isadigit{2}}\ {\isasymin}\ M\ {\isasymLongrightarrow}\ env\ {\isasymin}\ list{\isacharparenleft}{\kern0pt}M{\isacharparenright}{\kern0pt}\ {\isasymLongrightarrow}\ a{\isadigit{0}}\ {\isacharequal}{\kern0pt}\ G{\isacharparenleft}{\kern0pt}a{\isadigit{2}}{\isacharcomma}{\kern0pt}\ a{\isadigit{1}}{\isacharparenright}{\kern0pt}\ {\isasymlongleftrightarrow}\ sats{\isacharparenleft}{\kern0pt}M{\isacharcomma}{\kern0pt}\ Gfm{\isacharcomma}{\kern0pt}\ {\isacharbrackleft}{\kern0pt}a{\isadigit{0}}{\isacharcomma}{\kern0pt}\ a{\isadigit{1}}{\isacharcomma}{\kern0pt}\ a{\isadigit{2}}{\isacharbrackright}{\kern0pt}\ {\isacharat}{\kern0pt}\ env{\isacharparenright}{\kern0pt}{\isacharparenright}{\kern0pt}{\isachardoublequoteclose}\ \isanewline
\isanewline
\ \ \isakeyword{shows}\ {\isachardoublequoteopen}wftrec{\isacharparenleft}{\kern0pt}Memrel{\isacharparenleft}{\kern0pt}M{\isacharparenright}{\kern0pt}{\isacharcircum}{\kern0pt}{\isacharplus}{\kern0pt}{\isacharcomma}{\kern0pt}\ x{\isacharcomma}{\kern0pt}\ H{\isacharparenright}{\kern0pt}\ {\isasymin}\ M{\isachardoublequoteclose}\isanewline
%
\isadelimproof
\isanewline
\ \ %
\endisadelimproof
%
\isatagproof
\isacommand{apply}\isamarkupfalse%
{\isacharparenleft}{\kern0pt}rule{\isacharunderscore}{\kern0pt}tac\ b{\isacharequal}{\kern0pt}{\isachardoublequoteopen}Memrel{\isacharparenleft}{\kern0pt}M{\isacharparenright}{\kern0pt}{\isacharcircum}{\kern0pt}{\isacharplus}{\kern0pt}{\isachardoublequoteclose}\ \isakeyword{in}\ ssubst{\isacharcomma}{\kern0pt}\ rule\ eq{\isacharunderscore}{\kern0pt}flip{\isacharcomma}{\kern0pt}\ rule\ Rrel{\isacharunderscore}{\kern0pt}InEclose{\isacharparenright}{\kern0pt}\isanewline
\ \ \isacommand{apply}\isamarkupfalse%
{\isacharparenleft}{\kern0pt}rule{\isacharunderscore}{\kern0pt}tac\ a{\isacharequal}{\kern0pt}a\ \isakeyword{and}\ Gfm{\isacharequal}{\kern0pt}Gfm\ \isakeyword{in}\ Rrel{\isacharunderscore}{\kern0pt}wftrec{\isacharunderscore}{\kern0pt}in{\isacharunderscore}{\kern0pt}M{\isacharparenright}{\kern0pt}\ \isanewline
\ \ \isacommand{using}\isamarkupfalse%
\ assms\isanewline
\ \ \ \ \ \ \ \ \ \ \ \ \ \isacommand{apply}\isamarkupfalse%
\ auto{\isacharbrackleft}{\kern0pt}{\isadigit{2}}{\isacharbrackright}{\kern0pt}\isanewline
\ \ \ \ \ \ \ \ \ \ \ \isacommand{apply}\isamarkupfalse%
{\isacharparenleft}{\kern0pt}subst\ field{\isacharunderscore}{\kern0pt}Rrel{\isacharunderscore}{\kern0pt}InEclose{\isacharcomma}{\kern0pt}\ simp\ add{\isacharcolon}{\kern0pt}assms{\isacharparenright}{\kern0pt}\isanewline
\ \ \ \ \ \ \ \ \ \ \isacommand{apply}\isamarkupfalse%
{\isacharparenleft}{\kern0pt}subst\ Rrel{\isacharunderscore}{\kern0pt}InEclose{\isacharcomma}{\kern0pt}\ rule\ wf{\isacharunderscore}{\kern0pt}trancl{\isacharcomma}{\kern0pt}\ rule\ wf{\isacharunderscore}{\kern0pt}Memrel{\isacharparenright}{\kern0pt}\isanewline
\ \ \ \ \ \ \ \ \ \isacommand{apply}\isamarkupfalse%
{\isacharparenleft}{\kern0pt}subst\ Rrel{\isacharunderscore}{\kern0pt}InEclose{\isacharcomma}{\kern0pt}\ rule\ trans{\isacharunderscore}{\kern0pt}trancl{\isacharparenright}{\kern0pt}\isanewline
\ \ \ \ \ \ \ \ \isacommand{apply}\isamarkupfalse%
{\isacharparenleft}{\kern0pt}rule\ Relation{\isacharunderscore}{\kern0pt}fm{\isacharunderscore}{\kern0pt}InEclose{\isacharparenright}{\kern0pt}\isanewline
\ \ \ \ \ \ \ \isacommand{apply}\isamarkupfalse%
{\isacharparenleft}{\kern0pt}subst\ preds{\isacharunderscore}{\kern0pt}InEclose{\isacharunderscore}{\kern0pt}eq{\isacharcomma}{\kern0pt}\ simp\ add{\isacharcolon}{\kern0pt}assms{\isacharcomma}{\kern0pt}\ rule\ to{\isacharunderscore}{\kern0pt}rin{\isacharcomma}{\kern0pt}\ rule\ eclose{\isacharunderscore}{\kern0pt}closed{\isacharcomma}{\kern0pt}\ simp\ add{\isacharcolon}{\kern0pt}assms{\isacharparenright}{\kern0pt}\isanewline
\ \ \isacommand{using}\isamarkupfalse%
\ assms\isanewline
\ \ \ \ \ \ \isacommand{apply}\isamarkupfalse%
\ auto{\isacharbrackleft}{\kern0pt}{\isadigit{2}}{\isacharbrackright}{\kern0pt}\isanewline
\ \ \ \ \isacommand{apply}\isamarkupfalse%
{\isacharparenleft}{\kern0pt}rule\ sats{\isacharunderscore}{\kern0pt}Gfm{\isacharunderscore}{\kern0pt}iff{\isacharparenright}{\kern0pt}\isanewline
\ \ \ \ \ \ \ \isacommand{apply}\isamarkupfalse%
\ auto{\isacharbrackleft}{\kern0pt}{\isadigit{4}}{\isacharbrackright}{\kern0pt}\isanewline
\ \ \isacommand{using}\isamarkupfalse%
\ assms\ \isanewline
\ \ \ \isacommand{apply}\isamarkupfalse%
\ auto{\isacharbrackleft}{\kern0pt}{\isadigit{1}}{\isacharbrackright}{\kern0pt}\isanewline
\ \ \isacommand{apply}\isamarkupfalse%
{\isacharparenleft}{\kern0pt}rename{\isacharunderscore}{\kern0pt}tac\ h\ g\ x{\isacharcomma}{\kern0pt}\ subgoal{\isacharunderscore}{\kern0pt}tac\ {\isachardoublequoteopen}x\ {\isasymin}\ M{\isachardoublequoteclose}{\isacharparenright}{\kern0pt}\isanewline
\ \ \isacommand{apply}\isamarkupfalse%
{\isacharparenleft}{\kern0pt}rule\ HGeq{\isacharparenright}{\kern0pt}\isanewline
\ \ \ \ \ \ \ \isacommand{apply}\isamarkupfalse%
{\isacharparenleft}{\kern0pt}rename{\isacharunderscore}{\kern0pt}tac\ h\ g\ x{\isacharcomma}{\kern0pt}\ rule{\isacharunderscore}{\kern0pt}tac\ b{\isacharequal}{\kern0pt}{\isachardoublequoteopen}eclose{\isacharparenleft}{\kern0pt}x{\isacharparenright}{\kern0pt}{\isachardoublequoteclose}\ \isakeyword{and}\ a{\isacharequal}{\kern0pt}{\isachardoublequoteopen}Rrel{\isacharparenleft}{\kern0pt}InEclose{\isacharcomma}{\kern0pt}\ M{\isacharparenright}{\kern0pt}\ {\isacharminus}{\kern0pt}{\isacharbackquote}{\kern0pt}{\isacharbackquote}{\kern0pt}\ {\isacharbraceleft}{\kern0pt}x{\isacharbraceright}{\kern0pt}{\isachardoublequoteclose}\ \isakeyword{in}\ ssubst{\isacharparenright}{\kern0pt}\isanewline
\ \ \ \ \ \ \isacommand{apply}\isamarkupfalse%
{\isacharparenleft}{\kern0pt}rule\ eq{\isacharunderscore}{\kern0pt}flip{\isacharcomma}{\kern0pt}\ rule\ Rrel{\isacharunderscore}{\kern0pt}InEclose{\isacharunderscore}{\kern0pt}vimage{\isacharunderscore}{\kern0pt}eq{\isacharcomma}{\kern0pt}\ simp{\isacharcomma}{\kern0pt}\ simp{\isacharparenright}{\kern0pt}\isanewline
\ \ \ \ \ \ \isacommand{apply}\isamarkupfalse%
{\isacharparenleft}{\kern0pt}rename{\isacharunderscore}{\kern0pt}tac\ h\ g\ x{\isacharcomma}{\kern0pt}\ rule{\isacharunderscore}{\kern0pt}tac\ b{\isacharequal}{\kern0pt}{\isachardoublequoteopen}eclose{\isacharparenleft}{\kern0pt}x{\isacharparenright}{\kern0pt}{\isachardoublequoteclose}\ \isakeyword{and}\ a{\isacharequal}{\kern0pt}{\isachardoublequoteopen}Rrel{\isacharparenleft}{\kern0pt}InEclose{\isacharcomma}{\kern0pt}\ M{\isacharparenright}{\kern0pt}\ {\isacharminus}{\kern0pt}{\isacharbackquote}{\kern0pt}{\isacharbackquote}{\kern0pt}\ {\isacharbraceleft}{\kern0pt}x{\isacharbraceright}{\kern0pt}{\isachardoublequoteclose}\ \isakeyword{in}\ ssubst{\isacharparenright}{\kern0pt}\isanewline
\ \ \ \ \ \ \ \isacommand{apply}\isamarkupfalse%
{\isacharparenleft}{\kern0pt}rule\ eq{\isacharunderscore}{\kern0pt}flip{\isacharcomma}{\kern0pt}\ rule\ Rrel{\isacharunderscore}{\kern0pt}InEclose{\isacharunderscore}{\kern0pt}vimage{\isacharunderscore}{\kern0pt}eq{\isacharcomma}{\kern0pt}\ simp{\isacharcomma}{\kern0pt}\ simp{\isacharcomma}{\kern0pt}\ simp{\isacharcomma}{\kern0pt}\ simp{\isacharparenright}{\kern0pt}\isanewline
\ \ \ \isacommand{apply}\isamarkupfalse%
{\isacharparenleft}{\kern0pt}rename{\isacharunderscore}{\kern0pt}tac\ h\ g\ x\ y{\isacharcomma}{\kern0pt}\ subgoal{\isacharunderscore}{\kern0pt}tac\ {\isachardoublequoteopen}y\ {\isasymin}\ Rrel{\isacharparenleft}{\kern0pt}InEclose{\isacharcomma}{\kern0pt}\ M{\isacharparenright}{\kern0pt}\ {\isacharminus}{\kern0pt}{\isacharbackquote}{\kern0pt}{\isacharbackquote}{\kern0pt}\ {\isacharbraceleft}{\kern0pt}x{\isacharbraceright}{\kern0pt}{\isachardoublequoteclose}{\isacharcomma}{\kern0pt}\ force{\isacharparenright}{\kern0pt}\isanewline
\ \ \isacommand{using}\isamarkupfalse%
\ Rrel{\isacharunderscore}{\kern0pt}InEclose{\isacharunderscore}{\kern0pt}vimage{\isacharunderscore}{\kern0pt}eq\isanewline
\ \ \ \isacommand{apply}\isamarkupfalse%
\ force\isanewline
\ \ \isacommand{apply}\isamarkupfalse%
{\isacharparenleft}{\kern0pt}rule{\isacharunderscore}{\kern0pt}tac\ a{\isacharequal}{\kern0pt}{\isachardoublequoteopen}field{\isacharparenleft}{\kern0pt}Rrel{\isacharparenleft}{\kern0pt}InEclose{\isacharcomma}{\kern0pt}\ M{\isacharparenright}{\kern0pt}{\isacharparenright}{\kern0pt}{\isachardoublequoteclose}\ \isakeyword{and}\ b{\isacharequal}{\kern0pt}M\ \isakeyword{in}\ ssubst{\isacharcomma}{\kern0pt}\ rule\ eq{\isacharunderscore}{\kern0pt}flip{\isacharcomma}{\kern0pt}\ rule\ field{\isacharunderscore}{\kern0pt}Rrel{\isacharunderscore}{\kern0pt}InEclose{\isacharcomma}{\kern0pt}\ simp{\isacharparenright}{\kern0pt}\isanewline
\ \ \isacommand{done}\isamarkupfalse%
%
\endisatagproof
{\isafoldproof}%
%
\isadelimproof
\isanewline
%
\endisadelimproof
\ \ \isanewline
\isacommand{lemma}\isamarkupfalse%
\ domain{\isacharunderscore}{\kern0pt}elem{\isacharunderscore}{\kern0pt}Memrel{\isacharunderscore}{\kern0pt}trancl\ {\isacharcolon}{\kern0pt}\ \isanewline
\ \ \isakeyword{fixes}\ x\ y\ \isanewline
\ \ \isakeyword{assumes}\ {\isachardoublequoteopen}x\ {\isasymin}\ M{\isachardoublequoteclose}\ {\isachardoublequoteopen}y\ {\isasymin}\ domain{\isacharparenleft}{\kern0pt}x{\isacharparenright}{\kern0pt}{\isachardoublequoteclose}\ \isanewline
\ \ \isakeyword{shows}\ {\isachardoublequoteopen}{\isacharless}{\kern0pt}y{\isacharcomma}{\kern0pt}\ x{\isachargreater}{\kern0pt}\ {\isasymin}\ Memrel{\isacharparenleft}{\kern0pt}M{\isacharparenright}{\kern0pt}{\isacharcircum}{\kern0pt}{\isacharplus}{\kern0pt}{\isachardoublequoteclose}\ \isanewline
%
\isadelimproof
%
\endisadelimproof
%
\isatagproof
\isacommand{proof}\isamarkupfalse%
\ {\isacharminus}{\kern0pt}\ \isanewline
\ \ \isacommand{obtain}\isamarkupfalse%
\ z\ \isakeyword{where}\ zH\ {\isacharcolon}{\kern0pt}\ {\isachardoublequoteopen}{\isacharless}{\kern0pt}y{\isacharcomma}{\kern0pt}\ z{\isachargreater}{\kern0pt}\ {\isasymin}\ x{\isachardoublequoteclose}\ \isacommand{using}\isamarkupfalse%
\ assms\ \isacommand{by}\isamarkupfalse%
\ auto\ \isanewline
\ \ \isacommand{have}\isamarkupfalse%
\ singin\ {\isacharcolon}{\kern0pt}\ {\isachardoublequoteopen}{\isacharbraceleft}{\kern0pt}y{\isacharbraceright}{\kern0pt}\ {\isasymin}\ {\isacharless}{\kern0pt}y{\isacharcomma}{\kern0pt}\ z{\isachargreater}{\kern0pt}{\isachardoublequoteclose}\ \isacommand{using}\isamarkupfalse%
\ Pair{\isacharunderscore}{\kern0pt}def\ \isacommand{by}\isamarkupfalse%
\ auto\ \isanewline
\ \ \isacommand{have}\isamarkupfalse%
\ yinM\ {\isacharcolon}{\kern0pt}\ {\isachardoublequoteopen}y\ {\isasymin}\ M{\isachardoublequoteclose}\ \isacommand{using}\isamarkupfalse%
\ domain{\isacharunderscore}{\kern0pt}elem{\isacharunderscore}{\kern0pt}in{\isacharunderscore}{\kern0pt}M\ assms\ \isacommand{by}\isamarkupfalse%
\ auto\isanewline
\ \ \isacommand{have}\isamarkupfalse%
\ yzinM\ {\isacharcolon}{\kern0pt}\ {\isachardoublequoteopen}{\isacharless}{\kern0pt}y{\isacharcomma}{\kern0pt}\ z{\isachargreater}{\kern0pt}\ {\isasymin}\ M{\isachardoublequoteclose}\ \isacommand{using}\isamarkupfalse%
\ zH\ assms\ transM\ \isacommand{by}\isamarkupfalse%
\ auto\ \isanewline
\ \ \isacommand{then}\isamarkupfalse%
\ \isacommand{have}\isamarkupfalse%
\ zinM\ {\isacharcolon}{\kern0pt}\ {\isachardoublequoteopen}z\ {\isasymin}\ M{\isachardoublequoteclose}\ \isacommand{using}\isamarkupfalse%
\ pair{\isacharunderscore}{\kern0pt}in{\isacharunderscore}{\kern0pt}M{\isacharunderscore}{\kern0pt}iff\ \isacommand{by}\isamarkupfalse%
\ auto\ \isanewline
\ \ \isacommand{show}\isamarkupfalse%
\ {\isacharquery}{\kern0pt}thesis\isanewline
\ \ \ \ \isacommand{apply}\isamarkupfalse%
{\isacharparenleft}{\kern0pt}rule{\isacharunderscore}{\kern0pt}tac\ b{\isacharequal}{\kern0pt}{\isachardoublequoteopen}{\isacharless}{\kern0pt}y{\isacharcomma}{\kern0pt}\ z{\isachargreater}{\kern0pt}{\isachardoublequoteclose}\ \isakeyword{in}\ rtrancl{\isacharunderscore}{\kern0pt}into{\isacharunderscore}{\kern0pt}trancl{\isadigit{1}}{\isacharparenright}{\kern0pt}\isanewline
\ \ \ \ \ \isacommand{apply}\isamarkupfalse%
{\isacharparenleft}{\kern0pt}rule{\isacharunderscore}{\kern0pt}tac\ b{\isacharequal}{\kern0pt}{\isachardoublequoteopen}{\isacharbraceleft}{\kern0pt}y{\isacharbraceright}{\kern0pt}{\isachardoublequoteclose}\ \isakeyword{in}\ rtrancl{\isacharunderscore}{\kern0pt}into{\isacharunderscore}{\kern0pt}rtrancl{\isacharparenright}{\kern0pt}\isanewline
\ \ \ \ \ \ \isacommand{apply}\isamarkupfalse%
{\isacharparenleft}{\kern0pt}rule\ r{\isacharunderscore}{\kern0pt}into{\isacharunderscore}{\kern0pt}rtrancl{\isacharparenright}{\kern0pt}\isanewline
\ \ \ \ \isacommand{unfolding}\isamarkupfalse%
\ Memrel{\isacharunderscore}{\kern0pt}def\ \isanewline
\ \ \ \ \isacommand{using}\isamarkupfalse%
\ yinM\ singleton{\isacharunderscore}{\kern0pt}in{\isacharunderscore}{\kern0pt}M{\isacharunderscore}{\kern0pt}iff\ \isanewline
\ \ \ \ \ \ \isacommand{apply}\isamarkupfalse%
\ force\isanewline
\ \ \ \ \isacommand{using}\isamarkupfalse%
\ yinM\ singleton{\isacharunderscore}{\kern0pt}in{\isacharunderscore}{\kern0pt}M{\isacharunderscore}{\kern0pt}iff\ pair{\isacharunderscore}{\kern0pt}in{\isacharunderscore}{\kern0pt}M{\isacharunderscore}{\kern0pt}iff\ zH\ singin\ zinM\ \isanewline
\ \ \ \ \ \isacommand{apply}\isamarkupfalse%
\ force\isanewline
\ \ \ \ \isacommand{apply}\isamarkupfalse%
\ simp\isanewline
\ \ \ \ \isacommand{using}\isamarkupfalse%
\ zH\ yzinM\ assms\ \isanewline
\ \ \ \ \isacommand{by}\isamarkupfalse%
\ auto\isanewline
\isacommand{qed}\isamarkupfalse%
%
\endisatagproof
{\isafoldproof}%
%
\isadelimproof
\isanewline
%
\endisadelimproof
\isanewline
\isacommand{lemma}\isamarkupfalse%
\ domain{\isacharunderscore}{\kern0pt}elem{\isacharunderscore}{\kern0pt}in{\isacharunderscore}{\kern0pt}eclose\ {\isacharcolon}{\kern0pt}\ \isanewline
\ \ \isakeyword{fixes}\ x\ y\ \isanewline
\ \ \isakeyword{assumes}\ {\isachardoublequoteopen}y\ {\isasymin}\ domain{\isacharparenleft}{\kern0pt}x{\isacharparenright}{\kern0pt}{\isachardoublequoteclose}\ \isanewline
\ \ \isakeyword{shows}\ {\isachardoublequoteopen}y\ {\isasymin}\ eclose{\isacharparenleft}{\kern0pt}x{\isacharparenright}{\kern0pt}{\isachardoublequoteclose}\isanewline
%
\isadelimproof
%
\endisadelimproof
%
\isatagproof
\isacommand{proof}\isamarkupfalse%
\ {\isacharminus}{\kern0pt}\ \isanewline
\ \ \isacommand{obtain}\isamarkupfalse%
\ z\ \isakeyword{where}\ zH\ {\isacharcolon}{\kern0pt}\ {\isachardoublequoteopen}{\isacharless}{\kern0pt}y{\isacharcomma}{\kern0pt}\ z{\isachargreater}{\kern0pt}\ {\isasymin}\ x{\isachardoublequoteclose}\ \isacommand{using}\isamarkupfalse%
\ assms\ \isacommand{by}\isamarkupfalse%
\ auto\ \isanewline
\ \ \isacommand{show}\isamarkupfalse%
\ {\isacharquery}{\kern0pt}thesis\ \isanewline
\ \ \ \ \isacommand{apply}\isamarkupfalse%
{\isacharparenleft}{\kern0pt}subst\ eclose{\isacharunderscore}{\kern0pt}eq{\isacharunderscore}{\kern0pt}Union{\isacharcomma}{\kern0pt}\ simp{\isacharcomma}{\kern0pt}\ rule{\isacharunderscore}{\kern0pt}tac\ x{\isacharequal}{\kern0pt}{\isadigit{2}}\ \isakeyword{in}\ bexI{\isacharcomma}{\kern0pt}\ simp{\isacharparenright}{\kern0pt}\isanewline
\ \ \ \ \ \isacommand{apply}\isamarkupfalse%
{\isacharparenleft}{\kern0pt}rule{\isacharunderscore}{\kern0pt}tac\ x{\isacharequal}{\kern0pt}{\isachardoublequoteopen}{\isacharless}{\kern0pt}y{\isacharcomma}{\kern0pt}\ z{\isachargreater}{\kern0pt}{\isachardoublequoteclose}\ \isakeyword{in}\ bexI{\isacharparenright}{\kern0pt}\isanewline
\ \ \ \ \ \ \isacommand{apply}\isamarkupfalse%
{\isacharparenleft}{\kern0pt}rule{\isacharunderscore}{\kern0pt}tac\ x{\isacharequal}{\kern0pt}{\isachardoublequoteopen}{\isacharbraceleft}{\kern0pt}y{\isacharbraceright}{\kern0pt}{\isachardoublequoteclose}\ \isakeyword{in}\ bexI{\isacharcomma}{\kern0pt}\ simp{\isacharparenright}{\kern0pt}\isanewline
\ \ \ \ \ \ \isacommand{apply}\isamarkupfalse%
{\isacharparenleft}{\kern0pt}subst\ Pair{\isacharunderscore}{\kern0pt}def{\isacharcomma}{\kern0pt}\ simp{\isacharparenright}{\kern0pt}\isanewline
\ \ \ \ \isacommand{using}\isamarkupfalse%
\ zH\ \isanewline
\ \ \ \ \isacommand{by}\isamarkupfalse%
\ auto\isanewline
\isacommand{qed}\isamarkupfalse%
%
\endisatagproof
{\isafoldproof}%
%
\isadelimproof
\isanewline
%
\endisadelimproof
\isanewline
\isacommand{end}\isamarkupfalse%
\isanewline
%
\isadelimtheory
%
\endisadelimtheory
%
\isatagtheory
\isacommand{end}\isamarkupfalse%
%
\endisatagtheory
{\isafoldtheory}%
%
\isadelimtheory
%
\endisadelimtheory
%
\end{isabellebody}%
\endinput
%:%file=~/source/repos/ZF-notAC/code/RecFun_M_Memrel.thy%:%
%:%10=1%:%
%:%11=1%:%
%:%12=2%:%
%:%13=3%:%
%:%14=4%:%
%:%15=5%:%
%:%20=5%:%
%:%23=6%:%
%:%24=7%:%
%:%25=7%:%
%:%26=8%:%
%:%27=8%:%
%:%28=9%:%
%:%29=10%:%
%:%30=10%:%
%:%31=11%:%
%:%32=12%:%
%:%33=13%:%
%:%34=13%:%
%:%35=14%:%
%:%38=15%:%
%:%42=15%:%
%:%43=15%:%
%:%44=16%:%
%:%45=16%:%
%:%46=17%:%
%:%47=17%:%
%:%48=18%:%
%:%49=18%:%
%:%50=19%:%
%:%51=19%:%
%:%52=20%:%
%:%53=20%:%
%:%54=21%:%
%:%55=21%:%
%:%56=22%:%
%:%57=22%:%
%:%58=23%:%
%:%59=23%:%
%:%64=23%:%
%:%67=24%:%
%:%68=25%:%
%:%69=25%:%
%:%76=26%:%
%:%77=26%:%
%:%78=27%:%
%:%79=27%:%
%:%80=28%:%
%:%81=28%:%
%:%82=29%:%
%:%83=29%:%
%:%84=29%:%
%:%85=30%:%
%:%86=30%:%
%:%87=30%:%
%:%88=30%:%
%:%89=31%:%
%:%90=31%:%
%:%91=31%:%
%:%92=31%:%
%:%93=32%:%
%:%94=32%:%
%:%95=32%:%
%:%96=32%:%
%:%97=32%:%
%:%98=33%:%
%:%99=33%:%
%:%100=33%:%
%:%101=33%:%
%:%102=33%:%
%:%103=34%:%
%:%104=34%:%
%:%105=35%:%
%:%106=35%:%
%:%107=35%:%
%:%108=36%:%
%:%109=36%:%
%:%110=36%:%
%:%111=36%:%
%:%112=36%:%
%:%113=37%:%
%:%114=37%:%
%:%115=37%:%
%:%116=37%:%
%:%117=37%:%
%:%118=38%:%
%:%119=38%:%
%:%120=39%:%
%:%121=39%:%
%:%122=39%:%
%:%123=40%:%
%:%124=40%:%
%:%125=40%:%
%:%126=41%:%
%:%127=41%:%
%:%128=42%:%
%:%129=42%:%
%:%130=43%:%
%:%131=43%:%
%:%132=44%:%
%:%138=44%:%
%:%141=45%:%
%:%142=46%:%
%:%143=46%:%
%:%144=47%:%
%:%151=48%:%
%:%152=48%:%
%:%153=49%:%
%:%154=49%:%
%:%155=49%:%
%:%156=50%:%
%:%157=50%:%
%:%158=50%:%
%:%159=50%:%
%:%160=50%:%
%:%161=51%:%
%:%162=51%:%
%:%163=51%:%
%:%164=51%:%
%:%165=51%:%
%:%166=52%:%
%:%167=52%:%
%:%168=52%:%
%:%169=52%:%
%:%170=53%:%
%:%171=54%:%
%:%172=54%:%
%:%173=55%:%
%:%174=55%:%
%:%175=56%:%
%:%176=56%:%
%:%177=56%:%
%:%178=57%:%
%:%179=57%:%
%:%180=58%:%
%:%181=58%:%
%:%182=59%:%
%:%183=59%:%
%:%184=59%:%
%:%185=60%:%
%:%186=60%:%
%:%187=60%:%
%:%188=60%:%
%:%189=61%:%
%:%190=61%:%
%:%191=61%:%
%:%192=61%:%
%:%193=61%:%
%:%194=62%:%
%:%195=63%:%
%:%196=63%:%
%:%197=64%:%
%:%198=64%:%
%:%199=65%:%
%:%200=65%:%
%:%201=66%:%
%:%202=66%:%
%:%203=67%:%
%:%204=67%:%
%:%205=68%:%
%:%206=68%:%
%:%207=69%:%
%:%208=69%:%
%:%209=70%:%
%:%210=70%:%
%:%211=71%:%
%:%212=71%:%
%:%213=72%:%
%:%214=72%:%
%:%215=73%:%
%:%216=73%:%
%:%217=74%:%
%:%218=74%:%
%:%219=74%:%
%:%220=75%:%
%:%221=75%:%
%:%222=75%:%
%:%223=75%:%
%:%224=76%:%
%:%225=76%:%
%:%226=76%:%
%:%227=76%:%
%:%228=77%:%
%:%229=77%:%
%:%230=77%:%
%:%231=77%:%
%:%232=77%:%
%:%233=78%:%
%:%234=78%:%
%:%235=78%:%
%:%236=78%:%
%:%237=79%:%
%:%238=79%:%
%:%239=79%:%
%:%240=79%:%
%:%241=79%:%
%:%242=80%:%
%:%243=80%:%
%:%244=80%:%
%:%245=80%:%
%:%246=81%:%
%:%247=81%:%
%:%248=81%:%
%:%249=81%:%
%:%250=81%:%
%:%251=82%:%
%:%252=82%:%
%:%253=83%:%
%:%254=83%:%
%:%255=84%:%
%:%256=84%:%
%:%257=85%:%
%:%258=85%:%
%:%259=86%:%
%:%260=86%:%
%:%261=87%:%
%:%262=87%:%
%:%263=88%:%
%:%264=88%:%
%:%265=89%:%
%:%266=89%:%
%:%267=90%:%
%:%268=91%:%
%:%269=91%:%
%:%270=91%:%
%:%271=91%:%
%:%272=91%:%
%:%273=92%:%
%:%274=92%:%
%:%275=93%:%
%:%276=93%:%
%:%277=93%:%
%:%278=94%:%
%:%279=94%:%
%:%280=94%:%
%:%281=94%:%
%:%282=94%:%
%:%283=95%:%
%:%284=95%:%
%:%285=95%:%
%:%286=96%:%
%:%287=96%:%
%:%288=97%:%
%:%289=97%:%
%:%290=97%:%
%:%291=97%:%
%:%292=98%:%
%:%293=98%:%
%:%294=99%:%
%:%295=99%:%
%:%296=99%:%
%:%297=100%:%
%:%298=100%:%
%:%299=100%:%
%:%300=100%:%
%:%301=101%:%
%:%302=101%:%
%:%303=101%:%
%:%304=102%:%
%:%305=102%:%
%:%306=103%:%
%:%307=103%:%
%:%308=104%:%
%:%309=104%:%
%:%310=105%:%
%:%311=105%:%
%:%312=106%:%
%:%313=106%:%
%:%314=107%:%
%:%315=108%:%
%:%316=108%:%
%:%317=109%:%
%:%318=109%:%
%:%319=110%:%
%:%320=110%:%
%:%321=111%:%
%:%322=111%:%
%:%323=112%:%
%:%324=112%:%
%:%325=113%:%
%:%326=113%:%
%:%327=114%:%
%:%328=115%:%
%:%329=115%:%
%:%330=116%:%
%:%331=116%:%
%:%332=117%:%
%:%333=118%:%
%:%334=118%:%
%:%335=118%:%
%:%336=118%:%
%:%337=119%:%
%:%338=119%:%
%:%339=119%:%
%:%340=119%:%
%:%341=120%:%
%:%342=120%:%
%:%343=120%:%
%:%344=120%:%
%:%345=121%:%
%:%346=122%:%
%:%347=122%:%
%:%348=122%:%
%:%349=122%:%
%:%350=123%:%
%:%351=123%:%
%:%352=123%:%
%:%353=123%:%
%:%354=123%:%
%:%355=124%:%
%:%356=124%:%
%:%357=124%:%
%:%358=125%:%
%:%359=125%:%
%:%360=126%:%
%:%361=126%:%
%:%362=127%:%
%:%363=127%:%
%:%364=128%:%
%:%365=128%:%
%:%366=129%:%
%:%367=129%:%
%:%368=129%:%
%:%369=129%:%
%:%370=129%:%
%:%371=130%:%
%:%372=130%:%
%:%373=130%:%
%:%374=130%:%
%:%375=131%:%
%:%376=131%:%
%:%377=131%:%
%:%378=132%:%
%:%379=132%:%
%:%380=133%:%
%:%381=133%:%
%:%382=134%:%
%:%383=134%:%
%:%384=135%:%
%:%385=135%:%
%:%386=135%:%
%:%387=136%:%
%:%388=136%:%
%:%389=137%:%
%:%390=137%:%
%:%391=138%:%
%:%392=138%:%
%:%393=139%:%
%:%394=139%:%
%:%395=140%:%
%:%396=140%:%
%:%397=140%:%
%:%398=140%:%
%:%399=140%:%
%:%400=141%:%
%:%406=141%:%
%:%409=142%:%
%:%410=143%:%
%:%411=143%:%
%:%414=144%:%
%:%418=144%:%
%:%419=144%:%
%:%420=145%:%
%:%421=145%:%
%:%422=146%:%
%:%423=146%:%
%:%424=147%:%
%:%425=147%:%
%:%426=148%:%
%:%427=148%:%
%:%428=149%:%
%:%429=149%:%
%:%430=150%:%
%:%431=150%:%
%:%436=150%:%
%:%439=151%:%
%:%440=152%:%
%:%441=152%:%
%:%444=153%:%
%:%448=153%:%
%:%449=153%:%
%:%450=154%:%
%:%451=154%:%
%:%452=155%:%
%:%453=155%:%
%:%454=156%:%
%:%455=156%:%
%:%456=157%:%
%:%457=157%:%
%:%458=158%:%
%:%459=158%:%
%:%460=159%:%
%:%461=159%:%
%:%462=160%:%
%:%463=160%:%
%:%464=161%:%
%:%465=161%:%
%:%466=162%:%
%:%467=162%:%
%:%472=162%:%
%:%475=163%:%
%:%476=164%:%
%:%477=164%:%
%:%480=165%:%
%:%484=165%:%
%:%485=165%:%
%:%486=166%:%
%:%487=166%:%
%:%488=167%:%
%:%489=167%:%
%:%490=168%:%
%:%491=168%:%
%:%492=169%:%
%:%493=169%:%
%:%494=170%:%
%:%495=170%:%
%:%500=170%:%
%:%503=171%:%
%:%504=172%:%
%:%505=172%:%
%:%506=173%:%
%:%507=174%:%
%:%508=174%:%
%:%509=175%:%
%:%510=176%:%
%:%511=176%:%
%:%512=177%:%
%:%513=178%:%
%:%514=179%:%
%:%515=179%:%
%:%516=180%:%
%:%517=181%:%
%:%518=182%:%
%:%521=183%:%
%:%522=184%:%
%:%526=184%:%
%:%527=184%:%
%:%528=185%:%
%:%529=185%:%
%:%530=186%:%
%:%531=186%:%
%:%532=187%:%
%:%533=187%:%
%:%538=187%:%
%:%541=188%:%
%:%542=189%:%
%:%543=189%:%
%:%544=190%:%
%:%545=191%:%
%:%546=192%:%
%:%549=193%:%
%:%550=194%:%
%:%554=194%:%
%:%555=194%:%
%:%556=195%:%
%:%557=195%:%
%:%558=196%:%
%:%559=196%:%
%:%560=197%:%
%:%561=197%:%
%:%562=198%:%
%:%563=198%:%
%:%564=199%:%
%:%565=199%:%
%:%566=200%:%
%:%567=200%:%
%:%568=201%:%
%:%569=201%:%
%:%570=202%:%
%:%571=202%:%
%:%572=203%:%
%:%573=203%:%
%:%578=203%:%
%:%581=204%:%
%:%582=205%:%
%:%583=205%:%
%:%584=206%:%
%:%585=207%:%
%:%586=208%:%
%:%587=209%:%
%:%588=210%:%
%:%589=211%:%
%:%590=212%:%
%:%591=213%:%
%:%592=214%:%
%:%593=215%:%
%:%594=216%:%
%:%595=217%:%
%:%602=218%:%
%:%603=218%:%
%:%604=219%:%
%:%605=220%:%
%:%606=220%:%
%:%607=221%:%
%:%608=221%:%
%:%609=222%:%
%:%610=222%:%
%:%611=222%:%
%:%612=223%:%
%:%613=223%:%
%:%614=223%:%
%:%615=223%:%
%:%616=224%:%
%:%617=224%:%
%:%618=224%:%
%:%619=224%:%
%:%620=224%:%
%:%621=225%:%
%:%622=225%:%
%:%623=225%:%
%:%624=225%:%
%:%625=225%:%
%:%626=226%:%
%:%627=226%:%
%:%628=226%:%
%:%629=226%:%
%:%630=226%:%
%:%631=227%:%
%:%632=227%:%
%:%633=228%:%
%:%634=228%:%
%:%635=228%:%
%:%636=229%:%
%:%637=229%:%
%:%638=229%:%
%:%639=229%:%
%:%640=230%:%
%:%641=230%:%
%:%642=230%:%
%:%643=231%:%
%:%644=231%:%
%:%645=232%:%
%:%646=232%:%
%:%647=233%:%
%:%648=233%:%
%:%649=234%:%
%:%650=234%:%
%:%651=234%:%
%:%652=235%:%
%:%653=235%:%
%:%654=236%:%
%:%655=236%:%
%:%656=237%:%
%:%657=237%:%
%:%658=238%:%
%:%659=238%:%
%:%660=239%:%
%:%661=239%:%
%:%662=240%:%
%:%663=240%:%
%:%664=241%:%
%:%665=241%:%
%:%666=242%:%
%:%667=242%:%
%:%668=243%:%
%:%669=243%:%
%:%670=243%:%
%:%671=244%:%
%:%672=244%:%
%:%673=244%:%
%:%674=245%:%
%:%675=245%:%
%:%676=246%:%
%:%677=247%:%
%:%678=247%:%
%:%679=248%:%
%:%680=248%:%
%:%681=249%:%
%:%682=250%:%
%:%683=251%:%
%:%684=252%:%
%:%685=252%:%
%:%686=253%:%
%:%687=253%:%
%:%688=254%:%
%:%689=254%:%
%:%690=255%:%
%:%691=255%:%
%:%692=256%:%
%:%693=256%:%
%:%694=257%:%
%:%695=257%:%
%:%696=258%:%
%:%697=258%:%
%:%698=259%:%
%:%699=259%:%
%:%700=260%:%
%:%701=260%:%
%:%702=261%:%
%:%703=261%:%
%:%704=262%:%
%:%705=262%:%
%:%706=263%:%
%:%707=263%:%
%:%708=264%:%
%:%709=264%:%
%:%710=265%:%
%:%711=265%:%
%:%712=266%:%
%:%713=266%:%
%:%714=267%:%
%:%715=267%:%
%:%716=268%:%
%:%717=268%:%
%:%718=269%:%
%:%719=269%:%
%:%720=270%:%
%:%721=270%:%
%:%722=271%:%
%:%723=271%:%
%:%724=272%:%
%:%725=273%:%
%:%726=273%:%
%:%727=273%:%
%:%728=273%:%
%:%729=273%:%
%:%730=273%:%
%:%731=274%:%
%:%737=274%:%
%:%740=275%:%
%:%741=276%:%
%:%742=276%:%
%:%743=277%:%
%:%744=278%:%
%:%745=279%:%
%:%746=280%:%
%:%747=281%:%
%:%748=282%:%
%:%749=283%:%
%:%750=284%:%
%:%751=285%:%
%:%752=286%:%
%:%755=287%:%
%:%756=288%:%
%:%760=288%:%
%:%761=288%:%
%:%762=289%:%
%:%763=289%:%
%:%764=290%:%
%:%765=290%:%
%:%766=291%:%
%:%767=291%:%
%:%768=292%:%
%:%769=292%:%
%:%770=293%:%
%:%771=293%:%
%:%772=294%:%
%:%773=294%:%
%:%774=295%:%
%:%775=295%:%
%:%776=296%:%
%:%777=296%:%
%:%778=297%:%
%:%779=297%:%
%:%780=298%:%
%:%781=298%:%
%:%782=299%:%
%:%783=299%:%
%:%784=300%:%
%:%785=300%:%
%:%786=301%:%
%:%787=301%:%
%:%788=302%:%
%:%789=302%:%
%:%790=303%:%
%:%791=303%:%
%:%792=304%:%
%:%793=304%:%
%:%794=305%:%
%:%795=305%:%
%:%796=306%:%
%:%797=306%:%
%:%798=307%:%
%:%799=307%:%
%:%800=308%:%
%:%801=308%:%
%:%802=309%:%
%:%803=309%:%
%:%804=310%:%
%:%805=310%:%
%:%806=311%:%
%:%807=311%:%
%:%808=312%:%
%:%809=312%:%
%:%810=313%:%
%:%816=313%:%
%:%819=314%:%
%:%820=315%:%
%:%821=315%:%
%:%822=316%:%
%:%823=317%:%
%:%824=318%:%
%:%831=319%:%
%:%832=319%:%
%:%833=320%:%
%:%834=320%:%
%:%835=320%:%
%:%836=320%:%
%:%837=321%:%
%:%838=321%:%
%:%839=321%:%
%:%840=321%:%
%:%841=322%:%
%:%842=322%:%
%:%843=322%:%
%:%844=322%:%
%:%845=323%:%
%:%846=323%:%
%:%847=323%:%
%:%848=323%:%
%:%849=324%:%
%:%850=324%:%
%:%851=324%:%
%:%852=324%:%
%:%853=324%:%
%:%854=325%:%
%:%855=325%:%
%:%856=326%:%
%:%857=326%:%
%:%858=327%:%
%:%859=327%:%
%:%860=328%:%
%:%861=328%:%
%:%862=329%:%
%:%863=329%:%
%:%864=330%:%
%:%865=330%:%
%:%866=331%:%
%:%867=331%:%
%:%868=332%:%
%:%869=332%:%
%:%870=333%:%
%:%871=333%:%
%:%872=334%:%
%:%873=334%:%
%:%874=335%:%
%:%875=335%:%
%:%876=336%:%
%:%877=336%:%
%:%878=337%:%
%:%884=337%:%
%:%887=338%:%
%:%888=339%:%
%:%889=339%:%
%:%890=340%:%
%:%891=341%:%
%:%892=342%:%
%:%899=343%:%
%:%900=343%:%
%:%901=344%:%
%:%902=344%:%
%:%903=344%:%
%:%904=344%:%
%:%905=345%:%
%:%906=345%:%
%:%907=346%:%
%:%908=346%:%
%:%909=347%:%
%:%910=347%:%
%:%911=348%:%
%:%912=348%:%
%:%913=349%:%
%:%914=349%:%
%:%915=350%:%
%:%916=350%:%
%:%917=351%:%
%:%918=351%:%
%:%919=352%:%
%:%925=352%:%
%:%928=353%:%
%:%929=354%:%
%:%930=354%:%
%:%937=355%:%

%
\begin{isabellebody}%
\setisabellecontext{P{\isacharunderscore}{\kern0pt}Names}%
%
\isadelimtheory
%
\endisadelimtheory
%
\isatagtheory
\isacommand{theory}\isamarkupfalse%
\ P{\isacharunderscore}{\kern0pt}Names\isanewline
\ \ \isakeyword{imports}\ \isanewline
\ \ \ \ {\isachardoublequoteopen}Forcing{\isacharslash}{\kern0pt}Forcing{\isacharunderscore}{\kern0pt}Main{\isachardoublequoteclose}\isanewline
\ \ \ \ Utilities{\isacharunderscore}{\kern0pt}M\isanewline
\isakeyword{begin}%
\endisatagtheory
{\isafoldtheory}%
%
\isadelimtheory
\ \isanewline
%
\endisadelimtheory
\isanewline
\isacommand{context}\isamarkupfalse%
\ forcing{\isacharunderscore}{\kern0pt}data\isanewline
\isakeyword{begin}\ \isanewline
\isanewline
\isacommand{definition}\isamarkupfalse%
\ HP{\isacharunderscore}{\kern0pt}set{\isacharunderscore}{\kern0pt}succ\ {\isacharcolon}{\kern0pt}{\isacharcolon}{\kern0pt}\ {\isachardoublequoteopen}{\isacharbrackleft}{\kern0pt}i{\isacharcomma}{\kern0pt}\ i{\isacharbrackright}{\kern0pt}\ {\isasymRightarrow}\ i{\isachardoublequoteclose}\ \isakeyword{where}\ \isanewline
\ \ {\isachardoublequoteopen}HP{\isacharunderscore}{\kern0pt}set{\isacharunderscore}{\kern0pt}succ{\isacharparenleft}{\kern0pt}a{\isacharcomma}{\kern0pt}\ X{\isacharparenright}{\kern0pt}\ {\isasymequiv}\ Pow{\isacharparenleft}{\kern0pt}X\ {\isasymtimes}\ P{\isacharparenright}{\kern0pt}\ {\isasyminter}\ M{\isachardoublequoteclose}\ \isanewline
\isanewline
\isacommand{definition}\isamarkupfalse%
\ P{\isacharunderscore}{\kern0pt}set\ {\isacharcolon}{\kern0pt}{\isacharcolon}{\kern0pt}\ {\isachardoublequoteopen}i\ {\isasymRightarrow}\ i{\isachardoublequoteclose}\ \isakeyword{where}\ \isanewline
\ \ {\isachardoublequoteopen}P{\isacharunderscore}{\kern0pt}set{\isacharparenleft}{\kern0pt}a{\isacharparenright}{\kern0pt}\ {\isasymequiv}\ transrec{\isadigit{2}}{\isacharparenleft}{\kern0pt}a{\isacharcomma}{\kern0pt}\ {\isadigit{0}}{\isacharcomma}{\kern0pt}\ HP{\isacharunderscore}{\kern0pt}set{\isacharunderscore}{\kern0pt}succ{\isacharparenright}{\kern0pt}{\isachardoublequoteclose}\ \isanewline
\isanewline
\isacommand{lemma}\isamarkupfalse%
\ P{\isacharunderscore}{\kern0pt}set{\isacharunderscore}{\kern0pt}{\isadigit{0}}\ {\isacharcolon}{\kern0pt}\ {\isachardoublequoteopen}P{\isacharunderscore}{\kern0pt}set{\isacharparenleft}{\kern0pt}{\isadigit{0}}{\isacharparenright}{\kern0pt}\ {\isacharequal}{\kern0pt}\ {\isadigit{0}}{\isachardoublequoteclose}%
\isadelimproof
\ %
\endisadelimproof
%
\isatagproof
\isacommand{unfolding}\isamarkupfalse%
\ P{\isacharunderscore}{\kern0pt}set{\isacharunderscore}{\kern0pt}def\ \isacommand{using}\isamarkupfalse%
\ transrec{\isadigit{2}}{\isacharunderscore}{\kern0pt}{\isadigit{0}}\ \isacommand{by}\isamarkupfalse%
\ auto%
\endisatagproof
{\isafoldproof}%
%
\isadelimproof
%
\endisadelimproof
\isanewline
\isanewline
\isacommand{lemma}\isamarkupfalse%
\ P{\isacharunderscore}{\kern0pt}set{\isacharunderscore}{\kern0pt}succ\ {\isacharcolon}{\kern0pt}\ \isanewline
\ \ {\isachardoublequoteopen}P{\isacharunderscore}{\kern0pt}set{\isacharparenleft}{\kern0pt}succ{\isacharparenleft}{\kern0pt}a{\isacharparenright}{\kern0pt}{\isacharparenright}{\kern0pt}\ {\isacharequal}{\kern0pt}\ Pow{\isacharparenleft}{\kern0pt}P{\isacharunderscore}{\kern0pt}set{\isacharparenleft}{\kern0pt}a{\isacharparenright}{\kern0pt}\ {\isasymtimes}\ P{\isacharparenright}{\kern0pt}\ {\isasyminter}\ M{\isachardoublequoteclose}\ \isanewline
%
\isadelimproof
\ \ %
\endisadelimproof
%
\isatagproof
\isacommand{using}\isamarkupfalse%
\ transrec{\isadigit{2}}{\isacharunderscore}{\kern0pt}succ\isanewline
\isacommand{proof}\isamarkupfalse%
\ {\isacharminus}{\kern0pt}\ \isanewline
\ \ \isacommand{have}\isamarkupfalse%
\ {\isachardoublequoteopen}P{\isacharunderscore}{\kern0pt}set{\isacharparenleft}{\kern0pt}succ{\isacharparenleft}{\kern0pt}a{\isacharparenright}{\kern0pt}{\isacharparenright}{\kern0pt}\ {\isacharequal}{\kern0pt}\ HP{\isacharunderscore}{\kern0pt}set{\isacharunderscore}{\kern0pt}succ{\isacharparenleft}{\kern0pt}a{\isacharcomma}{\kern0pt}\ P{\isacharunderscore}{\kern0pt}set{\isacharparenleft}{\kern0pt}a{\isacharparenright}{\kern0pt}{\isacharparenright}{\kern0pt}{\isachardoublequoteclose}\ \isacommand{unfolding}\isamarkupfalse%
\ P{\isacharunderscore}{\kern0pt}set{\isacharunderscore}{\kern0pt}def\ \isacommand{using}\isamarkupfalse%
\ transrec{\isadigit{2}}{\isacharunderscore}{\kern0pt}succ\ \isacommand{by}\isamarkupfalse%
\ auto\isanewline
\ \ \isacommand{then}\isamarkupfalse%
\ \isacommand{show}\isamarkupfalse%
\ {\isacharquery}{\kern0pt}thesis\ \isacommand{unfolding}\isamarkupfalse%
\ HP{\isacharunderscore}{\kern0pt}set{\isacharunderscore}{\kern0pt}succ{\isacharunderscore}{\kern0pt}def\ \isacommand{by}\isamarkupfalse%
\ auto\isanewline
\isacommand{qed}\isamarkupfalse%
%
\endisatagproof
{\isafoldproof}%
%
\isadelimproof
\isanewline
%
\endisadelimproof
\ \ \isanewline
\isacommand{lemma}\isamarkupfalse%
\ P{\isacharunderscore}{\kern0pt}set{\isacharunderscore}{\kern0pt}lim\ {\isacharcolon}{\kern0pt}\ \isanewline
\ \ {\isachardoublequoteopen}Limit{\isacharparenleft}{\kern0pt}a{\isacharparenright}{\kern0pt}\ {\isasymLongrightarrow}\ P{\isacharunderscore}{\kern0pt}set{\isacharparenleft}{\kern0pt}a{\isacharparenright}{\kern0pt}\ {\isacharequal}{\kern0pt}\ {\isacharparenleft}{\kern0pt}{\isasymUnion}\ b\ {\isacharless}{\kern0pt}\ a{\isachardot}{\kern0pt}\ P{\isacharunderscore}{\kern0pt}set{\isacharparenleft}{\kern0pt}b{\isacharparenright}{\kern0pt}{\isacharparenright}{\kern0pt}{\isachardoublequoteclose}\ \isanewline
%
\isadelimproof
\ \ %
\endisadelimproof
%
\isatagproof
\isacommand{using}\isamarkupfalse%
\ transrec{\isadigit{2}}{\isacharunderscore}{\kern0pt}Limit\ \isacommand{unfolding}\isamarkupfalse%
\ P{\isacharunderscore}{\kern0pt}set{\isacharunderscore}{\kern0pt}def\ \isacommand{by}\isamarkupfalse%
\ auto%
\endisatagproof
{\isafoldproof}%
%
\isadelimproof
\ \isanewline
%
\endisadelimproof
\ \ \ \ \isanewline
\isacommand{definition}\isamarkupfalse%
\ P{\isacharunderscore}{\kern0pt}names\ {\isacharcolon}{\kern0pt}{\isacharcolon}{\kern0pt}\ {\isachardoublequoteopen}i{\isachardoublequoteclose}\ \isakeyword{where}\ {\isachardoublequoteopen}P{\isacharunderscore}{\kern0pt}names\ {\isasymequiv}\ {\isacharbraceleft}{\kern0pt}\ x\ {\isasymin}\ M\ {\isachardot}{\kern0pt}\ {\isasymexists}\ a{\isachardot}{\kern0pt}\ Ord{\isacharparenleft}{\kern0pt}a{\isacharparenright}{\kern0pt}\ {\isasymand}\ x\ {\isasymin}\ P{\isacharunderscore}{\kern0pt}set{\isacharparenleft}{\kern0pt}a{\isacharparenright}{\kern0pt}\ {\isacharbraceright}{\kern0pt}{\isachardoublequoteclose}\ \isanewline
\isanewline
\isacommand{definition}\isamarkupfalse%
\ P{\isacharunderscore}{\kern0pt}rank\ {\isacharcolon}{\kern0pt}{\isacharcolon}{\kern0pt}\ {\isachardoublequoteopen}i\ {\isasymRightarrow}\ i{\isachardoublequoteclose}\ \isakeyword{where}\ \isanewline
\ \ {\isachardoublequoteopen}P{\isacharunderscore}{\kern0pt}rank{\isacharparenleft}{\kern0pt}x{\isacharparenright}{\kern0pt}\ {\isasymequiv}\ {\isasymmu}\ a{\isachardot}{\kern0pt}\ Ord{\isacharparenleft}{\kern0pt}a{\isacharparenright}{\kern0pt}\ {\isasymand}\ x\ {\isasymin}\ P{\isacharunderscore}{\kern0pt}set{\isacharparenleft}{\kern0pt}succ{\isacharparenleft}{\kern0pt}a{\isacharparenright}{\kern0pt}{\isacharparenright}{\kern0pt}{\isachardoublequoteclose}\ \isanewline
\isanewline
\isacommand{lemma}\isamarkupfalse%
\ P{\isacharunderscore}{\kern0pt}rank{\isacharunderscore}{\kern0pt}ord\ {\isacharcolon}{\kern0pt}\ {\isachardoublequoteopen}Ord{\isacharparenleft}{\kern0pt}P{\isacharunderscore}{\kern0pt}rank{\isacharparenleft}{\kern0pt}x{\isacharparenright}{\kern0pt}{\isacharparenright}{\kern0pt}{\isachardoublequoteclose}\ \isanewline
%
\isadelimproof
\ \ %
\endisadelimproof
%
\isatagproof
\isacommand{unfolding}\isamarkupfalse%
\ P{\isacharunderscore}{\kern0pt}rank{\isacharunderscore}{\kern0pt}def\ \isacommand{by}\isamarkupfalse%
\ auto%
\endisatagproof
{\isafoldproof}%
%
\isadelimproof
\ \isanewline
%
\endisadelimproof
\isanewline
\isacommand{lemma}\isamarkupfalse%
\ P{\isacharunderscore}{\kern0pt}set{\isacharunderscore}{\kern0pt}elems{\isacharunderscore}{\kern0pt}in{\isacharunderscore}{\kern0pt}P{\isacharunderscore}{\kern0pt}set{\isacharunderscore}{\kern0pt}succ\ {\isacharcolon}{\kern0pt}\ \isanewline
\ \ {\isachardoublequoteopen}Ord{\isacharparenleft}{\kern0pt}a{\isacharparenright}{\kern0pt}\ {\isasymLongrightarrow}\ x\ {\isasymin}\ P{\isacharunderscore}{\kern0pt}set{\isacharparenleft}{\kern0pt}a{\isacharparenright}{\kern0pt}\ {\isasymLongrightarrow}\ {\isacharparenleft}{\kern0pt}{\isasymexists}b\ {\isacharless}{\kern0pt}\ a{\isachardot}{\kern0pt}\ x\ {\isasymin}\ P{\isacharunderscore}{\kern0pt}set{\isacharparenleft}{\kern0pt}succ{\isacharparenleft}{\kern0pt}b{\isacharparenright}{\kern0pt}{\isacharparenright}{\kern0pt}{\isacharparenright}{\kern0pt}{\isachardoublequoteclose}\ \isanewline
%
\isadelimproof
%
\endisadelimproof
%
\isatagproof
\isacommand{proof}\isamarkupfalse%
\ {\isacharminus}{\kern0pt}\ \isanewline
\ \ \isacommand{assume}\isamarkupfalse%
\ assms\ {\isacharcolon}{\kern0pt}\ {\isachardoublequoteopen}Ord{\isacharparenleft}{\kern0pt}a{\isacharparenright}{\kern0pt}{\isachardoublequoteclose}\ {\isachardoublequoteopen}x\ {\isasymin}\ P{\isacharunderscore}{\kern0pt}set{\isacharparenleft}{\kern0pt}a{\isacharparenright}{\kern0pt}{\isachardoublequoteclose}\isanewline
\ \ \isacommand{then}\isamarkupfalse%
\ \isacommand{have}\isamarkupfalse%
\ helper\ {\isacharcolon}{\kern0pt}\ {\isachardoublequoteopen}x\ {\isasymin}\ P{\isacharunderscore}{\kern0pt}set{\isacharparenleft}{\kern0pt}a{\isacharparenright}{\kern0pt}\ {\isasymlongrightarrow}\ {\isacharparenleft}{\kern0pt}{\isasymexists}b\ {\isacharless}{\kern0pt}\ a{\isachardot}{\kern0pt}\ x\ {\isasymin}\ P{\isacharunderscore}{\kern0pt}set{\isacharparenleft}{\kern0pt}succ{\isacharparenleft}{\kern0pt}b{\isacharparenright}{\kern0pt}{\isacharparenright}{\kern0pt}{\isacharparenright}{\kern0pt}{\isachardoublequoteclose}\ \ \isanewline
\ \ \ \ \isacommand{apply}\isamarkupfalse%
\ {\isacharparenleft}{\kern0pt}rule{\isacharunderscore}{\kern0pt}tac\ P{\isacharequal}{\kern0pt}{\isachardoublequoteopen}{\isasymlambda}a{\isachardot}{\kern0pt}\ x\ {\isasymin}\ P{\isacharunderscore}{\kern0pt}set{\isacharparenleft}{\kern0pt}a{\isacharparenright}{\kern0pt}\ {\isasymlongrightarrow}\ {\isacharparenleft}{\kern0pt}{\isasymexists}b\ {\isacharless}{\kern0pt}\ a{\isachardot}{\kern0pt}\ x\ {\isasymin}\ P{\isacharunderscore}{\kern0pt}set{\isacharparenleft}{\kern0pt}succ{\isacharparenleft}{\kern0pt}b{\isacharparenright}{\kern0pt}{\isacharparenright}{\kern0pt}{\isacharparenright}{\kern0pt}{\isachardoublequoteclose}\ \isanewline
\ \ \ \ \ \ \ \ \ \ \isakeyword{in}\ trans{\isacharunderscore}{\kern0pt}induct{\isadigit{3}}{\isacharunderscore}{\kern0pt}raw{\isacharparenright}{\kern0pt}\isanewline
\ \ \ \ \isacommand{apply}\isamarkupfalse%
\ simp\ \isanewline
\ \ \ \ \isacommand{using}\isamarkupfalse%
\ P{\isacharunderscore}{\kern0pt}set{\isacharunderscore}{\kern0pt}{\isadigit{0}}\ \isacommand{apply}\isamarkupfalse%
\ simp\ \isanewline
\ \ \ \ \isacommand{apply}\isamarkupfalse%
\ auto\ \isanewline
\ \ \isacommand{proof}\isamarkupfalse%
\ {\isacharminus}{\kern0pt}\ \isanewline
\ \ \ \ \isacommand{fix}\isamarkupfalse%
\ a\ \isacommand{assume}\isamarkupfalse%
\ \isanewline
\ \ \ \ assms\ {\isacharcolon}{\kern0pt}\ {\isachardoublequoteopen}Limit{\isacharparenleft}{\kern0pt}a{\isacharparenright}{\kern0pt}{\isachardoublequoteclose}\isanewline
\ \ \ \ \ \ \ \ \ \ \ \ {\isachardoublequoteopen}x\ {\isasymin}\ P{\isacharunderscore}{\kern0pt}set{\isacharparenleft}{\kern0pt}a{\isacharparenright}{\kern0pt}{\isachardoublequoteclose}\isanewline
\ \ \ \ \ \ \ \ \ \ \ \ {\isachardoublequoteopen}{\isasymforall}y{\isasymin}a{\isachardot}{\kern0pt}\ x\ {\isasymin}\ P{\isacharunderscore}{\kern0pt}set{\isacharparenleft}{\kern0pt}y{\isacharparenright}{\kern0pt}\ {\isasymlongrightarrow}\ {\isacharparenleft}{\kern0pt}{\isasymexists}b{\isacharless}{\kern0pt}y{\isachardot}{\kern0pt}\ x\ {\isasymin}\ P{\isacharunderscore}{\kern0pt}set{\isacharparenleft}{\kern0pt}succ{\isacharparenleft}{\kern0pt}b{\isacharparenright}{\kern0pt}{\isacharparenright}{\kern0pt}{\isacharparenright}{\kern0pt}{\isachardoublequoteclose}\ \isanewline
\ \ \ \ \isacommand{then}\isamarkupfalse%
\ \isacommand{have}\isamarkupfalse%
\ {\isachardoublequoteopen}x\ {\isasymin}\ {\isacharparenleft}{\kern0pt}{\isasymUnion}\ b\ {\isacharless}{\kern0pt}\ a\ {\isachardot}{\kern0pt}\ P{\isacharunderscore}{\kern0pt}set{\isacharparenleft}{\kern0pt}b{\isacharparenright}{\kern0pt}{\isacharparenright}{\kern0pt}{\isachardoublequoteclose}\ \isacommand{using}\isamarkupfalse%
\ P{\isacharunderscore}{\kern0pt}set{\isacharunderscore}{\kern0pt}lim\ assms\ \isacommand{by}\isamarkupfalse%
\ auto\ \isanewline
\ \ \ \ \isacommand{then}\isamarkupfalse%
\ \isacommand{obtain}\isamarkupfalse%
\ b\ \isakeyword{where}\ p{\isadigit{1}}\ {\isacharcolon}{\kern0pt}\ {\isachardoublequoteopen}b\ {\isacharless}{\kern0pt}\ a{\isachardoublequoteclose}\ {\isachardoublequoteopen}x\ {\isasymin}\ P{\isacharunderscore}{\kern0pt}set{\isacharparenleft}{\kern0pt}b{\isacharparenright}{\kern0pt}{\isachardoublequoteclose}\ \isacommand{by}\isamarkupfalse%
\ auto\ \isanewline
\ \ \ \ \isacommand{then}\isamarkupfalse%
\ \isacommand{have}\isamarkupfalse%
\ {\isachardoublequoteopen}b\ {\isasymin}\ a{\isachardoublequoteclose}\ \isacommand{using}\isamarkupfalse%
\ ltD\ p{\isadigit{1}}\ \isacommand{by}\isamarkupfalse%
\ auto\ \isanewline
\ \ \ \ \isacommand{then}\isamarkupfalse%
\ \isacommand{obtain}\isamarkupfalse%
\ c\ \isakeyword{where}\ p{\isadigit{2}}\ {\isacharcolon}{\kern0pt}\ {\isachardoublequoteopen}c\ {\isacharless}{\kern0pt}\ b{\isachardoublequoteclose}\ {\isachardoublequoteopen}x\ {\isasymin}\ P{\isacharunderscore}{\kern0pt}set{\isacharparenleft}{\kern0pt}succ{\isacharparenleft}{\kern0pt}c{\isacharparenright}{\kern0pt}{\isacharparenright}{\kern0pt}{\isachardoublequoteclose}\ \isacommand{using}\isamarkupfalse%
\ assms\ p{\isadigit{1}}\ \isacommand{by}\isamarkupfalse%
\ auto\isanewline
\ \ \ \ \isacommand{then}\isamarkupfalse%
\ \isacommand{have}\isamarkupfalse%
\ {\isachardoublequoteopen}c\ {\isacharless}{\kern0pt}\ a{\isachardoublequoteclose}\ \isacommand{using}\isamarkupfalse%
\ p{\isadigit{1}}\ lt{\isacharunderscore}{\kern0pt}trans\ \isacommand{by}\isamarkupfalse%
\ auto\ \isanewline
\ \ \ \ \isacommand{then}\isamarkupfalse%
\ \isacommand{show}\isamarkupfalse%
\ {\isachardoublequoteopen}{\isacharparenleft}{\kern0pt}{\isasymexists}c{\isacharless}{\kern0pt}a{\isachardot}{\kern0pt}\ x\ {\isasymin}\ P{\isacharunderscore}{\kern0pt}set{\isacharparenleft}{\kern0pt}succ{\isacharparenleft}{\kern0pt}c{\isacharparenright}{\kern0pt}{\isacharparenright}{\kern0pt}{\isacharparenright}{\kern0pt}{\isachardoublequoteclose}\ \isacommand{using}\isamarkupfalse%
\ p{\isadigit{2}}\ \isacommand{by}\isamarkupfalse%
\ auto\ \isanewline
\ \ \isacommand{qed}\isamarkupfalse%
\isanewline
\ \ \isacommand{then}\isamarkupfalse%
\ \isacommand{show}\isamarkupfalse%
\ {\isacharquery}{\kern0pt}thesis\ \isacommand{using}\isamarkupfalse%
\ assms\ \isacommand{by}\isamarkupfalse%
\ auto\ \isanewline
\isacommand{qed}\isamarkupfalse%
%
\endisatagproof
{\isafoldproof}%
%
\isadelimproof
\isanewline
%
\endisadelimproof
\isanewline
\isacommand{lemma}\isamarkupfalse%
\ P{\isacharunderscore}{\kern0pt}names{\isacharunderscore}{\kern0pt}in{\isacharunderscore}{\kern0pt}P{\isacharunderscore}{\kern0pt}set{\isacharunderscore}{\kern0pt}succ\ {\isacharcolon}{\kern0pt}\ {\isachardoublequoteopen}x\ {\isasymin}\ P{\isacharunderscore}{\kern0pt}names\ {\isasymLongrightarrow}\ {\isasymexists}a{\isachardot}{\kern0pt}\ Ord{\isacharparenleft}{\kern0pt}a{\isacharparenright}{\kern0pt}\ {\isasymand}\ \ x\ {\isasymin}\ P{\isacharunderscore}{\kern0pt}set{\isacharparenleft}{\kern0pt}succ{\isacharparenleft}{\kern0pt}a{\isacharparenright}{\kern0pt}{\isacharparenright}{\kern0pt}{\isachardoublequoteclose}\ \isanewline
%
\isadelimproof
%
\endisadelimproof
%
\isatagproof
\isacommand{proof}\isamarkupfalse%
\ {\isacharminus}{\kern0pt}\ \isanewline
\ \ \isacommand{assume}\isamarkupfalse%
\ {\isachardoublequoteopen}x\ {\isasymin}\ P{\isacharunderscore}{\kern0pt}names{\isachardoublequoteclose}\ \isanewline
\ \ \isacommand{then}\isamarkupfalse%
\ \isacommand{obtain}\isamarkupfalse%
\ a\ \isakeyword{where}\ {\isachardoublequoteopen}Ord{\isacharparenleft}{\kern0pt}a{\isacharparenright}{\kern0pt}{\isachardoublequoteclose}\ {\isachardoublequoteopen}x\ {\isasymin}\ P{\isacharunderscore}{\kern0pt}set{\isacharparenleft}{\kern0pt}a{\isacharparenright}{\kern0pt}{\isachardoublequoteclose}\ \isacommand{unfolding}\isamarkupfalse%
\ P{\isacharunderscore}{\kern0pt}names{\isacharunderscore}{\kern0pt}def\ \isacommand{by}\isamarkupfalse%
\ auto\isanewline
\ \ \isacommand{then}\isamarkupfalse%
\ \isacommand{obtain}\isamarkupfalse%
\ b\ \isakeyword{where}\ bp\ {\isacharcolon}{\kern0pt}\ {\isachardoublequoteopen}b\ {\isacharless}{\kern0pt}\ a{\isachardoublequoteclose}\ {\isachardoublequoteopen}x\ {\isasymin}\ P{\isacharunderscore}{\kern0pt}set{\isacharparenleft}{\kern0pt}succ{\isacharparenleft}{\kern0pt}b{\isacharparenright}{\kern0pt}{\isacharparenright}{\kern0pt}{\isachardoublequoteclose}\ \isacommand{using}\isamarkupfalse%
\ P{\isacharunderscore}{\kern0pt}set{\isacharunderscore}{\kern0pt}elems{\isacharunderscore}{\kern0pt}in{\isacharunderscore}{\kern0pt}P{\isacharunderscore}{\kern0pt}set{\isacharunderscore}{\kern0pt}succ\ \isacommand{by}\isamarkupfalse%
\ auto\ \isanewline
\ \ \isacommand{then}\isamarkupfalse%
\ \isacommand{have}\isamarkupfalse%
\ {\isachardoublequoteopen}Ord{\isacharparenleft}{\kern0pt}b{\isacharparenright}{\kern0pt}{\isachardoublequoteclose}\ \isacommand{using}\isamarkupfalse%
\ lt{\isacharunderscore}{\kern0pt}Ord\ \isacommand{by}\isamarkupfalse%
\ auto\ \isanewline
\ \ \isacommand{then}\isamarkupfalse%
\ \isacommand{show}\isamarkupfalse%
\ {\isacharquery}{\kern0pt}thesis\ \isacommand{using}\isamarkupfalse%
\ bp\ \isacommand{by}\isamarkupfalse%
\ auto\ \isanewline
\isacommand{qed}\isamarkupfalse%
%
\endisatagproof
{\isafoldproof}%
%
\isadelimproof
\ \ \ \isanewline
%
\endisadelimproof
\ \ \isanewline
\isacommand{lemma}\isamarkupfalse%
\ P{\isacharunderscore}{\kern0pt}rank{\isacharunderscore}{\kern0pt}works\ {\isacharcolon}{\kern0pt}\ \isanewline
\ \ {\isachardoublequoteopen}x\ {\isasymin}\ P{\isacharunderscore}{\kern0pt}names\ {\isasymLongrightarrow}\ x\ {\isasymin}\ P{\isacharunderscore}{\kern0pt}set{\isacharparenleft}{\kern0pt}succ{\isacharparenleft}{\kern0pt}P{\isacharunderscore}{\kern0pt}rank{\isacharparenleft}{\kern0pt}x{\isacharparenright}{\kern0pt}{\isacharparenright}{\kern0pt}{\isacharparenright}{\kern0pt}{\isachardoublequoteclose}\isanewline
%
\isadelimproof
%
\endisadelimproof
%
\isatagproof
\isacommand{proof}\isamarkupfalse%
\ {\isacharminus}{\kern0pt}\ \isanewline
\ \ \isacommand{assume}\isamarkupfalse%
\ {\isachardoublequoteopen}x\ {\isasymin}\ P{\isacharunderscore}{\kern0pt}names{\isachardoublequoteclose}\ \isanewline
\ \ \isacommand{then}\isamarkupfalse%
\ \isacommand{obtain}\isamarkupfalse%
\ a\ \isakeyword{where}\ ah\ {\isacharcolon}{\kern0pt}\ {\isachardoublequoteopen}Ord{\isacharparenleft}{\kern0pt}a{\isacharparenright}{\kern0pt}{\isachardoublequoteclose}\ {\isachardoublequoteopen}x\ {\isasymin}\ P{\isacharunderscore}{\kern0pt}set{\isacharparenleft}{\kern0pt}succ{\isacharparenleft}{\kern0pt}a{\isacharparenright}{\kern0pt}{\isacharparenright}{\kern0pt}{\isachardoublequoteclose}\ \isacommand{using}\isamarkupfalse%
\ P{\isacharunderscore}{\kern0pt}names{\isacharunderscore}{\kern0pt}in{\isacharunderscore}{\kern0pt}P{\isacharunderscore}{\kern0pt}set{\isacharunderscore}{\kern0pt}succ\ \isacommand{by}\isamarkupfalse%
\ auto\ \isanewline
\isanewline
\ \ \isacommand{define}\isamarkupfalse%
\ Q\ \isakeyword{where}\ {\isachardoublequoteopen}Q\ {\isasymequiv}\ {\isasymlambda}a{\isachardot}{\kern0pt}\ Ord{\isacharparenleft}{\kern0pt}a{\isacharparenright}{\kern0pt}\ {\isasymand}\ x\ {\isasymin}\ P{\isacharunderscore}{\kern0pt}set{\isacharparenleft}{\kern0pt}succ{\isacharparenleft}{\kern0pt}a{\isacharparenright}{\kern0pt}{\isacharparenright}{\kern0pt}{\isachardoublequoteclose}\ \isanewline
\ \ \isacommand{then}\isamarkupfalse%
\ \isacommand{have}\isamarkupfalse%
\ {\isachardoublequoteopen}Q{\isacharparenleft}{\kern0pt}a{\isacharparenright}{\kern0pt}{\isachardoublequoteclose}\ \isacommand{using}\isamarkupfalse%
\ ah\ \isacommand{by}\isamarkupfalse%
\ auto\isanewline
\ \ \isacommand{then}\isamarkupfalse%
\ \isacommand{have}\isamarkupfalse%
\ {\isachardoublequoteopen}Q{\isacharparenleft}{\kern0pt}{\isasymmu}\ a{\isachardot}{\kern0pt}\ Q{\isacharparenleft}{\kern0pt}a{\isacharparenright}{\kern0pt}{\isacharparenright}{\kern0pt}{\isachardoublequoteclose}\ \isanewline
\ \ \ \ \isacommand{apply}\isamarkupfalse%
\ {\isacharparenleft}{\kern0pt}rule{\isacharunderscore}{\kern0pt}tac\ i{\isacharequal}{\kern0pt}a\ \isakeyword{in}\ LeastI{\isacharparenright}{\kern0pt}\ \isacommand{using}\isamarkupfalse%
\ ah\ \isacommand{by}\isamarkupfalse%
\ auto\ \isanewline
\ \ \isacommand{then}\isamarkupfalse%
\ \isacommand{have}\isamarkupfalse%
\ {\isachardoublequoteopen}x\ {\isasymin}\ P{\isacharunderscore}{\kern0pt}set{\isacharparenleft}{\kern0pt}succ{\isacharparenleft}{\kern0pt}{\isasymmu}\ a{\isachardot}{\kern0pt}\ Q{\isacharparenleft}{\kern0pt}a{\isacharparenright}{\kern0pt}{\isacharparenright}{\kern0pt}{\isacharparenright}{\kern0pt}{\isachardoublequoteclose}\ \isacommand{unfolding}\isamarkupfalse%
\ Q{\isacharunderscore}{\kern0pt}def\ \isacommand{by}\isamarkupfalse%
\ auto\ \isanewline
\ \ \isacommand{then}\isamarkupfalse%
\ \isacommand{show}\isamarkupfalse%
\ {\isachardoublequoteopen}x\ {\isasymin}\ P{\isacharunderscore}{\kern0pt}set{\isacharparenleft}{\kern0pt}succ{\isacharparenleft}{\kern0pt}P{\isacharunderscore}{\kern0pt}rank{\isacharparenleft}{\kern0pt}x{\isacharparenright}{\kern0pt}{\isacharparenright}{\kern0pt}{\isacharparenright}{\kern0pt}{\isachardoublequoteclose}\ \isacommand{unfolding}\isamarkupfalse%
\ P{\isacharunderscore}{\kern0pt}rank{\isacharunderscore}{\kern0pt}def\ Q{\isacharunderscore}{\kern0pt}def\ \isacommand{by}\isamarkupfalse%
\ auto\ \isanewline
\isacommand{qed}\isamarkupfalse%
%
\endisatagproof
{\isafoldproof}%
%
\isadelimproof
\isanewline
%
\endisadelimproof
\isanewline
\isacommand{lemma}\isamarkupfalse%
\ P{\isacharunderscore}{\kern0pt}set{\isacharunderscore}{\kern0pt}elems{\isacharunderscore}{\kern0pt}in{\isacharunderscore}{\kern0pt}M\ {\isacharcolon}{\kern0pt}\ {\isachardoublequoteopen}Ord{\isacharparenleft}{\kern0pt}a{\isacharparenright}{\kern0pt}\ {\isasymLongrightarrow}\ x\ {\isasymin}\ P{\isacharunderscore}{\kern0pt}set{\isacharparenleft}{\kern0pt}a{\isacharparenright}{\kern0pt}\ {\isasymLongrightarrow}\ x\ {\isasymin}\ M{\isachardoublequoteclose}\ \isanewline
%
\isadelimproof
%
\endisadelimproof
%
\isatagproof
\isacommand{proof}\isamarkupfalse%
\ {\isacharminus}{\kern0pt}\ \isanewline
\ \ \isacommand{assume}\isamarkupfalse%
\ {\isachardoublequoteopen}x\ {\isasymin}\ P{\isacharunderscore}{\kern0pt}set{\isacharparenleft}{\kern0pt}a{\isacharparenright}{\kern0pt}{\isachardoublequoteclose}\ {\isachardoublequoteopen}Ord{\isacharparenleft}{\kern0pt}a{\isacharparenright}{\kern0pt}{\isachardoublequoteclose}\isanewline
\ \ \isacommand{then}\isamarkupfalse%
\ \isacommand{have}\isamarkupfalse%
\ {\isachardoublequoteopen}{\isasymexists}b\ {\isacharless}{\kern0pt}\ a{\isachardot}{\kern0pt}\ x\ {\isasymin}\ P{\isacharunderscore}{\kern0pt}set{\isacharparenleft}{\kern0pt}succ{\isacharparenleft}{\kern0pt}b{\isacharparenright}{\kern0pt}{\isacharparenright}{\kern0pt}{\isachardoublequoteclose}\ \isacommand{using}\isamarkupfalse%
\ P{\isacharunderscore}{\kern0pt}set{\isacharunderscore}{\kern0pt}elems{\isacharunderscore}{\kern0pt}in{\isacharunderscore}{\kern0pt}P{\isacharunderscore}{\kern0pt}set{\isacharunderscore}{\kern0pt}succ\ \isacommand{by}\isamarkupfalse%
\ auto\ \isanewline
\ \ \isacommand{then}\isamarkupfalse%
\ \isacommand{obtain}\isamarkupfalse%
\ b\ \isakeyword{where}\ bp\ {\isacharcolon}{\kern0pt}\ {\isachardoublequoteopen}b\ {\isacharless}{\kern0pt}\ a{\isachardoublequoteclose}\ {\isachardoublequoteopen}x\ {\isasymin}\ P{\isacharunderscore}{\kern0pt}set{\isacharparenleft}{\kern0pt}succ{\isacharparenleft}{\kern0pt}b{\isacharparenright}{\kern0pt}{\isacharparenright}{\kern0pt}{\isachardoublequoteclose}\ \isacommand{by}\isamarkupfalse%
\ auto\isanewline
\ \ \isacommand{then}\isamarkupfalse%
\ \isacommand{have}\isamarkupfalse%
\ {\isachardoublequoteopen}Ord{\isacharparenleft}{\kern0pt}b{\isacharparenright}{\kern0pt}{\isachardoublequoteclose}\ \isacommand{using}\isamarkupfalse%
\ lt{\isacharunderscore}{\kern0pt}Ord\ \isacommand{by}\isamarkupfalse%
\ auto\ \isanewline
\ \ \isacommand{then}\isamarkupfalse%
\ \isacommand{have}\isamarkupfalse%
\ {\isachardoublequoteopen}x\ {\isasymin}\ Pow{\isacharparenleft}{\kern0pt}P{\isacharunderscore}{\kern0pt}set{\isacharparenleft}{\kern0pt}b{\isacharparenright}{\kern0pt}\ {\isasymtimes}\ P{\isacharparenright}{\kern0pt}\ {\isasyminter}\ M{\isachardoublequoteclose}\ \isacommand{using}\isamarkupfalse%
\ bp\ P{\isacharunderscore}{\kern0pt}set{\isacharunderscore}{\kern0pt}succ\ \isacommand{by}\isamarkupfalse%
\ auto\ \isanewline
\ \ \isacommand{then}\isamarkupfalse%
\ \isacommand{show}\isamarkupfalse%
\ {\isachardoublequoteopen}x\ {\isasymin}\ M{\isachardoublequoteclose}\ \isacommand{by}\isamarkupfalse%
\ auto\ \isanewline
\isacommand{qed}\isamarkupfalse%
%
\endisatagproof
{\isafoldproof}%
%
\isadelimproof
\isanewline
%
\endisadelimproof
\isanewline
\isacommand{lemma}\isamarkupfalse%
\ P{\isacharunderscore}{\kern0pt}namesI\ {\isacharcolon}{\kern0pt}\ {\isachardoublequoteopen}Ord{\isacharparenleft}{\kern0pt}a{\isacharparenright}{\kern0pt}\ {\isasymLongrightarrow}\ x\ {\isasymin}\ P{\isacharunderscore}{\kern0pt}set{\isacharparenleft}{\kern0pt}a{\isacharparenright}{\kern0pt}\ {\isasymLongrightarrow}\ x\ {\isasymin}\ P{\isacharunderscore}{\kern0pt}names{\isachardoublequoteclose}\ \isanewline
%
\isadelimproof
\ \ %
\endisadelimproof
%
\isatagproof
\isacommand{unfolding}\isamarkupfalse%
\ P{\isacharunderscore}{\kern0pt}names{\isacharunderscore}{\kern0pt}def\ \isacommand{using}\isamarkupfalse%
\ P{\isacharunderscore}{\kern0pt}set{\isacharunderscore}{\kern0pt}elems{\isacharunderscore}{\kern0pt}in{\isacharunderscore}{\kern0pt}M\ \isacommand{by}\isamarkupfalse%
\ auto%
\endisatagproof
{\isafoldproof}%
%
\isadelimproof
\isanewline
%
\endisadelimproof
\isanewline
\isacommand{lemma}\isamarkupfalse%
\ P{\isacharunderscore}{\kern0pt}names{\isacharunderscore}{\kern0pt}in\ {\isacharcolon}{\kern0pt}\ {\isachardoublequoteopen}x\ {\isasymin}\ P{\isacharunderscore}{\kern0pt}names\ {\isasymLongrightarrow}\ x\ {\isasymin}\ Pow{\isacharparenleft}{\kern0pt}P{\isacharunderscore}{\kern0pt}set{\isacharparenleft}{\kern0pt}P{\isacharunderscore}{\kern0pt}rank{\isacharparenleft}{\kern0pt}x{\isacharparenright}{\kern0pt}{\isacharparenright}{\kern0pt}\ {\isasymtimes}\ P{\isacharparenright}{\kern0pt}\ {\isasyminter}\ M{\isachardoublequoteclose}\ \isanewline
%
\isadelimproof
%
\endisadelimproof
%
\isatagproof
\isacommand{proof}\isamarkupfalse%
\ {\isacharminus}{\kern0pt}\ \isanewline
\ \ \isacommand{assume}\isamarkupfalse%
\ {\isachardoublequoteopen}x\ {\isasymin}\ P{\isacharunderscore}{\kern0pt}names{\isachardoublequoteclose}\ \isanewline
\ \ \isacommand{then}\isamarkupfalse%
\ \isacommand{have}\isamarkupfalse%
\ {\isachardoublequoteopen}x\ {\isasymin}\ P{\isacharunderscore}{\kern0pt}set{\isacharparenleft}{\kern0pt}succ{\isacharparenleft}{\kern0pt}P{\isacharunderscore}{\kern0pt}rank{\isacharparenleft}{\kern0pt}x{\isacharparenright}{\kern0pt}{\isacharparenright}{\kern0pt}{\isacharparenright}{\kern0pt}{\isachardoublequoteclose}\ \isacommand{using}\isamarkupfalse%
\ P{\isacharunderscore}{\kern0pt}rank{\isacharunderscore}{\kern0pt}works\ \isacommand{by}\isamarkupfalse%
\ auto\ \isanewline
\ \ \isacommand{then}\isamarkupfalse%
\ \isacommand{show}\isamarkupfalse%
\ {\isachardoublequoteopen}x\ {\isasymin}\ Pow{\isacharparenleft}{\kern0pt}P{\isacharunderscore}{\kern0pt}set{\isacharparenleft}{\kern0pt}P{\isacharunderscore}{\kern0pt}rank{\isacharparenleft}{\kern0pt}x{\isacharparenright}{\kern0pt}{\isacharparenright}{\kern0pt}\ {\isasymtimes}\ P{\isacharparenright}{\kern0pt}\ {\isasyminter}\ M{\isachardoublequoteclose}\ \ \isanewline
\ \ \ \ \isacommand{using}\isamarkupfalse%
\ P{\isacharunderscore}{\kern0pt}set{\isacharunderscore}{\kern0pt}succ\ \isacommand{by}\isamarkupfalse%
\ auto\ \isanewline
\isacommand{qed}\isamarkupfalse%
%
\endisatagproof
{\isafoldproof}%
%
\isadelimproof
\isanewline
%
\endisadelimproof
\isanewline
\isacommand{lemma}\isamarkupfalse%
\ relation{\isacharunderscore}{\kern0pt}P{\isacharunderscore}{\kern0pt}name\ {\isacharcolon}{\kern0pt}\ {\isachardoublequoteopen}x\ {\isasymin}\ P{\isacharunderscore}{\kern0pt}names\ {\isasymLongrightarrow}\ relation{\isacharparenleft}{\kern0pt}x{\isacharparenright}{\kern0pt}{\isachardoublequoteclose}\ \isanewline
%
\isadelimproof
\ \ %
\endisadelimproof
%
\isatagproof
\isacommand{using}\isamarkupfalse%
\ P{\isacharunderscore}{\kern0pt}names{\isacharunderscore}{\kern0pt}in\ \isacommand{unfolding}\isamarkupfalse%
\ relation{\isacharunderscore}{\kern0pt}def\ \isacommand{by}\isamarkupfalse%
\ blast%
\endisatagproof
{\isafoldproof}%
%
\isadelimproof
\ \isanewline
%
\endisadelimproof
\isanewline
\isacommand{lemma}\isamarkupfalse%
\ P{\isacharunderscore}{\kern0pt}name{\isacharunderscore}{\kern0pt}in{\isacharunderscore}{\kern0pt}M\ {\isacharcolon}{\kern0pt}\ {\isachardoublequoteopen}x\ {\isasymin}\ P{\isacharunderscore}{\kern0pt}names\ {\isasymLongrightarrow}\ x\ {\isasymin}\ M{\isachardoublequoteclose}\ \isanewline
%
\isadelimproof
%
\endisadelimproof
%
\isatagproof
\isacommand{proof}\isamarkupfalse%
\ {\isacharminus}{\kern0pt}\ \isanewline
\ \ \isacommand{assume}\isamarkupfalse%
\ {\isachardoublequoteopen}x\ {\isasymin}\ P{\isacharunderscore}{\kern0pt}names{\isachardoublequoteclose}\ \isanewline
\ \ \isacommand{then}\isamarkupfalse%
\ \isacommand{obtain}\isamarkupfalse%
\ a\ \isakeyword{where}\ {\isachardoublequoteopen}Ord{\isacharparenleft}{\kern0pt}a{\isacharparenright}{\kern0pt}{\isachardoublequoteclose}\ {\isachardoublequoteopen}x\ {\isasymin}\ P{\isacharunderscore}{\kern0pt}set{\isacharparenleft}{\kern0pt}a{\isacharparenright}{\kern0pt}{\isachardoublequoteclose}\ \isacommand{unfolding}\isamarkupfalse%
\ P{\isacharunderscore}{\kern0pt}names{\isacharunderscore}{\kern0pt}def\ \isacommand{by}\isamarkupfalse%
\ auto\ \isanewline
\ \ \isacommand{then}\isamarkupfalse%
\ \isacommand{show}\isamarkupfalse%
\ {\isachardoublequoteopen}x\ {\isasymin}\ M{\isachardoublequoteclose}\ \isacommand{using}\isamarkupfalse%
\ P{\isacharunderscore}{\kern0pt}set{\isacharunderscore}{\kern0pt}elems{\isacharunderscore}{\kern0pt}in{\isacharunderscore}{\kern0pt}M\ \isacommand{by}\isamarkupfalse%
\ auto\ \isanewline
\isacommand{qed}\isamarkupfalse%
%
\endisatagproof
{\isafoldproof}%
%
\isadelimproof
\ \isanewline
%
\endisadelimproof
\isanewline
\isacommand{lemma}\isamarkupfalse%
\ P{\isacharunderscore}{\kern0pt}set{\isacharunderscore}{\kern0pt}domain{\isacharunderscore}{\kern0pt}P{\isacharunderscore}{\kern0pt}set\ {\isacharcolon}{\kern0pt}\ \isanewline
\ \ {\isachardoublequoteopen}Ord{\isacharparenleft}{\kern0pt}a{\isacharparenright}{\kern0pt}\ {\isasymLongrightarrow}\ x\ {\isasymin}\ P{\isacharunderscore}{\kern0pt}set{\isacharparenleft}{\kern0pt}a{\isacharparenright}{\kern0pt}\ {\isasymLongrightarrow}\ {\isacharless}{\kern0pt}y{\isacharcomma}{\kern0pt}\ p{\isachargreater}{\kern0pt}\ {\isasymin}\ x\ {\isasymLongrightarrow}\ {\isasymexists}b\ {\isacharless}{\kern0pt}\ a{\isachardot}{\kern0pt}\ Ord{\isacharparenleft}{\kern0pt}b{\isacharparenright}{\kern0pt}\ {\isasymand}\ y\ {\isasymin}\ P{\isacharunderscore}{\kern0pt}set{\isacharparenleft}{\kern0pt}b{\isacharparenright}{\kern0pt}{\isachardoublequoteclose}\ \isanewline
%
\isadelimproof
%
\endisadelimproof
%
\isatagproof
\isacommand{proof}\isamarkupfalse%
\ {\isacharminus}{\kern0pt}\isanewline
\ \ \isacommand{have}\isamarkupfalse%
\ helper\ {\isacharcolon}{\kern0pt}\ \ {\isachardoublequoteopen}Ord{\isacharparenleft}{\kern0pt}a{\isacharparenright}{\kern0pt}\ {\isasymLongrightarrow}\ {\isacharless}{\kern0pt}y{\isacharcomma}{\kern0pt}\ p{\isachargreater}{\kern0pt}\ {\isasymin}\ x\ {\isasymLongrightarrow}\ {\isacharparenleft}{\kern0pt}x\ {\isasymin}\ P{\isacharunderscore}{\kern0pt}set{\isacharparenleft}{\kern0pt}a{\isacharparenright}{\kern0pt}\ {\isasymlongrightarrow}\ {\isacharparenleft}{\kern0pt}{\isasymexists}b\ {\isacharless}{\kern0pt}\ a{\isachardot}{\kern0pt}\ Ord{\isacharparenleft}{\kern0pt}b{\isacharparenright}{\kern0pt}\ {\isasymand}\ y\ {\isasymin}\ P{\isacharunderscore}{\kern0pt}set{\isacharparenleft}{\kern0pt}b{\isacharparenright}{\kern0pt}{\isacharparenright}{\kern0pt}{\isacharparenright}{\kern0pt}{\isachardoublequoteclose}\ \isanewline
\ \ \ \ \isacommand{apply}\isamarkupfalse%
\ {\isacharparenleft}{\kern0pt}rule{\isacharunderscore}{\kern0pt}tac\ \isanewline
\ \ \ \ \ \ \ \ P{\isacharequal}{\kern0pt}{\isachardoublequoteopen}{\isasymlambda}a\ {\isachardot}{\kern0pt}\ {\isacharparenleft}{\kern0pt}x\ {\isasymin}\ P{\isacharunderscore}{\kern0pt}set{\isacharparenleft}{\kern0pt}a{\isacharparenright}{\kern0pt}\ {\isasymlongrightarrow}\ {\isacharparenleft}{\kern0pt}{\isasymexists}b\ {\isacharless}{\kern0pt}\ a{\isachardot}{\kern0pt}\ Ord{\isacharparenleft}{\kern0pt}b{\isacharparenright}{\kern0pt}\ {\isasymand}\ y\ {\isasymin}\ P{\isacharunderscore}{\kern0pt}set{\isacharparenleft}{\kern0pt}b{\isacharparenright}{\kern0pt}{\isacharparenright}{\kern0pt}{\isacharparenright}{\kern0pt}{\isachardoublequoteclose}\ \isakeyword{in}\ trans{\isacharunderscore}{\kern0pt}induct{\isadigit{3}}{\isacharunderscore}{\kern0pt}raw{\isacharparenright}{\kern0pt}\isanewline
\ \ \ \ \isacommand{apply}\isamarkupfalse%
\ simp\isanewline
\ \ \ \ \isacommand{using}\isamarkupfalse%
\ P{\isacharunderscore}{\kern0pt}set{\isacharunderscore}{\kern0pt}{\isadigit{0}}\ \isacommand{apply}\isamarkupfalse%
\ simp\ \isanewline
\ \ \isacommand{proof}\isamarkupfalse%
\ {\isacharparenleft}{\kern0pt}clarify{\isacharparenright}{\kern0pt}\isanewline
\ \ \ \ \isacommand{fix}\isamarkupfalse%
\ b\isanewline
\ \ \ \ \isacommand{assume}\isamarkupfalse%
\ assms\ {\isacharcolon}{\kern0pt}{\isachardoublequoteopen}x\ {\isasymin}\ P{\isacharunderscore}{\kern0pt}set{\isacharparenleft}{\kern0pt}succ{\isacharparenleft}{\kern0pt}b{\isacharparenright}{\kern0pt}{\isacharparenright}{\kern0pt}{\isachardoublequoteclose}\ {\isachardoublequoteopen}Ord{\isacharparenleft}{\kern0pt}b{\isacharparenright}{\kern0pt}{\isachardoublequoteclose}\ {\isachardoublequoteopen}{\isasymlangle}y{\isacharcomma}{\kern0pt}\ p{\isasymrangle}\ {\isasymin}\ x\ {\isachardoublequoteclose}\isanewline
\ \ \ \ \isacommand{then}\isamarkupfalse%
\ \isacommand{have}\isamarkupfalse%
\ p{\isadigit{1}}\ {\isacharcolon}{\kern0pt}\ {\isachardoublequoteopen}x\ {\isasymin}\ \ Pow{\isacharparenleft}{\kern0pt}P{\isacharunderscore}{\kern0pt}set{\isacharparenleft}{\kern0pt}b{\isacharparenright}{\kern0pt}\ {\isasymtimes}\ P{\isacharparenright}{\kern0pt}\ {\isasyminter}\ M{\isachardoublequoteclose}\ \isacommand{using}\isamarkupfalse%
\ P{\isacharunderscore}{\kern0pt}set{\isacharunderscore}{\kern0pt}succ\ \isacommand{by}\isamarkupfalse%
\ auto\ \isanewline
\ \ \ \ \isacommand{then}\isamarkupfalse%
\ \isacommand{have}\isamarkupfalse%
\ p{\isadigit{2}}\ {\isacharcolon}{\kern0pt}\ {\isachardoublequoteopen}x\ {\isasymsubseteq}\ P{\isacharunderscore}{\kern0pt}set{\isacharparenleft}{\kern0pt}b{\isacharparenright}{\kern0pt}\ {\isasymtimes}\ P{\isachardoublequoteclose}\ \isacommand{by}\isamarkupfalse%
\ auto\ \isanewline
\ \ \ \ \isacommand{then}\isamarkupfalse%
\ \isacommand{have}\isamarkupfalse%
\ p{\isadigit{2}}{\isadigit{1}}\ {\isacharcolon}{\kern0pt}\ {\isachardoublequoteopen}{\isacharless}{\kern0pt}y{\isacharcomma}{\kern0pt}\ p{\isachargreater}{\kern0pt}\ {\isasymin}\ P{\isacharunderscore}{\kern0pt}set{\isacharparenleft}{\kern0pt}b{\isacharparenright}{\kern0pt}\ {\isasymtimes}\ P{\isachardoublequoteclose}\ \isacommand{using}\isamarkupfalse%
\ assms\ \isacommand{by}\isamarkupfalse%
\ auto\ \isanewline
\ \ \ \ \isacommand{then}\isamarkupfalse%
\ \isacommand{have}\isamarkupfalse%
\ p{\isadigit{2}}{\isadigit{2}}\ {\isacharcolon}{\kern0pt}\ {\isachardoublequoteopen}y\ {\isasymin}\ P{\isacharunderscore}{\kern0pt}set{\isacharparenleft}{\kern0pt}b{\isacharparenright}{\kern0pt}{\isachardoublequoteclose}\ \isacommand{by}\isamarkupfalse%
\ auto\ \isanewline
\ \ \ \ \isacommand{have}\isamarkupfalse%
\ p{\isadigit{3}}\ {\isacharcolon}{\kern0pt}\ {\isachardoublequoteopen}x\ {\isasymin}\ M{\isachardoublequoteclose}\ \isacommand{using}\isamarkupfalse%
\ p{\isadigit{1}}\ \isacommand{by}\isamarkupfalse%
\ auto\ \isanewline
\ \ \ \ \isacommand{then}\isamarkupfalse%
\ \isacommand{have}\isamarkupfalse%
\ {\isachardoublequoteopen}{\isacharless}{\kern0pt}y{\isacharcomma}{\kern0pt}\ p{\isachargreater}{\kern0pt}\ {\isasymin}\ M{\isachardoublequoteclose}\ \isacommand{using}\isamarkupfalse%
\ transM\ assms\ \isacommand{by}\isamarkupfalse%
\ auto\ \isanewline
\ \ \ \ \isacommand{then}\isamarkupfalse%
\ \isacommand{have}\isamarkupfalse%
\ p{\isadigit{4}}\ {\isacharcolon}{\kern0pt}\ {\isachardoublequoteopen}y\ {\isasymin}\ M{\isachardoublequoteclose}\ \isacommand{using}\isamarkupfalse%
\ pair{\isacharunderscore}{\kern0pt}in{\isacharunderscore}{\kern0pt}M{\isacharunderscore}{\kern0pt}iff\ \ \isacommand{by}\isamarkupfalse%
\ auto\ \isanewline
\ \ \ \ \isacommand{then}\isamarkupfalse%
\ \isacommand{show}\isamarkupfalse%
\ {\isachardoublequoteopen}{\isasymexists}b{\isacharless}{\kern0pt}succ{\isacharparenleft}{\kern0pt}b{\isacharparenright}{\kern0pt}{\isachardot}{\kern0pt}\ Ord{\isacharparenleft}{\kern0pt}b{\isacharparenright}{\kern0pt}\ {\isasymand}\ y\ {\isasymin}\ P{\isacharunderscore}{\kern0pt}set{\isacharparenleft}{\kern0pt}b{\isacharparenright}{\kern0pt}{\isachardoublequoteclose}\ \isanewline
\ \ \ \ \ \ \isacommand{using}\isamarkupfalse%
\ p{\isadigit{2}}{\isadigit{2}}\ assms\ \isacommand{by}\isamarkupfalse%
\ auto\ \isanewline
\ \ \isacommand{next}\isamarkupfalse%
\isanewline
\ \ \ \ \isacommand{fix}\isamarkupfalse%
\ a\ \isacommand{assume}\isamarkupfalse%
\ assms\ {\isacharcolon}{\kern0pt}\ {\isachardoublequoteopen}Limit{\isacharparenleft}{\kern0pt}a{\isacharparenright}{\kern0pt}{\isachardoublequoteclose}\ {\isachardoublequoteopen}{\isasymlangle}y{\isacharcomma}{\kern0pt}\ p{\isasymrangle}\ {\isasymin}\ x{\isachardoublequoteclose}\ \isanewline
\ \ \ \ \ \ {\isachardoublequoteopen}{\isasymforall}b{\isasymin}a{\isachardot}{\kern0pt}\ x\ {\isasymin}\ P{\isacharunderscore}{\kern0pt}set{\isacharparenleft}{\kern0pt}b{\isacharparenright}{\kern0pt}\ {\isasymlongrightarrow}\ {\isacharparenleft}{\kern0pt}{\isasymexists}c{\isacharless}{\kern0pt}b{\isachardot}{\kern0pt}\ Ord{\isacharparenleft}{\kern0pt}c{\isacharparenright}{\kern0pt}\ {\isasymand}\ y\ {\isasymin}\ P{\isacharunderscore}{\kern0pt}set{\isacharparenleft}{\kern0pt}c{\isacharparenright}{\kern0pt}{\isacharparenright}{\kern0pt}{\isachardoublequoteclose}\isanewline
\ \ \ \ \isacommand{show}\isamarkupfalse%
\ {\isachardoublequoteopen}x\ {\isasymin}\ P{\isacharunderscore}{\kern0pt}set{\isacharparenleft}{\kern0pt}a{\isacharparenright}{\kern0pt}\ {\isasymlongrightarrow}\ {\isacharparenleft}{\kern0pt}{\isasymexists}b{\isacharless}{\kern0pt}a{\isachardot}{\kern0pt}\ Ord{\isacharparenleft}{\kern0pt}b{\isacharparenright}{\kern0pt}\ {\isasymand}\ y\ {\isasymin}\ P{\isacharunderscore}{\kern0pt}set{\isacharparenleft}{\kern0pt}b{\isacharparenright}{\kern0pt}{\isacharparenright}{\kern0pt}{\isachardoublequoteclose}\isanewline
\ \ \ \ \isacommand{proof}\isamarkupfalse%
\ {\isacharparenleft}{\kern0pt}clarify{\isacharparenright}{\kern0pt}\ \isanewline
\ \ \ \ \ \ \isacommand{assume}\isamarkupfalse%
\ assms{\isadigit{1}}\ {\isacharcolon}{\kern0pt}\ {\isachardoublequoteopen}x\ {\isasymin}\ P{\isacharunderscore}{\kern0pt}set{\isacharparenleft}{\kern0pt}a{\isacharparenright}{\kern0pt}{\isachardoublequoteclose}\ \isanewline
\ \ \ \ \ \ \isacommand{then}\isamarkupfalse%
\ \isacommand{have}\isamarkupfalse%
\ {\isachardoublequoteopen}x\ {\isasymin}\ {\isacharparenleft}{\kern0pt}{\isasymUnion}\ b\ {\isacharless}{\kern0pt}\ a\ {\isachardot}{\kern0pt}\ P{\isacharunderscore}{\kern0pt}set{\isacharparenleft}{\kern0pt}b{\isacharparenright}{\kern0pt}{\isacharparenright}{\kern0pt}{\isachardoublequoteclose}\ \isacommand{using}\isamarkupfalse%
\ P{\isacharunderscore}{\kern0pt}set{\isacharunderscore}{\kern0pt}lim\ assms\ \isacommand{by}\isamarkupfalse%
\ auto\ \isanewline
\ \ \ \ \ \ \isacommand{then}\isamarkupfalse%
\ \isacommand{obtain}\isamarkupfalse%
\ b\ \isakeyword{where}\ p{\isadigit{1}}\ {\isacharcolon}{\kern0pt}\ {\isachardoublequoteopen}b\ {\isacharless}{\kern0pt}\ a{\isachardoublequoteclose}\ {\isachardoublequoteopen}x\ {\isasymin}\ P{\isacharunderscore}{\kern0pt}set{\isacharparenleft}{\kern0pt}b{\isacharparenright}{\kern0pt}{\isachardoublequoteclose}\ \isacommand{by}\isamarkupfalse%
\ auto\ \isanewline
\ \ \ \ \ \ \isacommand{then}\isamarkupfalse%
\ \isacommand{have}\isamarkupfalse%
\ {\isachardoublequoteopen}b\ {\isasymin}\ a{\isachardoublequoteclose}\ \isacommand{using}\isamarkupfalse%
\ ltD\ p{\isadigit{1}}\ \isacommand{by}\isamarkupfalse%
\ auto\ \isanewline
\ \ \ \ \ \ \isacommand{then}\isamarkupfalse%
\ \isacommand{obtain}\isamarkupfalse%
\ c\ \isakeyword{where}\ cp\ {\isacharcolon}{\kern0pt}\ {\isachardoublequoteopen}c\ {\isacharless}{\kern0pt}\ b{\isachardoublequoteclose}\ {\isachardoublequoteopen}Ord{\isacharparenleft}{\kern0pt}c{\isacharparenright}{\kern0pt}\ {\isasymand}\ y\ {\isasymin}\ P{\isacharunderscore}{\kern0pt}set{\isacharparenleft}{\kern0pt}c{\isacharparenright}{\kern0pt}{\isachardoublequoteclose}\ \isacommand{using}\isamarkupfalse%
\ assms\ p{\isadigit{1}}\ \isacommand{by}\isamarkupfalse%
\ auto\ \isanewline
\ \ \ \ \ \ \isacommand{then}\isamarkupfalse%
\ \isacommand{have}\isamarkupfalse%
\ {\isachardoublequoteopen}c\ {\isacharless}{\kern0pt}\ a{\isachardoublequoteclose}\ \isacommand{using}\isamarkupfalse%
\ p{\isadigit{1}}\ lt{\isacharunderscore}{\kern0pt}trans\ \isacommand{by}\isamarkupfalse%
\ auto\ \isanewline
\ \ \ \ \ \ \isacommand{then}\isamarkupfalse%
\ \isacommand{show}\isamarkupfalse%
\ {\isachardoublequoteopen}{\isasymexists}b{\isacharless}{\kern0pt}a{\isachardot}{\kern0pt}\ Ord{\isacharparenleft}{\kern0pt}b{\isacharparenright}{\kern0pt}\ {\isasymand}\ y\ {\isasymin}\ P{\isacharunderscore}{\kern0pt}set{\isacharparenleft}{\kern0pt}b{\isacharparenright}{\kern0pt}{\isachardoublequoteclose}\ \isacommand{using}\isamarkupfalse%
\ cp\ \isacommand{by}\isamarkupfalse%
\ auto\ \isanewline
\ \ \ \ \isacommand{qed}\isamarkupfalse%
\isanewline
\ \ \isacommand{qed}\isamarkupfalse%
\isanewline
\ \ \isacommand{assume}\isamarkupfalse%
\ {\isachardoublequoteopen}Ord{\isacharparenleft}{\kern0pt}a{\isacharparenright}{\kern0pt}{\isachardoublequoteclose}\ {\isachardoublequoteopen}x\ {\isasymin}\ P{\isacharunderscore}{\kern0pt}set{\isacharparenleft}{\kern0pt}a{\isacharparenright}{\kern0pt}{\isachardoublequoteclose}\ {\isachardoublequoteopen}{\isacharless}{\kern0pt}y{\isacharcomma}{\kern0pt}\ p{\isachargreater}{\kern0pt}\ {\isasymin}\ x{\isachardoublequoteclose}\isanewline
\ \ \isacommand{then}\isamarkupfalse%
\ \isacommand{show}\isamarkupfalse%
\ {\isacharquery}{\kern0pt}thesis\ \isacommand{using}\isamarkupfalse%
\ helper\ \isacommand{by}\isamarkupfalse%
\ auto\ \isanewline
\isacommand{qed}\isamarkupfalse%
%
\endisatagproof
{\isafoldproof}%
%
\isadelimproof
\ \isanewline
%
\endisadelimproof
\isanewline
\isacommand{lemma}\isamarkupfalse%
\ P{\isacharunderscore}{\kern0pt}name{\isacharunderscore}{\kern0pt}domain{\isacharunderscore}{\kern0pt}P{\isacharunderscore}{\kern0pt}name\ {\isacharcolon}{\kern0pt}\ \isanewline
\ \ {\isachardoublequoteopen}x\ {\isasymin}\ P{\isacharunderscore}{\kern0pt}names\ {\isasymLongrightarrow}\ {\isacharless}{\kern0pt}y{\isacharcomma}{\kern0pt}\ p{\isachargreater}{\kern0pt}\ {\isasymin}\ x\ {\isasymLongrightarrow}\ y\ {\isasymin}\ P{\isacharunderscore}{\kern0pt}names{\isachardoublequoteclose}\ \isanewline
%
\isadelimproof
%
\endisadelimproof
%
\isatagproof
\isacommand{proof}\isamarkupfalse%
\ {\isacharminus}{\kern0pt}\ \isanewline
\ \ \isacommand{assume}\isamarkupfalse%
\ assms\ {\isacharcolon}{\kern0pt}\ {\isachardoublequoteopen}x\ {\isasymin}\ P{\isacharunderscore}{\kern0pt}names{\isachardoublequoteclose}\ {\isachardoublequoteopen}{\isacharless}{\kern0pt}y{\isacharcomma}{\kern0pt}\ p{\isachargreater}{\kern0pt}\ {\isasymin}\ x{\isachardoublequoteclose}\isanewline
\ \ \isacommand{then}\isamarkupfalse%
\ \isacommand{obtain}\isamarkupfalse%
\ a\ \isakeyword{where}\ {\isachardoublequoteopen}Ord{\isacharparenleft}{\kern0pt}a{\isacharparenright}{\kern0pt}{\isachardoublequoteclose}\ {\isachardoublequoteopen}x\ {\isasymin}\ P{\isacharunderscore}{\kern0pt}set{\isacharparenleft}{\kern0pt}a{\isacharparenright}{\kern0pt}{\isachardoublequoteclose}\ \isacommand{unfolding}\isamarkupfalse%
\ P{\isacharunderscore}{\kern0pt}names{\isacharunderscore}{\kern0pt}def\ \isacommand{by}\isamarkupfalse%
\ auto\ \isanewline
\ \ \isacommand{then}\isamarkupfalse%
\ \isacommand{have}\isamarkupfalse%
\ {\isachardoublequoteopen}{\isasymexists}b\ {\isacharless}{\kern0pt}\ a{\isachardot}{\kern0pt}\ Ord{\isacharparenleft}{\kern0pt}b{\isacharparenright}{\kern0pt}\ {\isasymand}\ y\ {\isasymin}\ P{\isacharunderscore}{\kern0pt}set{\isacharparenleft}{\kern0pt}b{\isacharparenright}{\kern0pt}{\isachardoublequoteclose}\ \isacommand{using}\isamarkupfalse%
\ P{\isacharunderscore}{\kern0pt}set{\isacharunderscore}{\kern0pt}domain{\isacharunderscore}{\kern0pt}P{\isacharunderscore}{\kern0pt}set\ assms\ \isacommand{by}\isamarkupfalse%
\ auto\ \isanewline
\ \ \isacommand{then}\isamarkupfalse%
\ \isacommand{obtain}\isamarkupfalse%
\ b\ \isakeyword{where}\ bp\ {\isacharcolon}{\kern0pt}\ \ {\isachardoublequoteopen}Ord{\isacharparenleft}{\kern0pt}b{\isacharparenright}{\kern0pt}{\isachardoublequoteclose}\ {\isachardoublequoteopen}y\ {\isasymin}\ P{\isacharunderscore}{\kern0pt}set{\isacharparenleft}{\kern0pt}b{\isacharparenright}{\kern0pt}{\isachardoublequoteclose}\ \isacommand{unfolding}\isamarkupfalse%
\ oex{\isacharunderscore}{\kern0pt}def\ \isacommand{by}\isamarkupfalse%
\ auto\ \isanewline
\ \ \isacommand{then}\isamarkupfalse%
\ \isacommand{have}\isamarkupfalse%
\ {\isachardoublequoteopen}y\ {\isasymin}\ M{\isachardoublequoteclose}\ \isacommand{using}\isamarkupfalse%
\ P{\isacharunderscore}{\kern0pt}set{\isacharunderscore}{\kern0pt}elems{\isacharunderscore}{\kern0pt}in{\isacharunderscore}{\kern0pt}M\ \isacommand{by}\isamarkupfalse%
\ auto\ \isanewline
\ \ \isacommand{then}\isamarkupfalse%
\ \isacommand{show}\isamarkupfalse%
\ {\isachardoublequoteopen}y\ {\isasymin}\ P{\isacharunderscore}{\kern0pt}names{\isachardoublequoteclose}\ \isacommand{unfolding}\isamarkupfalse%
\ P{\isacharunderscore}{\kern0pt}names{\isacharunderscore}{\kern0pt}def\ \isacommand{using}\isamarkupfalse%
\ bp\ \isacommand{by}\isamarkupfalse%
\ auto\ \isanewline
\isacommand{qed}\isamarkupfalse%
%
\endisatagproof
{\isafoldproof}%
%
\isadelimproof
\isanewline
%
\endisadelimproof
\isanewline
\isacommand{lemma}\isamarkupfalse%
\ P{\isacharunderscore}{\kern0pt}name{\isacharunderscore}{\kern0pt}domain{\isacharunderscore}{\kern0pt}P{\isacharunderscore}{\kern0pt}name{\isacharprime}{\kern0pt}\ {\isacharcolon}{\kern0pt}\ \isanewline
\ \ {\isachardoublequoteopen}x\ {\isasymin}\ P{\isacharunderscore}{\kern0pt}names\ {\isasymLongrightarrow}\ y\ {\isasymin}\ domain{\isacharparenleft}{\kern0pt}x{\isacharparenright}{\kern0pt}\ {\isasymLongrightarrow}\ y\ {\isasymin}\ P{\isacharunderscore}{\kern0pt}names{\isachardoublequoteclose}\ \isanewline
%
\isadelimproof
%
\endisadelimproof
%
\isatagproof
\isacommand{proof}\isamarkupfalse%
\ {\isacharminus}{\kern0pt}\ \isanewline
\ \ \isacommand{assume}\isamarkupfalse%
\ assms\ {\isacharcolon}{\kern0pt}\ {\isachardoublequoteopen}x\ {\isasymin}\ P{\isacharunderscore}{\kern0pt}names{\isachardoublequoteclose}\ {\isachardoublequoteopen}y\ {\isasymin}\ domain{\isacharparenleft}{\kern0pt}x{\isacharparenright}{\kern0pt}{\isachardoublequoteclose}\ \isanewline
\ \ \isacommand{then}\isamarkupfalse%
\ \isacommand{obtain}\isamarkupfalse%
\ p\ \isakeyword{where}\ {\isachardoublequoteopen}{\isacharless}{\kern0pt}y{\isacharcomma}{\kern0pt}\ p{\isachargreater}{\kern0pt}\ {\isasymin}\ x{\isachardoublequoteclose}\ \isacommand{by}\isamarkupfalse%
\ auto\ \isanewline
\ \ \isacommand{then}\isamarkupfalse%
\ \isacommand{show}\isamarkupfalse%
\ {\isachardoublequoteopen}y\ {\isasymin}\ P{\isacharunderscore}{\kern0pt}names\ {\isachardoublequoteclose}\ \isacommand{using}\isamarkupfalse%
\ assms\ P{\isacharunderscore}{\kern0pt}name{\isacharunderscore}{\kern0pt}domain{\isacharunderscore}{\kern0pt}P{\isacharunderscore}{\kern0pt}name\ \isacommand{by}\isamarkupfalse%
\ auto\ \isanewline
\isacommand{qed}\isamarkupfalse%
%
\endisatagproof
{\isafoldproof}%
%
\isadelimproof
\isanewline
%
\endisadelimproof
\isanewline
\isacommand{lemma}\isamarkupfalse%
\ P{\isacharunderscore}{\kern0pt}name{\isacharunderscore}{\kern0pt}range\ {\isacharcolon}{\kern0pt}\ \isanewline
\ \ {\isachardoublequoteopen}x\ {\isasymin}\ P{\isacharunderscore}{\kern0pt}names\ {\isasymLongrightarrow}\ {\isacharless}{\kern0pt}y{\isacharcomma}{\kern0pt}\ p{\isachargreater}{\kern0pt}\ {\isasymin}\ x\ {\isasymLongrightarrow}\ p\ {\isasymin}\ P{\isachardoublequoteclose}\ \isanewline
%
\isadelimproof
%
\endisadelimproof
%
\isatagproof
\isacommand{proof}\isamarkupfalse%
\ {\isacharminus}{\kern0pt}\ \isanewline
\ \ \isacommand{assume}\isamarkupfalse%
\ assms\ {\isacharcolon}{\kern0pt}\ {\isachardoublequoteopen}x\ {\isasymin}\ P{\isacharunderscore}{\kern0pt}names{\isachardoublequoteclose}\ {\isachardoublequoteopen}{\isacharless}{\kern0pt}y{\isacharcomma}{\kern0pt}\ p{\isachargreater}{\kern0pt}\ {\isasymin}\ x{\isachardoublequoteclose}\isanewline
\ \ \isacommand{then}\isamarkupfalse%
\ \isacommand{have}\isamarkupfalse%
\ {\isachardoublequoteopen}x\ {\isasymin}\ Pow{\isacharparenleft}{\kern0pt}P{\isacharunderscore}{\kern0pt}set{\isacharparenleft}{\kern0pt}P{\isacharunderscore}{\kern0pt}rank{\isacharparenleft}{\kern0pt}x{\isacharparenright}{\kern0pt}{\isacharparenright}{\kern0pt}\ {\isasymtimes}\ P{\isacharparenright}{\kern0pt}\ {\isasyminter}\ M{\isachardoublequoteclose}\ \isacommand{using}\isamarkupfalse%
\ P{\isacharunderscore}{\kern0pt}names{\isacharunderscore}{\kern0pt}in\ \isacommand{by}\isamarkupfalse%
\ auto\isanewline
\ \ \isacommand{then}\isamarkupfalse%
\ \isacommand{show}\isamarkupfalse%
\ {\isachardoublequoteopen}p\ {\isasymin}\ P{\isachardoublequoteclose}\ \isacommand{using}\isamarkupfalse%
\ assms\ \isacommand{by}\isamarkupfalse%
\ auto\isanewline
\isacommand{qed}\isamarkupfalse%
%
\endisatagproof
{\isafoldproof}%
%
\isadelimproof
\isanewline
%
\endisadelimproof
\isanewline
\isacommand{lemma}\isamarkupfalse%
\ domain{\isacharunderscore}{\kern0pt}P{\isacharunderscore}{\kern0pt}rank{\isacharunderscore}{\kern0pt}lt\ {\isacharcolon}{\kern0pt}\ \isanewline
\ \ {\isachardoublequoteopen}x\ {\isasymin}\ P{\isacharunderscore}{\kern0pt}names\ {\isasymLongrightarrow}\ {\isacharless}{\kern0pt}y{\isacharcomma}{\kern0pt}\ p{\isachargreater}{\kern0pt}\ {\isasymin}\ x\ {\isasymLongrightarrow}\ P{\isacharunderscore}{\kern0pt}rank{\isacharparenleft}{\kern0pt}y{\isacharparenright}{\kern0pt}\ {\isacharless}{\kern0pt}\ P{\isacharunderscore}{\kern0pt}rank{\isacharparenleft}{\kern0pt}x{\isacharparenright}{\kern0pt}{\isachardoublequoteclose}\ \isanewline
\ \ \isacommand{thm}\isamarkupfalse%
\ Least{\isacharunderscore}{\kern0pt}le\isanewline
%
\isadelimproof
%
\endisadelimproof
%
\isatagproof
\isacommand{proof}\isamarkupfalse%
\ {\isacharminus}{\kern0pt}\ \isanewline
\ \ \isacommand{assume}\isamarkupfalse%
\ assms\ {\isacharcolon}{\kern0pt}\ \ {\isachardoublequoteopen}x\ {\isasymin}\ P{\isacharunderscore}{\kern0pt}names{\isachardoublequoteclose}\ {\isachardoublequoteopen}{\isacharless}{\kern0pt}y{\isacharcomma}{\kern0pt}\ p{\isachargreater}{\kern0pt}\ {\isasymin}\ x{\isachardoublequoteclose}\isanewline
\ \ \isacommand{then}\isamarkupfalse%
\ \isacommand{have}\isamarkupfalse%
\ p{\isadigit{1}}\ {\isacharcolon}{\kern0pt}\ {\isachardoublequoteopen}x\ {\isasymin}\ P{\isacharunderscore}{\kern0pt}set{\isacharparenleft}{\kern0pt}succ{\isacharparenleft}{\kern0pt}P{\isacharunderscore}{\kern0pt}rank{\isacharparenleft}{\kern0pt}x{\isacharparenright}{\kern0pt}{\isacharparenright}{\kern0pt}{\isacharparenright}{\kern0pt}{\isachardoublequoteclose}\ \isacommand{using}\isamarkupfalse%
\ P{\isacharunderscore}{\kern0pt}rank{\isacharunderscore}{\kern0pt}works\ \isacommand{by}\isamarkupfalse%
\ auto\ \isanewline
\ \ \isacommand{have}\isamarkupfalse%
\ {\isachardoublequoteopen}Ord{\isacharparenleft}{\kern0pt}succ{\isacharparenleft}{\kern0pt}P{\isacharunderscore}{\kern0pt}rank{\isacharparenleft}{\kern0pt}x{\isacharparenright}{\kern0pt}{\isacharparenright}{\kern0pt}{\isacharparenright}{\kern0pt}{\isachardoublequoteclose}\ \isacommand{unfolding}\isamarkupfalse%
\ P{\isacharunderscore}{\kern0pt}rank{\isacharunderscore}{\kern0pt}def\ \isacommand{by}\isamarkupfalse%
\ auto\ \isanewline
\ \ \isacommand{then}\isamarkupfalse%
\ \isacommand{have}\isamarkupfalse%
\ {\isachardoublequoteopen}{\isasymexists}b\ {\isacharless}{\kern0pt}\ succ{\isacharparenleft}{\kern0pt}P{\isacharunderscore}{\kern0pt}rank{\isacharparenleft}{\kern0pt}x{\isacharparenright}{\kern0pt}{\isacharparenright}{\kern0pt}{\isachardot}{\kern0pt}\ Ord{\isacharparenleft}{\kern0pt}b{\isacharparenright}{\kern0pt}\ {\isasymand}\ y\ {\isasymin}\ P{\isacharunderscore}{\kern0pt}set{\isacharparenleft}{\kern0pt}b{\isacharparenright}{\kern0pt}{\isachardoublequoteclose}\isanewline
\ \ \ \ \isacommand{apply}\isamarkupfalse%
\ {\isacharparenleft}{\kern0pt}rule{\isacharunderscore}{\kern0pt}tac\ x{\isacharequal}{\kern0pt}x\ \isakeyword{and}\ a{\isacharequal}{\kern0pt}{\isachardoublequoteopen}succ{\isacharparenleft}{\kern0pt}P{\isacharunderscore}{\kern0pt}rank{\isacharparenleft}{\kern0pt}x{\isacharparenright}{\kern0pt}{\isacharparenright}{\kern0pt}{\isachardoublequoteclose}\ \isakeyword{in}\ P{\isacharunderscore}{\kern0pt}set{\isacharunderscore}{\kern0pt}domain{\isacharunderscore}{\kern0pt}P{\isacharunderscore}{\kern0pt}set{\isacharparenright}{\kern0pt}\isanewline
\ \ \ \ \isacommand{using}\isamarkupfalse%
\ assms\ p{\isadigit{1}}\ \isacommand{by}\isamarkupfalse%
\ auto\ \isanewline
\ \ \isacommand{then}\isamarkupfalse%
\ \isacommand{obtain}\isamarkupfalse%
\ b\ \isakeyword{where}\ bp\ {\isacharcolon}{\kern0pt}\ {\isachardoublequoteopen}b\ {\isacharless}{\kern0pt}\ succ{\isacharparenleft}{\kern0pt}P{\isacharunderscore}{\kern0pt}rank{\isacharparenleft}{\kern0pt}x{\isacharparenright}{\kern0pt}{\isacharparenright}{\kern0pt}{\isachardoublequoteclose}\ {\isachardoublequoteopen}Ord{\isacharparenleft}{\kern0pt}b{\isacharparenright}{\kern0pt}{\isachardoublequoteclose}\ {\isachardoublequoteopen}y\ {\isasymin}\ P{\isacharunderscore}{\kern0pt}set{\isacharparenleft}{\kern0pt}b{\isacharparenright}{\kern0pt}{\isachardoublequoteclose}\ \isacommand{unfolding}\isamarkupfalse%
\ oex{\isacharunderscore}{\kern0pt}def\ \isacommand{by}\isamarkupfalse%
\ auto\ \isanewline
\ \ \isacommand{then}\isamarkupfalse%
\ \isacommand{have}\isamarkupfalse%
\ {\isachardoublequoteopen}{\isasymexists}c\ {\isacharless}{\kern0pt}\ b{\isachardot}{\kern0pt}\ y\ {\isasymin}\ P{\isacharunderscore}{\kern0pt}set{\isacharparenleft}{\kern0pt}succ{\isacharparenleft}{\kern0pt}c{\isacharparenright}{\kern0pt}{\isacharparenright}{\kern0pt}{\isachardoublequoteclose}\isanewline
\ \ \ \ \isacommand{apply}\isamarkupfalse%
\ {\isacharparenleft}{\kern0pt}rule{\isacharunderscore}{\kern0pt}tac\ \ P{\isacharunderscore}{\kern0pt}set{\isacharunderscore}{\kern0pt}elems{\isacharunderscore}{\kern0pt}in{\isacharunderscore}{\kern0pt}P{\isacharunderscore}{\kern0pt}set{\isacharunderscore}{\kern0pt}succ{\isacharparenright}{\kern0pt}\ \isacommand{by}\isamarkupfalse%
\ auto\ \isanewline
\ \ \isacommand{then}\isamarkupfalse%
\ \isacommand{obtain}\isamarkupfalse%
\ c\ \isakeyword{where}\ cp\ {\isacharcolon}{\kern0pt}\ {\isachardoublequoteopen}c\ {\isacharless}{\kern0pt}\ b{\isachardoublequoteclose}\ {\isachardoublequoteopen}y\ {\isasymin}\ P{\isacharunderscore}{\kern0pt}set{\isacharparenleft}{\kern0pt}succ{\isacharparenleft}{\kern0pt}c{\isacharparenright}{\kern0pt}{\isacharparenright}{\kern0pt}{\isachardoublequoteclose}\ \isacommand{using}\isamarkupfalse%
\ oex{\isacharunderscore}{\kern0pt}def\ \isacommand{by}\isamarkupfalse%
\ auto\ \isanewline
\ \ \isacommand{have}\isamarkupfalse%
\ p{\isadigit{3}}\ {\isacharcolon}{\kern0pt}\ {\isachardoublequoteopen}c\ {\isacharless}{\kern0pt}\ P{\isacharunderscore}{\kern0pt}rank{\isacharparenleft}{\kern0pt}x{\isacharparenright}{\kern0pt}{\isachardoublequoteclose}\ \isacommand{using}\isamarkupfalse%
\ lt{\isacharunderscore}{\kern0pt}le{\isacharunderscore}{\kern0pt}lt\ cp\ bp\ \isacommand{by}\isamarkupfalse%
\ auto\ \isanewline
\ \ \isacommand{have}\isamarkupfalse%
\ {\isachardoublequoteopen}P{\isacharunderscore}{\kern0pt}rank{\isacharparenleft}{\kern0pt}y{\isacharparenright}{\kern0pt}\ {\isasymle}\ c{\isachardoublequoteclose}\ \isanewline
\ \ \ \ \isacommand{unfolding}\isamarkupfalse%
\ P{\isacharunderscore}{\kern0pt}rank{\isacharunderscore}{\kern0pt}def\ \isanewline
\ \ \ \ \isacommand{apply}\isamarkupfalse%
\ {\isacharparenleft}{\kern0pt}rule\ Least{\isacharunderscore}{\kern0pt}le{\isacharparenright}{\kern0pt}\ \isanewline
\ \ \ \ \isacommand{using}\isamarkupfalse%
\ cp\ lt{\isacharunderscore}{\kern0pt}Ord\ \isacommand{by}\isamarkupfalse%
\ auto\ \isanewline
\ \ \isacommand{then}\isamarkupfalse%
\ \isacommand{show}\isamarkupfalse%
\ {\isachardoublequoteopen}P{\isacharunderscore}{\kern0pt}rank{\isacharparenleft}{\kern0pt}y{\isacharparenright}{\kern0pt}\ {\isacharless}{\kern0pt}\ P{\isacharunderscore}{\kern0pt}rank{\isacharparenleft}{\kern0pt}x{\isacharparenright}{\kern0pt}{\isachardoublequoteclose}\ \isacommand{using}\isamarkupfalse%
\ p{\isadigit{3}}\ le{\isacharunderscore}{\kern0pt}lt{\isacharunderscore}{\kern0pt}lt\ \isacommand{by}\isamarkupfalse%
\ auto\ \isanewline
\isacommand{qed}\isamarkupfalse%
%
\endisatagproof
{\isafoldproof}%
%
\isadelimproof
\isanewline
%
\endisadelimproof
\isanewline
\isacommand{lemma}\isamarkupfalse%
\ P{\isacharunderscore}{\kern0pt}rank{\isacharunderscore}{\kern0pt}induct\ {\isacharcolon}{\kern0pt}\ \isanewline
\ \ {\isachardoublequoteopen}{\isacharparenleft}{\kern0pt}{\isasymforall}x\ {\isasymin}\ P{\isacharunderscore}{\kern0pt}names{\isachardot}{\kern0pt}\ {\isacharparenleft}{\kern0pt}{\isacharparenleft}{\kern0pt}{\isasymforall}\ y\ {\isasymin}\ P{\isacharunderscore}{\kern0pt}names{\isachardot}{\kern0pt}\ P{\isacharunderscore}{\kern0pt}rank{\isacharparenleft}{\kern0pt}y{\isacharparenright}{\kern0pt}\ {\isacharless}{\kern0pt}\ P{\isacharunderscore}{\kern0pt}rank{\isacharparenleft}{\kern0pt}x{\isacharparenright}{\kern0pt}\ {\isasymlongrightarrow}\ Q{\isacharparenleft}{\kern0pt}y{\isacharparenright}{\kern0pt}{\isacharparenright}{\kern0pt}\ {\isasymlongrightarrow}\ Q{\isacharparenleft}{\kern0pt}x{\isacharparenright}{\kern0pt}{\isacharparenright}{\kern0pt}{\isacharparenright}{\kern0pt}\isanewline
\ \ \ \ {\isasymLongrightarrow}\ x\ {\isasymin}\ P{\isacharunderscore}{\kern0pt}names\ {\isasymLongrightarrow}\ Q{\isacharparenleft}{\kern0pt}x{\isacharparenright}{\kern0pt}{\isachardoublequoteclose}\ \isanewline
%
\isadelimproof
%
\endisadelimproof
%
\isatagproof
\isacommand{proof}\isamarkupfalse%
\ {\isacharminus}{\kern0pt}\ \isanewline
\ \ \isacommand{fix}\isamarkupfalse%
\ x\ \isanewline
\ \ \isacommand{assume}\isamarkupfalse%
\ assms\ {\isacharcolon}{\kern0pt}\ {\isachardoublequoteopen}x\ {\isasymin}\ P{\isacharunderscore}{\kern0pt}names{\isachardoublequoteclose}\ {\isachardoublequoteopen}{\isasymforall}x{\isasymin}P{\isacharunderscore}{\kern0pt}names{\isachardot}{\kern0pt}\ {\isacharparenleft}{\kern0pt}{\isasymforall}y{\isasymin}P{\isacharunderscore}{\kern0pt}names{\isachardot}{\kern0pt}\ P{\isacharunderscore}{\kern0pt}rank{\isacharparenleft}{\kern0pt}y{\isacharparenright}{\kern0pt}\ {\isacharless}{\kern0pt}\ P{\isacharunderscore}{\kern0pt}rank{\isacharparenleft}{\kern0pt}x{\isacharparenright}{\kern0pt}\ {\isasymlongrightarrow}\ Q{\isacharparenleft}{\kern0pt}y{\isacharparenright}{\kern0pt}{\isacharparenright}{\kern0pt}\ {\isasymlongrightarrow}\ Q{\isacharparenleft}{\kern0pt}x{\isacharparenright}{\kern0pt}{\isachardoublequoteclose}\isanewline
\ \ \isacommand{define}\isamarkupfalse%
\ R\ {\isacharcolon}{\kern0pt}{\isacharcolon}{\kern0pt}\ {\isachardoublequoteopen}i{\isasymRightarrow}o{\isachardoublequoteclose}\ \isakeyword{where}\ {\isachardoublequoteopen}R\ {\isasymequiv}\ {\isasymlambda}a{\isachardot}{\kern0pt}\ {\isasymforall}y\ {\isasymin}\ P{\isacharunderscore}{\kern0pt}names{\isachardot}{\kern0pt}\ P{\isacharunderscore}{\kern0pt}rank{\isacharparenleft}{\kern0pt}y{\isacharparenright}{\kern0pt}\ {\isacharequal}{\kern0pt}\ a\ {\isasymlongrightarrow}\ Q{\isacharparenleft}{\kern0pt}y{\isacharparenright}{\kern0pt}{\isachardoublequoteclose}\ \isanewline
\ \ \isacommand{have}\isamarkupfalse%
\ {\isachardoublequoteopen}{\isacharparenleft}{\kern0pt}{\isasymAnd}a{\isachardot}{\kern0pt}\ {\isacharparenleft}{\kern0pt}{\isasymforall}b\ {\isasymin}\ a{\isachardot}{\kern0pt}\ R{\isacharparenleft}{\kern0pt}b{\isacharparenright}{\kern0pt}{\isacharparenright}{\kern0pt}\ {\isasymLongrightarrow}\ R{\isacharparenleft}{\kern0pt}a{\isacharparenright}{\kern0pt}{\isacharparenright}{\kern0pt}{\isachardoublequoteclose}\isanewline
\ \ \ \ \isacommand{unfolding}\isamarkupfalse%
\ R{\isacharunderscore}{\kern0pt}def\isanewline
\ \ \ \ \isacommand{apply}\isamarkupfalse%
\ clarify\ \isanewline
\ \ \isacommand{proof}\isamarkupfalse%
\ {\isacharminus}{\kern0pt}\ \isanewline
\ \ \ \ \isacommand{fix}\isamarkupfalse%
\ y\ \isanewline
\ \ \ \ \isacommand{assume}\isamarkupfalse%
\ assms{\isadigit{1}}\ {\isacharcolon}{\kern0pt}\ {\isachardoublequoteopen}{\isasymforall}b{\isasymin}P{\isacharunderscore}{\kern0pt}rank{\isacharparenleft}{\kern0pt}y{\isacharparenright}{\kern0pt}{\isachardot}{\kern0pt}\ {\isasymforall}z{\isasymin}P{\isacharunderscore}{\kern0pt}names{\isachardot}{\kern0pt}\ P{\isacharunderscore}{\kern0pt}rank{\isacharparenleft}{\kern0pt}z{\isacharparenright}{\kern0pt}\ {\isacharequal}{\kern0pt}\ b\ {\isasymlongrightarrow}\ Q{\isacharparenleft}{\kern0pt}z{\isacharparenright}{\kern0pt}{\isachardoublequoteclose}\ {\isachardoublequoteopen}y\ {\isasymin}\ P{\isacharunderscore}{\kern0pt}names{\isachardoublequoteclose}\ \isanewline
\ \ \ \ \isacommand{then}\isamarkupfalse%
\ \isacommand{have}\isamarkupfalse%
\ {\isachardoublequoteopen}{\isasymforall}z\ {\isasymin}\ P{\isacharunderscore}{\kern0pt}names{\isachardot}{\kern0pt}\ P{\isacharunderscore}{\kern0pt}rank{\isacharparenleft}{\kern0pt}z{\isacharparenright}{\kern0pt}\ {\isacharless}{\kern0pt}\ P{\isacharunderscore}{\kern0pt}rank{\isacharparenleft}{\kern0pt}y{\isacharparenright}{\kern0pt}\ {\isasymlongrightarrow}\ Q{\isacharparenleft}{\kern0pt}z{\isacharparenright}{\kern0pt}{\isachardoublequoteclose}\ \isanewline
\ \ \ \ \isacommand{proof}\isamarkupfalse%
\ {\isacharparenleft}{\kern0pt}clarify{\isacharparenright}{\kern0pt}\ \isanewline
\ \ \ \ \ \ \isacommand{fix}\isamarkupfalse%
\ z\ \isacommand{assume}\isamarkupfalse%
\ assms{\isadigit{2}}\ {\isacharcolon}{\kern0pt}\ \ {\isachardoublequoteopen}z\ {\isasymin}\ P{\isacharunderscore}{\kern0pt}names{\isachardoublequoteclose}\ {\isachardoublequoteopen}P{\isacharunderscore}{\kern0pt}rank{\isacharparenleft}{\kern0pt}z{\isacharparenright}{\kern0pt}\ {\isacharless}{\kern0pt}\ P{\isacharunderscore}{\kern0pt}rank{\isacharparenleft}{\kern0pt}y{\isacharparenright}{\kern0pt}{\isachardoublequoteclose}\ \isanewline
\ \ \ \ \ \ \isacommand{then}\isamarkupfalse%
\ \isacommand{have}\isamarkupfalse%
\ {\isachardoublequoteopen}P{\isacharunderscore}{\kern0pt}rank{\isacharparenleft}{\kern0pt}z{\isacharparenright}{\kern0pt}\ {\isasymin}\ P{\isacharunderscore}{\kern0pt}rank{\isacharparenleft}{\kern0pt}y{\isacharparenright}{\kern0pt}{\isachardoublequoteclose}\ \isacommand{using}\isamarkupfalse%
\ ltD\ \isacommand{by}\isamarkupfalse%
\ auto\ \isanewline
\ \ \ \ \ \ \isacommand{then}\isamarkupfalse%
\ \isacommand{show}\isamarkupfalse%
\ {\isachardoublequoteopen}Q{\isacharparenleft}{\kern0pt}z{\isacharparenright}{\kern0pt}{\isachardoublequoteclose}\ \isacommand{using}\isamarkupfalse%
\ assms{\isadigit{1}}\ assms{\isadigit{2}}\ \isacommand{by}\isamarkupfalse%
\ auto\ \isanewline
\ \ \ \ \isacommand{qed}\isamarkupfalse%
\isanewline
\ \ \ \ \isacommand{then}\isamarkupfalse%
\ \isacommand{show}\isamarkupfalse%
\ {\isachardoublequoteopen}Q{\isacharparenleft}{\kern0pt}y{\isacharparenright}{\kern0pt}{\isachardoublequoteclose}\ \isacommand{using}\isamarkupfalse%
\ assms\ assms{\isadigit{1}}\ \isacommand{by}\isamarkupfalse%
\ auto\ \isanewline
\ \ \isacommand{qed}\isamarkupfalse%
\isanewline
\ \ \isacommand{then}\isamarkupfalse%
\ \isacommand{have}\isamarkupfalse%
\ {\isachardoublequoteopen}R{\isacharparenleft}{\kern0pt}P{\isacharunderscore}{\kern0pt}rank{\isacharparenleft}{\kern0pt}x{\isacharparenright}{\kern0pt}{\isacharparenright}{\kern0pt}{\isachardoublequoteclose}\isanewline
\ \ \ \ \isacommand{apply}\isamarkupfalse%
\ {\isacharparenleft}{\kern0pt}rule{\isacharunderscore}{\kern0pt}tac\ eps{\isacharunderscore}{\kern0pt}induct{\isacharparenright}{\kern0pt}\ \isacommand{by}\isamarkupfalse%
\ auto\isanewline
\ \ \isacommand{then}\isamarkupfalse%
\ \isacommand{show}\isamarkupfalse%
\ {\isachardoublequoteopen}Q{\isacharparenleft}{\kern0pt}x{\isacharparenright}{\kern0pt}{\isachardoublequoteclose}\ \isacommand{unfolding}\isamarkupfalse%
\ R{\isacharunderscore}{\kern0pt}def\ \isacommand{using}\isamarkupfalse%
\ assms\ \isacommand{by}\isamarkupfalse%
\ auto\ \isanewline
\isacommand{qed}\isamarkupfalse%
%
\endisatagproof
{\isafoldproof}%
%
\isadelimproof
\isanewline
%
\endisadelimproof
\isanewline
\isacommand{lemma}\isamarkupfalse%
\ set{\isacharunderscore}{\kern0pt}of{\isacharunderscore}{\kern0pt}P{\isacharunderscore}{\kern0pt}names{\isacharunderscore}{\kern0pt}in{\isacharunderscore}{\kern0pt}P{\isacharunderscore}{\kern0pt}set\ {\isacharcolon}{\kern0pt}\isanewline
\ \ {\isachardoublequoteopen}A\ {\isasymsubseteq}\ P{\isacharunderscore}{\kern0pt}names\ {\isasymLongrightarrow}\ {\isacharparenleft}{\kern0pt}{\isasymexists}a{\isachardot}{\kern0pt}\ {\isacharparenleft}{\kern0pt}Limit{\isacharparenleft}{\kern0pt}a{\isacharparenright}{\kern0pt}\ {\isasymand}\ {\isacharparenleft}{\kern0pt}{\isasymforall}x\ {\isasymin}\ A{\isachardot}{\kern0pt}\ x\ {\isasymin}\ P{\isacharunderscore}{\kern0pt}set{\isacharparenleft}{\kern0pt}a{\isacharparenright}{\kern0pt}{\isacharparenright}{\kern0pt}{\isacharparenright}{\kern0pt}{\isacharparenright}{\kern0pt}{\isachardoublequoteclose}\isanewline
%
\isadelimproof
\ \ %
\endisadelimproof
%
\isatagproof
\isacommand{apply}\isamarkupfalse%
{\isacharparenleft}{\kern0pt}cases\ {\isachardoublequoteopen}A{\isacharequal}{\kern0pt}{\isadigit{0}}{\isachardoublequoteclose}{\isacharparenright}{\kern0pt}\ \isacommand{apply}\isamarkupfalse%
\ blast\isanewline
\isacommand{proof}\isamarkupfalse%
\ {\isacharminus}{\kern0pt}\ \isanewline
\ \ \isacommand{assume}\isamarkupfalse%
\ assm\ {\isacharcolon}{\kern0pt}\ {\isachardoublequoteopen}A\ {\isasymsubseteq}\ P{\isacharunderscore}{\kern0pt}names{\isachardoublequoteclose}\ {\isachardoublequoteopen}A\ {\isasymnoteq}\ {\isadigit{0}}{\isachardoublequoteclose}\isanewline
\ \ \isacommand{define}\isamarkupfalse%
\ F\ \isakeyword{where}\ {\isachardoublequoteopen}F\ {\isasymequiv}\ {\isasymlambda}x{\isachardot}{\kern0pt}\ {\isacharparenleft}{\kern0pt}{\isasymmu}\ a{\isachardot}{\kern0pt}\ x\ {\isasymin}\ P{\isacharunderscore}{\kern0pt}set{\isacharparenleft}{\kern0pt}a{\isacharparenright}{\kern0pt}\ {\isasymand}\ Limit{\isacharparenleft}{\kern0pt}a{\isacharparenright}{\kern0pt}{\isacharparenright}{\kern0pt}{\isachardoublequoteclose}\ \isanewline
\ \ \isacommand{then}\isamarkupfalse%
\ \isacommand{have}\isamarkupfalse%
\ p{\isadigit{1}}\ {\isacharcolon}{\kern0pt}\ {\isachardoublequoteopen}{\isasymAnd}x{\isachardot}{\kern0pt}\ x\ {\isasymin}\ A\ {\isasymLongrightarrow}\ x\ {\isasymin}\ P{\isacharunderscore}{\kern0pt}set{\isacharparenleft}{\kern0pt}F{\isacharparenleft}{\kern0pt}x{\isacharparenright}{\kern0pt}{\isacharparenright}{\kern0pt}\ {\isasymand}\ Limit{\isacharparenleft}{\kern0pt}F{\isacharparenleft}{\kern0pt}x{\isacharparenright}{\kern0pt}{\isacharparenright}{\kern0pt}{\isachardoublequoteclose}\isanewline
\ \ \isacommand{proof}\isamarkupfalse%
\ {\isacharminus}{\kern0pt}\ \isanewline
\ \ \ \ \isacommand{fix}\isamarkupfalse%
\ x\ \isacommand{assume}\isamarkupfalse%
\ {\isachardoublequoteopen}x\ {\isasymin}\ A{\isachardoublequoteclose}\ \isanewline
\ \ \ \ \isacommand{then}\isamarkupfalse%
\ \isacommand{have}\isamarkupfalse%
\ {\isachardoublequoteopen}x\ {\isasymin}\ P{\isacharunderscore}{\kern0pt}names{\isachardoublequoteclose}\ \isacommand{using}\isamarkupfalse%
\ assm\ \isacommand{by}\isamarkupfalse%
\ auto\ \isanewline
\ \ \ \ \isacommand{then}\isamarkupfalse%
\ \isacommand{obtain}\isamarkupfalse%
\ a\ \isakeyword{where}\ ap\ {\isacharcolon}{\kern0pt}\ {\isachardoublequoteopen}Ord{\isacharparenleft}{\kern0pt}a{\isacharparenright}{\kern0pt}{\isachardoublequoteclose}\ {\isachardoublequoteopen}x\ {\isasymin}\ P{\isacharunderscore}{\kern0pt}set{\isacharparenleft}{\kern0pt}a{\isacharparenright}{\kern0pt}{\isachardoublequoteclose}\ \isacommand{unfolding}\isamarkupfalse%
\ P{\isacharunderscore}{\kern0pt}names{\isacharunderscore}{\kern0pt}def\ \isacommand{by}\isamarkupfalse%
\ auto\isanewline
\ \ \ \ \isacommand{then}\isamarkupfalse%
\ \isacommand{obtain}\isamarkupfalse%
\ b\ \isakeyword{where}\ bp\ {\isacharcolon}{\kern0pt}\ {\isachardoublequoteopen}a\ {\isacharless}{\kern0pt}\ b{\isachardoublequoteclose}\ {\isachardoublequoteopen}Limit{\isacharparenleft}{\kern0pt}b{\isacharparenright}{\kern0pt}{\isachardoublequoteclose}\ \isacommand{using}\isamarkupfalse%
\ ex{\isacharunderscore}{\kern0pt}larger{\isacharunderscore}{\kern0pt}limit\ \isacommand{by}\isamarkupfalse%
\ auto\ \isanewline
\ \ \ \ \isacommand{then}\isamarkupfalse%
\ \isacommand{have}\isamarkupfalse%
\ p{\isadigit{1}}\ {\isacharcolon}{\kern0pt}\ {\isachardoublequoteopen}P{\isacharunderscore}{\kern0pt}set{\isacharparenleft}{\kern0pt}b{\isacharparenright}{\kern0pt}\ {\isacharequal}{\kern0pt}\ {\isacharparenleft}{\kern0pt}{\isasymUnion}c\ {\isacharless}{\kern0pt}\ b{\isachardot}{\kern0pt}\ P{\isacharunderscore}{\kern0pt}set{\isacharparenleft}{\kern0pt}c{\isacharparenright}{\kern0pt}{\isacharparenright}{\kern0pt}{\isachardoublequoteclose}\ \isacommand{using}\isamarkupfalse%
\ P{\isacharunderscore}{\kern0pt}set{\isacharunderscore}{\kern0pt}lim\ \isacommand{by}\isamarkupfalse%
\ auto\isanewline
\ \ \ \ \isacommand{have}\isamarkupfalse%
\ {\isachardoublequoteopen}x\ {\isasymin}\ {\isacharparenleft}{\kern0pt}{\isasymUnion}c\ {\isacharless}{\kern0pt}\ b{\isachardot}{\kern0pt}\ P{\isacharunderscore}{\kern0pt}set{\isacharparenleft}{\kern0pt}c{\isacharparenright}{\kern0pt}{\isacharparenright}{\kern0pt}{\isachardoublequoteclose}\ \isanewline
\ \ \ \ \ \ \isacommand{apply}\isamarkupfalse%
\ {\isacharparenleft}{\kern0pt}rule{\isacharunderscore}{\kern0pt}tac\ a{\isacharequal}{\kern0pt}a\ \isakeyword{in}\ OUN{\isacharunderscore}{\kern0pt}RepFunI{\isacharparenright}{\kern0pt}\ \isacommand{using}\isamarkupfalse%
\ ap\ bp\ \isacommand{by}\isamarkupfalse%
\ auto\isanewline
\ \ \ \ \isacommand{then}\isamarkupfalse%
\ \isacommand{have}\isamarkupfalse%
\ {\isachardoublequoteopen}x\ {\isasymin}\ P{\isacharunderscore}{\kern0pt}set{\isacharparenleft}{\kern0pt}b{\isacharparenright}{\kern0pt}{\isachardoublequoteclose}\ \isacommand{using}\isamarkupfalse%
\ p{\isadigit{1}}\ \isacommand{by}\isamarkupfalse%
\ auto\ \isanewline
\ \ \ \ \isacommand{then}\isamarkupfalse%
\ \isacommand{show}\isamarkupfalse%
\ {\isachardoublequoteopen}x\ {\isasymin}\ P{\isacharunderscore}{\kern0pt}set{\isacharparenleft}{\kern0pt}F{\isacharparenleft}{\kern0pt}x{\isacharparenright}{\kern0pt}{\isacharparenright}{\kern0pt}\ {\isasymand}\ Limit{\isacharparenleft}{\kern0pt}F{\isacharparenleft}{\kern0pt}x{\isacharparenright}{\kern0pt}{\isacharparenright}{\kern0pt}{\isachardoublequoteclose}\isanewline
\ \ \ \ \ \ \isacommand{unfolding}\isamarkupfalse%
\ F{\isacharunderscore}{\kern0pt}def\ \isanewline
\ \ \ \ \ \ \isacommand{apply}\isamarkupfalse%
\ {\isacharparenleft}{\kern0pt}rule{\isacharunderscore}{\kern0pt}tac\ i{\isacharequal}{\kern0pt}b\ \isakeyword{in}\ LeastI{\isacharparenright}{\kern0pt}\isanewline
\ \ \ \ \ \ \isacommand{using}\isamarkupfalse%
\ bp\ Limit{\isacharunderscore}{\kern0pt}is{\isacharunderscore}{\kern0pt}Ord\ \isacommand{by}\isamarkupfalse%
\ auto\ \isanewline
\ \ \isacommand{qed}\isamarkupfalse%
\isanewline
\ \ \isacommand{define}\isamarkupfalse%
\ I\ \isakeyword{where}\ {\isachardoublequoteopen}I\ {\isasymequiv}\ {\isasymUnion}{\isacharbraceleft}{\kern0pt}\ F{\isacharparenleft}{\kern0pt}x{\isacharparenright}{\kern0pt}{\isachardot}{\kern0pt}\ x\ {\isasymin}\ A\ {\isacharbraceright}{\kern0pt}{\isachardoublequoteclose}\ \isanewline
\ \ \isacommand{then}\isamarkupfalse%
\ \isacommand{have}\isamarkupfalse%
\ ilim\ {\isacharcolon}{\kern0pt}\ {\isachardoublequoteopen}Limit{\isacharparenleft}{\kern0pt}I{\isacharparenright}{\kern0pt}{\isachardoublequoteclose}\ \isanewline
\ \ \ \ \isacommand{unfolding}\isamarkupfalse%
\ I{\isacharunderscore}{\kern0pt}def\isanewline
\ \ \ \ \isacommand{apply}\isamarkupfalse%
\ {\isacharparenleft}{\kern0pt}rule{\isacharunderscore}{\kern0pt}tac\ Limit{\isacharunderscore}{\kern0pt}Union{\isacharparenright}{\kern0pt}\isanewline
\ \ \ \ \isacommand{using}\isamarkupfalse%
\ assm\ p{\isadigit{1}}\ \isacommand{by}\isamarkupfalse%
\ auto\isanewline
\ \ \isacommand{then}\isamarkupfalse%
\ \isacommand{have}\isamarkupfalse%
\ {\isachardoublequoteopen}{\isasymforall}\ x\ {\isasymin}\ A{\isachardot}{\kern0pt}\ F{\isacharparenleft}{\kern0pt}x{\isacharparenright}{\kern0pt}\ {\isasymle}\ I{\isachardoublequoteclose}\ \isanewline
\ \ \ \ \isacommand{apply}\isamarkupfalse%
\ clarify\isanewline
\ \ \ \ \isacommand{unfolding}\isamarkupfalse%
\ I{\isacharunderscore}{\kern0pt}def\isanewline
\ \ \ \ \isacommand{apply}\isamarkupfalse%
\ {\isacharparenleft}{\kern0pt}rule{\isacharunderscore}{\kern0pt}tac\ a{\isacharequal}{\kern0pt}x\ \isakeyword{and}\ b{\isacharequal}{\kern0pt}F\ \isakeyword{in}\ UN{\isacharunderscore}{\kern0pt}upper{\isacharunderscore}{\kern0pt}le{\isacharparenright}{\kern0pt}\ \isanewline
\ \ \ \ \isacommand{using}\isamarkupfalse%
\ Limit{\isacharunderscore}{\kern0pt}is{\isacharunderscore}{\kern0pt}Ord\ p{\isadigit{1}}\ \isacommand{by}\isamarkupfalse%
\ auto\isanewline
\ \ \isacommand{then}\isamarkupfalse%
\ \isacommand{have}\isamarkupfalse%
\ {\isachardoublequoteopen}{\isasymforall}x\ {\isasymin}\ A{\isachardot}{\kern0pt}\ x\ {\isasymin}\ P{\isacharunderscore}{\kern0pt}set{\isacharparenleft}{\kern0pt}I{\isacharparenright}{\kern0pt}{\isachardoublequoteclose}\isanewline
\ \ \ \ \isacommand{apply}\isamarkupfalse%
\ clarify\isanewline
\ \ \ \ \isacommand{apply}\isamarkupfalse%
\ {\isacharparenleft}{\kern0pt}rule{\isacharunderscore}{\kern0pt}tac\ i{\isacharequal}{\kern0pt}{\isachardoublequoteopen}F{\isacharparenleft}{\kern0pt}x{\isacharparenright}{\kern0pt}{\isachardoublequoteclose}\ \isakeyword{and}\ j{\isacharequal}{\kern0pt}I\ \isakeyword{in}\ leE{\isacharparenright}{\kern0pt}\isanewline
\ \ \isacommand{proof}\isamarkupfalse%
\ {\isacharparenleft}{\kern0pt}auto{\isacharparenright}{\kern0pt}\isanewline
\ \ \ \ \isacommand{fix}\isamarkupfalse%
\ x\ \isacommand{assume}\isamarkupfalse%
\ {\isachardoublequoteopen}x\ {\isasymin}\ A{\isachardoublequoteclose}\ {\isachardoublequoteopen}F{\isacharparenleft}{\kern0pt}x{\isacharparenright}{\kern0pt}\ {\isacharless}{\kern0pt}\ I{\isachardoublequoteclose}\ \isanewline
\ \ \ \ \isacommand{then}\isamarkupfalse%
\ \isacommand{have}\isamarkupfalse%
\ {\isachardoublequoteopen}x\ {\isasymin}\ {\isacharparenleft}{\kern0pt}{\isasymUnion}c\ {\isacharless}{\kern0pt}\ I{\isachardot}{\kern0pt}\ P{\isacharunderscore}{\kern0pt}set{\isacharparenleft}{\kern0pt}c{\isacharparenright}{\kern0pt}{\isacharparenright}{\kern0pt}{\isachardoublequoteclose}\ \isanewline
\ \ \ \ \ \ \isacommand{apply}\isamarkupfalse%
\ {\isacharparenleft}{\kern0pt}rule{\isacharunderscore}{\kern0pt}tac\ a{\isacharequal}{\kern0pt}{\isachardoublequoteopen}F{\isacharparenleft}{\kern0pt}x{\isacharparenright}{\kern0pt}{\isachardoublequoteclose}\ \isakeyword{in}\ OUN{\isacharunderscore}{\kern0pt}RepFunI{\isacharparenright}{\kern0pt}\isanewline
\ \ \ \ \ \ \isacommand{using}\isamarkupfalse%
\ p{\isadigit{1}}\ \isacommand{by}\isamarkupfalse%
\ auto\ \isanewline
\ \ \ \ \isacommand{then}\isamarkupfalse%
\ \isacommand{show}\isamarkupfalse%
\ {\isachardoublequoteopen}x\ {\isasymin}\ P{\isacharunderscore}{\kern0pt}set{\isacharparenleft}{\kern0pt}I{\isacharparenright}{\kern0pt}\ {\isachardoublequoteclose}\ \isacommand{using}\isamarkupfalse%
\ P{\isacharunderscore}{\kern0pt}set{\isacharunderscore}{\kern0pt}lim\ ilim\ \isacommand{by}\isamarkupfalse%
\ auto\isanewline
\ \ \isacommand{next}\isamarkupfalse%
\ \isanewline
\ \ \ \ \isacommand{fix}\isamarkupfalse%
\ x\ \isacommand{assume}\isamarkupfalse%
\ {\isachardoublequoteopen}x\ {\isasymin}\ A{\isachardoublequoteclose}\isanewline
\ \ \ \ \isacommand{then}\isamarkupfalse%
\ \isacommand{show}\isamarkupfalse%
\ {\isachardoublequoteopen}x\ {\isasymin}\ P{\isacharunderscore}{\kern0pt}set{\isacharparenleft}{\kern0pt}F{\isacharparenleft}{\kern0pt}x{\isacharparenright}{\kern0pt}{\isacharparenright}{\kern0pt}{\isachardoublequoteclose}\ \isacommand{using}\isamarkupfalse%
\ p{\isadigit{1}}\ \isacommand{by}\isamarkupfalse%
\ auto\ \isanewline
\ \ \isacommand{qed}\isamarkupfalse%
\isanewline
\ \ \isacommand{then}\isamarkupfalse%
\ \isacommand{show}\isamarkupfalse%
\ {\isacharquery}{\kern0pt}thesis\ \ \ \isanewline
\ \ \ \ \isacommand{apply}\isamarkupfalse%
\ {\isacharparenleft}{\kern0pt}rule{\isacharunderscore}{\kern0pt}tac\ x{\isacharequal}{\kern0pt}I\ \isakeyword{in}\ exI{\isacharparenright}{\kern0pt}\isanewline
\ \ \ \ \isacommand{using}\isamarkupfalse%
\ ilim\ \isacommand{by}\isamarkupfalse%
\ auto\isanewline
\isacommand{qed}\isamarkupfalse%
%
\endisatagproof
{\isafoldproof}%
%
\isadelimproof
\isanewline
%
\endisadelimproof
\isanewline
\isacommand{lemma}\isamarkupfalse%
\ P{\isacharunderscore}{\kern0pt}name{\isacharunderscore}{\kern0pt}iff\ {\isacharcolon}{\kern0pt}\ {\isachardoublequoteopen}x\ {\isasymin}\ P{\isacharunderscore}{\kern0pt}names\ {\isasymlongleftrightarrow}\ {\isacharparenleft}{\kern0pt}x\ {\isasymin}\ M\ {\isasymand}\ x\ {\isasymsubseteq}\ P{\isacharunderscore}{\kern0pt}names\ {\isasymtimes}\ P{\isacharparenright}{\kern0pt}{\isachardoublequoteclose}\ \isanewline
%
\isadelimproof
%
\endisadelimproof
%
\isatagproof
\isacommand{proof}\isamarkupfalse%
\ {\isacharparenleft}{\kern0pt}rule\ iffI{\isacharparenright}{\kern0pt}\isanewline
\ \ \isacommand{assume}\isamarkupfalse%
\ assm\ {\isacharcolon}{\kern0pt}\ {\isachardoublequoteopen}x\ {\isasymin}\ P{\isacharunderscore}{\kern0pt}names{\isachardoublequoteclose}\ \isanewline
\ \ \isacommand{then}\isamarkupfalse%
\ \isacommand{obtain}\isamarkupfalse%
\ a\ \isakeyword{where}\ aH{\isacharcolon}{\kern0pt}\ {\isachardoublequoteopen}Ord{\isacharparenleft}{\kern0pt}a{\isacharparenright}{\kern0pt}{\isachardoublequoteclose}\ {\isachardoublequoteopen}x\ {\isasymin}\ P{\isacharunderscore}{\kern0pt}set{\isacharparenleft}{\kern0pt}succ{\isacharparenleft}{\kern0pt}a{\isacharparenright}{\kern0pt}{\isacharparenright}{\kern0pt}{\isachardoublequoteclose}\ \isanewline
\ \ \ \ \isacommand{using}\isamarkupfalse%
\ P{\isacharunderscore}{\kern0pt}names{\isacharunderscore}{\kern0pt}in{\isacharunderscore}{\kern0pt}P{\isacharunderscore}{\kern0pt}set{\isacharunderscore}{\kern0pt}succ\ \isanewline
\ \ \ \ \isacommand{by}\isamarkupfalse%
\ auto\ \isanewline
\ \ \isacommand{then}\isamarkupfalse%
\ \isacommand{have}\isamarkupfalse%
\ xsubset\ {\isacharcolon}{\kern0pt}\ {\isachardoublequoteopen}x\ {\isasymsubseteq}\ P{\isacharunderscore}{\kern0pt}set{\isacharparenleft}{\kern0pt}a{\isacharparenright}{\kern0pt}\ {\isasymtimes}\ P{\isachardoublequoteclose}\ \isanewline
\ \ \ \ \isacommand{using}\isamarkupfalse%
\ P{\isacharunderscore}{\kern0pt}set{\isacharunderscore}{\kern0pt}succ\ \isanewline
\ \ \ \ \isacommand{by}\isamarkupfalse%
\ auto\ \isanewline
\ \ \isacommand{have}\isamarkupfalse%
\ {\isachardoublequoteopen}x\ {\isasymsubseteq}\ P{\isacharunderscore}{\kern0pt}names\ {\isasymtimes}\ P{\isachardoublequoteclose}\ \isanewline
\ \ \isacommand{proof}\isamarkupfalse%
\ {\isacharparenleft}{\kern0pt}rule\ subsetI{\isacharparenright}{\kern0pt}\isanewline
\ \ \ \ \isacommand{fix}\isamarkupfalse%
\ v\ \isacommand{assume}\isamarkupfalse%
\ {\isachardoublequoteopen}v\ {\isasymin}\ x{\isachardoublequoteclose}\ \isanewline
\ \ \ \ \isacommand{then}\isamarkupfalse%
\ \isacommand{obtain}\isamarkupfalse%
\ y\ p\ \isakeyword{where}\ ypH\ {\isacharcolon}{\kern0pt}\ {\isachardoublequoteopen}v\ {\isacharequal}{\kern0pt}\ {\isacharless}{\kern0pt}y{\isacharcomma}{\kern0pt}\ p{\isachargreater}{\kern0pt}{\isachardoublequoteclose}\ {\isachardoublequoteopen}y\ {\isasymin}\ P{\isacharunderscore}{\kern0pt}set{\isacharparenleft}{\kern0pt}a{\isacharparenright}{\kern0pt}{\isachardoublequoteclose}\ {\isachardoublequoteopen}p\ {\isasymin}\ P{\isachardoublequoteclose}\ \isacommand{using}\isamarkupfalse%
\ xsubset\ \isacommand{by}\isamarkupfalse%
\ auto\isanewline
\ \ \ \ \isacommand{then}\isamarkupfalse%
\ \isacommand{have}\isamarkupfalse%
\ {\isachardoublequoteopen}y\ {\isasymin}\ P{\isacharunderscore}{\kern0pt}names{\isachardoublequoteclose}\ \isacommand{using}\isamarkupfalse%
\ P{\isacharunderscore}{\kern0pt}namesI\ aH\ \isacommand{by}\isamarkupfalse%
\ auto\isanewline
\ \ \ \ \isacommand{then}\isamarkupfalse%
\ \isacommand{show}\isamarkupfalse%
\ {\isachardoublequoteopen}v\ {\isasymin}\ P{\isacharunderscore}{\kern0pt}names\ {\isasymtimes}\ P{\isachardoublequoteclose}\ \isacommand{using}\isamarkupfalse%
\ ypH\ \isacommand{by}\isamarkupfalse%
\ auto\isanewline
\ \ \isacommand{qed}\isamarkupfalse%
\isanewline
\ \ \isacommand{then}\isamarkupfalse%
\ \isacommand{show}\isamarkupfalse%
\ {\isachardoublequoteopen}x\ {\isasymin}\ M\ {\isasymand}\ x\ {\isasymsubseteq}\ P{\isacharunderscore}{\kern0pt}names\ {\isasymtimes}\ P{\isachardoublequoteclose}\ \isacommand{using}\isamarkupfalse%
\ P{\isacharunderscore}{\kern0pt}name{\isacharunderscore}{\kern0pt}in{\isacharunderscore}{\kern0pt}M\ assm\ \isacommand{by}\isamarkupfalse%
\ auto\isanewline
\isacommand{next}\isamarkupfalse%
\ \isanewline
\ \ \isacommand{assume}\isamarkupfalse%
\ assms\ {\isacharcolon}{\kern0pt}\ {\isachardoublequoteopen}\ x\ {\isasymin}\ M\ {\isasymand}\ x\ {\isasymsubseteq}\ P{\isacharunderscore}{\kern0pt}names\ {\isasymtimes}\ P\ {\isachardoublequoteclose}\ \isanewline
\ \ \isacommand{then}\isamarkupfalse%
\ \isacommand{have}\isamarkupfalse%
\ {\isachardoublequoteopen}domain{\isacharparenleft}{\kern0pt}x{\isacharparenright}{\kern0pt}\ {\isasymsubseteq}\ P{\isacharunderscore}{\kern0pt}names{\isachardoublequoteclose}\ \isacommand{by}\isamarkupfalse%
\ auto\ \isanewline
\ \ \isacommand{then}\isamarkupfalse%
\ \isacommand{obtain}\isamarkupfalse%
\ a\ \isakeyword{where}\ aH\ {\isacharcolon}{\kern0pt}\ {\isachardoublequoteopen}Limit{\isacharparenleft}{\kern0pt}a{\isacharparenright}{\kern0pt}{\isachardoublequoteclose}\ {\isachardoublequoteopen}{\isasymforall}y\ {\isasymin}\ domain{\isacharparenleft}{\kern0pt}x{\isacharparenright}{\kern0pt}{\isachardot}{\kern0pt}\ y\ {\isasymin}\ P{\isacharunderscore}{\kern0pt}set{\isacharparenleft}{\kern0pt}a{\isacharparenright}{\kern0pt}{\isachardoublequoteclose}\ \isanewline
\ \ \ \ \isacommand{using}\isamarkupfalse%
\ set{\isacharunderscore}{\kern0pt}of{\isacharunderscore}{\kern0pt}P{\isacharunderscore}{\kern0pt}names{\isacharunderscore}{\kern0pt}in{\isacharunderscore}{\kern0pt}P{\isacharunderscore}{\kern0pt}set\ \isacommand{by}\isamarkupfalse%
\ blast\isanewline
\ \ \isacommand{have}\isamarkupfalse%
\ {\isachardoublequoteopen}x\ {\isasymsubseteq}\ P{\isacharunderscore}{\kern0pt}set{\isacharparenleft}{\kern0pt}a{\isacharparenright}{\kern0pt}\ {\isasymtimes}\ P{\isachardoublequoteclose}\ \isanewline
\ \ \isacommand{proof}\isamarkupfalse%
\ {\isacharparenleft}{\kern0pt}rule\ subsetI{\isacharparenright}{\kern0pt}\isanewline
\ \ \ \ \isacommand{fix}\isamarkupfalse%
\ v\ \isacommand{assume}\isamarkupfalse%
\ vin\ {\isacharcolon}{\kern0pt}\ {\isachardoublequoteopen}v\ {\isasymin}\ x{\isachardoublequoteclose}\ \isanewline
\ \ \ \ \isacommand{then}\isamarkupfalse%
\ \isacommand{obtain}\isamarkupfalse%
\ y\ p\ \isakeyword{where}\ ypH\ {\isacharcolon}{\kern0pt}\ {\isachardoublequoteopen}y\ {\isasymin}\ P{\isacharunderscore}{\kern0pt}names{\isachardoublequoteclose}\ {\isachardoublequoteopen}p\ {\isasymin}\ P{\isachardoublequoteclose}\ {\isachardoublequoteopen}v\ {\isacharequal}{\kern0pt}\ {\isacharless}{\kern0pt}y{\isacharcomma}{\kern0pt}\ p{\isachargreater}{\kern0pt}{\isachardoublequoteclose}\ \isacommand{using}\isamarkupfalse%
\ assms\ \isacommand{by}\isamarkupfalse%
\ auto\ \isanewline
\ \ \ \ \isacommand{then}\isamarkupfalse%
\ \isacommand{have}\isamarkupfalse%
\ {\isachardoublequoteopen}y\ {\isasymin}\ P{\isacharunderscore}{\kern0pt}set{\isacharparenleft}{\kern0pt}a{\isacharparenright}{\kern0pt}{\isachardoublequoteclose}\ \isacommand{using}\isamarkupfalse%
\ vin\ ypH\ aH\ \isacommand{by}\isamarkupfalse%
\ auto\ \isanewline
\ \ \ \ \isacommand{then}\isamarkupfalse%
\ \isacommand{show}\isamarkupfalse%
\ {\isachardoublequoteopen}v\ {\isasymin}\ P{\isacharunderscore}{\kern0pt}set{\isacharparenleft}{\kern0pt}a{\isacharparenright}{\kern0pt}\ {\isasymtimes}\ P{\isachardoublequoteclose}\ \isacommand{using}\isamarkupfalse%
\ ypH\ vin\ \isacommand{by}\isamarkupfalse%
\ auto\isanewline
\ \ \isacommand{qed}\isamarkupfalse%
\isanewline
\ \ \isacommand{then}\isamarkupfalse%
\ \isacommand{have}\isamarkupfalse%
\ {\isachardoublequoteopen}x\ {\isasymin}\ P{\isacharunderscore}{\kern0pt}set{\isacharparenleft}{\kern0pt}succ{\isacharparenleft}{\kern0pt}a{\isacharparenright}{\kern0pt}{\isacharparenright}{\kern0pt}{\isachardoublequoteclose}\ \isacommand{using}\isamarkupfalse%
\ P{\isacharunderscore}{\kern0pt}set{\isacharunderscore}{\kern0pt}succ\ assms\ \isacommand{by}\isamarkupfalse%
\ auto\isanewline
\ \ \isacommand{then}\isamarkupfalse%
\ \isacommand{show}\isamarkupfalse%
\ {\isachardoublequoteopen}x\ {\isasymin}\ P{\isacharunderscore}{\kern0pt}names{\isachardoublequoteclose}\ \isanewline
\ \ \ \ \isacommand{apply}\isamarkupfalse%
{\isacharparenleft}{\kern0pt}rule{\isacharunderscore}{\kern0pt}tac\ P{\isacharunderscore}{\kern0pt}namesI{\isacharparenright}{\kern0pt}\isanewline
\ \ \ \ \isacommand{using}\isamarkupfalse%
\ Limit{\isacharunderscore}{\kern0pt}is{\isacharunderscore}{\kern0pt}Ord\ aH\ \isanewline
\ \ \ \ \isacommand{by}\isamarkupfalse%
\ auto\isanewline
\isacommand{qed}\isamarkupfalse%
%
\endisatagproof
{\isafoldproof}%
%
\isadelimproof
\isanewline
%
\endisadelimproof
\isanewline
\isacommand{lemma}\isamarkupfalse%
\ zero{\isacharunderscore}{\kern0pt}is{\isacharunderscore}{\kern0pt}P{\isacharunderscore}{\kern0pt}name\ {\isacharcolon}{\kern0pt}\ {\isachardoublequoteopen}{\isadigit{0}}\ {\isasymin}\ P{\isacharunderscore}{\kern0pt}names{\isachardoublequoteclose}\ \isanewline
%
\isadelimproof
%
\endisadelimproof
%
\isatagproof
\isacommand{proof}\isamarkupfalse%
\ {\isacharminus}{\kern0pt}\ \isanewline
\ \ \isacommand{have}\isamarkupfalse%
\ p{\isadigit{1}}\ {\isacharcolon}{\kern0pt}\ {\isachardoublequoteopen}{\isadigit{0}}\ {\isasymin}\ Pow{\isacharparenleft}{\kern0pt}{\isadigit{0}}{\isacharparenright}{\kern0pt}\ {\isasyminter}\ M{\isachardoublequoteclose}\ \isacommand{using}\isamarkupfalse%
\ zero{\isacharunderscore}{\kern0pt}in{\isacharunderscore}{\kern0pt}M\ \isacommand{by}\isamarkupfalse%
\ auto\isanewline
\ \ \isacommand{have}\isamarkupfalse%
\ p{\isadigit{2}}\ {\isacharcolon}{\kern0pt}\ {\isachardoublequoteopen}Pow{\isacharparenleft}{\kern0pt}{\isadigit{0}}{\isacharparenright}{\kern0pt}\ {\isasyminter}\ M\ {\isacharequal}{\kern0pt}\ P{\isacharunderscore}{\kern0pt}set{\isacharparenleft}{\kern0pt}{\isadigit{1}}{\isacharparenright}{\kern0pt}{\isachardoublequoteclose}\ \isacommand{using}\isamarkupfalse%
\ P{\isacharunderscore}{\kern0pt}set{\isacharunderscore}{\kern0pt}{\isadigit{0}}\ P{\isacharunderscore}{\kern0pt}set{\isacharunderscore}{\kern0pt}succ\ \isacommand{by}\isamarkupfalse%
\ auto\ \isanewline
\ \ \isacommand{then}\isamarkupfalse%
\ \isacommand{have}\isamarkupfalse%
\ {\isachardoublequoteopen}{\isadigit{0}}\ {\isasymin}\ P{\isacharunderscore}{\kern0pt}set{\isacharparenleft}{\kern0pt}{\isadigit{1}}{\isacharparenright}{\kern0pt}{\isachardoublequoteclose}\ \isacommand{using}\isamarkupfalse%
\ p{\isadigit{1}}\ \isacommand{by}\isamarkupfalse%
\ auto\ \isanewline
\ \ \isacommand{then}\isamarkupfalse%
\ \isacommand{show}\isamarkupfalse%
\ {\isachardoublequoteopen}{\isadigit{0}}\ {\isasymin}\ P{\isacharunderscore}{\kern0pt}names{\isachardoublequoteclose}\isanewline
\ \ \ \ \isacommand{unfolding}\isamarkupfalse%
\ P{\isacharunderscore}{\kern0pt}names{\isacharunderscore}{\kern0pt}def\ \isacommand{apply}\isamarkupfalse%
\ auto\ \ \isanewline
\ \ \ \ \isacommand{using}\isamarkupfalse%
\ zero{\isacharunderscore}{\kern0pt}in{\isacharunderscore}{\kern0pt}M\ \isacommand{by}\isamarkupfalse%
\ auto\ \isanewline
\isacommand{qed}\isamarkupfalse%
%
\endisatagproof
{\isafoldproof}%
%
\isadelimproof
\ \isanewline
%
\endisadelimproof
\isanewline
\isacommand{lemma}\isamarkupfalse%
\ P{\isacharunderscore}{\kern0pt}rank{\isacharunderscore}{\kern0pt}zero\ {\isacharcolon}{\kern0pt}\ {\isachardoublequoteopen}P{\isacharunderscore}{\kern0pt}rank{\isacharparenleft}{\kern0pt}{\isadigit{0}}{\isacharparenright}{\kern0pt}\ {\isacharequal}{\kern0pt}\ {\isadigit{0}}{\isachardoublequoteclose}\ \isanewline
%
\isadelimproof
%
\endisadelimproof
%
\isatagproof
\isacommand{proof}\isamarkupfalse%
\ {\isacharminus}{\kern0pt}\ \isanewline
\ \ \isacommand{have}\isamarkupfalse%
\ p{\isadigit{1}}\ {\isacharcolon}{\kern0pt}\ {\isachardoublequoteopen}{\isadigit{0}}\ {\isasymin}\ Pow{\isacharparenleft}{\kern0pt}{\isadigit{0}}{\isacharparenright}{\kern0pt}\ {\isasyminter}\ M{\isachardoublequoteclose}\ \isacommand{using}\isamarkupfalse%
\ zero{\isacharunderscore}{\kern0pt}in{\isacharunderscore}{\kern0pt}M\ \isacommand{by}\isamarkupfalse%
\ auto\isanewline
\ \ \isacommand{have}\isamarkupfalse%
\ p{\isadigit{2}}\ {\isacharcolon}{\kern0pt}\ {\isachardoublequoteopen}Pow{\isacharparenleft}{\kern0pt}{\isadigit{0}}{\isacharparenright}{\kern0pt}\ {\isasyminter}\ M\ {\isacharequal}{\kern0pt}\ P{\isacharunderscore}{\kern0pt}set{\isacharparenleft}{\kern0pt}{\isadigit{1}}{\isacharparenright}{\kern0pt}{\isachardoublequoteclose}\ \isacommand{using}\isamarkupfalse%
\ P{\isacharunderscore}{\kern0pt}set{\isacharunderscore}{\kern0pt}{\isadigit{0}}\ P{\isacharunderscore}{\kern0pt}set{\isacharunderscore}{\kern0pt}succ\ \isacommand{by}\isamarkupfalse%
\ auto\ \isanewline
\ \ \isacommand{then}\isamarkupfalse%
\ \isacommand{have}\isamarkupfalse%
\ {\isachardoublequoteopen}{\isadigit{0}}\ {\isasymin}\ P{\isacharunderscore}{\kern0pt}set{\isacharparenleft}{\kern0pt}{\isadigit{1}}{\isacharparenright}{\kern0pt}{\isachardoublequoteclose}\ \isacommand{using}\isamarkupfalse%
\ p{\isadigit{1}}\ \isacommand{by}\isamarkupfalse%
\ auto\ \isanewline
\ \ \isacommand{then}\isamarkupfalse%
\ \isacommand{show}\isamarkupfalse%
\ {\isachardoublequoteopen}P{\isacharunderscore}{\kern0pt}rank{\isacharparenleft}{\kern0pt}{\isadigit{0}}{\isacharparenright}{\kern0pt}\ {\isacharequal}{\kern0pt}\ {\isadigit{0}}{\isachardoublequoteclose}\ \isanewline
\ \ \ \ \isacommand{unfolding}\isamarkupfalse%
\ P{\isacharunderscore}{\kern0pt}rank{\isacharunderscore}{\kern0pt}def\ \isanewline
\ \ \ \ \isacommand{apply}\isamarkupfalse%
{\isacharparenleft}{\kern0pt}rule{\isacharunderscore}{\kern0pt}tac\ Least{\isacharunderscore}{\kern0pt}equality{\isacharcomma}{\kern0pt}\ auto{\isacharparenright}{\kern0pt}\isanewline
\ \ \ \ \isacommand{done}\isamarkupfalse%
\isanewline
\isacommand{qed}\isamarkupfalse%
%
\endisatagproof
{\isafoldproof}%
%
\isadelimproof
\isanewline
%
\endisadelimproof
\ \ \isanewline
\isacommand{lemma}\isamarkupfalse%
\ P{\isacharunderscore}{\kern0pt}name{\isacharunderscore}{\kern0pt}induct\ {\isacharcolon}{\kern0pt}\ \isanewline
\ \ {\isachardoublequoteopen}{\isasymforall}x\ {\isasymin}\ P{\isacharunderscore}{\kern0pt}names{\isachardot}{\kern0pt}\ {\isacharparenleft}{\kern0pt}{\isasymforall}y{\isachardot}{\kern0pt}\ {\isasymforall}\ p{\isachardot}{\kern0pt}\ {\isacharless}{\kern0pt}y{\isacharcomma}{\kern0pt}\ p{\isachargreater}{\kern0pt}\ {\isasymin}\ x\ {\isasymlongrightarrow}\ Q{\isacharparenleft}{\kern0pt}y{\isacharparenright}{\kern0pt}{\isacharparenright}{\kern0pt}\ {\isasymlongrightarrow}\ Q{\isacharparenleft}{\kern0pt}x{\isacharparenright}{\kern0pt}\ {\isasymLongrightarrow}\ x\ {\isasymin}\ P{\isacharunderscore}{\kern0pt}names\ {\isasymLongrightarrow}\ Q{\isacharparenleft}{\kern0pt}x{\isacharparenright}{\kern0pt}{\isachardoublequoteclose}\isanewline
%
\isadelimproof
\ \ %
\endisadelimproof
%
\isatagproof
\isacommand{apply}\isamarkupfalse%
\ {\isacharparenleft}{\kern0pt}rule{\isacharunderscore}{\kern0pt}tac\ P{\isacharunderscore}{\kern0pt}rank{\isacharunderscore}{\kern0pt}induct{\isacharparenright}{\kern0pt}\isanewline
\isacommand{proof}\isamarkupfalse%
\ {\isacharparenleft}{\kern0pt}clarify{\isacharparenright}{\kern0pt}\isanewline
\ \ \isacommand{fix}\isamarkupfalse%
\ x\ \isacommand{assume}\isamarkupfalse%
\ assms{\isacharcolon}{\kern0pt}\ \isanewline
\ \ \ \ {\isachardoublequoteopen}x\ {\isasymin}\ P{\isacharunderscore}{\kern0pt}names{\isachardoublequoteclose}\ {\isachardoublequoteopen}{\isasymforall}y{\isasymin}P{\isacharunderscore}{\kern0pt}names{\isachardot}{\kern0pt}\ P{\isacharunderscore}{\kern0pt}rank{\isacharparenleft}{\kern0pt}y{\isacharparenright}{\kern0pt}\ {\isacharless}{\kern0pt}\ P{\isacharunderscore}{\kern0pt}rank{\isacharparenleft}{\kern0pt}x{\isacharparenright}{\kern0pt}\ {\isasymlongrightarrow}\ Q{\isacharparenleft}{\kern0pt}y{\isacharparenright}{\kern0pt}{\isachardoublequoteclose}\isanewline
\ \ \ \ {\isachardoublequoteopen}{\isasymforall}x{\isasymin}P{\isacharunderscore}{\kern0pt}names{\isachardot}{\kern0pt}\ {\isacharparenleft}{\kern0pt}{\isasymforall}y\ p{\isachardot}{\kern0pt}\ {\isasymlangle}y{\isacharcomma}{\kern0pt}\ p{\isasymrangle}\ {\isasymin}\ x\ {\isasymlongrightarrow}\ Q{\isacharparenleft}{\kern0pt}y{\isacharparenright}{\kern0pt}{\isacharparenright}{\kern0pt}\ {\isasymlongrightarrow}\ Q{\isacharparenleft}{\kern0pt}x{\isacharparenright}{\kern0pt}{\isachardoublequoteclose}\ \isanewline
\ \ \isacommand{then}\isamarkupfalse%
\ \isacommand{have}\isamarkupfalse%
\ p{\isadigit{1}}\ {\isacharcolon}{\kern0pt}\ \ {\isachardoublequoteopen}{\isacharparenleft}{\kern0pt}{\isasymforall}y\ p{\isachardot}{\kern0pt}\ {\isasymlangle}y{\isacharcomma}{\kern0pt}\ p{\isasymrangle}\ {\isasymin}\ x\ {\isasymlongrightarrow}\ Q{\isacharparenleft}{\kern0pt}y{\isacharparenright}{\kern0pt}{\isacharparenright}{\kern0pt}\ {\isasymlongrightarrow}\ Q{\isacharparenleft}{\kern0pt}x{\isacharparenright}{\kern0pt}{\isachardoublequoteclose}\ \isacommand{by}\isamarkupfalse%
\ auto\ \isanewline
\ \ \isacommand{have}\isamarkupfalse%
\ p{\isadigit{2}}\ {\isacharcolon}{\kern0pt}\ {\isachardoublequoteopen}{\isasymforall}y\ p{\isachardot}{\kern0pt}\ {\isacharless}{\kern0pt}y{\isacharcomma}{\kern0pt}\ p{\isachargreater}{\kern0pt}\ {\isasymin}\ x\ {\isasymlongrightarrow}\ P{\isacharunderscore}{\kern0pt}rank{\isacharparenleft}{\kern0pt}y{\isacharparenright}{\kern0pt}\ {\isacharless}{\kern0pt}\ P{\isacharunderscore}{\kern0pt}rank{\isacharparenleft}{\kern0pt}x{\isacharparenright}{\kern0pt}{\isachardoublequoteclose}\ \isanewline
\ \ \ \ \isacommand{using}\isamarkupfalse%
\ domain{\isacharunderscore}{\kern0pt}P{\isacharunderscore}{\kern0pt}rank{\isacharunderscore}{\kern0pt}lt\ assms\ \isacommand{by}\isamarkupfalse%
\ auto\isanewline
\ \ \isanewline
\ \ \isacommand{then}\isamarkupfalse%
\ \isacommand{have}\isamarkupfalse%
\ {\isachardoublequoteopen}{\isasymforall}y\ p{\isachardot}{\kern0pt}\ {\isacharless}{\kern0pt}y{\isacharcomma}{\kern0pt}\ p{\isachargreater}{\kern0pt}\ {\isasymin}\ x\ {\isasymlongrightarrow}\ Q{\isacharparenleft}{\kern0pt}y{\isacharparenright}{\kern0pt}{\isachardoublequoteclose}\ \isanewline
\ \ \isacommand{proof}\isamarkupfalse%
\ {\isacharparenleft}{\kern0pt}clarify{\isacharparenright}{\kern0pt}\isanewline
\ \ \ \ \isacommand{fix}\isamarkupfalse%
\ y\ p\ \isacommand{assume}\isamarkupfalse%
\ p{\isadigit{3}}{\isadigit{0}}\ {\isacharcolon}{\kern0pt}\ {\isachardoublequoteopen}{\isacharless}{\kern0pt}y{\isacharcomma}{\kern0pt}\ p{\isachargreater}{\kern0pt}\ {\isasymin}\ x{\isachardoublequoteclose}\isanewline
\ \ \ \ \isacommand{then}\isamarkupfalse%
\ \isacommand{have}\isamarkupfalse%
\ p{\isadigit{3}}\ {\isacharcolon}{\kern0pt}\ {\isachardoublequoteopen}P{\isacharunderscore}{\kern0pt}rank{\isacharparenleft}{\kern0pt}y{\isacharparenright}{\kern0pt}\ {\isacharless}{\kern0pt}\ P{\isacharunderscore}{\kern0pt}rank{\isacharparenleft}{\kern0pt}x{\isacharparenright}{\kern0pt}{\isachardoublequoteclose}\ \isacommand{using}\isamarkupfalse%
\ p{\isadigit{2}}\ \isacommand{by}\isamarkupfalse%
\ auto\ \isanewline
\ \ \ \ \isacommand{then}\isamarkupfalse%
\ \isacommand{have}\isamarkupfalse%
\ {\isachardoublequoteopen}y\ {\isasymin}\ P{\isacharunderscore}{\kern0pt}names{\isachardoublequoteclose}\ \isacommand{using}\isamarkupfalse%
\ P{\isacharunderscore}{\kern0pt}name{\isacharunderscore}{\kern0pt}domain{\isacharunderscore}{\kern0pt}P{\isacharunderscore}{\kern0pt}name\ p{\isadigit{3}}{\isadigit{0}}\ assms\ \isacommand{by}\isamarkupfalse%
\ auto\ \isanewline
\ \ \ \ \isacommand{then}\isamarkupfalse%
\ \isacommand{show}\isamarkupfalse%
\ {\isachardoublequoteopen}Q{\isacharparenleft}{\kern0pt}y{\isacharparenright}{\kern0pt}{\isachardoublequoteclose}\ \isacommand{using}\isamarkupfalse%
\ assms\ p{\isadigit{3}}\ \isacommand{by}\isamarkupfalse%
\ auto\ \isanewline
\ \ \isacommand{qed}\isamarkupfalse%
\ \isanewline
\ \ \isacommand{then}\isamarkupfalse%
\ \isacommand{show}\isamarkupfalse%
\ {\isachardoublequoteopen}Q{\isacharparenleft}{\kern0pt}x{\isacharparenright}{\kern0pt}{\isachardoublequoteclose}\ \isacommand{using}\isamarkupfalse%
\ assms\ \isacommand{by}\isamarkupfalse%
\ auto\ \isanewline
\isacommand{qed}\isamarkupfalse%
%
\endisatagproof
{\isafoldproof}%
%
\isadelimproof
\isanewline
%
\endisadelimproof
\ \ \isanewline
\isanewline
\isacommand{lemma}\isamarkupfalse%
\ check{\isacharunderscore}{\kern0pt}P{\isacharunderscore}{\kern0pt}name\ {\isacharcolon}{\kern0pt}\ {\isachardoublequoteopen}x\ {\isasymin}\ M\ {\isasymlongrightarrow}\ check{\isacharparenleft}{\kern0pt}x{\isacharparenright}{\kern0pt}\ {\isasymin}\ P{\isacharunderscore}{\kern0pt}names{\isachardoublequoteclose}\ \isanewline
%
\isadelimproof
\ \ %
\endisadelimproof
%
\isatagproof
\isacommand{apply}\isamarkupfalse%
\ {\isacharparenleft}{\kern0pt}rule{\isacharunderscore}{\kern0pt}tac\ P\ {\isacharequal}{\kern0pt}\ {\isachardoublequoteopen}{\isasymlambda}x{\isachardot}{\kern0pt}\ x\ {\isasymin}\ M\ {\isasymlongrightarrow}\ check{\isacharparenleft}{\kern0pt}x{\isacharparenright}{\kern0pt}\ {\isasymin}\ P{\isacharunderscore}{\kern0pt}names{\isachardoublequoteclose}\ \isakeyword{in}\ eps{\isacharunderscore}{\kern0pt}induct{\isacharparenright}{\kern0pt}\ \isanewline
\ \ \isacommand{apply}\isamarkupfalse%
\ {\isacharparenleft}{\kern0pt}clarify{\isacharparenright}{\kern0pt}\isanewline
\isacommand{proof}\isamarkupfalse%
{\isacharminus}{\kern0pt}\isanewline
\ \ \isacommand{fix}\isamarkupfalse%
\ x\ \isacommand{assume}\isamarkupfalse%
\ assm\ {\isacharcolon}{\kern0pt}\ {\isachardoublequoteopen}{\isasymforall}y{\isasymin}x{\isachardot}{\kern0pt}\ y\ {\isasymin}\ M\ {\isasymlongrightarrow}\ check{\isacharparenleft}{\kern0pt}y{\isacharparenright}{\kern0pt}\ {\isasymin}\ P{\isacharunderscore}{\kern0pt}names{\isachardoublequoteclose}\ {\isachardoublequoteopen}x\ {\isasymin}\ M{\isachardoublequoteclose}\isanewline
\ \ \isacommand{have}\isamarkupfalse%
\ p{\isadigit{1}}{\isacharcolon}{\kern0pt}\ {\isachardoublequoteopen}check{\isacharparenleft}{\kern0pt}x{\isacharparenright}{\kern0pt}\ {\isacharequal}{\kern0pt}\ {\isacharbraceleft}{\kern0pt}\ {\isacharless}{\kern0pt}check{\isacharparenleft}{\kern0pt}y{\isacharparenright}{\kern0pt}{\isacharcomma}{\kern0pt}\ one{\isachargreater}{\kern0pt}\ {\isachardot}{\kern0pt}\ y\ {\isasymin}\ x\ {\isacharbraceright}{\kern0pt}{\isachardoublequoteclose}\ \isacommand{using}\isamarkupfalse%
\ def{\isacharunderscore}{\kern0pt}check\ \isacommand{by}\isamarkupfalse%
\ assumption\isanewline
\ \ \isacommand{then}\isamarkupfalse%
\ \isacommand{show}\isamarkupfalse%
\ {\isachardoublequoteopen}check{\isacharparenleft}{\kern0pt}x{\isacharparenright}{\kern0pt}\ {\isasymin}\ P{\isacharunderscore}{\kern0pt}names{\isachardoublequoteclose}\ \isanewline
\ \ \isacommand{proof}\isamarkupfalse%
\ {\isacharparenleft}{\kern0pt}cases\ {\isachardoublequoteopen}x\ {\isacharequal}{\kern0pt}\ {\isadigit{0}}{\isachardoublequoteclose}{\isacharparenright}{\kern0pt}\isanewline
\ \ \ \ \isacommand{case}\isamarkupfalse%
\ True\isanewline
\ \ \ \ \isacommand{then}\isamarkupfalse%
\ \isacommand{have}\isamarkupfalse%
\ {\isachardoublequoteopen}check{\isacharparenleft}{\kern0pt}x{\isacharparenright}{\kern0pt}\ {\isacharequal}{\kern0pt}\ {\isadigit{0}}{\isachardoublequoteclose}\ \isacommand{using}\isamarkupfalse%
\ p{\isadigit{1}}\ \isacommand{by}\isamarkupfalse%
\ auto\ \isanewline
\ \ \ \ \isacommand{then}\isamarkupfalse%
\ \isacommand{show}\isamarkupfalse%
\ {\isacharquery}{\kern0pt}thesis\ \isacommand{using}\isamarkupfalse%
\ zero{\isacharunderscore}{\kern0pt}is{\isacharunderscore}{\kern0pt}P{\isacharunderscore}{\kern0pt}name\ \isacommand{by}\isamarkupfalse%
\ auto\ \isanewline
\ \ \isacommand{next}\isamarkupfalse%
\isanewline
\ \ \ \ \isacommand{case}\isamarkupfalse%
\ False\isanewline
\ \ \ \ \isacommand{define}\isamarkupfalse%
\ A\ \isakeyword{where}\ {\isachardoublequoteopen}A\ {\isasymequiv}\ {\isacharbraceleft}{\kern0pt}\ check{\isacharparenleft}{\kern0pt}y{\isacharparenright}{\kern0pt}{\isachardot}{\kern0pt}\ y\ {\isasymin}\ x\ {\isacharbraceright}{\kern0pt}{\isachardoublequoteclose}\ \isanewline
\ \ \ \ \isacommand{then}\isamarkupfalse%
\ \isacommand{have}\isamarkupfalse%
\ asubset{\isacharcolon}{\kern0pt}\ \ {\isachardoublequoteopen}A\ {\isasymsubseteq}\ P{\isacharunderscore}{\kern0pt}names{\isachardoublequoteclose}\isanewline
\ \ \ \ \ \ \isacommand{apply}\isamarkupfalse%
\ {\isacharparenleft}{\kern0pt}rule{\isacharunderscore}{\kern0pt}tac\ subsetI{\isacharparenright}{\kern0pt}\ \isanewline
\ \ \ \ \ \ \isacommand{using}\isamarkupfalse%
\ assm\ transM\ \isacommand{by}\isamarkupfalse%
\ auto\isanewline
\ \ \ \ \isacommand{have}\isamarkupfalse%
\ {\isachardoublequoteopen}A\ {\isasymnoteq}\ {\isadigit{0}}{\isachardoublequoteclose}\ \isacommand{using}\isamarkupfalse%
\ {\isacartoucheopen}x\ {\isasymnoteq}\ {\isadigit{0}}{\isacartoucheclose}\ \isacommand{unfolding}\isamarkupfalse%
\ A{\isacharunderscore}{\kern0pt}def\ \isacommand{by}\isamarkupfalse%
\ auto\ \isanewline
\ \ \ \ \isacommand{then}\isamarkupfalse%
\ \isacommand{have}\isamarkupfalse%
\ {\isachardoublequoteopen}{\isasymexists}\ a{\isachardot}{\kern0pt}\ {\isacharparenleft}{\kern0pt}Limit{\isacharparenleft}{\kern0pt}a{\isacharparenright}{\kern0pt}\ {\isasymand}\ {\isacharparenleft}{\kern0pt}{\isasymforall}\ z\ {\isasymin}\ A{\isachardot}{\kern0pt}\ z\ {\isasymin}\ P{\isacharunderscore}{\kern0pt}set{\isacharparenleft}{\kern0pt}a{\isacharparenright}{\kern0pt}{\isacharparenright}{\kern0pt}{\isacharparenright}{\kern0pt}{\isachardoublequoteclose}\isanewline
\ \ \ \ \ \ \isacommand{using}\isamarkupfalse%
\ set{\isacharunderscore}{\kern0pt}of{\isacharunderscore}{\kern0pt}P{\isacharunderscore}{\kern0pt}names{\isacharunderscore}{\kern0pt}in{\isacharunderscore}{\kern0pt}P{\isacharunderscore}{\kern0pt}set\ asubset\ \isacommand{by}\isamarkupfalse%
\ auto\isanewline
\ \ \ \ \isacommand{then}\isamarkupfalse%
\ \isacommand{obtain}\isamarkupfalse%
\ a\ \isakeyword{where}\ ap\ {\isacharcolon}{\kern0pt}\ {\isachardoublequoteopen}Limit{\isacharparenleft}{\kern0pt}a{\isacharparenright}{\kern0pt}{\isachardoublequoteclose}\ {\isachardoublequoteopen}{\isacharparenleft}{\kern0pt}{\isasymforall}\ z\ {\isasymin}\ A{\isachardot}{\kern0pt}\ z\ {\isasymin}\ P{\isacharunderscore}{\kern0pt}set{\isacharparenleft}{\kern0pt}a{\isacharparenright}{\kern0pt}{\isacharparenright}{\kern0pt}{\isachardoublequoteclose}\ \isacommand{by}\isamarkupfalse%
\ auto\ \isanewline
\ \ \ \ \isacommand{then}\isamarkupfalse%
\ \isacommand{have}\isamarkupfalse%
\ p{\isadigit{1}}{\isacharcolon}{\kern0pt}\ \ {\isachardoublequoteopen}check{\isacharparenleft}{\kern0pt}x{\isacharparenright}{\kern0pt}\ {\isasymin}\ Pow{\isacharparenleft}{\kern0pt}P{\isacharunderscore}{\kern0pt}set{\isacharparenleft}{\kern0pt}a{\isacharparenright}{\kern0pt}\ {\isasymtimes}\ P{\isacharparenright}{\kern0pt}{\isachardoublequoteclose}\ \isanewline
\ \ \ \ \ \ \isacommand{using}\isamarkupfalse%
\ A{\isacharunderscore}{\kern0pt}def\ p{\isadigit{1}}\ one{\isacharunderscore}{\kern0pt}in{\isacharunderscore}{\kern0pt}P\ check{\isacharunderscore}{\kern0pt}in{\isacharunderscore}{\kern0pt}M\ \isacommand{by}\isamarkupfalse%
\ auto\ \isanewline
\ \ \ \ \isacommand{then}\isamarkupfalse%
\ \isacommand{have}\isamarkupfalse%
\ p{\isadigit{2}}\ {\isacharcolon}{\kern0pt}\ {\isachardoublequoteopen}check{\isacharparenleft}{\kern0pt}x{\isacharparenright}{\kern0pt}\ {\isasymin}\ Pow{\isacharparenleft}{\kern0pt}P{\isacharunderscore}{\kern0pt}set{\isacharparenleft}{\kern0pt}a{\isacharparenright}{\kern0pt}\ {\isasymtimes}\ P{\isacharparenright}{\kern0pt}\ {\isasyminter}\ M{\isachardoublequoteclose}\ \isanewline
\ \ \ \ \ \ \isacommand{using}\isamarkupfalse%
\ check{\isacharunderscore}{\kern0pt}in{\isacharunderscore}{\kern0pt}M\ assm\ \isacommand{by}\isamarkupfalse%
\ auto\ \isanewline
\ \ \ \ \isacommand{then}\isamarkupfalse%
\ \isacommand{have}\isamarkupfalse%
\ {\isachardoublequoteopen}check{\isacharparenleft}{\kern0pt}x{\isacharparenright}{\kern0pt}\ {\isasymin}\ P{\isacharunderscore}{\kern0pt}set{\isacharparenleft}{\kern0pt}succ{\isacharparenleft}{\kern0pt}a{\isacharparenright}{\kern0pt}{\isacharparenright}{\kern0pt}{\isachardoublequoteclose}\ \isanewline
\ \ \ \ \ \ \isacommand{using}\isamarkupfalse%
\ P{\isacharunderscore}{\kern0pt}set{\isacharunderscore}{\kern0pt}succ\ \isacommand{by}\isamarkupfalse%
\ auto\ \isanewline
\ \ \ \ \isacommand{then}\isamarkupfalse%
\ \isacommand{show}\isamarkupfalse%
\ {\isachardoublequoteopen}check{\isacharparenleft}{\kern0pt}x{\isacharparenright}{\kern0pt}\ {\isasymin}\ P{\isacharunderscore}{\kern0pt}names{\isachardoublequoteclose}\isanewline
\ \ \ \ \ \ \isacommand{unfolding}\isamarkupfalse%
\ P{\isacharunderscore}{\kern0pt}names{\isacharunderscore}{\kern0pt}def\ \isacommand{using}\isamarkupfalse%
\ p{\isadigit{2}}\ \isacommand{apply}\isamarkupfalse%
\ simp\ \isanewline
\ \ \ \ \ \ \isacommand{apply}\isamarkupfalse%
\ {\isacharparenleft}{\kern0pt}rule{\isacharunderscore}{\kern0pt}tac\ x{\isacharequal}{\kern0pt}{\isachardoublequoteopen}succ{\isacharparenleft}{\kern0pt}a{\isacharparenright}{\kern0pt}{\isachardoublequoteclose}\ \isakeyword{in}\ exI{\isacharparenright}{\kern0pt}\isanewline
\ \ \ \ \ \ \isacommand{apply}\isamarkupfalse%
\ simp\ \isanewline
\ \ \ \ \ \ \isacommand{using}\isamarkupfalse%
\ ap\ Limit{\isacharunderscore}{\kern0pt}is{\isacharunderscore}{\kern0pt}Ord\ \isacommand{by}\isamarkupfalse%
\ auto\ \isanewline
\ \ \isacommand{qed}\isamarkupfalse%
\isanewline
\isacommand{qed}\isamarkupfalse%
%
\endisatagproof
{\isafoldproof}%
%
\isadelimproof
\isanewline
%
\endisadelimproof
\isanewline
\isacommand{end}\isamarkupfalse%
\isanewline
%
\isadelimtheory
%
\endisadelimtheory
%
\isatagtheory
\isacommand{end}\isamarkupfalse%
%
\endisatagtheory
{\isafoldtheory}%
%
\isadelimtheory
%
\endisadelimtheory
%
\end{isabellebody}%
\endinput
%:%file=~/source/repos/ZF-notAC/code/P_Names.thy%:%
%:%10=1%:%
%:%11=1%:%
%:%12=2%:%
%:%13=3%:%
%:%14=4%:%
%:%15=5%:%
%:%20=5%:%
%:%23=6%:%
%:%24=7%:%
%:%25=7%:%
%:%26=8%:%
%:%27=9%:%
%:%28=10%:%
%:%29=10%:%
%:%30=11%:%
%:%31=12%:%
%:%32=13%:%
%:%33=13%:%
%:%34=14%:%
%:%35=15%:%
%:%36=16%:%
%:%37=16%:%
%:%39=16%:%
%:%43=16%:%
%:%44=16%:%
%:%45=16%:%
%:%46=16%:%
%:%53=16%:%
%:%54=17%:%
%:%55=18%:%
%:%56=18%:%
%:%57=19%:%
%:%60=20%:%
%:%64=20%:%
%:%65=20%:%
%:%66=21%:%
%:%67=21%:%
%:%68=22%:%
%:%69=22%:%
%:%70=22%:%
%:%71=22%:%
%:%72=22%:%
%:%73=23%:%
%:%74=23%:%
%:%75=23%:%
%:%76=23%:%
%:%77=23%:%
%:%78=24%:%
%:%84=24%:%
%:%87=25%:%
%:%88=26%:%
%:%89=26%:%
%:%90=27%:%
%:%93=28%:%
%:%97=28%:%
%:%98=28%:%
%:%99=28%:%
%:%100=28%:%
%:%105=28%:%
%:%108=29%:%
%:%109=30%:%
%:%110=30%:%
%:%111=31%:%
%:%112=32%:%
%:%113=32%:%
%:%114=33%:%
%:%115=34%:%
%:%116=35%:%
%:%117=35%:%
%:%120=36%:%
%:%124=36%:%
%:%125=36%:%
%:%126=36%:%
%:%131=36%:%
%:%134=37%:%
%:%135=38%:%
%:%136=38%:%
%:%137=39%:%
%:%144=40%:%
%:%145=40%:%
%:%146=41%:%
%:%147=41%:%
%:%148=42%:%
%:%149=42%:%
%:%150=42%:%
%:%151=43%:%
%:%152=43%:%
%:%153=44%:%
%:%154=45%:%
%:%155=45%:%
%:%156=46%:%
%:%157=46%:%
%:%158=46%:%
%:%159=47%:%
%:%160=47%:%
%:%161=48%:%
%:%162=48%:%
%:%163=49%:%
%:%164=49%:%
%:%165=49%:%
%:%166=50%:%
%:%167=51%:%
%:%168=52%:%
%:%169=53%:%
%:%170=53%:%
%:%171=53%:%
%:%172=53%:%
%:%173=53%:%
%:%174=54%:%
%:%175=54%:%
%:%176=54%:%
%:%177=54%:%
%:%178=55%:%
%:%179=55%:%
%:%180=55%:%
%:%181=55%:%
%:%182=55%:%
%:%183=56%:%
%:%184=56%:%
%:%185=56%:%
%:%186=56%:%
%:%187=56%:%
%:%188=57%:%
%:%189=57%:%
%:%190=57%:%
%:%191=57%:%
%:%192=57%:%
%:%193=58%:%
%:%194=58%:%
%:%195=58%:%
%:%196=58%:%
%:%197=58%:%
%:%198=59%:%
%:%199=59%:%
%:%200=60%:%
%:%201=60%:%
%:%202=60%:%
%:%203=60%:%
%:%204=60%:%
%:%205=61%:%
%:%211=61%:%
%:%214=62%:%
%:%215=63%:%
%:%216=63%:%
%:%223=64%:%
%:%224=64%:%
%:%225=65%:%
%:%226=65%:%
%:%227=66%:%
%:%228=66%:%
%:%229=66%:%
%:%230=66%:%
%:%231=66%:%
%:%232=67%:%
%:%233=67%:%
%:%234=67%:%
%:%235=67%:%
%:%236=67%:%
%:%237=68%:%
%:%238=68%:%
%:%239=68%:%
%:%240=68%:%
%:%241=68%:%
%:%242=69%:%
%:%243=69%:%
%:%244=69%:%
%:%245=69%:%
%:%246=69%:%
%:%247=70%:%
%:%253=70%:%
%:%256=71%:%
%:%257=72%:%
%:%258=72%:%
%:%259=73%:%
%:%266=74%:%
%:%267=74%:%
%:%268=75%:%
%:%269=75%:%
%:%270=76%:%
%:%271=76%:%
%:%272=76%:%
%:%273=76%:%
%:%274=76%:%
%:%275=77%:%
%:%276=78%:%
%:%277=78%:%
%:%278=79%:%
%:%279=79%:%
%:%280=79%:%
%:%281=79%:%
%:%282=79%:%
%:%283=80%:%
%:%284=80%:%
%:%285=80%:%
%:%286=81%:%
%:%287=81%:%
%:%288=81%:%
%:%289=81%:%
%:%290=82%:%
%:%291=82%:%
%:%292=82%:%
%:%293=82%:%
%:%294=82%:%
%:%295=83%:%
%:%296=83%:%
%:%297=83%:%
%:%298=83%:%
%:%299=83%:%
%:%300=84%:%
%:%306=84%:%
%:%309=85%:%
%:%310=86%:%
%:%311=86%:%
%:%318=87%:%
%:%319=87%:%
%:%320=88%:%
%:%321=88%:%
%:%322=89%:%
%:%323=89%:%
%:%324=89%:%
%:%325=89%:%
%:%326=89%:%
%:%327=90%:%
%:%328=90%:%
%:%329=90%:%
%:%330=90%:%
%:%331=91%:%
%:%332=91%:%
%:%333=91%:%
%:%334=91%:%
%:%335=91%:%
%:%336=92%:%
%:%337=92%:%
%:%338=92%:%
%:%339=92%:%
%:%340=92%:%
%:%341=93%:%
%:%342=93%:%
%:%343=93%:%
%:%344=93%:%
%:%345=94%:%
%:%351=94%:%
%:%354=95%:%
%:%355=96%:%
%:%356=96%:%
%:%359=97%:%
%:%363=97%:%
%:%364=97%:%
%:%365=97%:%
%:%366=97%:%
%:%371=97%:%
%:%374=98%:%
%:%375=99%:%
%:%376=99%:%
%:%383=100%:%
%:%384=100%:%
%:%385=101%:%
%:%386=101%:%
%:%387=102%:%
%:%388=102%:%
%:%389=102%:%
%:%390=102%:%
%:%391=102%:%
%:%392=103%:%
%:%393=103%:%
%:%394=103%:%
%:%395=104%:%
%:%396=104%:%
%:%397=104%:%
%:%398=105%:%
%:%404=105%:%
%:%407=106%:%
%:%408=107%:%
%:%409=107%:%
%:%412=108%:%
%:%416=108%:%
%:%417=108%:%
%:%418=108%:%
%:%419=108%:%
%:%424=108%:%
%:%427=109%:%
%:%428=110%:%
%:%429=110%:%
%:%436=111%:%
%:%437=111%:%
%:%438=112%:%
%:%439=112%:%
%:%440=113%:%
%:%441=113%:%
%:%442=113%:%
%:%443=113%:%
%:%444=113%:%
%:%445=114%:%
%:%446=114%:%
%:%447=114%:%
%:%448=114%:%
%:%449=114%:%
%:%450=115%:%
%:%456=115%:%
%:%459=116%:%
%:%460=117%:%
%:%461=117%:%
%:%462=118%:%
%:%469=119%:%
%:%470=119%:%
%:%471=120%:%
%:%472=120%:%
%:%473=121%:%
%:%474=121%:%
%:%475=122%:%
%:%476=123%:%
%:%477=123%:%
%:%478=124%:%
%:%479=124%:%
%:%480=124%:%
%:%481=125%:%
%:%482=125%:%
%:%483=126%:%
%:%484=126%:%
%:%485=127%:%
%:%486=127%:%
%:%487=128%:%
%:%488=128%:%
%:%489=128%:%
%:%490=128%:%
%:%491=128%:%
%:%492=129%:%
%:%493=129%:%
%:%494=129%:%
%:%495=129%:%
%:%496=130%:%
%:%497=130%:%
%:%498=130%:%
%:%499=130%:%
%:%500=130%:%
%:%501=131%:%
%:%502=131%:%
%:%503=131%:%
%:%504=131%:%
%:%505=132%:%
%:%506=132%:%
%:%507=132%:%
%:%508=132%:%
%:%509=133%:%
%:%510=133%:%
%:%511=133%:%
%:%512=133%:%
%:%513=133%:%
%:%514=134%:%
%:%515=134%:%
%:%516=134%:%
%:%517=134%:%
%:%518=134%:%
%:%519=135%:%
%:%520=135%:%
%:%521=135%:%
%:%522=136%:%
%:%523=136%:%
%:%524=136%:%
%:%525=137%:%
%:%526=137%:%
%:%527=138%:%
%:%528=138%:%
%:%529=138%:%
%:%530=139%:%
%:%531=140%:%
%:%532=140%:%
%:%533=141%:%
%:%534=141%:%
%:%535=142%:%
%:%536=142%:%
%:%537=143%:%
%:%538=143%:%
%:%539=143%:%
%:%540=143%:%
%:%541=143%:%
%:%542=144%:%
%:%543=144%:%
%:%544=144%:%
%:%545=144%:%
%:%546=145%:%
%:%547=145%:%
%:%548=145%:%
%:%549=145%:%
%:%550=145%:%
%:%551=146%:%
%:%552=146%:%
%:%553=146%:%
%:%554=146%:%
%:%555=146%:%
%:%556=147%:%
%:%557=147%:%
%:%558=147%:%
%:%559=147%:%
%:%560=147%:%
%:%561=148%:%
%:%562=148%:%
%:%563=148%:%
%:%564=148%:%
%:%565=148%:%
%:%566=149%:%
%:%567=149%:%
%:%568=150%:%
%:%569=150%:%
%:%570=151%:%
%:%571=151%:%
%:%572=152%:%
%:%573=152%:%
%:%574=152%:%
%:%575=152%:%
%:%576=152%:%
%:%577=153%:%
%:%583=153%:%
%:%586=154%:%
%:%587=155%:%
%:%588=155%:%
%:%589=156%:%
%:%596=157%:%
%:%597=157%:%
%:%598=158%:%
%:%599=158%:%
%:%600=159%:%
%:%601=159%:%
%:%602=159%:%
%:%603=159%:%
%:%604=159%:%
%:%605=160%:%
%:%606=160%:%
%:%607=160%:%
%:%608=160%:%
%:%609=160%:%
%:%610=161%:%
%:%611=161%:%
%:%612=161%:%
%:%613=161%:%
%:%614=161%:%
%:%615=162%:%
%:%616=162%:%
%:%617=162%:%
%:%618=162%:%
%:%619=162%:%
%:%620=163%:%
%:%621=163%:%
%:%622=163%:%
%:%623=163%:%
%:%624=163%:%
%:%625=163%:%
%:%626=164%:%
%:%632=164%:%
%:%635=165%:%
%:%636=166%:%
%:%637=166%:%
%:%638=167%:%
%:%645=168%:%
%:%646=168%:%
%:%647=169%:%
%:%648=169%:%
%:%649=170%:%
%:%650=170%:%
%:%651=170%:%
%:%652=170%:%
%:%653=171%:%
%:%654=171%:%
%:%655=171%:%
%:%656=171%:%
%:%657=171%:%
%:%658=172%:%
%:%664=172%:%
%:%667=173%:%
%:%668=174%:%
%:%669=174%:%
%:%670=175%:%
%:%677=176%:%
%:%678=176%:%
%:%679=177%:%
%:%680=177%:%
%:%681=178%:%
%:%682=178%:%
%:%683=178%:%
%:%684=178%:%
%:%685=178%:%
%:%686=179%:%
%:%687=179%:%
%:%688=179%:%
%:%689=179%:%
%:%690=179%:%
%:%691=180%:%
%:%697=180%:%
%:%700=181%:%
%:%701=182%:%
%:%702=182%:%
%:%703=183%:%
%:%704=184%:%
%:%705=184%:%
%:%712=185%:%
%:%713=185%:%
%:%714=186%:%
%:%715=186%:%
%:%716=187%:%
%:%717=187%:%
%:%718=187%:%
%:%719=187%:%
%:%720=187%:%
%:%721=188%:%
%:%722=188%:%
%:%723=188%:%
%:%724=188%:%
%:%725=189%:%
%:%726=189%:%
%:%727=189%:%
%:%728=190%:%
%:%729=190%:%
%:%730=191%:%
%:%731=191%:%
%:%732=191%:%
%:%733=192%:%
%:%734=192%:%
%:%735=192%:%
%:%736=192%:%
%:%737=192%:%
%:%738=193%:%
%:%739=193%:%
%:%740=193%:%
%:%741=194%:%
%:%742=194%:%
%:%743=194%:%
%:%744=195%:%
%:%745=195%:%
%:%746=195%:%
%:%747=195%:%
%:%748=195%:%
%:%749=196%:%
%:%750=196%:%
%:%751=196%:%
%:%752=196%:%
%:%753=197%:%
%:%754=197%:%
%:%755=198%:%
%:%756=198%:%
%:%757=199%:%
%:%758=199%:%
%:%759=200%:%
%:%760=200%:%
%:%761=200%:%
%:%762=201%:%
%:%763=201%:%
%:%764=201%:%
%:%765=201%:%
%:%766=201%:%
%:%767=202%:%
%:%773=202%:%
%:%776=203%:%
%:%777=204%:%
%:%778=204%:%
%:%779=205%:%
%:%780=206%:%
%:%787=207%:%
%:%788=207%:%
%:%789=208%:%
%:%790=208%:%
%:%791=209%:%
%:%792=209%:%
%:%793=210%:%
%:%794=210%:%
%:%795=211%:%
%:%796=211%:%
%:%797=212%:%
%:%798=212%:%
%:%799=213%:%
%:%800=213%:%
%:%801=214%:%
%:%802=214%:%
%:%803=215%:%
%:%804=215%:%
%:%805=216%:%
%:%806=216%:%
%:%807=217%:%
%:%808=217%:%
%:%809=217%:%
%:%810=218%:%
%:%811=218%:%
%:%812=219%:%
%:%813=219%:%
%:%814=219%:%
%:%815=220%:%
%:%816=220%:%
%:%817=220%:%
%:%818=220%:%
%:%819=220%:%
%:%820=221%:%
%:%821=221%:%
%:%822=221%:%
%:%823=221%:%
%:%824=221%:%
%:%825=222%:%
%:%826=222%:%
%:%827=223%:%
%:%828=223%:%
%:%829=223%:%
%:%830=223%:%
%:%831=223%:%
%:%832=224%:%
%:%833=224%:%
%:%834=225%:%
%:%835=225%:%
%:%836=225%:%
%:%837=226%:%
%:%838=226%:%
%:%839=226%:%
%:%840=227%:%
%:%841=227%:%
%:%842=227%:%
%:%843=227%:%
%:%844=227%:%
%:%845=227%:%
%:%846=228%:%
%:%852=228%:%
%:%855=229%:%
%:%856=230%:%
%:%857=230%:%
%:%858=231%:%
%:%861=232%:%
%:%865=232%:%
%:%866=232%:%
%:%867=232%:%
%:%868=233%:%
%:%869=233%:%
%:%870=234%:%
%:%871=234%:%
%:%872=235%:%
%:%873=235%:%
%:%874=236%:%
%:%875=236%:%
%:%876=236%:%
%:%877=237%:%
%:%878=237%:%
%:%879=238%:%
%:%880=238%:%
%:%881=238%:%
%:%882=239%:%
%:%883=239%:%
%:%884=239%:%
%:%885=239%:%
%:%886=239%:%
%:%887=240%:%
%:%888=240%:%
%:%889=240%:%
%:%890=240%:%
%:%891=240%:%
%:%892=241%:%
%:%893=241%:%
%:%894=241%:%
%:%895=241%:%
%:%896=241%:%
%:%897=242%:%
%:%898=242%:%
%:%899=242%:%
%:%900=242%:%
%:%901=242%:%
%:%902=243%:%
%:%903=243%:%
%:%904=244%:%
%:%905=244%:%
%:%906=244%:%
%:%907=244%:%
%:%908=245%:%
%:%909=245%:%
%:%910=245%:%
%:%911=245%:%
%:%912=245%:%
%:%913=246%:%
%:%914=246%:%
%:%915=246%:%
%:%916=247%:%
%:%917=247%:%
%:%918=248%:%
%:%919=248%:%
%:%920=249%:%
%:%921=249%:%
%:%922=249%:%
%:%923=250%:%
%:%924=250%:%
%:%925=251%:%
%:%926=251%:%
%:%927=252%:%
%:%928=252%:%
%:%929=252%:%
%:%930=253%:%
%:%931=253%:%
%:%932=254%:%
%:%933=254%:%
%:%934=255%:%
%:%935=255%:%
%:%936=255%:%
%:%937=256%:%
%:%938=256%:%
%:%939=256%:%
%:%940=257%:%
%:%941=257%:%
%:%942=258%:%
%:%943=258%:%
%:%944=259%:%
%:%945=259%:%
%:%946=260%:%
%:%947=260%:%
%:%948=260%:%
%:%949=261%:%
%:%950=261%:%
%:%951=261%:%
%:%952=262%:%
%:%953=262%:%
%:%954=263%:%
%:%955=263%:%
%:%956=264%:%
%:%957=264%:%
%:%958=265%:%
%:%959=265%:%
%:%960=265%:%
%:%961=266%:%
%:%962=266%:%
%:%963=266%:%
%:%964=267%:%
%:%965=267%:%
%:%966=268%:%
%:%967=268%:%
%:%968=268%:%
%:%969=269%:%
%:%970=269%:%
%:%971=269%:%
%:%972=269%:%
%:%973=269%:%
%:%974=270%:%
%:%975=270%:%
%:%976=271%:%
%:%977=271%:%
%:%978=271%:%
%:%979=272%:%
%:%980=272%:%
%:%981=272%:%
%:%982=272%:%
%:%983=272%:%
%:%984=273%:%
%:%985=273%:%
%:%986=274%:%
%:%987=274%:%
%:%988=274%:%
%:%989=275%:%
%:%990=275%:%
%:%991=276%:%
%:%992=276%:%
%:%993=276%:%
%:%994=277%:%
%:%1000=277%:%
%:%1003=278%:%
%:%1004=279%:%
%:%1005=279%:%
%:%1012=280%:%
%:%1013=280%:%
%:%1014=281%:%
%:%1015=281%:%
%:%1016=282%:%
%:%1017=282%:%
%:%1018=282%:%
%:%1019=283%:%
%:%1020=283%:%
%:%1021=284%:%
%:%1022=284%:%
%:%1023=285%:%
%:%1024=285%:%
%:%1025=285%:%
%:%1026=286%:%
%:%1027=286%:%
%:%1028=287%:%
%:%1029=287%:%
%:%1030=288%:%
%:%1031=288%:%
%:%1032=289%:%
%:%1033=289%:%
%:%1034=290%:%
%:%1035=290%:%
%:%1036=290%:%
%:%1037=291%:%
%:%1038=291%:%
%:%1039=291%:%
%:%1040=291%:%
%:%1041=291%:%
%:%1042=292%:%
%:%1043=292%:%
%:%1044=292%:%
%:%1045=292%:%
%:%1046=292%:%
%:%1047=293%:%
%:%1048=293%:%
%:%1049=293%:%
%:%1050=293%:%
%:%1051=293%:%
%:%1052=294%:%
%:%1053=294%:%
%:%1054=295%:%
%:%1055=295%:%
%:%1056=295%:%
%:%1057=295%:%
%:%1058=295%:%
%:%1059=296%:%
%:%1060=296%:%
%:%1061=297%:%
%:%1062=297%:%
%:%1063=298%:%
%:%1064=298%:%
%:%1065=298%:%
%:%1066=298%:%
%:%1067=299%:%
%:%1068=299%:%
%:%1069=299%:%
%:%1070=300%:%
%:%1071=300%:%
%:%1072=300%:%
%:%1073=301%:%
%:%1074=301%:%
%:%1075=302%:%
%:%1076=302%:%
%:%1077=303%:%
%:%1078=303%:%
%:%1079=303%:%
%:%1080=304%:%
%:%1081=304%:%
%:%1082=304%:%
%:%1083=304%:%
%:%1084=304%:%
%:%1085=305%:%
%:%1086=305%:%
%:%1087=305%:%
%:%1088=305%:%
%:%1089=305%:%
%:%1090=306%:%
%:%1091=306%:%
%:%1092=306%:%
%:%1093=306%:%
%:%1094=306%:%
%:%1095=307%:%
%:%1096=307%:%
%:%1097=308%:%
%:%1098=308%:%
%:%1099=308%:%
%:%1100=308%:%
%:%1101=308%:%
%:%1102=309%:%
%:%1103=309%:%
%:%1104=309%:%
%:%1105=310%:%
%:%1106=310%:%
%:%1107=311%:%
%:%1108=311%:%
%:%1109=312%:%
%:%1110=312%:%
%:%1111=313%:%
%:%1117=313%:%
%:%1120=314%:%
%:%1121=315%:%
%:%1122=315%:%
%:%1129=316%:%
%:%1130=316%:%
%:%1131=317%:%
%:%1132=317%:%
%:%1133=317%:%
%:%1134=317%:%
%:%1135=318%:%
%:%1136=318%:%
%:%1137=318%:%
%:%1138=318%:%
%:%1139=319%:%
%:%1140=319%:%
%:%1141=319%:%
%:%1142=319%:%
%:%1143=319%:%
%:%1144=320%:%
%:%1145=320%:%
%:%1146=320%:%
%:%1147=321%:%
%:%1148=321%:%
%:%1149=321%:%
%:%1150=322%:%
%:%1151=322%:%
%:%1152=322%:%
%:%1153=323%:%
%:%1159=323%:%
%:%1162=324%:%
%:%1163=325%:%
%:%1164=325%:%
%:%1171=326%:%
%:%1172=326%:%
%:%1173=327%:%
%:%1174=327%:%
%:%1175=327%:%
%:%1176=327%:%
%:%1177=328%:%
%:%1178=328%:%
%:%1179=328%:%
%:%1180=328%:%
%:%1181=329%:%
%:%1182=329%:%
%:%1183=329%:%
%:%1184=329%:%
%:%1185=329%:%
%:%1186=330%:%
%:%1187=330%:%
%:%1188=330%:%
%:%1189=331%:%
%:%1190=331%:%
%:%1191=332%:%
%:%1192=332%:%
%:%1193=333%:%
%:%1194=333%:%
%:%1195=334%:%
%:%1201=334%:%
%:%1204=335%:%
%:%1205=336%:%
%:%1206=336%:%
%:%1207=337%:%
%:%1210=338%:%
%:%1214=338%:%
%:%1215=338%:%
%:%1216=339%:%
%:%1217=339%:%
%:%1218=340%:%
%:%1219=340%:%
%:%1220=340%:%
%:%1221=341%:%
%:%1222=342%:%
%:%1223=343%:%
%:%1224=343%:%
%:%1225=343%:%
%:%1226=343%:%
%:%1227=344%:%
%:%1228=344%:%
%:%1229=345%:%
%:%1230=345%:%
%:%1231=345%:%
%:%1232=346%:%
%:%1233=347%:%
%:%1234=347%:%
%:%1235=347%:%
%:%1236=348%:%
%:%1237=348%:%
%:%1238=349%:%
%:%1239=349%:%
%:%1240=349%:%
%:%1241=350%:%
%:%1242=350%:%
%:%1243=350%:%
%:%1244=350%:%
%:%1245=350%:%
%:%1246=351%:%
%:%1247=351%:%
%:%1248=351%:%
%:%1249=351%:%
%:%1250=351%:%
%:%1251=352%:%
%:%1252=352%:%
%:%1253=352%:%
%:%1254=352%:%
%:%1255=352%:%
%:%1256=353%:%
%:%1257=353%:%
%:%1258=354%:%
%:%1259=354%:%
%:%1260=354%:%
%:%1261=354%:%
%:%1262=354%:%
%:%1263=355%:%
%:%1269=355%:%
%:%1272=356%:%
%:%1273=357%:%
%:%1274=358%:%
%:%1275=358%:%
%:%1278=359%:%
%:%1282=359%:%
%:%1283=359%:%
%:%1284=360%:%
%:%1285=360%:%
%:%1286=361%:%
%:%1287=361%:%
%:%1288=362%:%
%:%1289=362%:%
%:%1290=362%:%
%:%1291=363%:%
%:%1292=363%:%
%:%1293=363%:%
%:%1294=363%:%
%:%1295=364%:%
%:%1296=364%:%
%:%1297=364%:%
%:%1298=365%:%
%:%1299=365%:%
%:%1300=366%:%
%:%1301=366%:%
%:%1302=367%:%
%:%1303=367%:%
%:%1304=367%:%
%:%1305=367%:%
%:%1306=367%:%
%:%1307=368%:%
%:%1308=368%:%
%:%1309=368%:%
%:%1310=368%:%
%:%1311=368%:%
%:%1312=369%:%
%:%1313=369%:%
%:%1314=370%:%
%:%1315=370%:%
%:%1316=371%:%
%:%1317=371%:%
%:%1318=372%:%
%:%1319=372%:%
%:%1320=372%:%
%:%1321=373%:%
%:%1322=373%:%
%:%1323=374%:%
%:%1324=374%:%
%:%1325=374%:%
%:%1326=375%:%
%:%1327=375%:%
%:%1328=375%:%
%:%1329=375%:%
%:%1330=375%:%
%:%1331=376%:%
%:%1332=376%:%
%:%1333=376%:%
%:%1334=377%:%
%:%1335=377%:%
%:%1336=377%:%
%:%1337=378%:%
%:%1338=378%:%
%:%1339=378%:%
%:%1340=378%:%
%:%1341=379%:%
%:%1342=379%:%
%:%1343=379%:%
%:%1344=380%:%
%:%1345=380%:%
%:%1346=380%:%
%:%1347=381%:%
%:%1348=381%:%
%:%1349=381%:%
%:%1350=382%:%
%:%1351=382%:%
%:%1352=382%:%
%:%1353=383%:%
%:%1354=383%:%
%:%1355=383%:%
%:%1356=384%:%
%:%1357=384%:%
%:%1358=384%:%
%:%1359=385%:%
%:%1360=385%:%
%:%1361=385%:%
%:%1362=386%:%
%:%1363=386%:%
%:%1364=386%:%
%:%1365=386%:%
%:%1366=387%:%
%:%1367=387%:%
%:%1368=388%:%
%:%1369=388%:%
%:%1370=389%:%
%:%1371=389%:%
%:%1372=389%:%
%:%1373=390%:%
%:%1374=390%:%
%:%1375=391%:%
%:%1381=391%:%
%:%1384=392%:%
%:%1385=393%:%
%:%1386=393%:%
%:%1393=394%:%

%
\begin{isabellebody}%
\setisabellecontext{P{\isacharunderscore}{\kern0pt}Names{\isacharunderscore}{\kern0pt}M}%
%
\isadelimtheory
%
\endisadelimtheory
%
\isatagtheory
\isacommand{theory}\isamarkupfalse%
\ P{\isacharunderscore}{\kern0pt}Names{\isacharunderscore}{\kern0pt}M\isanewline
\ \ \isakeyword{imports}\ \isanewline
\ \ \ \ {\isachardoublequoteopen}Forcing{\isacharslash}{\kern0pt}Forcing{\isacharunderscore}{\kern0pt}Main{\isachardoublequoteclose}\isanewline
\ \ \ \ RecFun{\isacharunderscore}{\kern0pt}M{\isacharunderscore}{\kern0pt}Memrel\isanewline
\ \ \ \ P{\isacharunderscore}{\kern0pt}Names\isanewline
\isakeyword{begin}%
\endisatagtheory
{\isafoldtheory}%
%
\isadelimtheory
\ \isanewline
%
\endisadelimtheory
\isanewline
\isacommand{context}\isamarkupfalse%
\ forcing{\isacharunderscore}{\kern0pt}data\isanewline
\isakeyword{begin}\ \isanewline
\isanewline
\isacommand{definition}\isamarkupfalse%
\ is{\isacharunderscore}{\kern0pt}{\isadigit{1}}\ \isakeyword{where}\ {\isachardoublequoteopen}is{\isacharunderscore}{\kern0pt}{\isadigit{1}}{\isacharparenleft}{\kern0pt}x{\isacharparenright}{\kern0pt}\ {\isasymequiv}\ {\isasymforall}y\ {\isasymin}\ M{\isachardot}{\kern0pt}\ y\ {\isasymin}\ x\ {\isasymlongleftrightarrow}\ empty{\isacharparenleft}{\kern0pt}{\isacharhash}{\kern0pt}{\isacharhash}{\kern0pt}M{\isacharcomma}{\kern0pt}\ y{\isacharparenright}{\kern0pt}{\isachardoublequoteclose}\ \isanewline
\isanewline
\isacommand{lemma}\isamarkupfalse%
\ is{\isacharunderscore}{\kern0pt}{\isadigit{1}}D\ {\isacharcolon}{\kern0pt}\ {\isachardoublequoteopen}A\ {\isasymin}\ M\ {\isasymLongrightarrow}\ {\isasymforall}x\ {\isasymin}\ M{\isachardot}{\kern0pt}\ x\ {\isasymin}\ A\ {\isasymlongleftrightarrow}\ x\ {\isacharequal}{\kern0pt}\ {\isadigit{0}}\ {\isasymLongrightarrow}\ A\ {\isacharequal}{\kern0pt}\ {\isadigit{1}}{\isachardoublequoteclose}\ \isanewline
%
\isadelimproof
\ \ %
\endisadelimproof
%
\isatagproof
\isacommand{apply}\isamarkupfalse%
{\isacharparenleft}{\kern0pt}rule\ equality{\isacharunderscore}{\kern0pt}iffI{\isacharsemicolon}{\kern0pt}\ rule\ iffI{\isacharparenright}{\kern0pt}\ \isanewline
\ \ \isacommand{using}\isamarkupfalse%
\ transM\ \isanewline
\ \ \isacommand{by}\isamarkupfalse%
\ auto%
\endisatagproof
{\isafoldproof}%
%
\isadelimproof
\isanewline
%
\endisadelimproof
\isanewline
\isacommand{lemma}\isamarkupfalse%
\ is{\isacharunderscore}{\kern0pt}{\isadigit{1}}{\isacharunderscore}{\kern0pt}iff\ {\isacharcolon}{\kern0pt}\ \isanewline
\ \ \isakeyword{fixes}\ x\ \isanewline
\ \ \isakeyword{assumes}\ {\isachardoublequoteopen}x\ {\isasymin}\ M{\isachardoublequoteclose}\ \isanewline
\ \ \isakeyword{shows}\ {\isachardoublequoteopen}is{\isacharunderscore}{\kern0pt}{\isadigit{1}}{\isacharparenleft}{\kern0pt}x{\isacharparenright}{\kern0pt}\ {\isasymlongleftrightarrow}\ x\ {\isacharequal}{\kern0pt}\ {\isadigit{1}}{\isachardoublequoteclose}\ \isanewline
%
\isadelimproof
\ \ %
\endisadelimproof
%
\isatagproof
\isacommand{unfolding}\isamarkupfalse%
\ is{\isacharunderscore}{\kern0pt}{\isadigit{1}}{\isacharunderscore}{\kern0pt}def\ \isanewline
\ \ \isacommand{apply}\isamarkupfalse%
{\isacharparenleft}{\kern0pt}rule\ iffI{\isacharcomma}{\kern0pt}\ rule\ equality{\isacharunderscore}{\kern0pt}iffI{\isacharcomma}{\kern0pt}\ rule\ iffI{\isacharparenright}{\kern0pt}\isanewline
\ \ \isacommand{using}\isamarkupfalse%
\ transM\ assms\ zero{\isacharunderscore}{\kern0pt}in{\isacharunderscore}{\kern0pt}M\ \isanewline
\ \ \ \ \isacommand{apply}\isamarkupfalse%
\ {\isacharparenleft}{\kern0pt}force{\isacharcomma}{\kern0pt}\ force{\isacharparenright}{\kern0pt}\isanewline
\ \ \isacommand{by}\isamarkupfalse%
\ auto%
\endisatagproof
{\isafoldproof}%
%
\isadelimproof
\isanewline
%
\endisadelimproof
\isanewline
\isacommand{end}\isamarkupfalse%
\isanewline
\isanewline
\isacommand{definition}\isamarkupfalse%
\ is{\isacharunderscore}{\kern0pt}{\isadigit{1}}{\isacharunderscore}{\kern0pt}fm\ \isakeyword{where}\ {\isachardoublequoteopen}is{\isacharunderscore}{\kern0pt}{\isadigit{1}}{\isacharunderscore}{\kern0pt}fm{\isacharparenleft}{\kern0pt}x{\isacharparenright}{\kern0pt}\ {\isasymequiv}\ Forall{\isacharparenleft}{\kern0pt}Iff{\isacharparenleft}{\kern0pt}Member{\isacharparenleft}{\kern0pt}{\isadigit{0}}{\isacharcomma}{\kern0pt}\ x\ {\isacharhash}{\kern0pt}{\isacharplus}{\kern0pt}\ {\isadigit{1}}{\isacharparenright}{\kern0pt}{\isacharcomma}{\kern0pt}\ empty{\isacharunderscore}{\kern0pt}fm{\isacharparenleft}{\kern0pt}{\isadigit{0}}{\isacharparenright}{\kern0pt}{\isacharparenright}{\kern0pt}{\isacharparenright}{\kern0pt}{\isachardoublequoteclose}\isanewline
\isanewline
\isacommand{context}\isamarkupfalse%
\ forcing{\isacharunderscore}{\kern0pt}data\isanewline
\isakeyword{begin}\ \isanewline
\isanewline
\isacommand{lemma}\isamarkupfalse%
\ is{\isacharunderscore}{\kern0pt}{\isadigit{1}}{\isacharunderscore}{\kern0pt}fm{\isacharunderscore}{\kern0pt}type\ {\isacharcolon}{\kern0pt}\ \isanewline
\ \ \isakeyword{fixes}\ i\ \isanewline
\ \ \isakeyword{assumes}\ {\isachardoublequoteopen}i\ {\isasymin}\ nat{\isachardoublequoteclose}\ \isanewline
\ \ \isakeyword{shows}\ {\isachardoublequoteopen}is{\isacharunderscore}{\kern0pt}{\isadigit{1}}{\isacharunderscore}{\kern0pt}fm{\isacharparenleft}{\kern0pt}x{\isacharparenright}{\kern0pt}\ {\isasymin}\ formula{\isachardoublequoteclose}\ \isanewline
%
\isadelimproof
\ \ %
\endisadelimproof
%
\isatagproof
\isacommand{unfolding}\isamarkupfalse%
\ is{\isacharunderscore}{\kern0pt}{\isadigit{1}}{\isacharunderscore}{\kern0pt}fm{\isacharunderscore}{\kern0pt}def\ \isacommand{by}\isamarkupfalse%
\ auto%
\endisatagproof
{\isafoldproof}%
%
\isadelimproof
\isanewline
%
\endisadelimproof
\isanewline
\isacommand{lemma}\isamarkupfalse%
\ arity{\isacharunderscore}{\kern0pt}is{\isacharunderscore}{\kern0pt}{\isadigit{1}}{\isacharunderscore}{\kern0pt}fm\ {\isacharcolon}{\kern0pt}\ \isanewline
\ \ \isakeyword{fixes}\ i\ \ \isakeyword{assumes}\ {\isachardoublequoteopen}i\ {\isasymin}\ nat{\isachardoublequoteclose}\ \isanewline
\ \ \isakeyword{shows}\ {\isachardoublequoteopen}arity{\isacharparenleft}{\kern0pt}is{\isacharunderscore}{\kern0pt}{\isadigit{1}}{\isacharunderscore}{\kern0pt}fm{\isacharparenleft}{\kern0pt}i{\isacharparenright}{\kern0pt}{\isacharparenright}{\kern0pt}\ {\isacharequal}{\kern0pt}\ succ{\isacharparenleft}{\kern0pt}i{\isacharparenright}{\kern0pt}{\isachardoublequoteclose}\ \isanewline
%
\isadelimproof
\ \ %
\endisadelimproof
%
\isatagproof
\isacommand{unfolding}\isamarkupfalse%
\ is{\isacharunderscore}{\kern0pt}{\isadigit{1}}{\isacharunderscore}{\kern0pt}fm{\isacharunderscore}{\kern0pt}def\ \isanewline
\ \ \isacommand{using}\isamarkupfalse%
\ assms\ \isanewline
\ \ \isacommand{apply}\isamarkupfalse%
\ simp\isanewline
\ \ \isacommand{apply}\isamarkupfalse%
{\isacharparenleft}{\kern0pt}subst\ arity{\isacharunderscore}{\kern0pt}empty{\isacharunderscore}{\kern0pt}fm{\isacharcomma}{\kern0pt}\ simp{\isacharparenright}{\kern0pt}\isanewline
\ \ \isacommand{apply}\isamarkupfalse%
{\isacharparenleft}{\kern0pt}subst\ Ord{\isacharunderscore}{\kern0pt}un{\isacharunderscore}{\kern0pt}eq{\isadigit{2}}{\isacharcomma}{\kern0pt}\ simp{\isacharunderscore}{\kern0pt}all{\isacharcomma}{\kern0pt}\ subst\ Ord{\isacharunderscore}{\kern0pt}un{\isacharunderscore}{\kern0pt}eq{\isadigit{1}}{\isacharcomma}{\kern0pt}\ simp{\isacharunderscore}{\kern0pt}all{\isacharparenright}{\kern0pt}\isanewline
\ \ \isacommand{done}\isamarkupfalse%
%
\endisatagproof
{\isafoldproof}%
%
\isadelimproof
\isanewline
%
\endisadelimproof
\ \ \isanewline
\isacommand{lemma}\isamarkupfalse%
\ sats{\isacharunderscore}{\kern0pt}is{\isacharunderscore}{\kern0pt}{\isadigit{1}}{\isacharunderscore}{\kern0pt}fm{\isacharunderscore}{\kern0pt}iff\ {\isacharcolon}{\kern0pt}\ \isanewline
\ \ \isakeyword{fixes}\ i\ x\ env\ \isanewline
\ \ \isakeyword{assumes}\ {\isachardoublequoteopen}i\ {\isacharless}{\kern0pt}\ length{\isacharparenleft}{\kern0pt}env{\isacharparenright}{\kern0pt}{\isachardoublequoteclose}\ {\isachardoublequoteopen}nth{\isacharparenleft}{\kern0pt}i{\isacharcomma}{\kern0pt}\ env{\isacharparenright}{\kern0pt}\ {\isacharequal}{\kern0pt}\ x{\isachardoublequoteclose}\ {\isachardoublequoteopen}env\ {\isasymin}\ list{\isacharparenleft}{\kern0pt}M{\isacharparenright}{\kern0pt}{\isachardoublequoteclose}\isanewline
\ \ \isakeyword{shows}\ {\isachardoublequoteopen}sats{\isacharparenleft}{\kern0pt}M{\isacharcomma}{\kern0pt}\ is{\isacharunderscore}{\kern0pt}{\isadigit{1}}{\isacharunderscore}{\kern0pt}fm{\isacharparenleft}{\kern0pt}i{\isacharparenright}{\kern0pt}{\isacharcomma}{\kern0pt}\ env{\isacharparenright}{\kern0pt}\ {\isasymlongleftrightarrow}\ x\ {\isacharequal}{\kern0pt}\ {\isadigit{1}}{\isachardoublequoteclose}\ \isanewline
%
\isadelimproof
\isanewline
\ \ %
\endisadelimproof
%
\isatagproof
\isacommand{unfolding}\isamarkupfalse%
\ is{\isacharunderscore}{\kern0pt}{\isadigit{1}}{\isacharunderscore}{\kern0pt}fm{\isacharunderscore}{\kern0pt}def\ \isanewline
\ \ \isacommand{apply}\isamarkupfalse%
{\isacharparenleft}{\kern0pt}subgoal{\isacharunderscore}{\kern0pt}tac\ {\isachardoublequoteopen}i\ {\isasymin}\ nat{\isachardoublequoteclose}{\isacharparenright}{\kern0pt}\isanewline
\ \ \isacommand{apply}\isamarkupfalse%
{\isacharparenleft}{\kern0pt}rule{\isacharunderscore}{\kern0pt}tac\ Q{\isacharequal}{\kern0pt}{\isachardoublequoteopen}{\isasymforall}v\ {\isasymin}\ M{\isachardot}{\kern0pt}\ v\ {\isasymin}\ x\ {\isasymlongleftrightarrow}\ v\ {\isacharequal}{\kern0pt}\ {\isadigit{0}}{\isachardoublequoteclose}\ \isakeyword{in}\ iff{\isacharunderscore}{\kern0pt}trans{\isacharparenright}{\kern0pt}\isanewline
\ \ \ \isacommand{apply}\isamarkupfalse%
{\isacharparenleft}{\kern0pt}rule\ iff{\isacharunderscore}{\kern0pt}trans{\isacharcomma}{\kern0pt}\ rule\ sats{\isacharunderscore}{\kern0pt}Forall{\isacharunderscore}{\kern0pt}iff{\isacharcomma}{\kern0pt}\ simp\ add{\isacharcolon}{\kern0pt}assms{\isacharparenright}{\kern0pt}\isanewline
\ \ \ \isacommand{apply}\isamarkupfalse%
{\isacharparenleft}{\kern0pt}rule\ ball{\isacharunderscore}{\kern0pt}iff{\isacharcomma}{\kern0pt}\ rule\ iff{\isacharunderscore}{\kern0pt}trans{\isacharcomma}{\kern0pt}\ rule\ sats{\isacharunderscore}{\kern0pt}Iff{\isacharunderscore}{\kern0pt}iff{\isacharcomma}{\kern0pt}\ simp\ add{\isacharcolon}{\kern0pt}assms{\isacharcomma}{\kern0pt}\ rule\ iff{\isacharunderscore}{\kern0pt}iff{\isacharcomma}{\kern0pt}\ simp\ add{\isacharcolon}{\kern0pt}assms{\isacharparenright}{\kern0pt}\isanewline
\ \ \ \isacommand{apply}\isamarkupfalse%
{\isacharparenleft}{\kern0pt}rule\ iff{\isacharunderscore}{\kern0pt}trans{\isacharcomma}{\kern0pt}\ rule\ sats{\isacharunderscore}{\kern0pt}empty{\isacharunderscore}{\kern0pt}fm{\isacharcomma}{\kern0pt}\ simp{\isacharcomma}{\kern0pt}\ simp\ add{\isacharcolon}{\kern0pt}assms{\isacharcomma}{\kern0pt}\ simp\ add{\isacharcolon}{\kern0pt}assms{\isacharparenright}{\kern0pt}\isanewline
\ \ \isacommand{using}\isamarkupfalse%
\ is{\isacharunderscore}{\kern0pt}{\isadigit{1}}{\isacharunderscore}{\kern0pt}iff\ assms\isanewline
\ \ \isacommand{unfolding}\isamarkupfalse%
\ is{\isacharunderscore}{\kern0pt}{\isadigit{1}}{\isacharunderscore}{\kern0pt}def\ \isanewline
\ \ \ \isacommand{apply}\isamarkupfalse%
\ force\isanewline
\ \ \isacommand{apply}\isamarkupfalse%
{\isacharparenleft}{\kern0pt}rule\ lt{\isacharunderscore}{\kern0pt}nat{\isacharunderscore}{\kern0pt}in{\isacharunderscore}{\kern0pt}nat{\isacharparenright}{\kern0pt}\isanewline
\ \ \isacommand{using}\isamarkupfalse%
\ assms\ \isanewline
\ \ \isacommand{by}\isamarkupfalse%
\ auto%
\endisatagproof
{\isafoldproof}%
%
\isadelimproof
\isanewline
%
\endisadelimproof
\ \ \isanewline
\isacommand{definition}\isamarkupfalse%
\ His{\isacharunderscore}{\kern0pt}P{\isacharunderscore}{\kern0pt}name\ \isakeyword{where}\ {\isachardoublequoteopen}His{\isacharunderscore}{\kern0pt}P{\isacharunderscore}{\kern0pt}name{\isacharparenleft}{\kern0pt}x{\isacharcomma}{\kern0pt}\ g{\isacharparenright}{\kern0pt}\ {\isasymequiv}\ if\ relation{\isacharparenleft}{\kern0pt}x{\isacharparenright}{\kern0pt}\ {\isasymand}\ range{\isacharparenleft}{\kern0pt}x{\isacharparenright}{\kern0pt}\ {\isasymsubseteq}\ P\ {\isasymand}\ {\isacharparenleft}{\kern0pt}{\isasymforall}y\ {\isasymin}\ domain{\isacharparenleft}{\kern0pt}x{\isacharparenright}{\kern0pt}{\isachardot}{\kern0pt}\ g{\isacharbackquote}{\kern0pt}y\ {\isacharequal}{\kern0pt}\ {\isadigit{1}}{\isacharparenright}{\kern0pt}\ then\ {\isadigit{1}}\ else\ {\isadigit{0}}{\isachardoublequoteclose}\ \isanewline
\isanewline
\isacommand{definition}\isamarkupfalse%
\ is{\isacharunderscore}{\kern0pt}P{\isacharunderscore}{\kern0pt}name\ \isakeyword{where}\ {\isachardoublequoteopen}is{\isacharunderscore}{\kern0pt}P{\isacharunderscore}{\kern0pt}name{\isacharparenleft}{\kern0pt}x{\isacharparenright}{\kern0pt}\ {\isasymequiv}\ wftrec{\isacharparenleft}{\kern0pt}Memrel{\isacharparenleft}{\kern0pt}M{\isacharparenright}{\kern0pt}{\isacharcircum}{\kern0pt}{\isacharplus}{\kern0pt}{\isacharcomma}{\kern0pt}\ x{\isacharcomma}{\kern0pt}\ His{\isacharunderscore}{\kern0pt}P{\isacharunderscore}{\kern0pt}name{\isacharparenright}{\kern0pt}{\isachardoublequoteclose}\ \isanewline
\isanewline
\isacommand{lemma}\isamarkupfalse%
\ def{\isacharunderscore}{\kern0pt}is{\isacharunderscore}{\kern0pt}P{\isacharunderscore}{\kern0pt}name\ {\isacharcolon}{\kern0pt}\ \isanewline
\ \ \isakeyword{fixes}\ x\ \isanewline
\ \ \isakeyword{assumes}\ {\isachardoublequoteopen}x\ {\isasymin}\ M{\isachardoublequoteclose}\ \isanewline
\ \ \isakeyword{shows}\ {\isachardoublequoteopen}x\ {\isasymin}\ P{\isacharunderscore}{\kern0pt}names\ {\isasymlongleftrightarrow}\ is{\isacharunderscore}{\kern0pt}P{\isacharunderscore}{\kern0pt}name{\isacharparenleft}{\kern0pt}x{\isacharparenright}{\kern0pt}\ {\isacharequal}{\kern0pt}\ {\isadigit{1}}{\isachardoublequoteclose}\ \isanewline
%
\isadelimproof
\isanewline
%
\endisadelimproof
%
\isatagproof
\isacommand{proof}\isamarkupfalse%
{\isacharminus}{\kern0pt}\isanewline
\ \ \isacommand{have}\isamarkupfalse%
\ {\isachardoublequoteopen}{\isasymAnd}x{\isachardot}{\kern0pt}\ x\ {\isasymin}\ M\ {\isasymlongrightarrow}\ x\ {\isasymin}\ P{\isacharunderscore}{\kern0pt}names\ {\isasymlongleftrightarrow}\ is{\isacharunderscore}{\kern0pt}P{\isacharunderscore}{\kern0pt}name{\isacharparenleft}{\kern0pt}x{\isacharparenright}{\kern0pt}\ {\isacharequal}{\kern0pt}\ {\isadigit{1}}{\isachardoublequoteclose}\ \isanewline
\ \ \isacommand{proof}\isamarkupfalse%
{\isacharparenleft}{\kern0pt}rule{\isacharunderscore}{\kern0pt}tac\ Q{\isacharequal}{\kern0pt}{\isachardoublequoteopen}{\isasymlambda}x{\isachardot}{\kern0pt}\ x\ {\isasymin}\ M\ {\isasymlongrightarrow}\ x\ {\isasymin}\ P{\isacharunderscore}{\kern0pt}names\ {\isasymlongleftrightarrow}\ is{\isacharunderscore}{\kern0pt}P{\isacharunderscore}{\kern0pt}name{\isacharparenleft}{\kern0pt}x{\isacharparenright}{\kern0pt}\ {\isacharequal}{\kern0pt}\ {\isadigit{1}}{\isachardoublequoteclose}\ \isakeyword{in}\ ed{\isacharunderscore}{\kern0pt}induction{\isacharcomma}{\kern0pt}\ rule\ impI{\isacharparenright}{\kern0pt}\isanewline
\ \ \ \ \isacommand{fix}\isamarkupfalse%
\ x\ \isacommand{assume}\isamarkupfalse%
\ assms{\isadigit{1}}{\isacharcolon}{\kern0pt}\ {\isachardoublequoteopen}x\ {\isasymin}\ M{\isachardoublequoteclose}\ {\isachardoublequoteopen}{\isacharparenleft}{\kern0pt}{\isasymAnd}y{\isachardot}{\kern0pt}\ ed{\isacharparenleft}{\kern0pt}y{\isacharcomma}{\kern0pt}\ x{\isacharparenright}{\kern0pt}\ {\isasymLongrightarrow}\ y\ {\isasymin}\ M\ {\isasymlongrightarrow}\ y\ {\isasymin}\ P{\isacharunderscore}{\kern0pt}names\ {\isasymlongleftrightarrow}\ is{\isacharunderscore}{\kern0pt}P{\isacharunderscore}{\kern0pt}name{\isacharparenleft}{\kern0pt}y{\isacharparenright}{\kern0pt}\ {\isacharequal}{\kern0pt}\ {\isadigit{1}}{\isacharparenright}{\kern0pt}{\isachardoublequoteclose}\isanewline
\ \ \ \ \isacommand{have}\isamarkupfalse%
\ {\isachardoublequoteopen}is{\isacharunderscore}{\kern0pt}P{\isacharunderscore}{\kern0pt}name{\isacharparenleft}{\kern0pt}x{\isacharparenright}{\kern0pt}\ {\isacharequal}{\kern0pt}\ {\isadigit{1}}\ {\isasymlongleftrightarrow}\ His{\isacharunderscore}{\kern0pt}P{\isacharunderscore}{\kern0pt}name{\isacharparenleft}{\kern0pt}x{\isacharcomma}{\kern0pt}\ {\isasymlambda}y\ {\isasymin}\ Memrel{\isacharparenleft}{\kern0pt}M{\isacharparenright}{\kern0pt}{\isacharcircum}{\kern0pt}{\isacharplus}{\kern0pt}\ {\isacharminus}{\kern0pt}{\isacharbackquote}{\kern0pt}{\isacharbackquote}{\kern0pt}\ {\isacharbraceleft}{\kern0pt}x{\isacharbraceright}{\kern0pt}{\isachardot}{\kern0pt}\ is{\isacharunderscore}{\kern0pt}P{\isacharunderscore}{\kern0pt}name{\isacharparenleft}{\kern0pt}y{\isacharparenright}{\kern0pt}{\isacharparenright}{\kern0pt}\ {\isacharequal}{\kern0pt}\ {\isadigit{1}}{\isachardoublequoteclose}\isanewline
\ \ \ \ \ \ \isacommand{unfolding}\isamarkupfalse%
\ is{\isacharunderscore}{\kern0pt}P{\isacharunderscore}{\kern0pt}name{\isacharunderscore}{\kern0pt}def\ \isanewline
\ \ \ \ \ \ \isacommand{apply}\isamarkupfalse%
{\isacharparenleft}{\kern0pt}subst\ wftrec{\isacharparenright}{\kern0pt}\isanewline
\ \ \ \ \ \ \ \ \isacommand{apply}\isamarkupfalse%
{\isacharparenleft}{\kern0pt}rule\ wf{\isacharunderscore}{\kern0pt}trancl{\isacharcomma}{\kern0pt}\ rule\ wf{\isacharunderscore}{\kern0pt}Memrel{\isacharcomma}{\kern0pt}\ rule\ trans{\isacharunderscore}{\kern0pt}trancl{\isacharparenright}{\kern0pt}\isanewline
\ \ \ \ \ \ \isacommand{by}\isamarkupfalse%
\ simp\isanewline
\ \ \ \ \isacommand{also}\isamarkupfalse%
\ \isacommand{have}\isamarkupfalse%
\ {\isachardoublequoteopen}{\isachardot}{\kern0pt}{\isachardot}{\kern0pt}{\isachardot}{\kern0pt}\ {\isasymlongleftrightarrow}\ relation{\isacharparenleft}{\kern0pt}x{\isacharparenright}{\kern0pt}\ {\isasymand}\ range{\isacharparenleft}{\kern0pt}x{\isacharparenright}{\kern0pt}\ {\isasymsubseteq}\ P\ {\isasymand}\ {\isacharparenleft}{\kern0pt}{\isasymforall}y\ {\isasymin}\ domain{\isacharparenleft}{\kern0pt}x{\isacharparenright}{\kern0pt}{\isachardot}{\kern0pt}\ {\isacharparenleft}{\kern0pt}{\isasymlambda}y\ {\isasymin}\ Memrel{\isacharparenleft}{\kern0pt}M{\isacharparenright}{\kern0pt}{\isacharcircum}{\kern0pt}{\isacharplus}{\kern0pt}\ {\isacharminus}{\kern0pt}{\isacharbackquote}{\kern0pt}{\isacharbackquote}{\kern0pt}\ {\isacharbraceleft}{\kern0pt}x{\isacharbraceright}{\kern0pt}{\isachardot}{\kern0pt}\ is{\isacharunderscore}{\kern0pt}P{\isacharunderscore}{\kern0pt}name{\isacharparenleft}{\kern0pt}y{\isacharparenright}{\kern0pt}{\isacharparenright}{\kern0pt}{\isacharbackquote}{\kern0pt}y\ {\isacharequal}{\kern0pt}\ {\isadigit{1}}{\isacharparenright}{\kern0pt}{\isachardoublequoteclose}\ \isanewline
\ \ \ \ \ \ \isacommand{unfolding}\isamarkupfalse%
\ His{\isacharunderscore}{\kern0pt}P{\isacharunderscore}{\kern0pt}name{\isacharunderscore}{\kern0pt}def\ \isanewline
\ \ \ \ \ \ \isacommand{apply}\isamarkupfalse%
{\isacharparenleft}{\kern0pt}rule\ iffI{\isacharcomma}{\kern0pt}\ rule{\isacharunderscore}{\kern0pt}tac\ a{\isacharequal}{\kern0pt}{\isadigit{1}}\ \isakeyword{and}\ b{\isacharequal}{\kern0pt}{\isadigit{0}}\ \isakeyword{in}\ ifT{\isacharunderscore}{\kern0pt}eq{\isacharcomma}{\kern0pt}\ simp{\isacharcomma}{\kern0pt}\ simp{\isacharparenright}{\kern0pt}\isanewline
\ \ \ \ \ \ \isacommand{apply}\isamarkupfalse%
{\isacharparenleft}{\kern0pt}subst\ if{\isacharunderscore}{\kern0pt}P{\isacharcomma}{\kern0pt}\ auto{\isacharparenright}{\kern0pt}\isanewline
\ \ \ \ \ \ \isacommand{done}\isamarkupfalse%
\isanewline
\ \ \ \ \isacommand{also}\isamarkupfalse%
\ \isacommand{have}\isamarkupfalse%
\ {\isachardoublequoteopen}{\isachardot}{\kern0pt}{\isachardot}{\kern0pt}{\isachardot}{\kern0pt}\ {\isasymlongleftrightarrow}\ relation{\isacharparenleft}{\kern0pt}x{\isacharparenright}{\kern0pt}\ {\isasymand}\ range{\isacharparenleft}{\kern0pt}x{\isacharparenright}{\kern0pt}\ {\isasymsubseteq}\ P\ {\isasymand}\ {\isacharparenleft}{\kern0pt}{\isasymforall}y\ {\isasymin}\ domain{\isacharparenleft}{\kern0pt}x{\isacharparenright}{\kern0pt}{\isachardot}{\kern0pt}\ is{\isacharunderscore}{\kern0pt}P{\isacharunderscore}{\kern0pt}name{\isacharparenleft}{\kern0pt}y{\isacharparenright}{\kern0pt}\ {\isacharequal}{\kern0pt}\ {\isadigit{1}}{\isacharparenright}{\kern0pt}{\isachardoublequoteclose}\ \isanewline
\ \ \ \ \ \ \isacommand{apply}\isamarkupfalse%
{\isacharparenleft}{\kern0pt}rule\ iff{\isacharunderscore}{\kern0pt}conjI{\isacharcomma}{\kern0pt}\ simp{\isacharparenright}{\kern0pt}{\isacharplus}{\kern0pt}\isanewline
\ \ \ \ \ \ \isacommand{apply}\isamarkupfalse%
{\isacharparenleft}{\kern0pt}rule\ ball{\isacharunderscore}{\kern0pt}iff{\isacharparenright}{\kern0pt}\isanewline
\ \ \ \ \ \ \isacommand{apply}\isamarkupfalse%
{\isacharparenleft}{\kern0pt}rename{\isacharunderscore}{\kern0pt}tac\ y{\isacharcomma}{\kern0pt}\ subgoal{\isacharunderscore}{\kern0pt}tac\ {\isachardoublequoteopen}y\ {\isasymin}\ Memrel{\isacharparenleft}{\kern0pt}M{\isacharparenright}{\kern0pt}{\isacharcircum}{\kern0pt}{\isacharplus}{\kern0pt}\ {\isacharminus}{\kern0pt}{\isacharbackquote}{\kern0pt}{\isacharbackquote}{\kern0pt}\ {\isacharbraceleft}{\kern0pt}x{\isacharbraceright}{\kern0pt}{\isachardoublequoteclose}{\isacharcomma}{\kern0pt}\ simp{\isacharcomma}{\kern0pt}\ rule{\isacharunderscore}{\kern0pt}tac\ b{\isacharequal}{\kern0pt}x\ \isakeyword{in}\ vimageI{\isacharparenright}{\kern0pt}\isanewline
\ \ \ \ \ \ \ \isacommand{apply}\isamarkupfalse%
{\isacharparenleft}{\kern0pt}rule\ domain{\isacharunderscore}{\kern0pt}elem{\isacharunderscore}{\kern0pt}Memrel{\isacharunderscore}{\kern0pt}trancl{\isacharparenright}{\kern0pt}\isanewline
\ \ \ \ \ \ \isacommand{using}\isamarkupfalse%
\ assms{\isadigit{1}}\ \isanewline
\ \ \ \ \ \ \isacommand{by}\isamarkupfalse%
\ auto\ \isanewline
\ \ \ \ \isacommand{also}\isamarkupfalse%
\ \isacommand{have}\isamarkupfalse%
\ {\isachardoublequoteopen}{\isachardot}{\kern0pt}{\isachardot}{\kern0pt}{\isachardot}{\kern0pt}\ {\isasymlongleftrightarrow}\ relation{\isacharparenleft}{\kern0pt}x{\isacharparenright}{\kern0pt}\ {\isasymand}\ range{\isacharparenleft}{\kern0pt}x{\isacharparenright}{\kern0pt}\ {\isasymsubseteq}\ P\ {\isasymand}\ {\isacharparenleft}{\kern0pt}{\isasymforall}y\ {\isasymin}\ domain{\isacharparenleft}{\kern0pt}x{\isacharparenright}{\kern0pt}{\isachardot}{\kern0pt}\ y\ {\isasymin}\ P{\isacharunderscore}{\kern0pt}names{\isacharparenright}{\kern0pt}{\isachardoublequoteclose}\ \isanewline
\ \ \ \ \ \ \isacommand{apply}\isamarkupfalse%
{\isacharparenleft}{\kern0pt}rule\ iff{\isacharunderscore}{\kern0pt}conjI{\isacharcomma}{\kern0pt}\ simp{\isacharparenright}{\kern0pt}{\isacharplus}{\kern0pt}\ \isanewline
\ \ \ \ \ \ \isacommand{apply}\isamarkupfalse%
{\isacharparenleft}{\kern0pt}rule\ ball{\isacharunderscore}{\kern0pt}iff{\isacharparenright}{\kern0pt}\isanewline
\ \ \ \ \ \ \isacommand{apply}\isamarkupfalse%
{\isacharparenleft}{\kern0pt}rename{\isacharunderscore}{\kern0pt}tac\ y{\isacharcomma}{\kern0pt}\ subgoal{\isacharunderscore}{\kern0pt}tac\ {\isachardoublequoteopen}y\ {\isasymin}\ M{\isachardoublequoteclose}{\isacharparenright}{\kern0pt}\isanewline
\ \ \ \ \ \ \isacommand{using}\isamarkupfalse%
\ assms{\isadigit{1}}\ ed{\isacharunderscore}{\kern0pt}def\ \isanewline
\ \ \ \ \ \ \ \isacommand{apply}\isamarkupfalse%
\ blast\isanewline
\ \ \ \ \ \ \isacommand{using}\isamarkupfalse%
\ domain{\isacharunderscore}{\kern0pt}elem{\isacharunderscore}{\kern0pt}in{\isacharunderscore}{\kern0pt}M\ assms{\isadigit{1}}\isanewline
\ \ \ \ \ \ \isacommand{by}\isamarkupfalse%
\ auto\ \isanewline
\ \ \ \ \isacommand{also}\isamarkupfalse%
\ \isacommand{have}\isamarkupfalse%
\ {\isachardoublequoteopen}{\isachardot}{\kern0pt}{\isachardot}{\kern0pt}{\isachardot}{\kern0pt}\ {\isasymlongleftrightarrow}\ x\ {\isasymsubseteq}\ P{\isacharunderscore}{\kern0pt}names\ {\isasymtimes}\ P{\isachardoublequoteclose}\ \isanewline
\ \ \ \ \isacommand{proof}\isamarkupfalse%
{\isacharparenleft}{\kern0pt}rule\ iffI{\isacharcomma}{\kern0pt}\ rule\ subsetI{\isacharparenright}{\kern0pt}\isanewline
\ \ \ \ \ \ \isacommand{fix}\isamarkupfalse%
\ v\ \isacommand{assume}\isamarkupfalse%
\ assms{\isadigit{2}}{\isacharcolon}{\kern0pt}\ {\isachardoublequoteopen}relation{\isacharparenleft}{\kern0pt}x{\isacharparenright}{\kern0pt}\ {\isasymand}\ range{\isacharparenleft}{\kern0pt}x{\isacharparenright}{\kern0pt}\ {\isasymsubseteq}\ P\ {\isasymand}\ {\isacharparenleft}{\kern0pt}{\isasymforall}y{\isasymin}domain{\isacharparenleft}{\kern0pt}x{\isacharparenright}{\kern0pt}{\isachardot}{\kern0pt}\ y\ {\isasymin}\ P{\isacharunderscore}{\kern0pt}names{\isacharparenright}{\kern0pt}{\isachardoublequoteclose}\ {\isachardoublequoteopen}v\ {\isasymin}\ x{\isachardoublequoteclose}\ \isanewline
\ \ \ \ \ \ \isacommand{then}\isamarkupfalse%
\ \isacommand{obtain}\isamarkupfalse%
\ y\ p\ \isakeyword{where}\ ypH\ {\isacharcolon}{\kern0pt}\ {\isachardoublequoteopen}v\ {\isacharequal}{\kern0pt}\ {\isacharless}{\kern0pt}y{\isacharcomma}{\kern0pt}\ p{\isachargreater}{\kern0pt}{\isachardoublequoteclose}\ \isacommand{using}\isamarkupfalse%
\ assms{\isadigit{2}}\ relation{\isacharunderscore}{\kern0pt}def\ \isacommand{by}\isamarkupfalse%
\ auto\isanewline
\ \ \ \ \ \ \isacommand{then}\isamarkupfalse%
\ \isacommand{have}\isamarkupfalse%
\ H{\isadigit{1}}{\isacharcolon}{\kern0pt}\ {\isachardoublequoteopen}y\ {\isasymin}\ P{\isacharunderscore}{\kern0pt}names{\isachardoublequoteclose}\ \isacommand{using}\isamarkupfalse%
\ assms{\isadigit{2}}\ \isacommand{by}\isamarkupfalse%
\ auto\ \isanewline
\ \ \ \ \ \ \isacommand{then}\isamarkupfalse%
\ \isacommand{have}\isamarkupfalse%
\ H{\isadigit{2}}{\isacharcolon}{\kern0pt}\ {\isachardoublequoteopen}p\ {\isasymin}\ P{\isachardoublequoteclose}\ \isacommand{using}\isamarkupfalse%
\ assms{\isadigit{2}}\ ypH\ \isacommand{by}\isamarkupfalse%
\ auto\ \isanewline
\ \ \ \ \ \ \isacommand{then}\isamarkupfalse%
\ \isacommand{show}\isamarkupfalse%
\ {\isachardoublequoteopen}v\ {\isasymin}\ P{\isacharunderscore}{\kern0pt}names\ {\isasymtimes}\ P{\isachardoublequoteclose}\ \isacommand{using}\isamarkupfalse%
\ ypH\ H{\isadigit{1}}\ H{\isadigit{2}}\ \isacommand{by}\isamarkupfalse%
\ auto\ \isanewline
\ \ \ \ \isacommand{next}\isamarkupfalse%
\ \isanewline
\ \ \ \ \ \ \isacommand{assume}\isamarkupfalse%
\ assms{\isadigit{2}}{\isacharcolon}{\kern0pt}\ {\isachardoublequoteopen}x\ {\isasymsubseteq}\ P{\isacharunderscore}{\kern0pt}names\ {\isasymtimes}\ P{\isachardoublequoteclose}\ \isanewline
\ \ \ \ \ \ \isacommand{show}\isamarkupfalse%
\ {\isachardoublequoteopen}relation{\isacharparenleft}{\kern0pt}x{\isacharparenright}{\kern0pt}\ {\isasymand}\ range{\isacharparenleft}{\kern0pt}x{\isacharparenright}{\kern0pt}\ {\isasymsubseteq}\ P\ {\isasymand}\ {\isacharparenleft}{\kern0pt}{\isasymforall}y{\isasymin}domain{\isacharparenleft}{\kern0pt}x{\isacharparenright}{\kern0pt}{\isachardot}{\kern0pt}\ y\ {\isasymin}\ P{\isacharunderscore}{\kern0pt}names{\isacharparenright}{\kern0pt}{\isachardoublequoteclose}\ \isanewline
\ \ \ \ \ \ \ \ \isacommand{unfolding}\isamarkupfalse%
\ relation{\isacharunderscore}{\kern0pt}def\isanewline
\ \ \ \ \ \ \ \ \isacommand{using}\isamarkupfalse%
\ assms{\isadigit{2}}\ \isanewline
\ \ \ \ \ \ \ \ \isacommand{by}\isamarkupfalse%
\ auto\isanewline
\ \ \ \ \isacommand{qed}\isamarkupfalse%
\isanewline
\ \ \ \ \isacommand{also}\isamarkupfalse%
\ \isacommand{have}\isamarkupfalse%
\ {\isachardoublequoteopen}{\isachardot}{\kern0pt}{\isachardot}{\kern0pt}{\isachardot}{\kern0pt}\ {\isasymlongleftrightarrow}\ x\ {\isasymin}\ P{\isacharunderscore}{\kern0pt}names{\isachardoublequoteclose}\ \isacommand{using}\isamarkupfalse%
\ P{\isacharunderscore}{\kern0pt}name{\isacharunderscore}{\kern0pt}iff\ assms{\isadigit{1}}\ \isacommand{by}\isamarkupfalse%
\ auto\isanewline
\isanewline
\ \ \ \ \isacommand{finally}\isamarkupfalse%
\ \isacommand{show}\isamarkupfalse%
\ {\isachardoublequoteopen}x\ {\isasymin}\ P{\isacharunderscore}{\kern0pt}names\ {\isasymlongleftrightarrow}\ is{\isacharunderscore}{\kern0pt}P{\isacharunderscore}{\kern0pt}name{\isacharparenleft}{\kern0pt}x{\isacharparenright}{\kern0pt}\ {\isacharequal}{\kern0pt}\ {\isadigit{1}}{\isachardoublequoteclose}\ \isacommand{by}\isamarkupfalse%
\ simp\isanewline
\ \ \isacommand{qed}\isamarkupfalse%
\isanewline
\ \ \isacommand{then}\isamarkupfalse%
\ \isacommand{show}\isamarkupfalse%
\ {\isacharquery}{\kern0pt}thesis\ \isacommand{using}\isamarkupfalse%
\ assms\ \isacommand{by}\isamarkupfalse%
\ auto\ \isanewline
\isacommand{qed}\isamarkupfalse%
%
\endisatagproof
{\isafoldproof}%
%
\isadelimproof
\isanewline
%
\endisadelimproof
\isanewline
\isacommand{definition}\isamarkupfalse%
\ His{\isacharunderscore}{\kern0pt}P{\isacharunderscore}{\kern0pt}name{\isacharunderscore}{\kern0pt}M\ \isakeyword{where}\ \isanewline
\ \ {\isachardoublequoteopen}His{\isacharunderscore}{\kern0pt}P{\isacharunderscore}{\kern0pt}name{\isacharunderscore}{\kern0pt}M{\isacharparenleft}{\kern0pt}x{\isacharprime}{\kern0pt}{\isacharcomma}{\kern0pt}\ g{\isacharparenright}{\kern0pt}\ {\isasymequiv}\ {\isacharparenleft}{\kern0pt}if\ {\isasymexists}x\ {\isasymin}\ M{\isachardot}{\kern0pt}\ {\isasymexists}P\ {\isasymin}\ M{\isachardot}{\kern0pt}\ x{\isacharprime}{\kern0pt}\ {\isacharequal}{\kern0pt}\ {\isacharless}{\kern0pt}x{\isacharcomma}{\kern0pt}\ P{\isachargreater}{\kern0pt}\ {\isasymand}\ relation{\isacharparenleft}{\kern0pt}x{\isacharparenright}{\kern0pt}\ {\isasymand}\ range{\isacharparenleft}{\kern0pt}x{\isacharparenright}{\kern0pt}\ {\isasymsubseteq}\ P\ {\isasymand}\ {\isacharparenleft}{\kern0pt}{\isasymforall}y\ {\isasymin}\ domain{\isacharparenleft}{\kern0pt}x{\isacharparenright}{\kern0pt}{\isachardot}{\kern0pt}\ g{\isacharbackquote}{\kern0pt}{\isacharless}{\kern0pt}y{\isacharcomma}{\kern0pt}\ P{\isachargreater}{\kern0pt}\ {\isacharequal}{\kern0pt}\ {\isadigit{1}}{\isacharparenright}{\kern0pt}\ then\ {\isadigit{1}}\ else\ {\isadigit{0}}{\isacharparenright}{\kern0pt}{\isachardoublequoteclose}\ \isanewline
\isanewline
\isacommand{definition}\isamarkupfalse%
\ His{\isacharunderscore}{\kern0pt}P{\isacharunderscore}{\kern0pt}name{\isacharunderscore}{\kern0pt}M{\isacharunderscore}{\kern0pt}cond\ \isakeyword{where}\ \isanewline
\ \ {\isachardoublequoteopen}His{\isacharunderscore}{\kern0pt}P{\isacharunderscore}{\kern0pt}name{\isacharunderscore}{\kern0pt}M{\isacharunderscore}{\kern0pt}cond{\isacharparenleft}{\kern0pt}x{\isacharprime}{\kern0pt}{\isacharcomma}{\kern0pt}\ g{\isacharparenright}{\kern0pt}\ {\isasymequiv}\ \isanewline
\ \ \ \ \ \ {\isacharparenleft}{\kern0pt}{\isasymexists}x\ {\isasymin}\ M{\isachardot}{\kern0pt}\ {\isasymexists}P\ {\isasymin}\ M{\isachardot}{\kern0pt}\ {\isasymexists}xrange\ {\isasymin}\ M{\isachardot}{\kern0pt}\ {\isasymexists}xdomain\ {\isasymin}\ M{\isachardot}{\kern0pt}\isanewline
\ \ \ \ \ \ pair{\isacharparenleft}{\kern0pt}{\isacharhash}{\kern0pt}{\isacharhash}{\kern0pt}M{\isacharcomma}{\kern0pt}\ x{\isacharcomma}{\kern0pt}\ P{\isacharcomma}{\kern0pt}\ x{\isacharprime}{\kern0pt}{\isacharparenright}{\kern0pt}\ {\isasymand}\ is{\isacharunderscore}{\kern0pt}relation{\isacharparenleft}{\kern0pt}{\isacharhash}{\kern0pt}{\isacharhash}{\kern0pt}M{\isacharcomma}{\kern0pt}\ x{\isacharparenright}{\kern0pt}\ {\isasymand}\ is{\isacharunderscore}{\kern0pt}range{\isacharparenleft}{\kern0pt}{\isacharhash}{\kern0pt}{\isacharhash}{\kern0pt}M{\isacharcomma}{\kern0pt}\ x{\isacharcomma}{\kern0pt}\ xrange{\isacharparenright}{\kern0pt}\ {\isasymand}\ is{\isacharunderscore}{\kern0pt}domain{\isacharparenleft}{\kern0pt}{\isacharhash}{\kern0pt}{\isacharhash}{\kern0pt}M{\isacharcomma}{\kern0pt}\ x{\isacharcomma}{\kern0pt}\ xdomain{\isacharparenright}{\kern0pt}\ \isanewline
\ \ \ \ \ \ {\isasymand}\ subset{\isacharparenleft}{\kern0pt}{\isacharhash}{\kern0pt}{\isacharhash}{\kern0pt}M{\isacharcomma}{\kern0pt}\ xrange{\isacharcomma}{\kern0pt}\ P{\isacharparenright}{\kern0pt}\ {\isasymand}\ {\isacharparenleft}{\kern0pt}{\isasymforall}y\ {\isasymin}\ M{\isachardot}{\kern0pt}\ y\ {\isasymin}\ xdomain\ {\isasymlongrightarrow}\ {\isacharparenleft}{\kern0pt}{\isasymexists}y{\isacharunderscore}{\kern0pt}P\ {\isasymin}\ M{\isachardot}{\kern0pt}\ {\isasymexists}gy\ {\isasymin}\ M{\isachardot}{\kern0pt}\ pair{\isacharparenleft}{\kern0pt}{\isacharhash}{\kern0pt}{\isacharhash}{\kern0pt}M{\isacharcomma}{\kern0pt}\ y{\isacharcomma}{\kern0pt}\ P{\isacharcomma}{\kern0pt}\ y{\isacharunderscore}{\kern0pt}P{\isacharparenright}{\kern0pt}\ {\isasymand}\ fun{\isacharunderscore}{\kern0pt}apply{\isacharparenleft}{\kern0pt}{\isacharhash}{\kern0pt}{\isacharhash}{\kern0pt}M{\isacharcomma}{\kern0pt}\ g{\isacharcomma}{\kern0pt}\ y{\isacharunderscore}{\kern0pt}P{\isacharcomma}{\kern0pt}\ gy{\isacharparenright}{\kern0pt}\ {\isasymand}\ {\isacharparenleft}{\kern0pt}{\isasymforall}z\ {\isasymin}\ M{\isachardot}{\kern0pt}\ z\ {\isasymin}\ gy\ {\isasymlongleftrightarrow}\ empty{\isacharparenleft}{\kern0pt}{\isacharhash}{\kern0pt}{\isacharhash}{\kern0pt}M{\isacharcomma}{\kern0pt}\ z{\isacharparenright}{\kern0pt}{\isacharparenright}{\kern0pt}{\isacharparenright}{\kern0pt}{\isacharparenright}{\kern0pt}{\isacharparenright}{\kern0pt}{\isachardoublequoteclose}\isanewline
\isanewline
\isacommand{lemma}\isamarkupfalse%
\ His{\isacharunderscore}{\kern0pt}P{\isacharunderscore}{\kern0pt}name{\isacharunderscore}{\kern0pt}M{\isacharunderscore}{\kern0pt}cond{\isacharunderscore}{\kern0pt}iff\ {\isacharcolon}{\kern0pt}\ \isanewline
\ \ {\isachardoublequoteopen}{\isasymAnd}x{\isacharprime}{\kern0pt}\ g{\isachardot}{\kern0pt}\ x{\isacharprime}{\kern0pt}\ {\isasymin}\ M\ {\isasymLongrightarrow}\ g\ {\isasymin}\ M\ {\isasymLongrightarrow}\ \isanewline
\ \ \ \ {\isacharparenleft}{\kern0pt}{\isasymexists}x\ {\isasymin}\ M{\isachardot}{\kern0pt}\ {\isasymexists}P\ {\isasymin}\ M{\isachardot}{\kern0pt}\ x{\isacharprime}{\kern0pt}\ {\isacharequal}{\kern0pt}\ {\isacharless}{\kern0pt}x{\isacharcomma}{\kern0pt}\ P{\isachargreater}{\kern0pt}\ {\isasymand}\ relation{\isacharparenleft}{\kern0pt}x{\isacharparenright}{\kern0pt}\ {\isasymand}\ range{\isacharparenleft}{\kern0pt}x{\isacharparenright}{\kern0pt}\ {\isasymsubseteq}\ P\ {\isasymand}\ {\isacharparenleft}{\kern0pt}{\isasymforall}y\ {\isasymin}\ domain{\isacharparenleft}{\kern0pt}x{\isacharparenright}{\kern0pt}{\isachardot}{\kern0pt}\ g{\isacharbackquote}{\kern0pt}{\isacharless}{\kern0pt}y{\isacharcomma}{\kern0pt}\ P{\isachargreater}{\kern0pt}\ {\isacharequal}{\kern0pt}\ {\isadigit{1}}{\isacharparenright}{\kern0pt}{\isacharparenright}{\kern0pt}\ {\isasymlongleftrightarrow}\ His{\isacharunderscore}{\kern0pt}P{\isacharunderscore}{\kern0pt}name{\isacharunderscore}{\kern0pt}M{\isacharunderscore}{\kern0pt}cond{\isacharparenleft}{\kern0pt}x{\isacharprime}{\kern0pt}{\isacharcomma}{\kern0pt}\ g{\isacharparenright}{\kern0pt}{\isachardoublequoteclose}\ \isanewline
%
\isadelimproof
\ \ %
\endisadelimproof
%
\isatagproof
\isacommand{unfolding}\isamarkupfalse%
\ His{\isacharunderscore}{\kern0pt}P{\isacharunderscore}{\kern0pt}name{\isacharunderscore}{\kern0pt}M{\isacharunderscore}{\kern0pt}cond{\isacharunderscore}{\kern0pt}def\ \isanewline
\ \ \isacommand{using}\isamarkupfalse%
\ domain{\isacharunderscore}{\kern0pt}closed\ range{\isacharunderscore}{\kern0pt}closed\ pair{\isacharunderscore}{\kern0pt}in{\isacharunderscore}{\kern0pt}M{\isacharunderscore}{\kern0pt}iff\ \isanewline
\ \ \isacommand{apply}\isamarkupfalse%
\ auto\ \ \isanewline
\ \ \isacommand{apply}\isamarkupfalse%
{\isacharparenleft}{\kern0pt}rule{\isacharunderscore}{\kern0pt}tac\ is{\isacharunderscore}{\kern0pt}{\isadigit{1}}D{\isacharparenright}{\kern0pt}\ \isanewline
\ \ \ \isacommand{apply}\isamarkupfalse%
{\isacharparenleft}{\kern0pt}rule{\isacharunderscore}{\kern0pt}tac\ to{\isacharunderscore}{\kern0pt}rin{\isacharparenright}{\kern0pt}\ \isanewline
\ \ \ \isacommand{apply}\isamarkupfalse%
{\isacharparenleft}{\kern0pt}rule{\isacharunderscore}{\kern0pt}tac\ apply{\isacharunderscore}{\kern0pt}closed{\isacharparenright}{\kern0pt}\ \isanewline
\ \ \ \ \isacommand{apply}\isamarkupfalse%
\ simp\ \isanewline
\ \ \ \isacommand{apply}\isamarkupfalse%
\ simp\ \isanewline
\ \ \ \isacommand{apply}\isamarkupfalse%
{\isacharparenleft}{\kern0pt}rule{\isacharunderscore}{\kern0pt}tac\ x{\isacharequal}{\kern0pt}x\ \isakeyword{in}\ domain{\isacharunderscore}{\kern0pt}elem{\isacharunderscore}{\kern0pt}in{\isacharunderscore}{\kern0pt}M{\isacharparenright}{\kern0pt}\ \isanewline
\ \ \ \ \isacommand{apply}\isamarkupfalse%
\ simp\ \isanewline
\ \ \ \isacommand{apply}\isamarkupfalse%
{\isacharparenleft}{\kern0pt}rule{\isacharunderscore}{\kern0pt}tac\ P{\isacharequal}{\kern0pt}{\isachardoublequoteopen}y\ {\isasymin}\ M\ {\isasymand}\ y\ {\isasymin}\ domain{\isacharparenleft}{\kern0pt}x{\isacharparenright}{\kern0pt}{\isachardoublequoteclose}\ \isakeyword{in}\ mp{\isacharparenright}{\kern0pt}\ \isanewline
\ \ \ \ \isacommand{apply}\isamarkupfalse%
\ simp\ \isanewline
\ \ \isacommand{using}\isamarkupfalse%
\ domain{\isacharunderscore}{\kern0pt}elem{\isacharunderscore}{\kern0pt}in{\isacharunderscore}{\kern0pt}M\ domainI\ \isanewline
\ \ \isacommand{by}\isamarkupfalse%
\ auto%
\endisatagproof
{\isafoldproof}%
%
\isadelimproof
\ \isanewline
%
\endisadelimproof
\isanewline
\isacommand{schematic{\isacharunderscore}{\kern0pt}goal}\isamarkupfalse%
\ His{\isacharunderscore}{\kern0pt}P{\isacharunderscore}{\kern0pt}name{\isacharunderscore}{\kern0pt}M{\isacharunderscore}{\kern0pt}fm{\isacharunderscore}{\kern0pt}auto{\isacharcolon}{\kern0pt}\isanewline
\ \ \isakeyword{assumes}\isanewline
\ \ \ \ {\isachardoublequoteopen}nth{\isacharparenleft}{\kern0pt}{\isadigit{0}}{\isacharcomma}{\kern0pt}env{\isacharparenright}{\kern0pt}\ {\isacharequal}{\kern0pt}\ v{\isachardoublequoteclose}\ \isanewline
\ \ \ \ {\isachardoublequoteopen}nth{\isacharparenleft}{\kern0pt}{\isadigit{1}}{\isacharcomma}{\kern0pt}env{\isacharparenright}{\kern0pt}\ {\isacharequal}{\kern0pt}\ g{\isachardoublequoteclose}\ \isanewline
\ \ \ \ {\isachardoublequoteopen}nth{\isacharparenleft}{\kern0pt}{\isadigit{2}}{\isacharcomma}{\kern0pt}env{\isacharparenright}{\kern0pt}\ {\isacharequal}{\kern0pt}\ x{\isacharprime}{\kern0pt}{\isachardoublequoteclose}\ \ \ \isanewline
\ \ \ \ {\isachardoublequoteopen}env\ {\isasymin}\ list{\isacharparenleft}{\kern0pt}M{\isacharparenright}{\kern0pt}{\isachardoublequoteclose}\ \isanewline
\ \isakeyword{shows}\ \isanewline
\ \ \ \ {\isachardoublequoteopen}{\isacharparenleft}{\kern0pt}{\isasymforall}e\ {\isasymin}\ M{\isachardot}{\kern0pt}\ e\ {\isasymin}\ v\ {\isasymlongleftrightarrow}\ {\isacharparenleft}{\kern0pt}empty{\isacharparenleft}{\kern0pt}{\isacharhash}{\kern0pt}{\isacharhash}{\kern0pt}M{\isacharcomma}{\kern0pt}\ e{\isacharparenright}{\kern0pt}\ {\isasymand}\ His{\isacharunderscore}{\kern0pt}P{\isacharunderscore}{\kern0pt}name{\isacharunderscore}{\kern0pt}M{\isacharunderscore}{\kern0pt}cond{\isacharparenleft}{\kern0pt}x{\isacharprime}{\kern0pt}{\isacharcomma}{\kern0pt}\ g{\isacharparenright}{\kern0pt}{\isacharparenright}{\kern0pt}{\isacharparenright}{\kern0pt}\ {\isasymlongleftrightarrow}\ sats{\isacharparenleft}{\kern0pt}M{\isacharcomma}{\kern0pt}{\isacharquery}{\kern0pt}fm{\isacharparenleft}{\kern0pt}{\isadigit{0}}{\isacharcomma}{\kern0pt}{\isadigit{1}}{\isacharcomma}{\kern0pt}{\isadigit{2}}{\isacharparenright}{\kern0pt}{\isacharcomma}{\kern0pt}env{\isacharparenright}{\kern0pt}{\isachardoublequoteclose}\ \isanewline
%
\isadelimproof
\ \ %
\endisadelimproof
%
\isatagproof
\isacommand{unfolding}\isamarkupfalse%
\ His{\isacharunderscore}{\kern0pt}P{\isacharunderscore}{\kern0pt}name{\isacharunderscore}{\kern0pt}M{\isacharunderscore}{\kern0pt}cond{\isacharunderscore}{\kern0pt}def\ subset{\isacharunderscore}{\kern0pt}def\ \isanewline
\ \ \isacommand{by}\isamarkupfalse%
\ {\isacharparenleft}{\kern0pt}insert\ assms\ {\isacharsemicolon}{\kern0pt}\ {\isacharparenleft}{\kern0pt}rule\ sep{\isacharunderscore}{\kern0pt}rules\ {\isacharbar}{\kern0pt}\ simp{\isacharparenright}{\kern0pt}{\isacharplus}{\kern0pt}{\isacharparenright}{\kern0pt}%
\endisatagproof
{\isafoldproof}%
%
\isadelimproof
\ \isanewline
%
\endisadelimproof
\isanewline
\isacommand{end}\isamarkupfalse%
\ \isanewline
\isanewline
\isacommand{definition}\isamarkupfalse%
\ His{\isacharunderscore}{\kern0pt}P{\isacharunderscore}{\kern0pt}name{\isacharunderscore}{\kern0pt}M{\isacharunderscore}{\kern0pt}fm\ \isakeyword{where}\ {\isachardoublequoteopen}His{\isacharunderscore}{\kern0pt}P{\isacharunderscore}{\kern0pt}name{\isacharunderscore}{\kern0pt}M{\isacharunderscore}{\kern0pt}fm\ {\isasymequiv}\ \isanewline
\ \ \ \ \ \ \ Forall\isanewline
\ \ \ \ \ \ \ \ \ \ \ \ \ {\isacharparenleft}{\kern0pt}Iff{\isacharparenleft}{\kern0pt}Member{\isacharparenleft}{\kern0pt}{\isadigit{0}}{\isacharcomma}{\kern0pt}\ {\isadigit{1}}{\isacharparenright}{\kern0pt}{\isacharcomma}{\kern0pt}\isanewline
\ \ \ \ \ \ \ \ \ \ \ \ \ \ \ \ \ \ And{\isacharparenleft}{\kern0pt}empty{\isacharunderscore}{\kern0pt}fm{\isacharparenleft}{\kern0pt}{\isadigit{0}}{\isacharparenright}{\kern0pt}{\isacharcomma}{\kern0pt}\isanewline
\ \ \ \ \ \ \ \ \ \ \ \ \ \ \ \ \ \ \ \ \ \ Exists\isanewline
\ \ \ \ \ \ \ \ \ \ \ \ \ \ \ \ \ \ \ \ \ \ \ {\isacharparenleft}{\kern0pt}Exists\isanewline
\ \ \ \ \ \ \ \ \ \ \ \ \ \ \ \ \ \ \ \ \ \ \ \ \ {\isacharparenleft}{\kern0pt}Exists\isanewline
\ \ \ \ \ \ \ \ \ \ \ \ \ \ \ \ \ \ \ \ \ \ \ \ \ \ \ {\isacharparenleft}{\kern0pt}Exists\isanewline
\ \ \ \ \ \ \ \ \ \ \ \ \ \ \ \ \ \ \ \ \ \ \ \ \ \ \ \ \ {\isacharparenleft}{\kern0pt}And{\isacharparenleft}{\kern0pt}pair{\isacharunderscore}{\kern0pt}fm{\isacharparenleft}{\kern0pt}{\isadigit{3}}{\isacharcomma}{\kern0pt}\ {\isadigit{2}}{\isacharcomma}{\kern0pt}\ {\isadigit{7}}{\isacharparenright}{\kern0pt}{\isacharcomma}{\kern0pt}\isanewline
\ \ \ \ \ \ \ \ \ \ \ \ \ \ \ \ \ \ \ \ \ \ \ \ \ \ \ \ \ \ \ \ \ \ And{\isacharparenleft}{\kern0pt}relation{\isacharunderscore}{\kern0pt}fm{\isacharparenleft}{\kern0pt}{\isadigit{3}}{\isacharparenright}{\kern0pt}{\isacharcomma}{\kern0pt}\isanewline
\ \ \ \ \ \ \ \ \ \ \ \ \ \ \ \ \ \ \ \ \ \ \ \ \ \ \ \ \ \ \ \ \ \ \ \ \ \ And{\isacharparenleft}{\kern0pt}range{\isacharunderscore}{\kern0pt}fm{\isacharparenleft}{\kern0pt}{\isadigit{3}}{\isacharcomma}{\kern0pt}\ {\isadigit{1}}{\isacharparenright}{\kern0pt}{\isacharcomma}{\kern0pt}\isanewline
\ \ \ \ \ \ \ \ \ \ \ \ \ \ \ \ \ \ \ \ \ \ \ \ \ \ \ \ \ \ \ \ \ \ \ \ \ \ \ \ \ \ And{\isacharparenleft}{\kern0pt}domain{\isacharunderscore}{\kern0pt}fm{\isacharparenleft}{\kern0pt}{\isadigit{3}}{\isacharcomma}{\kern0pt}\ {\isadigit{0}}{\isacharparenright}{\kern0pt}{\isacharcomma}{\kern0pt}\isanewline
\ \ \ \ \ \ \ \ \ \ \ \ \ \ \ \ \ \ \ \ \ \ \ \ \ \ \ \ \ \ \ \ \ \ \ \ \ \ \ \ \ \ \ \ \ \ And{\isacharparenleft}{\kern0pt}Forall{\isacharparenleft}{\kern0pt}Implies{\isacharparenleft}{\kern0pt}Member{\isacharparenleft}{\kern0pt}{\isadigit{0}}{\isacharcomma}{\kern0pt}\ {\isadigit{2}}{\isacharparenright}{\kern0pt}{\isacharcomma}{\kern0pt}\ Member{\isacharparenleft}{\kern0pt}{\isadigit{0}}{\isacharcomma}{\kern0pt}\ {\isadigit{3}}{\isacharparenright}{\kern0pt}{\isacharparenright}{\kern0pt}{\isacharparenright}{\kern0pt}{\isacharcomma}{\kern0pt}\isanewline
\ \ \ \ \ \ \ \ \ \ \ \ \ \ \ \ \ \ \ \ \ \ \ \ \ \ \ \ \ \ \ \ \ \ \ \ \ \ \ \ \ \ \ \ \ \ \ \ \ \ Forall\isanewline
\ \ \ \ \ \ \ \ \ \ \ \ \ \ \ \ \ \ \ \ \ \ \ \ \ \ \ \ \ \ \ \ \ \ \ \ \ \ \ \ \ \ \ \ \ \ \ \ \ \ \ {\isacharparenleft}{\kern0pt}Implies\isanewline
\ \ \ \ \ \ \ \ \ \ \ \ \ \ \ \ \ \ \ \ \ \ \ \ \ \ \ \ \ \ \ \ \ \ \ \ \ \ \ \ \ \ \ \ \ \ \ \ \ \ \ \ \ {\isacharparenleft}{\kern0pt}Member{\isacharparenleft}{\kern0pt}{\isadigit{0}}{\isacharcomma}{\kern0pt}\ {\isadigit{1}}{\isacharparenright}{\kern0pt}{\isacharcomma}{\kern0pt}\isanewline
\ \ \ \ \ \ \ \ \ \ \ \ \ \ \ \ \ \ \ \ \ \ \ \ \ \ \ \ \ \ \ \ \ \ \ \ \ \ \ \ \ \ \ \ \ \ \ \ \ \ \ \ \ \ Exists\isanewline
\ \ \ \ \ \ \ \ \ \ \ \ \ \ \ \ \ \ \ \ \ \ \ \ \ \ \ \ \ \ \ \ \ \ \ \ \ \ \ \ \ \ \ \ \ \ \ \ \ \ \ \ \ \ \ {\isacharparenleft}{\kern0pt}Exists\isanewline
\ \ \ \ \ \ \ \ \ \ \ \ \ \ \ \ \ \ \ \ \ \ \ \ \ \ \ \ \ \ \ \ \ \ \ \ \ \ \ \ \ \ \ \ \ \ \ \ \ \ \ \ \ \ \ \ \ {\isacharparenleft}{\kern0pt}And{\isacharparenleft}{\kern0pt}pair{\isacharunderscore}{\kern0pt}fm{\isacharparenleft}{\kern0pt}{\isadigit{2}}{\isacharcomma}{\kern0pt}\ {\isadigit{5}}{\isacharcomma}{\kern0pt}\ {\isadigit{1}}{\isacharparenright}{\kern0pt}{\isacharcomma}{\kern0pt}\isanewline
\ \ \ \ \ \ \ \ \ \ \ \ \ \ \ \ \ \ \ \ \ \ \ \ \ \ \ \ \ \ \ \ \ \ \ \ \ \ \ \ \ \ \ \ \ \ \ \ \ \ \ \ \ \ \ \ \ \ \ \ \ \ And{\isacharparenleft}{\kern0pt}fun{\isacharunderscore}{\kern0pt}apply{\isacharunderscore}{\kern0pt}fm{\isacharparenleft}{\kern0pt}{\isadigit{9}}{\isacharcomma}{\kern0pt}\ {\isadigit{1}}{\isacharcomma}{\kern0pt}\ {\isadigit{0}}{\isacharparenright}{\kern0pt}{\isacharcomma}{\kern0pt}\isanewline
\ \ \ \ \ \ \ \ \ \ \ \ \ \ \ \ \ \ \ \ \ \ \ \ \ \ \ \ \ \ \ \ \ \ \ \ \ \ \ \ \ \ \ \ \ \ \ \ \ \ \ \ \ \ \ \ \ \ \ \ \ \ \ \ \ \ Forall{\isacharparenleft}{\kern0pt}Iff{\isacharparenleft}{\kern0pt}Member{\isacharparenleft}{\kern0pt}{\isadigit{0}}{\isacharcomma}{\kern0pt}\ {\isadigit{1}}{\isacharparenright}{\kern0pt}{\isacharcomma}{\kern0pt}\ empty{\isacharunderscore}{\kern0pt}fm{\isacharparenleft}{\kern0pt}{\isadigit{0}}{\isacharparenright}{\kern0pt}{\isacharparenright}{\kern0pt}{\isacharparenright}{\kern0pt}{\isacharparenright}{\kern0pt}{\isacharparenright}{\kern0pt}{\isacharparenright}{\kern0pt}{\isacharparenright}{\kern0pt}{\isacharparenright}{\kern0pt}{\isacharparenright}{\kern0pt}{\isacharparenright}{\kern0pt}{\isacharparenright}{\kern0pt}{\isacharparenright}{\kern0pt}{\isacharparenright}{\kern0pt}{\isacharparenright}{\kern0pt}{\isacharparenright}{\kern0pt}{\isacharparenright}{\kern0pt}{\isacharparenright}{\kern0pt}{\isacharparenright}{\kern0pt}{\isacharparenright}{\kern0pt}{\isacharparenright}{\kern0pt}{\isacharparenright}{\kern0pt}\ \ \ {\isachardoublequoteclose}\isanewline
\isanewline
\isanewline
\isacommand{context}\isamarkupfalse%
\ forcing{\isacharunderscore}{\kern0pt}data\isanewline
\isakeyword{begin}\ \isanewline
\isanewline
\isacommand{lemma}\isamarkupfalse%
\ His{\isacharunderscore}{\kern0pt}P{\isacharunderscore}{\kern0pt}name{\isacharunderscore}{\kern0pt}M{\isacharunderscore}{\kern0pt}fm{\isacharunderscore}{\kern0pt}iff{\isacharunderscore}{\kern0pt}sats\ {\isacharcolon}{\kern0pt}\ \isanewline
\ \ {\isachardoublequoteopen}v\ {\isasymin}\ M\ {\isasymLongrightarrow}\ g\ {\isasymin}\ M\ {\isasymLongrightarrow}\ x{\isacharprime}{\kern0pt}\ {\isasymin}\ M\ {\isasymLongrightarrow}\ env\ {\isasymin}\ list{\isacharparenleft}{\kern0pt}M{\isacharparenright}{\kern0pt}\ {\isasymLongrightarrow}\ sats{\isacharparenleft}{\kern0pt}M{\isacharcomma}{\kern0pt}\ His{\isacharunderscore}{\kern0pt}P{\isacharunderscore}{\kern0pt}name{\isacharunderscore}{\kern0pt}M{\isacharunderscore}{\kern0pt}fm{\isacharcomma}{\kern0pt}\ {\isacharbrackleft}{\kern0pt}v{\isacharcomma}{\kern0pt}\ g{\isacharcomma}{\kern0pt}\ x{\isacharprime}{\kern0pt}{\isacharbrackright}{\kern0pt}\ {\isacharat}{\kern0pt}\ env{\isacharparenright}{\kern0pt}\ {\isasymlongleftrightarrow}\ v\ {\isacharequal}{\kern0pt}\ His{\isacharunderscore}{\kern0pt}P{\isacharunderscore}{\kern0pt}name{\isacharunderscore}{\kern0pt}M{\isacharparenleft}{\kern0pt}x{\isacharprime}{\kern0pt}{\isacharcomma}{\kern0pt}\ g{\isacharparenright}{\kern0pt}{\isachardoublequoteclose}\ \isanewline
%
\isadelimproof
\isanewline
\ \ %
\endisadelimproof
%
\isatagproof
\isacommand{apply}\isamarkupfalse%
{\isacharparenleft}{\kern0pt}rule{\isacharunderscore}{\kern0pt}tac\ Q{\isacharequal}{\kern0pt}{\isachardoublequoteopen}{\isasymforall}e\ {\isasymin}\ M{\isachardot}{\kern0pt}\ e\ {\isasymin}\ v\ {\isasymlongleftrightarrow}\ {\isacharparenleft}{\kern0pt}empty{\isacharparenleft}{\kern0pt}{\isacharhash}{\kern0pt}{\isacharhash}{\kern0pt}M{\isacharcomma}{\kern0pt}\ e{\isacharparenright}{\kern0pt}\ {\isasymand}\ His{\isacharunderscore}{\kern0pt}P{\isacharunderscore}{\kern0pt}name{\isacharunderscore}{\kern0pt}M{\isacharunderscore}{\kern0pt}cond{\isacharparenleft}{\kern0pt}x{\isacharprime}{\kern0pt}{\isacharcomma}{\kern0pt}\ g{\isacharparenright}{\kern0pt}{\isacharparenright}{\kern0pt}{\isachardoublequoteclose}\ \isakeyword{in}\ iff{\isacharunderscore}{\kern0pt}trans{\isacharparenright}{\kern0pt}\isanewline
\ \ \ \isacommand{apply}\isamarkupfalse%
{\isacharparenleft}{\kern0pt}rule{\isacharunderscore}{\kern0pt}tac\ iff{\isacharunderscore}{\kern0pt}flip{\isacharparenright}{\kern0pt}\ \isanewline
\ \ \isacommand{unfolding}\isamarkupfalse%
\ His{\isacharunderscore}{\kern0pt}P{\isacharunderscore}{\kern0pt}name{\isacharunderscore}{\kern0pt}M{\isacharunderscore}{\kern0pt}fm{\isacharunderscore}{\kern0pt}def\ \isanewline
\ \ \ \isacommand{apply}\isamarkupfalse%
{\isacharparenleft}{\kern0pt}rule{\isacharunderscore}{\kern0pt}tac\ v{\isacharequal}{\kern0pt}v\ \isakeyword{in}\ His{\isacharunderscore}{\kern0pt}P{\isacharunderscore}{\kern0pt}name{\isacharunderscore}{\kern0pt}M{\isacharunderscore}{\kern0pt}fm{\isacharunderscore}{\kern0pt}auto{\isacharparenright}{\kern0pt}\ \isanewline
\ \ \ \ \ \ \isacommand{apply}\isamarkupfalse%
\ simp{\isacharunderscore}{\kern0pt}all\ \ \ \isanewline
\ \ \isacommand{unfolding}\isamarkupfalse%
\ His{\isacharunderscore}{\kern0pt}P{\isacharunderscore}{\kern0pt}name{\isacharunderscore}{\kern0pt}M{\isacharunderscore}{\kern0pt}def\ \isanewline
\ \ \isacommand{apply}\isamarkupfalse%
{\isacharparenleft}{\kern0pt}rule{\isacharunderscore}{\kern0pt}tac\ iffD{\isadigit{2}}{\isacharparenright}{\kern0pt}\ \isanewline
\ \ \ \isacommand{apply}\isamarkupfalse%
{\isacharparenleft}{\kern0pt}rule{\isacharunderscore}{\kern0pt}tac\ P{\isacharequal}{\kern0pt}{\isachardoublequoteopen}{\isasymlambda}X{\isachardot}{\kern0pt}\ {\isacharparenleft}{\kern0pt}{\isacharparenleft}{\kern0pt}{\isasymforall}e{\isasymin}M{\isachardot}{\kern0pt}\ e\ {\isasymin}\ v\ {\isasymlongleftrightarrow}\ e\ {\isacharequal}{\kern0pt}\ {\isadigit{0}}\ {\isasymand}\ His{\isacharunderscore}{\kern0pt}P{\isacharunderscore}{\kern0pt}name{\isacharunderscore}{\kern0pt}M{\isacharunderscore}{\kern0pt}cond{\isacharparenleft}{\kern0pt}x{\isacharprime}{\kern0pt}{\isacharcomma}{\kern0pt}\ g{\isacharparenright}{\kern0pt}{\isacharparenright}{\kern0pt}{\isacharparenright}{\kern0pt}\ {\isasymlongleftrightarrow}\ v\ {\isacharequal}{\kern0pt}\ X{\isachardoublequoteclose}\ \isakeyword{in}\ split{\isacharunderscore}{\kern0pt}if{\isacharparenright}{\kern0pt}\ \isanewline
\ \ \isacommand{apply}\isamarkupfalse%
{\isacharparenleft}{\kern0pt}rule\ conjI{\isacharparenright}{\kern0pt}\ \isanewline
\ \ \ \isacommand{apply}\isamarkupfalse%
{\isacharparenleft}{\kern0pt}rule\ impI{\isacharparenright}{\kern0pt}\ \isanewline
\ \ \ \isacommand{apply}\isamarkupfalse%
{\isacharparenleft}{\kern0pt}rule{\isacharunderscore}{\kern0pt}tac\ P{\isacharequal}{\kern0pt}{\isachardoublequoteopen}His{\isacharunderscore}{\kern0pt}P{\isacharunderscore}{\kern0pt}name{\isacharunderscore}{\kern0pt}M{\isacharunderscore}{\kern0pt}cond{\isacharparenleft}{\kern0pt}x{\isacharprime}{\kern0pt}{\isacharcomma}{\kern0pt}\ g{\isacharparenright}{\kern0pt}{\isachardoublequoteclose}\ \isakeyword{in}\ mp{\isacharparenright}{\kern0pt}\ \isanewline
\ \ \ \ \isacommand{apply}\isamarkupfalse%
\ {\isacharparenleft}{\kern0pt}rule\ impI{\isacharsemicolon}{\kern0pt}\ simp{\isacharparenright}{\kern0pt}\ \isanewline
\ \ \ \ \isacommand{apply}\isamarkupfalse%
{\isacharparenleft}{\kern0pt}rule\ iffI{\isacharparenright}{\kern0pt}\ \isanewline
\ \ \ \ \ \isacommand{apply}\isamarkupfalse%
{\isacharparenleft}{\kern0pt}rule{\isacharunderscore}{\kern0pt}tac\ is{\isacharunderscore}{\kern0pt}{\isadigit{1}}D{\isacharparenright}{\kern0pt}\ \isanewline
\ \ \ \ \ \ \isacommand{apply}\isamarkupfalse%
\ simp\ \isanewline
\ \ \ \ \ \isacommand{apply}\isamarkupfalse%
\ simp\ \isanewline
\ \ \ \ \isacommand{apply}\isamarkupfalse%
{\isacharparenleft}{\kern0pt}clarify{\isacharparenright}{\kern0pt}\ \isanewline
\ \ \ \ \isacommand{apply}\isamarkupfalse%
\ blast\ \isanewline
\ \ \isacommand{using}\isamarkupfalse%
\ His{\isacharunderscore}{\kern0pt}P{\isacharunderscore}{\kern0pt}name{\isacharunderscore}{\kern0pt}M{\isacharunderscore}{\kern0pt}cond{\isacharunderscore}{\kern0pt}iff\ \isanewline
\ \ \ \isacommand{apply}\isamarkupfalse%
\ blast\ \isanewline
\ \ \isacommand{apply}\isamarkupfalse%
\ clarify\isanewline
\ \ \isacommand{apply}\isamarkupfalse%
{\isacharparenleft}{\kern0pt}rule{\isacharunderscore}{\kern0pt}tac\ P{\isacharequal}{\kern0pt}{\isachardoublequoteopen}{\isasymnot}His{\isacharunderscore}{\kern0pt}P{\isacharunderscore}{\kern0pt}name{\isacharunderscore}{\kern0pt}M{\isacharunderscore}{\kern0pt}cond{\isacharparenleft}{\kern0pt}x{\isacharprime}{\kern0pt}{\isacharcomma}{\kern0pt}\ g{\isacharparenright}{\kern0pt}{\isachardoublequoteclose}\ \isakeyword{in}\ mp{\isacharparenright}{\kern0pt}\isanewline
\ \ \isacommand{using}\isamarkupfalse%
\ empty{\isacharunderscore}{\kern0pt}abs\ empty{\isacharunderscore}{\kern0pt}def\ \isanewline
\ \ \ \isacommand{apply}\isamarkupfalse%
\ simp\ \isanewline
\ \ \isacommand{using}\isamarkupfalse%
\ His{\isacharunderscore}{\kern0pt}P{\isacharunderscore}{\kern0pt}name{\isacharunderscore}{\kern0pt}M{\isacharunderscore}{\kern0pt}cond{\isacharunderscore}{\kern0pt}iff\ \isanewline
\ \ \isacommand{apply}\isamarkupfalse%
\ blast\ \isanewline
\ \ \isacommand{done}\isamarkupfalse%
%
\endisatagproof
{\isafoldproof}%
%
\isadelimproof
\isanewline
%
\endisadelimproof
\isanewline
\isacommand{end}\isamarkupfalse%
\isanewline
\isanewline
\isacommand{definition}\isamarkupfalse%
\ is{\isacharunderscore}{\kern0pt}P{\isacharunderscore}{\kern0pt}name{\isacharunderscore}{\kern0pt}fm\ \isakeyword{where}\ {\isachardoublequoteopen}is{\isacharunderscore}{\kern0pt}P{\isacharunderscore}{\kern0pt}name{\isacharunderscore}{\kern0pt}fm{\isacharparenleft}{\kern0pt}p{\isacharcomma}{\kern0pt}\ x{\isacharparenright}{\kern0pt}\ {\isasymequiv}\ Exists{\isacharparenleft}{\kern0pt}And{\isacharparenleft}{\kern0pt}is{\isacharunderscore}{\kern0pt}memrel{\isacharunderscore}{\kern0pt}wftrec{\isacharunderscore}{\kern0pt}fm{\isacharparenleft}{\kern0pt}His{\isacharunderscore}{\kern0pt}P{\isacharunderscore}{\kern0pt}name{\isacharunderscore}{\kern0pt}M{\isacharunderscore}{\kern0pt}fm{\isacharcomma}{\kern0pt}\ x\ {\isacharhash}{\kern0pt}{\isacharplus}{\kern0pt}\ {\isadigit{1}}{\isacharcomma}{\kern0pt}\ p\ {\isacharhash}{\kern0pt}{\isacharplus}{\kern0pt}\ {\isadigit{1}}{\isacharcomma}{\kern0pt}\ {\isadigit{0}}{\isacharparenright}{\kern0pt}{\isacharcomma}{\kern0pt}\ is{\isacharunderscore}{\kern0pt}{\isadigit{1}}{\isacharunderscore}{\kern0pt}fm{\isacharparenleft}{\kern0pt}{\isadigit{0}}{\isacharparenright}{\kern0pt}{\isacharparenright}{\kern0pt}{\isacharparenright}{\kern0pt}{\isachardoublequoteclose}\ \isanewline
\isanewline
\isacommand{context}\isamarkupfalse%
\ forcing{\isacharunderscore}{\kern0pt}data\isanewline
\isakeyword{begin}\ \isanewline
\isanewline
\isacommand{lemma}\isamarkupfalse%
\ is{\isacharunderscore}{\kern0pt}P{\isacharunderscore}{\kern0pt}name{\isacharunderscore}{\kern0pt}fm{\isacharunderscore}{\kern0pt}type\ {\isacharcolon}{\kern0pt}\ \isanewline
\ \ \isakeyword{fixes}\ i\ j\ \isanewline
\ \ \isakeyword{assumes}\ {\isachardoublequoteopen}i\ {\isasymin}\ nat{\isachardoublequoteclose}\ {\isachardoublequoteopen}j\ {\isasymin}\ nat{\isachardoublequoteclose}\ \isanewline
\ \ \isakeyword{shows}\ {\isachardoublequoteopen}is{\isacharunderscore}{\kern0pt}P{\isacharunderscore}{\kern0pt}name{\isacharunderscore}{\kern0pt}fm{\isacharparenleft}{\kern0pt}i{\isacharcomma}{\kern0pt}\ j{\isacharparenright}{\kern0pt}\ {\isasymin}\ formula{\isachardoublequoteclose}\ \isanewline
%
\isadelimproof
\ \ %
\endisadelimproof
%
\isatagproof
\isacommand{unfolding}\isamarkupfalse%
\ is{\isacharunderscore}{\kern0pt}P{\isacharunderscore}{\kern0pt}name{\isacharunderscore}{\kern0pt}fm{\isacharunderscore}{\kern0pt}def\ \isanewline
\ \ \isacommand{apply}\isamarkupfalse%
{\isacharparenleft}{\kern0pt}rule\ Exists{\isacharunderscore}{\kern0pt}type{\isacharcomma}{\kern0pt}\ rule\ And{\isacharunderscore}{\kern0pt}type{\isacharcomma}{\kern0pt}\ rule\ is{\isacharunderscore}{\kern0pt}memrel{\isacharunderscore}{\kern0pt}wftrec{\isacharunderscore}{\kern0pt}fm{\isacharunderscore}{\kern0pt}type{\isacharparenright}{\kern0pt}\isanewline
\ \ \isacommand{unfolding}\isamarkupfalse%
\ His{\isacharunderscore}{\kern0pt}P{\isacharunderscore}{\kern0pt}name{\isacharunderscore}{\kern0pt}M{\isacharunderscore}{\kern0pt}fm{\isacharunderscore}{\kern0pt}def\ \isanewline
\ \ \isacommand{using}\isamarkupfalse%
\ assms\ \isanewline
\ \ \ \ \ \ \isacommand{apply}\isamarkupfalse%
\ auto{\isacharbrackleft}{\kern0pt}{\isadigit{4}}{\isacharbrackright}{\kern0pt}\isanewline
\ \ \isacommand{apply}\isamarkupfalse%
{\isacharparenleft}{\kern0pt}rule\ is{\isacharunderscore}{\kern0pt}{\isadigit{1}}{\isacharunderscore}{\kern0pt}fm{\isacharunderscore}{\kern0pt}type{\isacharparenright}{\kern0pt}\isanewline
\ \ \isacommand{by}\isamarkupfalse%
\ auto%
\endisatagproof
{\isafoldproof}%
%
\isadelimproof
\isanewline
%
\endisadelimproof
\isanewline
\isacommand{lemma}\isamarkupfalse%
\ arity{\isacharunderscore}{\kern0pt}is{\isacharunderscore}{\kern0pt}P{\isacharunderscore}{\kern0pt}name{\isacharunderscore}{\kern0pt}fm\ {\isacharcolon}{\kern0pt}\ \isanewline
\ \ \isakeyword{fixes}\ i\ j\ \isanewline
\ \ \isakeyword{assumes}\ {\isachardoublequoteopen}i\ {\isasymin}\ nat{\isachardoublequoteclose}\ {\isachardoublequoteopen}j\ {\isasymin}\ nat{\isachardoublequoteclose}\ \isanewline
\ \ \isakeyword{shows}\ {\isachardoublequoteopen}arity{\isacharparenleft}{\kern0pt}is{\isacharunderscore}{\kern0pt}P{\isacharunderscore}{\kern0pt}name{\isacharunderscore}{\kern0pt}fm{\isacharparenleft}{\kern0pt}i{\isacharcomma}{\kern0pt}\ j{\isacharparenright}{\kern0pt}{\isacharparenright}{\kern0pt}\ {\isasymle}\ succ{\isacharparenleft}{\kern0pt}i{\isacharparenright}{\kern0pt}\ {\isasymunion}\ succ{\isacharparenleft}{\kern0pt}j{\isacharparenright}{\kern0pt}{\isachardoublequoteclose}\ \isanewline
%
\isadelimproof
\ \ %
\endisadelimproof
%
\isatagproof
\isacommand{apply}\isamarkupfalse%
{\isacharparenleft}{\kern0pt}subgoal{\isacharunderscore}{\kern0pt}tac\ {\isachardoublequoteopen}is{\isacharunderscore}{\kern0pt}memrel{\isacharunderscore}{\kern0pt}wftrec{\isacharunderscore}{\kern0pt}fm{\isacharparenleft}{\kern0pt}His{\isacharunderscore}{\kern0pt}P{\isacharunderscore}{\kern0pt}name{\isacharunderscore}{\kern0pt}M{\isacharunderscore}{\kern0pt}fm{\isacharcomma}{\kern0pt}\ succ{\isacharparenleft}{\kern0pt}j{\isacharparenright}{\kern0pt}{\isacharcomma}{\kern0pt}\ succ{\isacharparenleft}{\kern0pt}i{\isacharparenright}{\kern0pt}{\isacharcomma}{\kern0pt}\ {\isadigit{0}}{\isacharparenright}{\kern0pt}\ {\isasymin}\ formula{\isachardoublequoteclose}{\isacharparenright}{\kern0pt}\isanewline
\ \ \isacommand{apply}\isamarkupfalse%
{\isacharparenleft}{\kern0pt}subgoal{\isacharunderscore}{\kern0pt}tac\ {\isachardoublequoteopen}is{\isacharunderscore}{\kern0pt}{\isadigit{1}}{\isacharunderscore}{\kern0pt}fm{\isacharparenleft}{\kern0pt}{\isadigit{0}}{\isacharparenright}{\kern0pt}\ {\isasymin}\ formula{\isachardoublequoteclose}{\isacharparenright}{\kern0pt}\isanewline
\ \ \isacommand{unfolding}\isamarkupfalse%
\ is{\isacharunderscore}{\kern0pt}P{\isacharunderscore}{\kern0pt}name{\isacharunderscore}{\kern0pt}fm{\isacharunderscore}{\kern0pt}def\ \isanewline
\ \ \isacommand{apply}\isamarkupfalse%
\ simp\isanewline
\ \ \isacommand{using}\isamarkupfalse%
\ assms\isanewline
\ \ \ \ \isacommand{apply}\isamarkupfalse%
{\isacharparenleft}{\kern0pt}subst\ pred{\isacharunderscore}{\kern0pt}Un{\isacharunderscore}{\kern0pt}distrib{\isacharcomma}{\kern0pt}\ simp{\isacharunderscore}{\kern0pt}all{\isacharparenright}{\kern0pt}\isanewline
\ \ \ \ \isacommand{apply}\isamarkupfalse%
{\isacharparenleft}{\kern0pt}rule\ Un{\isacharunderscore}{\kern0pt}least{\isacharunderscore}{\kern0pt}lt{\isacharcomma}{\kern0pt}\ rule\ pred{\isacharunderscore}{\kern0pt}le{\isacharcomma}{\kern0pt}\ simp{\isacharcomma}{\kern0pt}\ simp{\isacharparenright}{\kern0pt}\isanewline
\ \ \ \ \ \isacommand{apply}\isamarkupfalse%
{\isacharparenleft}{\kern0pt}rule\ le{\isacharunderscore}{\kern0pt}trans{\isacharcomma}{\kern0pt}\ rule\ arity{\isacharunderscore}{\kern0pt}is{\isacharunderscore}{\kern0pt}memrel{\isacharunderscore}{\kern0pt}wftrec{\isacharunderscore}{\kern0pt}fm{\isacharparenright}{\kern0pt}\isanewline
\ \ \ \ \ \ \ \ \ \ \isacommand{apply}\isamarkupfalse%
{\isacharparenleft}{\kern0pt}simp\ add{\isacharcolon}{\kern0pt}His{\isacharunderscore}{\kern0pt}P{\isacharunderscore}{\kern0pt}name{\isacharunderscore}{\kern0pt}M{\isacharunderscore}{\kern0pt}fm{\isacharunderscore}{\kern0pt}def{\isacharparenright}{\kern0pt}{\isacharplus}{\kern0pt}\isanewline
\ \ \ \ \ \ \ \ \ \isacommand{apply}\isamarkupfalse%
{\isacharparenleft}{\kern0pt}simp\ del{\isacharcolon}{\kern0pt}FOL{\isacharunderscore}{\kern0pt}sats{\isacharunderscore}{\kern0pt}iff\ pair{\isacharunderscore}{\kern0pt}abs\ add{\isacharcolon}{\kern0pt}\ fm{\isacharunderscore}{\kern0pt}defs\ nat{\isacharunderscore}{\kern0pt}simp{\isacharunderscore}{\kern0pt}union{\isacharparenright}{\kern0pt}\ \isanewline
\ \ \ \ \ \ \ \ \isacommand{apply}\isamarkupfalse%
\ auto{\isacharbrackleft}{\kern0pt}{\isadigit{3}}{\isacharbrackright}{\kern0pt}\isanewline
\ \ \ \ \ \isacommand{apply}\isamarkupfalse%
{\isacharparenleft}{\kern0pt}rule\ Un{\isacharunderscore}{\kern0pt}least{\isacharunderscore}{\kern0pt}lt{\isacharparenright}{\kern0pt}{\isacharplus}{\kern0pt}\isanewline
\ \ \ \ \ \ \ \isacommand{apply}\isamarkupfalse%
{\isacharparenleft}{\kern0pt}simp{\isacharcomma}{\kern0pt}\ rule\ ltI{\isacharcomma}{\kern0pt}\ simp{\isacharcomma}{\kern0pt}\ simp{\isacharparenright}{\kern0pt}{\isacharplus}{\kern0pt}\isanewline
\ \ \ \ \ \isacommand{apply}\isamarkupfalse%
\ simp\isanewline
\ \ \ \ \isacommand{apply}\isamarkupfalse%
{\isacharparenleft}{\kern0pt}rule\ pred{\isacharunderscore}{\kern0pt}le{\isacharcomma}{\kern0pt}\ simp{\isacharunderscore}{\kern0pt}all{\isacharparenright}{\kern0pt}\isanewline
\ \ \ \ \isacommand{apply}\isamarkupfalse%
{\isacharparenleft}{\kern0pt}rule{\isacharunderscore}{\kern0pt}tac\ j{\isacharequal}{\kern0pt}{\isadigit{1}}\ \isakeyword{in}\ le{\isacharunderscore}{\kern0pt}trans{\isacharcomma}{\kern0pt}\ simp\ add{\isacharcolon}{\kern0pt}is{\isacharunderscore}{\kern0pt}{\isadigit{1}}{\isacharunderscore}{\kern0pt}fm{\isacharunderscore}{\kern0pt}def{\isacharparenright}{\kern0pt}\isanewline
\ \ \ \ \ \isacommand{apply}\isamarkupfalse%
{\isacharparenleft}{\kern0pt}subst\ arity{\isacharunderscore}{\kern0pt}empty{\isacharunderscore}{\kern0pt}fm{\isacharcomma}{\kern0pt}\ simp{\isacharparenright}{\kern0pt}\isanewline
\ \ \ \ \ \isacommand{apply}\isamarkupfalse%
{\isacharparenleft}{\kern0pt}simp\ del{\isacharcolon}{\kern0pt}FOL{\isacharunderscore}{\kern0pt}sats{\isacharunderscore}{\kern0pt}iff\ pair{\isacharunderscore}{\kern0pt}abs\ add{\isacharcolon}{\kern0pt}\ fm{\isacharunderscore}{\kern0pt}defs\ nat{\isacharunderscore}{\kern0pt}simp{\isacharunderscore}{\kern0pt}union{\isacharcomma}{\kern0pt}\ simp{\isacharparenright}{\kern0pt}\isanewline
\ \ \ \isacommand{apply}\isamarkupfalse%
{\isacharparenleft}{\kern0pt}rule\ is{\isacharunderscore}{\kern0pt}{\isadigit{1}}{\isacharunderscore}{\kern0pt}fm{\isacharunderscore}{\kern0pt}type{\isacharcomma}{\kern0pt}\ simp{\isacharparenright}{\kern0pt}\isanewline
\ \ \isacommand{apply}\isamarkupfalse%
{\isacharparenleft}{\kern0pt}rule\ is{\isacharunderscore}{\kern0pt}memrel{\isacharunderscore}{\kern0pt}wftrec{\isacharunderscore}{\kern0pt}fm{\isacharunderscore}{\kern0pt}type{\isacharparenright}{\kern0pt}\isanewline
\ \ \ \ \ \isacommand{apply}\isamarkupfalse%
{\isacharparenleft}{\kern0pt}simp\ add{\isacharcolon}{\kern0pt}His{\isacharunderscore}{\kern0pt}P{\isacharunderscore}{\kern0pt}name{\isacharunderscore}{\kern0pt}M{\isacharunderscore}{\kern0pt}fm{\isacharunderscore}{\kern0pt}def{\isacharparenright}{\kern0pt}\isanewline
\ \ \isacommand{by}\isamarkupfalse%
\ auto%
\endisatagproof
{\isafoldproof}%
%
\isadelimproof
\ \ \isanewline
%
\endisadelimproof
\isanewline
\isacommand{lemma}\isamarkupfalse%
\ sats{\isacharunderscore}{\kern0pt}is{\isacharunderscore}{\kern0pt}P{\isacharunderscore}{\kern0pt}name{\isacharunderscore}{\kern0pt}fm{\isacharunderscore}{\kern0pt}iff\ {\isacharcolon}{\kern0pt}\ \isanewline
\ \ \isakeyword{fixes}\ env\ i\ j\ x\ \isanewline
\ \ \isakeyword{assumes}\ {\isachardoublequoteopen}env\ {\isasymin}\ list{\isacharparenleft}{\kern0pt}M{\isacharparenright}{\kern0pt}{\isachardoublequoteclose}\ {\isachardoublequoteopen}i\ {\isacharless}{\kern0pt}\ length{\isacharparenleft}{\kern0pt}env{\isacharparenright}{\kern0pt}{\isachardoublequoteclose}\ {\isachardoublequoteopen}j\ {\isacharless}{\kern0pt}\ length{\isacharparenleft}{\kern0pt}env{\isacharparenright}{\kern0pt}{\isachardoublequoteclose}\ {\isachardoublequoteopen}nth{\isacharparenleft}{\kern0pt}i{\isacharcomma}{\kern0pt}\ env{\isacharparenright}{\kern0pt}\ {\isacharequal}{\kern0pt}\ P{\isachardoublequoteclose}\ {\isachardoublequoteopen}nth{\isacharparenleft}{\kern0pt}j{\isacharcomma}{\kern0pt}\ env{\isacharparenright}{\kern0pt}\ {\isacharequal}{\kern0pt}\ x{\isachardoublequoteclose}\ \ \isanewline
\ \ \isakeyword{shows}\ {\isachardoublequoteopen}sats{\isacharparenleft}{\kern0pt}M{\isacharcomma}{\kern0pt}\ is{\isacharunderscore}{\kern0pt}P{\isacharunderscore}{\kern0pt}name{\isacharunderscore}{\kern0pt}fm{\isacharparenleft}{\kern0pt}i{\isacharcomma}{\kern0pt}\ j{\isacharparenright}{\kern0pt}{\isacharcomma}{\kern0pt}\ env{\isacharparenright}{\kern0pt}\ {\isasymlongleftrightarrow}\ x\ {\isasymin}\ P{\isacharunderscore}{\kern0pt}names{\isachardoublequoteclose}\ \isanewline
%
\isadelimproof
%
\endisadelimproof
%
\isatagproof
\isacommand{proof}\isamarkupfalse%
\ {\isacharminus}{\kern0pt}\ \isanewline
\ \ \isacommand{have}\isamarkupfalse%
\ inat\ {\isacharcolon}{\kern0pt}\ {\isachardoublequoteopen}i\ {\isasymin}\ nat{\isachardoublequoteclose}\ \isacommand{using}\isamarkupfalse%
\ assms\ lt{\isacharunderscore}{\kern0pt}nat{\isacharunderscore}{\kern0pt}in{\isacharunderscore}{\kern0pt}nat\ \isacommand{by}\isamarkupfalse%
\ auto\isanewline
\ \ \isacommand{have}\isamarkupfalse%
\ jnat\ {\isacharcolon}{\kern0pt}\ {\isachardoublequoteopen}j\ {\isasymin}\ nat{\isachardoublequoteclose}\ \isacommand{using}\isamarkupfalse%
\ assms\ lt{\isacharunderscore}{\kern0pt}nat{\isacharunderscore}{\kern0pt}in{\isacharunderscore}{\kern0pt}nat\ \isacommand{by}\isamarkupfalse%
\ auto\isanewline
\ \ \isacommand{have}\isamarkupfalse%
\ xinM\ {\isacharcolon}{\kern0pt}\ {\isachardoublequoteopen}x\ {\isasymin}\ M{\isachardoublequoteclose}\ \isacommand{using}\isamarkupfalse%
\ assms\ \isacommand{by}\isamarkupfalse%
\ auto\isanewline
\isanewline
\ \ \isacommand{have}\isamarkupfalse%
\ Heq{\isacharcolon}{\kern0pt}\ {\isachardoublequoteopen}{\isasymAnd}h\ g\ x{\isachardot}{\kern0pt}\ h\ {\isasymin}\ eclose{\isacharparenleft}{\kern0pt}x{\isacharparenright}{\kern0pt}\ {\isasymrightarrow}\ M\ {\isasymLongrightarrow}\ g\ {\isasymin}\ eclose{\isacharparenleft}{\kern0pt}x{\isacharparenright}{\kern0pt}\ {\isasymtimes}\ {\isacharbraceleft}{\kern0pt}P{\isacharbraceright}{\kern0pt}\ {\isasymrightarrow}\ M\ {\isasymLongrightarrow}\ g\ {\isasymin}\ M\isanewline
\ \ \ \ \ \ \ \ \ \ \ \ \ \ \ {\isasymLongrightarrow}\ x\ {\isasymin}\ M\ {\isasymLongrightarrow}\ {\isacharparenleft}{\kern0pt}{\isasymAnd}y{\isachardot}{\kern0pt}\ y\ {\isasymin}\ eclose{\isacharparenleft}{\kern0pt}x{\isacharparenright}{\kern0pt}\ {\isasymLongrightarrow}\ h{\isacharbackquote}{\kern0pt}y\ {\isacharequal}{\kern0pt}\ g{\isacharbackquote}{\kern0pt}{\isacharless}{\kern0pt}y{\isacharcomma}{\kern0pt}\ P{\isachargreater}{\kern0pt}{\isacharparenright}{\kern0pt}\ {\isasymLongrightarrow}\ His{\isacharunderscore}{\kern0pt}P{\isacharunderscore}{\kern0pt}name{\isacharparenleft}{\kern0pt}x{\isacharcomma}{\kern0pt}\ h{\isacharparenright}{\kern0pt}\ {\isacharequal}{\kern0pt}\ His{\isacharunderscore}{\kern0pt}P{\isacharunderscore}{\kern0pt}name{\isacharunderscore}{\kern0pt}M{\isacharparenleft}{\kern0pt}{\isacharless}{\kern0pt}x{\isacharcomma}{\kern0pt}\ P{\isachargreater}{\kern0pt}{\isacharcomma}{\kern0pt}\ g{\isacharparenright}{\kern0pt}{\isachardoublequoteclose}\ \ \ \isanewline
\ \ \ \ \isacommand{unfolding}\isamarkupfalse%
\ His{\isacharunderscore}{\kern0pt}P{\isacharunderscore}{\kern0pt}name{\isacharunderscore}{\kern0pt}def\ His{\isacharunderscore}{\kern0pt}P{\isacharunderscore}{\kern0pt}name{\isacharunderscore}{\kern0pt}M{\isacharunderscore}{\kern0pt}def\ \isanewline
\ \ \ \ \isacommand{apply}\isamarkupfalse%
{\isacharparenleft}{\kern0pt}rule\ if{\isacharunderscore}{\kern0pt}cong{\isacharcomma}{\kern0pt}\ rule\ iffI{\isacharcomma}{\kern0pt}\ simp\ add{\isacharcolon}{\kern0pt}P{\isacharunderscore}{\kern0pt}in{\isacharunderscore}{\kern0pt}M{\isacharcomma}{\kern0pt}\ rule\ ballI{\isacharparenright}{\kern0pt}\isanewline
\ \ \ \ \ \ \ \isacommand{apply}\isamarkupfalse%
{\isacharparenleft}{\kern0pt}rename{\isacharunderscore}{\kern0pt}tac\ h\ g\ x\ y{\isacharcomma}{\kern0pt}\ subgoal{\isacharunderscore}{\kern0pt}tac\ {\isachardoublequoteopen}y\ {\isasymin}\ eclose{\isacharparenleft}{\kern0pt}x{\isacharparenright}{\kern0pt}{\isachardoublequoteclose}{\isacharparenright}{\kern0pt}\isanewline
\ \ \ \ \ \ \ \ \isacommand{apply}\isamarkupfalse%
\ auto{\isacharbrackleft}{\kern0pt}{\isadigit{2}}{\isacharbrackright}{\kern0pt}\isanewline
\ \ \ \ \ \ \ \isacommand{apply}\isamarkupfalse%
{\isacharparenleft}{\kern0pt}rule\ domain{\isacharunderscore}{\kern0pt}elem{\isacharunderscore}{\kern0pt}in{\isacharunderscore}{\kern0pt}eclose{\isacharcomma}{\kern0pt}\ force{\isacharparenright}{\kern0pt}\isanewline
\ \ \ \ \ \ \isacommand{apply}\isamarkupfalse%
\ simp\isanewline
\ \ \ \ \isacommand{apply}\isamarkupfalse%
{\isacharparenleft}{\kern0pt}rule\ ballI{\isacharparenright}{\kern0pt}\isanewline
\ \ \ \ \ \ \isacommand{apply}\isamarkupfalse%
{\isacharparenleft}{\kern0pt}rename{\isacharunderscore}{\kern0pt}tac\ h\ g\ x\ y{\isacharcomma}{\kern0pt}\ subgoal{\isacharunderscore}{\kern0pt}tac\ {\isachardoublequoteopen}y\ {\isasymin}\ eclose{\isacharparenleft}{\kern0pt}x{\isacharparenright}{\kern0pt}{\isachardoublequoteclose}{\isacharparenright}{\kern0pt}\isanewline
\ \ \ \ \ \ \ \isacommand{apply}\isamarkupfalse%
\ auto{\isacharbrackleft}{\kern0pt}{\isadigit{1}}{\isacharbrackright}{\kern0pt}\isanewline
\ \ \ \ \ \ \isacommand{apply}\isamarkupfalse%
{\isacharparenleft}{\kern0pt}rule\ domain{\isacharunderscore}{\kern0pt}elem{\isacharunderscore}{\kern0pt}in{\isacharunderscore}{\kern0pt}eclose{\isacharparenright}{\kern0pt}\isanewline
\ \ \ \ \isacommand{by}\isamarkupfalse%
\ auto\isanewline
\ \ \ \ \isanewline
\ \ \isacommand{have}\isamarkupfalse%
\ {\isachardoublequoteopen}sats{\isacharparenleft}{\kern0pt}M{\isacharcomma}{\kern0pt}\ is{\isacharunderscore}{\kern0pt}P{\isacharunderscore}{\kern0pt}name{\isacharunderscore}{\kern0pt}fm{\isacharparenleft}{\kern0pt}i{\isacharcomma}{\kern0pt}\ j{\isacharparenright}{\kern0pt}{\isacharcomma}{\kern0pt}\ env{\isacharparenright}{\kern0pt}\ {\isasymlongleftrightarrow}\ {\isacharparenleft}{\kern0pt}{\isasymexists}v\ {\isasymin}\ M{\isachardot}{\kern0pt}\ v\ {\isacharequal}{\kern0pt}\ wftrec{\isacharparenleft}{\kern0pt}Memrel{\isacharparenleft}{\kern0pt}M{\isacharparenright}{\kern0pt}{\isacharcircum}{\kern0pt}{\isacharplus}{\kern0pt}{\isacharcomma}{\kern0pt}\ x{\isacharcomma}{\kern0pt}\ His{\isacharunderscore}{\kern0pt}P{\isacharunderscore}{\kern0pt}name{\isacharparenright}{\kern0pt}\ {\isasymand}\ v\ {\isacharequal}{\kern0pt}\ {\isadigit{1}}{\isacharparenright}{\kern0pt}{\isachardoublequoteclose}\ \isanewline
\ \ \ \ \isacommand{unfolding}\isamarkupfalse%
\ is{\isacharunderscore}{\kern0pt}P{\isacharunderscore}{\kern0pt}name{\isacharunderscore}{\kern0pt}fm{\isacharunderscore}{\kern0pt}def\ \isanewline
\ \ \ \ \isacommand{apply}\isamarkupfalse%
{\isacharparenleft}{\kern0pt}rule\ iff{\isacharunderscore}{\kern0pt}trans{\isacharcomma}{\kern0pt}\ rule\ sats{\isacharunderscore}{\kern0pt}Exists{\isacharunderscore}{\kern0pt}iff{\isacharcomma}{\kern0pt}\ simp\ add{\isacharcolon}{\kern0pt}assms{\isacharcomma}{\kern0pt}\ rule\ bex{\isacharunderscore}{\kern0pt}iff{\isacharparenright}{\kern0pt}\isanewline
\ \ \ \ \isacommand{apply}\isamarkupfalse%
{\isacharparenleft}{\kern0pt}rule\ iff{\isacharunderscore}{\kern0pt}trans{\isacharcomma}{\kern0pt}\ rule\ sats{\isacharunderscore}{\kern0pt}And{\isacharunderscore}{\kern0pt}iff{\isacharcomma}{\kern0pt}\ simp\ add{\isacharcolon}{\kern0pt}assms{\isacharcomma}{\kern0pt}\ rule\ iff{\isacharunderscore}{\kern0pt}conjI{\isadigit{2}}{\isacharparenright}{\kern0pt}\isanewline
\ \ \ \ \ \isacommand{apply}\isamarkupfalse%
{\isacharparenleft}{\kern0pt}rule{\isacharunderscore}{\kern0pt}tac\ a{\isacharequal}{\kern0pt}P\ \isakeyword{and}\ G{\isacharequal}{\kern0pt}His{\isacharunderscore}{\kern0pt}P{\isacharunderscore}{\kern0pt}name{\isacharunderscore}{\kern0pt}M\ \isanewline
\ \ \ \ \ \ \ \ \isakeyword{in}\ sats{\isacharunderscore}{\kern0pt}is{\isacharunderscore}{\kern0pt}memrel{\isacharunderscore}{\kern0pt}wftrec{\isacharunderscore}{\kern0pt}fm{\isacharunderscore}{\kern0pt}iff{\isacharparenright}{\kern0pt}\isanewline
\ \ \ \ \isacommand{using}\isamarkupfalse%
\ assms\ singleton{\isacharunderscore}{\kern0pt}in{\isacharunderscore}{\kern0pt}M{\isacharunderscore}{\kern0pt}iff\ P{\isacharunderscore}{\kern0pt}in{\isacharunderscore}{\kern0pt}M\ xinM\ inat\ jnat\isanewline
\ \ \ \ \ \ \ \ \ \ \ \ \ \ \ \ \ \ \ \isacommand{apply}\isamarkupfalse%
\ auto{\isacharbrackleft}{\kern0pt}{\isadigit{1}}{\isadigit{0}}{\isacharbrackright}{\kern0pt}\isanewline
\ \ \ \ \ \ \ \ \ \isacommand{apply}\isamarkupfalse%
{\isacharparenleft}{\kern0pt}simp\ add{\isacharcolon}{\kern0pt}\ His{\isacharunderscore}{\kern0pt}P{\isacharunderscore}{\kern0pt}name{\isacharunderscore}{\kern0pt}M{\isacharunderscore}{\kern0pt}fm{\isacharunderscore}{\kern0pt}def{\isacharparenright}{\kern0pt}\isanewline
\ \ \ \ \ \ \ \ \isacommand{apply}\isamarkupfalse%
\ {\isacharparenleft}{\kern0pt}simp\ add{\isacharcolon}{\kern0pt}\ His{\isacharunderscore}{\kern0pt}P{\isacharunderscore}{\kern0pt}name{\isacharunderscore}{\kern0pt}M{\isacharunderscore}{\kern0pt}fm{\isacharunderscore}{\kern0pt}def{\isacharcomma}{\kern0pt}\ simp\ del{\isacharcolon}{\kern0pt}FOL{\isacharunderscore}{\kern0pt}sats{\isacharunderscore}{\kern0pt}iff\ pair{\isacharunderscore}{\kern0pt}abs\ add{\isacharcolon}{\kern0pt}\ fm{\isacharunderscore}{\kern0pt}defs\ nat{\isacharunderscore}{\kern0pt}simp{\isacharunderscore}{\kern0pt}union{\isacharparenright}{\kern0pt}\ \isanewline
\ \ \ \ \ \ \ \isacommand{apply}\isamarkupfalse%
{\isacharparenleft}{\kern0pt}simp\ add{\isacharcolon}{\kern0pt}\ His{\isacharunderscore}{\kern0pt}P{\isacharunderscore}{\kern0pt}name{\isacharunderscore}{\kern0pt}M{\isacharunderscore}{\kern0pt}def{\isacharcomma}{\kern0pt}\ rule\ impI{\isacharcomma}{\kern0pt}\ simp\ add{\isacharcolon}{\kern0pt}zero{\isacharunderscore}{\kern0pt}in{\isacharunderscore}{\kern0pt}M{\isacharparenright}{\kern0pt}\isanewline
\ \ \ \ \ \ \isacommand{apply}\isamarkupfalse%
{\isacharparenleft}{\kern0pt}rule\ Heq{\isacharparenright}{\kern0pt}\isanewline
\ \ \ \ \ \ \ \ \ \ \ \isacommand{apply}\isamarkupfalse%
\ auto{\isacharbrackleft}{\kern0pt}{\isadigit{5}}{\isacharbrackright}{\kern0pt}\isanewline
\ \ \ \ \ \isacommand{apply}\isamarkupfalse%
{\isacharparenleft}{\kern0pt}rule\ iff{\isacharunderscore}{\kern0pt}flip{\isacharcomma}{\kern0pt}\ rule\ His{\isacharunderscore}{\kern0pt}P{\isacharunderscore}{\kern0pt}name{\isacharunderscore}{\kern0pt}M{\isacharunderscore}{\kern0pt}fm{\isacharunderscore}{\kern0pt}iff{\isacharunderscore}{\kern0pt}sats{\isacharparenright}{\kern0pt}\isanewline
\ \ \ \ \ \ \ \ \isacommand{apply}\isamarkupfalse%
\ auto{\isacharbrackleft}{\kern0pt}{\isadigit{4}}{\isacharbrackright}{\kern0pt}\isanewline
\ \ \ \ \isacommand{apply}\isamarkupfalse%
{\isacharparenleft}{\kern0pt}rule\ sats{\isacharunderscore}{\kern0pt}is{\isacharunderscore}{\kern0pt}{\isadigit{1}}{\isacharunderscore}{\kern0pt}fm{\isacharunderscore}{\kern0pt}iff{\isacharparenright}{\kern0pt}\isanewline
\ \ \ \ \isacommand{using}\isamarkupfalse%
\ assms\ \isanewline
\ \ \ \ \isacommand{by}\isamarkupfalse%
\ auto\isanewline
\ \ \isacommand{also}\isamarkupfalse%
\ \isacommand{have}\isamarkupfalse%
\ {\isachardoublequoteopen}{\isachardot}{\kern0pt}{\isachardot}{\kern0pt}{\isachardot}{\kern0pt}\ {\isasymlongleftrightarrow}\ is{\isacharunderscore}{\kern0pt}P{\isacharunderscore}{\kern0pt}name{\isacharparenleft}{\kern0pt}x{\isacharparenright}{\kern0pt}\ {\isacharequal}{\kern0pt}\ {\isadigit{1}}{\isachardoublequoteclose}\ \isacommand{unfolding}\isamarkupfalse%
\ is{\isacharunderscore}{\kern0pt}P{\isacharunderscore}{\kern0pt}name{\isacharunderscore}{\kern0pt}def\ \isacommand{by}\isamarkupfalse%
\ auto\isanewline
\ \ \isacommand{also}\isamarkupfalse%
\ \isacommand{have}\isamarkupfalse%
\ {\isachardoublequoteopen}{\isachardot}{\kern0pt}{\isachardot}{\kern0pt}{\isachardot}{\kern0pt}\ {\isasymlongleftrightarrow}\ x\ {\isasymin}\ P{\isacharunderscore}{\kern0pt}names{\isachardoublequoteclose}\ \ \isanewline
\ \ \ \ \isacommand{apply}\isamarkupfalse%
{\isacharparenleft}{\kern0pt}rule\ iff{\isacharunderscore}{\kern0pt}flip{\isacharcomma}{\kern0pt}\ rule\ def{\isacharunderscore}{\kern0pt}is{\isacharunderscore}{\kern0pt}P{\isacharunderscore}{\kern0pt}name{\isacharparenright}{\kern0pt}\isanewline
\ \ \ \ \isacommand{using}\isamarkupfalse%
\ assms\isanewline
\ \ \ \ \isacommand{by}\isamarkupfalse%
\ auto\ \isanewline
\ \ \isacommand{finally}\isamarkupfalse%
\ \isacommand{show}\isamarkupfalse%
\ {\isacharquery}{\kern0pt}thesis\ \isacommand{by}\isamarkupfalse%
\ simp\isanewline
\isacommand{qed}\isamarkupfalse%
%
\endisatagproof
{\isafoldproof}%
%
\isadelimproof
\isanewline
%
\endisadelimproof
\isanewline
\isacommand{end}\isamarkupfalse%
\ \isanewline
%
\isadelimtheory
%
\endisadelimtheory
%
\isatagtheory
\isacommand{end}\isamarkupfalse%
%
\endisatagtheory
{\isafoldtheory}%
%
\isadelimtheory
%
\endisadelimtheory
%
\end{isabellebody}%
\endinput
%:%file=~/source/repos/ZF-notAC/code/P_Names_M.thy%:%
%:%10=1%:%
%:%11=1%:%
%:%12=2%:%
%:%13=3%:%
%:%14=4%:%
%:%15=5%:%
%:%16=6%:%
%:%21=6%:%
%:%24=7%:%
%:%25=8%:%
%:%26=8%:%
%:%27=9%:%
%:%28=10%:%
%:%29=11%:%
%:%30=11%:%
%:%31=12%:%
%:%32=13%:%
%:%33=13%:%
%:%36=14%:%
%:%40=14%:%
%:%41=14%:%
%:%42=15%:%
%:%43=15%:%
%:%44=16%:%
%:%45=16%:%
%:%50=16%:%
%:%53=17%:%
%:%54=18%:%
%:%55=18%:%
%:%56=19%:%
%:%57=20%:%
%:%58=21%:%
%:%61=22%:%
%:%65=22%:%
%:%66=22%:%
%:%67=23%:%
%:%68=23%:%
%:%69=24%:%
%:%70=24%:%
%:%71=25%:%
%:%72=25%:%
%:%73=26%:%
%:%74=26%:%
%:%79=26%:%
%:%82=27%:%
%:%83=28%:%
%:%84=28%:%
%:%85=29%:%
%:%86=30%:%
%:%87=30%:%
%:%88=31%:%
%:%89=32%:%
%:%90=32%:%
%:%91=33%:%
%:%92=34%:%
%:%93=35%:%
%:%94=35%:%
%:%95=36%:%
%:%96=37%:%
%:%97=38%:%
%:%100=39%:%
%:%104=39%:%
%:%105=39%:%
%:%106=39%:%
%:%111=39%:%
%:%114=40%:%
%:%115=41%:%
%:%116=41%:%
%:%117=42%:%
%:%118=43%:%
%:%121=44%:%
%:%125=44%:%
%:%126=44%:%
%:%127=45%:%
%:%128=45%:%
%:%129=46%:%
%:%130=46%:%
%:%131=47%:%
%:%132=47%:%
%:%133=48%:%
%:%134=48%:%
%:%135=49%:%
%:%141=49%:%
%:%144=50%:%
%:%145=51%:%
%:%146=51%:%
%:%147=52%:%
%:%148=53%:%
%:%149=54%:%
%:%152=55%:%
%:%153=56%:%
%:%157=56%:%
%:%158=56%:%
%:%159=57%:%
%:%160=57%:%
%:%161=58%:%
%:%162=58%:%
%:%163=59%:%
%:%164=59%:%
%:%165=60%:%
%:%166=60%:%
%:%167=61%:%
%:%168=61%:%
%:%169=62%:%
%:%170=62%:%
%:%171=63%:%
%:%172=63%:%
%:%173=64%:%
%:%174=64%:%
%:%175=65%:%
%:%176=65%:%
%:%177=66%:%
%:%178=66%:%
%:%179=67%:%
%:%180=67%:%
%:%185=67%:%
%:%188=68%:%
%:%189=69%:%
%:%190=69%:%
%:%191=70%:%
%:%192=71%:%
%:%193=71%:%
%:%194=72%:%
%:%195=73%:%
%:%196=73%:%
%:%197=74%:%
%:%198=75%:%
%:%199=76%:%
%:%202=77%:%
%:%207=78%:%
%:%208=78%:%
%:%209=79%:%
%:%210=79%:%
%:%211=80%:%
%:%212=80%:%
%:%213=81%:%
%:%214=81%:%
%:%215=81%:%
%:%216=82%:%
%:%217=82%:%
%:%218=83%:%
%:%219=83%:%
%:%220=84%:%
%:%221=84%:%
%:%222=85%:%
%:%223=85%:%
%:%224=86%:%
%:%225=86%:%
%:%226=87%:%
%:%227=87%:%
%:%228=87%:%
%:%229=88%:%
%:%230=88%:%
%:%231=89%:%
%:%232=89%:%
%:%233=90%:%
%:%234=90%:%
%:%235=91%:%
%:%236=91%:%
%:%237=92%:%
%:%238=92%:%
%:%239=92%:%
%:%240=93%:%
%:%241=93%:%
%:%242=94%:%
%:%243=94%:%
%:%244=95%:%
%:%245=95%:%
%:%246=96%:%
%:%247=96%:%
%:%248=97%:%
%:%249=97%:%
%:%250=98%:%
%:%251=98%:%
%:%252=99%:%
%:%253=99%:%
%:%254=99%:%
%:%255=100%:%
%:%256=100%:%
%:%257=101%:%
%:%258=101%:%
%:%259=102%:%
%:%260=102%:%
%:%261=103%:%
%:%262=103%:%
%:%263=104%:%
%:%264=104%:%
%:%265=105%:%
%:%266=105%:%
%:%267=106%:%
%:%268=106%:%
%:%269=107%:%
%:%270=107%:%
%:%271=107%:%
%:%272=108%:%
%:%273=108%:%
%:%274=109%:%
%:%275=109%:%
%:%276=109%:%
%:%277=110%:%
%:%278=110%:%
%:%279=110%:%
%:%280=110%:%
%:%281=110%:%
%:%282=111%:%
%:%283=111%:%
%:%284=111%:%
%:%285=111%:%
%:%286=111%:%
%:%287=112%:%
%:%288=112%:%
%:%289=112%:%
%:%290=112%:%
%:%291=112%:%
%:%292=113%:%
%:%293=113%:%
%:%294=113%:%
%:%295=113%:%
%:%296=113%:%
%:%297=114%:%
%:%298=114%:%
%:%299=115%:%
%:%300=115%:%
%:%301=116%:%
%:%302=116%:%
%:%303=117%:%
%:%304=117%:%
%:%305=118%:%
%:%306=118%:%
%:%307=119%:%
%:%308=119%:%
%:%309=120%:%
%:%310=120%:%
%:%311=121%:%
%:%312=121%:%
%:%313=121%:%
%:%314=121%:%
%:%315=121%:%
%:%316=122%:%
%:%317=123%:%
%:%318=123%:%
%:%319=123%:%
%:%320=123%:%
%:%321=124%:%
%:%322=124%:%
%:%323=125%:%
%:%324=125%:%
%:%325=125%:%
%:%326=125%:%
%:%327=125%:%
%:%328=126%:%
%:%334=126%:%
%:%337=127%:%
%:%338=128%:%
%:%339=128%:%
%:%340=129%:%
%:%341=130%:%
%:%342=131%:%
%:%343=131%:%
%:%344=132%:%
%:%347=135%:%
%:%348=136%:%
%:%349=137%:%
%:%350=137%:%
%:%351=138%:%
%:%352=139%:%
%:%355=140%:%
%:%359=140%:%
%:%360=140%:%
%:%361=141%:%
%:%362=141%:%
%:%363=142%:%
%:%364=142%:%
%:%365=143%:%
%:%366=143%:%
%:%367=144%:%
%:%368=144%:%
%:%369=145%:%
%:%370=145%:%
%:%371=146%:%
%:%372=146%:%
%:%373=147%:%
%:%374=147%:%
%:%375=148%:%
%:%376=148%:%
%:%377=149%:%
%:%378=149%:%
%:%379=150%:%
%:%380=150%:%
%:%381=151%:%
%:%382=151%:%
%:%383=152%:%
%:%384=152%:%
%:%385=153%:%
%:%386=153%:%
%:%391=153%:%
%:%394=154%:%
%:%395=155%:%
%:%396=155%:%
%:%397=156%:%
%:%398=157%:%
%:%399=158%:%
%:%400=159%:%
%:%401=160%:%
%:%402=161%:%
%:%403=162%:%
%:%406=163%:%
%:%410=163%:%
%:%411=163%:%
%:%412=164%:%
%:%413=164%:%
%:%418=164%:%
%:%421=165%:%
%:%422=166%:%
%:%423=166%:%
%:%424=167%:%
%:%425=168%:%
%:%426=168%:%
%:%446=188%:%
%:%447=189%:%
%:%448=190%:%
%:%449=191%:%
%:%450=191%:%
%:%451=192%:%
%:%452=193%:%
%:%453=194%:%
%:%454=194%:%
%:%455=195%:%
%:%458=196%:%
%:%459=197%:%
%:%463=197%:%
%:%464=197%:%
%:%465=198%:%
%:%466=198%:%
%:%467=199%:%
%:%468=199%:%
%:%469=200%:%
%:%470=200%:%
%:%471=201%:%
%:%472=201%:%
%:%473=202%:%
%:%474=202%:%
%:%475=203%:%
%:%476=203%:%
%:%477=204%:%
%:%478=204%:%
%:%479=205%:%
%:%480=205%:%
%:%481=206%:%
%:%482=206%:%
%:%483=207%:%
%:%484=207%:%
%:%485=208%:%
%:%486=208%:%
%:%487=209%:%
%:%488=209%:%
%:%489=210%:%
%:%490=210%:%
%:%491=211%:%
%:%492=211%:%
%:%493=212%:%
%:%494=212%:%
%:%495=213%:%
%:%496=213%:%
%:%497=214%:%
%:%498=214%:%
%:%499=215%:%
%:%500=215%:%
%:%501=216%:%
%:%502=216%:%
%:%503=217%:%
%:%504=217%:%
%:%505=218%:%
%:%506=218%:%
%:%507=219%:%
%:%508=219%:%
%:%509=220%:%
%:%510=220%:%
%:%511=221%:%
%:%512=221%:%
%:%513=222%:%
%:%514=222%:%
%:%515=223%:%
%:%521=223%:%
%:%524=224%:%
%:%525=225%:%
%:%526=225%:%
%:%527=226%:%
%:%528=227%:%
%:%529=227%:%
%:%530=228%:%
%:%531=229%:%
%:%532=229%:%
%:%533=230%:%
%:%534=231%:%
%:%535=232%:%
%:%536=232%:%
%:%537=233%:%
%:%538=234%:%
%:%539=235%:%
%:%542=236%:%
%:%546=236%:%
%:%547=236%:%
%:%548=237%:%
%:%549=237%:%
%:%550=238%:%
%:%551=238%:%
%:%552=239%:%
%:%553=239%:%
%:%554=240%:%
%:%555=240%:%
%:%556=241%:%
%:%557=241%:%
%:%558=242%:%
%:%559=242%:%
%:%564=242%:%
%:%567=243%:%
%:%568=244%:%
%:%569=244%:%
%:%570=245%:%
%:%571=246%:%
%:%572=247%:%
%:%575=248%:%
%:%579=248%:%
%:%580=248%:%
%:%581=249%:%
%:%582=249%:%
%:%583=250%:%
%:%584=250%:%
%:%585=251%:%
%:%586=251%:%
%:%587=252%:%
%:%588=252%:%
%:%589=253%:%
%:%590=253%:%
%:%591=254%:%
%:%592=254%:%
%:%593=255%:%
%:%594=255%:%
%:%595=256%:%
%:%596=256%:%
%:%597=257%:%
%:%598=257%:%
%:%599=258%:%
%:%600=258%:%
%:%601=259%:%
%:%602=259%:%
%:%603=260%:%
%:%604=260%:%
%:%605=261%:%
%:%606=261%:%
%:%607=262%:%
%:%608=262%:%
%:%609=263%:%
%:%610=263%:%
%:%611=264%:%
%:%612=264%:%
%:%613=265%:%
%:%614=265%:%
%:%615=266%:%
%:%616=266%:%
%:%617=267%:%
%:%618=267%:%
%:%619=268%:%
%:%620=268%:%
%:%621=269%:%
%:%622=269%:%
%:%627=269%:%
%:%630=270%:%
%:%631=271%:%
%:%632=271%:%
%:%633=272%:%
%:%634=273%:%
%:%635=274%:%
%:%642=275%:%
%:%643=275%:%
%:%644=276%:%
%:%645=276%:%
%:%646=276%:%
%:%647=276%:%
%:%648=277%:%
%:%649=277%:%
%:%650=277%:%
%:%651=277%:%
%:%652=278%:%
%:%653=278%:%
%:%654=278%:%
%:%655=278%:%
%:%656=279%:%
%:%657=280%:%
%:%658=280%:%
%:%659=281%:%
%:%660=282%:%
%:%661=282%:%
%:%662=283%:%
%:%663=283%:%
%:%664=284%:%
%:%665=284%:%
%:%666=285%:%
%:%667=285%:%
%:%668=286%:%
%:%669=286%:%
%:%670=287%:%
%:%671=287%:%
%:%672=288%:%
%:%673=288%:%
%:%674=289%:%
%:%675=289%:%
%:%676=290%:%
%:%677=290%:%
%:%678=291%:%
%:%679=291%:%
%:%680=292%:%
%:%681=292%:%
%:%682=293%:%
%:%683=294%:%
%:%684=294%:%
%:%685=295%:%
%:%686=295%:%
%:%687=296%:%
%:%688=296%:%
%:%689=297%:%
%:%690=297%:%
%:%691=298%:%
%:%692=298%:%
%:%693=299%:%
%:%694=300%:%
%:%695=300%:%
%:%696=301%:%
%:%697=301%:%
%:%698=302%:%
%:%699=302%:%
%:%700=303%:%
%:%701=303%:%
%:%702=304%:%
%:%703=304%:%
%:%704=305%:%
%:%705=305%:%
%:%706=306%:%
%:%707=306%:%
%:%708=307%:%
%:%709=307%:%
%:%710=308%:%
%:%711=308%:%
%:%712=309%:%
%:%713=309%:%
%:%714=310%:%
%:%715=310%:%
%:%716=311%:%
%:%717=311%:%
%:%718=312%:%
%:%719=312%:%
%:%720=312%:%
%:%721=312%:%
%:%722=312%:%
%:%723=313%:%
%:%724=313%:%
%:%725=313%:%
%:%726=314%:%
%:%727=314%:%
%:%728=315%:%
%:%729=315%:%
%:%730=316%:%
%:%731=316%:%
%:%732=317%:%
%:%733=317%:%
%:%734=317%:%
%:%735=317%:%
%:%736=318%:%
%:%742=318%:%
%:%745=319%:%
%:%746=320%:%
%:%747=320%:%
%:%754=321%:%

%
\begin{isabellebody}%
\setisabellecontext{Automorphism{\isacharunderscore}{\kern0pt}Definition}%
%
\isadelimtheory
%
\endisadelimtheory
%
\isatagtheory
\isacommand{theory}\isamarkupfalse%
\ Automorphism{\isacharunderscore}{\kern0pt}Definition\isanewline
\ \ \isakeyword{imports}\ \isanewline
\ \ \ \ {\isachardoublequoteopen}Forcing{\isacharslash}{\kern0pt}Forcing{\isacharunderscore}{\kern0pt}Main{\isachardoublequoteclose}\ \isanewline
\ \ \ \ P{\isacharunderscore}{\kern0pt}Names{\isacharunderscore}{\kern0pt}M\isanewline
\isakeyword{begin}%
\endisatagtheory
{\isafoldtheory}%
%
\isadelimtheory
\ \isanewline
%
\endisadelimtheory
\isanewline
\isacommand{locale}\isamarkupfalse%
\ forcing{\isacharunderscore}{\kern0pt}data{\isacharunderscore}{\kern0pt}partial\ {\isacharequal}{\kern0pt}\ \isanewline
\ \ forcing{\isacharunderscore}{\kern0pt}data\ {\isacharplus}{\kern0pt}\ \isanewline
\ \ \isakeyword{assumes}\ leq{\isacharunderscore}{\kern0pt}relation{\isacharunderscore}{\kern0pt}on{\isacharunderscore}{\kern0pt}P\ {\isacharcolon}{\kern0pt}\ {\isachardoublequoteopen}leq\ {\isasymin}\ Pow{\isacharparenleft}{\kern0pt}P\ {\isasymtimes}\ P{\isacharparenright}{\kern0pt}{\isachardoublequoteclose}\ \isanewline
\ \ \isakeyword{and}\ leq{\isacharunderscore}{\kern0pt}partial{\isacharunderscore}{\kern0pt}order\ {\isacharcolon}{\kern0pt}\ {\isachardoublequoteopen}partial{\isacharunderscore}{\kern0pt}order{\isacharunderscore}{\kern0pt}on{\isacharparenleft}{\kern0pt}P{\isacharcomma}{\kern0pt}\ leq{\isacharparenright}{\kern0pt}{\isachardoublequoteclose}\ \isanewline
\isakeyword{begin}\isanewline
\isanewline
\isacommand{lemma}\isamarkupfalse%
\ leq{\isacharunderscore}{\kern0pt}antisym\ {\isacharcolon}{\kern0pt}\ {\isachardoublequoteopen}antisym{\isacharparenleft}{\kern0pt}leq{\isacharparenright}{\kern0pt}{\isachardoublequoteclose}\ \isanewline
%
\isadelimproof
\ \ %
\endisadelimproof
%
\isatagproof
\isacommand{using}\isamarkupfalse%
\ leq{\isacharunderscore}{\kern0pt}partial{\isacharunderscore}{\kern0pt}order\ \isanewline
\ \ \isacommand{unfolding}\isamarkupfalse%
\ partial{\isacharunderscore}{\kern0pt}order{\isacharunderscore}{\kern0pt}on{\isacharunderscore}{\kern0pt}def\ \isanewline
\ \ \isacommand{by}\isamarkupfalse%
\ auto%
\endisatagproof
{\isafoldproof}%
%
\isadelimproof
\isanewline
%
\endisadelimproof
\isanewline
\isacommand{lemma}\isamarkupfalse%
\ one{\isacharunderscore}{\kern0pt}is{\isacharunderscore}{\kern0pt}unique{\isacharunderscore}{\kern0pt}max\ {\isacharcolon}{\kern0pt}\ {\isachardoublequoteopen}x\ {\isasymin}\ P\ {\isasymLongrightarrow}\ one\ {\isasympreceq}\ x\ {\isasymLongrightarrow}\ x\ {\isacharequal}{\kern0pt}\ one{\isachardoublequoteclose}\ \isanewline
%
\isadelimproof
%
\endisadelimproof
%
\isatagproof
\isacommand{proof}\isamarkupfalse%
{\isacharminus}{\kern0pt}\ \isanewline
\ \ \isacommand{assume}\isamarkupfalse%
\ assms\ {\isacharcolon}{\kern0pt}\ {\isachardoublequoteopen}x\ {\isasymin}\ P{\isachardoublequoteclose}\ {\isachardoublequoteopen}one\ {\isasympreceq}\ x{\isachardoublequoteclose}\isanewline
\ \ \isacommand{then}\isamarkupfalse%
\ \isacommand{have}\isamarkupfalse%
\ {\isachardoublequoteopen}x\ {\isasympreceq}\ one{\isachardoublequoteclose}\ \isacommand{using}\isamarkupfalse%
\ one{\isacharunderscore}{\kern0pt}max\ \isacommand{by}\isamarkupfalse%
\ auto\ \isanewline
\ \ \isacommand{then}\isamarkupfalse%
\ \isacommand{show}\isamarkupfalse%
\ {\isachardoublequoteopen}x\ {\isacharequal}{\kern0pt}\ one{\isachardoublequoteclose}\ \isacommand{using}\isamarkupfalse%
\ leq{\isacharunderscore}{\kern0pt}in{\isacharunderscore}{\kern0pt}M\ assms\ leq{\isacharunderscore}{\kern0pt}antisym\ \isacommand{unfolding}\isamarkupfalse%
\ antisym{\isacharunderscore}{\kern0pt}def\ \isacommand{by}\isamarkupfalse%
\ auto\isanewline
\isacommand{qed}\isamarkupfalse%
%
\endisatagproof
{\isafoldproof}%
%
\isadelimproof
\isanewline
%
\endisadelimproof
\isanewline
\isacommand{definition}\isamarkupfalse%
\ is{\isacharunderscore}{\kern0pt}P{\isacharunderscore}{\kern0pt}auto\ {\isacharcolon}{\kern0pt}{\isacharcolon}{\kern0pt}\ {\isachardoublequoteopen}i\ {\isasymRightarrow}\ o{\isachardoublequoteclose}\ \isakeyword{where}\isanewline
\ \ {\isachardoublequoteopen}is{\isacharunderscore}{\kern0pt}P{\isacharunderscore}{\kern0pt}auto{\isacharparenleft}{\kern0pt}{\isasympi}{\isacharparenright}{\kern0pt}\ {\isasymequiv}\ {\isasympi}\ {\isasymin}\ M\ {\isasymand}\ {\isasympi}\ {\isasymin}\ bij{\isacharparenleft}{\kern0pt}P{\isacharcomma}{\kern0pt}\ P{\isacharparenright}{\kern0pt}\ {\isasymand}\ {\isacharparenleft}{\kern0pt}{\isasymforall}\ p\ {\isasymin}\ P{\isachardot}{\kern0pt}\ {\isasymforall}q\ {\isasymin}\ P{\isachardot}{\kern0pt}\ p\ {\isasympreceq}\ q\ {\isasymlongleftrightarrow}\ {\isasympi}{\isacharbackquote}{\kern0pt}p\ {\isasympreceq}\ {\isasympi}{\isacharbackquote}{\kern0pt}q{\isacharparenright}{\kern0pt}{\isachardoublequoteclose}\ \ \isanewline
\isanewline
\isacommand{definition}\isamarkupfalse%
\ P{\isacharunderscore}{\kern0pt}auto\ \isakeyword{where}\ {\isachardoublequoteopen}P{\isacharunderscore}{\kern0pt}auto\ {\isasymequiv}\ {\isacharbraceleft}{\kern0pt}\ {\isasympi}\ {\isasymin}\ P\ {\isasymrightarrow}\ P{\isachardot}{\kern0pt}\ is{\isacharunderscore}{\kern0pt}P{\isacharunderscore}{\kern0pt}auto{\isacharparenleft}{\kern0pt}{\isasympi}{\isacharparenright}{\kern0pt}\ {\isacharbraceright}{\kern0pt}{\isachardoublequoteclose}\ \isanewline
\isanewline
\isacommand{lemma}\isamarkupfalse%
\ P{\isacharunderscore}{\kern0pt}auto{\isacharunderscore}{\kern0pt}type\ {\isacharcolon}{\kern0pt}\ {\isachardoublequoteopen}is{\isacharunderscore}{\kern0pt}P{\isacharunderscore}{\kern0pt}auto{\isacharparenleft}{\kern0pt}{\isasympi}{\isacharparenright}{\kern0pt}\ {\isasymLongrightarrow}\ {\isasympi}\ {\isasymin}\ P\ {\isasymrightarrow}\ P{\isachardoublequoteclose}%
\isadelimproof
\ %
\endisadelimproof
%
\isatagproof
\isacommand{unfolding}\isamarkupfalse%
\ is{\isacharunderscore}{\kern0pt}P{\isacharunderscore}{\kern0pt}auto{\isacharunderscore}{\kern0pt}def\ bij{\isacharunderscore}{\kern0pt}def\ surj{\isacharunderscore}{\kern0pt}def\ \isacommand{by}\isamarkupfalse%
\ auto%
\endisatagproof
{\isafoldproof}%
%
\isadelimproof
%
\endisadelimproof
\isanewline
\isanewline
\isacommand{lemma}\isamarkupfalse%
\ P{\isacharunderscore}{\kern0pt}auto{\isacharunderscore}{\kern0pt}is{\isacharunderscore}{\kern0pt}function\ {\isacharcolon}{\kern0pt}\ {\isachardoublequoteopen}is{\isacharunderscore}{\kern0pt}P{\isacharunderscore}{\kern0pt}auto{\isacharparenleft}{\kern0pt}{\isasympi}{\isacharparenright}{\kern0pt}\ {\isasymLongrightarrow}\ function{\isacharparenleft}{\kern0pt}{\isasympi}{\isacharparenright}{\kern0pt}{\isachardoublequoteclose}\ \isanewline
%
\isadelimproof
\ \ %
\endisadelimproof
%
\isatagproof
\isacommand{unfolding}\isamarkupfalse%
\ is{\isacharunderscore}{\kern0pt}P{\isacharunderscore}{\kern0pt}auto{\isacharunderscore}{\kern0pt}def\ \isanewline
\ \ \isacommand{using}\isamarkupfalse%
\ bij{\isacharunderscore}{\kern0pt}is{\isacharunderscore}{\kern0pt}fun\ bij{\isacharunderscore}{\kern0pt}is{\isacharunderscore}{\kern0pt}fun\ Pi{\isacharunderscore}{\kern0pt}def\ \isacommand{by}\isamarkupfalse%
\ auto%
\endisatagproof
{\isafoldproof}%
%
\isadelimproof
\isanewline
%
\endisadelimproof
\isanewline
\isacommand{lemma}\isamarkupfalse%
\ P{\isacharunderscore}{\kern0pt}auto{\isacharunderscore}{\kern0pt}domain\ {\isacharcolon}{\kern0pt}\ {\isachardoublequoteopen}is{\isacharunderscore}{\kern0pt}P{\isacharunderscore}{\kern0pt}auto{\isacharparenleft}{\kern0pt}{\isasympi}{\isacharparenright}{\kern0pt}\ {\isasymLongrightarrow}\ domain{\isacharparenleft}{\kern0pt}{\isasympi}{\isacharparenright}{\kern0pt}\ {\isacharequal}{\kern0pt}\ P{\isachardoublequoteclose}\ \isanewline
%
\isadelimproof
%
\endisadelimproof
%
\isatagproof
\isacommand{proof}\isamarkupfalse%
\ {\isacharminus}{\kern0pt}\ \isanewline
\ \ \isacommand{assume}\isamarkupfalse%
\ assms\ {\isacharcolon}{\kern0pt}\ {\isachardoublequoteopen}is{\isacharunderscore}{\kern0pt}P{\isacharunderscore}{\kern0pt}auto{\isacharparenleft}{\kern0pt}{\isasympi}{\isacharparenright}{\kern0pt}{\isachardoublequoteclose}\ \isanewline
\ \ \isacommand{have}\isamarkupfalse%
\ {\isachardoublequoteopen}{\isasympi}\ {\isasymin}\ P\ {\isasymrightarrow}\ P{\isachardoublequoteclose}\ \isacommand{using}\isamarkupfalse%
\ assms\ \isacommand{unfolding}\isamarkupfalse%
\ is{\isacharunderscore}{\kern0pt}P{\isacharunderscore}{\kern0pt}auto{\isacharunderscore}{\kern0pt}def\ \isacommand{using}\isamarkupfalse%
\ bij{\isacharunderscore}{\kern0pt}is{\isacharunderscore}{\kern0pt}fun\ \isacommand{by}\isamarkupfalse%
\ auto\ \isanewline
\ \ \isacommand{then}\isamarkupfalse%
\ \isacommand{show}\isamarkupfalse%
\ {\isachardoublequoteopen}domain{\isacharparenleft}{\kern0pt}{\isasympi}{\isacharparenright}{\kern0pt}\ {\isacharequal}{\kern0pt}\ P{\isachardoublequoteclose}\ \isacommand{unfolding}\isamarkupfalse%
\ Pi{\isacharunderscore}{\kern0pt}def\ \isacommand{by}\isamarkupfalse%
\ auto\ \isanewline
\isacommand{qed}\isamarkupfalse%
%
\endisatagproof
{\isafoldproof}%
%
\isadelimproof
\isanewline
%
\endisadelimproof
\isanewline
\isacommand{lemma}\isamarkupfalse%
\ P{\isacharunderscore}{\kern0pt}auto{\isacharunderscore}{\kern0pt}value\ {\isacharcolon}{\kern0pt}\ {\isachardoublequoteopen}is{\isacharunderscore}{\kern0pt}P{\isacharunderscore}{\kern0pt}auto{\isacharparenleft}{\kern0pt}{\isasympi}{\isacharparenright}{\kern0pt}\ {\isasymLongrightarrow}\ p\ {\isasymin}\ P\ {\isasymLongrightarrow}\ {\isasympi}{\isacharbackquote}{\kern0pt}p\ {\isasymin}\ P{\isachardoublequoteclose}\ \isanewline
%
\isadelimproof
%
\endisadelimproof
%
\isatagproof
\isacommand{proof}\isamarkupfalse%
\ {\isacharminus}{\kern0pt}\ \isanewline
\ \ \isacommand{assume}\isamarkupfalse%
\ {\isachardoublequoteopen}is{\isacharunderscore}{\kern0pt}P{\isacharunderscore}{\kern0pt}auto{\isacharparenleft}{\kern0pt}{\isasympi}{\isacharparenright}{\kern0pt}{\isachardoublequoteclose}\ {\isachardoublequoteopen}p\ {\isasymin}\ P\ {\isachardoublequoteclose}\isanewline
\ \ \isacommand{then}\isamarkupfalse%
\ \isacommand{have}\isamarkupfalse%
\ {\isachardoublequoteopen}{\isasympi}\ {\isasymin}\ P\ {\isasymrightarrow}\ P{\isachardoublequoteclose}\isanewline
\ \ \ \ \isacommand{unfolding}\isamarkupfalse%
\ is{\isacharunderscore}{\kern0pt}P{\isacharunderscore}{\kern0pt}auto{\isacharunderscore}{\kern0pt}def\ \isacommand{using}\isamarkupfalse%
\ bij{\isacharunderscore}{\kern0pt}is{\isacharunderscore}{\kern0pt}fun\ \isacommand{by}\isamarkupfalse%
\ auto\ \isanewline
\ \ \isacommand{then}\isamarkupfalse%
\ \isacommand{show}\isamarkupfalse%
\ {\isachardoublequoteopen}{\isasympi}{\isacharbackquote}{\kern0pt}p\ {\isasymin}\ P{\isachardoublequoteclose}\ \isacommand{using}\isamarkupfalse%
\ function{\isacharunderscore}{\kern0pt}value{\isacharunderscore}{\kern0pt}in\ {\isacartoucheopen}p\ {\isasymin}\ P{\isacartoucheclose}\ \isacommand{by}\isamarkupfalse%
\ auto\ \isanewline
\isacommand{qed}\isamarkupfalse%
%
\endisatagproof
{\isafoldproof}%
%
\isadelimproof
\ \isanewline
%
\endisadelimproof
\ \ \isanewline
\isacommand{lemma}\isamarkupfalse%
\ P{\isacharunderscore}{\kern0pt}auto{\isacharunderscore}{\kern0pt}preserves{\isacharunderscore}{\kern0pt}leq\ {\isacharcolon}{\kern0pt}\ \isanewline
\ \ \ {\isachardoublequoteopen}is{\isacharunderscore}{\kern0pt}P{\isacharunderscore}{\kern0pt}auto{\isacharparenleft}{\kern0pt}{\isasympi}{\isacharparenright}{\kern0pt}\ {\isasymLongrightarrow}\ p\ {\isasymin}\ P\ {\isasymLongrightarrow}\ q\ {\isasymin}\ P\ {\isasymLongrightarrow}\ p\ {\isasympreceq}\ q\ {\isasymLongrightarrow}\ {\isasympi}{\isacharbackquote}{\kern0pt}p\ {\isasympreceq}\ {\isasympi}{\isacharbackquote}{\kern0pt}q{\isachardoublequoteclose}\ \isanewline
%
\isadelimproof
\ \ %
\endisadelimproof
%
\isatagproof
\isacommand{unfolding}\isamarkupfalse%
\ is{\isacharunderscore}{\kern0pt}P{\isacharunderscore}{\kern0pt}auto{\isacharunderscore}{\kern0pt}def\ \isacommand{by}\isamarkupfalse%
\ auto%
\endisatagproof
{\isafoldproof}%
%
\isadelimproof
\ \isanewline
%
\endisadelimproof
\isanewline
\isacommand{lemma}\isamarkupfalse%
\ P{\isacharunderscore}{\kern0pt}auto{\isacharunderscore}{\kern0pt}preserves{\isacharunderscore}{\kern0pt}leq{\isacharprime}{\kern0pt}\ {\isacharcolon}{\kern0pt}\ \isanewline
\ \ {\isachardoublequoteopen}is{\isacharunderscore}{\kern0pt}P{\isacharunderscore}{\kern0pt}auto{\isacharparenleft}{\kern0pt}{\isasympi}{\isacharparenright}{\kern0pt}\ {\isasymLongrightarrow}\ p\ {\isasymin}\ P\ {\isasymLongrightarrow}\ q\ {\isasymin}\ P\ {\isasymLongrightarrow}\ {\isasympi}{\isacharbackquote}{\kern0pt}p\ {\isasympreceq}\ {\isasympi}{\isacharbackquote}{\kern0pt}q\ {\isasymLongrightarrow}\ p\ {\isasympreceq}\ q{\isachardoublequoteclose}\ \isanewline
%
\isadelimproof
\ \ %
\endisadelimproof
%
\isatagproof
\isacommand{unfolding}\isamarkupfalse%
\ is{\isacharunderscore}{\kern0pt}P{\isacharunderscore}{\kern0pt}auto{\isacharunderscore}{\kern0pt}def\ \isacommand{by}\isamarkupfalse%
\ auto%
\endisatagproof
{\isafoldproof}%
%
\isadelimproof
\ \isanewline
%
\endisadelimproof
\isanewline
\isacommand{lemma}\isamarkupfalse%
\ P{\isacharunderscore}{\kern0pt}auto{\isacharunderscore}{\kern0pt}idP\ {\isacharcolon}{\kern0pt}\ {\isachardoublequoteopen}is{\isacharunderscore}{\kern0pt}P{\isacharunderscore}{\kern0pt}auto{\isacharparenleft}{\kern0pt}id{\isacharparenleft}{\kern0pt}P{\isacharparenright}{\kern0pt}{\isacharparenright}{\kern0pt}{\isachardoublequoteclose}\isanewline
%
\isadelimproof
\ \ %
\endisadelimproof
%
\isatagproof
\isacommand{unfolding}\isamarkupfalse%
\ is{\isacharunderscore}{\kern0pt}P{\isacharunderscore}{\kern0pt}auto{\isacharunderscore}{\kern0pt}def\ \isanewline
\ \ \isacommand{apply}\isamarkupfalse%
\ {\isacharparenleft}{\kern0pt}rule\ conjI{\isacharparenright}{\kern0pt}\ \isacommand{using}\isamarkupfalse%
\ id{\isacharunderscore}{\kern0pt}closed\ P{\isacharunderscore}{\kern0pt}in{\isacharunderscore}{\kern0pt}M\ \isacommand{apply}\isamarkupfalse%
\ simp\ \isacommand{apply}\isamarkupfalse%
\ {\isacharparenleft}{\kern0pt}rule\ conjI{\isacharparenright}{\kern0pt}\ \isanewline
\ \ \isacommand{using}\isamarkupfalse%
\ id{\isacharunderscore}{\kern0pt}bij\ \isacommand{apply}\isamarkupfalse%
\ simp\ \isacommand{apply}\isamarkupfalse%
\ simp\ \isacommand{done}\isamarkupfalse%
%
\endisatagproof
{\isafoldproof}%
%
\isadelimproof
\ \isanewline
%
\endisadelimproof
\isanewline
\isacommand{lemma}\isamarkupfalse%
\ P{\isacharunderscore}{\kern0pt}auto{\isacharunderscore}{\kern0pt}converse\ {\isacharcolon}{\kern0pt}\ {\isachardoublequoteopen}is{\isacharunderscore}{\kern0pt}P{\isacharunderscore}{\kern0pt}auto{\isacharparenleft}{\kern0pt}{\isasympi}{\isacharparenright}{\kern0pt}\ {\isasymLongrightarrow}\ is{\isacharunderscore}{\kern0pt}P{\isacharunderscore}{\kern0pt}auto{\isacharparenleft}{\kern0pt}converse{\isacharparenleft}{\kern0pt}{\isasympi}{\isacharparenright}{\kern0pt}{\isacharparenright}{\kern0pt}{\isachardoublequoteclose}\ \isanewline
%
\isadelimproof
\ \ %
\endisadelimproof
%
\isatagproof
\isacommand{unfolding}\isamarkupfalse%
\ is{\isacharunderscore}{\kern0pt}P{\isacharunderscore}{\kern0pt}auto{\isacharunderscore}{\kern0pt}def\ \isacommand{apply}\isamarkupfalse%
\ {\isacharparenleft}{\kern0pt}rule{\isacharunderscore}{\kern0pt}tac\ conjI{\isacharparenright}{\kern0pt}\ \isacommand{using}\isamarkupfalse%
\ converse{\isacharunderscore}{\kern0pt}closed\ \isacommand{apply}\isamarkupfalse%
\ simp\ \isanewline
\ \ \isacommand{apply}\isamarkupfalse%
\ {\isacharparenleft}{\kern0pt}rule{\isacharunderscore}{\kern0pt}tac\ conjI{\isacharparenright}{\kern0pt}\ \isacommand{using}\isamarkupfalse%
\ bij{\isacharunderscore}{\kern0pt}converse{\isacharunderscore}{\kern0pt}bij\ \isacommand{apply}\isamarkupfalse%
\ simp\ \isacommand{apply}\isamarkupfalse%
\ {\isacharparenleft}{\kern0pt}clarify{\isacharparenright}{\kern0pt}\ \isanewline
\isacommand{proof}\isamarkupfalse%
\ {\isacharminus}{\kern0pt}\isanewline
\ \ \isacommand{fix}\isamarkupfalse%
\ p\ q\ \isanewline
\ \ \isacommand{assume}\isamarkupfalse%
\ assms\ {\isacharcolon}{\kern0pt}\ {\isachardoublequoteopen}{\isasympi}\ {\isasymin}\ bij{\isacharparenleft}{\kern0pt}P{\isacharcomma}{\kern0pt}\ P{\isacharparenright}{\kern0pt}{\isachardoublequoteclose}\ {\isachardoublequoteopen}p\ {\isasymin}\ P{\isachardoublequoteclose}\ {\isachardoublequoteopen}q\ {\isasymin}\ P{\isachardoublequoteclose}\ {\isachardoublequoteopen}{\isasymforall}p{\isasymin}P{\isachardot}{\kern0pt}\ {\isasymforall}q{\isasymin}P{\isachardot}{\kern0pt}\ p\ {\isasympreceq}\ q\ {\isasymlongleftrightarrow}\ {\isasympi}\ {\isacharbackquote}{\kern0pt}\ p\ {\isasympreceq}\ {\isasympi}\ {\isacharbackquote}{\kern0pt}\ q{\isachardoublequoteclose}\ \isanewline
\ \ \isacommand{then}\isamarkupfalse%
\ \isacommand{have}\isamarkupfalse%
\ pisurj\ {\isacharcolon}{\kern0pt}\ {\isachardoublequoteopen}{\isasympi}\ {\isasymin}\ surj{\isacharparenleft}{\kern0pt}P{\isacharcomma}{\kern0pt}P{\isacharparenright}{\kern0pt}{\isachardoublequoteclose}\ \isacommand{using}\isamarkupfalse%
\ bij{\isacharunderscore}{\kern0pt}def\ \isacommand{by}\isamarkupfalse%
\ auto\isanewline
\ \ \isacommand{then}\isamarkupfalse%
\ \isacommand{obtain}\isamarkupfalse%
\ p{\isacharprime}{\kern0pt}\ \isakeyword{where}\ p{\isacharprime}{\kern0pt}h\ {\isacharcolon}{\kern0pt}\ {\isachardoublequoteopen}p{\isacharprime}{\kern0pt}\ {\isasymin}\ P{\isachardoublequoteclose}\ {\isachardoublequoteopen}{\isasympi}{\isacharbackquote}{\kern0pt}p{\isacharprime}{\kern0pt}\ {\isacharequal}{\kern0pt}\ p{\isachardoublequoteclose}\ \isacommand{using}\isamarkupfalse%
\ assms\ \isacommand{unfolding}\isamarkupfalse%
\ surj{\isacharunderscore}{\kern0pt}def\ \isacommand{by}\isamarkupfalse%
\ auto\ \isanewline
\ \ \isacommand{then}\isamarkupfalse%
\ \isacommand{obtain}\isamarkupfalse%
\ q{\isacharprime}{\kern0pt}\ \isakeyword{where}\ q{\isacharprime}{\kern0pt}h\ {\isacharcolon}{\kern0pt}\ {\isachardoublequoteopen}q{\isacharprime}{\kern0pt}\ {\isasymin}\ P{\isachardoublequoteclose}\ {\isachardoublequoteopen}{\isasympi}{\isacharbackquote}{\kern0pt}q{\isacharprime}{\kern0pt}\ {\isacharequal}{\kern0pt}\ q{\isachardoublequoteclose}\ \isacommand{using}\isamarkupfalse%
\ assms\ pisurj\ \isacommand{unfolding}\isamarkupfalse%
\ surj{\isacharunderscore}{\kern0pt}def\ \isacommand{by}\isamarkupfalse%
\ auto\ \isanewline
\ \ \isacommand{have}\isamarkupfalse%
\ H{\isadigit{1}}{\isacharcolon}{\kern0pt}\ {\isachardoublequoteopen}converse{\isacharparenleft}{\kern0pt}{\isasympi}{\isacharparenright}{\kern0pt}\ {\isacharbackquote}{\kern0pt}\ p\ {\isacharequal}{\kern0pt}\ p{\isacharprime}{\kern0pt}{\isachardoublequoteclose}\ \isacommand{using}\isamarkupfalse%
\ converse{\isacharunderscore}{\kern0pt}apply\ assms\ p{\isacharprime}{\kern0pt}h\ \isacommand{by}\isamarkupfalse%
\ auto\isanewline
\ \ \isacommand{have}\isamarkupfalse%
\ H{\isadigit{2}}{\isacharcolon}{\kern0pt}\ {\isachardoublequoteopen}converse{\isacharparenleft}{\kern0pt}{\isasympi}{\isacharparenright}{\kern0pt}\ {\isacharbackquote}{\kern0pt}\ q\ {\isacharequal}{\kern0pt}\ q{\isacharprime}{\kern0pt}{\isachardoublequoteclose}\ \isacommand{using}\isamarkupfalse%
\ converse{\isacharunderscore}{\kern0pt}apply\ assms\ q{\isacharprime}{\kern0pt}h\ \isacommand{by}\isamarkupfalse%
\ auto\isanewline
\ \ \isacommand{show}\isamarkupfalse%
\ {\isachardoublequoteopen}p\ {\isasympreceq}\ q\ {\isasymlongleftrightarrow}\ converse{\isacharparenleft}{\kern0pt}{\isasympi}{\isacharparenright}{\kern0pt}\ {\isacharbackquote}{\kern0pt}\ p\ {\isasympreceq}\ converse{\isacharparenleft}{\kern0pt}{\isasympi}{\isacharparenright}{\kern0pt}\ {\isacharbackquote}{\kern0pt}\ q{\isachardoublequoteclose}\ \isanewline
\ \ \ \ \isacommand{apply}\isamarkupfalse%
\ {\isacharparenleft}{\kern0pt}rule\ iffI{\isacharparenright}{\kern0pt}\ \isanewline
\ \ \isacommand{proof}\isamarkupfalse%
\ {\isacharminus}{\kern0pt}\ \isanewline
\ \ \ \ \isacommand{assume}\isamarkupfalse%
\ assms{\isacharprime}{\kern0pt}\ {\isacharcolon}{\kern0pt}\ {\isachardoublequoteopen}\ p\ {\isasympreceq}\ q{\isachardoublequoteclose}\ \isanewline
\ \ \ \ \isacommand{have}\isamarkupfalse%
\ H{\isadigit{3}}{\isacharcolon}{\kern0pt}\ {\isachardoublequoteopen}p{\isacharprime}{\kern0pt}\ {\isasympreceq}\ q{\isacharprime}{\kern0pt}{\isachardoublequoteclose}\ \isacommand{using}\isamarkupfalse%
\ p{\isacharprime}{\kern0pt}h\ q{\isacharprime}{\kern0pt}h\ assms\ assms{\isacharprime}{\kern0pt}\ \isacommand{by}\isamarkupfalse%
\ auto\isanewline
\ \ \ \ \isacommand{show}\isamarkupfalse%
\ {\isachardoublequoteopen}\ converse{\isacharparenleft}{\kern0pt}{\isasympi}{\isacharparenright}{\kern0pt}\ {\isacharbackquote}{\kern0pt}\ p\ {\isasympreceq}\ converse{\isacharparenleft}{\kern0pt}{\isasympi}{\isacharparenright}{\kern0pt}\ {\isacharbackquote}{\kern0pt}\ q{\isachardoublequoteclose}\ \isacommand{using}\isamarkupfalse%
\ H{\isadigit{1}}\ H{\isadigit{2}}\ H{\isadigit{3}}\ \isacommand{by}\isamarkupfalse%
\ auto\ \isanewline
\ \ \isacommand{next}\isamarkupfalse%
\ \isanewline
\ \ \ \ \isacommand{assume}\isamarkupfalse%
\ assms{\isacharprime}{\kern0pt}\ {\isacharcolon}{\kern0pt}\ {\isachardoublequoteopen}converse{\isacharparenleft}{\kern0pt}{\isasympi}{\isacharparenright}{\kern0pt}\ {\isacharbackquote}{\kern0pt}\ p\ {\isasympreceq}\ converse{\isacharparenleft}{\kern0pt}{\isasympi}{\isacharparenright}{\kern0pt}\ {\isacharbackquote}{\kern0pt}\ q\ {\isachardoublequoteclose}\ \isanewline
\ \ \ \ \isacommand{have}\isamarkupfalse%
\ H{\isadigit{3}}{\isacharcolon}{\kern0pt}\ {\isachardoublequoteopen}p{\isacharprime}{\kern0pt}\ {\isasympreceq}\ q{\isacharprime}{\kern0pt}{\isachardoublequoteclose}\ \isacommand{using}\isamarkupfalse%
\ p{\isacharprime}{\kern0pt}h\ q{\isacharprime}{\kern0pt}h\ assms\ assms{\isacharprime}{\kern0pt}\ H{\isadigit{1}}\ H{\isadigit{2}}\ \isacommand{by}\isamarkupfalse%
\ auto\ \isanewline
\ \ \ \ \isacommand{then}\isamarkupfalse%
\ \isacommand{show}\isamarkupfalse%
\ {\isachardoublequoteopen}p\ {\isasympreceq}\ q{\isachardoublequoteclose}\ \isacommand{using}\isamarkupfalse%
\ assms\ p{\isacharprime}{\kern0pt}h\ q{\isacharprime}{\kern0pt}h\ \isacommand{by}\isamarkupfalse%
\ auto\ \isanewline
\ \ \isacommand{qed}\isamarkupfalse%
\isanewline
\isacommand{qed}\isamarkupfalse%
%
\endisatagproof
{\isafoldproof}%
%
\isadelimproof
\isanewline
%
\endisadelimproof
\isanewline
\isacommand{lemma}\isamarkupfalse%
\ P{\isacharunderscore}{\kern0pt}auto{\isacharunderscore}{\kern0pt}preserves{\isacharunderscore}{\kern0pt}one\ {\isacharcolon}{\kern0pt}\ {\isachardoublequoteopen}is{\isacharunderscore}{\kern0pt}P{\isacharunderscore}{\kern0pt}auto{\isacharparenleft}{\kern0pt}{\isasympi}{\isacharparenright}{\kern0pt}\ {\isasymLongrightarrow}\ {\isasympi}{\isacharbackquote}{\kern0pt}one\ {\isacharequal}{\kern0pt}\ one{\isachardoublequoteclose}\ \isanewline
%
\isadelimproof
%
\endisadelimproof
%
\isatagproof
\isacommand{proof}\isamarkupfalse%
\ {\isacharminus}{\kern0pt}\ \isanewline
\ \ \isacommand{assume}\isamarkupfalse%
\ assm\ {\isacharcolon}{\kern0pt}\ {\isachardoublequoteopen}is{\isacharunderscore}{\kern0pt}P{\isacharunderscore}{\kern0pt}auto{\isacharparenleft}{\kern0pt}{\isasympi}{\isacharparenright}{\kern0pt}{\isachardoublequoteclose}\ \isanewline
\ \ \isacommand{obtain}\isamarkupfalse%
\ p\ \isakeyword{where}\ pp\ {\isacharcolon}{\kern0pt}\ {\isachardoublequoteopen}p\ {\isasymin}\ P{\isachardoublequoteclose}\ {\isachardoublequoteopen}{\isasympi}{\isacharbackquote}{\kern0pt}p{\isacharequal}{\kern0pt}one{\isachardoublequoteclose}\ \isanewline
\ \ \ \ \isacommand{using}\isamarkupfalse%
\ one{\isacharunderscore}{\kern0pt}in{\isacharunderscore}{\kern0pt}P\ surj{\isacharunderscore}{\kern0pt}def\ assm\ \isacommand{unfolding}\isamarkupfalse%
\ is{\isacharunderscore}{\kern0pt}P{\isacharunderscore}{\kern0pt}auto{\isacharunderscore}{\kern0pt}def\ bij{\isacharunderscore}{\kern0pt}def\ \isacommand{by}\isamarkupfalse%
\ auto\ \isanewline
\ \ \isacommand{then}\isamarkupfalse%
\ \isacommand{have}\isamarkupfalse%
\ p{\isadigit{1}}{\isacharcolon}{\kern0pt}\ {\isachardoublequoteopen}p\ {\isasympreceq}\ one{\isachardoublequoteclose}\ \isacommand{using}\isamarkupfalse%
\ one{\isacharunderscore}{\kern0pt}max\ \isacommand{by}\isamarkupfalse%
\ auto\ \isanewline
\ \ \isacommand{then}\isamarkupfalse%
\ \isacommand{have}\isamarkupfalse%
\ {\isachardoublequoteopen}{\isasympi}{\isacharbackquote}{\kern0pt}p\ {\isasympreceq}\ {\isasympi}{\isacharbackquote}{\kern0pt}one{\isachardoublequoteclose}\ \isanewline
\ \ \ \ \isacommand{using}\isamarkupfalse%
\ P{\isacharunderscore}{\kern0pt}auto{\isacharunderscore}{\kern0pt}preserves{\isacharunderscore}{\kern0pt}leq\ pp\ one{\isacharunderscore}{\kern0pt}in{\isacharunderscore}{\kern0pt}P\ p{\isadigit{1}}\ assm\ \isacommand{by}\isamarkupfalse%
\ auto\ \isanewline
\ \ \isacommand{then}\isamarkupfalse%
\ \isacommand{have}\isamarkupfalse%
\ {\isachardoublequoteopen}one\ {\isasympreceq}\ {\isasympi}{\isacharbackquote}{\kern0pt}one{\isachardoublequoteclose}\ \isacommand{using}\isamarkupfalse%
\ pp\ \isacommand{by}\isamarkupfalse%
\ auto\ \isanewline
\ \ \isacommand{then}\isamarkupfalse%
\ \isacommand{show}\isamarkupfalse%
\ {\isachardoublequoteopen}{\isasympi}{\isacharbackquote}{\kern0pt}one\ {\isacharequal}{\kern0pt}\ one{\isachardoublequoteclose}\ \isacommand{using}\isamarkupfalse%
\ assm\ P{\isacharunderscore}{\kern0pt}auto{\isacharunderscore}{\kern0pt}value\ one{\isacharunderscore}{\kern0pt}in{\isacharunderscore}{\kern0pt}P\ one{\isacharunderscore}{\kern0pt}is{\isacharunderscore}{\kern0pt}unique{\isacharunderscore}{\kern0pt}max\ \isacommand{by}\isamarkupfalse%
\ auto\isanewline
\isacommand{qed}\isamarkupfalse%
%
\endisatagproof
{\isafoldproof}%
%
\isadelimproof
\isanewline
%
\endisadelimproof
\isanewline
\isacommand{lemma}\isamarkupfalse%
\ P{\isacharunderscore}{\kern0pt}auto{\isacharunderscore}{\kern0pt}preserves{\isacharunderscore}{\kern0pt}density\ {\isacharcolon}{\kern0pt}\ \isanewline
\ \ {\isachardoublequoteopen}is{\isacharunderscore}{\kern0pt}P{\isacharunderscore}{\kern0pt}auto{\isacharparenleft}{\kern0pt}{\isasympi}{\isacharparenright}{\kern0pt}\ {\isasymLongrightarrow}\ D\ {\isasymsubseteq}\ P\ {\isasymLongrightarrow}\ q\ {\isasymin}\ P\ {\isasymLongrightarrow}\ dense{\isacharunderscore}{\kern0pt}below{\isacharparenleft}{\kern0pt}D{\isacharcomma}{\kern0pt}\ q{\isacharparenright}{\kern0pt}\ {\isasymLongrightarrow}\ dense{\isacharunderscore}{\kern0pt}below{\isacharparenleft}{\kern0pt}{\isasympi}{\isacharbackquote}{\kern0pt}{\isacharbackquote}{\kern0pt}D{\isacharcomma}{\kern0pt}\ {\isasympi}{\isacharbackquote}{\kern0pt}q{\isacharparenright}{\kern0pt}{\isachardoublequoteclose}\ \isanewline
%
\isadelimproof
\ \ %
\endisadelimproof
%
\isatagproof
\isacommand{unfolding}\isamarkupfalse%
\ dense{\isacharunderscore}{\kern0pt}below{\isacharunderscore}{\kern0pt}def\ \isacommand{apply}\isamarkupfalse%
\ auto\ \isanewline
\isacommand{proof}\isamarkupfalse%
\ {\isacharminus}{\kern0pt}\ \isanewline
\ \ \isacommand{fix}\isamarkupfalse%
\ p\isanewline
\ \ \isacommand{assume}\isamarkupfalse%
\ assms\ {\isacharcolon}{\kern0pt}\ {\isachardoublequoteopen}is{\isacharunderscore}{\kern0pt}P{\isacharunderscore}{\kern0pt}auto{\isacharparenleft}{\kern0pt}{\isasympi}{\isacharparenright}{\kern0pt}{\isachardoublequoteclose}\ {\isachardoublequoteopen}{\isasymforall}p{\isasymin}P{\isachardot}{\kern0pt}\ p\ {\isasympreceq}\ q\ {\isasymlongrightarrow}\ {\isacharparenleft}{\kern0pt}{\isasymexists}d{\isasymin}D{\isachardot}{\kern0pt}\ d\ {\isasymin}\ P\ {\isasymand}\ d\ {\isasympreceq}\ p{\isacharparenright}{\kern0pt}{\isachardoublequoteclose}\ \isanewline
\ \ \ \ \ \ \ \ \ \ \ \ \ \ \ \ \ {\isachardoublequoteopen}p\ {\isasymin}\ P{\isachardoublequoteclose}\ {\isachardoublequoteopen}p\ {\isasympreceq}\ {\isasympi}{\isacharbackquote}{\kern0pt}q{\isachardoublequoteclose}\ {\isachardoublequoteopen}q\ {\isasymin}\ P{\isachardoublequoteclose}\ {\isachardoublequoteopen}D\ {\isasymsubseteq}\ P{\isachardoublequoteclose}\ \isanewline
\ \ \isacommand{have}\isamarkupfalse%
\ {\isachardoublequoteopen}{\isasympi}\ {\isasymin}\ surj{\isacharparenleft}{\kern0pt}P{\isacharcomma}{\kern0pt}\ P{\isacharparenright}{\kern0pt}{\isachardoublequoteclose}\ \isacommand{using}\isamarkupfalse%
\ assms\ \isacommand{unfolding}\isamarkupfalse%
\ is{\isacharunderscore}{\kern0pt}P{\isacharunderscore}{\kern0pt}auto{\isacharunderscore}{\kern0pt}def\ bij{\isacharunderscore}{\kern0pt}def\ \isacommand{by}\isamarkupfalse%
\ auto\ \isanewline
\ \ \isacommand{then}\isamarkupfalse%
\ \isacommand{obtain}\isamarkupfalse%
\ p{\isacharprime}{\kern0pt}\ \isakeyword{where}\ p{\isacharprime}{\kern0pt}h{\isacharcolon}{\kern0pt}\ {\isachardoublequoteopen}p{\isacharprime}{\kern0pt}\ {\isasymin}\ P{\isachardoublequoteclose}\ {\isachardoublequoteopen}{\isasympi}{\isacharbackquote}{\kern0pt}p{\isacharprime}{\kern0pt}\ {\isacharequal}{\kern0pt}\ p{\isachardoublequoteclose}\ \isacommand{using}\isamarkupfalse%
\ assms\ \isacommand{unfolding}\isamarkupfalse%
\ surj{\isacharunderscore}{\kern0pt}def\ \isacommand{by}\isamarkupfalse%
\ auto\ \isanewline
\ \ \isacommand{then}\isamarkupfalse%
\ \isacommand{have}\isamarkupfalse%
\ {\isachardoublequoteopen}p{\isacharprime}{\kern0pt}\ {\isasympreceq}\ q{\isachardoublequoteclose}\ \isacommand{apply}\isamarkupfalse%
\ {\isacharparenleft}{\kern0pt}rule{\isacharunderscore}{\kern0pt}tac\ {\isasympi}{\isacharequal}{\kern0pt}{\isasympi}\ \isakeyword{in}\ P{\isacharunderscore}{\kern0pt}auto{\isacharunderscore}{\kern0pt}preserves{\isacharunderscore}{\kern0pt}leq{\isacharprime}{\kern0pt}{\isacharparenright}{\kern0pt}\ \isacommand{using}\isamarkupfalse%
\ assms\ \isacommand{apply}\isamarkupfalse%
\ auto\ \isacommand{done}\isamarkupfalse%
\isanewline
\ \ \isacommand{then}\isamarkupfalse%
\ \isacommand{obtain}\isamarkupfalse%
\ d\ \isakeyword{where}\ dh{\isacharcolon}{\kern0pt}\ {\isachardoublequoteopen}d\ {\isasymin}\ D{\isachardoublequoteclose}\ {\isachardoublequoteopen}d\ {\isasympreceq}\ p{\isacharprime}{\kern0pt}{\isachardoublequoteclose}\ \isacommand{using}\isamarkupfalse%
\ p{\isacharprime}{\kern0pt}h\ assms\ \isacommand{by}\isamarkupfalse%
\ auto\ \isanewline
\ \ \isacommand{then}\isamarkupfalse%
\ \isacommand{have}\isamarkupfalse%
\ H{\isacharcolon}{\kern0pt}\ {\isachardoublequoteopen}{\isasympi}{\isacharbackquote}{\kern0pt}d\ {\isasymin}\ {\isasympi}{\isacharbackquote}{\kern0pt}{\isacharbackquote}{\kern0pt}D{\isachardoublequoteclose}\ \isacommand{apply}\isamarkupfalse%
\ {\isacharparenleft}{\kern0pt}rule{\isacharunderscore}{\kern0pt}tac\ a{\isacharequal}{\kern0pt}d\ \isakeyword{in}\ imageI{\isacharparenright}{\kern0pt}\ \isanewline
\ \ \ \ \isacommand{apply}\isamarkupfalse%
\ {\isacharparenleft}{\kern0pt}rule{\isacharunderscore}{\kern0pt}tac\ function{\isacharunderscore}{\kern0pt}apply{\isacharunderscore}{\kern0pt}Pair{\isacharparenright}{\kern0pt}\ \isanewline
\ \ \ \ \isacommand{using}\isamarkupfalse%
\ P{\isacharunderscore}{\kern0pt}auto{\isacharunderscore}{\kern0pt}is{\isacharunderscore}{\kern0pt}function\ assms\ P{\isacharunderscore}{\kern0pt}auto{\isacharunderscore}{\kern0pt}domain\ \isacommand{apply}\isamarkupfalse%
\ auto\ \isacommand{done}\isamarkupfalse%
\isanewline
\ \ \isacommand{have}\isamarkupfalse%
\ H{\isadigit{2}}\ {\isacharcolon}{\kern0pt}\ {\isachardoublequoteopen}{\isasympi}{\isacharbackquote}{\kern0pt}d\ {\isasympreceq}\ p{\isachardoublequoteclose}\ \isacommand{using}\isamarkupfalse%
\ P{\isacharunderscore}{\kern0pt}auto{\isacharunderscore}{\kern0pt}preserves{\isacharunderscore}{\kern0pt}leq\ dh\ p{\isacharprime}{\kern0pt}h\ assms\ \isacommand{apply}\isamarkupfalse%
\ auto\ \isacommand{done}\isamarkupfalse%
\ \isanewline
\ \ \isacommand{have}\isamarkupfalse%
\ {\isachardoublequoteopen}{\isasympi}{\isacharbackquote}{\kern0pt}d\ {\isasymin}\ P{\isachardoublequoteclose}\ \isacommand{using}\isamarkupfalse%
\ P{\isacharunderscore}{\kern0pt}auto{\isacharunderscore}{\kern0pt}value\ assms\ dh\ \isacommand{by}\isamarkupfalse%
\ auto\ \isanewline
\ \ \isacommand{then}\isamarkupfalse%
\ \isacommand{show}\isamarkupfalse%
\ {\isachardoublequoteopen}{\isasymexists}d{\isasymin}{\isasympi}\ {\isacharbackquote}{\kern0pt}{\isacharbackquote}{\kern0pt}\ D{\isachardot}{\kern0pt}\ d\ {\isasymin}\ P\ {\isasymand}\ d\ {\isasympreceq}\ p{\isachardoublequoteclose}\ \isacommand{using}\isamarkupfalse%
\ H\ H{\isadigit{2}}\ \isacommand{by}\isamarkupfalse%
\ auto\ \isanewline
\isacommand{qed}\isamarkupfalse%
%
\endisatagproof
{\isafoldproof}%
%
\isadelimproof
\isanewline
%
\endisadelimproof
\isanewline
\isacommand{lemma}\isamarkupfalse%
\ P{\isacharunderscore}{\kern0pt}auto{\isacharunderscore}{\kern0pt}preserves{\isacharunderscore}{\kern0pt}density{\isacharprime}{\kern0pt}\ {\isacharcolon}{\kern0pt}\ \isanewline
\ \ {\isachardoublequoteopen}is{\isacharunderscore}{\kern0pt}P{\isacharunderscore}{\kern0pt}auto{\isacharparenleft}{\kern0pt}{\isasympi}{\isacharparenright}{\kern0pt}\ {\isasymLongrightarrow}\ D\ {\isasymsubseteq}\ P\ {\isasymLongrightarrow}\ q\ {\isasymin}\ P\ {\isasymLongrightarrow}\ dense{\isacharunderscore}{\kern0pt}below{\isacharparenleft}{\kern0pt}D{\isacharcomma}{\kern0pt}\ q{\isacharparenright}{\kern0pt}\ {\isasymlongleftrightarrow}\ dense{\isacharunderscore}{\kern0pt}below{\isacharparenleft}{\kern0pt}{\isasympi}{\isacharbackquote}{\kern0pt}{\isacharbackquote}{\kern0pt}D{\isacharcomma}{\kern0pt}\ {\isasympi}{\isacharbackquote}{\kern0pt}q{\isacharparenright}{\kern0pt}{\isachardoublequoteclose}\ \isanewline
%
\isadelimproof
\ \ %
\endisadelimproof
%
\isatagproof
\isacommand{apply}\isamarkupfalse%
\ {\isacharparenleft}{\kern0pt}rule\ iffI{\isacharparenright}{\kern0pt}\ \isacommand{using}\isamarkupfalse%
\ P{\isacharunderscore}{\kern0pt}auto{\isacharunderscore}{\kern0pt}preserves{\isacharunderscore}{\kern0pt}density\ \isacommand{apply}\isamarkupfalse%
\ simp\ \isanewline
\ \ \isacommand{apply}\isamarkupfalse%
\ {\isacharparenleft}{\kern0pt}rule{\isacharunderscore}{\kern0pt}tac\ P{\isacharequal}{\kern0pt}{\isachardoublequoteopen}dense{\isacharunderscore}{\kern0pt}below{\isacharparenleft}{\kern0pt}converse{\isacharparenleft}{\kern0pt}{\isasympi}{\isacharparenright}{\kern0pt}\ {\isacharbackquote}{\kern0pt}{\isacharbackquote}{\kern0pt}\ {\isacharparenleft}{\kern0pt}{\isasympi}\ {\isacharbackquote}{\kern0pt}{\isacharbackquote}{\kern0pt}\ D{\isacharparenright}{\kern0pt}{\isacharcomma}{\kern0pt}\ converse{\isacharparenleft}{\kern0pt}{\isasympi}{\isacharparenright}{\kern0pt}\ {\isacharbackquote}{\kern0pt}\ {\isacharparenleft}{\kern0pt}{\isasympi}\ {\isacharbackquote}{\kern0pt}\ q{\isacharparenright}{\kern0pt}{\isacharparenright}{\kern0pt}{\isachardoublequoteclose}\ \isakeyword{in}\ mp{\isacharparenright}{\kern0pt}\isanewline
\ \ \isacommand{apply}\isamarkupfalse%
\ {\isacharparenleft}{\kern0pt}rule{\isacharunderscore}{\kern0pt}tac\ a{\isacharequal}{\kern0pt}D\ \isakeyword{and}\ b{\isacharequal}{\kern0pt}{\isachardoublequoteopen}converse{\isacharparenleft}{\kern0pt}{\isasympi}{\isacharparenright}{\kern0pt}\ {\isacharbackquote}{\kern0pt}{\isacharbackquote}{\kern0pt}\ {\isacharparenleft}{\kern0pt}{\isasympi}\ {\isacharbackquote}{\kern0pt}{\isacharbackquote}{\kern0pt}\ D{\isacharparenright}{\kern0pt}{\isachardoublequoteclose}\ \isakeyword{in}\ ssubst{\isacharparenright}{\kern0pt}\isanewline
\ \ \isacommand{apply}\isamarkupfalse%
\ {\isacharparenleft}{\kern0pt}rule{\isacharunderscore}{\kern0pt}tac\ A{\isacharequal}{\kern0pt}P\ \isakeyword{and}\ B{\isacharequal}{\kern0pt}P\ \isakeyword{in}\ image{\isacharunderscore}{\kern0pt}converse{\isacharunderscore}{\kern0pt}image{\isacharparenright}{\kern0pt}\ \isacommand{apply}\isamarkupfalse%
\ {\isacharparenleft}{\kern0pt}simp\ add\ {\isacharcolon}{\kern0pt}\ is{\isacharunderscore}{\kern0pt}P{\isacharunderscore}{\kern0pt}auto{\isacharunderscore}{\kern0pt}def{\isacharparenright}{\kern0pt}\ \isanewline
\ \ \isacommand{apply}\isamarkupfalse%
\ simp\ \isanewline
\ \ \isacommand{apply}\isamarkupfalse%
\ {\isacharparenleft}{\kern0pt}rule{\isacharunderscore}{\kern0pt}tac\ a{\isacharequal}{\kern0pt}q\ \isakeyword{and}\ b{\isacharequal}{\kern0pt}{\isachardoublequoteopen}converse{\isacharparenleft}{\kern0pt}{\isasympi}{\isacharparenright}{\kern0pt}\ {\isacharbackquote}{\kern0pt}\ {\isacharparenleft}{\kern0pt}{\isasympi}\ {\isacharbackquote}{\kern0pt}\ q{\isacharparenright}{\kern0pt}{\isachardoublequoteclose}\ \isakeyword{in}\ ssubst{\isacharparenright}{\kern0pt}\isanewline
\ \ \isacommand{apply}\isamarkupfalse%
\ {\isacharparenleft}{\kern0pt}rule{\isacharunderscore}{\kern0pt}tac\ A{\isacharequal}{\kern0pt}P\ \isakeyword{and}\ B{\isacharequal}{\kern0pt}P\ \isakeyword{in}\ converse{\isacharunderscore}{\kern0pt}apply{\isacharparenright}{\kern0pt}\ \ \ \isacommand{apply}\isamarkupfalse%
\ {\isacharparenleft}{\kern0pt}simp\ add\ {\isacharcolon}{\kern0pt}\ is{\isacharunderscore}{\kern0pt}P{\isacharunderscore}{\kern0pt}auto{\isacharunderscore}{\kern0pt}def{\isacharparenright}{\kern0pt}\ \isanewline
\ \ \isacommand{apply}\isamarkupfalse%
\ {\isacharparenleft}{\kern0pt}simp{\isacharunderscore}{\kern0pt}all{\isacharparenright}{\kern0pt}\isanewline
\isacommand{proof}\isamarkupfalse%
\ {\isacharminus}{\kern0pt}\ \isanewline
\ \ \isacommand{assume}\isamarkupfalse%
\ assms\ {\isacharcolon}{\kern0pt}\ \ {\isachardoublequoteopen}is{\isacharunderscore}{\kern0pt}P{\isacharunderscore}{\kern0pt}auto{\isacharparenleft}{\kern0pt}{\isasympi}{\isacharparenright}{\kern0pt}{\isachardoublequoteclose}\ {\isachardoublequoteopen}\ D\ {\isasymsubseteq}\ P{\isachardoublequoteclose}\ \ {\isachardoublequoteopen}q\ {\isasymin}\ P{\isachardoublequoteclose}\ {\isachardoublequoteopen}dense{\isacharunderscore}{\kern0pt}below{\isacharparenleft}{\kern0pt}{\isasympi}\ {\isacharbackquote}{\kern0pt}{\isacharbackquote}{\kern0pt}\ D{\isacharcomma}{\kern0pt}\ {\isasympi}\ {\isacharbackquote}{\kern0pt}\ q{\isacharparenright}{\kern0pt}\ {\isachardoublequoteclose}\isanewline
\ \ \isacommand{show}\isamarkupfalse%
\ {\isachardoublequoteopen}dense{\isacharunderscore}{\kern0pt}below{\isacharparenleft}{\kern0pt}converse{\isacharparenleft}{\kern0pt}{\isasympi}{\isacharparenright}{\kern0pt}\ {\isacharbackquote}{\kern0pt}{\isacharbackquote}{\kern0pt}\ {\isacharparenleft}{\kern0pt}{\isasympi}\ {\isacharbackquote}{\kern0pt}{\isacharbackquote}{\kern0pt}\ D{\isacharparenright}{\kern0pt}{\isacharcomma}{\kern0pt}\ converse{\isacharparenleft}{\kern0pt}{\isasympi}{\isacharparenright}{\kern0pt}\ {\isacharbackquote}{\kern0pt}\ {\isacharparenleft}{\kern0pt}{\isasympi}\ {\isacharbackquote}{\kern0pt}\ q{\isacharparenright}{\kern0pt}{\isacharparenright}{\kern0pt}{\isachardoublequoteclose}\ \isanewline
\ \ \ \ \isacommand{apply}\isamarkupfalse%
\ {\isacharparenleft}{\kern0pt}rule{\isacharunderscore}{\kern0pt}tac\ P{\isacharunderscore}{\kern0pt}auto{\isacharunderscore}{\kern0pt}preserves{\isacharunderscore}{\kern0pt}density{\isacharparenright}{\kern0pt}\ \isacommand{using}\isamarkupfalse%
\ assms\ P{\isacharunderscore}{\kern0pt}auto{\isacharunderscore}{\kern0pt}converse\ \isacommand{apply}\isamarkupfalse%
\ simp\isanewline
\ \ \ \ \isacommand{apply}\isamarkupfalse%
\ {\isacharparenleft}{\kern0pt}rule{\isacharunderscore}{\kern0pt}tac\ A{\isacharequal}{\kern0pt}P\ \isakeyword{in}\ image{\isacharunderscore}{\kern0pt}subset{\isacharparenright}{\kern0pt}\ \isacommand{using}\isamarkupfalse%
\ assms\ bij{\isacharunderscore}{\kern0pt}is{\isacharunderscore}{\kern0pt}fun\ \isacommand{unfolding}\isamarkupfalse%
\ is{\isacharunderscore}{\kern0pt}P{\isacharunderscore}{\kern0pt}auto{\isacharunderscore}{\kern0pt}def\ Pi{\isacharunderscore}{\kern0pt}def\ \isacommand{apply}\isamarkupfalse%
\ auto\ \isanewline
\ \ \ \ \isacommand{using}\isamarkupfalse%
\ assms\ local{\isachardot}{\kern0pt}P{\isacharunderscore}{\kern0pt}auto{\isacharunderscore}{\kern0pt}value\ \isacommand{by}\isamarkupfalse%
\ auto\ \isanewline
\isacommand{qed}\isamarkupfalse%
%
\endisatagproof
{\isafoldproof}%
%
\isadelimproof
\isanewline
%
\endisadelimproof
\isanewline
\isacommand{lemma}\isamarkupfalse%
\ P{\isacharunderscore}{\kern0pt}auto{\isacharunderscore}{\kern0pt}comp\ {\isacharcolon}{\kern0pt}\ {\isachardoublequoteopen}is{\isacharunderscore}{\kern0pt}P{\isacharunderscore}{\kern0pt}auto{\isacharparenleft}{\kern0pt}{\isasympi}{\isacharparenright}{\kern0pt}\ {\isasymLongrightarrow}\ is{\isacharunderscore}{\kern0pt}P{\isacharunderscore}{\kern0pt}auto{\isacharparenleft}{\kern0pt}{\isasymtau}{\isacharparenright}{\kern0pt}\ {\isasymLongrightarrow}\ is{\isacharunderscore}{\kern0pt}P{\isacharunderscore}{\kern0pt}auto{\isacharparenleft}{\kern0pt}{\isasympi}\ O\ {\isasymtau}{\isacharparenright}{\kern0pt}{\isachardoublequoteclose}\ \isanewline
%
\isadelimproof
\ \ %
\endisadelimproof
%
\isatagproof
\isacommand{unfolding}\isamarkupfalse%
\ is{\isacharunderscore}{\kern0pt}P{\isacharunderscore}{\kern0pt}auto{\isacharunderscore}{\kern0pt}def\ \isanewline
\ \ \isacommand{apply}\isamarkupfalse%
{\isacharparenleft}{\kern0pt}rule\ conjI{\isacharcomma}{\kern0pt}\ rule\ to{\isacharunderscore}{\kern0pt}rin{\isacharcomma}{\kern0pt}\ rule\ comp{\isacharunderscore}{\kern0pt}closed{\isacharcomma}{\kern0pt}\ simp{\isacharcomma}{\kern0pt}\ simp{\isacharparenright}{\kern0pt}\isanewline
\ \ \isacommand{apply}\isamarkupfalse%
{\isacharparenleft}{\kern0pt}rule\ conjI{\isacharcomma}{\kern0pt}\ rule\ comp{\isacharunderscore}{\kern0pt}bij{\isacharcomma}{\kern0pt}\ force{\isacharcomma}{\kern0pt}\ force{\isacharparenright}{\kern0pt}\isanewline
\ \ \isacommand{apply}\isamarkupfalse%
\ clarify\ \isanewline
\ \ \isacommand{apply}\isamarkupfalse%
{\isacharparenleft}{\kern0pt}subst\ comp{\isacharunderscore}{\kern0pt}fun{\isacharunderscore}{\kern0pt}apply{\isacharcomma}{\kern0pt}\ rule\ bij{\isacharunderscore}{\kern0pt}is{\isacharunderscore}{\kern0pt}fun{\isacharcomma}{\kern0pt}\ simp{\isacharcomma}{\kern0pt}\ simp{\isacharparenright}{\kern0pt}\isanewline
\ \ \isacommand{apply}\isamarkupfalse%
{\isacharparenleft}{\kern0pt}subst\ comp{\isacharunderscore}{\kern0pt}fun{\isacharunderscore}{\kern0pt}apply{\isacharcomma}{\kern0pt}\ rule\ bij{\isacharunderscore}{\kern0pt}is{\isacharunderscore}{\kern0pt}fun{\isacharcomma}{\kern0pt}\ simp{\isacharcomma}{\kern0pt}\ simp{\isacharparenright}{\kern0pt}\isanewline
\ \ \isacommand{apply}\isamarkupfalse%
{\isacharparenleft}{\kern0pt}rename{\isacharunderscore}{\kern0pt}tac\ p\ q{\isacharcomma}{\kern0pt}\ rule{\isacharunderscore}{\kern0pt}tac\ Q{\isacharequal}{\kern0pt}{\isachardoublequoteopen}{\isacharparenleft}{\kern0pt}{\isasymtau}{\isacharbackquote}{\kern0pt}p\ {\isasympreceq}\ {\isasymtau}{\isacharbackquote}{\kern0pt}q{\isacharparenright}{\kern0pt}{\isachardoublequoteclose}\ \isakeyword{in}\ iff{\isacharunderscore}{\kern0pt}trans{\isacharparenright}{\kern0pt}\ \isanewline
\ \ \ \isacommand{apply}\isamarkupfalse%
\ simp\isanewline
\ \ \isacommand{apply}\isamarkupfalse%
{\isacharparenleft}{\kern0pt}rename{\isacharunderscore}{\kern0pt}tac\ p\ q{\isacharcomma}{\kern0pt}\ subgoal{\isacharunderscore}{\kern0pt}tac\ {\isachardoublequoteopen}{\isasymtau}{\isacharbackquote}{\kern0pt}p\ {\isasymin}\ P\ {\isasymand}\ {\isasymtau}{\isacharbackquote}{\kern0pt}q\ {\isasymin}\ P{\isachardoublequoteclose}{\isacharparenright}{\kern0pt}\isanewline
\ \ \ \isacommand{apply}\isamarkupfalse%
\ force\isanewline
\ \ \isacommand{apply}\isamarkupfalse%
{\isacharparenleft}{\kern0pt}rule\ conjI{\isacharcomma}{\kern0pt}\ rule\ P{\isacharunderscore}{\kern0pt}auto{\isacharunderscore}{\kern0pt}value{\isacharcomma}{\kern0pt}\ simp\ add{\isacharcolon}{\kern0pt}is{\isacharunderscore}{\kern0pt}P{\isacharunderscore}{\kern0pt}auto{\isacharunderscore}{\kern0pt}def{\isacharcomma}{\kern0pt}\ simp{\isacharparenright}{\kern0pt}\isanewline
\ \ \isacommand{apply}\isamarkupfalse%
{\isacharparenleft}{\kern0pt}rule\ P{\isacharunderscore}{\kern0pt}auto{\isacharunderscore}{\kern0pt}value{\isacharcomma}{\kern0pt}\ simp\ add{\isacharcolon}{\kern0pt}is{\isacharunderscore}{\kern0pt}P{\isacharunderscore}{\kern0pt}auto{\isacharunderscore}{\kern0pt}def{\isacharcomma}{\kern0pt}\ simp{\isacharparenright}{\kern0pt}\isanewline
\ \ \isacommand{done}\isamarkupfalse%
%
\endisatagproof
{\isafoldproof}%
%
\isadelimproof
\isanewline
%
\endisadelimproof
\isanewline
\isacommand{definition}\isamarkupfalse%
\ HPn{\isacharunderscore}{\kern0pt}auto\ {\isacharcolon}{\kern0pt}{\isacharcolon}{\kern0pt}\ {\isachardoublequoteopen}{\isacharbrackleft}{\kern0pt}i{\isacharcomma}{\kern0pt}\ i{\isacharcomma}{\kern0pt}\ i{\isacharbrackright}{\kern0pt}\ {\isasymRightarrow}\ i{\isachardoublequoteclose}\ \isakeyword{where}\ \isanewline
\ \ {\isachardoublequoteopen}HPn{\isacharunderscore}{\kern0pt}auto{\isacharparenleft}{\kern0pt}{\isasympi}{\isacharcomma}{\kern0pt}\ x{\isacharcomma}{\kern0pt}\ H{\isacharparenright}{\kern0pt}\ {\isasymequiv}\ {\isacharbraceleft}{\kern0pt}\ {\isacharless}{\kern0pt}H{\isacharbackquote}{\kern0pt}fst{\isacharparenleft}{\kern0pt}v{\isacharparenright}{\kern0pt}{\isacharcomma}{\kern0pt}\ {\isasympi}{\isacharbackquote}{\kern0pt}snd{\isacharparenleft}{\kern0pt}v{\isacharparenright}{\kern0pt}{\isachargreater}{\kern0pt}\ {\isachardot}{\kern0pt}{\isachardot}{\kern0pt}\ v\ {\isasymin}\ x{\isacharcomma}{\kern0pt}\ {\isasymexists}y\ p{\isachardot}{\kern0pt}\ v\ {\isacharequal}{\kern0pt}\ {\isacharless}{\kern0pt}y{\isacharcomma}{\kern0pt}\ p{\isachargreater}{\kern0pt}\ \ {\isacharbraceright}{\kern0pt}{\isachardoublequoteclose}\ \isanewline
\isanewline
\isacommand{definition}\isamarkupfalse%
\ Pn{\isacharunderscore}{\kern0pt}auto\ {\isacharcolon}{\kern0pt}{\isacharcolon}{\kern0pt}\ {\isachardoublequoteopen}i\ {\isasymRightarrow}\ i{\isachardoublequoteclose}\ \isakeyword{where}\ \isanewline
\ \ {\isachardoublequoteopen}Pn{\isacharunderscore}{\kern0pt}auto{\isacharparenleft}{\kern0pt}{\isasympi}{\isacharparenright}{\kern0pt}\ {\isasymequiv}\ \isanewline
\ \ \ \ {\isacharbraceleft}{\kern0pt}\ {\isacharless}{\kern0pt}x{\isacharcomma}{\kern0pt}\ wftrec{\isacharparenleft}{\kern0pt}Memrel{\isacharparenleft}{\kern0pt}M{\isacharparenright}{\kern0pt}{\isacharcircum}{\kern0pt}{\isacharplus}{\kern0pt}{\isacharcomma}{\kern0pt}\ x{\isacharcomma}{\kern0pt}\ HPn{\isacharunderscore}{\kern0pt}auto{\isacharparenleft}{\kern0pt}{\isasympi}{\isacharparenright}{\kern0pt}{\isacharparenright}{\kern0pt}{\isachargreater}{\kern0pt}\ {\isachardot}{\kern0pt}\ x\ {\isasymin}\ P{\isacharunderscore}{\kern0pt}names\ {\isacharbraceright}{\kern0pt}{\isachardoublequoteclose}\ \isanewline
\isanewline
\isacommand{lemma}\isamarkupfalse%
\ Pn{\isacharunderscore}{\kern0pt}auto{\isacharunderscore}{\kern0pt}function\ {\isacharcolon}{\kern0pt}\ {\isachardoublequoteopen}function{\isacharparenleft}{\kern0pt}Pn{\isacharunderscore}{\kern0pt}auto{\isacharparenleft}{\kern0pt}{\isasympi}{\isacharparenright}{\kern0pt}{\isacharparenright}{\kern0pt}{\isachardoublequoteclose}\ \isanewline
%
\isadelimproof
\ \ %
\endisadelimproof
%
\isatagproof
\isacommand{unfolding}\isamarkupfalse%
\ Pn{\isacharunderscore}{\kern0pt}auto{\isacharunderscore}{\kern0pt}def\ function{\isacharunderscore}{\kern0pt}def\ \isacommand{by}\isamarkupfalse%
\ auto%
\endisatagproof
{\isafoldproof}%
%
\isadelimproof
\isanewline
%
\endisadelimproof
\isacommand{lemma}\isamarkupfalse%
\ Pn{\isacharunderscore}{\kern0pt}auto{\isacharunderscore}{\kern0pt}domain\ {\isacharcolon}{\kern0pt}\ {\isachardoublequoteopen}domain{\isacharparenleft}{\kern0pt}Pn{\isacharunderscore}{\kern0pt}auto{\isacharparenleft}{\kern0pt}{\isasympi}{\isacharparenright}{\kern0pt}{\isacharparenright}{\kern0pt}\ {\isacharequal}{\kern0pt}\ P{\isacharunderscore}{\kern0pt}names{\isachardoublequoteclose}\isanewline
%
\isadelimproof
\ \ %
\endisadelimproof
%
\isatagproof
\isacommand{unfolding}\isamarkupfalse%
\ Pn{\isacharunderscore}{\kern0pt}auto{\isacharunderscore}{\kern0pt}def\ domain{\isacharunderscore}{\kern0pt}def\ \isacommand{by}\isamarkupfalse%
\ auto%
\endisatagproof
{\isafoldproof}%
%
\isadelimproof
\ \isanewline
%
\endisadelimproof
\isanewline
\isacommand{lemma}\isamarkupfalse%
\ Pn{\isacharunderscore}{\kern0pt}auto\ {\isacharcolon}{\kern0pt}\ {\isachardoublequoteopen}x\ {\isasymin}\ P{\isacharunderscore}{\kern0pt}names\ {\isasymLongrightarrow}\ Pn{\isacharunderscore}{\kern0pt}auto{\isacharparenleft}{\kern0pt}{\isasympi}{\isacharparenright}{\kern0pt}{\isacharbackquote}{\kern0pt}x\ {\isacharequal}{\kern0pt}\ {\isacharbraceleft}{\kern0pt}\ {\isacharless}{\kern0pt}Pn{\isacharunderscore}{\kern0pt}auto{\isacharparenleft}{\kern0pt}{\isasympi}{\isacharparenright}{\kern0pt}{\isacharbackquote}{\kern0pt}y{\isacharcomma}{\kern0pt}\ {\isasympi}{\isacharbackquote}{\kern0pt}p{\isachargreater}{\kern0pt}\ {\isachardot}{\kern0pt}\ {\isacharless}{\kern0pt}y{\isacharcomma}{\kern0pt}\ p{\isachargreater}{\kern0pt}\ {\isasymin}\ x{\isacharbraceright}{\kern0pt}{\isachardoublequoteclose}\ \ \isanewline
%
\isadelimproof
%
\endisadelimproof
%
\isatagproof
\isacommand{proof}\isamarkupfalse%
\ {\isacharminus}{\kern0pt}\isanewline
\ \ \isacommand{define}\isamarkupfalse%
\ F\ \isakeyword{where}\ {\isachardoublequoteopen}F\ {\isasymequiv}\ {\isasymlambda}y\ {\isasymin}\ Memrel{\isacharparenleft}{\kern0pt}M{\isacharparenright}{\kern0pt}{\isacharcircum}{\kern0pt}{\isacharplus}{\kern0pt}\ {\isacharminus}{\kern0pt}{\isacharbackquote}{\kern0pt}{\isacharbackquote}{\kern0pt}{\isacharbraceleft}{\kern0pt}x{\isacharbraceright}{\kern0pt}{\isachardot}{\kern0pt}\ wftrec{\isacharparenleft}{\kern0pt}Memrel{\isacharparenleft}{\kern0pt}M{\isacharparenright}{\kern0pt}{\isacharcircum}{\kern0pt}{\isacharplus}{\kern0pt}{\isacharcomma}{\kern0pt}\ y{\isacharcomma}{\kern0pt}\ HPn{\isacharunderscore}{\kern0pt}auto{\isacharparenleft}{\kern0pt}{\isasympi}{\isacharparenright}{\kern0pt}{\isacharparenright}{\kern0pt}{\isachardoublequoteclose}\ \isanewline
\isanewline
\ \ \isacommand{assume}\isamarkupfalse%
\ assm\ {\isacharcolon}{\kern0pt}\ {\isachardoublequoteopen}x\ {\isasymin}\ P{\isacharunderscore}{\kern0pt}names{\isachardoublequoteclose}\isanewline
\ \ \isacommand{then}\isamarkupfalse%
\ \isacommand{have}\isamarkupfalse%
\ xin{\isacharcolon}{\kern0pt}\ {\isachardoublequoteopen}x\ {\isasymsubseteq}\ P{\isacharunderscore}{\kern0pt}set{\isacharparenleft}{\kern0pt}P{\isacharunderscore}{\kern0pt}rank{\isacharparenleft}{\kern0pt}x{\isacharparenright}{\kern0pt}{\isacharparenright}{\kern0pt}\ {\isasymtimes}\ P{\isachardoublequoteclose}\ \isacommand{using}\isamarkupfalse%
\ P{\isacharunderscore}{\kern0pt}names{\isacharunderscore}{\kern0pt}in\ \isacommand{by}\isamarkupfalse%
\ auto\ \isanewline
\ \ \isacommand{then}\isamarkupfalse%
\ \isacommand{have}\isamarkupfalse%
\ E{\isadigit{1}}{\isacharcolon}{\kern0pt}\ {\isachardoublequoteopen}Pn{\isacharunderscore}{\kern0pt}auto{\isacharparenleft}{\kern0pt}{\isasympi}{\isacharparenright}{\kern0pt}\ {\isacharbackquote}{\kern0pt}\ x\ {\isacharequal}{\kern0pt}\ wftrec{\isacharparenleft}{\kern0pt}Memrel{\isacharparenleft}{\kern0pt}M{\isacharparenright}{\kern0pt}{\isacharcircum}{\kern0pt}{\isacharplus}{\kern0pt}{\isacharcomma}{\kern0pt}\ x{\isacharcomma}{\kern0pt}\ HPn{\isacharunderscore}{\kern0pt}auto{\isacharparenleft}{\kern0pt}{\isasympi}{\isacharparenright}{\kern0pt}{\isacharparenright}{\kern0pt}{\isachardoublequoteclose}\ \isanewline
\ \ \ \ \isacommand{using}\isamarkupfalse%
\ function{\isacharunderscore}{\kern0pt}value\ assm\ \ \isacommand{unfolding}\isamarkupfalse%
\ Pn{\isacharunderscore}{\kern0pt}auto{\isacharunderscore}{\kern0pt}def\ \isacommand{by}\isamarkupfalse%
\ auto\ \isanewline
\ \ \isacommand{have}\isamarkupfalse%
\ E{\isadigit{2}}{\isacharcolon}{\kern0pt}\isanewline
\ \ \ \ {\isachardoublequoteopen}{\isachardot}{\kern0pt}{\isachardot}{\kern0pt}{\isachardot}{\kern0pt}\ {\isacharequal}{\kern0pt}\ HPn{\isacharunderscore}{\kern0pt}auto{\isacharparenleft}{\kern0pt}{\isasympi}{\isacharcomma}{\kern0pt}\ x{\isacharcomma}{\kern0pt}\ F{\isacharparenright}{\kern0pt}{\isachardoublequoteclose}\isanewline
\ \ \ \ \isacommand{unfolding}\isamarkupfalse%
\ F{\isacharunderscore}{\kern0pt}def\isanewline
\ \ \ \ \isacommand{apply}\isamarkupfalse%
\ {\isacharparenleft}{\kern0pt}subst\ wftrec{\isacharparenright}{\kern0pt}\isanewline
\ \ \ \ \ \ \isacommand{apply}\isamarkupfalse%
{\isacharparenleft}{\kern0pt}rule\ wf{\isacharunderscore}{\kern0pt}trancl{\isacharcomma}{\kern0pt}\ rule\ wf{\isacharunderscore}{\kern0pt}Memrel{\isacharcomma}{\kern0pt}\ rule\ trans{\isacharunderscore}{\kern0pt}trancl{\isacharcomma}{\kern0pt}\ simp{\isacharparenright}{\kern0pt}\isanewline
\ \ \ \ \isacommand{done}\isamarkupfalse%
\isanewline
\ \ \isacommand{have}\isamarkupfalse%
\ E{\isadigit{3}}{\isacharcolon}{\kern0pt}\isanewline
\ \ \ \ {\isachardoublequoteopen}{\isachardot}{\kern0pt}{\isachardot}{\kern0pt}{\isachardot}{\kern0pt}\ {\isacharequal}{\kern0pt}\ {\isacharbraceleft}{\kern0pt}\ {\isacharless}{\kern0pt}F{\isacharbackquote}{\kern0pt}fst{\isacharparenleft}{\kern0pt}v{\isacharparenright}{\kern0pt}{\isacharcomma}{\kern0pt}\ {\isasympi}{\isacharbackquote}{\kern0pt}snd{\isacharparenleft}{\kern0pt}v{\isacharparenright}{\kern0pt}{\isachargreater}{\kern0pt}\ {\isachardot}{\kern0pt}{\isachardot}{\kern0pt}\ v\ {\isasymin}\ x{\isacharcomma}{\kern0pt}\ {\isasymexists}y\ p{\isachardot}{\kern0pt}\ v\ {\isacharequal}{\kern0pt}\ {\isacharless}{\kern0pt}y{\isacharcomma}{\kern0pt}\ p{\isachargreater}{\kern0pt}\ {\isacharbraceright}{\kern0pt}{\isachardoublequoteclose}\ \isacommand{unfolding}\isamarkupfalse%
\ HPn{\isacharunderscore}{\kern0pt}auto{\isacharunderscore}{\kern0pt}def\ \isacommand{by}\isamarkupfalse%
\ auto\ \isanewline
\ \ \isacommand{have}\isamarkupfalse%
\ E{\isadigit{4}}{\isacharcolon}{\kern0pt}\isanewline
\ \ \ \ {\isachardoublequoteopen}{\isachardot}{\kern0pt}{\isachardot}{\kern0pt}{\isachardot}{\kern0pt}\ {\isacharequal}{\kern0pt}\ {\isacharbraceleft}{\kern0pt}\ {\isacharless}{\kern0pt}F{\isacharbackquote}{\kern0pt}y{\isacharcomma}{\kern0pt}\ {\isasympi}{\isacharbackquote}{\kern0pt}p{\isachargreater}{\kern0pt}{\isachardot}{\kern0pt}\ \ {\isacharless}{\kern0pt}y{\isacharcomma}{\kern0pt}\ p{\isachargreater}{\kern0pt}\ {\isasymin}\ x\ {\isacharbraceright}{\kern0pt}{\isachardoublequoteclose}\ \isanewline
\ \ \isacommand{proof}\isamarkupfalse%
{\isacharparenleft}{\kern0pt}rule\ equality{\isacharunderscore}{\kern0pt}iffI{\isacharcomma}{\kern0pt}\ rule\ iffI{\isacharparenright}{\kern0pt}\isanewline
\ \ \ \ \isacommand{fix}\isamarkupfalse%
\ w\ \isacommand{assume}\isamarkupfalse%
\ {\isachardoublequoteopen}w\ {\isasymin}\ {\isacharbraceleft}{\kern0pt}\ {\isacharless}{\kern0pt}F{\isacharbackquote}{\kern0pt}fst{\isacharparenleft}{\kern0pt}v{\isacharparenright}{\kern0pt}{\isacharcomma}{\kern0pt}\ {\isasympi}{\isacharbackquote}{\kern0pt}snd{\isacharparenleft}{\kern0pt}v{\isacharparenright}{\kern0pt}{\isachargreater}{\kern0pt}\ {\isachardot}{\kern0pt}{\isachardot}{\kern0pt}\ v\ {\isasymin}\ x{\isacharcomma}{\kern0pt}\ {\isasymexists}y\ p{\isachardot}{\kern0pt}\ v\ {\isacharequal}{\kern0pt}\ {\isacharless}{\kern0pt}y{\isacharcomma}{\kern0pt}\ p{\isachargreater}{\kern0pt}\ {\isacharbraceright}{\kern0pt}{\isachardoublequoteclose}\ \isanewline
\ \ \ \ \isacommand{then}\isamarkupfalse%
\ \isacommand{obtain}\isamarkupfalse%
\ v\ \isakeyword{where}\ {\isachardoublequoteopen}w\ {\isacharequal}{\kern0pt}\ {\isacharless}{\kern0pt}F{\isacharbackquote}{\kern0pt}fst{\isacharparenleft}{\kern0pt}v{\isacharparenright}{\kern0pt}{\isacharcomma}{\kern0pt}\ {\isasympi}{\isacharbackquote}{\kern0pt}snd{\isacharparenleft}{\kern0pt}v{\isacharparenright}{\kern0pt}{\isachargreater}{\kern0pt}{\isachardoublequoteclose}\ {\isachardoublequoteopen}v\ {\isasymin}\ x{\isachardoublequoteclose}\ {\isachardoublequoteopen}{\isasymexists}y\ p{\isachardot}{\kern0pt}\ v\ {\isacharequal}{\kern0pt}\ {\isacharless}{\kern0pt}y{\isacharcomma}{\kern0pt}\ p{\isachargreater}{\kern0pt}{\isachardoublequoteclose}\ \isacommand{by}\isamarkupfalse%
\ auto\ \isanewline
\ \ \ \ \isacommand{then}\isamarkupfalse%
\ \isacommand{obtain}\isamarkupfalse%
\ y\ p\ \isakeyword{where}\ {\isachardoublequoteopen}w\ {\isacharequal}{\kern0pt}\ {\isacharless}{\kern0pt}F{\isacharbackquote}{\kern0pt}y{\isacharcomma}{\kern0pt}\ {\isasympi}{\isacharbackquote}{\kern0pt}p{\isachargreater}{\kern0pt}{\isachardoublequoteclose}\ {\isachardoublequoteopen}{\isacharless}{\kern0pt}y{\isacharcomma}{\kern0pt}\ p{\isachargreater}{\kern0pt}\ {\isasymin}\ x{\isachardoublequoteclose}\ \isacommand{by}\isamarkupfalse%
\ auto\ \isanewline
\ \ \ \ \isacommand{then}\isamarkupfalse%
\ \isacommand{show}\isamarkupfalse%
\ {\isachardoublequoteopen}w\ {\isasymin}\ {\isacharbraceleft}{\kern0pt}{\isasymlangle}F\ {\isacharbackquote}{\kern0pt}\ y{\isacharcomma}{\kern0pt}\ {\isasympi}\ {\isacharbackquote}{\kern0pt}\ p{\isasymrangle}\ {\isachardot}{\kern0pt}\ {\isasymlangle}y{\isacharcomma}{\kern0pt}p{\isasymrangle}\ {\isasymin}\ x{\isacharbraceright}{\kern0pt}{\isachardoublequoteclose}\ \isanewline
\ \ \ \ \ \ \isacommand{apply}\isamarkupfalse%
\ clarify\isanewline
\ \ \ \ \ \ \isacommand{apply}\isamarkupfalse%
\ force\isanewline
\ \ \ \ \ \ \isacommand{done}\isamarkupfalse%
\isanewline
\ \ \isacommand{next}\isamarkupfalse%
\ \isanewline
\ \ \ \ \isacommand{fix}\isamarkupfalse%
\ v\ \isacommand{assume}\isamarkupfalse%
\ {\isachardoublequoteopen}v\ {\isasymin}\ {\isacharbraceleft}{\kern0pt}{\isasymlangle}F\ {\isacharbackquote}{\kern0pt}\ y{\isacharcomma}{\kern0pt}\ {\isasympi}\ {\isacharbackquote}{\kern0pt}\ p{\isasymrangle}\ {\isachardot}{\kern0pt}\ {\isasymlangle}y{\isacharcomma}{\kern0pt}p{\isasymrangle}\ {\isasymin}\ x{\isacharbraceright}{\kern0pt}{\isachardoublequoteclose}\ \ \isanewline
\ \ \ \ \isacommand{then}\isamarkupfalse%
\ \isacommand{have}\isamarkupfalse%
\ {\isachardoublequoteopen}{\isasymexists}y\ p{\isachardot}{\kern0pt}\ {\isacharless}{\kern0pt}y{\isacharcomma}{\kern0pt}\ p{\isachargreater}{\kern0pt}\ {\isasymin}\ x\ {\isasymand}\ v\ {\isacharequal}{\kern0pt}\ {\isasymlangle}F\ {\isacharbackquote}{\kern0pt}\ y{\isacharcomma}{\kern0pt}\ {\isasympi}\ {\isacharbackquote}{\kern0pt}\ p{\isasymrangle}{\isachardoublequoteclose}\ \isanewline
\ \ \ \ \ \ \isacommand{apply}\isamarkupfalse%
{\isacharparenleft}{\kern0pt}rule{\isacharunderscore}{\kern0pt}tac\ pair{\isacharunderscore}{\kern0pt}rel{\isacharunderscore}{\kern0pt}arg{\isacharparenright}{\kern0pt}\isanewline
\ \ \ \ \ \ \isacommand{using}\isamarkupfalse%
\ assm\ relation{\isacharunderscore}{\kern0pt}P{\isacharunderscore}{\kern0pt}name\ \isanewline
\ \ \ \ \ \ \isacommand{by}\isamarkupfalse%
\ auto\isanewline
\ \ \ \ \isacommand{then}\isamarkupfalse%
\ \isacommand{obtain}\isamarkupfalse%
\ y\ p\ \isakeyword{where}\ {\isachardoublequoteopen}{\isacharless}{\kern0pt}y{\isacharcomma}{\kern0pt}\ p{\isachargreater}{\kern0pt}\ {\isasymin}\ x{\isachardoublequoteclose}\ {\isachardoublequoteopen}v\ {\isacharequal}{\kern0pt}\ {\isasymlangle}F\ {\isacharbackquote}{\kern0pt}\ y{\isacharcomma}{\kern0pt}\ {\isasympi}\ {\isacharbackquote}{\kern0pt}\ p{\isasymrangle}{\isachardoublequoteclose}\ \isacommand{by}\isamarkupfalse%
\ blast\isanewline
\ \ \ \ \isacommand{then}\isamarkupfalse%
\ \isacommand{show}\isamarkupfalse%
\ {\isachardoublequoteopen}v\ {\isasymin}\ {\isacharbraceleft}{\kern0pt}\ {\isacharless}{\kern0pt}F{\isacharbackquote}{\kern0pt}fst{\isacharparenleft}{\kern0pt}v{\isacharparenright}{\kern0pt}{\isacharcomma}{\kern0pt}\ {\isasympi}{\isacharbackquote}{\kern0pt}snd{\isacharparenleft}{\kern0pt}v{\isacharparenright}{\kern0pt}{\isachargreater}{\kern0pt}\ {\isachardot}{\kern0pt}{\isachardot}{\kern0pt}\ v\ {\isasymin}\ x{\isacharcomma}{\kern0pt}\ {\isasymexists}y\ p{\isachardot}{\kern0pt}\ v\ {\isacharequal}{\kern0pt}\ {\isacharless}{\kern0pt}y{\isacharcomma}{\kern0pt}\ p{\isachargreater}{\kern0pt}\ {\isacharbraceright}{\kern0pt}{\isachardoublequoteclose}\ \isanewline
\ \ \ \ \ \ \isacommand{apply}\isamarkupfalse%
{\isacharparenleft}{\kern0pt}rule{\isacharunderscore}{\kern0pt}tac\ iffD{\isadigit{2}}{\isacharcomma}{\kern0pt}\ rule{\isacharunderscore}{\kern0pt}tac\ SepReplace{\isacharunderscore}{\kern0pt}iff{\isacharparenright}{\kern0pt}\isanewline
\ \ \ \ \ \ \isacommand{apply}\isamarkupfalse%
{\isacharparenleft}{\kern0pt}rule{\isacharunderscore}{\kern0pt}tac\ x{\isacharequal}{\kern0pt}{\isachardoublequoteopen}{\isacharless}{\kern0pt}y{\isacharcomma}{\kern0pt}\ p{\isachargreater}{\kern0pt}{\isachardoublequoteclose}\ \isakeyword{in}\ bexI{\isacharparenright}{\kern0pt}\isanewline
\ \ \ \ \ \ \isacommand{by}\isamarkupfalse%
\ auto\isanewline
\ \ \isacommand{qed}\isamarkupfalse%
\isanewline
\ \ \isacommand{have}\isamarkupfalse%
\ E{\isadigit{5}}{\isacharcolon}{\kern0pt}\isanewline
\ \ \ \ {\isachardoublequoteopen}{\isachardot}{\kern0pt}{\isachardot}{\kern0pt}{\isachardot}{\kern0pt}\ {\isacharequal}{\kern0pt}\ {\isacharbraceleft}{\kern0pt}\ {\isacharless}{\kern0pt}wftrec{\isacharparenleft}{\kern0pt}Memrel{\isacharparenleft}{\kern0pt}M{\isacharparenright}{\kern0pt}{\isacharcircum}{\kern0pt}{\isacharplus}{\kern0pt}{\isacharcomma}{\kern0pt}\ y{\isacharcomma}{\kern0pt}\ HPn{\isacharunderscore}{\kern0pt}auto{\isacharparenleft}{\kern0pt}{\isasympi}{\isacharparenright}{\kern0pt}{\isacharparenright}{\kern0pt}{\isacharcomma}{\kern0pt}\ {\isasympi}{\isacharbackquote}{\kern0pt}p{\isachargreater}{\kern0pt}{\isachardot}{\kern0pt}\ {\isacharless}{\kern0pt}y{\isacharcomma}{\kern0pt}\ p{\isachargreater}{\kern0pt}\ {\isasymin}\ x\ {\isacharbraceright}{\kern0pt}{\isachardoublequoteclose}\isanewline
\ \ \ \ \isacommand{unfolding}\isamarkupfalse%
\ F{\isacharunderscore}{\kern0pt}def\isanewline
\ \ \ \ \isacommand{apply}\isamarkupfalse%
\ {\isacharparenleft}{\kern0pt}rule{\isacharunderscore}{\kern0pt}tac\ pair{\isacharunderscore}{\kern0pt}rel{\isacharunderscore}{\kern0pt}eq{\isacharparenright}{\kern0pt}\isanewline
\ \ \ \ \isacommand{using}\isamarkupfalse%
\ xin\ assm\ relation{\isacharunderscore}{\kern0pt}P{\isacharunderscore}{\kern0pt}name\isanewline
\ \ \ \ \ \isacommand{apply}\isamarkupfalse%
\ force\isanewline
\ \ \ \ \isacommand{apply}\isamarkupfalse%
{\isacharparenleft}{\kern0pt}rule\ allI{\isacharparenright}{\kern0pt}{\isacharplus}{\kern0pt}\isanewline
\ \ \ \ \isacommand{apply}\isamarkupfalse%
{\isacharparenleft}{\kern0pt}rule\ impI{\isacharparenright}{\kern0pt}\isanewline
\ \ \ \ \isacommand{apply}\isamarkupfalse%
{\isacharparenleft}{\kern0pt}rename{\isacharunderscore}{\kern0pt}tac\ y\ p{\isacharcomma}{\kern0pt}\ subgoal{\isacharunderscore}{\kern0pt}tac\ {\isachardoublequoteopen}y{\isasymin}Memrel{\isacharparenleft}{\kern0pt}M{\isacharparenright}{\kern0pt}{\isacharcircum}{\kern0pt}{\isacharplus}{\kern0pt}\ {\isacharminus}{\kern0pt}{\isacharbackquote}{\kern0pt}{\isacharbackquote}{\kern0pt}\ {\isacharbraceleft}{\kern0pt}x{\isacharbraceright}{\kern0pt}{\isachardoublequoteclose}{\isacharparenright}{\kern0pt}\ \isanewline
\ \ \ \ \ \isacommand{apply}\isamarkupfalse%
\ force\isanewline
\ \ \ \ \isacommand{apply}\isamarkupfalse%
{\isacharparenleft}{\kern0pt}rule{\isacharunderscore}{\kern0pt}tac\ b{\isacharequal}{\kern0pt}x\ \isakeyword{in}\ vimageI{\isacharparenright}{\kern0pt}\ \isanewline
\ \ \ \ \ \isacommand{apply}\isamarkupfalse%
{\isacharparenleft}{\kern0pt}rule\ domain{\isacharunderscore}{\kern0pt}elem{\isacharunderscore}{\kern0pt}Memrel{\isacharunderscore}{\kern0pt}trancl{\isacharparenright}{\kern0pt}\isanewline
\ \ \ \ \isacommand{using}\isamarkupfalse%
\ assm\ P{\isacharunderscore}{\kern0pt}name{\isacharunderscore}{\kern0pt}in{\isacharunderscore}{\kern0pt}M\ \isanewline
\ \ \ \ \isacommand{by}\isamarkupfalse%
\ auto\isanewline
\ \ \isacommand{have}\isamarkupfalse%
\ E{\isadigit{6}}{\isacharcolon}{\kern0pt}\ \isanewline
\ \ \ \ {\isachardoublequoteopen}{\isachardot}{\kern0pt}{\isachardot}{\kern0pt}{\isachardot}{\kern0pt}\ {\isacharequal}{\kern0pt}\ {\isacharbraceleft}{\kern0pt}\ {\isacharless}{\kern0pt}Pn{\isacharunderscore}{\kern0pt}auto{\isacharparenleft}{\kern0pt}{\isasympi}{\isacharparenright}{\kern0pt}{\isacharbackquote}{\kern0pt}y{\isacharcomma}{\kern0pt}\ {\isasympi}{\isacharbackquote}{\kern0pt}p{\isachargreater}{\kern0pt}{\isachardot}{\kern0pt}\ {\isacharless}{\kern0pt}y{\isacharcomma}{\kern0pt}\ p{\isachargreater}{\kern0pt}\ {\isasymin}\ x\ {\isacharbraceright}{\kern0pt}{\isachardoublequoteclose}\isanewline
\ \ \ \ \ \isacommand{apply}\isamarkupfalse%
\ {\isacharparenleft}{\kern0pt}rule{\isacharunderscore}{\kern0pt}tac\ pair{\isacharunderscore}{\kern0pt}rel{\isacharunderscore}{\kern0pt}eq{\isacharparenright}{\kern0pt}\ \isanewline
\ \ \ \ \ \isacommand{using}\isamarkupfalse%
\ xin\ assm\ relation{\isacharunderscore}{\kern0pt}P{\isacharunderscore}{\kern0pt}name\ \isanewline
\ \ \ \ \ \ \isacommand{apply}\isamarkupfalse%
\ auto\isanewline
\ \ \ \ \isacommand{apply}\isamarkupfalse%
\ {\isacharparenleft}{\kern0pt}rule{\isacharunderscore}{\kern0pt}tac\ eq{\isacharunderscore}{\kern0pt}flip{\isacharparenright}{\kern0pt}\isanewline
\ \ \ \ \isacommand{unfolding}\isamarkupfalse%
\ Pn{\isacharunderscore}{\kern0pt}auto{\isacharunderscore}{\kern0pt}def\isanewline
\ \ \ \ \isacommand{apply}\isamarkupfalse%
\ {\isacharparenleft}{\kern0pt}rule{\isacharunderscore}{\kern0pt}tac\ function{\isacharunderscore}{\kern0pt}value{\isacharparenright}{\kern0pt}\isanewline
\ \ \ \ \isacommand{using}\isamarkupfalse%
\ P{\isacharunderscore}{\kern0pt}name{\isacharunderscore}{\kern0pt}domain{\isacharunderscore}{\kern0pt}P{\isacharunderscore}{\kern0pt}name\ assm\ \isanewline
\ \ \ \ \isacommand{by}\isamarkupfalse%
\ auto\ \isanewline
\ \ \isacommand{show}\isamarkupfalse%
\ {\isachardoublequoteopen}\ Pn{\isacharunderscore}{\kern0pt}auto{\isacharparenleft}{\kern0pt}{\isasympi}{\isacharparenright}{\kern0pt}{\isacharbackquote}{\kern0pt}x\ {\isacharequal}{\kern0pt}\ {\isacharbraceleft}{\kern0pt}\ {\isacharless}{\kern0pt}Pn{\isacharunderscore}{\kern0pt}auto{\isacharparenleft}{\kern0pt}{\isasympi}{\isacharparenright}{\kern0pt}{\isacharbackquote}{\kern0pt}y{\isacharcomma}{\kern0pt}\ {\isasympi}{\isacharbackquote}{\kern0pt}p{\isachargreater}{\kern0pt}\ {\isachardot}{\kern0pt}\ {\isacharless}{\kern0pt}y{\isacharcomma}{\kern0pt}\ p{\isachargreater}{\kern0pt}\ {\isasymin}\ x{\isacharbraceright}{\kern0pt}{\isachardoublequoteclose}\ \isanewline
\ \ \ \ \isacommand{using}\isamarkupfalse%
\ E{\isadigit{1}}\ E{\isadigit{2}}\ E{\isadigit{3}}\ E{\isadigit{4}}\ E{\isadigit{5}}\ E{\isadigit{6}}\isanewline
\ \ \ \ \isacommand{by}\isamarkupfalse%
\ auto\isanewline
\isacommand{qed}\isamarkupfalse%
%
\endisatagproof
{\isafoldproof}%
%
\isadelimproof
\ \isanewline
%
\endisadelimproof
\isanewline
\isacommand{end}\isamarkupfalse%
\isanewline
%
\isadelimtheory
%
\endisadelimtheory
%
\isatagtheory
\isacommand{end}\isamarkupfalse%
%
\endisatagtheory
{\isafoldtheory}%
%
\isadelimtheory
%
\endisadelimtheory
%
\end{isabellebody}%
\endinput
%:%file=~/source/repos/ZF-notAC/code/Automorphism_Definition.thy%:%
%:%10=1%:%
%:%11=1%:%
%:%12=2%:%
%:%13=3%:%
%:%14=4%:%
%:%15=5%:%
%:%20=5%:%
%:%23=6%:%
%:%24=7%:%
%:%25=7%:%
%:%26=8%:%
%:%27=9%:%
%:%28=10%:%
%:%29=11%:%
%:%30=12%:%
%:%31=13%:%
%:%32=13%:%
%:%35=14%:%
%:%39=14%:%
%:%40=14%:%
%:%41=15%:%
%:%42=15%:%
%:%43=16%:%
%:%44=16%:%
%:%49=16%:%
%:%52=17%:%
%:%53=18%:%
%:%54=18%:%
%:%61=19%:%
%:%62=19%:%
%:%63=20%:%
%:%64=20%:%
%:%65=21%:%
%:%66=21%:%
%:%67=21%:%
%:%68=21%:%
%:%69=21%:%
%:%70=22%:%
%:%71=22%:%
%:%72=22%:%
%:%73=22%:%
%:%74=22%:%
%:%75=22%:%
%:%76=23%:%
%:%82=23%:%
%:%85=24%:%
%:%86=25%:%
%:%87=25%:%
%:%88=26%:%
%:%89=27%:%
%:%90=28%:%
%:%91=28%:%
%:%92=29%:%
%:%93=30%:%
%:%94=30%:%
%:%96=30%:%
%:%100=30%:%
%:%101=30%:%
%:%102=30%:%
%:%109=30%:%
%:%110=31%:%
%:%111=32%:%
%:%112=32%:%
%:%115=33%:%
%:%119=33%:%
%:%120=33%:%
%:%121=34%:%
%:%122=34%:%
%:%123=34%:%
%:%128=34%:%
%:%131=35%:%
%:%132=36%:%
%:%133=36%:%
%:%140=37%:%
%:%141=37%:%
%:%142=38%:%
%:%143=38%:%
%:%144=39%:%
%:%145=39%:%
%:%146=39%:%
%:%147=39%:%
%:%148=39%:%
%:%149=39%:%
%:%150=40%:%
%:%151=40%:%
%:%152=40%:%
%:%153=40%:%
%:%154=40%:%
%:%155=41%:%
%:%161=41%:%
%:%164=42%:%
%:%165=43%:%
%:%166=43%:%
%:%173=44%:%
%:%174=44%:%
%:%175=45%:%
%:%176=45%:%
%:%177=46%:%
%:%178=46%:%
%:%179=46%:%
%:%180=47%:%
%:%181=47%:%
%:%182=47%:%
%:%183=47%:%
%:%184=48%:%
%:%185=48%:%
%:%186=48%:%
%:%187=48%:%
%:%188=48%:%
%:%189=49%:%
%:%195=49%:%
%:%198=50%:%
%:%199=51%:%
%:%200=51%:%
%:%201=52%:%
%:%204=53%:%
%:%208=53%:%
%:%209=53%:%
%:%210=53%:%
%:%215=53%:%
%:%218=54%:%
%:%219=55%:%
%:%220=55%:%
%:%221=56%:%
%:%224=57%:%
%:%228=57%:%
%:%229=57%:%
%:%230=57%:%
%:%235=57%:%
%:%238=58%:%
%:%239=59%:%
%:%240=59%:%
%:%243=60%:%
%:%247=60%:%
%:%248=60%:%
%:%249=61%:%
%:%250=61%:%
%:%251=61%:%
%:%252=61%:%
%:%253=61%:%
%:%254=62%:%
%:%255=62%:%
%:%256=62%:%
%:%257=62%:%
%:%263=62%:%
%:%266=63%:%
%:%267=64%:%
%:%268=64%:%
%:%271=65%:%
%:%275=65%:%
%:%276=65%:%
%:%277=65%:%
%:%278=65%:%
%:%279=65%:%
%:%280=66%:%
%:%281=66%:%
%:%282=66%:%
%:%283=66%:%
%:%284=66%:%
%:%285=67%:%
%:%286=67%:%
%:%287=68%:%
%:%288=68%:%
%:%289=69%:%
%:%290=69%:%
%:%291=70%:%
%:%292=70%:%
%:%293=70%:%
%:%294=70%:%
%:%295=70%:%
%:%296=71%:%
%:%297=71%:%
%:%298=71%:%
%:%299=71%:%
%:%300=71%:%
%:%301=71%:%
%:%302=72%:%
%:%303=72%:%
%:%304=72%:%
%:%305=72%:%
%:%306=72%:%
%:%307=72%:%
%:%308=73%:%
%:%309=73%:%
%:%310=73%:%
%:%311=73%:%
%:%312=74%:%
%:%313=74%:%
%:%314=74%:%
%:%315=74%:%
%:%316=75%:%
%:%317=75%:%
%:%318=76%:%
%:%319=76%:%
%:%320=77%:%
%:%321=77%:%
%:%322=78%:%
%:%323=78%:%
%:%324=79%:%
%:%325=79%:%
%:%326=79%:%
%:%327=79%:%
%:%328=80%:%
%:%329=80%:%
%:%330=80%:%
%:%331=80%:%
%:%332=81%:%
%:%333=81%:%
%:%334=82%:%
%:%335=82%:%
%:%336=83%:%
%:%337=83%:%
%:%338=83%:%
%:%339=83%:%
%:%340=84%:%
%:%341=84%:%
%:%342=84%:%
%:%343=84%:%
%:%344=84%:%
%:%345=85%:%
%:%346=85%:%
%:%347=86%:%
%:%353=86%:%
%:%356=87%:%
%:%357=88%:%
%:%358=88%:%
%:%365=89%:%
%:%366=89%:%
%:%367=90%:%
%:%368=90%:%
%:%369=91%:%
%:%370=91%:%
%:%371=92%:%
%:%372=92%:%
%:%373=92%:%
%:%374=92%:%
%:%375=93%:%
%:%376=93%:%
%:%377=93%:%
%:%378=93%:%
%:%379=93%:%
%:%380=94%:%
%:%381=94%:%
%:%382=94%:%
%:%383=95%:%
%:%384=95%:%
%:%385=95%:%
%:%386=96%:%
%:%387=96%:%
%:%388=96%:%
%:%389=96%:%
%:%390=96%:%
%:%391=97%:%
%:%392=97%:%
%:%393=97%:%
%:%394=97%:%
%:%395=97%:%
%:%396=98%:%
%:%402=98%:%
%:%405=99%:%
%:%406=100%:%
%:%407=100%:%
%:%408=101%:%
%:%411=102%:%
%:%415=102%:%
%:%416=102%:%
%:%417=102%:%
%:%418=103%:%
%:%419=103%:%
%:%420=104%:%
%:%421=104%:%
%:%422=105%:%
%:%423=105%:%
%:%424=106%:%
%:%425=107%:%
%:%426=107%:%
%:%427=107%:%
%:%428=107%:%
%:%429=107%:%
%:%430=108%:%
%:%431=108%:%
%:%432=108%:%
%:%433=108%:%
%:%434=108%:%
%:%435=108%:%
%:%436=109%:%
%:%437=109%:%
%:%438=109%:%
%:%439=109%:%
%:%440=109%:%
%:%441=109%:%
%:%442=109%:%
%:%443=110%:%
%:%444=110%:%
%:%445=110%:%
%:%446=110%:%
%:%447=110%:%
%:%448=111%:%
%:%449=111%:%
%:%450=111%:%
%:%451=111%:%
%:%452=112%:%
%:%453=112%:%
%:%454=113%:%
%:%455=113%:%
%:%456=113%:%
%:%457=113%:%
%:%458=114%:%
%:%459=114%:%
%:%460=114%:%
%:%461=114%:%
%:%462=114%:%
%:%463=115%:%
%:%464=115%:%
%:%465=115%:%
%:%466=115%:%
%:%467=116%:%
%:%468=116%:%
%:%469=116%:%
%:%470=116%:%
%:%471=116%:%
%:%472=117%:%
%:%478=117%:%
%:%481=118%:%
%:%482=119%:%
%:%483=119%:%
%:%484=120%:%
%:%487=121%:%
%:%491=121%:%
%:%492=121%:%
%:%493=121%:%
%:%494=121%:%
%:%495=122%:%
%:%496=122%:%
%:%497=123%:%
%:%498=123%:%
%:%499=124%:%
%:%500=124%:%
%:%501=124%:%
%:%502=125%:%
%:%503=125%:%
%:%504=126%:%
%:%505=126%:%
%:%506=127%:%
%:%507=127%:%
%:%508=127%:%
%:%509=128%:%
%:%510=128%:%
%:%511=129%:%
%:%512=129%:%
%:%513=130%:%
%:%514=130%:%
%:%515=131%:%
%:%516=131%:%
%:%517=132%:%
%:%518=132%:%
%:%519=132%:%
%:%520=132%:%
%:%521=133%:%
%:%522=133%:%
%:%523=133%:%
%:%524=133%:%
%:%525=133%:%
%:%526=134%:%
%:%527=134%:%
%:%528=134%:%
%:%529=135%:%
%:%535=135%:%
%:%538=136%:%
%:%539=137%:%
%:%540=137%:%
%:%543=138%:%
%:%547=138%:%
%:%548=138%:%
%:%549=139%:%
%:%550=139%:%
%:%551=140%:%
%:%552=140%:%
%:%553=141%:%
%:%554=141%:%
%:%555=142%:%
%:%556=142%:%
%:%557=143%:%
%:%558=143%:%
%:%559=144%:%
%:%560=144%:%
%:%561=145%:%
%:%562=145%:%
%:%563=146%:%
%:%564=146%:%
%:%565=147%:%
%:%566=147%:%
%:%567=148%:%
%:%568=148%:%
%:%569=149%:%
%:%570=149%:%
%:%571=150%:%
%:%577=150%:%
%:%580=151%:%
%:%581=152%:%
%:%582=152%:%
%:%583=153%:%
%:%584=154%:%
%:%585=155%:%
%:%586=155%:%
%:%587=156%:%
%:%588=157%:%
%:%589=158%:%
%:%590=159%:%
%:%591=159%:%
%:%594=160%:%
%:%598=160%:%
%:%599=160%:%
%:%600=160%:%
%:%605=160%:%
%:%608=161%:%
%:%609=161%:%
%:%612=162%:%
%:%616=162%:%
%:%617=162%:%
%:%618=162%:%
%:%623=162%:%
%:%626=163%:%
%:%627=164%:%
%:%628=164%:%
%:%635=165%:%
%:%636=165%:%
%:%637=166%:%
%:%638=166%:%
%:%639=167%:%
%:%640=168%:%
%:%641=168%:%
%:%642=169%:%
%:%643=169%:%
%:%644=169%:%
%:%645=169%:%
%:%646=169%:%
%:%647=170%:%
%:%648=170%:%
%:%649=170%:%
%:%650=171%:%
%:%651=171%:%
%:%652=171%:%
%:%653=171%:%
%:%654=172%:%
%:%655=172%:%
%:%656=173%:%
%:%657=174%:%
%:%658=174%:%
%:%659=175%:%
%:%660=175%:%
%:%661=176%:%
%:%662=176%:%
%:%663=177%:%
%:%664=177%:%
%:%665=178%:%
%:%666=178%:%
%:%667=179%:%
%:%668=179%:%
%:%669=179%:%
%:%670=180%:%
%:%671=180%:%
%:%672=181%:%
%:%673=182%:%
%:%674=182%:%
%:%675=183%:%
%:%676=183%:%
%:%677=183%:%
%:%678=184%:%
%:%679=184%:%
%:%680=184%:%
%:%681=184%:%
%:%682=185%:%
%:%683=185%:%
%:%684=185%:%
%:%685=185%:%
%:%686=186%:%
%:%687=186%:%
%:%688=186%:%
%:%689=187%:%
%:%690=187%:%
%:%691=188%:%
%:%692=188%:%
%:%693=189%:%
%:%694=189%:%
%:%695=190%:%
%:%696=190%:%
%:%697=191%:%
%:%698=191%:%
%:%699=191%:%
%:%700=192%:%
%:%701=192%:%
%:%702=192%:%
%:%703=193%:%
%:%704=193%:%
%:%705=194%:%
%:%706=194%:%
%:%707=195%:%
%:%708=195%:%
%:%709=196%:%
%:%710=196%:%
%:%711=196%:%
%:%712=196%:%
%:%713=197%:%
%:%714=197%:%
%:%715=197%:%
%:%716=198%:%
%:%717=198%:%
%:%718=199%:%
%:%719=199%:%
%:%720=200%:%
%:%721=200%:%
%:%722=201%:%
%:%723=201%:%
%:%724=202%:%
%:%725=202%:%
%:%726=203%:%
%:%727=204%:%
%:%728=204%:%
%:%729=205%:%
%:%730=205%:%
%:%731=206%:%
%:%732=206%:%
%:%733=207%:%
%:%734=207%:%
%:%735=208%:%
%:%736=208%:%
%:%737=209%:%
%:%738=209%:%
%:%739=210%:%
%:%740=210%:%
%:%741=211%:%
%:%742=211%:%
%:%743=212%:%
%:%744=212%:%
%:%745=213%:%
%:%746=213%:%
%:%747=214%:%
%:%748=214%:%
%:%749=215%:%
%:%750=215%:%
%:%751=216%:%
%:%752=216%:%
%:%753=217%:%
%:%754=218%:%
%:%755=218%:%
%:%756=219%:%
%:%757=219%:%
%:%758=220%:%
%:%759=220%:%
%:%760=221%:%
%:%761=221%:%
%:%762=222%:%
%:%763=222%:%
%:%764=223%:%
%:%765=223%:%
%:%766=224%:%
%:%767=224%:%
%:%768=225%:%
%:%769=225%:%
%:%770=226%:%
%:%771=226%:%
%:%772=227%:%
%:%773=227%:%
%:%774=228%:%
%:%775=228%:%
%:%776=229%:%
%:%782=229%:%
%:%785=230%:%
%:%786=231%:%
%:%787=231%:%
%:%794=232%:%

%
\begin{isabellebody}%
\setisabellecontext{Automorphism{\isacharunderscore}{\kern0pt}M}%
%
\isadelimtheory
%
\endisadelimtheory
%
\isatagtheory
\isacommand{theory}\isamarkupfalse%
\ Automorphism{\isacharunderscore}{\kern0pt}M\isanewline
\ \ \isakeyword{imports}\ \isanewline
\ \ \ \ {\isachardoublequoteopen}Forcing{\isacharslash}{\kern0pt}Forcing{\isacharunderscore}{\kern0pt}Main{\isachardoublequoteclose}\ \isanewline
\ \ \ \ P{\isacharunderscore}{\kern0pt}Names{\isacharunderscore}{\kern0pt}M\isanewline
\ \ \ \ Automorphism{\isacharunderscore}{\kern0pt}Definition\isanewline
\isakeyword{begin}%
\endisatagtheory
{\isafoldtheory}%
%
\isadelimtheory
\ \isanewline
%
\endisadelimtheory
\isanewline
\isacommand{schematic{\isacharunderscore}{\kern0pt}goal}\isamarkupfalse%
\ is{\isacharunderscore}{\kern0pt}P{\isacharunderscore}{\kern0pt}auto{\isacharunderscore}{\kern0pt}fm{\isacharunderscore}{\kern0pt}auto{\isacharcolon}{\kern0pt}\isanewline
\ \ \isakeyword{assumes}\isanewline
\ \ \ \ {\isachardoublequoteopen}nth{\isacharparenleft}{\kern0pt}{\isadigit{0}}{\isacharcomma}{\kern0pt}env{\isacharparenright}{\kern0pt}\ {\isacharequal}{\kern0pt}\ {\isasympi}{\isachardoublequoteclose}\ \isanewline
\ \ \ \ {\isachardoublequoteopen}nth{\isacharparenleft}{\kern0pt}{\isadigit{1}}{\isacharcomma}{\kern0pt}env{\isacharparenright}{\kern0pt}\ {\isacharequal}{\kern0pt}\ P{\isachardoublequoteclose}\ \isanewline
\ \ \ \ {\isachardoublequoteopen}nth{\isacharparenleft}{\kern0pt}{\isadigit{2}}{\isacharcomma}{\kern0pt}env{\isacharparenright}{\kern0pt}\ {\isacharequal}{\kern0pt}\ leq{\isachardoublequoteclose}\ \ \isanewline
\ \ \ \ {\isachardoublequoteopen}env\ {\isasymin}\ list{\isacharparenleft}{\kern0pt}M{\isacharparenright}{\kern0pt}{\isachardoublequoteclose}\isanewline
\ \isakeyword{shows}\ \isanewline
\ \ \ \ {\isachardoublequoteopen}{\isacharparenleft}{\kern0pt}bijection{\isacharparenleft}{\kern0pt}{\isacharhash}{\kern0pt}{\isacharhash}{\kern0pt}M{\isacharcomma}{\kern0pt}\ P{\isacharcomma}{\kern0pt}\ P{\isacharcomma}{\kern0pt}\ {\isasympi}{\isacharparenright}{\kern0pt}\ {\isasymand}\ \isanewline
\ \ \ \ \ \ {\isacharparenleft}{\kern0pt}{\isasymforall}p\ {\isasymin}\ M{\isachardot}{\kern0pt}\ {\isasymforall}q\ {\isasymin}\ M{\isachardot}{\kern0pt}\ {\isasymforall}p{\isacharunderscore}{\kern0pt}q\ {\isasymin}\ M{\isachardot}{\kern0pt}\ {\isasymforall}{\isasympi}p\ {\isasymin}\ M{\isachardot}{\kern0pt}\ {\isasymforall}{\isasympi}q\ {\isasymin}\ M{\isachardot}{\kern0pt}\ {\isasymforall}{\isasympi}p{\isacharunderscore}{\kern0pt}{\isasympi}q\ {\isasymin}\ M{\isachardot}{\kern0pt}\ \isanewline
\ \ \ \ \ \ \ \ p\ {\isasymin}\ P\ {\isasymlongrightarrow}\ q\ {\isasymin}\ P\ {\isasymlongrightarrow}\ fun{\isacharunderscore}{\kern0pt}apply{\isacharparenleft}{\kern0pt}{\isacharhash}{\kern0pt}{\isacharhash}{\kern0pt}M{\isacharcomma}{\kern0pt}\ {\isasympi}{\isacharcomma}{\kern0pt}\ p{\isacharcomma}{\kern0pt}\ {\isasympi}p{\isacharparenright}{\kern0pt}\ {\isasymlongrightarrow}\ fun{\isacharunderscore}{\kern0pt}apply{\isacharparenleft}{\kern0pt}{\isacharhash}{\kern0pt}{\isacharhash}{\kern0pt}M{\isacharcomma}{\kern0pt}\ {\isasympi}{\isacharcomma}{\kern0pt}\ q{\isacharcomma}{\kern0pt}\ {\isasympi}q{\isacharparenright}{\kern0pt}\ {\isasymlongrightarrow}\isanewline
\ \ \ \ \ \ \ \ pair{\isacharparenleft}{\kern0pt}{\isacharhash}{\kern0pt}{\isacharhash}{\kern0pt}M{\isacharcomma}{\kern0pt}\ p{\isacharcomma}{\kern0pt}\ q{\isacharcomma}{\kern0pt}\ p{\isacharunderscore}{\kern0pt}q{\isacharparenright}{\kern0pt}\ {\isasymlongrightarrow}\ pair{\isacharparenleft}{\kern0pt}{\isacharhash}{\kern0pt}{\isacharhash}{\kern0pt}M{\isacharcomma}{\kern0pt}\ {\isasympi}p{\isacharcomma}{\kern0pt}\ {\isasympi}q{\isacharcomma}{\kern0pt}\ {\isasympi}p{\isacharunderscore}{\kern0pt}{\isasympi}q{\isacharparenright}{\kern0pt}\ {\isasymlongrightarrow}\ {\isacharparenleft}{\kern0pt}p{\isacharunderscore}{\kern0pt}q\ {\isasymin}\ leq\ \ {\isasymlongleftrightarrow}\ {\isasympi}p{\isacharunderscore}{\kern0pt}{\isasympi}q\ {\isasymin}\ leq{\isacharparenright}{\kern0pt}{\isacharparenright}{\kern0pt}{\isacharparenright}{\kern0pt}\isanewline
\ \ \ \ \ {\isasymlongleftrightarrow}\ sats{\isacharparenleft}{\kern0pt}M{\isacharcomma}{\kern0pt}{\isacharquery}{\kern0pt}fm{\isacharparenleft}{\kern0pt}{\isadigit{0}}{\isacharcomma}{\kern0pt}{\isadigit{1}}{\isacharcomma}{\kern0pt}{\isadigit{2}}{\isacharparenright}{\kern0pt}{\isacharcomma}{\kern0pt}env{\isacharparenright}{\kern0pt}{\isachardoublequoteclose}\isanewline
%
\isadelimproof
\ \ %
\endisadelimproof
%
\isatagproof
\isacommand{by}\isamarkupfalse%
\ {\isacharparenleft}{\kern0pt}insert\ assms\ {\isacharsemicolon}{\kern0pt}\ {\isacharparenleft}{\kern0pt}rule\ sep{\isacharunderscore}{\kern0pt}rules\ {\isacharbar}{\kern0pt}\ simp{\isacharparenright}{\kern0pt}{\isacharplus}{\kern0pt}{\isacharparenright}{\kern0pt}%
\endisatagproof
{\isafoldproof}%
%
\isadelimproof
\isanewline
%
\endisadelimproof
\isanewline
\isacommand{definition}\isamarkupfalse%
\ is{\isacharunderscore}{\kern0pt}P{\isacharunderscore}{\kern0pt}auto{\isacharunderscore}{\kern0pt}fm\ \isakeyword{where}\ \isanewline
\ \ {\isachardoublequoteopen}is{\isacharunderscore}{\kern0pt}P{\isacharunderscore}{\kern0pt}auto{\isacharunderscore}{\kern0pt}fm\ {\isasymequiv}\ \ \isanewline
\ \ \ \ \ \ \ \ \ \ And{\isacharparenleft}{\kern0pt}bijection{\isacharunderscore}{\kern0pt}fm{\isacharparenleft}{\kern0pt}{\isadigit{1}}{\isacharcomma}{\kern0pt}\ {\isadigit{1}}{\isacharcomma}{\kern0pt}\ {\isadigit{0}}{\isacharparenright}{\kern0pt}{\isacharcomma}{\kern0pt}\isanewline
\ \ \ \ \ \ \ \ \ \ \ \ \ \ \ \ Forall\isanewline
\ \ \ \ \ \ \ \ \ \ \ \ \ \ \ \ \ {\isacharparenleft}{\kern0pt}Forall\isanewline
\ \ \ \ \ \ \ \ \ \ \ \ \ \ \ \ \ \ \ {\isacharparenleft}{\kern0pt}Forall\isanewline
\ \ \ \ \ \ \ \ \ \ \ \ \ \ \ \ \ \ \ \ \ {\isacharparenleft}{\kern0pt}Forall\isanewline
\ \ \ \ \ \ \ \ \ \ \ \ \ \ \ \ \ \ \ \ \ \ \ {\isacharparenleft}{\kern0pt}Forall\isanewline
\ \ \ \ \ \ \ \ \ \ \ \ \ \ \ \ \ \ \ \ \ \ \ \ \ {\isacharparenleft}{\kern0pt}Forall\isanewline
\ \ \ \ \ \ \ \ \ \ \ \ \ \ \ \ \ \ \ \ \ \ \ \ \ \ \ {\isacharparenleft}{\kern0pt}Implies\isanewline
\ \ \ \ \ \ \ \ \ \ \ \ \ \ \ \ \ \ \ \ \ \ \ \ \ \ \ \ \ {\isacharparenleft}{\kern0pt}Member{\isacharparenleft}{\kern0pt}{\isadigit{5}}{\isacharcomma}{\kern0pt}\ {\isadigit{7}}{\isacharparenright}{\kern0pt}{\isacharcomma}{\kern0pt}\isanewline
\ \ \ \ \ \ \ \ \ \ \ \ \ \ \ \ \ \ \ \ \ \ \ \ \ \ \ \ \ \ Implies\isanewline
\ \ \ \ \ \ \ \ \ \ \ \ \ \ \ \ \ \ \ \ \ \ \ \ \ \ \ \ \ \ \ {\isacharparenleft}{\kern0pt}Member{\isacharparenleft}{\kern0pt}{\isadigit{4}}{\isacharcomma}{\kern0pt}\ {\isadigit{7}}{\isacharparenright}{\kern0pt}{\isacharcomma}{\kern0pt}\isanewline
\ \ \ \ \ \ \ \ \ \ \ \ \ \ \ \ \ \ \ \ \ \ \ \ \ \ \ \ \ \ \ \ Implies\isanewline
\ \ \ \ \ \ \ \ \ \ \ \ \ \ \ \ \ \ \ \ \ \ \ \ \ \ \ \ \ \ \ \ \ {\isacharparenleft}{\kern0pt}fun{\isacharunderscore}{\kern0pt}apply{\isacharunderscore}{\kern0pt}fm{\isacharparenleft}{\kern0pt}{\isadigit{6}}{\isacharcomma}{\kern0pt}\ {\isadigit{5}}{\isacharcomma}{\kern0pt}\ {\isadigit{2}}{\isacharparenright}{\kern0pt}{\isacharcomma}{\kern0pt}\isanewline
\ \ \ \ \ \ \ \ \ \ \ \ \ \ \ \ \ \ \ \ \ \ \ \ \ \ \ \ \ \ \ \ \ \ Implies\isanewline
\ \ \ \ \ \ \ \ \ \ \ \ \ \ \ \ \ \ \ \ \ \ \ \ \ \ \ \ \ \ \ \ \ \ \ {\isacharparenleft}{\kern0pt}fun{\isacharunderscore}{\kern0pt}apply{\isacharunderscore}{\kern0pt}fm{\isacharparenleft}{\kern0pt}{\isadigit{6}}{\isacharcomma}{\kern0pt}\ {\isadigit{4}}{\isacharcomma}{\kern0pt}\ {\isadigit{1}}{\isacharparenright}{\kern0pt}{\isacharcomma}{\kern0pt}\isanewline
\ \ \ \ \ \ \ \ \ \ \ \ \ \ \ \ \ \ \ \ \ \ \ \ \ \ \ \ \ \ \ \ \ \ \ \ Implies{\isacharparenleft}{\kern0pt}pair{\isacharunderscore}{\kern0pt}fm{\isacharparenleft}{\kern0pt}{\isadigit{5}}{\isacharcomma}{\kern0pt}\ {\isadigit{4}}{\isacharcomma}{\kern0pt}\ {\isadigit{3}}{\isacharparenright}{\kern0pt}{\isacharcomma}{\kern0pt}\ Implies{\isacharparenleft}{\kern0pt}pair{\isacharunderscore}{\kern0pt}fm{\isacharparenleft}{\kern0pt}{\isadigit{2}}{\isacharcomma}{\kern0pt}\ {\isadigit{1}}{\isacharcomma}{\kern0pt}\ {\isadigit{0}}{\isacharparenright}{\kern0pt}{\isacharcomma}{\kern0pt}\ Iff{\isacharparenleft}{\kern0pt}Member{\isacharparenleft}{\kern0pt}{\isadigit{3}}{\isacharcomma}{\kern0pt}\ {\isadigit{8}}{\isacharparenright}{\kern0pt}{\isacharcomma}{\kern0pt}\ Member{\isacharparenleft}{\kern0pt}{\isadigit{0}}{\isacharcomma}{\kern0pt}\ {\isadigit{8}}{\isacharparenright}{\kern0pt}{\isacharparenright}{\kern0pt}{\isacharparenright}{\kern0pt}{\isacharparenright}{\kern0pt}{\isacharparenright}{\kern0pt}{\isacharparenright}{\kern0pt}{\isacharparenright}{\kern0pt}{\isacharparenright}{\kern0pt}{\isacharparenright}{\kern0pt}{\isacharparenright}{\kern0pt}{\isacharparenright}{\kern0pt}{\isacharparenright}{\kern0pt}{\isacharparenright}{\kern0pt}{\isacharparenright}{\kern0pt}{\isacharparenright}{\kern0pt}\ \ {\isachardoublequoteclose}\ \isanewline
\isanewline
\isacommand{context}\isamarkupfalse%
\ forcing{\isacharunderscore}{\kern0pt}data{\isacharunderscore}{\kern0pt}partial\ \isanewline
\isakeyword{begin}\ \isanewline
\isanewline
\isacommand{lemma}\isamarkupfalse%
\ is{\isacharunderscore}{\kern0pt}P{\isacharunderscore}{\kern0pt}auto{\isacharunderscore}{\kern0pt}fm{\isacharunderscore}{\kern0pt}sats{\isacharunderscore}{\kern0pt}iff\ {\isacharcolon}{\kern0pt}\ \isanewline
\ \ {\isachardoublequoteopen}{\isasympi}\ {\isasymin}\ M\ {\isasymLongrightarrow}\ sats{\isacharparenleft}{\kern0pt}M{\isacharcomma}{\kern0pt}\ is{\isacharunderscore}{\kern0pt}P{\isacharunderscore}{\kern0pt}auto{\isacharunderscore}{\kern0pt}fm{\isacharcomma}{\kern0pt}\ {\isacharbrackleft}{\kern0pt}{\isasympi}{\isacharcomma}{\kern0pt}\ P{\isacharcomma}{\kern0pt}\ leq{\isacharbrackright}{\kern0pt}{\isacharparenright}{\kern0pt}\ {\isasymlongleftrightarrow}\ is{\isacharunderscore}{\kern0pt}P{\isacharunderscore}{\kern0pt}auto{\isacharparenleft}{\kern0pt}{\isasympi}{\isacharparenright}{\kern0pt}{\isachardoublequoteclose}\ \isanewline
%
\isadelimproof
\isanewline
\ \ %
\endisadelimproof
%
\isatagproof
\isacommand{unfolding}\isamarkupfalse%
\ is{\isacharunderscore}{\kern0pt}P{\isacharunderscore}{\kern0pt}auto{\isacharunderscore}{\kern0pt}def\isanewline
\ \ \isacommand{apply}\isamarkupfalse%
{\isacharparenleft}{\kern0pt}rule{\isacharunderscore}{\kern0pt}tac\ Q{\isacharequal}{\kern0pt}{\isachardoublequoteopen}bijection{\isacharparenleft}{\kern0pt}{\isacharhash}{\kern0pt}{\isacharhash}{\kern0pt}M{\isacharcomma}{\kern0pt}\ P{\isacharcomma}{\kern0pt}\ P{\isacharcomma}{\kern0pt}\ {\isasympi}{\isacharparenright}{\kern0pt}\ {\isasymand}\ {\isacharparenleft}{\kern0pt}{\isasymforall}p\ {\isasymin}\ M{\isachardot}{\kern0pt}\ {\isasymforall}q\ {\isasymin}\ M{\isachardot}{\kern0pt}\ {\isasymforall}p{\isacharunderscore}{\kern0pt}q\ {\isasymin}\ M{\isachardot}{\kern0pt}\ {\isasymforall}{\isasympi}p\ {\isasymin}\ M{\isachardot}{\kern0pt}\ {\isasymforall}{\isasympi}q\ {\isasymin}\ M{\isachardot}{\kern0pt}\ {\isasymforall}{\isasympi}p{\isacharunderscore}{\kern0pt}{\isasympi}q\ {\isasymin}\ M{\isachardot}{\kern0pt}\ \isanewline
\ \ \ \ \ \ \ \ p\ {\isasymin}\ P\ {\isasymlongrightarrow}\ q\ {\isasymin}\ P\ {\isasymlongrightarrow}\ fun{\isacharunderscore}{\kern0pt}apply{\isacharparenleft}{\kern0pt}{\isacharhash}{\kern0pt}{\isacharhash}{\kern0pt}M{\isacharcomma}{\kern0pt}\ {\isasympi}{\isacharcomma}{\kern0pt}\ p{\isacharcomma}{\kern0pt}\ {\isasympi}p{\isacharparenright}{\kern0pt}\ {\isasymlongrightarrow}\ fun{\isacharunderscore}{\kern0pt}apply{\isacharparenleft}{\kern0pt}{\isacharhash}{\kern0pt}{\isacharhash}{\kern0pt}M{\isacharcomma}{\kern0pt}\ {\isasympi}{\isacharcomma}{\kern0pt}\ q{\isacharcomma}{\kern0pt}\ {\isasympi}q{\isacharparenright}{\kern0pt}\ {\isasymlongrightarrow}\isanewline
\ \ \ \ \ \ \ \ pair{\isacharparenleft}{\kern0pt}{\isacharhash}{\kern0pt}{\isacharhash}{\kern0pt}M{\isacharcomma}{\kern0pt}\ p{\isacharcomma}{\kern0pt}\ q{\isacharcomma}{\kern0pt}\ p{\isacharunderscore}{\kern0pt}q{\isacharparenright}{\kern0pt}\ {\isasymlongrightarrow}\ pair{\isacharparenleft}{\kern0pt}{\isacharhash}{\kern0pt}{\isacharhash}{\kern0pt}M{\isacharcomma}{\kern0pt}\ {\isasympi}p{\isacharcomma}{\kern0pt}\ {\isasympi}q{\isacharcomma}{\kern0pt}\ {\isasympi}p{\isacharunderscore}{\kern0pt}{\isasympi}q{\isacharparenright}{\kern0pt}\ {\isasymlongrightarrow}\ {\isacharparenleft}{\kern0pt}p{\isacharunderscore}{\kern0pt}q\ {\isasymin}\ leq\ \ {\isasymlongleftrightarrow}\ {\isasympi}p{\isacharunderscore}{\kern0pt}{\isasympi}q\ {\isasymin}\ leq{\isacharparenright}{\kern0pt}{\isacharparenright}{\kern0pt}{\isachardoublequoteclose}\ \isakeyword{in}\ iff{\isacharunderscore}{\kern0pt}trans{\isacharparenright}{\kern0pt}\ \isanewline
\ \ \ \isacommand{apply}\isamarkupfalse%
{\isacharparenleft}{\kern0pt}rule\ iff{\isacharunderscore}{\kern0pt}flip{\isacharparenright}{\kern0pt}\ \isanewline
\ \ \isacommand{unfolding}\isamarkupfalse%
\ is{\isacharunderscore}{\kern0pt}P{\isacharunderscore}{\kern0pt}auto{\isacharunderscore}{\kern0pt}fm{\isacharunderscore}{\kern0pt}def\ \isanewline
\ \ \ \isacommand{apply}\isamarkupfalse%
{\isacharparenleft}{\kern0pt}rule{\isacharunderscore}{\kern0pt}tac\ is{\isacharunderscore}{\kern0pt}P{\isacharunderscore}{\kern0pt}auto{\isacharunderscore}{\kern0pt}fm{\isacharunderscore}{\kern0pt}auto{\isacharparenright}{\kern0pt}\ \isanewline
\ \ \isacommand{using}\isamarkupfalse%
\ P{\isacharunderscore}{\kern0pt}in{\isacharunderscore}{\kern0pt}M\ leq{\isacharunderscore}{\kern0pt}in{\isacharunderscore}{\kern0pt}M\ \isanewline
\ \ \ \ \ \ \isacommand{apply}\isamarkupfalse%
\ simp{\isacharunderscore}{\kern0pt}all\ \isanewline
\ \ \isacommand{apply}\isamarkupfalse%
{\isacharparenleft}{\kern0pt}rule{\isacharunderscore}{\kern0pt}tac\ iffI{\isacharparenright}{\kern0pt}\ \isanewline
\ \ \ \isacommand{apply}\isamarkupfalse%
\ clarify\ \isanewline
\ \ \isacommand{apply}\isamarkupfalse%
{\isacharparenleft}{\kern0pt}rename{\isacharunderscore}{\kern0pt}tac\ p\ q{\isacharparenright}{\kern0pt}\isanewline
\ \ \ \isacommand{apply}\isamarkupfalse%
{\isacharparenleft}{\kern0pt}rule{\isacharunderscore}{\kern0pt}tac\ P{\isacharequal}{\kern0pt}{\isachardoublequoteopen}{\isasympi}\ {\isacharbackquote}{\kern0pt}\ p\ {\isasymin}\ M{\isachardoublequoteclose}\ \isakeyword{in}\ mp{\isacharparenright}{\kern0pt}\ \isanewline
\ \ \ \ \isacommand{apply}\isamarkupfalse%
{\isacharparenleft}{\kern0pt}rule{\isacharunderscore}{\kern0pt}tac\ P{\isacharequal}{\kern0pt}{\isachardoublequoteopen}{\isasympi}\ {\isacharbackquote}{\kern0pt}\ q\ {\isasymin}\ M{\isachardoublequoteclose}\ \isakeyword{in}\ mp{\isacharparenright}{\kern0pt}\ \isanewline
\ \ \ \ \ \isacommand{apply}\isamarkupfalse%
\ clarify\ \isanewline
\ \ \isacommand{using}\isamarkupfalse%
\ pair{\isacharunderscore}{\kern0pt}in{\isacharunderscore}{\kern0pt}M{\isacharunderscore}{\kern0pt}iff\ P{\isacharunderscore}{\kern0pt}in{\isacharunderscore}{\kern0pt}M\ transM\ \isanewline
\ \ \ \ \ \isacommand{apply}\isamarkupfalse%
\ auto\ \isanewline
\ \ \isacommand{using}\isamarkupfalse%
\ apply{\isacharunderscore}{\kern0pt}closed\ \isanewline
\ \ \isacommand{by}\isamarkupfalse%
\ auto%
\endisatagproof
{\isafoldproof}%
%
\isadelimproof
\isanewline
%
\endisadelimproof
\isanewline
\isacommand{lemma}\isamarkupfalse%
\ P{\isacharunderscore}{\kern0pt}auto{\isacharunderscore}{\kern0pt}in{\isacharunderscore}{\kern0pt}M\ {\isacharcolon}{\kern0pt}\ {\isachardoublequoteopen}P{\isacharunderscore}{\kern0pt}auto\ {\isasymin}\ M{\isachardoublequoteclose}\ \isanewline
%
\isadelimproof
%
\endisadelimproof
%
\isatagproof
\isacommand{proof}\isamarkupfalse%
\ {\isacharminus}{\kern0pt}\ \isanewline
\ \ \isacommand{have}\isamarkupfalse%
\ {\isachardoublequoteopen}separation{\isacharparenleft}{\kern0pt}{\isacharhash}{\kern0pt}{\isacharhash}{\kern0pt}M{\isacharcomma}{\kern0pt}\ {\isasymlambda}{\isasympi}{\isachardot}{\kern0pt}\ sats{\isacharparenleft}{\kern0pt}M{\isacharcomma}{\kern0pt}\ is{\isacharunderscore}{\kern0pt}P{\isacharunderscore}{\kern0pt}auto{\isacharunderscore}{\kern0pt}fm{\isacharcomma}{\kern0pt}\ {\isacharbrackleft}{\kern0pt}{\isasympi}{\isacharbrackright}{\kern0pt}{\isacharat}{\kern0pt}{\isacharbrackleft}{\kern0pt}P{\isacharcomma}{\kern0pt}\ leq{\isacharbrackright}{\kern0pt}{\isacharparenright}{\kern0pt}{\isacharparenright}{\kern0pt}{\isachardoublequoteclose}\ \isanewline
\ \ \ \ \isacommand{apply}\isamarkupfalse%
{\isacharparenleft}{\kern0pt}rule{\isacharunderscore}{\kern0pt}tac\ separation{\isacharunderscore}{\kern0pt}ax{\isacharparenright}{\kern0pt}\ \isanewline
\ \ \ \ \isacommand{unfolding}\isamarkupfalse%
\ is{\isacharunderscore}{\kern0pt}P{\isacharunderscore}{\kern0pt}auto{\isacharunderscore}{\kern0pt}fm{\isacharunderscore}{\kern0pt}def\isanewline
\ \ \ \ \isacommand{using}\isamarkupfalse%
\ P{\isacharunderscore}{\kern0pt}in{\isacharunderscore}{\kern0pt}M\ leq{\isacharunderscore}{\kern0pt}in{\isacharunderscore}{\kern0pt}M\ \isanewline
\ \ \ \ \ \ \isacommand{apply}\isamarkupfalse%
\ simp{\isacharunderscore}{\kern0pt}all\ \isanewline
\ \ \ \ \isacommand{unfolding}\isamarkupfalse%
\ bijection{\isacharunderscore}{\kern0pt}fm{\isacharunderscore}{\kern0pt}def\ injection{\isacharunderscore}{\kern0pt}fm{\isacharunderscore}{\kern0pt}def\ surjection{\isacharunderscore}{\kern0pt}fm{\isacharunderscore}{\kern0pt}def\ \isanewline
\ \ \ \ \isacommand{by}\isamarkupfalse%
\ {\isacharparenleft}{\kern0pt}simp\ del{\isacharcolon}{\kern0pt}FOL{\isacharunderscore}{\kern0pt}sats{\isacharunderscore}{\kern0pt}iff\ pair{\isacharunderscore}{\kern0pt}abs\ add{\isacharcolon}{\kern0pt}\ fm{\isacharunderscore}{\kern0pt}defs\ nat{\isacharunderscore}{\kern0pt}simp{\isacharunderscore}{\kern0pt}union{\isacharparenright}{\kern0pt}\ \ \isanewline
\ \ \isacommand{then}\isamarkupfalse%
\ \isacommand{have}\isamarkupfalse%
\ {\isachardoublequoteopen}separation{\isacharparenleft}{\kern0pt}{\isacharhash}{\kern0pt}{\isacharhash}{\kern0pt}M{\isacharcomma}{\kern0pt}\ {\isasymlambda}{\isasympi}{\isachardot}{\kern0pt}\ is{\isacharunderscore}{\kern0pt}P{\isacharunderscore}{\kern0pt}auto{\isacharparenleft}{\kern0pt}{\isasympi}{\isacharparenright}{\kern0pt}{\isacharparenright}{\kern0pt}{\isachardoublequoteclose}\isanewline
\ \ \ \ \isacommand{apply}\isamarkupfalse%
{\isacharparenleft}{\kern0pt}rule{\isacharunderscore}{\kern0pt}tac\ P{\isacharequal}{\kern0pt}{\isachardoublequoteopen}separation{\isacharparenleft}{\kern0pt}{\isacharhash}{\kern0pt}{\isacharhash}{\kern0pt}M{\isacharcomma}{\kern0pt}\ {\isasymlambda}{\isasympi}{\isachardot}{\kern0pt}\ sats{\isacharparenleft}{\kern0pt}M{\isacharcomma}{\kern0pt}\ is{\isacharunderscore}{\kern0pt}P{\isacharunderscore}{\kern0pt}auto{\isacharunderscore}{\kern0pt}fm{\isacharcomma}{\kern0pt}\ {\isacharbrackleft}{\kern0pt}{\isasympi}{\isacharbrackright}{\kern0pt}{\isacharat}{\kern0pt}{\isacharbrackleft}{\kern0pt}P{\isacharcomma}{\kern0pt}\ leq{\isacharbrackright}{\kern0pt}{\isacharparenright}{\kern0pt}{\isacharparenright}{\kern0pt}{\isachardoublequoteclose}\ \isakeyword{in}\ iffD{\isadigit{1}}{\isacharparenright}{\kern0pt}\isanewline
\ \ \ \ \ \isacommand{apply}\isamarkupfalse%
{\isacharparenleft}{\kern0pt}rule{\isacharunderscore}{\kern0pt}tac\ separation{\isacharunderscore}{\kern0pt}cong{\isacharparenright}{\kern0pt}\ \isanewline
\ \ \ \ \isacommand{using}\isamarkupfalse%
\ is{\isacharunderscore}{\kern0pt}P{\isacharunderscore}{\kern0pt}auto{\isacharunderscore}{\kern0pt}fm{\isacharunderscore}{\kern0pt}sats{\isacharunderscore}{\kern0pt}iff\ \isanewline
\ \ \ \ \isacommand{by}\isamarkupfalse%
\ auto\isanewline
\ \ \isacommand{then}\isamarkupfalse%
\ \isacommand{have}\isamarkupfalse%
\ {\isachardoublequoteopen}{\isacharbraceleft}{\kern0pt}\ {\isasympi}\ {\isasymin}\ Pow{\isacharparenleft}{\kern0pt}P\ {\isasymtimes}\ P{\isacharparenright}{\kern0pt}\ {\isasyminter}\ M{\isachardot}{\kern0pt}\ is{\isacharunderscore}{\kern0pt}P{\isacharunderscore}{\kern0pt}auto{\isacharparenleft}{\kern0pt}{\isasympi}{\isacharparenright}{\kern0pt}\ {\isacharbraceright}{\kern0pt}\ {\isasymin}\ M{\isachardoublequoteclose}\ \isanewline
\ \ \ \ \isacommand{apply}\isamarkupfalse%
{\isacharparenleft}{\kern0pt}rule{\isacharunderscore}{\kern0pt}tac\ separation{\isacharunderscore}{\kern0pt}notation{\isacharparenright}{\kern0pt}\ \isanewline
\ \ \ \ \ \isacommand{apply}\isamarkupfalse%
\ simp\ \isanewline
\ \ \ \ \isacommand{apply}\isamarkupfalse%
{\isacharparenleft}{\kern0pt}rule{\isacharunderscore}{\kern0pt}tac\ M{\isacharunderscore}{\kern0pt}powerset{\isacharparenright}{\kern0pt}\ \isanewline
\ \ \ \ \isacommand{using}\isamarkupfalse%
\ P{\isacharunderscore}{\kern0pt}in{\isacharunderscore}{\kern0pt}M\ cartprod{\isacharunderscore}{\kern0pt}closed\ \isanewline
\ \ \ \ \isacommand{by}\isamarkupfalse%
\ auto\ \isanewline
\ \ \isacommand{then}\isamarkupfalse%
\ \isacommand{show}\isamarkupfalse%
\ {\isachardoublequoteopen}P{\isacharunderscore}{\kern0pt}auto\ {\isasymin}\ M{\isachardoublequoteclose}\ \isanewline
\ \ \ \ \isacommand{apply}\isamarkupfalse%
{\isacharparenleft}{\kern0pt}rule{\isacharunderscore}{\kern0pt}tac\ b{\isacharequal}{\kern0pt}P{\isacharunderscore}{\kern0pt}auto\ \isakeyword{and}\ a\ {\isacharequal}{\kern0pt}\ {\isachardoublequoteopen}{\isacharbraceleft}{\kern0pt}\ {\isasympi}\ {\isasymin}\ Pow{\isacharparenleft}{\kern0pt}P\ {\isasymtimes}\ P{\isacharparenright}{\kern0pt}\ {\isasyminter}\ M{\isachardot}{\kern0pt}\ is{\isacharunderscore}{\kern0pt}P{\isacharunderscore}{\kern0pt}auto{\isacharparenleft}{\kern0pt}{\isasympi}{\isacharparenright}{\kern0pt}\ {\isacharbraceright}{\kern0pt}\ {\isachardoublequoteclose}\ \isakeyword{in}\ ssubst{\isacharparenright}{\kern0pt}\ \isanewline
\ \ \ \ \ \isacommand{apply}\isamarkupfalse%
{\isacharparenleft}{\kern0pt}rule\ equality{\isacharunderscore}{\kern0pt}iffI{\isacharsemicolon}{\kern0pt}\ rule\ iffI{\isacharparenright}{\kern0pt}\ \isanewline
\ \ \ \ \isacommand{unfolding}\isamarkupfalse%
\ P{\isacharunderscore}{\kern0pt}auto{\isacharunderscore}{\kern0pt}def\ \isanewline
\ \ \ \ \ \ \isacommand{apply}\isamarkupfalse%
\ auto\ \isanewline
\ \ \ \ \isacommand{using}\isamarkupfalse%
\ Pi{\isacharunderscore}{\kern0pt}iff\ \isanewline
\ \ \ \ \ \ \isacommand{apply}\isamarkupfalse%
\ force\ \isanewline
\ \ \ \ \isacommand{using}\isamarkupfalse%
\ P{\isacharunderscore}{\kern0pt}auto{\isacharunderscore}{\kern0pt}type\ \isanewline
\ \ \ \ \ \isacommand{apply}\isamarkupfalse%
\ auto\ \isanewline
\ \ \ \ \isacommand{unfolding}\isamarkupfalse%
\ is{\isacharunderscore}{\kern0pt}P{\isacharunderscore}{\kern0pt}auto{\isacharunderscore}{\kern0pt}def\ \isanewline
\ \ \ \ \isacommand{by}\isamarkupfalse%
\ auto\isanewline
\isacommand{qed}\isamarkupfalse%
%
\endisatagproof
{\isafoldproof}%
%
\isadelimproof
\isanewline
%
\endisadelimproof
\isanewline
\isacommand{definition}\isamarkupfalse%
\ HPn{\isacharunderscore}{\kern0pt}auto{\isacharunderscore}{\kern0pt}M{\isacharunderscore}{\kern0pt}cond\ \isakeyword{where}\ \isanewline
\ \ {\isachardoublequoteopen}HPn{\isacharunderscore}{\kern0pt}auto{\isacharunderscore}{\kern0pt}M{\isacharunderscore}{\kern0pt}cond{\isacharparenleft}{\kern0pt}x{\isacharunderscore}{\kern0pt}pi{\isacharcomma}{\kern0pt}\ g{\isacharcomma}{\kern0pt}\ elem{\isacharparenright}{\kern0pt}\ {\isasymequiv}\ \isanewline
\ \ \ \ \ {\isacharparenleft}{\kern0pt}{\isasymexists}g{\isacharunderscore}{\kern0pt}app{\isacharunderscore}{\kern0pt}y{\isacharunderscore}{\kern0pt}pi\ {\isasymin}\ M{\isachardot}{\kern0pt}\ {\isasymexists}pi{\isacharunderscore}{\kern0pt}app{\isacharunderscore}{\kern0pt}p\ {\isasymin}\ M{\isachardot}{\kern0pt}\ {\isasymexists}y{\isacharunderscore}{\kern0pt}p\ {\isasymin}\ M{\isachardot}{\kern0pt}\ {\isasymexists}y{\isacharunderscore}{\kern0pt}pi\ {\isasymin}\ M{\isachardot}{\kern0pt}\ \isanewline
\ \ \ \ \ \ \ {\isasymexists}y\ {\isasymin}\ M{\isachardot}{\kern0pt}\ {\isasymexists}p\ {\isasymin}\ M{\isachardot}{\kern0pt}\ {\isasymexists}x\ {\isasymin}\ M{\isachardot}{\kern0pt}\ {\isasymexists}pi\ {\isasymin}\ M{\isachardot}{\kern0pt}\ \isanewline
\ \ \ \ \ \ \ \ pair{\isacharparenleft}{\kern0pt}{\isacharhash}{\kern0pt}{\isacharhash}{\kern0pt}M{\isacharcomma}{\kern0pt}\ g{\isacharunderscore}{\kern0pt}app{\isacharunderscore}{\kern0pt}y{\isacharunderscore}{\kern0pt}pi{\isacharcomma}{\kern0pt}\ pi{\isacharunderscore}{\kern0pt}app{\isacharunderscore}{\kern0pt}p{\isacharcomma}{\kern0pt}\ elem{\isacharparenright}{\kern0pt}\ {\isasymand}\ \isanewline
\ \ \ \ \ \ \ \ pair{\isacharparenleft}{\kern0pt}{\isacharhash}{\kern0pt}{\isacharhash}{\kern0pt}M{\isacharcomma}{\kern0pt}\ x{\isacharcomma}{\kern0pt}\ pi{\isacharcomma}{\kern0pt}\ x{\isacharunderscore}{\kern0pt}pi{\isacharparenright}{\kern0pt}\ {\isasymand}\ \isanewline
\ \ \ \ \ \ \ \ pair{\isacharparenleft}{\kern0pt}{\isacharhash}{\kern0pt}{\isacharhash}{\kern0pt}M{\isacharcomma}{\kern0pt}\ y{\isacharcomma}{\kern0pt}\ pi{\isacharcomma}{\kern0pt}\ y{\isacharunderscore}{\kern0pt}pi{\isacharparenright}{\kern0pt}\ {\isasymand}\ \isanewline
\ \ \ \ \ \ \ \ pair{\isacharparenleft}{\kern0pt}{\isacharhash}{\kern0pt}{\isacharhash}{\kern0pt}M{\isacharcomma}{\kern0pt}\ y{\isacharcomma}{\kern0pt}\ p{\isacharcomma}{\kern0pt}\ y{\isacharunderscore}{\kern0pt}p{\isacharparenright}{\kern0pt}\ {\isasymand}\ \isanewline
\ \ \ \ \ \ \ \ y{\isacharunderscore}{\kern0pt}p\ {\isasymin}\ x\ {\isasymand}\ \isanewline
\ \ \ \ \ \ \ \ is{\isacharunderscore}{\kern0pt}function{\isacharparenleft}{\kern0pt}{\isacharhash}{\kern0pt}{\isacharhash}{\kern0pt}M{\isacharcomma}{\kern0pt}\ pi{\isacharparenright}{\kern0pt}\ {\isasymand}\isanewline
\ \ \ \ \ \ \ \ fun{\isacharunderscore}{\kern0pt}apply{\isacharparenleft}{\kern0pt}{\isacharhash}{\kern0pt}{\isacharhash}{\kern0pt}M{\isacharcomma}{\kern0pt}\ pi{\isacharcomma}{\kern0pt}\ p{\isacharcomma}{\kern0pt}\ pi{\isacharunderscore}{\kern0pt}app{\isacharunderscore}{\kern0pt}p{\isacharparenright}{\kern0pt}\ {\isasymand}\ \isanewline
\ \ \ \ \ \ \ \ fun{\isacharunderscore}{\kern0pt}apply{\isacharparenleft}{\kern0pt}{\isacharhash}{\kern0pt}{\isacharhash}{\kern0pt}M{\isacharcomma}{\kern0pt}\ g{\isacharcomma}{\kern0pt}\ y{\isacharunderscore}{\kern0pt}pi{\isacharcomma}{\kern0pt}\ g{\isacharunderscore}{\kern0pt}app{\isacharunderscore}{\kern0pt}y{\isacharunderscore}{\kern0pt}pi{\isacharparenright}{\kern0pt}\isanewline
\ \ \ \ \ {\isacharparenright}{\kern0pt}{\isachardoublequoteclose}\isanewline
\isanewline
\isacommand{lemma}\isamarkupfalse%
\ HPn{\isacharunderscore}{\kern0pt}auto{\isacharunderscore}{\kern0pt}M{\isacharunderscore}{\kern0pt}cond{\isacharunderscore}{\kern0pt}iff\ {\isacharcolon}{\kern0pt}\ \isanewline
\ \ {\isachardoublequoteopen}x{\isacharunderscore}{\kern0pt}pi\ {\isasymin}\ M\ {\isasymLongrightarrow}\ g\ {\isasymin}\ M\ {\isasymLongrightarrow}\ elem\ {\isasymin}\ M\ {\isasymLongrightarrow}\ \isanewline
\ \ \ \ HPn{\isacharunderscore}{\kern0pt}auto{\isacharunderscore}{\kern0pt}M{\isacharunderscore}{\kern0pt}cond{\isacharparenleft}{\kern0pt}x{\isacharunderscore}{\kern0pt}pi{\isacharcomma}{\kern0pt}\ g{\isacharcomma}{\kern0pt}\ elem{\isacharparenright}{\kern0pt}\ {\isasymlongleftrightarrow}\ \isanewline
\ \ \ \ \ \ {\isacharparenleft}{\kern0pt}{\isasymexists}y\ {\isasymin}\ M{\isachardot}{\kern0pt}\ {\isasymexists}p\ {\isasymin}\ M{\isachardot}{\kern0pt}\ {\isasymexists}x\ {\isasymin}\ M{\isachardot}{\kern0pt}\ {\isasymexists}pi\ {\isasymin}\ M{\isachardot}{\kern0pt}\ \isanewline
\ \ \ \ \ \ \ \ x{\isacharunderscore}{\kern0pt}pi\ {\isacharequal}{\kern0pt}\ {\isacharless}{\kern0pt}x{\isacharcomma}{\kern0pt}\ pi{\isachargreater}{\kern0pt}\ {\isasymand}\isanewline
\ \ \ \ \ \ \ \ elem\ {\isacharequal}{\kern0pt}\ {\isacharless}{\kern0pt}g{\isacharbackquote}{\kern0pt}{\isacharless}{\kern0pt}y{\isacharcomma}{\kern0pt}pi{\isachargreater}{\kern0pt}{\isacharcomma}{\kern0pt}\ pi{\isacharbackquote}{\kern0pt}p{\isachargreater}{\kern0pt}\ {\isasymand}\ \isanewline
\ \ \ \ \ \ \ \ {\isacharless}{\kern0pt}y{\isacharcomma}{\kern0pt}\ p{\isachargreater}{\kern0pt}\ {\isasymin}\ x\ {\isasymand}\ \isanewline
\ \ \ \ \ \ \ \ function{\isacharparenleft}{\kern0pt}pi{\isacharparenright}{\kern0pt}\isanewline
\ \ \ \ \ {\isacharparenright}{\kern0pt}{\isachardoublequoteclose}\isanewline
%
\isadelimproof
%
\endisadelimproof
%
\isatagproof
\isacommand{proof}\isamarkupfalse%
\ {\isacharparenleft}{\kern0pt}rule\ iffI{\isacharparenright}{\kern0pt}\isanewline
\ \ \isacommand{assume}\isamarkupfalse%
\ {\isachardoublequoteopen}HPn{\isacharunderscore}{\kern0pt}auto{\isacharunderscore}{\kern0pt}M{\isacharunderscore}{\kern0pt}cond{\isacharparenleft}{\kern0pt}x{\isacharunderscore}{\kern0pt}pi{\isacharcomma}{\kern0pt}\ g{\isacharcomma}{\kern0pt}\ elem{\isacharparenright}{\kern0pt}{\isachardoublequoteclose}\isanewline
\ \ \isakeyword{and}\ inM\ {\isacharcolon}{\kern0pt}\ {\isachardoublequoteopen}x{\isacharunderscore}{\kern0pt}pi\ {\isasymin}\ M{\isachardoublequoteclose}\ {\isachardoublequoteopen}g\ {\isasymin}\ M{\isachardoublequoteclose}\ {\isachardoublequoteopen}elem\ {\isasymin}\ M{\isachardoublequoteclose}\isanewline
\ \ \isacommand{then}\isamarkupfalse%
\ \isacommand{obtain}\isamarkupfalse%
\ g{\isacharunderscore}{\kern0pt}app{\isacharunderscore}{\kern0pt}y{\isacharunderscore}{\kern0pt}pi\ pi{\isacharunderscore}{\kern0pt}app{\isacharunderscore}{\kern0pt}p\ y{\isacharunderscore}{\kern0pt}p\ y{\isacharunderscore}{\kern0pt}pi\ y\ p\ x\ pi\ \isakeyword{where}\ \isanewline
\ \ \ \ \ \ H\ {\isacharcolon}{\kern0pt}\ \isanewline
\ \ \ \ \ \ \ \ {\isachardoublequoteopen}g{\isacharunderscore}{\kern0pt}app{\isacharunderscore}{\kern0pt}y{\isacharunderscore}{\kern0pt}pi\ {\isasymin}\ M{\isachardoublequoteclose}\ {\isachardoublequoteopen}pi{\isacharunderscore}{\kern0pt}app{\isacharunderscore}{\kern0pt}p\ {\isasymin}\ M{\isachardoublequoteclose}\ {\isachardoublequoteopen}y{\isacharunderscore}{\kern0pt}p\ {\isasymin}\ M{\isachardoublequoteclose}\ {\isachardoublequoteopen}y{\isacharunderscore}{\kern0pt}pi\ {\isasymin}\ M{\isachardoublequoteclose}\ \isanewline
\ \ \ \ \ \ \ \ {\isachardoublequoteopen}pair{\isacharparenleft}{\kern0pt}{\isacharhash}{\kern0pt}{\isacharhash}{\kern0pt}M{\isacharcomma}{\kern0pt}\ g{\isacharunderscore}{\kern0pt}app{\isacharunderscore}{\kern0pt}y{\isacharunderscore}{\kern0pt}pi{\isacharcomma}{\kern0pt}\ pi{\isacharunderscore}{\kern0pt}app{\isacharunderscore}{\kern0pt}p{\isacharcomma}{\kern0pt}\ elem{\isacharparenright}{\kern0pt}{\isachardoublequoteclose}\isanewline
\ \ \ \ \ \ \ \ {\isachardoublequoteopen}pair{\isacharparenleft}{\kern0pt}{\isacharhash}{\kern0pt}{\isacharhash}{\kern0pt}M{\isacharcomma}{\kern0pt}\ x{\isacharcomma}{\kern0pt}\ pi{\isacharcomma}{\kern0pt}\ x{\isacharunderscore}{\kern0pt}pi{\isacharparenright}{\kern0pt}{\isachardoublequoteclose}\ \isanewline
\ \ \ \ \ \ \ \ {\isachardoublequoteopen}pair{\isacharparenleft}{\kern0pt}{\isacharhash}{\kern0pt}{\isacharhash}{\kern0pt}M{\isacharcomma}{\kern0pt}\ y{\isacharcomma}{\kern0pt}\ pi{\isacharcomma}{\kern0pt}\ y{\isacharunderscore}{\kern0pt}pi{\isacharparenright}{\kern0pt}{\isachardoublequoteclose}\ \isanewline
\ \ \ \ \ \ \ \ {\isachardoublequoteopen}pair{\isacharparenleft}{\kern0pt}{\isacharhash}{\kern0pt}{\isacharhash}{\kern0pt}M{\isacharcomma}{\kern0pt}\ y{\isacharcomma}{\kern0pt}\ p{\isacharcomma}{\kern0pt}\ y{\isacharunderscore}{\kern0pt}p{\isacharparenright}{\kern0pt}{\isachardoublequoteclose}\ \isanewline
\ \ \ \ \ \ \ \ {\isachardoublequoteopen}y{\isacharunderscore}{\kern0pt}p\ {\isasymin}\ x{\isachardoublequoteclose}\ \isanewline
\ \ \ \ \ \ \ \ {\isachardoublequoteopen}is{\isacharunderscore}{\kern0pt}function{\isacharparenleft}{\kern0pt}{\isacharhash}{\kern0pt}{\isacharhash}{\kern0pt}M{\isacharcomma}{\kern0pt}\ pi{\isacharparenright}{\kern0pt}{\isachardoublequoteclose}\isanewline
\ \ \ \ \ \ \ \ {\isachardoublequoteopen}fun{\isacharunderscore}{\kern0pt}apply{\isacharparenleft}{\kern0pt}{\isacharhash}{\kern0pt}{\isacharhash}{\kern0pt}M{\isacharcomma}{\kern0pt}\ pi{\isacharcomma}{\kern0pt}\ p{\isacharcomma}{\kern0pt}\ pi{\isacharunderscore}{\kern0pt}app{\isacharunderscore}{\kern0pt}p{\isacharparenright}{\kern0pt}{\isachardoublequoteclose}\ \isanewline
\ \ \ \ \ \ \ \ {\isachardoublequoteopen}fun{\isacharunderscore}{\kern0pt}apply{\isacharparenleft}{\kern0pt}{\isacharhash}{\kern0pt}{\isacharhash}{\kern0pt}M{\isacharcomma}{\kern0pt}\ g{\isacharcomma}{\kern0pt}\ y{\isacharunderscore}{\kern0pt}pi{\isacharcomma}{\kern0pt}\ g{\isacharunderscore}{\kern0pt}app{\isacharunderscore}{\kern0pt}y{\isacharunderscore}{\kern0pt}pi{\isacharparenright}{\kern0pt}{\isachardoublequoteclose}\isanewline
\ \ \ \ \isacommand{unfolding}\isamarkupfalse%
\ HPn{\isacharunderscore}{\kern0pt}auto{\isacharunderscore}{\kern0pt}M{\isacharunderscore}{\kern0pt}cond{\isacharunderscore}{\kern0pt}def\ \isacommand{by}\isamarkupfalse%
\ blast\isanewline
\ \ \isacommand{have}\isamarkupfalse%
\ elemeq{\isacharcolon}{\kern0pt}\ {\isachardoublequoteopen}elem\ {\isacharequal}{\kern0pt}\ {\isacharless}{\kern0pt}g{\isacharunderscore}{\kern0pt}app{\isacharunderscore}{\kern0pt}y{\isacharunderscore}{\kern0pt}pi{\isacharcomma}{\kern0pt}\ pi{\isacharunderscore}{\kern0pt}app{\isacharunderscore}{\kern0pt}p{\isachargreater}{\kern0pt}{\isachardoublequoteclose}\ \isacommand{using}\isamarkupfalse%
\ H\ pair{\isacharunderscore}{\kern0pt}abs\ inM\ \isacommand{by}\isamarkupfalse%
\ auto\ \isanewline
\ \ \isacommand{have}\isamarkupfalse%
\ x{\isacharunderscore}{\kern0pt}pi{\isacharunderscore}{\kern0pt}eq\ {\isacharcolon}{\kern0pt}{\isachardoublequoteopen}x{\isacharunderscore}{\kern0pt}pi\ {\isacharequal}{\kern0pt}\ {\isacharless}{\kern0pt}x{\isacharcomma}{\kern0pt}\ pi{\isachargreater}{\kern0pt}{\isachardoublequoteclose}\ \isacommand{using}\isamarkupfalse%
\ H\ pair{\isacharunderscore}{\kern0pt}abs\ inM\ \isacommand{by}\isamarkupfalse%
\ auto\isanewline
\ \ \isacommand{have}\isamarkupfalse%
\ y{\isacharunderscore}{\kern0pt}pi{\isacharunderscore}{\kern0pt}eq\ {\isacharcolon}{\kern0pt}\ {\isachardoublequoteopen}y{\isacharunderscore}{\kern0pt}pi\ {\isacharequal}{\kern0pt}\ {\isacharless}{\kern0pt}y{\isacharcomma}{\kern0pt}\ pi{\isachargreater}{\kern0pt}{\isachardoublequoteclose}\ \isacommand{using}\isamarkupfalse%
\ H\ pair{\isacharunderscore}{\kern0pt}abs\ inM\ \isacommand{by}\isamarkupfalse%
\ auto\isanewline
\ \ \isacommand{have}\isamarkupfalse%
\ y{\isacharunderscore}{\kern0pt}p{\isacharunderscore}{\kern0pt}eq\ {\isacharcolon}{\kern0pt}\ {\isachardoublequoteopen}y{\isacharunderscore}{\kern0pt}p\ {\isacharequal}{\kern0pt}\ {\isacharless}{\kern0pt}y{\isacharcomma}{\kern0pt}\ p{\isachargreater}{\kern0pt}{\isachardoublequoteclose}\ \isacommand{using}\isamarkupfalse%
\ H\ pair{\isacharunderscore}{\kern0pt}abs\ inM\ \isacommand{by}\isamarkupfalse%
\ auto\isanewline
\ \ \isacommand{have}\isamarkupfalse%
\ xin\ {\isacharcolon}{\kern0pt}\ {\isachardoublequoteopen}x\ {\isasymin}\ M{\isachardoublequoteclose}\ \isacommand{using}\isamarkupfalse%
\ x{\isacharunderscore}{\kern0pt}pi{\isacharunderscore}{\kern0pt}eq\ inM\ pair{\isacharunderscore}{\kern0pt}in{\isacharunderscore}{\kern0pt}M{\isacharunderscore}{\kern0pt}iff\ \isacommand{by}\isamarkupfalse%
\ auto\ \isanewline
\ \ \isacommand{have}\isamarkupfalse%
\ yin\ {\isacharcolon}{\kern0pt}\ {\isachardoublequoteopen}y\ {\isasymin}\ M{\isachardoublequoteclose}\ \isacommand{using}\isamarkupfalse%
\ y{\isacharunderscore}{\kern0pt}pi{\isacharunderscore}{\kern0pt}eq\ H\ pair{\isacharunderscore}{\kern0pt}in{\isacharunderscore}{\kern0pt}M{\isacharunderscore}{\kern0pt}iff\ \isacommand{by}\isamarkupfalse%
\ auto\ \isanewline
\ \ \isacommand{have}\isamarkupfalse%
\ pin\ {\isacharcolon}{\kern0pt}\ {\isachardoublequoteopen}p\ {\isasymin}\ M{\isachardoublequoteclose}\ \isacommand{using}\isamarkupfalse%
\ y{\isacharunderscore}{\kern0pt}pi{\isacharunderscore}{\kern0pt}eq\ H\ pair{\isacharunderscore}{\kern0pt}in{\isacharunderscore}{\kern0pt}M{\isacharunderscore}{\kern0pt}iff\ \isacommand{by}\isamarkupfalse%
\ auto\ \isanewline
\ \ \isacommand{have}\isamarkupfalse%
\ piin\ {\isacharcolon}{\kern0pt}\ {\isachardoublequoteopen}pi\ {\isasymin}\ M{\isachardoublequoteclose}\ \isacommand{using}\isamarkupfalse%
\ y{\isacharunderscore}{\kern0pt}pi{\isacharunderscore}{\kern0pt}eq\ H\ pair{\isacharunderscore}{\kern0pt}in{\isacharunderscore}{\kern0pt}M{\isacharunderscore}{\kern0pt}iff\ \isacommand{by}\isamarkupfalse%
\ auto\ \isanewline
\ \ \isacommand{have}\isamarkupfalse%
\ pifun\ {\isacharcolon}{\kern0pt}\ {\isachardoublequoteopen}function{\isacharparenleft}{\kern0pt}pi{\isacharparenright}{\kern0pt}{\isachardoublequoteclose}\ \isacommand{using}\isamarkupfalse%
\ function{\isacharunderscore}{\kern0pt}abs\ piin\ H\ \isacommand{by}\isamarkupfalse%
\ auto\ \isanewline
\ \ \isacommand{have}\isamarkupfalse%
\ pi{\isacharunderscore}{\kern0pt}app{\isacharunderscore}{\kern0pt}p{\isacharunderscore}{\kern0pt}val\ {\isacharcolon}{\kern0pt}\ {\isachardoublequoteopen}pi{\isacharbackquote}{\kern0pt}p\ {\isacharequal}{\kern0pt}\ pi{\isacharunderscore}{\kern0pt}app{\isacharunderscore}{\kern0pt}p{\isachardoublequoteclose}\ \isacommand{using}\isamarkupfalse%
\ apply{\isacharunderscore}{\kern0pt}abs\ piin\ pin\ H\ \isacommand{by}\isamarkupfalse%
\ auto\isanewline
\ \ \isacommand{have}\isamarkupfalse%
\ g{\isacharunderscore}{\kern0pt}app{\isacharunderscore}{\kern0pt}val\ {\isacharcolon}{\kern0pt}\ {\isachardoublequoteopen}g{\isacharbackquote}{\kern0pt}{\isacharless}{\kern0pt}y{\isacharcomma}{\kern0pt}pi{\isachargreater}{\kern0pt}\ {\isacharequal}{\kern0pt}\ g{\isacharunderscore}{\kern0pt}app{\isacharunderscore}{\kern0pt}y{\isacharunderscore}{\kern0pt}pi{\isachardoublequoteclose}\ \ \ \ \isanewline
\ \ \ \ \isacommand{apply}\isamarkupfalse%
\ {\isacharparenleft}{\kern0pt}rule{\isacharunderscore}{\kern0pt}tac\ P{\isacharequal}{\kern0pt}{\isachardoublequoteopen}fun{\isacharunderscore}{\kern0pt}apply{\isacharparenleft}{\kern0pt}{\isacharhash}{\kern0pt}{\isacharhash}{\kern0pt}M{\isacharcomma}{\kern0pt}\ g{\isacharcomma}{\kern0pt}\ y{\isacharunderscore}{\kern0pt}pi{\isacharcomma}{\kern0pt}\ g{\isacharunderscore}{\kern0pt}app{\isacharunderscore}{\kern0pt}y{\isacharunderscore}{\kern0pt}pi{\isacharparenright}{\kern0pt}{\isachardoublequoteclose}\ \isakeyword{in}\ iffD{\isadigit{1}}{\isacharparenright}{\kern0pt}\isanewline
\ \ \ \ \isacommand{apply}\isamarkupfalse%
\ {\isacharparenleft}{\kern0pt}rule{\isacharunderscore}{\kern0pt}tac\ Q{\isacharequal}{\kern0pt}{\isachardoublequoteopen}g{\isacharbackquote}{\kern0pt}y{\isacharunderscore}{\kern0pt}pi{\isacharequal}{\kern0pt}g{\isacharunderscore}{\kern0pt}app{\isacharunderscore}{\kern0pt}y{\isacharunderscore}{\kern0pt}pi{\isachardoublequoteclose}\ \isakeyword{in}\ iff{\isacharunderscore}{\kern0pt}trans{\isacharparenright}{\kern0pt}\isanewline
\ \ \ \ \isacommand{apply}\isamarkupfalse%
\ {\isacharparenleft}{\kern0pt}rule{\isacharunderscore}{\kern0pt}tac\ apply{\isacharunderscore}{\kern0pt}abs{\isacharparenright}{\kern0pt}\ \isacommand{using}\isamarkupfalse%
\ H\ inM\ \isacommand{by}\isamarkupfalse%
\ auto\isanewline
\ \ \isacommand{show}\isamarkupfalse%
\ {\isachardoublequoteopen}{\isacharparenleft}{\kern0pt}{\isasymexists}y\ {\isasymin}\ M{\isachardot}{\kern0pt}\ {\isasymexists}p\ {\isasymin}\ M{\isachardot}{\kern0pt}\ {\isasymexists}x\ {\isasymin}\ M{\isachardot}{\kern0pt}\ {\isasymexists}pi\ {\isasymin}\ M{\isachardot}{\kern0pt}\ \isanewline
\ \ \ \ \ \ \ \ x{\isacharunderscore}{\kern0pt}pi\ {\isacharequal}{\kern0pt}\ {\isacharless}{\kern0pt}x{\isacharcomma}{\kern0pt}\ pi{\isachargreater}{\kern0pt}\ {\isasymand}\isanewline
\ \ \ \ \ \ \ \ elem\ {\isacharequal}{\kern0pt}\ {\isacharless}{\kern0pt}g{\isacharbackquote}{\kern0pt}{\isacharless}{\kern0pt}y{\isacharcomma}{\kern0pt}pi{\isachargreater}{\kern0pt}{\isacharcomma}{\kern0pt}\ pi{\isacharbackquote}{\kern0pt}p{\isachargreater}{\kern0pt}\ {\isasymand}\ \isanewline
\ \ \ \ \ \ \ \ {\isacharless}{\kern0pt}y{\isacharcomma}{\kern0pt}\ p{\isachargreater}{\kern0pt}\ {\isasymin}\ x\ {\isasymand}\ \isanewline
\ \ \ \ \ \ \ \ function{\isacharparenleft}{\kern0pt}pi{\isacharparenright}{\kern0pt}\isanewline
\ \ \ \ \ {\isacharparenright}{\kern0pt}{\isachardoublequoteclose}\isanewline
\ \ \ \ \isacommand{apply}\isamarkupfalse%
\ {\isacharparenleft}{\kern0pt}rule{\isacharunderscore}{\kern0pt}tac\ x{\isacharequal}{\kern0pt}y\ \isakeyword{in}\ bexI{\isacharparenright}{\kern0pt}\ \ \isacommand{apply}\isamarkupfalse%
\ {\isacharparenleft}{\kern0pt}rule{\isacharunderscore}{\kern0pt}tac\ x{\isacharequal}{\kern0pt}p\ \isakeyword{in}\ bexI{\isacharparenright}{\kern0pt}\ \isanewline
\ \ \ \ \isacommand{apply}\isamarkupfalse%
\ {\isacharparenleft}{\kern0pt}rule{\isacharunderscore}{\kern0pt}tac\ x{\isacharequal}{\kern0pt}x\ \isakeyword{in}\ bexI{\isacharparenright}{\kern0pt}\ \ \isacommand{apply}\isamarkupfalse%
\ {\isacharparenleft}{\kern0pt}rule{\isacharunderscore}{\kern0pt}tac\ x{\isacharequal}{\kern0pt}pi\ \isakeyword{in}\ bexI{\isacharparenright}{\kern0pt}\isanewline
\ \ \ \ \isacommand{using}\isamarkupfalse%
\ g{\isacharunderscore}{\kern0pt}app{\isacharunderscore}{\kern0pt}val\ pi{\isacharunderscore}{\kern0pt}app{\isacharunderscore}{\kern0pt}p{\isacharunderscore}{\kern0pt}val\ H\ elemeq\ pifun\ piin\ xin\ pin\ yin\ y{\isacharunderscore}{\kern0pt}p{\isacharunderscore}{\kern0pt}eq\ H\ x{\isacharunderscore}{\kern0pt}pi{\isacharunderscore}{\kern0pt}eq\ \isacommand{by}\isamarkupfalse%
\ auto\isanewline
\isacommand{next}\isamarkupfalse%
\ \isanewline
\ \ \isacommand{assume}\isamarkupfalse%
\ {\isachardoublequoteopen}{\isasymexists}y{\isasymin}M{\isachardot}{\kern0pt}\ {\isasymexists}p{\isasymin}M{\isachardot}{\kern0pt}\ {\isasymexists}x{\isasymin}M{\isachardot}{\kern0pt}\ {\isasymexists}pi{\isasymin}M{\isachardot}{\kern0pt}\ \isanewline
\ \ \ \ \ \ x{\isacharunderscore}{\kern0pt}pi\ {\isacharequal}{\kern0pt}\ {\isacharless}{\kern0pt}x{\isacharcomma}{\kern0pt}\ pi{\isachargreater}{\kern0pt}\ {\isasymand}\ elem\ {\isacharequal}{\kern0pt}\ {\isasymlangle}g\ {\isacharbackquote}{\kern0pt}\ {\isasymlangle}y{\isacharcomma}{\kern0pt}\ pi{\isasymrangle}{\isacharcomma}{\kern0pt}\ pi\ {\isacharbackquote}{\kern0pt}\ p{\isasymrangle}\ {\isasymand}\ {\isasymlangle}y{\isacharcomma}{\kern0pt}\ p{\isasymrangle}\ {\isasymin}\ x\ {\isasymand}\ function{\isacharparenleft}{\kern0pt}pi{\isacharparenright}{\kern0pt}{\isachardoublequoteclose}\isanewline
\ \ \isakeyword{and}\ inM\ {\isacharcolon}{\kern0pt}\ {\isachardoublequoteopen}x{\isacharunderscore}{\kern0pt}pi\ {\isasymin}\ M{\isachardoublequoteclose}\ {\isachardoublequoteopen}g\ {\isasymin}\ M{\isachardoublequoteclose}\ {\isachardoublequoteopen}elem\ {\isasymin}\ M{\isachardoublequoteclose}\isanewline
\isanewline
\ \ \isacommand{then}\isamarkupfalse%
\ \isacommand{obtain}\isamarkupfalse%
\ y\ p\ x\ pi\ \isakeyword{where}\ H\ {\isacharcolon}{\kern0pt}\ \isanewline
\ \ \ \ {\isachardoublequoteopen}y\ {\isasymin}\ M{\isachardoublequoteclose}\ {\isachardoublequoteopen}p\ {\isasymin}\ M{\isachardoublequoteclose}\ {\isachardoublequoteopen}x\ {\isasymin}\ M{\isachardoublequoteclose}\ {\isachardoublequoteopen}pi\ {\isasymin}\ M{\isachardoublequoteclose}\ \isanewline
\ \ \ \ {\isachardoublequoteopen}x{\isacharunderscore}{\kern0pt}pi\ {\isacharequal}{\kern0pt}\ {\isacharless}{\kern0pt}x{\isacharcomma}{\kern0pt}\ pi{\isachargreater}{\kern0pt}{\isachardoublequoteclose}\ {\isachardoublequoteopen}elem\ {\isacharequal}{\kern0pt}\ {\isasymlangle}g\ {\isacharbackquote}{\kern0pt}\ {\isasymlangle}y{\isacharcomma}{\kern0pt}\ pi{\isasymrangle}{\isacharcomma}{\kern0pt}\ pi\ {\isacharbackquote}{\kern0pt}\ p{\isasymrangle}{\isachardoublequoteclose}\ {\isachardoublequoteopen}{\isasymlangle}y{\isacharcomma}{\kern0pt}\ p{\isasymrangle}\ {\isasymin}\ x{\isachardoublequoteclose}\ {\isachardoublequoteopen}function{\isacharparenleft}{\kern0pt}pi{\isacharparenright}{\kern0pt}{\isachardoublequoteclose}\isanewline
\ \ \ \ \isacommand{by}\isamarkupfalse%
\ auto\ \isanewline
\isanewline
\ \ \isacommand{show}\isamarkupfalse%
\ {\isachardoublequoteopen}HPn{\isacharunderscore}{\kern0pt}auto{\isacharunderscore}{\kern0pt}M{\isacharunderscore}{\kern0pt}cond{\isacharparenleft}{\kern0pt}x{\isacharunderscore}{\kern0pt}pi{\isacharcomma}{\kern0pt}\ g{\isacharcomma}{\kern0pt}\ elem{\isacharparenright}{\kern0pt}{\isachardoublequoteclose}\isanewline
\ \ \ \ \isacommand{unfolding}\isamarkupfalse%
\ HPn{\isacharunderscore}{\kern0pt}auto{\isacharunderscore}{\kern0pt}M{\isacharunderscore}{\kern0pt}cond{\isacharunderscore}{\kern0pt}def\ \isanewline
\ \ \ \ \isacommand{apply}\isamarkupfalse%
\ {\isacharparenleft}{\kern0pt}rule{\isacharunderscore}{\kern0pt}tac\ x{\isacharequal}{\kern0pt}{\isachardoublequoteopen}g{\isacharbackquote}{\kern0pt}{\isacharless}{\kern0pt}y{\isacharcomma}{\kern0pt}\ pi{\isachargreater}{\kern0pt}{\isachardoublequoteclose}\ \isakeyword{in}\ bexI{\isacharparenright}{\kern0pt}\ \isanewline
\ \ \ \ \isacommand{apply}\isamarkupfalse%
\ {\isacharparenleft}{\kern0pt}rule{\isacharunderscore}{\kern0pt}tac\ x{\isacharequal}{\kern0pt}{\isachardoublequoteopen}pi{\isacharbackquote}{\kern0pt}p{\isachardoublequoteclose}\ \isakeyword{in}\ bexI{\isacharparenright}{\kern0pt}\ \isanewline
\ \ \ \ \isacommand{apply}\isamarkupfalse%
\ {\isacharparenleft}{\kern0pt}rule{\isacharunderscore}{\kern0pt}tac\ x{\isacharequal}{\kern0pt}{\isachardoublequoteopen}{\isacharless}{\kern0pt}y{\isacharcomma}{\kern0pt}p{\isachargreater}{\kern0pt}{\isachardoublequoteclose}\ \isakeyword{in}\ bexI{\isacharparenright}{\kern0pt}\ \isanewline
\ \ \ \ \isacommand{apply}\isamarkupfalse%
\ {\isacharparenleft}{\kern0pt}rule{\isacharunderscore}{\kern0pt}tac\ x{\isacharequal}{\kern0pt}{\isachardoublequoteopen}{\isacharless}{\kern0pt}y{\isacharcomma}{\kern0pt}pi{\isachargreater}{\kern0pt}{\isachardoublequoteclose}\ \isakeyword{in}\ bexI{\isacharparenright}{\kern0pt}\ \isanewline
\ \ \ \ \isacommand{apply}\isamarkupfalse%
\ {\isacharparenleft}{\kern0pt}rule{\isacharunderscore}{\kern0pt}tac\ x{\isacharequal}{\kern0pt}{\isachardoublequoteopen}y{\isachardoublequoteclose}\ \isakeyword{in}\ bexI{\isacharparenright}{\kern0pt}\ \isanewline
\ \ \ \ \isacommand{apply}\isamarkupfalse%
\ {\isacharparenleft}{\kern0pt}rule{\isacharunderscore}{\kern0pt}tac\ x{\isacharequal}{\kern0pt}{\isachardoublequoteopen}p{\isachardoublequoteclose}\ \isakeyword{in}\ bexI{\isacharparenright}{\kern0pt}\ \isanewline
\ \ \ \ \isacommand{apply}\isamarkupfalse%
\ {\isacharparenleft}{\kern0pt}rule{\isacharunderscore}{\kern0pt}tac\ x{\isacharequal}{\kern0pt}{\isachardoublequoteopen}x{\isachardoublequoteclose}\ \isakeyword{in}\ bexI{\isacharparenright}{\kern0pt}\ \isanewline
\ \ \ \ \isacommand{apply}\isamarkupfalse%
\ {\isacharparenleft}{\kern0pt}rule{\isacharunderscore}{\kern0pt}tac\ x{\isacharequal}{\kern0pt}{\isachardoublequoteopen}pi{\isachardoublequoteclose}\ \isakeyword{in}\ bexI{\isacharparenright}{\kern0pt}\ \isanewline
\ \ \ \ \isacommand{using}\isamarkupfalse%
\ inM\ pair{\isacharunderscore}{\kern0pt}in{\isacharunderscore}{\kern0pt}M{\isacharunderscore}{\kern0pt}iff\ H\ \isacommand{by}\isamarkupfalse%
\ simp{\isacharunderscore}{\kern0pt}all\isanewline
\isacommand{qed}\isamarkupfalse%
%
\endisatagproof
{\isafoldproof}%
%
\isadelimproof
\isanewline
%
\endisadelimproof
\isanewline
\isacommand{lemma}\isamarkupfalse%
\ HPn{\isacharunderscore}{\kern0pt}auto{\isacharunderscore}{\kern0pt}M{\isacharunderscore}{\kern0pt}cond{\isacharunderscore}{\kern0pt}elem{\isacharunderscore}{\kern0pt}in\ {\isacharcolon}{\kern0pt}\ \isanewline
\ \ {\isachardoublequoteopen}x{\isacharunderscore}{\kern0pt}pi\ {\isasymin}\ M\ {\isasymLongrightarrow}\ g\ {\isasymin}\ M\ {\isasymLongrightarrow}\ function{\isacharparenleft}{\kern0pt}g{\isacharparenright}{\kern0pt}\ {\isasymLongrightarrow}\ \isanewline
\ \ \ {\isasymexists}A\ {\isasymin}\ M{\isachardot}{\kern0pt}\ {\isasymforall}\ elem\ {\isasymin}\ M{\isachardot}{\kern0pt}\ HPn{\isacharunderscore}{\kern0pt}auto{\isacharunderscore}{\kern0pt}M{\isacharunderscore}{\kern0pt}cond{\isacharparenleft}{\kern0pt}x{\isacharunderscore}{\kern0pt}pi{\isacharcomma}{\kern0pt}\ g{\isacharcomma}{\kern0pt}\ elem{\isacharparenright}{\kern0pt}\ {\isasymlongrightarrow}\ elem\ {\isasymin}\ A{\isachardoublequoteclose}\ \isanewline
%
\isadelimproof
\ \ %
\endisadelimproof
%
\isatagproof
\isacommand{apply}\isamarkupfalse%
\ {\isacharparenleft}{\kern0pt}rule{\isacharunderscore}{\kern0pt}tac\ x{\isacharequal}{\kern0pt}{\isachardoublequoteopen}MVset{\isacharparenleft}{\kern0pt}succ{\isacharparenleft}{\kern0pt}rank{\isacharparenleft}{\kern0pt}g{\isacharparenright}{\kern0pt}{\isacharparenright}{\kern0pt}{\isacharparenright}{\kern0pt}\ {\isasymtimes}\ MVset{\isacharparenleft}{\kern0pt}succ{\isacharparenleft}{\kern0pt}rank{\isacharparenleft}{\kern0pt}x{\isacharunderscore}{\kern0pt}pi{\isacharparenright}{\kern0pt}{\isacharparenright}{\kern0pt}{\isacharparenright}{\kern0pt}{\isachardoublequoteclose}\ \isakeyword{in}\ bexI{\isacharparenright}{\kern0pt}\ \isanewline
\isacommand{proof}\isamarkupfalse%
\ {\isacharparenleft}{\kern0pt}clarify{\isacharparenright}{\kern0pt}\ \isanewline
\ \ \isacommand{fix}\isamarkupfalse%
\ elem\ \isanewline
\ \ \isacommand{assume}\isamarkupfalse%
\ inM\ {\isacharcolon}{\kern0pt}\ {\isachardoublequoteopen}x{\isacharunderscore}{\kern0pt}pi\ {\isasymin}\ M{\isachardoublequoteclose}\ {\isachardoublequoteopen}g\ {\isasymin}\ M{\isachardoublequoteclose}\ {\isachardoublequoteopen}elem\ {\isasymin}\ M{\isachardoublequoteclose}\ \isanewline
\ \ \isakeyword{and}\ gfun\ {\isacharcolon}{\kern0pt}\ {\isachardoublequoteopen}function{\isacharparenleft}{\kern0pt}g{\isacharparenright}{\kern0pt}{\isachardoublequoteclose}\ \isanewline
\ \ \isakeyword{and}\ cond\ {\isacharcolon}{\kern0pt}\ {\isachardoublequoteopen}HPn{\isacharunderscore}{\kern0pt}auto{\isacharunderscore}{\kern0pt}M{\isacharunderscore}{\kern0pt}cond{\isacharparenleft}{\kern0pt}x{\isacharunderscore}{\kern0pt}pi{\isacharcomma}{\kern0pt}\ g{\isacharcomma}{\kern0pt}\ elem{\isacharparenright}{\kern0pt}{\isachardoublequoteclose}\ \isanewline
\isanewline
\ \ \isacommand{then}\isamarkupfalse%
\ \isacommand{have}\isamarkupfalse%
\ {\isachardoublequoteopen}\ {\isacharparenleft}{\kern0pt}{\isasymexists}y\ {\isasymin}\ M{\isachardot}{\kern0pt}\ {\isasymexists}p\ {\isasymin}\ M{\isachardot}{\kern0pt}\ {\isasymexists}x\ {\isasymin}\ M{\isachardot}{\kern0pt}\ {\isasymexists}pi\ {\isasymin}\ M{\isachardot}{\kern0pt}\ \isanewline
\ \ \ \ \ \ \ \ x{\isacharunderscore}{\kern0pt}pi\ {\isacharequal}{\kern0pt}\ {\isacharless}{\kern0pt}x{\isacharcomma}{\kern0pt}\ pi{\isachargreater}{\kern0pt}\ {\isasymand}\isanewline
\ \ \ \ \ \ \ \ elem\ {\isacharequal}{\kern0pt}\ {\isacharless}{\kern0pt}g{\isacharbackquote}{\kern0pt}{\isacharless}{\kern0pt}y{\isacharcomma}{\kern0pt}pi{\isachargreater}{\kern0pt}{\isacharcomma}{\kern0pt}\ pi{\isacharbackquote}{\kern0pt}p{\isachargreater}{\kern0pt}\ {\isasymand}\ \isanewline
\ \ \ \ \ \ \ \ {\isacharless}{\kern0pt}y{\isacharcomma}{\kern0pt}\ p{\isachargreater}{\kern0pt}\ {\isasymin}\ x\ {\isasymand}\ \isanewline
\ \ \ \ \ \ \ \ function{\isacharparenleft}{\kern0pt}pi{\isacharparenright}{\kern0pt}\isanewline
\ \ \ \ \ {\isacharparenright}{\kern0pt}{\isachardoublequoteclose}\ \isacommand{using}\isamarkupfalse%
\ HPn{\isacharunderscore}{\kern0pt}auto{\isacharunderscore}{\kern0pt}M{\isacharunderscore}{\kern0pt}cond{\isacharunderscore}{\kern0pt}iff\ \isacommand{by}\isamarkupfalse%
\ auto\ \isanewline
\ \ \isacommand{then}\isamarkupfalse%
\ \isacommand{obtain}\isamarkupfalse%
\ y\ p\ x\ pi\ \isakeyword{where}\ H\ {\isacharcolon}{\kern0pt}\ \isanewline
\ \ \ \ {\isachardoublequoteopen}y\ {\isasymin}\ M{\isachardoublequoteclose}\ {\isachardoublequoteopen}p\ {\isasymin}\ M{\isachardoublequoteclose}\ {\isachardoublequoteopen}x\ {\isasymin}\ M{\isachardoublequoteclose}\ {\isachardoublequoteopen}pi\ {\isasymin}\ M{\isachardoublequoteclose}\ \isanewline
\ \ \ \ {\isachardoublequoteopen}x{\isacharunderscore}{\kern0pt}pi\ {\isacharequal}{\kern0pt}\ {\isacharless}{\kern0pt}x{\isacharcomma}{\kern0pt}\ pi{\isachargreater}{\kern0pt}{\isachardoublequoteclose}\ {\isachardoublequoteopen}elem\ {\isacharequal}{\kern0pt}\ {\isasymlangle}g\ {\isacharbackquote}{\kern0pt}\ {\isasymlangle}y{\isacharcomma}{\kern0pt}\ pi{\isasymrangle}{\isacharcomma}{\kern0pt}\ pi\ {\isacharbackquote}{\kern0pt}\ p{\isasymrangle}{\isachardoublequoteclose}\ {\isachardoublequoteopen}{\isasymlangle}y{\isacharcomma}{\kern0pt}\ p{\isasymrangle}\ {\isasymin}\ x{\isachardoublequoteclose}\ {\isachardoublequoteopen}function{\isacharparenleft}{\kern0pt}pi{\isacharparenright}{\kern0pt}{\isachardoublequoteclose}\isanewline
\ \ \ \ \ \isacommand{by}\isamarkupfalse%
\ auto\ \isanewline
\isanewline
\ \ \ \isacommand{have}\isamarkupfalse%
\ P{\isadigit{1}}{\isacharcolon}{\kern0pt}\ {\isachardoublequoteopen}g{\isacharbackquote}{\kern0pt}{\isacharless}{\kern0pt}y{\isacharcomma}{\kern0pt}\ pi{\isachargreater}{\kern0pt}\ {\isasymin}\ MVset{\isacharparenleft}{\kern0pt}succ{\isacharparenleft}{\kern0pt}rank{\isacharparenleft}{\kern0pt}g{\isacharparenright}{\kern0pt}{\isacharparenright}{\kern0pt}{\isacharparenright}{\kern0pt}{\isachardoublequoteclose}\ \isanewline
\ \ \ \ \ \isacommand{apply}\isamarkupfalse%
\ {\isacharparenleft}{\kern0pt}rule{\isacharunderscore}{\kern0pt}tac\ P{\isacharequal}{\kern0pt}{\isachardoublequoteopen}{\isasymlambda}x{\isachardot}{\kern0pt}\ x\ {\isasymin}\ MVset{\isacharparenleft}{\kern0pt}succ{\isacharparenleft}{\kern0pt}rank{\isacharparenleft}{\kern0pt}g{\isacharparenright}{\kern0pt}{\isacharparenright}{\kern0pt}{\isacharparenright}{\kern0pt}{\isachardoublequoteclose}\ \isakeyword{and}\ F{\isacharequal}{\kern0pt}g\ \isakeyword{and}\ x{\isacharequal}{\kern0pt}{\isachardoublequoteopen}{\isacharless}{\kern0pt}y{\isacharcomma}{\kern0pt}\ pi{\isachargreater}{\kern0pt}{\isachardoublequoteclose}\ \isakeyword{in}\ in{\isacharunderscore}{\kern0pt}dom{\isacharunderscore}{\kern0pt}or{\isacharunderscore}{\kern0pt}not{\isacharparenright}{\kern0pt}\isanewline
\ \ \ \ \ \isacommand{using}\isamarkupfalse%
\ gfun\ \isanewline
\ \ \ \ \ \ \ \isacommand{apply}\isamarkupfalse%
\ simp\ \isanewline
\ \ \ \ \ \isacommand{using}\isamarkupfalse%
\ zero{\isacharunderscore}{\kern0pt}in{\isacharunderscore}{\kern0pt}MVset\ Ord{\isacharunderscore}{\kern0pt}rank\ \isanewline
\ \ \ \ \ \ \isacommand{apply}\isamarkupfalse%
\ simp\ \isanewline
\ \ \ \isacommand{proof}\isamarkupfalse%
\ {\isacharminus}{\kern0pt}\ \isanewline
\ \ \ \ \ \isacommand{assume}\isamarkupfalse%
\ assm\ {\isacharcolon}{\kern0pt}\ {\isachardoublequoteopen}{\isasymlangle}y{\isacharcomma}{\kern0pt}\ pi{\isasymrangle}\ {\isasymin}\ domain{\isacharparenleft}{\kern0pt}g{\isacharparenright}{\kern0pt}{\isachardoublequoteclose}\ \isanewline
\ \ \ \ \ \isacommand{then}\isamarkupfalse%
\ \isacommand{obtain}\isamarkupfalse%
\ w\ \isakeyword{where}\ wh\ {\isacharcolon}{\kern0pt}\ {\isachardoublequoteopen}{\isacharless}{\kern0pt}{\isacharless}{\kern0pt}y{\isacharcomma}{\kern0pt}\ pi{\isachargreater}{\kern0pt}{\isacharcomma}{\kern0pt}\ w{\isachargreater}{\kern0pt}\ {\isasymin}\ g{\isachardoublequoteclose}\ \isacommand{by}\isamarkupfalse%
\ auto\ \isanewline
\ \ \ \ \ \isacommand{then}\isamarkupfalse%
\ \isacommand{have}\isamarkupfalse%
\ {\isachardoublequoteopen}{\isacharless}{\kern0pt}{\isacharless}{\kern0pt}y{\isacharcomma}{\kern0pt}\ pi{\isachargreater}{\kern0pt}{\isacharcomma}{\kern0pt}\ w{\isachargreater}{\kern0pt}\ {\isasymin}\ M{\isachardoublequoteclose}\ \isacommand{using}\isamarkupfalse%
\ transM\ inM\ \isacommand{by}\isamarkupfalse%
\ auto\isanewline
\ \ \ \ \ \isacommand{then}\isamarkupfalse%
\ \isacommand{have}\isamarkupfalse%
\ winM\ {\isacharcolon}{\kern0pt}\ {\isachardoublequoteopen}w\ {\isasymin}\ M{\isachardoublequoteclose}\ \isacommand{using}\isamarkupfalse%
\ pair{\isacharunderscore}{\kern0pt}in{\isacharunderscore}{\kern0pt}M{\isacharunderscore}{\kern0pt}iff\ \isacommand{by}\isamarkupfalse%
\ auto\ \isanewline
\ \ \ \ \ \isacommand{have}\isamarkupfalse%
\ weq\ {\isacharcolon}{\kern0pt}\ {\isachardoublequoteopen}g{\isacharbackquote}{\kern0pt}{\isacharless}{\kern0pt}y{\isacharcomma}{\kern0pt}\ pi{\isachargreater}{\kern0pt}\ {\isacharequal}{\kern0pt}\ w{\isachardoublequoteclose}\ \isacommand{apply}\isamarkupfalse%
\ {\isacharparenleft}{\kern0pt}rule{\isacharunderscore}{\kern0pt}tac\ function{\isacharunderscore}{\kern0pt}apply{\isacharunderscore}{\kern0pt}equality{\isacharparenright}{\kern0pt}\ \isacommand{using}\isamarkupfalse%
\ wh\ gfun\ \isacommand{by}\isamarkupfalse%
\ auto\isanewline
\ \ \ \ \ \isacommand{then}\isamarkupfalse%
\ \isacommand{have}\isamarkupfalse%
\ {\isachardoublequoteopen}g{\isacharbackquote}{\kern0pt}{\isacharless}{\kern0pt}y{\isacharcomma}{\kern0pt}pi{\isachargreater}{\kern0pt}\ {\isasymin}\ M{\isachardoublequoteclose}\ \isacommand{using}\isamarkupfalse%
\ winM\ \isacommand{by}\isamarkupfalse%
\ auto\isanewline
\ \ \ \ \ \isacommand{then}\isamarkupfalse%
\ \isacommand{show}\isamarkupfalse%
\ {\isachardoublequoteopen}g{\isacharbackquote}{\kern0pt}{\isacharless}{\kern0pt}y{\isacharcomma}{\kern0pt}\ pi{\isachargreater}{\kern0pt}\ {\isasymin}\ MVset{\isacharparenleft}{\kern0pt}succ{\isacharparenleft}{\kern0pt}rank{\isacharparenleft}{\kern0pt}g{\isacharparenright}{\kern0pt}{\isacharparenright}{\kern0pt}{\isacharparenright}{\kern0pt}{\isachardoublequoteclose}\ \isanewline
\ \ \ \ \ \ \ \isacommand{apply}\isamarkupfalse%
\ {\isacharparenleft}{\kern0pt}rule{\isacharunderscore}{\kern0pt}tac\ MVsetI{\isacharsemicolon}{\kern0pt}\ simp{\isacharparenright}{\kern0pt}\ \isanewline
\ \ \ \ \ \ \ \isacommand{apply}\isamarkupfalse%
\ {\isacharparenleft}{\kern0pt}rule\ lt{\isacharunderscore}{\kern0pt}succ{\isacharunderscore}{\kern0pt}lt{\isacharparenright}{\kern0pt}\ \isanewline
\ \ \ \ \ \ \ \isacommand{using}\isamarkupfalse%
\ Ord{\isacharunderscore}{\kern0pt}rank\ \isanewline
\ \ \ \ \ \ \ \ \isacommand{apply}\isamarkupfalse%
\ simp\ \isanewline
\ \ \ \ \ \ \ \isacommand{apply}\isamarkupfalse%
\ {\isacharparenleft}{\kern0pt}rule{\isacharunderscore}{\kern0pt}tac\ j{\isacharequal}{\kern0pt}{\isachardoublequoteopen}rank{\isacharparenleft}{\kern0pt}{\isacharless}{\kern0pt}{\isacharless}{\kern0pt}y{\isacharcomma}{\kern0pt}\ pi{\isachargreater}{\kern0pt}{\isacharcomma}{\kern0pt}\ w{\isachargreater}{\kern0pt}{\isacharparenright}{\kern0pt}{\isachardoublequoteclose}\ \isakeyword{in}\ lt{\isacharunderscore}{\kern0pt}trans{\isacharparenright}{\kern0pt}\isanewline
\ \ \ \ \ \ \ \isacommand{using}\isamarkupfalse%
\ weq\ rank{\isacharunderscore}{\kern0pt}pair{\isadigit{2}}\ \isanewline
\ \ \ \ \ \ \ \ \isacommand{apply}\isamarkupfalse%
\ simp\isanewline
\ \ \ \ \ \ \ \isacommand{using}\isamarkupfalse%
\ wh\ rank{\isacharunderscore}{\kern0pt}lt\ \isacommand{by}\isamarkupfalse%
\ simp\isanewline
\ \ \ \isacommand{qed}\isamarkupfalse%
\isanewline
\isanewline
\ \ \ \isacommand{have}\isamarkupfalse%
\ {\isachardoublequoteopen}pi\ {\isacharbackquote}{\kern0pt}\ p\ {\isasymin}\ MVset{\isacharparenleft}{\kern0pt}succ{\isacharparenleft}{\kern0pt}rank{\isacharparenleft}{\kern0pt}x{\isacharunderscore}{\kern0pt}pi{\isacharparenright}{\kern0pt}{\isacharparenright}{\kern0pt}{\isacharparenright}{\kern0pt}{\isachardoublequoteclose}\ \isanewline
\ \ \ \ \ \isacommand{apply}\isamarkupfalse%
\ {\isacharparenleft}{\kern0pt}rule{\isacharunderscore}{\kern0pt}tac\ P{\isacharequal}{\kern0pt}{\isachardoublequoteopen}{\isasymlambda}x{\isachardot}{\kern0pt}\ x\ {\isasymin}\ MVset{\isacharparenleft}{\kern0pt}succ{\isacharparenleft}{\kern0pt}rank{\isacharparenleft}{\kern0pt}x{\isacharunderscore}{\kern0pt}pi{\isacharparenright}{\kern0pt}{\isacharparenright}{\kern0pt}{\isacharparenright}{\kern0pt}{\isachardoublequoteclose}\ \isakeyword{in}\ in{\isacharunderscore}{\kern0pt}dom{\isacharunderscore}{\kern0pt}or{\isacharunderscore}{\kern0pt}not{\isacharparenright}{\kern0pt}\ \isanewline
\ \ \ \ \ \isacommand{using}\isamarkupfalse%
\ H\ \isanewline
\ \ \ \ \ \ \ \isacommand{apply}\isamarkupfalse%
\ simp\isanewline
\ \ \ \ \ \isacommand{using}\isamarkupfalse%
\ zero{\isacharunderscore}{\kern0pt}in{\isacharunderscore}{\kern0pt}MVset\ Ord{\isacharunderscore}{\kern0pt}rank\isanewline
\ \ \ \ \ \ \isacommand{apply}\isamarkupfalse%
\ simp\ \isanewline
\ \ \ \isacommand{proof}\isamarkupfalse%
\ {\isacharminus}{\kern0pt}\ \isanewline
\ \ \ \ \ \isacommand{assume}\isamarkupfalse%
\ {\isachardoublequoteopen}p\ {\isasymin}\ domain{\isacharparenleft}{\kern0pt}pi{\isacharparenright}{\kern0pt}{\isachardoublequoteclose}\ \isanewline
\ \ \ \ \ \isacommand{then}\isamarkupfalse%
\ \isacommand{obtain}\isamarkupfalse%
\ w\ \isakeyword{where}\ wh\ {\isacharcolon}{\kern0pt}\ {\isachardoublequoteopen}{\isacharless}{\kern0pt}p{\isacharcomma}{\kern0pt}\ w{\isachargreater}{\kern0pt}\ {\isasymin}\ pi{\isachardoublequoteclose}\ \isacommand{by}\isamarkupfalse%
\ auto\ \isanewline
\ \ \ \ \ \isacommand{then}\isamarkupfalse%
\ \isacommand{have}\isamarkupfalse%
\ weq\ {\isacharcolon}{\kern0pt}\ {\isachardoublequoteopen}pi{\isacharbackquote}{\kern0pt}p\ {\isacharequal}{\kern0pt}\ w{\isachardoublequoteclose}\ \isacommand{using}\isamarkupfalse%
\ function{\isacharunderscore}{\kern0pt}apply{\isacharunderscore}{\kern0pt}equality\ H\ \isacommand{by}\isamarkupfalse%
\ auto\ \isanewline
\ \ \ \ \ \isacommand{have}\isamarkupfalse%
\ {\isachardoublequoteopen}{\isacharless}{\kern0pt}p{\isacharcomma}{\kern0pt}\ w{\isachargreater}{\kern0pt}\ {\isasymin}\ M{\isachardoublequoteclose}\ \isacommand{using}\isamarkupfalse%
\ transM\ wh\ H\ \isacommand{by}\isamarkupfalse%
\ auto\ \isanewline
\ \ \ \ \ \isacommand{then}\isamarkupfalse%
\ \isacommand{have}\isamarkupfalse%
\ {\isachardoublequoteopen}w\ {\isasymin}\ M{\isachardoublequoteclose}\ \isacommand{using}\isamarkupfalse%
\ pair{\isacharunderscore}{\kern0pt}in{\isacharunderscore}{\kern0pt}M{\isacharunderscore}{\kern0pt}iff\ \isacommand{by}\isamarkupfalse%
\ auto\ \isanewline
\ \ \ \ \ \isacommand{then}\isamarkupfalse%
\ \isacommand{show}\isamarkupfalse%
\ {\isachardoublequoteopen}pi\ {\isacharbackquote}{\kern0pt}\ p\ {\isasymin}\ MVset{\isacharparenleft}{\kern0pt}succ{\isacharparenleft}{\kern0pt}rank{\isacharparenleft}{\kern0pt}x{\isacharunderscore}{\kern0pt}pi{\isacharparenright}{\kern0pt}{\isacharparenright}{\kern0pt}{\isacharparenright}{\kern0pt}{\isachardoublequoteclose}\isanewline
\ \ \ \ \ \ \ \isacommand{apply}\isamarkupfalse%
\ {\isacharparenleft}{\kern0pt}rule{\isacharunderscore}{\kern0pt}tac\ MVsetI{\isacharparenright}{\kern0pt}\isanewline
\ \ \ \ \ \ \ \isacommand{using}\isamarkupfalse%
\ weq\ \isanewline
\ \ \ \ \ \ \ \ \isacommand{apply}\isamarkupfalse%
\ simp\isanewline
\ \ \ \ \ \ \ \isacommand{apply}\isamarkupfalse%
\ {\isacharparenleft}{\kern0pt}rule{\isacharunderscore}{\kern0pt}tac\ lt{\isacharunderscore}{\kern0pt}succ{\isacharunderscore}{\kern0pt}lt{\isacharparenright}{\kern0pt}\ \isanewline
\ \ \ \ \ \ \ \isacommand{using}\isamarkupfalse%
\ Ord{\isacharunderscore}{\kern0pt}rank\ \isanewline
\ \ \ \ \ \ \ \ \isacommand{apply}\isamarkupfalse%
\ simp\isanewline
\ \ \ \ \ \ \ \isacommand{apply}\isamarkupfalse%
\ {\isacharparenleft}{\kern0pt}rule{\isacharunderscore}{\kern0pt}tac\ j{\isacharequal}{\kern0pt}{\isachardoublequoteopen}rank{\isacharparenleft}{\kern0pt}pi{\isacharparenright}{\kern0pt}{\isachardoublequoteclose}\ \isakeyword{in}\ lt{\isacharunderscore}{\kern0pt}trans{\isacharparenright}{\kern0pt}\ \isanewline
\ \ \ \ \ \ \ \isacommand{apply}\isamarkupfalse%
\ {\isacharparenleft}{\kern0pt}rule{\isacharunderscore}{\kern0pt}tac\ P{\isacharequal}{\kern0pt}{\isachardoublequoteopen}{\isasymlambda}x{\isachardot}{\kern0pt}\ rank{\isacharparenleft}{\kern0pt}x{\isacharparenright}{\kern0pt}\ {\isacharless}{\kern0pt}\ rank{\isacharparenleft}{\kern0pt}pi{\isacharparenright}{\kern0pt}{\isachardoublequoteclose}\ \isakeyword{and}\ a{\isacharequal}{\kern0pt}w\ \isakeyword{in}\ ssubst{\isacharparenright}{\kern0pt}\isanewline
\ \ \ \ \ \ \ \isacommand{using}\isamarkupfalse%
\ weq\ \isanewline
\ \ \ \ \ \ \ \ \ \isacommand{apply}\isamarkupfalse%
\ simp\isanewline
\ \ \ \ \ \ \ \isacommand{apply}\isamarkupfalse%
\ {\isacharparenleft}{\kern0pt}rule{\isacharunderscore}{\kern0pt}tac\ j{\isacharequal}{\kern0pt}{\isachardoublequoteopen}rank{\isacharparenleft}{\kern0pt}{\isacharless}{\kern0pt}p{\isacharcomma}{\kern0pt}\ w{\isachargreater}{\kern0pt}{\isacharparenright}{\kern0pt}{\isachardoublequoteclose}\ \isakeyword{in}\ lt{\isacharunderscore}{\kern0pt}trans{\isacharparenright}{\kern0pt}\isanewline
\ \ \ \ \ \ \ \isacommand{apply}\isamarkupfalse%
\ {\isacharparenleft}{\kern0pt}rule{\isacharunderscore}{\kern0pt}tac\ rank{\isacharunderscore}{\kern0pt}pair{\isadigit{2}}{\isacharparenright}{\kern0pt}\isanewline
\ \ \ \ \ \ \ \isacommand{using}\isamarkupfalse%
\ rank{\isacharunderscore}{\kern0pt}lt\ wh\ \isanewline
\ \ \ \ \ \ \ \ \isacommand{apply}\isamarkupfalse%
\ simp\ \isanewline
\ \ \ \ \ \ \ \isacommand{using}\isamarkupfalse%
\ H\ rank{\isacharunderscore}{\kern0pt}pair{\isadigit{2}}\ \isanewline
\ \ \ \ \ \ \ \isacommand{by}\isamarkupfalse%
\ auto\isanewline
\ \ \ \isacommand{qed}\isamarkupfalse%
\isanewline
\isanewline
\ \ \ \isacommand{then}\isamarkupfalse%
\ \isacommand{show}\isamarkupfalse%
\ {\isachardoublequoteopen}elem\ {\isasymin}\ MVset{\isacharparenleft}{\kern0pt}succ{\isacharparenleft}{\kern0pt}rank{\isacharparenleft}{\kern0pt}g{\isacharparenright}{\kern0pt}{\isacharparenright}{\kern0pt}{\isacharparenright}{\kern0pt}\ {\isasymtimes}\ MVset{\isacharparenleft}{\kern0pt}succ{\isacharparenleft}{\kern0pt}rank{\isacharparenleft}{\kern0pt}x{\isacharunderscore}{\kern0pt}pi{\isacharparenright}{\kern0pt}{\isacharparenright}{\kern0pt}{\isacharparenright}{\kern0pt}{\isachardoublequoteclose}\isanewline
\ \ \ \ \ \isacommand{using}\isamarkupfalse%
\ P{\isadigit{1}}\ H\ \isacommand{by}\isamarkupfalse%
\ auto\ \isanewline
\ \isacommand{next}\isamarkupfalse%
\isanewline
\ \ \ \isacommand{assume}\isamarkupfalse%
\ {\isachardoublequoteopen}x{\isacharunderscore}{\kern0pt}pi\ {\isasymin}\ M{\isachardoublequoteclose}\ {\isachardoublequoteopen}g\ {\isasymin}\ M{\isachardoublequoteclose}\isanewline
\ \ \ \isacommand{then}\isamarkupfalse%
\ \isacommand{have}\isamarkupfalse%
\ {\isachardoublequoteopen}{\isacharparenleft}{\kern0pt}{\isacharhash}{\kern0pt}{\isacharhash}{\kern0pt}M{\isacharparenright}{\kern0pt}{\isacharparenleft}{\kern0pt}MVset{\isacharparenleft}{\kern0pt}succ{\isacharparenleft}{\kern0pt}rank{\isacharparenleft}{\kern0pt}g{\isacharparenright}{\kern0pt}{\isacharparenright}{\kern0pt}{\isacharparenright}{\kern0pt}\ {\isasymtimes}\ MVset{\isacharparenleft}{\kern0pt}succ{\isacharparenleft}{\kern0pt}rank{\isacharparenleft}{\kern0pt}x{\isacharunderscore}{\kern0pt}pi{\isacharparenright}{\kern0pt}{\isacharparenright}{\kern0pt}{\isacharparenright}{\kern0pt}{\isacharparenright}{\kern0pt}{\isachardoublequoteclose}\ \isanewline
\ \ \ \ \ \isacommand{apply}\isamarkupfalse%
\ {\isacharparenleft}{\kern0pt}rule{\isacharunderscore}{\kern0pt}tac\ cartprod{\isacharunderscore}{\kern0pt}closed{\isacharparenright}{\kern0pt}\isanewline
\ \ \ \ \ \isacommand{apply}\isamarkupfalse%
\ simp{\isacharunderscore}{\kern0pt}all\ \isanewline
\ \ \ \ \ \isacommand{apply}\isamarkupfalse%
\ {\isacharparenleft}{\kern0pt}rule{\isacharunderscore}{\kern0pt}tac\ MVset{\isacharunderscore}{\kern0pt}in{\isacharunderscore}{\kern0pt}M{\isacharparenright}{\kern0pt}\ \isanewline
\ \ \ \ \ \isacommand{using}\isamarkupfalse%
\ rank{\isacharunderscore}{\kern0pt}closed\ Ord{\isacharunderscore}{\kern0pt}rank\ succ{\isacharunderscore}{\kern0pt}in{\isacharunderscore}{\kern0pt}MI\ Ord{\isacharunderscore}{\kern0pt}succ\ \isanewline
\ \ \ \ \ \ \ \isacommand{apply}\isamarkupfalse%
\ simp{\isacharunderscore}{\kern0pt}all\isanewline
\ \ \ \ \ \isacommand{apply}\isamarkupfalse%
\ {\isacharparenleft}{\kern0pt}rule{\isacharunderscore}{\kern0pt}tac\ MVset{\isacharunderscore}{\kern0pt}in{\isacharunderscore}{\kern0pt}M{\isacharparenright}{\kern0pt}\ \isanewline
\ \ \ \ \ \isacommand{using}\isamarkupfalse%
\ rank{\isacharunderscore}{\kern0pt}closed\ Ord{\isacharunderscore}{\kern0pt}rank\ succ{\isacharunderscore}{\kern0pt}in{\isacharunderscore}{\kern0pt}MI\ Ord{\isacharunderscore}{\kern0pt}succ\ \isanewline
\ \ \ \ \ \ \isacommand{apply}\isamarkupfalse%
\ simp{\isacharunderscore}{\kern0pt}all\isanewline
\ \ \ \ \ \isacommand{done}\isamarkupfalse%
\isanewline
\ \ \ \isacommand{then}\isamarkupfalse%
\ \isacommand{show}\isamarkupfalse%
\ {\isachardoublequoteopen}{\isacharparenleft}{\kern0pt}MVset{\isacharparenleft}{\kern0pt}succ{\isacharparenleft}{\kern0pt}rank{\isacharparenleft}{\kern0pt}g{\isacharparenright}{\kern0pt}{\isacharparenright}{\kern0pt}{\isacharparenright}{\kern0pt}\ {\isasymtimes}\ MVset{\isacharparenleft}{\kern0pt}succ{\isacharparenleft}{\kern0pt}rank{\isacharparenleft}{\kern0pt}x{\isacharunderscore}{\kern0pt}pi{\isacharparenright}{\kern0pt}{\isacharparenright}{\kern0pt}{\isacharparenright}{\kern0pt}{\isacharparenright}{\kern0pt}\ {\isasymin}\ M{\isachardoublequoteclose}\ \isacommand{by}\isamarkupfalse%
\ auto\isanewline
\ \isacommand{qed}\isamarkupfalse%
%
\endisatagproof
{\isafoldproof}%
%
\isadelimproof
\isanewline
%
\endisadelimproof
\isanewline
\isacommand{schematic{\isacharunderscore}{\kern0pt}goal}\isamarkupfalse%
\ HPn{\isacharunderscore}{\kern0pt}auto{\isacharunderscore}{\kern0pt}M{\isacharunderscore}{\kern0pt}fm{\isacharunderscore}{\kern0pt}auto{\isacharcolon}{\kern0pt}\isanewline
\ \ \isakeyword{assumes}\isanewline
\ \ \ \ {\isachardoublequoteopen}nth{\isacharparenleft}{\kern0pt}{\isadigit{0}}{\isacharcomma}{\kern0pt}env{\isacharparenright}{\kern0pt}\ {\isacharequal}{\kern0pt}\ elem{\isachardoublequoteclose}\ \isanewline
\ \ \ \ {\isachardoublequoteopen}nth{\isacharparenleft}{\kern0pt}{\isadigit{1}}{\isacharcomma}{\kern0pt}env{\isacharparenright}{\kern0pt}\ {\isacharequal}{\kern0pt}\ x{\isacharunderscore}{\kern0pt}pi{\isachardoublequoteclose}\ \isanewline
\ \ \ \ {\isachardoublequoteopen}nth{\isacharparenleft}{\kern0pt}{\isadigit{2}}{\isacharcomma}{\kern0pt}env{\isacharparenright}{\kern0pt}\ {\isacharequal}{\kern0pt}\ g{\isachardoublequoteclose}\ \ \isanewline
\ \ \ \ {\isachardoublequoteopen}env\ {\isasymin}\ list{\isacharparenleft}{\kern0pt}M{\isacharparenright}{\kern0pt}{\isachardoublequoteclose}\isanewline
\ \isakeyword{shows}\isanewline
\ \ \ \ {\isachardoublequoteopen}HPn{\isacharunderscore}{\kern0pt}auto{\isacharunderscore}{\kern0pt}M{\isacharunderscore}{\kern0pt}cond{\isacharparenleft}{\kern0pt}x{\isacharunderscore}{\kern0pt}pi{\isacharcomma}{\kern0pt}\ g{\isacharcomma}{\kern0pt}\ elem{\isacharparenright}{\kern0pt}\isanewline
\ \ \ \ \ {\isasymlongleftrightarrow}\ sats{\isacharparenleft}{\kern0pt}M{\isacharcomma}{\kern0pt}{\isacharquery}{\kern0pt}fm{\isacharparenleft}{\kern0pt}{\isadigit{0}}{\isacharcomma}{\kern0pt}{\isadigit{1}}{\isacharcomma}{\kern0pt}{\isadigit{2}}{\isacharparenright}{\kern0pt}{\isacharcomma}{\kern0pt}env{\isacharparenright}{\kern0pt}{\isachardoublequoteclose}\isanewline
%
\isadelimproof
\ \ %
\endisadelimproof
%
\isatagproof
\isacommand{unfolding}\isamarkupfalse%
\ HPn{\isacharunderscore}{\kern0pt}auto{\isacharunderscore}{\kern0pt}M{\isacharunderscore}{\kern0pt}cond{\isacharunderscore}{\kern0pt}def\isanewline
\ \ \isacommand{by}\isamarkupfalse%
\ {\isacharparenleft}{\kern0pt}insert\ assms\ {\isacharsemicolon}{\kern0pt}\ {\isacharparenleft}{\kern0pt}rule\ sep{\isacharunderscore}{\kern0pt}rules\ {\isacharbar}{\kern0pt}\ simp{\isacharparenright}{\kern0pt}{\isacharplus}{\kern0pt}{\isacharparenright}{\kern0pt}%
\endisatagproof
{\isafoldproof}%
%
\isadelimproof
\isanewline
%
\endisadelimproof
\isanewline
\isacommand{end}\isamarkupfalse%
\isanewline
\isanewline
\isacommand{definition}\isamarkupfalse%
\ HPn{\isacharunderscore}{\kern0pt}auto{\isacharunderscore}{\kern0pt}M{\isacharunderscore}{\kern0pt}fm\ \isakeyword{where}\ \isanewline
\ \ {\isachardoublequoteopen}HPn{\isacharunderscore}{\kern0pt}auto{\isacharunderscore}{\kern0pt}M{\isacharunderscore}{\kern0pt}fm\ {\isacharequal}{\kern0pt}\ \ Exists\isanewline
\ \ \ \ \ \ \ \ \ \ \ \ \ {\isacharparenleft}{\kern0pt}Exists\isanewline
\ \ \ \ \ \ \ \ \ \ \ \ \ \ \ {\isacharparenleft}{\kern0pt}Exists\isanewline
\ \ \ \ \ \ \ \ \ \ \ \ \ \ \ \ \ {\isacharparenleft}{\kern0pt}Exists\isanewline
\ \ \ \ \ \ \ \ \ \ \ \ \ \ \ \ \ \ \ {\isacharparenleft}{\kern0pt}Exists\isanewline
\ \ \ \ \ \ \ \ \ \ \ \ \ \ \ \ \ \ \ \ \ {\isacharparenleft}{\kern0pt}Exists\isanewline
\ \ \ \ \ \ \ \ \ \ \ \ \ \ \ \ \ \ \ \ \ \ \ {\isacharparenleft}{\kern0pt}Exists\isanewline
\ \ \ \ \ \ \ \ \ \ \ \ \ \ \ \ \ \ \ \ \ \ \ \ \ {\isacharparenleft}{\kern0pt}Exists\isanewline
\ \ \ \ \ \ \ \ \ \ \ \ \ \ \ \ \ \ \ \ \ \ \ \ \ \ \ {\isacharparenleft}{\kern0pt}And{\isacharparenleft}{\kern0pt}pair{\isacharunderscore}{\kern0pt}fm{\isacharparenleft}{\kern0pt}{\isadigit{7}}{\isacharcomma}{\kern0pt}\ {\isadigit{6}}{\isacharcomma}{\kern0pt}\ {\isadigit{8}}{\isacharparenright}{\kern0pt}{\isacharcomma}{\kern0pt}\isanewline
\ \ \ \ \ \ \ \ \ \ \ \ \ \ \ \ \ \ \ \ \ \ \ \ \ \ \ \ \ \ \ \ And{\isacharparenleft}{\kern0pt}pair{\isacharunderscore}{\kern0pt}fm{\isacharparenleft}{\kern0pt}{\isadigit{1}}{\isacharcomma}{\kern0pt}\ {\isadigit{0}}{\isacharcomma}{\kern0pt}\ {\isadigit{9}}{\isacharparenright}{\kern0pt}{\isacharcomma}{\kern0pt}\isanewline
\ \ \ \ \ \ \ \ \ \ \ \ \ \ \ \ \ \ \ \ \ \ \ \ \ \ \ \ \ \ \ \ \ \ \ \ And\isanewline
{\isacharparenleft}{\kern0pt}pair{\isacharunderscore}{\kern0pt}fm{\isacharparenleft}{\kern0pt}{\isadigit{3}}{\isacharcomma}{\kern0pt}\ {\isadigit{0}}{\isacharcomma}{\kern0pt}\ {\isadigit{4}}{\isacharparenright}{\kern0pt}{\isacharcomma}{\kern0pt}\isanewline
\ And{\isacharparenleft}{\kern0pt}pair{\isacharunderscore}{\kern0pt}fm{\isacharparenleft}{\kern0pt}{\isadigit{3}}{\isacharcomma}{\kern0pt}\ {\isadigit{2}}{\isacharcomma}{\kern0pt}\ {\isadigit{5}}{\isacharparenright}{\kern0pt}{\isacharcomma}{\kern0pt}\isanewline
\ \ \ \ \ And{\isacharparenleft}{\kern0pt}Member{\isacharparenleft}{\kern0pt}{\isadigit{5}}{\isacharcomma}{\kern0pt}\ {\isadigit{1}}{\isacharparenright}{\kern0pt}{\isacharcomma}{\kern0pt}\isanewline
\ \ \ \ \ \ \ \ \ And{\isacharparenleft}{\kern0pt}function{\isacharunderscore}{\kern0pt}fm{\isacharparenleft}{\kern0pt}{\isadigit{0}}{\isacharparenright}{\kern0pt}{\isacharcomma}{\kern0pt}\isanewline
\ \ \ \ \ \ \ \ \ \ \ \ \ And{\isacharparenleft}{\kern0pt}fun{\isacharunderscore}{\kern0pt}apply{\isacharunderscore}{\kern0pt}fm{\isacharparenleft}{\kern0pt}{\isadigit{0}}{\isacharcomma}{\kern0pt}\ {\isadigit{2}}{\isacharcomma}{\kern0pt}\ {\isadigit{6}}{\isacharparenright}{\kern0pt}{\isacharcomma}{\kern0pt}\isanewline
\ \ \ \ \ \ \ \ \ \ \ \ \ \ \ \ \ fun{\isacharunderscore}{\kern0pt}apply{\isacharunderscore}{\kern0pt}fm{\isacharparenleft}{\kern0pt}{\isadigit{1}}{\isadigit{0}}{\isacharcomma}{\kern0pt}\ {\isadigit{4}}{\isacharcomma}{\kern0pt}\ {\isadigit{7}}{\isacharparenright}{\kern0pt}{\isacharparenright}{\kern0pt}{\isacharparenright}{\kern0pt}{\isacharparenright}{\kern0pt}{\isacharparenright}{\kern0pt}{\isacharparenright}{\kern0pt}{\isacharparenright}{\kern0pt}{\isacharparenright}{\kern0pt}{\isacharparenright}{\kern0pt}{\isacharparenright}{\kern0pt}{\isacharparenright}{\kern0pt}{\isacharparenright}{\kern0pt}{\isacharparenright}{\kern0pt}{\isacharparenright}{\kern0pt}{\isacharparenright}{\kern0pt}{\isacharparenright}{\kern0pt}{\isachardoublequoteclose}\isanewline
\isanewline
\isacommand{context}\isamarkupfalse%
\ forcing{\isacharunderscore}{\kern0pt}data{\isacharunderscore}{\kern0pt}partial\ \isanewline
\isakeyword{begin}\ \isanewline
\isanewline
\isacommand{schematic{\isacharunderscore}{\kern0pt}goal}\isamarkupfalse%
\ HPn{\isacharunderscore}{\kern0pt}auto{\isacharunderscore}{\kern0pt}M{\isacharunderscore}{\kern0pt}fm{\isacharprime}{\kern0pt}{\isacharunderscore}{\kern0pt}auto{\isacharcolon}{\kern0pt}\isanewline
\ \ \isakeyword{assumes}\isanewline
\ \ \ \ {\isachardoublequoteopen}nth{\isacharparenleft}{\kern0pt}{\isadigit{0}}{\isacharcomma}{\kern0pt}env{\isacharparenright}{\kern0pt}\ {\isacharequal}{\kern0pt}\ z{\isachardoublequoteclose}\ \isanewline
\ \ \ \ {\isachardoublequoteopen}nth{\isacharparenleft}{\kern0pt}{\isadigit{1}}{\isacharcomma}{\kern0pt}env{\isacharparenright}{\kern0pt}\ {\isacharequal}{\kern0pt}\ g{\isachardoublequoteclose}\ \isanewline
\ \ \ \ {\isachardoublequoteopen}nth{\isacharparenleft}{\kern0pt}{\isadigit{2}}{\isacharcomma}{\kern0pt}env{\isacharparenright}{\kern0pt}\ {\isacharequal}{\kern0pt}\ x{\isacharunderscore}{\kern0pt}pi{\isachardoublequoteclose}\ \ \isanewline
\ \ \ \ {\isachardoublequoteopen}env\ {\isasymin}\ list{\isacharparenleft}{\kern0pt}M{\isacharparenright}{\kern0pt}{\isachardoublequoteclose}\isanewline
\ \isakeyword{shows}\isanewline
\ \ \ \ {\isachardoublequoteopen}{\isacharparenleft}{\kern0pt}{\isasymforall}elem\ {\isasymin}\ M{\isachardot}{\kern0pt}\ elem\ {\isasymin}\ z\ {\isasymlongleftrightarrow}\ HPn{\isacharunderscore}{\kern0pt}auto{\isacharunderscore}{\kern0pt}M{\isacharunderscore}{\kern0pt}cond{\isacharparenleft}{\kern0pt}x{\isacharunderscore}{\kern0pt}pi{\isacharcomma}{\kern0pt}\ g{\isacharcomma}{\kern0pt}\ elem{\isacharparenright}{\kern0pt}{\isacharparenright}{\kern0pt}\isanewline
\ \ \ \ \ {\isasymlongleftrightarrow}\ sats{\isacharparenleft}{\kern0pt}M{\isacharcomma}{\kern0pt}{\isacharquery}{\kern0pt}fm{\isacharparenleft}{\kern0pt}{\isadigit{0}}{\isacharcomma}{\kern0pt}{\isadigit{1}}{\isacharcomma}{\kern0pt}{\isadigit{2}}{\isacharparenright}{\kern0pt}{\isacharcomma}{\kern0pt}env{\isacharparenright}{\kern0pt}{\isachardoublequoteclose}\isanewline
%
\isadelimproof
\ \ %
\endisadelimproof
%
\isatagproof
\isacommand{unfolding}\isamarkupfalse%
\ HPn{\isacharunderscore}{\kern0pt}auto{\isacharunderscore}{\kern0pt}M{\isacharunderscore}{\kern0pt}cond{\isacharunderscore}{\kern0pt}def\isanewline
\ \ \isacommand{by}\isamarkupfalse%
\ {\isacharparenleft}{\kern0pt}insert\ assms\ {\isacharsemicolon}{\kern0pt}\ {\isacharparenleft}{\kern0pt}rule\ sep{\isacharunderscore}{\kern0pt}rules\ {\isacharbar}{\kern0pt}\ simp{\isacharparenright}{\kern0pt}{\isacharplus}{\kern0pt}{\isacharparenright}{\kern0pt}%
\endisatagproof
{\isafoldproof}%
%
\isadelimproof
\isanewline
%
\endisadelimproof
\isanewline
\isacommand{end}\isamarkupfalse%
\isanewline
\isanewline
\isacommand{definition}\isamarkupfalse%
\ HPn{\isacharunderscore}{\kern0pt}auto{\isacharunderscore}{\kern0pt}M{\isacharunderscore}{\kern0pt}fm{\isacharprime}{\kern0pt}\ \isakeyword{where}\ \isanewline
\ \ {\isachardoublequoteopen}HPn{\isacharunderscore}{\kern0pt}auto{\isacharunderscore}{\kern0pt}M{\isacharunderscore}{\kern0pt}fm{\isacharprime}{\kern0pt}\ {\isasymequiv}\ \isanewline
\ \ Forall\isanewline
\ \ \ \ \ \ \ \ \ \ \ \ \ {\isacharparenleft}{\kern0pt}Iff{\isacharparenleft}{\kern0pt}Member{\isacharparenleft}{\kern0pt}{\isadigit{0}}{\isacharcomma}{\kern0pt}\ {\isadigit{1}}{\isacharparenright}{\kern0pt}{\isacharcomma}{\kern0pt}\isanewline
\ \ \ \ \ \ \ \ \ \ \ \ \ \ \ \ \ \ Exists\isanewline
\ \ \ \ \ \ \ \ \ \ \ \ \ \ \ \ \ \ \ {\isacharparenleft}{\kern0pt}Exists\isanewline
\ \ \ \ \ \ \ \ \ \ \ \ \ \ \ \ \ \ \ \ \ {\isacharparenleft}{\kern0pt}Exists\isanewline
\ \ \ \ \ \ \ \ \ \ \ \ \ \ \ \ \ \ \ \ \ \ \ {\isacharparenleft}{\kern0pt}Exists\isanewline
\ \ \ \ \ \ \ \ \ \ \ \ \ \ \ \ \ \ \ \ \ \ \ \ \ {\isacharparenleft}{\kern0pt}Exists\isanewline
\ \ \ \ \ \ \ \ \ \ \ \ \ \ \ \ \ \ \ \ \ \ \ \ \ \ \ {\isacharparenleft}{\kern0pt}Exists\isanewline
\ \ \ \ \ \ \ \ \ \ \ \ \ \ \ \ \ \ \ \ \ \ \ \ \ \ \ \ \ {\isacharparenleft}{\kern0pt}Exists\isanewline
\ \ \ \ \ \ \ \ \ \ \ \ \ \ \ \ \ \ \ \ \ \ \ \ \ \ \ \ \ \ \ {\isacharparenleft}{\kern0pt}Exists\isanewline
\ \ \ \ \ \ \ \ \ \ \ \ \ \ \ \ \ \ \ \ \ \ \ \ \ \ \ \ \ \ \ \ \ {\isacharparenleft}{\kern0pt}And{\isacharparenleft}{\kern0pt}pair{\isacharunderscore}{\kern0pt}fm{\isacharparenleft}{\kern0pt}{\isadigit{7}}{\isacharcomma}{\kern0pt}\ {\isadigit{6}}{\isacharcomma}{\kern0pt}\ {\isadigit{8}}{\isacharparenright}{\kern0pt}{\isacharcomma}{\kern0pt}\isanewline
\ \ \ \ \ \ \ \ \ \ \ \ \ \ \ \ \ \ \ \ \ \ \ \ \ \ \ \ \ \ \ \ \ \ \ \ \ \ And{\isacharparenleft}{\kern0pt}pair{\isacharunderscore}{\kern0pt}fm{\isacharparenleft}{\kern0pt}{\isadigit{1}}{\isacharcomma}{\kern0pt}\ {\isadigit{0}}{\isacharcomma}{\kern0pt}\ {\isadigit{1}}{\isadigit{1}}{\isacharparenright}{\kern0pt}{\isacharcomma}{\kern0pt}\isanewline
\ \ \ \ \ \ \ \ \ \ \ \ \ \ \ \ \ \ \ \ \ \ \ \ \ \ \ \ \ \ \ \ \ \ \ \ \ \ \ \ \ \ And{\isacharparenleft}{\kern0pt}pair{\isacharunderscore}{\kern0pt}fm{\isacharparenleft}{\kern0pt}{\isadigit{3}}{\isacharcomma}{\kern0pt}\ {\isadigit{0}}{\isacharcomma}{\kern0pt}\ {\isadigit{4}}{\isacharparenright}{\kern0pt}{\isacharcomma}{\kern0pt}\isanewline
\ \ \ \ \ \ \ \ \ \ \ \ \ \ \ \ \ \ \ \ \ \ \ \ \ \ \ \ \ \ \ \ \ \ \ \ \ \ \ \ \ \ \ \ \ \ And{\isacharparenleft}{\kern0pt}pair{\isacharunderscore}{\kern0pt}fm{\isacharparenleft}{\kern0pt}{\isadigit{3}}{\isacharcomma}{\kern0pt}\ {\isadigit{2}}{\isacharcomma}{\kern0pt}\ {\isadigit{5}}{\isacharparenright}{\kern0pt}{\isacharcomma}{\kern0pt}\isanewline
\ And{\isacharparenleft}{\kern0pt}Member{\isacharparenleft}{\kern0pt}{\isadigit{5}}{\isacharcomma}{\kern0pt}\ {\isadigit{1}}{\isacharparenright}{\kern0pt}{\isacharcomma}{\kern0pt}\isanewline
\ \ \ \ \ And{\isacharparenleft}{\kern0pt}function{\isacharunderscore}{\kern0pt}fm{\isacharparenleft}{\kern0pt}{\isadigit{0}}{\isacharparenright}{\kern0pt}{\isacharcomma}{\kern0pt}\ And{\isacharparenleft}{\kern0pt}fun{\isacharunderscore}{\kern0pt}apply{\isacharunderscore}{\kern0pt}fm{\isacharparenleft}{\kern0pt}{\isadigit{0}}{\isacharcomma}{\kern0pt}\ {\isadigit{2}}{\isacharcomma}{\kern0pt}\ {\isadigit{6}}{\isacharparenright}{\kern0pt}{\isacharcomma}{\kern0pt}\ fun{\isacharunderscore}{\kern0pt}apply{\isacharunderscore}{\kern0pt}fm{\isacharparenleft}{\kern0pt}{\isadigit{1}}{\isadigit{0}}{\isacharcomma}{\kern0pt}\ {\isadigit{4}}{\isacharcomma}{\kern0pt}\ {\isadigit{7}}{\isacharparenright}{\kern0pt}{\isacharparenright}{\kern0pt}{\isacharparenright}{\kern0pt}{\isacharparenright}{\kern0pt}{\isacharparenright}{\kern0pt}{\isacharparenright}{\kern0pt}{\isacharparenright}{\kern0pt}{\isacharparenright}{\kern0pt}{\isacharparenright}{\kern0pt}{\isacharparenright}{\kern0pt}{\isacharparenright}{\kern0pt}{\isacharparenright}{\kern0pt}{\isacharparenright}{\kern0pt}{\isacharparenright}{\kern0pt}{\isacharparenright}{\kern0pt}{\isacharparenright}{\kern0pt}{\isacharparenright}{\kern0pt}{\isacharparenright}{\kern0pt}\ {\isachardoublequoteclose}\ \isanewline
\isanewline
\isanewline
\isacommand{context}\isamarkupfalse%
\ forcing{\isacharunderscore}{\kern0pt}data{\isacharunderscore}{\kern0pt}partial\ \isanewline
\isakeyword{begin}\ \isanewline
\isanewline
\isacommand{definition}\isamarkupfalse%
\ HPn{\isacharunderscore}{\kern0pt}auto{\isacharunderscore}{\kern0pt}M\ \isakeyword{where}\ \isanewline
\ \ {\isachardoublequoteopen}HPn{\isacharunderscore}{\kern0pt}auto{\isacharunderscore}{\kern0pt}M{\isacharparenleft}{\kern0pt}x{\isacharunderscore}{\kern0pt}pi{\isacharcomma}{\kern0pt}\ g{\isacharparenright}{\kern0pt}\ {\isasymequiv}\ {\isacharbraceleft}{\kern0pt}\ elem\ {\isasymin}\ M{\isachardot}{\kern0pt}\ HPn{\isacharunderscore}{\kern0pt}auto{\isacharunderscore}{\kern0pt}M{\isacharunderscore}{\kern0pt}cond{\isacharparenleft}{\kern0pt}x{\isacharunderscore}{\kern0pt}pi{\isacharcomma}{\kern0pt}\ g{\isacharcomma}{\kern0pt}\ elem{\isacharparenright}{\kern0pt}\ {\isacharbraceright}{\kern0pt}{\isachardoublequoteclose}\isanewline
\isanewline
\isacommand{lemma}\isamarkupfalse%
\ HPn{\isacharunderscore}{\kern0pt}auto{\isacharunderscore}{\kern0pt}M{\isacharunderscore}{\kern0pt}fm{\isacharunderscore}{\kern0pt}sats{\isacharunderscore}{\kern0pt}iff\ {\isacharcolon}{\kern0pt}\ \isanewline
\ \ {\isachardoublequoteopen}elem\ {\isasymin}\ M\ {\isasymLongrightarrow}\ x{\isacharunderscore}{\kern0pt}pi\ {\isasymin}\ M\ {\isasymLongrightarrow}\ g\ {\isasymin}\ M\ {\isasymLongrightarrow}\ \isanewline
\ \ sats{\isacharparenleft}{\kern0pt}M{\isacharcomma}{\kern0pt}\ HPn{\isacharunderscore}{\kern0pt}auto{\isacharunderscore}{\kern0pt}M{\isacharunderscore}{\kern0pt}fm{\isacharcomma}{\kern0pt}\ {\isacharbrackleft}{\kern0pt}elem{\isacharcomma}{\kern0pt}\ x{\isacharunderscore}{\kern0pt}pi{\isacharcomma}{\kern0pt}\ g{\isacharbrackright}{\kern0pt}{\isacharparenright}{\kern0pt}\ {\isasymlongleftrightarrow}\ HPn{\isacharunderscore}{\kern0pt}auto{\isacharunderscore}{\kern0pt}M{\isacharunderscore}{\kern0pt}cond{\isacharparenleft}{\kern0pt}x{\isacharunderscore}{\kern0pt}pi{\isacharcomma}{\kern0pt}\ g{\isacharcomma}{\kern0pt}\ elem{\isacharparenright}{\kern0pt}{\isachardoublequoteclose}\ \isanewline
%
\isadelimproof
\ \ %
\endisadelimproof
%
\isatagproof
\isacommand{apply}\isamarkupfalse%
\ {\isacharparenleft}{\kern0pt}rule\ iff{\isacharunderscore}{\kern0pt}flip{\isacharparenright}{\kern0pt}\ \isanewline
\ \ \isacommand{unfolding}\isamarkupfalse%
\ HPn{\isacharunderscore}{\kern0pt}auto{\isacharunderscore}{\kern0pt}M{\isacharunderscore}{\kern0pt}fm{\isacharunderscore}{\kern0pt}def\ \isanewline
\ \ \isacommand{apply}\isamarkupfalse%
\ {\isacharparenleft}{\kern0pt}rule{\isacharunderscore}{\kern0pt}tac\ HPn{\isacharunderscore}{\kern0pt}auto{\isacharunderscore}{\kern0pt}M{\isacharunderscore}{\kern0pt}fm{\isacharunderscore}{\kern0pt}auto{\isacharparenright}{\kern0pt}\isanewline
\ \ \isacommand{by}\isamarkupfalse%
\ auto%
\endisatagproof
{\isafoldproof}%
%
\isadelimproof
\isanewline
%
\endisadelimproof
\isanewline
\isacommand{lemma}\isamarkupfalse%
\ HPn{\isacharunderscore}{\kern0pt}auto{\isacharunderscore}{\kern0pt}M{\isacharunderscore}{\kern0pt}fm{\isacharprime}{\kern0pt}{\isacharunderscore}{\kern0pt}sats{\isacharunderscore}{\kern0pt}iff\ {\isacharcolon}{\kern0pt}\ \isanewline
\ \ {\isachardoublequoteopen}x{\isacharunderscore}{\kern0pt}pi\ {\isasymin}\ M\ {\isasymLongrightarrow}\ g\ {\isasymin}\ M\ {\isasymLongrightarrow}\ z\ {\isasymin}\ M\ {\isasymLongrightarrow}\ env\ {\isasymin}\ list{\isacharparenleft}{\kern0pt}M{\isacharparenright}{\kern0pt}\ {\isasymLongrightarrow}\ \isanewline
\ \ sats{\isacharparenleft}{\kern0pt}M{\isacharcomma}{\kern0pt}\ HPn{\isacharunderscore}{\kern0pt}auto{\isacharunderscore}{\kern0pt}M{\isacharunderscore}{\kern0pt}fm{\isacharprime}{\kern0pt}{\isacharcomma}{\kern0pt}\ {\isacharbrackleft}{\kern0pt}z{\isacharcomma}{\kern0pt}\ g{\isacharcomma}{\kern0pt}\ x{\isacharunderscore}{\kern0pt}pi{\isacharbrackright}{\kern0pt}\ {\isacharat}{\kern0pt}\ env{\isacharparenright}{\kern0pt}\ {\isasymlongleftrightarrow}\ z\ {\isacharequal}{\kern0pt}\ HPn{\isacharunderscore}{\kern0pt}auto{\isacharunderscore}{\kern0pt}M{\isacharparenleft}{\kern0pt}x{\isacharunderscore}{\kern0pt}pi{\isacharcomma}{\kern0pt}\ g{\isacharparenright}{\kern0pt}{\isachardoublequoteclose}\ \isanewline
%
\isadelimproof
\ \ %
\endisadelimproof
%
\isatagproof
\isacommand{apply}\isamarkupfalse%
\ {\isacharparenleft}{\kern0pt}rule{\isacharunderscore}{\kern0pt}tac\ Q{\isacharequal}{\kern0pt}{\isachardoublequoteopen}{\isasymforall}elem\ {\isasymin}\ M{\isachardot}{\kern0pt}\ elem\ {\isasymin}\ z\ {\isasymlongleftrightarrow}\ HPn{\isacharunderscore}{\kern0pt}auto{\isacharunderscore}{\kern0pt}M{\isacharunderscore}{\kern0pt}cond{\isacharparenleft}{\kern0pt}x{\isacharunderscore}{\kern0pt}pi{\isacharcomma}{\kern0pt}\ g{\isacharcomma}{\kern0pt}\ elem{\isacharparenright}{\kern0pt}{\isachardoublequoteclose}\ \isakeyword{in}\ iff{\isacharunderscore}{\kern0pt}trans{\isacharparenright}{\kern0pt}\ \isanewline
\ \ \isacommand{apply}\isamarkupfalse%
\ {\isacharparenleft}{\kern0pt}rule\ iff{\isacharunderscore}{\kern0pt}flip{\isacharparenright}{\kern0pt}\ \isanewline
\ \ \isacommand{unfolding}\isamarkupfalse%
\ HPn{\isacharunderscore}{\kern0pt}auto{\isacharunderscore}{\kern0pt}M{\isacharunderscore}{\kern0pt}fm{\isacharprime}{\kern0pt}{\isacharunderscore}{\kern0pt}def\ \isanewline
\ \ \ \isacommand{apply}\isamarkupfalse%
\ {\isacharparenleft}{\kern0pt}rule{\isacharunderscore}{\kern0pt}tac\ HPn{\isacharunderscore}{\kern0pt}auto{\isacharunderscore}{\kern0pt}M{\isacharunderscore}{\kern0pt}fm{\isacharprime}{\kern0pt}{\isacharunderscore}{\kern0pt}auto{\isacharparenright}{\kern0pt}\isanewline
\ \ \ \ \ \ \isacommand{apply}\isamarkupfalse%
\ simp{\isacharunderscore}{\kern0pt}all\ \isanewline
\ \ \isacommand{unfolding}\isamarkupfalse%
\ HPn{\isacharunderscore}{\kern0pt}auto{\isacharunderscore}{\kern0pt}M{\isacharunderscore}{\kern0pt}def\ \isanewline
\ \ \isacommand{using}\isamarkupfalse%
\ transM\ \isanewline
\ \ \isacommand{by}\isamarkupfalse%
\ auto%
\endisatagproof
{\isafoldproof}%
%
\isadelimproof
\isanewline
%
\endisadelimproof
\isanewline
\isacommand{lemma}\isamarkupfalse%
\ HPn{\isacharunderscore}{\kern0pt}auto{\isacharunderscore}{\kern0pt}M{\isacharunderscore}{\kern0pt}in{\isacharunderscore}{\kern0pt}M\ {\isacharcolon}{\kern0pt}\ \isanewline
\ \ {\isachardoublequoteopen}{\isasymAnd}x{\isacharunderscore}{\kern0pt}pi\ g{\isachardot}{\kern0pt}\ x{\isacharunderscore}{\kern0pt}pi\ {\isasymin}\ M\ {\isasymLongrightarrow}\ g\ {\isasymin}\ M\ {\isasymLongrightarrow}\ function{\isacharparenleft}{\kern0pt}g{\isacharparenright}{\kern0pt}\ {\isasymLongrightarrow}\ HPn{\isacharunderscore}{\kern0pt}auto{\isacharunderscore}{\kern0pt}M{\isacharparenleft}{\kern0pt}x{\isacharunderscore}{\kern0pt}pi{\isacharcomma}{\kern0pt}\ g{\isacharparenright}{\kern0pt}\ {\isasymin}\ M{\isachardoublequoteclose}\ \isanewline
%
\isadelimproof
%
\endisadelimproof
%
\isatagproof
\isacommand{proof}\isamarkupfalse%
\ {\isacharminus}{\kern0pt}\ \isanewline
\ \ \isacommand{fix}\isamarkupfalse%
\ x{\isacharunderscore}{\kern0pt}pi\ g\ \isanewline
\ \ \isacommand{assume}\isamarkupfalse%
\ assms\ {\isacharcolon}{\kern0pt}\ {\isachardoublequoteopen}x{\isacharunderscore}{\kern0pt}pi\ {\isasymin}\ M{\isachardoublequoteclose}\ {\isachardoublequoteopen}g\ {\isasymin}\ M{\isachardoublequoteclose}\ {\isachardoublequoteopen}function{\isacharparenleft}{\kern0pt}g{\isacharparenright}{\kern0pt}{\isachardoublequoteclose}\ \isanewline
\isanewline
\ \ \isacommand{then}\isamarkupfalse%
\ \isacommand{have}\isamarkupfalse%
\ {\isachardoublequoteopen}{\isasymexists}A{\isasymin}M{\isachardot}{\kern0pt}\ {\isasymforall}elem{\isasymin}M{\isachardot}{\kern0pt}\ HPn{\isacharunderscore}{\kern0pt}auto{\isacharunderscore}{\kern0pt}M{\isacharunderscore}{\kern0pt}cond{\isacharparenleft}{\kern0pt}x{\isacharunderscore}{\kern0pt}pi{\isacharcomma}{\kern0pt}\ g{\isacharcomma}{\kern0pt}\ elem{\isacharparenright}{\kern0pt}\ {\isasymlongrightarrow}\ elem\ {\isasymin}\ A{\isachardoublequoteclose}\ \isanewline
\ \ \ \isacommand{using}\isamarkupfalse%
\ HPn{\isacharunderscore}{\kern0pt}auto{\isacharunderscore}{\kern0pt}M{\isacharunderscore}{\kern0pt}cond{\isacharunderscore}{\kern0pt}elem{\isacharunderscore}{\kern0pt}in\ \isacommand{by}\isamarkupfalse%
\ auto\isanewline
\ \ \isacommand{then}\isamarkupfalse%
\ \isacommand{obtain}\isamarkupfalse%
\ A\ \isakeyword{where}\ Ah{\isacharcolon}{\kern0pt}\ {\isachardoublequoteopen}A\ {\isasymin}\ M{\isachardoublequoteclose}\ {\isachardoublequoteopen}{\isasymforall}\ elem\ {\isasymin}\ M{\isachardot}{\kern0pt}\ HPn{\isacharunderscore}{\kern0pt}auto{\isacharunderscore}{\kern0pt}M{\isacharunderscore}{\kern0pt}cond{\isacharparenleft}{\kern0pt}x{\isacharunderscore}{\kern0pt}pi{\isacharcomma}{\kern0pt}\ g{\isacharcomma}{\kern0pt}\ elem{\isacharparenright}{\kern0pt}\ {\isasymlongrightarrow}\ elem\ {\isasymin}\ A{\isachardoublequoteclose}\ \isacommand{by}\isamarkupfalse%
\ auto\isanewline
\isanewline
\ \ \isacommand{define}\isamarkupfalse%
\ S\ \isakeyword{where}\ {\isachardoublequoteopen}S\ {\isasymequiv}\ {\isacharbraceleft}{\kern0pt}\ elem\ {\isasymin}\ A{\isachardot}{\kern0pt}\ sats{\isacharparenleft}{\kern0pt}M{\isacharcomma}{\kern0pt}\ HPn{\isacharunderscore}{\kern0pt}auto{\isacharunderscore}{\kern0pt}M{\isacharunderscore}{\kern0pt}fm{\isacharcomma}{\kern0pt}\ {\isacharbrackleft}{\kern0pt}elem{\isacharbrackright}{\kern0pt}\ {\isacharat}{\kern0pt}\ {\isacharbrackleft}{\kern0pt}x{\isacharunderscore}{\kern0pt}pi{\isacharcomma}{\kern0pt}\ g{\isacharbrackright}{\kern0pt}{\isacharparenright}{\kern0pt}\ {\isacharbraceright}{\kern0pt}{\isachardoublequoteclose}\ \isanewline
\isanewline
\ \ \isacommand{have}\isamarkupfalse%
\ {\isachardoublequoteopen}separation{\isacharparenleft}{\kern0pt}{\isacharhash}{\kern0pt}{\isacharhash}{\kern0pt}M{\isacharcomma}{\kern0pt}\ {\isasymlambda}elem\ {\isachardot}{\kern0pt}\ sats{\isacharparenleft}{\kern0pt}M{\isacharcomma}{\kern0pt}\ HPn{\isacharunderscore}{\kern0pt}auto{\isacharunderscore}{\kern0pt}M{\isacharunderscore}{\kern0pt}fm{\isacharcomma}{\kern0pt}\ {\isacharbrackleft}{\kern0pt}elem{\isacharbrackright}{\kern0pt}\ {\isacharat}{\kern0pt}\ {\isacharbrackleft}{\kern0pt}x{\isacharunderscore}{\kern0pt}pi{\isacharcomma}{\kern0pt}\ g{\isacharbrackright}{\kern0pt}{\isacharparenright}{\kern0pt}{\isacharparenright}{\kern0pt}{\isachardoublequoteclose}\ \isanewline
\ \ \ \ \isacommand{apply}\isamarkupfalse%
\ {\isacharparenleft}{\kern0pt}rule{\isacharunderscore}{\kern0pt}tac\ separation{\isacharunderscore}{\kern0pt}ax{\isacharparenright}{\kern0pt}\ \isanewline
\ \ \ \ \isacommand{apply}\isamarkupfalse%
\ {\isacharparenleft}{\kern0pt}simp\ add\ {\isacharcolon}{\kern0pt}\ HPn{\isacharunderscore}{\kern0pt}auto{\isacharunderscore}{\kern0pt}M{\isacharunderscore}{\kern0pt}fm{\isacharunderscore}{\kern0pt}def{\isacharparenright}{\kern0pt}\ \isanewline
\ \ \ \ \isacommand{apply}\isamarkupfalse%
\ {\isacharparenleft}{\kern0pt}simp\ add\ {\isacharcolon}{\kern0pt}\ assms{\isacharparenright}{\kern0pt}\ \isanewline
\ \ \ \ \isacommand{unfolding}\isamarkupfalse%
\ HPn{\isacharunderscore}{\kern0pt}auto{\isacharunderscore}{\kern0pt}M{\isacharunderscore}{\kern0pt}fm{\isacharunderscore}{\kern0pt}def\ \isanewline
\ \ \ \ \isacommand{apply}\isamarkupfalse%
\ {\isacharparenleft}{\kern0pt}simp\ del{\isacharcolon}{\kern0pt}FOL{\isacharunderscore}{\kern0pt}sats{\isacharunderscore}{\kern0pt}iff\ pair{\isacharunderscore}{\kern0pt}abs\ add{\isacharcolon}{\kern0pt}\ fm{\isacharunderscore}{\kern0pt}defs\ nat{\isacharunderscore}{\kern0pt}simp{\isacharunderscore}{\kern0pt}union{\isacharparenright}{\kern0pt}\ \isanewline
\ \ \ \ \isacommand{done}\isamarkupfalse%
\ \isanewline
\ \ \isacommand{then}\isamarkupfalse%
\ \isacommand{have}\isamarkupfalse%
\ sinM\ {\isacharcolon}{\kern0pt}\ {\isachardoublequoteopen}S\ {\isasymin}\ M{\isachardoublequoteclose}\ \isanewline
\ \ \ \ \isacommand{unfolding}\isamarkupfalse%
\ S{\isacharunderscore}{\kern0pt}def\ \isanewline
\ \ \ \ \isacommand{apply}\isamarkupfalse%
\ {\isacharparenleft}{\kern0pt}rule{\isacharunderscore}{\kern0pt}tac\ separation{\isacharunderscore}{\kern0pt}notation{\isacharparenright}{\kern0pt}\ \isanewline
\ \ \ \ \isacommand{using}\isamarkupfalse%
\ Ah\ \isanewline
\ \ \ \ \isacommand{by}\isamarkupfalse%
\ auto\isanewline
\ \ \isanewline
\ \ \isacommand{have}\isamarkupfalse%
\ {\isachardoublequoteopen}S\ {\isacharequal}{\kern0pt}\ {\isacharbraceleft}{\kern0pt}\ elem\ {\isasymin}\ A{\isachardot}{\kern0pt}\ HPn{\isacharunderscore}{\kern0pt}auto{\isacharunderscore}{\kern0pt}M{\isacharunderscore}{\kern0pt}cond{\isacharparenleft}{\kern0pt}x{\isacharunderscore}{\kern0pt}pi{\isacharcomma}{\kern0pt}\ g{\isacharcomma}{\kern0pt}\ elem{\isacharparenright}{\kern0pt}\ {\isacharbraceright}{\kern0pt}{\isachardoublequoteclose}\ \isanewline
\ \ \ \ \isacommand{using}\isamarkupfalse%
\ HPn{\isacharunderscore}{\kern0pt}auto{\isacharunderscore}{\kern0pt}M{\isacharunderscore}{\kern0pt}fm{\isacharunderscore}{\kern0pt}sats{\isacharunderscore}{\kern0pt}iff\ assms\ Ah\ transM\ Ah\ \isanewline
\ \ \ \ \isacommand{unfolding}\isamarkupfalse%
\ S{\isacharunderscore}{\kern0pt}def\ \isanewline
\ \ \ \ \isacommand{by}\isamarkupfalse%
\ simp\ \isanewline
\ \ \isacommand{also}\isamarkupfalse%
\ \isacommand{have}\isamarkupfalse%
\ {\isachardoublequoteopen}{\isachardot}{\kern0pt}{\isachardot}{\kern0pt}{\isachardot}{\kern0pt}\ {\isacharequal}{\kern0pt}\ {\isacharbraceleft}{\kern0pt}\ \ elem\ {\isasymin}\ M{\isachardot}{\kern0pt}\ HPn{\isacharunderscore}{\kern0pt}auto{\isacharunderscore}{\kern0pt}M{\isacharunderscore}{\kern0pt}cond{\isacharparenleft}{\kern0pt}x{\isacharunderscore}{\kern0pt}pi{\isacharcomma}{\kern0pt}\ g{\isacharcomma}{\kern0pt}\ elem{\isacharparenright}{\kern0pt}\ {\isacharbraceright}{\kern0pt}{\isachardoublequoteclose}\ \isanewline
\ \ \ \ \isacommand{apply}\isamarkupfalse%
\ {\isacharparenleft}{\kern0pt}rule\ equality{\isacharunderscore}{\kern0pt}iffI{\isacharsemicolon}{\kern0pt}\ rule\ iffI{\isacharparenright}{\kern0pt}\isanewline
\ \ \ \ \isacommand{using}\isamarkupfalse%
\ transM\ Ah\ \isanewline
\ \ \ \ \isacommand{by}\isamarkupfalse%
\ auto\ \isanewline
\ \ \isacommand{also}\isamarkupfalse%
\ \isacommand{have}\isamarkupfalse%
\ {\isachardoublequoteopen}{\isachardot}{\kern0pt}{\isachardot}{\kern0pt}{\isachardot}{\kern0pt}\ {\isacharequal}{\kern0pt}\ HPn{\isacharunderscore}{\kern0pt}auto{\isacharunderscore}{\kern0pt}M{\isacharparenleft}{\kern0pt}x{\isacharunderscore}{\kern0pt}pi{\isacharcomma}{\kern0pt}\ g{\isacharparenright}{\kern0pt}{\isachardoublequoteclose}\ \isanewline
\ \ \ \ \isacommand{unfolding}\isamarkupfalse%
\ HPn{\isacharunderscore}{\kern0pt}auto{\isacharunderscore}{\kern0pt}M{\isacharunderscore}{\kern0pt}def\ \isacommand{by}\isamarkupfalse%
\ simp\ \isanewline
\ \ \isacommand{finally}\isamarkupfalse%
\ \isacommand{have}\isamarkupfalse%
\ {\isachardoublequoteopen}S\ {\isacharequal}{\kern0pt}\ HPn{\isacharunderscore}{\kern0pt}auto{\isacharunderscore}{\kern0pt}M{\isacharparenleft}{\kern0pt}x{\isacharunderscore}{\kern0pt}pi{\isacharcomma}{\kern0pt}\ g{\isacharparenright}{\kern0pt}{\isachardoublequoteclose}\ \ \isacommand{by}\isamarkupfalse%
\ simp\isanewline
\isanewline
\ \ \isacommand{then}\isamarkupfalse%
\ \isacommand{show}\isamarkupfalse%
\ {\isachardoublequoteopen}HPn{\isacharunderscore}{\kern0pt}auto{\isacharunderscore}{\kern0pt}M{\isacharparenleft}{\kern0pt}x{\isacharunderscore}{\kern0pt}pi{\isacharcomma}{\kern0pt}\ g{\isacharparenright}{\kern0pt}\ {\isasymin}\ M{\isachardoublequoteclose}\ \isanewline
\ \ \ \ \isacommand{using}\isamarkupfalse%
\ sinM\ \isacommand{by}\isamarkupfalse%
\ simp\ \isanewline
\isacommand{qed}\isamarkupfalse%
%
\endisatagproof
{\isafoldproof}%
%
\isadelimproof
\isanewline
%
\endisadelimproof
\isanewline
\isacommand{lemma}\isamarkupfalse%
\ HPn{\isacharunderscore}{\kern0pt}auto{\isacharunderscore}{\kern0pt}M{\isacharunderscore}{\kern0pt}eq\ {\isacharcolon}{\kern0pt}\ \ {\isachardoublequoteopen}{\isasymAnd}h\ g\ x\ {\isasympi}{\isachardot}{\kern0pt}\ {\isasympi}\ {\isasymin}\ P{\isacharunderscore}{\kern0pt}auto\ {\isasymLongrightarrow}\ h\ {\isasymin}\ eclose{\isacharparenleft}{\kern0pt}x{\isacharparenright}{\kern0pt}\ {\isasymrightarrow}\ M\ {\isasymLongrightarrow}\ g\ {\isasymin}\ eclose{\isacharparenleft}{\kern0pt}x{\isacharparenright}{\kern0pt}\ {\isasymtimes}\ {\isacharbraceleft}{\kern0pt}{\isasympi}{\isacharbraceright}{\kern0pt}\ {\isasymrightarrow}\ M\ {\isasymLongrightarrow}\ g\ {\isasymin}\ M\ \ \isanewline
\ \ \ \ \ \ \ \ \ \ \ \ \ \ \ {\isasymLongrightarrow}\ x\ {\isasymin}\ M\ {\isasymLongrightarrow}\ {\isacharparenleft}{\kern0pt}{\isasymAnd}y{\isachardot}{\kern0pt}\ y\ {\isasymin}\ eclose{\isacharparenleft}{\kern0pt}x{\isacharparenright}{\kern0pt}\ {\isasymLongrightarrow}\ h{\isacharbackquote}{\kern0pt}y\ {\isacharequal}{\kern0pt}\ g{\isacharbackquote}{\kern0pt}{\isacharless}{\kern0pt}y{\isacharcomma}{\kern0pt}\ {\isasympi}{\isachargreater}{\kern0pt}{\isacharparenright}{\kern0pt}\ {\isasymLongrightarrow}\ HPn{\isacharunderscore}{\kern0pt}auto{\isacharparenleft}{\kern0pt}{\isasympi}{\isacharcomma}{\kern0pt}\ x{\isacharcomma}{\kern0pt}\ h{\isacharparenright}{\kern0pt}\ {\isacharequal}{\kern0pt}\ HPn{\isacharunderscore}{\kern0pt}auto{\isacharunderscore}{\kern0pt}M{\isacharparenleft}{\kern0pt}{\isacharless}{\kern0pt}x{\isacharcomma}{\kern0pt}\ {\isasympi}{\isachargreater}{\kern0pt}{\isacharcomma}{\kern0pt}\ g{\isacharparenright}{\kern0pt}{\isachardoublequoteclose}\isanewline
%
\isadelimproof
%
\endisadelimproof
%
\isatagproof
\isacommand{proof}\isamarkupfalse%
\ {\isacharminus}{\kern0pt}\ \isanewline
\ \ \isacommand{fix}\isamarkupfalse%
\ h\ g\ x\ {\isasympi}\isanewline
\ \ \isacommand{assume}\isamarkupfalse%
\ assms{\isadigit{1}}\ {\isacharcolon}{\kern0pt}\ {\isachardoublequoteopen}x\ {\isasymin}\ M{\isachardoublequoteclose}\ {\isachardoublequoteopen}{\isasympi}\ {\isasymin}\ P{\isacharunderscore}{\kern0pt}auto{\isachardoublequoteclose}\ {\isachardoublequoteopen}h\ {\isasymin}\ eclose{\isacharparenleft}{\kern0pt}x{\isacharparenright}{\kern0pt}\ {\isasymrightarrow}\ M{\isachardoublequoteclose}\ {\isachardoublequoteopen}g\ {\isasymin}\ eclose{\isacharparenleft}{\kern0pt}x{\isacharparenright}{\kern0pt}\ {\isasymtimes}\ {\isacharbraceleft}{\kern0pt}{\isasympi}{\isacharbraceright}{\kern0pt}\ {\isasymrightarrow}\ M{\isachardoublequoteclose}\ {\isachardoublequoteopen}g\ {\isasymin}\ M{\isachardoublequoteclose}\ \isanewline
\ \ \ \ {\isachardoublequoteopen}{\isacharparenleft}{\kern0pt}{\isasymAnd}y{\isachardot}{\kern0pt}\ y\ {\isasymin}\ eclose{\isacharparenleft}{\kern0pt}x{\isacharparenright}{\kern0pt}\ {\isasymLongrightarrow}\ h\ {\isacharbackquote}{\kern0pt}\ y\ {\isacharequal}{\kern0pt}\ g\ {\isacharbackquote}{\kern0pt}\ {\isasymlangle}y{\isacharcomma}{\kern0pt}\ {\isasympi}{\isasymrangle}{\isacharparenright}{\kern0pt}{\isachardoublequoteclose}\isanewline
\isanewline
\ \ \isacommand{have}\isamarkupfalse%
\ piinM\ {\isacharcolon}{\kern0pt}\ {\isachardoublequoteopen}{\isasympi}\ {\isasymin}\ M{\isachardoublequoteclose}\ \isanewline
\ \ \ \ \isacommand{using}\isamarkupfalse%
\ pair{\isacharunderscore}{\kern0pt}in{\isacharunderscore}{\kern0pt}M{\isacharunderscore}{\kern0pt}iff\ assms{\isadigit{1}}\ P{\isacharunderscore}{\kern0pt}auto{\isacharunderscore}{\kern0pt}def\ is{\isacharunderscore}{\kern0pt}P{\isacharunderscore}{\kern0pt}auto{\isacharunderscore}{\kern0pt}def\ \isanewline
\ \ \ \ \isacommand{by}\isamarkupfalse%
\ auto\ \ \isanewline
\isanewline
\ \ \isacommand{have}\isamarkupfalse%
\ {\isachardoublequoteopen}HPn{\isacharunderscore}{\kern0pt}auto{\isacharunderscore}{\kern0pt}M{\isacharparenleft}{\kern0pt}{\isacharless}{\kern0pt}x{\isacharcomma}{\kern0pt}\ {\isasympi}{\isachargreater}{\kern0pt}{\isacharcomma}{\kern0pt}\ g{\isacharparenright}{\kern0pt}\ {\isasymin}\ M{\isachardoublequoteclose}\ \isanewline
\ \ \ \ \isacommand{apply}\isamarkupfalse%
{\isacharparenleft}{\kern0pt}rule\ HPn{\isacharunderscore}{\kern0pt}auto{\isacharunderscore}{\kern0pt}M{\isacharunderscore}{\kern0pt}in{\isacharunderscore}{\kern0pt}M{\isacharparenright}{\kern0pt}\isanewline
\ \ \ \ \isacommand{using}\isamarkupfalse%
\ assms{\isadigit{1}}\ pair{\isacharunderscore}{\kern0pt}in{\isacharunderscore}{\kern0pt}M{\isacharunderscore}{\kern0pt}iff\ P{\isacharunderscore}{\kern0pt}auto{\isacharunderscore}{\kern0pt}def\ is{\isacharunderscore}{\kern0pt}P{\isacharunderscore}{\kern0pt}auto{\isacharunderscore}{\kern0pt}def\ Pi{\isacharunderscore}{\kern0pt}def\isanewline
\ \ \ \ \isacommand{by}\isamarkupfalse%
\ auto\isanewline
\isanewline
\ \ \isacommand{show}\isamarkupfalse%
\ {\isachardoublequoteopen}HPn{\isacharunderscore}{\kern0pt}auto{\isacharparenleft}{\kern0pt}{\isasympi}{\isacharcomma}{\kern0pt}\ x{\isacharcomma}{\kern0pt}\ h{\isacharparenright}{\kern0pt}\ {\isacharequal}{\kern0pt}\ HPn{\isacharunderscore}{\kern0pt}auto{\isacharunderscore}{\kern0pt}M{\isacharparenleft}{\kern0pt}{\isacharless}{\kern0pt}x{\isacharcomma}{\kern0pt}\ {\isasympi}{\isachargreater}{\kern0pt}{\isacharcomma}{\kern0pt}\ g{\isacharparenright}{\kern0pt}{\isachardoublequoteclose}\ \isanewline
\ \ \isacommand{proof}\isamarkupfalse%
{\isacharparenleft}{\kern0pt}rule\ equality{\isacharunderscore}{\kern0pt}iffI{\isacharcomma}{\kern0pt}\ rule\ iffI{\isacharparenright}{\kern0pt}\isanewline
\ \ \ \ \isacommand{fix}\isamarkupfalse%
\ u\ \isacommand{assume}\isamarkupfalse%
\ {\isachardoublequoteopen}u\ {\isasymin}\ HPn{\isacharunderscore}{\kern0pt}auto{\isacharparenleft}{\kern0pt}{\isasympi}{\isacharcomma}{\kern0pt}\ x{\isacharcomma}{\kern0pt}\ h{\isacharparenright}{\kern0pt}{\isachardoublequoteclose}\ \isanewline
\ \ \ \ \isacommand{then}\isamarkupfalse%
\ \isacommand{obtain}\isamarkupfalse%
\ y\ p\ \isakeyword{where}\ ueq\ {\isacharcolon}{\kern0pt}\ {\isachardoublequoteopen}u\ {\isacharequal}{\kern0pt}\ {\isacharless}{\kern0pt}h{\isacharbackquote}{\kern0pt}y{\isacharcomma}{\kern0pt}\ {\isasympi}{\isacharbackquote}{\kern0pt}p{\isachargreater}{\kern0pt}{\isachardoublequoteclose}\ \isakeyword{and}\ ypin\ {\isacharcolon}{\kern0pt}\ {\isachardoublequoteopen}{\isacharless}{\kern0pt}y{\isacharcomma}{\kern0pt}\ p{\isachargreater}{\kern0pt}\ {\isasymin}\ x{\isachardoublequoteclose}\ \isacommand{unfolding}\isamarkupfalse%
\ HPn{\isacharunderscore}{\kern0pt}auto{\isacharunderscore}{\kern0pt}def\ \isacommand{by}\isamarkupfalse%
\ auto\isanewline
\ \ \ \ \isacommand{have}\isamarkupfalse%
\ {\isachardoublequoteopen}y\ {\isasymin}\ eclose{\isacharparenleft}{\kern0pt}x{\isacharparenright}{\kern0pt}{\isachardoublequoteclose}\ \isanewline
\ \ \ \ \ \ \isacommand{apply}\isamarkupfalse%
{\isacharparenleft}{\kern0pt}rule\ domain{\isacharunderscore}{\kern0pt}elem{\isacharunderscore}{\kern0pt}in{\isacharunderscore}{\kern0pt}eclose{\isacharparenright}{\kern0pt}\ \isanewline
\ \ \ \ \ \ \isacommand{using}\isamarkupfalse%
\ ypin\ \isanewline
\ \ \ \ \ \ \isacommand{by}\isamarkupfalse%
\ auto\isanewline
\ \ \ \ \isacommand{then}\isamarkupfalse%
\ \isacommand{have}\isamarkupfalse%
\ hyeq\ {\isacharcolon}{\kern0pt}\ {\isachardoublequoteopen}h{\isacharbackquote}{\kern0pt}y\ {\isacharequal}{\kern0pt}\ g{\isacharbackquote}{\kern0pt}{\isacharless}{\kern0pt}y{\isacharcomma}{\kern0pt}\ {\isasympi}{\isachargreater}{\kern0pt}{\isachardoublequoteclose}\ \isacommand{using}\isamarkupfalse%
\ assms{\isadigit{1}}\ \isacommand{by}\isamarkupfalse%
\ auto\isanewline
\isanewline
\ \ \ \ \isacommand{have}\isamarkupfalse%
\ yinM\ {\isacharcolon}{\kern0pt}\ {\isachardoublequoteopen}y\ {\isasymin}\ M{\isachardoublequoteclose}\isanewline
\ \ \ \ \ \ \isacommand{apply}\isamarkupfalse%
{\isacharparenleft}{\kern0pt}rule{\isacharunderscore}{\kern0pt}tac\ x{\isacharequal}{\kern0pt}x\ \isakeyword{in}\ domain{\isacharunderscore}{\kern0pt}elem{\isacharunderscore}{\kern0pt}in{\isacharunderscore}{\kern0pt}M{\isacharparenright}{\kern0pt}\isanewline
\ \ \ \ \ \ \isacommand{using}\isamarkupfalse%
\ assms{\isadigit{1}}\ ypin\ \isanewline
\ \ \ \ \ \ \isacommand{by}\isamarkupfalse%
\ auto\isanewline
\isanewline
\ \ \ \ \isacommand{have}\isamarkupfalse%
\ pinM\ {\isacharcolon}{\kern0pt}\ {\isachardoublequoteopen}p\ {\isasymin}\ M{\isachardoublequoteclose}\ \isanewline
\ \ \ \ \ \ \isacommand{apply}\isamarkupfalse%
{\isacharparenleft}{\kern0pt}rule\ to{\isacharunderscore}{\kern0pt}rin{\isacharcomma}{\kern0pt}\ rule{\isacharunderscore}{\kern0pt}tac\ x{\isacharequal}{\kern0pt}{\isachardoublequoteopen}range{\isacharparenleft}{\kern0pt}x{\isacharparenright}{\kern0pt}{\isachardoublequoteclose}\ \isakeyword{in}\ transM{\isacharparenright}{\kern0pt}\isanewline
\ \ \ \ \ \ \isacommand{using}\isamarkupfalse%
\ assms{\isadigit{1}}\ ypin\ range{\isacharunderscore}{\kern0pt}closed\ \isanewline
\ \ \ \ \ \ \isacommand{by}\isamarkupfalse%
\ auto\isanewline
\isanewline
\ \ \ \ \isacommand{have}\isamarkupfalse%
\ uinM\ {\isacharcolon}{\kern0pt}\ {\isachardoublequoteopen}u\ {\isasymin}\ M{\isachardoublequoteclose}\ \isanewline
\ \ \ \ \ \ \isacommand{apply}\isamarkupfalse%
{\isacharparenleft}{\kern0pt}subst\ ueq{\isacharparenright}{\kern0pt}\isanewline
\ \ \ \ \ \ \isacommand{apply}\isamarkupfalse%
{\isacharparenleft}{\kern0pt}subst\ hyeq{\isacharparenright}{\kern0pt}\isanewline
\ \ \ \ \ \ \isacommand{apply}\isamarkupfalse%
{\isacharparenleft}{\kern0pt}rule\ to{\isacharunderscore}{\kern0pt}rin{\isacharcomma}{\kern0pt}\ rule\ iffD{\isadigit{2}}{\isacharcomma}{\kern0pt}\ rule\ pair{\isacharunderscore}{\kern0pt}in{\isacharunderscore}{\kern0pt}M{\isacharunderscore}{\kern0pt}iff{\isacharparenright}{\kern0pt}\isanewline
\ \ \ \ \ \ \isacommand{apply}\isamarkupfalse%
{\isacharparenleft}{\kern0pt}rule\ conjI{\isacharcomma}{\kern0pt}\ rule\ apply{\isacharunderscore}{\kern0pt}closed{\isacharparenright}{\kern0pt}\isanewline
\ \ \ \ \ \ \isacommand{using}\isamarkupfalse%
\ assms{\isadigit{1}}\ pair{\isacharunderscore}{\kern0pt}in{\isacharunderscore}{\kern0pt}M{\isacharunderscore}{\kern0pt}iff\ yinM\ piinM\ \isanewline
\ \ \ \ \ \ \ \ \isacommand{apply}\isamarkupfalse%
\ auto{\isacharbrackleft}{\kern0pt}{\isadigit{2}}{\isacharbrackright}{\kern0pt}\isanewline
\ \ \ \ \ \ \isacommand{apply}\isamarkupfalse%
{\isacharparenleft}{\kern0pt}rule\ apply{\isacharunderscore}{\kern0pt}closed{\isacharparenright}{\kern0pt}\isanewline
\ \ \ \ \ \ \isacommand{using}\isamarkupfalse%
\ piinM\ pinM\isanewline
\ \ \ \ \ \ \isacommand{by}\isamarkupfalse%
\ auto\isanewline
\isanewline
\ \ \ \ \isacommand{have}\isamarkupfalse%
\ {\isachardoublequoteopen}HPn{\isacharunderscore}{\kern0pt}auto{\isacharunderscore}{\kern0pt}M{\isacharunderscore}{\kern0pt}cond{\isacharparenleft}{\kern0pt}{\isacharless}{\kern0pt}x{\isacharcomma}{\kern0pt}\ {\isasympi}{\isachargreater}{\kern0pt}{\isacharcomma}{\kern0pt}\ g{\isacharcomma}{\kern0pt}\ {\isacharless}{\kern0pt}h{\isacharbackquote}{\kern0pt}y{\isacharcomma}{\kern0pt}\ {\isasympi}{\isacharbackquote}{\kern0pt}p{\isachargreater}{\kern0pt}{\isacharparenright}{\kern0pt}{\isachardoublequoteclose}\ \isanewline
\ \ \ \ \ \ \isacommand{apply}\isamarkupfalse%
{\isacharparenleft}{\kern0pt}rule\ iffD{\isadigit{2}}{\isacharcomma}{\kern0pt}\ rule\ HPn{\isacharunderscore}{\kern0pt}auto{\isacharunderscore}{\kern0pt}M{\isacharunderscore}{\kern0pt}cond{\isacharunderscore}{\kern0pt}iff{\isacharparenright}{\kern0pt}\isanewline
\ \ \ \ \ \ \isacommand{using}\isamarkupfalse%
\ assms{\isadigit{1}}\ piinM\ pair{\isacharunderscore}{\kern0pt}in{\isacharunderscore}{\kern0pt}M{\isacharunderscore}{\kern0pt}iff\ \isanewline
\ \ \ \ \ \ \ \ \ \isacommand{apply}\isamarkupfalse%
\ auto{\isacharbrackleft}{\kern0pt}{\isadigit{2}}{\isacharbrackright}{\kern0pt}\isanewline
\ \ \ \ \ \ \ \isacommand{apply}\isamarkupfalse%
{\isacharparenleft}{\kern0pt}rule\ to{\isacharunderscore}{\kern0pt}rin{\isacharcomma}{\kern0pt}\ rule\ iffD{\isadigit{2}}{\isacharcomma}{\kern0pt}\ rule\ pair{\isacharunderscore}{\kern0pt}in{\isacharunderscore}{\kern0pt}M{\isacharunderscore}{\kern0pt}iff{\isacharcomma}{\kern0pt}\ rule\ conjI{\isacharcomma}{\kern0pt}\ subst\ hyeq{\isacharparenright}{\kern0pt}\isanewline
\ \ \ \ \ \ \ \ \isacommand{apply}\isamarkupfalse%
{\isacharparenleft}{\kern0pt}rule\ apply{\isacharunderscore}{\kern0pt}closed{\isacharparenright}{\kern0pt}\isanewline
\ \ \ \ \ \ \isacommand{using}\isamarkupfalse%
\ assms{\isadigit{1}}\ yinM\ piinM\ pair{\isacharunderscore}{\kern0pt}in{\isacharunderscore}{\kern0pt}M{\isacharunderscore}{\kern0pt}iff\ \isanewline
\ \ \ \ \ \ \ \ \ \isacommand{apply}\isamarkupfalse%
\ auto{\isacharbrackleft}{\kern0pt}{\isadigit{2}}{\isacharbrackright}{\kern0pt}\isanewline
\ \ \ \ \ \ \ \isacommand{apply}\isamarkupfalse%
{\isacharparenleft}{\kern0pt}rule\ apply{\isacharunderscore}{\kern0pt}closed{\isacharparenright}{\kern0pt}\isanewline
\ \ \ \ \ \ \isacommand{using}\isamarkupfalse%
\ piinM\ pinM\ \isanewline
\ \ \ \ \ \ \ \ \isacommand{apply}\isamarkupfalse%
\ auto{\isacharbrackleft}{\kern0pt}{\isadigit{2}}{\isacharbrackright}{\kern0pt}\isanewline
\ \ \ \ \ \ \isacommand{apply}\isamarkupfalse%
{\isacharparenleft}{\kern0pt}rule{\isacharunderscore}{\kern0pt}tac\ x{\isacharequal}{\kern0pt}y\ \isakeyword{in}\ bexI{\isacharcomma}{\kern0pt}\ rule{\isacharunderscore}{\kern0pt}tac\ x{\isacharequal}{\kern0pt}p\ \isakeyword{in}\ bexI{\isacharcomma}{\kern0pt}\ rule{\isacharunderscore}{\kern0pt}tac\ x{\isacharequal}{\kern0pt}x\ \isakeyword{in}\ bexI{\isacharcomma}{\kern0pt}\ rule{\isacharunderscore}{\kern0pt}tac\ x{\isacharequal}{\kern0pt}{\isasympi}\ \isakeyword{in}\ bexI{\isacharparenright}{\kern0pt}\isanewline
\ \ \ \ \ \ \ \ \ \ \isacommand{apply}\isamarkupfalse%
{\isacharparenleft}{\kern0pt}subst\ hyeq{\isacharparenright}{\kern0pt}\isanewline
\ \ \ \ \ \ \isacommand{using}\isamarkupfalse%
\ ypin\isanewline
\ \ \ \ \ \ \ \ \ \ \isacommand{apply}\isamarkupfalse%
\ simp\isanewline
\ \ \ \ \ \ \isacommand{using}\isamarkupfalse%
\ assms{\isadigit{1}}\ piinM\ yinM\ pinM\isanewline
\ \ \ \ \ \ \isacommand{unfolding}\isamarkupfalse%
\ P{\isacharunderscore}{\kern0pt}auto{\isacharunderscore}{\kern0pt}def\ Pi{\isacharunderscore}{\kern0pt}def\ \isanewline
\ \ \ \ \ \ \isacommand{by}\isamarkupfalse%
\ auto\isanewline
\isanewline
\ \ \ \ \isacommand{then}\isamarkupfalse%
\ \isacommand{show}\isamarkupfalse%
\ {\isachardoublequoteopen}u\ {\isasymin}\ HPn{\isacharunderscore}{\kern0pt}auto{\isacharunderscore}{\kern0pt}M{\isacharparenleft}{\kern0pt}{\isacharless}{\kern0pt}x{\isacharcomma}{\kern0pt}\ {\isasympi}{\isachargreater}{\kern0pt}{\isacharcomma}{\kern0pt}\ g{\isacharparenright}{\kern0pt}{\isachardoublequoteclose}\isanewline
\ \ \ \ \ \ \isacommand{unfolding}\isamarkupfalse%
\ HPn{\isacharunderscore}{\kern0pt}auto{\isacharunderscore}{\kern0pt}M{\isacharunderscore}{\kern0pt}def\ \isanewline
\ \ \ \ \ \ \isacommand{using}\isamarkupfalse%
\ ueq\ uinM\ \isanewline
\ \ \ \ \ \ \isacommand{by}\isamarkupfalse%
\ auto\isanewline
\ \ \isacommand{next}\isamarkupfalse%
\ \isanewline
\ \ \ \ \isacommand{fix}\isamarkupfalse%
\ u\ \isacommand{assume}\isamarkupfalse%
\ {\isachardoublequoteopen}u\ {\isasymin}\ HPn{\isacharunderscore}{\kern0pt}auto{\isacharunderscore}{\kern0pt}M{\isacharparenleft}{\kern0pt}{\isasymlangle}x{\isacharcomma}{\kern0pt}\ {\isasympi}{\isasymrangle}{\isacharcomma}{\kern0pt}\ g{\isacharparenright}{\kern0pt}{\isachardoublequoteclose}\ \isanewline
\ \ \ \ \isacommand{then}\isamarkupfalse%
\ \isacommand{have}\isamarkupfalse%
\ uH{\isacharcolon}{\kern0pt}\ {\isachardoublequoteopen}u\ {\isasymin}\ M{\isachardoublequoteclose}\ {\isachardoublequoteopen}HPn{\isacharunderscore}{\kern0pt}auto{\isacharunderscore}{\kern0pt}M{\isacharunderscore}{\kern0pt}cond{\isacharparenleft}{\kern0pt}{\isacharless}{\kern0pt}x{\isacharcomma}{\kern0pt}\ {\isasympi}{\isachargreater}{\kern0pt}{\isacharcomma}{\kern0pt}\ g{\isacharcomma}{\kern0pt}\ u{\isacharparenright}{\kern0pt}{\isachardoublequoteclose}\ \isacommand{unfolding}\isamarkupfalse%
\ HPn{\isacharunderscore}{\kern0pt}auto{\isacharunderscore}{\kern0pt}M{\isacharunderscore}{\kern0pt}def\ \isacommand{by}\isamarkupfalse%
\ auto\ \isanewline
\ \ \ \ \isacommand{have}\isamarkupfalse%
\ {\isachardoublequoteopen}{\isacharparenleft}{\kern0pt}{\isasymexists}y\ {\isasymin}\ M{\isachardot}{\kern0pt}\ {\isasymexists}p\ {\isasymin}\ M{\isachardot}{\kern0pt}\ {\isasymexists}xx\ {\isasymin}\ M{\isachardot}{\kern0pt}\ {\isasymexists}pi\ {\isasymin}\ M{\isachardot}{\kern0pt}\ \isanewline
\ \ \ \ \ \ {\isacharless}{\kern0pt}x{\isacharcomma}{\kern0pt}\ {\isasympi}{\isachargreater}{\kern0pt}\ {\isacharequal}{\kern0pt}\ {\isacharless}{\kern0pt}xx{\isacharcomma}{\kern0pt}\ pi{\isachargreater}{\kern0pt}\ {\isasymand}\isanewline
\ \ \ \ \ \ u\ {\isacharequal}{\kern0pt}\ {\isacharless}{\kern0pt}g{\isacharbackquote}{\kern0pt}{\isacharless}{\kern0pt}y{\isacharcomma}{\kern0pt}pi{\isachargreater}{\kern0pt}{\isacharcomma}{\kern0pt}\ pi{\isacharbackquote}{\kern0pt}p{\isachargreater}{\kern0pt}\ {\isasymand}\ \isanewline
\ \ \ \ \ \ {\isacharless}{\kern0pt}y{\isacharcomma}{\kern0pt}\ p{\isachargreater}{\kern0pt}\ {\isasymin}\ xx\ {\isasymand}\ \isanewline
\ \ \ \ \ \ function{\isacharparenleft}{\kern0pt}pi{\isacharparenright}{\kern0pt}\isanewline
\ \ \ \ {\isacharparenright}{\kern0pt}{\isachardoublequoteclose}\ \isanewline
\ \ \ \ \ \ \isacommand{apply}\isamarkupfalse%
{\isacharparenleft}{\kern0pt}rule\ iffD{\isadigit{1}}{\isacharcomma}{\kern0pt}\ rule{\isacharunderscore}{\kern0pt}tac\ HPn{\isacharunderscore}{\kern0pt}auto{\isacharunderscore}{\kern0pt}M{\isacharunderscore}{\kern0pt}cond{\isacharunderscore}{\kern0pt}iff{\isacharparenright}{\kern0pt}\isanewline
\ \ \ \ \ \ \isacommand{using}\isamarkupfalse%
\ pair{\isacharunderscore}{\kern0pt}in{\isacharunderscore}{\kern0pt}M{\isacharunderscore}{\kern0pt}iff\ assms{\isadigit{1}}\ piinM\ uH\isanewline
\ \ \ \ \ \ \isacommand{by}\isamarkupfalse%
\ auto\isanewline
\ \ \ \ \isacommand{then}\isamarkupfalse%
\ \isacommand{obtain}\isamarkupfalse%
\ y\ p\ \isakeyword{where}\ ueq\ {\isacharcolon}{\kern0pt}\ {\isachardoublequoteopen}u\ {\isacharequal}{\kern0pt}\ {\isacharless}{\kern0pt}g{\isacharbackquote}{\kern0pt}{\isacharless}{\kern0pt}y{\isacharcomma}{\kern0pt}\ {\isasympi}{\isachargreater}{\kern0pt}{\isacharcomma}{\kern0pt}\ {\isasympi}{\isacharbackquote}{\kern0pt}p{\isachargreater}{\kern0pt}{\isachardoublequoteclose}\ \isakeyword{and}\ ypin\ {\isacharcolon}{\kern0pt}\ {\isachardoublequoteopen}{\isacharless}{\kern0pt}y{\isacharcomma}{\kern0pt}\ p{\isachargreater}{\kern0pt}\ {\isasymin}\ x{\isachardoublequoteclose}\ \isacommand{by}\isamarkupfalse%
\ auto\ \isanewline
\ \ \ \ \isacommand{then}\isamarkupfalse%
\ \isacommand{show}\isamarkupfalse%
\ {\isachardoublequoteopen}u\ {\isasymin}\ HPn{\isacharunderscore}{\kern0pt}auto{\isacharparenleft}{\kern0pt}{\isasympi}{\isacharcomma}{\kern0pt}\ x{\isacharcomma}{\kern0pt}\ h{\isacharparenright}{\kern0pt}{\isachardoublequoteclose}\ \isanewline
\ \ \ \ \ \ \isacommand{unfolding}\isamarkupfalse%
\ HPn{\isacharunderscore}{\kern0pt}auto{\isacharunderscore}{\kern0pt}def\isanewline
\ \ \ \ \ \ \isacommand{apply}\isamarkupfalse%
\ simp\isanewline
\ \ \ \ \ \ \isacommand{apply}\isamarkupfalse%
{\isacharparenleft}{\kern0pt}rule{\isacharunderscore}{\kern0pt}tac\ x{\isacharequal}{\kern0pt}{\isachardoublequoteopen}{\isacharless}{\kern0pt}y{\isacharcomma}{\kern0pt}\ p{\isachargreater}{\kern0pt}{\isachardoublequoteclose}\ \isakeyword{in}\ bexI{\isacharparenright}{\kern0pt}\isanewline
\ \ \ \ \ \ \ \isacommand{apply}\isamarkupfalse%
{\isacharparenleft}{\kern0pt}subgoal{\isacharunderscore}{\kern0pt}tac\ {\isachardoublequoteopen}y\ {\isasymin}\ eclose{\isacharparenleft}{\kern0pt}x{\isacharparenright}{\kern0pt}{\isachardoublequoteclose}{\isacharparenright}{\kern0pt}\isanewline
\ \ \ \ \ \ \isacommand{using}\isamarkupfalse%
\ assms{\isadigit{1}}\ \isanewline
\ \ \ \ \ \ \ \ \isacommand{apply}\isamarkupfalse%
\ force\isanewline
\ \ \ \ \ \ \ \isacommand{apply}\isamarkupfalse%
{\isacharparenleft}{\kern0pt}rule\ domain{\isacharunderscore}{\kern0pt}elem{\isacharunderscore}{\kern0pt}in{\isacharunderscore}{\kern0pt}eclose{\isacharparenright}{\kern0pt}\isanewline
\ \ \ \ \ \ \isacommand{by}\isamarkupfalse%
\ auto\isanewline
\ \ \isacommand{qed}\isamarkupfalse%
\isanewline
\isacommand{qed}\isamarkupfalse%
%
\endisatagproof
{\isafoldproof}%
%
\isadelimproof
\isanewline
%
\endisadelimproof
\isanewline
\isacommand{end}\isamarkupfalse%
\isanewline
\isanewline
\isacommand{definition}\isamarkupfalse%
\ is{\isacharunderscore}{\kern0pt}Pn{\isacharunderscore}{\kern0pt}auto{\isacharunderscore}{\kern0pt}fm\ \isakeyword{where}\ {\isachardoublequoteopen}is{\isacharunderscore}{\kern0pt}Pn{\isacharunderscore}{\kern0pt}auto{\isacharunderscore}{\kern0pt}fm{\isacharparenleft}{\kern0pt}p{\isacharcomma}{\kern0pt}\ pi{\isacharcomma}{\kern0pt}\ x{\isacharcomma}{\kern0pt}\ v{\isacharparenright}{\kern0pt}\ {\isasymequiv}\ And{\isacharparenleft}{\kern0pt}is{\isacharunderscore}{\kern0pt}P{\isacharunderscore}{\kern0pt}name{\isacharunderscore}{\kern0pt}fm{\isacharparenleft}{\kern0pt}p{\isacharcomma}{\kern0pt}\ x{\isacharparenright}{\kern0pt}{\isacharcomma}{\kern0pt}\ is{\isacharunderscore}{\kern0pt}memrel{\isacharunderscore}{\kern0pt}wftrec{\isacharunderscore}{\kern0pt}fm{\isacharparenleft}{\kern0pt}HPn{\isacharunderscore}{\kern0pt}auto{\isacharunderscore}{\kern0pt}M{\isacharunderscore}{\kern0pt}fm{\isacharprime}{\kern0pt}{\isacharcomma}{\kern0pt}\ x{\isacharcomma}{\kern0pt}\ pi{\isacharcomma}{\kern0pt}\ v{\isacharparenright}{\kern0pt}{\isacharparenright}{\kern0pt}{\isachardoublequoteclose}\ \isanewline
\isanewline
\isanewline
\isacommand{context}\isamarkupfalse%
\ forcing{\isacharunderscore}{\kern0pt}data{\isacharunderscore}{\kern0pt}partial\ \isanewline
\isakeyword{begin}\ \isanewline
\isanewline
\isacommand{lemma}\isamarkupfalse%
\ is{\isacharunderscore}{\kern0pt}Pn{\isacharunderscore}{\kern0pt}auto{\isacharunderscore}{\kern0pt}fm{\isacharunderscore}{\kern0pt}type\ {\isacharcolon}{\kern0pt}\ \isanewline
\ \ \isakeyword{fixes}\ p\ pi\ x\ v\ \isanewline
\ \ \isakeyword{assumes}\ {\isachardoublequoteopen}p\ {\isasymin}\ nat{\isachardoublequoteclose}\ {\isachardoublequoteopen}pi\ {\isasymin}\ nat{\isachardoublequoteclose}\ {\isachardoublequoteopen}x\ {\isasymin}\ nat{\isachardoublequoteclose}\ {\isachardoublequoteopen}v\ {\isasymin}\ nat{\isachardoublequoteclose}\ \isanewline
\ \ \isakeyword{shows}\ {\isachardoublequoteopen}is{\isacharunderscore}{\kern0pt}Pn{\isacharunderscore}{\kern0pt}auto{\isacharunderscore}{\kern0pt}fm{\isacharparenleft}{\kern0pt}p{\isacharcomma}{\kern0pt}\ pi{\isacharcomma}{\kern0pt}\ x{\isacharcomma}{\kern0pt}\ v{\isacharparenright}{\kern0pt}\ {\isasymin}\ formula{\isachardoublequoteclose}\ \isanewline
%
\isadelimproof
\ \ %
\endisadelimproof
%
\isatagproof
\isacommand{unfolding}\isamarkupfalse%
\ is{\isacharunderscore}{\kern0pt}Pn{\isacharunderscore}{\kern0pt}auto{\isacharunderscore}{\kern0pt}fm{\isacharunderscore}{\kern0pt}def\ \isanewline
\ \ \isacommand{apply}\isamarkupfalse%
{\isacharparenleft}{\kern0pt}rule\ And{\isacharunderscore}{\kern0pt}type{\isacharcomma}{\kern0pt}\ rule\ is{\isacharunderscore}{\kern0pt}P{\isacharunderscore}{\kern0pt}name{\isacharunderscore}{\kern0pt}fm{\isacharunderscore}{\kern0pt}type{\isacharcomma}{\kern0pt}\ simp\ add{\isacharcolon}{\kern0pt}assms{\isacharcomma}{\kern0pt}\ simp\ add{\isacharcolon}{\kern0pt}assms{\isacharparenright}{\kern0pt}\isanewline
\ \ \isacommand{apply}\isamarkupfalse%
{\isacharparenleft}{\kern0pt}rule\ is{\isacharunderscore}{\kern0pt}memrel{\isacharunderscore}{\kern0pt}wftrec{\isacharunderscore}{\kern0pt}fm{\isacharunderscore}{\kern0pt}type{\isacharparenright}{\kern0pt}\ \isanewline
\ \ \isacommand{unfolding}\isamarkupfalse%
\ HPn{\isacharunderscore}{\kern0pt}auto{\isacharunderscore}{\kern0pt}M{\isacharunderscore}{\kern0pt}fm{\isacharprime}{\kern0pt}{\isacharunderscore}{\kern0pt}def\ \isanewline
\ \ \isacommand{using}\isamarkupfalse%
\ assms\isanewline
\ \ \isacommand{by}\isamarkupfalse%
\ auto%
\endisatagproof
{\isafoldproof}%
%
\isadelimproof
\isanewline
%
\endisadelimproof
\isanewline
\isacommand{lemma}\isamarkupfalse%
\ arity{\isacharunderscore}{\kern0pt}is{\isacharunderscore}{\kern0pt}Pn{\isacharunderscore}{\kern0pt}auto{\isacharunderscore}{\kern0pt}fm\ {\isacharcolon}{\kern0pt}\ \isanewline
\ \ \isakeyword{fixes}\ p\ pi\ x\ v\ \isanewline
\ \ \isakeyword{assumes}\ {\isachardoublequoteopen}p\ {\isasymin}\ nat{\isachardoublequoteclose}\ {\isachardoublequoteopen}pi\ {\isasymin}\ nat{\isachardoublequoteclose}\ {\isachardoublequoteopen}x\ {\isasymin}\ nat{\isachardoublequoteclose}\ {\isachardoublequoteopen}v\ {\isasymin}\ nat{\isachardoublequoteclose}\ \isanewline
\ \ \isakeyword{shows}\ {\isachardoublequoteopen}arity{\isacharparenleft}{\kern0pt}is{\isacharunderscore}{\kern0pt}Pn{\isacharunderscore}{\kern0pt}auto{\isacharunderscore}{\kern0pt}fm{\isacharparenleft}{\kern0pt}p{\isacharcomma}{\kern0pt}\ pi{\isacharcomma}{\kern0pt}\ x{\isacharcomma}{\kern0pt}\ v{\isacharparenright}{\kern0pt}{\isacharparenright}{\kern0pt}\ {\isasymle}\ succ{\isacharparenleft}{\kern0pt}p{\isacharparenright}{\kern0pt}\ {\isasymunion}\ succ{\isacharparenleft}{\kern0pt}pi{\isacharparenright}{\kern0pt}\ {\isasymunion}\ succ{\isacharparenleft}{\kern0pt}x{\isacharparenright}{\kern0pt}\ {\isasymunion}\ succ{\isacharparenleft}{\kern0pt}v{\isacharparenright}{\kern0pt}{\isachardoublequoteclose}\isanewline
%
\isadelimproof
\isanewline
\ \ %
\endisadelimproof
%
\isatagproof
\isacommand{unfolding}\isamarkupfalse%
\ is{\isacharunderscore}{\kern0pt}Pn{\isacharunderscore}{\kern0pt}auto{\isacharunderscore}{\kern0pt}fm{\isacharunderscore}{\kern0pt}def\ \isanewline
\ \ \isacommand{apply}\isamarkupfalse%
\ simp\isanewline
\ \ \isacommand{apply}\isamarkupfalse%
{\isacharparenleft}{\kern0pt}rule\ Un{\isacharunderscore}{\kern0pt}least{\isacharunderscore}{\kern0pt}lt{\isacharcomma}{\kern0pt}\ rule\ le{\isacharunderscore}{\kern0pt}trans{\isacharcomma}{\kern0pt}\ rule\ arity{\isacharunderscore}{\kern0pt}is{\isacharunderscore}{\kern0pt}P{\isacharunderscore}{\kern0pt}name{\isacharunderscore}{\kern0pt}fm{\isacharparenright}{\kern0pt}\isanewline
\ \ \isacommand{using}\isamarkupfalse%
\ assms\ \isanewline
\ \ \ \ \ \isacommand{apply}\isamarkupfalse%
\ auto{\isacharbrackleft}{\kern0pt}{\isadigit{2}}{\isacharbrackright}{\kern0pt}\isanewline
\ \ \ \isacommand{apply}\isamarkupfalse%
{\isacharparenleft}{\kern0pt}rule\ Un{\isacharunderscore}{\kern0pt}least{\isacharunderscore}{\kern0pt}lt{\isacharcomma}{\kern0pt}\ simp{\isacharcomma}{\kern0pt}\ rule\ ltI{\isacharcomma}{\kern0pt}\ simp{\isacharcomma}{\kern0pt}\ simp\ add{\isacharcolon}{\kern0pt}assms{\isacharparenright}{\kern0pt}\isanewline
\ \ \ \isacommand{apply}\isamarkupfalse%
{\isacharparenleft}{\kern0pt}simp{\isacharcomma}{\kern0pt}\ rule\ ltI{\isacharcomma}{\kern0pt}\ simp{\isacharcomma}{\kern0pt}\ simp\ add{\isacharcolon}{\kern0pt}assms{\isacharparenright}{\kern0pt}\isanewline
\ \ \isacommand{apply}\isamarkupfalse%
{\isacharparenleft}{\kern0pt}rule\ le{\isacharunderscore}{\kern0pt}trans{\isacharcomma}{\kern0pt}\ rule\ arity{\isacharunderscore}{\kern0pt}is{\isacharunderscore}{\kern0pt}memrel{\isacharunderscore}{\kern0pt}wftrec{\isacharunderscore}{\kern0pt}fm{\isacharparenright}{\kern0pt}\isanewline
\ \ \isacommand{unfolding}\isamarkupfalse%
\ HPn{\isacharunderscore}{\kern0pt}auto{\isacharunderscore}{\kern0pt}M{\isacharunderscore}{\kern0pt}fm{\isacharprime}{\kern0pt}{\isacharunderscore}{\kern0pt}def\isanewline
\ \ \isacommand{using}\isamarkupfalse%
\ assms\ \isanewline
\ \ \ \ \ \ \ \isacommand{apply}\isamarkupfalse%
\ simp\isanewline
\ \ \ \ \ \ \isacommand{apply}\isamarkupfalse%
\ {\isacharparenleft}{\kern0pt}simp\ del{\isacharcolon}{\kern0pt}FOL{\isacharunderscore}{\kern0pt}sats{\isacharunderscore}{\kern0pt}iff\ pair{\isacharunderscore}{\kern0pt}abs\ add{\isacharcolon}{\kern0pt}\ fm{\isacharunderscore}{\kern0pt}defs\ nat{\isacharunderscore}{\kern0pt}simp{\isacharunderscore}{\kern0pt}union{\isacharparenright}{\kern0pt}\isanewline
\ \ \isacommand{using}\isamarkupfalse%
\ assms\ \isanewline
\ \ \ \ \ \isacommand{apply}\isamarkupfalse%
\ auto{\isacharbrackleft}{\kern0pt}{\isadigit{3}}{\isacharbrackright}{\kern0pt}\isanewline
\ \ \isacommand{apply}\isamarkupfalse%
{\isacharparenleft}{\kern0pt}rule\ Un{\isacharunderscore}{\kern0pt}least{\isacharunderscore}{\kern0pt}lt{\isacharparenright}{\kern0pt}{\isacharplus}{\kern0pt}\isanewline
\ \ \ \ \isacommand{apply}\isamarkupfalse%
{\isacharparenleft}{\kern0pt}simp{\isacharcomma}{\kern0pt}\ rule\ ltI{\isacharcomma}{\kern0pt}\ simp{\isacharcomma}{\kern0pt}\ simp\ add{\isacharcolon}{\kern0pt}assms{\isacharparenright}{\kern0pt}{\isacharplus}{\kern0pt}\isanewline
\ \ \isacommand{done}\isamarkupfalse%
%
\endisatagproof
{\isafoldproof}%
%
\isadelimproof
\isanewline
%
\endisadelimproof
\isanewline
\isacommand{lemma}\isamarkupfalse%
\ sats{\isacharunderscore}{\kern0pt}is{\isacharunderscore}{\kern0pt}Pn{\isacharunderscore}{\kern0pt}auto{\isacharunderscore}{\kern0pt}fm{\isacharunderscore}{\kern0pt}iff\ {\isacharcolon}{\kern0pt}\isanewline
\ \ \isakeyword{fixes}\ x\ {\isasympi}\ v\ env\ i\ j\ k\ l\isanewline
\ \ \isakeyword{assumes}\ {\isachardoublequoteopen}i\ {\isacharless}{\kern0pt}\ length{\isacharparenleft}{\kern0pt}env{\isacharparenright}{\kern0pt}{\isachardoublequoteclose}\ {\isachardoublequoteopen}j\ {\isacharless}{\kern0pt}\ length{\isacharparenleft}{\kern0pt}env{\isacharparenright}{\kern0pt}{\isachardoublequoteclose}\ {\isachardoublequoteopen}k\ {\isacharless}{\kern0pt}\ length{\isacharparenleft}{\kern0pt}env{\isacharparenright}{\kern0pt}{\isachardoublequoteclose}\ {\isachardoublequoteopen}l\ {\isacharless}{\kern0pt}\ length{\isacharparenleft}{\kern0pt}env{\isacharparenright}{\kern0pt}{\isachardoublequoteclose}\ \isanewline
\ \ \ \ \ \ \ \ \ \ {\isachardoublequoteopen}nth{\isacharparenleft}{\kern0pt}i{\isacharcomma}{\kern0pt}\ env{\isacharparenright}{\kern0pt}\ {\isacharequal}{\kern0pt}\ P{\isachardoublequoteclose}\ {\isachardoublequoteopen}nth{\isacharparenleft}{\kern0pt}j{\isacharcomma}{\kern0pt}\ env{\isacharparenright}{\kern0pt}\ {\isacharequal}{\kern0pt}\ {\isasympi}{\isachardoublequoteclose}\ {\isachardoublequoteopen}nth{\isacharparenleft}{\kern0pt}k{\isacharcomma}{\kern0pt}\ env{\isacharparenright}{\kern0pt}\ {\isacharequal}{\kern0pt}\ x{\isachardoublequoteclose}\ {\isachardoublequoteopen}nth{\isacharparenleft}{\kern0pt}l{\isacharcomma}{\kern0pt}\ env{\isacharparenright}{\kern0pt}\ {\isacharequal}{\kern0pt}\ v{\isachardoublequoteclose}\ \isanewline
\ \ \ \ \ \ \ \ \ \ {\isachardoublequoteopen}env\ {\isasymin}\ list{\isacharparenleft}{\kern0pt}M{\isacharparenright}{\kern0pt}{\isachardoublequoteclose}\ {\isachardoublequoteopen}{\isasympi}\ {\isasymin}\ P{\isacharunderscore}{\kern0pt}auto{\isachardoublequoteclose}\ \isanewline
\ \ \isakeyword{shows}\ {\isachardoublequoteopen}sats{\isacharparenleft}{\kern0pt}M{\isacharcomma}{\kern0pt}\ is{\isacharunderscore}{\kern0pt}Pn{\isacharunderscore}{\kern0pt}auto{\isacharunderscore}{\kern0pt}fm{\isacharparenleft}{\kern0pt}i{\isacharcomma}{\kern0pt}\ j{\isacharcomma}{\kern0pt}\ k{\isacharcomma}{\kern0pt}\ l{\isacharparenright}{\kern0pt}{\isacharcomma}{\kern0pt}\ env{\isacharparenright}{\kern0pt}\ {\isasymlongleftrightarrow}\ x\ {\isasymin}\ P{\isacharunderscore}{\kern0pt}names\ {\isasymand}\ v\ {\isacharequal}{\kern0pt}\ Pn{\isacharunderscore}{\kern0pt}auto{\isacharparenleft}{\kern0pt}{\isasympi}{\isacharparenright}{\kern0pt}{\isacharbackquote}{\kern0pt}x{\isachardoublequoteclose}\ \isanewline
%
\isadelimproof
%
\endisadelimproof
%
\isatagproof
\isacommand{proof}\isamarkupfalse%
\ {\isacharminus}{\kern0pt}\ \isanewline
\isanewline
\ \ \isacommand{have}\isamarkupfalse%
\ piinM\ {\isacharcolon}{\kern0pt}\ {\isachardoublequoteopen}{\isasympi}\ {\isasymin}\ M{\isachardoublequoteclose}\ \isanewline
\ \ \ \ \isacommand{using}\isamarkupfalse%
\ pair{\isacharunderscore}{\kern0pt}in{\isacharunderscore}{\kern0pt}M{\isacharunderscore}{\kern0pt}iff\ assms\ P{\isacharunderscore}{\kern0pt}auto{\isacharunderscore}{\kern0pt}def\ is{\isacharunderscore}{\kern0pt}P{\isacharunderscore}{\kern0pt}auto{\isacharunderscore}{\kern0pt}def\ \isanewline
\ \ \ \ \isacommand{by}\isamarkupfalse%
\ auto\ \ \isanewline
\isanewline
\ \ \isacommand{have}\isamarkupfalse%
\ {\isachardoublequoteopen}sats{\isacharparenleft}{\kern0pt}M{\isacharcomma}{\kern0pt}\ is{\isacharunderscore}{\kern0pt}Pn{\isacharunderscore}{\kern0pt}auto{\isacharunderscore}{\kern0pt}fm{\isacharparenleft}{\kern0pt}i{\isacharcomma}{\kern0pt}\ j{\isacharcomma}{\kern0pt}\ k{\isacharcomma}{\kern0pt}\ l{\isacharparenright}{\kern0pt}{\isacharcomma}{\kern0pt}\ env{\isacharparenright}{\kern0pt}\ {\isasymlongleftrightarrow}\ x\ {\isasymin}\ P{\isacharunderscore}{\kern0pt}names\ {\isasymand}\ v\ {\isacharequal}{\kern0pt}\ wftrec{\isacharparenleft}{\kern0pt}Memrel{\isacharparenleft}{\kern0pt}M{\isacharparenright}{\kern0pt}{\isacharcircum}{\kern0pt}{\isacharplus}{\kern0pt}{\isacharcomma}{\kern0pt}\ x{\isacharcomma}{\kern0pt}\ HPn{\isacharunderscore}{\kern0pt}auto{\isacharparenleft}{\kern0pt}{\isasympi}{\isacharparenright}{\kern0pt}{\isacharparenright}{\kern0pt}{\isachardoublequoteclose}\isanewline
\ \ \ \ \isacommand{unfolding}\isamarkupfalse%
\ is{\isacharunderscore}{\kern0pt}Pn{\isacharunderscore}{\kern0pt}auto{\isacharunderscore}{\kern0pt}fm{\isacharunderscore}{\kern0pt}def\ \isanewline
\ \ \ \ \isacommand{apply}\isamarkupfalse%
{\isacharparenleft}{\kern0pt}rule\ iff{\isacharunderscore}{\kern0pt}trans{\isacharcomma}{\kern0pt}\ rule\ sats{\isacharunderscore}{\kern0pt}And{\isacharunderscore}{\kern0pt}iff{\isacharcomma}{\kern0pt}\ simp\ add{\isacharcolon}{\kern0pt}assms{\isacharcomma}{\kern0pt}\ rule\ iff{\isacharunderscore}{\kern0pt}conjI{\isacharparenright}{\kern0pt}\isanewline
\ \ \ \ \ \isacommand{apply}\isamarkupfalse%
{\isacharparenleft}{\kern0pt}rule\ sats{\isacharunderscore}{\kern0pt}is{\isacharunderscore}{\kern0pt}P{\isacharunderscore}{\kern0pt}name{\isacharunderscore}{\kern0pt}fm{\isacharunderscore}{\kern0pt}iff{\isacharparenright}{\kern0pt}\isanewline
\ \ \ \ \isacommand{using}\isamarkupfalse%
\ assms\ \isanewline
\ \ \ \ \ \ \ \ \ \isacommand{apply}\isamarkupfalse%
\ auto{\isacharbrackleft}{\kern0pt}{\isadigit{5}}{\isacharbrackright}{\kern0pt}\isanewline
\ \ \ \ \isacommand{apply}\isamarkupfalse%
{\isacharparenleft}{\kern0pt}rule{\isacharunderscore}{\kern0pt}tac\ a{\isacharequal}{\kern0pt}{\isasympi}\ \isakeyword{and}\ G{\isacharequal}{\kern0pt}HPn{\isacharunderscore}{\kern0pt}auto{\isacharunderscore}{\kern0pt}M\ \isakeyword{in}\ sats{\isacharunderscore}{\kern0pt}is{\isacharunderscore}{\kern0pt}memrel{\isacharunderscore}{\kern0pt}wftrec{\isacharunderscore}{\kern0pt}fm{\isacharunderscore}{\kern0pt}iff{\isacharparenright}{\kern0pt}\isanewline
\ \ \ \ \isacommand{using}\isamarkupfalse%
\ assms\ P{\isacharunderscore}{\kern0pt}auto{\isacharunderscore}{\kern0pt}in{\isacharunderscore}{\kern0pt}M\isanewline
\ \ \ \ \ \ \ \ \ \ \ \ \ \ \ \ \ \ \isacommand{apply}\isamarkupfalse%
\ auto{\isacharbrackleft}{\kern0pt}{\isadigit{1}}{\isadigit{0}}{\isacharbrackright}{\kern0pt}\isanewline
\ \ \ \ \ \ \ \ \isacommand{apply}\isamarkupfalse%
{\isacharparenleft}{\kern0pt}simp\ add{\isacharcolon}{\kern0pt}HPn{\isacharunderscore}{\kern0pt}auto{\isacharunderscore}{\kern0pt}M{\isacharunderscore}{\kern0pt}fm{\isacharprime}{\kern0pt}{\isacharunderscore}{\kern0pt}def{\isacharparenright}{\kern0pt}{\isacharplus}{\kern0pt}\isanewline
\ \ \ \ \isacommand{apply}\isamarkupfalse%
{\isacharparenleft}{\kern0pt}subst\ arity{\isacharunderscore}{\kern0pt}pair{\isacharunderscore}{\kern0pt}fm{\isacharcomma}{\kern0pt}\ simp{\isacharunderscore}{\kern0pt}all{\isacharparenright}{\kern0pt}{\isacharplus}{\kern0pt}\isanewline
\ \ \ \ \isacommand{apply}\isamarkupfalse%
{\isacharparenleft}{\kern0pt}subst\ arity{\isacharunderscore}{\kern0pt}function{\isacharunderscore}{\kern0pt}fm{\isacharcomma}{\kern0pt}\ simp{\isacharunderscore}{\kern0pt}all{\isacharparenright}{\kern0pt}\isanewline
\ \ \ \ \isacommand{apply}\isamarkupfalse%
{\isacharparenleft}{\kern0pt}subst\ arity{\isacharunderscore}{\kern0pt}fun{\isacharunderscore}{\kern0pt}apply{\isacharunderscore}{\kern0pt}fm{\isacharcomma}{\kern0pt}\ simp{\isacharunderscore}{\kern0pt}all{\isacharparenright}{\kern0pt}{\isacharplus}{\kern0pt}\isanewline
\ \ \ \ \ \ \ \isacommand{apply}\isamarkupfalse%
\ {\isacharparenleft}{\kern0pt}simp\ del{\isacharcolon}{\kern0pt}FOL{\isacharunderscore}{\kern0pt}sats{\isacharunderscore}{\kern0pt}iff\ pair{\isacharunderscore}{\kern0pt}abs\ add{\isacharcolon}{\kern0pt}\ fm{\isacharunderscore}{\kern0pt}defs\ nat{\isacharunderscore}{\kern0pt}simp{\isacharunderscore}{\kern0pt}union{\isacharparenright}{\kern0pt}\ \isanewline
\ \ \ \ \ \ \isacommand{apply}\isamarkupfalse%
{\isacharparenleft}{\kern0pt}rule\ HPn{\isacharunderscore}{\kern0pt}auto{\isacharunderscore}{\kern0pt}M{\isacharunderscore}{\kern0pt}in{\isacharunderscore}{\kern0pt}M{\isacharparenright}{\kern0pt}\isanewline
\ \ \ \ \ \ \ \ \isacommand{apply}\isamarkupfalse%
\ auto{\isacharbrackleft}{\kern0pt}{\isadigit{3}}{\isacharbrackright}{\kern0pt}\isanewline
\ \ \ \ \ \isacommand{apply}\isamarkupfalse%
{\isacharparenleft}{\kern0pt}rule\ HPn{\isacharunderscore}{\kern0pt}auto{\isacharunderscore}{\kern0pt}M{\isacharunderscore}{\kern0pt}eq{\isacharparenright}{\kern0pt}\isanewline
\ \ \ \ \isacommand{using}\isamarkupfalse%
\ assms\isanewline
\ \ \ \ \ \ \ \ \ \ \ \ \isacommand{apply}\isamarkupfalse%
\ auto{\isacharbrackleft}{\kern0pt}{\isadigit{6}}{\isacharbrackright}{\kern0pt}\isanewline
\ \ \ \ \isacommand{apply}\isamarkupfalse%
{\isacharparenleft}{\kern0pt}rule\ iff{\isacharunderscore}{\kern0pt}trans{\isacharcomma}{\kern0pt}\ rule\ iff{\isacharunderscore}{\kern0pt}flip{\isacharparenright}{\kern0pt}\isanewline
\ \ \ \ \ \isacommand{apply}\isamarkupfalse%
{\isacharparenleft}{\kern0pt}rule\ HPn{\isacharunderscore}{\kern0pt}auto{\isacharunderscore}{\kern0pt}M{\isacharunderscore}{\kern0pt}fm{\isacharprime}{\kern0pt}{\isacharunderscore}{\kern0pt}sats{\isacharunderscore}{\kern0pt}iff{\isacharparenright}{\kern0pt}\isanewline
\ \ \ \ \isacommand{by}\isamarkupfalse%
\ auto\isanewline
\ \ \isacommand{also}\isamarkupfalse%
\ \isacommand{have}\isamarkupfalse%
\ {\isachardoublequoteopen}{\isachardot}{\kern0pt}{\isachardot}{\kern0pt}{\isachardot}{\kern0pt}\ {\isasymlongleftrightarrow}\ x\ {\isasymin}\ P{\isacharunderscore}{\kern0pt}names\ {\isasymand}\ v\ {\isacharequal}{\kern0pt}\ Pn{\isacharunderscore}{\kern0pt}auto{\isacharparenleft}{\kern0pt}{\isasympi}{\isacharparenright}{\kern0pt}{\isacharbackquote}{\kern0pt}x{\isachardoublequoteclose}\ \isanewline
\ \ \ \ \isacommand{apply}\isamarkupfalse%
{\isacharparenleft}{\kern0pt}rule\ iff{\isacharunderscore}{\kern0pt}conjI{\isadigit{2}}{\isacharcomma}{\kern0pt}\ simp{\isacharparenright}{\kern0pt}\isanewline
\ \ \ \ \isacommand{unfolding}\isamarkupfalse%
\ Pn{\isacharunderscore}{\kern0pt}auto{\isacharunderscore}{\kern0pt}def\ \isanewline
\ \ \ \ \isacommand{apply}\isamarkupfalse%
{\isacharparenleft}{\kern0pt}subst\ function{\isacharunderscore}{\kern0pt}apply{\isacharunderscore}{\kern0pt}equality{\isacharparenright}{\kern0pt}\isanewline
\ \ \ \ \isacommand{unfolding}\isamarkupfalse%
\ function{\isacharunderscore}{\kern0pt}def\isanewline
\ \ \ \ \isacommand{by}\isamarkupfalse%
\ auto\ \isanewline
\ \ \isacommand{finally}\isamarkupfalse%
\ \isacommand{show}\isamarkupfalse%
\ {\isacharquery}{\kern0pt}thesis\ \isacommand{by}\isamarkupfalse%
\ simp\isanewline
\isacommand{qed}\isamarkupfalse%
%
\endisatagproof
{\isafoldproof}%
%
\isadelimproof
\isanewline
%
\endisadelimproof
\isanewline
\isacommand{lemma}\isamarkupfalse%
\ Pn{\isacharunderscore}{\kern0pt}auto{\isacharunderscore}{\kern0pt}value{\isacharunderscore}{\kern0pt}in{\isacharunderscore}{\kern0pt}M\ {\isacharcolon}{\kern0pt}\ \isanewline
\ \ {\isachardoublequoteopen}{\isasymAnd}x{\isachardot}{\kern0pt}\ is{\isacharunderscore}{\kern0pt}P{\isacharunderscore}{\kern0pt}auto{\isacharparenleft}{\kern0pt}{\isasympi}{\isacharparenright}{\kern0pt}\ {\isasymLongrightarrow}\ x\ {\isasymin}\ P{\isacharunderscore}{\kern0pt}names\ {\isasymLongrightarrow}\ Pn{\isacharunderscore}{\kern0pt}auto{\isacharparenleft}{\kern0pt}{\isasympi}{\isacharparenright}{\kern0pt}{\isacharbackquote}{\kern0pt}x\ {\isasymin}\ M{\isachardoublequoteclose}\ \isanewline
%
\isadelimproof
\isanewline
\ \ %
\endisadelimproof
%
\isatagproof
\isacommand{apply}\isamarkupfalse%
{\isacharparenleft}{\kern0pt}rename{\isacharunderscore}{\kern0pt}tac\ x{\isacharcomma}{\kern0pt}\ rule{\isacharunderscore}{\kern0pt}tac\ b{\isacharequal}{\kern0pt}{\isachardoublequoteopen}Pn{\isacharunderscore}{\kern0pt}auto{\isacharparenleft}{\kern0pt}{\isasympi}{\isacharparenright}{\kern0pt}{\isacharbackquote}{\kern0pt}x{\isachardoublequoteclose}\ \isakeyword{and}\ a{\isacharequal}{\kern0pt}{\isachardoublequoteopen}wftrec{\isacharparenleft}{\kern0pt}Memrel{\isacharparenleft}{\kern0pt}M{\isacharparenright}{\kern0pt}{\isacharcircum}{\kern0pt}{\isacharplus}{\kern0pt}{\isacharcomma}{\kern0pt}\ x{\isacharcomma}{\kern0pt}\ HPn{\isacharunderscore}{\kern0pt}auto{\isacharparenleft}{\kern0pt}{\isasympi}{\isacharparenright}{\kern0pt}{\isacharparenright}{\kern0pt}{\isachardoublequoteclose}\ \isakeyword{in}\ ssubst{\isacharparenright}{\kern0pt}\isanewline
\ \ \ \isacommand{apply}\isamarkupfalse%
{\isacharparenleft}{\kern0pt}rule\ function{\isacharunderscore}{\kern0pt}apply{\isacharunderscore}{\kern0pt}equality{\isacharparenright}{\kern0pt}\isanewline
\ \ \ \ \isacommand{apply}\isamarkupfalse%
{\isacharparenleft}{\kern0pt}simp\ add{\isacharcolon}{\kern0pt}Pn{\isacharunderscore}{\kern0pt}auto{\isacharunderscore}{\kern0pt}def{\isacharparenright}{\kern0pt}\isanewline
\ \ \ \isacommand{apply}\isamarkupfalse%
{\isacharparenleft}{\kern0pt}simp\ add{\isacharcolon}{\kern0pt}Pn{\isacharunderscore}{\kern0pt}auto{\isacharunderscore}{\kern0pt}def\ function{\isacharunderscore}{\kern0pt}def{\isacharparenright}{\kern0pt}\isanewline
\ \ \isacommand{apply}\isamarkupfalse%
{\isacharparenleft}{\kern0pt}rule{\isacharunderscore}{\kern0pt}tac\ G{\isacharequal}{\kern0pt}HPn{\isacharunderscore}{\kern0pt}auto{\isacharunderscore}{\kern0pt}M\ \isakeyword{and}\ Gfm\ {\isacharequal}{\kern0pt}\ HPn{\isacharunderscore}{\kern0pt}auto{\isacharunderscore}{\kern0pt}M{\isacharunderscore}{\kern0pt}fm{\isacharprime}{\kern0pt}\ \isakeyword{and}\ a{\isacharequal}{\kern0pt}{\isasympi}\ \isakeyword{in}\ memrel{\isacharunderscore}{\kern0pt}wftrec{\isacharunderscore}{\kern0pt}in{\isacharunderscore}{\kern0pt}M{\isacharparenright}{\kern0pt}\isanewline
\ \ \isacommand{using}\isamarkupfalse%
\ P{\isacharunderscore}{\kern0pt}name{\isacharunderscore}{\kern0pt}in{\isacharunderscore}{\kern0pt}M\ is{\isacharunderscore}{\kern0pt}P{\isacharunderscore}{\kern0pt}auto{\isacharunderscore}{\kern0pt}def\ \isanewline
\ \ \ \ \ \ \ \ \isacommand{apply}\isamarkupfalse%
\ auto{\isacharbrackleft}{\kern0pt}{\isadigit{2}}{\isacharbrackright}{\kern0pt}\isanewline
\ \ \ \ \ \ \isacommand{apply}\isamarkupfalse%
{\isacharparenleft}{\kern0pt}simp\ add{\isacharcolon}{\kern0pt}HPn{\isacharunderscore}{\kern0pt}auto{\isacharunderscore}{\kern0pt}M{\isacharunderscore}{\kern0pt}fm{\isacharprime}{\kern0pt}{\isacharunderscore}{\kern0pt}def{\isacharparenright}{\kern0pt}{\isacharplus}{\kern0pt}\isanewline
\ \ \ \ \ \isacommand{apply}\isamarkupfalse%
{\isacharparenleft}{\kern0pt}subst\ arity{\isacharunderscore}{\kern0pt}pair{\isacharunderscore}{\kern0pt}fm{\isacharcomma}{\kern0pt}\ simp{\isacharunderscore}{\kern0pt}all{\isacharparenright}{\kern0pt}{\isacharplus}{\kern0pt}\isanewline
\ \ \ \ \ \isacommand{apply}\isamarkupfalse%
{\isacharparenleft}{\kern0pt}subst\ arity{\isacharunderscore}{\kern0pt}function{\isacharunderscore}{\kern0pt}fm{\isacharcomma}{\kern0pt}\ simp{\isacharunderscore}{\kern0pt}all{\isacharparenright}{\kern0pt}\isanewline
\ \ \ \ \ \isacommand{apply}\isamarkupfalse%
{\isacharparenleft}{\kern0pt}subst\ arity{\isacharunderscore}{\kern0pt}fun{\isacharunderscore}{\kern0pt}apply{\isacharunderscore}{\kern0pt}fm{\isacharcomma}{\kern0pt}\ simp{\isacharunderscore}{\kern0pt}all{\isacharparenright}{\kern0pt}{\isacharplus}{\kern0pt}\isanewline
\ \ \ \ \ \isacommand{apply}\isamarkupfalse%
\ {\isacharparenleft}{\kern0pt}simp\ del{\isacharcolon}{\kern0pt}FOL{\isacharunderscore}{\kern0pt}sats{\isacharunderscore}{\kern0pt}iff\ pair{\isacharunderscore}{\kern0pt}abs\ add{\isacharcolon}{\kern0pt}\ fm{\isacharunderscore}{\kern0pt}defs\ nat{\isacharunderscore}{\kern0pt}simp{\isacharunderscore}{\kern0pt}union{\isacharparenright}{\kern0pt}\ \isanewline
\ \ \ \ \isacommand{apply}\isamarkupfalse%
{\isacharparenleft}{\kern0pt}rule\ HPn{\isacharunderscore}{\kern0pt}auto{\isacharunderscore}{\kern0pt}M{\isacharunderscore}{\kern0pt}in{\isacharunderscore}{\kern0pt}M{\isacharparenright}{\kern0pt}\isanewline
\ \ \ \ \ \ \isacommand{apply}\isamarkupfalse%
\ auto{\isacharbrackleft}{\kern0pt}{\isadigit{3}}{\isacharbrackright}{\kern0pt}\isanewline
\ \ \ \isacommand{apply}\isamarkupfalse%
{\isacharparenleft}{\kern0pt}rule\ HPn{\isacharunderscore}{\kern0pt}auto{\isacharunderscore}{\kern0pt}M{\isacharunderscore}{\kern0pt}eq{\isacharparenright}{\kern0pt}\isanewline
\ \ \ \ \ \ \ \ \ \ \ \isacommand{apply}\isamarkupfalse%
{\isacharparenleft}{\kern0pt}simp\ add{\isacharcolon}{\kern0pt}P{\isacharunderscore}{\kern0pt}auto{\isacharunderscore}{\kern0pt}def{\isacharparenright}{\kern0pt}\isanewline
\ \ \isacommand{unfolding}\isamarkupfalse%
\ is{\isacharunderscore}{\kern0pt}P{\isacharunderscore}{\kern0pt}auto{\isacharunderscore}{\kern0pt}def\ bij{\isacharunderscore}{\kern0pt}def\ inj{\isacharunderscore}{\kern0pt}def\ \isanewline
\ \ \isacommand{apply}\isamarkupfalse%
\ auto{\isacharbrackleft}{\kern0pt}{\isadigit{6}}{\isacharbrackright}{\kern0pt}\isanewline
\ \ \isacommand{apply}\isamarkupfalse%
{\isacharparenleft}{\kern0pt}rule\ iff{\isacharunderscore}{\kern0pt}trans{\isacharcomma}{\kern0pt}\ rule\ iff{\isacharunderscore}{\kern0pt}flip{\isacharcomma}{\kern0pt}\ rule\ HPn{\isacharunderscore}{\kern0pt}auto{\isacharunderscore}{\kern0pt}M{\isacharunderscore}{\kern0pt}fm{\isacharprime}{\kern0pt}{\isacharunderscore}{\kern0pt}sats{\isacharunderscore}{\kern0pt}iff{\isacharparenright}{\kern0pt}\isanewline
\ \ \ \ \ \ \isacommand{apply}\isamarkupfalse%
\ auto\isanewline
\ \ \isacommand{done}\isamarkupfalse%
%
\endisatagproof
{\isafoldproof}%
%
\isadelimproof
\isanewline
%
\endisadelimproof
\isanewline
\isacommand{end}\isamarkupfalse%
\isanewline
%
\isadelimtheory
%
\endisadelimtheory
%
\isatagtheory
\isacommand{end}\isamarkupfalse%
%
\endisatagtheory
{\isafoldtheory}%
%
\isadelimtheory
%
\endisadelimtheory
%
\end{isabellebody}%
\endinput
%:%file=~/source/repos/ZF-notAC/code/Automorphism_M.thy%:%
%:%10=1%:%
%:%11=1%:%
%:%12=2%:%
%:%13=3%:%
%:%14=4%:%
%:%15=5%:%
%:%16=6%:%
%:%21=6%:%
%:%24=7%:%
%:%25=8%:%
%:%26=8%:%
%:%27=9%:%
%:%28=10%:%
%:%29=11%:%
%:%30=12%:%
%:%31=13%:%
%:%32=14%:%
%:%33=15%:%
%:%37=19%:%
%:%40=20%:%
%:%44=20%:%
%:%45=20%:%
%:%50=20%:%
%:%53=21%:%
%:%54=22%:%
%:%55=22%:%
%:%56=23%:%
%:%72=39%:%
%:%73=40%:%
%:%74=41%:%
%:%75=41%:%
%:%76=42%:%
%:%77=43%:%
%:%78=44%:%
%:%79=44%:%
%:%80=45%:%
%:%83=46%:%
%:%84=47%:%
%:%88=47%:%
%:%89=47%:%
%:%90=48%:%
%:%91=48%:%
%:%93=50%:%
%:%94=51%:%
%:%95=51%:%
%:%96=52%:%
%:%97=52%:%
%:%98=53%:%
%:%99=53%:%
%:%100=54%:%
%:%101=54%:%
%:%102=55%:%
%:%103=55%:%
%:%104=56%:%
%:%105=56%:%
%:%106=57%:%
%:%107=57%:%
%:%108=58%:%
%:%109=58%:%
%:%110=59%:%
%:%111=59%:%
%:%112=60%:%
%:%113=60%:%
%:%114=61%:%
%:%115=61%:%
%:%116=62%:%
%:%117=62%:%
%:%118=63%:%
%:%119=63%:%
%:%120=64%:%
%:%121=64%:%
%:%122=65%:%
%:%123=65%:%
%:%128=65%:%
%:%131=66%:%
%:%132=67%:%
%:%133=67%:%
%:%140=68%:%
%:%141=68%:%
%:%142=69%:%
%:%143=69%:%
%:%144=70%:%
%:%145=70%:%
%:%146=71%:%
%:%147=71%:%
%:%148=72%:%
%:%149=72%:%
%:%150=73%:%
%:%151=73%:%
%:%152=74%:%
%:%153=74%:%
%:%154=75%:%
%:%155=75%:%
%:%156=76%:%
%:%157=76%:%
%:%158=76%:%
%:%159=77%:%
%:%160=77%:%
%:%161=78%:%
%:%162=78%:%
%:%163=79%:%
%:%164=79%:%
%:%165=80%:%
%:%166=80%:%
%:%167=81%:%
%:%168=81%:%
%:%169=81%:%
%:%170=82%:%
%:%171=82%:%
%:%172=83%:%
%:%173=83%:%
%:%174=84%:%
%:%175=84%:%
%:%176=85%:%
%:%177=85%:%
%:%178=86%:%
%:%179=86%:%
%:%180=87%:%
%:%181=87%:%
%:%182=87%:%
%:%183=88%:%
%:%184=88%:%
%:%185=89%:%
%:%186=89%:%
%:%187=90%:%
%:%188=90%:%
%:%189=91%:%
%:%190=91%:%
%:%191=92%:%
%:%192=92%:%
%:%193=93%:%
%:%194=93%:%
%:%195=94%:%
%:%196=94%:%
%:%197=95%:%
%:%198=95%:%
%:%199=96%:%
%:%200=96%:%
%:%201=97%:%
%:%202=97%:%
%:%203=98%:%
%:%209=98%:%
%:%212=99%:%
%:%213=100%:%
%:%214=100%:%
%:%215=101%:%
%:%226=112%:%
%:%227=113%:%
%:%228=114%:%
%:%229=114%:%
%:%230=115%:%
%:%237=122%:%
%:%244=123%:%
%:%245=123%:%
%:%246=124%:%
%:%247=124%:%
%:%248=125%:%
%:%249=126%:%
%:%250=126%:%
%:%251=126%:%
%:%252=127%:%
%:%253=128%:%
%:%254=129%:%
%:%255=130%:%
%:%256=131%:%
%:%257=132%:%
%:%258=133%:%
%:%259=134%:%
%:%260=135%:%
%:%261=136%:%
%:%262=137%:%
%:%263=137%:%
%:%264=137%:%
%:%265=138%:%
%:%266=138%:%
%:%267=138%:%
%:%268=138%:%
%:%269=139%:%
%:%270=139%:%
%:%271=139%:%
%:%272=139%:%
%:%273=140%:%
%:%274=140%:%
%:%275=140%:%
%:%276=140%:%
%:%277=141%:%
%:%278=141%:%
%:%279=141%:%
%:%280=141%:%
%:%281=142%:%
%:%282=142%:%
%:%283=142%:%
%:%284=142%:%
%:%285=143%:%
%:%286=143%:%
%:%287=143%:%
%:%288=143%:%
%:%289=144%:%
%:%290=144%:%
%:%291=144%:%
%:%292=144%:%
%:%293=145%:%
%:%294=145%:%
%:%295=145%:%
%:%296=145%:%
%:%297=146%:%
%:%298=146%:%
%:%299=146%:%
%:%300=146%:%
%:%301=147%:%
%:%302=147%:%
%:%303=147%:%
%:%304=147%:%
%:%305=148%:%
%:%306=148%:%
%:%307=149%:%
%:%308=149%:%
%:%309=150%:%
%:%310=150%:%
%:%311=151%:%
%:%312=151%:%
%:%313=151%:%
%:%314=151%:%
%:%315=152%:%
%:%316=152%:%
%:%321=157%:%
%:%322=158%:%
%:%323=158%:%
%:%324=158%:%
%:%325=159%:%
%:%326=159%:%
%:%327=159%:%
%:%328=160%:%
%:%329=160%:%
%:%330=160%:%
%:%331=161%:%
%:%332=161%:%
%:%333=162%:%
%:%334=162%:%
%:%335=163%:%
%:%336=164%:%
%:%337=165%:%
%:%338=166%:%
%:%339=166%:%
%:%340=166%:%
%:%341=167%:%
%:%342=168%:%
%:%343=169%:%
%:%344=169%:%
%:%345=170%:%
%:%346=171%:%
%:%347=171%:%
%:%348=172%:%
%:%349=172%:%
%:%350=173%:%
%:%351=173%:%
%:%352=174%:%
%:%353=174%:%
%:%354=175%:%
%:%355=175%:%
%:%356=176%:%
%:%357=176%:%
%:%358=177%:%
%:%359=177%:%
%:%360=178%:%
%:%361=178%:%
%:%362=179%:%
%:%363=179%:%
%:%364=180%:%
%:%365=180%:%
%:%366=181%:%
%:%367=181%:%
%:%368=181%:%
%:%369=182%:%
%:%375=182%:%
%:%378=183%:%
%:%379=184%:%
%:%380=184%:%
%:%381=185%:%
%:%382=186%:%
%:%385=187%:%
%:%389=187%:%
%:%390=187%:%
%:%391=188%:%
%:%392=188%:%
%:%393=189%:%
%:%394=189%:%
%:%395=190%:%
%:%396=190%:%
%:%397=191%:%
%:%398=192%:%
%:%399=193%:%
%:%400=194%:%
%:%401=194%:%
%:%402=194%:%
%:%407=199%:%
%:%408=199%:%
%:%409=199%:%
%:%410=200%:%
%:%411=200%:%
%:%412=200%:%
%:%413=201%:%
%:%414=202%:%
%:%415=203%:%
%:%416=203%:%
%:%417=204%:%
%:%418=205%:%
%:%419=205%:%
%:%420=206%:%
%:%421=206%:%
%:%422=207%:%
%:%423=207%:%
%:%424=208%:%
%:%425=208%:%
%:%426=209%:%
%:%427=209%:%
%:%428=210%:%
%:%429=210%:%
%:%430=211%:%
%:%431=211%:%
%:%432=212%:%
%:%433=212%:%
%:%434=213%:%
%:%435=213%:%
%:%436=213%:%
%:%437=213%:%
%:%438=214%:%
%:%439=214%:%
%:%440=214%:%
%:%441=214%:%
%:%442=214%:%
%:%443=215%:%
%:%444=215%:%
%:%445=215%:%
%:%446=215%:%
%:%447=215%:%
%:%448=216%:%
%:%449=216%:%
%:%450=216%:%
%:%451=216%:%
%:%452=216%:%
%:%453=217%:%
%:%454=217%:%
%:%455=217%:%
%:%456=217%:%
%:%457=217%:%
%:%458=218%:%
%:%459=218%:%
%:%460=218%:%
%:%461=219%:%
%:%462=219%:%
%:%463=220%:%
%:%464=220%:%
%:%465=221%:%
%:%466=221%:%
%:%467=222%:%
%:%468=222%:%
%:%469=223%:%
%:%470=223%:%
%:%471=224%:%
%:%472=224%:%
%:%473=225%:%
%:%474=225%:%
%:%475=226%:%
%:%476=226%:%
%:%477=226%:%
%:%478=227%:%
%:%479=227%:%
%:%480=228%:%
%:%481=229%:%
%:%482=229%:%
%:%483=230%:%
%:%484=230%:%
%:%485=231%:%
%:%486=231%:%
%:%487=232%:%
%:%488=232%:%
%:%489=233%:%
%:%490=233%:%
%:%491=234%:%
%:%492=234%:%
%:%493=235%:%
%:%494=235%:%
%:%495=236%:%
%:%496=236%:%
%:%497=237%:%
%:%498=237%:%
%:%499=237%:%
%:%500=237%:%
%:%501=238%:%
%:%502=238%:%
%:%503=238%:%
%:%504=238%:%
%:%505=238%:%
%:%506=239%:%
%:%507=239%:%
%:%508=239%:%
%:%509=239%:%
%:%510=240%:%
%:%511=240%:%
%:%512=240%:%
%:%513=240%:%
%:%514=240%:%
%:%515=241%:%
%:%516=241%:%
%:%517=241%:%
%:%518=242%:%
%:%519=242%:%
%:%520=243%:%
%:%521=243%:%
%:%522=244%:%
%:%523=244%:%
%:%524=245%:%
%:%525=245%:%
%:%526=246%:%
%:%527=246%:%
%:%528=247%:%
%:%529=247%:%
%:%530=248%:%
%:%531=248%:%
%:%532=249%:%
%:%533=249%:%
%:%534=250%:%
%:%535=250%:%
%:%536=251%:%
%:%537=251%:%
%:%538=252%:%
%:%539=252%:%
%:%540=253%:%
%:%541=253%:%
%:%542=254%:%
%:%543=254%:%
%:%544=255%:%
%:%545=255%:%
%:%546=256%:%
%:%547=256%:%
%:%548=257%:%
%:%549=257%:%
%:%550=258%:%
%:%551=258%:%
%:%552=259%:%
%:%553=260%:%
%:%554=260%:%
%:%555=260%:%
%:%556=261%:%
%:%557=261%:%
%:%558=261%:%
%:%559=262%:%
%:%560=262%:%
%:%561=263%:%
%:%562=263%:%
%:%563=264%:%
%:%564=264%:%
%:%565=264%:%
%:%566=265%:%
%:%567=265%:%
%:%568=266%:%
%:%569=266%:%
%:%570=267%:%
%:%571=267%:%
%:%572=268%:%
%:%573=268%:%
%:%574=269%:%
%:%575=269%:%
%:%576=270%:%
%:%577=270%:%
%:%578=271%:%
%:%579=271%:%
%:%580=272%:%
%:%581=272%:%
%:%582=273%:%
%:%583=273%:%
%:%584=274%:%
%:%585=274%:%
%:%586=274%:%
%:%587=274%:%
%:%588=275%:%
%:%594=275%:%
%:%597=276%:%
%:%598=277%:%
%:%599=277%:%
%:%600=278%:%
%:%601=279%:%
%:%602=280%:%
%:%603=281%:%
%:%604=282%:%
%:%605=283%:%
%:%606=284%:%
%:%607=285%:%
%:%610=286%:%
%:%614=286%:%
%:%615=286%:%
%:%616=287%:%
%:%617=287%:%
%:%622=287%:%
%:%625=288%:%
%:%626=289%:%
%:%627=289%:%
%:%628=290%:%
%:%629=291%:%
%:%630=291%:%
%:%631=292%:%
%:%647=308%:%
%:%648=309%:%
%:%649=310%:%
%:%650=310%:%
%:%651=311%:%
%:%652=312%:%
%:%653=313%:%
%:%654=313%:%
%:%655=314%:%
%:%656=315%:%
%:%657=316%:%
%:%658=317%:%
%:%659=318%:%
%:%660=319%:%
%:%661=320%:%
%:%662=321%:%
%:%665=322%:%
%:%669=322%:%
%:%670=322%:%
%:%671=323%:%
%:%672=323%:%
%:%677=323%:%
%:%680=324%:%
%:%681=325%:%
%:%682=325%:%
%:%683=326%:%
%:%684=327%:%
%:%685=327%:%
%:%686=328%:%
%:%702=344%:%
%:%703=345%:%
%:%704=346%:%
%:%705=347%:%
%:%706=347%:%
%:%707=348%:%
%:%708=349%:%
%:%709=350%:%
%:%710=350%:%
%:%711=351%:%
%:%712=352%:%
%:%713=353%:%
%:%714=353%:%
%:%715=354%:%
%:%716=355%:%
%:%719=356%:%
%:%723=356%:%
%:%724=356%:%
%:%725=357%:%
%:%726=357%:%
%:%727=358%:%
%:%728=358%:%
%:%729=359%:%
%:%730=359%:%
%:%735=359%:%
%:%738=360%:%
%:%739=361%:%
%:%740=361%:%
%:%741=362%:%
%:%742=363%:%
%:%745=364%:%
%:%749=364%:%
%:%750=364%:%
%:%751=365%:%
%:%752=365%:%
%:%753=366%:%
%:%754=366%:%
%:%755=367%:%
%:%756=367%:%
%:%757=368%:%
%:%758=368%:%
%:%759=369%:%
%:%760=369%:%
%:%761=370%:%
%:%762=370%:%
%:%763=371%:%
%:%764=371%:%
%:%769=371%:%
%:%772=372%:%
%:%773=373%:%
%:%774=373%:%
%:%775=374%:%
%:%782=375%:%
%:%783=375%:%
%:%784=376%:%
%:%785=376%:%
%:%786=377%:%
%:%787=377%:%
%:%788=378%:%
%:%789=379%:%
%:%790=379%:%
%:%791=379%:%
%:%792=380%:%
%:%793=380%:%
%:%794=380%:%
%:%795=381%:%
%:%796=381%:%
%:%797=381%:%
%:%798=381%:%
%:%799=382%:%
%:%800=383%:%
%:%801=383%:%
%:%802=384%:%
%:%803=385%:%
%:%804=385%:%
%:%805=386%:%
%:%806=386%:%
%:%807=387%:%
%:%808=387%:%
%:%809=388%:%
%:%810=388%:%
%:%811=389%:%
%:%812=389%:%
%:%813=390%:%
%:%814=390%:%
%:%815=391%:%
%:%816=391%:%
%:%817=392%:%
%:%818=392%:%
%:%819=392%:%
%:%820=393%:%
%:%821=393%:%
%:%822=394%:%
%:%823=394%:%
%:%824=395%:%
%:%825=395%:%
%:%826=396%:%
%:%827=396%:%
%:%828=397%:%
%:%829=398%:%
%:%830=398%:%
%:%831=399%:%
%:%832=399%:%
%:%833=400%:%
%:%834=400%:%
%:%835=401%:%
%:%836=401%:%
%:%837=402%:%
%:%838=402%:%
%:%839=402%:%
%:%840=403%:%
%:%841=403%:%
%:%842=404%:%
%:%843=404%:%
%:%844=405%:%
%:%845=405%:%
%:%846=406%:%
%:%847=406%:%
%:%848=406%:%
%:%849=407%:%
%:%850=407%:%
%:%851=407%:%
%:%852=408%:%
%:%853=408%:%
%:%854=408%:%
%:%855=408%:%
%:%856=409%:%
%:%857=410%:%
%:%858=410%:%
%:%859=410%:%
%:%860=411%:%
%:%861=411%:%
%:%862=411%:%
%:%863=412%:%
%:%869=412%:%
%:%872=413%:%
%:%873=414%:%
%:%874=414%:%
%:%875=415%:%
%:%882=416%:%
%:%883=416%:%
%:%884=417%:%
%:%885=417%:%
%:%886=418%:%
%:%887=418%:%
%:%888=419%:%
%:%889=420%:%
%:%890=421%:%
%:%891=421%:%
%:%892=422%:%
%:%893=422%:%
%:%894=423%:%
%:%895=423%:%
%:%896=424%:%
%:%897=425%:%
%:%898=425%:%
%:%899=426%:%
%:%900=426%:%
%:%901=427%:%
%:%902=427%:%
%:%903=428%:%
%:%904=428%:%
%:%905=429%:%
%:%906=430%:%
%:%907=430%:%
%:%908=431%:%
%:%909=431%:%
%:%910=432%:%
%:%911=432%:%
%:%912=432%:%
%:%913=433%:%
%:%914=433%:%
%:%915=433%:%
%:%916=433%:%
%:%917=433%:%
%:%918=434%:%
%:%919=434%:%
%:%920=435%:%
%:%921=435%:%
%:%922=436%:%
%:%923=436%:%
%:%924=437%:%
%:%925=437%:%
%:%926=438%:%
%:%927=438%:%
%:%928=438%:%
%:%929=438%:%
%:%930=438%:%
%:%931=439%:%
%:%932=440%:%
%:%933=440%:%
%:%934=441%:%
%:%935=441%:%
%:%936=442%:%
%:%937=442%:%
%:%938=443%:%
%:%939=443%:%
%:%940=444%:%
%:%941=445%:%
%:%942=445%:%
%:%943=446%:%
%:%944=446%:%
%:%945=447%:%
%:%946=447%:%
%:%947=448%:%
%:%948=448%:%
%:%949=449%:%
%:%950=450%:%
%:%951=450%:%
%:%952=451%:%
%:%953=451%:%
%:%954=452%:%
%:%955=452%:%
%:%956=453%:%
%:%957=453%:%
%:%958=454%:%
%:%959=454%:%
%:%960=455%:%
%:%961=455%:%
%:%962=456%:%
%:%963=456%:%
%:%964=457%:%
%:%965=457%:%
%:%966=458%:%
%:%967=458%:%
%:%968=459%:%
%:%969=459%:%
%:%970=460%:%
%:%971=461%:%
%:%972=461%:%
%:%973=462%:%
%:%974=462%:%
%:%975=463%:%
%:%976=463%:%
%:%977=464%:%
%:%978=464%:%
%:%979=465%:%
%:%980=465%:%
%:%981=466%:%
%:%982=466%:%
%:%983=467%:%
%:%984=467%:%
%:%985=468%:%
%:%986=468%:%
%:%987=469%:%
%:%988=469%:%
%:%989=470%:%
%:%990=470%:%
%:%991=471%:%
%:%992=471%:%
%:%993=472%:%
%:%994=472%:%
%:%995=473%:%
%:%996=473%:%
%:%997=474%:%
%:%998=474%:%
%:%999=475%:%
%:%1000=475%:%
%:%1001=476%:%
%:%1002=476%:%
%:%1003=477%:%
%:%1004=477%:%
%:%1005=478%:%
%:%1006=478%:%
%:%1007=479%:%
%:%1008=480%:%
%:%1009=480%:%
%:%1010=480%:%
%:%1011=481%:%
%:%1012=481%:%
%:%1013=482%:%
%:%1014=482%:%
%:%1015=483%:%
%:%1016=483%:%
%:%1017=484%:%
%:%1018=484%:%
%:%1019=485%:%
%:%1020=485%:%
%:%1021=485%:%
%:%1022=486%:%
%:%1023=486%:%
%:%1024=486%:%
%:%1025=486%:%
%:%1026=486%:%
%:%1027=487%:%
%:%1028=487%:%
%:%1033=492%:%
%:%1034=493%:%
%:%1035=493%:%
%:%1036=494%:%
%:%1037=494%:%
%:%1038=495%:%
%:%1039=495%:%
%:%1040=496%:%
%:%1041=496%:%
%:%1042=496%:%
%:%1043=496%:%
%:%1044=497%:%
%:%1045=497%:%
%:%1046=497%:%
%:%1047=498%:%
%:%1048=498%:%
%:%1049=499%:%
%:%1050=499%:%
%:%1051=500%:%
%:%1052=500%:%
%:%1053=501%:%
%:%1054=501%:%
%:%1055=502%:%
%:%1056=502%:%
%:%1057=503%:%
%:%1058=503%:%
%:%1059=504%:%
%:%1060=504%:%
%:%1061=505%:%
%:%1062=505%:%
%:%1063=506%:%
%:%1064=506%:%
%:%1065=507%:%
%:%1071=507%:%
%:%1074=508%:%
%:%1075=509%:%
%:%1076=509%:%
%:%1077=510%:%
%:%1078=511%:%
%:%1079=511%:%
%:%1080=512%:%
%:%1081=513%:%
%:%1082=514%:%
%:%1083=514%:%
%:%1084=515%:%
%:%1085=516%:%
%:%1086=517%:%
%:%1087=517%:%
%:%1088=518%:%
%:%1089=519%:%
%:%1090=520%:%
%:%1093=521%:%
%:%1097=521%:%
%:%1098=521%:%
%:%1099=522%:%
%:%1100=522%:%
%:%1101=523%:%
%:%1102=523%:%
%:%1103=524%:%
%:%1104=524%:%
%:%1105=525%:%
%:%1106=525%:%
%:%1107=526%:%
%:%1108=526%:%
%:%1113=526%:%
%:%1116=527%:%
%:%1117=528%:%
%:%1118=528%:%
%:%1119=529%:%
%:%1120=530%:%
%:%1121=531%:%
%:%1124=532%:%
%:%1125=533%:%
%:%1129=533%:%
%:%1130=533%:%
%:%1131=534%:%
%:%1132=534%:%
%:%1133=535%:%
%:%1134=535%:%
%:%1135=536%:%
%:%1136=536%:%
%:%1137=537%:%
%:%1138=537%:%
%:%1139=538%:%
%:%1140=538%:%
%:%1141=539%:%
%:%1142=539%:%
%:%1143=540%:%
%:%1144=540%:%
%:%1145=541%:%
%:%1146=541%:%
%:%1147=542%:%
%:%1148=542%:%
%:%1149=543%:%
%:%1150=543%:%
%:%1151=544%:%
%:%1152=544%:%
%:%1153=545%:%
%:%1154=545%:%
%:%1155=546%:%
%:%1156=546%:%
%:%1157=547%:%
%:%1158=547%:%
%:%1159=548%:%
%:%1160=548%:%
%:%1161=549%:%
%:%1167=549%:%
%:%1170=550%:%
%:%1171=551%:%
%:%1172=551%:%
%:%1173=552%:%
%:%1174=553%:%
%:%1175=554%:%
%:%1176=555%:%
%:%1177=556%:%
%:%1184=557%:%
%:%1185=557%:%
%:%1186=558%:%
%:%1187=559%:%
%:%1188=559%:%
%:%1189=560%:%
%:%1190=560%:%
%:%1191=561%:%
%:%1192=561%:%
%:%1193=562%:%
%:%1194=563%:%
%:%1195=563%:%
%:%1196=564%:%
%:%1197=564%:%
%:%1198=565%:%
%:%1199=565%:%
%:%1200=566%:%
%:%1201=566%:%
%:%1202=567%:%
%:%1203=567%:%
%:%1204=568%:%
%:%1205=568%:%
%:%1206=569%:%
%:%1207=569%:%
%:%1208=570%:%
%:%1209=570%:%
%:%1210=571%:%
%:%1211=571%:%
%:%1212=572%:%
%:%1213=572%:%
%:%1214=573%:%
%:%1215=573%:%
%:%1216=574%:%
%:%1217=574%:%
%:%1218=575%:%
%:%1219=575%:%
%:%1220=576%:%
%:%1221=576%:%
%:%1222=577%:%
%:%1223=577%:%
%:%1224=578%:%
%:%1225=578%:%
%:%1226=579%:%
%:%1227=579%:%
%:%1228=580%:%
%:%1229=580%:%
%:%1230=581%:%
%:%1231=581%:%
%:%1232=582%:%
%:%1233=582%:%
%:%1234=583%:%
%:%1235=583%:%
%:%1236=584%:%
%:%1237=584%:%
%:%1238=585%:%
%:%1239=585%:%
%:%1240=585%:%
%:%1241=586%:%
%:%1242=586%:%
%:%1243=587%:%
%:%1244=587%:%
%:%1245=588%:%
%:%1246=588%:%
%:%1247=589%:%
%:%1248=589%:%
%:%1249=590%:%
%:%1250=590%:%
%:%1251=591%:%
%:%1252=591%:%
%:%1253=591%:%
%:%1254=591%:%
%:%1255=592%:%
%:%1261=592%:%
%:%1264=593%:%
%:%1265=594%:%
%:%1266=594%:%
%:%1267=595%:%
%:%1270=596%:%
%:%1271=597%:%
%:%1275=597%:%
%:%1276=597%:%
%:%1277=598%:%
%:%1278=598%:%
%:%1279=599%:%
%:%1280=599%:%
%:%1281=600%:%
%:%1282=600%:%
%:%1283=601%:%
%:%1284=601%:%
%:%1285=602%:%
%:%1286=602%:%
%:%1287=603%:%
%:%1288=603%:%
%:%1289=604%:%
%:%1290=604%:%
%:%1291=605%:%
%:%1292=605%:%
%:%1293=606%:%
%:%1294=606%:%
%:%1295=607%:%
%:%1296=607%:%
%:%1297=608%:%
%:%1298=608%:%
%:%1299=609%:%
%:%1300=609%:%
%:%1301=610%:%
%:%1302=610%:%
%:%1303=611%:%
%:%1304=611%:%
%:%1305=612%:%
%:%1306=612%:%
%:%1307=613%:%
%:%1308=613%:%
%:%1309=614%:%
%:%1310=614%:%
%:%1311=615%:%
%:%1312=615%:%
%:%1313=616%:%
%:%1314=616%:%
%:%1315=617%:%
%:%1321=617%:%
%:%1324=618%:%
%:%1325=619%:%
%:%1326=619%:%
%:%1333=620%:%

%
\begin{isabellebody}%
\setisabellecontext{Automorphism{\isacharunderscore}{\kern0pt}Theorems}%
%
\isadelimtheory
%
\endisadelimtheory
%
\isatagtheory
\isacommand{theory}\isamarkupfalse%
\ Automorphism{\isacharunderscore}{\kern0pt}Theorems\isanewline
\ \ \isakeyword{imports}\ \isanewline
\ \ \ \ {\isachardoublequoteopen}Forcing{\isacharslash}{\kern0pt}Forcing{\isacharunderscore}{\kern0pt}Main{\isachardoublequoteclose}\ \isanewline
\ \ \ \ Automorphism{\isacharunderscore}{\kern0pt}M\isanewline
\isakeyword{begin}%
\endisatagtheory
{\isafoldtheory}%
%
\isadelimtheory
\ \isanewline
%
\endisadelimtheory
\isanewline
\isacommand{context}\isamarkupfalse%
\ forcing{\isacharunderscore}{\kern0pt}data{\isacharunderscore}{\kern0pt}partial\ \isanewline
\isakeyword{begin}\ \isanewline
\isanewline
\isacommand{lemma}\isamarkupfalse%
\ Pn{\isacharunderscore}{\kern0pt}auto{\isacharunderscore}{\kern0pt}value{\isacharunderscore}{\kern0pt}in{\isacharunderscore}{\kern0pt}P{\isacharunderscore}{\kern0pt}set\ {\isacharcolon}{\kern0pt}\ \isanewline
\ \ {\isachardoublequoteopen}is{\isacharunderscore}{\kern0pt}P{\isacharunderscore}{\kern0pt}auto{\isacharparenleft}{\kern0pt}{\isasympi}{\isacharparenright}{\kern0pt}\ {\isasymLongrightarrow}\ Ord{\isacharparenleft}{\kern0pt}a{\isacharparenright}{\kern0pt}\ {\isasymLongrightarrow}\ x\ {\isasymin}\ P{\isacharunderscore}{\kern0pt}names\ {\isasymLongrightarrow}\ {\isacharparenleft}{\kern0pt}x\ {\isasymin}\ P{\isacharunderscore}{\kern0pt}set{\isacharparenleft}{\kern0pt}a{\isacharparenright}{\kern0pt}\ {\isasymlongleftrightarrow}\ Pn{\isacharunderscore}{\kern0pt}auto{\isacharparenleft}{\kern0pt}{\isasympi}{\isacharparenright}{\kern0pt}{\isacharbackquote}{\kern0pt}x\ {\isasymin}\ P{\isacharunderscore}{\kern0pt}set{\isacharparenleft}{\kern0pt}a{\isacharparenright}{\kern0pt}{\isacharparenright}{\kern0pt}{\isachardoublequoteclose}\ \isanewline
%
\isadelimproof
%
\endisadelimproof
%
\isatagproof
\isacommand{proof}\isamarkupfalse%
\ {\isacharminus}{\kern0pt}\ \isanewline
\ \ \isacommand{assume}\isamarkupfalse%
\ assms\ {\isacharcolon}{\kern0pt}\ {\isachardoublequoteopen}is{\isacharunderscore}{\kern0pt}P{\isacharunderscore}{\kern0pt}auto{\isacharparenleft}{\kern0pt}{\isasympi}{\isacharparenright}{\kern0pt}{\isachardoublequoteclose}\ {\isachardoublequoteopen}x\ {\isasymin}\ P{\isacharunderscore}{\kern0pt}names{\isachardoublequoteclose}\ {\isachardoublequoteopen}Ord{\isacharparenleft}{\kern0pt}a{\isacharparenright}{\kern0pt}{\isachardoublequoteclose}\isanewline
\ \ \isacommand{define}\isamarkupfalse%
\ Q\ \isakeyword{where}\ {\isachardoublequoteopen}Q\ {\isasymequiv}\ {\isasymlambda}x\ a{\isachardot}{\kern0pt}\ x\ {\isasymin}\ P{\isacharunderscore}{\kern0pt}names\ {\isasymlongrightarrow}\ {\isacharparenleft}{\kern0pt}x\ {\isasymin}\ P{\isacharunderscore}{\kern0pt}set{\isacharparenleft}{\kern0pt}a{\isacharparenright}{\kern0pt}\ {\isasymlongleftrightarrow}\ Pn{\isacharunderscore}{\kern0pt}auto{\isacharparenleft}{\kern0pt}{\isasympi}{\isacharparenright}{\kern0pt}{\isacharbackquote}{\kern0pt}x\ {\isasymin}\ P{\isacharunderscore}{\kern0pt}set{\isacharparenleft}{\kern0pt}a{\isacharparenright}{\kern0pt}{\isacharparenright}{\kern0pt}{\isachardoublequoteclose}\ \isanewline
\ \ \isacommand{have}\isamarkupfalse%
\ main\ {\isacharcolon}{\kern0pt}\ \isanewline
\ \ \ \ {\isachardoublequoteopen}{\isasymAnd}x{\isachardot}{\kern0pt}\ {\isacharparenleft}{\kern0pt}{\isasymforall}a{\isachardot}{\kern0pt}\ Ord{\isacharparenleft}{\kern0pt}a{\isacharparenright}{\kern0pt}\ {\isasymlongrightarrow}\ Q{\isacharparenleft}{\kern0pt}x{\isacharcomma}{\kern0pt}\ a{\isacharparenright}{\kern0pt}{\isacharparenright}{\kern0pt}{\isachardoublequoteclose}\ \isanewline
\ \ \ \ \isacommand{apply}\isamarkupfalse%
\ {\isacharparenleft}{\kern0pt}rule{\isacharunderscore}{\kern0pt}tac\ Q{\isacharequal}{\kern0pt}{\isachardoublequoteopen}{\isasymlambda}x{\isachardot}{\kern0pt}\ {\isacharparenleft}{\kern0pt}{\isasymforall}a{\isachardot}{\kern0pt}\ Ord{\isacharparenleft}{\kern0pt}a{\isacharparenright}{\kern0pt}\ {\isasymlongrightarrow}\ Q{\isacharparenleft}{\kern0pt}x{\isacharcomma}{\kern0pt}\ a{\isacharparenright}{\kern0pt}{\isacharparenright}{\kern0pt}{\isachardoublequoteclose}\ \isakeyword{in}\ ed{\isacharunderscore}{\kern0pt}induction{\isacharparenright}{\kern0pt}\isanewline
\ \ \ \ \isacommand{apply}\isamarkupfalse%
\ {\isacharparenleft}{\kern0pt}clarify{\isacharparenright}{\kern0pt}\ \isanewline
\ \ \ \ \isacommand{apply}\isamarkupfalse%
\ {\isacharparenleft}{\kern0pt}rule{\isacharunderscore}{\kern0pt}tac\ trans{\isacharunderscore}{\kern0pt}induct{\isadigit{3}}{\isacharunderscore}{\kern0pt}raw{\isacharparenright}{\kern0pt}\isanewline
\ \ \ \ \isacommand{unfolding}\isamarkupfalse%
\ Q{\isacharunderscore}{\kern0pt}def\isanewline
\ \ \ \ \isacommand{apply}\isamarkupfalse%
\ simp\ \isacommand{apply}\isamarkupfalse%
\ {\isacharparenleft}{\kern0pt}simp\ add\ {\isacharcolon}{\kern0pt}\ Q{\isacharunderscore}{\kern0pt}def\ P{\isacharunderscore}{\kern0pt}set{\isacharunderscore}{\kern0pt}{\isadigit{0}}{\isacharparenright}{\kern0pt}\ \isanewline
\ \ \isacommand{proof}\isamarkupfalse%
\ {\isacharparenleft}{\kern0pt}clarify{\isacharparenright}{\kern0pt}\isanewline
\ \ \ \ \isacommand{fix}\isamarkupfalse%
\ b\ x\ \isanewline
\ \ \ \ \isacommand{assume}\isamarkupfalse%
\ bord\ {\isacharcolon}{\kern0pt}\ {\isachardoublequoteopen}Ord{\isacharparenleft}{\kern0pt}b{\isacharparenright}{\kern0pt}{\isachardoublequoteclose}\ \isanewline
\ \ \ \ \isakeyword{and}\ ih\ {\isacharcolon}{\kern0pt}\ {\isachardoublequoteopen}{\isasymAnd}y{\isachardot}{\kern0pt}\ ed{\isacharparenleft}{\kern0pt}y{\isacharcomma}{\kern0pt}\ x{\isacharparenright}{\kern0pt}\ {\isasymLongrightarrow}\ {\isasymforall}a{\isachardot}{\kern0pt}\ Ord{\isacharparenleft}{\kern0pt}a{\isacharparenright}{\kern0pt}\ {\isasymlongrightarrow}\ {\isacharparenleft}{\kern0pt}y\ {\isasymin}\ P{\isacharunderscore}{\kern0pt}names\ {\isasymlongrightarrow}\ y\ {\isasymin}\ P{\isacharunderscore}{\kern0pt}set{\isacharparenleft}{\kern0pt}a{\isacharparenright}{\kern0pt}\ {\isasymlongleftrightarrow}\ Pn{\isacharunderscore}{\kern0pt}auto{\isacharparenleft}{\kern0pt}{\isasympi}{\isacharparenright}{\kern0pt}\ {\isacharbackquote}{\kern0pt}\ y\ {\isasymin}\ P{\isacharunderscore}{\kern0pt}set{\isacharparenleft}{\kern0pt}a{\isacharparenright}{\kern0pt}{\isacharparenright}{\kern0pt}{\isachardoublequoteclose}\ \isanewline
\ \ \ \ \isakeyword{and}\ xpname\ {\isacharcolon}{\kern0pt}\ {\isachardoublequoteopen}x\ {\isasymin}\ P{\isacharunderscore}{\kern0pt}names{\isachardoublequoteclose}\ \isanewline
\isanewline
\ \ \ \ \isacommand{have}\isamarkupfalse%
\ xinM\ {\isacharcolon}{\kern0pt}\ {\isachardoublequoteopen}x\ {\isasymin}\ M{\isachardoublequoteclose}\ \isacommand{using}\isamarkupfalse%
\ P{\isacharunderscore}{\kern0pt}name{\isacharunderscore}{\kern0pt}in{\isacharunderscore}{\kern0pt}M\ xpname\ \isacommand{by}\isamarkupfalse%
\ auto\ \isanewline
\isanewline
\ \ \ \ \isacommand{have}\isamarkupfalse%
\ pinP\ {\isacharcolon}{\kern0pt}\ {\isachardoublequoteopen}{\isasymforall}{\isacharless}{\kern0pt}y{\isacharcomma}{\kern0pt}\ p{\isachargreater}{\kern0pt}\ {\isasymin}\ x{\isachardot}{\kern0pt}\ p\ {\isasymin}\ P{\isachardoublequoteclose}\ \isanewline
\ \ \ \ \ \ \isacommand{apply}\isamarkupfalse%
\ {\isacharparenleft}{\kern0pt}rule{\isacharunderscore}{\kern0pt}tac\ pair{\isacharunderscore}{\kern0pt}forallI{\isacharparenright}{\kern0pt}\ \isanewline
\ \ \ \ \ \ \isacommand{using}\isamarkupfalse%
\ P{\isacharunderscore}{\kern0pt}names{\isacharunderscore}{\kern0pt}in\ xpname\ relation{\isacharunderscore}{\kern0pt}P{\isacharunderscore}{\kern0pt}name\ \isacommand{by}\isamarkupfalse%
\ auto\ \isanewline
\isanewline
\ \ \ \ \isacommand{have}\isamarkupfalse%
\ ih\ {\isacharcolon}{\kern0pt}\ {\isachardoublequoteopen}{\isasymAnd}y\ a{\isachardot}{\kern0pt}\ ed{\isacharparenleft}{\kern0pt}y{\isacharcomma}{\kern0pt}\ x{\isacharparenright}{\kern0pt}\ {\isasymLongrightarrow}\ Ord{\isacharparenleft}{\kern0pt}a{\isacharparenright}{\kern0pt}\ {\isasymLongrightarrow}\ y\ {\isasymin}\ P{\isacharunderscore}{\kern0pt}names\ {\isasymLongrightarrow}\ {\isacharparenleft}{\kern0pt}y\ {\isasymin}\ P{\isacharunderscore}{\kern0pt}set{\isacharparenleft}{\kern0pt}a{\isacharparenright}{\kern0pt}\ {\isasymlongleftrightarrow}\ Pn{\isacharunderscore}{\kern0pt}auto{\isacharparenleft}{\kern0pt}{\isasympi}{\isacharparenright}{\kern0pt}\ {\isacharbackquote}{\kern0pt}\ y\ {\isasymin}\ P{\isacharunderscore}{\kern0pt}set{\isacharparenleft}{\kern0pt}a{\isacharparenright}{\kern0pt}{\isacharparenright}{\kern0pt}{\isachardoublequoteclose}\isanewline
\ \ \ \ \ \ \isacommand{using}\isamarkupfalse%
\ ih\ \isacommand{by}\isamarkupfalse%
\ auto\isanewline
\isanewline
\ \ \ \ \isacommand{have}\isamarkupfalse%
\ {\isachardoublequoteopen}x\ {\isasymin}\ P{\isacharunderscore}{\kern0pt}set{\isacharparenleft}{\kern0pt}succ{\isacharparenleft}{\kern0pt}b{\isacharparenright}{\kern0pt}{\isacharparenright}{\kern0pt}\ {\isasymlongleftrightarrow}\ x\ {\isasymin}\ Pow{\isacharparenleft}{\kern0pt}P{\isacharunderscore}{\kern0pt}set{\isacharparenleft}{\kern0pt}b{\isacharparenright}{\kern0pt}\ {\isasymtimes}\ P{\isacharparenright}{\kern0pt}\ {\isasyminter}\ M{\isachardoublequoteclose}\ \isacommand{using}\isamarkupfalse%
\ bord\ P{\isacharunderscore}{\kern0pt}set{\isacharunderscore}{\kern0pt}succ\ \isacommand{by}\isamarkupfalse%
\ auto\ \isanewline
\ \ \ \ \isacommand{also}\isamarkupfalse%
\ \isacommand{have}\isamarkupfalse%
\ {\isachardoublequoteopen}{\isachardot}{\kern0pt}{\isachardot}{\kern0pt}{\isachardot}{\kern0pt}\ {\isasymlongleftrightarrow}\ x\ {\isasymsubseteq}\ P{\isacharunderscore}{\kern0pt}set{\isacharparenleft}{\kern0pt}b{\isacharparenright}{\kern0pt}\ {\isasymtimes}\ P{\isachardoublequoteclose}\ \isacommand{using}\isamarkupfalse%
\ xinM\ \isacommand{by}\isamarkupfalse%
\ auto\ \isanewline
\ \ \ \ \isacommand{also}\isamarkupfalse%
\ \isacommand{have}\isamarkupfalse%
\ {\isachardoublequoteopen}{\isachardot}{\kern0pt}{\isachardot}{\kern0pt}{\isachardot}{\kern0pt}\ {\isasymlongleftrightarrow}\ {\isacharparenleft}{\kern0pt}{\isasymforall}{\isacharless}{\kern0pt}y{\isacharcomma}{\kern0pt}\ p{\isachargreater}{\kern0pt}\ {\isasymin}\ x{\isachardot}{\kern0pt}\ y\ {\isasymin}\ P{\isacharunderscore}{\kern0pt}set{\isacharparenleft}{\kern0pt}b{\isacharparenright}{\kern0pt}{\isacharparenright}{\kern0pt}{\isachardoublequoteclose}\ \isanewline
\ \ \ \ \ \ \isacommand{apply}\isamarkupfalse%
\ {\isacharparenleft}{\kern0pt}rule\ iffI{\isacharparenright}{\kern0pt}\ \isanewline
\ \ \ \ \ \ \isacommand{apply}\isamarkupfalse%
\ {\isacharparenleft}{\kern0pt}rule{\isacharunderscore}{\kern0pt}tac\ pair{\isacharunderscore}{\kern0pt}forallI{\isacharparenright}{\kern0pt}\ \isacommand{using}\isamarkupfalse%
\ relation{\isacharunderscore}{\kern0pt}P{\isacharunderscore}{\kern0pt}name\ xpname\ \isacommand{apply}\isamarkupfalse%
\ simp\ \isacommand{apply}\isamarkupfalse%
\ blast\ \isanewline
\ \ \ \ \ \ \isacommand{apply}\isamarkupfalse%
\ {\isacharparenleft}{\kern0pt}rule{\isacharunderscore}{\kern0pt}tac\ x{\isacharequal}{\kern0pt}x\ \isakeyword{and}\ A{\isacharequal}{\kern0pt}{\isachardoublequoteopen}{\isasymlambda}y\ p{\isachardot}{\kern0pt}\ y\ {\isasymin}\ P{\isacharunderscore}{\kern0pt}set{\isacharparenleft}{\kern0pt}b{\isacharparenright}{\kern0pt}{\isachardoublequoteclose}\ \isakeyword{in}\ pair{\isacharunderscore}{\kern0pt}forallE{\isacharparenright}{\kern0pt}\ \ \isanewline
\ \ \ \ \ \ \isacommand{using}\isamarkupfalse%
\ relation{\isacharunderscore}{\kern0pt}P{\isacharunderscore}{\kern0pt}name\ xpname\ \isacommand{apply}\isamarkupfalse%
\ simp\ \isacommand{apply}\isamarkupfalse%
\ simp\isanewline
\ \ \ \ \ \ \isacommand{apply}\isamarkupfalse%
\ {\isacharparenleft}{\kern0pt}rule\ subsetI{\isacharparenright}{\kern0pt}\isanewline
\ \ \ \ \isacommand{proof}\isamarkupfalse%
\ {\isacharminus}{\kern0pt}\ \isanewline
\ \ \ \ \ \ \isacommand{fix}\isamarkupfalse%
\ v\ \isacommand{assume}\isamarkupfalse%
\ assms\ {\isacharcolon}{\kern0pt}\ {\isachardoublequoteopen}v\ {\isasymin}\ x{\isachardoublequoteclose}\ {\isachardoublequoteopen}{\isacharparenleft}{\kern0pt}{\isasymAnd}y\ p{\isachardot}{\kern0pt}\ {\isasymlangle}y{\isacharcomma}{\kern0pt}\ p{\isasymrangle}\ {\isasymin}\ x\ {\isasymLongrightarrow}\ y\ {\isasymin}\ P{\isacharunderscore}{\kern0pt}set{\isacharparenleft}{\kern0pt}b{\isacharparenright}{\kern0pt}{\isacharparenright}{\kern0pt}{\isachardoublequoteclose}\ \isanewline
\ \ \ \ \ \ \isacommand{have}\isamarkupfalse%
\ xrel\ {\isacharcolon}{\kern0pt}\ {\isachardoublequoteopen}relation{\isacharparenleft}{\kern0pt}x{\isacharparenright}{\kern0pt}{\isachardoublequoteclose}\ \isacommand{using}\isamarkupfalse%
\ relation{\isacharunderscore}{\kern0pt}P{\isacharunderscore}{\kern0pt}name\ xpname\ \isacommand{by}\isamarkupfalse%
\ auto\ \isanewline
\ \ \ \ \ \ \isacommand{then}\isamarkupfalse%
\ \isacommand{obtain}\isamarkupfalse%
\ y\ p\ \isakeyword{where}\ yph{\isacharcolon}{\kern0pt}\ {\isachardoublequoteopen}v\ {\isacharequal}{\kern0pt}\ {\isacharless}{\kern0pt}y{\isacharcomma}{\kern0pt}\ p{\isachargreater}{\kern0pt}{\isachardoublequoteclose}\ \isacommand{unfolding}\isamarkupfalse%
\ relation{\isacharunderscore}{\kern0pt}def\ \isacommand{using}\isamarkupfalse%
\ assms\ \isacommand{by}\isamarkupfalse%
\ auto\ \isanewline
\ \ \ \ \ \ \isacommand{then}\isamarkupfalse%
\ \isacommand{have}\isamarkupfalse%
\ H\ {\isacharcolon}{\kern0pt}\ {\isachardoublequoteopen}\ y\ {\isasymin}\ P{\isacharunderscore}{\kern0pt}set{\isacharparenleft}{\kern0pt}b{\isacharparenright}{\kern0pt}{\isachardoublequoteclose}\ \isacommand{using}\isamarkupfalse%
\ assms\ \isacommand{by}\isamarkupfalse%
\ auto\ \isanewline
\ \ \ \ \ \ \isacommand{have}\isamarkupfalse%
\ {\isachardoublequoteopen}p\ {\isasymin}\ P{\isachardoublequoteclose}\ \isacommand{using}\isamarkupfalse%
\ assms\ yph\ pinP\ \isacommand{by}\isamarkupfalse%
\ auto\ \isanewline
\ \ \ \ \ \ \isacommand{then}\isamarkupfalse%
\ \isacommand{show}\isamarkupfalse%
\ {\isachardoublequoteopen}v\ {\isasymin}\ P{\isacharunderscore}{\kern0pt}set{\isacharparenleft}{\kern0pt}b{\isacharparenright}{\kern0pt}\ {\isasymtimes}\ P{\isachardoublequoteclose}\ \isacommand{using}\isamarkupfalse%
\ yph\ H\ \isacommand{by}\isamarkupfalse%
\ auto\ \isanewline
\ \ \ \ \isacommand{qed}\isamarkupfalse%
\isanewline
\ \ \ \ \isacommand{also}\isamarkupfalse%
\ \isacommand{have}\isamarkupfalse%
\ {\isachardoublequoteopen}{\isachardot}{\kern0pt}{\isachardot}{\kern0pt}{\isachardot}{\kern0pt}\ {\isasymlongleftrightarrow}\ {\isacharparenleft}{\kern0pt}{\isasymforall}{\isacharless}{\kern0pt}y{\isacharcomma}{\kern0pt}\ p{\isachargreater}{\kern0pt}\ {\isasymin}\ x{\isachardot}{\kern0pt}\ Pn{\isacharunderscore}{\kern0pt}auto{\isacharparenleft}{\kern0pt}{\isasympi}{\isacharparenright}{\kern0pt}{\isacharbackquote}{\kern0pt}y\ {\isasymin}\ P{\isacharunderscore}{\kern0pt}set{\isacharparenleft}{\kern0pt}b{\isacharparenright}{\kern0pt}{\isacharparenright}{\kern0pt}{\isachardoublequoteclose}\ \isanewline
\ \ \ \ \ \ \isacommand{apply}\isamarkupfalse%
\ {\isacharparenleft}{\kern0pt}rule\ iffI{\isacharparenright}{\kern0pt}\ \isanewline
\ \ \ \ \ \ \isacommand{apply}\isamarkupfalse%
\ {\isacharparenleft}{\kern0pt}rule{\isacharunderscore}{\kern0pt}tac\ pair{\isacharunderscore}{\kern0pt}forallI{\isacharparenright}{\kern0pt}\isanewline
\ \ \ \ \ \ \isacommand{using}\isamarkupfalse%
\ relation{\isacharunderscore}{\kern0pt}P{\isacharunderscore}{\kern0pt}name\ xpname\ \isacommand{apply}\isamarkupfalse%
\ simp\ \isanewline
\ \ \ \ \ \ \isacommand{apply}\isamarkupfalse%
\ {\isacharparenleft}{\kern0pt}rule{\isacharunderscore}{\kern0pt}tac\ x{\isacharequal}{\kern0pt}x\ \isakeyword{and}\ A{\isacharequal}{\kern0pt}{\isachardoublequoteopen}{\isasymlambda}y\ p{\isachardot}{\kern0pt}\ y\ {\isasymin}\ P{\isacharunderscore}{\kern0pt}set{\isacharparenleft}{\kern0pt}b{\isacharparenright}{\kern0pt}{\isachardoublequoteclose}\ \isakeyword{in}\ pair{\isacharunderscore}{\kern0pt}forallE{\isacharparenright}{\kern0pt}\isanewline
\ \ \ \ \ \ \isacommand{using}\isamarkupfalse%
\ relation{\isacharunderscore}{\kern0pt}P{\isacharunderscore}{\kern0pt}name\ xpname\ \isacommand{apply}\isamarkupfalse%
\ simp\ \isacommand{apply}\isamarkupfalse%
\ simp\ \isanewline
\ \ \ \ \ \ \isacommand{apply}\isamarkupfalse%
\ {\isacharparenleft}{\kern0pt}rename{\isacharunderscore}{\kern0pt}tac\ y\ p{\isacharparenright}{\kern0pt}\isanewline
\ \ \ \ \ \ \isacommand{apply}\isamarkupfalse%
\ {\isacharparenleft}{\kern0pt}rule{\isacharunderscore}{\kern0pt}tac\ P{\isacharequal}{\kern0pt}{\isachardoublequoteopen}y\ {\isasymin}\ P{\isacharunderscore}{\kern0pt}set{\isacharparenleft}{\kern0pt}b{\isacharparenright}{\kern0pt}{\isachardoublequoteclose}\ \isakeyword{in}\ iffD{\isadigit{1}}{\isacharparenright}{\kern0pt}\isanewline
\ \ \ \ \ \ \isacommand{apply}\isamarkupfalse%
\ {\isacharparenleft}{\kern0pt}rule{\isacharunderscore}{\kern0pt}tac\ ih{\isacharparenright}{\kern0pt}\ \isacommand{unfolding}\isamarkupfalse%
\ ed{\isacharunderscore}{\kern0pt}def\ domain{\isacharunderscore}{\kern0pt}def\ \isacommand{apply}\isamarkupfalse%
\ blast\ \isanewline
\ \ \ \ \ \ \isacommand{using}\isamarkupfalse%
\ bord\ \isacommand{apply}\isamarkupfalse%
\ simp\ \isacommand{using}\isamarkupfalse%
\ P{\isacharunderscore}{\kern0pt}name{\isacharunderscore}{\kern0pt}domain{\isacharunderscore}{\kern0pt}P{\isacharunderscore}{\kern0pt}name\ xpname\ \isacommand{apply}\isamarkupfalse%
\ simp\ \isanewline
\ \ \ \ \ \ \isacommand{apply}\isamarkupfalse%
\ simp\ \isanewline
\ \ \ \ \ \ \isacommand{apply}\isamarkupfalse%
\ {\isacharparenleft}{\kern0pt}rule{\isacharunderscore}{\kern0pt}tac\ pair{\isacharunderscore}{\kern0pt}forallI{\isacharparenright}{\kern0pt}\isanewline
\ \ \ \ \ \ \isacommand{using}\isamarkupfalse%
\ relation{\isacharunderscore}{\kern0pt}P{\isacharunderscore}{\kern0pt}name\ xpname\ \isacommand{apply}\isamarkupfalse%
\ simp\ \isanewline
\ \ \ \ \ \ \isacommand{apply}\isamarkupfalse%
\ {\isacharparenleft}{\kern0pt}rule{\isacharunderscore}{\kern0pt}tac\ x{\isacharequal}{\kern0pt}x\ \isakeyword{and}\ A{\isacharequal}{\kern0pt}{\isachardoublequoteopen}{\isasymlambda}y\ p{\isachardot}{\kern0pt}\ Pn{\isacharunderscore}{\kern0pt}auto{\isacharparenleft}{\kern0pt}{\isasympi}{\isacharparenright}{\kern0pt}{\isacharbackquote}{\kern0pt}y\ {\isasymin}\ P{\isacharunderscore}{\kern0pt}set{\isacharparenleft}{\kern0pt}b{\isacharparenright}{\kern0pt}{\isachardoublequoteclose}\ \isakeyword{in}\ pair{\isacharunderscore}{\kern0pt}forallE{\isacharparenright}{\kern0pt}\ \isanewline
\ \ \ \ \ \ \isacommand{using}\isamarkupfalse%
\ relation{\isacharunderscore}{\kern0pt}P{\isacharunderscore}{\kern0pt}name\ xpname\ \isacommand{apply}\isamarkupfalse%
\ simp\ \isanewline
\ \ \ \ \ \ \isacommand{apply}\isamarkupfalse%
\ simp\ \isanewline
\ \ \ \ \ \ \isacommand{apply}\isamarkupfalse%
\ {\isacharparenleft}{\kern0pt}rule{\isacharunderscore}{\kern0pt}tac\ Q{\isacharequal}{\kern0pt}{\isachardoublequoteopen}Pn{\isacharunderscore}{\kern0pt}auto{\isacharparenleft}{\kern0pt}{\isasympi}{\isacharparenright}{\kern0pt}{\isacharbackquote}{\kern0pt}y\ {\isasymin}\ P{\isacharunderscore}{\kern0pt}set{\isacharparenleft}{\kern0pt}b{\isacharparenright}{\kern0pt}{\isachardoublequoteclose}\ \isakeyword{in}\ iffD{\isadigit{2}}{\isacharparenright}{\kern0pt}\isanewline
\ \ \ \ \ \ \isacommand{apply}\isamarkupfalse%
\ {\isacharparenleft}{\kern0pt}rule{\isacharunderscore}{\kern0pt}tac\ ih{\isacharparenright}{\kern0pt}\ \isacommand{unfolding}\isamarkupfalse%
\ ed{\isacharunderscore}{\kern0pt}def\ domain{\isacharunderscore}{\kern0pt}def\ \isacommand{apply}\isamarkupfalse%
\ blast\ \isanewline
\ \ \ \ \ \ \isacommand{using}\isamarkupfalse%
\ bord\ \isacommand{apply}\isamarkupfalse%
\ simp\ \isacommand{using}\isamarkupfalse%
\ P{\isacharunderscore}{\kern0pt}name{\isacharunderscore}{\kern0pt}domain{\isacharunderscore}{\kern0pt}P{\isacharunderscore}{\kern0pt}name\ xpname\ \isacommand{apply}\isamarkupfalse%
\ simp\ \isanewline
\ \ \ \ \ \ \isacommand{apply}\isamarkupfalse%
\ simp\isanewline
\ \ \ \ \ \ \isacommand{done}\isamarkupfalse%
\ \isanewline
\ \ \ \ \isacommand{also}\isamarkupfalse%
\ \isacommand{have}\isamarkupfalse%
\ {\isachardoublequoteopen}{\isachardot}{\kern0pt}{\isachardot}{\kern0pt}{\isachardot}{\kern0pt}\ {\isasymlongleftrightarrow}\ Pn{\isacharunderscore}{\kern0pt}auto{\isacharparenleft}{\kern0pt}{\isasympi}{\isacharparenright}{\kern0pt}{\isacharbackquote}{\kern0pt}x\ {\isasymin}\ P{\isacharunderscore}{\kern0pt}set{\isacharparenleft}{\kern0pt}succ{\isacharparenleft}{\kern0pt}b{\isacharparenright}{\kern0pt}{\isacharparenright}{\kern0pt}{\isachardoublequoteclose}\ \isanewline
\ \ \ \ \ \ \isacommand{apply}\isamarkupfalse%
\ {\isacharparenleft}{\kern0pt}rule\ iffI{\isacharparenright}{\kern0pt}\ \isanewline
\ \ \ \ \ \ \isacommand{apply}\isamarkupfalse%
\ {\isacharparenleft}{\kern0pt}rule{\isacharunderscore}{\kern0pt}tac\ x{\isacharequal}{\kern0pt}x\ \isakeyword{and}\ A{\isacharequal}{\kern0pt}{\isachardoublequoteopen}{\isasymlambda}y\ p{\isachardot}{\kern0pt}\ Pn{\isacharunderscore}{\kern0pt}auto{\isacharparenleft}{\kern0pt}{\isasympi}{\isacharparenright}{\kern0pt}\ {\isacharbackquote}{\kern0pt}\ y\ {\isasymin}\ P{\isacharunderscore}{\kern0pt}set{\isacharparenleft}{\kern0pt}b{\isacharparenright}{\kern0pt}{\isachardoublequoteclose}\ \isakeyword{in}\ pair{\isacharunderscore}{\kern0pt}forallE{\isacharparenright}{\kern0pt}\ \isanewline
\ \ \ \ \ \ \isacommand{using}\isamarkupfalse%
\ relation{\isacharunderscore}{\kern0pt}P{\isacharunderscore}{\kern0pt}name\ xpname\ \isacommand{apply}\isamarkupfalse%
\ simp\ \isacommand{apply}\isamarkupfalse%
\ simp\ \isanewline
\ \ \ \ \isacommand{proof}\isamarkupfalse%
\ {\isacharminus}{\kern0pt}\ \isanewline
\ \ \ \ \ \ \isacommand{assume}\isamarkupfalse%
\ assm\ {\isacharcolon}{\kern0pt}\ {\isachardoublequoteopen}{\isacharparenleft}{\kern0pt}{\isasymAnd}y\ p{\isachardot}{\kern0pt}\ {\isasymlangle}y{\isacharcomma}{\kern0pt}\ p{\isasymrangle}\ {\isasymin}\ x\ {\isasymLongrightarrow}\ Pn{\isacharunderscore}{\kern0pt}auto{\isacharparenleft}{\kern0pt}{\isasympi}{\isacharparenright}{\kern0pt}\ {\isacharbackquote}{\kern0pt}\ y\ {\isasymin}\ P{\isacharunderscore}{\kern0pt}set{\isacharparenleft}{\kern0pt}b{\isacharparenright}{\kern0pt}{\isacharparenright}{\kern0pt}{\isachardoublequoteclose}\ \isanewline
\ \ \ \ \ \ \isacommand{have}\isamarkupfalse%
\ {\isachardoublequoteopen}{\isasymAnd}y\ p{\isachardot}{\kern0pt}\ {\isasymlangle}y{\isacharcomma}{\kern0pt}\ p{\isasymrangle}\ {\isasymin}\ x\ {\isasymLongrightarrow}\ {\isasympi}{\isacharbackquote}{\kern0pt}p\ {\isasymin}\ P{\isachardoublequoteclose}\ \isanewline
\ \ \ \ \ \ \ \ \isacommand{using}\isamarkupfalse%
\ pinP\ P{\isacharunderscore}{\kern0pt}auto{\isacharunderscore}{\kern0pt}value\ assms\ \isacommand{by}\isamarkupfalse%
\ auto\ \isanewline
\ \ \ \ \ \ \isacommand{then}\isamarkupfalse%
\ \isacommand{have}\isamarkupfalse%
\ H{\isacharcolon}{\kern0pt}\ {\isachardoublequoteopen}{\isasymAnd}y\ p{\isachardot}{\kern0pt}\ {\isasymlangle}y{\isacharcomma}{\kern0pt}\ p{\isasymrangle}\ {\isasymin}\ x\ {\isasymLongrightarrow}\ {\isacharless}{\kern0pt}Pn{\isacharunderscore}{\kern0pt}auto{\isacharparenleft}{\kern0pt}{\isasympi}{\isacharparenright}{\kern0pt}{\isacharbackquote}{\kern0pt}y{\isacharcomma}{\kern0pt}\ {\isasympi}{\isacharbackquote}{\kern0pt}p{\isachargreater}{\kern0pt}\ {\isasymin}\ P{\isacharunderscore}{\kern0pt}set{\isacharparenleft}{\kern0pt}b{\isacharparenright}{\kern0pt}\ {\isasymtimes}\ P{\isachardoublequoteclose}\ \isacommand{using}\isamarkupfalse%
\ assm\ \isacommand{by}\isamarkupfalse%
\ auto\isanewline
\ \ \ \ \ \ \isacommand{then}\isamarkupfalse%
\ \isacommand{have}\isamarkupfalse%
\ {\isachardoublequoteopen}{\isacharbraceleft}{\kern0pt}\ {\isacharless}{\kern0pt}Pn{\isacharunderscore}{\kern0pt}auto{\isacharparenleft}{\kern0pt}{\isasympi}{\isacharparenright}{\kern0pt}{\isacharbackquote}{\kern0pt}y{\isacharcomma}{\kern0pt}\ {\isasympi}{\isacharbackquote}{\kern0pt}p{\isachargreater}{\kern0pt}\ {\isachardot}{\kern0pt}\ {\isacharless}{\kern0pt}y{\isacharcomma}{\kern0pt}\ p{\isachargreater}{\kern0pt}\ {\isasymin}\ x\ {\isacharbraceright}{\kern0pt}\ {\isasymsubseteq}\ P{\isacharunderscore}{\kern0pt}set{\isacharparenleft}{\kern0pt}b{\isacharparenright}{\kern0pt}\ {\isasymtimes}\ P{\isachardoublequoteclose}\ \isanewline
\ \ \ \ \ \ \ \ \isacommand{apply}\isamarkupfalse%
\ {\isacharparenleft}{\kern0pt}rule{\isacharunderscore}{\kern0pt}tac\ subsetI{\isacharparenright}{\kern0pt}\ \isanewline
\ \ \ \ \ \ \isacommand{proof}\isamarkupfalse%
\ {\isacharminus}{\kern0pt}\ \isanewline
\ \ \ \ \ \ \ \ \isacommand{fix}\isamarkupfalse%
\ v\isanewline
\ \ \ \ \ \ \ \ \isacommand{assume}\isamarkupfalse%
\ assm{\isadigit{1}}\ {\isacharcolon}{\kern0pt}\ {\isachardoublequoteopen}v\ {\isasymin}\ {\isacharbraceleft}{\kern0pt}{\isasymlangle}Pn{\isacharunderscore}{\kern0pt}auto{\isacharparenleft}{\kern0pt}{\isasympi}{\isacharparenright}{\kern0pt}\ {\isacharbackquote}{\kern0pt}\ y{\isacharcomma}{\kern0pt}\ {\isasympi}\ {\isacharbackquote}{\kern0pt}\ p{\isasymrangle}\ {\isachardot}{\kern0pt}\ {\isasymlangle}y{\isacharcomma}{\kern0pt}p{\isasymrangle}\ {\isasymin}\ x{\isacharbraceright}{\kern0pt}{\isachardoublequoteclose}\ \isanewline
\ \ \ \ \ \ \ \ \isacommand{have}\isamarkupfalse%
\ {\isachardoublequoteopen}relation{\isacharparenleft}{\kern0pt}x{\isacharparenright}{\kern0pt}{\isachardoublequoteclose}\ \isacommand{using}\isamarkupfalse%
\ relation{\isacharunderscore}{\kern0pt}P{\isacharunderscore}{\kern0pt}name\ xpname\ \isacommand{by}\isamarkupfalse%
\ auto\ \isanewline
\ \ \ \ \ \ \ \ \isacommand{then}\isamarkupfalse%
\ \isacommand{have}\isamarkupfalse%
\ {\isachardoublequoteopen}{\isasymexists}y\ p{\isachardot}{\kern0pt}\ {\isasymlangle}y{\isacharcomma}{\kern0pt}\ p{\isasymrangle}\ {\isasymin}\ x\ {\isasymand}\ v\ {\isacharequal}{\kern0pt}\ {\isasymlangle}Pn{\isacharunderscore}{\kern0pt}auto{\isacharparenleft}{\kern0pt}{\isasympi}{\isacharparenright}{\kern0pt}\ {\isacharbackquote}{\kern0pt}\ y{\isacharcomma}{\kern0pt}\ {\isasympi}\ {\isacharbackquote}{\kern0pt}\ p{\isasymrangle}{\isachardoublequoteclose}\ \isanewline
\ \ \ \ \ \ \ \ \ \ \isacommand{using}\isamarkupfalse%
\ pair{\isacharunderscore}{\kern0pt}rel{\isacharunderscore}{\kern0pt}arg\ assm{\isadigit{1}}\ \isacommand{by}\isamarkupfalse%
\ auto\ \isanewline
\ \ \ \ \ \ \ \ \isacommand{then}\isamarkupfalse%
\ \isacommand{obtain}\isamarkupfalse%
\ y\ p\ \isakeyword{where}\ yph\ {\isacharcolon}{\kern0pt}\ {\isachardoublequoteopen}{\isacharless}{\kern0pt}y{\isacharcomma}{\kern0pt}\ p{\isachargreater}{\kern0pt}\ {\isasymin}\ x{\isachardoublequoteclose}\ {\isachardoublequoteopen}v\ {\isacharequal}{\kern0pt}\ {\isasymlangle}Pn{\isacharunderscore}{\kern0pt}auto{\isacharparenleft}{\kern0pt}{\isasympi}{\isacharparenright}{\kern0pt}\ {\isacharbackquote}{\kern0pt}\ y{\isacharcomma}{\kern0pt}\ {\isasympi}\ {\isacharbackquote}{\kern0pt}\ p{\isasymrangle}{\isachardoublequoteclose}\ \isacommand{by}\isamarkupfalse%
\ auto\ \isanewline
\ \ \ \ \ \ \ \ \isacommand{then}\isamarkupfalse%
\ \isacommand{have}\isamarkupfalse%
\ H\ {\isacharcolon}{\kern0pt}\ {\isachardoublequoteopen}Pn{\isacharunderscore}{\kern0pt}auto{\isacharparenleft}{\kern0pt}{\isasympi}{\isacharparenright}{\kern0pt}\ {\isacharbackquote}{\kern0pt}\ y\ {\isasymin}\ P{\isacharunderscore}{\kern0pt}set{\isacharparenleft}{\kern0pt}b{\isacharparenright}{\kern0pt}{\isachardoublequoteclose}\ \isacommand{using}\isamarkupfalse%
\ yph\ assm\ \isacommand{by}\isamarkupfalse%
\ auto\ \isanewline
\ \ \ \ \ \ \ \ \isacommand{have}\isamarkupfalse%
\ {\isachardoublequoteopen}p\ {\isasymin}\ P{\isachardoublequoteclose}\ \isacommand{using}\isamarkupfalse%
\ pinP\ yph\ \isacommand{by}\isamarkupfalse%
\ auto\ \isanewline
\ \ \ \ \ \ \ \ \isacommand{then}\isamarkupfalse%
\ \isacommand{have}\isamarkupfalse%
\ {\isachardoublequoteopen}{\isasympi}{\isacharbackquote}{\kern0pt}p\ {\isasymin}\ P{\isachardoublequoteclose}\ \isacommand{using}\isamarkupfalse%
\ P{\isacharunderscore}{\kern0pt}auto{\isacharunderscore}{\kern0pt}value\ assms\ \isacommand{by}\isamarkupfalse%
\ auto\ \isanewline
\ \ \ \ \ \ \ \ \isacommand{then}\isamarkupfalse%
\ \isacommand{show}\isamarkupfalse%
\ \ {\isachardoublequoteopen}v\ {\isasymin}\ P{\isacharunderscore}{\kern0pt}set{\isacharparenleft}{\kern0pt}b{\isacharparenright}{\kern0pt}\ {\isasymtimes}\ P{\isachardoublequoteclose}\ \isacommand{using}\isamarkupfalse%
\ H\ yph\ \isacommand{by}\isamarkupfalse%
\ auto\ \isanewline
\ \ \ \ \ \ \isacommand{qed}\isamarkupfalse%
\ \ \ \ \ \isanewline
\ \ \ \ \ \ \isacommand{then}\isamarkupfalse%
\ \isacommand{have}\isamarkupfalse%
\ H{\isacharcolon}{\kern0pt}\ {\isachardoublequoteopen}Pn{\isacharunderscore}{\kern0pt}auto{\isacharparenleft}{\kern0pt}{\isasympi}{\isacharparenright}{\kern0pt}\ {\isacharbackquote}{\kern0pt}\ x\ {\isasymin}\ Pow{\isacharparenleft}{\kern0pt}P{\isacharunderscore}{\kern0pt}set{\isacharparenleft}{\kern0pt}b{\isacharparenright}{\kern0pt}\ {\isasymtimes}\ P{\isacharparenright}{\kern0pt}{\isachardoublequoteclose}\ \isacommand{using}\isamarkupfalse%
\ xpname\ Pn{\isacharunderscore}{\kern0pt}auto\ \isacommand{by}\isamarkupfalse%
\ auto\ \isanewline
\ \ \ \ \ \ \isacommand{have}\isamarkupfalse%
\ {\isachardoublequoteopen}Pn{\isacharunderscore}{\kern0pt}auto{\isacharparenleft}{\kern0pt}{\isasympi}{\isacharparenright}{\kern0pt}\ {\isacharbackquote}{\kern0pt}\ x\ \ {\isasymin}\ M{\isachardoublequoteclose}\ \isacommand{using}\isamarkupfalse%
\ Pn{\isacharunderscore}{\kern0pt}auto{\isacharunderscore}{\kern0pt}value{\isacharunderscore}{\kern0pt}in{\isacharunderscore}{\kern0pt}M\ assms\ xpname\ \isacommand{by}\isamarkupfalse%
\ auto\ \isanewline
\ \ \ \ \ \ \isacommand{then}\isamarkupfalse%
\ \isacommand{have}\isamarkupfalse%
\ {\isachardoublequoteopen}Pn{\isacharunderscore}{\kern0pt}auto{\isacharparenleft}{\kern0pt}{\isasympi}{\isacharparenright}{\kern0pt}\ {\isacharbackquote}{\kern0pt}\ x\ {\isasymin}\ Pow{\isacharparenleft}{\kern0pt}P{\isacharunderscore}{\kern0pt}set{\isacharparenleft}{\kern0pt}b{\isacharparenright}{\kern0pt}\ {\isasymtimes}\ P{\isacharparenright}{\kern0pt}\ {\isasyminter}\ M{\isachardoublequoteclose}\ \isacommand{using}\isamarkupfalse%
\ H\ \isacommand{by}\isamarkupfalse%
\ auto\ \isanewline
\ \ \ \ \ \ \isacommand{then}\isamarkupfalse%
\ \isacommand{show}\isamarkupfalse%
\ {\isachardoublequoteopen}Pn{\isacharunderscore}{\kern0pt}auto{\isacharparenleft}{\kern0pt}{\isasympi}{\isacharparenright}{\kern0pt}\ {\isacharbackquote}{\kern0pt}\ x\ {\isasymin}\ P{\isacharunderscore}{\kern0pt}set{\isacharparenleft}{\kern0pt}succ{\isacharparenleft}{\kern0pt}b{\isacharparenright}{\kern0pt}{\isacharparenright}{\kern0pt}{\isachardoublequoteclose}\ \isacommand{using}\isamarkupfalse%
\ P{\isacharunderscore}{\kern0pt}set{\isacharunderscore}{\kern0pt}succ\ \isacommand{by}\isamarkupfalse%
\ auto\ \isanewline
\ \ \ \ \isacommand{next}\isamarkupfalse%
\ \isanewline
\ \ \ \ \ \ \isacommand{assume}\isamarkupfalse%
\ assm\ {\isacharcolon}{\kern0pt}\ {\isachardoublequoteopen}Pn{\isacharunderscore}{\kern0pt}auto{\isacharparenleft}{\kern0pt}{\isasympi}{\isacharparenright}{\kern0pt}\ {\isacharbackquote}{\kern0pt}\ x\ {\isasymin}\ P{\isacharunderscore}{\kern0pt}set{\isacharparenleft}{\kern0pt}succ{\isacharparenleft}{\kern0pt}b{\isacharparenright}{\kern0pt}{\isacharparenright}{\kern0pt}{\isachardoublequoteclose}\ \isanewline
\ \ \ \ \ \ \isacommand{show}\isamarkupfalse%
\ {\isachardoublequoteopen}{\isasymforall}{\isasymlangle}y{\isacharcomma}{\kern0pt}p{\isasymrangle}{\isasymin}x{\isachardot}{\kern0pt}\ Pn{\isacharunderscore}{\kern0pt}auto{\isacharparenleft}{\kern0pt}{\isasympi}{\isacharparenright}{\kern0pt}\ {\isacharbackquote}{\kern0pt}\ y\ {\isasymin}\ P{\isacharunderscore}{\kern0pt}set{\isacharparenleft}{\kern0pt}b{\isacharparenright}{\kern0pt}{\isachardoublequoteclose}\ \isanewline
\ \ \ \ \ \ \ \ \isacommand{apply}\isamarkupfalse%
\ {\isacharparenleft}{\kern0pt}rule{\isacharunderscore}{\kern0pt}tac\ pair{\isacharunderscore}{\kern0pt}forallI{\isacharparenright}{\kern0pt}\ \isanewline
\ \ \ \ \ \ \ \ \isacommand{using}\isamarkupfalse%
\ relation{\isacharunderscore}{\kern0pt}P{\isacharunderscore}{\kern0pt}name\ xpname\ \isacommand{apply}\isamarkupfalse%
\ auto\ \isanewline
\ \ \ \ \ \ \isacommand{proof}\isamarkupfalse%
\ {\isacharminus}{\kern0pt}\ \isanewline
\ \ \ \ \ \ \ \ \isacommand{fix}\isamarkupfalse%
\ y\ p\ \isacommand{assume}\isamarkupfalse%
\ ypin\ {\isacharcolon}{\kern0pt}\ {\isachardoublequoteopen}{\isacharless}{\kern0pt}y{\isacharcomma}{\kern0pt}\ p{\isachargreater}{\kern0pt}\ {\isasymin}\ x{\isachardoublequoteclose}\isanewline
\ \ \ \ \ \ \ \ \isacommand{then}\isamarkupfalse%
\ \isacommand{have}\isamarkupfalse%
\ {\isachardoublequoteopen}{\isacharless}{\kern0pt}Pn{\isacharunderscore}{\kern0pt}auto{\isacharparenleft}{\kern0pt}{\isasympi}{\isacharparenright}{\kern0pt}{\isacharbackquote}{\kern0pt}y{\isacharcomma}{\kern0pt}\ {\isasympi}{\isacharbackquote}{\kern0pt}p{\isachargreater}{\kern0pt}\ {\isasymin}\ {\isacharbraceleft}{\kern0pt}\ {\isacharless}{\kern0pt}Pn{\isacharunderscore}{\kern0pt}auto{\isacharparenleft}{\kern0pt}{\isasympi}{\isacharparenright}{\kern0pt}\ {\isacharbackquote}{\kern0pt}\ y{\isacharcomma}{\kern0pt}\ {\isasympi}{\isacharbackquote}{\kern0pt}p{\isachargreater}{\kern0pt}\ {\isachardot}{\kern0pt}\ {\isacharless}{\kern0pt}y{\isacharcomma}{\kern0pt}\ p{\isachargreater}{\kern0pt}\ {\isasymin}\ x\ {\isacharbraceright}{\kern0pt}{\isachardoublequoteclose}\ \isanewline
\ \ \ \ \ \ \ \ \ \ \isacommand{apply}\isamarkupfalse%
\ auto\ \isacommand{apply}\isamarkupfalse%
{\isacharparenleft}{\kern0pt}rule{\isacharunderscore}{\kern0pt}tac\ x{\isacharequal}{\kern0pt}{\isachardoublequoteopen}{\isacharless}{\kern0pt}y{\isacharcomma}{\kern0pt}\ p{\isachargreater}{\kern0pt}{\isachardoublequoteclose}\ \isakeyword{in}\ bexI{\isacharparenright}{\kern0pt}\ \isacommand{by}\isamarkupfalse%
\ auto\isanewline
\ \ \ \ \ \ \ \ \isacommand{then}\isamarkupfalse%
\ \isacommand{have}\isamarkupfalse%
\ H{\isacharcolon}{\kern0pt}{\isachardoublequoteopen}{\isacharless}{\kern0pt}Pn{\isacharunderscore}{\kern0pt}auto{\isacharparenleft}{\kern0pt}{\isasympi}{\isacharparenright}{\kern0pt}{\isacharbackquote}{\kern0pt}y{\isacharcomma}{\kern0pt}\ {\isasympi}{\isacharbackquote}{\kern0pt}p{\isachargreater}{\kern0pt}\ {\isasymin}\ Pn{\isacharunderscore}{\kern0pt}auto{\isacharparenleft}{\kern0pt}{\isasympi}{\isacharparenright}{\kern0pt}\ {\isacharbackquote}{\kern0pt}\ x\ {\isachardoublequoteclose}\ \isanewline
\ \ \ \ \ \ \ \ \ \ \isacommand{using}\isamarkupfalse%
\ xpname\ Pn{\isacharunderscore}{\kern0pt}auto\ \isacommand{by}\isamarkupfalse%
\ auto\isanewline
\isanewline
\ \ \ \ \ \ \ \ \isacommand{have}\isamarkupfalse%
\ {\isachardoublequoteopen}P{\isacharunderscore}{\kern0pt}set{\isacharparenleft}{\kern0pt}succ{\isacharparenleft}{\kern0pt}b{\isacharparenright}{\kern0pt}{\isacharparenright}{\kern0pt}\ {\isacharequal}{\kern0pt}\ Pow{\isacharparenleft}{\kern0pt}P{\isacharunderscore}{\kern0pt}set{\isacharparenleft}{\kern0pt}b{\isacharparenright}{\kern0pt}\ {\isasymtimes}\ P{\isacharparenright}{\kern0pt}\ {\isasyminter}\ M{\isachardoublequoteclose}\ \isacommand{using}\isamarkupfalse%
\ P{\isacharunderscore}{\kern0pt}set{\isacharunderscore}{\kern0pt}succ\ \isacommand{by}\isamarkupfalse%
\ auto\ \isanewline
\ \ \ \ \ \ \ \ \isacommand{then}\isamarkupfalse%
\ \isacommand{have}\isamarkupfalse%
\ {\isachardoublequoteopen}Pn{\isacharunderscore}{\kern0pt}auto{\isacharparenleft}{\kern0pt}{\isasympi}{\isacharparenright}{\kern0pt}\ {\isacharbackquote}{\kern0pt}\ x\ {\isasymin}\ Pow{\isacharparenleft}{\kern0pt}P{\isacharunderscore}{\kern0pt}set{\isacharparenleft}{\kern0pt}b{\isacharparenright}{\kern0pt}\ {\isasymtimes}\ P{\isacharparenright}{\kern0pt}\ {\isasyminter}\ M{\isachardoublequoteclose}\ \isacommand{using}\isamarkupfalse%
\ assm\ \isacommand{by}\isamarkupfalse%
\ auto\ \isanewline
\ \ \ \ \ \ \ \ \isacommand{then}\isamarkupfalse%
\ \isacommand{show}\isamarkupfalse%
\ {\isachardoublequoteopen}\ Pn{\isacharunderscore}{\kern0pt}auto{\isacharparenleft}{\kern0pt}{\isasympi}{\isacharparenright}{\kern0pt}\ {\isacharbackquote}{\kern0pt}\ y\ {\isasymin}\ P{\isacharunderscore}{\kern0pt}set{\isacharparenleft}{\kern0pt}b{\isacharparenright}{\kern0pt}{\isachardoublequoteclose}\ \isacommand{using}\isamarkupfalse%
\ H\ \isacommand{by}\isamarkupfalse%
\ auto\ \isanewline
\ \ \ \ \ \ \isacommand{qed}\isamarkupfalse%
\isanewline
\ \ \ \ \isacommand{qed}\isamarkupfalse%
\isanewline
\ \ \ \ \isacommand{finally}\isamarkupfalse%
\ \isacommand{show}\isamarkupfalse%
\ {\isachardoublequoteopen}x\ {\isasymin}\ P{\isacharunderscore}{\kern0pt}set{\isacharparenleft}{\kern0pt}succ{\isacharparenleft}{\kern0pt}b{\isacharparenright}{\kern0pt}{\isacharparenright}{\kern0pt}\ {\isasymlongleftrightarrow}\ Pn{\isacharunderscore}{\kern0pt}auto{\isacharparenleft}{\kern0pt}{\isasympi}{\isacharparenright}{\kern0pt}\ {\isacharbackquote}{\kern0pt}\ x\ {\isasymin}\ P{\isacharunderscore}{\kern0pt}set{\isacharparenleft}{\kern0pt}succ{\isacharparenleft}{\kern0pt}b{\isacharparenright}{\kern0pt}{\isacharparenright}{\kern0pt}{\isachardoublequoteclose}\ \isacommand{by}\isamarkupfalse%
\ auto\isanewline
\ \ \isacommand{next}\isamarkupfalse%
\ \isanewline
\ \ \ \ \isacommand{fix}\isamarkupfalse%
\ b\ x\ \isacommand{assume}\isamarkupfalse%
\ blim\ {\isacharcolon}{\kern0pt}\ {\isachardoublequoteopen}Limit{\isacharparenleft}{\kern0pt}b{\isacharparenright}{\kern0pt}{\isachardoublequoteclose}\isanewline
\ \ \ \ \isakeyword{and}\ ih\ {\isacharcolon}{\kern0pt}\ {\isachardoublequoteopen}{\isasymforall}a{\isasymin}b{\isachardot}{\kern0pt}\ x\ {\isasymin}\ P{\isacharunderscore}{\kern0pt}names\ {\isasymlongrightarrow}\ x\ {\isasymin}\ P{\isacharunderscore}{\kern0pt}set{\isacharparenleft}{\kern0pt}a{\isacharparenright}{\kern0pt}\ {\isasymlongleftrightarrow}\ Pn{\isacharunderscore}{\kern0pt}auto{\isacharparenleft}{\kern0pt}{\isasympi}{\isacharparenright}{\kern0pt}\ {\isacharbackquote}{\kern0pt}\ x\ {\isasymin}\ P{\isacharunderscore}{\kern0pt}set{\isacharparenleft}{\kern0pt}a{\isacharparenright}{\kern0pt}{\isachardoublequoteclose}\isanewline
\isanewline
\ \ \ \ \isacommand{have}\isamarkupfalse%
\ psetb\ {\isacharcolon}{\kern0pt}\ {\isachardoublequoteopen}P{\isacharunderscore}{\kern0pt}set{\isacharparenleft}{\kern0pt}b{\isacharparenright}{\kern0pt}\ {\isacharequal}{\kern0pt}\ {\isacharparenleft}{\kern0pt}{\isasymUnion}a{\isacharless}{\kern0pt}b{\isachardot}{\kern0pt}\ P{\isacharunderscore}{\kern0pt}set{\isacharparenleft}{\kern0pt}a{\isacharparenright}{\kern0pt}{\isacharparenright}{\kern0pt}{\isachardoublequoteclose}\ \isacommand{using}\isamarkupfalse%
\ P{\isacharunderscore}{\kern0pt}set{\isacharunderscore}{\kern0pt}lim\ blim\ \isacommand{by}\isamarkupfalse%
\ auto\isanewline
\isanewline
\ \ \ \ \isacommand{show}\isamarkupfalse%
\ {\isachardoublequoteopen}x\ {\isasymin}\ P{\isacharunderscore}{\kern0pt}names\ {\isasymlongrightarrow}\ x\ {\isasymin}\ P{\isacharunderscore}{\kern0pt}set{\isacharparenleft}{\kern0pt}b{\isacharparenright}{\kern0pt}\ {\isasymlongleftrightarrow}\ Pn{\isacharunderscore}{\kern0pt}auto{\isacharparenleft}{\kern0pt}{\isasympi}{\isacharparenright}{\kern0pt}\ {\isacharbackquote}{\kern0pt}\ x\ {\isasymin}\ P{\isacharunderscore}{\kern0pt}set{\isacharparenleft}{\kern0pt}b{\isacharparenright}{\kern0pt}{\isachardoublequoteclose}\isanewline
\ \ \ \ \isacommand{proof}\isamarkupfalse%
\ {\isacharparenleft}{\kern0pt}rule\ impI{\isacharsemicolon}{\kern0pt}\ rule\ iffI{\isacharparenright}{\kern0pt}\isanewline
\ \ \ \ \ \ \isacommand{assume}\isamarkupfalse%
\ assm\ {\isacharcolon}{\kern0pt}\ {\isachardoublequoteopen}x\ {\isasymin}\ P{\isacharunderscore}{\kern0pt}names{\isachardoublequoteclose}\ {\isachardoublequoteopen}x\ {\isasymin}\ P{\isacharunderscore}{\kern0pt}set{\isacharparenleft}{\kern0pt}b{\isacharparenright}{\kern0pt}{\isachardoublequoteclose}\ \isanewline
\ \ \ \ \ \ \isacommand{then}\isamarkupfalse%
\ \isacommand{obtain}\isamarkupfalse%
\ a\ \isakeyword{where}\ ah\ {\isacharcolon}{\kern0pt}\ {\isachardoublequoteopen}a\ {\isacharless}{\kern0pt}\ b{\isachardoublequoteclose}\ {\isachardoublequoteopen}x\ {\isasymin}\ P{\isacharunderscore}{\kern0pt}set{\isacharparenleft}{\kern0pt}a{\isacharparenright}{\kern0pt}{\isachardoublequoteclose}\ \ \isacommand{using}\isamarkupfalse%
\ psetb\ \isacommand{by}\isamarkupfalse%
\ auto\isanewline
\ \ \ \ \ \ \isacommand{then}\isamarkupfalse%
\ \isacommand{have}\isamarkupfalse%
\ {\isachardoublequoteopen}Pn{\isacharunderscore}{\kern0pt}auto{\isacharparenleft}{\kern0pt}{\isasympi}{\isacharparenright}{\kern0pt}\ {\isacharbackquote}{\kern0pt}\ x\ {\isasymin}\ P{\isacharunderscore}{\kern0pt}set{\isacharparenleft}{\kern0pt}a{\isacharparenright}{\kern0pt}{\isachardoublequoteclose}\ \isacommand{using}\isamarkupfalse%
\ ih\ ltD\ assm\ \isacommand{by}\isamarkupfalse%
\ auto\ \isanewline
\ \ \ \ \ \ \isacommand{then}\isamarkupfalse%
\ \isacommand{have}\isamarkupfalse%
\ {\isachardoublequoteopen}Pn{\isacharunderscore}{\kern0pt}auto{\isacharparenleft}{\kern0pt}{\isasympi}{\isacharparenright}{\kern0pt}\ {\isacharbackquote}{\kern0pt}\ x\ {\isasymin}\ {\isacharparenleft}{\kern0pt}{\isasymUnion}a{\isacharless}{\kern0pt}b{\isachardot}{\kern0pt}\ P{\isacharunderscore}{\kern0pt}set{\isacharparenleft}{\kern0pt}a{\isacharparenright}{\kern0pt}{\isacharparenright}{\kern0pt}{\isachardoublequoteclose}\ \isacommand{using}\isamarkupfalse%
\ ah\ \isacommand{by}\isamarkupfalse%
\ auto\ \isanewline
\ \ \ \ \ \ \isacommand{then}\isamarkupfalse%
\ \isacommand{show}\isamarkupfalse%
\ {\isachardoublequoteopen}Pn{\isacharunderscore}{\kern0pt}auto{\isacharparenleft}{\kern0pt}{\isasympi}{\isacharparenright}{\kern0pt}\ {\isacharbackquote}{\kern0pt}\ x\ {\isasymin}\ P{\isacharunderscore}{\kern0pt}set{\isacharparenleft}{\kern0pt}b{\isacharparenright}{\kern0pt}{\isachardoublequoteclose}\ \isacommand{using}\isamarkupfalse%
\ psetb\ \isacommand{by}\isamarkupfalse%
\ auto\isanewline
\ \ \ \ \isacommand{next}\isamarkupfalse%
\ \isanewline
\ \ \ \ \ \ \isacommand{assume}\isamarkupfalse%
\ assm\ {\isacharcolon}{\kern0pt}\ {\isachardoublequoteopen}x\ {\isasymin}\ P{\isacharunderscore}{\kern0pt}names{\isachardoublequoteclose}\ {\isachardoublequoteopen}Pn{\isacharunderscore}{\kern0pt}auto{\isacharparenleft}{\kern0pt}{\isasympi}{\isacharparenright}{\kern0pt}\ {\isacharbackquote}{\kern0pt}\ x\ {\isasymin}\ P{\isacharunderscore}{\kern0pt}set{\isacharparenleft}{\kern0pt}b{\isacharparenright}{\kern0pt}{\isachardoublequoteclose}\ \isanewline
\ \ \ \ \ \ \isacommand{then}\isamarkupfalse%
\ \isacommand{obtain}\isamarkupfalse%
\ a\ \isakeyword{where}\ ah\ {\isacharcolon}{\kern0pt}\ {\isachardoublequoteopen}a\ {\isacharless}{\kern0pt}\ b{\isachardoublequoteclose}\ {\isachardoublequoteopen}Pn{\isacharunderscore}{\kern0pt}auto{\isacharparenleft}{\kern0pt}{\isasympi}{\isacharparenright}{\kern0pt}\ {\isacharbackquote}{\kern0pt}\ x\ {\isasymin}\ P{\isacharunderscore}{\kern0pt}set{\isacharparenleft}{\kern0pt}a{\isacharparenright}{\kern0pt}{\isachardoublequoteclose}\ \ \isacommand{using}\isamarkupfalse%
\ psetb\ \isacommand{by}\isamarkupfalse%
\ auto\isanewline
\ \ \ \ \ \ \isacommand{then}\isamarkupfalse%
\ \isacommand{have}\isamarkupfalse%
\ {\isachardoublequoteopen}x\ {\isasymin}\ P{\isacharunderscore}{\kern0pt}set{\isacharparenleft}{\kern0pt}a{\isacharparenright}{\kern0pt}{\isachardoublequoteclose}\ \isacommand{using}\isamarkupfalse%
\ ih\ ltD\ assm\ \isacommand{by}\isamarkupfalse%
\ auto\ \isanewline
\ \ \ \ \ \ \isacommand{then}\isamarkupfalse%
\ \isacommand{have}\isamarkupfalse%
\ {\isachardoublequoteopen}x\ {\isasymin}\ {\isacharparenleft}{\kern0pt}{\isasymUnion}a{\isacharless}{\kern0pt}b{\isachardot}{\kern0pt}\ P{\isacharunderscore}{\kern0pt}set{\isacharparenleft}{\kern0pt}a{\isacharparenright}{\kern0pt}{\isacharparenright}{\kern0pt}{\isachardoublequoteclose}\ \isacommand{using}\isamarkupfalse%
\ ah\ \isacommand{by}\isamarkupfalse%
\ auto\ \isanewline
\ \ \ \ \ \ \isacommand{then}\isamarkupfalse%
\ \isacommand{show}\isamarkupfalse%
\ {\isachardoublequoteopen}x\ {\isasymin}\ P{\isacharunderscore}{\kern0pt}set{\isacharparenleft}{\kern0pt}b{\isacharparenright}{\kern0pt}{\isachardoublequoteclose}\ \isacommand{using}\isamarkupfalse%
\ psetb\ \isacommand{by}\isamarkupfalse%
\ auto\isanewline
\ \ \ \ \isacommand{qed}\isamarkupfalse%
\isanewline
\ \ \isacommand{qed}\isamarkupfalse%
\isanewline
\ \ \isacommand{then}\isamarkupfalse%
\ \isacommand{show}\isamarkupfalse%
\ {\isachardoublequoteopen}x\ {\isasymin}\ P{\isacharunderscore}{\kern0pt}set{\isacharparenleft}{\kern0pt}a{\isacharparenright}{\kern0pt}\ {\isasymlongleftrightarrow}\ Pn{\isacharunderscore}{\kern0pt}auto{\isacharparenleft}{\kern0pt}{\isasympi}{\isacharparenright}{\kern0pt}{\isacharbackquote}{\kern0pt}x\ {\isasymin}\ P{\isacharunderscore}{\kern0pt}set{\isacharparenleft}{\kern0pt}a{\isacharparenright}{\kern0pt}{\isachardoublequoteclose}\ \isanewline
\ \ \ \ \isacommand{using}\isamarkupfalse%
\ assms\ main\ \isacommand{unfolding}\isamarkupfalse%
\ Q{\isacharunderscore}{\kern0pt}def\ \isacommand{by}\isamarkupfalse%
\ auto\ \isanewline
\isacommand{qed}\isamarkupfalse%
%
\endisatagproof
{\isafoldproof}%
%
\isadelimproof
\isanewline
%
\endisadelimproof
\isanewline
\isacommand{lemma}\isamarkupfalse%
\ Pn{\isacharunderscore}{\kern0pt}auto{\isacharunderscore}{\kern0pt}value\ {\isacharcolon}{\kern0pt}\ {\isachardoublequoteopen}is{\isacharunderscore}{\kern0pt}P{\isacharunderscore}{\kern0pt}auto{\isacharparenleft}{\kern0pt}{\isasympi}{\isacharparenright}{\kern0pt}\ {\isasymLongrightarrow}\ x\ {\isasymin}\ P{\isacharunderscore}{\kern0pt}names\ {\isasymLongrightarrow}\ Pn{\isacharunderscore}{\kern0pt}auto{\isacharparenleft}{\kern0pt}{\isasympi}{\isacharparenright}{\kern0pt}{\isacharbackquote}{\kern0pt}x\ {\isasymin}\ P{\isacharunderscore}{\kern0pt}names{\isachardoublequoteclose}\ \isanewline
%
\isadelimproof
%
\endisadelimproof
%
\isatagproof
\isacommand{proof}\isamarkupfalse%
\ {\isacharminus}{\kern0pt}\ \isanewline
\ \ \isacommand{assume}\isamarkupfalse%
\ assms\ {\isacharcolon}{\kern0pt}\ {\isachardoublequoteopen}is{\isacharunderscore}{\kern0pt}P{\isacharunderscore}{\kern0pt}auto{\isacharparenleft}{\kern0pt}{\isasympi}{\isacharparenright}{\kern0pt}{\isachardoublequoteclose}\ {\isachardoublequoteopen}x\ {\isasymin}\ P{\isacharunderscore}{\kern0pt}names{\isachardoublequoteclose}\ \isanewline
\ \ \isacommand{then}\isamarkupfalse%
\ \isacommand{obtain}\isamarkupfalse%
\ a\ \isakeyword{where}\ ah\ {\isacharcolon}{\kern0pt}{\isachardoublequoteopen}Ord{\isacharparenleft}{\kern0pt}a{\isacharparenright}{\kern0pt}{\isachardoublequoteclose}\ {\isachardoublequoteopen}x\ {\isasymin}\ P{\isacharunderscore}{\kern0pt}set{\isacharparenleft}{\kern0pt}succ{\isacharparenleft}{\kern0pt}a{\isacharparenright}{\kern0pt}{\isacharparenright}{\kern0pt}{\isachardoublequoteclose}\ \isacommand{using}\isamarkupfalse%
\ P{\isacharunderscore}{\kern0pt}names{\isacharunderscore}{\kern0pt}in{\isacharunderscore}{\kern0pt}P{\isacharunderscore}{\kern0pt}set{\isacharunderscore}{\kern0pt}succ\ \isacommand{by}\isamarkupfalse%
\ auto\isanewline
\ \ \isacommand{then}\isamarkupfalse%
\ \isacommand{have}\isamarkupfalse%
\ {\isachardoublequoteopen}Pn{\isacharunderscore}{\kern0pt}auto{\isacharparenleft}{\kern0pt}{\isasympi}{\isacharparenright}{\kern0pt}{\isacharbackquote}{\kern0pt}x\ {\isasymin}\ P{\isacharunderscore}{\kern0pt}set{\isacharparenleft}{\kern0pt}succ{\isacharparenleft}{\kern0pt}a{\isacharparenright}{\kern0pt}{\isacharparenright}{\kern0pt}{\isachardoublequoteclose}\ \isacommand{using}\isamarkupfalse%
\ Pn{\isacharunderscore}{\kern0pt}auto{\isacharunderscore}{\kern0pt}value{\isacharunderscore}{\kern0pt}in{\isacharunderscore}{\kern0pt}P{\isacharunderscore}{\kern0pt}set\ assms\ \isacommand{by}\isamarkupfalse%
\ auto\ \isanewline
\ \ \isacommand{then}\isamarkupfalse%
\ \isacommand{show}\isamarkupfalse%
\ {\isachardoublequoteopen}Pn{\isacharunderscore}{\kern0pt}auto{\isacharparenleft}{\kern0pt}{\isasympi}{\isacharparenright}{\kern0pt}{\isacharbackquote}{\kern0pt}x\ {\isasymin}\ P{\isacharunderscore}{\kern0pt}names{\isachardoublequoteclose}\ \isanewline
\ \ \ \ \isacommand{unfolding}\isamarkupfalse%
\ P{\isacharunderscore}{\kern0pt}names{\isacharunderscore}{\kern0pt}def\ \isacommand{using}\isamarkupfalse%
\ ah\ Pn{\isacharunderscore}{\kern0pt}auto{\isacharunderscore}{\kern0pt}value{\isacharunderscore}{\kern0pt}in{\isacharunderscore}{\kern0pt}M\ assms\ \isacommand{by}\isamarkupfalse%
\ auto\ \isanewline
\isacommand{qed}\isamarkupfalse%
%
\endisatagproof
{\isafoldproof}%
%
\isadelimproof
\isanewline
%
\endisadelimproof
\isanewline
\isacommand{lemma}\isamarkupfalse%
\ Pn{\isacharunderscore}{\kern0pt}auto{\isacharunderscore}{\kern0pt}type\ {\isacharcolon}{\kern0pt}\ {\isachardoublequoteopen}is{\isacharunderscore}{\kern0pt}P{\isacharunderscore}{\kern0pt}auto{\isacharparenleft}{\kern0pt}{\isasympi}{\isacharparenright}{\kern0pt}\ {\isasymLongrightarrow}\ Pn{\isacharunderscore}{\kern0pt}auto{\isacharparenleft}{\kern0pt}{\isasympi}{\isacharparenright}{\kern0pt}\ {\isasymin}\ P{\isacharunderscore}{\kern0pt}names\ {\isasymrightarrow}\ P{\isacharunderscore}{\kern0pt}names{\isachardoublequoteclose}\ \isanewline
%
\isadelimproof
\ \ %
\endisadelimproof
%
\isatagproof
\isacommand{unfolding}\isamarkupfalse%
\ Pi{\isacharunderscore}{\kern0pt}def\ \isacommand{apply}\isamarkupfalse%
\ auto\ \isacommand{using}\isamarkupfalse%
\ Pn{\isacharunderscore}{\kern0pt}auto{\isacharunderscore}{\kern0pt}function\ Pn{\isacharunderscore}{\kern0pt}auto{\isacharunderscore}{\kern0pt}domain\ \isacommand{apply}\isamarkupfalse%
\ simp{\isacharunderscore}{\kern0pt}all\isanewline
\isacommand{proof}\isamarkupfalse%
\ {\isacharminus}{\kern0pt}\ \isanewline
\ \ \isacommand{fix}\isamarkupfalse%
\ x\ \isacommand{assume}\isamarkupfalse%
\ assm\ {\isacharcolon}{\kern0pt}\ {\isachardoublequoteopen}x\ {\isasymin}\ Pn{\isacharunderscore}{\kern0pt}auto{\isacharparenleft}{\kern0pt}{\isasympi}{\isacharparenright}{\kern0pt}{\isachardoublequoteclose}\ {\isachardoublequoteopen}is{\isacharunderscore}{\kern0pt}P{\isacharunderscore}{\kern0pt}auto{\isacharparenleft}{\kern0pt}{\isasympi}{\isacharparenright}{\kern0pt}{\isachardoublequoteclose}\isanewline
\ \ \isacommand{then}\isamarkupfalse%
\ \isacommand{obtain}\isamarkupfalse%
\ a\ b\ \isakeyword{where}\ abH{\isacharcolon}{\kern0pt}\ {\isachardoublequoteopen}x\ {\isacharequal}{\kern0pt}\ {\isacharless}{\kern0pt}a{\isacharcomma}{\kern0pt}\ b{\isachargreater}{\kern0pt}{\isachardoublequoteclose}\ \isacommand{unfolding}\isamarkupfalse%
\ Pn{\isacharunderscore}{\kern0pt}auto{\isacharunderscore}{\kern0pt}def\ \isacommand{by}\isamarkupfalse%
\ auto\ \isanewline
\ \ \isacommand{have}\isamarkupfalse%
\ H{\isacharcolon}{\kern0pt}\ {\isachardoublequoteopen}a\ {\isasymin}\ P{\isacharunderscore}{\kern0pt}names{\isachardoublequoteclose}\ \isacommand{using}\isamarkupfalse%
\ abH\ assm\ \isacommand{unfolding}\isamarkupfalse%
\ Pn{\isacharunderscore}{\kern0pt}auto{\isacharunderscore}{\kern0pt}def\ \isacommand{by}\isamarkupfalse%
\ auto\ \isanewline
\ \ \isacommand{have}\isamarkupfalse%
\ {\isachardoublequoteopen}b\ {\isacharequal}{\kern0pt}\ Pn{\isacharunderscore}{\kern0pt}auto{\isacharparenleft}{\kern0pt}{\isasympi}{\isacharparenright}{\kern0pt}{\isacharbackquote}{\kern0pt}a{\isachardoublequoteclose}\ \isacommand{using}\isamarkupfalse%
\ function{\isacharunderscore}{\kern0pt}apply{\isacharunderscore}{\kern0pt}equality\ abH\ assm\ Pn{\isacharunderscore}{\kern0pt}auto{\isacharunderscore}{\kern0pt}function\ \isacommand{by}\isamarkupfalse%
\ auto\ \isanewline
\ \ \isacommand{then}\isamarkupfalse%
\ \isacommand{have}\isamarkupfalse%
\ {\isachardoublequoteopen}b\ {\isasymin}\ P{\isacharunderscore}{\kern0pt}names{\isachardoublequoteclose}\ \isacommand{using}\isamarkupfalse%
\ Pn{\isacharunderscore}{\kern0pt}auto{\isacharunderscore}{\kern0pt}value\ H\ assm\ \isacommand{by}\isamarkupfalse%
\ auto\ \isanewline
\ \ \isacommand{then}\isamarkupfalse%
\ \isacommand{show}\isamarkupfalse%
\ {\isachardoublequoteopen}x\ {\isasymin}\ P{\isacharunderscore}{\kern0pt}names\ {\isasymtimes}\ P{\isacharunderscore}{\kern0pt}names{\isachardoublequoteclose}\ \isacommand{using}\isamarkupfalse%
\ H\ abH\ \isacommand{by}\isamarkupfalse%
\ auto\isanewline
\isacommand{qed}\isamarkupfalse%
%
\endisatagproof
{\isafoldproof}%
%
\isadelimproof
\isanewline
%
\endisadelimproof
\isanewline
\isacommand{lemma}\isamarkupfalse%
\ Pn{\isacharunderscore}{\kern0pt}auto{\isacharunderscore}{\kern0pt}comp\ {\isacharcolon}{\kern0pt}\ {\isachardoublequoteopen}is{\isacharunderscore}{\kern0pt}P{\isacharunderscore}{\kern0pt}auto{\isacharparenleft}{\kern0pt}{\isasympi}{\isacharparenright}{\kern0pt}\ {\isasymLongrightarrow}\ is{\isacharunderscore}{\kern0pt}P{\isacharunderscore}{\kern0pt}auto{\isacharparenleft}{\kern0pt}{\isasymtau}{\isacharparenright}{\kern0pt}\ {\isasymLongrightarrow}\ Pn{\isacharunderscore}{\kern0pt}auto{\isacharparenleft}{\kern0pt}{\isasympi}\ O\ {\isasymtau}{\isacharparenright}{\kern0pt}\ {\isacharequal}{\kern0pt}\ Pn{\isacharunderscore}{\kern0pt}auto{\isacharparenleft}{\kern0pt}{\isasympi}{\isacharparenright}{\kern0pt}\ O\ Pn{\isacharunderscore}{\kern0pt}auto{\isacharparenleft}{\kern0pt}{\isasymtau}{\isacharparenright}{\kern0pt}{\isachardoublequoteclose}\ \isanewline
%
\isadelimproof
%
\endisadelimproof
%
\isatagproof
\isacommand{proof}\isamarkupfalse%
\ {\isacharminus}{\kern0pt}\ \isanewline
\ \ \isacommand{assume}\isamarkupfalse%
\ assms\ {\isacharcolon}{\kern0pt}\ {\isachardoublequoteopen}is{\isacharunderscore}{\kern0pt}P{\isacharunderscore}{\kern0pt}auto{\isacharparenleft}{\kern0pt}{\isasympi}{\isacharparenright}{\kern0pt}{\isachardoublequoteclose}\ {\isachardoublequoteopen}is{\isacharunderscore}{\kern0pt}P{\isacharunderscore}{\kern0pt}auto{\isacharparenleft}{\kern0pt}{\isasymtau}{\isacharparenright}{\kern0pt}{\isachardoublequoteclose}\isanewline
\ \ \isacommand{have}\isamarkupfalse%
\ main\ {\isacharcolon}{\kern0pt}\ {\isachardoublequoteopen}{\isasymAnd}x{\isachardot}{\kern0pt}\ x\ {\isasymin}\ P{\isacharunderscore}{\kern0pt}names\ {\isasymlongrightarrow}\ Pn{\isacharunderscore}{\kern0pt}auto{\isacharparenleft}{\kern0pt}{\isasympi}\ O\ {\isasymtau}{\isacharparenright}{\kern0pt}{\isacharbackquote}{\kern0pt}x\ {\isacharequal}{\kern0pt}\ {\isacharparenleft}{\kern0pt}Pn{\isacharunderscore}{\kern0pt}auto{\isacharparenleft}{\kern0pt}{\isasympi}{\isacharparenright}{\kern0pt}\ O\ Pn{\isacharunderscore}{\kern0pt}auto{\isacharparenleft}{\kern0pt}{\isasymtau}{\isacharparenright}{\kern0pt}{\isacharparenright}{\kern0pt}{\isacharbackquote}{\kern0pt}x{\isachardoublequoteclose}\isanewline
\ \ \ \ \isacommand{apply}\isamarkupfalse%
\ {\isacharparenleft}{\kern0pt}rule{\isacharunderscore}{\kern0pt}tac\ Q{\isacharequal}{\kern0pt}{\isachardoublequoteopen}{\isasymlambda}x{\isachardot}{\kern0pt}\ x\ {\isasymin}\ P{\isacharunderscore}{\kern0pt}names\ {\isasymlongrightarrow}\ Pn{\isacharunderscore}{\kern0pt}auto{\isacharparenleft}{\kern0pt}{\isasympi}\ O\ {\isasymtau}{\isacharparenright}{\kern0pt}{\isacharbackquote}{\kern0pt}x\ {\isacharequal}{\kern0pt}\ {\isacharparenleft}{\kern0pt}Pn{\isacharunderscore}{\kern0pt}auto{\isacharparenleft}{\kern0pt}{\isasympi}{\isacharparenright}{\kern0pt}\ O\ Pn{\isacharunderscore}{\kern0pt}auto{\isacharparenleft}{\kern0pt}{\isasymtau}{\isacharparenright}{\kern0pt}{\isacharparenright}{\kern0pt}{\isacharbackquote}{\kern0pt}x{\isachardoublequoteclose}\ \isakeyword{in}\ ed{\isacharunderscore}{\kern0pt}induction{\isacharparenright}{\kern0pt}\isanewline
\ \ \isacommand{proof}\isamarkupfalse%
\ {\isacharparenleft}{\kern0pt}clarify{\isacharparenright}{\kern0pt}\isanewline
\ \ \ \ \isacommand{fix}\isamarkupfalse%
\ x\ \isacommand{assume}\isamarkupfalse%
\ {\isachardoublequoteopen}{\isasymAnd}y{\isachardot}{\kern0pt}\ ed{\isacharparenleft}{\kern0pt}y{\isacharcomma}{\kern0pt}\ x{\isacharparenright}{\kern0pt}\ {\isasymLongrightarrow}\ y\ {\isasymin}\ P{\isacharunderscore}{\kern0pt}names\ {\isasymlongrightarrow}\ Pn{\isacharunderscore}{\kern0pt}auto{\isacharparenleft}{\kern0pt}{\isasympi}\ O\ {\isasymtau}{\isacharparenright}{\kern0pt}\ {\isacharbackquote}{\kern0pt}\ y\ {\isacharequal}{\kern0pt}\ {\isacharparenleft}{\kern0pt}Pn{\isacharunderscore}{\kern0pt}auto{\isacharparenleft}{\kern0pt}{\isasympi}{\isacharparenright}{\kern0pt}\ O\ Pn{\isacharunderscore}{\kern0pt}auto{\isacharparenleft}{\kern0pt}{\isasymtau}{\isacharparenright}{\kern0pt}{\isacharparenright}{\kern0pt}\ {\isacharbackquote}{\kern0pt}\ y{\isachardoublequoteclose}\isanewline
\ \ \ \ \isakeyword{and}\ xpname\ {\isacharcolon}{\kern0pt}\ {\isachardoublequoteopen}x\ {\isasymin}\ P{\isacharunderscore}{\kern0pt}names{\isachardoublequoteclose}\ \isanewline
\isanewline
\ \ \ \ \isacommand{then}\isamarkupfalse%
\ \isacommand{have}\isamarkupfalse%
\ ih{\isacharcolon}{\kern0pt}\ {\isachardoublequoteopen}{\isasymAnd}y{\isachardot}{\kern0pt}\ ed{\isacharparenleft}{\kern0pt}y{\isacharcomma}{\kern0pt}\ x{\isacharparenright}{\kern0pt}\ {\isasymLongrightarrow}\ y\ {\isasymin}\ P{\isacharunderscore}{\kern0pt}names\ {\isasymLongrightarrow}\ Pn{\isacharunderscore}{\kern0pt}auto{\isacharparenleft}{\kern0pt}{\isasympi}\ O\ {\isasymtau}{\isacharparenright}{\kern0pt}\ {\isacharbackquote}{\kern0pt}\ y\ {\isacharequal}{\kern0pt}\ {\isacharparenleft}{\kern0pt}Pn{\isacharunderscore}{\kern0pt}auto{\isacharparenleft}{\kern0pt}{\isasympi}{\isacharparenright}{\kern0pt}\ O\ Pn{\isacharunderscore}{\kern0pt}auto{\isacharparenleft}{\kern0pt}{\isasymtau}{\isacharparenright}{\kern0pt}{\isacharparenright}{\kern0pt}\ {\isacharbackquote}{\kern0pt}\ y{\isachardoublequoteclose}\ \isacommand{by}\isamarkupfalse%
\ auto\isanewline
\isanewline
\ \ \ \ \isacommand{have}\isamarkupfalse%
\ {\isachardoublequoteopen}Pn{\isacharunderscore}{\kern0pt}auto{\isacharparenleft}{\kern0pt}{\isasympi}\ O\ {\isasymtau}{\isacharparenright}{\kern0pt}\ {\isacharbackquote}{\kern0pt}\ x\ {\isacharequal}{\kern0pt}\ {\isacharbraceleft}{\kern0pt}\ {\isacharless}{\kern0pt}Pn{\isacharunderscore}{\kern0pt}auto{\isacharparenleft}{\kern0pt}{\isasympi}\ O\ {\isasymtau}{\isacharparenright}{\kern0pt}{\isacharbackquote}{\kern0pt}y{\isacharcomma}{\kern0pt}\ {\isacharparenleft}{\kern0pt}{\isasympi}\ O\ {\isasymtau}{\isacharparenright}{\kern0pt}{\isacharbackquote}{\kern0pt}\ p{\isachargreater}{\kern0pt}\ {\isachardot}{\kern0pt}\ {\isacharless}{\kern0pt}y{\isacharcomma}{\kern0pt}\ p{\isachargreater}{\kern0pt}\ {\isasymin}\ x\ {\isacharbraceright}{\kern0pt}{\isachardoublequoteclose}\ \isanewline
\ \ \ \ \ \ \isacommand{using}\isamarkupfalse%
\ Pn{\isacharunderscore}{\kern0pt}auto\ xpname\ \isacommand{by}\isamarkupfalse%
\ auto\ \isanewline
\ \ \ \ \isacommand{also}\isamarkupfalse%
\ \isacommand{have}\isamarkupfalse%
\ {\isachardoublequoteopen}{\isachardot}{\kern0pt}{\isachardot}{\kern0pt}{\isachardot}{\kern0pt}\ {\isacharequal}{\kern0pt}\ {\isacharbraceleft}{\kern0pt}\ {\isacharless}{\kern0pt}{\isacharparenleft}{\kern0pt}Pn{\isacharunderscore}{\kern0pt}auto{\isacharparenleft}{\kern0pt}{\isasympi}{\isacharparenright}{\kern0pt}\ O\ Pn{\isacharunderscore}{\kern0pt}auto{\isacharparenleft}{\kern0pt}{\isasymtau}{\isacharparenright}{\kern0pt}{\isacharparenright}{\kern0pt}\ {\isacharbackquote}{\kern0pt}\ y{\isacharcomma}{\kern0pt}\ {\isacharparenleft}{\kern0pt}{\isasympi}\ O\ {\isasymtau}{\isacharparenright}{\kern0pt}{\isacharbackquote}{\kern0pt}\ p{\isachargreater}{\kern0pt}\ {\isachardot}{\kern0pt}\ {\isacharless}{\kern0pt}y{\isacharcomma}{\kern0pt}\ p{\isachargreater}{\kern0pt}\ {\isasymin}\ x\ {\isacharbraceright}{\kern0pt}{\isachardoublequoteclose}\isanewline
\ \ \ \ \ \ \isacommand{apply}\isamarkupfalse%
\ {\isacharparenleft}{\kern0pt}rule{\isacharunderscore}{\kern0pt}tac\ pair{\isacharunderscore}{\kern0pt}rel{\isacharunderscore}{\kern0pt}eq{\isacharparenright}{\kern0pt}\ \isacommand{using}\isamarkupfalse%
\ xpname\ relation{\isacharunderscore}{\kern0pt}P{\isacharunderscore}{\kern0pt}name\ \isacommand{apply}\isamarkupfalse%
\ simp\ \isanewline
\ \ \ \ \ \ \isacommand{apply}\isamarkupfalse%
\ auto\ \isacommand{apply}\isamarkupfalse%
\ {\isacharparenleft}{\kern0pt}rule{\isacharunderscore}{\kern0pt}tac\ ih{\isacharparenright}{\kern0pt}\ \isacommand{unfolding}\isamarkupfalse%
\ ed{\isacharunderscore}{\kern0pt}def\ \isacommand{apply}\isamarkupfalse%
\ auto\ \isacommand{using}\isamarkupfalse%
\ xpname\ P{\isacharunderscore}{\kern0pt}name{\isacharunderscore}{\kern0pt}domain{\isacharunderscore}{\kern0pt}P{\isacharunderscore}{\kern0pt}name\ \isacommand{by}\isamarkupfalse%
\ auto\ \isanewline
\ \ \ \ \isacommand{also}\isamarkupfalse%
\ \isacommand{have}\isamarkupfalse%
\ {\isachardoublequoteopen}{\isachardot}{\kern0pt}{\isachardot}{\kern0pt}{\isachardot}{\kern0pt}\ {\isacharequal}{\kern0pt}\ {\isacharbraceleft}{\kern0pt}\ {\isacharless}{\kern0pt}Pn{\isacharunderscore}{\kern0pt}auto{\isacharparenleft}{\kern0pt}{\isasympi}{\isacharparenright}{\kern0pt}\ {\isacharbackquote}{\kern0pt}\ {\isacharparenleft}{\kern0pt}Pn{\isacharunderscore}{\kern0pt}auto{\isacharparenleft}{\kern0pt}{\isasymtau}{\isacharparenright}{\kern0pt}\ {\isacharbackquote}{\kern0pt}\ y{\isacharparenright}{\kern0pt}{\isacharcomma}{\kern0pt}\ {\isasympi}\ {\isacharbackquote}{\kern0pt}\ {\isacharparenleft}{\kern0pt}{\isasymtau}\ {\isacharbackquote}{\kern0pt}\ p{\isacharparenright}{\kern0pt}{\isachargreater}{\kern0pt}\ {\isachardot}{\kern0pt}\ {\isacharless}{\kern0pt}y{\isacharcomma}{\kern0pt}\ p{\isachargreater}{\kern0pt}\ {\isasymin}\ x\ {\isacharbraceright}{\kern0pt}{\isachardoublequoteclose}\isanewline
\ \ \ \ \ \ \isacommand{apply}\isamarkupfalse%
\ {\isacharparenleft}{\kern0pt}rule{\isacharunderscore}{\kern0pt}tac\ pair{\isacharunderscore}{\kern0pt}rel{\isacharunderscore}{\kern0pt}eq{\isacharparenright}{\kern0pt}\ \isacommand{using}\isamarkupfalse%
\ xpname\ relation{\isacharunderscore}{\kern0pt}P{\isacharunderscore}{\kern0pt}name\ \isacommand{apply}\isamarkupfalse%
\ auto\ \isanewline
\ \ \ \ \ \ \isacommand{apply}\isamarkupfalse%
\ {\isacharparenleft}{\kern0pt}rule{\isacharunderscore}{\kern0pt}tac\ A{\isacharequal}{\kern0pt}P{\isacharunderscore}{\kern0pt}names\ \isakeyword{and}\ B{\isacharequal}{\kern0pt}P{\isacharunderscore}{\kern0pt}names\ \isakeyword{in}\ comp{\isacharunderscore}{\kern0pt}fun{\isacharunderscore}{\kern0pt}apply{\isacharparenright}{\kern0pt}\ \isacommand{using}\isamarkupfalse%
\ assms\ Pn{\isacharunderscore}{\kern0pt}auto{\isacharunderscore}{\kern0pt}type\ \isacommand{apply}\isamarkupfalse%
\ simp\ \isanewline
\ \ \ \ \ \ \isacommand{using}\isamarkupfalse%
\ P{\isacharunderscore}{\kern0pt}name{\isacharunderscore}{\kern0pt}domain{\isacharunderscore}{\kern0pt}P{\isacharunderscore}{\kern0pt}name\ \isacommand{apply}\isamarkupfalse%
\ simp\ \isanewline
\ \ \ \ \ \ \isacommand{apply}\isamarkupfalse%
\ {\isacharparenleft}{\kern0pt}rule{\isacharunderscore}{\kern0pt}tac\ A{\isacharequal}{\kern0pt}P\ \isakeyword{and}\ B{\isacharequal}{\kern0pt}P\ \isakeyword{in}\ comp{\isacharunderscore}{\kern0pt}fun{\isacharunderscore}{\kern0pt}apply{\isacharparenright}{\kern0pt}\ \isacommand{using}\isamarkupfalse%
\ assms\ \isacommand{unfolding}\isamarkupfalse%
\ is{\isacharunderscore}{\kern0pt}P{\isacharunderscore}{\kern0pt}auto{\isacharunderscore}{\kern0pt}def\ bij{\isacharunderscore}{\kern0pt}def\ inj{\isacharunderscore}{\kern0pt}def\ \isacommand{apply}\isamarkupfalse%
\ simp\ \isanewline
\ \ \ \ \ \ \isacommand{using}\isamarkupfalse%
\ P{\isacharunderscore}{\kern0pt}name{\isacharunderscore}{\kern0pt}range\ \isacommand{by}\isamarkupfalse%
\ auto\isanewline
\ \ \ \ \isacommand{also}\isamarkupfalse%
\ \isacommand{have}\isamarkupfalse%
\ {\isachardoublequoteopen}{\isachardot}{\kern0pt}{\isachardot}{\kern0pt}{\isachardot}{\kern0pt}\ {\isacharequal}{\kern0pt}\ {\isacharbraceleft}{\kern0pt}\ {\isacharless}{\kern0pt}Pn{\isacharunderscore}{\kern0pt}auto{\isacharparenleft}{\kern0pt}{\isasympi}{\isacharparenright}{\kern0pt}\ {\isacharbackquote}{\kern0pt}\ y{\isacharprime}{\kern0pt}{\isacharcomma}{\kern0pt}\ {\isasympi}\ {\isacharbackquote}{\kern0pt}\ p{\isacharprime}{\kern0pt}{\isachargreater}{\kern0pt}\ {\isachardot}{\kern0pt}\ {\isacharless}{\kern0pt}y{\isacharprime}{\kern0pt}{\isacharcomma}{\kern0pt}\ p{\isacharprime}{\kern0pt}{\isachargreater}{\kern0pt}\ {\isasymin}\ Pn{\isacharunderscore}{\kern0pt}auto{\isacharparenleft}{\kern0pt}{\isasymtau}{\isacharparenright}{\kern0pt}{\isacharbackquote}{\kern0pt}x\ {\isacharbraceright}{\kern0pt}{\isachardoublequoteclose}\ \isanewline
\ \ \ \ \ \ \isacommand{apply}\isamarkupfalse%
\ {\isacharparenleft}{\kern0pt}rule{\isacharunderscore}{\kern0pt}tac\ equality{\isacharunderscore}{\kern0pt}iffI{\isacharsemicolon}{\kern0pt}\ rule\ iffI{\isacharparenright}{\kern0pt}\ \isanewline
\ \ \ \ \isacommand{proof}\isamarkupfalse%
\ {\isacharminus}{\kern0pt}\ \isanewline
\ \ \ \ \ \ \isacommand{fix}\isamarkupfalse%
\ v\ \isacommand{assume}\isamarkupfalse%
\ vin\ {\isacharcolon}{\kern0pt}\ {\isachardoublequoteopen}v\ {\isasymin}\ {\isacharbraceleft}{\kern0pt}{\isasymlangle}Pn{\isacharunderscore}{\kern0pt}auto{\isacharparenleft}{\kern0pt}{\isasympi}{\isacharparenright}{\kern0pt}\ {\isacharbackquote}{\kern0pt}\ {\isacharparenleft}{\kern0pt}Pn{\isacharunderscore}{\kern0pt}auto{\isacharparenleft}{\kern0pt}{\isasymtau}{\isacharparenright}{\kern0pt}\ {\isacharbackquote}{\kern0pt}\ y{\isacharparenright}{\kern0pt}{\isacharcomma}{\kern0pt}\ {\isasympi}\ {\isacharbackquote}{\kern0pt}\ {\isacharparenleft}{\kern0pt}{\isasymtau}\ {\isacharbackquote}{\kern0pt}\ p{\isacharparenright}{\kern0pt}{\isasymrangle}\ {\isachardot}{\kern0pt}\ {\isasymlangle}y{\isacharcomma}{\kern0pt}p{\isasymrangle}\ {\isasymin}\ x{\isacharbraceright}{\kern0pt}{\isachardoublequoteclose}\isanewline
\ \ \ \ \ \ \isacommand{then}\isamarkupfalse%
\ \isacommand{have}\isamarkupfalse%
\ {\isachardoublequoteopen}{\isasymexists}y\ p{\isachardot}{\kern0pt}\ {\isacharless}{\kern0pt}y{\isacharcomma}{\kern0pt}\ p{\isachargreater}{\kern0pt}\ {\isasymin}\ x\ {\isasymand}\ v\ {\isacharequal}{\kern0pt}\ {\isasymlangle}Pn{\isacharunderscore}{\kern0pt}auto{\isacharparenleft}{\kern0pt}{\isasympi}{\isacharparenright}{\kern0pt}\ {\isacharbackquote}{\kern0pt}\ {\isacharparenleft}{\kern0pt}Pn{\isacharunderscore}{\kern0pt}auto{\isacharparenleft}{\kern0pt}{\isasymtau}{\isacharparenright}{\kern0pt}\ {\isacharbackquote}{\kern0pt}\ y{\isacharparenright}{\kern0pt}{\isacharcomma}{\kern0pt}\ {\isasympi}\ {\isacharbackquote}{\kern0pt}\ {\isacharparenleft}{\kern0pt}{\isasymtau}\ {\isacharbackquote}{\kern0pt}\ p{\isacharparenright}{\kern0pt}{\isasymrangle}{\isachardoublequoteclose}\ \isanewline
\ \ \ \ \ \ \ \ \isacommand{apply}\isamarkupfalse%
\ {\isacharparenleft}{\kern0pt}rule{\isacharunderscore}{\kern0pt}tac\ pair{\isacharunderscore}{\kern0pt}rel{\isacharunderscore}{\kern0pt}arg{\isacharparenright}{\kern0pt}\ \isacommand{using}\isamarkupfalse%
\ xpname\ relation{\isacharunderscore}{\kern0pt}P{\isacharunderscore}{\kern0pt}name\ \isacommand{apply}\isamarkupfalse%
\ simp\ \isacommand{by}\isamarkupfalse%
\ simp\ \isanewline
\ \ \ \ \ \ \isacommand{then}\isamarkupfalse%
\ \isacommand{obtain}\isamarkupfalse%
\ y\ p\ \isakeyword{where}\ yph\ {\isacharcolon}{\kern0pt}\ {\isachardoublequoteopen}{\isacharless}{\kern0pt}y{\isacharcomma}{\kern0pt}\ p{\isachargreater}{\kern0pt}\ {\isasymin}\ x{\isachardoublequoteclose}\ {\isachardoublequoteopen}v\ {\isacharequal}{\kern0pt}\ {\isasymlangle}Pn{\isacharunderscore}{\kern0pt}auto{\isacharparenleft}{\kern0pt}{\isasympi}{\isacharparenright}{\kern0pt}\ {\isacharbackquote}{\kern0pt}\ {\isacharparenleft}{\kern0pt}Pn{\isacharunderscore}{\kern0pt}auto{\isacharparenleft}{\kern0pt}{\isasymtau}{\isacharparenright}{\kern0pt}\ {\isacharbackquote}{\kern0pt}\ y{\isacharparenright}{\kern0pt}{\isacharcomma}{\kern0pt}\ {\isasympi}\ {\isacharbackquote}{\kern0pt}\ {\isacharparenleft}{\kern0pt}{\isasymtau}\ {\isacharbackquote}{\kern0pt}\ p{\isacharparenright}{\kern0pt}{\isasymrangle}{\isachardoublequoteclose}\ \isacommand{by}\isamarkupfalse%
\ auto\ \isanewline
\ \ \ \ \ \ \isacommand{then}\isamarkupfalse%
\ \isacommand{have}\isamarkupfalse%
\ {\isachardoublequoteopen}{\isasymlangle}Pn{\isacharunderscore}{\kern0pt}auto{\isacharparenleft}{\kern0pt}{\isasympi}{\isacharparenright}{\kern0pt}\ {\isacharbackquote}{\kern0pt}\ {\isacharparenleft}{\kern0pt}Pn{\isacharunderscore}{\kern0pt}auto{\isacharparenleft}{\kern0pt}{\isasymtau}{\isacharparenright}{\kern0pt}\ {\isacharbackquote}{\kern0pt}\ y{\isacharparenright}{\kern0pt}{\isacharcomma}{\kern0pt}\ {\isasympi}\ {\isacharbackquote}{\kern0pt}\ {\isacharparenleft}{\kern0pt}{\isasymtau}\ {\isacharbackquote}{\kern0pt}\ p{\isacharparenright}{\kern0pt}{\isasymrangle}\ {\isasymin}\ {\isacharbraceleft}{\kern0pt}\ {\isacharless}{\kern0pt}Pn{\isacharunderscore}{\kern0pt}auto{\isacharparenleft}{\kern0pt}{\isasympi}{\isacharparenright}{\kern0pt}\ {\isacharbackquote}{\kern0pt}\ y{\isacharprime}{\kern0pt}{\isacharcomma}{\kern0pt}\ {\isasympi}\ {\isacharbackquote}{\kern0pt}\ p{\isacharprime}{\kern0pt}{\isachargreater}{\kern0pt}\ {\isachardot}{\kern0pt}\ {\isacharless}{\kern0pt}y{\isacharprime}{\kern0pt}{\isacharcomma}{\kern0pt}\ p{\isacharprime}{\kern0pt}{\isachargreater}{\kern0pt}\ {\isasymin}\ Pn{\isacharunderscore}{\kern0pt}auto{\isacharparenleft}{\kern0pt}{\isasymtau}{\isacharparenright}{\kern0pt}{\isacharbackquote}{\kern0pt}x\ {\isacharbraceright}{\kern0pt}{\isachardoublequoteclose}\ \ \isanewline
\ \ \ \ \ \ \ \ \isacommand{apply}\isamarkupfalse%
\ {\isacharparenleft}{\kern0pt}rule{\isacharunderscore}{\kern0pt}tac\ pair{\isacharunderscore}{\kern0pt}relI{\isacharparenright}{\kern0pt}\ \isanewline
\ \ \ \ \ \ \ \ \isacommand{apply}\isamarkupfalse%
\ {\isacharparenleft}{\kern0pt}rule{\isacharunderscore}{\kern0pt}tac\ a{\isacharequal}{\kern0pt}{\isachardoublequoteopen}{\isacharbraceleft}{\kern0pt}\ {\isacharless}{\kern0pt}Pn{\isacharunderscore}{\kern0pt}auto{\isacharparenleft}{\kern0pt}{\isasymtau}{\isacharparenright}{\kern0pt}{\isacharbackquote}{\kern0pt}y{\isacharcomma}{\kern0pt}\ {\isasymtau}{\isacharbackquote}{\kern0pt}p{\isachargreater}{\kern0pt}\ {\isachardot}{\kern0pt}\ {\isacharless}{\kern0pt}y{\isacharcomma}{\kern0pt}\ p{\isachargreater}{\kern0pt}\ {\isasymin}\ x\ {\isacharbraceright}{\kern0pt}{\isachardoublequoteclose}\ \isakeyword{and}\ b\ {\isacharequal}{\kern0pt}\ {\isachardoublequoteopen}Pn{\isacharunderscore}{\kern0pt}auto{\isacharparenleft}{\kern0pt}{\isasymtau}{\isacharparenright}{\kern0pt}{\isacharbackquote}{\kern0pt}x{\isachardoublequoteclose}\ \isakeyword{in}\ ssubst{\isacharparenright}{\kern0pt}\ \isanewline
\ \ \ \ \ \ \ \ \isacommand{using}\isamarkupfalse%
\ Pn{\isacharunderscore}{\kern0pt}auto\ xpname\ \isacommand{apply}\isamarkupfalse%
\ simp\ \isanewline
\ \ \ \ \ \ \ \ \isacommand{apply}\isamarkupfalse%
\ {\isacharparenleft}{\kern0pt}rule{\isacharunderscore}{\kern0pt}tac\ pair{\isacharunderscore}{\kern0pt}relI{\isacharparenright}{\kern0pt}\ \isacommand{by}\isamarkupfalse%
\ simp\ \isanewline
\ \ \ \ \ \ \isacommand{then}\isamarkupfalse%
\ \isacommand{show}\isamarkupfalse%
\ {\isachardoublequoteopen}v\ {\isasymin}\ {\isacharbraceleft}{\kern0pt}\ {\isacharless}{\kern0pt}Pn{\isacharunderscore}{\kern0pt}auto{\isacharparenleft}{\kern0pt}{\isasympi}{\isacharparenright}{\kern0pt}\ {\isacharbackquote}{\kern0pt}\ y{\isacharprime}{\kern0pt}{\isacharcomma}{\kern0pt}\ {\isasympi}\ {\isacharbackquote}{\kern0pt}\ p{\isacharprime}{\kern0pt}{\isachargreater}{\kern0pt}\ {\isachardot}{\kern0pt}\ {\isacharless}{\kern0pt}y{\isacharprime}{\kern0pt}{\isacharcomma}{\kern0pt}\ p{\isacharprime}{\kern0pt}{\isachargreater}{\kern0pt}\ {\isasymin}\ Pn{\isacharunderscore}{\kern0pt}auto{\isacharparenleft}{\kern0pt}{\isasymtau}{\isacharparenright}{\kern0pt}{\isacharbackquote}{\kern0pt}x\ {\isacharbraceright}{\kern0pt}{\isachardoublequoteclose}\ \ \isacommand{using}\isamarkupfalse%
\ yph\ \isacommand{by}\isamarkupfalse%
\ auto\ \isanewline
\ \ \ \ \isacommand{next}\isamarkupfalse%
\ \isanewline
\ \ \ \ \ \ \isacommand{fix}\isamarkupfalse%
\ v\ \isacommand{assume}\isamarkupfalse%
\ vin\ {\isacharcolon}{\kern0pt}\ {\isachardoublequoteopen}v\ {\isasymin}\ {\isacharbraceleft}{\kern0pt}{\isasymlangle}Pn{\isacharunderscore}{\kern0pt}auto{\isacharparenleft}{\kern0pt}{\isasympi}{\isacharparenright}{\kern0pt}\ {\isacharbackquote}{\kern0pt}\ y{\isacharprime}{\kern0pt}{\isacharcomma}{\kern0pt}\ {\isasympi}\ {\isacharbackquote}{\kern0pt}\ p{\isacharprime}{\kern0pt}{\isasymrangle}\ {\isachardot}{\kern0pt}\ {\isasymlangle}y{\isacharprime}{\kern0pt}{\isacharcomma}{\kern0pt}p{\isacharprime}{\kern0pt}{\isasymrangle}\ {\isasymin}\ Pn{\isacharunderscore}{\kern0pt}auto{\isacharparenleft}{\kern0pt}{\isasymtau}{\isacharparenright}{\kern0pt}\ {\isacharbackquote}{\kern0pt}\ x{\isacharbraceright}{\kern0pt}{\isachardoublequoteclose}\isanewline
\ \ \ \ \ \ \isacommand{then}\isamarkupfalse%
\ \isacommand{have}\isamarkupfalse%
\ {\isachardoublequoteopen}{\isasymexists}y{\isacharprime}{\kern0pt}\ p{\isacharprime}{\kern0pt}{\isachardot}{\kern0pt}\ {\isacharless}{\kern0pt}y{\isacharprime}{\kern0pt}{\isacharcomma}{\kern0pt}\ p{\isacharprime}{\kern0pt}{\isachargreater}{\kern0pt}\ {\isasymin}\ Pn{\isacharunderscore}{\kern0pt}auto{\isacharparenleft}{\kern0pt}{\isasymtau}{\isacharparenright}{\kern0pt}{\isacharbackquote}{\kern0pt}x\ {\isasymand}\ v\ {\isacharequal}{\kern0pt}\ {\isacharless}{\kern0pt}Pn{\isacharunderscore}{\kern0pt}auto{\isacharparenleft}{\kern0pt}{\isasympi}{\isacharparenright}{\kern0pt}{\isacharbackquote}{\kern0pt}y{\isacharprime}{\kern0pt}{\isacharcomma}{\kern0pt}\ {\isasympi}{\isacharbackquote}{\kern0pt}p{\isacharprime}{\kern0pt}{\isachargreater}{\kern0pt}{\isachardoublequoteclose}\ \isanewline
\ \ \ \ \ \ \ \ \isacommand{apply}\isamarkupfalse%
\ {\isacharparenleft}{\kern0pt}rule{\isacharunderscore}{\kern0pt}tac\ pair{\isacharunderscore}{\kern0pt}rel{\isacharunderscore}{\kern0pt}arg{\isacharparenright}{\kern0pt}\ \isacommand{apply}\isamarkupfalse%
{\isacharparenleft}{\kern0pt}rule{\isacharunderscore}{\kern0pt}tac\ relation{\isacharunderscore}{\kern0pt}P{\isacharunderscore}{\kern0pt}name{\isacharparenright}{\kern0pt}\ \isanewline
\ \ \ \ \ \ \ \ \isacommand{using}\isamarkupfalse%
\ Pn{\isacharunderscore}{\kern0pt}auto{\isacharunderscore}{\kern0pt}value\ xpname\ assms\ \isacommand{by}\isamarkupfalse%
\ auto\ \isanewline
\ \ \ \ \ \ \isacommand{then}\isamarkupfalse%
\ \isacommand{obtain}\isamarkupfalse%
\ y{\isacharprime}{\kern0pt}\ p{\isacharprime}{\kern0pt}\ \isakeyword{where}\ y{\isacharprime}{\kern0pt}p{\isacharprime}{\kern0pt}H\ {\isacharcolon}{\kern0pt}\ {\isachardoublequoteopen}{\isacharless}{\kern0pt}y{\isacharprime}{\kern0pt}{\isacharcomma}{\kern0pt}\ p{\isacharprime}{\kern0pt}{\isachargreater}{\kern0pt}\ {\isasymin}\ Pn{\isacharunderscore}{\kern0pt}auto{\isacharparenleft}{\kern0pt}{\isasymtau}{\isacharparenright}{\kern0pt}{\isacharbackquote}{\kern0pt}x{\isachardoublequoteclose}\ {\isachardoublequoteopen}v\ {\isacharequal}{\kern0pt}\ {\isacharless}{\kern0pt}Pn{\isacharunderscore}{\kern0pt}auto{\isacharparenleft}{\kern0pt}{\isasympi}{\isacharparenright}{\kern0pt}{\isacharbackquote}{\kern0pt}y{\isacharprime}{\kern0pt}{\isacharcomma}{\kern0pt}\ {\isasympi}{\isacharbackquote}{\kern0pt}p{\isacharprime}{\kern0pt}{\isachargreater}{\kern0pt}{\isachardoublequoteclose}\ \isacommand{by}\isamarkupfalse%
\ auto\ \isanewline
\ \ \ \ \ \ \isacommand{then}\isamarkupfalse%
\ \isacommand{have}\isamarkupfalse%
\ {\isachardoublequoteopen}{\isacharless}{\kern0pt}y{\isacharprime}{\kern0pt}{\isacharcomma}{\kern0pt}\ p{\isacharprime}{\kern0pt}{\isachargreater}{\kern0pt}\ {\isasymin}\ {\isacharbraceleft}{\kern0pt}\ {\isacharless}{\kern0pt}Pn{\isacharunderscore}{\kern0pt}auto{\isacharparenleft}{\kern0pt}{\isasymtau}{\isacharparenright}{\kern0pt}{\isacharbackquote}{\kern0pt}y{\isacharcomma}{\kern0pt}\ {\isasymtau}{\isacharbackquote}{\kern0pt}p{\isachargreater}{\kern0pt}\ {\isachardot}{\kern0pt}\ {\isacharless}{\kern0pt}y{\isacharcomma}{\kern0pt}\ p{\isachargreater}{\kern0pt}\ {\isasymin}\ x\ {\isacharbraceright}{\kern0pt}{\isachardoublequoteclose}\ \isacommand{using}\isamarkupfalse%
\ Pn{\isacharunderscore}{\kern0pt}auto\ assms\ xpname\ \isacommand{by}\isamarkupfalse%
\ auto\isanewline
\ \ \ \ \ \ \isacommand{then}\isamarkupfalse%
\ \isacommand{have}\isamarkupfalse%
\ {\isachardoublequoteopen}{\isasymexists}y\ p{\isachardot}{\kern0pt}\ {\isacharless}{\kern0pt}y{\isacharcomma}{\kern0pt}\ p{\isachargreater}{\kern0pt}\ {\isasymin}\ x\ {\isasymand}\ {\isacharless}{\kern0pt}y{\isacharprime}{\kern0pt}{\isacharcomma}{\kern0pt}\ p{\isacharprime}{\kern0pt}{\isachargreater}{\kern0pt}\ {\isacharequal}{\kern0pt}\ {\isacharless}{\kern0pt}Pn{\isacharunderscore}{\kern0pt}auto{\isacharparenleft}{\kern0pt}{\isasymtau}{\isacharparenright}{\kern0pt}{\isacharbackquote}{\kern0pt}y{\isacharcomma}{\kern0pt}\ {\isasymtau}{\isacharbackquote}{\kern0pt}p{\isachargreater}{\kern0pt}{\isachardoublequoteclose}\ \isanewline
\ \ \ \ \ \ \ \ \isacommand{apply}\isamarkupfalse%
\ {\isacharparenleft}{\kern0pt}rule{\isacharunderscore}{\kern0pt}tac\ pair{\isacharunderscore}{\kern0pt}rel{\isacharunderscore}{\kern0pt}arg{\isacharparenright}{\kern0pt}\ \isacommand{using}\isamarkupfalse%
\ xpname\ relation{\isacharunderscore}{\kern0pt}P{\isacharunderscore}{\kern0pt}name\ \isacommand{by}\isamarkupfalse%
\ auto\ \isanewline
\ \ \ \ \ \ \isacommand{then}\isamarkupfalse%
\ \isacommand{obtain}\isamarkupfalse%
\ y\ p\ \isakeyword{where}\ ypH\ {\isacharcolon}{\kern0pt}\ {\isachardoublequoteopen}{\isacharless}{\kern0pt}y{\isacharcomma}{\kern0pt}\ p{\isachargreater}{\kern0pt}\ {\isasymin}\ x{\isachardoublequoteclose}\ {\isachardoublequoteopen}y{\isacharprime}{\kern0pt}\ {\isacharequal}{\kern0pt}\ Pn{\isacharunderscore}{\kern0pt}auto{\isacharparenleft}{\kern0pt}{\isasymtau}{\isacharparenright}{\kern0pt}{\isacharbackquote}{\kern0pt}y{\isachardoublequoteclose}\ {\isachardoublequoteopen}p{\isacharprime}{\kern0pt}\ {\isacharequal}{\kern0pt}\ {\isasymtau}{\isacharbackquote}{\kern0pt}p{\isachardoublequoteclose}\ \isacommand{by}\isamarkupfalse%
\ auto\ \isanewline
\ \ \ \ \ \ \isacommand{then}\isamarkupfalse%
\ \isacommand{have}\isamarkupfalse%
\ {\isachardoublequoteopen}{\isacharless}{\kern0pt}Pn{\isacharunderscore}{\kern0pt}auto{\isacharparenleft}{\kern0pt}{\isasympi}{\isacharparenright}{\kern0pt}\ {\isacharbackquote}{\kern0pt}\ {\isacharparenleft}{\kern0pt}Pn{\isacharunderscore}{\kern0pt}auto{\isacharparenleft}{\kern0pt}{\isasymtau}{\isacharparenright}{\kern0pt}\ {\isacharbackquote}{\kern0pt}\ y{\isacharparenright}{\kern0pt}{\isacharcomma}{\kern0pt}\ {\isasympi}\ {\isacharbackquote}{\kern0pt}\ {\isacharparenleft}{\kern0pt}{\isasymtau}\ {\isacharbackquote}{\kern0pt}\ p{\isacharparenright}{\kern0pt}{\isachargreater}{\kern0pt}\ {\isasymin}\ {\isacharbraceleft}{\kern0pt}{\isasymlangle}Pn{\isacharunderscore}{\kern0pt}auto{\isacharparenleft}{\kern0pt}{\isasympi}{\isacharparenright}{\kern0pt}\ {\isacharbackquote}{\kern0pt}\ {\isacharparenleft}{\kern0pt}Pn{\isacharunderscore}{\kern0pt}auto{\isacharparenleft}{\kern0pt}{\isasymtau}{\isacharparenright}{\kern0pt}\ {\isacharbackquote}{\kern0pt}\ y{\isacharparenright}{\kern0pt}{\isacharcomma}{\kern0pt}\ {\isasympi}\ {\isacharbackquote}{\kern0pt}\ {\isacharparenleft}{\kern0pt}{\isasymtau}\ {\isacharbackquote}{\kern0pt}\ p{\isacharparenright}{\kern0pt}{\isasymrangle}\ {\isachardot}{\kern0pt}\ {\isasymlangle}y{\isacharcomma}{\kern0pt}p{\isasymrangle}\ {\isasymin}\ x{\isacharbraceright}{\kern0pt}{\isachardoublequoteclose}\ \isanewline
\ \ \ \ \ \ \ \ \isacommand{apply}\isamarkupfalse%
\ {\isacharparenleft}{\kern0pt}rule{\isacharunderscore}{\kern0pt}tac\ pair{\isacharunderscore}{\kern0pt}relI{\isacharparenright}{\kern0pt}\ \isacommand{by}\isamarkupfalse%
\ auto\ \isanewline
\ \ \ \ \ \ \isacommand{then}\isamarkupfalse%
\ \isacommand{show}\isamarkupfalse%
\ {\isachardoublequoteopen}v\ {\isasymin}\ {\isacharbraceleft}{\kern0pt}{\isasymlangle}Pn{\isacharunderscore}{\kern0pt}auto{\isacharparenleft}{\kern0pt}{\isasympi}{\isacharparenright}{\kern0pt}\ {\isacharbackquote}{\kern0pt}\ {\isacharparenleft}{\kern0pt}Pn{\isacharunderscore}{\kern0pt}auto{\isacharparenleft}{\kern0pt}{\isasymtau}{\isacharparenright}{\kern0pt}\ {\isacharbackquote}{\kern0pt}\ y{\isacharparenright}{\kern0pt}{\isacharcomma}{\kern0pt}\ {\isasympi}\ {\isacharbackquote}{\kern0pt}\ {\isacharparenleft}{\kern0pt}{\isasymtau}\ {\isacharbackquote}{\kern0pt}\ p{\isacharparenright}{\kern0pt}{\isasymrangle}\ {\isachardot}{\kern0pt}\ {\isasymlangle}y{\isacharcomma}{\kern0pt}p{\isasymrangle}\ {\isasymin}\ x{\isacharbraceright}{\kern0pt}{\isachardoublequoteclose}\isanewline
\ \ \ \ \ \ \ \ \isacommand{using}\isamarkupfalse%
\ y{\isacharprime}{\kern0pt}p{\isacharprime}{\kern0pt}H\ ypH\ \isacommand{by}\isamarkupfalse%
\ auto\ \isanewline
\ \ \ \ \isacommand{qed}\isamarkupfalse%
\isanewline
\ \ \ \ \isacommand{also}\isamarkupfalse%
\ \isacommand{have}\isamarkupfalse%
\ {\isachardoublequoteopen}{\isachardot}{\kern0pt}{\isachardot}{\kern0pt}{\isachardot}{\kern0pt}\ {\isacharequal}{\kern0pt}\ Pn{\isacharunderscore}{\kern0pt}auto{\isacharparenleft}{\kern0pt}{\isasympi}{\isacharparenright}{\kern0pt}\ {\isacharbackquote}{\kern0pt}\ {\isacharparenleft}{\kern0pt}Pn{\isacharunderscore}{\kern0pt}auto{\isacharparenleft}{\kern0pt}{\isasymtau}{\isacharparenright}{\kern0pt}\ {\isacharbackquote}{\kern0pt}\ x{\isacharparenright}{\kern0pt}{\isachardoublequoteclose}\ \isanewline
\ \ \ \ \ \ \isacommand{thm}\isamarkupfalse%
\ Pn{\isacharunderscore}{\kern0pt}auto\ \isanewline
\ \ \ \ \ \ \isacommand{apply}\isamarkupfalse%
\ {\isacharparenleft}{\kern0pt}rule{\isacharunderscore}{\kern0pt}tac\ eq{\isacharunderscore}{\kern0pt}flip{\isacharparenright}{\kern0pt}\isanewline
\ \ \ \ \ \ \isacommand{apply}\isamarkupfalse%
\ {\isacharparenleft}{\kern0pt}rule{\isacharunderscore}{\kern0pt}tac\ Pn{\isacharunderscore}{\kern0pt}auto{\isacharparenright}{\kern0pt}\ \isacommand{using}\isamarkupfalse%
\ Pn{\isacharunderscore}{\kern0pt}auto{\isacharunderscore}{\kern0pt}value\ assms\ xpname\ \isacommand{by}\isamarkupfalse%
\ auto\ \isanewline
\ \ \ \ \isacommand{also}\isamarkupfalse%
\ \isacommand{have}\isamarkupfalse%
\ {\isachardoublequoteopen}{\isachardot}{\kern0pt}{\isachardot}{\kern0pt}{\isachardot}{\kern0pt}\ {\isacharequal}{\kern0pt}\ {\isacharparenleft}{\kern0pt}Pn{\isacharunderscore}{\kern0pt}auto{\isacharparenleft}{\kern0pt}{\isasympi}{\isacharparenright}{\kern0pt}\ O\ Pn{\isacharunderscore}{\kern0pt}auto{\isacharparenleft}{\kern0pt}{\isasymtau}{\isacharparenright}{\kern0pt}{\isacharparenright}{\kern0pt}\ {\isacharbackquote}{\kern0pt}\ x{\isachardoublequoteclose}\ \isanewline
\ \ \ \ \ \ \isacommand{apply}\isamarkupfalse%
\ {\isacharparenleft}{\kern0pt}rule\ eq{\isacharunderscore}{\kern0pt}flip{\isacharparenright}{\kern0pt}\ \isacommand{apply}\isamarkupfalse%
{\isacharparenleft}{\kern0pt}rule{\isacharunderscore}{\kern0pt}tac\ A{\isacharequal}{\kern0pt}P{\isacharunderscore}{\kern0pt}names\ \isakeyword{and}\ B{\isacharequal}{\kern0pt}P{\isacharunderscore}{\kern0pt}names\ \isakeyword{in}\ comp{\isacharunderscore}{\kern0pt}fun{\isacharunderscore}{\kern0pt}apply{\isacharparenright}{\kern0pt}\ \isanewline
\ \ \ \ \ \ \isacommand{using}\isamarkupfalse%
\ assms\ xpname\ Pn{\isacharunderscore}{\kern0pt}auto{\isacharunderscore}{\kern0pt}type\ \isacommand{by}\isamarkupfalse%
\ auto\ \isanewline
\ \ \ \ \isacommand{finally}\isamarkupfalse%
\ \isacommand{show}\isamarkupfalse%
\ {\isachardoublequoteopen}\ Pn{\isacharunderscore}{\kern0pt}auto{\isacharparenleft}{\kern0pt}{\isasympi}\ O\ {\isasymtau}{\isacharparenright}{\kern0pt}\ {\isacharbackquote}{\kern0pt}\ x\ {\isacharequal}{\kern0pt}\ {\isacharparenleft}{\kern0pt}Pn{\isacharunderscore}{\kern0pt}auto{\isacharparenleft}{\kern0pt}{\isasympi}{\isacharparenright}{\kern0pt}\ O\ Pn{\isacharunderscore}{\kern0pt}auto{\isacharparenleft}{\kern0pt}{\isasymtau}{\isacharparenright}{\kern0pt}{\isacharparenright}{\kern0pt}\ {\isacharbackquote}{\kern0pt}\ x\ {\isachardoublequoteclose}\ \isacommand{by}\isamarkupfalse%
\ simp\ \isanewline
\ \ \isacommand{qed}\isamarkupfalse%
\isanewline
\isanewline
\ \ \isacommand{have}\isamarkupfalse%
\ {\isachardoublequoteopen}is{\isacharunderscore}{\kern0pt}P{\isacharunderscore}{\kern0pt}auto{\isacharparenleft}{\kern0pt}{\isasympi}\ O\ {\isasymtau}{\isacharparenright}{\kern0pt}{\isachardoublequoteclose}\ \isanewline
\ \ \ \ \isacommand{using}\isamarkupfalse%
\ assms\isanewline
\ \ \ \ \isacommand{unfolding}\isamarkupfalse%
\ is{\isacharunderscore}{\kern0pt}P{\isacharunderscore}{\kern0pt}auto{\isacharunderscore}{\kern0pt}def\ \isacommand{apply}\isamarkupfalse%
\ {\isacharparenleft}{\kern0pt}rule{\isacharunderscore}{\kern0pt}tac\ conjI{\isacharparenright}{\kern0pt}\ \isanewline
\ \ \ \ \isacommand{using}\isamarkupfalse%
\ comp{\isacharunderscore}{\kern0pt}closed\ \isacommand{apply}\isamarkupfalse%
\ simp\ \isacommand{apply}\isamarkupfalse%
\ {\isacharparenleft}{\kern0pt}rule{\isacharunderscore}{\kern0pt}tac\ conjI{\isacharparenright}{\kern0pt}\ \isanewline
\ \ \ \ \isacommand{using}\isamarkupfalse%
\ comp{\isacharunderscore}{\kern0pt}bij\ \isacommand{apply}\isamarkupfalse%
\ blast\ \isacommand{apply}\isamarkupfalse%
\ clarify\ \isacommand{apply}\isamarkupfalse%
\ {\isacharparenleft}{\kern0pt}rule{\isacharunderscore}{\kern0pt}tac\ Q{\isacharequal}{\kern0pt}{\isachardoublequoteopen}{\isasympi}{\isacharbackquote}{\kern0pt}{\isacharparenleft}{\kern0pt}{\isasymtau}{\isacharbackquote}{\kern0pt}p{\isacharparenright}{\kern0pt}\ {\isasympreceq}\ {\isasympi}{\isacharbackquote}{\kern0pt}{\isacharparenleft}{\kern0pt}{\isasymtau}{\isacharbackquote}{\kern0pt}q{\isacharparenright}{\kern0pt}{\isachardoublequoteclose}\ \isakeyword{in}\ iff{\isacharunderscore}{\kern0pt}trans{\isacharparenright}{\kern0pt}\ \isanewline
\ \ \ \ \isacommand{apply}\isamarkupfalse%
\ {\isacharparenleft}{\kern0pt}rule{\isacharunderscore}{\kern0pt}tac\ Q{\isacharequal}{\kern0pt}{\isachardoublequoteopen}{\isacharparenleft}{\kern0pt}{\isasymtau}{\isacharbackquote}{\kern0pt}p{\isacharparenright}{\kern0pt}\ {\isasympreceq}\ {\isacharparenleft}{\kern0pt}{\isasymtau}{\isacharbackquote}{\kern0pt}q{\isacharparenright}{\kern0pt}{\isachardoublequoteclose}\ \isakeyword{in}\ iff{\isacharunderscore}{\kern0pt}trans{\isacharparenright}{\kern0pt}\ \isacommand{apply}\isamarkupfalse%
\ simp\ \isanewline
\ \ \ \ \isacommand{apply}\isamarkupfalse%
\ {\isacharparenleft}{\kern0pt}rule{\isacharunderscore}{\kern0pt}tac\ P{\isacharequal}{\kern0pt}{\isachardoublequoteopen}{\isasymtau}{\isacharbackquote}{\kern0pt}p\ {\isasymin}\ P\ {\isasymand}\ {\isasymtau}{\isacharbackquote}{\kern0pt}q\ {\isasymin}\ P{\isachardoublequoteclose}\ \isakeyword{in}\ mp{\isacharparenright}{\kern0pt}\ \isacommand{apply}\isamarkupfalse%
\ simp\ \isacommand{using}\isamarkupfalse%
\ assms\ P{\isacharunderscore}{\kern0pt}auto{\isacharunderscore}{\kern0pt}value\ \isacommand{apply}\isamarkupfalse%
\ simp\isanewline
\ \ \ \ \isacommand{apply}\isamarkupfalse%
\ {\isacharparenleft}{\kern0pt}rule{\isacharunderscore}{\kern0pt}tac\ P{\isacharequal}{\kern0pt}{\isachardoublequoteopen}{\isacharparenleft}{\kern0pt}{\isasympi}\ O\ {\isasymtau}{\isacharparenright}{\kern0pt}{\isacharbackquote}{\kern0pt}\ p\ {\isacharequal}{\kern0pt}\ {\isasympi}\ {\isacharbackquote}{\kern0pt}\ {\isacharparenleft}{\kern0pt}{\isasymtau}\ {\isacharbackquote}{\kern0pt}\ p{\isacharparenright}{\kern0pt}\ {\isasymand}\ {\isacharparenleft}{\kern0pt}{\isasympi}\ O\ {\isasymtau}{\isacharparenright}{\kern0pt}{\isacharbackquote}{\kern0pt}\ q\ {\isacharequal}{\kern0pt}\ {\isasympi}\ {\isacharbackquote}{\kern0pt}\ {\isacharparenleft}{\kern0pt}{\isasymtau}\ {\isacharbackquote}{\kern0pt}\ q{\isacharparenright}{\kern0pt}{\isachardoublequoteclose}\ \isakeyword{in}\ mp{\isacharparenright}{\kern0pt}\ \isacommand{apply}\isamarkupfalse%
\ simp\isanewline
\ \ \ \ \isacommand{apply}\isamarkupfalse%
\ {\isacharparenleft}{\kern0pt}rule\ conjI{\isacharparenright}{\kern0pt}\isanewline
\ \ \ \ \isacommand{apply}\isamarkupfalse%
\ {\isacharparenleft}{\kern0pt}rule{\isacharunderscore}{\kern0pt}tac\ A{\isacharequal}{\kern0pt}P\ \isakeyword{and}\ B{\isacharequal}{\kern0pt}P\ \isakeyword{in}\ comp{\isacharunderscore}{\kern0pt}fun{\isacharunderscore}{\kern0pt}apply{\isacharparenright}{\kern0pt}\ \isacommand{using}\isamarkupfalse%
\ assms\ is{\isacharunderscore}{\kern0pt}P{\isacharunderscore}{\kern0pt}auto{\isacharunderscore}{\kern0pt}def\ bij{\isacharunderscore}{\kern0pt}def\ inj{\isacharunderscore}{\kern0pt}def\ \isacommand{apply}\isamarkupfalse%
\ simp{\isacharunderscore}{\kern0pt}all\ \isanewline
\ \ \ \ \isacommand{apply}\isamarkupfalse%
\ {\isacharparenleft}{\kern0pt}rule{\isacharunderscore}{\kern0pt}tac\ A{\isacharequal}{\kern0pt}P\ \isakeyword{and}\ B{\isacharequal}{\kern0pt}P\ \isakeyword{in}\ comp{\isacharunderscore}{\kern0pt}fun{\isacharunderscore}{\kern0pt}apply{\isacharparenright}{\kern0pt}\ \isacommand{using}\isamarkupfalse%
\ assms\ is{\isacharunderscore}{\kern0pt}P{\isacharunderscore}{\kern0pt}auto{\isacharunderscore}{\kern0pt}def\ bij{\isacharunderscore}{\kern0pt}def\ inj{\isacharunderscore}{\kern0pt}def\ \isacommand{apply}\isamarkupfalse%
\ simp{\isacharunderscore}{\kern0pt}all\ \isacommand{done}\isamarkupfalse%
\ \isanewline
\ \ \isacommand{then}\isamarkupfalse%
\ \isacommand{have}\isamarkupfalse%
\ comp{\isacharunderscore}{\kern0pt}type{\isacharcolon}{\kern0pt}\ {\isachardoublequoteopen}Pn{\isacharunderscore}{\kern0pt}auto{\isacharparenleft}{\kern0pt}{\isasympi}\ O\ {\isasymtau}{\isacharparenright}{\kern0pt}\ {\isasymin}\ P{\isacharunderscore}{\kern0pt}names\ {\isasymrightarrow}\ P{\isacharunderscore}{\kern0pt}names{\isachardoublequoteclose}\ \isacommand{using}\isamarkupfalse%
\ Pn{\isacharunderscore}{\kern0pt}auto{\isacharunderscore}{\kern0pt}type\ \isacommand{by}\isamarkupfalse%
\ auto\ \isanewline
\isanewline
\ \ \isacommand{have}\isamarkupfalse%
\ range{\isacharunderscore}{\kern0pt}subset{\isacharcolon}{\kern0pt}\ {\isachardoublequoteopen}range{\isacharparenleft}{\kern0pt}Pn{\isacharunderscore}{\kern0pt}auto{\isacharparenleft}{\kern0pt}{\isasymtau}{\isacharparenright}{\kern0pt}{\isacharparenright}{\kern0pt}\ {\isasymsubseteq}\ domain{\isacharparenleft}{\kern0pt}Pn{\isacharunderscore}{\kern0pt}auto{\isacharparenleft}{\kern0pt}{\isasympi}{\isacharparenright}{\kern0pt}{\isacharparenright}{\kern0pt}{\isachardoublequoteclose}\ \isanewline
\ \ \isacommand{proof}\isamarkupfalse%
{\isacharparenleft}{\kern0pt}rule\ subsetI{\isacharparenright}{\kern0pt}\isanewline
\ \ \ \ \isacommand{fix}\isamarkupfalse%
\ v\ \isacommand{assume}\isamarkupfalse%
\ {\isachardoublequoteopen}v\ {\isasymin}\ range{\isacharparenleft}{\kern0pt}Pn{\isacharunderscore}{\kern0pt}auto{\isacharparenleft}{\kern0pt}{\isasymtau}{\isacharparenright}{\kern0pt}{\isacharparenright}{\kern0pt}{\isachardoublequoteclose}\ \isanewline
\ \ \ \ \isacommand{then}\isamarkupfalse%
\ \isacommand{obtain}\isamarkupfalse%
\ x\ \isakeyword{where}\ xH{\isacharcolon}{\kern0pt}\ {\isachardoublequoteopen}{\isacharless}{\kern0pt}x{\isacharcomma}{\kern0pt}\ v{\isachargreater}{\kern0pt}\ {\isasymin}\ Pn{\isacharunderscore}{\kern0pt}auto{\isacharparenleft}{\kern0pt}{\isasymtau}{\isacharparenright}{\kern0pt}{\isachardoublequoteclose}\ \isacommand{by}\isamarkupfalse%
\ auto\ \isanewline
\ \ \ \ \isacommand{then}\isamarkupfalse%
\ \isacommand{have}\isamarkupfalse%
\ veq{\isacharcolon}{\kern0pt}\ {\isachardoublequoteopen}Pn{\isacharunderscore}{\kern0pt}auto{\isacharparenleft}{\kern0pt}{\isasymtau}{\isacharparenright}{\kern0pt}{\isacharbackquote}{\kern0pt}x\ {\isacharequal}{\kern0pt}\ v{\isachardoublequoteclose}\ \isanewline
\ \ \ \ \ \ \isacommand{apply}\isamarkupfalse%
{\isacharparenleft}{\kern0pt}rule\ function{\isacharunderscore}{\kern0pt}apply{\isacharunderscore}{\kern0pt}equality{\isacharparenright}{\kern0pt}\isanewline
\ \ \ \ \ \ \isacommand{apply}\isamarkupfalse%
{\isacharparenleft}{\kern0pt}rule\ Pn{\isacharunderscore}{\kern0pt}auto{\isacharunderscore}{\kern0pt}function{\isacharparenright}{\kern0pt}\isanewline
\ \ \ \ \ \ \isacommand{done}\isamarkupfalse%
\isanewline
\ \ \ \ \isacommand{have}\isamarkupfalse%
\ {\isachardoublequoteopen}Pn{\isacharunderscore}{\kern0pt}auto{\isacharparenleft}{\kern0pt}{\isasymtau}{\isacharparenright}{\kern0pt}{\isacharbackquote}{\kern0pt}x\ {\isasymin}\ P{\isacharunderscore}{\kern0pt}names{\isachardoublequoteclose}\ \isanewline
\ \ \ \ \ \ \isacommand{apply}\isamarkupfalse%
{\isacharparenleft}{\kern0pt}rule\ Pn{\isacharunderscore}{\kern0pt}auto{\isacharunderscore}{\kern0pt}value{\isacharparenright}{\kern0pt}\isanewline
\ \ \ \ \ \ \isacommand{using}\isamarkupfalse%
\ assms\ xH\ Pn{\isacharunderscore}{\kern0pt}auto{\isacharunderscore}{\kern0pt}def\ \isanewline
\ \ \ \ \ \ \isacommand{by}\isamarkupfalse%
\ auto\isanewline
\ \ \ \ \isacommand{then}\isamarkupfalse%
\ \isacommand{have}\isamarkupfalse%
\ {\isachardoublequoteopen}v\ {\isasymin}\ P{\isacharunderscore}{\kern0pt}names{\isachardoublequoteclose}\ \isacommand{using}\isamarkupfalse%
\ veq\ \isacommand{by}\isamarkupfalse%
\ auto\ \isanewline
\ \ \ \ \isacommand{then}\isamarkupfalse%
\ \isacommand{show}\isamarkupfalse%
\ {\isachardoublequoteopen}v\ {\isasymin}\ domain{\isacharparenleft}{\kern0pt}Pn{\isacharunderscore}{\kern0pt}auto{\isacharparenleft}{\kern0pt}{\isasympi}{\isacharparenright}{\kern0pt}{\isacharparenright}{\kern0pt}{\isachardoublequoteclose}\ \isacommand{using}\isamarkupfalse%
\ Pn{\isacharunderscore}{\kern0pt}auto{\isacharunderscore}{\kern0pt}domain\ \isacommand{by}\isamarkupfalse%
\ auto\isanewline
\ \ \isacommand{qed}\isamarkupfalse%
\isanewline
\isanewline
\ \ \isacommand{show}\isamarkupfalse%
\ {\isachardoublequoteopen}Pn{\isacharunderscore}{\kern0pt}auto{\isacharparenleft}{\kern0pt}{\isasympi}\ O\ {\isasymtau}{\isacharparenright}{\kern0pt}\ {\isacharequal}{\kern0pt}\ Pn{\isacharunderscore}{\kern0pt}auto{\isacharparenleft}{\kern0pt}{\isasympi}{\isacharparenright}{\kern0pt}\ O\ Pn{\isacharunderscore}{\kern0pt}auto{\isacharparenleft}{\kern0pt}{\isasymtau}{\isacharparenright}{\kern0pt}{\isachardoublequoteclose}\ \isacommand{thm}\isamarkupfalse%
\ function{\isacharunderscore}{\kern0pt}eq\isanewline
\ \ \ \ \isacommand{apply}\isamarkupfalse%
\ {\isacharparenleft}{\kern0pt}rule{\isacharunderscore}{\kern0pt}tac\ function{\isacharunderscore}{\kern0pt}eq{\isacharparenright}{\kern0pt}\isanewline
\ \ \ \ \isacommand{unfolding}\isamarkupfalse%
\ relation{\isacharunderscore}{\kern0pt}def\isanewline
\ \ \ \ \isacommand{using}\isamarkupfalse%
\ Pi{\isacharunderscore}{\kern0pt}def\ comp{\isacharunderscore}{\kern0pt}type\isanewline
\ \ \ \ \ \ \ \ \ \isacommand{apply}\isamarkupfalse%
\ force\isanewline
\ \ \ \ \ \ \ \ \isacommand{apply}\isamarkupfalse%
{\isacharparenleft}{\kern0pt}simp\ add{\isacharcolon}{\kern0pt}comp{\isacharunderscore}{\kern0pt}def{\isacharparenright}{\kern0pt}\isanewline
\ \ \ \ \isacommand{using}\isamarkupfalse%
\ Pi{\isacharunderscore}{\kern0pt}def\ comp{\isacharunderscore}{\kern0pt}type\ \isanewline
\ \ \ \ \ \ \ \isacommand{apply}\isamarkupfalse%
\ simp\isanewline
\ \ \ \ \ \ \isacommand{apply}\isamarkupfalse%
{\isacharparenleft}{\kern0pt}rule\ comp{\isacharunderscore}{\kern0pt}function{\isacharparenright}{\kern0pt}\isanewline
\ \ \ \ \ \ \ \isacommand{apply}\isamarkupfalse%
{\isacharparenleft}{\kern0pt}rule\ Pn{\isacharunderscore}{\kern0pt}auto{\isacharunderscore}{\kern0pt}function{\isacharparenright}{\kern0pt}{\isacharplus}{\kern0pt}\isanewline
\ \ \ \ \ \isacommand{apply}\isamarkupfalse%
{\isacharparenleft}{\kern0pt}subst\ domain{\isacharunderscore}{\kern0pt}comp{\isacharunderscore}{\kern0pt}eq{\isacharcomma}{\kern0pt}\ rule\ range{\isacharunderscore}{\kern0pt}subset{\isacharparenright}{\kern0pt}\isanewline
\ \ \ \ \ \isacommand{apply}\isamarkupfalse%
{\isacharparenleft}{\kern0pt}subst\ Pn{\isacharunderscore}{\kern0pt}auto{\isacharunderscore}{\kern0pt}domain{\isacharparenright}{\kern0pt}{\isacharplus}{\kern0pt}\isanewline
\ \ \ \ \ \isacommand{apply}\isamarkupfalse%
\ simp\isanewline
\ \ \ \ \isacommand{apply}\isamarkupfalse%
{\isacharparenleft}{\kern0pt}rule\ mp{\isacharcomma}{\kern0pt}\ rule\ main{\isacharparenright}{\kern0pt}\isanewline
\ \ \ \ \isacommand{using}\isamarkupfalse%
\ Pn{\isacharunderscore}{\kern0pt}auto{\isacharunderscore}{\kern0pt}domain\ \isanewline
\ \ \ \ \isacommand{by}\isamarkupfalse%
\ auto\ \isanewline
\isacommand{qed}\isamarkupfalse%
%
\endisatagproof
{\isafoldproof}%
%
\isadelimproof
\isanewline
%
\endisadelimproof
\isanewline
\isacommand{lemma}\isamarkupfalse%
\ Pn{\isacharunderscore}{\kern0pt}auto{\isacharunderscore}{\kern0pt}id\ {\isacharcolon}{\kern0pt}\ {\isachardoublequoteopen}Pn{\isacharunderscore}{\kern0pt}auto{\isacharparenleft}{\kern0pt}id{\isacharparenleft}{\kern0pt}P{\isacharparenright}{\kern0pt}{\isacharparenright}{\kern0pt}\ {\isacharequal}{\kern0pt}\ id{\isacharparenleft}{\kern0pt}P{\isacharunderscore}{\kern0pt}names{\isacharparenright}{\kern0pt}{\isachardoublequoteclose}\ \isanewline
%
\isadelimproof
%
\endisadelimproof
%
\isatagproof
\isacommand{proof}\isamarkupfalse%
\ {\isacharminus}{\kern0pt}\isanewline
\ \ \isacommand{have}\isamarkupfalse%
\ main\ {\isacharcolon}{\kern0pt}\ {\isachardoublequoteopen}{\isasymAnd}x{\isachardot}{\kern0pt}\ x\ {\isasymin}\ P{\isacharunderscore}{\kern0pt}names\ {\isasymlongrightarrow}\ Pn{\isacharunderscore}{\kern0pt}auto{\isacharparenleft}{\kern0pt}id{\isacharparenleft}{\kern0pt}P{\isacharparenright}{\kern0pt}{\isacharparenright}{\kern0pt}{\isacharbackquote}{\kern0pt}x\ {\isacharequal}{\kern0pt}\ id{\isacharparenleft}{\kern0pt}P{\isacharunderscore}{\kern0pt}names{\isacharparenright}{\kern0pt}\ {\isacharbackquote}{\kern0pt}\ x{\isachardoublequoteclose}\ \isanewline
\ \ \ \ \isacommand{apply}\isamarkupfalse%
\ {\isacharparenleft}{\kern0pt}rule{\isacharunderscore}{\kern0pt}tac\ Q{\isacharequal}{\kern0pt}{\isachardoublequoteopen}{\isasymlambda}x{\isachardot}{\kern0pt}\ x\ {\isasymin}\ P{\isacharunderscore}{\kern0pt}names\ {\isasymlongrightarrow}\ Pn{\isacharunderscore}{\kern0pt}auto{\isacharparenleft}{\kern0pt}id{\isacharparenleft}{\kern0pt}P{\isacharparenright}{\kern0pt}{\isacharparenright}{\kern0pt}{\isacharbackquote}{\kern0pt}x\ {\isacharequal}{\kern0pt}\ id{\isacharparenleft}{\kern0pt}P{\isacharunderscore}{\kern0pt}names{\isacharparenright}{\kern0pt}\ {\isacharbackquote}{\kern0pt}\ \ x{\isachardoublequoteclose}\ \isakeyword{in}\ ed{\isacharunderscore}{\kern0pt}induction{\isacharparenright}{\kern0pt}\isanewline
\ \ \isacommand{proof}\isamarkupfalse%
\ {\isacharparenleft}{\kern0pt}clarify{\isacharparenright}{\kern0pt}\isanewline
\ \ \ \ \isacommand{fix}\isamarkupfalse%
\ x\ \isacommand{assume}\isamarkupfalse%
\ ih\ {\isacharcolon}{\kern0pt}\ {\isachardoublequoteopen}{\isacharparenleft}{\kern0pt}{\isasymAnd}y{\isachardot}{\kern0pt}\ ed{\isacharparenleft}{\kern0pt}y{\isacharcomma}{\kern0pt}\ x{\isacharparenright}{\kern0pt}\ {\isasymLongrightarrow}\ y\ {\isasymin}\ P{\isacharunderscore}{\kern0pt}names\ {\isasymlongrightarrow}\ Pn{\isacharunderscore}{\kern0pt}auto{\isacharparenleft}{\kern0pt}id{\isacharparenleft}{\kern0pt}P{\isacharparenright}{\kern0pt}{\isacharparenright}{\kern0pt}\ {\isacharbackquote}{\kern0pt}\ y\ {\isacharequal}{\kern0pt}\ id{\isacharparenleft}{\kern0pt}P{\isacharunderscore}{\kern0pt}names{\isacharparenright}{\kern0pt}\ {\isacharbackquote}{\kern0pt}\ y{\isacharparenright}{\kern0pt}{\isachardoublequoteclose}\isanewline
\ \ \ \ \isakeyword{and}\ xpname\ {\isacharcolon}{\kern0pt}\ {\isachardoublequoteopen}x\ {\isasymin}\ P{\isacharunderscore}{\kern0pt}names{\isachardoublequoteclose}\ \isanewline
\isanewline
\ \ \ \ \isacommand{have}\isamarkupfalse%
\ {\isachardoublequoteopen}\ Pn{\isacharunderscore}{\kern0pt}auto{\isacharparenleft}{\kern0pt}id{\isacharparenleft}{\kern0pt}P{\isacharparenright}{\kern0pt}{\isacharparenright}{\kern0pt}{\isacharbackquote}{\kern0pt}x\ {\isacharequal}{\kern0pt}\ {\isacharbraceleft}{\kern0pt}\ {\isacharless}{\kern0pt}Pn{\isacharunderscore}{\kern0pt}auto{\isacharparenleft}{\kern0pt}id{\isacharparenleft}{\kern0pt}P{\isacharparenright}{\kern0pt}{\isacharparenright}{\kern0pt}{\isacharbackquote}{\kern0pt}y{\isacharcomma}{\kern0pt}\ id{\isacharparenleft}{\kern0pt}P{\isacharparenright}{\kern0pt}{\isacharbackquote}{\kern0pt}p{\isachargreater}{\kern0pt}{\isachardot}{\kern0pt}\ {\isacharless}{\kern0pt}y{\isacharcomma}{\kern0pt}\ p{\isachargreater}{\kern0pt}\ {\isasymin}\ x\ {\isacharbraceright}{\kern0pt}{\isachardoublequoteclose}\ \isanewline
\ \ \ \ \ \ \isacommand{using}\isamarkupfalse%
\ Pn{\isacharunderscore}{\kern0pt}auto\ xpname\ \isacommand{by}\isamarkupfalse%
\ auto\ \isanewline
\ \ \ \ \isacommand{also}\isamarkupfalse%
\ \isacommand{have}\isamarkupfalse%
\ {\isachardoublequoteopen}{\isachardot}{\kern0pt}{\isachardot}{\kern0pt}{\isachardot}{\kern0pt}\ {\isacharequal}{\kern0pt}\ {\isacharbraceleft}{\kern0pt}\ {\isacharless}{\kern0pt}y{\isacharcomma}{\kern0pt}\ p{\isachargreater}{\kern0pt}\ {\isachardot}{\kern0pt}\ {\isacharless}{\kern0pt}y{\isacharcomma}{\kern0pt}\ p{\isachargreater}{\kern0pt}\ {\isasymin}\ x\ {\isacharbraceright}{\kern0pt}{\isachardoublequoteclose}\ \isanewline
\ \ \ \ \ \ \isacommand{apply}\isamarkupfalse%
\ {\isacharparenleft}{\kern0pt}rule{\isacharunderscore}{\kern0pt}tac\ pair{\isacharunderscore}{\kern0pt}rel{\isacharunderscore}{\kern0pt}eq{\isacharparenright}{\kern0pt}\ \isacommand{using}\isamarkupfalse%
\ xpname\ relation{\isacharunderscore}{\kern0pt}P{\isacharunderscore}{\kern0pt}name\ \isacommand{apply}\isamarkupfalse%
\ simp\ \isanewline
\ \ \ \ \ \ \isacommand{apply}\isamarkupfalse%
\ auto\ \isacommand{apply}\isamarkupfalse%
{\isacharparenleft}{\kern0pt}rule{\isacharunderscore}{\kern0pt}tac\ P{\isacharequal}{\kern0pt}{\isachardoublequoteopen}y\ {\isasymin}\ P{\isacharunderscore}{\kern0pt}names\ {\isasymand}\ ed{\isacharparenleft}{\kern0pt}y{\isacharcomma}{\kern0pt}\ x{\isacharparenright}{\kern0pt}{\isachardoublequoteclose}\ \isakeyword{in}\ mp{\isacharparenright}{\kern0pt}\ \isanewline
\ \ \ \ \ \ \isacommand{using}\isamarkupfalse%
\ ih\ \isacommand{apply}\isamarkupfalse%
\ simp\ \isacommand{apply}\isamarkupfalse%
\ {\isacharparenleft}{\kern0pt}rule\ conjI{\isacharparenright}{\kern0pt}\ \isacommand{using}\isamarkupfalse%
\ xpname\ P{\isacharunderscore}{\kern0pt}name{\isacharunderscore}{\kern0pt}domain{\isacharunderscore}{\kern0pt}P{\isacharunderscore}{\kern0pt}name\ \isacommand{apply}\isamarkupfalse%
\ simp\ \isanewline
\ \ \ \ \ \ \isacommand{unfolding}\isamarkupfalse%
\ ed{\isacharunderscore}{\kern0pt}def\ domain{\isacharunderscore}{\kern0pt}def\ \isacommand{apply}\isamarkupfalse%
\ auto\ \isacommand{unfolding}\isamarkupfalse%
\ id{\isacharunderscore}{\kern0pt}def\ \isacommand{using}\isamarkupfalse%
\ xpname\ P{\isacharunderscore}{\kern0pt}name{\isacharunderscore}{\kern0pt}range\ \isacommand{apply}\isamarkupfalse%
\ simp\ \isacommand{done}\isamarkupfalse%
\ \isanewline
\ \ \ \ \isacommand{also}\isamarkupfalse%
\ \isacommand{have}\isamarkupfalse%
\ {\isachardoublequoteopen}{\isachardot}{\kern0pt}{\isachardot}{\kern0pt}{\isachardot}{\kern0pt}\ {\isacharequal}{\kern0pt}\ x{\isachardoublequoteclose}\ \ \ \isanewline
\ \ \ \ \ \ \isacommand{apply}\isamarkupfalse%
\ {\isacharparenleft}{\kern0pt}rule\ equality{\isacharunderscore}{\kern0pt}iffI{\isacharsemicolon}{\kern0pt}\ rule\ iffI{\isacharparenright}{\kern0pt}\isanewline
\ \ \ \ \isacommand{proof}\isamarkupfalse%
\ {\isacharminus}{\kern0pt}\ \isanewline
\ \ \ \ \ \ \isacommand{fix}\isamarkupfalse%
\ v\ \isacommand{assume}\isamarkupfalse%
\ {\isachardoublequoteopen}v\ {\isasymin}\ {\isacharbraceleft}{\kern0pt}\ {\isacharless}{\kern0pt}y{\isacharcomma}{\kern0pt}\ p{\isachargreater}{\kern0pt}{\isachardot}{\kern0pt}\ {\isacharless}{\kern0pt}y{\isacharcomma}{\kern0pt}\ p{\isachargreater}{\kern0pt}\ {\isasymin}\ x\ {\isacharbraceright}{\kern0pt}{\isachardoublequoteclose}\ \isanewline
\ \ \ \ \ \ \isacommand{then}\isamarkupfalse%
\ \isacommand{have}\isamarkupfalse%
\ {\isachardoublequoteopen}{\isasymexists}y\ p{\isachardot}{\kern0pt}\ {\isacharless}{\kern0pt}y{\isacharcomma}{\kern0pt}\ p{\isachargreater}{\kern0pt}\ {\isasymin}\ x\ {\isasymand}\ v\ {\isacharequal}{\kern0pt}\ {\isacharless}{\kern0pt}y{\isacharcomma}{\kern0pt}\ p{\isachargreater}{\kern0pt}{\isachardoublequoteclose}\ \isacommand{apply}\isamarkupfalse%
\ {\isacharparenleft}{\kern0pt}rule{\isacharunderscore}{\kern0pt}tac\ pair{\isacharunderscore}{\kern0pt}rel{\isacharunderscore}{\kern0pt}arg{\isacharparenright}{\kern0pt}\ \isacommand{using}\isamarkupfalse%
\ xpname\ relation{\isacharunderscore}{\kern0pt}P{\isacharunderscore}{\kern0pt}name\ \isacommand{by}\isamarkupfalse%
\ auto\ \isanewline
\ \ \ \ \ \ \isacommand{then}\isamarkupfalse%
\ \isacommand{obtain}\isamarkupfalse%
\ y\ p\ \isakeyword{where}\ {\isachardoublequoteopen}v\ {\isacharequal}{\kern0pt}\ {\isacharless}{\kern0pt}y{\isacharcomma}{\kern0pt}\ p{\isachargreater}{\kern0pt}{\isachardoublequoteclose}\ {\isachardoublequoteopen}{\isacharless}{\kern0pt}y{\isacharcomma}{\kern0pt}\ p{\isachargreater}{\kern0pt}\ {\isasymin}\ x{\isachardoublequoteclose}\ \isacommand{by}\isamarkupfalse%
\ auto\ \isanewline
\ \ \ \ \ \ \isacommand{then}\isamarkupfalse%
\ \isacommand{show}\isamarkupfalse%
\ {\isachardoublequoteopen}v\ {\isasymin}\ x{\isachardoublequoteclose}\ \isacommand{by}\isamarkupfalse%
\ auto\ \isanewline
\ \ \ \ \isacommand{next}\isamarkupfalse%
\ \isanewline
\ \ \ \ \ \ \isacommand{fix}\isamarkupfalse%
\ v\ \isacommand{assume}\isamarkupfalse%
\ assm\ {\isacharcolon}{\kern0pt}\ {\isachardoublequoteopen}v\ {\isasymin}\ x{\isachardoublequoteclose}\ \isacommand{then}\isamarkupfalse%
\ \isacommand{obtain}\isamarkupfalse%
\ y\ p\ \isakeyword{where}\ ypH{\isacharcolon}{\kern0pt}\ {\isachardoublequoteopen}v\ {\isacharequal}{\kern0pt}\ {\isacharless}{\kern0pt}y{\isacharcomma}{\kern0pt}\ p{\isachargreater}{\kern0pt}{\isachardoublequoteclose}\ \isanewline
\ \ \ \ \ \ \ \ \isacommand{using}\isamarkupfalse%
\ xpname\ relation{\isacharunderscore}{\kern0pt}P{\isacharunderscore}{\kern0pt}name\ \isacommand{unfolding}\isamarkupfalse%
\ relation{\isacharunderscore}{\kern0pt}def\ \isacommand{by}\isamarkupfalse%
\ blast\isanewline
\ \ \ \ \ \ \isacommand{then}\isamarkupfalse%
\ \isacommand{have}\isamarkupfalse%
\ {\isachardoublequoteopen}{\isacharless}{\kern0pt}y{\isacharcomma}{\kern0pt}\ p{\isachargreater}{\kern0pt}\ {\isasymin}\ {\isacharbraceleft}{\kern0pt}\ {\isacharless}{\kern0pt}y{\isacharcomma}{\kern0pt}\ p{\isachargreater}{\kern0pt}\ {\isachardot}{\kern0pt}\ {\isacharless}{\kern0pt}y{\isacharcomma}{\kern0pt}\ p{\isachargreater}{\kern0pt}\ {\isasymin}\ x\ {\isacharbraceright}{\kern0pt}{\isachardoublequoteclose}\ \isacommand{apply}\isamarkupfalse%
\ {\isacharparenleft}{\kern0pt}rule{\isacharunderscore}{\kern0pt}tac\ pair{\isacharunderscore}{\kern0pt}relI{\isacharparenright}{\kern0pt}\ \isacommand{using}\isamarkupfalse%
\ assm\ \isacommand{by}\isamarkupfalse%
\ auto\ \isanewline
\ \ \ \ \ \ \isacommand{then}\isamarkupfalse%
\ \isacommand{show}\isamarkupfalse%
\ {\isachardoublequoteopen}v\ {\isasymin}\ {\isacharbraceleft}{\kern0pt}\ {\isacharless}{\kern0pt}y{\isacharcomma}{\kern0pt}\ p{\isachargreater}{\kern0pt}\ {\isachardot}{\kern0pt}\ {\isacharless}{\kern0pt}y{\isacharcomma}{\kern0pt}\ p{\isachargreater}{\kern0pt}\ {\isasymin}\ x\ {\isacharbraceright}{\kern0pt}{\isachardoublequoteclose}\ \isacommand{using}\isamarkupfalse%
\ ypH\ \isacommand{by}\isamarkupfalse%
\ auto\ \isanewline
\ \ \ \ \isacommand{qed}\isamarkupfalse%
\isanewline
\ \ \ \ \isacommand{also}\isamarkupfalse%
\ \isacommand{have}\isamarkupfalse%
\ {\isachardoublequoteopen}{\isachardot}{\kern0pt}{\isachardot}{\kern0pt}{\isachardot}{\kern0pt}\ {\isacharequal}{\kern0pt}\ id{\isacharparenleft}{\kern0pt}P{\isacharunderscore}{\kern0pt}names{\isacharparenright}{\kern0pt}{\isacharbackquote}{\kern0pt}x{\isachardoublequoteclose}\ \isacommand{unfolding}\isamarkupfalse%
\ id{\isacharunderscore}{\kern0pt}def\ \isacommand{using}\isamarkupfalse%
\ xpname\ \isacommand{by}\isamarkupfalse%
\ auto\isanewline
\ \ \ \ \isacommand{finally}\isamarkupfalse%
\ \isacommand{show}\isamarkupfalse%
\ {\isachardoublequoteopen}Pn{\isacharunderscore}{\kern0pt}auto{\isacharparenleft}{\kern0pt}id{\isacharparenleft}{\kern0pt}P{\isacharparenright}{\kern0pt}{\isacharparenright}{\kern0pt}\ {\isacharbackquote}{\kern0pt}\ x\ {\isacharequal}{\kern0pt}\ id{\isacharparenleft}{\kern0pt}P{\isacharunderscore}{\kern0pt}names{\isacharparenright}{\kern0pt}{\isacharbackquote}{\kern0pt}x{\isachardoublequoteclose}\ \isacommand{by}\isamarkupfalse%
\ simp\ \isanewline
\ \ \isacommand{qed}\isamarkupfalse%
\isanewline
\isanewline
\ \ \isacommand{show}\isamarkupfalse%
\ {\isachardoublequoteopen}Pn{\isacharunderscore}{\kern0pt}auto{\isacharparenleft}{\kern0pt}id{\isacharparenleft}{\kern0pt}P{\isacharparenright}{\kern0pt}{\isacharparenright}{\kern0pt}\ {\isacharequal}{\kern0pt}\ id{\isacharparenleft}{\kern0pt}P{\isacharunderscore}{\kern0pt}names{\isacharparenright}{\kern0pt}{\isachardoublequoteclose}\ \isanewline
\ \ \ \ \isacommand{apply}\isamarkupfalse%
\ {\isacharparenleft}{\kern0pt}rule\ function{\isacharunderscore}{\kern0pt}eq{\isacharparenright}{\kern0pt}\ \isanewline
\ \ \ \ \ \ \ \ \ \isacommand{apply}\isamarkupfalse%
{\isacharparenleft}{\kern0pt}simp\ add{\isacharcolon}{\kern0pt}relation{\isacharunderscore}{\kern0pt}def\ Pn{\isacharunderscore}{\kern0pt}auto{\isacharunderscore}{\kern0pt}def{\isacharparenright}{\kern0pt}\isanewline
\ \ \ \ \isacommand{using}\isamarkupfalse%
\ relation{\isacharunderscore}{\kern0pt}def\ id{\isacharunderscore}{\kern0pt}def\ relation{\isacharunderscore}{\kern0pt}lam\ \isanewline
\ \ \ \ \ \ \ \ \isacommand{apply}\isamarkupfalse%
\ force\isanewline
\ \ \ \ \ \ \ \isacommand{apply}\isamarkupfalse%
{\isacharparenleft}{\kern0pt}rule\ Pn{\isacharunderscore}{\kern0pt}auto{\isacharunderscore}{\kern0pt}function{\isacharparenright}{\kern0pt}\isanewline
\ \ \ \ \ \ \isacommand{apply}\isamarkupfalse%
{\isacharparenleft}{\kern0pt}simp\ add{\isacharcolon}{\kern0pt}id{\isacharunderscore}{\kern0pt}def{\isacharcomma}{\kern0pt}\ rule\ function{\isacharunderscore}{\kern0pt}lam{\isacharparenright}{\kern0pt}\isanewline
\ \ \ \ \ \isacommand{apply}\isamarkupfalse%
{\isacharparenleft}{\kern0pt}subst\ Pn{\isacharunderscore}{\kern0pt}auto{\isacharunderscore}{\kern0pt}domain{\isacharcomma}{\kern0pt}\ simp\ add{\isacharcolon}{\kern0pt}id{\isacharunderscore}{\kern0pt}def{\isacharparenright}{\kern0pt}\isanewline
\ \ \ \ \isacommand{apply}\isamarkupfalse%
{\isacharparenleft}{\kern0pt}rule\ mp{\isacharcomma}{\kern0pt}\ rule\ main{\isacharparenright}{\kern0pt}\isanewline
\ \ \ \ \isacommand{using}\isamarkupfalse%
\ Pn{\isacharunderscore}{\kern0pt}auto{\isacharunderscore}{\kern0pt}domain\ \isanewline
\ \ \ \ \isacommand{by}\isamarkupfalse%
\ auto\isanewline
\isacommand{qed}\isamarkupfalse%
%
\endisatagproof
{\isafoldproof}%
%
\isadelimproof
\isanewline
%
\endisadelimproof
\isanewline
\isacommand{lemma}\isamarkupfalse%
\ Pn{\isacharunderscore}{\kern0pt}auto{\isacharunderscore}{\kern0pt}bij\ {\isacharcolon}{\kern0pt}\ {\isachardoublequoteopen}is{\isacharunderscore}{\kern0pt}P{\isacharunderscore}{\kern0pt}auto{\isacharparenleft}{\kern0pt}{\isasympi}{\isacharparenright}{\kern0pt}\ {\isasymLongrightarrow}\ Pn{\isacharunderscore}{\kern0pt}auto{\isacharparenleft}{\kern0pt}{\isasympi}{\isacharparenright}{\kern0pt}\ {\isasymin}\ bij{\isacharparenleft}{\kern0pt}P{\isacharunderscore}{\kern0pt}names{\isacharcomma}{\kern0pt}\ P{\isacharunderscore}{\kern0pt}names{\isacharparenright}{\kern0pt}{\isachardoublequoteclose}\isanewline
%
\isadelimproof
\ \ %
\endisadelimproof
%
\isatagproof
\isacommand{apply}\isamarkupfalse%
\ {\isacharparenleft}{\kern0pt}rule{\isacharunderscore}{\kern0pt}tac\ P{\isacharequal}{\kern0pt}{\isachardoublequoteopen}{\isasympi}\ {\isasymin}\ bij{\isacharparenleft}{\kern0pt}P{\isacharcomma}{\kern0pt}\ P{\isacharparenright}{\kern0pt}{\isachardoublequoteclose}\ \isakeyword{in}\ mp{\isacharparenright}{\kern0pt}\ \isacommand{apply}\isamarkupfalse%
{\isacharparenleft}{\kern0pt}rule\ impI{\isacharparenright}{\kern0pt}\isanewline
\ \ \isacommand{apply}\isamarkupfalse%
\ {\isacharparenleft}{\kern0pt}rule{\isacharunderscore}{\kern0pt}tac\ g{\isacharequal}{\kern0pt}{\isachardoublequoteopen}Pn{\isacharunderscore}{\kern0pt}auto{\isacharparenleft}{\kern0pt}converse{\isacharparenleft}{\kern0pt}{\isasympi}{\isacharparenright}{\kern0pt}{\isacharparenright}{\kern0pt}{\isachardoublequoteclose}\ \isakeyword{in}\ fg{\isacharunderscore}{\kern0pt}imp{\isacharunderscore}{\kern0pt}bijective{\isacharparenright}{\kern0pt}\ \isanewline
\ \ \isacommand{using}\isamarkupfalse%
\ Pn{\isacharunderscore}{\kern0pt}auto{\isacharunderscore}{\kern0pt}type\ P{\isacharunderscore}{\kern0pt}auto{\isacharunderscore}{\kern0pt}converse\ \isacommand{apply}\isamarkupfalse%
\ simp{\isacharunderscore}{\kern0pt}all\ \isanewline
\ \ \isacommand{apply}\isamarkupfalse%
\ {\isacharparenleft}{\kern0pt}rule{\isacharunderscore}{\kern0pt}tac\ b{\isacharequal}{\kern0pt}{\isachardoublequoteopen}Pn{\isacharunderscore}{\kern0pt}auto{\isacharparenleft}{\kern0pt}{\isasympi}{\isacharparenright}{\kern0pt}\ O\ Pn{\isacharunderscore}{\kern0pt}auto{\isacharparenleft}{\kern0pt}converse{\isacharparenleft}{\kern0pt}{\isasympi}{\isacharparenright}{\kern0pt}{\isacharparenright}{\kern0pt}{\isachardoublequoteclose}\ \isakeyword{and}\ a{\isacharequal}{\kern0pt}{\isachardoublequoteopen}Pn{\isacharunderscore}{\kern0pt}auto{\isacharparenleft}{\kern0pt}{\isasympi}\ O\ converse{\isacharparenleft}{\kern0pt}{\isasympi}{\isacharparenright}{\kern0pt}{\isacharparenright}{\kern0pt}{\isachardoublequoteclose}\ \isakeyword{in}\ ssubst{\isacharparenright}{\kern0pt}\ \isanewline
\ \ \isacommand{using}\isamarkupfalse%
\ Pn{\isacharunderscore}{\kern0pt}auto{\isacharunderscore}{\kern0pt}comp\ P{\isacharunderscore}{\kern0pt}auto{\isacharunderscore}{\kern0pt}converse\ \isacommand{apply}\isamarkupfalse%
\ simp\ \isanewline
\ \ \isacommand{apply}\isamarkupfalse%
\ {\isacharparenleft}{\kern0pt}rule{\isacharunderscore}{\kern0pt}tac\ b{\isacharequal}{\kern0pt}{\isachardoublequoteopen}{\isasympi}\ O\ converse{\isacharparenleft}{\kern0pt}{\isasympi}{\isacharparenright}{\kern0pt}{\isachardoublequoteclose}\ \isakeyword{and}\ a\ {\isacharequal}{\kern0pt}\ {\isachardoublequoteopen}id{\isacharparenleft}{\kern0pt}P{\isacharparenright}{\kern0pt}{\isachardoublequoteclose}\ \isakeyword{in}\ ssubst{\isacharparenright}{\kern0pt}\ \isanewline
\ \ \isacommand{apply}\isamarkupfalse%
\ {\isacharparenleft}{\kern0pt}rule{\isacharunderscore}{\kern0pt}tac\ A{\isacharequal}{\kern0pt}P\ \isakeyword{and}\ B{\isacharequal}{\kern0pt}P\ \isakeyword{in}\ right{\isacharunderscore}{\kern0pt}comp{\isacharunderscore}{\kern0pt}inverse{\isacharparenright}{\kern0pt}\ \isacommand{using}\isamarkupfalse%
\ bij{\isacharunderscore}{\kern0pt}is{\isacharunderscore}{\kern0pt}surj\ \isacommand{apply}\isamarkupfalse%
\ simp\ \isanewline
\ \ \isacommand{using}\isamarkupfalse%
\ Pn{\isacharunderscore}{\kern0pt}auto{\isacharunderscore}{\kern0pt}id\ \isacommand{apply}\isamarkupfalse%
\ simp\ \isanewline
\ \ \isacommand{apply}\isamarkupfalse%
\ {\isacharparenleft}{\kern0pt}rule{\isacharunderscore}{\kern0pt}tac\ b{\isacharequal}{\kern0pt}{\isachardoublequoteopen}Pn{\isacharunderscore}{\kern0pt}auto{\isacharparenleft}{\kern0pt}converse{\isacharparenleft}{\kern0pt}{\isasympi}{\isacharparenright}{\kern0pt}{\isacharparenright}{\kern0pt}\ O\ Pn{\isacharunderscore}{\kern0pt}auto{\isacharparenleft}{\kern0pt}{\isasympi}{\isacharparenright}{\kern0pt}{\isachardoublequoteclose}\ \isakeyword{and}\ a{\isacharequal}{\kern0pt}{\isachardoublequoteopen}Pn{\isacharunderscore}{\kern0pt}auto{\isacharparenleft}{\kern0pt}converse{\isacharparenleft}{\kern0pt}{\isasympi}{\isacharparenright}{\kern0pt}\ O\ {\isasympi}{\isacharparenright}{\kern0pt}{\isachardoublequoteclose}\ \isakeyword{in}\ ssubst{\isacharparenright}{\kern0pt}\ \isanewline
\ \ \isacommand{using}\isamarkupfalse%
\ Pn{\isacharunderscore}{\kern0pt}auto{\isacharunderscore}{\kern0pt}comp\ P{\isacharunderscore}{\kern0pt}auto{\isacharunderscore}{\kern0pt}converse\ \isacommand{apply}\isamarkupfalse%
\ simp\ \isanewline
\ \ \isacommand{apply}\isamarkupfalse%
\ {\isacharparenleft}{\kern0pt}rule{\isacharunderscore}{\kern0pt}tac\ b{\isacharequal}{\kern0pt}{\isachardoublequoteopen}converse{\isacharparenleft}{\kern0pt}{\isasympi}{\isacharparenright}{\kern0pt}\ O\ {\isasympi}{\isachardoublequoteclose}\ \isakeyword{and}\ a\ {\isacharequal}{\kern0pt}\ {\isachardoublequoteopen}id{\isacharparenleft}{\kern0pt}P{\isacharparenright}{\kern0pt}{\isachardoublequoteclose}\ \isakeyword{in}\ ssubst{\isacharparenright}{\kern0pt}\ \isanewline
\ \ \isacommand{apply}\isamarkupfalse%
\ {\isacharparenleft}{\kern0pt}rule{\isacharunderscore}{\kern0pt}tac\ A{\isacharequal}{\kern0pt}P\ \isakeyword{and}\ B{\isacharequal}{\kern0pt}P\ \isakeyword{in}\ left{\isacharunderscore}{\kern0pt}comp{\isacharunderscore}{\kern0pt}inverse{\isacharparenright}{\kern0pt}\ \isacommand{using}\isamarkupfalse%
\ bij{\isacharunderscore}{\kern0pt}is{\isacharunderscore}{\kern0pt}inj\ \isacommand{apply}\isamarkupfalse%
\ simp\ \isanewline
\ \ \isacommand{using}\isamarkupfalse%
\ Pn{\isacharunderscore}{\kern0pt}auto{\isacharunderscore}{\kern0pt}id\ \isacommand{apply}\isamarkupfalse%
\ simp\ \isanewline
\isacommand{proof}\isamarkupfalse%
\ {\isacharminus}{\kern0pt}\ \isanewline
\ \ \isacommand{assume}\isamarkupfalse%
\ assm\ {\isacharcolon}{\kern0pt}\ {\isachardoublequoteopen}is{\isacharunderscore}{\kern0pt}P{\isacharunderscore}{\kern0pt}auto{\isacharparenleft}{\kern0pt}{\isasympi}{\isacharparenright}{\kern0pt}{\isachardoublequoteclose}\ \isanewline
\ \ \isacommand{then}\isamarkupfalse%
\ \isacommand{show}\isamarkupfalse%
\ {\isachardoublequoteopen}{\isasympi}\ {\isasymin}\ bij{\isacharparenleft}{\kern0pt}P{\isacharcomma}{\kern0pt}\ P{\isacharparenright}{\kern0pt}{\isachardoublequoteclose}\ \isacommand{unfolding}\isamarkupfalse%
\ is{\isacharunderscore}{\kern0pt}P{\isacharunderscore}{\kern0pt}auto{\isacharunderscore}{\kern0pt}def\ \isacommand{by}\isamarkupfalse%
\ auto\ \isanewline
\isacommand{qed}\isamarkupfalse%
%
\endisatagproof
{\isafoldproof}%
%
\isadelimproof
\isanewline
%
\endisadelimproof
\isanewline
\isacommand{lemma}\isamarkupfalse%
\ Pn{\isacharunderscore}{\kern0pt}auto{\isacharunderscore}{\kern0pt}converse\ {\isacharcolon}{\kern0pt}\ {\isachardoublequoteopen}is{\isacharunderscore}{\kern0pt}P{\isacharunderscore}{\kern0pt}auto{\isacharparenleft}{\kern0pt}{\isasympi}{\isacharparenright}{\kern0pt}\ {\isasymLongrightarrow}\ Pn{\isacharunderscore}{\kern0pt}auto{\isacharparenleft}{\kern0pt}converse{\isacharparenleft}{\kern0pt}{\isasympi}{\isacharparenright}{\kern0pt}{\isacharparenright}{\kern0pt}\ {\isacharequal}{\kern0pt}\ converse{\isacharparenleft}{\kern0pt}Pn{\isacharunderscore}{\kern0pt}auto{\isacharparenleft}{\kern0pt}{\isasympi}{\isacharparenright}{\kern0pt}{\isacharparenright}{\kern0pt}{\isachardoublequoteclose}\ \isanewline
%
\isadelimproof
%
\endisadelimproof
%
\isatagproof
\isacommand{proof}\isamarkupfalse%
\ {\isacharminus}{\kern0pt}\ \isanewline
\ \ \isacommand{assume}\isamarkupfalse%
\ assms{\isacharcolon}{\kern0pt}\ {\isachardoublequoteopen}is{\isacharunderscore}{\kern0pt}P{\isacharunderscore}{\kern0pt}auto{\isacharparenleft}{\kern0pt}{\isasympi}{\isacharparenright}{\kern0pt}{\isachardoublequoteclose}\ \isanewline
\ \ \isacommand{have}\isamarkupfalse%
\ {\isachardoublequoteopen}Pn{\isacharunderscore}{\kern0pt}auto{\isacharparenleft}{\kern0pt}{\isasympi}{\isacharparenright}{\kern0pt}\ O\ Pn{\isacharunderscore}{\kern0pt}auto{\isacharparenleft}{\kern0pt}converse{\isacharparenleft}{\kern0pt}{\isasympi}{\isacharparenright}{\kern0pt}{\isacharparenright}{\kern0pt}\ {\isacharequal}{\kern0pt}\ Pn{\isacharunderscore}{\kern0pt}auto{\isacharparenleft}{\kern0pt}{\isasympi}\ O\ converse{\isacharparenleft}{\kern0pt}{\isasympi}{\isacharparenright}{\kern0pt}{\isacharparenright}{\kern0pt}{\isachardoublequoteclose}\ \isanewline
\ \ \ \ \isacommand{apply}\isamarkupfalse%
{\isacharparenleft}{\kern0pt}subst\ Pn{\isacharunderscore}{\kern0pt}auto{\isacharunderscore}{\kern0pt}comp{\isacharparenright}{\kern0pt}\isanewline
\ \ \ \ \isacommand{using}\isamarkupfalse%
\ assms\ P{\isacharunderscore}{\kern0pt}auto{\isacharunderscore}{\kern0pt}converse\ \isanewline
\ \ \ \ \isacommand{by}\isamarkupfalse%
\ auto\isanewline
\ \ \isacommand{also}\isamarkupfalse%
\ \isacommand{have}\isamarkupfalse%
\ {\isachardoublequoteopen}{\isachardot}{\kern0pt}{\isachardot}{\kern0pt}{\isachardot}{\kern0pt}\ {\isacharequal}{\kern0pt}\ Pn{\isacharunderscore}{\kern0pt}auto{\isacharparenleft}{\kern0pt}id{\isacharparenleft}{\kern0pt}P{\isacharparenright}{\kern0pt}{\isacharparenright}{\kern0pt}{\isachardoublequoteclose}\ \isanewline
\ \ \ \ \isacommand{apply}\isamarkupfalse%
{\isacharparenleft}{\kern0pt}subst\ right{\isacharunderscore}{\kern0pt}comp{\isacharunderscore}{\kern0pt}inverse{\isacharparenright}{\kern0pt}\isanewline
\ \ \ \ \isacommand{using}\isamarkupfalse%
\ assms\ is{\isacharunderscore}{\kern0pt}P{\isacharunderscore}{\kern0pt}auto{\isacharunderscore}{\kern0pt}def\ bij{\isacharunderscore}{\kern0pt}is{\isacharunderscore}{\kern0pt}surj\ \isanewline
\ \ \ \ \isacommand{by}\isamarkupfalse%
\ auto\isanewline
\ \ \isacommand{also}\isamarkupfalse%
\ \isacommand{have}\isamarkupfalse%
\ {\isachardoublequoteopen}{\isachardot}{\kern0pt}{\isachardot}{\kern0pt}{\isachardot}{\kern0pt}\ {\isacharequal}{\kern0pt}\ id{\isacharparenleft}{\kern0pt}P{\isacharunderscore}{\kern0pt}names{\isacharparenright}{\kern0pt}{\isachardoublequoteclose}\isanewline
\ \ \ \ \isacommand{using}\isamarkupfalse%
\ Pn{\isacharunderscore}{\kern0pt}auto{\isacharunderscore}{\kern0pt}id\ \isanewline
\ \ \ \ \isacommand{by}\isamarkupfalse%
\ auto\isanewline
\ \ \isacommand{finally}\isamarkupfalse%
\ \isacommand{have}\isamarkupfalse%
\ eq{\isacharcolon}{\kern0pt}\ {\isachardoublequoteopen}Pn{\isacharunderscore}{\kern0pt}auto{\isacharparenleft}{\kern0pt}{\isasympi}{\isacharparenright}{\kern0pt}\ O\ Pn{\isacharunderscore}{\kern0pt}auto{\isacharparenleft}{\kern0pt}converse{\isacharparenleft}{\kern0pt}{\isasympi}{\isacharparenright}{\kern0pt}{\isacharparenright}{\kern0pt}\ {\isacharequal}{\kern0pt}\ id{\isacharparenleft}{\kern0pt}P{\isacharunderscore}{\kern0pt}names{\isacharparenright}{\kern0pt}\ {\isachardoublequoteclose}\ \isacommand{by}\isamarkupfalse%
\ simp\isanewline
\isanewline
\ \ \isacommand{have}\isamarkupfalse%
\ {\isachardoublequoteopen}Pn{\isacharunderscore}{\kern0pt}auto{\isacharparenleft}{\kern0pt}converse{\isacharparenleft}{\kern0pt}{\isasympi}{\isacharparenright}{\kern0pt}{\isacharparenright}{\kern0pt}\ {\isacharequal}{\kern0pt}\ id{\isacharparenleft}{\kern0pt}P{\isacharunderscore}{\kern0pt}names{\isacharparenright}{\kern0pt}\ O\ Pn{\isacharunderscore}{\kern0pt}auto{\isacharparenleft}{\kern0pt}converse{\isacharparenleft}{\kern0pt}{\isasympi}{\isacharparenright}{\kern0pt}{\isacharparenright}{\kern0pt}{\isachardoublequoteclose}\ \isanewline
\ \ \ \ \isacommand{apply}\isamarkupfalse%
{\isacharparenleft}{\kern0pt}rule\ eq{\isacharunderscore}{\kern0pt}flip{\isacharcomma}{\kern0pt}\ rule{\isacharunderscore}{\kern0pt}tac\ A{\isacharequal}{\kern0pt}P{\isacharunderscore}{\kern0pt}names\ \isakeyword{in}\ left{\isacharunderscore}{\kern0pt}comp{\isacharunderscore}{\kern0pt}id{\isacharparenright}{\kern0pt}\isanewline
\ \ \ \ \isacommand{using}\isamarkupfalse%
\ Pn{\isacharunderscore}{\kern0pt}auto{\isacharunderscore}{\kern0pt}type\ assms\ P{\isacharunderscore}{\kern0pt}auto{\isacharunderscore}{\kern0pt}converse\ Pi{\isacharunderscore}{\kern0pt}def\ \isanewline
\ \ \ \ \isacommand{by}\isamarkupfalse%
\ force\isanewline
\ \ \isacommand{also}\isamarkupfalse%
\ \isacommand{have}\isamarkupfalse%
\ {\isachardoublequoteopen}{\isachardot}{\kern0pt}{\isachardot}{\kern0pt}{\isachardot}{\kern0pt}\ {\isacharequal}{\kern0pt}\ {\isacharparenleft}{\kern0pt}converse{\isacharparenleft}{\kern0pt}Pn{\isacharunderscore}{\kern0pt}auto{\isacharparenleft}{\kern0pt}{\isasympi}{\isacharparenright}{\kern0pt}{\isacharparenright}{\kern0pt}\ O\ Pn{\isacharunderscore}{\kern0pt}auto{\isacharparenleft}{\kern0pt}{\isasympi}{\isacharparenright}{\kern0pt}{\isacharparenright}{\kern0pt}\ O\ Pn{\isacharunderscore}{\kern0pt}auto{\isacharparenleft}{\kern0pt}converse{\isacharparenleft}{\kern0pt}{\isasympi}{\isacharparenright}{\kern0pt}{\isacharparenright}{\kern0pt}{\isachardoublequoteclose}\ \isanewline
\ \ \ \ \isacommand{apply}\isamarkupfalse%
{\isacharparenleft}{\kern0pt}rule\ eq{\isacharunderscore}{\kern0pt}flip{\isacharparenright}{\kern0pt}\isanewline
\ \ \ \ \isacommand{apply}\isamarkupfalse%
{\isacharparenleft}{\kern0pt}subst\ left{\isacharunderscore}{\kern0pt}comp{\isacharunderscore}{\kern0pt}inverse{\isacharparenright}{\kern0pt}\isanewline
\ \ \ \ \isacommand{using}\isamarkupfalse%
\ Pn{\isacharunderscore}{\kern0pt}auto{\isacharunderscore}{\kern0pt}bij\ assms\ bij{\isacharunderscore}{\kern0pt}is{\isacharunderscore}{\kern0pt}inj\ \isanewline
\ \ \ \ \isacommand{by}\isamarkupfalse%
\ auto\isanewline
\ \ \isacommand{also}\isamarkupfalse%
\ \isacommand{have}\isamarkupfalse%
\ {\isachardoublequoteopen}{\isachardot}{\kern0pt}{\isachardot}{\kern0pt}{\isachardot}{\kern0pt}\ {\isacharequal}{\kern0pt}\ converse{\isacharparenleft}{\kern0pt}Pn{\isacharunderscore}{\kern0pt}auto{\isacharparenleft}{\kern0pt}{\isasympi}{\isacharparenright}{\kern0pt}{\isacharparenright}{\kern0pt}\ O\ {\isacharparenleft}{\kern0pt}Pn{\isacharunderscore}{\kern0pt}auto{\isacharparenleft}{\kern0pt}{\isasympi}{\isacharparenright}{\kern0pt}\ O\ Pn{\isacharunderscore}{\kern0pt}auto{\isacharparenleft}{\kern0pt}converse{\isacharparenleft}{\kern0pt}{\isasympi}{\isacharparenright}{\kern0pt}{\isacharparenright}{\kern0pt}{\isacharparenright}{\kern0pt}{\isachardoublequoteclose}\ \isanewline
\ \ \ \ \isacommand{using}\isamarkupfalse%
\ comp{\isacharunderscore}{\kern0pt}assoc\ \isacommand{by}\isamarkupfalse%
\ auto\isanewline
\ \ \isacommand{also}\isamarkupfalse%
\ \isacommand{have}\isamarkupfalse%
\ {\isachardoublequoteopen}{\isachardot}{\kern0pt}{\isachardot}{\kern0pt}{\isachardot}{\kern0pt}\ {\isacharequal}{\kern0pt}\ converse{\isacharparenleft}{\kern0pt}Pn{\isacharunderscore}{\kern0pt}auto{\isacharparenleft}{\kern0pt}{\isasympi}{\isacharparenright}{\kern0pt}{\isacharparenright}{\kern0pt}\ O\ Pn{\isacharunderscore}{\kern0pt}auto{\isacharparenleft}{\kern0pt}{\isasympi}\ O\ converse{\isacharparenleft}{\kern0pt}{\isasympi}{\isacharparenright}{\kern0pt}{\isacharparenright}{\kern0pt}{\isachardoublequoteclose}\ \isanewline
\ \ \ \ \isacommand{apply}\isamarkupfalse%
{\isacharparenleft}{\kern0pt}subst\ Pn{\isacharunderscore}{\kern0pt}auto{\isacharunderscore}{\kern0pt}comp{\isacharparenright}{\kern0pt}\isanewline
\ \ \ \ \isacommand{using}\isamarkupfalse%
\ assms\ P{\isacharunderscore}{\kern0pt}auto{\isacharunderscore}{\kern0pt}converse\ \ \ \isanewline
\ \ \ \ \isacommand{by}\isamarkupfalse%
\ auto\ \isanewline
\ \ \isacommand{also}\isamarkupfalse%
\ \isacommand{have}\isamarkupfalse%
\ {\isachardoublequoteopen}{\isachardot}{\kern0pt}{\isachardot}{\kern0pt}{\isachardot}{\kern0pt}\ {\isacharequal}{\kern0pt}\ converse{\isacharparenleft}{\kern0pt}Pn{\isacharunderscore}{\kern0pt}auto{\isacharparenleft}{\kern0pt}{\isasympi}{\isacharparenright}{\kern0pt}{\isacharparenright}{\kern0pt}\ O\ Pn{\isacharunderscore}{\kern0pt}auto{\isacharparenleft}{\kern0pt}id{\isacharparenleft}{\kern0pt}P{\isacharparenright}{\kern0pt}{\isacharparenright}{\kern0pt}{\isachardoublequoteclose}\ \isanewline
\ \ \ \ \isacommand{apply}\isamarkupfalse%
{\isacharparenleft}{\kern0pt}subst\ right{\isacharunderscore}{\kern0pt}comp{\isacharunderscore}{\kern0pt}inverse{\isacharparenright}{\kern0pt}\isanewline
\ \ \ \ \isacommand{using}\isamarkupfalse%
\ assms\ is{\isacharunderscore}{\kern0pt}P{\isacharunderscore}{\kern0pt}auto{\isacharunderscore}{\kern0pt}def\ bij{\isacharunderscore}{\kern0pt}is{\isacharunderscore}{\kern0pt}surj\ \isanewline
\ \ \ \ \isacommand{by}\isamarkupfalse%
\ auto\isanewline
\ \ \isacommand{also}\isamarkupfalse%
\ \isacommand{have}\isamarkupfalse%
\ {\isachardoublequoteopen}{\isachardot}{\kern0pt}{\isachardot}{\kern0pt}{\isachardot}{\kern0pt}\ {\isacharequal}{\kern0pt}\ converse{\isacharparenleft}{\kern0pt}Pn{\isacharunderscore}{\kern0pt}auto{\isacharparenleft}{\kern0pt}{\isasympi}{\isacharparenright}{\kern0pt}{\isacharparenright}{\kern0pt}\ O\ id{\isacharparenleft}{\kern0pt}P{\isacharunderscore}{\kern0pt}names{\isacharparenright}{\kern0pt}{\isachardoublequoteclose}\ \isanewline
\ \ \ \ \isacommand{using}\isamarkupfalse%
\ Pn{\isacharunderscore}{\kern0pt}auto{\isacharunderscore}{\kern0pt}id\ \isanewline
\ \ \ \ \isacommand{by}\isamarkupfalse%
\ auto\ \isanewline
\ \ \isacommand{also}\isamarkupfalse%
\ \isacommand{have}\isamarkupfalse%
\ {\isachardoublequoteopen}{\isachardot}{\kern0pt}{\isachardot}{\kern0pt}{\isachardot}{\kern0pt}\ {\isacharequal}{\kern0pt}\ converse{\isacharparenleft}{\kern0pt}Pn{\isacharunderscore}{\kern0pt}auto{\isacharparenleft}{\kern0pt}{\isasympi}{\isacharparenright}{\kern0pt}{\isacharparenright}{\kern0pt}{\isachardoublequoteclose}\ \isanewline
\ \ \ \ \isacommand{apply}\isamarkupfalse%
{\isacharparenleft}{\kern0pt}rule\ right{\isacharunderscore}{\kern0pt}comp{\isacharunderscore}{\kern0pt}id{\isacharcomma}{\kern0pt}\ rule\ converse{\isacharunderscore}{\kern0pt}type{\isacharparenright}{\kern0pt}\isanewline
\ \ \ \ \isacommand{using}\isamarkupfalse%
\ Pn{\isacharunderscore}{\kern0pt}auto{\isacharunderscore}{\kern0pt}type\ Pi{\isacharunderscore}{\kern0pt}def\ assms\ \isanewline
\ \ \ \ \isacommand{by}\isamarkupfalse%
\ auto\isanewline
\ \ \isacommand{finally}\isamarkupfalse%
\ \isacommand{show}\isamarkupfalse%
\ {\isachardoublequoteopen}Pn{\isacharunderscore}{\kern0pt}auto{\isacharparenleft}{\kern0pt}converse{\isacharparenleft}{\kern0pt}{\isasympi}{\isacharparenright}{\kern0pt}{\isacharparenright}{\kern0pt}\ {\isacharequal}{\kern0pt}\ converse{\isacharparenleft}{\kern0pt}Pn{\isacharunderscore}{\kern0pt}auto{\isacharparenleft}{\kern0pt}{\isasympi}{\isacharparenright}{\kern0pt}{\isacharparenright}{\kern0pt}\ {\isachardoublequoteclose}\ \isacommand{by}\isamarkupfalse%
\ simp\isanewline
\isacommand{qed}\isamarkupfalse%
%
\endisatagproof
{\isafoldproof}%
%
\isadelimproof
\isanewline
%
\endisadelimproof
\isanewline
\isacommand{lemma}\isamarkupfalse%
\ Pn{\isacharunderscore}{\kern0pt}auto{\isacharunderscore}{\kern0pt}preserves{\isacharunderscore}{\kern0pt}P{\isacharunderscore}{\kern0pt}rank\ {\isacharcolon}{\kern0pt}\ \isanewline
\ \ {\isachardoublequoteopen}is{\isacharunderscore}{\kern0pt}P{\isacharunderscore}{\kern0pt}auto{\isacharparenleft}{\kern0pt}{\isasympi}{\isacharparenright}{\kern0pt}\ {\isasymLongrightarrow}\ x\ {\isasymin}\ P{\isacharunderscore}{\kern0pt}names\ {\isasymLongrightarrow}\ P{\isacharunderscore}{\kern0pt}rank{\isacharparenleft}{\kern0pt}x{\isacharparenright}{\kern0pt}\ {\isacharequal}{\kern0pt}\ P{\isacharunderscore}{\kern0pt}rank{\isacharparenleft}{\kern0pt}Pn{\isacharunderscore}{\kern0pt}auto{\isacharparenleft}{\kern0pt}{\isasympi}{\isacharparenright}{\kern0pt}{\isacharbackquote}{\kern0pt}x{\isacharparenright}{\kern0pt}{\isachardoublequoteclose}\ \isanewline
%
\isadelimproof
%
\endisadelimproof
%
\isatagproof
\isacommand{proof}\isamarkupfalse%
{\isacharminus}{\kern0pt}\isanewline
\ \ \isacommand{assume}\isamarkupfalse%
\ assms\ {\isacharcolon}{\kern0pt}\ {\isachardoublequoteopen}is{\isacharunderscore}{\kern0pt}P{\isacharunderscore}{\kern0pt}auto{\isacharparenleft}{\kern0pt}{\isasympi}{\isacharparenright}{\kern0pt}{\isachardoublequoteclose}\ {\isachardoublequoteopen}x\ {\isasymin}\ P{\isacharunderscore}{\kern0pt}names{\isachardoublequoteclose}\ \isanewline
\ \ \isacommand{have}\isamarkupfalse%
\ least{\isacharunderscore}{\kern0pt}iff\ {\isacharcolon}{\kern0pt}\ {\isachardoublequoteopen}{\isasymAnd}P\ Q{\isachardot}{\kern0pt}\ {\isasymforall}x{\isachardot}{\kern0pt}\ P{\isacharparenleft}{\kern0pt}x{\isacharparenright}{\kern0pt}\ {\isasymlongleftrightarrow}\ Q{\isacharparenleft}{\kern0pt}x{\isacharparenright}{\kern0pt}\ {\isasymLongrightarrow}\ {\isacharparenleft}{\kern0pt}{\isasymmu}\ x{\isachardot}{\kern0pt}\ P{\isacharparenleft}{\kern0pt}x{\isacharparenright}{\kern0pt}{\isacharparenright}{\kern0pt}\ {\isacharequal}{\kern0pt}\ {\isacharparenleft}{\kern0pt}{\isasymmu}\ x{\isachardot}{\kern0pt}\ Q{\isacharparenleft}{\kern0pt}x{\isacharparenright}{\kern0pt}{\isacharparenright}{\kern0pt}{\isachardoublequoteclose}\ \isacommand{by}\isamarkupfalse%
\ auto\isanewline
\ \ \isacommand{have}\isamarkupfalse%
\ iff{\isacharunderscore}{\kern0pt}lemma\ {\isacharcolon}{\kern0pt}\ {\isachardoublequoteopen}{\isasymAnd}P\ Q\ R{\isachardot}{\kern0pt}\ {\isacharparenleft}{\kern0pt}P\ {\isasymLongrightarrow}\ Q\ {\isasymlongleftrightarrow}\ R{\isacharparenright}{\kern0pt}\ {\isasymLongrightarrow}\ P\ {\isasymand}\ Q\ {\isasymlongleftrightarrow}\ P\ {\isasymand}\ R{\isachardoublequoteclose}\ \isacommand{by}\isamarkupfalse%
\ auto\ \isanewline
\ \ \isacommand{show}\isamarkupfalse%
\ {\isachardoublequoteopen}P{\isacharunderscore}{\kern0pt}rank{\isacharparenleft}{\kern0pt}x{\isacharparenright}{\kern0pt}\ {\isacharequal}{\kern0pt}\ P{\isacharunderscore}{\kern0pt}rank{\isacharparenleft}{\kern0pt}Pn{\isacharunderscore}{\kern0pt}auto{\isacharparenleft}{\kern0pt}{\isasympi}{\isacharparenright}{\kern0pt}\ {\isacharbackquote}{\kern0pt}\ x{\isacharparenright}{\kern0pt}{\isachardoublequoteclose}\ \isanewline
\ \ \ \ \isacommand{unfolding}\isamarkupfalse%
\ P{\isacharunderscore}{\kern0pt}rank{\isacharunderscore}{\kern0pt}def\ \isanewline
\ \ \ \ \isacommand{apply}\isamarkupfalse%
\ {\isacharparenleft}{\kern0pt}rule{\isacharunderscore}{\kern0pt}tac\ least{\isacharunderscore}{\kern0pt}iff{\isacharsemicolon}{\kern0pt}\ rule\ allI{\isacharsemicolon}{\kern0pt}\ rule{\isacharunderscore}{\kern0pt}tac\ iff{\isacharunderscore}{\kern0pt}lemma{\isacharparenright}{\kern0pt}\isanewline
\ \ \ \ \isacommand{apply}\isamarkupfalse%
\ {\isacharparenleft}{\kern0pt}rule{\isacharunderscore}{\kern0pt}tac\ Pn{\isacharunderscore}{\kern0pt}auto{\isacharunderscore}{\kern0pt}value{\isacharunderscore}{\kern0pt}in{\isacharunderscore}{\kern0pt}P{\isacharunderscore}{\kern0pt}set{\isacharparenright}{\kern0pt}\ \isacommand{using}\isamarkupfalse%
\ assms\ \isacommand{by}\isamarkupfalse%
\ auto\isanewline
\isacommand{qed}\isamarkupfalse%
%
\endisatagproof
{\isafoldproof}%
%
\isadelimproof
\isanewline
%
\endisadelimproof
\isanewline
\isacommand{lemma}\isamarkupfalse%
\ check{\isacharunderscore}{\kern0pt}fixpoint\ {\isacharcolon}{\kern0pt}\ \isanewline
\ \ {\isachardoublequoteopen}is{\isacharunderscore}{\kern0pt}P{\isacharunderscore}{\kern0pt}auto{\isacharparenleft}{\kern0pt}{\isasympi}{\isacharparenright}{\kern0pt}\ {\isasymLongrightarrow}\ x\ {\isasymin}\ M\ {\isasymLongrightarrow}\ Pn{\isacharunderscore}{\kern0pt}auto{\isacharparenleft}{\kern0pt}{\isasympi}{\isacharparenright}{\kern0pt}{\isacharbackquote}{\kern0pt}{\isacharparenleft}{\kern0pt}check{\isacharparenleft}{\kern0pt}x{\isacharparenright}{\kern0pt}{\isacharparenright}{\kern0pt}\ {\isacharequal}{\kern0pt}\ check{\isacharparenleft}{\kern0pt}x{\isacharparenright}{\kern0pt}{\isachardoublequoteclose}\isanewline
%
\isadelimproof
%
\endisadelimproof
%
\isatagproof
\isacommand{proof}\isamarkupfalse%
\ {\isacharminus}{\kern0pt}\ \isanewline
\ \ \isacommand{assume}\isamarkupfalse%
\ piauto{\isacharcolon}{\kern0pt}\ {\isachardoublequoteopen}is{\isacharunderscore}{\kern0pt}P{\isacharunderscore}{\kern0pt}auto{\isacharparenleft}{\kern0pt}{\isasympi}{\isacharparenright}{\kern0pt}{\isachardoublequoteclose}\ \isanewline
\ \ \isacommand{have}\isamarkupfalse%
\ mainlemma{\isacharcolon}{\kern0pt}\ {\isachardoublequoteopen}{\isasymAnd}v{\isachardot}{\kern0pt}\ v\ {\isasymin}\ P{\isacharunderscore}{\kern0pt}names\ {\isasymLongrightarrow}\ {\isasymforall}x\ {\isasymin}\ M{\isachardot}{\kern0pt}\ {\isacharparenleft}{\kern0pt}check{\isacharparenleft}{\kern0pt}x{\isacharparenright}{\kern0pt}\ {\isacharequal}{\kern0pt}\ v\ {\isasymlongrightarrow}\ Pn{\isacharunderscore}{\kern0pt}auto{\isacharparenleft}{\kern0pt}{\isasympi}{\isacharparenright}{\kern0pt}{\isacharbackquote}{\kern0pt}{\isacharparenleft}{\kern0pt}v{\isacharparenright}{\kern0pt}\ {\isacharequal}{\kern0pt}\ v{\isacharparenright}{\kern0pt}{\isachardoublequoteclose}\ \isanewline
\ \ \ \ \isacommand{thm}\isamarkupfalse%
\ P{\isacharunderscore}{\kern0pt}name{\isacharunderscore}{\kern0pt}induct\isanewline
\ \ \ \ \isacommand{apply}\isamarkupfalse%
\ {\isacharparenleft}{\kern0pt}rule{\isacharunderscore}{\kern0pt}tac\ Q{\isacharequal}{\kern0pt}{\isachardoublequoteopen}{\isasymlambda}v{\isachardot}{\kern0pt}\ {\isasymforall}x\ {\isasymin}\ M{\isachardot}{\kern0pt}\ {\isacharparenleft}{\kern0pt}check{\isacharparenleft}{\kern0pt}x{\isacharparenright}{\kern0pt}\ {\isacharequal}{\kern0pt}\ v\ {\isasymlongrightarrow}\ Pn{\isacharunderscore}{\kern0pt}auto{\isacharparenleft}{\kern0pt}{\isasympi}{\isacharparenright}{\kern0pt}{\isacharbackquote}{\kern0pt}{\isacharparenleft}{\kern0pt}v{\isacharparenright}{\kern0pt}\ {\isacharequal}{\kern0pt}\ v{\isacharparenright}{\kern0pt}{\isachardoublequoteclose}\ \isakeyword{in}\ P{\isacharunderscore}{\kern0pt}name{\isacharunderscore}{\kern0pt}induct{\isacharparenright}{\kern0pt}\ \isanewline
\ \ \ \ \isacommand{apply}\isamarkupfalse%
\ {\isacharparenleft}{\kern0pt}auto{\isacharparenright}{\kern0pt}\isanewline
\ \ \isacommand{proof}\isamarkupfalse%
\ {\isacharminus}{\kern0pt}\isanewline
\ \ \ \ \isacommand{fix}\isamarkupfalse%
\ x\ \isacommand{assume}\isamarkupfalse%
\ assms\ {\isacharcolon}{\kern0pt}\ \isanewline
\ \ \ \ \ \ \ \ \ \ \ {\isachardoublequoteopen}{\isasymforall}y{\isachardot}{\kern0pt}\ {\isacharparenleft}{\kern0pt}{\isasymexists}p{\isachardot}{\kern0pt}\ {\isasymlangle}y{\isacharcomma}{\kern0pt}\ p{\isasymrangle}\ {\isasymin}\ check{\isacharparenleft}{\kern0pt}x{\isacharparenright}{\kern0pt}{\isacharparenright}{\kern0pt}\ {\isasymlongrightarrow}\ {\isacharparenleft}{\kern0pt}{\isasymexists}x{\isasymin}M{\isachardot}{\kern0pt}\ check{\isacharparenleft}{\kern0pt}x{\isacharparenright}{\kern0pt}\ {\isacharequal}{\kern0pt}\ y{\isacharparenright}{\kern0pt}\ {\isasymlongrightarrow}\ Pn{\isacharunderscore}{\kern0pt}auto{\isacharparenleft}{\kern0pt}{\isasympi}{\isacharparenright}{\kern0pt}\ {\isacharbackquote}{\kern0pt}\ y\ {\isacharequal}{\kern0pt}\ y{\isachardoublequoteclose}\isanewline
\ \ \ \ \ \ \ \ \ \ \ {\isachardoublequoteopen}x\ {\isasymin}\ M{\isachardoublequoteclose}\isanewline
\ \ \ \ \isacommand{have}\isamarkupfalse%
\ {\isachardoublequoteopen}check{\isacharparenleft}{\kern0pt}x{\isacharparenright}{\kern0pt}\ {\isasymin}\ Pow{\isacharparenleft}{\kern0pt}P{\isacharunderscore}{\kern0pt}set{\isacharparenleft}{\kern0pt}P{\isacharunderscore}{\kern0pt}rank{\isacharparenleft}{\kern0pt}check{\isacharparenleft}{\kern0pt}x{\isacharparenright}{\kern0pt}{\isacharparenright}{\kern0pt}{\isacharparenright}{\kern0pt}\ {\isasymtimes}\ P{\isacharparenright}{\kern0pt}\ {\isasyminter}\ M{\isachardoublequoteclose}\ \isanewline
\ \ \ \ \ \ \isacommand{using}\isamarkupfalse%
\ P{\isacharunderscore}{\kern0pt}names{\isacharunderscore}{\kern0pt}in\ check{\isacharunderscore}{\kern0pt}P{\isacharunderscore}{\kern0pt}name\ assms\ \isacommand{by}\isamarkupfalse%
\ auto\ \isanewline
\ \ \ \ \isacommand{then}\isamarkupfalse%
\ \isacommand{have}\isamarkupfalse%
\ p{\isadigit{1}}\ {\isacharcolon}{\kern0pt}\ {\isachardoublequoteopen}check{\isacharparenleft}{\kern0pt}x{\isacharparenright}{\kern0pt}\ {\isasymsubseteq}\ P{\isacharunderscore}{\kern0pt}set{\isacharparenleft}{\kern0pt}P{\isacharunderscore}{\kern0pt}rank{\isacharparenleft}{\kern0pt}check{\isacharparenleft}{\kern0pt}x{\isacharparenright}{\kern0pt}{\isacharparenright}{\kern0pt}{\isacharparenright}{\kern0pt}\ {\isasymtimes}\ P{\isachardoublequoteclose}\ \isacommand{by}\isamarkupfalse%
\ auto\ \isanewline
\ \ \ \ \isacommand{have}\isamarkupfalse%
\ p{\isadigit{2}}\ {\isacharcolon}{\kern0pt}\ {\isachardoublequoteopen}check{\isacharparenleft}{\kern0pt}x{\isacharparenright}{\kern0pt}\ {\isacharequal}{\kern0pt}\ {\isacharbraceleft}{\kern0pt}\ {\isacharless}{\kern0pt}check{\isacharparenleft}{\kern0pt}y{\isacharparenright}{\kern0pt}{\isacharcomma}{\kern0pt}\ one{\isachargreater}{\kern0pt}\ {\isachardot}{\kern0pt}\ y\ {\isasymin}\ x\ {\isacharbraceright}{\kern0pt}{\isachardoublequoteclose}\ \isanewline
\ \ \ \ \ \ \isacommand{by}\isamarkupfalse%
\ {\isacharparenleft}{\kern0pt}rule{\isacharunderscore}{\kern0pt}tac\ def{\isacharunderscore}{\kern0pt}check{\isacharparenright}{\kern0pt}\isanewline
\ \ \ \ \isacommand{have}\isamarkupfalse%
\ {\isachardoublequoteopen}Pn{\isacharunderscore}{\kern0pt}auto{\isacharparenleft}{\kern0pt}{\isasympi}{\isacharparenright}{\kern0pt}\ {\isacharbackquote}{\kern0pt}\ check{\isacharparenleft}{\kern0pt}x{\isacharparenright}{\kern0pt}\ {\isacharequal}{\kern0pt}\ {\isacharbraceleft}{\kern0pt}\ {\isacharless}{\kern0pt}Pn{\isacharunderscore}{\kern0pt}auto{\isacharparenleft}{\kern0pt}{\isasympi}{\isacharparenright}{\kern0pt}{\isacharbackquote}{\kern0pt}y{\isacharcomma}{\kern0pt}\ {\isasympi}{\isacharbackquote}{\kern0pt}p{\isachargreater}{\kern0pt}\ {\isachardot}{\kern0pt}\ {\isacharless}{\kern0pt}y{\isacharcomma}{\kern0pt}\ p{\isachargreater}{\kern0pt}\ {\isasymin}\ check{\isacharparenleft}{\kern0pt}x{\isacharparenright}{\kern0pt}\ {\isacharbraceright}{\kern0pt}{\isachardoublequoteclose}\isanewline
\ \ \ \ \ \ \isacommand{using}\isamarkupfalse%
\ Pn{\isacharunderscore}{\kern0pt}auto\ check{\isacharunderscore}{\kern0pt}P{\isacharunderscore}{\kern0pt}name\ assms\ \isacommand{by}\isamarkupfalse%
\ auto\ \isanewline
\ \ \ \ \isacommand{also}\isamarkupfalse%
\ \isacommand{have}\isamarkupfalse%
\ {\isachardoublequoteopen}{\isachardot}{\kern0pt}{\isachardot}{\kern0pt}{\isachardot}{\kern0pt}\ {\isacharequal}{\kern0pt}\ \ {\isacharbraceleft}{\kern0pt}\ {\isacharless}{\kern0pt}Pn{\isacharunderscore}{\kern0pt}auto{\isacharparenleft}{\kern0pt}{\isasympi}{\isacharparenright}{\kern0pt}{\isacharbackquote}{\kern0pt}y{\isacharcomma}{\kern0pt}\ {\isasympi}{\isacharbackquote}{\kern0pt}one{\isachargreater}{\kern0pt}\ {\isachardot}{\kern0pt}\ {\isacharless}{\kern0pt}y{\isacharcomma}{\kern0pt}\ p{\isachargreater}{\kern0pt}\ {\isasymin}\ check{\isacharparenleft}{\kern0pt}x{\isacharparenright}{\kern0pt}\ {\isacharbraceright}{\kern0pt}{\isachardoublequoteclose}\isanewline
\ \ \ \ \ \ \isacommand{apply}\isamarkupfalse%
\ {\isacharparenleft}{\kern0pt}rule{\isacharunderscore}{\kern0pt}tac\ pair{\isacharunderscore}{\kern0pt}rel{\isacharunderscore}{\kern0pt}eq{\isacharparenright}{\kern0pt}\isanewline
\ \ \ \ \ \ \isacommand{using}\isamarkupfalse%
\ p{\isadigit{1}}\ \isacommand{apply}\isamarkupfalse%
\ auto\ \isanewline
\ \ \ \ \ \ \isacommand{using}\isamarkupfalse%
\ check{\isacharunderscore}{\kern0pt}P{\isacharunderscore}{\kern0pt}name\ assms\ relation{\isacharunderscore}{\kern0pt}P{\isacharunderscore}{\kern0pt}name\ \isacommand{apply}\isamarkupfalse%
\ simp\isanewline
\ \ \ \ \isacommand{proof}\isamarkupfalse%
\ {\isacharminus}{\kern0pt}\ \isanewline
\ \ \ \ \ \ \isacommand{fix}\isamarkupfalse%
\ y\ p\ \isacommand{assume}\isamarkupfalse%
\ {\isachardoublequoteopen}{\isacharless}{\kern0pt}y{\isacharcomma}{\kern0pt}\ p{\isachargreater}{\kern0pt}\ {\isasymin}\ check{\isacharparenleft}{\kern0pt}x{\isacharparenright}{\kern0pt}{\isachardoublequoteclose}\ \isanewline
\ \ \ \ \ \ \isacommand{then}\isamarkupfalse%
\ \isacommand{have}\isamarkupfalse%
\ {\isachardoublequoteopen}p\ {\isacharequal}{\kern0pt}\ one{\isachardoublequoteclose}\ \isacommand{using}\isamarkupfalse%
\ p{\isadigit{2}}\ \isacommand{by}\isamarkupfalse%
\ auto\ \isanewline
\ \ \ \ \ \ \isacommand{then}\isamarkupfalse%
\ \isacommand{show}\isamarkupfalse%
\ {\isachardoublequoteopen}{\isasympi}{\isacharbackquote}{\kern0pt}p\ {\isacharequal}{\kern0pt}\ {\isasympi}{\isacharbackquote}{\kern0pt}one{\isachardoublequoteclose}\ \isacommand{by}\isamarkupfalse%
\ auto\ \isanewline
\ \ \ \ \isacommand{qed}\isamarkupfalse%
\ \isanewline
\ \ \ \ \isacommand{also}\isamarkupfalse%
\ \isacommand{have}\isamarkupfalse%
\ {\isachardoublequoteopen}{\isachardot}{\kern0pt}{\isachardot}{\kern0pt}{\isachardot}{\kern0pt}\ {\isacharequal}{\kern0pt}\ \ {\isacharbraceleft}{\kern0pt}\ {\isacharless}{\kern0pt}Pn{\isacharunderscore}{\kern0pt}auto{\isacharparenleft}{\kern0pt}{\isasympi}{\isacharparenright}{\kern0pt}{\isacharbackquote}{\kern0pt}y{\isacharcomma}{\kern0pt}\ one{\isachargreater}{\kern0pt}\ {\isachardot}{\kern0pt}\ {\isacharless}{\kern0pt}y{\isacharcomma}{\kern0pt}\ p{\isachargreater}{\kern0pt}\ {\isasymin}\ check{\isacharparenleft}{\kern0pt}x{\isacharparenright}{\kern0pt}\ {\isacharbraceright}{\kern0pt}{\isachardoublequoteclose}\ \isanewline
\ \ \ \ \ \ \isacommand{using}\isamarkupfalse%
\ piauto\ P{\isacharunderscore}{\kern0pt}auto{\isacharunderscore}{\kern0pt}preserves{\isacharunderscore}{\kern0pt}one\ \isacommand{by}\isamarkupfalse%
\ auto\ \isanewline
\ \ \ \ \isacommand{also}\isamarkupfalse%
\ \isacommand{have}\isamarkupfalse%
\ {\isachardoublequoteopen}{\isachardot}{\kern0pt}{\isachardot}{\kern0pt}{\isachardot}{\kern0pt}\ {\isacharequal}{\kern0pt}\ \ {\isacharbraceleft}{\kern0pt}\ {\isacharless}{\kern0pt}y{\isacharcomma}{\kern0pt}\ one{\isachargreater}{\kern0pt}\ {\isachardot}{\kern0pt}\ \ {\isacharless}{\kern0pt}y{\isacharcomma}{\kern0pt}\ p{\isachargreater}{\kern0pt}\ {\isasymin}\ check{\isacharparenleft}{\kern0pt}x{\isacharparenright}{\kern0pt}\ {\isacharbraceright}{\kern0pt}{\isachardoublequoteclose}\isanewline
\ \ \ \ \ \ \isacommand{apply}\isamarkupfalse%
\ {\isacharparenleft}{\kern0pt}rule{\isacharunderscore}{\kern0pt}tac\ pair{\isacharunderscore}{\kern0pt}rel{\isacharunderscore}{\kern0pt}eq{\isacharparenright}{\kern0pt}\isanewline
\ \ \ \ \ \ \isacommand{using}\isamarkupfalse%
\ p{\isadigit{1}}\ \isacommand{apply}\isamarkupfalse%
\ auto\ \isanewline
\ \ \ \ \ \ \isacommand{using}\isamarkupfalse%
\ check{\isacharunderscore}{\kern0pt}P{\isacharunderscore}{\kern0pt}name\ assms\ relation{\isacharunderscore}{\kern0pt}P{\isacharunderscore}{\kern0pt}name\ \isacommand{apply}\isamarkupfalse%
\ simp\isanewline
\ \ \ \ \isacommand{proof}\isamarkupfalse%
\ {\isacharminus}{\kern0pt}\isanewline
\ \ \ \ \ \ \isacommand{fix}\isamarkupfalse%
\ y\ p\ \isacommand{assume}\isamarkupfalse%
\ assm{\isadigit{1}}{\isacharcolon}{\kern0pt}\ \ {\isachardoublequoteopen}{\isasymlangle}y{\isacharcomma}{\kern0pt}\ p{\isasymrangle}\ {\isasymin}\ check{\isacharparenleft}{\kern0pt}x{\isacharparenright}{\kern0pt}{\isachardoublequoteclose}\ \isanewline
\ \ \ \ \ \ \isacommand{then}\isamarkupfalse%
\ \isacommand{obtain}\isamarkupfalse%
\ yy\ \isakeyword{where}\ p{\isadigit{0}}{\isacharcolon}{\kern0pt}\ {\isachardoublequoteopen}yy\ {\isasymin}\ x{\isachardoublequoteclose}\ {\isachardoublequoteopen}{\isacharless}{\kern0pt}y{\isacharcomma}{\kern0pt}\ p{\isachargreater}{\kern0pt}\ {\isacharequal}{\kern0pt}\ {\isacharless}{\kern0pt}check{\isacharparenleft}{\kern0pt}yy{\isacharparenright}{\kern0pt}{\isacharcomma}{\kern0pt}\ one{\isachargreater}{\kern0pt}{\isachardoublequoteclose}\ \isacommand{using}\isamarkupfalse%
\ p{\isadigit{2}}\ \isacommand{by}\isamarkupfalse%
\ auto\ \isanewline
\ \ \ \ \ \ \isacommand{then}\isamarkupfalse%
\ \isacommand{have}\isamarkupfalse%
\ p{\isadigit{1}}{\isacharcolon}{\kern0pt}\ {\isachardoublequoteopen}y\ {\isacharequal}{\kern0pt}\ check{\isacharparenleft}{\kern0pt}yy{\isacharparenright}{\kern0pt}{\isachardoublequoteclose}\ \isacommand{by}\isamarkupfalse%
\ auto\ \isanewline
\ \ \ \ \ \ \isacommand{have}\isamarkupfalse%
\ p{\isadigit{2}}\ {\isacharcolon}{\kern0pt}\ {\isachardoublequoteopen}yy\ {\isasymin}\ M{\isachardoublequoteclose}\ \isacommand{using}\isamarkupfalse%
\ transM\ assms\ p{\isadigit{0}}\ \isacommand{by}\isamarkupfalse%
\ auto\ \isanewline
\ \ \ \ \ \ \isacommand{then}\isamarkupfalse%
\ \isacommand{show}\isamarkupfalse%
\ {\isachardoublequoteopen}Pn{\isacharunderscore}{\kern0pt}auto{\isacharparenleft}{\kern0pt}{\isasympi}{\isacharparenright}{\kern0pt}\ {\isacharbackquote}{\kern0pt}\ y\ {\isacharequal}{\kern0pt}\ y{\isachardoublequoteclose}\ \isacommand{using}\isamarkupfalse%
\ assms\ assm{\isadigit{1}}\ p{\isadigit{1}}\ p{\isadigit{2}}\ \isacommand{by}\isamarkupfalse%
\ auto\ \isanewline
\ \ \ \ \isacommand{qed}\isamarkupfalse%
\ \isanewline
\ \ \ \ \isacommand{also}\isamarkupfalse%
\ \isacommand{have}\isamarkupfalse%
{\isachardoublequoteopen}{\isachardot}{\kern0pt}{\isachardot}{\kern0pt}{\isachardot}{\kern0pt}\ {\isacharequal}{\kern0pt}\ {\isacharbraceleft}{\kern0pt}\ {\isacharless}{\kern0pt}check{\isacharparenleft}{\kern0pt}y{\isacharparenright}{\kern0pt}{\isacharcomma}{\kern0pt}\ one{\isachargreater}{\kern0pt}\ {\isachardot}{\kern0pt}\ y\ {\isasymin}\ x\ {\isacharbraceright}{\kern0pt}{\isachardoublequoteclose}\ \isanewline
\ \ \ \ \ \ \isacommand{apply}\isamarkupfalse%
\ {\isacharparenleft}{\kern0pt}rule{\isacharunderscore}{\kern0pt}tac\ RepFun{\isacharunderscore}{\kern0pt}eq{\isacharsemicolon}{\kern0pt}\ auto{\isacharparenright}{\kern0pt}\isanewline
\ \ \ \ \isacommand{proof}\isamarkupfalse%
\ {\isacharminus}{\kern0pt}\ \isanewline
\ \ \ \ \ \ \isacommand{fix}\isamarkupfalse%
\ v\ \isacommand{assume}\isamarkupfalse%
\ {\isachardoublequoteopen}v\ {\isasymin}\ check{\isacharparenleft}{\kern0pt}x{\isacharparenright}{\kern0pt}{\isachardoublequoteclose}\ \isanewline
\ \ \ \ \ \ \isacommand{then}\isamarkupfalse%
\ \isacommand{have}\isamarkupfalse%
\ {\isachardoublequoteopen}v\ {\isasymin}\ {\isacharbraceleft}{\kern0pt}\ {\isacharless}{\kern0pt}check{\isacharparenleft}{\kern0pt}y{\isacharparenright}{\kern0pt}{\isacharcomma}{\kern0pt}\ one{\isachargreater}{\kern0pt}\ {\isachardot}{\kern0pt}\ y\ {\isasymin}\ x\ {\isacharbraceright}{\kern0pt}{\isachardoublequoteclose}\ \isacommand{using}\isamarkupfalse%
\ p{\isadigit{2}}\ \isacommand{by}\isamarkupfalse%
\ auto\ \isanewline
\ \ \ \ \ \ \isacommand{then}\isamarkupfalse%
\ \isacommand{obtain}\isamarkupfalse%
\ y\ \isakeyword{where}\ yp\ {\isacharcolon}{\kern0pt}\ {\isachardoublequoteopen}y\ {\isasymin}\ x{\isachardoublequoteclose}\ {\isachardoublequoteopen}v\ {\isacharequal}{\kern0pt}\ {\isacharless}{\kern0pt}check{\isacharparenleft}{\kern0pt}y{\isacharparenright}{\kern0pt}{\isacharcomma}{\kern0pt}\ one{\isachargreater}{\kern0pt}{\isachardoublequoteclose}\ \isacommand{by}\isamarkupfalse%
\ auto\ \isanewline
\ \ \ \ \ \ \isacommand{then}\isamarkupfalse%
\ \isacommand{show}\isamarkupfalse%
\ {\isachardoublequoteopen}{\isasymexists}\ y\ {\isasymin}\ x{\isachardot}{\kern0pt}\ {\isacharparenleft}{\kern0pt}{\isasymlambda}{\isasymlangle}y{\isacharcomma}{\kern0pt}p{\isasymrangle}{\isachardot}{\kern0pt}\ {\isasymlangle}y{\isacharcomma}{\kern0pt}\ one{\isasymrangle}{\isacharparenright}{\kern0pt}{\isacharparenleft}{\kern0pt}v{\isacharparenright}{\kern0pt}\ {\isacharequal}{\kern0pt}\ {\isacharless}{\kern0pt}check{\isacharparenleft}{\kern0pt}y{\isacharparenright}{\kern0pt}{\isacharcomma}{\kern0pt}\ one{\isachargreater}{\kern0pt}{\isachardoublequoteclose}\ \isacommand{by}\isamarkupfalse%
\ auto\ \isanewline
\ \ \ \ \isacommand{next}\isamarkupfalse%
\ \isanewline
\ \ \ \ \ \ \isacommand{fix}\isamarkupfalse%
\ y\ \isacommand{assume}\isamarkupfalse%
\ {\isachardoublequoteopen}y\ {\isasymin}\ x{\isachardoublequoteclose}\ \isanewline
\ \ \ \ \ \ \isacommand{then}\isamarkupfalse%
\ \isacommand{show}\isamarkupfalse%
\ {\isachardoublequoteopen}{\isasymexists}v\ {\isasymin}\ check{\isacharparenleft}{\kern0pt}x{\isacharparenright}{\kern0pt}{\isachardot}{\kern0pt}\ {\isasymlangle}check{\isacharparenleft}{\kern0pt}y{\isacharparenright}{\kern0pt}{\isacharcomma}{\kern0pt}\ one{\isasymrangle}\ {\isacharequal}{\kern0pt}\ {\isacharparenleft}{\kern0pt}{\isasymlambda}{\isasymlangle}y{\isacharcomma}{\kern0pt}p{\isasymrangle}{\isachardot}{\kern0pt}\ {\isasymlangle}y{\isacharcomma}{\kern0pt}\ one{\isasymrangle}{\isacharparenright}{\kern0pt}{\isacharparenleft}{\kern0pt}v{\isacharparenright}{\kern0pt}{\isachardoublequoteclose}\isanewline
\ \ \ \ \ \ \ \ \isacommand{apply}\isamarkupfalse%
\ {\isacharparenleft}{\kern0pt}rule{\isacharunderscore}{\kern0pt}tac\ x{\isacharequal}{\kern0pt}{\isachardoublequoteopen}{\isacharless}{\kern0pt}check{\isacharparenleft}{\kern0pt}y{\isacharparenright}{\kern0pt}{\isacharcomma}{\kern0pt}\ one{\isachargreater}{\kern0pt}{\isachardoublequoteclose}\ \isakeyword{in}\ bexI{\isacharparenright}{\kern0pt}\isanewline
\ \ \ \ \ \ \ \ \isacommand{using}\isamarkupfalse%
\ p{\isadigit{2}}\ \isacommand{by}\isamarkupfalse%
\ auto\isanewline
\ \ \ \ \isacommand{qed}\isamarkupfalse%
\isanewline
\ \ \ \ \isacommand{also}\isamarkupfalse%
\ \isacommand{have}\isamarkupfalse%
\ {\isachardoublequoteopen}\ {\isachardot}{\kern0pt}{\isachardot}{\kern0pt}{\isachardot}{\kern0pt}\ {\isacharequal}{\kern0pt}\ check{\isacharparenleft}{\kern0pt}x{\isacharparenright}{\kern0pt}{\isachardoublequoteclose}\ \isacommand{using}\isamarkupfalse%
\ p{\isadigit{2}}\ \isacommand{by}\isamarkupfalse%
\ auto\ \isanewline
\ \ \ \ \isacommand{finally}\isamarkupfalse%
\ \isacommand{show}\isamarkupfalse%
\ {\isachardoublequoteopen}Pn{\isacharunderscore}{\kern0pt}auto{\isacharparenleft}{\kern0pt}{\isasympi}{\isacharparenright}{\kern0pt}\ {\isacharbackquote}{\kern0pt}\ check{\isacharparenleft}{\kern0pt}x{\isacharparenright}{\kern0pt}\ {\isacharequal}{\kern0pt}\ check{\isacharparenleft}{\kern0pt}x{\isacharparenright}{\kern0pt}{\isachardoublequoteclose}\ \isacommand{by}\isamarkupfalse%
\ auto\ \isanewline
\ \ \isacommand{qed}\isamarkupfalse%
\isanewline
\ \ \isacommand{assume}\isamarkupfalse%
\ {\isachardoublequoteopen}is{\isacharunderscore}{\kern0pt}P{\isacharunderscore}{\kern0pt}auto{\isacharparenleft}{\kern0pt}{\isasympi}{\isacharparenright}{\kern0pt}{\isachardoublequoteclose}\ {\isachardoublequoteopen}x\ {\isasymin}\ M{\isachardoublequoteclose}\isanewline
\ \ \isacommand{then}\isamarkupfalse%
\ \isacommand{show}\isamarkupfalse%
\ {\isachardoublequoteopen}Pn{\isacharunderscore}{\kern0pt}auto{\isacharparenleft}{\kern0pt}{\isasympi}{\isacharparenright}{\kern0pt}\ {\isacharbackquote}{\kern0pt}\ check{\isacharparenleft}{\kern0pt}x{\isacharparenright}{\kern0pt}\ {\isacharequal}{\kern0pt}\ check{\isacharparenleft}{\kern0pt}x{\isacharparenright}{\kern0pt}{\isachardoublequoteclose}\isanewline
\ \ \ \ \isacommand{using}\isamarkupfalse%
\ mainlemma\ check{\isacharunderscore}{\kern0pt}P{\isacharunderscore}{\kern0pt}name\ \isacommand{by}\isamarkupfalse%
\ auto\isanewline
\isacommand{qed}\isamarkupfalse%
%
\endisatagproof
{\isafoldproof}%
%
\isadelimproof
\isanewline
%
\endisadelimproof
\isanewline
\isacommand{end}\isamarkupfalse%
\ \isanewline
%
\isadelimtheory
%
\endisadelimtheory
%
\isatagtheory
\isacommand{end}\isamarkupfalse%
%
\endisatagtheory
{\isafoldtheory}%
%
\isadelimtheory
%
\endisadelimtheory
%
\end{isabellebody}%
\endinput
%:%file=~/source/repos/ZF-notAC/code/Automorphism_Theorems.thy%:%
%:%10=1%:%
%:%11=1%:%
%:%12=2%:%
%:%13=3%:%
%:%14=4%:%
%:%15=5%:%
%:%20=5%:%
%:%23=6%:%
%:%24=7%:%
%:%25=7%:%
%:%26=8%:%
%:%27=9%:%
%:%28=10%:%
%:%29=10%:%
%:%30=11%:%
%:%37=12%:%
%:%38=12%:%
%:%39=13%:%
%:%40=13%:%
%:%41=14%:%
%:%42=14%:%
%:%43=15%:%
%:%44=15%:%
%:%45=16%:%
%:%46=17%:%
%:%47=17%:%
%:%48=18%:%
%:%49=18%:%
%:%50=19%:%
%:%51=19%:%
%:%52=20%:%
%:%53=20%:%
%:%54=21%:%
%:%55=21%:%
%:%56=21%:%
%:%57=22%:%
%:%58=22%:%
%:%59=23%:%
%:%60=23%:%
%:%61=24%:%
%:%62=24%:%
%:%63=25%:%
%:%64=26%:%
%:%65=27%:%
%:%66=28%:%
%:%67=28%:%
%:%68=28%:%
%:%69=28%:%
%:%70=29%:%
%:%71=30%:%
%:%72=30%:%
%:%73=31%:%
%:%74=31%:%
%:%75=32%:%
%:%76=32%:%
%:%77=32%:%
%:%78=33%:%
%:%79=34%:%
%:%80=34%:%
%:%81=35%:%
%:%82=35%:%
%:%83=35%:%
%:%84=36%:%
%:%85=37%:%
%:%86=37%:%
%:%87=37%:%
%:%88=37%:%
%:%89=38%:%
%:%90=38%:%
%:%91=38%:%
%:%92=38%:%
%:%93=38%:%
%:%94=39%:%
%:%95=39%:%
%:%96=39%:%
%:%97=40%:%
%:%98=40%:%
%:%99=41%:%
%:%100=41%:%
%:%101=41%:%
%:%102=41%:%
%:%103=41%:%
%:%104=42%:%
%:%105=42%:%
%:%106=43%:%
%:%107=43%:%
%:%108=43%:%
%:%109=43%:%
%:%110=44%:%
%:%111=44%:%
%:%112=45%:%
%:%113=45%:%
%:%114=46%:%
%:%115=46%:%
%:%116=46%:%
%:%117=47%:%
%:%118=47%:%
%:%119=47%:%
%:%120=47%:%
%:%121=48%:%
%:%122=48%:%
%:%123=48%:%
%:%124=48%:%
%:%125=48%:%
%:%126=48%:%
%:%127=49%:%
%:%128=49%:%
%:%129=49%:%
%:%130=49%:%
%:%131=49%:%
%:%132=50%:%
%:%133=50%:%
%:%134=50%:%
%:%135=50%:%
%:%136=51%:%
%:%137=51%:%
%:%138=51%:%
%:%139=51%:%
%:%140=51%:%
%:%141=52%:%
%:%142=52%:%
%:%143=53%:%
%:%144=53%:%
%:%145=53%:%
%:%146=54%:%
%:%147=54%:%
%:%148=55%:%
%:%149=55%:%
%:%150=56%:%
%:%151=56%:%
%:%152=56%:%
%:%153=57%:%
%:%154=57%:%
%:%155=58%:%
%:%156=58%:%
%:%157=58%:%
%:%158=58%:%
%:%159=59%:%
%:%160=59%:%
%:%161=60%:%
%:%162=60%:%
%:%163=61%:%
%:%164=61%:%
%:%165=61%:%
%:%166=61%:%
%:%167=62%:%
%:%168=62%:%
%:%169=62%:%
%:%170=62%:%
%:%171=62%:%
%:%172=63%:%
%:%173=63%:%
%:%174=64%:%
%:%175=64%:%
%:%176=65%:%
%:%177=65%:%
%:%178=65%:%
%:%179=66%:%
%:%180=66%:%
%:%181=67%:%
%:%182=67%:%
%:%183=67%:%
%:%184=68%:%
%:%185=68%:%
%:%186=69%:%
%:%187=69%:%
%:%188=70%:%
%:%189=70%:%
%:%190=70%:%
%:%191=70%:%
%:%192=71%:%
%:%193=71%:%
%:%194=71%:%
%:%195=71%:%
%:%196=71%:%
%:%197=72%:%
%:%198=72%:%
%:%199=73%:%
%:%200=73%:%
%:%201=74%:%
%:%202=74%:%
%:%203=74%:%
%:%204=75%:%
%:%205=75%:%
%:%206=76%:%
%:%207=76%:%
%:%208=77%:%
%:%209=77%:%
%:%210=77%:%
%:%211=77%:%
%:%212=78%:%
%:%213=78%:%
%:%214=79%:%
%:%215=79%:%
%:%216=80%:%
%:%217=80%:%
%:%218=81%:%
%:%219=81%:%
%:%220=81%:%
%:%221=82%:%
%:%222=82%:%
%:%223=82%:%
%:%224=82%:%
%:%225=82%:%
%:%226=83%:%
%:%227=83%:%
%:%228=83%:%
%:%229=84%:%
%:%230=84%:%
%:%231=85%:%
%:%232=85%:%
%:%233=86%:%
%:%234=86%:%
%:%235=87%:%
%:%236=87%:%
%:%237=88%:%
%:%238=88%:%
%:%239=88%:%
%:%240=88%:%
%:%241=89%:%
%:%242=89%:%
%:%243=89%:%
%:%244=90%:%
%:%245=90%:%
%:%246=90%:%
%:%247=91%:%
%:%248=91%:%
%:%249=91%:%
%:%250=91%:%
%:%251=92%:%
%:%252=92%:%
%:%253=92%:%
%:%254=92%:%
%:%255=92%:%
%:%256=93%:%
%:%257=93%:%
%:%258=93%:%
%:%259=93%:%
%:%260=94%:%
%:%261=94%:%
%:%262=94%:%
%:%263=94%:%
%:%264=94%:%
%:%265=95%:%
%:%266=95%:%
%:%267=95%:%
%:%268=95%:%
%:%269=95%:%
%:%270=96%:%
%:%271=96%:%
%:%272=97%:%
%:%273=97%:%
%:%274=97%:%
%:%275=97%:%
%:%276=97%:%
%:%277=98%:%
%:%278=98%:%
%:%279=98%:%
%:%280=98%:%
%:%281=99%:%
%:%282=99%:%
%:%283=99%:%
%:%284=99%:%
%:%285=99%:%
%:%286=100%:%
%:%287=100%:%
%:%288=100%:%
%:%289=100%:%
%:%290=100%:%
%:%291=101%:%
%:%292=101%:%
%:%293=102%:%
%:%294=102%:%
%:%295=103%:%
%:%296=103%:%
%:%297=104%:%
%:%298=104%:%
%:%299=105%:%
%:%300=105%:%
%:%301=105%:%
%:%302=106%:%
%:%303=106%:%
%:%304=107%:%
%:%305=107%:%
%:%306=107%:%
%:%307=108%:%
%:%308=108%:%
%:%309=108%:%
%:%310=109%:%
%:%311=109%:%
%:%312=109%:%
%:%313=109%:%
%:%314=110%:%
%:%315=110%:%
%:%316=110%:%
%:%317=111%:%
%:%318=111%:%
%:%319=111%:%
%:%320=112%:%
%:%321=113%:%
%:%322=113%:%
%:%323=113%:%
%:%324=113%:%
%:%325=114%:%
%:%326=114%:%
%:%327=114%:%
%:%328=114%:%
%:%329=114%:%
%:%330=115%:%
%:%331=115%:%
%:%332=115%:%
%:%333=115%:%
%:%334=115%:%
%:%335=116%:%
%:%336=116%:%
%:%337=117%:%
%:%338=117%:%
%:%339=118%:%
%:%340=118%:%
%:%341=118%:%
%:%342=118%:%
%:%343=119%:%
%:%344=119%:%
%:%345=120%:%
%:%346=120%:%
%:%347=120%:%
%:%348=121%:%
%:%349=122%:%
%:%350=123%:%
%:%351=123%:%
%:%352=123%:%
%:%353=123%:%
%:%354=124%:%
%:%355=125%:%
%:%356=125%:%
%:%357=126%:%
%:%358=126%:%
%:%359=127%:%
%:%360=127%:%
%:%361=128%:%
%:%362=128%:%
%:%363=128%:%
%:%364=128%:%
%:%365=128%:%
%:%366=129%:%
%:%367=129%:%
%:%368=129%:%
%:%369=129%:%
%:%370=129%:%
%:%371=130%:%
%:%372=130%:%
%:%373=130%:%
%:%374=130%:%
%:%375=130%:%
%:%376=131%:%
%:%377=131%:%
%:%378=131%:%
%:%379=131%:%
%:%380=131%:%
%:%381=132%:%
%:%382=132%:%
%:%383=133%:%
%:%384=133%:%
%:%385=134%:%
%:%386=134%:%
%:%387=134%:%
%:%388=134%:%
%:%389=134%:%
%:%390=135%:%
%:%391=135%:%
%:%392=135%:%
%:%393=135%:%
%:%394=135%:%
%:%395=136%:%
%:%396=136%:%
%:%397=136%:%
%:%398=136%:%
%:%399=136%:%
%:%400=137%:%
%:%401=137%:%
%:%402=137%:%
%:%403=137%:%
%:%404=137%:%
%:%405=138%:%
%:%406=138%:%
%:%407=139%:%
%:%408=139%:%
%:%409=140%:%
%:%410=140%:%
%:%411=140%:%
%:%412=141%:%
%:%413=141%:%
%:%414=141%:%
%:%415=141%:%
%:%416=142%:%
%:%422=142%:%
%:%425=143%:%
%:%426=144%:%
%:%427=144%:%
%:%434=145%:%
%:%435=145%:%
%:%436=146%:%
%:%437=146%:%
%:%438=147%:%
%:%439=147%:%
%:%440=147%:%
%:%441=147%:%
%:%442=147%:%
%:%443=148%:%
%:%444=148%:%
%:%445=148%:%
%:%446=148%:%
%:%447=148%:%
%:%448=149%:%
%:%449=149%:%
%:%450=149%:%
%:%451=150%:%
%:%452=150%:%
%:%453=150%:%
%:%454=150%:%
%:%455=151%:%
%:%461=151%:%
%:%464=152%:%
%:%465=153%:%
%:%466=153%:%
%:%469=154%:%
%:%473=154%:%
%:%474=154%:%
%:%475=154%:%
%:%476=154%:%
%:%477=154%:%
%:%478=155%:%
%:%479=155%:%
%:%480=156%:%
%:%481=156%:%
%:%482=156%:%
%:%483=157%:%
%:%484=157%:%
%:%485=157%:%
%:%486=157%:%
%:%487=157%:%
%:%488=158%:%
%:%489=158%:%
%:%490=158%:%
%:%491=158%:%
%:%492=158%:%
%:%493=159%:%
%:%494=159%:%
%:%495=159%:%
%:%496=159%:%
%:%497=160%:%
%:%498=160%:%
%:%499=160%:%
%:%500=160%:%
%:%501=160%:%
%:%502=161%:%
%:%503=161%:%
%:%504=161%:%
%:%505=161%:%
%:%506=161%:%
%:%507=162%:%
%:%513=162%:%
%:%516=163%:%
%:%517=164%:%
%:%518=164%:%
%:%525=165%:%
%:%526=165%:%
%:%527=166%:%
%:%528=166%:%
%:%529=167%:%
%:%530=167%:%
%:%531=168%:%
%:%532=168%:%
%:%533=169%:%
%:%534=169%:%
%:%535=170%:%
%:%536=170%:%
%:%537=170%:%
%:%538=171%:%
%:%539=172%:%
%:%540=173%:%
%:%541=173%:%
%:%542=173%:%
%:%543=173%:%
%:%544=174%:%
%:%545=175%:%
%:%546=175%:%
%:%547=176%:%
%:%548=176%:%
%:%549=176%:%
%:%550=177%:%
%:%551=177%:%
%:%552=177%:%
%:%553=178%:%
%:%554=178%:%
%:%555=178%:%
%:%556=178%:%
%:%557=179%:%
%:%558=179%:%
%:%559=179%:%
%:%560=179%:%
%:%561=179%:%
%:%562=179%:%
%:%563=179%:%
%:%564=180%:%
%:%565=180%:%
%:%566=180%:%
%:%567=181%:%
%:%568=181%:%
%:%569=181%:%
%:%570=181%:%
%:%571=182%:%
%:%572=182%:%
%:%573=182%:%
%:%574=182%:%
%:%575=183%:%
%:%576=183%:%
%:%577=183%:%
%:%578=184%:%
%:%579=184%:%
%:%580=184%:%
%:%581=184%:%
%:%582=184%:%
%:%583=185%:%
%:%584=185%:%
%:%585=185%:%
%:%586=186%:%
%:%587=186%:%
%:%588=186%:%
%:%589=187%:%
%:%590=187%:%
%:%591=188%:%
%:%592=188%:%
%:%593=189%:%
%:%594=189%:%
%:%595=189%:%
%:%596=190%:%
%:%597=190%:%
%:%598=190%:%
%:%599=191%:%
%:%600=191%:%
%:%601=191%:%
%:%602=191%:%
%:%603=191%:%
%:%604=192%:%
%:%605=192%:%
%:%606=192%:%
%:%607=192%:%
%:%608=193%:%
%:%609=193%:%
%:%610=193%:%
%:%611=194%:%
%:%612=194%:%
%:%613=195%:%
%:%614=195%:%
%:%615=196%:%
%:%616=196%:%
%:%617=196%:%
%:%618=197%:%
%:%619=197%:%
%:%620=197%:%
%:%621=198%:%
%:%622=198%:%
%:%623=198%:%
%:%624=198%:%
%:%625=198%:%
%:%626=199%:%
%:%627=199%:%
%:%628=200%:%
%:%629=200%:%
%:%630=200%:%
%:%631=201%:%
%:%632=201%:%
%:%633=201%:%
%:%634=202%:%
%:%635=202%:%
%:%636=202%:%
%:%637=203%:%
%:%638=203%:%
%:%639=203%:%
%:%640=204%:%
%:%641=204%:%
%:%642=204%:%
%:%643=204%:%
%:%644=205%:%
%:%645=205%:%
%:%646=205%:%
%:%647=205%:%
%:%648=205%:%
%:%649=206%:%
%:%650=206%:%
%:%651=206%:%
%:%652=207%:%
%:%653=207%:%
%:%654=207%:%
%:%655=207%:%
%:%656=208%:%
%:%657=208%:%
%:%658=208%:%
%:%659=208%:%
%:%660=209%:%
%:%661=209%:%
%:%662=209%:%
%:%663=210%:%
%:%664=210%:%
%:%665=210%:%
%:%666=211%:%
%:%667=211%:%
%:%668=211%:%
%:%669=212%:%
%:%670=212%:%
%:%671=212%:%
%:%672=213%:%
%:%673=213%:%
%:%674=214%:%
%:%675=214%:%
%:%676=214%:%
%:%677=215%:%
%:%678=215%:%
%:%679=216%:%
%:%680=216%:%
%:%681=217%:%
%:%682=217%:%
%:%683=217%:%
%:%684=217%:%
%:%685=218%:%
%:%686=218%:%
%:%687=218%:%
%:%688=219%:%
%:%689=219%:%
%:%690=219%:%
%:%691=220%:%
%:%692=220%:%
%:%693=220%:%
%:%694=221%:%
%:%695=221%:%
%:%696=221%:%
%:%697=221%:%
%:%698=222%:%
%:%699=222%:%
%:%700=223%:%
%:%701=224%:%
%:%702=224%:%
%:%703=225%:%
%:%704=225%:%
%:%705=226%:%
%:%706=226%:%
%:%707=226%:%
%:%708=227%:%
%:%709=227%:%
%:%710=227%:%
%:%711=227%:%
%:%712=228%:%
%:%713=228%:%
%:%714=228%:%
%:%715=228%:%
%:%716=228%:%
%:%717=229%:%
%:%718=229%:%
%:%719=229%:%
%:%720=230%:%
%:%721=230%:%
%:%722=230%:%
%:%723=230%:%
%:%724=230%:%
%:%725=231%:%
%:%726=231%:%
%:%727=231%:%
%:%728=232%:%
%:%729=232%:%
%:%730=233%:%
%:%731=233%:%
%:%732=233%:%
%:%733=233%:%
%:%734=234%:%
%:%735=234%:%
%:%736=234%:%
%:%737=234%:%
%:%738=234%:%
%:%739=235%:%
%:%740=235%:%
%:%741=235%:%
%:%742=235%:%
%:%743=235%:%
%:%744=236%:%
%:%745=237%:%
%:%746=237%:%
%:%747=238%:%
%:%748=238%:%
%:%749=239%:%
%:%750=239%:%
%:%751=239%:%
%:%752=240%:%
%:%753=240%:%
%:%754=240%:%
%:%755=240%:%
%:%756=241%:%
%:%757=241%:%
%:%758=241%:%
%:%759=242%:%
%:%760=242%:%
%:%761=243%:%
%:%762=243%:%
%:%763=244%:%
%:%764=244%:%
%:%765=245%:%
%:%766=245%:%
%:%767=246%:%
%:%768=246%:%
%:%769=247%:%
%:%770=247%:%
%:%771=248%:%
%:%772=248%:%
%:%773=249%:%
%:%774=249%:%
%:%775=249%:%
%:%776=249%:%
%:%777=249%:%
%:%778=250%:%
%:%779=250%:%
%:%780=250%:%
%:%781=250%:%
%:%782=250%:%
%:%783=251%:%
%:%784=251%:%
%:%785=252%:%
%:%786=253%:%
%:%787=253%:%
%:%788=253%:%
%:%789=254%:%
%:%790=254%:%
%:%791=255%:%
%:%792=255%:%
%:%793=256%:%
%:%794=256%:%
%:%795=257%:%
%:%796=257%:%
%:%797=258%:%
%:%798=258%:%
%:%799=259%:%
%:%800=259%:%
%:%801=260%:%
%:%802=260%:%
%:%803=261%:%
%:%804=261%:%
%:%805=262%:%
%:%806=262%:%
%:%807=263%:%
%:%808=263%:%
%:%809=264%:%
%:%810=264%:%
%:%811=265%:%
%:%812=265%:%
%:%813=266%:%
%:%814=266%:%
%:%815=267%:%
%:%816=267%:%
%:%817=268%:%
%:%818=268%:%
%:%819=269%:%
%:%825=269%:%
%:%828=270%:%
%:%829=271%:%
%:%830=271%:%
%:%837=272%:%
%:%838=272%:%
%:%839=273%:%
%:%840=273%:%
%:%841=274%:%
%:%842=274%:%
%:%843=275%:%
%:%844=275%:%
%:%845=276%:%
%:%846=276%:%
%:%847=276%:%
%:%848=277%:%
%:%849=278%:%
%:%850=279%:%
%:%851=279%:%
%:%852=280%:%
%:%853=280%:%
%:%854=280%:%
%:%855=281%:%
%:%856=281%:%
%:%857=281%:%
%:%858=282%:%
%:%859=282%:%
%:%860=282%:%
%:%861=282%:%
%:%862=283%:%
%:%863=283%:%
%:%864=283%:%
%:%865=284%:%
%:%866=284%:%
%:%867=284%:%
%:%868=284%:%
%:%869=284%:%
%:%870=284%:%
%:%871=285%:%
%:%872=285%:%
%:%873=285%:%
%:%874=285%:%
%:%875=285%:%
%:%876=285%:%
%:%877=285%:%
%:%878=286%:%
%:%879=286%:%
%:%880=286%:%
%:%881=287%:%
%:%882=287%:%
%:%883=288%:%
%:%884=288%:%
%:%885=289%:%
%:%886=289%:%
%:%887=289%:%
%:%888=290%:%
%:%889=290%:%
%:%890=290%:%
%:%891=290%:%
%:%892=290%:%
%:%893=290%:%
%:%894=291%:%
%:%895=291%:%
%:%896=291%:%
%:%897=291%:%
%:%898=292%:%
%:%899=292%:%
%:%900=292%:%
%:%901=292%:%
%:%902=293%:%
%:%903=293%:%
%:%904=294%:%
%:%905=294%:%
%:%906=294%:%
%:%907=294%:%
%:%908=294%:%
%:%909=295%:%
%:%910=295%:%
%:%911=295%:%
%:%912=295%:%
%:%913=296%:%
%:%914=296%:%
%:%915=296%:%
%:%916=296%:%
%:%917=296%:%
%:%918=296%:%
%:%919=297%:%
%:%920=297%:%
%:%921=297%:%
%:%922=297%:%
%:%923=297%:%
%:%924=298%:%
%:%925=298%:%
%:%926=299%:%
%:%927=299%:%
%:%928=299%:%
%:%929=299%:%
%:%930=299%:%
%:%931=299%:%
%:%932=300%:%
%:%933=300%:%
%:%934=300%:%
%:%935=300%:%
%:%936=301%:%
%:%937=301%:%
%:%938=302%:%
%:%939=303%:%
%:%940=303%:%
%:%941=304%:%
%:%942=304%:%
%:%943=305%:%
%:%944=305%:%
%:%945=306%:%
%:%946=306%:%
%:%947=307%:%
%:%948=307%:%
%:%949=308%:%
%:%950=308%:%
%:%951=309%:%
%:%952=309%:%
%:%953=310%:%
%:%954=310%:%
%:%955=311%:%
%:%956=311%:%
%:%957=312%:%
%:%958=312%:%
%:%959=313%:%
%:%960=313%:%
%:%961=314%:%
%:%967=314%:%
%:%970=315%:%
%:%971=316%:%
%:%972=316%:%
%:%975=317%:%
%:%979=317%:%
%:%980=317%:%
%:%981=317%:%
%:%982=318%:%
%:%983=318%:%
%:%984=319%:%
%:%985=319%:%
%:%986=319%:%
%:%987=320%:%
%:%988=320%:%
%:%989=321%:%
%:%990=321%:%
%:%991=321%:%
%:%992=322%:%
%:%993=322%:%
%:%994=323%:%
%:%995=323%:%
%:%996=323%:%
%:%997=323%:%
%:%998=324%:%
%:%999=324%:%
%:%1000=324%:%
%:%1001=325%:%
%:%1002=325%:%
%:%1003=326%:%
%:%1004=326%:%
%:%1005=326%:%
%:%1006=327%:%
%:%1007=327%:%
%:%1008=328%:%
%:%1009=328%:%
%:%1010=328%:%
%:%1011=328%:%
%:%1012=329%:%
%:%1013=329%:%
%:%1014=329%:%
%:%1015=330%:%
%:%1016=330%:%
%:%1017=331%:%
%:%1018=331%:%
%:%1019=332%:%
%:%1020=332%:%
%:%1021=332%:%
%:%1022=332%:%
%:%1023=332%:%
%:%1024=333%:%
%:%1030=333%:%
%:%1033=334%:%
%:%1034=335%:%
%:%1035=335%:%
%:%1042=336%:%
%:%1043=336%:%
%:%1044=337%:%
%:%1045=337%:%
%:%1046=338%:%
%:%1047=338%:%
%:%1048=339%:%
%:%1049=339%:%
%:%1050=340%:%
%:%1051=340%:%
%:%1052=341%:%
%:%1053=341%:%
%:%1054=342%:%
%:%1055=342%:%
%:%1056=342%:%
%:%1057=343%:%
%:%1058=343%:%
%:%1059=344%:%
%:%1060=344%:%
%:%1061=345%:%
%:%1062=345%:%
%:%1063=346%:%
%:%1064=346%:%
%:%1065=346%:%
%:%1066=347%:%
%:%1067=347%:%
%:%1068=348%:%
%:%1069=348%:%
%:%1070=349%:%
%:%1071=349%:%
%:%1072=349%:%
%:%1073=349%:%
%:%1074=350%:%
%:%1075=351%:%
%:%1076=351%:%
%:%1077=352%:%
%:%1078=352%:%
%:%1079=353%:%
%:%1080=353%:%
%:%1081=354%:%
%:%1082=354%:%
%:%1083=355%:%
%:%1084=355%:%
%:%1085=355%:%
%:%1086=356%:%
%:%1087=356%:%
%:%1088=357%:%
%:%1089=357%:%
%:%1090=358%:%
%:%1091=358%:%
%:%1092=359%:%
%:%1093=359%:%
%:%1094=360%:%
%:%1095=360%:%
%:%1096=360%:%
%:%1097=361%:%
%:%1098=361%:%
%:%1099=361%:%
%:%1100=362%:%
%:%1101=362%:%
%:%1102=362%:%
%:%1103=363%:%
%:%1104=363%:%
%:%1105=364%:%
%:%1106=364%:%
%:%1107=365%:%
%:%1108=365%:%
%:%1109=366%:%
%:%1110=366%:%
%:%1111=366%:%
%:%1112=367%:%
%:%1113=367%:%
%:%1114=368%:%
%:%1115=368%:%
%:%1116=369%:%
%:%1117=369%:%
%:%1118=370%:%
%:%1119=370%:%
%:%1120=370%:%
%:%1121=371%:%
%:%1122=371%:%
%:%1123=372%:%
%:%1124=372%:%
%:%1125=373%:%
%:%1126=373%:%
%:%1127=373%:%
%:%1128=374%:%
%:%1129=374%:%
%:%1130=375%:%
%:%1131=375%:%
%:%1132=376%:%
%:%1133=376%:%
%:%1134=377%:%
%:%1135=377%:%
%:%1136=377%:%
%:%1137=377%:%
%:%1138=378%:%
%:%1144=378%:%
%:%1147=379%:%
%:%1148=380%:%
%:%1149=380%:%
%:%1150=381%:%
%:%1157=382%:%
%:%1158=382%:%
%:%1159=383%:%
%:%1160=383%:%
%:%1161=384%:%
%:%1162=384%:%
%:%1163=384%:%
%:%1164=385%:%
%:%1165=385%:%
%:%1166=385%:%
%:%1167=386%:%
%:%1168=386%:%
%:%1169=387%:%
%:%1170=387%:%
%:%1171=388%:%
%:%1172=388%:%
%:%1173=389%:%
%:%1174=389%:%
%:%1175=389%:%
%:%1176=389%:%
%:%1177=390%:%
%:%1183=390%:%
%:%1186=391%:%
%:%1187=392%:%
%:%1188=392%:%
%:%1189=393%:%
%:%1196=394%:%
%:%1197=394%:%
%:%1198=395%:%
%:%1199=395%:%
%:%1200=396%:%
%:%1201=396%:%
%:%1202=397%:%
%:%1203=397%:%
%:%1204=398%:%
%:%1205=398%:%
%:%1206=399%:%
%:%1207=399%:%
%:%1208=400%:%
%:%1209=400%:%
%:%1210=401%:%
%:%1211=401%:%
%:%1212=401%:%
%:%1213=402%:%
%:%1214=403%:%
%:%1215=404%:%
%:%1216=404%:%
%:%1217=405%:%
%:%1218=405%:%
%:%1219=405%:%
%:%1220=406%:%
%:%1221=406%:%
%:%1222=406%:%
%:%1223=406%:%
%:%1224=407%:%
%:%1225=407%:%
%:%1226=408%:%
%:%1227=408%:%
%:%1228=409%:%
%:%1229=409%:%
%:%1230=410%:%
%:%1231=410%:%
%:%1232=410%:%
%:%1233=411%:%
%:%1234=411%:%
%:%1235=411%:%
%:%1236=412%:%
%:%1237=412%:%
%:%1238=413%:%
%:%1239=413%:%
%:%1240=413%:%
%:%1241=414%:%
%:%1242=414%:%
%:%1243=414%:%
%:%1244=415%:%
%:%1245=415%:%
%:%1246=416%:%
%:%1247=416%:%
%:%1248=416%:%
%:%1249=417%:%
%:%1250=417%:%
%:%1251=417%:%
%:%1252=417%:%
%:%1253=417%:%
%:%1254=418%:%
%:%1255=418%:%
%:%1256=418%:%
%:%1257=418%:%
%:%1258=419%:%
%:%1259=419%:%
%:%1260=420%:%
%:%1261=420%:%
%:%1262=420%:%
%:%1263=421%:%
%:%1264=421%:%
%:%1265=421%:%
%:%1266=422%:%
%:%1267=422%:%
%:%1268=422%:%
%:%1269=423%:%
%:%1270=423%:%
%:%1271=424%:%
%:%1272=424%:%
%:%1273=424%:%
%:%1274=425%:%
%:%1275=425%:%
%:%1276=425%:%
%:%1277=426%:%
%:%1278=426%:%
%:%1279=427%:%
%:%1280=427%:%
%:%1281=427%:%
%:%1282=428%:%
%:%1283=428%:%
%:%1284=428%:%
%:%1285=428%:%
%:%1286=428%:%
%:%1287=429%:%
%:%1288=429%:%
%:%1289=429%:%
%:%1290=429%:%
%:%1291=430%:%
%:%1292=430%:%
%:%1293=430%:%
%:%1294=430%:%
%:%1295=431%:%
%:%1296=431%:%
%:%1297=431%:%
%:%1298=431%:%
%:%1299=431%:%
%:%1300=432%:%
%:%1301=432%:%
%:%1302=433%:%
%:%1303=433%:%
%:%1304=433%:%
%:%1305=434%:%
%:%1306=434%:%
%:%1307=435%:%
%:%1308=435%:%
%:%1309=436%:%
%:%1310=436%:%
%:%1311=436%:%
%:%1312=437%:%
%:%1313=437%:%
%:%1314=437%:%
%:%1315=437%:%
%:%1316=437%:%
%:%1317=438%:%
%:%1318=438%:%
%:%1319=438%:%
%:%1320=438%:%
%:%1321=439%:%
%:%1322=439%:%
%:%1323=439%:%
%:%1324=439%:%
%:%1325=440%:%
%:%1326=440%:%
%:%1327=441%:%
%:%1328=441%:%
%:%1329=441%:%
%:%1330=442%:%
%:%1331=442%:%
%:%1332=442%:%
%:%1333=443%:%
%:%1334=443%:%
%:%1335=444%:%
%:%1336=444%:%
%:%1337=444%:%
%:%1338=445%:%
%:%1339=445%:%
%:%1340=446%:%
%:%1341=446%:%
%:%1342=446%:%
%:%1343=446%:%
%:%1344=446%:%
%:%1345=447%:%
%:%1346=447%:%
%:%1347=447%:%
%:%1348=447%:%
%:%1349=448%:%
%:%1350=448%:%
%:%1351=449%:%
%:%1352=449%:%
%:%1353=450%:%
%:%1354=450%:%
%:%1355=450%:%
%:%1356=451%:%
%:%1357=451%:%
%:%1358=451%:%
%:%1359=452%:%
%:%1365=452%:%
%:%1368=453%:%
%:%1369=454%:%
%:%1370=454%:%
%:%1377=455%:%

%
\begin{isabellebody}%
\setisabellecontext{HS{\isacharunderscore}{\kern0pt}Definition}%
%
\isadelimtheory
%
\endisadelimtheory
%
\isatagtheory
\isacommand{theory}\isamarkupfalse%
\ HS{\isacharunderscore}{\kern0pt}Definition\isanewline
\ \ \isakeyword{imports}\ \isanewline
\ \ \ \ {\isachardoublequoteopen}Forcing{\isacharslash}{\kern0pt}Forcing{\isacharunderscore}{\kern0pt}Main{\isachardoublequoteclose}\ \isanewline
\ \ \ \ Automorphism{\isacharunderscore}{\kern0pt}Theorems\isanewline
\isakeyword{begin}%
\endisatagtheory
{\isafoldtheory}%
%
\isadelimtheory
\ \isanewline
%
\endisadelimtheory
\isanewline
\isacommand{context}\isamarkupfalse%
\ forcing{\isacharunderscore}{\kern0pt}data{\isacharunderscore}{\kern0pt}partial\isanewline
\isakeyword{begin}\isanewline
\isanewline
\isacommand{definition}\isamarkupfalse%
\ is{\isacharunderscore}{\kern0pt}P{\isacharunderscore}{\kern0pt}auto{\isacharunderscore}{\kern0pt}group\ \isakeyword{where}\ \isanewline
\ \ {\isachardoublequoteopen}is{\isacharunderscore}{\kern0pt}P{\isacharunderscore}{\kern0pt}auto{\isacharunderscore}{\kern0pt}group{\isacharparenleft}{\kern0pt}G{\isacharparenright}{\kern0pt}\ {\isasymequiv}\ \isanewline
\ \ \ \ G\ {\isasymsubseteq}\ {\isacharbraceleft}{\kern0pt}\ {\isasympi}\ {\isasymin}\ P\ {\isasymrightarrow}\ P{\isachardot}{\kern0pt}\ is{\isacharunderscore}{\kern0pt}P{\isacharunderscore}{\kern0pt}auto{\isacharparenleft}{\kern0pt}{\isasympi}{\isacharparenright}{\kern0pt}\ {\isacharbraceright}{\kern0pt}\ \isanewline
\ \ {\isasymand}\ {\isacharparenleft}{\kern0pt}{\isasymforall}{\isasympi}\ {\isasymin}\ G{\isachardot}{\kern0pt}\ {\isasymforall}{\isasymtau}\ {\isasymin}\ G{\isachardot}{\kern0pt}\ {\isasympi}\ O\ {\isasymtau}\ {\isasymin}\ G{\isacharparenright}{\kern0pt}\ \isanewline
\ \ {\isasymand}\ {\isacharparenleft}{\kern0pt}{\isasymforall}{\isasympi}\ {\isasymin}\ G{\isachardot}{\kern0pt}\ converse{\isacharparenleft}{\kern0pt}{\isasympi}{\isacharparenright}{\kern0pt}\ {\isasymin}\ G{\isacharparenright}{\kern0pt}{\isachardoublequoteclose}\ \ \ \isanewline
\isanewline
\isacommand{definition}\isamarkupfalse%
\ P{\isacharunderscore}{\kern0pt}auto{\isacharunderscore}{\kern0pt}subgroups\ \isakeyword{where}\ {\isachardoublequoteopen}P{\isacharunderscore}{\kern0pt}auto{\isacharunderscore}{\kern0pt}subgroups{\isacharparenleft}{\kern0pt}G{\isacharparenright}{\kern0pt}\ {\isasymequiv}\ {\isacharbraceleft}{\kern0pt}\ H\ {\isasymin}\ Pow{\isacharparenleft}{\kern0pt}G{\isacharparenright}{\kern0pt}\ {\isasyminter}\ M{\isachardot}{\kern0pt}\ is{\isacharunderscore}{\kern0pt}P{\isacharunderscore}{\kern0pt}auto{\isacharunderscore}{\kern0pt}group{\isacharparenleft}{\kern0pt}H{\isacharparenright}{\kern0pt}\ {\isacharbraceright}{\kern0pt}{\isachardoublequoteclose}\ \isanewline
\isanewline
\isacommand{end}\isamarkupfalse%
\isanewline
\isanewline
\isacommand{locale}\isamarkupfalse%
\ M{\isacharunderscore}{\kern0pt}symmetric{\isacharunderscore}{\kern0pt}system\ {\isacharequal}{\kern0pt}\ forcing{\isacharunderscore}{\kern0pt}data{\isacharunderscore}{\kern0pt}partial\ {\isacharplus}{\kern0pt}\ \isanewline
\ \ \isakeyword{fixes}\ {\isasymG}\ {\isasymF}\ \isanewline
\ \ \isakeyword{assumes}\ {\isasymG}{\isacharunderscore}{\kern0pt}in{\isacharunderscore}{\kern0pt}M\ {\isacharcolon}{\kern0pt}\ {\isachardoublequoteopen}{\isasymG}\ {\isasymin}\ M{\isachardoublequoteclose}\ \ \isanewline
\ \ \isakeyword{and}\ {\isasymG}{\isacharunderscore}{\kern0pt}P{\isacharunderscore}{\kern0pt}auto{\isacharunderscore}{\kern0pt}group\ {\isacharcolon}{\kern0pt}\ {\isachardoublequoteopen}is{\isacharunderscore}{\kern0pt}P{\isacharunderscore}{\kern0pt}auto{\isacharunderscore}{\kern0pt}group{\isacharparenleft}{\kern0pt}{\isasymG}{\isacharparenright}{\kern0pt}{\isachardoublequoteclose}\ \ \ \isanewline
\ \ \isakeyword{and}\ {\isasymF}{\isacharunderscore}{\kern0pt}in{\isacharunderscore}{\kern0pt}M\ {\isacharcolon}{\kern0pt}\ {\isachardoublequoteopen}{\isasymF}\ {\isasymin}\ M{\isachardoublequoteclose}\isanewline
\ \ \isakeyword{and}\ {\isasymF}{\isacharunderscore}{\kern0pt}subset\ {\isacharcolon}{\kern0pt}\ {\isachardoublequoteopen}{\isasymF}\ {\isasymsubseteq}\ P{\isacharunderscore}{\kern0pt}auto{\isacharunderscore}{\kern0pt}subgroups{\isacharparenleft}{\kern0pt}{\isasymG}{\isacharparenright}{\kern0pt}{\isachardoublequoteclose}\ \isanewline
\ \ \isakeyword{and}\ {\isasymF}{\isacharunderscore}{\kern0pt}nonempty\ {\isacharcolon}{\kern0pt}\ {\isachardoublequoteopen}{\isasymF}\ {\isasymnoteq}\ {\isadigit{0}}{\isachardoublequoteclose}\ \isanewline
\ \ \isakeyword{and}\ {\isasymF}{\isacharunderscore}{\kern0pt}closed{\isacharunderscore}{\kern0pt}under{\isacharunderscore}{\kern0pt}intersection\ {\isacharcolon}{\kern0pt}\ {\isachardoublequoteopen}{\isasymforall}A\ {\isasymin}\ {\isasymF}{\isachardot}{\kern0pt}\ {\isasymforall}B\ {\isasymin}\ {\isasymF}{\isachardot}{\kern0pt}\ A\ {\isasyminter}\ B\ {\isasymin}\ {\isasymF}{\isachardoublequoteclose}\ \isanewline
\ \ \isakeyword{and}\ {\isasymF}{\isacharunderscore}{\kern0pt}closed{\isacharunderscore}{\kern0pt}under{\isacharunderscore}{\kern0pt}supergroup\ {\isacharcolon}{\kern0pt}\ {\isachardoublequoteopen}{\isasymforall}A\ {\isasymin}\ {\isasymF}{\isachardot}{\kern0pt}\ {\isasymforall}B\ {\isasymin}\ P{\isacharunderscore}{\kern0pt}auto{\isacharunderscore}{\kern0pt}subgroups{\isacharparenleft}{\kern0pt}{\isasymG}{\isacharparenright}{\kern0pt}{\isachardot}{\kern0pt}\ A\ {\isasymsubseteq}\ B\ {\isasymlongrightarrow}\ B\ {\isasymin}\ {\isasymF}{\isachardoublequoteclose}\ \isanewline
\ \ \isakeyword{and}\ {\isasymF}{\isacharunderscore}{\kern0pt}normal\ {\isacharcolon}{\kern0pt}\ {\isachardoublequoteopen}{\isasymforall}H\ {\isasymin}\ {\isasymF}{\isachardot}{\kern0pt}\ {\isasymforall}{\isasympi}\ {\isasymin}\ {\isasymG}{\isachardot}{\kern0pt}\ {\isacharbraceleft}{\kern0pt}\ {\isasympi}\ O\ {\isasymtau}\ O\ converse{\isacharparenleft}{\kern0pt}{\isasympi}{\isacharparenright}{\kern0pt}{\isachardot}{\kern0pt}\ {\isasymtau}\ {\isasymin}\ H\ {\isacharbraceright}{\kern0pt}\ {\isasymin}\ {\isasymF}{\isachardoublequoteclose}\ \isanewline
\isakeyword{begin}\isanewline
\isanewline
\isacommand{definition}\isamarkupfalse%
\ sym\ \isakeyword{where}\ {\isachardoublequoteopen}sym{\isacharparenleft}{\kern0pt}x{\isacharparenright}{\kern0pt}\ {\isasymequiv}\ {\isacharbraceleft}{\kern0pt}\ {\isasympi}\ {\isasymin}\ {\isasymG}{\isachardot}{\kern0pt}\ Pn{\isacharunderscore}{\kern0pt}auto{\isacharparenleft}{\kern0pt}{\isasympi}{\isacharparenright}{\kern0pt}{\isacharbackquote}{\kern0pt}x\ {\isacharequal}{\kern0pt}\ x\ {\isacharbraceright}{\kern0pt}{\isachardoublequoteclose}\ \ \isanewline
\isanewline
\isacommand{definition}\isamarkupfalse%
\ symmetric\ \isakeyword{where}\ {\isachardoublequoteopen}symmetric{\isacharparenleft}{\kern0pt}x{\isacharparenright}{\kern0pt}\ {\isasymequiv}\ sym{\isacharparenleft}{\kern0pt}x{\isacharparenright}{\kern0pt}\ {\isasymin}\ {\isasymF}{\isachardoublequoteclose}\ \ \isanewline
\isanewline
\isacommand{definition}\isamarkupfalse%
\ HHS{\isacharunderscore}{\kern0pt}set{\isacharunderscore}{\kern0pt}succ\ \isakeyword{where}\ {\isachardoublequoteopen}HHS{\isacharunderscore}{\kern0pt}set{\isacharunderscore}{\kern0pt}succ{\isacharparenleft}{\kern0pt}a{\isacharcomma}{\kern0pt}\ X{\isacharparenright}{\kern0pt}\ {\isasymequiv}\ {\isacharbraceleft}{\kern0pt}\ x\ {\isasymin}\ P{\isacharunderscore}{\kern0pt}set{\isacharparenleft}{\kern0pt}succ{\isacharparenleft}{\kern0pt}a{\isacharparenright}{\kern0pt}{\isacharparenright}{\kern0pt}{\isachardot}{\kern0pt}\ domain{\isacharparenleft}{\kern0pt}x{\isacharparenright}{\kern0pt}\ {\isasymsubseteq}\ X\ {\isasymand}\ symmetric{\isacharparenleft}{\kern0pt}x{\isacharparenright}{\kern0pt}\ {\isacharbraceright}{\kern0pt}{\isachardoublequoteclose}\ \isanewline
\isanewline
\isacommand{definition}\isamarkupfalse%
\ HS{\isacharunderscore}{\kern0pt}set\ \isakeyword{where}\ {\isachardoublequoteopen}HS{\isacharunderscore}{\kern0pt}set{\isacharparenleft}{\kern0pt}a{\isacharparenright}{\kern0pt}\ {\isasymequiv}\ transrec{\isadigit{2}}{\isacharparenleft}{\kern0pt}a{\isacharcomma}{\kern0pt}\ {\isadigit{0}}{\isacharcomma}{\kern0pt}\ HHS{\isacharunderscore}{\kern0pt}set{\isacharunderscore}{\kern0pt}succ{\isacharparenright}{\kern0pt}{\isachardoublequoteclose}\ \isanewline
\isacommand{definition}\isamarkupfalse%
\ HS\ \isakeyword{where}\ {\isachardoublequoteopen}HS\ {\isasymequiv}\ {\isacharbraceleft}{\kern0pt}\ x\ {\isasymin}\ P{\isacharunderscore}{\kern0pt}names{\isachardot}{\kern0pt}\ {\isasymexists}a{\isachardot}{\kern0pt}\ Ord{\isacharparenleft}{\kern0pt}a{\isacharparenright}{\kern0pt}\ {\isasymand}\ x\ {\isasymin}\ HS{\isacharunderscore}{\kern0pt}set{\isacharparenleft}{\kern0pt}a{\isacharparenright}{\kern0pt}\ {\isacharbraceright}{\kern0pt}{\isachardoublequoteclose}\ \isanewline
\isanewline
\isacommand{lemma}\isamarkupfalse%
\ HS{\isacharunderscore}{\kern0pt}in{\isacharunderscore}{\kern0pt}HS{\isacharunderscore}{\kern0pt}set{\isacharunderscore}{\kern0pt}succ{\isacharprime}{\kern0pt}\ {\isacharcolon}{\kern0pt}\ {\isachardoublequoteopen}Ord{\isacharparenleft}{\kern0pt}a{\isacharparenright}{\kern0pt}\ {\isasymLongrightarrow}\ x\ {\isasymin}\ HS{\isacharunderscore}{\kern0pt}set{\isacharparenleft}{\kern0pt}a{\isacharparenright}{\kern0pt}\ {\isasymLongrightarrow}\ {\isasymexists}b\ {\isacharless}{\kern0pt}\ a{\isachardot}{\kern0pt}\ Ord{\isacharparenleft}{\kern0pt}b{\isacharparenright}{\kern0pt}\ {\isasymand}\ x\ {\isasymin}\ HS{\isacharunderscore}{\kern0pt}set{\isacharparenleft}{\kern0pt}succ{\isacharparenleft}{\kern0pt}b{\isacharparenright}{\kern0pt}{\isacharparenright}{\kern0pt}{\isachardoublequoteclose}\ \isanewline
%
\isadelimproof
\ \ %
\endisadelimproof
%
\isatagproof
\isacommand{apply}\isamarkupfalse%
{\isacharparenleft}{\kern0pt}rule{\isacharunderscore}{\kern0pt}tac\ P{\isacharequal}{\kern0pt}{\isachardoublequoteopen}{\isasymforall}x{\isachardot}{\kern0pt}\ x\ {\isasymin}\ HS{\isacharunderscore}{\kern0pt}set{\isacharparenleft}{\kern0pt}a{\isacharparenright}{\kern0pt}\ {\isasymlongrightarrow}\ {\isacharparenleft}{\kern0pt}{\isasymexists}b\ {\isacharless}{\kern0pt}\ a{\isachardot}{\kern0pt}\ Ord{\isacharparenleft}{\kern0pt}b{\isacharparenright}{\kern0pt}\ {\isasymand}\ x\ {\isasymin}\ HS{\isacharunderscore}{\kern0pt}set{\isacharparenleft}{\kern0pt}succ{\isacharparenleft}{\kern0pt}b{\isacharparenright}{\kern0pt}{\isacharparenright}{\kern0pt}{\isacharparenright}{\kern0pt}{\isachardoublequoteclose}\ \isakeyword{in}\ mp{\isacharparenright}{\kern0pt}\ \isanewline
\ \ \ \isacommand{apply}\isamarkupfalse%
\ simp\ \isanewline
\ \ \isacommand{apply}\isamarkupfalse%
{\isacharparenleft}{\kern0pt}rule{\isacharunderscore}{\kern0pt}tac\ P{\isacharequal}{\kern0pt}{\isachardoublequoteopen}{\isasymlambda}a{\isachardot}{\kern0pt}\ {\isasymforall}x{\isachardot}{\kern0pt}\ x\ {\isasymin}\ HS{\isacharunderscore}{\kern0pt}set{\isacharparenleft}{\kern0pt}a{\isacharparenright}{\kern0pt}\ {\isasymlongrightarrow}\ {\isacharparenleft}{\kern0pt}{\isasymexists}b\ {\isacharless}{\kern0pt}\ a{\isachardot}{\kern0pt}\ Ord{\isacharparenleft}{\kern0pt}b{\isacharparenright}{\kern0pt}\ {\isasymand}\ x\ {\isasymin}\ HS{\isacharunderscore}{\kern0pt}set{\isacharparenleft}{\kern0pt}succ{\isacharparenleft}{\kern0pt}b{\isacharparenright}{\kern0pt}{\isacharparenright}{\kern0pt}{\isacharparenright}{\kern0pt}{\isachardoublequoteclose}\ \isakeyword{in}\ trans{\isacharunderscore}{\kern0pt}induct{\isadigit{3}}{\isacharunderscore}{\kern0pt}raw{\isacharparenright}{\kern0pt}\ \ \isanewline
\ \ \ \ \ \isacommand{apply}\isamarkupfalse%
\ simp\ \isanewline
\ \ \ \ \isacommand{apply}\isamarkupfalse%
{\isacharparenleft}{\kern0pt}simp\ add{\isacharcolon}{\kern0pt}HS{\isacharunderscore}{\kern0pt}set{\isacharunderscore}{\kern0pt}def{\isacharparenright}{\kern0pt}\ \isanewline
\ \ \ \isacommand{apply}\isamarkupfalse%
\ blast\ \isanewline
\ \ \isacommand{apply}\isamarkupfalse%
\ clarify\ \isanewline
\isacommand{proof}\isamarkupfalse%
\ {\isacharminus}{\kern0pt}\ \isanewline
\ \ \isacommand{fix}\isamarkupfalse%
\ a\ x\ \isacommand{assume}\isamarkupfalse%
\ assms\ {\isacharcolon}{\kern0pt}\ {\isachardoublequoteopen}Limit{\isacharparenleft}{\kern0pt}a{\isacharparenright}{\kern0pt}{\isachardoublequoteclose}\ {\isachardoublequoteopen}{\isasymforall}b{\isasymin}a{\isachardot}{\kern0pt}\ {\isasymforall}x{\isachardot}{\kern0pt}\ x\ {\isasymin}\ HS{\isacharunderscore}{\kern0pt}set{\isacharparenleft}{\kern0pt}b{\isacharparenright}{\kern0pt}\ {\isasymlongrightarrow}\ {\isacharparenleft}{\kern0pt}{\isasymexists}c\ {\isacharless}{\kern0pt}\ b{\isachardot}{\kern0pt}\ Ord{\isacharparenleft}{\kern0pt}c{\isacharparenright}{\kern0pt}\ {\isasymand}\ x\ {\isasymin}\ HS{\isacharunderscore}{\kern0pt}set{\isacharparenleft}{\kern0pt}succ{\isacharparenleft}{\kern0pt}c{\isacharparenright}{\kern0pt}{\isacharparenright}{\kern0pt}{\isacharparenright}{\kern0pt}{\isachardoublequoteclose}\ {\isachardoublequoteopen}x\ {\isasymin}\ HS{\isacharunderscore}{\kern0pt}set{\isacharparenleft}{\kern0pt}a{\isacharparenright}{\kern0pt}{\isachardoublequoteclose}\ \isanewline
\ \ \isacommand{have}\isamarkupfalse%
\ {\isachardoublequoteopen}HS{\isacharunderscore}{\kern0pt}set{\isacharparenleft}{\kern0pt}a{\isacharparenright}{\kern0pt}\ {\isacharequal}{\kern0pt}\ {\isacharparenleft}{\kern0pt}{\isasymUnion}b{\isacharless}{\kern0pt}a{\isachardot}{\kern0pt}\ HS{\isacharunderscore}{\kern0pt}set{\isacharparenleft}{\kern0pt}b{\isacharparenright}{\kern0pt}{\isacharparenright}{\kern0pt}{\isachardoublequoteclose}\ \isanewline
\ \ \ \ \isacommand{unfolding}\isamarkupfalse%
\ HS{\isacharunderscore}{\kern0pt}set{\isacharunderscore}{\kern0pt}def\ \isacommand{using}\isamarkupfalse%
\ transrec{\isadigit{2}}{\isacharunderscore}{\kern0pt}Limit\ assms\ \isacommand{by}\isamarkupfalse%
\ auto\ \isanewline
\ \ \isacommand{then}\isamarkupfalse%
\ \isacommand{obtain}\isamarkupfalse%
\ b\ \isakeyword{where}\ bH\ {\isacharcolon}{\kern0pt}\ {\isachardoublequoteopen}b\ {\isacharless}{\kern0pt}\ a{\isachardoublequoteclose}\ {\isachardoublequoteopen}x\ {\isasymin}\ HS{\isacharunderscore}{\kern0pt}set{\isacharparenleft}{\kern0pt}b{\isacharparenright}{\kern0pt}{\isachardoublequoteclose}\ \isacommand{using}\isamarkupfalse%
\ assms\ \isacommand{by}\isamarkupfalse%
\ auto\ \isanewline
\ \ \isacommand{then}\isamarkupfalse%
\ \isacommand{obtain}\isamarkupfalse%
\ c\ \isakeyword{where}\ cH\ {\isacharcolon}{\kern0pt}\ {\isachardoublequoteopen}c\ {\isacharless}{\kern0pt}\ b{\isachardoublequoteclose}\ {\isachardoublequoteopen}Ord{\isacharparenleft}{\kern0pt}c{\isacharparenright}{\kern0pt}\ {\isasymand}\ x\ {\isasymin}\ HS{\isacharunderscore}{\kern0pt}set{\isacharparenleft}{\kern0pt}succ{\isacharparenleft}{\kern0pt}c{\isacharparenright}{\kern0pt}{\isacharparenright}{\kern0pt}{\isachardoublequoteclose}\ \isacommand{using}\isamarkupfalse%
\ assms\ ltD\ \isacommand{by}\isamarkupfalse%
\ blast\ \ \isanewline
\ \ \isacommand{then}\isamarkupfalse%
\ \isacommand{have}\isamarkupfalse%
\ {\isachardoublequoteopen}c\ {\isacharless}{\kern0pt}\ a{\isachardoublequoteclose}\ \isacommand{using}\isamarkupfalse%
\ lt{\isacharunderscore}{\kern0pt}trans\ bH\ \isacommand{by}\isamarkupfalse%
\ auto\isanewline
\ \ \isacommand{then}\isamarkupfalse%
\ \isacommand{show}\isamarkupfalse%
\ {\isachardoublequoteopen}{\isasymexists}c\ {\isacharless}{\kern0pt}\ a{\isachardot}{\kern0pt}\ Ord{\isacharparenleft}{\kern0pt}c{\isacharparenright}{\kern0pt}\ {\isasymand}\ x\ {\isasymin}\ HS{\isacharunderscore}{\kern0pt}set{\isacharparenleft}{\kern0pt}succ{\isacharparenleft}{\kern0pt}c{\isacharparenright}{\kern0pt}{\isacharparenright}{\kern0pt}{\isachardoublequoteclose}\ \isacommand{using}\isamarkupfalse%
\ cH\ \isacommand{by}\isamarkupfalse%
\ auto\ \isanewline
\isacommand{qed}\isamarkupfalse%
%
\endisatagproof
{\isafoldproof}%
%
\isadelimproof
\isanewline
%
\endisadelimproof
\isanewline
\isacommand{lemma}\isamarkupfalse%
\ HS{\isacharunderscore}{\kern0pt}in{\isacharunderscore}{\kern0pt}HS{\isacharunderscore}{\kern0pt}set{\isacharunderscore}{\kern0pt}succ\ {\isacharcolon}{\kern0pt}\ {\isachardoublequoteopen}x\ {\isasymin}\ HS\ {\isasymLongrightarrow}\ {\isasymexists}b{\isachardot}{\kern0pt}\ Ord{\isacharparenleft}{\kern0pt}b{\isacharparenright}{\kern0pt}\ {\isasymand}\ x\ {\isasymin}\ HS{\isacharunderscore}{\kern0pt}set{\isacharparenleft}{\kern0pt}succ{\isacharparenleft}{\kern0pt}b{\isacharparenright}{\kern0pt}{\isacharparenright}{\kern0pt}{\isachardoublequoteclose}\ \isanewline
%
\isadelimproof
%
\endisadelimproof
%
\isatagproof
\isacommand{proof}\isamarkupfalse%
\ {\isacharminus}{\kern0pt}\ \isanewline
\ \ \isacommand{assume}\isamarkupfalse%
\ {\isachardoublequoteopen}x\ {\isasymin}\ HS{\isachardoublequoteclose}\ \isanewline
\ \ \isacommand{then}\isamarkupfalse%
\ \isacommand{obtain}\isamarkupfalse%
\ a\ \isakeyword{where}\ {\isachardoublequoteopen}Ord{\isacharparenleft}{\kern0pt}a{\isacharparenright}{\kern0pt}{\isachardoublequoteclose}\ {\isachardoublequoteopen}x\ {\isasymin}\ HS{\isacharunderscore}{\kern0pt}set{\isacharparenleft}{\kern0pt}a{\isacharparenright}{\kern0pt}{\isachardoublequoteclose}\ \isacommand{unfolding}\isamarkupfalse%
\ HS{\isacharunderscore}{\kern0pt}def\ \isacommand{by}\isamarkupfalse%
\ auto\ \isanewline
\ \ \isacommand{then}\isamarkupfalse%
\ \isacommand{show}\isamarkupfalse%
\ {\isacharquery}{\kern0pt}thesis\ \isacommand{using}\isamarkupfalse%
\ HS{\isacharunderscore}{\kern0pt}in{\isacharunderscore}{\kern0pt}HS{\isacharunderscore}{\kern0pt}set{\isacharunderscore}{\kern0pt}succ{\isacharprime}{\kern0pt}\ \isacommand{by}\isamarkupfalse%
\ auto\isanewline
\isacommand{qed}\isamarkupfalse%
%
\endisatagproof
{\isafoldproof}%
%
\isadelimproof
\isanewline
%
\endisadelimproof
\isanewline
\isacommand{lemma}\isamarkupfalse%
\ HS{\isacharunderscore}{\kern0pt}set{\isacharunderscore}{\kern0pt}succ\ {\isacharcolon}{\kern0pt}\ {\isachardoublequoteopen}Ord{\isacharparenleft}{\kern0pt}a{\isacharparenright}{\kern0pt}\ {\isasymLongrightarrow}\ HS{\isacharunderscore}{\kern0pt}set{\isacharparenleft}{\kern0pt}succ{\isacharparenleft}{\kern0pt}a{\isacharparenright}{\kern0pt}{\isacharparenright}{\kern0pt}\ {\isacharequal}{\kern0pt}\ {\isacharbraceleft}{\kern0pt}\ x\ {\isasymin}\ P{\isacharunderscore}{\kern0pt}set{\isacharparenleft}{\kern0pt}succ{\isacharparenleft}{\kern0pt}a{\isacharparenright}{\kern0pt}{\isacharparenright}{\kern0pt}{\isachardot}{\kern0pt}\ domain{\isacharparenleft}{\kern0pt}x{\isacharparenright}{\kern0pt}\ {\isasymsubseteq}\ HS{\isacharunderscore}{\kern0pt}set{\isacharparenleft}{\kern0pt}a{\isacharparenright}{\kern0pt}\ {\isasymand}\ symmetric{\isacharparenleft}{\kern0pt}x{\isacharparenright}{\kern0pt}\ {\isacharbraceright}{\kern0pt}{\isachardoublequoteclose}\ \isanewline
%
\isadelimproof
\ \ %
\endisadelimproof
%
\isatagproof
\isacommand{unfolding}\isamarkupfalse%
\ HS{\isacharunderscore}{\kern0pt}set{\isacharunderscore}{\kern0pt}def\ \isanewline
\ \ \isacommand{apply}\isamarkupfalse%
{\isacharparenleft}{\kern0pt}rule{\isacharunderscore}{\kern0pt}tac\ b{\isacharequal}{\kern0pt}{\isachardoublequoteopen}transrec{\isadigit{2}}{\isacharparenleft}{\kern0pt}succ{\isacharparenleft}{\kern0pt}a{\isacharparenright}{\kern0pt}{\isacharcomma}{\kern0pt}\ {\isadigit{0}}{\isacharcomma}{\kern0pt}\ HHS{\isacharunderscore}{\kern0pt}set{\isacharunderscore}{\kern0pt}succ{\isacharparenright}{\kern0pt}{\isachardoublequoteclose}\ \isakeyword{and}\ a{\isacharequal}{\kern0pt}{\isachardoublequoteopen}HHS{\isacharunderscore}{\kern0pt}set{\isacharunderscore}{\kern0pt}succ{\isacharparenleft}{\kern0pt}a{\isacharcomma}{\kern0pt}\ transrec{\isadigit{2}}{\isacharparenleft}{\kern0pt}a{\isacharcomma}{\kern0pt}\ {\isadigit{0}}{\isacharcomma}{\kern0pt}\ HHS{\isacharunderscore}{\kern0pt}set{\isacharunderscore}{\kern0pt}succ{\isacharparenright}{\kern0pt}{\isacharparenright}{\kern0pt}{\isachardoublequoteclose}\ \isakeyword{in}\ ssubst{\isacharparenright}{\kern0pt}\isanewline
\ \ \ \isacommand{apply}\isamarkupfalse%
{\isacharparenleft}{\kern0pt}rule\ transrec{\isadigit{2}}{\isacharunderscore}{\kern0pt}succ{\isacharparenright}{\kern0pt}\ \isanewline
\ \ \isacommand{unfolding}\isamarkupfalse%
\ HHS{\isacharunderscore}{\kern0pt}set{\isacharunderscore}{\kern0pt}succ{\isacharunderscore}{\kern0pt}def\ \isanewline
\ \ \isacommand{by}\isamarkupfalse%
\ simp%
\endisatagproof
{\isafoldproof}%
%
\isadelimproof
\isanewline
%
\endisadelimproof
\isanewline
\isacommand{lemma}\isamarkupfalse%
\ HS{\isacharunderscore}{\kern0pt}set{\isacharunderscore}{\kern0pt}subset\ {\isacharcolon}{\kern0pt}\ {\isachardoublequoteopen}Ord{\isacharparenleft}{\kern0pt}a{\isacharparenright}{\kern0pt}\ {\isasymLongrightarrow}\ HS{\isacharunderscore}{\kern0pt}set{\isacharparenleft}{\kern0pt}a{\isacharparenright}{\kern0pt}\ {\isasymsubseteq}\ P{\isacharunderscore}{\kern0pt}set{\isacharparenleft}{\kern0pt}a{\isacharparenright}{\kern0pt}{\isachardoublequoteclose}\ \isanewline
%
\isadelimproof
\ \ %
\endisadelimproof
%
\isatagproof
\isacommand{apply}\isamarkupfalse%
{\isacharparenleft}{\kern0pt}rule{\isacharunderscore}{\kern0pt}tac\ P{\isacharequal}{\kern0pt}{\isachardoublequoteopen}Ord{\isacharparenleft}{\kern0pt}a{\isacharparenright}{\kern0pt}\ {\isasymlongrightarrow}\ HS{\isacharunderscore}{\kern0pt}set{\isacharparenleft}{\kern0pt}a{\isacharparenright}{\kern0pt}\ {\isasymsubseteq}\ P{\isacharunderscore}{\kern0pt}set{\isacharparenleft}{\kern0pt}a{\isacharparenright}{\kern0pt}{\isachardoublequoteclose}\ \isakeyword{in}\ mp{\isacharparenright}{\kern0pt}\ \isanewline
\ \ \ \isacommand{apply}\isamarkupfalse%
\ simp\ \isanewline
\ \ \isacommand{apply}\isamarkupfalse%
{\isacharparenleft}{\kern0pt}rule{\isacharunderscore}{\kern0pt}tac\ P{\isacharequal}{\kern0pt}{\isachardoublequoteopen}{\isasymlambda}a{\isachardot}{\kern0pt}\ Ord{\isacharparenleft}{\kern0pt}a{\isacharparenright}{\kern0pt}\ {\isasymlongrightarrow}\ HS{\isacharunderscore}{\kern0pt}set{\isacharparenleft}{\kern0pt}a{\isacharparenright}{\kern0pt}\ {\isasymsubseteq}\ P{\isacharunderscore}{\kern0pt}set{\isacharparenleft}{\kern0pt}a{\isacharparenright}{\kern0pt}{\isachardoublequoteclose}\ \isakeyword{in}\ trans{\isacharunderscore}{\kern0pt}induct{\isadigit{3}}{\isacharunderscore}{\kern0pt}raw{\isacharparenright}{\kern0pt}\ \isanewline
\ \ \ \ \ \isacommand{apply}\isamarkupfalse%
\ simp\ \isanewline
\ \ \ \ \isacommand{apply}\isamarkupfalse%
\ {\isacharparenleft}{\kern0pt}simp\ add{\isacharcolon}{\kern0pt}HS{\isacharunderscore}{\kern0pt}set{\isacharunderscore}{\kern0pt}def{\isacharparenright}{\kern0pt}\ \isanewline
\ \ \ \isacommand{apply}\isamarkupfalse%
\ {\isacharparenleft}{\kern0pt}simp\ add{\isacharcolon}{\kern0pt}HS{\isacharunderscore}{\kern0pt}set{\isacharunderscore}{\kern0pt}def\ HHS{\isacharunderscore}{\kern0pt}set{\isacharunderscore}{\kern0pt}succ{\isacharunderscore}{\kern0pt}def{\isacharparenright}{\kern0pt}\ \isanewline
\ \ \ \isacommand{apply}\isamarkupfalse%
\ blast\ \isanewline
\ \ \isacommand{apply}\isamarkupfalse%
\ auto\ \isanewline
\isacommand{proof}\isamarkupfalse%
\ {\isacharminus}{\kern0pt}\ \isanewline
\ \ \isacommand{fix}\isamarkupfalse%
\ x\ a\ \isacommand{assume}\isamarkupfalse%
\ assms\ {\isacharcolon}{\kern0pt}\ {\isachardoublequoteopen}Limit{\isacharparenleft}{\kern0pt}a{\isacharparenright}{\kern0pt}{\isachardoublequoteclose}\ {\isachardoublequoteopen}x\ {\isasymin}\ HS{\isacharunderscore}{\kern0pt}set{\isacharparenleft}{\kern0pt}a{\isacharparenright}{\kern0pt}{\isachardoublequoteclose}\ {\isachardoublequoteopen}{\isasymforall}b{\isasymin}a{\isachardot}{\kern0pt}\ Ord{\isacharparenleft}{\kern0pt}b{\isacharparenright}{\kern0pt}\ {\isasymlongrightarrow}\ HS{\isacharunderscore}{\kern0pt}set{\isacharparenleft}{\kern0pt}b{\isacharparenright}{\kern0pt}\ {\isasymsubseteq}\ P{\isacharunderscore}{\kern0pt}set{\isacharparenleft}{\kern0pt}b{\isacharparenright}{\kern0pt}{\isachardoublequoteclose}\isanewline
\ \ \isacommand{then}\isamarkupfalse%
\ \isacommand{have}\isamarkupfalse%
\ {\isachardoublequoteopen}x\ {\isasymin}\ {\isacharparenleft}{\kern0pt}{\isasymUnion}b\ {\isacharless}{\kern0pt}\ a{\isachardot}{\kern0pt}\ HS{\isacharunderscore}{\kern0pt}set{\isacharparenleft}{\kern0pt}b{\isacharparenright}{\kern0pt}{\isacharparenright}{\kern0pt}{\isachardoublequoteclose}\ \isacommand{unfolding}\isamarkupfalse%
\ HS{\isacharunderscore}{\kern0pt}set{\isacharunderscore}{\kern0pt}def\ \isacommand{using}\isamarkupfalse%
\ transrec{\isadigit{2}}{\isacharunderscore}{\kern0pt}Limit\ \isacommand{by}\isamarkupfalse%
\ auto\ \isanewline
\ \ \isacommand{then}\isamarkupfalse%
\ \isacommand{obtain}\isamarkupfalse%
\ b\ \isakeyword{where}\ bH\ {\isacharcolon}{\kern0pt}\ {\isachardoublequoteopen}b\ {\isacharless}{\kern0pt}\ a{\isachardoublequoteclose}\ {\isachardoublequoteopen}x\ {\isasymin}\ HS{\isacharunderscore}{\kern0pt}set{\isacharparenleft}{\kern0pt}b{\isacharparenright}{\kern0pt}{\isachardoublequoteclose}\ {\isachardoublequoteopen}Ord{\isacharparenleft}{\kern0pt}b{\isacharparenright}{\kern0pt}{\isachardoublequoteclose}\ \isacommand{using}\isamarkupfalse%
\ assms\ Limit{\isacharunderscore}{\kern0pt}is{\isacharunderscore}{\kern0pt}Ord\ lt{\isacharunderscore}{\kern0pt}Ord\ \isacommand{by}\isamarkupfalse%
\ auto\ \isanewline
\ \ \isacommand{then}\isamarkupfalse%
\ \isacommand{have}\isamarkupfalse%
\ {\isachardoublequoteopen}x\ {\isasymin}\ P{\isacharunderscore}{\kern0pt}set{\isacharparenleft}{\kern0pt}b{\isacharparenright}{\kern0pt}{\isachardoublequoteclose}\ \isacommand{using}\isamarkupfalse%
\ assms\ ltD\ \isacommand{by}\isamarkupfalse%
\ auto\ \isanewline
\ \ \isacommand{then}\isamarkupfalse%
\ \isacommand{show}\isamarkupfalse%
\ {\isachardoublequoteopen}x\ {\isasymin}\ P{\isacharunderscore}{\kern0pt}set{\isacharparenleft}{\kern0pt}a{\isacharparenright}{\kern0pt}{\isachardoublequoteclose}\ \isacommand{using}\isamarkupfalse%
\ P{\isacharunderscore}{\kern0pt}set{\isacharunderscore}{\kern0pt}lim\ assms\ bH\ \isacommand{by}\isamarkupfalse%
\ auto\ \isanewline
\isacommand{qed}\isamarkupfalse%
%
\endisatagproof
{\isafoldproof}%
%
\isadelimproof
\isanewline
%
\endisadelimproof
\isanewline
\isacommand{lemma}\isamarkupfalse%
\ HS{\isacharunderscore}{\kern0pt}iff{\isacharcolon}{\kern0pt}\ {\isachardoublequoteopen}x\ {\isasymin}\ HS\ {\isasymlongleftrightarrow}\ x\ {\isasymin}\ P{\isacharunderscore}{\kern0pt}names\ {\isasymand}\ domain{\isacharparenleft}{\kern0pt}x{\isacharparenright}{\kern0pt}\ {\isasymsubseteq}\ HS\ {\isasymand}\ symmetric{\isacharparenleft}{\kern0pt}x{\isacharparenright}{\kern0pt}{\isachardoublequoteclose}\ \isanewline
%
\isadelimproof
\ \ %
\endisadelimproof
%
\isatagproof
\isacommand{apply}\isamarkupfalse%
{\isacharparenleft}{\kern0pt}rule{\isacharunderscore}{\kern0pt}tac\ Q{\isacharequal}{\kern0pt}{\isachardoublequoteopen}{\isasymlambda}x{\isachardot}{\kern0pt}\ x\ {\isasymin}\ HS\ {\isasymlongleftrightarrow}\ x\ {\isasymin}\ P{\isacharunderscore}{\kern0pt}names\ {\isasymand}\ domain{\isacharparenleft}{\kern0pt}x{\isacharparenright}{\kern0pt}\ {\isasymsubseteq}\ HS\ {\isasymand}\ symmetric{\isacharparenleft}{\kern0pt}x{\isacharparenright}{\kern0pt}{\isachardoublequoteclose}\ \isakeyword{in}\ ed{\isacharunderscore}{\kern0pt}induction{\isacharparenright}{\kern0pt}\ \isanewline
\ \ \isacommand{apply}\isamarkupfalse%
\ {\isacharparenleft}{\kern0pt}rule\ iffI{\isacharparenright}{\kern0pt}\isanewline
\isacommand{proof}\isamarkupfalse%
\ {\isacharminus}{\kern0pt}\ \isanewline
\ \ \isacommand{fix}\isamarkupfalse%
\ x\ \isacommand{assume}\isamarkupfalse%
\ assms\ {\isacharcolon}{\kern0pt}\ {\isachardoublequoteopen}{\isacharparenleft}{\kern0pt}{\isasymAnd}y{\isachardot}{\kern0pt}\ ed{\isacharparenleft}{\kern0pt}y{\isacharcomma}{\kern0pt}\ x{\isacharparenright}{\kern0pt}\ {\isasymLongrightarrow}\ y\ {\isasymin}\ HS\ {\isasymlongleftrightarrow}\ y\ {\isasymin}\ P{\isacharunderscore}{\kern0pt}names\ {\isasymand}\ domain{\isacharparenleft}{\kern0pt}y{\isacharparenright}{\kern0pt}\ {\isasymsubseteq}\ HS\ {\isasymand}\ symmetric{\isacharparenleft}{\kern0pt}y{\isacharparenright}{\kern0pt}{\isacharparenright}{\kern0pt}{\isachardoublequoteclose}\ {\isachardoublequoteopen}x\ {\isasymin}\ HS{\isachardoublequoteclose}\isanewline
\ \ \isanewline
\ \ \isacommand{obtain}\isamarkupfalse%
\ a\ \isakeyword{where}\ aH\ {\isacharcolon}{\kern0pt}\ {\isachardoublequoteopen}Ord{\isacharparenleft}{\kern0pt}a{\isacharparenright}{\kern0pt}{\isachardoublequoteclose}\ {\isachardoublequoteopen}x\ {\isasymin}\ HS{\isacharunderscore}{\kern0pt}set{\isacharparenleft}{\kern0pt}succ{\isacharparenleft}{\kern0pt}a{\isacharparenright}{\kern0pt}{\isacharparenright}{\kern0pt}{\isachardoublequoteclose}\ \isacommand{using}\isamarkupfalse%
\ assms\ HS{\isacharunderscore}{\kern0pt}in{\isacharunderscore}{\kern0pt}HS{\isacharunderscore}{\kern0pt}set{\isacharunderscore}{\kern0pt}succ\ \isacommand{by}\isamarkupfalse%
\ blast\isanewline
\ \ \isacommand{then}\isamarkupfalse%
\ \isacommand{have}\isamarkupfalse%
\ {\isachardoublequoteopen}domain{\isacharparenleft}{\kern0pt}x{\isacharparenright}{\kern0pt}\ {\isasymsubseteq}\ HS{\isacharunderscore}{\kern0pt}set{\isacharparenleft}{\kern0pt}a{\isacharparenright}{\kern0pt}{\isachardoublequoteclose}\ {\isachardoublequoteopen}symmetric{\isacharparenleft}{\kern0pt}x{\isacharparenright}{\kern0pt}{\isachardoublequoteclose}\ \isacommand{using}\isamarkupfalse%
\ HS{\isacharunderscore}{\kern0pt}set{\isacharunderscore}{\kern0pt}succ\ \isacommand{by}\isamarkupfalse%
\ auto\ \isanewline
\ \ \isacommand{then}\isamarkupfalse%
\ \isacommand{show}\isamarkupfalse%
\ {\isachardoublequoteopen}x\ {\isasymin}\ P{\isacharunderscore}{\kern0pt}names\ {\isasymand}\ domain{\isacharparenleft}{\kern0pt}x{\isacharparenright}{\kern0pt}\ {\isasymsubseteq}\ HS\ {\isasymand}\ symmetric{\isacharparenleft}{\kern0pt}x{\isacharparenright}{\kern0pt}{\isachardoublequoteclose}\ \isanewline
\ \ \ \ \isacommand{apply}\isamarkupfalse%
\ auto\ \isanewline
\ \ \ \ \isacommand{using}\isamarkupfalse%
\ assms\ \isanewline
\ \ \ \ \ \isacommand{apply}\isamarkupfalse%
{\isacharparenleft}{\kern0pt}simp\ add{\isacharcolon}{\kern0pt}HS{\isacharunderscore}{\kern0pt}def{\isacharparenright}{\kern0pt}\ \isanewline
\ \ \ \ \isacommand{unfolding}\isamarkupfalse%
\ HS{\isacharunderscore}{\kern0pt}def\ \isanewline
\ \ \ \ \isacommand{using}\isamarkupfalse%
\ aH\ \isanewline
\ \ \ \ \isacommand{apply}\isamarkupfalse%
\ auto\isanewline
\ \ \isacommand{proof}\isamarkupfalse%
\ {\isacharminus}{\kern0pt}\ \isanewline
\ \ \ \ \isacommand{fix}\isamarkupfalse%
\ y\ p\ \isacommand{assume}\isamarkupfalse%
\ {\isachardoublequoteopen}{\isacharless}{\kern0pt}y{\isacharcomma}{\kern0pt}\ p{\isachargreater}{\kern0pt}\ {\isasymin}\ x{\isachardoublequoteclose}\ \isacommand{then}\isamarkupfalse%
\ \isacommand{show}\isamarkupfalse%
\ {\isachardoublequoteopen}y\ {\isasymin}\ P{\isacharunderscore}{\kern0pt}names{\isachardoublequoteclose}\ \isacommand{using}\isamarkupfalse%
\ P{\isacharunderscore}{\kern0pt}name{\isacharunderscore}{\kern0pt}domain{\isacharunderscore}{\kern0pt}P{\isacharunderscore}{\kern0pt}name\ assms\ \isacommand{unfolding}\isamarkupfalse%
\ HS{\isacharunderscore}{\kern0pt}def\ \isacommand{by}\isamarkupfalse%
\ auto\isanewline
\ \ \isacommand{qed}\isamarkupfalse%
\isanewline
\isacommand{next}\isamarkupfalse%
\ \isanewline
\ \ \isacommand{fix}\isamarkupfalse%
\ x\ \isacommand{assume}\isamarkupfalse%
\ assms\ {\isacharcolon}{\kern0pt}\ {\isachardoublequoteopen}{\isacharparenleft}{\kern0pt}{\isasymAnd}y{\isachardot}{\kern0pt}\ ed{\isacharparenleft}{\kern0pt}y{\isacharcomma}{\kern0pt}\ x{\isacharparenright}{\kern0pt}\ {\isasymLongrightarrow}\ y\ {\isasymin}\ HS\ {\isasymlongleftrightarrow}\ y\ {\isasymin}\ P{\isacharunderscore}{\kern0pt}names\ {\isasymand}\ domain{\isacharparenleft}{\kern0pt}y{\isacharparenright}{\kern0pt}\ {\isasymsubseteq}\ HS\ {\isasymand}\ symmetric{\isacharparenleft}{\kern0pt}y{\isacharparenright}{\kern0pt}{\isacharparenright}{\kern0pt}{\isachardoublequoteclose}\ \isanewline
\ \ \ \ \ \ \ \ \ \ \ \ \ \ \ \ \ \ \ \ \ \ \ {\isachardoublequoteopen}x\ {\isasymin}\ P{\isacharunderscore}{\kern0pt}names\ {\isasymand}\ domain{\isacharparenleft}{\kern0pt}x{\isacharparenright}{\kern0pt}\ {\isasymsubseteq}\ HS\ {\isasymand}\ symmetric{\isacharparenleft}{\kern0pt}x{\isacharparenright}{\kern0pt}{\isachardoublequoteclose}\isanewline
\isanewline
\ \ \isacommand{define}\isamarkupfalse%
\ I\ \isakeyword{where}\ {\isachardoublequoteopen}I\ {\isasymequiv}\ {\isacharbraceleft}{\kern0pt}\ {\isacharparenleft}{\kern0pt}{\isasymmu}\ a{\isachardot}{\kern0pt}\ y\ {\isasymin}\ HS{\isacharunderscore}{\kern0pt}set{\isacharparenleft}{\kern0pt}a{\isacharparenright}{\kern0pt}{\isacharparenright}{\kern0pt}\ {\isachardot}{\kern0pt}\ y\ {\isasymin}\ domain{\isacharparenleft}{\kern0pt}x{\isacharparenright}{\kern0pt}\ {\isacharbraceright}{\kern0pt}{\isachardoublequoteclose}\ \isanewline
\ \ \isacommand{then}\isamarkupfalse%
\ \isacommand{have}\isamarkupfalse%
\ {\isachardoublequoteopen}Ord{\isacharparenleft}{\kern0pt}{\isasymUnion}I{\isacharparenright}{\kern0pt}{\isachardoublequoteclose}\isanewline
\ \ \ \ \isacommand{apply}\isamarkupfalse%
{\isacharparenleft}{\kern0pt}rule{\isacharunderscore}{\kern0pt}tac\ Ord{\isacharunderscore}{\kern0pt}Union{\isacharparenright}{\kern0pt}\ \isanewline
\ \ \ \ \isacommand{by}\isamarkupfalse%
\ auto\ \isanewline
\ \ \isacommand{then}\isamarkupfalse%
\ \isacommand{obtain}\isamarkupfalse%
\ L\ \isakeyword{where}\ LH\ {\isacharcolon}{\kern0pt}\ {\isachardoublequoteopen}{\isacharparenleft}{\kern0pt}{\isasymUnion}I{\isacharparenright}{\kern0pt}\ {\isacharless}{\kern0pt}\ L{\isachardoublequoteclose}\ \ {\isachardoublequoteopen}Limit{\isacharparenleft}{\kern0pt}L{\isacharparenright}{\kern0pt}{\isachardoublequoteclose}\ \isanewline
\ \ \ \ \isacommand{using}\isamarkupfalse%
\ ex{\isacharunderscore}{\kern0pt}larger{\isacharunderscore}{\kern0pt}limit\ \isacommand{by}\isamarkupfalse%
\ auto\ \isanewline
\ \ \isacommand{then}\isamarkupfalse%
\ \isacommand{have}\isamarkupfalse%
\ {\isachardoublequoteopen}HS{\isacharunderscore}{\kern0pt}set{\isacharparenleft}{\kern0pt}L{\isacharparenright}{\kern0pt}\ {\isacharequal}{\kern0pt}\ {\isacharparenleft}{\kern0pt}{\isasymUnion}a\ {\isacharless}{\kern0pt}\ L{\isachardot}{\kern0pt}\ HS{\isacharunderscore}{\kern0pt}set{\isacharparenleft}{\kern0pt}a{\isacharparenright}{\kern0pt}{\isacharparenright}{\kern0pt}{\isachardoublequoteclose}\ \isacommand{unfolding}\isamarkupfalse%
\ HS{\isacharunderscore}{\kern0pt}set{\isacharunderscore}{\kern0pt}def\ \isacommand{using}\isamarkupfalse%
\ transrec{\isadigit{2}}{\isacharunderscore}{\kern0pt}Limit\ \isacommand{by}\isamarkupfalse%
\ auto\ \isanewline
\ \ \isacommand{then}\isamarkupfalse%
\ \isacommand{have}\isamarkupfalse%
\ H\ {\isacharcolon}{\kern0pt}\ {\isachardoublequoteopen}domain{\isacharparenleft}{\kern0pt}x{\isacharparenright}{\kern0pt}\ {\isasymsubseteq}\ HS{\isacharunderscore}{\kern0pt}set{\isacharparenleft}{\kern0pt}L{\isacharparenright}{\kern0pt}{\isachardoublequoteclose}\ \isanewline
\ \ \isacommand{proof}\isamarkupfalse%
\ {\isacharparenleft}{\kern0pt}auto{\isacharparenright}{\kern0pt}\ \isanewline
\ \ \ \ \isacommand{fix}\isamarkupfalse%
\ y\ p\ \isacommand{assume}\isamarkupfalse%
\ assm\ {\isacharcolon}{\kern0pt}\ {\isachardoublequoteopen}{\isacharless}{\kern0pt}y{\isacharcomma}{\kern0pt}\ p{\isachargreater}{\kern0pt}\ {\isasymin}\ x{\isachardoublequoteclose}\isanewline
\ \ \ \ \isacommand{then}\isamarkupfalse%
\ \isacommand{have}\isamarkupfalse%
\ {\isachardoublequoteopen}y\ {\isasymin}\ HS{\isachardoublequoteclose}\ \isacommand{using}\isamarkupfalse%
\ assms\ \isacommand{by}\isamarkupfalse%
\ auto\ \isanewline
\ \ \ \ \isacommand{then}\isamarkupfalse%
\ \isacommand{obtain}\isamarkupfalse%
\ a\ \isakeyword{where}\ {\isachardoublequoteopen}Ord{\isacharparenleft}{\kern0pt}a{\isacharparenright}{\kern0pt}{\isachardoublequoteclose}\ {\isachardoublequoteopen}y\ {\isasymin}\ HS{\isacharunderscore}{\kern0pt}set{\isacharparenleft}{\kern0pt}a{\isacharparenright}{\kern0pt}{\isachardoublequoteclose}\ \isacommand{unfolding}\isamarkupfalse%
\ HS{\isacharunderscore}{\kern0pt}def\ \isacommand{by}\isamarkupfalse%
\ auto\ \isanewline
\ \ \ \ \isacommand{then}\isamarkupfalse%
\ \isacommand{have}\isamarkupfalse%
\ H{\isacharcolon}{\kern0pt}\ {\isachardoublequoteopen}y\ {\isasymin}\ HS{\isacharunderscore}{\kern0pt}set{\isacharparenleft}{\kern0pt}{\isasymmu}\ a{\isachardot}{\kern0pt}\ y\ {\isasymin}\ HS{\isacharunderscore}{\kern0pt}set{\isacharparenleft}{\kern0pt}a{\isacharparenright}{\kern0pt}{\isacharparenright}{\kern0pt}{\isachardoublequoteclose}\ \isanewline
\ \ \ \ \ \ \isacommand{apply}\isamarkupfalse%
{\isacharparenleft}{\kern0pt}rule{\isacharunderscore}{\kern0pt}tac\ LeastI{\isacharparenright}{\kern0pt}\ \isanewline
\ \ \ \ \ \ \isacommand{by}\isamarkupfalse%
\ auto\isanewline
\ \isanewline
\ \ \ \ \isacommand{have}\isamarkupfalse%
\ {\isachardoublequoteopen}{\isacharparenleft}{\kern0pt}{\isasymmu}\ a{\isachardot}{\kern0pt}\ y\ {\isasymin}\ HS{\isacharunderscore}{\kern0pt}set{\isacharparenleft}{\kern0pt}a{\isacharparenright}{\kern0pt}{\isacharparenright}{\kern0pt}\ {\isasymle}\ {\isacharparenleft}{\kern0pt}{\isasymUnion}y\ {\isasymin}\ domain{\isacharparenleft}{\kern0pt}x{\isacharparenright}{\kern0pt}{\isachardot}{\kern0pt}\ {\isacharparenleft}{\kern0pt}{\isasymmu}\ a{\isachardot}{\kern0pt}\ y\ {\isasymin}\ HS{\isacharunderscore}{\kern0pt}set{\isacharparenleft}{\kern0pt}a{\isacharparenright}{\kern0pt}{\isacharparenright}{\kern0pt}{\isacharparenright}{\kern0pt}{\isachardoublequoteclose}\ \ \isanewline
\ \ \ \ \ \ \isacommand{apply}\isamarkupfalse%
{\isacharparenleft}{\kern0pt}rule{\isacharunderscore}{\kern0pt}tac\ j{\isacharequal}{\kern0pt}{\isachardoublequoteopen}{\isacharparenleft}{\kern0pt}{\isasymmu}\ a{\isachardot}{\kern0pt}\ y\ {\isasymin}\ HS{\isacharunderscore}{\kern0pt}set{\isacharparenleft}{\kern0pt}a{\isacharparenright}{\kern0pt}{\isacharparenright}{\kern0pt}{\isachardoublequoteclose}\ \isakeyword{in}\ Union{\isacharunderscore}{\kern0pt}upper{\isacharunderscore}{\kern0pt}le{\isacharparenright}{\kern0pt}\ \isanewline
\ \ \ \ \ \ \isacommand{using}\isamarkupfalse%
\ assm\ le{\isacharunderscore}{\kern0pt}refl\ \isanewline
\ \ \ \ \ \ \ \ \isacommand{apply}\isamarkupfalse%
\ auto\ \isanewline
\ \ \ \ \ \ \isacommand{done}\isamarkupfalse%
\ \isanewline
\ \ \ \ \isacommand{then}\isamarkupfalse%
\ \isacommand{have}\isamarkupfalse%
\ {\isachardoublequoteopen}{\isacharparenleft}{\kern0pt}{\isasymmu}\ a{\isachardot}{\kern0pt}\ y\ {\isasymin}\ HS{\isacharunderscore}{\kern0pt}set{\isacharparenleft}{\kern0pt}a{\isacharparenright}{\kern0pt}{\isacharparenright}{\kern0pt}\ {\isasymle}\ {\isacharparenleft}{\kern0pt}{\isasymUnion}I{\isacharparenright}{\kern0pt}{\isachardoublequoteclose}\ \isacommand{unfolding}\isamarkupfalse%
\ I{\isacharunderscore}{\kern0pt}def\ \isacommand{by}\isamarkupfalse%
\ auto\ \isanewline
\ \ \ \ \isacommand{then}\isamarkupfalse%
\ \isacommand{have}\isamarkupfalse%
\ {\isachardoublequoteopen}{\isacharparenleft}{\kern0pt}{\isasymmu}\ a{\isachardot}{\kern0pt}\ y\ {\isasymin}\ HS{\isacharunderscore}{\kern0pt}set{\isacharparenleft}{\kern0pt}a{\isacharparenright}{\kern0pt}{\isacharparenright}{\kern0pt}\ {\isacharless}{\kern0pt}\ L{\isachardoublequoteclose}\ \isanewline
\ \ \ \ \ \ \isacommand{apply}\isamarkupfalse%
{\isacharparenleft}{\kern0pt}rule{\isacharunderscore}{\kern0pt}tac\ b{\isacharequal}{\kern0pt}{\isachardoublequoteopen}{\isasymUnion}I{\isachardoublequoteclose}\ \isakeyword{in}\ le{\isacharunderscore}{\kern0pt}lt{\isacharunderscore}{\kern0pt}lt{\isacharparenright}{\kern0pt}\ \isanewline
\ \ \ \ \ \ \isacommand{using}\isamarkupfalse%
\ LH\ \isanewline
\ \ \ \ \ \ \isacommand{by}\isamarkupfalse%
\ auto\ \isanewline
\ \ \ \ \isacommand{then}\isamarkupfalse%
\ \isacommand{show}\isamarkupfalse%
\ {\isachardoublequoteopen}y\ {\isasymin}\ {\isacharparenleft}{\kern0pt}{\isasymUnion}a\ {\isacharless}{\kern0pt}\ L{\isachardot}{\kern0pt}\ HS{\isacharunderscore}{\kern0pt}set{\isacharparenleft}{\kern0pt}a{\isacharparenright}{\kern0pt}{\isacharparenright}{\kern0pt}{\isachardoublequoteclose}\isanewline
\ \ \ \ \ \ \isacommand{apply}\isamarkupfalse%
{\isacharparenleft}{\kern0pt}rule{\isacharunderscore}{\kern0pt}tac\ OUN{\isacharunderscore}{\kern0pt}I{\isacharparenright}{\kern0pt}\ \isanewline
\ \ \ \ \ \ \isacommand{using}\isamarkupfalse%
\ H\ \isanewline
\ \ \ \ \ \ \isacommand{by}\isamarkupfalse%
\ auto\isanewline
\ \ \isacommand{qed}\isamarkupfalse%
\isanewline
\ \ \isacommand{then}\isamarkupfalse%
\ \isacommand{have}\isamarkupfalse%
\ {\isachardoublequoteopen}x\ {\isasymin}\ {\isacharbraceleft}{\kern0pt}\ x\ {\isasymin}\ P{\isacharunderscore}{\kern0pt}set{\isacharparenleft}{\kern0pt}succ{\isacharparenleft}{\kern0pt}L{\isacharparenright}{\kern0pt}{\isacharparenright}{\kern0pt}{\isachardot}{\kern0pt}\ domain{\isacharparenleft}{\kern0pt}x{\isacharparenright}{\kern0pt}\ {\isasymsubseteq}\ HS{\isacharunderscore}{\kern0pt}set{\isacharparenleft}{\kern0pt}L{\isacharparenright}{\kern0pt}\ {\isasymand}\ symmetric{\isacharparenleft}{\kern0pt}x{\isacharparenright}{\kern0pt}\ {\isacharbraceright}{\kern0pt}{\isachardoublequoteclose}\isanewline
\ \ \ \ \isacommand{using}\isamarkupfalse%
\ assms\ \isanewline
\ \ \ \ \isacommand{apply}\isamarkupfalse%
\ auto\ \isanewline
\ \ \ \ \isacommand{apply}\isamarkupfalse%
{\isacharparenleft}{\kern0pt}rule{\isacharunderscore}{\kern0pt}tac\ P{\isacharequal}{\kern0pt}{\isachardoublequoteopen}x\ {\isasymin}\ Pow{\isacharparenleft}{\kern0pt}P{\isacharunderscore}{\kern0pt}set{\isacharparenleft}{\kern0pt}L{\isacharparenright}{\kern0pt}\ {\isasymtimes}\ P{\isacharparenright}{\kern0pt}\ {\isasyminter}\ M{\isachardoublequoteclose}\ \isakeyword{in}\ mp{\isacharparenright}{\kern0pt}\isanewline
\ \ \ \ \isacommand{using}\isamarkupfalse%
\ P{\isacharunderscore}{\kern0pt}set{\isacharunderscore}{\kern0pt}succ\ P{\isacharunderscore}{\kern0pt}name{\isacharunderscore}{\kern0pt}in{\isacharunderscore}{\kern0pt}M\ \isanewline
\ \ \ \ \ \isacommand{apply}\isamarkupfalse%
\ auto\ \ \isanewline
\ \ \isacommand{proof}\isamarkupfalse%
\ {\isacharminus}{\kern0pt}\ \isanewline
\ \ \ \ \isacommand{fix}\isamarkupfalse%
\ v\ \isacommand{assume}\isamarkupfalse%
\ vin\ {\isacharcolon}{\kern0pt}\ {\isachardoublequoteopen}v\ {\isasymin}\ x{\isachardoublequoteclose}\ \isanewline
\ \ \ \ \isacommand{then}\isamarkupfalse%
\ \isacommand{obtain}\isamarkupfalse%
\ y\ p\ \isakeyword{where}\ ypH\ {\isacharcolon}{\kern0pt}\ {\isachardoublequoteopen}v\ {\isacharequal}{\kern0pt}\ {\isacharless}{\kern0pt}y{\isacharcomma}{\kern0pt}\ p{\isachargreater}{\kern0pt}{\isachardoublequoteclose}\ \isacommand{using}\isamarkupfalse%
\ assms\ relation{\isacharunderscore}{\kern0pt}P{\isacharunderscore}{\kern0pt}name\ \isacommand{unfolding}\isamarkupfalse%
\ relation{\isacharunderscore}{\kern0pt}def\ \isacommand{by}\isamarkupfalse%
\ blast\ \isanewline
\ \ \ \ \isacommand{then}\isamarkupfalse%
\ \isacommand{have}\isamarkupfalse%
\ pin{\isacharcolon}{\kern0pt}{\isachardoublequoteopen}p\ {\isasymin}\ P{\isachardoublequoteclose}\ \isacommand{using}\isamarkupfalse%
\ assms\ ypH\ vin\ P{\isacharunderscore}{\kern0pt}name{\isacharunderscore}{\kern0pt}range\ \isacommand{by}\isamarkupfalse%
\ auto\ \isanewline
\ \ \ \ \isacommand{have}\isamarkupfalse%
\ {\isachardoublequoteopen}y\ {\isasymin}\ P{\isacharunderscore}{\kern0pt}set{\isacharparenleft}{\kern0pt}L{\isacharparenright}{\kern0pt}{\isachardoublequoteclose}\ \isacommand{using}\isamarkupfalse%
\ H\ ypH\ vin\ HS{\isacharunderscore}{\kern0pt}set{\isacharunderscore}{\kern0pt}subset\ LH\ Limit{\isacharunderscore}{\kern0pt}is{\isacharunderscore}{\kern0pt}Ord\ \isacommand{by}\isamarkupfalse%
\ auto\ \isanewline
\ \ \ \ \isacommand{then}\isamarkupfalse%
\ \isacommand{show}\isamarkupfalse%
\ {\isachardoublequoteopen}v\ {\isasymin}\ P{\isacharunderscore}{\kern0pt}set{\isacharparenleft}{\kern0pt}L{\isacharparenright}{\kern0pt}\ {\isasymtimes}\ P{\isachardoublequoteclose}\ \isacommand{using}\isamarkupfalse%
\ pin\ ypH\ \isacommand{by}\isamarkupfalse%
\ auto\ \isanewline
\ \ \isacommand{qed}\isamarkupfalse%
\isanewline
\ \ \isacommand{then}\isamarkupfalse%
\ \isacommand{have}\isamarkupfalse%
\ {\isachardoublequoteopen}x\ {\isasymin}\ HS{\isacharunderscore}{\kern0pt}set{\isacharparenleft}{\kern0pt}succ{\isacharparenleft}{\kern0pt}L{\isacharparenright}{\kern0pt}{\isacharparenright}{\kern0pt}{\isachardoublequoteclose}\ \isanewline
\ \ \ \ \isacommand{using}\isamarkupfalse%
\ HS{\isacharunderscore}{\kern0pt}set{\isacharunderscore}{\kern0pt}succ\ LH\ Limit{\isacharunderscore}{\kern0pt}is{\isacharunderscore}{\kern0pt}Ord\ \isacommand{by}\isamarkupfalse%
\ auto\ \isanewline
\ \ \isacommand{then}\isamarkupfalse%
\ \isacommand{show}\isamarkupfalse%
\ {\isachardoublequoteopen}x\ {\isasymin}\ HS{\isachardoublequoteclose}\ \isanewline
\ \ \ \ \isacommand{unfolding}\isamarkupfalse%
\ HS{\isacharunderscore}{\kern0pt}def\ \isanewline
\ \ \ \ \isacommand{using}\isamarkupfalse%
\ assms\ \isanewline
\ \ \ \ \isacommand{apply}\isamarkupfalse%
\ simp\ \isanewline
\ \ \ \ \isacommand{apply}\isamarkupfalse%
{\isacharparenleft}{\kern0pt}rule{\isacharunderscore}{\kern0pt}tac\ x{\isacharequal}{\kern0pt}{\isachardoublequoteopen}succ{\isacharparenleft}{\kern0pt}L{\isacharparenright}{\kern0pt}{\isachardoublequoteclose}\ \isakeyword{in}\ exI{\isacharparenright}{\kern0pt}\ \isanewline
\ \ \ \ \isacommand{using}\isamarkupfalse%
\ LH\ Limit{\isacharunderscore}{\kern0pt}is{\isacharunderscore}{\kern0pt}Ord\ \isanewline
\ \ \ \ \isacommand{by}\isamarkupfalse%
\ auto\isanewline
\isacommand{qed}\isamarkupfalse%
%
\endisatagproof
{\isafoldproof}%
%
\isadelimproof
\isanewline
%
\endisadelimproof
\isanewline
\isacommand{lemma}\isamarkupfalse%
\ {\isasymG}{\isacharunderscore}{\kern0pt}is{\isacharunderscore}{\kern0pt}subgroup{\isacharunderscore}{\kern0pt}of{\isacharunderscore}{\kern0pt}{\isasymG}\ {\isacharcolon}{\kern0pt}\ {\isachardoublequoteopen}{\isasymG}\ {\isasymin}\ P{\isacharunderscore}{\kern0pt}auto{\isacharunderscore}{\kern0pt}subgroups{\isacharparenleft}{\kern0pt}{\isasymG}{\isacharparenright}{\kern0pt}{\isachardoublequoteclose}\ \isanewline
%
\isadelimproof
\ \ %
\endisadelimproof
%
\isatagproof
\isacommand{unfolding}\isamarkupfalse%
\ P{\isacharunderscore}{\kern0pt}auto{\isacharunderscore}{\kern0pt}subgroups{\isacharunderscore}{\kern0pt}def\ \isanewline
\ \ \isacommand{using}\isamarkupfalse%
\ {\isasymG}{\isacharunderscore}{\kern0pt}in{\isacharunderscore}{\kern0pt}M\ local{\isachardot}{\kern0pt}{\isasymG}{\isacharunderscore}{\kern0pt}P{\isacharunderscore}{\kern0pt}auto{\isacharunderscore}{\kern0pt}group\ \isanewline
\ \ \isacommand{by}\isamarkupfalse%
\ auto%
\endisatagproof
{\isafoldproof}%
%
\isadelimproof
\isanewline
%
\endisadelimproof
\isanewline
\isacommand{lemma}\isamarkupfalse%
\ {\isasymG}{\isacharunderscore}{\kern0pt}in{\isacharunderscore}{\kern0pt}{\isasymF}\ {\isacharcolon}{\kern0pt}\ {\isachardoublequoteopen}{\isasymG}\ {\isasymin}\ {\isasymF}{\isachardoublequoteclose}\ \isanewline
%
\isadelimproof
%
\endisadelimproof
%
\isatagproof
\isacommand{proof}\isamarkupfalse%
\ {\isacharminus}{\kern0pt}\ \isanewline
\ \ \isacommand{obtain}\isamarkupfalse%
\ {\isasymI}\ \isakeyword{where}\ H{\isacharcolon}{\kern0pt}\ {\isachardoublequoteopen}{\isasymI}\ {\isasymin}\ {\isasymF}{\isachardoublequoteclose}\ {\isachardoublequoteopen}{\isasymI}\ {\isasymin}\ P{\isacharunderscore}{\kern0pt}auto{\isacharunderscore}{\kern0pt}subgroups{\isacharparenleft}{\kern0pt}{\isasymG}{\isacharparenright}{\kern0pt}{\isachardoublequoteclose}\ \ \isacommand{using}\isamarkupfalse%
\ {\isasymF}{\isacharunderscore}{\kern0pt}nonempty\ {\isasymF}{\isacharunderscore}{\kern0pt}subset\ \isacommand{by}\isamarkupfalse%
\ auto\ \isanewline
\ \ \isacommand{then}\isamarkupfalse%
\ \isacommand{have}\isamarkupfalse%
\ {\isachardoublequoteopen}{\isasymI}\ {\isasymsubseteq}\ {\isasymG}{\isachardoublequoteclose}\ \isacommand{using}\isamarkupfalse%
\ P{\isacharunderscore}{\kern0pt}auto{\isacharunderscore}{\kern0pt}subgroups{\isacharunderscore}{\kern0pt}def\ \isacommand{by}\isamarkupfalse%
\ auto\ \isanewline
\ \ \isacommand{then}\isamarkupfalse%
\ \isacommand{show}\isamarkupfalse%
\ {\isachardoublequoteopen}{\isasymG}\ {\isasymin}\ {\isasymF}{\isachardoublequoteclose}\ \isacommand{using}\isamarkupfalse%
\ {\isasymF}{\isacharunderscore}{\kern0pt}closed{\isacharunderscore}{\kern0pt}under{\isacharunderscore}{\kern0pt}supergroup\ {\isasymG}{\isacharunderscore}{\kern0pt}is{\isacharunderscore}{\kern0pt}subgroup{\isacharunderscore}{\kern0pt}of{\isacharunderscore}{\kern0pt}{\isasymG}\ H\ \isacommand{by}\isamarkupfalse%
\ auto\isanewline
\isacommand{qed}\isamarkupfalse%
%
\endisatagproof
{\isafoldproof}%
%
\isadelimproof
\isanewline
%
\endisadelimproof
\isanewline
\isacommand{end}\isamarkupfalse%
\isanewline
%
\isadelimtheory
%
\endisadelimtheory
%
\isatagtheory
\isacommand{end}\isamarkupfalse%
%
\endisatagtheory
{\isafoldtheory}%
%
\isadelimtheory
%
\endisadelimtheory
%
\end{isabellebody}%
\endinput
%:%file=~/source/repos/ZF-notAC/code/HS_Definition.thy%:%
%:%10=1%:%
%:%11=1%:%
%:%12=2%:%
%:%13=3%:%
%:%14=4%:%
%:%15=5%:%
%:%20=5%:%
%:%23=6%:%
%:%24=7%:%
%:%25=7%:%
%:%26=8%:%
%:%27=9%:%
%:%28=10%:%
%:%29=10%:%
%:%30=11%:%
%:%33=14%:%
%:%34=15%:%
%:%35=16%:%
%:%36=16%:%
%:%37=17%:%
%:%38=18%:%
%:%39=18%:%
%:%40=19%:%
%:%41=20%:%
%:%42=20%:%
%:%43=21%:%
%:%44=22%:%
%:%45=23%:%
%:%46=24%:%
%:%47=25%:%
%:%48=26%:%
%:%49=27%:%
%:%50=28%:%
%:%51=29%:%
%:%52=30%:%
%:%53=31%:%
%:%54=32%:%
%:%55=32%:%
%:%56=33%:%
%:%57=34%:%
%:%58=34%:%
%:%59=35%:%
%:%60=36%:%
%:%61=36%:%
%:%62=37%:%
%:%63=38%:%
%:%64=38%:%
%:%65=39%:%
%:%66=39%:%
%:%67=40%:%
%:%68=41%:%
%:%69=41%:%
%:%72=42%:%
%:%76=42%:%
%:%77=42%:%
%:%78=43%:%
%:%79=43%:%
%:%80=44%:%
%:%81=44%:%
%:%82=45%:%
%:%83=45%:%
%:%84=46%:%
%:%85=46%:%
%:%86=47%:%
%:%87=47%:%
%:%88=48%:%
%:%89=48%:%
%:%90=49%:%
%:%91=49%:%
%:%92=50%:%
%:%93=50%:%
%:%94=50%:%
%:%95=51%:%
%:%96=51%:%
%:%97=52%:%
%:%98=52%:%
%:%99=52%:%
%:%100=52%:%
%:%101=53%:%
%:%102=53%:%
%:%103=53%:%
%:%104=53%:%
%:%105=53%:%
%:%106=54%:%
%:%107=54%:%
%:%108=54%:%
%:%109=54%:%
%:%110=54%:%
%:%111=55%:%
%:%112=55%:%
%:%113=55%:%
%:%114=55%:%
%:%115=55%:%
%:%116=56%:%
%:%117=56%:%
%:%118=56%:%
%:%119=56%:%
%:%120=56%:%
%:%121=57%:%
%:%127=57%:%
%:%130=58%:%
%:%131=59%:%
%:%132=59%:%
%:%139=60%:%
%:%140=60%:%
%:%141=61%:%
%:%142=61%:%
%:%143=62%:%
%:%144=62%:%
%:%145=62%:%
%:%146=62%:%
%:%147=62%:%
%:%148=63%:%
%:%149=63%:%
%:%150=63%:%
%:%151=63%:%
%:%152=63%:%
%:%153=64%:%
%:%159=64%:%
%:%162=65%:%
%:%163=66%:%
%:%164=66%:%
%:%167=67%:%
%:%171=67%:%
%:%172=67%:%
%:%173=68%:%
%:%174=68%:%
%:%175=69%:%
%:%176=69%:%
%:%177=70%:%
%:%178=70%:%
%:%179=71%:%
%:%180=71%:%
%:%185=71%:%
%:%188=72%:%
%:%189=73%:%
%:%190=73%:%
%:%193=74%:%
%:%197=74%:%
%:%198=74%:%
%:%199=75%:%
%:%200=75%:%
%:%201=76%:%
%:%202=76%:%
%:%203=77%:%
%:%204=77%:%
%:%205=78%:%
%:%206=78%:%
%:%207=79%:%
%:%208=79%:%
%:%209=80%:%
%:%210=80%:%
%:%211=81%:%
%:%212=81%:%
%:%213=82%:%
%:%214=82%:%
%:%215=83%:%
%:%216=83%:%
%:%217=83%:%
%:%218=84%:%
%:%219=84%:%
%:%220=84%:%
%:%221=84%:%
%:%222=84%:%
%:%223=84%:%
%:%224=85%:%
%:%225=85%:%
%:%226=85%:%
%:%227=85%:%
%:%228=85%:%
%:%229=86%:%
%:%230=86%:%
%:%231=86%:%
%:%232=86%:%
%:%233=86%:%
%:%234=87%:%
%:%235=87%:%
%:%236=87%:%
%:%237=87%:%
%:%238=87%:%
%:%239=88%:%
%:%245=88%:%
%:%248=89%:%
%:%249=90%:%
%:%250=90%:%
%:%253=91%:%
%:%257=91%:%
%:%258=91%:%
%:%259=92%:%
%:%260=92%:%
%:%261=93%:%
%:%262=93%:%
%:%263=94%:%
%:%264=94%:%
%:%265=94%:%
%:%266=95%:%
%:%267=96%:%
%:%268=96%:%
%:%269=96%:%
%:%270=96%:%
%:%271=97%:%
%:%272=97%:%
%:%273=97%:%
%:%274=97%:%
%:%275=97%:%
%:%276=98%:%
%:%277=98%:%
%:%278=98%:%
%:%279=99%:%
%:%280=99%:%
%:%281=100%:%
%:%282=100%:%
%:%283=101%:%
%:%284=101%:%
%:%285=102%:%
%:%286=102%:%
%:%287=103%:%
%:%288=103%:%
%:%289=104%:%
%:%290=104%:%
%:%291=105%:%
%:%292=105%:%
%:%293=106%:%
%:%294=106%:%
%:%295=106%:%
%:%296=106%:%
%:%297=106%:%
%:%298=106%:%
%:%299=106%:%
%:%300=106%:%
%:%301=107%:%
%:%302=107%:%
%:%303=108%:%
%:%304=108%:%
%:%305=109%:%
%:%306=109%:%
%:%307=109%:%
%:%308=110%:%
%:%309=111%:%
%:%310=112%:%
%:%311=112%:%
%:%312=113%:%
%:%313=113%:%
%:%314=113%:%
%:%315=114%:%
%:%316=114%:%
%:%317=115%:%
%:%318=115%:%
%:%319=116%:%
%:%320=116%:%
%:%321=116%:%
%:%322=117%:%
%:%323=117%:%
%:%324=117%:%
%:%325=118%:%
%:%326=118%:%
%:%327=118%:%
%:%328=118%:%
%:%329=118%:%
%:%330=118%:%
%:%331=119%:%
%:%332=119%:%
%:%333=119%:%
%:%334=120%:%
%:%335=120%:%
%:%336=121%:%
%:%337=121%:%
%:%338=121%:%
%:%339=122%:%
%:%340=122%:%
%:%341=122%:%
%:%342=122%:%
%:%343=122%:%
%:%344=123%:%
%:%345=123%:%
%:%346=123%:%
%:%347=123%:%
%:%348=123%:%
%:%349=124%:%
%:%350=124%:%
%:%351=124%:%
%:%352=125%:%
%:%353=125%:%
%:%354=126%:%
%:%355=126%:%
%:%356=127%:%
%:%357=128%:%
%:%358=128%:%
%:%359=129%:%
%:%360=129%:%
%:%361=130%:%
%:%362=130%:%
%:%363=131%:%
%:%364=131%:%
%:%365=132%:%
%:%366=132%:%
%:%367=133%:%
%:%368=133%:%
%:%369=133%:%
%:%370=133%:%
%:%371=133%:%
%:%372=134%:%
%:%373=134%:%
%:%374=134%:%
%:%375=135%:%
%:%376=135%:%
%:%377=136%:%
%:%378=136%:%
%:%379=137%:%
%:%380=137%:%
%:%381=138%:%
%:%382=138%:%
%:%383=138%:%
%:%384=139%:%
%:%385=139%:%
%:%386=140%:%
%:%387=140%:%
%:%388=141%:%
%:%389=141%:%
%:%390=142%:%
%:%391=142%:%
%:%392=143%:%
%:%393=143%:%
%:%394=143%:%
%:%395=144%:%
%:%396=144%:%
%:%397=145%:%
%:%398=145%:%
%:%399=146%:%
%:%400=146%:%
%:%401=147%:%
%:%402=147%:%
%:%403=148%:%
%:%404=148%:%
%:%405=149%:%
%:%406=149%:%
%:%407=150%:%
%:%408=150%:%
%:%409=150%:%
%:%410=151%:%
%:%411=151%:%
%:%412=151%:%
%:%413=151%:%
%:%414=151%:%
%:%415=151%:%
%:%416=152%:%
%:%417=152%:%
%:%418=152%:%
%:%419=152%:%
%:%420=152%:%
%:%421=153%:%
%:%422=153%:%
%:%423=153%:%
%:%424=153%:%
%:%425=154%:%
%:%426=154%:%
%:%427=154%:%
%:%428=154%:%
%:%429=154%:%
%:%430=155%:%
%:%431=155%:%
%:%432=156%:%
%:%433=156%:%
%:%434=156%:%
%:%435=157%:%
%:%436=157%:%
%:%437=157%:%
%:%438=158%:%
%:%439=158%:%
%:%440=158%:%
%:%441=159%:%
%:%442=159%:%
%:%443=160%:%
%:%444=160%:%
%:%445=161%:%
%:%446=161%:%
%:%447=162%:%
%:%448=162%:%
%:%449=163%:%
%:%450=163%:%
%:%451=164%:%
%:%452=164%:%
%:%453=165%:%
%:%459=165%:%
%:%462=166%:%
%:%463=167%:%
%:%464=167%:%
%:%467=168%:%
%:%471=168%:%
%:%472=168%:%
%:%473=169%:%
%:%474=169%:%
%:%475=170%:%
%:%476=170%:%
%:%481=170%:%
%:%484=171%:%
%:%485=172%:%
%:%486=172%:%
%:%493=173%:%
%:%494=173%:%
%:%495=174%:%
%:%496=174%:%
%:%497=174%:%
%:%498=174%:%
%:%499=175%:%
%:%500=175%:%
%:%501=175%:%
%:%502=175%:%
%:%503=175%:%
%:%504=176%:%
%:%505=176%:%
%:%506=176%:%
%:%507=176%:%
%:%508=176%:%
%:%509=177%:%
%:%515=177%:%
%:%518=178%:%
%:%519=179%:%
%:%520=179%:%
%:%527=180%:%

%
\begin{isabellebody}%
\setisabellecontext{HS{\isacharunderscore}{\kern0pt}M}%
%
\isadelimtheory
%
\endisadelimtheory
%
\isatagtheory
\isacommand{theory}\isamarkupfalse%
\ HS{\isacharunderscore}{\kern0pt}M\isanewline
\ \ \isakeyword{imports}\ \isanewline
\ \ \ \ {\isachardoublequoteopen}Forcing{\isacharslash}{\kern0pt}Forcing{\isacharunderscore}{\kern0pt}Main{\isachardoublequoteclose}\ \isanewline
\ \ \ \ HS{\isacharunderscore}{\kern0pt}Definition\isanewline
\isakeyword{begin}%
\endisatagtheory
{\isafoldtheory}%
%
\isadelimtheory
\ \isanewline
%
\endisadelimtheory
\isanewline
\isanewline
\isacommand{definition}\isamarkupfalse%
\ apply{\isacharunderscore}{\kern0pt}all{\isacharunderscore}{\kern0pt}{\isadigit{1}}{\isacharunderscore}{\kern0pt}fm\ \isakeyword{where}\ \isanewline
\ \ {\isachardoublequoteopen}apply{\isacharunderscore}{\kern0pt}all{\isacharunderscore}{\kern0pt}{\isadigit{1}}{\isacharunderscore}{\kern0pt}fm{\isacharparenleft}{\kern0pt}g{\isacharcomma}{\kern0pt}\ s{\isacharcomma}{\kern0pt}\ param{\isacharparenright}{\kern0pt}\ {\isasymequiv}\ \isanewline
\ \ \ \ Forall{\isacharparenleft}{\kern0pt}Implies{\isacharparenleft}{\kern0pt}Member{\isacharparenleft}{\kern0pt}{\isadigit{0}}{\isacharcomma}{\kern0pt}\ s\ {\isacharhash}{\kern0pt}{\isacharplus}{\kern0pt}\ {\isadigit{1}}{\isacharparenright}{\kern0pt}{\isacharcomma}{\kern0pt}\ Exists{\isacharparenleft}{\kern0pt}Exists{\isacharparenleft}{\kern0pt}And{\isacharparenleft}{\kern0pt}pair{\isacharunderscore}{\kern0pt}fm{\isacharparenleft}{\kern0pt}{\isadigit{2}}{\isacharcomma}{\kern0pt}\ param\ {\isacharhash}{\kern0pt}{\isacharplus}{\kern0pt}\ {\isadigit{3}}{\isacharcomma}{\kern0pt}\ {\isadigit{1}}{\isacharparenright}{\kern0pt}{\isacharcomma}{\kern0pt}\ And{\isacharparenleft}{\kern0pt}fun{\isacharunderscore}{\kern0pt}apply{\isacharunderscore}{\kern0pt}fm{\isacharparenleft}{\kern0pt}g\ {\isacharhash}{\kern0pt}{\isacharplus}{\kern0pt}\ {\isadigit{3}}{\isacharcomma}{\kern0pt}\ {\isadigit{1}}{\isacharcomma}{\kern0pt}\ {\isadigit{0}}{\isacharparenright}{\kern0pt}{\isacharcomma}{\kern0pt}\ is{\isacharunderscore}{\kern0pt}{\isadigit{1}}{\isacharunderscore}{\kern0pt}fm{\isacharparenleft}{\kern0pt}{\isadigit{0}}{\isacharparenright}{\kern0pt}{\isacharparenright}{\kern0pt}{\isacharparenright}{\kern0pt}{\isacharparenright}{\kern0pt}{\isacharparenright}{\kern0pt}{\isacharparenright}{\kern0pt}{\isacharparenright}{\kern0pt}{\isachardoublequoteclose}\ \isanewline
\isanewline
\isacommand{context}\isamarkupfalse%
\ M{\isacharunderscore}{\kern0pt}symmetric{\isacharunderscore}{\kern0pt}system\isanewline
\isakeyword{begin}\isanewline
\isanewline
\isacommand{lemma}\isamarkupfalse%
\ apply{\isacharunderscore}{\kern0pt}all{\isacharunderscore}{\kern0pt}{\isadigit{1}}{\isacharunderscore}{\kern0pt}fm{\isacharunderscore}{\kern0pt}type\ {\isacharcolon}{\kern0pt}\ \isanewline
\ \ \isakeyword{fixes}\ g\ s\ param\isanewline
\ \ \isakeyword{assumes}\ {\isachardoublequoteopen}g\ {\isasymin}\ nat{\isachardoublequoteclose}\ {\isachardoublequoteopen}s\ {\isasymin}\ nat{\isachardoublequoteclose}\ {\isachardoublequoteopen}param\ {\isasymin}\ nat{\isachardoublequoteclose}\ \isanewline
\ \ \isakeyword{shows}\ {\isachardoublequoteopen}apply{\isacharunderscore}{\kern0pt}all{\isacharunderscore}{\kern0pt}{\isadigit{1}}{\isacharunderscore}{\kern0pt}fm{\isacharparenleft}{\kern0pt}g{\isacharcomma}{\kern0pt}\ s{\isacharcomma}{\kern0pt}\ param{\isacharparenright}{\kern0pt}\ {\isasymin}\ formula{\isachardoublequoteclose}\ \isanewline
%
\isadelimproof
\ \ %
\endisadelimproof
%
\isatagproof
\isacommand{unfolding}\isamarkupfalse%
\ apply{\isacharunderscore}{\kern0pt}all{\isacharunderscore}{\kern0pt}{\isadigit{1}}{\isacharunderscore}{\kern0pt}fm{\isacharunderscore}{\kern0pt}def\ \isanewline
\ \ \isacommand{apply}\isamarkupfalse%
{\isacharparenleft}{\kern0pt}clarify{\isacharcomma}{\kern0pt}\ rule\ Implies{\isacharunderscore}{\kern0pt}type{\isacharcomma}{\kern0pt}\ simp{\isacharcomma}{\kern0pt}\ rule\ Exists{\isacharunderscore}{\kern0pt}type{\isacharcomma}{\kern0pt}\ rule\ Exists{\isacharunderscore}{\kern0pt}type{\isacharcomma}{\kern0pt}\ rule\ And{\isacharunderscore}{\kern0pt}type{\isacharparenright}{\kern0pt}\isanewline
\ \ \ \isacommand{apply}\isamarkupfalse%
\ simp\isanewline
\ \ \isacommand{apply}\isamarkupfalse%
{\isacharparenleft}{\kern0pt}rule\ And{\isacharunderscore}{\kern0pt}type{\isacharparenright}{\kern0pt}\isanewline
\ \ \isacommand{using}\isamarkupfalse%
\ is{\isacharunderscore}{\kern0pt}{\isadigit{1}}{\isacharunderscore}{\kern0pt}fm{\isacharunderscore}{\kern0pt}type\ assms\ \isanewline
\ \ \isacommand{by}\isamarkupfalse%
\ auto%
\endisatagproof
{\isafoldproof}%
%
\isadelimproof
\isanewline
%
\endisadelimproof
\isanewline
\isacommand{lemma}\isamarkupfalse%
\ arity{\isacharunderscore}{\kern0pt}apply{\isacharunderscore}{\kern0pt}all{\isacharunderscore}{\kern0pt}{\isadigit{1}}{\isacharunderscore}{\kern0pt}fm\ {\isacharcolon}{\kern0pt}\isanewline
\ \ \isakeyword{fixes}\ g\ s\ param\isanewline
\ \ \isakeyword{assumes}\ {\isachardoublequoteopen}g\ {\isasymin}\ nat{\isachardoublequoteclose}\ {\isachardoublequoteopen}s\ {\isasymin}\ nat{\isachardoublequoteclose}\ {\isachardoublequoteopen}param\ {\isasymin}\ nat{\isachardoublequoteclose}\isanewline
\ \ \isakeyword{shows}\ {\isachardoublequoteopen}arity{\isacharparenleft}{\kern0pt}apply{\isacharunderscore}{\kern0pt}all{\isacharunderscore}{\kern0pt}{\isadigit{1}}{\isacharunderscore}{\kern0pt}fm{\isacharparenleft}{\kern0pt}g{\isacharcomma}{\kern0pt}\ s{\isacharcomma}{\kern0pt}\ param{\isacharparenright}{\kern0pt}{\isacharparenright}{\kern0pt}\ {\isasymle}\ succ{\isacharparenleft}{\kern0pt}g{\isacharparenright}{\kern0pt}\ {\isasymunion}\ succ{\isacharparenleft}{\kern0pt}s{\isacharparenright}{\kern0pt}\ {\isasymunion}\ succ{\isacharparenleft}{\kern0pt}param{\isacharparenright}{\kern0pt}{\isachardoublequoteclose}\isanewline
%
\isadelimproof
\ \ %
\endisadelimproof
%
\isatagproof
\isacommand{apply}\isamarkupfalse%
{\isacharparenleft}{\kern0pt}subgoal{\isacharunderscore}{\kern0pt}tac\ {\isachardoublequoteopen}is{\isacharunderscore}{\kern0pt}{\isadigit{1}}{\isacharunderscore}{\kern0pt}fm{\isacharparenleft}{\kern0pt}{\isadigit{0}}{\isacharparenright}{\kern0pt}\ {\isasymin}\ formula{\isachardoublequoteclose}{\isacharparenright}{\kern0pt}\isanewline
\ \ \isacommand{unfolding}\isamarkupfalse%
\ apply{\isacharunderscore}{\kern0pt}all{\isacharunderscore}{\kern0pt}{\isadigit{1}}{\isacharunderscore}{\kern0pt}fm{\isacharunderscore}{\kern0pt}def\ \isanewline
\ \ \ \isacommand{apply}\isamarkupfalse%
\ simp\ \isanewline
\ \ \isacommand{using}\isamarkupfalse%
\ assms\isanewline
\ \ \ \isacommand{apply}\isamarkupfalse%
{\isacharparenleft}{\kern0pt}subst\ pred{\isacharunderscore}{\kern0pt}Un{\isacharunderscore}{\kern0pt}distrib{\isacharcomma}{\kern0pt}\ simp{\isacharunderscore}{\kern0pt}all{\isacharparenright}{\kern0pt}{\isacharplus}{\kern0pt}\isanewline
\ \ \ \isacommand{apply}\isamarkupfalse%
{\isacharparenleft}{\kern0pt}rule\ Un{\isacharunderscore}{\kern0pt}least{\isacharunderscore}{\kern0pt}lt{\isacharcomma}{\kern0pt}\ simp{\isacharcomma}{\kern0pt}\ rule\ ltI{\isacharcomma}{\kern0pt}\ simp{\isacharcomma}{\kern0pt}\ simp\ add{\isacharcolon}{\kern0pt}assms{\isacharparenright}{\kern0pt}\isanewline
\ \ \ \isacommand{apply}\isamarkupfalse%
{\isacharparenleft}{\kern0pt}rule\ Un{\isacharunderscore}{\kern0pt}least{\isacharunderscore}{\kern0pt}lt{\isacharcomma}{\kern0pt}\ rule{\isacharunderscore}{\kern0pt}tac\ j{\isacharequal}{\kern0pt}{\isachardoublequoteopen}succ{\isacharparenleft}{\kern0pt}param{\isacharparenright}{\kern0pt}{\isachardoublequoteclose}\ \isakeyword{in}\ le{\isacharunderscore}{\kern0pt}trans{\isacharparenright}{\kern0pt}\isanewline
\ \ \ \ \ \isacommand{apply}\isamarkupfalse%
{\isacharparenleft}{\kern0pt}rule\ pred{\isacharunderscore}{\kern0pt}le{\isacharcomma}{\kern0pt}\ simp{\isacharunderscore}{\kern0pt}all{\isacharparenright}{\kern0pt}{\isacharplus}{\kern0pt}\isanewline
\ \ \ \ \ \isacommand{apply}\isamarkupfalse%
{\isacharparenleft}{\kern0pt}subst\ arity{\isacharunderscore}{\kern0pt}pair{\isacharunderscore}{\kern0pt}fm{\isacharparenright}{\kern0pt}\isanewline
\ \ \isacommand{apply}\isamarkupfalse%
\ auto{\isacharbrackleft}{\kern0pt}{\isadigit{4}}{\isacharbrackright}{\kern0pt}\isanewline
\ \ \ \ \ \isacommand{apply}\isamarkupfalse%
{\isacharparenleft}{\kern0pt}rule\ Un{\isacharunderscore}{\kern0pt}least{\isacharunderscore}{\kern0pt}lt{\isacharparenright}{\kern0pt}{\isacharplus}{\kern0pt}\isanewline
\ \ \ \ \ \ \isacommand{apply}\isamarkupfalse%
\ auto{\isacharbrackleft}{\kern0pt}{\isadigit{1}}{\isacharbrackright}{\kern0pt}\isanewline
\ \ \ \ \ \isacommand{apply}\isamarkupfalse%
{\isacharparenleft}{\kern0pt}rule\ Un{\isacharunderscore}{\kern0pt}least{\isacharunderscore}{\kern0pt}lt{\isacharcomma}{\kern0pt}\ simp{\isacharcomma}{\kern0pt}\ simp{\isacharparenright}{\kern0pt}\isanewline
\ \ \ \ \isacommand{apply}\isamarkupfalse%
{\isacharparenleft}{\kern0pt}rule\ ltI{\isacharcomma}{\kern0pt}\ simp{\isacharcomma}{\kern0pt}\ simp{\isacharparenright}{\kern0pt}\isanewline
\ \ \ \ \ \isacommand{apply}\isamarkupfalse%
{\isacharparenleft}{\kern0pt}rule\ Un{\isacharunderscore}{\kern0pt}least{\isacharunderscore}{\kern0pt}lt{\isacharparenright}{\kern0pt}{\isacharplus}{\kern0pt}\isanewline
\ \ \ \ \ \isacommand{apply}\isamarkupfalse%
{\isacharparenleft}{\kern0pt}subst\ arity{\isacharunderscore}{\kern0pt}fun{\isacharunderscore}{\kern0pt}apply{\isacharunderscore}{\kern0pt}fm{\isacharparenright}{\kern0pt}\isanewline
\ \ \isacommand{using}\isamarkupfalse%
\ assms\ \isanewline
\ \ \ \ \ \ \ \ \isacommand{apply}\isamarkupfalse%
\ auto{\isacharbrackleft}{\kern0pt}{\isadigit{3}}{\isacharbrackright}{\kern0pt}\isanewline
\ \ \ \ \isacommand{apply}\isamarkupfalse%
{\isacharparenleft}{\kern0pt}rule{\isacharunderscore}{\kern0pt}tac\ j{\isacharequal}{\kern0pt}{\isachardoublequoteopen}succ{\isacharparenleft}{\kern0pt}g{\isacharparenright}{\kern0pt}{\isachardoublequoteclose}\ \isakeyword{in}\ le{\isacharunderscore}{\kern0pt}trans{\isacharparenright}{\kern0pt}\isanewline
\ \ \ \ \ \isacommand{apply}\isamarkupfalse%
{\isacharparenleft}{\kern0pt}rule\ pred{\isacharunderscore}{\kern0pt}le{\isacharcomma}{\kern0pt}\ simp{\isacharunderscore}{\kern0pt}all{\isacharparenright}{\kern0pt}{\isacharplus}{\kern0pt}\isanewline
\ \ \ \ \ \isacommand{apply}\isamarkupfalse%
{\isacharparenleft}{\kern0pt}rule\ Un{\isacharunderscore}{\kern0pt}least{\isacharunderscore}{\kern0pt}lt{\isacharparenright}{\kern0pt}{\isacharplus}{\kern0pt}\isanewline
\ \ \ \ \ \ \ \isacommand{apply}\isamarkupfalse%
{\isacharparenleft}{\kern0pt}simp{\isacharcomma}{\kern0pt}\ simp{\isacharcomma}{\kern0pt}\ simp{\isacharparenright}{\kern0pt}\isanewline
\ \ \ \ \isacommand{apply}\isamarkupfalse%
{\isacharparenleft}{\kern0pt}rule\ ltI{\isacharcomma}{\kern0pt}\ simp\ {\isacharcomma}{\kern0pt}\ simp{\isacharparenright}{\kern0pt}\isanewline
\ \ \ \isacommand{apply}\isamarkupfalse%
{\isacharparenleft}{\kern0pt}subst\ arity{\isacharunderscore}{\kern0pt}is{\isacharunderscore}{\kern0pt}{\isadigit{1}}{\isacharunderscore}{\kern0pt}fm{\isacharcomma}{\kern0pt}\ simp{\isacharparenright}{\kern0pt}\isanewline
\ \ \ \isacommand{apply}\isamarkupfalse%
\ simp\isanewline
\ \ \isacommand{apply}\isamarkupfalse%
{\isacharparenleft}{\kern0pt}rule\ is{\isacharunderscore}{\kern0pt}{\isadigit{1}}{\isacharunderscore}{\kern0pt}fm{\isacharunderscore}{\kern0pt}type{\isacharparenright}{\kern0pt}\isanewline
\ \ \isacommand{by}\isamarkupfalse%
\ auto%
\endisatagproof
{\isafoldproof}%
%
\isadelimproof
\isanewline
%
\endisadelimproof
\isanewline
\isacommand{lemma}\isamarkupfalse%
\ sats{\isacharunderscore}{\kern0pt}apply{\isacharunderscore}{\kern0pt}all{\isacharunderscore}{\kern0pt}{\isadigit{1}}{\isacharunderscore}{\kern0pt}fm{\isacharunderscore}{\kern0pt}iff\ {\isacharcolon}{\kern0pt}\isanewline
\ \ \isakeyword{fixes}\ env\ i\ j\ k\ g\ s\ param\isanewline
\ \ \isakeyword{assumes}\ {\isachardoublequoteopen}i\ {\isacharless}{\kern0pt}\ length{\isacharparenleft}{\kern0pt}env{\isacharparenright}{\kern0pt}{\isachardoublequoteclose}\ {\isachardoublequoteopen}j\ {\isacharless}{\kern0pt}\ length{\isacharparenleft}{\kern0pt}env{\isacharparenright}{\kern0pt}{\isachardoublequoteclose}\ {\isachardoublequoteopen}k\ {\isacharless}{\kern0pt}\ length{\isacharparenleft}{\kern0pt}env{\isacharparenright}{\kern0pt}{\isachardoublequoteclose}\ {\isachardoublequoteopen}env\ {\isasymin}\ list{\isacharparenleft}{\kern0pt}M{\isacharparenright}{\kern0pt}{\isachardoublequoteclose}\ {\isachardoublequoteopen}nth{\isacharparenleft}{\kern0pt}i{\isacharcomma}{\kern0pt}\ env{\isacharparenright}{\kern0pt}\ {\isacharequal}{\kern0pt}\ g{\isachardoublequoteclose}\ {\isachardoublequoteopen}nth{\isacharparenleft}{\kern0pt}j{\isacharcomma}{\kern0pt}\ env{\isacharparenright}{\kern0pt}\ {\isacharequal}{\kern0pt}\ s{\isachardoublequoteclose}\ {\isachardoublequoteopen}nth{\isacharparenleft}{\kern0pt}k{\isacharcomma}{\kern0pt}\ env{\isacharparenright}{\kern0pt}\ {\isacharequal}{\kern0pt}\ param{\isachardoublequoteclose}\ \isanewline
\ \ \isakeyword{shows}\ {\isachardoublequoteopen}sats{\isacharparenleft}{\kern0pt}M{\isacharcomma}{\kern0pt}\ apply{\isacharunderscore}{\kern0pt}all{\isacharunderscore}{\kern0pt}{\isadigit{1}}{\isacharunderscore}{\kern0pt}fm{\isacharparenleft}{\kern0pt}i{\isacharcomma}{\kern0pt}\ j{\isacharcomma}{\kern0pt}\ k{\isacharparenright}{\kern0pt}{\isacharcomma}{\kern0pt}\ env{\isacharparenright}{\kern0pt}\ {\isasymlongleftrightarrow}\ {\isacharparenleft}{\kern0pt}{\isasymforall}y{\isachardot}{\kern0pt}\ y\ {\isasymin}\ s\ {\isasymlongrightarrow}\ g{\isacharbackquote}{\kern0pt}{\isacharless}{\kern0pt}y{\isacharcomma}{\kern0pt}\ param{\isachargreater}{\kern0pt}\ {\isacharequal}{\kern0pt}\ {\isadigit{1}}{\isacharparenright}{\kern0pt}{\isachardoublequoteclose}\ \isanewline
%
\isadelimproof
%
\endisadelimproof
%
\isatagproof
\isacommand{proof}\isamarkupfalse%
\ {\isacharminus}{\kern0pt}\ \isanewline
\ \ \isacommand{have}\isamarkupfalse%
\ {\isachardoublequoteopen}sats{\isacharparenleft}{\kern0pt}M{\isacharcomma}{\kern0pt}\ apply{\isacharunderscore}{\kern0pt}all{\isacharunderscore}{\kern0pt}{\isadigit{1}}{\isacharunderscore}{\kern0pt}fm{\isacharparenleft}{\kern0pt}i{\isacharcomma}{\kern0pt}\ j{\isacharcomma}{\kern0pt}\ k{\isacharparenright}{\kern0pt}{\isacharcomma}{\kern0pt}\ env{\isacharparenright}{\kern0pt}\ {\isasymlongleftrightarrow}\ {\isacharparenleft}{\kern0pt}{\isasymforall}y\ {\isasymin}\ M{\isachardot}{\kern0pt}\ y\ {\isasymin}\ s\ {\isasymlongrightarrow}\ g{\isacharbackquote}{\kern0pt}{\isacharless}{\kern0pt}y{\isacharcomma}{\kern0pt}\ param{\isachargreater}{\kern0pt}\ {\isacharequal}{\kern0pt}\ {\isadigit{1}}{\isacharparenright}{\kern0pt}{\isachardoublequoteclose}\ \isanewline
\ \ \ \ \isacommand{unfolding}\isamarkupfalse%
\ apply{\isacharunderscore}{\kern0pt}all{\isacharunderscore}{\kern0pt}{\isadigit{1}}{\isacharunderscore}{\kern0pt}fm{\isacharunderscore}{\kern0pt}def\ \isanewline
\ \ \ \ \isacommand{apply}\isamarkupfalse%
{\isacharparenleft}{\kern0pt}rule\ iff{\isacharunderscore}{\kern0pt}trans{\isacharcomma}{\kern0pt}\ rule\ sats{\isacharunderscore}{\kern0pt}Forall{\isacharunderscore}{\kern0pt}iff{\isacharcomma}{\kern0pt}\ simp\ add{\isacharcolon}{\kern0pt}assms{\isacharparenright}{\kern0pt}\isanewline
\ \ \ \ \isacommand{apply}\isamarkupfalse%
{\isacharparenleft}{\kern0pt}rule\ ball{\isacharunderscore}{\kern0pt}iff{\isacharparenright}{\kern0pt}\isanewline
\ \ \ \ \isacommand{apply}\isamarkupfalse%
{\isacharparenleft}{\kern0pt}rule\ iff{\isacharunderscore}{\kern0pt}trans{\isacharcomma}{\kern0pt}\ rule\ sats{\isacharunderscore}{\kern0pt}Implies{\isacharunderscore}{\kern0pt}iff{\isacharcomma}{\kern0pt}\ simp\ add{\isacharcolon}{\kern0pt}assms{\isacharparenright}{\kern0pt}\isanewline
\ \ \ \ \isacommand{apply}\isamarkupfalse%
{\isacharparenleft}{\kern0pt}rule\ imp{\isacharunderscore}{\kern0pt}iff{\isacharparenright}{\kern0pt}\isanewline
\ \ \ \ \isacommand{apply}\isamarkupfalse%
{\isacharparenleft}{\kern0pt}subgoal{\isacharunderscore}{\kern0pt}tac\ {\isachardoublequoteopen}j\ {\isasymin}\ nat{\isachardoublequoteclose}{\isacharparenright}{\kern0pt}\isanewline
\ \ \ \ \isacommand{using}\isamarkupfalse%
\ assms\ \isanewline
\ \ \ \ \ \ \isacommand{apply}\isamarkupfalse%
\ simp\isanewline
\ \ \ \ \isacommand{using}\isamarkupfalse%
\ assms\ lt{\isacharunderscore}{\kern0pt}nat{\isacharunderscore}{\kern0pt}in{\isacharunderscore}{\kern0pt}nat\ \isanewline
\ \ \ \ \ \isacommand{apply}\isamarkupfalse%
\ force\isanewline
\ \ \ \ \isacommand{apply}\isamarkupfalse%
{\isacharparenleft}{\kern0pt}rename{\isacharunderscore}{\kern0pt}tac\ x{\isacharcomma}{\kern0pt}\ rule{\isacharunderscore}{\kern0pt}tac\ Q{\isacharequal}{\kern0pt}\ {\isachardoublequoteopen}{\isasymexists}a\ {\isasymin}\ M{\isachardot}{\kern0pt}\ {\isasymexists}v\ {\isasymin}\ M{\isachardot}{\kern0pt}\ a\ {\isacharequal}{\kern0pt}\ {\isacharless}{\kern0pt}x{\isacharcomma}{\kern0pt}\ param{\isachargreater}{\kern0pt}\ {\isasymand}\ \ g{\isacharbackquote}{\kern0pt}a\ {\isacharequal}{\kern0pt}\ v\ {\isasymand}\ v\ {\isacharequal}{\kern0pt}\ {\isadigit{1}}{\isachardoublequoteclose}\ \isakeyword{in}\ iff{\isacharunderscore}{\kern0pt}trans{\isacharparenright}{\kern0pt}\isanewline
\ \ \ \ \ \isacommand{apply}\isamarkupfalse%
{\isacharparenleft}{\kern0pt}rule\ iff{\isacharunderscore}{\kern0pt}trans{\isacharcomma}{\kern0pt}\ rule\ sats{\isacharunderscore}{\kern0pt}Exists{\isacharunderscore}{\kern0pt}iff{\isacharcomma}{\kern0pt}\ simp\ add{\isacharcolon}{\kern0pt}assms{\isacharparenright}{\kern0pt}\isanewline
\ \ \ \ \ \isacommand{apply}\isamarkupfalse%
{\isacharparenleft}{\kern0pt}rule\ bex{\isacharunderscore}{\kern0pt}iff{\isacharcomma}{\kern0pt}\ rule\ iff{\isacharunderscore}{\kern0pt}trans{\isacharcomma}{\kern0pt}\ rule\ sats{\isacharunderscore}{\kern0pt}Exists{\isacharunderscore}{\kern0pt}iff{\isacharcomma}{\kern0pt}\ simp\ add{\isacharcolon}{\kern0pt}assms{\isacharparenright}{\kern0pt}\isanewline
\ \ \ \ \ \isacommand{apply}\isamarkupfalse%
{\isacharparenleft}{\kern0pt}rule\ bex{\isacharunderscore}{\kern0pt}iff{\isacharcomma}{\kern0pt}\ rule\ iff{\isacharunderscore}{\kern0pt}trans{\isacharcomma}{\kern0pt}\ rule\ sats{\isacharunderscore}{\kern0pt}And{\isacharunderscore}{\kern0pt}iff{\isacharcomma}{\kern0pt}\ simp\ add{\isacharcolon}{\kern0pt}assms{\isacharparenright}{\kern0pt}\isanewline
\ \ \ \ \ \isacommand{apply}\isamarkupfalse%
{\isacharparenleft}{\kern0pt}rule\ iff{\isacharunderscore}{\kern0pt}conjI{\isacharparenright}{\kern0pt}\isanewline
\ \ \ \ \isacommand{apply}\isamarkupfalse%
{\isacharparenleft}{\kern0pt}subgoal{\isacharunderscore}{\kern0pt}tac\ {\isachardoublequoteopen}i\ {\isasymin}\ nat\ {\isasymand}\ k\ {\isasymin}\ nat\ {\isasymand}\ g\ {\isasymin}\ M{\isachardoublequoteclose}{\isacharparenright}{\kern0pt}\isanewline
\ \ \ \ \isacommand{using}\isamarkupfalse%
\ assms\ \isanewline
\ \ \ \ \ \ \ \isacommand{apply}\isamarkupfalse%
\ simp\isanewline
\ \ \ \ \ \ \isacommand{apply}\isamarkupfalse%
{\isacharparenleft}{\kern0pt}rule\ conjI{\isacharparenright}{\kern0pt}\isanewline
\ \ \ \ \isacommand{using}\isamarkupfalse%
\ assms\ lt{\isacharunderscore}{\kern0pt}nat{\isacharunderscore}{\kern0pt}in{\isacharunderscore}{\kern0pt}nat\ \isanewline
\ \ \ \ \ \ \ \isacommand{apply}\isamarkupfalse%
\ force\isanewline
\ \ \ \ \ \ \isacommand{apply}\isamarkupfalse%
{\isacharparenleft}{\kern0pt}rule\ conjI{\isacharparenright}{\kern0pt}\isanewline
\ \ \ \ \isacommand{using}\isamarkupfalse%
\ assms\ lt{\isacharunderscore}{\kern0pt}nat{\isacharunderscore}{\kern0pt}in{\isacharunderscore}{\kern0pt}nat\ \isanewline
\ \ \ \ \ \ \ \isacommand{apply}\isamarkupfalse%
\ force\isanewline
\ \ \ \ \isacommand{using}\isamarkupfalse%
\ assms\ nth{\isacharunderscore}{\kern0pt}type\ \isanewline
\ \ \ \ \ \ \isacommand{apply}\isamarkupfalse%
\ force\isanewline
\ \ \ \ \ \isacommand{apply}\isamarkupfalse%
{\isacharparenleft}{\kern0pt}rule\ iff{\isacharunderscore}{\kern0pt}trans{\isacharcomma}{\kern0pt}\ rule\ sats{\isacharunderscore}{\kern0pt}And{\isacharunderscore}{\kern0pt}iff{\isacharcomma}{\kern0pt}\ simp\ add{\isacharcolon}{\kern0pt}assms{\isacharparenright}{\kern0pt}\isanewline
\ \ \ \ \ \isacommand{apply}\isamarkupfalse%
{\isacharparenleft}{\kern0pt}rule\ iff{\isacharunderscore}{\kern0pt}conjI{\isacharparenright}{\kern0pt}\isanewline
\ \ \ \ \ \ \isacommand{apply}\isamarkupfalse%
{\isacharparenleft}{\kern0pt}rule\ iff{\isacharunderscore}{\kern0pt}trans{\isacharcomma}{\kern0pt}\ rule\ sats{\isacharunderscore}{\kern0pt}fun{\isacharunderscore}{\kern0pt}apply{\isacharunderscore}{\kern0pt}fm{\isacharparenright}{\kern0pt}\isanewline
\ \ \ \ \isacommand{using}\isamarkupfalse%
\ assms\ lt{\isacharunderscore}{\kern0pt}nat{\isacharunderscore}{\kern0pt}in{\isacharunderscore}{\kern0pt}nat\ \isanewline
\ \ \ \ \ \ \ \ \ \ \isacommand{apply}\isamarkupfalse%
\ auto{\isacharbrackleft}{\kern0pt}{\isadigit{5}}{\isacharbrackright}{\kern0pt}\isanewline
\ \ \ \ \ \isacommand{apply}\isamarkupfalse%
{\isacharparenleft}{\kern0pt}rule\ sats{\isacharunderscore}{\kern0pt}is{\isacharunderscore}{\kern0pt}{\isadigit{1}}{\isacharunderscore}{\kern0pt}fm{\isacharunderscore}{\kern0pt}iff{\isacharparenright}{\kern0pt}\isanewline
\ \ \ \ \isacommand{using}\isamarkupfalse%
\ assms\isanewline
\ \ \ \ \ \ \ \isacommand{apply}\isamarkupfalse%
\ auto{\isacharbrackleft}{\kern0pt}{\isadigit{3}}{\isacharbrackright}{\kern0pt}\isanewline
\ \ \ \ \isacommand{apply}\isamarkupfalse%
{\isacharparenleft}{\kern0pt}rule\ iffI{\isacharcomma}{\kern0pt}\ simp{\isacharparenright}{\kern0pt}\isanewline
\ \ \ \ \isacommand{apply}\isamarkupfalse%
{\isacharparenleft}{\kern0pt}rename{\isacharunderscore}{\kern0pt}tac\ x{\isacharcomma}{\kern0pt}\ rule{\isacharunderscore}{\kern0pt}tac\ x{\isacharequal}{\kern0pt}{\isachardoublequoteopen}{\isacharless}{\kern0pt}x{\isacharcomma}{\kern0pt}\ param{\isachargreater}{\kern0pt}{\isachardoublequoteclose}\ \isakeyword{in}\ bexI{\isacharcomma}{\kern0pt}\ simp{\isacharparenright}{\kern0pt}\isanewline
\ \ \ \ \isacommand{using}\isamarkupfalse%
\ assms\ nth{\isacharunderscore}{\kern0pt}type\ pair{\isacharunderscore}{\kern0pt}in{\isacharunderscore}{\kern0pt}M{\isacharunderscore}{\kern0pt}iff\isanewline
\ \ \ \ \isacommand{by}\isamarkupfalse%
\ auto\isanewline
\ \ \isacommand{also}\isamarkupfalse%
\ \isacommand{have}\isamarkupfalse%
\ {\isachardoublequoteopen}{\isachardot}{\kern0pt}{\isachardot}{\kern0pt}{\isachardot}{\kern0pt}\ {\isasymlongleftrightarrow}\ {\isacharparenleft}{\kern0pt}{\isasymforall}y{\isachardot}{\kern0pt}\ y\ {\isasymin}\ s\ {\isasymlongrightarrow}\ g{\isacharbackquote}{\kern0pt}{\isacharless}{\kern0pt}y{\isacharcomma}{\kern0pt}\ param{\isachargreater}{\kern0pt}\ {\isacharequal}{\kern0pt}\ {\isadigit{1}}{\isacharparenright}{\kern0pt}{\isachardoublequoteclose}\ \isanewline
\ \ \ \ \isacommand{apply}\isamarkupfalse%
{\isacharparenleft}{\kern0pt}rule\ iffI{\isacharcomma}{\kern0pt}\ clarify{\isacharparenright}{\kern0pt}\isanewline
\ \ \ \ \ \isacommand{apply}\isamarkupfalse%
{\isacharparenleft}{\kern0pt}rename{\isacharunderscore}{\kern0pt}tac\ y{\isacharcomma}{\kern0pt}\ subgoal{\isacharunderscore}{\kern0pt}tac\ {\isachardoublequoteopen}y\ {\isasymin}\ M{\isachardoublequoteclose}{\isacharcomma}{\kern0pt}\ force{\isacharparenright}{\kern0pt}\isanewline
\ \ \ \ \ \isacommand{apply}\isamarkupfalse%
{\isacharparenleft}{\kern0pt}subgoal{\isacharunderscore}{\kern0pt}tac\ {\isachardoublequoteopen}s\ {\isasymin}\ M{\isachardoublequoteclose}{\isacharparenright}{\kern0pt}\isanewline
\ \ \ \ \isacommand{using}\isamarkupfalse%
\ transM\ \isanewline
\ \ \ \ \ \ \isacommand{apply}\isamarkupfalse%
\ force\isanewline
\ \ \ \ \isacommand{using}\isamarkupfalse%
\ assms\ nth{\isacharunderscore}{\kern0pt}type\ \isanewline
\ \ \ \ \isacommand{by}\isamarkupfalse%
\ auto\ \isanewline
\ \ \isacommand{finally}\isamarkupfalse%
\ \isacommand{show}\isamarkupfalse%
\ {\isacharquery}{\kern0pt}thesis\ \isacommand{by}\isamarkupfalse%
\ auto\isanewline
\isacommand{qed}\isamarkupfalse%
%
\endisatagproof
{\isafoldproof}%
%
\isadelimproof
\isanewline
%
\endisadelimproof
\isanewline
\isacommand{end}\isamarkupfalse%
\isanewline
\isanewline
\isacommand{definition}\isamarkupfalse%
\ is{\isacharunderscore}{\kern0pt}sym{\isacharunderscore}{\kern0pt}elem{\isacharunderscore}{\kern0pt}fm\ \isakeyword{where}\isanewline
\ \ {\isachardoublequoteopen}is{\isacharunderscore}{\kern0pt}sym{\isacharunderscore}{\kern0pt}elem{\isacharunderscore}{\kern0pt}fm{\isacharparenleft}{\kern0pt}p{\isacharcomma}{\kern0pt}\ G{\isacharcomma}{\kern0pt}\ x{\isacharcomma}{\kern0pt}\ v{\isacharparenright}{\kern0pt}\ {\isasymequiv}\ And{\isacharparenleft}{\kern0pt}Member{\isacharparenleft}{\kern0pt}v{\isacharcomma}{\kern0pt}\ G{\isacharparenright}{\kern0pt}{\isacharcomma}{\kern0pt}\ Exists{\isacharparenleft}{\kern0pt}And{\isacharparenleft}{\kern0pt}is{\isacharunderscore}{\kern0pt}Pn{\isacharunderscore}{\kern0pt}auto{\isacharunderscore}{\kern0pt}fm{\isacharparenleft}{\kern0pt}p\ {\isacharhash}{\kern0pt}{\isacharplus}{\kern0pt}\ {\isadigit{1}}{\isacharcomma}{\kern0pt}\ v\ {\isacharhash}{\kern0pt}{\isacharplus}{\kern0pt}\ {\isadigit{1}}{\isacharcomma}{\kern0pt}\ x\ {\isacharhash}{\kern0pt}{\isacharplus}{\kern0pt}\ {\isadigit{1}}{\isacharcomma}{\kern0pt}\ {\isadigit{0}}{\isacharparenright}{\kern0pt}{\isacharcomma}{\kern0pt}\ Equal{\isacharparenleft}{\kern0pt}{\isadigit{0}}{\isacharcomma}{\kern0pt}\ x\ {\isacharhash}{\kern0pt}{\isacharplus}{\kern0pt}\ {\isadigit{1}}{\isacharparenright}{\kern0pt}{\isacharparenright}{\kern0pt}{\isacharparenright}{\kern0pt}{\isacharparenright}{\kern0pt}{\isachardoublequoteclose}\ \isanewline
\isanewline
\isanewline
\isacommand{context}\isamarkupfalse%
\ M{\isacharunderscore}{\kern0pt}symmetric{\isacharunderscore}{\kern0pt}system\isanewline
\isakeyword{begin}\isanewline
\isanewline
\isacommand{lemma}\isamarkupfalse%
\ is{\isacharunderscore}{\kern0pt}sym{\isacharunderscore}{\kern0pt}elem{\isacharunderscore}{\kern0pt}fm{\isacharunderscore}{\kern0pt}type\ {\isacharcolon}{\kern0pt}\ \ \ \ \ \ \ \isanewline
\ \ \isakeyword{fixes}\ p\ G\ x\ v\isanewline
\ \ \isakeyword{assumes}\ {\isachardoublequoteopen}p\ {\isasymin}\ nat{\isachardoublequoteclose}\ {\isachardoublequoteopen}G\ {\isasymin}\ nat{\isachardoublequoteclose}\ {\isachardoublequoteopen}x\ {\isasymin}\ nat{\isachardoublequoteclose}\ {\isachardoublequoteopen}v\ {\isasymin}\ nat{\isachardoublequoteclose}\ \isanewline
\ \ \isakeyword{shows}\ {\isachardoublequoteopen}is{\isacharunderscore}{\kern0pt}sym{\isacharunderscore}{\kern0pt}elem{\isacharunderscore}{\kern0pt}fm{\isacharparenleft}{\kern0pt}p{\isacharcomma}{\kern0pt}\ G{\isacharcomma}{\kern0pt}\ x{\isacharcomma}{\kern0pt}\ v{\isacharparenright}{\kern0pt}\ {\isasymin}\ formula{\isachardoublequoteclose}\ \isanewline
%
\isadelimproof
\isanewline
\ \ %
\endisadelimproof
%
\isatagproof
\isacommand{unfolding}\isamarkupfalse%
\ is{\isacharunderscore}{\kern0pt}sym{\isacharunderscore}{\kern0pt}elem{\isacharunderscore}{\kern0pt}fm{\isacharunderscore}{\kern0pt}def\ \isanewline
\ \ \isacommand{apply}\isamarkupfalse%
{\isacharparenleft}{\kern0pt}rule\ And{\isacharunderscore}{\kern0pt}type{\isacharparenright}{\kern0pt}\isanewline
\ \ \isacommand{using}\isamarkupfalse%
\ assms\ \isanewline
\ \ \ \isacommand{apply}\isamarkupfalse%
\ simp\isanewline
\ \ \isacommand{apply}\isamarkupfalse%
{\isacharparenleft}{\kern0pt}rule\ Exists{\isacharunderscore}{\kern0pt}type{\isacharcomma}{\kern0pt}\ rule\ And{\isacharunderscore}{\kern0pt}type{\isacharcomma}{\kern0pt}\ rule\ is{\isacharunderscore}{\kern0pt}Pn{\isacharunderscore}{\kern0pt}auto{\isacharunderscore}{\kern0pt}fm{\isacharunderscore}{\kern0pt}type{\isacharparenright}{\kern0pt}\isanewline
\ \ \isacommand{using}\isamarkupfalse%
\ assms\isanewline
\ \ \isacommand{by}\isamarkupfalse%
\ auto%
\endisatagproof
{\isafoldproof}%
%
\isadelimproof
\isanewline
%
\endisadelimproof
\isanewline
\isacommand{lemma}\isamarkupfalse%
\ arity{\isacharunderscore}{\kern0pt}is{\isacharunderscore}{\kern0pt}sym{\isacharunderscore}{\kern0pt}elem{\isacharunderscore}{\kern0pt}fm\ {\isacharcolon}{\kern0pt}\ \ \ \ \isanewline
\ \ \isakeyword{fixes}\ p\ G\ x\ v\isanewline
\ \ \isakeyword{assumes}\ {\isachardoublequoteopen}p\ {\isasymin}\ nat{\isachardoublequoteclose}\ {\isachardoublequoteopen}G\ {\isasymin}\ nat{\isachardoublequoteclose}\ {\isachardoublequoteopen}x\ {\isasymin}\ nat{\isachardoublequoteclose}\ {\isachardoublequoteopen}v\ {\isasymin}\ nat{\isachardoublequoteclose}\ \isanewline
\ \ \isakeyword{shows}\ {\isachardoublequoteopen}arity{\isacharparenleft}{\kern0pt}is{\isacharunderscore}{\kern0pt}sym{\isacharunderscore}{\kern0pt}elem{\isacharunderscore}{\kern0pt}fm{\isacharparenleft}{\kern0pt}p{\isacharcomma}{\kern0pt}\ G{\isacharcomma}{\kern0pt}\ x{\isacharcomma}{\kern0pt}\ v{\isacharparenright}{\kern0pt}{\isacharparenright}{\kern0pt}\ {\isasymle}\ \ succ{\isacharparenleft}{\kern0pt}p{\isacharparenright}{\kern0pt}\ {\isasymunion}\ succ{\isacharparenleft}{\kern0pt}G{\isacharparenright}{\kern0pt}\ {\isasymunion}\ succ{\isacharparenleft}{\kern0pt}x{\isacharparenright}{\kern0pt}\ {\isasymunion}\ succ{\isacharparenleft}{\kern0pt}v{\isacharparenright}{\kern0pt}{\isachardoublequoteclose}\isanewline
%
\isadelimproof
\isanewline
\ \ %
\endisadelimproof
%
\isatagproof
\isacommand{apply}\isamarkupfalse%
{\isacharparenleft}{\kern0pt}subgoal{\isacharunderscore}{\kern0pt}tac\ {\isachardoublequoteopen}is{\isacharunderscore}{\kern0pt}Pn{\isacharunderscore}{\kern0pt}auto{\isacharunderscore}{\kern0pt}fm{\isacharparenleft}{\kern0pt}succ{\isacharparenleft}{\kern0pt}p{\isacharparenright}{\kern0pt}{\isacharcomma}{\kern0pt}\ succ{\isacharparenleft}{\kern0pt}v{\isacharparenright}{\kern0pt}{\isacharcomma}{\kern0pt}\ succ{\isacharparenleft}{\kern0pt}x{\isacharparenright}{\kern0pt}{\isacharcomma}{\kern0pt}\ {\isadigit{0}}{\isacharparenright}{\kern0pt}\ {\isasymin}\ formula{\isachardoublequoteclose}{\isacharparenright}{\kern0pt}\isanewline
\ \ \isacommand{unfolding}\isamarkupfalse%
\ is{\isacharunderscore}{\kern0pt}sym{\isacharunderscore}{\kern0pt}elem{\isacharunderscore}{\kern0pt}fm{\isacharunderscore}{\kern0pt}def\isanewline
\ \ \isacommand{apply}\isamarkupfalse%
\ simp\ \isanewline
\ \ \isacommand{apply}\isamarkupfalse%
{\isacharparenleft}{\kern0pt}rule\ Un{\isacharunderscore}{\kern0pt}least{\isacharunderscore}{\kern0pt}lt{\isacharparenright}{\kern0pt}{\isacharplus}{\kern0pt}\isanewline
\ \ \ \ \isacommand{apply}\isamarkupfalse%
{\isacharparenleft}{\kern0pt}rule\ ltI{\isacharparenright}{\kern0pt}\isanewline
\ \ \isacommand{using}\isamarkupfalse%
\ assms\isanewline
\ \ \ \ \ \isacommand{apply}\isamarkupfalse%
{\isacharparenleft}{\kern0pt}subst\ succ{\isacharunderscore}{\kern0pt}Un{\isacharunderscore}{\kern0pt}distrib{\isacharcomma}{\kern0pt}\ simp{\isacharunderscore}{\kern0pt}all{\isacharparenright}{\kern0pt}{\isacharplus}{\kern0pt}\isanewline
\ \ \ \isacommand{apply}\isamarkupfalse%
{\isacharparenleft}{\kern0pt}rule\ ltI{\isacharcomma}{\kern0pt}\ simp{\isacharcomma}{\kern0pt}\ simp{\isacharparenright}{\kern0pt}\isanewline
\ \ \isacommand{apply}\isamarkupfalse%
{\isacharparenleft}{\kern0pt}rule\ pred{\isacharunderscore}{\kern0pt}le{\isacharcomma}{\kern0pt}\ simp{\isacharcomma}{\kern0pt}\ simp{\isacharparenright}{\kern0pt}\isanewline
\ \ \ \isacommand{apply}\isamarkupfalse%
{\isacharparenleft}{\kern0pt}rule\ Un{\isacharunderscore}{\kern0pt}least{\isacharunderscore}{\kern0pt}lt{\isacharparenright}{\kern0pt}{\isacharplus}{\kern0pt}\isanewline
\ \ \ \ \isacommand{apply}\isamarkupfalse%
{\isacharparenleft}{\kern0pt}rule\ le{\isacharunderscore}{\kern0pt}trans{\isacharcomma}{\kern0pt}\ rule\ arity{\isacharunderscore}{\kern0pt}is{\isacharunderscore}{\kern0pt}Pn{\isacharunderscore}{\kern0pt}auto{\isacharunderscore}{\kern0pt}fm{\isacharparenright}{\kern0pt}\isanewline
\ \ \ \ \ \ \ \ \isacommand{apply}\isamarkupfalse%
\ auto{\isacharbrackleft}{\kern0pt}{\isadigit{4}}{\isacharbrackright}{\kern0pt}\isanewline
\ \ \ \ \isacommand{apply}\isamarkupfalse%
{\isacharparenleft}{\kern0pt}subst\ succ{\isacharunderscore}{\kern0pt}Un{\isacharunderscore}{\kern0pt}distrib{\isacharcomma}{\kern0pt}\ simp{\isacharunderscore}{\kern0pt}all{\isacharparenright}{\kern0pt}{\isacharplus}{\kern0pt}\isanewline
\ \ \ \isacommand{apply}\isamarkupfalse%
{\isacharparenleft}{\kern0pt}rule\ Un{\isacharunderscore}{\kern0pt}least{\isacharunderscore}{\kern0pt}lt{\isacharparenright}{\kern0pt}{\isacharplus}{\kern0pt}\isanewline
\ \ \ \ \ \ \ \isacommand{apply}\isamarkupfalse%
{\isacharparenleft}{\kern0pt}rule\ ltI{\isacharcomma}{\kern0pt}\ simp{\isacharunderscore}{\kern0pt}all{\isacharparenright}{\kern0pt}{\isacharplus}{\kern0pt}\isanewline
\ \ \ \ \isacommand{apply}\isamarkupfalse%
{\isacharparenleft}{\kern0pt}rule\ disjI{\isadigit{1}}{\isacharcomma}{\kern0pt}\ rule\ ltD{\isacharcomma}{\kern0pt}\ simp{\isacharparenright}{\kern0pt}\isanewline
\ \ \isacommand{apply}\isamarkupfalse%
{\isacharparenleft}{\kern0pt}rule\ Un{\isacharunderscore}{\kern0pt}least{\isacharunderscore}{\kern0pt}lt{\isacharcomma}{\kern0pt}\ simp{\isacharparenright}{\kern0pt}\isanewline
\ \ \ \isacommand{apply}\isamarkupfalse%
{\isacharparenleft}{\kern0pt}subst\ succ{\isacharunderscore}{\kern0pt}Un{\isacharunderscore}{\kern0pt}distrib{\isacharcomma}{\kern0pt}\ simp{\isacharunderscore}{\kern0pt}all{\isacharparenright}{\kern0pt}{\isacharplus}{\kern0pt}\isanewline
\ \ \ \isacommand{apply}\isamarkupfalse%
{\isacharparenleft}{\kern0pt}rule\ ltI{\isacharcomma}{\kern0pt}\ simp{\isacharcomma}{\kern0pt}\ simp{\isacharparenright}{\kern0pt}\isanewline
\ \ \isacommand{apply}\isamarkupfalse%
{\isacharparenleft}{\kern0pt}rule\ is{\isacharunderscore}{\kern0pt}Pn{\isacharunderscore}{\kern0pt}auto{\isacharunderscore}{\kern0pt}fm{\isacharunderscore}{\kern0pt}type{\isacharparenright}{\kern0pt}\isanewline
\ \ \isacommand{by}\isamarkupfalse%
\ auto%
\endisatagproof
{\isafoldproof}%
%
\isadelimproof
\isanewline
%
\endisadelimproof
\isanewline
\isacommand{lemma}\isamarkupfalse%
\ sats{\isacharunderscore}{\kern0pt}is{\isacharunderscore}{\kern0pt}sym{\isacharunderscore}{\kern0pt}elem{\isacharunderscore}{\kern0pt}fm{\isacharunderscore}{\kern0pt}iff\ {\isacharcolon}{\kern0pt}\ \isanewline
\ \ \isakeyword{fixes}\ env\ p\ G\ x\ v\ i\ j\ k\ l\ \isanewline
\ \ \isakeyword{assumes}\ {\isachardoublequoteopen}env\ {\isasymin}\ list{\isacharparenleft}{\kern0pt}M{\isacharparenright}{\kern0pt}{\isachardoublequoteclose}\ {\isachardoublequoteopen}i\ {\isacharless}{\kern0pt}\ length{\isacharparenleft}{\kern0pt}env{\isacharparenright}{\kern0pt}{\isachardoublequoteclose}\ {\isachardoublequoteopen}j\ {\isacharless}{\kern0pt}\ length{\isacharparenleft}{\kern0pt}env{\isacharparenright}{\kern0pt}{\isachardoublequoteclose}\ {\isachardoublequoteopen}k\ {\isacharless}{\kern0pt}\ length{\isacharparenleft}{\kern0pt}env{\isacharparenright}{\kern0pt}{\isachardoublequoteclose}\ {\isachardoublequoteopen}l\ {\isacharless}{\kern0pt}\ length{\isacharparenleft}{\kern0pt}env{\isacharparenright}{\kern0pt}{\isachardoublequoteclose}\isanewline
\ \ \ \ \ \ \ \ \ \ {\isachardoublequoteopen}nth{\isacharparenleft}{\kern0pt}i{\isacharcomma}{\kern0pt}\ env{\isacharparenright}{\kern0pt}\ {\isacharequal}{\kern0pt}\ P{\isachardoublequoteclose}\ {\isachardoublequoteopen}nth{\isacharparenleft}{\kern0pt}j{\isacharcomma}{\kern0pt}\ env{\isacharparenright}{\kern0pt}\ {\isacharequal}{\kern0pt}\ {\isasymG}{\isachardoublequoteclose}\ {\isachardoublequoteopen}nth{\isacharparenleft}{\kern0pt}k{\isacharcomma}{\kern0pt}\ env{\isacharparenright}{\kern0pt}\ {\isacharequal}{\kern0pt}\ x{\isachardoublequoteclose}\ {\isachardoublequoteopen}nth{\isacharparenleft}{\kern0pt}l{\isacharcomma}{\kern0pt}\ env{\isacharparenright}{\kern0pt}\ {\isacharequal}{\kern0pt}\ v{\isachardoublequoteclose}\ \isanewline
\ \ \ \ \ \ \ \ \ \ {\isachardoublequoteopen}x\ {\isasymin}\ P{\isacharunderscore}{\kern0pt}names{\isachardoublequoteclose}\ \isanewline
\ \ \isakeyword{shows}\ {\isachardoublequoteopen}sats{\isacharparenleft}{\kern0pt}M{\isacharcomma}{\kern0pt}\ is{\isacharunderscore}{\kern0pt}sym{\isacharunderscore}{\kern0pt}elem{\isacharunderscore}{\kern0pt}fm{\isacharparenleft}{\kern0pt}i{\isacharcomma}{\kern0pt}\ j{\isacharcomma}{\kern0pt}\ k{\isacharcomma}{\kern0pt}\ l{\isacharparenright}{\kern0pt}{\isacharcomma}{\kern0pt}\ env{\isacharparenright}{\kern0pt}\ {\isasymlongleftrightarrow}\ v\ {\isasymin}\ sym{\isacharparenleft}{\kern0pt}x{\isacharparenright}{\kern0pt}{\isachardoublequoteclose}\ \isanewline
%
\isadelimproof
%
\endisadelimproof
%
\isatagproof
\isacommand{proof}\isamarkupfalse%
\ {\isacharminus}{\kern0pt}\ \isanewline
\ \ \isacommand{have}\isamarkupfalse%
\ {\isachardoublequoteopen}{\isacharparenleft}{\kern0pt}M{\isacharcomma}{\kern0pt}\ env\ {\isasymTurnstile}\ is{\isacharunderscore}{\kern0pt}sym{\isacharunderscore}{\kern0pt}elem{\isacharunderscore}{\kern0pt}fm{\isacharparenleft}{\kern0pt}i{\isacharcomma}{\kern0pt}\ j{\isacharcomma}{\kern0pt}\ k{\isacharcomma}{\kern0pt}\ l{\isacharparenright}{\kern0pt}{\isacharparenright}{\kern0pt}\ {\isasymlongleftrightarrow}\ {\isacharparenleft}{\kern0pt}v\ {\isasymin}\ {\isasymG}\ {\isasymand}\ {\isacharparenleft}{\kern0pt}{\isasymexists}u\ {\isasymin}\ M{\isachardot}{\kern0pt}\ {\isacharparenleft}{\kern0pt}x\ {\isasymin}\ P{\isacharunderscore}{\kern0pt}names\ {\isasymand}\ u\ {\isacharequal}{\kern0pt}\ Pn{\isacharunderscore}{\kern0pt}auto{\isacharparenleft}{\kern0pt}v{\isacharparenright}{\kern0pt}{\isacharbackquote}{\kern0pt}x{\isacharparenright}{\kern0pt}\ {\isasymand}\ u\ {\isacharequal}{\kern0pt}\ x{\isacharparenright}{\kern0pt}{\isacharparenright}{\kern0pt}{\isachardoublequoteclose}\ \isanewline
\ \ \ \ \isacommand{unfolding}\isamarkupfalse%
\ is{\isacharunderscore}{\kern0pt}sym{\isacharunderscore}{\kern0pt}elem{\isacharunderscore}{\kern0pt}fm{\isacharunderscore}{\kern0pt}def\ \isanewline
\ \ \ \ \isacommand{apply}\isamarkupfalse%
{\isacharparenleft}{\kern0pt}rule\ iff{\isacharunderscore}{\kern0pt}trans{\isacharcomma}{\kern0pt}\ rule\ sats{\isacharunderscore}{\kern0pt}And{\isacharunderscore}{\kern0pt}iff{\isacharcomma}{\kern0pt}\ simp\ add{\isacharcolon}{\kern0pt}assms{\isacharcomma}{\kern0pt}\ rule\ iff{\isacharunderscore}{\kern0pt}conjI{\isadigit{2}}{\isacharcomma}{\kern0pt}\ simp\ add{\isacharcolon}{\kern0pt}assms{\isacharparenright}{\kern0pt}\isanewline
\ \ \ \ \isacommand{apply}\isamarkupfalse%
{\isacharparenleft}{\kern0pt}rule\ iff{\isacharunderscore}{\kern0pt}trans{\isacharcomma}{\kern0pt}\ rule\ sats{\isacharunderscore}{\kern0pt}Exists{\isacharunderscore}{\kern0pt}iff{\isacharcomma}{\kern0pt}\ simp\ add{\isacharcolon}{\kern0pt}assms{\isacharcomma}{\kern0pt}\ rule\ bex{\isacharunderscore}{\kern0pt}iff{\isacharparenright}{\kern0pt}\ \ \isanewline
\ \ \ \ \isacommand{apply}\isamarkupfalse%
{\isacharparenleft}{\kern0pt}rule\ iff{\isacharunderscore}{\kern0pt}trans{\isacharcomma}{\kern0pt}\ rule\ sats{\isacharunderscore}{\kern0pt}And{\isacharunderscore}{\kern0pt}iff{\isacharcomma}{\kern0pt}\ simp\ add{\isacharcolon}{\kern0pt}assms{\isacharcomma}{\kern0pt}\ rule\ iff{\isacharunderscore}{\kern0pt}conjI{\isadigit{2}}{\isacharparenright}{\kern0pt}\isanewline
\ \ \ \ \ \isacommand{apply}\isamarkupfalse%
{\isacharparenleft}{\kern0pt}rule\ sats{\isacharunderscore}{\kern0pt}is{\isacharunderscore}{\kern0pt}Pn{\isacharunderscore}{\kern0pt}auto{\isacharunderscore}{\kern0pt}fm{\isacharunderscore}{\kern0pt}iff{\isacharparenright}{\kern0pt}\ \isanewline
\ \ \ \ \isacommand{using}\isamarkupfalse%
\ assms\ lt{\isacharunderscore}{\kern0pt}nat{\isacharunderscore}{\kern0pt}in{\isacharunderscore}{\kern0pt}nat\ nth{\isacharunderscore}{\kern0pt}type\ \isacommand{apply}\isamarkupfalse%
\ auto{\isacharbrackleft}{\kern0pt}{\isadigit{9}}{\isacharbrackright}{\kern0pt}\isanewline
\ \ \ \ \isacommand{using}\isamarkupfalse%
\ {\isasymG}{\isacharunderscore}{\kern0pt}P{\isacharunderscore}{\kern0pt}auto{\isacharunderscore}{\kern0pt}group\ P{\isacharunderscore}{\kern0pt}auto{\isacharunderscore}{\kern0pt}def\ is{\isacharunderscore}{\kern0pt}P{\isacharunderscore}{\kern0pt}auto{\isacharunderscore}{\kern0pt}group{\isacharunderscore}{\kern0pt}def\ \isanewline
\ \ \ \ \ \isacommand{apply}\isamarkupfalse%
\ force\isanewline
\ \ \ \ \isacommand{apply}\isamarkupfalse%
{\isacharparenleft}{\kern0pt}subgoal{\isacharunderscore}{\kern0pt}tac\ {\isachardoublequoteopen}k\ {\isasymin}\ nat{\isachardoublequoteclose}{\isacharcomma}{\kern0pt}\ simp\ add{\isacharcolon}{\kern0pt}assms{\isacharparenright}{\kern0pt}\isanewline
\ \ \ \ \isacommand{using}\isamarkupfalse%
\ lt{\isacharunderscore}{\kern0pt}nat{\isacharunderscore}{\kern0pt}in{\isacharunderscore}{\kern0pt}nat\ assms\isanewline
\ \ \ \ \isacommand{by}\isamarkupfalse%
\ auto\ \isanewline
\ \ \isacommand{also}\isamarkupfalse%
\ \isacommand{have}\isamarkupfalse%
\ {\isachardoublequoteopen}{\isachardot}{\kern0pt}{\isachardot}{\kern0pt}{\isachardot}{\kern0pt}\ {\isasymlongleftrightarrow}\ {\isacharparenleft}{\kern0pt}v\ {\isasymin}\ {\isasymG}\ {\isasymand}\ Pn{\isacharunderscore}{\kern0pt}auto{\isacharparenleft}{\kern0pt}v{\isacharparenright}{\kern0pt}{\isacharbackquote}{\kern0pt}x\ {\isacharequal}{\kern0pt}\ x{\isacharparenright}{\kern0pt}{\isachardoublequoteclose}\ \isanewline
\ \ \ \ \isacommand{apply}\isamarkupfalse%
{\isacharparenleft}{\kern0pt}rule\ iffI{\isacharcomma}{\kern0pt}\ simp{\isacharparenright}{\kern0pt}\isanewline
\ \ \ \ \isacommand{apply}\isamarkupfalse%
{\isacharparenleft}{\kern0pt}rule\ conjI{\isacharcomma}{\kern0pt}\ simp{\isacharparenright}{\kern0pt}\isanewline
\ \ \ \ \isacommand{apply}\isamarkupfalse%
{\isacharparenleft}{\kern0pt}rule{\isacharunderscore}{\kern0pt}tac\ x{\isacharequal}{\kern0pt}x\ \isakeyword{in}\ bexI{\isacharparenright}{\kern0pt}\isanewline
\ \ \ \ \isacommand{using}\isamarkupfalse%
\ assms\ P{\isacharunderscore}{\kern0pt}name{\isacharunderscore}{\kern0pt}in{\isacharunderscore}{\kern0pt}M\ \isanewline
\ \ \ \ \isacommand{by}\isamarkupfalse%
\ auto\isanewline
\ \ \isacommand{also}\isamarkupfalse%
\ \isacommand{have}\isamarkupfalse%
\ {\isachardoublequoteopen}{\isachardot}{\kern0pt}{\isachardot}{\kern0pt}{\isachardot}{\kern0pt}\ {\isasymlongleftrightarrow}\ v\ {\isasymin}\ sym{\isacharparenleft}{\kern0pt}x{\isacharparenright}{\kern0pt}{\isachardoublequoteclose}\ \isanewline
\ \ \ \ \isacommand{unfolding}\isamarkupfalse%
\ sym{\isacharunderscore}{\kern0pt}def\ \isanewline
\ \ \ \ \isacommand{apply}\isamarkupfalse%
\ force\isanewline
\ \ \ \ \isacommand{done}\isamarkupfalse%
\isanewline
\ \ \isacommand{finally}\isamarkupfalse%
\ \isacommand{show}\isamarkupfalse%
\ {\isacharquery}{\kern0pt}thesis\ \isacommand{by}\isamarkupfalse%
\ simp\isanewline
\isacommand{qed}\isamarkupfalse%
%
\endisatagproof
{\isafoldproof}%
%
\isadelimproof
\isanewline
%
\endisadelimproof
\isacommand{end}\isamarkupfalse%
\isanewline
\isanewline
\isacommand{definition}\isamarkupfalse%
\ is{\isacharunderscore}{\kern0pt}sym{\isacharunderscore}{\kern0pt}fm\ \isakeyword{where}\ \isanewline
\ \ {\isachardoublequoteopen}is{\isacharunderscore}{\kern0pt}sym{\isacharunderscore}{\kern0pt}fm{\isacharparenleft}{\kern0pt}p{\isacharcomma}{\kern0pt}\ G{\isacharcomma}{\kern0pt}\ x{\isacharcomma}{\kern0pt}\ s{\isacharparenright}{\kern0pt}\ {\isasymequiv}\ \isanewline
\ \ \ \ Forall{\isacharparenleft}{\kern0pt}Iff{\isacharparenleft}{\kern0pt}Member{\isacharparenleft}{\kern0pt}{\isadigit{0}}{\isacharcomma}{\kern0pt}\ s\ {\isacharhash}{\kern0pt}{\isacharplus}{\kern0pt}\ {\isadigit{1}}{\isacharparenright}{\kern0pt}{\isacharcomma}{\kern0pt}\ is{\isacharunderscore}{\kern0pt}sym{\isacharunderscore}{\kern0pt}elem{\isacharunderscore}{\kern0pt}fm{\isacharparenleft}{\kern0pt}p\ {\isacharhash}{\kern0pt}{\isacharplus}{\kern0pt}\ {\isadigit{1}}{\isacharcomma}{\kern0pt}\ G\ {\isacharhash}{\kern0pt}{\isacharplus}{\kern0pt}\ {\isadigit{1}}{\isacharcomma}{\kern0pt}\ x\ {\isacharhash}{\kern0pt}{\isacharplus}{\kern0pt}\ {\isadigit{1}}{\isacharcomma}{\kern0pt}\ {\isadigit{0}}{\isacharparenright}{\kern0pt}{\isacharparenright}{\kern0pt}{\isacharparenright}{\kern0pt}{\isachardoublequoteclose}\ \isanewline
\isanewline
\isanewline
\isacommand{context}\isamarkupfalse%
\ M{\isacharunderscore}{\kern0pt}symmetric{\isacharunderscore}{\kern0pt}system\isanewline
\isakeyword{begin}\isanewline
\isanewline
\isacommand{lemma}\isamarkupfalse%
\ is{\isacharunderscore}{\kern0pt}sym{\isacharunderscore}{\kern0pt}fm{\isacharunderscore}{\kern0pt}type\ {\isacharcolon}{\kern0pt}\ \isanewline
\ \ \isakeyword{fixes}\ p\ G\ x\ s\ \isanewline
\ \ \isakeyword{assumes}\ {\isachardoublequoteopen}p\ {\isasymin}\ nat{\isachardoublequoteclose}\ {\isachardoublequoteopen}G\ {\isasymin}\ nat{\isachardoublequoteclose}\ {\isachardoublequoteopen}x\ {\isasymin}\ nat{\isachardoublequoteclose}\ {\isachardoublequoteopen}s\ {\isasymin}\ nat{\isachardoublequoteclose}\ \isanewline
\ \ \isakeyword{shows}\ {\isachardoublequoteopen}is{\isacharunderscore}{\kern0pt}sym{\isacharunderscore}{\kern0pt}fm{\isacharparenleft}{\kern0pt}p{\isacharcomma}{\kern0pt}\ G{\isacharcomma}{\kern0pt}\ x{\isacharcomma}{\kern0pt}\ s{\isacharparenright}{\kern0pt}\ {\isasymin}\ formula{\isachardoublequoteclose}\ \isanewline
%
\isadelimproof
\ \ %
\endisadelimproof
%
\isatagproof
\isacommand{unfolding}\isamarkupfalse%
\ is{\isacharunderscore}{\kern0pt}sym{\isacharunderscore}{\kern0pt}fm{\isacharunderscore}{\kern0pt}def\ \isanewline
\ \ \isacommand{apply}\isamarkupfalse%
{\isacharparenleft}{\kern0pt}clarify{\isacharcomma}{\kern0pt}\ rule\ Iff{\isacharunderscore}{\kern0pt}type{\isacharparenright}{\kern0pt}\isanewline
\ \ \isacommand{using}\isamarkupfalse%
\ assms\isanewline
\ \ \ \isacommand{apply}\isamarkupfalse%
\ force\isanewline
\ \ \isacommand{apply}\isamarkupfalse%
{\isacharparenleft}{\kern0pt}rule\ is{\isacharunderscore}{\kern0pt}sym{\isacharunderscore}{\kern0pt}elem{\isacharunderscore}{\kern0pt}fm{\isacharunderscore}{\kern0pt}type{\isacharparenright}{\kern0pt}\isanewline
\ \ \isacommand{using}\isamarkupfalse%
\ assms\isanewline
\ \ \isacommand{by}\isamarkupfalse%
\ auto%
\endisatagproof
{\isafoldproof}%
%
\isadelimproof
\isanewline
%
\endisadelimproof
\isanewline
\isacommand{lemma}\isamarkupfalse%
\ arity{\isacharunderscore}{\kern0pt}is{\isacharunderscore}{\kern0pt}sym{\isacharunderscore}{\kern0pt}fm\ {\isacharcolon}{\kern0pt}\ \isanewline
\ \ \isakeyword{fixes}\ p\ G\ x\ s\ \isanewline
\ \ \isakeyword{assumes}\ {\isachardoublequoteopen}p\ {\isasymin}\ nat{\isachardoublequoteclose}\ {\isachardoublequoteopen}G\ {\isasymin}\ nat{\isachardoublequoteclose}\ {\isachardoublequoteopen}x\ {\isasymin}\ nat{\isachardoublequoteclose}\ {\isachardoublequoteopen}s\ {\isasymin}\ nat{\isachardoublequoteclose}\ \isanewline
\ \ \isakeyword{shows}\ {\isachardoublequoteopen}arity{\isacharparenleft}{\kern0pt}is{\isacharunderscore}{\kern0pt}sym{\isacharunderscore}{\kern0pt}fm{\isacharparenleft}{\kern0pt}p{\isacharcomma}{\kern0pt}\ G{\isacharcomma}{\kern0pt}\ x{\isacharcomma}{\kern0pt}\ s{\isacharparenright}{\kern0pt}{\isacharparenright}{\kern0pt}\ {\isasymle}\ succ{\isacharparenleft}{\kern0pt}p{\isacharparenright}{\kern0pt}\ {\isasymunion}\ succ{\isacharparenleft}{\kern0pt}G{\isacharparenright}{\kern0pt}\ {\isasymunion}\ succ{\isacharparenleft}{\kern0pt}x{\isacharparenright}{\kern0pt}\ {\isasymunion}\ succ{\isacharparenleft}{\kern0pt}s{\isacharparenright}{\kern0pt}{\isachardoublequoteclose}\isanewline
%
\isadelimproof
\isanewline
\ \ %
\endisadelimproof
%
\isatagproof
\isacommand{apply}\isamarkupfalse%
{\isacharparenleft}{\kern0pt}subgoal{\isacharunderscore}{\kern0pt}tac\ {\isachardoublequoteopen}is{\isacharunderscore}{\kern0pt}sym{\isacharunderscore}{\kern0pt}elem{\isacharunderscore}{\kern0pt}fm{\isacharparenleft}{\kern0pt}p\ {\isacharhash}{\kern0pt}{\isacharplus}{\kern0pt}\ {\isadigit{1}}{\isacharcomma}{\kern0pt}\ G\ {\isacharhash}{\kern0pt}{\isacharplus}{\kern0pt}\ {\isadigit{1}}{\isacharcomma}{\kern0pt}\ x\ {\isacharhash}{\kern0pt}{\isacharplus}{\kern0pt}\ {\isadigit{1}}{\isacharcomma}{\kern0pt}\ {\isadigit{0}}{\isacharparenright}{\kern0pt}\ {\isasymin}\ formula{\isachardoublequoteclose}{\isacharparenright}{\kern0pt}\isanewline
\ \ \isacommand{unfolding}\isamarkupfalse%
\ is{\isacharunderscore}{\kern0pt}sym{\isacharunderscore}{\kern0pt}fm{\isacharunderscore}{\kern0pt}def\ \isanewline
\ \ \isacommand{using}\isamarkupfalse%
\ assms\isanewline
\ \ \isacommand{apply}\isamarkupfalse%
\ simp\isanewline
\ \ \ \isacommand{apply}\isamarkupfalse%
{\isacharparenleft}{\kern0pt}rule\ pred{\isacharunderscore}{\kern0pt}le{\isacharcomma}{\kern0pt}\ simp{\isacharunderscore}{\kern0pt}all{\isacharparenright}{\kern0pt}{\isacharplus}{\kern0pt}\isanewline
\ \ \ \isacommand{apply}\isamarkupfalse%
{\isacharparenleft}{\kern0pt}rule\ Un{\isacharunderscore}{\kern0pt}least{\isacharunderscore}{\kern0pt}lt{\isacharparenright}{\kern0pt}{\isacharplus}{\kern0pt}\isanewline
\ \ \ \ \ \isacommand{apply}\isamarkupfalse%
\ simp\isanewline
\ \ \ \ \isacommand{apply}\isamarkupfalse%
{\isacharparenleft}{\kern0pt}simp{\isacharcomma}{\kern0pt}\ rule\ ltI{\isacharcomma}{\kern0pt}\ simp{\isacharcomma}{\kern0pt}\ simp{\isacharparenright}{\kern0pt}\isanewline
\ \ \ \isacommand{apply}\isamarkupfalse%
{\isacharparenleft}{\kern0pt}rule\ le{\isacharunderscore}{\kern0pt}trans{\isacharcomma}{\kern0pt}\ rule\ arity{\isacharunderscore}{\kern0pt}is{\isacharunderscore}{\kern0pt}sym{\isacharunderscore}{\kern0pt}elem{\isacharunderscore}{\kern0pt}fm{\isacharparenright}{\kern0pt}\isanewline
\ \ \ \ \ \ \ \isacommand{apply}\isamarkupfalse%
\ auto{\isacharbrackleft}{\kern0pt}{\isadigit{4}}{\isacharbrackright}{\kern0pt}\isanewline
\ \ \ \isacommand{apply}\isamarkupfalse%
{\isacharparenleft}{\kern0pt}subst\ succ{\isacharunderscore}{\kern0pt}Un{\isacharunderscore}{\kern0pt}distrib{\isacharcomma}{\kern0pt}\ simp{\isacharunderscore}{\kern0pt}all{\isacharparenright}{\kern0pt}{\isacharplus}{\kern0pt}\isanewline
\ \ \ \isacommand{apply}\isamarkupfalse%
{\isacharparenleft}{\kern0pt}rule\ Un{\isacharunderscore}{\kern0pt}least{\isacharunderscore}{\kern0pt}lt{\isacharparenright}{\kern0pt}{\isacharplus}{\kern0pt}\isanewline
\ \ \ \ \ \ \isacommand{apply}\isamarkupfalse%
{\isacharparenleft}{\kern0pt}rule\ ltI{\isacharcomma}{\kern0pt}\ simp{\isacharcomma}{\kern0pt}\ simp{\isacharparenright}{\kern0pt}{\isacharplus}{\kern0pt}\isanewline
\ \ \ \isacommand{apply}\isamarkupfalse%
{\isacharparenleft}{\kern0pt}rule\ ltI{\isacharcomma}{\kern0pt}\ simp{\isacharcomma}{\kern0pt}\ rule\ disjI{\isadigit{1}}{\isacharcomma}{\kern0pt}\ rule\ ltD{\isacharcomma}{\kern0pt}\ simp{\isacharcomma}{\kern0pt}\ simp{\isacharparenright}{\kern0pt}\isanewline
\ \ \isacommand{apply}\isamarkupfalse%
{\isacharparenleft}{\kern0pt}rule\ is{\isacharunderscore}{\kern0pt}sym{\isacharunderscore}{\kern0pt}elem{\isacharunderscore}{\kern0pt}fm{\isacharunderscore}{\kern0pt}type{\isacharparenright}{\kern0pt}\isanewline
\ \ \isacommand{using}\isamarkupfalse%
\ assms\ \isanewline
\ \ \isacommand{by}\isamarkupfalse%
\ auto%
\endisatagproof
{\isafoldproof}%
%
\isadelimproof
\isanewline
%
\endisadelimproof
\isanewline
\isacommand{lemma}\isamarkupfalse%
\ sats{\isacharunderscore}{\kern0pt}is{\isacharunderscore}{\kern0pt}sym{\isacharunderscore}{\kern0pt}fm{\isacharunderscore}{\kern0pt}iff\ {\isacharcolon}{\kern0pt}\ \isanewline
\ \ \isakeyword{fixes}\ env\ i\ j\ k\ l\ p\ G\ x\ s\ \isanewline
\ \ \isakeyword{assumes}\ {\isachardoublequoteopen}env\ {\isasymin}\ list{\isacharparenleft}{\kern0pt}M{\isacharparenright}{\kern0pt}{\isachardoublequoteclose}\ {\isachardoublequoteopen}x\ {\isasymin}\ P{\isacharunderscore}{\kern0pt}names{\isachardoublequoteclose}\ {\isachardoublequoteopen}i\ {\isacharless}{\kern0pt}\ length{\isacharparenleft}{\kern0pt}env{\isacharparenright}{\kern0pt}{\isachardoublequoteclose}\ {\isachardoublequoteopen}j\ {\isacharless}{\kern0pt}\ length{\isacharparenleft}{\kern0pt}env{\isacharparenright}{\kern0pt}{\isachardoublequoteclose}\ {\isachardoublequoteopen}k\ {\isacharless}{\kern0pt}\ length{\isacharparenleft}{\kern0pt}env{\isacharparenright}{\kern0pt}{\isachardoublequoteclose}\ {\isachardoublequoteopen}l\ {\isacharless}{\kern0pt}\ length{\isacharparenleft}{\kern0pt}env{\isacharparenright}{\kern0pt}{\isachardoublequoteclose}\ \isanewline
\ \ \ \ \ \ \ \ \ \ {\isachardoublequoteopen}nth{\isacharparenleft}{\kern0pt}i{\isacharcomma}{\kern0pt}\ env{\isacharparenright}{\kern0pt}\ {\isacharequal}{\kern0pt}\ P{\isachardoublequoteclose}\ {\isachardoublequoteopen}nth{\isacharparenleft}{\kern0pt}j{\isacharcomma}{\kern0pt}\ env{\isacharparenright}{\kern0pt}\ {\isacharequal}{\kern0pt}\ {\isasymG}{\isachardoublequoteclose}\ {\isachardoublequoteopen}nth{\isacharparenleft}{\kern0pt}k{\isacharcomma}{\kern0pt}\ env{\isacharparenright}{\kern0pt}\ {\isacharequal}{\kern0pt}\ x{\isachardoublequoteclose}\ {\isachardoublequoteopen}nth{\isacharparenleft}{\kern0pt}l{\isacharcomma}{\kern0pt}\ env{\isacharparenright}{\kern0pt}\ {\isacharequal}{\kern0pt}\ s{\isachardoublequoteclose}\ \ \isanewline
\ \ \isakeyword{shows}\ {\isachardoublequoteopen}sats{\isacharparenleft}{\kern0pt}M{\isacharcomma}{\kern0pt}\ is{\isacharunderscore}{\kern0pt}sym{\isacharunderscore}{\kern0pt}fm{\isacharparenleft}{\kern0pt}i{\isacharcomma}{\kern0pt}\ j{\isacharcomma}{\kern0pt}\ k{\isacharcomma}{\kern0pt}\ l{\isacharparenright}{\kern0pt}{\isacharcomma}{\kern0pt}\ env{\isacharparenright}{\kern0pt}\ {\isasymlongleftrightarrow}\ s\ {\isacharequal}{\kern0pt}\ sym{\isacharparenleft}{\kern0pt}x{\isacharparenright}{\kern0pt}{\isachardoublequoteclose}\ \isanewline
%
\isadelimproof
%
\endisadelimproof
%
\isatagproof
\isacommand{proof}\isamarkupfalse%
\ {\isacharminus}{\kern0pt}\ \isanewline
\isanewline
\ \ \isacommand{have}\isamarkupfalse%
\ iff{\isacharunderscore}{\kern0pt}lemma\ {\isacharcolon}{\kern0pt}\ {\isachardoublequoteopen}{\isasymAnd}a\ b\ c{\isachardot}{\kern0pt}\ b\ {\isacharequal}{\kern0pt}\ c\ {\isasymLongrightarrow}\ a\ {\isacharequal}{\kern0pt}\ b\ {\isasymlongleftrightarrow}\ a\ {\isacharequal}{\kern0pt}\ c{\isachardoublequoteclose}\ \isacommand{by}\isamarkupfalse%
\ auto\isanewline
\isanewline
\ \ \isacommand{have}\isamarkupfalse%
\ innat\ {\isacharcolon}{\kern0pt}\ {\isachardoublequoteopen}i\ {\isasymin}\ nat\ {\isasymand}\ j\ {\isasymin}\ nat\ {\isasymand}\ k\ {\isasymin}\ nat\ {\isasymand}\ l\ {\isasymin}\ nat{\isachardoublequoteclose}\ \isanewline
\ \ \ \ \isacommand{using}\isamarkupfalse%
\ assms\ lt{\isacharunderscore}{\kern0pt}nat{\isacharunderscore}{\kern0pt}in{\isacharunderscore}{\kern0pt}nat\ \isacommand{by}\isamarkupfalse%
\ auto\isanewline
\isanewline
\ \ \isacommand{have}\isamarkupfalse%
\ {\isachardoublequoteopen}{\isacharparenleft}{\kern0pt}M{\isacharcomma}{\kern0pt}\ env\ {\isasymTurnstile}\ is{\isacharunderscore}{\kern0pt}sym{\isacharunderscore}{\kern0pt}fm{\isacharparenleft}{\kern0pt}i{\isacharcomma}{\kern0pt}\ j{\isacharcomma}{\kern0pt}\ k{\isacharcomma}{\kern0pt}\ l{\isacharparenright}{\kern0pt}{\isacharparenright}{\kern0pt}\ {\isasymlongleftrightarrow}\ {\isacharparenleft}{\kern0pt}{\isasymforall}{\isasympi}\ {\isasymin}\ M{\isachardot}{\kern0pt}\ {\isasympi}\ {\isasymin}\ s\ {\isasymlongleftrightarrow}\ {\isasympi}\ {\isasymin}\ sym{\isacharparenleft}{\kern0pt}x{\isacharparenright}{\kern0pt}{\isacharparenright}{\kern0pt}{\isachardoublequoteclose}\ \isanewline
\ \ \ \ \isacommand{unfolding}\isamarkupfalse%
\ is{\isacharunderscore}{\kern0pt}sym{\isacharunderscore}{\kern0pt}fm{\isacharunderscore}{\kern0pt}def\ \isanewline
\ \ \ \ \isacommand{apply}\isamarkupfalse%
{\isacharparenleft}{\kern0pt}rule\ iff{\isacharunderscore}{\kern0pt}trans{\isacharcomma}{\kern0pt}\ rule\ sats{\isacharunderscore}{\kern0pt}Forall{\isacharunderscore}{\kern0pt}iff{\isacharcomma}{\kern0pt}\ simp\ add{\isacharcolon}{\kern0pt}assms{\isacharparenright}{\kern0pt}\isanewline
\ \ \ \ \isacommand{apply}\isamarkupfalse%
{\isacharparenleft}{\kern0pt}rule\ ball{\isacharunderscore}{\kern0pt}iff{\isacharcomma}{\kern0pt}\ rule\ iff{\isacharunderscore}{\kern0pt}trans{\isacharcomma}{\kern0pt}\ rule\ sats{\isacharunderscore}{\kern0pt}Iff{\isacharunderscore}{\kern0pt}iff{\isacharcomma}{\kern0pt}\ simp\ add{\isacharcolon}{\kern0pt}assms{\isacharparenright}{\kern0pt}\isanewline
\ \ \ \ \isacommand{apply}\isamarkupfalse%
{\isacharparenleft}{\kern0pt}rule\ iff{\isacharunderscore}{\kern0pt}iff{\isacharparenright}{\kern0pt}\isanewline
\ \ \ \ \isacommand{using}\isamarkupfalse%
\ assms\ innat\ \isanewline
\ \ \ \ \ \isacommand{apply}\isamarkupfalse%
\ force\isanewline
\ \ \ \ \isacommand{apply}\isamarkupfalse%
{\isacharparenleft}{\kern0pt}rule\ sats{\isacharunderscore}{\kern0pt}is{\isacharunderscore}{\kern0pt}sym{\isacharunderscore}{\kern0pt}elem{\isacharunderscore}{\kern0pt}fm{\isacharunderscore}{\kern0pt}iff{\isacharparenright}{\kern0pt}\isanewline
\ \ \ \ \isacommand{using}\isamarkupfalse%
\ assms\ innat\ \isanewline
\ \ \ \ \isacommand{by}\isamarkupfalse%
\ auto\ \isanewline
\ \ \isacommand{also}\isamarkupfalse%
\ \isacommand{have}\isamarkupfalse%
\ {\isachardoublequoteopen}{\isachardot}{\kern0pt}{\isachardot}{\kern0pt}{\isachardot}{\kern0pt}\ {\isasymlongleftrightarrow}\ s\ {\isacharequal}{\kern0pt}\ sym{\isacharparenleft}{\kern0pt}x{\isacharparenright}{\kern0pt}{\isachardoublequoteclose}\ \isanewline
\ \ \ \ \isacommand{apply}\isamarkupfalse%
{\isacharparenleft}{\kern0pt}rule\ iffI{\isacharcomma}{\kern0pt}\ rule\ equality{\isacharunderscore}{\kern0pt}iffI{\isacharcomma}{\kern0pt}\ rule\ iffI{\isacharparenright}{\kern0pt}\isanewline
\ \ \ \ \ \ \isacommand{apply}\isamarkupfalse%
{\isacharparenleft}{\kern0pt}rename{\isacharunderscore}{\kern0pt}tac\ {\isasymtau}{\isacharcomma}{\kern0pt}\ subgoal{\isacharunderscore}{\kern0pt}tac\ {\isachardoublequoteopen}{\isasymtau}\ {\isasymin}\ M{\isachardoublequoteclose}{\isacharcomma}{\kern0pt}\ force{\isacharparenright}{\kern0pt}\isanewline
\ \ \ \ \ \ \isacommand{apply}\isamarkupfalse%
{\isacharparenleft}{\kern0pt}subgoal{\isacharunderscore}{\kern0pt}tac\ {\isachardoublequoteopen}s\ {\isasymin}\ M{\isachardoublequoteclose}{\isacharparenright}{\kern0pt}\isanewline
\ \ \ \ \isacommand{using}\isamarkupfalse%
\ transM\ assms\ nth{\isacharunderscore}{\kern0pt}type\isanewline
\ \ \ \ \ \ \ \isacommand{apply}\isamarkupfalse%
\ auto{\isacharbrackleft}{\kern0pt}{\isadigit{2}}{\isacharbrackright}{\kern0pt}\isanewline
\ \ \ \ \isacommand{apply}\isamarkupfalse%
{\isacharparenleft}{\kern0pt}rename{\isacharunderscore}{\kern0pt}tac\ {\isasymtau}{\isacharcomma}{\kern0pt}\ subgoal{\isacharunderscore}{\kern0pt}tac\ {\isachardoublequoteopen}{\isasymtau}\ {\isasymin}\ M{\isachardoublequoteclose}{\isacharcomma}{\kern0pt}\ force{\isacharparenright}{\kern0pt}\isanewline
\ \ \ \ \isacommand{using}\isamarkupfalse%
\ transM\ {\isasymG}{\isacharunderscore}{\kern0pt}in{\isacharunderscore}{\kern0pt}M\ sym{\isacharunderscore}{\kern0pt}def\isanewline
\ \ \ \ \ \isacommand{apply}\isamarkupfalse%
\ force\isanewline
\ \ \ \ \isacommand{by}\isamarkupfalse%
\ auto\ \isanewline
\ \ \isacommand{finally}\isamarkupfalse%
\ \isacommand{show}\isamarkupfalse%
\ {\isacharquery}{\kern0pt}thesis\ \isacommand{by}\isamarkupfalse%
\ auto\isanewline
\isacommand{qed}\isamarkupfalse%
%
\endisatagproof
{\isafoldproof}%
%
\isadelimproof
\isanewline
%
\endisadelimproof
\isanewline
\isacommand{lemma}\isamarkupfalse%
\ sym{\isacharunderscore}{\kern0pt}in{\isacharunderscore}{\kern0pt}M\ {\isacharcolon}{\kern0pt}\ \isanewline
\ \ \isakeyword{fixes}\ x\isanewline
\ \ \isakeyword{assumes}\ {\isachardoublequoteopen}x\ {\isasymin}\ P{\isacharunderscore}{\kern0pt}names{\isachardoublequoteclose}\ \isanewline
\ \ \isakeyword{shows}\ {\isachardoublequoteopen}sym{\isacharparenleft}{\kern0pt}x{\isacharparenright}{\kern0pt}\ {\isasymin}\ M{\isachardoublequoteclose}\ \isanewline
%
\isadelimproof
\isanewline
%
\endisadelimproof
%
\isatagproof
\isacommand{proof}\isamarkupfalse%
\ {\isacharminus}{\kern0pt}\ \isanewline
\isanewline
\ \ \isacommand{define}\isamarkupfalse%
\ S\ \isakeyword{where}\ {\isachardoublequoteopen}S\ {\isasymequiv}\ {\isacharbraceleft}{\kern0pt}\ v\ {\isasymin}\ {\isasymG}{\isachardot}{\kern0pt}\ sats{\isacharparenleft}{\kern0pt}M{\isacharcomma}{\kern0pt}\ is{\isacharunderscore}{\kern0pt}sym{\isacharunderscore}{\kern0pt}elem{\isacharunderscore}{\kern0pt}fm{\isacharparenleft}{\kern0pt}{\isadigit{1}}{\isacharcomma}{\kern0pt}\ {\isadigit{2}}{\isacharcomma}{\kern0pt}\ {\isadigit{3}}{\isacharcomma}{\kern0pt}\ {\isadigit{0}}{\isacharparenright}{\kern0pt}{\isacharcomma}{\kern0pt}\ {\isacharbrackleft}{\kern0pt}v{\isacharbrackright}{\kern0pt}\ {\isacharat}{\kern0pt}\ {\isacharbrackleft}{\kern0pt}P{\isacharcomma}{\kern0pt}\ {\isasymG}{\isacharcomma}{\kern0pt}\ x{\isacharbrackright}{\kern0pt}{\isacharparenright}{\kern0pt}\ {\isacharbraceright}{\kern0pt}{\isachardoublequoteclose}\ \isanewline
\isanewline
\ \ \isacommand{have}\isamarkupfalse%
\ {\isachardoublequoteopen}separation{\isacharparenleft}{\kern0pt}{\isacharhash}{\kern0pt}{\isacharhash}{\kern0pt}M{\isacharcomma}{\kern0pt}\ {\isasymlambda}v{\isachardot}{\kern0pt}\ sats{\isacharparenleft}{\kern0pt}M{\isacharcomma}{\kern0pt}\ is{\isacharunderscore}{\kern0pt}sym{\isacharunderscore}{\kern0pt}elem{\isacharunderscore}{\kern0pt}fm{\isacharparenleft}{\kern0pt}{\isadigit{1}}{\isacharcomma}{\kern0pt}\ {\isadigit{2}}{\isacharcomma}{\kern0pt}\ {\isadigit{3}}{\isacharcomma}{\kern0pt}\ {\isadigit{0}}{\isacharparenright}{\kern0pt}{\isacharcomma}{\kern0pt}\ {\isacharbrackleft}{\kern0pt}v{\isacharbrackright}{\kern0pt}\ {\isacharat}{\kern0pt}\ {\isacharbrackleft}{\kern0pt}P{\isacharcomma}{\kern0pt}\ {\isasymG}{\isacharcomma}{\kern0pt}\ x{\isacharbrackright}{\kern0pt}{\isacharparenright}{\kern0pt}{\isacharparenright}{\kern0pt}{\isachardoublequoteclose}\ \isanewline
\ \ \ \ \isacommand{apply}\isamarkupfalse%
{\isacharparenleft}{\kern0pt}rule\ separation{\isacharunderscore}{\kern0pt}ax{\isacharparenright}{\kern0pt}\isanewline
\ \ \ \ \ \ \isacommand{apply}\isamarkupfalse%
{\isacharparenleft}{\kern0pt}rule\ is{\isacharunderscore}{\kern0pt}sym{\isacharunderscore}{\kern0pt}elem{\isacharunderscore}{\kern0pt}fm{\isacharunderscore}{\kern0pt}type{\isacharparenright}{\kern0pt}\isanewline
\ \ \ \ \isacommand{using}\isamarkupfalse%
\ P{\isacharunderscore}{\kern0pt}in{\isacharunderscore}{\kern0pt}M\ {\isasymG}{\isacharunderscore}{\kern0pt}in{\isacharunderscore}{\kern0pt}M\ assms\ P{\isacharunderscore}{\kern0pt}name{\isacharunderscore}{\kern0pt}in{\isacharunderscore}{\kern0pt}M\isanewline
\ \ \ \ \ \ \ \ \ \isacommand{apply}\isamarkupfalse%
\ auto{\isacharbrackleft}{\kern0pt}{\isadigit{5}}{\isacharbrackright}{\kern0pt}\isanewline
\ \ \ \ \isacommand{apply}\isamarkupfalse%
{\isacharparenleft}{\kern0pt}rule\ le{\isacharunderscore}{\kern0pt}trans{\isacharcomma}{\kern0pt}\ rule\ arity{\isacharunderscore}{\kern0pt}is{\isacharunderscore}{\kern0pt}sym{\isacharunderscore}{\kern0pt}elem{\isacharunderscore}{\kern0pt}fm{\isacharparenright}{\kern0pt}\isanewline
\ \ \ \ \ \ \ \ \isacommand{apply}\isamarkupfalse%
\ auto{\isacharbrackleft}{\kern0pt}{\isadigit{4}}{\isacharbrackright}{\kern0pt}\isanewline
\ \ \ \ \isacommand{apply}\isamarkupfalse%
\ simp\ \isanewline
\ \ \ \ \isacommand{apply}\isamarkupfalse%
{\isacharparenleft}{\kern0pt}rule\ Un{\isacharunderscore}{\kern0pt}least{\isacharunderscore}{\kern0pt}lt{\isacharparenright}{\kern0pt}{\isacharplus}{\kern0pt}\isanewline
\ \ \ \ \isacommand{by}\isamarkupfalse%
\ auto\isanewline
\ \ \isanewline
\ \ \isacommand{then}\isamarkupfalse%
\ \isacommand{have}\isamarkupfalse%
\ SinM\ {\isacharcolon}{\kern0pt}\ {\isachardoublequoteopen}S\ {\isasymin}\ M{\isachardoublequoteclose}\isanewline
\ \ \ \ \isacommand{unfolding}\isamarkupfalse%
\ S{\isacharunderscore}{\kern0pt}def\ \isanewline
\ \ \ \ \isacommand{apply}\isamarkupfalse%
{\isacharparenleft}{\kern0pt}rule\ separation{\isacharunderscore}{\kern0pt}notation{\isacharparenright}{\kern0pt}\isanewline
\ \ \ \ \isacommand{using}\isamarkupfalse%
\ {\isasymG}{\isacharunderscore}{\kern0pt}in{\isacharunderscore}{\kern0pt}M\isanewline
\ \ \ \ \isacommand{by}\isamarkupfalse%
\ auto\ \isanewline
\isanewline
\ \ \isacommand{have}\isamarkupfalse%
\ {\isachardoublequoteopen}S\ {\isacharequal}{\kern0pt}\ sym{\isacharparenleft}{\kern0pt}x{\isacharparenright}{\kern0pt}{\isachardoublequoteclose}\ \isanewline
\ \ \ \ \isacommand{unfolding}\isamarkupfalse%
\ S{\isacharunderscore}{\kern0pt}def\ sym{\isacharunderscore}{\kern0pt}def\ \isanewline
\ \ \ \ \isacommand{apply}\isamarkupfalse%
{\isacharparenleft}{\kern0pt}rule\ iff{\isacharunderscore}{\kern0pt}eq{\isacharparenright}{\kern0pt}\isanewline
\ \ \ \ \isacommand{apply}\isamarkupfalse%
{\isacharparenleft}{\kern0pt}rule\ iff{\isacharunderscore}{\kern0pt}trans{\isacharcomma}{\kern0pt}\ rule\ sats{\isacharunderscore}{\kern0pt}is{\isacharunderscore}{\kern0pt}sym{\isacharunderscore}{\kern0pt}elem{\isacharunderscore}{\kern0pt}fm{\isacharunderscore}{\kern0pt}iff{\isacharparenright}{\kern0pt}\isanewline
\ \ \ \ \isacommand{using}\isamarkupfalse%
\ assms\ {\isasymG}{\isacharunderscore}{\kern0pt}in{\isacharunderscore}{\kern0pt}M\ transM\ P{\isacharunderscore}{\kern0pt}in{\isacharunderscore}{\kern0pt}M\ P{\isacharunderscore}{\kern0pt}name{\isacharunderscore}{\kern0pt}in{\isacharunderscore}{\kern0pt}M\ \isanewline
\ \ \ \ \ \ \ \ \ \ \ \ \ \ \isacommand{apply}\isamarkupfalse%
\ auto{\isacharbrackleft}{\kern0pt}{\isadigit{1}}{\isadigit{0}}{\isacharbrackright}{\kern0pt}\isanewline
\ \ \ \ \isacommand{unfolding}\isamarkupfalse%
\ sym{\isacharunderscore}{\kern0pt}def\ \isanewline
\ \ \ \ \isacommand{by}\isamarkupfalse%
\ auto\isanewline
\ \ \isacommand{then}\isamarkupfalse%
\ \isacommand{show}\isamarkupfalse%
\ {\isacharquery}{\kern0pt}thesis\ \isacommand{using}\isamarkupfalse%
\ SinM\ \isacommand{by}\isamarkupfalse%
\ auto\isanewline
\isacommand{qed}\isamarkupfalse%
%
\endisatagproof
{\isafoldproof}%
%
\isadelimproof
\isanewline
%
\endisadelimproof
\isanewline
\isacommand{end}\isamarkupfalse%
\isanewline
\isanewline
\ \isanewline
\isanewline
\isacommand{definition}\isamarkupfalse%
\ His{\isacharunderscore}{\kern0pt}HS{\isacharunderscore}{\kern0pt}M{\isacharunderscore}{\kern0pt}cond{\isacharunderscore}{\kern0pt}fm\ \isakeyword{where}\ \isanewline
\ \ {\isachardoublequoteopen}His{\isacharunderscore}{\kern0pt}HS{\isacharunderscore}{\kern0pt}M{\isacharunderscore}{\kern0pt}cond{\isacharunderscore}{\kern0pt}fm{\isacharparenleft}{\kern0pt}x{\isacharprime}{\kern0pt}{\isacharcomma}{\kern0pt}\ g{\isacharparenright}{\kern0pt}\ {\isasymequiv}\ \isanewline
\ \ \ \ Exists{\isacharparenleft}{\kern0pt}Exists{\isacharparenleft}{\kern0pt}Exists{\isacharparenleft}{\kern0pt}Exists{\isacharparenleft}{\kern0pt}Exists{\isacharparenleft}{\kern0pt}Exists{\isacharparenleft}{\kern0pt}Exists{\isacharparenleft}{\kern0pt}Exists{\isacharparenleft}{\kern0pt}Exists{\isacharparenleft}{\kern0pt}Exists{\isacharparenleft}{\kern0pt}\isanewline
\ \ \ \ \ \ And{\isacharparenleft}{\kern0pt}pair{\isacharunderscore}{\kern0pt}fm{\isacharparenleft}{\kern0pt}{\isadigit{0}}{\isacharcomma}{\kern0pt}\ {\isadigit{1}}{\isacharcomma}{\kern0pt}\ x{\isacharprime}{\kern0pt}\ {\isacharhash}{\kern0pt}{\isacharplus}{\kern0pt}\ {\isadigit{1}}{\isadigit{0}}{\isacharparenright}{\kern0pt}{\isacharcomma}{\kern0pt}\ \isanewline
\ \ \ \ \ \ And{\isacharparenleft}{\kern0pt}pair{\isacharunderscore}{\kern0pt}fm{\isacharparenleft}{\kern0pt}{\isadigit{2}}{\isacharcomma}{\kern0pt}\ {\isadigit{3}}{\isacharcomma}{\kern0pt}\ {\isadigit{1}}{\isacharparenright}{\kern0pt}{\isacharcomma}{\kern0pt}\ \isanewline
\ \ \ \ \ \ And{\isacharparenleft}{\kern0pt}pair{\isacharunderscore}{\kern0pt}fm{\isacharparenleft}{\kern0pt}{\isadigit{4}}{\isacharcomma}{\kern0pt}\ {\isadigit{5}}{\isacharcomma}{\kern0pt}\ {\isadigit{3}}{\isacharparenright}{\kern0pt}{\isacharcomma}{\kern0pt}\ \isanewline
\ \ \ \ \ \ And{\isacharparenleft}{\kern0pt}pair{\isacharunderscore}{\kern0pt}fm{\isacharparenleft}{\kern0pt}{\isadigit{6}}{\isacharcomma}{\kern0pt}\ {\isadigit{7}}{\isacharcomma}{\kern0pt}\ {\isadigit{5}}{\isacharparenright}{\kern0pt}{\isacharcomma}{\kern0pt}\ \isanewline
\ \ \ \ \ \ And{\isacharparenleft}{\kern0pt}domain{\isacharunderscore}{\kern0pt}fm{\isacharparenleft}{\kern0pt}{\isadigit{0}}{\isacharcomma}{\kern0pt}\ {\isadigit{8}}{\isacharparenright}{\kern0pt}{\isacharcomma}{\kern0pt}\ \isanewline
\ \ \ \ \ \ And{\isacharparenleft}{\kern0pt}is{\isacharunderscore}{\kern0pt}P{\isacharunderscore}{\kern0pt}name{\isacharunderscore}{\kern0pt}fm{\isacharparenleft}{\kern0pt}{\isadigit{6}}{\isacharcomma}{\kern0pt}\ {\isadigit{0}}{\isacharparenright}{\kern0pt}{\isacharcomma}{\kern0pt}\ \isanewline
\ \ \ \ \ \ And{\isacharparenleft}{\kern0pt}apply{\isacharunderscore}{\kern0pt}all{\isacharunderscore}{\kern0pt}{\isadigit{1}}{\isacharunderscore}{\kern0pt}fm{\isacharparenleft}{\kern0pt}g\ {\isacharhash}{\kern0pt}{\isacharplus}{\kern0pt}\ {\isadigit{1}}{\isadigit{0}}{\isacharcomma}{\kern0pt}\ {\isadigit{8}}{\isacharcomma}{\kern0pt}\ {\isadigit{1}}{\isacharparenright}{\kern0pt}{\isacharcomma}{\kern0pt}\ \isanewline
\ \ \ \ \ \ And{\isacharparenleft}{\kern0pt}is{\isacharunderscore}{\kern0pt}sym{\isacharunderscore}{\kern0pt}fm{\isacharparenleft}{\kern0pt}{\isadigit{6}}{\isacharcomma}{\kern0pt}\ {\isadigit{4}}{\isacharcomma}{\kern0pt}\ {\isadigit{0}}{\isacharcomma}{\kern0pt}\ {\isadigit{9}}{\isacharparenright}{\kern0pt}{\isacharcomma}{\kern0pt}\ Member{\isacharparenleft}{\kern0pt}{\isadigit{9}}{\isacharcomma}{\kern0pt}\ {\isadigit{2}}{\isacharparenright}{\kern0pt}{\isacharparenright}{\kern0pt}{\isacharparenright}{\kern0pt}{\isacharparenright}{\kern0pt}{\isacharparenright}{\kern0pt}{\isacharparenright}{\kern0pt}{\isacharparenright}{\kern0pt}{\isacharparenright}{\kern0pt}{\isacharparenright}{\kern0pt}{\isacharparenright}{\kern0pt}{\isacharparenright}{\kern0pt}{\isacharparenright}{\kern0pt}{\isacharparenright}{\kern0pt}{\isacharparenright}{\kern0pt}{\isacharparenright}{\kern0pt}{\isacharparenright}{\kern0pt}{\isacharparenright}{\kern0pt}{\isacharparenright}{\kern0pt}{\isacharparenright}{\kern0pt}{\isachardoublequoteclose}\ \isanewline
\isanewline
\isanewline
\isacommand{context}\isamarkupfalse%
\ M{\isacharunderscore}{\kern0pt}symmetric{\isacharunderscore}{\kern0pt}system\isanewline
\isakeyword{begin}\isanewline
\isanewline
\isacommand{lemma}\isamarkupfalse%
\ His{\isacharunderscore}{\kern0pt}HS{\isacharunderscore}{\kern0pt}M{\isacharunderscore}{\kern0pt}cond{\isacharunderscore}{\kern0pt}fm{\isacharunderscore}{\kern0pt}type\ {\isacharcolon}{\kern0pt}\ \isanewline
\ \ \isakeyword{fixes}\ x{\isacharprime}{\kern0pt}\ g\ \isanewline
\ \ \isakeyword{assumes}\ {\isachardoublequoteopen}x{\isacharprime}{\kern0pt}\ {\isasymin}\ nat{\isachardoublequoteclose}\ {\isachardoublequoteopen}g\ {\isasymin}\ nat{\isachardoublequoteclose}\ \isanewline
\ \ \isakeyword{shows}\ {\isachardoublequoteopen}His{\isacharunderscore}{\kern0pt}HS{\isacharunderscore}{\kern0pt}M{\isacharunderscore}{\kern0pt}cond{\isacharunderscore}{\kern0pt}fm{\isacharparenleft}{\kern0pt}x{\isacharprime}{\kern0pt}{\isacharcomma}{\kern0pt}\ g{\isacharparenright}{\kern0pt}\ {\isasymin}\ formula{\isachardoublequoteclose}\ \isanewline
%
\isadelimproof
\isanewline
\ \ %
\endisadelimproof
%
\isatagproof
\isacommand{unfolding}\isamarkupfalse%
\ His{\isacharunderscore}{\kern0pt}HS{\isacharunderscore}{\kern0pt}M{\isacharunderscore}{\kern0pt}cond{\isacharunderscore}{\kern0pt}fm{\isacharunderscore}{\kern0pt}def\isanewline
\ \ \isacommand{apply}\isamarkupfalse%
\ {\isacharparenleft}{\kern0pt}rule\ Exists{\isacharunderscore}{\kern0pt}type{\isacharparenright}{\kern0pt}{\isacharplus}{\kern0pt}\isanewline
\ \ \isacommand{apply}\isamarkupfalse%
{\isacharparenleft}{\kern0pt}rule\ And{\isacharunderscore}{\kern0pt}type{\isacharcomma}{\kern0pt}\ simp{\isacharparenright}{\kern0pt}{\isacharplus}{\kern0pt}\isanewline
\ \ \isacommand{apply}\isamarkupfalse%
{\isacharparenleft}{\kern0pt}rule\ And{\isacharunderscore}{\kern0pt}type{\isacharcomma}{\kern0pt}\ rule\ is{\isacharunderscore}{\kern0pt}P{\isacharunderscore}{\kern0pt}name{\isacharunderscore}{\kern0pt}fm{\isacharunderscore}{\kern0pt}type{\isacharcomma}{\kern0pt}\ simp{\isacharcomma}{\kern0pt}\ simp{\isacharparenright}{\kern0pt}\isanewline
\ \ \ \isacommand{apply}\isamarkupfalse%
{\isacharparenleft}{\kern0pt}rule\ And{\isacharunderscore}{\kern0pt}type{\isacharcomma}{\kern0pt}\ rule\ apply{\isacharunderscore}{\kern0pt}all{\isacharunderscore}{\kern0pt}{\isadigit{1}}{\isacharunderscore}{\kern0pt}fm{\isacharunderscore}{\kern0pt}type{\isacharcomma}{\kern0pt}\ simp{\isacharcomma}{\kern0pt}\ simp{\isacharcomma}{\kern0pt}\ simp{\isacharparenright}{\kern0pt}\isanewline
\ \ \isacommand{apply}\isamarkupfalse%
{\isacharparenleft}{\kern0pt}rule\ And{\isacharunderscore}{\kern0pt}type{\isacharcomma}{\kern0pt}\ rule\ is{\isacharunderscore}{\kern0pt}sym{\isacharunderscore}{\kern0pt}fm{\isacharunderscore}{\kern0pt}type{\isacharparenright}{\kern0pt}\isanewline
\ \ \isacommand{by}\isamarkupfalse%
\ auto%
\endisatagproof
{\isafoldproof}%
%
\isadelimproof
\isanewline
%
\endisadelimproof
\isanewline
\isacommand{lemma}\isamarkupfalse%
\ arity{\isacharunderscore}{\kern0pt}His{\isacharunderscore}{\kern0pt}HS{\isacharunderscore}{\kern0pt}M{\isacharunderscore}{\kern0pt}cond{\isacharunderscore}{\kern0pt}fm\ {\isacharcolon}{\kern0pt}\ \isanewline
\ \ \isakeyword{fixes}\ x{\isacharprime}{\kern0pt}\ g\ \isanewline
\ \ \isakeyword{assumes}\ {\isachardoublequoteopen}x{\isacharprime}{\kern0pt}\ {\isasymin}\ nat{\isachardoublequoteclose}\ {\isachardoublequoteopen}g\ {\isasymin}\ nat{\isachardoublequoteclose}\ \isanewline
\ \ \isakeyword{shows}\ {\isachardoublequoteopen}arity{\isacharparenleft}{\kern0pt}His{\isacharunderscore}{\kern0pt}HS{\isacharunderscore}{\kern0pt}M{\isacharunderscore}{\kern0pt}cond{\isacharunderscore}{\kern0pt}fm{\isacharparenleft}{\kern0pt}x{\isacharprime}{\kern0pt}{\isacharcomma}{\kern0pt}\ g{\isacharparenright}{\kern0pt}{\isacharparenright}{\kern0pt}\ {\isasymle}\ succ{\isacharparenleft}{\kern0pt}x{\isacharprime}{\kern0pt}{\isacharparenright}{\kern0pt}\ {\isasymunion}\ succ{\isacharparenleft}{\kern0pt}g{\isacharparenright}{\kern0pt}{\isachardoublequoteclose}\isanewline
%
\isadelimproof
\isanewline
\ \ %
\endisadelimproof
%
\isatagproof
\isacommand{unfolding}\isamarkupfalse%
\ His{\isacharunderscore}{\kern0pt}HS{\isacharunderscore}{\kern0pt}M{\isacharunderscore}{\kern0pt}cond{\isacharunderscore}{\kern0pt}fm{\isacharunderscore}{\kern0pt}def\ \isanewline
\ \ \isacommand{apply}\isamarkupfalse%
\ {\isacharparenleft}{\kern0pt}simp\ del{\isacharcolon}{\kern0pt}FOL{\isacharunderscore}{\kern0pt}sats{\isacharunderscore}{\kern0pt}iff\ pair{\isacharunderscore}{\kern0pt}abs\ add{\isacharcolon}{\kern0pt}\ fm{\isacharunderscore}{\kern0pt}defs\ nat{\isacharunderscore}{\kern0pt}simp{\isacharunderscore}{\kern0pt}union{\isacharparenright}{\kern0pt}\isanewline
\ \ \isacommand{using}\isamarkupfalse%
\ assms\isanewline
\ \ \isacommand{apply}\isamarkupfalse%
\ simp\isanewline
\ \ \isacommand{apply}\isamarkupfalse%
{\isacharparenleft}{\kern0pt}subgoal{\isacharunderscore}{\kern0pt}tac\ {\isachardoublequoteopen}apply{\isacharunderscore}{\kern0pt}all{\isacharunderscore}{\kern0pt}{\isadigit{1}}{\isacharunderscore}{\kern0pt}fm{\isacharparenleft}{\kern0pt}succ{\isacharparenleft}{\kern0pt}succ{\isacharparenleft}{\kern0pt}succ{\isacharparenleft}{\kern0pt}succ{\isacharparenleft}{\kern0pt}succ{\isacharparenleft}{\kern0pt}succ{\isacharparenleft}{\kern0pt}succ{\isacharparenleft}{\kern0pt}succ{\isacharparenleft}{\kern0pt}succ{\isacharparenleft}{\kern0pt}succ{\isacharparenleft}{\kern0pt}natify{\isacharparenleft}{\kern0pt}g{\isacharparenright}{\kern0pt}{\isacharparenright}{\kern0pt}{\isacharparenright}{\kern0pt}{\isacharparenright}{\kern0pt}{\isacharparenright}{\kern0pt}{\isacharparenright}{\kern0pt}{\isacharparenright}{\kern0pt}{\isacharparenright}{\kern0pt}{\isacharparenright}{\kern0pt}{\isacharparenright}{\kern0pt}{\isacharparenright}{\kern0pt}{\isacharcomma}{\kern0pt}\ {\isadigit{8}}{\isacharcomma}{\kern0pt}\ {\isadigit{1}}{\isacharparenright}{\kern0pt}\ {\isasymin}\ formula{\isachardoublequoteclose}{\isacharparenright}{\kern0pt}\ \isanewline
\ \ \isacommand{apply}\isamarkupfalse%
{\isacharparenleft}{\kern0pt}subgoal{\isacharunderscore}{\kern0pt}tac\ {\isachardoublequoteopen}is{\isacharunderscore}{\kern0pt}sym{\isacharunderscore}{\kern0pt}fm{\isacharparenleft}{\kern0pt}{\isadigit{6}}{\isacharcomma}{\kern0pt}\ {\isadigit{4}}{\isacharcomma}{\kern0pt}\ {\isadigit{0}}{\isacharcomma}{\kern0pt}\ {\isadigit{9}}{\isacharparenright}{\kern0pt}\ {\isasymin}\ formula{\isachardoublequoteclose}{\isacharparenright}{\kern0pt}\isanewline
\ \ \isacommand{apply}\isamarkupfalse%
{\isacharparenleft}{\kern0pt}subgoal{\isacharunderscore}{\kern0pt}tac\ {\isachardoublequoteopen}is{\isacharunderscore}{\kern0pt}P{\isacharunderscore}{\kern0pt}name{\isacharunderscore}{\kern0pt}fm{\isacharparenleft}{\kern0pt}{\isadigit{6}}{\isacharcomma}{\kern0pt}\ {\isadigit{0}}{\isacharparenright}{\kern0pt}\ {\isasymin}\ formula{\isachardoublequoteclose}{\isacharparenright}{\kern0pt}\isanewline
\ \ \ \ \isacommand{apply}\isamarkupfalse%
{\isacharparenleft}{\kern0pt}rule{\isacharunderscore}{\kern0pt}tac\ pred{\isacharunderscore}{\kern0pt}le{\isacharcomma}{\kern0pt}\ simp{\isacharunderscore}{\kern0pt}all{\isacharparenright}{\kern0pt}{\isacharplus}{\kern0pt}\isanewline
\ \ \ \ \isacommand{apply}\isamarkupfalse%
{\isacharparenleft}{\kern0pt}subst\ succ{\isacharunderscore}{\kern0pt}Un{\isacharunderscore}{\kern0pt}distrib{\isacharcomma}{\kern0pt}\ simp{\isacharunderscore}{\kern0pt}all{\isacharparenright}{\kern0pt}{\isacharplus}{\kern0pt}\isanewline
\ \ \ \ \isacommand{apply}\isamarkupfalse%
{\isacharparenleft}{\kern0pt}rule\ Un{\isacharunderscore}{\kern0pt}least{\isacharunderscore}{\kern0pt}lt{\isacharcomma}{\kern0pt}\ rule\ ltI{\isacharcomma}{\kern0pt}\ simp{\isacharcomma}{\kern0pt}\ simp{\isacharparenright}{\kern0pt}\isanewline
\ \ \ \ \ \isacommand{apply}\isamarkupfalse%
{\isacharparenleft}{\kern0pt}rule\ Un{\isacharunderscore}{\kern0pt}least{\isacharunderscore}{\kern0pt}lt{\isacharcomma}{\kern0pt}\ rule\ ltI{\isacharcomma}{\kern0pt}\ simp{\isacharcomma}{\kern0pt}\ rule\ disjI{\isadigit{1}}{\isacharcomma}{\kern0pt}\ rule\ ltD{\isacharcomma}{\kern0pt}\ simp{\isacharcomma}{\kern0pt}\ simp{\isacharparenright}{\kern0pt}{\isacharplus}{\kern0pt}\isanewline
\ \ \ \ \ \isacommand{apply}\isamarkupfalse%
{\isacharparenleft}{\kern0pt}rule\ Un{\isacharunderscore}{\kern0pt}least{\isacharunderscore}{\kern0pt}lt{\isacharcomma}{\kern0pt}\ rule\ le{\isacharunderscore}{\kern0pt}lt{\isacharunderscore}{\kern0pt}lt{\isacharcomma}{\kern0pt}\ rule\ arity{\isacharunderscore}{\kern0pt}is{\isacharunderscore}{\kern0pt}P{\isacharunderscore}{\kern0pt}name{\isacharunderscore}{\kern0pt}fm{\isacharcomma}{\kern0pt}\ simp{\isacharcomma}{\kern0pt}\ simp{\isacharparenright}{\kern0pt}\isanewline
\ \ \ \ \ \ \isacommand{apply}\isamarkupfalse%
{\isacharparenleft}{\kern0pt}rule\ Un{\isacharunderscore}{\kern0pt}least{\isacharunderscore}{\kern0pt}lt{\isacharcomma}{\kern0pt}\ rule\ ltI{\isacharcomma}{\kern0pt}\ simp{\isacharcomma}{\kern0pt}\ rule\ disjI{\isadigit{1}}{\isacharcomma}{\kern0pt}\ rule\ ltD{\isacharcomma}{\kern0pt}\ simp{\isacharcomma}{\kern0pt}\ simp{\isacharparenright}{\kern0pt}\isanewline
\ \ \ \ \ \ \isacommand{apply}\isamarkupfalse%
{\isacharparenleft}{\kern0pt}rule\ ltI{\isacharcomma}{\kern0pt}\ simp\ {\isacharcomma}{\kern0pt}rule\ disjI{\isadigit{1}}{\isacharcomma}{\kern0pt}\ rule\ ltD{\isacharcomma}{\kern0pt}\ simp{\isacharcomma}{\kern0pt}\ simp{\isacharparenright}{\kern0pt}\isanewline
\ \ \ \ \ \isacommand{apply}\isamarkupfalse%
{\isacharparenleft}{\kern0pt}rule\ Un{\isacharunderscore}{\kern0pt}least{\isacharunderscore}{\kern0pt}lt{\isacharcomma}{\kern0pt}\ rule\ le{\isacharunderscore}{\kern0pt}lt{\isacharunderscore}{\kern0pt}lt{\isacharcomma}{\kern0pt}\ rule\ arity{\isacharunderscore}{\kern0pt}apply{\isacharunderscore}{\kern0pt}all{\isacharunderscore}{\kern0pt}{\isadigit{1}}{\isacharunderscore}{\kern0pt}fm{\isacharcomma}{\kern0pt}\ simp{\isacharcomma}{\kern0pt}\ simp{\isacharcomma}{\kern0pt}\ simp{\isacharparenright}{\kern0pt}\isanewline
\ \ \ \ \ \ \isacommand{apply}\isamarkupfalse%
{\isacharparenleft}{\kern0pt}rule\ Un{\isacharunderscore}{\kern0pt}least{\isacharunderscore}{\kern0pt}lt{\isacharparenright}{\kern0pt}{\isacharplus}{\kern0pt}\isanewline
\ \ \ \ \ \ \ \ \isacommand{apply}\isamarkupfalse%
{\isacharparenleft}{\kern0pt}rule\ ltI{\isacharcomma}{\kern0pt}\ simp{\isacharcomma}{\kern0pt}\ simp{\isacharparenright}{\kern0pt}\isanewline
\ \ \ \ \ \ \ \isacommand{apply}\isamarkupfalse%
{\isacharparenleft}{\kern0pt}rule\ ltI{\isacharcomma}{\kern0pt}\ simp{\isacharcomma}{\kern0pt}\ rule\ disjI{\isadigit{1}}{\isacharcomma}{\kern0pt}\ rule\ ltD{\isacharcomma}{\kern0pt}\ simp{\isacharcomma}{\kern0pt}\ simp{\isacharparenright}{\kern0pt}{\isacharplus}{\kern0pt}\isanewline
\ \ \ \ \ \isacommand{apply}\isamarkupfalse%
{\isacharparenleft}{\kern0pt}rule\ Un{\isacharunderscore}{\kern0pt}least{\isacharunderscore}{\kern0pt}lt{\isacharparenright}{\kern0pt}{\isacharplus}{\kern0pt}\isanewline
\ \ \ \ \ \ \isacommand{apply}\isamarkupfalse%
{\isacharparenleft}{\kern0pt}rule\ le{\isacharunderscore}{\kern0pt}lt{\isacharunderscore}{\kern0pt}lt{\isacharparenright}{\kern0pt}\isanewline
\ \ \ \ \ \ \ \isacommand{apply}\isamarkupfalse%
{\isacharparenleft}{\kern0pt}rule\ arity{\isacharunderscore}{\kern0pt}is{\isacharunderscore}{\kern0pt}sym{\isacharunderscore}{\kern0pt}fm{\isacharparenright}{\kern0pt}\isanewline
\ \ \ \ \ \ \ \ \ \ \isacommand{apply}\isamarkupfalse%
\ auto{\isacharbrackleft}{\kern0pt}{\isadigit{4}}{\isacharbrackright}{\kern0pt}\isanewline
\ \ \ \ \ \ \isacommand{apply}\isamarkupfalse%
{\isacharparenleft}{\kern0pt}rule\ Un{\isacharunderscore}{\kern0pt}least{\isacharunderscore}{\kern0pt}lt{\isacharparenright}{\kern0pt}{\isacharplus}{\kern0pt}\isanewline
\ \ \ \ \ \ \ \ \ \isacommand{apply}\isamarkupfalse%
{\isacharparenleft}{\kern0pt}rule\ ltI{\isacharcomma}{\kern0pt}\ simp{\isacharcomma}{\kern0pt}\ rule\ disjI{\isadigit{1}}{\isacharcomma}{\kern0pt}\ rule\ ltD{\isacharcomma}{\kern0pt}\ simp{\isacharcomma}{\kern0pt}\ simp{\isacharparenright}{\kern0pt}{\isacharplus}{\kern0pt}\isanewline
\ \ \ \ \isacommand{apply}\isamarkupfalse%
{\isacharparenleft}{\kern0pt}rule\ is{\isacharunderscore}{\kern0pt}P{\isacharunderscore}{\kern0pt}name{\isacharunderscore}{\kern0pt}fm{\isacharunderscore}{\kern0pt}type{\isacharparenright}{\kern0pt}\isanewline
\ \ \ \ \ \isacommand{apply}\isamarkupfalse%
\ auto{\isacharbrackleft}{\kern0pt}{\isadigit{2}}{\isacharbrackright}{\kern0pt}\isanewline
\ \ \ \isacommand{apply}\isamarkupfalse%
{\isacharparenleft}{\kern0pt}rule\ is{\isacharunderscore}{\kern0pt}sym{\isacharunderscore}{\kern0pt}fm{\isacharunderscore}{\kern0pt}type{\isacharparenright}{\kern0pt}\isanewline
\ \ \ \ \ \ \isacommand{apply}\isamarkupfalse%
\ auto{\isacharbrackleft}{\kern0pt}{\isadigit{4}}{\isacharbrackright}{\kern0pt}\isanewline
\ \ \isacommand{apply}\isamarkupfalse%
{\isacharparenleft}{\kern0pt}rule\ apply{\isacharunderscore}{\kern0pt}all{\isacharunderscore}{\kern0pt}{\isadigit{1}}{\isacharunderscore}{\kern0pt}fm{\isacharunderscore}{\kern0pt}type{\isacharparenright}{\kern0pt}\isanewline
\ \ \isacommand{using}\isamarkupfalse%
\ assms\ \isanewline
\ \ \isacommand{by}\isamarkupfalse%
\ auto%
\endisatagproof
{\isafoldproof}%
%
\isadelimproof
\isanewline
%
\endisadelimproof
\isanewline
\isacommand{end}\isamarkupfalse%
\isanewline
\isanewline
\isacommand{definition}\isamarkupfalse%
\ His{\isacharunderscore}{\kern0pt}HS{\isacharunderscore}{\kern0pt}M{\isacharunderscore}{\kern0pt}cond{\isacharunderscore}{\kern0pt}fm{\isacharunderscore}{\kern0pt}ren\ \isakeyword{where}\ {\isachardoublequoteopen}His{\isacharunderscore}{\kern0pt}HS{\isacharunderscore}{\kern0pt}M{\isacharunderscore}{\kern0pt}cond{\isacharunderscore}{\kern0pt}fm{\isacharunderscore}{\kern0pt}ren{\isacharparenleft}{\kern0pt}i{\isacharcomma}{\kern0pt}\ j{\isacharparenright}{\kern0pt}\ {\isasymequiv}\ {\isacharbraceleft}{\kern0pt}\ {\isacharless}{\kern0pt}{\isadigit{0}}{\isacharcomma}{\kern0pt}\ i{\isachargreater}{\kern0pt}{\isacharcomma}{\kern0pt}\ {\isacharless}{\kern0pt}{\isadigit{1}}{\isacharcomma}{\kern0pt}\ j{\isachargreater}{\kern0pt}\ {\isacharbraceright}{\kern0pt}{\isachardoublequoteclose}\ \isanewline
\isacommand{definition}\isamarkupfalse%
\ His{\isacharunderscore}{\kern0pt}HS{\isacharunderscore}{\kern0pt}M{\isacharunderscore}{\kern0pt}cond{\isacharunderscore}{\kern0pt}fm{\isacharprime}{\kern0pt}\ \isakeyword{where}\ {\isachardoublequoteopen}His{\isacharunderscore}{\kern0pt}HS{\isacharunderscore}{\kern0pt}M{\isacharunderscore}{\kern0pt}cond{\isacharunderscore}{\kern0pt}fm{\isacharprime}{\kern0pt}{\isacharparenleft}{\kern0pt}i{\isacharcomma}{\kern0pt}\ j{\isacharparenright}{\kern0pt}\ {\isasymequiv}\ ren{\isacharparenleft}{\kern0pt}His{\isacharunderscore}{\kern0pt}HS{\isacharunderscore}{\kern0pt}M{\isacharunderscore}{\kern0pt}cond{\isacharunderscore}{\kern0pt}fm{\isacharparenleft}{\kern0pt}{\isadigit{0}}{\isacharcomma}{\kern0pt}\ {\isadigit{1}}{\isacharparenright}{\kern0pt}{\isacharparenright}{\kern0pt}{\isacharbackquote}{\kern0pt}{\isadigit{2}}{\isacharbackquote}{\kern0pt}{\isacharparenleft}{\kern0pt}succ{\isacharparenleft}{\kern0pt}i{\isacharparenright}{\kern0pt}\ {\isasymunion}\ succ{\isacharparenleft}{\kern0pt}j{\isacharparenright}{\kern0pt}{\isacharparenright}{\kern0pt}{\isacharbackquote}{\kern0pt}His{\isacharunderscore}{\kern0pt}HS{\isacharunderscore}{\kern0pt}M{\isacharunderscore}{\kern0pt}cond{\isacharunderscore}{\kern0pt}fm{\isacharunderscore}{\kern0pt}ren{\isacharparenleft}{\kern0pt}i{\isacharcomma}{\kern0pt}\ j{\isacharparenright}{\kern0pt}{\isachardoublequoteclose}\ \isanewline
\isanewline
\isacommand{context}\isamarkupfalse%
\ M{\isacharunderscore}{\kern0pt}symmetric{\isacharunderscore}{\kern0pt}system\isanewline
\isakeyword{begin}\isanewline
\isanewline
\isacommand{lemma}\isamarkupfalse%
\ His{\isacharunderscore}{\kern0pt}HS{\isacharunderscore}{\kern0pt}M{\isacharunderscore}{\kern0pt}cond{\isacharunderscore}{\kern0pt}fm{\isacharprime}{\kern0pt}{\isacharunderscore}{\kern0pt}type\ {\isacharcolon}{\kern0pt}\ \isanewline
\ \ \isakeyword{fixes}\ i\ j\ \isanewline
\ \ \isakeyword{assumes}\ {\isachardoublequoteopen}i\ {\isasymin}\ nat{\isachardoublequoteclose}\ {\isachardoublequoteopen}j\ {\isasymin}\ nat{\isachardoublequoteclose}\isanewline
\ \ \isakeyword{shows}\ {\isachardoublequoteopen}His{\isacharunderscore}{\kern0pt}HS{\isacharunderscore}{\kern0pt}M{\isacharunderscore}{\kern0pt}cond{\isacharunderscore}{\kern0pt}fm{\isacharprime}{\kern0pt}{\isacharparenleft}{\kern0pt}i{\isacharcomma}{\kern0pt}\ j{\isacharparenright}{\kern0pt}\ {\isasymin}\ formula{\isachardoublequoteclose}\ \isanewline
%
\isadelimproof
\ \ %
\endisadelimproof
%
\isatagproof
\isacommand{unfolding}\isamarkupfalse%
\ His{\isacharunderscore}{\kern0pt}HS{\isacharunderscore}{\kern0pt}M{\isacharunderscore}{\kern0pt}cond{\isacharunderscore}{\kern0pt}fm{\isacharprime}{\kern0pt}{\isacharunderscore}{\kern0pt}def\ \isanewline
\ \ \isacommand{apply}\isamarkupfalse%
{\isacharparenleft}{\kern0pt}rule\ ren{\isacharunderscore}{\kern0pt}tc{\isacharparenright}{\kern0pt}\isanewline
\ \ \ \ \ \isacommand{apply}\isamarkupfalse%
{\isacharparenleft}{\kern0pt}rule\ His{\isacharunderscore}{\kern0pt}HS{\isacharunderscore}{\kern0pt}M{\isacharunderscore}{\kern0pt}cond{\isacharunderscore}{\kern0pt}fm{\isacharunderscore}{\kern0pt}type{\isacharparenright}{\kern0pt}\isanewline
\ \ \isacommand{using}\isamarkupfalse%
\ assms\ \isanewline
\ \ \ \ \ \ \isacommand{apply}\isamarkupfalse%
\ auto{\isacharbrackleft}{\kern0pt}{\isadigit{4}}{\isacharbrackright}{\kern0pt}\ \isanewline
\ \ \isacommand{apply}\isamarkupfalse%
{\isacharparenleft}{\kern0pt}rule\ Pi{\isacharunderscore}{\kern0pt}memberI{\isacharparenright}{\kern0pt}\isanewline
\ \ \isacommand{unfolding}\isamarkupfalse%
\ relation{\isacharunderscore}{\kern0pt}def\ function{\isacharunderscore}{\kern0pt}def\ domain{\isacharunderscore}{\kern0pt}def\ range{\isacharunderscore}{\kern0pt}def\ His{\isacharunderscore}{\kern0pt}HS{\isacharunderscore}{\kern0pt}M{\isacharunderscore}{\kern0pt}cond{\isacharunderscore}{\kern0pt}fm{\isacharunderscore}{\kern0pt}ren{\isacharunderscore}{\kern0pt}def\ \isanewline
\ \ \isacommand{by}\isamarkupfalse%
\ auto%
\endisatagproof
{\isafoldproof}%
%
\isadelimproof
\isanewline
%
\endisadelimproof
\isanewline
\isacommand{lemma}\isamarkupfalse%
\ arity{\isacharunderscore}{\kern0pt}His{\isacharunderscore}{\kern0pt}HS{\isacharunderscore}{\kern0pt}M{\isacharunderscore}{\kern0pt}cond{\isacharunderscore}{\kern0pt}fm{\isacharprime}{\kern0pt}\ {\isacharcolon}{\kern0pt}\ \ \isanewline
\ \ \isakeyword{fixes}\ i\ j\ \isanewline
\ \ \isakeyword{assumes}\ {\isachardoublequoteopen}i\ {\isasymin}\ nat{\isachardoublequoteclose}\ {\isachardoublequoteopen}j\ {\isasymin}\ nat{\isachardoublequoteclose}\isanewline
\ \ \isakeyword{shows}\ {\isachardoublequoteopen}arity{\isacharparenleft}{\kern0pt}His{\isacharunderscore}{\kern0pt}HS{\isacharunderscore}{\kern0pt}M{\isacharunderscore}{\kern0pt}cond{\isacharunderscore}{\kern0pt}fm{\isacharprime}{\kern0pt}{\isacharparenleft}{\kern0pt}i{\isacharcomma}{\kern0pt}\ j{\isacharparenright}{\kern0pt}{\isacharparenright}{\kern0pt}\ {\isasymle}\ succ{\isacharparenleft}{\kern0pt}i{\isacharparenright}{\kern0pt}\ {\isasymunion}\ succ{\isacharparenleft}{\kern0pt}j{\isacharparenright}{\kern0pt}{\isachardoublequoteclose}\isanewline
%
\isadelimproof
\ \ %
\endisadelimproof
%
\isatagproof
\isacommand{unfolding}\isamarkupfalse%
\ His{\isacharunderscore}{\kern0pt}HS{\isacharunderscore}{\kern0pt}M{\isacharunderscore}{\kern0pt}cond{\isacharunderscore}{\kern0pt}fm{\isacharprime}{\kern0pt}{\isacharunderscore}{\kern0pt}def\ \isanewline
\ \ \isacommand{apply}\isamarkupfalse%
{\isacharparenleft}{\kern0pt}rule\ arity{\isacharunderscore}{\kern0pt}ren{\isacharparenright}{\kern0pt}\isanewline
\ \ \ \ \ \ \isacommand{apply}\isamarkupfalse%
{\isacharparenleft}{\kern0pt}rule\ His{\isacharunderscore}{\kern0pt}HS{\isacharunderscore}{\kern0pt}M{\isacharunderscore}{\kern0pt}cond{\isacharunderscore}{\kern0pt}fm{\isacharunderscore}{\kern0pt}type{\isacharparenright}{\kern0pt}\isanewline
\ \ \isacommand{using}\isamarkupfalse%
\ assms\ \isanewline
\ \ \ \ \ \ \ \isacommand{apply}\isamarkupfalse%
\ auto{\isacharbrackleft}{\kern0pt}{\isadigit{4}}{\isacharbrackright}{\kern0pt}\ \isanewline
\ \ \ \isacommand{apply}\isamarkupfalse%
{\isacharparenleft}{\kern0pt}rule\ Pi{\isacharunderscore}{\kern0pt}memberI{\isacharparenright}{\kern0pt}\isanewline
\ \ \isacommand{unfolding}\isamarkupfalse%
\ relation{\isacharunderscore}{\kern0pt}def\ function{\isacharunderscore}{\kern0pt}def\ domain{\isacharunderscore}{\kern0pt}def\ range{\isacharunderscore}{\kern0pt}def\ His{\isacharunderscore}{\kern0pt}HS{\isacharunderscore}{\kern0pt}M{\isacharunderscore}{\kern0pt}cond{\isacharunderscore}{\kern0pt}fm{\isacharunderscore}{\kern0pt}ren{\isacharunderscore}{\kern0pt}def\ \isanewline
\ \ \ \ \ \ \isacommand{apply}\isamarkupfalse%
\ auto{\isacharbrackleft}{\kern0pt}{\isadigit{4}}{\isacharbrackright}{\kern0pt}\isanewline
\ \ \isacommand{apply}\isamarkupfalse%
{\isacharparenleft}{\kern0pt}rule\ le{\isacharunderscore}{\kern0pt}trans{\isacharcomma}{\kern0pt}\ rule\ arity{\isacharunderscore}{\kern0pt}His{\isacharunderscore}{\kern0pt}HS{\isacharunderscore}{\kern0pt}M{\isacharunderscore}{\kern0pt}cond{\isacharunderscore}{\kern0pt}fm{\isacharparenright}{\kern0pt}\isanewline
\ \ \ \ \isacommand{apply}\isamarkupfalse%
\ auto{\isacharbrackleft}{\kern0pt}{\isadigit{2}}{\isacharbrackright}{\kern0pt}\isanewline
\ \ \isacommand{apply}\isamarkupfalse%
{\isacharparenleft}{\kern0pt}rule\ Un{\isacharunderscore}{\kern0pt}least{\isacharunderscore}{\kern0pt}lt{\isacharparenright}{\kern0pt}\isanewline
\ \ \isacommand{by}\isamarkupfalse%
\ auto%
\endisatagproof
{\isafoldproof}%
%
\isadelimproof
\isanewline
%
\endisadelimproof
\isanewline
\isacommand{lemma}\isamarkupfalse%
\ sats{\isacharunderscore}{\kern0pt}His{\isacharunderscore}{\kern0pt}HS{\isacharunderscore}{\kern0pt}M{\isacharunderscore}{\kern0pt}cond{\isacharunderscore}{\kern0pt}fm{\isacharprime}{\kern0pt}{\isacharunderscore}{\kern0pt}iff\ {\isacharcolon}{\kern0pt}\ \isanewline
\ \ \isakeyword{fixes}\ env\ x{\isacharprime}{\kern0pt}\ g\ i\ j\ \isanewline
\ \ \isakeyword{assumes}\ {\isachardoublequoteopen}env\ {\isasymin}\ list{\isacharparenleft}{\kern0pt}M{\isacharparenright}{\kern0pt}{\isachardoublequoteclose}\ {\isachardoublequoteopen}i\ {\isacharless}{\kern0pt}\ length{\isacharparenleft}{\kern0pt}env{\isacharparenright}{\kern0pt}{\isachardoublequoteclose}\ {\isachardoublequoteopen}j\ {\isacharless}{\kern0pt}\ length{\isacharparenleft}{\kern0pt}env{\isacharparenright}{\kern0pt}{\isachardoublequoteclose}\ {\isachardoublequoteopen}nth{\isacharparenleft}{\kern0pt}i{\isacharcomma}{\kern0pt}\ env{\isacharparenright}{\kern0pt}\ {\isacharequal}{\kern0pt}\ x{\isacharprime}{\kern0pt}{\isachardoublequoteclose}\ {\isachardoublequoteopen}nth{\isacharparenleft}{\kern0pt}j{\isacharcomma}{\kern0pt}\ env{\isacharparenright}{\kern0pt}\ {\isacharequal}{\kern0pt}\ g{\isachardoublequoteclose}\ \isanewline
\ \ \isakeyword{shows}\ {\isachardoublequoteopen}sats{\isacharparenleft}{\kern0pt}M{\isacharcomma}{\kern0pt}\ His{\isacharunderscore}{\kern0pt}HS{\isacharunderscore}{\kern0pt}M{\isacharunderscore}{\kern0pt}cond{\isacharunderscore}{\kern0pt}fm{\isacharparenleft}{\kern0pt}{\isadigit{0}}{\isacharcomma}{\kern0pt}\ {\isadigit{1}}{\isacharparenright}{\kern0pt}{\isacharcomma}{\kern0pt}\ {\isacharbrackleft}{\kern0pt}x{\isacharprime}{\kern0pt}{\isacharcomma}{\kern0pt}\ g{\isacharbrackright}{\kern0pt}{\isacharparenright}{\kern0pt}\ {\isasymlongleftrightarrow}\ sats{\isacharparenleft}{\kern0pt}M{\isacharcomma}{\kern0pt}\ His{\isacharunderscore}{\kern0pt}HS{\isacharunderscore}{\kern0pt}M{\isacharunderscore}{\kern0pt}cond{\isacharunderscore}{\kern0pt}fm{\isacharprime}{\kern0pt}{\isacharparenleft}{\kern0pt}i{\isacharcomma}{\kern0pt}\ j{\isacharparenright}{\kern0pt}{\isacharcomma}{\kern0pt}\ env{\isacharparenright}{\kern0pt}{\isachardoublequoteclose}\ \isanewline
%
\isadelimproof
\isanewline
\ \ %
\endisadelimproof
%
\isatagproof
\isacommand{unfolding}\isamarkupfalse%
\ His{\isacharunderscore}{\kern0pt}HS{\isacharunderscore}{\kern0pt}M{\isacharunderscore}{\kern0pt}cond{\isacharunderscore}{\kern0pt}fm{\isacharprime}{\kern0pt}{\isacharunderscore}{\kern0pt}def\isanewline
\ \ \isacommand{apply}\isamarkupfalse%
{\isacharparenleft}{\kern0pt}rule\ sats{\isacharunderscore}{\kern0pt}iff{\isacharunderscore}{\kern0pt}sats{\isacharunderscore}{\kern0pt}ren{\isacharparenright}{\kern0pt}\isanewline
\ \ \ \ \ \ \ \ \ \isacommand{apply}\isamarkupfalse%
{\isacharparenleft}{\kern0pt}rule\ His{\isacharunderscore}{\kern0pt}HS{\isacharunderscore}{\kern0pt}M{\isacharunderscore}{\kern0pt}cond{\isacharunderscore}{\kern0pt}fm{\isacharunderscore}{\kern0pt}type{\isacharparenright}{\kern0pt}\isanewline
\ \ \isacommand{using}\isamarkupfalse%
\ assms\ nth{\isacharunderscore}{\kern0pt}type\ \ \isanewline
\ \ \ \ \ \ \ \ \ \ \isacommand{apply}\isamarkupfalse%
\ auto{\isacharbrackleft}{\kern0pt}{\isadigit{6}}{\isacharbrackright}{\kern0pt}\isanewline
\ \ \ \ \ \isacommand{apply}\isamarkupfalse%
{\isacharparenleft}{\kern0pt}subgoal{\isacharunderscore}{\kern0pt}tac\ {\isachardoublequoteopen}i\ {\isasymin}\ nat\ {\isasymand}\ j\ {\isasymin}\ nat{\isachardoublequoteclose}{\isacharparenright}{\kern0pt}\isanewline
\ \ \ \ \ \ \isacommand{apply}\isamarkupfalse%
\ force\isanewline
\ \ \isacommand{using}\isamarkupfalse%
\ lt{\isacharunderscore}{\kern0pt}nat{\isacharunderscore}{\kern0pt}in{\isacharunderscore}{\kern0pt}nat\ \isanewline
\ \ \ \ \ \isacommand{apply}\isamarkupfalse%
\ simp\isanewline
\ \ \isacommand{unfolding}\isamarkupfalse%
\ His{\isacharunderscore}{\kern0pt}HS{\isacharunderscore}{\kern0pt}M{\isacharunderscore}{\kern0pt}cond{\isacharunderscore}{\kern0pt}fm{\isacharunderscore}{\kern0pt}ren{\isacharunderscore}{\kern0pt}def\ \isanewline
\ \ \ \ \isacommand{apply}\isamarkupfalse%
{\isacharparenleft}{\kern0pt}rule\ Pi{\isacharunderscore}{\kern0pt}memberI{\isacharparenright}{\kern0pt}\isanewline
\ \ \isacommand{unfolding}\isamarkupfalse%
\ relation{\isacharunderscore}{\kern0pt}def\ function{\isacharunderscore}{\kern0pt}def\ \isanewline
\ \ \ \ \ \ \ \isacommand{apply}\isamarkupfalse%
\ auto{\isacharbrackleft}{\kern0pt}{\isadigit{4}}{\isacharbrackright}{\kern0pt}\isanewline
\ \ \ \isacommand{apply}\isamarkupfalse%
{\isacharparenleft}{\kern0pt}rule\ le{\isacharunderscore}{\kern0pt}trans{\isacharcomma}{\kern0pt}\ rule\ arity{\isacharunderscore}{\kern0pt}His{\isacharunderscore}{\kern0pt}HS{\isacharunderscore}{\kern0pt}M{\isacharunderscore}{\kern0pt}cond{\isacharunderscore}{\kern0pt}fm{\isacharcomma}{\kern0pt}\ simp{\isacharcomma}{\kern0pt}\ simp{\isacharparenright}{\kern0pt}\isanewline
\ \ \ \isacommand{apply}\isamarkupfalse%
{\isacharparenleft}{\kern0pt}rule\ Un{\isacharunderscore}{\kern0pt}least{\isacharunderscore}{\kern0pt}lt{\isacharcomma}{\kern0pt}\ simp{\isacharcomma}{\kern0pt}\ simp{\isacharparenright}{\kern0pt}\isanewline
\ \ \isacommand{apply}\isamarkupfalse%
{\isacharparenleft}{\kern0pt}rename{\isacharunderscore}{\kern0pt}tac\ k{\isacharcomma}{\kern0pt}\ subgoal{\isacharunderscore}{\kern0pt}tac\ {\isachardoublequoteopen}k\ {\isacharequal}{\kern0pt}\ {\isadigit{0}}\ {\isasymor}\ k\ {\isacharequal}{\kern0pt}\ {\isadigit{1}}{\isachardoublequoteclose}{\isacharparenright}{\kern0pt}\isanewline
\ \ \ \isacommand{apply}\isamarkupfalse%
\ auto{\isacharbrackleft}{\kern0pt}{\isadigit{1}}{\isacharbrackright}{\kern0pt}\isanewline
\ \ \ \ \isacommand{apply}\isamarkupfalse%
{\isacharparenleft}{\kern0pt}rule{\isacharunderscore}{\kern0pt}tac\ b{\isacharequal}{\kern0pt}{\isachardoublequoteopen}{\isacharbraceleft}{\kern0pt}{\isasymlangle}{\isadigit{0}}{\isacharcomma}{\kern0pt}\ i{\isasymrangle}{\isacharcomma}{\kern0pt}\ {\isasymlangle}{\isadigit{1}}{\isacharcomma}{\kern0pt}\ j{\isasymrangle}{\isacharbraceright}{\kern0pt}\ {\isacharbackquote}{\kern0pt}\ {\isadigit{0}}{\isachardoublequoteclose}\ \isakeyword{and}\ a{\isacharequal}{\kern0pt}i\ \isakeyword{in}\ ssubst{\isacharparenright}{\kern0pt}\isanewline
\ \ \ \ \ \isacommand{apply}\isamarkupfalse%
{\isacharparenleft}{\kern0pt}rule\ function{\isacharunderscore}{\kern0pt}apply{\isacharunderscore}{\kern0pt}equality{\isacharcomma}{\kern0pt}\ simp{\isacharcomma}{\kern0pt}\ simp\ add{\isacharcolon}{\kern0pt}function{\isacharunderscore}{\kern0pt}def{\isacharcomma}{\kern0pt}\ force{\isacharcomma}{\kern0pt}\ simp\ add{\isacharcolon}{\kern0pt}assms{\isacharparenright}{\kern0pt}\isanewline
\ \ \ \ \ \isacommand{apply}\isamarkupfalse%
{\isacharparenleft}{\kern0pt}rule{\isacharunderscore}{\kern0pt}tac\ b{\isacharequal}{\kern0pt}{\isachardoublequoteopen}{\isacharbraceleft}{\kern0pt}{\isasymlangle}{\isadigit{0}}{\isacharcomma}{\kern0pt}\ i{\isasymrangle}{\isacharcomma}{\kern0pt}\ {\isasymlangle}{\isadigit{1}}{\isacharcomma}{\kern0pt}\ j{\isasymrangle}{\isacharbraceright}{\kern0pt}\ {\isacharbackquote}{\kern0pt}\ {\isadigit{1}}{\isachardoublequoteclose}\ \isakeyword{and}\ a{\isacharequal}{\kern0pt}j\ \isakeyword{in}\ ssubst{\isacharparenright}{\kern0pt}\isanewline
\ \ \ \ \isacommand{apply}\isamarkupfalse%
{\isacharparenleft}{\kern0pt}rule\ function{\isacharunderscore}{\kern0pt}apply{\isacharunderscore}{\kern0pt}equality{\isacharcomma}{\kern0pt}\ simp{\isacharcomma}{\kern0pt}\ simp\ add{\isacharcolon}{\kern0pt}function{\isacharunderscore}{\kern0pt}def{\isacharcomma}{\kern0pt}\ force{\isacharcomma}{\kern0pt}\ simp\ add{\isacharcolon}{\kern0pt}assms{\isacharparenright}{\kern0pt}\isanewline
\ \ \isacommand{using}\isamarkupfalse%
\ le{\isacharunderscore}{\kern0pt}iff\ \isanewline
\ \ \isacommand{apply}\isamarkupfalse%
\ force\isanewline
\ \ \isacommand{done}\isamarkupfalse%
%
\endisatagproof
{\isafoldproof}%
%
\isadelimproof
\ \ \ \ \ \ \ \ \ \ \ \ \ \ \ \ \ \ \ \ \ \ \ \ \ \ \ \ \isanewline
%
\endisadelimproof
\isanewline
\isacommand{end}\isamarkupfalse%
\isanewline
\isanewline
\isacommand{definition}\isamarkupfalse%
\ His{\isacharunderscore}{\kern0pt}HS{\isacharunderscore}{\kern0pt}M{\isacharunderscore}{\kern0pt}fm\ \isakeyword{where}\ {\isachardoublequoteopen}His{\isacharunderscore}{\kern0pt}HS{\isacharunderscore}{\kern0pt}M{\isacharunderscore}{\kern0pt}fm{\isacharparenleft}{\kern0pt}x{\isacharprime}{\kern0pt}{\isacharcomma}{\kern0pt}\ g{\isacharcomma}{\kern0pt}\ v{\isacharparenright}{\kern0pt}\ {\isasymequiv}\ Forall{\isacharparenleft}{\kern0pt}Iff{\isacharparenleft}{\kern0pt}Member{\isacharparenleft}{\kern0pt}{\isadigit{0}}{\isacharcomma}{\kern0pt}\ v\ {\isacharhash}{\kern0pt}{\isacharplus}{\kern0pt}\ {\isadigit{1}}{\isacharparenright}{\kern0pt}{\isacharcomma}{\kern0pt}\ And{\isacharparenleft}{\kern0pt}His{\isacharunderscore}{\kern0pt}HS{\isacharunderscore}{\kern0pt}M{\isacharunderscore}{\kern0pt}cond{\isacharunderscore}{\kern0pt}fm{\isacharprime}{\kern0pt}{\isacharparenleft}{\kern0pt}x{\isacharprime}{\kern0pt}\ {\isacharhash}{\kern0pt}{\isacharplus}{\kern0pt}\ {\isadigit{1}}{\isacharcomma}{\kern0pt}\ g\ {\isacharhash}{\kern0pt}{\isacharplus}{\kern0pt}\ {\isadigit{1}}{\isacharparenright}{\kern0pt}{\isacharcomma}{\kern0pt}\ empty{\isacharunderscore}{\kern0pt}fm{\isacharparenleft}{\kern0pt}{\isadigit{0}}{\isacharparenright}{\kern0pt}{\isacharparenright}{\kern0pt}{\isacharparenright}{\kern0pt}{\isacharparenright}{\kern0pt}{\isachardoublequoteclose}\ \isanewline
\isanewline
\isacommand{context}\isamarkupfalse%
\ M{\isacharunderscore}{\kern0pt}symmetric{\isacharunderscore}{\kern0pt}system\isanewline
\isakeyword{begin}\isanewline
\isanewline
\isacommand{lemma}\isamarkupfalse%
\ His{\isacharunderscore}{\kern0pt}HS{\isacharunderscore}{\kern0pt}M{\isacharunderscore}{\kern0pt}fm{\isacharunderscore}{\kern0pt}type\ {\isacharcolon}{\kern0pt}\ \isanewline
\ \ \isakeyword{fixes}\ x{\isacharprime}{\kern0pt}\ g\ v\ \isanewline
\ \ \isakeyword{assumes}\ {\isachardoublequoteopen}x{\isacharprime}{\kern0pt}\ {\isasymin}\ nat{\isachardoublequoteclose}\ {\isachardoublequoteopen}g\ {\isasymin}\ nat{\isachardoublequoteclose}\ {\isachardoublequoteopen}v\ {\isasymin}\ nat{\isachardoublequoteclose}\ \isanewline
\ \ \isakeyword{shows}\ {\isachardoublequoteopen}His{\isacharunderscore}{\kern0pt}HS{\isacharunderscore}{\kern0pt}M{\isacharunderscore}{\kern0pt}fm{\isacharparenleft}{\kern0pt}x{\isacharprime}{\kern0pt}{\isacharcomma}{\kern0pt}\ g{\isacharcomma}{\kern0pt}\ v{\isacharparenright}{\kern0pt}\ {\isasymin}\ formula{\isachardoublequoteclose}\ \isanewline
%
\isadelimproof
\ \ %
\endisadelimproof
%
\isatagproof
\isacommand{unfolding}\isamarkupfalse%
\ His{\isacharunderscore}{\kern0pt}HS{\isacharunderscore}{\kern0pt}M{\isacharunderscore}{\kern0pt}fm{\isacharunderscore}{\kern0pt}def\ \isanewline
\ \ \isacommand{apply}\isamarkupfalse%
{\isacharparenleft}{\kern0pt}subgoal{\isacharunderscore}{\kern0pt}tac\ {\isachardoublequoteopen}His{\isacharunderscore}{\kern0pt}HS{\isacharunderscore}{\kern0pt}M{\isacharunderscore}{\kern0pt}cond{\isacharunderscore}{\kern0pt}fm{\isacharprime}{\kern0pt}{\isacharparenleft}{\kern0pt}x{\isacharprime}{\kern0pt}\ {\isacharhash}{\kern0pt}{\isacharplus}{\kern0pt}\ {\isadigit{1}}{\isacharcomma}{\kern0pt}\ g\ {\isacharhash}{\kern0pt}{\isacharplus}{\kern0pt}\ {\isadigit{1}}{\isacharparenright}{\kern0pt}\ {\isasymin}\ formula{\isachardoublequoteclose}{\isacharparenright}{\kern0pt}\isanewline
\ \ \ \isacommand{apply}\isamarkupfalse%
\ simp\isanewline
\ \ \isacommand{apply}\isamarkupfalse%
{\isacharparenleft}{\kern0pt}rule\ His{\isacharunderscore}{\kern0pt}HS{\isacharunderscore}{\kern0pt}M{\isacharunderscore}{\kern0pt}cond{\isacharunderscore}{\kern0pt}fm{\isacharprime}{\kern0pt}{\isacharunderscore}{\kern0pt}type{\isacharparenright}{\kern0pt}\isanewline
\ \ \isacommand{using}\isamarkupfalse%
\ assms\ \isanewline
\ \ \isacommand{by}\isamarkupfalse%
\ auto%
\endisatagproof
{\isafoldproof}%
%
\isadelimproof
\isanewline
%
\endisadelimproof
\isanewline
\isacommand{lemma}\isamarkupfalse%
\ arity{\isacharunderscore}{\kern0pt}His{\isacharunderscore}{\kern0pt}HS{\isacharunderscore}{\kern0pt}M{\isacharunderscore}{\kern0pt}fm\ {\isacharcolon}{\kern0pt}\ \isanewline
\ \ \isakeyword{fixes}\ x{\isacharprime}{\kern0pt}\ g\ v\ \isanewline
\ \ \isakeyword{assumes}\ {\isachardoublequoteopen}x{\isacharprime}{\kern0pt}\ {\isasymin}\ nat{\isachardoublequoteclose}\ {\isachardoublequoteopen}g\ {\isasymin}\ nat{\isachardoublequoteclose}\ {\isachardoublequoteopen}v\ {\isasymin}\ nat{\isachardoublequoteclose}\ \isanewline
\ \ \isakeyword{shows}\ {\isachardoublequoteopen}arity{\isacharparenleft}{\kern0pt}His{\isacharunderscore}{\kern0pt}HS{\isacharunderscore}{\kern0pt}M{\isacharunderscore}{\kern0pt}fm{\isacharparenleft}{\kern0pt}x{\isacharprime}{\kern0pt}{\isacharcomma}{\kern0pt}\ g{\isacharcomma}{\kern0pt}\ v{\isacharparenright}{\kern0pt}{\isacharparenright}{\kern0pt}\ {\isasymle}\ succ{\isacharparenleft}{\kern0pt}x{\isacharprime}{\kern0pt}{\isacharparenright}{\kern0pt}\ {\isasymunion}\ succ{\isacharparenleft}{\kern0pt}g{\isacharparenright}{\kern0pt}\ {\isasymunion}\ succ{\isacharparenleft}{\kern0pt}v{\isacharparenright}{\kern0pt}{\isachardoublequoteclose}\ \isanewline
%
\isadelimproof
\ \ %
\endisadelimproof
%
\isatagproof
\isacommand{unfolding}\isamarkupfalse%
\ His{\isacharunderscore}{\kern0pt}HS{\isacharunderscore}{\kern0pt}M{\isacharunderscore}{\kern0pt}fm{\isacharunderscore}{\kern0pt}def\ \isanewline
\ \ \isacommand{apply}\isamarkupfalse%
{\isacharparenleft}{\kern0pt}subgoal{\isacharunderscore}{\kern0pt}tac\ {\isachardoublequoteopen}His{\isacharunderscore}{\kern0pt}HS{\isacharunderscore}{\kern0pt}M{\isacharunderscore}{\kern0pt}cond{\isacharunderscore}{\kern0pt}fm{\isacharprime}{\kern0pt}{\isacharparenleft}{\kern0pt}succ{\isacharparenleft}{\kern0pt}x{\isacharprime}{\kern0pt}{\isacharparenright}{\kern0pt}{\isacharcomma}{\kern0pt}\ succ{\isacharparenleft}{\kern0pt}g{\isacharparenright}{\kern0pt}{\isacharparenright}{\kern0pt}\ {\isasymin}\ formula{\isachardoublequoteclose}{\isacharparenright}{\kern0pt}\isanewline
\ \ \isacommand{apply}\isamarkupfalse%
\ {\isacharparenleft}{\kern0pt}simp\ add{\isacharcolon}{\kern0pt}assms{\isacharparenright}{\kern0pt}\isanewline
\ \ \ \isacommand{apply}\isamarkupfalse%
{\isacharparenleft}{\kern0pt}rule\ pred{\isacharunderscore}{\kern0pt}le{\isacharparenright}{\kern0pt}\isanewline
\ \ \ \ \ \isacommand{apply}\isamarkupfalse%
\ {\isacharparenleft}{\kern0pt}rule\ union{\isacharunderscore}{\kern0pt}in{\isacharunderscore}{\kern0pt}nat{\isacharparenright}{\kern0pt}{\isacharplus}{\kern0pt}\isanewline
\ \ \isacommand{using}\isamarkupfalse%
\ assms\isanewline
\ \ \ \ \ \ \ \isacommand{apply}\isamarkupfalse%
\ auto{\isacharbrackleft}{\kern0pt}{\isadigit{4}}{\isacharbrackright}{\kern0pt}\isanewline
\ \ \ \isacommand{apply}\isamarkupfalse%
{\isacharparenleft}{\kern0pt}rule\ Un{\isacharunderscore}{\kern0pt}least{\isacharunderscore}{\kern0pt}lt{\isacharparenright}{\kern0pt}{\isacharplus}{\kern0pt}\isanewline
\ \ \ \ \ \isacommand{apply}\isamarkupfalse%
\ {\isacharparenleft}{\kern0pt}simp{\isacharcomma}{\kern0pt}\ rule\ ltI{\isacharparenright}{\kern0pt}\isanewline
\ \ \ \ \ \ \isacommand{apply}\isamarkupfalse%
{\isacharparenleft}{\kern0pt}subst\ succ{\isacharunderscore}{\kern0pt}Un{\isacharunderscore}{\kern0pt}distrib{\isacharcomma}{\kern0pt}\ simp\ add{\isacharcolon}{\kern0pt}assms{\isacharcomma}{\kern0pt}\ simp\ add{\isacharcolon}{\kern0pt}assms{\isacharparenright}{\kern0pt}{\isacharplus}{\kern0pt}\isanewline
\ \ \ \ \ \ \isacommand{apply}\isamarkupfalse%
\ {\isacharparenleft}{\kern0pt}simp{\isacharcomma}{\kern0pt}\ rule\ disjI{\isadigit{1}}{\isacharcomma}{\kern0pt}\ rule\ ltD{\isacharcomma}{\kern0pt}\ simp\ add{\isacharcolon}{\kern0pt}assms{\isacharcomma}{\kern0pt}\ simp\ add{\isacharcolon}{\kern0pt}assms{\isacharparenright}{\kern0pt}\isanewline
\ \ \ \ \isacommand{apply}\isamarkupfalse%
{\isacharparenleft}{\kern0pt}subst\ succ{\isacharunderscore}{\kern0pt}Un{\isacharunderscore}{\kern0pt}distrib{\isacharcomma}{\kern0pt}\ simp\ add{\isacharcolon}{\kern0pt}assms{\isacharcomma}{\kern0pt}\ simp\ add{\isacharcolon}{\kern0pt}assms{\isacharparenright}{\kern0pt}{\isacharplus}{\kern0pt}\isanewline
\ \ \ \ \isacommand{apply}\isamarkupfalse%
{\isacharparenleft}{\kern0pt}rule\ ltI{\isacharcomma}{\kern0pt}\ simp{\isacharcomma}{\kern0pt}\ simp\ add{\isacharcolon}{\kern0pt}assms{\isacharparenright}{\kern0pt}\isanewline
\ \ \ \isacommand{apply}\isamarkupfalse%
{\isacharparenleft}{\kern0pt}rule\ Un{\isacharunderscore}{\kern0pt}least{\isacharunderscore}{\kern0pt}lt{\isacharcomma}{\kern0pt}\ rule\ le{\isacharunderscore}{\kern0pt}trans{\isacharcomma}{\kern0pt}\ rule\ arity{\isacharunderscore}{\kern0pt}His{\isacharunderscore}{\kern0pt}HS{\isacharunderscore}{\kern0pt}M{\isacharunderscore}{\kern0pt}cond{\isacharunderscore}{\kern0pt}fm{\isacharprime}{\kern0pt}{\isacharcomma}{\kern0pt}\ simp\ add{\isacharcolon}{\kern0pt}assms{\isacharcomma}{\kern0pt}\ simp\ add{\isacharcolon}{\kern0pt}assms{\isacharparenright}{\kern0pt}\isanewline
\ \ \ \ \isacommand{apply}\isamarkupfalse%
{\isacharparenleft}{\kern0pt}rule\ Un{\isacharunderscore}{\kern0pt}least{\isacharunderscore}{\kern0pt}lt{\isacharparenright}{\kern0pt}\isanewline
\ \ \ \ \ \isacommand{apply}\isamarkupfalse%
{\isacharparenleft}{\kern0pt}subst\ succ{\isacharunderscore}{\kern0pt}Un{\isacharunderscore}{\kern0pt}distrib{\isacharcomma}{\kern0pt}\ simp\ add{\isacharcolon}{\kern0pt}assms{\isacharcomma}{\kern0pt}\ simp\ add{\isacharcolon}{\kern0pt}assms{\isacharparenright}{\kern0pt}{\isacharplus}{\kern0pt}\isanewline
\ \ \ \ \ \isacommand{apply}\isamarkupfalse%
{\isacharparenleft}{\kern0pt}rule\ ltI{\isacharcomma}{\kern0pt}\ simp{\isacharcomma}{\kern0pt}\ simp\ add{\isacharcolon}{\kern0pt}assms{\isacharparenright}{\kern0pt}\isanewline
\ \ \ \ \isacommand{apply}\isamarkupfalse%
{\isacharparenleft}{\kern0pt}subst\ succ{\isacharunderscore}{\kern0pt}Un{\isacharunderscore}{\kern0pt}distrib{\isacharcomma}{\kern0pt}\ simp\ add{\isacharcolon}{\kern0pt}assms{\isacharcomma}{\kern0pt}\ simp\ add{\isacharcolon}{\kern0pt}assms{\isacharparenright}{\kern0pt}{\isacharplus}{\kern0pt}\isanewline
\ \ \ \ \isacommand{apply}\isamarkupfalse%
{\isacharparenleft}{\kern0pt}rule\ ltI{\isacharcomma}{\kern0pt}\ simp{\isacharcomma}{\kern0pt}\ simp\ add{\isacharcolon}{\kern0pt}assms{\isacharparenright}{\kern0pt}\isanewline
\ \ \ \isacommand{apply}\isamarkupfalse%
\ {\isacharparenleft}{\kern0pt}subst\ arity{\isacharunderscore}{\kern0pt}empty{\isacharunderscore}{\kern0pt}fm{\isacharcomma}{\kern0pt}\ simp{\isacharcomma}{\kern0pt}\ simp{\isacharparenright}{\kern0pt}\isanewline
\ \ \isacommand{apply}\isamarkupfalse%
\ {\isacharparenleft}{\kern0pt}rule\ ltI{\isacharparenright}{\kern0pt}\isanewline
\ \ \ \ \isacommand{apply}\isamarkupfalse%
{\isacharparenleft}{\kern0pt}subst\ succ{\isacharunderscore}{\kern0pt}Un{\isacharunderscore}{\kern0pt}distrib{\isacharcomma}{\kern0pt}\ simp\ add{\isacharcolon}{\kern0pt}assms{\isacharcomma}{\kern0pt}\ simp\ add{\isacharcolon}{\kern0pt}assms{\isacharparenright}{\kern0pt}{\isacharplus}{\kern0pt}\isanewline
\ \ \ \ \isacommand{apply}\isamarkupfalse%
{\isacharparenleft}{\kern0pt}simp{\isacharcomma}{\kern0pt}\ rule\ disjI{\isadigit{1}}{\isacharcomma}{\kern0pt}\ rule\ ltD{\isacharcomma}{\kern0pt}\ simp\ add{\isacharcolon}{\kern0pt}assms{\isacharcomma}{\kern0pt}\ simp\ add{\isacharcolon}{\kern0pt}assms{\isacharparenright}{\kern0pt}\isanewline
\ \ \isacommand{apply}\isamarkupfalse%
{\isacharparenleft}{\kern0pt}rule\ His{\isacharunderscore}{\kern0pt}HS{\isacharunderscore}{\kern0pt}M{\isacharunderscore}{\kern0pt}cond{\isacharunderscore}{\kern0pt}fm{\isacharprime}{\kern0pt}{\isacharunderscore}{\kern0pt}type{\isacharparenright}{\kern0pt}\isanewline
\ \ \isacommand{using}\isamarkupfalse%
\ assms\isanewline
\ \ \isacommand{by}\isamarkupfalse%
\ auto%
\endisatagproof
{\isafoldproof}%
%
\isadelimproof
\isanewline
%
\endisadelimproof
\isanewline
\isacommand{definition}\isamarkupfalse%
\ His{\isacharunderscore}{\kern0pt}HS{\isacharunderscore}{\kern0pt}M\ \isakeyword{where}\ \ \ \ \ \ \ \ \ \ \ \ \ \ \ \ \ \ \ \ \ \ \isanewline
\ \ {\isachardoublequoteopen}His{\isacharunderscore}{\kern0pt}HS{\isacharunderscore}{\kern0pt}M{\isacharparenleft}{\kern0pt}x{\isacharprime}{\kern0pt}{\isacharcomma}{\kern0pt}\ g{\isacharparenright}{\kern0pt}\ {\isacharequal}{\kern0pt}\ {\isacharparenleft}{\kern0pt}if\ {\isacharparenleft}{\kern0pt}sats{\isacharparenleft}{\kern0pt}M{\isacharcomma}{\kern0pt}\ His{\isacharunderscore}{\kern0pt}HS{\isacharunderscore}{\kern0pt}M{\isacharunderscore}{\kern0pt}cond{\isacharunderscore}{\kern0pt}fm{\isacharparenleft}{\kern0pt}{\isadigit{0}}{\isacharcomma}{\kern0pt}\ {\isadigit{1}}{\isacharparenright}{\kern0pt}{\isacharcomma}{\kern0pt}\ {\isacharbrackleft}{\kern0pt}x{\isacharprime}{\kern0pt}{\isacharcomma}{\kern0pt}\ g{\isacharbrackright}{\kern0pt}{\isacharparenright}{\kern0pt}{\isacharparenright}{\kern0pt}\ then\ {\isadigit{1}}\ else\ {\isadigit{0}}{\isacharparenright}{\kern0pt}{\isachardoublequoteclose}\ \ \isanewline
\isanewline
\isacommand{lemma}\isamarkupfalse%
\ sats{\isacharunderscore}{\kern0pt}His{\isacharunderscore}{\kern0pt}HS{\isacharunderscore}{\kern0pt}M{\isacharunderscore}{\kern0pt}fm{\isacharunderscore}{\kern0pt}iff\ {\isacharcolon}{\kern0pt}\ \isanewline
\ \ \isakeyword{fixes}\ env\ i\ j\ k\ x{\isacharprime}{\kern0pt}\ g\ v\isanewline
\ \ \isakeyword{assumes}\ {\isachardoublequoteopen}env\ {\isasymin}\ list{\isacharparenleft}{\kern0pt}M{\isacharparenright}{\kern0pt}{\isachardoublequoteclose}\ {\isachardoublequoteopen}i\ {\isacharless}{\kern0pt}\ length{\isacharparenleft}{\kern0pt}env{\isacharparenright}{\kern0pt}{\isachardoublequoteclose}\ {\isachardoublequoteopen}j\ {\isacharless}{\kern0pt}\ length{\isacharparenleft}{\kern0pt}env{\isacharparenright}{\kern0pt}{\isachardoublequoteclose}\ {\isachardoublequoteopen}k\ {\isacharless}{\kern0pt}\ length{\isacharparenleft}{\kern0pt}env{\isacharparenright}{\kern0pt}{\isachardoublequoteclose}\ {\isachardoublequoteopen}nth{\isacharparenleft}{\kern0pt}i{\isacharcomma}{\kern0pt}\ env{\isacharparenright}{\kern0pt}\ {\isacharequal}{\kern0pt}\ x{\isacharprime}{\kern0pt}{\isachardoublequoteclose}\ {\isachardoublequoteopen}nth{\isacharparenleft}{\kern0pt}j{\isacharcomma}{\kern0pt}\ env{\isacharparenright}{\kern0pt}\ {\isacharequal}{\kern0pt}\ g{\isachardoublequoteclose}\ {\isachardoublequoteopen}nth{\isacharparenleft}{\kern0pt}k{\isacharcomma}{\kern0pt}\ env{\isacharparenright}{\kern0pt}\ {\isacharequal}{\kern0pt}\ v{\isachardoublequoteclose}\isanewline
\ \ \isakeyword{shows}\ {\isachardoublequoteopen}sats{\isacharparenleft}{\kern0pt}M{\isacharcomma}{\kern0pt}\ His{\isacharunderscore}{\kern0pt}HS{\isacharunderscore}{\kern0pt}M{\isacharunderscore}{\kern0pt}fm{\isacharparenleft}{\kern0pt}i{\isacharcomma}{\kern0pt}\ j{\isacharcomma}{\kern0pt}\ k{\isacharparenright}{\kern0pt}{\isacharcomma}{\kern0pt}\ env{\isacharparenright}{\kern0pt}\ {\isasymlongleftrightarrow}\ v\ {\isacharequal}{\kern0pt}\ His{\isacharunderscore}{\kern0pt}HS{\isacharunderscore}{\kern0pt}M{\isacharparenleft}{\kern0pt}x{\isacharprime}{\kern0pt}{\isacharcomma}{\kern0pt}\ g{\isacharparenright}{\kern0pt}{\isachardoublequoteclose}\ \isanewline
%
\isadelimproof
\isanewline
%
\endisadelimproof
%
\isatagproof
\isacommand{proof}\isamarkupfalse%
\ {\isacharminus}{\kern0pt}\ \isanewline
\ \ \isacommand{have}\isamarkupfalse%
\ innat\ {\isacharcolon}{\kern0pt}\ {\isachardoublequoteopen}i\ {\isasymin}\ nat\ {\isasymand}\ j\ {\isasymin}\ nat\ {\isasymand}\ k\ {\isasymin}\ nat{\isachardoublequoteclose}\ \isacommand{using}\isamarkupfalse%
\ assms\ lt{\isacharunderscore}{\kern0pt}nat{\isacharunderscore}{\kern0pt}in{\isacharunderscore}{\kern0pt}nat\ \isacommand{by}\isamarkupfalse%
\ auto\isanewline
\ \ \isacommand{have}\isamarkupfalse%
\ inM\ {\isacharcolon}{\kern0pt}\ {\isachardoublequoteopen}x{\isacharprime}{\kern0pt}\ {\isasymin}\ M\ {\isasymand}\ g\ {\isasymin}\ M\ {\isasymand}\ v\ {\isasymin}\ M{\isachardoublequoteclose}\ \isacommand{using}\isamarkupfalse%
\ nth{\isacharunderscore}{\kern0pt}type\ assms\ \isacommand{by}\isamarkupfalse%
\ auto\isanewline
\isanewline
\ \ \isacommand{have}\isamarkupfalse%
\ {\isachardoublequoteopen}sats{\isacharparenleft}{\kern0pt}M{\isacharcomma}{\kern0pt}\ His{\isacharunderscore}{\kern0pt}HS{\isacharunderscore}{\kern0pt}M{\isacharunderscore}{\kern0pt}fm{\isacharparenleft}{\kern0pt}i{\isacharcomma}{\kern0pt}\ j{\isacharcomma}{\kern0pt}\ k{\isacharparenright}{\kern0pt}{\isacharcomma}{\kern0pt}\ env{\isacharparenright}{\kern0pt}\ {\isasymlongleftrightarrow}\ \isanewline
\ \ \ \ \ \ \ \ {\isacharparenleft}{\kern0pt}{\isasymforall}s\ {\isasymin}\ M{\isachardot}{\kern0pt}\ s\ {\isasymin}\ v\ {\isasymlongleftrightarrow}\ sats{\isacharparenleft}{\kern0pt}M{\isacharcomma}{\kern0pt}\ His{\isacharunderscore}{\kern0pt}HS{\isacharunderscore}{\kern0pt}M{\isacharunderscore}{\kern0pt}cond{\isacharunderscore}{\kern0pt}fm{\isacharprime}{\kern0pt}{\isacharparenleft}{\kern0pt}i\ {\isacharhash}{\kern0pt}{\isacharplus}{\kern0pt}\ {\isadigit{1}}{\isacharcomma}{\kern0pt}\ j\ {\isacharhash}{\kern0pt}{\isacharplus}{\kern0pt}\ {\isadigit{1}}{\isacharparenright}{\kern0pt}{\isacharcomma}{\kern0pt}\ Cons{\isacharparenleft}{\kern0pt}s{\isacharcomma}{\kern0pt}\ env{\isacharparenright}{\kern0pt}{\isacharparenright}{\kern0pt}\ {\isasymand}\ s\ {\isacharequal}{\kern0pt}\ {\isadigit{0}}{\isacharparenright}{\kern0pt}{\isachardoublequoteclose}\ \isanewline
\ \ \ \ \isacommand{unfolding}\isamarkupfalse%
\ His{\isacharunderscore}{\kern0pt}HS{\isacharunderscore}{\kern0pt}M{\isacharunderscore}{\kern0pt}fm{\isacharunderscore}{\kern0pt}def\ \isanewline
\ \ \ \ \isacommand{apply}\isamarkupfalse%
{\isacharparenleft}{\kern0pt}rule\ iff{\isacharunderscore}{\kern0pt}trans{\isacharcomma}{\kern0pt}\ rule\ sats{\isacharunderscore}{\kern0pt}Forall{\isacharunderscore}{\kern0pt}iff{\isacharcomma}{\kern0pt}\ simp\ add{\isacharcolon}{\kern0pt}assms{\isacharcomma}{\kern0pt}\ rule\ ball{\isacharunderscore}{\kern0pt}iff{\isacharparenright}{\kern0pt}\isanewline
\ \ \ \ \isacommand{apply}\isamarkupfalse%
{\isacharparenleft}{\kern0pt}rule\ iff{\isacharunderscore}{\kern0pt}trans{\isacharcomma}{\kern0pt}\ rule\ sats{\isacharunderscore}{\kern0pt}Iff{\isacharunderscore}{\kern0pt}iff{\isacharcomma}{\kern0pt}\ simp\ add{\isacharcolon}{\kern0pt}assms{\isacharcomma}{\kern0pt}\ rule\ iff{\isacharunderscore}{\kern0pt}iff{\isacharparenright}{\kern0pt}\isanewline
\ \ \ \ \ \isacommand{apply}\isamarkupfalse%
{\isacharparenleft}{\kern0pt}simp\ add{\isacharcolon}{\kern0pt}assms\ innat\ inM{\isacharparenright}{\kern0pt}\isanewline
\ \ \ \ \isacommand{apply}\isamarkupfalse%
{\isacharparenleft}{\kern0pt}rule\ iff{\isacharunderscore}{\kern0pt}trans{\isacharcomma}{\kern0pt}\ rule\ sats{\isacharunderscore}{\kern0pt}And{\isacharunderscore}{\kern0pt}iff{\isacharcomma}{\kern0pt}\ simp\ add{\isacharcolon}{\kern0pt}assms{\isacharcomma}{\kern0pt}\ rule\ iff{\isacharunderscore}{\kern0pt}conjI{\isacharcomma}{\kern0pt}\ simp{\isacharparenright}{\kern0pt}\isanewline
\ \ \ \ \isacommand{apply}\isamarkupfalse%
{\isacharparenleft}{\kern0pt}rule\ iff{\isacharunderscore}{\kern0pt}trans{\isacharcomma}{\kern0pt}\ rule\ sats{\isacharunderscore}{\kern0pt}empty{\isacharunderscore}{\kern0pt}fm{\isacharcomma}{\kern0pt}\ simp{\isacharcomma}{\kern0pt}\ simp\ add{\isacharcolon}{\kern0pt}assms{\isacharparenright}{\kern0pt}\isanewline
\ \ \ \ \isacommand{using}\isamarkupfalse%
\ assms\ \isanewline
\ \ \ \ \isacommand{apply}\isamarkupfalse%
\ simp\isanewline
\ \ \ \ \isacommand{done}\isamarkupfalse%
\isanewline
\isanewline
\ \ \isacommand{also}\isamarkupfalse%
\ \isacommand{have}\isamarkupfalse%
\ {\isachardoublequoteopen}{\isachardot}{\kern0pt}{\isachardot}{\kern0pt}{\isachardot}{\kern0pt}\ {\isasymlongleftrightarrow}\ {\isacharparenleft}{\kern0pt}{\isasymforall}s\ {\isasymin}\ M{\isachardot}{\kern0pt}\ s\ {\isasymin}\ v\ {\isasymlongleftrightarrow}\ sats{\isacharparenleft}{\kern0pt}M{\isacharcomma}{\kern0pt}\ His{\isacharunderscore}{\kern0pt}HS{\isacharunderscore}{\kern0pt}M{\isacharunderscore}{\kern0pt}cond{\isacharunderscore}{\kern0pt}fm{\isacharparenleft}{\kern0pt}{\isadigit{0}}{\isacharcomma}{\kern0pt}\ {\isadigit{1}}{\isacharparenright}{\kern0pt}{\isacharcomma}{\kern0pt}\ {\isacharbrackleft}{\kern0pt}x{\isacharprime}{\kern0pt}{\isacharcomma}{\kern0pt}\ g{\isacharbrackright}{\kern0pt}{\isacharparenright}{\kern0pt}\ {\isasymand}\ s\ {\isacharequal}{\kern0pt}\ {\isadigit{0}}{\isacharparenright}{\kern0pt}{\isachardoublequoteclose}\ \isanewline
\ \ \ \ \isacommand{apply}\isamarkupfalse%
{\isacharparenleft}{\kern0pt}rule\ ball{\isacharunderscore}{\kern0pt}iff{\isacharcomma}{\kern0pt}\ rule\ iff{\isacharunderscore}{\kern0pt}iff{\isacharcomma}{\kern0pt}\ simp{\isacharcomma}{\kern0pt}\ rule\ iff{\isacharunderscore}{\kern0pt}conjI{\isacharparenright}{\kern0pt}\isanewline
\ \ \ \ \ \isacommand{apply}\isamarkupfalse%
{\isacharparenleft}{\kern0pt}rule\ iff{\isacharunderscore}{\kern0pt}flip{\isacharcomma}{\kern0pt}\ rule\ sats{\isacharunderscore}{\kern0pt}His{\isacharunderscore}{\kern0pt}HS{\isacharunderscore}{\kern0pt}M{\isacharunderscore}{\kern0pt}cond{\isacharunderscore}{\kern0pt}fm{\isacharprime}{\kern0pt}{\isacharunderscore}{\kern0pt}iff{\isacharparenright}{\kern0pt}\isanewline
\ \ \ \ \isacommand{using}\isamarkupfalse%
\ assms\ innat\ inM\isanewline
\ \ \ \ \ \ \ \ \ \isacommand{apply}\isamarkupfalse%
\ auto{\isacharbrackleft}{\kern0pt}{\isadigit{6}}{\isacharbrackright}{\kern0pt}\isanewline
\ \ \ \ \isacommand{done}\isamarkupfalse%
\isanewline
\isanewline
\ \ \isacommand{also}\isamarkupfalse%
\ \isacommand{have}\isamarkupfalse%
\ {\isachardoublequoteopen}{\isachardot}{\kern0pt}{\isachardot}{\kern0pt}{\isachardot}{\kern0pt}\ {\isasymlongleftrightarrow}\ {\isacharparenleft}{\kern0pt}{\isasymforall}s{\isachardot}{\kern0pt}\ s\ {\isasymin}\ v\ {\isasymlongleftrightarrow}\ sats{\isacharparenleft}{\kern0pt}M{\isacharcomma}{\kern0pt}\ His{\isacharunderscore}{\kern0pt}HS{\isacharunderscore}{\kern0pt}M{\isacharunderscore}{\kern0pt}cond{\isacharunderscore}{\kern0pt}fm{\isacharparenleft}{\kern0pt}{\isadigit{0}}{\isacharcomma}{\kern0pt}\ {\isadigit{1}}{\isacharparenright}{\kern0pt}{\isacharcomma}{\kern0pt}\ {\isacharbrackleft}{\kern0pt}x{\isacharprime}{\kern0pt}{\isacharcomma}{\kern0pt}\ g{\isacharbrackright}{\kern0pt}{\isacharparenright}{\kern0pt}\ {\isasymand}\ s\ {\isacharequal}{\kern0pt}\ {\isadigit{0}}{\isacharparenright}{\kern0pt}{\isachardoublequoteclose}\ \isanewline
\ \ \ \ \isacommand{apply}\isamarkupfalse%
{\isacharparenleft}{\kern0pt}rule\ iffI{\isacharcomma}{\kern0pt}\ clarify{\isacharcomma}{\kern0pt}\ rule\ iffI{\isacharparenright}{\kern0pt}\isanewline
\ \ \ \ \ \ \isacommand{apply}\isamarkupfalse%
{\isacharparenleft}{\kern0pt}rename{\isacharunderscore}{\kern0pt}tac\ s{\isacharcomma}{\kern0pt}\ subgoal{\isacharunderscore}{\kern0pt}tac\ {\isachardoublequoteopen}s\ {\isasymin}\ M{\isachardoublequoteclose}{\isacharcomma}{\kern0pt}\ force{\isacharparenright}{\kern0pt}\isanewline
\ \ \ \ \isacommand{using}\isamarkupfalse%
\ inM\ transM\ \isanewline
\ \ \ \ \ \ \isacommand{apply}\isamarkupfalse%
\ force\isanewline
\ \ \ \ \isacommand{using}\isamarkupfalse%
\ zero{\isacharunderscore}{\kern0pt}in{\isacharunderscore}{\kern0pt}M\ \isanewline
\ \ \ \ \ \isacommand{apply}\isamarkupfalse%
\ force\isanewline
\ \ \ \ \isacommand{apply}\isamarkupfalse%
\ auto\isanewline
\ \ \ \ \isacommand{done}\isamarkupfalse%
\isanewline
\isanewline
\ \ \isacommand{also}\isamarkupfalse%
\ \isacommand{have}\isamarkupfalse%
\ {\isachardoublequoteopen}{\isachardot}{\kern0pt}{\isachardot}{\kern0pt}{\isachardot}{\kern0pt}\ {\isasymlongleftrightarrow}\ v\ {\isacharequal}{\kern0pt}\ His{\isacharunderscore}{\kern0pt}HS{\isacharunderscore}{\kern0pt}M{\isacharparenleft}{\kern0pt}x{\isacharprime}{\kern0pt}{\isacharcomma}{\kern0pt}\ g{\isacharparenright}{\kern0pt}{\isachardoublequoteclose}\ \isanewline
\ \ \ \ \isacommand{unfolding}\isamarkupfalse%
\ His{\isacharunderscore}{\kern0pt}HS{\isacharunderscore}{\kern0pt}M{\isacharunderscore}{\kern0pt}def\ \isanewline
\ \ \ \ \isacommand{by}\isamarkupfalse%
\ auto\ \isanewline
\isanewline
\ \ \isacommand{finally}\isamarkupfalse%
\ \isacommand{show}\isamarkupfalse%
\ {\isacharquery}{\kern0pt}thesis\ \isacommand{by}\isamarkupfalse%
\ simp\isanewline
\isacommand{qed}\isamarkupfalse%
%
\endisatagproof
{\isafoldproof}%
%
\isadelimproof
\isanewline
%
\endisadelimproof
\isanewline
\isacommand{end}\isamarkupfalse%
\isanewline
\isanewline
\isacommand{definition}\isamarkupfalse%
\ is{\isacharunderscore}{\kern0pt}HS{\isacharunderscore}{\kern0pt}fm\ \isakeyword{where}\ \isanewline
\ \ {\isachardoublequoteopen}is{\isacharunderscore}{\kern0pt}HS{\isacharunderscore}{\kern0pt}fm{\isacharparenleft}{\kern0pt}FGppauto{\isacharcomma}{\kern0pt}\ x{\isacharparenright}{\kern0pt}\ {\isasymequiv}\ Exists{\isacharparenleft}{\kern0pt}And{\isacharparenleft}{\kern0pt}is{\isacharunderscore}{\kern0pt}memrel{\isacharunderscore}{\kern0pt}wftrec{\isacharunderscore}{\kern0pt}fm{\isacharparenleft}{\kern0pt}His{\isacharunderscore}{\kern0pt}HS{\isacharunderscore}{\kern0pt}M{\isacharunderscore}{\kern0pt}fm{\isacharparenleft}{\kern0pt}{\isadigit{2}}{\isacharcomma}{\kern0pt}\ {\isadigit{1}}{\isacharcomma}{\kern0pt}\ {\isadigit{0}}{\isacharparenright}{\kern0pt}{\isacharcomma}{\kern0pt}\ x\ {\isacharhash}{\kern0pt}{\isacharplus}{\kern0pt}\ {\isadigit{1}}{\isacharcomma}{\kern0pt}\ FGppauto\ {\isacharhash}{\kern0pt}{\isacharplus}{\kern0pt}\ {\isadigit{1}}{\isacharcomma}{\kern0pt}\ {\isadigit{0}}{\isacharparenright}{\kern0pt}{\isacharcomma}{\kern0pt}\ is{\isacharunderscore}{\kern0pt}{\isadigit{1}}{\isacharunderscore}{\kern0pt}fm{\isacharparenleft}{\kern0pt}{\isadigit{0}}{\isacharparenright}{\kern0pt}{\isacharparenright}{\kern0pt}{\isacharparenright}{\kern0pt}{\isachardoublequoteclose}\ \isanewline
\isanewline
\isacommand{context}\isamarkupfalse%
\ M{\isacharunderscore}{\kern0pt}symmetric{\isacharunderscore}{\kern0pt}system\isanewline
\isakeyword{begin}\isanewline
\isanewline
\isacommand{lemma}\isamarkupfalse%
\ is{\isacharunderscore}{\kern0pt}HS{\isacharunderscore}{\kern0pt}fm{\isacharunderscore}{\kern0pt}type\ {\isacharcolon}{\kern0pt}\ \isanewline
\ \ \isakeyword{fixes}\ i\ j\ \isanewline
\ \ \isakeyword{assumes}\ {\isachardoublequoteopen}i\ {\isasymin}\ nat{\isachardoublequoteclose}\ {\isachardoublequoteopen}j\ {\isasymin}\ nat{\isachardoublequoteclose}\ \isanewline
\ \ \isakeyword{shows}\ {\isachardoublequoteopen}is{\isacharunderscore}{\kern0pt}HS{\isacharunderscore}{\kern0pt}fm{\isacharparenleft}{\kern0pt}i{\isacharcomma}{\kern0pt}\ j{\isacharparenright}{\kern0pt}\ {\isasymin}\ formula{\isachardoublequoteclose}\ \isanewline
%
\isadelimproof
\ \ %
\endisadelimproof
%
\isatagproof
\isacommand{unfolding}\isamarkupfalse%
\ is{\isacharunderscore}{\kern0pt}HS{\isacharunderscore}{\kern0pt}fm{\isacharunderscore}{\kern0pt}def\ \isanewline
\ \ \isacommand{apply}\isamarkupfalse%
{\isacharparenleft}{\kern0pt}rule\ Exists{\isacharunderscore}{\kern0pt}type{\isacharcomma}{\kern0pt}\ rule\ And{\isacharunderscore}{\kern0pt}type{\isacharcomma}{\kern0pt}\ rule\ is{\isacharunderscore}{\kern0pt}memrel{\isacharunderscore}{\kern0pt}wftrec{\isacharunderscore}{\kern0pt}fm{\isacharunderscore}{\kern0pt}type{\isacharparenright}{\kern0pt}\isanewline
\ \ \ \ \ \ \isacommand{apply}\isamarkupfalse%
{\isacharparenleft}{\kern0pt}rule\ His{\isacharunderscore}{\kern0pt}HS{\isacharunderscore}{\kern0pt}M{\isacharunderscore}{\kern0pt}fm{\isacharunderscore}{\kern0pt}type{\isacharparenright}{\kern0pt}\isanewline
\ \ \isacommand{using}\isamarkupfalse%
\ assms\ is{\isacharunderscore}{\kern0pt}{\isadigit{1}}{\isacharunderscore}{\kern0pt}fm{\isacharunderscore}{\kern0pt}type\ \isanewline
\ \ \isacommand{by}\isamarkupfalse%
\ auto%
\endisatagproof
{\isafoldproof}%
%
\isadelimproof
\isanewline
%
\endisadelimproof
\isanewline
\isacommand{lemma}\isamarkupfalse%
\ arity{\isacharunderscore}{\kern0pt}is{\isacharunderscore}{\kern0pt}HS{\isacharunderscore}{\kern0pt}fm\ {\isacharcolon}{\kern0pt}\ \isanewline
\ \ \isakeyword{fixes}\ i\ j\ \isanewline
\ \ \isakeyword{assumes}\ {\isachardoublequoteopen}i\ {\isasymin}\ nat{\isachardoublequoteclose}\ {\isachardoublequoteopen}j\ {\isasymin}\ nat{\isachardoublequoteclose}\ \isanewline
\ \ \isakeyword{shows}\ {\isachardoublequoteopen}arity{\isacharparenleft}{\kern0pt}is{\isacharunderscore}{\kern0pt}HS{\isacharunderscore}{\kern0pt}fm{\isacharparenleft}{\kern0pt}i{\isacharcomma}{\kern0pt}\ j{\isacharparenright}{\kern0pt}{\isacharparenright}{\kern0pt}\ {\isasymle}\ succ{\isacharparenleft}{\kern0pt}i{\isacharparenright}{\kern0pt}\ {\isasymunion}\ succ{\isacharparenleft}{\kern0pt}j{\isacharparenright}{\kern0pt}{\isachardoublequoteclose}\ \isanewline
%
\isadelimproof
\ \ %
\endisadelimproof
%
\isatagproof
\isacommand{unfolding}\isamarkupfalse%
\ is{\isacharunderscore}{\kern0pt}HS{\isacharunderscore}{\kern0pt}fm{\isacharunderscore}{\kern0pt}def\isanewline
\ \ \isacommand{apply}\isamarkupfalse%
\ simp\isanewline
\ \ \isacommand{apply}\isamarkupfalse%
{\isacharparenleft}{\kern0pt}subgoal{\isacharunderscore}{\kern0pt}tac\ {\isachardoublequoteopen}is{\isacharunderscore}{\kern0pt}memrel{\isacharunderscore}{\kern0pt}wftrec{\isacharunderscore}{\kern0pt}fm{\isacharparenleft}{\kern0pt}His{\isacharunderscore}{\kern0pt}HS{\isacharunderscore}{\kern0pt}M{\isacharunderscore}{\kern0pt}fm{\isacharparenleft}{\kern0pt}{\isadigit{2}}{\isacharcomma}{\kern0pt}\ {\isadigit{1}}{\isacharcomma}{\kern0pt}\ {\isadigit{0}}{\isacharparenright}{\kern0pt}{\isacharcomma}{\kern0pt}\ succ{\isacharparenleft}{\kern0pt}natify{\isacharparenleft}{\kern0pt}j{\isacharparenright}{\kern0pt}{\isacharparenright}{\kern0pt}{\isacharcomma}{\kern0pt}\ succ{\isacharparenleft}{\kern0pt}natify{\isacharparenleft}{\kern0pt}i{\isacharparenright}{\kern0pt}{\isacharparenright}{\kern0pt}{\isacharcomma}{\kern0pt}\ {\isadigit{0}}{\isacharparenright}{\kern0pt}\ {\isasymin}\ formula{\isachardoublequoteclose}{\isacharparenright}{\kern0pt}\isanewline
\ \ \isacommand{apply}\isamarkupfalse%
{\isacharparenleft}{\kern0pt}subgoal{\isacharunderscore}{\kern0pt}tac\ {\isachardoublequoteopen}is{\isacharunderscore}{\kern0pt}{\isadigit{1}}{\isacharunderscore}{\kern0pt}fm{\isacharparenleft}{\kern0pt}{\isadigit{0}}{\isacharparenright}{\kern0pt}\ {\isasymin}\ formula{\isachardoublequoteclose}{\isacharparenright}{\kern0pt}\ \isanewline
\ \ \ \ \isacommand{apply}\isamarkupfalse%
{\isacharparenleft}{\kern0pt}rule\ pred{\isacharunderscore}{\kern0pt}le{\isacharcomma}{\kern0pt}\ simp\ add{\isacharcolon}{\kern0pt}assms{\isacharcomma}{\kern0pt}\ rule\ union{\isacharunderscore}{\kern0pt}in{\isacharunderscore}{\kern0pt}nat{\isacharparenright}{\kern0pt}\isanewline
\ \ \isacommand{using}\isamarkupfalse%
\ assms\ \isanewline
\ \ \ \ \ \ \isacommand{apply}\isamarkupfalse%
\ auto{\isacharbrackleft}{\kern0pt}{\isadigit{2}}{\isacharbrackright}{\kern0pt}\isanewline
\ \ \ \ \isacommand{apply}\isamarkupfalse%
{\isacharparenleft}{\kern0pt}rule\ Un{\isacharunderscore}{\kern0pt}least{\isacharunderscore}{\kern0pt}lt{\isacharparenright}{\kern0pt}\isanewline
\ \ \isacommand{using}\isamarkupfalse%
\ assms\ \isanewline
\ \ \ \ \ \isacommand{apply}\isamarkupfalse%
\ simp\isanewline
\ \ \ \ \ \isacommand{apply}\isamarkupfalse%
{\isacharparenleft}{\kern0pt}subst\ succ{\isacharunderscore}{\kern0pt}Un{\isacharunderscore}{\kern0pt}distrib{\isacharcomma}{\kern0pt}\ simp\ add{\isacharcolon}{\kern0pt}assms{\isacharcomma}{\kern0pt}\ simp\ add{\isacharcolon}{\kern0pt}assms{\isacharparenright}{\kern0pt}\isanewline
\ \ \ \ \ \isacommand{apply}\isamarkupfalse%
{\isacharparenleft}{\kern0pt}rule\ le{\isacharunderscore}{\kern0pt}trans{\isacharcomma}{\kern0pt}\ rule\ arity{\isacharunderscore}{\kern0pt}is{\isacharunderscore}{\kern0pt}memrel{\isacharunderscore}{\kern0pt}wftrec{\isacharunderscore}{\kern0pt}fm{\isacharparenright}{\kern0pt}\isanewline
\ \ \ \ \ \ \ \ \ \ \isacommand{apply}\isamarkupfalse%
{\isacharparenleft}{\kern0pt}rule\ His{\isacharunderscore}{\kern0pt}HS{\isacharunderscore}{\kern0pt}M{\isacharunderscore}{\kern0pt}fm{\isacharunderscore}{\kern0pt}type{\isacharparenright}{\kern0pt}\isanewline
\ \ \ \ \ \ \ \ \ \ \ \ \isacommand{apply}\isamarkupfalse%
\ auto{\isacharbrackleft}{\kern0pt}{\isadigit{3}}{\isacharbrackright}{\kern0pt}\isanewline
\ \ \ \ \ \ \ \ \ \isacommand{apply}\isamarkupfalse%
{\isacharparenleft}{\kern0pt}rule\ le{\isacharunderscore}{\kern0pt}trans{\isacharcomma}{\kern0pt}\ rule\ arity{\isacharunderscore}{\kern0pt}His{\isacharunderscore}{\kern0pt}HS{\isacharunderscore}{\kern0pt}M{\isacharunderscore}{\kern0pt}fm{\isacharparenright}{\kern0pt}\isanewline
\ \ \ \ \ \ \ \ \ \ \ \ \isacommand{apply}\isamarkupfalse%
\ auto{\isacharbrackleft}{\kern0pt}{\isadigit{3}}{\isacharbrackright}{\kern0pt}\isanewline
\ \ \ \ \ \ \ \ \ \isacommand{apply}\isamarkupfalse%
{\isacharparenleft}{\kern0pt}rule\ Un{\isacharunderscore}{\kern0pt}least{\isacharunderscore}{\kern0pt}lt{\isacharparenright}{\kern0pt}{\isacharplus}{\kern0pt}\isanewline
\ \ \ \ \ \ \ \ \ \ \ \isacommand{apply}\isamarkupfalse%
\ auto{\isacharbrackleft}{\kern0pt}{\isadigit{6}}{\isacharbrackright}{\kern0pt}\isanewline
\ \ \ \ \ \isacommand{apply}\isamarkupfalse%
{\isacharparenleft}{\kern0pt}rule\ Un{\isacharunderscore}{\kern0pt}least{\isacharunderscore}{\kern0pt}lt{\isacharparenright}{\kern0pt}{\isacharplus}{\kern0pt}\isanewline
\ \ \ \ \ \ \ \isacommand{apply}\isamarkupfalse%
{\isacharparenleft}{\kern0pt}simp{\isacharcomma}{\kern0pt}\ rule\ ltI{\isacharcomma}{\kern0pt}\ simp{\isacharcomma}{\kern0pt}\ simp{\isacharcomma}{\kern0pt}\ simp{\isacharcomma}{\kern0pt}\ rule\ ltI{\isacharcomma}{\kern0pt}\ simp{\isacharcomma}{\kern0pt}\ simp{\isacharcomma}{\kern0pt}\ simp{\isacharparenright}{\kern0pt}\isanewline
\ \ \ \ \ \isacommand{apply}\isamarkupfalse%
{\isacharparenleft}{\kern0pt}rule\ ltI{\isacharcomma}{\kern0pt}\ simp{\isacharcomma}{\kern0pt}\ rule\ disjI{\isadigit{1}}{\isacharcomma}{\kern0pt}\ rule\ ltD{\isacharcomma}{\kern0pt}\ simp{\isacharcomma}{\kern0pt}\ simp{\isacharparenright}{\kern0pt}\isanewline
\ \ \ \ \isacommand{apply}\isamarkupfalse%
{\isacharparenleft}{\kern0pt}subst\ arity{\isacharunderscore}{\kern0pt}is{\isacharunderscore}{\kern0pt}{\isadigit{1}}{\isacharunderscore}{\kern0pt}fm{\isacharcomma}{\kern0pt}\ simp{\isacharcomma}{\kern0pt}\ simp{\isacharparenright}{\kern0pt}\isanewline
\ \ \ \ \isacommand{apply}\isamarkupfalse%
{\isacharparenleft}{\kern0pt}rule{\isacharunderscore}{\kern0pt}tac\ j{\isacharequal}{\kern0pt}{\isachardoublequoteopen}succ{\isacharparenleft}{\kern0pt}i{\isacharparenright}{\kern0pt}{\isachardoublequoteclose}\ \isakeyword{in}\ le{\isacharunderscore}{\kern0pt}trans{\isacharcomma}{\kern0pt}\ simp{\isacharcomma}{\kern0pt}\ simp\ add{\isacharcolon}{\kern0pt}assms{\isacharcomma}{\kern0pt}\ simp{\isacharcomma}{\kern0pt}\ rule\ ltI{\isacharcomma}{\kern0pt}\ simp{\isacharcomma}{\kern0pt}\ simp\ add{\isacharcolon}{\kern0pt}assms{\isacharparenright}{\kern0pt}\isanewline
\ \ \ \isacommand{apply}\isamarkupfalse%
{\isacharparenleft}{\kern0pt}rule{\isacharunderscore}{\kern0pt}tac\ i{\isacharequal}{\kern0pt}{\isadigit{0}}\ \isakeyword{in}\ is{\isacharunderscore}{\kern0pt}{\isadigit{1}}{\isacharunderscore}{\kern0pt}fm{\isacharunderscore}{\kern0pt}type{\isacharcomma}{\kern0pt}\ simp{\isacharparenright}{\kern0pt}\isanewline
\ \ \isacommand{apply}\isamarkupfalse%
{\isacharparenleft}{\kern0pt}rule\ is{\isacharunderscore}{\kern0pt}memrel{\isacharunderscore}{\kern0pt}wftrec{\isacharunderscore}{\kern0pt}fm{\isacharunderscore}{\kern0pt}type{\isacharparenright}{\kern0pt}\isanewline
\ \ \ \ \ \isacommand{apply}\isamarkupfalse%
{\isacharparenleft}{\kern0pt}rule\ His{\isacharunderscore}{\kern0pt}HS{\isacharunderscore}{\kern0pt}M{\isacharunderscore}{\kern0pt}fm{\isacharunderscore}{\kern0pt}type{\isacharparenright}{\kern0pt}\isanewline
\ \ \isacommand{using}\isamarkupfalse%
\ assms\ \isanewline
\ \ \isacommand{by}\isamarkupfalse%
\ auto%
\endisatagproof
{\isafoldproof}%
%
\isadelimproof
\isanewline
%
\endisadelimproof
\isanewline
\isacommand{definition}\isamarkupfalse%
\ His{\isacharunderscore}{\kern0pt}HS\ \isakeyword{where}\ \isanewline
\ \ {\isachardoublequoteopen}His{\isacharunderscore}{\kern0pt}HS{\isacharparenleft}{\kern0pt}x{\isacharcomma}{\kern0pt}\ g{\isacharparenright}{\kern0pt}\ {\isasymequiv}\ if\ {\isacharparenleft}{\kern0pt}x\ {\isasymin}\ P{\isacharunderscore}{\kern0pt}names\ {\isasymand}\ {\isacharparenleft}{\kern0pt}{\isasymforall}y\ {\isasymin}\ domain{\isacharparenleft}{\kern0pt}x{\isacharparenright}{\kern0pt}{\isachardot}{\kern0pt}\ g{\isacharbackquote}{\kern0pt}y\ {\isacharequal}{\kern0pt}\ {\isadigit{1}}{\isacharparenright}{\kern0pt}\ {\isasymand}\ sym{\isacharparenleft}{\kern0pt}x{\isacharparenright}{\kern0pt}\ {\isasymin}\ {\isasymF}{\isacharparenright}{\kern0pt}\ then\ {\isadigit{1}}\ else\ {\isadigit{0}}{\isachardoublequoteclose}\ \isanewline
\isanewline
\isacommand{lemma}\isamarkupfalse%
\ His{\isacharunderscore}{\kern0pt}HS{\isacharunderscore}{\kern0pt}eq\ {\isacharcolon}{\kern0pt}\ \isanewline
\ \ \ \ {\isachardoublequoteopen}{\isasymAnd}h\ g\ u{\isachardot}{\kern0pt}\isanewline
\ \ \ \ \ u\ {\isasymin}\ M\ {\isasymLongrightarrow}\isanewline
\ \ \ \ \ h\ {\isasymin}\ eclose{\isacharparenleft}{\kern0pt}u{\isacharparenright}{\kern0pt}\ {\isasymrightarrow}\ M\ {\isasymLongrightarrow}\isanewline
\ \ \ \ \ g\ {\isasymin}\ eclose{\isacharparenleft}{\kern0pt}u{\isacharparenright}{\kern0pt}\ {\isasymtimes}\ {\isacharbraceleft}{\kern0pt}{\isacharless}{\kern0pt}{\isasymF}{\isacharcomma}{\kern0pt}\ {\isasymG}{\isacharcomma}{\kern0pt}\ P{\isacharcomma}{\kern0pt}\ P{\isacharunderscore}{\kern0pt}auto{\isachargreater}{\kern0pt}{\isacharbraceright}{\kern0pt}\ {\isasymrightarrow}\ M\ {\isasymLongrightarrow}\isanewline
\ \ \ \ \ g\ {\isasymin}\ M\ {\isasymLongrightarrow}\ {\isacharparenleft}{\kern0pt}{\isasymAnd}y{\isachardot}{\kern0pt}\ y\ {\isasymin}\ eclose{\isacharparenleft}{\kern0pt}u{\isacharparenright}{\kern0pt}\ {\isasymLongrightarrow}\ h\ {\isacharbackquote}{\kern0pt}\ y\ {\isacharequal}{\kern0pt}\ g\ {\isacharbackquote}{\kern0pt}\ {\isasymlangle}y{\isacharcomma}{\kern0pt}\ {\isasymF}{\isacharcomma}{\kern0pt}\ {\isasymG}{\isacharcomma}{\kern0pt}\ P{\isacharcomma}{\kern0pt}\ P{\isacharunderscore}{\kern0pt}auto{\isasymrangle}{\isacharparenright}{\kern0pt}\ {\isasymLongrightarrow}\ His{\isacharunderscore}{\kern0pt}HS{\isacharparenleft}{\kern0pt}u{\isacharcomma}{\kern0pt}\ h{\isacharparenright}{\kern0pt}\ {\isacharequal}{\kern0pt}\ His{\isacharunderscore}{\kern0pt}HS{\isacharunderscore}{\kern0pt}M{\isacharparenleft}{\kern0pt}{\isasymlangle}u{\isacharcomma}{\kern0pt}\ {\isasymF}{\isacharcomma}{\kern0pt}\ {\isasymG}{\isacharcomma}{\kern0pt}\ P{\isacharcomma}{\kern0pt}\ P{\isacharunderscore}{\kern0pt}auto{\isasymrangle}{\isacharcomma}{\kern0pt}\ g{\isacharparenright}{\kern0pt}{\isachardoublequoteclose}\ \isanewline
%
\isadelimproof
%
\endisadelimproof
%
\isatagproof
\isacommand{proof}\isamarkupfalse%
\ {\isacharminus}{\kern0pt}\ \isanewline
\ \ \isacommand{fix}\isamarkupfalse%
\ h\ g\ u\isanewline
\ \ \isacommand{assume}\isamarkupfalse%
\ assms{\isacharcolon}{\kern0pt}\ {\isachardoublequoteopen}u\ {\isasymin}\ M{\isachardoublequoteclose}\ {\isachardoublequoteopen}h\ {\isasymin}\ eclose{\isacharparenleft}{\kern0pt}u{\isacharparenright}{\kern0pt}\ {\isasymrightarrow}\ M{\isachardoublequoteclose}\ {\isachardoublequoteopen}g\ {\isasymin}\ eclose{\isacharparenleft}{\kern0pt}u{\isacharparenright}{\kern0pt}\ {\isasymtimes}\ {\isacharbraceleft}{\kern0pt}{\isacharless}{\kern0pt}{\isasymF}{\isacharcomma}{\kern0pt}\ {\isasymG}{\isacharcomma}{\kern0pt}\ P{\isacharcomma}{\kern0pt}\ P{\isacharunderscore}{\kern0pt}auto{\isachargreater}{\kern0pt}{\isacharbraceright}{\kern0pt}\ {\isasymrightarrow}\ M{\isachardoublequoteclose}\ {\isachardoublequoteopen}g\ {\isasymin}\ M{\isachardoublequoteclose}\ \isanewline
\ \ \ \ \ \ \ \ \ \ \ \ \ \ \ \ \ {\isachardoublequoteopen}{\isacharparenleft}{\kern0pt}{\isasymAnd}y{\isachardot}{\kern0pt}\ y\ {\isasymin}\ eclose{\isacharparenleft}{\kern0pt}u{\isacharparenright}{\kern0pt}\ {\isasymLongrightarrow}\ h\ {\isacharbackquote}{\kern0pt}\ y\ {\isacharequal}{\kern0pt}\ g\ {\isacharbackquote}{\kern0pt}\ {\isasymlangle}y{\isacharcomma}{\kern0pt}\ {\isasymF}{\isacharcomma}{\kern0pt}\ {\isasymG}{\isacharcomma}{\kern0pt}\ P{\isacharcomma}{\kern0pt}\ P{\isacharunderscore}{\kern0pt}auto{\isasymrangle}{\isacharparenright}{\kern0pt}{\isachardoublequoteclose}\ \isanewline
\ \isanewline
\ \ \isacommand{have}\isamarkupfalse%
\ {\isachardoublequoteopen}M{\isacharcomma}{\kern0pt}\ {\isacharbrackleft}{\kern0pt}{\isasymlangle}u{\isacharcomma}{\kern0pt}\ {\isasymF}{\isacharcomma}{\kern0pt}\ {\isasymG}{\isacharcomma}{\kern0pt}\ P{\isacharcomma}{\kern0pt}\ P{\isacharunderscore}{\kern0pt}auto{\isasymrangle}{\isacharcomma}{\kern0pt}\ g{\isacharbrackright}{\kern0pt}\ {\isasymTurnstile}\ His{\isacharunderscore}{\kern0pt}HS{\isacharunderscore}{\kern0pt}M{\isacharunderscore}{\kern0pt}cond{\isacharunderscore}{\kern0pt}fm{\isacharparenleft}{\kern0pt}{\isadigit{0}}{\isacharcomma}{\kern0pt}\ {\isadigit{1}}{\isacharparenright}{\kern0pt}\ {\isasymlongleftrightarrow}\ \isanewline
\ \ \ \ \ \ \ \ {\isacharparenleft}{\kern0pt}{\isasymexists}symx\ {\isasymin}\ M{\isachardot}{\kern0pt}\ {\isasymexists}domx\ {\isasymin}\ M{\isachardot}{\kern0pt}\ {\isasymexists}pauto\ {\isasymin}\ M{\isachardot}{\kern0pt}\ {\isasymexists}p\ {\isasymin}\ M{\isachardot}{\kern0pt}\ {\isasymexists}ppauto\ {\isasymin}\ M{\isachardot}{\kern0pt}\ \ {\isasymexists}G\ {\isasymin}\ M{\isachardot}{\kern0pt}\ \ {\isasymexists}Gppauto\ {\isasymin}\ M{\isachardot}{\kern0pt}\ {\isasymexists}F\ {\isasymin}\ M{\isachardot}{\kern0pt}\ {\isasymexists}FGppauto\ {\isasymin}\ M{\isachardot}{\kern0pt}\ {\isasymexists}x\ {\isasymin}\ M{\isachardot}{\kern0pt}\isanewline
\ \ \ \ \ \ \ \ \ \ {\isasymlangle}u{\isacharcomma}{\kern0pt}\ {\isasymF}{\isacharcomma}{\kern0pt}\ {\isasymG}{\isacharcomma}{\kern0pt}\ P{\isacharcomma}{\kern0pt}\ P{\isacharunderscore}{\kern0pt}auto{\isasymrangle}\ {\isacharequal}{\kern0pt}\ {\isacharless}{\kern0pt}x{\isacharcomma}{\kern0pt}\ FGppauto{\isachargreater}{\kern0pt}\ {\isasymand}\ FGppauto\ {\isacharequal}{\kern0pt}\ {\isacharless}{\kern0pt}F{\isacharcomma}{\kern0pt}\ Gppauto{\isachargreater}{\kern0pt}\ {\isasymand}\ Gppauto\ {\isacharequal}{\kern0pt}\ {\isacharless}{\kern0pt}G{\isacharcomma}{\kern0pt}\ ppauto{\isachargreater}{\kern0pt}\ {\isasymand}\ ppauto\ {\isacharequal}{\kern0pt}\ {\isacharless}{\kern0pt}p{\isacharcomma}{\kern0pt}\ pauto{\isachargreater}{\kern0pt}\ {\isasymand}\ \isanewline
\ \ \ \ \ \ \ \ \ \ domx\ {\isacharequal}{\kern0pt}\ domain{\isacharparenleft}{\kern0pt}x{\isacharparenright}{\kern0pt}\ {\isasymand}\ x\ {\isasymin}\ P{\isacharunderscore}{\kern0pt}names\ {\isasymand}\ {\isacharparenleft}{\kern0pt}{\isasymforall}y{\isachardot}{\kern0pt}\ y\ {\isasymin}\ domx\ {\isasymlongrightarrow}\ g{\isacharbackquote}{\kern0pt}{\isasymlangle}y{\isacharcomma}{\kern0pt}\ {\isasymF}{\isacharcomma}{\kern0pt}\ {\isasymG}{\isacharcomma}{\kern0pt}\ P{\isacharcomma}{\kern0pt}\ P{\isacharunderscore}{\kern0pt}auto{\isasymrangle}\ {\isacharequal}{\kern0pt}\ {\isadigit{1}}{\isacharparenright}{\kern0pt}\ {\isasymand}\ symx\ {\isacharequal}{\kern0pt}\ sym{\isacharparenleft}{\kern0pt}x{\isacharparenright}{\kern0pt}\ {\isasymand}\ symx\ {\isasymin}\ {\isasymF}{\isacharparenright}{\kern0pt}{\isachardoublequoteclose}\ \ \ \isanewline
\ \ \ \ \isacommand{unfolding}\isamarkupfalse%
\ His{\isacharunderscore}{\kern0pt}HS{\isacharunderscore}{\kern0pt}M{\isacharunderscore}{\kern0pt}cond{\isacharunderscore}{\kern0pt}fm{\isacharunderscore}{\kern0pt}def\ \isanewline
\ \ \ \ \isacommand{apply}\isamarkupfalse%
{\isacharparenleft}{\kern0pt}subgoal{\isacharunderscore}{\kern0pt}tac\ {\isachardoublequoteopen}{\isasymlangle}u{\isacharcomma}{\kern0pt}\ {\isasymF}{\isacharcomma}{\kern0pt}\ {\isasymG}{\isacharcomma}{\kern0pt}\ P{\isacharcomma}{\kern0pt}\ P{\isacharunderscore}{\kern0pt}auto{\isasymrangle}\ {\isasymin}\ M{\isachardoublequoteclose}{\isacharparenright}{\kern0pt}\isanewline
\ \ \ \ \ \isacommand{apply}\isamarkupfalse%
{\isacharparenleft}{\kern0pt}rule{\isacharunderscore}{\kern0pt}tac\ iff{\isacharunderscore}{\kern0pt}trans{\isacharcomma}{\kern0pt}\ rule{\isacharunderscore}{\kern0pt}tac\ sats{\isacharunderscore}{\kern0pt}Exists{\isacharunderscore}{\kern0pt}iff{\isacharcomma}{\kern0pt}\ simp\ add{\isacharcolon}{\kern0pt}assms{\isacharcomma}{\kern0pt}\ rule{\isacharunderscore}{\kern0pt}tac\ bex{\isacharunderscore}{\kern0pt}iff{\isacharparenright}{\kern0pt}{\isacharplus}{\kern0pt}\isanewline
\ \ \ \ \ \isacommand{apply}\isamarkupfalse%
{\isacharparenleft}{\kern0pt}rule\ iff{\isacharunderscore}{\kern0pt}trans{\isacharcomma}{\kern0pt}\ rule\ sats{\isacharunderscore}{\kern0pt}And{\isacharunderscore}{\kern0pt}iff{\isacharcomma}{\kern0pt}\ simp\ add{\isacharcolon}{\kern0pt}assms{\isacharcomma}{\kern0pt}\ rule\ iff{\isacharunderscore}{\kern0pt}conjI{\isadigit{2}}{\isacharcomma}{\kern0pt}\ simp\ add{\isacharcolon}{\kern0pt}assms{\isacharparenright}{\kern0pt}{\isacharplus}{\kern0pt}\isanewline
\ \ \ \ \ \ \isacommand{apply}\isamarkupfalse%
{\isacharparenleft}{\kern0pt}rename{\isacharunderscore}{\kern0pt}tac\ symx\ domx\ pauto\ p\ ppauto\ G\ Gppauto\ F\ FGppauto\ x{\isacharparenright}{\kern0pt}\isanewline
\ \ \ \ \ \ \isacommand{apply}\isamarkupfalse%
{\isacharparenleft}{\kern0pt}rule{\isacharunderscore}{\kern0pt}tac\ b{\isacharequal}{\kern0pt}{\isachardoublequoteopen}forcing{\isacharunderscore}{\kern0pt}data{\isachardot}{\kern0pt}P{\isacharunderscore}{\kern0pt}names{\isacharparenleft}{\kern0pt}p{\isacharcomma}{\kern0pt}\ M{\isacharparenright}{\kern0pt}{\isachardoublequoteclose}\ \isakeyword{and}\ a{\isacharequal}{\kern0pt}P{\isacharunderscore}{\kern0pt}names\ \isakeyword{in}\ ssubst{\isacharcomma}{\kern0pt}\ force{\isacharparenright}{\kern0pt}\isanewline
\ \ \ \ \ \ \isacommand{apply}\isamarkupfalse%
{\isacharparenleft}{\kern0pt}rule\ sats{\isacharunderscore}{\kern0pt}is{\isacharunderscore}{\kern0pt}P{\isacharunderscore}{\kern0pt}name{\isacharunderscore}{\kern0pt}fm{\isacharunderscore}{\kern0pt}iff{\isacharparenright}{\kern0pt}\isanewline
\ \ \ \ \isacommand{using}\isamarkupfalse%
\ assms\isanewline
\ \ \ \ \ \ \ \ \ \ \isacommand{apply}\isamarkupfalse%
\ auto{\isacharbrackleft}{\kern0pt}{\isadigit{5}}{\isacharbrackright}{\kern0pt}\isanewline
\ \ \ \ \ \isacommand{apply}\isamarkupfalse%
{\isacharparenleft}{\kern0pt}rule\ iff{\isacharunderscore}{\kern0pt}trans{\isacharcomma}{\kern0pt}\ rule\ sats{\isacharunderscore}{\kern0pt}And{\isacharunderscore}{\kern0pt}iff{\isacharcomma}{\kern0pt}\ simp\ add{\isacharcolon}{\kern0pt}assms{\isacharcomma}{\kern0pt}\ rule\ iff{\isacharunderscore}{\kern0pt}conjI{\isadigit{2}}{\isacharparenright}{\kern0pt}\isanewline
\ \ \ \ \ \ \isacommand{apply}\isamarkupfalse%
{\isacharparenleft}{\kern0pt}rule\ sats{\isacharunderscore}{\kern0pt}apply{\isacharunderscore}{\kern0pt}all{\isacharunderscore}{\kern0pt}{\isadigit{1}}{\isacharunderscore}{\kern0pt}fm{\isacharunderscore}{\kern0pt}iff{\isacharparenright}{\kern0pt}\isanewline
\ \ \ \ \isacommand{using}\isamarkupfalse%
\ assms\ \isanewline
\ \ \ \ \ \ \ \ \ \ \ \ \isacommand{apply}\isamarkupfalse%
\ auto{\isacharbrackleft}{\kern0pt}{\isadigit{7}}{\isacharbrackright}{\kern0pt}\isanewline
\ \ \ \ \ \isacommand{apply}\isamarkupfalse%
{\isacharparenleft}{\kern0pt}rule\ iff{\isacharunderscore}{\kern0pt}trans{\isacharcomma}{\kern0pt}\ rule\ sats{\isacharunderscore}{\kern0pt}And{\isacharunderscore}{\kern0pt}iff{\isacharcomma}{\kern0pt}\ simp\ add{\isacharcolon}{\kern0pt}assms{\isacharcomma}{\kern0pt}\ rule\ iff{\isacharunderscore}{\kern0pt}conjI{\isadigit{2}}{\isacharparenright}{\kern0pt}\isanewline
\ \ \ \ \ \ \isacommand{apply}\isamarkupfalse%
{\isacharparenleft}{\kern0pt}rule\ sats{\isacharunderscore}{\kern0pt}is{\isacharunderscore}{\kern0pt}sym{\isacharunderscore}{\kern0pt}fm{\isacharunderscore}{\kern0pt}iff{\isacharcomma}{\kern0pt}\ simp\ add{\isacharcolon}{\kern0pt}assms{\isacharparenright}{\kern0pt}\isanewline
\ \ \ \ \isacommand{using}\isamarkupfalse%
\ assms\isanewline
\ \ \ \ \ \ \ \ \ \ \ \ \ \ \isacommand{apply}\isamarkupfalse%
\ auto{\isacharbrackleft}{\kern0pt}{\isadigit{1}}{\isadigit{0}}{\isacharbrackright}{\kern0pt}\isanewline
\ \ \ \ \isacommand{using}\isamarkupfalse%
\ assms\ {\isasymF}{\isacharunderscore}{\kern0pt}in{\isacharunderscore}{\kern0pt}M\ {\isasymG}{\isacharunderscore}{\kern0pt}in{\isacharunderscore}{\kern0pt}M\ P{\isacharunderscore}{\kern0pt}in{\isacharunderscore}{\kern0pt}M\ P{\isacharunderscore}{\kern0pt}auto{\isacharunderscore}{\kern0pt}in{\isacharunderscore}{\kern0pt}M\ pair{\isacharunderscore}{\kern0pt}in{\isacharunderscore}{\kern0pt}M{\isacharunderscore}{\kern0pt}iff\ \isanewline
\ \ \ \ \isacommand{by}\isamarkupfalse%
\ auto\ \isanewline
\ \ \isacommand{also}\isamarkupfalse%
\ \isacommand{have}\isamarkupfalse%
\ {\isachardoublequoteopen}{\isachardot}{\kern0pt}{\isachardot}{\kern0pt}{\isachardot}{\kern0pt}\ {\isasymlongleftrightarrow}\ u\ {\isasymin}\ P{\isacharunderscore}{\kern0pt}names\ {\isasymand}\ {\isacharparenleft}{\kern0pt}{\isasymforall}y\ {\isasymin}\ domain{\isacharparenleft}{\kern0pt}u{\isacharparenright}{\kern0pt}{\isachardot}{\kern0pt}\ g{\isacharbackquote}{\kern0pt}{\isasymlangle}y{\isacharcomma}{\kern0pt}\ {\isasymF}{\isacharcomma}{\kern0pt}\ {\isasymG}{\isacharcomma}{\kern0pt}\ P{\isacharcomma}{\kern0pt}\ P{\isacharunderscore}{\kern0pt}auto{\isasymrangle}\ {\isacharequal}{\kern0pt}\ {\isadigit{1}}{\isacharparenright}{\kern0pt}\ {\isasymand}\ sym{\isacharparenleft}{\kern0pt}u{\isacharparenright}{\kern0pt}\ {\isasymin}\ {\isasymF}{\isachardoublequoteclose}\ \isanewline
\ \ \ \ \isacommand{apply}\isamarkupfalse%
{\isacharparenleft}{\kern0pt}rule\ iffI{\isacharcomma}{\kern0pt}\ clarify{\isacharparenright}{\kern0pt}\isanewline
\ \ \ \ \isacommand{using}\isamarkupfalse%
\ pair{\isacharunderscore}{\kern0pt}in{\isacharunderscore}{\kern0pt}M{\isacharunderscore}{\kern0pt}iff\ {\isasymF}{\isacharunderscore}{\kern0pt}in{\isacharunderscore}{\kern0pt}M\ {\isasymG}{\isacharunderscore}{\kern0pt}in{\isacharunderscore}{\kern0pt}M\ P{\isacharunderscore}{\kern0pt}in{\isacharunderscore}{\kern0pt}M\ P{\isacharunderscore}{\kern0pt}auto{\isacharunderscore}{\kern0pt}in{\isacharunderscore}{\kern0pt}M\ assms\ domain{\isacharunderscore}{\kern0pt}closed\ transM\ \isanewline
\ \ \ \ \isacommand{by}\isamarkupfalse%
\ auto\isanewline
\ \ \isacommand{also}\isamarkupfalse%
\ \isacommand{have}\isamarkupfalse%
\ {\isachardoublequoteopen}{\isachardot}{\kern0pt}{\isachardot}{\kern0pt}{\isachardot}{\kern0pt}\ {\isasymlongleftrightarrow}\ u\ {\isasymin}\ P{\isacharunderscore}{\kern0pt}names\ {\isasymand}\ {\isacharparenleft}{\kern0pt}{\isasymforall}y\ {\isasymin}\ domain{\isacharparenleft}{\kern0pt}u{\isacharparenright}{\kern0pt}{\isachardot}{\kern0pt}\ h{\isacharbackquote}{\kern0pt}y\ {\isacharequal}{\kern0pt}\ {\isadigit{1}}{\isacharparenright}{\kern0pt}\ {\isasymand}\ sym{\isacharparenleft}{\kern0pt}u{\isacharparenright}{\kern0pt}\ {\isasymin}\ {\isasymF}{\isachardoublequoteclose}\ \isanewline
\ \ \ \ \isacommand{apply}\isamarkupfalse%
{\isacharparenleft}{\kern0pt}rule\ iff{\isacharunderscore}{\kern0pt}conjI{\isacharcomma}{\kern0pt}\ simp{\isacharcomma}{\kern0pt}\ rule\ iff{\isacharunderscore}{\kern0pt}conjI{\isacharcomma}{\kern0pt}\ rule\ ball{\isacharunderscore}{\kern0pt}iff{\isacharparenright}{\kern0pt}\isanewline
\ \ \ \ \isacommand{using}\isamarkupfalse%
\ domain{\isacharunderscore}{\kern0pt}elem{\isacharunderscore}{\kern0pt}in{\isacharunderscore}{\kern0pt}eclose\ assms\ \isanewline
\ \ \ \ \ \isacommand{apply}\isamarkupfalse%
\ force\isanewline
\ \ \ \ \isacommand{by}\isamarkupfalse%
\ auto\ \isanewline
\ \ \isacommand{finally}\isamarkupfalse%
\ \isacommand{show}\isamarkupfalse%
\ {\isachardoublequoteopen}His{\isacharunderscore}{\kern0pt}HS{\isacharparenleft}{\kern0pt}u{\isacharcomma}{\kern0pt}\ h{\isacharparenright}{\kern0pt}\ {\isacharequal}{\kern0pt}\ His{\isacharunderscore}{\kern0pt}HS{\isacharunderscore}{\kern0pt}M{\isacharparenleft}{\kern0pt}{\isasymlangle}u{\isacharcomma}{\kern0pt}\ {\isasymF}{\isacharcomma}{\kern0pt}\ {\isasymG}{\isacharcomma}{\kern0pt}\ P{\isacharcomma}{\kern0pt}\ P{\isacharunderscore}{\kern0pt}auto{\isasymrangle}{\isacharcomma}{\kern0pt}\ g{\isacharparenright}{\kern0pt}{\isachardoublequoteclose}\ \isanewline
\ \ \ \ \isacommand{unfolding}\isamarkupfalse%
\ His{\isacharunderscore}{\kern0pt}HS{\isacharunderscore}{\kern0pt}def\ His{\isacharunderscore}{\kern0pt}HS{\isacharunderscore}{\kern0pt}M{\isacharunderscore}{\kern0pt}def\isanewline
\ \ \ \ \isacommand{apply}\isamarkupfalse%
{\isacharparenleft}{\kern0pt}rule{\isacharunderscore}{\kern0pt}tac\ if{\isacharunderscore}{\kern0pt}cong{\isacharparenright}{\kern0pt}\isanewline
\ \ \ \ \isacommand{by}\isamarkupfalse%
\ auto\ \isanewline
\isacommand{qed}\isamarkupfalse%
%
\endisatagproof
{\isafoldproof}%
%
\isadelimproof
\isanewline
%
\endisadelimproof
\isanewline
\isacommand{definition}\isamarkupfalse%
\ is{\isacharunderscore}{\kern0pt}HS\ \isakeyword{where}\ {\isachardoublequoteopen}is{\isacharunderscore}{\kern0pt}HS{\isacharparenleft}{\kern0pt}x{\isacharparenright}{\kern0pt}\ {\isasymequiv}\ wftrec{\isacharparenleft}{\kern0pt}Memrel{\isacharparenleft}{\kern0pt}M{\isacharparenright}{\kern0pt}{\isacharcircum}{\kern0pt}{\isacharplus}{\kern0pt}{\isacharcomma}{\kern0pt}\ x{\isacharcomma}{\kern0pt}\ His{\isacharunderscore}{\kern0pt}HS{\isacharparenright}{\kern0pt}{\isachardoublequoteclose}\ \isanewline
\isanewline
\isacommand{lemma}\isamarkupfalse%
\ def{\isacharunderscore}{\kern0pt}is{\isacharunderscore}{\kern0pt}HS\ {\isacharcolon}{\kern0pt}\ \isanewline
\ \ \isakeyword{fixes}\ x\ \isanewline
\ \ \isakeyword{assumes}\ {\isachardoublequoteopen}x\ {\isasymin}\ M{\isachardoublequoteclose}\ \isanewline
\ \ \isakeyword{shows}\ {\isachardoublequoteopen}is{\isacharunderscore}{\kern0pt}HS{\isacharparenleft}{\kern0pt}x{\isacharparenright}{\kern0pt}\ {\isacharequal}{\kern0pt}\ {\isadigit{1}}\ {\isasymlongleftrightarrow}\ x\ {\isasymin}\ HS{\isachardoublequoteclose}\ \isanewline
%
\isadelimproof
\isanewline
%
\endisadelimproof
%
\isatagproof
\isacommand{proof}\isamarkupfalse%
\ {\isacharminus}{\kern0pt}\ \isanewline
\ \ \isacommand{have}\isamarkupfalse%
\ {\isachardoublequoteopen}{\isasymAnd}u{\isachardot}{\kern0pt}\ u\ {\isasymin}\ M\ {\isasymlongrightarrow}\ is{\isacharunderscore}{\kern0pt}HS{\isacharparenleft}{\kern0pt}u{\isacharparenright}{\kern0pt}\ {\isacharequal}{\kern0pt}\ {\isadigit{1}}\ {\isasymlongleftrightarrow}\ u\ {\isasymin}\ HS{\isachardoublequoteclose}\ \isanewline
\ \ \isacommand{proof}\isamarkupfalse%
\ {\isacharparenleft}{\kern0pt}rule{\isacharunderscore}{\kern0pt}tac\ Q{\isacharequal}{\kern0pt}{\isachardoublequoteopen}{\isasymlambda}u{\isachardot}{\kern0pt}\ u\ {\isasymin}\ M\ {\isasymlongrightarrow}\ is{\isacharunderscore}{\kern0pt}HS{\isacharparenleft}{\kern0pt}u{\isacharparenright}{\kern0pt}\ {\isacharequal}{\kern0pt}\ {\isadigit{1}}\ {\isasymlongleftrightarrow}\ u\ {\isasymin}\ HS{\isachardoublequoteclose}\ \isakeyword{in}\ ed{\isacharunderscore}{\kern0pt}induction{\isacharcomma}{\kern0pt}\ rule\ impI{\isacharparenright}{\kern0pt}\isanewline
\ \ \ \ \isacommand{fix}\isamarkupfalse%
\ u\ \isanewline
\ \ \ \ \isacommand{assume}\isamarkupfalse%
\ assms{\isadigit{1}}{\isacharcolon}{\kern0pt}\ {\isachardoublequoteopen}{\isacharparenleft}{\kern0pt}{\isasymAnd}y{\isachardot}{\kern0pt}\ ed{\isacharparenleft}{\kern0pt}y{\isacharcomma}{\kern0pt}\ u{\isacharparenright}{\kern0pt}\ {\isasymLongrightarrow}\ y\ {\isasymin}\ M\ {\isasymlongrightarrow}\ is{\isacharunderscore}{\kern0pt}HS{\isacharparenleft}{\kern0pt}y{\isacharparenright}{\kern0pt}\ {\isacharequal}{\kern0pt}\ {\isadigit{1}}\ {\isasymlongleftrightarrow}\ y\ {\isasymin}\ HS{\isacharparenright}{\kern0pt}{\isachardoublequoteclose}\ {\isachardoublequoteopen}u\ {\isasymin}\ M{\isachardoublequoteclose}\isanewline
\isanewline
\ \ \ \ \isacommand{define}\isamarkupfalse%
\ F\ \isakeyword{where}\ {\isachardoublequoteopen}F\ {\isasymequiv}\ {\isasymlambda}y\ {\isasymin}\ Memrel{\isacharparenleft}{\kern0pt}M{\isacharparenright}{\kern0pt}{\isacharcircum}{\kern0pt}{\isacharplus}{\kern0pt}\ {\isacharminus}{\kern0pt}{\isacharbackquote}{\kern0pt}{\isacharbackquote}{\kern0pt}\ {\isacharbraceleft}{\kern0pt}u{\isacharbraceright}{\kern0pt}{\isachardot}{\kern0pt}\ wftrec{\isacharparenleft}{\kern0pt}Memrel{\isacharparenleft}{\kern0pt}M{\isacharparenright}{\kern0pt}{\isacharcircum}{\kern0pt}{\isacharplus}{\kern0pt}{\isacharcomma}{\kern0pt}\ y{\isacharcomma}{\kern0pt}\ His{\isacharunderscore}{\kern0pt}HS{\isacharparenright}{\kern0pt}{\isachardoublequoteclose}\ \isanewline
\isanewline
\ \ \ \ \isacommand{have}\isamarkupfalse%
\ iff{\isacharunderscore}{\kern0pt}lemma\ {\isacharcolon}{\kern0pt}\ {\isachardoublequoteopen}{\isasymAnd}a\ b\ c{\isachardot}{\kern0pt}\ a\ {\isacharequal}{\kern0pt}\ b\ {\isasymLongrightarrow}\ a\ {\isacharequal}{\kern0pt}\ c\ {\isasymlongleftrightarrow}\ b\ {\isacharequal}{\kern0pt}\ c{\isachardoublequoteclose}\ \isacommand{by}\isamarkupfalse%
\ auto\isanewline
\isanewline
\ \ \ \ \isacommand{have}\isamarkupfalse%
\ {\isachardoublequoteopen}is{\isacharunderscore}{\kern0pt}HS{\isacharparenleft}{\kern0pt}u{\isacharparenright}{\kern0pt}\ {\isacharequal}{\kern0pt}\ {\isadigit{1}}\ {\isasymlongleftrightarrow}\ His{\isacharunderscore}{\kern0pt}HS{\isacharparenleft}{\kern0pt}u{\isacharcomma}{\kern0pt}\ F{\isacharparenright}{\kern0pt}\ {\isacharequal}{\kern0pt}\ {\isadigit{1}}{\isachardoublequoteclose}\isanewline
\ \ \ \ \ \ \isacommand{unfolding}\isamarkupfalse%
\ is{\isacharunderscore}{\kern0pt}HS{\isacharunderscore}{\kern0pt}def\ F{\isacharunderscore}{\kern0pt}def\isanewline
\ \ \ \ \ \ \isacommand{apply}\isamarkupfalse%
{\isacharparenleft}{\kern0pt}rule\ iff{\isacharunderscore}{\kern0pt}lemma{\isacharcomma}{\kern0pt}\ rule\ wftrec{\isacharparenright}{\kern0pt}\isanewline
\ \ \ \ \ \ \ \isacommand{apply}\isamarkupfalse%
{\isacharparenleft}{\kern0pt}rule\ wf{\isacharunderscore}{\kern0pt}trancl{\isacharcomma}{\kern0pt}\ rule\ wf{\isacharunderscore}{\kern0pt}Memrel{\isacharcomma}{\kern0pt}\ rule\ trans{\isacharunderscore}{\kern0pt}trancl{\isacharparenright}{\kern0pt}\isanewline
\ \ \ \ \ \ \isacommand{done}\isamarkupfalse%
\ \isanewline
\ \ \ \ \isacommand{also}\isamarkupfalse%
\ \isacommand{have}\isamarkupfalse%
\ {\isachardoublequoteopen}{\isachardot}{\kern0pt}{\isachardot}{\kern0pt}{\isachardot}{\kern0pt}\ {\isasymlongleftrightarrow}\ u\ {\isasymin}\ P{\isacharunderscore}{\kern0pt}names\ {\isasymand}\ {\isacharparenleft}{\kern0pt}{\isasymforall}y\ {\isasymin}\ domain{\isacharparenleft}{\kern0pt}u{\isacharparenright}{\kern0pt}{\isachardot}{\kern0pt}\ F{\isacharbackquote}{\kern0pt}y\ {\isacharequal}{\kern0pt}\ {\isadigit{1}}{\isacharparenright}{\kern0pt}\ {\isasymand}\ sym{\isacharparenleft}{\kern0pt}u{\isacharparenright}{\kern0pt}\ {\isasymin}\ {\isasymF}{\isachardoublequoteclose}\ \isanewline
\ \ \ \ \ \ \isacommand{unfolding}\isamarkupfalse%
\ His{\isacharunderscore}{\kern0pt}HS{\isacharunderscore}{\kern0pt}def\ \isanewline
\ \ \ \ \ \ \isacommand{apply}\isamarkupfalse%
{\isacharparenleft}{\kern0pt}rule\ iffI{\isacharcomma}{\kern0pt}\ rule{\isacharunderscore}{\kern0pt}tac\ a{\isacharequal}{\kern0pt}{\isadigit{1}}\ \isakeyword{and}\ b{\isacharequal}{\kern0pt}{\isadigit{0}}\ \isakeyword{in}\ ifT{\isacharunderscore}{\kern0pt}eq{\isacharcomma}{\kern0pt}\ simp{\isacharcomma}{\kern0pt}\ simp{\isacharcomma}{\kern0pt}\ simp{\isacharparenright}{\kern0pt}\isanewline
\ \ \ \ \ \ \isacommand{done}\isamarkupfalse%
\ \isanewline
\ \ \ \ \isacommand{also}\isamarkupfalse%
\ \isacommand{have}\isamarkupfalse%
\ {\isachardoublequoteopen}{\isachardot}{\kern0pt}{\isachardot}{\kern0pt}{\isachardot}{\kern0pt}\ {\isasymlongleftrightarrow}\ u\ {\isasymin}\ P{\isacharunderscore}{\kern0pt}names\ {\isasymand}\ {\isacharparenleft}{\kern0pt}{\isasymforall}y\ {\isasymin}\ domain{\isacharparenleft}{\kern0pt}u{\isacharparenright}{\kern0pt}{\isachardot}{\kern0pt}\ is{\isacharunderscore}{\kern0pt}HS{\isacharparenleft}{\kern0pt}y{\isacharparenright}{\kern0pt}\ {\isacharequal}{\kern0pt}\ {\isadigit{1}}{\isacharparenright}{\kern0pt}\ {\isasymand}\ sym{\isacharparenleft}{\kern0pt}u{\isacharparenright}{\kern0pt}\ {\isasymin}\ {\isasymF}{\isachardoublequoteclose}\ \isanewline
\ \ \ \ \ \ \isacommand{apply}\isamarkupfalse%
{\isacharparenleft}{\kern0pt}rule\ iff{\isacharunderscore}{\kern0pt}conjI{\isacharcomma}{\kern0pt}\ simp{\isacharcomma}{\kern0pt}\ rule\ iff{\isacharunderscore}{\kern0pt}conjI{\isacharcomma}{\kern0pt}\ rule\ ball{\isacharunderscore}{\kern0pt}iff{\isacharparenright}{\kern0pt}\isanewline
\ \ \ \ \ \ \ \isacommand{apply}\isamarkupfalse%
{\isacharparenleft}{\kern0pt}rule\ iff{\isacharunderscore}{\kern0pt}lemma{\isacharparenright}{\kern0pt}\isanewline
\ \ \ \ \ \ \isacommand{unfolding}\isamarkupfalse%
\ F{\isacharunderscore}{\kern0pt}def\ is{\isacharunderscore}{\kern0pt}HS{\isacharunderscore}{\kern0pt}def\ \isanewline
\ \ \ \ \ \ \ \isacommand{apply}\isamarkupfalse%
{\isacharparenleft}{\kern0pt}rename{\isacharunderscore}{\kern0pt}tac\ y{\isacharcomma}{\kern0pt}\ subgoal{\isacharunderscore}{\kern0pt}tac\ {\isachardoublequoteopen}y\ {\isasymin}\ Memrel{\isacharparenleft}{\kern0pt}M{\isacharparenright}{\kern0pt}{\isacharcircum}{\kern0pt}{\isacharplus}{\kern0pt}\ {\isacharminus}{\kern0pt}{\isacharbackquote}{\kern0pt}{\isacharbackquote}{\kern0pt}\ {\isacharbraceleft}{\kern0pt}u{\isacharbraceright}{\kern0pt}{\isachardoublequoteclose}{\isacharcomma}{\kern0pt}\ simp{\isacharparenright}{\kern0pt}\isanewline
\ \ \ \ \ \ \ \isacommand{apply}\isamarkupfalse%
{\isacharparenleft}{\kern0pt}rule{\isacharunderscore}{\kern0pt}tac\ b{\isacharequal}{\kern0pt}u\ \isakeyword{in}\ vimageI{\isacharcomma}{\kern0pt}\ rule\ domain{\isacharunderscore}{\kern0pt}elem{\isacharunderscore}{\kern0pt}Memrel{\isacharunderscore}{\kern0pt}trancl{\isacharparenright}{\kern0pt}\isanewline
\ \ \ \ \ \ \isacommand{using}\isamarkupfalse%
\ assms{\isadigit{1}}\ \isanewline
\ \ \ \ \ \ \isacommand{by}\isamarkupfalse%
\ auto\ \isanewline
\ \ \ \ \isacommand{also}\isamarkupfalse%
\ \isacommand{have}\isamarkupfalse%
\ {\isachardoublequoteopen}{\isachardot}{\kern0pt}{\isachardot}{\kern0pt}{\isachardot}{\kern0pt}\ {\isasymlongleftrightarrow}\ u\ {\isasymin}\ P{\isacharunderscore}{\kern0pt}names\ {\isasymand}\ {\isacharparenleft}{\kern0pt}{\isasymforall}y\ {\isasymin}\ domain{\isacharparenleft}{\kern0pt}u{\isacharparenright}{\kern0pt}{\isachardot}{\kern0pt}\ y\ {\isasymin}\ HS{\isacharparenright}{\kern0pt}\ {\isasymand}\ sym{\isacharparenleft}{\kern0pt}u{\isacharparenright}{\kern0pt}\ {\isasymin}\ {\isasymF}{\isachardoublequoteclose}\ \isanewline
\ \ \ \ \ \ \isacommand{apply}\isamarkupfalse%
{\isacharparenleft}{\kern0pt}rule\ iff{\isacharunderscore}{\kern0pt}conjI{\isacharcomma}{\kern0pt}\ simp{\isacharcomma}{\kern0pt}\ rule\ iff{\isacharunderscore}{\kern0pt}conjI{\isacharcomma}{\kern0pt}\ rule\ ball{\isacharunderscore}{\kern0pt}iff{\isacharparenright}{\kern0pt}\isanewline
\ \ \ \ \ \ \isacommand{apply}\isamarkupfalse%
{\isacharparenleft}{\kern0pt}rename{\isacharunderscore}{\kern0pt}tac\ y{\isacharcomma}{\kern0pt}\ subgoal{\isacharunderscore}{\kern0pt}tac\ {\isachardoublequoteopen}y\ {\isasymin}\ M{\isachardoublequoteclose}{\isacharparenright}{\kern0pt}\isanewline
\ \ \ \ \ \ \isacommand{using}\isamarkupfalse%
\ assms{\isadigit{1}}\ \isacommand{unfolding}\isamarkupfalse%
\ ed{\isacharunderscore}{\kern0pt}def\ \ \isanewline
\ \ \ \ \ \ \ \ \isacommand{apply}\isamarkupfalse%
\ blast\isanewline
\ \ \ \ \ \ \isacommand{using}\isamarkupfalse%
\ domain{\isacharunderscore}{\kern0pt}elem{\isacharunderscore}{\kern0pt}in{\isacharunderscore}{\kern0pt}M\ assms{\isadigit{1}}\ \isanewline
\ \ \ \ \ \ \isacommand{by}\isamarkupfalse%
\ auto\ \isanewline
\ \ \ \ \isacommand{also}\isamarkupfalse%
\ \isacommand{have}\isamarkupfalse%
\ {\isachardoublequoteopen}{\isachardot}{\kern0pt}{\isachardot}{\kern0pt}{\isachardot}{\kern0pt}\ {\isasymlongleftrightarrow}\ u\ {\isasymin}\ HS{\isachardoublequoteclose}\isanewline
\ \ \ \ \ \ \isacommand{apply}\isamarkupfalse%
{\isacharparenleft}{\kern0pt}rule\ iff{\isacharunderscore}{\kern0pt}flip{\isacharcomma}{\kern0pt}\ rule\ iff{\isacharunderscore}{\kern0pt}trans{\isacharcomma}{\kern0pt}\ rule\ HS{\isacharunderscore}{\kern0pt}iff{\isacharparenright}{\kern0pt}\isanewline
\ \ \ \ \ \ \isacommand{unfolding}\isamarkupfalse%
\ symmetric{\isacharunderscore}{\kern0pt}def\ \isanewline
\ \ \ \ \ \ \isacommand{by}\isamarkupfalse%
\ auto\ \isanewline
\ \ \ \ \isacommand{finally}\isamarkupfalse%
\ \isacommand{show}\isamarkupfalse%
\ {\isachardoublequoteopen}is{\isacharunderscore}{\kern0pt}HS{\isacharparenleft}{\kern0pt}u{\isacharparenright}{\kern0pt}\ {\isacharequal}{\kern0pt}\ {\isadigit{1}}\ {\isasymlongleftrightarrow}\ u\ {\isasymin}\ HS{\isachardoublequoteclose}\ \isacommand{by}\isamarkupfalse%
\ simp\isanewline
\ \ \isacommand{qed}\isamarkupfalse%
\isanewline
\ \ \isacommand{then}\isamarkupfalse%
\ \isacommand{show}\isamarkupfalse%
\ {\isacharquery}{\kern0pt}thesis\ \isacommand{using}\isamarkupfalse%
\ assms\ \isacommand{by}\isamarkupfalse%
\ auto\isanewline
\isacommand{qed}\isamarkupfalse%
%
\endisatagproof
{\isafoldproof}%
%
\isadelimproof
\isanewline
%
\endisadelimproof
\isanewline
\isacommand{lemma}\isamarkupfalse%
\ sats{\isacharunderscore}{\kern0pt}is{\isacharunderscore}{\kern0pt}HS{\isacharunderscore}{\kern0pt}fm{\isacharunderscore}{\kern0pt}iff\ {\isacharcolon}{\kern0pt}\ \isanewline
\ \ \isakeyword{fixes}\ x\ i\ j\ env\ \isanewline
\ \ \isakeyword{assumes}\ {\isachardoublequoteopen}env\ {\isasymin}\ list{\isacharparenleft}{\kern0pt}M{\isacharparenright}{\kern0pt}{\isachardoublequoteclose}\ {\isachardoublequoteopen}i\ {\isacharless}{\kern0pt}\ length{\isacharparenleft}{\kern0pt}env{\isacharparenright}{\kern0pt}{\isachardoublequoteclose}\ {\isachardoublequoteopen}j\ {\isacharless}{\kern0pt}\ length{\isacharparenleft}{\kern0pt}env{\isacharparenright}{\kern0pt}{\isachardoublequoteclose}\ {\isachardoublequoteopen}nth{\isacharparenleft}{\kern0pt}i{\isacharcomma}{\kern0pt}\ env{\isacharparenright}{\kern0pt}\ {\isacharequal}{\kern0pt}\ {\isacharless}{\kern0pt}{\isasymF}{\isacharcomma}{\kern0pt}\ {\isasymG}{\isacharcomma}{\kern0pt}\ P{\isacharcomma}{\kern0pt}\ P{\isacharunderscore}{\kern0pt}auto{\isachargreater}{\kern0pt}{\isachardoublequoteclose}\ {\isachardoublequoteopen}nth{\isacharparenleft}{\kern0pt}j{\isacharcomma}{\kern0pt}\ env{\isacharparenright}{\kern0pt}\ {\isacharequal}{\kern0pt}\ x{\isachardoublequoteclose}\ \isanewline
\ \ \isakeyword{shows}\ {\isachardoublequoteopen}sats{\isacharparenleft}{\kern0pt}M{\isacharcomma}{\kern0pt}\ is{\isacharunderscore}{\kern0pt}HS{\isacharunderscore}{\kern0pt}fm{\isacharparenleft}{\kern0pt}i{\isacharcomma}{\kern0pt}\ j{\isacharparenright}{\kern0pt}{\isacharcomma}{\kern0pt}\ env{\isacharparenright}{\kern0pt}\ {\isasymlongleftrightarrow}\ x\ {\isasymin}\ HS{\isachardoublequoteclose}\ \isanewline
%
\isadelimproof
\isanewline
%
\endisadelimproof
%
\isatagproof
\isacommand{proof}\isamarkupfalse%
\ {\isacharminus}{\kern0pt}\ \isanewline
\ \ \isacommand{have}\isamarkupfalse%
\ innat\ {\isacharcolon}{\kern0pt}\ {\isachardoublequoteopen}i\ {\isasymin}\ nat\ {\isasymand}\ j\ {\isasymin}\ nat{\isachardoublequoteclose}\ \isacommand{using}\isamarkupfalse%
\ lt{\isacharunderscore}{\kern0pt}nat{\isacharunderscore}{\kern0pt}in{\isacharunderscore}{\kern0pt}nat\ assms\ \isacommand{by}\isamarkupfalse%
\ auto\ \isanewline
\isanewline
\ \ \isacommand{have}\isamarkupfalse%
\ {\isachardoublequoteopen}sats{\isacharparenleft}{\kern0pt}M{\isacharcomma}{\kern0pt}\ is{\isacharunderscore}{\kern0pt}HS{\isacharunderscore}{\kern0pt}fm{\isacharparenleft}{\kern0pt}i{\isacharcomma}{\kern0pt}\ j{\isacharparenright}{\kern0pt}{\isacharcomma}{\kern0pt}\ env{\isacharparenright}{\kern0pt}\ {\isasymlongleftrightarrow}\ {\isacharparenleft}{\kern0pt}{\isasymexists}v\ {\isasymin}\ M{\isachardot}{\kern0pt}\ v\ {\isacharequal}{\kern0pt}\ wftrec{\isacharparenleft}{\kern0pt}Memrel{\isacharparenleft}{\kern0pt}M{\isacharparenright}{\kern0pt}{\isacharcircum}{\kern0pt}{\isacharplus}{\kern0pt}{\isacharcomma}{\kern0pt}\ x{\isacharcomma}{\kern0pt}\ His{\isacharunderscore}{\kern0pt}HS{\isacharparenright}{\kern0pt}\ {\isasymand}\ v\ {\isacharequal}{\kern0pt}\ {\isadigit{1}}{\isacharparenright}{\kern0pt}{\isachardoublequoteclose}\ \isanewline
\ \ \ \ \isacommand{unfolding}\isamarkupfalse%
\ is{\isacharunderscore}{\kern0pt}HS{\isacharunderscore}{\kern0pt}fm{\isacharunderscore}{\kern0pt}def\ \isanewline
\ \ \ \ \isacommand{apply}\isamarkupfalse%
{\isacharparenleft}{\kern0pt}rule\ iff{\isacharunderscore}{\kern0pt}trans{\isacharcomma}{\kern0pt}\ rule\ sats{\isacharunderscore}{\kern0pt}Exists{\isacharunderscore}{\kern0pt}iff{\isacharcomma}{\kern0pt}\ simp\ add{\isacharcolon}{\kern0pt}assms{\isacharcomma}{\kern0pt}\ rule\ bex{\isacharunderscore}{\kern0pt}iff{\isacharparenright}{\kern0pt}\isanewline
\ \ \ \ \isacommand{apply}\isamarkupfalse%
{\isacharparenleft}{\kern0pt}rule\ iff{\isacharunderscore}{\kern0pt}trans{\isacharcomma}{\kern0pt}\ rule\ sats{\isacharunderscore}{\kern0pt}And{\isacharunderscore}{\kern0pt}iff{\isacharcomma}{\kern0pt}\ simp\ add{\isacharcolon}{\kern0pt}assms{\isacharcomma}{\kern0pt}\ rule\ iff{\isacharunderscore}{\kern0pt}conjI{\isacharparenright}{\kern0pt}\isanewline
\ \ \ \ \ \isacommand{apply}\isamarkupfalse%
{\isacharparenleft}{\kern0pt}rule{\isacharunderscore}{\kern0pt}tac\ a{\isacharequal}{\kern0pt}{\isachardoublequoteopen}{\isacharless}{\kern0pt}{\isasymF}{\isacharcomma}{\kern0pt}\ {\isasymG}{\isacharcomma}{\kern0pt}\ P{\isacharcomma}{\kern0pt}\ P{\isacharunderscore}{\kern0pt}auto{\isachargreater}{\kern0pt}{\isachardoublequoteclose}\ \isakeyword{and}\ G{\isacharequal}{\kern0pt}{\isachardoublequoteopen}His{\isacharunderscore}{\kern0pt}HS{\isacharunderscore}{\kern0pt}M{\isachardoublequoteclose}\ \isakeyword{in}\ sats{\isacharunderscore}{\kern0pt}is{\isacharunderscore}{\kern0pt}memrel{\isacharunderscore}{\kern0pt}wftrec{\isacharunderscore}{\kern0pt}fm{\isacharunderscore}{\kern0pt}iff{\isacharparenright}{\kern0pt}\isanewline
\ \ \ \ \isacommand{using}\isamarkupfalse%
\ assms\ {\isasymF}{\isacharunderscore}{\kern0pt}in{\isacharunderscore}{\kern0pt}M\ {\isasymG}{\isacharunderscore}{\kern0pt}in{\isacharunderscore}{\kern0pt}M\ P{\isacharunderscore}{\kern0pt}in{\isacharunderscore}{\kern0pt}M\ P{\isacharunderscore}{\kern0pt}auto{\isacharunderscore}{\kern0pt}in{\isacharunderscore}{\kern0pt}M\ pair{\isacharunderscore}{\kern0pt}in{\isacharunderscore}{\kern0pt}M{\isacharunderscore}{\kern0pt}iff\ innat\isanewline
\ \ \ \ \ \ \ \ \ \ \ \ \ \ \ \ \ \ \isacommand{apply}\isamarkupfalse%
\ auto{\isacharbrackleft}{\kern0pt}{\isadigit{9}}{\isacharbrackright}{\kern0pt}\isanewline
\ \ \ \ \isacommand{apply}\isamarkupfalse%
{\isacharparenleft}{\kern0pt}rule\ His{\isacharunderscore}{\kern0pt}HS{\isacharunderscore}{\kern0pt}M{\isacharunderscore}{\kern0pt}fm{\isacharunderscore}{\kern0pt}type{\isacharparenright}{\kern0pt}\isanewline
\ \ \ \ \ \ \ \ \ \ \ \isacommand{apply}\isamarkupfalse%
\ auto{\isacharbrackleft}{\kern0pt}{\isadigit{3}}{\isacharbrackright}{\kern0pt}\isanewline
\ \ \ \ \ \ \ \ \isacommand{apply}\isamarkupfalse%
{\isacharparenleft}{\kern0pt}rule\ le{\isacharunderscore}{\kern0pt}trans{\isacharcomma}{\kern0pt}\ rule\ arity{\isacharunderscore}{\kern0pt}His{\isacharunderscore}{\kern0pt}HS{\isacharunderscore}{\kern0pt}M{\isacharunderscore}{\kern0pt}fm{\isacharparenright}{\kern0pt}\isanewline
\ \ \ \ \ \ \ \ \ \ \ \isacommand{apply}\isamarkupfalse%
\ auto{\isacharbrackleft}{\kern0pt}{\isadigit{3}}{\isacharbrackright}{\kern0pt}\isanewline
\ \ \ \ \ \ \ \ \isacommand{apply}\isamarkupfalse%
{\isacharparenleft}{\kern0pt}rule\ Un{\isacharunderscore}{\kern0pt}least{\isacharunderscore}{\kern0pt}lt{\isacharparenright}{\kern0pt}{\isacharplus}{\kern0pt}\isanewline
\ \ \ \ \ \ \ \ \ \ \isacommand{apply}\isamarkupfalse%
\ auto{\isacharbrackleft}{\kern0pt}{\isadigit{3}}{\isacharbrackright}{\kern0pt}\isanewline
\ \ \ \ \isacommand{using}\isamarkupfalse%
\ zero{\isacharunderscore}{\kern0pt}in{\isacharunderscore}{\kern0pt}M\ succ{\isacharunderscore}{\kern0pt}in{\isacharunderscore}{\kern0pt}MI\ \isanewline
\ \ \ \ \ \ \ \isacommand{apply}\isamarkupfalse%
{\isacharparenleft}{\kern0pt}simp\ add{\isacharcolon}{\kern0pt}His{\isacharunderscore}{\kern0pt}HS{\isacharunderscore}{\kern0pt}M{\isacharunderscore}{\kern0pt}def{\isacharparenright}{\kern0pt}\isanewline
\ \ \ \ \ \ \isacommand{apply}\isamarkupfalse%
{\isacharparenleft}{\kern0pt}rule\ His{\isacharunderscore}{\kern0pt}HS{\isacharunderscore}{\kern0pt}eq{\isacharparenright}{\kern0pt}\isanewline
\ \ \ \ \ \ \ \ \ \ \ \ \ \isacommand{apply}\isamarkupfalse%
\ auto{\isacharbrackleft}{\kern0pt}{\isadigit{5}}{\isacharbrackright}{\kern0pt}\isanewline
\ \ \ \ \ \isacommand{apply}\isamarkupfalse%
{\isacharparenleft}{\kern0pt}rule\ iff{\isacharunderscore}{\kern0pt}flip{\isacharcomma}{\kern0pt}\ rule\ sats{\isacharunderscore}{\kern0pt}His{\isacharunderscore}{\kern0pt}HS{\isacharunderscore}{\kern0pt}M{\isacharunderscore}{\kern0pt}fm{\isacharunderscore}{\kern0pt}iff{\isacharparenright}{\kern0pt}\isanewline
\ \ \ \ \ \ \ \ \ \ \ \isacommand{apply}\isamarkupfalse%
\ auto{\isacharbrackleft}{\kern0pt}{\isadigit{7}}{\isacharbrackright}{\kern0pt}\isanewline
\ \ \ \ \isacommand{apply}\isamarkupfalse%
{\isacharparenleft}{\kern0pt}rule\ sats{\isacharunderscore}{\kern0pt}is{\isacharunderscore}{\kern0pt}{\isadigit{1}}{\isacharunderscore}{\kern0pt}fm{\isacharunderscore}{\kern0pt}iff{\isacharparenright}{\kern0pt}\isanewline
\ \ \ \ \isacommand{using}\isamarkupfalse%
\ assms\ \isanewline
\ \ \ \ \isacommand{by}\isamarkupfalse%
\ auto\isanewline
\ \ \isacommand{also}\isamarkupfalse%
\ \isacommand{have}\isamarkupfalse%
\ {\isachardoublequoteopen}{\isachardot}{\kern0pt}{\isachardot}{\kern0pt}{\isachardot}{\kern0pt}\ {\isasymlongleftrightarrow}\ is{\isacharunderscore}{\kern0pt}HS{\isacharparenleft}{\kern0pt}x{\isacharparenright}{\kern0pt}\ {\isacharequal}{\kern0pt}\ {\isadigit{1}}{\isachardoublequoteclose}\ \isacommand{unfolding}\isamarkupfalse%
\ is{\isacharunderscore}{\kern0pt}HS{\isacharunderscore}{\kern0pt}def\ \isacommand{by}\isamarkupfalse%
\ auto\ \isanewline
\ \ \isacommand{also}\isamarkupfalse%
\ \isacommand{have}\isamarkupfalse%
\ {\isachardoublequoteopen}{\isachardot}{\kern0pt}{\isachardot}{\kern0pt}{\isachardot}{\kern0pt}\ {\isasymlongleftrightarrow}\ x\ {\isasymin}\ HS{\isachardoublequoteclose}\ \isacommand{using}\isamarkupfalse%
\ def{\isacharunderscore}{\kern0pt}is{\isacharunderscore}{\kern0pt}HS\ assms\ nth{\isacharunderscore}{\kern0pt}type\ \isacommand{by}\isamarkupfalse%
\ auto\ \isanewline
\ \ \isacommand{finally}\isamarkupfalse%
\ \isacommand{show}\isamarkupfalse%
\ {\isacharquery}{\kern0pt}thesis\ \isacommand{by}\isamarkupfalse%
\ simp\isanewline
\isacommand{qed}\isamarkupfalse%
%
\endisatagproof
{\isafoldproof}%
%
\isadelimproof
\isanewline
%
\endisadelimproof
\isanewline
\isacommand{lemma}\isamarkupfalse%
\ HS{\isacharunderscore}{\kern0pt}separation\ {\isacharcolon}{\kern0pt}\ \isanewline
\ \ \isakeyword{fixes}\ A\ \isanewline
\ \ \isakeyword{assumes}\ {\isachardoublequoteopen}A\ {\isasymin}\ M{\isachardoublequoteclose}\ \isanewline
\ \ \isakeyword{shows}\ {\isachardoublequoteopen}A\ {\isasyminter}\ HS\ {\isasymin}\ M{\isachardoublequoteclose}\ \isanewline
%
\isadelimproof
%
\endisadelimproof
%
\isatagproof
\isacommand{proof}\isamarkupfalse%
\ {\isacharminus}{\kern0pt}\ \isanewline
\ \ \isacommand{have}\isamarkupfalse%
\ sep\ {\isacharcolon}{\kern0pt}\ {\isachardoublequoteopen}separation{\isacharparenleft}{\kern0pt}{\isacharhash}{\kern0pt}{\isacharhash}{\kern0pt}M{\isacharcomma}{\kern0pt}\ {\isasymlambda}x{\isachardot}{\kern0pt}\ sats{\isacharparenleft}{\kern0pt}M{\isacharcomma}{\kern0pt}\ is{\isacharunderscore}{\kern0pt}HS{\isacharunderscore}{\kern0pt}fm{\isacharparenleft}{\kern0pt}{\isadigit{1}}{\isacharcomma}{\kern0pt}\ {\isadigit{0}}{\isacharparenright}{\kern0pt}{\isacharcomma}{\kern0pt}\ {\isacharbrackleft}{\kern0pt}x{\isacharbrackright}{\kern0pt}\ {\isacharat}{\kern0pt}\ {\isacharbrackleft}{\kern0pt}{\isacharless}{\kern0pt}{\isasymF}{\isacharcomma}{\kern0pt}\ {\isasymG}{\isacharcomma}{\kern0pt}\ P{\isacharcomma}{\kern0pt}\ P{\isacharunderscore}{\kern0pt}auto{\isachargreater}{\kern0pt}{\isacharbrackright}{\kern0pt}{\isacharparenright}{\kern0pt}{\isacharparenright}{\kern0pt}{\isachardoublequoteclose}\isanewline
\ \ \ \ \isacommand{apply}\isamarkupfalse%
{\isacharparenleft}{\kern0pt}rule\ separation{\isacharunderscore}{\kern0pt}ax{\isacharparenright}{\kern0pt}\isanewline
\ \ \ \ \isacommand{apply}\isamarkupfalse%
{\isacharparenleft}{\kern0pt}rule\ is{\isacharunderscore}{\kern0pt}HS{\isacharunderscore}{\kern0pt}fm{\isacharunderscore}{\kern0pt}type{\isacharparenright}{\kern0pt}\isanewline
\ \ \ \ \isacommand{using}\isamarkupfalse%
\ assms\ {\isasymF}{\isacharunderscore}{\kern0pt}in{\isacharunderscore}{\kern0pt}M\ {\isasymG}{\isacharunderscore}{\kern0pt}in{\isacharunderscore}{\kern0pt}M\ P{\isacharunderscore}{\kern0pt}in{\isacharunderscore}{\kern0pt}M\ P{\isacharunderscore}{\kern0pt}auto{\isacharunderscore}{\kern0pt}in{\isacharunderscore}{\kern0pt}M\ pair{\isacharunderscore}{\kern0pt}in{\isacharunderscore}{\kern0pt}M{\isacharunderscore}{\kern0pt}iff\ \ \isanewline
\ \ \ \ \ \ \ \isacommand{apply}\isamarkupfalse%
\ auto{\isacharbrackleft}{\kern0pt}{\isadigit{4}}{\isacharbrackright}{\kern0pt}\isanewline
\ \ \ \ \isacommand{apply}\isamarkupfalse%
{\isacharparenleft}{\kern0pt}rule\ le{\isacharunderscore}{\kern0pt}trans{\isacharcomma}{\kern0pt}\ rule\ arity{\isacharunderscore}{\kern0pt}is{\isacharunderscore}{\kern0pt}HS{\isacharunderscore}{\kern0pt}fm{\isacharcomma}{\kern0pt}\ simp{\isacharcomma}{\kern0pt}\ simp{\isacharparenright}{\kern0pt}\isanewline
\ \ \ \ \isacommand{apply}\isamarkupfalse%
{\isacharparenleft}{\kern0pt}rule\ Un{\isacharunderscore}{\kern0pt}least{\isacharunderscore}{\kern0pt}lt{\isacharparenright}{\kern0pt}\isanewline
\ \ \ \ \isacommand{by}\isamarkupfalse%
\ auto\isanewline
\isanewline
\ \ \isacommand{define}\isamarkupfalse%
\ S\ \isakeyword{where}\ {\isachardoublequoteopen}S\ {\isasymequiv}\ {\isacharbraceleft}{\kern0pt}\ x\ {\isasymin}\ A{\isachardot}{\kern0pt}\ sats{\isacharparenleft}{\kern0pt}M{\isacharcomma}{\kern0pt}\ is{\isacharunderscore}{\kern0pt}HS{\isacharunderscore}{\kern0pt}fm{\isacharparenleft}{\kern0pt}{\isadigit{1}}{\isacharcomma}{\kern0pt}\ {\isadigit{0}}{\isacharparenright}{\kern0pt}{\isacharcomma}{\kern0pt}\ {\isacharbrackleft}{\kern0pt}x{\isacharbrackright}{\kern0pt}\ {\isacharat}{\kern0pt}\ {\isacharbrackleft}{\kern0pt}{\isacharless}{\kern0pt}{\isasymF}{\isacharcomma}{\kern0pt}\ {\isasymG}{\isacharcomma}{\kern0pt}\ P{\isacharcomma}{\kern0pt}\ P{\isacharunderscore}{\kern0pt}auto{\isachargreater}{\kern0pt}{\isacharbrackright}{\kern0pt}{\isacharparenright}{\kern0pt}\ {\isacharbraceright}{\kern0pt}{\isachardoublequoteclose}\ \isanewline
\ \ \isacommand{have}\isamarkupfalse%
\ SinM\ {\isacharcolon}{\kern0pt}\ {\isachardoublequoteopen}S\ {\isasymin}\ M{\isachardoublequoteclose}\ \isanewline
\ \ \ \ \isacommand{unfolding}\isamarkupfalse%
\ S{\isacharunderscore}{\kern0pt}def\ \isanewline
\ \ \ \ \isacommand{apply}\isamarkupfalse%
{\isacharparenleft}{\kern0pt}rule\ separation{\isacharunderscore}{\kern0pt}notation{\isacharparenright}{\kern0pt}\isanewline
\ \ \ \ \isacommand{using}\isamarkupfalse%
\ assms\ sep\ \isanewline
\ \ \ \ \isacommand{by}\isamarkupfalse%
\ auto\isanewline
\isanewline
\ \ \isacommand{have}\isamarkupfalse%
\ {\isachardoublequoteopen}S\ {\isacharequal}{\kern0pt}\ {\isacharbraceleft}{\kern0pt}\ x\ {\isasymin}\ A{\isachardot}{\kern0pt}\ x\ {\isasymin}\ HS\ {\isacharbraceright}{\kern0pt}{\isachardoublequoteclose}\ \isanewline
\ \ \ \ \isacommand{unfolding}\isamarkupfalse%
\ S{\isacharunderscore}{\kern0pt}def\isanewline
\ \ \ \ \isacommand{apply}\isamarkupfalse%
{\isacharparenleft}{\kern0pt}rule\ iff{\isacharunderscore}{\kern0pt}eq{\isacharparenright}{\kern0pt}\isanewline
\ \ \ \ \isacommand{apply}\isamarkupfalse%
{\isacharparenleft}{\kern0pt}rule\ sats{\isacharunderscore}{\kern0pt}is{\isacharunderscore}{\kern0pt}HS{\isacharunderscore}{\kern0pt}fm{\isacharunderscore}{\kern0pt}iff{\isacharparenright}{\kern0pt}\isanewline
\ \ \ \ \isacommand{using}\isamarkupfalse%
\ assms\ {\isasymF}{\isacharunderscore}{\kern0pt}in{\isacharunderscore}{\kern0pt}M\ {\isasymG}{\isacharunderscore}{\kern0pt}in{\isacharunderscore}{\kern0pt}M\ P{\isacharunderscore}{\kern0pt}in{\isacharunderscore}{\kern0pt}M\ P{\isacharunderscore}{\kern0pt}auto{\isacharunderscore}{\kern0pt}in{\isacharunderscore}{\kern0pt}M\ pair{\isacharunderscore}{\kern0pt}in{\isacharunderscore}{\kern0pt}M{\isacharunderscore}{\kern0pt}iff\ transM\isanewline
\ \ \ \ \ \ \ \ \isacommand{apply}\isamarkupfalse%
\ auto{\isacharbrackleft}{\kern0pt}{\isadigit{5}}{\isacharbrackright}{\kern0pt}\isanewline
\ \ \ \ \isacommand{done}\isamarkupfalse%
\ \isanewline
\isanewline
\ \ \isacommand{also}\isamarkupfalse%
\ \isacommand{have}\isamarkupfalse%
\ {\isachardoublequoteopen}{\isachardot}{\kern0pt}{\isachardot}{\kern0pt}{\isachardot}{\kern0pt}\ {\isacharequal}{\kern0pt}\ A\ {\isasyminter}\ HS{\isachardoublequoteclose}\ \isacommand{by}\isamarkupfalse%
\ auto\ \isanewline
\isanewline
\ \ \isacommand{finally}\isamarkupfalse%
\ \isacommand{show}\isamarkupfalse%
\ {\isacharquery}{\kern0pt}thesis\ \isacommand{using}\isamarkupfalse%
\ SinM\ \isacommand{by}\isamarkupfalse%
\ auto\isanewline
\isacommand{qed}\isamarkupfalse%
%
\endisatagproof
{\isafoldproof}%
%
\isadelimproof
\isanewline
%
\endisadelimproof
\ \ \isanewline
\isacommand{end}\isamarkupfalse%
\isanewline
%
\isadelimtheory
%
\endisadelimtheory
%
\isatagtheory
\isacommand{end}\isamarkupfalse%
%
\endisatagtheory
{\isafoldtheory}%
%
\isadelimtheory
%
\endisadelimtheory
%
\end{isabellebody}%
\endinput
%:%file=~/source/repos/ZF-notAC/code/HS_M.thy%:%
%:%10=1%:%
%:%11=1%:%
%:%12=2%:%
%:%13=3%:%
%:%14=4%:%
%:%15=5%:%
%:%20=5%:%
%:%23=6%:%
%:%24=7%:%
%:%25=8%:%
%:%26=8%:%
%:%27=9%:%
%:%28=10%:%
%:%29=11%:%
%:%30=12%:%
%:%31=12%:%
%:%32=13%:%
%:%33=14%:%
%:%34=15%:%
%:%35=15%:%
%:%36=16%:%
%:%37=17%:%
%:%38=18%:%
%:%41=19%:%
%:%45=19%:%
%:%46=19%:%
%:%47=20%:%
%:%48=20%:%
%:%49=21%:%
%:%50=21%:%
%:%51=22%:%
%:%52=22%:%
%:%53=23%:%
%:%54=23%:%
%:%55=24%:%
%:%56=24%:%
%:%61=24%:%
%:%64=25%:%
%:%65=26%:%
%:%66=26%:%
%:%67=27%:%
%:%68=28%:%
%:%69=29%:%
%:%72=30%:%
%:%76=30%:%
%:%77=30%:%
%:%78=31%:%
%:%79=31%:%
%:%80=32%:%
%:%81=32%:%
%:%82=33%:%
%:%83=33%:%
%:%84=34%:%
%:%85=34%:%
%:%86=35%:%
%:%87=35%:%
%:%88=36%:%
%:%89=36%:%
%:%90=37%:%
%:%91=37%:%
%:%92=38%:%
%:%93=38%:%
%:%94=39%:%
%:%95=39%:%
%:%96=40%:%
%:%97=40%:%
%:%98=41%:%
%:%99=41%:%
%:%100=42%:%
%:%101=42%:%
%:%102=43%:%
%:%103=43%:%
%:%104=44%:%
%:%105=44%:%
%:%106=45%:%
%:%107=45%:%
%:%108=46%:%
%:%109=46%:%
%:%110=47%:%
%:%111=47%:%
%:%112=48%:%
%:%113=48%:%
%:%114=49%:%
%:%115=49%:%
%:%116=50%:%
%:%117=50%:%
%:%118=51%:%
%:%119=51%:%
%:%120=52%:%
%:%121=52%:%
%:%122=53%:%
%:%123=53%:%
%:%124=54%:%
%:%125=54%:%
%:%126=55%:%
%:%127=55%:%
%:%128=56%:%
%:%129=56%:%
%:%134=56%:%
%:%137=57%:%
%:%138=58%:%
%:%139=58%:%
%:%140=59%:%
%:%141=60%:%
%:%142=61%:%
%:%149=62%:%
%:%150=62%:%
%:%151=63%:%
%:%152=63%:%
%:%153=64%:%
%:%154=64%:%
%:%155=65%:%
%:%156=65%:%
%:%157=66%:%
%:%158=66%:%
%:%159=67%:%
%:%160=67%:%
%:%161=68%:%
%:%162=68%:%
%:%163=69%:%
%:%164=69%:%
%:%165=70%:%
%:%166=70%:%
%:%167=71%:%
%:%168=71%:%
%:%169=72%:%
%:%170=72%:%
%:%171=73%:%
%:%172=73%:%
%:%173=74%:%
%:%174=74%:%
%:%175=75%:%
%:%176=75%:%
%:%177=76%:%
%:%178=76%:%
%:%179=77%:%
%:%180=77%:%
%:%181=78%:%
%:%182=78%:%
%:%183=79%:%
%:%184=79%:%
%:%185=80%:%
%:%186=80%:%
%:%187=81%:%
%:%188=81%:%
%:%189=82%:%
%:%190=82%:%
%:%191=83%:%
%:%192=83%:%
%:%193=84%:%
%:%194=84%:%
%:%195=85%:%
%:%196=85%:%
%:%197=86%:%
%:%198=86%:%
%:%199=87%:%
%:%200=87%:%
%:%201=88%:%
%:%202=88%:%
%:%203=89%:%
%:%204=89%:%
%:%205=90%:%
%:%206=90%:%
%:%207=91%:%
%:%208=91%:%
%:%209=92%:%
%:%210=92%:%
%:%211=93%:%
%:%212=93%:%
%:%213=94%:%
%:%214=94%:%
%:%215=95%:%
%:%216=95%:%
%:%217=96%:%
%:%218=96%:%
%:%219=97%:%
%:%220=97%:%
%:%221=98%:%
%:%222=98%:%
%:%223=99%:%
%:%224=99%:%
%:%225=100%:%
%:%226=100%:%
%:%227=101%:%
%:%228=101%:%
%:%229=102%:%
%:%230=102%:%
%:%231=102%:%
%:%232=103%:%
%:%233=103%:%
%:%234=104%:%
%:%235=104%:%
%:%236=105%:%
%:%237=105%:%
%:%238=106%:%
%:%239=106%:%
%:%240=107%:%
%:%241=107%:%
%:%242=108%:%
%:%243=108%:%
%:%244=109%:%
%:%245=109%:%
%:%246=110%:%
%:%247=110%:%
%:%248=110%:%
%:%249=110%:%
%:%250=111%:%
%:%256=111%:%
%:%259=112%:%
%:%260=113%:%
%:%261=113%:%
%:%262=114%:%
%:%263=115%:%
%:%264=115%:%
%:%265=116%:%
%:%266=117%:%
%:%267=118%:%
%:%268=119%:%
%:%269=119%:%
%:%270=120%:%
%:%271=121%:%
%:%272=122%:%
%:%273=122%:%
%:%274=123%:%
%:%275=124%:%
%:%276=125%:%
%:%279=126%:%
%:%280=127%:%
%:%284=127%:%
%:%285=127%:%
%:%286=128%:%
%:%287=128%:%
%:%288=129%:%
%:%289=129%:%
%:%290=130%:%
%:%291=130%:%
%:%292=131%:%
%:%293=131%:%
%:%294=132%:%
%:%295=132%:%
%:%296=133%:%
%:%297=133%:%
%:%302=133%:%
%:%305=134%:%
%:%306=135%:%
%:%307=135%:%
%:%308=136%:%
%:%309=137%:%
%:%310=138%:%
%:%313=139%:%
%:%314=140%:%
%:%318=140%:%
%:%319=140%:%
%:%320=141%:%
%:%321=141%:%
%:%322=142%:%
%:%323=142%:%
%:%324=143%:%
%:%325=143%:%
%:%326=144%:%
%:%327=144%:%
%:%328=145%:%
%:%329=145%:%
%:%330=146%:%
%:%331=146%:%
%:%332=147%:%
%:%333=147%:%
%:%334=148%:%
%:%335=148%:%
%:%336=149%:%
%:%337=149%:%
%:%338=150%:%
%:%339=150%:%
%:%340=151%:%
%:%341=151%:%
%:%342=152%:%
%:%343=152%:%
%:%344=153%:%
%:%345=153%:%
%:%346=154%:%
%:%347=154%:%
%:%348=155%:%
%:%349=155%:%
%:%350=156%:%
%:%351=156%:%
%:%352=157%:%
%:%353=157%:%
%:%354=158%:%
%:%355=158%:%
%:%356=159%:%
%:%357=159%:%
%:%358=160%:%
%:%359=160%:%
%:%364=160%:%
%:%367=161%:%
%:%368=162%:%
%:%369=162%:%
%:%370=163%:%
%:%371=164%:%
%:%372=165%:%
%:%373=166%:%
%:%374=167%:%
%:%381=168%:%
%:%382=168%:%
%:%383=169%:%
%:%384=169%:%
%:%385=170%:%
%:%386=170%:%
%:%387=171%:%
%:%388=171%:%
%:%389=172%:%
%:%390=172%:%
%:%391=173%:%
%:%392=173%:%
%:%393=174%:%
%:%394=174%:%
%:%395=175%:%
%:%396=175%:%
%:%397=175%:%
%:%398=176%:%
%:%399=176%:%
%:%400=177%:%
%:%401=177%:%
%:%402=178%:%
%:%403=178%:%
%:%404=179%:%
%:%405=179%:%
%:%406=180%:%
%:%407=180%:%
%:%408=181%:%
%:%409=181%:%
%:%410=181%:%
%:%411=182%:%
%:%412=182%:%
%:%413=183%:%
%:%414=183%:%
%:%415=184%:%
%:%416=184%:%
%:%417=185%:%
%:%418=185%:%
%:%419=186%:%
%:%420=186%:%
%:%421=187%:%
%:%422=187%:%
%:%423=187%:%
%:%424=188%:%
%:%425=188%:%
%:%426=189%:%
%:%427=189%:%
%:%428=190%:%
%:%429=190%:%
%:%430=191%:%
%:%431=191%:%
%:%432=191%:%
%:%433=191%:%
%:%434=192%:%
%:%440=192%:%
%:%443=193%:%
%:%444=193%:%
%:%445=194%:%
%:%446=195%:%
%:%447=195%:%
%:%448=196%:%
%:%449=197%:%
%:%450=198%:%
%:%451=199%:%
%:%452=200%:%
%:%453=200%:%
%:%454=201%:%
%:%455=202%:%
%:%456=203%:%
%:%457=203%:%
%:%458=204%:%
%:%459=205%:%
%:%460=206%:%
%:%463=207%:%
%:%467=207%:%
%:%468=207%:%
%:%469=208%:%
%:%470=208%:%
%:%471=209%:%
%:%472=209%:%
%:%473=210%:%
%:%474=210%:%
%:%475=211%:%
%:%476=211%:%
%:%477=212%:%
%:%478=212%:%
%:%479=213%:%
%:%480=213%:%
%:%485=213%:%
%:%488=214%:%
%:%489=215%:%
%:%490=215%:%
%:%491=216%:%
%:%492=217%:%
%:%493=218%:%
%:%496=219%:%
%:%497=220%:%
%:%501=220%:%
%:%502=220%:%
%:%503=221%:%
%:%504=221%:%
%:%505=222%:%
%:%506=222%:%
%:%507=223%:%
%:%508=223%:%
%:%509=224%:%
%:%510=224%:%
%:%511=225%:%
%:%512=225%:%
%:%513=226%:%
%:%514=226%:%
%:%515=227%:%
%:%516=227%:%
%:%517=228%:%
%:%518=228%:%
%:%519=229%:%
%:%520=229%:%
%:%521=230%:%
%:%522=230%:%
%:%523=231%:%
%:%524=231%:%
%:%525=232%:%
%:%526=232%:%
%:%527=233%:%
%:%528=233%:%
%:%529=234%:%
%:%530=234%:%
%:%531=235%:%
%:%532=235%:%
%:%533=236%:%
%:%534=236%:%
%:%539=236%:%
%:%542=237%:%
%:%543=238%:%
%:%544=238%:%
%:%545=239%:%
%:%546=240%:%
%:%547=241%:%
%:%548=242%:%
%:%555=243%:%
%:%556=243%:%
%:%557=244%:%
%:%558=245%:%
%:%559=245%:%
%:%560=245%:%
%:%561=246%:%
%:%562=247%:%
%:%563=247%:%
%:%564=248%:%
%:%565=248%:%
%:%566=248%:%
%:%567=249%:%
%:%568=250%:%
%:%569=250%:%
%:%570=251%:%
%:%571=251%:%
%:%572=252%:%
%:%573=252%:%
%:%574=253%:%
%:%575=253%:%
%:%576=254%:%
%:%577=254%:%
%:%578=255%:%
%:%579=255%:%
%:%580=256%:%
%:%581=256%:%
%:%582=257%:%
%:%583=257%:%
%:%584=258%:%
%:%585=258%:%
%:%586=259%:%
%:%587=259%:%
%:%588=260%:%
%:%589=260%:%
%:%590=260%:%
%:%591=261%:%
%:%592=261%:%
%:%593=262%:%
%:%594=262%:%
%:%595=263%:%
%:%596=263%:%
%:%597=264%:%
%:%598=264%:%
%:%599=265%:%
%:%600=265%:%
%:%601=266%:%
%:%602=266%:%
%:%603=267%:%
%:%604=267%:%
%:%605=268%:%
%:%606=268%:%
%:%607=269%:%
%:%608=269%:%
%:%609=270%:%
%:%610=270%:%
%:%611=270%:%
%:%612=270%:%
%:%613=271%:%
%:%619=271%:%
%:%622=272%:%
%:%623=273%:%
%:%624=273%:%
%:%625=274%:%
%:%626=275%:%
%:%627=276%:%
%:%630=277%:%
%:%635=278%:%
%:%636=278%:%
%:%637=279%:%
%:%638=280%:%
%:%639=280%:%
%:%640=281%:%
%:%641=282%:%
%:%642=282%:%
%:%643=283%:%
%:%644=283%:%
%:%645=284%:%
%:%646=284%:%
%:%647=285%:%
%:%648=285%:%
%:%649=286%:%
%:%650=286%:%
%:%651=287%:%
%:%652=287%:%
%:%653=288%:%
%:%654=288%:%
%:%655=289%:%
%:%656=289%:%
%:%657=290%:%
%:%658=290%:%
%:%659=291%:%
%:%660=291%:%
%:%661=292%:%
%:%662=293%:%
%:%663=293%:%
%:%664=293%:%
%:%665=294%:%
%:%666=294%:%
%:%667=295%:%
%:%668=295%:%
%:%669=296%:%
%:%670=296%:%
%:%671=297%:%
%:%672=297%:%
%:%673=298%:%
%:%674=299%:%
%:%675=299%:%
%:%676=300%:%
%:%677=300%:%
%:%678=301%:%
%:%679=301%:%
%:%680=302%:%
%:%681=302%:%
%:%682=303%:%
%:%683=303%:%
%:%684=304%:%
%:%685=304%:%
%:%686=305%:%
%:%687=305%:%
%:%688=306%:%
%:%689=306%:%
%:%690=307%:%
%:%691=307%:%
%:%692=307%:%
%:%693=307%:%
%:%694=307%:%
%:%695=308%:%
%:%701=308%:%
%:%704=309%:%
%:%705=310%:%
%:%706=310%:%
%:%707=311%:%
%:%708=323%:%
%:%709=324%:%
%:%710=325%:%
%:%711=325%:%
%:%712=326%:%
%:%721=335%:%
%:%722=336%:%
%:%723=337%:%
%:%724=338%:%
%:%725=338%:%
%:%726=339%:%
%:%727=340%:%
%:%728=341%:%
%:%729=341%:%
%:%730=342%:%
%:%731=343%:%
%:%732=344%:%
%:%735=345%:%
%:%736=346%:%
%:%740=346%:%
%:%741=346%:%
%:%742=347%:%
%:%743=347%:%
%:%744=348%:%
%:%745=348%:%
%:%746=349%:%
%:%747=349%:%
%:%748=350%:%
%:%749=350%:%
%:%750=351%:%
%:%751=351%:%
%:%752=352%:%
%:%753=352%:%
%:%758=352%:%
%:%761=353%:%
%:%762=354%:%
%:%763=354%:%
%:%764=355%:%
%:%765=356%:%
%:%766=357%:%
%:%769=358%:%
%:%770=359%:%
%:%774=359%:%
%:%775=359%:%
%:%776=360%:%
%:%777=360%:%
%:%778=361%:%
%:%779=361%:%
%:%780=362%:%
%:%781=362%:%
%:%782=363%:%
%:%783=363%:%
%:%784=364%:%
%:%785=364%:%
%:%786=365%:%
%:%787=365%:%
%:%788=366%:%
%:%789=366%:%
%:%790=367%:%
%:%791=367%:%
%:%792=368%:%
%:%793=368%:%
%:%794=369%:%
%:%795=369%:%
%:%796=370%:%
%:%797=370%:%
%:%798=371%:%
%:%799=371%:%
%:%800=372%:%
%:%801=372%:%
%:%802=373%:%
%:%803=373%:%
%:%804=374%:%
%:%805=374%:%
%:%806=375%:%
%:%807=375%:%
%:%808=376%:%
%:%809=376%:%
%:%810=377%:%
%:%811=377%:%
%:%812=378%:%
%:%813=378%:%
%:%814=379%:%
%:%815=379%:%
%:%816=380%:%
%:%817=380%:%
%:%818=381%:%
%:%819=381%:%
%:%820=382%:%
%:%821=382%:%
%:%822=383%:%
%:%823=383%:%
%:%824=384%:%
%:%825=384%:%
%:%826=385%:%
%:%827=385%:%
%:%828=386%:%
%:%829=386%:%
%:%830=387%:%
%:%831=387%:%
%:%832=388%:%
%:%833=388%:%
%:%834=389%:%
%:%835=389%:%
%:%840=389%:%
%:%843=390%:%
%:%844=391%:%
%:%845=391%:%
%:%846=392%:%
%:%847=393%:%
%:%848=393%:%
%:%849=394%:%
%:%850=394%:%
%:%851=395%:%
%:%852=396%:%
%:%853=396%:%
%:%854=397%:%
%:%855=398%:%
%:%856=399%:%
%:%857=399%:%
%:%858=400%:%
%:%859=401%:%
%:%860=402%:%
%:%863=403%:%
%:%867=403%:%
%:%868=403%:%
%:%869=404%:%
%:%870=404%:%
%:%871=405%:%
%:%872=405%:%
%:%873=406%:%
%:%874=406%:%
%:%875=407%:%
%:%876=407%:%
%:%877=408%:%
%:%878=408%:%
%:%879=409%:%
%:%880=409%:%
%:%881=410%:%
%:%882=410%:%
%:%887=410%:%
%:%890=411%:%
%:%891=412%:%
%:%892=412%:%
%:%893=413%:%
%:%894=414%:%
%:%895=415%:%
%:%898=416%:%
%:%902=416%:%
%:%903=416%:%
%:%904=417%:%
%:%905=417%:%
%:%906=418%:%
%:%907=418%:%
%:%908=419%:%
%:%909=419%:%
%:%910=420%:%
%:%911=420%:%
%:%912=421%:%
%:%913=421%:%
%:%914=422%:%
%:%915=422%:%
%:%916=423%:%
%:%917=423%:%
%:%918=424%:%
%:%919=424%:%
%:%920=425%:%
%:%921=425%:%
%:%922=426%:%
%:%923=426%:%
%:%924=427%:%
%:%925=427%:%
%:%930=427%:%
%:%933=428%:%
%:%934=429%:%
%:%935=429%:%
%:%936=430%:%
%:%937=431%:%
%:%938=432%:%
%:%941=433%:%
%:%942=434%:%
%:%946=434%:%
%:%947=434%:%
%:%948=435%:%
%:%949=435%:%
%:%950=436%:%
%:%951=436%:%
%:%952=437%:%
%:%953=437%:%
%:%954=438%:%
%:%955=438%:%
%:%956=439%:%
%:%957=439%:%
%:%958=440%:%
%:%959=440%:%
%:%960=441%:%
%:%961=441%:%
%:%962=442%:%
%:%963=442%:%
%:%964=443%:%
%:%965=443%:%
%:%966=444%:%
%:%967=444%:%
%:%968=445%:%
%:%969=445%:%
%:%970=446%:%
%:%971=446%:%
%:%972=447%:%
%:%973=447%:%
%:%974=448%:%
%:%975=448%:%
%:%976=449%:%
%:%977=449%:%
%:%978=450%:%
%:%979=450%:%
%:%980=451%:%
%:%981=451%:%
%:%982=452%:%
%:%983=452%:%
%:%984=453%:%
%:%985=453%:%
%:%986=454%:%
%:%987=454%:%
%:%988=455%:%
%:%989=455%:%
%:%990=456%:%
%:%991=456%:%
%:%992=457%:%
%:%998=457%:%
%:%1001=458%:%
%:%1002=459%:%
%:%1003=459%:%
%:%1004=460%:%
%:%1005=461%:%
%:%1006=461%:%
%:%1007=462%:%
%:%1008=463%:%
%:%1009=463%:%
%:%1010=464%:%
%:%1011=465%:%
%:%1012=466%:%
%:%1013=466%:%
%:%1014=467%:%
%:%1015=468%:%
%:%1016=469%:%
%:%1019=470%:%
%:%1023=470%:%
%:%1024=470%:%
%:%1025=471%:%
%:%1026=471%:%
%:%1027=472%:%
%:%1028=472%:%
%:%1029=473%:%
%:%1030=473%:%
%:%1031=474%:%
%:%1032=474%:%
%:%1033=475%:%
%:%1034=475%:%
%:%1039=475%:%
%:%1042=476%:%
%:%1043=477%:%
%:%1044=477%:%
%:%1045=478%:%
%:%1046=479%:%
%:%1047=480%:%
%:%1050=481%:%
%:%1054=481%:%
%:%1055=481%:%
%:%1056=482%:%
%:%1057=482%:%
%:%1058=483%:%
%:%1059=483%:%
%:%1060=484%:%
%:%1061=484%:%
%:%1062=485%:%
%:%1063=485%:%
%:%1064=486%:%
%:%1065=486%:%
%:%1066=487%:%
%:%1067=487%:%
%:%1068=488%:%
%:%1069=488%:%
%:%1070=489%:%
%:%1071=489%:%
%:%1072=490%:%
%:%1073=490%:%
%:%1074=491%:%
%:%1075=491%:%
%:%1076=492%:%
%:%1077=492%:%
%:%1078=493%:%
%:%1079=493%:%
%:%1080=494%:%
%:%1081=494%:%
%:%1082=495%:%
%:%1083=495%:%
%:%1084=496%:%
%:%1085=496%:%
%:%1086=497%:%
%:%1087=497%:%
%:%1088=498%:%
%:%1089=498%:%
%:%1090=499%:%
%:%1091=499%:%
%:%1092=500%:%
%:%1093=500%:%
%:%1094=501%:%
%:%1095=501%:%
%:%1096=502%:%
%:%1097=502%:%
%:%1098=503%:%
%:%1099=503%:%
%:%1100=504%:%
%:%1101=504%:%
%:%1102=505%:%
%:%1103=505%:%
%:%1104=506%:%
%:%1105=506%:%
%:%1110=506%:%
%:%1113=507%:%
%:%1114=508%:%
%:%1115=508%:%
%:%1116=509%:%
%:%1117=510%:%
%:%1118=511%:%
%:%1119=511%:%
%:%1120=512%:%
%:%1121=513%:%
%:%1122=514%:%
%:%1125=515%:%
%:%1130=516%:%
%:%1131=516%:%
%:%1132=517%:%
%:%1133=517%:%
%:%1134=517%:%
%:%1135=517%:%
%:%1136=518%:%
%:%1137=518%:%
%:%1138=518%:%
%:%1139=518%:%
%:%1140=519%:%
%:%1141=520%:%
%:%1142=520%:%
%:%1143=521%:%
%:%1144=522%:%
%:%1145=522%:%
%:%1146=523%:%
%:%1147=523%:%
%:%1148=524%:%
%:%1149=524%:%
%:%1150=525%:%
%:%1151=525%:%
%:%1152=526%:%
%:%1153=526%:%
%:%1154=527%:%
%:%1155=527%:%
%:%1156=528%:%
%:%1157=528%:%
%:%1158=529%:%
%:%1159=529%:%
%:%1160=530%:%
%:%1161=530%:%
%:%1162=531%:%
%:%1163=532%:%
%:%1164=532%:%
%:%1165=532%:%
%:%1166=533%:%
%:%1167=533%:%
%:%1168=534%:%
%:%1169=534%:%
%:%1170=535%:%
%:%1171=535%:%
%:%1172=536%:%
%:%1173=536%:%
%:%1174=537%:%
%:%1175=537%:%
%:%1176=538%:%
%:%1177=539%:%
%:%1178=539%:%
%:%1179=539%:%
%:%1180=540%:%
%:%1181=540%:%
%:%1182=541%:%
%:%1183=541%:%
%:%1184=542%:%
%:%1185=542%:%
%:%1186=543%:%
%:%1187=543%:%
%:%1188=544%:%
%:%1189=544%:%
%:%1190=545%:%
%:%1191=545%:%
%:%1192=546%:%
%:%1193=546%:%
%:%1194=547%:%
%:%1195=547%:%
%:%1196=548%:%
%:%1197=549%:%
%:%1198=549%:%
%:%1199=549%:%
%:%1200=550%:%
%:%1201=550%:%
%:%1202=551%:%
%:%1203=551%:%
%:%1204=552%:%
%:%1205=553%:%
%:%1206=553%:%
%:%1207=553%:%
%:%1208=553%:%
%:%1209=554%:%
%:%1215=554%:%
%:%1218=555%:%
%:%1219=556%:%
%:%1220=556%:%
%:%1221=557%:%
%:%1222=558%:%
%:%1223=558%:%
%:%1224=559%:%
%:%1225=560%:%
%:%1226=561%:%
%:%1227=561%:%
%:%1228=562%:%
%:%1229=563%:%
%:%1230=564%:%
%:%1231=564%:%
%:%1232=565%:%
%:%1233=566%:%
%:%1234=567%:%
%:%1237=568%:%
%:%1241=568%:%
%:%1242=568%:%
%:%1243=569%:%
%:%1244=569%:%
%:%1245=570%:%
%:%1246=570%:%
%:%1247=571%:%
%:%1248=571%:%
%:%1249=572%:%
%:%1250=572%:%
%:%1255=572%:%
%:%1258=573%:%
%:%1259=574%:%
%:%1260=574%:%
%:%1261=575%:%
%:%1262=576%:%
%:%1263=577%:%
%:%1266=578%:%
%:%1270=578%:%
%:%1271=578%:%
%:%1272=579%:%
%:%1273=579%:%
%:%1274=580%:%
%:%1275=580%:%
%:%1276=581%:%
%:%1277=581%:%
%:%1278=582%:%
%:%1279=582%:%
%:%1280=583%:%
%:%1281=583%:%
%:%1282=584%:%
%:%1283=584%:%
%:%1284=585%:%
%:%1285=585%:%
%:%1286=586%:%
%:%1287=586%:%
%:%1288=587%:%
%:%1289=587%:%
%:%1290=588%:%
%:%1291=588%:%
%:%1292=589%:%
%:%1293=589%:%
%:%1294=590%:%
%:%1295=590%:%
%:%1296=591%:%
%:%1297=591%:%
%:%1298=592%:%
%:%1299=592%:%
%:%1300=593%:%
%:%1301=593%:%
%:%1302=594%:%
%:%1303=594%:%
%:%1304=595%:%
%:%1305=595%:%
%:%1306=596%:%
%:%1307=596%:%
%:%1308=597%:%
%:%1309=597%:%
%:%1310=598%:%
%:%1311=598%:%
%:%1312=599%:%
%:%1313=599%:%
%:%1314=600%:%
%:%1315=600%:%
%:%1316=601%:%
%:%1317=601%:%
%:%1318=602%:%
%:%1319=602%:%
%:%1320=603%:%
%:%1321=603%:%
%:%1322=604%:%
%:%1323=604%:%
%:%1324=605%:%
%:%1325=605%:%
%:%1330=605%:%
%:%1333=606%:%
%:%1334=607%:%
%:%1335=607%:%
%:%1336=608%:%
%:%1337=609%:%
%:%1338=610%:%
%:%1339=610%:%
%:%1340=611%:%
%:%1344=615%:%
%:%1351=616%:%
%:%1352=616%:%
%:%1353=617%:%
%:%1354=617%:%
%:%1355=618%:%
%:%1356=618%:%
%:%1357=619%:%
%:%1358=620%:%
%:%1359=621%:%
%:%1360=621%:%
%:%1363=624%:%
%:%1364=625%:%
%:%1365=625%:%
%:%1366=626%:%
%:%1367=626%:%
%:%1368=627%:%
%:%1369=627%:%
%:%1370=628%:%
%:%1371=628%:%
%:%1372=629%:%
%:%1373=629%:%
%:%1374=630%:%
%:%1375=630%:%
%:%1376=631%:%
%:%1377=631%:%
%:%1378=632%:%
%:%1379=632%:%
%:%1380=633%:%
%:%1381=633%:%
%:%1382=634%:%
%:%1383=634%:%
%:%1384=635%:%
%:%1385=635%:%
%:%1386=636%:%
%:%1387=636%:%
%:%1388=637%:%
%:%1389=637%:%
%:%1390=638%:%
%:%1391=638%:%
%:%1392=639%:%
%:%1393=639%:%
%:%1394=640%:%
%:%1395=640%:%
%:%1396=641%:%
%:%1397=641%:%
%:%1398=642%:%
%:%1399=642%:%
%:%1400=643%:%
%:%1401=643%:%
%:%1402=644%:%
%:%1403=644%:%
%:%1404=644%:%
%:%1405=645%:%
%:%1406=645%:%
%:%1407=646%:%
%:%1408=646%:%
%:%1409=647%:%
%:%1410=647%:%
%:%1411=648%:%
%:%1412=648%:%
%:%1413=648%:%
%:%1414=649%:%
%:%1415=649%:%
%:%1416=650%:%
%:%1417=650%:%
%:%1418=651%:%
%:%1419=651%:%
%:%1420=652%:%
%:%1421=652%:%
%:%1422=653%:%
%:%1423=653%:%
%:%1424=653%:%
%:%1425=654%:%
%:%1426=654%:%
%:%1427=655%:%
%:%1428=655%:%
%:%1429=656%:%
%:%1430=656%:%
%:%1431=657%:%
%:%1437=657%:%
%:%1440=658%:%
%:%1441=659%:%
%:%1442=659%:%
%:%1443=660%:%
%:%1444=661%:%
%:%1445=661%:%
%:%1446=662%:%
%:%1447=663%:%
%:%1448=664%:%
%:%1451=665%:%
%:%1456=666%:%
%:%1457=666%:%
%:%1458=667%:%
%:%1459=667%:%
%:%1460=668%:%
%:%1461=668%:%
%:%1462=669%:%
%:%1463=669%:%
%:%1464=670%:%
%:%1465=670%:%
%:%1466=671%:%
%:%1467=672%:%
%:%1468=672%:%
%:%1469=673%:%
%:%1470=674%:%
%:%1471=674%:%
%:%1472=674%:%
%:%1473=675%:%
%:%1474=676%:%
%:%1475=676%:%
%:%1476=677%:%
%:%1477=677%:%
%:%1478=678%:%
%:%1479=678%:%
%:%1480=679%:%
%:%1481=679%:%
%:%1482=680%:%
%:%1483=680%:%
%:%1484=681%:%
%:%1485=681%:%
%:%1486=681%:%
%:%1487=682%:%
%:%1488=682%:%
%:%1489=683%:%
%:%1490=683%:%
%:%1491=684%:%
%:%1492=684%:%
%:%1493=685%:%
%:%1494=685%:%
%:%1495=685%:%
%:%1496=686%:%
%:%1497=686%:%
%:%1498=687%:%
%:%1499=687%:%
%:%1500=688%:%
%:%1501=688%:%
%:%1502=689%:%
%:%1503=689%:%
%:%1504=690%:%
%:%1505=690%:%
%:%1506=691%:%
%:%1507=691%:%
%:%1508=692%:%
%:%1509=692%:%
%:%1510=693%:%
%:%1511=693%:%
%:%1512=693%:%
%:%1513=694%:%
%:%1514=694%:%
%:%1515=695%:%
%:%1516=695%:%
%:%1517=696%:%
%:%1518=696%:%
%:%1519=696%:%
%:%1520=697%:%
%:%1521=697%:%
%:%1522=698%:%
%:%1523=698%:%
%:%1524=699%:%
%:%1525=699%:%
%:%1526=700%:%
%:%1527=700%:%
%:%1528=700%:%
%:%1529=701%:%
%:%1530=701%:%
%:%1531=702%:%
%:%1532=702%:%
%:%1533=703%:%
%:%1534=703%:%
%:%1535=704%:%
%:%1536=704%:%
%:%1537=704%:%
%:%1538=704%:%
%:%1539=705%:%
%:%1540=705%:%
%:%1541=706%:%
%:%1542=706%:%
%:%1543=706%:%
%:%1544=706%:%
%:%1545=706%:%
%:%1546=707%:%
%:%1552=707%:%
%:%1555=708%:%
%:%1556=709%:%
%:%1557=709%:%
%:%1558=710%:%
%:%1559=711%:%
%:%1560=712%:%
%:%1563=713%:%
%:%1568=714%:%
%:%1569=714%:%
%:%1570=715%:%
%:%1571=715%:%
%:%1572=715%:%
%:%1573=715%:%
%:%1574=716%:%
%:%1575=717%:%
%:%1576=717%:%
%:%1577=718%:%
%:%1578=718%:%
%:%1579=719%:%
%:%1580=719%:%
%:%1581=720%:%
%:%1582=720%:%
%:%1583=721%:%
%:%1584=721%:%
%:%1585=722%:%
%:%1586=722%:%
%:%1587=723%:%
%:%1588=723%:%
%:%1589=724%:%
%:%1590=724%:%
%:%1591=725%:%
%:%1592=725%:%
%:%1593=726%:%
%:%1594=726%:%
%:%1595=727%:%
%:%1596=727%:%
%:%1597=728%:%
%:%1598=728%:%
%:%1599=729%:%
%:%1600=729%:%
%:%1601=730%:%
%:%1602=730%:%
%:%1603=731%:%
%:%1604=731%:%
%:%1605=732%:%
%:%1606=732%:%
%:%1607=733%:%
%:%1608=733%:%
%:%1609=734%:%
%:%1610=734%:%
%:%1611=735%:%
%:%1612=735%:%
%:%1613=736%:%
%:%1614=736%:%
%:%1615=737%:%
%:%1616=737%:%
%:%1617=738%:%
%:%1618=738%:%
%:%1619=739%:%
%:%1620=739%:%
%:%1621=739%:%
%:%1622=739%:%
%:%1623=739%:%
%:%1624=740%:%
%:%1625=740%:%
%:%1626=740%:%
%:%1627=740%:%
%:%1628=740%:%
%:%1629=741%:%
%:%1630=741%:%
%:%1631=741%:%
%:%1632=741%:%
%:%1633=742%:%
%:%1639=742%:%
%:%1642=743%:%
%:%1643=744%:%
%:%1644=744%:%
%:%1645=745%:%
%:%1646=746%:%
%:%1647=747%:%
%:%1654=748%:%
%:%1655=748%:%
%:%1656=749%:%
%:%1657=749%:%
%:%1658=750%:%
%:%1659=750%:%
%:%1660=751%:%
%:%1661=751%:%
%:%1662=752%:%
%:%1663=752%:%
%:%1664=753%:%
%:%1665=753%:%
%:%1666=754%:%
%:%1667=754%:%
%:%1668=755%:%
%:%1669=755%:%
%:%1670=756%:%
%:%1671=756%:%
%:%1672=757%:%
%:%1673=758%:%
%:%1674=758%:%
%:%1675=759%:%
%:%1676=759%:%
%:%1677=760%:%
%:%1678=760%:%
%:%1679=761%:%
%:%1680=761%:%
%:%1681=762%:%
%:%1682=762%:%
%:%1683=763%:%
%:%1684=763%:%
%:%1685=764%:%
%:%1686=765%:%
%:%1687=765%:%
%:%1688=766%:%
%:%1689=766%:%
%:%1690=767%:%
%:%1691=767%:%
%:%1692=768%:%
%:%1693=768%:%
%:%1694=769%:%
%:%1695=769%:%
%:%1696=770%:%
%:%1697=770%:%
%:%1698=771%:%
%:%1699=771%:%
%:%1700=772%:%
%:%1701=773%:%
%:%1702=773%:%
%:%1703=773%:%
%:%1704=773%:%
%:%1705=774%:%
%:%1706=775%:%
%:%1707=775%:%
%:%1708=775%:%
%:%1709=775%:%
%:%1710=775%:%
%:%1711=776%:%
%:%1717=776%:%
%:%1720=777%:%
%:%1721=778%:%
%:%1722=778%:%
%:%1729=779%:%

%
\begin{isabellebody}%
\setisabellecontext{HS{\isacharunderscore}{\kern0pt}Theorems}%
%
\isadelimtheory
%
\endisadelimtheory
%
\isatagtheory
\isacommand{theory}\isamarkupfalse%
\ HS{\isacharunderscore}{\kern0pt}Theorems\isanewline
\ \ \isakeyword{imports}\ \isanewline
\ \ \ \ {\isachardoublequoteopen}Forcing{\isacharslash}{\kern0pt}Forcing{\isacharunderscore}{\kern0pt}Main{\isachardoublequoteclose}\ \isanewline
\ \ \ \ HS{\isacharunderscore}{\kern0pt}M\isanewline
\isakeyword{begin}%
\endisatagtheory
{\isafoldtheory}%
%
\isadelimtheory
\ \isanewline
%
\endisadelimtheory
\isanewline
\isacommand{context}\isamarkupfalse%
\ M{\isacharunderscore}{\kern0pt}symmetric{\isacharunderscore}{\kern0pt}system\isanewline
\isakeyword{begin}\isanewline
\ \ \ \ \isanewline
\isacommand{lemma}\isamarkupfalse%
\ check{\isacharunderscore}{\kern0pt}in{\isacharunderscore}{\kern0pt}HS\ {\isacharcolon}{\kern0pt}\ {\isachardoublequoteopen}x\ {\isasymin}\ M\ {\isasymLongrightarrow}\ check{\isacharparenleft}{\kern0pt}x{\isacharparenright}{\kern0pt}\ {\isasymin}\ HS{\isachardoublequoteclose}\ \isanewline
%
\isadelimproof
%
\endisadelimproof
%
\isatagproof
\isacommand{proof}\isamarkupfalse%
\ {\isacharminus}{\kern0pt}\ \isanewline
\ \ \isacommand{assume}\isamarkupfalse%
\ {\isachardoublequoteopen}x\ {\isasymin}\ M{\isachardoublequoteclose}\ \isanewline
\ \ \isacommand{have}\isamarkupfalse%
\ {\isachardoublequoteopen}{\isasymAnd}x{\isachardot}{\kern0pt}\ x\ {\isasymin}\ M\ {\isasymlongrightarrow}\ check{\isacharparenleft}{\kern0pt}x{\isacharparenright}{\kern0pt}\ {\isasymin}\ HS{\isachardoublequoteclose}\ \isanewline
\ \ \isacommand{proof}\isamarkupfalse%
{\isacharparenleft}{\kern0pt}rule{\isacharunderscore}{\kern0pt}tac\ P{\isacharequal}{\kern0pt}{\isachardoublequoteopen}{\isasymlambda}x{\isachardot}{\kern0pt}\ x\ {\isasymin}\ M\ {\isasymlongrightarrow}\ check{\isacharparenleft}{\kern0pt}x{\isacharparenright}{\kern0pt}\ {\isasymin}\ HS{\isachardoublequoteclose}\ \isakeyword{in}\ eps{\isacharunderscore}{\kern0pt}induct{\isacharcomma}{\kern0pt}\ rule\ impI{\isacharparenright}{\kern0pt}\ \isanewline
\ \ \ \ \isacommand{fix}\isamarkupfalse%
\ x\ \isacommand{assume}\isamarkupfalse%
\ assms\ {\isacharcolon}{\kern0pt}\ {\isachardoublequoteopen}x\ {\isasymin}\ M{\isachardoublequoteclose}\ {\isachardoublequoteopen}{\isasymforall}y\ {\isasymin}\ x{\isachardot}{\kern0pt}\ y\ {\isasymin}\ M\ {\isasymlongrightarrow}\ check{\isacharparenleft}{\kern0pt}y{\isacharparenright}{\kern0pt}\ {\isasymin}\ HS{\isachardoublequoteclose}\isanewline
\ \ \ \ \isacommand{thm}\isamarkupfalse%
\ HS{\isacharunderscore}{\kern0pt}iff\ \isanewline
\ \ \ \ \isacommand{have}\isamarkupfalse%
\ H{\isadigit{1}}{\isacharcolon}{\kern0pt}\ {\isachardoublequoteopen}check{\isacharparenleft}{\kern0pt}x{\isacharparenright}{\kern0pt}\ {\isasymin}\ P{\isacharunderscore}{\kern0pt}names{\isachardoublequoteclose}\ \ \isacommand{using}\isamarkupfalse%
\ check{\isacharunderscore}{\kern0pt}P{\isacharunderscore}{\kern0pt}name\ assms\ \isacommand{by}\isamarkupfalse%
\ auto\isanewline
\ \ \ \ \isacommand{have}\isamarkupfalse%
\ {\isachardoublequoteopen}{\isasymAnd}{\isasympi}{\isachardot}{\kern0pt}\ {\isasympi}\ {\isasymin}\ {\isasymG}\ {\isasymLongrightarrow}\ Pn{\isacharunderscore}{\kern0pt}auto{\isacharparenleft}{\kern0pt}{\isasympi}{\isacharparenright}{\kern0pt}{\isacharbackquote}{\kern0pt}check{\isacharparenleft}{\kern0pt}x{\isacharparenright}{\kern0pt}\ {\isacharequal}{\kern0pt}\ check{\isacharparenleft}{\kern0pt}x{\isacharparenright}{\kern0pt}{\isachardoublequoteclose}\ \isanewline
\ \ \ \ \ \ \isacommand{apply}\isamarkupfalse%
{\isacharparenleft}{\kern0pt}rule\ check{\isacharunderscore}{\kern0pt}fixpoint{\isacharparenright}{\kern0pt}\ \isanewline
\ \ \ \ \ \ \isacommand{using}\isamarkupfalse%
\ {\isasymG}{\isacharunderscore}{\kern0pt}P{\isacharunderscore}{\kern0pt}auto{\isacharunderscore}{\kern0pt}group\ assms\isanewline
\ \ \ \ \ \ \isacommand{unfolding}\isamarkupfalse%
\ is{\isacharunderscore}{\kern0pt}P{\isacharunderscore}{\kern0pt}auto{\isacharunderscore}{\kern0pt}group{\isacharunderscore}{\kern0pt}def\ \isanewline
\ \ \ \ \ \ \isacommand{by}\isamarkupfalse%
\ auto\isanewline
\ \ \ \ \isacommand{then}\isamarkupfalse%
\ \isacommand{have}\isamarkupfalse%
\ {\isachardoublequoteopen}sym{\isacharparenleft}{\kern0pt}check{\isacharparenleft}{\kern0pt}x{\isacharparenright}{\kern0pt}{\isacharparenright}{\kern0pt}\ {\isacharequal}{\kern0pt}\ {\isasymG}{\isachardoublequoteclose}\isanewline
\ \ \ \ \ \ \isacommand{using}\isamarkupfalse%
\ check{\isacharunderscore}{\kern0pt}fixpoint\ {\isasymG}{\isacharunderscore}{\kern0pt}P{\isacharunderscore}{\kern0pt}auto{\isacharunderscore}{\kern0pt}group\ \ \isanewline
\ \ \ \ \ \ \isacommand{unfolding}\isamarkupfalse%
\ is{\isacharunderscore}{\kern0pt}P{\isacharunderscore}{\kern0pt}auto{\isacharunderscore}{\kern0pt}group{\isacharunderscore}{\kern0pt}def\ sym{\isacharunderscore}{\kern0pt}def\ \isanewline
\ \ \ \ \ \ \isacommand{by}\isamarkupfalse%
\ auto\isanewline
\ \ \ \ \isacommand{then}\isamarkupfalse%
\ \isacommand{have}\isamarkupfalse%
\ H{\isadigit{2}}{\isacharcolon}{\kern0pt}\ {\isachardoublequoteopen}symmetric{\isacharparenleft}{\kern0pt}check{\isacharparenleft}{\kern0pt}x{\isacharparenright}{\kern0pt}{\isacharparenright}{\kern0pt}{\isachardoublequoteclose}\ \isanewline
\ \ \ \ \ \ \isacommand{unfolding}\isamarkupfalse%
\ symmetric{\isacharunderscore}{\kern0pt}def\ \isanewline
\ \ \ \ \ \ \isacommand{using}\isamarkupfalse%
\ {\isasymG}{\isacharunderscore}{\kern0pt}in{\isacharunderscore}{\kern0pt}{\isasymF}\isanewline
\ \ \ \ \ \ \isacommand{by}\isamarkupfalse%
\ auto\ \isanewline
\ \ \ \ \isacommand{have}\isamarkupfalse%
\ {\isachardoublequoteopen}domain{\isacharparenleft}{\kern0pt}check{\isacharparenleft}{\kern0pt}x{\isacharparenright}{\kern0pt}{\isacharparenright}{\kern0pt}\ {\isasymsubseteq}\ HS{\isachardoublequoteclose}\ \isanewline
\ \ \ \ \isacommand{proof}\isamarkupfalse%
{\isacharparenleft}{\kern0pt}rule\ subsetI{\isacharparenright}{\kern0pt}\isanewline
\ \ \ \ \ \ \isacommand{fix}\isamarkupfalse%
\ v\ \isacommand{assume}\isamarkupfalse%
\ assm{\isadigit{1}}{\isacharcolon}{\kern0pt}\ {\isachardoublequoteopen}v\ {\isasymin}\ domain{\isacharparenleft}{\kern0pt}check{\isacharparenleft}{\kern0pt}x{\isacharparenright}{\kern0pt}{\isacharparenright}{\kern0pt}{\isachardoublequoteclose}\ \isanewline
\ \ \ \ \ \ \isacommand{then}\isamarkupfalse%
\ \isacommand{obtain}\isamarkupfalse%
\ p\ \isakeyword{where}\ vpH{\isacharcolon}{\kern0pt}\ {\isachardoublequoteopen}{\isacharless}{\kern0pt}v{\isacharcomma}{\kern0pt}\ p{\isachargreater}{\kern0pt}\ {\isasymin}\ check{\isacharparenleft}{\kern0pt}x{\isacharparenright}{\kern0pt}{\isachardoublequoteclose}\ \isacommand{by}\isamarkupfalse%
\ auto\ \isanewline
\ \ \ \ \ \ \isacommand{have}\isamarkupfalse%
\ {\isachardoublequoteopen}check{\isacharparenleft}{\kern0pt}x{\isacharparenright}{\kern0pt}\ {\isacharequal}{\kern0pt}\ {\isacharbraceleft}{\kern0pt}\ {\isacharless}{\kern0pt}check{\isacharparenleft}{\kern0pt}y{\isacharparenright}{\kern0pt}{\isacharcomma}{\kern0pt}\ one{\isachargreater}{\kern0pt}\ {\isachardot}{\kern0pt}\ y\ {\isasymin}\ x\ {\isacharbraceright}{\kern0pt}{\isachardoublequoteclose}\ \isacommand{by}\isamarkupfalse%
{\isacharparenleft}{\kern0pt}rule\ def{\isacharunderscore}{\kern0pt}check{\isacharparenright}{\kern0pt}\ \isanewline
\ \ \ \ \ \ \isacommand{then}\isamarkupfalse%
\ \isacommand{obtain}\isamarkupfalse%
\ y\ \isakeyword{where}\ {\isachardoublequoteopen}v\ {\isacharequal}{\kern0pt}\ check{\isacharparenleft}{\kern0pt}y{\isacharparenright}{\kern0pt}{\isachardoublequoteclose}\ {\isachardoublequoteopen}y\ {\isasymin}\ x{\isachardoublequoteclose}\ \isacommand{using}\isamarkupfalse%
\ vpH\ \isacommand{by}\isamarkupfalse%
\ auto\ \isanewline
\ \ \ \ \ \ \isacommand{then}\isamarkupfalse%
\ \isacommand{show}\isamarkupfalse%
\ {\isachardoublequoteopen}v\ {\isasymin}\ HS{\isachardoublequoteclose}\ \isacommand{using}\isamarkupfalse%
\ assms\ transM\ \isacommand{by}\isamarkupfalse%
\ auto\isanewline
\ \ \ \ \isacommand{qed}\isamarkupfalse%
\isanewline
\isanewline
\ \ \ \ \isacommand{then}\isamarkupfalse%
\ \isacommand{show}\isamarkupfalse%
\ {\isachardoublequoteopen}check{\isacharparenleft}{\kern0pt}x{\isacharparenright}{\kern0pt}\ {\isasymin}\ HS{\isachardoublequoteclose}\ \isacommand{using}\isamarkupfalse%
\ HS{\isacharunderscore}{\kern0pt}iff\ H{\isadigit{1}}\ H{\isadigit{2}}\ \isacommand{by}\isamarkupfalse%
\ auto\isanewline
\ \ \isacommand{qed}\isamarkupfalse%
\isanewline
\ \ \isacommand{then}\isamarkupfalse%
\ \isacommand{show}\isamarkupfalse%
\ {\isacharquery}{\kern0pt}thesis\ \isacommand{using}\isamarkupfalse%
\ {\isacartoucheopen}x\ {\isasymin}\ M{\isacartoucheclose}\ \isacommand{by}\isamarkupfalse%
\ auto\isanewline
\isacommand{qed}\isamarkupfalse%
%
\endisatagproof
{\isafoldproof}%
%
\isadelimproof
\isanewline
%
\endisadelimproof
\isanewline
\isacommand{lemma}\isamarkupfalse%
\ comp{\isacharunderscore}{\kern0pt}in{\isacharunderscore}{\kern0pt}sym\ {\isacharcolon}{\kern0pt}\ {\isachardoublequoteopen}x\ {\isasymin}\ P{\isacharunderscore}{\kern0pt}names\ {\isasymLongrightarrow}\ {\isasympi}\ {\isasymin}\ sym{\isacharparenleft}{\kern0pt}x{\isacharparenright}{\kern0pt}\ {\isasymLongrightarrow}\ {\isasymtau}\ {\isasymin}\ sym{\isacharparenleft}{\kern0pt}x{\isacharparenright}{\kern0pt}\ {\isasymLongrightarrow}\ {\isasympi}\ O\ {\isasymtau}\ {\isasymin}\ sym{\isacharparenleft}{\kern0pt}x{\isacharparenright}{\kern0pt}{\isachardoublequoteclose}\ \isanewline
%
\isadelimproof
\ \ %
\endisadelimproof
%
\isatagproof
\isacommand{unfolding}\isamarkupfalse%
\ sym{\isacharunderscore}{\kern0pt}def\ \isanewline
\ \ \isacommand{apply}\isamarkupfalse%
\ simp\isanewline
\ \ \isacommand{apply}\isamarkupfalse%
{\isacharparenleft}{\kern0pt}rule\ conjI{\isacharparenright}{\kern0pt}\isanewline
\ \ \isacommand{using}\isamarkupfalse%
\ {\isasymG}{\isacharunderscore}{\kern0pt}P{\isacharunderscore}{\kern0pt}auto{\isacharunderscore}{\kern0pt}group\ is{\isacharunderscore}{\kern0pt}P{\isacharunderscore}{\kern0pt}auto{\isacharunderscore}{\kern0pt}group{\isacharunderscore}{\kern0pt}def\ \isanewline
\ \ \ \isacommand{apply}\isamarkupfalse%
\ force\isanewline
\ \ \isacommand{apply}\isamarkupfalse%
{\isacharparenleft}{\kern0pt}subst\ Pn{\isacharunderscore}{\kern0pt}auto{\isacharunderscore}{\kern0pt}comp{\isacharparenright}{\kern0pt}\isanewline
\ \ \isacommand{using}\isamarkupfalse%
\ {\isasymG}{\isacharunderscore}{\kern0pt}P{\isacharunderscore}{\kern0pt}auto{\isacharunderscore}{\kern0pt}group\ is{\isacharunderscore}{\kern0pt}P{\isacharunderscore}{\kern0pt}auto{\isacharunderscore}{\kern0pt}group{\isacharunderscore}{\kern0pt}def\ \isanewline
\ \ \ \ \isacommand{apply}\isamarkupfalse%
\ {\isacharparenleft}{\kern0pt}force{\isacharcomma}{\kern0pt}\ force{\isacharparenright}{\kern0pt}\isanewline
\ \ \isacommand{apply}\isamarkupfalse%
{\isacharparenleft}{\kern0pt}subst\ comp{\isacharunderscore}{\kern0pt}fun{\isacharunderscore}{\kern0pt}apply{\isacharparenright}{\kern0pt}\ \isanewline
\ \ \isacommand{apply}\isamarkupfalse%
{\isacharparenleft}{\kern0pt}rule\ Pn{\isacharunderscore}{\kern0pt}auto{\isacharunderscore}{\kern0pt}type{\isacharparenright}{\kern0pt}\isanewline
\ \ \isacommand{using}\isamarkupfalse%
\ {\isasymG}{\isacharunderscore}{\kern0pt}P{\isacharunderscore}{\kern0pt}auto{\isacharunderscore}{\kern0pt}group\ is{\isacharunderscore}{\kern0pt}P{\isacharunderscore}{\kern0pt}auto{\isacharunderscore}{\kern0pt}group{\isacharunderscore}{\kern0pt}def\ \isanewline
\ \ \ \ \isacommand{apply}\isamarkupfalse%
\ {\isacharparenleft}{\kern0pt}force{\isacharcomma}{\kern0pt}\ force{\isacharparenright}{\kern0pt}\isanewline
\ \ \isacommand{by}\isamarkupfalse%
\ auto%
\endisatagproof
{\isafoldproof}%
%
\isadelimproof
\isanewline
%
\endisadelimproof
\isanewline
\isacommand{lemma}\isamarkupfalse%
\ converse{\isacharunderscore}{\kern0pt}in{\isacharunderscore}{\kern0pt}sym\ {\isacharcolon}{\kern0pt}\ {\isachardoublequoteopen}x\ {\isasymin}\ P{\isacharunderscore}{\kern0pt}names\ {\isasymLongrightarrow}\ {\isasympi}\ {\isasymin}\ sym{\isacharparenleft}{\kern0pt}x{\isacharparenright}{\kern0pt}\ {\isasymLongrightarrow}\ converse{\isacharparenleft}{\kern0pt}{\isasympi}{\isacharparenright}{\kern0pt}\ {\isasymin}\ sym{\isacharparenleft}{\kern0pt}x{\isacharparenright}{\kern0pt}{\isachardoublequoteclose}\ \isanewline
%
\isadelimproof
%
\endisadelimproof
%
\isatagproof
\isacommand{proof}\isamarkupfalse%
\ {\isacharminus}{\kern0pt}\ \isanewline
\ \ \isacommand{assume}\isamarkupfalse%
\ assms\ {\isacharcolon}{\kern0pt}\ {\isachardoublequoteopen}x\ {\isasymin}\ P{\isacharunderscore}{\kern0pt}names{\isachardoublequoteclose}\ {\isachardoublequoteopen}{\isasympi}\ {\isasymin}\ sym{\isacharparenleft}{\kern0pt}x{\isacharparenright}{\kern0pt}{\isachardoublequoteclose}\ \isanewline
\isanewline
\ \ \isacommand{have}\isamarkupfalse%
\ {\isachardoublequoteopen}{\isacharless}{\kern0pt}x{\isacharcomma}{\kern0pt}\ Pn{\isacharunderscore}{\kern0pt}auto{\isacharparenleft}{\kern0pt}{\isasympi}{\isacharparenright}{\kern0pt}{\isacharbackquote}{\kern0pt}x{\isachargreater}{\kern0pt}\ {\isasymin}\ Pn{\isacharunderscore}{\kern0pt}auto{\isacharparenleft}{\kern0pt}{\isasympi}{\isacharparenright}{\kern0pt}{\isachardoublequoteclose}\ \isanewline
\ \ \ \ \isacommand{apply}\isamarkupfalse%
{\isacharparenleft}{\kern0pt}rule\ function{\isacharunderscore}{\kern0pt}apply{\isacharunderscore}{\kern0pt}Pair{\isacharparenright}{\kern0pt}\isanewline
\ \ \ \ \isacommand{using}\isamarkupfalse%
\ Pn{\isacharunderscore}{\kern0pt}auto{\isacharunderscore}{\kern0pt}function\ Pn{\isacharunderscore}{\kern0pt}auto{\isacharunderscore}{\kern0pt}domain\ assms\ \isanewline
\ \ \ \ \isacommand{by}\isamarkupfalse%
\ auto\ \isanewline
\ \ \isacommand{then}\isamarkupfalse%
\ \isacommand{have}\isamarkupfalse%
\ {\isachardoublequoteopen}{\isacharless}{\kern0pt}x{\isacharcomma}{\kern0pt}\ x{\isachargreater}{\kern0pt}\ {\isasymin}\ Pn{\isacharunderscore}{\kern0pt}auto{\isacharparenleft}{\kern0pt}{\isasympi}{\isacharparenright}{\kern0pt}{\isachardoublequoteclose}\ \isanewline
\ \ \ \ \isacommand{using}\isamarkupfalse%
\ assms\ \isanewline
\ \ \ \ \isacommand{unfolding}\isamarkupfalse%
\ sym{\isacharunderscore}{\kern0pt}def\ \isanewline
\ \ \ \ \isacommand{by}\isamarkupfalse%
\ auto\isanewline
\ \ \isacommand{then}\isamarkupfalse%
\ \isacommand{have}\isamarkupfalse%
\ {\isachardoublequoteopen}{\isacharless}{\kern0pt}x{\isacharcomma}{\kern0pt}\ x{\isachargreater}{\kern0pt}\ {\isasymin}\ converse{\isacharparenleft}{\kern0pt}Pn{\isacharunderscore}{\kern0pt}auto{\isacharparenleft}{\kern0pt}{\isasympi}{\isacharparenright}{\kern0pt}{\isacharparenright}{\kern0pt}{\isachardoublequoteclose}\ \isacommand{by}\isamarkupfalse%
\ auto\ \isanewline
\ \ \isacommand{then}\isamarkupfalse%
\ \isacommand{have}\isamarkupfalse%
\ {\isachardoublequoteopen}{\isacharless}{\kern0pt}x{\isacharcomma}{\kern0pt}\ x{\isachargreater}{\kern0pt}\ {\isasymin}\ Pn{\isacharunderscore}{\kern0pt}auto{\isacharparenleft}{\kern0pt}converse{\isacharparenleft}{\kern0pt}{\isasympi}{\isacharparenright}{\kern0pt}{\isacharparenright}{\kern0pt}{\isachardoublequoteclose}\ \isanewline
\ \ \ \ \isacommand{using}\isamarkupfalse%
\ Pn{\isacharunderscore}{\kern0pt}auto{\isacharunderscore}{\kern0pt}converse\ assms\ sym{\isacharunderscore}{\kern0pt}def\ {\isasymG}{\isacharunderscore}{\kern0pt}P{\isacharunderscore}{\kern0pt}auto{\isacharunderscore}{\kern0pt}group\ is{\isacharunderscore}{\kern0pt}P{\isacharunderscore}{\kern0pt}auto{\isacharunderscore}{\kern0pt}group{\isacharunderscore}{\kern0pt}def\ \isanewline
\ \ \ \ \isacommand{by}\isamarkupfalse%
\ auto\isanewline
\ \ \isacommand{then}\isamarkupfalse%
\ \isacommand{have}\isamarkupfalse%
\ H\ {\isacharcolon}{\kern0pt}\ {\isachardoublequoteopen}Pn{\isacharunderscore}{\kern0pt}auto{\isacharparenleft}{\kern0pt}converse{\isacharparenleft}{\kern0pt}{\isasympi}{\isacharparenright}{\kern0pt}{\isacharparenright}{\kern0pt}{\isacharbackquote}{\kern0pt}x\ {\isacharequal}{\kern0pt}\ x{\isachardoublequoteclose}\ \isanewline
\ \ \ \ \isacommand{apply}\isamarkupfalse%
{\isacharparenleft}{\kern0pt}rule\ function{\isacharunderscore}{\kern0pt}apply{\isacharunderscore}{\kern0pt}equality{\isacharparenright}{\kern0pt}\isanewline
\ \ \ \ \isacommand{by}\isamarkupfalse%
{\isacharparenleft}{\kern0pt}rule\ Pn{\isacharunderscore}{\kern0pt}auto{\isacharunderscore}{\kern0pt}function{\isacharparenright}{\kern0pt}\isanewline
\isanewline
\ \ \isacommand{have}\isamarkupfalse%
\ {\isachardoublequoteopen}{\isasympi}\ {\isasymin}\ {\isasymG}{\isachardoublequoteclose}\ \isanewline
\ \ \ \ \isacommand{using}\isamarkupfalse%
\ assms\isanewline
\ \ \ \ \isacommand{unfolding}\isamarkupfalse%
\ sym{\isacharunderscore}{\kern0pt}def\ \isanewline
\ \ \ \ \isacommand{by}\isamarkupfalse%
\ auto\ \isanewline
\ \ \isacommand{then}\isamarkupfalse%
\ \isacommand{have}\isamarkupfalse%
\ {\isachardoublequoteopen}converse{\isacharparenleft}{\kern0pt}{\isasympi}{\isacharparenright}{\kern0pt}\ {\isasymin}\ {\isasymG}{\isachardoublequoteclose}\ \isanewline
\ \ \ \ \isacommand{using}\isamarkupfalse%
\ {\isasymG}{\isacharunderscore}{\kern0pt}P{\isacharunderscore}{\kern0pt}auto{\isacharunderscore}{\kern0pt}group\ \isanewline
\ \ \ \ \isacommand{unfolding}\isamarkupfalse%
\ is{\isacharunderscore}{\kern0pt}P{\isacharunderscore}{\kern0pt}auto{\isacharunderscore}{\kern0pt}group{\isacharunderscore}{\kern0pt}def\ \isanewline
\ \ \ \ \isacommand{by}\isamarkupfalse%
\ auto\ \isanewline
\isanewline
\ \ \isacommand{then}\isamarkupfalse%
\ \isacommand{show}\isamarkupfalse%
\ {\isachardoublequoteopen}converse{\isacharparenleft}{\kern0pt}{\isasympi}{\isacharparenright}{\kern0pt}\ {\isasymin}\ sym{\isacharparenleft}{\kern0pt}x{\isacharparenright}{\kern0pt}{\isachardoublequoteclose}\ \isanewline
\ \ \ \ \isacommand{unfolding}\isamarkupfalse%
\ sym{\isacharunderscore}{\kern0pt}def\isanewline
\ \ \ \ \isacommand{using}\isamarkupfalse%
\ H\ \isanewline
\ \ \ \ \isacommand{by}\isamarkupfalse%
\ auto\isanewline
\isacommand{qed}\isamarkupfalse%
%
\endisatagproof
{\isafoldproof}%
%
\isadelimproof
\isanewline
%
\endisadelimproof
\isanewline
\isacommand{lemma}\isamarkupfalse%
\ sym{\isacharunderscore}{\kern0pt}P{\isacharunderscore}{\kern0pt}auto{\isacharunderscore}{\kern0pt}subgroup\ {\isacharcolon}{\kern0pt}\ {\isachardoublequoteopen}x\ {\isasymin}\ P{\isacharunderscore}{\kern0pt}names\ {\isasymLongrightarrow}\ sym{\isacharparenleft}{\kern0pt}x{\isacharparenright}{\kern0pt}\ {\isasymin}\ P{\isacharunderscore}{\kern0pt}auto{\isacharunderscore}{\kern0pt}subgroups{\isacharparenleft}{\kern0pt}{\isasymG}{\isacharparenright}{\kern0pt}{\isachardoublequoteclose}\ \isanewline
%
\isadelimproof
%
\endisadelimproof
%
\isatagproof
\isacommand{proof}\isamarkupfalse%
\ {\isacharminus}{\kern0pt}\ \isanewline
\ \ \isacommand{assume}\isamarkupfalse%
\ assm\ {\isacharcolon}{\kern0pt}\ {\isachardoublequoteopen}x\ {\isasymin}\ P{\isacharunderscore}{\kern0pt}names{\isachardoublequoteclose}\ \isanewline
\ \ \isacommand{thm}\isamarkupfalse%
\ P{\isacharunderscore}{\kern0pt}auto{\isacharunderscore}{\kern0pt}subgroups{\isacharunderscore}{\kern0pt}def\ is{\isacharunderscore}{\kern0pt}P{\isacharunderscore}{\kern0pt}auto{\isacharunderscore}{\kern0pt}group{\isacharunderscore}{\kern0pt}def\isanewline
\isanewline
\ \ \isacommand{have}\isamarkupfalse%
\ {\isachardoublequoteopen}sym{\isacharparenleft}{\kern0pt}x{\isacharparenright}{\kern0pt}\ {\isasymsubseteq}\ {\isacharbraceleft}{\kern0pt}\ {\isasympi}\ {\isasymin}\ P\ {\isasymrightarrow}\ P{\isachardot}{\kern0pt}\ is{\isacharunderscore}{\kern0pt}P{\isacharunderscore}{\kern0pt}auto{\isacharparenleft}{\kern0pt}{\isasympi}{\isacharparenright}{\kern0pt}\ {\isacharbraceright}{\kern0pt}{\isachardoublequoteclose}\ \isanewline
\ \ \ \ \isacommand{unfolding}\isamarkupfalse%
\ sym{\isacharunderscore}{\kern0pt}def\ \isanewline
\ \ \ \ \isacommand{using}\isamarkupfalse%
\ {\isasymG}{\isacharunderscore}{\kern0pt}P{\isacharunderscore}{\kern0pt}auto{\isacharunderscore}{\kern0pt}group\ is{\isacharunderscore}{\kern0pt}P{\isacharunderscore}{\kern0pt}auto{\isacharunderscore}{\kern0pt}group{\isacharunderscore}{\kern0pt}def\ \isanewline
\ \ \ \ \isacommand{by}\isamarkupfalse%
\ auto\isanewline
\isanewline
\ \ \isacommand{then}\isamarkupfalse%
\ \isacommand{show}\isamarkupfalse%
\ {\isachardoublequoteopen}sym{\isacharparenleft}{\kern0pt}x{\isacharparenright}{\kern0pt}\ {\isasymin}\ P{\isacharunderscore}{\kern0pt}auto{\isacharunderscore}{\kern0pt}subgroups{\isacharparenleft}{\kern0pt}{\isasymG}{\isacharparenright}{\kern0pt}{\isachardoublequoteclose}\ \isanewline
\ \ \ \ \isacommand{unfolding}\isamarkupfalse%
\ P{\isacharunderscore}{\kern0pt}auto{\isacharunderscore}{\kern0pt}subgroups{\isacharunderscore}{\kern0pt}def\ is{\isacharunderscore}{\kern0pt}P{\isacharunderscore}{\kern0pt}auto{\isacharunderscore}{\kern0pt}group{\isacharunderscore}{\kern0pt}def\isanewline
\ \ \ \ \isacommand{apply}\isamarkupfalse%
\ simp\ \isanewline
\ \ \ \ \isacommand{apply}\isamarkupfalse%
{\isacharparenleft}{\kern0pt}rule\ conjI{\isacharcomma}{\kern0pt}\ simp\ add{\isacharcolon}{\kern0pt}sym{\isacharunderscore}{\kern0pt}def{\isacharcomma}{\kern0pt}\ force{\isacharparenright}{\kern0pt}\isanewline
\ \ \ \ \isacommand{apply}\isamarkupfalse%
{\isacharparenleft}{\kern0pt}rule\ conjI{\isacharcomma}{\kern0pt}\ rule\ sym{\isacharunderscore}{\kern0pt}in{\isacharunderscore}{\kern0pt}M{\isacharcomma}{\kern0pt}\ simp\ add{\isacharcolon}{\kern0pt}assm{\isacharparenright}{\kern0pt}\isanewline
\ \ \ \ \isacommand{apply}\isamarkupfalse%
{\isacharparenleft}{\kern0pt}rule\ conjI{\isacharparenright}{\kern0pt}\isanewline
\ \ \ \ \isacommand{using}\isamarkupfalse%
\ comp{\isacharunderscore}{\kern0pt}in{\isacharunderscore}{\kern0pt}sym\ assm\ converse{\isacharunderscore}{\kern0pt}in{\isacharunderscore}{\kern0pt}sym\isanewline
\ \ \ \ \isacommand{by}\isamarkupfalse%
\ auto\isanewline
\isacommand{qed}\isamarkupfalse%
%
\endisatagproof
{\isafoldproof}%
%
\isadelimproof
\isanewline
%
\endisadelimproof
\isanewline
\isacommand{lemma}\isamarkupfalse%
\ HS{\isacharunderscore}{\kern0pt}Pn{\isacharunderscore}{\kern0pt}auto{\isacharunderscore}{\kern0pt}helper\ {\isacharcolon}{\kern0pt}\ {\isachardoublequoteopen}{\isasympi}\ {\isasymin}\ {\isasymG}\ {\isasymLongrightarrow}\ x\ {\isasymin}\ HS\ {\isasymLongrightarrow}\ Pn{\isacharunderscore}{\kern0pt}auto{\isacharparenleft}{\kern0pt}{\isasympi}{\isacharparenright}{\kern0pt}{\isacharbackquote}{\kern0pt}x\ {\isasymin}\ HS{\isachardoublequoteclose}\ \isanewline
%
\isadelimproof
%
\endisadelimproof
%
\isatagproof
\isacommand{proof}\isamarkupfalse%
\ {\isacharminus}{\kern0pt}\ \isanewline
\ \ \isacommand{assume}\isamarkupfalse%
\ assms\ {\isacharcolon}{\kern0pt}\ {\isachardoublequoteopen}{\isasympi}\ {\isasymin}\ {\isasymG}{\isachardoublequoteclose}\ {\isachardoublequoteopen}x\ {\isasymin}\ HS{\isachardoublequoteclose}\ \isanewline
\isanewline
\ \ \isacommand{have}\isamarkupfalse%
\ main\ {\isacharcolon}{\kern0pt}\ {\isachardoublequoteopen}x\ {\isasymin}\ HS\ {\isasymlongrightarrow}\ Pn{\isacharunderscore}{\kern0pt}auto{\isacharparenleft}{\kern0pt}{\isasympi}{\isacharparenright}{\kern0pt}{\isacharbackquote}{\kern0pt}x\ {\isasymin}\ HS{\isachardoublequoteclose}\ \isanewline
\ \ \isacommand{proof}\isamarkupfalse%
\ {\isacharparenleft}{\kern0pt}rule{\isacharunderscore}{\kern0pt}tac\ Q{\isacharequal}{\kern0pt}{\isachardoublequoteopen}{\isasymlambda}x{\isachardot}{\kern0pt}\ x\ {\isasymin}\ HS\ {\isasymlongrightarrow}\ Pn{\isacharunderscore}{\kern0pt}auto{\isacharparenleft}{\kern0pt}{\isasympi}{\isacharparenright}{\kern0pt}{\isacharbackquote}{\kern0pt}x\ {\isasymin}\ HS{\isachardoublequoteclose}\ \isakeyword{in}\ ed{\isacharunderscore}{\kern0pt}induction{\isacharcomma}{\kern0pt}\ rule\ impI{\isacharparenright}{\kern0pt}\isanewline
\ \ \ \ \isacommand{fix}\isamarkupfalse%
\ x\ \isacommand{assume}\isamarkupfalse%
\ assms{\isadigit{1}}\ {\isacharcolon}{\kern0pt}\ {\isachardoublequoteopen}{\isacharparenleft}{\kern0pt}{\isasymAnd}y{\isachardot}{\kern0pt}\ ed{\isacharparenleft}{\kern0pt}y{\isacharcomma}{\kern0pt}\ x{\isacharparenright}{\kern0pt}\ {\isasymLongrightarrow}\ y\ {\isasymin}\ HS\ {\isasymlongrightarrow}\ Pn{\isacharunderscore}{\kern0pt}auto{\isacharparenleft}{\kern0pt}{\isasympi}{\isacharparenright}{\kern0pt}\ {\isacharbackquote}{\kern0pt}\ y\ {\isasymin}\ HS{\isacharparenright}{\kern0pt}{\isachardoublequoteclose}\ {\isachardoublequoteopen}x\ {\isasymin}\ HS{\isachardoublequoteclose}\ \isanewline
\isanewline
\ \ \ \ \isacommand{have}\isamarkupfalse%
\ pi{\isacharunderscore}{\kern0pt}pauto\ {\isacharcolon}{\kern0pt}\ {\isachardoublequoteopen}{\isasympi}\ {\isasymin}\ P{\isacharunderscore}{\kern0pt}auto{\isachardoublequoteclose}\ \isanewline
\ \ \ \ \ \ \isacommand{using}\isamarkupfalse%
\ assms\ {\isasymG}{\isacharunderscore}{\kern0pt}P{\isacharunderscore}{\kern0pt}auto{\isacharunderscore}{\kern0pt}group\ is{\isacharunderscore}{\kern0pt}P{\isacharunderscore}{\kern0pt}auto{\isacharunderscore}{\kern0pt}group{\isacharunderscore}{\kern0pt}def\ P{\isacharunderscore}{\kern0pt}auto{\isacharunderscore}{\kern0pt}def\ \isacommand{by}\isamarkupfalse%
\ auto\ \isanewline
\isanewline
\ \ \ \ \isacommand{have}\isamarkupfalse%
\ {\isachardoublequoteopen}Pn{\isacharunderscore}{\kern0pt}auto{\isacharparenleft}{\kern0pt}{\isasympi}{\isacharparenright}{\kern0pt}{\isacharbackquote}{\kern0pt}x\ {\isasymin}\ P{\isacharunderscore}{\kern0pt}names{\isachardoublequoteclose}\ \isacommand{using}\isamarkupfalse%
\ Pn{\isacharunderscore}{\kern0pt}auto{\isacharunderscore}{\kern0pt}value\ assms\ P{\isacharunderscore}{\kern0pt}auto{\isacharunderscore}{\kern0pt}def\ assms{\isadigit{1}}\ HS{\isacharunderscore}{\kern0pt}iff\ pi{\isacharunderscore}{\kern0pt}pauto\ \isacommand{by}\isamarkupfalse%
\ auto\isanewline
\isanewline
\ \ \ \ \isacommand{have}\isamarkupfalse%
\ eq{\isacharcolon}{\kern0pt}{\isachardoublequoteopen}Pn{\isacharunderscore}{\kern0pt}auto{\isacharparenleft}{\kern0pt}{\isasympi}{\isacharparenright}{\kern0pt}{\isacharbackquote}{\kern0pt}x\ {\isacharequal}{\kern0pt}\ {\isacharbraceleft}{\kern0pt}\ {\isacharless}{\kern0pt}Pn{\isacharunderscore}{\kern0pt}auto{\isacharparenleft}{\kern0pt}{\isasympi}{\isacharparenright}{\kern0pt}{\isacharbackquote}{\kern0pt}y{\isacharcomma}{\kern0pt}\ {\isasympi}{\isacharbackquote}{\kern0pt}p{\isachargreater}{\kern0pt}\ {\isachardot}{\kern0pt}\ {\isacharless}{\kern0pt}y{\isacharcomma}{\kern0pt}p{\isachargreater}{\kern0pt}\ {\isasymin}\ x\ {\isacharbraceright}{\kern0pt}{\isachardoublequoteclose}\ \isacommand{using}\isamarkupfalse%
\ Pn{\isacharunderscore}{\kern0pt}auto\ assms\ P{\isacharunderscore}{\kern0pt}auto{\isacharunderscore}{\kern0pt}def\ assms{\isadigit{1}}\ HS{\isacharunderscore}{\kern0pt}iff\ \isacommand{by}\isamarkupfalse%
\ auto\isanewline
\ \ \ \ \isacommand{have}\isamarkupfalse%
\ domsubset\ {\isacharcolon}{\kern0pt}\ {\isachardoublequoteopen}domain{\isacharparenleft}{\kern0pt}Pn{\isacharunderscore}{\kern0pt}auto{\isacharparenleft}{\kern0pt}{\isasympi}{\isacharparenright}{\kern0pt}{\isacharbackquote}{\kern0pt}x{\isacharparenright}{\kern0pt}\ {\isasymsubseteq}\ HS{\isachardoublequoteclose}\ \isanewline
\ \ \ \ \isacommand{proof}\isamarkupfalse%
\ {\isacharparenleft}{\kern0pt}rule\ subsetI{\isacharparenright}{\kern0pt}\isanewline
\ \ \ \ \ \ \isacommand{fix}\isamarkupfalse%
\ v\ \isacommand{assume}\isamarkupfalse%
\ vin\ {\isacharcolon}{\kern0pt}\ {\isachardoublequoteopen}v\ {\isasymin}\ domain{\isacharparenleft}{\kern0pt}Pn{\isacharunderscore}{\kern0pt}auto{\isacharparenleft}{\kern0pt}{\isasympi}{\isacharparenright}{\kern0pt}{\isacharbackquote}{\kern0pt}x{\isacharparenright}{\kern0pt}{\isachardoublequoteclose}\ \isanewline
\ \ \ \ \ \ \isacommand{then}\isamarkupfalse%
\ \isacommand{obtain}\isamarkupfalse%
\ q\ \isakeyword{where}\ qH{\isacharcolon}{\kern0pt}\ {\isachardoublequoteopen}{\isacharless}{\kern0pt}v{\isacharcomma}{\kern0pt}\ q{\isachargreater}{\kern0pt}\ {\isasymin}\ Pn{\isacharunderscore}{\kern0pt}auto{\isacharparenleft}{\kern0pt}{\isasympi}{\isacharparenright}{\kern0pt}{\isacharbackquote}{\kern0pt}x{\isachardoublequoteclose}\ \isacommand{by}\isamarkupfalse%
\ auto\isanewline
\ \ \ \ \ \ \isacommand{have}\isamarkupfalse%
\ {\isachardoublequoteopen}{\isasymexists}y{\isachardot}{\kern0pt}\ {\isasymexists}p{\isachardot}{\kern0pt}\ {\isacharless}{\kern0pt}y{\isacharcomma}{\kern0pt}\ p{\isachargreater}{\kern0pt}\ {\isasymin}\ x\ {\isasymand}\ {\isacharless}{\kern0pt}v{\isacharcomma}{\kern0pt}\ q{\isachargreater}{\kern0pt}\ {\isacharequal}{\kern0pt}\ {\isacharless}{\kern0pt}Pn{\isacharunderscore}{\kern0pt}auto{\isacharparenleft}{\kern0pt}{\isasympi}{\isacharparenright}{\kern0pt}{\isacharbackquote}{\kern0pt}y{\isacharcomma}{\kern0pt}\ {\isasympi}{\isacharbackquote}{\kern0pt}p{\isachargreater}{\kern0pt}{\isachardoublequoteclose}\ \isanewline
\ \ \ \ \ \ \ \ \isacommand{apply}\isamarkupfalse%
{\isacharparenleft}{\kern0pt}rule\ pair{\isacharunderscore}{\kern0pt}rel{\isacharunderscore}{\kern0pt}arg{\isacharparenright}{\kern0pt}\isanewline
\ \ \ \ \ \ \ \ \isacommand{using}\isamarkupfalse%
\ assms{\isadigit{1}}\ HS{\isacharunderscore}{\kern0pt}iff\ relation{\isacharunderscore}{\kern0pt}P{\isacharunderscore}{\kern0pt}name\ eq\ qH\ \isanewline
\ \ \ \ \ \ \ \ \isacommand{by}\isamarkupfalse%
\ auto\isanewline
\ \ \ \ \ \ \isacommand{then}\isamarkupfalse%
\ \isacommand{obtain}\isamarkupfalse%
\ y\ \isakeyword{where}\ yH{\isacharcolon}{\kern0pt}\ {\isachardoublequoteopen}y\ {\isasymin}\ domain{\isacharparenleft}{\kern0pt}x{\isacharparenright}{\kern0pt}{\isachardoublequoteclose}\ {\isachardoublequoteopen}v\ {\isacharequal}{\kern0pt}\ Pn{\isacharunderscore}{\kern0pt}auto{\isacharparenleft}{\kern0pt}{\isasympi}{\isacharparenright}{\kern0pt}{\isacharbackquote}{\kern0pt}y{\isachardoublequoteclose}\ \isacommand{by}\isamarkupfalse%
\ auto\isanewline
\ \ \ \ \ \ \isacommand{then}\isamarkupfalse%
\ \isacommand{have}\isamarkupfalse%
\ {\isachardoublequoteopen}y\ {\isasymin}\ HS{\isachardoublequoteclose}\ \isacommand{using}\isamarkupfalse%
\ assms{\isadigit{1}}\ HS{\isacharunderscore}{\kern0pt}iff\ \isacommand{by}\isamarkupfalse%
\ blast\isanewline
\ \ \ \ \ \ \isacommand{then}\isamarkupfalse%
\ \isacommand{show}\isamarkupfalse%
\ {\isachardoublequoteopen}v\ {\isasymin}\ HS{\isachardoublequoteclose}\ \isacommand{using}\isamarkupfalse%
\ yH\ assms{\isadigit{1}}\ ed{\isacharunderscore}{\kern0pt}def\ \isacommand{by}\isamarkupfalse%
\ auto\isanewline
\ \ \ \ \isacommand{qed}\isamarkupfalse%
\isanewline
\isanewline
\ \ \ \ \isacommand{then}\isamarkupfalse%
\ \isacommand{have}\isamarkupfalse%
\ H{\isadigit{1}}{\isacharcolon}{\kern0pt}\ {\isachardoublequoteopen}{\isacharbraceleft}{\kern0pt}\ {\isasympi}\ O\ {\isasymtau}\ O\ converse{\isacharparenleft}{\kern0pt}{\isasympi}{\isacharparenright}{\kern0pt}{\isachardot}{\kern0pt}\ {\isasymtau}\ {\isasymin}\ sym{\isacharparenleft}{\kern0pt}x{\isacharparenright}{\kern0pt}\ {\isacharbraceright}{\kern0pt}\ {\isasymin}\ {\isasymF}{\isachardoublequoteclose}\ \isanewline
\ \ \ \ \ \ \isacommand{using}\isamarkupfalse%
\ assms{\isadigit{1}}\ HS{\isacharunderscore}{\kern0pt}iff\ symmetric{\isacharunderscore}{\kern0pt}def\ {\isasymF}{\isacharunderscore}{\kern0pt}normal\ assms\ \isacommand{by}\isamarkupfalse%
\ auto\ \isanewline
\isanewline
\ \ \ \ \isacommand{have}\isamarkupfalse%
\ H{\isadigit{2}}\ {\isacharcolon}{\kern0pt}\ {\isachardoublequoteopen}{\isacharbraceleft}{\kern0pt}\ {\isasympi}\ O\ {\isasymtau}\ O\ converse{\isacharparenleft}{\kern0pt}{\isasympi}{\isacharparenright}{\kern0pt}{\isachardot}{\kern0pt}\ {\isasymtau}\ {\isasymin}\ sym{\isacharparenleft}{\kern0pt}x{\isacharparenright}{\kern0pt}\ {\isacharbraceright}{\kern0pt}\ {\isasymsubseteq}\ sym{\isacharparenleft}{\kern0pt}Pn{\isacharunderscore}{\kern0pt}auto{\isacharparenleft}{\kern0pt}{\isasympi}{\isacharparenright}{\kern0pt}{\isacharbackquote}{\kern0pt}x{\isacharparenright}{\kern0pt}{\isachardoublequoteclose}\ \isanewline
\ \ \ \ \isacommand{proof}\isamarkupfalse%
\ {\isacharparenleft}{\kern0pt}rule\ subsetI{\isacharparenright}{\kern0pt}\ \isanewline
\ \ \ \ \ \ \isacommand{fix}\isamarkupfalse%
\ {\isasymsigma}\ \isacommand{assume}\isamarkupfalse%
\ assm\ {\isacharcolon}{\kern0pt}\ {\isachardoublequoteopen}{\isasymsigma}\ {\isasymin}\ {\isacharbraceleft}{\kern0pt}{\isasympi}\ O\ {\isasymtau}\ O\ converse{\isacharparenleft}{\kern0pt}{\isasympi}{\isacharparenright}{\kern0pt}\ {\isachardot}{\kern0pt}\ {\isasymtau}\ {\isasymin}\ sym{\isacharparenleft}{\kern0pt}x{\isacharparenright}{\kern0pt}{\isacharbraceright}{\kern0pt}{\isachardoublequoteclose}\isanewline
\isanewline
\ \ \ \ \ \ \isacommand{have}\isamarkupfalse%
\ sigmain\ {\isacharcolon}{\kern0pt}\ {\isachardoublequoteopen}{\isasymsigma}\ {\isasymin}\ {\isasymG}{\isachardoublequoteclose}\ \isacommand{using}\isamarkupfalse%
\ {\isasymF}{\isacharunderscore}{\kern0pt}subset\ P{\isacharunderscore}{\kern0pt}auto{\isacharunderscore}{\kern0pt}subgroups{\isacharunderscore}{\kern0pt}def\ H{\isadigit{1}}\ assm\ \isacommand{by}\isamarkupfalse%
\ auto\isanewline
\isanewline
\ \ \ \ \ \ \isacommand{obtain}\isamarkupfalse%
\ {\isasymtau}\ \isakeyword{where}\ tauH\ {\isacharcolon}{\kern0pt}\ {\isachardoublequoteopen}{\isasymsigma}\ {\isacharequal}{\kern0pt}\ {\isasympi}\ O\ {\isasymtau}\ O\ converse{\isacharparenleft}{\kern0pt}{\isasympi}{\isacharparenright}{\kern0pt}{\isachardoublequoteclose}\ {\isachardoublequoteopen}{\isasymtau}\ {\isasymin}\ sym{\isacharparenleft}{\kern0pt}x{\isacharparenright}{\kern0pt}{\isachardoublequoteclose}\ \isacommand{using}\isamarkupfalse%
\ assm\ \isacommand{by}\isamarkupfalse%
\ auto\ \isanewline
\isanewline
\ \ \ \ \ \ \isacommand{then}\isamarkupfalse%
\ \isacommand{have}\isamarkupfalse%
\ {\isachardoublequoteopen}Pn{\isacharunderscore}{\kern0pt}auto{\isacharparenleft}{\kern0pt}{\isasymsigma}{\isacharparenright}{\kern0pt}\ {\isacharbackquote}{\kern0pt}\ {\isacharparenleft}{\kern0pt}Pn{\isacharunderscore}{\kern0pt}auto{\isacharparenleft}{\kern0pt}{\isasympi}{\isacharparenright}{\kern0pt}{\isacharbackquote}{\kern0pt}x{\isacharparenright}{\kern0pt}\ {\isacharequal}{\kern0pt}\ Pn{\isacharunderscore}{\kern0pt}auto{\isacharparenleft}{\kern0pt}{\isasympi}\ O\ {\isasymtau}\ O\ converse{\isacharparenleft}{\kern0pt}{\isasympi}{\isacharparenright}{\kern0pt}{\isacharparenright}{\kern0pt}\ {\isacharbackquote}{\kern0pt}\ {\isacharparenleft}{\kern0pt}Pn{\isacharunderscore}{\kern0pt}auto{\isacharparenleft}{\kern0pt}{\isasympi}{\isacharparenright}{\kern0pt}{\isacharbackquote}{\kern0pt}x{\isacharparenright}{\kern0pt}{\isachardoublequoteclose}\ \isacommand{by}\isamarkupfalse%
\ auto\ \isanewline
\ \ \ \ \ \ \isacommand{also}\isamarkupfalse%
\ \isacommand{have}\isamarkupfalse%
\ {\isachardoublequoteopen}{\isachardot}{\kern0pt}{\isachardot}{\kern0pt}{\isachardot}{\kern0pt}\ {\isacharequal}{\kern0pt}\ {\isacharparenleft}{\kern0pt}Pn{\isacharunderscore}{\kern0pt}auto{\isacharparenleft}{\kern0pt}{\isasympi}\ O\ {\isasymtau}\ O\ converse{\isacharparenleft}{\kern0pt}{\isasympi}{\isacharparenright}{\kern0pt}{\isacharparenright}{\kern0pt}\ O\ Pn{\isacharunderscore}{\kern0pt}auto{\isacharparenleft}{\kern0pt}{\isasympi}{\isacharparenright}{\kern0pt}{\isacharparenright}{\kern0pt}{\isacharbackquote}{\kern0pt}x{\isachardoublequoteclose}\ \isanewline
\ \ \ \ \ \ \ \ \isacommand{apply}\isamarkupfalse%
{\isacharparenleft}{\kern0pt}rule\ eq{\isacharunderscore}{\kern0pt}flip{\isacharcomma}{\kern0pt}\ rule{\isacharunderscore}{\kern0pt}tac\ A{\isacharequal}{\kern0pt}P{\isacharunderscore}{\kern0pt}names\ \isakeyword{in}\ comp{\isacharunderscore}{\kern0pt}fun{\isacharunderscore}{\kern0pt}apply{\isacharparenright}{\kern0pt}\isanewline
\ \ \ \ \ \ \ \ \isacommand{using}\isamarkupfalse%
\ Pn{\isacharunderscore}{\kern0pt}auto{\isacharunderscore}{\kern0pt}type\ assms\ P{\isacharunderscore}{\kern0pt}auto{\isacharunderscore}{\kern0pt}def\ assms{\isadigit{1}}\ HS{\isacharunderscore}{\kern0pt}iff\ pi{\isacharunderscore}{\kern0pt}pauto\isanewline
\ \ \ \ \ \ \ \ \isacommand{by}\isamarkupfalse%
\ auto\isanewline
\ \ \ \ \ \ \isacommand{also}\isamarkupfalse%
\ \isacommand{have}\isamarkupfalse%
\ {\isachardoublequoteopen}{\isachardot}{\kern0pt}{\isachardot}{\kern0pt}{\isachardot}{\kern0pt}\ {\isacharequal}{\kern0pt}\ Pn{\isacharunderscore}{\kern0pt}auto{\isacharparenleft}{\kern0pt}{\isacharparenleft}{\kern0pt}{\isasympi}\ O\ {\isasymtau}\ O\ converse{\isacharparenleft}{\kern0pt}{\isasympi}{\isacharparenright}{\kern0pt}{\isacharparenright}{\kern0pt}\ O\ {\isasympi}{\isacharparenright}{\kern0pt}{\isacharbackquote}{\kern0pt}x{\isachardoublequoteclose}\ \isanewline
\ \ \ \ \ \ \ \ \isacommand{apply}\isamarkupfalse%
{\isacharparenleft}{\kern0pt}rule\ eq{\isacharunderscore}{\kern0pt}flip{\isacharparenright}{\kern0pt}\isanewline
\ \ \ \ \ \ \ \ \isacommand{apply}\isamarkupfalse%
{\isacharparenleft}{\kern0pt}subst\ Pn{\isacharunderscore}{\kern0pt}auto{\isacharunderscore}{\kern0pt}comp{\isacharparenright}{\kern0pt}\isanewline
\ \ \ \ \ \ \ \ \isacommand{apply}\isamarkupfalse%
{\isacharparenleft}{\kern0pt}rule\ P{\isacharunderscore}{\kern0pt}auto{\isacharunderscore}{\kern0pt}comp{\isacharparenright}{\kern0pt}\isanewline
\ \ \ \ \ \ \ \ \isacommand{using}\isamarkupfalse%
\ assms\ P{\isacharunderscore}{\kern0pt}auto{\isacharunderscore}{\kern0pt}def\ pi{\isacharunderscore}{\kern0pt}pauto\isanewline
\ \ \ \ \ \ \ \ \ \ \ \isacommand{apply}\isamarkupfalse%
\ simp\isanewline
\ \ \ \ \ \ \ \ \ \ \isacommand{apply}\isamarkupfalse%
{\isacharparenleft}{\kern0pt}rule\ P{\isacharunderscore}{\kern0pt}auto{\isacharunderscore}{\kern0pt}comp{\isacharparenright}{\kern0pt}\isanewline
\ \ \ \ \ \ \ \ \isacommand{using}\isamarkupfalse%
\ tauH\ {\isasymG}{\isacharunderscore}{\kern0pt}P{\isacharunderscore}{\kern0pt}auto{\isacharunderscore}{\kern0pt}group\ \isanewline
\ \ \ \ \ \ \ \ \isacommand{unfolding}\isamarkupfalse%
\ sym{\isacharunderscore}{\kern0pt}def\ is{\isacharunderscore}{\kern0pt}P{\isacharunderscore}{\kern0pt}auto{\isacharunderscore}{\kern0pt}group{\isacharunderscore}{\kern0pt}def\isanewline
\ \ \ \ \ \ \ \ \ \ \ \isacommand{apply}\isamarkupfalse%
\ force\isanewline
\ \ \ \ \ \ \ \ \ \ \isacommand{apply}\isamarkupfalse%
{\isacharparenleft}{\kern0pt}rule\ P{\isacharunderscore}{\kern0pt}auto{\isacharunderscore}{\kern0pt}converse{\isacharparenright}{\kern0pt}\isanewline
\ \ \ \ \ \ \ \ \isacommand{using}\isamarkupfalse%
\ assms\ P{\isacharunderscore}{\kern0pt}auto{\isacharunderscore}{\kern0pt}def\ pi{\isacharunderscore}{\kern0pt}pauto\isanewline
\ \ \ \ \ \ \ \ \isacommand{by}\isamarkupfalse%
\ auto\isanewline
\ \ \ \ \ \ \isacommand{also}\isamarkupfalse%
\ \isacommand{have}\isamarkupfalse%
\ {\isachardoublequoteopen}{\isachardot}{\kern0pt}{\isachardot}{\kern0pt}{\isachardot}{\kern0pt}\ {\isacharequal}{\kern0pt}\ Pn{\isacharunderscore}{\kern0pt}auto{\isacharparenleft}{\kern0pt}{\isasympi}\ O\ {\isasymtau}\ O\ {\isacharparenleft}{\kern0pt}converse{\isacharparenleft}{\kern0pt}{\isasympi}{\isacharparenright}{\kern0pt}\ O\ {\isasympi}{\isacharparenright}{\kern0pt}{\isacharparenright}{\kern0pt}{\isacharbackquote}{\kern0pt}x{\isachardoublequoteclose}\ \isanewline
\ \ \ \ \ \ \ \ \isacommand{using}\isamarkupfalse%
\ comp{\isacharunderscore}{\kern0pt}assoc\ \isacommand{by}\isamarkupfalse%
\ auto\isanewline
\ \ \ \ \ \ \isacommand{also}\isamarkupfalse%
\ \isacommand{have}\isamarkupfalse%
\ {\isachardoublequoteopen}{\isachardot}{\kern0pt}{\isachardot}{\kern0pt}{\isachardot}{\kern0pt}\ {\isacharequal}{\kern0pt}\ Pn{\isacharunderscore}{\kern0pt}auto{\isacharparenleft}{\kern0pt}{\isasympi}\ O\ {\isacharparenleft}{\kern0pt}{\isasymtau}\ O\ id{\isacharparenleft}{\kern0pt}P{\isacharparenright}{\kern0pt}{\isacharparenright}{\kern0pt}{\isacharparenright}{\kern0pt}{\isacharbackquote}{\kern0pt}x{\isachardoublequoteclose}\isanewline
\ \ \ \ \ \ \ \ \isacommand{apply}\isamarkupfalse%
{\isacharparenleft}{\kern0pt}subst\ left{\isacharunderscore}{\kern0pt}comp{\isacharunderscore}{\kern0pt}inverse{\isacharcomma}{\kern0pt}\ rule\ bij{\isacharunderscore}{\kern0pt}is{\isacharunderscore}{\kern0pt}inj{\isacharparenright}{\kern0pt}\isanewline
\ \ \ \ \ \ \ \ \isacommand{using}\isamarkupfalse%
\ assms\ P{\isacharunderscore}{\kern0pt}auto{\isacharunderscore}{\kern0pt}def\ is{\isacharunderscore}{\kern0pt}P{\isacharunderscore}{\kern0pt}auto{\isacharunderscore}{\kern0pt}def\ pi{\isacharunderscore}{\kern0pt}pauto\isanewline
\ \ \ \ \ \ \ \ \isacommand{by}\isamarkupfalse%
\ auto\isanewline
\ \ \ \ \ \ \isacommand{also}\isamarkupfalse%
\ \isacommand{have}\isamarkupfalse%
\ {\isachardoublequoteopen}{\isachardot}{\kern0pt}{\isachardot}{\kern0pt}{\isachardot}{\kern0pt}\ {\isacharequal}{\kern0pt}\ Pn{\isacharunderscore}{\kern0pt}auto{\isacharparenleft}{\kern0pt}{\isasympi}\ O\ {\isasymtau}{\isacharparenright}{\kern0pt}{\isacharbackquote}{\kern0pt}x{\isachardoublequoteclose}\ \isanewline
\ \ \ \ \ \ \ \ \isacommand{apply}\isamarkupfalse%
{\isacharparenleft}{\kern0pt}subst\ right{\isacharunderscore}{\kern0pt}comp{\isacharunderscore}{\kern0pt}id{\isacharparenright}{\kern0pt}\isanewline
\ \ \ \ \ \ \ \ \isacommand{using}\isamarkupfalse%
\ tauH\ sym{\isacharunderscore}{\kern0pt}def\ {\isasymG}{\isacharunderscore}{\kern0pt}P{\isacharunderscore}{\kern0pt}auto{\isacharunderscore}{\kern0pt}group\ is{\isacharunderscore}{\kern0pt}P{\isacharunderscore}{\kern0pt}auto{\isacharunderscore}{\kern0pt}group{\isacharunderscore}{\kern0pt}def\ is{\isacharunderscore}{\kern0pt}P{\isacharunderscore}{\kern0pt}auto{\isacharunderscore}{\kern0pt}def\ bij{\isacharunderscore}{\kern0pt}def\ inj{\isacharunderscore}{\kern0pt}def\ Pi{\isacharunderscore}{\kern0pt}def\ \isanewline
\ \ \ \ \ \ \ \ \isacommand{by}\isamarkupfalse%
\ auto\isanewline
\ \ \ \ \ \ \isacommand{also}\isamarkupfalse%
\ \isacommand{have}\isamarkupfalse%
\ {\isachardoublequoteopen}{\isachardot}{\kern0pt}{\isachardot}{\kern0pt}{\isachardot}{\kern0pt}\ {\isacharequal}{\kern0pt}\ {\isacharparenleft}{\kern0pt}Pn{\isacharunderscore}{\kern0pt}auto{\isacharparenleft}{\kern0pt}{\isasympi}{\isacharparenright}{\kern0pt}\ O\ Pn{\isacharunderscore}{\kern0pt}auto{\isacharparenleft}{\kern0pt}{\isasymtau}{\isacharparenright}{\kern0pt}{\isacharparenright}{\kern0pt}\ {\isacharbackquote}{\kern0pt}\ x{\isachardoublequoteclose}\isanewline
\ \ \ \ \ \ \ \ \isacommand{apply}\isamarkupfalse%
{\isacharparenleft}{\kern0pt}subst\ Pn{\isacharunderscore}{\kern0pt}auto{\isacharunderscore}{\kern0pt}comp{\isacharparenright}{\kern0pt}\isanewline
\ \ \ \ \ \ \ \ \isacommand{using}\isamarkupfalse%
\ assms\ P{\isacharunderscore}{\kern0pt}auto{\isacharunderscore}{\kern0pt}def\ is{\isacharunderscore}{\kern0pt}P{\isacharunderscore}{\kern0pt}auto{\isacharunderscore}{\kern0pt}def\ pi{\isacharunderscore}{\kern0pt}pauto\isanewline
\ \ \ \ \ \ \ \ \isacommand{apply}\isamarkupfalse%
\ force\isanewline
\ \ \ \ \ \ \ \ \isacommand{using}\isamarkupfalse%
\ tauH\ {\isasymG}{\isacharunderscore}{\kern0pt}P{\isacharunderscore}{\kern0pt}auto{\isacharunderscore}{\kern0pt}group\ \isanewline
\ \ \ \ \ \ \ \ \isacommand{unfolding}\isamarkupfalse%
\ sym{\isacharunderscore}{\kern0pt}def\ is{\isacharunderscore}{\kern0pt}P{\isacharunderscore}{\kern0pt}auto{\isacharunderscore}{\kern0pt}group{\isacharunderscore}{\kern0pt}def\isanewline
\ \ \ \ \ \ \ \ \isacommand{by}\isamarkupfalse%
\ auto\isanewline
\ \ \ \ \ \ \isacommand{also}\isamarkupfalse%
\ \isacommand{have}\isamarkupfalse%
\ {\isachardoublequoteopen}{\isachardot}{\kern0pt}{\isachardot}{\kern0pt}{\isachardot}{\kern0pt}\ {\isacharequal}{\kern0pt}\ Pn{\isacharunderscore}{\kern0pt}auto{\isacharparenleft}{\kern0pt}{\isasympi}{\isacharparenright}{\kern0pt}\ {\isacharbackquote}{\kern0pt}\ {\isacharparenleft}{\kern0pt}Pn{\isacharunderscore}{\kern0pt}auto{\isacharparenleft}{\kern0pt}{\isasymtau}{\isacharparenright}{\kern0pt}{\isacharbackquote}{\kern0pt}\ x{\isacharparenright}{\kern0pt}{\isachardoublequoteclose}\isanewline
\ \ \ \ \ \ \ \ \isacommand{apply}\isamarkupfalse%
{\isacharparenleft}{\kern0pt}subst\ comp{\isacharunderscore}{\kern0pt}fun{\isacharunderscore}{\kern0pt}apply{\isacharparenright}{\kern0pt}\isanewline
\ \ \ \ \ \ \ \ \isacommand{apply}\isamarkupfalse%
{\isacharparenleft}{\kern0pt}rule\ Pn{\isacharunderscore}{\kern0pt}auto{\isacharunderscore}{\kern0pt}type{\isacharparenright}{\kern0pt}\isanewline
\ \ \ \ \ \ \ \ \isacommand{using}\isamarkupfalse%
\ tauH\ sym{\isacharunderscore}{\kern0pt}def\ {\isasymG}{\isacharunderscore}{\kern0pt}P{\isacharunderscore}{\kern0pt}auto{\isacharunderscore}{\kern0pt}group\ is{\isacharunderscore}{\kern0pt}P{\isacharunderscore}{\kern0pt}auto{\isacharunderscore}{\kern0pt}group{\isacharunderscore}{\kern0pt}def\ assms{\isadigit{1}}\ HS{\isacharunderscore}{\kern0pt}iff\ \isanewline
\ \ \ \ \ \ \ \ \isacommand{by}\isamarkupfalse%
\ auto\isanewline
\ \ \ \ \ \ \isacommand{also}\isamarkupfalse%
\ \isacommand{have}\isamarkupfalse%
\ {\isachardoublequoteopen}{\isachardot}{\kern0pt}{\isachardot}{\kern0pt}{\isachardot}{\kern0pt}\ {\isacharequal}{\kern0pt}\ Pn{\isacharunderscore}{\kern0pt}auto{\isacharparenleft}{\kern0pt}{\isasympi}{\isacharparenright}{\kern0pt}\ {\isacharbackquote}{\kern0pt}\ x{\isachardoublequoteclose}\ \isanewline
\ \ \ \ \ \ \ \ \isacommand{using}\isamarkupfalse%
\ tauH\ sym{\isacharunderscore}{\kern0pt}def\ \isacommand{by}\isamarkupfalse%
\ auto\ \isanewline
\ \ \ \ \ \ \isacommand{finally}\isamarkupfalse%
\ \isacommand{show}\isamarkupfalse%
\ {\isachardoublequoteopen}{\isasymsigma}\ {\isasymin}\ sym{\isacharparenleft}{\kern0pt}Pn{\isacharunderscore}{\kern0pt}auto{\isacharparenleft}{\kern0pt}{\isasympi}{\isacharparenright}{\kern0pt}\ {\isacharbackquote}{\kern0pt}\ x{\isacharparenright}{\kern0pt}{\isachardoublequoteclose}\ \isanewline
\ \ \ \ \ \ \ \ \isacommand{unfolding}\isamarkupfalse%
\ sym{\isacharunderscore}{\kern0pt}def\ \isanewline
\ \ \ \ \ \ \ \ \isacommand{using}\isamarkupfalse%
\ sigmain\isanewline
\ \ \ \ \ \ \ \ \isacommand{by}\isamarkupfalse%
\ auto\ \isanewline
\ \ \ \ \isacommand{qed}\isamarkupfalse%
\isanewline
\isanewline
\ \ \ \ \isacommand{have}\isamarkupfalse%
\ H{\isadigit{3}}\ {\isacharcolon}{\kern0pt}\ {\isachardoublequoteopen}sym{\isacharparenleft}{\kern0pt}Pn{\isacharunderscore}{\kern0pt}auto{\isacharparenleft}{\kern0pt}{\isasympi}{\isacharparenright}{\kern0pt}{\isacharbackquote}{\kern0pt}x{\isacharparenright}{\kern0pt}\ {\isasymin}\ P{\isacharunderscore}{\kern0pt}auto{\isacharunderscore}{\kern0pt}subgroups{\isacharparenleft}{\kern0pt}{\isasymG}{\isacharparenright}{\kern0pt}{\isachardoublequoteclose}\ \isanewline
\ \ \ \ \ \ \isacommand{apply}\isamarkupfalse%
{\isacharparenleft}{\kern0pt}rule\ sym{\isacharunderscore}{\kern0pt}P{\isacharunderscore}{\kern0pt}auto{\isacharunderscore}{\kern0pt}subgroup{\isacharparenright}{\kern0pt}\isanewline
\ \ \ \ \ \ \isacommand{apply}\isamarkupfalse%
{\isacharparenleft}{\kern0pt}rule\ Pn{\isacharunderscore}{\kern0pt}auto{\isacharunderscore}{\kern0pt}value{\isacharparenright}{\kern0pt}\isanewline
\ \ \ \ \ \ \isacommand{using}\isamarkupfalse%
\ assms\ {\isasymG}{\isacharunderscore}{\kern0pt}P{\isacharunderscore}{\kern0pt}auto{\isacharunderscore}{\kern0pt}group\ is{\isacharunderscore}{\kern0pt}P{\isacharunderscore}{\kern0pt}auto{\isacharunderscore}{\kern0pt}group{\isacharunderscore}{\kern0pt}def\ assms{\isadigit{1}}\ HS{\isacharunderscore}{\kern0pt}iff\ \isanewline
\ \ \ \ \ \ \isacommand{by}\isamarkupfalse%
\ auto\isanewline
\isanewline
\ \ \ \ \isacommand{then}\isamarkupfalse%
\ \isacommand{have}\isamarkupfalse%
\ {\isachardoublequoteopen}symmetric{\isacharparenleft}{\kern0pt}Pn{\isacharunderscore}{\kern0pt}auto{\isacharparenleft}{\kern0pt}{\isasympi}{\isacharparenright}{\kern0pt}{\isacharbackquote}{\kern0pt}x{\isacharparenright}{\kern0pt}{\isachardoublequoteclose}\ \isanewline
\ \ \ \ \ \ \isacommand{unfolding}\isamarkupfalse%
\ symmetric{\isacharunderscore}{\kern0pt}def\ \isanewline
\ \ \ \ \ \ \isacommand{using}\isamarkupfalse%
\ H{\isadigit{1}}\ H{\isadigit{2}}\ {\isasymF}{\isacharunderscore}{\kern0pt}closed{\isacharunderscore}{\kern0pt}under{\isacharunderscore}{\kern0pt}supergroup\ sym{\isacharunderscore}{\kern0pt}P{\isacharunderscore}{\kern0pt}auto{\isacharunderscore}{\kern0pt}subgroup\ assms{\isadigit{1}}\ Pn{\isacharunderscore}{\kern0pt}auto{\isacharunderscore}{\kern0pt}value\ HS{\isacharunderscore}{\kern0pt}iff\ \isanewline
\ \ \ \ \ \ \isacommand{by}\isamarkupfalse%
\ auto\isanewline
\isanewline
\ \ \ \ \isacommand{then}\isamarkupfalse%
\ \isacommand{show}\isamarkupfalse%
\ {\isachardoublequoteopen}Pn{\isacharunderscore}{\kern0pt}auto{\isacharparenleft}{\kern0pt}{\isasympi}{\isacharparenright}{\kern0pt}\ {\isacharbackquote}{\kern0pt}\ x\ {\isasymin}\ HS{\isachardoublequoteclose}\ \isanewline
\ \ \ \ \ \ \isacommand{apply}\isamarkupfalse%
{\isacharparenleft}{\kern0pt}rule{\isacharunderscore}{\kern0pt}tac\ iffD{\isadigit{2}}{\isacharcomma}{\kern0pt}\ rule{\isacharunderscore}{\kern0pt}tac\ HS{\isacharunderscore}{\kern0pt}iff{\isacharparenright}{\kern0pt}\isanewline
\ \ \ \ \ \ \isacommand{apply}\isamarkupfalse%
{\isacharparenleft}{\kern0pt}rule\ conjI{\isacharcomma}{\kern0pt}\ rule\ Pn{\isacharunderscore}{\kern0pt}auto{\isacharunderscore}{\kern0pt}value{\isacharparenright}{\kern0pt}\isanewline
\ \ \ \ \ \ \isacommand{using}\isamarkupfalse%
\ assms\ {\isasymG}{\isacharunderscore}{\kern0pt}P{\isacharunderscore}{\kern0pt}auto{\isacharunderscore}{\kern0pt}group\ is{\isacharunderscore}{\kern0pt}P{\isacharunderscore}{\kern0pt}auto{\isacharunderscore}{\kern0pt}group{\isacharunderscore}{\kern0pt}def\ \isanewline
\ \ \ \ \ \ \ \ \isacommand{apply}\isamarkupfalse%
\ force\isanewline
\ \ \ \ \ \ \isacommand{using}\isamarkupfalse%
\ assms{\isadigit{1}}\ HS{\isacharunderscore}{\kern0pt}iff\ \isanewline
\ \ \ \ \ \ \ \isacommand{apply}\isamarkupfalse%
\ force\isanewline
\ \ \ \ \ \ \isacommand{apply}\isamarkupfalse%
{\isacharparenleft}{\kern0pt}rule\ conjI{\isacharcomma}{\kern0pt}\ rule\ domsubset{\isacharcomma}{\kern0pt}\ simp{\isacharparenright}{\kern0pt}\isanewline
\ \ \ \ \ \ \isacommand{done}\isamarkupfalse%
\isanewline
\ \ \isacommand{qed}\isamarkupfalse%
\isanewline
\isanewline
\ \ \isacommand{then}\isamarkupfalse%
\ \isacommand{show}\isamarkupfalse%
\ {\isachardoublequoteopen}Pn{\isacharunderscore}{\kern0pt}auto{\isacharparenleft}{\kern0pt}{\isasympi}{\isacharparenright}{\kern0pt}\ {\isacharbackquote}{\kern0pt}\ x\ {\isasymin}\ HS{\isachardoublequoteclose}\ \isanewline
\ \ \ \ \isacommand{using}\isamarkupfalse%
\ assms\ \isanewline
\ \ \ \ \isacommand{by}\isamarkupfalse%
\ auto\isanewline
\isacommand{qed}\isamarkupfalse%
%
\endisatagproof
{\isafoldproof}%
%
\isadelimproof
\isanewline
%
\endisadelimproof
\isanewline
\isacommand{lemma}\isamarkupfalse%
\ HS{\isacharunderscore}{\kern0pt}Pn{\isacharunderscore}{\kern0pt}auto\ {\isacharcolon}{\kern0pt}\ {\isachardoublequoteopen}{\isasympi}\ {\isasymin}\ {\isasymG}\ {\isasymLongrightarrow}\ x\ {\isasymin}\ P{\isacharunderscore}{\kern0pt}names\ {\isasymLongrightarrow}\ x\ {\isasymin}\ HS\ {\isasymlongleftrightarrow}\ Pn{\isacharunderscore}{\kern0pt}auto{\isacharparenleft}{\kern0pt}{\isasympi}{\isacharparenright}{\kern0pt}{\isacharbackquote}{\kern0pt}x\ {\isasymin}\ HS{\isachardoublequoteclose}\ \isanewline
%
\isadelimproof
%
\endisadelimproof
%
\isatagproof
\isacommand{proof}\isamarkupfalse%
{\isacharparenleft}{\kern0pt}rule\ iffI{\isacharparenright}{\kern0pt}\isanewline
\ \ \isacommand{fix}\isamarkupfalse%
\ {\isasympi}\ \isacommand{assume}\isamarkupfalse%
\ {\isachardoublequoteopen}{\isasympi}\ {\isasymin}\ {\isasymG}{\isachardoublequoteclose}\ {\isachardoublequoteopen}x\ {\isasymin}\ HS{\isachardoublequoteclose}\ \isanewline
\ \ \isacommand{then}\isamarkupfalse%
\ \isacommand{show}\isamarkupfalse%
\ {\isachardoublequoteopen}Pn{\isacharunderscore}{\kern0pt}auto{\isacharparenleft}{\kern0pt}{\isasympi}{\isacharparenright}{\kern0pt}{\isacharbackquote}{\kern0pt}x\ {\isasymin}\ HS{\isachardoublequoteclose}\ \isanewline
\ \ \ \ \isacommand{using}\isamarkupfalse%
\ HS{\isacharunderscore}{\kern0pt}Pn{\isacharunderscore}{\kern0pt}auto{\isacharunderscore}{\kern0pt}helper\ \isacommand{by}\isamarkupfalse%
\ auto\isanewline
\isacommand{next}\isamarkupfalse%
\ \isanewline
\ \ \isacommand{fix}\isamarkupfalse%
\ {\isasympi}\ \isacommand{assume}\isamarkupfalse%
\ assms\ {\isacharcolon}{\kern0pt}\ {\isachardoublequoteopen}{\isasympi}\ {\isasymin}\ {\isasymG}{\isachardoublequoteclose}\ {\isachardoublequoteopen}Pn{\isacharunderscore}{\kern0pt}auto{\isacharparenleft}{\kern0pt}{\isasympi}{\isacharparenright}{\kern0pt}{\isacharbackquote}{\kern0pt}x\ {\isasymin}\ HS{\isachardoublequoteclose}\ {\isachardoublequoteopen}x\ {\isasymin}\ P{\isacharunderscore}{\kern0pt}names{\isachardoublequoteclose}\isanewline
\ \ \isacommand{have}\isamarkupfalse%
\ H\ {\isacharcolon}{\kern0pt}\ {\isachardoublequoteopen}Pn{\isacharunderscore}{\kern0pt}auto{\isacharparenleft}{\kern0pt}converse{\isacharparenleft}{\kern0pt}{\isasympi}{\isacharparenright}{\kern0pt}{\isacharparenright}{\kern0pt}\ {\isacharbackquote}{\kern0pt}\ {\isacharparenleft}{\kern0pt}Pn{\isacharunderscore}{\kern0pt}auto{\isacharparenleft}{\kern0pt}{\isasympi}{\isacharparenright}{\kern0pt}{\isacharbackquote}{\kern0pt}x{\isacharparenright}{\kern0pt}\ {\isasymin}\ HS{\isachardoublequoteclose}\ \isanewline
\ \ \ \ \isacommand{apply}\isamarkupfalse%
{\isacharparenleft}{\kern0pt}rule\ HS{\isacharunderscore}{\kern0pt}Pn{\isacharunderscore}{\kern0pt}auto{\isacharunderscore}{\kern0pt}helper{\isacharparenright}{\kern0pt}\isanewline
\ \ \ \ \isacommand{using}\isamarkupfalse%
\ {\isasymG}{\isacharunderscore}{\kern0pt}P{\isacharunderscore}{\kern0pt}auto{\isacharunderscore}{\kern0pt}group\ is{\isacharunderscore}{\kern0pt}P{\isacharunderscore}{\kern0pt}auto{\isacharunderscore}{\kern0pt}group{\isacharunderscore}{\kern0pt}def\ assms\ \isanewline
\ \ \ \ \isacommand{by}\isamarkupfalse%
\ auto\isanewline
\ \ \isacommand{have}\isamarkupfalse%
\ {\isachardoublequoteopen}Pn{\isacharunderscore}{\kern0pt}auto{\isacharparenleft}{\kern0pt}converse{\isacharparenleft}{\kern0pt}{\isasympi}{\isacharparenright}{\kern0pt}{\isacharparenright}{\kern0pt}\ {\isacharbackquote}{\kern0pt}\ {\isacharparenleft}{\kern0pt}Pn{\isacharunderscore}{\kern0pt}auto{\isacharparenleft}{\kern0pt}{\isasympi}{\isacharparenright}{\kern0pt}{\isacharbackquote}{\kern0pt}x{\isacharparenright}{\kern0pt}\ {\isacharequal}{\kern0pt}\ converse{\isacharparenleft}{\kern0pt}Pn{\isacharunderscore}{\kern0pt}auto{\isacharparenleft}{\kern0pt}{\isasympi}{\isacharparenright}{\kern0pt}{\isacharparenright}{\kern0pt}\ {\isacharbackquote}{\kern0pt}\ {\isacharparenleft}{\kern0pt}Pn{\isacharunderscore}{\kern0pt}auto{\isacharparenleft}{\kern0pt}{\isasympi}{\isacharparenright}{\kern0pt}{\isacharbackquote}{\kern0pt}x{\isacharparenright}{\kern0pt}{\isachardoublequoteclose}\isanewline
\ \ \ \ \isacommand{apply}\isamarkupfalse%
{\isacharparenleft}{\kern0pt}subst\ Pn{\isacharunderscore}{\kern0pt}auto{\isacharunderscore}{\kern0pt}converse{\isacharparenright}{\kern0pt}\isanewline
\ \ \ \ \isacommand{using}\isamarkupfalse%
\ assms\ {\isasymG}{\isacharunderscore}{\kern0pt}P{\isacharunderscore}{\kern0pt}auto{\isacharunderscore}{\kern0pt}group\ is{\isacharunderscore}{\kern0pt}P{\isacharunderscore}{\kern0pt}auto{\isacharunderscore}{\kern0pt}group{\isacharunderscore}{\kern0pt}def\ \isanewline
\ \ \ \ \isacommand{by}\isamarkupfalse%
\ auto\isanewline
\ \ \isacommand{also}\isamarkupfalse%
\ \isacommand{have}\isamarkupfalse%
\ {\isachardoublequoteopen}{\isachardot}{\kern0pt}{\isachardot}{\kern0pt}{\isachardot}{\kern0pt}\ {\isacharequal}{\kern0pt}\ {\isacharparenleft}{\kern0pt}converse{\isacharparenleft}{\kern0pt}Pn{\isacharunderscore}{\kern0pt}auto{\isacharparenleft}{\kern0pt}{\isasympi}{\isacharparenright}{\kern0pt}{\isacharparenright}{\kern0pt}\ O\ Pn{\isacharunderscore}{\kern0pt}auto{\isacharparenleft}{\kern0pt}{\isasympi}{\isacharparenright}{\kern0pt}{\isacharparenright}{\kern0pt}\ {\isacharbackquote}{\kern0pt}\ x{\isachardoublequoteclose}\ \isanewline
\ \ \ \ \isacommand{apply}\isamarkupfalse%
{\isacharparenleft}{\kern0pt}rule\ sym{\isacharcomma}{\kern0pt}\ rule\ comp{\isacharunderscore}{\kern0pt}fun{\isacharunderscore}{\kern0pt}apply{\isacharparenright}{\kern0pt}\isanewline
\ \ \ \ \ \isacommand{apply}\isamarkupfalse%
{\isacharparenleft}{\kern0pt}rule\ Pn{\isacharunderscore}{\kern0pt}auto{\isacharunderscore}{\kern0pt}type{\isacharparenright}{\kern0pt}\isanewline
\ \ \ \ \isacommand{using}\isamarkupfalse%
\ assms\ {\isasymG}{\isacharunderscore}{\kern0pt}P{\isacharunderscore}{\kern0pt}auto{\isacharunderscore}{\kern0pt}group\ is{\isacharunderscore}{\kern0pt}P{\isacharunderscore}{\kern0pt}auto{\isacharunderscore}{\kern0pt}group{\isacharunderscore}{\kern0pt}def\ \ \isanewline
\ \ \ \ \isacommand{by}\isamarkupfalse%
\ auto\ \isanewline
\ \ \isacommand{also}\isamarkupfalse%
\ \isacommand{have}\isamarkupfalse%
\ {\isachardoublequoteopen}{\isachardot}{\kern0pt}{\isachardot}{\kern0pt}{\isachardot}{\kern0pt}\ {\isacharequal}{\kern0pt}\ Pn{\isacharunderscore}{\kern0pt}auto{\isacharparenleft}{\kern0pt}id{\isacharparenleft}{\kern0pt}P{\isacharparenright}{\kern0pt}{\isacharparenright}{\kern0pt}\ {\isacharbackquote}{\kern0pt}\ x{\isachardoublequoteclose}\ \isanewline
\ \ \ \ \isacommand{apply}\isamarkupfalse%
{\isacharparenleft}{\kern0pt}subst\ left{\isacharunderscore}{\kern0pt}comp{\isacharunderscore}{\kern0pt}inverse{\isacharparenright}{\kern0pt}\isanewline
\ \ \ \ \ \isacommand{apply}\isamarkupfalse%
{\isacharparenleft}{\kern0pt}rule\ bij{\isacharunderscore}{\kern0pt}is{\isacharunderscore}{\kern0pt}inj{\isacharparenright}{\kern0pt}\isanewline
\ \ \ \ \isacommand{apply}\isamarkupfalse%
{\isacharparenleft}{\kern0pt}rule\ Pn{\isacharunderscore}{\kern0pt}auto{\isacharunderscore}{\kern0pt}bij{\isacharparenright}{\kern0pt}\isanewline
\ \ \ \ \isacommand{using}\isamarkupfalse%
\ assms\ {\isasymG}{\isacharunderscore}{\kern0pt}P{\isacharunderscore}{\kern0pt}auto{\isacharunderscore}{\kern0pt}group\ is{\isacharunderscore}{\kern0pt}P{\isacharunderscore}{\kern0pt}auto{\isacharunderscore}{\kern0pt}group{\isacharunderscore}{\kern0pt}def\ Pn{\isacharunderscore}{\kern0pt}auto{\isacharunderscore}{\kern0pt}id\isanewline
\ \ \ \ \isacommand{by}\isamarkupfalse%
\ auto\isanewline
\ \ \isacommand{also}\isamarkupfalse%
\ \isacommand{have}\isamarkupfalse%
\ {\isachardoublequoteopen}{\isachardot}{\kern0pt}{\isachardot}{\kern0pt}{\isachardot}{\kern0pt}\ {\isacharequal}{\kern0pt}\ id{\isacharparenleft}{\kern0pt}P{\isacharunderscore}{\kern0pt}names{\isacharparenright}{\kern0pt}\ {\isacharbackquote}{\kern0pt}\ x{\isachardoublequoteclose}\ \isanewline
\ \ \ \ \isacommand{by}\isamarkupfalse%
{\isacharparenleft}{\kern0pt}subst\ Pn{\isacharunderscore}{\kern0pt}auto{\isacharunderscore}{\kern0pt}id{\isacharcomma}{\kern0pt}\ simp{\isacharparenright}{\kern0pt}\isanewline
\ \ \isacommand{also}\isamarkupfalse%
\ \isacommand{have}\isamarkupfalse%
\ {\isachardoublequoteopen}{\isachardot}{\kern0pt}{\isachardot}{\kern0pt}{\isachardot}{\kern0pt}\ {\isacharequal}{\kern0pt}\ x{\isachardoublequoteclose}\ \isanewline
\ \ \ \ \isacommand{using}\isamarkupfalse%
\ assms\ \isanewline
\ \ \ \ \isacommand{by}\isamarkupfalse%
\ auto\ \isanewline
\ \ \isacommand{finally}\isamarkupfalse%
\ \isacommand{show}\isamarkupfalse%
\ {\isachardoublequoteopen}x\ {\isasymin}\ HS{\isachardoublequoteclose}\ \isacommand{using}\isamarkupfalse%
\ H\ \isacommand{by}\isamarkupfalse%
\ auto\isanewline
\isacommand{qed}\isamarkupfalse%
%
\endisatagproof
{\isafoldproof}%
%
\isadelimproof
\isanewline
%
\endisadelimproof
\isanewline
\isacommand{lemma}\isamarkupfalse%
\ Pn{\isacharunderscore}{\kern0pt}auto{\isacharunderscore}{\kern0pt}domain{\isacharunderscore}{\kern0pt}closed{\isacharunderscore}{\kern0pt}helper\ {\isacharcolon}{\kern0pt}\isanewline
\ \ \isakeyword{fixes}\ x\ y\ {\isasympi}\isanewline
\ \ \isakeyword{assumes}\ {\isachardoublequoteopen}x\ {\isasymin}\ P{\isacharunderscore}{\kern0pt}names{\isachardoublequoteclose}\ {\isachardoublequoteopen}y\ {\isasymin}\ domain{\isacharparenleft}{\kern0pt}x{\isacharparenright}{\kern0pt}{\isachardoublequoteclose}\ {\isachardoublequoteopen}{\isasympi}\ {\isasymin}\ sym{\isacharparenleft}{\kern0pt}x{\isacharparenright}{\kern0pt}{\isachardoublequoteclose}\ \isanewline
\ \ \isakeyword{shows}\ {\isachardoublequoteopen}Pn{\isacharunderscore}{\kern0pt}auto{\isacharparenleft}{\kern0pt}{\isasympi}{\isacharparenright}{\kern0pt}{\isacharbackquote}{\kern0pt}y\ {\isasymin}\ domain{\isacharparenleft}{\kern0pt}x{\isacharparenright}{\kern0pt}{\isachardoublequoteclose}\ \isanewline
%
\isadelimproof
%
\endisadelimproof
%
\isatagproof
\isacommand{proof}\isamarkupfalse%
\ {\isacharminus}{\kern0pt}\ \isanewline
\ \ \isacommand{obtain}\isamarkupfalse%
\ p\ \isakeyword{where}\ pH{\isacharcolon}{\kern0pt}\ {\isachardoublequoteopen}{\isacharless}{\kern0pt}y{\isacharcomma}{\kern0pt}\ p{\isachargreater}{\kern0pt}\ {\isasymin}\ x{\isachardoublequoteclose}\ \isacommand{using}\isamarkupfalse%
\ assms\ \isacommand{by}\isamarkupfalse%
\ auto\isanewline
\ \ \isacommand{have}\isamarkupfalse%
\ {\isachardoublequoteopen}{\isacharless}{\kern0pt}Pn{\isacharunderscore}{\kern0pt}auto{\isacharparenleft}{\kern0pt}{\isasympi}{\isacharparenright}{\kern0pt}{\isacharbackquote}{\kern0pt}y{\isacharcomma}{\kern0pt}\ {\isasympi}{\isacharbackquote}{\kern0pt}p{\isachargreater}{\kern0pt}\ {\isasymin}\ Pn{\isacharunderscore}{\kern0pt}auto{\isacharparenleft}{\kern0pt}{\isasympi}{\isacharparenright}{\kern0pt}{\isacharbackquote}{\kern0pt}x{\isachardoublequoteclose}\isanewline
\ \ \ \ \isacommand{apply}\isamarkupfalse%
{\isacharparenleft}{\kern0pt}subst\ {\isacharparenleft}{\kern0pt}{\isadigit{2}}{\isacharparenright}{\kern0pt}\ Pn{\isacharunderscore}{\kern0pt}auto{\isacharparenright}{\kern0pt}\isanewline
\ \ \ \ \isacommand{using}\isamarkupfalse%
\ assms\ \isanewline
\ \ \ \ \ \isacommand{apply}\isamarkupfalse%
\ simp\isanewline
\ \ \ \ \isacommand{apply}\isamarkupfalse%
{\isacharparenleft}{\kern0pt}rule\ pair{\isacharunderscore}{\kern0pt}relI{\isacharparenright}{\kern0pt}\isanewline
\ \ \ \ \isacommand{using}\isamarkupfalse%
\ pH\isanewline
\ \ \ \ \isacommand{by}\isamarkupfalse%
\ auto\isanewline
\ \ \isacommand{then}\isamarkupfalse%
\ \isacommand{have}\isamarkupfalse%
\ {\isachardoublequoteopen}{\isacharless}{\kern0pt}Pn{\isacharunderscore}{\kern0pt}auto{\isacharparenleft}{\kern0pt}{\isasympi}{\isacharparenright}{\kern0pt}{\isacharbackquote}{\kern0pt}y{\isacharcomma}{\kern0pt}\ {\isasympi}{\isacharbackquote}{\kern0pt}p{\isachargreater}{\kern0pt}\ {\isasymin}\ x{\isachardoublequoteclose}\ \isanewline
\ \ \ \ \isacommand{using}\isamarkupfalse%
\ assms\ sym{\isacharunderscore}{\kern0pt}def\ \isanewline
\ \ \ \ \isacommand{by}\isamarkupfalse%
\ auto\isanewline
\ \ \isacommand{then}\isamarkupfalse%
\ \isacommand{show}\isamarkupfalse%
\ {\isacharquery}{\kern0pt}thesis\ \isacommand{by}\isamarkupfalse%
\ auto\isanewline
\isacommand{qed}\isamarkupfalse%
%
\endisatagproof
{\isafoldproof}%
%
\isadelimproof
\isanewline
%
\endisadelimproof
\isanewline
\isacommand{lemma}\isamarkupfalse%
\ Pn{\isacharunderscore}{\kern0pt}auto{\isacharunderscore}{\kern0pt}domain{\isacharunderscore}{\kern0pt}closed{\isacharcolon}{\kern0pt}\isanewline
\ \ \isakeyword{fixes}\ x\ y\ {\isasympi}\isanewline
\ \ \isakeyword{assumes}\ {\isachardoublequoteopen}x\ {\isasymin}\ P{\isacharunderscore}{\kern0pt}names{\isachardoublequoteclose}\ {\isachardoublequoteopen}{\isasympi}\ {\isasymin}\ sym{\isacharparenleft}{\kern0pt}x{\isacharparenright}{\kern0pt}{\isachardoublequoteclose}\ {\isachardoublequoteopen}y\ {\isasymin}\ P{\isacharunderscore}{\kern0pt}names{\isachardoublequoteclose}\ \isanewline
\ \ \isakeyword{shows}\ {\isachardoublequoteopen}y\ {\isasymin}\ domain{\isacharparenleft}{\kern0pt}x{\isacharparenright}{\kern0pt}\ {\isasymlongleftrightarrow}\ Pn{\isacharunderscore}{\kern0pt}auto{\isacharparenleft}{\kern0pt}{\isasympi}{\isacharparenright}{\kern0pt}{\isacharbackquote}{\kern0pt}y\ {\isasymin}\ domain{\isacharparenleft}{\kern0pt}x{\isacharparenright}{\kern0pt}{\isachardoublequoteclose}\ \isanewline
%
\isadelimproof
%
\endisadelimproof
%
\isatagproof
\isacommand{proof}\isamarkupfalse%
{\isacharparenleft}{\kern0pt}rule\ iffI{\isacharparenright}{\kern0pt}\isanewline
\ \ \isacommand{assume}\isamarkupfalse%
\ {\isachardoublequoteopen}y\ {\isasymin}\ domain{\isacharparenleft}{\kern0pt}x{\isacharparenright}{\kern0pt}{\isachardoublequoteclose}\ \isanewline
\ \ \isacommand{then}\isamarkupfalse%
\ \isacommand{obtain}\isamarkupfalse%
\ p\ \isakeyword{where}\ {\isachardoublequoteopen}{\isacharless}{\kern0pt}y{\isacharcomma}{\kern0pt}\ p{\isachargreater}{\kern0pt}\ {\isasymin}\ x{\isachardoublequoteclose}\ \isacommand{by}\isamarkupfalse%
\ auto\isanewline
\ \ \isacommand{then}\isamarkupfalse%
\ \isacommand{show}\isamarkupfalse%
\ {\isachardoublequoteopen}Pn{\isacharunderscore}{\kern0pt}auto{\isacharparenleft}{\kern0pt}{\isasympi}{\isacharparenright}{\kern0pt}\ {\isacharbackquote}{\kern0pt}\ y\ {\isasymin}\ domain{\isacharparenleft}{\kern0pt}x{\isacharparenright}{\kern0pt}{\isachardoublequoteclose}\ \isanewline
\ \ \ \ \isacommand{apply}\isamarkupfalse%
{\isacharparenleft}{\kern0pt}rule{\isacharunderscore}{\kern0pt}tac\ Pn{\isacharunderscore}{\kern0pt}auto{\isacharunderscore}{\kern0pt}domain{\isacharunderscore}{\kern0pt}closed{\isacharunderscore}{\kern0pt}helper{\isacharparenright}{\kern0pt}\isanewline
\ \ \ \ \isacommand{using}\isamarkupfalse%
\ assms\ P{\isacharunderscore}{\kern0pt}name{\isacharunderscore}{\kern0pt}domain{\isacharunderscore}{\kern0pt}P{\isacharunderscore}{\kern0pt}name\ \isanewline
\ \ \ \ \isacommand{by}\isamarkupfalse%
\ auto\isanewline
\isacommand{next}\isamarkupfalse%
\ \isanewline
\ \ \isacommand{assume}\isamarkupfalse%
\ assms{\isadigit{1}}\ {\isacharcolon}{\kern0pt}\ {\isachardoublequoteopen}Pn{\isacharunderscore}{\kern0pt}auto{\isacharparenleft}{\kern0pt}{\isasympi}{\isacharparenright}{\kern0pt}\ {\isacharbackquote}{\kern0pt}\ y\ {\isasymin}\ domain{\isacharparenleft}{\kern0pt}x{\isacharparenright}{\kern0pt}{\isachardoublequoteclose}\ \isanewline
\isanewline
\ \ \isacommand{have}\isamarkupfalse%
\ H\ {\isacharcolon}{\kern0pt}\ {\isachardoublequoteopen}Pn{\isacharunderscore}{\kern0pt}auto{\isacharparenleft}{\kern0pt}converse{\isacharparenleft}{\kern0pt}{\isasympi}{\isacharparenright}{\kern0pt}{\isacharparenright}{\kern0pt}{\isacharbackquote}{\kern0pt}{\isacharparenleft}{\kern0pt}Pn{\isacharunderscore}{\kern0pt}auto{\isacharparenleft}{\kern0pt}{\isasympi}{\isacharparenright}{\kern0pt}{\isacharbackquote}{\kern0pt}y{\isacharparenright}{\kern0pt}\ {\isasymin}\ domain{\isacharparenleft}{\kern0pt}x{\isacharparenright}{\kern0pt}{\isachardoublequoteclose}\ \isanewline
\ \ \ \ \isacommand{apply}\isamarkupfalse%
{\isacharparenleft}{\kern0pt}rule\ Pn{\isacharunderscore}{\kern0pt}auto{\isacharunderscore}{\kern0pt}domain{\isacharunderscore}{\kern0pt}closed{\isacharunderscore}{\kern0pt}helper{\isacharparenright}{\kern0pt}\isanewline
\ \ \ \ \isacommand{using}\isamarkupfalse%
\ assms\ assms{\isadigit{1}}\ converse{\isacharunderscore}{\kern0pt}in{\isacharunderscore}{\kern0pt}sym\ assms\ \isanewline
\ \ \ \ \isacommand{by}\isamarkupfalse%
\ auto\isanewline
\ \ \isacommand{have}\isamarkupfalse%
\ {\isachardoublequoteopen}Pn{\isacharunderscore}{\kern0pt}auto{\isacharparenleft}{\kern0pt}converse{\isacharparenleft}{\kern0pt}{\isasympi}{\isacharparenright}{\kern0pt}{\isacharparenright}{\kern0pt}\ {\isacharbackquote}{\kern0pt}\ {\isacharparenleft}{\kern0pt}Pn{\isacharunderscore}{\kern0pt}auto{\isacharparenleft}{\kern0pt}{\isasympi}{\isacharparenright}{\kern0pt}{\isacharbackquote}{\kern0pt}y{\isacharparenright}{\kern0pt}\ {\isacharequal}{\kern0pt}\ converse{\isacharparenleft}{\kern0pt}Pn{\isacharunderscore}{\kern0pt}auto{\isacharparenleft}{\kern0pt}{\isasympi}{\isacharparenright}{\kern0pt}{\isacharparenright}{\kern0pt}\ {\isacharbackquote}{\kern0pt}\ {\isacharparenleft}{\kern0pt}Pn{\isacharunderscore}{\kern0pt}auto{\isacharparenleft}{\kern0pt}{\isasympi}{\isacharparenright}{\kern0pt}{\isacharbackquote}{\kern0pt}y{\isacharparenright}{\kern0pt}{\isachardoublequoteclose}\isanewline
\ \ \ \ \isacommand{apply}\isamarkupfalse%
{\isacharparenleft}{\kern0pt}subst\ Pn{\isacharunderscore}{\kern0pt}auto{\isacharunderscore}{\kern0pt}converse{\isacharparenright}{\kern0pt}\isanewline
\ \ \ \ \isacommand{using}\isamarkupfalse%
\ assms\ {\isasymG}{\isacharunderscore}{\kern0pt}P{\isacharunderscore}{\kern0pt}auto{\isacharunderscore}{\kern0pt}group\ is{\isacharunderscore}{\kern0pt}P{\isacharunderscore}{\kern0pt}auto{\isacharunderscore}{\kern0pt}group{\isacharunderscore}{\kern0pt}def\ sym{\isacharunderscore}{\kern0pt}def\isanewline
\ \ \ \ \isacommand{by}\isamarkupfalse%
\ auto\isanewline
\ \ \isacommand{also}\isamarkupfalse%
\ \isacommand{have}\isamarkupfalse%
\ {\isachardoublequoteopen}{\isachardot}{\kern0pt}{\isachardot}{\kern0pt}{\isachardot}{\kern0pt}\ {\isacharequal}{\kern0pt}\ {\isacharparenleft}{\kern0pt}converse{\isacharparenleft}{\kern0pt}Pn{\isacharunderscore}{\kern0pt}auto{\isacharparenleft}{\kern0pt}{\isasympi}{\isacharparenright}{\kern0pt}{\isacharparenright}{\kern0pt}\ O\ Pn{\isacharunderscore}{\kern0pt}auto{\isacharparenleft}{\kern0pt}{\isasympi}{\isacharparenright}{\kern0pt}{\isacharparenright}{\kern0pt}\ {\isacharbackquote}{\kern0pt}\ y{\isachardoublequoteclose}\ \isanewline
\ \ \ \ \isacommand{apply}\isamarkupfalse%
{\isacharparenleft}{\kern0pt}rule\ sym{\isacharcomma}{\kern0pt}\ rule\ comp{\isacharunderscore}{\kern0pt}fun{\isacharunderscore}{\kern0pt}apply{\isacharparenright}{\kern0pt}\isanewline
\ \ \ \ \ \isacommand{apply}\isamarkupfalse%
{\isacharparenleft}{\kern0pt}rule\ Pn{\isacharunderscore}{\kern0pt}auto{\isacharunderscore}{\kern0pt}type{\isacharparenright}{\kern0pt}\isanewline
\ \ \ \ \isacommand{using}\isamarkupfalse%
\ assms\ {\isasymG}{\isacharunderscore}{\kern0pt}P{\isacharunderscore}{\kern0pt}auto{\isacharunderscore}{\kern0pt}group\ is{\isacharunderscore}{\kern0pt}P{\isacharunderscore}{\kern0pt}auto{\isacharunderscore}{\kern0pt}group{\isacharunderscore}{\kern0pt}def\ sym{\isacharunderscore}{\kern0pt}def\isanewline
\ \ \ \ \isacommand{by}\isamarkupfalse%
\ auto\ \isanewline
\ \ \isacommand{also}\isamarkupfalse%
\ \isacommand{have}\isamarkupfalse%
\ {\isachardoublequoteopen}{\isachardot}{\kern0pt}{\isachardot}{\kern0pt}{\isachardot}{\kern0pt}\ {\isacharequal}{\kern0pt}\ Pn{\isacharunderscore}{\kern0pt}auto{\isacharparenleft}{\kern0pt}id{\isacharparenleft}{\kern0pt}P{\isacharparenright}{\kern0pt}{\isacharparenright}{\kern0pt}\ {\isacharbackquote}{\kern0pt}\ y{\isachardoublequoteclose}\ \isanewline
\ \ \ \ \isacommand{apply}\isamarkupfalse%
{\isacharparenleft}{\kern0pt}subst\ left{\isacharunderscore}{\kern0pt}comp{\isacharunderscore}{\kern0pt}inverse{\isacharparenright}{\kern0pt}\isanewline
\ \ \ \ \ \isacommand{apply}\isamarkupfalse%
{\isacharparenleft}{\kern0pt}rule\ bij{\isacharunderscore}{\kern0pt}is{\isacharunderscore}{\kern0pt}inj{\isacharparenright}{\kern0pt}\isanewline
\ \ \ \ \isacommand{apply}\isamarkupfalse%
{\isacharparenleft}{\kern0pt}rule\ Pn{\isacharunderscore}{\kern0pt}auto{\isacharunderscore}{\kern0pt}bij{\isacharparenright}{\kern0pt}\isanewline
\ \ \ \ \isacommand{using}\isamarkupfalse%
\ assms\ {\isasymG}{\isacharunderscore}{\kern0pt}P{\isacharunderscore}{\kern0pt}auto{\isacharunderscore}{\kern0pt}group\ is{\isacharunderscore}{\kern0pt}P{\isacharunderscore}{\kern0pt}auto{\isacharunderscore}{\kern0pt}group{\isacharunderscore}{\kern0pt}def\ Pn{\isacharunderscore}{\kern0pt}auto{\isacharunderscore}{\kern0pt}id\ sym{\isacharunderscore}{\kern0pt}def\isanewline
\ \ \ \ \isacommand{by}\isamarkupfalse%
\ auto\isanewline
\ \ \isacommand{also}\isamarkupfalse%
\ \isacommand{have}\isamarkupfalse%
\ {\isachardoublequoteopen}{\isachardot}{\kern0pt}{\isachardot}{\kern0pt}{\isachardot}{\kern0pt}\ {\isacharequal}{\kern0pt}\ id{\isacharparenleft}{\kern0pt}P{\isacharunderscore}{\kern0pt}names{\isacharparenright}{\kern0pt}\ {\isacharbackquote}{\kern0pt}\ y{\isachardoublequoteclose}\ \isanewline
\ \ \ \ \isacommand{by}\isamarkupfalse%
{\isacharparenleft}{\kern0pt}subst\ Pn{\isacharunderscore}{\kern0pt}auto{\isacharunderscore}{\kern0pt}id{\isacharcomma}{\kern0pt}\ simp{\isacharparenright}{\kern0pt}\isanewline
\ \ \isacommand{also}\isamarkupfalse%
\ \isacommand{have}\isamarkupfalse%
\ {\isachardoublequoteopen}{\isachardot}{\kern0pt}{\isachardot}{\kern0pt}{\isachardot}{\kern0pt}\ {\isacharequal}{\kern0pt}\ y{\isachardoublequoteclose}\ \isanewline
\ \ \ \ \isacommand{using}\isamarkupfalse%
\ assms\ \isanewline
\ \ \ \ \isacommand{by}\isamarkupfalse%
\ auto\ \isanewline
\ \ \isacommand{finally}\isamarkupfalse%
\ \isacommand{show}\isamarkupfalse%
\ {\isachardoublequoteopen}y\ {\isasymin}\ domain{\isacharparenleft}{\kern0pt}x{\isacharparenright}{\kern0pt}{\isachardoublequoteclose}\ \isanewline
\ \ \ \ \isacommand{using}\isamarkupfalse%
\ H\ \isacommand{by}\isamarkupfalse%
\ auto\isanewline
\isacommand{qed}\isamarkupfalse%
%
\endisatagproof
{\isafoldproof}%
%
\isadelimproof
\isanewline
%
\endisadelimproof
\isanewline
\isacommand{end}\isamarkupfalse%
\isanewline
%
\isadelimtheory
%
\endisadelimtheory
%
\isatagtheory
\isacommand{end}\isamarkupfalse%
%
\endisatagtheory
{\isafoldtheory}%
%
\isadelimtheory
%
\endisadelimtheory
%
\end{isabellebody}%
\endinput
%:%file=~/source/repos/ZF-notAC/code/HS_Theorems.thy%:%
%:%10=1%:%
%:%11=1%:%
%:%12=2%:%
%:%13=3%:%
%:%14=4%:%
%:%15=5%:%
%:%20=5%:%
%:%23=6%:%
%:%24=7%:%
%:%25=7%:%
%:%26=8%:%
%:%27=9%:%
%:%28=10%:%
%:%29=10%:%
%:%36=11%:%
%:%37=11%:%
%:%38=12%:%
%:%39=12%:%
%:%40=13%:%
%:%41=13%:%
%:%42=14%:%
%:%43=14%:%
%:%44=15%:%
%:%45=15%:%
%:%46=15%:%
%:%47=16%:%
%:%48=16%:%
%:%49=17%:%
%:%50=17%:%
%:%51=17%:%
%:%52=17%:%
%:%53=18%:%
%:%54=18%:%
%:%55=19%:%
%:%56=19%:%
%:%57=20%:%
%:%58=20%:%
%:%59=21%:%
%:%60=21%:%
%:%61=22%:%
%:%62=22%:%
%:%63=23%:%
%:%64=23%:%
%:%65=23%:%
%:%66=24%:%
%:%67=24%:%
%:%68=25%:%
%:%69=25%:%
%:%70=26%:%
%:%71=26%:%
%:%72=27%:%
%:%73=27%:%
%:%74=27%:%
%:%75=28%:%
%:%76=28%:%
%:%77=29%:%
%:%78=29%:%
%:%79=30%:%
%:%80=30%:%
%:%81=31%:%
%:%82=31%:%
%:%83=32%:%
%:%84=32%:%
%:%85=33%:%
%:%86=33%:%
%:%87=33%:%
%:%88=34%:%
%:%89=34%:%
%:%90=34%:%
%:%91=34%:%
%:%92=35%:%
%:%93=35%:%
%:%94=35%:%
%:%95=36%:%
%:%96=36%:%
%:%97=36%:%
%:%98=36%:%
%:%99=36%:%
%:%100=37%:%
%:%101=37%:%
%:%102=37%:%
%:%103=37%:%
%:%104=37%:%
%:%105=38%:%
%:%106=38%:%
%:%107=39%:%
%:%108=40%:%
%:%109=40%:%
%:%110=40%:%
%:%111=40%:%
%:%112=40%:%
%:%113=41%:%
%:%114=41%:%
%:%115=42%:%
%:%116=42%:%
%:%117=42%:%
%:%118=42%:%
%:%119=42%:%
%:%120=43%:%
%:%126=43%:%
%:%129=44%:%
%:%130=45%:%
%:%131=45%:%
%:%134=46%:%
%:%138=46%:%
%:%139=46%:%
%:%140=47%:%
%:%141=47%:%
%:%142=48%:%
%:%143=48%:%
%:%144=49%:%
%:%145=49%:%
%:%146=50%:%
%:%147=50%:%
%:%148=51%:%
%:%149=51%:%
%:%150=52%:%
%:%151=52%:%
%:%152=53%:%
%:%153=53%:%
%:%154=54%:%
%:%155=54%:%
%:%156=55%:%
%:%157=55%:%
%:%158=56%:%
%:%159=56%:%
%:%160=57%:%
%:%161=57%:%
%:%162=58%:%
%:%163=58%:%
%:%168=58%:%
%:%171=59%:%
%:%172=60%:%
%:%173=60%:%
%:%180=61%:%
%:%181=61%:%
%:%182=62%:%
%:%183=62%:%
%:%184=63%:%
%:%185=64%:%
%:%186=64%:%
%:%187=65%:%
%:%188=65%:%
%:%189=66%:%
%:%190=66%:%
%:%191=67%:%
%:%192=67%:%
%:%193=68%:%
%:%194=68%:%
%:%195=68%:%
%:%196=69%:%
%:%197=69%:%
%:%198=70%:%
%:%199=70%:%
%:%200=71%:%
%:%201=71%:%
%:%202=72%:%
%:%203=72%:%
%:%204=72%:%
%:%205=72%:%
%:%206=73%:%
%:%207=73%:%
%:%208=73%:%
%:%209=74%:%
%:%210=74%:%
%:%211=75%:%
%:%212=75%:%
%:%213=76%:%
%:%214=76%:%
%:%215=76%:%
%:%216=77%:%
%:%217=77%:%
%:%218=78%:%
%:%219=78%:%
%:%220=79%:%
%:%221=80%:%
%:%222=80%:%
%:%223=81%:%
%:%224=81%:%
%:%225=82%:%
%:%226=82%:%
%:%227=83%:%
%:%228=83%:%
%:%229=84%:%
%:%230=84%:%
%:%231=84%:%
%:%232=85%:%
%:%233=85%:%
%:%234=86%:%
%:%235=86%:%
%:%236=87%:%
%:%237=87%:%
%:%238=88%:%
%:%239=89%:%
%:%240=89%:%
%:%241=89%:%
%:%242=90%:%
%:%243=90%:%
%:%244=91%:%
%:%245=91%:%
%:%246=92%:%
%:%247=92%:%
%:%248=93%:%
%:%254=93%:%
%:%257=94%:%
%:%258=95%:%
%:%259=95%:%
%:%266=96%:%
%:%267=96%:%
%:%268=97%:%
%:%269=97%:%
%:%270=98%:%
%:%271=98%:%
%:%272=99%:%
%:%273=100%:%
%:%274=100%:%
%:%275=101%:%
%:%276=101%:%
%:%277=102%:%
%:%278=102%:%
%:%279=103%:%
%:%280=103%:%
%:%281=104%:%
%:%282=105%:%
%:%283=105%:%
%:%284=105%:%
%:%285=106%:%
%:%286=106%:%
%:%287=107%:%
%:%288=107%:%
%:%289=108%:%
%:%290=108%:%
%:%291=109%:%
%:%292=109%:%
%:%293=110%:%
%:%294=110%:%
%:%295=111%:%
%:%296=111%:%
%:%297=112%:%
%:%298=112%:%
%:%299=113%:%
%:%305=113%:%
%:%308=114%:%
%:%309=115%:%
%:%310=115%:%
%:%317=116%:%
%:%318=116%:%
%:%319=117%:%
%:%320=117%:%
%:%321=118%:%
%:%322=119%:%
%:%323=119%:%
%:%324=120%:%
%:%325=120%:%
%:%326=121%:%
%:%327=121%:%
%:%328=121%:%
%:%329=122%:%
%:%330=123%:%
%:%331=123%:%
%:%332=124%:%
%:%333=124%:%
%:%334=124%:%
%:%335=125%:%
%:%336=126%:%
%:%337=126%:%
%:%338=126%:%
%:%339=126%:%
%:%340=127%:%
%:%341=128%:%
%:%342=128%:%
%:%343=128%:%
%:%344=128%:%
%:%345=129%:%
%:%346=129%:%
%:%347=130%:%
%:%348=130%:%
%:%349=131%:%
%:%350=131%:%
%:%351=131%:%
%:%352=132%:%
%:%353=132%:%
%:%354=132%:%
%:%355=132%:%
%:%356=133%:%
%:%357=133%:%
%:%358=134%:%
%:%359=134%:%
%:%360=135%:%
%:%361=135%:%
%:%362=136%:%
%:%363=136%:%
%:%364=137%:%
%:%365=137%:%
%:%366=137%:%
%:%367=137%:%
%:%368=138%:%
%:%369=138%:%
%:%370=138%:%
%:%371=138%:%
%:%372=138%:%
%:%373=139%:%
%:%374=139%:%
%:%375=139%:%
%:%376=139%:%
%:%377=139%:%
%:%378=140%:%
%:%379=140%:%
%:%380=141%:%
%:%381=142%:%
%:%382=142%:%
%:%383=142%:%
%:%384=143%:%
%:%385=143%:%
%:%386=143%:%
%:%387=144%:%
%:%388=145%:%
%:%389=145%:%
%:%390=146%:%
%:%391=146%:%
%:%392=147%:%
%:%393=147%:%
%:%394=147%:%
%:%395=148%:%
%:%396=149%:%
%:%397=149%:%
%:%398=149%:%
%:%399=149%:%
%:%400=150%:%
%:%401=151%:%
%:%402=151%:%
%:%403=151%:%
%:%404=151%:%
%:%405=152%:%
%:%406=153%:%
%:%407=153%:%
%:%408=153%:%
%:%409=153%:%
%:%410=154%:%
%:%411=154%:%
%:%412=154%:%
%:%413=155%:%
%:%414=155%:%
%:%415=156%:%
%:%416=156%:%
%:%417=157%:%
%:%418=157%:%
%:%419=158%:%
%:%420=158%:%
%:%421=158%:%
%:%422=159%:%
%:%423=159%:%
%:%424=160%:%
%:%425=160%:%
%:%426=161%:%
%:%427=161%:%
%:%428=162%:%
%:%429=162%:%
%:%430=163%:%
%:%431=163%:%
%:%432=164%:%
%:%433=164%:%
%:%434=165%:%
%:%435=165%:%
%:%436=166%:%
%:%437=166%:%
%:%438=167%:%
%:%439=167%:%
%:%440=168%:%
%:%441=168%:%
%:%442=169%:%
%:%443=169%:%
%:%444=170%:%
%:%445=170%:%
%:%446=171%:%
%:%447=171%:%
%:%448=171%:%
%:%449=172%:%
%:%450=172%:%
%:%451=172%:%
%:%452=173%:%
%:%453=173%:%
%:%454=173%:%
%:%455=174%:%
%:%456=174%:%
%:%457=175%:%
%:%458=175%:%
%:%459=176%:%
%:%460=176%:%
%:%461=177%:%
%:%462=177%:%
%:%463=177%:%
%:%464=178%:%
%:%465=178%:%
%:%466=179%:%
%:%467=179%:%
%:%468=180%:%
%:%469=180%:%
%:%470=181%:%
%:%471=181%:%
%:%472=181%:%
%:%473=182%:%
%:%474=182%:%
%:%475=183%:%
%:%476=183%:%
%:%477=184%:%
%:%478=184%:%
%:%479=185%:%
%:%480=185%:%
%:%481=186%:%
%:%482=186%:%
%:%483=187%:%
%:%484=187%:%
%:%485=188%:%
%:%486=188%:%
%:%487=188%:%
%:%488=189%:%
%:%489=189%:%
%:%490=190%:%
%:%491=190%:%
%:%492=191%:%
%:%493=191%:%
%:%494=192%:%
%:%495=192%:%
%:%496=193%:%
%:%497=193%:%
%:%498=193%:%
%:%499=194%:%
%:%500=194%:%
%:%501=194%:%
%:%502=195%:%
%:%503=195%:%
%:%504=195%:%
%:%505=196%:%
%:%506=196%:%
%:%507=197%:%
%:%508=197%:%
%:%509=198%:%
%:%510=198%:%
%:%511=199%:%
%:%512=199%:%
%:%513=200%:%
%:%514=201%:%
%:%515=201%:%
%:%516=202%:%
%:%517=202%:%
%:%518=203%:%
%:%519=203%:%
%:%520=204%:%
%:%521=204%:%
%:%522=205%:%
%:%523=205%:%
%:%524=206%:%
%:%525=207%:%
%:%526=207%:%
%:%527=207%:%
%:%528=208%:%
%:%529=208%:%
%:%530=209%:%
%:%531=209%:%
%:%532=210%:%
%:%533=210%:%
%:%534=211%:%
%:%535=212%:%
%:%536=212%:%
%:%537=212%:%
%:%538=213%:%
%:%539=213%:%
%:%540=214%:%
%:%541=214%:%
%:%542=215%:%
%:%543=215%:%
%:%544=216%:%
%:%545=216%:%
%:%546=217%:%
%:%547=217%:%
%:%548=218%:%
%:%549=218%:%
%:%550=219%:%
%:%551=219%:%
%:%552=220%:%
%:%553=220%:%
%:%554=221%:%
%:%555=221%:%
%:%556=222%:%
%:%557=223%:%
%:%558=223%:%
%:%559=223%:%
%:%560=224%:%
%:%561=224%:%
%:%562=225%:%
%:%563=225%:%
%:%564=226%:%
%:%570=226%:%
%:%573=227%:%
%:%574=228%:%
%:%575=228%:%
%:%582=229%:%
%:%583=229%:%
%:%584=230%:%
%:%585=230%:%
%:%586=230%:%
%:%587=231%:%
%:%588=231%:%
%:%589=231%:%
%:%590=232%:%
%:%591=232%:%
%:%592=232%:%
%:%593=233%:%
%:%594=233%:%
%:%595=234%:%
%:%596=234%:%
%:%597=234%:%
%:%598=235%:%
%:%599=235%:%
%:%600=236%:%
%:%601=236%:%
%:%602=237%:%
%:%603=237%:%
%:%604=238%:%
%:%605=238%:%
%:%606=239%:%
%:%607=239%:%
%:%608=240%:%
%:%609=240%:%
%:%610=241%:%
%:%611=241%:%
%:%612=242%:%
%:%613=242%:%
%:%614=243%:%
%:%615=243%:%
%:%616=243%:%
%:%617=244%:%
%:%618=244%:%
%:%619=245%:%
%:%620=245%:%
%:%621=246%:%
%:%622=246%:%
%:%623=247%:%
%:%624=247%:%
%:%625=248%:%
%:%626=248%:%
%:%627=248%:%
%:%628=249%:%
%:%629=249%:%
%:%630=250%:%
%:%631=250%:%
%:%632=251%:%
%:%633=251%:%
%:%634=252%:%
%:%635=252%:%
%:%636=253%:%
%:%637=253%:%
%:%638=254%:%
%:%639=254%:%
%:%640=254%:%
%:%641=255%:%
%:%642=255%:%
%:%643=256%:%
%:%644=256%:%
%:%645=256%:%
%:%646=257%:%
%:%647=257%:%
%:%648=258%:%
%:%649=258%:%
%:%650=259%:%
%:%651=259%:%
%:%652=259%:%
%:%653=259%:%
%:%654=259%:%
%:%655=260%:%
%:%661=260%:%
%:%664=261%:%
%:%665=262%:%
%:%666=262%:%
%:%667=263%:%
%:%668=264%:%
%:%669=265%:%
%:%676=266%:%
%:%677=266%:%
%:%678=267%:%
%:%679=267%:%
%:%680=267%:%
%:%681=267%:%
%:%682=268%:%
%:%683=268%:%
%:%684=269%:%
%:%685=269%:%
%:%686=270%:%
%:%687=270%:%
%:%688=271%:%
%:%689=271%:%
%:%690=272%:%
%:%691=272%:%
%:%692=273%:%
%:%693=273%:%
%:%694=274%:%
%:%695=274%:%
%:%696=275%:%
%:%697=275%:%
%:%698=275%:%
%:%699=276%:%
%:%700=276%:%
%:%701=277%:%
%:%702=277%:%
%:%703=278%:%
%:%704=278%:%
%:%705=278%:%
%:%706=278%:%
%:%707=279%:%
%:%713=279%:%
%:%716=280%:%
%:%717=281%:%
%:%718=281%:%
%:%719=282%:%
%:%720=283%:%
%:%721=284%:%
%:%728=285%:%
%:%729=285%:%
%:%730=286%:%
%:%731=286%:%
%:%732=287%:%
%:%733=287%:%
%:%734=287%:%
%:%735=287%:%
%:%736=288%:%
%:%737=288%:%
%:%738=288%:%
%:%739=289%:%
%:%740=289%:%
%:%741=290%:%
%:%742=290%:%
%:%743=291%:%
%:%744=291%:%
%:%745=292%:%
%:%746=292%:%
%:%747=293%:%
%:%748=293%:%
%:%749=294%:%
%:%750=295%:%
%:%751=295%:%
%:%752=296%:%
%:%753=296%:%
%:%754=297%:%
%:%755=297%:%
%:%756=298%:%
%:%757=298%:%
%:%758=299%:%
%:%759=299%:%
%:%760=300%:%
%:%761=300%:%
%:%762=301%:%
%:%763=301%:%
%:%764=302%:%
%:%765=302%:%
%:%766=303%:%
%:%767=303%:%
%:%768=303%:%
%:%769=304%:%
%:%770=304%:%
%:%771=305%:%
%:%772=305%:%
%:%773=306%:%
%:%774=306%:%
%:%775=307%:%
%:%776=307%:%
%:%777=308%:%
%:%778=308%:%
%:%779=308%:%
%:%780=309%:%
%:%781=309%:%
%:%782=310%:%
%:%783=310%:%
%:%784=311%:%
%:%785=311%:%
%:%786=312%:%
%:%787=312%:%
%:%788=313%:%
%:%789=313%:%
%:%790=314%:%
%:%791=314%:%
%:%792=314%:%
%:%793=315%:%
%:%794=315%:%
%:%795=316%:%
%:%796=316%:%
%:%797=316%:%
%:%798=317%:%
%:%799=317%:%
%:%800=318%:%
%:%801=318%:%
%:%802=319%:%
%:%803=319%:%
%:%804=319%:%
%:%805=320%:%
%:%806=320%:%
%:%807=320%:%
%:%808=321%:%
%:%814=321%:%
%:%817=322%:%
%:%818=323%:%
%:%819=323%:%
%:%826=324%:%

%
\begin{isabellebody}%
\setisabellecontext{SymExt{\isacharunderscore}{\kern0pt}Definition}%
%
\isadelimtheory
%
\endisadelimtheory
%
\isatagtheory
\isacommand{theory}\isamarkupfalse%
\ SymExt{\isacharunderscore}{\kern0pt}Definition\isanewline
\ \ \isakeyword{imports}\ \isanewline
\ \ \ \ {\isachardoublequoteopen}Forcing{\isacharslash}{\kern0pt}Forcing{\isacharunderscore}{\kern0pt}Main{\isachardoublequoteclose}\ \isanewline
\ \ \ \ HS{\isacharunderscore}{\kern0pt}Theorems\isanewline
\isakeyword{begin}%
\endisatagtheory
{\isafoldtheory}%
%
\isadelimtheory
\ \isanewline
%
\endisadelimtheory
\isanewline
\isacommand{context}\isamarkupfalse%
\ M{\isacharunderscore}{\kern0pt}symmetric{\isacharunderscore}{\kern0pt}system\ \isanewline
\isakeyword{begin}\ \isanewline
\isanewline
\isacommand{definition}\isamarkupfalse%
\ SymExt\ \isakeyword{where}\ {\isachardoublequoteopen}SymExt{\isacharparenleft}{\kern0pt}G{\isacharparenright}{\kern0pt}\ {\isasymequiv}\ {\isacharbraceleft}{\kern0pt}\ val{\isacharparenleft}{\kern0pt}G{\isacharcomma}{\kern0pt}\ x{\isacharparenright}{\kern0pt}{\isachardot}{\kern0pt}\ x\ {\isasymin}\ HS\ {\isacharbraceright}{\kern0pt}{\isachardoublequoteclose}\ \isanewline
\isanewline
\isacommand{end}\isamarkupfalse%
\ \isanewline
\isanewline
\isanewline
\isacommand{locale}\isamarkupfalse%
\ M{\isacharunderscore}{\kern0pt}symmetric{\isacharunderscore}{\kern0pt}system{\isacharunderscore}{\kern0pt}G{\isacharunderscore}{\kern0pt}generic\ {\isacharequal}{\kern0pt}\ M{\isacharunderscore}{\kern0pt}symmetric{\isacharunderscore}{\kern0pt}system\ {\isacharplus}{\kern0pt}\ G{\isacharunderscore}{\kern0pt}generic\isanewline
\isakeyword{begin}\isanewline
\isanewline
\isacommand{lemma}\isamarkupfalse%
\ M{\isacharunderscore}{\kern0pt}subset{\isacharunderscore}{\kern0pt}SymExt\ {\isacharcolon}{\kern0pt}\ {\isachardoublequoteopen}M\ {\isasymsubseteq}\ SymExt{\isacharparenleft}{\kern0pt}G{\isacharparenright}{\kern0pt}{\isachardoublequoteclose}\ \isanewline
%
\isadelimproof
%
\endisadelimproof
%
\isatagproof
\isacommand{proof}\isamarkupfalse%
\ {\isacharparenleft}{\kern0pt}rule\ subsetI{\isacharparenright}{\kern0pt}\isanewline
\ \ \isacommand{fix}\isamarkupfalse%
\ x\ \isacommand{assume}\isamarkupfalse%
\ assms\ {\isacharcolon}{\kern0pt}\ {\isachardoublequoteopen}x\ {\isasymin}\ M{\isachardoublequoteclose}\ \isanewline
\ \ \isacommand{then}\isamarkupfalse%
\ \isacommand{have}\isamarkupfalse%
\ {\isachardoublequoteopen}check{\isacharparenleft}{\kern0pt}x{\isacharparenright}{\kern0pt}\ {\isasymin}\ HS{\isachardoublequoteclose}\ \isacommand{using}\isamarkupfalse%
\ check{\isacharunderscore}{\kern0pt}in{\isacharunderscore}{\kern0pt}HS\ \isacommand{by}\isamarkupfalse%
\ auto\ \isanewline
\ \ \isacommand{then}\isamarkupfalse%
\ \isacommand{have}\isamarkupfalse%
\ {\isachardoublequoteopen}val{\isacharparenleft}{\kern0pt}G{\isacharcomma}{\kern0pt}\ check{\isacharparenleft}{\kern0pt}x{\isacharparenright}{\kern0pt}{\isacharparenright}{\kern0pt}\ {\isasymin}\ SymExt{\isacharparenleft}{\kern0pt}G{\isacharparenright}{\kern0pt}{\isachardoublequoteclose}\ \isacommand{unfolding}\isamarkupfalse%
\ SymExt{\isacharunderscore}{\kern0pt}def\ \isacommand{by}\isamarkupfalse%
\ auto\ \isanewline
\ \ \isacommand{then}\isamarkupfalse%
\ \isacommand{show}\isamarkupfalse%
\ {\isachardoublequoteopen}x\ {\isasymin}\ SymExt{\isacharparenleft}{\kern0pt}G{\isacharparenright}{\kern0pt}{\isachardoublequoteclose}\ \isacommand{using}\isamarkupfalse%
\ valcheck\ one{\isacharunderscore}{\kern0pt}in{\isacharunderscore}{\kern0pt}G\ assms\ one{\isacharunderscore}{\kern0pt}in{\isacharunderscore}{\kern0pt}P\ generic\ \isacommand{by}\isamarkupfalse%
\ auto\ \isanewline
\isacommand{qed}\isamarkupfalse%
%
\endisatagproof
{\isafoldproof}%
%
\isadelimproof
\isanewline
%
\endisadelimproof
\isanewline
\isacommand{lemma}\isamarkupfalse%
\ SymExt{\isacharunderscore}{\kern0pt}subset{\isacharunderscore}{\kern0pt}GenExt\ {\isacharcolon}{\kern0pt}\ {\isachardoublequoteopen}SymExt{\isacharparenleft}{\kern0pt}G{\isacharparenright}{\kern0pt}\ {\isasymsubseteq}\ GenExt{\isacharparenleft}{\kern0pt}G{\isacharparenright}{\kern0pt}{\isachardoublequoteclose}\ \isanewline
%
\isadelimproof
\ \ %
\endisadelimproof
%
\isatagproof
\isacommand{apply}\isamarkupfalse%
{\isacharparenleft}{\kern0pt}rule\ subsetI{\isacharparenright}{\kern0pt}\isanewline
\ \ \isacommand{unfolding}\isamarkupfalse%
\ SymExt{\isacharunderscore}{\kern0pt}def\ GenExt{\isacharunderscore}{\kern0pt}def\ \isanewline
\ \ \isacommand{using}\isamarkupfalse%
\ HS{\isacharunderscore}{\kern0pt}iff\ P{\isacharunderscore}{\kern0pt}name{\isacharunderscore}{\kern0pt}in{\isacharunderscore}{\kern0pt}M\ \isanewline
\ \ \isacommand{by}\isamarkupfalse%
\ auto%
\endisatagproof
{\isafoldproof}%
%
\isadelimproof
\isanewline
%
\endisadelimproof
\isanewline
\isacommand{lemma}\isamarkupfalse%
\ Transset{\isacharunderscore}{\kern0pt}SymExt\ {\isacharcolon}{\kern0pt}\ {\isachardoublequoteopen}\ Transset{\isacharparenleft}{\kern0pt}SymExt{\isacharparenleft}{\kern0pt}G{\isacharparenright}{\kern0pt}{\isacharparenright}{\kern0pt}{\isachardoublequoteclose}\ \isanewline
%
\isadelimproof
\ \ %
\endisadelimproof
%
\isatagproof
\isacommand{unfolding}\isamarkupfalse%
\ Transset{\isacharunderscore}{\kern0pt}def\ \isanewline
\isacommand{proof}\isamarkupfalse%
\ {\isacharparenleft}{\kern0pt}rule\ ballI{\isacharsemicolon}{\kern0pt}\ rule\ subsetI{\isacharparenright}{\kern0pt}\isanewline
\ \ \isacommand{fix}\isamarkupfalse%
\ x\ y\ \isacommand{assume}\isamarkupfalse%
\ assms\ {\isacharcolon}{\kern0pt}\ {\isachardoublequoteopen}x\ {\isasymin}\ SymExt{\isacharparenleft}{\kern0pt}G{\isacharparenright}{\kern0pt}{\isachardoublequoteclose}\ {\isachardoublequoteopen}y\ {\isasymin}\ x{\isachardoublequoteclose}\ \isanewline
\ \ \isacommand{then}\isamarkupfalse%
\ \isacommand{obtain}\isamarkupfalse%
\ x{\isacharprime}{\kern0pt}\ \isakeyword{where}\ x{\isacharprime}{\kern0pt}H\ {\isacharcolon}{\kern0pt}\ {\isachardoublequoteopen}x\ {\isacharequal}{\kern0pt}\ val{\isacharparenleft}{\kern0pt}G{\isacharcomma}{\kern0pt}\ x{\isacharprime}{\kern0pt}{\isacharparenright}{\kern0pt}{\isachardoublequoteclose}\ {\isachardoublequoteopen}x{\isacharprime}{\kern0pt}\ {\isasymin}\ HS{\isachardoublequoteclose}\ \isacommand{unfolding}\isamarkupfalse%
\ SymExt{\isacharunderscore}{\kern0pt}def\ \isacommand{by}\isamarkupfalse%
\ auto\isanewline
\ \ \isacommand{then}\isamarkupfalse%
\ \isacommand{have}\isamarkupfalse%
\ {\isachardoublequoteopen}{\isasymexists}y{\isacharprime}{\kern0pt}\ {\isasymin}\ domain{\isacharparenleft}{\kern0pt}x{\isacharprime}{\kern0pt}{\isacharparenright}{\kern0pt}\ {\isachardot}{\kern0pt}\ y\ {\isacharequal}{\kern0pt}\ val{\isacharparenleft}{\kern0pt}G{\isacharcomma}{\kern0pt}\ y{\isacharprime}{\kern0pt}{\isacharparenright}{\kern0pt}{\isachardoublequoteclose}\ \isacommand{using}\isamarkupfalse%
\ assms\ elem{\isacharunderscore}{\kern0pt}of{\isacharunderscore}{\kern0pt}val\ \isacommand{by}\isamarkupfalse%
\ blast\ \isanewline
\ \ \isacommand{then}\isamarkupfalse%
\ \isacommand{obtain}\isamarkupfalse%
\ y{\isacharprime}{\kern0pt}\ \isakeyword{where}\ y{\isacharprime}{\kern0pt}H\ {\isacharcolon}{\kern0pt}\ {\isachardoublequoteopen}y{\isacharprime}{\kern0pt}\ {\isasymin}\ domain{\isacharparenleft}{\kern0pt}x{\isacharprime}{\kern0pt}{\isacharparenright}{\kern0pt}{\isachardoublequoteclose}\ {\isachardoublequoteopen}y\ {\isacharequal}{\kern0pt}\ val{\isacharparenleft}{\kern0pt}G{\isacharcomma}{\kern0pt}\ y{\isacharprime}{\kern0pt}{\isacharparenright}{\kern0pt}{\isachardoublequoteclose}\ \isacommand{using}\isamarkupfalse%
\ elem{\isacharunderscore}{\kern0pt}of{\isacharunderscore}{\kern0pt}val\ assms\ \isacommand{by}\isamarkupfalse%
\ auto\isanewline
\ \ \isacommand{then}\isamarkupfalse%
\ \isacommand{have}\isamarkupfalse%
\ y{\isacharprime}{\kern0pt}inHS\ {\isacharcolon}{\kern0pt}\ {\isachardoublequoteopen}y{\isacharprime}{\kern0pt}\ {\isasymin}\ HS{\isachardoublequoteclose}\ \isacommand{using}\isamarkupfalse%
\ HS{\isacharunderscore}{\kern0pt}iff\ x{\isacharprime}{\kern0pt}H\ \isacommand{by}\isamarkupfalse%
\ auto\ \isanewline
\ \ \isacommand{then}\isamarkupfalse%
\ \isacommand{show}\isamarkupfalse%
\ {\isachardoublequoteopen}y\ {\isasymin}\ SymExt{\isacharparenleft}{\kern0pt}G{\isacharparenright}{\kern0pt}{\isachardoublequoteclose}\ \isacommand{unfolding}\isamarkupfalse%
\ SymExt{\isacharunderscore}{\kern0pt}def\ y{\isacharprime}{\kern0pt}H\ \isacommand{by}\isamarkupfalse%
\ auto\isanewline
\isacommand{qed}\isamarkupfalse%
%
\endisatagproof
{\isafoldproof}%
%
\isadelimproof
\isanewline
%
\endisadelimproof
\isanewline
\isacommand{lemma}\isamarkupfalse%
\ nonempty{\isacharunderscore}{\kern0pt}SymExt\ {\isacharcolon}{\kern0pt}\ {\isachardoublequoteopen}SymExt{\isacharparenleft}{\kern0pt}G{\isacharparenright}{\kern0pt}\ {\isasymnoteq}\ {\isadigit{0}}{\isachardoublequoteclose}\ \isanewline
%
\isadelimproof
\ \ %
\endisadelimproof
%
\isatagproof
\isacommand{apply}\isamarkupfalse%
{\isacharparenleft}{\kern0pt}subgoal{\isacharunderscore}{\kern0pt}tac\ {\isachardoublequoteopen}{\isadigit{0}}\ {\isasymin}\ SymExt{\isacharparenleft}{\kern0pt}G{\isacharparenright}{\kern0pt}{\isachardoublequoteclose}{\isacharcomma}{\kern0pt}\ force{\isacharparenright}{\kern0pt}\isanewline
\ \ \isacommand{using}\isamarkupfalse%
\ zero{\isacharunderscore}{\kern0pt}in{\isacharunderscore}{\kern0pt}M\ M{\isacharunderscore}{\kern0pt}subset{\isacharunderscore}{\kern0pt}SymExt\isanewline
\ \ \isacommand{by}\isamarkupfalse%
\ auto%
\endisatagproof
{\isafoldproof}%
%
\isadelimproof
\isanewline
%
\endisadelimproof
\isanewline
\isanewline
\isacommand{lemma}\isamarkupfalse%
\ ex{\isacharunderscore}{\kern0pt}HS{\isacharunderscore}{\kern0pt}name{\isacharunderscore}{\kern0pt}list\ {\isacharcolon}{\kern0pt}\ \isanewline
\ \ \isakeyword{fixes}\ l\ \isanewline
\ \ \isakeyword{assumes}\ {\isachardoublequoteopen}l\ {\isasymin}\ list{\isacharparenleft}{\kern0pt}SymExt{\isacharparenleft}{\kern0pt}G{\isacharparenright}{\kern0pt}{\isacharparenright}{\kern0pt}{\isachardoublequoteclose}\ \isanewline
\ \ \isakeyword{shows}\ {\isachardoublequoteopen}{\isasymexists}l{\isacharprime}{\kern0pt}\ {\isasymin}\ list{\isacharparenleft}{\kern0pt}HS{\isacharparenright}{\kern0pt}{\isachardot}{\kern0pt}\ map{\isacharparenleft}{\kern0pt}val{\isacharparenleft}{\kern0pt}G{\isacharparenright}{\kern0pt}{\isacharcomma}{\kern0pt}\ l{\isacharprime}{\kern0pt}{\isacharparenright}{\kern0pt}\ {\isacharequal}{\kern0pt}\ l{\isachardoublequoteclose}\ \isanewline
%
\isadelimproof
\isanewline
\ \ %
\endisadelimproof
%
\isatagproof
\isacommand{using}\isamarkupfalse%
\ assms\isanewline
\isacommand{proof}\isamarkupfalse%
{\isacharparenleft}{\kern0pt}induct{\isacharparenright}{\kern0pt}\isanewline
\ \ \isacommand{case}\isamarkupfalse%
\ Nil\isanewline
\ \ \isacommand{then}\isamarkupfalse%
\ \isacommand{show}\isamarkupfalse%
\ {\isacharquery}{\kern0pt}case\ \isanewline
\ \ \ \ \isacommand{apply}\isamarkupfalse%
{\isacharparenleft}{\kern0pt}rule{\isacharunderscore}{\kern0pt}tac\ x{\isacharequal}{\kern0pt}Nil\ \isakeyword{in}\ bexI{\isacharparenright}{\kern0pt}\isanewline
\ \ \ \ \isacommand{by}\isamarkupfalse%
\ auto\isanewline
\isacommand{next}\isamarkupfalse%
\isanewline
\ \ \isacommand{case}\isamarkupfalse%
\ {\isacharparenleft}{\kern0pt}Cons\ h\ t{\isacharparenright}{\kern0pt}\isanewline
\ \ \isacommand{then}\isamarkupfalse%
\ \isacommand{have}\isamarkupfalse%
\ H{\isacharcolon}{\kern0pt}\ {\isachardoublequoteopen}h\ {\isasymin}\ SymExt{\isacharparenleft}{\kern0pt}G{\isacharparenright}{\kern0pt}\ {\isasymand}\ {\isacharparenleft}{\kern0pt}{\isasymexists}t{\isacharprime}{\kern0pt}{\isasymin}list{\isacharparenleft}{\kern0pt}HS{\isacharparenright}{\kern0pt}{\isachardot}{\kern0pt}\ map{\isacharparenleft}{\kern0pt}{\isasymlambda}a{\isachardot}{\kern0pt}\ val{\isacharparenleft}{\kern0pt}G{\isacharcomma}{\kern0pt}\ a{\isacharparenright}{\kern0pt}{\isacharcomma}{\kern0pt}\ t{\isacharprime}{\kern0pt}{\isacharparenright}{\kern0pt}\ {\isacharequal}{\kern0pt}\ t{\isacharparenright}{\kern0pt}{\isachardoublequoteclose}\ \isacommand{by}\isamarkupfalse%
\ auto\ \isanewline
\ \ \isacommand{then}\isamarkupfalse%
\ \isacommand{obtain}\isamarkupfalse%
\ t{\isacharprime}{\kern0pt}\ \isakeyword{where}\ t{\isacharprime}{\kern0pt}H{\isacharcolon}{\kern0pt}\ {\isachardoublequoteopen}t{\isacharprime}{\kern0pt}\ {\isasymin}\ list{\isacharparenleft}{\kern0pt}HS{\isacharparenright}{\kern0pt}{\isachardoublequoteclose}\ {\isachardoublequoteopen}map{\isacharparenleft}{\kern0pt}{\isasymlambda}a{\isachardot}{\kern0pt}\ val{\isacharparenleft}{\kern0pt}G{\isacharcomma}{\kern0pt}\ a{\isacharparenright}{\kern0pt}{\isacharcomma}{\kern0pt}\ t{\isacharprime}{\kern0pt}{\isacharparenright}{\kern0pt}\ {\isacharequal}{\kern0pt}\ t{\isachardoublequoteclose}\ \isacommand{by}\isamarkupfalse%
\ auto\ \isanewline
\ \ \isacommand{obtain}\isamarkupfalse%
\ h{\isacharprime}{\kern0pt}\ \isakeyword{where}\ h{\isacharprime}{\kern0pt}H\ {\isacharcolon}{\kern0pt}{\isachardoublequoteopen}val{\isacharparenleft}{\kern0pt}G{\isacharcomma}{\kern0pt}\ h{\isacharprime}{\kern0pt}{\isacharparenright}{\kern0pt}\ {\isacharequal}{\kern0pt}\ h{\isachardoublequoteclose}\ {\isachardoublequoteopen}h{\isacharprime}{\kern0pt}\ {\isasymin}\ HS{\isachardoublequoteclose}\ \isanewline
\ \ \ \ \isacommand{using}\isamarkupfalse%
\ H\isanewline
\ \ \ \ \isacommand{unfolding}\isamarkupfalse%
\ SymExt{\isacharunderscore}{\kern0pt}def\isanewline
\ \ \ \ \isacommand{by}\isamarkupfalse%
\ auto\ \isanewline
\ \ \isacommand{show}\isamarkupfalse%
\ {\isacharquery}{\kern0pt}case\ \isanewline
\ \ \ \ \isacommand{apply}\isamarkupfalse%
{\isacharparenleft}{\kern0pt}rule{\isacharunderscore}{\kern0pt}tac\ x{\isacharequal}{\kern0pt}{\isachardoublequoteopen}Cons{\isacharparenleft}{\kern0pt}h{\isacharprime}{\kern0pt}{\isacharcomma}{\kern0pt}\ t{\isacharprime}{\kern0pt}{\isacharparenright}{\kern0pt}{\isachardoublequoteclose}\ \isakeyword{in}\ bexI{\isacharparenright}{\kern0pt}\isanewline
\ \ \ \ \isacommand{using}\isamarkupfalse%
\ h{\isacharprime}{\kern0pt}H\ t{\isacharprime}{\kern0pt}H\ \isanewline
\ \ \ \ \isacommand{by}\isamarkupfalse%
\ auto\isanewline
\isacommand{qed}\isamarkupfalse%
%
\endisatagproof
{\isafoldproof}%
%
\isadelimproof
\isanewline
%
\endisadelimproof
\isanewline
\isacommand{lemma}\isamarkupfalse%
\ SymExt{\isacharunderscore}{\kern0pt}trans\ {\isacharcolon}{\kern0pt}\ {\isachardoublequoteopen}x\ {\isasymin}\ SymExt{\isacharparenleft}{\kern0pt}G{\isacharparenright}{\kern0pt}\ {\isasymLongrightarrow}\ y\ {\isasymin}\ x\ {\isasymLongrightarrow}\ y\ {\isasymin}\ SymExt{\isacharparenleft}{\kern0pt}G{\isacharparenright}{\kern0pt}{\isachardoublequoteclose}\ \isanewline
%
\isadelimproof
\ \ %
\endisadelimproof
%
\isatagproof
\isacommand{using}\isamarkupfalse%
\ Transset{\isacharunderscore}{\kern0pt}SymExt\ \isanewline
\ \ \isacommand{unfolding}\isamarkupfalse%
\ Transset{\isacharunderscore}{\kern0pt}def\isanewline
\ \ \isacommand{by}\isamarkupfalse%
\ auto%
\endisatagproof
{\isafoldproof}%
%
\isadelimproof
\isanewline
%
\endisadelimproof
\isanewline
\isacommand{end}\isamarkupfalse%
\isanewline
%
\isadelimtheory
%
\endisadelimtheory
%
\isatagtheory
\isacommand{end}\isamarkupfalse%
%
\endisatagtheory
{\isafoldtheory}%
%
\isadelimtheory
%
\endisadelimtheory
%
\end{isabellebody}%
\endinput
%:%file=~/source/repos/ZF-notAC/code/SymExt_Definition.thy%:%
%:%10=1%:%
%:%11=1%:%
%:%12=2%:%
%:%13=3%:%
%:%14=4%:%
%:%15=5%:%
%:%20=5%:%
%:%23=6%:%
%:%24=7%:%
%:%25=7%:%
%:%26=8%:%
%:%27=9%:%
%:%28=10%:%
%:%29=10%:%
%:%30=11%:%
%:%31=12%:%
%:%32=12%:%
%:%33=13%:%
%:%34=14%:%
%:%35=15%:%
%:%36=15%:%
%:%37=16%:%
%:%38=17%:%
%:%39=18%:%
%:%40=18%:%
%:%47=19%:%
%:%48=19%:%
%:%49=20%:%
%:%50=20%:%
%:%51=20%:%
%:%52=21%:%
%:%53=21%:%
%:%54=21%:%
%:%55=21%:%
%:%56=21%:%
%:%57=22%:%
%:%58=22%:%
%:%59=22%:%
%:%60=22%:%
%:%61=22%:%
%:%62=23%:%
%:%63=23%:%
%:%64=23%:%
%:%65=23%:%
%:%66=23%:%
%:%67=24%:%
%:%73=24%:%
%:%76=25%:%
%:%77=26%:%
%:%78=26%:%
%:%81=27%:%
%:%85=27%:%
%:%86=27%:%
%:%87=28%:%
%:%88=28%:%
%:%89=29%:%
%:%90=29%:%
%:%91=30%:%
%:%92=30%:%
%:%97=30%:%
%:%100=31%:%
%:%101=32%:%
%:%102=32%:%
%:%105=33%:%
%:%109=33%:%
%:%110=33%:%
%:%111=34%:%
%:%112=34%:%
%:%113=35%:%
%:%114=35%:%
%:%115=35%:%
%:%116=36%:%
%:%117=36%:%
%:%118=36%:%
%:%119=36%:%
%:%120=36%:%
%:%121=37%:%
%:%122=37%:%
%:%123=37%:%
%:%124=37%:%
%:%125=37%:%
%:%126=38%:%
%:%127=38%:%
%:%128=38%:%
%:%129=38%:%
%:%130=38%:%
%:%131=39%:%
%:%132=39%:%
%:%133=39%:%
%:%134=39%:%
%:%135=39%:%
%:%136=40%:%
%:%137=40%:%
%:%138=40%:%
%:%139=40%:%
%:%140=40%:%
%:%141=41%:%
%:%147=41%:%
%:%150=42%:%
%:%151=43%:%
%:%152=43%:%
%:%155=44%:%
%:%159=44%:%
%:%160=44%:%
%:%161=45%:%
%:%162=45%:%
%:%163=46%:%
%:%164=46%:%
%:%169=46%:%
%:%172=47%:%
%:%173=48%:%
%:%174=49%:%
%:%175=49%:%
%:%176=50%:%
%:%177=51%:%
%:%178=52%:%
%:%181=53%:%
%:%182=54%:%
%:%186=54%:%
%:%187=54%:%
%:%188=55%:%
%:%189=55%:%
%:%190=56%:%
%:%191=56%:%
%:%192=57%:%
%:%193=57%:%
%:%194=57%:%
%:%195=58%:%
%:%196=58%:%
%:%197=59%:%
%:%198=59%:%
%:%199=60%:%
%:%200=60%:%
%:%201=61%:%
%:%202=61%:%
%:%203=62%:%
%:%204=62%:%
%:%205=62%:%
%:%206=62%:%
%:%207=63%:%
%:%208=63%:%
%:%209=63%:%
%:%210=63%:%
%:%211=64%:%
%:%212=64%:%
%:%213=65%:%
%:%214=65%:%
%:%215=66%:%
%:%216=66%:%
%:%217=67%:%
%:%218=67%:%
%:%219=68%:%
%:%220=68%:%
%:%221=69%:%
%:%222=69%:%
%:%223=70%:%
%:%224=70%:%
%:%225=71%:%
%:%226=71%:%
%:%227=72%:%
%:%233=72%:%
%:%236=73%:%
%:%237=74%:%
%:%238=74%:%
%:%241=75%:%
%:%245=75%:%
%:%246=75%:%
%:%247=76%:%
%:%248=76%:%
%:%249=77%:%
%:%250=77%:%
%:%255=77%:%
%:%258=78%:%
%:%259=79%:%
%:%260=79%:%
%:%267=80%:%

%
\begin{isabellebody}%
\setisabellecontext{Delta{\isadigit{0}}}%
%
\isadelimtheory
%
\endisadelimtheory
%
\isatagtheory
\isacommand{theory}\isamarkupfalse%
\ Delta{\isadigit{0}}\isanewline
\ \ \isakeyword{imports}\ \isanewline
\ \ \ \ Utilities{\isacharunderscore}{\kern0pt}M\isanewline
\isakeyword{begin}%
\endisatagtheory
{\isafoldtheory}%
%
\isadelimtheory
\ \isanewline
%
\endisadelimtheory
\isanewline
\isacommand{definition}\isamarkupfalse%
\ BExists{\isacharprime}{\kern0pt}\ \isakeyword{where}\ {\isachardoublequoteopen}BExists{\isacharprime}{\kern0pt}{\isacharparenleft}{\kern0pt}n{\isacharcomma}{\kern0pt}\ {\isasymphi}{\isacharparenright}{\kern0pt}\ {\isasymequiv}\ Exists{\isacharparenleft}{\kern0pt}And{\isacharparenleft}{\kern0pt}Member{\isacharparenleft}{\kern0pt}{\isadigit{0}}{\isacharcomma}{\kern0pt}\ n{\isacharparenright}{\kern0pt}{\isacharcomma}{\kern0pt}\ {\isasymphi}{\isacharparenright}{\kern0pt}{\isacharparenright}{\kern0pt}{\isachardoublequoteclose}\ \ \isanewline
\isacommand{definition}\isamarkupfalse%
\ BExists\ \isakeyword{where}\ {\isachardoublequoteopen}BExists{\isacharparenleft}{\kern0pt}n{\isacharcomma}{\kern0pt}\ {\isasymphi}{\isacharparenright}{\kern0pt}\ {\isasymequiv}\ BExists{\isacharprime}{\kern0pt}{\isacharparenleft}{\kern0pt}succ{\isacharparenleft}{\kern0pt}n{\isacharparenright}{\kern0pt}{\isacharcomma}{\kern0pt}\ {\isasymphi}{\isacharparenright}{\kern0pt}{\isachardoublequoteclose}\ \isanewline
\isacommand{definition}\isamarkupfalse%
\ BForall\ \isakeyword{where}\ {\isachardoublequoteopen}BForall{\isacharparenleft}{\kern0pt}n{\isacharcomma}{\kern0pt}\ {\isasymphi}{\isacharparenright}{\kern0pt}\ {\isasymequiv}\ Neg{\isacharparenleft}{\kern0pt}BExists{\isacharparenleft}{\kern0pt}n{\isacharcomma}{\kern0pt}\ Neg{\isacharparenleft}{\kern0pt}{\isasymphi}{\isacharparenright}{\kern0pt}{\isacharparenright}{\kern0pt}{\isacharparenright}{\kern0pt}{\isachardoublequoteclose}\ \isanewline
\isanewline
\isacommand{definition}\isamarkupfalse%
\ {\isasymDelta}{\isadigit{0}}{\isacharunderscore}{\kern0pt}from\ \isakeyword{where}\ {\isachardoublequoteopen}{\isasymDelta}{\isadigit{0}}{\isacharunderscore}{\kern0pt}from{\isacharparenleft}{\kern0pt}X{\isacharparenright}{\kern0pt}\ {\isasymequiv}\ X\ {\isasymunion}\ {\isacharbraceleft}{\kern0pt}\ Nand{\isacharparenleft}{\kern0pt}{\isasymphi}{\isacharcomma}{\kern0pt}\ {\isasympsi}{\isacharparenright}{\kern0pt}{\isachardot}{\kern0pt}\ {\isacharless}{\kern0pt}{\isasymphi}{\isacharcomma}{\kern0pt}\ {\isasympsi}{\isachargreater}{\kern0pt}\ {\isasymin}\ X\ {\isasymtimes}\ X\ {\isacharbraceright}{\kern0pt}\ {\isasymunion}\ {\isacharbraceleft}{\kern0pt}\ {\isasympsi}\ {\isasymin}\ formula{\isachardot}{\kern0pt}\ {\isasymexists}n\ {\isasymin}\ nat{\isachardot}{\kern0pt}\ {\isasymexists}{\isasymphi}\ {\isasymin}\ X{\isachardot}{\kern0pt}\ n\ {\isasymnoteq}\ {\isadigit{0}}\ {\isasymand}\ {\isasympsi}\ {\isacharequal}{\kern0pt}\ BExists{\isacharprime}{\kern0pt}{\isacharparenleft}{\kern0pt}n{\isacharcomma}{\kern0pt}\ {\isasymphi}{\isacharparenright}{\kern0pt}\ {\isacharbraceright}{\kern0pt}{\isachardoublequoteclose}\ \isanewline
\isacommand{definition}\isamarkupfalse%
\ {\isasymDelta}{\isadigit{0}}{\isacharunderscore}{\kern0pt}base\ \isakeyword{where}\ {\isachardoublequoteopen}{\isasymDelta}{\isadigit{0}}{\isacharunderscore}{\kern0pt}base\ {\isasymequiv}\ {\isacharbraceleft}{\kern0pt}\ {\isasymphi}\ {\isasymin}\ formula{\isachardot}{\kern0pt}\ {\isasymexists}\ n\ {\isasymin}\ nat{\isachardot}{\kern0pt}\ {\isasymexists}\ m\ {\isasymin}\ nat{\isachardot}{\kern0pt}\ {\isasymphi}\ {\isacharequal}{\kern0pt}\ Member{\isacharparenleft}{\kern0pt}n{\isacharcomma}{\kern0pt}\ m{\isacharparenright}{\kern0pt}\ {\isasymor}\ {\isasymphi}\ {\isacharequal}{\kern0pt}\ Equal{\isacharparenleft}{\kern0pt}n{\isacharcomma}{\kern0pt}\ m{\isacharparenright}{\kern0pt}\ {\isacharbraceright}{\kern0pt}{\isachardoublequoteclose}\ \isanewline
\isanewline
\isacommand{definition}\isamarkupfalse%
\ {\isasymDelta}{\isadigit{0}}\ \isakeyword{where}\ {\isachardoublequoteopen}{\isasymDelta}{\isadigit{0}}\ {\isasymequiv}\ {\isasymUnion}n\ {\isasymin}\ nat{\isachardot}{\kern0pt}\ {\isasymDelta}{\isadigit{0}}{\isacharunderscore}{\kern0pt}from{\isacharcircum}{\kern0pt}n\ {\isacharparenleft}{\kern0pt}{\isasymDelta}{\isadigit{0}}{\isacharunderscore}{\kern0pt}base{\isacharparenright}{\kern0pt}{\isachardoublequoteclose}\ \isanewline
\isanewline
\isacommand{lemma}\isamarkupfalse%
\ {\isasymDelta}{\isadigit{0}}{\isacharunderscore}{\kern0pt}subset\ {\isacharcolon}{\kern0pt}\ {\isachardoublequoteopen}{\isasymDelta}{\isadigit{0}}\ {\isasymsubseteq}\ formula{\isachardoublequoteclose}\ \isanewline
%
\isadelimproof
%
\endisadelimproof
%
\isatagproof
\isacommand{proof}\isamarkupfalse%
\ {\isacharminus}{\kern0pt}\isanewline
\ \ \isacommand{have}\isamarkupfalse%
\ main\ {\isacharcolon}{\kern0pt}\ {\isachardoublequoteopen}{\isasymAnd}n{\isachardot}{\kern0pt}\ n\ {\isasymin}\ nat\ {\isasymLongrightarrow}\ {\isasymDelta}{\isadigit{0}}{\isacharunderscore}{\kern0pt}from{\isacharcircum}{\kern0pt}n\ {\isacharparenleft}{\kern0pt}{\isasymDelta}{\isadigit{0}}{\isacharunderscore}{\kern0pt}base{\isacharparenright}{\kern0pt}\ {\isasymsubseteq}\ formula{\isachardoublequoteclose}\ \isanewline
\ \ \ \ \isacommand{apply}\isamarkupfalse%
\ {\isacharparenleft}{\kern0pt}rule{\isacharunderscore}{\kern0pt}tac\ P{\isacharequal}{\kern0pt}{\isachardoublequoteopen}{\isasymlambda}n{\isachardot}{\kern0pt}\ {\isasymDelta}{\isadigit{0}}{\isacharunderscore}{\kern0pt}from{\isacharcircum}{\kern0pt}n\ {\isacharparenleft}{\kern0pt}{\isasymDelta}{\isadigit{0}}{\isacharunderscore}{\kern0pt}base{\isacharparenright}{\kern0pt}\ {\isasymsubseteq}\ formula{\isachardoublequoteclose}\ \isakeyword{in}\ nat{\isacharunderscore}{\kern0pt}induct{\isacharcomma}{\kern0pt}\ simp{\isacharparenright}{\kern0pt}\isanewline
\ \ \ \ \isacommand{unfolding}\isamarkupfalse%
\ {\isasymDelta}{\isadigit{0}}{\isacharunderscore}{\kern0pt}from{\isacharunderscore}{\kern0pt}def\ {\isasymDelta}{\isadigit{0}}{\isacharunderscore}{\kern0pt}base{\isacharunderscore}{\kern0pt}def\isanewline
\ \ \ \ \isacommand{by}\isamarkupfalse%
\ auto\isanewline
\ \ \isacommand{then}\isamarkupfalse%
\ \isacommand{show}\isamarkupfalse%
\ {\isacharquery}{\kern0pt}thesis\ \isanewline
\ \ \ \ \isacommand{unfolding}\isamarkupfalse%
\ {\isasymDelta}{\isadigit{0}}{\isacharunderscore}{\kern0pt}def\ \isanewline
\ \ \ \ \isacommand{by}\isamarkupfalse%
\ auto\isanewline
\isacommand{qed}\isamarkupfalse%
%
\endisatagproof
{\isafoldproof}%
%
\isadelimproof
\isanewline
%
\endisadelimproof
\isanewline
\isacommand{lemma}\isamarkupfalse%
\ {\isasymDelta}{\isadigit{0}}{\isacharunderscore}{\kern0pt}from{\isacharunderscore}{\kern0pt}increasing\ {\isacharcolon}{\kern0pt}\ \isanewline
\ \ \isakeyword{fixes}\ n\ m\ \isanewline
\ \ \isakeyword{assumes}\ {\isachardoublequoteopen}n\ {\isasymin}\ nat{\isachardoublequoteclose}\ {\isachardoublequoteopen}m\ {\isasymin}\ nat{\isachardoublequoteclose}\ {\isachardoublequoteopen}n\ {\isasymle}\ m{\isachardoublequoteclose}\isanewline
\ \ \isakeyword{shows}\ {\isachardoublequoteopen}{\isasymDelta}{\isadigit{0}}{\isacharunderscore}{\kern0pt}from{\isacharcircum}{\kern0pt}n{\isacharparenleft}{\kern0pt}{\isasymDelta}{\isadigit{0}}{\isacharunderscore}{\kern0pt}base{\isacharparenright}{\kern0pt}\ {\isasymsubseteq}\ {\isasymDelta}{\isadigit{0}}{\isacharunderscore}{\kern0pt}from{\isacharcircum}{\kern0pt}m{\isacharparenleft}{\kern0pt}{\isasymDelta}{\isadigit{0}}{\isacharunderscore}{\kern0pt}base{\isacharparenright}{\kern0pt}{\isachardoublequoteclose}\ \isanewline
%
\isadelimproof
%
\endisadelimproof
%
\isatagproof
\isacommand{proof}\isamarkupfalse%
{\isacharminus}{\kern0pt}\ \isanewline
\ \ \isacommand{have}\isamarkupfalse%
\ main\ {\isacharcolon}{\kern0pt}\ {\isachardoublequoteopen}{\isasymforall}n\ {\isasymin}\ nat{\isachardot}{\kern0pt}\ n\ {\isasymle}\ m\ {\isasymlongrightarrow}\ {\isasymDelta}{\isadigit{0}}{\isacharunderscore}{\kern0pt}from{\isacharcircum}{\kern0pt}n{\isacharparenleft}{\kern0pt}{\isasymDelta}{\isadigit{0}}{\isacharunderscore}{\kern0pt}base{\isacharparenright}{\kern0pt}\ {\isasymsubseteq}\ {\isasymDelta}{\isadigit{0}}{\isacharunderscore}{\kern0pt}from{\isacharcircum}{\kern0pt}m{\isacharparenleft}{\kern0pt}{\isasymDelta}{\isadigit{0}}{\isacharunderscore}{\kern0pt}base{\isacharparenright}{\kern0pt}{\isachardoublequoteclose}\ \isanewline
\ \ \ \ \isacommand{apply}\isamarkupfalse%
{\isacharparenleft}{\kern0pt}rule{\isacharunderscore}{\kern0pt}tac\ P{\isacharequal}{\kern0pt}{\isachardoublequoteopen}{\isasymlambda}m{\isachardot}{\kern0pt}\ {\isasymforall}n\ {\isasymin}\ nat{\isachardot}{\kern0pt}\ n\ {\isasymle}\ m\ {\isasymlongrightarrow}\ {\isasymDelta}{\isadigit{0}}{\isacharunderscore}{\kern0pt}from{\isacharcircum}{\kern0pt}n{\isacharparenleft}{\kern0pt}{\isasymDelta}{\isadigit{0}}{\isacharunderscore}{\kern0pt}base{\isacharparenright}{\kern0pt}\ {\isasymsubseteq}\ {\isasymDelta}{\isadigit{0}}{\isacharunderscore}{\kern0pt}from{\isacharcircum}{\kern0pt}m{\isacharparenleft}{\kern0pt}{\isasymDelta}{\isadigit{0}}{\isacharunderscore}{\kern0pt}base{\isacharparenright}{\kern0pt}{\isachardoublequoteclose}\ \isakeyword{in}\ nat{\isacharunderscore}{\kern0pt}induct{\isacharparenright}{\kern0pt}\isanewline
\ \ \ \ \isacommand{apply}\isamarkupfalse%
{\isacharparenleft}{\kern0pt}simp\ add{\isacharcolon}{\kern0pt}assms{\isacharcomma}{\kern0pt}\ force{\isacharparenright}{\kern0pt}\isanewline
\ \ \ \ \isacommand{apply}\isamarkupfalse%
{\isacharparenleft}{\kern0pt}rule\ ballI{\isacharcomma}{\kern0pt}\ rule\ impI{\isacharparenright}{\kern0pt}\isanewline
\ \ \ \ \isacommand{apply}\isamarkupfalse%
{\isacharparenleft}{\kern0pt}rename{\isacharunderscore}{\kern0pt}tac\ k\ n{\isacharcomma}{\kern0pt}\ rule{\isacharunderscore}{\kern0pt}tac\ P{\isacharequal}{\kern0pt}{\isachardoublequoteopen}n\ {\isasymle}\ k{\isachardoublequoteclose}\ \isakeyword{and}\ Q{\isacharequal}{\kern0pt}{\isachardoublequoteopen}n\ {\isacharequal}{\kern0pt}\ succ{\isacharparenleft}{\kern0pt}k{\isacharparenright}{\kern0pt}{\isachardoublequoteclose}\ \isakeyword{in}\ disjE{\isacharparenright}{\kern0pt}\isanewline
\ \ \ \ \isacommand{using}\isamarkupfalse%
\ le{\isacharunderscore}{\kern0pt}iff\isanewline
\ \ \ \ \ \ \isacommand{apply}\isamarkupfalse%
\ force\isanewline
\ \ \ \ \ \isacommand{apply}\isamarkupfalse%
{\isacharparenleft}{\kern0pt}simp{\isacharcomma}{\kern0pt}\ subst\ {\isasymDelta}{\isadigit{0}}{\isacharunderscore}{\kern0pt}from{\isacharunderscore}{\kern0pt}def{\isacharparenright}{\kern0pt}\isanewline
\ \ \ \ \ \isacommand{apply}\isamarkupfalse%
{\isacharparenleft}{\kern0pt}rule\ subsetI{\isacharcomma}{\kern0pt}\ simp{\isacharcomma}{\kern0pt}\ rule\ disjI{\isadigit{1}}{\isacharcomma}{\kern0pt}\ force{\isacharcomma}{\kern0pt}\ force{\isacharparenright}{\kern0pt}\isanewline
\ \ \ \ \isacommand{done}\isamarkupfalse%
\isanewline
\ \ \isacommand{then}\isamarkupfalse%
\ \isacommand{show}\isamarkupfalse%
\ {\isacharquery}{\kern0pt}thesis\ \isanewline
\ \ \ \ \isacommand{using}\isamarkupfalse%
\ assms\isanewline
\ \ \ \ \isacommand{by}\isamarkupfalse%
\ auto\isanewline
\isacommand{qed}\isamarkupfalse%
%
\endisatagproof
{\isafoldproof}%
%
\isadelimproof
\isanewline
%
\endisadelimproof
\isanewline
\isacommand{lemma}\isamarkupfalse%
\ Member{\isacharunderscore}{\kern0pt}{\isasymDelta}{\isadigit{0}}\ {\isacharbrackleft}{\kern0pt}simp{\isacharbrackright}{\kern0pt}\ {\isacharcolon}{\kern0pt}\ \isanewline
\ \ \isakeyword{fixes}\ n\ m\ \isanewline
\ \ \isakeyword{assumes}\ {\isachardoublequoteopen}n\ {\isasymin}\ nat{\isachardoublequoteclose}\ {\isachardoublequoteopen}m\ {\isasymin}\ nat{\isachardoublequoteclose}\ \isanewline
\ \ \isakeyword{shows}\ {\isachardoublequoteopen}Member{\isacharparenleft}{\kern0pt}n{\isacharcomma}{\kern0pt}\ m{\isacharparenright}{\kern0pt}\ {\isasymin}\ {\isasymDelta}{\isadigit{0}}{\isachardoublequoteclose}\ \isanewline
%
\isadelimproof
\ \ %
\endisadelimproof
%
\isatagproof
\isacommand{unfolding}\isamarkupfalse%
\ {\isasymDelta}{\isadigit{0}}{\isacharunderscore}{\kern0pt}def\ \isanewline
\ \ \isacommand{apply}\isamarkupfalse%
\ simp\isanewline
\ \ \isacommand{apply}\isamarkupfalse%
{\isacharparenleft}{\kern0pt}rule{\isacharunderscore}{\kern0pt}tac\ x{\isacharequal}{\kern0pt}{\isadigit{0}}\ \isakeyword{in}\ bexI{\isacharparenright}{\kern0pt}\isanewline
\ \ \isacommand{unfolding}\isamarkupfalse%
\ {\isasymDelta}{\isadigit{0}}{\isacharunderscore}{\kern0pt}base{\isacharunderscore}{\kern0pt}def\ \isanewline
\ \ \isacommand{using}\isamarkupfalse%
\ assms\isanewline
\ \ \isacommand{by}\isamarkupfalse%
\ auto%
\endisatagproof
{\isafoldproof}%
%
\isadelimproof
\isanewline
%
\endisadelimproof
\isanewline
\isacommand{lemma}\isamarkupfalse%
\ Equal{\isacharunderscore}{\kern0pt}{\isasymDelta}{\isadigit{0}}\ {\isacharbrackleft}{\kern0pt}simp{\isacharbrackright}{\kern0pt}\ {\isacharcolon}{\kern0pt}\ \isanewline
\ \ \isakeyword{fixes}\ n\ m\ \isanewline
\ \ \isakeyword{assumes}\ {\isachardoublequoteopen}n\ {\isasymin}\ nat{\isachardoublequoteclose}\ {\isachardoublequoteopen}m\ {\isasymin}\ nat{\isachardoublequoteclose}\ \isanewline
\ \ \isakeyword{shows}\ {\isachardoublequoteopen}Equal{\isacharparenleft}{\kern0pt}n{\isacharcomma}{\kern0pt}\ m{\isacharparenright}{\kern0pt}\ {\isasymin}\ {\isasymDelta}{\isadigit{0}}{\isachardoublequoteclose}\ \isanewline
%
\isadelimproof
\ \ %
\endisadelimproof
%
\isatagproof
\isacommand{unfolding}\isamarkupfalse%
\ {\isasymDelta}{\isadigit{0}}{\isacharunderscore}{\kern0pt}def\ \isanewline
\ \ \isacommand{apply}\isamarkupfalse%
\ simp\isanewline
\ \ \isacommand{apply}\isamarkupfalse%
{\isacharparenleft}{\kern0pt}rule{\isacharunderscore}{\kern0pt}tac\ x{\isacharequal}{\kern0pt}{\isadigit{0}}\ \isakeyword{in}\ bexI{\isacharparenright}{\kern0pt}\isanewline
\ \ \isacommand{unfolding}\isamarkupfalse%
\ {\isasymDelta}{\isadigit{0}}{\isacharunderscore}{\kern0pt}base{\isacharunderscore}{\kern0pt}def\ \isanewline
\ \ \isacommand{using}\isamarkupfalse%
\ assms\isanewline
\ \ \isacommand{by}\isamarkupfalse%
\ auto%
\endisatagproof
{\isafoldproof}%
%
\isadelimproof
\isanewline
%
\endisadelimproof
\isanewline
\isacommand{lemma}\isamarkupfalse%
\ Nand{\isacharunderscore}{\kern0pt}{\isasymDelta}{\isadigit{0}}\ {\isacharbrackleft}{\kern0pt}simp{\isacharbrackright}{\kern0pt}\ {\isacharcolon}{\kern0pt}\ \isanewline
\ \ \isakeyword{fixes}\ {\isasymphi}\ {\isasympsi}\isanewline
\ \ \isakeyword{assumes}\ {\isachardoublequoteopen}{\isasymphi}\ {\isasymin}\ {\isasymDelta}{\isadigit{0}}{\isachardoublequoteclose}\ {\isachardoublequoteopen}{\isasympsi}\ {\isasymin}\ {\isasymDelta}{\isadigit{0}}{\isachardoublequoteclose}\ \isanewline
\ \ \isakeyword{shows}\ {\isachardoublequoteopen}Nand{\isacharparenleft}{\kern0pt}{\isasymphi}{\isacharcomma}{\kern0pt}\ {\isasympsi}{\isacharparenright}{\kern0pt}\ {\isasymin}\ {\isasymDelta}{\isadigit{0}}{\isachardoublequoteclose}\ \isanewline
%
\isadelimproof
%
\endisadelimproof
%
\isatagproof
\isacommand{proof}\isamarkupfalse%
\ {\isacharminus}{\kern0pt}\isanewline
\ \ \isacommand{obtain}\isamarkupfalse%
\ n\ \isakeyword{where}\ nH{\isacharcolon}{\kern0pt}{\isachardoublequoteopen}{\isasymphi}\ {\isasymin}\ {\isasymDelta}{\isadigit{0}}{\isacharunderscore}{\kern0pt}from{\isacharcircum}{\kern0pt}n{\isacharparenleft}{\kern0pt}{\isasymDelta}{\isadigit{0}}{\isacharunderscore}{\kern0pt}base{\isacharparenright}{\kern0pt}{\isachardoublequoteclose}\ {\isachardoublequoteopen}n\ {\isasymin}\ nat{\isachardoublequoteclose}\ \isanewline
\ \ \ \ \isacommand{using}\isamarkupfalse%
\ assms\isanewline
\ \ \ \ \isacommand{unfolding}\isamarkupfalse%
\ {\isasymDelta}{\isadigit{0}}{\isacharunderscore}{\kern0pt}def\ \isanewline
\ \ \ \ \isacommand{by}\isamarkupfalse%
\ auto\isanewline
\isanewline
\ \ \isacommand{obtain}\isamarkupfalse%
\ m\ \isakeyword{where}\ mH{\isacharcolon}{\kern0pt}{\isachardoublequoteopen}{\isasympsi}\ {\isasymin}\ {\isasymDelta}{\isadigit{0}}{\isacharunderscore}{\kern0pt}from{\isacharcircum}{\kern0pt}m{\isacharparenleft}{\kern0pt}{\isasymDelta}{\isadigit{0}}{\isacharunderscore}{\kern0pt}base{\isacharparenright}{\kern0pt}{\isachardoublequoteclose}\ {\isachardoublequoteopen}m\ {\isasymin}\ nat{\isachardoublequoteclose}\ \isanewline
\ \ \ \ \isacommand{using}\isamarkupfalse%
\ assms\isanewline
\ \ \ \ \isacommand{unfolding}\isamarkupfalse%
\ {\isasymDelta}{\isadigit{0}}{\isacharunderscore}{\kern0pt}def\ \isanewline
\ \ \ \ \isacommand{by}\isamarkupfalse%
\ auto\isanewline
\isanewline
\ \ \isacommand{define}\isamarkupfalse%
\ k\ \isakeyword{where}\ {\isachardoublequoteopen}k\ {\isasymequiv}\ n\ {\isasymunion}\ m{\isachardoublequoteclose}\isanewline
\isanewline
\ \ \isacommand{have}\isamarkupfalse%
\ knat{\isacharcolon}{\kern0pt}\ {\isachardoublequoteopen}k\ {\isasymin}\ nat{\isachardoublequoteclose}\ \isacommand{using}\isamarkupfalse%
\ Un{\isacharunderscore}{\kern0pt}nat{\isacharunderscore}{\kern0pt}type\ nH\ mH\ k{\isacharunderscore}{\kern0pt}def\ \isacommand{by}\isamarkupfalse%
\ auto\isanewline
\ \ \isacommand{have}\isamarkupfalse%
\ mnle{\isacharcolon}{\kern0pt}\ {\isachardoublequoteopen}m\ {\isasymle}\ k\ {\isasymand}\ n\ {\isasymle}\ k{\isachardoublequoteclose}\ \isacommand{using}\isamarkupfalse%
\ k{\isacharunderscore}{\kern0pt}def\ max{\isacharunderscore}{\kern0pt}le{\isadigit{1}}\ max{\isacharunderscore}{\kern0pt}le{\isadigit{2}}\ nH\ mH\ \isacommand{by}\isamarkupfalse%
\ auto\isanewline
\ \ \isanewline
\ \ \isacommand{have}\isamarkupfalse%
\ {\isachardoublequoteopen}{\isasymphi}\ {\isasymin}\ {\isasymDelta}{\isadigit{0}}{\isacharunderscore}{\kern0pt}from{\isacharcircum}{\kern0pt}k{\isacharparenleft}{\kern0pt}{\isasymDelta}{\isadigit{0}}{\isacharunderscore}{\kern0pt}base{\isacharparenright}{\kern0pt}\ {\isasymand}\ {\isasympsi}\ {\isasymin}\ {\isasymDelta}{\isadigit{0}}{\isacharunderscore}{\kern0pt}from{\isacharcircum}{\kern0pt}k{\isacharparenleft}{\kern0pt}{\isasymDelta}{\isadigit{0}}{\isacharunderscore}{\kern0pt}base{\isacharparenright}{\kern0pt}{\isachardoublequoteclose}\isanewline
\ \ \ \ \isacommand{using}\isamarkupfalse%
\ mnle\ {\isasymDelta}{\isadigit{0}}{\isacharunderscore}{\kern0pt}from{\isacharunderscore}{\kern0pt}increasing\ knat\ mH\ nH\ \isanewline
\ \ \ \ \isacommand{by}\isamarkupfalse%
\ auto\isanewline
\ \ \isacommand{then}\isamarkupfalse%
\ \isacommand{have}\isamarkupfalse%
\ {\isachardoublequoteopen}Nand{\isacharparenleft}{\kern0pt}{\isasymphi}{\isacharcomma}{\kern0pt}\ {\isasympsi}{\isacharparenright}{\kern0pt}\ {\isasymin}\ {\isasymDelta}{\isadigit{0}}{\isacharunderscore}{\kern0pt}from{\isacharcircum}{\kern0pt}succ{\isacharparenleft}{\kern0pt}k{\isacharparenright}{\kern0pt}{\isacharparenleft}{\kern0pt}{\isasymDelta}{\isadigit{0}}{\isacharunderscore}{\kern0pt}base{\isacharparenright}{\kern0pt}{\isachardoublequoteclose}\isanewline
\ \ \ \ \isacommand{apply}\isamarkupfalse%
\ simp\isanewline
\ \ \ \ \isacommand{apply}\isamarkupfalse%
{\isacharparenleft}{\kern0pt}subst\ {\isasymDelta}{\isadigit{0}}{\isacharunderscore}{\kern0pt}from{\isacharunderscore}{\kern0pt}def{\isacharparenright}{\kern0pt}\isanewline
\ \ \ \ \isacommand{by}\isamarkupfalse%
\ auto\isanewline
\ \ \isacommand{then}\isamarkupfalse%
\ \isacommand{show}\isamarkupfalse%
\ {\isacharquery}{\kern0pt}thesis\ \isanewline
\ \ \ \ \isacommand{unfolding}\isamarkupfalse%
\ {\isasymDelta}{\isadigit{0}}{\isacharunderscore}{\kern0pt}def\isanewline
\ \ \ \ \isacommand{using}\isamarkupfalse%
\ knat\ \isanewline
\ \ \ \ \isacommand{by}\isamarkupfalse%
\ blast\isanewline
\isacommand{qed}\isamarkupfalse%
%
\endisatagproof
{\isafoldproof}%
%
\isadelimproof
\ \isanewline
%
\endisadelimproof
\isanewline
\isacommand{lemma}\isamarkupfalse%
\ Neg{\isacharunderscore}{\kern0pt}{\isasymDelta}{\isadigit{0}}\ {\isacharbrackleft}{\kern0pt}simp{\isacharbrackright}{\kern0pt}{\isacharcolon}{\kern0pt}\ \isanewline
\ \ \isakeyword{fixes}\ {\isasymphi}\ \isanewline
\ \ \isakeyword{assumes}\ {\isachardoublequoteopen}{\isasymphi}\ {\isasymin}\ {\isasymDelta}{\isadigit{0}}{\isachardoublequoteclose}\ \isanewline
\ \ \isakeyword{shows}\ {\isachardoublequoteopen}Neg{\isacharparenleft}{\kern0pt}{\isasymphi}{\isacharparenright}{\kern0pt}\ {\isasymin}\ {\isasymDelta}{\isadigit{0}}{\isachardoublequoteclose}\ \isanewline
%
\isadelimproof
\ \ %
\endisadelimproof
%
\isatagproof
\isacommand{unfolding}\isamarkupfalse%
\ Neg{\isacharunderscore}{\kern0pt}def\ \isanewline
\ \ \isacommand{using}\isamarkupfalse%
\ assms\isanewline
\ \ \isacommand{by}\isamarkupfalse%
\ auto%
\endisatagproof
{\isafoldproof}%
%
\isadelimproof
\isanewline
%
\endisadelimproof
\isanewline
\isacommand{lemma}\isamarkupfalse%
\ And{\isacharunderscore}{\kern0pt}{\isasymDelta}{\isadigit{0}}\ {\isacharbrackleft}{\kern0pt}simp{\isacharbrackright}{\kern0pt}{\isacharcolon}{\kern0pt}\ \isanewline
\ \ \isakeyword{fixes}\ {\isasymphi}\ {\isasympsi}\isanewline
\ \ \isakeyword{assumes}\ {\isachardoublequoteopen}{\isasymphi}\ {\isasymin}\ {\isasymDelta}{\isadigit{0}}{\isachardoublequoteclose}\ {\isachardoublequoteopen}{\isasympsi}\ {\isasymin}\ {\isasymDelta}{\isadigit{0}}{\isachardoublequoteclose}\isanewline
\ \ \isakeyword{shows}\ {\isachardoublequoteopen}And{\isacharparenleft}{\kern0pt}{\isasymphi}{\isacharcomma}{\kern0pt}\ {\isasympsi}{\isacharparenright}{\kern0pt}\ {\isasymin}\ {\isasymDelta}{\isadigit{0}}{\isachardoublequoteclose}\ \isanewline
%
\isadelimproof
\ \ %
\endisadelimproof
%
\isatagproof
\isacommand{unfolding}\isamarkupfalse%
\ And{\isacharunderscore}{\kern0pt}def\ \isanewline
\ \ \isacommand{using}\isamarkupfalse%
\ assms\isanewline
\ \ \isacommand{by}\isamarkupfalse%
\ auto%
\endisatagproof
{\isafoldproof}%
%
\isadelimproof
\isanewline
%
\endisadelimproof
\isanewline
\isacommand{lemma}\isamarkupfalse%
\ Or{\isacharunderscore}{\kern0pt}{\isasymDelta}{\isadigit{0}}\ {\isacharbrackleft}{\kern0pt}simp{\isacharbrackright}{\kern0pt}\ {\isacharcolon}{\kern0pt}\ \isanewline
\ \ \isakeyword{fixes}\ {\isasymphi}\ {\isasympsi}\isanewline
\ \ \isakeyword{assumes}\ {\isachardoublequoteopen}{\isasymphi}\ {\isasymin}\ {\isasymDelta}{\isadigit{0}}{\isachardoublequoteclose}\ {\isachardoublequoteopen}{\isasympsi}\ {\isasymin}\ {\isasymDelta}{\isadigit{0}}{\isachardoublequoteclose}\isanewline
\ \ \isakeyword{shows}\ {\isachardoublequoteopen}Or{\isacharparenleft}{\kern0pt}{\isasymphi}{\isacharcomma}{\kern0pt}\ {\isasympsi}{\isacharparenright}{\kern0pt}\ {\isasymin}\ {\isasymDelta}{\isadigit{0}}{\isachardoublequoteclose}\ \isanewline
%
\isadelimproof
\ \ %
\endisadelimproof
%
\isatagproof
\isacommand{unfolding}\isamarkupfalse%
\ Or{\isacharunderscore}{\kern0pt}def\ \isanewline
\ \ \isacommand{using}\isamarkupfalse%
\ assms\isanewline
\ \ \isacommand{by}\isamarkupfalse%
\ auto%
\endisatagproof
{\isafoldproof}%
%
\isadelimproof
\isanewline
%
\endisadelimproof
\isanewline
\isacommand{lemma}\isamarkupfalse%
\ Implies{\isacharunderscore}{\kern0pt}{\isasymDelta}{\isadigit{0}}\ {\isacharbrackleft}{\kern0pt}simp{\isacharbrackright}{\kern0pt}\ {\isacharcolon}{\kern0pt}\ \isanewline
\ \ \isakeyword{fixes}\ {\isasymphi}\ {\isasympsi}\isanewline
\ \ \isakeyword{assumes}\ {\isachardoublequoteopen}{\isasymphi}\ {\isasymin}\ {\isasymDelta}{\isadigit{0}}{\isachardoublequoteclose}\ {\isachardoublequoteopen}{\isasympsi}\ {\isasymin}\ {\isasymDelta}{\isadigit{0}}{\isachardoublequoteclose}\isanewline
\ \ \isakeyword{shows}\ {\isachardoublequoteopen}Implies{\isacharparenleft}{\kern0pt}{\isasymphi}{\isacharcomma}{\kern0pt}\ {\isasympsi}{\isacharparenright}{\kern0pt}\ {\isasymin}\ {\isasymDelta}{\isadigit{0}}{\isachardoublequoteclose}\ \isanewline
%
\isadelimproof
\ \ %
\endisadelimproof
%
\isatagproof
\isacommand{unfolding}\isamarkupfalse%
\ Implies{\isacharunderscore}{\kern0pt}def\ \isanewline
\ \ \isacommand{using}\isamarkupfalse%
\ assms\isanewline
\ \ \isacommand{by}\isamarkupfalse%
\ auto%
\endisatagproof
{\isafoldproof}%
%
\isadelimproof
\isanewline
%
\endisadelimproof
\ \ \isanewline
\isacommand{lemma}\isamarkupfalse%
\ Iff{\isacharunderscore}{\kern0pt}{\isasymDelta}{\isadigit{0}}\ {\isacharbrackleft}{\kern0pt}simp{\isacharbrackright}{\kern0pt}\ {\isacharcolon}{\kern0pt}\ \isanewline
\ \ \isakeyword{fixes}\ {\isasymphi}\ {\isasympsi}\isanewline
\ \ \isakeyword{assumes}\ {\isachardoublequoteopen}{\isasymphi}\ {\isasymin}\ {\isasymDelta}{\isadigit{0}}{\isachardoublequoteclose}\ {\isachardoublequoteopen}{\isasympsi}\ {\isasymin}\ {\isasymDelta}{\isadigit{0}}{\isachardoublequoteclose}\isanewline
\ \ \isakeyword{shows}\ {\isachardoublequoteopen}Iff{\isacharparenleft}{\kern0pt}{\isasymphi}{\isacharcomma}{\kern0pt}\ {\isasympsi}{\isacharparenright}{\kern0pt}\ {\isasymin}\ {\isasymDelta}{\isadigit{0}}{\isachardoublequoteclose}\ \isanewline
%
\isadelimproof
\ \ %
\endisadelimproof
%
\isatagproof
\isacommand{unfolding}\isamarkupfalse%
\ Iff{\isacharunderscore}{\kern0pt}def\ \isanewline
\ \ \isacommand{using}\isamarkupfalse%
\ assms\isanewline
\ \ \isacommand{by}\isamarkupfalse%
\ auto%
\endisatagproof
{\isafoldproof}%
%
\isadelimproof
\isanewline
%
\endisadelimproof
\isanewline
\isacommand{lemma}\isamarkupfalse%
\ BExists{\isacharunderscore}{\kern0pt}{\isasymDelta}{\isadigit{0}}{\isacharbrackleft}{\kern0pt}simp{\isacharbrackright}{\kern0pt}\ {\isacharcolon}{\kern0pt}\ \isanewline
\ \ \isakeyword{fixes}\ {\isasymphi}\ n\ \isanewline
\ \ \isakeyword{assumes}\ {\isachardoublequoteopen}{\isasymphi}\ {\isasymin}\ {\isasymDelta}{\isadigit{0}}{\isachardoublequoteclose}\ {\isachardoublequoteopen}n\ {\isasymin}\ nat{\isachardoublequoteclose}\ \isanewline
\ \ \isakeyword{shows}\ {\isachardoublequoteopen}BExists{\isacharparenleft}{\kern0pt}n{\isacharcomma}{\kern0pt}\ {\isasymphi}{\isacharparenright}{\kern0pt}\ {\isasymin}\ {\isasymDelta}{\isadigit{0}}{\isachardoublequoteclose}\ \isanewline
%
\isadelimproof
%
\endisadelimproof
%
\isatagproof
\isacommand{proof}\isamarkupfalse%
\ {\isacharminus}{\kern0pt}\ \isanewline
\ \ \isacommand{obtain}\isamarkupfalse%
\ m\ \isakeyword{where}\ mH{\isacharcolon}{\kern0pt}\ {\isachardoublequoteopen}{\isasymphi}\ {\isasymin}\ {\isasymDelta}{\isadigit{0}}{\isacharunderscore}{\kern0pt}from{\isacharcircum}{\kern0pt}m{\isacharparenleft}{\kern0pt}{\isasymDelta}{\isadigit{0}}{\isacharunderscore}{\kern0pt}base{\isacharparenright}{\kern0pt}{\isachardoublequoteclose}\ {\isachardoublequoteopen}m\ {\isasymin}\ nat{\isachardoublequoteclose}\ \isacommand{using}\isamarkupfalse%
\ assms\ \isacommand{unfolding}\isamarkupfalse%
\ {\isasymDelta}{\isadigit{0}}{\isacharunderscore}{\kern0pt}def\ \isacommand{by}\isamarkupfalse%
\ auto\isanewline
\ \ \isacommand{then}\isamarkupfalse%
\ \isacommand{have}\isamarkupfalse%
\ {\isachardoublequoteopen}BExists{\isacharparenleft}{\kern0pt}n{\isacharcomma}{\kern0pt}\ {\isasymphi}{\isacharparenright}{\kern0pt}\ {\isasymin}\ {\isasymDelta}{\isadigit{0}}{\isacharunderscore}{\kern0pt}from{\isacharcircum}{\kern0pt}succ{\isacharparenleft}{\kern0pt}m{\isacharparenright}{\kern0pt}{\isacharparenleft}{\kern0pt}{\isasymDelta}{\isadigit{0}}{\isacharunderscore}{\kern0pt}base{\isacharparenright}{\kern0pt}{\isachardoublequoteclose}\isanewline
\ \ \ \ \isacommand{apply}\isamarkupfalse%
\ simp\isanewline
\ \ \ \ \isacommand{apply}\isamarkupfalse%
{\isacharparenleft}{\kern0pt}subst\ {\isasymDelta}{\isadigit{0}}{\isacharunderscore}{\kern0pt}from{\isacharunderscore}{\kern0pt}def{\isacharcomma}{\kern0pt}\ simp{\isacharcomma}{\kern0pt}\ rule\ disjI{\isadigit{2}}{\isacharcomma}{\kern0pt}\ rule\ disjI{\isadigit{2}}{\isacharparenright}{\kern0pt}\isanewline
\ \ \ \ \isacommand{unfolding}\isamarkupfalse%
\ BExists{\isacharunderscore}{\kern0pt}def\isanewline
\ \ \ \ \isacommand{apply}\isamarkupfalse%
{\isacharparenleft}{\kern0pt}rule\ conjI{\isacharparenright}{\kern0pt}\isanewline
\ \ \ \ \isacommand{unfolding}\isamarkupfalse%
\ BExists{\isacharprime}{\kern0pt}{\isacharunderscore}{\kern0pt}def\isanewline
\ \ \ \ \isacommand{using}\isamarkupfalse%
\ assms\ {\isasymDelta}{\isadigit{0}}{\isacharunderscore}{\kern0pt}subset\isanewline
\ \ \ \ \ \isacommand{apply}\isamarkupfalse%
\ force\isanewline
\ \ \ \ \isacommand{apply}\isamarkupfalse%
{\isacharparenleft}{\kern0pt}rule{\isacharunderscore}{\kern0pt}tac\ x{\isacharequal}{\kern0pt}{\isachardoublequoteopen}succ{\isacharparenleft}{\kern0pt}n{\isacharparenright}{\kern0pt}{\isachardoublequoteclose}\ \isakeyword{in}\ bexI{\isacharcomma}{\kern0pt}\ rule\ conjI{\isacharcomma}{\kern0pt}\ simp{\isacharparenright}{\kern0pt}\isanewline
\ \ \ \ \ \isacommand{apply}\isamarkupfalse%
{\isacharparenleft}{\kern0pt}rule{\isacharunderscore}{\kern0pt}tac\ x{\isacharequal}{\kern0pt}{\isachardoublequoteopen}{\isasymphi}{\isachardoublequoteclose}\ \isakeyword{in}\ bexI{\isacharcomma}{\kern0pt}\ simp{\isacharcomma}{\kern0pt}\ simp{\isacharcomma}{\kern0pt}\ simp\ add{\isacharcolon}{\kern0pt}assms{\isacharparenright}{\kern0pt}\isanewline
\ \ \ \ \isacommand{done}\isamarkupfalse%
\isanewline
\ \ \isacommand{then}\isamarkupfalse%
\ \isacommand{show}\isamarkupfalse%
\ {\isacharquery}{\kern0pt}thesis\ \isanewline
\ \ \ \ \isacommand{using}\isamarkupfalse%
\ assms\ mH\isanewline
\ \ \ \ \isacommand{unfolding}\isamarkupfalse%
\ {\isasymDelta}{\isadigit{0}}{\isacharunderscore}{\kern0pt}def\isanewline
\ \ \ \ \isacommand{by}\isamarkupfalse%
\ blast\isanewline
\isacommand{qed}\isamarkupfalse%
%
\endisatagproof
{\isafoldproof}%
%
\isadelimproof
\isanewline
%
\endisadelimproof
\isanewline
\isacommand{lemma}\isamarkupfalse%
\ BExists{\isacharunderscore}{\kern0pt}formula\ {\isacharcolon}{\kern0pt}\ \isanewline
\ \ \isakeyword{fixes}\ {\isasymphi}\ n\ \isanewline
\ \ \isakeyword{assumes}\ {\isachardoublequoteopen}{\isasymphi}\ {\isasymin}\ {\isasymDelta}{\isadigit{0}}{\isachardoublequoteclose}\ {\isachardoublequoteopen}n\ {\isasymin}\ nat{\isachardoublequoteclose}\ \isanewline
\ \ \isakeyword{shows}\ {\isachardoublequoteopen}BExists{\isacharparenleft}{\kern0pt}n{\isacharcomma}{\kern0pt}\ {\isasymphi}{\isacharparenright}{\kern0pt}\ {\isasymin}\ formula{\isachardoublequoteclose}\ \isanewline
%
\isadelimproof
\ \ %
\endisadelimproof
%
\isatagproof
\isacommand{using}\isamarkupfalse%
\ assms\ {\isasymDelta}{\isadigit{0}}{\isacharunderscore}{\kern0pt}subset\isanewline
\ \ \isacommand{by}\isamarkupfalse%
\ auto%
\endisatagproof
{\isafoldproof}%
%
\isadelimproof
\isanewline
%
\endisadelimproof
\isanewline
\isacommand{lemma}\isamarkupfalse%
\ BForall{\isacharunderscore}{\kern0pt}{\isasymDelta}{\isadigit{0}}\ {\isacharbrackleft}{\kern0pt}simp{\isacharbrackright}{\kern0pt}\ {\isacharcolon}{\kern0pt}\ \isanewline
\ \ \isakeyword{fixes}\ {\isasymphi}\ n\ \isanewline
\ \ \isakeyword{assumes}\ {\isachardoublequoteopen}{\isasymphi}\ {\isasymin}\ {\isasymDelta}{\isadigit{0}}{\isachardoublequoteclose}\ {\isachardoublequoteopen}n\ {\isasymin}\ nat{\isachardoublequoteclose}\ \isanewline
\ \ \isakeyword{shows}\ {\isachardoublequoteopen}BForall{\isacharparenleft}{\kern0pt}n{\isacharcomma}{\kern0pt}\ {\isasymphi}{\isacharparenright}{\kern0pt}\ {\isasymin}\ {\isasymDelta}{\isadigit{0}}{\isachardoublequoteclose}\ \isanewline
%
\isadelimproof
\ \ %
\endisadelimproof
%
\isatagproof
\isacommand{unfolding}\isamarkupfalse%
\ BForall{\isacharunderscore}{\kern0pt}def\ \isanewline
\ \ \isacommand{using}\isamarkupfalse%
\ assms\isanewline
\ \ \isacommand{by}\isamarkupfalse%
\ auto%
\endisatagproof
{\isafoldproof}%
%
\isadelimproof
\isanewline
%
\endisadelimproof
\isanewline
\isacommand{lemma}\isamarkupfalse%
\ {\isasymDelta}{\isadigit{0}}{\isacharunderscore}{\kern0pt}sats{\isacharunderscore}{\kern0pt}iff\ {\isacharcolon}{\kern0pt}\ \isanewline
\ \ \isakeyword{fixes}\ A\ B\ env\ {\isasymphi}\isanewline
\ \ \isakeyword{assumes}\ {\isachardoublequoteopen}A\ {\isasymsubseteq}\ B{\isachardoublequoteclose}\ {\isachardoublequoteopen}env\ {\isasymin}\ list{\isacharparenleft}{\kern0pt}A{\isacharparenright}{\kern0pt}{\isachardoublequoteclose}\ {\isachardoublequoteopen}{\isasymphi}\ {\isasymin}\ {\isasymDelta}{\isadigit{0}}{\isachardoublequoteclose}\ {\isachardoublequoteopen}Transset{\isacharparenleft}{\kern0pt}A{\isacharparenright}{\kern0pt}{\isachardoublequoteclose}\ {\isachardoublequoteopen}arity{\isacharparenleft}{\kern0pt}{\isasymphi}{\isacharparenright}{\kern0pt}\ {\isasymle}\ length{\isacharparenleft}{\kern0pt}env{\isacharparenright}{\kern0pt}{\isachardoublequoteclose}\ \isanewline
\ \ \isakeyword{shows}\ {\isachardoublequoteopen}sats{\isacharparenleft}{\kern0pt}A{\isacharcomma}{\kern0pt}\ {\isasymphi}{\isacharcomma}{\kern0pt}\ env{\isacharparenright}{\kern0pt}\ {\isasymlongleftrightarrow}\ sats{\isacharparenleft}{\kern0pt}B{\isacharcomma}{\kern0pt}\ {\isasymphi}{\isacharcomma}{\kern0pt}\ env{\isacharparenright}{\kern0pt}{\isachardoublequoteclose}\ \isanewline
%
\isadelimproof
%
\endisadelimproof
%
\isatagproof
\isacommand{proof}\isamarkupfalse%
\ {\isacharminus}{\kern0pt}\ \isanewline
\ \ \isacommand{have}\isamarkupfalse%
\ main\ {\isacharcolon}{\kern0pt}\ {\isachardoublequoteopen}{\isasymAnd}n{\isachardot}{\kern0pt}\ n\ {\isasymin}\ nat\ {\isasymLongrightarrow}\ {\isasymforall}{\isasymphi}\ {\isasymin}\ {\isasymDelta}{\isadigit{0}}{\isacharunderscore}{\kern0pt}from{\isacharcircum}{\kern0pt}n\ {\isacharparenleft}{\kern0pt}{\isasymDelta}{\isadigit{0}}{\isacharunderscore}{\kern0pt}base{\isacharparenright}{\kern0pt}{\isachardot}{\kern0pt}\ {\isacharparenleft}{\kern0pt}{\isasymforall}env\ {\isasymin}\ list{\isacharparenleft}{\kern0pt}A{\isacharparenright}{\kern0pt}{\isachardot}{\kern0pt}\ arity{\isacharparenleft}{\kern0pt}{\isasymphi}{\isacharparenright}{\kern0pt}\ {\isasymle}\ length{\isacharparenleft}{\kern0pt}env{\isacharparenright}{\kern0pt}\ {\isasymlongrightarrow}\ {\isacharparenleft}{\kern0pt}sats{\isacharparenleft}{\kern0pt}A{\isacharcomma}{\kern0pt}\ {\isasymphi}{\isacharcomma}{\kern0pt}\ env{\isacharparenright}{\kern0pt}\ {\isasymlongleftrightarrow}\ sats{\isacharparenleft}{\kern0pt}B{\isacharcomma}{\kern0pt}\ {\isasymphi}{\isacharcomma}{\kern0pt}\ env{\isacharparenright}{\kern0pt}{\isacharparenright}{\kern0pt}{\isacharparenright}{\kern0pt}{\isachardoublequoteclose}\ \isanewline
\ \ \isacommand{proof}\isamarkupfalse%
\ {\isacharparenleft}{\kern0pt}rule{\isacharunderscore}{\kern0pt}tac\ P{\isacharequal}{\kern0pt}{\isachardoublequoteopen}{\isasymlambda}n{\isachardot}{\kern0pt}\ {\isasymforall}{\isasymphi}\ {\isasymin}\ {\isasymDelta}{\isadigit{0}}{\isacharunderscore}{\kern0pt}from{\isacharcircum}{\kern0pt}n\ {\isacharparenleft}{\kern0pt}{\isasymDelta}{\isadigit{0}}{\isacharunderscore}{\kern0pt}base{\isacharparenright}{\kern0pt}{\isachardot}{\kern0pt}\ {\isacharparenleft}{\kern0pt}{\isasymforall}env\ {\isasymin}\ list{\isacharparenleft}{\kern0pt}A{\isacharparenright}{\kern0pt}{\isachardot}{\kern0pt}\ arity{\isacharparenleft}{\kern0pt}{\isasymphi}{\isacharparenright}{\kern0pt}\ {\isasymle}\ length{\isacharparenleft}{\kern0pt}env{\isacharparenright}{\kern0pt}\ {\isasymlongrightarrow}\ {\isacharparenleft}{\kern0pt}sats{\isacharparenleft}{\kern0pt}A{\isacharcomma}{\kern0pt}\ {\isasymphi}{\isacharcomma}{\kern0pt}\ env{\isacharparenright}{\kern0pt}\ {\isasymlongleftrightarrow}\ sats{\isacharparenleft}{\kern0pt}B{\isacharcomma}{\kern0pt}\ {\isasymphi}{\isacharcomma}{\kern0pt}\ env{\isacharparenright}{\kern0pt}{\isacharparenright}{\kern0pt}{\isacharparenright}{\kern0pt}{\isachardoublequoteclose}\ \isakeyword{in}\ nat{\isacharunderscore}{\kern0pt}induct{\isacharcomma}{\kern0pt}\ simp{\isacharparenright}{\kern0pt}\isanewline
\ \ \ \ \isacommand{show}\isamarkupfalse%
\ {\isachardoublequoteopen}{\isasymforall}{\isasymphi}\ {\isasymin}\ {\isasymDelta}{\isadigit{0}}{\isacharunderscore}{\kern0pt}from{\isacharcircum}{\kern0pt}{\isadigit{0}}\ {\isacharparenleft}{\kern0pt}{\isasymDelta}{\isadigit{0}}{\isacharunderscore}{\kern0pt}base{\isacharparenright}{\kern0pt}{\isachardot}{\kern0pt}\ {\isacharparenleft}{\kern0pt}{\isasymforall}env\ {\isasymin}\ list{\isacharparenleft}{\kern0pt}A{\isacharparenright}{\kern0pt}{\isachardot}{\kern0pt}\ arity{\isacharparenleft}{\kern0pt}{\isasymphi}{\isacharparenright}{\kern0pt}\ {\isasymle}\ length{\isacharparenleft}{\kern0pt}env{\isacharparenright}{\kern0pt}\ {\isasymlongrightarrow}\ {\isacharparenleft}{\kern0pt}sats{\isacharparenleft}{\kern0pt}A{\isacharcomma}{\kern0pt}\ {\isasymphi}{\isacharcomma}{\kern0pt}\ env{\isacharparenright}{\kern0pt}\ {\isasymlongleftrightarrow}\ sats{\isacharparenleft}{\kern0pt}B{\isacharcomma}{\kern0pt}\ {\isasymphi}{\isacharcomma}{\kern0pt}\ env{\isacharparenright}{\kern0pt}{\isacharparenright}{\kern0pt}{\isacharparenright}{\kern0pt}{\isachardoublequoteclose}\ \isanewline
\ \ \ \ \ \ \isacommand{unfolding}\isamarkupfalse%
\ {\isasymDelta}{\isadigit{0}}{\isacharunderscore}{\kern0pt}base{\isacharunderscore}{\kern0pt}def\isanewline
\ \ \ \ \ \ \isacommand{apply}\isamarkupfalse%
\ {\isacharparenleft}{\kern0pt}simp{\isacharcomma}{\kern0pt}\ clarify{\isacharparenright}{\kern0pt}\isanewline
\ \ \ \ \ \ \isacommand{apply}\isamarkupfalse%
{\isacharparenleft}{\kern0pt}rename{\isacharunderscore}{\kern0pt}tac\ {\isasymphi}\ n\ m\ env{\isacharcomma}{\kern0pt}\ subgoal{\isacharunderscore}{\kern0pt}tac\ {\isachardoublequoteopen}env\ {\isasymin}\ list{\isacharparenleft}{\kern0pt}B{\isacharparenright}{\kern0pt}{\isachardoublequoteclose}{\isacharparenright}{\kern0pt}\ \isanewline
\ \ \ \ \ \ \ \isacommand{apply}\isamarkupfalse%
{\isacharparenleft}{\kern0pt}rename{\isacharunderscore}{\kern0pt}tac\ {\isasymphi}\ n\ m\ env{\isacharcomma}{\kern0pt}\ case{\isacharunderscore}{\kern0pt}tac\ {\isachardoublequoteopen}{\isasymphi}\ {\isacharequal}{\kern0pt}\ Member{\isacharparenleft}{\kern0pt}n{\isacharcomma}{\kern0pt}\ m{\isacharparenright}{\kern0pt}{\isachardoublequoteclose}{\isacharcomma}{\kern0pt}\ simp{\isacharcomma}{\kern0pt}\ simp{\isacharparenright}{\kern0pt}\isanewline
\ \ \ \ \ \ \isacommand{apply}\isamarkupfalse%
{\isacharparenleft}{\kern0pt}rule\ subsetD{\isacharcomma}{\kern0pt}\ rule\ list{\isacharunderscore}{\kern0pt}mono{\isacharparenright}{\kern0pt}\isanewline
\ \ \ \ \ \ \isacommand{using}\isamarkupfalse%
\ assms\ \isanewline
\ \ \ \ \ \ \isacommand{by}\isamarkupfalse%
\ auto\isanewline
\ \ \isacommand{next}\isamarkupfalse%
\ \isanewline
\ \ \ \ \isacommand{fix}\isamarkupfalse%
\ n\ \isanewline
\ \ \ \ \isacommand{assume}\isamarkupfalse%
\ assms{\isadigit{1}}\ {\isacharcolon}{\kern0pt}\ {\isachardoublequoteopen}n\ {\isasymin}\ nat{\isachardoublequoteclose}\ {\isachardoublequoteopen}{\isasymforall}{\isasymphi}{\isasymin}\ {\isasymDelta}{\isadigit{0}}{\isacharunderscore}{\kern0pt}from{\isacharcircum}{\kern0pt}n\ {\isacharparenleft}{\kern0pt}{\isasymDelta}{\isadigit{0}}{\isacharunderscore}{\kern0pt}base{\isacharparenright}{\kern0pt}{\isachardot}{\kern0pt}{\isacharparenleft}{\kern0pt}{\isasymforall}env{\isasymin}list{\isacharparenleft}{\kern0pt}A{\isacharparenright}{\kern0pt}{\isachardot}{\kern0pt}\ arity{\isacharparenleft}{\kern0pt}{\isasymphi}{\isacharparenright}{\kern0pt}\ {\isasymle}\ length{\isacharparenleft}{\kern0pt}env{\isacharparenright}{\kern0pt}\ {\isasymlongrightarrow}\ {\isacharparenleft}{\kern0pt}A{\isacharcomma}{\kern0pt}\ env\ {\isasymTurnstile}\ {\isasymphi}\ {\isasymlongleftrightarrow}\ B{\isacharcomma}{\kern0pt}\ env\ {\isasymTurnstile}\ {\isasymphi}{\isacharparenright}{\kern0pt}{\isacharparenright}{\kern0pt}{\isachardoublequoteclose}\isanewline
\isanewline
\ \ \ \ \isacommand{define}\isamarkupfalse%
\ X\ \isakeyword{where}\ {\isachardoublequoteopen}X\ {\isasymequiv}\ {\isasymDelta}{\isadigit{0}}{\isacharunderscore}{\kern0pt}from{\isacharcircum}{\kern0pt}n\ {\isacharparenleft}{\kern0pt}{\isasymDelta}{\isadigit{0}}{\isacharunderscore}{\kern0pt}base{\isacharparenright}{\kern0pt}{\isachardoublequoteclose}\isanewline
\isanewline
\ \ \ \ \isacommand{have}\isamarkupfalse%
\ Xsubset\ {\isacharcolon}{\kern0pt}\ {\isachardoublequoteopen}X\ {\isasymsubseteq}\ formula{\isachardoublequoteclose}\ \isanewline
\ \ \ \ \ \ \isacommand{using}\isamarkupfalse%
\ {\isasymDelta}{\isadigit{0}}{\isacharunderscore}{\kern0pt}subset\ assms{\isadigit{1}}\isanewline
\ \ \ \ \ \ \isacommand{unfolding}\isamarkupfalse%
\ {\isasymDelta}{\isadigit{0}}{\isacharunderscore}{\kern0pt}def\ X{\isacharunderscore}{\kern0pt}def\ \isanewline
\ \ \ \ \ \ \isacommand{by}\isamarkupfalse%
\ auto\isanewline
\isanewline
\ \ \ \ \isacommand{have}\isamarkupfalse%
\ ih{\isadigit{1}}\ {\isacharcolon}{\kern0pt}\ {\isachardoublequoteopen}{\isasymAnd}env\ {\isasymphi}{\isachardot}{\kern0pt}\ {\isasymphi}\ {\isasymin}\ X\ {\isasymLongrightarrow}\ env\ {\isasymin}\ list{\isacharparenleft}{\kern0pt}A{\isacharparenright}{\kern0pt}\ {\isasymLongrightarrow}\ arity{\isacharparenleft}{\kern0pt}{\isasymphi}{\isacharparenright}{\kern0pt}\ {\isasymle}\ length{\isacharparenleft}{\kern0pt}env{\isacharparenright}{\kern0pt}\ {\isasymLongrightarrow}\ A{\isacharcomma}{\kern0pt}\ env\ {\isasymTurnstile}\ {\isasymphi}\ {\isasymLongrightarrow}\ B{\isacharcomma}{\kern0pt}\ env\ {\isasymTurnstile}\ {\isasymphi}{\isachardoublequoteclose}\ \isacommand{using}\isamarkupfalse%
\ assms{\isadigit{1}}\ X{\isacharunderscore}{\kern0pt}def\ \isacommand{by}\isamarkupfalse%
\ auto\isanewline
\ \ \ \ \isacommand{have}\isamarkupfalse%
\ ih{\isadigit{2}}\ {\isacharcolon}{\kern0pt}\ {\isachardoublequoteopen}{\isasymAnd}env\ {\isasymphi}{\isachardot}{\kern0pt}\ {\isasymphi}\ {\isasymin}\ X\ {\isasymLongrightarrow}\ env\ {\isasymin}\ list{\isacharparenleft}{\kern0pt}A{\isacharparenright}{\kern0pt}\ {\isasymLongrightarrow}\ arity{\isacharparenleft}{\kern0pt}{\isasymphi}{\isacharparenright}{\kern0pt}\ {\isasymle}\ length{\isacharparenleft}{\kern0pt}env{\isacharparenright}{\kern0pt}\ {\isasymLongrightarrow}\ B{\isacharcomma}{\kern0pt}\ env\ {\isasymTurnstile}\ {\isasymphi}\ {\isasymLongrightarrow}\ A{\isacharcomma}{\kern0pt}\ env\ {\isasymTurnstile}\ {\isasymphi}{\isachardoublequoteclose}\ \isacommand{using}\isamarkupfalse%
\ assms{\isadigit{1}}\ X{\isacharunderscore}{\kern0pt}def\ \isacommand{by}\isamarkupfalse%
\ auto\isanewline
\isanewline
\ \ \ \ \isacommand{have}\isamarkupfalse%
\ Hnand{\isacharcolon}{\kern0pt}\ {\isachardoublequoteopen}{\isasymAnd}{\isasymphi}\ {\isasympsi}\ env{\isachardot}{\kern0pt}\ {\isasymphi}\ {\isasymin}\ X\ {\isasymLongrightarrow}\ {\isasympsi}\ {\isasymin}\ X\ {\isasymLongrightarrow}\ env\ {\isasymin}\ list{\isacharparenleft}{\kern0pt}A{\isacharparenright}{\kern0pt}\ {\isasymLongrightarrow}\ arity{\isacharparenleft}{\kern0pt}Nand{\isacharparenleft}{\kern0pt}{\isasymphi}{\isacharcomma}{\kern0pt}\ {\isasympsi}{\isacharparenright}{\kern0pt}{\isacharparenright}{\kern0pt}\ {\isasymle}\ length{\isacharparenleft}{\kern0pt}env{\isacharparenright}{\kern0pt}\ {\isasymLongrightarrow}\ A{\isacharcomma}{\kern0pt}\ env\ {\isasymTurnstile}\ Nand{\isacharparenleft}{\kern0pt}{\isasymphi}{\isacharcomma}{\kern0pt}\ {\isasympsi}{\isacharparenright}{\kern0pt}\ {\isasymlongleftrightarrow}\ B{\isacharcomma}{\kern0pt}\ env\ {\isasymTurnstile}\ Nand{\isacharparenleft}{\kern0pt}{\isasymphi}{\isacharcomma}{\kern0pt}\ {\isasympsi}{\isacharparenright}{\kern0pt}{\isachardoublequoteclose}\ \isanewline
\ \ \ \ \isacommand{proof}\isamarkupfalse%
\ {\isacharminus}{\kern0pt}\ \isanewline
\ \ \ \ \ \ \isacommand{fix}\isamarkupfalse%
\ {\isasymphi}\ {\isasympsi}\ env\ \isanewline
\ \ \ \ \ \ \isacommand{assume}\isamarkupfalse%
\ assms{\isadigit{2}}\ {\isacharcolon}{\kern0pt}\ {\isachardoublequoteopen}{\isasymphi}\ {\isasymin}\ X{\isachardoublequoteclose}\ {\isachardoublequoteopen}{\isasympsi}\ {\isasymin}\ X{\isachardoublequoteclose}\ {\isachardoublequoteopen}env\ {\isasymin}\ list{\isacharparenleft}{\kern0pt}A{\isacharparenright}{\kern0pt}{\isachardoublequoteclose}\ {\isachardoublequoteopen}arity{\isacharparenleft}{\kern0pt}Nand{\isacharparenleft}{\kern0pt}{\isasymphi}{\isacharcomma}{\kern0pt}\ {\isasympsi}{\isacharparenright}{\kern0pt}{\isacharparenright}{\kern0pt}\ {\isasymle}\ length{\isacharparenleft}{\kern0pt}env{\isacharparenright}{\kern0pt}{\isachardoublequoteclose}\ \isanewline
\ \ \ \ \ \ \isacommand{have}\isamarkupfalse%
\ {\isachardoublequoteopen}arity{\isacharparenleft}{\kern0pt}{\isasymphi}{\isacharparenright}{\kern0pt}\ {\isasymle}\ length{\isacharparenleft}{\kern0pt}env{\isacharparenright}{\kern0pt}{\isachardoublequoteclose}\ \isanewline
\ \ \ \ \ \ \ \ \isacommand{apply}\isamarkupfalse%
{\isacharparenleft}{\kern0pt}rule{\isacharunderscore}{\kern0pt}tac\ j{\isacharequal}{\kern0pt}{\isachardoublequoteopen}arity{\isacharparenleft}{\kern0pt}Nand{\isacharparenleft}{\kern0pt}{\isasymphi}{\isacharcomma}{\kern0pt}\ {\isasympsi}{\isacharparenright}{\kern0pt}{\isacharparenright}{\kern0pt}{\isachardoublequoteclose}\ \isakeyword{in}\ le{\isacharunderscore}{\kern0pt}trans{\isacharparenright}{\kern0pt}\isanewline
\ \ \ \ \ \ \ \ \ \isacommand{apply}\isamarkupfalse%
\ simp\isanewline
\ \ \ \ \ \ \ \ \ \isacommand{apply}\isamarkupfalse%
{\isacharparenleft}{\kern0pt}rule\ max{\isacharunderscore}{\kern0pt}le{\isadigit{1}}{\isacharparenright}{\kern0pt}\isanewline
\ \ \ \ \ \ \ \ \isacommand{using}\isamarkupfalse%
\ assms{\isadigit{2}}\ Xsubset\ \isanewline
\ \ \ \ \ \ \ \ \isacommand{by}\isamarkupfalse%
\ auto\isanewline
\ \ \ \ \ \ \isacommand{then}\isamarkupfalse%
\ \isacommand{have}\isamarkupfalse%
\ iff{\isadigit{1}}{\isacharcolon}{\kern0pt}\ {\isachardoublequoteopen}A{\isacharcomma}{\kern0pt}\ env\ {\isasymTurnstile}\ {\isasymphi}\ {\isasymlongleftrightarrow}\ B{\isacharcomma}{\kern0pt}\ env\ {\isasymTurnstile}\ {\isasymphi}{\isachardoublequoteclose}\ \isanewline
\ \ \ \ \ \ \ \ \isacommand{using}\isamarkupfalse%
\ ih{\isadigit{1}}\ ih{\isadigit{2}}\ assms{\isadigit{2}}\ \isacommand{by}\isamarkupfalse%
\ auto\isanewline
\isanewline
\ \ \ \ \ \ \isacommand{have}\isamarkupfalse%
\ {\isachardoublequoteopen}arity{\isacharparenleft}{\kern0pt}{\isasympsi}{\isacharparenright}{\kern0pt}\ {\isasymle}\ length{\isacharparenleft}{\kern0pt}env{\isacharparenright}{\kern0pt}{\isachardoublequoteclose}\ \isanewline
\ \ \ \ \ \ \ \ \isacommand{apply}\isamarkupfalse%
{\isacharparenleft}{\kern0pt}rule{\isacharunderscore}{\kern0pt}tac\ j{\isacharequal}{\kern0pt}{\isachardoublequoteopen}arity{\isacharparenleft}{\kern0pt}Nand{\isacharparenleft}{\kern0pt}{\isasymphi}{\isacharcomma}{\kern0pt}\ {\isasympsi}{\isacharparenright}{\kern0pt}{\isacharparenright}{\kern0pt}{\isachardoublequoteclose}\ \isakeyword{in}\ le{\isacharunderscore}{\kern0pt}trans{\isacharparenright}{\kern0pt}\isanewline
\ \ \ \ \ \ \ \ \ \isacommand{apply}\isamarkupfalse%
\ simp\isanewline
\ \ \ \ \ \ \ \ \ \isacommand{apply}\isamarkupfalse%
{\isacharparenleft}{\kern0pt}rule\ max{\isacharunderscore}{\kern0pt}le{\isadigit{2}}{\isacharparenright}{\kern0pt}\isanewline
\ \ \ \ \ \ \ \ \isacommand{using}\isamarkupfalse%
\ assms{\isadigit{2}}\ Xsubset\ \isanewline
\ \ \ \ \ \ \ \ \isacommand{by}\isamarkupfalse%
\ auto\isanewline
\ \ \ \ \ \ \isacommand{then}\isamarkupfalse%
\ \isacommand{have}\isamarkupfalse%
\ iff{\isadigit{2}}{\isacharcolon}{\kern0pt}\ {\isachardoublequoteopen}A{\isacharcomma}{\kern0pt}\ env\ {\isasymTurnstile}\ {\isasympsi}\ {\isasymlongleftrightarrow}\ B{\isacharcomma}{\kern0pt}\ env\ {\isasymTurnstile}\ {\isasympsi}{\isachardoublequoteclose}\ \isanewline
\ \ \ \ \ \ \ \ \isacommand{using}\isamarkupfalse%
\ ih{\isadigit{1}}\ ih{\isadigit{2}}\ assms{\isadigit{2}}\ \isacommand{by}\isamarkupfalse%
\ auto\isanewline
\ \ \ \ \ \ \isanewline
\ \ \ \ \ \ \isacommand{have}\isamarkupfalse%
\ envinBlist\ {\isacharcolon}{\kern0pt}\ {\isachardoublequoteopen}env\ {\isasymin}\ list{\isacharparenleft}{\kern0pt}B{\isacharparenright}{\kern0pt}{\isachardoublequoteclose}\isanewline
\ \ \ \ \ \ \ \ \isacommand{apply}\isamarkupfalse%
{\isacharparenleft}{\kern0pt}rule\ subsetD{\isacharcomma}{\kern0pt}\ rule\ list{\isacharunderscore}{\kern0pt}mono{\isacharparenright}{\kern0pt}\isanewline
\ \ \ \ \ \ \ \ \isacommand{using}\isamarkupfalse%
\ assms{\isadigit{2}}\ assms\isanewline
\ \ \ \ \ \ \ \ \isacommand{by}\isamarkupfalse%
\ auto\isanewline
\isanewline
\ \ \ \ \ \ \isacommand{then}\isamarkupfalse%
\ \isacommand{show}\isamarkupfalse%
\ {\isachardoublequoteopen}A{\isacharcomma}{\kern0pt}\ env\ {\isasymTurnstile}\ Nand{\isacharparenleft}{\kern0pt}{\isasymphi}{\isacharcomma}{\kern0pt}\ {\isasympsi}{\isacharparenright}{\kern0pt}\ {\isasymlongleftrightarrow}\ B{\isacharcomma}{\kern0pt}\ env\ {\isasymTurnstile}\ Nand{\isacharparenleft}{\kern0pt}{\isasymphi}{\isacharcomma}{\kern0pt}\ {\isasympsi}{\isacharparenright}{\kern0pt}{\isachardoublequoteclose}\ \isanewline
\ \ \ \ \ \ \ \ \isacommand{using}\isamarkupfalse%
\ iff{\isadigit{1}}\ iff{\isadigit{2}}\ assms{\isadigit{2}}\ Xsubset\ \isanewline
\ \ \ \ \ \ \ \ \isacommand{by}\isamarkupfalse%
\ auto\isanewline
\ \ \ \ \isacommand{qed}\isamarkupfalse%
\isanewline
\isanewline
\ \ \ \ \isacommand{have}\isamarkupfalse%
\ Hbex\ {\isacharcolon}{\kern0pt}\ {\isachardoublequoteopen}{\isasymAnd}m\ {\isasymphi}\ env{\isachardot}{\kern0pt}\ m\ {\isasymin}\ nat\ {\isasymLongrightarrow}\ {\isasymphi}\ {\isasymin}\ X\ {\isasymLongrightarrow}\ env\ {\isasymin}\ list{\isacharparenleft}{\kern0pt}A{\isacharparenright}{\kern0pt}\ {\isasymLongrightarrow}\ arity{\isacharparenleft}{\kern0pt}BExists{\isacharprime}{\kern0pt}{\isacharparenleft}{\kern0pt}m{\isacharcomma}{\kern0pt}\ {\isasymphi}{\isacharparenright}{\kern0pt}{\isacharparenright}{\kern0pt}\ {\isasymle}\ length{\isacharparenleft}{\kern0pt}env{\isacharparenright}{\kern0pt}\ {\isasymLongrightarrow}\ A{\isacharcomma}{\kern0pt}\ env\ {\isasymTurnstile}\ BExists{\isacharprime}{\kern0pt}{\isacharparenleft}{\kern0pt}m{\isacharcomma}{\kern0pt}\ {\isasymphi}{\isacharparenright}{\kern0pt}\ {\isasymlongleftrightarrow}\ B{\isacharcomma}{\kern0pt}\ env\ {\isasymTurnstile}\ BExists{\isacharprime}{\kern0pt}{\isacharparenleft}{\kern0pt}m{\isacharcomma}{\kern0pt}\ {\isasymphi}{\isacharparenright}{\kern0pt}{\isachardoublequoteclose}\isanewline
\ \ \ \ \isacommand{proof}\isamarkupfalse%
\ {\isacharminus}{\kern0pt}\ \isanewline
\ \ \ \ \ \ \isacommand{fix}\isamarkupfalse%
\ m\ {\isasymphi}\ env\isanewline
\ \ \ \ \ \ \isacommand{assume}\isamarkupfalse%
\ assms{\isadigit{2}}\ {\isacharcolon}{\kern0pt}\ {\isachardoublequoteopen}m\ {\isasymin}\ nat{\isachardoublequoteclose}\ {\isachardoublequoteopen}{\isasymphi}\ {\isasymin}\ X{\isachardoublequoteclose}\ {\isachardoublequoteopen}env\ {\isasymin}\ list{\isacharparenleft}{\kern0pt}A{\isacharparenright}{\kern0pt}{\isachardoublequoteclose}\ {\isachardoublequoteopen}arity{\isacharparenleft}{\kern0pt}BExists{\isacharprime}{\kern0pt}{\isacharparenleft}{\kern0pt}m{\isacharcomma}{\kern0pt}\ {\isasymphi}{\isacharparenright}{\kern0pt}{\isacharparenright}{\kern0pt}\ {\isasymle}\ length{\isacharparenleft}{\kern0pt}env{\isacharparenright}{\kern0pt}{\isachardoublequoteclose}\isanewline
\isanewline
\ \ \ \ \ \ \isacommand{have}\isamarkupfalse%
\ {\isachardoublequoteopen}{\isasymphi}\ {\isasymin}\ {\isasymDelta}{\isadigit{0}}{\isachardoublequoteclose}\ \isanewline
\ \ \ \ \ \ \ \ \isacommand{unfolding}\isamarkupfalse%
\ {\isasymDelta}{\isadigit{0}}{\isacharunderscore}{\kern0pt}def\ \isanewline
\ \ \ \ \ \ \ \ \isacommand{using}\isamarkupfalse%
\ assms{\isadigit{1}}\ assms{\isadigit{2}}\ X{\isacharunderscore}{\kern0pt}def\ \isanewline
\ \ \ \ \ \ \ \ \isacommand{by}\isamarkupfalse%
\ auto\isanewline
\ \ \ \ \ \ \isacommand{then}\isamarkupfalse%
\ \isacommand{have}\isamarkupfalse%
\ phiformula{\isacharcolon}{\kern0pt}\ {\isachardoublequoteopen}{\isasymphi}\ {\isasymin}\ formula{\isachardoublequoteclose}\ \isanewline
\ \ \ \ \ \ \ \ \isacommand{using}\isamarkupfalse%
\ {\isasymDelta}{\isadigit{0}}{\isacharunderscore}{\kern0pt}subset\ \isanewline
\ \ \ \ \ \ \ \ \isacommand{by}\isamarkupfalse%
\ auto\isanewline
\isanewline
\ \ \ \ \ \ \isacommand{have}\isamarkupfalse%
\ envinBlist\ {\isacharcolon}{\kern0pt}\ {\isachardoublequoteopen}env\ {\isasymin}\ list{\isacharparenleft}{\kern0pt}B{\isacharparenright}{\kern0pt}{\isachardoublequoteclose}\isanewline
\ \ \ \ \ \ \ \ \isacommand{apply}\isamarkupfalse%
{\isacharparenleft}{\kern0pt}rule\ subsetD{\isacharcomma}{\kern0pt}\ rule\ list{\isacharunderscore}{\kern0pt}mono{\isacharparenright}{\kern0pt}\isanewline
\ \ \ \ \ \ \ \ \isacommand{using}\isamarkupfalse%
\ assms{\isadigit{2}}\ assms\isanewline
\ \ \ \ \ \ \ \ \isacommand{by}\isamarkupfalse%
\ auto\isanewline
\isanewline
\ \ \ \ \ \ \isacommand{have}\isamarkupfalse%
\ eq{\isacharcolon}{\kern0pt}\ {\isachardoublequoteopen}succ{\isacharparenleft}{\kern0pt}arity{\isacharparenleft}{\kern0pt}BExists{\isacharprime}{\kern0pt}{\isacharparenleft}{\kern0pt}m{\isacharcomma}{\kern0pt}\ {\isasymphi}{\isacharparenright}{\kern0pt}{\isacharparenright}{\kern0pt}{\isacharparenright}{\kern0pt}\ {\isacharequal}{\kern0pt}\ {\isadigit{1}}\ {\isasymunion}\ succ{\isacharparenleft}{\kern0pt}m{\isacharparenright}{\kern0pt}\ {\isasymunion}\ arity{\isacharparenleft}{\kern0pt}{\isasymphi}{\isacharparenright}{\kern0pt}{\isachardoublequoteclose}\isanewline
\ \ \ \ \ \ \ \ \isacommand{unfolding}\isamarkupfalse%
\ BExists{\isacharprime}{\kern0pt}{\isacharunderscore}{\kern0pt}def\ \isanewline
\ \ \ \ \ \ \ \ \isacommand{apply}\isamarkupfalse%
\ simp\isanewline
\ \ \ \ \ \ \ \ \isacommand{apply}\isamarkupfalse%
{\isacharparenleft}{\kern0pt}subst\ succ{\isacharunderscore}{\kern0pt}pred{\isacharunderscore}{\kern0pt}eq{\isacharparenright}{\kern0pt}\isanewline
\ \ \ \ \ \ \ \ \isacommand{apply}\isamarkupfalse%
{\isacharparenleft}{\kern0pt}rule\ Un{\isacharunderscore}{\kern0pt}nat{\isacharunderscore}{\kern0pt}type{\isacharcomma}{\kern0pt}\ rule\ Un{\isacharunderscore}{\kern0pt}nat{\isacharunderscore}{\kern0pt}type{\isacharcomma}{\kern0pt}\ simp{\isacharcomma}{\kern0pt}\ simp\ add{\isacharcolon}{\kern0pt}assms{\isadigit{2}}{\isacharparenright}{\kern0pt}\isanewline
\ \ \ \ \ \ \ \ \isacommand{using}\isamarkupfalse%
\ phiformula\ \isanewline
\ \ \ \ \ \ \ \ \isacommand{by}\isamarkupfalse%
\ auto\isanewline
\ \ \ \ \ \ \isacommand{have}\isamarkupfalse%
\ {\isachardoublequoteopen}arity{\isacharparenleft}{\kern0pt}{\isasymphi}{\isacharparenright}{\kern0pt}\ {\isasymle}\ succ{\isacharparenleft}{\kern0pt}arity{\isacharparenleft}{\kern0pt}BExists{\isacharprime}{\kern0pt}{\isacharparenleft}{\kern0pt}m{\isacharcomma}{\kern0pt}\ {\isasymphi}{\isacharparenright}{\kern0pt}{\isacharparenright}{\kern0pt}{\isacharparenright}{\kern0pt}{\isachardoublequoteclose}\isanewline
\ \ \ \ \ \ \ \ \isacommand{apply}\isamarkupfalse%
{\isacharparenleft}{\kern0pt}subst\ eq{\isacharparenright}{\kern0pt}\ \isanewline
\ \ \ \ \ \ \ \ \isacommand{apply}\isamarkupfalse%
{\isacharparenleft}{\kern0pt}rule\ max{\isacharunderscore}{\kern0pt}le{\isadigit{2}}{\isacharcomma}{\kern0pt}\ rule\ Ord{\isacharunderscore}{\kern0pt}Un{\isacharcomma}{\kern0pt}\ simp{\isacharcomma}{\kern0pt}\ simp{\isacharcomma}{\kern0pt}\ simp\ add{\isacharcolon}{\kern0pt}assms{\isadigit{2}}{\isacharparenright}{\kern0pt}\isanewline
\ \ \ \ \ \ \ \ \isacommand{using}\isamarkupfalse%
\ phiformula\ \isanewline
\ \ \ \ \ \ \ \ \isacommand{by}\isamarkupfalse%
\ auto\isanewline
\ \ \ \ \ \ \isacommand{then}\isamarkupfalse%
\ \isacommand{have}\isamarkupfalse%
\ arityle{\isacharcolon}{\kern0pt}\ \ {\isachardoublequoteopen}arity{\isacharparenleft}{\kern0pt}{\isasymphi}{\isacharparenright}{\kern0pt}\ {\isasymle}\ succ{\isacharparenleft}{\kern0pt}length{\isacharparenleft}{\kern0pt}env{\isacharparenright}{\kern0pt}{\isacharparenright}{\kern0pt}{\isachardoublequoteclose}\ \isanewline
\ \ \ \ \ \ \ \ \isacommand{by}\isamarkupfalse%
{\isacharparenleft}{\kern0pt}rule{\isacharunderscore}{\kern0pt}tac\ j{\isacharequal}{\kern0pt}{\isachardoublequoteopen}succ{\isacharparenleft}{\kern0pt}arity{\isacharparenleft}{\kern0pt}BExists{\isacharprime}{\kern0pt}{\isacharparenleft}{\kern0pt}m{\isacharcomma}{\kern0pt}\ {\isasymphi}{\isacharparenright}{\kern0pt}{\isacharparenright}{\kern0pt}{\isacharparenright}{\kern0pt}{\isachardoublequoteclose}\ \isakeyword{in}\ le{\isacharunderscore}{\kern0pt}trans{\isacharcomma}{\kern0pt}\ simp{\isacharcomma}{\kern0pt}\ simp\ add{\isacharcolon}{\kern0pt}assms{\isadigit{2}}{\isacharparenright}{\kern0pt}\isanewline
\isanewline
\ \ \ \ \ \ \isacommand{have}\isamarkupfalse%
\ mle\ {\isacharcolon}{\kern0pt}\ {\isachardoublequoteopen}m\ {\isasymle}\ length{\isacharparenleft}{\kern0pt}env{\isacharparenright}{\kern0pt}{\isachardoublequoteclose}\isanewline
\ \ \ \ \ \ \ \ \isacommand{apply}\isamarkupfalse%
{\isacharparenleft}{\kern0pt}rule{\isacharunderscore}{\kern0pt}tac\ b{\isacharequal}{\kern0pt}{\isachardoublequoteopen}succ{\isacharparenleft}{\kern0pt}arity{\isacharparenleft}{\kern0pt}BExists{\isacharprime}{\kern0pt}{\isacharparenleft}{\kern0pt}m{\isacharcomma}{\kern0pt}\ {\isasymphi}{\isacharparenright}{\kern0pt}{\isacharparenright}{\kern0pt}{\isacharparenright}{\kern0pt}{\isachardoublequoteclose}\ \isakeyword{in}\ lt{\isacharunderscore}{\kern0pt}le{\isacharunderscore}{\kern0pt}lt{\isacharparenright}{\kern0pt}\isanewline
\ \ \ \ \ \ \ \ \ \isacommand{apply}\isamarkupfalse%
{\isacharparenleft}{\kern0pt}subst\ eq{\isacharcomma}{\kern0pt}\ rule\ ltI{\isacharcomma}{\kern0pt}\ simp{\isacharparenright}{\kern0pt}\isanewline
\ \ \ \ \ \ \ \ \ \isacommand{apply}\isamarkupfalse%
{\isacharparenleft}{\kern0pt}rule\ Ord{\isacharunderscore}{\kern0pt}Un{\isacharcomma}{\kern0pt}\ rule\ Ord{\isacharunderscore}{\kern0pt}Un{\isacharcomma}{\kern0pt}\ simp{\isacharcomma}{\kern0pt}\ simp\ add{\isacharcolon}{\kern0pt}assms{\isadigit{2}}{\isacharparenright}{\kern0pt}\isanewline
\ \ \ \ \ \ \ \ \isacommand{using}\isamarkupfalse%
\ phiformula\ assms{\isadigit{2}}\ \isanewline
\ \ \ \ \ \ \ \ \isacommand{by}\isamarkupfalse%
\ auto\isanewline
\isanewline
\ \ \ \ \ \ \isacommand{have}\isamarkupfalse%
\ I{\isadigit{1}}\ {\isacharcolon}{\kern0pt}\ {\isachardoublequoteopen}A{\isacharcomma}{\kern0pt}\ env\ {\isasymTurnstile}\ BExists{\isacharprime}{\kern0pt}{\isacharparenleft}{\kern0pt}m{\isacharcomma}{\kern0pt}\ {\isasymphi}{\isacharparenright}{\kern0pt}\ {\isasymlongleftrightarrow}\ {\isacharparenleft}{\kern0pt}{\isasymexists}x\ {\isasymin}\ A{\isachardot}{\kern0pt}\ x\ {\isasymin}\ nth{\isacharparenleft}{\kern0pt}m{\isacharcomma}{\kern0pt}\ Cons{\isacharparenleft}{\kern0pt}x{\isacharcomma}{\kern0pt}\ env{\isacharparenright}{\kern0pt}{\isacharparenright}{\kern0pt}\ {\isasymand}\ sats{\isacharparenleft}{\kern0pt}A{\isacharcomma}{\kern0pt}\ {\isasymphi}{\isacharcomma}{\kern0pt}\ Cons{\isacharparenleft}{\kern0pt}x{\isacharcomma}{\kern0pt}\ env{\isacharparenright}{\kern0pt}{\isacharparenright}{\kern0pt}{\isacharparenright}{\kern0pt}{\isachardoublequoteclose}\isanewline
\ \ \ \ \ \ \ \ \isacommand{unfolding}\isamarkupfalse%
\ BExists{\isacharprime}{\kern0pt}{\isacharunderscore}{\kern0pt}def\ \isanewline
\ \ \ \ \ \ \ \ \isacommand{using}\isamarkupfalse%
\ assms{\isadigit{2}}\isanewline
\ \ \ \ \ \ \ \ \isacommand{by}\isamarkupfalse%
\ auto\isanewline
\ \ \ \ \ \ \isacommand{have}\isamarkupfalse%
\ I{\isadigit{2}}\ {\isacharcolon}{\kern0pt}\ {\isachardoublequoteopen}{\isachardot}{\kern0pt}{\isachardot}{\kern0pt}{\isachardot}{\kern0pt}\ {\isasymlongleftrightarrow}\ {\isacharparenleft}{\kern0pt}{\isasymexists}x\ {\isasymin}\ B{\isachardot}{\kern0pt}\ x\ {\isasymin}\ nth{\isacharparenleft}{\kern0pt}m{\isacharcomma}{\kern0pt}\ Cons{\isacharparenleft}{\kern0pt}x{\isacharcomma}{\kern0pt}\ env{\isacharparenright}{\kern0pt}{\isacharparenright}{\kern0pt}\ {\isasymand}\ sats{\isacharparenleft}{\kern0pt}B{\isacharcomma}{\kern0pt}\ {\isasymphi}{\isacharcomma}{\kern0pt}\ Cons{\isacharparenleft}{\kern0pt}x{\isacharcomma}{\kern0pt}\ env{\isacharparenright}{\kern0pt}{\isacharparenright}{\kern0pt}{\isacharparenright}{\kern0pt}{\isachardoublequoteclose}\ \isanewline
\ \ \ \ \ \ \ \ \isacommand{apply}\isamarkupfalse%
{\isacharparenleft}{\kern0pt}rule\ iffI{\isacharcomma}{\kern0pt}\ clarify{\isacharparenright}{\kern0pt}\isanewline
\ \ \ \ \ \ \ \ \ \isacommand{apply}\isamarkupfalse%
{\isacharparenleft}{\kern0pt}rename{\isacharunderscore}{\kern0pt}tac\ x{\isacharcomma}{\kern0pt}\ rule{\isacharunderscore}{\kern0pt}tac\ x{\isacharequal}{\kern0pt}x\ \isakeyword{in}\ bexI{\isacharcomma}{\kern0pt}\ rule\ conjI{\isacharcomma}{\kern0pt}\ simp{\isacharcomma}{\kern0pt}\ rule\ ih{\isadigit{1}}{\isacharcomma}{\kern0pt}\ simp\ add{\isacharcolon}{\kern0pt}assms{\isadigit{2}}{\isacharcomma}{\kern0pt}\ simp\ add{\isacharcolon}{\kern0pt}assms{\isadigit{2}}{\isacharcomma}{\kern0pt}\ simp{\isacharcomma}{\kern0pt}\ simp\ add{\isacharcolon}{\kern0pt}arityle{\isacharcomma}{\kern0pt}\ simp{\isacharparenright}{\kern0pt}\isanewline
\ \ \ \ \ \ \ \ \isacommand{using}\isamarkupfalse%
\ assms\ \isanewline
\ \ \ \ \ \ \ \ \ \isacommand{apply}\isamarkupfalse%
\ force\isanewline
\ \ \ \ \ \ \ \ \isacommand{apply}\isamarkupfalse%
\ clarify\ \isanewline
\ \ \ \ \ \ \ \ \isacommand{apply}\isamarkupfalse%
{\isacharparenleft}{\kern0pt}rename{\isacharunderscore}{\kern0pt}tac\ x{\isacharcomma}{\kern0pt}\ subgoal{\isacharunderscore}{\kern0pt}tac\ {\isachardoublequoteopen}x\ {\isasymin}\ A{\isachardoublequoteclose}{\isacharcomma}{\kern0pt}\ rule{\isacharunderscore}{\kern0pt}tac\ x{\isacharequal}{\kern0pt}x\ \isakeyword{in}\ bexI{\isacharcomma}{\kern0pt}\ simp{\isacharparenright}{\kern0pt}\isanewline
\ \ \ \ \ \ \ \ \ \ \isacommand{apply}\isamarkupfalse%
{\isacharparenleft}{\kern0pt}rule\ ih{\isadigit{2}}{\isacharcomma}{\kern0pt}\ simp\ add{\isacharcolon}{\kern0pt}assms{\isadigit{2}}{\isacharcomma}{\kern0pt}\ simp{\isacharcomma}{\kern0pt}\ simp\ add{\isacharcolon}{\kern0pt}assms{\isadigit{2}}{\isacharcomma}{\kern0pt}\ simp{\isacharcomma}{\kern0pt}\ simp\ add{\isacharcolon}{\kern0pt}arityle{\isacharcomma}{\kern0pt}\ simp{\isacharcomma}{\kern0pt}\ simp{\isacharparenright}{\kern0pt}\isanewline
\ \ \ \ \ \ \ \ \isacommand{apply}\isamarkupfalse%
{\isacharparenleft}{\kern0pt}rule{\isacharunderscore}{\kern0pt}tac\ n{\isacharequal}{\kern0pt}m\ \isakeyword{in}\ natE{\isacharcomma}{\kern0pt}\ simp\ add{\isacharcolon}{\kern0pt}assms{\isadigit{2}}{\isacharparenright}{\kern0pt}\isanewline
\ \ \ \ \ \ \ \ \ \isacommand{apply}\isamarkupfalse%
{\isacharparenleft}{\kern0pt}rule\ mem{\isacharunderscore}{\kern0pt}irrefl{\isacharcomma}{\kern0pt}\ simp{\isacharparenright}{\kern0pt}\isanewline
\ \ \ \ \ \ \ \ \isacommand{apply}\isamarkupfalse%
\ simp\ \isanewline
\ \ \ \ \ \ \ \ \isacommand{apply}\isamarkupfalse%
{\isacharparenleft}{\kern0pt}rename{\isacharunderscore}{\kern0pt}tac\ x\ m{\isacharprime}{\kern0pt}{\isacharcomma}{\kern0pt}\ subgoal{\isacharunderscore}{\kern0pt}tac\ {\isachardoublequoteopen}nth{\isacharparenleft}{\kern0pt}m{\isacharprime}{\kern0pt}{\isacharcomma}{\kern0pt}\ env{\isacharparenright}{\kern0pt}\ {\isasymin}\ A{\isachardoublequoteclose}{\isacharparenright}{\kern0pt}\ \isanewline
\ \ \ \ \ \ \ \ \isacommand{using}\isamarkupfalse%
\ assms\ Transset{\isacharunderscore}{\kern0pt}def\ \isanewline
\ \ \ \ \ \ \ \ \ \isacommand{apply}\isamarkupfalse%
\ force\isanewline
\ \ \ \ \ \ \ \ \isacommand{apply}\isamarkupfalse%
{\isacharparenleft}{\kern0pt}rule\ nth{\isacharunderscore}{\kern0pt}type{\isacharcomma}{\kern0pt}\ simp\ add{\isacharcolon}{\kern0pt}assms{\isadigit{2}}{\isacharcomma}{\kern0pt}\ rule{\isacharunderscore}{\kern0pt}tac\ b{\isacharequal}{\kern0pt}m\ \isakeyword{in}\ lt{\isacharunderscore}{\kern0pt}le{\isacharunderscore}{\kern0pt}lt{\isacharcomma}{\kern0pt}\ simp{\isacharcomma}{\kern0pt}\ rule\ mle{\isacharparenright}{\kern0pt}\isanewline
\ \ \ \ \ \ \ \ \isacommand{done}\isamarkupfalse%
\isanewline
\ \ \ \ \ \ \isacommand{have}\isamarkupfalse%
\ I{\isadigit{3}}\ {\isacharcolon}{\kern0pt}\ {\isachardoublequoteopen}{\isachardot}{\kern0pt}{\isachardot}{\kern0pt}{\isachardot}{\kern0pt}\ {\isasymlongleftrightarrow}\ B{\isacharcomma}{\kern0pt}\ env\ {\isasymTurnstile}\ BExists{\isacharprime}{\kern0pt}{\isacharparenleft}{\kern0pt}m{\isacharcomma}{\kern0pt}\ {\isasymphi}{\isacharparenright}{\kern0pt}{\isachardoublequoteclose}\isanewline
\ \ \ \ \ \ \ \ \isacommand{using}\isamarkupfalse%
\ envinBlist\ BExists{\isacharprime}{\kern0pt}{\isacharunderscore}{\kern0pt}def\ \isacommand{by}\isamarkupfalse%
\ auto\isanewline
\ \ \ \ \ \ \isacommand{show}\isamarkupfalse%
\ {\isachardoublequoteopen}\ A{\isacharcomma}{\kern0pt}\ env\ {\isasymTurnstile}\ BExists{\isacharprime}{\kern0pt}{\isacharparenleft}{\kern0pt}m{\isacharcomma}{\kern0pt}\ {\isasymphi}{\isacharparenright}{\kern0pt}\ {\isasymlongleftrightarrow}\ B{\isacharcomma}{\kern0pt}\ env\ {\isasymTurnstile}\ BExists{\isacharprime}{\kern0pt}{\isacharparenleft}{\kern0pt}m{\isacharcomma}{\kern0pt}\ {\isasymphi}{\isacharparenright}{\kern0pt}{\isachardoublequoteclose}\ \isanewline
\ \ \ \ \ \ \ \ \isacommand{using}\isamarkupfalse%
\ I{\isadigit{1}}\ I{\isadigit{2}}\ I{\isadigit{3}}\ \isanewline
\ \ \ \ \ \ \ \ \isacommand{by}\isamarkupfalse%
\ auto\ \isanewline
\ \ \ \ \isacommand{qed}\isamarkupfalse%
\isanewline
\ \ \ \ \ \ \ \ \isanewline
\ \ \ \ \isacommand{show}\isamarkupfalse%
\ {\isachardoublequoteopen}{\isasymforall}{\isasymphi}{\isasymin}{\isasymDelta}{\isadigit{0}}{\isacharunderscore}{\kern0pt}from{\isacharcircum}{\kern0pt}succ{\isacharparenleft}{\kern0pt}n{\isacharparenright}{\kern0pt}\ {\isacharparenleft}{\kern0pt}{\isasymDelta}{\isadigit{0}}{\isacharunderscore}{\kern0pt}base{\isacharparenright}{\kern0pt}{\isachardot}{\kern0pt}\ {\isasymforall}env{\isasymin}list{\isacharparenleft}{\kern0pt}A{\isacharparenright}{\kern0pt}{\isachardot}{\kern0pt}\ arity{\isacharparenleft}{\kern0pt}{\isasymphi}{\isacharparenright}{\kern0pt}\ {\isasymle}\ length{\isacharparenleft}{\kern0pt}env{\isacharparenright}{\kern0pt}\ {\isasymlongrightarrow}\ A{\isacharcomma}{\kern0pt}\ env\ {\isasymTurnstile}\ {\isasymphi}\ {\isasymlongleftrightarrow}\ B{\isacharcomma}{\kern0pt}\ env\ {\isasymTurnstile}\ {\isasymphi}{\isachardoublequoteclose}\ \isanewline
\ \ \ \ \ \ \isacommand{apply}\isamarkupfalse%
{\isacharparenleft}{\kern0pt}simp{\isacharcomma}{\kern0pt}\ rule{\isacharunderscore}{\kern0pt}tac\ b{\isacharequal}{\kern0pt}{\isachardoublequoteopen}{\isasymDelta}{\isadigit{0}}{\isacharunderscore}{\kern0pt}from{\isacharcircum}{\kern0pt}n\ {\isacharparenleft}{\kern0pt}{\isasymDelta}{\isadigit{0}}{\isacharunderscore}{\kern0pt}base{\isacharparenright}{\kern0pt}{\isachardoublequoteclose}\ \isakeyword{and}\ a{\isacharequal}{\kern0pt}X\ \isakeyword{in}\ ssubst{\isacharcomma}{\kern0pt}\ simp\ add{\isacharcolon}{\kern0pt}X{\isacharunderscore}{\kern0pt}def{\isacharcomma}{\kern0pt}\ clarify{\isacharparenright}{\kern0pt}\isanewline
\ \ \ \ \ \ \isacommand{unfolding}\isamarkupfalse%
\ {\isasymDelta}{\isadigit{0}}{\isacharunderscore}{\kern0pt}from{\isacharunderscore}{\kern0pt}def\ \isanewline
\ \ \ \ \ \ \isacommand{apply}\isamarkupfalse%
\ simp\isanewline
\ \ \ \ \ \ \isacommand{apply}\isamarkupfalse%
{\isacharparenleft}{\kern0pt}rename{\isacharunderscore}{\kern0pt}tac\ {\isasymphi}\ env{\isacharcomma}{\kern0pt}\ case{\isacharunderscore}{\kern0pt}tac\ {\isachardoublequoteopen}{\isasymphi}\ {\isasymin}\ X{\isachardoublequoteclose}{\isacharcomma}{\kern0pt}\ simp{\isacharparenright}{\kern0pt}\isanewline
\ \ \ \ \ \ \isacommand{using}\isamarkupfalse%
\ ih{\isadigit{1}}\ ih{\isadigit{2}}\ \isanewline
\ \ \ \ \ \ \ \isacommand{apply}\isamarkupfalse%
\ force\isanewline
\ \ \ \ \ \ \isacommand{apply}\isamarkupfalse%
\ clarify\isanewline
\ \ \ \ \ \ \isacommand{apply}\isamarkupfalse%
{\isacharparenleft}{\kern0pt}rule\ disjE{\isacharcomma}{\kern0pt}\ simp{\isacharparenright}{\kern0pt}\isanewline
\ \ \ \ \ \ \isacommand{apply}\isamarkupfalse%
{\isacharparenleft}{\kern0pt}clarify{\isacharcomma}{\kern0pt}\ rule\ Hnand{\isacharcomma}{\kern0pt}\ simp{\isacharcomma}{\kern0pt}\ simp{\isacharcomma}{\kern0pt}\ simp{\isacharcomma}{\kern0pt}\ simp{\isacharparenright}{\kern0pt}\isanewline
\ \ \ \ \ \ \isacommand{apply}\isamarkupfalse%
{\isacharparenleft}{\kern0pt}clarify{\isacharcomma}{\kern0pt}\ rule\ Hbex{\isacharcomma}{\kern0pt}\ simp{\isacharcomma}{\kern0pt}\ simp{\isacharcomma}{\kern0pt}\ simp{\isacharparenright}{\kern0pt}\isanewline
\ \ \ \ \ \ \isacommand{unfolding}\isamarkupfalse%
\ BExists{\isacharprime}{\kern0pt}{\isacharunderscore}{\kern0pt}def\ \isanewline
\ \ \ \ \ \ \isacommand{by}\isamarkupfalse%
\ simp\isanewline
\ \ \isacommand{qed}\isamarkupfalse%
\isanewline
\isanewline
\ \ \isacommand{then}\isamarkupfalse%
\ \isacommand{show}\isamarkupfalse%
\ {\isachardoublequoteopen}A{\isacharcomma}{\kern0pt}\ env\ {\isasymTurnstile}\ {\isasymphi}\ {\isasymlongleftrightarrow}\ B{\isacharcomma}{\kern0pt}\ env\ {\isasymTurnstile}\ {\isasymphi}{\isachardoublequoteclose}\ \isanewline
\ \ \ \ \isacommand{using}\isamarkupfalse%
\ assms\ \isanewline
\ \ \ \ \isacommand{unfolding}\isamarkupfalse%
\ {\isasymDelta}{\isadigit{0}}{\isacharunderscore}{\kern0pt}def\ \isanewline
\ \ \ \ \isacommand{by}\isamarkupfalse%
\ auto\isanewline
\isacommand{qed}\isamarkupfalse%
%
\endisatagproof
{\isafoldproof}%
%
\isadelimproof
\isanewline
%
\endisadelimproof
\isanewline
\isacommand{lemma}\isamarkupfalse%
\ ren{\isacharunderscore}{\kern0pt}{\isasymDelta}{\isadigit{0}}\ {\isacharcolon}{\kern0pt}\ \isanewline
\ \ \isakeyword{fixes}\ {\isasymphi}\ m\ n\ f\isanewline
\ \ \isakeyword{assumes}\ {\isachardoublequoteopen}{\isasymphi}\ {\isasymin}\ {\isasymDelta}{\isadigit{0}}{\isachardoublequoteclose}\ {\isachardoublequoteopen}m\ {\isasymin}\ nat{\isachardoublequoteclose}\ {\isachardoublequoteopen}n\ {\isasymin}\ nat{\isachardoublequoteclose}\ {\isachardoublequoteopen}f\ {\isasymin}\ m\ {\isasymrightarrow}\ n{\isachardoublequoteclose}\ {\isachardoublequoteopen}arity{\isacharparenleft}{\kern0pt}{\isasymphi}{\isacharparenright}{\kern0pt}\ {\isasymle}\ m{\isachardoublequoteclose}\isanewline
\ \ \isakeyword{shows}\ {\isachardoublequoteopen}ren{\isacharparenleft}{\kern0pt}{\isasymphi}{\isacharparenright}{\kern0pt}{\isacharbackquote}{\kern0pt}m{\isacharbackquote}{\kern0pt}n{\isacharbackquote}{\kern0pt}f\ {\isasymin}\ {\isasymDelta}{\isadigit{0}}{\isachardoublequoteclose}\ \isanewline
%
\isadelimproof
%
\endisadelimproof
%
\isatagproof
\isacommand{proof}\isamarkupfalse%
\ {\isacharminus}{\kern0pt}\ \isanewline
\isanewline
\ \ \isacommand{have}\isamarkupfalse%
\ value{\isacharunderscore}{\kern0pt}type{\isacharcolon}{\kern0pt}\ {\isachardoublequoteopen}{\isasymAnd}k\ f\ m\ n{\isachardot}{\kern0pt}\ m\ {\isasymin}\ nat\ {\isasymLongrightarrow}\ n\ {\isasymin}\ nat\ {\isasymLongrightarrow}\ f\ {\isasymin}\ m\ {\isasymrightarrow}\ n\ {\isasymLongrightarrow}\ f{\isacharbackquote}{\kern0pt}k\ {\isasymin}\ nat{\isachardoublequoteclose}\ \isanewline
\ \ \isacommand{proof}\isamarkupfalse%
{\isacharparenleft}{\kern0pt}rename{\isacharunderscore}{\kern0pt}tac\ k\ f\ m\ n{\isacharcomma}{\kern0pt}\ case{\isacharunderscore}{\kern0pt}tac\ {\isachardoublequoteopen}k\ {\isasymin}\ m{\isachardoublequoteclose}{\isacharparenright}{\kern0pt}\isanewline
\ \ \ \ \isacommand{fix}\isamarkupfalse%
\ k\ f\ m\ n\ \isacommand{assume}\isamarkupfalse%
\ assms{\isadigit{2}}\ {\isacharcolon}{\kern0pt}\ {\isachardoublequoteopen}k\ {\isasymin}\ m{\isachardoublequoteclose}\ {\isachardoublequoteopen}m\ {\isasymin}\ nat{\isachardoublequoteclose}\ {\isachardoublequoteopen}n\ {\isasymin}\ nat{\isachardoublequoteclose}\ {\isachardoublequoteopen}f\ {\isasymin}\ m\ {\isasymrightarrow}\ n{\isachardoublequoteclose}\ \ \ \ \isanewline
\ \ \ \ \isacommand{then}\isamarkupfalse%
\ \isacommand{have}\isamarkupfalse%
\ {\isachardoublequoteopen}f{\isacharbackquote}{\kern0pt}k\ {\isasymin}\ range{\isacharparenleft}{\kern0pt}f{\isacharparenright}{\kern0pt}{\isachardoublequoteclose}\ \isacommand{using}\isamarkupfalse%
\ apply{\isacharunderscore}{\kern0pt}rangeI\ assms{\isadigit{2}}\ Pi{\isacharunderscore}{\kern0pt}def\ \isacommand{by}\isamarkupfalse%
\ auto\ \isanewline
\ \ \ \ \isacommand{then}\isamarkupfalse%
\ \isacommand{have}\isamarkupfalse%
\ {\isachardoublequoteopen}f{\isacharbackquote}{\kern0pt}k\ {\isasymin}\ n{\isachardoublequoteclose}\ \isacommand{using}\isamarkupfalse%
\ assms{\isadigit{2}}\ Pi{\isacharunderscore}{\kern0pt}def\ \isacommand{by}\isamarkupfalse%
\ auto\ \isanewline
\ \ \ \ \isacommand{then}\isamarkupfalse%
\ \isacommand{have}\isamarkupfalse%
\ {\isachardoublequoteopen}f{\isacharbackquote}{\kern0pt}k\ {\isacharless}{\kern0pt}\ n{\isachardoublequoteclose}\ \isacommand{using}\isamarkupfalse%
\ ltI\ assms{\isadigit{2}}\ \isacommand{by}\isamarkupfalse%
\ auto\isanewline
\ \ \ \ \isacommand{then}\isamarkupfalse%
\ \isacommand{show}\isamarkupfalse%
\ {\isachardoublequoteopen}f{\isacharbackquote}{\kern0pt}k\ {\isasymin}\ nat{\isachardoublequoteclose}\ \isacommand{using}\isamarkupfalse%
\ assms{\isadigit{2}}\ lt{\isacharunderscore}{\kern0pt}nat{\isacharunderscore}{\kern0pt}in{\isacharunderscore}{\kern0pt}nat\ \isacommand{by}\isamarkupfalse%
\ auto\isanewline
\ \ \isacommand{next}\isamarkupfalse%
\ \isanewline
\ \ \ \ \isacommand{fix}\isamarkupfalse%
\ k\ f\ m\ n\ \isacommand{assume}\isamarkupfalse%
\ assms{\isadigit{2}}\ {\isacharcolon}{\kern0pt}\ {\isachardoublequoteopen}k\ {\isasymnotin}\ m{\isachardoublequoteclose}\ {\isachardoublequoteopen}m\ {\isasymin}\ nat{\isachardoublequoteclose}\ {\isachardoublequoteopen}n\ {\isasymin}\ nat{\isachardoublequoteclose}\ {\isachardoublequoteopen}f\ {\isasymin}\ m\ {\isasymrightarrow}\ n{\isachardoublequoteclose}\isanewline
\ \ \ \ \isacommand{then}\isamarkupfalse%
\ \isacommand{have}\isamarkupfalse%
\ {\isachardoublequoteopen}k\ {\isasymnotin}\ domain{\isacharparenleft}{\kern0pt}f{\isacharparenright}{\kern0pt}{\isachardoublequoteclose}\ \isacommand{using}\isamarkupfalse%
\ assms{\isadigit{2}}\ Pi{\isacharunderscore}{\kern0pt}def\ \isacommand{by}\isamarkupfalse%
\ auto\ \isanewline
\ \ \ \ \isacommand{then}\isamarkupfalse%
\ \isacommand{have}\isamarkupfalse%
\ {\isachardoublequoteopen}f{\isacharbackquote}{\kern0pt}k\ {\isacharequal}{\kern0pt}\ {\isadigit{0}}{\isachardoublequoteclose}\ \isacommand{using}\isamarkupfalse%
\ apply{\isacharunderscore}{\kern0pt}{\isadigit{0}}\ \isacommand{by}\isamarkupfalse%
\ auto\ \isanewline
\ \ \ \ \isacommand{then}\isamarkupfalse%
\ \isacommand{show}\isamarkupfalse%
\ {\isachardoublequoteopen}f{\isacharbackquote}{\kern0pt}k\ {\isasymin}\ nat{\isachardoublequoteclose}\ \isacommand{by}\isamarkupfalse%
\ auto\isanewline
\ \ \isacommand{qed}\isamarkupfalse%
\isanewline
\isanewline
\ \ \isacommand{have}\isamarkupfalse%
\ {\isachardoublequoteopen}{\isasymAnd}i{\isachardot}{\kern0pt}\ i\ {\isasymin}\ nat\ {\isasymLongrightarrow}\ \ {\isasymforall}m\ {\isasymin}\ nat{\isachardot}{\kern0pt}\ {\isasymforall}n\ {\isasymin}\ nat{\isachardot}{\kern0pt}\ {\isasymforall}f\ {\isasymin}\ m\ {\isasymrightarrow}\ n{\isachardot}{\kern0pt}\ {\isasymforall}{\isasymphi}\ {\isasymin}\ formula{\isachardot}{\kern0pt}\ {\isasymphi}\ {\isasymin}\ {\isasymDelta}{\isadigit{0}}{\isacharunderscore}{\kern0pt}from{\isacharcircum}{\kern0pt}i{\isacharparenleft}{\kern0pt}{\isasymDelta}{\isadigit{0}}{\isacharunderscore}{\kern0pt}base{\isacharparenright}{\kern0pt}\ {\isasymlongrightarrow}\ arity{\isacharparenleft}{\kern0pt}{\isasymphi}{\isacharparenright}{\kern0pt}\ {\isasymle}\ m\ {\isasymlongrightarrow}\ ren{\isacharparenleft}{\kern0pt}{\isasymphi}{\isacharparenright}{\kern0pt}{\isacharbackquote}{\kern0pt}m{\isacharbackquote}{\kern0pt}n{\isacharbackquote}{\kern0pt}f\ {\isasymin}\ {\isasymDelta}{\isadigit{0}}{\isacharunderscore}{\kern0pt}from{\isacharcircum}{\kern0pt}i{\isacharparenleft}{\kern0pt}{\isasymDelta}{\isadigit{0}}{\isacharunderscore}{\kern0pt}base{\isacharparenright}{\kern0pt}{\isachardoublequoteclose}\isanewline
\ \ \isacommand{proof}\isamarkupfalse%
{\isacharparenleft}{\kern0pt}rule{\isacharunderscore}{\kern0pt}tac\ P{\isacharequal}{\kern0pt}{\isachardoublequoteopen}{\isasymlambda}i{\isachardot}{\kern0pt}\ {\isasymforall}m\ {\isasymin}\ nat{\isachardot}{\kern0pt}\ {\isasymforall}n\ {\isasymin}\ nat{\isachardot}{\kern0pt}\ {\isasymforall}f\ {\isasymin}\ m\ {\isasymrightarrow}\ n{\isachardot}{\kern0pt}\ {\isasymforall}{\isasymphi}\ {\isasymin}\ formula{\isachardot}{\kern0pt}\ {\isasymphi}\ {\isasymin}\ {\isasymDelta}{\isadigit{0}}{\isacharunderscore}{\kern0pt}from{\isacharcircum}{\kern0pt}i{\isacharparenleft}{\kern0pt}{\isasymDelta}{\isadigit{0}}{\isacharunderscore}{\kern0pt}base{\isacharparenright}{\kern0pt}\ {\isasymlongrightarrow}\ arity{\isacharparenleft}{\kern0pt}{\isasymphi}{\isacharparenright}{\kern0pt}\ {\isasymle}\ m\ {\isasymlongrightarrow}\ ren{\isacharparenleft}{\kern0pt}{\isasymphi}{\isacharparenright}{\kern0pt}{\isacharbackquote}{\kern0pt}m{\isacharbackquote}{\kern0pt}n{\isacharbackquote}{\kern0pt}f\ {\isasymin}\ {\isasymDelta}{\isadigit{0}}{\isacharunderscore}{\kern0pt}from{\isacharcircum}{\kern0pt}i{\isacharparenleft}{\kern0pt}{\isasymDelta}{\isadigit{0}}{\isacharunderscore}{\kern0pt}base{\isacharparenright}{\kern0pt}{\isachardoublequoteclose}\ \isakeyword{in}\ nat{\isacharunderscore}{\kern0pt}induct{\isacharcomma}{\kern0pt}\ simp{\isacharparenright}{\kern0pt}\isanewline
\ \ \ \ \isacommand{fix}\isamarkupfalse%
\ i\ \isanewline
\ \ \ \ \isacommand{assume}\isamarkupfalse%
\ assms{\isadigit{1}}\ {\isacharcolon}{\kern0pt}\ {\isachardoublequoteopen}i\ {\isasymin}\ nat{\isachardoublequoteclose}\ \isanewline
\ \ \ \ \isacommand{show}\isamarkupfalse%
\ {\isachardoublequoteopen}{\isasymforall}m{\isasymin}nat{\isachardot}{\kern0pt}\ {\isasymforall}n{\isasymin}nat{\isachardot}{\kern0pt}\ {\isasymforall}f{\isasymin}m\ {\isasymrightarrow}\ n{\isachardot}{\kern0pt}\ {\isasymforall}{\isasymphi}{\isasymin}formula{\isachardot}{\kern0pt}\ {\isasymphi}\ {\isasymin}\ {\isasymDelta}{\isadigit{0}}{\isacharunderscore}{\kern0pt}from{\isacharcircum}{\kern0pt}{\isadigit{0}}\ {\isacharparenleft}{\kern0pt}{\isasymDelta}{\isadigit{0}}{\isacharunderscore}{\kern0pt}base{\isacharparenright}{\kern0pt}\ {\isasymlongrightarrow}\ arity{\isacharparenleft}{\kern0pt}{\isasymphi}{\isacharparenright}{\kern0pt}\ {\isasymle}\ m\ {\isasymlongrightarrow}\ ren{\isacharparenleft}{\kern0pt}{\isasymphi}{\isacharparenright}{\kern0pt}\ {\isacharbackquote}{\kern0pt}\ m\ {\isacharbackquote}{\kern0pt}\ n\ {\isacharbackquote}{\kern0pt}\ f\ {\isasymin}\ {\isasymDelta}{\isadigit{0}}{\isacharunderscore}{\kern0pt}from{\isacharcircum}{\kern0pt}{\isadigit{0}}\ {\isacharparenleft}{\kern0pt}{\isasymDelta}{\isadigit{0}}{\isacharunderscore}{\kern0pt}base{\isacharparenright}{\kern0pt}{\isachardoublequoteclose}\ \isanewline
\ \ \ \ \isacommand{proof}\isamarkupfalse%
{\isacharparenleft}{\kern0pt}rule\ ballI{\isacharcomma}{\kern0pt}\ rule\ ballI{\isacharcomma}{\kern0pt}\ rule\ ballI{\isacharcomma}{\kern0pt}\ rule\ ballI\ {\isacharcomma}{\kern0pt}\ rule\ impI{\isacharcomma}{\kern0pt}\ rule\ impI{\isacharparenright}{\kern0pt}\isanewline
\ \ \ \ \ \ \isacommand{fix}\isamarkupfalse%
\ m\ n\ f\ {\isasymphi}\ \isanewline
\ \ \ \ \ \ \isacommand{assume}\isamarkupfalse%
\ assms{\isadigit{2}}{\isacharcolon}{\kern0pt}\ {\isachardoublequoteopen}m\ {\isasymin}\ nat{\isachardoublequoteclose}\ {\isachardoublequoteopen}n\ {\isasymin}\ nat{\isachardoublequoteclose}\ {\isachardoublequoteopen}f\ {\isasymin}\ m\ {\isasymrightarrow}\ n{\isachardoublequoteclose}\ {\isachardoublequoteopen}{\isasymphi}\ {\isasymin}\ formula{\isachardoublequoteclose}\ {\isachardoublequoteopen}{\isasymphi}\ {\isasymin}\ {\isasymDelta}{\isadigit{0}}{\isacharunderscore}{\kern0pt}from{\isacharcircum}{\kern0pt}{\isadigit{0}}\ {\isacharparenleft}{\kern0pt}{\isasymDelta}{\isadigit{0}}{\isacharunderscore}{\kern0pt}base{\isacharparenright}{\kern0pt}{\isachardoublequoteclose}\ \isanewline
\ \ \ \ \ \ \isacommand{show}\isamarkupfalse%
\ {\isachardoublequoteopen}ren{\isacharparenleft}{\kern0pt}{\isasymphi}{\isacharparenright}{\kern0pt}\ {\isacharbackquote}{\kern0pt}\ m\ {\isacharbackquote}{\kern0pt}\ n\ {\isacharbackquote}{\kern0pt}\ f\ {\isasymin}\ {\isasymDelta}{\isadigit{0}}{\isacharunderscore}{\kern0pt}from{\isacharcircum}{\kern0pt}{\isadigit{0}}\ {\isacharparenleft}{\kern0pt}{\isasymDelta}{\isadigit{0}}{\isacharunderscore}{\kern0pt}base{\isacharparenright}{\kern0pt}{\isachardoublequoteclose}\isanewline
\ \ \ \ \ \ \ \ \isacommand{using}\isamarkupfalse%
\ assms{\isadigit{2}}\ \isanewline
\ \ \ \ \ \ \ \ \isacommand{unfolding}\isamarkupfalse%
\ {\isasymDelta}{\isadigit{0}}{\isacharunderscore}{\kern0pt}base{\isacharunderscore}{\kern0pt}def\isanewline
\ \ \ \ \ \ \ \ \isacommand{apply}\isamarkupfalse%
\ auto\isanewline
\ \ \ \ \ \ \ \ \ \ \ \ \ \ \ \isacommand{apply}\isamarkupfalse%
{\isacharparenleft}{\kern0pt}rule\ value{\isacharunderscore}{\kern0pt}type{\isacharcomma}{\kern0pt}\ simp{\isacharunderscore}{\kern0pt}all{\isacharparenright}{\kern0pt}{\isacharplus}{\kern0pt}\isanewline
\ \ \ \ \ \ \ \ \isacommand{done}\isamarkupfalse%
\isanewline
\ \ \ \ \isacommand{qed}\isamarkupfalse%
\isanewline
\ \ \isacommand{next}\isamarkupfalse%
\ \isanewline
\ \ \ \ \isacommand{fix}\isamarkupfalse%
\ i\ \isanewline
\ \ \ \ \isacommand{assume}\isamarkupfalse%
\ assms{\isadigit{1}}{\isacharcolon}{\kern0pt}\ {\isachardoublequoteopen}i\ {\isasymin}\ nat{\isachardoublequoteclose}\ {\isachardoublequoteopen}{\isasymforall}m{\isasymin}nat{\isachardot}{\kern0pt}\ {\isasymforall}n{\isasymin}nat{\isachardot}{\kern0pt}\ {\isasymforall}f{\isasymin}m\ {\isasymrightarrow}\ n{\isachardot}{\kern0pt}\ {\isasymforall}{\isasymphi}{\isasymin}formula{\isachardot}{\kern0pt}\ {\isasymphi}\ {\isasymin}\ {\isasymDelta}{\isadigit{0}}{\isacharunderscore}{\kern0pt}from{\isacharcircum}{\kern0pt}i\ {\isacharparenleft}{\kern0pt}{\isasymDelta}{\isadigit{0}}{\isacharunderscore}{\kern0pt}base{\isacharparenright}{\kern0pt}\ {\isasymlongrightarrow}\ arity{\isacharparenleft}{\kern0pt}{\isasymphi}{\isacharparenright}{\kern0pt}\ {\isasymle}\ m\ {\isasymlongrightarrow}\ ren{\isacharparenleft}{\kern0pt}{\isasymphi}{\isacharparenright}{\kern0pt}\ {\isacharbackquote}{\kern0pt}\ m\ {\isacharbackquote}{\kern0pt}\ n\ {\isacharbackquote}{\kern0pt}\ f\ {\isasymin}\ {\isasymDelta}{\isadigit{0}}{\isacharunderscore}{\kern0pt}from{\isacharcircum}{\kern0pt}i\ {\isacharparenleft}{\kern0pt}{\isasymDelta}{\isadigit{0}}{\isacharunderscore}{\kern0pt}base{\isacharparenright}{\kern0pt}{\isachardoublequoteclose}\isanewline
\isanewline
\ \ \ \ \isacommand{have}\isamarkupfalse%
\ ih\ {\isacharcolon}{\kern0pt}\ {\isachardoublequoteopen}{\isasymAnd}m\ n\ f\ {\isasymphi}{\isachardot}{\kern0pt}\ m\ {\isasymin}\ nat\ {\isasymLongrightarrow}\ n\ {\isasymin}\ nat\ {\isasymLongrightarrow}\ f\ {\isasymin}\ m\ {\isasymrightarrow}\ n\ {\isasymLongrightarrow}\ {\isasymphi}\ {\isasymin}\ {\isasymDelta}{\isadigit{0}}{\isacharunderscore}{\kern0pt}from{\isacharcircum}{\kern0pt}i{\isacharparenleft}{\kern0pt}{\isasymDelta}{\isadigit{0}}{\isacharunderscore}{\kern0pt}base{\isacharparenright}{\kern0pt}\ {\isasymLongrightarrow}\ arity{\isacharparenleft}{\kern0pt}{\isasymphi}{\isacharparenright}{\kern0pt}\ {\isasymle}\ m\ {\isasymLongrightarrow}\ ren{\isacharparenleft}{\kern0pt}{\isasymphi}{\isacharparenright}{\kern0pt}{\isacharbackquote}{\kern0pt}m{\isacharbackquote}{\kern0pt}n{\isacharbackquote}{\kern0pt}f\ {\isasymin}\ {\isasymDelta}{\isadigit{0}}{\isacharunderscore}{\kern0pt}from{\isacharcircum}{\kern0pt}i{\isacharparenleft}{\kern0pt}{\isasymDelta}{\isadigit{0}}{\isacharunderscore}{\kern0pt}base{\isacharparenright}{\kern0pt}{\isachardoublequoteclose}\ \isanewline
\ \ \ \ \ \ \isacommand{apply}\isamarkupfalse%
{\isacharparenleft}{\kern0pt}rename{\isacharunderscore}{\kern0pt}tac\ m\ n\ f\ {\isasymphi}{\isacharcomma}{\kern0pt}\ subgoal{\isacharunderscore}{\kern0pt}tac\ {\isachardoublequoteopen}{\isasymphi}\ {\isasymin}\ {\isasymDelta}{\isadigit{0}}{\isachardoublequoteclose}{\isacharparenright}{\kern0pt}\isanewline
\ \ \ \ \ \ \isacommand{using}\isamarkupfalse%
\ {\isasymDelta}{\isadigit{0}}{\isacharunderscore}{\kern0pt}subset\ assms{\isadigit{1}}\isanewline
\ \ \ \ \ \ \ \isacommand{apply}\isamarkupfalse%
\ blast\isanewline
\ \ \ \ \ \ \isacommand{unfolding}\isamarkupfalse%
\ {\isasymDelta}{\isadigit{0}}{\isacharunderscore}{\kern0pt}def\ \isanewline
\ \ \ \ \ \ \isacommand{using}\isamarkupfalse%
\ assms{\isadigit{1}}\isanewline
\ \ \ \ \ \ \isacommand{by}\isamarkupfalse%
\ auto\isanewline
\isanewline
\ \ \ \ \isacommand{show}\isamarkupfalse%
\ {\isachardoublequoteopen}\ {\isasymforall}m{\isasymin}nat{\isachardot}{\kern0pt}\ {\isasymforall}n{\isasymin}nat{\isachardot}{\kern0pt}\ {\isasymforall}f{\isasymin}m\ {\isasymrightarrow}\ n{\isachardot}{\kern0pt}\ {\isasymforall}{\isasymphi}{\isasymin}formula{\isachardot}{\kern0pt}\ {\isasymphi}\ {\isasymin}\ {\isasymDelta}{\isadigit{0}}{\isacharunderscore}{\kern0pt}from{\isacharcircum}{\kern0pt}succ{\isacharparenleft}{\kern0pt}i{\isacharparenright}{\kern0pt}\ {\isacharparenleft}{\kern0pt}{\isasymDelta}{\isadigit{0}}{\isacharunderscore}{\kern0pt}base{\isacharparenright}{\kern0pt}\ {\isasymlongrightarrow}\ arity{\isacharparenleft}{\kern0pt}{\isasymphi}{\isacharparenright}{\kern0pt}\ {\isasymle}\ m\ {\isasymlongrightarrow}\ ren{\isacharparenleft}{\kern0pt}{\isasymphi}{\isacharparenright}{\kern0pt}\ {\isacharbackquote}{\kern0pt}\ m\ {\isacharbackquote}{\kern0pt}\ n\ {\isacharbackquote}{\kern0pt}\ f\ {\isasymin}\ {\isasymDelta}{\isadigit{0}}{\isacharunderscore}{\kern0pt}from{\isacharcircum}{\kern0pt}succ{\isacharparenleft}{\kern0pt}i{\isacharparenright}{\kern0pt}\ {\isacharparenleft}{\kern0pt}{\isasymDelta}{\isadigit{0}}{\isacharunderscore}{\kern0pt}base{\isacharparenright}{\kern0pt}{\isachardoublequoteclose}\ \isanewline
\ \ \ \ \isacommand{proof}\isamarkupfalse%
\ {\isacharparenleft}{\kern0pt}rule\ ballI{\isacharcomma}{\kern0pt}\ rule\ ballI{\isacharcomma}{\kern0pt}\ rule\ ballI{\isacharcomma}{\kern0pt}\ rule\ ballI\ {\isacharcomma}{\kern0pt}\ rule\ impI{\isacharcomma}{\kern0pt}\ rule\ impI{\isacharparenright}{\kern0pt}\isanewline
\ \ \ \ \ \ \isacommand{fix}\isamarkupfalse%
\ m\ n\ f\ {\isasymphi}\ \isanewline
\ \ \ \ \ \ \isacommand{assume}\isamarkupfalse%
\ assms{\isadigit{2}}{\isacharcolon}{\kern0pt}\ {\isachardoublequoteopen}m\ {\isasymin}\ nat{\isachardoublequoteclose}\ {\isachardoublequoteopen}n\ {\isasymin}\ nat{\isachardoublequoteclose}\ {\isachardoublequoteopen}f\ {\isasymin}\ m\ {\isasymrightarrow}\ n{\isachardoublequoteclose}\ {\isachardoublequoteopen}{\isasymphi}\ {\isasymin}\ formula{\isachardoublequoteclose}\ {\isachardoublequoteopen}{\isasymphi}\ {\isasymin}\ {\isasymDelta}{\isadigit{0}}{\isacharunderscore}{\kern0pt}from{\isacharcircum}{\kern0pt}succ{\isacharparenleft}{\kern0pt}i{\isacharparenright}{\kern0pt}\ {\isacharparenleft}{\kern0pt}{\isasymDelta}{\isadigit{0}}{\isacharunderscore}{\kern0pt}base{\isacharparenright}{\kern0pt}{\isachardoublequoteclose}\ {\isachardoublequoteopen}arity{\isacharparenleft}{\kern0pt}{\isasymphi}{\isacharparenright}{\kern0pt}\ {\isasymle}\ m{\isachardoublequoteclose}\isanewline
\isanewline
\ \ \ \ \ \ \isacommand{have}\isamarkupfalse%
\ H{\isacharcolon}{\kern0pt}\ {\isachardoublequoteopen}{\isasymphi}\ {\isasymin}\ {\isasymDelta}{\isadigit{0}}{\isacharunderscore}{\kern0pt}from{\isacharcircum}{\kern0pt}i\ {\isacharparenleft}{\kern0pt}{\isasymDelta}{\isadigit{0}}{\isacharunderscore}{\kern0pt}base{\isacharparenright}{\kern0pt}\ {\isasymor}\ {\isacharparenleft}{\kern0pt}{\isasymexists}x{\isasymin}{\isasymDelta}{\isadigit{0}}{\isacharunderscore}{\kern0pt}from{\isacharcircum}{\kern0pt}i\ {\isacharparenleft}{\kern0pt}{\isasymDelta}{\isadigit{0}}{\isacharunderscore}{\kern0pt}base{\isacharparenright}{\kern0pt}{\isachardot}{\kern0pt}\ {\isasymexists}y{\isasymin}{\isasymDelta}{\isadigit{0}}{\isacharunderscore}{\kern0pt}from{\isacharcircum}{\kern0pt}i\ {\isacharparenleft}{\kern0pt}{\isasymDelta}{\isadigit{0}}{\isacharunderscore}{\kern0pt}base{\isacharparenright}{\kern0pt}{\isachardot}{\kern0pt}\ {\isasymphi}\ {\isacharequal}{\kern0pt}\ Nand{\isacharparenleft}{\kern0pt}x{\isacharcomma}{\kern0pt}\ y{\isacharparenright}{\kern0pt}{\isacharparenright}{\kern0pt}\ {\isasymor}\isanewline
\ \ \ \ \ \ \ \ \ \ \ \ {\isacharparenleft}{\kern0pt}{\isasymexists}j{\isasymin}nat{\isachardot}{\kern0pt}\ {\isasymexists}{\isasympsi}{\isasymin}{\isasymDelta}{\isadigit{0}}{\isacharunderscore}{\kern0pt}from{\isacharcircum}{\kern0pt}i\ {\isacharparenleft}{\kern0pt}{\isasymDelta}{\isadigit{0}}{\isacharunderscore}{\kern0pt}base{\isacharparenright}{\kern0pt}{\isachardot}{\kern0pt}\ j\ {\isasymnoteq}\ {\isadigit{0}}\ {\isasymand}\ {\isasymphi}\ {\isacharequal}{\kern0pt}\ BExists{\isacharprime}{\kern0pt}{\isacharparenleft}{\kern0pt}j{\isacharcomma}{\kern0pt}\ {\isasympsi}{\isacharparenright}{\kern0pt}{\isacharparenright}{\kern0pt}{\isachardoublequoteclose}\isanewline
\ \ \ \ \ \ \ \ \isacommand{using}\isamarkupfalse%
\ assms{\isadigit{2}}\ {\isasymDelta}{\isadigit{0}}{\isacharunderscore}{\kern0pt}from{\isacharunderscore}{\kern0pt}def\isanewline
\ \ \ \ \ \ \ \ \isacommand{by}\isamarkupfalse%
\ auto\isanewline
\isanewline
\ \ \ \ \ \ \isacommand{show}\isamarkupfalse%
\ {\isachardoublequoteopen}ren{\isacharparenleft}{\kern0pt}{\isasymphi}{\isacharparenright}{\kern0pt}\ {\isacharbackquote}{\kern0pt}\ m\ {\isacharbackquote}{\kern0pt}\ n\ {\isacharbackquote}{\kern0pt}\ f\ {\isasymin}\ {\isasymDelta}{\isadigit{0}}{\isacharunderscore}{\kern0pt}from{\isacharcircum}{\kern0pt}succ{\isacharparenleft}{\kern0pt}i{\isacharparenright}{\kern0pt}\ {\isacharparenleft}{\kern0pt}{\isasymDelta}{\isadigit{0}}{\isacharunderscore}{\kern0pt}base{\isacharparenright}{\kern0pt}{\isachardoublequoteclose}\ \isanewline
\ \ \ \ \ \ \ \ \isacommand{using}\isamarkupfalse%
\ H\ \isanewline
\ \ \ \ \ \ \ \ \isacommand{apply}\isamarkupfalse%
{\isacharparenleft}{\kern0pt}rule\ disjE{\isacharcomma}{\kern0pt}\ simp{\isacharparenright}{\kern0pt}\isanewline
\ \ \ \ \ \ \ \ \isacommand{using}\isamarkupfalse%
\ assms{\isadigit{1}}\ assms{\isadigit{2}}\ {\isasymDelta}{\isadigit{0}}{\isacharunderscore}{\kern0pt}from{\isacharunderscore}{\kern0pt}def\isanewline
\ \ \ \ \ \ \ \ \ \isacommand{apply}\isamarkupfalse%
\ force\isanewline
\ \ \ \ \ \ \ \ \isacommand{apply}\isamarkupfalse%
{\isacharparenleft}{\kern0pt}rule\ disjE{\isacharcomma}{\kern0pt}\ simp{\isacharcomma}{\kern0pt}\ simp{\isacharparenright}{\kern0pt}\isanewline
\ \ \ \ \ \ \ \ \ \isacommand{apply}\isamarkupfalse%
\ clarify\isanewline
\ \ \ \ \ \ \ \ \ \isacommand{apply}\isamarkupfalse%
{\isacharparenleft}{\kern0pt}rename{\isacharunderscore}{\kern0pt}tac\ x\ y{\isacharcomma}{\kern0pt}\ subgoal{\isacharunderscore}{\kern0pt}tac\ {\isachardoublequoteopen}x\ {\isasymin}\ {\isasymDelta}{\isadigit{0}}\ {\isasymand}\ y\ {\isasymin}\ {\isasymDelta}{\isadigit{0}}{\isachardoublequoteclose}{\isacharparenright}{\kern0pt}\isanewline
\ \ \ \ \ \ \ \ \isacommand{using}\isamarkupfalse%
\ assms{\isadigit{2}}\isanewline
\ \ \ \ \ \ \ \ \ \isacommand{apply}\isamarkupfalse%
\ simp\isanewline
\ \ \ \ \ \ \ \ \ \isacommand{apply}\isamarkupfalse%
{\isacharparenleft}{\kern0pt}rename{\isacharunderscore}{\kern0pt}tac\ x\ y{\isacharcomma}{\kern0pt}\ subgoal{\isacharunderscore}{\kern0pt}tac\ {\isachardoublequoteopen}ren{\isacharparenleft}{\kern0pt}x{\isacharparenright}{\kern0pt}{\isacharbackquote}{\kern0pt}m{\isacharbackquote}{\kern0pt}n{\isacharbackquote}{\kern0pt}f\ {\isasymin}\ {\isasymDelta}{\isadigit{0}}{\isacharunderscore}{\kern0pt}from{\isacharcircum}{\kern0pt}i\ {\isacharparenleft}{\kern0pt}{\isasymDelta}{\isadigit{0}}{\isacharunderscore}{\kern0pt}base{\isacharparenright}{\kern0pt}\ {\isasymand}\ ren{\isacharparenleft}{\kern0pt}y{\isacharparenright}{\kern0pt}{\isacharbackquote}{\kern0pt}m{\isacharbackquote}{\kern0pt}n{\isacharbackquote}{\kern0pt}f\ {\isasymin}\ {\isasymDelta}{\isadigit{0}}{\isacharunderscore}{\kern0pt}from{\isacharcircum}{\kern0pt}i\ {\isacharparenleft}{\kern0pt}{\isasymDelta}{\isadigit{0}}{\isacharunderscore}{\kern0pt}base{\isacharparenright}{\kern0pt}{\isachardoublequoteclose}{\isacharparenright}{\kern0pt}\isanewline
\ \ \ \ \ \ \ \ \ \ \isacommand{apply}\isamarkupfalse%
{\isacharparenleft}{\kern0pt}subst\ {\isasymDelta}{\isadigit{0}}{\isacharunderscore}{\kern0pt}from{\isacharunderscore}{\kern0pt}def{\isacharparenright}{\kern0pt}\isanewline
\ \ \ \ \ \ \ \ \isacommand{using}\isamarkupfalse%
\ assms{\isadigit{2}}\ assms{\isadigit{1}}\isanewline
\ \ \ \ \ \ \ \ \ \ \isacommand{apply}\isamarkupfalse%
\ force\isanewline
\ \ \ \ \ \ \ \ \ \isacommand{apply}\isamarkupfalse%
{\isacharparenleft}{\kern0pt}rule\ conjI{\isacharcomma}{\kern0pt}\ rule\ ih{\isacharparenright}{\kern0pt}\isanewline
\ \ \ \ \ \ \ \ \ \ \ \ \ \ \isacommand{apply}\isamarkupfalse%
\ auto{\isacharbrackleft}{\kern0pt}{\isadigit{4}}{\isacharbrackright}{\kern0pt}\isanewline
\ \ \ \ \ \ \ \ \ \ \isacommand{apply}\isamarkupfalse%
{\isacharparenleft}{\kern0pt}rename{\isacharunderscore}{\kern0pt}tac\ x\ y{\isacharcomma}{\kern0pt}\ rule{\isacharunderscore}{\kern0pt}tac\ b{\isacharequal}{\kern0pt}{\isachardoublequoteopen}arity{\isacharparenleft}{\kern0pt}x{\isacharparenright}{\kern0pt}\ {\isasymunion}\ arity{\isacharparenleft}{\kern0pt}y{\isacharparenright}{\kern0pt}{\isachardoublequoteclose}\ \isakeyword{in}\ le{\isacharunderscore}{\kern0pt}lt{\isacharunderscore}{\kern0pt}lt{\isacharparenright}{\kern0pt}\isanewline
\ \ \ \ \ \ \ \ \ \ \ \ \isacommand{apply}\isamarkupfalse%
{\isacharparenleft}{\kern0pt}rule\ max{\isacharunderscore}{\kern0pt}le{\isadigit{1}}{\isacharparenright}{\kern0pt}\isanewline
\ \ \ \ \ \ \ \ \isacommand{using}\isamarkupfalse%
\ {\isasymDelta}{\isadigit{0}}{\isacharunderscore}{\kern0pt}subset\isanewline
\ \ \ \ \ \ \ \ \ \ \ \ \ \isacommand{apply}\isamarkupfalse%
\ auto{\isacharbrackleft}{\kern0pt}{\isadigit{3}}{\isacharbrackright}{\kern0pt}\isanewline
\ \ \ \ \ \ \ \ \ \ \isacommand{apply}\isamarkupfalse%
{\isacharparenleft}{\kern0pt}rule\ ih{\isacharparenright}{\kern0pt}\isanewline
\ \ \ \ \ \ \ \ \ \ \ \ \ \ \isacommand{apply}\isamarkupfalse%
\ auto{\isacharbrackleft}{\kern0pt}{\isadigit{4}}{\isacharbrackright}{\kern0pt}\isanewline
\ \ \ \ \ \ \ \ \ \ \isacommand{apply}\isamarkupfalse%
{\isacharparenleft}{\kern0pt}rename{\isacharunderscore}{\kern0pt}tac\ x\ y{\isacharcomma}{\kern0pt}\ rule{\isacharunderscore}{\kern0pt}tac\ b{\isacharequal}{\kern0pt}{\isachardoublequoteopen}arity{\isacharparenleft}{\kern0pt}x{\isacharparenright}{\kern0pt}\ {\isasymunion}\ arity{\isacharparenleft}{\kern0pt}y{\isacharparenright}{\kern0pt}{\isachardoublequoteclose}\ \isakeyword{in}\ le{\isacharunderscore}{\kern0pt}lt{\isacharunderscore}{\kern0pt}lt{\isacharparenright}{\kern0pt}\isanewline
\ \ \ \ \ \ \ \ \ \ \ \isacommand{apply}\isamarkupfalse%
{\isacharparenleft}{\kern0pt}rule\ max{\isacharunderscore}{\kern0pt}le{\isadigit{2}}{\isacharparenright}{\kern0pt}\isanewline
\ \ \ \ \ \ \ \ \isacommand{using}\isamarkupfalse%
\ {\isasymDelta}{\isadigit{0}}{\isacharunderscore}{\kern0pt}subset\isanewline
\ \ \ \ \ \ \ \ \ \ \ \ \isacommand{apply}\isamarkupfalse%
\ auto{\isacharbrackleft}{\kern0pt}{\isadigit{3}}{\isacharbrackright}{\kern0pt}\isanewline
\ \ \ \ \ \ \ \ \ \isacommand{apply}\isamarkupfalse%
{\isacharparenleft}{\kern0pt}simp\ add{\isacharcolon}{\kern0pt}{\isasymDelta}{\isadigit{0}}{\isacharunderscore}{\kern0pt}def{\isacharparenright}{\kern0pt}\isanewline
\ \ \ \ \ \ \ \ \isacommand{using}\isamarkupfalse%
\ assms{\isadigit{1}}\isanewline
\ \ \ \ \ \ \ \ \ \isacommand{apply}\isamarkupfalse%
\ force\isanewline
\ \ \ \ \ \ \ \ \isacommand{apply}\isamarkupfalse%
\ simp\isanewline
\ \ \ \ \ \ \ \ \isacommand{apply}\isamarkupfalse%
\ clarify\ \isanewline
\ \ \ \ \ \ \ \ \isacommand{apply}\isamarkupfalse%
{\isacharparenleft}{\kern0pt}rename{\isacharunderscore}{\kern0pt}tac\ j\ {\isasympsi}{\isacharcomma}{\kern0pt}\ subgoal{\isacharunderscore}{\kern0pt}tac\ {\isachardoublequoteopen}{\isasympsi}\ {\isasymin}\ {\isasymDelta}{\isadigit{0}}{\isachardoublequoteclose}{\isacharparenright}{\kern0pt}\isanewline
\ \ \ \ \ \ \ \ \ \isacommand{apply}\isamarkupfalse%
{\isacharparenleft}{\kern0pt}rename{\isacharunderscore}{\kern0pt}tac\ j\ {\isasympsi}{\isacharcomma}{\kern0pt}\ subgoal{\isacharunderscore}{\kern0pt}tac\ {\isachardoublequoteopen}j\ {\isasymle}\ m{\isachardoublequoteclose}{\isacharparenright}{\kern0pt}\isanewline
\ \ \ \ \ \ \ \ \ \ \isacommand{apply}\isamarkupfalse%
{\isacharparenleft}{\kern0pt}subst\ {\isasymDelta}{\isadigit{0}}{\isacharunderscore}{\kern0pt}from{\isacharunderscore}{\kern0pt}def{\isacharcomma}{\kern0pt}\ simp{\isacharcomma}{\kern0pt}\ rule\ disjI{\isadigit{2}}{\isacharcomma}{\kern0pt}\ rule\ disjI{\isadigit{2}}{\isacharcomma}{\kern0pt}\ rule\ conjI{\isacharparenright}{\kern0pt}\isanewline
\ \ \ \ \ \ \ \ \ \ \ \isacommand{apply}\isamarkupfalse%
{\isacharparenleft}{\kern0pt}rule\ ren{\isacharunderscore}{\kern0pt}tc{\isacharcomma}{\kern0pt}\ simp\ add{\isacharcolon}{\kern0pt}BExists{\isacharprime}{\kern0pt}{\isacharunderscore}{\kern0pt}def{\isacharparenright}{\kern0pt}\isanewline
\ \ \ \ \ \ \ \ \isacommand{using}\isamarkupfalse%
\ {\isasymDelta}{\isadigit{0}}{\isacharunderscore}{\kern0pt}subset\isanewline
\ \ \ \ \ \ \ \ \ \ \ \ \ \ \isacommand{apply}\isamarkupfalse%
\ force\isanewline
\ \ \ \ \ \ \ \ \isacommand{using}\isamarkupfalse%
\ assms{\isadigit{2}}\isanewline
\ \ \ \ \ \ \ \ \ \ \ \ \ \isacommand{apply}\isamarkupfalse%
\ auto{\isacharbrackleft}{\kern0pt}{\isadigit{3}}{\isacharbrackright}{\kern0pt}\isanewline
\ \ \ \ \ \ \ \ \ \ \isacommand{apply}\isamarkupfalse%
{\isacharparenleft}{\kern0pt}rename{\isacharunderscore}{\kern0pt}tac\ j\ {\isasympsi}{\isacharcomma}{\kern0pt}\ rule{\isacharunderscore}{\kern0pt}tac\ x{\isacharequal}{\kern0pt}{\isachardoublequoteopen}sum{\isacharunderscore}{\kern0pt}id{\isacharparenleft}{\kern0pt}m{\isacharcomma}{\kern0pt}\ f{\isacharparenright}{\kern0pt}{\isacharbackquote}{\kern0pt}j{\isachardoublequoteclose}\ \isakeyword{in}\ bexI{\isacharcomma}{\kern0pt}\ rule\ conjI{\isacharparenright}{\kern0pt}\isanewline
\ \ \ \ \ \ \ \ \ \ \ \ \isacommand{apply}\isamarkupfalse%
{\isacharparenleft}{\kern0pt}rename{\isacharunderscore}{\kern0pt}tac\ j\ {\isasympsi}{\isacharcomma}{\kern0pt}\ rule{\isacharunderscore}{\kern0pt}tac\ n{\isacharequal}{\kern0pt}j\ \isakeyword{in}\ natE{\isacharcomma}{\kern0pt}\ simp{\isacharcomma}{\kern0pt}\ simp{\isacharcomma}{\kern0pt}\ simp{\isacharparenright}{\kern0pt}\isanewline
\ \ \ \ \ \ \ \ \ \ \ \ \isacommand{apply}\isamarkupfalse%
{\isacharparenleft}{\kern0pt}subst\ sum{\isacharunderscore}{\kern0pt}idS{\isacharbrackleft}{\kern0pt}\isakeyword{where}\ q{\isacharequal}{\kern0pt}n{\isacharbrackright}{\kern0pt}{\isacharparenright}{\kern0pt}\isanewline
\ \ \ \ \ \ \ \ \isacommand{using}\isamarkupfalse%
\ assms{\isadigit{2}}\isanewline
\ \ \ \ \ \ \ \ \ \ \ \ \ \ \ \ \isacommand{apply}\isamarkupfalse%
\ auto{\isacharbrackleft}{\kern0pt}{\isadigit{3}}{\isacharbrackright}{\kern0pt}\isanewline
\ \ \ \ \ \ \ \ \ \ \ \ \ \isacommand{apply}\isamarkupfalse%
{\isacharparenleft}{\kern0pt}rule\ ltD{\isacharparenright}{\kern0pt}\isanewline
\ \ \ \ \ \ \ \ \ \ \ \ \ \isacommand{apply}\isamarkupfalse%
\ auto{\isacharbrackleft}{\kern0pt}{\isadigit{3}}{\isacharbrackright}{\kern0pt}\isanewline
\ \ \ \ \ \ \ \ \ \ \ \isacommand{apply}\isamarkupfalse%
{\isacharparenleft}{\kern0pt}rule{\isacharunderscore}{\kern0pt}tac\ x{\isacharequal}{\kern0pt}{\isachardoublequoteopen}ren{\isacharparenleft}{\kern0pt}{\isasympsi}{\isacharparenright}{\kern0pt}\ {\isacharbackquote}{\kern0pt}\ succ{\isacharparenleft}{\kern0pt}m{\isacharparenright}{\kern0pt}\ {\isacharbackquote}{\kern0pt}\ succ{\isacharparenleft}{\kern0pt}n{\isacharparenright}{\kern0pt}\ {\isacharbackquote}{\kern0pt}\ sum{\isacharunderscore}{\kern0pt}id{\isacharparenleft}{\kern0pt}m{\isacharcomma}{\kern0pt}\ f{\isacharparenright}{\kern0pt}{\isachardoublequoteclose}\ \isakeyword{in}\ bexI{\isacharparenright}{\kern0pt}\isanewline
\ \ \ \ \ \ \ \ \isacommand{unfolding}\isamarkupfalse%
\ BExists{\isacharprime}{\kern0pt}{\isacharunderscore}{\kern0pt}def\ Exists{\isacharunderscore}{\kern0pt}def\ Neg{\isacharunderscore}{\kern0pt}def\ And{\isacharunderscore}{\kern0pt}def\isanewline
\ \ \ \ \ \ \ \ \ \ \ \ \isacommand{apply}\isamarkupfalse%
{\isacharparenleft}{\kern0pt}subgoal{\isacharunderscore}{\kern0pt}tac\ {\isachardoublequoteopen}sum{\isacharunderscore}{\kern0pt}id{\isacharparenleft}{\kern0pt}m{\isacharcomma}{\kern0pt}\ f{\isacharparenright}{\kern0pt}\ {\isasymin}\ succ{\isacharparenleft}{\kern0pt}m{\isacharparenright}{\kern0pt}\ {\isasymrightarrow}\ succ{\isacharparenleft}{\kern0pt}n{\isacharparenright}{\kern0pt}{\isachardoublequoteclose}{\isacharcomma}{\kern0pt}\ simp\ add{\isacharcolon}{\kern0pt}assms{\isadigit{2}}{\isacharparenright}{\kern0pt}\isanewline
\ \ \ \ \ \ \ \ \ \ \ \ \ \isacommand{apply}\isamarkupfalse%
{\isacharparenleft}{\kern0pt}rule\ sum{\isacharunderscore}{\kern0pt}id{\isadigit{0}}{\isacharcomma}{\kern0pt}\ simp\ add{\isacharcolon}{\kern0pt}assms{\isadigit{2}}{\isacharcomma}{\kern0pt}\ rule\ sum{\isacharunderscore}{\kern0pt}id{\isacharunderscore}{\kern0pt}tc{\isacharparenright}{\kern0pt}\isanewline
\ \ \ \ \ \ \ \ \isacommand{using}\isamarkupfalse%
\ assms{\isadigit{2}}\isanewline
\ \ \ \ \ \ \ \ \ \ \ \ \ \ \isacommand{apply}\isamarkupfalse%
\ auto{\isacharbrackleft}{\kern0pt}{\isadigit{3}}{\isacharbrackright}{\kern0pt}\isanewline
\ \ \ \ \ \ \ \ \ \ \ \isacommand{apply}\isamarkupfalse%
{\isacharparenleft}{\kern0pt}rule\ ih{\isacharparenright}{\kern0pt}\isanewline
\ \ \ \ \ \ \ \ \isacommand{using}\isamarkupfalse%
\ assms{\isadigit{2}}\isanewline
\ \ \ \ \ \ \ \ \ \ \ \ \ \ \ \isacommand{apply}\isamarkupfalse%
\ auto{\isacharbrackleft}{\kern0pt}{\isadigit{2}}{\isacharbrackright}{\kern0pt}\isanewline
\ \ \ \ \ \ \ \ \ \ \ \ \ \isacommand{apply}\isamarkupfalse%
{\isacharparenleft}{\kern0pt}rule\ sum{\isacharunderscore}{\kern0pt}id{\isacharunderscore}{\kern0pt}tc{\isacharparenright}{\kern0pt}\isanewline
\ \ \ \ \ \ \ \ \isacommand{using}\isamarkupfalse%
\ assms{\isadigit{2}}\isanewline
\ \ \ \ \ \ \ \ \ \ \ \ \ \ \ \isacommand{apply}\isamarkupfalse%
\ auto{\isacharbrackleft}{\kern0pt}{\isadigit{4}}{\isacharbrackright}{\kern0pt}\isanewline
\ \ \ \ \ \ \ \ \ \ \ \isacommand{apply}\isamarkupfalse%
{\isacharparenleft}{\kern0pt}rename{\isacharunderscore}{\kern0pt}tac\ j\ {\isasympsi}{\isacharcomma}{\kern0pt}\ subgoal{\isacharunderscore}{\kern0pt}tac\ {\isachardoublequoteopen}arity{\isacharparenleft}{\kern0pt}{\isasympsi}{\isacharparenright}{\kern0pt}\ {\isasymle}\ succ{\isacharparenleft}{\kern0pt}arity{\isacharparenleft}{\kern0pt}{\isasymphi}{\isacharparenright}{\kern0pt}{\isacharparenright}{\kern0pt}{\isachardoublequoteclose}{\isacharparenright}{\kern0pt}\isanewline
\ \ \ \ \ \ \ \ \ \ \ \ \isacommand{apply}\isamarkupfalse%
{\isacharparenleft}{\kern0pt}rename{\isacharunderscore}{\kern0pt}tac\ j\ {\isasympsi}{\isacharcomma}{\kern0pt}\ rule{\isacharunderscore}{\kern0pt}tac\ j{\isacharequal}{\kern0pt}{\isachardoublequoteopen}succ{\isacharparenleft}{\kern0pt}arity{\isacharparenleft}{\kern0pt}{\isasymphi}{\isacharparenright}{\kern0pt}{\isacharparenright}{\kern0pt}{\isachardoublequoteclose}\ \isakeyword{in}\ le{\isacharunderscore}{\kern0pt}trans{\isacharcomma}{\kern0pt}\ simp{\isacharparenright}{\kern0pt}\isanewline
\ \ \ \ \ \ \ \ \isacommand{using}\isamarkupfalse%
\ assms{\isadigit{2}}\isanewline
\ \ \ \ \ \ \ \ \ \ \ \ \isacommand{apply}\isamarkupfalse%
\ {\isacharparenleft}{\kern0pt}simp{\isacharcomma}{\kern0pt}\ simp{\isacharparenright}{\kern0pt}\isanewline
\ \ \ \ \ \ \ \ \ \ \ \isacommand{apply}\isamarkupfalse%
{\isacharparenleft}{\kern0pt}subst\ succ{\isacharunderscore}{\kern0pt}pred{\isacharunderscore}{\kern0pt}eq{\isacharparenright}{\kern0pt}\isanewline
\ \ \ \ \ \ \ \ \isacommand{using}\isamarkupfalse%
\ {\isasymDelta}{\isadigit{0}}{\isacharunderscore}{\kern0pt}subset\isanewline
\ \ \ \ \ \ \ \ \ \ \ \ \ \isacommand{apply}\isamarkupfalse%
\ force\isanewline
\ \ \ \ \ \ \ \ \ \ \ \ \isacommand{apply}\isamarkupfalse%
\ simp\isanewline
\ \ \ \ \ \ \ \ \ \ \ \isacommand{apply}\isamarkupfalse%
{\isacharparenleft}{\kern0pt}rule\ ltI{\isacharparenright}{\kern0pt}\isanewline
\ \ \ \ \ \ \ \ \isacommand{using}\isamarkupfalse%
\ {\isasymDelta}{\isadigit{0}}{\isacharunderscore}{\kern0pt}subset\isanewline
\ \ \ \ \ \ \ \ \ \ \ \ \isacommand{apply}\isamarkupfalse%
{\isacharparenleft}{\kern0pt}subst\ succ{\isacharunderscore}{\kern0pt}Un{\isacharunderscore}{\kern0pt}distrib{\isacharcomma}{\kern0pt}\ simp{\isacharcomma}{\kern0pt}\ force{\isacharparenright}{\kern0pt}{\isacharplus}{\kern0pt}\isanewline
\ \ \ \ \ \ \ \ \ \ \ \ \isacommand{apply}\isamarkupfalse%
\ simp\isanewline
\ \ \ \ \ \ \ \ \isacommand{using}\isamarkupfalse%
\ {\isasymDelta}{\isadigit{0}}{\isacharunderscore}{\kern0pt}subset\isanewline
\ \ \ \ \ \ \ \ \ \ \ \isacommand{apply}\isamarkupfalse%
\ force\isanewline
\ \ \ \ \ \ \ \ \ \ \isacommand{apply}\isamarkupfalse%
{\isacharparenleft}{\kern0pt}rule{\isacharunderscore}{\kern0pt}tac\ m{\isacharequal}{\kern0pt}{\isachardoublequoteopen}succ{\isacharparenleft}{\kern0pt}m{\isacharparenright}{\kern0pt}{\isachardoublequoteclose}\ \isakeyword{and}\ n{\isacharequal}{\kern0pt}{\isachardoublequoteopen}succ{\isacharparenleft}{\kern0pt}n{\isacharparenright}{\kern0pt}{\isachardoublequoteclose}\ \isakeyword{in}\ value{\isacharunderscore}{\kern0pt}type{\isacharparenright}{\kern0pt}\isanewline
\ \ \ \ \ \ \ \ \isacommand{using}\isamarkupfalse%
\ assms{\isadigit{2}}\isanewline
\ \ \ \ \ \ \ \ \ \ \ \ \isacommand{apply}\isamarkupfalse%
\ auto{\isacharbrackleft}{\kern0pt}{\isadigit{2}}{\isacharbrackright}{\kern0pt}\isanewline
\ \ \ \ \ \ \ \ \ \ \isacommand{apply}\isamarkupfalse%
{\isacharparenleft}{\kern0pt}rule\ sum{\isacharunderscore}{\kern0pt}id{\isacharunderscore}{\kern0pt}tc{\isacharparenright}{\kern0pt}\isanewline
\ \ \ \ \ \ \ \ \isacommand{using}\isamarkupfalse%
\ assms{\isadigit{2}}\isanewline
\ \ \ \ \ \ \ \ \ \ \ \ \isacommand{apply}\isamarkupfalse%
\ auto{\isacharbrackleft}{\kern0pt}{\isadigit{3}}{\isacharbrackright}{\kern0pt}\isanewline
\ \ \ \ \ \ \ \ \ \isacommand{apply}\isamarkupfalse%
{\isacharparenleft}{\kern0pt}rule{\isacharunderscore}{\kern0pt}tac\ b{\isacharequal}{\kern0pt}{\isachardoublequoteopen}arity{\isacharparenleft}{\kern0pt}{\isasymphi}{\isacharparenright}{\kern0pt}{\isachardoublequoteclose}\ \isakeyword{in}\ le{\isacharunderscore}{\kern0pt}lt{\isacharunderscore}{\kern0pt}lt{\isacharparenright}{\kern0pt}\isanewline
\ \ \ \ \ \ \ \ \ \ \isacommand{apply}\isamarkupfalse%
\ simp\isanewline
\ \ \ \ \ \ \ \ \ \ \isacommand{apply}\isamarkupfalse%
{\isacharparenleft}{\kern0pt}rename{\isacharunderscore}{\kern0pt}tac\ j\ {\isasympsi}{\isacharcomma}{\kern0pt}\ subgoal{\isacharunderscore}{\kern0pt}tac\ {\isachardoublequoteopen}succ{\isacharparenleft}{\kern0pt}j{\isacharparenright}{\kern0pt}\ {\isasymle}\ succ{\isacharparenleft}{\kern0pt}pred{\isacharparenleft}{\kern0pt}{\isadigit{1}}\ {\isasymunion}\ succ{\isacharparenleft}{\kern0pt}j{\isacharparenright}{\kern0pt}\ {\isasymunion}\ arity{\isacharparenleft}{\kern0pt}{\isasympsi}{\isacharparenright}{\kern0pt}{\isacharparenright}{\kern0pt}{\isacharparenright}{\kern0pt}{\isachardoublequoteclose}{\isacharparenright}{\kern0pt}\isanewline
\ \ \ \ \ \ \ \ \ \ \ \isacommand{apply}\isamarkupfalse%
\ simp\isanewline
\ \ \ \ \ \ \ \ \ \ \isacommand{apply}\isamarkupfalse%
{\isacharparenleft}{\kern0pt}subst\ succ{\isacharunderscore}{\kern0pt}pred{\isacharunderscore}{\kern0pt}eq{\isacharparenright}{\kern0pt}\isanewline
\ \ \ \ \ \ \ \ \isacommand{using}\isamarkupfalse%
\ {\isasymDelta}{\isadigit{0}}{\isacharunderscore}{\kern0pt}subset\isanewline
\ \ \ \ \ \ \ \ \ \ \ \ \isacommand{apply}\isamarkupfalse%
\ force\isanewline
\ \ \ \ \ \ \ \ \ \ \ \isacommand{apply}\isamarkupfalse%
\ simp\isanewline
\ \ \ \ \ \ \ \ \isacommand{apply}\isamarkupfalse%
{\isacharparenleft}{\kern0pt}rule\ ltI{\isacharparenright}{\kern0pt}\isanewline
\ \ \ \ \ \ \ \ \isacommand{using}\isamarkupfalse%
\ {\isasymDelta}{\isadigit{0}}{\isacharunderscore}{\kern0pt}subset\isanewline
\ \ \ \ \ \ \ \ \ \ \ \isacommand{apply}\isamarkupfalse%
{\isacharparenleft}{\kern0pt}subst\ succ{\isacharunderscore}{\kern0pt}Un{\isacharunderscore}{\kern0pt}distrib{\isacharcomma}{\kern0pt}\ simp{\isacharcomma}{\kern0pt}\ force{\isacharparenright}{\kern0pt}{\isacharplus}{\kern0pt}\isanewline
\ \ \ \ \ \ \ \ \ \ \ \isacommand{apply}\isamarkupfalse%
\ simp\isanewline
\ \ \ \ \ \ \ \ \isacommand{using}\isamarkupfalse%
\ {\isasymDelta}{\isadigit{0}}{\isacharunderscore}{\kern0pt}subset\ assms{\isadigit{2}}\isanewline
\ \ \ \ \ \ \ \ \ \ \isacommand{apply}\isamarkupfalse%
\ auto{\isacharbrackleft}{\kern0pt}{\isadigit{2}}{\isacharbrackright}{\kern0pt}\isanewline
\ \ \ \ \ \ \ \ \isacommand{unfolding}\isamarkupfalse%
\ {\isasymDelta}{\isadigit{0}}{\isacharunderscore}{\kern0pt}def\isanewline
\ \ \ \ \ \ \ \ \isacommand{using}\isamarkupfalse%
\ assms{\isadigit{1}}\isanewline
\ \ \ \ \ \ \ \ \isacommand{by}\isamarkupfalse%
\ auto\isanewline
\ \ \ \ \isacommand{qed}\isamarkupfalse%
\isanewline
\ \ \isacommand{qed}\isamarkupfalse%
\isanewline
\ \ \isacommand{then}\isamarkupfalse%
\ \isacommand{have}\isamarkupfalse%
\ main\ {\isacharcolon}{\kern0pt}\ {\isachardoublequoteopen}{\isasymAnd}i\ m\ n\ f\ {\isasymphi}{\isachardot}{\kern0pt}\ i\ {\isasymin}\ nat\ {\isasymLongrightarrow}\ m\ {\isasymin}\ nat\ {\isasymLongrightarrow}\ n\ {\isasymin}\ nat\ {\isasymLongrightarrow}\ f\ {\isasymin}\ m\ {\isasymrightarrow}\ n\ {\isasymLongrightarrow}\ arity{\isacharparenleft}{\kern0pt}{\isasymphi}{\isacharparenright}{\kern0pt}\ {\isasymle}\ \ m\ {\isasymLongrightarrow}\ {\isasymphi}\ {\isasymin}\ formula\ {\isasymLongrightarrow}\ {\isasymphi}\ {\isasymin}\ {\isasymDelta}{\isadigit{0}}{\isacharunderscore}{\kern0pt}from{\isacharcircum}{\kern0pt}i{\isacharparenleft}{\kern0pt}{\isasymDelta}{\isadigit{0}}{\isacharunderscore}{\kern0pt}base{\isacharparenright}{\kern0pt}\ {\isasymLongrightarrow}\ ren{\isacharparenleft}{\kern0pt}{\isasymphi}{\isacharparenright}{\kern0pt}{\isacharbackquote}{\kern0pt}m{\isacharbackquote}{\kern0pt}n{\isacharbackquote}{\kern0pt}f\ {\isasymin}\ {\isasymDelta}{\isadigit{0}}{\isacharunderscore}{\kern0pt}from{\isacharcircum}{\kern0pt}i{\isacharparenleft}{\kern0pt}{\isasymDelta}{\isadigit{0}}{\isacharunderscore}{\kern0pt}base{\isacharparenright}{\kern0pt}{\isachardoublequoteclose}\isanewline
\ \ \ \ \isacommand{by}\isamarkupfalse%
\ auto\isanewline
\isanewline
\ \ \isacommand{have}\isamarkupfalse%
\ {\isachardoublequoteopen}{\isasymexists}i\ {\isasymin}\ nat{\isachardot}{\kern0pt}\ {\isasymphi}\ {\isasymin}\ {\isasymDelta}{\isadigit{0}}{\isacharunderscore}{\kern0pt}from{\isacharcircum}{\kern0pt}i{\isacharparenleft}{\kern0pt}{\isasymDelta}{\isadigit{0}}{\isacharunderscore}{\kern0pt}base{\isacharparenright}{\kern0pt}{\isachardoublequoteclose}\ \isanewline
\ \ \ \ \isacommand{using}\isamarkupfalse%
\ assms\isanewline
\ \ \ \ \isacommand{unfolding}\isamarkupfalse%
\ {\isasymDelta}{\isadigit{0}}{\isacharunderscore}{\kern0pt}def\isanewline
\ \ \ \ \isacommand{by}\isamarkupfalse%
\ auto\isanewline
\ \ \isacommand{then}\isamarkupfalse%
\ \isacommand{obtain}\isamarkupfalse%
\ i\ \isakeyword{where}\ iH{\isacharcolon}{\kern0pt}\ {\isachardoublequoteopen}i\ {\isasymin}\ nat{\isachardoublequoteclose}\ {\isachardoublequoteopen}{\isasymphi}\ {\isasymin}\ {\isasymDelta}{\isadigit{0}}{\isacharunderscore}{\kern0pt}from{\isacharcircum}{\kern0pt}i{\isacharparenleft}{\kern0pt}{\isasymDelta}{\isadigit{0}}{\isacharunderscore}{\kern0pt}base{\isacharparenright}{\kern0pt}{\isachardoublequoteclose}\ \isacommand{by}\isamarkupfalse%
\ auto\isanewline
\isanewline
\ \ \isacommand{show}\isamarkupfalse%
\ {\isacharquery}{\kern0pt}thesis\ \isanewline
\ \ \ \ \isacommand{unfolding}\isamarkupfalse%
\ {\isasymDelta}{\isadigit{0}}{\isacharunderscore}{\kern0pt}def\ \isanewline
\ \ \ \ \isacommand{apply}\isamarkupfalse%
\ simp\isanewline
\ \ \ \ \isacommand{apply}\isamarkupfalse%
{\isacharparenleft}{\kern0pt}rule{\isacharunderscore}{\kern0pt}tac\ x{\isacharequal}{\kern0pt}i\ \isakeyword{in}\ bexI{\isacharparenright}{\kern0pt}\isanewline
\ \ \ \ \ \isacommand{apply}\isamarkupfalse%
{\isacharparenleft}{\kern0pt}rule\ main{\isacharparenright}{\kern0pt}\isanewline
\ \ \ \ \isacommand{using}\isamarkupfalse%
\ assms\ iH\ {\isasymDelta}{\isadigit{0}}{\isacharunderscore}{\kern0pt}subset\ \isanewline
\ \ \ \ \isacommand{apply}\isamarkupfalse%
\ auto\isanewline
\ \ \ \ \isacommand{done}\isamarkupfalse%
\isanewline
\isacommand{qed}\isamarkupfalse%
%
\endisatagproof
{\isafoldproof}%
%
\isadelimproof
\isanewline
%
\endisadelimproof
\isanewline
\isacommand{lemma}\isamarkupfalse%
\ sats{\isacharunderscore}{\kern0pt}BExists{\isacharunderscore}{\kern0pt}iff\ {\isacharcolon}{\kern0pt}\isanewline
\ \ \isakeyword{fixes}\ A\ {\isasymphi}\ env\ n\ \isanewline
\ \ \isakeyword{assumes}\ {\isachardoublequoteopen}{\isasymphi}\ {\isasymin}\ formula{\isachardoublequoteclose}\ {\isachardoublequoteopen}n\ {\isacharless}{\kern0pt}\ length{\isacharparenleft}{\kern0pt}env{\isacharparenright}{\kern0pt}{\isachardoublequoteclose}\ {\isachardoublequoteopen}env\ {\isasymin}\ list{\isacharparenleft}{\kern0pt}A{\isacharparenright}{\kern0pt}{\isachardoublequoteclose}\ \isanewline
\ \ \isakeyword{shows}\ {\isachardoublequoteopen}sats{\isacharparenleft}{\kern0pt}A{\isacharcomma}{\kern0pt}\ BExists{\isacharparenleft}{\kern0pt}n{\isacharcomma}{\kern0pt}\ {\isasymphi}{\isacharparenright}{\kern0pt}{\isacharcomma}{\kern0pt}\ env{\isacharparenright}{\kern0pt}\ {\isasymlongleftrightarrow}\ {\isacharparenleft}{\kern0pt}{\isasymexists}x\ {\isasymin}\ A{\isachardot}{\kern0pt}\ x\ {\isasymin}\ nth{\isacharparenleft}{\kern0pt}n{\isacharcomma}{\kern0pt}\ env{\isacharparenright}{\kern0pt}\ {\isasymand}\ sats{\isacharparenleft}{\kern0pt}A{\isacharcomma}{\kern0pt}\ {\isasymphi}{\isacharcomma}{\kern0pt}\ Cons{\isacharparenleft}{\kern0pt}x{\isacharcomma}{\kern0pt}\ env{\isacharparenright}{\kern0pt}{\isacharparenright}{\kern0pt}{\isacharparenright}{\kern0pt}{\isachardoublequoteclose}\ \isanewline
%
\isadelimproof
\ \ %
\endisadelimproof
%
\isatagproof
\isacommand{unfolding}\isamarkupfalse%
\ BExists{\isacharunderscore}{\kern0pt}def\ BExists{\isacharprime}{\kern0pt}{\isacharunderscore}{\kern0pt}def\ \isanewline
\ \ \isacommand{using}\isamarkupfalse%
\ assms\ lt{\isacharunderscore}{\kern0pt}nat{\isacharunderscore}{\kern0pt}in{\isacharunderscore}{\kern0pt}nat\isanewline
\ \ \isacommand{by}\isamarkupfalse%
\ auto%
\endisatagproof
{\isafoldproof}%
%
\isadelimproof
\isanewline
%
\endisadelimproof
\ \ \isanewline
\isacommand{lemma}\isamarkupfalse%
\ sats{\isacharunderscore}{\kern0pt}BForall{\isacharunderscore}{\kern0pt}iff\ {\isacharcolon}{\kern0pt}\ \isanewline
\ \ \isakeyword{fixes}\ A\ {\isasymphi}\ env\ n\ \isanewline
\ \ \isakeyword{assumes}\ {\isachardoublequoteopen}{\isasymphi}\ {\isasymin}\ formula{\isachardoublequoteclose}\ {\isachardoublequoteopen}n\ {\isacharless}{\kern0pt}\ length{\isacharparenleft}{\kern0pt}env{\isacharparenright}{\kern0pt}{\isachardoublequoteclose}\ {\isachardoublequoteopen}env\ {\isasymin}\ list{\isacharparenleft}{\kern0pt}A{\isacharparenright}{\kern0pt}{\isachardoublequoteclose}\ \isanewline
\ \ \isakeyword{shows}\ {\isachardoublequoteopen}sats{\isacharparenleft}{\kern0pt}A{\isacharcomma}{\kern0pt}\ BForall{\isacharparenleft}{\kern0pt}n{\isacharcomma}{\kern0pt}\ {\isasymphi}{\isacharparenright}{\kern0pt}{\isacharcomma}{\kern0pt}\ env{\isacharparenright}{\kern0pt}\ {\isasymlongleftrightarrow}\ {\isacharparenleft}{\kern0pt}{\isasymforall}x\ {\isasymin}\ A{\isachardot}{\kern0pt}\ x\ {\isasymin}\ nth{\isacharparenleft}{\kern0pt}n{\isacharcomma}{\kern0pt}\ env{\isacharparenright}{\kern0pt}\ {\isasymlongrightarrow}\ sats{\isacharparenleft}{\kern0pt}A{\isacharcomma}{\kern0pt}\ {\isasymphi}{\isacharcomma}{\kern0pt}\ Cons{\isacharparenleft}{\kern0pt}x{\isacharcomma}{\kern0pt}\ env{\isacharparenright}{\kern0pt}{\isacharparenright}{\kern0pt}{\isacharparenright}{\kern0pt}{\isachardoublequoteclose}\ \isanewline
%
\isadelimproof
\ \ %
\endisadelimproof
%
\isatagproof
\isacommand{unfolding}\isamarkupfalse%
\ BForall{\isacharunderscore}{\kern0pt}def\ BExists{\isacharunderscore}{\kern0pt}def\ BExists{\isacharprime}{\kern0pt}{\isacharunderscore}{\kern0pt}def\ \isanewline
\ \ \isacommand{using}\isamarkupfalse%
\ assms\ lt{\isacharunderscore}{\kern0pt}nat{\isacharunderscore}{\kern0pt}in{\isacharunderscore}{\kern0pt}nat\isanewline
\ \ \isacommand{by}\isamarkupfalse%
\ auto%
\endisatagproof
{\isafoldproof}%
%
\isadelimproof
\isanewline
%
\endisadelimproof
\isanewline
\isacommand{lemma}\isamarkupfalse%
\ arity{\isacharunderscore}{\kern0pt}BExists\ {\isacharcolon}{\kern0pt}\ \isanewline
\ \ \isakeyword{fixes}\ {\isasymphi}\ n\ \isanewline
\ \ \isakeyword{assumes}\ {\isachardoublequoteopen}n\ {\isasymin}\ nat{\isachardoublequoteclose}\ {\isachardoublequoteopen}{\isasymphi}\ {\isasymin}\ formula{\isachardoublequoteclose}\ \isanewline
\ \ \isakeyword{shows}\ {\isachardoublequoteopen}arity{\isacharparenleft}{\kern0pt}BExists{\isacharparenleft}{\kern0pt}n{\isacharcomma}{\kern0pt}\ {\isasymphi}{\isacharparenright}{\kern0pt}{\isacharparenright}{\kern0pt}\ {\isasymle}\ succ{\isacharparenleft}{\kern0pt}n{\isacharparenright}{\kern0pt}\ {\isasymunion}\ pred{\isacharparenleft}{\kern0pt}arity{\isacharparenleft}{\kern0pt}{\isasymphi}{\isacharparenright}{\kern0pt}{\isacharparenright}{\kern0pt}{\isachardoublequoteclose}\ \isanewline
%
\isadelimproof
\isanewline
\ \ %
\endisadelimproof
%
\isatagproof
\isacommand{unfolding}\isamarkupfalse%
\ BExists{\isacharunderscore}{\kern0pt}def\ BExists{\isacharprime}{\kern0pt}{\isacharunderscore}{\kern0pt}def\ \isanewline
\ \ \isacommand{using}\isamarkupfalse%
\ assms\isanewline
\ \ \isacommand{apply}\isamarkupfalse%
\ simp\isanewline
\ \ \isacommand{apply}\isamarkupfalse%
{\isacharparenleft}{\kern0pt}rule\ pred{\isacharunderscore}{\kern0pt}le{\isacharcomma}{\kern0pt}\ simp{\isacharunderscore}{\kern0pt}all{\isacharparenright}{\kern0pt}\isanewline
\ \ \isacommand{apply}\isamarkupfalse%
{\isacharparenleft}{\kern0pt}rule\ Un{\isacharunderscore}{\kern0pt}least{\isacharunderscore}{\kern0pt}lt{\isacharparenright}{\kern0pt}{\isacharplus}{\kern0pt}\isanewline
\ \ \ \ \isacommand{apply}\isamarkupfalse%
\ simp{\isacharunderscore}{\kern0pt}all\isanewline
\ \ \ \isacommand{apply}\isamarkupfalse%
{\isacharparenleft}{\kern0pt}rule\ ltI{\isacharcomma}{\kern0pt}\ simp{\isacharunderscore}{\kern0pt}all{\isacharparenright}{\kern0pt}\isanewline
\ \ \isacommand{apply}\isamarkupfalse%
{\isacharparenleft}{\kern0pt}subst\ succ{\isacharunderscore}{\kern0pt}Un{\isacharunderscore}{\kern0pt}distrib{\isacharcomma}{\kern0pt}\ simp{\isacharunderscore}{\kern0pt}all{\isacharparenright}{\kern0pt}\isanewline
\ \ \isacommand{apply}\isamarkupfalse%
{\isacharparenleft}{\kern0pt}rule{\isacharunderscore}{\kern0pt}tac\ n{\isacharequal}{\kern0pt}{\isachardoublequoteopen}arity{\isacharparenleft}{\kern0pt}{\isasymphi}{\isacharparenright}{\kern0pt}{\isachardoublequoteclose}\ \isakeyword{in}\ natE{\isacharcomma}{\kern0pt}\ simp{\isacharunderscore}{\kern0pt}all{\isacharparenright}{\kern0pt}\isanewline
\ \ \isacommand{apply}\isamarkupfalse%
{\isacharparenleft}{\kern0pt}rule\ ltI{\isacharcomma}{\kern0pt}\ simp{\isacharunderscore}{\kern0pt}all{\isacharparenright}{\kern0pt}\isanewline
\ \ \isacommand{done}\isamarkupfalse%
%
\endisatagproof
{\isafoldproof}%
%
\isadelimproof
\isanewline
%
\endisadelimproof
%
\isadelimtheory
\isanewline
%
\endisadelimtheory
%
\isatagtheory
\isacommand{end}\isamarkupfalse%
%
\endisatagtheory
{\isafoldtheory}%
%
\isadelimtheory
%
\endisadelimtheory
%
\end{isabellebody}%
\endinput
%:%file=~/source/repos/ZF-notAC/code/Delta0.thy%:%
%:%10=1%:%
%:%11=1%:%
%:%12=2%:%
%:%13=3%:%
%:%14=4%:%
%:%19=4%:%
%:%22=5%:%
%:%23=6%:%
%:%24=6%:%
%:%25=7%:%
%:%26=7%:%
%:%27=8%:%
%:%28=8%:%
%:%29=9%:%
%:%30=10%:%
%:%31=10%:%
%:%32=11%:%
%:%33=11%:%
%:%34=12%:%
%:%35=13%:%
%:%36=13%:%
%:%37=14%:%
%:%38=15%:%
%:%39=15%:%
%:%46=16%:%
%:%47=16%:%
%:%48=17%:%
%:%49=17%:%
%:%50=18%:%
%:%51=18%:%
%:%52=19%:%
%:%53=19%:%
%:%54=20%:%
%:%55=20%:%
%:%56=21%:%
%:%57=21%:%
%:%58=21%:%
%:%59=22%:%
%:%60=22%:%
%:%61=23%:%
%:%62=23%:%
%:%63=24%:%
%:%69=24%:%
%:%72=25%:%
%:%73=26%:%
%:%74=26%:%
%:%75=27%:%
%:%76=28%:%
%:%77=29%:%
%:%84=30%:%
%:%85=30%:%
%:%86=31%:%
%:%87=31%:%
%:%88=32%:%
%:%89=32%:%
%:%90=33%:%
%:%91=33%:%
%:%92=34%:%
%:%93=34%:%
%:%94=35%:%
%:%95=35%:%
%:%96=36%:%
%:%97=36%:%
%:%98=37%:%
%:%99=37%:%
%:%100=38%:%
%:%101=38%:%
%:%102=39%:%
%:%103=39%:%
%:%104=40%:%
%:%105=40%:%
%:%106=41%:%
%:%107=41%:%
%:%108=41%:%
%:%109=42%:%
%:%110=42%:%
%:%111=43%:%
%:%112=43%:%
%:%113=44%:%
%:%119=44%:%
%:%122=45%:%
%:%123=46%:%
%:%124=46%:%
%:%125=47%:%
%:%126=48%:%
%:%127=49%:%
%:%130=50%:%
%:%134=50%:%
%:%135=50%:%
%:%136=51%:%
%:%137=51%:%
%:%138=52%:%
%:%139=52%:%
%:%140=53%:%
%:%141=53%:%
%:%142=54%:%
%:%143=54%:%
%:%144=55%:%
%:%145=55%:%
%:%150=55%:%
%:%153=56%:%
%:%154=57%:%
%:%155=57%:%
%:%156=58%:%
%:%157=59%:%
%:%158=60%:%
%:%161=61%:%
%:%165=61%:%
%:%166=61%:%
%:%167=62%:%
%:%168=62%:%
%:%169=63%:%
%:%170=63%:%
%:%171=64%:%
%:%172=64%:%
%:%173=65%:%
%:%174=65%:%
%:%175=66%:%
%:%176=66%:%
%:%181=66%:%
%:%184=67%:%
%:%185=68%:%
%:%186=68%:%
%:%187=69%:%
%:%188=70%:%
%:%189=71%:%
%:%196=72%:%
%:%197=72%:%
%:%198=73%:%
%:%199=73%:%
%:%200=74%:%
%:%201=74%:%
%:%202=75%:%
%:%203=75%:%
%:%204=76%:%
%:%205=76%:%
%:%206=77%:%
%:%207=78%:%
%:%208=78%:%
%:%209=79%:%
%:%210=79%:%
%:%211=80%:%
%:%212=80%:%
%:%213=81%:%
%:%214=81%:%
%:%215=82%:%
%:%216=83%:%
%:%217=83%:%
%:%218=84%:%
%:%219=85%:%
%:%220=85%:%
%:%221=85%:%
%:%222=85%:%
%:%223=86%:%
%:%224=86%:%
%:%225=86%:%
%:%226=86%:%
%:%227=87%:%
%:%228=88%:%
%:%229=88%:%
%:%230=89%:%
%:%231=89%:%
%:%232=90%:%
%:%233=90%:%
%:%234=91%:%
%:%235=91%:%
%:%236=91%:%
%:%237=92%:%
%:%238=92%:%
%:%239=93%:%
%:%240=93%:%
%:%241=94%:%
%:%242=94%:%
%:%243=95%:%
%:%244=95%:%
%:%245=95%:%
%:%246=96%:%
%:%247=96%:%
%:%248=97%:%
%:%249=97%:%
%:%250=98%:%
%:%251=98%:%
%:%252=99%:%
%:%258=99%:%
%:%261=100%:%
%:%262=101%:%
%:%263=101%:%
%:%264=102%:%
%:%265=103%:%
%:%266=104%:%
%:%269=105%:%
%:%273=105%:%
%:%274=105%:%
%:%275=106%:%
%:%276=106%:%
%:%277=107%:%
%:%278=107%:%
%:%283=107%:%
%:%286=108%:%
%:%287=109%:%
%:%288=109%:%
%:%289=110%:%
%:%290=111%:%
%:%291=112%:%
%:%294=113%:%
%:%298=113%:%
%:%299=113%:%
%:%300=114%:%
%:%301=114%:%
%:%302=115%:%
%:%303=115%:%
%:%308=115%:%
%:%311=116%:%
%:%312=117%:%
%:%313=117%:%
%:%314=118%:%
%:%315=119%:%
%:%316=120%:%
%:%319=121%:%
%:%323=121%:%
%:%324=121%:%
%:%325=122%:%
%:%326=122%:%
%:%327=123%:%
%:%328=123%:%
%:%333=123%:%
%:%336=124%:%
%:%337=125%:%
%:%338=125%:%
%:%339=126%:%
%:%340=127%:%
%:%341=128%:%
%:%344=129%:%
%:%348=129%:%
%:%349=129%:%
%:%350=130%:%
%:%351=130%:%
%:%352=131%:%
%:%353=131%:%
%:%358=131%:%
%:%361=132%:%
%:%362=133%:%
%:%363=133%:%
%:%364=134%:%
%:%365=135%:%
%:%366=136%:%
%:%369=137%:%
%:%373=137%:%
%:%374=137%:%
%:%375=138%:%
%:%376=138%:%
%:%377=139%:%
%:%378=139%:%
%:%383=139%:%
%:%386=140%:%
%:%387=141%:%
%:%388=141%:%
%:%389=142%:%
%:%390=143%:%
%:%391=144%:%
%:%398=145%:%
%:%399=145%:%
%:%400=146%:%
%:%401=146%:%
%:%402=146%:%
%:%403=146%:%
%:%404=146%:%
%:%405=147%:%
%:%406=147%:%
%:%407=147%:%
%:%408=148%:%
%:%409=148%:%
%:%410=149%:%
%:%411=149%:%
%:%412=150%:%
%:%413=150%:%
%:%414=151%:%
%:%415=151%:%
%:%416=152%:%
%:%417=152%:%
%:%418=153%:%
%:%419=153%:%
%:%420=154%:%
%:%421=154%:%
%:%422=155%:%
%:%423=155%:%
%:%424=156%:%
%:%425=156%:%
%:%426=157%:%
%:%427=157%:%
%:%428=158%:%
%:%429=158%:%
%:%430=158%:%
%:%431=159%:%
%:%432=159%:%
%:%433=160%:%
%:%434=160%:%
%:%435=161%:%
%:%436=161%:%
%:%437=162%:%
%:%443=162%:%
%:%446=163%:%
%:%447=164%:%
%:%448=164%:%
%:%449=165%:%
%:%450=166%:%
%:%451=167%:%
%:%454=168%:%
%:%458=168%:%
%:%459=168%:%
%:%460=169%:%
%:%461=169%:%
%:%466=169%:%
%:%469=170%:%
%:%470=171%:%
%:%471=171%:%
%:%472=172%:%
%:%473=173%:%
%:%474=174%:%
%:%477=175%:%
%:%481=175%:%
%:%482=175%:%
%:%483=176%:%
%:%484=176%:%
%:%485=177%:%
%:%486=177%:%
%:%491=177%:%
%:%494=178%:%
%:%495=179%:%
%:%496=179%:%
%:%497=180%:%
%:%498=181%:%
%:%499=182%:%
%:%506=183%:%
%:%507=183%:%
%:%508=184%:%
%:%509=184%:%
%:%510=185%:%
%:%511=185%:%
%:%512=186%:%
%:%513=186%:%
%:%514=187%:%
%:%515=187%:%
%:%516=188%:%
%:%517=188%:%
%:%518=189%:%
%:%519=189%:%
%:%520=190%:%
%:%521=190%:%
%:%522=191%:%
%:%523=191%:%
%:%524=192%:%
%:%525=192%:%
%:%526=193%:%
%:%527=193%:%
%:%528=194%:%
%:%529=194%:%
%:%530=195%:%
%:%531=195%:%
%:%532=196%:%
%:%533=196%:%
%:%534=197%:%
%:%535=198%:%
%:%536=198%:%
%:%537=199%:%
%:%538=200%:%
%:%539=200%:%
%:%540=201%:%
%:%541=201%:%
%:%542=202%:%
%:%543=202%:%
%:%544=203%:%
%:%545=203%:%
%:%546=204%:%
%:%547=205%:%
%:%548=205%:%
%:%549=205%:%
%:%550=205%:%
%:%551=206%:%
%:%552=206%:%
%:%553=206%:%
%:%554=206%:%
%:%555=207%:%
%:%556=208%:%
%:%557=208%:%
%:%558=209%:%
%:%559=209%:%
%:%560=210%:%
%:%561=210%:%
%:%562=211%:%
%:%563=211%:%
%:%564=212%:%
%:%565=212%:%
%:%566=213%:%
%:%567=213%:%
%:%568=214%:%
%:%569=214%:%
%:%570=215%:%
%:%571=215%:%
%:%572=216%:%
%:%573=216%:%
%:%574=217%:%
%:%575=217%:%
%:%576=218%:%
%:%577=218%:%
%:%578=218%:%
%:%579=219%:%
%:%580=219%:%
%:%581=219%:%
%:%582=220%:%
%:%583=221%:%
%:%584=221%:%
%:%585=222%:%
%:%586=222%:%
%:%587=223%:%
%:%588=223%:%
%:%589=224%:%
%:%590=224%:%
%:%591=225%:%
%:%592=225%:%
%:%593=226%:%
%:%594=226%:%
%:%595=227%:%
%:%596=227%:%
%:%597=227%:%
%:%598=228%:%
%:%599=228%:%
%:%600=228%:%
%:%601=229%:%
%:%602=230%:%
%:%603=230%:%
%:%604=231%:%
%:%605=231%:%
%:%606=232%:%
%:%607=232%:%
%:%608=233%:%
%:%609=233%:%
%:%610=234%:%
%:%611=235%:%
%:%612=235%:%
%:%613=235%:%
%:%614=236%:%
%:%615=236%:%
%:%616=237%:%
%:%617=237%:%
%:%618=238%:%
%:%619=238%:%
%:%620=239%:%
%:%621=240%:%
%:%622=240%:%
%:%623=241%:%
%:%624=241%:%
%:%625=242%:%
%:%626=242%:%
%:%627=243%:%
%:%628=243%:%
%:%629=244%:%
%:%630=245%:%
%:%631=245%:%
%:%632=246%:%
%:%633=246%:%
%:%634=247%:%
%:%635=247%:%
%:%636=248%:%
%:%637=248%:%
%:%638=249%:%
%:%639=249%:%
%:%640=249%:%
%:%641=250%:%
%:%642=250%:%
%:%643=251%:%
%:%644=251%:%
%:%645=252%:%
%:%646=253%:%
%:%647=253%:%
%:%648=254%:%
%:%649=254%:%
%:%650=255%:%
%:%651=255%:%
%:%652=256%:%
%:%653=256%:%
%:%654=257%:%
%:%655=258%:%
%:%656=258%:%
%:%657=259%:%
%:%658=259%:%
%:%659=260%:%
%:%660=260%:%
%:%661=261%:%
%:%662=261%:%
%:%663=262%:%
%:%664=262%:%
%:%665=263%:%
%:%666=263%:%
%:%667=264%:%
%:%668=264%:%
%:%669=265%:%
%:%670=265%:%
%:%671=266%:%
%:%672=266%:%
%:%673=267%:%
%:%674=267%:%
%:%675=268%:%
%:%676=268%:%
%:%677=269%:%
%:%678=269%:%
%:%679=270%:%
%:%680=270%:%
%:%681=270%:%
%:%682=271%:%
%:%683=271%:%
%:%684=272%:%
%:%685=273%:%
%:%686=273%:%
%:%687=274%:%
%:%688=274%:%
%:%689=275%:%
%:%690=275%:%
%:%691=276%:%
%:%692=276%:%
%:%693=277%:%
%:%694=277%:%
%:%695=278%:%
%:%696=278%:%
%:%697=279%:%
%:%698=280%:%
%:%699=280%:%
%:%700=281%:%
%:%701=281%:%
%:%702=282%:%
%:%703=282%:%
%:%704=283%:%
%:%705=283%:%
%:%706=284%:%
%:%707=284%:%
%:%708=285%:%
%:%709=285%:%
%:%710=286%:%
%:%711=286%:%
%:%712=287%:%
%:%713=287%:%
%:%714=288%:%
%:%715=288%:%
%:%716=289%:%
%:%717=289%:%
%:%718=290%:%
%:%719=290%:%
%:%720=291%:%
%:%721=291%:%
%:%722=292%:%
%:%723=292%:%
%:%724=293%:%
%:%725=293%:%
%:%726=294%:%
%:%727=294%:%
%:%728=295%:%
%:%729=295%:%
%:%730=296%:%
%:%731=296%:%
%:%732=297%:%
%:%733=297%:%
%:%734=298%:%
%:%735=298%:%
%:%736=299%:%
%:%737=299%:%
%:%738=300%:%
%:%739=300%:%
%:%740=301%:%
%:%741=301%:%
%:%742=301%:%
%:%743=302%:%
%:%744=302%:%
%:%745=303%:%
%:%746=303%:%
%:%747=304%:%
%:%748=304%:%
%:%749=305%:%
%:%750=305%:%
%:%751=306%:%
%:%752=307%:%
%:%753=307%:%
%:%754=308%:%
%:%755=308%:%
%:%756=309%:%
%:%757=309%:%
%:%758=310%:%
%:%759=310%:%
%:%760=311%:%
%:%761=311%:%
%:%762=312%:%
%:%763=312%:%
%:%764=313%:%
%:%765=313%:%
%:%766=314%:%
%:%767=314%:%
%:%768=315%:%
%:%769=315%:%
%:%770=316%:%
%:%771=316%:%
%:%772=317%:%
%:%773=317%:%
%:%774=318%:%
%:%775=318%:%
%:%776=319%:%
%:%777=319%:%
%:%778=320%:%
%:%779=320%:%
%:%780=321%:%
%:%781=322%:%
%:%782=322%:%
%:%783=322%:%
%:%784=323%:%
%:%785=323%:%
%:%786=324%:%
%:%787=324%:%
%:%788=325%:%
%:%789=325%:%
%:%790=326%:%
%:%796=326%:%
%:%799=327%:%
%:%800=328%:%
%:%801=328%:%
%:%802=329%:%
%:%803=330%:%
%:%804=331%:%
%:%811=332%:%
%:%812=332%:%
%:%813=333%:%
%:%814=334%:%
%:%815=334%:%
%:%816=335%:%
%:%817=335%:%
%:%818=336%:%
%:%819=336%:%
%:%820=336%:%
%:%821=337%:%
%:%822=337%:%
%:%823=337%:%
%:%824=337%:%
%:%825=337%:%
%:%826=338%:%
%:%827=338%:%
%:%828=338%:%
%:%829=338%:%
%:%830=338%:%
%:%831=339%:%
%:%832=339%:%
%:%833=339%:%
%:%834=339%:%
%:%835=339%:%
%:%836=340%:%
%:%837=340%:%
%:%838=340%:%
%:%839=340%:%
%:%840=340%:%
%:%841=341%:%
%:%842=341%:%
%:%843=342%:%
%:%844=342%:%
%:%845=342%:%
%:%846=343%:%
%:%847=343%:%
%:%848=343%:%
%:%849=343%:%
%:%850=343%:%
%:%851=344%:%
%:%852=344%:%
%:%853=344%:%
%:%854=344%:%
%:%855=344%:%
%:%856=345%:%
%:%857=345%:%
%:%858=345%:%
%:%859=345%:%
%:%860=346%:%
%:%861=346%:%
%:%862=347%:%
%:%863=348%:%
%:%864=348%:%
%:%865=349%:%
%:%866=349%:%
%:%867=350%:%
%:%868=350%:%
%:%869=351%:%
%:%870=351%:%
%:%871=352%:%
%:%872=352%:%
%:%873=353%:%
%:%874=353%:%
%:%875=354%:%
%:%876=354%:%
%:%877=355%:%
%:%878=355%:%
%:%879=356%:%
%:%880=356%:%
%:%881=357%:%
%:%882=357%:%
%:%883=358%:%
%:%884=358%:%
%:%885=359%:%
%:%886=359%:%
%:%887=360%:%
%:%888=360%:%
%:%889=361%:%
%:%890=361%:%
%:%891=362%:%
%:%892=362%:%
%:%893=363%:%
%:%894=363%:%
%:%895=364%:%
%:%896=364%:%
%:%897=365%:%
%:%898=365%:%
%:%899=366%:%
%:%900=367%:%
%:%901=367%:%
%:%902=368%:%
%:%903=368%:%
%:%904=369%:%
%:%905=369%:%
%:%906=370%:%
%:%907=370%:%
%:%908=371%:%
%:%909=371%:%
%:%910=372%:%
%:%911=372%:%
%:%912=373%:%
%:%913=373%:%
%:%914=374%:%
%:%915=375%:%
%:%916=375%:%
%:%917=376%:%
%:%918=376%:%
%:%919=377%:%
%:%920=377%:%
%:%921=378%:%
%:%922=378%:%
%:%923=379%:%
%:%924=380%:%
%:%925=380%:%
%:%926=381%:%
%:%927=382%:%
%:%928=382%:%
%:%929=383%:%
%:%930=383%:%
%:%931=384%:%
%:%932=385%:%
%:%933=385%:%
%:%934=386%:%
%:%935=386%:%
%:%936=387%:%
%:%937=387%:%
%:%938=388%:%
%:%939=388%:%
%:%940=389%:%
%:%941=389%:%
%:%942=390%:%
%:%943=390%:%
%:%944=391%:%
%:%945=391%:%
%:%946=392%:%
%:%947=392%:%
%:%948=393%:%
%:%949=393%:%
%:%950=394%:%
%:%951=394%:%
%:%952=395%:%
%:%953=395%:%
%:%954=396%:%
%:%955=396%:%
%:%956=397%:%
%:%957=397%:%
%:%958=398%:%
%:%959=398%:%
%:%960=399%:%
%:%961=399%:%
%:%962=400%:%
%:%963=400%:%
%:%964=401%:%
%:%965=401%:%
%:%966=402%:%
%:%967=402%:%
%:%968=403%:%
%:%969=403%:%
%:%970=404%:%
%:%971=404%:%
%:%972=405%:%
%:%973=405%:%
%:%974=406%:%
%:%975=406%:%
%:%976=407%:%
%:%977=407%:%
%:%978=408%:%
%:%979=408%:%
%:%980=409%:%
%:%981=409%:%
%:%982=410%:%
%:%983=410%:%
%:%984=411%:%
%:%985=411%:%
%:%986=412%:%
%:%987=412%:%
%:%988=413%:%
%:%989=413%:%
%:%990=414%:%
%:%991=414%:%
%:%992=415%:%
%:%993=415%:%
%:%994=416%:%
%:%995=416%:%
%:%996=417%:%
%:%997=417%:%
%:%998=418%:%
%:%999=418%:%
%:%1000=419%:%
%:%1001=419%:%
%:%1002=420%:%
%:%1003=420%:%
%:%1004=421%:%
%:%1005=421%:%
%:%1006=422%:%
%:%1007=422%:%
%:%1008=423%:%
%:%1009=423%:%
%:%1010=424%:%
%:%1011=424%:%
%:%1012=425%:%
%:%1013=425%:%
%:%1014=426%:%
%:%1015=426%:%
%:%1016=427%:%
%:%1017=427%:%
%:%1018=428%:%
%:%1019=428%:%
%:%1020=429%:%
%:%1021=429%:%
%:%1022=430%:%
%:%1023=430%:%
%:%1024=431%:%
%:%1025=431%:%
%:%1026=432%:%
%:%1027=432%:%
%:%1028=433%:%
%:%1029=433%:%
%:%1030=434%:%
%:%1031=434%:%
%:%1032=435%:%
%:%1033=435%:%
%:%1034=436%:%
%:%1035=436%:%
%:%1036=437%:%
%:%1037=437%:%
%:%1038=438%:%
%:%1039=438%:%
%:%1040=439%:%
%:%1041=439%:%
%:%1042=440%:%
%:%1043=440%:%
%:%1044=441%:%
%:%1045=441%:%
%:%1046=442%:%
%:%1047=442%:%
%:%1048=443%:%
%:%1049=443%:%
%:%1050=444%:%
%:%1051=444%:%
%:%1052=445%:%
%:%1053=445%:%
%:%1054=446%:%
%:%1055=446%:%
%:%1056=447%:%
%:%1057=447%:%
%:%1058=448%:%
%:%1059=448%:%
%:%1060=449%:%
%:%1061=449%:%
%:%1062=450%:%
%:%1063=450%:%
%:%1064=451%:%
%:%1065=451%:%
%:%1066=452%:%
%:%1067=452%:%
%:%1068=453%:%
%:%1069=453%:%
%:%1070=454%:%
%:%1071=454%:%
%:%1072=455%:%
%:%1073=455%:%
%:%1074=456%:%
%:%1075=456%:%
%:%1076=457%:%
%:%1077=457%:%
%:%1078=458%:%
%:%1079=458%:%
%:%1080=459%:%
%:%1081=459%:%
%:%1082=460%:%
%:%1083=460%:%
%:%1084=461%:%
%:%1085=461%:%
%:%1086=462%:%
%:%1087=462%:%
%:%1088=463%:%
%:%1089=463%:%
%:%1090=464%:%
%:%1091=464%:%
%:%1092=465%:%
%:%1093=465%:%
%:%1094=466%:%
%:%1095=466%:%
%:%1096=467%:%
%:%1097=467%:%
%:%1098=468%:%
%:%1099=468%:%
%:%1100=469%:%
%:%1101=469%:%
%:%1102=470%:%
%:%1103=470%:%
%:%1104=471%:%
%:%1105=471%:%
%:%1106=472%:%
%:%1107=472%:%
%:%1108=473%:%
%:%1109=473%:%
%:%1110=474%:%
%:%1111=474%:%
%:%1112=475%:%
%:%1113=475%:%
%:%1114=476%:%
%:%1115=476%:%
%:%1116=477%:%
%:%1117=477%:%
%:%1118=478%:%
%:%1119=478%:%
%:%1120=479%:%
%:%1121=479%:%
%:%1122=480%:%
%:%1123=480%:%
%:%1124=481%:%
%:%1125=481%:%
%:%1126=482%:%
%:%1127=482%:%
%:%1128=482%:%
%:%1129=483%:%
%:%1130=483%:%
%:%1131=484%:%
%:%1132=485%:%
%:%1133=485%:%
%:%1134=486%:%
%:%1135=486%:%
%:%1136=487%:%
%:%1137=487%:%
%:%1138=488%:%
%:%1139=488%:%
%:%1140=489%:%
%:%1141=489%:%
%:%1142=489%:%
%:%1143=489%:%
%:%1144=490%:%
%:%1145=491%:%
%:%1146=491%:%
%:%1147=492%:%
%:%1148=492%:%
%:%1149=493%:%
%:%1150=493%:%
%:%1151=494%:%
%:%1152=494%:%
%:%1153=495%:%
%:%1154=495%:%
%:%1155=496%:%
%:%1156=496%:%
%:%1157=497%:%
%:%1158=497%:%
%:%1159=498%:%
%:%1160=498%:%
%:%1161=499%:%
%:%1167=499%:%
%:%1170=500%:%
%:%1171=501%:%
%:%1172=501%:%
%:%1173=502%:%
%:%1174=503%:%
%:%1175=504%:%
%:%1178=505%:%
%:%1182=505%:%
%:%1183=505%:%
%:%1184=506%:%
%:%1185=506%:%
%:%1186=507%:%
%:%1187=507%:%
%:%1192=507%:%
%:%1195=508%:%
%:%1196=509%:%
%:%1197=509%:%
%:%1198=510%:%
%:%1199=511%:%
%:%1200=512%:%
%:%1203=513%:%
%:%1207=513%:%
%:%1208=513%:%
%:%1209=514%:%
%:%1210=514%:%
%:%1211=515%:%
%:%1212=515%:%
%:%1217=515%:%
%:%1220=516%:%
%:%1221=517%:%
%:%1222=517%:%
%:%1223=518%:%
%:%1224=519%:%
%:%1225=520%:%
%:%1228=521%:%
%:%1229=522%:%
%:%1233=522%:%
%:%1234=522%:%
%:%1235=523%:%
%:%1236=523%:%
%:%1237=524%:%
%:%1238=524%:%
%:%1239=525%:%
%:%1240=525%:%
%:%1241=526%:%
%:%1242=526%:%
%:%1243=527%:%
%:%1244=527%:%
%:%1245=528%:%
%:%1246=528%:%
%:%1247=529%:%
%:%1248=529%:%
%:%1249=530%:%
%:%1250=530%:%
%:%1251=531%:%
%:%1252=531%:%
%:%1253=532%:%
%:%1259=532%:%
%:%1264=533%:%
%:%1269=534%:%

%
\begin{isabellebody}%
\setisabellecontext{HS{\isacharunderscore}{\kern0pt}Forces}%
%
\isadelimtheory
%
\endisadelimtheory
%
\isatagtheory
\isacommand{theory}\isamarkupfalse%
\ HS{\isacharunderscore}{\kern0pt}Forces\ \isanewline
\ \ \isakeyword{imports}\ SymExt{\isacharunderscore}{\kern0pt}Definition\ Delta{\isadigit{0}}\isanewline
\isakeyword{begin}%
\endisatagtheory
{\isafoldtheory}%
%
\isadelimtheory
\isanewline
%
\endisadelimtheory
\ \ \isanewline
\isacommand{definition}\isamarkupfalse%
\isanewline
\ \ ren{\isacharunderscore}{\kern0pt}forcesHS{\isacharunderscore}{\kern0pt}forall\ {\isacharcolon}{\kern0pt}{\isacharcolon}{\kern0pt}\ {\isachardoublequoteopen}i{\isasymRightarrow}i{\isachardoublequoteclose}\ \isakeyword{where}\isanewline
\ \ {\isachardoublequoteopen}ren{\isacharunderscore}{\kern0pt}forcesHS{\isacharunderscore}{\kern0pt}forall{\isacharparenleft}{\kern0pt}n{\isacharparenright}{\kern0pt}\ {\isasymequiv}\ {\isacharbraceleft}{\kern0pt}\ {\isacharless}{\kern0pt}{\isadigit{0}}{\isacharcomma}{\kern0pt}\ {\isadigit{1}}{\isachargreater}{\kern0pt}{\isacharcomma}{\kern0pt}\ {\isacharless}{\kern0pt}{\isadigit{1}}{\isacharcomma}{\kern0pt}\ {\isadigit{2}}{\isachargreater}{\kern0pt}{\isacharcomma}{\kern0pt}\ {\isacharless}{\kern0pt}{\isadigit{2}}{\isacharcomma}{\kern0pt}\ {\isadigit{3}}{\isachargreater}{\kern0pt}{\isacharcomma}{\kern0pt}\ {\isacharless}{\kern0pt}{\isadigit{3}}{\isacharcomma}{\kern0pt}\ {\isadigit{4}}{\isachargreater}{\kern0pt}{\isacharcomma}{\kern0pt}\ {\isacharless}{\kern0pt}{\isadigit{4}}{\isacharcomma}{\kern0pt}\ {\isadigit{5}}{\isachargreater}{\kern0pt}{\isacharcomma}{\kern0pt}\ {\isacharless}{\kern0pt}{\isadigit{5}}{\isacharcomma}{\kern0pt}\ {\isadigit{0}}{\isachargreater}{\kern0pt}\ {\isacharbraceright}{\kern0pt}\ {\isasymunion}\ {\isacharbraceleft}{\kern0pt}\ {\isacharless}{\kern0pt}k{\isacharcomma}{\kern0pt}\ k{\isachargreater}{\kern0pt}{\isachardot}{\kern0pt}{\isachardot}{\kern0pt}\ k\ {\isasymin}\ n{\isacharcomma}{\kern0pt}\ {\isadigit{6}}\ {\isasymle}\ k\ {\isacharbraceright}{\kern0pt}{\isachardoublequoteclose}\ \isanewline
\isanewline
\isacommand{consts}\isamarkupfalse%
\ forcesHS{\isacharprime}{\kern0pt}\ {\isacharcolon}{\kern0pt}{\isacharcolon}{\kern0pt}\ {\isachardoublequoteopen}i{\isasymRightarrow}i{\isachardoublequoteclose}\isanewline
\isacommand{primrec}\isamarkupfalse%
\isanewline
\ \ {\isachardoublequoteopen}forcesHS{\isacharprime}{\kern0pt}{\isacharparenleft}{\kern0pt}Member{\isacharparenleft}{\kern0pt}x{\isacharcomma}{\kern0pt}y{\isacharparenright}{\kern0pt}{\isacharparenright}{\kern0pt}\ {\isacharequal}{\kern0pt}\ incr{\isacharunderscore}{\kern0pt}bv{\isacharparenleft}{\kern0pt}forces{\isacharprime}{\kern0pt}{\isacharparenleft}{\kern0pt}Member{\isacharparenleft}{\kern0pt}x{\isacharcomma}{\kern0pt}\ y{\isacharparenright}{\kern0pt}{\isacharparenright}{\kern0pt}{\isacharparenright}{\kern0pt}{\isacharbackquote}{\kern0pt}{\isadigit{4}}{\isachardoublequoteclose}\isanewline
\ \ {\isachardoublequoteopen}forcesHS{\isacharprime}{\kern0pt}{\isacharparenleft}{\kern0pt}Equal{\isacharparenleft}{\kern0pt}x{\isacharcomma}{\kern0pt}y{\isacharparenright}{\kern0pt}{\isacharparenright}{\kern0pt}\ \ {\isacharequal}{\kern0pt}\ incr{\isacharunderscore}{\kern0pt}bv{\isacharparenleft}{\kern0pt}forces{\isacharprime}{\kern0pt}{\isacharparenleft}{\kern0pt}Equal{\isacharparenleft}{\kern0pt}x{\isacharcomma}{\kern0pt}\ y{\isacharparenright}{\kern0pt}{\isacharparenright}{\kern0pt}{\isacharparenright}{\kern0pt}{\isacharbackquote}{\kern0pt}{\isadigit{4}}{\isachardoublequoteclose}\isanewline
\ \ {\isachardoublequoteopen}forcesHS{\isacharprime}{\kern0pt}{\isacharparenleft}{\kern0pt}Nand{\isacharparenleft}{\kern0pt}p{\isacharcomma}{\kern0pt}q{\isacharparenright}{\kern0pt}{\isacharparenright}{\kern0pt}\ \ \ {\isacharequal}{\kern0pt}\isanewline
\ \ \ \ \ \ \ \ Neg{\isacharparenleft}{\kern0pt}Exists{\isacharparenleft}{\kern0pt}And{\isacharparenleft}{\kern0pt}Member{\isacharparenleft}{\kern0pt}{\isadigit{0}}{\isacharcomma}{\kern0pt}{\isadigit{2}}{\isacharparenright}{\kern0pt}{\isacharcomma}{\kern0pt}And{\isacharparenleft}{\kern0pt}leq{\isacharunderscore}{\kern0pt}fm{\isacharparenleft}{\kern0pt}{\isadigit{3}}{\isacharcomma}{\kern0pt}{\isadigit{0}}{\isacharcomma}{\kern0pt}{\isadigit{1}}{\isacharparenright}{\kern0pt}{\isacharcomma}{\kern0pt}And{\isacharparenleft}{\kern0pt}ren{\isacharunderscore}{\kern0pt}forces{\isacharunderscore}{\kern0pt}nand{\isacharparenleft}{\kern0pt}forcesHS{\isacharprime}{\kern0pt}{\isacharparenleft}{\kern0pt}p{\isacharparenright}{\kern0pt}{\isacharparenright}{\kern0pt}{\isacharcomma}{\kern0pt}\isanewline
\ \ \ \ \ \ \ \ \ \ \ \ \ \ \ \ \ \ \ \ \ \ \ \ \ \ \ \ \ \ \ \ \ \ \ \ \ \ \ \ \ ren{\isacharunderscore}{\kern0pt}forces{\isacharunderscore}{\kern0pt}nand{\isacharparenleft}{\kern0pt}forcesHS{\isacharprime}{\kern0pt}{\isacharparenleft}{\kern0pt}q{\isacharparenright}{\kern0pt}{\isacharparenright}{\kern0pt}{\isacharparenright}{\kern0pt}{\isacharparenright}{\kern0pt}{\isacharparenright}{\kern0pt}{\isacharparenright}{\kern0pt}{\isacharparenright}{\kern0pt}{\isachardoublequoteclose}\isanewline
\ \ {\isachardoublequoteopen}forcesHS{\isacharprime}{\kern0pt}{\isacharparenleft}{\kern0pt}Forall{\isacharparenleft}{\kern0pt}p{\isacharparenright}{\kern0pt}{\isacharparenright}{\kern0pt}\ \ \ {\isacharequal}{\kern0pt}\ Forall{\isacharparenleft}{\kern0pt}Implies{\isacharparenleft}{\kern0pt}is{\isacharunderscore}{\kern0pt}HS{\isacharunderscore}{\kern0pt}fm{\isacharparenleft}{\kern0pt}{\isadigit{5}}{\isacharcomma}{\kern0pt}\ {\isadigit{0}}{\isacharparenright}{\kern0pt}{\isacharcomma}{\kern0pt}\ ren{\isacharparenleft}{\kern0pt}forcesHS{\isacharprime}{\kern0pt}{\isacharparenleft}{\kern0pt}p{\isacharparenright}{\kern0pt}{\isacharparenright}{\kern0pt}{\isacharbackquote}{\kern0pt}{\isacharparenleft}{\kern0pt}{\isadigit{6}}{\isasymunion}arity{\isacharparenleft}{\kern0pt}forcesHS{\isacharprime}{\kern0pt}{\isacharparenleft}{\kern0pt}p{\isacharparenright}{\kern0pt}{\isacharparenright}{\kern0pt}{\isacharparenright}{\kern0pt}{\isacharbackquote}{\kern0pt}{\isacharparenleft}{\kern0pt}{\isadigit{6}}{\isasymunion}arity{\isacharparenleft}{\kern0pt}forcesHS{\isacharprime}{\kern0pt}{\isacharparenleft}{\kern0pt}p{\isacharparenright}{\kern0pt}{\isacharparenright}{\kern0pt}{\isacharparenright}{\kern0pt}{\isacharbackquote}{\kern0pt}ren{\isacharunderscore}{\kern0pt}forcesHS{\isacharunderscore}{\kern0pt}forall{\isacharparenleft}{\kern0pt}arity{\isacharparenleft}{\kern0pt}forcesHS{\isacharprime}{\kern0pt}{\isacharparenleft}{\kern0pt}p{\isacharparenright}{\kern0pt}{\isacharparenright}{\kern0pt}{\isacharparenright}{\kern0pt}{\isacharparenright}{\kern0pt}{\isacharparenright}{\kern0pt}{\isachardoublequoteclose}\isanewline
\isanewline
\isacommand{definition}\isamarkupfalse%
\isanewline
\ \ forcesHS\ {\isacharcolon}{\kern0pt}{\isacharcolon}{\kern0pt}\ {\isachardoublequoteopen}i{\isasymRightarrow}i{\isachardoublequoteclose}\ \isakeyword{where}\isanewline
\ \ {\isachardoublequoteopen}forcesHS{\isacharparenleft}{\kern0pt}{\isasymphi}{\isacharparenright}{\kern0pt}\ {\isasymequiv}\ And{\isacharparenleft}{\kern0pt}Member{\isacharparenleft}{\kern0pt}{\isadigit{0}}{\isacharcomma}{\kern0pt}{\isadigit{1}}{\isacharparenright}{\kern0pt}{\isacharcomma}{\kern0pt}forcesHS{\isacharprime}{\kern0pt}{\isacharparenleft}{\kern0pt}{\isasymphi}{\isacharparenright}{\kern0pt}{\isacharparenright}{\kern0pt}{\isachardoublequoteclose}\isanewline
\isanewline
\isacommand{context}\isamarkupfalse%
\ M{\isacharunderscore}{\kern0pt}symmetric{\isacharunderscore}{\kern0pt}system\ \isakeyword{begin}\ \isanewline
\isanewline
\isacommand{abbreviation}\isamarkupfalse%
\ ForcesHS\ {\isacharcolon}{\kern0pt}{\isacharcolon}{\kern0pt}\ {\isachardoublequoteopen}{\isacharbrackleft}{\kern0pt}i{\isacharcomma}{\kern0pt}\ i{\isacharcomma}{\kern0pt}\ i{\isacharbrackright}{\kern0pt}\ {\isasymRightarrow}\ o{\isachardoublequoteclose}\ \ {\isacharparenleft}{\kern0pt}{\isachardoublequoteopen}{\isacharunderscore}{\kern0pt}\ {\isasymtturnstile}HS\ {\isacharunderscore}{\kern0pt}\ {\isacharunderscore}{\kern0pt}{\isachardoublequoteclose}\ {\isacharbrackleft}{\kern0pt}{\isadigit{3}}{\isadigit{6}}{\isacharcomma}{\kern0pt}{\isadigit{3}}{\isadigit{6}}{\isacharcomma}{\kern0pt}{\isadigit{3}}{\isadigit{6}}{\isacharbrackright}{\kern0pt}\ {\isadigit{6}}{\isadigit{0}}{\isacharparenright}{\kern0pt}\ \isakeyword{where}\isanewline
\ \ {\isachardoublequoteopen}p\ {\isasymtturnstile}HS\ {\isasymphi}\ env\ \ \ {\isasymequiv}\ \ \ M{\isacharcomma}{\kern0pt}\ {\isacharparenleft}{\kern0pt}{\isacharbrackleft}{\kern0pt}p{\isacharcomma}{\kern0pt}P{\isacharcomma}{\kern0pt}leq{\isacharcomma}{\kern0pt}one{\isacharcomma}{\kern0pt}\ {\isasymlangle}{\isasymF}{\isacharcomma}{\kern0pt}\ {\isasymG}{\isacharcomma}{\kern0pt}\ P{\isacharcomma}{\kern0pt}\ P{\isacharunderscore}{\kern0pt}auto{\isasymrangle}{\isacharbrackright}{\kern0pt}\ {\isacharat}{\kern0pt}\ env{\isacharparenright}{\kern0pt}\ {\isasymTurnstile}\ forcesHS{\isacharparenleft}{\kern0pt}{\isasymphi}{\isacharparenright}{\kern0pt}{\isachardoublequoteclose}\isanewline
\isanewline
\isacommand{lemma}\isamarkupfalse%
\ ren{\isacharunderscore}{\kern0pt}forcesHS{\isacharunderscore}{\kern0pt}forall{\isacharunderscore}{\kern0pt}type\ {\isacharcolon}{\kern0pt}\ \isanewline
\ \ \isakeyword{fixes}\ n\ \isanewline
\ \ \isakeyword{assumes}\ {\isachardoublequoteopen}n\ {\isasymin}\ nat{\isachardoublequoteclose}\ \isanewline
\ \ \isakeyword{shows}\ {\isachardoublequoteopen}ren{\isacharunderscore}{\kern0pt}forcesHS{\isacharunderscore}{\kern0pt}forall{\isacharparenleft}{\kern0pt}n{\isacharparenright}{\kern0pt}\ {\isasymin}\ {\isacharparenleft}{\kern0pt}{\isadigit{6}}{\isasymunion}n{\isacharparenright}{\kern0pt}\ {\isasymrightarrow}\ {\isacharparenleft}{\kern0pt}{\isadigit{6}}{\isasymunion}n{\isacharparenright}{\kern0pt}{\isachardoublequoteclose}\ \isanewline
%
\isadelimproof
\isanewline
\ \ %
\endisadelimproof
%
\isatagproof
\isacommand{apply}\isamarkupfalse%
{\isacharparenleft}{\kern0pt}rule\ Pi{\isacharunderscore}{\kern0pt}memberI{\isacharparenright}{\kern0pt}\isanewline
\ \ \isacommand{unfolding}\isamarkupfalse%
\ relation{\isacharunderscore}{\kern0pt}def\ ren{\isacharunderscore}{\kern0pt}forcesHS{\isacharunderscore}{\kern0pt}forall{\isacharunderscore}{\kern0pt}def\ function{\isacharunderscore}{\kern0pt}def\ \isanewline
\ \ \ \ \ \isacommand{apply}\isamarkupfalse%
\ auto{\isacharbrackleft}{\kern0pt}{\isadigit{2}}{\isacharbrackright}{\kern0pt}\isanewline
\ \ \ \isacommand{apply}\isamarkupfalse%
{\isacharparenleft}{\kern0pt}rule\ equality{\isacharunderscore}{\kern0pt}iffI{\isacharcomma}{\kern0pt}\ rule\ iffI{\isacharcomma}{\kern0pt}\ simp{\isacharparenright}{\kern0pt}\isanewline
\ \ \ \ \isacommand{apply}\isamarkupfalse%
{\isacharparenleft}{\kern0pt}rename{\isacharunderscore}{\kern0pt}tac\ k{\isacharcomma}{\kern0pt}\ case{\isacharunderscore}{\kern0pt}tac\ {\isachardoublequoteopen}{\isadigit{6}}\ {\isasymle}\ k{\isachardoublequoteclose}{\isacharparenright}{\kern0pt}\isanewline
\ \ \ \ \ \isacommand{apply}\isamarkupfalse%
{\isacharparenleft}{\kern0pt}rule\ disjI{\isadigit{2}}{\isacharparenright}{\kern0pt}{\isacharplus}{\kern0pt}\isanewline
\ \ \ \ \ \isacommand{apply}\isamarkupfalse%
{\isacharparenleft}{\kern0pt}rename{\isacharunderscore}{\kern0pt}tac\ k{\isacharcomma}{\kern0pt}\ subgoal{\isacharunderscore}{\kern0pt}tac\ {\isachardoublequoteopen}k\ {\isasymnotin}\ {\isadigit{6}}{\isachardoublequoteclose}{\isacharparenright}{\kern0pt}\isanewline
\ \ \ \ \ \ \isacommand{apply}\isamarkupfalse%
{\isacharparenleft}{\kern0pt}force{\isacharparenright}{\kern0pt}\isanewline
\ \ \ \ \ \isacommand{apply}\isamarkupfalse%
{\isacharparenleft}{\kern0pt}rule\ ccontr{\isacharcomma}{\kern0pt}\ simp{\isacharparenright}{\kern0pt}\isanewline
\ \ \ \ \ \isacommand{apply}\isamarkupfalse%
{\isacharparenleft}{\kern0pt}rename{\isacharunderscore}{\kern0pt}tac\ k{\isacharcomma}{\kern0pt}\ subgoal{\isacharunderscore}{\kern0pt}tac\ {\isachardoublequoteopen}k\ {\isacharless}{\kern0pt}\ {\isadigit{6}}{\isachardoublequoteclose}{\isacharcomma}{\kern0pt}\ force{\isacharparenright}{\kern0pt}\isanewline
\ \ \isacommand{using}\isamarkupfalse%
\ not{\isacharunderscore}{\kern0pt}le{\isacharunderscore}{\kern0pt}iff{\isacharunderscore}{\kern0pt}lt\ \isanewline
\ \ \ \ \ \isacommand{apply}\isamarkupfalse%
\ force\isanewline
\ \ \ \ \isacommand{apply}\isamarkupfalse%
{\isacharparenleft}{\kern0pt}rename{\isacharunderscore}{\kern0pt}tac\ k{\isacharcomma}{\kern0pt}\ subgoal{\isacharunderscore}{\kern0pt}tac\ {\isachardoublequoteopen}k\ {\isacharless}{\kern0pt}\ {\isadigit{6}}{\isachardoublequoteclose}{\isacharparenright}{\kern0pt}\isanewline
\ \ \isacommand{using}\isamarkupfalse%
\ le{\isacharunderscore}{\kern0pt}iff\ \isanewline
\ \ \ \ \ \isacommand{apply}\isamarkupfalse%
\ force\ \isanewline
\ \ \ \ \isacommand{apply}\isamarkupfalse%
{\isacharparenleft}{\kern0pt}rename{\isacharunderscore}{\kern0pt}tac\ k{\isacharcomma}{\kern0pt}\ subgoal{\isacharunderscore}{\kern0pt}tac\ {\isachardoublequoteopen}k\ {\isasymin}\ nat{\isachardoublequoteclose}{\isacharparenright}{\kern0pt}\isanewline
\ \ \isacommand{using}\isamarkupfalse%
\ not{\isacharunderscore}{\kern0pt}le{\isacharunderscore}{\kern0pt}iff{\isacharunderscore}{\kern0pt}lt\ lt{\isacharunderscore}{\kern0pt}nat{\isacharunderscore}{\kern0pt}in{\isacharunderscore}{\kern0pt}nat\ ltI\ \isanewline
\ \ \ \ \ \isacommand{apply}\isamarkupfalse%
\ force\ \isanewline
\ \ \ \ \isacommand{apply}\isamarkupfalse%
{\isacharparenleft}{\kern0pt}rule\ disjE{\isacharcomma}{\kern0pt}\ assumption{\isacharparenright}{\kern0pt}\isanewline
\ \ \isacommand{using}\isamarkupfalse%
\ not{\isacharunderscore}{\kern0pt}le{\isacharunderscore}{\kern0pt}iff{\isacharunderscore}{\kern0pt}lt\ lt{\isacharunderscore}{\kern0pt}nat{\isacharunderscore}{\kern0pt}in{\isacharunderscore}{\kern0pt}nat\ ltI\ assms\isanewline
\ \ \ \ \ \isacommand{apply}\isamarkupfalse%
\ force\ \isanewline
\ \ \ \ \isacommand{apply}\isamarkupfalse%
{\isacharparenleft}{\kern0pt}rule{\isacharunderscore}{\kern0pt}tac\ n{\isacharequal}{\kern0pt}n\ \isakeyword{in}\ lt{\isacharunderscore}{\kern0pt}nat{\isacharunderscore}{\kern0pt}in{\isacharunderscore}{\kern0pt}nat{\isacharcomma}{\kern0pt}\ rule\ ltI{\isacharparenright}{\kern0pt}\isanewline
\ \ \isacommand{using}\isamarkupfalse%
\ assms\isanewline
\ \ \ \ \ \ \isacommand{apply}\isamarkupfalse%
\ auto{\isacharbrackleft}{\kern0pt}{\isadigit{5}}{\isacharbrackright}{\kern0pt}\isanewline
\ \ \isacommand{done}\isamarkupfalse%
%
\endisatagproof
{\isafoldproof}%
%
\isadelimproof
\isanewline
%
\endisadelimproof
\isanewline
\isacommand{lemma}\isamarkupfalse%
\ sats{\isacharunderscore}{\kern0pt}ren{\isacharunderscore}{\kern0pt}forcesHS{\isacharunderscore}{\kern0pt}forall{\isacharunderscore}{\kern0pt}iff\ {\isacharcolon}{\kern0pt}\ \isanewline
\ \ \isakeyword{fixes}\ {\isasymphi}\ n\ a\ b\ c\ d\ e\ f\ env\isanewline
\ \ \isakeyword{assumes}\ {\isachardoublequoteopen}{\isasymphi}\ {\isasymin}\ formula{\isachardoublequoteclose}\ {\isachardoublequoteopen}arity{\isacharparenleft}{\kern0pt}{\isasymphi}{\isacharparenright}{\kern0pt}\ {\isasymle}\ {\isadigit{6}}\ {\isasymunion}\ n{\isachardoublequoteclose}\ {\isachardoublequoteopen}n\ {\isasymin}\ nat{\isachardoublequoteclose}\ {\isachardoublequoteopen}a\ {\isasymin}\ M{\isachardoublequoteclose}\ {\isachardoublequoteopen}b\ {\isasymin}\ M{\isachardoublequoteclose}\ {\isachardoublequoteopen}c\ {\isasymin}\ M{\isachardoublequoteclose}\ {\isachardoublequoteopen}d\ {\isasymin}\ M{\isachardoublequoteclose}\ {\isachardoublequoteopen}e\ {\isasymin}\ M{\isachardoublequoteclose}\ {\isachardoublequoteopen}f\ {\isasymin}\ M{\isachardoublequoteclose}\ {\isachardoublequoteopen}env\ {\isasymin}\ list{\isacharparenleft}{\kern0pt}M{\isacharparenright}{\kern0pt}{\isachardoublequoteclose}\isanewline
\ \ \isakeyword{shows}\ {\isachardoublequoteopen}sats{\isacharparenleft}{\kern0pt}M{\isacharcomma}{\kern0pt}\ {\isasymphi}{\isacharcomma}{\kern0pt}\ {\isacharbrackleft}{\kern0pt}a{\isacharcomma}{\kern0pt}\ b{\isacharcomma}{\kern0pt}\ c{\isacharcomma}{\kern0pt}\ d{\isacharcomma}{\kern0pt}\ e{\isacharcomma}{\kern0pt}\ f{\isacharbrackright}{\kern0pt}\ {\isacharat}{\kern0pt}\ env{\isacharparenright}{\kern0pt}\ {\isasymlongleftrightarrow}\ sats{\isacharparenleft}{\kern0pt}M{\isacharcomma}{\kern0pt}\ ren{\isacharparenleft}{\kern0pt}{\isasymphi}{\isacharparenright}{\kern0pt}{\isacharbackquote}{\kern0pt}{\isacharparenleft}{\kern0pt}{\isadigit{6}}{\isasymunion}n{\isacharparenright}{\kern0pt}{\isacharbackquote}{\kern0pt}{\isacharparenleft}{\kern0pt}{\isadigit{6}}{\isasymunion}n{\isacharparenright}{\kern0pt}{\isacharbackquote}{\kern0pt}ren{\isacharunderscore}{\kern0pt}forcesHS{\isacharunderscore}{\kern0pt}forall{\isacharparenleft}{\kern0pt}n{\isacharparenright}{\kern0pt}{\isacharcomma}{\kern0pt}\ {\isacharbrackleft}{\kern0pt}f{\isacharcomma}{\kern0pt}\ a{\isacharcomma}{\kern0pt}\ b{\isacharcomma}{\kern0pt}\ c{\isacharcomma}{\kern0pt}\ d{\isacharcomma}{\kern0pt}\ e{\isacharbrackright}{\kern0pt}\ {\isacharat}{\kern0pt}\ env{\isacharparenright}{\kern0pt}{\isachardoublequoteclose}\isanewline
%
\isadelimproof
\isanewline
\ \ %
\endisadelimproof
%
\isatagproof
\isacommand{apply}\isamarkupfalse%
{\isacharparenleft}{\kern0pt}rule\ sats{\isacharunderscore}{\kern0pt}iff{\isacharunderscore}{\kern0pt}sats{\isacharunderscore}{\kern0pt}ren{\isacharparenright}{\kern0pt}\isanewline
\ \ \isacommand{using}\isamarkupfalse%
\ assms\isanewline
\ \ \ \ \ \ \ \ \ \isacommand{apply}\isamarkupfalse%
\ auto{\isacharbrackleft}{\kern0pt}{\isadigit{5}}{\isacharbrackright}{\kern0pt}\isanewline
\ \ \ \ \isacommand{apply}\isamarkupfalse%
{\isacharparenleft}{\kern0pt}rule\ ren{\isacharunderscore}{\kern0pt}forcesHS{\isacharunderscore}{\kern0pt}forall{\isacharunderscore}{\kern0pt}type{\isacharparenright}{\kern0pt}\isanewline
\ \ \isacommand{using}\isamarkupfalse%
\ assms\ \isanewline
\ \ \ \ \isacommand{apply}\isamarkupfalse%
\ auto{\isacharbrackleft}{\kern0pt}{\isadigit{2}}{\isacharbrackright}{\kern0pt}\isanewline
\ \ \isacommand{apply}\isamarkupfalse%
{\isacharparenleft}{\kern0pt}rename{\isacharunderscore}{\kern0pt}tac\ k{\isacharcomma}{\kern0pt}\ subgoal{\isacharunderscore}{\kern0pt}tac\ {\isachardoublequoteopen}k\ {\isasymin}\ {\isadigit{6}}\ {\isasymor}\ k\ {\isasymin}\ n{\isachardoublequoteclose}{\isacharparenright}{\kern0pt}\isanewline
\ \ \ \isacommand{apply}\isamarkupfalse%
{\isacharparenleft}{\kern0pt}rename{\isacharunderscore}{\kern0pt}tac\ k{\isacharcomma}{\kern0pt}\ case{\isacharunderscore}{\kern0pt}tac\ {\isachardoublequoteopen}k\ {\isasymin}\ {\isadigit{6}}{\isachardoublequoteclose}{\isacharcomma}{\kern0pt}\ simp{\isacharparenright}{\kern0pt}\isanewline
\ \ \ \ \isacommand{apply}\isamarkupfalse%
{\isacharparenleft}{\kern0pt}rename{\isacharunderscore}{\kern0pt}tac\ k{\isacharcomma}{\kern0pt}\ subgoal{\isacharunderscore}{\kern0pt}tac\ {\isachardoublequoteopen}k\ {\isacharless}{\kern0pt}\ {\isadigit{6}}{\isachardoublequoteclose}{\isacharparenright}{\kern0pt}\isanewline
\ \ \ \ \ \isacommand{apply}\isamarkupfalse%
{\isacharparenleft}{\kern0pt}rename{\isacharunderscore}{\kern0pt}tac\ k{\isacharcomma}{\kern0pt}\ subgoal{\isacharunderscore}{\kern0pt}tac\ {\isachardoublequoteopen}k\ {\isacharequal}{\kern0pt}\ {\isadigit{0}}\ {\isasymor}\ k\ {\isacharequal}{\kern0pt}\ {\isadigit{1}}\ {\isasymor}\ k\ {\isacharequal}{\kern0pt}\ {\isadigit{2}}\ {\isasymor}\ k\ {\isacharequal}{\kern0pt}\ {\isadigit{3}}\ {\isasymor}\ k\ {\isacharequal}{\kern0pt}\ {\isadigit{4}}\ {\isasymor}\ k\ {\isacharequal}{\kern0pt}\ {\isadigit{5}}{\isachardoublequoteclose}{\isacharparenright}{\kern0pt}\ \isanewline
\ \ \ \ \ \ \isacommand{apply}\isamarkupfalse%
{\isacharparenleft}{\kern0pt}subgoal{\isacharunderscore}{\kern0pt}tac\ {\isachardoublequoteopen}function{\isacharparenleft}{\kern0pt}ren{\isacharunderscore}{\kern0pt}forcesHS{\isacharunderscore}{\kern0pt}forall{\isacharparenleft}{\kern0pt}n{\isacharparenright}{\kern0pt}{\isacharparenright}{\kern0pt}{\isachardoublequoteclose}{\isacharparenright}{\kern0pt}\isanewline
\ \ \isacommand{unfolding}\isamarkupfalse%
\ ren{\isacharunderscore}{\kern0pt}forcesHS{\isacharunderscore}{\kern0pt}forall{\isacharunderscore}{\kern0pt}def\isanewline
\ \ \ \ \ \ \isacommand{apply}\isamarkupfalse%
{\isacharparenleft}{\kern0pt}rule\ disjE{\isacharcomma}{\kern0pt}\ assumption{\isacharcomma}{\kern0pt}\ subst\ function{\isacharunderscore}{\kern0pt}apply{\isacharunderscore}{\kern0pt}equality{\isacharcomma}{\kern0pt}\ simp{\isacharcomma}{\kern0pt}\ assumption{\isacharcomma}{\kern0pt}\ simp{\isacharcomma}{\kern0pt}\ simp{\isacharparenright}{\kern0pt}\isanewline
\ \ \ \ \ \ \ \isacommand{apply}\isamarkupfalse%
{\isacharparenleft}{\kern0pt}rule\ disjE{\isacharcomma}{\kern0pt}\ assumption{\isacharcomma}{\kern0pt}\ subst\ function{\isacharunderscore}{\kern0pt}apply{\isacharunderscore}{\kern0pt}equality{\isacharcomma}{\kern0pt}\ rule\ UnI{\isadigit{1}}{\isacharcomma}{\kern0pt}\ simp{\isacharcomma}{\kern0pt}\ assumption{\isacharcomma}{\kern0pt}\ simp{\isacharcomma}{\kern0pt}\ simp{\isacharparenright}{\kern0pt}{\isacharplus}{\kern0pt}\isanewline
\ \ \ \ \ \ \ \isacommand{apply}\isamarkupfalse%
{\isacharparenleft}{\kern0pt}subst\ function{\isacharunderscore}{\kern0pt}apply{\isacharunderscore}{\kern0pt}equality{\isacharcomma}{\kern0pt}\ rule\ UnI{\isadigit{1}}{\isacharcomma}{\kern0pt}\ simp{\isacharcomma}{\kern0pt}\ assumption{\isacharcomma}{\kern0pt}\ force{\isacharparenright}{\kern0pt}\isanewline
\ \ \isacommand{using}\isamarkupfalse%
\ ren{\isacharunderscore}{\kern0pt}forcesHS{\isacharunderscore}{\kern0pt}forall{\isacharunderscore}{\kern0pt}type\ assms\ Pi{\isacharunderscore}{\kern0pt}def\ ren{\isacharunderscore}{\kern0pt}forcesHS{\isacharunderscore}{\kern0pt}forall{\isacharunderscore}{\kern0pt}def\ \isanewline
\ \ \ \ \ \ \isacommand{apply}\isamarkupfalse%
\ force\ \isanewline
\ \ \isacommand{using}\isamarkupfalse%
\ le{\isacharunderscore}{\kern0pt}iff\ \isanewline
\ \ \ \ \ \isacommand{apply}\isamarkupfalse%
\ force\ \isanewline
\ \ \ \ \isacommand{apply}\isamarkupfalse%
{\isacharparenleft}{\kern0pt}rule\ ltI{\isacharcomma}{\kern0pt}\ simp{\isacharcomma}{\kern0pt}\ simp{\isacharparenright}{\kern0pt}\isanewline
\ \ \isacommand{apply}\isamarkupfalse%
{\isacharparenleft}{\kern0pt}rename{\isacharunderscore}{\kern0pt}tac\ k{\isacharcomma}{\kern0pt}\ subgoal{\isacharunderscore}{\kern0pt}tac\ {\isachardoublequoteopen}{\isadigit{5}}\ {\isacharless}{\kern0pt}\ k{\isachardoublequoteclose}{\isacharparenright}{\kern0pt}\isanewline
\ \ \ \ \isacommand{apply}\isamarkupfalse%
{\isacharparenleft}{\kern0pt}subst\ function{\isacharunderscore}{\kern0pt}apply{\isacharunderscore}{\kern0pt}equality{\isacharcomma}{\kern0pt}\ rule\ UnI{\isadigit{2}}{\isacharcomma}{\kern0pt}\ force{\isacharparenright}{\kern0pt}\isanewline
\ \ \isacommand{using}\isamarkupfalse%
\ ren{\isacharunderscore}{\kern0pt}forcesHS{\isacharunderscore}{\kern0pt}forall{\isacharunderscore}{\kern0pt}type\ assms\ Pi{\isacharunderscore}{\kern0pt}def\ ren{\isacharunderscore}{\kern0pt}forcesHS{\isacharunderscore}{\kern0pt}forall{\isacharunderscore}{\kern0pt}def\ \isanewline
\ \ \ \ \ \isacommand{apply}\isamarkupfalse%
\ force\ \isanewline
\ \ \ \ \isacommand{apply}\isamarkupfalse%
{\isacharparenleft}{\kern0pt}subst\ nth{\isacharunderscore}{\kern0pt}append{\isacharparenright}{\kern0pt}\isanewline
\ \ \isacommand{using}\isamarkupfalse%
\ assms\ lt{\isacharunderscore}{\kern0pt}nat{\isacharunderscore}{\kern0pt}in{\isacharunderscore}{\kern0pt}nat\ assms\isanewline
\ \ \ \ \ \ \isacommand{apply}\isamarkupfalse%
\ auto{\isacharbrackleft}{\kern0pt}{\isadigit{2}}{\isacharbrackright}{\kern0pt}\isanewline
\ \ \ \ \isacommand{apply}\isamarkupfalse%
{\isacharparenleft}{\kern0pt}subst\ {\isacharparenleft}{\kern0pt}{\isadigit{1}}{\isacharparenright}{\kern0pt}\ nth{\isacharunderscore}{\kern0pt}append{\isacharparenright}{\kern0pt}\isanewline
\ \ \isacommand{using}\isamarkupfalse%
\ assms\ lt{\isacharunderscore}{\kern0pt}nat{\isacharunderscore}{\kern0pt}in{\isacharunderscore}{\kern0pt}nat\ assms\isanewline
\ \ \ \ \ \ \isacommand{apply}\isamarkupfalse%
\ auto{\isacharbrackleft}{\kern0pt}{\isadigit{2}}{\isacharbrackright}{\kern0pt}\isanewline
\ \ \ \ \isacommand{apply}\isamarkupfalse%
{\isacharparenleft}{\kern0pt}subst\ if{\isacharunderscore}{\kern0pt}not{\isacharunderscore}{\kern0pt}P{\isacharcomma}{\kern0pt}\ simp{\isacharparenright}{\kern0pt}\isanewline
\ \ \isacommand{apply}\isamarkupfalse%
{\isacharparenleft}{\kern0pt}rule\ iffD{\isadigit{2}}{\isacharcomma}{\kern0pt}\ rule\ not{\isacharunderscore}{\kern0pt}le{\isacharunderscore}{\kern0pt}iff{\isacharunderscore}{\kern0pt}lt{\isacharparenright}{\kern0pt}\isanewline
\ \ \isacommand{using}\isamarkupfalse%
\ assms\ lt{\isacharunderscore}{\kern0pt}nat{\isacharunderscore}{\kern0pt}in{\isacharunderscore}{\kern0pt}nat\ assms\isanewline
\ \ \ \ \ \ \ \isacommand{apply}\isamarkupfalse%
\ auto{\isacharbrackleft}{\kern0pt}{\isadigit{3}}{\isacharbrackright}{\kern0pt}\isanewline
\ \ \ \ \isacommand{apply}\isamarkupfalse%
{\isacharparenleft}{\kern0pt}subst\ if{\isacharunderscore}{\kern0pt}not{\isacharunderscore}{\kern0pt}P{\isacharcomma}{\kern0pt}\ simp{\isacharparenright}{\kern0pt}\isanewline
\ \ \isacommand{apply}\isamarkupfalse%
{\isacharparenleft}{\kern0pt}rule\ iffD{\isadigit{2}}{\isacharcomma}{\kern0pt}\ rule\ not{\isacharunderscore}{\kern0pt}le{\isacharunderscore}{\kern0pt}iff{\isacharunderscore}{\kern0pt}lt{\isacharparenright}{\kern0pt}\isanewline
\ \ \isacommand{using}\isamarkupfalse%
\ assms\ lt{\isacharunderscore}{\kern0pt}nat{\isacharunderscore}{\kern0pt}in{\isacharunderscore}{\kern0pt}nat\ assms\isanewline
\ \ \ \ \ \ \ \isacommand{apply}\isamarkupfalse%
\ auto{\isacharbrackleft}{\kern0pt}{\isadigit{4}}{\isacharbrackright}{\kern0pt}\isanewline
\ \ \ \isacommand{apply}\isamarkupfalse%
{\isacharparenleft}{\kern0pt}rule\ ccontr{\isacharparenright}{\kern0pt}\isanewline
\ \ \ \isacommand{apply}\isamarkupfalse%
{\isacharparenleft}{\kern0pt}rename{\isacharunderscore}{\kern0pt}tac\ k{\isacharcomma}{\kern0pt}\ subgoal{\isacharunderscore}{\kern0pt}tac\ {\isachardoublequoteopen}k\ {\isasymle}\ {\isadigit{5}}{\isachardoublequoteclose}{\isacharparenright}{\kern0pt}\isanewline
\ \ \isacommand{using}\isamarkupfalse%
\ ltD\ \isanewline
\ \ \ \ \isacommand{apply}\isamarkupfalse%
\ force\isanewline
\ \ \isacommand{apply}\isamarkupfalse%
{\isacharparenleft}{\kern0pt}rule\ iffD{\isadigit{1}}{\isacharcomma}{\kern0pt}\ rule\ not{\isacharunderscore}{\kern0pt}lt{\isacharunderscore}{\kern0pt}iff{\isacharunderscore}{\kern0pt}le{\isacharparenright}{\kern0pt}\isanewline
\ \ \isacommand{using}\isamarkupfalse%
\ assms\ lt{\isacharunderscore}{\kern0pt}nat{\isacharunderscore}{\kern0pt}in{\isacharunderscore}{\kern0pt}nat\ assms\isanewline
\ \ \ \ \ \isacommand{apply}\isamarkupfalse%
\ auto{\isacharbrackleft}{\kern0pt}{\isadigit{3}}{\isacharbrackright}{\kern0pt}\isanewline
\ \ \isacommand{apply}\isamarkupfalse%
{\isacharparenleft}{\kern0pt}rename{\isacharunderscore}{\kern0pt}tac\ k{\isacharcomma}{\kern0pt}\ subgoal{\isacharunderscore}{\kern0pt}tac\ {\isachardoublequoteopen}k\ {\isasymin}\ {\isadigit{6}}\ {\isasymunion}\ n{\isachardoublequoteclose}{\isacharcomma}{\kern0pt}\ force{\isacharparenright}{\kern0pt}\isanewline
\ \ \isacommand{apply}\isamarkupfalse%
{\isacharparenleft}{\kern0pt}rule\ ltD{\isacharcomma}{\kern0pt}\ simp{\isacharparenright}{\kern0pt}\isanewline
\ \ \isacommand{done}\isamarkupfalse%
%
\endisatagproof
{\isafoldproof}%
%
\isadelimproof
\isanewline
%
\endisadelimproof
\isanewline
\isacommand{lemma}\isamarkupfalse%
\ forcesHS{\isacharprime}{\kern0pt}{\isacharunderscore}{\kern0pt}type\ {\isacharcolon}{\kern0pt}\ \isanewline
\ \ \isakeyword{fixes}\ {\isasymphi}\isanewline
\ \ \isakeyword{assumes}\ {\isachardoublequoteopen}{\isasymphi}\ {\isasymin}\ formula{\isachardoublequoteclose}\ \isanewline
\ \ \isakeyword{shows}\ {\isachardoublequoteopen}forcesHS{\isacharprime}{\kern0pt}{\isacharparenleft}{\kern0pt}{\isasymphi}{\isacharparenright}{\kern0pt}\ {\isasymin}\ formula{\isachardoublequoteclose}\ \isanewline
%
\isadelimproof
\ \ %
\endisadelimproof
%
\isatagproof
\isacommand{using}\isamarkupfalse%
\ assms\isanewline
\ \ \isacommand{apply}\isamarkupfalse%
{\isacharparenleft}{\kern0pt}induct{\isacharparenright}{\kern0pt}\isanewline
\ \ \ \ \ \isacommand{apply}\isamarkupfalse%
\ auto{\isacharbrackleft}{\kern0pt}{\isadigit{3}}{\isacharbrackright}{\kern0pt}\isanewline
\ \ \isacommand{apply}\isamarkupfalse%
\ simp\isanewline
\ \ \isacommand{apply}\isamarkupfalse%
{\isacharparenleft}{\kern0pt}clarify{\isacharcomma}{\kern0pt}\ rule\ Implies{\isacharunderscore}{\kern0pt}type{\isacharparenright}{\kern0pt}\isanewline
\ \ \ \isacommand{apply}\isamarkupfalse%
{\isacharparenleft}{\kern0pt}rule\ is{\isacharunderscore}{\kern0pt}HS{\isacharunderscore}{\kern0pt}fm{\isacharunderscore}{\kern0pt}type{\isacharparenright}{\kern0pt}\isanewline
\ \ \ \ \isacommand{apply}\isamarkupfalse%
\ auto{\isacharbrackleft}{\kern0pt}{\isadigit{2}}{\isacharbrackright}{\kern0pt}\isanewline
\ \ \isacommand{apply}\isamarkupfalse%
{\isacharparenleft}{\kern0pt}rule\ ren{\isacharunderscore}{\kern0pt}tc{\isacharparenright}{\kern0pt}\isanewline
\ \ \isacommand{using}\isamarkupfalse%
\ assms\isanewline
\ \ \ \ \ \isacommand{apply}\isamarkupfalse%
\ auto{\isacharbrackleft}{\kern0pt}{\isadigit{3}}{\isacharbrackright}{\kern0pt}\isanewline
\ \ \isacommand{apply}\isamarkupfalse%
{\isacharparenleft}{\kern0pt}rule\ ren{\isacharunderscore}{\kern0pt}forcesHS{\isacharunderscore}{\kern0pt}forall{\isacharunderscore}{\kern0pt}type{\isacharparenright}{\kern0pt}\isanewline
\ \ \isacommand{by}\isamarkupfalse%
\ simp%
\endisatagproof
{\isafoldproof}%
%
\isadelimproof
\isanewline
%
\endisadelimproof
\isanewline
\isacommand{lemma}\isamarkupfalse%
\ forcesHS{\isacharunderscore}{\kern0pt}type\ {\isacharcolon}{\kern0pt}\ \isanewline
\ \ \isakeyword{fixes}\ {\isasymphi}\isanewline
\ \ \isakeyword{assumes}\ {\isachardoublequoteopen}{\isasymphi}\ {\isasymin}\ formula{\isachardoublequoteclose}\ \isanewline
\ \ \isakeyword{shows}\ {\isachardoublequoteopen}forcesHS{\isacharparenleft}{\kern0pt}{\isasymphi}{\isacharparenright}{\kern0pt}\ {\isasymin}\ formula{\isachardoublequoteclose}\ \isanewline
%
\isadelimproof
\ \ %
\endisadelimproof
%
\isatagproof
\isacommand{unfolding}\isamarkupfalse%
\ forcesHS{\isacharunderscore}{\kern0pt}def\isanewline
\ \ \isacommand{apply}\isamarkupfalse%
{\isacharparenleft}{\kern0pt}rule\ And{\isacharunderscore}{\kern0pt}type{\isacharcomma}{\kern0pt}\ simp{\isacharparenright}{\kern0pt}\isanewline
\ \ \isacommand{using}\isamarkupfalse%
\ forcesHS{\isacharprime}{\kern0pt}{\isacharunderscore}{\kern0pt}type\ assms\isanewline
\ \ \isacommand{by}\isamarkupfalse%
\ auto%
\endisatagproof
{\isafoldproof}%
%
\isadelimproof
\isanewline
%
\endisadelimproof
\isanewline
\isacommand{lemma}\isamarkupfalse%
\ arity{\isacharunderscore}{\kern0pt}forcesHS{\isacharprime}{\kern0pt}\ {\isacharcolon}{\kern0pt}\ \isanewline
\ \ \isakeyword{fixes}\ {\isasymphi}\isanewline
\ \ \isakeyword{assumes}\ {\isachardoublequoteopen}{\isasymphi}\ {\isasymin}\ formula{\isachardoublequoteclose}\ \isanewline
\ \ \isakeyword{shows}\ {\isachardoublequoteopen}arity{\isacharparenleft}{\kern0pt}forcesHS{\isacharprime}{\kern0pt}{\isacharparenleft}{\kern0pt}{\isasymphi}{\isacharparenright}{\kern0pt}{\isacharparenright}{\kern0pt}\ {\isasymle}\ arity{\isacharparenleft}{\kern0pt}{\isasymphi}{\isacharparenright}{\kern0pt}\ {\isacharhash}{\kern0pt}{\isacharplus}{\kern0pt}\ {\isadigit{5}}{\isachardoublequoteclose}\ \isanewline
%
\isadelimproof
\ \ %
\endisadelimproof
%
\isatagproof
\isacommand{using}\isamarkupfalse%
\ assms\isanewline
\ \ \isacommand{apply}\isamarkupfalse%
{\isacharparenleft}{\kern0pt}induct{\isacharparenright}{\kern0pt}\isanewline
\ \ \isacommand{apply}\isamarkupfalse%
\ simp\ \isanewline
\ \ \ \ \ \isacommand{apply}\isamarkupfalse%
{\isacharparenleft}{\kern0pt}subst\ arity{\isacharunderscore}{\kern0pt}incr{\isacharunderscore}{\kern0pt}bv{\isacharunderscore}{\kern0pt}lemma{\isacharparenright}{\kern0pt}\isanewline
\ \ \ \ \ \ \ \isacommand{apply}\isamarkupfalse%
{\isacharparenleft}{\kern0pt}rule\ forces{\isacharunderscore}{\kern0pt}mem{\isacharunderscore}{\kern0pt}fm{\isacharunderscore}{\kern0pt}type{\isacharparenright}{\kern0pt}\isanewline
\ \ \ \ \ \ \ \ \ \ \ \isacommand{apply}\isamarkupfalse%
\ auto{\isacharbrackleft}{\kern0pt}{\isadigit{6}}{\isacharbrackright}{\kern0pt}\isanewline
\ \ \ \ \ \isacommand{apply}\isamarkupfalse%
\ {\isacharparenleft}{\kern0pt}subst\ if{\isacharunderscore}{\kern0pt}P{\isacharcomma}{\kern0pt}\ subst\ arity{\isacharunderscore}{\kern0pt}forces{\isacharunderscore}{\kern0pt}mem{\isacharunderscore}{\kern0pt}fm{\isacharparenright}{\kern0pt}\isanewline
\ \ \ \ \ \ \ \ \ \ \ \isacommand{apply}\isamarkupfalse%
\ auto{\isacharbrackleft}{\kern0pt}{\isadigit{5}}{\isacharbrackright}{\kern0pt}\isanewline
\ \ \ \ \ \ \isacommand{apply}\isamarkupfalse%
{\isacharparenleft}{\kern0pt}rule\ ltI{\isacharcomma}{\kern0pt}\ simp{\isacharcomma}{\kern0pt}\ rule\ disjI{\isadigit{1}}{\isacharcomma}{\kern0pt}\ rule\ ltD{\isacharcomma}{\kern0pt}\ simp{\isacharcomma}{\kern0pt}\ simp{\isacharcomma}{\kern0pt}\ simp{\isacharparenright}{\kern0pt}\isanewline
\ \ \ \ \ \isacommand{apply}\isamarkupfalse%
{\isacharparenleft}{\kern0pt}subst\ arity{\isacharunderscore}{\kern0pt}forces{\isacharunderscore}{\kern0pt}mem{\isacharunderscore}{\kern0pt}fm{\isacharparenright}{\kern0pt}\isanewline
\ \ \ \ \ \ \ \ \ \ \isacommand{apply}\isamarkupfalse%
\ auto{\isacharbrackleft}{\kern0pt}{\isadigit{5}}{\isacharbrackright}{\kern0pt}\isanewline
\ \ \ \ \ \isacommand{apply}\isamarkupfalse%
{\isacharparenleft}{\kern0pt}subst\ succ{\isacharunderscore}{\kern0pt}Un{\isacharunderscore}{\kern0pt}distrib{\isacharcomma}{\kern0pt}\ simp{\isacharcomma}{\kern0pt}\ simp{\isacharparenright}{\kern0pt}{\isacharplus}{\kern0pt}\isanewline
\ \ \ \ \ \isacommand{apply}\isamarkupfalse%
{\isacharparenleft}{\kern0pt}rule\ Un{\isacharunderscore}{\kern0pt}least{\isacharunderscore}{\kern0pt}lt{\isacharparenright}{\kern0pt}{\isacharplus}{\kern0pt}\isanewline
\ \ \ \ \ \ \ \ \ \isacommand{apply}\isamarkupfalse%
{\isacharparenleft}{\kern0pt}rule\ ltI{\isacharcomma}{\kern0pt}\ simp{\isacharcomma}{\kern0pt}\ simp{\isacharparenright}{\kern0pt}{\isacharplus}{\kern0pt}\isanewline
\ \ \ \ \ \ \ \isacommand{apply}\isamarkupfalse%
{\isacharparenleft}{\kern0pt}rule\ ltI{\isacharcomma}{\kern0pt}\ simp{\isacharcomma}{\kern0pt}\ rule\ disjI{\isadigit{1}}{\isacharcomma}{\kern0pt}\ rule\ ltD{\isacharcomma}{\kern0pt}\ simp{\isacharcomma}{\kern0pt}\ simp{\isacharparenright}{\kern0pt}{\isacharplus}{\kern0pt}\isanewline
\ \ \isacommand{apply}\isamarkupfalse%
\ simp\ \isanewline
\ \ \ \ \ \isacommand{apply}\isamarkupfalse%
{\isacharparenleft}{\kern0pt}subst\ arity{\isacharunderscore}{\kern0pt}incr{\isacharunderscore}{\kern0pt}bv{\isacharunderscore}{\kern0pt}lemma{\isacharparenright}{\kern0pt}\isanewline
\ \ \ \ \ \ \ \isacommand{apply}\isamarkupfalse%
{\isacharparenleft}{\kern0pt}rule\ forces{\isacharunderscore}{\kern0pt}eq{\isacharunderscore}{\kern0pt}fm{\isacharunderscore}{\kern0pt}type{\isacharparenright}{\kern0pt}\isanewline
\ \ \ \ \ \ \ \ \ \ \ \isacommand{apply}\isamarkupfalse%
\ auto{\isacharbrackleft}{\kern0pt}{\isadigit{6}}{\isacharbrackright}{\kern0pt}\isanewline
\ \ \ \ \ \isacommand{apply}\isamarkupfalse%
\ {\isacharparenleft}{\kern0pt}subst\ if{\isacharunderscore}{\kern0pt}P{\isacharcomma}{\kern0pt}\ subst\ arity{\isacharunderscore}{\kern0pt}forces{\isacharunderscore}{\kern0pt}eq{\isacharunderscore}{\kern0pt}fm{\isacharparenright}{\kern0pt}\isanewline
\ \ \ \ \ \ \ \ \ \ \ \isacommand{apply}\isamarkupfalse%
\ auto{\isacharbrackleft}{\kern0pt}{\isadigit{5}}{\isacharbrackright}{\kern0pt}\isanewline
\ \ \ \ \ \ \isacommand{apply}\isamarkupfalse%
{\isacharparenleft}{\kern0pt}rule\ ltI{\isacharcomma}{\kern0pt}\ simp{\isacharcomma}{\kern0pt}\ rule\ disjI{\isadigit{1}}{\isacharcomma}{\kern0pt}\ rule\ ltD{\isacharcomma}{\kern0pt}\ simp{\isacharcomma}{\kern0pt}\ simp{\isacharcomma}{\kern0pt}\ simp{\isacharparenright}{\kern0pt}\isanewline
\ \ \ \ \ \isacommand{apply}\isamarkupfalse%
{\isacharparenleft}{\kern0pt}subst\ arity{\isacharunderscore}{\kern0pt}forces{\isacharunderscore}{\kern0pt}eq{\isacharunderscore}{\kern0pt}fm{\isacharparenright}{\kern0pt}\isanewline
\ \ \ \ \ \ \ \ \ \ \isacommand{apply}\isamarkupfalse%
\ auto{\isacharbrackleft}{\kern0pt}{\isadigit{5}}{\isacharbrackright}{\kern0pt}\isanewline
\ \ \ \ \ \isacommand{apply}\isamarkupfalse%
{\isacharparenleft}{\kern0pt}subst\ succ{\isacharunderscore}{\kern0pt}Un{\isacharunderscore}{\kern0pt}distrib{\isacharcomma}{\kern0pt}\ simp{\isacharcomma}{\kern0pt}\ simp{\isacharparenright}{\kern0pt}{\isacharplus}{\kern0pt}\isanewline
\ \ \ \ \ \isacommand{apply}\isamarkupfalse%
{\isacharparenleft}{\kern0pt}rule\ Un{\isacharunderscore}{\kern0pt}least{\isacharunderscore}{\kern0pt}lt{\isacharparenright}{\kern0pt}{\isacharplus}{\kern0pt}\isanewline
\ \ \ \ \ \ \ \ \ \isacommand{apply}\isamarkupfalse%
{\isacharparenleft}{\kern0pt}rule\ ltI{\isacharcomma}{\kern0pt}\ simp{\isacharcomma}{\kern0pt}\ simp{\isacharparenright}{\kern0pt}{\isacharplus}{\kern0pt}\isanewline
\ \ \ \ \ \ \isacommand{apply}\isamarkupfalse%
{\isacharparenleft}{\kern0pt}rule\ ltI{\isacharcomma}{\kern0pt}\ simp{\isacharcomma}{\kern0pt}\ rule\ disjI{\isadigit{1}}{\isacharcomma}{\kern0pt}\ rule\ ltD{\isacharcomma}{\kern0pt}\ simp{\isacharcomma}{\kern0pt}\ simp{\isacharparenright}{\kern0pt}{\isacharplus}{\kern0pt}\isanewline
\ \ \ \isacommand{apply}\isamarkupfalse%
\ simp\isanewline
\ \ \ \isacommand{apply}\isamarkupfalse%
{\isacharparenleft}{\kern0pt}rule\ pred{\isacharunderscore}{\kern0pt}le{\isacharparenright}{\kern0pt}\isanewline
\ \ \ \ \ \isacommand{apply}\isamarkupfalse%
{\isacharparenleft}{\kern0pt}rule\ union{\isacharunderscore}{\kern0pt}in{\isacharunderscore}{\kern0pt}nat{\isacharcomma}{\kern0pt}\ simp{\isacharparenright}{\kern0pt}{\isacharplus}{\kern0pt}\isanewline
\ \ \ \ \ \isacommand{apply}\isamarkupfalse%
{\isacharparenleft}{\kern0pt}rule\ union{\isacharunderscore}{\kern0pt}in{\isacharunderscore}{\kern0pt}nat{\isacharcomma}{\kern0pt}\ rule\ arity{\isacharunderscore}{\kern0pt}type{\isacharcomma}{\kern0pt}\ rule\ ren{\isacharunderscore}{\kern0pt}forces{\isacharunderscore}{\kern0pt}nand{\isacharunderscore}{\kern0pt}type{\isacharcomma}{\kern0pt}\ rule\ forcesHS{\isacharprime}{\kern0pt}{\isacharunderscore}{\kern0pt}type{\isacharcomma}{\kern0pt}\ simp{\isacharparenright}{\kern0pt}\isanewline
\ \ \ \ \ \isacommand{apply}\isamarkupfalse%
{\isacharparenleft}{\kern0pt}rule\ arity{\isacharunderscore}{\kern0pt}type{\isacharcomma}{\kern0pt}\ rule\ ren{\isacharunderscore}{\kern0pt}forces{\isacharunderscore}{\kern0pt}nand{\isacharunderscore}{\kern0pt}type{\isacharcomma}{\kern0pt}\ rule\ forcesHS{\isacharprime}{\kern0pt}{\isacharunderscore}{\kern0pt}type{\isacharcomma}{\kern0pt}\ simp{\isacharparenright}{\kern0pt}\isanewline
\ \ \ \ \isacommand{apply}\isamarkupfalse%
\ simp\isanewline
\ \ \ \isacommand{apply}\isamarkupfalse%
{\isacharparenleft}{\kern0pt}rule\ Un{\isacharunderscore}{\kern0pt}least{\isacharunderscore}{\kern0pt}lt{\isacharparenright}{\kern0pt}{\isacharplus}{\kern0pt}\isanewline
\ \ \ \ \ \isacommand{apply}\isamarkupfalse%
\ auto{\isacharbrackleft}{\kern0pt}{\isadigit{2}}{\isacharbrackright}{\kern0pt}\isanewline
\ \ \ \isacommand{apply}\isamarkupfalse%
{\isacharparenleft}{\kern0pt}rule\ Un{\isacharunderscore}{\kern0pt}least{\isacharunderscore}{\kern0pt}lt{\isacharparenright}{\kern0pt}{\isacharplus}{\kern0pt}\isanewline
\ \ \ \ \isacommand{apply}\isamarkupfalse%
{\isacharparenleft}{\kern0pt}subst\ arity{\isacharunderscore}{\kern0pt}leq{\isacharunderscore}{\kern0pt}fm{\isacharparenright}{\kern0pt}\isanewline
\ \ \ \ \ \ \ \isacommand{apply}\isamarkupfalse%
\ auto{\isacharbrackleft}{\kern0pt}{\isadigit{3}}{\isacharbrackright}{\kern0pt}\isanewline
\ \ \ \ \isacommand{apply}\isamarkupfalse%
{\isacharparenleft}{\kern0pt}rule\ Un{\isacharunderscore}{\kern0pt}least{\isacharunderscore}{\kern0pt}lt{\isacharparenright}{\kern0pt}{\isacharplus}{\kern0pt}\isanewline
\ \ \ \ \ \ \isacommand{apply}\isamarkupfalse%
\ auto{\isacharbrackleft}{\kern0pt}{\isadigit{3}}{\isacharbrackright}{\kern0pt}\isanewline
\ \ \ \isacommand{apply}\isamarkupfalse%
{\isacharparenleft}{\kern0pt}rule\ Un{\isacharunderscore}{\kern0pt}least{\isacharunderscore}{\kern0pt}lt{\isacharparenright}{\kern0pt}{\isacharplus}{\kern0pt}\ \isanewline
\ \ \ \ \isacommand{apply}\isamarkupfalse%
{\isacharparenleft}{\kern0pt}rule\ le{\isacharunderscore}{\kern0pt}trans{\isacharcomma}{\kern0pt}\ rule\ arity{\isacharunderscore}{\kern0pt}ren{\isacharunderscore}{\kern0pt}forces{\isacharunderscore}{\kern0pt}nand{\isacharcomma}{\kern0pt}\ rule\ forcesHS{\isacharprime}{\kern0pt}{\isacharunderscore}{\kern0pt}type{\isacharcomma}{\kern0pt}\ simp{\isacharcomma}{\kern0pt}\ simp{\isacharparenright}{\kern0pt}\isanewline
\ \ \ \ \isacommand{apply}\isamarkupfalse%
{\isacharparenleft}{\kern0pt}subst\ succ{\isacharunderscore}{\kern0pt}Un{\isacharunderscore}{\kern0pt}distrib{\isacharcomma}{\kern0pt}\ simp{\isacharcomma}{\kern0pt}\ simp{\isacharparenright}{\kern0pt}{\isacharplus}{\kern0pt}\isanewline
\ \ \ \ \isacommand{apply}\isamarkupfalse%
{\isacharparenleft}{\kern0pt}rule\ ltI{\isacharcomma}{\kern0pt}\ simp\ {\isacharcomma}{\kern0pt}rule\ disjI{\isadigit{1}}{\isacharcomma}{\kern0pt}\ rule\ ltD{\isacharcomma}{\kern0pt}\ simp{\isacharcomma}{\kern0pt}\ simp{\isacharparenright}{\kern0pt}\isanewline
\ \ \ \isacommand{apply}\isamarkupfalse%
{\isacharparenleft}{\kern0pt}rule{\isacharunderscore}{\kern0pt}tac\ le{\isacharunderscore}{\kern0pt}trans{\isacharcomma}{\kern0pt}\ rule\ arity{\isacharunderscore}{\kern0pt}ren{\isacharunderscore}{\kern0pt}forces{\isacharunderscore}{\kern0pt}nand{\isacharcomma}{\kern0pt}\ rule\ forcesHS{\isacharprime}{\kern0pt}{\isacharunderscore}{\kern0pt}type{\isacharcomma}{\kern0pt}\ simp{\isacharcomma}{\kern0pt}\ simp{\isacharparenright}{\kern0pt}\isanewline
\ \ \ \isacommand{apply}\isamarkupfalse%
{\isacharparenleft}{\kern0pt}subst\ succ{\isacharunderscore}{\kern0pt}Un{\isacharunderscore}{\kern0pt}distrib{\isacharcomma}{\kern0pt}\ simp{\isacharcomma}{\kern0pt}\ simp{\isacharparenright}{\kern0pt}{\isacharplus}{\kern0pt}\isanewline
\ \ \ \isacommand{apply}\isamarkupfalse%
{\isacharparenleft}{\kern0pt}rule\ ltI{\isacharcomma}{\kern0pt}\ simp\ {\isacharcomma}{\kern0pt}rule\ disjI{\isadigit{2}}{\isacharcomma}{\kern0pt}\ rule\ ltD{\isacharcomma}{\kern0pt}\ simp{\isacharcomma}{\kern0pt}\ simp{\isacharparenright}{\kern0pt}\isanewline
\ \ \isacommand{apply}\isamarkupfalse%
\ simp\isanewline
\ \ \isacommand{apply}\isamarkupfalse%
{\isacharparenleft}{\kern0pt}rename{\isacharunderscore}{\kern0pt}tac\ p{\isacharcomma}{\kern0pt}\ subgoal{\isacharunderscore}{\kern0pt}tac\ {\isachardoublequoteopen}forcesHS{\isacharprime}{\kern0pt}{\isacharparenleft}{\kern0pt}p{\isacharparenright}{\kern0pt}\ {\isasymin}\ formula\ {\isasymand}\ is{\isacharunderscore}{\kern0pt}HS{\isacharunderscore}{\kern0pt}fm{\isacharparenleft}{\kern0pt}{\isadigit{5}}{\isacharcomma}{\kern0pt}\ {\isadigit{0}}{\isacharparenright}{\kern0pt}\ {\isasymin}\ formula\ {\isasymand}\ ren{\isacharparenleft}{\kern0pt}forcesHS{\isacharprime}{\kern0pt}{\isacharparenleft}{\kern0pt}p{\isacharparenright}{\kern0pt}{\isacharparenright}{\kern0pt}\ {\isacharbackquote}{\kern0pt}\ {\isacharparenleft}{\kern0pt}{\isadigit{6}}\ {\isasymunion}\ arity{\isacharparenleft}{\kern0pt}forcesHS{\isacharprime}{\kern0pt}{\isacharparenleft}{\kern0pt}p{\isacharparenright}{\kern0pt}{\isacharparenright}{\kern0pt}{\isacharparenright}{\kern0pt}\ {\isacharbackquote}{\kern0pt}\ {\isacharparenleft}{\kern0pt}{\isadigit{6}}\ {\isasymunion}\ arity{\isacharparenleft}{\kern0pt}forcesHS{\isacharprime}{\kern0pt}{\isacharparenleft}{\kern0pt}p{\isacharparenright}{\kern0pt}{\isacharparenright}{\kern0pt}{\isacharparenright}{\kern0pt}\ {\isacharbackquote}{\kern0pt}\ ren{\isacharunderscore}{\kern0pt}forcesHS{\isacharunderscore}{\kern0pt}forall{\isacharparenleft}{\kern0pt}arity{\isacharparenleft}{\kern0pt}forcesHS{\isacharprime}{\kern0pt}{\isacharparenleft}{\kern0pt}p{\isacharparenright}{\kern0pt}{\isacharparenright}{\kern0pt}{\isacharparenright}{\kern0pt}\ {\isasymin}\ formula{\isachardoublequoteclose}{\isacharparenright}{\kern0pt}\isanewline
\ \ \ \isacommand{apply}\isamarkupfalse%
{\isacharparenleft}{\kern0pt}rule\ pred{\isacharunderscore}{\kern0pt}le{\isacharcomma}{\kern0pt}\ force{\isacharcomma}{\kern0pt}\ simp{\isacharparenright}{\kern0pt}\isanewline
\ \ \ \isacommand{apply}\isamarkupfalse%
{\isacharparenleft}{\kern0pt}rule\ Un{\isacharunderscore}{\kern0pt}least{\isacharunderscore}{\kern0pt}lt{\isacharparenright}{\kern0pt}\isanewline
\ \ \ \ \isacommand{apply}\isamarkupfalse%
{\isacharparenleft}{\kern0pt}rule\ le{\isacharunderscore}{\kern0pt}trans{\isacharcomma}{\kern0pt}\ rule\ arity{\isacharunderscore}{\kern0pt}is{\isacharunderscore}{\kern0pt}HS{\isacharunderscore}{\kern0pt}fm{\isacharcomma}{\kern0pt}\ simp{\isacharcomma}{\kern0pt}\ simp{\isacharparenright}{\kern0pt}\isanewline
\ \ \ \ \isacommand{apply}\isamarkupfalse%
{\isacharparenleft}{\kern0pt}rule\ Un{\isacharunderscore}{\kern0pt}least{\isacharunderscore}{\kern0pt}lt{\isacharcomma}{\kern0pt}\ simp{\isacharcomma}{\kern0pt}\ simp{\isacharparenright}{\kern0pt}\isanewline
\ \ \ \isacommand{apply}\isamarkupfalse%
{\isacharparenleft}{\kern0pt}rule\ le{\isacharunderscore}{\kern0pt}trans{\isacharcomma}{\kern0pt}\ rule\ arity{\isacharunderscore}{\kern0pt}ren{\isacharcomma}{\kern0pt}\ simp{\isacharparenright}{\kern0pt}\isanewline
\ \ \ \ \ \ \ \isacommand{apply}\isamarkupfalse%
{\isacharparenleft}{\kern0pt}rule\ union{\isacharunderscore}{\kern0pt}in{\isacharunderscore}{\kern0pt}nat{\isacharcomma}{\kern0pt}\ simp{\isacharcomma}{\kern0pt}\ force{\isacharparenright}{\kern0pt}{\isacharplus}{\kern0pt}\isanewline
\ \ \ \ \ \isacommand{apply}\isamarkupfalse%
{\isacharparenleft}{\kern0pt}rule\ ren{\isacharunderscore}{\kern0pt}forcesHS{\isacharunderscore}{\kern0pt}forall{\isacharunderscore}{\kern0pt}type{\isacharparenright}{\kern0pt}\isanewline
\ \ \ \ \ \isacommand{apply}\isamarkupfalse%
\ force\isanewline
\ \ \ \ \isacommand{apply}\isamarkupfalse%
{\isacharparenleft}{\kern0pt}rule\ max{\isacharunderscore}{\kern0pt}le{\isadigit{2}}{\isacharcomma}{\kern0pt}\ simp{\isacharcomma}{\kern0pt}\ force{\isacharparenright}{\kern0pt}\isanewline
\ \ \ \isacommand{apply}\isamarkupfalse%
{\isacharparenleft}{\kern0pt}rule\ Un{\isacharunderscore}{\kern0pt}least{\isacharunderscore}{\kern0pt}lt{\isacharcomma}{\kern0pt}\ simp{\isacharparenright}{\kern0pt}\isanewline
\ \ \ \isacommand{apply}\isamarkupfalse%
{\isacharparenleft}{\kern0pt}rename{\isacharunderscore}{\kern0pt}tac\ p{\isacharcomma}{\kern0pt}\ rule{\isacharunderscore}{\kern0pt}tac\ n{\isacharequal}{\kern0pt}{\isachardoublequoteopen}arity{\isacharparenleft}{\kern0pt}p{\isacharparenright}{\kern0pt}{\isachardoublequoteclose}\ \isakeyword{in}\ natE{\isacharcomma}{\kern0pt}\ simp{\isacharcomma}{\kern0pt}\ simp{\isacharparenright}{\kern0pt}\isanewline
\ \ \ \ \isacommand{apply}\isamarkupfalse%
{\isacharparenleft}{\kern0pt}rule{\isacharunderscore}{\kern0pt}tac\ j{\isacharequal}{\kern0pt}{\isadigit{5}}\ \isakeyword{in}\ le{\isacharunderscore}{\kern0pt}trans{\isacharcomma}{\kern0pt}\ simp{\isacharcomma}{\kern0pt}\ simp{\isacharparenright}{\kern0pt}\isanewline
\ \ \ \isacommand{apply}\isamarkupfalse%
\ force\ \isanewline
\ \ \isacommand{apply}\isamarkupfalse%
{\isacharparenleft}{\kern0pt}rule\ conjI{\isacharcomma}{\kern0pt}\ rule\ forcesHS{\isacharprime}{\kern0pt}{\isacharunderscore}{\kern0pt}type{\isacharcomma}{\kern0pt}\ simp{\isacharcomma}{\kern0pt}\ rule\ conjI{\isacharcomma}{\kern0pt}\ rule\ is{\isacharunderscore}{\kern0pt}HS{\isacharunderscore}{\kern0pt}fm{\isacharunderscore}{\kern0pt}type{\isacharcomma}{\kern0pt}\ simp{\isacharcomma}{\kern0pt}\ simp{\isacharparenright}{\kern0pt}\isanewline
\ \ \isacommand{apply}\isamarkupfalse%
{\isacharparenleft}{\kern0pt}rule\ ren{\isacharunderscore}{\kern0pt}tc{\isacharcomma}{\kern0pt}\ rule\ forcesHS{\isacharprime}{\kern0pt}{\isacharunderscore}{\kern0pt}type{\isacharcomma}{\kern0pt}\ simp{\isacharparenright}{\kern0pt}\isanewline
\ \ \ \ \isacommand{apply}\isamarkupfalse%
{\isacharparenleft}{\kern0pt}rule\ union{\isacharunderscore}{\kern0pt}in{\isacharunderscore}{\kern0pt}nat{\isacharcomma}{\kern0pt}\ simp{\isacharcomma}{\kern0pt}\ rule\ arity{\isacharunderscore}{\kern0pt}type{\isacharcomma}{\kern0pt}\ rule\ forcesHS{\isacharprime}{\kern0pt}{\isacharunderscore}{\kern0pt}type{\isacharcomma}{\kern0pt}\ simp{\isacharparenright}{\kern0pt}{\isacharplus}{\kern0pt}\isanewline
\ \ \isacommand{apply}\isamarkupfalse%
{\isacharparenleft}{\kern0pt}rule\ ren{\isacharunderscore}{\kern0pt}forcesHS{\isacharunderscore}{\kern0pt}forall{\isacharunderscore}{\kern0pt}type{\isacharparenright}{\kern0pt}\isanewline
\ \ \isacommand{apply}\isamarkupfalse%
{\isacharparenleft}{\kern0pt}rule\ arity{\isacharunderscore}{\kern0pt}type{\isacharcomma}{\kern0pt}\ rule\ forcesHS{\isacharprime}{\kern0pt}{\isacharunderscore}{\kern0pt}type{\isacharcomma}{\kern0pt}\ simp{\isacharparenright}{\kern0pt}\isanewline
\ \ \isacommand{done}\isamarkupfalse%
%
\endisatagproof
{\isafoldproof}%
%
\isadelimproof
\isanewline
%
\endisadelimproof
\isanewline
\isacommand{lemma}\isamarkupfalse%
\ arity{\isacharunderscore}{\kern0pt}forcesHS\ {\isacharcolon}{\kern0pt}\ \isanewline
\ \ \isakeyword{fixes}\ {\isasymphi}\isanewline
\ \ \isakeyword{assumes}\ {\isachardoublequoteopen}{\isasymphi}\ {\isasymin}\ formula{\isachardoublequoteclose}\ \isanewline
\ \ \isakeyword{shows}\ {\isachardoublequoteopen}arity{\isacharparenleft}{\kern0pt}forcesHS{\isacharparenleft}{\kern0pt}{\isasymphi}{\isacharparenright}{\kern0pt}{\isacharparenright}{\kern0pt}\ {\isasymle}\ arity{\isacharparenleft}{\kern0pt}{\isasymphi}{\isacharparenright}{\kern0pt}\ {\isacharhash}{\kern0pt}{\isacharplus}{\kern0pt}\ {\isadigit{5}}{\isachardoublequoteclose}\ \isanewline
%
\isadelimproof
\ \ %
\endisadelimproof
%
\isatagproof
\isacommand{unfolding}\isamarkupfalse%
\ forcesHS{\isacharunderscore}{\kern0pt}def\ \isanewline
\ \ \isacommand{using}\isamarkupfalse%
\ forcesHS{\isacharprime}{\kern0pt}{\isacharunderscore}{\kern0pt}type\ assms\ \isanewline
\ \ \isacommand{apply}\isamarkupfalse%
\ simp\isanewline
\ \ \isacommand{apply}\isamarkupfalse%
{\isacharparenleft}{\kern0pt}rule\ Un{\isacharunderscore}{\kern0pt}least{\isacharunderscore}{\kern0pt}lt{\isacharparenright}{\kern0pt}{\isacharplus}{\kern0pt}\isanewline
\ \ \ \ \isacommand{apply}\isamarkupfalse%
\ auto{\isacharbrackleft}{\kern0pt}{\isadigit{2}}{\isacharbrackright}{\kern0pt}\isanewline
\ \ \isacommand{using}\isamarkupfalse%
\ arity{\isacharunderscore}{\kern0pt}forcesHS{\isacharprime}{\kern0pt}\ assms\isanewline
\ \ \isacommand{by}\isamarkupfalse%
\ auto%
\endisatagproof
{\isafoldproof}%
%
\isadelimproof
\isanewline
%
\endisadelimproof
\isanewline
\isacommand{lemma}\isamarkupfalse%
\ sats{\isacharunderscore}{\kern0pt}forcesHS{\isacharunderscore}{\kern0pt}Member\ {\isacharcolon}{\kern0pt}\isanewline
\ \ \isakeyword{assumes}\ \ {\isachardoublequoteopen}x{\isasymin}nat{\isachardoublequoteclose}\ {\isachardoublequoteopen}y{\isasymin}nat{\isachardoublequoteclose}\ {\isachardoublequoteopen}env{\isasymin}list{\isacharparenleft}{\kern0pt}M{\isacharparenright}{\kern0pt}{\isachardoublequoteclose}\ {\isachardoublequoteopen}A\ {\isasymin}\ M{\isachardoublequoteclose}\ {\isachardoublequoteopen}q{\isasymin}M{\isachardoublequoteclose}\isanewline
\ \ \isakeyword{shows}\ {\isachardoublequoteopen}sats{\isacharparenleft}{\kern0pt}M{\isacharcomma}{\kern0pt}forcesHS{\isacharparenleft}{\kern0pt}Member{\isacharparenleft}{\kern0pt}x{\isacharcomma}{\kern0pt}y{\isacharparenright}{\kern0pt}{\isacharparenright}{\kern0pt}{\isacharcomma}{\kern0pt}{\isacharbrackleft}{\kern0pt}q{\isacharcomma}{\kern0pt}P{\isacharcomma}{\kern0pt}leq{\isacharcomma}{\kern0pt}one{\isacharcomma}{\kern0pt}A{\isacharbrackright}{\kern0pt}{\isacharat}{\kern0pt}env{\isacharparenright}{\kern0pt}\ {\isasymlongleftrightarrow}\isanewline
\ \ \ \ \ \ \ \ \ sats{\isacharparenleft}{\kern0pt}M{\isacharcomma}{\kern0pt}forces{\isacharparenleft}{\kern0pt}Member{\isacharparenleft}{\kern0pt}x{\isacharcomma}{\kern0pt}y{\isacharparenright}{\kern0pt}{\isacharparenright}{\kern0pt}{\isacharcomma}{\kern0pt}{\isacharbrackleft}{\kern0pt}q{\isacharcomma}{\kern0pt}P{\isacharcomma}{\kern0pt}leq{\isacharcomma}{\kern0pt}one{\isacharbrackright}{\kern0pt}{\isacharat}{\kern0pt}env{\isacharparenright}{\kern0pt}{\isachardoublequoteclose}\ \isanewline
%
\isadelimproof
\ \ %
\endisadelimproof
%
\isatagproof
\isacommand{apply}\isamarkupfalse%
{\isacharparenleft}{\kern0pt}subgoal{\isacharunderscore}{\kern0pt}tac\ {\isachardoublequoteopen}P\ {\isasymin}\ M\ {\isasymand}\ leq\ {\isasymin}\ M\ {\isasymand}\ one\ {\isasymin}\ M{\isachardoublequoteclose}{\isacharparenright}{\kern0pt}\isanewline
\ \ \isacommand{unfolding}\isamarkupfalse%
\ forcesHS{\isacharunderscore}{\kern0pt}def\ forces{\isacharunderscore}{\kern0pt}def\ \isanewline
\ \ \isacommand{using}\isamarkupfalse%
\ assms\isanewline
\ \ \ \isacommand{apply}\isamarkupfalse%
\ simp\isanewline
\ \ \ \isacommand{apply}\isamarkupfalse%
{\isacharparenleft}{\kern0pt}rule\ iff{\isacharunderscore}{\kern0pt}conjI{\isadigit{2}}{\isacharcomma}{\kern0pt}\ simp{\isacharparenright}{\kern0pt}\isanewline
\ \ \ \isacommand{apply}\isamarkupfalse%
{\isacharparenleft}{\kern0pt}rule{\isacharunderscore}{\kern0pt}tac\ Q{\isacharequal}{\kern0pt}{\isachardoublequoteopen}M{\isacharcomma}{\kern0pt}\ {\isacharbrackleft}{\kern0pt}q{\isacharcomma}{\kern0pt}\ P{\isacharcomma}{\kern0pt}\ leq{\isacharcomma}{\kern0pt}\ one{\isacharbrackright}{\kern0pt}\ {\isacharat}{\kern0pt}\ Cons{\isacharparenleft}{\kern0pt}A{\isacharcomma}{\kern0pt}\ env{\isacharparenright}{\kern0pt}\ {\isasymTurnstile}\ \isanewline
\ \ \ \ \ \ \ \ \ \ \ \ \ \ \ \ \ \ \ \ \ \ incr{\isacharunderscore}{\kern0pt}bv{\isacharparenleft}{\kern0pt}forces{\isacharunderscore}{\kern0pt}mem{\isacharunderscore}{\kern0pt}fm{\isacharparenleft}{\kern0pt}{\isadigit{1}}{\isacharcomma}{\kern0pt}\ {\isadigit{2}}{\isacharcomma}{\kern0pt}\ {\isadigit{0}}{\isacharcomma}{\kern0pt}\ succ{\isacharparenleft}{\kern0pt}succ{\isacharparenleft}{\kern0pt}succ{\isacharparenleft}{\kern0pt}succ{\isacharparenleft}{\kern0pt}x{\isacharparenright}{\kern0pt}{\isacharparenright}{\kern0pt}{\isacharparenright}{\kern0pt}{\isacharparenright}{\kern0pt}{\isacharcomma}{\kern0pt}\ succ{\isacharparenleft}{\kern0pt}succ{\isacharparenleft}{\kern0pt}succ{\isacharparenleft}{\kern0pt}succ{\isacharparenleft}{\kern0pt}y{\isacharparenright}{\kern0pt}{\isacharparenright}{\kern0pt}{\isacharparenright}{\kern0pt}{\isacharparenright}{\kern0pt}{\isacharparenright}{\kern0pt}{\isacharparenright}{\kern0pt}\ {\isacharbackquote}{\kern0pt}\ length{\isacharparenleft}{\kern0pt}{\isacharbrackleft}{\kern0pt}q{\isacharcomma}{\kern0pt}\ P{\isacharcomma}{\kern0pt}\ leq{\isacharcomma}{\kern0pt}\ one{\isacharbrackright}{\kern0pt}{\isacharparenright}{\kern0pt}{\isachardoublequoteclose}\ \isakeyword{in}\ iff{\isacharunderscore}{\kern0pt}trans{\isacharparenright}{\kern0pt}\isanewline
\ \ \isacommand{apply}\isamarkupfalse%
\ force\ \isanewline
\ \ \ \isacommand{apply}\isamarkupfalse%
{\isacharparenleft}{\kern0pt}rule\ iff{\isacharunderscore}{\kern0pt}trans{\isacharcomma}{\kern0pt}\ rule\ sats{\isacharunderscore}{\kern0pt}incr{\isacharunderscore}{\kern0pt}bv{\isacharunderscore}{\kern0pt}iff{\isacharparenright}{\kern0pt}\isanewline
\ \ \ \ \ \ \ \isacommand{apply}\isamarkupfalse%
\ auto{\isacharbrackleft}{\kern0pt}{\isadigit{5}}{\isacharbrackright}{\kern0pt}\isanewline
\ \ \isacommand{using}\isamarkupfalse%
\ P{\isacharunderscore}{\kern0pt}in{\isacharunderscore}{\kern0pt}M\ leq{\isacharunderscore}{\kern0pt}in{\isacharunderscore}{\kern0pt}M\ one{\isacharunderscore}{\kern0pt}in{\isacharunderscore}{\kern0pt}M\isanewline
\ \ \isacommand{by}\isamarkupfalse%
\ auto%
\endisatagproof
{\isafoldproof}%
%
\isadelimproof
\isanewline
%
\endisadelimproof
\isanewline
\isacommand{lemma}\isamarkupfalse%
\ ForcesHS{\isacharunderscore}{\kern0pt}Member\ {\isacharcolon}{\kern0pt}\ \isanewline
\ \ \isakeyword{fixes}\ x\ y\ env\ p\isanewline
\ \ \isakeyword{assumes}\ {\isachardoublequoteopen}x{\isasymin}nat{\isachardoublequoteclose}\ {\isachardoublequoteopen}y{\isasymin}nat{\isachardoublequoteclose}\ {\isachardoublequoteopen}env{\isasymin}list{\isacharparenleft}{\kern0pt}M{\isacharparenright}{\kern0pt}{\isachardoublequoteclose}\ {\isachardoublequoteopen}p\ {\isasymin}\ M{\isachardoublequoteclose}\isanewline
\ \ \isakeyword{shows}\ {\isachardoublequoteopen}p\ {\isasymtturnstile}\ Member{\isacharparenleft}{\kern0pt}x{\isacharcomma}{\kern0pt}\ y{\isacharparenright}{\kern0pt}\ env\ {\isasymlongleftrightarrow}\ p\ {\isasymtturnstile}HS\ Member{\isacharparenleft}{\kern0pt}x{\isacharcomma}{\kern0pt}\ y{\isacharparenright}{\kern0pt}\ env{\isachardoublequoteclose}\ \isanewline
%
\isadelimproof
\ \ %
\endisadelimproof
%
\isatagproof
\isacommand{apply}\isamarkupfalse%
{\isacharparenleft}{\kern0pt}rule\ iff{\isacharunderscore}{\kern0pt}flip{\isacharcomma}{\kern0pt}\ rule\ sats{\isacharunderscore}{\kern0pt}forcesHS{\isacharunderscore}{\kern0pt}Member{\isacharparenright}{\kern0pt}\isanewline
\ \ \isacommand{using}\isamarkupfalse%
\ assms\isanewline
\ \ \ \ \ \ \isacommand{apply}\isamarkupfalse%
\ auto{\isacharbrackleft}{\kern0pt}{\isadigit{2}}{\isacharbrackright}{\kern0pt}\isanewline
\ \ \isacommand{using}\isamarkupfalse%
\ assms\ {\isasymF}{\isacharunderscore}{\kern0pt}in{\isacharunderscore}{\kern0pt}M\ {\isasymG}{\isacharunderscore}{\kern0pt}in{\isacharunderscore}{\kern0pt}M\ P{\isacharunderscore}{\kern0pt}auto{\isacharunderscore}{\kern0pt}in{\isacharunderscore}{\kern0pt}M\ P{\isacharunderscore}{\kern0pt}in{\isacharunderscore}{\kern0pt}M\ pair{\isacharunderscore}{\kern0pt}in{\isacharunderscore}{\kern0pt}M{\isacharunderscore}{\kern0pt}iff\isanewline
\ \ \isacommand{by}\isamarkupfalse%
\ auto%
\endisatagproof
{\isafoldproof}%
%
\isadelimproof
\isanewline
%
\endisadelimproof
\isanewline
\isacommand{lemma}\isamarkupfalse%
\ sats{\isacharunderscore}{\kern0pt}forcesHS{\isacharunderscore}{\kern0pt}Equal\ {\isacharcolon}{\kern0pt}\isanewline
\ \ \isakeyword{assumes}\ \ {\isachardoublequoteopen}x{\isasymin}nat{\isachardoublequoteclose}\ {\isachardoublequoteopen}y{\isasymin}nat{\isachardoublequoteclose}\ {\isachardoublequoteopen}env{\isasymin}list{\isacharparenleft}{\kern0pt}M{\isacharparenright}{\kern0pt}{\isachardoublequoteclose}\ {\isachardoublequoteopen}A\ {\isasymin}\ M{\isachardoublequoteclose}\ {\isachardoublequoteopen}q{\isasymin}M{\isachardoublequoteclose}\isanewline
\ \ \isakeyword{shows}\ {\isachardoublequoteopen}sats{\isacharparenleft}{\kern0pt}M{\isacharcomma}{\kern0pt}forcesHS{\isacharparenleft}{\kern0pt}Equal{\isacharparenleft}{\kern0pt}x{\isacharcomma}{\kern0pt}y{\isacharparenright}{\kern0pt}{\isacharparenright}{\kern0pt}{\isacharcomma}{\kern0pt}{\isacharbrackleft}{\kern0pt}q{\isacharcomma}{\kern0pt}P{\isacharcomma}{\kern0pt}leq{\isacharcomma}{\kern0pt}one{\isacharcomma}{\kern0pt}A{\isacharbrackright}{\kern0pt}{\isacharat}{\kern0pt}env{\isacharparenright}{\kern0pt}\ {\isasymlongleftrightarrow}\isanewline
\ \ \ \ \ \ \ \ \ sats{\isacharparenleft}{\kern0pt}M{\isacharcomma}{\kern0pt}forces{\isacharparenleft}{\kern0pt}Equal{\isacharparenleft}{\kern0pt}x{\isacharcomma}{\kern0pt}y{\isacharparenright}{\kern0pt}{\isacharparenright}{\kern0pt}{\isacharcomma}{\kern0pt}{\isacharbrackleft}{\kern0pt}q{\isacharcomma}{\kern0pt}P{\isacharcomma}{\kern0pt}leq{\isacharcomma}{\kern0pt}one{\isacharbrackright}{\kern0pt}{\isacharat}{\kern0pt}env{\isacharparenright}{\kern0pt}{\isachardoublequoteclose}\isanewline
%
\isadelimproof
\ \ %
\endisadelimproof
%
\isatagproof
\isacommand{apply}\isamarkupfalse%
{\isacharparenleft}{\kern0pt}subgoal{\isacharunderscore}{\kern0pt}tac\ {\isachardoublequoteopen}P\ {\isasymin}\ M\ {\isasymand}\ leq\ {\isasymin}\ M\ {\isasymand}\ one\ {\isasymin}\ M{\isachardoublequoteclose}{\isacharparenright}{\kern0pt}\isanewline
\ \ \isacommand{unfolding}\isamarkupfalse%
\ forcesHS{\isacharunderscore}{\kern0pt}def\ forces{\isacharunderscore}{\kern0pt}def\ \isanewline
\ \ \isacommand{using}\isamarkupfalse%
\ assms\isanewline
\ \ \ \isacommand{apply}\isamarkupfalse%
\ simp\isanewline
\ \ \ \isacommand{apply}\isamarkupfalse%
{\isacharparenleft}{\kern0pt}rule\ iff{\isacharunderscore}{\kern0pt}conjI{\isadigit{2}}{\isacharcomma}{\kern0pt}\ simp{\isacharparenright}{\kern0pt}\isanewline
\ \ \ \isacommand{apply}\isamarkupfalse%
{\isacharparenleft}{\kern0pt}rule{\isacharunderscore}{\kern0pt}tac\ Q{\isacharequal}{\kern0pt}{\isachardoublequoteopen}M{\isacharcomma}{\kern0pt}\ {\isacharbrackleft}{\kern0pt}q{\isacharcomma}{\kern0pt}\ P{\isacharcomma}{\kern0pt}\ leq{\isacharcomma}{\kern0pt}\ one{\isacharbrackright}{\kern0pt}\ {\isacharat}{\kern0pt}\ Cons{\isacharparenleft}{\kern0pt}A{\isacharcomma}{\kern0pt}\ env{\isacharparenright}{\kern0pt}\ {\isasymTurnstile}\ \isanewline
\ \ \ \ \ \ \ \ \ \ \ \ \ \ \ \ \ \ \ \ \ \ incr{\isacharunderscore}{\kern0pt}bv{\isacharparenleft}{\kern0pt}forces{\isacharunderscore}{\kern0pt}eq{\isacharunderscore}{\kern0pt}fm{\isacharparenleft}{\kern0pt}{\isadigit{1}}{\isacharcomma}{\kern0pt}\ {\isadigit{2}}{\isacharcomma}{\kern0pt}\ {\isadigit{0}}{\isacharcomma}{\kern0pt}\ succ{\isacharparenleft}{\kern0pt}succ{\isacharparenleft}{\kern0pt}succ{\isacharparenleft}{\kern0pt}succ{\isacharparenleft}{\kern0pt}x{\isacharparenright}{\kern0pt}{\isacharparenright}{\kern0pt}{\isacharparenright}{\kern0pt}{\isacharparenright}{\kern0pt}{\isacharcomma}{\kern0pt}\ succ{\isacharparenleft}{\kern0pt}succ{\isacharparenleft}{\kern0pt}succ{\isacharparenleft}{\kern0pt}succ{\isacharparenleft}{\kern0pt}y{\isacharparenright}{\kern0pt}{\isacharparenright}{\kern0pt}{\isacharparenright}{\kern0pt}{\isacharparenright}{\kern0pt}{\isacharparenright}{\kern0pt}{\isacharparenright}{\kern0pt}\ {\isacharbackquote}{\kern0pt}\ length{\isacharparenleft}{\kern0pt}{\isacharbrackleft}{\kern0pt}q{\isacharcomma}{\kern0pt}\ P{\isacharcomma}{\kern0pt}\ leq{\isacharcomma}{\kern0pt}\ one{\isacharbrackright}{\kern0pt}{\isacharparenright}{\kern0pt}{\isachardoublequoteclose}\ \isakeyword{in}\ iff{\isacharunderscore}{\kern0pt}trans{\isacharparenright}{\kern0pt}\isanewline
\ \ \ \ \isacommand{apply}\isamarkupfalse%
\ force\ \isanewline
\ \ \ \isacommand{apply}\isamarkupfalse%
{\isacharparenleft}{\kern0pt}rule\ iff{\isacharunderscore}{\kern0pt}trans{\isacharcomma}{\kern0pt}\ rule\ sats{\isacharunderscore}{\kern0pt}incr{\isacharunderscore}{\kern0pt}bv{\isacharunderscore}{\kern0pt}iff{\isacharparenright}{\kern0pt}\isanewline
\ \ \ \ \ \ \ \isacommand{apply}\isamarkupfalse%
\ auto{\isacharbrackleft}{\kern0pt}{\isadigit{5}}{\isacharbrackright}{\kern0pt}\isanewline
\ \ \isacommand{using}\isamarkupfalse%
\ P{\isacharunderscore}{\kern0pt}in{\isacharunderscore}{\kern0pt}M\ leq{\isacharunderscore}{\kern0pt}in{\isacharunderscore}{\kern0pt}M\ one{\isacharunderscore}{\kern0pt}in{\isacharunderscore}{\kern0pt}M\isanewline
\ \ \isacommand{by}\isamarkupfalse%
\ auto%
\endisatagproof
{\isafoldproof}%
%
\isadelimproof
\isanewline
%
\endisadelimproof
\isanewline
\isacommand{lemma}\isamarkupfalse%
\ ForcesHS{\isacharunderscore}{\kern0pt}Equal\ {\isacharcolon}{\kern0pt}\ \isanewline
\ \ \isakeyword{fixes}\ x\ y\ env\ p\isanewline
\ \ \isakeyword{assumes}\ {\isachardoublequoteopen}x{\isasymin}nat{\isachardoublequoteclose}\ {\isachardoublequoteopen}y{\isasymin}nat{\isachardoublequoteclose}\ {\isachardoublequoteopen}env{\isasymin}list{\isacharparenleft}{\kern0pt}M{\isacharparenright}{\kern0pt}{\isachardoublequoteclose}\ {\isachardoublequoteopen}p\ {\isasymin}\ M{\isachardoublequoteclose}\isanewline
\ \ \isakeyword{shows}\ {\isachardoublequoteopen}p\ {\isasymtturnstile}\ Equal{\isacharparenleft}{\kern0pt}x{\isacharcomma}{\kern0pt}\ y{\isacharparenright}{\kern0pt}\ env\ {\isasymlongleftrightarrow}\ p\ {\isasymtturnstile}HS\ Equal{\isacharparenleft}{\kern0pt}x{\isacharcomma}{\kern0pt}\ y{\isacharparenright}{\kern0pt}\ env{\isachardoublequoteclose}\ \isanewline
%
\isadelimproof
\ \ %
\endisadelimproof
%
\isatagproof
\isacommand{apply}\isamarkupfalse%
{\isacharparenleft}{\kern0pt}rule\ iff{\isacharunderscore}{\kern0pt}flip{\isacharcomma}{\kern0pt}\ rule\ sats{\isacharunderscore}{\kern0pt}forcesHS{\isacharunderscore}{\kern0pt}Equal{\isacharparenright}{\kern0pt}\isanewline
\ \ \isacommand{using}\isamarkupfalse%
\ assms\isanewline
\ \ \ \ \ \ \isacommand{apply}\isamarkupfalse%
\ auto{\isacharbrackleft}{\kern0pt}{\isadigit{2}}{\isacharbrackright}{\kern0pt}\isanewline
\ \ \isacommand{using}\isamarkupfalse%
\ assms\ {\isasymF}{\isacharunderscore}{\kern0pt}in{\isacharunderscore}{\kern0pt}M\ {\isasymG}{\isacharunderscore}{\kern0pt}in{\isacharunderscore}{\kern0pt}M\ P{\isacharunderscore}{\kern0pt}auto{\isacharunderscore}{\kern0pt}in{\isacharunderscore}{\kern0pt}M\ P{\isacharunderscore}{\kern0pt}in{\isacharunderscore}{\kern0pt}M\ pair{\isacharunderscore}{\kern0pt}in{\isacharunderscore}{\kern0pt}M{\isacharunderscore}{\kern0pt}iff\isanewline
\ \ \isacommand{by}\isamarkupfalse%
\ auto%
\endisatagproof
{\isafoldproof}%
%
\isadelimproof
\isanewline
%
\endisadelimproof
\isanewline
\isacommand{lemma}\isamarkupfalse%
\ sats{\isacharunderscore}{\kern0pt}forcesHS{\isacharunderscore}{\kern0pt}Nand\ {\isacharcolon}{\kern0pt}\isanewline
\ \ \isakeyword{assumes}\ \ {\isachardoublequoteopen}{\isasymphi}{\isasymin}formula{\isachardoublequoteclose}\ {\isachardoublequoteopen}{\isasympsi}{\isasymin}formula{\isachardoublequoteclose}\ {\isachardoublequoteopen}env{\isasymin}list{\isacharparenleft}{\kern0pt}M{\isacharparenright}{\kern0pt}{\isachardoublequoteclose}\ {\isachardoublequoteopen}p{\isasymin}M{\isachardoublequoteclose}\ {\isachardoublequoteopen}A\ {\isasymin}\ M{\isachardoublequoteclose}\ \isanewline
\ \ \isakeyword{shows}\ {\isachardoublequoteopen}sats{\isacharparenleft}{\kern0pt}M{\isacharcomma}{\kern0pt}forcesHS{\isacharparenleft}{\kern0pt}Nand{\isacharparenleft}{\kern0pt}{\isasymphi}{\isacharcomma}{\kern0pt}{\isasympsi}{\isacharparenright}{\kern0pt}{\isacharparenright}{\kern0pt}{\isacharcomma}{\kern0pt}{\isacharbrackleft}{\kern0pt}p{\isacharcomma}{\kern0pt}P{\isacharcomma}{\kern0pt}leq{\isacharcomma}{\kern0pt}one{\isacharcomma}{\kern0pt}A{\isacharbrackright}{\kern0pt}{\isacharat}{\kern0pt}env{\isacharparenright}{\kern0pt}\ {\isasymlongleftrightarrow}\isanewline
\ \ \ \ \ \ \ \ \ {\isacharparenleft}{\kern0pt}p{\isasymin}P\ {\isasymand}\ {\isasymnot}{\isacharparenleft}{\kern0pt}{\isasymexists}q{\isasymin}M{\isachardot}{\kern0pt}\ q{\isasymin}P\ {\isasymand}\ is{\isacharunderscore}{\kern0pt}leq{\isacharparenleft}{\kern0pt}{\isacharhash}{\kern0pt}{\isacharhash}{\kern0pt}M{\isacharcomma}{\kern0pt}leq{\isacharcomma}{\kern0pt}q{\isacharcomma}{\kern0pt}p{\isacharparenright}{\kern0pt}\ {\isasymand}\isanewline
\ \ \ \ \ \ \ \ \ \ \ \ \ \ \ {\isacharparenleft}{\kern0pt}sats{\isacharparenleft}{\kern0pt}M{\isacharcomma}{\kern0pt}forcesHS{\isacharprime}{\kern0pt}{\isacharparenleft}{\kern0pt}{\isasymphi}{\isacharparenright}{\kern0pt}{\isacharcomma}{\kern0pt}{\isacharbrackleft}{\kern0pt}q{\isacharcomma}{\kern0pt}P{\isacharcomma}{\kern0pt}leq{\isacharcomma}{\kern0pt}one{\isacharcomma}{\kern0pt}A{\isacharbrackright}{\kern0pt}{\isacharat}{\kern0pt}env{\isacharparenright}{\kern0pt}\ {\isasymand}\ sats{\isacharparenleft}{\kern0pt}M{\isacharcomma}{\kern0pt}forcesHS{\isacharprime}{\kern0pt}{\isacharparenleft}{\kern0pt}{\isasympsi}{\isacharparenright}{\kern0pt}{\isacharcomma}{\kern0pt}{\isacharbrackleft}{\kern0pt}q{\isacharcomma}{\kern0pt}P{\isacharcomma}{\kern0pt}leq{\isacharcomma}{\kern0pt}one{\isacharcomma}{\kern0pt}A{\isacharbrackright}{\kern0pt}{\isacharat}{\kern0pt}env{\isacharparenright}{\kern0pt}{\isacharparenright}{\kern0pt}{\isacharparenright}{\kern0pt}{\isacharparenright}{\kern0pt}{\isachardoublequoteclose}\isanewline
%
\isadelimproof
\ \ %
\endisadelimproof
%
\isatagproof
\isacommand{unfolding}\isamarkupfalse%
\ forcesHS{\isacharunderscore}{\kern0pt}def\ \isacommand{using}\isamarkupfalse%
\ sats{\isacharunderscore}{\kern0pt}leq{\isacharunderscore}{\kern0pt}fm\ assms\ P{\isacharunderscore}{\kern0pt}in{\isacharunderscore}{\kern0pt}M\ leq{\isacharunderscore}{\kern0pt}in{\isacharunderscore}{\kern0pt}M\ one{\isacharunderscore}{\kern0pt}in{\isacharunderscore}{\kern0pt}M\isanewline
\ \ \isacommand{apply}\isamarkupfalse%
\ simp\ \isanewline
\ \ \isacommand{apply}\isamarkupfalse%
{\isacharparenleft}{\kern0pt}rule\ iff{\isacharunderscore}{\kern0pt}conjI{\isadigit{2}}{\isacharcomma}{\kern0pt}\ simp{\isacharparenright}{\kern0pt}\isanewline
\ \ \isacommand{apply}\isamarkupfalse%
{\isacharparenleft}{\kern0pt}rule\ ball{\isacharunderscore}{\kern0pt}iff{\isacharcomma}{\kern0pt}\ rule\ iff{\isacharunderscore}{\kern0pt}disjI{\isacharcomma}{\kern0pt}\ simp{\isacharcomma}{\kern0pt}\ rule\ iff{\isacharunderscore}{\kern0pt}disjI{\isacharcomma}{\kern0pt}\ simp{\isacharcomma}{\kern0pt}\ rule\ iff{\isacharunderscore}{\kern0pt}disjI{\isacharparenright}{\kern0pt}\isanewline
\ \ \ \isacommand{apply}\isamarkupfalse%
{\isacharparenleft}{\kern0pt}rule\ notnot{\isacharunderscore}{\kern0pt}iff{\isacharparenright}{\kern0pt}\isanewline
\ \ \isacommand{apply}\isamarkupfalse%
{\isacharparenleft}{\kern0pt}rename{\isacharunderscore}{\kern0pt}tac\ x{\isacharcomma}{\kern0pt}\ subgoal{\isacharunderscore}{\kern0pt}tac\ {\isachardoublequoteopen}M{\isacharcomma}{\kern0pt}\ {\isacharbrackleft}{\kern0pt}x{\isacharcomma}{\kern0pt}\ p{\isacharcomma}{\kern0pt}\ P{\isacharcomma}{\kern0pt}\ leq{\isacharcomma}{\kern0pt}\ one{\isacharbrackright}{\kern0pt}\ {\isacharat}{\kern0pt}\ {\isacharparenleft}{\kern0pt}{\isacharbrackleft}{\kern0pt}A{\isacharbrackright}{\kern0pt}\ {\isacharat}{\kern0pt}\ env{\isacharparenright}{\kern0pt}\ {\isasymTurnstile}\ ren{\isacharunderscore}{\kern0pt}forces{\isacharunderscore}{\kern0pt}nand{\isacharparenleft}{\kern0pt}forcesHS{\isacharprime}{\kern0pt}{\isacharparenleft}{\kern0pt}{\isasymphi}{\isacharparenright}{\kern0pt}{\isacharparenright}{\kern0pt}\ {\isasymlongleftrightarrow}\isanewline
\ \ \ \ \ \ \ \ \ \ \ \ \ \ \ \ \ \ \ \ \ \ \ \ \ \ \ \ \ \ \ \ \ \ \ M{\isacharcomma}{\kern0pt}\ {\isacharbrackleft}{\kern0pt}x{\isacharcomma}{\kern0pt}\ P{\isacharcomma}{\kern0pt}\ leq{\isacharcomma}{\kern0pt}\ one{\isacharbrackright}{\kern0pt}\ {\isacharat}{\kern0pt}\ {\isacharparenleft}{\kern0pt}{\isacharbrackleft}{\kern0pt}A{\isacharbrackright}{\kern0pt}\ {\isacharat}{\kern0pt}\ env{\isacharparenright}{\kern0pt}\ {\isasymTurnstile}\ forcesHS{\isacharprime}{\kern0pt}{\isacharparenleft}{\kern0pt}{\isasymphi}{\isacharparenright}{\kern0pt}{\isachardoublequoteclose}{\isacharparenright}{\kern0pt}\isanewline
\ \ \ \ \isacommand{apply}\isamarkupfalse%
\ force\ \isanewline
\ \ \ \isacommand{apply}\isamarkupfalse%
{\isacharparenleft}{\kern0pt}rule\ sats{\isacharunderscore}{\kern0pt}ren{\isacharunderscore}{\kern0pt}forces{\isacharunderscore}{\kern0pt}nand{\isacharcomma}{\kern0pt}\ force{\isacharcomma}{\kern0pt}\ rule\ forcesHS{\isacharprime}{\kern0pt}{\isacharunderscore}{\kern0pt}type{\isacharcomma}{\kern0pt}\ simp{\isacharparenright}{\kern0pt}\isanewline
\ \ \ \isacommand{apply}\isamarkupfalse%
{\isacharparenleft}{\kern0pt}rule\ notnot{\isacharunderscore}{\kern0pt}iff{\isacharparenright}{\kern0pt}\isanewline
\ \ \isacommand{apply}\isamarkupfalse%
{\isacharparenleft}{\kern0pt}rename{\isacharunderscore}{\kern0pt}tac\ x{\isacharcomma}{\kern0pt}\ subgoal{\isacharunderscore}{\kern0pt}tac\ {\isachardoublequoteopen}M{\isacharcomma}{\kern0pt}\ {\isacharbrackleft}{\kern0pt}x{\isacharcomma}{\kern0pt}\ p{\isacharcomma}{\kern0pt}\ P{\isacharcomma}{\kern0pt}\ leq{\isacharcomma}{\kern0pt}\ one{\isacharbrackright}{\kern0pt}\ {\isacharat}{\kern0pt}\ {\isacharparenleft}{\kern0pt}{\isacharbrackleft}{\kern0pt}A{\isacharbrackright}{\kern0pt}\ {\isacharat}{\kern0pt}\ env{\isacharparenright}{\kern0pt}\ {\isasymTurnstile}\ ren{\isacharunderscore}{\kern0pt}forces{\isacharunderscore}{\kern0pt}nand{\isacharparenleft}{\kern0pt}forcesHS{\isacharprime}{\kern0pt}{\isacharparenleft}{\kern0pt}{\isasympsi}{\isacharparenright}{\kern0pt}{\isacharparenright}{\kern0pt}\ {\isasymlongleftrightarrow}\isanewline
\ \ \ \ \ \ \ \ \ \ \ \ \ \ \ \ \ \ \ \ \ \ \ \ \ \ \ \ \ \ \ \ \ \ \ M{\isacharcomma}{\kern0pt}\ {\isacharbrackleft}{\kern0pt}x{\isacharcomma}{\kern0pt}\ P{\isacharcomma}{\kern0pt}\ leq{\isacharcomma}{\kern0pt}\ one{\isacharbrackright}{\kern0pt}\ {\isacharat}{\kern0pt}\ {\isacharparenleft}{\kern0pt}{\isacharbrackleft}{\kern0pt}A{\isacharbrackright}{\kern0pt}\ {\isacharat}{\kern0pt}\ env{\isacharparenright}{\kern0pt}\ {\isasymTurnstile}\ forcesHS{\isacharprime}{\kern0pt}{\isacharparenleft}{\kern0pt}{\isasympsi}{\isacharparenright}{\kern0pt}{\isachardoublequoteclose}{\isacharparenright}{\kern0pt}\isanewline
\ \ \ \ \isacommand{apply}\isamarkupfalse%
\ force\ \isanewline
\ \ \isacommand{apply}\isamarkupfalse%
{\isacharparenleft}{\kern0pt}rule\ sats{\isacharunderscore}{\kern0pt}ren{\isacharunderscore}{\kern0pt}forces{\isacharunderscore}{\kern0pt}nand{\isacharcomma}{\kern0pt}\ force{\isacharcomma}{\kern0pt}\ rule\ forcesHS{\isacharprime}{\kern0pt}{\isacharunderscore}{\kern0pt}type{\isacharcomma}{\kern0pt}\ simp{\isacharparenright}{\kern0pt}\isanewline
\ \ \isacommand{done}\isamarkupfalse%
%
\endisatagproof
{\isafoldproof}%
%
\isadelimproof
\isanewline
%
\endisadelimproof
\isanewline
\isacommand{lemma}\isamarkupfalse%
\ sats{\isacharunderscore}{\kern0pt}forcesHS{\isacharunderscore}{\kern0pt}Nand{\isacharprime}{\kern0pt}{\isacharcolon}{\kern0pt}\isanewline
\ \ \isakeyword{assumes}\isanewline
\ \ \ \ {\isachardoublequoteopen}p{\isasymin}P{\isachardoublequoteclose}\ {\isachardoublequoteopen}{\isasymphi}{\isasymin}formula{\isachardoublequoteclose}\ {\isachardoublequoteopen}{\isasympsi}{\isasymin}formula{\isachardoublequoteclose}\ {\isachardoublequoteopen}env\ {\isasymin}\ list{\isacharparenleft}{\kern0pt}M{\isacharparenright}{\kern0pt}{\isachardoublequoteclose}\ \isanewline
\ \ \isakeyword{shows}\isanewline
\ \ \ \ {\isachardoublequoteopen}M{\isacharcomma}{\kern0pt}\ {\isacharbrackleft}{\kern0pt}p{\isacharcomma}{\kern0pt}P{\isacharcomma}{\kern0pt}leq{\isacharcomma}{\kern0pt}one{\isacharcomma}{\kern0pt}{\isacharless}{\kern0pt}{\isasymF}{\isacharcomma}{\kern0pt}\ {\isasymG}{\isacharcomma}{\kern0pt}\ P{\isacharcomma}{\kern0pt}\ P{\isacharunderscore}{\kern0pt}auto{\isachargreater}{\kern0pt}{\isacharbrackright}{\kern0pt}\ {\isacharat}{\kern0pt}\ env\ {\isasymTurnstile}\ forcesHS{\isacharparenleft}{\kern0pt}Nand{\isacharparenleft}{\kern0pt}{\isasymphi}{\isacharcomma}{\kern0pt}{\isasympsi}{\isacharparenright}{\kern0pt}{\isacharparenright}{\kern0pt}\ {\isasymlongleftrightarrow}\ \isanewline
\ \ \ \ \ {\isasymnot}{\isacharparenleft}{\kern0pt}{\isasymexists}q{\isasymin}M{\isachardot}{\kern0pt}\ q{\isasymin}P\ {\isasymand}\ is{\isacharunderscore}{\kern0pt}leq{\isacharparenleft}{\kern0pt}{\isacharhash}{\kern0pt}{\isacharhash}{\kern0pt}M{\isacharcomma}{\kern0pt}leq{\isacharcomma}{\kern0pt}q{\isacharcomma}{\kern0pt}p{\isacharparenright}{\kern0pt}\ {\isasymand}\ \isanewline
\ \ \ \ \ \ \ \ \ \ \ M{\isacharcomma}{\kern0pt}\ {\isacharbrackleft}{\kern0pt}q{\isacharcomma}{\kern0pt}P{\isacharcomma}{\kern0pt}leq{\isacharcomma}{\kern0pt}one{\isacharcomma}{\kern0pt}{\isacharless}{\kern0pt}{\isasymF}{\isacharcomma}{\kern0pt}\ {\isasymG}{\isacharcomma}{\kern0pt}\ P{\isacharcomma}{\kern0pt}\ P{\isacharunderscore}{\kern0pt}auto{\isachargreater}{\kern0pt}{\isacharbrackright}{\kern0pt}\ {\isacharat}{\kern0pt}\ env\ {\isasymTurnstile}\ forcesHS{\isacharparenleft}{\kern0pt}{\isasymphi}{\isacharparenright}{\kern0pt}\ {\isasymand}\ \isanewline
\ \ \ \ \ \ \ \ \ \ \ M{\isacharcomma}{\kern0pt}\ {\isacharbrackleft}{\kern0pt}q{\isacharcomma}{\kern0pt}P{\isacharcomma}{\kern0pt}leq{\isacharcomma}{\kern0pt}one{\isacharcomma}{\kern0pt}{\isacharless}{\kern0pt}{\isasymF}{\isacharcomma}{\kern0pt}\ {\isasymG}{\isacharcomma}{\kern0pt}\ P{\isacharcomma}{\kern0pt}\ P{\isacharunderscore}{\kern0pt}auto{\isachargreater}{\kern0pt}{\isacharbrackright}{\kern0pt}\ {\isacharat}{\kern0pt}\ env\ {\isasymTurnstile}\ forcesHS{\isacharparenleft}{\kern0pt}{\isasympsi}{\isacharparenright}{\kern0pt}{\isacharparenright}{\kern0pt}{\isachardoublequoteclose}\isanewline
%
\isadelimproof
\ \ %
\endisadelimproof
%
\isatagproof
\isacommand{apply}\isamarkupfalse%
{\isacharparenleft}{\kern0pt}subgoal{\isacharunderscore}{\kern0pt}tac\ {\isachardoublequoteopen}P\ {\isasymin}\ M\ {\isasymand}\ leq\ {\isasymin}\ M\ {\isasymand}\ one\ {\isasymin}\ M\ {\isasymand}\ {\isacharless}{\kern0pt}{\isasymF}{\isacharcomma}{\kern0pt}\ {\isasymG}{\isacharcomma}{\kern0pt}\ P{\isacharcomma}{\kern0pt}\ P{\isacharunderscore}{\kern0pt}auto{\isachargreater}{\kern0pt}\ {\isasymin}\ M{\isachardoublequoteclose}{\isacharparenright}{\kern0pt}\isanewline
\ \ \isacommand{using}\isamarkupfalse%
\ assms\ sats{\isacharunderscore}{\kern0pt}forcesHS{\isacharunderscore}{\kern0pt}Nand{\isacharbrackleft}{\kern0pt}OF\ assms{\isacharparenleft}{\kern0pt}{\isadigit{2}}{\isacharminus}{\kern0pt}{\isadigit{4}}{\isacharparenright}{\kern0pt}\ transitivity{\isacharbrackleft}{\kern0pt}OF\ {\isacartoucheopen}p{\isasymin}P{\isacartoucheclose}{\isacharbrackright}{\kern0pt}{\isacharbrackright}{\kern0pt}\ P{\isacharunderscore}{\kern0pt}in{\isacharunderscore}{\kern0pt}M\ leq{\isacharunderscore}{\kern0pt}in{\isacharunderscore}{\kern0pt}M\ one{\isacharunderscore}{\kern0pt}in{\isacharunderscore}{\kern0pt}M\ \isanewline
\ \ \isacommand{unfolding}\isamarkupfalse%
\ forcesHS{\isacharunderscore}{\kern0pt}def\isanewline
\ \ \ \isacommand{apply}\isamarkupfalse%
\ simp\isanewline
\ \ \isacommand{using}\isamarkupfalse%
\ P{\isacharunderscore}{\kern0pt}in{\isacharunderscore}{\kern0pt}M\ leq{\isacharunderscore}{\kern0pt}in{\isacharunderscore}{\kern0pt}M\ one{\isacharunderscore}{\kern0pt}in{\isacharunderscore}{\kern0pt}M\ {\isasymF}{\isacharunderscore}{\kern0pt}in{\isacharunderscore}{\kern0pt}M\ {\isasymG}{\isacharunderscore}{\kern0pt}in{\isacharunderscore}{\kern0pt}M\ P{\isacharunderscore}{\kern0pt}auto{\isacharunderscore}{\kern0pt}in{\isacharunderscore}{\kern0pt}M\ pair{\isacharunderscore}{\kern0pt}in{\isacharunderscore}{\kern0pt}M{\isacharunderscore}{\kern0pt}iff\isanewline
\ \ \isacommand{by}\isamarkupfalse%
\ auto%
\endisatagproof
{\isafoldproof}%
%
\isadelimproof
\isanewline
%
\endisadelimproof
\isanewline
\isacommand{lemma}\isamarkupfalse%
\ ForcesHS{\isacharunderscore}{\kern0pt}Nand{\isacharcolon}{\kern0pt}\isanewline
\ \ \isakeyword{assumes}\isanewline
\ \ \ \ {\isachardoublequoteopen}p{\isasymin}P{\isachardoublequoteclose}\ {\isachardoublequoteopen}env\ {\isasymin}\ list{\isacharparenleft}{\kern0pt}M{\isacharparenright}{\kern0pt}{\isachardoublequoteclose}\ {\isachardoublequoteopen}{\isasymphi}{\isasymin}formula{\isachardoublequoteclose}\ {\isachardoublequoteopen}{\isasympsi}{\isasymin}formula{\isachardoublequoteclose}\isanewline
\ \ \isakeyword{shows}\isanewline
\ \ \ \ {\isachardoublequoteopen}{\isacharparenleft}{\kern0pt}p\ {\isasymtturnstile}HS\ Nand{\isacharparenleft}{\kern0pt}{\isasymphi}{\isacharcomma}{\kern0pt}{\isasympsi}{\isacharparenright}{\kern0pt}\ env{\isacharparenright}{\kern0pt}\ {\isasymlongleftrightarrow}\ {\isasymnot}{\isacharparenleft}{\kern0pt}{\isasymexists}q{\isasymin}M{\isachardot}{\kern0pt}\ q{\isasymin}P\ {\isasymand}\ q{\isasympreceq}p\ {\isasymand}\ {\isacharparenleft}{\kern0pt}q\ {\isasymtturnstile}HS\ {\isasymphi}\ env{\isacharparenright}{\kern0pt}\ {\isasymand}\ {\isacharparenleft}{\kern0pt}q\ {\isasymtturnstile}HS\ {\isasympsi}\ env{\isacharparenright}{\kern0pt}{\isacharparenright}{\kern0pt}{\isachardoublequoteclose}\isanewline
%
\isadelimproof
\ \ %
\endisadelimproof
%
\isatagproof
\isacommand{using}\isamarkupfalse%
\ assms\ sats{\isacharunderscore}{\kern0pt}forcesHS{\isacharunderscore}{\kern0pt}Nand{\isacharprime}{\kern0pt}\ transitivity\ P{\isacharunderscore}{\kern0pt}in{\isacharunderscore}{\kern0pt}M\ pair{\isacharunderscore}{\kern0pt}in{\isacharunderscore}{\kern0pt}M{\isacharunderscore}{\kern0pt}iff\ leq{\isacharunderscore}{\kern0pt}in{\isacharunderscore}{\kern0pt}M\ leq{\isacharunderscore}{\kern0pt}abs\ \isanewline
\ \ \isacommand{by}\isamarkupfalse%
\ simp%
\endisatagproof
{\isafoldproof}%
%
\isadelimproof
\isanewline
%
\endisadelimproof
\isanewline
\isacommand{lemma}\isamarkupfalse%
\ ForcesHS{\isacharunderscore}{\kern0pt}Neg{\isacharcolon}{\kern0pt}\isanewline
\ \ \isakeyword{assumes}\isanewline
\ \ \ \ {\isachardoublequoteopen}p{\isasymin}P{\isachardoublequoteclose}\ {\isachardoublequoteopen}env\ {\isasymin}\ list{\isacharparenleft}{\kern0pt}M{\isacharparenright}{\kern0pt}{\isachardoublequoteclose}\ {\isachardoublequoteopen}{\isasymphi}{\isasymin}formula{\isachardoublequoteclose}\ \isanewline
\ \ \isakeyword{shows}\isanewline
\ \ \ \ {\isachardoublequoteopen}{\isacharparenleft}{\kern0pt}p\ {\isasymtturnstile}HS\ Neg{\isacharparenleft}{\kern0pt}{\isasymphi}{\isacharparenright}{\kern0pt}\ env{\isacharparenright}{\kern0pt}\ {\isasymlongleftrightarrow}\ {\isasymnot}{\isacharparenleft}{\kern0pt}{\isasymexists}q{\isasymin}M{\isachardot}{\kern0pt}\ q{\isasymin}P\ {\isasymand}\ q{\isasympreceq}p\ {\isasymand}\ {\isacharparenleft}{\kern0pt}q\ {\isasymtturnstile}HS\ {\isasymphi}\ env{\isacharparenright}{\kern0pt}{\isacharparenright}{\kern0pt}{\isachardoublequoteclose}\isanewline
%
\isadelimproof
\ \ %
\endisadelimproof
%
\isatagproof
\isacommand{unfolding}\isamarkupfalse%
\ Neg{\isacharunderscore}{\kern0pt}def\isanewline
\ \ \isacommand{using}\isamarkupfalse%
\ assms\ sats{\isacharunderscore}{\kern0pt}forcesHS{\isacharunderscore}{\kern0pt}Nand{\isacharprime}{\kern0pt}\ transitivity\ P{\isacharunderscore}{\kern0pt}in{\isacharunderscore}{\kern0pt}M\ pair{\isacharunderscore}{\kern0pt}in{\isacharunderscore}{\kern0pt}M{\isacharunderscore}{\kern0pt}iff\ leq{\isacharunderscore}{\kern0pt}in{\isacharunderscore}{\kern0pt}M\ leq{\isacharunderscore}{\kern0pt}abs\isanewline
\ \ \isacommand{by}\isamarkupfalse%
\ simp%
\endisatagproof
{\isafoldproof}%
%
\isadelimproof
\isanewline
%
\endisadelimproof
\isanewline
\isacommand{lemma}\isamarkupfalse%
\ sats{\isacharunderscore}{\kern0pt}forcesHS{\isacharunderscore}{\kern0pt}Forall\ {\isacharcolon}{\kern0pt}\isanewline
\ \ \isakeyword{assumes}\ \ {\isachardoublequoteopen}{\isasymphi}{\isasymin}formula{\isachardoublequoteclose}\ {\isachardoublequoteopen}env{\isasymin}list{\isacharparenleft}{\kern0pt}M{\isacharparenright}{\kern0pt}{\isachardoublequoteclose}\ {\isachardoublequoteopen}p{\isasymin}M{\isachardoublequoteclose}\ \ \isanewline
\ \ \isakeyword{shows}\ {\isachardoublequoteopen}sats{\isacharparenleft}{\kern0pt}M{\isacharcomma}{\kern0pt}forcesHS{\isacharparenleft}{\kern0pt}Forall{\isacharparenleft}{\kern0pt}{\isasymphi}{\isacharparenright}{\kern0pt}{\isacharparenright}{\kern0pt}{\isacharcomma}{\kern0pt}{\isacharbrackleft}{\kern0pt}p{\isacharcomma}{\kern0pt}P{\isacharcomma}{\kern0pt}leq{\isacharcomma}{\kern0pt}one{\isacharcomma}{\kern0pt}{\isacharless}{\kern0pt}{\isasymF}{\isacharcomma}{\kern0pt}\ {\isasymG}{\isacharcomma}{\kern0pt}\ P{\isacharcomma}{\kern0pt}\ P{\isacharunderscore}{\kern0pt}auto{\isachargreater}{\kern0pt}{\isacharbrackright}{\kern0pt}{\isacharat}{\kern0pt}env{\isacharparenright}{\kern0pt}\ {\isasymlongleftrightarrow}\isanewline
\ \ \ \ \ \ \ \ \ p{\isasymin}P\ {\isasymand}\ {\isacharparenleft}{\kern0pt}{\isasymforall}x{\isasymin}HS{\isachardot}{\kern0pt}\ sats{\isacharparenleft}{\kern0pt}M{\isacharcomma}{\kern0pt}forcesHS{\isacharprime}{\kern0pt}{\isacharparenleft}{\kern0pt}{\isasymphi}{\isacharparenright}{\kern0pt}{\isacharcomma}{\kern0pt}{\isacharbrackleft}{\kern0pt}p{\isacharcomma}{\kern0pt}P{\isacharcomma}{\kern0pt}leq{\isacharcomma}{\kern0pt}one{\isacharcomma}{\kern0pt}{\isacharless}{\kern0pt}{\isasymF}{\isacharcomma}{\kern0pt}\ {\isasymG}{\isacharcomma}{\kern0pt}\ P{\isacharcomma}{\kern0pt}\ P{\isacharunderscore}{\kern0pt}auto{\isachargreater}{\kern0pt}{\isacharcomma}{\kern0pt}x{\isacharbrackright}{\kern0pt}{\isacharat}{\kern0pt}env{\isacharparenright}{\kern0pt}{\isacharparenright}{\kern0pt}{\isachardoublequoteclose}\isanewline
%
\isadelimproof
\ \ %
\endisadelimproof
%
\isatagproof
\isacommand{unfolding}\isamarkupfalse%
\ forcesHS{\isacharunderscore}{\kern0pt}def\ \isanewline
\ \ \isacommand{using}\isamarkupfalse%
\ assms\ P{\isacharunderscore}{\kern0pt}in{\isacharunderscore}{\kern0pt}M\ leq{\isacharunderscore}{\kern0pt}in{\isacharunderscore}{\kern0pt}M\ one{\isacharunderscore}{\kern0pt}in{\isacharunderscore}{\kern0pt}M\isanewline
\ \ \isacommand{apply}\isamarkupfalse%
{\isacharparenleft}{\kern0pt}subgoal{\isacharunderscore}{\kern0pt}tac\ {\isachardoublequoteopen}{\isacharless}{\kern0pt}{\isasymF}{\isacharcomma}{\kern0pt}\ {\isasymG}{\isacharcomma}{\kern0pt}\ P{\isacharcomma}{\kern0pt}\ P{\isacharunderscore}{\kern0pt}auto{\isachargreater}{\kern0pt}\ {\isasymin}\ M{\isachardoublequoteclose}{\isacharparenright}{\kern0pt}\isanewline
\ \ \isacommand{apply}\isamarkupfalse%
\ simp\isanewline
\ \ \ \isacommand{apply}\isamarkupfalse%
{\isacharparenleft}{\kern0pt}rule\ iff{\isacharunderscore}{\kern0pt}conjI{\isadigit{2}}{\isacharcomma}{\kern0pt}\ simp{\isacharcomma}{\kern0pt}\ rule\ iffI{\isacharcomma}{\kern0pt}\ rule\ ballI{\isacharparenright}{\kern0pt}\isanewline
\ \ \ \ \isacommand{apply}\isamarkupfalse%
{\isacharparenleft}{\kern0pt}rename{\isacharunderscore}{\kern0pt}tac\ x{\isacharcomma}{\kern0pt}\ subgoal{\isacharunderscore}{\kern0pt}tac\ {\isachardoublequoteopen}M{\isacharcomma}{\kern0pt}\ {\isacharbrackleft}{\kern0pt}p{\isacharcomma}{\kern0pt}\ P{\isacharcomma}{\kern0pt}\ leq{\isacharcomma}{\kern0pt}\ one{\isacharcomma}{\kern0pt}\ {\isasymlangle}{\isasymF}{\isacharcomma}{\kern0pt}\ {\isasymG}{\isacharcomma}{\kern0pt}\ P{\isacharcomma}{\kern0pt}\ P{\isacharunderscore}{\kern0pt}auto{\isasymrangle}{\isacharcomma}{\kern0pt}\ x{\isacharbrackright}{\kern0pt}\ {\isacharat}{\kern0pt}\ env\ {\isasymTurnstile}\ forcesHS{\isacharprime}{\kern0pt}{\isacharparenleft}{\kern0pt}{\isasymphi}{\isacharparenright}{\kern0pt}{\isachardoublequoteclose}{\isacharcomma}{\kern0pt}\ force{\isacharparenright}{\kern0pt}\isanewline
\ \ \ \ \isacommand{apply}\isamarkupfalse%
{\isacharparenleft}{\kern0pt}rule\ iffD{\isadigit{2}}{\isacharcomma}{\kern0pt}\ rule{\isacharunderscore}{\kern0pt}tac\ n{\isacharequal}{\kern0pt}{\isachardoublequoteopen}arity{\isacharparenleft}{\kern0pt}forcesHS{\isacharprime}{\kern0pt}{\isacharparenleft}{\kern0pt}{\isasymphi}{\isacharparenright}{\kern0pt}{\isacharparenright}{\kern0pt}{\isachardoublequoteclose}\ \isakeyword{in}\ sats{\isacharunderscore}{\kern0pt}ren{\isacharunderscore}{\kern0pt}forcesHS{\isacharunderscore}{\kern0pt}forall{\isacharunderscore}{\kern0pt}iff{\isacharparenright}{\kern0pt}\isanewline
\ \ \ \ \ \ \ \ \ \ \ \ \ \ \isacommand{apply}\isamarkupfalse%
{\isacharparenleft}{\kern0pt}rule\ forcesHS{\isacharprime}{\kern0pt}{\isacharunderscore}{\kern0pt}type{\isacharcomma}{\kern0pt}\ simp{\isacharcomma}{\kern0pt}\ rule\ max{\isacharunderscore}{\kern0pt}le{\isadigit{2}}{\isacharcomma}{\kern0pt}\ simp{\isacharparenright}{\kern0pt}\isanewline
\ \ \isacommand{using}\isamarkupfalse%
\ forcesHS{\isacharprime}{\kern0pt}{\isacharunderscore}{\kern0pt}type\ arity{\isacharunderscore}{\kern0pt}type\ pair{\isacharunderscore}{\kern0pt}in{\isacharunderscore}{\kern0pt}M{\isacharunderscore}{\kern0pt}iff\ HS{\isacharunderscore}{\kern0pt}iff\ P{\isacharunderscore}{\kern0pt}name{\isacharunderscore}{\kern0pt}in{\isacharunderscore}{\kern0pt}M\isanewline
\ \ \ \ \ \ \ \ \ \ \ \ \ \isacommand{apply}\isamarkupfalse%
\ auto{\isacharbrackleft}{\kern0pt}{\isadigit{9}}{\isacharbrackright}{\kern0pt}\isanewline
\ \ \ \ \isacommand{apply}\isamarkupfalse%
{\isacharparenleft}{\kern0pt}rename{\isacharunderscore}{\kern0pt}tac\ x{\isacharcomma}{\kern0pt}\ subgoal{\isacharunderscore}{\kern0pt}tac\ {\isachardoublequoteopen}M{\isacharcomma}{\kern0pt}\ Cons{\isacharparenleft}{\kern0pt}x{\isacharcomma}{\kern0pt}\ Cons{\isacharparenleft}{\kern0pt}p{\isacharcomma}{\kern0pt}\ Cons{\isacharparenleft}{\kern0pt}P{\isacharcomma}{\kern0pt}\ Cons{\isacharparenleft}{\kern0pt}leq{\isacharcomma}{\kern0pt}\ Cons{\isacharparenleft}{\kern0pt}one{\isacharcomma}{\kern0pt}\ Cons{\isacharparenleft}{\kern0pt}{\isasymlangle}{\isasymF}{\isacharcomma}{\kern0pt}\ {\isasymG}{\isacharcomma}{\kern0pt}\ P{\isacharcomma}{\kern0pt}\ P{\isacharunderscore}{\kern0pt}auto{\isasymrangle}{\isacharcomma}{\kern0pt}\ env{\isacharparenright}{\kern0pt}{\isacharparenright}{\kern0pt}{\isacharparenright}{\kern0pt}{\isacharparenright}{\kern0pt}{\isacharparenright}{\kern0pt}{\isacharparenright}{\kern0pt}\ {\isasymTurnstile}\ is{\isacharunderscore}{\kern0pt}HS{\isacharunderscore}{\kern0pt}fm{\isacharparenleft}{\kern0pt}{\isadigit{5}}{\isacharcomma}{\kern0pt}\ {\isadigit{0}}{\isacharparenright}{\kern0pt}{\isachardoublequoteclose}{\isacharparenright}{\kern0pt}\isanewline
\ \ \isacommand{using}\isamarkupfalse%
\ HS{\isacharunderscore}{\kern0pt}iff\ P{\isacharunderscore}{\kern0pt}name{\isacharunderscore}{\kern0pt}in{\isacharunderscore}{\kern0pt}M\ \isanewline
\ \ \ \ \ \isacommand{apply}\isamarkupfalse%
\ force\ \ \isanewline
\ \ \ \ \isacommand{apply}\isamarkupfalse%
{\isacharparenleft}{\kern0pt}rule\ iffD{\isadigit{2}}{\isacharcomma}{\kern0pt}\ rule\ sats{\isacharunderscore}{\kern0pt}is{\isacharunderscore}{\kern0pt}HS{\isacharunderscore}{\kern0pt}fm{\isacharunderscore}{\kern0pt}iff{\isacharparenright}{\kern0pt}\isanewline
\ \ \isacommand{using}\isamarkupfalse%
\ HS{\isacharunderscore}{\kern0pt}iff\ P{\isacharunderscore}{\kern0pt}name{\isacharunderscore}{\kern0pt}in{\isacharunderscore}{\kern0pt}M\ \isanewline
\ \ \ \ \ \ \ \ \ \isacommand{apply}\isamarkupfalse%
\ auto{\isacharbrackleft}{\kern0pt}{\isadigit{6}}{\isacharbrackright}{\kern0pt}\isanewline
\ \ \ \isacommand{apply}\isamarkupfalse%
{\isacharparenleft}{\kern0pt}rule\ ballI{\isacharcomma}{\kern0pt}\ rule\ impI{\isacharparenright}{\kern0pt}\isanewline
\ \ \isacommand{apply}\isamarkupfalse%
{\isacharparenleft}{\kern0pt}rename{\isacharunderscore}{\kern0pt}tac\ x{\isacharcomma}{\kern0pt}\ \isanewline
\ \ \ \ \ \ \ \ subgoal{\isacharunderscore}{\kern0pt}tac\ {\isachardoublequoteopen}\ M{\isacharcomma}{\kern0pt}\ {\isacharbrackleft}{\kern0pt}x{\isacharcomma}{\kern0pt}\ p{\isacharcomma}{\kern0pt}\ P{\isacharcomma}{\kern0pt}\ leq{\isacharcomma}{\kern0pt}\ one{\isacharcomma}{\kern0pt}\ {\isasymlangle}{\isasymF}{\isacharcomma}{\kern0pt}\ {\isasymG}{\isacharcomma}{\kern0pt}\ P{\isacharcomma}{\kern0pt}\ P{\isacharunderscore}{\kern0pt}auto{\isasymrangle}{\isacharbrackright}{\kern0pt}\ {\isacharat}{\kern0pt}\ env\ {\isasymTurnstile}\ \isanewline
\ \ \ \ \ \ \ \ \ \ \ \ \ \ \ \ \ \ \ \ \ \ \ \ ren{\isacharparenleft}{\kern0pt}forcesHS{\isacharprime}{\kern0pt}{\isacharparenleft}{\kern0pt}{\isasymphi}{\isacharparenright}{\kern0pt}{\isacharparenright}{\kern0pt}\ {\isacharbackquote}{\kern0pt}\ {\isacharparenleft}{\kern0pt}{\isadigit{6}}\ {\isasymunion}\ arity{\isacharparenleft}{\kern0pt}forcesHS{\isacharprime}{\kern0pt}{\isacharparenleft}{\kern0pt}{\isasymphi}{\isacharparenright}{\kern0pt}{\isacharparenright}{\kern0pt}{\isacharparenright}{\kern0pt}\ {\isacharbackquote}{\kern0pt}\ {\isacharparenleft}{\kern0pt}{\isadigit{6}}\ {\isasymunion}\ arity{\isacharparenleft}{\kern0pt}forcesHS{\isacharprime}{\kern0pt}{\isacharparenleft}{\kern0pt}{\isasymphi}{\isacharparenright}{\kern0pt}{\isacharparenright}{\kern0pt}{\isacharparenright}{\kern0pt}\ {\isacharbackquote}{\kern0pt}\ ren{\isacharunderscore}{\kern0pt}forcesHS{\isacharunderscore}{\kern0pt}forall{\isacharparenleft}{\kern0pt}arity{\isacharparenleft}{\kern0pt}forcesHS{\isacharprime}{\kern0pt}{\isacharparenleft}{\kern0pt}{\isasymphi}{\isacharparenright}{\kern0pt}{\isacharparenright}{\kern0pt}{\isacharparenright}{\kern0pt}{\isachardoublequoteclose}{\isacharparenright}{\kern0pt}\isanewline
\ \ \ \ \isacommand{apply}\isamarkupfalse%
\ force\isanewline
\ \ \ \isacommand{apply}\isamarkupfalse%
{\isacharparenleft}{\kern0pt}rule\ iffD{\isadigit{1}}{\isacharcomma}{\kern0pt}\ rule{\isacharunderscore}{\kern0pt}tac\ n{\isacharequal}{\kern0pt}{\isachardoublequoteopen}arity{\isacharparenleft}{\kern0pt}forcesHS{\isacharprime}{\kern0pt}{\isacharparenleft}{\kern0pt}{\isasymphi}{\isacharparenright}{\kern0pt}{\isacharparenright}{\kern0pt}{\isachardoublequoteclose}\ \isakeyword{in}\ sats{\isacharunderscore}{\kern0pt}ren{\isacharunderscore}{\kern0pt}forcesHS{\isacharunderscore}{\kern0pt}forall{\isacharunderscore}{\kern0pt}iff{\isacharparenright}{\kern0pt}\isanewline
\ \ \ \ \ \ \ \ \ \ \ \ \ \isacommand{apply}\isamarkupfalse%
{\isacharparenleft}{\kern0pt}rule\ forcesHS{\isacharprime}{\kern0pt}{\isacharunderscore}{\kern0pt}type{\isacharcomma}{\kern0pt}\ simp{\isacharcomma}{\kern0pt}\ rule\ max{\isacharunderscore}{\kern0pt}le{\isadigit{2}}{\isacharcomma}{\kern0pt}\ simp{\isacharparenright}{\kern0pt}\isanewline
\ \ \isacommand{using}\isamarkupfalse%
\ forcesHS{\isacharprime}{\kern0pt}{\isacharunderscore}{\kern0pt}type\ arity{\isacharunderscore}{\kern0pt}type\ pair{\isacharunderscore}{\kern0pt}in{\isacharunderscore}{\kern0pt}M{\isacharunderscore}{\kern0pt}iff\ HS{\isacharunderscore}{\kern0pt}iff\ P{\isacharunderscore}{\kern0pt}name{\isacharunderscore}{\kern0pt}in{\isacharunderscore}{\kern0pt}M\isanewline
\ \ \ \ \ \ \ \ \ \ \ \ \isacommand{apply}\isamarkupfalse%
\ auto{\isacharbrackleft}{\kern0pt}{\isadigit{9}}{\isacharbrackright}{\kern0pt}\isanewline
\ \ \ \isacommand{apply}\isamarkupfalse%
{\isacharparenleft}{\kern0pt}rename{\isacharunderscore}{\kern0pt}tac\ x{\isacharcomma}{\kern0pt}\ subgoal{\isacharunderscore}{\kern0pt}tac\ {\isachardoublequoteopen}x\ {\isasymin}\ HS{\isachardoublequoteclose}{\isacharcomma}{\kern0pt}\ force{\isacharparenright}{\kern0pt}\isanewline
\ \ \ \isacommand{apply}\isamarkupfalse%
{\isacharparenleft}{\kern0pt}rename{\isacharunderscore}{\kern0pt}tac\ x{\isacharcomma}{\kern0pt}\ rule{\isacharunderscore}{\kern0pt}tac\ P{\isacharequal}{\kern0pt}{\isachardoublequoteopen}M{\isacharcomma}{\kern0pt}\ Cons{\isacharparenleft}{\kern0pt}x{\isacharcomma}{\kern0pt}\ Cons{\isacharparenleft}{\kern0pt}p{\isacharcomma}{\kern0pt}\ Cons{\isacharparenleft}{\kern0pt}P{\isacharcomma}{\kern0pt}\ Cons{\isacharparenleft}{\kern0pt}leq{\isacharcomma}{\kern0pt}\ Cons{\isacharparenleft}{\kern0pt}one{\isacharcomma}{\kern0pt}\ Cons{\isacharparenleft}{\kern0pt}{\isasymlangle}{\isasymF}{\isacharcomma}{\kern0pt}\ {\isasymG}{\isacharcomma}{\kern0pt}\ P{\isacharcomma}{\kern0pt}\ P{\isacharunderscore}{\kern0pt}auto{\isasymrangle}{\isacharcomma}{\kern0pt}\ env{\isacharparenright}{\kern0pt}{\isacharparenright}{\kern0pt}{\isacharparenright}{\kern0pt}{\isacharparenright}{\kern0pt}{\isacharparenright}{\kern0pt}{\isacharparenright}{\kern0pt}\ {\isasymTurnstile}\ is{\isacharunderscore}{\kern0pt}HS{\isacharunderscore}{\kern0pt}fm{\isacharparenleft}{\kern0pt}{\isadigit{5}}{\isacharcomma}{\kern0pt}\ {\isadigit{0}}{\isacharparenright}{\kern0pt}{\isachardoublequoteclose}\ \isakeyword{in}\ iffD{\isadigit{1}}{\isacharcomma}{\kern0pt}\ rule\ sats{\isacharunderscore}{\kern0pt}is{\isacharunderscore}{\kern0pt}HS{\isacharunderscore}{\kern0pt}fm{\isacharunderscore}{\kern0pt}iff{\isacharparenright}{\kern0pt}\isanewline
\ \ \ \ \ \ \ \ \isacommand{apply}\isamarkupfalse%
\ auto{\isacharbrackleft}{\kern0pt}{\isadigit{6}}{\isacharbrackright}{\kern0pt}\isanewline
\ \ \isacommand{using}\isamarkupfalse%
\ {\isasymF}{\isacharunderscore}{\kern0pt}in{\isacharunderscore}{\kern0pt}M\ {\isasymG}{\isacharunderscore}{\kern0pt}in{\isacharunderscore}{\kern0pt}M\ P{\isacharunderscore}{\kern0pt}auto{\isacharunderscore}{\kern0pt}in{\isacharunderscore}{\kern0pt}M\ pair{\isacharunderscore}{\kern0pt}in{\isacharunderscore}{\kern0pt}M{\isacharunderscore}{\kern0pt}iff\isanewline
\ \ \isacommand{by}\isamarkupfalse%
\ auto%
\endisatagproof
{\isafoldproof}%
%
\isadelimproof
\isanewline
%
\endisadelimproof
\isanewline
\isacommand{lemma}\isamarkupfalse%
\ sats{\isacharunderscore}{\kern0pt}forcesHS{\isacharunderscore}{\kern0pt}Forall{\isacharprime}{\kern0pt}{\isacharcolon}{\kern0pt}\isanewline
\ \ \isakeyword{assumes}\isanewline
\ \ \ \ {\isachardoublequoteopen}p{\isasymin}P{\isachardoublequoteclose}\ {\isachardoublequoteopen}env\ {\isasymin}\ list{\isacharparenleft}{\kern0pt}M{\isacharparenright}{\kern0pt}{\isachardoublequoteclose}\ {\isachardoublequoteopen}{\isasymphi}{\isasymin}formula{\isachardoublequoteclose}\isanewline
\ \ \isakeyword{shows}\isanewline
\ \ \ \ {\isachardoublequoteopen}M{\isacharcomma}{\kern0pt}{\isacharbrackleft}{\kern0pt}p{\isacharcomma}{\kern0pt}P{\isacharcomma}{\kern0pt}leq{\isacharcomma}{\kern0pt}one{\isacharcomma}{\kern0pt}{\isasymlangle}{\isasymF}{\isacharcomma}{\kern0pt}\ {\isasymG}{\isacharcomma}{\kern0pt}\ P{\isacharcomma}{\kern0pt}\ P{\isacharunderscore}{\kern0pt}auto{\isasymrangle}{\isacharbrackright}{\kern0pt}\ {\isacharat}{\kern0pt}\ env\ {\isasymTurnstile}\ forcesHS{\isacharparenleft}{\kern0pt}Forall{\isacharparenleft}{\kern0pt}{\isasymphi}{\isacharparenright}{\kern0pt}{\isacharparenright}{\kern0pt}\ {\isasymlongleftrightarrow}\ {\isacharparenleft}{\kern0pt}{\isasymforall}x{\isasymin}HS{\isachardot}{\kern0pt}\ M{\isacharcomma}{\kern0pt}\ {\isacharbrackleft}{\kern0pt}p{\isacharcomma}{\kern0pt}P{\isacharcomma}{\kern0pt}leq{\isacharcomma}{\kern0pt}one{\isacharcomma}{\kern0pt}{\isasymlangle}{\isasymF}{\isacharcomma}{\kern0pt}\ {\isasymG}{\isacharcomma}{\kern0pt}\ P{\isacharcomma}{\kern0pt}\ P{\isacharunderscore}{\kern0pt}auto{\isasymrangle}{\isacharcomma}{\kern0pt}x{\isacharbrackright}{\kern0pt}\ {\isacharat}{\kern0pt}\ env\ {\isasymTurnstile}\ forcesHS{\isacharparenleft}{\kern0pt}{\isasymphi}{\isacharparenright}{\kern0pt}{\isacharparenright}{\kern0pt}{\isachardoublequoteclose}\isanewline
%
\isadelimproof
\ \ %
\endisadelimproof
%
\isatagproof
\isacommand{apply}\isamarkupfalse%
{\isacharparenleft}{\kern0pt}rule\ iff{\isacharunderscore}{\kern0pt}trans{\isacharcomma}{\kern0pt}\ rule\ sats{\isacharunderscore}{\kern0pt}forcesHS{\isacharunderscore}{\kern0pt}Forall{\isacharparenright}{\kern0pt}\isanewline
\ \ \isacommand{using}\isamarkupfalse%
\ assms\ P{\isacharunderscore}{\kern0pt}in{\isacharunderscore}{\kern0pt}M\ transM\ \isanewline
\ \ \ \ \ \isacommand{apply}\isamarkupfalse%
\ auto{\isacharbrackleft}{\kern0pt}{\isadigit{3}}{\isacharbrackright}{\kern0pt}\isanewline
\ \ \isacommand{using}\isamarkupfalse%
\ assms\isanewline
\ \ \isacommand{apply}\isamarkupfalse%
\ simp\isanewline
\ \ \isacommand{apply}\isamarkupfalse%
{\isacharparenleft}{\kern0pt}rule\ ball{\isacharunderscore}{\kern0pt}iff{\isacharparenright}{\kern0pt}\isanewline
\ \ \isacommand{unfolding}\isamarkupfalse%
\ forcesHS{\isacharunderscore}{\kern0pt}def\ \isanewline
\ \ \isacommand{apply}\isamarkupfalse%
{\isacharparenleft}{\kern0pt}subgoal{\isacharunderscore}{\kern0pt}tac\ {\isachardoublequoteopen}P\ {\isasymin}\ M\ {\isasymand}\ leq\ {\isasymin}\ M\ {\isasymand}\ one\ {\isasymin}\ M\ {\isasymand}\ {\isasymlangle}{\isasymF}{\isacharcomma}{\kern0pt}\ {\isasymG}{\isacharcomma}{\kern0pt}\ P{\isacharcomma}{\kern0pt}\ P{\isacharunderscore}{\kern0pt}auto{\isasymrangle}\ {\isasymin}\ M\ {\isasymand}\ p\ {\isasymin}\ M\ {\isasymand}\ HS\ {\isasymsubseteq}\ M{\isachardoublequoteclose}{\isacharparenright}{\kern0pt}\isanewline
\ \ \ \isacommand{apply}\isamarkupfalse%
\ force\ \isanewline
\ \ \isacommand{using}\isamarkupfalse%
\ {\isasymF}{\isacharunderscore}{\kern0pt}in{\isacharunderscore}{\kern0pt}M\ {\isasymG}{\isacharunderscore}{\kern0pt}in{\isacharunderscore}{\kern0pt}M\ P{\isacharunderscore}{\kern0pt}auto{\isacharunderscore}{\kern0pt}in{\isacharunderscore}{\kern0pt}M\ pair{\isacharunderscore}{\kern0pt}in{\isacharunderscore}{\kern0pt}M{\isacharunderscore}{\kern0pt}iff\ P{\isacharunderscore}{\kern0pt}in{\isacharunderscore}{\kern0pt}M\ leq{\isacharunderscore}{\kern0pt}in{\isacharunderscore}{\kern0pt}M\ one{\isacharunderscore}{\kern0pt}in{\isacharunderscore}{\kern0pt}M\ HS{\isacharunderscore}{\kern0pt}iff\ P{\isacharunderscore}{\kern0pt}name{\isacharunderscore}{\kern0pt}in{\isacharunderscore}{\kern0pt}M\ assms\ transM\isanewline
\ \ \isacommand{by}\isamarkupfalse%
\ auto%
\endisatagproof
{\isafoldproof}%
%
\isadelimproof
\isanewline
%
\endisadelimproof
\isanewline
\isacommand{lemma}\isamarkupfalse%
\ ForcesHS{\isacharunderscore}{\kern0pt}Forall{\isacharcolon}{\kern0pt}\isanewline
\ \ \isakeyword{assumes}\isanewline
\ \ \ \ {\isachardoublequoteopen}p{\isasymin}P{\isachardoublequoteclose}\ {\isachardoublequoteopen}env\ {\isasymin}\ list{\isacharparenleft}{\kern0pt}M{\isacharparenright}{\kern0pt}{\isachardoublequoteclose}\ {\isachardoublequoteopen}{\isasymphi}{\isasymin}formula{\isachardoublequoteclose}\isanewline
\ \ \isakeyword{shows}\isanewline
\ \ \ \ {\isachardoublequoteopen}{\isacharparenleft}{\kern0pt}p\ {\isasymtturnstile}HS\ Forall{\isacharparenleft}{\kern0pt}{\isasymphi}{\isacharparenright}{\kern0pt}\ env{\isacharparenright}{\kern0pt}\ {\isasymlongleftrightarrow}\ {\isacharparenleft}{\kern0pt}{\isasymforall}x{\isasymin}HS{\isachardot}{\kern0pt}\ {\isacharparenleft}{\kern0pt}p\ {\isasymtturnstile}HS\ {\isasymphi}\ {\isacharparenleft}{\kern0pt}{\isacharbrackleft}{\kern0pt}x{\isacharbrackright}{\kern0pt}\ {\isacharat}{\kern0pt}\ env{\isacharparenright}{\kern0pt}{\isacharparenright}{\kern0pt}{\isacharparenright}{\kern0pt}{\isachardoublequoteclose}\isanewline
%
\isadelimproof
\ \ \ %
\endisadelimproof
%
\isatagproof
\isacommand{using}\isamarkupfalse%
\ sats{\isacharunderscore}{\kern0pt}forcesHS{\isacharunderscore}{\kern0pt}Forall{\isacharprime}{\kern0pt}\ assms\ \isacommand{by}\isamarkupfalse%
\ simp%
\endisatagproof
{\isafoldproof}%
%
\isadelimproof
\isanewline
%
\endisadelimproof
\isanewline
\isacommand{lemma}\isamarkupfalse%
\ HS{\isacharunderscore}{\kern0pt}strengthening{\isacharunderscore}{\kern0pt}lemma{\isacharcolon}{\kern0pt}\isanewline
\ \ \isakeyword{assumes}\ \isanewline
\ \ \ \ {\isachardoublequoteopen}p{\isasymin}P{\isachardoublequoteclose}\ {\isachardoublequoteopen}{\isasymphi}{\isasymin}formula{\isachardoublequoteclose}\ {\isachardoublequoteopen}r{\isasymin}P{\isachardoublequoteclose}\ {\isachardoublequoteopen}r{\isasympreceq}p{\isachardoublequoteclose}\isanewline
\ \ \isakeyword{shows}\isanewline
\ \ \ \ {\isachardoublequoteopen}{\isasymAnd}env{\isachardot}{\kern0pt}\ env{\isasymin}list{\isacharparenleft}{\kern0pt}M{\isacharparenright}{\kern0pt}\ {\isasymLongrightarrow}\ arity{\isacharparenleft}{\kern0pt}{\isasymphi}{\isacharparenright}{\kern0pt}{\isasymle}length{\isacharparenleft}{\kern0pt}env{\isacharparenright}{\kern0pt}\ {\isasymLongrightarrow}\ p\ {\isasymtturnstile}HS\ {\isasymphi}\ env\ {\isasymLongrightarrow}\ r\ {\isasymtturnstile}HS\ {\isasymphi}\ env{\isachardoublequoteclose}\isanewline
%
\isadelimproof
\ \ %
\endisadelimproof
%
\isatagproof
\isacommand{using}\isamarkupfalse%
\ assms{\isacharparenleft}{\kern0pt}{\isadigit{2}}{\isacharparenright}{\kern0pt}\isanewline
\isacommand{proof}\isamarkupfalse%
\ {\isacharparenleft}{\kern0pt}induct{\isacharparenright}{\kern0pt}\isanewline
\ \ \isacommand{case}\isamarkupfalse%
\ {\isacharparenleft}{\kern0pt}Member\ n\ m{\isacharparenright}{\kern0pt}\isanewline
\ \ \isacommand{then}\isamarkupfalse%
\ \isacommand{have}\isamarkupfalse%
\ assms{\isadigit{1}}\ {\isacharcolon}{\kern0pt}\ \isanewline
\ \ \ \ {\isachardoublequoteopen}n\ {\isasymin}\ nat{\isachardoublequoteclose}\ {\isachardoublequoteopen}m\ {\isasymin}\ nat{\isachardoublequoteclose}\ {\isachardoublequoteopen}env\ {\isasymin}\ list{\isacharparenleft}{\kern0pt}M{\isacharparenright}{\kern0pt}{\isachardoublequoteclose}\isanewline
\ \ \ \ {\isachardoublequoteopen}arity{\isacharparenleft}{\kern0pt}Member{\isacharparenleft}{\kern0pt}n{\isacharcomma}{\kern0pt}\ m{\isacharparenright}{\kern0pt}{\isacharparenright}{\kern0pt}\ {\isasymle}\ length{\isacharparenleft}{\kern0pt}env{\isacharparenright}{\kern0pt}{\isachardoublequoteclose}\ {\isachardoublequoteopen}M{\isacharcomma}{\kern0pt}\ {\isacharbrackleft}{\kern0pt}p{\isacharcomma}{\kern0pt}\ P{\isacharcomma}{\kern0pt}\ leq{\isacharcomma}{\kern0pt}\ one{\isacharcomma}{\kern0pt}\ {\isasymlangle}{\isasymF}{\isacharcomma}{\kern0pt}\ {\isasymG}{\isacharcomma}{\kern0pt}\ P{\isacharcomma}{\kern0pt}\ P{\isacharunderscore}{\kern0pt}auto{\isasymrangle}{\isacharbrackright}{\kern0pt}\ {\isacharat}{\kern0pt}\ env\ {\isasymTurnstile}\ forcesHS{\isacharparenleft}{\kern0pt}Member{\isacharparenleft}{\kern0pt}n{\isacharcomma}{\kern0pt}\ m{\isacharparenright}{\kern0pt}{\isacharparenright}{\kern0pt}{\isachardoublequoteclose}\ \isanewline
\ \ \ \ \isacommand{by}\isamarkupfalse%
\ auto\isanewline
\ \ \isacommand{then}\isamarkupfalse%
\ \isacommand{have}\isamarkupfalse%
\ {\isachardoublequoteopen}p\ {\isasymtturnstile}\ {\isacharparenleft}{\kern0pt}Member{\isacharparenleft}{\kern0pt}n{\isacharcomma}{\kern0pt}\ m{\isacharparenright}{\kern0pt}{\isacharparenright}{\kern0pt}\ env{\isachardoublequoteclose}\ \isanewline
\ \ \ \ \isacommand{apply}\isamarkupfalse%
{\isacharparenleft}{\kern0pt}rule{\isacharunderscore}{\kern0pt}tac\ iffD{\isadigit{2}}{\isacharparenright}{\kern0pt}\isanewline
\ \ \ \ \ \isacommand{apply}\isamarkupfalse%
{\isacharparenleft}{\kern0pt}rule\ ForcesHS{\isacharunderscore}{\kern0pt}Member{\isacharparenright}{\kern0pt}\isanewline
\ \ \ \ \isacommand{using}\isamarkupfalse%
\ P{\isacharunderscore}{\kern0pt}in{\isacharunderscore}{\kern0pt}M\ transM\ assms\isanewline
\ \ \ \ \isacommand{by}\isamarkupfalse%
\ auto\isanewline
\ \ \isacommand{then}\isamarkupfalse%
\ \isacommand{have}\isamarkupfalse%
\ {\isachardoublequoteopen}r\ {\isasymtturnstile}\ {\isacharparenleft}{\kern0pt}Member{\isacharparenleft}{\kern0pt}n{\isacharcomma}{\kern0pt}\ m{\isacharparenright}{\kern0pt}{\isacharparenright}{\kern0pt}\ env{\isachardoublequoteclose}\ \isanewline
\ \ \ \ \isacommand{apply}\isamarkupfalse%
{\isacharparenleft}{\kern0pt}rule{\isacharunderscore}{\kern0pt}tac\ strengthening{\isacharunderscore}{\kern0pt}lemma{\isacharparenright}{\kern0pt}\isanewline
\ \ \ \ \isacommand{using}\isamarkupfalse%
\ assms\ assms{\isadigit{1}}\ \isanewline
\ \ \ \ \isacommand{by}\isamarkupfalse%
\ auto\isanewline
\ \ \isacommand{then}\isamarkupfalse%
\ \isacommand{show}\isamarkupfalse%
\ {\isachardoublequoteopen}r\ {\isasymtturnstile}HS\ {\isacharparenleft}{\kern0pt}Member{\isacharparenleft}{\kern0pt}n{\isacharcomma}{\kern0pt}\ m{\isacharparenright}{\kern0pt}{\isacharparenright}{\kern0pt}\ env{\isachardoublequoteclose}\ \isanewline
\ \ \ \ \isacommand{apply}\isamarkupfalse%
{\isacharparenleft}{\kern0pt}rule{\isacharunderscore}{\kern0pt}tac\ iffD{\isadigit{1}}{\isacharparenright}{\kern0pt}\isanewline
\ \ \ \ \ \isacommand{apply}\isamarkupfalse%
{\isacharparenleft}{\kern0pt}rule\ ForcesHS{\isacharunderscore}{\kern0pt}Member{\isacharparenright}{\kern0pt}\isanewline
\ \ \ \ \isacommand{using}\isamarkupfalse%
\ P{\isacharunderscore}{\kern0pt}in{\isacharunderscore}{\kern0pt}M\ transM\ assms\ assms{\isadigit{1}}\isanewline
\ \ \ \ \isacommand{by}\isamarkupfalse%
\ auto\isanewline
\isacommand{next}\isamarkupfalse%
\isanewline
\ \ \isacommand{case}\isamarkupfalse%
\ {\isacharparenleft}{\kern0pt}Equal\ n\ m{\isacharparenright}{\kern0pt}\isanewline
\ \ \isacommand{then}\isamarkupfalse%
\ \isacommand{have}\isamarkupfalse%
\ assms{\isadigit{1}}\ {\isacharcolon}{\kern0pt}\ \isanewline
\ \ \ \ {\isachardoublequoteopen}n\ {\isasymin}\ nat{\isachardoublequoteclose}\ {\isachardoublequoteopen}m\ {\isasymin}\ nat{\isachardoublequoteclose}\ {\isachardoublequoteopen}env\ {\isasymin}\ list{\isacharparenleft}{\kern0pt}M{\isacharparenright}{\kern0pt}{\isachardoublequoteclose}\isanewline
\ \ \ \ {\isachardoublequoteopen}arity{\isacharparenleft}{\kern0pt}Member{\isacharparenleft}{\kern0pt}n{\isacharcomma}{\kern0pt}\ m{\isacharparenright}{\kern0pt}{\isacharparenright}{\kern0pt}\ {\isasymle}\ length{\isacharparenleft}{\kern0pt}env{\isacharparenright}{\kern0pt}{\isachardoublequoteclose}\ {\isachardoublequoteopen}M{\isacharcomma}{\kern0pt}\ {\isacharbrackleft}{\kern0pt}p{\isacharcomma}{\kern0pt}\ P{\isacharcomma}{\kern0pt}\ leq{\isacharcomma}{\kern0pt}\ one{\isacharcomma}{\kern0pt}\ {\isasymlangle}{\isasymF}{\isacharcomma}{\kern0pt}\ {\isasymG}{\isacharcomma}{\kern0pt}\ P{\isacharcomma}{\kern0pt}\ P{\isacharunderscore}{\kern0pt}auto{\isasymrangle}{\isacharbrackright}{\kern0pt}\ {\isacharat}{\kern0pt}\ env\ {\isasymTurnstile}\ forcesHS{\isacharparenleft}{\kern0pt}Equal{\isacharparenleft}{\kern0pt}n{\isacharcomma}{\kern0pt}\ m{\isacharparenright}{\kern0pt}{\isacharparenright}{\kern0pt}{\isachardoublequoteclose}\ \isanewline
\ \ \ \ \isacommand{by}\isamarkupfalse%
\ auto\isanewline
\ \ \isacommand{then}\isamarkupfalse%
\ \isacommand{have}\isamarkupfalse%
\ {\isachardoublequoteopen}p\ {\isasymtturnstile}\ {\isacharparenleft}{\kern0pt}Equal{\isacharparenleft}{\kern0pt}n{\isacharcomma}{\kern0pt}\ m{\isacharparenright}{\kern0pt}{\isacharparenright}{\kern0pt}\ env{\isachardoublequoteclose}\ \isanewline
\ \ \ \ \isacommand{apply}\isamarkupfalse%
{\isacharparenleft}{\kern0pt}rule{\isacharunderscore}{\kern0pt}tac\ iffD{\isadigit{2}}{\isacharparenright}{\kern0pt}\isanewline
\ \ \ \ \ \isacommand{apply}\isamarkupfalse%
{\isacharparenleft}{\kern0pt}rule\ ForcesHS{\isacharunderscore}{\kern0pt}Equal{\isacharparenright}{\kern0pt}\isanewline
\ \ \ \ \isacommand{using}\isamarkupfalse%
\ P{\isacharunderscore}{\kern0pt}in{\isacharunderscore}{\kern0pt}M\ transM\ assms\isanewline
\ \ \ \ \isacommand{by}\isamarkupfalse%
\ auto\isanewline
\ \ \isacommand{then}\isamarkupfalse%
\ \isacommand{have}\isamarkupfalse%
\ {\isachardoublequoteopen}r\ {\isasymtturnstile}\ {\isacharparenleft}{\kern0pt}Equal{\isacharparenleft}{\kern0pt}n{\isacharcomma}{\kern0pt}\ m{\isacharparenright}{\kern0pt}{\isacharparenright}{\kern0pt}\ env{\isachardoublequoteclose}\ \isanewline
\ \ \ \ \isacommand{apply}\isamarkupfalse%
{\isacharparenleft}{\kern0pt}rule{\isacharunderscore}{\kern0pt}tac\ strengthening{\isacharunderscore}{\kern0pt}lemma{\isacharparenright}{\kern0pt}\isanewline
\ \ \ \ \isacommand{using}\isamarkupfalse%
\ assms\ assms{\isadigit{1}}\ \isanewline
\ \ \ \ \isacommand{by}\isamarkupfalse%
\ auto\isanewline
\ \ \isacommand{then}\isamarkupfalse%
\ \isacommand{show}\isamarkupfalse%
\ {\isachardoublequoteopen}r\ {\isasymtturnstile}HS\ {\isacharparenleft}{\kern0pt}Equal{\isacharparenleft}{\kern0pt}n{\isacharcomma}{\kern0pt}\ m{\isacharparenright}{\kern0pt}{\isacharparenright}{\kern0pt}\ env{\isachardoublequoteclose}\ \isanewline
\ \ \ \ \isacommand{apply}\isamarkupfalse%
{\isacharparenleft}{\kern0pt}rule{\isacharunderscore}{\kern0pt}tac\ iffD{\isadigit{1}}{\isacharparenright}{\kern0pt}\isanewline
\ \ \ \ \ \isacommand{apply}\isamarkupfalse%
{\isacharparenleft}{\kern0pt}rule\ ForcesHS{\isacharunderscore}{\kern0pt}Equal{\isacharparenright}{\kern0pt}\isanewline
\ \ \ \ \isacommand{using}\isamarkupfalse%
\ P{\isacharunderscore}{\kern0pt}in{\isacharunderscore}{\kern0pt}M\ transM\ assms\ assms{\isadigit{1}}\isanewline
\ \ \ \ \isacommand{by}\isamarkupfalse%
\ auto\isanewline
\isacommand{next}\isamarkupfalse%
\isanewline
\ \ \isacommand{case}\isamarkupfalse%
\ {\isacharparenleft}{\kern0pt}Nand\ {\isasymphi}\ {\isasympsi}{\isacharparenright}{\kern0pt}\isanewline
\ \ \isacommand{with}\isamarkupfalse%
\ assms\isanewline
\ \ \isacommand{show}\isamarkupfalse%
\ {\isacharquery}{\kern0pt}case\ \isanewline
\ \ \ \ \isacommand{using}\isamarkupfalse%
\ ForcesHS{\isacharunderscore}{\kern0pt}Nand\ transitivity{\isacharbrackleft}{\kern0pt}OF\ {\isacharunderscore}{\kern0pt}\ P{\isacharunderscore}{\kern0pt}in{\isacharunderscore}{\kern0pt}M{\isacharbrackright}{\kern0pt}\ pair{\isacharunderscore}{\kern0pt}in{\isacharunderscore}{\kern0pt}M{\isacharunderscore}{\kern0pt}iff\ \isanewline
\ \ \ \ \ \ transitivity{\isacharbrackleft}{\kern0pt}OF\ {\isacharunderscore}{\kern0pt}\ leq{\isacharunderscore}{\kern0pt}in{\isacharunderscore}{\kern0pt}M{\isacharbrackright}{\kern0pt}\ leq{\isacharunderscore}{\kern0pt}transD\ \isacommand{by}\isamarkupfalse%
\ auto\isanewline
\isacommand{next}\isamarkupfalse%
\isanewline
\ \ \isacommand{case}\isamarkupfalse%
\ {\isacharparenleft}{\kern0pt}Forall\ {\isasymphi}{\isacharparenright}{\kern0pt}\isanewline
\ \ \isacommand{with}\isamarkupfalse%
\ assms\isanewline
\ \ \isacommand{have}\isamarkupfalse%
\ {\isachardoublequoteopen}p\ {\isasymtturnstile}HS\ {\isasymphi}\ {\isacharparenleft}{\kern0pt}{\isacharbrackleft}{\kern0pt}x{\isacharbrackright}{\kern0pt}\ {\isacharat}{\kern0pt}\ env{\isacharparenright}{\kern0pt}{\isachardoublequoteclose}\ \isakeyword{if}\ {\isachardoublequoteopen}x{\isasymin}HS{\isachardoublequoteclose}\ \isakeyword{for}\ x\isanewline
\ \ \ \ \isacommand{using}\isamarkupfalse%
\ that\ ForcesHS{\isacharunderscore}{\kern0pt}Forall\ \isacommand{by}\isamarkupfalse%
\ simp\isanewline
\ \ \isanewline
\ \ \isacommand{with}\isamarkupfalse%
\ Forall\ \isanewline
\ \ \isacommand{have}\isamarkupfalse%
\ {\isachardoublequoteopen}r\ {\isasymtturnstile}HS\ {\isasymphi}\ {\isacharparenleft}{\kern0pt}{\isacharbrackleft}{\kern0pt}x{\isacharbrackright}{\kern0pt}\ {\isacharat}{\kern0pt}\ env{\isacharparenright}{\kern0pt}{\isachardoublequoteclose}\ \isakeyword{if}\ {\isachardoublequoteopen}x{\isasymin}HS{\isachardoublequoteclose}\ \isakeyword{for}\ x\isanewline
\ \ \ \ \isacommand{using}\isamarkupfalse%
\ that\ pred{\isacharunderscore}{\kern0pt}le{\isadigit{2}}\ HS{\isacharunderscore}{\kern0pt}iff\ P{\isacharunderscore}{\kern0pt}name{\isacharunderscore}{\kern0pt}in{\isacharunderscore}{\kern0pt}M\ \isacommand{by}\isamarkupfalse%
\ simp\isanewline
\ \ \isacommand{with}\isamarkupfalse%
\ assms\ Forall\isanewline
\ \ \isacommand{show}\isamarkupfalse%
\ {\isacharquery}{\kern0pt}case\ \isanewline
\ \ \ \ \isacommand{using}\isamarkupfalse%
\ ForcesHS{\isacharunderscore}{\kern0pt}Forall\ \isacommand{by}\isamarkupfalse%
\ simp\isanewline
\isacommand{qed}\isamarkupfalse%
%
\endisatagproof
{\isafoldproof}%
%
\isadelimproof
\isanewline
%
\endisadelimproof
\isanewline
\isanewline
\isanewline
\isacommand{lemma}\isamarkupfalse%
\ ForcesHS{\isacharunderscore}{\kern0pt}separation\ {\isacharcolon}{\kern0pt}\ \isanewline
\ \ \isakeyword{assumes}\ {\isachardoublequoteopen}{\isasymphi}{\isasymin}formula{\isachardoublequoteclose}\ {\isachardoublequoteopen}arity{\isacharparenleft}{\kern0pt}{\isasymphi}{\isacharparenright}{\kern0pt}{\isasymle}length{\isacharparenleft}{\kern0pt}env{\isacharparenright}{\kern0pt}{\isachardoublequoteclose}\ {\isachardoublequoteopen}env{\isasymin}list{\isacharparenleft}{\kern0pt}M{\isacharparenright}{\kern0pt}{\isachardoublequoteclose}\isanewline
\ \ \isakeyword{shows}\ {\isachardoublequoteopen}separation{\isacharparenleft}{\kern0pt}{\isacharhash}{\kern0pt}{\isacharhash}{\kern0pt}M{\isacharcomma}{\kern0pt}\ {\isasymlambda}p{\isachardot}{\kern0pt}\ p\ {\isasymtturnstile}HS\ {\isasymphi}\ env{\isacharparenright}{\kern0pt}{\isachardoublequoteclose}\ \isanewline
%
\isadelimproof
%
\endisadelimproof
%
\isatagproof
\isacommand{proof}\isamarkupfalse%
\ {\isacharminus}{\kern0pt}\ \isanewline
\ \ \isacommand{have}\isamarkupfalse%
\ {\isachardoublequoteopen}z{\isasymin}P\ {\isasymLongrightarrow}\ z{\isasymin}M{\isachardoublequoteclose}\ \isakeyword{for}\ z\isanewline
\ \ \ \ \isacommand{using}\isamarkupfalse%
\ P{\isacharunderscore}{\kern0pt}in{\isacharunderscore}{\kern0pt}M\ transitivity{\isacharbrackleft}{\kern0pt}of\ z\ P{\isacharbrackright}{\kern0pt}\ \isacommand{by}\isamarkupfalse%
\ simp\isanewline
\ \ \isacommand{moreover}\isamarkupfalse%
\isanewline
\ \ \isacommand{have}\isamarkupfalse%
\ {\isachardoublequoteopen}separation{\isacharparenleft}{\kern0pt}{\isacharhash}{\kern0pt}{\isacharhash}{\kern0pt}M{\isacharcomma}{\kern0pt}\ {\isasymlambda}p{\isachardot}{\kern0pt}\ {\isacharparenleft}{\kern0pt}M{\isacharcomma}{\kern0pt}\ {\isacharbrackleft}{\kern0pt}p{\isacharbrackright}{\kern0pt}\ {\isacharat}{\kern0pt}\ {\isacharbrackleft}{\kern0pt}P{\isacharcomma}{\kern0pt}\ leq{\isacharcomma}{\kern0pt}\ one{\isacharcomma}{\kern0pt}\ {\isasymlangle}{\isasymF}{\isacharcomma}{\kern0pt}\ {\isasymG}{\isacharcomma}{\kern0pt}\ P{\isacharcomma}{\kern0pt}\ P{\isacharunderscore}{\kern0pt}auto{\isasymrangle}{\isacharbrackright}{\kern0pt}\ {\isacharat}{\kern0pt}\ env\ {\isasymTurnstile}\ forcesHS{\isacharparenleft}{\kern0pt}{\isasymphi}{\isacharparenright}{\kern0pt}{\isacharparenright}{\kern0pt}{\isacharparenright}{\kern0pt}{\isachardoublequoteclose}\isanewline
\ \ \ \ \isacommand{apply}\isamarkupfalse%
{\isacharparenleft}{\kern0pt}rule\ separation{\isacharunderscore}{\kern0pt}ax{\isacharparenright}{\kern0pt}\isanewline
\ \ \ \ \ \ \isacommand{apply}\isamarkupfalse%
{\isacharparenleft}{\kern0pt}rule\ forcesHS{\isacharunderscore}{\kern0pt}type{\isacharcomma}{\kern0pt}\ simp\ add{\isacharcolon}{\kern0pt}assms{\isacharparenright}{\kern0pt}\isanewline
\ \ \ \ \isacommand{using}\isamarkupfalse%
\ P{\isacharunderscore}{\kern0pt}in{\isacharunderscore}{\kern0pt}M\ leq{\isacharunderscore}{\kern0pt}in{\isacharunderscore}{\kern0pt}M\ one{\isacharunderscore}{\kern0pt}in{\isacharunderscore}{\kern0pt}M\ {\isasymF}{\isacharunderscore}{\kern0pt}in{\isacharunderscore}{\kern0pt}M\ {\isasymG}{\isacharunderscore}{\kern0pt}in{\isacharunderscore}{\kern0pt}M\ P{\isacharunderscore}{\kern0pt}auto{\isacharunderscore}{\kern0pt}in{\isacharunderscore}{\kern0pt}M\ pair{\isacharunderscore}{\kern0pt}in{\isacharunderscore}{\kern0pt}M{\isacharunderscore}{\kern0pt}iff\ assms\ \isanewline
\ \ \ \ \ \isacommand{apply}\isamarkupfalse%
\ force\ \isanewline
\ \ \ \ \isacommand{apply}\isamarkupfalse%
\ simp\isanewline
\ \ \ \ \isacommand{apply}\isamarkupfalse%
{\isacharparenleft}{\kern0pt}rule\ le{\isacharunderscore}{\kern0pt}trans{\isacharcomma}{\kern0pt}\ rule\ arity{\isacharunderscore}{\kern0pt}forcesHS{\isacharcomma}{\kern0pt}\ simp\ add{\isacharcolon}{\kern0pt}assms{\isacharparenright}{\kern0pt}\isanewline
\ \ \ \ \isacommand{using}\isamarkupfalse%
\ assms\ \isanewline
\ \ \ \ \isacommand{by}\isamarkupfalse%
\ auto\isanewline
\ \ \isacommand{then}\isamarkupfalse%
\ \isacommand{show}\isamarkupfalse%
\ {\isacharquery}{\kern0pt}thesis\ \isacommand{by}\isamarkupfalse%
\ auto\isanewline
\isacommand{qed}\isamarkupfalse%
%
\endisatagproof
{\isafoldproof}%
%
\isadelimproof
\isanewline
%
\endisadelimproof
\isanewline
\isacommand{lemma}\isamarkupfalse%
\ Collect{\isacharunderscore}{\kern0pt}ForcesHS\ {\isacharcolon}{\kern0pt}\isanewline
\ \ \isakeyword{assumes}\ \isanewline
\ \ \ \ fty{\isacharcolon}{\kern0pt}\ {\isachardoublequoteopen}{\isasymphi}{\isasymin}formula{\isachardoublequoteclose}\ \isakeyword{and}\isanewline
\ \ \ \ far{\isacharcolon}{\kern0pt}\ {\isachardoublequoteopen}arity{\isacharparenleft}{\kern0pt}{\isasymphi}{\isacharparenright}{\kern0pt}{\isasymle}length{\isacharparenleft}{\kern0pt}env{\isacharparenright}{\kern0pt}{\isachardoublequoteclose}\ \isakeyword{and}\isanewline
\ \ \ \ envty{\isacharcolon}{\kern0pt}\ {\isachardoublequoteopen}env{\isasymin}list{\isacharparenleft}{\kern0pt}M{\isacharparenright}{\kern0pt}{\isachardoublequoteclose}\isanewline
\ \ \isakeyword{shows}\isanewline
\ \ \ \ {\isachardoublequoteopen}{\isacharbraceleft}{\kern0pt}p{\isasymin}P\ {\isachardot}{\kern0pt}\ p\ {\isasymtturnstile}HS\ {\isasymphi}\ env{\isacharbraceright}{\kern0pt}\ {\isasymin}\ M{\isachardoublequoteclose}\isanewline
%
\isadelimproof
%
\endisadelimproof
%
\isatagproof
\isacommand{proof}\isamarkupfalse%
\ {\isacharminus}{\kern0pt}\isanewline
\ \ \isacommand{have}\isamarkupfalse%
\ {\isachardoublequoteopen}z{\isasymin}P\ {\isasymLongrightarrow}\ z{\isasymin}M{\isachardoublequoteclose}\ \isakeyword{for}\ z\isanewline
\ \ \ \ \isacommand{using}\isamarkupfalse%
\ P{\isacharunderscore}{\kern0pt}in{\isacharunderscore}{\kern0pt}M\ transitivity{\isacharbrackleft}{\kern0pt}of\ z\ P{\isacharbrackright}{\kern0pt}\ \isacommand{by}\isamarkupfalse%
\ simp\isanewline
\ \ \isacommand{moreover}\isamarkupfalse%
\isanewline
\ \ \isacommand{have}\isamarkupfalse%
\ {\isachardoublequoteopen}separation{\isacharparenleft}{\kern0pt}{\isacharhash}{\kern0pt}{\isacharhash}{\kern0pt}M{\isacharcomma}{\kern0pt}{\isasymlambda}p{\isachardot}{\kern0pt}\ {\isacharparenleft}{\kern0pt}p\ {\isasymtturnstile}HS\ {\isasymphi}\ env{\isacharparenright}{\kern0pt}{\isacharparenright}{\kern0pt}{\isachardoublequoteclose}\isanewline
\ \ \ \ \isacommand{apply}\isamarkupfalse%
{\isacharparenleft}{\kern0pt}rule\ ForcesHS{\isacharunderscore}{\kern0pt}separation{\isacharparenright}{\kern0pt}\isanewline
\ \ \ \ \isacommand{using}\isamarkupfalse%
\ assms\isanewline
\ \ \ \ \isacommand{by}\isamarkupfalse%
\ auto\isanewline
\ \ \isacommand{then}\isamarkupfalse%
\ \isanewline
\ \ \isacommand{have}\isamarkupfalse%
\ {\isachardoublequoteopen}Collect{\isacharparenleft}{\kern0pt}P{\isacharcomma}{\kern0pt}{\isasymlambda}p{\isachardot}{\kern0pt}\ {\isacharparenleft}{\kern0pt}p\ {\isasymtturnstile}HS\ {\isasymphi}\ env{\isacharparenright}{\kern0pt}{\isacharparenright}{\kern0pt}{\isasymin}M{\isachardoublequoteclose}\isanewline
\ \ \ \ \isacommand{using}\isamarkupfalse%
\ separation{\isacharunderscore}{\kern0pt}closed\ P{\isacharunderscore}{\kern0pt}in{\isacharunderscore}{\kern0pt}M\ \isacommand{by}\isamarkupfalse%
\ simp\isanewline
\ \ \isacommand{then}\isamarkupfalse%
\ \isacommand{show}\isamarkupfalse%
\ {\isacharquery}{\kern0pt}thesis\ \isacommand{by}\isamarkupfalse%
\ simp\isanewline
\isacommand{qed}\isamarkupfalse%
%
\endisatagproof
{\isafoldproof}%
%
\isadelimproof
\isanewline
%
\endisadelimproof
\isanewline
\isacommand{lemma}\isamarkupfalse%
\ ForcesHS{\isacharunderscore}{\kern0pt}And{\isacharunderscore}{\kern0pt}aux{\isacharcolon}{\kern0pt}\isanewline
\ \ \isakeyword{assumes}\isanewline
\ \ \ \ {\isachardoublequoteopen}p{\isasymin}P{\isachardoublequoteclose}\ {\isachardoublequoteopen}env\ {\isasymin}\ list{\isacharparenleft}{\kern0pt}M{\isacharparenright}{\kern0pt}{\isachardoublequoteclose}\ {\isachardoublequoteopen}{\isasymphi}{\isasymin}formula{\isachardoublequoteclose}\ {\isachardoublequoteopen}{\isasympsi}{\isasymin}formula{\isachardoublequoteclose}\isanewline
\ \ \isakeyword{shows}\isanewline
\ \ \ \ {\isachardoublequoteopen}p\ {\isasymtturnstile}HS\ And{\isacharparenleft}{\kern0pt}{\isasymphi}{\isacharcomma}{\kern0pt}{\isasympsi}{\isacharparenright}{\kern0pt}\ env\ \ \ {\isasymlongleftrightarrow}\ \isanewline
\ \ \ \ {\isacharparenleft}{\kern0pt}{\isasymforall}q{\isasymin}M{\isachardot}{\kern0pt}\ q{\isasymin}P\ {\isasymand}\ q{\isasympreceq}p\ {\isasymlongrightarrow}\ {\isacharparenleft}{\kern0pt}{\isasymexists}r{\isasymin}M{\isachardot}{\kern0pt}\ r{\isasymin}P\ {\isasymand}\ r{\isasympreceq}q\ {\isasymand}\ {\isacharparenleft}{\kern0pt}r\ {\isasymtturnstile}HS\ {\isasymphi}\ env{\isacharparenright}{\kern0pt}\ {\isasymand}\ {\isacharparenleft}{\kern0pt}r\ {\isasymtturnstile}HS\ {\isasympsi}\ env{\isacharparenright}{\kern0pt}{\isacharparenright}{\kern0pt}{\isacharparenright}{\kern0pt}{\isachardoublequoteclose}\isanewline
%
\isadelimproof
\ \ %
\endisadelimproof
%
\isatagproof
\isacommand{unfolding}\isamarkupfalse%
\ And{\isacharunderscore}{\kern0pt}def\ \isacommand{using}\isamarkupfalse%
\ assms\ ForcesHS{\isacharunderscore}{\kern0pt}Neg\ ForcesHS{\isacharunderscore}{\kern0pt}Nand\ \isacommand{by}\isamarkupfalse%
\ {\isacharparenleft}{\kern0pt}auto\ simp\ only{\isacharcolon}{\kern0pt}{\isacharparenright}{\kern0pt}%
\endisatagproof
{\isafoldproof}%
%
\isadelimproof
\isanewline
%
\endisadelimproof
\isanewline
\isacommand{lemma}\isamarkupfalse%
\ ForcesHS{\isacharunderscore}{\kern0pt}And{\isacharunderscore}{\kern0pt}iff{\isacharunderscore}{\kern0pt}dense{\isacharunderscore}{\kern0pt}below{\isacharcolon}{\kern0pt}\isanewline
\ \ \isakeyword{assumes}\isanewline
\ \ \ \ {\isachardoublequoteopen}p{\isasymin}P{\isachardoublequoteclose}\ {\isachardoublequoteopen}env\ {\isasymin}\ list{\isacharparenleft}{\kern0pt}M{\isacharparenright}{\kern0pt}{\isachardoublequoteclose}\ {\isachardoublequoteopen}{\isasymphi}{\isasymin}formula{\isachardoublequoteclose}\ {\isachardoublequoteopen}{\isasympsi}{\isasymin}formula{\isachardoublequoteclose}\isanewline
\ \ \isakeyword{shows}\isanewline
\ \ \ \ {\isachardoublequoteopen}{\isacharparenleft}{\kern0pt}p\ {\isasymtturnstile}HS\ And{\isacharparenleft}{\kern0pt}{\isasymphi}{\isacharcomma}{\kern0pt}{\isasympsi}{\isacharparenright}{\kern0pt}\ env{\isacharparenright}{\kern0pt}\ {\isasymlongleftrightarrow}\ dense{\isacharunderscore}{\kern0pt}below{\isacharparenleft}{\kern0pt}{\isacharbraceleft}{\kern0pt}r{\isasymin}P{\isachardot}{\kern0pt}\ {\isacharparenleft}{\kern0pt}r\ {\isasymtturnstile}HS\ {\isasymphi}\ env{\isacharparenright}{\kern0pt}\ {\isasymand}\ {\isacharparenleft}{\kern0pt}r\ {\isasymtturnstile}HS\ {\isasympsi}\ env{\isacharparenright}{\kern0pt}\ {\isacharbraceright}{\kern0pt}{\isacharcomma}{\kern0pt}p{\isacharparenright}{\kern0pt}{\isachardoublequoteclose}\isanewline
%
\isadelimproof
\ \ %
\endisadelimproof
%
\isatagproof
\isacommand{unfolding}\isamarkupfalse%
\ dense{\isacharunderscore}{\kern0pt}below{\isacharunderscore}{\kern0pt}def\ \isacommand{using}\isamarkupfalse%
\ ForcesHS{\isacharunderscore}{\kern0pt}And{\isacharunderscore}{\kern0pt}aux\ assms\isanewline
\ \ \ \ \isacommand{by}\isamarkupfalse%
\ {\isacharparenleft}{\kern0pt}auto\ dest{\isacharcolon}{\kern0pt}transitivity{\isacharbrackleft}{\kern0pt}OF\ {\isacharunderscore}{\kern0pt}\ P{\isacharunderscore}{\kern0pt}in{\isacharunderscore}{\kern0pt}M{\isacharbrackright}{\kern0pt}{\isacharsemicolon}{\kern0pt}\ rename{\isacharunderscore}{\kern0pt}tac\ q{\isacharsemicolon}{\kern0pt}\ drule{\isacharunderscore}{\kern0pt}tac\ x{\isacharequal}{\kern0pt}q\ \isakeyword{in}\ bspec{\isacharparenright}{\kern0pt}{\isacharplus}{\kern0pt}%
\endisatagproof
{\isafoldproof}%
%
\isadelimproof
\isanewline
%
\endisadelimproof
\isanewline
\isacommand{lemma}\isamarkupfalse%
\ dense{\isacharunderscore}{\kern0pt}below{\isacharunderscore}{\kern0pt}imp{\isacharunderscore}{\kern0pt}forcesHS{\isacharcolon}{\kern0pt}\isanewline
\ \ \isakeyword{assumes}\ \isanewline
\ \ \ \ {\isachardoublequoteopen}p{\isasymin}P{\isachardoublequoteclose}\ {\isachardoublequoteopen}{\isasymphi}{\isasymin}formula{\isachardoublequoteclose}\isanewline
\ \ \isakeyword{shows}\isanewline
\ \ \ \ {\isachardoublequoteopen}{\isasymAnd}env{\isachardot}{\kern0pt}\ env{\isasymin}list{\isacharparenleft}{\kern0pt}M{\isacharparenright}{\kern0pt}\ {\isasymLongrightarrow}\ arity{\isacharparenleft}{\kern0pt}{\isasymphi}{\isacharparenright}{\kern0pt}{\isasymle}length{\isacharparenleft}{\kern0pt}env{\isacharparenright}{\kern0pt}\ {\isasymLongrightarrow}\isanewline
\ \ \ \ \ dense{\isacharunderscore}{\kern0pt}below{\isacharparenleft}{\kern0pt}{\isacharbraceleft}{\kern0pt}q{\isasymin}P{\isachardot}{\kern0pt}\ {\isacharparenleft}{\kern0pt}q\ {\isasymtturnstile}HS\ {\isasymphi}\ env{\isacharparenright}{\kern0pt}{\isacharbraceright}{\kern0pt}{\isacharcomma}{\kern0pt}p{\isacharparenright}{\kern0pt}\ {\isasymLongrightarrow}\ {\isacharparenleft}{\kern0pt}p\ {\isasymtturnstile}HS\ {\isasymphi}\ env{\isacharparenright}{\kern0pt}{\isachardoublequoteclose}\isanewline
%
\isadelimproof
\ \ %
\endisadelimproof
%
\isatagproof
\isacommand{using}\isamarkupfalse%
\ assms{\isacharparenleft}{\kern0pt}{\isadigit{2}}{\isacharparenright}{\kern0pt}\isanewline
\isacommand{proof}\isamarkupfalse%
\ {\isacharparenleft}{\kern0pt}induct{\isacharparenright}{\kern0pt}\isanewline
\ \ \isacommand{case}\isamarkupfalse%
\ {\isacharparenleft}{\kern0pt}Member\ n\ m{\isacharparenright}{\kern0pt}\isanewline
\ \ \isacommand{then}\isamarkupfalse%
\ \isacommand{have}\isamarkupfalse%
\ {\isachardoublequoteopen}{\isacharbraceleft}{\kern0pt}q{\isasymin}P{\isachardot}{\kern0pt}\ {\isacharparenleft}{\kern0pt}q\ {\isasymtturnstile}HS\ {\isacharparenleft}{\kern0pt}Member{\isacharparenleft}{\kern0pt}n{\isacharcomma}{\kern0pt}\ m{\isacharparenright}{\kern0pt}{\isacharparenright}{\kern0pt}\ env{\isacharparenright}{\kern0pt}{\isacharbraceright}{\kern0pt}\ {\isacharequal}{\kern0pt}\ {\isacharbraceleft}{\kern0pt}q{\isasymin}P{\isachardot}{\kern0pt}\ {\isacharparenleft}{\kern0pt}q\ {\isasymtturnstile}\ {\isacharparenleft}{\kern0pt}Member{\isacharparenleft}{\kern0pt}n{\isacharcomma}{\kern0pt}\ m{\isacharparenright}{\kern0pt}{\isacharparenright}{\kern0pt}\ env{\isacharparenright}{\kern0pt}{\isacharbraceright}{\kern0pt}{\isachardoublequoteclose}\ \isanewline
\ \ \ \ \isacommand{apply}\isamarkupfalse%
{\isacharparenleft}{\kern0pt}rule{\isacharunderscore}{\kern0pt}tac\ iff{\isacharunderscore}{\kern0pt}eq{\isacharparenright}{\kern0pt}\isanewline
\ \ \ \ \isacommand{apply}\isamarkupfalse%
{\isacharparenleft}{\kern0pt}rule\ iff{\isacharunderscore}{\kern0pt}flip{\isacharparenright}{\kern0pt}\isanewline
\ \ \ \ \isacommand{apply}\isamarkupfalse%
{\isacharparenleft}{\kern0pt}rule\ ForcesHS{\isacharunderscore}{\kern0pt}Member{\isacharparenright}{\kern0pt}\isanewline
\ \ \ \ \isacommand{using}\isamarkupfalse%
\ transM\ P{\isacharunderscore}{\kern0pt}in{\isacharunderscore}{\kern0pt}M\isanewline
\ \ \ \ \isacommand{by}\isamarkupfalse%
\ auto\isanewline
\ \ \isacommand{then}\isamarkupfalse%
\ \isacommand{have}\isamarkupfalse%
\ {\isachardoublequoteopen}dense{\isacharunderscore}{\kern0pt}below{\isacharparenleft}{\kern0pt}{\isacharbraceleft}{\kern0pt}q{\isasymin}P{\isachardot}{\kern0pt}\ {\isacharparenleft}{\kern0pt}q\ {\isasymtturnstile}\ {\isacharparenleft}{\kern0pt}Member{\isacharparenleft}{\kern0pt}n{\isacharcomma}{\kern0pt}\ m{\isacharparenright}{\kern0pt}{\isacharparenright}{\kern0pt}\ env{\isacharparenright}{\kern0pt}{\isacharbraceright}{\kern0pt}{\isacharcomma}{\kern0pt}\ p{\isacharparenright}{\kern0pt}{\isachardoublequoteclose}\ \isanewline
\ \ \ \ \isacommand{using}\isamarkupfalse%
\ Member\ \isanewline
\ \ \ \ \isacommand{by}\isamarkupfalse%
\ auto\isanewline
\ \ \isacommand{then}\isamarkupfalse%
\ \isacommand{have}\isamarkupfalse%
\ {\isachardoublequoteopen}p\ {\isasymtturnstile}\ Member{\isacharparenleft}{\kern0pt}n{\isacharcomma}{\kern0pt}\ m{\isacharparenright}{\kern0pt}\ env{\isachardoublequoteclose}\ \isanewline
\ \ \ \ \isacommand{apply}\isamarkupfalse%
{\isacharparenleft}{\kern0pt}rule{\isacharunderscore}{\kern0pt}tac\ dense{\isacharunderscore}{\kern0pt}below{\isacharunderscore}{\kern0pt}imp{\isacharunderscore}{\kern0pt}forces{\isacharparenright}{\kern0pt}\isanewline
\ \ \ \ \isacommand{using}\isamarkupfalse%
\ Member\ assms\isanewline
\ \ \ \ \isacommand{by}\isamarkupfalse%
\ auto\isanewline
\ \ \isacommand{then}\isamarkupfalse%
\ \isacommand{show}\isamarkupfalse%
\ {\isachardoublequoteopen}p\ {\isasymtturnstile}HS\ Member{\isacharparenleft}{\kern0pt}n{\isacharcomma}{\kern0pt}\ m{\isacharparenright}{\kern0pt}\ env{\isachardoublequoteclose}\ \isanewline
\ \ \ \ \isacommand{apply}\isamarkupfalse%
{\isacharparenleft}{\kern0pt}rule{\isacharunderscore}{\kern0pt}tac\ iffD{\isadigit{1}}{\isacharparenright}{\kern0pt}\isanewline
\ \ \ \ \isacommand{apply}\isamarkupfalse%
{\isacharparenleft}{\kern0pt}rule\ ForcesHS{\isacharunderscore}{\kern0pt}Member{\isacharparenright}{\kern0pt}\isanewline
\ \ \ \ \isacommand{using}\isamarkupfalse%
\ Member\ transM\ P{\isacharunderscore}{\kern0pt}in{\isacharunderscore}{\kern0pt}M\ assms\ \isanewline
\ \ \ \ \isacommand{by}\isamarkupfalse%
\ auto\isanewline
\isacommand{next}\isamarkupfalse%
\isanewline
\ \ \isacommand{case}\isamarkupfalse%
\ {\isacharparenleft}{\kern0pt}Equal\ n\ m{\isacharparenright}{\kern0pt}\isanewline
\ \ \isacommand{then}\isamarkupfalse%
\ \isacommand{have}\isamarkupfalse%
\ {\isachardoublequoteopen}{\isacharbraceleft}{\kern0pt}q{\isasymin}P{\isachardot}{\kern0pt}\ {\isacharparenleft}{\kern0pt}q\ {\isasymtturnstile}HS\ {\isacharparenleft}{\kern0pt}Equal{\isacharparenleft}{\kern0pt}n{\isacharcomma}{\kern0pt}\ m{\isacharparenright}{\kern0pt}{\isacharparenright}{\kern0pt}\ env{\isacharparenright}{\kern0pt}{\isacharbraceright}{\kern0pt}\ {\isacharequal}{\kern0pt}\ {\isacharbraceleft}{\kern0pt}q{\isasymin}P{\isachardot}{\kern0pt}\ {\isacharparenleft}{\kern0pt}q\ {\isasymtturnstile}\ {\isacharparenleft}{\kern0pt}Equal{\isacharparenleft}{\kern0pt}n{\isacharcomma}{\kern0pt}\ m{\isacharparenright}{\kern0pt}{\isacharparenright}{\kern0pt}\ env{\isacharparenright}{\kern0pt}{\isacharbraceright}{\kern0pt}{\isachardoublequoteclose}\ \isanewline
\ \ \ \ \isacommand{apply}\isamarkupfalse%
{\isacharparenleft}{\kern0pt}rule{\isacharunderscore}{\kern0pt}tac\ iff{\isacharunderscore}{\kern0pt}eq{\isacharparenright}{\kern0pt}\isanewline
\ \ \ \ \isacommand{apply}\isamarkupfalse%
{\isacharparenleft}{\kern0pt}rule\ iff{\isacharunderscore}{\kern0pt}flip{\isacharparenright}{\kern0pt}\isanewline
\ \ \ \ \isacommand{apply}\isamarkupfalse%
{\isacharparenleft}{\kern0pt}rule\ ForcesHS{\isacharunderscore}{\kern0pt}Equal{\isacharparenright}{\kern0pt}\isanewline
\ \ \ \ \isacommand{using}\isamarkupfalse%
\ transM\ P{\isacharunderscore}{\kern0pt}in{\isacharunderscore}{\kern0pt}M\isanewline
\ \ \ \ \isacommand{by}\isamarkupfalse%
\ auto\isanewline
\ \ \isacommand{then}\isamarkupfalse%
\ \isacommand{have}\isamarkupfalse%
\ {\isachardoublequoteopen}dense{\isacharunderscore}{\kern0pt}below{\isacharparenleft}{\kern0pt}{\isacharbraceleft}{\kern0pt}q{\isasymin}P{\isachardot}{\kern0pt}\ {\isacharparenleft}{\kern0pt}q\ {\isasymtturnstile}\ {\isacharparenleft}{\kern0pt}Equal{\isacharparenleft}{\kern0pt}n{\isacharcomma}{\kern0pt}\ m{\isacharparenright}{\kern0pt}{\isacharparenright}{\kern0pt}\ env{\isacharparenright}{\kern0pt}{\isacharbraceright}{\kern0pt}{\isacharcomma}{\kern0pt}\ p{\isacharparenright}{\kern0pt}{\isachardoublequoteclose}\ \isanewline
\ \ \ \ \isacommand{using}\isamarkupfalse%
\ Equal\ \isanewline
\ \ \ \ \isacommand{by}\isamarkupfalse%
\ auto\isanewline
\ \ \isacommand{then}\isamarkupfalse%
\ \isacommand{have}\isamarkupfalse%
\ {\isachardoublequoteopen}p\ {\isasymtturnstile}\ Equal{\isacharparenleft}{\kern0pt}n{\isacharcomma}{\kern0pt}\ m{\isacharparenright}{\kern0pt}\ env{\isachardoublequoteclose}\ \isanewline
\ \ \ \ \isacommand{apply}\isamarkupfalse%
{\isacharparenleft}{\kern0pt}rule{\isacharunderscore}{\kern0pt}tac\ dense{\isacharunderscore}{\kern0pt}below{\isacharunderscore}{\kern0pt}imp{\isacharunderscore}{\kern0pt}forces{\isacharparenright}{\kern0pt}\isanewline
\ \ \ \ \isacommand{using}\isamarkupfalse%
\ Equal\ assms\isanewline
\ \ \ \ \isacommand{by}\isamarkupfalse%
\ auto\isanewline
\ \ \isacommand{then}\isamarkupfalse%
\ \isacommand{show}\isamarkupfalse%
\ {\isachardoublequoteopen}p\ {\isasymtturnstile}HS\ Equal{\isacharparenleft}{\kern0pt}n{\isacharcomma}{\kern0pt}\ m{\isacharparenright}{\kern0pt}\ env{\isachardoublequoteclose}\ \isanewline
\ \ \ \ \isacommand{apply}\isamarkupfalse%
{\isacharparenleft}{\kern0pt}rule{\isacharunderscore}{\kern0pt}tac\ iffD{\isadigit{1}}{\isacharparenright}{\kern0pt}\isanewline
\ \ \ \ \isacommand{apply}\isamarkupfalse%
{\isacharparenleft}{\kern0pt}rule\ ForcesHS{\isacharunderscore}{\kern0pt}Equal{\isacharparenright}{\kern0pt}\isanewline
\ \ \ \ \isacommand{using}\isamarkupfalse%
\ Equal\ transM\ P{\isacharunderscore}{\kern0pt}in{\isacharunderscore}{\kern0pt}M\ assms\ \isanewline
\ \ \ \ \isacommand{by}\isamarkupfalse%
\ auto\isanewline
\isacommand{next}\isamarkupfalse%
\isanewline
\isacommand{case}\isamarkupfalse%
\ {\isacharparenleft}{\kern0pt}Nand\ {\isasymphi}\ {\isasympsi}{\isacharparenright}{\kern0pt}\isanewline
\ \ \isacommand{{\isacharbraceleft}{\kern0pt}}\isamarkupfalse%
\ \ \isanewline
\ \ \ \ \isacommand{fix}\isamarkupfalse%
\ q\isanewline
\ \ \ \ \isacommand{assume}\isamarkupfalse%
\ {\isachardoublequoteopen}q{\isasymin}M{\isachardoublequoteclose}\ {\isachardoublequoteopen}q{\isasymin}P{\isachardoublequoteclose}\ {\isachardoublequoteopen}q{\isasympreceq}\ p{\isachardoublequoteclose}\ {\isachardoublequoteopen}q\ {\isasymtturnstile}HS\ {\isasymphi}\ env{\isachardoublequoteclose}\isanewline
\ \ \ \ \isacommand{moreover}\isamarkupfalse%
\ \isanewline
\ \ \ \ \isacommand{note}\isamarkupfalse%
\ Nand\isanewline
\ \ \ \ \isacommand{moreover}\isamarkupfalse%
\ \isacommand{from}\isamarkupfalse%
\ calculation\isanewline
\ \ \ \ \isacommand{obtain}\isamarkupfalse%
\ d\ \isakeyword{where}\ {\isachardoublequoteopen}d{\isasymin}P{\isachardoublequoteclose}\ {\isachardoublequoteopen}d\ {\isasymtturnstile}HS\ Nand{\isacharparenleft}{\kern0pt}{\isasymphi}{\isacharcomma}{\kern0pt}\ {\isasympsi}{\isacharparenright}{\kern0pt}\ env{\isachardoublequoteclose}\ {\isachardoublequoteopen}d{\isasympreceq}\ q{\isachardoublequoteclose}\isanewline
\ \ \ \ \ \ \isacommand{using}\isamarkupfalse%
\ dense{\isacharunderscore}{\kern0pt}belowI\ \isacommand{by}\isamarkupfalse%
\ auto\isanewline
\ \ \ \ \isacommand{moreover}\isamarkupfalse%
\ \isacommand{from}\isamarkupfalse%
\ calculation\isanewline
\ \ \ \ \isacommand{have}\isamarkupfalse%
\ {\isachardoublequoteopen}{\isasymnot}{\isacharparenleft}{\kern0pt}d{\isasymtturnstile}HS\ {\isasympsi}\ env{\isacharparenright}{\kern0pt}{\isachardoublequoteclose}\ \isakeyword{if}\ {\isachardoublequoteopen}d\ {\isasymtturnstile}HS\ {\isasymphi}\ env{\isachardoublequoteclose}\isanewline
\ \ \ \ \ \ \isacommand{using}\isamarkupfalse%
\ that\ ForcesHS{\isacharunderscore}{\kern0pt}Nand\ leq{\isacharunderscore}{\kern0pt}reflI\ transitivity{\isacharbrackleft}{\kern0pt}OF\ {\isacharunderscore}{\kern0pt}\ P{\isacharunderscore}{\kern0pt}in{\isacharunderscore}{\kern0pt}M{\isacharcomma}{\kern0pt}\ of\ d{\isacharbrackright}{\kern0pt}\ \isacommand{by}\isamarkupfalse%
\ auto\isanewline
\ \ \ \ \isacommand{moreover}\isamarkupfalse%
\ \isanewline
\ \ \ \ \isacommand{note}\isamarkupfalse%
\ arity{\isacharunderscore}{\kern0pt}Nand{\isacharunderscore}{\kern0pt}le{\isacharbrackleft}{\kern0pt}of\ {\isasymphi}\ {\isasympsi}{\isacharbrackright}{\kern0pt}\isanewline
\ \ \ \ \isacommand{moreover}\isamarkupfalse%
\ \isacommand{from}\isamarkupfalse%
\ calculation\isanewline
\ \ \ \ \isacommand{have}\isamarkupfalse%
\ {\isachardoublequoteopen}d\ {\isasymtturnstile}HS\ {\isasymphi}\ env{\isachardoublequoteclose}\ \isanewline
\ \ \ \ \ \ \ \isacommand{using}\isamarkupfalse%
\ HS{\isacharunderscore}{\kern0pt}strengthening{\isacharunderscore}{\kern0pt}lemma{\isacharbrackleft}{\kern0pt}of\ q\ {\isasymphi}\ d\ env{\isacharbrackright}{\kern0pt}\ Un{\isacharunderscore}{\kern0pt}leD{\isadigit{1}}\ \isacommand{by}\isamarkupfalse%
\ auto\isanewline
\ \ \ \ \isacommand{ultimately}\isamarkupfalse%
\isanewline
\ \ \ \ \isacommand{have}\isamarkupfalse%
\ {\isachardoublequoteopen}{\isasymnot}\ {\isacharparenleft}{\kern0pt}q\ {\isasymtturnstile}HS\ {\isasympsi}\ env{\isacharparenright}{\kern0pt}{\isachardoublequoteclose}\isanewline
\ \ \ \ \ \ \isacommand{using}\isamarkupfalse%
\ HS{\isacharunderscore}{\kern0pt}strengthening{\isacharunderscore}{\kern0pt}lemma{\isacharbrackleft}{\kern0pt}of\ q\ {\isasympsi}\ d\ env{\isacharbrackright}{\kern0pt}\ \isacommand{by}\isamarkupfalse%
\ auto\isanewline
\ \ \isacommand{{\isacharbraceright}{\kern0pt}}\isamarkupfalse%
\isanewline
\ \ \isacommand{with}\isamarkupfalse%
\ {\isacartoucheopen}p{\isasymin}P{\isacartoucheclose}\isanewline
\ \ \isacommand{show}\isamarkupfalse%
\ {\isacharquery}{\kern0pt}case\isanewline
\ \ \ \ \isacommand{using}\isamarkupfalse%
\ ForcesHS{\isacharunderscore}{\kern0pt}Nand{\isacharbrackleft}{\kern0pt}symmetric{\isacharcomma}{\kern0pt}\ OF\ {\isacharunderscore}{\kern0pt}\ Nand{\isacharparenleft}{\kern0pt}{\isadigit{5}}{\isacharcomma}{\kern0pt}{\isadigit{1}}{\isacharcomma}{\kern0pt}{\isadigit{3}}{\isacharparenright}{\kern0pt}{\isacharbrackright}{\kern0pt}\ \isacommand{by}\isamarkupfalse%
\ blast\isanewline
\isacommand{next}\isamarkupfalse%
\isanewline
\ \ \isacommand{case}\isamarkupfalse%
\ {\isacharparenleft}{\kern0pt}Forall\ {\isasymphi}{\isacharparenright}{\kern0pt}\isanewline
\isanewline
\ \ \isacommand{then}\isamarkupfalse%
\ \isacommand{have}\isamarkupfalse%
\ H\ {\isacharcolon}{\kern0pt}\ {\isachardoublequoteopen}{\isasymAnd}env{\isachardot}{\kern0pt}\ env\ {\isasymin}\ list{\isacharparenleft}{\kern0pt}M{\isacharparenright}{\kern0pt}\ {\isasymLongrightarrow}\isanewline
\ \ \ \ \ \ \ \ \ \ \ \ arity{\isacharparenleft}{\kern0pt}{\isasymphi}{\isacharparenright}{\kern0pt}\ {\isasymle}\ length{\isacharparenleft}{\kern0pt}env{\isacharparenright}{\kern0pt}\ {\isasymLongrightarrow}\isanewline
\ \ \ \ \ \ \ \ \ \ \ \ local{\isachardot}{\kern0pt}dense{\isacharunderscore}{\kern0pt}below{\isacharparenleft}{\kern0pt}{\isacharbraceleft}{\kern0pt}q\ {\isasymin}\ P\ {\isachardot}{\kern0pt}\ M{\isacharcomma}{\kern0pt}\ {\isacharbrackleft}{\kern0pt}q{\isacharcomma}{\kern0pt}\ P{\isacharcomma}{\kern0pt}\ leq{\isacharcomma}{\kern0pt}\ one{\isacharcomma}{\kern0pt}\ {\isasymlangle}{\isasymF}{\isacharcomma}{\kern0pt}\ {\isasymG}{\isacharcomma}{\kern0pt}\ P{\isacharcomma}{\kern0pt}\ P{\isacharunderscore}{\kern0pt}auto{\isasymrangle}{\isacharbrackright}{\kern0pt}\ {\isacharat}{\kern0pt}\ env\ {\isasymTurnstile}\ forcesHS{\isacharparenleft}{\kern0pt}{\isasymphi}{\isacharparenright}{\kern0pt}{\isacharbraceright}{\kern0pt}{\isacharcomma}{\kern0pt}\ p{\isacharparenright}{\kern0pt}\ {\isasymLongrightarrow}\isanewline
\ \ \ \ \ \ \ \ \ \ \ \ M{\isacharcomma}{\kern0pt}\ Cons{\isacharparenleft}{\kern0pt}p{\isacharcomma}{\kern0pt}\ Cons{\isacharparenleft}{\kern0pt}P{\isacharcomma}{\kern0pt}\ Cons{\isacharparenleft}{\kern0pt}leq{\isacharcomma}{\kern0pt}\ Cons{\isacharparenleft}{\kern0pt}one{\isacharcomma}{\kern0pt}\ Cons{\isacharparenleft}{\kern0pt}{\isasymlangle}{\isasymF}{\isacharcomma}{\kern0pt}\ {\isasymG}{\isacharcomma}{\kern0pt}\ P{\isacharcomma}{\kern0pt}\ P{\isacharunderscore}{\kern0pt}auto{\isasymrangle}{\isacharcomma}{\kern0pt}\ env{\isacharparenright}{\kern0pt}{\isacharparenright}{\kern0pt}{\isacharparenright}{\kern0pt}{\isacharparenright}{\kern0pt}{\isacharparenright}{\kern0pt}\ {\isasymTurnstile}\ forcesHS{\isacharparenleft}{\kern0pt}{\isasymphi}{\isacharparenright}{\kern0pt}{\isachardoublequoteclose}\ \isacommand{by}\isamarkupfalse%
\ auto\isanewline
\isanewline
\ \ \isacommand{have}\isamarkupfalse%
\ {\isachardoublequoteopen}dense{\isacharunderscore}{\kern0pt}below{\isacharparenleft}{\kern0pt}{\isacharbraceleft}{\kern0pt}q{\isasymin}P{\isachardot}{\kern0pt}\ q\ {\isasymtturnstile}HS\ {\isasymphi}\ {\isacharparenleft}{\kern0pt}{\isacharbrackleft}{\kern0pt}a{\isacharbrackright}{\kern0pt}{\isacharat}{\kern0pt}env{\isacharparenright}{\kern0pt}{\isacharbraceright}{\kern0pt}{\isacharcomma}{\kern0pt}p{\isacharparenright}{\kern0pt}{\isachardoublequoteclose}\ \isakeyword{if}\ {\isachardoublequoteopen}a{\isasymin}HS{\isachardoublequoteclose}\ \isakeyword{for}\ a\isanewline
\ \ \isacommand{proof}\isamarkupfalse%
\isanewline
\ \ \ \ \isacommand{fix}\isamarkupfalse%
\ r\isanewline
\ \ \ \ \isacommand{assume}\isamarkupfalse%
\ {\isachardoublequoteopen}r{\isasymin}P{\isachardoublequoteclose}\ {\isachardoublequoteopen}r{\isasympreceq}p{\isachardoublequoteclose}\isanewline
\ \ \ \ \isacommand{with}\isamarkupfalse%
\ {\isacartoucheopen}dense{\isacharunderscore}{\kern0pt}below{\isacharparenleft}{\kern0pt}{\isacharunderscore}{\kern0pt}{\isacharcomma}{\kern0pt}p{\isacharparenright}{\kern0pt}{\isacartoucheclose}\isanewline
\ \ \ \ \isacommand{obtain}\isamarkupfalse%
\ q\ \isakeyword{where}\ {\isachardoublequoteopen}q{\isasymin}P{\isachardoublequoteclose}\ {\isachardoublequoteopen}q{\isasympreceq}r{\isachardoublequoteclose}\ {\isachardoublequoteopen}q\ {\isasymtturnstile}HS\ Forall{\isacharparenleft}{\kern0pt}{\isasymphi}{\isacharparenright}{\kern0pt}\ env{\isachardoublequoteclose}\isanewline
\ \ \ \ \ \ \isacommand{by}\isamarkupfalse%
\ blast\isanewline
\ \ \ \ \isacommand{moreover}\isamarkupfalse%
\isanewline
\ \ \ \ \isacommand{note}\isamarkupfalse%
\ Forall\ {\isacartoucheopen}a{\isasymin}HS{\isacartoucheclose}\isanewline
\ \ \ \ \isacommand{moreover}\isamarkupfalse%
\ \isacommand{from}\isamarkupfalse%
\ calculation\isanewline
\ \ \ \ \isacommand{have}\isamarkupfalse%
\ {\isachardoublequoteopen}q\ {\isasymtturnstile}HS\ {\isasymphi}\ {\isacharparenleft}{\kern0pt}{\isacharbrackleft}{\kern0pt}a{\isacharbrackright}{\kern0pt}{\isacharat}{\kern0pt}env{\isacharparenright}{\kern0pt}{\isachardoublequoteclose}\isanewline
\ \ \ \ \ \ \isacommand{using}\isamarkupfalse%
\ ForcesHS{\isacharunderscore}{\kern0pt}Forall\ \isacommand{by}\isamarkupfalse%
\ simp\isanewline
\ \ \ \ \isacommand{ultimately}\isamarkupfalse%
\isanewline
\ \ \ \ \isacommand{show}\isamarkupfalse%
\ {\isachardoublequoteopen}{\isasymexists}d\ {\isasymin}\ {\isacharbraceleft}{\kern0pt}q{\isasymin}P{\isachardot}{\kern0pt}\ q\ {\isasymtturnstile}HS\ {\isasymphi}\ {\isacharparenleft}{\kern0pt}{\isacharbrackleft}{\kern0pt}a{\isacharbrackright}{\kern0pt}{\isacharat}{\kern0pt}env{\isacharparenright}{\kern0pt}{\isacharbraceright}{\kern0pt}{\isachardot}{\kern0pt}\ d\ {\isasymin}\ P\ {\isasymand}\ d{\isasympreceq}r{\isachardoublequoteclose}\isanewline
\ \ \ \ \ \ \isacommand{by}\isamarkupfalse%
\ auto\isanewline
\ \ \isacommand{qed}\isamarkupfalse%
\isanewline
\isanewline
\ \ \isacommand{moreover}\isamarkupfalse%
\isanewline
\ \ \isacommand{note}\isamarkupfalse%
\ Forall{\isacharparenleft}{\kern0pt}{\isadigit{2}}{\isacharparenright}{\kern0pt}{\isacharbrackleft}{\kern0pt}of\ {\isachardoublequoteopen}Cons{\isacharparenleft}{\kern0pt}{\isacharunderscore}{\kern0pt}{\isacharcomma}{\kern0pt}env{\isacharparenright}{\kern0pt}{\isachardoublequoteclose}{\isacharbrackright}{\kern0pt}\ Forall{\isacharparenleft}{\kern0pt}{\isadigit{1}}{\isacharcomma}{\kern0pt}{\isadigit{3}}{\isacharminus}{\kern0pt}{\isadigit{5}}{\isacharparenright}{\kern0pt}\isanewline
\ \ \isacommand{ultimately}\isamarkupfalse%
\isanewline
\ \ \isacommand{have}\isamarkupfalse%
\ {\isachardoublequoteopen}p\ {\isasymtturnstile}HS\ {\isasymphi}\ {\isacharparenleft}{\kern0pt}{\isacharbrackleft}{\kern0pt}a{\isacharbrackright}{\kern0pt}{\isacharat}{\kern0pt}env{\isacharparenright}{\kern0pt}{\isachardoublequoteclose}\ \isakeyword{if}\ {\isachardoublequoteopen}a{\isasymin}HS{\isachardoublequoteclose}\ \isakeyword{for}\ a\isanewline
\ \ \ \ \isacommand{apply}\isamarkupfalse%
\ simp\isanewline
\ \ \ \ \isacommand{apply}\isamarkupfalse%
{\isacharparenleft}{\kern0pt}rule\ H{\isacharparenright}{\kern0pt}\isanewline
\ \ \ \ \isacommand{using}\isamarkupfalse%
\ that\ HS{\isacharunderscore}{\kern0pt}iff\ P{\isacharunderscore}{\kern0pt}name{\isacharunderscore}{\kern0pt}in{\isacharunderscore}{\kern0pt}M\ pred{\isacharunderscore}{\kern0pt}le\isanewline
\ \ \ \ \ \ \isacommand{apply}\isamarkupfalse%
\ force\isanewline
\ \ \ \ \isacommand{using}\isamarkupfalse%
\ that\ HS{\isacharunderscore}{\kern0pt}iff\ P{\isacharunderscore}{\kern0pt}name{\isacharunderscore}{\kern0pt}in{\isacharunderscore}{\kern0pt}M\ pred{\isacharunderscore}{\kern0pt}le\isanewline
\ \ \ \ \ \isacommand{apply}\isamarkupfalse%
\ simp\isanewline
\ \ \ \ \ \isacommand{apply}\isamarkupfalse%
{\isacharparenleft}{\kern0pt}rule{\isacharunderscore}{\kern0pt}tac\ n{\isacharequal}{\kern0pt}{\isachardoublequoteopen}arity{\isacharparenleft}{\kern0pt}{\isasymphi}{\isacharparenright}{\kern0pt}{\isachardoublequoteclose}\ \isakeyword{in}\ natE{\isacharcomma}{\kern0pt}\ simp{\isacharcomma}{\kern0pt}\ force{\isacharcomma}{\kern0pt}\ force{\isacharparenright}{\kern0pt}\isanewline
\ \ \ \ \isacommand{using}\isamarkupfalse%
\ that\ pred{\isacharunderscore}{\kern0pt}le{\isadigit{2}}\ \isanewline
\ \ \ \ \isacommand{by}\isamarkupfalse%
\ simp\isanewline
\ \ \isacommand{with}\isamarkupfalse%
\ assms\ Forall\isanewline
\ \ \isacommand{show}\isamarkupfalse%
\ {\isacharquery}{\kern0pt}case\ \isacommand{using}\isamarkupfalse%
\ ForcesHS{\isacharunderscore}{\kern0pt}Forall\ \isacommand{by}\isamarkupfalse%
\ simp\isanewline
\isacommand{qed}\isamarkupfalse%
%
\endisatagproof
{\isafoldproof}%
%
\isadelimproof
\isanewline
%
\endisadelimproof
\isanewline
\isacommand{lemma}\isamarkupfalse%
\ HS{\isacharunderscore}{\kern0pt}density{\isacharunderscore}{\kern0pt}lemma{\isacharcolon}{\kern0pt}\isanewline
\ \ \isakeyword{assumes}\isanewline
\ \ \ \ {\isachardoublequoteopen}p{\isasymin}P{\isachardoublequoteclose}\ {\isachardoublequoteopen}{\isasymphi}{\isasymin}formula{\isachardoublequoteclose}\ {\isachardoublequoteopen}env{\isasymin}list{\isacharparenleft}{\kern0pt}M{\isacharparenright}{\kern0pt}{\isachardoublequoteclose}\ {\isachardoublequoteopen}arity{\isacharparenleft}{\kern0pt}{\isasymphi}{\isacharparenright}{\kern0pt}{\isasymle}length{\isacharparenleft}{\kern0pt}env{\isacharparenright}{\kern0pt}{\isachardoublequoteclose}\isanewline
\ \ \isakeyword{shows}\isanewline
\ \ \ \ {\isachardoublequoteopen}p\ {\isasymtturnstile}HS\ {\isasymphi}\ env\ \ \ {\isasymlongleftrightarrow}\ \ \ dense{\isacharunderscore}{\kern0pt}below{\isacharparenleft}{\kern0pt}{\isacharbraceleft}{\kern0pt}q{\isasymin}P{\isachardot}{\kern0pt}\ {\isacharparenleft}{\kern0pt}q\ {\isasymtturnstile}HS\ {\isasymphi}\ env{\isacharparenright}{\kern0pt}{\isacharbraceright}{\kern0pt}{\isacharcomma}{\kern0pt}p{\isacharparenright}{\kern0pt}{\isachardoublequoteclose}\isanewline
%
\isadelimproof
%
\endisadelimproof
%
\isatagproof
\isacommand{proof}\isamarkupfalse%
\isanewline
\ \ \isacommand{assume}\isamarkupfalse%
\ {\isachardoublequoteopen}dense{\isacharunderscore}{\kern0pt}below{\isacharparenleft}{\kern0pt}{\isacharbraceleft}{\kern0pt}q{\isasymin}P{\isachardot}{\kern0pt}\ {\isacharparenleft}{\kern0pt}q\ {\isasymtturnstile}HS\ {\isasymphi}\ env{\isacharparenright}{\kern0pt}{\isacharbraceright}{\kern0pt}{\isacharcomma}{\kern0pt}p{\isacharparenright}{\kern0pt}{\isachardoublequoteclose}\isanewline
\ \ \isacommand{with}\isamarkupfalse%
\ assms\isanewline
\ \ \isacommand{show}\isamarkupfalse%
\ \ {\isachardoublequoteopen}{\isacharparenleft}{\kern0pt}p\ {\isasymtturnstile}HS\ {\isasymphi}\ env{\isacharparenright}{\kern0pt}{\isachardoublequoteclose}\isanewline
\ \ \ \ \isacommand{using}\isamarkupfalse%
\ dense{\isacharunderscore}{\kern0pt}below{\isacharunderscore}{\kern0pt}imp{\isacharunderscore}{\kern0pt}forcesHS\ \isacommand{by}\isamarkupfalse%
\ simp\isanewline
\isacommand{next}\isamarkupfalse%
\isanewline
\ \ \isacommand{assume}\isamarkupfalse%
\ {\isachardoublequoteopen}p\ {\isasymtturnstile}HS\ {\isasymphi}\ env{\isachardoublequoteclose}\isanewline
\ \ \isacommand{with}\isamarkupfalse%
\ assms\isanewline
\ \ \isacommand{show}\isamarkupfalse%
\ {\isachardoublequoteopen}dense{\isacharunderscore}{\kern0pt}below{\isacharparenleft}{\kern0pt}{\isacharbraceleft}{\kern0pt}q{\isasymin}P{\isachardot}{\kern0pt}\ q\ {\isasymtturnstile}HS\ {\isasymphi}\ env{\isacharbraceright}{\kern0pt}{\isacharcomma}{\kern0pt}p{\isacharparenright}{\kern0pt}{\isachardoublequoteclose}\isanewline
\ \ \ \ \isacommand{using}\isamarkupfalse%
\ HS{\isacharunderscore}{\kern0pt}strengthening{\isacharunderscore}{\kern0pt}lemma\ leq{\isacharunderscore}{\kern0pt}reflI\ \isacommand{by}\isamarkupfalse%
\ auto\isanewline
\isacommand{qed}\isamarkupfalse%
%
\endisatagproof
{\isafoldproof}%
%
\isadelimproof
\isanewline
%
\endisadelimproof
\isanewline
\isacommand{lemma}\isamarkupfalse%
\ ForcesHS{\isacharunderscore}{\kern0pt}And{\isacharcolon}{\kern0pt}\isanewline
\ \ \isakeyword{assumes}\isanewline
\ \ \ \ {\isachardoublequoteopen}p{\isasymin}P{\isachardoublequoteclose}\ {\isachardoublequoteopen}env\ {\isasymin}\ list{\isacharparenleft}{\kern0pt}M{\isacharparenright}{\kern0pt}{\isachardoublequoteclose}\ {\isachardoublequoteopen}{\isasymphi}{\isasymin}formula{\isachardoublequoteclose}\ {\isachardoublequoteopen}{\isasympsi}{\isasymin}formula{\isachardoublequoteclose}\ \isanewline
\ \ \ \ {\isachardoublequoteopen}arity{\isacharparenleft}{\kern0pt}{\isasymphi}{\isacharparenright}{\kern0pt}\ {\isasymle}\ length{\isacharparenleft}{\kern0pt}env{\isacharparenright}{\kern0pt}{\isachardoublequoteclose}\ {\isachardoublequoteopen}arity{\isacharparenleft}{\kern0pt}{\isasympsi}{\isacharparenright}{\kern0pt}\ {\isasymle}\ length{\isacharparenleft}{\kern0pt}env{\isacharparenright}{\kern0pt}{\isachardoublequoteclose}\isanewline
\ \ \isakeyword{shows}\isanewline
\ \ \ \ {\isachardoublequoteopen}p\ {\isasymtturnstile}HS\ And{\isacharparenleft}{\kern0pt}{\isasymphi}{\isacharcomma}{\kern0pt}{\isasympsi}{\isacharparenright}{\kern0pt}\ env\ \ \ {\isasymlongleftrightarrow}\ \ {\isacharparenleft}{\kern0pt}p\ {\isasymtturnstile}HS\ {\isasymphi}\ env{\isacharparenright}{\kern0pt}\ {\isasymand}\ {\isacharparenleft}{\kern0pt}p\ {\isasymtturnstile}HS\ {\isasympsi}\ env{\isacharparenright}{\kern0pt}{\isachardoublequoteclose}\isanewline
%
\isadelimproof
%
\endisadelimproof
%
\isatagproof
\isacommand{proof}\isamarkupfalse%
\isanewline
\ \ \isacommand{assume}\isamarkupfalse%
\ {\isachardoublequoteopen}p\ {\isasymtturnstile}HS\ And{\isacharparenleft}{\kern0pt}{\isasymphi}{\isacharcomma}{\kern0pt}\ {\isasympsi}{\isacharparenright}{\kern0pt}\ env{\isachardoublequoteclose}\isanewline
\ \ \isacommand{with}\isamarkupfalse%
\ assms\isanewline
\ \ \isacommand{have}\isamarkupfalse%
\ {\isachardoublequoteopen}dense{\isacharunderscore}{\kern0pt}below{\isacharparenleft}{\kern0pt}{\isacharbraceleft}{\kern0pt}r\ {\isasymin}\ P\ {\isachardot}{\kern0pt}\ {\isacharparenleft}{\kern0pt}r\ {\isasymtturnstile}HS\ {\isasymphi}\ env{\isacharparenright}{\kern0pt}\ {\isasymand}\ {\isacharparenleft}{\kern0pt}r\ {\isasymtturnstile}HS\ {\isasympsi}\ env{\isacharparenright}{\kern0pt}{\isacharbraceright}{\kern0pt}{\isacharcomma}{\kern0pt}\ p{\isacharparenright}{\kern0pt}{\isachardoublequoteclose}\isanewline
\ \ \ \ \isacommand{using}\isamarkupfalse%
\ ForcesHS{\isacharunderscore}{\kern0pt}And{\isacharunderscore}{\kern0pt}iff{\isacharunderscore}{\kern0pt}dense{\isacharunderscore}{\kern0pt}below\ \isacommand{by}\isamarkupfalse%
\ simp\isanewline
\ \ \isacommand{then}\isamarkupfalse%
\isanewline
\ \ \isacommand{have}\isamarkupfalse%
\ {\isachardoublequoteopen}dense{\isacharunderscore}{\kern0pt}below{\isacharparenleft}{\kern0pt}{\isacharbraceleft}{\kern0pt}r\ {\isasymin}\ P\ {\isachardot}{\kern0pt}\ {\isacharparenleft}{\kern0pt}r\ {\isasymtturnstile}HS\ {\isasymphi}\ env{\isacharparenright}{\kern0pt}{\isacharbraceright}{\kern0pt}{\isacharcomma}{\kern0pt}\ p{\isacharparenright}{\kern0pt}{\isachardoublequoteclose}\ {\isachardoublequoteopen}dense{\isacharunderscore}{\kern0pt}below{\isacharparenleft}{\kern0pt}{\isacharbraceleft}{\kern0pt}r\ {\isasymin}\ P\ {\isachardot}{\kern0pt}\ {\isacharparenleft}{\kern0pt}r\ {\isasymtturnstile}HS\ {\isasympsi}\ env{\isacharparenright}{\kern0pt}{\isacharbraceright}{\kern0pt}{\isacharcomma}{\kern0pt}\ p{\isacharparenright}{\kern0pt}{\isachardoublequoteclose}\isanewline
\ \ \ \ \isacommand{by}\isamarkupfalse%
\ blast{\isacharplus}{\kern0pt}\isanewline
\ \ \isacommand{with}\isamarkupfalse%
\ assms\isanewline
\ \ \isacommand{show}\isamarkupfalse%
\ {\isachardoublequoteopen}{\isacharparenleft}{\kern0pt}p\ {\isasymtturnstile}HS\ {\isasymphi}\ env{\isacharparenright}{\kern0pt}\ {\isasymand}\ {\isacharparenleft}{\kern0pt}p\ {\isasymtturnstile}HS\ {\isasympsi}\ env{\isacharparenright}{\kern0pt}{\isachardoublequoteclose}\isanewline
\ \ \ \ \isacommand{using}\isamarkupfalse%
\ HS{\isacharunderscore}{\kern0pt}density{\isacharunderscore}{\kern0pt}lemma{\isacharbrackleft}{\kern0pt}symmetric{\isacharbrackright}{\kern0pt}\ \isacommand{by}\isamarkupfalse%
\ simp\isanewline
\isacommand{next}\isamarkupfalse%
\isanewline
\ \ \isacommand{assume}\isamarkupfalse%
\ {\isachardoublequoteopen}{\isacharparenleft}{\kern0pt}p\ {\isasymtturnstile}HS\ {\isasymphi}\ env{\isacharparenright}{\kern0pt}\ {\isasymand}\ {\isacharparenleft}{\kern0pt}p\ {\isasymtturnstile}HS\ {\isasympsi}\ env{\isacharparenright}{\kern0pt}{\isachardoublequoteclose}\isanewline
\ \ \isacommand{have}\isamarkupfalse%
\ {\isachardoublequoteopen}dense{\isacharunderscore}{\kern0pt}below{\isacharparenleft}{\kern0pt}{\isacharbraceleft}{\kern0pt}r\ {\isasymin}\ P\ {\isachardot}{\kern0pt}\ {\isacharparenleft}{\kern0pt}r\ {\isasymtturnstile}HS\ {\isasymphi}\ env{\isacharparenright}{\kern0pt}\ {\isasymand}\ {\isacharparenleft}{\kern0pt}r\ {\isasymtturnstile}HS\ {\isasympsi}\ env{\isacharparenright}{\kern0pt}{\isacharbraceright}{\kern0pt}{\isacharcomma}{\kern0pt}\ p{\isacharparenright}{\kern0pt}{\isachardoublequoteclose}\isanewline
\ \ \isacommand{proof}\isamarkupfalse%
\ {\isacharparenleft}{\kern0pt}intro\ dense{\isacharunderscore}{\kern0pt}belowI\ bexI\ conjI{\isacharcomma}{\kern0pt}\ assumption{\isacharparenright}{\kern0pt}\isanewline
\ \ \ \ \isacommand{fix}\isamarkupfalse%
\ q\isanewline
\ \ \ \ \isacommand{assume}\isamarkupfalse%
\ {\isachardoublequoteopen}q{\isasymin}P{\isachardoublequoteclose}\ {\isachardoublequoteopen}q{\isasympreceq}\ p{\isachardoublequoteclose}\isanewline
\ \ \ \ \isacommand{with}\isamarkupfalse%
\ assms\ {\isacartoucheopen}{\isacharparenleft}{\kern0pt}p\ {\isasymtturnstile}HS\ {\isasymphi}\ env{\isacharparenright}{\kern0pt}\ {\isasymand}\ {\isacharparenleft}{\kern0pt}p\ {\isasymtturnstile}HS\ {\isasympsi}\ env{\isacharparenright}{\kern0pt}{\isacartoucheclose}\isanewline
\ \ \ \ \isacommand{show}\isamarkupfalse%
\ {\isachardoublequoteopen}q{\isasymin}{\isacharbraceleft}{\kern0pt}r\ {\isasymin}\ P\ {\isachardot}{\kern0pt}\ {\isacharparenleft}{\kern0pt}r\ {\isasymtturnstile}HS\ {\isasymphi}\ env{\isacharparenright}{\kern0pt}\ {\isasymand}\ {\isacharparenleft}{\kern0pt}r\ {\isasymtturnstile}HS\ {\isasympsi}\ env{\isacharparenright}{\kern0pt}{\isacharbraceright}{\kern0pt}{\isachardoublequoteclose}\ {\isachardoublequoteopen}q{\isasympreceq}\ q{\isachardoublequoteclose}\isanewline
\ \ \ \ \ \ \isacommand{using}\isamarkupfalse%
\ HS{\isacharunderscore}{\kern0pt}strengthening{\isacharunderscore}{\kern0pt}lemma\ leq{\isacharunderscore}{\kern0pt}reflI\ \isacommand{by}\isamarkupfalse%
\ auto\isanewline
\ \ \isacommand{qed}\isamarkupfalse%
\isanewline
\ \ \isacommand{with}\isamarkupfalse%
\ assms\isanewline
\ \ \isacommand{show}\isamarkupfalse%
\ {\isachardoublequoteopen}p\ {\isasymtturnstile}HS\ And{\isacharparenleft}{\kern0pt}{\isasymphi}{\isacharcomma}{\kern0pt}{\isasympsi}{\isacharparenright}{\kern0pt}\ env{\isachardoublequoteclose}\isanewline
\ \ \ \ \isacommand{using}\isamarkupfalse%
\ ForcesHS{\isacharunderscore}{\kern0pt}And{\isacharunderscore}{\kern0pt}iff{\isacharunderscore}{\kern0pt}dense{\isacharunderscore}{\kern0pt}below\ \isacommand{by}\isamarkupfalse%
\ simp\isanewline
\isacommand{qed}\isamarkupfalse%
%
\endisatagproof
{\isafoldproof}%
%
\isadelimproof
\isanewline
%
\endisadelimproof
\isanewline
\isacommand{lemma}\isamarkupfalse%
\ ForcesHS{\isacharunderscore}{\kern0pt}Nand{\isacharunderscore}{\kern0pt}alt{\isacharcolon}{\kern0pt}\isanewline
\ \ \isakeyword{assumes}\isanewline
\ \ \ \ {\isachardoublequoteopen}p{\isasymin}P{\isachardoublequoteclose}\ {\isachardoublequoteopen}env\ {\isasymin}\ list{\isacharparenleft}{\kern0pt}M{\isacharparenright}{\kern0pt}{\isachardoublequoteclose}\ {\isachardoublequoteopen}{\isasymphi}{\isasymin}formula{\isachardoublequoteclose}\ {\isachardoublequoteopen}{\isasympsi}{\isasymin}formula{\isachardoublequoteclose}\ \isanewline
\ \ \ \ {\isachardoublequoteopen}arity{\isacharparenleft}{\kern0pt}{\isasymphi}{\isacharparenright}{\kern0pt}\ {\isasymle}\ length{\isacharparenleft}{\kern0pt}env{\isacharparenright}{\kern0pt}{\isachardoublequoteclose}\ {\isachardoublequoteopen}arity{\isacharparenleft}{\kern0pt}{\isasympsi}{\isacharparenright}{\kern0pt}\ {\isasymle}\ length{\isacharparenleft}{\kern0pt}env{\isacharparenright}{\kern0pt}{\isachardoublequoteclose}\isanewline
\ \ \isakeyword{shows}\isanewline
\ \ \ \ {\isachardoublequoteopen}{\isacharparenleft}{\kern0pt}p\ {\isasymtturnstile}HS\ Nand{\isacharparenleft}{\kern0pt}{\isasymphi}{\isacharcomma}{\kern0pt}{\isasympsi}{\isacharparenright}{\kern0pt}\ env{\isacharparenright}{\kern0pt}\ {\isasymlongleftrightarrow}\ {\isacharparenleft}{\kern0pt}p\ {\isasymtturnstile}HS\ Neg{\isacharparenleft}{\kern0pt}And{\isacharparenleft}{\kern0pt}{\isasymphi}{\isacharcomma}{\kern0pt}{\isasympsi}{\isacharparenright}{\kern0pt}{\isacharparenright}{\kern0pt}\ env{\isacharparenright}{\kern0pt}{\isachardoublequoteclose}\isanewline
%
\isadelimproof
\ \ %
\endisadelimproof
%
\isatagproof
\isacommand{using}\isamarkupfalse%
\ assms\ ForcesHS{\isacharunderscore}{\kern0pt}Nand\ ForcesHS{\isacharunderscore}{\kern0pt}And\ ForcesHS{\isacharunderscore}{\kern0pt}Neg\ \isacommand{by}\isamarkupfalse%
\ auto%
\endisatagproof
{\isafoldproof}%
%
\isadelimproof
\isanewline
%
\endisadelimproof
\isanewline
\isacommand{lemma}\isamarkupfalse%
\ HS{\isacharunderscore}{\kern0pt}truth{\isacharunderscore}{\kern0pt}lemma{\isacharunderscore}{\kern0pt}Neg{\isacharcolon}{\kern0pt}\isanewline
\ \ \isakeyword{assumes}\ \isanewline
\ \ \ \ {\isachardoublequoteopen}{\isasymphi}{\isasymin}formula{\isachardoublequoteclose}\ {\isachardoublequoteopen}M{\isacharunderscore}{\kern0pt}generic{\isacharparenleft}{\kern0pt}G{\isacharparenright}{\kern0pt}{\isachardoublequoteclose}\ {\isachardoublequoteopen}env{\isasymin}list{\isacharparenleft}{\kern0pt}HS{\isacharparenright}{\kern0pt}{\isachardoublequoteclose}\ {\isachardoublequoteopen}arity{\isacharparenleft}{\kern0pt}{\isasymphi}{\isacharparenright}{\kern0pt}{\isasymle}length{\isacharparenleft}{\kern0pt}env{\isacharparenright}{\kern0pt}{\isachardoublequoteclose}\ \isakeyword{and}\isanewline
\ \ \ \ IH{\isacharcolon}{\kern0pt}\ {\isachardoublequoteopen}{\isacharparenleft}{\kern0pt}{\isasymexists}p{\isasymin}G{\isachardot}{\kern0pt}\ p\ {\isasymtturnstile}HS\ {\isasymphi}\ env{\isacharparenright}{\kern0pt}\ {\isasymlongleftrightarrow}\ SymExt{\isacharparenleft}{\kern0pt}G{\isacharparenright}{\kern0pt}{\isacharcomma}{\kern0pt}\ map{\isacharparenleft}{\kern0pt}val{\isacharparenleft}{\kern0pt}G{\isacharparenright}{\kern0pt}{\isacharcomma}{\kern0pt}env{\isacharparenright}{\kern0pt}\ {\isasymTurnstile}\ {\isasymphi}{\isachardoublequoteclose}\isanewline
\ \ \isakeyword{shows}\isanewline
\ \ \ \ {\isachardoublequoteopen}{\isacharparenleft}{\kern0pt}{\isasymexists}p{\isasymin}G{\isachardot}{\kern0pt}\ p\ {\isasymtturnstile}HS\ Neg{\isacharparenleft}{\kern0pt}{\isasymphi}{\isacharparenright}{\kern0pt}\ env{\isacharparenright}{\kern0pt}\ \ {\isasymlongleftrightarrow}\ \ SymExt{\isacharparenleft}{\kern0pt}G{\isacharparenright}{\kern0pt}{\isacharcomma}{\kern0pt}\ map{\isacharparenleft}{\kern0pt}val{\isacharparenleft}{\kern0pt}G{\isacharparenright}{\kern0pt}{\isacharcomma}{\kern0pt}env{\isacharparenright}{\kern0pt}\ {\isasymTurnstile}\ Neg{\isacharparenleft}{\kern0pt}{\isasymphi}{\isacharparenright}{\kern0pt}{\isachardoublequoteclose}\isanewline
%
\isadelimproof
%
\endisadelimproof
%
\isatagproof
\isacommand{proof}\isamarkupfalse%
\ {\isacharparenleft}{\kern0pt}intro\ iffI{\isacharcomma}{\kern0pt}\ elim\ bexE{\isacharcomma}{\kern0pt}\ rule\ ccontr{\isacharparenright}{\kern0pt}\ \isanewline
\ \ \isanewline
\ \ \isacommand{fix}\isamarkupfalse%
\ p\ \isanewline
\ \ \isacommand{assume}\isamarkupfalse%
\ assms{\isadigit{1}}\ {\isacharcolon}{\kern0pt}\ {\isachardoublequoteopen}p{\isasymin}G{\isachardoublequoteclose}\ {\isachardoublequoteopen}p\ {\isasymtturnstile}HS\ Neg{\isacharparenleft}{\kern0pt}{\isasymphi}{\isacharparenright}{\kern0pt}\ env{\isachardoublequoteclose}\ {\isachardoublequoteopen}{\isasymnot}{\isacharparenleft}{\kern0pt}SymExt{\isacharparenleft}{\kern0pt}G{\isacharparenright}{\kern0pt}{\isacharcomma}{\kern0pt}map{\isacharparenleft}{\kern0pt}val{\isacharparenleft}{\kern0pt}G{\isacharparenright}{\kern0pt}{\isacharcomma}{\kern0pt}env{\isacharparenright}{\kern0pt}\ {\isasymTurnstile}\ Neg{\isacharparenleft}{\kern0pt}{\isasymphi}{\isacharparenright}{\kern0pt}{\isacharparenright}{\kern0pt}{\isachardoublequoteclose}\isanewline
\isanewline
\ \ \isacommand{have}\isamarkupfalse%
\ envin\ {\isacharcolon}{\kern0pt}\ {\isachardoublequoteopen}env\ {\isasymin}\ list{\isacharparenleft}{\kern0pt}M{\isacharparenright}{\kern0pt}{\isachardoublequoteclose}\ \isanewline
\ \ \ \ \isacommand{apply}\isamarkupfalse%
{\isacharparenleft}{\kern0pt}rule{\isacharunderscore}{\kern0pt}tac\ A{\isacharequal}{\kern0pt}{\isachardoublequoteopen}list{\isacharparenleft}{\kern0pt}HS{\isacharparenright}{\kern0pt}{\isachardoublequoteclose}\ \isakeyword{in}\ subsetD{\isacharcomma}{\kern0pt}\ rule\ list{\isacharunderscore}{\kern0pt}mono{\isacharparenright}{\kern0pt}\isanewline
\ \ \ \ \isacommand{using}\isamarkupfalse%
\ HS{\isacharunderscore}{\kern0pt}iff\ P{\isacharunderscore}{\kern0pt}name{\isacharunderscore}{\kern0pt}in{\isacharunderscore}{\kern0pt}M\ assms\isanewline
\ \ \ \ \isacommand{by}\isamarkupfalse%
\ auto\isanewline
\ \ \isacommand{have}\isamarkupfalse%
\ mapin\ {\isacharcolon}{\kern0pt}\ {\isachardoublequoteopen}map{\isacharparenleft}{\kern0pt}val{\isacharparenleft}{\kern0pt}G{\isacharparenright}{\kern0pt}{\isacharcomma}{\kern0pt}\ env{\isacharparenright}{\kern0pt}\ {\isasymin}\ list{\isacharparenleft}{\kern0pt}SymExt{\isacharparenleft}{\kern0pt}G{\isacharparenright}{\kern0pt}{\isacharparenright}{\kern0pt}{\isachardoublequoteclose}\ \isanewline
\ \ \ \ \isacommand{apply}\isamarkupfalse%
{\isacharparenleft}{\kern0pt}rule\ map{\isacharunderscore}{\kern0pt}type{\isacharparenright}{\kern0pt}\isanewline
\ \ \ \ \isacommand{using}\isamarkupfalse%
\ assms\ SymExt{\isacharunderscore}{\kern0pt}def\isanewline
\ \ \ \ \isacommand{by}\isamarkupfalse%
\ auto\isanewline
\ \isanewline
\ \ \isacommand{note}\isamarkupfalse%
\ assms\ envin\ mapin\ assms{\isadigit{1}}\isanewline
\ \ \isacommand{moreover}\isamarkupfalse%
\ \isacommand{from}\isamarkupfalse%
\ calculation\isanewline
\ \ \isacommand{have}\isamarkupfalse%
\ {\isachardoublequoteopen}SymExt{\isacharparenleft}{\kern0pt}G{\isacharparenright}{\kern0pt}{\isacharcomma}{\kern0pt}\ map{\isacharparenleft}{\kern0pt}val{\isacharparenleft}{\kern0pt}G{\isacharparenright}{\kern0pt}{\isacharcomma}{\kern0pt}env{\isacharparenright}{\kern0pt}\ {\isasymTurnstile}\ {\isasymphi}{\isachardoublequoteclose}\isanewline
\ \ \ \ \isacommand{by}\isamarkupfalse%
\ auto\isanewline
\ \ \isacommand{with}\isamarkupfalse%
\ IH\isanewline
\ \ \isacommand{obtain}\isamarkupfalse%
\ r\ \isakeyword{where}\ {\isachardoublequoteopen}r\ {\isasymtturnstile}HS\ {\isasymphi}\ env{\isachardoublequoteclose}\ {\isachardoublequoteopen}r{\isasymin}G{\isachardoublequoteclose}\ \isacommand{by}\isamarkupfalse%
\ auto\isanewline
\ \ \isacommand{moreover}\isamarkupfalse%
\ \isacommand{from}\isamarkupfalse%
\ this\ \isakeyword{and}\ {\isacartoucheopen}M{\isacharunderscore}{\kern0pt}generic{\isacharparenleft}{\kern0pt}G{\isacharparenright}{\kern0pt}{\isacartoucheclose}\ {\isacartoucheopen}p{\isasymin}G{\isacartoucheclose}\isanewline
\ \ \isacommand{obtain}\isamarkupfalse%
\ q\ \isakeyword{where}\ {\isachardoublequoteopen}q{\isasympreceq}p{\isachardoublequoteclose}\ {\isachardoublequoteopen}q{\isasympreceq}r{\isachardoublequoteclose}\ {\isachardoublequoteopen}q{\isasymin}G{\isachardoublequoteclose}\isanewline
\ \ \ \ \isacommand{by}\isamarkupfalse%
\ blast\isanewline
\ \ \isacommand{moreover}\isamarkupfalse%
\ \isacommand{from}\isamarkupfalse%
\ calculation\ \isanewline
\ \ \isacommand{have}\isamarkupfalse%
\ {\isachardoublequoteopen}q\ {\isasymtturnstile}HS\ {\isasymphi}\ env{\isachardoublequoteclose}\isanewline
\ \ \ \ \isacommand{using}\isamarkupfalse%
\ HS{\isacharunderscore}{\kern0pt}strengthening{\isacharunderscore}{\kern0pt}lemma{\isacharbrackleft}{\kern0pt}\isakeyword{where}\ {\isasymphi}{\isacharequal}{\kern0pt}{\isasymphi}{\isacharbrackright}{\kern0pt}\ \isacommand{by}\isamarkupfalse%
\ blast\isanewline
\ \ \isacommand{ultimately}\isamarkupfalse%
\isanewline
\ \ \isacommand{show}\isamarkupfalse%
\ {\isachardoublequoteopen}False{\isachardoublequoteclose}\isanewline
\ \ \ \ \isacommand{using}\isamarkupfalse%
\ ForcesHS{\isacharunderscore}{\kern0pt}Neg{\isacharbrackleft}{\kern0pt}\isakeyword{where}\ {\isasymphi}{\isacharequal}{\kern0pt}{\isasymphi}{\isacharbrackright}{\kern0pt}\ transitivity{\isacharbrackleft}{\kern0pt}OF\ {\isacharunderscore}{\kern0pt}\ P{\isacharunderscore}{\kern0pt}in{\isacharunderscore}{\kern0pt}M{\isacharbrackright}{\kern0pt}\ \isacommand{by}\isamarkupfalse%
\ blast\isanewline
\isacommand{next}\isamarkupfalse%
\isanewline
\ \ \isacommand{have}\isamarkupfalse%
\ envin\ {\isacharcolon}{\kern0pt}\ {\isachardoublequoteopen}env\ {\isasymin}\ list{\isacharparenleft}{\kern0pt}M{\isacharparenright}{\kern0pt}{\isachardoublequoteclose}\ \isanewline
\ \ \ \ \isacommand{apply}\isamarkupfalse%
{\isacharparenleft}{\kern0pt}rule{\isacharunderscore}{\kern0pt}tac\ A{\isacharequal}{\kern0pt}{\isachardoublequoteopen}list{\isacharparenleft}{\kern0pt}HS{\isacharparenright}{\kern0pt}{\isachardoublequoteclose}\ \isakeyword{in}\ subsetD{\isacharcomma}{\kern0pt}\ rule\ list{\isacharunderscore}{\kern0pt}mono{\isacharparenright}{\kern0pt}\isanewline
\ \ \ \ \isacommand{using}\isamarkupfalse%
\ HS{\isacharunderscore}{\kern0pt}iff\ P{\isacharunderscore}{\kern0pt}name{\isacharunderscore}{\kern0pt}in{\isacharunderscore}{\kern0pt}M\ assms\isanewline
\ \ \ \ \isacommand{by}\isamarkupfalse%
\ auto\isanewline
\ \ \isacommand{have}\isamarkupfalse%
\ mapin\ {\isacharcolon}{\kern0pt}\ {\isachardoublequoteopen}map{\isacharparenleft}{\kern0pt}val{\isacharparenleft}{\kern0pt}G{\isacharparenright}{\kern0pt}{\isacharcomma}{\kern0pt}\ env{\isacharparenright}{\kern0pt}\ {\isasymin}\ list{\isacharparenleft}{\kern0pt}SymExt{\isacharparenleft}{\kern0pt}G{\isacharparenright}{\kern0pt}{\isacharparenright}{\kern0pt}{\isachardoublequoteclose}\ \isanewline
\ \ \ \ \isacommand{apply}\isamarkupfalse%
{\isacharparenleft}{\kern0pt}rule\ map{\isacharunderscore}{\kern0pt}type{\isacharparenright}{\kern0pt}\isanewline
\ \ \ \ \isacommand{using}\isamarkupfalse%
\ assms\ SymExt{\isacharunderscore}{\kern0pt}def\isanewline
\ \ \ \ \isacommand{by}\isamarkupfalse%
\ auto\isanewline
\isanewline
\ \ \isacommand{assume}\isamarkupfalse%
\ {\isachardoublequoteopen}SymExt{\isacharparenleft}{\kern0pt}G{\isacharparenright}{\kern0pt}{\isacharcomma}{\kern0pt}\ map{\isacharparenleft}{\kern0pt}val{\isacharparenleft}{\kern0pt}G{\isacharparenright}{\kern0pt}{\isacharcomma}{\kern0pt}env{\isacharparenright}{\kern0pt}\ {\isasymTurnstile}\ Neg{\isacharparenleft}{\kern0pt}{\isasymphi}{\isacharparenright}{\kern0pt}{\isachardoublequoteclose}\isanewline
\ \ \isacommand{with}\isamarkupfalse%
\ assms\ \isanewline
\ \ \isacommand{have}\isamarkupfalse%
\ {\isachardoublequoteopen}{\isasymnot}\ {\isacharparenleft}{\kern0pt}SymExt{\isacharparenleft}{\kern0pt}G{\isacharparenright}{\kern0pt}{\isacharcomma}{\kern0pt}\ map{\isacharparenleft}{\kern0pt}val{\isacharparenleft}{\kern0pt}G{\isacharparenright}{\kern0pt}{\isacharcomma}{\kern0pt}env{\isacharparenright}{\kern0pt}\ {\isasymTurnstile}\ {\isasymphi}{\isacharparenright}{\kern0pt}{\isachardoublequoteclose}\isanewline
\ \ \ \ \isacommand{using}\isamarkupfalse%
\ mapin\ \isacommand{by}\isamarkupfalse%
\ simp\isanewline
\ \ \isacommand{let}\isamarkupfalse%
\ {\isacharquery}{\kern0pt}D{\isacharequal}{\kern0pt}{\isachardoublequoteopen}{\isacharbraceleft}{\kern0pt}p{\isasymin}P{\isachardot}{\kern0pt}\ {\isacharparenleft}{\kern0pt}p\ {\isasymtturnstile}HS\ {\isasymphi}\ env{\isacharparenright}{\kern0pt}\ {\isasymor}\ {\isacharparenleft}{\kern0pt}p\ {\isasymtturnstile}HS\ Neg{\isacharparenleft}{\kern0pt}{\isasymphi}{\isacharparenright}{\kern0pt}\ env{\isacharparenright}{\kern0pt}{\isacharbraceright}{\kern0pt}{\isachardoublequoteclose}\isanewline
\ \ \isacommand{have}\isamarkupfalse%
\ {\isachardoublequoteopen}separation{\isacharparenleft}{\kern0pt}{\isacharhash}{\kern0pt}{\isacharhash}{\kern0pt}M{\isacharcomma}{\kern0pt}{\isasymlambda}p{\isachardot}{\kern0pt}\ {\isacharparenleft}{\kern0pt}p\ {\isasymtturnstile}HS\ {\isasymphi}\ env{\isacharparenright}{\kern0pt}{\isacharparenright}{\kern0pt}{\isachardoublequoteclose}\ \isanewline
\ \ \ \ \isacommand{apply}\isamarkupfalse%
{\isacharparenleft}{\kern0pt}rule\ ForcesHS{\isacharunderscore}{\kern0pt}separation{\isacharparenright}{\kern0pt}\isanewline
\ \ \ \ \isacommand{using}\isamarkupfalse%
\ assms\ envin\isanewline
\ \ \ \ \isacommand{by}\isamarkupfalse%
\ auto\isanewline
\ \ \isacommand{moreover}\isamarkupfalse%
\isanewline
\ \ \isacommand{have}\isamarkupfalse%
\ {\isachardoublequoteopen}separation{\isacharparenleft}{\kern0pt}{\isacharhash}{\kern0pt}{\isacharhash}{\kern0pt}M{\isacharcomma}{\kern0pt}{\isasymlambda}p{\isachardot}{\kern0pt}\ {\isacharparenleft}{\kern0pt}p\ {\isasymtturnstile}HS\ Neg{\isacharparenleft}{\kern0pt}{\isasymphi}{\isacharparenright}{\kern0pt}\ env{\isacharparenright}{\kern0pt}{\isacharparenright}{\kern0pt}{\isachardoublequoteclose}\isanewline
\ \ \ \ \isacommand{apply}\isamarkupfalse%
{\isacharparenleft}{\kern0pt}rule\ ForcesHS{\isacharunderscore}{\kern0pt}separation{\isacharparenright}{\kern0pt}\isanewline
\ \ \ \ \isacommand{using}\isamarkupfalse%
\ assms\ envin\isanewline
\ \ \ \ \isacommand{by}\isamarkupfalse%
\ auto\isanewline
\ \ \isacommand{ultimately}\isamarkupfalse%
\isanewline
\ \ \isacommand{have}\isamarkupfalse%
\ {\isachardoublequoteopen}separation{\isacharparenleft}{\kern0pt}{\isacharhash}{\kern0pt}{\isacharhash}{\kern0pt}M{\isacharcomma}{\kern0pt}{\isasymlambda}p{\isachardot}{\kern0pt}\ {\isacharparenleft}{\kern0pt}p\ {\isasymtturnstile}HS\ {\isasymphi}\ env{\isacharparenright}{\kern0pt}\ {\isasymor}\ {\isacharparenleft}{\kern0pt}p\ {\isasymtturnstile}HS\ Neg{\isacharparenleft}{\kern0pt}{\isasymphi}{\isacharparenright}{\kern0pt}\ env{\isacharparenright}{\kern0pt}{\isacharparenright}{\kern0pt}{\isachardoublequoteclose}\ \isanewline
\ \ \ \ \isacommand{using}\isamarkupfalse%
\ separation{\isacharunderscore}{\kern0pt}disj\ \isacommand{by}\isamarkupfalse%
\ simp\isanewline
\ \ \isacommand{then}\isamarkupfalse%
\ \isanewline
\ \ \isacommand{have}\isamarkupfalse%
\ {\isachardoublequoteopen}{\isacharquery}{\kern0pt}D\ {\isasymin}\ M{\isachardoublequoteclose}\ \isanewline
\ \ \ \ \isacommand{using}\isamarkupfalse%
\ separation{\isacharunderscore}{\kern0pt}closed\ P{\isacharunderscore}{\kern0pt}in{\isacharunderscore}{\kern0pt}M\ \isacommand{by}\isamarkupfalse%
\ simp\isanewline
\ \ \isacommand{moreover}\isamarkupfalse%
\isanewline
\ \ \isacommand{have}\isamarkupfalse%
\ {\isachardoublequoteopen}{\isacharquery}{\kern0pt}D\ {\isasymsubseteq}\ P{\isachardoublequoteclose}\ \isacommand{by}\isamarkupfalse%
\ auto\isanewline
\ \ \isacommand{moreover}\isamarkupfalse%
\isanewline
\ \ \isacommand{have}\isamarkupfalse%
\ {\isachardoublequoteopen}dense{\isacharparenleft}{\kern0pt}{\isacharquery}{\kern0pt}D{\isacharparenright}{\kern0pt}{\isachardoublequoteclose}\isanewline
\ \ \isacommand{proof}\isamarkupfalse%
\isanewline
\ \ \ \ \isacommand{fix}\isamarkupfalse%
\ q\isanewline
\ \ \ \ \isacommand{assume}\isamarkupfalse%
\ {\isachardoublequoteopen}q{\isasymin}P{\isachardoublequoteclose}\isanewline
\ \ \ \ \isacommand{show}\isamarkupfalse%
\ {\isachardoublequoteopen}{\isasymexists}d{\isasymin}{\isacharbraceleft}{\kern0pt}p\ {\isasymin}\ P\ {\isachardot}{\kern0pt}\ {\isacharparenleft}{\kern0pt}p\ {\isasymtturnstile}HS\ {\isasymphi}\ env{\isacharparenright}{\kern0pt}\ {\isasymor}\ {\isacharparenleft}{\kern0pt}p\ {\isasymtturnstile}HS\ Neg{\isacharparenleft}{\kern0pt}{\isasymphi}{\isacharparenright}{\kern0pt}\ env{\isacharparenright}{\kern0pt}{\isacharbraceright}{\kern0pt}{\isachardot}{\kern0pt}\ d{\isasympreceq}\ q{\isachardoublequoteclose}\isanewline
\ \ \ \ \isacommand{proof}\isamarkupfalse%
\ {\isacharparenleft}{\kern0pt}cases\ {\isachardoublequoteopen}q\ {\isasymtturnstile}HS\ Neg{\isacharparenleft}{\kern0pt}{\isasymphi}{\isacharparenright}{\kern0pt}\ env{\isachardoublequoteclose}{\isacharparenright}{\kern0pt}\isanewline
\ \ \ \ \ \ \isacommand{case}\isamarkupfalse%
\ True\isanewline
\ \ \ \ \ \ \isacommand{with}\isamarkupfalse%
\ {\isacartoucheopen}q{\isasymin}P{\isacartoucheclose}\isanewline
\ \ \ \ \ \ \isacommand{show}\isamarkupfalse%
\ {\isacharquery}{\kern0pt}thesis\ \isacommand{using}\isamarkupfalse%
\ leq{\isacharunderscore}{\kern0pt}reflI\ \isacommand{by}\isamarkupfalse%
\ blast\isanewline
\ \ \ \ \isacommand{next}\isamarkupfalse%
\isanewline
\ \ \ \ \ \ \isacommand{case}\isamarkupfalse%
\ False\isanewline
\ \ \ \ \ \ \isacommand{with}\isamarkupfalse%
\ {\isacartoucheopen}q{\isasymin}P{\isacartoucheclose}\ \isakeyword{and}\ assms\isanewline
\ \ \ \ \ \ \isacommand{show}\isamarkupfalse%
\ {\isacharquery}{\kern0pt}thesis\ \isacommand{using}\isamarkupfalse%
\ ForcesHS{\isacharunderscore}{\kern0pt}Neg\ envin\ \isacommand{by}\isamarkupfalse%
\ auto\isanewline
\ \ \ \ \isacommand{qed}\isamarkupfalse%
\isanewline
\ \ \isacommand{qed}\isamarkupfalse%
\isanewline
\ \ \isacommand{moreover}\isamarkupfalse%
\isanewline
\ \ \isacommand{note}\isamarkupfalse%
\ {\isacartoucheopen}M{\isacharunderscore}{\kern0pt}generic{\isacharparenleft}{\kern0pt}G{\isacharparenright}{\kern0pt}{\isacartoucheclose}\isanewline
\ \ \isacommand{ultimately}\isamarkupfalse%
\isanewline
\ \ \isacommand{obtain}\isamarkupfalse%
\ p\ \isakeyword{where}\ {\isachardoublequoteopen}p{\isasymin}G{\isachardoublequoteclose}\ {\isachardoublequoteopen}{\isacharparenleft}{\kern0pt}p\ {\isasymtturnstile}HS\ {\isasymphi}\ env{\isacharparenright}{\kern0pt}\ {\isasymor}\ {\isacharparenleft}{\kern0pt}p\ {\isasymtturnstile}HS\ Neg{\isacharparenleft}{\kern0pt}{\isasymphi}{\isacharparenright}{\kern0pt}\ env{\isacharparenright}{\kern0pt}{\isachardoublequoteclose}\isanewline
\ \ \ \ \isacommand{by}\isamarkupfalse%
\ blast\isanewline
\ \ \isacommand{then}\isamarkupfalse%
\isanewline
\ \ \isacommand{consider}\isamarkupfalse%
\ {\isacharparenleft}{\kern0pt}{\isadigit{1}}{\isacharparenright}{\kern0pt}\ {\isachardoublequoteopen}p\ {\isasymtturnstile}HS\ {\isasymphi}\ env{\isachardoublequoteclose}\ {\isacharbar}{\kern0pt}\ {\isacharparenleft}{\kern0pt}{\isadigit{2}}{\isacharparenright}{\kern0pt}\ {\isachardoublequoteopen}p\ {\isasymtturnstile}HS\ Neg{\isacharparenleft}{\kern0pt}{\isasymphi}{\isacharparenright}{\kern0pt}\ env{\isachardoublequoteclose}\ \isacommand{by}\isamarkupfalse%
\ blast\isanewline
\ \ \isacommand{then}\isamarkupfalse%
\isanewline
\ \ \isacommand{show}\isamarkupfalse%
\ {\isachardoublequoteopen}{\isasymexists}p{\isasymin}G{\isachardot}{\kern0pt}\ {\isacharparenleft}{\kern0pt}p\ {\isasymtturnstile}HS\ Neg{\isacharparenleft}{\kern0pt}{\isasymphi}{\isacharparenright}{\kern0pt}\ env{\isacharparenright}{\kern0pt}{\isachardoublequoteclose}\isanewline
\ \ \isacommand{proof}\isamarkupfalse%
\ {\isacharparenleft}{\kern0pt}cases{\isacharparenright}{\kern0pt}\isanewline
\ \ \ \ \isacommand{case}\isamarkupfalse%
\ {\isadigit{1}}\isanewline
\ \ \ \ \isacommand{with}\isamarkupfalse%
\ {\isacartoucheopen}{\isasymnot}\ {\isacharparenleft}{\kern0pt}SymExt{\isacharparenleft}{\kern0pt}G{\isacharparenright}{\kern0pt}{\isacharcomma}{\kern0pt}map{\isacharparenleft}{\kern0pt}val{\isacharparenleft}{\kern0pt}G{\isacharparenright}{\kern0pt}{\isacharcomma}{\kern0pt}env{\isacharparenright}{\kern0pt}\ {\isasymTurnstile}\ {\isasymphi}{\isacharparenright}{\kern0pt}{\isacartoucheclose}\ {\isacartoucheopen}p{\isasymin}G{\isacartoucheclose}\ IH\isanewline
\ \ \ \ \isacommand{show}\isamarkupfalse%
\ {\isacharquery}{\kern0pt}thesis\isanewline
\ \ \ \ \ \ \isacommand{by}\isamarkupfalse%
\ blast\isanewline
\ \ \isacommand{next}\isamarkupfalse%
\isanewline
\ \ \ \ \isacommand{case}\isamarkupfalse%
\ {\isadigit{2}}\isanewline
\ \ \ \ \isacommand{with}\isamarkupfalse%
\ {\isacartoucheopen}p{\isasymin}G{\isacartoucheclose}\ \isanewline
\ \ \ \ \isacommand{show}\isamarkupfalse%
\ {\isacharquery}{\kern0pt}thesis\ \isacommand{by}\isamarkupfalse%
\ blast\isanewline
\ \ \isacommand{qed}\isamarkupfalse%
\isanewline
\isacommand{qed}\isamarkupfalse%
%
\endisatagproof
{\isafoldproof}%
%
\isadelimproof
\ \isanewline
%
\endisadelimproof
\isanewline
\isacommand{lemma}\isamarkupfalse%
\ HS{\isacharunderscore}{\kern0pt}truth{\isacharunderscore}{\kern0pt}lemma{\isacharunderscore}{\kern0pt}And{\isacharcolon}{\kern0pt}\isanewline
\ \ \isakeyword{assumes}\ \isanewline
\ \ \ \ {\isachardoublequoteopen}env{\isasymin}list{\isacharparenleft}{\kern0pt}HS{\isacharparenright}{\kern0pt}{\isachardoublequoteclose}\ {\isachardoublequoteopen}{\isasymphi}{\isasymin}formula{\isachardoublequoteclose}\ {\isachardoublequoteopen}{\isasympsi}{\isasymin}formula{\isachardoublequoteclose}\isanewline
\ \ \ \ {\isachardoublequoteopen}arity{\isacharparenleft}{\kern0pt}{\isasymphi}{\isacharparenright}{\kern0pt}{\isasymle}length{\isacharparenleft}{\kern0pt}env{\isacharparenright}{\kern0pt}{\isachardoublequoteclose}\ {\isachardoublequoteopen}arity{\isacharparenleft}{\kern0pt}{\isasympsi}{\isacharparenright}{\kern0pt}\ {\isasymle}\ length{\isacharparenleft}{\kern0pt}env{\isacharparenright}{\kern0pt}{\isachardoublequoteclose}\ {\isachardoublequoteopen}M{\isacharunderscore}{\kern0pt}generic{\isacharparenleft}{\kern0pt}G{\isacharparenright}{\kern0pt}{\isachardoublequoteclose}\isanewline
\ \ \ \ \isakeyword{and}\isanewline
\ \ \ \ IH{\isacharcolon}{\kern0pt}\ {\isachardoublequoteopen}{\isacharparenleft}{\kern0pt}{\isasymexists}p{\isasymin}G{\isachardot}{\kern0pt}\ p\ {\isasymtturnstile}HS\ {\isasymphi}\ env{\isacharparenright}{\kern0pt}\ \ {\isasymlongleftrightarrow}\ \ \ SymExt{\isacharparenleft}{\kern0pt}G{\isacharparenright}{\kern0pt}{\isacharcomma}{\kern0pt}\ map{\isacharparenleft}{\kern0pt}val{\isacharparenleft}{\kern0pt}G{\isacharparenright}{\kern0pt}{\isacharcomma}{\kern0pt}env{\isacharparenright}{\kern0pt}\ {\isasymTurnstile}\ {\isasymphi}{\isachardoublequoteclose}\isanewline
\ \ \ \ \ \ \ \ {\isachardoublequoteopen}{\isacharparenleft}{\kern0pt}{\isasymexists}p{\isasymin}G{\isachardot}{\kern0pt}\ p\ {\isasymtturnstile}HS\ {\isasympsi}\ env{\isacharparenright}{\kern0pt}\ \ {\isasymlongleftrightarrow}\ \ \ SymExt{\isacharparenleft}{\kern0pt}G{\isacharparenright}{\kern0pt}{\isacharcomma}{\kern0pt}\ map{\isacharparenleft}{\kern0pt}val{\isacharparenleft}{\kern0pt}G{\isacharparenright}{\kern0pt}{\isacharcomma}{\kern0pt}env{\isacharparenright}{\kern0pt}\ {\isasymTurnstile}\ {\isasympsi}{\isachardoublequoteclose}\isanewline
\ \ \isakeyword{shows}\isanewline
\ \ \ \ {\isachardoublequoteopen}{\isacharparenleft}{\kern0pt}{\isasymexists}p{\isasymin}G{\isachardot}{\kern0pt}\ {\isacharparenleft}{\kern0pt}p\ {\isasymtturnstile}HS\ And{\isacharparenleft}{\kern0pt}{\isasymphi}{\isacharcomma}{\kern0pt}{\isasympsi}{\isacharparenright}{\kern0pt}\ env{\isacharparenright}{\kern0pt}{\isacharparenright}{\kern0pt}\ {\isasymlongleftrightarrow}\ SymExt{\isacharparenleft}{\kern0pt}G{\isacharparenright}{\kern0pt}\ {\isacharcomma}{\kern0pt}\ map{\isacharparenleft}{\kern0pt}val{\isacharparenleft}{\kern0pt}G{\isacharparenright}{\kern0pt}{\isacharcomma}{\kern0pt}env{\isacharparenright}{\kern0pt}\ {\isasymTurnstile}\ And{\isacharparenleft}{\kern0pt}{\isasymphi}{\isacharcomma}{\kern0pt}{\isasympsi}{\isacharparenright}{\kern0pt}{\isachardoublequoteclose}\isanewline
%
\isadelimproof
\ \isanewline
%
\endisadelimproof
%
\isatagproof
\isacommand{proof}\isamarkupfalse%
\ {\isacharparenleft}{\kern0pt}intro\ iffI{\isacharcomma}{\kern0pt}\ elim\ bexE{\isacharparenright}{\kern0pt}\isanewline
\ \ \isacommand{have}\isamarkupfalse%
\ envin\ {\isacharcolon}{\kern0pt}\ {\isachardoublequoteopen}env\ {\isasymin}\ list{\isacharparenleft}{\kern0pt}M{\isacharparenright}{\kern0pt}{\isachardoublequoteclose}\ \isanewline
\ \ \ \ \isacommand{apply}\isamarkupfalse%
{\isacharparenleft}{\kern0pt}rule{\isacharunderscore}{\kern0pt}tac\ A{\isacharequal}{\kern0pt}{\isachardoublequoteopen}list{\isacharparenleft}{\kern0pt}HS{\isacharparenright}{\kern0pt}{\isachardoublequoteclose}\ \isakeyword{in}\ subsetD{\isacharcomma}{\kern0pt}\ rule\ list{\isacharunderscore}{\kern0pt}mono{\isacharparenright}{\kern0pt}\isanewline
\ \ \ \ \isacommand{using}\isamarkupfalse%
\ HS{\isacharunderscore}{\kern0pt}iff\ P{\isacharunderscore}{\kern0pt}name{\isacharunderscore}{\kern0pt}in{\isacharunderscore}{\kern0pt}M\ assms\isanewline
\ \ \ \ \isacommand{by}\isamarkupfalse%
\ auto\isanewline
\ \ \isacommand{have}\isamarkupfalse%
\ mapin\ {\isacharcolon}{\kern0pt}\ {\isachardoublequoteopen}map{\isacharparenleft}{\kern0pt}val{\isacharparenleft}{\kern0pt}G{\isacharparenright}{\kern0pt}{\isacharcomma}{\kern0pt}\ env{\isacharparenright}{\kern0pt}\ {\isasymin}\ list{\isacharparenleft}{\kern0pt}SymExt{\isacharparenleft}{\kern0pt}G{\isacharparenright}{\kern0pt}{\isacharparenright}{\kern0pt}{\isachardoublequoteclose}\ \isanewline
\ \ \ \ \isacommand{apply}\isamarkupfalse%
{\isacharparenleft}{\kern0pt}rule\ map{\isacharunderscore}{\kern0pt}type{\isacharparenright}{\kern0pt}\isanewline
\ \ \ \ \isacommand{using}\isamarkupfalse%
\ assms\ SymExt{\isacharunderscore}{\kern0pt}def\isanewline
\ \ \ \ \isacommand{by}\isamarkupfalse%
\ auto\isanewline
\isanewline
\ \ \isacommand{fix}\isamarkupfalse%
\ p\isanewline
\ \ \isacommand{assume}\isamarkupfalse%
\ {\isachardoublequoteopen}p{\isasymin}G{\isachardoublequoteclose}\ {\isachardoublequoteopen}p\ {\isasymtturnstile}HS\ And{\isacharparenleft}{\kern0pt}{\isasymphi}{\isacharcomma}{\kern0pt}{\isasympsi}{\isacharparenright}{\kern0pt}\ env{\isachardoublequoteclose}\isanewline
\ \ \isacommand{with}\isamarkupfalse%
\ assms\isanewline
\ \ \isacommand{show}\isamarkupfalse%
\ {\isachardoublequoteopen}SymExt{\isacharparenleft}{\kern0pt}G{\isacharparenright}{\kern0pt}{\isacharcomma}{\kern0pt}\ map{\isacharparenleft}{\kern0pt}val{\isacharparenleft}{\kern0pt}G{\isacharparenright}{\kern0pt}{\isacharcomma}{\kern0pt}env{\isacharparenright}{\kern0pt}\ {\isasymTurnstile}\ And{\isacharparenleft}{\kern0pt}{\isasymphi}{\isacharcomma}{\kern0pt}{\isasympsi}{\isacharparenright}{\kern0pt}{\isachardoublequoteclose}\ \isanewline
\ \ \ \ \isacommand{using}\isamarkupfalse%
\ ForcesHS{\isacharunderscore}{\kern0pt}And{\isacharbrackleft}{\kern0pt}OF\ M{\isacharunderscore}{\kern0pt}genericD{\isacharcomma}{\kern0pt}\ of\ {\isacharunderscore}{\kern0pt}\ {\isacharunderscore}{\kern0pt}\ {\isacharunderscore}{\kern0pt}\ {\isasymphi}\ {\isasympsi}{\isacharbrackright}{\kern0pt}\ mapin\ envin\ \isacommand{by}\isamarkupfalse%
\ auto\isanewline
\isacommand{next}\isamarkupfalse%
\ \isanewline
\ \ \isacommand{have}\isamarkupfalse%
\ envin\ {\isacharcolon}{\kern0pt}\ {\isachardoublequoteopen}env\ {\isasymin}\ list{\isacharparenleft}{\kern0pt}M{\isacharparenright}{\kern0pt}{\isachardoublequoteclose}\ \isanewline
\ \ \ \ \isacommand{apply}\isamarkupfalse%
{\isacharparenleft}{\kern0pt}rule{\isacharunderscore}{\kern0pt}tac\ A{\isacharequal}{\kern0pt}{\isachardoublequoteopen}list{\isacharparenleft}{\kern0pt}HS{\isacharparenright}{\kern0pt}{\isachardoublequoteclose}\ \isakeyword{in}\ subsetD{\isacharcomma}{\kern0pt}\ rule\ list{\isacharunderscore}{\kern0pt}mono{\isacharparenright}{\kern0pt}\isanewline
\ \ \ \ \isacommand{using}\isamarkupfalse%
\ HS{\isacharunderscore}{\kern0pt}iff\ P{\isacharunderscore}{\kern0pt}name{\isacharunderscore}{\kern0pt}in{\isacharunderscore}{\kern0pt}M\ assms\isanewline
\ \ \ \ \isacommand{by}\isamarkupfalse%
\ auto\isanewline
\ \ \isacommand{have}\isamarkupfalse%
\ mapin\ {\isacharcolon}{\kern0pt}\ {\isachardoublequoteopen}map{\isacharparenleft}{\kern0pt}val{\isacharparenleft}{\kern0pt}G{\isacharparenright}{\kern0pt}{\isacharcomma}{\kern0pt}\ env{\isacharparenright}{\kern0pt}\ {\isasymin}\ list{\isacharparenleft}{\kern0pt}SymExt{\isacharparenleft}{\kern0pt}G{\isacharparenright}{\kern0pt}{\isacharparenright}{\kern0pt}{\isachardoublequoteclose}\ \isanewline
\ \ \ \ \isacommand{apply}\isamarkupfalse%
{\isacharparenleft}{\kern0pt}rule\ map{\isacharunderscore}{\kern0pt}type{\isacharparenright}{\kern0pt}\isanewline
\ \ \ \ \isacommand{using}\isamarkupfalse%
\ assms\ SymExt{\isacharunderscore}{\kern0pt}def\isanewline
\ \ \ \ \isacommand{by}\isamarkupfalse%
\ auto\isanewline
\isanewline
\ \ \isacommand{assume}\isamarkupfalse%
\ {\isachardoublequoteopen}SymExt{\isacharparenleft}{\kern0pt}G{\isacharparenright}{\kern0pt}{\isacharcomma}{\kern0pt}\ map{\isacharparenleft}{\kern0pt}val{\isacharparenleft}{\kern0pt}G{\isacharparenright}{\kern0pt}{\isacharcomma}{\kern0pt}env{\isacharparenright}{\kern0pt}\ {\isasymTurnstile}\ And{\isacharparenleft}{\kern0pt}{\isasymphi}{\isacharcomma}{\kern0pt}{\isasympsi}{\isacharparenright}{\kern0pt}{\isachardoublequoteclose}\isanewline
\ \ \isacommand{moreover}\isamarkupfalse%
\isanewline
\ \ \isacommand{note}\isamarkupfalse%
\ assms\isanewline
\ \ \isacommand{moreover}\isamarkupfalse%
\ \isacommand{from}\isamarkupfalse%
\ calculation\isanewline
\ \ \isacommand{obtain}\isamarkupfalse%
\ q\ r\ \isakeyword{where}\ {\isachardoublequoteopen}q\ {\isasymtturnstile}HS\ {\isasymphi}\ env{\isachardoublequoteclose}\ {\isachardoublequoteopen}r\ {\isasymtturnstile}HS\ {\isasympsi}\ env{\isachardoublequoteclose}\ {\isachardoublequoteopen}q{\isasymin}G{\isachardoublequoteclose}\ {\isachardoublequoteopen}r{\isasymin}G{\isachardoublequoteclose}\isanewline
\ \ \ \ \isacommand{using}\isamarkupfalse%
\ mapin\ envin\ ForcesHS{\isacharunderscore}{\kern0pt}And\ \isacommand{by}\isamarkupfalse%
\ auto\isanewline
\ \ \isacommand{moreover}\isamarkupfalse%
\ \isacommand{from}\isamarkupfalse%
\ calculation\isanewline
\ \ \isacommand{obtain}\isamarkupfalse%
\ p\ \isakeyword{where}\ {\isachardoublequoteopen}p{\isasympreceq}q{\isachardoublequoteclose}\ {\isachardoublequoteopen}p{\isasympreceq}r{\isachardoublequoteclose}\ {\isachardoublequoteopen}p{\isasymin}G{\isachardoublequoteclose}\isanewline
\ \ \ \ \isacommand{by}\isamarkupfalse%
\ blast\isanewline
\ \ \isacommand{moreover}\isamarkupfalse%
\ \isacommand{from}\isamarkupfalse%
\ calculation\isanewline
\ \ \isacommand{have}\isamarkupfalse%
\ {\isachardoublequoteopen}{\isacharparenleft}{\kern0pt}p\ {\isasymtturnstile}HS\ {\isasymphi}\ env{\isacharparenright}{\kern0pt}\ {\isasymand}\ {\isacharparenleft}{\kern0pt}p\ {\isasymtturnstile}HS\ {\isasympsi}\ env{\isacharparenright}{\kern0pt}{\isachardoublequoteclose}\isanewline
\ \ \ \ \isacommand{using}\isamarkupfalse%
\ HS{\isacharunderscore}{\kern0pt}strengthening{\isacharunderscore}{\kern0pt}lemma\ envin\ \isacommand{by}\isamarkupfalse%
\ {\isacharparenleft}{\kern0pt}blast{\isacharparenright}{\kern0pt}\isanewline
\ \ \isacommand{ultimately}\isamarkupfalse%
\isanewline
\ \ \isacommand{show}\isamarkupfalse%
\ {\isachardoublequoteopen}{\isasymexists}p{\isasymin}G{\isachardot}{\kern0pt}\ {\isacharparenleft}{\kern0pt}p\ {\isasymtturnstile}HS\ And{\isacharparenleft}{\kern0pt}{\isasymphi}{\isacharcomma}{\kern0pt}{\isasympsi}{\isacharparenright}{\kern0pt}\ env{\isacharparenright}{\kern0pt}{\isachardoublequoteclose}\isanewline
\ \ \ \ \isacommand{apply}\isamarkupfalse%
{\isacharparenleft}{\kern0pt}rule{\isacharunderscore}{\kern0pt}tac\ x{\isacharequal}{\kern0pt}p\ \isakeyword{in}\ bexI{\isacharparenright}{\kern0pt}\isanewline
\ \ \ \ \ \isacommand{apply}\isamarkupfalse%
{\isacharparenleft}{\kern0pt}rule\ iffD{\isadigit{2}}{\isacharcomma}{\kern0pt}\ rule\ ForcesHS{\isacharunderscore}{\kern0pt}And{\isacharparenright}{\kern0pt}\isanewline
\ \ \ \ \isacommand{using}\isamarkupfalse%
\ assms\ M{\isacharunderscore}{\kern0pt}genericD\ envin\ \isanewline
\ \ \ \ \isacommand{by}\isamarkupfalse%
\ auto\isanewline
\isacommand{qed}\isamarkupfalse%
%
\endisatagproof
{\isafoldproof}%
%
\isadelimproof
\ \isanewline
%
\endisadelimproof
\isanewline
\isacommand{definition}\isamarkupfalse%
\ \isanewline
\ \ ren{\isacharunderscore}{\kern0pt}HS{\isacharunderscore}{\kern0pt}truth{\isacharunderscore}{\kern0pt}lemma\ {\isacharcolon}{\kern0pt}{\isacharcolon}{\kern0pt}\ {\isachardoublequoteopen}i{\isasymRightarrow}i{\isachardoublequoteclose}\ \isakeyword{where}\isanewline
\ \ {\isachardoublequoteopen}ren{\isacharunderscore}{\kern0pt}HS{\isacharunderscore}{\kern0pt}truth{\isacharunderscore}{\kern0pt}lemma{\isacharparenleft}{\kern0pt}{\isasymphi}{\isacharparenright}{\kern0pt}\ {\isasymequiv}\ \isanewline
\ \ \ \ Exists{\isacharparenleft}{\kern0pt}Exists{\isacharparenleft}{\kern0pt}Exists{\isacharparenleft}{\kern0pt}Exists{\isacharparenleft}{\kern0pt}Exists{\isacharparenleft}{\kern0pt}Exists{\isacharparenleft}{\kern0pt}\isanewline
\ \ \ \ And{\isacharparenleft}{\kern0pt}Equal{\isacharparenleft}{\kern0pt}{\isadigit{0}}{\isacharcomma}{\kern0pt}{\isadigit{6}}{\isacharparenright}{\kern0pt}{\isacharcomma}{\kern0pt}And{\isacharparenleft}{\kern0pt}Equal{\isacharparenleft}{\kern0pt}{\isadigit{1}}{\isacharcomma}{\kern0pt}{\isadigit{9}}{\isacharparenright}{\kern0pt}{\isacharcomma}{\kern0pt}And{\isacharparenleft}{\kern0pt}Equal{\isacharparenleft}{\kern0pt}{\isadigit{2}}{\isacharcomma}{\kern0pt}{\isadigit{1}}{\isadigit{0}}{\isacharparenright}{\kern0pt}{\isacharcomma}{\kern0pt}And{\isacharparenleft}{\kern0pt}Equal{\isacharparenleft}{\kern0pt}{\isadigit{3}}{\isacharcomma}{\kern0pt}{\isadigit{1}}{\isadigit{1}}{\isacharparenright}{\kern0pt}{\isacharcomma}{\kern0pt}And{\isacharparenleft}{\kern0pt}Equal{\isacharparenleft}{\kern0pt}{\isadigit{4}}{\isacharcomma}{\kern0pt}{\isadigit{1}}{\isadigit{2}}{\isacharparenright}{\kern0pt}{\isacharcomma}{\kern0pt}And{\isacharparenleft}{\kern0pt}Equal{\isacharparenleft}{\kern0pt}{\isadigit{5}}{\isacharcomma}{\kern0pt}{\isadigit{7}}{\isacharparenright}{\kern0pt}{\isacharcomma}{\kern0pt}\isanewline
\ \ \ \ iterates{\isacharparenleft}{\kern0pt}{\isasymlambda}p{\isachardot}{\kern0pt}\ incr{\isacharunderscore}{\kern0pt}bv{\isacharparenleft}{\kern0pt}p{\isacharparenright}{\kern0pt}{\isacharbackquote}{\kern0pt}{\isadigit{6}}\ {\isacharcomma}{\kern0pt}\ {\isadigit{7}}{\isacharcomma}{\kern0pt}\ {\isasymphi}{\isacharparenright}{\kern0pt}{\isacharparenright}{\kern0pt}{\isacharparenright}{\kern0pt}{\isacharparenright}{\kern0pt}{\isacharparenright}{\kern0pt}{\isacharparenright}{\kern0pt}{\isacharparenright}{\kern0pt}{\isacharparenright}{\kern0pt}{\isacharparenright}{\kern0pt}{\isacharparenright}{\kern0pt}{\isacharparenright}{\kern0pt}{\isacharparenright}{\kern0pt}{\isacharparenright}{\kern0pt}{\isachardoublequoteclose}\isanewline
\isanewline
\isacommand{lemma}\isamarkupfalse%
\ ren{\isacharunderscore}{\kern0pt}HS{\isacharunderscore}{\kern0pt}truth{\isacharunderscore}{\kern0pt}lemma{\isacharunderscore}{\kern0pt}type{\isacharbrackleft}{\kern0pt}TC{\isacharbrackright}{\kern0pt}\ {\isacharcolon}{\kern0pt}\isanewline
\ \ {\isachardoublequoteopen}{\isasymphi}{\isasymin}formula\ {\isasymLongrightarrow}\ ren{\isacharunderscore}{\kern0pt}HS{\isacharunderscore}{\kern0pt}truth{\isacharunderscore}{\kern0pt}lemma{\isacharparenleft}{\kern0pt}{\isasymphi}{\isacharparenright}{\kern0pt}\ {\isasymin}formula{\isachardoublequoteclose}\ \isanewline
%
\isadelimproof
\ \ %
\endisadelimproof
%
\isatagproof
\isacommand{unfolding}\isamarkupfalse%
\ ren{\isacharunderscore}{\kern0pt}HS{\isacharunderscore}{\kern0pt}truth{\isacharunderscore}{\kern0pt}lemma{\isacharunderscore}{\kern0pt}def\isanewline
\ \ \isacommand{by}\isamarkupfalse%
\ simp%
\endisatagproof
{\isafoldproof}%
%
\isadelimproof
\isanewline
%
\endisadelimproof
\isanewline
\isacommand{lemma}\isamarkupfalse%
\ sats{\isacharunderscore}{\kern0pt}ren{\isacharunderscore}{\kern0pt}HS{\isacharunderscore}{\kern0pt}truth{\isacharunderscore}{\kern0pt}lemma{\isacharcolon}{\kern0pt}\isanewline
\ \ {\isachardoublequoteopen}{\isacharbrackleft}{\kern0pt}q{\isacharcomma}{\kern0pt}b{\isacharcomma}{\kern0pt}d{\isacharcomma}{\kern0pt}a{\isadigit{1}}{\isacharcomma}{\kern0pt}a{\isadigit{2}}{\isacharcomma}{\kern0pt}a{\isadigit{3}}{\isacharcomma}{\kern0pt}a{\isadigit{4}}{\isacharbrackright}{\kern0pt}\ {\isacharat}{\kern0pt}\ env\ {\isasymin}\ list{\isacharparenleft}{\kern0pt}M{\isacharparenright}{\kern0pt}\ {\isasymLongrightarrow}\ {\isasymphi}\ {\isasymin}\ formula\ {\isasymLongrightarrow}\isanewline
\ \ \ {\isacharparenleft}{\kern0pt}M{\isacharcomma}{\kern0pt}\ {\isacharbrackleft}{\kern0pt}q{\isacharcomma}{\kern0pt}b{\isacharcomma}{\kern0pt}d{\isacharcomma}{\kern0pt}a{\isadigit{1}}{\isacharcomma}{\kern0pt}a{\isadigit{2}}{\isacharcomma}{\kern0pt}a{\isadigit{3}}{\isacharcomma}{\kern0pt}a{\isadigit{4}}{\isacharbrackright}{\kern0pt}\ {\isacharat}{\kern0pt}\ env\ {\isasymTurnstile}\ ren{\isacharunderscore}{\kern0pt}HS{\isacharunderscore}{\kern0pt}truth{\isacharunderscore}{\kern0pt}lemma{\isacharparenleft}{\kern0pt}{\isasymphi}{\isacharparenright}{\kern0pt}\ {\isacharparenright}{\kern0pt}\ {\isasymlongleftrightarrow}\isanewline
\ \ \ {\isacharparenleft}{\kern0pt}M{\isacharcomma}{\kern0pt}\ {\isacharbrackleft}{\kern0pt}q{\isacharcomma}{\kern0pt}a{\isadigit{1}}{\isacharcomma}{\kern0pt}a{\isadigit{2}}{\isacharcomma}{\kern0pt}a{\isadigit{3}}{\isacharcomma}{\kern0pt}a{\isadigit{4}}{\isacharcomma}{\kern0pt}b{\isacharbrackright}{\kern0pt}\ {\isacharat}{\kern0pt}\ env\ {\isasymTurnstile}\ {\isasymphi}{\isacharparenright}{\kern0pt}{\isachardoublequoteclose}\isanewline
%
\isadelimproof
\ \ %
\endisadelimproof
%
\isatagproof
\isacommand{unfolding}\isamarkupfalse%
\ ren{\isacharunderscore}{\kern0pt}HS{\isacharunderscore}{\kern0pt}truth{\isacharunderscore}{\kern0pt}lemma{\isacharunderscore}{\kern0pt}def\isanewline
\ \ \isacommand{by}\isamarkupfalse%
\ {\isacharparenleft}{\kern0pt}insert\ sats{\isacharunderscore}{\kern0pt}incr{\isacharunderscore}{\kern0pt}bv{\isacharunderscore}{\kern0pt}iff\ {\isacharbrackleft}{\kern0pt}of\ {\isacharunderscore}{\kern0pt}\ {\isacharunderscore}{\kern0pt}\ M\ {\isacharunderscore}{\kern0pt}\ {\isachardoublequoteopen}{\isacharbrackleft}{\kern0pt}q{\isacharcomma}{\kern0pt}a{\isadigit{1}}{\isacharcomma}{\kern0pt}a{\isadigit{2}}{\isacharcomma}{\kern0pt}a{\isadigit{3}}{\isacharcomma}{\kern0pt}a{\isadigit{4}}{\isacharcomma}{\kern0pt}b{\isacharbrackright}{\kern0pt}{\isachardoublequoteclose}{\isacharbrackright}{\kern0pt}{\isacharcomma}{\kern0pt}\ simp{\isacharparenright}{\kern0pt}%
\endisatagproof
{\isafoldproof}%
%
\isadelimproof
\isanewline
%
\endisadelimproof
\isanewline
\isacommand{lemma}\isamarkupfalse%
\ arity{\isacharunderscore}{\kern0pt}incr{\isacharunderscore}{\kern0pt}bv{\isacharunderscore}{\kern0pt}le\ {\isacharcolon}{\kern0pt}\ \isanewline
\ \ \isakeyword{assumes}\ {\isachardoublequoteopen}{\isasymphi}\ {\isasymin}\ formula{\isachardoublequoteclose}\ {\isachardoublequoteopen}m\ {\isasymin}\ nat{\isachardoublequoteclose}\ {\isachardoublequoteopen}n\ {\isasymin}\ nat{\isachardoublequoteclose}\isanewline
\ \ \isakeyword{shows}\ {\isachardoublequoteopen}arity{\isacharparenleft}{\kern0pt}{\isacharparenleft}{\kern0pt}{\isasymlambda}p{\isachardot}{\kern0pt}\ incr{\isacharunderscore}{\kern0pt}bv{\isacharparenleft}{\kern0pt}p{\isacharparenright}{\kern0pt}\ {\isacharbackquote}{\kern0pt}\ n{\isacharparenright}{\kern0pt}{\isacharcircum}{\kern0pt}m\ {\isacharparenleft}{\kern0pt}{\isasymphi}{\isacharparenright}{\kern0pt}\ {\isacharparenright}{\kern0pt}\ {\isasymle}\ m\ {\isacharhash}{\kern0pt}{\isacharplus}{\kern0pt}\ arity{\isacharparenleft}{\kern0pt}{\isasymphi}{\isacharparenright}{\kern0pt}{\isachardoublequoteclose}\ \isanewline
%
\isadelimproof
\ \ %
\endisadelimproof
%
\isatagproof
\isacommand{using}\isamarkupfalse%
\ {\isacartoucheopen}m\ {\isasymin}\ nat{\isacartoucheclose}\isanewline
\isacommand{proof}\isamarkupfalse%
{\isacharparenleft}{\kern0pt}induct{\isacharparenright}{\kern0pt}\isanewline
\ \ \isacommand{case}\isamarkupfalse%
\ {\isadigit{0}}\isanewline
\ \ \isacommand{then}\isamarkupfalse%
\ \isacommand{show}\isamarkupfalse%
\ {\isacharquery}{\kern0pt}case\ \isacommand{using}\isamarkupfalse%
\ assms\ \isacommand{by}\isamarkupfalse%
\ auto\isanewline
\isacommand{next}\isamarkupfalse%
\isanewline
\ \ \isacommand{case}\isamarkupfalse%
\ {\isacharparenleft}{\kern0pt}succ\ x{\isacharparenright}{\kern0pt}\isanewline
\ \ \isacommand{then}\isamarkupfalse%
\ \isacommand{show}\isamarkupfalse%
\ {\isacharquery}{\kern0pt}case\ \isanewline
\ \ \ \ \isacommand{apply}\isamarkupfalse%
\ simp\isanewline
\ \ \ \ \isacommand{apply}\isamarkupfalse%
{\isacharparenleft}{\kern0pt}subst\ arity{\isacharunderscore}{\kern0pt}incr{\isacharunderscore}{\kern0pt}bv{\isacharunderscore}{\kern0pt}lemma{\isacharparenright}{\kern0pt}\isanewline
\ \ \ \ \ \ \isacommand{apply}\isamarkupfalse%
{\isacharparenleft}{\kern0pt}rule\ iterates{\isacharunderscore}{\kern0pt}type{\isacharcomma}{\kern0pt}\ simp{\isacharcomma}{\kern0pt}\ simp\ add{\isacharcolon}{\kern0pt}assms{\isacharparenright}{\kern0pt}\isanewline
\ \ \ \ \ \ \isacommand{apply}\isamarkupfalse%
{\isacharparenleft}{\kern0pt}rule\ function{\isacharunderscore}{\kern0pt}value{\isacharunderscore}{\kern0pt}in{\isacharparenright}{\kern0pt}\isanewline
\ \ \ \ \ \ \ \isacommand{apply}\isamarkupfalse%
{\isacharparenleft}{\kern0pt}rule\ incr{\isacharunderscore}{\kern0pt}bv{\isacharunderscore}{\kern0pt}type{\isacharparenright}{\kern0pt}\isanewline
\ \ \ \ \isacommand{using}\isamarkupfalse%
\ assms\isanewline
\ \ \ \ \ \ \ \isacommand{apply}\isamarkupfalse%
\ auto{\isacharbrackleft}{\kern0pt}{\isadigit{3}}{\isacharbrackright}{\kern0pt}\isanewline
\ \ \ \ \isacommand{apply}\isamarkupfalse%
{\isacharparenleft}{\kern0pt}cases\ {\isachardoublequoteopen}n\ {\isacharless}{\kern0pt}\ arity{\isacharparenleft}{\kern0pt}{\isacharparenleft}{\kern0pt}{\isasymlambda}p{\isachardot}{\kern0pt}\ incr{\isacharunderscore}{\kern0pt}bv{\isacharparenleft}{\kern0pt}p{\isacharparenright}{\kern0pt}\ {\isacharbackquote}{\kern0pt}\ n{\isacharparenright}{\kern0pt}{\isacharcircum}{\kern0pt}x\ {\isacharparenleft}{\kern0pt}{\isasymphi}{\isacharparenright}{\kern0pt}{\isacharparenright}{\kern0pt}{\isachardoublequoteclose}{\isacharparenright}{\kern0pt}\isanewline
\ \ \ \ \ \isacommand{apply}\isamarkupfalse%
\ simp\isanewline
\ \ \ \ \isacommand{apply}\isamarkupfalse%
\ simp\isanewline
\ \ \ \ \isacommand{apply}\isamarkupfalse%
{\isacharparenleft}{\kern0pt}rule{\isacharunderscore}{\kern0pt}tac\ j{\isacharequal}{\kern0pt}{\isachardoublequoteopen}x\ {\isacharhash}{\kern0pt}{\isacharplus}{\kern0pt}\ arity{\isacharparenleft}{\kern0pt}{\isasymphi}{\isacharparenright}{\kern0pt}{\isachardoublequoteclose}\ \isakeyword{in}\ le{\isacharunderscore}{\kern0pt}trans{\isacharcomma}{\kern0pt}\ simp{\isacharcomma}{\kern0pt}\ simp{\isacharparenright}{\kern0pt}\isanewline
\ \ \ \ \isacommand{done}\isamarkupfalse%
\isanewline
\isacommand{qed}\isamarkupfalse%
%
\endisatagproof
{\isafoldproof}%
%
\isadelimproof
\isanewline
%
\endisadelimproof
\isanewline
\isacommand{lemma}\isamarkupfalse%
\ arity{\isacharunderscore}{\kern0pt}ren{\isacharunderscore}{\kern0pt}HS{\isacharunderscore}{\kern0pt}truth{\isacharunderscore}{\kern0pt}lemma\ {\isacharcolon}{\kern0pt}\ \isanewline
\ \ \isakeyword{assumes}\ {\isachardoublequoteopen}{\isasymphi}{\isasymin}formula{\isachardoublequoteclose}\isanewline
\ \ \isakeyword{shows}\ {\isachardoublequoteopen}arity{\isacharparenleft}{\kern0pt}ren{\isacharunderscore}{\kern0pt}HS{\isacharunderscore}{\kern0pt}truth{\isacharunderscore}{\kern0pt}lemma{\isacharparenleft}{\kern0pt}{\isasymphi}{\isacharparenright}{\kern0pt}{\isacharparenright}{\kern0pt}\ {\isasymle}\ {\isadigit{7}}\ {\isasymunion}\ succ{\isacharparenleft}{\kern0pt}arity{\isacharparenleft}{\kern0pt}{\isasymphi}{\isacharparenright}{\kern0pt}{\isacharparenright}{\kern0pt}{\isachardoublequoteclose}\isanewline
%
\isadelimproof
\isanewline
\ \ %
\endisadelimproof
%
\isatagproof
\isacommand{unfolding}\isamarkupfalse%
\ ren{\isacharunderscore}{\kern0pt}HS{\isacharunderscore}{\kern0pt}truth{\isacharunderscore}{\kern0pt}lemma{\isacharunderscore}{\kern0pt}def\ \isanewline
\ \ \isacommand{apply}\isamarkupfalse%
{\isacharparenleft}{\kern0pt}subgoal{\isacharunderscore}{\kern0pt}tac\ {\isachardoublequoteopen}{\isacharparenleft}{\kern0pt}{\isasymlambda}p{\isachardot}{\kern0pt}\ incr{\isacharunderscore}{\kern0pt}bv{\isacharparenleft}{\kern0pt}p{\isacharparenright}{\kern0pt}\ {\isacharbackquote}{\kern0pt}\ {\isadigit{6}}{\isacharparenright}{\kern0pt}{\isacharcircum}{\kern0pt}{\isadigit{7}}\ {\isacharparenleft}{\kern0pt}{\isasymphi}{\isacharparenright}{\kern0pt}\ {\isasymin}\ formula{\isachardoublequoteclose}{\isacharparenright}{\kern0pt}\isanewline
\ \ \isacommand{using}\isamarkupfalse%
\ assms\ \isanewline
\ \ \ \isacommand{apply}\isamarkupfalse%
\ simp\isanewline
\ \ \ \isacommand{apply}\isamarkupfalse%
{\isacharparenleft}{\kern0pt}rule\ pred{\isacharunderscore}{\kern0pt}le{\isacharcomma}{\kern0pt}\ simp{\isacharcomma}{\kern0pt}\ simp{\isacharparenright}{\kern0pt}{\isacharplus}{\kern0pt}\isanewline
\ \ \ \isacommand{apply}\isamarkupfalse%
{\isacharparenleft}{\kern0pt}subst\ succ{\isacharunderscore}{\kern0pt}Un{\isacharunderscore}{\kern0pt}distrib{\isacharcomma}{\kern0pt}\ simp{\isacharcomma}{\kern0pt}\ simp{\isacharparenright}{\kern0pt}{\isacharplus}{\kern0pt}\isanewline
\ \ \ \isacommand{apply}\isamarkupfalse%
{\isacharparenleft}{\kern0pt}simp\ add{\isacharcolon}{\kern0pt}nat{\isacharunderscore}{\kern0pt}union{\isacharunderscore}{\kern0pt}abs{\isadigit{1}}{\isacharparenright}{\kern0pt}\isanewline
\ \ \ \isacommand{apply}\isamarkupfalse%
{\isacharparenleft}{\kern0pt}rule\ Un{\isacharunderscore}{\kern0pt}least{\isacharunderscore}{\kern0pt}lt{\isacharcomma}{\kern0pt}\ rule\ ltI{\isacharcomma}{\kern0pt}\ simp{\isacharcomma}{\kern0pt}\ rule\ disjI{\isadigit{1}}{\isacharcomma}{\kern0pt}\ rule\ ltD{\isacharcomma}{\kern0pt}\ simp{\isacharcomma}{\kern0pt}\ simp{\isacharparenright}{\kern0pt}{\isacharplus}{\kern0pt}\isanewline
\ \ \ \isacommand{apply}\isamarkupfalse%
{\isacharparenleft}{\kern0pt}rule\ Un{\isacharunderscore}{\kern0pt}least{\isacharunderscore}{\kern0pt}lt{\isacharcomma}{\kern0pt}\ rule\ ltI{\isacharcomma}{\kern0pt}\ simp{\isacharcomma}{\kern0pt}\ simp{\isacharparenright}{\kern0pt}\isanewline
\ \ \ \isacommand{apply}\isamarkupfalse%
{\isacharparenleft}{\kern0pt}rule\ Un{\isacharunderscore}{\kern0pt}least{\isacharunderscore}{\kern0pt}lt{\isacharcomma}{\kern0pt}\ rule\ ltI{\isacharcomma}{\kern0pt}\ simp{\isacharcomma}{\kern0pt}\ rule\ disjI{\isadigit{1}}{\isacharcomma}{\kern0pt}\ rule\ ltD{\isacharcomma}{\kern0pt}\ simp{\isacharcomma}{\kern0pt}\ simp{\isacharparenright}{\kern0pt}\isanewline
\ \ \ \isacommand{apply}\isamarkupfalse%
{\isacharparenleft}{\kern0pt}rule{\isacharunderscore}{\kern0pt}tac\ b{\isacharequal}{\kern0pt}{\isachardoublequoteopen}arity{\isacharparenleft}{\kern0pt}{\isacharparenleft}{\kern0pt}{\isasymlambda}p{\isachardot}{\kern0pt}\ incr{\isacharunderscore}{\kern0pt}bv{\isacharparenleft}{\kern0pt}p{\isacharparenright}{\kern0pt}\ {\isacharbackquote}{\kern0pt}\ {\isadigit{6}}{\isacharparenright}{\kern0pt}{\isacharcircum}{\kern0pt}{\isadigit{7}}\ {\isacharparenleft}{\kern0pt}{\isasymphi}{\isacharparenright}{\kern0pt}{\isacharparenright}{\kern0pt}{\isachardoublequoteclose}\ \isakeyword{in}\ le{\isacharunderscore}{\kern0pt}lt{\isacharunderscore}{\kern0pt}lt{\isacharcomma}{\kern0pt}\ force{\isacharparenright}{\kern0pt}\isanewline
\ \ \ \isacommand{apply}\isamarkupfalse%
{\isacharparenleft}{\kern0pt}rule\ le{\isacharunderscore}{\kern0pt}lt{\isacharunderscore}{\kern0pt}lt{\isacharcomma}{\kern0pt}\ rule\ arity{\isacharunderscore}{\kern0pt}incr{\isacharunderscore}{\kern0pt}bv{\isacharunderscore}{\kern0pt}le{\isacharparenright}{\kern0pt}\isanewline
\ \ \ \ \ \ \isacommand{apply}\isamarkupfalse%
\ auto{\isacharbrackleft}{\kern0pt}{\isadigit{3}}{\isacharbrackright}{\kern0pt}\isanewline
\ \ \ \isacommand{apply}\isamarkupfalse%
\ simp\isanewline
\ \ \ \isacommand{apply}\isamarkupfalse%
{\isacharparenleft}{\kern0pt}rule\ ltI{\isacharcomma}{\kern0pt}\ simp{\isacharcomma}{\kern0pt}\ simp{\isacharparenright}{\kern0pt}\isanewline
\ \ \isacommand{apply}\isamarkupfalse%
{\isacharparenleft}{\kern0pt}rule\ iterates{\isacharunderscore}{\kern0pt}type{\isacharcomma}{\kern0pt}\ simp{\isacharcomma}{\kern0pt}\ simp\ add{\isacharcolon}{\kern0pt}assms{\isacharparenright}{\kern0pt}\isanewline
\ \ \isacommand{apply}\isamarkupfalse%
{\isacharparenleft}{\kern0pt}rule\ function{\isacharunderscore}{\kern0pt}value{\isacharunderscore}{\kern0pt}in{\isacharparenright}{\kern0pt}\isanewline
\ \ \ \isacommand{apply}\isamarkupfalse%
{\isacharparenleft}{\kern0pt}rule\ incr{\isacharunderscore}{\kern0pt}bv{\isacharunderscore}{\kern0pt}type{\isacharparenright}{\kern0pt}\isanewline
\ \ \isacommand{by}\isamarkupfalse%
\ auto%
\endisatagproof
{\isafoldproof}%
%
\isadelimproof
\isanewline
%
\endisadelimproof
\isanewline
\isacommand{lemma}\isamarkupfalse%
\ HS{\isacharunderscore}{\kern0pt}truth{\isacharunderscore}{\kern0pt}lemma{\isacharprime}{\kern0pt}\ {\isacharcolon}{\kern0pt}\isanewline
\ \ \isakeyword{assumes}\isanewline
\ \ \ \ {\isachardoublequoteopen}{\isasymphi}{\isasymin}formula{\isachardoublequoteclose}\ {\isachardoublequoteopen}env{\isasymin}list{\isacharparenleft}{\kern0pt}M{\isacharparenright}{\kern0pt}{\isachardoublequoteclose}\ {\isachardoublequoteopen}arity{\isacharparenleft}{\kern0pt}{\isasymphi}{\isacharparenright}{\kern0pt}\ {\isasymle}\ succ{\isacharparenleft}{\kern0pt}length{\isacharparenleft}{\kern0pt}env{\isacharparenright}{\kern0pt}{\isacharparenright}{\kern0pt}{\isachardoublequoteclose}\ \isanewline
\ \ \isakeyword{shows}\isanewline
\ \ \ \ {\isachardoublequoteopen}separation{\isacharparenleft}{\kern0pt}{\isacharhash}{\kern0pt}{\isacharhash}{\kern0pt}M{\isacharcomma}{\kern0pt}{\isasymlambda}d{\isachardot}{\kern0pt}\ {\isasymexists}b{\isasymin}HS{\isachardot}{\kern0pt}\ {\isasymforall}q{\isasymin}P{\isachardot}{\kern0pt}\ q{\isasympreceq}d\ {\isasymlongrightarrow}\ {\isasymnot}{\isacharparenleft}{\kern0pt}q\ {\isasymtturnstile}HS\ {\isasymphi}\ {\isacharparenleft}{\kern0pt}{\isacharbrackleft}{\kern0pt}b{\isacharbrackright}{\kern0pt}{\isacharat}{\kern0pt}env{\isacharparenright}{\kern0pt}{\isacharparenright}{\kern0pt}{\isacharparenright}{\kern0pt}{\isachardoublequoteclose}\isanewline
%
\isadelimproof
%
\endisadelimproof
%
\isatagproof
\isacommand{proof}\isamarkupfalse%
\ {\isacharminus}{\kern0pt}\isanewline
\isanewline
\ \ \isacommand{have}\isamarkupfalse%
\ iff{\isacharunderscore}{\kern0pt}lemma{\isacharcolon}{\kern0pt}\ {\isachardoublequoteopen}{\isasymAnd}P\ Q\ R{\isachardot}{\kern0pt}\ {\isacharparenleft}{\kern0pt}{\isasymAnd}x{\isachardot}{\kern0pt}\ x\ {\isasymin}\ M\ {\isasymLongrightarrow}\ x\ {\isasymin}\ HS\ {\isasymlongleftrightarrow}\ P{\isacharparenleft}{\kern0pt}x{\isacharparenright}{\kern0pt}{\isacharparenright}{\kern0pt}\ {\isasymLongrightarrow}\ {\isacharparenleft}{\kern0pt}{\isasymAnd}x{\isachardot}{\kern0pt}\ x\ {\isasymin}\ HS\ {\isasymLongrightarrow}\ Q{\isacharparenleft}{\kern0pt}x{\isacharparenright}{\kern0pt}\ {\isasymlongleftrightarrow}\ R{\isacharparenleft}{\kern0pt}x{\isacharparenright}{\kern0pt}{\isacharparenright}{\kern0pt}\ {\isasymLongrightarrow}\ {\isacharparenleft}{\kern0pt}{\isasymexists}x\ {\isasymin}\ M{\isachardot}{\kern0pt}\ P{\isacharparenleft}{\kern0pt}x{\isacharparenright}{\kern0pt}\ {\isasymand}\ Q{\isacharparenleft}{\kern0pt}x{\isacharparenright}{\kern0pt}{\isacharparenright}{\kern0pt}\ {\isasymlongleftrightarrow}\ {\isacharparenleft}{\kern0pt}{\isasymexists}x\ {\isasymin}\ HS{\isachardot}{\kern0pt}\ R{\isacharparenleft}{\kern0pt}x{\isacharparenright}{\kern0pt}{\isacharparenright}{\kern0pt}{\isachardoublequoteclose}\isanewline
\ \ \ \ \isacommand{apply}\isamarkupfalse%
{\isacharparenleft}{\kern0pt}rule\ iffI{\isacharparenright}{\kern0pt}\isanewline
\ \ \ \ \ \isacommand{apply}\isamarkupfalse%
\ clarify\ \isanewline
\ \ \ \ \ \isacommand{apply}\isamarkupfalse%
{\isacharparenleft}{\kern0pt}rename{\isacharunderscore}{\kern0pt}tac\ x{\isacharcomma}{\kern0pt}\ rule{\isacharunderscore}{\kern0pt}tac\ x{\isacharequal}{\kern0pt}x\ \isakeyword{in}\ bexI{\isacharparenright}{\kern0pt}\isanewline
\ \ \ \ \ \ \isacommand{apply}\isamarkupfalse%
{\isacharparenleft}{\kern0pt}rename{\isacharunderscore}{\kern0pt}tac\ x{\isacharcomma}{\kern0pt}\ subgoal{\isacharunderscore}{\kern0pt}tac\ {\isachardoublequoteopen}x\ {\isasymin}\ HS{\isachardoublequoteclose}{\isacharcomma}{\kern0pt}\ force{\isacharcomma}{\kern0pt}\ force{\isacharcomma}{\kern0pt}\ force{\isacharparenright}{\kern0pt}\isanewline
\ \ \ \ \isacommand{apply}\isamarkupfalse%
\ clarify\ \isanewline
\ \ \ \ \isacommand{apply}\isamarkupfalse%
{\isacharparenleft}{\kern0pt}rename{\isacharunderscore}{\kern0pt}tac\ x{\isacharcomma}{\kern0pt}\ rule{\isacharunderscore}{\kern0pt}tac\ x{\isacharequal}{\kern0pt}x\ \isakeyword{in}\ bexI{\isacharparenright}{\kern0pt}\isanewline
\ \ \ \ \isacommand{using}\isamarkupfalse%
\ HS{\isacharunderscore}{\kern0pt}iff\ P{\isacharunderscore}{\kern0pt}name{\isacharunderscore}{\kern0pt}in{\isacharunderscore}{\kern0pt}M\isanewline
\ \ \ \ \ \isacommand{apply}\isamarkupfalse%
\ force\ \isanewline
\ \ \ \ \isacommand{using}\isamarkupfalse%
\ HS{\isacharunderscore}{\kern0pt}iff\ P{\isacharunderscore}{\kern0pt}name{\isacharunderscore}{\kern0pt}in{\isacharunderscore}{\kern0pt}M\isanewline
\ \ \ \ \ \isacommand{apply}\isamarkupfalse%
\ force\ \isanewline
\ \ \ \ \isacommand{done}\isamarkupfalse%
\isanewline
\isanewline
\ \ \isacommand{let}\isamarkupfalse%
\ {\isacharquery}{\kern0pt}rel{\isacharunderscore}{\kern0pt}pred{\isacharequal}{\kern0pt}{\isachardoublequoteopen}{\isasymlambda}M\ x\ a{\isadigit{1}}\ a{\isadigit{2}}\ a{\isadigit{3}}\ a{\isadigit{4}}{\isachardot}{\kern0pt}\ {\isasymexists}b{\isasymin}HS{\isachardot}{\kern0pt}\ {\isasymforall}q{\isasymin}M{\isachardot}{\kern0pt}\ q{\isasymin}a{\isadigit{1}}\ {\isasymand}\ is{\isacharunderscore}{\kern0pt}leq{\isacharparenleft}{\kern0pt}{\isacharhash}{\kern0pt}{\isacharhash}{\kern0pt}M{\isacharcomma}{\kern0pt}a{\isadigit{2}}{\isacharcomma}{\kern0pt}q{\isacharcomma}{\kern0pt}x{\isacharparenright}{\kern0pt}\ {\isasymlongrightarrow}\ \isanewline
\ \ \ \ \ \ \ \ \ \ \ \ \ \ \ \ \ \ {\isasymnot}{\isacharparenleft}{\kern0pt}M{\isacharcomma}{\kern0pt}\ {\isacharbrackleft}{\kern0pt}q{\isacharcomma}{\kern0pt}a{\isadigit{1}}{\isacharcomma}{\kern0pt}a{\isadigit{2}}{\isacharcomma}{\kern0pt}a{\isadigit{3}}{\isacharcomma}{\kern0pt}a{\isadigit{4}}{\isacharcomma}{\kern0pt}b{\isacharbrackright}{\kern0pt}\ {\isacharat}{\kern0pt}\ env\ {\isasymTurnstile}\ forcesHS{\isacharparenleft}{\kern0pt}{\isasymphi}{\isacharparenright}{\kern0pt}{\isacharparenright}{\kern0pt}{\isachardoublequoteclose}\ \isanewline
\ \ \isacommand{let}\isamarkupfalse%
\ {\isacharquery}{\kern0pt}{\isasympsi}{\isacharequal}{\kern0pt}{\isachardoublequoteopen}Exists{\isacharparenleft}{\kern0pt}And{\isacharparenleft}{\kern0pt}\isanewline
\ \ \ \ \ \ \ \ \ \ \ \ is{\isacharunderscore}{\kern0pt}HS{\isacharunderscore}{\kern0pt}fm{\isacharparenleft}{\kern0pt}{\isadigit{5}}{\isacharcomma}{\kern0pt}\ {\isadigit{0}}{\isacharparenright}{\kern0pt}{\isacharcomma}{\kern0pt}\ \isanewline
\ \ \ \ \ \ \ \ \ \ \ \ Forall{\isacharparenleft}{\kern0pt}Implies{\isacharparenleft}{\kern0pt}And{\isacharparenleft}{\kern0pt}Member{\isacharparenleft}{\kern0pt}{\isadigit{0}}{\isacharcomma}{\kern0pt}{\isadigit{3}}{\isacharparenright}{\kern0pt}{\isacharcomma}{\kern0pt}leq{\isacharunderscore}{\kern0pt}fm{\isacharparenleft}{\kern0pt}{\isadigit{4}}{\isacharcomma}{\kern0pt}{\isadigit{0}}{\isacharcomma}{\kern0pt}{\isadigit{2}}{\isacharparenright}{\kern0pt}{\isacharparenright}{\kern0pt}{\isacharcomma}{\kern0pt}\isanewline
\ \ \ \ \ \ \ \ \ \ \ \ Neg{\isacharparenleft}{\kern0pt}ren{\isacharunderscore}{\kern0pt}HS{\isacharunderscore}{\kern0pt}truth{\isacharunderscore}{\kern0pt}lemma{\isacharparenleft}{\kern0pt}forcesHS{\isacharparenleft}{\kern0pt}{\isasymphi}{\isacharparenright}{\kern0pt}{\isacharparenright}{\kern0pt}{\isacharparenright}{\kern0pt}{\isacharparenright}{\kern0pt}{\isacharparenright}{\kern0pt}{\isacharparenright}{\kern0pt}{\isacharparenright}{\kern0pt}{\isachardoublequoteclose}\isanewline
\ \ \isacommand{have}\isamarkupfalse%
\ {\isachardoublequoteopen}q{\isasymin}M{\isachardoublequoteclose}\ \isakeyword{if}\ {\isachardoublequoteopen}q{\isasymin}P{\isachardoublequoteclose}\ \isakeyword{for}\ q\ \isacommand{using}\isamarkupfalse%
\ that\ transitivity{\isacharbrackleft}{\kern0pt}OF\ {\isacharunderscore}{\kern0pt}\ P{\isacharunderscore}{\kern0pt}in{\isacharunderscore}{\kern0pt}M{\isacharbrackright}{\kern0pt}\ \isacommand{by}\isamarkupfalse%
\ simp\isanewline
\ \ \isacommand{then}\isamarkupfalse%
\isanewline
\ \ \isacommand{have}\isamarkupfalse%
\ {\isadigit{1}}{\isacharcolon}{\kern0pt}{\isachardoublequoteopen}{\isasymforall}q{\isasymin}M{\isachardot}{\kern0pt}\ q{\isasymin}P\ {\isasymand}\ R{\isacharparenleft}{\kern0pt}q{\isacharparenright}{\kern0pt}\ {\isasymlongrightarrow}\ Q{\isacharparenleft}{\kern0pt}q{\isacharparenright}{\kern0pt}\ {\isasymLongrightarrow}\ {\isacharparenleft}{\kern0pt}{\isasymforall}q{\isasymin}P{\isachardot}{\kern0pt}\ R{\isacharparenleft}{\kern0pt}q{\isacharparenright}{\kern0pt}\ {\isasymlongrightarrow}\ Q{\isacharparenleft}{\kern0pt}q{\isacharparenright}{\kern0pt}{\isacharparenright}{\kern0pt}{\isachardoublequoteclose}\ \isakeyword{for}\ R\ Q\ \isanewline
\ \ \ \ \isacommand{by}\isamarkupfalse%
\ auto\isanewline
\ \ \isacommand{then}\isamarkupfalse%
\isanewline
\ \ \isacommand{have}\isamarkupfalse%
\ {\isachardoublequoteopen}{\isasymlbrakk}b\ {\isasymin}\ HS{\isacharsemicolon}{\kern0pt}\ {\isasymforall}q{\isasymin}M{\isachardot}{\kern0pt}\ q\ {\isasymin}\ P\ {\isasymand}\ q\ {\isasympreceq}\ d\ {\isasymlongrightarrow}\ {\isasymnot}{\isacharparenleft}{\kern0pt}q\ {\isasymtturnstile}HS\ {\isasymphi}\ {\isacharparenleft}{\kern0pt}{\isacharbrackleft}{\kern0pt}b{\isacharbrackright}{\kern0pt}{\isacharat}{\kern0pt}env{\isacharparenright}{\kern0pt}{\isacharparenright}{\kern0pt}{\isasymrbrakk}\ {\isasymLongrightarrow}\isanewline
\ \ \ \ \ \ \ \ \ {\isasymexists}c{\isasymin}HS{\isachardot}{\kern0pt}\ {\isasymforall}q{\isasymin}P{\isachardot}{\kern0pt}\ q\ {\isasympreceq}\ d\ {\isasymlongrightarrow}\ {\isasymnot}{\isacharparenleft}{\kern0pt}q\ {\isasymtturnstile}HS\ {\isasymphi}\ {\isacharparenleft}{\kern0pt}{\isacharbrackleft}{\kern0pt}c{\isacharbrackright}{\kern0pt}{\isacharat}{\kern0pt}env{\isacharparenright}{\kern0pt}{\isacharparenright}{\kern0pt}{\isachardoublequoteclose}\ \isakeyword{for}\ b\ d\isanewline
\ \ \ \ \isacommand{by}\isamarkupfalse%
\ {\isacharparenleft}{\kern0pt}rule\ bexI{\isacharcomma}{\kern0pt}simp{\isacharunderscore}{\kern0pt}all{\isacharparenright}{\kern0pt}\isanewline
\ \ \isacommand{then}\isamarkupfalse%
\isanewline
\ \ \isacommand{have}\isamarkupfalse%
\ {\isachardoublequoteopen}{\isacharquery}{\kern0pt}rel{\isacharunderscore}{\kern0pt}pred{\isacharparenleft}{\kern0pt}M{\isacharcomma}{\kern0pt}d{\isacharcomma}{\kern0pt}P{\isacharcomma}{\kern0pt}leq{\isacharcomma}{\kern0pt}one{\isacharcomma}{\kern0pt}{\isasymlangle}{\isasymF}{\isacharcomma}{\kern0pt}\ {\isasymG}{\isacharcomma}{\kern0pt}\ P{\isacharcomma}{\kern0pt}\ P{\isacharunderscore}{\kern0pt}auto{\isasymrangle}{\isacharparenright}{\kern0pt}\ {\isasymlongleftrightarrow}\ {\isacharparenleft}{\kern0pt}{\isasymexists}b{\isasymin}HS{\isachardot}{\kern0pt}\ {\isasymforall}q{\isasymin}P{\isachardot}{\kern0pt}\ q{\isasympreceq}d\ {\isasymlongrightarrow}\ {\isasymnot}{\isacharparenleft}{\kern0pt}q\ {\isasymtturnstile}HS\ {\isasymphi}\ {\isacharparenleft}{\kern0pt}{\isacharbrackleft}{\kern0pt}b{\isacharbrackright}{\kern0pt}{\isacharat}{\kern0pt}env{\isacharparenright}{\kern0pt}{\isacharparenright}{\kern0pt}{\isacharparenright}{\kern0pt}{\isachardoublequoteclose}\ \isakeyword{if}\ {\isachardoublequoteopen}d{\isasymin}M{\isachardoublequoteclose}\ \isakeyword{for}\ d\isanewline
\ \ \ \ \isacommand{using}\isamarkupfalse%
\ that\ leq{\isacharunderscore}{\kern0pt}abs\ leq{\isacharunderscore}{\kern0pt}in{\isacharunderscore}{\kern0pt}M\ P{\isacharunderscore}{\kern0pt}in{\isacharunderscore}{\kern0pt}M\ one{\isacharunderscore}{\kern0pt}in{\isacharunderscore}{\kern0pt}M\ assms\ {\isasymF}{\isacharunderscore}{\kern0pt}in{\isacharunderscore}{\kern0pt}M\ {\isasymG}{\isacharunderscore}{\kern0pt}in{\isacharunderscore}{\kern0pt}M\ P{\isacharunderscore}{\kern0pt}auto{\isacharunderscore}{\kern0pt}in{\isacharunderscore}{\kern0pt}M\ pair{\isacharunderscore}{\kern0pt}in{\isacharunderscore}{\kern0pt}M{\isacharunderscore}{\kern0pt}iff\ \isanewline
\ \ \ \ \isacommand{by}\isamarkupfalse%
\ auto\isanewline
\ \ \isacommand{moreover}\isamarkupfalse%
\isanewline
\ \ \isacommand{have}\isamarkupfalse%
\ {\isachardoublequoteopen}{\isacharquery}{\kern0pt}{\isasympsi}{\isasymin}formula{\isachardoublequoteclose}\ \isanewline
\ \ \ \ \isacommand{apply}\isamarkupfalse%
{\isacharparenleft}{\kern0pt}subgoal{\isacharunderscore}{\kern0pt}tac\ {\isachardoublequoteopen}is{\isacharunderscore}{\kern0pt}HS{\isacharunderscore}{\kern0pt}fm{\isacharparenleft}{\kern0pt}{\isadigit{5}}{\isacharcomma}{\kern0pt}\ {\isadigit{0}}{\isacharparenright}{\kern0pt}\ {\isasymin}\ formula\ {\isasymand}\ ren{\isacharunderscore}{\kern0pt}HS{\isacharunderscore}{\kern0pt}truth{\isacharunderscore}{\kern0pt}lemma{\isacharparenleft}{\kern0pt}forcesHS{\isacharparenleft}{\kern0pt}{\isasymphi}{\isacharparenright}{\kern0pt}{\isacharparenright}{\kern0pt}\ {\isasymin}\ formula{\isachardoublequoteclose}{\isacharcomma}{\kern0pt}\ force{\isacharparenright}{\kern0pt}\isanewline
\ \ \ \ \isacommand{unfolding}\isamarkupfalse%
\ ren{\isacharunderscore}{\kern0pt}HS{\isacharunderscore}{\kern0pt}truth{\isacharunderscore}{\kern0pt}lemma{\isacharunderscore}{\kern0pt}def\ \isanewline
\ \ \ \ \isacommand{apply}\isamarkupfalse%
{\isacharparenleft}{\kern0pt}rule\ conjI{\isacharcomma}{\kern0pt}\ rule\ is{\isacharunderscore}{\kern0pt}HS{\isacharunderscore}{\kern0pt}fm{\isacharunderscore}{\kern0pt}type{\isacharcomma}{\kern0pt}\ simp{\isacharcomma}{\kern0pt}\ simp{\isacharparenright}{\kern0pt}\isanewline
\ \ \ \ \isacommand{apply}\isamarkupfalse%
{\isacharparenleft}{\kern0pt}rule\ Exists{\isacharunderscore}{\kern0pt}type{\isacharparenright}{\kern0pt}{\isacharplus}{\kern0pt}\isanewline
\ \ \ \ \isacommand{apply}\isamarkupfalse%
{\isacharparenleft}{\kern0pt}rule\ And{\isacharunderscore}{\kern0pt}type{\isacharcomma}{\kern0pt}\ simp{\isacharparenright}{\kern0pt}{\isacharplus}{\kern0pt}\isanewline
\ \ \ \ \isacommand{apply}\isamarkupfalse%
{\isacharparenleft}{\kern0pt}rule\ iterates{\isacharunderscore}{\kern0pt}type{\isacharcomma}{\kern0pt}\ simp{\isacharcomma}{\kern0pt}\ rule\ forcesHS{\isacharunderscore}{\kern0pt}type{\isacharcomma}{\kern0pt}\ simp\ add{\isacharcolon}{\kern0pt}assms{\isacharparenright}{\kern0pt}\isanewline
\ \ \ \ \isacommand{apply}\isamarkupfalse%
{\isacharparenleft}{\kern0pt}rule\ function{\isacharunderscore}{\kern0pt}value{\isacharunderscore}{\kern0pt}in{\isacharparenright}{\kern0pt}\isanewline
\ \ \ \ \ \isacommand{apply}\isamarkupfalse%
{\isacharparenleft}{\kern0pt}rule\ incr{\isacharunderscore}{\kern0pt}bv{\isacharunderscore}{\kern0pt}type{\isacharparenright}{\kern0pt}\isanewline
\ \ \ \ \isacommand{by}\isamarkupfalse%
\ auto\isanewline
\ \ \isacommand{moreover}\isamarkupfalse%
\isanewline
\ \ \isacommand{have}\isamarkupfalse%
\ H{\isacharcolon}{\kern0pt}\ {\isachardoublequoteopen}{\isacharparenleft}{\kern0pt}M{\isacharcomma}{\kern0pt}\ {\isacharbrackleft}{\kern0pt}d{\isacharcomma}{\kern0pt}P{\isacharcomma}{\kern0pt}leq{\isacharcomma}{\kern0pt}one{\isacharcomma}{\kern0pt}{\isasymlangle}{\isasymF}{\isacharcomma}{\kern0pt}\ {\isasymG}{\isacharcomma}{\kern0pt}\ P{\isacharcomma}{\kern0pt}\ P{\isacharunderscore}{\kern0pt}auto{\isasymrangle}{\isacharbrackright}{\kern0pt}{\isacharat}{\kern0pt}env\ {\isasymTurnstile}\ {\isacharquery}{\kern0pt}{\isasympsi}{\isacharparenright}{\kern0pt}\ {\isasymlongleftrightarrow}\ {\isacharquery}{\kern0pt}rel{\isacharunderscore}{\kern0pt}pred{\isacharparenleft}{\kern0pt}M{\isacharcomma}{\kern0pt}d{\isacharcomma}{\kern0pt}P{\isacharcomma}{\kern0pt}leq{\isacharcomma}{\kern0pt}one{\isacharcomma}{\kern0pt}{\isasymlangle}{\isasymF}{\isacharcomma}{\kern0pt}\ {\isasymG}{\isacharcomma}{\kern0pt}\ P{\isacharcomma}{\kern0pt}\ P{\isacharunderscore}{\kern0pt}auto{\isasymrangle}{\isacharparenright}{\kern0pt}{\isachardoublequoteclose}\ \isakeyword{if}\ {\isachardoublequoteopen}d{\isasymin}M{\isachardoublequoteclose}\ \isakeyword{for}\ d\isanewline
\ \ \ \ \isacommand{apply}\isamarkupfalse%
{\isacharparenleft}{\kern0pt}subgoal{\isacharunderscore}{\kern0pt}tac\ {\isachardoublequoteopen}{\isasymlangle}{\isasymF}{\isacharcomma}{\kern0pt}\ {\isasymG}{\isacharcomma}{\kern0pt}\ P{\isacharcomma}{\kern0pt}\ P{\isacharunderscore}{\kern0pt}auto{\isasymrangle}\ {\isasymin}\ M{\isachardoublequoteclose}{\isacharparenright}{\kern0pt}\isanewline
\ \ \ \ \isacommand{using}\isamarkupfalse%
\ assms\ that\ P{\isacharunderscore}{\kern0pt}in{\isacharunderscore}{\kern0pt}M\ leq{\isacharunderscore}{\kern0pt}in{\isacharunderscore}{\kern0pt}M\ one{\isacharunderscore}{\kern0pt}in{\isacharunderscore}{\kern0pt}M\ sats{\isacharunderscore}{\kern0pt}leq{\isacharunderscore}{\kern0pt}fm\ \ \isanewline
\ \ \ \ \ \isacommand{apply}\isamarkupfalse%
\ simp\isanewline
\ \ \ \ \ \isacommand{apply}\isamarkupfalse%
{\isacharparenleft}{\kern0pt}rule\ iff{\isacharunderscore}{\kern0pt}lemma{\isacharparenright}{\kern0pt}\isanewline
\ \ \ \ \ \ \isacommand{apply}\isamarkupfalse%
{\isacharparenleft}{\kern0pt}rule\ iff{\isacharunderscore}{\kern0pt}flip{\isacharcomma}{\kern0pt}\ rule\ sats{\isacharunderscore}{\kern0pt}is{\isacharunderscore}{\kern0pt}HS{\isacharunderscore}{\kern0pt}fm{\isacharunderscore}{\kern0pt}iff{\isacharcomma}{\kern0pt}\ force{\isacharcomma}{\kern0pt}\ force{\isacharcomma}{\kern0pt}\ force{\isacharcomma}{\kern0pt}\ force{\isacharcomma}{\kern0pt}\ force{\isacharparenright}{\kern0pt}\isanewline
\ \ \ \ \ \isacommand{apply}\isamarkupfalse%
{\isacharparenleft}{\kern0pt}rule\ ball{\isacharunderscore}{\kern0pt}iff{\isacharcomma}{\kern0pt}\ rule\ imp{\isacharunderscore}{\kern0pt}iff{\isacharcomma}{\kern0pt}\ simp{\isacharcomma}{\kern0pt}\ rule\ notnot{\isacharunderscore}{\kern0pt}iff{\isacharparenright}{\kern0pt}\isanewline
\ \ \ \ \ \isacommand{apply}\isamarkupfalse%
{\isacharparenleft}{\kern0pt}rename{\isacharunderscore}{\kern0pt}tac\ x\ y{\isacharcomma}{\kern0pt}\ rule{\isacharunderscore}{\kern0pt}tac\ Q{\isacharequal}{\kern0pt}{\isachardoublequoteopen}M{\isacharcomma}{\kern0pt}\ {\isacharbrackleft}{\kern0pt}y{\isacharcomma}{\kern0pt}\ x{\isacharcomma}{\kern0pt}\ d{\isacharcomma}{\kern0pt}\ P{\isacharcomma}{\kern0pt}\ leq{\isacharcomma}{\kern0pt}\ one{\isacharcomma}{\kern0pt}\ {\isasymlangle}{\isasymF}{\isacharcomma}{\kern0pt}\ {\isasymG}{\isacharcomma}{\kern0pt}\ P{\isacharcomma}{\kern0pt}\ P{\isacharunderscore}{\kern0pt}auto{\isasymrangle}{\isacharbrackright}{\kern0pt}\ {\isacharat}{\kern0pt}\ env\ {\isasymTurnstile}\ ren{\isacharunderscore}{\kern0pt}HS{\isacharunderscore}{\kern0pt}truth{\isacharunderscore}{\kern0pt}lemma{\isacharparenleft}{\kern0pt}forcesHS{\isacharparenleft}{\kern0pt}{\isasymphi}{\isacharparenright}{\kern0pt}{\isacharparenright}{\kern0pt}{\isachardoublequoteclose}\ \isakeyword{in}\ iff{\isacharunderscore}{\kern0pt}trans{\isacharparenright}{\kern0pt}\isanewline
\ \ \ \ \ \ \isacommand{apply}\isamarkupfalse%
\ force\isanewline
\ \ \ \ \ \isacommand{apply}\isamarkupfalse%
{\isacharparenleft}{\kern0pt}rule\ iff{\isacharunderscore}{\kern0pt}trans{\isacharcomma}{\kern0pt}\ rule\ sats{\isacharunderscore}{\kern0pt}ren{\isacharunderscore}{\kern0pt}HS{\isacharunderscore}{\kern0pt}truth{\isacharunderscore}{\kern0pt}lemma{\isacharparenright}{\kern0pt}\isanewline
\ \ \ \ \isacommand{using}\isamarkupfalse%
\ HS{\isacharunderscore}{\kern0pt}iff\ P{\isacharunderscore}{\kern0pt}name{\isacharunderscore}{\kern0pt}in{\isacharunderscore}{\kern0pt}M\isanewline
\ \ \ \ \ \ \ \isacommand{apply}\isamarkupfalse%
\ force\isanewline
\ \ \ \ \ \ \isacommand{apply}\isamarkupfalse%
{\isacharparenleft}{\kern0pt}rule\ forcesHS{\isacharunderscore}{\kern0pt}type{\isacharcomma}{\kern0pt}\ simp{\isacharcomma}{\kern0pt}\ force{\isacharparenright}{\kern0pt}\isanewline
\ \ \ \ \isacommand{using}\isamarkupfalse%
\ {\isasymF}{\isacharunderscore}{\kern0pt}in{\isacharunderscore}{\kern0pt}M\ {\isasymG}{\isacharunderscore}{\kern0pt}in{\isacharunderscore}{\kern0pt}M\ P{\isacharunderscore}{\kern0pt}in{\isacharunderscore}{\kern0pt}M\ P{\isacharunderscore}{\kern0pt}auto{\isacharunderscore}{\kern0pt}in{\isacharunderscore}{\kern0pt}M\ pair{\isacharunderscore}{\kern0pt}in{\isacharunderscore}{\kern0pt}M{\isacharunderscore}{\kern0pt}iff\isanewline
\ \ \ \ \isacommand{by}\isamarkupfalse%
\ auto\isanewline
\isanewline
\ \ \isacommand{have}\isamarkupfalse%
\ I{\isadigit{1}}{\isacharcolon}{\kern0pt}\ {\isachardoublequoteopen}separation{\isacharparenleft}{\kern0pt}{\isacharhash}{\kern0pt}{\isacharhash}{\kern0pt}M{\isacharcomma}{\kern0pt}{\isasymlambda}d{\isachardot}{\kern0pt}\ {\isasymexists}b{\isasymin}HS{\isachardot}{\kern0pt}\ {\isasymforall}q{\isasymin}P{\isachardot}{\kern0pt}\ q{\isasympreceq}d\ {\isasymlongrightarrow}\ {\isasymnot}{\isacharparenleft}{\kern0pt}q\ {\isasymtturnstile}HS\ {\isasymphi}\ {\isacharparenleft}{\kern0pt}{\isacharbrackleft}{\kern0pt}b{\isacharbrackright}{\kern0pt}{\isacharat}{\kern0pt}env{\isacharparenright}{\kern0pt}{\isacharparenright}{\kern0pt}{\isacharparenright}{\kern0pt}\ {\isasymlongleftrightarrow}\ \isanewline
\ \ \ \ \ \ \ \ separation{\isacharparenleft}{\kern0pt}{\isacharhash}{\kern0pt}{\isacharhash}{\kern0pt}M{\isacharcomma}{\kern0pt}{\isasymlambda}d{\isachardot}{\kern0pt}\ {\isacharquery}{\kern0pt}rel{\isacharunderscore}{\kern0pt}pred{\isacharparenleft}{\kern0pt}M{\isacharcomma}{\kern0pt}d{\isacharcomma}{\kern0pt}P{\isacharcomma}{\kern0pt}leq{\isacharcomma}{\kern0pt}one{\isacharcomma}{\kern0pt}{\isasymlangle}{\isasymF}{\isacharcomma}{\kern0pt}\ {\isasymG}{\isacharcomma}{\kern0pt}\ P{\isacharcomma}{\kern0pt}\ P{\isacharunderscore}{\kern0pt}auto{\isasymrangle}{\isacharparenright}{\kern0pt}{\isacharparenright}{\kern0pt}{\isachardoublequoteclose}\ \isanewline
\ \ \ \ \isacommand{apply}\isamarkupfalse%
{\isacharparenleft}{\kern0pt}rule\ separation{\isacharunderscore}{\kern0pt}cong{\isacharparenright}{\kern0pt}\isanewline
\ \ \ \ \isacommand{using}\isamarkupfalse%
\ P{\isacharunderscore}{\kern0pt}in{\isacharunderscore}{\kern0pt}M\ leq{\isacharunderscore}{\kern0pt}in{\isacharunderscore}{\kern0pt}M\ transM\ \isanewline
\ \ \ \ \isacommand{by}\isamarkupfalse%
\ auto\isanewline
\ \ \isacommand{have}\isamarkupfalse%
\ I{\isadigit{2}}{\isacharcolon}{\kern0pt}\ {\isachardoublequoteopen}{\isachardot}{\kern0pt}{\isachardot}{\kern0pt}{\isachardot}{\kern0pt}\ {\isasymlongleftrightarrow}\ separation{\isacharparenleft}{\kern0pt}{\isacharhash}{\kern0pt}{\isacharhash}{\kern0pt}M{\isacharcomma}{\kern0pt}{\isasymlambda}d{\isachardot}{\kern0pt}\ {\isacharparenleft}{\kern0pt}M{\isacharcomma}{\kern0pt}\ {\isacharbrackleft}{\kern0pt}d{\isacharcomma}{\kern0pt}P{\isacharcomma}{\kern0pt}leq{\isacharcomma}{\kern0pt}one{\isacharcomma}{\kern0pt}{\isasymlangle}{\isasymF}{\isacharcomma}{\kern0pt}\ {\isasymG}{\isacharcomma}{\kern0pt}\ P{\isacharcomma}{\kern0pt}\ P{\isacharunderscore}{\kern0pt}auto{\isasymrangle}{\isacharbrackright}{\kern0pt}{\isacharat}{\kern0pt}env\ {\isasymTurnstile}\ {\isacharquery}{\kern0pt}{\isasympsi}{\isacharparenright}{\kern0pt}{\isacharparenright}{\kern0pt}{\isachardoublequoteclose}\isanewline
\ \ \ \ \isacommand{apply}\isamarkupfalse%
{\isacharparenleft}{\kern0pt}rule\ separation{\isacharunderscore}{\kern0pt}cong{\isacharparenright}{\kern0pt}\isanewline
\ \ \ \ \isacommand{apply}\isamarkupfalse%
{\isacharparenleft}{\kern0pt}rule\ iff{\isacharunderscore}{\kern0pt}flip{\isacharcomma}{\kern0pt}\ rule\ H{\isacharcomma}{\kern0pt}\ force{\isacharparenright}{\kern0pt}\isanewline
\ \ \ \ \isacommand{done}\isamarkupfalse%
\isanewline
\ \ \isacommand{have}\isamarkupfalse%
\ I{\isadigit{3}}{\isacharcolon}{\kern0pt}\ {\isachardoublequoteopen}{\isachardot}{\kern0pt}{\isachardot}{\kern0pt}{\isachardot}{\kern0pt}\ {\isasymlongleftrightarrow}\ separation{\isacharparenleft}{\kern0pt}{\isacharhash}{\kern0pt}{\isacharhash}{\kern0pt}M{\isacharcomma}{\kern0pt}{\isasymlambda}d{\isachardot}{\kern0pt}\ {\isacharparenleft}{\kern0pt}M{\isacharcomma}{\kern0pt}\ {\isacharbrackleft}{\kern0pt}d{\isacharbrackright}{\kern0pt}\ {\isacharat}{\kern0pt}\ {\isacharbrackleft}{\kern0pt}P{\isacharcomma}{\kern0pt}leq{\isacharcomma}{\kern0pt}one{\isacharcomma}{\kern0pt}{\isasymlangle}{\isasymF}{\isacharcomma}{\kern0pt}\ {\isasymG}{\isacharcomma}{\kern0pt}\ P{\isacharcomma}{\kern0pt}\ P{\isacharunderscore}{\kern0pt}auto{\isasymrangle}{\isacharbrackright}{\kern0pt}{\isacharat}{\kern0pt}env\ {\isasymTurnstile}\ {\isacharquery}{\kern0pt}{\isasympsi}{\isacharparenright}{\kern0pt}{\isacharparenright}{\kern0pt}{\isachardoublequoteclose}\ \isacommand{by}\isamarkupfalse%
\ auto\isanewline
\isanewline
\ \ \isacommand{have}\isamarkupfalse%
\ {\isachardoublequoteopen}separation{\isacharparenleft}{\kern0pt}{\isacharhash}{\kern0pt}{\isacharhash}{\kern0pt}M{\isacharcomma}{\kern0pt}{\isasymlambda}d{\isachardot}{\kern0pt}\ {\isacharparenleft}{\kern0pt}M{\isacharcomma}{\kern0pt}\ {\isacharbrackleft}{\kern0pt}d{\isacharbrackright}{\kern0pt}\ {\isacharat}{\kern0pt}\ {\isacharbrackleft}{\kern0pt}P{\isacharcomma}{\kern0pt}leq{\isacharcomma}{\kern0pt}one{\isacharcomma}{\kern0pt}{\isasymlangle}{\isasymF}{\isacharcomma}{\kern0pt}\ {\isasymG}{\isacharcomma}{\kern0pt}\ P{\isacharcomma}{\kern0pt}\ P{\isacharunderscore}{\kern0pt}auto{\isasymrangle}{\isacharbrackright}{\kern0pt}{\isacharat}{\kern0pt}env\ {\isasymTurnstile}\ {\isacharquery}{\kern0pt}{\isasympsi}{\isacharparenright}{\kern0pt}{\isacharparenright}{\kern0pt}{\isachardoublequoteclose}\isanewline
\ \ \ \ \isacommand{apply}\isamarkupfalse%
{\isacharparenleft}{\kern0pt}subgoal{\isacharunderscore}{\kern0pt}tac\ {\isachardoublequoteopen}is{\isacharunderscore}{\kern0pt}HS{\isacharunderscore}{\kern0pt}fm{\isacharparenleft}{\kern0pt}{\isadigit{5}}{\isacharcomma}{\kern0pt}\ {\isadigit{0}}{\isacharparenright}{\kern0pt}\ {\isasymin}\ formula{\isachardoublequoteclose}{\isacharparenright}{\kern0pt}\isanewline
\ \ \ \ \isacommand{apply}\isamarkupfalse%
{\isacharparenleft}{\kern0pt}rule\ separation{\isacharunderscore}{\kern0pt}ax{\isacharparenright}{\kern0pt}\isanewline
\ \ \ \ \isacommand{apply}\isamarkupfalse%
{\isacharparenleft}{\kern0pt}rule\ Exists{\isacharunderscore}{\kern0pt}type{\isacharcomma}{\kern0pt}\ rule\ And{\isacharunderscore}{\kern0pt}type{\isacharcomma}{\kern0pt}\ rule\ is{\isacharunderscore}{\kern0pt}HS{\isacharunderscore}{\kern0pt}fm{\isacharunderscore}{\kern0pt}type{\isacharcomma}{\kern0pt}\ simp{\isacharcomma}{\kern0pt}\ simp{\isacharparenright}{\kern0pt}\isanewline
\ \ \ \ \isacommand{using}\isamarkupfalse%
\ assms\ forcesHS{\isacharunderscore}{\kern0pt}type\ is{\isacharunderscore}{\kern0pt}HS{\isacharunderscore}{\kern0pt}fm{\isacharunderscore}{\kern0pt}type\ \isanewline
\ \ \ \ \ \ \isacommand{apply}\isamarkupfalse%
\ force\isanewline
\ \ \ \ \isacommand{using}\isamarkupfalse%
\ P{\isacharunderscore}{\kern0pt}in{\isacharunderscore}{\kern0pt}M\ leq{\isacharunderscore}{\kern0pt}in{\isacharunderscore}{\kern0pt}M\ one{\isacharunderscore}{\kern0pt}in{\isacharunderscore}{\kern0pt}M\ {\isasymF}{\isacharunderscore}{\kern0pt}in{\isacharunderscore}{\kern0pt}M\ {\isasymG}{\isacharunderscore}{\kern0pt}in{\isacharunderscore}{\kern0pt}M\ P{\isacharunderscore}{\kern0pt}auto{\isacharunderscore}{\kern0pt}in{\isacharunderscore}{\kern0pt}M\ pair{\isacharunderscore}{\kern0pt}in{\isacharunderscore}{\kern0pt}M{\isacharunderscore}{\kern0pt}iff\ assms\isanewline
\ \ \ \ \ \isacommand{apply}\isamarkupfalse%
\ force\ \isanewline
\ \ \ \ \isacommand{using}\isamarkupfalse%
\ assms\isanewline
\ \ \ \ \isacommand{apply}\isamarkupfalse%
\ simp\isanewline
\ \ \ \ \isacommand{apply}\isamarkupfalse%
{\isacharparenleft}{\kern0pt}rule\ pred{\isacharunderscore}{\kern0pt}le{\isacharparenright}{\kern0pt}\isanewline
\ \ \ \ \isacommand{using}\isamarkupfalse%
\ forcesHS{\isacharunderscore}{\kern0pt}type\ assms\isanewline
\ \ \ \ \ \ \isacommand{apply}\isamarkupfalse%
\ force\ \isanewline
\ \ \ \ \ \ \isacommand{apply}\isamarkupfalse%
\ force\ \isanewline
\ \ \ \ \ \isacommand{apply}\isamarkupfalse%
{\isacharparenleft}{\kern0pt}rule\ Un{\isacharunderscore}{\kern0pt}least{\isacharunderscore}{\kern0pt}lt{\isacharcomma}{\kern0pt}\ rule\ le{\isacharunderscore}{\kern0pt}trans{\isacharcomma}{\kern0pt}\ rule\ arity{\isacharunderscore}{\kern0pt}is{\isacharunderscore}{\kern0pt}HS{\isacharunderscore}{\kern0pt}fm{\isacharcomma}{\kern0pt}\ simp{\isacharcomma}{\kern0pt}\ simp{\isacharparenright}{\kern0pt}\isanewline
\ \ \ \ \ \ \isacommand{apply}\isamarkupfalse%
{\isacharparenleft}{\kern0pt}rule\ Un{\isacharunderscore}{\kern0pt}least{\isacharunderscore}{\kern0pt}lt{\isacharcomma}{\kern0pt}\ simp\ {\isacharcomma}{\kern0pt}simp{\isacharparenright}{\kern0pt}\isanewline
\ \ \ \ \isacommand{apply}\isamarkupfalse%
{\isacharparenleft}{\kern0pt}rule\ pred{\isacharunderscore}{\kern0pt}le{\isacharparenright}{\kern0pt}\isanewline
\ \ \ \ \isacommand{using}\isamarkupfalse%
\ forcesHS{\isacharunderscore}{\kern0pt}type\ assms\isanewline
\ \ \ \ \ \ \isacommand{apply}\isamarkupfalse%
\ force\ \isanewline
\ \ \ \ \ \isacommand{apply}\isamarkupfalse%
\ force\ \isanewline
\ \ \ \ \isacommand{apply}\isamarkupfalse%
{\isacharparenleft}{\kern0pt}rule\ Un{\isacharunderscore}{\kern0pt}least{\isacharunderscore}{\kern0pt}lt{\isacharparenright}{\kern0pt}{\isacharplus}{\kern0pt}\isanewline
\ \ \ \ \ \ \ \isacommand{apply}\isamarkupfalse%
\ auto{\isacharbrackleft}{\kern0pt}{\isadigit{2}}{\isacharbrackright}{\kern0pt}\isanewline
\ \ \ \ \ \isacommand{apply}\isamarkupfalse%
{\isacharparenleft}{\kern0pt}subst\ arity{\isacharunderscore}{\kern0pt}leq{\isacharunderscore}{\kern0pt}fm{\isacharparenright}{\kern0pt}\isanewline
\ \ \ \ \ \ \ \ \isacommand{apply}\isamarkupfalse%
\ auto{\isacharbrackleft}{\kern0pt}{\isadigit{3}}{\isacharbrackright}{\kern0pt}\isanewline
\ \ \ \ \ \isacommand{apply}\isamarkupfalse%
{\isacharparenleft}{\kern0pt}rule\ Un{\isacharunderscore}{\kern0pt}least{\isacharunderscore}{\kern0pt}lt{\isacharparenright}{\kern0pt}{\isacharplus}{\kern0pt}\isanewline
\ \ \ \ \ \ \ \isacommand{apply}\isamarkupfalse%
\ auto{\isacharbrackleft}{\kern0pt}{\isadigit{3}}{\isacharbrackright}{\kern0pt}\isanewline
\ \ \ \ \isacommand{apply}\isamarkupfalse%
{\isacharparenleft}{\kern0pt}rule\ le{\isacharunderscore}{\kern0pt}trans{\isacharcomma}{\kern0pt}\ rule\ arity{\isacharunderscore}{\kern0pt}ren{\isacharunderscore}{\kern0pt}HS{\isacharunderscore}{\kern0pt}truth{\isacharunderscore}{\kern0pt}lemma{\isacharparenright}{\kern0pt}\isanewline
\ \ \ \ \isacommand{using}\isamarkupfalse%
\ forcesHS{\isacharunderscore}{\kern0pt}type\ \isanewline
\ \ \ \ \ \isacommand{apply}\isamarkupfalse%
\ force\ \isanewline
\ \ \ \ \isacommand{apply}\isamarkupfalse%
{\isacharparenleft}{\kern0pt}rule\ Un{\isacharunderscore}{\kern0pt}least{\isacharunderscore}{\kern0pt}lt{\isacharcomma}{\kern0pt}\ simp{\isacharcomma}{\kern0pt}\ simp{\isacharparenright}{\kern0pt}\isanewline
\ \ \ \ \isacommand{apply}\isamarkupfalse%
{\isacharparenleft}{\kern0pt}rule\ le{\isacharunderscore}{\kern0pt}trans{\isacharcomma}{\kern0pt}\ rule\ arity{\isacharunderscore}{\kern0pt}forcesHS{\isacharcomma}{\kern0pt}\ simp{\isacharparenright}{\kern0pt}\isanewline
\ \ \ \ \ \isacommand{apply}\isamarkupfalse%
\ simp\isanewline
\ \ \ \ \isacommand{apply}\isamarkupfalse%
{\isacharparenleft}{\kern0pt}rule\ is{\isacharunderscore}{\kern0pt}HS{\isacharunderscore}{\kern0pt}fm{\isacharunderscore}{\kern0pt}type{\isacharparenright}{\kern0pt}\isanewline
\ \ \ \ \isacommand{by}\isamarkupfalse%
\ auto\isanewline
\ \ \isacommand{then}\isamarkupfalse%
\ \isacommand{show}\isamarkupfalse%
\ {\isachardoublequoteopen}separation{\isacharparenleft}{\kern0pt}{\isacharhash}{\kern0pt}{\isacharhash}{\kern0pt}M{\isacharcomma}{\kern0pt}\ {\isasymlambda}d{\isachardot}{\kern0pt}\ {\isasymexists}b{\isasymin}HS{\isachardot}{\kern0pt}\ {\isasymforall}q{\isasymin}P{\isachardot}{\kern0pt}\ q\ {\isasympreceq}\ d\ {\isasymlongrightarrow}\ satisfies{\isacharparenleft}{\kern0pt}M{\isacharcomma}{\kern0pt}\ forcesHS{\isacharparenleft}{\kern0pt}{\isasymphi}{\isacharparenright}{\kern0pt}{\isacharparenright}{\kern0pt}\ {\isacharbackquote}{\kern0pt}\ {\isacharparenleft}{\kern0pt}{\isacharbrackleft}{\kern0pt}q{\isacharcomma}{\kern0pt}\ P{\isacharcomma}{\kern0pt}\ leq{\isacharcomma}{\kern0pt}\ one{\isacharcomma}{\kern0pt}\ {\isasymlangle}{\isasymF}{\isacharcomma}{\kern0pt}\ {\isasymG}{\isacharcomma}{\kern0pt}\ P{\isacharcomma}{\kern0pt}\ P{\isacharunderscore}{\kern0pt}auto{\isasymrangle}{\isacharbrackright}{\kern0pt}\ {\isacharat}{\kern0pt}\ {\isacharbrackleft}{\kern0pt}b{\isacharbrackright}{\kern0pt}\ {\isacharat}{\kern0pt}\ env{\isacharparenright}{\kern0pt}\ {\isasymnoteq}\ {\isadigit{1}}{\isacharparenright}{\kern0pt}{\isachardoublequoteclose}\ \isanewline
\ \ \ \ \isacommand{using}\isamarkupfalse%
\ I{\isadigit{1}}\ I{\isadigit{2}}\ I{\isadigit{3}}\ \isanewline
\ \ \ \ \isacommand{by}\isamarkupfalse%
\ auto\isanewline
\isacommand{qed}\isamarkupfalse%
%
\endisatagproof
{\isafoldproof}%
%
\isadelimproof
\isanewline
%
\endisadelimproof
\isanewline
\isacommand{lemma}\isamarkupfalse%
\ HS{\isacharunderscore}{\kern0pt}truth{\isacharunderscore}{\kern0pt}lemma{\isacharcolon}{\kern0pt}\isanewline
\ \ \isakeyword{assumes}\ \isanewline
\ \ \ \ {\isachardoublequoteopen}{\isasymphi}{\isasymin}formula{\isachardoublequoteclose}\ {\isachardoublequoteopen}M{\isacharunderscore}{\kern0pt}generic{\isacharparenleft}{\kern0pt}G{\isacharparenright}{\kern0pt}{\isachardoublequoteclose}\isanewline
\ \ \isakeyword{shows}\ \isanewline
\ \ \ \ \ {\isachardoublequoteopen}{\isasymAnd}env{\isachardot}{\kern0pt}\ env{\isasymin}list{\isacharparenleft}{\kern0pt}HS{\isacharparenright}{\kern0pt}\ {\isasymLongrightarrow}\ arity{\isacharparenleft}{\kern0pt}{\isasymphi}{\isacharparenright}{\kern0pt}{\isasymle}length{\isacharparenleft}{\kern0pt}env{\isacharparenright}{\kern0pt}\ {\isasymLongrightarrow}\ \isanewline
\ \ \ \ \ \ {\isacharparenleft}{\kern0pt}{\isasymexists}p{\isasymin}G{\isachardot}{\kern0pt}\ p\ {\isasymtturnstile}HS\ {\isasymphi}\ env{\isacharparenright}{\kern0pt}\ {\isasymlongleftrightarrow}\ SymExt{\isacharparenleft}{\kern0pt}G{\isacharparenright}{\kern0pt}{\isacharcomma}{\kern0pt}\ map{\isacharparenleft}{\kern0pt}val{\isacharparenleft}{\kern0pt}G{\isacharparenright}{\kern0pt}{\isacharcomma}{\kern0pt}env{\isacharparenright}{\kern0pt}\ {\isasymTurnstile}\ {\isasymphi}{\isachardoublequoteclose}\isanewline
%
\isadelimproof
\ \ %
\endisadelimproof
%
\isatagproof
\isacommand{using}\isamarkupfalse%
\ assms{\isacharparenleft}{\kern0pt}{\isadigit{1}}{\isacharparenright}{\kern0pt}\isanewline
\isacommand{proof}\isamarkupfalse%
\ {\isacharparenleft}{\kern0pt}induct{\isacharparenright}{\kern0pt}\isanewline
\ \ \isacommand{case}\isamarkupfalse%
\ {\isacharparenleft}{\kern0pt}Member\ x\ y{\isacharparenright}{\kern0pt}\isanewline
\ \ \isacommand{then}\isamarkupfalse%
\ \isacommand{have}\isamarkupfalse%
\ assms{\isadigit{1}}{\isacharcolon}{\kern0pt}\ {\isachardoublequoteopen}x\ {\isasymin}\ nat{\isachardoublequoteclose}\ {\isachardoublequoteopen}y\ {\isasymin}\ nat{\isachardoublequoteclose}\ {\isachardoublequoteopen}env\ {\isasymin}\ list{\isacharparenleft}{\kern0pt}HS{\isacharparenright}{\kern0pt}{\isachardoublequoteclose}\ {\isachardoublequoteopen}arity{\isacharparenleft}{\kern0pt}Member{\isacharparenleft}{\kern0pt}x{\isacharcomma}{\kern0pt}\ y{\isacharparenright}{\kern0pt}{\isacharparenright}{\kern0pt}\ {\isasymle}\ length{\isacharparenleft}{\kern0pt}env{\isacharparenright}{\kern0pt}{\isachardoublequoteclose}\ \isacommand{by}\isamarkupfalse%
\ auto\isanewline
\isanewline
\ \ \isacommand{have}\isamarkupfalse%
\ GsubsetM\ {\isacharcolon}{\kern0pt}\ {\isachardoublequoteopen}G\ {\isasymsubseteq}\ M{\isachardoublequoteclose}\ \isanewline
\ \ \ \ \isacommand{apply}\isamarkupfalse%
{\isacharparenleft}{\kern0pt}rule{\isacharunderscore}{\kern0pt}tac\ B{\isacharequal}{\kern0pt}P\ \isakeyword{in}\ subset{\isacharunderscore}{\kern0pt}trans{\isacharparenright}{\kern0pt}\isanewline
\ \ \ \ \isacommand{using}\isamarkupfalse%
\ assms\ M{\isacharunderscore}{\kern0pt}generic{\isacharunderscore}{\kern0pt}def\ filter{\isacharunderscore}{\kern0pt}def\ P{\isacharunderscore}{\kern0pt}in{\isacharunderscore}{\kern0pt}M\ transM\ \isanewline
\ \ \ \ \isacommand{by}\isamarkupfalse%
\ auto\isanewline
\isanewline
\ \ \isacommand{have}\isamarkupfalse%
\ envin\ {\isacharcolon}{\kern0pt}\ {\isachardoublequoteopen}env\ {\isasymin}\ list{\isacharparenleft}{\kern0pt}M{\isacharparenright}{\kern0pt}{\isachardoublequoteclose}\ \isanewline
\ \ \ \ \isacommand{apply}\isamarkupfalse%
{\isacharparenleft}{\kern0pt}rule{\isacharunderscore}{\kern0pt}tac\ A{\isacharequal}{\kern0pt}{\isachardoublequoteopen}list{\isacharparenleft}{\kern0pt}HS{\isacharparenright}{\kern0pt}{\isachardoublequoteclose}\ \isakeyword{in}\ subsetD{\isacharparenright}{\kern0pt}\isanewline
\ \ \ \ \ \isacommand{apply}\isamarkupfalse%
{\isacharparenleft}{\kern0pt}rule\ list{\isacharunderscore}{\kern0pt}mono{\isacharparenright}{\kern0pt}\isanewline
\ \ \ \ \isacommand{using}\isamarkupfalse%
\ assms{\isadigit{1}}\ HS{\isacharunderscore}{\kern0pt}iff\ P{\isacharunderscore}{\kern0pt}name{\isacharunderscore}{\kern0pt}in{\isacharunderscore}{\kern0pt}M\ \isanewline
\ \ \ \ \isacommand{by}\isamarkupfalse%
\ auto\isanewline
\isanewline
\ \ \isacommand{have}\isamarkupfalse%
\ envmaptype\ {\isacharcolon}{\kern0pt}\ {\isachardoublequoteopen}map{\isacharparenleft}{\kern0pt}val{\isacharparenleft}{\kern0pt}G{\isacharparenright}{\kern0pt}{\isacharcomma}{\kern0pt}\ env{\isacharparenright}{\kern0pt}\ {\isasymin}\ list{\isacharparenleft}{\kern0pt}SymExt{\isacharparenleft}{\kern0pt}G{\isacharparenright}{\kern0pt}{\isacharparenright}{\kern0pt}{\isachardoublequoteclose}\isanewline
\ \ \ \ \isacommand{apply}\isamarkupfalse%
{\isacharparenleft}{\kern0pt}rule\ map{\isacharunderscore}{\kern0pt}type{\isacharparenright}{\kern0pt}\isanewline
\ \ \ \ \isacommand{using}\isamarkupfalse%
\ assms{\isadigit{1}}\ SymExt{\isacharunderscore}{\kern0pt}def\isanewline
\ \ \ \ \isacommand{by}\isamarkupfalse%
\ auto\isanewline
\ \ \ \ \ \ \ \ \ \ \ \ \ \ \ \ \ \ \ \ \ \ \ \ \ \ \ \ \ \ \ \ \ \ \ \ \ \ \ \ \ \ \ \ \ \isanewline
\ \ \isacommand{have}\isamarkupfalse%
\ H{\isacharcolon}{\kern0pt}\ {\isachardoublequoteopen}M{\isacharunderscore}{\kern0pt}symmetric{\isacharunderscore}{\kern0pt}system{\isacharunderscore}{\kern0pt}G{\isacharunderscore}{\kern0pt}generic{\isacharparenleft}{\kern0pt}P{\isacharcomma}{\kern0pt}\ leq{\isacharcomma}{\kern0pt}\ one{\isacharcomma}{\kern0pt}\ M{\isacharcomma}{\kern0pt}\ enum{\isacharcomma}{\kern0pt}\ {\isasymG}{\isacharcomma}{\kern0pt}\ {\isasymF}{\isacharcomma}{\kern0pt}\ G{\isacharparenright}{\kern0pt}{\isachardoublequoteclose}\isanewline
\ \ \ \ \isacommand{unfolding}\isamarkupfalse%
\ M{\isacharunderscore}{\kern0pt}symmetric{\isacharunderscore}{\kern0pt}system{\isacharunderscore}{\kern0pt}G{\isacharunderscore}{\kern0pt}generic{\isacharunderscore}{\kern0pt}def\ G{\isacharunderscore}{\kern0pt}generic{\isacharunderscore}{\kern0pt}def\ G{\isacharunderscore}{\kern0pt}generic{\isacharunderscore}{\kern0pt}axioms{\isacharunderscore}{\kern0pt}def\isanewline
\ \ \ \ \isacommand{using}\isamarkupfalse%
\ M{\isacharunderscore}{\kern0pt}symmetric{\isacharunderscore}{\kern0pt}system{\isacharunderscore}{\kern0pt}axioms\ forcing{\isacharunderscore}{\kern0pt}data{\isacharunderscore}{\kern0pt}axioms\ assms\isanewline
\ \ \ \ \isacommand{by}\isamarkupfalse%
\ auto\isanewline
\isanewline
\ \ \isacommand{have}\isamarkupfalse%
\ I{\isadigit{1}}{\isacharcolon}{\kern0pt}\ {\isachardoublequoteopen}{\isacharparenleft}{\kern0pt}{\isasymexists}p{\isasymin}G{\isachardot}{\kern0pt}\ M{\isacharcomma}{\kern0pt}\ {\isacharbrackleft}{\kern0pt}p{\isacharcomma}{\kern0pt}\ P{\isacharcomma}{\kern0pt}\ leq{\isacharcomma}{\kern0pt}\ one{\isacharcomma}{\kern0pt}\ {\isasymlangle}{\isasymF}{\isacharcomma}{\kern0pt}\ {\isasymG}{\isacharcomma}{\kern0pt}\ P{\isacharcomma}{\kern0pt}\ P{\isacharunderscore}{\kern0pt}auto{\isasymrangle}{\isacharbrackright}{\kern0pt}\ {\isacharat}{\kern0pt}\ env\ {\isasymTurnstile}\ forcesHS{\isacharparenleft}{\kern0pt}Member{\isacharparenleft}{\kern0pt}x{\isacharcomma}{\kern0pt}\ y{\isacharparenright}{\kern0pt}{\isacharparenright}{\kern0pt}{\isacharparenright}{\kern0pt}\ {\isasymlongleftrightarrow}\ \isanewline
\ \ \ \ \ \ \ \ \ \ \ \ \ {\isacharparenleft}{\kern0pt}{\isasymexists}p{\isasymin}G{\isachardot}{\kern0pt}\ M{\isacharcomma}{\kern0pt}\ {\isacharbrackleft}{\kern0pt}p{\isacharcomma}{\kern0pt}\ P{\isacharcomma}{\kern0pt}\ leq{\isacharcomma}{\kern0pt}\ one{\isacharbrackright}{\kern0pt}\ {\isacharat}{\kern0pt}\ env\ {\isasymTurnstile}\ forces{\isacharparenleft}{\kern0pt}Member{\isacharparenleft}{\kern0pt}x{\isacharcomma}{\kern0pt}\ y{\isacharparenright}{\kern0pt}{\isacharparenright}{\kern0pt}{\isacharparenright}{\kern0pt}{\isachardoublequoteclose}\ \isanewline
\ \ \ \ \isacommand{apply}\isamarkupfalse%
{\isacharparenleft}{\kern0pt}rule\ bex{\isacharunderscore}{\kern0pt}iff{\isacharcomma}{\kern0pt}\ rule\ sats{\isacharunderscore}{\kern0pt}forcesHS{\isacharunderscore}{\kern0pt}Member{\isacharparenright}{\kern0pt}\isanewline
\ \ \ \ \isacommand{using}\isamarkupfalse%
\ assms{\isadigit{1}}\ {\isasymF}{\isacharunderscore}{\kern0pt}in{\isacharunderscore}{\kern0pt}M\ {\isasymG}{\isacharunderscore}{\kern0pt}in{\isacharunderscore}{\kern0pt}M\ P{\isacharunderscore}{\kern0pt}in{\isacharunderscore}{\kern0pt}M\ P{\isacharunderscore}{\kern0pt}auto{\isacharunderscore}{\kern0pt}in{\isacharunderscore}{\kern0pt}M\ pair{\isacharunderscore}{\kern0pt}in{\isacharunderscore}{\kern0pt}M{\isacharunderscore}{\kern0pt}iff\ envin\ GsubsetM\isanewline
\ \ \ \ \isacommand{by}\isamarkupfalse%
\ auto\isanewline
\ \ \isacommand{have}\isamarkupfalse%
\ I{\isadigit{2}}{\isacharcolon}{\kern0pt}\ {\isachardoublequoteopen}{\isachardot}{\kern0pt}{\isachardot}{\kern0pt}{\isachardot}{\kern0pt}\ {\isasymlongleftrightarrow}\ {\isacharparenleft}{\kern0pt}{\isasymexists}p\ {\isasymin}\ G{\isachardot}{\kern0pt}\ p\ {\isasymtturnstile}\ Member{\isacharparenleft}{\kern0pt}x{\isacharcomma}{\kern0pt}\ y{\isacharparenright}{\kern0pt}\ env{\isacharparenright}{\kern0pt}{\isachardoublequoteclose}\ \isacommand{by}\isamarkupfalse%
\ auto\isanewline
\ \ \isacommand{have}\isamarkupfalse%
\ I{\isadigit{3}}{\isacharcolon}{\kern0pt}\ {\isachardoublequoteopen}{\isachardot}{\kern0pt}{\isachardot}{\kern0pt}{\isachardot}{\kern0pt}\ {\isasymlongleftrightarrow}\ M{\isacharbrackleft}{\kern0pt}G{\isacharbrackright}{\kern0pt}{\isacharcomma}{\kern0pt}\ map{\isacharparenleft}{\kern0pt}val{\isacharparenleft}{\kern0pt}G{\isacharparenright}{\kern0pt}{\isacharcomma}{\kern0pt}\ env{\isacharparenright}{\kern0pt}\ {\isasymTurnstile}\ Member{\isacharparenleft}{\kern0pt}x{\isacharcomma}{\kern0pt}\ y{\isacharparenright}{\kern0pt}{\isachardoublequoteclose}\ \isanewline
\ \ \ \ \isacommand{apply}\isamarkupfalse%
{\isacharparenleft}{\kern0pt}rule\ truth{\isacharunderscore}{\kern0pt}lemma{\isacharparenright}{\kern0pt}\isanewline
\ \ \ \ \isacommand{using}\isamarkupfalse%
\ assms{\isadigit{1}}\ assms\ envin\ \isanewline
\ \ \ \ \isacommand{by}\isamarkupfalse%
\ auto\isanewline
\ \ \isacommand{have}\isamarkupfalse%
\ I{\isadigit{4}}\ {\isacharcolon}{\kern0pt}\ {\isachardoublequoteopen}{\isachardot}{\kern0pt}{\isachardot}{\kern0pt}{\isachardot}{\kern0pt}\ {\isasymlongleftrightarrow}\ SymExt{\isacharparenleft}{\kern0pt}G{\isacharparenright}{\kern0pt}{\isacharcomma}{\kern0pt}\ map{\isacharparenleft}{\kern0pt}val{\isacharparenleft}{\kern0pt}G{\isacharparenright}{\kern0pt}{\isacharcomma}{\kern0pt}\ env{\isacharparenright}{\kern0pt}\ {\isasymTurnstile}\ Member{\isacharparenleft}{\kern0pt}x{\isacharcomma}{\kern0pt}\ y{\isacharparenright}{\kern0pt}{\isachardoublequoteclose}\ \isanewline
\ \ \ \ \isacommand{apply}\isamarkupfalse%
{\isacharparenleft}{\kern0pt}rule\ iff{\isacharunderscore}{\kern0pt}flip{\isacharcomma}{\kern0pt}\ rule\ {\isasymDelta}{\isadigit{0}}{\isacharunderscore}{\kern0pt}sats{\isacharunderscore}{\kern0pt}iff{\isacharparenright}{\kern0pt}\isanewline
\ \ \ \ \isacommand{using}\isamarkupfalse%
\ H\ M{\isacharunderscore}{\kern0pt}symmetric{\isacharunderscore}{\kern0pt}system{\isacharunderscore}{\kern0pt}G{\isacharunderscore}{\kern0pt}generic{\isachardot}{\kern0pt}SymExt{\isacharunderscore}{\kern0pt}subset{\isacharunderscore}{\kern0pt}GenExt\ envmaptype\ Member{\isacharunderscore}{\kern0pt}{\isasymDelta}{\isadigit{0}}\ assms{\isadigit{1}}\ M{\isacharunderscore}{\kern0pt}symmetric{\isacharunderscore}{\kern0pt}system{\isacharunderscore}{\kern0pt}G{\isacharunderscore}{\kern0pt}generic{\isachardot}{\kern0pt}Transset{\isacharunderscore}{\kern0pt}SymExt\ length{\isacharunderscore}{\kern0pt}map\ \isanewline
\ \ \ \ \isacommand{by}\isamarkupfalse%
\ auto\isanewline
\ \ \isacommand{show}\isamarkupfalse%
\ {\isachardoublequoteopen}{\isacharparenleft}{\kern0pt}{\isasymexists}p{\isasymin}G{\isachardot}{\kern0pt}\ M{\isacharcomma}{\kern0pt}\ {\isacharbrackleft}{\kern0pt}p{\isacharcomma}{\kern0pt}\ P{\isacharcomma}{\kern0pt}\ leq{\isacharcomma}{\kern0pt}\ one{\isacharcomma}{\kern0pt}\ {\isasymlangle}{\isasymF}{\isacharcomma}{\kern0pt}\ {\isasymG}{\isacharcomma}{\kern0pt}\ P{\isacharcomma}{\kern0pt}\ P{\isacharunderscore}{\kern0pt}auto{\isasymrangle}{\isacharbrackright}{\kern0pt}\ {\isacharat}{\kern0pt}\ env\ {\isasymTurnstile}\ forcesHS{\isacharparenleft}{\kern0pt}Member{\isacharparenleft}{\kern0pt}x{\isacharcomma}{\kern0pt}\ y{\isacharparenright}{\kern0pt}{\isacharparenright}{\kern0pt}{\isacharparenright}{\kern0pt}\ {\isasymlongleftrightarrow}\isanewline
\ \ \ \ \ \ \ \ \ \ SymExt{\isacharparenleft}{\kern0pt}G{\isacharparenright}{\kern0pt}{\isacharcomma}{\kern0pt}\ map{\isacharparenleft}{\kern0pt}val{\isacharparenleft}{\kern0pt}G{\isacharparenright}{\kern0pt}{\isacharcomma}{\kern0pt}\ env{\isacharparenright}{\kern0pt}\ {\isasymTurnstile}\ Member{\isacharparenleft}{\kern0pt}x{\isacharcomma}{\kern0pt}\ y{\isacharparenright}{\kern0pt}{\isachardoublequoteclose}\isanewline
\ \ \ \ \isacommand{using}\isamarkupfalse%
\ I{\isadigit{1}}\ I{\isadigit{2}}\ I{\isadigit{3}}\ I{\isadigit{4}}\ \isanewline
\ \ \ \ \isacommand{by}\isamarkupfalse%
\ auto\isanewline
\isanewline
\isacommand{next}\isamarkupfalse%
\isanewline
\ \ \isacommand{case}\isamarkupfalse%
\ {\isacharparenleft}{\kern0pt}Equal\ x\ y{\isacharparenright}{\kern0pt}\isanewline
\ \ \isacommand{then}\isamarkupfalse%
\ \isacommand{have}\isamarkupfalse%
\ assms{\isadigit{1}}{\isacharcolon}{\kern0pt}\ {\isachardoublequoteopen}x\ {\isasymin}\ nat{\isachardoublequoteclose}\ {\isachardoublequoteopen}y\ {\isasymin}\ nat{\isachardoublequoteclose}\ {\isachardoublequoteopen}env\ {\isasymin}\ list{\isacharparenleft}{\kern0pt}HS{\isacharparenright}{\kern0pt}{\isachardoublequoteclose}\ {\isachardoublequoteopen}arity{\isacharparenleft}{\kern0pt}Equal{\isacharparenleft}{\kern0pt}x{\isacharcomma}{\kern0pt}\ y{\isacharparenright}{\kern0pt}{\isacharparenright}{\kern0pt}\ {\isasymle}\ length{\isacharparenleft}{\kern0pt}env{\isacharparenright}{\kern0pt}{\isachardoublequoteclose}\ \isacommand{by}\isamarkupfalse%
\ auto\isanewline
\ \ \ \ \ \ \ \ \ \ \ \ \ \ \ \ \ \ \ \ \ \ \ \ \ \ \ \ \ \ \ \ \ \ \ \ \ \ \ \ \ \ \ \ \ \isanewline
\ \ \isacommand{have}\isamarkupfalse%
\ H{\isacharcolon}{\kern0pt}\ {\isachardoublequoteopen}M{\isacharunderscore}{\kern0pt}symmetric{\isacharunderscore}{\kern0pt}system{\isacharunderscore}{\kern0pt}G{\isacharunderscore}{\kern0pt}generic{\isacharparenleft}{\kern0pt}P{\isacharcomma}{\kern0pt}\ leq{\isacharcomma}{\kern0pt}\ one{\isacharcomma}{\kern0pt}\ M{\isacharcomma}{\kern0pt}\ enum{\isacharcomma}{\kern0pt}\ {\isasymG}{\isacharcomma}{\kern0pt}\ {\isasymF}{\isacharcomma}{\kern0pt}\ G{\isacharparenright}{\kern0pt}{\isachardoublequoteclose}\isanewline
\ \ \ \ \isacommand{unfolding}\isamarkupfalse%
\ M{\isacharunderscore}{\kern0pt}symmetric{\isacharunderscore}{\kern0pt}system{\isacharunderscore}{\kern0pt}G{\isacharunderscore}{\kern0pt}generic{\isacharunderscore}{\kern0pt}def\ G{\isacharunderscore}{\kern0pt}generic{\isacharunderscore}{\kern0pt}def\ G{\isacharunderscore}{\kern0pt}generic{\isacharunderscore}{\kern0pt}axioms{\isacharunderscore}{\kern0pt}def\isanewline
\ \ \ \ \isacommand{using}\isamarkupfalse%
\ M{\isacharunderscore}{\kern0pt}symmetric{\isacharunderscore}{\kern0pt}system{\isacharunderscore}{\kern0pt}axioms\ forcing{\isacharunderscore}{\kern0pt}data{\isacharunderscore}{\kern0pt}axioms\ assms\isanewline
\ \ \ \ \isacommand{by}\isamarkupfalse%
\ auto\isanewline
\isanewline
\ \ \isacommand{have}\isamarkupfalse%
\ GsubsetM\ {\isacharcolon}{\kern0pt}\ {\isachardoublequoteopen}G\ {\isasymsubseteq}\ M{\isachardoublequoteclose}\ \isanewline
\ \ \ \ \isacommand{apply}\isamarkupfalse%
{\isacharparenleft}{\kern0pt}rule{\isacharunderscore}{\kern0pt}tac\ B{\isacharequal}{\kern0pt}P\ \isakeyword{in}\ subset{\isacharunderscore}{\kern0pt}trans{\isacharparenright}{\kern0pt}\isanewline
\ \ \ \ \isacommand{using}\isamarkupfalse%
\ assms\ M{\isacharunderscore}{\kern0pt}generic{\isacharunderscore}{\kern0pt}def\ filter{\isacharunderscore}{\kern0pt}def\ P{\isacharunderscore}{\kern0pt}in{\isacharunderscore}{\kern0pt}M\ transM\ \isanewline
\ \ \ \ \isacommand{by}\isamarkupfalse%
\ auto\isanewline
\isanewline
\ \ \isacommand{have}\isamarkupfalse%
\ envin\ {\isacharcolon}{\kern0pt}\ {\isachardoublequoteopen}env\ {\isasymin}\ list{\isacharparenleft}{\kern0pt}M{\isacharparenright}{\kern0pt}{\isachardoublequoteclose}\ \isanewline
\ \ \ \ \isacommand{apply}\isamarkupfalse%
{\isacharparenleft}{\kern0pt}rule{\isacharunderscore}{\kern0pt}tac\ A{\isacharequal}{\kern0pt}{\isachardoublequoteopen}list{\isacharparenleft}{\kern0pt}HS{\isacharparenright}{\kern0pt}{\isachardoublequoteclose}\ \isakeyword{in}\ subsetD{\isacharparenright}{\kern0pt}\isanewline
\ \ \ \ \ \isacommand{apply}\isamarkupfalse%
{\isacharparenleft}{\kern0pt}rule\ list{\isacharunderscore}{\kern0pt}mono{\isacharparenright}{\kern0pt}\isanewline
\ \ \ \ \isacommand{using}\isamarkupfalse%
\ assms{\isadigit{1}}\ HS{\isacharunderscore}{\kern0pt}iff\ P{\isacharunderscore}{\kern0pt}name{\isacharunderscore}{\kern0pt}in{\isacharunderscore}{\kern0pt}M\ \isanewline
\ \ \ \ \isacommand{by}\isamarkupfalse%
\ auto\isanewline
\isanewline
\ \ \isacommand{have}\isamarkupfalse%
\ envmaptype\ {\isacharcolon}{\kern0pt}\ {\isachardoublequoteopen}map{\isacharparenleft}{\kern0pt}val{\isacharparenleft}{\kern0pt}G{\isacharparenright}{\kern0pt}{\isacharcomma}{\kern0pt}\ env{\isacharparenright}{\kern0pt}\ {\isasymin}\ list{\isacharparenleft}{\kern0pt}SymExt{\isacharparenleft}{\kern0pt}G{\isacharparenright}{\kern0pt}{\isacharparenright}{\kern0pt}{\isachardoublequoteclose}\isanewline
\ \ \ \ \isacommand{apply}\isamarkupfalse%
{\isacharparenleft}{\kern0pt}rule\ map{\isacharunderscore}{\kern0pt}type{\isacharparenright}{\kern0pt}\isanewline
\ \ \ \ \isacommand{using}\isamarkupfalse%
\ assms{\isadigit{1}}\ SymExt{\isacharunderscore}{\kern0pt}def\isanewline
\ \ \ \ \isacommand{by}\isamarkupfalse%
\ auto\isanewline
\isanewline
\ \ \isacommand{have}\isamarkupfalse%
\ I{\isadigit{1}}{\isacharcolon}{\kern0pt}\ {\isachardoublequoteopen}{\isacharparenleft}{\kern0pt}{\isasymexists}p{\isasymin}G{\isachardot}{\kern0pt}\ M{\isacharcomma}{\kern0pt}\ {\isacharbrackleft}{\kern0pt}p{\isacharcomma}{\kern0pt}\ P{\isacharcomma}{\kern0pt}\ leq{\isacharcomma}{\kern0pt}\ one{\isacharcomma}{\kern0pt}\ {\isasymlangle}{\isasymF}{\isacharcomma}{\kern0pt}\ {\isasymG}{\isacharcomma}{\kern0pt}\ P{\isacharcomma}{\kern0pt}\ P{\isacharunderscore}{\kern0pt}auto{\isasymrangle}{\isacharbrackright}{\kern0pt}\ {\isacharat}{\kern0pt}\ env\ {\isasymTurnstile}\ forcesHS{\isacharparenleft}{\kern0pt}Equal{\isacharparenleft}{\kern0pt}x{\isacharcomma}{\kern0pt}\ y{\isacharparenright}{\kern0pt}{\isacharparenright}{\kern0pt}{\isacharparenright}{\kern0pt}\ {\isasymlongleftrightarrow}\ \isanewline
\ \ \ \ \ \ \ \ \ \ \ \ \ {\isacharparenleft}{\kern0pt}{\isasymexists}p{\isasymin}G{\isachardot}{\kern0pt}\ M{\isacharcomma}{\kern0pt}\ {\isacharbrackleft}{\kern0pt}p{\isacharcomma}{\kern0pt}\ P{\isacharcomma}{\kern0pt}\ leq{\isacharcomma}{\kern0pt}\ one{\isacharbrackright}{\kern0pt}\ {\isacharat}{\kern0pt}\ env\ {\isasymTurnstile}\ forces{\isacharparenleft}{\kern0pt}Equal{\isacharparenleft}{\kern0pt}x{\isacharcomma}{\kern0pt}\ y{\isacharparenright}{\kern0pt}{\isacharparenright}{\kern0pt}{\isacharparenright}{\kern0pt}{\isachardoublequoteclose}\ \isanewline
\ \ \ \ \isacommand{apply}\isamarkupfalse%
{\isacharparenleft}{\kern0pt}rule\ bex{\isacharunderscore}{\kern0pt}iff{\isacharcomma}{\kern0pt}\ rule\ sats{\isacharunderscore}{\kern0pt}forcesHS{\isacharunderscore}{\kern0pt}Equal{\isacharparenright}{\kern0pt}\isanewline
\ \ \ \ \isacommand{using}\isamarkupfalse%
\ assms{\isadigit{1}}\ {\isasymF}{\isacharunderscore}{\kern0pt}in{\isacharunderscore}{\kern0pt}M\ {\isasymG}{\isacharunderscore}{\kern0pt}in{\isacharunderscore}{\kern0pt}M\ P{\isacharunderscore}{\kern0pt}in{\isacharunderscore}{\kern0pt}M\ P{\isacharunderscore}{\kern0pt}auto{\isacharunderscore}{\kern0pt}in{\isacharunderscore}{\kern0pt}M\ pair{\isacharunderscore}{\kern0pt}in{\isacharunderscore}{\kern0pt}M{\isacharunderscore}{\kern0pt}iff\ envin\ GsubsetM\isanewline
\ \ \ \ \isacommand{by}\isamarkupfalse%
\ auto\isanewline
\ \ \isacommand{have}\isamarkupfalse%
\ I{\isadigit{2}}{\isacharcolon}{\kern0pt}\ {\isachardoublequoteopen}{\isachardot}{\kern0pt}{\isachardot}{\kern0pt}{\isachardot}{\kern0pt}\ {\isasymlongleftrightarrow}\ {\isacharparenleft}{\kern0pt}{\isasymexists}p\ {\isasymin}\ G{\isachardot}{\kern0pt}\ p\ {\isasymtturnstile}\ Equal{\isacharparenleft}{\kern0pt}x{\isacharcomma}{\kern0pt}\ y{\isacharparenright}{\kern0pt}\ env{\isacharparenright}{\kern0pt}{\isachardoublequoteclose}\ \isacommand{by}\isamarkupfalse%
\ auto\isanewline
\ \ \isacommand{have}\isamarkupfalse%
\ I{\isadigit{3}}{\isacharcolon}{\kern0pt}\ {\isachardoublequoteopen}{\isachardot}{\kern0pt}{\isachardot}{\kern0pt}{\isachardot}{\kern0pt}\ {\isasymlongleftrightarrow}\ M{\isacharbrackleft}{\kern0pt}G{\isacharbrackright}{\kern0pt}{\isacharcomma}{\kern0pt}\ map{\isacharparenleft}{\kern0pt}val{\isacharparenleft}{\kern0pt}G{\isacharparenright}{\kern0pt}{\isacharcomma}{\kern0pt}\ env{\isacharparenright}{\kern0pt}\ {\isasymTurnstile}\ Equal{\isacharparenleft}{\kern0pt}x{\isacharcomma}{\kern0pt}\ y{\isacharparenright}{\kern0pt}{\isachardoublequoteclose}\ \isanewline
\ \ \ \ \isacommand{apply}\isamarkupfalse%
{\isacharparenleft}{\kern0pt}rule\ truth{\isacharunderscore}{\kern0pt}lemma{\isacharparenright}{\kern0pt}\isanewline
\ \ \ \ \isacommand{using}\isamarkupfalse%
\ assms{\isadigit{1}}\ assms\ envin\ \isanewline
\ \ \ \ \isacommand{by}\isamarkupfalse%
\ auto\isanewline
\ \ \isacommand{have}\isamarkupfalse%
\ I{\isadigit{4}}\ {\isacharcolon}{\kern0pt}\ {\isachardoublequoteopen}{\isachardot}{\kern0pt}{\isachardot}{\kern0pt}{\isachardot}{\kern0pt}\ {\isasymlongleftrightarrow}\ SymExt{\isacharparenleft}{\kern0pt}G{\isacharparenright}{\kern0pt}{\isacharcomma}{\kern0pt}\ map{\isacharparenleft}{\kern0pt}val{\isacharparenleft}{\kern0pt}G{\isacharparenright}{\kern0pt}{\isacharcomma}{\kern0pt}\ env{\isacharparenright}{\kern0pt}\ {\isasymTurnstile}\ Equal{\isacharparenleft}{\kern0pt}x{\isacharcomma}{\kern0pt}\ y{\isacharparenright}{\kern0pt}{\isachardoublequoteclose}\ \isanewline
\ \ \ \ \isacommand{apply}\isamarkupfalse%
{\isacharparenleft}{\kern0pt}rule\ iff{\isacharunderscore}{\kern0pt}flip{\isacharcomma}{\kern0pt}\ rule\ {\isasymDelta}{\isadigit{0}}{\isacharunderscore}{\kern0pt}sats{\isacharunderscore}{\kern0pt}iff{\isacharparenright}{\kern0pt}\isanewline
\ \ \ \ \isacommand{using}\isamarkupfalse%
\ H\ M{\isacharunderscore}{\kern0pt}symmetric{\isacharunderscore}{\kern0pt}system{\isacharunderscore}{\kern0pt}G{\isacharunderscore}{\kern0pt}generic{\isachardot}{\kern0pt}SymExt{\isacharunderscore}{\kern0pt}subset{\isacharunderscore}{\kern0pt}GenExt\ envmaptype\ Member{\isacharunderscore}{\kern0pt}{\isasymDelta}{\isadigit{0}}\ assms{\isadigit{1}}\ M{\isacharunderscore}{\kern0pt}symmetric{\isacharunderscore}{\kern0pt}system{\isacharunderscore}{\kern0pt}G{\isacharunderscore}{\kern0pt}generic{\isachardot}{\kern0pt}Transset{\isacharunderscore}{\kern0pt}SymExt\ length{\isacharunderscore}{\kern0pt}map\ \isanewline
\ \ \ \ \isacommand{by}\isamarkupfalse%
\ auto\isanewline
\ \ \isacommand{show}\isamarkupfalse%
\ {\isachardoublequoteopen}{\isacharparenleft}{\kern0pt}{\isasymexists}p{\isasymin}G{\isachardot}{\kern0pt}\ M{\isacharcomma}{\kern0pt}\ {\isacharbrackleft}{\kern0pt}p{\isacharcomma}{\kern0pt}\ P{\isacharcomma}{\kern0pt}\ leq{\isacharcomma}{\kern0pt}\ one{\isacharcomma}{\kern0pt}\ {\isasymlangle}{\isasymF}{\isacharcomma}{\kern0pt}\ {\isasymG}{\isacharcomma}{\kern0pt}\ P{\isacharcomma}{\kern0pt}\ P{\isacharunderscore}{\kern0pt}auto{\isasymrangle}{\isacharbrackright}{\kern0pt}\ {\isacharat}{\kern0pt}\ env\ {\isasymTurnstile}\ forcesHS{\isacharparenleft}{\kern0pt}Equal{\isacharparenleft}{\kern0pt}x{\isacharcomma}{\kern0pt}\ y{\isacharparenright}{\kern0pt}{\isacharparenright}{\kern0pt}{\isacharparenright}{\kern0pt}\ {\isasymlongleftrightarrow}\isanewline
\ \ \ \ \ \ \ \ \ \ SymExt{\isacharparenleft}{\kern0pt}G{\isacharparenright}{\kern0pt}{\isacharcomma}{\kern0pt}\ map{\isacharparenleft}{\kern0pt}val{\isacharparenleft}{\kern0pt}G{\isacharparenright}{\kern0pt}{\isacharcomma}{\kern0pt}\ env{\isacharparenright}{\kern0pt}\ {\isasymTurnstile}\ Equal{\isacharparenleft}{\kern0pt}x{\isacharcomma}{\kern0pt}\ y{\isacharparenright}{\kern0pt}{\isachardoublequoteclose}\isanewline
\ \ \ \ \isacommand{using}\isamarkupfalse%
\ I{\isadigit{1}}\ I{\isadigit{2}}\ I{\isadigit{3}}\ I{\isadigit{4}}\ \isanewline
\ \ \ \ \isacommand{by}\isamarkupfalse%
\ auto\isanewline
\isacommand{next}\isamarkupfalse%
\isanewline
\ \ \isacommand{case}\isamarkupfalse%
\ {\isacharparenleft}{\kern0pt}Nand\ {\isasymphi}\ {\isasympsi}{\isacharparenright}{\kern0pt}\isanewline
\isanewline
\ \ \isacommand{have}\isamarkupfalse%
\ envin\ {\isacharcolon}{\kern0pt}\ {\isachardoublequoteopen}env\ {\isasymin}\ list{\isacharparenleft}{\kern0pt}M{\isacharparenright}{\kern0pt}{\isachardoublequoteclose}\ \isanewline
\ \ \ \ \isacommand{apply}\isamarkupfalse%
{\isacharparenleft}{\kern0pt}rule{\isacharunderscore}{\kern0pt}tac\ A{\isacharequal}{\kern0pt}{\isachardoublequoteopen}list{\isacharparenleft}{\kern0pt}HS{\isacharparenright}{\kern0pt}{\isachardoublequoteclose}\ \isakeyword{in}\ subsetD{\isacharcomma}{\kern0pt}\ rule\ list{\isacharunderscore}{\kern0pt}mono{\isacharparenright}{\kern0pt}\isanewline
\ \ \ \ \isacommand{using}\isamarkupfalse%
\ HS{\isacharunderscore}{\kern0pt}iff\ P{\isacharunderscore}{\kern0pt}name{\isacharunderscore}{\kern0pt}in{\isacharunderscore}{\kern0pt}M\ assms\ Nand\isanewline
\ \ \ \ \isacommand{by}\isamarkupfalse%
\ auto\isanewline
\ \ \isacommand{have}\isamarkupfalse%
\ mapin\ {\isacharcolon}{\kern0pt}\ {\isachardoublequoteopen}map{\isacharparenleft}{\kern0pt}val{\isacharparenleft}{\kern0pt}G{\isacharparenright}{\kern0pt}{\isacharcomma}{\kern0pt}\ env{\isacharparenright}{\kern0pt}\ {\isasymin}\ list{\isacharparenleft}{\kern0pt}SymExt{\isacharparenleft}{\kern0pt}G{\isacharparenright}{\kern0pt}{\isacharparenright}{\kern0pt}{\isachardoublequoteclose}\ \isanewline
\ \ \ \ \isacommand{apply}\isamarkupfalse%
{\isacharparenleft}{\kern0pt}rule\ map{\isacharunderscore}{\kern0pt}type{\isacharparenright}{\kern0pt}\isanewline
\ \ \ \ \isacommand{using}\isamarkupfalse%
\ assms\ SymExt{\isacharunderscore}{\kern0pt}def\ Nand\isanewline
\ \ \ \ \isacommand{by}\isamarkupfalse%
\ auto\isanewline
\ \ \isacommand{have}\isamarkupfalse%
\ arityle\ {\isacharcolon}{\kern0pt}\ {\isachardoublequoteopen}arity{\isacharparenleft}{\kern0pt}{\isasymphi}{\isacharparenright}{\kern0pt}\ {\isasymle}\ length{\isacharparenleft}{\kern0pt}env{\isacharparenright}{\kern0pt}\ {\isasymand}\ arity{\isacharparenleft}{\kern0pt}{\isasympsi}{\isacharparenright}{\kern0pt}\ {\isasymle}\ length{\isacharparenleft}{\kern0pt}env{\isacharparenright}{\kern0pt}{\isachardoublequoteclose}\isanewline
\ \ \ \ \isacommand{apply}\isamarkupfalse%
{\isacharparenleft}{\kern0pt}rule\ conjI{\isacharparenright}{\kern0pt}\isanewline
\ \ \ \ \ \ \isacommand{apply}\isamarkupfalse%
{\isacharparenleft}{\kern0pt}rule{\isacharunderscore}{\kern0pt}tac\ j{\isacharequal}{\kern0pt}{\isachardoublequoteopen}arity{\isacharparenleft}{\kern0pt}{\isasymphi}{\isacharparenright}{\kern0pt}\ {\isasymunion}\ arity{\isacharparenleft}{\kern0pt}{\isasympsi}{\isacharparenright}{\kern0pt}{\isachardoublequoteclose}\ \isakeyword{in}\ le{\isacharunderscore}{\kern0pt}trans{\isacharparenright}{\kern0pt}\isanewline
\ \ \ \ \isacommand{using}\isamarkupfalse%
\ Nand\ max{\isacharunderscore}{\kern0pt}le{\isadigit{1}}\isanewline
\ \ \ \ \ \ \ \isacommand{apply}\isamarkupfalse%
\ auto{\isacharbrackleft}{\kern0pt}{\isadigit{2}}{\isacharbrackright}{\kern0pt}\isanewline
\ \ \ \ \ \ \isacommand{apply}\isamarkupfalse%
{\isacharparenleft}{\kern0pt}rule{\isacharunderscore}{\kern0pt}tac\ j{\isacharequal}{\kern0pt}{\isachardoublequoteopen}arity{\isacharparenleft}{\kern0pt}{\isasymphi}{\isacharparenright}{\kern0pt}\ {\isasymunion}\ arity{\isacharparenleft}{\kern0pt}{\isasympsi}{\isacharparenright}{\kern0pt}{\isachardoublequoteclose}\ \isakeyword{in}\ le{\isacharunderscore}{\kern0pt}trans{\isacharparenright}{\kern0pt}\isanewline
\ \ \ \ \isacommand{using}\isamarkupfalse%
\ Nand\ max{\isacharunderscore}{\kern0pt}le{\isadigit{2}}\isanewline
\ \ \ \ \ \isacommand{apply}\isamarkupfalse%
\ auto{\isacharbrackleft}{\kern0pt}{\isadigit{2}}{\isacharbrackright}{\kern0pt}\isanewline
\ \ \ \ \isacommand{done}\isamarkupfalse%
\isanewline
\isanewline
\ \ \isacommand{moreover}\isamarkupfalse%
\ \isanewline
\ \ \isacommand{note}\isamarkupfalse%
\ {\isacartoucheopen}M{\isacharunderscore}{\kern0pt}generic{\isacharparenleft}{\kern0pt}G{\isacharparenright}{\kern0pt}{\isacartoucheclose}\isanewline
\ \ \isacommand{ultimately}\isamarkupfalse%
\isanewline
\ \ \isacommand{show}\isamarkupfalse%
\ {\isacharquery}{\kern0pt}case\ \isanewline
\ \ \ \ \isacommand{apply}\isamarkupfalse%
{\isacharparenleft}{\kern0pt}rule{\isacharunderscore}{\kern0pt}tac\ iff{\isacharunderscore}{\kern0pt}trans{\isacharparenright}{\kern0pt}\ \isanewline
\ \ \ \ \ \isacommand{apply}\isamarkupfalse%
{\isacharparenleft}{\kern0pt}rule\ bex{\isacharunderscore}{\kern0pt}iff{\isacharcomma}{\kern0pt}\ rule\ ForcesHS{\isacharunderscore}{\kern0pt}Nand{\isacharunderscore}{\kern0pt}alt{\isacharparenright}{\kern0pt}\isanewline
\ \ \ \ \isacommand{using}\isamarkupfalse%
\ M{\isacharunderscore}{\kern0pt}genericD\ assms\ envin\ Nand\ arityle\isanewline
\ \ \ \ \ \ \ \ \ \ \isacommand{apply}\isamarkupfalse%
\ auto{\isacharbrackleft}{\kern0pt}{\isadigit{6}}{\isacharbrackright}{\kern0pt}\isanewline
\ \ \ \ \isacommand{apply}\isamarkupfalse%
{\isacharparenleft}{\kern0pt}rule\ iff{\isacharunderscore}{\kern0pt}trans{\isacharcomma}{\kern0pt}\ rule\ HS{\isacharunderscore}{\kern0pt}truth{\isacharunderscore}{\kern0pt}lemma{\isacharunderscore}{\kern0pt}Neg{\isacharparenright}{\kern0pt}\isanewline
\ \ \ \ \isacommand{using}\isamarkupfalse%
\ Nand\ assms\ \isanewline
\ \ \ \ \ \ \ \ \ \isacommand{apply}\isamarkupfalse%
\ auto{\isacharbrackleft}{\kern0pt}{\isadigit{4}}{\isacharbrackright}{\kern0pt}\isanewline
\ \ \ \ \ \isacommand{apply}\isamarkupfalse%
{\isacharparenleft}{\kern0pt}rule\ HS{\isacharunderscore}{\kern0pt}truth{\isacharunderscore}{\kern0pt}lemma{\isacharunderscore}{\kern0pt}And{\isacharparenright}{\kern0pt}\isanewline
\ \ \ \ \isacommand{using}\isamarkupfalse%
\ Nand\ arityle\ \isanewline
\ \ \ \ \ \ \ \ \ \ \ \ \isacommand{apply}\isamarkupfalse%
\ auto{\isacharbrackleft}{\kern0pt}{\isadigit{8}}{\isacharbrackright}{\kern0pt}\isanewline
\ \ \ \ \isacommand{using}\isamarkupfalse%
\ mapin\ Nand\isanewline
\ \ \ \ \isacommand{by}\isamarkupfalse%
\ auto\isanewline
\isacommand{next}\isamarkupfalse%
\isanewline
\ \ \isacommand{case}\isamarkupfalse%
\ {\isacharparenleft}{\kern0pt}Forall\ {\isasymphi}{\isacharparenright}{\kern0pt}\isanewline
\isanewline
\ \ \isacommand{have}\isamarkupfalse%
\ envin\ {\isacharcolon}{\kern0pt}\ {\isachardoublequoteopen}env\ {\isasymin}\ list{\isacharparenleft}{\kern0pt}M{\isacharparenright}{\kern0pt}{\isachardoublequoteclose}\ \isanewline
\ \ \ \ \isacommand{apply}\isamarkupfalse%
{\isacharparenleft}{\kern0pt}rule{\isacharunderscore}{\kern0pt}tac\ A{\isacharequal}{\kern0pt}{\isachardoublequoteopen}list{\isacharparenleft}{\kern0pt}HS{\isacharparenright}{\kern0pt}{\isachardoublequoteclose}\ \isakeyword{in}\ subsetD{\isacharcomma}{\kern0pt}\ rule\ list{\isacharunderscore}{\kern0pt}mono{\isacharparenright}{\kern0pt}\isanewline
\ \ \ \ \isacommand{using}\isamarkupfalse%
\ HS{\isacharunderscore}{\kern0pt}iff\ P{\isacharunderscore}{\kern0pt}name{\isacharunderscore}{\kern0pt}in{\isacharunderscore}{\kern0pt}M\ assms\ Forall\isanewline
\ \ \ \ \isacommand{by}\isamarkupfalse%
\ auto\isanewline
\ \ \isacommand{have}\isamarkupfalse%
\ mapin\ {\isacharcolon}{\kern0pt}\ {\isachardoublequoteopen}map{\isacharparenleft}{\kern0pt}val{\isacharparenleft}{\kern0pt}G{\isacharparenright}{\kern0pt}{\isacharcomma}{\kern0pt}\ env{\isacharparenright}{\kern0pt}\ {\isasymin}\ list{\isacharparenleft}{\kern0pt}SymExt{\isacharparenleft}{\kern0pt}G{\isacharparenright}{\kern0pt}{\isacharparenright}{\kern0pt}{\isachardoublequoteclose}\ \isanewline
\ \ \ \ \isacommand{apply}\isamarkupfalse%
{\isacharparenleft}{\kern0pt}rule\ map{\isacharunderscore}{\kern0pt}type{\isacharparenright}{\kern0pt}\isanewline
\ \ \ \ \isacommand{using}\isamarkupfalse%
\ assms\ SymExt{\isacharunderscore}{\kern0pt}def\ Forall\isanewline
\ \ \ \ \isacommand{by}\isamarkupfalse%
\ auto\isanewline
\isanewline
\ \ \isacommand{have}\isamarkupfalse%
\ ihE\ {\isacharcolon}{\kern0pt}\ {\isachardoublequoteopen}{\isasymAnd}env{\isachardot}{\kern0pt}\ env\ {\isasymin}\ list{\isacharparenleft}{\kern0pt}HS{\isacharparenright}{\kern0pt}\ {\isasymLongrightarrow}\ arity{\isacharparenleft}{\kern0pt}{\isasymphi}{\isacharparenright}{\kern0pt}\ {\isasymle}\ length{\isacharparenleft}{\kern0pt}env{\isacharparenright}{\kern0pt}\ {\isasymLongrightarrow}\isanewline
\ \ \ \ {\isacharparenleft}{\kern0pt}{\isasymexists}p{\isasymin}G{\isachardot}{\kern0pt}\ M{\isacharcomma}{\kern0pt}\ {\isacharbrackleft}{\kern0pt}p{\isacharcomma}{\kern0pt}\ P{\isacharcomma}{\kern0pt}\ leq{\isacharcomma}{\kern0pt}\ one{\isacharcomma}{\kern0pt}\ {\isasymlangle}{\isasymF}{\isacharcomma}{\kern0pt}\ {\isasymG}{\isacharcomma}{\kern0pt}\ P{\isacharcomma}{\kern0pt}\ P{\isacharunderscore}{\kern0pt}auto{\isasymrangle}{\isacharbrackright}{\kern0pt}\ {\isacharat}{\kern0pt}\ env\ {\isasymTurnstile}\ forcesHS{\isacharparenleft}{\kern0pt}{\isasymphi}{\isacharparenright}{\kern0pt}{\isacharparenright}{\kern0pt}\ {\isasymlongleftrightarrow}\ SymExt{\isacharparenleft}{\kern0pt}G{\isacharparenright}{\kern0pt}{\isacharcomma}{\kern0pt}\ map{\isacharparenleft}{\kern0pt}{\isasymlambda}a{\isachardot}{\kern0pt}\ val{\isacharparenleft}{\kern0pt}G{\isacharcomma}{\kern0pt}\ a{\isacharparenright}{\kern0pt}{\isacharcomma}{\kern0pt}\ env{\isacharparenright}{\kern0pt}\ {\isasymTurnstile}\ {\isasymphi}{\isachardoublequoteclose}\isanewline
\ \ \ \ \isacommand{using}\isamarkupfalse%
\ Forall\ \isanewline
\ \ \ \ \isacommand{by}\isamarkupfalse%
\ auto\isanewline
\isanewline
\ \ \isacommand{with}\isamarkupfalse%
\ {\isacartoucheopen}M{\isacharunderscore}{\kern0pt}generic{\isacharparenleft}{\kern0pt}G{\isacharparenright}{\kern0pt}{\isacartoucheclose}\isanewline
\ \ \isacommand{show}\isamarkupfalse%
\ {\isacharquery}{\kern0pt}case\isanewline
\ \ \isacommand{proof}\isamarkupfalse%
\ {\isacharparenleft}{\kern0pt}intro\ iffI{\isacharparenright}{\kern0pt}\isanewline
\ \ \ \ \isacommand{assume}\isamarkupfalse%
\ {\isachardoublequoteopen}{\isasymexists}p{\isasymin}G{\isachardot}{\kern0pt}\ {\isacharparenleft}{\kern0pt}p\ {\isasymtturnstile}HS\ Forall{\isacharparenleft}{\kern0pt}{\isasymphi}{\isacharparenright}{\kern0pt}\ env{\isacharparenright}{\kern0pt}{\isachardoublequoteclose}\isanewline
\ \ \ \ \isacommand{with}\isamarkupfalse%
\ {\isacartoucheopen}M{\isacharunderscore}{\kern0pt}generic{\isacharparenleft}{\kern0pt}G{\isacharparenright}{\kern0pt}{\isacartoucheclose}\isanewline
\ \ \ \ \isacommand{obtain}\isamarkupfalse%
\ p\ \isakeyword{where}\ pH{\isacharcolon}{\kern0pt}\ {\isachardoublequoteopen}p{\isasymin}G{\isachardoublequoteclose}\ {\isachardoublequoteopen}p{\isasymin}M{\isachardoublequoteclose}\ {\isachardoublequoteopen}p{\isasymin}P{\isachardoublequoteclose}\ {\isachardoublequoteopen}p\ {\isasymtturnstile}HS\ Forall{\isacharparenleft}{\kern0pt}{\isasymphi}{\isacharparenright}{\kern0pt}\ env{\isachardoublequoteclose}\isanewline
\ \ \ \ \ \ \isacommand{using}\isamarkupfalse%
\ transitivity{\isacharbrackleft}{\kern0pt}OF\ {\isacharunderscore}{\kern0pt}\ P{\isacharunderscore}{\kern0pt}in{\isacharunderscore}{\kern0pt}M{\isacharbrackright}{\kern0pt}\ \isacommand{by}\isamarkupfalse%
\ auto\isanewline
\ \ \ \ \isacommand{with}\isamarkupfalse%
\ {\isacartoucheopen}env{\isasymin}list{\isacharparenleft}{\kern0pt}HS{\isacharparenright}{\kern0pt}{\isacartoucheclose}\ {\isacartoucheopen}{\isasymphi}{\isasymin}formula{\isacartoucheclose}\isanewline
\ \ \ \ \isacommand{have}\isamarkupfalse%
\ {\isachardoublequoteopen}p\ {\isasymtturnstile}HS\ {\isasymphi}\ {\isacharparenleft}{\kern0pt}Cons{\isacharparenleft}{\kern0pt}x{\isacharcomma}{\kern0pt}\ env{\isacharparenright}{\kern0pt}{\isacharparenright}{\kern0pt}{\isachardoublequoteclose}\ \isakeyword{if}\ {\isachardoublequoteopen}x{\isasymin}HS{\isachardoublequoteclose}\ \isakeyword{for}\ x\isanewline
\ \ \ \ \ \ \isacommand{using}\isamarkupfalse%
\ that\ ForcesHS{\isacharunderscore}{\kern0pt}Forall\ envin\ \isacommand{by}\isamarkupfalse%
\ simp\isanewline
\ \ \ \ \isacommand{then}\isamarkupfalse%
\ \isacommand{have}\isamarkupfalse%
\ {\isachardoublequoteopen}SymExt{\isacharparenleft}{\kern0pt}G{\isacharparenright}{\kern0pt}{\isacharcomma}{\kern0pt}\ map{\isacharparenleft}{\kern0pt}{\isasymlambda}a{\isachardot}{\kern0pt}\ val{\isacharparenleft}{\kern0pt}G{\isacharcomma}{\kern0pt}\ a{\isacharparenright}{\kern0pt}{\isacharcomma}{\kern0pt}\ Cons{\isacharparenleft}{\kern0pt}x{\isacharcomma}{\kern0pt}\ env{\isacharparenright}{\kern0pt}{\isacharparenright}{\kern0pt}\ {\isasymTurnstile}\ {\isasymphi}{\isachardoublequoteclose}\ \isakeyword{if}\ {\isachardoublequoteopen}x\ {\isasymin}\ HS{\isachardoublequoteclose}\ \isakeyword{for}\ x\ \isanewline
\ \ \ \ \ \ \isacommand{using}\isamarkupfalse%
\ that\ ForcesHS{\isacharunderscore}{\kern0pt}Forall\ envin\ Forall\ pH\ \isanewline
\ \ \ \ \ \ \isacommand{apply}\isamarkupfalse%
{\isacharparenleft}{\kern0pt}rule{\isacharunderscore}{\kern0pt}tac\ iffD{\isadigit{1}}{\isacharcomma}{\kern0pt}\ rule{\isacharunderscore}{\kern0pt}tac\ ihE{\isacharparenright}{\kern0pt}\isanewline
\ \ \ \ \ \ \ \ \isacommand{apply}\isamarkupfalse%
\ auto{\isacharbrackleft}{\kern0pt}{\isadigit{1}}{\isacharbrackright}{\kern0pt}\isanewline
\ \ \ \ \ \ \ \isacommand{apply}\isamarkupfalse%
{\isacharparenleft}{\kern0pt}rule{\isacharunderscore}{\kern0pt}tac\ n{\isacharequal}{\kern0pt}{\isachardoublequoteopen}arity{\isacharparenleft}{\kern0pt}{\isasymphi}{\isacharparenright}{\kern0pt}{\isachardoublequoteclose}\ \isakeyword{in}\ natE{\isacharcomma}{\kern0pt}\ simp{\isacharcomma}{\kern0pt}\ force{\isacharcomma}{\kern0pt}\ force{\isacharparenright}{\kern0pt}\isanewline
\ \ \ \ \ \ \isacommand{apply}\isamarkupfalse%
{\isacharparenleft}{\kern0pt}rule{\isacharunderscore}{\kern0pt}tac\ x{\isacharequal}{\kern0pt}p\ \isakeyword{in}\ bexI{\isacharparenright}{\kern0pt}\isanewline
\ \ \ \ \ \ \isacommand{by}\isamarkupfalse%
\ auto\isanewline
\ \ \ \ \isacommand{then}\isamarkupfalse%
\ \isacommand{show}\isamarkupfalse%
\ {\isachardoublequoteopen}SymExt{\isacharparenleft}{\kern0pt}G{\isacharparenright}{\kern0pt}{\isacharcomma}{\kern0pt}\ map{\isacharparenleft}{\kern0pt}val{\isacharparenleft}{\kern0pt}G{\isacharparenright}{\kern0pt}{\isacharcomma}{\kern0pt}env{\isacharparenright}{\kern0pt}\ {\isasymTurnstile}\ Forall{\isacharparenleft}{\kern0pt}{\isasymphi}{\isacharparenright}{\kern0pt}{\isachardoublequoteclose}\ \isanewline
\ \ \ \ \ \ \isacommand{using}\isamarkupfalse%
\ mapin\ \isanewline
\ \ \ \ \ \ \isacommand{apply}\isamarkupfalse%
\ simp\isanewline
\ \ \ \ \ \ \isacommand{apply}\isamarkupfalse%
{\isacharparenleft}{\kern0pt}rule\ ballI{\isacharparenright}{\kern0pt}\isanewline
\ \ \ \ \ \ \isacommand{apply}\isamarkupfalse%
{\isacharparenleft}{\kern0pt}rename{\isacharunderscore}{\kern0pt}tac\ x{\isacharcomma}{\kern0pt}\ subgoal{\isacharunderscore}{\kern0pt}tac\ {\isachardoublequoteopen}{\isasymexists}x{\isacharprime}{\kern0pt}\ {\isasymin}\ HS{\isachardot}{\kern0pt}\ val{\isacharparenleft}{\kern0pt}G{\isacharcomma}{\kern0pt}\ x{\isacharprime}{\kern0pt}{\isacharparenright}{\kern0pt}\ {\isacharequal}{\kern0pt}\ x{\isachardoublequoteclose}{\isacharcomma}{\kern0pt}\ force{\isacharparenright}{\kern0pt}\isanewline
\ \ \ \ \ \ \isacommand{unfolding}\isamarkupfalse%
\ SymExt{\isacharunderscore}{\kern0pt}def\isanewline
\ \ \ \ \ \ \isacommand{by}\isamarkupfalse%
\ auto\isanewline
\ \ \isacommand{next}\isamarkupfalse%
\isanewline
\ \ \ \ \isacommand{assume}\isamarkupfalse%
\ satsforall\ {\isacharcolon}{\kern0pt}\ {\isachardoublequoteopen}SymExt{\isacharparenleft}{\kern0pt}G{\isacharparenright}{\kern0pt}{\isacharcomma}{\kern0pt}\ map{\isacharparenleft}{\kern0pt}val{\isacharparenleft}{\kern0pt}G{\isacharparenright}{\kern0pt}{\isacharcomma}{\kern0pt}env{\isacharparenright}{\kern0pt}\ {\isasymTurnstile}\ Forall{\isacharparenleft}{\kern0pt}{\isasymphi}{\isacharparenright}{\kern0pt}{\isachardoublequoteclose}\isanewline
\isanewline
\ \ \ \ \isacommand{let}\isamarkupfalse%
\ {\isacharquery}{\kern0pt}D{\isadigit{1}}{\isacharequal}{\kern0pt}{\isachardoublequoteopen}{\isacharbraceleft}{\kern0pt}d{\isasymin}P{\isachardot}{\kern0pt}\ {\isacharparenleft}{\kern0pt}d\ {\isasymtturnstile}HS\ Forall{\isacharparenleft}{\kern0pt}{\isasymphi}{\isacharparenright}{\kern0pt}\ env{\isacharparenright}{\kern0pt}{\isacharbraceright}{\kern0pt}{\isachardoublequoteclose}\isanewline
\ \ \ \ \isacommand{let}\isamarkupfalse%
\ {\isacharquery}{\kern0pt}D{\isadigit{2}}{\isacharequal}{\kern0pt}{\isachardoublequoteopen}{\isacharbraceleft}{\kern0pt}d{\isasymin}P{\isachardot}{\kern0pt}\ {\isasymexists}b{\isasymin}HS{\isachardot}{\kern0pt}\ {\isasymforall}q{\isasymin}P{\isachardot}{\kern0pt}\ q{\isasympreceq}d\ {\isasymlongrightarrow}\ {\isasymnot}{\isacharparenleft}{\kern0pt}q\ {\isasymtturnstile}HS\ {\isasymphi}\ {\isacharparenleft}{\kern0pt}{\isacharbrackleft}{\kern0pt}b{\isacharbrackright}{\kern0pt}{\isacharat}{\kern0pt}env{\isacharparenright}{\kern0pt}{\isacharparenright}{\kern0pt}{\isacharbraceright}{\kern0pt}{\isachardoublequoteclose}\isanewline
\ \ \ \ \isacommand{define}\isamarkupfalse%
\ D\ \isakeyword{where}\ {\isachardoublequoteopen}D\ {\isasymequiv}\ {\isacharquery}{\kern0pt}D{\isadigit{1}}\ {\isasymunion}\ {\isacharquery}{\kern0pt}D{\isadigit{2}}{\isachardoublequoteclose}\isanewline
\ \ \ \ \isacommand{have}\isamarkupfalse%
\ ar{\isasymphi}{\isacharcolon}{\kern0pt}{\isachardoublequoteopen}arity{\isacharparenleft}{\kern0pt}{\isasymphi}{\isacharparenright}{\kern0pt}{\isasymle}succ{\isacharparenleft}{\kern0pt}length{\isacharparenleft}{\kern0pt}env{\isacharparenright}{\kern0pt}{\isacharparenright}{\kern0pt}{\isachardoublequoteclose}\ \isanewline
\ \ \ \ \ \ \isacommand{using}\isamarkupfalse%
\ assms\ {\isacartoucheopen}arity{\isacharparenleft}{\kern0pt}Forall{\isacharparenleft}{\kern0pt}{\isasymphi}{\isacharparenright}{\kern0pt}{\isacharparenright}{\kern0pt}\ {\isasymle}\ length{\isacharparenleft}{\kern0pt}env{\isacharparenright}{\kern0pt}{\isacartoucheclose}\ {\isacartoucheopen}{\isasymphi}{\isasymin}formula{\isacartoucheclose}\ {\isacartoucheopen}env{\isasymin}list{\isacharparenleft}{\kern0pt}M{\isacharparenright}{\kern0pt}{\isacartoucheclose}\ pred{\isacharunderscore}{\kern0pt}le{\isadigit{2}}\ \isanewline
\ \ \ \ \ \ \isacommand{by}\isamarkupfalse%
\ simp\isanewline
\ \ \ \ \isacommand{then}\isamarkupfalse%
\isanewline
\ \ \ \ \isacommand{have}\isamarkupfalse%
\ {\isachardoublequoteopen}arity{\isacharparenleft}{\kern0pt}Forall{\isacharparenleft}{\kern0pt}{\isasymphi}{\isacharparenright}{\kern0pt}{\isacharparenright}{\kern0pt}\ {\isasymle}\ length{\isacharparenleft}{\kern0pt}env{\isacharparenright}{\kern0pt}{\isachardoublequoteclose}\ \isanewline
\ \ \ \ \ \ \isacommand{using}\isamarkupfalse%
\ pred{\isacharunderscore}{\kern0pt}le\ {\isacartoucheopen}{\isasymphi}{\isasymin}formula{\isacartoucheclose}\ {\isacartoucheopen}env{\isasymin}list{\isacharparenleft}{\kern0pt}M{\isacharparenright}{\kern0pt}{\isacartoucheclose}\ \isacommand{by}\isamarkupfalse%
\ simp\isanewline
\ \ \ \ \isacommand{then}\isamarkupfalse%
\isanewline
\ \ \ \ \isacommand{have}\isamarkupfalse%
\ {\isachardoublequoteopen}{\isacharquery}{\kern0pt}D{\isadigit{1}}{\isasymin}M{\isachardoublequoteclose}\ \isacommand{using}\isamarkupfalse%
\ Collect{\isacharunderscore}{\kern0pt}ForcesHS\ ar{\isasymphi}\ {\isacartoucheopen}{\isasymphi}{\isasymin}formula{\isacartoucheclose}\ {\isacartoucheopen}env{\isasymin}list{\isacharparenleft}{\kern0pt}M{\isacharparenright}{\kern0pt}{\isacartoucheclose}\ \isacommand{by}\isamarkupfalse%
\ simp\isanewline
\ \ \ \ \isacommand{moreover}\isamarkupfalse%
\isanewline
\ \ \ \ \isacommand{have}\isamarkupfalse%
\ {\isachardoublequoteopen}{\isacharquery}{\kern0pt}D{\isadigit{2}}{\isasymin}M{\isachardoublequoteclose}\ \isacommand{using}\isamarkupfalse%
\ {\isacartoucheopen}env{\isasymin}list{\isacharparenleft}{\kern0pt}M{\isacharparenright}{\kern0pt}{\isacartoucheclose}\ {\isacartoucheopen}{\isasymphi}{\isasymin}formula{\isacartoucheclose}\ \ HS{\isacharunderscore}{\kern0pt}truth{\isacharunderscore}{\kern0pt}lemma{\isacharprime}{\kern0pt}\ separation{\isacharunderscore}{\kern0pt}closed\ ar{\isasymphi}\isanewline
\ \ \ \ \ \ \ \ \ \ \ \ \ \ \ \ \ \ \ \ \ \ \ \ P{\isacharunderscore}{\kern0pt}in{\isacharunderscore}{\kern0pt}M\isanewline
\ \ \ \ \ \ \isacommand{by}\isamarkupfalse%
\ simp\isanewline
\ \ \ \ \isacommand{ultimately}\isamarkupfalse%
\isanewline
\ \ \ \ \isacommand{have}\isamarkupfalse%
\ {\isachardoublequoteopen}D{\isasymin}M{\isachardoublequoteclose}\ \isacommand{unfolding}\isamarkupfalse%
\ D{\isacharunderscore}{\kern0pt}def\ \isacommand{using}\isamarkupfalse%
\ Un{\isacharunderscore}{\kern0pt}closed\ \isacommand{by}\isamarkupfalse%
\ simp\isanewline
\ \ \ \ \isacommand{moreover}\isamarkupfalse%
\isanewline
\ \ \ \ \isacommand{have}\isamarkupfalse%
\ {\isachardoublequoteopen}D\ {\isasymsubseteq}\ P{\isachardoublequoteclose}\ \isacommand{unfolding}\isamarkupfalse%
\ D{\isacharunderscore}{\kern0pt}def\ \isacommand{by}\isamarkupfalse%
\ auto\isanewline
\ \ \ \ \isacommand{moreover}\isamarkupfalse%
\isanewline
\ \ \ \ \isacommand{have}\isamarkupfalse%
\ {\isachardoublequoteopen}dense{\isacharparenleft}{\kern0pt}D{\isacharparenright}{\kern0pt}{\isachardoublequoteclose}\ \isanewline
\ \ \ \ \isacommand{proof}\isamarkupfalse%
\isanewline
\ \ \ \ \ \ \isacommand{fix}\isamarkupfalse%
\ p\isanewline
\ \ \ \ \ \ \isacommand{assume}\isamarkupfalse%
\ {\isachardoublequoteopen}p{\isasymin}P{\isachardoublequoteclose}\isanewline
\ \ \ \ \ \ \isacommand{show}\isamarkupfalse%
\ {\isachardoublequoteopen}{\isasymexists}d{\isasymin}D{\isachardot}{\kern0pt}\ d{\isasympreceq}\ p{\isachardoublequoteclose}\isanewline
\ \ \ \ \ \ \isacommand{proof}\isamarkupfalse%
\ {\isacharparenleft}{\kern0pt}cases\ {\isachardoublequoteopen}p\ {\isasymtturnstile}HS\ Forall{\isacharparenleft}{\kern0pt}{\isasymphi}{\isacharparenright}{\kern0pt}\ env{\isachardoublequoteclose}{\isacharparenright}{\kern0pt}\isanewline
\ \ \ \ \ \ \ \ \isacommand{case}\isamarkupfalse%
\ True\isanewline
\ \ \ \ \ \ \ \ \isacommand{with}\isamarkupfalse%
\ {\isacartoucheopen}p{\isasymin}P{\isacartoucheclose}\ \isanewline
\ \ \ \ \ \ \ \ \isacommand{show}\isamarkupfalse%
\ {\isacharquery}{\kern0pt}thesis\ \isacommand{unfolding}\isamarkupfalse%
\ D{\isacharunderscore}{\kern0pt}def\ \isacommand{using}\isamarkupfalse%
\ leq{\isacharunderscore}{\kern0pt}reflI\ \isacommand{by}\isamarkupfalse%
\ blast\isanewline
\ \ \ \ \ \ \isacommand{next}\isamarkupfalse%
\isanewline
\ \ \ \ \ \ \ \ \isacommand{case}\isamarkupfalse%
\ False\isanewline
\isanewline
\ \ \ \ \ \ \ \ \isacommand{have}\isamarkupfalse%
\ {\isachardoublequoteopen}{\isacharparenleft}{\kern0pt}M{\isacharcomma}{\kern0pt}\ {\isacharbrackleft}{\kern0pt}p{\isacharcomma}{\kern0pt}\ P{\isacharcomma}{\kern0pt}\ leq{\isacharcomma}{\kern0pt}\ one{\isacharcomma}{\kern0pt}\ {\isasymlangle}{\isasymF}{\isacharcomma}{\kern0pt}\ {\isasymG}{\isacharcomma}{\kern0pt}\ P{\isacharcomma}{\kern0pt}\ P{\isacharunderscore}{\kern0pt}auto{\isasymrangle}{\isacharbrackright}{\kern0pt}\ {\isacharat}{\kern0pt}\ env\ {\isasymTurnstile}\ forcesHS{\isacharparenleft}{\kern0pt}Forall{\isacharparenleft}{\kern0pt}{\isasymphi}{\isacharparenright}{\kern0pt}{\isacharparenright}{\kern0pt}{\isacharparenright}{\kern0pt}\ {\isasymlongleftrightarrow}\ \isanewline
\ \ \ \ \ \ \ \ \ \ \ \ \ \ {\isacharparenleft}{\kern0pt}{\isasymforall}x{\isasymin}HS{\isachardot}{\kern0pt}\ M{\isacharcomma}{\kern0pt}\ {\isacharbrackleft}{\kern0pt}p{\isacharcomma}{\kern0pt}\ P{\isacharcomma}{\kern0pt}\ leq{\isacharcomma}{\kern0pt}\ one{\isacharcomma}{\kern0pt}\ {\isasymlangle}{\isasymF}{\isacharcomma}{\kern0pt}\ {\isasymG}{\isacharcomma}{\kern0pt}\ P{\isacharcomma}{\kern0pt}\ P{\isacharunderscore}{\kern0pt}auto{\isasymrangle}{\isacharbrackright}{\kern0pt}\ {\isacharat}{\kern0pt}\ {\isacharbrackleft}{\kern0pt}x{\isacharbrackright}{\kern0pt}\ {\isacharat}{\kern0pt}\ env\ {\isasymTurnstile}\ forcesHS{\isacharparenleft}{\kern0pt}{\isasymphi}{\isacharparenright}{\kern0pt}{\isacharparenright}{\kern0pt}{\isachardoublequoteclose}\ \isanewline
\ \ \ \ \ \ \ \ \ \ \isacommand{apply}\isamarkupfalse%
{\isacharparenleft}{\kern0pt}rule\ ForcesHS{\isacharunderscore}{\kern0pt}Forall{\isacharparenright}{\kern0pt}\isanewline
\ \ \ \ \ \ \ \ \ \ \isacommand{using}\isamarkupfalse%
\ {\isacartoucheopen}p\ {\isasymin}\ P{\isacartoucheclose}\ Forall\ envin\isanewline
\ \ \ \ \ \ \ \ \ \ \isacommand{by}\isamarkupfalse%
\ auto\isanewline
\ \ \ \ \ \ \ \ \isacommand{then}\isamarkupfalse%
\ \isacommand{have}\isamarkupfalse%
\ {\isachardoublequoteopen}{\isasymnot}\ {\isacharparenleft}{\kern0pt}{\isasymforall}x{\isasymin}HS{\isachardot}{\kern0pt}\ M{\isacharcomma}{\kern0pt}\ {\isacharbrackleft}{\kern0pt}p{\isacharcomma}{\kern0pt}\ P{\isacharcomma}{\kern0pt}\ leq{\isacharcomma}{\kern0pt}\ one{\isacharcomma}{\kern0pt}\ {\isasymlangle}{\isasymF}{\isacharcomma}{\kern0pt}\ {\isasymG}{\isacharcomma}{\kern0pt}\ P{\isacharcomma}{\kern0pt}\ P{\isacharunderscore}{\kern0pt}auto{\isasymrangle}{\isacharbrackright}{\kern0pt}\ {\isacharat}{\kern0pt}\ {\isacharbrackleft}{\kern0pt}x{\isacharbrackright}{\kern0pt}\ {\isacharat}{\kern0pt}\ env\ {\isasymTurnstile}\ forcesHS{\isacharparenleft}{\kern0pt}{\isasymphi}{\isacharparenright}{\kern0pt}{\isacharparenright}{\kern0pt}{\isachardoublequoteclose}\ \isacommand{using}\isamarkupfalse%
\ False\ \isacommand{by}\isamarkupfalse%
\ auto\isanewline
\ \ \ \ \ \ \ \ \isacommand{then}\isamarkupfalse%
\ \isacommand{obtain}\isamarkupfalse%
\ b\ \isakeyword{where}\ {\isachardoublequoteopen}b{\isasymin}HS{\isachardoublequoteclose}\ {\isachardoublequoteopen}{\isasymnot}{\isacharparenleft}{\kern0pt}p\ {\isasymtturnstile}HS\ {\isasymphi}\ {\isacharparenleft}{\kern0pt}{\isacharbrackleft}{\kern0pt}b{\isacharbrackright}{\kern0pt}{\isacharat}{\kern0pt}env{\isacharparenright}{\kern0pt}{\isacharparenright}{\kern0pt}{\isachardoublequoteclose}\ \isacommand{by}\isamarkupfalse%
\ auto\isanewline
\ \ \ \ \ \ \ \ \isacommand{moreover}\isamarkupfalse%
\ \isacommand{from}\isamarkupfalse%
\ this\ {\isacartoucheopen}p{\isasymin}P{\isacartoucheclose}\ Forall\isanewline
\ \ \ \ \ \ \ \ \isacommand{have}\isamarkupfalse%
\ {\isachardoublequoteopen}{\isasymnot}dense{\isacharunderscore}{\kern0pt}below{\isacharparenleft}{\kern0pt}{\isacharbraceleft}{\kern0pt}q{\isasymin}P{\isachardot}{\kern0pt}\ q\ {\isasymtturnstile}HS\ {\isasymphi}\ {\isacharparenleft}{\kern0pt}{\isacharbrackleft}{\kern0pt}b{\isacharbrackright}{\kern0pt}{\isacharat}{\kern0pt}env{\isacharparenright}{\kern0pt}{\isacharbraceright}{\kern0pt}{\isacharcomma}{\kern0pt}p{\isacharparenright}{\kern0pt}{\isachardoublequoteclose}\isanewline
\ \ \ \ \ \ \ \ \ \ \isacommand{apply}\isamarkupfalse%
{\isacharparenleft}{\kern0pt}rule{\isacharunderscore}{\kern0pt}tac\ iffD{\isadigit{1}}{\isacharparenright}{\kern0pt}\isanewline
\ \ \ \ \ \ \ \ \ \ \ \isacommand{apply}\isamarkupfalse%
{\isacharparenleft}{\kern0pt}rule\ notnot{\isacharunderscore}{\kern0pt}iff{\isacharcomma}{\kern0pt}\ rule\ HS{\isacharunderscore}{\kern0pt}density{\isacharunderscore}{\kern0pt}lemma{\isacharcomma}{\kern0pt}\ simp{\isacharcomma}{\kern0pt}\ simp{\isacharparenright}{\kern0pt}\isanewline
\ \ \ \ \ \ \ \ \ \ \isacommand{using}\isamarkupfalse%
\ envin\ HS{\isacharunderscore}{\kern0pt}iff\ P{\isacharunderscore}{\kern0pt}name{\isacharunderscore}{\kern0pt}in{\isacharunderscore}{\kern0pt}M\isanewline
\ \ \ \ \ \ \ \ \ \ \ \ \isacommand{apply}\isamarkupfalse%
\ force\ \isanewline
\ \ \ \ \ \ \ \ \ \ \ \isacommand{apply}\isamarkupfalse%
\ simp\isanewline
\ \ \ \ \ \ \ \ \ \ \ \isacommand{apply}\isamarkupfalse%
{\isacharparenleft}{\kern0pt}rule{\isacharunderscore}{\kern0pt}tac\ n{\isacharequal}{\kern0pt}{\isachardoublequoteopen}arity{\isacharparenleft}{\kern0pt}{\isasymphi}{\isacharparenright}{\kern0pt}{\isachardoublequoteclose}\ \isakeyword{in}\ natE{\isacharcomma}{\kern0pt}\ simp{\isacharcomma}{\kern0pt}\ simp{\isacharcomma}{\kern0pt}\ force{\isacharparenright}{\kern0pt}\isanewline
\ \ \ \ \ \ \ \ \ \ \isacommand{using}\isamarkupfalse%
\ HS{\isacharunderscore}{\kern0pt}density{\isacharunderscore}{\kern0pt}lemma\ pred{\isacharunderscore}{\kern0pt}le{\isadigit{2}}\ \isanewline
\ \ \ \ \ \ \ \ \ \ \isacommand{by}\isamarkupfalse%
\ auto\isanewline
\ \ \ \ \ \ \ \ \isacommand{moreover}\isamarkupfalse%
\ \isacommand{from}\isamarkupfalse%
\ this\isanewline
\ \ \ \ \ \ \ \ \isacommand{obtain}\isamarkupfalse%
\ d\ \isakeyword{where}\ {\isachardoublequoteopen}d{\isasympreceq}p{\isachardoublequoteclose}\ {\isachardoublequoteopen}{\isasymforall}q{\isasymin}P{\isachardot}{\kern0pt}\ q{\isasympreceq}d\ {\isasymlongrightarrow}\ {\isasymnot}{\isacharparenleft}{\kern0pt}q\ {\isasymtturnstile}HS\ {\isasymphi}\ {\isacharparenleft}{\kern0pt}{\isacharbrackleft}{\kern0pt}b{\isacharbrackright}{\kern0pt}\ {\isacharat}{\kern0pt}\ env{\isacharparenright}{\kern0pt}{\isacharparenright}{\kern0pt}{\isachardoublequoteclose}\isanewline
\ \ \ \ \ \ \ \ \ \ {\isachardoublequoteopen}d{\isasymin}P{\isachardoublequoteclose}\ \isacommand{by}\isamarkupfalse%
\ blast\isanewline
\ \ \ \ \ \ \ \ \isacommand{ultimately}\isamarkupfalse%
\isanewline
\ \ \ \ \ \ \ \ \isacommand{show}\isamarkupfalse%
\ {\isacharquery}{\kern0pt}thesis\ \isanewline
\ \ \ \ \ \ \ \ \ \ \isacommand{unfolding}\isamarkupfalse%
\ D{\isacharunderscore}{\kern0pt}def\isanewline
\ \ \ \ \ \ \ \ \ \ \isacommand{apply}\isamarkupfalse%
{\isacharparenleft}{\kern0pt}rule{\isacharunderscore}{\kern0pt}tac\ x{\isacharequal}{\kern0pt}d\ \isakeyword{in}\ bexI{\isacharcomma}{\kern0pt}\ simp{\isacharparenright}{\kern0pt}\isanewline
\ \ \ \ \ \ \ \ \ \ \isacommand{apply}\isamarkupfalse%
\ simp\isanewline
\ \ \ \ \ \ \ \ \ \ \isacommand{apply}\isamarkupfalse%
{\isacharparenleft}{\kern0pt}rule\ disjI{\isadigit{2}}{\isacharcomma}{\kern0pt}\ rule{\isacharunderscore}{\kern0pt}tac\ x{\isacharequal}{\kern0pt}b\ \isakeyword{in}\ bexI{\isacharcomma}{\kern0pt}\ force{\isacharparenright}{\kern0pt}\isanewline
\ \ \ \ \ \ \ \ \ \ \isacommand{using}\isamarkupfalse%
\ HS{\isacharunderscore}{\kern0pt}iff\ P{\isacharunderscore}{\kern0pt}name{\isacharunderscore}{\kern0pt}in{\isacharunderscore}{\kern0pt}M\isanewline
\ \ \ \ \ \ \ \ \ \ \isacommand{by}\isamarkupfalse%
\ auto\isanewline
\ \ \ \ \ \ \isacommand{qed}\isamarkupfalse%
\isanewline
\ \ \ \ \isacommand{qed}\isamarkupfalse%
\isanewline
\ \ \ \ \isacommand{moreover}\isamarkupfalse%
\isanewline
\ \ \ \ \isacommand{note}\isamarkupfalse%
\ {\isacartoucheopen}M{\isacharunderscore}{\kern0pt}generic{\isacharparenleft}{\kern0pt}G{\isacharparenright}{\kern0pt}{\isacartoucheclose}\isanewline
\ \ \ \ \isacommand{ultimately}\isamarkupfalse%
\isanewline
\ \ \ \ \isacommand{obtain}\isamarkupfalse%
\ d\ \isakeyword{where}\ {\isachardoublequoteopen}d\ {\isasymin}\ D{\isachardoublequoteclose}\ \ {\isachardoublequoteopen}d\ {\isasymin}\ G{\isachardoublequoteclose}\ \isacommand{by}\isamarkupfalse%
\ blast\isanewline
\ \ \ \ \isacommand{then}\isamarkupfalse%
\isanewline
\ \ \ \ \isacommand{consider}\isamarkupfalse%
\ {\isacharparenleft}{\kern0pt}{\isadigit{1}}{\isacharparenright}{\kern0pt}\ {\isachardoublequoteopen}d{\isasymin}{\isacharquery}{\kern0pt}D{\isadigit{1}}{\isachardoublequoteclose}\ {\isacharbar}{\kern0pt}\ {\isacharparenleft}{\kern0pt}{\isadigit{2}}{\isacharparenright}{\kern0pt}\ {\isachardoublequoteopen}d{\isasymin}{\isacharquery}{\kern0pt}D{\isadigit{2}}{\isachardoublequoteclose}\ \isacommand{unfolding}\isamarkupfalse%
\ D{\isacharunderscore}{\kern0pt}def\ \isacommand{by}\isamarkupfalse%
\ blast\isanewline
\ \ \ \ \isacommand{then}\isamarkupfalse%
\isanewline
\ \ \ \ \isacommand{show}\isamarkupfalse%
\ {\isachardoublequoteopen}{\isasymexists}p{\isasymin}G{\isachardot}{\kern0pt}\ {\isacharparenleft}{\kern0pt}p\ {\isasymtturnstile}HS\ Forall{\isacharparenleft}{\kern0pt}{\isasymphi}{\isacharparenright}{\kern0pt}\ env{\isacharparenright}{\kern0pt}{\isachardoublequoteclose}\isanewline
\ \ \ \ \isacommand{proof}\isamarkupfalse%
\ {\isacharparenleft}{\kern0pt}cases{\isacharparenright}{\kern0pt}\isanewline
\ \ \ \ \ \ \isacommand{case}\isamarkupfalse%
\ {\isadigit{1}}\isanewline
\ \ \ \ \ \ \isacommand{with}\isamarkupfalse%
\ {\isacartoucheopen}d{\isasymin}G{\isacartoucheclose}\isanewline
\ \ \ \ \ \ \isacommand{show}\isamarkupfalse%
\ {\isacharquery}{\kern0pt}thesis\ \isacommand{by}\isamarkupfalse%
\ blast\isanewline
\ \ \ \ \isacommand{next}\isamarkupfalse%
\isanewline
\ \ \ \ \ \ \isacommand{case}\isamarkupfalse%
\ {\isadigit{2}}\isanewline
\ \ \ \ \ \ \isacommand{then}\isamarkupfalse%
\isanewline
\ \ \ \ \ \ \isacommand{obtain}\isamarkupfalse%
\ b\ \isakeyword{where}\ {\isachardoublequoteopen}b{\isasymin}HS{\isachardoublequoteclose}\ {\isachardoublequoteopen}{\isasymforall}q{\isasymin}P{\isachardot}{\kern0pt}\ q{\isasympreceq}d\ {\isasymlongrightarrow}{\isasymnot}{\isacharparenleft}{\kern0pt}q\ {\isasymtturnstile}HS\ {\isasymphi}\ {\isacharparenleft}{\kern0pt}{\isacharbrackleft}{\kern0pt}b{\isacharbrackright}{\kern0pt}\ {\isacharat}{\kern0pt}\ env{\isacharparenright}{\kern0pt}{\isacharparenright}{\kern0pt}{\isachardoublequoteclose}\isanewline
\ \ \ \ \ \ \ \ \isacommand{by}\isamarkupfalse%
\ blast\isanewline
\ \ \ \ \ \ \isacommand{then}\isamarkupfalse%
\ \isanewline
\isanewline
\ \ \ \ \ \ \isacommand{have}\isamarkupfalse%
\ {\isachardoublequoteopen}{\isasymAnd}x{\isachardot}{\kern0pt}\ x\ {\isasymin}\ SymExt{\isacharparenleft}{\kern0pt}G{\isacharparenright}{\kern0pt}\ {\isasymLongrightarrow}\ sats{\isacharparenleft}{\kern0pt}SymExt{\isacharparenleft}{\kern0pt}G{\isacharparenright}{\kern0pt}{\isacharcomma}{\kern0pt}\ {\isasymphi}{\isacharcomma}{\kern0pt}\ Cons{\isacharparenleft}{\kern0pt}x{\isacharcomma}{\kern0pt}\ map{\isacharparenleft}{\kern0pt}val{\isacharparenleft}{\kern0pt}G{\isacharparenright}{\kern0pt}{\isacharcomma}{\kern0pt}\ env{\isacharparenright}{\kern0pt}{\isacharparenright}{\kern0pt}{\isacharparenright}{\kern0pt}{\isachardoublequoteclose}\ \isanewline
\ \ \ \ \ \ \ \ \isacommand{using}\isamarkupfalse%
\ satsforall\ mapin\ \isacommand{by}\isamarkupfalse%
\ simp\ \isanewline
\ \ \ \ \ \ \isacommand{then}\isamarkupfalse%
\ \isacommand{have}\isamarkupfalse%
\ {\isachardoublequoteopen}sats{\isacharparenleft}{\kern0pt}SymExt{\isacharparenleft}{\kern0pt}G{\isacharparenright}{\kern0pt}{\isacharcomma}{\kern0pt}\ {\isasymphi}{\isacharcomma}{\kern0pt}\ map{\isacharparenleft}{\kern0pt}val{\isacharparenleft}{\kern0pt}G{\isacharparenright}{\kern0pt}{\isacharcomma}{\kern0pt}\ Cons{\isacharparenleft}{\kern0pt}b{\isacharcomma}{\kern0pt}\ env{\isacharparenright}{\kern0pt}{\isacharparenright}{\kern0pt}{\isacharparenright}{\kern0pt}{\isachardoublequoteclose}\ \isanewline
\ \ \ \ \ \ \ \ \isacommand{apply}\isamarkupfalse%
{\isacharparenleft}{\kern0pt}subgoal{\isacharunderscore}{\kern0pt}tac\ {\isachardoublequoteopen}val{\isacharparenleft}{\kern0pt}G{\isacharcomma}{\kern0pt}\ b{\isacharparenright}{\kern0pt}\ {\isasymin}\ SymExt{\isacharparenleft}{\kern0pt}G{\isacharparenright}{\kern0pt}{\isachardoublequoteclose}{\isacharparenright}{\kern0pt}\ \isanewline
\ \ \ \ \ \ \ \ \ \isacommand{apply}\isamarkupfalse%
\ force\ \isanewline
\ \ \ \ \ \ \ \ \isacommand{unfolding}\isamarkupfalse%
\ SymExt{\isacharunderscore}{\kern0pt}def\ \isanewline
\ \ \ \ \ \ \ \ \isacommand{using}\isamarkupfalse%
\ {\isacartoucheopen}b\ {\isasymin}\ HS{\isacartoucheclose}\ \isanewline
\ \ \ \ \ \ \ \ \isacommand{by}\isamarkupfalse%
\ auto\isanewline
\ \ \ \ \ \ \isacommand{then}\isamarkupfalse%
\ \isacommand{have}\isamarkupfalse%
\ {\isachardoublequoteopen}{\isasymexists}p\ {\isasymin}\ G{\isachardot}{\kern0pt}\ p\ {\isasymtturnstile}HS\ {\isasymphi}\ {\isacharparenleft}{\kern0pt}{\isacharbrackleft}{\kern0pt}b{\isacharbrackright}{\kern0pt}\ {\isacharat}{\kern0pt}\ env{\isacharparenright}{\kern0pt}{\isachardoublequoteclose}\ \ \isanewline
\ \ \ \ \ \ \ \ \isacommand{apply}\isamarkupfalse%
{\isacharparenleft}{\kern0pt}rule{\isacharunderscore}{\kern0pt}tac\ iffD{\isadigit{2}}{\isacharparenright}{\kern0pt}\isanewline
\ \ \ \ \ \ \ \ \ \isacommand{apply}\isamarkupfalse%
{\isacharparenleft}{\kern0pt}rule\ ihE{\isacharparenright}{\kern0pt}\isanewline
\ \ \ \ \ \ \ \ \isacommand{using}\isamarkupfalse%
\ {\isacartoucheopen}b\ {\isasymin}\ HS{\isacartoucheclose}\ Forall\isanewline
\ \ \ \ \ \ \ \ \ \ \isacommand{apply}\isamarkupfalse%
\ force\ \isanewline
\ \ \ \ \ \ \ \ \isacommand{apply}\isamarkupfalse%
{\isacharparenleft}{\kern0pt}rule{\isacharunderscore}{\kern0pt}tac\ n{\isacharequal}{\kern0pt}{\isachardoublequoteopen}arity{\isacharparenleft}{\kern0pt}{\isasymphi}{\isacharparenright}{\kern0pt}{\isachardoublequoteclose}\ \isakeyword{in}\ natE{\isacharparenright}{\kern0pt}\isanewline
\ \ \ \ \ \ \ \ \isacommand{using}\isamarkupfalse%
\ Forall\ \isanewline
\ \ \ \ \ \ \ \ \isacommand{by}\isamarkupfalse%
\ auto\isanewline
\ \ \ \ \ \ \isacommand{then}\isamarkupfalse%
\ \isacommand{obtain}\isamarkupfalse%
\ p\ \isakeyword{where}\ {\isachardoublequoteopen}p{\isasymin}G{\isachardoublequoteclose}\ {\isachardoublequoteopen}p{\isasymin}P{\isachardoublequoteclose}\ {\isachardoublequoteopen}p\ {\isasymtturnstile}HS\ {\isasymphi}\ {\isacharparenleft}{\kern0pt}{\isacharbrackleft}{\kern0pt}b{\isacharbrackright}{\kern0pt}\ {\isacharat}{\kern0pt}\ env{\isacharparenright}{\kern0pt}{\isachardoublequoteclose}\ \isanewline
\ \ \ \ \ \ \ \ \isacommand{using}\isamarkupfalse%
\ M{\isacharunderscore}{\kern0pt}genericD\ assms\ \isanewline
\ \ \ \ \ \ \ \ \isacommand{by}\isamarkupfalse%
\ auto\isanewline
\ \ \ \ \ \ \isacommand{moreover}\isamarkupfalse%
\isanewline
\ \ \ \ \ \ \isacommand{note}\isamarkupfalse%
\ {\isacartoucheopen}d{\isasymin}G{\isacartoucheclose}\ {\isacartoucheopen}M{\isacharunderscore}{\kern0pt}generic{\isacharparenleft}{\kern0pt}G{\isacharparenright}{\kern0pt}{\isacartoucheclose}\isanewline
\ \ \ \ \ \ \isacommand{ultimately}\isamarkupfalse%
\isanewline
\ \ \ \ \ \ \isacommand{obtain}\isamarkupfalse%
\ q\ \isakeyword{where}\ {\isachardoublequoteopen}q{\isasymin}G{\isachardoublequoteclose}\ {\isachardoublequoteopen}q{\isasymin}P{\isachardoublequoteclose}\ {\isachardoublequoteopen}q{\isasympreceq}d{\isachardoublequoteclose}\ {\isachardoublequoteopen}q{\isasympreceq}p{\isachardoublequoteclose}\ \isacommand{by}\isamarkupfalse%
\ blast\isanewline
\ \ \ \ \ \ \isacommand{moreover}\isamarkupfalse%
\ \isacommand{from}\isamarkupfalse%
\ this\ \isakeyword{and}\ \ {\isacartoucheopen}p\ {\isasymtturnstile}HS\ {\isasymphi}\ {\isacharparenleft}{\kern0pt}{\isacharbrackleft}{\kern0pt}b{\isacharbrackright}{\kern0pt}\ {\isacharat}{\kern0pt}\ env{\isacharparenright}{\kern0pt}{\isacartoucheclose}\ \isanewline
\ \ \ \ \ \ \ \ Forall\ \ {\isacartoucheopen}b{\isasymin}HS{\isacartoucheclose}\ {\isacartoucheopen}p{\isasymin}P{\isacartoucheclose}\isanewline
\ \ \ \ \ \ \isacommand{have}\isamarkupfalse%
\ {\isachardoublequoteopen}q\ {\isasymtturnstile}HS\ {\isasymphi}\ {\isacharparenleft}{\kern0pt}{\isacharbrackleft}{\kern0pt}b{\isacharbrackright}{\kern0pt}\ {\isacharat}{\kern0pt}\ env{\isacharparenright}{\kern0pt}{\isachardoublequoteclose}\isanewline
\ \ \ \ \ \ \ \ \isacommand{apply}\isamarkupfalse%
{\isacharparenleft}{\kern0pt}rule{\isacharunderscore}{\kern0pt}tac\ p{\isacharequal}{\kern0pt}p\ \isakeyword{in}\ HS{\isacharunderscore}{\kern0pt}strengthening{\isacharunderscore}{\kern0pt}lemma{\isacharparenright}{\kern0pt}\isanewline
\ \ \ \ \ \ \ \ \isacommand{using}\isamarkupfalse%
\ {\isacartoucheopen}b{\isasymin}HS{\isacartoucheclose}\ HS{\isacharunderscore}{\kern0pt}iff\ P{\isacharunderscore}{\kern0pt}name{\isacharunderscore}{\kern0pt}in{\isacharunderscore}{\kern0pt}M\ envin\isanewline
\ \ \ \ \ \ \ \ \ \ \ \ \ \ \isacommand{apply}\isamarkupfalse%
\ auto{\isacharbrackleft}{\kern0pt}{\isadigit{6}}{\isacharbrackright}{\kern0pt}\isanewline
\ \ \ \ \ \ \ \ \ \isacommand{apply}\isamarkupfalse%
{\isacharparenleft}{\kern0pt}rule{\isacharunderscore}{\kern0pt}tac\ n{\isacharequal}{\kern0pt}{\isachardoublequoteopen}arity{\isacharparenleft}{\kern0pt}{\isasymphi}{\isacharparenright}{\kern0pt}{\isachardoublequoteclose}\ \isakeyword{in}\ natE{\isacharcomma}{\kern0pt}\ simp{\isacharcomma}{\kern0pt}\ simp{\isacharcomma}{\kern0pt}\ force{\isacharcomma}{\kern0pt}\ force{\isacharparenright}{\kern0pt}\isanewline
\ \ \ \ \ \ \ \ \isacommand{done}\isamarkupfalse%
\isanewline
\ \ \ \ \ \ \isacommand{moreover}\isamarkupfalse%
\ \isanewline
\ \ \ \ \ \ \isacommand{note}\isamarkupfalse%
\ {\isacartoucheopen}{\isasymforall}q{\isasymin}P{\isachardot}{\kern0pt}\ q{\isasympreceq}d\ {\isasymlongrightarrow}{\isasymnot}{\isacharparenleft}{\kern0pt}q\ {\isasymtturnstile}HS\ {\isasymphi}\ {\isacharparenleft}{\kern0pt}{\isacharbrackleft}{\kern0pt}b{\isacharbrackright}{\kern0pt}\ {\isacharat}{\kern0pt}\ env{\isacharparenright}{\kern0pt}{\isacharparenright}{\kern0pt}{\isacartoucheclose}\isanewline
\ \ \ \ \ \ \isacommand{ultimately}\isamarkupfalse%
\isanewline
\ \ \ \ \ \ \isacommand{show}\isamarkupfalse%
\ {\isacharquery}{\kern0pt}thesis\ \isacommand{by}\isamarkupfalse%
\ simp\isanewline
\ \ \ \ \isacommand{qed}\isamarkupfalse%
\isanewline
\ \ \isacommand{qed}\isamarkupfalse%
\isanewline
\isacommand{qed}\isamarkupfalse%
%
\endisatagproof
{\isafoldproof}%
%
\isadelimproof
\isanewline
%
\endisadelimproof
\isanewline
\isacommand{lemma}\isamarkupfalse%
\ definition{\isacharunderscore}{\kern0pt}of{\isacharunderscore}{\kern0pt}forcing{\isacharunderscore}{\kern0pt}HS{\isacharcolon}{\kern0pt}\isanewline
\ \ \isakeyword{assumes}\isanewline
\ \ \ \ {\isachardoublequoteopen}p{\isasymin}P{\isachardoublequoteclose}\ {\isachardoublequoteopen}{\isasymphi}{\isasymin}formula{\isachardoublequoteclose}\ {\isachardoublequoteopen}env{\isasymin}list{\isacharparenleft}{\kern0pt}HS{\isacharparenright}{\kern0pt}{\isachardoublequoteclose}\ {\isachardoublequoteopen}arity{\isacharparenleft}{\kern0pt}{\isasymphi}{\isacharparenright}{\kern0pt}{\isasymle}length{\isacharparenleft}{\kern0pt}env{\isacharparenright}{\kern0pt}{\isachardoublequoteclose}\isanewline
\ \ \isakeyword{shows}\isanewline
\ \ \ \ {\isachardoublequoteopen}{\isacharparenleft}{\kern0pt}p\ {\isasymtturnstile}HS\ {\isasymphi}\ env{\isacharparenright}{\kern0pt}\ {\isasymlongleftrightarrow}\isanewline
\ \ \ \ \ {\isacharparenleft}{\kern0pt}{\isasymforall}G{\isachardot}{\kern0pt}\ M{\isacharunderscore}{\kern0pt}generic{\isacharparenleft}{\kern0pt}G{\isacharparenright}{\kern0pt}\ {\isasymand}\ p{\isasymin}G\ \ {\isasymlongrightarrow}\ \ SymExt{\isacharparenleft}{\kern0pt}G{\isacharparenright}{\kern0pt}{\isacharcomma}{\kern0pt}\ map{\isacharparenleft}{\kern0pt}val{\isacharparenleft}{\kern0pt}G{\isacharparenright}{\kern0pt}{\isacharcomma}{\kern0pt}env{\isacharparenright}{\kern0pt}\ {\isasymTurnstile}\ {\isasymphi}{\isacharparenright}{\kern0pt}{\isachardoublequoteclose}\isanewline
%
\isadelimproof
%
\endisadelimproof
%
\isatagproof
\isacommand{proof}\isamarkupfalse%
\ {\isacharparenleft}{\kern0pt}intro\ iffI\ allI\ impI{\isacharcomma}{\kern0pt}\ elim\ conjE{\isacharparenright}{\kern0pt}\isanewline
\ \ \isacommand{fix}\isamarkupfalse%
\ G\isanewline
\ \ \isacommand{assume}\isamarkupfalse%
\ assms{\isadigit{1}}{\isacharcolon}{\kern0pt}{\isachardoublequoteopen}{\isacharparenleft}{\kern0pt}p\ {\isasymtturnstile}HS\ {\isasymphi}\ env{\isacharparenright}{\kern0pt}{\isachardoublequoteclose}\ {\isachardoublequoteopen}M{\isacharunderscore}{\kern0pt}generic{\isacharparenleft}{\kern0pt}G{\isacharparenright}{\kern0pt}{\isachardoublequoteclose}\ {\isachardoublequoteopen}p\ {\isasymin}\ G{\isachardoublequoteclose}\isanewline
\ \ \isacommand{with}\isamarkupfalse%
\ assms\ \isanewline
\ \ \isacommand{show}\isamarkupfalse%
\ {\isachardoublequoteopen}SymExt{\isacharparenleft}{\kern0pt}G{\isacharparenright}{\kern0pt}{\isacharcomma}{\kern0pt}\ map{\isacharparenleft}{\kern0pt}val{\isacharparenleft}{\kern0pt}G{\isacharparenright}{\kern0pt}{\isacharcomma}{\kern0pt}env{\isacharparenright}{\kern0pt}\ {\isasymTurnstile}\ {\isasymphi}{\isachardoublequoteclose}\isanewline
\ \ \ \ \isacommand{apply}\isamarkupfalse%
{\isacharparenleft}{\kern0pt}rule{\isacharunderscore}{\kern0pt}tac\ iffD{\isadigit{1}}{\isacharparenright}{\kern0pt}\isanewline
\ \ \ \ \ \isacommand{apply}\isamarkupfalse%
{\isacharparenleft}{\kern0pt}rule\ HS{\isacharunderscore}{\kern0pt}truth{\isacharunderscore}{\kern0pt}lemma{\isacharparenright}{\kern0pt}\isanewline
\ \ \ \ \isacommand{by}\isamarkupfalse%
\ auto\isanewline
\isacommand{next}\isamarkupfalse%
\isanewline
\ \ \isanewline
\ \ \isacommand{have}\isamarkupfalse%
\ envin\ {\isacharcolon}{\kern0pt}\ {\isachardoublequoteopen}env\ {\isasymin}\ list{\isacharparenleft}{\kern0pt}M{\isacharparenright}{\kern0pt}{\isachardoublequoteclose}\ \isanewline
\ \ \ \ \isacommand{apply}\isamarkupfalse%
{\isacharparenleft}{\kern0pt}rule{\isacharunderscore}{\kern0pt}tac\ A{\isacharequal}{\kern0pt}{\isachardoublequoteopen}list{\isacharparenleft}{\kern0pt}HS{\isacharparenright}{\kern0pt}{\isachardoublequoteclose}\ \isakeyword{in}\ subsetD{\isacharcomma}{\kern0pt}\ rule\ list{\isacharunderscore}{\kern0pt}mono{\isacharparenright}{\kern0pt}\isanewline
\ \ \ \ \isacommand{using}\isamarkupfalse%
\ HS{\isacharunderscore}{\kern0pt}iff\ P{\isacharunderscore}{\kern0pt}name{\isacharunderscore}{\kern0pt}in{\isacharunderscore}{\kern0pt}M\ assms\ Forall\isanewline
\ \ \ \ \isacommand{by}\isamarkupfalse%
\ auto\isanewline
\isanewline
\ \ \isacommand{assume}\isamarkupfalse%
\ {\isadigit{1}}{\isacharcolon}{\kern0pt}\ {\isachardoublequoteopen}{\isasymforall}G{\isachardot}{\kern0pt}{\isacharparenleft}{\kern0pt}M{\isacharunderscore}{\kern0pt}generic{\isacharparenleft}{\kern0pt}G{\isacharparenright}{\kern0pt}{\isasymand}\ p{\isasymin}G{\isacharparenright}{\kern0pt}{\isasymlongrightarrow}\ SymExt{\isacharparenleft}{\kern0pt}G{\isacharparenright}{\kern0pt}\ {\isacharcomma}{\kern0pt}\ map{\isacharparenleft}{\kern0pt}val{\isacharparenleft}{\kern0pt}G{\isacharparenright}{\kern0pt}{\isacharcomma}{\kern0pt}env{\isacharparenright}{\kern0pt}\ {\isasymTurnstile}\ {\isasymphi}{\isachardoublequoteclose}\isanewline
\ \ \isacommand{{\isacharbraceleft}{\kern0pt}}\isamarkupfalse%
\isanewline
\ \ \ \ \isacommand{fix}\isamarkupfalse%
\ r\ \isanewline
\ \ \ \ \isacommand{assume}\isamarkupfalse%
\ {\isadigit{2}}{\isacharcolon}{\kern0pt}\ {\isachardoublequoteopen}r{\isasymin}P{\isachardoublequoteclose}\ {\isachardoublequoteopen}r{\isasympreceq}p{\isachardoublequoteclose}\isanewline
\ \ \ \ \isacommand{then}\isamarkupfalse%
\ \isanewline
\ \ \ \ \isacommand{obtain}\isamarkupfalse%
\ G\ \isakeyword{where}\ {\isachardoublequoteopen}r{\isasymin}G{\isachardoublequoteclose}\ {\isachardoublequoteopen}M{\isacharunderscore}{\kern0pt}generic{\isacharparenleft}{\kern0pt}G{\isacharparenright}{\kern0pt}{\isachardoublequoteclose}\isanewline
\ \ \ \ \ \ \isacommand{using}\isamarkupfalse%
\ generic{\isacharunderscore}{\kern0pt}filter{\isacharunderscore}{\kern0pt}existence\ \isacommand{by}\isamarkupfalse%
\ auto\isanewline
\isanewline
\ \ \ \ \isacommand{moreover}\isamarkupfalse%
\ \isacommand{from}\isamarkupfalse%
\ calculation\ {\isadigit{2}}\ {\isacartoucheopen}p{\isasymin}P{\isacartoucheclose}\ \isanewline
\ \ \ \ \isacommand{have}\isamarkupfalse%
\ {\isachardoublequoteopen}p{\isasymin}G{\isachardoublequoteclose}\ \isanewline
\ \ \ \ \ \ \isacommand{unfolding}\isamarkupfalse%
\ M{\isacharunderscore}{\kern0pt}generic{\isacharunderscore}{\kern0pt}def\ \isacommand{using}\isamarkupfalse%
\ filter{\isacharunderscore}{\kern0pt}leqD\ \isacommand{by}\isamarkupfalse%
\ simp\isanewline
\ \ \ \ \isacommand{moreover}\isamarkupfalse%
\ \isacommand{note}\isamarkupfalse%
\ {\isadigit{1}}\isanewline
\ \ \ \ \isacommand{ultimately}\isamarkupfalse%
\isanewline
\ \ \ \ \isacommand{have}\isamarkupfalse%
\ satsphi\ {\isacharcolon}{\kern0pt}\ {\isachardoublequoteopen}SymExt{\isacharparenleft}{\kern0pt}G{\isacharparenright}{\kern0pt}{\isacharcomma}{\kern0pt}\ map{\isacharparenleft}{\kern0pt}val{\isacharparenleft}{\kern0pt}G{\isacharparenright}{\kern0pt}{\isacharcomma}{\kern0pt}env{\isacharparenright}{\kern0pt}\ {\isasymTurnstile}\ {\isasymphi}{\isachardoublequoteclose}\isanewline
\ \ \ \ \ \ \isacommand{by}\isamarkupfalse%
\ simp\isanewline
\isanewline
\ \ \ \ \isacommand{have}\isamarkupfalse%
\ {\isachardoublequoteopen}{\isasymexists}s\ {\isasymin}\ G{\isachardot}{\kern0pt}\ s\ {\isasymtturnstile}HS\ {\isasymphi}\ env{\isachardoublequoteclose}\isanewline
\ \ \ \ \ \ \isacommand{apply}\isamarkupfalse%
{\isacharparenleft}{\kern0pt}rule\ iffD{\isadigit{2}}{\isacharparenright}{\kern0pt}\isanewline
\ \ \ \ \ \ \ \isacommand{apply}\isamarkupfalse%
{\isacharparenleft}{\kern0pt}rule\ HS{\isacharunderscore}{\kern0pt}truth{\isacharunderscore}{\kern0pt}lemma{\isacharparenright}{\kern0pt}\isanewline
\ \ \ \ \ \ \isacommand{using}\isamarkupfalse%
\ assms\ satsphi\ {\isacartoucheopen}M{\isacharunderscore}{\kern0pt}generic{\isacharparenleft}{\kern0pt}G{\isacharparenright}{\kern0pt}{\isacartoucheclose}\isanewline
\ \ \ \ \ \ \isacommand{by}\isamarkupfalse%
\ auto\isanewline
\ \ \ \ \isacommand{then}\isamarkupfalse%
\ \isacommand{obtain}\isamarkupfalse%
\ s\ \isakeyword{where}\ {\isachardoublequoteopen}s{\isasymin}G{\isachardoublequoteclose}\ {\isachardoublequoteopen}{\isacharparenleft}{\kern0pt}s\ {\isasymtturnstile}HS\ {\isasymphi}\ env{\isacharparenright}{\kern0pt}{\isachardoublequoteclose}\isanewline
\ \ \ \ \ \ \isacommand{using}\isamarkupfalse%
\ satsphi\ \isacommand{by}\isamarkupfalse%
\ blast\isanewline
\ \ \ \ \isacommand{moreover}\isamarkupfalse%
\ \isacommand{from}\isamarkupfalse%
\ this\ \isakeyword{and}\ \ {\isacartoucheopen}M{\isacharunderscore}{\kern0pt}generic{\isacharparenleft}{\kern0pt}G{\isacharparenright}{\kern0pt}{\isacartoucheclose}\ {\isacartoucheopen}r{\isasymin}G{\isacartoucheclose}\ \isanewline
\ \ \ \ \isacommand{obtain}\isamarkupfalse%
\ q\ \isakeyword{where}\ {\isachardoublequoteopen}q{\isasymin}G{\isachardoublequoteclose}\ {\isachardoublequoteopen}q\ {\isasympreceq}\ s{\isachardoublequoteclose}\ {\isachardoublequoteopen}q\ {\isasympreceq}\ r{\isachardoublequoteclose}\isanewline
\ \ \ \ \ \ \isacommand{by}\isamarkupfalse%
\ blast\isanewline
\ \ \ \ \isacommand{moreover}\isamarkupfalse%
\ \isacommand{from}\isamarkupfalse%
\ calculation\ {\isacartoucheopen}s{\isasymin}G{\isacartoucheclose}\ {\isacartoucheopen}M{\isacharunderscore}{\kern0pt}generic{\isacharparenleft}{\kern0pt}G{\isacharparenright}{\kern0pt}{\isacartoucheclose}\ \isanewline
\ \ \ \ \isacommand{have}\isamarkupfalse%
\ {\isachardoublequoteopen}s{\isasymin}P{\isachardoublequoteclose}\ {\isachardoublequoteopen}q{\isasymin}P{\isachardoublequoteclose}\ \isanewline
\ \ \ \ \ \ \isacommand{unfolding}\isamarkupfalse%
\ M{\isacharunderscore}{\kern0pt}generic{\isacharunderscore}{\kern0pt}def\ filter{\isacharunderscore}{\kern0pt}def\ \isacommand{by}\isamarkupfalse%
\ auto\isanewline
\ \ \ \ \isacommand{moreover}\isamarkupfalse%
\ \isanewline
\ \ \ \ \isacommand{note}\isamarkupfalse%
\ assms\isanewline
\ \ \ \ \isacommand{ultimately}\isamarkupfalse%
\ \isanewline
\ \ \ \ \isacommand{have}\isamarkupfalse%
\ {\isachardoublequoteopen}{\isasymexists}q{\isasymin}P{\isachardot}{\kern0pt}\ q{\isasympreceq}r\ {\isasymand}\ {\isacharparenleft}{\kern0pt}q\ {\isasymtturnstile}HS\ {\isasymphi}\ env{\isacharparenright}{\kern0pt}{\isachardoublequoteclose}\isanewline
\ \ \ \ \ \ \isacommand{apply}\isamarkupfalse%
{\isacharparenleft}{\kern0pt}rule{\isacharunderscore}{\kern0pt}tac\ x{\isacharequal}{\kern0pt}q\ \isakeyword{in}\ bexI{\isacharparenright}{\kern0pt}\isanewline
\ \ \ \ \ \ \ \isacommand{apply}\isamarkupfalse%
{\isacharparenleft}{\kern0pt}rule\ conjI{\isacharparenright}{\kern0pt}\ \isanewline
\ \ \ \ \ \ \isacommand{using}\isamarkupfalse%
\ {\isacartoucheopen}q\ {\isasympreceq}\ r{\isacartoucheclose}\ \isanewline
\ \ \ \ \ \ \ \ \isacommand{apply}\isamarkupfalse%
\ simp\isanewline
\ \ \ \ \ \ \ \isacommand{apply}\isamarkupfalse%
{\isacharparenleft}{\kern0pt}rule{\isacharunderscore}{\kern0pt}tac\ p{\isacharequal}{\kern0pt}s\ \isakeyword{in}\ HS{\isacharunderscore}{\kern0pt}strengthening{\isacharunderscore}{\kern0pt}lemma{\isacharparenright}{\kern0pt}\isanewline
\ \ \ \ \ \ \isacommand{using}\isamarkupfalse%
\ M{\isacharunderscore}{\kern0pt}genericD\ {\isacartoucheopen}M{\isacharunderscore}{\kern0pt}generic{\isacharparenleft}{\kern0pt}G{\isacharparenright}{\kern0pt}{\isacartoucheclose}\ {\isacartoucheopen}s{\isasymin}G{\isacartoucheclose}\ envin\ \isanewline
\ \ \ \ \ \ \isacommand{by}\isamarkupfalse%
\ auto\isanewline
\ \ \isacommand{{\isacharbraceright}{\kern0pt}}\isamarkupfalse%
\isanewline
\ \ \isacommand{then}\isamarkupfalse%
\isanewline
\ \ \isacommand{have}\isamarkupfalse%
\ {\isachardoublequoteopen}dense{\isacharunderscore}{\kern0pt}below{\isacharparenleft}{\kern0pt}{\isacharbraceleft}{\kern0pt}q{\isasymin}P{\isachardot}{\kern0pt}\ {\isacharparenleft}{\kern0pt}q\ {\isasymtturnstile}HS\ {\isasymphi}\ env{\isacharparenright}{\kern0pt}{\isacharbraceright}{\kern0pt}{\isacharcomma}{\kern0pt}p{\isacharparenright}{\kern0pt}{\isachardoublequoteclose}\isanewline
\ \ \ \ \isacommand{unfolding}\isamarkupfalse%
\ dense{\isacharunderscore}{\kern0pt}below{\isacharunderscore}{\kern0pt}def\ \isacommand{by}\isamarkupfalse%
\ blast\isanewline
\ \ \isacommand{with}\isamarkupfalse%
\ assms\isanewline
\ \ \isacommand{show}\isamarkupfalse%
\ {\isachardoublequoteopen}{\isacharparenleft}{\kern0pt}p\ {\isasymtturnstile}HS\ {\isasymphi}\ env{\isacharparenright}{\kern0pt}{\isachardoublequoteclose}\isanewline
\ \ \ \ \isacommand{apply}\isamarkupfalse%
{\isacharparenleft}{\kern0pt}rule{\isacharunderscore}{\kern0pt}tac\ iffD{\isadigit{2}}{\isacharparenright}{\kern0pt}\isanewline
\ \ \ \ \ \isacommand{apply}\isamarkupfalse%
{\isacharparenleft}{\kern0pt}rule\ HS{\isacharunderscore}{\kern0pt}density{\isacharunderscore}{\kern0pt}lemma{\isacharparenright}{\kern0pt}\ \isanewline
\ \ \ \ \isacommand{using}\isamarkupfalse%
\ envin\isanewline
\ \ \ \ \isacommand{by}\isamarkupfalse%
\ auto\isanewline
\isacommand{qed}\isamarkupfalse%
%
\endisatagproof
{\isafoldproof}%
%
\isadelimproof
\isanewline
%
\endisadelimproof
\isanewline
\isanewline
\isacommand{end}\isamarkupfalse%
\isanewline
%
\isadelimtheory
%
\endisadelimtheory
%
\isatagtheory
\isacommand{end}\isamarkupfalse%
%
\endisatagtheory
{\isafoldtheory}%
%
\isadelimtheory
%
\endisadelimtheory
%
\end{isabellebody}%
\endinput
%:%file=~/source/repos/ZF-notAC/code/HS_Forces.thy%:%
%:%10=1%:%
%:%11=1%:%
%:%12=2%:%
%:%13=3%:%
%:%18=3%:%
%:%21=4%:%
%:%22=5%:%
%:%23=5%:%
%:%24=6%:%
%:%25=7%:%
%:%26=8%:%
%:%27=9%:%
%:%28=9%:%
%:%29=10%:%
%:%30=10%:%
%:%31=11%:%
%:%32=12%:%
%:%33=13%:%
%:%35=15%:%
%:%36=16%:%
%:%37=17%:%
%:%38=18%:%
%:%39=18%:%
%:%40=19%:%
%:%41=20%:%
%:%42=21%:%
%:%43=22%:%
%:%44=22%:%
%:%45=23%:%
%:%46=24%:%
%:%47=24%:%
%:%48=25%:%
%:%49=26%:%
%:%50=27%:%
%:%51=27%:%
%:%52=28%:%
%:%53=29%:%
%:%54=30%:%
%:%57=31%:%
%:%58=32%:%
%:%62=32%:%
%:%63=32%:%
%:%64=33%:%
%:%65=33%:%
%:%66=34%:%
%:%67=34%:%
%:%68=35%:%
%:%69=35%:%
%:%70=36%:%
%:%71=36%:%
%:%72=37%:%
%:%73=37%:%
%:%74=38%:%
%:%75=38%:%
%:%76=39%:%
%:%77=39%:%
%:%78=40%:%
%:%79=40%:%
%:%80=41%:%
%:%81=41%:%
%:%82=42%:%
%:%83=42%:%
%:%84=43%:%
%:%85=43%:%
%:%86=44%:%
%:%87=44%:%
%:%88=45%:%
%:%89=45%:%
%:%90=46%:%
%:%91=46%:%
%:%92=47%:%
%:%93=47%:%
%:%94=48%:%
%:%95=48%:%
%:%96=49%:%
%:%97=49%:%
%:%98=50%:%
%:%99=50%:%
%:%100=51%:%
%:%101=51%:%
%:%102=52%:%
%:%103=52%:%
%:%104=53%:%
%:%105=53%:%
%:%106=54%:%
%:%107=54%:%
%:%108=55%:%
%:%109=55%:%
%:%110=56%:%
%:%116=56%:%
%:%119=57%:%
%:%120=58%:%
%:%121=58%:%
%:%122=59%:%
%:%123=60%:%
%:%124=61%:%
%:%127=62%:%
%:%128=63%:%
%:%132=63%:%
%:%133=63%:%
%:%134=64%:%
%:%135=64%:%
%:%136=65%:%
%:%137=65%:%
%:%138=66%:%
%:%139=66%:%
%:%140=67%:%
%:%141=67%:%
%:%142=68%:%
%:%143=68%:%
%:%144=69%:%
%:%145=69%:%
%:%146=70%:%
%:%147=70%:%
%:%148=71%:%
%:%149=71%:%
%:%150=72%:%
%:%151=72%:%
%:%152=73%:%
%:%153=73%:%
%:%154=74%:%
%:%155=74%:%
%:%156=75%:%
%:%157=75%:%
%:%158=76%:%
%:%159=76%:%
%:%160=77%:%
%:%161=77%:%
%:%162=78%:%
%:%163=78%:%
%:%164=79%:%
%:%165=79%:%
%:%166=80%:%
%:%167=80%:%
%:%168=81%:%
%:%169=81%:%
%:%170=82%:%
%:%171=82%:%
%:%172=83%:%
%:%173=83%:%
%:%174=84%:%
%:%175=84%:%
%:%176=85%:%
%:%177=85%:%
%:%178=86%:%
%:%179=86%:%
%:%180=87%:%
%:%181=87%:%
%:%182=88%:%
%:%183=88%:%
%:%184=89%:%
%:%185=89%:%
%:%186=90%:%
%:%187=90%:%
%:%188=91%:%
%:%189=91%:%
%:%190=92%:%
%:%191=92%:%
%:%192=93%:%
%:%193=93%:%
%:%194=94%:%
%:%195=94%:%
%:%196=95%:%
%:%197=95%:%
%:%198=96%:%
%:%199=96%:%
%:%200=97%:%
%:%201=97%:%
%:%202=98%:%
%:%203=98%:%
%:%204=99%:%
%:%205=99%:%
%:%206=100%:%
%:%207=100%:%
%:%208=101%:%
%:%209=101%:%
%:%210=102%:%
%:%211=102%:%
%:%212=103%:%
%:%213=103%:%
%:%214=104%:%
%:%215=104%:%
%:%216=105%:%
%:%217=105%:%
%:%218=106%:%
%:%219=106%:%
%:%220=107%:%
%:%221=107%:%
%:%222=108%:%
%:%223=108%:%
%:%224=109%:%
%:%225=109%:%
%:%226=110%:%
%:%232=110%:%
%:%235=111%:%
%:%236=112%:%
%:%237=112%:%
%:%238=113%:%
%:%239=114%:%
%:%240=115%:%
%:%243=116%:%
%:%247=116%:%
%:%248=116%:%
%:%249=117%:%
%:%250=117%:%
%:%251=118%:%
%:%252=118%:%
%:%253=119%:%
%:%254=119%:%
%:%255=120%:%
%:%256=120%:%
%:%257=121%:%
%:%258=121%:%
%:%259=122%:%
%:%260=122%:%
%:%261=123%:%
%:%262=123%:%
%:%263=124%:%
%:%264=124%:%
%:%265=125%:%
%:%266=125%:%
%:%267=126%:%
%:%268=126%:%
%:%269=127%:%
%:%270=127%:%
%:%275=127%:%
%:%278=128%:%
%:%279=129%:%
%:%280=129%:%
%:%281=130%:%
%:%282=131%:%
%:%283=132%:%
%:%286=133%:%
%:%290=133%:%
%:%291=133%:%
%:%292=134%:%
%:%293=134%:%
%:%294=135%:%
%:%295=135%:%
%:%296=136%:%
%:%297=136%:%
%:%302=136%:%
%:%305=137%:%
%:%306=138%:%
%:%307=138%:%
%:%308=139%:%
%:%309=140%:%
%:%310=141%:%
%:%313=142%:%
%:%317=142%:%
%:%318=142%:%
%:%319=143%:%
%:%320=143%:%
%:%321=144%:%
%:%322=144%:%
%:%323=145%:%
%:%324=145%:%
%:%325=146%:%
%:%326=146%:%
%:%327=147%:%
%:%328=147%:%
%:%329=148%:%
%:%330=148%:%
%:%331=149%:%
%:%332=149%:%
%:%333=150%:%
%:%334=150%:%
%:%335=151%:%
%:%336=151%:%
%:%337=152%:%
%:%338=152%:%
%:%339=153%:%
%:%340=153%:%
%:%341=154%:%
%:%342=154%:%
%:%343=155%:%
%:%344=155%:%
%:%345=156%:%
%:%346=156%:%
%:%347=157%:%
%:%348=157%:%
%:%349=158%:%
%:%350=158%:%
%:%351=159%:%
%:%352=159%:%
%:%353=160%:%
%:%354=160%:%
%:%355=161%:%
%:%356=161%:%
%:%357=162%:%
%:%358=162%:%
%:%359=163%:%
%:%360=163%:%
%:%361=164%:%
%:%362=164%:%
%:%363=165%:%
%:%364=165%:%
%:%365=166%:%
%:%366=166%:%
%:%367=167%:%
%:%368=167%:%
%:%369=168%:%
%:%370=168%:%
%:%371=169%:%
%:%372=169%:%
%:%373=170%:%
%:%374=170%:%
%:%375=171%:%
%:%376=171%:%
%:%377=172%:%
%:%378=172%:%
%:%379=173%:%
%:%380=173%:%
%:%381=174%:%
%:%382=174%:%
%:%383=175%:%
%:%384=175%:%
%:%385=176%:%
%:%386=176%:%
%:%387=177%:%
%:%388=177%:%
%:%389=178%:%
%:%390=178%:%
%:%391=179%:%
%:%392=179%:%
%:%393=180%:%
%:%394=180%:%
%:%395=181%:%
%:%396=181%:%
%:%397=182%:%
%:%398=182%:%
%:%399=183%:%
%:%400=183%:%
%:%401=184%:%
%:%402=184%:%
%:%403=185%:%
%:%404=185%:%
%:%405=186%:%
%:%406=186%:%
%:%407=187%:%
%:%408=187%:%
%:%409=188%:%
%:%410=188%:%
%:%411=189%:%
%:%412=189%:%
%:%413=190%:%
%:%414=190%:%
%:%415=191%:%
%:%416=191%:%
%:%417=192%:%
%:%418=192%:%
%:%419=193%:%
%:%420=193%:%
%:%421=194%:%
%:%422=194%:%
%:%423=195%:%
%:%424=195%:%
%:%425=196%:%
%:%426=196%:%
%:%427=197%:%
%:%428=197%:%
%:%429=198%:%
%:%430=198%:%
%:%431=199%:%
%:%432=199%:%
%:%433=200%:%
%:%434=200%:%
%:%435=201%:%
%:%436=201%:%
%:%437=202%:%
%:%438=202%:%
%:%439=203%:%
%:%440=203%:%
%:%441=204%:%
%:%442=204%:%
%:%443=205%:%
%:%444=205%:%
%:%445=206%:%
%:%446=206%:%
%:%447=207%:%
%:%448=207%:%
%:%449=208%:%
%:%450=208%:%
%:%451=209%:%
%:%452=209%:%
%:%453=210%:%
%:%459=210%:%
%:%462=211%:%
%:%463=212%:%
%:%464=212%:%
%:%465=213%:%
%:%466=214%:%
%:%467=215%:%
%:%470=216%:%
%:%474=216%:%
%:%475=216%:%
%:%476=217%:%
%:%477=217%:%
%:%478=218%:%
%:%479=218%:%
%:%480=219%:%
%:%481=219%:%
%:%482=220%:%
%:%483=220%:%
%:%484=221%:%
%:%485=221%:%
%:%486=222%:%
%:%487=222%:%
%:%492=222%:%
%:%495=223%:%
%:%496=224%:%
%:%497=224%:%
%:%498=225%:%
%:%499=226%:%
%:%500=227%:%
%:%503=228%:%
%:%507=228%:%
%:%508=228%:%
%:%509=229%:%
%:%510=229%:%
%:%511=230%:%
%:%512=230%:%
%:%513=231%:%
%:%514=231%:%
%:%515=232%:%
%:%516=232%:%
%:%517=233%:%
%:%518=233%:%
%:%519=234%:%
%:%520=235%:%
%:%521=235%:%
%:%522=236%:%
%:%523=236%:%
%:%524=237%:%
%:%525=237%:%
%:%526=238%:%
%:%527=238%:%
%:%528=239%:%
%:%529=239%:%
%:%534=239%:%
%:%537=240%:%
%:%538=241%:%
%:%539=241%:%
%:%540=242%:%
%:%541=243%:%
%:%542=244%:%
%:%545=245%:%
%:%549=245%:%
%:%550=245%:%
%:%551=246%:%
%:%552=246%:%
%:%553=247%:%
%:%554=247%:%
%:%555=248%:%
%:%556=248%:%
%:%557=249%:%
%:%558=249%:%
%:%563=249%:%
%:%566=250%:%
%:%567=251%:%
%:%568=251%:%
%:%569=252%:%
%:%570=253%:%
%:%571=254%:%
%:%574=255%:%
%:%578=255%:%
%:%579=255%:%
%:%580=256%:%
%:%581=256%:%
%:%582=257%:%
%:%583=257%:%
%:%584=258%:%
%:%585=258%:%
%:%586=259%:%
%:%587=259%:%
%:%588=260%:%
%:%589=260%:%
%:%590=261%:%
%:%591=262%:%
%:%592=262%:%
%:%593=263%:%
%:%594=263%:%
%:%595=264%:%
%:%596=264%:%
%:%597=265%:%
%:%598=265%:%
%:%599=266%:%
%:%600=266%:%
%:%605=266%:%
%:%608=267%:%
%:%609=268%:%
%:%610=268%:%
%:%611=269%:%
%:%612=270%:%
%:%613=271%:%
%:%616=272%:%
%:%620=272%:%
%:%621=272%:%
%:%622=273%:%
%:%623=273%:%
%:%624=274%:%
%:%625=274%:%
%:%626=275%:%
%:%627=275%:%
%:%628=276%:%
%:%629=276%:%
%:%634=276%:%
%:%637=277%:%
%:%638=278%:%
%:%639=278%:%
%:%640=279%:%
%:%641=280%:%
%:%643=282%:%
%:%646=283%:%
%:%650=283%:%
%:%651=283%:%
%:%652=283%:%
%:%653=284%:%
%:%654=284%:%
%:%655=285%:%
%:%656=285%:%
%:%657=286%:%
%:%658=286%:%
%:%659=287%:%
%:%660=287%:%
%:%661=288%:%
%:%662=288%:%
%:%663=289%:%
%:%664=290%:%
%:%665=290%:%
%:%666=291%:%
%:%667=291%:%
%:%668=292%:%
%:%669=292%:%
%:%670=293%:%
%:%671=293%:%
%:%672=294%:%
%:%673=295%:%
%:%674=295%:%
%:%675=296%:%
%:%676=296%:%
%:%677=297%:%
%:%683=297%:%
%:%686=298%:%
%:%687=299%:%
%:%688=299%:%
%:%689=300%:%
%:%690=301%:%
%:%691=302%:%
%:%692=303%:%
%:%695=306%:%
%:%698=307%:%
%:%702=307%:%
%:%703=307%:%
%:%704=308%:%
%:%705=308%:%
%:%706=309%:%
%:%707=309%:%
%:%708=310%:%
%:%709=310%:%
%:%710=311%:%
%:%711=311%:%
%:%712=312%:%
%:%713=312%:%
%:%718=312%:%
%:%721=313%:%
%:%722=314%:%
%:%723=314%:%
%:%724=315%:%
%:%725=316%:%
%:%726=317%:%
%:%727=318%:%
%:%730=319%:%
%:%734=319%:%
%:%735=319%:%
%:%736=320%:%
%:%737=320%:%
%:%742=320%:%
%:%745=321%:%
%:%746=322%:%
%:%747=322%:%
%:%748=323%:%
%:%749=324%:%
%:%750=325%:%
%:%751=326%:%
%:%754=327%:%
%:%758=327%:%
%:%759=327%:%
%:%760=328%:%
%:%761=328%:%
%:%762=329%:%
%:%763=329%:%
%:%768=329%:%
%:%771=330%:%
%:%772=331%:%
%:%773=331%:%
%:%774=332%:%
%:%775=333%:%
%:%776=334%:%
%:%779=335%:%
%:%783=335%:%
%:%784=335%:%
%:%785=336%:%
%:%786=336%:%
%:%787=337%:%
%:%788=337%:%
%:%789=338%:%
%:%790=338%:%
%:%791=339%:%
%:%792=339%:%
%:%793=340%:%
%:%794=340%:%
%:%795=341%:%
%:%796=341%:%
%:%797=342%:%
%:%798=342%:%
%:%799=343%:%
%:%800=343%:%
%:%801=344%:%
%:%802=344%:%
%:%803=345%:%
%:%804=345%:%
%:%805=346%:%
%:%806=346%:%
%:%807=347%:%
%:%808=347%:%
%:%809=348%:%
%:%810=348%:%
%:%811=349%:%
%:%812=349%:%
%:%813=350%:%
%:%814=350%:%
%:%815=351%:%
%:%816=351%:%
%:%817=352%:%
%:%818=352%:%
%:%819=353%:%
%:%820=354%:%
%:%821=355%:%
%:%822=355%:%
%:%823=356%:%
%:%824=356%:%
%:%825=357%:%
%:%826=357%:%
%:%827=358%:%
%:%828=358%:%
%:%829=359%:%
%:%830=359%:%
%:%831=360%:%
%:%832=360%:%
%:%833=361%:%
%:%834=361%:%
%:%835=362%:%
%:%836=362%:%
%:%837=363%:%
%:%838=363%:%
%:%839=364%:%
%:%840=364%:%
%:%845=364%:%
%:%848=365%:%
%:%849=366%:%
%:%850=366%:%
%:%851=367%:%
%:%852=368%:%
%:%853=369%:%
%:%854=370%:%
%:%857=371%:%
%:%861=371%:%
%:%862=371%:%
%:%863=372%:%
%:%864=372%:%
%:%865=373%:%
%:%866=373%:%
%:%867=374%:%
%:%868=374%:%
%:%869=375%:%
%:%870=375%:%
%:%871=376%:%
%:%872=376%:%
%:%873=377%:%
%:%874=377%:%
%:%875=378%:%
%:%876=378%:%
%:%877=379%:%
%:%878=379%:%
%:%879=380%:%
%:%880=380%:%
%:%881=381%:%
%:%882=381%:%
%:%887=381%:%
%:%890=382%:%
%:%891=383%:%
%:%892=383%:%
%:%893=384%:%
%:%894=385%:%
%:%895=386%:%
%:%896=387%:%
%:%899=388%:%
%:%903=388%:%
%:%904=388%:%
%:%905=388%:%
%:%910=388%:%
%:%913=389%:%
%:%914=390%:%
%:%915=390%:%
%:%916=391%:%
%:%917=392%:%
%:%918=393%:%
%:%919=394%:%
%:%922=395%:%
%:%926=395%:%
%:%927=395%:%
%:%928=396%:%
%:%929=396%:%
%:%930=397%:%
%:%931=397%:%
%:%932=398%:%
%:%933=398%:%
%:%934=398%:%
%:%935=399%:%
%:%936=400%:%
%:%937=401%:%
%:%938=401%:%
%:%939=402%:%
%:%940=402%:%
%:%941=402%:%
%:%942=403%:%
%:%943=403%:%
%:%944=404%:%
%:%945=404%:%
%:%946=405%:%
%:%947=405%:%
%:%948=406%:%
%:%949=406%:%
%:%950=407%:%
%:%951=407%:%
%:%952=407%:%
%:%953=408%:%
%:%954=408%:%
%:%955=409%:%
%:%956=409%:%
%:%957=410%:%
%:%958=410%:%
%:%959=411%:%
%:%960=411%:%
%:%961=411%:%
%:%962=412%:%
%:%963=412%:%
%:%964=413%:%
%:%965=413%:%
%:%966=414%:%
%:%967=414%:%
%:%968=415%:%
%:%969=415%:%
%:%970=416%:%
%:%971=416%:%
%:%972=417%:%
%:%973=417%:%
%:%974=418%:%
%:%975=418%:%
%:%976=418%:%
%:%977=419%:%
%:%978=420%:%
%:%979=421%:%
%:%980=421%:%
%:%981=422%:%
%:%982=422%:%
%:%983=422%:%
%:%984=423%:%
%:%985=423%:%
%:%986=424%:%
%:%987=424%:%
%:%988=425%:%
%:%989=425%:%
%:%990=426%:%
%:%991=426%:%
%:%992=427%:%
%:%993=427%:%
%:%994=427%:%
%:%995=428%:%
%:%996=428%:%
%:%997=429%:%
%:%998=429%:%
%:%999=430%:%
%:%1000=430%:%
%:%1001=431%:%
%:%1002=431%:%
%:%1003=431%:%
%:%1004=432%:%
%:%1005=432%:%
%:%1006=433%:%
%:%1007=433%:%
%:%1008=434%:%
%:%1009=434%:%
%:%1010=435%:%
%:%1011=435%:%
%:%1012=436%:%
%:%1013=436%:%
%:%1014=437%:%
%:%1015=437%:%
%:%1016=438%:%
%:%1017=438%:%
%:%1018=439%:%
%:%1019=439%:%
%:%1020=440%:%
%:%1021=440%:%
%:%1022=441%:%
%:%1023=441%:%
%:%1024=442%:%
%:%1025=442%:%
%:%1026=443%:%
%:%1027=443%:%
%:%1028=444%:%
%:%1029=444%:%
%:%1030=445%:%
%:%1031=445%:%
%:%1032=446%:%
%:%1033=446%:%
%:%1034=446%:%
%:%1035=447%:%
%:%1036=448%:%
%:%1037=448%:%
%:%1038=449%:%
%:%1039=449%:%
%:%1040=450%:%
%:%1041=450%:%
%:%1042=450%:%
%:%1043=451%:%
%:%1044=451%:%
%:%1045=452%:%
%:%1046=452%:%
%:%1047=453%:%
%:%1048=453%:%
%:%1049=453%:%
%:%1050=454%:%
%:%1056=454%:%
%:%1059=455%:%
%:%1060=456%:%
%:%1061=457%:%
%:%1062=458%:%
%:%1063=458%:%
%:%1064=459%:%
%:%1065=460%:%
%:%1072=461%:%
%:%1073=461%:%
%:%1074=462%:%
%:%1075=462%:%
%:%1076=463%:%
%:%1077=463%:%
%:%1078=463%:%
%:%1079=464%:%
%:%1080=464%:%
%:%1081=465%:%
%:%1082=465%:%
%:%1083=466%:%
%:%1084=466%:%
%:%1085=467%:%
%:%1086=467%:%
%:%1087=468%:%
%:%1088=468%:%
%:%1089=469%:%
%:%1090=469%:%
%:%1091=470%:%
%:%1092=470%:%
%:%1093=471%:%
%:%1094=471%:%
%:%1095=472%:%
%:%1096=472%:%
%:%1097=473%:%
%:%1098=473%:%
%:%1099=474%:%
%:%1100=474%:%
%:%1101=474%:%
%:%1102=474%:%
%:%1103=475%:%
%:%1109=475%:%
%:%1112=476%:%
%:%1113=477%:%
%:%1114=477%:%
%:%1115=478%:%
%:%1116=479%:%
%:%1117=480%:%
%:%1118=481%:%
%:%1119=482%:%
%:%1120=483%:%
%:%1127=484%:%
%:%1128=484%:%
%:%1129=485%:%
%:%1130=485%:%
%:%1131=486%:%
%:%1132=486%:%
%:%1133=486%:%
%:%1134=487%:%
%:%1135=487%:%
%:%1136=488%:%
%:%1137=488%:%
%:%1138=489%:%
%:%1139=489%:%
%:%1140=490%:%
%:%1141=490%:%
%:%1142=491%:%
%:%1143=491%:%
%:%1144=492%:%
%:%1145=492%:%
%:%1146=493%:%
%:%1147=493%:%
%:%1148=494%:%
%:%1149=494%:%
%:%1150=494%:%
%:%1151=495%:%
%:%1152=495%:%
%:%1153=495%:%
%:%1154=495%:%
%:%1155=496%:%
%:%1161=496%:%
%:%1164=497%:%
%:%1165=498%:%
%:%1166=498%:%
%:%1167=499%:%
%:%1168=500%:%
%:%1169=501%:%
%:%1170=502%:%
%:%1171=503%:%
%:%1174=504%:%
%:%1178=504%:%
%:%1179=504%:%
%:%1180=504%:%
%:%1181=504%:%
%:%1186=504%:%
%:%1189=505%:%
%:%1190=506%:%
%:%1191=506%:%
%:%1192=507%:%
%:%1193=508%:%
%:%1194=509%:%
%:%1195=510%:%
%:%1198=511%:%
%:%1202=511%:%
%:%1203=511%:%
%:%1204=511%:%
%:%1205=512%:%
%:%1206=512%:%
%:%1211=512%:%
%:%1214=513%:%
%:%1215=514%:%
%:%1216=514%:%
%:%1217=515%:%
%:%1218=516%:%
%:%1219=517%:%
%:%1220=518%:%
%:%1221=519%:%
%:%1224=520%:%
%:%1228=520%:%
%:%1229=520%:%
%:%1230=521%:%
%:%1231=521%:%
%:%1232=522%:%
%:%1233=522%:%
%:%1234=523%:%
%:%1235=523%:%
%:%1236=523%:%
%:%1237=524%:%
%:%1238=524%:%
%:%1239=525%:%
%:%1240=525%:%
%:%1241=526%:%
%:%1242=526%:%
%:%1243=527%:%
%:%1244=527%:%
%:%1245=528%:%
%:%1246=528%:%
%:%1247=529%:%
%:%1248=529%:%
%:%1249=529%:%
%:%1250=530%:%
%:%1251=530%:%
%:%1252=531%:%
%:%1253=531%:%
%:%1254=532%:%
%:%1255=532%:%
%:%1256=532%:%
%:%1257=533%:%
%:%1258=533%:%
%:%1259=534%:%
%:%1260=534%:%
%:%1261=535%:%
%:%1262=535%:%
%:%1263=536%:%
%:%1264=536%:%
%:%1265=536%:%
%:%1266=537%:%
%:%1267=537%:%
%:%1268=538%:%
%:%1269=538%:%
%:%1270=539%:%
%:%1271=539%:%
%:%1272=540%:%
%:%1273=540%:%
%:%1274=541%:%
%:%1275=541%:%
%:%1276=542%:%
%:%1277=542%:%
%:%1278=543%:%
%:%1279=543%:%
%:%1280=543%:%
%:%1281=544%:%
%:%1282=544%:%
%:%1283=545%:%
%:%1284=545%:%
%:%1285=546%:%
%:%1286=546%:%
%:%1287=547%:%
%:%1288=547%:%
%:%1289=548%:%
%:%1290=548%:%
%:%1291=549%:%
%:%1292=549%:%
%:%1293=549%:%
%:%1294=550%:%
%:%1295=550%:%
%:%1296=551%:%
%:%1297=551%:%
%:%1298=552%:%
%:%1299=552%:%
%:%1300=552%:%
%:%1301=553%:%
%:%1302=553%:%
%:%1303=554%:%
%:%1304=554%:%
%:%1305=555%:%
%:%1306=555%:%
%:%1307=556%:%
%:%1308=556%:%
%:%1309=556%:%
%:%1310=557%:%
%:%1311=557%:%
%:%1312=558%:%
%:%1313=558%:%
%:%1314=559%:%
%:%1315=559%:%
%:%1316=560%:%
%:%1317=560%:%
%:%1318=561%:%
%:%1319=561%:%
%:%1320=562%:%
%:%1321=562%:%
%:%1322=563%:%
%:%1323=563%:%
%:%1324=564%:%
%:%1325=564%:%
%:%1326=565%:%
%:%1327=565%:%
%:%1328=566%:%
%:%1329=566%:%
%:%1330=567%:%
%:%1331=567%:%
%:%1332=568%:%
%:%1333=568%:%
%:%1334=568%:%
%:%1335=569%:%
%:%1336=569%:%
%:%1337=570%:%
%:%1338=570%:%
%:%1339=570%:%
%:%1340=571%:%
%:%1341=571%:%
%:%1342=571%:%
%:%1343=572%:%
%:%1344=572%:%
%:%1345=573%:%
%:%1346=573%:%
%:%1347=573%:%
%:%1348=574%:%
%:%1349=574%:%
%:%1350=575%:%
%:%1351=575%:%
%:%1352=576%:%
%:%1353=576%:%
%:%1354=576%:%
%:%1355=577%:%
%:%1356=577%:%
%:%1357=578%:%
%:%1358=578%:%
%:%1359=578%:%
%:%1360=579%:%
%:%1361=579%:%
%:%1362=580%:%
%:%1363=580%:%
%:%1364=581%:%
%:%1365=581%:%
%:%1366=581%:%
%:%1367=582%:%
%:%1368=582%:%
%:%1369=583%:%
%:%1370=583%:%
%:%1371=584%:%
%:%1372=584%:%
%:%1373=585%:%
%:%1374=585%:%
%:%1375=585%:%
%:%1376=586%:%
%:%1377=586%:%
%:%1378=587%:%
%:%1379=587%:%
%:%1380=588%:%
%:%1381=589%:%
%:%1382=589%:%
%:%1383=589%:%
%:%1386=592%:%
%:%1387=592%:%
%:%1388=593%:%
%:%1389=594%:%
%:%1390=594%:%
%:%1391=595%:%
%:%1392=595%:%
%:%1393=596%:%
%:%1394=596%:%
%:%1395=597%:%
%:%1396=597%:%
%:%1397=598%:%
%:%1398=598%:%
%:%1399=599%:%
%:%1400=599%:%
%:%1401=600%:%
%:%1402=600%:%
%:%1403=601%:%
%:%1404=601%:%
%:%1405=602%:%
%:%1406=602%:%
%:%1407=603%:%
%:%1408=603%:%
%:%1409=603%:%
%:%1410=604%:%
%:%1411=604%:%
%:%1412=605%:%
%:%1413=605%:%
%:%1414=605%:%
%:%1415=606%:%
%:%1416=606%:%
%:%1417=607%:%
%:%1418=607%:%
%:%1419=608%:%
%:%1420=608%:%
%:%1421=609%:%
%:%1422=609%:%
%:%1423=610%:%
%:%1424=611%:%
%:%1425=611%:%
%:%1426=612%:%
%:%1427=612%:%
%:%1428=613%:%
%:%1429=613%:%
%:%1430=614%:%
%:%1431=614%:%
%:%1432=615%:%
%:%1433=615%:%
%:%1434=616%:%
%:%1435=616%:%
%:%1436=617%:%
%:%1437=617%:%
%:%1438=618%:%
%:%1439=618%:%
%:%1440=619%:%
%:%1441=619%:%
%:%1442=620%:%
%:%1443=620%:%
%:%1444=621%:%
%:%1445=621%:%
%:%1446=622%:%
%:%1447=622%:%
%:%1448=623%:%
%:%1449=623%:%
%:%1450=624%:%
%:%1451=624%:%
%:%1452=625%:%
%:%1453=625%:%
%:%1454=625%:%
%:%1455=625%:%
%:%1456=626%:%
%:%1462=626%:%
%:%1465=627%:%
%:%1466=628%:%
%:%1467=628%:%
%:%1468=629%:%
%:%1469=630%:%
%:%1470=631%:%
%:%1471=632%:%
%:%1478=633%:%
%:%1479=633%:%
%:%1480=634%:%
%:%1481=634%:%
%:%1482=635%:%
%:%1483=635%:%
%:%1484=636%:%
%:%1485=636%:%
%:%1486=637%:%
%:%1487=637%:%
%:%1488=637%:%
%:%1489=638%:%
%:%1490=638%:%
%:%1491=639%:%
%:%1492=639%:%
%:%1493=640%:%
%:%1494=640%:%
%:%1495=641%:%
%:%1496=641%:%
%:%1497=642%:%
%:%1498=642%:%
%:%1499=642%:%
%:%1500=643%:%
%:%1506=643%:%
%:%1509=644%:%
%:%1510=645%:%
%:%1511=645%:%
%:%1512=646%:%
%:%1513=647%:%
%:%1514=648%:%
%:%1515=649%:%
%:%1516=650%:%
%:%1523=651%:%
%:%1524=651%:%
%:%1525=652%:%
%:%1526=652%:%
%:%1527=653%:%
%:%1528=653%:%
%:%1529=654%:%
%:%1530=654%:%
%:%1531=655%:%
%:%1532=655%:%
%:%1533=655%:%
%:%1534=656%:%
%:%1535=656%:%
%:%1536=657%:%
%:%1537=657%:%
%:%1538=658%:%
%:%1539=658%:%
%:%1540=659%:%
%:%1541=659%:%
%:%1542=660%:%
%:%1543=660%:%
%:%1544=661%:%
%:%1545=661%:%
%:%1546=661%:%
%:%1547=662%:%
%:%1548=662%:%
%:%1549=663%:%
%:%1550=663%:%
%:%1551=664%:%
%:%1552=664%:%
%:%1553=665%:%
%:%1554=665%:%
%:%1555=666%:%
%:%1556=666%:%
%:%1557=667%:%
%:%1558=667%:%
%:%1559=668%:%
%:%1560=668%:%
%:%1561=669%:%
%:%1562=669%:%
%:%1563=670%:%
%:%1564=670%:%
%:%1565=670%:%
%:%1566=671%:%
%:%1567=671%:%
%:%1568=672%:%
%:%1569=672%:%
%:%1570=673%:%
%:%1571=673%:%
%:%1572=674%:%
%:%1573=674%:%
%:%1574=674%:%
%:%1575=675%:%
%:%1581=675%:%
%:%1584=676%:%
%:%1585=677%:%
%:%1586=677%:%
%:%1587=678%:%
%:%1588=679%:%
%:%1589=680%:%
%:%1590=681%:%
%:%1591=682%:%
%:%1594=683%:%
%:%1598=683%:%
%:%1599=683%:%
%:%1600=683%:%
%:%1605=683%:%
%:%1608=684%:%
%:%1609=685%:%
%:%1610=685%:%
%:%1611=686%:%
%:%1612=687%:%
%:%1613=688%:%
%:%1614=689%:%
%:%1615=690%:%
%:%1622=691%:%
%:%1623=691%:%
%:%1624=692%:%
%:%1625=693%:%
%:%1626=693%:%
%:%1627=694%:%
%:%1628=694%:%
%:%1629=695%:%
%:%1630=696%:%
%:%1631=696%:%
%:%1632=697%:%
%:%1633=697%:%
%:%1634=698%:%
%:%1635=698%:%
%:%1636=699%:%
%:%1637=699%:%
%:%1638=700%:%
%:%1639=700%:%
%:%1640=701%:%
%:%1641=701%:%
%:%1642=702%:%
%:%1643=702%:%
%:%1644=703%:%
%:%1645=703%:%
%:%1646=704%:%
%:%1647=705%:%
%:%1648=705%:%
%:%1649=706%:%
%:%1650=706%:%
%:%1651=706%:%
%:%1652=707%:%
%:%1653=707%:%
%:%1654=708%:%
%:%1655=708%:%
%:%1656=709%:%
%:%1657=709%:%
%:%1658=710%:%
%:%1659=710%:%
%:%1660=710%:%
%:%1661=711%:%
%:%1662=711%:%
%:%1663=711%:%
%:%1664=712%:%
%:%1665=712%:%
%:%1666=713%:%
%:%1667=713%:%
%:%1668=714%:%
%:%1669=714%:%
%:%1670=714%:%
%:%1671=715%:%
%:%1672=715%:%
%:%1673=716%:%
%:%1674=716%:%
%:%1675=716%:%
%:%1676=717%:%
%:%1677=717%:%
%:%1678=718%:%
%:%1679=718%:%
%:%1680=719%:%
%:%1681=719%:%
%:%1682=719%:%
%:%1683=720%:%
%:%1684=720%:%
%:%1685=721%:%
%:%1686=721%:%
%:%1687=722%:%
%:%1688=722%:%
%:%1689=723%:%
%:%1690=723%:%
%:%1691=724%:%
%:%1692=724%:%
%:%1693=725%:%
%:%1694=725%:%
%:%1695=726%:%
%:%1696=726%:%
%:%1697=727%:%
%:%1698=727%:%
%:%1699=728%:%
%:%1700=728%:%
%:%1701=729%:%
%:%1702=730%:%
%:%1703=730%:%
%:%1704=731%:%
%:%1705=731%:%
%:%1706=732%:%
%:%1707=732%:%
%:%1708=733%:%
%:%1709=733%:%
%:%1710=733%:%
%:%1711=734%:%
%:%1712=734%:%
%:%1713=735%:%
%:%1714=735%:%
%:%1715=736%:%
%:%1716=736%:%
%:%1717=737%:%
%:%1718=737%:%
%:%1719=738%:%
%:%1720=738%:%
%:%1721=739%:%
%:%1722=739%:%
%:%1723=740%:%
%:%1724=740%:%
%:%1725=741%:%
%:%1726=741%:%
%:%1727=742%:%
%:%1728=742%:%
%:%1729=743%:%
%:%1730=743%:%
%:%1731=744%:%
%:%1732=744%:%
%:%1733=745%:%
%:%1734=745%:%
%:%1735=746%:%
%:%1736=746%:%
%:%1737=746%:%
%:%1738=747%:%
%:%1739=747%:%
%:%1740=748%:%
%:%1741=748%:%
%:%1742=749%:%
%:%1743=749%:%
%:%1744=749%:%
%:%1745=750%:%
%:%1746=750%:%
%:%1747=751%:%
%:%1748=751%:%
%:%1749=751%:%
%:%1750=752%:%
%:%1751=752%:%
%:%1752=753%:%
%:%1753=753%:%
%:%1754=754%:%
%:%1755=754%:%
%:%1756=755%:%
%:%1757=755%:%
%:%1758=756%:%
%:%1759=756%:%
%:%1760=757%:%
%:%1761=757%:%
%:%1762=758%:%
%:%1763=758%:%
%:%1764=759%:%
%:%1765=759%:%
%:%1766=760%:%
%:%1767=760%:%
%:%1768=761%:%
%:%1769=761%:%
%:%1770=761%:%
%:%1771=761%:%
%:%1772=762%:%
%:%1773=762%:%
%:%1774=763%:%
%:%1775=763%:%
%:%1776=764%:%
%:%1777=764%:%
%:%1778=765%:%
%:%1779=765%:%
%:%1780=765%:%
%:%1781=765%:%
%:%1782=766%:%
%:%1783=766%:%
%:%1784=767%:%
%:%1785=767%:%
%:%1786=768%:%
%:%1787=768%:%
%:%1788=769%:%
%:%1789=769%:%
%:%1790=770%:%
%:%1791=770%:%
%:%1792=771%:%
%:%1793=771%:%
%:%1794=772%:%
%:%1795=772%:%
%:%1796=773%:%
%:%1797=773%:%
%:%1798=774%:%
%:%1799=774%:%
%:%1800=774%:%
%:%1801=775%:%
%:%1802=775%:%
%:%1803=776%:%
%:%1804=776%:%
%:%1805=777%:%
%:%1806=777%:%
%:%1807=778%:%
%:%1808=778%:%
%:%1809=779%:%
%:%1810=779%:%
%:%1811=780%:%
%:%1812=780%:%
%:%1813=781%:%
%:%1814=781%:%
%:%1815=782%:%
%:%1816=782%:%
%:%1817=783%:%
%:%1818=783%:%
%:%1819=784%:%
%:%1820=784%:%
%:%1821=785%:%
%:%1822=785%:%
%:%1823=785%:%
%:%1824=786%:%
%:%1825=786%:%
%:%1826=787%:%
%:%1832=787%:%
%:%1835=788%:%
%:%1836=789%:%
%:%1837=789%:%
%:%1838=790%:%
%:%1839=791%:%
%:%1840=792%:%
%:%1841=793%:%
%:%1842=794%:%
%:%1843=795%:%
%:%1844=796%:%
%:%1845=797%:%
%:%1848=798%:%
%:%1853=799%:%
%:%1854=799%:%
%:%1855=800%:%
%:%1856=800%:%
%:%1857=801%:%
%:%1858=801%:%
%:%1859=802%:%
%:%1860=802%:%
%:%1861=803%:%
%:%1862=803%:%
%:%1863=804%:%
%:%1864=804%:%
%:%1865=805%:%
%:%1866=805%:%
%:%1867=806%:%
%:%1868=806%:%
%:%1869=807%:%
%:%1870=807%:%
%:%1871=808%:%
%:%1872=809%:%
%:%1873=809%:%
%:%1874=810%:%
%:%1875=810%:%
%:%1876=811%:%
%:%1877=811%:%
%:%1878=812%:%
%:%1879=812%:%
%:%1880=813%:%
%:%1881=813%:%
%:%1882=813%:%
%:%1883=814%:%
%:%1884=814%:%
%:%1885=815%:%
%:%1886=815%:%
%:%1887=816%:%
%:%1888=816%:%
%:%1889=817%:%
%:%1890=817%:%
%:%1891=818%:%
%:%1892=818%:%
%:%1893=819%:%
%:%1894=819%:%
%:%1895=820%:%
%:%1896=820%:%
%:%1897=821%:%
%:%1898=821%:%
%:%1899=822%:%
%:%1900=822%:%
%:%1901=823%:%
%:%1902=824%:%
%:%1903=824%:%
%:%1904=825%:%
%:%1905=825%:%
%:%1906=826%:%
%:%1907=826%:%
%:%1908=827%:%
%:%1909=827%:%
%:%1910=827%:%
%:%1911=828%:%
%:%1912=828%:%
%:%1913=829%:%
%:%1914=829%:%
%:%1915=829%:%
%:%1916=830%:%
%:%1917=830%:%
%:%1918=830%:%
%:%1919=831%:%
%:%1920=831%:%
%:%1921=832%:%
%:%1922=832%:%
%:%1923=833%:%
%:%1924=833%:%
%:%1925=833%:%
%:%1926=834%:%
%:%1927=834%:%
%:%1928=835%:%
%:%1929=835%:%
%:%1930=835%:%
%:%1931=836%:%
%:%1932=836%:%
%:%1933=837%:%
%:%1934=837%:%
%:%1935=838%:%
%:%1936=838%:%
%:%1937=839%:%
%:%1938=839%:%
%:%1939=840%:%
%:%1940=840%:%
%:%1941=841%:%
%:%1942=841%:%
%:%1943=842%:%
%:%1949=842%:%
%:%1952=843%:%
%:%1953=844%:%
%:%1954=844%:%
%:%1955=845%:%
%:%1956=846%:%
%:%1959=849%:%
%:%1960=850%:%
%:%1961=851%:%
%:%1962=851%:%
%:%1963=852%:%
%:%1966=853%:%
%:%1970=853%:%
%:%1971=853%:%
%:%1972=854%:%
%:%1973=854%:%
%:%1978=854%:%
%:%1981=855%:%
%:%1982=856%:%
%:%1983=856%:%
%:%1984=857%:%
%:%1986=859%:%
%:%1989=860%:%
%:%1993=860%:%
%:%1994=860%:%
%:%1995=861%:%
%:%1996=861%:%
%:%2001=861%:%
%:%2004=862%:%
%:%2005=863%:%
%:%2006=863%:%
%:%2007=864%:%
%:%2008=865%:%
%:%2011=866%:%
%:%2015=866%:%
%:%2016=866%:%
%:%2017=867%:%
%:%2018=867%:%
%:%2019=868%:%
%:%2020=868%:%
%:%2021=869%:%
%:%2022=869%:%
%:%2023=869%:%
%:%2024=869%:%
%:%2025=869%:%
%:%2026=870%:%
%:%2027=870%:%
%:%2028=871%:%
%:%2029=871%:%
%:%2030=872%:%
%:%2031=872%:%
%:%2032=872%:%
%:%2033=873%:%
%:%2034=873%:%
%:%2035=874%:%
%:%2036=874%:%
%:%2037=875%:%
%:%2038=875%:%
%:%2039=876%:%
%:%2040=876%:%
%:%2041=877%:%
%:%2042=877%:%
%:%2043=878%:%
%:%2044=878%:%
%:%2045=879%:%
%:%2046=879%:%
%:%2047=880%:%
%:%2048=880%:%
%:%2049=881%:%
%:%2050=881%:%
%:%2051=882%:%
%:%2052=882%:%
%:%2053=883%:%
%:%2054=883%:%
%:%2055=884%:%
%:%2056=884%:%
%:%2057=885%:%
%:%2063=885%:%
%:%2066=886%:%
%:%2067=887%:%
%:%2068=887%:%
%:%2069=888%:%
%:%2070=889%:%
%:%2073=890%:%
%:%2074=891%:%
%:%2078=891%:%
%:%2079=891%:%
%:%2080=892%:%
%:%2081=892%:%
%:%2082=893%:%
%:%2083=893%:%
%:%2084=894%:%
%:%2085=894%:%
%:%2086=895%:%
%:%2087=895%:%
%:%2088=896%:%
%:%2089=896%:%
%:%2090=897%:%
%:%2091=897%:%
%:%2092=898%:%
%:%2093=898%:%
%:%2094=899%:%
%:%2095=899%:%
%:%2096=900%:%
%:%2097=900%:%
%:%2098=901%:%
%:%2099=901%:%
%:%2100=902%:%
%:%2101=902%:%
%:%2102=903%:%
%:%2103=903%:%
%:%2104=904%:%
%:%2105=904%:%
%:%2106=905%:%
%:%2107=905%:%
%:%2108=906%:%
%:%2109=906%:%
%:%2110=907%:%
%:%2111=907%:%
%:%2112=908%:%
%:%2113=908%:%
%:%2114=909%:%
%:%2115=909%:%
%:%2120=909%:%
%:%2123=910%:%
%:%2124=911%:%
%:%2125=911%:%
%:%2126=912%:%
%:%2127=913%:%
%:%2128=914%:%
%:%2129=915%:%
%:%2136=916%:%
%:%2137=916%:%
%:%2138=917%:%
%:%2139=918%:%
%:%2140=918%:%
%:%2141=919%:%
%:%2142=919%:%
%:%2143=920%:%
%:%2144=920%:%
%:%2145=921%:%
%:%2146=921%:%
%:%2147=922%:%
%:%2148=922%:%
%:%2149=923%:%
%:%2150=923%:%
%:%2151=924%:%
%:%2152=924%:%
%:%2153=925%:%
%:%2154=925%:%
%:%2155=926%:%
%:%2156=926%:%
%:%2157=927%:%
%:%2158=927%:%
%:%2159=928%:%
%:%2160=928%:%
%:%2161=929%:%
%:%2162=929%:%
%:%2163=930%:%
%:%2164=931%:%
%:%2165=931%:%
%:%2166=932%:%
%:%2167=933%:%
%:%2168=933%:%
%:%2171=936%:%
%:%2172=937%:%
%:%2173=937%:%
%:%2174=937%:%
%:%2175=937%:%
%:%2176=938%:%
%:%2177=938%:%
%:%2178=939%:%
%:%2179=939%:%
%:%2180=940%:%
%:%2181=940%:%
%:%2182=941%:%
%:%2183=941%:%
%:%2184=942%:%
%:%2185=942%:%
%:%2186=943%:%
%:%2187=944%:%
%:%2188=944%:%
%:%2189=945%:%
%:%2190=945%:%
%:%2191=946%:%
%:%2192=946%:%
%:%2193=947%:%
%:%2194=947%:%
%:%2195=948%:%
%:%2196=948%:%
%:%2197=949%:%
%:%2198=949%:%
%:%2199=950%:%
%:%2200=950%:%
%:%2201=951%:%
%:%2202=951%:%
%:%2203=952%:%
%:%2204=952%:%
%:%2205=953%:%
%:%2206=953%:%
%:%2207=954%:%
%:%2208=954%:%
%:%2209=955%:%
%:%2210=955%:%
%:%2211=956%:%
%:%2212=956%:%
%:%2213=957%:%
%:%2214=957%:%
%:%2215=958%:%
%:%2216=958%:%
%:%2217=959%:%
%:%2218=959%:%
%:%2219=960%:%
%:%2220=960%:%
%:%2221=961%:%
%:%2222=961%:%
%:%2223=962%:%
%:%2224=962%:%
%:%2225=963%:%
%:%2226=963%:%
%:%2227=964%:%
%:%2228=964%:%
%:%2229=965%:%
%:%2230=965%:%
%:%2231=966%:%
%:%2232=966%:%
%:%2233=967%:%
%:%2234=967%:%
%:%2235=968%:%
%:%2236=968%:%
%:%2237=969%:%
%:%2238=969%:%
%:%2239=970%:%
%:%2240=970%:%
%:%2241=971%:%
%:%2242=971%:%
%:%2243=972%:%
%:%2244=972%:%
%:%2245=973%:%
%:%2246=973%:%
%:%2247=974%:%
%:%2248=974%:%
%:%2249=975%:%
%:%2250=975%:%
%:%2251=976%:%
%:%2252=977%:%
%:%2253=977%:%
%:%2254=978%:%
%:%2255=979%:%
%:%2256=979%:%
%:%2257=980%:%
%:%2258=980%:%
%:%2259=981%:%
%:%2260=981%:%
%:%2261=982%:%
%:%2262=982%:%
%:%2263=983%:%
%:%2264=983%:%
%:%2265=984%:%
%:%2266=984%:%
%:%2267=985%:%
%:%2268=985%:%
%:%2269=986%:%
%:%2270=986%:%
%:%2271=986%:%
%:%2272=987%:%
%:%2273=988%:%
%:%2274=988%:%
%:%2275=989%:%
%:%2276=989%:%
%:%2277=990%:%
%:%2278=990%:%
%:%2279=991%:%
%:%2280=991%:%
%:%2281=992%:%
%:%2282=992%:%
%:%2283=993%:%
%:%2284=993%:%
%:%2285=994%:%
%:%2286=994%:%
%:%2287=995%:%
%:%2288=995%:%
%:%2289=996%:%
%:%2290=996%:%
%:%2291=997%:%
%:%2292=997%:%
%:%2293=998%:%
%:%2294=998%:%
%:%2295=999%:%
%:%2296=999%:%
%:%2297=1000%:%
%:%2298=1000%:%
%:%2299=1001%:%
%:%2300=1001%:%
%:%2301=1002%:%
%:%2302=1002%:%
%:%2303=1003%:%
%:%2304=1003%:%
%:%2305=1004%:%
%:%2306=1004%:%
%:%2307=1005%:%
%:%2308=1005%:%
%:%2309=1006%:%
%:%2310=1006%:%
%:%2311=1007%:%
%:%2312=1007%:%
%:%2313=1008%:%
%:%2314=1008%:%
%:%2315=1009%:%
%:%2316=1009%:%
%:%2317=1010%:%
%:%2318=1010%:%
%:%2319=1011%:%
%:%2320=1011%:%
%:%2321=1012%:%
%:%2322=1012%:%
%:%2323=1013%:%
%:%2324=1013%:%
%:%2325=1014%:%
%:%2326=1014%:%
%:%2327=1015%:%
%:%2328=1015%:%
%:%2329=1016%:%
%:%2330=1016%:%
%:%2331=1017%:%
%:%2332=1017%:%
%:%2333=1018%:%
%:%2334=1018%:%
%:%2335=1019%:%
%:%2336=1019%:%
%:%2337=1020%:%
%:%2338=1020%:%
%:%2339=1021%:%
%:%2340=1021%:%
%:%2341=1022%:%
%:%2342=1022%:%
%:%2343=1022%:%
%:%2344=1023%:%
%:%2345=1023%:%
%:%2346=1024%:%
%:%2347=1024%:%
%:%2348=1025%:%
%:%2354=1025%:%
%:%2357=1026%:%
%:%2358=1027%:%
%:%2359=1027%:%
%:%2360=1028%:%
%:%2361=1029%:%
%:%2362=1030%:%
%:%2363=1031%:%
%:%2364=1032%:%
%:%2367=1033%:%
%:%2371=1033%:%
%:%2372=1033%:%
%:%2373=1034%:%
%:%2374=1034%:%
%:%2375=1035%:%
%:%2376=1035%:%
%:%2377=1036%:%
%:%2378=1036%:%
%:%2379=1036%:%
%:%2380=1036%:%
%:%2381=1037%:%
%:%2382=1038%:%
%:%2383=1038%:%
%:%2384=1039%:%
%:%2385=1039%:%
%:%2386=1040%:%
%:%2387=1040%:%
%:%2388=1041%:%
%:%2389=1041%:%
%:%2390=1042%:%
%:%2391=1043%:%
%:%2392=1043%:%
%:%2393=1044%:%
%:%2394=1044%:%
%:%2395=1045%:%
%:%2396=1045%:%
%:%2397=1046%:%
%:%2398=1046%:%
%:%2399=1047%:%
%:%2400=1047%:%
%:%2401=1048%:%
%:%2402=1049%:%
%:%2403=1049%:%
%:%2404=1050%:%
%:%2405=1050%:%
%:%2406=1051%:%
%:%2407=1051%:%
%:%2408=1052%:%
%:%2409=1052%:%
%:%2410=1053%:%
%:%2411=1054%:%
%:%2412=1054%:%
%:%2413=1055%:%
%:%2414=1055%:%
%:%2415=1056%:%
%:%2416=1056%:%
%:%2417=1057%:%
%:%2418=1057%:%
%:%2419=1058%:%
%:%2420=1059%:%
%:%2421=1059%:%
%:%2422=1060%:%
%:%2423=1061%:%
%:%2424=1061%:%
%:%2425=1062%:%
%:%2426=1062%:%
%:%2427=1063%:%
%:%2428=1063%:%
%:%2429=1064%:%
%:%2430=1064%:%
%:%2431=1064%:%
%:%2432=1065%:%
%:%2433=1065%:%
%:%2434=1066%:%
%:%2435=1066%:%
%:%2436=1067%:%
%:%2437=1067%:%
%:%2438=1068%:%
%:%2439=1068%:%
%:%2440=1069%:%
%:%2441=1069%:%
%:%2442=1070%:%
%:%2443=1070%:%
%:%2444=1071%:%
%:%2445=1071%:%
%:%2446=1072%:%
%:%2447=1072%:%
%:%2448=1073%:%
%:%2449=1073%:%
%:%2450=1074%:%
%:%2451=1075%:%
%:%2452=1075%:%
%:%2453=1076%:%
%:%2454=1076%:%
%:%2455=1077%:%
%:%2456=1078%:%
%:%2457=1078%:%
%:%2458=1079%:%
%:%2459=1079%:%
%:%2460=1080%:%
%:%2461=1080%:%
%:%2462=1080%:%
%:%2463=1080%:%
%:%2464=1081%:%
%:%2465=1082%:%
%:%2466=1082%:%
%:%2467=1083%:%
%:%2468=1083%:%
%:%2469=1084%:%
%:%2470=1084%:%
%:%2471=1085%:%
%:%2472=1085%:%
%:%2473=1086%:%
%:%2474=1087%:%
%:%2475=1087%:%
%:%2476=1088%:%
%:%2477=1088%:%
%:%2478=1089%:%
%:%2479=1089%:%
%:%2480=1090%:%
%:%2481=1090%:%
%:%2482=1091%:%
%:%2483=1092%:%
%:%2484=1092%:%
%:%2485=1093%:%
%:%2486=1093%:%
%:%2487=1094%:%
%:%2488=1094%:%
%:%2489=1095%:%
%:%2490=1095%:%
%:%2491=1096%:%
%:%2492=1096%:%
%:%2493=1097%:%
%:%2494=1098%:%
%:%2495=1098%:%
%:%2496=1099%:%
%:%2497=1099%:%
%:%2498=1100%:%
%:%2499=1100%:%
%:%2500=1101%:%
%:%2501=1101%:%
%:%2502=1102%:%
%:%2503=1103%:%
%:%2504=1103%:%
%:%2505=1104%:%
%:%2506=1105%:%
%:%2507=1105%:%
%:%2508=1106%:%
%:%2509=1106%:%
%:%2510=1107%:%
%:%2511=1107%:%
%:%2512=1108%:%
%:%2513=1108%:%
%:%2514=1108%:%
%:%2515=1109%:%
%:%2516=1109%:%
%:%2517=1110%:%
%:%2518=1110%:%
%:%2519=1111%:%
%:%2520=1111%:%
%:%2521=1112%:%
%:%2522=1112%:%
%:%2523=1113%:%
%:%2524=1113%:%
%:%2525=1114%:%
%:%2526=1114%:%
%:%2527=1115%:%
%:%2528=1115%:%
%:%2529=1116%:%
%:%2530=1116%:%
%:%2531=1117%:%
%:%2532=1117%:%
%:%2533=1118%:%
%:%2534=1119%:%
%:%2535=1119%:%
%:%2536=1120%:%
%:%2537=1120%:%
%:%2538=1121%:%
%:%2539=1121%:%
%:%2540=1122%:%
%:%2541=1122%:%
%:%2542=1123%:%
%:%2543=1124%:%
%:%2544=1124%:%
%:%2545=1125%:%
%:%2546=1125%:%
%:%2547=1126%:%
%:%2548=1126%:%
%:%2549=1127%:%
%:%2550=1127%:%
%:%2551=1128%:%
%:%2552=1128%:%
%:%2553=1129%:%
%:%2554=1129%:%
%:%2555=1130%:%
%:%2556=1130%:%
%:%2557=1131%:%
%:%2558=1131%:%
%:%2559=1132%:%
%:%2560=1132%:%
%:%2561=1133%:%
%:%2562=1133%:%
%:%2563=1134%:%
%:%2564=1134%:%
%:%2565=1135%:%
%:%2566=1135%:%
%:%2567=1136%:%
%:%2568=1136%:%
%:%2569=1137%:%
%:%2570=1137%:%
%:%2571=1138%:%
%:%2572=1138%:%
%:%2573=1139%:%
%:%2574=1139%:%
%:%2575=1140%:%
%:%2576=1140%:%
%:%2577=1141%:%
%:%2578=1142%:%
%:%2579=1142%:%
%:%2580=1143%:%
%:%2581=1143%:%
%:%2582=1144%:%
%:%2583=1144%:%
%:%2584=1145%:%
%:%2585=1145%:%
%:%2586=1146%:%
%:%2587=1146%:%
%:%2588=1147%:%
%:%2589=1147%:%
%:%2590=1148%:%
%:%2591=1148%:%
%:%2592=1149%:%
%:%2593=1149%:%
%:%2594=1150%:%
%:%2595=1150%:%
%:%2596=1151%:%
%:%2597=1151%:%
%:%2598=1152%:%
%:%2599=1152%:%
%:%2600=1153%:%
%:%2601=1153%:%
%:%2602=1154%:%
%:%2603=1154%:%
%:%2604=1155%:%
%:%2605=1155%:%
%:%2606=1156%:%
%:%2607=1156%:%
%:%2608=1157%:%
%:%2609=1157%:%
%:%2610=1158%:%
%:%2611=1158%:%
%:%2612=1159%:%
%:%2613=1159%:%
%:%2614=1160%:%
%:%2615=1161%:%
%:%2616=1161%:%
%:%2617=1162%:%
%:%2618=1162%:%
%:%2619=1163%:%
%:%2620=1163%:%
%:%2621=1164%:%
%:%2622=1164%:%
%:%2623=1165%:%
%:%2624=1165%:%
%:%2625=1166%:%
%:%2626=1166%:%
%:%2627=1167%:%
%:%2628=1167%:%
%:%2629=1168%:%
%:%2630=1168%:%
%:%2631=1169%:%
%:%2632=1170%:%
%:%2633=1170%:%
%:%2634=1171%:%
%:%2635=1172%:%
%:%2636=1172%:%
%:%2637=1173%:%
%:%2638=1173%:%
%:%2639=1174%:%
%:%2640=1175%:%
%:%2641=1175%:%
%:%2642=1176%:%
%:%2643=1176%:%
%:%2644=1177%:%
%:%2645=1177%:%
%:%2646=1178%:%
%:%2647=1178%:%
%:%2648=1179%:%
%:%2649=1179%:%
%:%2650=1180%:%
%:%2651=1180%:%
%:%2652=1181%:%
%:%2653=1181%:%
%:%2654=1181%:%
%:%2655=1182%:%
%:%2656=1182%:%
%:%2657=1183%:%
%:%2658=1183%:%
%:%2659=1184%:%
%:%2660=1184%:%
%:%2661=1184%:%
%:%2662=1185%:%
%:%2663=1185%:%
%:%2664=1185%:%
%:%2665=1186%:%
%:%2666=1186%:%
%:%2667=1187%:%
%:%2668=1187%:%
%:%2669=1188%:%
%:%2670=1188%:%
%:%2671=1189%:%
%:%2672=1189%:%
%:%2673=1190%:%
%:%2674=1190%:%
%:%2675=1191%:%
%:%2676=1191%:%
%:%2677=1192%:%
%:%2678=1192%:%
%:%2679=1192%:%
%:%2680=1193%:%
%:%2681=1193%:%
%:%2682=1194%:%
%:%2683=1194%:%
%:%2684=1195%:%
%:%2685=1195%:%
%:%2686=1196%:%
%:%2687=1196%:%
%:%2688=1197%:%
%:%2689=1197%:%
%:%2690=1198%:%
%:%2691=1198%:%
%:%2692=1199%:%
%:%2693=1199%:%
%:%2694=1200%:%
%:%2695=1200%:%
%:%2696=1201%:%
%:%2697=1202%:%
%:%2698=1202%:%
%:%2699=1203%:%
%:%2700=1203%:%
%:%2701=1204%:%
%:%2702=1204%:%
%:%2703=1205%:%
%:%2704=1205%:%
%:%2705=1206%:%
%:%2706=1206%:%
%:%2707=1207%:%
%:%2708=1207%:%
%:%2709=1208%:%
%:%2710=1208%:%
%:%2711=1209%:%
%:%2712=1209%:%
%:%2713=1210%:%
%:%2714=1210%:%
%:%2715=1210%:%
%:%2716=1211%:%
%:%2717=1211%:%
%:%2718=1212%:%
%:%2719=1212%:%
%:%2720=1212%:%
%:%2721=1212%:%
%:%2722=1213%:%
%:%2723=1213%:%
%:%2724=1214%:%
%:%2725=1214%:%
%:%2726=1214%:%
%:%2727=1215%:%
%:%2728=1216%:%
%:%2729=1216%:%
%:%2730=1217%:%
%:%2731=1217%:%
%:%2732=1218%:%
%:%2733=1218%:%
%:%2734=1218%:%
%:%2735=1218%:%
%:%2736=1218%:%
%:%2737=1219%:%
%:%2738=1219%:%
%:%2739=1220%:%
%:%2740=1220%:%
%:%2741=1220%:%
%:%2742=1220%:%
%:%2743=1221%:%
%:%2744=1221%:%
%:%2745=1222%:%
%:%2746=1222%:%
%:%2747=1223%:%
%:%2748=1223%:%
%:%2749=1224%:%
%:%2750=1224%:%
%:%2751=1225%:%
%:%2752=1225%:%
%:%2753=1226%:%
%:%2754=1226%:%
%:%2755=1227%:%
%:%2756=1227%:%
%:%2757=1228%:%
%:%2758=1228%:%
%:%2759=1229%:%
%:%2760=1229%:%
%:%2761=1230%:%
%:%2762=1230%:%
%:%2763=1230%:%
%:%2764=1230%:%
%:%2765=1230%:%
%:%2766=1231%:%
%:%2767=1231%:%
%:%2768=1232%:%
%:%2769=1232%:%
%:%2770=1233%:%
%:%2771=1234%:%
%:%2772=1234%:%
%:%2773=1235%:%
%:%2774=1236%:%
%:%2775=1236%:%
%:%2776=1237%:%
%:%2777=1237%:%
%:%2778=1238%:%
%:%2779=1238%:%
%:%2780=1239%:%
%:%2781=1239%:%
%:%2782=1239%:%
%:%2783=1239%:%
%:%2784=1239%:%
%:%2785=1240%:%
%:%2786=1240%:%
%:%2787=1240%:%
%:%2788=1240%:%
%:%2789=1241%:%
%:%2790=1241%:%
%:%2791=1241%:%
%:%2792=1242%:%
%:%2793=1242%:%
%:%2794=1243%:%
%:%2795=1243%:%
%:%2796=1244%:%
%:%2797=1244%:%
%:%2798=1245%:%
%:%2799=1245%:%
%:%2800=1246%:%
%:%2801=1246%:%
%:%2802=1247%:%
%:%2803=1247%:%
%:%2804=1248%:%
%:%2805=1248%:%
%:%2806=1249%:%
%:%2807=1249%:%
%:%2808=1250%:%
%:%2809=1250%:%
%:%2810=1251%:%
%:%2811=1251%:%
%:%2812=1251%:%
%:%2813=1252%:%
%:%2814=1252%:%
%:%2815=1253%:%
%:%2816=1253%:%
%:%2817=1254%:%
%:%2818=1254%:%
%:%2819=1255%:%
%:%2820=1255%:%
%:%2821=1256%:%
%:%2822=1256%:%
%:%2823=1257%:%
%:%2824=1257%:%
%:%2825=1258%:%
%:%2826=1258%:%
%:%2827=1259%:%
%:%2828=1259%:%
%:%2829=1260%:%
%:%2830=1260%:%
%:%2831=1261%:%
%:%2832=1261%:%
%:%2833=1262%:%
%:%2834=1262%:%
%:%2835=1263%:%
%:%2836=1263%:%
%:%2837=1264%:%
%:%2838=1264%:%
%:%2839=1265%:%
%:%2840=1265%:%
%:%2841=1266%:%
%:%2842=1266%:%
%:%2843=1267%:%
%:%2844=1267%:%
%:%2845=1267%:%
%:%2846=1268%:%
%:%2847=1268%:%
%:%2848=1269%:%
%:%2849=1269%:%
%:%2850=1269%:%
%:%2851=1269%:%
%:%2852=1270%:%
%:%2853=1270%:%
%:%2854=1271%:%
%:%2855=1271%:%
%:%2856=1272%:%
%:%2857=1272%:%
%:%2858=1273%:%
%:%2859=1273%:%
%:%2860=1274%:%
%:%2861=1274%:%
%:%2862=1275%:%
%:%2863=1275%:%
%:%2864=1275%:%
%:%2865=1276%:%
%:%2866=1276%:%
%:%2867=1277%:%
%:%2868=1277%:%
%:%2869=1278%:%
%:%2870=1278%:%
%:%2871=1279%:%
%:%2872=1279%:%
%:%2873=1280%:%
%:%2874=1280%:%
%:%2875=1281%:%
%:%2876=1281%:%
%:%2877=1282%:%
%:%2878=1283%:%
%:%2879=1283%:%
%:%2880=1284%:%
%:%2881=1284%:%
%:%2882=1284%:%
%:%2883=1285%:%
%:%2884=1285%:%
%:%2885=1285%:%
%:%2886=1286%:%
%:%2887=1286%:%
%:%2888=1287%:%
%:%2889=1287%:%
%:%2890=1288%:%
%:%2891=1288%:%
%:%2892=1289%:%
%:%2893=1289%:%
%:%2894=1290%:%
%:%2895=1290%:%
%:%2896=1291%:%
%:%2897=1291%:%
%:%2898=1291%:%
%:%2899=1292%:%
%:%2900=1292%:%
%:%2901=1293%:%
%:%2902=1293%:%
%:%2903=1294%:%
%:%2904=1294%:%
%:%2905=1295%:%
%:%2906=1295%:%
%:%2907=1296%:%
%:%2908=1296%:%
%:%2909=1297%:%
%:%2910=1297%:%
%:%2911=1298%:%
%:%2912=1298%:%
%:%2913=1299%:%
%:%2914=1299%:%
%:%2915=1299%:%
%:%2916=1300%:%
%:%2917=1300%:%
%:%2918=1301%:%
%:%2919=1301%:%
%:%2920=1302%:%
%:%2921=1302%:%
%:%2922=1303%:%
%:%2923=1303%:%
%:%2924=1304%:%
%:%2925=1304%:%
%:%2926=1305%:%
%:%2927=1305%:%
%:%2928=1305%:%
%:%2929=1306%:%
%:%2930=1306%:%
%:%2931=1306%:%
%:%2932=1307%:%
%:%2933=1308%:%
%:%2934=1308%:%
%:%2935=1309%:%
%:%2936=1309%:%
%:%2937=1310%:%
%:%2938=1310%:%
%:%2939=1311%:%
%:%2940=1311%:%
%:%2941=1312%:%
%:%2942=1312%:%
%:%2943=1313%:%
%:%2944=1313%:%
%:%2945=1314%:%
%:%2946=1314%:%
%:%2947=1315%:%
%:%2948=1315%:%
%:%2949=1316%:%
%:%2950=1316%:%
%:%2951=1317%:%
%:%2952=1317%:%
%:%2953=1317%:%
%:%2954=1318%:%
%:%2955=1318%:%
%:%2956=1319%:%
%:%2957=1319%:%
%:%2958=1320%:%
%:%2964=1320%:%
%:%2967=1321%:%
%:%2968=1322%:%
%:%2969=1322%:%
%:%2970=1323%:%
%:%2971=1324%:%
%:%2972=1325%:%
%:%2973=1326%:%
%:%2974=1327%:%
%:%2981=1328%:%
%:%2982=1328%:%
%:%2983=1329%:%
%:%2984=1329%:%
%:%2985=1330%:%
%:%2986=1330%:%
%:%2987=1331%:%
%:%2988=1331%:%
%:%2989=1332%:%
%:%2990=1332%:%
%:%2991=1333%:%
%:%2992=1333%:%
%:%2993=1334%:%
%:%2994=1334%:%
%:%2995=1335%:%
%:%2996=1335%:%
%:%2997=1336%:%
%:%2998=1336%:%
%:%2999=1337%:%
%:%3000=1338%:%
%:%3001=1338%:%
%:%3002=1339%:%
%:%3003=1339%:%
%:%3004=1340%:%
%:%3005=1340%:%
%:%3006=1341%:%
%:%3007=1341%:%
%:%3008=1342%:%
%:%3009=1343%:%
%:%3010=1343%:%
%:%3011=1344%:%
%:%3012=1344%:%
%:%3013=1345%:%
%:%3014=1345%:%
%:%3015=1346%:%
%:%3016=1346%:%
%:%3017=1347%:%
%:%3018=1347%:%
%:%3019=1348%:%
%:%3020=1348%:%
%:%3021=1349%:%
%:%3022=1349%:%
%:%3023=1349%:%
%:%3024=1350%:%
%:%3025=1351%:%
%:%3026=1351%:%
%:%3027=1351%:%
%:%3028=1352%:%
%:%3029=1352%:%
%:%3030=1353%:%
%:%3031=1353%:%
%:%3032=1353%:%
%:%3033=1353%:%
%:%3034=1354%:%
%:%3035=1354%:%
%:%3036=1354%:%
%:%3037=1355%:%
%:%3038=1355%:%
%:%3039=1356%:%
%:%3040=1356%:%
%:%3041=1357%:%
%:%3042=1357%:%
%:%3043=1358%:%
%:%3044=1359%:%
%:%3045=1359%:%
%:%3046=1360%:%
%:%3047=1360%:%
%:%3048=1361%:%
%:%3049=1361%:%
%:%3050=1362%:%
%:%3051=1362%:%
%:%3052=1363%:%
%:%3053=1363%:%
%:%3054=1364%:%
%:%3055=1364%:%
%:%3056=1364%:%
%:%3057=1365%:%
%:%3058=1365%:%
%:%3059=1365%:%
%:%3060=1366%:%
%:%3061=1366%:%
%:%3062=1366%:%
%:%3063=1367%:%
%:%3064=1367%:%
%:%3065=1368%:%
%:%3066=1368%:%
%:%3067=1369%:%
%:%3068=1369%:%
%:%3069=1369%:%
%:%3070=1370%:%
%:%3071=1370%:%
%:%3072=1371%:%
%:%3073=1371%:%
%:%3074=1371%:%
%:%3075=1372%:%
%:%3076=1372%:%
%:%3077=1373%:%
%:%3078=1373%:%
%:%3079=1374%:%
%:%3080=1374%:%
%:%3081=1375%:%
%:%3082=1375%:%
%:%3083=1376%:%
%:%3084=1376%:%
%:%3085=1377%:%
%:%3086=1377%:%
%:%3087=1378%:%
%:%3088=1378%:%
%:%3089=1379%:%
%:%3090=1379%:%
%:%3091=1380%:%
%:%3092=1380%:%
%:%3093=1381%:%
%:%3094=1381%:%
%:%3095=1382%:%
%:%3096=1382%:%
%:%3097=1383%:%
%:%3098=1383%:%
%:%3099=1384%:%
%:%3100=1384%:%
%:%3101=1385%:%
%:%3102=1385%:%
%:%3103=1386%:%
%:%3104=1386%:%
%:%3105=1386%:%
%:%3106=1387%:%
%:%3107=1387%:%
%:%3108=1388%:%
%:%3109=1388%:%
%:%3110=1389%:%
%:%3111=1389%:%
%:%3112=1390%:%
%:%3113=1390%:%
%:%3114=1391%:%
%:%3115=1391%:%
%:%3116=1392%:%
%:%3117=1392%:%
%:%3118=1393%:%
%:%3124=1393%:%
%:%3127=1394%:%
%:%3128=1395%:%
%:%3129=1396%:%
%:%3130=1396%:%
%:%3137=1397%:%

%
\begin{isabellebody}%
\setisabellecontext{Symmetry{\isacharunderscore}{\kern0pt}Lemma}%
%
\isadelimtheory
%
\endisadelimtheory
%
\isatagtheory
\isacommand{theory}\isamarkupfalse%
\ Symmetry{\isacharunderscore}{\kern0pt}Lemma\isanewline
\ \ \isakeyword{imports}\ \isanewline
\ \ \ \ {\isachardoublequoteopen}Forcing{\isacharslash}{\kern0pt}Forcing{\isacharunderscore}{\kern0pt}Main{\isachardoublequoteclose}\ \isanewline
\ \ \ \ HS{\isacharunderscore}{\kern0pt}Forces\isanewline
\isakeyword{begin}%
\endisatagtheory
{\isafoldtheory}%
%
\isadelimtheory
\ \isanewline
%
\endisadelimtheory
\isanewline
\isacommand{context}\isamarkupfalse%
\ M{\isacharunderscore}{\kern0pt}symmetric{\isacharunderscore}{\kern0pt}system\isanewline
\isakeyword{begin}\isanewline
\isanewline
\isacommand{lemma}\isamarkupfalse%
\ symmetry{\isacharunderscore}{\kern0pt}lemma{\isacharunderscore}{\kern0pt}base\ {\isacharcolon}{\kern0pt}\ \isanewline
\ \ {\isachardoublequoteopen}{\isasymAnd}{\isasympi}\ p\ x\ y{\isachardot}{\kern0pt}\ is{\isacharunderscore}{\kern0pt}P{\isacharunderscore}{\kern0pt}auto{\isacharparenleft}{\kern0pt}{\isasympi}{\isacharparenright}{\kern0pt}\ {\isasymLongrightarrow}\ p\ {\isasymin}\ P\ {\isasymLongrightarrow}\ x\ {\isasymin}\ P{\isacharunderscore}{\kern0pt}names\ {\isasymLongrightarrow}\ y\ {\isasymin}\ P{\isacharunderscore}{\kern0pt}names\ \isanewline
\ \ \ \ {\isasymLongrightarrow}\ {\isacharparenleft}{\kern0pt}forces{\isacharunderscore}{\kern0pt}mem{\isacharparenleft}{\kern0pt}p{\isacharcomma}{\kern0pt}\ x{\isacharcomma}{\kern0pt}\ y{\isacharparenright}{\kern0pt}\ {\isasymlongleftrightarrow}\ forces{\isacharunderscore}{\kern0pt}mem{\isacharparenleft}{\kern0pt}{\isasympi}{\isacharbackquote}{\kern0pt}p{\isacharcomma}{\kern0pt}\ Pn{\isacharunderscore}{\kern0pt}auto{\isacharparenleft}{\kern0pt}{\isasympi}{\isacharparenright}{\kern0pt}{\isacharbackquote}{\kern0pt}x{\isacharcomma}{\kern0pt}\ Pn{\isacharunderscore}{\kern0pt}auto{\isacharparenleft}{\kern0pt}{\isasympi}{\isacharparenright}{\kern0pt}{\isacharbackquote}{\kern0pt}y{\isacharparenright}{\kern0pt}{\isacharparenright}{\kern0pt}\isanewline
\ \ \ \ \ \ {\isasymand}\ {\isacharparenleft}{\kern0pt}forces{\isacharunderscore}{\kern0pt}eq{\isacharparenleft}{\kern0pt}p{\isacharcomma}{\kern0pt}\ x{\isacharcomma}{\kern0pt}\ y{\isacharparenright}{\kern0pt}\ {\isasymlongleftrightarrow}\ forces{\isacharunderscore}{\kern0pt}eq{\isacharparenleft}{\kern0pt}{\isasympi}{\isacharbackquote}{\kern0pt}p{\isacharcomma}{\kern0pt}\ Pn{\isacharunderscore}{\kern0pt}auto{\isacharparenleft}{\kern0pt}{\isasympi}{\isacharparenright}{\kern0pt}{\isacharbackquote}{\kern0pt}x{\isacharcomma}{\kern0pt}\ Pn{\isacharunderscore}{\kern0pt}auto{\isacharparenleft}{\kern0pt}{\isasympi}{\isacharparenright}{\kern0pt}{\isacharbackquote}{\kern0pt}y{\isacharparenright}{\kern0pt}{\isacharparenright}{\kern0pt}{\isachardoublequoteclose}\isanewline
%
\isadelimproof
%
\endisadelimproof
%
\isatagproof
\isacommand{proof}\isamarkupfalse%
\ {\isacharminus}{\kern0pt}\ \isanewline
\ \ \isacommand{fix}\isamarkupfalse%
\ {\isasympi}\ \isanewline
\ \ \isacommand{assume}\isamarkupfalse%
\ piauto\ {\isacharcolon}{\kern0pt}\ {\isachardoublequoteopen}is{\isacharunderscore}{\kern0pt}P{\isacharunderscore}{\kern0pt}auto{\isacharparenleft}{\kern0pt}{\isasympi}{\isacharparenright}{\kern0pt}{\isachardoublequoteclose}\isanewline
\isanewline
\ \ \isacommand{define}\isamarkupfalse%
\ MEM\ \isakeyword{where}\ {\isachardoublequoteopen}MEM\ {\isasymequiv}\ {\isasymlambda}\ p\ x\ y\ {\isachardot}{\kern0pt}\ {\isacharparenleft}{\kern0pt}forces{\isacharunderscore}{\kern0pt}mem{\isacharparenleft}{\kern0pt}p{\isacharcomma}{\kern0pt}\ x{\isacharcomma}{\kern0pt}\ y{\isacharparenright}{\kern0pt}\ {\isasymlongleftrightarrow}\ forces{\isacharunderscore}{\kern0pt}mem{\isacharparenleft}{\kern0pt}{\isasympi}{\isacharbackquote}{\kern0pt}p{\isacharcomma}{\kern0pt}\ Pn{\isacharunderscore}{\kern0pt}auto{\isacharparenleft}{\kern0pt}{\isasympi}{\isacharparenright}{\kern0pt}{\isacharbackquote}{\kern0pt}x{\isacharcomma}{\kern0pt}\ Pn{\isacharunderscore}{\kern0pt}auto{\isacharparenleft}{\kern0pt}{\isasympi}{\isacharparenright}{\kern0pt}{\isacharbackquote}{\kern0pt}y{\isacharparenright}{\kern0pt}{\isacharparenright}{\kern0pt}{\isachardoublequoteclose}\isanewline
\ \ \isacommand{define}\isamarkupfalse%
\ EQ\ \isakeyword{where}\ {\isachardoublequoteopen}EQ\ {\isasymequiv}\ {\isasymlambda}\ p\ x\ y\ {\isachardot}{\kern0pt}\ {\isacharparenleft}{\kern0pt}forces{\isacharunderscore}{\kern0pt}eq{\isacharparenleft}{\kern0pt}p{\isacharcomma}{\kern0pt}\ x{\isacharcomma}{\kern0pt}\ y{\isacharparenright}{\kern0pt}\ {\isasymlongleftrightarrow}\ forces{\isacharunderscore}{\kern0pt}eq{\isacharparenleft}{\kern0pt}{\isasympi}{\isacharbackquote}{\kern0pt}p{\isacharcomma}{\kern0pt}\ Pn{\isacharunderscore}{\kern0pt}auto{\isacharparenleft}{\kern0pt}{\isasympi}{\isacharparenright}{\kern0pt}{\isacharbackquote}{\kern0pt}x{\isacharcomma}{\kern0pt}\ Pn{\isacharunderscore}{\kern0pt}auto{\isacharparenleft}{\kern0pt}{\isasympi}{\isacharparenright}{\kern0pt}{\isacharbackquote}{\kern0pt}y{\isacharparenright}{\kern0pt}{\isacharparenright}{\kern0pt}{\isachardoublequoteclose}\isanewline
\ \ \isacommand{define}\isamarkupfalse%
\ Q\ \isakeyword{where}\ {\isachardoublequoteopen}Q\ {\isasymequiv}\ {\isasymlambda}\ Q\ a\ b{\isachardot}{\kern0pt}\ {\isasymforall}x\ {\isasymin}\ P{\isacharunderscore}{\kern0pt}names{\isachardot}{\kern0pt}\ {\isasymforall}y\ {\isasymin}\ P{\isacharunderscore}{\kern0pt}names{\isachardot}{\kern0pt}\ {\isasymforall}p\ {\isasymin}\ P{\isachardot}{\kern0pt}\ P{\isacharunderscore}{\kern0pt}rank{\isacharparenleft}{\kern0pt}x{\isacharparenright}{\kern0pt}\ {\isasymle}\ a\ {\isasymlongrightarrow}\ P{\isacharunderscore}{\kern0pt}rank{\isacharparenleft}{\kern0pt}y{\isacharparenright}{\kern0pt}\ {\isasymle}\ b\ {\isasymlongrightarrow}\ Q{\isacharparenleft}{\kern0pt}p{\isacharcomma}{\kern0pt}\ x{\isacharcomma}{\kern0pt}\ y{\isacharparenright}{\kern0pt}{\isachardoublequoteclose}\ \isanewline
\isanewline
\ \ \isacommand{have}\isamarkupfalse%
\ MEM{\isacharunderscore}{\kern0pt}step{\isacharcolon}{\kern0pt}\ {\isachardoublequoteopen}{\isasymAnd}a{\isachardot}{\kern0pt}\ Ord{\isacharparenleft}{\kern0pt}a{\isacharparenright}{\kern0pt}\ {\isasymLongrightarrow}\ {\isasymforall}b\ {\isasymin}\ a{\isachardot}{\kern0pt}\ Q{\isacharparenleft}{\kern0pt}EQ{\isacharcomma}{\kern0pt}\ b{\isacharcomma}{\kern0pt}\ b{\isacharparenright}{\kern0pt}\ {\isasymLongrightarrow}\ {\isasymforall}b\ {\isasymin}\ a{\isachardot}{\kern0pt}\ Q{\isacharparenleft}{\kern0pt}MEM{\isacharcomma}{\kern0pt}\ b{\isacharcomma}{\kern0pt}\ a{\isacharparenright}{\kern0pt}{\isachardoublequoteclose}\ \ \isanewline
\ \ \ \ \isacommand{unfolding}\isamarkupfalse%
\ Q{\isacharunderscore}{\kern0pt}def\ \isanewline
\ \ \ \ \isacommand{apply}\isamarkupfalse%
\ {\isacharparenleft}{\kern0pt}clarify{\isacharparenright}{\kern0pt}\ \isanewline
\ \ \isacommand{proof}\isamarkupfalse%
\ {\isacharminus}{\kern0pt}\ \isanewline
\ \ \ \ \isacommand{fix}\isamarkupfalse%
\ a\ b\ x\ y\ p\ \isanewline
\ \ \ \ \isacommand{assume}\isamarkupfalse%
\ xrank{\isacharcolon}{\kern0pt}\ {\isachardoublequoteopen}P{\isacharunderscore}{\kern0pt}rank{\isacharparenleft}{\kern0pt}x{\isacharparenright}{\kern0pt}\ {\isasymle}\ b{\isachardoublequoteclose}\ \isakeyword{and}\ yrank{\isacharcolon}{\kern0pt}{\isachardoublequoteopen}P{\isacharunderscore}{\kern0pt}rank{\isacharparenleft}{\kern0pt}y{\isacharparenright}{\kern0pt}\ {\isasymle}\ a{\isachardoublequoteclose}\ \isakeyword{and}\ bina\ {\isacharcolon}{\kern0pt}\ {\isachardoublequoteopen}b\ {\isasymin}\ a{\isachardoublequoteclose}\ \isanewline
\ \ \ \ \isakeyword{and}\ orda\ {\isacharcolon}{\kern0pt}\ {\isachardoublequoteopen}Ord{\isacharparenleft}{\kern0pt}a{\isacharparenright}{\kern0pt}{\isachardoublequoteclose}\ \isakeyword{and}\ pinP\ {\isacharcolon}{\kern0pt}\ {\isachardoublequoteopen}p\ {\isasymin}\ P{\isachardoublequoteclose}\ \isakeyword{and}\ xpname\ {\isacharcolon}{\kern0pt}\ {\isachardoublequoteopen}x\ {\isasymin}\ P{\isacharunderscore}{\kern0pt}names{\isachardoublequoteclose}\ \isakeyword{and}\ ypname\ {\isacharcolon}{\kern0pt}\ {\isachardoublequoteopen}y\ {\isasymin}\ P{\isacharunderscore}{\kern0pt}names{\isachardoublequoteclose}\ \isanewline
\ \ \ \ \isakeyword{and}\ {\isachardoublequoteopen}{\isasymforall}b{\isasymin}a{\isachardot}{\kern0pt}\ {\isasymforall}x{\isasymin}P{\isacharunderscore}{\kern0pt}names{\isachardot}{\kern0pt}\ {\isasymforall}y{\isasymin}P{\isacharunderscore}{\kern0pt}names{\isachardot}{\kern0pt}\ {\isasymforall}p{\isasymin}P{\isachardot}{\kern0pt}\ P{\isacharunderscore}{\kern0pt}rank{\isacharparenleft}{\kern0pt}x{\isacharparenright}{\kern0pt}\ {\isasymle}\ b\ {\isasymlongrightarrow}\ P{\isacharunderscore}{\kern0pt}rank{\isacharparenleft}{\kern0pt}y{\isacharparenright}{\kern0pt}\ {\isasymle}\ b\ {\isasymlongrightarrow}\ EQ{\isacharparenleft}{\kern0pt}p{\isacharcomma}{\kern0pt}\ x{\isacharcomma}{\kern0pt}\ y{\isacharparenright}{\kern0pt}\ {\isachardoublequoteclose}\isanewline
\isanewline
\ \ \ \ \isacommand{then}\isamarkupfalse%
\ \isacommand{have}\isamarkupfalse%
\ EQH\ {\isacharcolon}{\kern0pt}\ {\isachardoublequoteopen}{\isasymAnd}b\ x\ y\ p{\isachardot}{\kern0pt}\ b\ {\isasymin}\ a\ {\isasymLongrightarrow}\ x\ {\isasymin}\ P{\isacharunderscore}{\kern0pt}names\ {\isasymLongrightarrow}\ y\ {\isasymin}\ P{\isacharunderscore}{\kern0pt}names\ {\isasymLongrightarrow}\ p\ {\isasymin}\ P\ {\isasymLongrightarrow}\ P{\isacharunderscore}{\kern0pt}rank{\isacharparenleft}{\kern0pt}x{\isacharparenright}{\kern0pt}\ {\isasymle}\ b\ {\isasymLongrightarrow}\ P{\isacharunderscore}{\kern0pt}rank{\isacharparenleft}{\kern0pt}y{\isacharparenright}{\kern0pt}\ {\isasymle}\ b\ {\isasymLongrightarrow}\ EQ{\isacharparenleft}{\kern0pt}p{\isacharcomma}{\kern0pt}x{\isacharcomma}{\kern0pt}y{\isacharparenright}{\kern0pt}{\isachardoublequoteclose}\ \isacommand{by}\isamarkupfalse%
\ auto\ \ \isanewline
\isanewline
\ \ \ \ \isacommand{have}\isamarkupfalse%
\ {\isachardoublequoteopen}forces{\isacharunderscore}{\kern0pt}mem{\isacharparenleft}{\kern0pt}p{\isacharcomma}{\kern0pt}\ x{\isacharcomma}{\kern0pt}\ y{\isacharparenright}{\kern0pt}\ {\isasymlongleftrightarrow}\ \ dense{\isacharunderscore}{\kern0pt}below{\isacharparenleft}{\kern0pt}{\isacharbraceleft}{\kern0pt}\ q{\isasymin}P{\isachardot}{\kern0pt}\ {\isasymexists}s{\isachardot}{\kern0pt}\ {\isasymexists}r{\isachardot}{\kern0pt}\ r{\isasymin}P\ {\isasymand}\ {\isasymlangle}s{\isacharcomma}{\kern0pt}r{\isasymrangle}\ {\isasymin}\ y\ {\isasymand}\ q{\isasympreceq}r\ {\isasymand}\ forces{\isacharunderscore}{\kern0pt}eq{\isacharparenleft}{\kern0pt}q{\isacharcomma}{\kern0pt}x{\isacharcomma}{\kern0pt}s{\isacharparenright}{\kern0pt}{\isacharbraceright}{\kern0pt}{\isacharcomma}{\kern0pt}\ p{\isacharparenright}{\kern0pt}{\isachardoublequoteclose}\isanewline
\ \ \ \ \ \ \isacommand{using}\isamarkupfalse%
\ forces{\isacharunderscore}{\kern0pt}mem{\isacharunderscore}{\kern0pt}iff{\isacharunderscore}{\kern0pt}dense{\isacharunderscore}{\kern0pt}below\ pinP\ \isanewline
\ \ \ \ \ \ \isacommand{apply}\isamarkupfalse%
\ auto\ \isanewline
\ \ \ \ \ \ \isacommand{done}\isamarkupfalse%
\ \isanewline
\ \ \ \ \isacommand{also}\isamarkupfalse%
\ \isacommand{have}\isamarkupfalse%
\ {\isachardoublequoteopen}{\isachardot}{\kern0pt}{\isachardot}{\kern0pt}{\isachardot}{\kern0pt}\ {\isasymlongleftrightarrow}\ dense{\isacharunderscore}{\kern0pt}below{\isacharparenleft}{\kern0pt}{\isasympi}{\isacharbackquote}{\kern0pt}{\isacharbackquote}{\kern0pt}{\isacharbraceleft}{\kern0pt}\ q{\isasymin}P{\isachardot}{\kern0pt}\ {\isasymexists}s{\isachardot}{\kern0pt}\ {\isasymexists}r{\isachardot}{\kern0pt}\ r{\isasymin}P\ {\isasymand}\ {\isasymlangle}s{\isacharcomma}{\kern0pt}r{\isasymrangle}\ {\isasymin}\ y\ {\isasymand}\ q{\isasympreceq}r\ {\isasymand}\ forces{\isacharunderscore}{\kern0pt}eq{\isacharparenleft}{\kern0pt}q{\isacharcomma}{\kern0pt}x{\isacharcomma}{\kern0pt}s{\isacharparenright}{\kern0pt}{\isacharbraceright}{\kern0pt}{\isacharcomma}{\kern0pt}\ {\isasympi}{\isacharbackquote}{\kern0pt}p{\isacharparenright}{\kern0pt}{\isachardoublequoteclose}\ \isanewline
\ \ \ \ \ \ \isacommand{apply}\isamarkupfalse%
\ {\isacharparenleft}{\kern0pt}rule{\isacharunderscore}{\kern0pt}tac\ P{\isacharunderscore}{\kern0pt}auto{\isacharunderscore}{\kern0pt}preserves{\isacharunderscore}{\kern0pt}density{\isacharprime}{\kern0pt}{\isacharparenright}{\kern0pt}\ \isanewline
\ \ \ \ \ \ \isacommand{using}\isamarkupfalse%
\ piauto\ pinP\ \isanewline
\ \ \ \ \ \ \isacommand{by}\isamarkupfalse%
\ auto\ \isanewline
\ \ \ \ \isacommand{also}\isamarkupfalse%
\ \isacommand{have}\isamarkupfalse%
\ {\isachardoublequoteopen}{\isachardot}{\kern0pt}{\isachardot}{\kern0pt}{\isachardot}{\kern0pt}\ {\isasymlongleftrightarrow}\ dense{\isacharunderscore}{\kern0pt}below{\isacharparenleft}{\kern0pt}{\isacharbraceleft}{\kern0pt}\ q{\isasymin}P{\isachardot}{\kern0pt}\ {\isasymexists}s{\isachardot}{\kern0pt}\ {\isasymexists}r{\isachardot}{\kern0pt}\ r{\isasymin}P\ {\isasymand}\ {\isasymlangle}s{\isacharcomma}{\kern0pt}r{\isasymrangle}\ {\isasymin}\ Pn{\isacharunderscore}{\kern0pt}auto{\isacharparenleft}{\kern0pt}{\isasympi}{\isacharparenright}{\kern0pt}{\isacharbackquote}{\kern0pt}y\ {\isasymand}\ q{\isasympreceq}r\ {\isasymand}\ forces{\isacharunderscore}{\kern0pt}eq{\isacharparenleft}{\kern0pt}q{\isacharcomma}{\kern0pt}Pn{\isacharunderscore}{\kern0pt}auto{\isacharparenleft}{\kern0pt}{\isasympi}{\isacharparenright}{\kern0pt}{\isacharbackquote}{\kern0pt}x{\isacharcomma}{\kern0pt}s{\isacharparenright}{\kern0pt}{\isacharbraceright}{\kern0pt}{\isacharcomma}{\kern0pt}\ {\isasympi}{\isacharbackquote}{\kern0pt}p{\isacharparenright}{\kern0pt}{\isachardoublequoteclose}\ \isanewline
\ \ \ \ \isacommand{proof}\isamarkupfalse%
\ {\isacharminus}{\kern0pt}\ \isanewline
\ \ \ \ \ \ \isacommand{define}\isamarkupfalse%
\ X\ \isakeyword{where}\ {\isachardoublequoteopen}X\ {\isasymequiv}\ {\isacharbraceleft}{\kern0pt}\ q{\isasymin}P{\isachardot}{\kern0pt}\ {\isasymexists}s{\isachardot}{\kern0pt}\ {\isasymexists}r{\isachardot}{\kern0pt}\ r{\isasymin}P\ {\isasymand}\ {\isasymlangle}s{\isacharcomma}{\kern0pt}r{\isasymrangle}\ {\isasymin}\ y\ {\isasymand}\ q{\isasympreceq}r\ {\isasymand}\ forces{\isacharunderscore}{\kern0pt}eq{\isacharparenleft}{\kern0pt}q{\isacharcomma}{\kern0pt}x{\isacharcomma}{\kern0pt}s{\isacharparenright}{\kern0pt}{\isacharbraceright}{\kern0pt}\ {\isachardoublequoteclose}\isanewline
\ \ \ \ \ \ \isacommand{define}\isamarkupfalse%
\ Y\ \isakeyword{where}\ {\isachardoublequoteopen}Y\ {\isasymequiv}\ {\isacharbraceleft}{\kern0pt}\ q{\isasymin}P{\isachardot}{\kern0pt}\ {\isasymexists}s{\isachardot}{\kern0pt}\ {\isasymexists}r{\isachardot}{\kern0pt}\ r{\isasymin}P\ {\isasymand}\ {\isasymlangle}s{\isacharcomma}{\kern0pt}r{\isasymrangle}\ {\isasymin}\ Pn{\isacharunderscore}{\kern0pt}auto{\isacharparenleft}{\kern0pt}{\isasympi}{\isacharparenright}{\kern0pt}{\isacharbackquote}{\kern0pt}y\ {\isasymand}\ q{\isasympreceq}r\ {\isasymand}\ forces{\isacharunderscore}{\kern0pt}eq{\isacharparenleft}{\kern0pt}q{\isacharcomma}{\kern0pt}Pn{\isacharunderscore}{\kern0pt}auto{\isacharparenleft}{\kern0pt}{\isasympi}{\isacharparenright}{\kern0pt}{\isacharbackquote}{\kern0pt}x{\isacharcomma}{\kern0pt}s{\isacharparenright}{\kern0pt}{\isacharbraceright}{\kern0pt}\ {\isachardoublequoteclose}\isanewline
\ \ \ \ \ \ \isacommand{have}\isamarkupfalse%
\ {\isachardoublequoteopen}{\isasympi}{\isacharbackquote}{\kern0pt}{\isacharbackquote}{\kern0pt}X\ {\isacharequal}{\kern0pt}\ Y{\isachardoublequoteclose}\ \isanewline
\ \ \ \ \ \ \ \ \isacommand{apply}\isamarkupfalse%
\ {\isacharparenleft}{\kern0pt}rule\ equality{\isacharunderscore}{\kern0pt}iffI{\isacharsemicolon}{\kern0pt}\ rule\ iffI{\isacharparenright}{\kern0pt}\isanewline
\ \ \ \ \ \ \ \ \ \isacommand{apply}\isamarkupfalse%
\ {\isacharparenleft}{\kern0pt}rule{\isacharunderscore}{\kern0pt}tac\ A{\isacharequal}{\kern0pt}X\ \isakeyword{and}\ r{\isacharequal}{\kern0pt}{\isasympi}\ \isakeyword{and}\ b{\isacharequal}{\kern0pt}x\ \isakeyword{in}\ imageE{\isacharparenright}{\kern0pt}\ \isanewline
\ \ \ \ \ \ \ \ \ \ \isacommand{apply}\isamarkupfalse%
\ simp\ \isanewline
\ \ \ \ \ \ \ \ \isacommand{apply}\isamarkupfalse%
\ {\isacharparenleft}{\kern0pt}simp\ add{\isacharcolon}{\kern0pt}Y{\isacharunderscore}{\kern0pt}def{\isacharsemicolon}{\kern0pt}\ rule\ conjI{\isacharparenright}{\kern0pt}\isanewline
\ \ \ \ \ \ \ \ \isacommand{using}\isamarkupfalse%
\ piauto\ bij{\isacharunderscore}{\kern0pt}is{\isacharunderscore}{\kern0pt}fun\ \isanewline
\ \ \ \ \ \ \ \ \isacommand{unfolding}\isamarkupfalse%
\ is{\isacharunderscore}{\kern0pt}P{\isacharunderscore}{\kern0pt}auto{\isacharunderscore}{\kern0pt}def\ Pi{\isacharunderscore}{\kern0pt}def\ \isanewline
\ \ \ \ \ \ \ \ \ \ \isacommand{apply}\isamarkupfalse%
\ blast\ \isanewline
\ \ \ \ \ \ \isacommand{proof}\isamarkupfalse%
\ {\isacharminus}{\kern0pt}\ \isanewline
\ \ \ \ \ \ \ \ \isacommand{fix}\isamarkupfalse%
\ q\ piq\ \isacommand{assume}\isamarkupfalse%
\ assms\ {\isacharcolon}{\kern0pt}\ {\isachardoublequoteopen}{\isacharless}{\kern0pt}q{\isacharcomma}{\kern0pt}\ piq{\isachargreater}{\kern0pt}\ {\isasymin}\ {\isasympi}{\isachardoublequoteclose}\ \ {\isachardoublequoteopen}q\ {\isasymin}\ X{\isachardoublequoteclose}\ \isanewline
\ \ \ \ \ \ \ \ \isacommand{then}\isamarkupfalse%
\ \isacommand{obtain}\isamarkupfalse%
\ s\ r\ \isakeyword{where}\ srH\ {\isacharcolon}{\kern0pt}\ \ {\isachardoublequoteopen}r\ {\isasymin}\ P{\isachardoublequoteclose}\ {\isachardoublequoteopen}{\isacharless}{\kern0pt}s{\isacharcomma}{\kern0pt}\ r{\isachargreater}{\kern0pt}\ {\isasymin}\ y{\isachardoublequoteclose}\ {\isachardoublequoteopen}q{\isasympreceq}r{\isachardoublequoteclose}\ {\isachardoublequoteopen}forces{\isacharunderscore}{\kern0pt}eq{\isacharparenleft}{\kern0pt}q{\isacharcomma}{\kern0pt}x{\isacharcomma}{\kern0pt}s{\isacharparenright}{\kern0pt}{\isachardoublequoteclose}\ \isacommand{unfolding}\isamarkupfalse%
\ X{\isacharunderscore}{\kern0pt}def\ \isacommand{by}\isamarkupfalse%
\ auto\ \isanewline
\ \ \ \ \ \ \ \ \isacommand{have}\isamarkupfalse%
\ H{\isacharcolon}{\kern0pt}\ {\isachardoublequoteopen}q\ {\isasymin}\ P\ {\isasymand}\ {\isasympi}{\isacharbackquote}{\kern0pt}q\ {\isacharequal}{\kern0pt}\ piq{\isachardoublequoteclose}\ \isanewline
\ \ \ \ \ \ \ \ \ \ \isacommand{apply}\isamarkupfalse%
\ {\isacharparenleft}{\kern0pt}rule{\isacharunderscore}{\kern0pt}tac\ B{\isacharequal}{\kern0pt}{\isachardoublequoteopen}{\isasymlambda}{\isacharunderscore}{\kern0pt}{\isachardot}{\kern0pt}P{\isachardoublequoteclose}\ \isakeyword{in}\ apply{\isacharunderscore}{\kern0pt}fun{\isacharparenright}{\kern0pt}\ \isanewline
\ \ \ \ \ \ \ \ \ \ \isacommand{using}\isamarkupfalse%
\ bij{\isacharunderscore}{\kern0pt}is{\isacharunderscore}{\kern0pt}fun\ piauto\ \isanewline
\ \ \ \ \ \ \ \ \ \ \isacommand{unfolding}\isamarkupfalse%
\ is{\isacharunderscore}{\kern0pt}P{\isacharunderscore}{\kern0pt}auto{\isacharunderscore}{\kern0pt}def\ assms\ \isanewline
\ \ \ \ \ \ \ \ \ \ \ \isacommand{apply}\isamarkupfalse%
\ blast\ \isanewline
\ \ \ \ \ \ \ \ \ \ \isacommand{using}\isamarkupfalse%
\ assms\ \isanewline
\ \ \ \ \ \ \ \ \ \ \isacommand{by}\isamarkupfalse%
\ simp\ \isanewline
\ \ \ \ \ \ \ \ \isacommand{then}\isamarkupfalse%
\ \isacommand{have}\isamarkupfalse%
\ H{\isadigit{2}}{\isacharcolon}{\kern0pt}{\isachardoublequoteopen}{\isasympi}{\isacharbackquote}{\kern0pt}q\ {\isacharequal}{\kern0pt}\ piq{\isachardoublequoteclose}\ \isacommand{by}\isamarkupfalse%
\ auto\ \isanewline
\ \ \ \ \ \ \ \ \isacommand{then}\isamarkupfalse%
\ \isacommand{have}\isamarkupfalse%
\ H{\isadigit{3}}{\isacharcolon}{\kern0pt}{\isachardoublequoteopen}{\isasympi}{\isacharbackquote}{\kern0pt}q\ {\isasympreceq}\ {\isasympi}\ {\isacharbackquote}{\kern0pt}\ r{\isachardoublequoteclose}\isanewline
\ \ \ \ \ \ \ \ \ \ \isacommand{apply}\isamarkupfalse%
\ {\isacharparenleft}{\kern0pt}rule{\isacharunderscore}{\kern0pt}tac\ P{\isacharunderscore}{\kern0pt}auto{\isacharunderscore}{\kern0pt}preserves{\isacharunderscore}{\kern0pt}leq{\isacharparenright}{\kern0pt}\isanewline
\ \ \ \ \ \ \ \ \ \ \isacommand{using}\isamarkupfalse%
\ piauto\ H\ srH\ \isanewline
\ \ \ \ \ \ \ \ \ \ \isacommand{by}\isamarkupfalse%
\ auto\ \isanewline
\ \ \ \ \ \ \ \ \isacommand{show}\isamarkupfalse%
\ {\isachardoublequoteopen}{\isasymexists}s\ r{\isachardot}{\kern0pt}\ r\ {\isasymin}\ P\ {\isasymand}\ {\isasymlangle}s{\isacharcomma}{\kern0pt}\ r{\isasymrangle}\ {\isasymin}\ Pn{\isacharunderscore}{\kern0pt}auto{\isacharparenleft}{\kern0pt}{\isasympi}{\isacharparenright}{\kern0pt}\ {\isacharbackquote}{\kern0pt}\ y\ {\isasymand}\ piq\ {\isasympreceq}\ r\ {\isasymand}\ forces{\isacharunderscore}{\kern0pt}eq{\isacharparenleft}{\kern0pt}piq{\isacharcomma}{\kern0pt}\ Pn{\isacharunderscore}{\kern0pt}auto{\isacharparenleft}{\kern0pt}{\isasympi}{\isacharparenright}{\kern0pt}{\isacharbackquote}{\kern0pt}x{\isacharcomma}{\kern0pt}\ s{\isacharparenright}{\kern0pt}{\isachardoublequoteclose}\isanewline
\ \ \ \ \ \ \ \ \ \ \isacommand{apply}\isamarkupfalse%
\ {\isacharparenleft}{\kern0pt}rule{\isacharunderscore}{\kern0pt}tac\ x{\isacharequal}{\kern0pt}{\isachardoublequoteopen}Pn{\isacharunderscore}{\kern0pt}auto{\isacharparenleft}{\kern0pt}{\isasympi}{\isacharparenright}{\kern0pt}{\isacharbackquote}{\kern0pt}s{\isachardoublequoteclose}\ \isakeyword{in}\ exI{\isacharparenright}{\kern0pt}\ \isanewline
\ \ \ \ \ \ \ \ \ \ \isacommand{apply}\isamarkupfalse%
\ {\isacharparenleft}{\kern0pt}rule{\isacharunderscore}{\kern0pt}tac\ x{\isacharequal}{\kern0pt}{\isachardoublequoteopen}{\isasympi}{\isacharbackquote}{\kern0pt}r{\isachardoublequoteclose}\ \isakeyword{in}\ exI{\isacharparenright}{\kern0pt}\ \isanewline
\ \ \ \ \ \ \ \ \ \ \isacommand{apply}\isamarkupfalse%
\ {\isacharparenleft}{\kern0pt}auto{\isacharparenright}{\kern0pt}\ \isanewline
\ \ \ \ \ \ \ \ \ \ \isacommand{using}\isamarkupfalse%
\ P{\isacharunderscore}{\kern0pt}auto{\isacharunderscore}{\kern0pt}value\ piauto\ srH\ \isanewline
\ \ \ \ \ \ \ \ \ \ \ \ \ \isacommand{apply}\isamarkupfalse%
\ simp\ \isanewline
\ \ \ \ \ \ \ \ \ \ \isacommand{apply}\isamarkupfalse%
\ {\isacharparenleft}{\kern0pt}rule{\isacharunderscore}{\kern0pt}tac\ a{\isacharequal}{\kern0pt}{\isachardoublequoteopen}{\isacharbraceleft}{\kern0pt}{\isacharless}{\kern0pt}Pn{\isacharunderscore}{\kern0pt}auto{\isacharparenleft}{\kern0pt}{\isasympi}{\isacharparenright}{\kern0pt}{\isacharbackquote}{\kern0pt}z{\isacharcomma}{\kern0pt}\ {\isasympi}{\isacharbackquote}{\kern0pt}v{\isachargreater}{\kern0pt}{\isachardot}{\kern0pt}\ {\isacharless}{\kern0pt}z{\isacharcomma}{\kern0pt}\ v{\isachargreater}{\kern0pt}\ {\isasymin}\ y{\isacharbraceright}{\kern0pt}{\isachardoublequoteclose}\ \isakeyword{and}\ b{\isacharequal}{\kern0pt}{\isachardoublequoteopen}Pn{\isacharunderscore}{\kern0pt}auto{\isacharparenleft}{\kern0pt}{\isasympi}{\isacharparenright}{\kern0pt}\ {\isacharbackquote}{\kern0pt}\ y{\isachardoublequoteclose}\ \isakeyword{in}\ ssubst{\isacharparenright}{\kern0pt}\isanewline
\ \ \ \ \ \ \ \ \ \ \isacommand{using}\isamarkupfalse%
\ Pn{\isacharunderscore}{\kern0pt}auto\ ypname\ \isanewline
\ \ \ \ \ \ \ \ \ \ \ \ \ \isacommand{apply}\isamarkupfalse%
\ simp\ \isanewline
\ \ \ \ \ \ \ \ \ \ \ \ \isacommand{apply}\isamarkupfalse%
\ {\isacharparenleft}{\kern0pt}rule{\isacharunderscore}{\kern0pt}tac\ pair{\isacharunderscore}{\kern0pt}relI{\isacharparenright}{\kern0pt}\ \isanewline
\ \ \ \ \ \ \ \ \ \ \isacommand{using}\isamarkupfalse%
\ srH\ H\ H{\isadigit{2}}\ H{\isadigit{3}}\ \isanewline
\ \ \ \ \ \ \ \ \ \ \ \ \isacommand{apply}\isamarkupfalse%
\ simp{\isacharunderscore}{\kern0pt}all\isanewline
\ \ \ \ \ \ \ \ \ \ \isacommand{apply}\isamarkupfalse%
\ {\isacharparenleft}{\kern0pt}rule{\isacharunderscore}{\kern0pt}tac\ a{\isacharequal}{\kern0pt}{\isachardoublequoteopen}{\isasympi}{\isacharbackquote}{\kern0pt}q{\isachardoublequoteclose}\ \isakeyword{and}\ b\ {\isacharequal}{\kern0pt}\ {\isachardoublequoteopen}piq{\isachardoublequoteclose}\ \isakeyword{in}\ ssubst{\isacharparenright}{\kern0pt}\ \isanewline
\ \ \ \ \ \ \ \ \ \ \isacommand{using}\isamarkupfalse%
\ H\ \isanewline
\ \ \ \ \ \ \ \ \ \ \ \isacommand{apply}\isamarkupfalse%
\ simp\ \isanewline
\ \ \ \ \ \ \ \ \ \ \isacommand{apply}\isamarkupfalse%
\ {\isacharparenleft}{\kern0pt}rule{\isacharunderscore}{\kern0pt}tac\ P{\isacharequal}{\kern0pt}{\isachardoublequoteopen}EQ{\isacharparenleft}{\kern0pt}q{\isacharcomma}{\kern0pt}\ x{\isacharcomma}{\kern0pt}\ s{\isacharparenright}{\kern0pt}{\isachardoublequoteclose}\ \isakeyword{in}\ mp{\isacharparenright}{\kern0pt}\ \isanewline
\ \ \ \ \ \ \ \ \ \ \isacommand{apply}\isamarkupfalse%
\ {\isacharparenleft}{\kern0pt}simp\ add{\isacharcolon}{\kern0pt}\ EQ{\isacharunderscore}{\kern0pt}def{\isacharparenright}{\kern0pt}\ \isanewline
\ \ \ \ \ \ \ \ \ \ \isacommand{apply}\isamarkupfalse%
\ {\isacharparenleft}{\kern0pt}cases\ {\isachardoublequoteopen}P{\isacharunderscore}{\kern0pt}rank{\isacharparenleft}{\kern0pt}s{\isacharparenright}{\kern0pt}\ {\isasymle}\ b{\isachardoublequoteclose}{\isacharparenright}{\kern0pt}\ \isanewline
\ \ \ \ \ \ \ \ \ \ \ \isacommand{apply}\isamarkupfalse%
\ {\isacharparenleft}{\kern0pt}rule{\isacharunderscore}{\kern0pt}tac\ EQH{\isacharbrackleft}{\kern0pt}of\ b{\isacharbrackright}{\kern0pt}{\isacharparenright}{\kern0pt}\ \isacommand{using}\isamarkupfalse%
\ bina\ xpname\ ypname\ H\ xrank\ \isanewline
\ \ \ \ \ \ \ \ \ \ \ \ \ \ \ \ \isacommand{apply}\isamarkupfalse%
\ simp{\isacharunderscore}{\kern0pt}all\ \isanewline
\ \ \ \ \ \ \ \ \ \ \isacommand{using}\isamarkupfalse%
\ ypname\ srH\ P{\isacharunderscore}{\kern0pt}name{\isacharunderscore}{\kern0pt}domain{\isacharunderscore}{\kern0pt}P{\isacharunderscore}{\kern0pt}name\ \isanewline
\ \ \ \ \ \ \ \ \ \ \ \isacommand{apply}\isamarkupfalse%
\ simp\ \isanewline
\ \ \ \ \ \ \ \ \ \ \isacommand{apply}\isamarkupfalse%
\ {\isacharparenleft}{\kern0pt}rule{\isacharunderscore}{\kern0pt}tac\ P{\isacharequal}{\kern0pt}{\isachardoublequoteopen}b\ {\isasymle}\ P{\isacharunderscore}{\kern0pt}rank{\isacharparenleft}{\kern0pt}s{\isacharparenright}{\kern0pt}{\isachardoublequoteclose}\ \isakeyword{in}\ mp{\isacharparenright}{\kern0pt}\ \isanewline
\ \ \ \ \ \ \ \ \ \ \ \isacommand{apply}\isamarkupfalse%
\ {\isacharparenleft}{\kern0pt}rule\ impI{\isacharparenright}{\kern0pt}\ \isanewline
\ \ \ \ \ \ \ \ \ \ \ \isacommand{apply}\isamarkupfalse%
\ {\isacharparenleft}{\kern0pt}rule{\isacharunderscore}{\kern0pt}tac\ EQH{\isacharbrackleft}{\kern0pt}of\ {\isachardoublequoteopen}P{\isacharunderscore}{\kern0pt}rank{\isacharparenleft}{\kern0pt}s{\isacharparenright}{\kern0pt}{\isachardoublequoteclose}{\isacharbrackright}{\kern0pt}{\isacharparenright}{\kern0pt}\ \isanewline
\ \ \ \ \ \ \ \ \ \ \ \ \ \ \ \ \isacommand{apply}\isamarkupfalse%
\ {\isacharparenleft}{\kern0pt}rule{\isacharunderscore}{\kern0pt}tac\ ltD{\isacharparenright}{\kern0pt}\ \isanewline
\ \ \ \ \ \ \ \ \ \ \ \ \ \ \ \ \isacommand{apply}\isamarkupfalse%
{\isacharparenleft}{\kern0pt}rule{\isacharunderscore}{\kern0pt}tac\ b{\isacharequal}{\kern0pt}{\isachardoublequoteopen}P{\isacharunderscore}{\kern0pt}rank{\isacharparenleft}{\kern0pt}y{\isacharparenright}{\kern0pt}{\isachardoublequoteclose}\ \isakeyword{in}\ lt{\isacharunderscore}{\kern0pt}le{\isacharunderscore}{\kern0pt}lt{\isacharparenright}{\kern0pt}\ \isanewline
\ \ \ \ \ \ \ \ \ \ \isacommand{using}\isamarkupfalse%
\ ypname\ domain{\isacharunderscore}{\kern0pt}P{\isacharunderscore}{\kern0pt}rank{\isacharunderscore}{\kern0pt}lt\ yrank\ P{\isacharunderscore}{\kern0pt}name{\isacharunderscore}{\kern0pt}domain{\isacharunderscore}{\kern0pt}P{\isacharunderscore}{\kern0pt}name\ P{\isacharunderscore}{\kern0pt}rank{\isacharunderscore}{\kern0pt}ord\ \isanewline
\ \ \ \ \ \ \ \ \ \ \ \ \ \ \ \ \ \isacommand{apply}\isamarkupfalse%
\ simp{\isacharunderscore}{\kern0pt}all\ \ \ \isanewline
\ \ \ \ \ \ \ \ \ \ \ \isacommand{apply}\isamarkupfalse%
\ {\isacharparenleft}{\kern0pt}rule{\isacharunderscore}{\kern0pt}tac\ j{\isacharequal}{\kern0pt}b\ \isakeyword{in}\ le{\isacharunderscore}{\kern0pt}trans{\isacharparenright}{\kern0pt}\ \isanewline
\ \ \ \ \ \ \ \ \ \ \isacommand{using}\isamarkupfalse%
\ xrank\ \isanewline
\ \ \ \ \ \ \ \ \ \ \ \ \isacommand{apply}\isamarkupfalse%
\ simp{\isacharunderscore}{\kern0pt}all\ \isanewline
\ \ \ \ \ \ \ \ \ \ \isacommand{apply}\isamarkupfalse%
{\isacharparenleft}{\kern0pt}rule{\isacharunderscore}{\kern0pt}tac\ P{\isacharequal}{\kern0pt}{\isachardoublequoteopen}b\ {\isacharless}{\kern0pt}\ P{\isacharunderscore}{\kern0pt}rank{\isacharparenleft}{\kern0pt}s{\isacharparenright}{\kern0pt}{\isachardoublequoteclose}\ \isakeyword{in}\ mp{\isacharparenright}{\kern0pt}\ \isanewline
\ \ \ \ \ \ \ \ \ \ \isacommand{using}\isamarkupfalse%
\ lt{\isacharunderscore}{\kern0pt}succ{\isacharunderscore}{\kern0pt}lt\ \isanewline
\ \ \ \ \ \ \ \ \ \ \ \isacommand{apply}\isamarkupfalse%
\ simp\ \isanewline
\ \ \ \ \ \ \ \ \ \ \isacommand{using}\isamarkupfalse%
\ not{\isacharunderscore}{\kern0pt}le{\isacharunderscore}{\kern0pt}iff{\isacharunderscore}{\kern0pt}lt\ orda\ Ord{\isacharunderscore}{\kern0pt}in{\isacharunderscore}{\kern0pt}Ord\ P{\isacharunderscore}{\kern0pt}rank{\isacharunderscore}{\kern0pt}ord\ \isanewline
\ \ \ \ \ \ \ \ \ \ \isacommand{apply}\isamarkupfalse%
\ auto\ \isanewline
\ \ \ \ \ \ \ \ \ \ \isacommand{done}\isamarkupfalse%
\isanewline
\ \ \ \ \ \ \isacommand{next}\isamarkupfalse%
\ \isanewline
\ \ \ \ \ \ \ \ \isacommand{fix}\isamarkupfalse%
\ q{\isacharprime}{\kern0pt}\ \isacommand{assume}\isamarkupfalse%
\ qinY\ {\isacharcolon}{\kern0pt}\ {\isachardoublequoteopen}q{\isacharprime}{\kern0pt}\ {\isasymin}\ Y{\isachardoublequoteclose}\ \isanewline
\ \ \ \ \ \ \ \ \isacommand{then}\isamarkupfalse%
\ \isacommand{have}\isamarkupfalse%
\ q{\isacharprime}{\kern0pt}inP\ {\isacharcolon}{\kern0pt}\ {\isachardoublequoteopen}q{\isacharprime}{\kern0pt}\ {\isasymin}\ P{\isachardoublequoteclose}\ \isacommand{unfolding}\isamarkupfalse%
\ Y{\isacharunderscore}{\kern0pt}def\ \isacommand{by}\isamarkupfalse%
\ auto\ \isanewline
\ \ \ \ \ \ \ \ \isacommand{obtain}\isamarkupfalse%
\ s{\isacharprime}{\kern0pt}\ r{\isacharprime}{\kern0pt}\ \isanewline
\ \ \ \ \ \ \ \ \ \ \isakeyword{where}\ s{\isacharprime}{\kern0pt}r{\isacharprime}{\kern0pt}H\ {\isacharcolon}{\kern0pt}\ {\isachardoublequoteopen}r{\isacharprime}{\kern0pt}\ {\isasymin}\ P{\isachardoublequoteclose}\ {\isachardoublequoteopen}{\isacharless}{\kern0pt}s{\isacharprime}{\kern0pt}{\isacharcomma}{\kern0pt}\ r{\isacharprime}{\kern0pt}{\isachargreater}{\kern0pt}\ {\isasymin}\ Pn{\isacharunderscore}{\kern0pt}auto{\isacharparenleft}{\kern0pt}{\isasympi}{\isacharparenright}{\kern0pt}{\isacharbackquote}{\kern0pt}y{\isachardoublequoteclose}\ {\isachardoublequoteopen}q{\isacharprime}{\kern0pt}{\isasympreceq}r{\isacharprime}{\kern0pt}{\isachardoublequoteclose}\ {\isachardoublequoteopen}forces{\isacharunderscore}{\kern0pt}eq{\isacharparenleft}{\kern0pt}q{\isacharprime}{\kern0pt}{\isacharcomma}{\kern0pt}Pn{\isacharunderscore}{\kern0pt}auto{\isacharparenleft}{\kern0pt}{\isasympi}{\isacharparenright}{\kern0pt}{\isacharbackquote}{\kern0pt}x{\isacharcomma}{\kern0pt}s{\isacharprime}{\kern0pt}{\isacharparenright}{\kern0pt}{\isachardoublequoteclose}\ \isanewline
\ \ \ \ \ \ \ \ \ \ \isacommand{using}\isamarkupfalse%
\ qinY\ \isacommand{unfolding}\isamarkupfalse%
\ Y{\isacharunderscore}{\kern0pt}def\ \isacommand{by}\isamarkupfalse%
\ auto\ \isanewline
\ \ \ \ \ \ \ \ \isacommand{have}\isamarkupfalse%
\ {\isachardoublequoteopen}{\isasymexists}s\ r{\isachardot}{\kern0pt}\ {\isacharless}{\kern0pt}s{\isacharcomma}{\kern0pt}\ r{\isachargreater}{\kern0pt}\ {\isasymin}\ y\ {\isasymand}\ \ {\isacharless}{\kern0pt}s{\isacharprime}{\kern0pt}{\isacharcomma}{\kern0pt}\ r{\isacharprime}{\kern0pt}{\isachargreater}{\kern0pt}\ {\isacharequal}{\kern0pt}\ {\isacharless}{\kern0pt}Pn{\isacharunderscore}{\kern0pt}auto{\isacharparenleft}{\kern0pt}{\isasympi}{\isacharparenright}{\kern0pt}{\isacharbackquote}{\kern0pt}s{\isacharcomma}{\kern0pt}\ {\isasympi}{\isacharbackquote}{\kern0pt}r{\isachargreater}{\kern0pt}{\isachardoublequoteclose}\ \isanewline
\ \ \ \ \ \ \ \ \ \ \isacommand{apply}\isamarkupfalse%
\ {\isacharparenleft}{\kern0pt}rule{\isacharunderscore}{\kern0pt}tac\ pair{\isacharunderscore}{\kern0pt}rel{\isacharunderscore}{\kern0pt}arg{\isacharparenright}{\kern0pt}\ \isanewline
\ \ \ \ \ \ \ \ \ \ \isacommand{using}\isamarkupfalse%
\ relation{\isacharunderscore}{\kern0pt}P{\isacharunderscore}{\kern0pt}name\ ypname\ s{\isacharprime}{\kern0pt}r{\isacharprime}{\kern0pt}H\ Pn{\isacharunderscore}{\kern0pt}auto\ \isanewline
\ \ \ \ \ \ \ \ \ \ \isacommand{by}\isamarkupfalse%
\ auto\isanewline
\ \ \ \ \ \ \ \ \isacommand{then}\isamarkupfalse%
\ \isacommand{obtain}\isamarkupfalse%
\ s\ r\ \isakeyword{where}\ srH\ {\isacharcolon}{\kern0pt}\ {\isachardoublequoteopen}{\isacharless}{\kern0pt}s{\isacharcomma}{\kern0pt}\ r{\isachargreater}{\kern0pt}{\isasymin}\ y{\isachardoublequoteclose}\ {\isachardoublequoteopen}s{\isacharprime}{\kern0pt}\ {\isacharequal}{\kern0pt}\ Pn{\isacharunderscore}{\kern0pt}auto{\isacharparenleft}{\kern0pt}{\isasympi}{\isacharparenright}{\kern0pt}{\isacharbackquote}{\kern0pt}s{\isachardoublequoteclose}\ {\isachardoublequoteopen}r{\isacharprime}{\kern0pt}\ {\isacharequal}{\kern0pt}\ {\isasympi}{\isacharbackquote}{\kern0pt}r{\isachardoublequoteclose}\ \isacommand{by}\isamarkupfalse%
\ auto\ \isanewline
\isanewline
\ \ \ \ \ \ \ \ \isacommand{have}\isamarkupfalse%
\ {\isachardoublequoteopen}{\isasympi}\ {\isasymin}\ surj{\isacharparenleft}{\kern0pt}P{\isacharcomma}{\kern0pt}\ P{\isacharparenright}{\kern0pt}{\isachardoublequoteclose}\ \isanewline
\ \ \ \ \ \ \ \ \ \ \isacommand{using}\isamarkupfalse%
\ piauto\ bij{\isacharunderscore}{\kern0pt}is{\isacharunderscore}{\kern0pt}surj\ \isacommand{unfolding}\isamarkupfalse%
\ is{\isacharunderscore}{\kern0pt}P{\isacharunderscore}{\kern0pt}auto{\isacharunderscore}{\kern0pt}def\ \isacommand{by}\isamarkupfalse%
\ auto\ \isanewline
\ \ \ \ \ \ \ \ \isacommand{then}\isamarkupfalse%
\ \isacommand{obtain}\isamarkupfalse%
\ q\ \isakeyword{where}\ qH{\isacharcolon}{\kern0pt}\ {\isachardoublequoteopen}q\ {\isasymin}\ P{\isachardoublequoteclose}\ {\isachardoublequoteopen}{\isasympi}{\isacharbackquote}{\kern0pt}q\ {\isacharequal}{\kern0pt}\ q{\isacharprime}{\kern0pt}{\isachardoublequoteclose}\ \isacommand{unfolding}\isamarkupfalse%
\ surj{\isacharunderscore}{\kern0pt}def\ \isacommand{using}\isamarkupfalse%
\ q{\isacharprime}{\kern0pt}inP\ \isacommand{by}\isamarkupfalse%
\ auto\isanewline
\isanewline
\ \ \ \ \ \ \ \ \isacommand{have}\isamarkupfalse%
\ rinP\ {\isacharcolon}{\kern0pt}\ {\isachardoublequoteopen}r\ {\isasymin}\ P{\isachardoublequoteclose}\ \isanewline
\ \ \ \ \ \ \ \ \ \ \isacommand{apply}\isamarkupfalse%
\ {\isacharparenleft}{\kern0pt}rule{\isacharunderscore}{\kern0pt}tac\ x{\isacharequal}{\kern0pt}y\ \isakeyword{and}\ y{\isacharequal}{\kern0pt}s\ \isakeyword{in}\ P{\isacharunderscore}{\kern0pt}name{\isacharunderscore}{\kern0pt}range{\isacharparenright}{\kern0pt}\ \isanewline
\ \ \ \ \ \ \ \ \ \ \isacommand{using}\isamarkupfalse%
\ ypname\ srH\ qH\ \isanewline
\ \ \ \ \ \ \ \ \ \ \ \isacommand{apply}\isamarkupfalse%
\ simp{\isacharunderscore}{\kern0pt}all\ \isanewline
\ \ \ \ \ \ \ \ \ \ \isacommand{done}\isamarkupfalse%
\ \isanewline
\ \ \ \ \ \ \ \ \isacommand{then}\isamarkupfalse%
\ \isacommand{have}\isamarkupfalse%
\ qr\ {\isacharcolon}{\kern0pt}\ {\isachardoublequoteopen}q{\isasympreceq}r{\isachardoublequoteclose}\ \isacommand{using}\isamarkupfalse%
\ s{\isacharprime}{\kern0pt}r{\isacharprime}{\kern0pt}H\ srH\ qH\ P{\isacharunderscore}{\kern0pt}auto{\isacharunderscore}{\kern0pt}preserves{\isacharunderscore}{\kern0pt}leq{\isacharprime}{\kern0pt}\ piauto\ \isacommand{by}\isamarkupfalse%
\ auto\isanewline
\isanewline
\ \ \ \ \ \ \ \ \isacommand{have}\isamarkupfalse%
\ fH{\isacharcolon}{\kern0pt}\ {\isachardoublequoteopen}forces{\isacharunderscore}{\kern0pt}eq{\isacharparenleft}{\kern0pt}{\isasympi}{\isacharbackquote}{\kern0pt}q{\isacharcomma}{\kern0pt}Pn{\isacharunderscore}{\kern0pt}auto{\isacharparenleft}{\kern0pt}{\isasympi}{\isacharparenright}{\kern0pt}{\isacharbackquote}{\kern0pt}x{\isacharcomma}{\kern0pt}\ Pn{\isacharunderscore}{\kern0pt}auto{\isacharparenleft}{\kern0pt}{\isasympi}{\isacharparenright}{\kern0pt}{\isacharbackquote}{\kern0pt}s{\isacharparenright}{\kern0pt}{\isachardoublequoteclose}\ \isacommand{using}\isamarkupfalse%
\ s{\isacharprime}{\kern0pt}r{\isacharprime}{\kern0pt}H\ srH\ qH\ \isacommand{by}\isamarkupfalse%
\ auto\ \isanewline
\isanewline
\ \ \ \ \ \ \ \ \isacommand{have}\isamarkupfalse%
\ spname\ {\isacharcolon}{\kern0pt}\ {\isachardoublequoteopen}s\ {\isasymin}\ P{\isacharunderscore}{\kern0pt}names{\isachardoublequoteclose}\ \isacommand{using}\isamarkupfalse%
\ ypname\ srH\ P{\isacharunderscore}{\kern0pt}name{\isacharunderscore}{\kern0pt}domain{\isacharunderscore}{\kern0pt}P{\isacharunderscore}{\kern0pt}name\ \isacommand{by}\isamarkupfalse%
\ auto\ \isanewline
\ \ \ \ \ \ \ \ \isanewline
\ \ \ \ \ \ \ \ \isacommand{then}\isamarkupfalse%
\ \isacommand{show}\isamarkupfalse%
\ {\isachardoublequoteopen}q{\isacharprime}{\kern0pt}\ {\isasymin}\ {\isasympi}{\isacharbackquote}{\kern0pt}{\isacharbackquote}{\kern0pt}X{\isachardoublequoteclose}\ \isanewline
\ \ \ \ \ \ \ \ \ \ \isacommand{apply}\isamarkupfalse%
\ {\isacharparenleft}{\kern0pt}rule{\isacharunderscore}{\kern0pt}tac\ a{\isacharequal}{\kern0pt}q\ \isakeyword{in}\ imageI{\isacharparenright}{\kern0pt}\ \isanewline
\ \ \ \ \ \ \ \ \ \ \ \isacommand{apply}\isamarkupfalse%
\ {\isacharparenleft}{\kern0pt}rule{\isacharunderscore}{\kern0pt}tac\ a{\isacharequal}{\kern0pt}{\isachardoublequoteopen}{\isasympi}{\isacharbackquote}{\kern0pt}q{\isachardoublequoteclose}\ \isakeyword{and}\ b{\isacharequal}{\kern0pt}q{\isacharprime}{\kern0pt}\ \isakeyword{in}\ ssubst{\isacharparenright}{\kern0pt}\isanewline
\ \ \ \ \ \ \ \ \ \ \isacommand{using}\isamarkupfalse%
\ qH\ \isanewline
\ \ \ \ \ \ \ \ \ \ \ \ \isacommand{apply}\isamarkupfalse%
\ simp\ \isanewline
\ \ \ \ \ \ \ \ \ \ \ \isacommand{apply}\isamarkupfalse%
\ {\isacharparenleft}{\kern0pt}rule{\isacharunderscore}{\kern0pt}tac\ function{\isacharunderscore}{\kern0pt}apply{\isacharunderscore}{\kern0pt}Pair{\isacharparenright}{\kern0pt}\ \isanewline
\ \ \ \ \ \ \ \ \ \ \isacommand{using}\isamarkupfalse%
\ P{\isacharunderscore}{\kern0pt}auto{\isacharunderscore}{\kern0pt}is{\isacharunderscore}{\kern0pt}function\ P{\isacharunderscore}{\kern0pt}auto{\isacharunderscore}{\kern0pt}domain\ qH\ piauto\ \isanewline
\ \ \ \ \ \ \ \ \ \ \ \ \isacommand{apply}\isamarkupfalse%
\ simp{\isacharunderscore}{\kern0pt}all\isanewline
\ \ \ \ \ \ \ \ \ \ \isacommand{unfolding}\isamarkupfalse%
\ X{\isacharunderscore}{\kern0pt}def\ \isanewline
\ \ \ \ \ \ \ \ \ \ \isacommand{apply}\isamarkupfalse%
\ simp\ \isanewline
\ \ \ \ \ \ \ \ \ \ \isacommand{apply}\isamarkupfalse%
\ {\isacharparenleft}{\kern0pt}rule{\isacharunderscore}{\kern0pt}tac\ x{\isacharequal}{\kern0pt}s\ \isakeyword{in}\ exI{\isacharparenright}{\kern0pt}\isanewline
\ \ \ \ \ \ \ \ \ \ \isacommand{apply}\isamarkupfalse%
\ {\isacharparenleft}{\kern0pt}rule{\isacharunderscore}{\kern0pt}tac\ x{\isacharequal}{\kern0pt}r\ \isakeyword{in}\ exI{\isacharparenright}{\kern0pt}\isanewline
\ \ \ \ \ \ \ \ \ \ \isacommand{apply}\isamarkupfalse%
\ {\isacharparenleft}{\kern0pt}rule\ conjI{\isacharparenright}{\kern0pt}\ \isanewline
\ \ \ \ \ \ \ \ \ \ \isacommand{using}\isamarkupfalse%
\ rinP\ \isanewline
\ \ \ \ \ \ \ \ \ \ \ \isacommand{apply}\isamarkupfalse%
\ simp\ \isanewline
\ \ \ \ \ \ \ \ \ \ \isacommand{apply}\isamarkupfalse%
\ {\isacharparenleft}{\kern0pt}rule\ conjI{\isacharparenright}{\kern0pt}\ \isanewline
\ \ \ \ \ \ \ \ \ \ \isacommand{using}\isamarkupfalse%
\ srH\ qH\ \isanewline
\ \ \ \ \ \ \ \ \ \ \ \isacommand{apply}\isamarkupfalse%
\ simp{\isacharunderscore}{\kern0pt}all\ \isanewline
\ \ \ \ \ \ \ \ \ \ \isacommand{apply}\isamarkupfalse%
\ {\isacharparenleft}{\kern0pt}rule\ conjI{\isacharparenright}{\kern0pt}\ \isanewline
\ \ \ \ \ \ \ \ \ \ \isacommand{using}\isamarkupfalse%
\ qr\ \isanewline
\ \ \ \ \ \ \ \ \ \ \ \isacommand{apply}\isamarkupfalse%
\ simp\ \isanewline
\ \ \ \ \ \ \ \ \ \ \isacommand{apply}\isamarkupfalse%
\ {\isacharparenleft}{\kern0pt}rule{\isacharunderscore}{\kern0pt}tac\ P{\isacharequal}{\kern0pt}{\isachardoublequoteopen}EQ{\isacharparenleft}{\kern0pt}q{\isacharcomma}{\kern0pt}\ x{\isacharcomma}{\kern0pt}\ s{\isacharparenright}{\kern0pt}{\isachardoublequoteclose}\ \isakeyword{in}\ mp{\isacharparenright}{\kern0pt}\ \isanewline
\ \ \ \ \ \ \ \ \ \ \isacommand{using}\isamarkupfalse%
\ fH\ \isanewline
\ \ \ \ \ \ \ \ \ \ \ \isacommand{apply}\isamarkupfalse%
\ {\isacharparenleft}{\kern0pt}simp\ add{\isacharcolon}{\kern0pt}\ EQ{\isacharunderscore}{\kern0pt}def{\isacharparenright}{\kern0pt}\ \isanewline
\ \ \ \ \ \ \ \ \ \ \isacommand{apply}\isamarkupfalse%
\ {\isacharparenleft}{\kern0pt}cases\ {\isachardoublequoteopen}P{\isacharunderscore}{\kern0pt}rank{\isacharparenleft}{\kern0pt}s{\isacharparenright}{\kern0pt}\ {\isasymle}\ b{\isachardoublequoteclose}{\isacharparenright}{\kern0pt}\ \isanewline
\ \ \ \ \ \ \ \ \ \ \ \isacommand{apply}\isamarkupfalse%
\ {\isacharparenleft}{\kern0pt}rule{\isacharunderscore}{\kern0pt}tac\ EQH{\isacharbrackleft}{\kern0pt}of\ b{\isacharbrackright}{\kern0pt}{\isacharparenright}{\kern0pt}\ \isanewline
\ \ \ \ \ \ \ \ \ \ \isacommand{using}\isamarkupfalse%
\ bina\ xpname\ ypname\ qH\ xrank\ \isanewline
\ \ \ \ \ \ \ \ \ \ \ \ \ \ \ \ \isacommand{apply}\isamarkupfalse%
\ simp{\isacharunderscore}{\kern0pt}all\ \isanewline
\ \ \ \ \ \ \ \ \ \ \isacommand{using}\isamarkupfalse%
\ ypname\ srH\ P{\isacharunderscore}{\kern0pt}name{\isacharunderscore}{\kern0pt}domain{\isacharunderscore}{\kern0pt}P{\isacharunderscore}{\kern0pt}name\ \isanewline
\ \ \ \ \ \ \ \ \ \ \isacommand{apply}\isamarkupfalse%
\ simp\ \isanewline
\ \ \ \ \ \ \ \ \ \ \isacommand{apply}\isamarkupfalse%
\ {\isacharparenleft}{\kern0pt}rule{\isacharunderscore}{\kern0pt}tac\ P{\isacharequal}{\kern0pt}{\isachardoublequoteopen}b\ {\isasymle}\ P{\isacharunderscore}{\kern0pt}rank{\isacharparenleft}{\kern0pt}s{\isacharparenright}{\kern0pt}{\isachardoublequoteclose}\ \isakeyword{in}\ mp{\isacharparenright}{\kern0pt}\ \isanewline
\ \ \ \ \ \ \ \ \ \ \ \isacommand{apply}\isamarkupfalse%
\ {\isacharparenleft}{\kern0pt}rule\ impI{\isacharparenright}{\kern0pt}\ \isanewline
\ \ \ \ \ \ \ \ \ \ \isacommand{apply}\isamarkupfalse%
\ {\isacharparenleft}{\kern0pt}rule\ EQH\ {\isacharbrackleft}{\kern0pt}of\ {\isachardoublequoteopen}P{\isacharunderscore}{\kern0pt}rank{\isacharparenleft}{\kern0pt}s{\isacharparenright}{\kern0pt}{\isachardoublequoteclose}{\isacharbrackright}{\kern0pt}{\isacharparenright}{\kern0pt}\ \isanewline
\ \ \ \ \ \ \ \ \ \ \ \ \ \ \ \ \isacommand{apply}\isamarkupfalse%
\ {\isacharparenleft}{\kern0pt}rule{\isacharunderscore}{\kern0pt}tac\ ltD{\isacharparenright}{\kern0pt}\ \isanewline
\ \ \ \ \ \ \ \ \ \ \ \ \ \ \ \ \isacommand{apply}\isamarkupfalse%
{\isacharparenleft}{\kern0pt}rule{\isacharunderscore}{\kern0pt}tac\ b{\isacharequal}{\kern0pt}{\isachardoublequoteopen}P{\isacharunderscore}{\kern0pt}rank{\isacharparenleft}{\kern0pt}y{\isacharparenright}{\kern0pt}{\isachardoublequoteclose}\ \isakeyword{in}\ lt{\isacharunderscore}{\kern0pt}le{\isacharunderscore}{\kern0pt}lt{\isacharparenright}{\kern0pt}\ \isanewline
\ \ \ \ \ \ \ \ \ \ \isacommand{using}\isamarkupfalse%
\ ypname\ domain{\isacharunderscore}{\kern0pt}P{\isacharunderscore}{\kern0pt}rank{\isacharunderscore}{\kern0pt}lt\ yrank\ P{\isacharunderscore}{\kern0pt}name{\isacharunderscore}{\kern0pt}domain{\isacharunderscore}{\kern0pt}P{\isacharunderscore}{\kern0pt}name\ P{\isacharunderscore}{\kern0pt}rank{\isacharunderscore}{\kern0pt}ord\ \isanewline
\ \ \ \ \ \ \ \ \ \ \ \ \ \ \ \ \ \isacommand{apply}\isamarkupfalse%
\ simp{\isacharunderscore}{\kern0pt}all\ \ \ \isanewline
\ \ \ \ \ \ \ \ \ \ \ \isacommand{apply}\isamarkupfalse%
\ {\isacharparenleft}{\kern0pt}rule{\isacharunderscore}{\kern0pt}tac\ j{\isacharequal}{\kern0pt}b\ \isakeyword{in}\ le{\isacharunderscore}{\kern0pt}trans{\isacharparenright}{\kern0pt}\ \isanewline
\ \ \ \ \ \ \ \ \ \ \isacommand{using}\isamarkupfalse%
\ xrank\ \isanewline
\ \ \ \ \ \ \ \ \ \ \ \ \isacommand{apply}\isamarkupfalse%
\ simp{\isacharunderscore}{\kern0pt}all\ \isanewline
\ \ \ \ \ \ \ \ \ \ \isacommand{apply}\isamarkupfalse%
{\isacharparenleft}{\kern0pt}rule{\isacharunderscore}{\kern0pt}tac\ P{\isacharequal}{\kern0pt}{\isachardoublequoteopen}b\ {\isacharless}{\kern0pt}\ P{\isacharunderscore}{\kern0pt}rank{\isacharparenleft}{\kern0pt}s{\isacharparenright}{\kern0pt}{\isachardoublequoteclose}\ \isakeyword{in}\ mp{\isacharparenright}{\kern0pt}\ \isanewline
\ \ \ \ \ \ \ \ \ \ \isacommand{using}\isamarkupfalse%
\ lt{\isacharunderscore}{\kern0pt}succ{\isacharunderscore}{\kern0pt}lt\ \isanewline
\ \ \ \ \ \ \ \ \ \ \ \isacommand{apply}\isamarkupfalse%
\ simp\ \isanewline
\ \ \ \ \ \ \ \ \ \ \isacommand{using}\isamarkupfalse%
\ not{\isacharunderscore}{\kern0pt}le{\isacharunderscore}{\kern0pt}iff{\isacharunderscore}{\kern0pt}lt\ orda\ Ord{\isacharunderscore}{\kern0pt}in{\isacharunderscore}{\kern0pt}Ord\ P{\isacharunderscore}{\kern0pt}rank{\isacharunderscore}{\kern0pt}ord\ \isanewline
\ \ \ \ \ \ \ \ \ \ \isacommand{apply}\isamarkupfalse%
\ auto\ \isanewline
\ \ \ \ \ \ \ \ \ \ \isacommand{done}\isamarkupfalse%
\isanewline
\ \ \ \ \ \ \isacommand{qed}\isamarkupfalse%
\isanewline
\ \ \ \ \ \ \isacommand{then}\isamarkupfalse%
\ \isacommand{show}\isamarkupfalse%
\ {\isacharquery}{\kern0pt}thesis\ \isacommand{unfolding}\isamarkupfalse%
\ X{\isacharunderscore}{\kern0pt}def\ Y{\isacharunderscore}{\kern0pt}def\ \isacommand{by}\isamarkupfalse%
\ simp\ \isanewline
\ \ \ \ \ \ \isacommand{qed}\isamarkupfalse%
\isanewline
\ \ \ \ \ \ \isacommand{also}\isamarkupfalse%
\ \isacommand{have}\isamarkupfalse%
\ {\isachardoublequoteopen}{\isachardot}{\kern0pt}{\isachardot}{\kern0pt}{\isachardot}{\kern0pt}\ {\isasymlongleftrightarrow}\ forces{\isacharunderscore}{\kern0pt}mem{\isacharparenleft}{\kern0pt}{\isasympi}{\isacharbackquote}{\kern0pt}p{\isacharcomma}{\kern0pt}\ Pn{\isacharunderscore}{\kern0pt}auto{\isacharparenleft}{\kern0pt}{\isasympi}{\isacharparenright}{\kern0pt}{\isacharbackquote}{\kern0pt}x{\isacharcomma}{\kern0pt}\ Pn{\isacharunderscore}{\kern0pt}auto{\isacharparenleft}{\kern0pt}{\isasympi}{\isacharparenright}{\kern0pt}{\isacharbackquote}{\kern0pt}y{\isacharparenright}{\kern0pt}{\isachardoublequoteclose}\ \isanewline
\ \ \ \ \ \ \ \ \isacommand{apply}\isamarkupfalse%
\ {\isacharparenleft}{\kern0pt}rule\ iff{\isacharunderscore}{\kern0pt}flip{\isacharparenright}{\kern0pt}\isanewline
\ \ \ \ \ \ \ \ \isacommand{apply}\isamarkupfalse%
\ {\isacharparenleft}{\kern0pt}rule{\isacharunderscore}{\kern0pt}tac\ forces{\isacharunderscore}{\kern0pt}mem{\isacharunderscore}{\kern0pt}iff{\isacharunderscore}{\kern0pt}dense{\isacharunderscore}{\kern0pt}below{\isacharparenright}{\kern0pt}\isanewline
\ \ \ \ \ \ \ \ \isacommand{using}\isamarkupfalse%
\ P{\isacharunderscore}{\kern0pt}auto{\isacharunderscore}{\kern0pt}value\ pinP\ piauto\ \isanewline
\ \ \ \ \ \ \ \ \isacommand{by}\isamarkupfalse%
\ auto\ \isanewline
\ \ \ \ \ \ \isacommand{finally}\isamarkupfalse%
\ \isacommand{show}\isamarkupfalse%
\ {\isachardoublequoteopen}MEM{\isacharparenleft}{\kern0pt}p{\isacharcomma}{\kern0pt}\ x{\isacharcomma}{\kern0pt}\ y{\isacharparenright}{\kern0pt}{\isachardoublequoteclose}\ \isacommand{unfolding}\isamarkupfalse%
\ MEM{\isacharunderscore}{\kern0pt}def\ \isacommand{by}\isamarkupfalse%
\ simp\ \isanewline
\ \ \ \ \isacommand{qed}\isamarkupfalse%
\isanewline
\isanewline
\ \ \ \ \isacommand{have}\isamarkupfalse%
\ EQ{\isacharunderscore}{\kern0pt}step\ {\isacharcolon}{\kern0pt}\ {\isachardoublequoteopen}{\isasymAnd}a{\isachardot}{\kern0pt}\ Ord{\isacharparenleft}{\kern0pt}a{\isacharparenright}{\kern0pt}\ {\isasymLongrightarrow}\ {\isasymforall}b\ {\isasymin}\ a{\isachardot}{\kern0pt}\ Q{\isacharparenleft}{\kern0pt}MEM{\isacharcomma}{\kern0pt}\ b{\isacharcomma}{\kern0pt}\ a{\isacharparenright}{\kern0pt}\ {\isasymLongrightarrow}\ Q{\isacharparenleft}{\kern0pt}EQ{\isacharcomma}{\kern0pt}\ a{\isacharcomma}{\kern0pt}\ a{\isacharparenright}{\kern0pt}{\isachardoublequoteclose}\ \ \ \isanewline
\ \ \ \ \ \ \isacommand{unfolding}\isamarkupfalse%
\ Q{\isacharunderscore}{\kern0pt}def\ \isacommand{apply}\isamarkupfalse%
\ clarify\ \isanewline
\ \ \ \ \isacommand{proof}\isamarkupfalse%
\ {\isacharminus}{\kern0pt}\ \isanewline
\ \ \ \ \ \ \isacommand{fix}\isamarkupfalse%
\ a\ x\ y\ p\ \isanewline
\ \ \ \ \ \ \isacommand{assume}\isamarkupfalse%
\ orda\ {\isacharcolon}{\kern0pt}\ {\isachardoublequoteopen}Ord{\isacharparenleft}{\kern0pt}a{\isacharparenright}{\kern0pt}{\isachardoublequoteclose}\ \isanewline
\ \ \ \ \ \ \isakeyword{and}\ {\isachardoublequoteopen}{\isasymforall}b{\isasymin}a{\isachardot}{\kern0pt}\ {\isasymforall}x{\isasymin}P{\isacharunderscore}{\kern0pt}names{\isachardot}{\kern0pt}\ {\isasymforall}y{\isasymin}P{\isacharunderscore}{\kern0pt}names{\isachardot}{\kern0pt}\ {\isasymforall}p{\isasymin}P{\isachardot}{\kern0pt}\ P{\isacharunderscore}{\kern0pt}rank{\isacharparenleft}{\kern0pt}x{\isacharparenright}{\kern0pt}\ {\isasymle}\ b\ {\isasymlongrightarrow}\ P{\isacharunderscore}{\kern0pt}rank{\isacharparenleft}{\kern0pt}y{\isacharparenright}{\kern0pt}\ {\isasymle}\ a\ {\isasymlongrightarrow}\ MEM{\isacharparenleft}{\kern0pt}p{\isacharcomma}{\kern0pt}\ x{\isacharcomma}{\kern0pt}\ y{\isacharparenright}{\kern0pt}{\isachardoublequoteclose}\ \isanewline
\ \ \ \ \ \ \isakeyword{and}\ xpname\ {\isacharcolon}{\kern0pt}\ {\isachardoublequoteopen}x\ {\isasymin}\ P{\isacharunderscore}{\kern0pt}names{\isachardoublequoteclose}\ \isakeyword{and}\ ypname\ {\isacharcolon}{\kern0pt}\ {\isachardoublequoteopen}y\ {\isasymin}\ P{\isacharunderscore}{\kern0pt}names{\isachardoublequoteclose}\ \isanewline
\ \ \ \ \ \ \isakeyword{and}\ pinP\ {\isacharcolon}{\kern0pt}\ {\isachardoublequoteopen}p\ {\isasymin}\ P{\isachardoublequoteclose}\ \isakeyword{and}\ xrank\ {\isacharcolon}{\kern0pt}\ {\isachardoublequoteopen}P{\isacharunderscore}{\kern0pt}rank{\isacharparenleft}{\kern0pt}x{\isacharparenright}{\kern0pt}\ {\isasymle}\ a{\isachardoublequoteclose}\ \isakeyword{and}\ yrank\ {\isacharcolon}{\kern0pt}\ {\isachardoublequoteopen}P{\isacharunderscore}{\kern0pt}rank{\isacharparenleft}{\kern0pt}y{\isacharparenright}{\kern0pt}\ {\isasymle}\ a{\isachardoublequoteclose}\ \isanewline
\isanewline
\ \ \ \ \ \ \isacommand{then}\isamarkupfalse%
\ \isacommand{have}\isamarkupfalse%
\ MEMH\ {\isacharcolon}{\kern0pt}\ \isanewline
\ \ \ \ \ \ \ \ {\isachardoublequoteopen}{\isasymAnd}b\ x\ y\ p{\isachardot}{\kern0pt}\ b\ {\isasymin}\ a\ {\isasymLongrightarrow}\ x\ {\isasymin}\ P{\isacharunderscore}{\kern0pt}names\ {\isasymLongrightarrow}\ y\ {\isasymin}\ P{\isacharunderscore}{\kern0pt}names\ {\isasymLongrightarrow}\ p\ {\isasymin}\ P\ {\isasymLongrightarrow}\ P{\isacharunderscore}{\kern0pt}rank{\isacharparenleft}{\kern0pt}x{\isacharparenright}{\kern0pt}\ {\isasymle}\ b\ {\isasymLongrightarrow}\ P{\isacharunderscore}{\kern0pt}rank{\isacharparenleft}{\kern0pt}y{\isacharparenright}{\kern0pt}\ {\isasymle}\ a\ {\isasymLongrightarrow}\ MEM{\isacharparenleft}{\kern0pt}p{\isacharcomma}{\kern0pt}x{\isacharcomma}{\kern0pt}y{\isacharparenright}{\kern0pt}{\isachardoublequoteclose}\ \isanewline
\ \ \ \ \ \ \ \ \isacommand{by}\isamarkupfalse%
\ auto\ \isanewline
\isanewline
\ \ \ \ \ \ \isacommand{have}\isamarkupfalse%
\ srank{\isacharunderscore}{\kern0pt}lemma\ {\isacharcolon}{\kern0pt}\ {\isachardoublequoteopen}{\isasymAnd}s{\isachardot}{\kern0pt}\ s\ {\isasymin}\ domain{\isacharparenleft}{\kern0pt}x{\isacharparenright}{\kern0pt}\ {\isasymunion}\ domain{\isacharparenleft}{\kern0pt}y{\isacharparenright}{\kern0pt}\ {\isasymLongrightarrow}\ P{\isacharunderscore}{\kern0pt}rank{\isacharparenleft}{\kern0pt}s{\isacharparenright}{\kern0pt}\ {\isacharless}{\kern0pt}\ a{\isachardoublequoteclose}\ \isanewline
\ \ \ \ \ \ \ \ \isacommand{apply}\isamarkupfalse%
\ {\isacharparenleft}{\kern0pt}rule{\isacharunderscore}{\kern0pt}tac\ A{\isacharequal}{\kern0pt}{\isachardoublequoteopen}domain{\isacharparenleft}{\kern0pt}x{\isacharparenright}{\kern0pt}{\isachardoublequoteclose}\ \isakeyword{and}\ B{\isacharequal}{\kern0pt}{\isachardoublequoteopen}domain{\isacharparenleft}{\kern0pt}y{\isacharparenright}{\kern0pt}{\isachardoublequoteclose}\ \isakeyword{and}\ c{\isacharequal}{\kern0pt}s\ \isakeyword{in}\ UnE{\isacharparenright}{\kern0pt}\isanewline
\ \ \ \ \ \ \ \ \ \ \isacommand{apply}\isamarkupfalse%
\ simp\ \isanewline
\ \ \ \ \ \ \ \ \ \isacommand{apply}\isamarkupfalse%
\ {\isacharparenleft}{\kern0pt}rule{\isacharunderscore}{\kern0pt}tac\ b{\isacharequal}{\kern0pt}{\isachardoublequoteopen}P{\isacharunderscore}{\kern0pt}rank{\isacharparenleft}{\kern0pt}x{\isacharparenright}{\kern0pt}{\isachardoublequoteclose}\ \isakeyword{in}\ lt{\isacharunderscore}{\kern0pt}le{\isacharunderscore}{\kern0pt}lt{\isacharparenright}{\kern0pt}\ \isanewline
\ \ \ \ \ \ \ \ \ \ \isacommand{apply}\isamarkupfalse%
\ {\isacharparenleft}{\kern0pt}rule{\isacharunderscore}{\kern0pt}tac\ P{\isacharequal}{\kern0pt}{\isachardoublequoteopen}{\isasymexists}p{\isachardot}{\kern0pt}\ {\isacharless}{\kern0pt}s{\isacharcomma}{\kern0pt}\ p{\isachargreater}{\kern0pt}\ {\isasymin}\ x{\isachardoublequoteclose}\ \isakeyword{in}\ mp{\isacharparenright}{\kern0pt}\ \isanewline
\ \ \ \ \ \ \ \ \ \ \ \isacommand{apply}\isamarkupfalse%
\ clarify\ \isanewline
\ \ \ \ \ \ \ \ \isacommand{using}\isamarkupfalse%
\ xpname\ domain{\isacharunderscore}{\kern0pt}P{\isacharunderscore}{\kern0pt}rank{\isacharunderscore}{\kern0pt}lt\ \isanewline
\ \ \ \ \ \ \ \ \ \ \ \isacommand{apply}\isamarkupfalse%
\ simp\ \isanewline
\ \ \ \ \ \ \ \ \ \ \isacommand{apply}\isamarkupfalse%
\ {\isacharparenleft}{\kern0pt}simp\ add{\isacharcolon}{\kern0pt}domain{\isacharunderscore}{\kern0pt}def{\isacharparenright}{\kern0pt}\ \isanewline
\ \ \ \ \ \ \ \ \ \ \isacommand{apply}\isamarkupfalse%
\ blast\ \isanewline
\ \ \ \ \ \ \ \ \isacommand{using}\isamarkupfalse%
\ xrank\ \isanewline
\ \ \ \ \ \ \ \ \ \isacommand{apply}\isamarkupfalse%
\ simp\ \isanewline
\ \ \ \ \ \ \ \ \isacommand{apply}\isamarkupfalse%
\ {\isacharparenleft}{\kern0pt}rule{\isacharunderscore}{\kern0pt}tac\ b{\isacharequal}{\kern0pt}{\isachardoublequoteopen}P{\isacharunderscore}{\kern0pt}rank{\isacharparenleft}{\kern0pt}y{\isacharparenright}{\kern0pt}{\isachardoublequoteclose}\ \isakeyword{in}\ lt{\isacharunderscore}{\kern0pt}le{\isacharunderscore}{\kern0pt}lt{\isacharparenright}{\kern0pt}\ \isanewline
\ \ \ \ \ \ \ \ \ \isacommand{apply}\isamarkupfalse%
\ {\isacharparenleft}{\kern0pt}rule{\isacharunderscore}{\kern0pt}tac\ P{\isacharequal}{\kern0pt}{\isachardoublequoteopen}{\isasymexists}p{\isachardot}{\kern0pt}\ {\isacharless}{\kern0pt}s{\isacharcomma}{\kern0pt}\ p{\isachargreater}{\kern0pt}\ {\isasymin}\ y{\isachardoublequoteclose}\ \isakeyword{in}\ mp{\isacharparenright}{\kern0pt}\ \isanewline
\ \ \ \ \ \ \ \ \ \ \isacommand{apply}\isamarkupfalse%
\ clarify\ \isanewline
\ \ \ \ \ \ \ \ \isacommand{using}\isamarkupfalse%
\ ypname\ domain{\isacharunderscore}{\kern0pt}P{\isacharunderscore}{\kern0pt}rank{\isacharunderscore}{\kern0pt}lt\ \isanewline
\ \ \ \ \ \ \ \ \ \ \isacommand{apply}\isamarkupfalse%
\ simp\ \isanewline
\ \ \ \ \ \ \ \ \ \isacommand{apply}\isamarkupfalse%
\ {\isacharparenleft}{\kern0pt}simp\ add{\isacharcolon}{\kern0pt}domain{\isacharunderscore}{\kern0pt}def{\isacharparenright}{\kern0pt}\ \isanewline
\ \ \ \ \ \ \ \ \ \isacommand{apply}\isamarkupfalse%
\ blast\ \isanewline
\ \ \ \ \ \ \ \ \isacommand{using}\isamarkupfalse%
\ yrank\ \isanewline
\ \ \ \ \ \ \ \ \isacommand{apply}\isamarkupfalse%
\ simp\ \isanewline
\ \ \ \ \ \ \ \ \isacommand{done}\isamarkupfalse%
\ \isanewline
\isanewline
\ \ \ \ \ \ \isacommand{have}\isamarkupfalse%
\ {\isachardoublequoteopen}forces{\isacharunderscore}{\kern0pt}eq{\isacharparenleft}{\kern0pt}p{\isacharcomma}{\kern0pt}\ x{\isacharcomma}{\kern0pt}\ y{\isacharparenright}{\kern0pt}\ {\isasymlongleftrightarrow}\ {\isacharparenleft}{\kern0pt}{\isasymforall}s{\isasymin}domain{\isacharparenleft}{\kern0pt}x{\isacharparenright}{\kern0pt}\ {\isasymunion}\ domain{\isacharparenleft}{\kern0pt}y{\isacharparenright}{\kern0pt}{\isachardot}{\kern0pt}\ {\isasymforall}q{\isachardot}{\kern0pt}\ q{\isasymin}P\ {\isasymand}\ q\ {\isasympreceq}\ p\ {\isasymlongrightarrow}\ {\isacharparenleft}{\kern0pt}forces{\isacharunderscore}{\kern0pt}mem{\isacharparenleft}{\kern0pt}q{\isacharcomma}{\kern0pt}s{\isacharcomma}{\kern0pt}x{\isacharparenright}{\kern0pt}\ {\isasymlongleftrightarrow}\ forces{\isacharunderscore}{\kern0pt}mem{\isacharparenleft}{\kern0pt}q{\isacharcomma}{\kern0pt}s{\isacharcomma}{\kern0pt}y{\isacharparenright}{\kern0pt}{\isacharparenright}{\kern0pt}{\isacharparenright}{\kern0pt}{\isachardoublequoteclose}\ \isanewline
\ \ \ \ \ \ \ \ \isacommand{using}\isamarkupfalse%
\ def{\isacharunderscore}{\kern0pt}forces{\isacharunderscore}{\kern0pt}eq\ pinP\ \isacommand{by}\isamarkupfalse%
\ auto\isanewline
\ \ \ \ \ \ \isacommand{also}\isamarkupfalse%
\ \isacommand{have}\isamarkupfalse%
\ {\isachardoublequoteopen}{\isachardot}{\kern0pt}{\isachardot}{\kern0pt}{\isachardot}{\kern0pt}\ {\isasymlongleftrightarrow}\ \isanewline
\ \ \ \ \ \ \ \ {\isacharparenleft}{\kern0pt}{\isasymforall}s{\isasymin}domain{\isacharparenleft}{\kern0pt}Pn{\isacharunderscore}{\kern0pt}auto{\isacharparenleft}{\kern0pt}{\isasympi}{\isacharparenright}{\kern0pt}{\isacharbackquote}{\kern0pt}x{\isacharparenright}{\kern0pt}\ {\isasymunion}\ domain{\isacharparenleft}{\kern0pt}Pn{\isacharunderscore}{\kern0pt}auto{\isacharparenleft}{\kern0pt}{\isasympi}{\isacharparenright}{\kern0pt}{\isacharbackquote}{\kern0pt}y{\isacharparenright}{\kern0pt}{\isachardot}{\kern0pt}\ {\isasymforall}q{\isachardot}{\kern0pt}\ q{\isasymin}P\ {\isasymand}\ q\ {\isasympreceq}\ {\isasympi}{\isacharbackquote}{\kern0pt}p\ \isanewline
\ \ \ \ \ \ \ \ \ \ {\isasymlongrightarrow}\ {\isacharparenleft}{\kern0pt}forces{\isacharunderscore}{\kern0pt}mem{\isacharparenleft}{\kern0pt}q{\isacharcomma}{\kern0pt}s{\isacharcomma}{\kern0pt}Pn{\isacharunderscore}{\kern0pt}auto{\isacharparenleft}{\kern0pt}{\isasympi}{\isacharparenright}{\kern0pt}{\isacharbackquote}{\kern0pt}x{\isacharparenright}{\kern0pt}\ {\isasymlongleftrightarrow}\ forces{\isacharunderscore}{\kern0pt}mem{\isacharparenleft}{\kern0pt}q{\isacharcomma}{\kern0pt}s{\isacharcomma}{\kern0pt}Pn{\isacharunderscore}{\kern0pt}auto{\isacharparenleft}{\kern0pt}{\isasympi}{\isacharparenright}{\kern0pt}{\isacharbackquote}{\kern0pt}y{\isacharparenright}{\kern0pt}{\isacharparenright}{\kern0pt}{\isacharparenright}{\kern0pt}{\isachardoublequoteclose}\ \isanewline
\ \ \ \ \ \ \ \ \isacommand{apply}\isamarkupfalse%
\ {\isacharparenleft}{\kern0pt}rule\ iffI{\isacharsemicolon}{\kern0pt}\ clarify{\isacharparenright}{\kern0pt}\isanewline
\ \ \ \ \ \ \isacommand{proof}\isamarkupfalse%
\ {\isacharminus}{\kern0pt}\ \isanewline
\ \ \ \ \ \ \ \ \isacommand{fix}\isamarkupfalse%
\ s{\isacharprime}{\kern0pt}\ q{\isacharprime}{\kern0pt}\ \ \isacommand{assume}\isamarkupfalse%
\ assm{\isacharcolon}{\kern0pt}\ \ {\isachardoublequoteopen}{\isasymforall}s{\isasymin}domain{\isacharparenleft}{\kern0pt}x{\isacharparenright}{\kern0pt}\ {\isasymunion}\ domain{\isacharparenleft}{\kern0pt}y{\isacharparenright}{\kern0pt}{\isachardot}{\kern0pt}\ {\isasymforall}q{\isachardot}{\kern0pt}\ q\ {\isasymin}\ P\ {\isasymand}\ q\ {\isasympreceq}\ p\ {\isasymlongrightarrow}\ forces{\isacharunderscore}{\kern0pt}mem{\isacharparenleft}{\kern0pt}q{\isacharcomma}{\kern0pt}\ s{\isacharcomma}{\kern0pt}\ x{\isacharparenright}{\kern0pt}\ {\isasymlongleftrightarrow}\ forces{\isacharunderscore}{\kern0pt}mem{\isacharparenleft}{\kern0pt}q{\isacharcomma}{\kern0pt}\ s{\isacharcomma}{\kern0pt}\ y{\isacharparenright}{\kern0pt}{\isachardoublequoteclose}\ \isanewline
\ \ \ \ \ \ \ \ \isakeyword{and}\ s{\isacharprime}{\kern0pt}in\ {\isacharcolon}{\kern0pt}\ {\isachardoublequoteopen}s{\isacharprime}{\kern0pt}\ {\isasymin}\ domain{\isacharparenleft}{\kern0pt}Pn{\isacharunderscore}{\kern0pt}auto{\isacharparenleft}{\kern0pt}{\isasympi}{\isacharparenright}{\kern0pt}{\isacharbackquote}{\kern0pt}x{\isacharparenright}{\kern0pt}\ {\isasymunion}\ domain{\isacharparenleft}{\kern0pt}Pn{\isacharunderscore}{\kern0pt}auto{\isacharparenleft}{\kern0pt}{\isasympi}{\isacharparenright}{\kern0pt}{\isacharbackquote}{\kern0pt}y{\isacharparenright}{\kern0pt}{\isachardoublequoteclose}\ \isanewline
\ \ \ \ \ \ \ \ \isakeyword{and}\ q{\isacharprime}{\kern0pt}inP\ {\isacharcolon}{\kern0pt}\ {\isachardoublequoteopen}q{\isacharprime}{\kern0pt}\ {\isasymin}\ P{\isachardoublequoteclose}\ \isakeyword{and}\ q{\isacharprime}{\kern0pt}le\ {\isacharcolon}{\kern0pt}\ {\isachardoublequoteopen}q{\isacharprime}{\kern0pt}\ {\isasympreceq}\ {\isasympi}{\isacharbackquote}{\kern0pt}p{\isachardoublequoteclose}\ \isanewline
\isanewline
\ \ \ \ \ \ \ \ \isacommand{obtain}\isamarkupfalse%
\ q\ \isakeyword{where}\ qH\ {\isacharcolon}{\kern0pt}\ {\isachardoublequoteopen}q\ {\isasymin}\ P{\isachardoublequoteclose}\ {\isachardoublequoteopen}q{\isacharprime}{\kern0pt}\ {\isacharequal}{\kern0pt}\ {\isasympi}{\isacharbackquote}{\kern0pt}q{\isachardoublequoteclose}\ \isanewline
\ \ \ \ \ \ \ \ \ \ \isacommand{using}\isamarkupfalse%
\ piauto\ \isacommand{unfolding}\isamarkupfalse%
\ is{\isacharunderscore}{\kern0pt}P{\isacharunderscore}{\kern0pt}auto{\isacharunderscore}{\kern0pt}def\ \isacommand{using}\isamarkupfalse%
\ bij{\isacharunderscore}{\kern0pt}is{\isacharunderscore}{\kern0pt}surj\ \ q{\isacharprime}{\kern0pt}inP\ \isacommand{unfolding}\isamarkupfalse%
\ surj{\isacharunderscore}{\kern0pt}def\ \isacommand{by}\isamarkupfalse%
\ blast\ \isanewline
\ \ \ \ \ \ \ \ \isacommand{then}\isamarkupfalse%
\ \isacommand{have}\isamarkupfalse%
\ qlep\ {\isacharcolon}{\kern0pt}\ {\isachardoublequoteopen}q\ {\isasympreceq}\ p{\isachardoublequoteclose}\ \ \ \isanewline
\ \ \ \ \ \ \ \ \ \ \isacommand{apply}\isamarkupfalse%
\ {\isacharparenleft}{\kern0pt}rule{\isacharunderscore}{\kern0pt}tac\ {\isasympi}{\isacharequal}{\kern0pt}{\isasympi}\ \isakeyword{in}\ P{\isacharunderscore}{\kern0pt}auto{\isacharunderscore}{\kern0pt}preserves{\isacharunderscore}{\kern0pt}leq{\isacharprime}{\kern0pt}{\isacharparenright}{\kern0pt}\ \isanewline
\ \ \ \ \ \ \ \ \ \ \isacommand{using}\isamarkupfalse%
\ piauto\ pinP\ q{\isacharprime}{\kern0pt}le\ \isanewline
\ \ \ \ \ \ \ \ \ \ \isacommand{by}\isamarkupfalse%
\ auto\ \isanewline
\isanewline
\ \ \ \ \ \ \ \ \isacommand{have}\isamarkupfalse%
\ s{\isacharprime}{\kern0pt}pname\ {\isacharcolon}{\kern0pt}\ {\isachardoublequoteopen}s{\isacharprime}{\kern0pt}\ {\isasymin}\ P{\isacharunderscore}{\kern0pt}names{\isachardoublequoteclose}\ \isanewline
\ \ \ \ \ \ \ \ \ \ \isacommand{apply}\isamarkupfalse%
\ {\isacharparenleft}{\kern0pt}rule{\isacharunderscore}{\kern0pt}tac\ A{\isacharequal}{\kern0pt}{\isachardoublequoteopen}domain{\isacharparenleft}{\kern0pt}Pn{\isacharunderscore}{\kern0pt}auto{\isacharparenleft}{\kern0pt}{\isasympi}{\isacharparenright}{\kern0pt}{\isacharbackquote}{\kern0pt}x{\isacharparenright}{\kern0pt}{\isachardoublequoteclose}\ \isakeyword{and}\ B{\isacharequal}{\kern0pt}{\isachardoublequoteopen}domain{\isacharparenleft}{\kern0pt}Pn{\isacharunderscore}{\kern0pt}auto{\isacharparenleft}{\kern0pt}{\isasympi}{\isacharparenright}{\kern0pt}{\isacharbackquote}{\kern0pt}y{\isacharparenright}{\kern0pt}{\isachardoublequoteclose}\ \isakeyword{and}\ c{\isacharequal}{\kern0pt}s{\isacharprime}{\kern0pt}\ \isakeyword{in}\ UnE{\isacharparenright}{\kern0pt}\isanewline
\ \ \ \ \ \ \ \ \ \ \isacommand{using}\isamarkupfalse%
\ s{\isacharprime}{\kern0pt}in\ \isanewline
\ \ \ \ \ \ \ \ \ \ \ \ \isacommand{apply}\isamarkupfalse%
\ simp\ \isanewline
\ \ \ \ \ \ \ \ \ \ \ \isacommand{apply}\isamarkupfalse%
\ {\isacharparenleft}{\kern0pt}rule{\isacharunderscore}{\kern0pt}tac\ x{\isacharequal}{\kern0pt}{\isachardoublequoteopen}{\isacharparenleft}{\kern0pt}Pn{\isacharunderscore}{\kern0pt}auto{\isacharparenleft}{\kern0pt}{\isasympi}{\isacharparenright}{\kern0pt}\ {\isacharbackquote}{\kern0pt}\ x{\isacharparenright}{\kern0pt}{\isachardoublequoteclose}\ \isakeyword{in}\ P{\isacharunderscore}{\kern0pt}name{\isacharunderscore}{\kern0pt}domain{\isacharunderscore}{\kern0pt}P{\isacharunderscore}{\kern0pt}name{\isacharprime}{\kern0pt}{\isacharparenright}{\kern0pt}\ \isanewline
\ \ \ \ \ \ \ \ \ \ \isacommand{using}\isamarkupfalse%
\ xpname\ Pn{\isacharunderscore}{\kern0pt}auto{\isacharunderscore}{\kern0pt}value\ piauto\ \isanewline
\ \ \ \ \ \ \ \ \ \ \ \ \isacommand{apply}\isamarkupfalse%
\ simp{\isacharunderscore}{\kern0pt}all\isanewline
\ \ \ \ \ \ \ \ \ \ \isacommand{apply}\isamarkupfalse%
\ {\isacharparenleft}{\kern0pt}rule{\isacharunderscore}{\kern0pt}tac\ x{\isacharequal}{\kern0pt}{\isachardoublequoteopen}{\isacharparenleft}{\kern0pt}Pn{\isacharunderscore}{\kern0pt}auto{\isacharparenleft}{\kern0pt}{\isasympi}{\isacharparenright}{\kern0pt}\ {\isacharbackquote}{\kern0pt}\ y{\isacharparenright}{\kern0pt}{\isachardoublequoteclose}\ \isakeyword{in}\ P{\isacharunderscore}{\kern0pt}name{\isacharunderscore}{\kern0pt}domain{\isacharunderscore}{\kern0pt}P{\isacharunderscore}{\kern0pt}name{\isacharprime}{\kern0pt}{\isacharparenright}{\kern0pt}\ \isanewline
\ \ \ \ \ \ \ \ \ \ \isacommand{using}\isamarkupfalse%
\ ypname\ Pn{\isacharunderscore}{\kern0pt}auto{\isacharunderscore}{\kern0pt}value\ piauto\ \isanewline
\ \ \ \ \ \ \ \ \ \ \ \isacommand{apply}\isamarkupfalse%
\ simp{\isacharunderscore}{\kern0pt}all\ \isanewline
\ \ \ \ \ \ \ \ \ \ \isacommand{done}\isamarkupfalse%
\ \isanewline
\ \ \ \ \ \ \ \ \isacommand{then}\isamarkupfalse%
\ \isacommand{obtain}\isamarkupfalse%
\ s\ \isakeyword{where}\ sH\ {\isacharcolon}{\kern0pt}\ {\isachardoublequoteopen}s\ {\isasymin}\ P{\isacharunderscore}{\kern0pt}names{\isachardoublequoteclose}\ {\isachardoublequoteopen}s{\isacharprime}{\kern0pt}\ {\isacharequal}{\kern0pt}\ Pn{\isacharunderscore}{\kern0pt}auto{\isacharparenleft}{\kern0pt}{\isasympi}{\isacharparenright}{\kern0pt}{\isacharbackquote}{\kern0pt}s{\isachardoublequoteclose}\ \isanewline
\ \ \ \ \ \ \ \ \ \ \isacommand{using}\isamarkupfalse%
\ piauto\ Pn{\isacharunderscore}{\kern0pt}auto{\isacharunderscore}{\kern0pt}bij\ bij{\isacharunderscore}{\kern0pt}is{\isacharunderscore}{\kern0pt}surj\ \isacommand{unfolding}\isamarkupfalse%
\ surj{\isacharunderscore}{\kern0pt}def\ \isacommand{by}\isamarkupfalse%
\ blast\ \isanewline
\isanewline
\ \ \ \ \ \ \ \ \isacommand{have}\isamarkupfalse%
\ domain{\isacharunderscore}{\kern0pt}lemma\ {\isacharcolon}{\kern0pt}\ {\isachardoublequoteopen}{\isasymAnd}x{\isachardot}{\kern0pt}\ x\ {\isasymin}\ P{\isacharunderscore}{\kern0pt}names\ {\isasymLongrightarrow}\ s{\isacharprime}{\kern0pt}\ {\isasymin}\ domain{\isacharparenleft}{\kern0pt}Pn{\isacharunderscore}{\kern0pt}auto{\isacharparenleft}{\kern0pt}{\isasympi}{\isacharparenright}{\kern0pt}{\isacharbackquote}{\kern0pt}x{\isacharparenright}{\kern0pt}\ {\isasymLongrightarrow}\ s\ {\isasymin}\ domain{\isacharparenleft}{\kern0pt}x{\isacharparenright}{\kern0pt}{\isachardoublequoteclose}\ \isanewline
\ \ \ \ \ \ \ \ \isacommand{proof}\isamarkupfalse%
\ {\isacharminus}{\kern0pt}\ \isanewline
\ \ \ \ \ \ \ \ \ \ \isacommand{fix}\isamarkupfalse%
\ x\ \isacommand{assume}\isamarkupfalse%
\ xpname\ {\isacharcolon}{\kern0pt}\ {\isachardoublequoteopen}x\ {\isasymin}\ P{\isacharunderscore}{\kern0pt}names{\isachardoublequoteclose}\ \isakeyword{and}\ s{\isacharprime}{\kern0pt}in\ {\isacharcolon}{\kern0pt}\ {\isachardoublequoteopen}s{\isacharprime}{\kern0pt}\ {\isasymin}\ domain{\isacharparenleft}{\kern0pt}Pn{\isacharunderscore}{\kern0pt}auto{\isacharparenleft}{\kern0pt}{\isasympi}{\isacharparenright}{\kern0pt}{\isacharbackquote}{\kern0pt}x{\isacharparenright}{\kern0pt}{\isachardoublequoteclose}\ \isanewline
\ \ \ \ \ \ \ \ \ \ \isacommand{then}\isamarkupfalse%
\ \isacommand{obtain}\isamarkupfalse%
\ p{\isacharprime}{\kern0pt}\ \isakeyword{where}\ {\isachardoublequoteopen}{\isacharless}{\kern0pt}s{\isacharprime}{\kern0pt}{\isacharcomma}{\kern0pt}\ p{\isacharprime}{\kern0pt}{\isachargreater}{\kern0pt}\ {\isasymin}\ Pn{\isacharunderscore}{\kern0pt}auto{\isacharparenleft}{\kern0pt}{\isasympi}{\isacharparenright}{\kern0pt}{\isacharbackquote}{\kern0pt}x{\isachardoublequoteclose}\ \isacommand{by}\isamarkupfalse%
\ auto\ \isanewline
\ \ \ \ \ \ \ \ \ \ \isacommand{then}\isamarkupfalse%
\ \isacommand{have}\isamarkupfalse%
\ {\isachardoublequoteopen}{\isacharless}{\kern0pt}s{\isacharprime}{\kern0pt}{\isacharcomma}{\kern0pt}\ p{\isacharprime}{\kern0pt}{\isachargreater}{\kern0pt}\ {\isasymin}\ {\isacharbraceleft}{\kern0pt}\ {\isacharless}{\kern0pt}Pn{\isacharunderscore}{\kern0pt}auto{\isacharparenleft}{\kern0pt}{\isasympi}{\isacharparenright}{\kern0pt}{\isacharbackquote}{\kern0pt}s{\isacharcomma}{\kern0pt}\ {\isasympi}{\isacharbackquote}{\kern0pt}p{\isachargreater}{\kern0pt}{\isachardot}{\kern0pt}\ {\isacharless}{\kern0pt}s{\isacharcomma}{\kern0pt}\ p{\isachargreater}{\kern0pt}\ {\isasymin}\ x\ {\isacharbraceright}{\kern0pt}{\isachardoublequoteclose}\ \isanewline
\ \ \ \ \ \ \ \ \ \ \ \ \isacommand{using}\isamarkupfalse%
\ Pn{\isacharunderscore}{\kern0pt}auto\ xpname\ \isacommand{by}\isamarkupfalse%
\ auto\isanewline
\ \ \ \ \ \ \ \ \ \ \isacommand{then}\isamarkupfalse%
\ \isacommand{have}\isamarkupfalse%
\ {\isachardoublequoteopen}{\isasymexists}t\ p{\isachardot}{\kern0pt}\ {\isacharless}{\kern0pt}t{\isacharcomma}{\kern0pt}\ p{\isachargreater}{\kern0pt}\ {\isasymin}\ x\ {\isasymand}\ {\isacharless}{\kern0pt}s{\isacharprime}{\kern0pt}{\isacharcomma}{\kern0pt}\ p{\isacharprime}{\kern0pt}{\isachargreater}{\kern0pt}\ {\isacharequal}{\kern0pt}\ {\isacharless}{\kern0pt}Pn{\isacharunderscore}{\kern0pt}auto{\isacharparenleft}{\kern0pt}{\isasympi}{\isacharparenright}{\kern0pt}{\isacharbackquote}{\kern0pt}t{\isacharcomma}{\kern0pt}\ {\isasympi}{\isacharbackquote}{\kern0pt}p{\isachargreater}{\kern0pt}{\isachardoublequoteclose}\ \isanewline
\ \ \ \ \ \ \ \ \ \ \ \ \isacommand{apply}\isamarkupfalse%
\ {\isacharparenleft}{\kern0pt}rule{\isacharunderscore}{\kern0pt}tac\ pair{\isacharunderscore}{\kern0pt}rel{\isacharunderscore}{\kern0pt}arg{\isacharparenright}{\kern0pt}\ \isanewline
\ \ \ \ \ \ \ \ \ \ \ \ \isacommand{using}\isamarkupfalse%
\ xpname\ relation{\isacharunderscore}{\kern0pt}P{\isacharunderscore}{\kern0pt}name\ \isanewline
\ \ \ \ \ \ \ \ \ \ \ \ \ \isacommand{apply}\isamarkupfalse%
\ simp{\isacharunderscore}{\kern0pt}all\ \isanewline
\ \ \ \ \ \ \ \ \ \ \ \ \isacommand{done}\isamarkupfalse%
\ \isanewline
\ \ \ \ \ \ \ \ \ \ \isacommand{then}\isamarkupfalse%
\ \isacommand{obtain}\isamarkupfalse%
\ t\ p\ \isakeyword{where}\ tpH\ {\isacharcolon}{\kern0pt}\ {\isachardoublequoteopen}{\isacharless}{\kern0pt}t{\isacharcomma}{\kern0pt}\ p{\isachargreater}{\kern0pt}\ {\isasymin}\ x{\isachardoublequoteclose}\ {\isachardoublequoteopen}s{\isacharprime}{\kern0pt}\ {\isacharequal}{\kern0pt}\ Pn{\isacharunderscore}{\kern0pt}auto{\isacharparenleft}{\kern0pt}{\isasympi}{\isacharparenright}{\kern0pt}{\isacharbackquote}{\kern0pt}t{\isachardoublequoteclose}\ \isacommand{by}\isamarkupfalse%
\ auto\ \isanewline
\ \ \ \ \ \ \ \ \ \ \isacommand{then}\isamarkupfalse%
\ \isacommand{have}\isamarkupfalse%
\ {\isachardoublequoteopen}s\ {\isacharequal}{\kern0pt}\ t{\isachardoublequoteclose}\ \isanewline
\ \ \ \ \ \ \ \ \ \ \ \ \isacommand{apply}\isamarkupfalse%
\ {\isacharparenleft}{\kern0pt}rule{\isacharunderscore}{\kern0pt}tac\ f{\isacharequal}{\kern0pt}{\isachardoublequoteopen}Pn{\isacharunderscore}{\kern0pt}auto{\isacharparenleft}{\kern0pt}{\isasympi}{\isacharparenright}{\kern0pt}{\isachardoublequoteclose}\ \isakeyword{and}\ b{\isacharequal}{\kern0pt}{\isachardoublequoteopen}s{\isacharprime}{\kern0pt}{\isachardoublequoteclose}\ \isakeyword{and}\ A{\isacharequal}{\kern0pt}\ P{\isacharunderscore}{\kern0pt}names\ \isakeyword{and}\ B{\isacharequal}{\kern0pt}P{\isacharunderscore}{\kern0pt}names\ \isakeyword{in}\ inj{\isacharunderscore}{\kern0pt}equality{\isacharparenright}{\kern0pt}\isanewline
\ \ \ \ \ \ \ \ \ \ \ \ \isacommand{apply}\isamarkupfalse%
\ {\isacharparenleft}{\kern0pt}rule{\isacharunderscore}{\kern0pt}tac\ a{\isacharequal}{\kern0pt}{\isachardoublequoteopen}Pn{\isacharunderscore}{\kern0pt}auto{\isacharparenleft}{\kern0pt}{\isasympi}{\isacharparenright}{\kern0pt}{\isacharbackquote}{\kern0pt}s{\isachardoublequoteclose}\ \isakeyword{and}\ b{\isacharequal}{\kern0pt}s{\isacharprime}{\kern0pt}\ \isakeyword{in}\ ssubst{\isacharparenright}{\kern0pt}\ \isanewline
\ \ \ \ \ \ \ \ \ \ \ \ \isacommand{using}\isamarkupfalse%
\ sH\ \isanewline
\ \ \ \ \ \ \ \ \ \ \ \ \ \ \ \isacommand{apply}\isamarkupfalse%
\ simp\ \isanewline
\ \ \ \ \ \ \ \ \ \ \ \ \ \ \isacommand{apply}\isamarkupfalse%
\ {\isacharparenleft}{\kern0pt}rule{\isacharunderscore}{\kern0pt}tac\ function{\isacharunderscore}{\kern0pt}apply{\isacharunderscore}{\kern0pt}Pair{\isacharparenright}{\kern0pt}\ \isanewline
\ \ \ \ \ \ \ \ \ \ \ \ \isacommand{using}\isamarkupfalse%
\ piauto\ Pn{\isacharunderscore}{\kern0pt}auto{\isacharunderscore}{\kern0pt}function\ \isanewline
\ \ \ \ \ \ \ \ \ \ \ \ \ \ \ \isacommand{apply}\isamarkupfalse%
\ simp\ \isanewline
\ \ \ \ \ \ \ \ \ \ \ \ \isacommand{using}\isamarkupfalse%
\ sH\ Pn{\isacharunderscore}{\kern0pt}auto{\isacharunderscore}{\kern0pt}domain\ \isanewline
\ \ \ \ \ \ \ \ \ \ \ \ \ \ \isacommand{apply}\isamarkupfalse%
\ simp\ \isanewline
\ \ \ \ \ \ \ \ \ \ \ \ \isacommand{apply}\isamarkupfalse%
\ {\isacharparenleft}{\kern0pt}rule{\isacharunderscore}{\kern0pt}tac\ a{\isacharequal}{\kern0pt}{\isachardoublequoteopen}Pn{\isacharunderscore}{\kern0pt}auto{\isacharparenleft}{\kern0pt}{\isasympi}{\isacharparenright}{\kern0pt}{\isacharbackquote}{\kern0pt}t{\isachardoublequoteclose}\ \isakeyword{and}\ b{\isacharequal}{\kern0pt}s{\isacharprime}{\kern0pt}\ \isakeyword{in}\ ssubst{\isacharparenright}{\kern0pt}\ \isanewline
\ \ \ \ \ \ \ \ \ \ \ \ \isacommand{using}\isamarkupfalse%
\ tpH\ \isanewline
\ \ \ \ \ \ \ \ \ \ \ \ \ \ \isacommand{apply}\isamarkupfalse%
\ simp\ \isanewline
\ \ \ \ \ \ \ \ \ \ \ \ \ \isacommand{apply}\isamarkupfalse%
\ {\isacharparenleft}{\kern0pt}rule{\isacharunderscore}{\kern0pt}tac\ function{\isacharunderscore}{\kern0pt}apply{\isacharunderscore}{\kern0pt}Pair{\isacharparenright}{\kern0pt}\ \isanewline
\ \ \ \ \ \ \ \ \ \ \ \ \isacommand{using}\isamarkupfalse%
\ piauto\ Pn{\isacharunderscore}{\kern0pt}auto{\isacharunderscore}{\kern0pt}function\ \isanewline
\ \ \ \ \ \ \ \ \ \ \ \ \ \ \isacommand{apply}\isamarkupfalse%
\ simp\ \isanewline
\ \ \ \ \ \ \ \ \ \ \ \ \isacommand{using}\isamarkupfalse%
\ sH\ Pn{\isacharunderscore}{\kern0pt}auto{\isacharunderscore}{\kern0pt}domain\ \isanewline
\ \ \ \ \ \ \ \ \ \ \ \ \ \isacommand{apply}\isamarkupfalse%
\ simp\ \isanewline
\ \ \ \ \ \ \ \ \ \ \ \ \isacommand{using}\isamarkupfalse%
\ Pn{\isacharunderscore}{\kern0pt}auto{\isacharunderscore}{\kern0pt}bij\ piauto\ bij{\isacharunderscore}{\kern0pt}is{\isacharunderscore}{\kern0pt}inj\ \isanewline
\ \ \ \ \ \ \ \ \ \ \ \ \ \isacommand{apply}\isamarkupfalse%
\ auto\ \isanewline
\ \ \ \ \ \ \ \ \ \ \ \ \isacommand{using}\isamarkupfalse%
\ tpH\ xpname\ P{\isacharunderscore}{\kern0pt}name{\isacharunderscore}{\kern0pt}domain{\isacharunderscore}{\kern0pt}P{\isacharunderscore}{\kern0pt}name\ \isanewline
\ \ \ \ \ \ \ \ \ \ \ \ \isacommand{by}\isamarkupfalse%
\ simp\ \isanewline
\ \ \ \ \ \ \ \ \ \ \isacommand{then}\isamarkupfalse%
\ \isacommand{have}\isamarkupfalse%
\ {\isachardoublequoteopen}{\isacharless}{\kern0pt}s{\isacharcomma}{\kern0pt}\ p{\isachargreater}{\kern0pt}\ {\isasymin}\ x{\isachardoublequoteclose}\ \isacommand{using}\isamarkupfalse%
\ tpH\ \isacommand{by}\isamarkupfalse%
\ auto\ \isanewline
\ \ \ \ \ \ \ \ \ \ \isacommand{then}\isamarkupfalse%
\ \isacommand{show}\isamarkupfalse%
\ {\isachardoublequoteopen}s\ {\isasymin}\ domain{\isacharparenleft}{\kern0pt}x{\isacharparenright}{\kern0pt}{\isachardoublequoteclose}\ \isacommand{by}\isamarkupfalse%
\ auto\ \isanewline
\ \ \ \ \ \ \ \ \isacommand{qed}\isamarkupfalse%
\isanewline
\ \ \ \ \ \ \ \ \isanewline
\ \ \ \ \ \ \ \ \isacommand{have}\isamarkupfalse%
\ sin\ {\isacharcolon}{\kern0pt}\ {\isachardoublequoteopen}s\ {\isasymin}\ domain{\isacharparenleft}{\kern0pt}x{\isacharparenright}{\kern0pt}\ {\isasymunion}\ domain{\isacharparenleft}{\kern0pt}y{\isacharparenright}{\kern0pt}{\isachardoublequoteclose}\ \isanewline
\ \ \ \ \ \ \ \ \ \ \isacommand{using}\isamarkupfalse%
\ s{\isacharprime}{\kern0pt}in\ \isacommand{apply}\isamarkupfalse%
\ simp\ \isanewline
\ \ \ \ \ \ \ \ \ \ \isacommand{using}\isamarkupfalse%
\ domain{\isacharunderscore}{\kern0pt}lemma\ s{\isacharprime}{\kern0pt}pname\ xpname\ ypname\ \isacommand{by}\isamarkupfalse%
\ auto\ \isanewline
\isanewline
\ \ \ \ \ \ \ \ \isacommand{then}\isamarkupfalse%
\ \isacommand{have}\isamarkupfalse%
\ srank\ {\isacharcolon}{\kern0pt}\ {\isachardoublequoteopen}P{\isacharunderscore}{\kern0pt}rank{\isacharparenleft}{\kern0pt}s{\isacharparenright}{\kern0pt}\ {\isacharless}{\kern0pt}\ a{\isachardoublequoteclose}\ \isanewline
\ \ \ \ \ \ \ \ \ \ \isacommand{using}\isamarkupfalse%
\ srank{\isacharunderscore}{\kern0pt}lemma\ \isacommand{by}\isamarkupfalse%
\ auto\ \isanewline
\isanewline
\ \ \ \ \ \ \ \ \isacommand{have}\isamarkupfalse%
\ {\isachardoublequoteopen}forces{\isacharunderscore}{\kern0pt}mem{\isacharparenleft}{\kern0pt}q{\isacharprime}{\kern0pt}{\isacharcomma}{\kern0pt}\ s{\isacharprime}{\kern0pt}{\isacharcomma}{\kern0pt}\ Pn{\isacharunderscore}{\kern0pt}auto{\isacharparenleft}{\kern0pt}{\isasympi}{\isacharparenright}{\kern0pt}\ {\isacharbackquote}{\kern0pt}\ x{\isacharparenright}{\kern0pt}\ \isanewline
\ \ \ \ \ \ \ \ \ \ \ \ {\isasymlongleftrightarrow}\ forces{\isacharunderscore}{\kern0pt}mem{\isacharparenleft}{\kern0pt}{\isasympi}{\isacharbackquote}{\kern0pt}q{\isacharcomma}{\kern0pt}\ Pn{\isacharunderscore}{\kern0pt}auto{\isacharparenleft}{\kern0pt}{\isasympi}{\isacharparenright}{\kern0pt}\ {\isacharbackquote}{\kern0pt}\ s{\isacharcomma}{\kern0pt}\ Pn{\isacharunderscore}{\kern0pt}auto{\isacharparenleft}{\kern0pt}{\isasympi}{\isacharparenright}{\kern0pt}\ {\isacharbackquote}{\kern0pt}\ x{\isacharparenright}{\kern0pt}{\isachardoublequoteclose}\ \isacommand{using}\isamarkupfalse%
\ sH\ qH\ \isacommand{by}\isamarkupfalse%
\ auto\isanewline
\ \ \ \ \ \ \ \ \isacommand{also}\isamarkupfalse%
\ \isacommand{have}\isamarkupfalse%
\ {\isachardoublequoteopen}{\isachardot}{\kern0pt}{\isachardot}{\kern0pt}{\isachardot}{\kern0pt}\ {\isasymlongleftrightarrow}\ forces{\isacharunderscore}{\kern0pt}mem{\isacharparenleft}{\kern0pt}q{\isacharcomma}{\kern0pt}\ s{\isacharcomma}{\kern0pt}\ x{\isacharparenright}{\kern0pt}{\isachardoublequoteclose}\ \isanewline
\ \ \ \ \ \ \ \ \ \ \isacommand{apply}\isamarkupfalse%
\ {\isacharparenleft}{\kern0pt}rule{\isacharunderscore}{\kern0pt}tac\ P{\isacharequal}{\kern0pt}{\isachardoublequoteopen}MEM{\isacharparenleft}{\kern0pt}q{\isacharcomma}{\kern0pt}s{\isacharcomma}{\kern0pt}x{\isacharparenright}{\kern0pt}{\isachardoublequoteclose}\ \isakeyword{in}\ mp{\isacharparenright}{\kern0pt}\ \isanewline
\ \ \ \ \ \ \ \ \ \ \ \isacommand{apply}\isamarkupfalse%
\ {\isacharparenleft}{\kern0pt}simp\ add{\isacharcolon}{\kern0pt}MEM{\isacharunderscore}{\kern0pt}def{\isacharparenright}{\kern0pt}\ \isanewline
\ \ \ \ \ \ \ \ \ \ \isacommand{apply}\isamarkupfalse%
{\isacharparenleft}{\kern0pt}rule\ MEMH\ {\isacharbrackleft}{\kern0pt}of\ {\isachardoublequoteopen}P{\isacharunderscore}{\kern0pt}rank{\isacharparenleft}{\kern0pt}s{\isacharparenright}{\kern0pt}{\isachardoublequoteclose}{\isacharbrackright}{\kern0pt}{\isacharparenright}{\kern0pt}\ \isanewline
\ \ \ \ \ \ \ \ \ \ \isacommand{using}\isamarkupfalse%
\ srank\ ltD\ sH\ xpname\ qH\ le{\isacharunderscore}{\kern0pt}refl\ xrank\ P{\isacharunderscore}{\kern0pt}rank{\isacharunderscore}{\kern0pt}ord\ \isanewline
\ \ \ \ \ \ \ \ \ \ \isacommand{by}\isamarkupfalse%
\ auto\ \ \ \ \ \ \ \ \isanewline
\ \ \ \ \ \ \ \ \isacommand{also}\isamarkupfalse%
\ \isacommand{have}\isamarkupfalse%
\ {\isachardoublequoteopen}{\isachardot}{\kern0pt}{\isachardot}{\kern0pt}{\isachardot}{\kern0pt}\ {\isasymlongleftrightarrow}\ forces{\isacharunderscore}{\kern0pt}mem{\isacharparenleft}{\kern0pt}q{\isacharcomma}{\kern0pt}\ s{\isacharcomma}{\kern0pt}\ y{\isacharparenright}{\kern0pt}{\isachardoublequoteclose}\ \isanewline
\ \ \ \ \ \ \ \ \ \ \isacommand{using}\isamarkupfalse%
\ qH\ qlep\ assm\ sin\ \isacommand{by}\isamarkupfalse%
\ auto\ \isanewline
\ \ \ \ \ \ \ \ \isacommand{also}\isamarkupfalse%
\ \isacommand{have}\isamarkupfalse%
\ {\isachardoublequoteopen}{\isachardot}{\kern0pt}{\isachardot}{\kern0pt}{\isachardot}{\kern0pt}\ {\isasymlongleftrightarrow}\ forces{\isacharunderscore}{\kern0pt}mem{\isacharparenleft}{\kern0pt}{\isasympi}{\isacharbackquote}{\kern0pt}q{\isacharcomma}{\kern0pt}\ Pn{\isacharunderscore}{\kern0pt}auto{\isacharparenleft}{\kern0pt}{\isasympi}{\isacharparenright}{\kern0pt}\ {\isacharbackquote}{\kern0pt}\ s{\isacharcomma}{\kern0pt}\ Pn{\isacharunderscore}{\kern0pt}auto{\isacharparenleft}{\kern0pt}{\isasympi}{\isacharparenright}{\kern0pt}\ {\isacharbackquote}{\kern0pt}\ y{\isacharparenright}{\kern0pt}{\isachardoublequoteclose}\ \isanewline
\ \ \ \ \ \ \ \ \ \ \isacommand{apply}\isamarkupfalse%
\ {\isacharparenleft}{\kern0pt}rule{\isacharunderscore}{\kern0pt}tac\ P{\isacharequal}{\kern0pt}{\isachardoublequoteopen}MEM{\isacharparenleft}{\kern0pt}q{\isacharcomma}{\kern0pt}s{\isacharcomma}{\kern0pt}y{\isacharparenright}{\kern0pt}{\isachardoublequoteclose}\ \isakeyword{in}\ mp{\isacharparenright}{\kern0pt}\ \isanewline
\ \ \ \ \ \ \ \ \ \ \ \isacommand{apply}\isamarkupfalse%
\ {\isacharparenleft}{\kern0pt}simp\ add{\isacharcolon}{\kern0pt}MEM{\isacharunderscore}{\kern0pt}def{\isacharparenright}{\kern0pt}\ \isanewline
\ \ \ \ \ \ \ \ \ \ \isacommand{apply}\isamarkupfalse%
{\isacharparenleft}{\kern0pt}rule\ MEMH{\isacharbrackleft}{\kern0pt}of\ {\isachardoublequoteopen}P{\isacharunderscore}{\kern0pt}rank{\isacharparenleft}{\kern0pt}s{\isacharparenright}{\kern0pt}{\isachardoublequoteclose}{\isacharbrackright}{\kern0pt}{\isacharparenright}{\kern0pt}\ \isanewline
\ \ \ \ \ \ \ \ \ \ \isacommand{using}\isamarkupfalse%
\ srank\ ltD\ sH\ ypname\ qH\ le{\isacharunderscore}{\kern0pt}refl\ yrank\ P{\isacharunderscore}{\kern0pt}rank{\isacharunderscore}{\kern0pt}ord\ \isanewline
\ \ \ \ \ \ \ \ \ \ \isacommand{by}\isamarkupfalse%
\ auto\ \ \isanewline
\ \ \ \ \ \ \ \ \isacommand{also}\isamarkupfalse%
\ \isacommand{have}\isamarkupfalse%
\ {\isachardoublequoteopen}{\isachardot}{\kern0pt}{\isachardot}{\kern0pt}{\isachardot}{\kern0pt}\ {\isasymlongleftrightarrow}\ forces{\isacharunderscore}{\kern0pt}mem{\isacharparenleft}{\kern0pt}q{\isacharprime}{\kern0pt}{\isacharcomma}{\kern0pt}\ s{\isacharprime}{\kern0pt}{\isacharcomma}{\kern0pt}\ Pn{\isacharunderscore}{\kern0pt}auto{\isacharparenleft}{\kern0pt}{\isasympi}{\isacharparenright}{\kern0pt}\ {\isacharbackquote}{\kern0pt}\ y{\isacharparenright}{\kern0pt}{\isachardoublequoteclose}\ \isacommand{using}\isamarkupfalse%
\ sH\ qH\ \isacommand{by}\isamarkupfalse%
\ auto\isanewline
\ \ \ \ \ \ \ \ \isacommand{finally}\isamarkupfalse%
\ \isacommand{show}\isamarkupfalse%
\ {\isachardoublequoteopen}forces{\isacharunderscore}{\kern0pt}mem{\isacharparenleft}{\kern0pt}q{\isacharprime}{\kern0pt}{\isacharcomma}{\kern0pt}\ s{\isacharprime}{\kern0pt}{\isacharcomma}{\kern0pt}\ Pn{\isacharunderscore}{\kern0pt}auto{\isacharparenleft}{\kern0pt}{\isasympi}{\isacharparenright}{\kern0pt}\ {\isacharbackquote}{\kern0pt}\ x{\isacharparenright}{\kern0pt}\ {\isasymlongleftrightarrow}\ forces{\isacharunderscore}{\kern0pt}mem{\isacharparenleft}{\kern0pt}q{\isacharprime}{\kern0pt}{\isacharcomma}{\kern0pt}\ s{\isacharprime}{\kern0pt}{\isacharcomma}{\kern0pt}\ Pn{\isacharunderscore}{\kern0pt}auto{\isacharparenleft}{\kern0pt}{\isasympi}{\isacharparenright}{\kern0pt}\ {\isacharbackquote}{\kern0pt}\ y{\isacharparenright}{\kern0pt}\ {\isachardoublequoteclose}\ \isacommand{by}\isamarkupfalse%
\ auto\ \isanewline
\ \ \ \ \ \ \isacommand{next}\isamarkupfalse%
\isanewline
\ \ \ \ \ \ \ \ \isacommand{fix}\isamarkupfalse%
\ s\ q\ \isanewline
\ \ \ \ \ \ \ \ \isacommand{assume}\isamarkupfalse%
\ assm{\isacharcolon}{\kern0pt}\ {\isachardoublequoteopen}{\isasymforall}s{\isacharprime}{\kern0pt}{\isasymin}domain{\isacharparenleft}{\kern0pt}Pn{\isacharunderscore}{\kern0pt}auto{\isacharparenleft}{\kern0pt}{\isasympi}{\isacharparenright}{\kern0pt}\ {\isacharbackquote}{\kern0pt}\ x{\isacharparenright}{\kern0pt}\ {\isasymunion}\ domain{\isacharparenleft}{\kern0pt}Pn{\isacharunderscore}{\kern0pt}auto{\isacharparenleft}{\kern0pt}{\isasympi}{\isacharparenright}{\kern0pt}\ {\isacharbackquote}{\kern0pt}\ y{\isacharparenright}{\kern0pt}{\isachardot}{\kern0pt}\isanewline
\ \ \ \ \ \ \ \ \ \ \ \ \ \ {\isasymforall}q{\isacharprime}{\kern0pt}{\isachardot}{\kern0pt}\ q{\isacharprime}{\kern0pt}\ {\isasymin}\ P\ {\isasymand}\ q{\isacharprime}{\kern0pt}\ {\isasympreceq}\ {\isasympi}\ {\isacharbackquote}{\kern0pt}\ p\ {\isasymlongrightarrow}\ forces{\isacharunderscore}{\kern0pt}mem{\isacharparenleft}{\kern0pt}q{\isacharprime}{\kern0pt}{\isacharcomma}{\kern0pt}\ s{\isacharprime}{\kern0pt}{\isacharcomma}{\kern0pt}\ Pn{\isacharunderscore}{\kern0pt}auto{\isacharparenleft}{\kern0pt}{\isasympi}{\isacharparenright}{\kern0pt}\ {\isacharbackquote}{\kern0pt}\ x{\isacharparenright}{\kern0pt}\ {\isasymlongleftrightarrow}\ forces{\isacharunderscore}{\kern0pt}mem{\isacharparenleft}{\kern0pt}q{\isacharprime}{\kern0pt}{\isacharcomma}{\kern0pt}\ s{\isacharprime}{\kern0pt}{\isacharcomma}{\kern0pt}\ Pn{\isacharunderscore}{\kern0pt}auto{\isacharparenleft}{\kern0pt}{\isasympi}{\isacharparenright}{\kern0pt}\ {\isacharbackquote}{\kern0pt}\ y{\isacharparenright}{\kern0pt}{\isachardoublequoteclose}\ \isanewline
\ \ \ \ \ \ \ \ \isakeyword{and}\ sin\ {\isacharcolon}{\kern0pt}\ {\isachardoublequoteopen}s\ {\isasymin}\ domain{\isacharparenleft}{\kern0pt}x{\isacharparenright}{\kern0pt}\ {\isasymunion}\ domain{\isacharparenleft}{\kern0pt}y{\isacharparenright}{\kern0pt}{\isachardoublequoteclose}\ \isanewline
\ \ \ \ \ \ \ \ \isakeyword{and}\ qinP\ {\isacharcolon}{\kern0pt}\ {\isachardoublequoteopen}q\ {\isasymin}\ P{\isachardoublequoteclose}\ \isakeyword{and}\ qlep\ {\isacharcolon}{\kern0pt}\ {\isachardoublequoteopen}q\ {\isasympreceq}\ p{\isachardoublequoteclose}\ \isanewline
\isanewline
\ \ \ \ \ \ \ \ \isacommand{have}\isamarkupfalse%
\ srank\ {\isacharcolon}{\kern0pt}\ {\isachardoublequoteopen}P{\isacharunderscore}{\kern0pt}rank{\isacharparenleft}{\kern0pt}s{\isacharparenright}{\kern0pt}\ {\isacharless}{\kern0pt}\ a{\isachardoublequoteclose}\ \isacommand{using}\isamarkupfalse%
\ srank{\isacharunderscore}{\kern0pt}lemma\ sin\ \isacommand{by}\isamarkupfalse%
\ auto\ \isanewline
\isanewline
\ \ \ \ \ \ \ \ \isacommand{have}\isamarkupfalse%
\ domain{\isacharunderscore}{\kern0pt}lemma\ {\isacharcolon}{\kern0pt}\ {\isachardoublequoteopen}{\isasymAnd}x{\isachardot}{\kern0pt}\ x\ {\isasymin}\ P{\isacharunderscore}{\kern0pt}names\ {\isasymLongrightarrow}\ \ s{\isasymin}domain{\isacharparenleft}{\kern0pt}x{\isacharparenright}{\kern0pt}\ {\isasymLongrightarrow}\ Pn{\isacharunderscore}{\kern0pt}auto{\isacharparenleft}{\kern0pt}{\isasympi}{\isacharparenright}{\kern0pt}{\isacharbackquote}{\kern0pt}s\ {\isasymin}\ domain{\isacharparenleft}{\kern0pt}Pn{\isacharunderscore}{\kern0pt}auto{\isacharparenleft}{\kern0pt}{\isasympi}{\isacharparenright}{\kern0pt}{\isacharbackquote}{\kern0pt}x{\isacharparenright}{\kern0pt}{\isachardoublequoteclose}\ \isanewline
\ \ \ \ \ \ \ \ \isacommand{proof}\isamarkupfalse%
\ {\isacharminus}{\kern0pt}\ \isanewline
\ \ \ \ \ \ \ \ \ \ \isacommand{fix}\isamarkupfalse%
\ x\ \isacommand{assume}\isamarkupfalse%
\ assms\ {\isacharcolon}{\kern0pt}\ {\isachardoublequoteopen}s\ {\isasymin}\ domain{\isacharparenleft}{\kern0pt}x{\isacharparenright}{\kern0pt}{\isachardoublequoteclose}\ {\isachardoublequoteopen}x\ {\isasymin}\ P{\isacharunderscore}{\kern0pt}names{\isachardoublequoteclose}\ \isanewline
\ \ \ \ \ \ \ \ \ \ \isacommand{then}\isamarkupfalse%
\ \isacommand{obtain}\isamarkupfalse%
\ p\ \isakeyword{where}\ {\isachardoublequoteopen}{\isacharless}{\kern0pt}s{\isacharcomma}{\kern0pt}\ p{\isachargreater}{\kern0pt}\ {\isasymin}\ x{\isachardoublequoteclose}\ \isacommand{unfolding}\isamarkupfalse%
\ domain{\isacharunderscore}{\kern0pt}def\ \isacommand{by}\isamarkupfalse%
\ auto\ \isanewline
\ \ \ \ \ \ \ \ \ \ \isacommand{then}\isamarkupfalse%
\ \isacommand{have}\isamarkupfalse%
\ {\isachardoublequoteopen}{\isacharless}{\kern0pt}Pn{\isacharunderscore}{\kern0pt}auto{\isacharparenleft}{\kern0pt}{\isasympi}{\isacharparenright}{\kern0pt}{\isacharbackquote}{\kern0pt}s{\isacharcomma}{\kern0pt}\ {\isasympi}{\isacharbackquote}{\kern0pt}p{\isachargreater}{\kern0pt}\ {\isasymin}\ {\isacharbraceleft}{\kern0pt}\ {\isacharless}{\kern0pt}Pn{\isacharunderscore}{\kern0pt}auto{\isacharparenleft}{\kern0pt}{\isasympi}{\isacharparenright}{\kern0pt}{\isacharbackquote}{\kern0pt}s{\isacharcomma}{\kern0pt}\ {\isasympi}{\isacharbackquote}{\kern0pt}p{\isachargreater}{\kern0pt}{\isachardot}{\kern0pt}\ {\isacharless}{\kern0pt}s{\isacharcomma}{\kern0pt}\ p{\isachargreater}{\kern0pt}\ {\isasymin}\ x\ {\isacharbraceright}{\kern0pt}{\isachardoublequoteclose}\ \isanewline
\ \ \ \ \ \ \ \ \ \ \ \ \isacommand{apply}\isamarkupfalse%
\ {\isacharparenleft}{\kern0pt}rule{\isacharunderscore}{\kern0pt}tac\ pair{\isacharunderscore}{\kern0pt}relI{\isacharparenright}{\kern0pt}\ \isanewline
\ \ \ \ \ \ \ \ \ \ \ \ \isacommand{by}\isamarkupfalse%
\ auto\ \isanewline
\ \ \ \ \ \ \ \ \ \ \isacommand{then}\isamarkupfalse%
\ \isacommand{have}\isamarkupfalse%
\ {\isachardoublequoteopen}{\isacharless}{\kern0pt}Pn{\isacharunderscore}{\kern0pt}auto{\isacharparenleft}{\kern0pt}{\isasympi}{\isacharparenright}{\kern0pt}{\isacharbackquote}{\kern0pt}s{\isacharcomma}{\kern0pt}\ {\isasympi}{\isacharbackquote}{\kern0pt}p{\isachargreater}{\kern0pt}\ {\isasymin}\ Pn{\isacharunderscore}{\kern0pt}auto{\isacharparenleft}{\kern0pt}{\isasympi}{\isacharparenright}{\kern0pt}{\isacharbackquote}{\kern0pt}x{\isachardoublequoteclose}\ \isanewline
\ \ \ \ \ \ \ \ \ \ \ \ \isacommand{using}\isamarkupfalse%
\ assms\ piauto\ Pn{\isacharunderscore}{\kern0pt}auto\ \isanewline
\ \ \ \ \ \ \ \ \ \ \ \ \isacommand{by}\isamarkupfalse%
\ auto\ \isanewline
\ \ \ \ \ \ \ \ \ \ \isacommand{then}\isamarkupfalse%
\ \isacommand{show}\isamarkupfalse%
\ {\isachardoublequoteopen}Pn{\isacharunderscore}{\kern0pt}auto{\isacharparenleft}{\kern0pt}{\isasympi}{\isacharparenright}{\kern0pt}{\isacharbackquote}{\kern0pt}s\ {\isasymin}\ domain{\isacharparenleft}{\kern0pt}Pn{\isacharunderscore}{\kern0pt}auto{\isacharparenleft}{\kern0pt}{\isasympi}{\isacharparenright}{\kern0pt}{\isacharbackquote}{\kern0pt}x{\isacharparenright}{\kern0pt}{\isachardoublequoteclose}\ \isacommand{by}\isamarkupfalse%
\ auto\ \isanewline
\ \ \ \ \ \ \ \ \isacommand{qed}\isamarkupfalse%
\isanewline
\isanewline
\ \ \ \ \ \ \ \ \isacommand{have}\isamarkupfalse%
\ s{\isacharprime}{\kern0pt}in\ {\isacharcolon}{\kern0pt}\ {\isachardoublequoteopen}Pn{\isacharunderscore}{\kern0pt}auto{\isacharparenleft}{\kern0pt}{\isasympi}{\isacharparenright}{\kern0pt}{\isacharbackquote}{\kern0pt}s\ {\isasymin}\ domain{\isacharparenleft}{\kern0pt}Pn{\isacharunderscore}{\kern0pt}auto{\isacharparenleft}{\kern0pt}{\isasympi}{\isacharparenright}{\kern0pt}{\isacharbackquote}{\kern0pt}x{\isacharparenright}{\kern0pt}\ {\isasymunion}\ domain{\isacharparenleft}{\kern0pt}Pn{\isacharunderscore}{\kern0pt}auto{\isacharparenleft}{\kern0pt}{\isasympi}{\isacharparenright}{\kern0pt}\ {\isacharbackquote}{\kern0pt}\ y{\isacharparenright}{\kern0pt}{\isachardoublequoteclose}\ \isanewline
\ \ \ \ \ \ \ \ \ \ \isacommand{apply}\isamarkupfalse%
\ {\isacharparenleft}{\kern0pt}rule{\isacharunderscore}{\kern0pt}tac\ A{\isacharequal}{\kern0pt}{\isachardoublequoteopen}domain{\isacharparenleft}{\kern0pt}x{\isacharparenright}{\kern0pt}{\isachardoublequoteclose}\ \isakeyword{and}\ B{\isacharequal}{\kern0pt}{\isachardoublequoteopen}domain{\isacharparenleft}{\kern0pt}y{\isacharparenright}{\kern0pt}{\isachardoublequoteclose}\ \isakeyword{and}\ c{\isacharequal}{\kern0pt}s\ \isakeyword{in}\ UnE{\isacharparenright}{\kern0pt}\ \ \isanewline
\ \ \ \ \ \ \ \ \ \ \isacommand{using}\isamarkupfalse%
\ sin\ \isanewline
\ \ \ \ \ \ \ \ \ \ \ \ \isacommand{apply}\isamarkupfalse%
\ simp\ \isanewline
\ \ \ \ \ \ \ \ \ \ \ \isacommand{apply}\isamarkupfalse%
\ {\isacharparenleft}{\kern0pt}rule{\isacharunderscore}{\kern0pt}tac\ UnI{\isadigit{1}}{\isacharparenright}{\kern0pt}\ \isanewline
\ \ \ \ \ \ \ \ \ \ \ \isacommand{apply}\isamarkupfalse%
\ {\isacharparenleft}{\kern0pt}rule{\isacharunderscore}{\kern0pt}tac\ domain{\isacharunderscore}{\kern0pt}lemma{\isacharparenright}{\kern0pt}\ \isanewline
\ \ \ \ \ \ \ \ \ \ \isacommand{using}\isamarkupfalse%
\ xpname\ \isanewline
\ \ \ \ \ \ \ \ \ \ \ \ \isacommand{apply}\isamarkupfalse%
\ simp\ \isanewline
\ \ \ \ \ \ \ \ \ \ \ \isacommand{apply}\isamarkupfalse%
\ simp\ \isanewline
\ \ \ \ \ \ \ \ \ \ \isacommand{apply}\isamarkupfalse%
\ {\isacharparenleft}{\kern0pt}rule{\isacharunderscore}{\kern0pt}tac\ UnI{\isadigit{2}}{\isacharparenright}{\kern0pt}\ \isanewline
\ \ \ \ \ \ \ \ \ \ \isacommand{apply}\isamarkupfalse%
\ {\isacharparenleft}{\kern0pt}rule{\isacharunderscore}{\kern0pt}tac\ domain{\isacharunderscore}{\kern0pt}lemma{\isacharparenright}{\kern0pt}\ \isanewline
\ \ \ \ \ \ \ \ \ \ \isacommand{using}\isamarkupfalse%
\ ypname\ \isanewline
\ \ \ \ \ \ \ \ \ \ \ \isacommand{apply}\isamarkupfalse%
\ simp{\isacharunderscore}{\kern0pt}all\ \isanewline
\ \ \ \ \ \ \ \ \ \ \isacommand{done}\isamarkupfalse%
\ \isanewline
\isanewline
\ \ \ \ \ \ \ \ \isacommand{have}\isamarkupfalse%
\ q{\isacharprime}{\kern0pt}H\ {\isacharcolon}{\kern0pt}\ {\isachardoublequoteopen}{\isasympi}{\isacharbackquote}{\kern0pt}q\ {\isasymin}\ P\ {\isasymand}\ {\isasympi}{\isacharbackquote}{\kern0pt}q\ {\isasympreceq}\ {\isasympi}{\isacharbackquote}{\kern0pt}p{\isachardoublequoteclose}\ \isanewline
\ \ \ \ \ \ \ \ \ \ \isacommand{using}\isamarkupfalse%
\ qinP\ qlep\ P{\isacharunderscore}{\kern0pt}auto{\isacharunderscore}{\kern0pt}value\ piauto\ P{\isacharunderscore}{\kern0pt}auto{\isacharunderscore}{\kern0pt}preserves{\isacharunderscore}{\kern0pt}leq\ pinP\ \isacommand{by}\isamarkupfalse%
\ auto\ \isanewline
\isanewline
\ \ \ \ \ \ \ \ \isacommand{have}\isamarkupfalse%
\ {\isachardoublequoteopen}forces{\isacharunderscore}{\kern0pt}mem{\isacharparenleft}{\kern0pt}q{\isacharcomma}{\kern0pt}\ s{\isacharcomma}{\kern0pt}\ x{\isacharparenright}{\kern0pt}\ {\isasymlongleftrightarrow}\ forces{\isacharunderscore}{\kern0pt}mem{\isacharparenleft}{\kern0pt}{\isasympi}{\isacharbackquote}{\kern0pt}q{\isacharcomma}{\kern0pt}\ Pn{\isacharunderscore}{\kern0pt}auto{\isacharparenleft}{\kern0pt}{\isasympi}{\isacharparenright}{\kern0pt}{\isacharbackquote}{\kern0pt}s{\isacharcomma}{\kern0pt}\ Pn{\isacharunderscore}{\kern0pt}auto{\isacharparenleft}{\kern0pt}{\isasympi}{\isacharparenright}{\kern0pt}\ {\isacharbackquote}{\kern0pt}\ x{\isacharparenright}{\kern0pt}{\isachardoublequoteclose}\ \isanewline
\ \ \ \ \ \ \ \ \ \ \isacommand{apply}\isamarkupfalse%
\ {\isacharparenleft}{\kern0pt}rule{\isacharunderscore}{\kern0pt}tac\ P{\isacharequal}{\kern0pt}{\isachardoublequoteopen}MEM{\isacharparenleft}{\kern0pt}q{\isacharcomma}{\kern0pt}s{\isacharcomma}{\kern0pt}x{\isacharparenright}{\kern0pt}{\isachardoublequoteclose}\ \isakeyword{in}\ mp{\isacharparenright}{\kern0pt}\ \isanewline
\ \ \ \ \ \ \ \ \ \ \isacommand{apply}\isamarkupfalse%
\ {\isacharparenleft}{\kern0pt}simp\ add{\isacharcolon}{\kern0pt}MEM{\isacharunderscore}{\kern0pt}def{\isacharparenright}{\kern0pt}\ \isanewline
\ \ \ \ \ \ \ \ \ \ \isacommand{apply}\isamarkupfalse%
\ {\isacharparenleft}{\kern0pt}rule\ MEMH{\isacharbrackleft}{\kern0pt}of\ {\isachardoublequoteopen}P{\isacharunderscore}{\kern0pt}rank{\isacharparenleft}{\kern0pt}s{\isacharparenright}{\kern0pt}{\isachardoublequoteclose}{\isacharbrackright}{\kern0pt}{\isacharparenright}{\kern0pt}\isanewline
\ \ \ \ \ \ \ \ \ \ \isacommand{using}\isamarkupfalse%
\ srank\ ltD\ \isanewline
\ \ \ \ \ \ \ \ \ \ \ \ \ \ \ \isacommand{apply}\isamarkupfalse%
\ simp\ \isanewline
\ \ \ \ \ \ \ \ \ \ \isacommand{using}\isamarkupfalse%
\ P{\isacharunderscore}{\kern0pt}name{\isacharunderscore}{\kern0pt}domain{\isacharunderscore}{\kern0pt}P{\isacharunderscore}{\kern0pt}name{\isacharprime}{\kern0pt}\ sin\ xpname\ ypname\ \isanewline
\ \ \ \ \ \ \ \ \ \ \ \ \ \ \isacommand{apply}\isamarkupfalse%
\ blast\ \isanewline
\ \ \ \ \ \ \ \ \ \ \isacommand{using}\isamarkupfalse%
\ xpname\ qinP\ le{\isacharunderscore}{\kern0pt}refl\ xrank\ P{\isacharunderscore}{\kern0pt}rank{\isacharunderscore}{\kern0pt}ord\ \isanewline
\ \ \ \ \ \ \ \ \ \ \ \ \ \isacommand{apply}\isamarkupfalse%
\ simp{\isacharunderscore}{\kern0pt}all\ \isanewline
\ \ \ \ \ \ \ \ \ \ \isacommand{done}\isamarkupfalse%
\ \isanewline
\ \ \ \ \ \ \ \ \isacommand{also}\isamarkupfalse%
\ \isacommand{have}\isamarkupfalse%
\ {\isachardoublequoteopen}{\isachardot}{\kern0pt}{\isachardot}{\kern0pt}{\isachardot}{\kern0pt}\ {\isasymlongleftrightarrow}\ forces{\isacharunderscore}{\kern0pt}mem{\isacharparenleft}{\kern0pt}{\isasympi}{\isacharbackquote}{\kern0pt}q{\isacharcomma}{\kern0pt}\ Pn{\isacharunderscore}{\kern0pt}auto{\isacharparenleft}{\kern0pt}{\isasympi}{\isacharparenright}{\kern0pt}{\isacharbackquote}{\kern0pt}s{\isacharcomma}{\kern0pt}\ Pn{\isacharunderscore}{\kern0pt}auto{\isacharparenleft}{\kern0pt}{\isasympi}{\isacharparenright}{\kern0pt}\ {\isacharbackquote}{\kern0pt}\ y{\isacharparenright}{\kern0pt}{\isachardoublequoteclose}\ \isanewline
\ \ \ \ \ \ \ \ \ \ \isacommand{using}\isamarkupfalse%
\ assm\ q{\isacharprime}{\kern0pt}H\ s{\isacharprime}{\kern0pt}in\ \isacommand{by}\isamarkupfalse%
\ auto\ \isanewline
\ \ \ \ \ \ \ \ \isacommand{also}\isamarkupfalse%
\ \isacommand{have}\isamarkupfalse%
\ {\isachardoublequoteopen}{\isachardot}{\kern0pt}{\isachardot}{\kern0pt}{\isachardot}{\kern0pt}\ {\isasymlongleftrightarrow}\ forces{\isacharunderscore}{\kern0pt}mem{\isacharparenleft}{\kern0pt}q{\isacharcomma}{\kern0pt}\ s{\isacharcomma}{\kern0pt}\ y{\isacharparenright}{\kern0pt}{\isachardoublequoteclose}\ \isanewline
\ \ \ \ \ \ \ \ \ \ \isacommand{apply}\isamarkupfalse%
\ {\isacharparenleft}{\kern0pt}rule{\isacharunderscore}{\kern0pt}tac\ P{\isacharequal}{\kern0pt}{\isachardoublequoteopen}MEM{\isacharparenleft}{\kern0pt}q{\isacharcomma}{\kern0pt}s{\isacharcomma}{\kern0pt}y{\isacharparenright}{\kern0pt}{\isachardoublequoteclose}\ \isakeyword{in}\ mp{\isacharparenright}{\kern0pt}\ \isanewline
\ \ \ \ \ \ \ \ \ \ \isacommand{apply}\isamarkupfalse%
\ {\isacharparenleft}{\kern0pt}simp\ add{\isacharcolon}{\kern0pt}MEM{\isacharunderscore}{\kern0pt}def{\isacharparenright}{\kern0pt}\ \isanewline
\ \ \ \ \ \ \ \ \ \ \isacommand{apply}\isamarkupfalse%
\ {\isacharparenleft}{\kern0pt}rule\ MEMH{\isacharbrackleft}{\kern0pt}of\ {\isachardoublequoteopen}P{\isacharunderscore}{\kern0pt}rank{\isacharparenleft}{\kern0pt}s{\isacharparenright}{\kern0pt}{\isachardoublequoteclose}{\isacharbrackright}{\kern0pt}{\isacharparenright}{\kern0pt}\isanewline
\ \ \ \ \ \ \ \ \ \ \isacommand{using}\isamarkupfalse%
\ srank\ ltD\ \isanewline
\ \ \ \ \ \ \ \ \ \ \ \ \ \ \ \isacommand{apply}\isamarkupfalse%
\ simp\ \isanewline
\ \ \ \ \ \ \ \ \ \ \isacommand{using}\isamarkupfalse%
\ P{\isacharunderscore}{\kern0pt}name{\isacharunderscore}{\kern0pt}domain{\isacharunderscore}{\kern0pt}P{\isacharunderscore}{\kern0pt}name{\isacharprime}{\kern0pt}\ sin\ xpname\ ypname\ \isanewline
\ \ \ \ \ \ \ \ \ \ \ \ \ \ \isacommand{apply}\isamarkupfalse%
\ blast\ \isanewline
\ \ \ \ \ \ \ \ \ \ \isacommand{using}\isamarkupfalse%
\ ypname\ qinP\ le{\isacharunderscore}{\kern0pt}refl\ yrank\ P{\isacharunderscore}{\kern0pt}rank{\isacharunderscore}{\kern0pt}ord\ \isanewline
\ \ \ \ \ \ \ \ \ \ \ \ \ \isacommand{apply}\isamarkupfalse%
\ simp{\isacharunderscore}{\kern0pt}all\ \isanewline
\ \ \ \ \ \ \ \ \ \ \isacommand{done}\isamarkupfalse%
\ \isanewline
\ \ \ \ \ \ \ \ \isacommand{finally}\isamarkupfalse%
\ \isacommand{show}\isamarkupfalse%
\ {\isachardoublequoteopen}forces{\isacharunderscore}{\kern0pt}mem{\isacharparenleft}{\kern0pt}q{\isacharcomma}{\kern0pt}\ s{\isacharcomma}{\kern0pt}\ x{\isacharparenright}{\kern0pt}\ {\isasymlongleftrightarrow}\ forces{\isacharunderscore}{\kern0pt}mem{\isacharparenleft}{\kern0pt}q{\isacharcomma}{\kern0pt}\ s{\isacharcomma}{\kern0pt}\ y{\isacharparenright}{\kern0pt}\ {\isachardoublequoteclose}\ \isacommand{by}\isamarkupfalse%
\ simp\ \isanewline
\ \ \ \ \ \ \isacommand{qed}\isamarkupfalse%
\isanewline
\ \ \ \ \ \ \isacommand{also}\isamarkupfalse%
\ \isacommand{have}\isamarkupfalse%
\ {\isachardoublequoteopen}{\isachardot}{\kern0pt}{\isachardot}{\kern0pt}{\isachardot}{\kern0pt}\ {\isasymlongleftrightarrow}\ forces{\isacharunderscore}{\kern0pt}eq{\isacharparenleft}{\kern0pt}{\isasympi}{\isacharbackquote}{\kern0pt}p{\isacharcomma}{\kern0pt}\ Pn{\isacharunderscore}{\kern0pt}auto{\isacharparenleft}{\kern0pt}{\isasympi}{\isacharparenright}{\kern0pt}{\isacharbackquote}{\kern0pt}x{\isacharcomma}{\kern0pt}\ Pn{\isacharunderscore}{\kern0pt}auto{\isacharparenleft}{\kern0pt}{\isasympi}{\isacharparenright}{\kern0pt}{\isacharbackquote}{\kern0pt}y{\isacharparenright}{\kern0pt}{\isachardoublequoteclose}\ \isanewline
\ \ \ \ \ \ \ \ \isacommand{apply}\isamarkupfalse%
{\isacharparenleft}{\kern0pt}rule\ iff{\isacharunderscore}{\kern0pt}flip{\isacharparenright}{\kern0pt}\isanewline
\ \ \ \ \ \ \ \ \isacommand{apply}\isamarkupfalse%
{\isacharparenleft}{\kern0pt}rule{\isacharunderscore}{\kern0pt}tac\ def{\isacharunderscore}{\kern0pt}forces{\isacharunderscore}{\kern0pt}eq{\isacharparenright}{\kern0pt}\ \isanewline
\ \ \ \ \ \ \ \ \isacommand{using}\isamarkupfalse%
\ pinP\ piauto\ P{\isacharunderscore}{\kern0pt}auto{\isacharunderscore}{\kern0pt}value\ \isanewline
\ \ \ \ \ \ \ \ \isacommand{by}\isamarkupfalse%
\ auto\ \isanewline
\ \ \ \ \ \ \isacommand{finally}\isamarkupfalse%
\ \isacommand{show}\isamarkupfalse%
\ {\isachardoublequoteopen}EQ{\isacharparenleft}{\kern0pt}p{\isacharcomma}{\kern0pt}x{\isacharcomma}{\kern0pt}y{\isacharparenright}{\kern0pt}{\isachardoublequoteclose}\isanewline
\ \ \ \ \ \ \ \ \isacommand{unfolding}\isamarkupfalse%
\ EQ{\isacharunderscore}{\kern0pt}def\ \isacommand{by}\isamarkupfalse%
\ auto\ \isanewline
\ \ \ \ \isacommand{qed}\isamarkupfalse%
\isanewline
\isanewline
\ \ \ \ \isacommand{have}\isamarkupfalse%
\ main\ {\isacharcolon}{\kern0pt}\ {\isachardoublequoteopen}{\isasymAnd}a{\isachardot}{\kern0pt}\ Ord{\isacharparenleft}{\kern0pt}a{\isacharparenright}{\kern0pt}\ {\isasymlongrightarrow}\ Q{\isacharparenleft}{\kern0pt}MEM{\isacharcomma}{\kern0pt}\ a{\isacharcomma}{\kern0pt}\ a{\isacharparenright}{\kern0pt}\ {\isasymand}\ Q{\isacharparenleft}{\kern0pt}EQ{\isacharcomma}{\kern0pt}\ a{\isacharcomma}{\kern0pt}\ a{\isacharparenright}{\kern0pt}{\isachardoublequoteclose}\ \isanewline
\ \ \ \ \ \ \isacommand{apply}\isamarkupfalse%
\ {\isacharparenleft}{\kern0pt}rule{\isacharunderscore}{\kern0pt}tac\ eps{\isacharunderscore}{\kern0pt}induct{\isacharparenright}{\kern0pt}\ \isanewline
\ \ \ \ \ \ \isacommand{apply}\isamarkupfalse%
\ {\isacharparenleft}{\kern0pt}rule\ impI{\isacharparenright}{\kern0pt}\isanewline
\ \ \ \ \isacommand{proof}\isamarkupfalse%
\ {\isacharminus}{\kern0pt}\ \isanewline
\ \ \ \ \ \ \isacommand{fix}\isamarkupfalse%
\ a\ \isacommand{assume}\isamarkupfalse%
\ ih{\isacharcolon}{\kern0pt}\ {\isachardoublequoteopen}{\isasymforall}b\ {\isasymin}\ a{\isachardot}{\kern0pt}\ Ord{\isacharparenleft}{\kern0pt}b{\isacharparenright}{\kern0pt}\ {\isasymlongrightarrow}\ Q{\isacharparenleft}{\kern0pt}MEM{\isacharcomma}{\kern0pt}\ b{\isacharcomma}{\kern0pt}b{\isacharparenright}{\kern0pt}\ {\isasymand}\ Q{\isacharparenleft}{\kern0pt}EQ{\isacharcomma}{\kern0pt}\ b{\isacharcomma}{\kern0pt}b{\isacharparenright}{\kern0pt}{\isachardoublequoteclose}\ \isanewline
\ \ \ \ \ \ \isakeyword{and}\ orda\ {\isacharcolon}{\kern0pt}\ {\isachardoublequoteopen}Ord{\isacharparenleft}{\kern0pt}a{\isacharparenright}{\kern0pt}{\isachardoublequoteclose}\ \isanewline
\ \ \ \ \ \ \isacommand{then}\isamarkupfalse%
\ \isacommand{have}\isamarkupfalse%
\ {\isachardoublequoteopen}{\isasymforall}b\ {\isasymin}\ a{\isachardot}{\kern0pt}\ Q{\isacharparenleft}{\kern0pt}MEM{\isacharcomma}{\kern0pt}\ b{\isacharcomma}{\kern0pt}\ a{\isacharparenright}{\kern0pt}{\isachardoublequoteclose}\ \isanewline
\ \ \ \ \ \ \ \ \isacommand{apply}\isamarkupfalse%
\ {\isacharparenleft}{\kern0pt}rule{\isacharunderscore}{\kern0pt}tac\ MEM{\isacharunderscore}{\kern0pt}step{\isacharparenright}{\kern0pt}\ \isanewline
\ \ \ \ \ \ \ \ \ \isacommand{apply}\isamarkupfalse%
\ simp\ \isanewline
\ \ \ \ \ \ \ \ \isacommand{using}\isamarkupfalse%
\ orda\ Ord{\isacharunderscore}{\kern0pt}in{\isacharunderscore}{\kern0pt}Ord\ \isanewline
\ \ \ \ \ \ \ \ \isacommand{by}\isamarkupfalse%
\ auto\ \isanewline
\ \ \ \ \ \ \isacommand{then}\isamarkupfalse%
\ \isacommand{have}\isamarkupfalse%
\ H\ {\isacharcolon}{\kern0pt}\ {\isachardoublequoteopen}Q{\isacharparenleft}{\kern0pt}EQ{\isacharcomma}{\kern0pt}\ a{\isacharcomma}{\kern0pt}a{\isacharparenright}{\kern0pt}{\isachardoublequoteclose}\ \isanewline
\ \ \ \ \ \ \ \ \isacommand{apply}\isamarkupfalse%
\ {\isacharparenleft}{\kern0pt}rule{\isacharunderscore}{\kern0pt}tac\ EQ{\isacharunderscore}{\kern0pt}step{\isacharparenright}{\kern0pt}\ \isanewline
\ \ \ \ \ \ \ \ \isacommand{using}\isamarkupfalse%
\ orda\ \isanewline
\ \ \ \ \ \ \ \ \ \isacommand{apply}\isamarkupfalse%
\ auto\ \isanewline
\ \ \ \ \ \ \ \ \isacommand{done}\isamarkupfalse%
\ \isanewline
\ \ \ \ \ \ \isacommand{then}\isamarkupfalse%
\ \isacommand{have}\isamarkupfalse%
\ {\isachardoublequoteopen}{\isasymforall}b\ {\isasymin}\ succ{\isacharparenleft}{\kern0pt}a{\isacharparenright}{\kern0pt}{\isachardot}{\kern0pt}\ Q{\isacharparenleft}{\kern0pt}EQ{\isacharcomma}{\kern0pt}\ b{\isacharcomma}{\kern0pt}\ b{\isacharparenright}{\kern0pt}{\isachardoublequoteclose}\ \isanewline
\ \ \ \ \ \ \ \ \isacommand{apply}\isamarkupfalse%
\ clarify\ \isanewline
\ \ \ \ \ \ \ \ \isacommand{apply}\isamarkupfalse%
\ {\isacharparenleft}{\kern0pt}rule{\isacharunderscore}{\kern0pt}tac\ i{\isacharequal}{\kern0pt}b\ \isakeyword{and}\ j{\isacharequal}{\kern0pt}a\ \isakeyword{in}\ leE{\isacharparenright}{\kern0pt}\ \isanewline
\ \ \ \ \ \ \ \ \isacommand{using}\isamarkupfalse%
\ ltI\ orda\ \isanewline
\ \ \ \ \ \ \ \ \ \ \isacommand{apply}\isamarkupfalse%
\ simp\ \isanewline
\ \ \ \ \ \ \ \ \isacommand{using}\isamarkupfalse%
\ ltD\ ih\ orda\ lt{\isacharunderscore}{\kern0pt}Ord\ \isanewline
\ \ \ \ \ \ \ \ \ \isacommand{apply}\isamarkupfalse%
\ auto\ \isanewline
\ \ \ \ \ \ \ \ \isacommand{done}\isamarkupfalse%
\ \isanewline
\ \ \ \ \ \ \isacommand{then}\isamarkupfalse%
\ \isacommand{have}\isamarkupfalse%
\ {\isachardoublequoteopen}{\isasymforall}b\ {\isasymin}\ succ{\isacharparenleft}{\kern0pt}a{\isacharparenright}{\kern0pt}{\isachardot}{\kern0pt}\ Q{\isacharparenleft}{\kern0pt}MEM{\isacharcomma}{\kern0pt}\ b{\isacharcomma}{\kern0pt}\ succ{\isacharparenleft}{\kern0pt}a{\isacharparenright}{\kern0pt}{\isacharparenright}{\kern0pt}{\isachardoublequoteclose}\isanewline
\ \ \ \ \ \ \ \ \isacommand{apply}\isamarkupfalse%
\ {\isacharparenleft}{\kern0pt}rule{\isacharunderscore}{\kern0pt}tac\ MEM{\isacharunderscore}{\kern0pt}step{\isacharparenright}{\kern0pt}\ \isanewline
\ \ \ \ \ \ \ \ \isacommand{using}\isamarkupfalse%
\ orda\ \isanewline
\ \ \ \ \ \ \ \ \ \isacommand{apply}\isamarkupfalse%
\ auto\ \isanewline
\ \ \ \ \ \ \ \ \isacommand{done}\isamarkupfalse%
\ \isanewline
\ \ \ \ \ \ \isacommand{then}\isamarkupfalse%
\ \isacommand{have}\isamarkupfalse%
\ {\isachardoublequoteopen}Q{\isacharparenleft}{\kern0pt}MEM{\isacharcomma}{\kern0pt}\ a{\isacharcomma}{\kern0pt}\ succ{\isacharparenleft}{\kern0pt}a{\isacharparenright}{\kern0pt}{\isacharparenright}{\kern0pt}{\isachardoublequoteclose}\ \isacommand{by}\isamarkupfalse%
\ auto\ \isanewline
\ \ \ \ \ \ \isacommand{then}\isamarkupfalse%
\ \isacommand{have}\isamarkupfalse%
\ {\isachardoublequoteopen}Q{\isacharparenleft}{\kern0pt}MEM{\isacharcomma}{\kern0pt}\ a{\isacharcomma}{\kern0pt}\ a{\isacharparenright}{\kern0pt}{\isachardoublequoteclose}\ \isanewline
\ \ \ \ \ \ \ \ \isacommand{unfolding}\isamarkupfalse%
\ Q{\isacharunderscore}{\kern0pt}def\ \isanewline
\ \ \ \ \ \ \ \ \isacommand{apply}\isamarkupfalse%
\ clarify\ \isanewline
\ \ \ \ \ \ \ \ \isacommand{apply}\isamarkupfalse%
\ {\isacharparenleft}{\kern0pt}rule{\isacharunderscore}{\kern0pt}tac\ P{\isacharequal}{\kern0pt}{\isachardoublequoteopen}P{\isacharunderscore}{\kern0pt}rank{\isacharparenleft}{\kern0pt}y{\isacharparenright}{\kern0pt}\ {\isasymle}\ succ{\isacharparenleft}{\kern0pt}a{\isacharparenright}{\kern0pt}{\isachardoublequoteclose}\ \isakeyword{in}\ mp{\isacharparenright}{\kern0pt}\ \isanewline
\ \ \ \ \ \ \ \ \ \isacommand{apply}\isamarkupfalse%
\ auto\ \isanewline
\ \ \ \ \ \ \ \ \isacommand{apply}\isamarkupfalse%
\ {\isacharparenleft}{\kern0pt}rule{\isacharunderscore}{\kern0pt}tac\ j{\isacharequal}{\kern0pt}a\ \isakeyword{in}\ le{\isacharunderscore}{\kern0pt}trans{\isacharparenright}{\kern0pt}\ \isanewline
\ \ \ \ \ \ \ \ \ \isacommand{apply}\isamarkupfalse%
\ simp\ \isanewline
\ \ \ \ \ \ \ \ \isacommand{apply}\isamarkupfalse%
\ {\isacharparenleft}{\kern0pt}rule{\isacharunderscore}{\kern0pt}tac\ j{\isacharequal}{\kern0pt}{\isachardoublequoteopen}succ{\isacharparenleft}{\kern0pt}a{\isacharparenright}{\kern0pt}{\isachardoublequoteclose}\ \isakeyword{in}\ lt{\isacharunderscore}{\kern0pt}trans{\isacharparenright}{\kern0pt}\ \isanewline
\ \ \ \ \ \ \ \ \isacommand{using}\isamarkupfalse%
\ le{\isacharunderscore}{\kern0pt}refl\ orda\ \isanewline
\ \ \ \ \ \ \ \ \isacommand{by}\isamarkupfalse%
\ auto\ \isanewline
\ \ \ \ \ \ \isacommand{then}\isamarkupfalse%
\ \isacommand{show}\isamarkupfalse%
\ {\isachardoublequoteopen}\ Q{\isacharparenleft}{\kern0pt}{\isasymlambda}a\ b\ c{\isachardot}{\kern0pt}\ MEM{\isacharparenleft}{\kern0pt}a{\isacharcomma}{\kern0pt}\ b{\isacharcomma}{\kern0pt}\ c{\isacharparenright}{\kern0pt}{\isacharcomma}{\kern0pt}\ a{\isacharcomma}{\kern0pt}\ a{\isacharparenright}{\kern0pt}\ {\isasymand}\ Q{\isacharparenleft}{\kern0pt}{\isasymlambda}a\ b\ c{\isachardot}{\kern0pt}\ EQ{\isacharparenleft}{\kern0pt}a{\isacharcomma}{\kern0pt}\ b{\isacharcomma}{\kern0pt}\ c{\isacharparenright}{\kern0pt}{\isacharcomma}{\kern0pt}\ a{\isacharcomma}{\kern0pt}\ a{\isacharparenright}{\kern0pt}{\isachardoublequoteclose}\ \isanewline
\ \ \ \ \ \ \ \ \isacommand{using}\isamarkupfalse%
\ H\ \isacommand{by}\isamarkupfalse%
\ auto\ \isanewline
\ \ \ \ \isacommand{qed}\isamarkupfalse%
\isanewline
\ \ \ \ \ \ \isanewline
\ \ \ \ \isacommand{fix}\isamarkupfalse%
\ x\ y\ p\ \isacommand{assume}\isamarkupfalse%
\ pinP\ {\isacharcolon}{\kern0pt}\ {\isachardoublequoteopen}p\ {\isasymin}\ P{\isachardoublequoteclose}\ \isakeyword{and}\ xpname\ {\isacharcolon}{\kern0pt}\ {\isachardoublequoteopen}x\ {\isasymin}\ P{\isacharunderscore}{\kern0pt}names{\isachardoublequoteclose}\ \isakeyword{and}\ ypname{\isacharcolon}{\kern0pt}\ {\isachardoublequoteopen}y\ {\isasymin}\ P{\isacharunderscore}{\kern0pt}names{\isachardoublequoteclose}\ \isanewline
\isanewline
\ \ \ \ \isacommand{define}\isamarkupfalse%
\ r\ \isakeyword{where}\ {\isachardoublequoteopen}r\ {\isasymequiv}\ if\ P{\isacharunderscore}{\kern0pt}rank{\isacharparenleft}{\kern0pt}x{\isacharparenright}{\kern0pt}\ {\isasymle}\ P{\isacharunderscore}{\kern0pt}rank{\isacharparenleft}{\kern0pt}y{\isacharparenright}{\kern0pt}\ then\ P{\isacharunderscore}{\kern0pt}rank{\isacharparenleft}{\kern0pt}y{\isacharparenright}{\kern0pt}\ else\ P{\isacharunderscore}{\kern0pt}rank{\isacharparenleft}{\kern0pt}x{\isacharparenright}{\kern0pt}{\isachardoublequoteclose}\ \isanewline
\ \ \ \ \isacommand{have}\isamarkupfalse%
\ roed\ {\isacharcolon}{\kern0pt}\ {\isachardoublequoteopen}Ord{\isacharparenleft}{\kern0pt}r{\isacharparenright}{\kern0pt}{\isachardoublequoteclose}\ \isanewline
\ \ \ \ \ \ \isacommand{apply}\isamarkupfalse%
\ {\isacharparenleft}{\kern0pt}cases\ {\isachardoublequoteopen}P{\isacharunderscore}{\kern0pt}rank{\isacharparenleft}{\kern0pt}x{\isacharparenright}{\kern0pt}\ {\isasymle}\ P{\isacharunderscore}{\kern0pt}rank{\isacharparenleft}{\kern0pt}y{\isacharparenright}{\kern0pt}{\isachardoublequoteclose}{\isacharparenright}{\kern0pt}\ \ \isanewline
\ \ \ \ \ \ \isacommand{using}\isamarkupfalse%
\ r{\isacharunderscore}{\kern0pt}def\ P{\isacharunderscore}{\kern0pt}rank{\isacharunderscore}{\kern0pt}ord\ \isacommand{by}\isamarkupfalse%
\ auto\ \isanewline
\isanewline
\ \ \ \ \isacommand{have}\isamarkupfalse%
\ rle{\isacharcolon}{\kern0pt}\ {\isachardoublequoteopen}P{\isacharunderscore}{\kern0pt}rank{\isacharparenleft}{\kern0pt}x{\isacharparenright}{\kern0pt}\ {\isasymle}\ r\ {\isasymand}\ P{\isacharunderscore}{\kern0pt}rank{\isacharparenleft}{\kern0pt}y{\isacharparenright}{\kern0pt}\ {\isasymle}\ r{\isachardoublequoteclose}\ \isanewline
\ \ \ \ \ \ \isacommand{apply}\isamarkupfalse%
\ {\isacharparenleft}{\kern0pt}cases\ {\isachardoublequoteopen}P{\isacharunderscore}{\kern0pt}rank{\isacharparenleft}{\kern0pt}x{\isacharparenright}{\kern0pt}\ {\isasymle}\ P{\isacharunderscore}{\kern0pt}rank{\isacharparenleft}{\kern0pt}y{\isacharparenright}{\kern0pt}{\isachardoublequoteclose}{\isacharparenright}{\kern0pt}\ \isanewline
\ \ \ \ \ \ \isacommand{apply}\isamarkupfalse%
\ {\isacharparenleft}{\kern0pt}rule{\isacharunderscore}{\kern0pt}tac\ P{\isacharequal}{\kern0pt}{\isachardoublequoteopen}r\ {\isacharequal}{\kern0pt}\ P{\isacharunderscore}{\kern0pt}rank{\isacharparenleft}{\kern0pt}y{\isacharparenright}{\kern0pt}{\isachardoublequoteclose}\ \isakeyword{in}\ mp{\isacharparenright}{\kern0pt}\ \isanewline
\ \ \ \ \ \ \ \ \isacommand{apply}\isamarkupfalse%
\ simp\ \isanewline
\ \ \ \ \ \ \isacommand{using}\isamarkupfalse%
\ P{\isacharunderscore}{\kern0pt}rank{\isacharunderscore}{\kern0pt}ord\ \isanewline
\ \ \ \ \ \ \ \ \isacommand{apply}\isamarkupfalse%
\ simp\ \isanewline
\ \ \ \ \ \ \ \isacommand{apply}\isamarkupfalse%
\ {\isacharparenleft}{\kern0pt}simp\ add{\isacharcolon}{\kern0pt}r{\isacharunderscore}{\kern0pt}def{\isacharparenright}{\kern0pt}\ \isanewline
\ \ \ \ \ \ \isacommand{apply}\isamarkupfalse%
\ {\isacharparenleft}{\kern0pt}rule{\isacharunderscore}{\kern0pt}tac\ P{\isacharequal}{\kern0pt}{\isachardoublequoteopen}r\ {\isacharequal}{\kern0pt}\ P{\isacharunderscore}{\kern0pt}rank{\isacharparenleft}{\kern0pt}x{\isacharparenright}{\kern0pt}{\isachardoublequoteclose}\ \isakeyword{in}\ mp{\isacharparenright}{\kern0pt}\ \isanewline
\ \ \ \ \ \ \isacommand{using}\isamarkupfalse%
\ P{\isacharunderscore}{\kern0pt}rank{\isacharunderscore}{\kern0pt}ord\ \isanewline
\ \ \ \ \ \ \ \isacommand{apply}\isamarkupfalse%
\ simp\ \isanewline
\ \ \ \ \ \ \isacommand{apply}\isamarkupfalse%
\ {\isacharparenleft}{\kern0pt}rule{\isacharunderscore}{\kern0pt}tac\ P{\isacharequal}{\kern0pt}{\isachardoublequoteopen}P{\isacharunderscore}{\kern0pt}rank{\isacharparenleft}{\kern0pt}y{\isacharparenright}{\kern0pt}\ {\isacharless}{\kern0pt}\ P{\isacharunderscore}{\kern0pt}rank{\isacharparenleft}{\kern0pt}x{\isacharparenright}{\kern0pt}{\isachardoublequoteclose}\ \isakeyword{in}\ mp{\isacharparenright}{\kern0pt}\ \isanewline
\ \ \ \ \ \ \isacommand{using}\isamarkupfalse%
\ lt{\isacharunderscore}{\kern0pt}succ{\isacharunderscore}{\kern0pt}lt\ \isanewline
\ \ \ \ \ \ \ \ \isacommand{apply}\isamarkupfalse%
\ simp\ \isanewline
\ \ \ \ \ \ \isacommand{using}\isamarkupfalse%
\ not{\isacharunderscore}{\kern0pt}le{\isacharunderscore}{\kern0pt}iff{\isacharunderscore}{\kern0pt}lt\ P{\isacharunderscore}{\kern0pt}rank{\isacharunderscore}{\kern0pt}ord\ \isanewline
\ \ \ \ \ \ \ \isacommand{apply}\isamarkupfalse%
\ simp\ \isanewline
\ \ \ \ \ \ \isacommand{apply}\isamarkupfalse%
\ {\isacharparenleft}{\kern0pt}simp\ add{\isacharcolon}{\kern0pt}r{\isacharunderscore}{\kern0pt}def{\isacharparenright}{\kern0pt}\ \isanewline
\ \ \ \ \ \ \isacommand{done}\isamarkupfalse%
\ \isanewline
\ \ \ \ \isanewline
\ \ \ \ \isacommand{have}\isamarkupfalse%
\ H\ {\isacharcolon}{\kern0pt}\ {\isachardoublequoteopen}Q{\isacharparenleft}{\kern0pt}MEM{\isacharcomma}{\kern0pt}\ r{\isacharcomma}{\kern0pt}r{\isacharparenright}{\kern0pt}\ {\isasymand}\ Q{\isacharparenleft}{\kern0pt}EQ{\isacharcomma}{\kern0pt}\ r{\isacharcomma}{\kern0pt}r{\isacharparenright}{\kern0pt}{\isachardoublequoteclose}\ \isanewline
\ \ \ \ \ \ \isacommand{using}\isamarkupfalse%
\ roed\ main\ \isacommand{by}\isamarkupfalse%
\ auto\isanewline
\ \ \ \ \isacommand{have}\isamarkupfalse%
\ H{\isadigit{1}}{\isacharcolon}{\kern0pt}\ {\isachardoublequoteopen}MEM{\isacharparenleft}{\kern0pt}p{\isacharcomma}{\kern0pt}\ x{\isacharcomma}{\kern0pt}\ y{\isacharparenright}{\kern0pt}{\isachardoublequoteclose}\ \isanewline
\ \ \ \ \ \ \isacommand{using}\isamarkupfalse%
\ H\ \isacommand{unfolding}\isamarkupfalse%
\ Q{\isacharunderscore}{\kern0pt}def\ \isacommand{using}\isamarkupfalse%
\ xpname\ ypname\ pinP\ rle\ \isacommand{apply}\isamarkupfalse%
\ auto\ \isacommand{done}\isamarkupfalse%
\ \isanewline
\ \ \ \ \isacommand{have}\isamarkupfalse%
\ H{\isadigit{2}}{\isacharcolon}{\kern0pt}\ {\isachardoublequoteopen}EQ{\isacharparenleft}{\kern0pt}p{\isacharcomma}{\kern0pt}\ x{\isacharcomma}{\kern0pt}\ y{\isacharparenright}{\kern0pt}{\isachardoublequoteclose}\ \isanewline
\ \ \ \ \ \ \isacommand{using}\isamarkupfalse%
\ H\ \isacommand{unfolding}\isamarkupfalse%
\ Q{\isacharunderscore}{\kern0pt}def\ \isacommand{using}\isamarkupfalse%
\ xpname\ ypname\ pinP\ rle\ \isacommand{apply}\isamarkupfalse%
\ auto\ \isacommand{done}\isamarkupfalse%
\ \isanewline
\ \ \ \ \isanewline
\ \ \ \ \isacommand{show}\isamarkupfalse%
\ {\isachardoublequoteopen}{\isacharparenleft}{\kern0pt}forces{\isacharunderscore}{\kern0pt}mem{\isacharparenleft}{\kern0pt}p{\isacharcomma}{\kern0pt}\ x{\isacharcomma}{\kern0pt}\ y{\isacharparenright}{\kern0pt}\ {\isasymlongleftrightarrow}\ forces{\isacharunderscore}{\kern0pt}mem{\isacharparenleft}{\kern0pt}{\isasympi}\ {\isacharbackquote}{\kern0pt}\ p{\isacharcomma}{\kern0pt}\ Pn{\isacharunderscore}{\kern0pt}auto{\isacharparenleft}{\kern0pt}{\isasympi}{\isacharparenright}{\kern0pt}\ {\isacharbackquote}{\kern0pt}\ x{\isacharcomma}{\kern0pt}\ Pn{\isacharunderscore}{\kern0pt}auto{\isacharparenleft}{\kern0pt}{\isasympi}{\isacharparenright}{\kern0pt}\ {\isacharbackquote}{\kern0pt}\ y{\isacharparenright}{\kern0pt}{\isacharparenright}{\kern0pt}\ {\isasymand}\isanewline
\ \ \ \ \ \ \ {\isacharparenleft}{\kern0pt}forces{\isacharunderscore}{\kern0pt}eq{\isacharparenleft}{\kern0pt}p{\isacharcomma}{\kern0pt}\ x{\isacharcomma}{\kern0pt}\ y{\isacharparenright}{\kern0pt}\ {\isasymlongleftrightarrow}\ forces{\isacharunderscore}{\kern0pt}eq{\isacharparenleft}{\kern0pt}{\isasympi}\ {\isacharbackquote}{\kern0pt}\ p{\isacharcomma}{\kern0pt}\ Pn{\isacharunderscore}{\kern0pt}auto{\isacharparenleft}{\kern0pt}{\isasympi}{\isacharparenright}{\kern0pt}\ {\isacharbackquote}{\kern0pt}\ x{\isacharcomma}{\kern0pt}\ Pn{\isacharunderscore}{\kern0pt}auto{\isacharparenleft}{\kern0pt}{\isasympi}{\isacharparenright}{\kern0pt}\ {\isacharbackquote}{\kern0pt}\ y{\isacharparenright}{\kern0pt}{\isacharparenright}{\kern0pt}{\isachardoublequoteclose}\ \isanewline
\ \ \ \ \ \ \isacommand{using}\isamarkupfalse%
\ H{\isadigit{1}}\ H{\isadigit{2}}\ \isacommand{unfolding}\isamarkupfalse%
\ MEM{\isacharunderscore}{\kern0pt}def\ EQ{\isacharunderscore}{\kern0pt}def\ \isacommand{by}\isamarkupfalse%
\ auto\ \isanewline
\ \ \isacommand{qed}\isamarkupfalse%
%
\endisatagproof
{\isafoldproof}%
%
\isadelimproof
\isanewline
%
\endisadelimproof
\isanewline
\isanewline
\isacommand{lemma}\isamarkupfalse%
\ symmetry{\isacharunderscore}{\kern0pt}lemma{\isacharunderscore}{\kern0pt}mem\ {\isacharcolon}{\kern0pt}\ \isanewline
\ \ \isakeyword{fixes}\ {\isasympi}\ p\ i\ j\ env\isanewline
\ \ \isakeyword{assumes}\ {\isachardoublequoteopen}is{\isacharunderscore}{\kern0pt}P{\isacharunderscore}{\kern0pt}auto{\isacharparenleft}{\kern0pt}{\isasympi}{\isacharparenright}{\kern0pt}{\isachardoublequoteclose}\ {\isachardoublequoteopen}p\ {\isasymin}\ P{\isachardoublequoteclose}\ {\isachardoublequoteopen}env\ {\isasymin}\ list{\isacharparenleft}{\kern0pt}HS{\isacharparenright}{\kern0pt}{\isachardoublequoteclose}\ {\isachardoublequoteopen}i\ {\isacharless}{\kern0pt}\ length{\isacharparenleft}{\kern0pt}env{\isacharparenright}{\kern0pt}{\isachardoublequoteclose}\ {\isachardoublequoteopen}j\ {\isacharless}{\kern0pt}\ length{\isacharparenleft}{\kern0pt}env{\isacharparenright}{\kern0pt}{\isachardoublequoteclose}\ \isanewline
\ \ \isakeyword{shows}\ {\isachardoublequoteopen}p\ {\isasymtturnstile}HS\ Member{\isacharparenleft}{\kern0pt}i{\isacharcomma}{\kern0pt}\ j{\isacharparenright}{\kern0pt}\ env\ {\isasymlongleftrightarrow}\ {\isasympi}{\isacharbackquote}{\kern0pt}p\ {\isasymtturnstile}HS\ Member{\isacharparenleft}{\kern0pt}i{\isacharcomma}{\kern0pt}\ j{\isacharparenright}{\kern0pt}\ map{\isacharparenleft}{\kern0pt}{\isasymlambda}x{\isachardot}{\kern0pt}\ Pn{\isacharunderscore}{\kern0pt}auto{\isacharparenleft}{\kern0pt}{\isasympi}{\isacharparenright}{\kern0pt}{\isacharbackquote}{\kern0pt}x{\isacharcomma}{\kern0pt}\ env{\isacharparenright}{\kern0pt}{\isachardoublequoteclose}\ \isanewline
%
\isadelimproof
%
\endisadelimproof
%
\isatagproof
\isacommand{proof}\isamarkupfalse%
\ {\isacharminus}{\kern0pt}\ \isanewline
\ \ \isacommand{have}\isamarkupfalse%
\ H{\isacharcolon}{\kern0pt}\ {\isachardoublequoteopen}{\isasymAnd}{\isasympi}\ p\ x\ y{\isachardot}{\kern0pt}\ is{\isacharunderscore}{\kern0pt}P{\isacharunderscore}{\kern0pt}auto{\isacharparenleft}{\kern0pt}{\isasympi}{\isacharparenright}{\kern0pt}\ {\isasymLongrightarrow}\ p\ {\isasymin}\ P\ {\isasymLongrightarrow}\ x\ {\isasymin}\ HS\ {\isasymLongrightarrow}\ y\ {\isasymin}\ HS\ \isanewline
\ \ \ \ \ \ \ \ \ \ \ \ {\isasymLongrightarrow}\ {\isacharparenleft}{\kern0pt}forces{\isacharunderscore}{\kern0pt}mem{\isacharparenleft}{\kern0pt}p{\isacharcomma}{\kern0pt}\ x{\isacharcomma}{\kern0pt}\ y{\isacharparenright}{\kern0pt}\ {\isasymlongleftrightarrow}\ forces{\isacharunderscore}{\kern0pt}mem{\isacharparenleft}{\kern0pt}{\isasympi}{\isacharbackquote}{\kern0pt}p{\isacharcomma}{\kern0pt}\ Pn{\isacharunderscore}{\kern0pt}auto{\isacharparenleft}{\kern0pt}{\isasympi}{\isacharparenright}{\kern0pt}{\isacharbackquote}{\kern0pt}x{\isacharcomma}{\kern0pt}\ Pn{\isacharunderscore}{\kern0pt}auto{\isacharparenleft}{\kern0pt}{\isasympi}{\isacharparenright}{\kern0pt}{\isacharbackquote}{\kern0pt}y{\isacharparenright}{\kern0pt}{\isacharparenright}{\kern0pt}{\isachardoublequoteclose}\ \isanewline
\ \ \ \ \isacommand{using}\isamarkupfalse%
\ symmetry{\isacharunderscore}{\kern0pt}lemma{\isacharunderscore}{\kern0pt}base\ HS{\isacharunderscore}{\kern0pt}iff\ \isacommand{by}\isamarkupfalse%
\ auto\ \isanewline
\isanewline
\ \ \isacommand{have}\isamarkupfalse%
\ envin\ {\isacharcolon}{\kern0pt}\ {\isachardoublequoteopen}env\ {\isasymin}\ list{\isacharparenleft}{\kern0pt}M{\isacharparenright}{\kern0pt}{\isachardoublequoteclose}\ \isanewline
\ \ \ \ \isacommand{apply}\isamarkupfalse%
{\isacharparenleft}{\kern0pt}rule{\isacharunderscore}{\kern0pt}tac\ A{\isacharequal}{\kern0pt}\ {\isachardoublequoteopen}list{\isacharparenleft}{\kern0pt}HS{\isacharparenright}{\kern0pt}{\isachardoublequoteclose}\ \isakeyword{in}\ subsetD{\isacharcomma}{\kern0pt}\ rule\ list{\isacharunderscore}{\kern0pt}mono{\isacharparenright}{\kern0pt}\isanewline
\ \ \ \ \isacommand{using}\isamarkupfalse%
\ HS{\isacharunderscore}{\kern0pt}iff\ P{\isacharunderscore}{\kern0pt}name{\isacharunderscore}{\kern0pt}in{\isacharunderscore}{\kern0pt}M\ assms\ \isanewline
\ \ \ \ \isacommand{by}\isamarkupfalse%
\ auto\isanewline
\isanewline
\ \ \isacommand{have}\isamarkupfalse%
\ mapin\ {\isacharcolon}{\kern0pt}\ {\isachardoublequoteopen}map{\isacharparenleft}{\kern0pt}{\isasymlambda}x{\isachardot}{\kern0pt}\ Pn{\isacharunderscore}{\kern0pt}auto{\isacharparenleft}{\kern0pt}{\isasympi}{\isacharparenright}{\kern0pt}\ {\isacharbackquote}{\kern0pt}\ x{\isacharcomma}{\kern0pt}\ env{\isacharparenright}{\kern0pt}\ {\isasymin}\ list{\isacharparenleft}{\kern0pt}M{\isacharparenright}{\kern0pt}{\isachardoublequoteclose}\ \ \isanewline
\ \ \ \ \isacommand{apply}\isamarkupfalse%
{\isacharparenleft}{\kern0pt}rule\ map{\isacharunderscore}{\kern0pt}type{\isacharparenright}{\kern0pt}\isanewline
\ \ \ \ \isacommand{using}\isamarkupfalse%
\ assms\ \isanewline
\ \ \ \ \ \isacommand{apply}\isamarkupfalse%
\ simp\isanewline
\ \ \ \ \isacommand{apply}\isamarkupfalse%
{\isacharparenleft}{\kern0pt}rule\ Pn{\isacharunderscore}{\kern0pt}auto{\isacharunderscore}{\kern0pt}value{\isacharunderscore}{\kern0pt}in{\isacharunderscore}{\kern0pt}M{\isacharparenright}{\kern0pt}\isanewline
\ \ \ \ \isacommand{using}\isamarkupfalse%
\ assms\ HS{\isacharunderscore}{\kern0pt}iff\isanewline
\ \ \ \ \isacommand{by}\isamarkupfalse%
\ auto\isanewline
\isanewline
\ \ \isacommand{have}\isamarkupfalse%
\ mapnthin\ {\isacharcolon}{\kern0pt}\ {\isachardoublequoteopen}{\isasymAnd}i{\isachardot}{\kern0pt}\ i\ {\isacharless}{\kern0pt}\ length{\isacharparenleft}{\kern0pt}env{\isacharparenright}{\kern0pt}\ {\isasymLongrightarrow}\ nth{\isacharparenleft}{\kern0pt}i{\isacharcomma}{\kern0pt}\ map{\isacharparenleft}{\kern0pt}{\isasymlambda}x{\isachardot}{\kern0pt}\ Pn{\isacharunderscore}{\kern0pt}auto{\isacharparenleft}{\kern0pt}{\isasympi}{\isacharparenright}{\kern0pt}\ {\isacharbackquote}{\kern0pt}\ x{\isacharcomma}{\kern0pt}\ env{\isacharparenright}{\kern0pt}{\isacharparenright}{\kern0pt}\ {\isasymin}\ M{\isachardoublequoteclose}\ \isanewline
\ \ \ \ \isacommand{apply}\isamarkupfalse%
{\isacharparenleft}{\kern0pt}rule\ nth{\isacharunderscore}{\kern0pt}type{\isacharparenright}{\kern0pt}\isanewline
\ \ \ \ \isacommand{using}\isamarkupfalse%
\ assms\ mapin\ length{\isacharunderscore}{\kern0pt}map\ \isanewline
\ \ \ \ \isacommand{by}\isamarkupfalse%
\ auto\isanewline
\isanewline
\ \ \isacommand{have}\isamarkupfalse%
\ {\isachardoublequoteopen}p\ {\isasymtturnstile}HS\ Member{\isacharparenleft}{\kern0pt}i{\isacharcomma}{\kern0pt}\ j{\isacharparenright}{\kern0pt}\ env\ {\isasymlongleftrightarrow}\ p\ {\isasymtturnstile}\ Member{\isacharparenleft}{\kern0pt}i{\isacharcomma}{\kern0pt}\ j{\isacharparenright}{\kern0pt}\ env{\isachardoublequoteclose}\ \isanewline
\ \ \ \ \isacommand{apply}\isamarkupfalse%
{\isacharparenleft}{\kern0pt}rule\ iff{\isacharunderscore}{\kern0pt}flip{\isacharcomma}{\kern0pt}\ rule\ ForcesHS{\isacharunderscore}{\kern0pt}Member{\isacharparenright}{\kern0pt}\isanewline
\ \ \ \ \isacommand{using}\isamarkupfalse%
\ assms\ envin\ P{\isacharunderscore}{\kern0pt}in{\isacharunderscore}{\kern0pt}M\ transM\ lt{\isacharunderscore}{\kern0pt}nat{\isacharunderscore}{\kern0pt}in{\isacharunderscore}{\kern0pt}nat\isanewline
\ \ \ \ \isacommand{by}\isamarkupfalse%
\ auto\isanewline
\ \ \isacommand{also}\isamarkupfalse%
\ \isacommand{have}\isamarkupfalse%
\ {\isachardoublequoteopen}{\isachardot}{\kern0pt}{\isachardot}{\kern0pt}{\isachardot}{\kern0pt}\ {\isasymlongleftrightarrow}\ forces{\isacharunderscore}{\kern0pt}mem{\isacharparenleft}{\kern0pt}p{\isacharcomma}{\kern0pt}\ nth{\isacharparenleft}{\kern0pt}i{\isacharcomma}{\kern0pt}\ env{\isacharparenright}{\kern0pt}{\isacharcomma}{\kern0pt}\ nth{\isacharparenleft}{\kern0pt}j{\isacharcomma}{\kern0pt}\ env{\isacharparenright}{\kern0pt}{\isacharparenright}{\kern0pt}{\isachardoublequoteclose}\ \isanewline
\ \ \ \ \isacommand{apply}\isamarkupfalse%
{\isacharparenleft}{\kern0pt}rule\ Forces{\isacharunderscore}{\kern0pt}Member{\isacharparenright}{\kern0pt}\isanewline
\ \ \ \ \isacommand{using}\isamarkupfalse%
\ assms\ envin\ Forces{\isacharunderscore}{\kern0pt}Member\ nth{\isacharunderscore}{\kern0pt}type\ HS{\isacharunderscore}{\kern0pt}iff\ lt{\isacharunderscore}{\kern0pt}nat{\isacharunderscore}{\kern0pt}in{\isacharunderscore}{\kern0pt}nat\isanewline
\ \ \ \ \isacommand{by}\isamarkupfalse%
\ auto\isanewline
\ \ \isacommand{also}\isamarkupfalse%
\ \isacommand{have}\isamarkupfalse%
\ {\isachardoublequoteopen}{\isachardot}{\kern0pt}{\isachardot}{\kern0pt}{\isachardot}{\kern0pt}\ {\isasymlongleftrightarrow}\ forces{\isacharunderscore}{\kern0pt}mem{\isacharparenleft}{\kern0pt}{\isasympi}{\isacharbackquote}{\kern0pt}p{\isacharcomma}{\kern0pt}\ Pn{\isacharunderscore}{\kern0pt}auto{\isacharparenleft}{\kern0pt}{\isasympi}{\isacharparenright}{\kern0pt}{\isacharbackquote}{\kern0pt}nth{\isacharparenleft}{\kern0pt}i{\isacharcomma}{\kern0pt}\ env{\isacharparenright}{\kern0pt}{\isacharcomma}{\kern0pt}\ Pn{\isacharunderscore}{\kern0pt}auto{\isacharparenleft}{\kern0pt}{\isasympi}{\isacharparenright}{\kern0pt}{\isacharbackquote}{\kern0pt}nth{\isacharparenleft}{\kern0pt}j{\isacharcomma}{\kern0pt}\ env{\isacharparenright}{\kern0pt}{\isacharparenright}{\kern0pt}{\isachardoublequoteclose}\ \isanewline
\ \ \ \ \isacommand{using}\isamarkupfalse%
\ assms\ nth{\isacharunderscore}{\kern0pt}type\ H\isanewline
\ \ \ \ \isacommand{by}\isamarkupfalse%
\ auto\isanewline
\ \ \isacommand{also}\isamarkupfalse%
\ \isacommand{have}\isamarkupfalse%
\ {\isachardoublequoteopen}{\isachardot}{\kern0pt}{\isachardot}{\kern0pt}{\isachardot}{\kern0pt}\ {\isasymlongleftrightarrow}\ forces{\isacharunderscore}{\kern0pt}mem{\isacharparenleft}{\kern0pt}{\isasympi}{\isacharbackquote}{\kern0pt}p{\isacharcomma}{\kern0pt}\ nth{\isacharparenleft}{\kern0pt}i{\isacharcomma}{\kern0pt}\ map{\isacharparenleft}{\kern0pt}{\isasymlambda}x{\isachardot}{\kern0pt}\ Pn{\isacharunderscore}{\kern0pt}auto{\isacharparenleft}{\kern0pt}{\isasympi}{\isacharparenright}{\kern0pt}{\isacharbackquote}{\kern0pt}x{\isacharcomma}{\kern0pt}\ env{\isacharparenright}{\kern0pt}{\isacharparenright}{\kern0pt}{\isacharcomma}{\kern0pt}\ nth{\isacharparenleft}{\kern0pt}j{\isacharcomma}{\kern0pt}\ map{\isacharparenleft}{\kern0pt}{\isasymlambda}x{\isachardot}{\kern0pt}\ Pn{\isacharunderscore}{\kern0pt}auto{\isacharparenleft}{\kern0pt}{\isasympi}{\isacharparenright}{\kern0pt}{\isacharbackquote}{\kern0pt}x{\isacharcomma}{\kern0pt}\ env{\isacharparenright}{\kern0pt}{\isacharparenright}{\kern0pt}{\isacharparenright}{\kern0pt}{\isachardoublequoteclose}\ \isanewline
\ \ \ \ \isacommand{using}\isamarkupfalse%
\ assms\ lt{\isacharunderscore}{\kern0pt}nat{\isacharunderscore}{\kern0pt}in{\isacharunderscore}{\kern0pt}nat\ nth{\isacharunderscore}{\kern0pt}map\isanewline
\ \ \ \ \isacommand{by}\isamarkupfalse%
\ auto\isanewline
\ \ \isacommand{also}\isamarkupfalse%
\ \isacommand{have}\isamarkupfalse%
\ {\isachardoublequoteopen}{\isachardot}{\kern0pt}{\isachardot}{\kern0pt}{\isachardot}{\kern0pt}\ {\isasymlongleftrightarrow}\ {\isasympi}{\isacharbackquote}{\kern0pt}p\ {\isasymtturnstile}\ Member{\isacharparenleft}{\kern0pt}i{\isacharcomma}{\kern0pt}\ j{\isacharparenright}{\kern0pt}\ map{\isacharparenleft}{\kern0pt}{\isasymlambda}x{\isachardot}{\kern0pt}\ Pn{\isacharunderscore}{\kern0pt}auto{\isacharparenleft}{\kern0pt}{\isasympi}{\isacharparenright}{\kern0pt}{\isacharbackquote}{\kern0pt}x{\isacharcomma}{\kern0pt}\ env{\isacharparenright}{\kern0pt}{\isachardoublequoteclose}\isanewline
\ \ \ \ \isacommand{apply}\isamarkupfalse%
{\isacharparenleft}{\kern0pt}rule\ iff{\isacharunderscore}{\kern0pt}flip{\isacharcomma}{\kern0pt}\ rule\ Forces{\isacharunderscore}{\kern0pt}Member{\isacharparenright}{\kern0pt}\isanewline
\ \ \ \ \isacommand{using}\isamarkupfalse%
\ assms\ mapin\ lt{\isacharunderscore}{\kern0pt}nat{\isacharunderscore}{\kern0pt}in{\isacharunderscore}{\kern0pt}nat\ nth{\isacharunderscore}{\kern0pt}type\ mapnthin\ P{\isacharunderscore}{\kern0pt}auto{\isacharunderscore}{\kern0pt}value\isanewline
\ \ \ \ \isacommand{by}\isamarkupfalse%
\ auto\isanewline
\ \ \isacommand{also}\isamarkupfalse%
\ \isacommand{have}\isamarkupfalse%
\ {\isachardoublequoteopen}{\isachardot}{\kern0pt}{\isachardot}{\kern0pt}{\isachardot}{\kern0pt}\ {\isasymlongleftrightarrow}\ {\isasympi}{\isacharbackquote}{\kern0pt}p\ {\isasymtturnstile}HS\ Member{\isacharparenleft}{\kern0pt}i{\isacharcomma}{\kern0pt}\ j{\isacharparenright}{\kern0pt}\ map{\isacharparenleft}{\kern0pt}{\isasymlambda}x{\isachardot}{\kern0pt}\ Pn{\isacharunderscore}{\kern0pt}auto{\isacharparenleft}{\kern0pt}{\isasympi}{\isacharparenright}{\kern0pt}{\isacharbackquote}{\kern0pt}x{\isacharcomma}{\kern0pt}\ env{\isacharparenright}{\kern0pt}{\isachardoublequoteclose}\isanewline
\ \ \ \ \isacommand{apply}\isamarkupfalse%
{\isacharparenleft}{\kern0pt}rule\ ForcesHS{\isacharunderscore}{\kern0pt}Member{\isacharparenright}{\kern0pt}\isanewline
\ \ \ \ \isacommand{using}\isamarkupfalse%
\ assms\ lt{\isacharunderscore}{\kern0pt}nat{\isacharunderscore}{\kern0pt}in{\isacharunderscore}{\kern0pt}nat\ mapin\ P{\isacharunderscore}{\kern0pt}auto{\isacharunderscore}{\kern0pt}value\isanewline
\ \ \ \ \ \ \ \isacommand{apply}\isamarkupfalse%
\ auto{\isacharbrackleft}{\kern0pt}{\isadigit{3}}{\isacharbrackright}{\kern0pt}\isanewline
\ \ \ \ \isacommand{apply}\isamarkupfalse%
{\isacharparenleft}{\kern0pt}rule{\isacharunderscore}{\kern0pt}tac\ A{\isacharequal}{\kern0pt}P\ \isakeyword{in}\ subsetD{\isacharparenright}{\kern0pt}\isanewline
\ \ \ \ \isacommand{using}\isamarkupfalse%
\ P{\isacharunderscore}{\kern0pt}in{\isacharunderscore}{\kern0pt}M\ transM\ \isanewline
\ \ \ \ \ \isacommand{apply}\isamarkupfalse%
\ force\ \isanewline
\ \ \ \ \isacommand{apply}\isamarkupfalse%
{\isacharparenleft}{\kern0pt}rule\ P{\isacharunderscore}{\kern0pt}auto{\isacharunderscore}{\kern0pt}value{\isacharparenright}{\kern0pt}\isanewline
\ \ \ \ \isacommand{using}\isamarkupfalse%
\ assms\ \isanewline
\ \ \ \ \isacommand{by}\isamarkupfalse%
\ auto\isanewline
\ \ \isacommand{finally}\isamarkupfalse%
\ \isacommand{show}\isamarkupfalse%
\ {\isacharquery}{\kern0pt}thesis\ \isanewline
\ \ \ \ \isacommand{by}\isamarkupfalse%
\ auto\isanewline
\isacommand{qed}\isamarkupfalse%
%
\endisatagproof
{\isafoldproof}%
%
\isadelimproof
\isanewline
%
\endisadelimproof
\isanewline
\isacommand{lemma}\isamarkupfalse%
\ symmetry{\isacharunderscore}{\kern0pt}lemma{\isacharunderscore}{\kern0pt}eq\ {\isacharcolon}{\kern0pt}\ \isanewline
\ \ \isakeyword{fixes}\ {\isasympi}\ p\ i\ j\ env\isanewline
\ \ \isakeyword{assumes}\ {\isachardoublequoteopen}is{\isacharunderscore}{\kern0pt}P{\isacharunderscore}{\kern0pt}auto{\isacharparenleft}{\kern0pt}{\isasympi}{\isacharparenright}{\kern0pt}{\isachardoublequoteclose}\ {\isachardoublequoteopen}p\ {\isasymin}\ P{\isachardoublequoteclose}\ {\isachardoublequoteopen}env\ {\isasymin}\ list{\isacharparenleft}{\kern0pt}HS{\isacharparenright}{\kern0pt}{\isachardoublequoteclose}\ {\isachardoublequoteopen}i\ {\isacharless}{\kern0pt}\ length{\isacharparenleft}{\kern0pt}env{\isacharparenright}{\kern0pt}{\isachardoublequoteclose}\ {\isachardoublequoteopen}j\ {\isacharless}{\kern0pt}\ length{\isacharparenleft}{\kern0pt}env{\isacharparenright}{\kern0pt}{\isachardoublequoteclose}\ \isanewline
\ \ \isakeyword{shows}\ {\isachardoublequoteopen}p\ {\isasymtturnstile}HS\ Equal{\isacharparenleft}{\kern0pt}i{\isacharcomma}{\kern0pt}\ j{\isacharparenright}{\kern0pt}\ env\ {\isasymlongleftrightarrow}\ {\isasympi}{\isacharbackquote}{\kern0pt}p\ {\isasymtturnstile}HS\ Equal{\isacharparenleft}{\kern0pt}i{\isacharcomma}{\kern0pt}\ j{\isacharparenright}{\kern0pt}\ map{\isacharparenleft}{\kern0pt}{\isasymlambda}x{\isachardot}{\kern0pt}\ Pn{\isacharunderscore}{\kern0pt}auto{\isacharparenleft}{\kern0pt}{\isasympi}{\isacharparenright}{\kern0pt}{\isacharbackquote}{\kern0pt}x{\isacharcomma}{\kern0pt}\ env{\isacharparenright}{\kern0pt}{\isachardoublequoteclose}\ \isanewline
%
\isadelimproof
%
\endisadelimproof
%
\isatagproof
\isacommand{proof}\isamarkupfalse%
\ {\isacharminus}{\kern0pt}\ \isanewline
\ \ \isacommand{have}\isamarkupfalse%
\ H{\isacharcolon}{\kern0pt}\ {\isachardoublequoteopen}{\isasymAnd}{\isasympi}\ p\ x\ y{\isachardot}{\kern0pt}\ is{\isacharunderscore}{\kern0pt}P{\isacharunderscore}{\kern0pt}auto{\isacharparenleft}{\kern0pt}{\isasympi}{\isacharparenright}{\kern0pt}\ {\isasymLongrightarrow}\ p\ {\isasymin}\ P\ {\isasymLongrightarrow}\ x\ {\isasymin}\ HS\ {\isasymLongrightarrow}\ y\ {\isasymin}\ HS\ \isanewline
\ \ \ \ \ \ \ \ \ \ \ \ {\isasymLongrightarrow}\ {\isacharparenleft}{\kern0pt}forces{\isacharunderscore}{\kern0pt}eq{\isacharparenleft}{\kern0pt}p{\isacharcomma}{\kern0pt}\ x{\isacharcomma}{\kern0pt}\ y{\isacharparenright}{\kern0pt}\ {\isasymlongleftrightarrow}\ forces{\isacharunderscore}{\kern0pt}eq{\isacharparenleft}{\kern0pt}{\isasympi}{\isacharbackquote}{\kern0pt}p{\isacharcomma}{\kern0pt}\ Pn{\isacharunderscore}{\kern0pt}auto{\isacharparenleft}{\kern0pt}{\isasympi}{\isacharparenright}{\kern0pt}{\isacharbackquote}{\kern0pt}x{\isacharcomma}{\kern0pt}\ Pn{\isacharunderscore}{\kern0pt}auto{\isacharparenleft}{\kern0pt}{\isasympi}{\isacharparenright}{\kern0pt}{\isacharbackquote}{\kern0pt}y{\isacharparenright}{\kern0pt}{\isacharparenright}{\kern0pt}{\isachardoublequoteclose}\ \isanewline
\ \ \ \ \isacommand{using}\isamarkupfalse%
\ symmetry{\isacharunderscore}{\kern0pt}lemma{\isacharunderscore}{\kern0pt}base\ HS{\isacharunderscore}{\kern0pt}iff\ \isacommand{by}\isamarkupfalse%
\ auto\ \isanewline
\isanewline
\ \ \isacommand{have}\isamarkupfalse%
\ envin\ {\isacharcolon}{\kern0pt}\ {\isachardoublequoteopen}env\ {\isasymin}\ list{\isacharparenleft}{\kern0pt}M{\isacharparenright}{\kern0pt}{\isachardoublequoteclose}\ \isanewline
\ \ \ \ \isacommand{apply}\isamarkupfalse%
{\isacharparenleft}{\kern0pt}rule{\isacharunderscore}{\kern0pt}tac\ A{\isacharequal}{\kern0pt}\ {\isachardoublequoteopen}list{\isacharparenleft}{\kern0pt}HS{\isacharparenright}{\kern0pt}{\isachardoublequoteclose}\ \isakeyword{in}\ subsetD{\isacharcomma}{\kern0pt}\ rule\ list{\isacharunderscore}{\kern0pt}mono{\isacharparenright}{\kern0pt}\isanewline
\ \ \ \ \isacommand{using}\isamarkupfalse%
\ HS{\isacharunderscore}{\kern0pt}iff\ P{\isacharunderscore}{\kern0pt}name{\isacharunderscore}{\kern0pt}in{\isacharunderscore}{\kern0pt}M\ assms\ \isanewline
\ \ \ \ \isacommand{by}\isamarkupfalse%
\ auto\isanewline
\isanewline
\ \ \isacommand{have}\isamarkupfalse%
\ mapin\ {\isacharcolon}{\kern0pt}\ {\isachardoublequoteopen}map{\isacharparenleft}{\kern0pt}{\isasymlambda}x{\isachardot}{\kern0pt}\ Pn{\isacharunderscore}{\kern0pt}auto{\isacharparenleft}{\kern0pt}{\isasympi}{\isacharparenright}{\kern0pt}\ {\isacharbackquote}{\kern0pt}\ x{\isacharcomma}{\kern0pt}\ env{\isacharparenright}{\kern0pt}\ {\isasymin}\ list{\isacharparenleft}{\kern0pt}M{\isacharparenright}{\kern0pt}{\isachardoublequoteclose}\ \ \isanewline
\ \ \ \ \isacommand{apply}\isamarkupfalse%
{\isacharparenleft}{\kern0pt}rule\ map{\isacharunderscore}{\kern0pt}type{\isacharparenright}{\kern0pt}\isanewline
\ \ \ \ \isacommand{using}\isamarkupfalse%
\ assms\ \isanewline
\ \ \ \ \ \isacommand{apply}\isamarkupfalse%
\ simp\isanewline
\ \ \ \ \isacommand{apply}\isamarkupfalse%
{\isacharparenleft}{\kern0pt}rule\ Pn{\isacharunderscore}{\kern0pt}auto{\isacharunderscore}{\kern0pt}value{\isacharunderscore}{\kern0pt}in{\isacharunderscore}{\kern0pt}M{\isacharparenright}{\kern0pt}\isanewline
\ \ \ \ \isacommand{using}\isamarkupfalse%
\ assms\ HS{\isacharunderscore}{\kern0pt}iff\isanewline
\ \ \ \ \isacommand{by}\isamarkupfalse%
\ auto\isanewline
\isanewline
\ \ \isacommand{have}\isamarkupfalse%
\ mapnthin\ {\isacharcolon}{\kern0pt}\ {\isachardoublequoteopen}{\isasymAnd}i{\isachardot}{\kern0pt}\ i\ {\isacharless}{\kern0pt}\ length{\isacharparenleft}{\kern0pt}env{\isacharparenright}{\kern0pt}\ {\isasymLongrightarrow}\ nth{\isacharparenleft}{\kern0pt}i{\isacharcomma}{\kern0pt}\ map{\isacharparenleft}{\kern0pt}{\isasymlambda}x{\isachardot}{\kern0pt}\ Pn{\isacharunderscore}{\kern0pt}auto{\isacharparenleft}{\kern0pt}{\isasympi}{\isacharparenright}{\kern0pt}\ {\isacharbackquote}{\kern0pt}\ x{\isacharcomma}{\kern0pt}\ env{\isacharparenright}{\kern0pt}{\isacharparenright}{\kern0pt}\ {\isasymin}\ M{\isachardoublequoteclose}\ \isanewline
\ \ \ \ \isacommand{apply}\isamarkupfalse%
{\isacharparenleft}{\kern0pt}rule\ nth{\isacharunderscore}{\kern0pt}type{\isacharparenright}{\kern0pt}\isanewline
\ \ \ \ \isacommand{using}\isamarkupfalse%
\ assms\ mapin\ length{\isacharunderscore}{\kern0pt}map\ \isanewline
\ \ \ \ \isacommand{by}\isamarkupfalse%
\ auto\isanewline
\isanewline
\ \ \isacommand{have}\isamarkupfalse%
\ {\isachardoublequoteopen}p\ {\isasymtturnstile}HS\ Equal{\isacharparenleft}{\kern0pt}i{\isacharcomma}{\kern0pt}\ j{\isacharparenright}{\kern0pt}\ env\ {\isasymlongleftrightarrow}\ p\ {\isasymtturnstile}\ Equal{\isacharparenleft}{\kern0pt}i{\isacharcomma}{\kern0pt}\ j{\isacharparenright}{\kern0pt}\ env{\isachardoublequoteclose}\ \isanewline
\ \ \ \ \isacommand{apply}\isamarkupfalse%
{\isacharparenleft}{\kern0pt}rule\ iff{\isacharunderscore}{\kern0pt}flip{\isacharcomma}{\kern0pt}\ rule\ ForcesHS{\isacharunderscore}{\kern0pt}Equal{\isacharparenright}{\kern0pt}\isanewline
\ \ \ \ \isacommand{using}\isamarkupfalse%
\ assms\ envin\ P{\isacharunderscore}{\kern0pt}in{\isacharunderscore}{\kern0pt}M\ transM\ lt{\isacharunderscore}{\kern0pt}nat{\isacharunderscore}{\kern0pt}in{\isacharunderscore}{\kern0pt}nat\isanewline
\ \ \ \ \isacommand{by}\isamarkupfalse%
\ auto\isanewline
\ \ \isacommand{also}\isamarkupfalse%
\ \isacommand{have}\isamarkupfalse%
\ {\isachardoublequoteopen}{\isachardot}{\kern0pt}{\isachardot}{\kern0pt}{\isachardot}{\kern0pt}\ {\isasymlongleftrightarrow}\ forces{\isacharunderscore}{\kern0pt}eq{\isacharparenleft}{\kern0pt}p{\isacharcomma}{\kern0pt}\ nth{\isacharparenleft}{\kern0pt}i{\isacharcomma}{\kern0pt}\ env{\isacharparenright}{\kern0pt}{\isacharcomma}{\kern0pt}\ nth{\isacharparenleft}{\kern0pt}j{\isacharcomma}{\kern0pt}\ env{\isacharparenright}{\kern0pt}{\isacharparenright}{\kern0pt}{\isachardoublequoteclose}\ \isanewline
\ \ \ \ \isacommand{apply}\isamarkupfalse%
{\isacharparenleft}{\kern0pt}rule\ Forces{\isacharunderscore}{\kern0pt}Equal{\isacharparenright}{\kern0pt}\isanewline
\ \ \ \ \isacommand{using}\isamarkupfalse%
\ assms\ envin\ Forces{\isacharunderscore}{\kern0pt}Member\ nth{\isacharunderscore}{\kern0pt}type\ HS{\isacharunderscore}{\kern0pt}iff\ lt{\isacharunderscore}{\kern0pt}nat{\isacharunderscore}{\kern0pt}in{\isacharunderscore}{\kern0pt}nat\isanewline
\ \ \ \ \isacommand{by}\isamarkupfalse%
\ auto\isanewline
\ \ \isacommand{also}\isamarkupfalse%
\ \isacommand{have}\isamarkupfalse%
\ {\isachardoublequoteopen}{\isachardot}{\kern0pt}{\isachardot}{\kern0pt}{\isachardot}{\kern0pt}\ {\isasymlongleftrightarrow}\ forces{\isacharunderscore}{\kern0pt}eq{\isacharparenleft}{\kern0pt}{\isasympi}{\isacharbackquote}{\kern0pt}p{\isacharcomma}{\kern0pt}\ Pn{\isacharunderscore}{\kern0pt}auto{\isacharparenleft}{\kern0pt}{\isasympi}{\isacharparenright}{\kern0pt}{\isacharbackquote}{\kern0pt}nth{\isacharparenleft}{\kern0pt}i{\isacharcomma}{\kern0pt}\ env{\isacharparenright}{\kern0pt}{\isacharcomma}{\kern0pt}\ Pn{\isacharunderscore}{\kern0pt}auto{\isacharparenleft}{\kern0pt}{\isasympi}{\isacharparenright}{\kern0pt}{\isacharbackquote}{\kern0pt}nth{\isacharparenleft}{\kern0pt}j{\isacharcomma}{\kern0pt}\ env{\isacharparenright}{\kern0pt}{\isacharparenright}{\kern0pt}{\isachardoublequoteclose}\ \isanewline
\ \ \ \ \isacommand{using}\isamarkupfalse%
\ assms\ nth{\isacharunderscore}{\kern0pt}type\ H\isanewline
\ \ \ \ \isacommand{by}\isamarkupfalse%
\ auto\isanewline
\ \ \isacommand{also}\isamarkupfalse%
\ \isacommand{have}\isamarkupfalse%
\ {\isachardoublequoteopen}{\isachardot}{\kern0pt}{\isachardot}{\kern0pt}{\isachardot}{\kern0pt}\ {\isasymlongleftrightarrow}\ forces{\isacharunderscore}{\kern0pt}eq{\isacharparenleft}{\kern0pt}{\isasympi}{\isacharbackquote}{\kern0pt}p{\isacharcomma}{\kern0pt}\ nth{\isacharparenleft}{\kern0pt}i{\isacharcomma}{\kern0pt}\ map{\isacharparenleft}{\kern0pt}{\isasymlambda}x{\isachardot}{\kern0pt}\ Pn{\isacharunderscore}{\kern0pt}auto{\isacharparenleft}{\kern0pt}{\isasympi}{\isacharparenright}{\kern0pt}{\isacharbackquote}{\kern0pt}x{\isacharcomma}{\kern0pt}\ env{\isacharparenright}{\kern0pt}{\isacharparenright}{\kern0pt}{\isacharcomma}{\kern0pt}\ nth{\isacharparenleft}{\kern0pt}j{\isacharcomma}{\kern0pt}\ map{\isacharparenleft}{\kern0pt}{\isasymlambda}x{\isachardot}{\kern0pt}\ Pn{\isacharunderscore}{\kern0pt}auto{\isacharparenleft}{\kern0pt}{\isasympi}{\isacharparenright}{\kern0pt}{\isacharbackquote}{\kern0pt}x{\isacharcomma}{\kern0pt}\ env{\isacharparenright}{\kern0pt}{\isacharparenright}{\kern0pt}{\isacharparenright}{\kern0pt}{\isachardoublequoteclose}\ \isanewline
\ \ \ \ \isacommand{using}\isamarkupfalse%
\ assms\ lt{\isacharunderscore}{\kern0pt}nat{\isacharunderscore}{\kern0pt}in{\isacharunderscore}{\kern0pt}nat\ nth{\isacharunderscore}{\kern0pt}map\isanewline
\ \ \ \ \isacommand{by}\isamarkupfalse%
\ auto\isanewline
\ \ \isacommand{also}\isamarkupfalse%
\ \isacommand{have}\isamarkupfalse%
\ {\isachardoublequoteopen}{\isachardot}{\kern0pt}{\isachardot}{\kern0pt}{\isachardot}{\kern0pt}\ {\isasymlongleftrightarrow}\ {\isasympi}{\isacharbackquote}{\kern0pt}p\ {\isasymtturnstile}\ Equal{\isacharparenleft}{\kern0pt}i{\isacharcomma}{\kern0pt}\ j{\isacharparenright}{\kern0pt}\ map{\isacharparenleft}{\kern0pt}{\isasymlambda}x{\isachardot}{\kern0pt}\ Pn{\isacharunderscore}{\kern0pt}auto{\isacharparenleft}{\kern0pt}{\isasympi}{\isacharparenright}{\kern0pt}{\isacharbackquote}{\kern0pt}x{\isacharcomma}{\kern0pt}\ env{\isacharparenright}{\kern0pt}{\isachardoublequoteclose}\isanewline
\ \ \ \ \isacommand{apply}\isamarkupfalse%
{\isacharparenleft}{\kern0pt}rule\ iff{\isacharunderscore}{\kern0pt}flip{\isacharcomma}{\kern0pt}\ rule\ Forces{\isacharunderscore}{\kern0pt}Equal{\isacharparenright}{\kern0pt}\isanewline
\ \ \ \ \isacommand{using}\isamarkupfalse%
\ assms\ mapin\ lt{\isacharunderscore}{\kern0pt}nat{\isacharunderscore}{\kern0pt}in{\isacharunderscore}{\kern0pt}nat\ nth{\isacharunderscore}{\kern0pt}type\ mapnthin\ P{\isacharunderscore}{\kern0pt}auto{\isacharunderscore}{\kern0pt}value\isanewline
\ \ \ \ \isacommand{by}\isamarkupfalse%
\ auto\isanewline
\ \ \isacommand{also}\isamarkupfalse%
\ \isacommand{have}\isamarkupfalse%
\ {\isachardoublequoteopen}{\isachardot}{\kern0pt}{\isachardot}{\kern0pt}{\isachardot}{\kern0pt}\ {\isasymlongleftrightarrow}\ {\isasympi}{\isacharbackquote}{\kern0pt}p\ {\isasymtturnstile}HS\ Equal{\isacharparenleft}{\kern0pt}i{\isacharcomma}{\kern0pt}\ j{\isacharparenright}{\kern0pt}\ map{\isacharparenleft}{\kern0pt}{\isasymlambda}x{\isachardot}{\kern0pt}\ Pn{\isacharunderscore}{\kern0pt}auto{\isacharparenleft}{\kern0pt}{\isasympi}{\isacharparenright}{\kern0pt}{\isacharbackquote}{\kern0pt}x{\isacharcomma}{\kern0pt}\ env{\isacharparenright}{\kern0pt}{\isachardoublequoteclose}\isanewline
\ \ \ \ \isacommand{apply}\isamarkupfalse%
{\isacharparenleft}{\kern0pt}rule\ ForcesHS{\isacharunderscore}{\kern0pt}Equal{\isacharparenright}{\kern0pt}\isanewline
\ \ \ \ \isacommand{using}\isamarkupfalse%
\ assms\ lt{\isacharunderscore}{\kern0pt}nat{\isacharunderscore}{\kern0pt}in{\isacharunderscore}{\kern0pt}nat\ mapin\ P{\isacharunderscore}{\kern0pt}auto{\isacharunderscore}{\kern0pt}value\isanewline
\ \ \ \ \ \ \ \isacommand{apply}\isamarkupfalse%
\ auto{\isacharbrackleft}{\kern0pt}{\isadigit{3}}{\isacharbrackright}{\kern0pt}\isanewline
\ \ \ \ \isacommand{apply}\isamarkupfalse%
{\isacharparenleft}{\kern0pt}rule{\isacharunderscore}{\kern0pt}tac\ A{\isacharequal}{\kern0pt}P\ \isakeyword{in}\ subsetD{\isacharparenright}{\kern0pt}\isanewline
\ \ \ \ \isacommand{using}\isamarkupfalse%
\ P{\isacharunderscore}{\kern0pt}in{\isacharunderscore}{\kern0pt}M\ transM\ \isanewline
\ \ \ \ \ \isacommand{apply}\isamarkupfalse%
\ force\ \isanewline
\ \ \ \ \isacommand{apply}\isamarkupfalse%
{\isacharparenleft}{\kern0pt}rule\ P{\isacharunderscore}{\kern0pt}auto{\isacharunderscore}{\kern0pt}value{\isacharparenright}{\kern0pt}\isanewline
\ \ \ \ \isacommand{using}\isamarkupfalse%
\ assms\ \isanewline
\ \ \ \ \isacommand{by}\isamarkupfalse%
\ auto\isanewline
\ \ \isacommand{finally}\isamarkupfalse%
\ \isacommand{show}\isamarkupfalse%
\ {\isacharquery}{\kern0pt}thesis\ \isanewline
\ \ \ \ \isacommand{by}\isamarkupfalse%
\ auto\isanewline
\isacommand{qed}\isamarkupfalse%
%
\endisatagproof
{\isafoldproof}%
%
\isadelimproof
\isanewline
%
\endisadelimproof
\isanewline
\isacommand{lemma}\isamarkupfalse%
\ symmetry{\isacharunderscore}{\kern0pt}lemma{\isacharcolon}{\kern0pt}\isanewline
\ \ \isakeyword{fixes}\ {\isasymphi}\ {\isasympi}\ \ \isanewline
\ \ \isakeyword{assumes}\ {\isachardoublequoteopen}{\isasymphi}\ {\isasymin}\ formula{\isachardoublequoteclose}\ {\isachardoublequoteopen}is{\isacharunderscore}{\kern0pt}P{\isacharunderscore}{\kern0pt}auto{\isacharparenleft}{\kern0pt}{\isasympi}{\isacharparenright}{\kern0pt}{\isachardoublequoteclose}\ {\isachardoublequoteopen}{\isasympi}\ {\isasymin}\ {\isasymG}{\isachardoublequoteclose}\ \isanewline
\ \ \isakeyword{shows}\ {\isachardoublequoteopen}{\isasymAnd}env\ p{\isachardot}{\kern0pt}\ env\ {\isasymin}\ list{\isacharparenleft}{\kern0pt}HS{\isacharparenright}{\kern0pt}\ {\isasymLongrightarrow}\ arity{\isacharparenleft}{\kern0pt}{\isasymphi}{\isacharparenright}{\kern0pt}\ {\isasymle}\ length{\isacharparenleft}{\kern0pt}env{\isacharparenright}{\kern0pt}\ {\isasymLongrightarrow}\ p\ {\isasymin}\ P\ {\isasymLongrightarrow}\ p\ {\isasymtturnstile}HS\ {\isasymphi}\ env\ {\isasymlongleftrightarrow}\ {\isasympi}{\isacharbackquote}{\kern0pt}p\ {\isasymtturnstile}HS\ {\isasymphi}\ map{\isacharparenleft}{\kern0pt}{\isasymlambda}x{\isachardot}{\kern0pt}\ Pn{\isacharunderscore}{\kern0pt}auto{\isacharparenleft}{\kern0pt}{\isasympi}{\isacharparenright}{\kern0pt}{\isacharbackquote}{\kern0pt}x{\isacharcomma}{\kern0pt}\ env{\isacharparenright}{\kern0pt}{\isachardoublequoteclose}\ \isanewline
%
\isadelimproof
\ \ %
\endisadelimproof
%
\isatagproof
\isacommand{using}\isamarkupfalse%
\ {\isacartoucheopen}{\isasymphi}\ {\isasymin}\ formula{\isacartoucheclose}\isanewline
\isacommand{proof}\isamarkupfalse%
{\isacharparenleft}{\kern0pt}induct{\isacharparenright}{\kern0pt}\isanewline
\ \ \isacommand{case}\isamarkupfalse%
\ {\isacharparenleft}{\kern0pt}Member\ x\ y{\isacharparenright}{\kern0pt}\isanewline
\ \ \isacommand{then}\isamarkupfalse%
\ \isacommand{show}\isamarkupfalse%
\ {\isacharquery}{\kern0pt}case\ \isanewline
\ \ \ \ \isacommand{apply}\isamarkupfalse%
{\isacharparenleft}{\kern0pt}rule{\isacharunderscore}{\kern0pt}tac\ symmetry{\isacharunderscore}{\kern0pt}lemma{\isacharunderscore}{\kern0pt}mem{\isacharparenright}{\kern0pt}\isanewline
\ \ \ \ \isacommand{using}\isamarkupfalse%
\ assms\isanewline
\ \ \ \ \ \ \ \ \isacommand{apply}\isamarkupfalse%
\ auto{\isacharbrackleft}{\kern0pt}{\isadigit{3}}{\isacharbrackright}{\kern0pt}\isanewline
\ \ \ \ \ \isacommand{apply}\isamarkupfalse%
\ simp\isanewline
\ \ \ \ \ \isacommand{apply}\isamarkupfalse%
{\isacharparenleft}{\kern0pt}rule{\isacharunderscore}{\kern0pt}tac\ b{\isacharequal}{\kern0pt}{\isachardoublequoteopen}succ{\isacharparenleft}{\kern0pt}x{\isacharparenright}{\kern0pt}\ {\isasymunion}\ succ{\isacharparenleft}{\kern0pt}y{\isacharparenright}{\kern0pt}{\isachardoublequoteclose}\ \isakeyword{in}\ lt{\isacharunderscore}{\kern0pt}le{\isacharunderscore}{\kern0pt}lt{\isacharcomma}{\kern0pt}\ rule\ ltI{\isacharcomma}{\kern0pt}\ simp{\isacharcomma}{\kern0pt}\ simp{\isacharcomma}{\kern0pt}\ simp{\isacharparenright}{\kern0pt}{\isacharplus}{\kern0pt}\isanewline
\ \ \ \ \isacommand{done}\isamarkupfalse%
\isanewline
\isacommand{next}\isamarkupfalse%
\isanewline
\ \ \isacommand{case}\isamarkupfalse%
\ {\isacharparenleft}{\kern0pt}Equal\ x\ y{\isacharparenright}{\kern0pt}\isanewline
\ \ \isacommand{then}\isamarkupfalse%
\ \isacommand{show}\isamarkupfalse%
\ {\isacharquery}{\kern0pt}case\isanewline
\ \ \ \ \isacommand{apply}\isamarkupfalse%
{\isacharparenleft}{\kern0pt}rule{\isacharunderscore}{\kern0pt}tac\ symmetry{\isacharunderscore}{\kern0pt}lemma{\isacharunderscore}{\kern0pt}eq{\isacharparenright}{\kern0pt}\isanewline
\ \ \ \ \isacommand{using}\isamarkupfalse%
\ assms\isanewline
\ \ \ \ \ \ \ \ \isacommand{apply}\isamarkupfalse%
\ auto{\isacharbrackleft}{\kern0pt}{\isadigit{3}}{\isacharbrackright}{\kern0pt}\isanewline
\ \ \ \ \ \isacommand{apply}\isamarkupfalse%
\ simp\isanewline
\ \ \ \ \ \isacommand{apply}\isamarkupfalse%
{\isacharparenleft}{\kern0pt}rule{\isacharunderscore}{\kern0pt}tac\ b{\isacharequal}{\kern0pt}{\isachardoublequoteopen}succ{\isacharparenleft}{\kern0pt}x{\isacharparenright}{\kern0pt}\ {\isasymunion}\ succ{\isacharparenleft}{\kern0pt}y{\isacharparenright}{\kern0pt}{\isachardoublequoteclose}\ \isakeyword{in}\ lt{\isacharunderscore}{\kern0pt}le{\isacharunderscore}{\kern0pt}lt{\isacharcomma}{\kern0pt}\ rule\ ltI{\isacharcomma}{\kern0pt}\ simp{\isacharcomma}{\kern0pt}\ simp{\isacharcomma}{\kern0pt}\ simp{\isacharparenright}{\kern0pt}{\isacharplus}{\kern0pt}\isanewline
\ \ \ \ \isacommand{done}\isamarkupfalse%
\isanewline
\isacommand{next}\isamarkupfalse%
\isanewline
\ \ \isacommand{case}\isamarkupfalse%
\ {\isacharparenleft}{\kern0pt}Nand\ {\isasymphi}\ {\isasympsi}{\isacharparenright}{\kern0pt}\isanewline
\isanewline
\ \ \isacommand{have}\isamarkupfalse%
\ envin\ {\isacharcolon}{\kern0pt}\ {\isachardoublequoteopen}env\ {\isasymin}\ list{\isacharparenleft}{\kern0pt}M{\isacharparenright}{\kern0pt}{\isachardoublequoteclose}\ \isanewline
\ \ \ \ \isacommand{apply}\isamarkupfalse%
{\isacharparenleft}{\kern0pt}rule{\isacharunderscore}{\kern0pt}tac\ A{\isacharequal}{\kern0pt}\ {\isachardoublequoteopen}list{\isacharparenleft}{\kern0pt}HS{\isacharparenright}{\kern0pt}{\isachardoublequoteclose}\ \isakeyword{in}\ subsetD{\isacharcomma}{\kern0pt}\ rule\ list{\isacharunderscore}{\kern0pt}mono{\isacharparenright}{\kern0pt}\isanewline
\ \ \ \ \isacommand{using}\isamarkupfalse%
\ HS{\isacharunderscore}{\kern0pt}iff\ P{\isacharunderscore}{\kern0pt}name{\isacharunderscore}{\kern0pt}in{\isacharunderscore}{\kern0pt}M\ assms\ Nand\isanewline
\ \ \ \ \isacommand{by}\isamarkupfalse%
\ auto\isanewline
\ \ \isacommand{have}\isamarkupfalse%
\ mapin\ {\isacharcolon}{\kern0pt}\ {\isachardoublequoteopen}map{\isacharparenleft}{\kern0pt}{\isasymlambda}x{\isachardot}{\kern0pt}\ Pn{\isacharunderscore}{\kern0pt}auto{\isacharparenleft}{\kern0pt}{\isasympi}{\isacharparenright}{\kern0pt}\ {\isacharbackquote}{\kern0pt}\ x{\isacharcomma}{\kern0pt}\ env{\isacharparenright}{\kern0pt}\ {\isasymin}\ list{\isacharparenleft}{\kern0pt}M{\isacharparenright}{\kern0pt}{\isachardoublequoteclose}\ \ \isanewline
\ \ \ \ \isacommand{apply}\isamarkupfalse%
{\isacharparenleft}{\kern0pt}rule\ map{\isacharunderscore}{\kern0pt}type{\isacharparenright}{\kern0pt}\isanewline
\ \ \ \ \isacommand{using}\isamarkupfalse%
\ assms\ Nand\isanewline
\ \ \ \ \ \isacommand{apply}\isamarkupfalse%
\ simp\isanewline
\ \ \ \ \isacommand{apply}\isamarkupfalse%
{\isacharparenleft}{\kern0pt}rule\ Pn{\isacharunderscore}{\kern0pt}auto{\isacharunderscore}{\kern0pt}value{\isacharunderscore}{\kern0pt}in{\isacharunderscore}{\kern0pt}M{\isacharparenright}{\kern0pt}\isanewline
\ \ \ \ \isacommand{using}\isamarkupfalse%
\ assms\ HS{\isacharunderscore}{\kern0pt}iff\isanewline
\ \ \ \ \isacommand{by}\isamarkupfalse%
\ auto\isanewline
\isanewline
\ \ \isacommand{have}\isamarkupfalse%
\ arityle\ {\isacharcolon}{\kern0pt}\ {\isachardoublequoteopen}arity{\isacharparenleft}{\kern0pt}{\isasymphi}{\isacharparenright}{\kern0pt}\ {\isasymle}\ length{\isacharparenleft}{\kern0pt}env{\isacharparenright}{\kern0pt}\ {\isasymand}\ arity{\isacharparenleft}{\kern0pt}{\isasympsi}{\isacharparenright}{\kern0pt}\ {\isasymle}\ length{\isacharparenleft}{\kern0pt}env{\isacharparenright}{\kern0pt}{\isachardoublequoteclose}\isanewline
\ \ \ \ \isacommand{apply}\isamarkupfalse%
{\isacharparenleft}{\kern0pt}rule\ conjI{\isacharcomma}{\kern0pt}\ rule{\isacharunderscore}{\kern0pt}tac\ j{\isacharequal}{\kern0pt}{\isachardoublequoteopen}arity{\isacharparenleft}{\kern0pt}{\isasymphi}{\isacharparenright}{\kern0pt}\ {\isasymunion}\ arity{\isacharparenleft}{\kern0pt}{\isasympsi}{\isacharparenright}{\kern0pt}{\isachardoublequoteclose}\ \isakeyword{in}\ le{\isacharunderscore}{\kern0pt}trans{\isacharcomma}{\kern0pt}\ rule\ max{\isacharunderscore}{\kern0pt}le{\isadigit{1}}{\isacharparenright}{\kern0pt}\isanewline
\ \ \ \ \isacommand{using}\isamarkupfalse%
\ Nand\isanewline
\ \ \ \ \ \ \ \isacommand{apply}\isamarkupfalse%
\ auto{\isacharbrackleft}{\kern0pt}{\isadigit{3}}{\isacharbrackright}{\kern0pt}\isanewline
\ \ \ \ \ \isacommand{apply}\isamarkupfalse%
{\isacharparenleft}{\kern0pt}rule{\isacharunderscore}{\kern0pt}tac\ j{\isacharequal}{\kern0pt}{\isachardoublequoteopen}arity{\isacharparenleft}{\kern0pt}{\isasymphi}{\isacharparenright}{\kern0pt}\ {\isasymunion}\ arity{\isacharparenleft}{\kern0pt}{\isasympsi}{\isacharparenright}{\kern0pt}{\isachardoublequoteclose}\ \isakeyword{in}\ le{\isacharunderscore}{\kern0pt}trans{\isacharparenright}{\kern0pt}\isanewline
\ \ \ \ \isacommand{using}\isamarkupfalse%
\ Nand\ max{\isacharunderscore}{\kern0pt}le{\isadigit{2}}\isanewline
\ \ \ \ \isacommand{by}\isamarkupfalse%
\ auto\isanewline
\isanewline
\ \ \isacommand{then}\isamarkupfalse%
\ \isacommand{show}\isamarkupfalse%
\ {\isacharquery}{\kern0pt}case\isanewline
\ \ \isacommand{proof}\isamarkupfalse%
\ {\isacharminus}{\kern0pt}\ \isanewline
\ \ \ \ \isacommand{have}\isamarkupfalse%
\ {\isachardoublequoteopen}p\ {\isasymtturnstile}HS\ Nand{\isacharparenleft}{\kern0pt}{\isasymphi}{\isacharcomma}{\kern0pt}\ {\isasympsi}{\isacharparenright}{\kern0pt}\ env\ {\isasymlongleftrightarrow}\ {\isasymnot}{\isacharparenleft}{\kern0pt}{\isasymexists}q{\isasymin}M{\isachardot}{\kern0pt}\ q{\isasymin}P\ {\isasymand}\ q{\isasympreceq}p\ {\isasymand}\ {\isacharparenleft}{\kern0pt}q\ {\isasymtturnstile}HS\ {\isasymphi}\ env{\isacharparenright}{\kern0pt}\ {\isasymand}\ {\isacharparenleft}{\kern0pt}q\ {\isasymtturnstile}HS\ {\isasympsi}\ env{\isacharparenright}{\kern0pt}{\isacharparenright}{\kern0pt}{\isachardoublequoteclose}\isanewline
\ \ \ \ \ \ \isacommand{using}\isamarkupfalse%
\ ForcesHS{\isacharunderscore}{\kern0pt}Nand\ Nand\ envin\ arityle\ \isanewline
\ \ \ \ \ \ \isacommand{by}\isamarkupfalse%
\ auto\isanewline
\ \ \ \ \isacommand{also}\isamarkupfalse%
\ \isacommand{have}\isamarkupfalse%
\ {\isachardoublequoteopen}{\isachardot}{\kern0pt}{\isachardot}{\kern0pt}{\isachardot}{\kern0pt}\ {\isasymlongleftrightarrow}\ {\isasymnot}{\isacharparenleft}{\kern0pt}{\isasymexists}q{\isasymin}M{\isachardot}{\kern0pt}\ q{\isasymin}P\ {\isasymand}\ q{\isasympreceq}p\ {\isasymand}\ {\isacharparenleft}{\kern0pt}{\isasympi}{\isacharbackquote}{\kern0pt}q\ {\isasymtturnstile}HS\ {\isasymphi}\ map{\isacharparenleft}{\kern0pt}{\isasymlambda}x{\isachardot}{\kern0pt}\ Pn{\isacharunderscore}{\kern0pt}auto{\isacharparenleft}{\kern0pt}{\isasympi}{\isacharparenright}{\kern0pt}\ {\isacharbackquote}{\kern0pt}\ x{\isacharcomma}{\kern0pt}\ env{\isacharparenright}{\kern0pt}{\isacharparenright}{\kern0pt}\ {\isasymand}\ {\isacharparenleft}{\kern0pt}{\isasympi}{\isacharbackquote}{\kern0pt}q\ {\isasymtturnstile}HS\ {\isasympsi}\ map{\isacharparenleft}{\kern0pt}{\isasymlambda}x{\isachardot}{\kern0pt}\ Pn{\isacharunderscore}{\kern0pt}auto{\isacharparenleft}{\kern0pt}{\isasympi}{\isacharparenright}{\kern0pt}\ {\isacharbackquote}{\kern0pt}\ x{\isacharcomma}{\kern0pt}\ env{\isacharparenright}{\kern0pt}{\isacharparenright}{\kern0pt}{\isacharparenright}{\kern0pt}{\isachardoublequoteclose}\isanewline
\ \ \ \ \ \ \isacommand{using}\isamarkupfalse%
\ Nand\ arityle\ \isacommand{by}\isamarkupfalse%
\ auto\isanewline
\ \ \ \ \isacommand{also}\isamarkupfalse%
\ \isacommand{have}\isamarkupfalse%
\ {\isachardoublequoteopen}{\isachardot}{\kern0pt}{\isachardot}{\kern0pt}{\isachardot}{\kern0pt}\ {\isasymlongleftrightarrow}\ {\isasymnot}{\isacharparenleft}{\kern0pt}{\isasymexists}q{\isasymin}M{\isachardot}{\kern0pt}\ q{\isasymin}P\ {\isasymand}\ {\isasympi}{\isacharbackquote}{\kern0pt}q{\isasympreceq}{\isasympi}{\isacharbackquote}{\kern0pt}p\ {\isasymand}\ {\isacharparenleft}{\kern0pt}{\isasympi}{\isacharbackquote}{\kern0pt}q\ {\isasymtturnstile}HS\ {\isasymphi}\ map{\isacharparenleft}{\kern0pt}{\isasymlambda}x{\isachardot}{\kern0pt}\ Pn{\isacharunderscore}{\kern0pt}auto{\isacharparenleft}{\kern0pt}{\isasympi}{\isacharparenright}{\kern0pt}\ {\isacharbackquote}{\kern0pt}\ x{\isacharcomma}{\kern0pt}\ env{\isacharparenright}{\kern0pt}{\isacharparenright}{\kern0pt}\ {\isasymand}\ {\isacharparenleft}{\kern0pt}{\isasympi}{\isacharbackquote}{\kern0pt}q\ {\isasymtturnstile}HS\ {\isasympsi}\ map{\isacharparenleft}{\kern0pt}{\isasymlambda}x{\isachardot}{\kern0pt}\ Pn{\isacharunderscore}{\kern0pt}auto{\isacharparenleft}{\kern0pt}{\isasympi}{\isacharparenright}{\kern0pt}\ {\isacharbackquote}{\kern0pt}\ x{\isacharcomma}{\kern0pt}\ env{\isacharparenright}{\kern0pt}{\isacharparenright}{\kern0pt}{\isacharparenright}{\kern0pt}{\isachardoublequoteclose}\isanewline
\ \ \ \ \ \ \isacommand{using}\isamarkupfalse%
\ is{\isacharunderscore}{\kern0pt}P{\isacharunderscore}{\kern0pt}auto{\isacharunderscore}{\kern0pt}def\ assms\ Nand\ \isacommand{by}\isamarkupfalse%
\ auto\isanewline
\ \ \ \ \isacommand{also}\isamarkupfalse%
\ \isacommand{have}\isamarkupfalse%
\ {\isachardoublequoteopen}{\isachardot}{\kern0pt}{\isachardot}{\kern0pt}{\isachardot}{\kern0pt}\ {\isasymlongleftrightarrow}\ {\isasymnot}{\isacharparenleft}{\kern0pt}{\isasymexists}q{\isasymin}M{\isachardot}{\kern0pt}\ q{\isasymin}P\ {\isasymand}\ q{\isasympreceq}{\isasympi}{\isacharbackquote}{\kern0pt}p\ {\isasymand}\ {\isacharparenleft}{\kern0pt}q\ {\isasymtturnstile}HS\ {\isasymphi}\ map{\isacharparenleft}{\kern0pt}{\isasymlambda}x{\isachardot}{\kern0pt}\ Pn{\isacharunderscore}{\kern0pt}auto{\isacharparenleft}{\kern0pt}{\isasympi}{\isacharparenright}{\kern0pt}\ {\isacharbackquote}{\kern0pt}\ x{\isacharcomma}{\kern0pt}\ env{\isacharparenright}{\kern0pt}{\isacharparenright}{\kern0pt}\ {\isasymand}\ {\isacharparenleft}{\kern0pt}q\ {\isasymtturnstile}HS\ {\isasympsi}\ map{\isacharparenleft}{\kern0pt}{\isasymlambda}x{\isachardot}{\kern0pt}\ Pn{\isacharunderscore}{\kern0pt}auto{\isacharparenleft}{\kern0pt}{\isasympi}{\isacharparenright}{\kern0pt}\ {\isacharbackquote}{\kern0pt}\ x{\isacharcomma}{\kern0pt}\ env{\isacharparenright}{\kern0pt}{\isacharparenright}{\kern0pt}{\isacharparenright}{\kern0pt}{\isachardoublequoteclose}\isanewline
\ \ \ \ \ \ \isacommand{apply}\isamarkupfalse%
{\isacharparenleft}{\kern0pt}subgoal{\isacharunderscore}{\kern0pt}tac\ {\isachardoublequoteopen}P\ {\isasymsubseteq}\ M{\isachardoublequoteclose}{\isacharparenright}{\kern0pt}\isanewline
\ \ \ \ \ \ \ \isacommand{apply}\isamarkupfalse%
{\isacharparenleft}{\kern0pt}rule\ notnot{\isacharunderscore}{\kern0pt}iff{\isacharcomma}{\kern0pt}\ rule\ iffI{\isacharcomma}{\kern0pt}\ clarify{\isacharparenright}{\kern0pt}\isanewline
\ \ \ \ \ \ \ \ \isacommand{apply}\isamarkupfalse%
{\isacharparenleft}{\kern0pt}rename{\isacharunderscore}{\kern0pt}tac\ q{\isacharcomma}{\kern0pt}\ rule{\isacharunderscore}{\kern0pt}tac\ x{\isacharequal}{\kern0pt}{\isachardoublequoteopen}{\isasympi}{\isacharbackquote}{\kern0pt}q{\isachardoublequoteclose}\ \isakeyword{in}\ bexI{\isacharparenright}{\kern0pt}\isanewline
\ \ \ \ \ \ \isacommand{using}\isamarkupfalse%
\ assms\ P{\isacharunderscore}{\kern0pt}auto{\isacharunderscore}{\kern0pt}value\ \isanewline
\ \ \ \ \ \ \ \ \ \isacommand{apply}\isamarkupfalse%
\ auto{\isacharbrackleft}{\kern0pt}{\isadigit{2}}{\isacharbrackright}{\kern0pt}\isanewline
\ \ \ \ \ \ \ \isacommand{apply}\isamarkupfalse%
\ clarify\ \isanewline
\ \ \ \ \ \ \ \isacommand{apply}\isamarkupfalse%
{\isacharparenleft}{\kern0pt}rename{\isacharunderscore}{\kern0pt}tac\ q{\isacharcomma}{\kern0pt}\ subgoal{\isacharunderscore}{\kern0pt}tac\ {\isachardoublequoteopen}{\isasymexists}r\ {\isasymin}\ P{\isachardot}{\kern0pt}\ {\isasympi}{\isacharbackquote}{\kern0pt}r\ {\isacharequal}{\kern0pt}\ q{\isachardoublequoteclose}{\isacharparenright}{\kern0pt}\ \isanewline
\ \ \ \ \ \ \ \ \isacommand{apply}\isamarkupfalse%
\ force\isanewline
\ \ \ \ \ \ \ \isacommand{apply}\isamarkupfalse%
{\isacharparenleft}{\kern0pt}subgoal{\isacharunderscore}{\kern0pt}tac\ {\isachardoublequoteopen}{\isasympi}\ {\isasymin}\ surj{\isacharparenleft}{\kern0pt}P{\isacharcomma}{\kern0pt}\ P{\isacharparenright}{\kern0pt}{\isachardoublequoteclose}{\isacharparenright}{\kern0pt}\isanewline
\ \ \ \ \ \ \isacommand{using}\isamarkupfalse%
\ surj{\isacharunderscore}{\kern0pt}def\ \isanewline
\ \ \ \ \ \ \ \ \isacommand{apply}\isamarkupfalse%
\ force\ \isanewline
\ \ \ \ \ \ \isacommand{using}\isamarkupfalse%
\ assms\ is{\isacharunderscore}{\kern0pt}P{\isacharunderscore}{\kern0pt}auto{\isacharunderscore}{\kern0pt}def\ bij{\isacharunderscore}{\kern0pt}is{\isacharunderscore}{\kern0pt}surj\ transM\ P{\isacharunderscore}{\kern0pt}in{\isacharunderscore}{\kern0pt}M\isanewline
\ \ \ \ \ \ \isacommand{by}\isamarkupfalse%
\ auto\isanewline
\ \ \ \ \isacommand{also}\isamarkupfalse%
\ \isacommand{have}\isamarkupfalse%
\ {\isachardoublequoteopen}{\isachardot}{\kern0pt}{\isachardot}{\kern0pt}{\isachardot}{\kern0pt}\ {\isasymlongleftrightarrow}\ {\isasympi}{\isacharbackquote}{\kern0pt}p\ {\isasymtturnstile}HS\ Nand{\isacharparenleft}{\kern0pt}{\isasymphi}{\isacharcomma}{\kern0pt}\ {\isasympsi}{\isacharparenright}{\kern0pt}\ map{\isacharparenleft}{\kern0pt}{\isasymlambda}x{\isachardot}{\kern0pt}\ Pn{\isacharunderscore}{\kern0pt}auto{\isacharparenleft}{\kern0pt}{\isasympi}{\isacharparenright}{\kern0pt}\ {\isacharbackquote}{\kern0pt}\ x{\isacharcomma}{\kern0pt}\ env{\isacharparenright}{\kern0pt}{\isachardoublequoteclose}\ \isanewline
\ \ \ \ \ \ \isacommand{apply}\isamarkupfalse%
{\isacharparenleft}{\kern0pt}rule\ iff{\isacharunderscore}{\kern0pt}flip{\isacharcomma}{\kern0pt}\ rule\ ForcesHS{\isacharunderscore}{\kern0pt}Nand{\isacharparenright}{\kern0pt}\isanewline
\ \ \ \ \ \ \isacommand{using}\isamarkupfalse%
\ Nand\ mapin\ P{\isacharunderscore}{\kern0pt}auto{\isacharunderscore}{\kern0pt}value\ assms\isanewline
\ \ \ \ \ \ \isacommand{by}\isamarkupfalse%
\ auto\isanewline
\ \ \ \ \isacommand{finally}\isamarkupfalse%
\ \isacommand{show}\isamarkupfalse%
\ {\isachardoublequoteopen}M{\isacharcomma}{\kern0pt}\ {\isacharbrackleft}{\kern0pt}p{\isacharcomma}{\kern0pt}\ P{\isacharcomma}{\kern0pt}\ leq{\isacharcomma}{\kern0pt}\ one{\isacharcomma}{\kern0pt}\ {\isasymlangle}{\isasymF}{\isacharcomma}{\kern0pt}\ {\isasymG}{\isacharcomma}{\kern0pt}\ P{\isacharcomma}{\kern0pt}\ P{\isacharunderscore}{\kern0pt}auto{\isasymrangle}{\isacharbrackright}{\kern0pt}\ {\isacharat}{\kern0pt}\ env\ {\isasymTurnstile}\ forcesHS{\isacharparenleft}{\kern0pt}Nand{\isacharparenleft}{\kern0pt}{\isasymphi}{\isacharcomma}{\kern0pt}\ {\isasympsi}{\isacharparenright}{\kern0pt}{\isacharparenright}{\kern0pt}\ {\isasymlongleftrightarrow}\isanewline
\ \ \ \ \ \ \ \ \ \ \ \ \ \ \ \ \ \ M{\isacharcomma}{\kern0pt}\ {\isacharbrackleft}{\kern0pt}{\isasympi}\ {\isacharbackquote}{\kern0pt}\ p{\isacharcomma}{\kern0pt}\ P{\isacharcomma}{\kern0pt}\ leq{\isacharcomma}{\kern0pt}\ one{\isacharcomma}{\kern0pt}\ {\isasymlangle}{\isasymF}{\isacharcomma}{\kern0pt}\ {\isasymG}{\isacharcomma}{\kern0pt}\ P{\isacharcomma}{\kern0pt}\ P{\isacharunderscore}{\kern0pt}auto{\isasymrangle}{\isacharbrackright}{\kern0pt}\ {\isacharat}{\kern0pt}\ map{\isacharparenleft}{\kern0pt}{\isasymlambda}a{\isachardot}{\kern0pt}\ Pn{\isacharunderscore}{\kern0pt}auto{\isacharparenleft}{\kern0pt}{\isasympi}{\isacharparenright}{\kern0pt}\ {\isacharbackquote}{\kern0pt}\ a{\isacharcomma}{\kern0pt}\ env{\isacharparenright}{\kern0pt}\ {\isasymTurnstile}\ forcesHS{\isacharparenleft}{\kern0pt}Nand{\isacharparenleft}{\kern0pt}{\isasymphi}{\isacharcomma}{\kern0pt}\ {\isasympsi}{\isacharparenright}{\kern0pt}{\isacharparenright}{\kern0pt}{\isachardoublequoteclose}\ \isanewline
\ \ \ \ \ \ \isacommand{by}\isamarkupfalse%
\ simp\isanewline
\ \ \isacommand{qed}\isamarkupfalse%
\isanewline
\isacommand{next}\isamarkupfalse%
\isanewline
\ \ \isacommand{case}\isamarkupfalse%
\ {\isacharparenleft}{\kern0pt}Forall\ {\isasymphi}{\isacharparenright}{\kern0pt}\isanewline
\isanewline
\ \ \isacommand{have}\isamarkupfalse%
\ envin\ {\isacharcolon}{\kern0pt}\ {\isachardoublequoteopen}env\ {\isasymin}\ list{\isacharparenleft}{\kern0pt}M{\isacharparenright}{\kern0pt}{\isachardoublequoteclose}\ \isanewline
\ \ \ \ \isacommand{apply}\isamarkupfalse%
{\isacharparenleft}{\kern0pt}rule{\isacharunderscore}{\kern0pt}tac\ A{\isacharequal}{\kern0pt}\ {\isachardoublequoteopen}list{\isacharparenleft}{\kern0pt}HS{\isacharparenright}{\kern0pt}{\isachardoublequoteclose}\ \isakeyword{in}\ subsetD{\isacharcomma}{\kern0pt}\ rule\ list{\isacharunderscore}{\kern0pt}mono{\isacharparenright}{\kern0pt}\isanewline
\ \ \ \ \isacommand{using}\isamarkupfalse%
\ HS{\isacharunderscore}{\kern0pt}iff\ P{\isacharunderscore}{\kern0pt}name{\isacharunderscore}{\kern0pt}in{\isacharunderscore}{\kern0pt}M\ assms\ Forall\isanewline
\ \ \ \ \isacommand{by}\isamarkupfalse%
\ auto\isanewline
\ \ \isacommand{have}\isamarkupfalse%
\ mapin\ {\isacharcolon}{\kern0pt}\ {\isachardoublequoteopen}map{\isacharparenleft}{\kern0pt}{\isasymlambda}x{\isachardot}{\kern0pt}\ Pn{\isacharunderscore}{\kern0pt}auto{\isacharparenleft}{\kern0pt}{\isasympi}{\isacharparenright}{\kern0pt}\ {\isacharbackquote}{\kern0pt}\ x{\isacharcomma}{\kern0pt}\ env{\isacharparenright}{\kern0pt}\ {\isasymin}\ list{\isacharparenleft}{\kern0pt}M{\isacharparenright}{\kern0pt}{\isachardoublequoteclose}\ \ \isanewline
\ \ \ \ \isacommand{apply}\isamarkupfalse%
{\isacharparenleft}{\kern0pt}rule\ map{\isacharunderscore}{\kern0pt}type{\isacharparenright}{\kern0pt}\isanewline
\ \ \ \ \isacommand{using}\isamarkupfalse%
\ assms\ Forall\isanewline
\ \ \ \ \ \isacommand{apply}\isamarkupfalse%
\ simp\isanewline
\ \ \ \ \isacommand{apply}\isamarkupfalse%
{\isacharparenleft}{\kern0pt}rule\ Pn{\isacharunderscore}{\kern0pt}auto{\isacharunderscore}{\kern0pt}value{\isacharunderscore}{\kern0pt}in{\isacharunderscore}{\kern0pt}M{\isacharparenright}{\kern0pt}\isanewline
\ \ \ \ \isacommand{using}\isamarkupfalse%
\ assms\ HS{\isacharunderscore}{\kern0pt}iff\isanewline
\ \ \ \ \isacommand{by}\isamarkupfalse%
\ auto\isanewline
\isanewline
\ \ \isacommand{have}\isamarkupfalse%
\ arityle\ {\isacharcolon}{\kern0pt}\ {\isachardoublequoteopen}arity{\isacharparenleft}{\kern0pt}{\isasymphi}{\isacharparenright}{\kern0pt}\ {\isasymle}\ succ{\isacharparenleft}{\kern0pt}length{\isacharparenleft}{\kern0pt}env{\isacharparenright}{\kern0pt}{\isacharparenright}{\kern0pt}{\isachardoublequoteclose}\isanewline
\ \ \ \ \isacommand{apply}\isamarkupfalse%
{\isacharparenleft}{\kern0pt}rule{\isacharunderscore}{\kern0pt}tac\ n{\isacharequal}{\kern0pt}{\isachardoublequoteopen}arity{\isacharparenleft}{\kern0pt}{\isasymphi}{\isacharparenright}{\kern0pt}{\isachardoublequoteclose}\ \isakeyword{in}\ natE{\isacharparenright}{\kern0pt}\isanewline
\ \ \ \ \isacommand{using}\isamarkupfalse%
\ Forall\ \isanewline
\ \ \ \ \isacommand{by}\isamarkupfalse%
\ auto\isanewline
\isanewline
\ \ \isacommand{then}\isamarkupfalse%
\ \isacommand{show}\isamarkupfalse%
\ {\isacharquery}{\kern0pt}case\ \isanewline
\ \ \isacommand{proof}\isamarkupfalse%
\ {\isacharminus}{\kern0pt}\ \isanewline
\ \ \ \ \isacommand{have}\isamarkupfalse%
\ {\isachardoublequoteopen}p\ {\isasymtturnstile}HS\ Forall{\isacharparenleft}{\kern0pt}{\isasymphi}{\isacharparenright}{\kern0pt}\ env\ {\isasymlongleftrightarrow}\ {\isacharparenleft}{\kern0pt}{\isasymforall}x\ {\isasymin}\ HS{\isachardot}{\kern0pt}\ p\ {\isasymtturnstile}HS\ {\isasymphi}\ {\isacharbrackleft}{\kern0pt}x{\isacharbrackright}{\kern0pt}\ {\isacharat}{\kern0pt}\ env{\isacharparenright}{\kern0pt}{\isachardoublequoteclose}\isanewline
\ \ \ \ \ \ \isacommand{using}\isamarkupfalse%
\ Forall\ ForcesHS{\isacharunderscore}{\kern0pt}Forall\ envin\ \isacommand{by}\isamarkupfalse%
\ auto\isanewline
\ \ \ \ \isacommand{also}\isamarkupfalse%
\ \isacommand{have}\isamarkupfalse%
\ {\isachardoublequoteopen}{\isachardot}{\kern0pt}{\isachardot}{\kern0pt}{\isachardot}{\kern0pt}\ {\isasymlongleftrightarrow}\ {\isacharparenleft}{\kern0pt}{\isasymforall}x\ {\isasymin}\ HS{\isachardot}{\kern0pt}\ {\isasympi}{\isacharbackquote}{\kern0pt}p\ {\isasymtturnstile}HS\ {\isasymphi}\ map{\isacharparenleft}{\kern0pt}{\isasymlambda}x{\isachardot}{\kern0pt}\ Pn{\isacharunderscore}{\kern0pt}auto{\isacharparenleft}{\kern0pt}{\isasympi}{\isacharparenright}{\kern0pt}{\isacharbackquote}{\kern0pt}x{\isacharcomma}{\kern0pt}\ {\isacharbrackleft}{\kern0pt}x{\isacharbrackright}{\kern0pt}\ {\isacharat}{\kern0pt}\ env{\isacharparenright}{\kern0pt}{\isacharparenright}{\kern0pt}{\isachardoublequoteclose}\isanewline
\ \ \ \ \ \ \isacommand{using}\isamarkupfalse%
\ Forall\ arityle\ \isacommand{by}\isamarkupfalse%
\ auto\isanewline
\ \ \ \ \isacommand{also}\isamarkupfalse%
\ \isacommand{have}\isamarkupfalse%
\ {\isachardoublequoteopen}{\isachardot}{\kern0pt}{\isachardot}{\kern0pt}{\isachardot}{\kern0pt}\ {\isasymlongleftrightarrow}\ {\isacharparenleft}{\kern0pt}{\isasymforall}x\ {\isasymin}\ HS{\isachardot}{\kern0pt}\ {\isasympi}{\isacharbackquote}{\kern0pt}p\ {\isasymtturnstile}HS\ {\isasymphi}\ {\isacharbrackleft}{\kern0pt}Pn{\isacharunderscore}{\kern0pt}auto{\isacharparenleft}{\kern0pt}{\isasympi}{\isacharparenright}{\kern0pt}{\isacharbackquote}{\kern0pt}x{\isacharbrackright}{\kern0pt}\ {\isacharat}{\kern0pt}\ map{\isacharparenleft}{\kern0pt}{\isasymlambda}x{\isachardot}{\kern0pt}\ Pn{\isacharunderscore}{\kern0pt}auto{\isacharparenleft}{\kern0pt}{\isasympi}{\isacharparenright}{\kern0pt}{\isacharbackquote}{\kern0pt}x{\isacharcomma}{\kern0pt}\ env{\isacharparenright}{\kern0pt}{\isacharparenright}{\kern0pt}{\isachardoublequoteclose}\isanewline
\ \ \ \ \ \ \isacommand{by}\isamarkupfalse%
\ auto\isanewline
\ \ \ \ \isacommand{also}\isamarkupfalse%
\ \isacommand{have}\isamarkupfalse%
\ {\isachardoublequoteopen}{\isachardot}{\kern0pt}{\isachardot}{\kern0pt}{\isachardot}{\kern0pt}\ {\isasymlongleftrightarrow}\ {\isacharparenleft}{\kern0pt}{\isasymforall}x\ {\isasymin}\ HS{\isachardot}{\kern0pt}\ {\isasympi}{\isacharbackquote}{\kern0pt}p\ {\isasymtturnstile}HS\ {\isasymphi}\ {\isacharbrackleft}{\kern0pt}x{\isacharbrackright}{\kern0pt}\ {\isacharat}{\kern0pt}\ map{\isacharparenleft}{\kern0pt}{\isasymlambda}x{\isachardot}{\kern0pt}\ Pn{\isacharunderscore}{\kern0pt}auto{\isacharparenleft}{\kern0pt}{\isasympi}{\isacharparenright}{\kern0pt}{\isacharbackquote}{\kern0pt}x{\isacharcomma}{\kern0pt}\ env{\isacharparenright}{\kern0pt}{\isacharparenright}{\kern0pt}{\isachardoublequoteclose}\isanewline
\ \ \ \ \ \ \isacommand{apply}\isamarkupfalse%
{\isacharparenleft}{\kern0pt}rule\ iffI{\isacharcomma}{\kern0pt}\ rule\ ballI{\isacharparenright}{\kern0pt}\isanewline
\ \ \ \ \ \ \ \isacommand{apply}\isamarkupfalse%
{\isacharparenleft}{\kern0pt}rename{\isacharunderscore}{\kern0pt}tac\ x{\isacharcomma}{\kern0pt}\ subgoal{\isacharunderscore}{\kern0pt}tac\ {\isachardoublequoteopen}{\isasymexists}y\ {\isasymin}\ P{\isacharunderscore}{\kern0pt}names{\isachardot}{\kern0pt}\ Pn{\isacharunderscore}{\kern0pt}auto{\isacharparenleft}{\kern0pt}{\isasympi}{\isacharparenright}{\kern0pt}{\isacharbackquote}{\kern0pt}y\ {\isacharequal}{\kern0pt}\ x{\isachardoublequoteclose}{\isacharparenright}{\kern0pt}\isanewline
\ \ \ \ \ \ \isacommand{apply}\isamarkupfalse%
\ clarify\isanewline
\ \ \ \ \ \ \isacommand{using}\isamarkupfalse%
\ HS{\isacharunderscore}{\kern0pt}Pn{\isacharunderscore}{\kern0pt}auto\ assms\ \isanewline
\ \ \ \ \ \ \ \ \isacommand{apply}\isamarkupfalse%
\ force\ \isanewline
\ \ \ \ \ \ \ \isacommand{apply}\isamarkupfalse%
{\isacharparenleft}{\kern0pt}rename{\isacharunderscore}{\kern0pt}tac\ x{\isacharcomma}{\kern0pt}\ subgoal{\isacharunderscore}{\kern0pt}tac\ {\isachardoublequoteopen}Pn{\isacharunderscore}{\kern0pt}auto{\isacharparenleft}{\kern0pt}{\isasympi}{\isacharparenright}{\kern0pt}\ {\isasymin}\ bij{\isacharparenleft}{\kern0pt}P{\isacharunderscore}{\kern0pt}names{\isacharcomma}{\kern0pt}\ P{\isacharunderscore}{\kern0pt}names{\isacharparenright}{\kern0pt}{\isachardoublequoteclose}{\isacharparenright}{\kern0pt}\ \isanewline
\ \ \ \ \ \ \isacommand{using}\isamarkupfalse%
\ bij{\isacharunderscore}{\kern0pt}is{\isacharunderscore}{\kern0pt}surj\ surj{\isacharunderscore}{\kern0pt}def\ HS{\isacharunderscore}{\kern0pt}iff\ \isanewline
\ \ \ \ \ \ \ \ \isacommand{apply}\isamarkupfalse%
\ force\ \isanewline
\ \ \ \ \ \ \ \isacommand{apply}\isamarkupfalse%
{\isacharparenleft}{\kern0pt}rule\ Pn{\isacharunderscore}{\kern0pt}auto{\isacharunderscore}{\kern0pt}bij{\isacharparenright}{\kern0pt}\isanewline
\ \ \ \ \ \ \isacommand{using}\isamarkupfalse%
\ assms\ \isanewline
\ \ \ \ \ \ \ \isacommand{apply}\isamarkupfalse%
\ simp\isanewline
\ \ \ \ \ \ \isacommand{apply}\isamarkupfalse%
{\isacharparenleft}{\kern0pt}rule\ ballI{\isacharparenright}{\kern0pt}\isanewline
\ \ \ \ \ \ \isacommand{apply}\isamarkupfalse%
{\isacharparenleft}{\kern0pt}rename{\isacharunderscore}{\kern0pt}tac\ x{\isacharcomma}{\kern0pt}\ subgoal{\isacharunderscore}{\kern0pt}tac\ {\isachardoublequoteopen}Pn{\isacharunderscore}{\kern0pt}auto{\isacharparenleft}{\kern0pt}{\isasympi}{\isacharparenright}{\kern0pt}{\isacharbackquote}{\kern0pt}x\ {\isasymin}\ HS{\isachardoublequoteclose}{\isacharparenright}{\kern0pt}\isanewline
\ \ \ \ \ \ \isacommand{using}\isamarkupfalse%
\ assms\ HS{\isacharunderscore}{\kern0pt}iff\isanewline
\ \ \ \ \ \ \ \isacommand{apply}\isamarkupfalse%
\ force\ \isanewline
\ \ \ \ \ \ \isacommand{apply}\isamarkupfalse%
{\isacharparenleft}{\kern0pt}rule\ iffD{\isadigit{1}}{\isacharcomma}{\kern0pt}\ rule\ HS{\isacharunderscore}{\kern0pt}Pn{\isacharunderscore}{\kern0pt}auto{\isacharparenright}{\kern0pt}\isanewline
\ \ \ \ \ \ \isacommand{using}\isamarkupfalse%
\ assms\ HS{\isacharunderscore}{\kern0pt}iff\isanewline
\ \ \ \ \ \ \isacommand{by}\isamarkupfalse%
\ auto\isanewline
\ \ \ \ \isacommand{also}\isamarkupfalse%
\ \isacommand{have}\isamarkupfalse%
\ {\isachardoublequoteopen}{\isachardot}{\kern0pt}{\isachardot}{\kern0pt}{\isachardot}{\kern0pt}\ {\isasymlongleftrightarrow}\ {\isasympi}{\isacharbackquote}{\kern0pt}p\ {\isasymtturnstile}HS\ Forall{\isacharparenleft}{\kern0pt}{\isasymphi}{\isacharparenright}{\kern0pt}\ map{\isacharparenleft}{\kern0pt}{\isasymlambda}x{\isachardot}{\kern0pt}\ Pn{\isacharunderscore}{\kern0pt}auto{\isacharparenleft}{\kern0pt}{\isasympi}{\isacharparenright}{\kern0pt}{\isacharbackquote}{\kern0pt}x{\isacharcomma}{\kern0pt}\ env{\isacharparenright}{\kern0pt}{\isachardoublequoteclose}\ \isanewline
\ \ \ \ \ \ \isacommand{apply}\isamarkupfalse%
{\isacharparenleft}{\kern0pt}rule\ iff{\isacharunderscore}{\kern0pt}flip{\isacharcomma}{\kern0pt}\ rule\ ForcesHS{\isacharunderscore}{\kern0pt}Forall{\isacharparenright}{\kern0pt}\isanewline
\ \ \ \ \ \ \isacommand{using}\isamarkupfalse%
\ P{\isacharunderscore}{\kern0pt}auto{\isacharunderscore}{\kern0pt}value\ Forall\ assms\ mapin\ \isanewline
\ \ \ \ \ \ \isacommand{by}\isamarkupfalse%
\ auto\isanewline
\ \ \ \ \isacommand{finally}\isamarkupfalse%
\ \isacommand{show}\isamarkupfalse%
\ {\isachardoublequoteopen}M{\isacharcomma}{\kern0pt}\ {\isacharbrackleft}{\kern0pt}p{\isacharcomma}{\kern0pt}\ P{\isacharcomma}{\kern0pt}\ leq{\isacharcomma}{\kern0pt}\ one{\isacharcomma}{\kern0pt}\ {\isasymlangle}{\isasymF}{\isacharcomma}{\kern0pt}\ {\isasymG}{\isacharcomma}{\kern0pt}\ P{\isacharcomma}{\kern0pt}\ P{\isacharunderscore}{\kern0pt}auto{\isasymrangle}{\isacharbrackright}{\kern0pt}\ {\isacharat}{\kern0pt}\ env\ {\isasymTurnstile}\ forcesHS{\isacharparenleft}{\kern0pt}Forall{\isacharparenleft}{\kern0pt}{\isasymphi}{\isacharparenright}{\kern0pt}{\isacharparenright}{\kern0pt}\ {\isasymlongleftrightarrow}\isanewline
\ \ \ \ \ \ \ \ \ \ \ \ \ \ \ \ \ \ M{\isacharcomma}{\kern0pt}\ {\isacharbrackleft}{\kern0pt}{\isasympi}\ {\isacharbackquote}{\kern0pt}\ p{\isacharcomma}{\kern0pt}\ P{\isacharcomma}{\kern0pt}\ leq{\isacharcomma}{\kern0pt}\ one{\isacharcomma}{\kern0pt}\ {\isasymlangle}{\isasymF}{\isacharcomma}{\kern0pt}\ {\isasymG}{\isacharcomma}{\kern0pt}\ P{\isacharcomma}{\kern0pt}\ P{\isacharunderscore}{\kern0pt}auto{\isasymrangle}{\isacharbrackright}{\kern0pt}\ {\isacharat}{\kern0pt}\ map{\isacharparenleft}{\kern0pt}{\isasymlambda}x{\isachardot}{\kern0pt}\ Pn{\isacharunderscore}{\kern0pt}auto{\isacharparenleft}{\kern0pt}{\isasympi}{\isacharparenright}{\kern0pt}\ {\isacharbackquote}{\kern0pt}\ x{\isacharcomma}{\kern0pt}\ env{\isacharparenright}{\kern0pt}\ {\isasymTurnstile}\ forcesHS{\isacharparenleft}{\kern0pt}Forall{\isacharparenleft}{\kern0pt}{\isasymphi}{\isacharparenright}{\kern0pt}{\isacharparenright}{\kern0pt}{\isachardoublequoteclose}\isanewline
\ \ \ \ \ \ \isacommand{by}\isamarkupfalse%
\ simp\isanewline
\ \ \isacommand{qed}\isamarkupfalse%
\isanewline
\isacommand{qed}\isamarkupfalse%
%
\endisatagproof
{\isafoldproof}%
%
\isadelimproof
\ \isanewline
%
\endisadelimproof
\isanewline
\isacommand{end}\isamarkupfalse%
\isanewline
%
\isadelimtheory
%
\endisadelimtheory
%
\isatagtheory
\isacommand{end}\isamarkupfalse%
%
\endisatagtheory
{\isafoldtheory}%
%
\isadelimtheory
%
\endisadelimtheory
%
\end{isabellebody}%
\endinput
%:%file=~/source/repos/ZF-notAC/code/Symmetry_Lemma.thy%:%
%:%10=1%:%
%:%11=1%:%
%:%12=2%:%
%:%13=3%:%
%:%14=4%:%
%:%15=5%:%
%:%20=5%:%
%:%23=6%:%
%:%24=7%:%
%:%25=7%:%
%:%26=8%:%
%:%27=9%:%
%:%28=10%:%
%:%29=10%:%
%:%30=11%:%
%:%32=13%:%
%:%39=14%:%
%:%40=14%:%
%:%41=15%:%
%:%42=15%:%
%:%43=16%:%
%:%44=16%:%
%:%45=17%:%
%:%46=18%:%
%:%47=18%:%
%:%48=19%:%
%:%49=19%:%
%:%50=20%:%
%:%51=20%:%
%:%52=21%:%
%:%53=22%:%
%:%54=22%:%
%:%55=23%:%
%:%56=23%:%
%:%57=24%:%
%:%58=24%:%
%:%59=25%:%
%:%60=25%:%
%:%61=26%:%
%:%62=26%:%
%:%63=27%:%
%:%64=27%:%
%:%65=28%:%
%:%66=29%:%
%:%67=30%:%
%:%68=31%:%
%:%69=31%:%
%:%70=31%:%
%:%71=31%:%
%:%72=32%:%
%:%73=33%:%
%:%74=33%:%
%:%75=34%:%
%:%76=34%:%
%:%77=35%:%
%:%78=35%:%
%:%79=36%:%
%:%80=36%:%
%:%81=37%:%
%:%82=37%:%
%:%83=37%:%
%:%84=38%:%
%:%85=38%:%
%:%86=39%:%
%:%87=39%:%
%:%88=40%:%
%:%89=40%:%
%:%90=41%:%
%:%91=41%:%
%:%92=41%:%
%:%93=42%:%
%:%94=42%:%
%:%95=43%:%
%:%96=43%:%
%:%97=44%:%
%:%98=44%:%
%:%99=45%:%
%:%100=45%:%
%:%101=46%:%
%:%102=46%:%
%:%103=47%:%
%:%104=47%:%
%:%105=48%:%
%:%106=48%:%
%:%107=49%:%
%:%108=49%:%
%:%109=50%:%
%:%110=50%:%
%:%111=51%:%
%:%112=51%:%
%:%113=52%:%
%:%114=52%:%
%:%115=53%:%
%:%116=53%:%
%:%117=54%:%
%:%118=54%:%
%:%119=54%:%
%:%120=55%:%
%:%121=55%:%
%:%122=55%:%
%:%123=55%:%
%:%124=55%:%
%:%125=56%:%
%:%126=56%:%
%:%127=57%:%
%:%128=57%:%
%:%129=58%:%
%:%130=58%:%
%:%131=59%:%
%:%132=59%:%
%:%133=60%:%
%:%134=60%:%
%:%135=61%:%
%:%136=61%:%
%:%137=62%:%
%:%138=62%:%
%:%139=63%:%
%:%140=63%:%
%:%141=63%:%
%:%142=63%:%
%:%143=64%:%
%:%144=64%:%
%:%145=64%:%
%:%146=65%:%
%:%147=65%:%
%:%148=66%:%
%:%149=66%:%
%:%150=67%:%
%:%151=67%:%
%:%152=68%:%
%:%153=68%:%
%:%154=69%:%
%:%155=69%:%
%:%156=70%:%
%:%157=70%:%
%:%158=71%:%
%:%159=71%:%
%:%160=72%:%
%:%161=72%:%
%:%162=73%:%
%:%163=73%:%
%:%164=74%:%
%:%165=74%:%
%:%166=75%:%
%:%167=75%:%
%:%168=76%:%
%:%169=76%:%
%:%170=77%:%
%:%171=77%:%
%:%172=78%:%
%:%173=78%:%
%:%174=79%:%
%:%175=79%:%
%:%176=80%:%
%:%177=80%:%
%:%178=81%:%
%:%179=81%:%
%:%180=82%:%
%:%181=82%:%
%:%182=83%:%
%:%183=83%:%
%:%184=84%:%
%:%185=84%:%
%:%186=85%:%
%:%187=85%:%
%:%188=86%:%
%:%189=86%:%
%:%190=86%:%
%:%191=87%:%
%:%192=87%:%
%:%193=88%:%
%:%194=88%:%
%:%195=89%:%
%:%196=89%:%
%:%197=90%:%
%:%198=90%:%
%:%199=91%:%
%:%200=91%:%
%:%201=92%:%
%:%202=92%:%
%:%203=93%:%
%:%204=93%:%
%:%205=94%:%
%:%206=94%:%
%:%207=95%:%
%:%208=95%:%
%:%209=96%:%
%:%210=96%:%
%:%211=97%:%
%:%212=97%:%
%:%213=98%:%
%:%214=98%:%
%:%215=99%:%
%:%216=99%:%
%:%217=100%:%
%:%218=100%:%
%:%219=101%:%
%:%220=101%:%
%:%221=102%:%
%:%222=102%:%
%:%223=103%:%
%:%224=103%:%
%:%225=104%:%
%:%226=104%:%
%:%227=105%:%
%:%228=105%:%
%:%229=106%:%
%:%230=106%:%
%:%231=107%:%
%:%232=107%:%
%:%233=107%:%
%:%234=108%:%
%:%235=108%:%
%:%236=108%:%
%:%237=108%:%
%:%238=108%:%
%:%239=109%:%
%:%240=109%:%
%:%241=110%:%
%:%242=111%:%
%:%243=111%:%
%:%244=111%:%
%:%245=111%:%
%:%246=112%:%
%:%247=112%:%
%:%248=113%:%
%:%249=113%:%
%:%250=114%:%
%:%251=114%:%
%:%252=115%:%
%:%253=115%:%
%:%254=116%:%
%:%255=116%:%
%:%256=116%:%
%:%257=116%:%
%:%258=117%:%
%:%259=118%:%
%:%260=118%:%
%:%261=119%:%
%:%262=119%:%
%:%263=119%:%
%:%264=119%:%
%:%265=120%:%
%:%266=120%:%
%:%267=120%:%
%:%268=120%:%
%:%269=120%:%
%:%270=120%:%
%:%271=121%:%
%:%272=122%:%
%:%273=122%:%
%:%274=123%:%
%:%275=123%:%
%:%276=124%:%
%:%277=124%:%
%:%278=125%:%
%:%279=125%:%
%:%280=126%:%
%:%281=126%:%
%:%282=127%:%
%:%283=127%:%
%:%284=127%:%
%:%285=127%:%
%:%286=127%:%
%:%287=128%:%
%:%288=129%:%
%:%289=129%:%
%:%290=129%:%
%:%291=129%:%
%:%292=130%:%
%:%293=131%:%
%:%294=131%:%
%:%295=131%:%
%:%296=131%:%
%:%297=132%:%
%:%298=133%:%
%:%299=133%:%
%:%300=133%:%
%:%301=134%:%
%:%302=134%:%
%:%303=135%:%
%:%304=135%:%
%:%305=136%:%
%:%306=136%:%
%:%307=137%:%
%:%308=137%:%
%:%309=138%:%
%:%310=138%:%
%:%311=139%:%
%:%312=139%:%
%:%313=140%:%
%:%314=140%:%
%:%315=141%:%
%:%316=141%:%
%:%317=142%:%
%:%318=142%:%
%:%319=143%:%
%:%320=143%:%
%:%321=144%:%
%:%322=144%:%
%:%323=145%:%
%:%324=145%:%
%:%325=146%:%
%:%326=146%:%
%:%327=147%:%
%:%328=147%:%
%:%329=148%:%
%:%330=148%:%
%:%331=149%:%
%:%332=149%:%
%:%333=150%:%
%:%334=150%:%
%:%335=151%:%
%:%336=151%:%
%:%337=152%:%
%:%338=152%:%
%:%339=153%:%
%:%340=153%:%
%:%341=154%:%
%:%342=154%:%
%:%343=155%:%
%:%344=155%:%
%:%345=156%:%
%:%346=156%:%
%:%347=157%:%
%:%348=157%:%
%:%349=158%:%
%:%350=158%:%
%:%351=159%:%
%:%352=159%:%
%:%353=160%:%
%:%354=160%:%
%:%355=161%:%
%:%356=161%:%
%:%357=162%:%
%:%358=162%:%
%:%359=163%:%
%:%360=163%:%
%:%361=164%:%
%:%362=164%:%
%:%363=165%:%
%:%364=165%:%
%:%365=166%:%
%:%366=166%:%
%:%367=167%:%
%:%368=167%:%
%:%369=168%:%
%:%370=168%:%
%:%371=169%:%
%:%372=169%:%
%:%373=170%:%
%:%374=170%:%
%:%375=171%:%
%:%376=171%:%
%:%377=172%:%
%:%378=172%:%
%:%379=173%:%
%:%380=173%:%
%:%381=174%:%
%:%382=174%:%
%:%383=175%:%
%:%384=175%:%
%:%385=176%:%
%:%386=176%:%
%:%387=177%:%
%:%388=177%:%
%:%389=178%:%
%:%390=178%:%
%:%391=179%:%
%:%392=179%:%
%:%393=180%:%
%:%394=180%:%
%:%395=180%:%
%:%396=180%:%
%:%397=180%:%
%:%398=181%:%
%:%399=181%:%
%:%400=182%:%
%:%401=182%:%
%:%402=182%:%
%:%403=183%:%
%:%404=183%:%
%:%405=184%:%
%:%406=184%:%
%:%407=185%:%
%:%408=185%:%
%:%409=186%:%
%:%410=186%:%
%:%411=187%:%
%:%412=187%:%
%:%413=187%:%
%:%414=187%:%
%:%415=187%:%
%:%416=188%:%
%:%417=188%:%
%:%418=189%:%
%:%419=190%:%
%:%420=190%:%
%:%421=191%:%
%:%422=191%:%
%:%423=191%:%
%:%424=192%:%
%:%425=192%:%
%:%426=193%:%
%:%427=193%:%
%:%428=194%:%
%:%429=194%:%
%:%430=195%:%
%:%431=196%:%
%:%432=197%:%
%:%433=198%:%
%:%434=199%:%
%:%435=199%:%
%:%436=199%:%
%:%437=200%:%
%:%438=201%:%
%:%439=201%:%
%:%440=202%:%
%:%441=203%:%
%:%442=203%:%
%:%443=204%:%
%:%444=204%:%
%:%445=205%:%
%:%446=205%:%
%:%447=206%:%
%:%448=206%:%
%:%449=207%:%
%:%450=207%:%
%:%451=208%:%
%:%452=208%:%
%:%453=209%:%
%:%454=209%:%
%:%455=210%:%
%:%456=210%:%
%:%457=211%:%
%:%458=211%:%
%:%459=212%:%
%:%460=212%:%
%:%461=213%:%
%:%462=213%:%
%:%463=214%:%
%:%464=214%:%
%:%465=215%:%
%:%466=215%:%
%:%467=216%:%
%:%468=216%:%
%:%469=217%:%
%:%470=217%:%
%:%471=218%:%
%:%472=218%:%
%:%473=219%:%
%:%474=219%:%
%:%475=220%:%
%:%476=220%:%
%:%477=221%:%
%:%478=221%:%
%:%479=222%:%
%:%480=222%:%
%:%481=223%:%
%:%482=223%:%
%:%483=224%:%
%:%484=224%:%
%:%485=225%:%
%:%486=226%:%
%:%487=226%:%
%:%488=227%:%
%:%489=227%:%
%:%490=227%:%
%:%491=228%:%
%:%492=228%:%
%:%493=228%:%
%:%495=230%:%
%:%496=231%:%
%:%497=231%:%
%:%498=232%:%
%:%499=232%:%
%:%500=233%:%
%:%501=233%:%
%:%502=233%:%
%:%503=234%:%
%:%504=235%:%
%:%505=236%:%
%:%506=237%:%
%:%507=237%:%
%:%508=238%:%
%:%509=238%:%
%:%510=238%:%
%:%511=238%:%
%:%512=238%:%
%:%513=238%:%
%:%514=239%:%
%:%515=239%:%
%:%516=239%:%
%:%517=240%:%
%:%518=240%:%
%:%519=241%:%
%:%520=241%:%
%:%521=242%:%
%:%522=242%:%
%:%523=243%:%
%:%524=244%:%
%:%525=244%:%
%:%526=245%:%
%:%527=245%:%
%:%528=246%:%
%:%529=246%:%
%:%530=247%:%
%:%531=247%:%
%:%532=248%:%
%:%533=248%:%
%:%534=249%:%
%:%535=249%:%
%:%536=250%:%
%:%537=250%:%
%:%538=251%:%
%:%539=251%:%
%:%540=252%:%
%:%541=252%:%
%:%542=253%:%
%:%543=253%:%
%:%544=254%:%
%:%545=254%:%
%:%546=255%:%
%:%547=255%:%
%:%548=255%:%
%:%549=256%:%
%:%550=256%:%
%:%551=256%:%
%:%552=256%:%
%:%553=257%:%
%:%554=258%:%
%:%555=258%:%
%:%556=259%:%
%:%557=259%:%
%:%558=260%:%
%:%559=260%:%
%:%560=260%:%
%:%561=261%:%
%:%562=261%:%
%:%563=261%:%
%:%564=261%:%
%:%565=262%:%
%:%566=262%:%
%:%567=262%:%
%:%568=263%:%
%:%569=263%:%
%:%570=263%:%
%:%571=264%:%
%:%572=264%:%
%:%573=264%:%
%:%574=265%:%
%:%575=265%:%
%:%576=266%:%
%:%577=266%:%
%:%578=267%:%
%:%579=267%:%
%:%580=268%:%
%:%581=268%:%
%:%582=269%:%
%:%583=269%:%
%:%584=269%:%
%:%585=269%:%
%:%586=270%:%
%:%587=270%:%
%:%588=270%:%
%:%589=271%:%
%:%590=271%:%
%:%591=272%:%
%:%592=272%:%
%:%593=273%:%
%:%594=273%:%
%:%595=274%:%
%:%596=274%:%
%:%597=275%:%
%:%598=275%:%
%:%599=276%:%
%:%600=276%:%
%:%601=277%:%
%:%602=277%:%
%:%603=278%:%
%:%604=278%:%
%:%605=279%:%
%:%606=279%:%
%:%607=280%:%
%:%608=280%:%
%:%609=281%:%
%:%610=281%:%
%:%611=282%:%
%:%612=282%:%
%:%613=283%:%
%:%614=283%:%
%:%615=284%:%
%:%616=284%:%
%:%617=285%:%
%:%618=285%:%
%:%619=286%:%
%:%620=286%:%
%:%621=287%:%
%:%622=287%:%
%:%623=288%:%
%:%624=288%:%
%:%625=289%:%
%:%626=289%:%
%:%627=290%:%
%:%628=290%:%
%:%629=291%:%
%:%630=291%:%
%:%631=292%:%
%:%632=292%:%
%:%633=292%:%
%:%634=292%:%
%:%635=292%:%
%:%636=293%:%
%:%637=293%:%
%:%638=293%:%
%:%639=293%:%
%:%640=294%:%
%:%641=294%:%
%:%642=295%:%
%:%643=296%:%
%:%644=296%:%
%:%645=297%:%
%:%646=297%:%
%:%647=297%:%
%:%648=298%:%
%:%649=298%:%
%:%650=298%:%
%:%651=299%:%
%:%652=300%:%
%:%653=300%:%
%:%654=300%:%
%:%655=301%:%
%:%656=301%:%
%:%657=301%:%
%:%658=302%:%
%:%659=303%:%
%:%660=303%:%
%:%661=304%:%
%:%662=304%:%
%:%663=304%:%
%:%664=305%:%
%:%665=305%:%
%:%666=305%:%
%:%667=306%:%
%:%668=306%:%
%:%669=307%:%
%:%670=307%:%
%:%671=308%:%
%:%672=308%:%
%:%673=309%:%
%:%674=309%:%
%:%675=310%:%
%:%676=310%:%
%:%677=311%:%
%:%678=311%:%
%:%679=311%:%
%:%680=312%:%
%:%681=312%:%
%:%682=312%:%
%:%683=313%:%
%:%684=313%:%
%:%685=313%:%
%:%686=314%:%
%:%687=314%:%
%:%688=315%:%
%:%689=315%:%
%:%690=316%:%
%:%691=316%:%
%:%692=317%:%
%:%693=317%:%
%:%694=318%:%
%:%695=318%:%
%:%696=319%:%
%:%697=319%:%
%:%698=319%:%
%:%699=319%:%
%:%700=319%:%
%:%701=320%:%
%:%702=320%:%
%:%703=320%:%
%:%704=320%:%
%:%705=321%:%
%:%706=321%:%
%:%707=322%:%
%:%708=322%:%
%:%709=323%:%
%:%710=323%:%
%:%711=324%:%
%:%712=325%:%
%:%713=326%:%
%:%714=327%:%
%:%715=328%:%
%:%716=328%:%
%:%717=328%:%
%:%718=328%:%
%:%719=329%:%
%:%720=330%:%
%:%721=330%:%
%:%722=331%:%
%:%723=331%:%
%:%724=332%:%
%:%725=332%:%
%:%726=332%:%
%:%727=333%:%
%:%728=333%:%
%:%729=333%:%
%:%730=333%:%
%:%731=333%:%
%:%732=334%:%
%:%733=334%:%
%:%734=334%:%
%:%735=335%:%
%:%736=335%:%
%:%737=336%:%
%:%738=336%:%
%:%739=337%:%
%:%740=337%:%
%:%741=337%:%
%:%742=338%:%
%:%743=338%:%
%:%744=339%:%
%:%745=339%:%
%:%746=340%:%
%:%747=340%:%
%:%748=340%:%
%:%749=340%:%
%:%750=341%:%
%:%751=341%:%
%:%752=342%:%
%:%753=343%:%
%:%754=343%:%
%:%755=344%:%
%:%756=344%:%
%:%757=345%:%
%:%758=345%:%
%:%759=346%:%
%:%760=346%:%
%:%761=347%:%
%:%762=347%:%
%:%763=348%:%
%:%764=348%:%
%:%765=349%:%
%:%766=349%:%
%:%767=350%:%
%:%768=350%:%
%:%769=351%:%
%:%770=351%:%
%:%771=352%:%
%:%772=352%:%
%:%773=353%:%
%:%774=353%:%
%:%775=354%:%
%:%776=354%:%
%:%777=355%:%
%:%778=355%:%
%:%779=356%:%
%:%780=356%:%
%:%781=357%:%
%:%782=358%:%
%:%783=358%:%
%:%784=359%:%
%:%785=359%:%
%:%786=359%:%
%:%787=360%:%
%:%788=361%:%
%:%789=361%:%
%:%790=362%:%
%:%791=362%:%
%:%792=363%:%
%:%793=363%:%
%:%794=364%:%
%:%795=364%:%
%:%796=365%:%
%:%797=365%:%
%:%798=366%:%
%:%799=366%:%
%:%800=367%:%
%:%801=367%:%
%:%802=368%:%
%:%803=368%:%
%:%804=369%:%
%:%805=369%:%
%:%806=370%:%
%:%807=370%:%
%:%808=371%:%
%:%809=371%:%
%:%810=372%:%
%:%811=372%:%
%:%812=372%:%
%:%813=373%:%
%:%814=373%:%
%:%815=373%:%
%:%816=374%:%
%:%817=374%:%
%:%818=374%:%
%:%819=375%:%
%:%820=375%:%
%:%821=376%:%
%:%822=376%:%
%:%823=377%:%
%:%824=377%:%
%:%825=378%:%
%:%826=378%:%
%:%827=379%:%
%:%828=379%:%
%:%829=380%:%
%:%830=380%:%
%:%831=381%:%
%:%832=381%:%
%:%833=382%:%
%:%834=382%:%
%:%835=383%:%
%:%836=383%:%
%:%837=384%:%
%:%838=384%:%
%:%839=385%:%
%:%840=385%:%
%:%841=385%:%
%:%842=385%:%
%:%843=386%:%
%:%844=386%:%
%:%845=387%:%
%:%846=387%:%
%:%847=387%:%
%:%848=388%:%
%:%849=388%:%
%:%850=389%:%
%:%851=389%:%
%:%852=390%:%
%:%853=390%:%
%:%854=391%:%
%:%855=391%:%
%:%856=392%:%
%:%857=392%:%
%:%858=392%:%
%:%859=393%:%
%:%860=393%:%
%:%861=393%:%
%:%862=394%:%
%:%863=394%:%
%:%864=395%:%
%:%865=396%:%
%:%866=396%:%
%:%867=397%:%
%:%868=397%:%
%:%869=398%:%
%:%870=398%:%
%:%871=399%:%
%:%872=399%:%
%:%873=400%:%
%:%874=400%:%
%:%875=400%:%
%:%876=401%:%
%:%877=402%:%
%:%878=402%:%
%:%879=402%:%
%:%880=403%:%
%:%881=403%:%
%:%882=404%:%
%:%883=404%:%
%:%884=405%:%
%:%885=405%:%
%:%886=406%:%
%:%887=406%:%
%:%888=407%:%
%:%889=407%:%
%:%890=407%:%
%:%891=408%:%
%:%892=408%:%
%:%893=409%:%
%:%894=409%:%
%:%895=410%:%
%:%896=410%:%
%:%897=411%:%
%:%898=411%:%
%:%899=412%:%
%:%900=412%:%
%:%901=412%:%
%:%902=413%:%
%:%903=413%:%
%:%904=414%:%
%:%905=414%:%
%:%906=415%:%
%:%907=415%:%
%:%908=416%:%
%:%909=416%:%
%:%910=417%:%
%:%911=417%:%
%:%912=418%:%
%:%913=418%:%
%:%914=419%:%
%:%915=419%:%
%:%916=420%:%
%:%917=420%:%
%:%918=420%:%
%:%919=421%:%
%:%920=421%:%
%:%921=422%:%
%:%922=422%:%
%:%923=423%:%
%:%924=423%:%
%:%925=424%:%
%:%926=424%:%
%:%927=425%:%
%:%928=425%:%
%:%929=425%:%
%:%930=425%:%
%:%931=426%:%
%:%932=426%:%
%:%933=426%:%
%:%934=427%:%
%:%935=427%:%
%:%936=428%:%
%:%937=428%:%
%:%938=429%:%
%:%939=429%:%
%:%940=430%:%
%:%941=430%:%
%:%942=431%:%
%:%943=431%:%
%:%944=432%:%
%:%945=432%:%
%:%946=433%:%
%:%947=433%:%
%:%948=434%:%
%:%949=434%:%
%:%950=435%:%
%:%951=435%:%
%:%952=436%:%
%:%953=436%:%
%:%954=436%:%
%:%955=437%:%
%:%956=437%:%
%:%957=437%:%
%:%958=438%:%
%:%959=438%:%
%:%960=439%:%
%:%961=440%:%
%:%962=440%:%
%:%963=440%:%
%:%964=441%:%
%:%965=442%:%
%:%966=442%:%
%:%967=443%:%
%:%968=443%:%
%:%969=444%:%
%:%970=444%:%
%:%971=445%:%
%:%972=445%:%
%:%973=445%:%
%:%974=446%:%
%:%975=447%:%
%:%976=447%:%
%:%977=448%:%
%:%978=448%:%
%:%979=449%:%
%:%980=449%:%
%:%981=450%:%
%:%982=450%:%
%:%983=451%:%
%:%984=451%:%
%:%985=452%:%
%:%986=452%:%
%:%987=453%:%
%:%988=453%:%
%:%989=454%:%
%:%990=454%:%
%:%991=455%:%
%:%992=455%:%
%:%993=456%:%
%:%994=456%:%
%:%995=457%:%
%:%996=457%:%
%:%997=458%:%
%:%998=458%:%
%:%999=459%:%
%:%1000=459%:%
%:%1001=460%:%
%:%1002=460%:%
%:%1003=461%:%
%:%1004=461%:%
%:%1005=462%:%
%:%1006=462%:%
%:%1007=463%:%
%:%1008=463%:%
%:%1009=464%:%
%:%1010=465%:%
%:%1011=465%:%
%:%1012=466%:%
%:%1013=466%:%
%:%1014=466%:%
%:%1015=467%:%
%:%1016=467%:%
%:%1017=468%:%
%:%1018=468%:%
%:%1019=468%:%
%:%1020=468%:%
%:%1021=468%:%
%:%1022=468%:%
%:%1023=469%:%
%:%1024=469%:%
%:%1025=470%:%
%:%1026=470%:%
%:%1027=470%:%
%:%1028=470%:%
%:%1029=470%:%
%:%1030=470%:%
%:%1031=471%:%
%:%1032=472%:%
%:%1033=472%:%
%:%1034=473%:%
%:%1035=474%:%
%:%1036=474%:%
%:%1037=474%:%
%:%1038=474%:%
%:%1039=475%:%
%:%1045=475%:%
%:%1048=476%:%
%:%1049=477%:%
%:%1050=478%:%
%:%1051=478%:%
%:%1052=479%:%
%:%1053=480%:%
%:%1054=481%:%
%:%1061=482%:%
%:%1062=482%:%
%:%1063=483%:%
%:%1064=483%:%
%:%1065=484%:%
%:%1066=485%:%
%:%1067=485%:%
%:%1068=485%:%
%:%1069=486%:%
%:%1070=487%:%
%:%1071=487%:%
%:%1072=488%:%
%:%1073=488%:%
%:%1074=489%:%
%:%1075=489%:%
%:%1076=490%:%
%:%1077=490%:%
%:%1078=491%:%
%:%1079=492%:%
%:%1080=492%:%
%:%1081=493%:%
%:%1082=493%:%
%:%1083=494%:%
%:%1084=494%:%
%:%1085=495%:%
%:%1086=495%:%
%:%1087=496%:%
%:%1088=496%:%
%:%1089=497%:%
%:%1090=497%:%
%:%1091=498%:%
%:%1092=498%:%
%:%1093=499%:%
%:%1094=500%:%
%:%1095=500%:%
%:%1096=501%:%
%:%1097=501%:%
%:%1098=502%:%
%:%1099=502%:%
%:%1100=503%:%
%:%1101=503%:%
%:%1102=504%:%
%:%1103=505%:%
%:%1104=505%:%
%:%1105=506%:%
%:%1106=506%:%
%:%1107=507%:%
%:%1108=507%:%
%:%1109=508%:%
%:%1110=508%:%
%:%1111=509%:%
%:%1112=509%:%
%:%1113=509%:%
%:%1114=510%:%
%:%1115=510%:%
%:%1116=511%:%
%:%1117=511%:%
%:%1118=512%:%
%:%1119=512%:%
%:%1120=513%:%
%:%1121=513%:%
%:%1122=513%:%
%:%1123=514%:%
%:%1124=514%:%
%:%1125=515%:%
%:%1126=515%:%
%:%1127=516%:%
%:%1128=516%:%
%:%1129=516%:%
%:%1130=517%:%
%:%1131=517%:%
%:%1132=518%:%
%:%1133=518%:%
%:%1134=519%:%
%:%1135=519%:%
%:%1136=519%:%
%:%1137=520%:%
%:%1138=520%:%
%:%1139=521%:%
%:%1140=521%:%
%:%1141=522%:%
%:%1142=522%:%
%:%1143=523%:%
%:%1144=523%:%
%:%1145=523%:%
%:%1146=524%:%
%:%1147=524%:%
%:%1148=525%:%
%:%1149=525%:%
%:%1150=526%:%
%:%1151=526%:%
%:%1152=527%:%
%:%1153=527%:%
%:%1154=528%:%
%:%1155=528%:%
%:%1156=529%:%
%:%1157=529%:%
%:%1158=530%:%
%:%1159=530%:%
%:%1160=531%:%
%:%1161=531%:%
%:%1162=532%:%
%:%1163=532%:%
%:%1164=533%:%
%:%1165=533%:%
%:%1166=533%:%
%:%1167=534%:%
%:%1168=534%:%
%:%1169=535%:%
%:%1175=535%:%
%:%1178=536%:%
%:%1179=537%:%
%:%1180=537%:%
%:%1181=538%:%
%:%1182=539%:%
%:%1183=540%:%
%:%1190=541%:%
%:%1191=541%:%
%:%1192=542%:%
%:%1193=542%:%
%:%1194=543%:%
%:%1195=544%:%
%:%1196=544%:%
%:%1197=544%:%
%:%1198=545%:%
%:%1199=546%:%
%:%1200=546%:%
%:%1201=547%:%
%:%1202=547%:%
%:%1203=548%:%
%:%1204=548%:%
%:%1205=549%:%
%:%1206=549%:%
%:%1207=550%:%
%:%1208=551%:%
%:%1209=551%:%
%:%1210=552%:%
%:%1211=552%:%
%:%1212=553%:%
%:%1213=553%:%
%:%1214=554%:%
%:%1215=554%:%
%:%1216=555%:%
%:%1217=555%:%
%:%1218=556%:%
%:%1219=556%:%
%:%1220=557%:%
%:%1221=557%:%
%:%1222=558%:%
%:%1223=559%:%
%:%1224=559%:%
%:%1225=560%:%
%:%1226=560%:%
%:%1227=561%:%
%:%1228=561%:%
%:%1229=562%:%
%:%1230=562%:%
%:%1231=563%:%
%:%1232=564%:%
%:%1233=564%:%
%:%1234=565%:%
%:%1235=565%:%
%:%1236=566%:%
%:%1237=566%:%
%:%1238=567%:%
%:%1239=567%:%
%:%1240=568%:%
%:%1241=568%:%
%:%1242=568%:%
%:%1243=569%:%
%:%1244=569%:%
%:%1245=570%:%
%:%1246=570%:%
%:%1247=571%:%
%:%1248=571%:%
%:%1249=572%:%
%:%1250=572%:%
%:%1251=572%:%
%:%1252=573%:%
%:%1253=573%:%
%:%1254=574%:%
%:%1255=574%:%
%:%1256=575%:%
%:%1257=575%:%
%:%1258=575%:%
%:%1259=576%:%
%:%1260=576%:%
%:%1261=577%:%
%:%1262=577%:%
%:%1263=578%:%
%:%1264=578%:%
%:%1265=578%:%
%:%1266=579%:%
%:%1267=579%:%
%:%1268=580%:%
%:%1269=580%:%
%:%1270=581%:%
%:%1271=581%:%
%:%1272=582%:%
%:%1273=582%:%
%:%1274=582%:%
%:%1275=583%:%
%:%1276=583%:%
%:%1277=584%:%
%:%1278=584%:%
%:%1279=585%:%
%:%1280=585%:%
%:%1281=586%:%
%:%1282=586%:%
%:%1283=587%:%
%:%1284=587%:%
%:%1285=588%:%
%:%1286=588%:%
%:%1287=589%:%
%:%1288=589%:%
%:%1289=590%:%
%:%1290=590%:%
%:%1291=591%:%
%:%1292=591%:%
%:%1293=592%:%
%:%1294=592%:%
%:%1295=592%:%
%:%1296=593%:%
%:%1297=593%:%
%:%1298=594%:%
%:%1304=594%:%
%:%1307=595%:%
%:%1308=596%:%
%:%1309=596%:%
%:%1310=597%:%
%:%1311=598%:%
%:%1312=599%:%
%:%1315=600%:%
%:%1319=600%:%
%:%1320=600%:%
%:%1321=601%:%
%:%1322=601%:%
%:%1323=602%:%
%:%1324=602%:%
%:%1325=603%:%
%:%1326=603%:%
%:%1327=603%:%
%:%1328=604%:%
%:%1329=604%:%
%:%1330=605%:%
%:%1331=605%:%
%:%1332=606%:%
%:%1333=606%:%
%:%1334=607%:%
%:%1335=607%:%
%:%1336=608%:%
%:%1337=608%:%
%:%1338=609%:%
%:%1339=609%:%
%:%1340=610%:%
%:%1341=610%:%
%:%1342=611%:%
%:%1343=611%:%
%:%1344=612%:%
%:%1345=612%:%
%:%1346=612%:%
%:%1347=613%:%
%:%1348=613%:%
%:%1349=614%:%
%:%1350=614%:%
%:%1351=615%:%
%:%1352=615%:%
%:%1353=616%:%
%:%1354=616%:%
%:%1355=617%:%
%:%1356=617%:%
%:%1357=618%:%
%:%1358=618%:%
%:%1359=619%:%
%:%1360=619%:%
%:%1361=620%:%
%:%1362=620%:%
%:%1363=621%:%
%:%1364=622%:%
%:%1365=622%:%
%:%1366=623%:%
%:%1367=623%:%
%:%1368=624%:%
%:%1369=624%:%
%:%1370=625%:%
%:%1371=625%:%
%:%1372=626%:%
%:%1373=626%:%
%:%1374=627%:%
%:%1375=627%:%
%:%1376=628%:%
%:%1377=628%:%
%:%1378=629%:%
%:%1379=629%:%
%:%1380=630%:%
%:%1381=630%:%
%:%1382=631%:%
%:%1383=631%:%
%:%1384=632%:%
%:%1385=632%:%
%:%1386=633%:%
%:%1387=634%:%
%:%1388=634%:%
%:%1389=635%:%
%:%1390=635%:%
%:%1391=636%:%
%:%1392=636%:%
%:%1393=637%:%
%:%1394=637%:%
%:%1395=638%:%
%:%1396=638%:%
%:%1397=639%:%
%:%1398=639%:%
%:%1399=640%:%
%:%1400=640%:%
%:%1401=641%:%
%:%1402=642%:%
%:%1403=642%:%
%:%1404=642%:%
%:%1405=643%:%
%:%1406=643%:%
%:%1407=644%:%
%:%1408=644%:%
%:%1409=645%:%
%:%1410=645%:%
%:%1411=646%:%
%:%1412=646%:%
%:%1413=647%:%
%:%1414=647%:%
%:%1415=647%:%
%:%1416=648%:%
%:%1417=648%:%
%:%1418=648%:%
%:%1419=649%:%
%:%1420=649%:%
%:%1421=649%:%
%:%1422=650%:%
%:%1423=650%:%
%:%1424=650%:%
%:%1425=651%:%
%:%1426=651%:%
%:%1427=651%:%
%:%1428=652%:%
%:%1429=652%:%
%:%1430=653%:%
%:%1431=653%:%
%:%1432=654%:%
%:%1433=654%:%
%:%1434=655%:%
%:%1435=655%:%
%:%1436=656%:%
%:%1437=656%:%
%:%1438=657%:%
%:%1439=657%:%
%:%1440=658%:%
%:%1441=658%:%
%:%1442=659%:%
%:%1443=659%:%
%:%1444=660%:%
%:%1445=660%:%
%:%1446=661%:%
%:%1447=661%:%
%:%1448=662%:%
%:%1449=662%:%
%:%1450=663%:%
%:%1451=663%:%
%:%1452=664%:%
%:%1453=664%:%
%:%1454=665%:%
%:%1455=665%:%
%:%1456=665%:%
%:%1457=666%:%
%:%1458=666%:%
%:%1459=667%:%
%:%1460=667%:%
%:%1461=668%:%
%:%1462=668%:%
%:%1463=669%:%
%:%1464=669%:%
%:%1465=669%:%
%:%1466=670%:%
%:%1467=671%:%
%:%1468=671%:%
%:%1469=672%:%
%:%1470=672%:%
%:%1471=673%:%
%:%1472=673%:%
%:%1473=674%:%
%:%1474=674%:%
%:%1475=675%:%
%:%1476=676%:%
%:%1477=676%:%
%:%1478=677%:%
%:%1479=677%:%
%:%1480=678%:%
%:%1481=678%:%
%:%1482=679%:%
%:%1483=679%:%
%:%1484=680%:%
%:%1485=680%:%
%:%1486=681%:%
%:%1487=681%:%
%:%1488=682%:%
%:%1489=682%:%
%:%1490=683%:%
%:%1491=683%:%
%:%1492=684%:%
%:%1493=684%:%
%:%1494=685%:%
%:%1495=685%:%
%:%1496=686%:%
%:%1497=686%:%
%:%1498=687%:%
%:%1499=688%:%
%:%1500=688%:%
%:%1501=689%:%
%:%1502=689%:%
%:%1503=690%:%
%:%1504=690%:%
%:%1505=691%:%
%:%1506=691%:%
%:%1507=692%:%
%:%1508=693%:%
%:%1509=693%:%
%:%1510=693%:%
%:%1511=694%:%
%:%1512=694%:%
%:%1513=695%:%
%:%1514=695%:%
%:%1515=696%:%
%:%1516=696%:%
%:%1517=696%:%
%:%1518=697%:%
%:%1519=697%:%
%:%1520=697%:%
%:%1521=698%:%
%:%1522=698%:%
%:%1523=698%:%
%:%1524=699%:%
%:%1525=699%:%
%:%1526=699%:%
%:%1527=700%:%
%:%1528=700%:%
%:%1529=701%:%
%:%1530=701%:%
%:%1531=701%:%
%:%1532=702%:%
%:%1533=702%:%
%:%1534=703%:%
%:%1535=703%:%
%:%1536=704%:%
%:%1537=704%:%
%:%1538=705%:%
%:%1539=705%:%
%:%1540=706%:%
%:%1541=706%:%
%:%1542=707%:%
%:%1543=707%:%
%:%1544=708%:%
%:%1545=708%:%
%:%1546=709%:%
%:%1547=709%:%
%:%1548=710%:%
%:%1549=710%:%
%:%1550=711%:%
%:%1551=711%:%
%:%1552=712%:%
%:%1553=712%:%
%:%1554=713%:%
%:%1555=713%:%
%:%1556=714%:%
%:%1557=714%:%
%:%1558=715%:%
%:%1559=715%:%
%:%1560=716%:%
%:%1561=716%:%
%:%1562=717%:%
%:%1563=717%:%
%:%1564=718%:%
%:%1565=718%:%
%:%1566=719%:%
%:%1567=719%:%
%:%1568=720%:%
%:%1569=720%:%
%:%1570=720%:%
%:%1571=721%:%
%:%1572=721%:%
%:%1573=722%:%
%:%1574=722%:%
%:%1575=723%:%
%:%1576=723%:%
%:%1577=724%:%
%:%1578=724%:%
%:%1579=724%:%
%:%1580=725%:%
%:%1581=726%:%
%:%1582=726%:%
%:%1583=727%:%
%:%1584=727%:%
%:%1585=728%:%
%:%1591=728%:%
%:%1594=729%:%
%:%1595=730%:%
%:%1596=730%:%
%:%1603=731%:%

%
\begin{isabellebody}%
\setisabellecontext{SymExt{\isacharunderscore}{\kern0pt}Separation}%
%
\isadelimtheory
%
\endisadelimtheory
%
\isatagtheory
\isacommand{theory}\isamarkupfalse%
\ SymExt{\isacharunderscore}{\kern0pt}Separation\isanewline
\ \ \isakeyword{imports}\ \isanewline
\ \ \ \ {\isachardoublequoteopen}Forcing{\isacharslash}{\kern0pt}Forcing{\isacharunderscore}{\kern0pt}Main{\isachardoublequoteclose}\ \isanewline
\ \ \ \ HS{\isacharunderscore}{\kern0pt}Forces\isanewline
\ \ \ \ Symmetry{\isacharunderscore}{\kern0pt}Lemma\isanewline
\isakeyword{begin}%
\endisatagtheory
{\isafoldtheory}%
%
\isadelimtheory
\ \isanewline
%
\endisadelimtheory
\isanewline
\isacommand{context}\isamarkupfalse%
\ M{\isacharunderscore}{\kern0pt}symmetric{\isacharunderscore}{\kern0pt}system{\isacharunderscore}{\kern0pt}G{\isacharunderscore}{\kern0pt}generic\isanewline
\isakeyword{begin}\isanewline
\isanewline
\isanewline
\isacommand{definition}\isamarkupfalse%
\ INTsym\ \isakeyword{where}\ {\isachardoublequoteopen}INTsym{\isacharparenleft}{\kern0pt}l{\isacharparenright}{\kern0pt}\ {\isasymequiv}\ {\isacharparenleft}{\kern0pt}{\isasymInter}H\ {\isasymin}\ set{\isacharunderscore}{\kern0pt}of{\isacharunderscore}{\kern0pt}list{\isacharparenleft}{\kern0pt}map{\isacharparenleft}{\kern0pt}sym{\isacharcomma}{\kern0pt}\ l{\isacharparenright}{\kern0pt}{\isacharparenright}{\kern0pt}{\isachardot}{\kern0pt}\ H{\isacharparenright}{\kern0pt}{\isachardoublequoteclose}\isanewline
\isanewline
\isacommand{lemma}\isamarkupfalse%
\ INT{\isacharunderscore}{\kern0pt}set{\isacharunderscore}{\kern0pt}of{\isacharunderscore}{\kern0pt}list{\isacharunderscore}{\kern0pt}of{\isacharunderscore}{\kern0pt}{\isasymF}{\isacharunderscore}{\kern0pt}lemma\ {\isacharcolon}{\kern0pt}\ \isanewline
\ \ \isakeyword{fixes}\ l\ \isanewline
\ \ \isakeyword{assumes}\ {\isachardoublequoteopen}l\ {\isasymin}\ list{\isacharparenleft}{\kern0pt}HS{\isacharparenright}{\kern0pt}{\isachardoublequoteclose}\ {\isachardoublequoteopen}l\ {\isasymnoteq}\ Nil{\isachardoublequoteclose}\isanewline
\ \ \isakeyword{shows}\ {\isachardoublequoteopen}INTsym{\isacharparenleft}{\kern0pt}l{\isacharparenright}{\kern0pt}\ {\isasymin}\ {\isasymF}\ {\isasymand}\ {\isacharparenleft}{\kern0pt}{\isasymforall}{\isasympi}\ {\isasymin}\ INTsym{\isacharparenleft}{\kern0pt}l{\isacharparenright}{\kern0pt}{\isachardot}{\kern0pt}\ map{\isacharparenleft}{\kern0pt}{\isasymlambda}x{\isachardot}{\kern0pt}\ Pn{\isacharunderscore}{\kern0pt}auto{\isacharparenleft}{\kern0pt}{\isasympi}{\isacharparenright}{\kern0pt}{\isacharbackquote}{\kern0pt}x{\isacharcomma}{\kern0pt}\ l{\isacharparenright}{\kern0pt}\ {\isacharequal}{\kern0pt}\ l{\isacharparenright}{\kern0pt}{\isachardoublequoteclose}\isanewline
%
\isadelimproof
\isanewline
\ \ %
\endisadelimproof
%
\isatagproof
\isacommand{unfolding}\isamarkupfalse%
\ INTsym{\isacharunderscore}{\kern0pt}def\isanewline
\ \ \isacommand{using}\isamarkupfalse%
\ assms\ \isanewline
\isacommand{proof}\isamarkupfalse%
\ {\isacharparenleft}{\kern0pt}induct{\isacharparenright}{\kern0pt}\isanewline
\ \ \isacommand{case}\isamarkupfalse%
\ Nil\isanewline
\ \ \isacommand{then}\isamarkupfalse%
\ \isacommand{show}\isamarkupfalse%
\ {\isacharquery}{\kern0pt}case\ \isacommand{by}\isamarkupfalse%
\ auto\ \isanewline
\isacommand{next}\isamarkupfalse%
\ \isanewline
\ \ \isacommand{case}\isamarkupfalse%
\ {\isacharparenleft}{\kern0pt}Cons\ a\ l{\isacharparenright}{\kern0pt}\isanewline
\ \ \isacommand{then}\isamarkupfalse%
\ \isacommand{show}\isamarkupfalse%
\ {\isacharquery}{\kern0pt}case\isanewline
\ \ \isacommand{proof}\isamarkupfalse%
\ {\isacharminus}{\kern0pt}\ \isanewline
\ \ \ \ \isacommand{assume}\isamarkupfalse%
\ assms\ {\isacharcolon}{\kern0pt}\ {\isachardoublequoteopen}a\ {\isasymin}\ HS{\isachardoublequoteclose}\ {\isachardoublequoteopen}l\ {\isasymin}\ list{\isacharparenleft}{\kern0pt}HS{\isacharparenright}{\kern0pt}{\isachardoublequoteclose}\isanewline
\ \ \ \ \ \ {\isachardoublequoteopen}{\isacharparenleft}{\kern0pt}l\ {\isasymnoteq}\ {\isacharbrackleft}{\kern0pt}{\isacharbrackright}{\kern0pt}\ {\isasymLongrightarrow}\ {\isacharparenleft}{\kern0pt}{\isasymInter}H{\isasymin}set{\isacharunderscore}{\kern0pt}of{\isacharunderscore}{\kern0pt}list{\isacharparenleft}{\kern0pt}map{\isacharparenleft}{\kern0pt}sym{\isacharcomma}{\kern0pt}\ l{\isacharparenright}{\kern0pt}{\isacharparenright}{\kern0pt}{\isachardot}{\kern0pt}\ H{\isacharparenright}{\kern0pt}\ {\isasymin}\ {\isasymF}\ {\isasymand}\ {\isacharparenleft}{\kern0pt}{\isasymforall}{\isasympi}{\isasymin}{\isasymInter}H{\isasymin}set{\isacharunderscore}{\kern0pt}of{\isacharunderscore}{\kern0pt}list{\isacharparenleft}{\kern0pt}map{\isacharparenleft}{\kern0pt}sym{\isacharcomma}{\kern0pt}\ l{\isacharparenright}{\kern0pt}{\isacharparenright}{\kern0pt}{\isachardot}{\kern0pt}\ H{\isachardot}{\kern0pt}\ map{\isacharparenleft}{\kern0pt}{\isasymlambda}a{\isachardot}{\kern0pt}\ Pn{\isacharunderscore}{\kern0pt}auto{\isacharparenleft}{\kern0pt}{\isasympi}{\isacharparenright}{\kern0pt}\ {\isacharbackquote}{\kern0pt}\ a{\isacharcomma}{\kern0pt}\ l{\isacharparenright}{\kern0pt}\ {\isacharequal}{\kern0pt}\ l{\isacharparenright}{\kern0pt}{\isacharparenright}{\kern0pt}{\isachardoublequoteclose}\ \isanewline
\ \ \ \ \isacommand{show}\isamarkupfalse%
\ {\isacharquery}{\kern0pt}thesis\ \isanewline
\ \ \ \ \ \ \isacommand{using}\isamarkupfalse%
\ {\isacartoucheopen}l\ {\isasymin}\ list{\isacharparenleft}{\kern0pt}HS{\isacharparenright}{\kern0pt}{\isacartoucheclose}\ \isanewline
\ \ \ \ \isacommand{proof}\isamarkupfalse%
{\isacharparenleft}{\kern0pt}cases{\isacharparenright}{\kern0pt}\isanewline
\ \ \ \ \ \ \isacommand{case}\isamarkupfalse%
\ Nil\isanewline
\ \ \ \ \ \ \isacommand{then}\isamarkupfalse%
\ \isacommand{have}\isamarkupfalse%
\ lnil\ {\isacharcolon}{\kern0pt}\ {\isachardoublequoteopen}l\ {\isacharequal}{\kern0pt}\ Nil{\isachardoublequoteclose}\ \isacommand{by}\isamarkupfalse%
\ simp\isanewline
\ \ \ \ \ \ \isacommand{then}\isamarkupfalse%
\ \isacommand{have}\isamarkupfalse%
\ {\isachardoublequoteopen}{\isacharparenleft}{\kern0pt}{\isasymInter}H{\isasymin}set{\isacharunderscore}{\kern0pt}of{\isacharunderscore}{\kern0pt}list{\isacharparenleft}{\kern0pt}Cons{\isacharparenleft}{\kern0pt}a{\isacharcomma}{\kern0pt}\ l{\isacharparenright}{\kern0pt}{\isacharparenright}{\kern0pt}{\isachardot}{\kern0pt}\ H{\isacharparenright}{\kern0pt}\ {\isacharequal}{\kern0pt}\ a{\isachardoublequoteclose}\ \isacommand{using}\isamarkupfalse%
\ assms\ \isacommand{by}\isamarkupfalse%
\ auto\ \isanewline
\ \ \ \ \ \ \isacommand{then}\isamarkupfalse%
\ \isacommand{show}\isamarkupfalse%
\ {\isacharquery}{\kern0pt}thesis\ \isacommand{using}\isamarkupfalse%
\ lnil\ assms\ HS{\isacharunderscore}{\kern0pt}iff\ symmetric{\isacharunderscore}{\kern0pt}def\ sym{\isacharunderscore}{\kern0pt}def\ \isacommand{by}\isamarkupfalse%
\ auto\isanewline
\ \ \ \ \isacommand{next}\isamarkupfalse%
\isanewline
\ \ \ \ \ \ \isacommand{case}\isamarkupfalse%
\ {\isacharparenleft}{\kern0pt}Cons\ b\ l{\isacharprime}{\kern0pt}{\isacharparenright}{\kern0pt}\isanewline
\isanewline
\ \ \ \ \ \ \isacommand{then}\isamarkupfalse%
\ \isacommand{have}\isamarkupfalse%
\ bl{\isacharprime}{\kern0pt}H\ {\isacharcolon}{\kern0pt}\ {\isachardoublequoteopen}l\ {\isacharequal}{\kern0pt}\ Cons{\isacharparenleft}{\kern0pt}b{\isacharcomma}{\kern0pt}\ l{\isacharprime}{\kern0pt}{\isacharparenright}{\kern0pt}{\isachardoublequoteclose}\ {\isachardoublequoteopen}b\ {\isasymin}\ HS{\isachardoublequoteclose}\ {\isachardoublequoteopen}l{\isacharprime}{\kern0pt}\ {\isasymin}\ list{\isacharparenleft}{\kern0pt}HS{\isacharparenright}{\kern0pt}{\isachardoublequoteclose}\ \isacommand{by}\isamarkupfalse%
\ auto\isanewline
\ \ \ \ \ \ \isacommand{then}\isamarkupfalse%
\ \isacommand{have}\isamarkupfalse%
\ neq{\isadigit{0}}\ {\isacharcolon}{\kern0pt}\ {\isachardoublequoteopen}set{\isacharunderscore}{\kern0pt}of{\isacharunderscore}{\kern0pt}list{\isacharparenleft}{\kern0pt}l{\isacharparenright}{\kern0pt}\ {\isasymnoteq}\ {\isadigit{0}}{\isachardoublequoteclose}\ \isacommand{by}\isamarkupfalse%
\ auto\isanewline
\isanewline
\ \ \ \ \ \ \isacommand{have}\isamarkupfalse%
\ {\isachardoublequoteopen}{\isacharparenleft}{\kern0pt}{\isasymInter}H{\isasymin}set{\isacharunderscore}{\kern0pt}of{\isacharunderscore}{\kern0pt}list{\isacharparenleft}{\kern0pt}map{\isacharparenleft}{\kern0pt}sym{\isacharcomma}{\kern0pt}\ Cons{\isacharparenleft}{\kern0pt}a{\isacharcomma}{\kern0pt}\ l{\isacharparenright}{\kern0pt}{\isacharparenright}{\kern0pt}{\isacharparenright}{\kern0pt}{\isachardot}{\kern0pt}\ H{\isacharparenright}{\kern0pt}\ {\isacharequal}{\kern0pt}\ {\isacharparenleft}{\kern0pt}{\isasymInter}H{\isasymin}{\isacharbraceleft}{\kern0pt}sym{\isacharparenleft}{\kern0pt}a{\isacharparenright}{\kern0pt}{\isacharbraceright}{\kern0pt}\ {\isasymunion}\ set{\isacharunderscore}{\kern0pt}of{\isacharunderscore}{\kern0pt}list{\isacharparenleft}{\kern0pt}map{\isacharparenleft}{\kern0pt}sym{\isacharcomma}{\kern0pt}\ l{\isacharparenright}{\kern0pt}{\isacharparenright}{\kern0pt}{\isachardot}{\kern0pt}\ H{\isacharparenright}{\kern0pt}{\isachardoublequoteclose}\ \isacommand{by}\isamarkupfalse%
\ auto\ \isanewline
\ \ \ \ \ \ \isacommand{also}\isamarkupfalse%
\ \isacommand{have}\isamarkupfalse%
\ {\isachardoublequoteopen}{\isachardot}{\kern0pt}{\isachardot}{\kern0pt}{\isachardot}{\kern0pt}\ {\isacharequal}{\kern0pt}\ {\isacharparenleft}{\kern0pt}{\isasymInter}H{\isasymin}{\isacharbraceleft}{\kern0pt}sym{\isacharparenleft}{\kern0pt}a{\isacharparenright}{\kern0pt}{\isacharbraceright}{\kern0pt}{\isachardot}{\kern0pt}\ H{\isacharparenright}{\kern0pt}\ {\isasyminter}\ {\isacharparenleft}{\kern0pt}{\isasymInter}H{\isasymin}set{\isacharunderscore}{\kern0pt}of{\isacharunderscore}{\kern0pt}list{\isacharparenleft}{\kern0pt}map{\isacharparenleft}{\kern0pt}sym{\isacharcomma}{\kern0pt}\ l{\isacharparenright}{\kern0pt}{\isacharparenright}{\kern0pt}{\isachardot}{\kern0pt}\ H{\isacharparenright}{\kern0pt}{\isachardoublequoteclose}\ \isanewline
\ \ \ \ \ \ \ \ \isacommand{apply}\isamarkupfalse%
\ {\isacharparenleft}{\kern0pt}subst\ INT{\isacharunderscore}{\kern0pt}Un{\isacharparenright}{\kern0pt}\ \isanewline
\ \ \ \ \ \ \ \ \isacommand{using}\isamarkupfalse%
\ bl{\isacharprime}{\kern0pt}H\ \isanewline
\ \ \ \ \ \ \ \ \isacommand{by}\isamarkupfalse%
\ auto\isanewline
\ \ \ \ \ \ \isacommand{finally}\isamarkupfalse%
\ \isacommand{have}\isamarkupfalse%
\ eq{\isacharcolon}{\kern0pt}\ {\isachardoublequoteopen}{\isacharparenleft}{\kern0pt}{\isasymInter}H{\isasymin}set{\isacharunderscore}{\kern0pt}of{\isacharunderscore}{\kern0pt}list{\isacharparenleft}{\kern0pt}map{\isacharparenleft}{\kern0pt}sym{\isacharcomma}{\kern0pt}\ Cons{\isacharparenleft}{\kern0pt}a{\isacharcomma}{\kern0pt}\ l{\isacharparenright}{\kern0pt}{\isacharparenright}{\kern0pt}{\isacharparenright}{\kern0pt}{\isachardot}{\kern0pt}\ H{\isacharparenright}{\kern0pt}\ {\isacharequal}{\kern0pt}\ sym{\isacharparenleft}{\kern0pt}a{\isacharparenright}{\kern0pt}\ {\isasyminter}\ {\isacharparenleft}{\kern0pt}{\isasymInter}H{\isasymin}set{\isacharunderscore}{\kern0pt}of{\isacharunderscore}{\kern0pt}list{\isacharparenleft}{\kern0pt}map{\isacharparenleft}{\kern0pt}sym{\isacharcomma}{\kern0pt}\ l{\isacharparenright}{\kern0pt}{\isacharparenright}{\kern0pt}{\isachardot}{\kern0pt}\ H{\isacharparenright}{\kern0pt}{\isachardoublequoteclose}\ \isacommand{by}\isamarkupfalse%
\ auto\isanewline
\isanewline
\ \ \ \ \ \ \isacommand{have}\isamarkupfalse%
\ {\isachardoublequoteopen}\ {\isacharparenleft}{\kern0pt}{\isasymInter}H{\isasymin}set{\isacharunderscore}{\kern0pt}of{\isacharunderscore}{\kern0pt}list{\isacharparenleft}{\kern0pt}map{\isacharparenleft}{\kern0pt}sym{\isacharcomma}{\kern0pt}\ l{\isacharparenright}{\kern0pt}{\isacharparenright}{\kern0pt}{\isachardot}{\kern0pt}\ H{\isacharparenright}{\kern0pt}\ {\isasymin}\ {\isasymF}{\isachardoublequoteclose}\ \isacommand{using}\isamarkupfalse%
\ assms\ bl{\isacharprime}{\kern0pt}H\ \isacommand{by}\isamarkupfalse%
\ auto\isanewline
\ \ \ \ \ \ \isacommand{then}\isamarkupfalse%
\ \isacommand{have}\isamarkupfalse%
\ in{\isasymF}\ {\isacharcolon}{\kern0pt}\ {\isachardoublequoteopen}{\isacharparenleft}{\kern0pt}{\isasymInter}H{\isasymin}set{\isacharunderscore}{\kern0pt}of{\isacharunderscore}{\kern0pt}list{\isacharparenleft}{\kern0pt}map{\isacharparenleft}{\kern0pt}sym{\isacharcomma}{\kern0pt}\ Cons{\isacharparenleft}{\kern0pt}a{\isacharcomma}{\kern0pt}\ l{\isacharparenright}{\kern0pt}{\isacharparenright}{\kern0pt}{\isacharparenright}{\kern0pt}{\isachardot}{\kern0pt}\ H{\isacharparenright}{\kern0pt}\ {\isasymin}\ {\isasymF}{\isachardoublequoteclose}\ \isanewline
\ \ \ \ \ \ \ \ \isacommand{apply}\isamarkupfalse%
{\isacharparenleft}{\kern0pt}subst\ eq{\isacharparenright}{\kern0pt}\isanewline
\ \ \ \ \ \ \ \ \isacommand{using}\isamarkupfalse%
\ {\isasymF}{\isacharunderscore}{\kern0pt}closed{\isacharunderscore}{\kern0pt}under{\isacharunderscore}{\kern0pt}intersection\ assms\ HS{\isacharunderscore}{\kern0pt}iff\ symmetric{\isacharunderscore}{\kern0pt}def\ bl{\isacharprime}{\kern0pt}H\ \isanewline
\ \ \ \ \ \ \ \ \isacommand{by}\isamarkupfalse%
\ auto\isanewline
\isanewline
\ \ \ \ \ \ \isacommand{have}\isamarkupfalse%
\ {\isachardoublequoteopen}{\isasymforall}{\isasympi}{\isasymin}{\isasymInter}H{\isasymin}set{\isacharunderscore}{\kern0pt}of{\isacharunderscore}{\kern0pt}list{\isacharparenleft}{\kern0pt}map{\isacharparenleft}{\kern0pt}sym{\isacharcomma}{\kern0pt}\ Cons{\isacharparenleft}{\kern0pt}a{\isacharcomma}{\kern0pt}\ l{\isacharparenright}{\kern0pt}{\isacharparenright}{\kern0pt}{\isacharparenright}{\kern0pt}{\isachardot}{\kern0pt}\ H{\isachardot}{\kern0pt}\ map{\isacharparenleft}{\kern0pt}{\isasymlambda}a{\isachardot}{\kern0pt}\ Pn{\isacharunderscore}{\kern0pt}auto{\isacharparenleft}{\kern0pt}{\isasympi}{\isacharparenright}{\kern0pt}\ {\isacharbackquote}{\kern0pt}\ a{\isacharcomma}{\kern0pt}\ Cons{\isacharparenleft}{\kern0pt}a{\isacharcomma}{\kern0pt}\ l{\isacharparenright}{\kern0pt}{\isacharparenright}{\kern0pt}\ {\isacharequal}{\kern0pt}\ Cons{\isacharparenleft}{\kern0pt}a{\isacharcomma}{\kern0pt}\ l{\isacharparenright}{\kern0pt}{\isachardoublequoteclose}\isanewline
\ \ \ \ \ \ \isacommand{proof}\isamarkupfalse%
\ {\isacharparenleft}{\kern0pt}rule\ ballI{\isacharparenright}{\kern0pt}\isanewline
\ \ \ \ \ \ \ \ \isacommand{fix}\isamarkupfalse%
\ {\isasympi}\ \isacommand{assume}\isamarkupfalse%
\ piH\ {\isacharcolon}{\kern0pt}\ {\isachardoublequoteopen}{\isasympi}{\isasymin}{\isacharparenleft}{\kern0pt}{\isasymInter}H{\isasymin}set{\isacharunderscore}{\kern0pt}of{\isacharunderscore}{\kern0pt}list{\isacharparenleft}{\kern0pt}map{\isacharparenleft}{\kern0pt}sym{\isacharcomma}{\kern0pt}\ Cons{\isacharparenleft}{\kern0pt}a{\isacharcomma}{\kern0pt}\ l{\isacharparenright}{\kern0pt}{\isacharparenright}{\kern0pt}{\isacharparenright}{\kern0pt}{\isachardot}{\kern0pt}\ H{\isacharparenright}{\kern0pt}{\isachardoublequoteclose}\isanewline
\ \ \ \ \ \ \ \ \isacommand{then}\isamarkupfalse%
\ \isacommand{have}\isamarkupfalse%
\ fixa\ {\isacharcolon}{\kern0pt}\ {\isachardoublequoteopen}Pn{\isacharunderscore}{\kern0pt}auto{\isacharparenleft}{\kern0pt}{\isasympi}{\isacharparenright}{\kern0pt}{\isacharbackquote}{\kern0pt}a\ {\isacharequal}{\kern0pt}\ a{\isachardoublequoteclose}\ \isacommand{using}\isamarkupfalse%
\ eq\ sym{\isacharunderscore}{\kern0pt}def\ \isacommand{by}\isamarkupfalse%
\ auto\ \isanewline
\ \ \ \ \ \ \ \ \isacommand{have}\isamarkupfalse%
\ {\isachardoublequoteopen}map{\isacharparenleft}{\kern0pt}{\isasymlambda}a{\isachardot}{\kern0pt}\ Pn{\isacharunderscore}{\kern0pt}auto{\isacharparenleft}{\kern0pt}{\isasympi}{\isacharparenright}{\kern0pt}{\isacharbackquote}{\kern0pt}a{\isacharcomma}{\kern0pt}\ l{\isacharparenright}{\kern0pt}\ {\isacharequal}{\kern0pt}\ l{\isachardoublequoteclose}\ \isacommand{using}\isamarkupfalse%
\ assms\ bl{\isacharprime}{\kern0pt}H\ piH\ eq\ \isacommand{by}\isamarkupfalse%
\ auto\isanewline
\ \ \ \ \ \ \ \ \isacommand{then}\isamarkupfalse%
\ \isacommand{show}\isamarkupfalse%
\ {\isachardoublequoteopen}map{\isacharparenleft}{\kern0pt}{\isasymlambda}a{\isachardot}{\kern0pt}\ Pn{\isacharunderscore}{\kern0pt}auto{\isacharparenleft}{\kern0pt}{\isasympi}{\isacharparenright}{\kern0pt}\ {\isacharbackquote}{\kern0pt}\ a{\isacharcomma}{\kern0pt}\ Cons{\isacharparenleft}{\kern0pt}a{\isacharcomma}{\kern0pt}\ l{\isacharparenright}{\kern0pt}{\isacharparenright}{\kern0pt}\ {\isacharequal}{\kern0pt}\ Cons{\isacharparenleft}{\kern0pt}a{\isacharcomma}{\kern0pt}\ l{\isacharparenright}{\kern0pt}{\isachardoublequoteclose}\ \isacommand{using}\isamarkupfalse%
\ fixa\ \isacommand{by}\isamarkupfalse%
\ auto\isanewline
\ \ \ \ \ \ \isacommand{qed}\isamarkupfalse%
\isanewline
\ \ \ \ \ \ \isacommand{then}\isamarkupfalse%
\ \isacommand{show}\isamarkupfalse%
\ {\isacharquery}{\kern0pt}thesis\ \isacommand{using}\isamarkupfalse%
\ in{\isasymF}\ \isacommand{by}\isamarkupfalse%
\ auto\isanewline
\ \ \ \ \isacommand{qed}\isamarkupfalse%
\isanewline
\ \ \isacommand{qed}\isamarkupfalse%
\isanewline
\isacommand{qed}\isamarkupfalse%
%
\endisatagproof
{\isafoldproof}%
%
\isadelimproof
\isanewline
%
\endisadelimproof
\isanewline
\isacommand{definition}\isamarkupfalse%
\ ren{\isacharunderscore}{\kern0pt}sep{\isacharunderscore}{\kern0pt}forces\ \isakeyword{where}\isanewline
\ \ {\isachardoublequoteopen}ren{\isacharunderscore}{\kern0pt}sep{\isacharunderscore}{\kern0pt}forces{\isacharparenleft}{\kern0pt}{\isasymphi}{\isacharparenright}{\kern0pt}\ {\isasymequiv}\ Exists{\isacharparenleft}{\kern0pt}Exists{\isacharparenleft}{\kern0pt}Exists{\isacharparenleft}{\kern0pt}Exists{\isacharparenleft}{\kern0pt}Exists{\isacharparenleft}{\kern0pt}\isanewline
\ \ \ \ \ \ And{\isacharparenleft}{\kern0pt}Equal{\isacharparenleft}{\kern0pt}{\isadigit{0}}{\isacharcomma}{\kern0pt}\ {\isadigit{6}}{\isacharparenright}{\kern0pt}{\isacharcomma}{\kern0pt}\ And{\isacharparenleft}{\kern0pt}Equal{\isacharparenleft}{\kern0pt}{\isadigit{1}}{\isacharcomma}{\kern0pt}\ {\isadigit{8}}{\isacharparenright}{\kern0pt}{\isacharcomma}{\kern0pt}\ And{\isacharparenleft}{\kern0pt}Equal{\isacharparenleft}{\kern0pt}{\isadigit{2}}{\isacharcomma}{\kern0pt}\ {\isadigit{9}}{\isacharparenright}{\kern0pt}{\isacharcomma}{\kern0pt}\ And{\isacharparenleft}{\kern0pt}Equal{\isacharparenleft}{\kern0pt}{\isadigit{3}}{\isacharcomma}{\kern0pt}\ {\isadigit{1}}{\isadigit{0}}{\isacharparenright}{\kern0pt}{\isacharcomma}{\kern0pt}\ And{\isacharparenleft}{\kern0pt}Equal{\isacharparenleft}{\kern0pt}{\isadigit{4}}{\isacharcomma}{\kern0pt}\ {\isadigit{1}}{\isadigit{1}}{\isacharparenright}{\kern0pt}{\isacharcomma}{\kern0pt}\ \isanewline
\ \ \ \ \ \ iterates{\isacharparenleft}{\kern0pt}{\isasymlambda}p{\isachardot}{\kern0pt}\ incr{\isacharunderscore}{\kern0pt}bv{\isacharparenleft}{\kern0pt}p{\isacharparenright}{\kern0pt}{\isacharbackquote}{\kern0pt}{\isadigit{6}}\ {\isacharcomma}{\kern0pt}\ {\isadigit{6}}{\isacharcomma}{\kern0pt}\ {\isasymphi}{\isacharparenright}{\kern0pt}{\isacharparenright}{\kern0pt}{\isacharparenright}{\kern0pt}{\isacharparenright}{\kern0pt}{\isacharparenright}{\kern0pt}{\isacharparenright}{\kern0pt}{\isacharparenright}{\kern0pt}{\isacharparenright}{\kern0pt}{\isacharparenright}{\kern0pt}{\isacharparenright}{\kern0pt}{\isacharparenright}{\kern0pt}{\isachardoublequoteclose}\ \isanewline
\isanewline
\isacommand{lemma}\isamarkupfalse%
\ ren{\isacharunderscore}{\kern0pt}sep{\isacharunderscore}{\kern0pt}forces{\isacharunderscore}{\kern0pt}type\ {\isacharcolon}{\kern0pt}\ {\isachardoublequoteopen}{\isasymphi}\ {\isasymin}\ formula\ {\isasymLongrightarrow}\ ren{\isacharunderscore}{\kern0pt}sep{\isacharunderscore}{\kern0pt}forces{\isacharparenleft}{\kern0pt}{\isasymphi}{\isacharparenright}{\kern0pt}\ {\isasymin}\ formula{\isachardoublequoteclose}\ \isanewline
%
\isadelimproof
\ \ %
\endisadelimproof
%
\isatagproof
\isacommand{unfolding}\isamarkupfalse%
\ ren{\isacharunderscore}{\kern0pt}sep{\isacharunderscore}{\kern0pt}forces{\isacharunderscore}{\kern0pt}def\ \isanewline
\ \ \isacommand{apply}\isamarkupfalse%
{\isacharparenleft}{\kern0pt}subgoal{\isacharunderscore}{\kern0pt}tac\ {\isachardoublequoteopen}{\isacharparenleft}{\kern0pt}{\isasymlambda}p{\isachardot}{\kern0pt}\ incr{\isacharunderscore}{\kern0pt}bv{\isacharparenleft}{\kern0pt}p{\isacharparenright}{\kern0pt}\ {\isacharbackquote}{\kern0pt}\ {\isadigit{6}}{\isacharparenright}{\kern0pt}{\isacharcircum}{\kern0pt}{\isadigit{6}}\ {\isacharparenleft}{\kern0pt}{\isasymphi}{\isacharparenright}{\kern0pt}\ {\isasymin}\ formula{\isachardoublequoteclose}{\isacharparenright}{\kern0pt}\isanewline
\ \ \ \isacommand{apply}\isamarkupfalse%
\ force\ \isanewline
\ \ \isacommand{apply}\isamarkupfalse%
{\isacharparenleft}{\kern0pt}rule\ iterates{\isacharunderscore}{\kern0pt}type{\isacharcomma}{\kern0pt}\ simp{\isacharcomma}{\kern0pt}\ simp{\isacharparenright}{\kern0pt}\isanewline
\ \ \isacommand{apply}\isamarkupfalse%
{\isacharparenleft}{\kern0pt}rule\ function{\isacharunderscore}{\kern0pt}value{\isacharunderscore}{\kern0pt}in{\isacharcomma}{\kern0pt}\ rule\ incr{\isacharunderscore}{\kern0pt}bv{\isacharunderscore}{\kern0pt}type{\isacharparenright}{\kern0pt}\isanewline
\ \ \isacommand{by}\isamarkupfalse%
\ auto%
\endisatagproof
{\isafoldproof}%
%
\isadelimproof
\isanewline
%
\endisadelimproof
\isanewline
\isacommand{lemma}\isamarkupfalse%
\ arity{\isacharunderscore}{\kern0pt}ren{\isacharunderscore}{\kern0pt}sep{\isacharunderscore}{\kern0pt}forces\ {\isacharcolon}{\kern0pt}\ \isanewline
\ \ \isakeyword{fixes}\ {\isasymphi}\ \isanewline
\ \ \isakeyword{assumes}\ {\isachardoublequoteopen}{\isasymphi}\ {\isasymin}\ formula{\isachardoublequoteclose}\ \isanewline
\ \ \isakeyword{shows}\ {\isachardoublequoteopen}arity{\isacharparenleft}{\kern0pt}ren{\isacharunderscore}{\kern0pt}sep{\isacharunderscore}{\kern0pt}forces{\isacharparenleft}{\kern0pt}{\isasymphi}{\isacharparenright}{\kern0pt}{\isacharparenright}{\kern0pt}\ {\isasymle}\ succ{\isacharparenleft}{\kern0pt}arity{\isacharparenleft}{\kern0pt}{\isasymphi}{\isacharparenright}{\kern0pt}{\isacharparenright}{\kern0pt}\ {\isasymunion}\ {\isadigit{7}}{\isachardoublequoteclose}\ \isanewline
%
\isadelimproof
\isanewline
\ \ %
\endisadelimproof
%
\isatagproof
\isacommand{unfolding}\isamarkupfalse%
\ ren{\isacharunderscore}{\kern0pt}sep{\isacharunderscore}{\kern0pt}forces{\isacharunderscore}{\kern0pt}def\ \isanewline
\ \ \isacommand{apply}\isamarkupfalse%
{\isacharparenleft}{\kern0pt}subgoal{\isacharunderscore}{\kern0pt}tac\ {\isachardoublequoteopen}{\isacharparenleft}{\kern0pt}{\isasymlambda}p{\isachardot}{\kern0pt}\ incr{\isacharunderscore}{\kern0pt}bv{\isacharparenleft}{\kern0pt}p{\isacharparenright}{\kern0pt}\ {\isacharbackquote}{\kern0pt}\ {\isadigit{6}}{\isacharparenright}{\kern0pt}{\isacharcircum}{\kern0pt}{\isadigit{6}}\ {\isacharparenleft}{\kern0pt}{\isasymphi}{\isacharparenright}{\kern0pt}\ {\isasymin}\ formula{\isachardoublequoteclose}{\isacharparenright}{\kern0pt}\isanewline
\ \ \isacommand{using}\isamarkupfalse%
\ assms\isanewline
\ \ \ \isacommand{apply}\isamarkupfalse%
\ simp\isanewline
\ \ \ \isacommand{apply}\isamarkupfalse%
{\isacharparenleft}{\kern0pt}rule\ pred{\isacharunderscore}{\kern0pt}le{\isacharcomma}{\kern0pt}\ simp{\isacharcomma}{\kern0pt}\ simp{\isacharparenright}{\kern0pt}{\isacharplus}{\kern0pt}\isanewline
\ \ \ \isacommand{apply}\isamarkupfalse%
{\isacharparenleft}{\kern0pt}rule\ Un{\isacharunderscore}{\kern0pt}least{\isacharunderscore}{\kern0pt}lt{\isacharcomma}{\kern0pt}\ rule\ Un{\isacharunderscore}{\kern0pt}least{\isacharunderscore}{\kern0pt}lt{\isacharcomma}{\kern0pt}\ simp{\isacharcomma}{\kern0pt}\ simp{\isacharcomma}{\kern0pt}\ rule\ union{\isacharunderscore}{\kern0pt}lt{\isadigit{2}}{\isacharcomma}{\kern0pt}\ simp{\isacharcomma}{\kern0pt}\ simp{\isacharcomma}{\kern0pt}\ simp{\isacharcomma}{\kern0pt}\ simp{\isacharparenright}{\kern0pt}{\isacharplus}{\kern0pt}\isanewline
\ \ \ \isacommand{apply}\isamarkupfalse%
{\isacharparenleft}{\kern0pt}rule{\isacharunderscore}{\kern0pt}tac\ j{\isacharequal}{\kern0pt}{\isachardoublequoteopen}arity{\isacharparenleft}{\kern0pt}{\isacharparenleft}{\kern0pt}{\isasymlambda}p{\isachardot}{\kern0pt}\ incr{\isacharunderscore}{\kern0pt}bv{\isacharparenleft}{\kern0pt}p{\isacharparenright}{\kern0pt}\ {\isacharbackquote}{\kern0pt}\ {\isadigit{6}}{\isacharparenright}{\kern0pt}{\isacharcircum}{\kern0pt}{\isadigit{6}}\ {\isacharparenleft}{\kern0pt}{\isasymphi}{\isacharparenright}{\kern0pt}{\isacharparenright}{\kern0pt}{\isachardoublequoteclose}\ \isakeyword{in}\ le{\isacharunderscore}{\kern0pt}trans{\isacharcomma}{\kern0pt}\ force{\isacharparenright}{\kern0pt}\isanewline
\ \ \ \isacommand{apply}\isamarkupfalse%
{\isacharparenleft}{\kern0pt}rule\ le{\isacharunderscore}{\kern0pt}trans{\isacharcomma}{\kern0pt}\ rule\ arity{\isacharunderscore}{\kern0pt}incr{\isacharunderscore}{\kern0pt}bv{\isacharunderscore}{\kern0pt}le{\isacharparenright}{\kern0pt}\isanewline
\ \ \isacommand{using}\isamarkupfalse%
\ assms\isanewline
\ \ \ \ \ \ \isacommand{apply}\isamarkupfalse%
\ auto{\isacharbrackleft}{\kern0pt}{\isadigit{4}}{\isacharbrackright}{\kern0pt}\isanewline
\ \ \ \isacommand{apply}\isamarkupfalse%
{\isacharparenleft}{\kern0pt}rule\ union{\isacharunderscore}{\kern0pt}lt{\isadigit{1}}{\isacharcomma}{\kern0pt}\ simp{\isacharcomma}{\kern0pt}\ simp{\isacharcomma}{\kern0pt}\ simp{\isacharcomma}{\kern0pt}\ simp{\isacharparenright}{\kern0pt}\isanewline
\ \ \isacommand{apply}\isamarkupfalse%
{\isacharparenleft}{\kern0pt}rule\ iterates{\isacharunderscore}{\kern0pt}type{\isacharcomma}{\kern0pt}\ simp{\isacharcomma}{\kern0pt}\ simp\ add{\isacharcolon}{\kern0pt}assms{\isacharparenright}{\kern0pt}\isanewline
\ \ \isacommand{apply}\isamarkupfalse%
{\isacharparenleft}{\kern0pt}rule\ function{\isacharunderscore}{\kern0pt}value{\isacharunderscore}{\kern0pt}in{\isacharparenright}{\kern0pt}\isanewline
\ \ \isacommand{using}\isamarkupfalse%
\ incr{\isacharunderscore}{\kern0pt}bv{\isacharunderscore}{\kern0pt}type\ \isanewline
\ \ \isacommand{by}\isamarkupfalse%
\ auto%
\endisatagproof
{\isafoldproof}%
%
\isadelimproof
\isanewline
%
\endisadelimproof
\isanewline
\isacommand{lemma}\isamarkupfalse%
\ sats{\isacharunderscore}{\kern0pt}ren{\isacharunderscore}{\kern0pt}sep{\isacharunderscore}{\kern0pt}forces{\isacharunderscore}{\kern0pt}fm{\isacharunderscore}{\kern0pt}iff\ \ {\isacharcolon}{\kern0pt}\ \isanewline
\ \ \isakeyword{fixes}\ {\isasymphi}\ a\ b\ c\ d\ e\ f\ g\ env\ \isanewline
\ \ \isakeyword{assumes}\ {\isachardoublequoteopen}{\isasymphi}\ {\isasymin}\ formula{\isachardoublequoteclose}\ {\isachardoublequoteopen}{\isacharbrackleft}{\kern0pt}a{\isacharcomma}{\kern0pt}\ b{\isacharcomma}{\kern0pt}\ c{\isacharcomma}{\kern0pt}\ d{\isacharcomma}{\kern0pt}\ e{\isacharcomma}{\kern0pt}\ f{\isacharcomma}{\kern0pt}\ g{\isacharbrackright}{\kern0pt}\ {\isacharat}{\kern0pt}\ env\ {\isasymin}\ list{\isacharparenleft}{\kern0pt}M{\isacharparenright}{\kern0pt}{\isachardoublequoteclose}\ \isanewline
\ \ \isakeyword{shows}\ {\isachardoublequoteopen}sats{\isacharparenleft}{\kern0pt}M{\isacharcomma}{\kern0pt}\ ren{\isacharunderscore}{\kern0pt}sep{\isacharunderscore}{\kern0pt}forces{\isacharparenleft}{\kern0pt}{\isasymphi}{\isacharparenright}{\kern0pt}{\isacharcomma}{\kern0pt}\ {\isacharbrackleft}{\kern0pt}a{\isacharcomma}{\kern0pt}\ b{\isacharcomma}{\kern0pt}\ c{\isacharcomma}{\kern0pt}\ d{\isacharcomma}{\kern0pt}\ e{\isacharcomma}{\kern0pt}\ f{\isacharcomma}{\kern0pt}\ g{\isacharbrackright}{\kern0pt}\ {\isacharat}{\kern0pt}\ env{\isacharparenright}{\kern0pt}\ {\isasymlongleftrightarrow}\ sats{\isacharparenleft}{\kern0pt}M{\isacharcomma}{\kern0pt}\ {\isasymphi}{\isacharcomma}{\kern0pt}\ {\isacharbrackleft}{\kern0pt}b{\isacharcomma}{\kern0pt}\ d{\isacharcomma}{\kern0pt}\ e{\isacharcomma}{\kern0pt}\ f{\isacharcomma}{\kern0pt}\ g{\isacharcomma}{\kern0pt}\ a{\isacharbrackright}{\kern0pt}\ {\isacharat}{\kern0pt}\ env{\isacharparenright}{\kern0pt}{\isachardoublequoteclose}\ \isanewline
%
\isadelimproof
\ \ %
\endisadelimproof
%
\isatagproof
\isacommand{unfolding}\isamarkupfalse%
\ ren{\isacharunderscore}{\kern0pt}sep{\isacharunderscore}{\kern0pt}forces{\isacharunderscore}{\kern0pt}def\ \isanewline
\ \ \ \isacommand{apply}\isamarkupfalse%
\ {\isacharparenleft}{\kern0pt}insert\ sats{\isacharunderscore}{\kern0pt}incr{\isacharunderscore}{\kern0pt}bv{\isacharunderscore}{\kern0pt}iff\ {\isacharbrackleft}{\kern0pt}of\ {\isacharunderscore}{\kern0pt}\ {\isacharunderscore}{\kern0pt}\ M\ {\isacharunderscore}{\kern0pt}\ {\isachardoublequoteopen}{\isacharbrackleft}{\kern0pt}b{\isacharcomma}{\kern0pt}\ \ d{\isacharcomma}{\kern0pt}\ e{\isacharcomma}{\kern0pt}\ f{\isacharcomma}{\kern0pt}\ g{\isacharcomma}{\kern0pt}\ a{\isacharbrackright}{\kern0pt}{\isachardoublequoteclose}{\isacharbrackright}{\kern0pt}{\isacharparenright}{\kern0pt}\isanewline
\ \ \isacommand{using}\isamarkupfalse%
\ assms\isanewline
\ \ \isacommand{by}\isamarkupfalse%
\ simp%
\endisatagproof
{\isafoldproof}%
%
\isadelimproof
\isanewline
%
\endisadelimproof
\isanewline
\isacommand{definition}\isamarkupfalse%
\ sep{\isacharunderscore}{\kern0pt}forces{\isacharunderscore}{\kern0pt}pair{\isacharunderscore}{\kern0pt}fm\ \isakeyword{where}\ \isanewline
\ \ {\isachardoublequoteopen}sep{\isacharunderscore}{\kern0pt}forces{\isacharunderscore}{\kern0pt}pair{\isacharunderscore}{\kern0pt}fm{\isacharparenleft}{\kern0pt}{\isasymphi}{\isacharparenright}{\kern0pt}\ {\isasymequiv}\ Exists{\isacharparenleft}{\kern0pt}Exists{\isacharparenleft}{\kern0pt}And{\isacharparenleft}{\kern0pt}pair{\isacharunderscore}{\kern0pt}fm{\isacharparenleft}{\kern0pt}{\isadigit{0}}{\isacharcomma}{\kern0pt}\ {\isadigit{1}}{\isacharcomma}{\kern0pt}\ {\isadigit{2}}{\isacharparenright}{\kern0pt}{\isacharcomma}{\kern0pt}\ ren{\isacharunderscore}{\kern0pt}sep{\isacharunderscore}{\kern0pt}forces{\isacharparenleft}{\kern0pt}forcesHS{\isacharparenleft}{\kern0pt}{\isasymphi}{\isacharparenright}{\kern0pt}{\isacharparenright}{\kern0pt}{\isacharparenright}{\kern0pt}{\isacharparenright}{\kern0pt}{\isacharparenright}{\kern0pt}{\isachardoublequoteclose}\ \isanewline
\isanewline
\isacommand{lemma}\isamarkupfalse%
\ sats{\isacharunderscore}{\kern0pt}sep{\isacharunderscore}{\kern0pt}forces{\isacharunderscore}{\kern0pt}pair{\isacharunderscore}{\kern0pt}fm{\isacharunderscore}{\kern0pt}iff\ {\isacharcolon}{\kern0pt}\ \isanewline
\ \ \isakeyword{fixes}\ {\isasymphi}\ y\ p\ env\isanewline
\ \ \isakeyword{assumes}\ {\isachardoublequoteopen}{\isasymphi}\ {\isasymin}\ formula{\isachardoublequoteclose}\ {\isachardoublequoteopen}y\ {\isasymin}\ HS{\isachardoublequoteclose}\ {\isachardoublequoteopen}p\ {\isasymin}\ P{\isachardoublequoteclose}\ {\isachardoublequoteopen}env\ {\isasymin}\ list{\isacharparenleft}{\kern0pt}HS{\isacharparenright}{\kern0pt}{\isachardoublequoteclose}\ \ \isanewline
\ \ \isakeyword{shows}\ {\isachardoublequoteopen}sats{\isacharparenleft}{\kern0pt}M{\isacharcomma}{\kern0pt}\ sep{\isacharunderscore}{\kern0pt}forces{\isacharunderscore}{\kern0pt}pair{\isacharunderscore}{\kern0pt}fm{\isacharparenleft}{\kern0pt}{\isasymphi}{\isacharparenright}{\kern0pt}{\isacharcomma}{\kern0pt}\ {\isacharbrackleft}{\kern0pt}{\isacharless}{\kern0pt}y{\isacharcomma}{\kern0pt}\ p{\isachargreater}{\kern0pt}{\isacharcomma}{\kern0pt}\ P{\isacharcomma}{\kern0pt}\ leq{\isacharcomma}{\kern0pt}\ one{\isacharcomma}{\kern0pt}\ {\isacharless}{\kern0pt}{\isasymF}{\isacharcomma}{\kern0pt}\ {\isasymG}{\isacharcomma}{\kern0pt}\ P{\isacharcomma}{\kern0pt}\ P{\isacharunderscore}{\kern0pt}auto{\isachargreater}{\kern0pt}{\isacharbrackright}{\kern0pt}\ {\isacharat}{\kern0pt}\ env{\isacharparenright}{\kern0pt}\ {\isasymlongleftrightarrow}\ p\ {\isasymtturnstile}HS\ {\isasymphi}\ {\isacharbrackleft}{\kern0pt}y{\isacharbrackright}{\kern0pt}\ {\isacharat}{\kern0pt}\ env{\isachardoublequoteclose}\ \isanewline
%
\isadelimproof
\isanewline
\ \ %
\endisadelimproof
%
\isatagproof
\isacommand{unfolding}\isamarkupfalse%
\ sep{\isacharunderscore}{\kern0pt}forces{\isacharunderscore}{\kern0pt}pair{\isacharunderscore}{\kern0pt}fm{\isacharunderscore}{\kern0pt}def\ \isanewline
\ \ \isacommand{apply}\isamarkupfalse%
{\isacharparenleft}{\kern0pt}subgoal{\isacharunderscore}{\kern0pt}tac\ {\isachardoublequoteopen}y\ {\isasymin}\ M\ {\isasymand}\ p\ {\isasymin}\ M\ {\isasymand}\ {\isacharless}{\kern0pt}y{\isacharcomma}{\kern0pt}\ p{\isachargreater}{\kern0pt}\ {\isasymin}\ M\ {\isasymand}\ env\ {\isasymin}\ list{\isacharparenleft}{\kern0pt}M{\isacharparenright}{\kern0pt}\ {\isasymand}\ P\ {\isasymin}\ M\ {\isasymand}\ leq\ {\isasymin}\ M\ {\isasymand}\ one\ {\isasymin}\ M\ {\isasymand}\ {\isacharless}{\kern0pt}{\isasymF}{\isacharcomma}{\kern0pt}\ {\isasymG}{\isacharcomma}{\kern0pt}\ P{\isacharcomma}{\kern0pt}\ P{\isacharunderscore}{\kern0pt}auto{\isachargreater}{\kern0pt}\ {\isasymin}\ M{\isachardoublequoteclose}{\isacharparenright}{\kern0pt}\isanewline
\ \ \ \isacommand{apply}\isamarkupfalse%
\ simp\isanewline
\ \ \ \isacommand{apply}\isamarkupfalse%
{\isacharparenleft}{\kern0pt}rule{\isacharunderscore}{\kern0pt}tac\ Q{\isacharequal}{\kern0pt}{\isachardoublequoteopen}sats{\isacharparenleft}{\kern0pt}M{\isacharcomma}{\kern0pt}\ ren{\isacharunderscore}{\kern0pt}sep{\isacharunderscore}{\kern0pt}forces{\isacharparenleft}{\kern0pt}forcesHS{\isacharparenleft}{\kern0pt}{\isasymphi}{\isacharparenright}{\kern0pt}{\isacharparenright}{\kern0pt}{\isacharcomma}{\kern0pt}\ {\isacharbrackleft}{\kern0pt}y{\isacharcomma}{\kern0pt}\ p{\isacharcomma}{\kern0pt}\ {\isacharless}{\kern0pt}y{\isacharcomma}{\kern0pt}\ p{\isachargreater}{\kern0pt}{\isacharcomma}{\kern0pt}\ P{\isacharcomma}{\kern0pt}\ leq{\isacharcomma}{\kern0pt}\ one{\isacharcomma}{\kern0pt}\ {\isacharless}{\kern0pt}{\isasymF}{\isacharcomma}{\kern0pt}\ {\isasymG}{\isacharcomma}{\kern0pt}\ P{\isacharcomma}{\kern0pt}\ P{\isacharunderscore}{\kern0pt}auto{\isachargreater}{\kern0pt}{\isacharbrackright}{\kern0pt}\ {\isacharat}{\kern0pt}\ env{\isacharparenright}{\kern0pt}{\isachardoublequoteclose}\ \isakeyword{in}\ iff{\isacharunderscore}{\kern0pt}trans{\isacharparenright}{\kern0pt}\isanewline
\ \ \ \ \isacommand{apply}\isamarkupfalse%
\ force\ \isanewline
\ \ \ \isacommand{apply}\isamarkupfalse%
{\isacharparenleft}{\kern0pt}rule\ iff{\isacharunderscore}{\kern0pt}trans{\isacharcomma}{\kern0pt}\ rule\ sats{\isacharunderscore}{\kern0pt}ren{\isacharunderscore}{\kern0pt}sep{\isacharunderscore}{\kern0pt}forces{\isacharunderscore}{\kern0pt}fm{\isacharunderscore}{\kern0pt}iff{\isacharparenright}{\kern0pt}\isanewline
\ \ \isacommand{using}\isamarkupfalse%
\ forcesHS{\isacharunderscore}{\kern0pt}type\ assms\ \isanewline
\ \ \ \ \ \isacommand{apply}\isamarkupfalse%
\ auto{\isacharbrackleft}{\kern0pt}{\isadigit{3}}{\isacharbrackright}{\kern0pt}\isanewline
\ \ \isacommand{using}\isamarkupfalse%
\ assms\ HS{\isacharunderscore}{\kern0pt}iff\ P{\isacharunderscore}{\kern0pt}in{\isacharunderscore}{\kern0pt}M\ transM\ pair{\isacharunderscore}{\kern0pt}in{\isacharunderscore}{\kern0pt}M{\isacharunderscore}{\kern0pt}iff\ leq{\isacharunderscore}{\kern0pt}in{\isacharunderscore}{\kern0pt}M\ one{\isacharunderscore}{\kern0pt}in{\isacharunderscore}{\kern0pt}M\ {\isasymF}{\isacharunderscore}{\kern0pt}in{\isacharunderscore}{\kern0pt}M\ {\isasymG}{\isacharunderscore}{\kern0pt}in{\isacharunderscore}{\kern0pt}M\ P{\isacharunderscore}{\kern0pt}auto{\isacharunderscore}{\kern0pt}in{\isacharunderscore}{\kern0pt}M\ P{\isacharunderscore}{\kern0pt}name{\isacharunderscore}{\kern0pt}in{\isacharunderscore}{\kern0pt}M\ \isanewline
\ \ \isacommand{apply}\isamarkupfalse%
\ simp\isanewline
\ \ \isacommand{apply}\isamarkupfalse%
{\isacharparenleft}{\kern0pt}rule{\isacharunderscore}{\kern0pt}tac\ A{\isacharequal}{\kern0pt}{\isachardoublequoteopen}list{\isacharparenleft}{\kern0pt}HS{\isacharparenright}{\kern0pt}{\isachardoublequoteclose}\ \isakeyword{in}\ subsetD{\isacharcomma}{\kern0pt}\ rule\ list{\isacharunderscore}{\kern0pt}mono{\isacharparenright}{\kern0pt}\isanewline
\ \ \isacommand{using}\isamarkupfalse%
\ assms\ HS{\isacharunderscore}{\kern0pt}iff\ P{\isacharunderscore}{\kern0pt}name{\isacharunderscore}{\kern0pt}in{\isacharunderscore}{\kern0pt}M\isanewline
\ \ \isacommand{by}\isamarkupfalse%
\ auto%
\endisatagproof
{\isafoldproof}%
%
\isadelimproof
\isanewline
%
\endisadelimproof
\isanewline
\isacommand{lemma}\isamarkupfalse%
\ sep{\isacharunderscore}{\kern0pt}forces{\isacharunderscore}{\kern0pt}pair{\isacharunderscore}{\kern0pt}in{\isacharunderscore}{\kern0pt}HS\ {\isacharcolon}{\kern0pt}\ \isanewline
\ \ \isakeyword{fixes}\ x\ env\ {\isasymphi}\isanewline
\ \ \isakeyword{assumes}\ {\isachardoublequoteopen}x\ {\isasymin}\ HS{\isachardoublequoteclose}\ {\isachardoublequoteopen}env\ {\isasymin}\ list{\isacharparenleft}{\kern0pt}HS{\isacharparenright}{\kern0pt}{\isachardoublequoteclose}\ {\isachardoublequoteopen}{\isasymphi}\ {\isasymin}\ formula{\isachardoublequoteclose}\ {\isachardoublequoteopen}arity{\isacharparenleft}{\kern0pt}{\isasymphi}{\isacharparenright}{\kern0pt}\ {\isasymle}\ succ{\isacharparenleft}{\kern0pt}length{\isacharparenleft}{\kern0pt}env{\isacharparenright}{\kern0pt}{\isacharparenright}{\kern0pt}{\isachardoublequoteclose}\ \isanewline
\ \ \isakeyword{shows}\ {\isachardoublequoteopen}{\isacharbraceleft}{\kern0pt}\ {\isacharless}{\kern0pt}y{\isacharcomma}{\kern0pt}\ p{\isachargreater}{\kern0pt}\ {\isasymin}\ domain{\isacharparenleft}{\kern0pt}x{\isacharparenright}{\kern0pt}\ {\isasymtimes}\ P{\isachardot}{\kern0pt}\ p\ {\isasymtturnstile}HS\ {\isasymphi}\ {\isacharbrackleft}{\kern0pt}y{\isacharbrackright}{\kern0pt}\ {\isacharat}{\kern0pt}\ env\ {\isacharbraceright}{\kern0pt}\ {\isasymin}\ HS{\isachardoublequoteclose}\isanewline
%
\isadelimproof
%
\endisadelimproof
%
\isatagproof
\isacommand{proof}\isamarkupfalse%
\ {\isacharminus}{\kern0pt}\ \isanewline
\ \ \isacommand{define}\isamarkupfalse%
\ X\ \isakeyword{where}\ {\isachardoublequoteopen}X\ {\isasymequiv}\ {\isacharbraceleft}{\kern0pt}\ {\isacharless}{\kern0pt}y{\isacharcomma}{\kern0pt}\ p{\isachargreater}{\kern0pt}\ {\isasymin}\ domain{\isacharparenleft}{\kern0pt}x{\isacharparenright}{\kern0pt}\ {\isasymtimes}\ P{\isachardot}{\kern0pt}\ p\ {\isasymtturnstile}HS\ {\isasymphi}\ {\isacharbrackleft}{\kern0pt}y{\isacharbrackright}{\kern0pt}\ {\isacharat}{\kern0pt}\ env\ {\isacharbraceright}{\kern0pt}{\isachardoublequoteclose}\ \isanewline
\isanewline
\ \ \isacommand{have}\isamarkupfalse%
\ envin\ {\isacharcolon}{\kern0pt}\ {\isachardoublequoteopen}env\ {\isasymin}\ list{\isacharparenleft}{\kern0pt}M{\isacharparenright}{\kern0pt}{\isachardoublequoteclose}\ \isanewline
\ \ \ \ \isacommand{apply}\isamarkupfalse%
{\isacharparenleft}{\kern0pt}rule{\isacharunderscore}{\kern0pt}tac\ A{\isacharequal}{\kern0pt}{\isachardoublequoteopen}list{\isacharparenleft}{\kern0pt}HS{\isacharparenright}{\kern0pt}{\isachardoublequoteclose}\ \isakeyword{in}\ subsetD{\isacharparenright}{\kern0pt}\isanewline
\ \ \ \ \ \isacommand{apply}\isamarkupfalse%
{\isacharparenleft}{\kern0pt}rule\ list{\isacharunderscore}{\kern0pt}mono{\isacharparenright}{\kern0pt}\isanewline
\ \ \ \ \isacommand{using}\isamarkupfalse%
\ HS{\isacharunderscore}{\kern0pt}iff\ assms\ P{\isacharunderscore}{\kern0pt}name{\isacharunderscore}{\kern0pt}in{\isacharunderscore}{\kern0pt}M\ \isanewline
\ \ \ \ \isacommand{by}\isamarkupfalse%
\ auto\isanewline
\isanewline
\ \ \isacommand{have}\isamarkupfalse%
\ XinM\ {\isacharcolon}{\kern0pt}\ {\isachardoublequoteopen}X\ {\isasymin}\ M{\isachardoublequoteclose}\ \isanewline
\ \ \isacommand{proof}\isamarkupfalse%
\ {\isacharminus}{\kern0pt}\ \isanewline
\ \ \ \ \isacommand{have}\isamarkupfalse%
\ inM\ {\isacharcolon}{\kern0pt}\ {\isachardoublequoteopen}{\isacharbraceleft}{\kern0pt}\ z\ {\isasymin}\ domain{\isacharparenleft}{\kern0pt}x{\isacharparenright}{\kern0pt}\ {\isasymtimes}\ P{\isachardot}{\kern0pt}\ sats{\isacharparenleft}{\kern0pt}M{\isacharcomma}{\kern0pt}\ sep{\isacharunderscore}{\kern0pt}forces{\isacharunderscore}{\kern0pt}pair{\isacharunderscore}{\kern0pt}fm{\isacharparenleft}{\kern0pt}{\isasymphi}{\isacharparenright}{\kern0pt}{\isacharcomma}{\kern0pt}\ {\isacharbrackleft}{\kern0pt}z{\isacharbrackright}{\kern0pt}\ {\isacharat}{\kern0pt}\ {\isacharbrackleft}{\kern0pt}P{\isacharcomma}{\kern0pt}\ leq{\isacharcomma}{\kern0pt}\ one{\isacharcomma}{\kern0pt}\ {\isacharless}{\kern0pt}{\isasymF}{\isacharcomma}{\kern0pt}\ {\isasymG}{\isacharcomma}{\kern0pt}\ P{\isacharcomma}{\kern0pt}\ P{\isacharunderscore}{\kern0pt}auto{\isachargreater}{\kern0pt}{\isacharbrackright}{\kern0pt}\ {\isacharat}{\kern0pt}\ env{\isacharparenright}{\kern0pt}\ {\isacharbraceright}{\kern0pt}\ {\isasymin}\ M{\isachardoublequoteclose}\ \isanewline
\ \ \ \ \ \ \isacommand{apply}\isamarkupfalse%
{\isacharparenleft}{\kern0pt}rule\ separation{\isacharunderscore}{\kern0pt}notation{\isacharparenright}{\kern0pt}\isanewline
\ \ \ \ \ \ \ \isacommand{apply}\isamarkupfalse%
{\isacharparenleft}{\kern0pt}rule\ separation{\isacharunderscore}{\kern0pt}ax{\isacharparenright}{\kern0pt}\isanewline
\ \ \ \ \ \ \isacommand{unfolding}\isamarkupfalse%
\ sep{\isacharunderscore}{\kern0pt}forces{\isacharunderscore}{\kern0pt}pair{\isacharunderscore}{\kern0pt}fm{\isacharunderscore}{\kern0pt}def\isanewline
\ \ \ \ \ \ \isacommand{using}\isamarkupfalse%
\ forcesHS{\isacharunderscore}{\kern0pt}type\ ren{\isacharunderscore}{\kern0pt}sep{\isacharunderscore}{\kern0pt}forces{\isacharunderscore}{\kern0pt}type\ assms\ \isanewline
\ \ \ \ \ \ \ \ \ \isacommand{apply}\isamarkupfalse%
\ force\ \isanewline
\ \ \ \ \ \ \isacommand{using}\isamarkupfalse%
\ pair{\isacharunderscore}{\kern0pt}in{\isacharunderscore}{\kern0pt}M{\isacharunderscore}{\kern0pt}iff\ P{\isacharunderscore}{\kern0pt}in{\isacharunderscore}{\kern0pt}M\ leq{\isacharunderscore}{\kern0pt}in{\isacharunderscore}{\kern0pt}M\ one{\isacharunderscore}{\kern0pt}in{\isacharunderscore}{\kern0pt}M\ {\isasymF}{\isacharunderscore}{\kern0pt}in{\isacharunderscore}{\kern0pt}M\ {\isasymG}{\isacharunderscore}{\kern0pt}in{\isacharunderscore}{\kern0pt}M\ P{\isacharunderscore}{\kern0pt}auto{\isacharunderscore}{\kern0pt}in{\isacharunderscore}{\kern0pt}M\ assms\ envin\ \isanewline
\ \ \ \ \ \ \ \ \isacommand{apply}\isamarkupfalse%
\ force\ \isanewline
\ \ \ \ \ \ \ \isacommand{apply}\isamarkupfalse%
\ simp\isanewline
\ \ \ \ \ \ \isacommand{using}\isamarkupfalse%
\ forcesHS{\isacharunderscore}{\kern0pt}type\ ren{\isacharunderscore}{\kern0pt}sep{\isacharunderscore}{\kern0pt}forces{\isacharunderscore}{\kern0pt}type\ assms\ \isanewline
\ \ \ \ \ \ \ \isacommand{apply}\isamarkupfalse%
{\isacharparenleft}{\kern0pt}rule{\isacharunderscore}{\kern0pt}tac\ pred{\isacharunderscore}{\kern0pt}le{\isacharcomma}{\kern0pt}\ force{\isacharcomma}{\kern0pt}\ force{\isacharparenright}{\kern0pt}{\isacharplus}{\kern0pt}\isanewline
\ \ \ \ \ \ \ \isacommand{apply}\isamarkupfalse%
{\isacharparenleft}{\kern0pt}rule\ Un{\isacharunderscore}{\kern0pt}least{\isacharunderscore}{\kern0pt}lt{\isacharparenright}{\kern0pt}\ \ \ \ \ \ \ \ \ \ \ \ \ \ \ \ \ \ \ \ \ \ \ \ \ \ \ \ \ \ \ \ \ \ \ \ \ \ \ \ \ \ \ \ \ \ \ \ \ \ \ \ \ \ \ \ \ \ \isanewline
\ \ \ \ \ \ \ \ \isacommand{apply}\isamarkupfalse%
\ {\isacharparenleft}{\kern0pt}subst\ arity{\isacharunderscore}{\kern0pt}pair{\isacharunderscore}{\kern0pt}fm{\isacharparenright}{\kern0pt}\isanewline
\ \ \ \ \ \ \ \ \ \ \ \isacommand{apply}\isamarkupfalse%
\ auto{\isacharbrackleft}{\kern0pt}{\isadigit{3}}{\isacharbrackright}{\kern0pt}\isanewline
\ \ \ \ \ \ \ \ \isacommand{apply}\isamarkupfalse%
\ simp\isanewline
\ \ \ \ \ \ \ \ \isacommand{apply}\isamarkupfalse%
{\isacharparenleft}{\kern0pt}rule\ Un{\isacharunderscore}{\kern0pt}least{\isacharunderscore}{\kern0pt}lt{\isacharcomma}{\kern0pt}\ simp{\isacharparenright}{\kern0pt}{\isacharplus}{\kern0pt}\isanewline
\ \ \ \ \ \ \ \ \isacommand{apply}\isamarkupfalse%
\ simp\isanewline
\ \ \ \ \ \ \ \isacommand{apply}\isamarkupfalse%
{\isacharparenleft}{\kern0pt}rule\ le{\isacharunderscore}{\kern0pt}trans{\isacharcomma}{\kern0pt}\ rule\ arity{\isacharunderscore}{\kern0pt}ren{\isacharunderscore}{\kern0pt}sep{\isacharunderscore}{\kern0pt}forces{\isacharparenright}{\kern0pt}\isanewline
\ \ \ \ \ \ \isacommand{using}\isamarkupfalse%
\ forcesHS{\isacharunderscore}{\kern0pt}type\ assms\ \isanewline
\ \ \ \ \ \ \ \ \isacommand{apply}\isamarkupfalse%
\ force\ \isanewline
\ \ \ \ \ \ \ \isacommand{apply}\isamarkupfalse%
{\isacharparenleft}{\kern0pt}rule\ Un{\isacharunderscore}{\kern0pt}least{\isacharunderscore}{\kern0pt}lt{\isacharcomma}{\kern0pt}\ simp{\isacharparenright}{\kern0pt}\isanewline
\ \ \ \ \ \ \ \ \isacommand{apply}\isamarkupfalse%
{\isacharparenleft}{\kern0pt}rule{\isacharunderscore}{\kern0pt}tac\ le{\isacharunderscore}{\kern0pt}trans{\isacharcomma}{\kern0pt}\ rule\ arity{\isacharunderscore}{\kern0pt}forcesHS{\isacharcomma}{\kern0pt}\ simp{\isacharcomma}{\kern0pt}\ simp{\isacharcomma}{\kern0pt}\ simp{\isacharparenright}{\kern0pt}\isanewline
\ \ \ \ \ \ \isacommand{using}\isamarkupfalse%
\ domain{\isacharunderscore}{\kern0pt}closed\ P{\isacharunderscore}{\kern0pt}in{\isacharunderscore}{\kern0pt}M\ cartprod{\isacharunderscore}{\kern0pt}closed\ assms\ HS{\isacharunderscore}{\kern0pt}iff\ P{\isacharunderscore}{\kern0pt}name{\isacharunderscore}{\kern0pt}in{\isacharunderscore}{\kern0pt}M\isanewline
\ \ \ \ \ \ \isacommand{by}\isamarkupfalse%
\ auto\isanewline
\isanewline
\ \ \ \ \isacommand{have}\isamarkupfalse%
\ {\isachardoublequoteopen}X\ {\isacharequal}{\kern0pt}\ {\isacharbraceleft}{\kern0pt}\ {\isacharless}{\kern0pt}y{\isacharcomma}{\kern0pt}\ p{\isachargreater}{\kern0pt}\ {\isasymin}\ domain{\isacharparenleft}{\kern0pt}x{\isacharparenright}{\kern0pt}\ {\isasymtimes}\ P{\isachardot}{\kern0pt}\ sats{\isacharparenleft}{\kern0pt}M{\isacharcomma}{\kern0pt}\ sep{\isacharunderscore}{\kern0pt}forces{\isacharunderscore}{\kern0pt}pair{\isacharunderscore}{\kern0pt}fm{\isacharparenleft}{\kern0pt}{\isasymphi}{\isacharparenright}{\kern0pt}{\isacharcomma}{\kern0pt}\ {\isacharbrackleft}{\kern0pt}{\isacharless}{\kern0pt}y{\isacharcomma}{\kern0pt}\ p{\isachargreater}{\kern0pt}{\isacharcomma}{\kern0pt}\ P{\isacharcomma}{\kern0pt}\ leq{\isacharcomma}{\kern0pt}\ one{\isacharcomma}{\kern0pt}\ {\isacharless}{\kern0pt}{\isasymF}{\isacharcomma}{\kern0pt}\ {\isasymG}{\isacharcomma}{\kern0pt}\ P{\isacharcomma}{\kern0pt}\ P{\isacharunderscore}{\kern0pt}auto{\isachargreater}{\kern0pt}{\isacharbrackright}{\kern0pt}\ {\isacharat}{\kern0pt}\ env{\isacharparenright}{\kern0pt}\ {\isacharbraceright}{\kern0pt}{\isachardoublequoteclose}\ \isanewline
\ \ \ \ \ \ \isacommand{unfolding}\isamarkupfalse%
\ X{\isacharunderscore}{\kern0pt}def\ \isanewline
\ \ \ \ \ \ \isacommand{apply}\isamarkupfalse%
{\isacharparenleft}{\kern0pt}rule\ iff{\isacharunderscore}{\kern0pt}eq{\isacharparenright}{\kern0pt}\isanewline
\ \ \ \ \ \ \isacommand{apply}\isamarkupfalse%
\ clarsimp\isanewline
\ \ \ \ \ \ \isacommand{apply}\isamarkupfalse%
{\isacharparenleft}{\kern0pt}rename{\isacharunderscore}{\kern0pt}tac\ y\ p\ q{\isacharcomma}{\kern0pt}\ subgoal{\isacharunderscore}{\kern0pt}tac\ {\isachardoublequoteopen}y\ {\isasymin}\ HS{\isachardoublequoteclose}{\isacharparenright}{\kern0pt}\isanewline
\ \ \ \ \ \ \isacommand{using}\isamarkupfalse%
\ sats{\isacharunderscore}{\kern0pt}sep{\isacharunderscore}{\kern0pt}forces{\isacharunderscore}{\kern0pt}pair{\isacharunderscore}{\kern0pt}fm{\isacharunderscore}{\kern0pt}iff\ assms\ \isanewline
\ \ \ \ \ \ \ \isacommand{apply}\isamarkupfalse%
\ force\ \isanewline
\ \ \ \ \ \ \isacommand{using}\isamarkupfalse%
\ HS{\isacharunderscore}{\kern0pt}iff\ assms\ \isanewline
\ \ \ \ \ \ \isacommand{by}\isamarkupfalse%
\ auto\isanewline
\ \ \ \ \isacommand{also}\isamarkupfalse%
\ \isacommand{have}\isamarkupfalse%
\ {\isachardoublequoteopen}{\isachardot}{\kern0pt}{\isachardot}{\kern0pt}{\isachardot}{\kern0pt}\ {\isacharequal}{\kern0pt}\ {\isacharbraceleft}{\kern0pt}\ z\ {\isasymin}\ domain{\isacharparenleft}{\kern0pt}x{\isacharparenright}{\kern0pt}\ {\isasymtimes}\ P{\isachardot}{\kern0pt}\ sats{\isacharparenleft}{\kern0pt}M{\isacharcomma}{\kern0pt}\ sep{\isacharunderscore}{\kern0pt}forces{\isacharunderscore}{\kern0pt}pair{\isacharunderscore}{\kern0pt}fm{\isacharparenleft}{\kern0pt}{\isasymphi}{\isacharparenright}{\kern0pt}{\isacharcomma}{\kern0pt}\ {\isacharbrackleft}{\kern0pt}z{\isacharcomma}{\kern0pt}\ P{\isacharcomma}{\kern0pt}\ leq{\isacharcomma}{\kern0pt}\ one{\isacharcomma}{\kern0pt}\ {\isacharless}{\kern0pt}{\isasymF}{\isacharcomma}{\kern0pt}\ {\isasymG}{\isacharcomma}{\kern0pt}\ P{\isacharcomma}{\kern0pt}\ P{\isacharunderscore}{\kern0pt}auto{\isachargreater}{\kern0pt}{\isacharbrackright}{\kern0pt}\ {\isacharat}{\kern0pt}\ env{\isacharparenright}{\kern0pt}\ {\isacharbraceright}{\kern0pt}{\isachardoublequoteclose}\ \isanewline
\ \ \ \ \ \ \isacommand{by}\isamarkupfalse%
\ auto\isanewline
\ \ \ \ \isacommand{finally}\isamarkupfalse%
\ \isacommand{show}\isamarkupfalse%
\ {\isachardoublequoteopen}X\ {\isasymin}\ M{\isachardoublequoteclose}\ \isacommand{using}\isamarkupfalse%
\ inM\ \isacommand{by}\isamarkupfalse%
\ auto\isanewline
\ \ \isacommand{qed}\isamarkupfalse%
\isanewline
\isanewline
\ \ \isacommand{have}\isamarkupfalse%
\ Xpname\ {\isacharcolon}{\kern0pt}\ {\isachardoublequoteopen}X\ {\isasymin}\ P{\isacharunderscore}{\kern0pt}names{\isachardoublequoteclose}\isanewline
\ \ \ \ \isacommand{apply}\isamarkupfalse%
{\isacharparenleft}{\kern0pt}rule\ iffD{\isadigit{2}}{\isacharcomma}{\kern0pt}\ rule\ P{\isacharunderscore}{\kern0pt}name{\isacharunderscore}{\kern0pt}iff{\isacharcomma}{\kern0pt}\ rule\ conjI{\isacharcomma}{\kern0pt}\ rule\ XinM{\isacharparenright}{\kern0pt}\isanewline
\ \ \ \ \isacommand{unfolding}\isamarkupfalse%
\ X{\isacharunderscore}{\kern0pt}def\ \isanewline
\ \ \ \ \isacommand{using}\isamarkupfalse%
\ assms\ HS{\isacharunderscore}{\kern0pt}iff\ \isanewline
\ \ \ \ \isacommand{by}\isamarkupfalse%
\ auto\isanewline
\isanewline
\ \ \isacommand{define}\isamarkupfalse%
\ S\ \isakeyword{where}\ {\isachardoublequoteopen}S\ {\isasymequiv}\ INTsym{\isacharparenleft}{\kern0pt}Cons{\isacharparenleft}{\kern0pt}x{\isacharcomma}{\kern0pt}\ env{\isacharparenright}{\kern0pt}{\isacharparenright}{\kern0pt}{\isachardoublequoteclose}\ \isanewline
\ \ \isacommand{have}\isamarkupfalse%
\ SH\ {\isacharcolon}{\kern0pt}\ {\isachardoublequoteopen}S\ {\isasymin}\ {\isasymF}\ {\isasymand}\ {\isacharparenleft}{\kern0pt}{\isasymforall}{\isasympi}\ {\isasymin}\ S{\isachardot}{\kern0pt}\ map{\isacharparenleft}{\kern0pt}{\isasymlambda}x{\isachardot}{\kern0pt}\ Pn{\isacharunderscore}{\kern0pt}auto{\isacharparenleft}{\kern0pt}{\isasympi}{\isacharparenright}{\kern0pt}{\isacharbackquote}{\kern0pt}x{\isacharcomma}{\kern0pt}\ Cons{\isacharparenleft}{\kern0pt}x{\isacharcomma}{\kern0pt}\ env{\isacharparenright}{\kern0pt}{\isacharparenright}{\kern0pt}\ {\isacharequal}{\kern0pt}\ Cons{\isacharparenleft}{\kern0pt}x{\isacharcomma}{\kern0pt}\ env{\isacharparenright}{\kern0pt}{\isacharparenright}{\kern0pt}{\isachardoublequoteclose}\ \isanewline
\ \ \ \ \isacommand{unfolding}\isamarkupfalse%
\ S{\isacharunderscore}{\kern0pt}def\isanewline
\ \ \ \ \isacommand{apply}\isamarkupfalse%
{\isacharparenleft}{\kern0pt}rule\ INT{\isacharunderscore}{\kern0pt}set{\isacharunderscore}{\kern0pt}of{\isacharunderscore}{\kern0pt}list{\isacharunderscore}{\kern0pt}of{\isacharunderscore}{\kern0pt}{\isasymF}{\isacharunderscore}{\kern0pt}lemma{\isacharparenright}{\kern0pt}\ \isanewline
\ \ \ \ \isacommand{using}\isamarkupfalse%
\ assms\ \isanewline
\ \ \ \ \isacommand{by}\isamarkupfalse%
\ auto\isanewline
\ \ \isacommand{then}\isamarkupfalse%
\ \isacommand{have}\isamarkupfalse%
\ {\isachardoublequoteopen}{\isasymforall}{\isasympi}\ {\isasymin}\ S{\isachardot}{\kern0pt}\ Pn{\isacharunderscore}{\kern0pt}auto{\isacharparenleft}{\kern0pt}{\isasympi}{\isacharparenright}{\kern0pt}{\isacharbackquote}{\kern0pt}x\ {\isacharequal}{\kern0pt}\ x{\isachardoublequoteclose}\ \isacommand{by}\isamarkupfalse%
\ auto\isanewline
\ \ \isacommand{then}\isamarkupfalse%
\ \isacommand{have}\isamarkupfalse%
\ Ssubsetsymx\ {\isacharcolon}{\kern0pt}\ {\isachardoublequoteopen}S\ {\isasymsubseteq}\ sym{\isacharparenleft}{\kern0pt}x{\isacharparenright}{\kern0pt}{\isachardoublequoteclose}\ \isacommand{unfolding}\isamarkupfalse%
\ sym{\isacharunderscore}{\kern0pt}def\ \isacommand{using}\isamarkupfalse%
\ {\isasymF}{\isacharunderscore}{\kern0pt}subset\ SH\ P{\isacharunderscore}{\kern0pt}auto{\isacharunderscore}{\kern0pt}subgroups{\isacharunderscore}{\kern0pt}def\ \isacommand{by}\isamarkupfalse%
\ auto\ \isanewline
\ \ \isacommand{then}\isamarkupfalse%
\ \isacommand{have}\isamarkupfalse%
\ envmapeq\ {\isacharcolon}{\kern0pt}\ {\isachardoublequoteopen}{\isasymforall}{\isasympi}\ {\isasymin}\ S{\isachardot}{\kern0pt}\ map{\isacharparenleft}{\kern0pt}{\isasymlambda}x{\isachardot}{\kern0pt}\ Pn{\isacharunderscore}{\kern0pt}auto{\isacharparenleft}{\kern0pt}{\isasympi}{\isacharparenright}{\kern0pt}{\isacharbackquote}{\kern0pt}x{\isacharcomma}{\kern0pt}\ env{\isacharparenright}{\kern0pt}\ {\isacharequal}{\kern0pt}\ env{\isachardoublequoteclose}\ \isacommand{using}\isamarkupfalse%
\ SH\ \isacommand{by}\isamarkupfalse%
\ auto\isanewline
\isanewline
\ \ \isacommand{have}\isamarkupfalse%
\ {\isachardoublequoteopen}S\ {\isasymsubseteq}\ sym{\isacharparenleft}{\kern0pt}X{\isacharparenright}{\kern0pt}{\isachardoublequoteclose}\ \isanewline
\ \ \isacommand{proof}\isamarkupfalse%
{\isacharparenleft}{\kern0pt}rule\ subsetI{\isacharparenright}{\kern0pt}\isanewline
\ \ \ \ \isacommand{fix}\isamarkupfalse%
\ {\isasympi}\ \isanewline
\ \ \ \ \isacommand{assume}\isamarkupfalse%
\ piin\ {\isacharcolon}{\kern0pt}\ {\isachardoublequoteopen}{\isasympi}\ {\isasymin}\ S{\isachardoublequoteclose}\ \isanewline
\ \ \ \ \isacommand{then}\isamarkupfalse%
\ \isacommand{have}\isamarkupfalse%
\ piin{\isasymG}\ {\isacharcolon}{\kern0pt}\ {\isachardoublequoteopen}{\isasympi}\ {\isasymin}\ {\isasymG}{\isachardoublequoteclose}\ \isacommand{using}\isamarkupfalse%
\ S{\isacharunderscore}{\kern0pt}def\ SH\ {\isasymF}{\isacharunderscore}{\kern0pt}subset\ P{\isacharunderscore}{\kern0pt}auto{\isacharunderscore}{\kern0pt}subgroups{\isacharunderscore}{\kern0pt}def\ \isacommand{by}\isamarkupfalse%
\ force\isanewline
\isanewline
\ \ \ \ \isacommand{have}\isamarkupfalse%
\ {\isachardoublequoteopen}Pn{\isacharunderscore}{\kern0pt}auto{\isacharparenleft}{\kern0pt}{\isasympi}{\isacharparenright}{\kern0pt}{\isacharbackquote}{\kern0pt}X\ {\isacharequal}{\kern0pt}\ {\isacharbraceleft}{\kern0pt}\ {\isacharless}{\kern0pt}Pn{\isacharunderscore}{\kern0pt}auto{\isacharparenleft}{\kern0pt}{\isasympi}{\isacharparenright}{\kern0pt}{\isacharbackquote}{\kern0pt}y{\isacharcomma}{\kern0pt}\ {\isasympi}{\isacharbackquote}{\kern0pt}p{\isachargreater}{\kern0pt}\ {\isachardot}{\kern0pt}\ {\isacharless}{\kern0pt}y{\isacharcomma}{\kern0pt}\ p{\isachargreater}{\kern0pt}\ {\isasymin}\ X\ {\isacharbraceright}{\kern0pt}{\isachardoublequoteclose}\ \isanewline
\ \ \ \ \ \ \isacommand{apply}\isamarkupfalse%
{\isacharparenleft}{\kern0pt}subst\ Pn{\isacharunderscore}{\kern0pt}auto{\isacharparenright}{\kern0pt}\isanewline
\ \ \ \ \ \ \isacommand{using}\isamarkupfalse%
\ Xpname\isanewline
\ \ \ \ \ \ \isacommand{by}\isamarkupfalse%
\ auto\isanewline
\ \ \ \ \isacommand{also}\isamarkupfalse%
\ \isacommand{have}\isamarkupfalse%
\ {\isachardoublequoteopen}{\isachardot}{\kern0pt}{\isachardot}{\kern0pt}{\isachardot}{\kern0pt}\ {\isacharequal}{\kern0pt}\ X{\isachardoublequoteclose}\ \isanewline
\ \ \ \ \isacommand{proof}\isamarkupfalse%
\ {\isacharparenleft}{\kern0pt}rule\ equality{\isacharunderscore}{\kern0pt}iffI{\isacharcomma}{\kern0pt}\ rule\ iffI{\isacharparenright}{\kern0pt}\isanewline
\ \ \ \ \ \ \isacommand{fix}\isamarkupfalse%
\ v\ \isacommand{assume}\isamarkupfalse%
\ {\isachardoublequoteopen}v\ {\isasymin}\ {\isacharbraceleft}{\kern0pt}{\isasymlangle}Pn{\isacharunderscore}{\kern0pt}auto{\isacharparenleft}{\kern0pt}{\isasympi}{\isacharparenright}{\kern0pt}\ {\isacharbackquote}{\kern0pt}\ y{\isacharcomma}{\kern0pt}\ {\isasympi}\ {\isacharbackquote}{\kern0pt}\ p{\isasymrangle}\ {\isachardot}{\kern0pt}\ {\isasymlangle}y{\isacharcomma}{\kern0pt}p{\isasymrangle}\ {\isasymin}\ X{\isacharbraceright}{\kern0pt}{\isachardoublequoteclose}\ \ \isanewline
\ \ \ \ \ \ \isacommand{then}\isamarkupfalse%
\ \isacommand{have}\isamarkupfalse%
\ {\isachardoublequoteopen}{\isasymexists}y\ p{\isachardot}{\kern0pt}\ {\isacharless}{\kern0pt}y{\isacharcomma}{\kern0pt}\ p{\isachargreater}{\kern0pt}\ {\isasymin}\ X\ {\isasymand}\ v\ {\isacharequal}{\kern0pt}\ {\isasymlangle}Pn{\isacharunderscore}{\kern0pt}auto{\isacharparenleft}{\kern0pt}{\isasympi}{\isacharparenright}{\kern0pt}\ {\isacharbackquote}{\kern0pt}\ y{\isacharcomma}{\kern0pt}\ {\isasympi}\ {\isacharbackquote}{\kern0pt}\ p{\isasymrangle}{\isachardoublequoteclose}\ \isanewline
\ \ \ \ \ \ \ \ \isacommand{apply}\isamarkupfalse%
{\isacharparenleft}{\kern0pt}rule{\isacharunderscore}{\kern0pt}tac\ pair{\isacharunderscore}{\kern0pt}rel{\isacharunderscore}{\kern0pt}arg{\isacharparenright}{\kern0pt}\isanewline
\ \ \ \ \ \ \ \ \isacommand{using}\isamarkupfalse%
\ Xpname\ relation{\isacharunderscore}{\kern0pt}def\ P{\isacharunderscore}{\kern0pt}name{\isacharunderscore}{\kern0pt}iff\isanewline
\ \ \ \ \ \ \ \ \isacommand{by}\isamarkupfalse%
\ auto\isanewline
\ \ \ \ \ \ \isacommand{then}\isamarkupfalse%
\ \isacommand{obtain}\isamarkupfalse%
\ y\ p\ \isakeyword{where}\ ypH{\isacharcolon}{\kern0pt}\ {\isachardoublequoteopen}{\isacharless}{\kern0pt}y{\isacharcomma}{\kern0pt}\ p{\isachargreater}{\kern0pt}\ {\isasymin}\ X{\isachardoublequoteclose}\ {\isachardoublequoteopen}v\ {\isacharequal}{\kern0pt}\ {\isasymlangle}Pn{\isacharunderscore}{\kern0pt}auto{\isacharparenleft}{\kern0pt}{\isasympi}{\isacharparenright}{\kern0pt}\ {\isacharbackquote}{\kern0pt}\ y{\isacharcomma}{\kern0pt}\ {\isasympi}\ {\isacharbackquote}{\kern0pt}\ p{\isasymrangle}{\isachardoublequoteclose}\ {\isachardoublequoteopen}p\ {\isasymin}\ P{\isachardoublequoteclose}\isanewline
\ \ \ \ \ \ \ \ \isacommand{using}\isamarkupfalse%
\ Xpname\ P{\isacharunderscore}{\kern0pt}name{\isacharunderscore}{\kern0pt}iff\ \isanewline
\ \ \ \ \ \ \ \ \isacommand{by}\isamarkupfalse%
\ auto\isanewline
\ \ \ \ \ \ \isacommand{then}\isamarkupfalse%
\ \isacommand{have}\isamarkupfalse%
\ yindom\ {\isacharcolon}{\kern0pt}\ {\isachardoublequoteopen}y\ {\isasymin}\ domain{\isacharparenleft}{\kern0pt}x{\isacharparenright}{\kern0pt}{\isachardoublequoteclose}\ \isacommand{using}\isamarkupfalse%
\ ypH\ X{\isacharunderscore}{\kern0pt}def\ assms\ \isacommand{by}\isamarkupfalse%
\ force\ \isanewline
\ \ \ \ \ \ \isacommand{then}\isamarkupfalse%
\ \isacommand{have}\isamarkupfalse%
\ yinHS\ {\isacharcolon}{\kern0pt}\ {\isachardoublequoteopen}y\ {\isasymin}\ HS{\isachardoublequoteclose}\ \isacommand{using}\isamarkupfalse%
\ HS{\isacharunderscore}{\kern0pt}iff\ assms\ \isacommand{by}\isamarkupfalse%
\ auto\isanewline
\isanewline
\ \ \ \ \ \ \isacommand{have}\isamarkupfalse%
\ indom\ {\isacharcolon}{\kern0pt}\ \ {\isachardoublequoteopen}Pn{\isacharunderscore}{\kern0pt}auto{\isacharparenleft}{\kern0pt}{\isasympi}{\isacharparenright}{\kern0pt}{\isacharbackquote}{\kern0pt}y\ {\isasymin}\ domain{\isacharparenleft}{\kern0pt}x{\isacharparenright}{\kern0pt}{\isachardoublequoteclose}\isanewline
\ \ \ \ \ \ \ \ \isacommand{apply}\isamarkupfalse%
{\isacharparenleft}{\kern0pt}rule\ iffD{\isadigit{1}}{\isacharcomma}{\kern0pt}\ rule\ Pn{\isacharunderscore}{\kern0pt}auto{\isacharunderscore}{\kern0pt}domain{\isacharunderscore}{\kern0pt}closed{\isacharparenright}{\kern0pt}\isanewline
\ \ \ \ \ \ \ \ \isacommand{using}\isamarkupfalse%
\ assms\ HS{\isacharunderscore}{\kern0pt}iff\ piin\ yindom\ Ssubsetsymx\isanewline
\ \ \ \ \ \ \ \ \isacommand{by}\isamarkupfalse%
\ auto\isanewline
\isanewline
\ \ \ \ \ \ \isacommand{have}\isamarkupfalse%
\ {\isachardoublequoteopen}{\isasympi}{\isacharbackquote}{\kern0pt}p\ {\isasymtturnstile}HS\ {\isasymphi}\ {\isacharbrackleft}{\kern0pt}Pn{\isacharunderscore}{\kern0pt}auto{\isacharparenleft}{\kern0pt}{\isasympi}{\isacharparenright}{\kern0pt}{\isacharbackquote}{\kern0pt}y{\isacharbrackright}{\kern0pt}\ {\isacharat}{\kern0pt}\ env\ {\isasymlongleftrightarrow}\ {\isasympi}{\isacharbackquote}{\kern0pt}p\ {\isasymtturnstile}HS\ {\isasymphi}\ {\isacharbrackleft}{\kern0pt}Pn{\isacharunderscore}{\kern0pt}auto{\isacharparenleft}{\kern0pt}{\isasympi}{\isacharparenright}{\kern0pt}{\isacharbackquote}{\kern0pt}y{\isacharbrackright}{\kern0pt}\ {\isacharat}{\kern0pt}\ map{\isacharparenleft}{\kern0pt}{\isasymlambda}x{\isachardot}{\kern0pt}\ Pn{\isacharunderscore}{\kern0pt}auto{\isacharparenleft}{\kern0pt}{\isasympi}{\isacharparenright}{\kern0pt}{\isacharbackquote}{\kern0pt}x{\isacharcomma}{\kern0pt}\ env{\isacharparenright}{\kern0pt}{\isachardoublequoteclose}\ {\isacharparenleft}{\kern0pt}\isakeyword{is}\ {\isachardoublequoteopen}{\isacharquery}{\kern0pt}A\ {\isasymlongleftrightarrow}\ {\isacharunderscore}{\kern0pt}{\isachardoublequoteclose}{\isacharparenright}{\kern0pt}\ \isanewline
\ \ \ \ \ \ \ \ \isacommand{apply}\isamarkupfalse%
{\isacharparenleft}{\kern0pt}subst\ envmapeq{\isacharparenright}{\kern0pt}\isanewline
\ \ \ \ \ \ \ \ \isacommand{using}\isamarkupfalse%
\ piin\ \isanewline
\ \ \ \ \ \ \ \ \isacommand{by}\isamarkupfalse%
\ auto\isanewline
\ \ \ \ \ \ \isacommand{also}\isamarkupfalse%
\ \isacommand{have}\isamarkupfalse%
\ {\isachardoublequoteopen}{\isachardot}{\kern0pt}{\isachardot}{\kern0pt}{\isachardot}{\kern0pt}\ {\isasymlongleftrightarrow}\ \ {\isasympi}{\isacharbackquote}{\kern0pt}p\ {\isasymtturnstile}HS\ {\isasymphi}\ map{\isacharparenleft}{\kern0pt}{\isasymlambda}x{\isachardot}{\kern0pt}\ Pn{\isacharunderscore}{\kern0pt}auto{\isacharparenleft}{\kern0pt}{\isasympi}{\isacharparenright}{\kern0pt}{\isacharbackquote}{\kern0pt}x{\isacharcomma}{\kern0pt}\ Cons{\isacharparenleft}{\kern0pt}y{\isacharcomma}{\kern0pt}\ env{\isacharparenright}{\kern0pt}{\isacharparenright}{\kern0pt}\ {\isachardoublequoteclose}\isanewline
\ \ \ \ \ \ \ \ \isacommand{by}\isamarkupfalse%
\ auto\isanewline
\ \ \ \ \ \ \isacommand{also}\isamarkupfalse%
\ \isacommand{have}\isamarkupfalse%
\ {\isachardoublequoteopen}{\isachardot}{\kern0pt}{\isachardot}{\kern0pt}{\isachardot}{\kern0pt}\ {\isasymlongleftrightarrow}\ p\ {\isasymtturnstile}HS\ {\isasymphi}\ Cons{\isacharparenleft}{\kern0pt}y{\isacharcomma}{\kern0pt}\ env{\isacharparenright}{\kern0pt}{\isachardoublequoteclose}\ \isanewline
\ \ \ \ \ \ \ \ \isacommand{apply}\isamarkupfalse%
{\isacharparenleft}{\kern0pt}rule\ iff{\isacharunderscore}{\kern0pt}flip{\isacharparenright}{\kern0pt}\isanewline
\ \ \ \ \ \ \ \ \isacommand{apply}\isamarkupfalse%
{\isacharparenleft}{\kern0pt}rule\ symmetry{\isacharunderscore}{\kern0pt}lemma{\isacharparenright}{\kern0pt}\isanewline
\ \ \ \ \ \ \ \ \isacommand{using}\isamarkupfalse%
\ assms\ piin{\isasymG}\ {\isasymG}{\isacharunderscore}{\kern0pt}P{\isacharunderscore}{\kern0pt}auto{\isacharunderscore}{\kern0pt}group\ is{\isacharunderscore}{\kern0pt}P{\isacharunderscore}{\kern0pt}auto{\isacharunderscore}{\kern0pt}group{\isacharunderscore}{\kern0pt}def\ yinHS\ ypH\ \isanewline
\ \ \ \ \ \ \ \ \isacommand{by}\isamarkupfalse%
\ auto\isanewline
\ \ \ \ \ \ \isacommand{finally}\isamarkupfalse%
\ \isacommand{have}\isamarkupfalse%
\ {\isachardoublequoteopen}{\isacharquery}{\kern0pt}A{\isachardoublequoteclose}\ \isanewline
\ \ \ \ \ \ \ \ \isacommand{using}\isamarkupfalse%
\ ypH\ X{\isacharunderscore}{\kern0pt}def\ \isanewline
\ \ \ \ \ \ \ \ \isacommand{by}\isamarkupfalse%
\ auto\isanewline
\ \ \ \ \ \ \isacommand{then}\isamarkupfalse%
\ \isacommand{have}\isamarkupfalse%
\ {\isachardoublequoteopen}{\isacharless}{\kern0pt}Pn{\isacharunderscore}{\kern0pt}auto{\isacharparenleft}{\kern0pt}{\isasympi}{\isacharparenright}{\kern0pt}{\isacharbackquote}{\kern0pt}y{\isacharcomma}{\kern0pt}\ {\isasympi}{\isacharbackquote}{\kern0pt}p{\isachargreater}{\kern0pt}\ {\isasymin}\ X{\isachardoublequoteclose}\ \isanewline
\ \ \ \ \ \ \ \ \isacommand{unfolding}\isamarkupfalse%
\ X{\isacharunderscore}{\kern0pt}def\ \isanewline
\ \ \ \ \ \ \ \ \isacommand{using}\isamarkupfalse%
\ indom\ P{\isacharunderscore}{\kern0pt}auto{\isacharunderscore}{\kern0pt}value\ ypH\ piin{\isasymG}\ {\isasymG}{\isacharunderscore}{\kern0pt}P{\isacharunderscore}{\kern0pt}auto{\isacharunderscore}{\kern0pt}group\ is{\isacharunderscore}{\kern0pt}P{\isacharunderscore}{\kern0pt}auto{\isacharunderscore}{\kern0pt}group{\isacharunderscore}{\kern0pt}def\isanewline
\ \ \ \ \ \ \ \ \isacommand{by}\isamarkupfalse%
\ auto\isanewline
\ \ \ \ \ \ \isacommand{then}\isamarkupfalse%
\ \isacommand{show}\isamarkupfalse%
\ {\isachardoublequoteopen}v\ {\isasymin}\ X{\isachardoublequoteclose}\ \isacommand{using}\isamarkupfalse%
\ ypH\ \isacommand{by}\isamarkupfalse%
\ auto\isanewline
\ \ \ \ \isacommand{next}\isamarkupfalse%
\ \isanewline
\ \ \ \ \ \ \isacommand{fix}\isamarkupfalse%
\ v\ \isanewline
\ \ \ \ \ \ \isacommand{assume}\isamarkupfalse%
\ {\isachardoublequoteopen}v\ {\isasymin}\ X{\isachardoublequoteclose}\ \isanewline
\ \ \ \ \ \ \isacommand{then}\isamarkupfalse%
\ \isacommand{have}\isamarkupfalse%
\ {\isachardoublequoteopen}{\isasymexists}y\ p{\isachardot}{\kern0pt}\ y\ {\isasymin}\ domain{\isacharparenleft}{\kern0pt}x{\isacharparenright}{\kern0pt}\ {\isasymand}\ p\ {\isasymin}\ P\ {\isasymand}\ {\isacharless}{\kern0pt}y{\isacharcomma}{\kern0pt}\ p{\isachargreater}{\kern0pt}\ {\isacharequal}{\kern0pt}\ v\ {\isasymand}\ p\ {\isasymtturnstile}HS\ {\isasymphi}\ {\isacharbrackleft}{\kern0pt}y{\isacharbrackright}{\kern0pt}{\isacharat}{\kern0pt}env{\isachardoublequoteclose}\ \isanewline
\ \ \ \ \ \ \ \ \isacommand{unfolding}\isamarkupfalse%
\ X{\isacharunderscore}{\kern0pt}def\ \isacommand{by}\isamarkupfalse%
\ auto\isanewline
\ \ \ \ \ \ \isacommand{then}\isamarkupfalse%
\ \isacommand{obtain}\isamarkupfalse%
\ y\ p\ \isakeyword{where}\ ypH\ {\isacharcolon}{\kern0pt}\ {\isachardoublequoteopen}{\isacharless}{\kern0pt}y{\isacharcomma}{\kern0pt}\ p{\isachargreater}{\kern0pt}\ {\isacharequal}{\kern0pt}\ v{\isachardoublequoteclose}\ {\isachardoublequoteopen}y\ {\isasymin}\ domain{\isacharparenleft}{\kern0pt}x{\isacharparenright}{\kern0pt}{\isachardoublequoteclose}\ {\isachardoublequoteopen}p\ {\isasymin}\ P{\isachardoublequoteclose}\ {\isachardoublequoteopen}p\ {\isasymtturnstile}HS\ {\isasymphi}\ {\isacharbrackleft}{\kern0pt}y{\isacharbrackright}{\kern0pt}{\isacharat}{\kern0pt}env{\isachardoublequoteclose}\ \isanewline
\ \ \ \ \ \ \ \ \isacommand{unfolding}\isamarkupfalse%
\ X{\isacharunderscore}{\kern0pt}def\ \isacommand{by}\isamarkupfalse%
\ blast\isanewline
\ \ \ \ \ \ \isacommand{then}\isamarkupfalse%
\ \isacommand{have}\isamarkupfalse%
\ yinHS\ {\isacharcolon}{\kern0pt}\ {\isachardoublequoteopen}y\ {\isasymin}\ HS{\isachardoublequoteclose}\ \isacommand{using}\isamarkupfalse%
\ HS{\isacharunderscore}{\kern0pt}iff\ assms\ \isacommand{by}\isamarkupfalse%
\ auto\isanewline
\isanewline
\ \ \ \ \ \ \isacommand{have}\isamarkupfalse%
\ {\isachardoublequoteopen}Pn{\isacharunderscore}{\kern0pt}auto{\isacharparenleft}{\kern0pt}{\isasympi}{\isacharparenright}{\kern0pt}\ {\isasymin}\ bij{\isacharparenleft}{\kern0pt}P{\isacharunderscore}{\kern0pt}names{\isacharcomma}{\kern0pt}\ P{\isacharunderscore}{\kern0pt}names{\isacharparenright}{\kern0pt}{\isachardoublequoteclose}\ \isanewline
\ \ \ \ \ \ \ \ \isacommand{apply}\isamarkupfalse%
{\isacharparenleft}{\kern0pt}rule\ Pn{\isacharunderscore}{\kern0pt}auto{\isacharunderscore}{\kern0pt}bij{\isacharparenright}{\kern0pt}\isanewline
\ \ \ \ \ \ \ \ \isacommand{using}\isamarkupfalse%
\ assms\ piin{\isasymG}\ {\isasymG}{\isacharunderscore}{\kern0pt}P{\isacharunderscore}{\kern0pt}auto{\isacharunderscore}{\kern0pt}group\ is{\isacharunderscore}{\kern0pt}P{\isacharunderscore}{\kern0pt}auto{\isacharunderscore}{\kern0pt}group{\isacharunderscore}{\kern0pt}def\ yinHS\ ypH\ \isanewline
\ \ \ \ \ \ \ \ \isacommand{by}\isamarkupfalse%
\ auto\isanewline
\ \ \ \ \ \ \isacommand{then}\isamarkupfalse%
\ \isacommand{have}\isamarkupfalse%
\ {\isachardoublequoteopen}Pn{\isacharunderscore}{\kern0pt}auto{\isacharparenleft}{\kern0pt}{\isasympi}{\isacharparenright}{\kern0pt}\ {\isasymin}\ surj{\isacharparenleft}{\kern0pt}P{\isacharunderscore}{\kern0pt}names{\isacharcomma}{\kern0pt}\ P{\isacharunderscore}{\kern0pt}names{\isacharparenright}{\kern0pt}{\isachardoublequoteclose}\ \isacommand{using}\isamarkupfalse%
\ bij{\isacharunderscore}{\kern0pt}is{\isacharunderscore}{\kern0pt}surj\ \isacommand{by}\isamarkupfalse%
\ auto\isanewline
\ \ \ \ \ \ \isacommand{then}\isamarkupfalse%
\ \isacommand{obtain}\isamarkupfalse%
\ y{\isacharprime}{\kern0pt}\ \isakeyword{where}\ y{\isacharprime}{\kern0pt}H\ {\isacharcolon}{\kern0pt}\ {\isachardoublequoteopen}Pn{\isacharunderscore}{\kern0pt}auto{\isacharparenleft}{\kern0pt}{\isasympi}{\isacharparenright}{\kern0pt}{\isacharbackquote}{\kern0pt}y{\isacharprime}{\kern0pt}\ {\isacharequal}{\kern0pt}\ y{\isachardoublequoteclose}\ {\isachardoublequoteopen}y{\isacharprime}{\kern0pt}\ {\isasymin}\ P{\isacharunderscore}{\kern0pt}names{\isachardoublequoteclose}\ \isacommand{unfolding}\isamarkupfalse%
\ surj{\isacharunderscore}{\kern0pt}def\ \isacommand{using}\isamarkupfalse%
\ yinHS\ HS{\isacharunderscore}{\kern0pt}iff\ \isacommand{by}\isamarkupfalse%
\ auto\isanewline
\ \ \ \ \ \ \isacommand{then}\isamarkupfalse%
\ \isacommand{have}\isamarkupfalse%
\ y{\isacharprime}{\kern0pt}inHS\ {\isacharcolon}{\kern0pt}\ {\isachardoublequoteopen}y{\isacharprime}{\kern0pt}\ {\isasymin}\ HS{\isachardoublequoteclose}\ \isanewline
\ \ \ \ \ \ \ \ \isacommand{apply}\isamarkupfalse%
{\isacharparenleft}{\kern0pt}rule{\isacharunderscore}{\kern0pt}tac\ iffD{\isadigit{2}}{\isacharcomma}{\kern0pt}\ rule{\isacharunderscore}{\kern0pt}tac\ HS{\isacharunderscore}{\kern0pt}Pn{\isacharunderscore}{\kern0pt}auto{\isacharparenright}{\kern0pt}\isanewline
\ \ \ \ \ \ \ \ \isacommand{using}\isamarkupfalse%
\ piin{\isasymG}\ yinHS\ \isanewline
\ \ \ \ \ \ \ \ \isacommand{by}\isamarkupfalse%
\ auto\isanewline
\ \ \ \ \ \ \isacommand{then}\isamarkupfalse%
\ \isacommand{have}\isamarkupfalse%
\ y{\isacharprime}{\kern0pt}indom\ {\isacharcolon}{\kern0pt}\ {\isachardoublequoteopen}y{\isacharprime}{\kern0pt}\ {\isasymin}\ domain{\isacharparenleft}{\kern0pt}x{\isacharparenright}{\kern0pt}{\isachardoublequoteclose}\ \isanewline
\ \ \ \ \ \ \ \ \isacommand{apply}\isamarkupfalse%
{\isacharparenleft}{\kern0pt}rule{\isacharunderscore}{\kern0pt}tac\ iffD{\isadigit{2}}{\isacharparenright}{\kern0pt}\isanewline
\ \ \ \ \ \ \ \ \ \isacommand{apply}\isamarkupfalse%
{\isacharparenleft}{\kern0pt}rule\ Pn{\isacharunderscore}{\kern0pt}auto{\isacharunderscore}{\kern0pt}domain{\isacharunderscore}{\kern0pt}closed{\isacharparenright}{\kern0pt}\isanewline
\ \ \ \ \ \ \ \ \isacommand{using}\isamarkupfalse%
\ assms\ HS{\isacharunderscore}{\kern0pt}iff\ piin\ Ssubsetsymx\ y{\isacharprime}{\kern0pt}H\ ypH\ \isanewline
\ \ \ \ \ \ \ \ \isacommand{by}\isamarkupfalse%
\ auto\isanewline
\isanewline
\ \ \ \ \ \ \isacommand{have}\isamarkupfalse%
\ {\isachardoublequoteopen}{\isasympi}\ {\isasymin}\ bij{\isacharparenleft}{\kern0pt}P{\isacharcomma}{\kern0pt}\ P{\isacharparenright}{\kern0pt}{\isachardoublequoteclose}\ \isacommand{using}\isamarkupfalse%
\ piin{\isasymG}\ {\isasymG}{\isacharunderscore}{\kern0pt}P{\isacharunderscore}{\kern0pt}auto{\isacharunderscore}{\kern0pt}group\ is{\isacharunderscore}{\kern0pt}P{\isacharunderscore}{\kern0pt}auto{\isacharunderscore}{\kern0pt}group{\isacharunderscore}{\kern0pt}def\ is{\isacharunderscore}{\kern0pt}P{\isacharunderscore}{\kern0pt}auto{\isacharunderscore}{\kern0pt}def\ \isacommand{by}\isamarkupfalse%
\ auto\isanewline
\ \ \ \ \ \ \isacommand{then}\isamarkupfalse%
\ \isacommand{have}\isamarkupfalse%
\ {\isachardoublequoteopen}{\isasympi}\ {\isasymin}\ surj{\isacharparenleft}{\kern0pt}P{\isacharcomma}{\kern0pt}\ P{\isacharparenright}{\kern0pt}{\isachardoublequoteclose}\ \isacommand{using}\isamarkupfalse%
\ bij{\isacharunderscore}{\kern0pt}is{\isacharunderscore}{\kern0pt}surj\ \isacommand{by}\isamarkupfalse%
\ auto\isanewline
\ \ \ \ \ \ \isacommand{then}\isamarkupfalse%
\ \isacommand{obtain}\isamarkupfalse%
\ p{\isacharprime}{\kern0pt}\ \isakeyword{where}\ p{\isacharprime}{\kern0pt}H\ {\isacharcolon}{\kern0pt}\ {\isachardoublequoteopen}p{\isacharprime}{\kern0pt}\ {\isasymin}\ P{\isachardoublequoteclose}\ {\isachardoublequoteopen}{\isasympi}{\isacharbackquote}{\kern0pt}p{\isacharprime}{\kern0pt}\ {\isacharequal}{\kern0pt}\ p{\isachardoublequoteclose}\ \isacommand{using}\isamarkupfalse%
\ ypH\ surj{\isacharunderscore}{\kern0pt}def\ \isacommand{by}\isamarkupfalse%
\ auto\isanewline
\isanewline
\ \ \ \ \ \ \isacommand{then}\isamarkupfalse%
\ \isacommand{have}\isamarkupfalse%
\ {\isachardoublequoteopen}{\isasympi}{\isacharbackquote}{\kern0pt}p{\isacharprime}{\kern0pt}\ {\isasymtturnstile}HS\ {\isasymphi}\ {\isacharbrackleft}{\kern0pt}Pn{\isacharunderscore}{\kern0pt}auto{\isacharparenleft}{\kern0pt}{\isasympi}{\isacharparenright}{\kern0pt}{\isacharbackquote}{\kern0pt}y{\isacharprime}{\kern0pt}{\isacharbrackright}{\kern0pt}{\isacharat}{\kern0pt}env{\isachardoublequoteclose}\ \isacommand{using}\isamarkupfalse%
\ ypH\ y{\isacharprime}{\kern0pt}H\ p{\isacharprime}{\kern0pt}H\ \isacommand{by}\isamarkupfalse%
\ auto\ \isanewline
\ \ \ \ \ \ \isacommand{then}\isamarkupfalse%
\ \isacommand{have}\isamarkupfalse%
\ {\isachardoublequoteopen}{\isasympi}{\isacharbackquote}{\kern0pt}p{\isacharprime}{\kern0pt}\ {\isasymtturnstile}HS\ {\isasymphi}\ {\isacharbrackleft}{\kern0pt}Pn{\isacharunderscore}{\kern0pt}auto{\isacharparenleft}{\kern0pt}{\isasympi}{\isacharparenright}{\kern0pt}{\isacharbackquote}{\kern0pt}y{\isacharprime}{\kern0pt}{\isacharbrackright}{\kern0pt}{\isacharat}{\kern0pt}map{\isacharparenleft}{\kern0pt}{\isasymlambda}x{\isachardot}{\kern0pt}\ Pn{\isacharunderscore}{\kern0pt}auto{\isacharparenleft}{\kern0pt}{\isasympi}{\isacharparenright}{\kern0pt}{\isacharbackquote}{\kern0pt}x{\isacharcomma}{\kern0pt}\ env{\isacharparenright}{\kern0pt}{\isachardoublequoteclose}\ \isacommand{using}\isamarkupfalse%
\ envmapeq\ piin\ \isacommand{by}\isamarkupfalse%
\ auto\isanewline
\ \ \ \ \ \ \isacommand{then}\isamarkupfalse%
\ \isacommand{have}\isamarkupfalse%
\ {\isachardoublequoteopen}{\isasympi}{\isacharbackquote}{\kern0pt}p{\isacharprime}{\kern0pt}\ {\isasymtturnstile}HS\ {\isasymphi}\ map{\isacharparenleft}{\kern0pt}{\isasymlambda}x{\isachardot}{\kern0pt}\ Pn{\isacharunderscore}{\kern0pt}auto{\isacharparenleft}{\kern0pt}{\isasympi}{\isacharparenright}{\kern0pt}{\isacharbackquote}{\kern0pt}x{\isacharcomma}{\kern0pt}\ Cons{\isacharparenleft}{\kern0pt}y{\isacharprime}{\kern0pt}{\isacharcomma}{\kern0pt}\ env{\isacharparenright}{\kern0pt}{\isacharparenright}{\kern0pt}{\isachardoublequoteclose}\ \isacommand{by}\isamarkupfalse%
\ auto\isanewline
\ \ \ \ \ \ \isacommand{then}\isamarkupfalse%
\ \isacommand{have}\isamarkupfalse%
\ {\isachardoublequoteopen}p{\isacharprime}{\kern0pt}\ {\isasymtturnstile}HS\ {\isasymphi}\ Cons{\isacharparenleft}{\kern0pt}y{\isacharprime}{\kern0pt}{\isacharcomma}{\kern0pt}\ env{\isacharparenright}{\kern0pt}{\isachardoublequoteclose}\ \isanewline
\ \ \ \ \ \ \ \ \isacommand{apply}\isamarkupfalse%
{\isacharparenleft}{\kern0pt}rule{\isacharunderscore}{\kern0pt}tac\ iffD{\isadigit{2}}{\isacharcomma}{\kern0pt}\ rule{\isacharunderscore}{\kern0pt}tac\ symmetry{\isacharunderscore}{\kern0pt}lemma{\isacharparenright}{\kern0pt}\isanewline
\ \ \ \ \ \ \ \ \isacommand{using}\isamarkupfalse%
\ assms\ piin{\isasymG}\ {\isasymG}{\isacharunderscore}{\kern0pt}P{\isacharunderscore}{\kern0pt}auto{\isacharunderscore}{\kern0pt}group\ is{\isacharunderscore}{\kern0pt}P{\isacharunderscore}{\kern0pt}auto{\isacharunderscore}{\kern0pt}group{\isacharunderscore}{\kern0pt}def\ p{\isacharprime}{\kern0pt}H\ y{\isacharprime}{\kern0pt}inHS\isanewline
\ \ \ \ \ \ \ \ \isacommand{by}\isamarkupfalse%
\ auto\isanewline
\ \ \ \ \ \ \isacommand{then}\isamarkupfalse%
\ \isacommand{have}\isamarkupfalse%
\ {\isachardoublequoteopen}{\isacharless}{\kern0pt}y{\isacharprime}{\kern0pt}{\isacharcomma}{\kern0pt}\ p{\isacharprime}{\kern0pt}{\isachargreater}{\kern0pt}\ {\isasymin}\ X{\isachardoublequoteclose}\ \isanewline
\ \ \ \ \ \ \ \ \isacommand{unfolding}\isamarkupfalse%
\ X{\isacharunderscore}{\kern0pt}def\ \isanewline
\ \ \ \ \ \ \ \ \isacommand{using}\isamarkupfalse%
\ y{\isacharprime}{\kern0pt}indom\ p{\isacharprime}{\kern0pt}H\ \isanewline
\ \ \ \ \ \ \ \ \isacommand{by}\isamarkupfalse%
\ auto\isanewline
\ \ \ \ \ \ \isacommand{then}\isamarkupfalse%
\ \isacommand{have}\isamarkupfalse%
\ {\isachardoublequoteopen}{\isacharless}{\kern0pt}Pn{\isacharunderscore}{\kern0pt}auto{\isacharparenleft}{\kern0pt}{\isasympi}{\isacharparenright}{\kern0pt}{\isacharbackquote}{\kern0pt}y{\isacharprime}{\kern0pt}{\isacharcomma}{\kern0pt}\ {\isasympi}{\isacharbackquote}{\kern0pt}p{\isacharprime}{\kern0pt}{\isachargreater}{\kern0pt}\ {\isasymin}\ {\isacharbraceleft}{\kern0pt}{\isasymlangle}Pn{\isacharunderscore}{\kern0pt}auto{\isacharparenleft}{\kern0pt}{\isasympi}{\isacharparenright}{\kern0pt}\ {\isacharbackquote}{\kern0pt}\ y{\isacharcomma}{\kern0pt}\ {\isasympi}\ {\isacharbackquote}{\kern0pt}\ p{\isasymrangle}\ {\isachardot}{\kern0pt}\ {\isasymlangle}y{\isacharcomma}{\kern0pt}p{\isasymrangle}\ {\isasymin}\ X{\isacharbraceright}{\kern0pt}{\isachardoublequoteclose}\ \isanewline
\ \ \ \ \ \ \ \ \isacommand{apply}\isamarkupfalse%
{\isacharparenleft}{\kern0pt}rule{\isacharunderscore}{\kern0pt}tac\ pair{\isacharunderscore}{\kern0pt}relI{\isacharparenright}{\kern0pt}\isanewline
\ \ \ \ \ \ \ \ \isacommand{by}\isamarkupfalse%
\ simp\isanewline
\ \ \ \ \ \ \isacommand{then}\isamarkupfalse%
\ \isacommand{show}\isamarkupfalse%
\ {\isachardoublequoteopen}v\ {\isasymin}\ {\isacharbraceleft}{\kern0pt}{\isasymlangle}Pn{\isacharunderscore}{\kern0pt}auto{\isacharparenleft}{\kern0pt}{\isasympi}{\isacharparenright}{\kern0pt}\ {\isacharbackquote}{\kern0pt}\ y{\isacharcomma}{\kern0pt}\ {\isasympi}\ {\isacharbackquote}{\kern0pt}\ p{\isasymrangle}\ {\isachardot}{\kern0pt}\ {\isasymlangle}y{\isacharcomma}{\kern0pt}p{\isasymrangle}\ {\isasymin}\ X{\isacharbraceright}{\kern0pt}{\isachardoublequoteclose}\ \isacommand{using}\isamarkupfalse%
\ ypH\ y{\isacharprime}{\kern0pt}H\ p{\isacharprime}{\kern0pt}H\ \isacommand{by}\isamarkupfalse%
\ auto\isanewline
\ \ \ \ \isacommand{qed}\isamarkupfalse%
\isanewline
\ \ \ \ \isacommand{finally}\isamarkupfalse%
\ \isacommand{show}\isamarkupfalse%
\ {\isachardoublequoteopen}{\isasympi}\ {\isasymin}\ sym{\isacharparenleft}{\kern0pt}X{\isacharparenright}{\kern0pt}{\isachardoublequoteclose}\ \ \isanewline
\ \ \ \ \ \ \isacommand{unfolding}\isamarkupfalse%
\ sym{\isacharunderscore}{\kern0pt}def\ \isanewline
\ \ \ \ \ \ \isacommand{using}\isamarkupfalse%
\ piin{\isasymG}\isanewline
\ \ \ \ \ \ \isacommand{by}\isamarkupfalse%
\ auto\isanewline
\ \ \isacommand{qed}\isamarkupfalse%
\isanewline
\isanewline
\ \ \isacommand{have}\isamarkupfalse%
\ {\isachardoublequoteopen}sym{\isacharparenleft}{\kern0pt}X{\isacharparenright}{\kern0pt}\ {\isasymin}\ P{\isacharunderscore}{\kern0pt}auto{\isacharunderscore}{\kern0pt}subgroups{\isacharparenleft}{\kern0pt}{\isasymG}{\isacharparenright}{\kern0pt}{\isachardoublequoteclose}\ \isanewline
\ \ \ \ \isacommand{apply}\isamarkupfalse%
{\isacharparenleft}{\kern0pt}rule\ sym{\isacharunderscore}{\kern0pt}P{\isacharunderscore}{\kern0pt}auto{\isacharunderscore}{\kern0pt}subgroup{\isacharparenright}{\kern0pt}\isanewline
\ \ \ \ \isacommand{using}\isamarkupfalse%
\ Xpname\ \isanewline
\ \ \ \ \isacommand{by}\isamarkupfalse%
\ auto\isanewline
\ \ \ \ \isanewline
\ \ \isacommand{then}\isamarkupfalse%
\ \isacommand{have}\isamarkupfalse%
\ {\isachardoublequoteopen}sym{\isacharparenleft}{\kern0pt}X{\isacharparenright}{\kern0pt}\ {\isasymin}\ {\isasymF}{\isachardoublequoteclose}\ \isanewline
\ \ \ \ \isacommand{using}\isamarkupfalse%
\ {\isasymF}{\isacharunderscore}{\kern0pt}closed{\isacharunderscore}{\kern0pt}under{\isacharunderscore}{\kern0pt}supergroup\ SH\ {\isacartoucheopen}S\ {\isasymsubseteq}\ sym{\isacharparenleft}{\kern0pt}X{\isacharparenright}{\kern0pt}{\isacartoucheclose}\isanewline
\ \ \ \ \isacommand{by}\isamarkupfalse%
\ auto\isanewline
\isanewline
\ \ \isacommand{then}\isamarkupfalse%
\ \isacommand{have}\isamarkupfalse%
\ {\isachardoublequoteopen}X\ {\isasymin}\ HS{\isachardoublequoteclose}\ \isanewline
\ \ \ \ \isacommand{using}\isamarkupfalse%
\ X{\isacharunderscore}{\kern0pt}def\ HS{\isacharunderscore}{\kern0pt}iff\ assms\ sym{\isacharunderscore}{\kern0pt}def\ symmetric{\isacharunderscore}{\kern0pt}def\ Xpname\ \isanewline
\ \ \ \ \isacommand{by}\isamarkupfalse%
\ auto\isanewline
\ \ \isanewline
\ \ \isacommand{then}\isamarkupfalse%
\ \isacommand{show}\isamarkupfalse%
\ {\isacharquery}{\kern0pt}thesis\ \isanewline
\ \ \ \ \isacommand{unfolding}\isamarkupfalse%
\ X{\isacharunderscore}{\kern0pt}def\ \isanewline
\ \ \ \ \isacommand{by}\isamarkupfalse%
\ simp\isanewline
\isacommand{qed}\isamarkupfalse%
%
\endisatagproof
{\isafoldproof}%
%
\isadelimproof
\isanewline
%
\endisadelimproof
\isanewline
\isacommand{lemma}\isamarkupfalse%
\ SymExt{\isacharunderscore}{\kern0pt}separation\ {\isacharcolon}{\kern0pt}\ \isanewline
\ \ \isakeyword{fixes}\ x\ env\ {\isasymphi}\ \isanewline
\ \ \isakeyword{assumes}\ {\isachardoublequoteopen}x\ {\isasymin}\ SymExt{\isacharparenleft}{\kern0pt}G{\isacharparenright}{\kern0pt}{\isachardoublequoteclose}\ {\isachardoublequoteopen}env\ {\isasymin}\ list{\isacharparenleft}{\kern0pt}SymExt{\isacharparenleft}{\kern0pt}G{\isacharparenright}{\kern0pt}{\isacharparenright}{\kern0pt}{\isachardoublequoteclose}\ {\isachardoublequoteopen}{\isasymphi}\ {\isasymin}\ formula{\isachardoublequoteclose}\ {\isachardoublequoteopen}arity{\isacharparenleft}{\kern0pt}{\isasymphi}{\isacharparenright}{\kern0pt}\ {\isasymle}\ succ{\isacharparenleft}{\kern0pt}length{\isacharparenleft}{\kern0pt}env{\isacharparenright}{\kern0pt}{\isacharparenright}{\kern0pt}{\isachardoublequoteclose}\ \isanewline
\ \ \isakeyword{shows}\ {\isachardoublequoteopen}{\isacharbraceleft}{\kern0pt}\ y\ {\isasymin}\ x{\isachardot}{\kern0pt}\ sats{\isacharparenleft}{\kern0pt}SymExt{\isacharparenleft}{\kern0pt}G{\isacharparenright}{\kern0pt}{\isacharcomma}{\kern0pt}\ {\isasymphi}{\isacharcomma}{\kern0pt}\ {\isacharbrackleft}{\kern0pt}y{\isacharbrackright}{\kern0pt}\ {\isacharat}{\kern0pt}\ env{\isacharparenright}{\kern0pt}\ {\isacharbraceright}{\kern0pt}\ {\isasymin}\ SymExt{\isacharparenleft}{\kern0pt}G{\isacharparenright}{\kern0pt}{\isachardoublequoteclose}\isanewline
%
\isadelimproof
%
\endisadelimproof
%
\isatagproof
\isacommand{proof}\isamarkupfalse%
\ {\isacharminus}{\kern0pt}\ \isanewline
\ \ \isacommand{have}\isamarkupfalse%
\ {\isachardoublequoteopen}{\isasymexists}env{\isacharprime}{\kern0pt}{\isasymin}list{\isacharparenleft}{\kern0pt}HS{\isacharparenright}{\kern0pt}{\isachardot}{\kern0pt}\ map{\isacharparenleft}{\kern0pt}val{\isacharparenleft}{\kern0pt}G{\isacharparenright}{\kern0pt}{\isacharcomma}{\kern0pt}\ env{\isacharprime}{\kern0pt}{\isacharparenright}{\kern0pt}\ {\isacharequal}{\kern0pt}\ env{\isachardoublequoteclose}\ \isanewline
\ \ \ \ \isacommand{apply}\isamarkupfalse%
{\isacharparenleft}{\kern0pt}rule\ ex{\isacharunderscore}{\kern0pt}HS{\isacharunderscore}{\kern0pt}name{\isacharunderscore}{\kern0pt}list{\isacharparenright}{\kern0pt}\isanewline
\ \ \ \ \isacommand{using}\isamarkupfalse%
\ assms\isanewline
\ \ \ \ \isacommand{by}\isamarkupfalse%
\ auto\isanewline
\ \ \isacommand{then}\isamarkupfalse%
\ \isacommand{obtain}\isamarkupfalse%
\ env{\isacharprime}{\kern0pt}\ \isakeyword{where}\ env{\isacharprime}{\kern0pt}H{\isacharcolon}{\kern0pt}\ {\isachardoublequoteopen}env{\isacharprime}{\kern0pt}\ {\isasymin}\ list{\isacharparenleft}{\kern0pt}HS{\isacharparenright}{\kern0pt}{\isachardoublequoteclose}\ {\isachardoublequoteopen}map{\isacharparenleft}{\kern0pt}val{\isacharparenleft}{\kern0pt}G{\isacharparenright}{\kern0pt}{\isacharcomma}{\kern0pt}\ env{\isacharprime}{\kern0pt}{\isacharparenright}{\kern0pt}\ {\isacharequal}{\kern0pt}\ env{\isachardoublequoteclose}\ \isacommand{by}\isamarkupfalse%
\ auto\isanewline
\isanewline
\ \ \isacommand{have}\isamarkupfalse%
\ env{\isacharprime}{\kern0pt}in\ {\isacharcolon}{\kern0pt}\ {\isachardoublequoteopen}env{\isacharprime}{\kern0pt}\ {\isasymin}\ list{\isacharparenleft}{\kern0pt}M{\isacharparenright}{\kern0pt}{\isachardoublequoteclose}\ \isanewline
\ \ \ \ \isacommand{apply}\isamarkupfalse%
{\isacharparenleft}{\kern0pt}rule{\isacharunderscore}{\kern0pt}tac\ A{\isacharequal}{\kern0pt}{\isachardoublequoteopen}list{\isacharparenleft}{\kern0pt}HS{\isacharparenright}{\kern0pt}{\isachardoublequoteclose}\ \isakeyword{in}\ subsetD{\isacharcomma}{\kern0pt}\ rule\ list{\isacharunderscore}{\kern0pt}mono{\isacharparenright}{\kern0pt}\isanewline
\ \ \ \ \isacommand{using}\isamarkupfalse%
\ HS{\isacharunderscore}{\kern0pt}iff\ P{\isacharunderscore}{\kern0pt}name{\isacharunderscore}{\kern0pt}in{\isacharunderscore}{\kern0pt}M\ env{\isacharprime}{\kern0pt}H\ \isanewline
\ \ \ \ \isacommand{by}\isamarkupfalse%
\ auto\isanewline
\isanewline
\ \ \isacommand{have}\isamarkupfalse%
\ leneq\ {\isacharcolon}{\kern0pt}\ {\isachardoublequoteopen}length{\isacharparenleft}{\kern0pt}env{\isacharprime}{\kern0pt}{\isacharparenright}{\kern0pt}\ {\isacharequal}{\kern0pt}\ length{\isacharparenleft}{\kern0pt}env{\isacharparenright}{\kern0pt}{\isachardoublequoteclose}\ \isacommand{using}\isamarkupfalse%
\ env{\isacharprime}{\kern0pt}H\ length{\isacharunderscore}{\kern0pt}map\ \isacommand{by}\isamarkupfalse%
\ auto\isanewline
\isanewline
\ \ \isacommand{obtain}\isamarkupfalse%
\ x{\isacharprime}{\kern0pt}\ \isakeyword{where}\ x{\isacharprime}{\kern0pt}H\ {\isacharcolon}{\kern0pt}\ {\isachardoublequoteopen}val{\isacharparenleft}{\kern0pt}G{\isacharcomma}{\kern0pt}\ x{\isacharprime}{\kern0pt}{\isacharparenright}{\kern0pt}\ {\isacharequal}{\kern0pt}\ x{\isachardoublequoteclose}\ {\isachardoublequoteopen}x{\isacharprime}{\kern0pt}\ {\isasymin}\ HS{\isachardoublequoteclose}\ \isacommand{using}\isamarkupfalse%
\ assms\ SymExt{\isacharunderscore}{\kern0pt}def\ \isacommand{by}\isamarkupfalse%
\ auto\ \isanewline
\isanewline
\ \ \isacommand{define}\isamarkupfalse%
\ X\ \isakeyword{where}\ {\isachardoublequoteopen}X\ {\isasymequiv}\ {\isacharbraceleft}{\kern0pt}\ {\isacharless}{\kern0pt}y{\isacharcomma}{\kern0pt}\ p{\isachargreater}{\kern0pt}\ {\isasymin}\ domain{\isacharparenleft}{\kern0pt}x{\isacharprime}{\kern0pt}{\isacharparenright}{\kern0pt}\ {\isasymtimes}\ P{\isachardot}{\kern0pt}\ p\ {\isasymtturnstile}HS\ And{\isacharparenleft}{\kern0pt}Member{\isacharparenleft}{\kern0pt}{\isadigit{0}}{\isacharcomma}{\kern0pt}\ succ{\isacharparenleft}{\kern0pt}length{\isacharparenleft}{\kern0pt}env{\isacharprime}{\kern0pt}{\isacharparenright}{\kern0pt}{\isacharparenright}{\kern0pt}{\isacharparenright}{\kern0pt}{\isacharcomma}{\kern0pt}\ {\isasymphi}{\isacharparenright}{\kern0pt}\ {\isacharbrackleft}{\kern0pt}y{\isacharbrackright}{\kern0pt}\ {\isacharat}{\kern0pt}\ env{\isacharprime}{\kern0pt}\ {\isacharat}{\kern0pt}\ {\isacharbrackleft}{\kern0pt}x{\isacharprime}{\kern0pt}{\isacharbrackright}{\kern0pt}\ {\isacharbraceright}{\kern0pt}{\isachardoublequoteclose}\ \ \isanewline
\ \ \isacommand{have}\isamarkupfalse%
\ {\isachardoublequoteopen}X\ {\isasymin}\ HS{\isachardoublequoteclose}\ \isanewline
\ \ \ \ \isacommand{unfolding}\isamarkupfalse%
\ X{\isacharunderscore}{\kern0pt}def\ \isanewline
\ \ \ \ \isacommand{apply}\isamarkupfalse%
{\isacharparenleft}{\kern0pt}rule\ sep{\isacharunderscore}{\kern0pt}forces{\isacharunderscore}{\kern0pt}pair{\isacharunderscore}{\kern0pt}in{\isacharunderscore}{\kern0pt}HS{\isacharparenright}{\kern0pt}\isanewline
\ \ \ \ \isacommand{using}\isamarkupfalse%
\ assms\ x{\isacharprime}{\kern0pt}H\ env{\isacharprime}{\kern0pt}H\ leneq\isanewline
\ \ \ \ \ \ \ \isacommand{apply}\isamarkupfalse%
\ auto{\isacharbrackleft}{\kern0pt}{\isadigit{3}}{\isacharbrackright}{\kern0pt}\isanewline
\ \ \ \ \isacommand{apply}\isamarkupfalse%
\ simp\isanewline
\ \ \ \ \isacommand{apply}\isamarkupfalse%
{\isacharparenleft}{\kern0pt}rule\ Un{\isacharunderscore}{\kern0pt}least{\isacharunderscore}{\kern0pt}lt{\isacharparenright}{\kern0pt}{\isacharplus}{\kern0pt}\isanewline
\ \ \ \ \isacommand{using}\isamarkupfalse%
\ assms\ x{\isacharprime}{\kern0pt}H\ env{\isacharprime}{\kern0pt}H\ leneq\isanewline
\ \ \ \ \ \ \isacommand{apply}\isamarkupfalse%
\ auto{\isacharbrackleft}{\kern0pt}{\isadigit{2}}{\isacharbrackright}{\kern0pt}\isanewline
\ \ \ \ \isacommand{apply}\isamarkupfalse%
{\isacharparenleft}{\kern0pt}rule{\isacharunderscore}{\kern0pt}tac\ j{\isacharequal}{\kern0pt}{\isachardoublequoteopen}succ{\isacharparenleft}{\kern0pt}length{\isacharparenleft}{\kern0pt}env{\isacharparenright}{\kern0pt}{\isacharparenright}{\kern0pt}{\isachardoublequoteclose}\ \isakeyword{in}\ le{\isacharunderscore}{\kern0pt}trans{\isacharparenright}{\kern0pt}\isanewline
\ \ \ \ \isacommand{using}\isamarkupfalse%
\ assms\ x{\isacharprime}{\kern0pt}H\ env{\isacharprime}{\kern0pt}H\ leneq\isanewline
\ \ \ \ \isacommand{by}\isamarkupfalse%
\ auto\isanewline
\ \ \isacommand{then}\isamarkupfalse%
\ \isacommand{have}\isamarkupfalse%
\ {\isachardoublequoteopen}val{\isacharparenleft}{\kern0pt}G{\isacharcomma}{\kern0pt}\ X{\isacharparenright}{\kern0pt}\ {\isasymin}\ SymExt{\isacharparenleft}{\kern0pt}G{\isacharparenright}{\kern0pt}{\isachardoublequoteclose}\ \isacommand{using}\isamarkupfalse%
\ SymExt{\isacharunderscore}{\kern0pt}def\ \isacommand{by}\isamarkupfalse%
\ auto\isanewline
\isanewline
\ \ \isacommand{define}\isamarkupfalse%
\ Y\ \isakeyword{where}\ {\isachardoublequoteopen}Y\ {\isasymequiv}\ {\isacharbraceleft}{\kern0pt}\ y\ {\isasymin}\ x{\isachardot}{\kern0pt}\ sats{\isacharparenleft}{\kern0pt}SymExt{\isacharparenleft}{\kern0pt}G{\isacharparenright}{\kern0pt}{\isacharcomma}{\kern0pt}\ {\isasymphi}{\isacharcomma}{\kern0pt}\ {\isacharbrackleft}{\kern0pt}y{\isacharbrackright}{\kern0pt}\ {\isacharat}{\kern0pt}\ env{\isacharparenright}{\kern0pt}\ {\isacharbraceright}{\kern0pt}{\isachardoublequoteclose}\ \isanewline
\isanewline
\ \ \isacommand{have}\isamarkupfalse%
\ {\isachardoublequoteopen}val{\isacharparenleft}{\kern0pt}G{\isacharcomma}{\kern0pt}\ X{\isacharparenright}{\kern0pt}\ {\isacharequal}{\kern0pt}\ Y{\isachardoublequoteclose}\ \isanewline
\ \ \isacommand{proof}\isamarkupfalse%
\ {\isacharparenleft}{\kern0pt}rule\ equality{\isacharunderscore}{\kern0pt}iffI{\isacharparenright}{\kern0pt}\isanewline
\ \ \ \ \isacommand{fix}\isamarkupfalse%
\ y\isanewline
\isanewline
\ \ \ \ \isacommand{have}\isamarkupfalse%
\ {\isachardoublequoteopen}y\ {\isasymin}\ val{\isacharparenleft}{\kern0pt}G{\isacharcomma}{\kern0pt}\ X{\isacharparenright}{\kern0pt}\ {\isasymlongleftrightarrow}\ {\isacharparenleft}{\kern0pt}{\isasymexists}y{\isacharprime}{\kern0pt}\ {\isasymin}\ domain{\isacharparenleft}{\kern0pt}x{\isacharprime}{\kern0pt}{\isacharparenright}{\kern0pt}{\isachardot}{\kern0pt}\ val{\isacharparenleft}{\kern0pt}G{\isacharcomma}{\kern0pt}\ y{\isacharprime}{\kern0pt}{\isacharparenright}{\kern0pt}\ {\isacharequal}{\kern0pt}\ y\ {\isasymand}\ {\isacharparenleft}{\kern0pt}{\isasymexists}p\ {\isasymin}\ G{\isachardot}{\kern0pt}\ {\isacharless}{\kern0pt}y{\isacharprime}{\kern0pt}{\isacharcomma}{\kern0pt}\ p{\isachargreater}{\kern0pt}\ {\isasymin}\ X{\isacharparenright}{\kern0pt}{\isacharparenright}{\kern0pt}{\isachardoublequoteclose}\isanewline
\ \ \ \ \ \ \isacommand{apply}\isamarkupfalse%
{\isacharparenleft}{\kern0pt}subst\ def{\isacharunderscore}{\kern0pt}val{\isacharcomma}{\kern0pt}\ rule\ iffI{\isacharparenright}{\kern0pt}\isanewline
\ \ \ \ \ \ \isacommand{using}\isamarkupfalse%
\ X{\isacharunderscore}{\kern0pt}def\isanewline
\ \ \ \ \ \ \ \isacommand{apply}\isamarkupfalse%
\ force\isanewline
\ \ \ \ \ \ \isacommand{apply}\isamarkupfalse%
\ clarsimp\isanewline
\ \ \ \ \ \ \isacommand{apply}\isamarkupfalse%
{\isacharparenleft}{\kern0pt}rename{\isacharunderscore}{\kern0pt}tac\ y{\isacharprime}{\kern0pt}\ p\ q{\isacharcomma}{\kern0pt}\ rule{\isacharunderscore}{\kern0pt}tac\ x{\isacharequal}{\kern0pt}y{\isacharprime}{\kern0pt}\ \isakeyword{in}\ bexI{\isacharcomma}{\kern0pt}\ rule\ conjI{\isacharcomma}{\kern0pt}\ simp{\isacharparenright}{\kern0pt}\isanewline
\ \ \ \ \ \ \ \isacommand{apply}\isamarkupfalse%
{\isacharparenleft}{\kern0pt}rename{\isacharunderscore}{\kern0pt}tac\ y{\isacharprime}{\kern0pt}\ p\ q{\isacharcomma}{\kern0pt}\ rule{\isacharunderscore}{\kern0pt}tac\ x{\isacharequal}{\kern0pt}q\ \isakeyword{in}\ bexI{\isacharcomma}{\kern0pt}\ simp{\isacharparenright}{\kern0pt}\isanewline
\ \ \ \ \ \ \isacommand{using}\isamarkupfalse%
\ M{\isacharunderscore}{\kern0pt}genericD\ generic\ \isanewline
\ \ \ \ \ \ \isacommand{by}\isamarkupfalse%
\ auto\isanewline
\ \ \ \ \isacommand{also}\isamarkupfalse%
\ \isacommand{have}\isamarkupfalse%
\ {\isachardoublequoteopen}{\isachardot}{\kern0pt}{\isachardot}{\kern0pt}{\isachardot}{\kern0pt}\ {\isasymlongleftrightarrow}\ {\isacharparenleft}{\kern0pt}{\isasymexists}y{\isacharprime}{\kern0pt}\ {\isasymin}\ domain{\isacharparenleft}{\kern0pt}x{\isacharprime}{\kern0pt}{\isacharparenright}{\kern0pt}{\isachardot}{\kern0pt}\ val{\isacharparenleft}{\kern0pt}G{\isacharcomma}{\kern0pt}\ y{\isacharprime}{\kern0pt}{\isacharparenright}{\kern0pt}\ {\isacharequal}{\kern0pt}\ y\ {\isasymand}\ {\isacharparenleft}{\kern0pt}{\isasymexists}p\ {\isasymin}\ G{\isachardot}{\kern0pt}\ p\ {\isasymtturnstile}HS\ And{\isacharparenleft}{\kern0pt}Member{\isacharparenleft}{\kern0pt}{\isadigit{0}}{\isacharcomma}{\kern0pt}\ succ{\isacharparenleft}{\kern0pt}length{\isacharparenleft}{\kern0pt}env{\isacharprime}{\kern0pt}{\isacharparenright}{\kern0pt}{\isacharparenright}{\kern0pt}{\isacharparenright}{\kern0pt}{\isacharcomma}{\kern0pt}\ {\isasymphi}{\isacharparenright}{\kern0pt}\ {\isacharbrackleft}{\kern0pt}y{\isacharprime}{\kern0pt}{\isacharbrackright}{\kern0pt}\ {\isacharat}{\kern0pt}\ env{\isacharprime}{\kern0pt}\ {\isacharat}{\kern0pt}\ {\isacharbrackleft}{\kern0pt}x{\isacharprime}{\kern0pt}{\isacharbrackright}{\kern0pt}{\isacharparenright}{\kern0pt}{\isacharparenright}{\kern0pt}{\isachardoublequoteclose}\ \isanewline
\ \ \ \ \ \ \isacommand{unfolding}\isamarkupfalse%
\ X{\isacharunderscore}{\kern0pt}def\isanewline
\ \ \ \ \ \ \isacommand{using}\isamarkupfalse%
\ M{\isacharunderscore}{\kern0pt}genericD\ generic\ \isanewline
\ \ \ \ \ \ \isacommand{by}\isamarkupfalse%
\ auto\isanewline
\ \ \ \ \isacommand{also}\isamarkupfalse%
\ \isacommand{have}\isamarkupfalse%
\ {\isachardoublequoteopen}{\isachardot}{\kern0pt}{\isachardot}{\kern0pt}{\isachardot}{\kern0pt}\ {\isasymlongleftrightarrow}\ {\isacharparenleft}{\kern0pt}{\isasymexists}y{\isacharprime}{\kern0pt}\ {\isasymin}\ domain{\isacharparenleft}{\kern0pt}x{\isacharprime}{\kern0pt}{\isacharparenright}{\kern0pt}{\isachardot}{\kern0pt}\ val{\isacharparenleft}{\kern0pt}G{\isacharcomma}{\kern0pt}\ y{\isacharprime}{\kern0pt}{\isacharparenright}{\kern0pt}\ {\isacharequal}{\kern0pt}\ y\ {\isasymand}\ sats{\isacharparenleft}{\kern0pt}SymExt{\isacharparenleft}{\kern0pt}G{\isacharparenright}{\kern0pt}{\isacharcomma}{\kern0pt}\ And{\isacharparenleft}{\kern0pt}Member{\isacharparenleft}{\kern0pt}{\isadigit{0}}{\isacharcomma}{\kern0pt}\ succ{\isacharparenleft}{\kern0pt}length{\isacharparenleft}{\kern0pt}env{\isacharprime}{\kern0pt}{\isacharparenright}{\kern0pt}{\isacharparenright}{\kern0pt}{\isacharparenright}{\kern0pt}{\isacharcomma}{\kern0pt}\ {\isasymphi}{\isacharparenright}{\kern0pt}{\isacharcomma}{\kern0pt}\ map{\isacharparenleft}{\kern0pt}val{\isacharparenleft}{\kern0pt}G{\isacharparenright}{\kern0pt}{\isacharcomma}{\kern0pt}\ {\isacharbrackleft}{\kern0pt}y{\isacharprime}{\kern0pt}{\isacharbrackright}{\kern0pt}\ {\isacharat}{\kern0pt}\ env{\isacharprime}{\kern0pt}\ {\isacharat}{\kern0pt}\ {\isacharbrackleft}{\kern0pt}x{\isacharprime}{\kern0pt}{\isacharbrackright}{\kern0pt}{\isacharparenright}{\kern0pt}{\isacharparenright}{\kern0pt}{\isacharparenright}{\kern0pt}{\isachardoublequoteclose}\isanewline
\ \ \ \ \ \ \isacommand{apply}\isamarkupfalse%
{\isacharparenleft}{\kern0pt}rule\ bex{\isacharunderscore}{\kern0pt}iff{\isacharcomma}{\kern0pt}\ rule\ iff{\isacharunderscore}{\kern0pt}conjI{\isadigit{2}}{\isacharcomma}{\kern0pt}\ simp{\isacharparenright}{\kern0pt}\ \isanewline
\ \ \ \ \ \ \isacommand{apply}\isamarkupfalse%
{\isacharparenleft}{\kern0pt}rule\ HS{\isacharunderscore}{\kern0pt}truth{\isacharunderscore}{\kern0pt}lemma{\isacharparenright}{\kern0pt}\isanewline
\ \ \ \ \ \ \isacommand{using}\isamarkupfalse%
\ env{\isacharprime}{\kern0pt}H\ assms\ generic\ x{\isacharprime}{\kern0pt}H\ HS{\isacharunderscore}{\kern0pt}iff\isanewline
\ \ \ \ \ \ \ \ \ \isacommand{apply}\isamarkupfalse%
\ auto{\isacharbrackleft}{\kern0pt}{\isadigit{2}}{\isacharbrackright}{\kern0pt}\ \isanewline
\ \ \ \ \ \ \isacommand{using}\isamarkupfalse%
\ env{\isacharprime}{\kern0pt}H\ assms\ generic\ x{\isacharprime}{\kern0pt}H\ \isanewline
\ \ \ \ \ \ \ \isacommand{apply}\isamarkupfalse%
\ simp\isanewline
\ \ \ \ \ \ \ \isacommand{apply}\isamarkupfalse%
{\isacharparenleft}{\kern0pt}rule\ conjI{\isacharparenright}{\kern0pt}\isanewline
\ \ \ \ \ \ \isacommand{using}\isamarkupfalse%
\ HS{\isacharunderscore}{\kern0pt}iff\ \isanewline
\ \ \ \ \ \ \ \ \isacommand{apply}\isamarkupfalse%
\ blast\isanewline
\ \ \ \ \ \ \ \isacommand{apply}\isamarkupfalse%
{\isacharparenleft}{\kern0pt}rule\ app{\isacharunderscore}{\kern0pt}type{\isacharparenright}{\kern0pt}\isanewline
\ \ \ \ \ \ \ \ \isacommand{apply}\isamarkupfalse%
\ auto{\isacharbrackleft}{\kern0pt}{\isadigit{2}}{\isacharbrackright}{\kern0pt}\isanewline
\ \ \ \ \ \ \ \isacommand{apply}\isamarkupfalse%
\ simp\isanewline
\ \ \ \ \ \ \isacommand{apply}\isamarkupfalse%
{\isacharparenleft}{\kern0pt}rule\ Un{\isacharunderscore}{\kern0pt}least{\isacharunderscore}{\kern0pt}lt{\isacharparenright}{\kern0pt}{\isacharplus}{\kern0pt}\isanewline
\ \ \ \ \ \ \isacommand{using}\isamarkupfalse%
\ env{\isacharprime}{\kern0pt}H\ x{\isacharprime}{\kern0pt}H\ length{\isacharunderscore}{\kern0pt}app\ \isanewline
\ \ \ \ \ \ \ \ \isacommand{apply}\isamarkupfalse%
\ auto{\isacharbrackleft}{\kern0pt}{\isadigit{2}}{\isacharbrackright}{\kern0pt}\isanewline
\ \ \ \ \ \ \isacommand{apply}\isamarkupfalse%
{\isacharparenleft}{\kern0pt}rule{\isacharunderscore}{\kern0pt}tac\ j{\isacharequal}{\kern0pt}{\isachardoublequoteopen}succ{\isacharparenleft}{\kern0pt}length{\isacharparenleft}{\kern0pt}env{\isacharparenright}{\kern0pt}{\isacharparenright}{\kern0pt}{\isachardoublequoteclose}\ \isakeyword{in}\ le{\isacharunderscore}{\kern0pt}trans{\isacharparenright}{\kern0pt}\isanewline
\ \ \ \ \ \ \isacommand{using}\isamarkupfalse%
\ assms\ env{\isacharprime}{\kern0pt}H\ x{\isacharprime}{\kern0pt}H\ \isanewline
\ \ \ \ \ \ \isacommand{by}\isamarkupfalse%
\ auto\isanewline
\ \ \ \ \isacommand{also}\isamarkupfalse%
\ \isacommand{have}\isamarkupfalse%
\ {\isachardoublequoteopen}{\isachardot}{\kern0pt}{\isachardot}{\kern0pt}{\isachardot}{\kern0pt}\ {\isasymlongleftrightarrow}\ {\isacharparenleft}{\kern0pt}{\isasymexists}y{\isacharprime}{\kern0pt}\ {\isasymin}\ domain{\isacharparenleft}{\kern0pt}x{\isacharprime}{\kern0pt}{\isacharparenright}{\kern0pt}{\isachardot}{\kern0pt}\ val{\isacharparenleft}{\kern0pt}G{\isacharcomma}{\kern0pt}\ y{\isacharprime}{\kern0pt}{\isacharparenright}{\kern0pt}\ {\isacharequal}{\kern0pt}\ y\ {\isasymand}\ y\ {\isasymin}\ x\ {\isasymand}\ sats{\isacharparenleft}{\kern0pt}SymExt{\isacharparenleft}{\kern0pt}G{\isacharparenright}{\kern0pt}{\isacharcomma}{\kern0pt}\ {\isasymphi}{\isacharcomma}{\kern0pt}\ {\isacharbrackleft}{\kern0pt}y{\isacharbrackright}{\kern0pt}\ {\isacharat}{\kern0pt}\ map{\isacharparenleft}{\kern0pt}val{\isacharparenleft}{\kern0pt}G{\isacharparenright}{\kern0pt}{\isacharcomma}{\kern0pt}\ env{\isacharprime}{\kern0pt}\ {\isacharat}{\kern0pt}\ {\isacharbrackleft}{\kern0pt}x{\isacharprime}{\kern0pt}{\isacharbrackright}{\kern0pt}{\isacharparenright}{\kern0pt}{\isacharparenright}{\kern0pt}{\isacharparenright}{\kern0pt}{\isachardoublequoteclose}\isanewline
\ \ \ \ \ \ \isacommand{apply}\isamarkupfalse%
{\isacharparenleft}{\kern0pt}rule\ bex{\isacharunderscore}{\kern0pt}iff{\isacharcomma}{\kern0pt}\ rule\ iff{\isacharunderscore}{\kern0pt}conjI{\isadigit{2}}{\isacharcomma}{\kern0pt}\ simp{\isacharparenright}{\kern0pt}\isanewline
\ \ \ \ \ \ \isacommand{apply}\isamarkupfalse%
{\isacharparenleft}{\kern0pt}rename{\isacharunderscore}{\kern0pt}tac\ y{\isacharprime}{\kern0pt}{\isacharcomma}{\kern0pt}\ subgoal{\isacharunderscore}{\kern0pt}tac\ {\isachardoublequoteopen}map{\isacharparenleft}{\kern0pt}val{\isacharparenleft}{\kern0pt}G{\isacharparenright}{\kern0pt}{\isacharcomma}{\kern0pt}\ {\isacharbrackleft}{\kern0pt}y{\isacharprime}{\kern0pt}{\isacharbrackright}{\kern0pt}\ {\isacharat}{\kern0pt}\ env{\isacharprime}{\kern0pt}\ {\isacharat}{\kern0pt}\ {\isacharbrackleft}{\kern0pt}x{\isacharprime}{\kern0pt}{\isacharbrackright}{\kern0pt}{\isacharparenright}{\kern0pt}\ {\isasymin}\ list{\isacharparenleft}{\kern0pt}SymExt{\isacharparenleft}{\kern0pt}G{\isacharparenright}{\kern0pt}{\isacharparenright}{\kern0pt}{\isachardoublequoteclose}{\isacharparenright}{\kern0pt}\ \isanewline
\ \ \ \ \ \ \ \isacommand{apply}\isamarkupfalse%
\ simp\isanewline
\ \ \ \ \ \ \ \isacommand{apply}\isamarkupfalse%
\ {\isacharparenleft}{\kern0pt}rule\ iff{\isacharunderscore}{\kern0pt}conjI{\isadigit{2}}{\isacharparenright}{\kern0pt}\isanewline
\ \ \ \ \ \ \isacommand{using}\isamarkupfalse%
\ env{\isacharprime}{\kern0pt}H\ \isanewline
\ \ \ \ \ \ \ \ \isacommand{apply}\isamarkupfalse%
\ simp\isanewline
\ \ \ \ \ \ \ \ \isacommand{apply}\isamarkupfalse%
{\isacharparenleft}{\kern0pt}subst\ nth{\isacharunderscore}{\kern0pt}map{\isacharparenright}{\kern0pt}\isanewline
\ \ \ \ \ \ \isacommand{using}\isamarkupfalse%
\ env{\isacharprime}{\kern0pt}H\ x{\isacharprime}{\kern0pt}H\ \isanewline
\ \ \ \ \ \ \ \ \ \ \ \isacommand{apply}\isamarkupfalse%
\ auto{\isacharbrackleft}{\kern0pt}{\isadigit{3}}{\isacharbrackright}{\kern0pt}\isanewline
\ \ \ \ \ \ \ \ \isacommand{apply}\isamarkupfalse%
{\isacharparenleft}{\kern0pt}subst\ nth{\isacharunderscore}{\kern0pt}append{\isacharparenright}{\kern0pt}\isanewline
\ \ \ \ \ \ \isacommand{using}\isamarkupfalse%
\ env{\isacharprime}{\kern0pt}H\ x{\isacharprime}{\kern0pt}H\isanewline
\ \ \ \ \ \ \ \ \ \ \isacommand{apply}\isamarkupfalse%
\ auto{\isacharbrackleft}{\kern0pt}{\isadigit{4}}{\isacharbrackright}{\kern0pt}\isanewline
\ \ \ \ \ \ \isacommand{apply}\isamarkupfalse%
{\isacharparenleft}{\kern0pt}rule{\isacharunderscore}{\kern0pt}tac\ A{\isacharequal}{\kern0pt}HS\ \isakeyword{in}\ map{\isacharunderscore}{\kern0pt}type{\isacharparenright}{\kern0pt}\isanewline
\ \ \ \ \ \ \ \isacommand{apply}\isamarkupfalse%
\ simp\isanewline
\ \ \ \ \ \ \ \isacommand{apply}\isamarkupfalse%
{\isacharparenleft}{\kern0pt}rule\ conjI{\isacharparenright}{\kern0pt}\isanewline
\ \ \ \ \ \ \isacommand{using}\isamarkupfalse%
\ HS{\isacharunderscore}{\kern0pt}iff\ x{\isacharprime}{\kern0pt}H\ env{\isacharprime}{\kern0pt}H\isanewline
\ \ \ \ \ \ \ \ \isacommand{apply}\isamarkupfalse%
\ blast\isanewline
\ \ \ \ \ \ \ \isacommand{apply}\isamarkupfalse%
{\isacharparenleft}{\kern0pt}rule\ app{\isacharunderscore}{\kern0pt}type{\isacharparenright}{\kern0pt}\isanewline
\ \ \ \ \ \ \isacommand{using}\isamarkupfalse%
\ env{\isacharprime}{\kern0pt}H\ x{\isacharprime}{\kern0pt}H\ SymExt{\isacharunderscore}{\kern0pt}def\ \isanewline
\ \ \ \ \ \ \isacommand{by}\isamarkupfalse%
\ auto\isanewline
\ \ \ \ \isacommand{also}\isamarkupfalse%
\ \isacommand{have}\isamarkupfalse%
\ {\isachardoublequoteopen}{\isachardot}{\kern0pt}{\isachardot}{\kern0pt}{\isachardot}{\kern0pt}\ {\isasymlongleftrightarrow}\ {\isacharparenleft}{\kern0pt}{\isasymexists}y{\isacharprime}{\kern0pt}\ {\isasymin}\ domain{\isacharparenleft}{\kern0pt}x{\isacharprime}{\kern0pt}{\isacharparenright}{\kern0pt}{\isachardot}{\kern0pt}\ val{\isacharparenleft}{\kern0pt}G{\isacharcomma}{\kern0pt}\ y{\isacharprime}{\kern0pt}{\isacharparenright}{\kern0pt}\ {\isacharequal}{\kern0pt}\ y\ {\isasymand}\ y\ {\isasymin}\ x\ {\isasymand}\ sats{\isacharparenleft}{\kern0pt}SymExt{\isacharparenleft}{\kern0pt}G{\isacharparenright}{\kern0pt}{\isacharcomma}{\kern0pt}\ {\isasymphi}{\isacharcomma}{\kern0pt}\ {\isacharbrackleft}{\kern0pt}y{\isacharbrackright}{\kern0pt}\ {\isacharat}{\kern0pt}\ env\ {\isacharat}{\kern0pt}\ {\isacharbrackleft}{\kern0pt}x{\isacharbrackright}{\kern0pt}{\isacharparenright}{\kern0pt}{\isacharparenright}{\kern0pt}{\isachardoublequoteclose}\isanewline
\ \ \ \ \ \ \isacommand{apply}\isamarkupfalse%
{\isacharparenleft}{\kern0pt}rule\ bex{\isacharunderscore}{\kern0pt}iff{\isacharparenright}{\kern0pt}\isanewline
\ \ \ \ \ \ \isacommand{apply}\isamarkupfalse%
{\isacharparenleft}{\kern0pt}rule\ iff{\isacharunderscore}{\kern0pt}conjI{\isadigit{2}}{\isacharcomma}{\kern0pt}\ simp{\isacharparenright}{\kern0pt}{\isacharplus}{\kern0pt}\isanewline
\ \ \ \ \ \ \isacommand{apply}\isamarkupfalse%
{\isacharparenleft}{\kern0pt}subst\ map{\isacharunderscore}{\kern0pt}app{\isacharunderscore}{\kern0pt}distrib{\isacharbrackleft}{\kern0pt}\isakeyword{where}\ A{\isacharequal}{\kern0pt}HS{\isacharbrackright}{\kern0pt}{\isacharparenright}{\kern0pt}\isanewline
\ \ \ \ \ \ \isacommand{using}\isamarkupfalse%
\ env{\isacharprime}{\kern0pt}H\ x{\isacharprime}{\kern0pt}H\ \isanewline
\ \ \ \ \ \ \isacommand{by}\isamarkupfalse%
\ auto\isanewline
\ \ \ \ \isacommand{also}\isamarkupfalse%
\ \isacommand{have}\isamarkupfalse%
\ {\isachardoublequoteopen}{\isachardot}{\kern0pt}{\isachardot}{\kern0pt}{\isachardot}{\kern0pt}\ {\isasymlongleftrightarrow}\ y\ {\isasymin}\ x\ {\isasymand}\ sats{\isacharparenleft}{\kern0pt}SymExt{\isacharparenleft}{\kern0pt}G{\isacharparenright}{\kern0pt}{\isacharcomma}{\kern0pt}\ {\isasymphi}{\isacharcomma}{\kern0pt}\ {\isacharparenleft}{\kern0pt}{\isacharbrackleft}{\kern0pt}y{\isacharbrackright}{\kern0pt}\ {\isacharat}{\kern0pt}\ env{\isacharparenright}{\kern0pt}\ {\isacharat}{\kern0pt}\ {\isacharbrackleft}{\kern0pt}x{\isacharbrackright}{\kern0pt}{\isacharparenright}{\kern0pt}{\isachardoublequoteclose}\isanewline
\ \ \ \ \ \ \isacommand{apply}\isamarkupfalse%
{\isacharparenleft}{\kern0pt}rule\ iffI{\isacharcomma}{\kern0pt}\ force{\isacharcomma}{\kern0pt}\ simp{\isacharparenright}{\kern0pt}\isanewline
\ \ \ \ \ \ \isacommand{apply}\isamarkupfalse%
{\isacharparenleft}{\kern0pt}rule{\isacharunderscore}{\kern0pt}tac\ P{\isacharequal}{\kern0pt}{\isachardoublequoteopen}y\ {\isasymin}\ val{\isacharparenleft}{\kern0pt}G{\isacharcomma}{\kern0pt}\ x{\isacharprime}{\kern0pt}{\isacharparenright}{\kern0pt}{\isachardoublequoteclose}\ \isakeyword{in}\ mp{\isacharparenright}{\kern0pt}\ \isanewline
\ \ \ \ \ \ \ \isacommand{apply}\isamarkupfalse%
{\isacharparenleft}{\kern0pt}subst\ def{\isacharunderscore}{\kern0pt}val{\isacharcomma}{\kern0pt}\ force{\isacharparenright}{\kern0pt}\isanewline
\ \ \ \ \ \ \isacommand{using}\isamarkupfalse%
\ x{\isacharprime}{\kern0pt}H\ \isanewline
\ \ \ \ \ \ \isacommand{by}\isamarkupfalse%
\ auto\isanewline
\ \ \ \ \isacommand{also}\isamarkupfalse%
\ \isacommand{have}\isamarkupfalse%
\ {\isachardoublequoteopen}{\isachardot}{\kern0pt}{\isachardot}{\kern0pt}{\isachardot}{\kern0pt}\ {\isasymlongleftrightarrow}\ y\ {\isasymin}\ x\ {\isasymand}\ sats{\isacharparenleft}{\kern0pt}SymExt{\isacharparenleft}{\kern0pt}G{\isacharparenright}{\kern0pt}{\isacharcomma}{\kern0pt}\ {\isasymphi}{\isacharcomma}{\kern0pt}\ {\isacharbrackleft}{\kern0pt}y{\isacharbrackright}{\kern0pt}\ {\isacharat}{\kern0pt}\ env{\isacharparenright}{\kern0pt}{\isachardoublequoteclose}\isanewline
\ \ \ \ \ \ \isacommand{apply}\isamarkupfalse%
{\isacharparenleft}{\kern0pt}rule\ iff{\isacharunderscore}{\kern0pt}conjI{\isadigit{2}}{\isacharcomma}{\kern0pt}\ simp{\isacharparenright}{\kern0pt}\ \isanewline
\ \ \ \ \ \ \isacommand{apply}\isamarkupfalse%
{\isacharparenleft}{\kern0pt}rule\ arity{\isacharunderscore}{\kern0pt}sats{\isacharunderscore}{\kern0pt}iff{\isacharparenright}{\kern0pt}\isanewline
\ \ \ \ \ \ \isacommand{using}\isamarkupfalse%
\ assms\ SymExt{\isacharunderscore}{\kern0pt}trans\isanewline
\ \ \ \ \ \ \isacommand{by}\isamarkupfalse%
\ auto\isanewline
\ \ \ \ \isacommand{also}\isamarkupfalse%
\ \isacommand{have}\isamarkupfalse%
\ {\isachardoublequoteopen}{\isachardot}{\kern0pt}{\isachardot}{\kern0pt}{\isachardot}{\kern0pt}\ {\isasymlongleftrightarrow}\ y\ {\isasymin}\ Y{\isachardoublequoteclose}\ \isanewline
\ \ \ \ \ \ \isacommand{unfolding}\isamarkupfalse%
\ Y{\isacharunderscore}{\kern0pt}def\ \isanewline
\ \ \ \ \ \ \isacommand{by}\isamarkupfalse%
\ auto\isanewline
\ \ \ \ \isacommand{finally}\isamarkupfalse%
\ \isacommand{show}\isamarkupfalse%
\ {\isachardoublequoteopen}y\ {\isasymin}\ val{\isacharparenleft}{\kern0pt}G{\isacharcomma}{\kern0pt}\ X{\isacharparenright}{\kern0pt}\ {\isasymlongleftrightarrow}\ y\ {\isasymin}\ Y\ {\isachardoublequoteclose}\ \isacommand{by}\isamarkupfalse%
\ auto\isanewline
\ \ \isacommand{qed}\isamarkupfalse%
\isanewline
\isanewline
\ \ \isacommand{then}\isamarkupfalse%
\ \isacommand{have}\isamarkupfalse%
\ {\isachardoublequoteopen}Y\ {\isasymin}\ SymExt{\isacharparenleft}{\kern0pt}G{\isacharparenright}{\kern0pt}{\isachardoublequoteclose}\ \isanewline
\ \ \ \ \isacommand{using}\isamarkupfalse%
\ {\isacartoucheopen}X\ {\isasymin}\ HS{\isacartoucheclose}\ SymExt{\isacharunderscore}{\kern0pt}def\isanewline
\ \ \ \ \isacommand{by}\isamarkupfalse%
\ auto\isanewline
\ \ \isacommand{then}\isamarkupfalse%
\ \isacommand{show}\isamarkupfalse%
\ {\isacharquery}{\kern0pt}thesis\ \isanewline
\ \ \ \ \isacommand{unfolding}\isamarkupfalse%
\ Y{\isacharunderscore}{\kern0pt}def\isanewline
\ \ \ \ \isacommand{by}\isamarkupfalse%
\ auto\isanewline
\isacommand{qed}\isamarkupfalse%
%
\endisatagproof
{\isafoldproof}%
%
\isadelimproof
\isanewline
%
\endisadelimproof
\isanewline
\isacommand{end}\isamarkupfalse%
\isanewline
%
\isadelimtheory
%
\endisadelimtheory
%
\isatagtheory
\isacommand{end}\isamarkupfalse%
%
\endisatagtheory
{\isafoldtheory}%
%
\isadelimtheory
%
\endisadelimtheory
%
\end{isabellebody}%
\endinput
%:%file=~/source/repos/ZF-notAC/code/SymExt_Separation.thy%:%
%:%10=1%:%
%:%11=1%:%
%:%12=2%:%
%:%13=3%:%
%:%14=4%:%
%:%15=5%:%
%:%16=6%:%
%:%21=6%:%
%:%24=7%:%
%:%25=8%:%
%:%26=8%:%
%:%27=9%:%
%:%28=10%:%
%:%29=11%:%
%:%30=12%:%
%:%31=12%:%
%:%32=13%:%
%:%33=14%:%
%:%34=14%:%
%:%35=15%:%
%:%36=16%:%
%:%37=17%:%
%:%40=18%:%
%:%41=19%:%
%:%45=19%:%
%:%46=19%:%
%:%47=20%:%
%:%48=20%:%
%:%49=21%:%
%:%50=21%:%
%:%51=22%:%
%:%52=22%:%
%:%53=23%:%
%:%54=23%:%
%:%55=23%:%
%:%56=23%:%
%:%57=24%:%
%:%58=24%:%
%:%59=25%:%
%:%60=25%:%
%:%61=26%:%
%:%62=26%:%
%:%63=26%:%
%:%64=27%:%
%:%65=27%:%
%:%66=28%:%
%:%67=28%:%
%:%68=29%:%
%:%69=30%:%
%:%70=30%:%
%:%71=31%:%
%:%72=31%:%
%:%73=32%:%
%:%74=32%:%
%:%75=33%:%
%:%76=33%:%
%:%77=34%:%
%:%78=34%:%
%:%79=34%:%
%:%80=34%:%
%:%81=35%:%
%:%82=35%:%
%:%83=35%:%
%:%84=35%:%
%:%85=35%:%
%:%86=36%:%
%:%87=36%:%
%:%88=36%:%
%:%89=36%:%
%:%90=36%:%
%:%91=37%:%
%:%92=37%:%
%:%93=38%:%
%:%94=38%:%
%:%95=39%:%
%:%96=40%:%
%:%97=40%:%
%:%98=40%:%
%:%99=40%:%
%:%100=41%:%
%:%101=41%:%
%:%102=41%:%
%:%103=41%:%
%:%104=42%:%
%:%105=43%:%
%:%106=43%:%
%:%107=43%:%
%:%108=44%:%
%:%109=44%:%
%:%110=44%:%
%:%111=45%:%
%:%112=45%:%
%:%113=46%:%
%:%114=46%:%
%:%115=47%:%
%:%116=47%:%
%:%117=48%:%
%:%118=48%:%
%:%119=48%:%
%:%120=48%:%
%:%121=49%:%
%:%122=50%:%
%:%123=50%:%
%:%124=50%:%
%:%125=50%:%
%:%126=51%:%
%:%127=51%:%
%:%128=51%:%
%:%129=52%:%
%:%130=52%:%
%:%131=53%:%
%:%132=53%:%
%:%133=54%:%
%:%134=54%:%
%:%135=55%:%
%:%136=56%:%
%:%137=56%:%
%:%138=57%:%
%:%139=57%:%
%:%140=58%:%
%:%141=58%:%
%:%142=58%:%
%:%143=59%:%
%:%144=59%:%
%:%145=59%:%
%:%146=59%:%
%:%147=59%:%
%:%148=60%:%
%:%149=60%:%
%:%150=60%:%
%:%151=60%:%
%:%152=61%:%
%:%153=61%:%
%:%154=61%:%
%:%155=61%:%
%:%156=61%:%
%:%157=62%:%
%:%158=62%:%
%:%159=63%:%
%:%160=63%:%
%:%161=63%:%
%:%162=63%:%
%:%163=63%:%
%:%164=64%:%
%:%165=64%:%
%:%166=65%:%
%:%167=65%:%
%:%168=66%:%
%:%174=66%:%
%:%177=67%:%
%:%178=68%:%
%:%179=68%:%
%:%180=69%:%
%:%182=71%:%
%:%183=72%:%
%:%184=73%:%
%:%185=73%:%
%:%188=74%:%
%:%192=74%:%
%:%193=74%:%
%:%194=75%:%
%:%195=75%:%
%:%196=76%:%
%:%197=76%:%
%:%198=77%:%
%:%199=77%:%
%:%200=78%:%
%:%201=78%:%
%:%202=79%:%
%:%203=79%:%
%:%208=79%:%
%:%211=80%:%
%:%212=81%:%
%:%213=81%:%
%:%214=82%:%
%:%215=83%:%
%:%216=84%:%
%:%219=85%:%
%:%220=86%:%
%:%224=86%:%
%:%225=86%:%
%:%226=87%:%
%:%227=87%:%
%:%228=88%:%
%:%229=88%:%
%:%230=89%:%
%:%231=89%:%
%:%232=90%:%
%:%233=90%:%
%:%234=91%:%
%:%235=91%:%
%:%236=92%:%
%:%237=92%:%
%:%238=93%:%
%:%239=93%:%
%:%240=94%:%
%:%241=94%:%
%:%242=95%:%
%:%243=95%:%
%:%244=96%:%
%:%245=96%:%
%:%246=97%:%
%:%247=97%:%
%:%248=98%:%
%:%249=98%:%
%:%250=99%:%
%:%251=99%:%
%:%252=100%:%
%:%253=100%:%
%:%258=100%:%
%:%261=101%:%
%:%262=102%:%
%:%263=102%:%
%:%264=103%:%
%:%265=104%:%
%:%266=105%:%
%:%269=106%:%
%:%273=106%:%
%:%274=106%:%
%:%275=107%:%
%:%276=107%:%
%:%277=108%:%
%:%278=108%:%
%:%279=109%:%
%:%280=109%:%
%:%285=109%:%
%:%288=110%:%
%:%289=111%:%
%:%290=111%:%
%:%291=112%:%
%:%292=113%:%
%:%293=114%:%
%:%294=114%:%
%:%295=115%:%
%:%296=116%:%
%:%297=117%:%
%:%300=118%:%
%:%301=119%:%
%:%305=119%:%
%:%306=119%:%
%:%307=120%:%
%:%308=120%:%
%:%309=121%:%
%:%310=121%:%
%:%311=122%:%
%:%312=122%:%
%:%313=123%:%
%:%314=123%:%
%:%315=124%:%
%:%316=124%:%
%:%317=125%:%
%:%318=125%:%
%:%319=126%:%
%:%320=126%:%
%:%321=127%:%
%:%322=127%:%
%:%323=128%:%
%:%324=128%:%
%:%325=129%:%
%:%326=129%:%
%:%327=130%:%
%:%328=130%:%
%:%329=131%:%
%:%330=131%:%
%:%335=131%:%
%:%338=132%:%
%:%339=133%:%
%:%340=133%:%
%:%341=134%:%
%:%342=135%:%
%:%343=136%:%
%:%350=137%:%
%:%351=137%:%
%:%352=138%:%
%:%353=138%:%
%:%354=139%:%
%:%355=140%:%
%:%356=140%:%
%:%357=141%:%
%:%358=141%:%
%:%359=142%:%
%:%360=142%:%
%:%361=143%:%
%:%362=143%:%
%:%363=144%:%
%:%364=144%:%
%:%365=145%:%
%:%366=146%:%
%:%367=146%:%
%:%368=147%:%
%:%369=147%:%
%:%370=148%:%
%:%371=148%:%
%:%372=149%:%
%:%373=149%:%
%:%374=150%:%
%:%375=150%:%
%:%376=151%:%
%:%377=151%:%
%:%378=152%:%
%:%379=152%:%
%:%380=153%:%
%:%381=153%:%
%:%382=154%:%
%:%383=154%:%
%:%384=155%:%
%:%385=155%:%
%:%386=156%:%
%:%387=156%:%
%:%388=157%:%
%:%389=157%:%
%:%390=158%:%
%:%391=158%:%
%:%392=159%:%
%:%393=159%:%
%:%394=160%:%
%:%395=160%:%
%:%396=161%:%
%:%397=161%:%
%:%398=162%:%
%:%399=162%:%
%:%400=163%:%
%:%401=163%:%
%:%402=164%:%
%:%403=164%:%
%:%404=165%:%
%:%405=165%:%
%:%406=166%:%
%:%407=166%:%
%:%408=167%:%
%:%409=167%:%
%:%410=168%:%
%:%411=168%:%
%:%412=169%:%
%:%413=169%:%
%:%414=170%:%
%:%415=170%:%
%:%416=171%:%
%:%417=171%:%
%:%418=172%:%
%:%419=173%:%
%:%420=173%:%
%:%421=174%:%
%:%422=174%:%
%:%423=175%:%
%:%424=175%:%
%:%425=176%:%
%:%426=176%:%
%:%427=177%:%
%:%428=177%:%
%:%429=178%:%
%:%430=178%:%
%:%431=179%:%
%:%432=179%:%
%:%433=180%:%
%:%434=180%:%
%:%435=181%:%
%:%436=181%:%
%:%437=182%:%
%:%438=182%:%
%:%439=182%:%
%:%440=183%:%
%:%441=183%:%
%:%442=184%:%
%:%443=184%:%
%:%444=184%:%
%:%445=184%:%
%:%446=184%:%
%:%447=185%:%
%:%448=185%:%
%:%449=186%:%
%:%450=187%:%
%:%451=187%:%
%:%452=188%:%
%:%453=188%:%
%:%454=189%:%
%:%455=189%:%
%:%456=190%:%
%:%457=190%:%
%:%458=191%:%
%:%459=191%:%
%:%460=192%:%
%:%461=193%:%
%:%462=193%:%
%:%463=194%:%
%:%464=194%:%
%:%465=195%:%
%:%466=195%:%
%:%467=196%:%
%:%468=196%:%
%:%469=197%:%
%:%470=197%:%
%:%471=198%:%
%:%472=198%:%
%:%473=199%:%
%:%474=199%:%
%:%475=199%:%
%:%476=199%:%
%:%477=200%:%
%:%478=200%:%
%:%479=200%:%
%:%480=200%:%
%:%481=200%:%
%:%482=200%:%
%:%483=201%:%
%:%484=201%:%
%:%485=201%:%
%:%486=201%:%
%:%487=201%:%
%:%488=202%:%
%:%489=203%:%
%:%490=203%:%
%:%491=204%:%
%:%492=204%:%
%:%493=205%:%
%:%494=205%:%
%:%495=206%:%
%:%496=206%:%
%:%497=207%:%
%:%498=207%:%
%:%499=207%:%
%:%500=207%:%
%:%501=207%:%
%:%502=208%:%
%:%503=209%:%
%:%504=209%:%
%:%505=210%:%
%:%506=210%:%
%:%507=211%:%
%:%508=211%:%
%:%509=212%:%
%:%510=212%:%
%:%511=213%:%
%:%512=213%:%
%:%513=213%:%
%:%514=214%:%
%:%515=214%:%
%:%516=215%:%
%:%517=215%:%
%:%518=215%:%
%:%519=216%:%
%:%520=216%:%
%:%521=216%:%
%:%522=217%:%
%:%523=217%:%
%:%524=218%:%
%:%525=218%:%
%:%526=219%:%
%:%527=219%:%
%:%528=220%:%
%:%529=220%:%
%:%530=220%:%
%:%531=221%:%
%:%532=221%:%
%:%533=222%:%
%:%534=222%:%
%:%535=223%:%
%:%536=223%:%
%:%537=223%:%
%:%538=223%:%
%:%539=223%:%
%:%540=224%:%
%:%541=224%:%
%:%542=224%:%
%:%543=224%:%
%:%544=224%:%
%:%545=225%:%
%:%546=226%:%
%:%547=226%:%
%:%548=227%:%
%:%549=227%:%
%:%550=228%:%
%:%551=228%:%
%:%552=229%:%
%:%553=229%:%
%:%554=230%:%
%:%555=231%:%
%:%556=231%:%
%:%557=232%:%
%:%558=232%:%
%:%559=233%:%
%:%560=233%:%
%:%561=234%:%
%:%562=234%:%
%:%563=235%:%
%:%564=235%:%
%:%565=235%:%
%:%566=236%:%
%:%567=236%:%
%:%568=237%:%
%:%569=237%:%
%:%570=237%:%
%:%571=238%:%
%:%572=238%:%
%:%573=239%:%
%:%574=239%:%
%:%575=240%:%
%:%576=240%:%
%:%577=241%:%
%:%578=241%:%
%:%579=242%:%
%:%580=242%:%
%:%581=242%:%
%:%582=243%:%
%:%583=243%:%
%:%584=244%:%
%:%585=244%:%
%:%586=245%:%
%:%587=245%:%
%:%588=245%:%
%:%589=246%:%
%:%590=246%:%
%:%591=247%:%
%:%592=247%:%
%:%593=248%:%
%:%594=248%:%
%:%595=249%:%
%:%596=249%:%
%:%597=249%:%
%:%598=249%:%
%:%599=249%:%
%:%600=250%:%
%:%601=250%:%
%:%602=251%:%
%:%603=251%:%
%:%604=252%:%
%:%605=252%:%
%:%606=253%:%
%:%607=253%:%
%:%608=253%:%
%:%609=254%:%
%:%610=254%:%
%:%611=254%:%
%:%612=255%:%
%:%613=255%:%
%:%614=255%:%
%:%615=256%:%
%:%616=256%:%
%:%617=256%:%
%:%618=257%:%
%:%619=257%:%
%:%620=257%:%
%:%621=257%:%
%:%622=257%:%
%:%623=258%:%
%:%624=259%:%
%:%625=259%:%
%:%626=260%:%
%:%627=260%:%
%:%628=261%:%
%:%629=261%:%
%:%630=262%:%
%:%631=262%:%
%:%632=263%:%
%:%633=263%:%
%:%634=263%:%
%:%635=263%:%
%:%636=263%:%
%:%637=264%:%
%:%638=264%:%
%:%639=264%:%
%:%640=264%:%
%:%641=264%:%
%:%642=264%:%
%:%643=265%:%
%:%644=265%:%
%:%645=265%:%
%:%646=266%:%
%:%647=266%:%
%:%648=267%:%
%:%649=267%:%
%:%650=268%:%
%:%651=268%:%
%:%652=269%:%
%:%653=269%:%
%:%654=269%:%
%:%655=270%:%
%:%656=270%:%
%:%657=271%:%
%:%658=271%:%
%:%659=272%:%
%:%660=272%:%
%:%661=273%:%
%:%662=273%:%
%:%663=274%:%
%:%664=275%:%
%:%665=275%:%
%:%666=275%:%
%:%667=275%:%
%:%668=276%:%
%:%669=276%:%
%:%670=276%:%
%:%671=276%:%
%:%672=276%:%
%:%673=277%:%
%:%674=277%:%
%:%675=277%:%
%:%676=277%:%
%:%677=277%:%
%:%678=278%:%
%:%679=279%:%
%:%680=279%:%
%:%681=279%:%
%:%682=279%:%
%:%683=279%:%
%:%684=280%:%
%:%685=280%:%
%:%686=280%:%
%:%687=280%:%
%:%688=280%:%
%:%689=281%:%
%:%690=281%:%
%:%691=281%:%
%:%692=281%:%
%:%693=282%:%
%:%694=282%:%
%:%695=282%:%
%:%696=283%:%
%:%697=283%:%
%:%698=284%:%
%:%699=284%:%
%:%700=285%:%
%:%701=285%:%
%:%702=286%:%
%:%703=286%:%
%:%704=286%:%
%:%705=287%:%
%:%706=287%:%
%:%707=288%:%
%:%708=288%:%
%:%709=289%:%
%:%710=289%:%
%:%711=290%:%
%:%712=290%:%
%:%713=290%:%
%:%714=291%:%
%:%715=291%:%
%:%716=292%:%
%:%717=292%:%
%:%718=293%:%
%:%719=293%:%
%:%720=293%:%
%:%721=293%:%
%:%722=293%:%
%:%723=294%:%
%:%724=294%:%
%:%725=295%:%
%:%726=295%:%
%:%727=295%:%
%:%728=296%:%
%:%729=296%:%
%:%730=297%:%
%:%731=297%:%
%:%732=298%:%
%:%733=298%:%
%:%734=299%:%
%:%735=299%:%
%:%736=300%:%
%:%737=301%:%
%:%738=301%:%
%:%739=302%:%
%:%740=302%:%
%:%741=303%:%
%:%742=303%:%
%:%743=304%:%
%:%744=304%:%
%:%745=305%:%
%:%746=306%:%
%:%747=306%:%
%:%748=306%:%
%:%749=307%:%
%:%750=307%:%
%:%751=308%:%
%:%752=308%:%
%:%753=309%:%
%:%754=310%:%
%:%755=310%:%
%:%756=310%:%
%:%757=311%:%
%:%758=311%:%
%:%759=312%:%
%:%760=312%:%
%:%761=313%:%
%:%762=314%:%
%:%763=314%:%
%:%764=314%:%
%:%765=315%:%
%:%766=315%:%
%:%767=316%:%
%:%768=316%:%
%:%769=317%:%
%:%775=317%:%
%:%778=318%:%
%:%779=319%:%
%:%780=319%:%
%:%781=320%:%
%:%782=321%:%
%:%783=322%:%
%:%790=323%:%
%:%791=323%:%
%:%792=324%:%
%:%793=324%:%
%:%794=325%:%
%:%795=325%:%
%:%796=326%:%
%:%797=326%:%
%:%798=327%:%
%:%799=327%:%
%:%800=328%:%
%:%801=328%:%
%:%802=328%:%
%:%803=328%:%
%:%804=329%:%
%:%805=330%:%
%:%806=330%:%
%:%807=331%:%
%:%808=331%:%
%:%809=332%:%
%:%810=332%:%
%:%811=333%:%
%:%812=333%:%
%:%813=334%:%
%:%814=335%:%
%:%815=335%:%
%:%816=335%:%
%:%817=335%:%
%:%818=336%:%
%:%819=337%:%
%:%820=337%:%
%:%821=337%:%
%:%822=337%:%
%:%823=338%:%
%:%824=339%:%
%:%825=339%:%
%:%826=340%:%
%:%827=340%:%
%:%828=341%:%
%:%829=341%:%
%:%830=342%:%
%:%831=342%:%
%:%832=343%:%
%:%833=343%:%
%:%834=344%:%
%:%835=344%:%
%:%836=345%:%
%:%837=345%:%
%:%838=346%:%
%:%839=346%:%
%:%840=347%:%
%:%841=347%:%
%:%842=348%:%
%:%843=348%:%
%:%844=349%:%
%:%845=349%:%
%:%846=350%:%
%:%847=350%:%
%:%848=351%:%
%:%849=351%:%
%:%850=352%:%
%:%851=352%:%
%:%852=352%:%
%:%853=352%:%
%:%854=352%:%
%:%855=353%:%
%:%856=354%:%
%:%857=354%:%
%:%858=355%:%
%:%859=356%:%
%:%860=356%:%
%:%861=357%:%
%:%862=357%:%
%:%863=358%:%
%:%864=358%:%
%:%865=359%:%
%:%866=360%:%
%:%867=360%:%
%:%868=361%:%
%:%869=361%:%
%:%870=362%:%
%:%871=362%:%
%:%872=363%:%
%:%873=363%:%
%:%874=364%:%
%:%875=364%:%
%:%876=365%:%
%:%877=365%:%
%:%878=366%:%
%:%879=366%:%
%:%880=367%:%
%:%881=367%:%
%:%882=368%:%
%:%883=368%:%
%:%884=369%:%
%:%885=369%:%
%:%886=369%:%
%:%887=370%:%
%:%888=370%:%
%:%889=371%:%
%:%890=371%:%
%:%891=372%:%
%:%892=372%:%
%:%893=373%:%
%:%894=373%:%
%:%895=373%:%
%:%896=374%:%
%:%897=374%:%
%:%898=375%:%
%:%899=375%:%
%:%900=376%:%
%:%901=376%:%
%:%902=377%:%
%:%903=377%:%
%:%904=378%:%
%:%905=378%:%
%:%906=379%:%
%:%907=379%:%
%:%908=380%:%
%:%909=380%:%
%:%910=381%:%
%:%911=381%:%
%:%912=382%:%
%:%913=382%:%
%:%914=383%:%
%:%915=383%:%
%:%916=384%:%
%:%917=384%:%
%:%918=385%:%
%:%919=385%:%
%:%920=386%:%
%:%921=386%:%
%:%922=387%:%
%:%923=387%:%
%:%924=388%:%
%:%925=388%:%
%:%926=389%:%
%:%927=389%:%
%:%928=390%:%
%:%929=390%:%
%:%930=391%:%
%:%931=391%:%
%:%932=392%:%
%:%933=392%:%
%:%934=392%:%
%:%935=393%:%
%:%936=393%:%
%:%937=394%:%
%:%938=394%:%
%:%939=395%:%
%:%940=395%:%
%:%941=396%:%
%:%942=396%:%
%:%943=397%:%
%:%944=397%:%
%:%945=398%:%
%:%946=398%:%
%:%947=399%:%
%:%948=399%:%
%:%949=400%:%
%:%950=400%:%
%:%951=401%:%
%:%952=401%:%
%:%953=402%:%
%:%954=402%:%
%:%955=403%:%
%:%956=403%:%
%:%957=404%:%
%:%958=404%:%
%:%959=405%:%
%:%960=405%:%
%:%961=406%:%
%:%962=406%:%
%:%963=407%:%
%:%964=407%:%
%:%965=408%:%
%:%966=408%:%
%:%967=409%:%
%:%968=409%:%
%:%969=410%:%
%:%970=410%:%
%:%971=411%:%
%:%972=411%:%
%:%973=412%:%
%:%974=412%:%
%:%975=413%:%
%:%976=413%:%
%:%977=413%:%
%:%978=414%:%
%:%979=414%:%
%:%980=415%:%
%:%981=415%:%
%:%982=416%:%
%:%983=416%:%
%:%984=417%:%
%:%985=417%:%
%:%986=418%:%
%:%987=418%:%
%:%988=419%:%
%:%989=419%:%
%:%990=419%:%
%:%991=420%:%
%:%992=420%:%
%:%993=421%:%
%:%994=421%:%
%:%995=422%:%
%:%996=422%:%
%:%997=423%:%
%:%998=423%:%
%:%999=424%:%
%:%1000=424%:%
%:%1001=425%:%
%:%1002=425%:%
%:%1003=425%:%
%:%1004=426%:%
%:%1005=426%:%
%:%1006=427%:%
%:%1007=427%:%
%:%1008=428%:%
%:%1009=428%:%
%:%1010=429%:%
%:%1011=429%:%
%:%1012=430%:%
%:%1013=430%:%
%:%1014=430%:%
%:%1015=431%:%
%:%1016=431%:%
%:%1017=432%:%
%:%1018=432%:%
%:%1019=433%:%
%:%1020=433%:%
%:%1021=433%:%
%:%1022=433%:%
%:%1023=434%:%
%:%1024=434%:%
%:%1025=435%:%
%:%1026=436%:%
%:%1027=436%:%
%:%1028=436%:%
%:%1029=437%:%
%:%1030=437%:%
%:%1031=438%:%
%:%1032=438%:%
%:%1033=439%:%
%:%1034=439%:%
%:%1035=439%:%
%:%1036=440%:%
%:%1037=440%:%
%:%1038=441%:%
%:%1039=441%:%
%:%1040=442%:%
%:%1046=442%:%
%:%1049=443%:%
%:%1050=444%:%
%:%1051=444%:%
%:%1058=445%:%

%
\begin{isabellebody}%
\setisabellecontext{SymExt{\isacharunderscore}{\kern0pt}Separation{\isacharunderscore}{\kern0pt}Base}%
%
\isadelimtheory
%
\endisadelimtheory
%
\isatagtheory
\isacommand{theory}\isamarkupfalse%
\ SymExt{\isacharunderscore}{\kern0pt}Separation{\isacharunderscore}{\kern0pt}Base\isanewline
\ \ \isakeyword{imports}\ \isanewline
\ \ \ \ {\isachardoublequoteopen}Forcing{\isacharslash}{\kern0pt}Forcing{\isacharunderscore}{\kern0pt}Main{\isachardoublequoteclose}\ \isanewline
\ \ \ \ SymExt{\isacharunderscore}{\kern0pt}Definition\isanewline
\isakeyword{begin}%
\endisatagtheory
{\isafoldtheory}%
%
\isadelimtheory
\ \isanewline
%
\endisadelimtheory
\isanewline
\isacommand{context}\isamarkupfalse%
\ M{\isacharunderscore}{\kern0pt}symmetric{\isacharunderscore}{\kern0pt}system{\isacharunderscore}{\kern0pt}G{\isacharunderscore}{\kern0pt}generic\isanewline
\isakeyword{begin}\isanewline
\isanewline
\isanewline
\isacommand{definition}\isamarkupfalse%
\ cartprod{\isacharprime}{\kern0pt}\ \isakeyword{where}\ {\isachardoublequoteopen}cartprod{\isacharprime}{\kern0pt}{\isacharparenleft}{\kern0pt}A{\isacharcomma}{\kern0pt}\ B{\isacharcomma}{\kern0pt}\ Z{\isacharparenright}{\kern0pt}\ {\isasymequiv}\ {\isasymforall}u\ {\isasymin}\ M{\isachardot}{\kern0pt}\ u\ {\isasymin}\ Z\ {\isasymlongleftrightarrow}\ {\isacharparenleft}{\kern0pt}{\isasymexists}x\ {\isasymin}\ M{\isachardot}{\kern0pt}\ x\ {\isasymin}\ A\ {\isasymand}\ {\isacharparenleft}{\kern0pt}{\isasymexists}y\ {\isasymin}\ M{\isachardot}{\kern0pt}\ y\ {\isasymin}\ B\ {\isasymand}\ pair{\isacharparenleft}{\kern0pt}{\isacharhash}{\kern0pt}{\isacharhash}{\kern0pt}M{\isacharcomma}{\kern0pt}\ x{\isacharcomma}{\kern0pt}\ y{\isacharcomma}{\kern0pt}\ u{\isacharparenright}{\kern0pt}{\isacharparenright}{\kern0pt}{\isacharparenright}{\kern0pt}{\isachardoublequoteclose}\ \isanewline
\isacommand{definition}\isamarkupfalse%
\ powerset{\isacharprime}{\kern0pt}\ \isakeyword{where}\ {\isachardoublequoteopen}powerset{\isacharprime}{\kern0pt}{\isacharparenleft}{\kern0pt}A{\isacharcomma}{\kern0pt}\ Z{\isacharparenright}{\kern0pt}\ {\isasymequiv}\ {\isasymforall}x\ {\isasymin}\ M{\isachardot}{\kern0pt}\ x\ {\isasymin}\ Z\ {\isasymlongleftrightarrow}\ subset{\isacharparenleft}{\kern0pt}{\isacharhash}{\kern0pt}{\isacharhash}{\kern0pt}M{\isacharcomma}{\kern0pt}\ x{\isacharcomma}{\kern0pt}\ A{\isacharparenright}{\kern0pt}{\isachardoublequoteclose}\isanewline
\isacommand{definition}\isamarkupfalse%
\ is{\isacharunderscore}{\kern0pt}singleton{\isacharprime}{\kern0pt}\ \isakeyword{where}\ {\isachardoublequoteopen}is{\isacharunderscore}{\kern0pt}singleton{\isacharprime}{\kern0pt}{\isacharparenleft}{\kern0pt}x{\isacharcomma}{\kern0pt}\ Z{\isacharparenright}{\kern0pt}\ {\isasymequiv}\ {\isasymforall}a\ {\isasymin}\ M{\isachardot}{\kern0pt}\ a\ {\isasymin}\ Z\ {\isasymlongleftrightarrow}\ a\ {\isacharequal}{\kern0pt}\ x{\isachardoublequoteclose}\isanewline
\isacommand{lemma}\isamarkupfalse%
\ is{\isacharunderscore}{\kern0pt}singleton{\isacharprime}{\kern0pt}{\isacharunderscore}{\kern0pt}iff\ {\isacharcolon}{\kern0pt}\ {\isachardoublequoteopen}x\ {\isasymin}\ M\ {\isasymLongrightarrow}\ Z\ {\isasymin}\ M\ {\isasymLongrightarrow}\ is{\isacharunderscore}{\kern0pt}singleton{\isacharprime}{\kern0pt}{\isacharparenleft}{\kern0pt}x{\isacharcomma}{\kern0pt}\ Z{\isacharparenright}{\kern0pt}\ {\isasymlongleftrightarrow}\ is{\isacharunderscore}{\kern0pt}singleton{\isacharparenleft}{\kern0pt}{\isacharhash}{\kern0pt}{\isacharhash}{\kern0pt}M{\isacharcomma}{\kern0pt}\ x{\isacharcomma}{\kern0pt}\ Z{\isacharparenright}{\kern0pt}{\isachardoublequoteclose}\ \isanewline
%
\isadelimproof
\ \ %
\endisadelimproof
%
\isatagproof
\isacommand{unfolding}\isamarkupfalse%
\ is{\isacharunderscore}{\kern0pt}singleton{\isacharprime}{\kern0pt}{\isacharunderscore}{\kern0pt}def\ \isanewline
\ \ \isacommand{apply}\isamarkupfalse%
{\isacharparenleft}{\kern0pt}simp{\isacharcomma}{\kern0pt}\ rule\ iffI{\isacharcomma}{\kern0pt}\ rule\ equality{\isacharunderscore}{\kern0pt}iffI{\isacharcomma}{\kern0pt}\ rule\ iffI{\isacharparenright}{\kern0pt}\isanewline
\ \ \isacommand{using}\isamarkupfalse%
\ transM\ \isanewline
\ \ \isacommand{by}\isamarkupfalse%
\ auto%
\endisatagproof
{\isafoldproof}%
%
\isadelimproof
\isanewline
%
\endisadelimproof
\isanewline
\isacommand{definition}\isamarkupfalse%
\ Hsep{\isacharunderscore}{\kern0pt}base{\isacharunderscore}{\kern0pt}M\ \isakeyword{where}\ {\isachardoublequoteopen}Hsep{\isacharunderscore}{\kern0pt}base{\isacharunderscore}{\kern0pt}M{\isacharparenleft}{\kern0pt}xP{\isacharcomma}{\kern0pt}\ g{\isacharparenright}{\kern0pt}\ {\isasymequiv}\ {\isacharbraceleft}{\kern0pt}\ v\ {\isasymin}\ M{\isachardot}{\kern0pt}\ {\isasymexists}x{\isachardot}{\kern0pt}\ {\isasymexists}P{\isachardot}{\kern0pt}\ xP\ {\isacharequal}{\kern0pt}\ {\isacharless}{\kern0pt}x{\isacharcomma}{\kern0pt}\ P{\isachargreater}{\kern0pt}\ {\isasymand}\ v\ {\isasymin}\ Pow{\isacharparenleft}{\kern0pt}{\isacharparenleft}{\kern0pt}{\isasymUnion}{\isacharparenleft}{\kern0pt}g{\isacharbackquote}{\kern0pt}{\isacharbackquote}{\kern0pt}{\isacharparenleft}{\kern0pt}domain{\isacharparenleft}{\kern0pt}x{\isacharparenright}{\kern0pt}\ {\isasymtimes}\ {\isacharbraceleft}{\kern0pt}P{\isacharbraceright}{\kern0pt}{\isacharparenright}{\kern0pt}{\isacharparenright}{\kern0pt}{\isacharparenright}{\kern0pt}\ {\isasymtimes}\ P{\isacharparenright}{\kern0pt}\ {\isasyminter}\ M\ {\isacharbraceright}{\kern0pt}{\isachardoublequoteclose}\ \isanewline
\isacommand{definition}\isamarkupfalse%
\ Hsep{\isacharunderscore}{\kern0pt}base{\isacharunderscore}{\kern0pt}M{\isacharunderscore}{\kern0pt}cond\ \isakeyword{where}\ {\isachardoublequoteopen}Hsep{\isacharunderscore}{\kern0pt}base{\isacharunderscore}{\kern0pt}M{\isacharunderscore}{\kern0pt}cond{\isacharparenleft}{\kern0pt}v{\isacharcomma}{\kern0pt}\ xP{\isacharcomma}{\kern0pt}\ g{\isacharparenright}{\kern0pt}\ {\isasymequiv}\isanewline
\ \ \ {\isasymexists}x\ {\isasymin}\ M{\isachardot}{\kern0pt}\ {\isasymexists}P\ {\isasymin}\ M{\isachardot}{\kern0pt}\ {\isasymexists}domx\ {\isasymin}\ M{\isachardot}{\kern0pt}\ {\isasymexists}Ps\ {\isasymin}\ M{\isachardot}{\kern0pt}\ {\isasymexists}dPs\ {\isasymin}\ M{\isachardot}{\kern0pt}\ {\isasymexists}gi\ {\isasymin}\ M{\isachardot}{\kern0pt}\ {\isasymexists}gu\ {\isasymin}\ M{\isachardot}{\kern0pt}\ {\isasymexists}guP\ {\isasymin}\ M{\isachardot}{\kern0pt}\ {\isasymexists}PguP\ {\isasymin}\ M{\isachardot}{\kern0pt}\ \isanewline
\ \ pair{\isacharparenleft}{\kern0pt}{\isacharhash}{\kern0pt}{\isacharhash}{\kern0pt}M{\isacharcomma}{\kern0pt}\ x{\isacharcomma}{\kern0pt}\ P{\isacharcomma}{\kern0pt}\ xP{\isacharparenright}{\kern0pt}\ {\isasymand}\ is{\isacharunderscore}{\kern0pt}domain{\isacharparenleft}{\kern0pt}{\isacharhash}{\kern0pt}{\isacharhash}{\kern0pt}M{\isacharcomma}{\kern0pt}\ x{\isacharcomma}{\kern0pt}\ domx{\isacharparenright}{\kern0pt}\ {\isasymand}\ is{\isacharunderscore}{\kern0pt}singleton{\isacharprime}{\kern0pt}{\isacharparenleft}{\kern0pt}P{\isacharcomma}{\kern0pt}\ Ps{\isacharparenright}{\kern0pt}\ {\isasymand}\ cartprod{\isacharprime}{\kern0pt}{\isacharparenleft}{\kern0pt}domx{\isacharcomma}{\kern0pt}\ Ps{\isacharcomma}{\kern0pt}\ dPs{\isacharparenright}{\kern0pt}\ {\isasymand}\ image{\isacharparenleft}{\kern0pt}{\isacharhash}{\kern0pt}{\isacharhash}{\kern0pt}M{\isacharcomma}{\kern0pt}\ g{\isacharcomma}{\kern0pt}\ dPs{\isacharcomma}{\kern0pt}\ gi{\isacharparenright}{\kern0pt}\ {\isasymand}\ big{\isacharunderscore}{\kern0pt}union{\isacharparenleft}{\kern0pt}{\isacharhash}{\kern0pt}{\isacharhash}{\kern0pt}M{\isacharcomma}{\kern0pt}\ gi{\isacharcomma}{\kern0pt}\ gu{\isacharparenright}{\kern0pt}\ {\isasymand}\ cartprod{\isacharprime}{\kern0pt}{\isacharparenleft}{\kern0pt}gu{\isacharcomma}{\kern0pt}\ P{\isacharcomma}{\kern0pt}\ guP{\isacharparenright}{\kern0pt}\ {\isasymand}\ powerset{\isacharprime}{\kern0pt}{\isacharparenleft}{\kern0pt}guP{\isacharcomma}{\kern0pt}\ PguP{\isacharparenright}{\kern0pt}\ {\isasymand}\ v\ {\isasymin}\ PguP{\isachardoublequoteclose}\ \isanewline
\isanewline
\isacommand{lemma}\isamarkupfalse%
\ Hsep{\isacharunderscore}{\kern0pt}base{\isacharunderscore}{\kern0pt}M{\isacharunderscore}{\kern0pt}eq\ {\isacharcolon}{\kern0pt}\ \isanewline
\ \ {\isachardoublequoteopen}xP\ {\isasymin}\ M\ {\isasymLongrightarrow}\ g\ {\isasymin}\ M\ {\isasymLongrightarrow}\ Hsep{\isacharunderscore}{\kern0pt}base{\isacharunderscore}{\kern0pt}M{\isacharparenleft}{\kern0pt}xP{\isacharcomma}{\kern0pt}\ g{\isacharparenright}{\kern0pt}\ {\isacharequal}{\kern0pt}\ {\isacharbraceleft}{\kern0pt}\ v\ {\isasymin}\ M{\isachardot}{\kern0pt}\ Hsep{\isacharunderscore}{\kern0pt}base{\isacharunderscore}{\kern0pt}M{\isacharunderscore}{\kern0pt}cond{\isacharparenleft}{\kern0pt}v{\isacharcomma}{\kern0pt}\ xP{\isacharcomma}{\kern0pt}\ g{\isacharparenright}{\kern0pt}\ {\isacharbraceright}{\kern0pt}{\isachardoublequoteclose}\ \isanewline
%
\isadelimproof
\ \ %
\endisadelimproof
%
\isatagproof
\isacommand{unfolding}\isamarkupfalse%
\ Hsep{\isacharunderscore}{\kern0pt}base{\isacharunderscore}{\kern0pt}M{\isacharunderscore}{\kern0pt}def\ Hsep{\isacharunderscore}{\kern0pt}base{\isacharunderscore}{\kern0pt}M{\isacharunderscore}{\kern0pt}cond{\isacharunderscore}{\kern0pt}def\isanewline
\ \ \isacommand{apply}\isamarkupfalse%
{\isacharparenleft}{\kern0pt}rule\ iff{\isacharunderscore}{\kern0pt}eq{\isacharcomma}{\kern0pt}\ rule\ iff{\isacharunderscore}{\kern0pt}flip{\isacharparenright}{\kern0pt}\ \isanewline
\ \ \isacommand{apply}\isamarkupfalse%
{\isacharparenleft}{\kern0pt}rule{\isacharunderscore}{\kern0pt}tac\ Q{\isacharequal}{\kern0pt}{\isachardoublequoteopen}{\isasymexists}x\ {\isasymin}\ M{\isachardot}{\kern0pt}\ {\isasymexists}P\ {\isasymin}\ M{\isachardot}{\kern0pt}\ {\isasymexists}domx\ {\isasymin}\ M{\isachardot}{\kern0pt}\ {\isasymexists}Ps\ {\isasymin}\ M{\isachardot}{\kern0pt}\ {\isasymexists}dPs\ {\isasymin}\ M{\isachardot}{\kern0pt}\ {\isasymexists}gi\ {\isasymin}\ M{\isachardot}{\kern0pt}\ {\isasymexists}gu\ {\isasymin}\ M{\isachardot}{\kern0pt}\ {\isasymexists}guP\ {\isasymin}\ M{\isachardot}{\kern0pt}\ {\isasymexists}PguP\ {\isasymin}\ M{\isachardot}{\kern0pt}\ \isanewline
\ \ \ \ \ \ \ \ \ \ \ \ \ \ \ \ pair{\isacharparenleft}{\kern0pt}{\isacharhash}{\kern0pt}{\isacharhash}{\kern0pt}M{\isacharcomma}{\kern0pt}\ x{\isacharcomma}{\kern0pt}\ P{\isacharcomma}{\kern0pt}\ xP{\isacharparenright}{\kern0pt}\ {\isasymand}\ is{\isacharunderscore}{\kern0pt}domain{\isacharparenleft}{\kern0pt}{\isacharhash}{\kern0pt}{\isacharhash}{\kern0pt}M{\isacharcomma}{\kern0pt}\ x{\isacharcomma}{\kern0pt}\ domx{\isacharparenright}{\kern0pt}\ {\isasymand}\ is{\isacharunderscore}{\kern0pt}singleton{\isacharparenleft}{\kern0pt}{\isacharhash}{\kern0pt}{\isacharhash}{\kern0pt}M{\isacharcomma}{\kern0pt}\ P{\isacharcomma}{\kern0pt}\ Ps{\isacharparenright}{\kern0pt}\ {\isasymand}\ cartprod{\isacharparenleft}{\kern0pt}{\isacharhash}{\kern0pt}{\isacharhash}{\kern0pt}M{\isacharcomma}{\kern0pt}\ domx{\isacharcomma}{\kern0pt}\ Ps{\isacharcomma}{\kern0pt}\ dPs{\isacharparenright}{\kern0pt}\ {\isasymand}\ image{\isacharparenleft}{\kern0pt}{\isacharhash}{\kern0pt}{\isacharhash}{\kern0pt}M{\isacharcomma}{\kern0pt}\ g{\isacharcomma}{\kern0pt}\ dPs{\isacharcomma}{\kern0pt}\ gi{\isacharparenright}{\kern0pt}\ {\isasymand}\ big{\isacharunderscore}{\kern0pt}union{\isacharparenleft}{\kern0pt}{\isacharhash}{\kern0pt}{\isacharhash}{\kern0pt}M{\isacharcomma}{\kern0pt}\ gi{\isacharcomma}{\kern0pt}\ gu{\isacharparenright}{\kern0pt}\ {\isasymand}\ cartprod{\isacharparenleft}{\kern0pt}{\isacharhash}{\kern0pt}{\isacharhash}{\kern0pt}M{\isacharcomma}{\kern0pt}\ gu{\isacharcomma}{\kern0pt}\ P{\isacharcomma}{\kern0pt}\ guP{\isacharparenright}{\kern0pt}\ {\isasymand}\ powerset{\isacharparenleft}{\kern0pt}{\isacharhash}{\kern0pt}{\isacharhash}{\kern0pt}M{\isacharcomma}{\kern0pt}\ guP{\isacharcomma}{\kern0pt}\ PguP{\isacharparenright}{\kern0pt}\ {\isasymand}\ v\ {\isasymin}\ PguP{\isachardoublequoteclose}\ \isakeyword{in}\ iff{\isacharunderscore}{\kern0pt}trans{\isacharparenright}{\kern0pt}\isanewline
\ \ \isacommand{apply}\isamarkupfalse%
{\isacharparenleft}{\kern0pt}simp\ add{\isacharcolon}{\kern0pt}\ cartprod{\isacharunderscore}{\kern0pt}def\ cartprod{\isacharprime}{\kern0pt}{\isacharunderscore}{\kern0pt}def\ powerset{\isacharunderscore}{\kern0pt}def\ powerset{\isacharprime}{\kern0pt}{\isacharunderscore}{\kern0pt}def\ is{\isacharunderscore}{\kern0pt}singleton{\isacharprime}{\kern0pt}{\isacharunderscore}{\kern0pt}iff{\isacharparenright}{\kern0pt}\isanewline
\ \ \isacommand{apply}\isamarkupfalse%
\ simp\isanewline
\ \ \isacommand{apply}\isamarkupfalse%
{\isacharparenleft}{\kern0pt}rule\ iffI{\isacharcomma}{\kern0pt}\ clarsimp{\isacharparenright}{\kern0pt}\isanewline
\ \ \isacommand{apply}\isamarkupfalse%
\ clarsimp\ \isanewline
\ \ \isacommand{using}\isamarkupfalse%
\ pair{\isacharunderscore}{\kern0pt}in{\isacharunderscore}{\kern0pt}M{\isacharunderscore}{\kern0pt}iff\ \isanewline
\ \ \ \isacommand{apply}\isamarkupfalse%
\ simp\isanewline
\ \ \isacommand{using}\isamarkupfalse%
\ pair{\isacharunderscore}{\kern0pt}in{\isacharunderscore}{\kern0pt}M{\isacharunderscore}{\kern0pt}iff\ domain{\isacharunderscore}{\kern0pt}closed\ singleton{\isacharunderscore}{\kern0pt}in{\isacharunderscore}{\kern0pt}M{\isacharunderscore}{\kern0pt}iff\ cartprod{\isacharunderscore}{\kern0pt}closed\ image{\isacharunderscore}{\kern0pt}closed\ Union{\isacharunderscore}{\kern0pt}closed\isanewline
\ \ \isacommand{apply}\isamarkupfalse%
\ clarsimp\ \isanewline
\ \ \isacommand{apply}\isamarkupfalse%
{\isacharparenleft}{\kern0pt}rename{\isacharunderscore}{\kern0pt}tac\ v\ x\ P{\isacharcomma}{\kern0pt}\ rule{\isacharunderscore}{\kern0pt}tac\ b{\isacharequal}{\kern0pt}{\isachardoublequoteopen}{\isacharbraceleft}{\kern0pt}a\ {\isasymin}\ Pow{\isacharparenleft}{\kern0pt}{\isacharparenleft}{\kern0pt}{\isasymUnion}{\isacharparenleft}{\kern0pt}g\ {\isacharbackquote}{\kern0pt}{\isacharbackquote}{\kern0pt}\ {\isacharparenleft}{\kern0pt}domain{\isacharparenleft}{\kern0pt}x{\isacharparenright}{\kern0pt}\ {\isasymtimes}\ {\isacharbraceleft}{\kern0pt}P{\isacharbraceright}{\kern0pt}{\isacharparenright}{\kern0pt}{\isacharparenright}{\kern0pt}{\isacharparenright}{\kern0pt}\ {\isasymtimes}\ P{\isacharparenright}{\kern0pt}\ {\isachardot}{\kern0pt}\ a\ {\isasymin}\ M{\isacharbraceright}{\kern0pt}{\isachardoublequoteclose}\ \isakeyword{and}\ a{\isacharequal}{\kern0pt}{\isachardoublequoteopen}Pow{\isacharparenleft}{\kern0pt}{\isacharparenleft}{\kern0pt}{\isasymUnion}{\isacharparenleft}{\kern0pt}g\ {\isacharbackquote}{\kern0pt}{\isacharbackquote}{\kern0pt}\ {\isacharparenleft}{\kern0pt}domain{\isacharparenleft}{\kern0pt}x{\isacharparenright}{\kern0pt}\ {\isasymtimes}\ {\isacharbraceleft}{\kern0pt}P{\isacharbraceright}{\kern0pt}{\isacharparenright}{\kern0pt}{\isacharparenright}{\kern0pt}{\isacharparenright}{\kern0pt}\ {\isasymtimes}\ P{\isacharparenright}{\kern0pt}\ {\isasyminter}\ M{\isachardoublequoteclose}\ \isakeyword{in}\ ssubst{\isacharparenright}{\kern0pt}\isanewline
\ \ \ \isacommand{apply}\isamarkupfalse%
\ force\isanewline
\ \ \isacommand{apply}\isamarkupfalse%
{\isacharparenleft}{\kern0pt}rule\ M{\isacharunderscore}{\kern0pt}powerset{\isacharparenright}{\kern0pt}\isanewline
\ \ \isacommand{using}\isamarkupfalse%
\ pair{\isacharunderscore}{\kern0pt}in{\isacharunderscore}{\kern0pt}M{\isacharunderscore}{\kern0pt}iff\ domain{\isacharunderscore}{\kern0pt}closed\ singleton{\isacharunderscore}{\kern0pt}in{\isacharunderscore}{\kern0pt}M{\isacharunderscore}{\kern0pt}iff\ cartprod{\isacharunderscore}{\kern0pt}closed\ image{\isacharunderscore}{\kern0pt}closed\ Union{\isacharunderscore}{\kern0pt}closed\isanewline
\ \ \isacommand{by}\isamarkupfalse%
\ clarsimp%
\endisatagproof
{\isafoldproof}%
%
\isadelimproof
\ \isanewline
%
\endisadelimproof
\ \ \isanewline
\isacommand{schematic{\isacharunderscore}{\kern0pt}goal}\isamarkupfalse%
\ Hsep{\isacharunderscore}{\kern0pt}base{\isacharunderscore}{\kern0pt}M{\isacharunderscore}{\kern0pt}fm{\isacharunderscore}{\kern0pt}auto{\isacharcolon}{\kern0pt}\isanewline
\ \ \isakeyword{assumes}\isanewline
\ \ \ \ {\isachardoublequoteopen}s\ {\isasymin}\ M{\isachardoublequoteclose}\ {\isachardoublequoteopen}g\ {\isasymin}\ M{\isachardoublequoteclose}\ {\isachardoublequoteopen}xP\ {\isasymin}\ M{\isachardoublequoteclose}\isanewline
\ \ \ \ {\isachardoublequoteopen}nth{\isacharparenleft}{\kern0pt}{\isadigit{0}}{\isacharcomma}{\kern0pt}env{\isacharparenright}{\kern0pt}\ {\isacharequal}{\kern0pt}\ s{\isachardoublequoteclose}\ \ \ \ \isanewline
\ \ \ \ {\isachardoublequoteopen}nth{\isacharparenleft}{\kern0pt}{\isadigit{1}}{\isacharcomma}{\kern0pt}env{\isacharparenright}{\kern0pt}\ {\isacharequal}{\kern0pt}\ g{\isachardoublequoteclose}\ \ \ \ \isanewline
\ \ \ \ {\isachardoublequoteopen}nth{\isacharparenleft}{\kern0pt}{\isadigit{2}}{\isacharcomma}{\kern0pt}env{\isacharparenright}{\kern0pt}\ {\isacharequal}{\kern0pt}\ xP{\isachardoublequoteclose}\ \ \ \ \isanewline
\ \ \ \ {\isachardoublequoteopen}env\ {\isasymin}\ list{\isacharparenleft}{\kern0pt}M{\isacharparenright}{\kern0pt}{\isachardoublequoteclose}\ \isanewline
\ \isakeyword{shows}\ \isanewline
\ \ \ \ {\isachardoublequoteopen}s\ {\isacharequal}{\kern0pt}\ Hsep{\isacharunderscore}{\kern0pt}base{\isacharunderscore}{\kern0pt}M{\isacharparenleft}{\kern0pt}xP{\isacharcomma}{\kern0pt}\ g{\isacharparenright}{\kern0pt}\ {\isasymlongleftrightarrow}\ sats{\isacharparenleft}{\kern0pt}M{\isacharcomma}{\kern0pt}{\isacharquery}{\kern0pt}fm{\isacharparenleft}{\kern0pt}{\isadigit{0}}{\isacharcomma}{\kern0pt}{\isadigit{1}}{\isacharcomma}{\kern0pt}{\isadigit{2}}{\isacharparenright}{\kern0pt}{\isacharcomma}{\kern0pt}env{\isacharparenright}{\kern0pt}{\isachardoublequoteclose}\isanewline
%
\isadelimproof
\isanewline
\ \ %
\endisadelimproof
%
\isatagproof
\isacommand{apply}\isamarkupfalse%
{\isacharparenleft}{\kern0pt}rule{\isacharunderscore}{\kern0pt}tac\ Q{\isacharequal}{\kern0pt}{\isachardoublequoteopen}{\isasymforall}v\ {\isasymin}\ M{\isachardot}{\kern0pt}\ v\ {\isasymin}\ s\ {\isasymlongleftrightarrow}\ Hsep{\isacharunderscore}{\kern0pt}base{\isacharunderscore}{\kern0pt}M{\isacharunderscore}{\kern0pt}cond{\isacharparenleft}{\kern0pt}v{\isacharcomma}{\kern0pt}\ xP{\isacharcomma}{\kern0pt}\ g{\isacharparenright}{\kern0pt}{\isachardoublequoteclose}\ \isakeyword{in}\ iff{\isacharunderscore}{\kern0pt}trans{\isacharparenright}{\kern0pt}\ \isanewline
\ \ \ \isacommand{apply}\isamarkupfalse%
{\isacharparenleft}{\kern0pt}rule\ iffI{\isacharcomma}{\kern0pt}\ rule\ ballI{\isacharcomma}{\kern0pt}\ simp{\isacharparenright}{\kern0pt}\isanewline
\ \ \isacommand{using}\isamarkupfalse%
\ Hsep{\isacharunderscore}{\kern0pt}base{\isacharunderscore}{\kern0pt}M{\isacharunderscore}{\kern0pt}eq\ assms\ \isanewline
\ \ \ \ \isacommand{apply}\isamarkupfalse%
\ force\isanewline
\ \ \isacommand{apply}\isamarkupfalse%
{\isacharparenleft}{\kern0pt}rule\ equality{\isacharunderscore}{\kern0pt}iffI{\isacharcomma}{\kern0pt}\ rule\ iffI{\isacharparenright}{\kern0pt}\ \ \isanewline
\ \ \isacommand{using}\isamarkupfalse%
\ Hsep{\isacharunderscore}{\kern0pt}base{\isacharunderscore}{\kern0pt}M{\isacharunderscore}{\kern0pt}eq\ assms\ transM\isanewline
\ \ \ \ \isacommand{apply}\isamarkupfalse%
\ {\isacharparenleft}{\kern0pt}force{\isacharcomma}{\kern0pt}\ force{\isacharparenright}{\kern0pt}\isanewline
\ \ \isacommand{unfolding}\isamarkupfalse%
\ \ Hsep{\isacharunderscore}{\kern0pt}base{\isacharunderscore}{\kern0pt}M{\isacharunderscore}{\kern0pt}cond{\isacharunderscore}{\kern0pt}def\ cartprod{\isacharprime}{\kern0pt}{\isacharunderscore}{\kern0pt}def\ powerset{\isacharprime}{\kern0pt}{\isacharunderscore}{\kern0pt}def\ is{\isacharunderscore}{\kern0pt}singleton{\isacharprime}{\kern0pt}{\isacharunderscore}{\kern0pt}def\ subset{\isacharunderscore}{\kern0pt}def\isanewline
\ \ \isacommand{by}\isamarkupfalse%
\ {\isacharparenleft}{\kern0pt}insert\ assms\ {\isacharsemicolon}{\kern0pt}\ {\isacharparenleft}{\kern0pt}rule\ sep{\isacharunderscore}{\kern0pt}rules\ {\isacharbar}{\kern0pt}\ simp{\isacharparenright}{\kern0pt}{\isacharplus}{\kern0pt}{\isacharparenright}{\kern0pt}%
\endisatagproof
{\isafoldproof}%
%
\isadelimproof
\ \isanewline
%
\endisadelimproof
\isanewline
\isacommand{definition}\isamarkupfalse%
\ Hsep{\isacharunderscore}{\kern0pt}base{\isacharunderscore}{\kern0pt}M{\isacharunderscore}{\kern0pt}fm\ \isakeyword{where}\ \isanewline
\ \ {\isachardoublequoteopen}Hsep{\isacharunderscore}{\kern0pt}base{\isacharunderscore}{\kern0pt}M{\isacharunderscore}{\kern0pt}fm\ {\isasymequiv}\ Forall\isanewline
\ \ \ \ \ \ \ \ \ \ \ \ \ {\isacharparenleft}{\kern0pt}Iff{\isacharparenleft}{\kern0pt}Member{\isacharparenleft}{\kern0pt}{\isadigit{0}}{\isacharcomma}{\kern0pt}\ {\isadigit{1}}{\isacharparenright}{\kern0pt}{\isacharcomma}{\kern0pt}\isanewline
\ \ \ \ \ \ \ \ \ \ \ \ \ \ \ \ \ \ Exists\isanewline
\ \ \ \ \ \ \ \ \ \ \ \ \ \ \ \ \ \ \ {\isacharparenleft}{\kern0pt}Exists\isanewline
\ \ \ \ \ \ \ \ \ \ \ \ \ \ \ \ \ \ \ \ \ {\isacharparenleft}{\kern0pt}Exists\isanewline
\ \ \ \ \ \ \ \ \ \ \ \ \ \ \ \ \ \ \ \ \ \ \ {\isacharparenleft}{\kern0pt}Exists\isanewline
\ \ \ \ \ \ \ \ \ \ \ \ \ \ \ \ \ \ \ \ \ \ \ \ \ {\isacharparenleft}{\kern0pt}Exists\isanewline
\ \ \ \ \ \ \ \ \ \ \ \ \ \ \ \ \ \ \ \ \ \ \ \ \ \ \ {\isacharparenleft}{\kern0pt}Exists\isanewline
\ \ \ \ \ \ \ \ \ \ \ \ \ \ \ \ \ \ \ \ \ \ \ \ \ \ \ \ \ {\isacharparenleft}{\kern0pt}Exists\isanewline
\ \ \ \ \ \ \ \ \ \ \ \ \ \ \ \ \ \ \ \ \ \ \ \ \ \ \ \ \ \ \ {\isacharparenleft}{\kern0pt}Exists\isanewline
\ \ \ \ \ \ \ \ \ \ \ \ \ \ \ \ \ \ \ \ \ \ \ \ \ \ \ \ \ \ \ \ \ {\isacharparenleft}{\kern0pt}Exists\isanewline
\ \ \ \ \ \ \ \ \ \ \ \ \ \ \ \ \ \ \ \ \ \ \ \ \ \ \ \ \ \ \ \ \ \ \ {\isacharparenleft}{\kern0pt}And{\isacharparenleft}{\kern0pt}pair{\isacharunderscore}{\kern0pt}fm{\isacharparenleft}{\kern0pt}{\isadigit{8}}{\isacharcomma}{\kern0pt}\ {\isadigit{7}}{\isacharcomma}{\kern0pt}\ {\isadigit{1}}{\isadigit{2}}{\isacharparenright}{\kern0pt}{\isacharcomma}{\kern0pt}\isanewline
\ \ \ \ \ \ \ \ \ \ \ \ \ \ \ \ \ \ \ \ \ \ \ \ \ \ \ \ \ \ \ \ \ \ \ \ \ \ \ \ And{\isacharparenleft}{\kern0pt}domain{\isacharunderscore}{\kern0pt}fm{\isacharparenleft}{\kern0pt}{\isadigit{8}}{\isacharcomma}{\kern0pt}\ {\isadigit{6}}{\isacharparenright}{\kern0pt}{\isacharcomma}{\kern0pt}\isanewline
\ \ \ \ \ \ \ \ \ \ \ \ \ \ \ \ \ \ \ \ \ \ \ \ \ \ \ \ \ \ \ \ \ \ \ \ \ \ \ \ \ \ \ \ And{\isacharparenleft}{\kern0pt}Forall{\isacharparenleft}{\kern0pt}Iff{\isacharparenleft}{\kern0pt}Member{\isacharparenleft}{\kern0pt}{\isadigit{0}}{\isacharcomma}{\kern0pt}\ {\isadigit{6}}{\isacharparenright}{\kern0pt}{\isacharcomma}{\kern0pt}\ Equal{\isacharparenleft}{\kern0pt}{\isadigit{0}}{\isacharcomma}{\kern0pt}\ {\isadigit{8}}{\isacharparenright}{\kern0pt}{\isacharparenright}{\kern0pt}{\isacharparenright}{\kern0pt}{\isacharcomma}{\kern0pt}\isanewline
\ \ \ \ \ \ \ \ \ \ \ \ \ \ \ \ \ \ \ \ \ \ \ \ \ \ \ \ \ \ \ \ \ \ \ \ \ \ \ \ \ \ \ \ \ \ \ \ And{\isacharparenleft}{\kern0pt}Forall\isanewline
\ \ \ \ \ \ \ \ \ \ \ \ \ \ \ \ \ \ \ \ \ \ \ \ \ \ \ \ \ \ \ \ \ \ \ \ \ \ \ \ \ \ \ \ \ \ \ \ \ \ \ \ \ {\isacharparenleft}{\kern0pt}Iff{\isacharparenleft}{\kern0pt}Member{\isacharparenleft}{\kern0pt}{\isadigit{0}}{\isacharcomma}{\kern0pt}\ {\isadigit{5}}{\isacharparenright}{\kern0pt}{\isacharcomma}{\kern0pt}\isanewline
\ \ \ \ \ \ \ \ \ \ \ \ \ \ \ \ \ \ \ \ \ \ \ \ \ \ \ \ \ \ \ \ \ \ \ \ \ \ \ \ \ \ \ \ \ \ \ \ \ \ \ \ \ \ \ \ \ \ Exists{\isacharparenleft}{\kern0pt}And{\isacharparenleft}{\kern0pt}Member{\isacharparenleft}{\kern0pt}{\isadigit{0}}{\isacharcomma}{\kern0pt}\ {\isadigit{8}}{\isacharparenright}{\kern0pt}{\isacharcomma}{\kern0pt}\ Exists{\isacharparenleft}{\kern0pt}And{\isacharparenleft}{\kern0pt}Member{\isacharparenleft}{\kern0pt}{\isadigit{0}}{\isacharcomma}{\kern0pt}\ {\isadigit{8}}{\isacharparenright}{\kern0pt}{\isacharcomma}{\kern0pt}\ pair{\isacharunderscore}{\kern0pt}fm{\isacharparenleft}{\kern0pt}{\isadigit{1}}{\isacharcomma}{\kern0pt}\ {\isadigit{0}}{\isacharcomma}{\kern0pt}\ {\isadigit{2}}{\isacharparenright}{\kern0pt}{\isacharparenright}{\kern0pt}{\isacharparenright}{\kern0pt}{\isacharparenright}{\kern0pt}{\isacharparenright}{\kern0pt}{\isacharparenright}{\kern0pt}{\isacharparenright}{\kern0pt}{\isacharcomma}{\kern0pt}\isanewline
\ \ \ \ \ \ \ \ \ \ \ \ \ \ \ \ \ \ \ \ \ \ \ \ \ \ \ \ \ \ \ \ \ \ \ \ \ \ \ \ \ \ \ \ \ \ \ \ \ \ \ \ And{\isacharparenleft}{\kern0pt}image{\isacharunderscore}{\kern0pt}fm{\isacharparenleft}{\kern0pt}{\isadigit{1}}{\isadigit{1}}{\isacharcomma}{\kern0pt}\ {\isadigit{4}}{\isacharcomma}{\kern0pt}\ {\isadigit{3}}{\isacharparenright}{\kern0pt}{\isacharcomma}{\kern0pt}\isanewline
\ \ \ \ \ \ \ \ \ \ \ \ \ \ \ \ \ \ \ \ \ \ \ \ \ \ \ \ \ \ \ \ \ \ \ \ \ \ \ \ \ \ \ \ \ \ \ \ \ \ \ \ \ \ \ \ And{\isacharparenleft}{\kern0pt}big{\isacharunderscore}{\kern0pt}union{\isacharunderscore}{\kern0pt}fm{\isacharparenleft}{\kern0pt}{\isadigit{3}}{\isacharcomma}{\kern0pt}\ {\isadigit{2}}{\isacharparenright}{\kern0pt}{\isacharcomma}{\kern0pt}\isanewline
\ \ \ \ \ \ \ \ \ \ \ \ \ \ \ \ \ \ \ \ \ \ \ \ \ \ \ \ \ \ \ \ \ \ \ \ \ \ \ \ \ \ \ \ \ \ \ \ \ \ \ \ \ \ \ \ \ \ \ \ And{\isacharparenleft}{\kern0pt}Forall\isanewline
\ \ \ \ \ \ \ \ \ \ \ \ \ \ \ \ \ \ \ \ \ \ \ \ \ \ \ \ \ \ \ \ \ \ \ \ \ \ \ \ \ \ \ \ \ \ \ \ \ \ \ \ \ \ \ \ \ \ \ \ \ \ \ \ \ {\isacharparenleft}{\kern0pt}Iff{\isacharparenleft}{\kern0pt}Member{\isacharparenleft}{\kern0pt}{\isadigit{0}}{\isacharcomma}{\kern0pt}\ {\isadigit{2}}{\isacharparenright}{\kern0pt}{\isacharcomma}{\kern0pt}\isanewline
\ \ \ \ \ \ \ \ \ \ \ \ \ \ \ \ \ \ \ \ \ \ \ \ \ \ \ \ \ \ \ \ \ \ \ \ \ \ \ \ \ \ \ \ \ \ \ \ \ \ \ \ \ \ \ \ \ \ \ \ \ \ \ \ \ \ \ \ \ \ Exists\isanewline
\ \ \ \ \ \ \ \ \ \ \ \ \ \ \ \ \ \ \ \ \ \ \ \ \ \ \ \ \ \ \ \ \ \ \ \ \ \ \ \ \ \ \ \ \ \ \ \ \ \ \ \ \ \ \ \ \ \ \ \ \ \ \ \ \ \ \ \ \ \ \ {\isacharparenleft}{\kern0pt}And{\isacharparenleft}{\kern0pt}Member{\isacharparenleft}{\kern0pt}{\isadigit{0}}{\isacharcomma}{\kern0pt}\ {\isadigit{4}}{\isacharparenright}{\kern0pt}{\isacharcomma}{\kern0pt}\ Exists{\isacharparenleft}{\kern0pt}And{\isacharparenleft}{\kern0pt}Member{\isacharparenleft}{\kern0pt}{\isadigit{0}}{\isacharcomma}{\kern0pt}\ {\isadigit{1}}{\isadigit{0}}{\isacharparenright}{\kern0pt}{\isacharcomma}{\kern0pt}\ pair{\isacharunderscore}{\kern0pt}fm{\isacharparenleft}{\kern0pt}{\isadigit{1}}{\isacharcomma}{\kern0pt}\ {\isadigit{0}}{\isacharcomma}{\kern0pt}\ {\isadigit{2}}{\isacharparenright}{\kern0pt}{\isacharparenright}{\kern0pt}{\isacharparenright}{\kern0pt}{\isacharparenright}{\kern0pt}{\isacharparenright}{\kern0pt}{\isacharparenright}{\kern0pt}{\isacharparenright}{\kern0pt}{\isacharcomma}{\kern0pt}\isanewline
\ \ \ \ \ \ \ \ \ \ \ \ \ \ \ \ \ \ \ \ \ \ \ \ \ \ \ \ \ \ \ \ \ \ \ \ \ \ \ \ \ \ \ \ \ \ \ \ \ \ \ \ \ \ \ \ \ \ \ \ \ \ \ \ And{\isacharparenleft}{\kern0pt}Forall{\isacharparenleft}{\kern0pt}Iff{\isacharparenleft}{\kern0pt}Member{\isacharparenleft}{\kern0pt}{\isadigit{0}}{\isacharcomma}{\kern0pt}\ {\isadigit{1}}{\isacharparenright}{\kern0pt}{\isacharcomma}{\kern0pt}\ Forall{\isacharparenleft}{\kern0pt}Implies{\isacharparenleft}{\kern0pt}Member{\isacharparenleft}{\kern0pt}{\isadigit{0}}{\isacharcomma}{\kern0pt}\ {\isadigit{1}}{\isacharparenright}{\kern0pt}{\isacharcomma}{\kern0pt}\ Member{\isacharparenleft}{\kern0pt}{\isadigit{0}}{\isacharcomma}{\kern0pt}\ {\isadigit{3}}{\isacharparenright}{\kern0pt}{\isacharparenright}{\kern0pt}{\isacharparenright}{\kern0pt}{\isacharparenright}{\kern0pt}{\isacharparenright}{\kern0pt}{\isacharcomma}{\kern0pt}\isanewline
\ \ \ \ \ \ \ \ \ \ \ \ \ \ \ \ \ \ \ \ \ \ \ \ \ \ \ \ \ \ \ \ \ \ \ \ \ \ \ \ \ \ \ \ \ \ \ \ \ \ \ \ \ \ \ \ \ \ \ \ \ \ \ \ \ \ \ \ Member{\isacharparenleft}{\kern0pt}{\isadigit{9}}{\isacharcomma}{\kern0pt}\ {\isadigit{0}}{\isacharparenright}{\kern0pt}{\isacharparenright}{\kern0pt}{\isacharparenright}{\kern0pt}{\isacharparenright}{\kern0pt}{\isacharparenright}{\kern0pt}{\isacharparenright}{\kern0pt}{\isacharparenright}{\kern0pt}{\isacharparenright}{\kern0pt}{\isacharparenright}{\kern0pt}{\isacharparenright}{\kern0pt}{\isacharparenright}{\kern0pt}{\isacharparenright}{\kern0pt}{\isacharparenright}{\kern0pt}{\isacharparenright}{\kern0pt}{\isacharparenright}{\kern0pt}{\isacharparenright}{\kern0pt}{\isacharparenright}{\kern0pt}{\isacharparenright}{\kern0pt}{\isacharparenright}{\kern0pt}{\isacharparenright}{\kern0pt}\ \ {\isachardoublequoteclose}\ \isanewline
\isanewline
\isacommand{lemma}\isamarkupfalse%
\ Hsep{\isacharunderscore}{\kern0pt}base{\isacharunderscore}{\kern0pt}M{\isacharunderscore}{\kern0pt}in{\isacharunderscore}{\kern0pt}M\ {\isacharcolon}{\kern0pt}\ {\isachardoublequoteopen}xP\ {\isasymin}\ M\ {\isasymLongrightarrow}\ g\ {\isasymin}\ M\ {\isasymLongrightarrow}\ function{\isacharparenleft}{\kern0pt}g{\isacharparenright}{\kern0pt}\ {\isasymLongrightarrow}\ Hsep{\isacharunderscore}{\kern0pt}base{\isacharunderscore}{\kern0pt}M{\isacharparenleft}{\kern0pt}xP{\isacharcomma}{\kern0pt}\ g{\isacharparenright}{\kern0pt}\ {\isasymin}\ M{\isachardoublequoteclose}\ \isanewline
%
\isadelimproof
%
\endisadelimproof
%
\isatagproof
\isacommand{proof}\isamarkupfalse%
\ {\isacharparenleft}{\kern0pt}cases\ {\isachardoublequoteopen}{\isasymexists}x{\isachardot}{\kern0pt}\ {\isasymexists}P{\isachardot}{\kern0pt}\ xP\ {\isacharequal}{\kern0pt}\ {\isacharless}{\kern0pt}x{\isacharcomma}{\kern0pt}\ P{\isachargreater}{\kern0pt}{\isachardoublequoteclose}{\isacharparenright}{\kern0pt}\isanewline
\ \ \isacommand{case}\isamarkupfalse%
\ False\isanewline
\ \ \isacommand{then}\isamarkupfalse%
\ \isacommand{have}\isamarkupfalse%
\ {\isachardoublequoteopen}Hsep{\isacharunderscore}{\kern0pt}base{\isacharunderscore}{\kern0pt}M{\isacharparenleft}{\kern0pt}xP{\isacharcomma}{\kern0pt}\ g{\isacharparenright}{\kern0pt}\ {\isacharequal}{\kern0pt}\ {\isadigit{0}}{\isachardoublequoteclose}\ \isacommand{unfolding}\isamarkupfalse%
\ Hsep{\isacharunderscore}{\kern0pt}base{\isacharunderscore}{\kern0pt}M{\isacharunderscore}{\kern0pt}def\ \isacommand{by}\isamarkupfalse%
\ auto\isanewline
\ \ \isacommand{then}\isamarkupfalse%
\ \isacommand{show}\isamarkupfalse%
\ {\isacharquery}{\kern0pt}thesis\ \isacommand{using}\isamarkupfalse%
\ zero{\isacharunderscore}{\kern0pt}in{\isacharunderscore}{\kern0pt}M\ \isacommand{by}\isamarkupfalse%
\ auto\isanewline
\isacommand{next}\isamarkupfalse%
\isanewline
\ \ \isacommand{assume}\isamarkupfalse%
\ assms\ {\isacharcolon}{\kern0pt}\ {\isachardoublequoteopen}xP\ {\isasymin}\ M{\isachardoublequoteclose}\ {\isachardoublequoteopen}g\ {\isasymin}\ M{\isachardoublequoteclose}\ {\isachardoublequoteopen}function{\isacharparenleft}{\kern0pt}g{\isacharparenright}{\kern0pt}{\isachardoublequoteclose}\ \isanewline
\isanewline
\ \ \isacommand{case}\isamarkupfalse%
\ True\isanewline
\ \ \isacommand{then}\isamarkupfalse%
\ \isacommand{obtain}\isamarkupfalse%
\ x\ P\ \isakeyword{where}\ xPH{\isacharcolon}{\kern0pt}\ {\isachardoublequoteopen}xP\ {\isacharequal}{\kern0pt}\ {\isacharless}{\kern0pt}x{\isacharcomma}{\kern0pt}\ P{\isachargreater}{\kern0pt}{\isachardoublequoteclose}\ \isacommand{by}\isamarkupfalse%
\ auto\ \isanewline
\isanewline
\ \ \isacommand{have}\isamarkupfalse%
\ H{\isacharcolon}{\kern0pt}\ {\isachardoublequoteopen}Hsep{\isacharunderscore}{\kern0pt}base{\isacharunderscore}{\kern0pt}M{\isacharparenleft}{\kern0pt}xP{\isacharcomma}{\kern0pt}\ g{\isacharparenright}{\kern0pt}\ {\isacharequal}{\kern0pt}\ Pow{\isacharparenleft}{\kern0pt}{\isacharparenleft}{\kern0pt}{\isasymUnion}{\isacharparenleft}{\kern0pt}g{\isacharbackquote}{\kern0pt}{\isacharbackquote}{\kern0pt}{\isacharparenleft}{\kern0pt}domain{\isacharparenleft}{\kern0pt}x{\isacharparenright}{\kern0pt}\ {\isasymtimes}\ {\isacharbraceleft}{\kern0pt}P{\isacharbraceright}{\kern0pt}{\isacharparenright}{\kern0pt}{\isacharparenright}{\kern0pt}{\isacharparenright}{\kern0pt}\ {\isasymtimes}\ P{\isacharparenright}{\kern0pt}\ {\isasyminter}\ M{\isachardoublequoteclose}\isanewline
\ \ \ \ \isacommand{unfolding}\isamarkupfalse%
\ Hsep{\isacharunderscore}{\kern0pt}base{\isacharunderscore}{\kern0pt}M{\isacharunderscore}{\kern0pt}def\ \isanewline
\ \ \ \ \isacommand{using}\isamarkupfalse%
\ xPH\ \isanewline
\ \ \ \ \isacommand{by}\isamarkupfalse%
\ auto\isanewline
\isanewline
\ \ \isacommand{have}\isamarkupfalse%
\ {\isachardoublequoteopen}Pow{\isacharparenleft}{\kern0pt}{\isacharparenleft}{\kern0pt}{\isasymUnion}{\isacharparenleft}{\kern0pt}g{\isacharbackquote}{\kern0pt}{\isacharbackquote}{\kern0pt}{\isacharparenleft}{\kern0pt}domain{\isacharparenleft}{\kern0pt}x{\isacharparenright}{\kern0pt}\ {\isasymtimes}\ {\isacharbraceleft}{\kern0pt}P{\isacharbraceright}{\kern0pt}{\isacharparenright}{\kern0pt}{\isacharparenright}{\kern0pt}{\isacharparenright}{\kern0pt}\ {\isasymtimes}\ P{\isacharparenright}{\kern0pt}\ {\isasyminter}\ M\ {\isasymin}\ M{\isachardoublequoteclose}\ \isanewline
\ \ \ \ \isacommand{apply}\isamarkupfalse%
{\isacharparenleft}{\kern0pt}rule\ M{\isacharunderscore}{\kern0pt}powerset{\isacharparenright}{\kern0pt}\isanewline
\ \ \ \ \isacommand{using}\isamarkupfalse%
\ xPH\ assms\ pair{\isacharunderscore}{\kern0pt}in{\isacharunderscore}{\kern0pt}M{\isacharunderscore}{\kern0pt}iff\ singleton{\isacharunderscore}{\kern0pt}in{\isacharunderscore}{\kern0pt}M{\isacharunderscore}{\kern0pt}iff\ domain{\isacharunderscore}{\kern0pt}closed\ cartprod{\isacharunderscore}{\kern0pt}closed\ image{\isacharunderscore}{\kern0pt}closed\ Union{\isacharunderscore}{\kern0pt}closed\isanewline
\ \ \ \ \isacommand{by}\isamarkupfalse%
\ auto\ \ \isanewline
\isanewline
\ \ \isacommand{then}\isamarkupfalse%
\ \isacommand{show}\isamarkupfalse%
\ {\isacharquery}{\kern0pt}thesis\ \isanewline
\ \ \ \ \isacommand{using}\isamarkupfalse%
\ H\ zero{\isacharunderscore}{\kern0pt}in{\isacharunderscore}{\kern0pt}M\ singleton{\isacharunderscore}{\kern0pt}in{\isacharunderscore}{\kern0pt}M{\isacharunderscore}{\kern0pt}iff\ \isacommand{by}\isamarkupfalse%
\ force\isanewline
\isacommand{qed}\isamarkupfalse%
%
\endisatagproof
{\isafoldproof}%
%
\isadelimproof
\isanewline
%
\endisadelimproof
\isanewline
\isacommand{definition}\isamarkupfalse%
\ Hsep{\isacharunderscore}{\kern0pt}base\ \isakeyword{where}\ {\isachardoublequoteopen}Hsep{\isacharunderscore}{\kern0pt}base{\isacharparenleft}{\kern0pt}x{\isacharcomma}{\kern0pt}\ g{\isacharparenright}{\kern0pt}\ {\isasymequiv}\ Pow{\isacharparenleft}{\kern0pt}{\isacharparenleft}{\kern0pt}{\isasymUnion}{\isacharparenleft}{\kern0pt}g{\isacharbackquote}{\kern0pt}{\isacharbackquote}{\kern0pt}domain{\isacharparenleft}{\kern0pt}x{\isacharparenright}{\kern0pt}{\isacharparenright}{\kern0pt}{\isacharparenright}{\kern0pt}\ {\isasymtimes}\ P{\isacharparenright}{\kern0pt}\ {\isasyminter}\ M{\isachardoublequoteclose}\isanewline
\isanewline
\isacommand{definition}\isamarkupfalse%
\ sep{\isacharunderscore}{\kern0pt}base\ \isakeyword{where}\ {\isachardoublequoteopen}sep{\isacharunderscore}{\kern0pt}base{\isacharparenleft}{\kern0pt}x{\isacharparenright}{\kern0pt}\ {\isasymequiv}\ wftrec{\isacharparenleft}{\kern0pt}Memrel{\isacharparenleft}{\kern0pt}M{\isacharparenright}{\kern0pt}{\isacharcircum}{\kern0pt}{\isacharplus}{\kern0pt}{\isacharcomma}{\kern0pt}\ x{\isacharcomma}{\kern0pt}\ Hsep{\isacharunderscore}{\kern0pt}base{\isacharparenright}{\kern0pt}{\isachardoublequoteclose}\ \isanewline
\isanewline
\isacommand{lemma}\isamarkupfalse%
\ def{\isacharunderscore}{\kern0pt}sep{\isacharunderscore}{\kern0pt}base\ {\isacharcolon}{\kern0pt}\ \isanewline
\ \ \isakeyword{fixes}\ x\ \isanewline
\ \ \isakeyword{assumes}\ {\isachardoublequoteopen}x\ {\isasymin}\ M{\isachardoublequoteclose}\ \isanewline
\ \ \isakeyword{shows}\ {\isachardoublequoteopen}sep{\isacharunderscore}{\kern0pt}base{\isacharparenleft}{\kern0pt}x{\isacharparenright}{\kern0pt}\ {\isacharequal}{\kern0pt}\ Pow{\isacharparenleft}{\kern0pt}{\isacharparenleft}{\kern0pt}{\isasymUnion}{\isacharbraceleft}{\kern0pt}\ sep{\isacharunderscore}{\kern0pt}base{\isacharparenleft}{\kern0pt}y{\isacharparenright}{\kern0pt}{\isachardot}{\kern0pt}\ y\ {\isasymin}\ domain{\isacharparenleft}{\kern0pt}x{\isacharparenright}{\kern0pt}\ {\isacharbraceright}{\kern0pt}{\isacharparenright}{\kern0pt}\ {\isasymtimes}\ P{\isacharparenright}{\kern0pt}\ {\isasyminter}\ M{\isachardoublequoteclose}\ \isanewline
%
\isadelimproof
%
\endisadelimproof
%
\isatagproof
\isacommand{proof}\isamarkupfalse%
\ {\isacharminus}{\kern0pt}\ \isanewline
\isanewline
\ \ \isacommand{define}\isamarkupfalse%
\ F\ \isakeyword{where}\ {\isachardoublequoteopen}F\ {\isasymequiv}\ {\isasymlambda}y\ {\isasymin}\ Memrel{\isacharparenleft}{\kern0pt}M{\isacharparenright}{\kern0pt}{\isacharcircum}{\kern0pt}{\isacharplus}{\kern0pt}\ {\isacharminus}{\kern0pt}{\isacharbackquote}{\kern0pt}{\isacharbackquote}{\kern0pt}\ {\isacharbraceleft}{\kern0pt}x{\isacharbraceright}{\kern0pt}{\isachardot}{\kern0pt}\ sep{\isacharunderscore}{\kern0pt}base{\isacharparenleft}{\kern0pt}y{\isacharparenright}{\kern0pt}{\isachardoublequoteclose}\isanewline
\ \ \isacommand{define}\isamarkupfalse%
\ S\ \isakeyword{where}\ {\isachardoublequoteopen}S\ {\isasymequiv}\ {\isacharbraceleft}{\kern0pt}\ sep{\isacharunderscore}{\kern0pt}base{\isacharparenleft}{\kern0pt}y{\isacharparenright}{\kern0pt}{\isachardot}{\kern0pt}\ y\ {\isasymin}\ domain{\isacharparenleft}{\kern0pt}x{\isacharparenright}{\kern0pt}\ {\isacharbraceright}{\kern0pt}{\isachardoublequoteclose}\ \isanewline
\isanewline
\ \ \isacommand{have}\isamarkupfalse%
\ H{\isadigit{1}}{\isacharcolon}{\kern0pt}\ {\isachardoublequoteopen}sep{\isacharunderscore}{\kern0pt}base{\isacharparenleft}{\kern0pt}x{\isacharparenright}{\kern0pt}\ {\isacharequal}{\kern0pt}\ Pow{\isacharparenleft}{\kern0pt}{\isacharparenleft}{\kern0pt}{\isasymUnion}{\isacharparenleft}{\kern0pt}F{\isacharbackquote}{\kern0pt}{\isacharbackquote}{\kern0pt}domain{\isacharparenleft}{\kern0pt}x{\isacharparenright}{\kern0pt}{\isacharparenright}{\kern0pt}{\isacharparenright}{\kern0pt}\ {\isasymtimes}\ P{\isacharparenright}{\kern0pt}\ {\isasyminter}\ M{\isachardoublequoteclose}\ \isanewline
\ \ \ \ \isacommand{unfolding}\isamarkupfalse%
\ sep{\isacharunderscore}{\kern0pt}base{\isacharunderscore}{\kern0pt}def\ \isanewline
\ \ \ \ \isacommand{apply}\isamarkupfalse%
{\isacharparenleft}{\kern0pt}subst\ wftrec{\isacharparenright}{\kern0pt}\isanewline
\ \ \ \ \ \ \isacommand{apply}\isamarkupfalse%
{\isacharparenleft}{\kern0pt}rule\ wf{\isacharunderscore}{\kern0pt}trancl{\isacharcomma}{\kern0pt}rule\ wf{\isacharunderscore}{\kern0pt}Memrel{\isacharcomma}{\kern0pt}\ rule\ trans{\isacharunderscore}{\kern0pt}trancl{\isacharparenright}{\kern0pt}\isanewline
\ \ \ \ \isacommand{unfolding}\isamarkupfalse%
\ Hsep{\isacharunderscore}{\kern0pt}base{\isacharunderscore}{\kern0pt}def\ F{\isacharunderscore}{\kern0pt}def\ sep{\isacharunderscore}{\kern0pt}base{\isacharunderscore}{\kern0pt}def\isanewline
\ \ \ \ \isacommand{by}\isamarkupfalse%
\ simp\isanewline
\isanewline
\ \ \isacommand{have}\isamarkupfalse%
\ H{\isadigit{2}}{\isacharcolon}{\kern0pt}\ {\isachardoublequoteopen}F{\isacharbackquote}{\kern0pt}{\isacharbackquote}{\kern0pt}domain{\isacharparenleft}{\kern0pt}x{\isacharparenright}{\kern0pt}\ {\isacharequal}{\kern0pt}\ S{\isachardoublequoteclose}\ \isanewline
\ \ \isacommand{proof}\isamarkupfalse%
{\isacharparenleft}{\kern0pt}rule\ equality{\isacharunderscore}{\kern0pt}iffI{\isacharcomma}{\kern0pt}\ rule\ iffI{\isacharparenright}{\kern0pt}\isanewline
\ \ \ \ \isacommand{fix}\isamarkupfalse%
\ v\ \isacommand{assume}\isamarkupfalse%
\ {\isachardoublequoteopen}v\ {\isasymin}\ F{\isacharbackquote}{\kern0pt}{\isacharbackquote}{\kern0pt}domain{\isacharparenleft}{\kern0pt}x{\isacharparenright}{\kern0pt}{\isachardoublequoteclose}\isanewline
\ \ \ \ \isacommand{then}\isamarkupfalse%
\ \isacommand{obtain}\isamarkupfalse%
\ u\ \isakeyword{where}\ uH{\isacharcolon}{\kern0pt}\ {\isachardoublequoteopen}{\isacharless}{\kern0pt}u{\isacharcomma}{\kern0pt}\ v{\isachargreater}{\kern0pt}\ {\isasymin}\ F{\isachardoublequoteclose}\ {\isachardoublequoteopen}u\ {\isasymin}\ domain{\isacharparenleft}{\kern0pt}x{\isacharparenright}{\kern0pt}{\isachardoublequoteclose}\ \isacommand{by}\isamarkupfalse%
\ auto\ \isanewline
\ \ \ \ \isacommand{have}\isamarkupfalse%
\ Fu\ {\isacharcolon}{\kern0pt}\ {\isachardoublequoteopen}F{\isacharbackquote}{\kern0pt}u\ {\isacharequal}{\kern0pt}\ v{\isachardoublequoteclose}\ \isanewline
\ \ \ \ \ \ \isacommand{apply}\isamarkupfalse%
{\isacharparenleft}{\kern0pt}rule\ function{\isacharunderscore}{\kern0pt}apply{\isacharunderscore}{\kern0pt}equality{\isacharparenright}{\kern0pt}\isanewline
\ \ \ \ \ \ \isacommand{apply}\isamarkupfalse%
{\isacharparenleft}{\kern0pt}simp\ add{\isacharcolon}{\kern0pt}uH{\isacharparenright}{\kern0pt}\isanewline
\ \ \ \ \ \ \isacommand{unfolding}\isamarkupfalse%
\ F{\isacharunderscore}{\kern0pt}def\isanewline
\ \ \ \ \ \ \isacommand{apply}\isamarkupfalse%
{\isacharparenleft}{\kern0pt}rule\ function{\isacharunderscore}{\kern0pt}lam{\isacharparenright}{\kern0pt}\isanewline
\ \ \ \ \ \ \isacommand{done}\isamarkupfalse%
\isanewline
\ \ \ \ \isacommand{have}\isamarkupfalse%
\ {\isachardoublequoteopen}u\ {\isasymin}\ Memrel{\isacharparenleft}{\kern0pt}M{\isacharparenright}{\kern0pt}{\isacharcircum}{\kern0pt}{\isacharplus}{\kern0pt}\ {\isacharminus}{\kern0pt}{\isacharbackquote}{\kern0pt}{\isacharbackquote}{\kern0pt}{\isacharbraceleft}{\kern0pt}x{\isacharbraceright}{\kern0pt}{\isachardoublequoteclose}\ \isacommand{using}\isamarkupfalse%
\ uH\ domain{\isacharunderscore}{\kern0pt}elem{\isacharunderscore}{\kern0pt}Memrel{\isacharunderscore}{\kern0pt}trancl\ assms\ \isacommand{by}\isamarkupfalse%
\ auto\isanewline
\ \ \ \ \isacommand{then}\isamarkupfalse%
\ \isacommand{have}\isamarkupfalse%
\ {\isachardoublequoteopen}v\ {\isacharequal}{\kern0pt}\ sep{\isacharunderscore}{\kern0pt}base{\isacharparenleft}{\kern0pt}u{\isacharparenright}{\kern0pt}{\isachardoublequoteclose}\ \isacommand{using}\isamarkupfalse%
\ Fu\ \isacommand{unfolding}\isamarkupfalse%
\ F{\isacharunderscore}{\kern0pt}def\ \isacommand{by}\isamarkupfalse%
\ auto\ \isanewline
\ \ \ \ \isacommand{then}\isamarkupfalse%
\ \isacommand{show}\isamarkupfalse%
\ {\isachardoublequoteopen}v\ {\isasymin}\ S{\isachardoublequoteclose}\ \isanewline
\ \ \ \ \ \ \isacommand{unfolding}\isamarkupfalse%
\ S{\isacharunderscore}{\kern0pt}def\ \isanewline
\ \ \ \ \ \ \isacommand{using}\isamarkupfalse%
\ uH\ \isanewline
\ \ \ \ \ \ \isacommand{by}\isamarkupfalse%
\ auto\isanewline
\ \ \isacommand{next}\isamarkupfalse%
\ \isanewline
\ \ \ \ \isacommand{fix}\isamarkupfalse%
\ v\ \isacommand{assume}\isamarkupfalse%
\ {\isachardoublequoteopen}v\ {\isasymin}\ S{\isachardoublequoteclose}\ \isanewline
\ \ \ \ \isacommand{then}\isamarkupfalse%
\ \isacommand{obtain}\isamarkupfalse%
\ u\ \isakeyword{where}\ uH\ {\isacharcolon}{\kern0pt}\ {\isachardoublequoteopen}u\ {\isasymin}\ domain{\isacharparenleft}{\kern0pt}x{\isacharparenright}{\kern0pt}{\isachardoublequoteclose}\ {\isachardoublequoteopen}v\ {\isacharequal}{\kern0pt}\ sep{\isacharunderscore}{\kern0pt}base{\isacharparenleft}{\kern0pt}u{\isacharparenright}{\kern0pt}{\isachardoublequoteclose}\ \isacommand{unfolding}\isamarkupfalse%
\ S{\isacharunderscore}{\kern0pt}def\ \isacommand{by}\isamarkupfalse%
\ auto\ \isanewline
\ \ \ \ \isacommand{then}\isamarkupfalse%
\ \isacommand{have}\isamarkupfalse%
\ uin\ {\isacharcolon}{\kern0pt}\ {\isachardoublequoteopen}u\ {\isasymin}\ Memrel{\isacharparenleft}{\kern0pt}M{\isacharparenright}{\kern0pt}{\isacharcircum}{\kern0pt}{\isacharplus}{\kern0pt}\ {\isacharminus}{\kern0pt}{\isacharbackquote}{\kern0pt}{\isacharbackquote}{\kern0pt}\ {\isacharbraceleft}{\kern0pt}x{\isacharbraceright}{\kern0pt}{\isachardoublequoteclose}\ \isacommand{using}\isamarkupfalse%
\ assms\ domain{\isacharunderscore}{\kern0pt}elem{\isacharunderscore}{\kern0pt}Memrel{\isacharunderscore}{\kern0pt}trancl\ uH\ \isacommand{by}\isamarkupfalse%
\ auto\ \isanewline
\ \ \ \ \isacommand{have}\isamarkupfalse%
\ Fu{\isacharcolon}{\kern0pt}\ {\isachardoublequoteopen}F{\isacharbackquote}{\kern0pt}u\ {\isacharequal}{\kern0pt}\ v{\isachardoublequoteclose}\ \isanewline
\ \ \ \ \ \ \isacommand{unfolding}\isamarkupfalse%
\ F{\isacharunderscore}{\kern0pt}def\ \isanewline
\ \ \ \ \ \ \isacommand{using}\isamarkupfalse%
\ uin\ uH\ \isanewline
\ \ \ \ \ \ \isacommand{by}\isamarkupfalse%
\ auto\ \isanewline
\ \ \ \ \isacommand{have}\isamarkupfalse%
\ {\isachardoublequoteopen}{\isacharless}{\kern0pt}u{\isacharcomma}{\kern0pt}\ F{\isacharbackquote}{\kern0pt}u{\isachargreater}{\kern0pt}\ {\isasymin}\ F{\isachardoublequoteclose}\ \isanewline
\ \ \ \ \ \ \isacommand{apply}\isamarkupfalse%
{\isacharparenleft}{\kern0pt}rule\ function{\isacharunderscore}{\kern0pt}apply{\isacharunderscore}{\kern0pt}Pair{\isacharparenright}{\kern0pt}\isanewline
\ \ \ \ \ \ \isacommand{unfolding}\isamarkupfalse%
\ F{\isacharunderscore}{\kern0pt}def\isanewline
\ \ \ \ \ \ \ \isacommand{apply}\isamarkupfalse%
{\isacharparenleft}{\kern0pt}rule\ function{\isacharunderscore}{\kern0pt}lam{\isacharcomma}{\kern0pt}\ subst\ domain{\isacharunderscore}{\kern0pt}lam{\isacharcomma}{\kern0pt}\ simp\ add{\isacharcolon}{\kern0pt}uin{\isacharparenright}{\kern0pt}\isanewline
\ \ \ \ \ \ \isacommand{done}\isamarkupfalse%
\isanewline
\ \ \ \ \isacommand{then}\isamarkupfalse%
\ \isacommand{show}\isamarkupfalse%
\ {\isachardoublequoteopen}v\ {\isasymin}\ F{\isacharbackquote}{\kern0pt}{\isacharbackquote}{\kern0pt}domain{\isacharparenleft}{\kern0pt}x{\isacharparenright}{\kern0pt}{\isachardoublequoteclose}\ \isanewline
\ \ \ \ \ \ \isacommand{using}\isamarkupfalse%
\ uin\ uH\ Fu\ \isacommand{by}\isamarkupfalse%
\ auto\isanewline
\ \ \isacommand{qed}\isamarkupfalse%
\isanewline
\isanewline
\ \ \isacommand{show}\isamarkupfalse%
\ {\isacharquery}{\kern0pt}thesis\ \isanewline
\ \ \ \ \isacommand{using}\isamarkupfalse%
\ H{\isadigit{1}}\ H{\isadigit{2}}\ \isanewline
\ \ \ \ \isacommand{unfolding}\isamarkupfalse%
\ F{\isacharunderscore}{\kern0pt}def\ S{\isacharunderscore}{\kern0pt}def\ \isanewline
\ \ \ \ \isacommand{by}\isamarkupfalse%
\ auto\isanewline
\isacommand{qed}\isamarkupfalse%
%
\endisatagproof
{\isafoldproof}%
%
\isadelimproof
\isanewline
%
\endisadelimproof
\isanewline
\isacommand{definition}\isamarkupfalse%
\ is{\isacharunderscore}{\kern0pt}sep{\isacharunderscore}{\kern0pt}base{\isacharunderscore}{\kern0pt}fm\ \isakeyword{where}\ {\isachardoublequoteopen}is{\isacharunderscore}{\kern0pt}sep{\isacharunderscore}{\kern0pt}base{\isacharunderscore}{\kern0pt}fm{\isacharparenleft}{\kern0pt}x{\isacharcomma}{\kern0pt}\ p{\isacharcomma}{\kern0pt}\ s{\isacharparenright}{\kern0pt}\ {\isasymequiv}\ is{\isacharunderscore}{\kern0pt}memrel{\isacharunderscore}{\kern0pt}wftrec{\isacharunderscore}{\kern0pt}fm{\isacharparenleft}{\kern0pt}Hsep{\isacharunderscore}{\kern0pt}base{\isacharunderscore}{\kern0pt}M{\isacharunderscore}{\kern0pt}fm{\isacharcomma}{\kern0pt}\ x{\isacharcomma}{\kern0pt}\ p{\isacharcomma}{\kern0pt}\ s{\isacharparenright}{\kern0pt}{\isachardoublequoteclose}\ \isanewline
\isanewline
\isacommand{lemma}\isamarkupfalse%
\ Hsep{\isacharunderscore}{\kern0pt}base{\isacharunderscore}{\kern0pt}eq\ {\isacharcolon}{\kern0pt}\ \isanewline
\ \ \isakeyword{fixes}\ h\ g\ x\ \isanewline
\ \ \isakeyword{assumes}\ {\isachardoublequoteopen}h\ {\isasymin}\ eclose{\isacharparenleft}{\kern0pt}x{\isacharparenright}{\kern0pt}\ {\isasymrightarrow}\ M{\isachardoublequoteclose}\ {\isachardoublequoteopen}g\ {\isasymin}\ eclose{\isacharparenleft}{\kern0pt}x{\isacharparenright}{\kern0pt}\ {\isasymtimes}\ {\isacharbraceleft}{\kern0pt}P{\isacharbraceright}{\kern0pt}\ {\isasymrightarrow}\ M{\isachardoublequoteclose}\ {\isachardoublequoteopen}g\ {\isasymin}\ M{\isachardoublequoteclose}\ {\isachardoublequoteopen}x\ {\isasymin}\ M{\isachardoublequoteclose}\ \isanewline
\ \ \ \ \ \ \ \ \ \ {\isachardoublequoteopen}{\isasymAnd}y{\isachardot}{\kern0pt}\ y\ {\isasymin}\ eclose{\isacharparenleft}{\kern0pt}x{\isacharparenright}{\kern0pt}\ {\isasymLongrightarrow}\ h\ {\isacharbackquote}{\kern0pt}\ y\ {\isacharequal}{\kern0pt}\ g\ {\isacharbackquote}{\kern0pt}\ {\isasymlangle}y{\isacharcomma}{\kern0pt}\ P{\isasymrangle}{\isachardoublequoteclose}\ \isanewline
\ \ \isakeyword{shows}\ {\isachardoublequoteopen}Hsep{\isacharunderscore}{\kern0pt}base{\isacharparenleft}{\kern0pt}x{\isacharcomma}{\kern0pt}\ h{\isacharparenright}{\kern0pt}\ {\isacharequal}{\kern0pt}\ Hsep{\isacharunderscore}{\kern0pt}base{\isacharunderscore}{\kern0pt}M{\isacharparenleft}{\kern0pt}{\isasymlangle}x{\isacharcomma}{\kern0pt}\ P{\isasymrangle}{\isacharcomma}{\kern0pt}\ g{\isacharparenright}{\kern0pt}{\isachardoublequoteclose}\isanewline
%
\isadelimproof
%
\endisadelimproof
%
\isatagproof
\isacommand{proof}\isamarkupfalse%
\ {\isacharminus}{\kern0pt}\ \isanewline
\isanewline
\ \ \isacommand{have}\isamarkupfalse%
\ iff{\isacharunderscore}{\kern0pt}lemma\ {\isacharcolon}{\kern0pt}\ {\isachardoublequoteopen}{\isasymAnd}a\ b\ c{\isachardot}{\kern0pt}\ b\ {\isacharequal}{\kern0pt}\ c\ {\isasymLongrightarrow}\ a\ {\isacharequal}{\kern0pt}\ b\ {\isasymlongleftrightarrow}\ a\ {\isacharequal}{\kern0pt}\ c{\isachardoublequoteclose}\ \isacommand{by}\isamarkupfalse%
\ auto\isanewline
\isanewline
\ \ \isacommand{have}\isamarkupfalse%
\ image{\isacharunderscore}{\kern0pt}lemma\ {\isacharcolon}{\kern0pt}\ {\isachardoublequoteopen}{\isasymAnd}f\ d\ v{\isachardot}{\kern0pt}\ function{\isacharparenleft}{\kern0pt}f{\isacharparenright}{\kern0pt}\ {\isasymLongrightarrow}\ d\ {\isasymsubseteq}\ domain{\isacharparenleft}{\kern0pt}f{\isacharparenright}{\kern0pt}\ {\isasymLongrightarrow}\ v\ {\isasymin}\ f{\isacharbackquote}{\kern0pt}{\isacharbackquote}{\kern0pt}d\ {\isasymlongleftrightarrow}\ {\isacharparenleft}{\kern0pt}{\isasymexists}a\ {\isasymin}\ d{\isachardot}{\kern0pt}\ v\ {\isacharequal}{\kern0pt}\ f{\isacharbackquote}{\kern0pt}a{\isacharparenright}{\kern0pt}{\isachardoublequoteclose}\ \isanewline
\ \ \isacommand{proof}\isamarkupfalse%
{\isacharparenleft}{\kern0pt}rule\ iffI{\isacharparenright}{\kern0pt}\isanewline
\ \ \ \ \isacommand{fix}\isamarkupfalse%
\ f\ d\ v\ \isacommand{assume}\isamarkupfalse%
\ assms{\isadigit{1}}{\isacharcolon}{\kern0pt}\ {\isachardoublequoteopen}function{\isacharparenleft}{\kern0pt}f{\isacharparenright}{\kern0pt}{\isachardoublequoteclose}\ {\isachardoublequoteopen}v\ {\isasymin}\ f{\isacharbackquote}{\kern0pt}{\isacharbackquote}{\kern0pt}d{\isachardoublequoteclose}\ \isanewline
\ \ \ \ \isacommand{then}\isamarkupfalse%
\ \isacommand{obtain}\isamarkupfalse%
\ a\ \isakeyword{where}\ aH\ {\isacharcolon}{\kern0pt}\ {\isachardoublequoteopen}{\isacharless}{\kern0pt}a{\isacharcomma}{\kern0pt}\ v{\isachargreater}{\kern0pt}\ {\isasymin}\ f{\isachardoublequoteclose}\ {\isachardoublequoteopen}a\ {\isasymin}\ d{\isachardoublequoteclose}\ \isacommand{by}\isamarkupfalse%
\ auto\ \isanewline
\ \ \ \ \isacommand{have}\isamarkupfalse%
\ {\isachardoublequoteopen}f{\isacharbackquote}{\kern0pt}a\ {\isacharequal}{\kern0pt}\ v{\isachardoublequoteclose}\ \isacommand{by}\isamarkupfalse%
{\isacharparenleft}{\kern0pt}rule\ function{\isacharunderscore}{\kern0pt}apply{\isacharunderscore}{\kern0pt}equality{\isacharcomma}{\kern0pt}\ simp\ add{\isacharcolon}{\kern0pt}aH{\isacharcomma}{\kern0pt}\ simp\ add{\isacharcolon}{\kern0pt}assms{\isadigit{1}}{\isacharparenright}{\kern0pt}\isanewline
\ \ \ \ \isacommand{then}\isamarkupfalse%
\ \isacommand{show}\isamarkupfalse%
\ {\isachardoublequoteopen}{\isasymexists}a{\isasymin}d{\isachardot}{\kern0pt}\ v\ {\isacharequal}{\kern0pt}\ f\ {\isacharbackquote}{\kern0pt}\ a{\isachardoublequoteclose}\ \isacommand{using}\isamarkupfalse%
\ aH\ \isacommand{by}\isamarkupfalse%
\ auto\ \isanewline
\ \ \isacommand{next}\isamarkupfalse%
\ \isanewline
\ \ \ \ \isacommand{fix}\isamarkupfalse%
\ f\ d\ v\ \isacommand{assume}\isamarkupfalse%
\ assms{\isadigit{1}}{\isacharcolon}{\kern0pt}\ {\isachardoublequoteopen}function{\isacharparenleft}{\kern0pt}f{\isacharparenright}{\kern0pt}{\isachardoublequoteclose}\ {\isachardoublequoteopen}d\ {\isasymsubseteq}\ domain{\isacharparenleft}{\kern0pt}f{\isacharparenright}{\kern0pt}{\isachardoublequoteclose}\ {\isachardoublequoteopen}{\isasymexists}a{\isasymin}d{\isachardot}{\kern0pt}\ v\ {\isacharequal}{\kern0pt}\ f\ {\isacharbackquote}{\kern0pt}\ a{\isachardoublequoteclose}\ \isanewline
\ \ \ \ \isacommand{then}\isamarkupfalse%
\ \isacommand{obtain}\isamarkupfalse%
\ a\ \isakeyword{where}\ aH\ {\isacharcolon}{\kern0pt}\ {\isachardoublequoteopen}a\ {\isasymin}\ d{\isachardoublequoteclose}\ {\isachardoublequoteopen}v\ {\isacharequal}{\kern0pt}\ f{\isacharbackquote}{\kern0pt}a{\isachardoublequoteclose}\ \isacommand{by}\isamarkupfalse%
\ auto\ \isanewline
\ \ \ \ \isacommand{have}\isamarkupfalse%
\ {\isachardoublequoteopen}{\isacharless}{\kern0pt}a{\isacharcomma}{\kern0pt}\ f{\isacharbackquote}{\kern0pt}a{\isachargreater}{\kern0pt}\ {\isasymin}\ f{\isachardoublequoteclose}\ \isanewline
\ \ \ \ \ \ \isacommand{apply}\isamarkupfalse%
{\isacharparenleft}{\kern0pt}rule\ function{\isacharunderscore}{\kern0pt}apply{\isacharunderscore}{\kern0pt}Pair{\isacharparenright}{\kern0pt}\ \isanewline
\ \ \ \ \ \ \isacommand{using}\isamarkupfalse%
\ assms{\isadigit{1}}\ aH\ \isanewline
\ \ \ \ \ \ \isacommand{by}\isamarkupfalse%
\ auto\isanewline
\ \ \ \ \isacommand{then}\isamarkupfalse%
\ \isacommand{show}\isamarkupfalse%
\ {\isachardoublequoteopen}v\ {\isasymin}\ f\ {\isacharbackquote}{\kern0pt}{\isacharbackquote}{\kern0pt}\ d{\isachardoublequoteclose}\ \isanewline
\ \ \ \ \ \ \isacommand{using}\isamarkupfalse%
\ aH\ \isanewline
\ \ \ \ \ \ \isacommand{by}\isamarkupfalse%
\ auto\isanewline
\ \ \isacommand{qed}\isamarkupfalse%
\isanewline
\isanewline
\ \ \isacommand{have}\isamarkupfalse%
\ {\isachardoublequoteopen}h\ {\isacharbackquote}{\kern0pt}{\isacharbackquote}{\kern0pt}\ domain{\isacharparenleft}{\kern0pt}x{\isacharparenright}{\kern0pt}\ {\isacharequal}{\kern0pt}\ g\ {\isacharbackquote}{\kern0pt}{\isacharbackquote}{\kern0pt}\ {\isacharparenleft}{\kern0pt}domain{\isacharparenleft}{\kern0pt}x{\isacharparenright}{\kern0pt}\ {\isasymtimes}\ {\isacharbraceleft}{\kern0pt}P{\isacharbraceright}{\kern0pt}{\isacharparenright}{\kern0pt}{\isachardoublequoteclose}\ \isanewline
\ \ \isacommand{proof}\isamarkupfalse%
{\isacharparenleft}{\kern0pt}rule\ equality{\isacharunderscore}{\kern0pt}iffI{\isacharparenright}{\kern0pt}\isanewline
\ \ \ \ \isacommand{fix}\isamarkupfalse%
\ v\isanewline
\ \ \ \ \isacommand{have}\isamarkupfalse%
\ {\isachardoublequoteopen}v\ {\isasymin}\ h\ {\isacharbackquote}{\kern0pt}{\isacharbackquote}{\kern0pt}\ domain{\isacharparenleft}{\kern0pt}x{\isacharparenright}{\kern0pt}\ {\isasymlongleftrightarrow}\ {\isacharparenleft}{\kern0pt}{\isasymexists}y\ {\isasymin}\ domain{\isacharparenleft}{\kern0pt}x{\isacharparenright}{\kern0pt}{\isachardot}{\kern0pt}\ v\ {\isacharequal}{\kern0pt}\ h{\isacharbackquote}{\kern0pt}y{\isacharparenright}{\kern0pt}{\isachardoublequoteclose}\isanewline
\ \ \ \ \ \ \isacommand{apply}\isamarkupfalse%
{\isacharparenleft}{\kern0pt}rule\ image{\isacharunderscore}{\kern0pt}lemma{\isacharparenright}{\kern0pt}\isanewline
\ \ \ \ \ \ \isacommand{using}\isamarkupfalse%
\ assms\ Pi{\isacharunderscore}{\kern0pt}def\ \isanewline
\ \ \ \ \ \ \ \isacommand{apply}\isamarkupfalse%
\ simp\isanewline
\ \ \ \ \ \ \isacommand{apply}\isamarkupfalse%
{\isacharparenleft}{\kern0pt}rule{\isacharunderscore}{\kern0pt}tac\ b{\isacharequal}{\kern0pt}{\isachardoublequoteopen}domain{\isacharparenleft}{\kern0pt}h{\isacharparenright}{\kern0pt}{\isachardoublequoteclose}\ \isakeyword{and}\ a{\isacharequal}{\kern0pt}{\isachardoublequoteopen}eclose{\isacharparenleft}{\kern0pt}x{\isacharparenright}{\kern0pt}{\isachardoublequoteclose}\ \isakeyword{in}\ ssubst{\isacharparenright}{\kern0pt}\ \isanewline
\ \ \ \ \ \ \ \isacommand{apply}\isamarkupfalse%
{\isacharparenleft}{\kern0pt}rule{\isacharunderscore}{\kern0pt}tac\ B{\isacharequal}{\kern0pt}M\ \isakeyword{in}\ domain{\isacharunderscore}{\kern0pt}Pi{\isacharcomma}{\kern0pt}\ simp\ add{\isacharcolon}{\kern0pt}assms{\isacharparenright}{\kern0pt}\isanewline
\ \ \ \ \ \ \isacommand{apply}\isamarkupfalse%
{\isacharparenleft}{\kern0pt}rule\ subsetI{\isacharparenright}{\kern0pt}\isanewline
\ \ \ \ \ \ \isacommand{using}\isamarkupfalse%
\ domain{\isacharunderscore}{\kern0pt}elem{\isacharunderscore}{\kern0pt}in{\isacharunderscore}{\kern0pt}eclose\ \isanewline
\ \ \ \ \ \ \isacommand{by}\isamarkupfalse%
\ auto\isanewline
\ \ \ \ \isacommand{also}\isamarkupfalse%
\ \isacommand{have}\isamarkupfalse%
\ {\isachardoublequoteopen}{\isachardot}{\kern0pt}{\isachardot}{\kern0pt}{\isachardot}{\kern0pt}\ {\isasymlongleftrightarrow}\ {\isacharparenleft}{\kern0pt}{\isasymexists}y\ {\isasymin}\ domain{\isacharparenleft}{\kern0pt}x{\isacharparenright}{\kern0pt}{\isachardot}{\kern0pt}\ v\ {\isacharequal}{\kern0pt}\ g{\isacharbackquote}{\kern0pt}{\isacharless}{\kern0pt}y{\isacharcomma}{\kern0pt}\ P{\isachargreater}{\kern0pt}{\isacharparenright}{\kern0pt}{\isachardoublequoteclose}\ \isanewline
\ \ \ \ \ \ \isacommand{apply}\isamarkupfalse%
{\isacharparenleft}{\kern0pt}rule\ bex{\isacharunderscore}{\kern0pt}iff{\isacharcomma}{\kern0pt}\ rule\ iff{\isacharunderscore}{\kern0pt}lemma{\isacharparenright}{\kern0pt}\isanewline
\ \ \ \ \ \ \isacommand{apply}\isamarkupfalse%
{\isacharparenleft}{\kern0pt}rename{\isacharunderscore}{\kern0pt}tac\ y{\isacharcomma}{\kern0pt}\ subgoal{\isacharunderscore}{\kern0pt}tac\ {\isachardoublequoteopen}y\ {\isasymin}\ eclose{\isacharparenleft}{\kern0pt}x{\isacharparenright}{\kern0pt}{\isachardoublequoteclose}{\isacharparenright}{\kern0pt}\ \isanewline
\ \ \ \ \ \ \isacommand{using}\isamarkupfalse%
\ assms\ \isanewline
\ \ \ \ \ \ \ \isacommand{apply}\isamarkupfalse%
\ force\isanewline
\ \ \ \ \ \ \isacommand{using}\isamarkupfalse%
\ domain{\isacharunderscore}{\kern0pt}elem{\isacharunderscore}{\kern0pt}in{\isacharunderscore}{\kern0pt}eclose\ assms\ \isanewline
\ \ \ \ \ \ \isacommand{by}\isamarkupfalse%
\ auto\isanewline
\ \ \ \ \isacommand{also}\isamarkupfalse%
\ \isacommand{have}\isamarkupfalse%
\ {\isachardoublequoteopen}{\isachardot}{\kern0pt}{\isachardot}{\kern0pt}{\isachardot}{\kern0pt}\ {\isasymlongleftrightarrow}\ {\isacharparenleft}{\kern0pt}{\isasymexists}yP\ {\isasymin}\ domain{\isacharparenleft}{\kern0pt}x{\isacharparenright}{\kern0pt}\ {\isasymtimes}\ {\isacharbraceleft}{\kern0pt}P{\isacharbraceright}{\kern0pt}{\isachardot}{\kern0pt}\ v\ {\isacharequal}{\kern0pt}\ g{\isacharbackquote}{\kern0pt}yP{\isacharparenright}{\kern0pt}{\isachardoublequoteclose}\ \isacommand{by}\isamarkupfalse%
\ auto\ \isanewline
\ \ \ \ \isacommand{also}\isamarkupfalse%
\ \isacommand{have}\isamarkupfalse%
\ {\isachardoublequoteopen}{\isachardot}{\kern0pt}{\isachardot}{\kern0pt}{\isachardot}{\kern0pt}\ {\isasymlongleftrightarrow}\ v\ {\isasymin}\ g\ {\isacharbackquote}{\kern0pt}{\isacharbackquote}{\kern0pt}\ {\isacharparenleft}{\kern0pt}domain{\isacharparenleft}{\kern0pt}x{\isacharparenright}{\kern0pt}\ {\isasymtimes}\ {\isacharbraceleft}{\kern0pt}P{\isacharbraceright}{\kern0pt}{\isacharparenright}{\kern0pt}{\isachardoublequoteclose}\ \isanewline
\ \ \ \ \ \ \isacommand{apply}\isamarkupfalse%
{\isacharparenleft}{\kern0pt}rule\ iff{\isacharunderscore}{\kern0pt}flip{\isacharcomma}{\kern0pt}\ rule\ image{\isacharunderscore}{\kern0pt}lemma{\isacharparenright}{\kern0pt}\isanewline
\ \ \ \ \ \ \isacommand{using}\isamarkupfalse%
\ assms\ Pi{\isacharunderscore}{\kern0pt}def\ \isanewline
\ \ \ \ \ \ \ \isacommand{apply}\isamarkupfalse%
\ simp\isanewline
\ \ \ \ \ \ \isacommand{apply}\isamarkupfalse%
{\isacharparenleft}{\kern0pt}rule{\isacharunderscore}{\kern0pt}tac\ b{\isacharequal}{\kern0pt}{\isachardoublequoteopen}domain{\isacharparenleft}{\kern0pt}g{\isacharparenright}{\kern0pt}{\isachardoublequoteclose}\ \isakeyword{and}\ a{\isacharequal}{\kern0pt}{\isachardoublequoteopen}eclose{\isacharparenleft}{\kern0pt}x{\isacharparenright}{\kern0pt}\ {\isasymtimes}\ {\isacharbraceleft}{\kern0pt}P{\isacharbraceright}{\kern0pt}{\isachardoublequoteclose}\ \isakeyword{in}\ ssubst{\isacharparenright}{\kern0pt}\ \isanewline
\ \ \ \ \ \ \isacommand{apply}\isamarkupfalse%
{\isacharparenleft}{\kern0pt}rule{\isacharunderscore}{\kern0pt}tac\ B{\isacharequal}{\kern0pt}M\ \isakeyword{in}\ domain{\isacharunderscore}{\kern0pt}Pi{\isacharparenright}{\kern0pt}\isanewline
\ \ \ \ \ \ \isacommand{using}\isamarkupfalse%
\ assms\ domain{\isacharunderscore}{\kern0pt}elem{\isacharunderscore}{\kern0pt}in{\isacharunderscore}{\kern0pt}eclose\ \isanewline
\ \ \ \ \ \ \isacommand{by}\isamarkupfalse%
\ auto\ \isanewline
\ \ \ \ \isacommand{finally}\isamarkupfalse%
\ \isacommand{show}\isamarkupfalse%
\ {\isachardoublequoteopen}v\ {\isasymin}\ h\ {\isacharbackquote}{\kern0pt}{\isacharbackquote}{\kern0pt}\ domain{\isacharparenleft}{\kern0pt}x{\isacharparenright}{\kern0pt}\ {\isasymlongleftrightarrow}\ v\ {\isasymin}\ g\ {\isacharbackquote}{\kern0pt}{\isacharbackquote}{\kern0pt}\ {\isacharparenleft}{\kern0pt}domain{\isacharparenleft}{\kern0pt}x{\isacharparenright}{\kern0pt}\ {\isasymtimes}\ {\isacharbraceleft}{\kern0pt}P{\isacharbraceright}{\kern0pt}{\isacharparenright}{\kern0pt}\ {\isachardoublequoteclose}\ \isacommand{by}\isamarkupfalse%
\ simp\isanewline
\ \ \isacommand{qed}\isamarkupfalse%
\isanewline
\isanewline
\ \ \isacommand{then}\isamarkupfalse%
\ \isacommand{show}\isamarkupfalse%
\ {\isacharquery}{\kern0pt}thesis\ \isanewline
\ \ \ \ \isacommand{unfolding}\isamarkupfalse%
\ Hsep{\isacharunderscore}{\kern0pt}base{\isacharunderscore}{\kern0pt}def\ Hsep{\isacharunderscore}{\kern0pt}base{\isacharunderscore}{\kern0pt}M{\isacharunderscore}{\kern0pt}def\isanewline
\ \ \ \ \isacommand{apply}\isamarkupfalse%
\ simp\isanewline
\ \ \ \ \isacommand{apply}\isamarkupfalse%
{\isacharparenleft}{\kern0pt}subgoal{\isacharunderscore}{\kern0pt}tac\ {\isachardoublequoteopen}h\ {\isacharbackquote}{\kern0pt}{\isacharbackquote}{\kern0pt}\ domain{\isacharparenleft}{\kern0pt}x{\isacharparenright}{\kern0pt}\ {\isacharequal}{\kern0pt}\ g\ {\isacharbackquote}{\kern0pt}{\isacharbackquote}{\kern0pt}\ {\isacharparenleft}{\kern0pt}domain{\isacharparenleft}{\kern0pt}x{\isacharparenright}{\kern0pt}\ {\isasymtimes}\ {\isacharbraceleft}{\kern0pt}P{\isacharbraceright}{\kern0pt}{\isacharparenright}{\kern0pt}{\isachardoublequoteclose}{\isacharcomma}{\kern0pt}\ force{\isacharparenright}{\kern0pt}\isanewline
\ \ \ \ \isacommand{by}\isamarkupfalse%
\ simp\isanewline
\isacommand{qed}\isamarkupfalse%
%
\endisatagproof
{\isafoldproof}%
%
\isadelimproof
\isanewline
%
\endisadelimproof
\isanewline
\isacommand{lemma}\isamarkupfalse%
\ sats{\isacharunderscore}{\kern0pt}is{\isacharunderscore}{\kern0pt}sep{\isacharunderscore}{\kern0pt}base{\isacharunderscore}{\kern0pt}fm{\isacharunderscore}{\kern0pt}iff\ {\isacharcolon}{\kern0pt}\ \isanewline
\ \ \isakeyword{fixes}\ env\ i\ j\ k\ x\ s\ \isanewline
\ \ \isakeyword{assumes}\ {\isachardoublequoteopen}env\ {\isasymin}\ list{\isacharparenleft}{\kern0pt}M{\isacharparenright}{\kern0pt}{\isachardoublequoteclose}\ {\isachardoublequoteopen}i\ {\isacharless}{\kern0pt}\ length{\isacharparenleft}{\kern0pt}env{\isacharparenright}{\kern0pt}{\isachardoublequoteclose}\ {\isachardoublequoteopen}j\ {\isacharless}{\kern0pt}\ length{\isacharparenleft}{\kern0pt}env{\isacharparenright}{\kern0pt}{\isachardoublequoteclose}\ {\isachardoublequoteopen}k\ {\isacharless}{\kern0pt}\ length{\isacharparenleft}{\kern0pt}env{\isacharparenright}{\kern0pt}{\isachardoublequoteclose}\isanewline
\ \ \ \ \ \ \ \ \ \ {\isachardoublequoteopen}nth{\isacharparenleft}{\kern0pt}i{\isacharcomma}{\kern0pt}\ env{\isacharparenright}{\kern0pt}\ {\isacharequal}{\kern0pt}\ x{\isachardoublequoteclose}\ {\isachardoublequoteopen}nth{\isacharparenleft}{\kern0pt}j{\isacharcomma}{\kern0pt}\ env{\isacharparenright}{\kern0pt}\ {\isacharequal}{\kern0pt}\ P{\isachardoublequoteclose}\ {\isachardoublequoteopen}nth{\isacharparenleft}{\kern0pt}k{\isacharcomma}{\kern0pt}\ env{\isacharparenright}{\kern0pt}\ {\isacharequal}{\kern0pt}\ s{\isachardoublequoteclose}\ \isanewline
\ \ \isakeyword{shows}\ {\isachardoublequoteopen}sats{\isacharparenleft}{\kern0pt}M{\isacharcomma}{\kern0pt}\ is{\isacharunderscore}{\kern0pt}sep{\isacharunderscore}{\kern0pt}base{\isacharunderscore}{\kern0pt}fm{\isacharparenleft}{\kern0pt}i{\isacharcomma}{\kern0pt}\ j{\isacharcomma}{\kern0pt}\ k{\isacharparenright}{\kern0pt}{\isacharcomma}{\kern0pt}\ env{\isacharparenright}{\kern0pt}\ {\isasymlongleftrightarrow}\ s\ {\isacharequal}{\kern0pt}\ sep{\isacharunderscore}{\kern0pt}base{\isacharparenleft}{\kern0pt}x{\isacharparenright}{\kern0pt}{\isachardoublequoteclose}\ \isanewline
%
\isadelimproof
\isanewline
\ \ %
\endisadelimproof
%
\isatagproof
\isacommand{unfolding}\isamarkupfalse%
\ sep{\isacharunderscore}{\kern0pt}base{\isacharunderscore}{\kern0pt}def\ is{\isacharunderscore}{\kern0pt}sep{\isacharunderscore}{\kern0pt}base{\isacharunderscore}{\kern0pt}fm{\isacharunderscore}{\kern0pt}def\isanewline
\ \ \isacommand{apply}\isamarkupfalse%
{\isacharparenleft}{\kern0pt}rule{\isacharunderscore}{\kern0pt}tac\ a{\isacharequal}{\kern0pt}P\ \isakeyword{and}\ G{\isacharequal}{\kern0pt}Hsep{\isacharunderscore}{\kern0pt}base{\isacharunderscore}{\kern0pt}M\ \isakeyword{in}\ sats{\isacharunderscore}{\kern0pt}is{\isacharunderscore}{\kern0pt}memrel{\isacharunderscore}{\kern0pt}wftrec{\isacharunderscore}{\kern0pt}fm{\isacharunderscore}{\kern0pt}iff{\isacharparenright}{\kern0pt}\isanewline
\ \ \isacommand{using}\isamarkupfalse%
\ assms\ \isanewline
\ \ \ \ \ \ \ \ \ \ \ \ \ \ \ \isacommand{apply}\isamarkupfalse%
\ auto{\isacharbrackleft}{\kern0pt}{\isadigit{9}}{\isacharbrackright}{\kern0pt}\isanewline
\ \ \ \ \ \ \isacommand{apply}\isamarkupfalse%
{\isacharparenleft}{\kern0pt}simp\ add{\isacharcolon}{\kern0pt}Hsep{\isacharunderscore}{\kern0pt}base{\isacharunderscore}{\kern0pt}M{\isacharunderscore}{\kern0pt}fm{\isacharunderscore}{\kern0pt}def{\isacharparenright}{\kern0pt}\isanewline
\ \ \ \ \ \isacommand{apply}\isamarkupfalse%
{\isacharparenleft}{\kern0pt}simp\ add{\isacharcolon}{\kern0pt}Hsep{\isacharunderscore}{\kern0pt}base{\isacharunderscore}{\kern0pt}M{\isacharunderscore}{\kern0pt}fm{\isacharunderscore}{\kern0pt}def{\isacharparenright}{\kern0pt}\isanewline
\ \ \ \ \ \isacommand{apply}\isamarkupfalse%
{\isacharparenleft}{\kern0pt}simp\ del{\isacharcolon}{\kern0pt}FOL{\isacharunderscore}{\kern0pt}sats{\isacharunderscore}{\kern0pt}iff\ pair{\isacharunderscore}{\kern0pt}abs\ add{\isacharcolon}{\kern0pt}\ fm{\isacharunderscore}{\kern0pt}defs\ nat{\isacharunderscore}{\kern0pt}simp{\isacharunderscore}{\kern0pt}union{\isacharparenright}{\kern0pt}\ \ \isanewline
\ \ \ \ \isacommand{apply}\isamarkupfalse%
{\isacharparenleft}{\kern0pt}rule\ Hsep{\isacharunderscore}{\kern0pt}base{\isacharunderscore}{\kern0pt}M{\isacharunderscore}{\kern0pt}in{\isacharunderscore}{\kern0pt}M{\isacharparenright}{\kern0pt}\isanewline
\ \ \ \ \ \ \isacommand{apply}\isamarkupfalse%
\ auto{\isacharbrackleft}{\kern0pt}{\isadigit{3}}{\isacharbrackright}{\kern0pt}\isanewline
\ \ \ \isacommand{apply}\isamarkupfalse%
{\isacharparenleft}{\kern0pt}rule\ Hsep{\isacharunderscore}{\kern0pt}base{\isacharunderscore}{\kern0pt}eq{\isacharparenright}{\kern0pt}\isanewline
\ \ \ \ \ \ \ \isacommand{apply}\isamarkupfalse%
\ auto{\isacharbrackleft}{\kern0pt}{\isadigit{5}}{\isacharbrackright}{\kern0pt}\isanewline
\ \ \isacommand{unfolding}\isamarkupfalse%
\ Hsep{\isacharunderscore}{\kern0pt}base{\isacharunderscore}{\kern0pt}M{\isacharunderscore}{\kern0pt}fm{\isacharunderscore}{\kern0pt}def\isanewline
\ \ \isacommand{apply}\isamarkupfalse%
{\isacharparenleft}{\kern0pt}rule\ Hsep{\isacharunderscore}{\kern0pt}base{\isacharunderscore}{\kern0pt}M{\isacharunderscore}{\kern0pt}fm{\isacharunderscore}{\kern0pt}auto{\isacharparenright}{\kern0pt}\isanewline
\ \ \isacommand{by}\isamarkupfalse%
\ auto%
\endisatagproof
{\isafoldproof}%
%
\isadelimproof
\isanewline
%
\endisadelimproof
\isanewline
\isacommand{lemma}\isamarkupfalse%
\ sep{\isacharunderscore}{\kern0pt}base{\isacharunderscore}{\kern0pt}in{\isacharunderscore}{\kern0pt}M\ {\isacharcolon}{\kern0pt}\ \isanewline
\ \ \isakeyword{fixes}\ x\ \isanewline
\ \ \isakeyword{assumes}\ {\isachardoublequoteopen}x\ {\isasymin}\ M{\isachardoublequoteclose}\ \isanewline
\ \ \isakeyword{shows}\ {\isachardoublequoteopen}sep{\isacharunderscore}{\kern0pt}base{\isacharparenleft}{\kern0pt}x{\isacharparenright}{\kern0pt}\ {\isasymin}\ M{\isachardoublequoteclose}\ \isanewline
%
\isadelimproof
\isanewline
\ \ %
\endisadelimproof
%
\isatagproof
\isacommand{unfolding}\isamarkupfalse%
\ sep{\isacharunderscore}{\kern0pt}base{\isacharunderscore}{\kern0pt}def\ \isanewline
\ \ \isacommand{apply}\isamarkupfalse%
{\isacharparenleft}{\kern0pt}rule{\isacharunderscore}{\kern0pt}tac\ a{\isacharequal}{\kern0pt}P\ \isakeyword{and}\ G\ {\isacharequal}{\kern0pt}\ Hsep{\isacharunderscore}{\kern0pt}base{\isacharunderscore}{\kern0pt}M\ \isakeyword{and}\ Gfm\ {\isacharequal}{\kern0pt}\ Hsep{\isacharunderscore}{\kern0pt}base{\isacharunderscore}{\kern0pt}M{\isacharunderscore}{\kern0pt}fm\ \isakeyword{in}\ \ memrel{\isacharunderscore}{\kern0pt}wftrec{\isacharunderscore}{\kern0pt}in{\isacharunderscore}{\kern0pt}M{\isacharparenright}{\kern0pt}\isanewline
\ \ \isacommand{using}\isamarkupfalse%
\ assms\ P{\isacharunderscore}{\kern0pt}in{\isacharunderscore}{\kern0pt}M\ \isanewline
\ \ \ \ \ \ \ \ \isacommand{apply}\isamarkupfalse%
\ auto{\isacharbrackleft}{\kern0pt}{\isadigit{2}}{\isacharbrackright}{\kern0pt}\isanewline
\ \ \ \ \ \ \isacommand{apply}\isamarkupfalse%
{\isacharparenleft}{\kern0pt}simp\ add{\isacharcolon}{\kern0pt}Hsep{\isacharunderscore}{\kern0pt}base{\isacharunderscore}{\kern0pt}M{\isacharunderscore}{\kern0pt}fm{\isacharunderscore}{\kern0pt}def{\isacharparenright}{\kern0pt}{\isacharplus}{\kern0pt}\isanewline
\ \ \ \ \ \isacommand{apply}\isamarkupfalse%
{\isacharparenleft}{\kern0pt}simp\ del{\isacharcolon}{\kern0pt}FOL{\isacharunderscore}{\kern0pt}sats{\isacharunderscore}{\kern0pt}iff\ pair{\isacharunderscore}{\kern0pt}abs\ add{\isacharcolon}{\kern0pt}\ fm{\isacharunderscore}{\kern0pt}defs\ nat{\isacharunderscore}{\kern0pt}simp{\isacharunderscore}{\kern0pt}union{\isacharparenright}{\kern0pt}\isanewline
\ \ \ \ \isacommand{apply}\isamarkupfalse%
{\isacharparenleft}{\kern0pt}rule\ Hsep{\isacharunderscore}{\kern0pt}base{\isacharunderscore}{\kern0pt}M{\isacharunderscore}{\kern0pt}in{\isacharunderscore}{\kern0pt}M{\isacharparenright}{\kern0pt}\isanewline
\ \ \ \ \ \ \isacommand{apply}\isamarkupfalse%
\ auto{\isacharbrackleft}{\kern0pt}{\isadigit{3}}{\isacharbrackright}{\kern0pt}\isanewline
\ \ \ \isacommand{apply}\isamarkupfalse%
{\isacharparenleft}{\kern0pt}rule\ Hsep{\isacharunderscore}{\kern0pt}base{\isacharunderscore}{\kern0pt}eq{\isacharparenright}{\kern0pt}\isanewline
\ \ \ \ \ \ \ \isacommand{apply}\isamarkupfalse%
\ auto{\isacharbrackleft}{\kern0pt}{\isadigit{5}}{\isacharbrackright}{\kern0pt}\isanewline
\ \ \isacommand{unfolding}\isamarkupfalse%
\ Hsep{\isacharunderscore}{\kern0pt}base{\isacharunderscore}{\kern0pt}M{\isacharunderscore}{\kern0pt}fm{\isacharunderscore}{\kern0pt}def\isanewline
\ \ \isacommand{apply}\isamarkupfalse%
{\isacharparenleft}{\kern0pt}rule\ Hsep{\isacharunderscore}{\kern0pt}base{\isacharunderscore}{\kern0pt}M{\isacharunderscore}{\kern0pt}fm{\isacharunderscore}{\kern0pt}auto{\isacharparenright}{\kern0pt}\isanewline
\ \ \isacommand{by}\isamarkupfalse%
\ auto%
\endisatagproof
{\isafoldproof}%
%
\isadelimproof
\isanewline
%
\endisadelimproof
\isanewline
\isacommand{lemma}\isamarkupfalse%
\ sep{\isacharunderscore}{\kern0pt}base{\isacharunderscore}{\kern0pt}in\ {\isacharcolon}{\kern0pt}\ \isanewline
\ \ \isakeyword{fixes}\ x\ \isanewline
\ \ \isakeyword{assumes}\ {\isachardoublequoteopen}x\ {\isasymin}\ P{\isacharunderscore}{\kern0pt}names{\isachardoublequoteclose}\ \isanewline
\ \ \isakeyword{shows}\ {\isachardoublequoteopen}x\ {\isasymin}\ sep{\isacharunderscore}{\kern0pt}base{\isacharparenleft}{\kern0pt}x{\isacharparenright}{\kern0pt}{\isachardoublequoteclose}\ \isanewline
%
\isadelimproof
%
\endisadelimproof
%
\isatagproof
\isacommand{proof}\isamarkupfalse%
\ {\isacharminus}{\kern0pt}\ \isanewline
\ \ \isacommand{have}\isamarkupfalse%
\ main\ {\isacharcolon}{\kern0pt}\ {\isachardoublequoteopen}{\isasymAnd}x{\isachardot}{\kern0pt}\ x\ {\isasymin}\ P{\isacharunderscore}{\kern0pt}names\ {\isasymlongrightarrow}\ x\ {\isasymin}\ sep{\isacharunderscore}{\kern0pt}base{\isacharparenleft}{\kern0pt}x{\isacharparenright}{\kern0pt}{\isachardoublequoteclose}\isanewline
\ \ \isacommand{proof}\isamarkupfalse%
\ {\isacharparenleft}{\kern0pt}rule{\isacharunderscore}{\kern0pt}tac\ Q{\isacharequal}{\kern0pt}{\isachardoublequoteopen}{\isasymlambda}x{\isachardot}{\kern0pt}\ x\ {\isasymin}\ P{\isacharunderscore}{\kern0pt}names\ {\isasymlongrightarrow}\ x\ {\isasymin}\ sep{\isacharunderscore}{\kern0pt}base{\isacharparenleft}{\kern0pt}x{\isacharparenright}{\kern0pt}{\isachardoublequoteclose}\ \isakeyword{in}\ ed{\isacharunderscore}{\kern0pt}induction{\isacharcomma}{\kern0pt}\ rule\ impI{\isacharparenright}{\kern0pt}\isanewline
\ \ \ \ \isacommand{fix}\isamarkupfalse%
\ x\ \isacommand{assume}\isamarkupfalse%
\ assms{\isacharcolon}{\kern0pt}\ {\isachardoublequoteopen}{\isacharparenleft}{\kern0pt}{\isasymAnd}y{\isachardot}{\kern0pt}\ ed{\isacharparenleft}{\kern0pt}y{\isacharcomma}{\kern0pt}\ x{\isacharparenright}{\kern0pt}\ {\isasymLongrightarrow}\ y\ {\isasymin}\ P{\isacharunderscore}{\kern0pt}names\ {\isasymlongrightarrow}\ y\ {\isasymin}\ sep{\isacharunderscore}{\kern0pt}base{\isacharparenleft}{\kern0pt}y{\isacharparenright}{\kern0pt}{\isacharparenright}{\kern0pt}{\isachardoublequoteclose}\ {\isachardoublequoteopen}x\ {\isasymin}\ P{\isacharunderscore}{\kern0pt}names{\isachardoublequoteclose}\ \isanewline
\isanewline
\ \ \ \ \isacommand{have}\isamarkupfalse%
\ domH{\isacharcolon}{\kern0pt}\ {\isachardoublequoteopen}domain{\isacharparenleft}{\kern0pt}x{\isacharparenright}{\kern0pt}\ {\isasymsubseteq}\ {\isasymUnion}RepFun{\isacharparenleft}{\kern0pt}domain{\isacharparenleft}{\kern0pt}x{\isacharparenright}{\kern0pt}{\isacharcomma}{\kern0pt}\ sep{\isacharunderscore}{\kern0pt}base{\isacharparenright}{\kern0pt}{\isachardoublequoteclose}\ \isanewline
\ \ \ \ \ \ \isacommand{apply}\isamarkupfalse%
{\isacharparenleft}{\kern0pt}rule\ subsetI{\isacharcomma}{\kern0pt}\ simp{\isacharparenright}{\kern0pt}\isanewline
\ \ \ \ \ \ \isacommand{using}\isamarkupfalse%
\ assms\ P{\isacharunderscore}{\kern0pt}name{\isacharunderscore}{\kern0pt}domain{\isacharunderscore}{\kern0pt}P{\isacharunderscore}{\kern0pt}name\ \isanewline
\ \ \ \ \ \ \isacommand{by}\isamarkupfalse%
\ blast\isanewline
\isanewline
\ \ \ \ \isacommand{have}\isamarkupfalse%
\ ranH{\isacharcolon}{\kern0pt}\ {\isachardoublequoteopen}range{\isacharparenleft}{\kern0pt}x{\isacharparenright}{\kern0pt}\ {\isasymsubseteq}\ P{\isachardoublequoteclose}\ \isanewline
\ \ \ \ \ \ \isacommand{using}\isamarkupfalse%
\ P{\isacharunderscore}{\kern0pt}name{\isacharunderscore}{\kern0pt}iff\ assms\ \isanewline
\ \ \ \ \ \ \isacommand{by}\isamarkupfalse%
\ auto\isanewline
\isanewline
\ \ \ \ \isacommand{have}\isamarkupfalse%
\ H\ {\isacharcolon}{\kern0pt}\ {\isachardoublequoteopen}x\ {\isasymsubseteq}\ {\isasymUnion}RepFun{\isacharparenleft}{\kern0pt}domain{\isacharparenleft}{\kern0pt}x{\isacharparenright}{\kern0pt}{\isacharcomma}{\kern0pt}\ sep{\isacharunderscore}{\kern0pt}base{\isacharparenright}{\kern0pt}\ {\isasymtimes}\ P{\isachardoublequoteclose}\ \isanewline
\ \ \ \ \isacommand{proof}\isamarkupfalse%
\ {\isacharparenleft}{\kern0pt}rule\ subsetI{\isacharparenright}{\kern0pt}\isanewline
\ \ \ \ \ \ \isacommand{fix}\isamarkupfalse%
\ v\ \isacommand{assume}\isamarkupfalse%
\ vin\ {\isacharcolon}{\kern0pt}\ {\isachardoublequoteopen}v\ {\isasymin}\ x{\isachardoublequoteclose}\ \isanewline
\ \ \ \ \ \ \isacommand{then}\isamarkupfalse%
\ \isacommand{obtain}\isamarkupfalse%
\ y\ p\ \isakeyword{where}\ ypH\ {\isacharcolon}{\kern0pt}\ {\isachardoublequoteopen}v\ {\isacharequal}{\kern0pt}\ {\isacharless}{\kern0pt}y{\isacharcomma}{\kern0pt}\ p{\isachargreater}{\kern0pt}{\isachardoublequoteclose}\ \isacommand{using}\isamarkupfalse%
\ assms\ P{\isacharunderscore}{\kern0pt}name{\isacharunderscore}{\kern0pt}iff\ \isacommand{by}\isamarkupfalse%
\ blast\isanewline
\ \ \ \ \ \ \isacommand{have}\isamarkupfalse%
\ {\isachardoublequoteopen}y\ {\isasymin}\ domain{\isacharparenleft}{\kern0pt}x{\isacharparenright}{\kern0pt}{\isachardoublequoteclose}\ \isacommand{using}\isamarkupfalse%
\ vin\ ypH\ \isacommand{by}\isamarkupfalse%
\ auto\ \isanewline
\ \ \ \ \ \ \isacommand{then}\isamarkupfalse%
\ \isacommand{have}\isamarkupfalse%
\ yin\ {\isacharcolon}{\kern0pt}\ {\isachardoublequoteopen}y\ {\isasymin}\ {\isasymUnion}RepFun{\isacharparenleft}{\kern0pt}domain{\isacharparenleft}{\kern0pt}x{\isacharparenright}{\kern0pt}{\isacharcomma}{\kern0pt}\ sep{\isacharunderscore}{\kern0pt}base{\isacharparenright}{\kern0pt}{\isachardoublequoteclose}\ \isacommand{using}\isamarkupfalse%
\ domH\ \isacommand{by}\isamarkupfalse%
\ auto\isanewline
\ \ \ \ \ \ \isacommand{have}\isamarkupfalse%
\ {\isachardoublequoteopen}p\ {\isasymin}\ range{\isacharparenleft}{\kern0pt}x{\isacharparenright}{\kern0pt}{\isachardoublequoteclose}\ \isacommand{using}\isamarkupfalse%
\ vin\ ypH\ \isacommand{by}\isamarkupfalse%
\ auto\ \isanewline
\ \ \ \ \ \ \isacommand{then}\isamarkupfalse%
\ \isacommand{have}\isamarkupfalse%
\ pin\ {\isacharcolon}{\kern0pt}\ {\isachardoublequoteopen}p\ {\isasymin}\ P{\isachardoublequoteclose}\ \isacommand{using}\isamarkupfalse%
\ ranH\ \isacommand{by}\isamarkupfalse%
\ auto\isanewline
\ \ \ \ \ \ \isacommand{show}\isamarkupfalse%
\ {\isachardoublequoteopen}v\ {\isasymin}\ {\isasymUnion}RepFun{\isacharparenleft}{\kern0pt}domain{\isacharparenleft}{\kern0pt}x{\isacharparenright}{\kern0pt}{\isacharcomma}{\kern0pt}\ sep{\isacharunderscore}{\kern0pt}base{\isacharparenright}{\kern0pt}\ {\isasymtimes}\ P{\isachardoublequoteclose}\ \isacommand{using}\isamarkupfalse%
\ ypH\ yin\ pin\ \isacommand{by}\isamarkupfalse%
\ auto\isanewline
\ \ \ \ \isacommand{qed}\isamarkupfalse%
\isanewline
\ \ \ \ \isacommand{then}\isamarkupfalse%
\ \isacommand{show}\isamarkupfalse%
\ {\isachardoublequoteopen}x\ {\isasymin}\ sep{\isacharunderscore}{\kern0pt}base{\isacharparenleft}{\kern0pt}x{\isacharparenright}{\kern0pt}{\isachardoublequoteclose}\isanewline
\ \ \ \ \ \ \isacommand{apply}\isamarkupfalse%
{\isacharparenleft}{\kern0pt}subst\ def{\isacharunderscore}{\kern0pt}sep{\isacharunderscore}{\kern0pt}base{\isacharparenright}{\kern0pt}\isanewline
\ \ \ \ \ \ \isacommand{using}\isamarkupfalse%
\ assms\ P{\isacharunderscore}{\kern0pt}name{\isacharunderscore}{\kern0pt}in{\isacharunderscore}{\kern0pt}M\ \isanewline
\ \ \ \ \ \ \isacommand{by}\isamarkupfalse%
\ auto\isanewline
\ \ \isacommand{qed}\isamarkupfalse%
\isanewline
\isanewline
\ \ \isacommand{then}\isamarkupfalse%
\ \isacommand{show}\isamarkupfalse%
\ {\isacharquery}{\kern0pt}thesis\ \isacommand{using}\isamarkupfalse%
\ assms\ \isacommand{by}\isamarkupfalse%
\ auto\isanewline
\isacommand{qed}\isamarkupfalse%
%
\endisatagproof
{\isafoldproof}%
%
\isadelimproof
\isanewline
%
\endisadelimproof
\isanewline
\isacommand{lemma}\isamarkupfalse%
\ sep{\isacharunderscore}{\kern0pt}base{\isacharunderscore}{\kern0pt}subset\ {\isacharcolon}{\kern0pt}\ \isanewline
\ \ \isakeyword{fixes}\ x\ \isanewline
\ \ \isakeyword{assumes}\ {\isachardoublequoteopen}x\ {\isasymin}\ P{\isacharunderscore}{\kern0pt}names{\isachardoublequoteclose}\ \isanewline
\ \ \isakeyword{shows}\ {\isachardoublequoteopen}sep{\isacharunderscore}{\kern0pt}base{\isacharparenleft}{\kern0pt}x{\isacharparenright}{\kern0pt}\ {\isasymsubseteq}\ P{\isacharunderscore}{\kern0pt}names{\isachardoublequoteclose}\isanewline
%
\isadelimproof
%
\endisadelimproof
%
\isatagproof
\isacommand{proof}\isamarkupfalse%
{\isacharminus}{\kern0pt}\ \isanewline
\ \ \isacommand{have}\isamarkupfalse%
\ main\ {\isacharcolon}{\kern0pt}\ {\isachardoublequoteopen}{\isasymAnd}x{\isachardot}{\kern0pt}\ x\ {\isasymin}\ P{\isacharunderscore}{\kern0pt}names\ {\isasymlongrightarrow}\ sep{\isacharunderscore}{\kern0pt}base{\isacharparenleft}{\kern0pt}x{\isacharparenright}{\kern0pt}\ {\isasymsubseteq}\ P{\isacharunderscore}{\kern0pt}names{\isachardoublequoteclose}\ \isanewline
\ \ \isacommand{proof}\isamarkupfalse%
{\isacharparenleft}{\kern0pt}rule{\isacharunderscore}{\kern0pt}tac\ Q{\isacharequal}{\kern0pt}{\isachardoublequoteopen}{\isasymlambda}\ x{\isachardot}{\kern0pt}\ x\ {\isasymin}\ P{\isacharunderscore}{\kern0pt}names\ {\isasymlongrightarrow}\ sep{\isacharunderscore}{\kern0pt}base{\isacharparenleft}{\kern0pt}x{\isacharparenright}{\kern0pt}\ {\isasymsubseteq}\ P{\isacharunderscore}{\kern0pt}names{\isachardoublequoteclose}\ \isakeyword{in}\ ed{\isacharunderscore}{\kern0pt}induction{\isacharcomma}{\kern0pt}\ rule\ impI{\isacharparenright}{\kern0pt}\isanewline
\ \ \ \ \isacommand{fix}\isamarkupfalse%
\ x\ \isanewline
\ \ \ \ \isacommand{assume}\isamarkupfalse%
\ assms\ {\isacharcolon}{\kern0pt}\ {\isachardoublequoteopen}{\isacharparenleft}{\kern0pt}{\isasymAnd}y{\isachardot}{\kern0pt}\ ed{\isacharparenleft}{\kern0pt}y{\isacharcomma}{\kern0pt}\ x{\isacharparenright}{\kern0pt}\ {\isasymLongrightarrow}\ y\ {\isasymin}\ P{\isacharunderscore}{\kern0pt}names\ {\isasymlongrightarrow}\ sep{\isacharunderscore}{\kern0pt}base{\isacharparenleft}{\kern0pt}y{\isacharparenright}{\kern0pt}\ {\isasymsubseteq}\ P{\isacharunderscore}{\kern0pt}names{\isacharparenright}{\kern0pt}{\isachardoublequoteclose}\ {\isachardoublequoteopen}x\ {\isasymin}\ P{\isacharunderscore}{\kern0pt}names{\isachardoublequoteclose}\isanewline
\ \ \ \ \isacommand{have}\isamarkupfalse%
\ {\isachardoublequoteopen}{\isasymUnion}RepFun{\isacharparenleft}{\kern0pt}domain{\isacharparenleft}{\kern0pt}x{\isacharparenright}{\kern0pt}{\isacharcomma}{\kern0pt}\ sep{\isacharunderscore}{\kern0pt}base{\isacharparenright}{\kern0pt}\ {\isasymsubseteq}\ P{\isacharunderscore}{\kern0pt}names{\isachardoublequoteclose}\ \isanewline
\ \ \ \ \ \ \isacommand{apply}\isamarkupfalse%
{\isacharparenleft}{\kern0pt}rule\ subsetI{\isacharcomma}{\kern0pt}\ simp{\isacharcomma}{\kern0pt}\ clarify{\isacharparenright}{\kern0pt}\isanewline
\ \ \ \ \ \ \isacommand{apply}\isamarkupfalse%
{\isacharparenleft}{\kern0pt}rename{\isacharunderscore}{\kern0pt}tac\ z\ y\ p{\isacharcomma}{\kern0pt}\ subgoal{\isacharunderscore}{\kern0pt}tac\ {\isachardoublequoteopen}y\ {\isasymin}\ P{\isacharunderscore}{\kern0pt}names{\isachardoublequoteclose}{\isacharparenright}{\kern0pt}\isanewline
\ \ \ \ \ \ \isacommand{using}\isamarkupfalse%
\ assms\ P{\isacharunderscore}{\kern0pt}name{\isacharunderscore}{\kern0pt}domain{\isacharunderscore}{\kern0pt}P{\isacharunderscore}{\kern0pt}name\isanewline
\ \ \ \ \ \ \isacommand{by}\isamarkupfalse%
\ auto\isanewline
\ \ \ \ \isacommand{then}\isamarkupfalse%
\ \isacommand{have}\isamarkupfalse%
\ {\isachardoublequoteopen}sep{\isacharunderscore}{\kern0pt}base{\isacharparenleft}{\kern0pt}x{\isacharparenright}{\kern0pt}\ {\isasymsubseteq}\ Pow{\isacharparenleft}{\kern0pt}P{\isacharunderscore}{\kern0pt}names\ {\isasymtimes}\ P{\isacharparenright}{\kern0pt}\ {\isasyminter}\ M{\isachardoublequoteclose}\ \isanewline
\ \ \ \ \ \ \isacommand{apply}\isamarkupfalse%
{\isacharparenleft}{\kern0pt}subst\ def{\isacharunderscore}{\kern0pt}sep{\isacharunderscore}{\kern0pt}base{\isacharparenright}{\kern0pt}\isanewline
\ \ \ \ \ \ \isacommand{using}\isamarkupfalse%
\ assms\ P{\isacharunderscore}{\kern0pt}name{\isacharunderscore}{\kern0pt}in{\isacharunderscore}{\kern0pt}M\ \isanewline
\ \ \ \ \ \ \ \isacommand{apply}\isamarkupfalse%
\ force\isanewline
\ \ \ \ \ \ \isacommand{apply}\isamarkupfalse%
\ force\isanewline
\ \ \ \ \ \ \isacommand{done}\isamarkupfalse%
\isanewline
\ \ \ \ \isacommand{then}\isamarkupfalse%
\ \isacommand{show}\isamarkupfalse%
\ {\isachardoublequoteopen}sep{\isacharunderscore}{\kern0pt}base{\isacharparenleft}{\kern0pt}x{\isacharparenright}{\kern0pt}\ {\isasymsubseteq}\ P{\isacharunderscore}{\kern0pt}names{\isachardoublequoteclose}\ \isanewline
\ \ \ \ \ \ \isacommand{using}\isamarkupfalse%
\ P{\isacharunderscore}{\kern0pt}name{\isacharunderscore}{\kern0pt}iff\ \isacommand{by}\isamarkupfalse%
\ auto\isanewline
\ \ \isacommand{qed}\isamarkupfalse%
\isanewline
\ \ \isacommand{then}\isamarkupfalse%
\ \isacommand{show}\isamarkupfalse%
\ {\isacharquery}{\kern0pt}thesis\ \isacommand{using}\isamarkupfalse%
\ assms\ \isacommand{by}\isamarkupfalse%
\ auto\isanewline
\isacommand{qed}\isamarkupfalse%
%
\endisatagproof
{\isafoldproof}%
%
\isadelimproof
\isanewline
%
\endisadelimproof
\ \ \isanewline
\isacommand{lemma}\isamarkupfalse%
\ sep{\isacharunderscore}{\kern0pt}base{\isacharunderscore}{\kern0pt}closed{\isacharunderscore}{\kern0pt}under{\isacharunderscore}{\kern0pt}Pn{\isacharunderscore}{\kern0pt}auto\ {\isacharcolon}{\kern0pt}\ \isanewline
\ \ \isakeyword{fixes}\ x{\isadigit{0}}\ x\ {\isasympi}\ \isanewline
\ \ \isakeyword{assumes}\ {\isachardoublequoteopen}x{\isadigit{0}}\ {\isasymin}\ P{\isacharunderscore}{\kern0pt}names{\isachardoublequoteclose}\ {\isachardoublequoteopen}x\ {\isasymin}\ sep{\isacharunderscore}{\kern0pt}base{\isacharparenleft}{\kern0pt}x{\isadigit{0}}{\isacharparenright}{\kern0pt}{\isachardoublequoteclose}\ {\isachardoublequoteopen}{\isasympi}\ {\isasymin}\ P{\isacharunderscore}{\kern0pt}auto{\isachardoublequoteclose}\ \ \ \isanewline
\ \ \isakeyword{shows}\ {\isachardoublequoteopen}Pn{\isacharunderscore}{\kern0pt}auto{\isacharparenleft}{\kern0pt}{\isasympi}{\isacharparenright}{\kern0pt}{\isacharbackquote}{\kern0pt}x\ {\isasymin}\ sep{\isacharunderscore}{\kern0pt}base{\isacharparenleft}{\kern0pt}x{\isadigit{0}}{\isacharparenright}{\kern0pt}{\isachardoublequoteclose}\ \isanewline
%
\isadelimproof
%
\endisadelimproof
%
\isatagproof
\isacommand{proof}\isamarkupfalse%
\ {\isacharminus}{\kern0pt}\ \isanewline
\ \ \isacommand{have}\isamarkupfalse%
\ main\ {\isacharcolon}{\kern0pt}\ {\isachardoublequoteopen}{\isasymAnd}x{\isadigit{0}}\ {\isachardot}{\kern0pt}\ x{\isadigit{0}}\ {\isasymin}\ P{\isacharunderscore}{\kern0pt}names\ {\isasymlongrightarrow}\ {\isacharparenleft}{\kern0pt}{\isasymforall}\ x\ {\isasymin}\ sep{\isacharunderscore}{\kern0pt}base{\isacharparenleft}{\kern0pt}x{\isadigit{0}}{\isacharparenright}{\kern0pt}{\isachardot}{\kern0pt}\ Pn{\isacharunderscore}{\kern0pt}auto{\isacharparenleft}{\kern0pt}{\isasympi}{\isacharparenright}{\kern0pt}{\isacharbackquote}{\kern0pt}x\ {\isasymin}\ sep{\isacharunderscore}{\kern0pt}base{\isacharparenleft}{\kern0pt}x{\isadigit{0}}{\isacharparenright}{\kern0pt}{\isacharparenright}{\kern0pt}{\isachardoublequoteclose}\ \isanewline
\ \ \isacommand{proof}\isamarkupfalse%
{\isacharparenleft}{\kern0pt}rule{\isacharunderscore}{\kern0pt}tac\ Q{\isacharequal}{\kern0pt}{\isachardoublequoteopen}{\isasymlambda}x{\isadigit{0}}\ {\isachardot}{\kern0pt}\ x{\isadigit{0}}\ {\isasymin}\ P{\isacharunderscore}{\kern0pt}names\ {\isasymlongrightarrow}\ {\isacharparenleft}{\kern0pt}{\isasymforall}\ x\ {\isasymin}\ sep{\isacharunderscore}{\kern0pt}base{\isacharparenleft}{\kern0pt}x{\isadigit{0}}{\isacharparenright}{\kern0pt}{\isachardot}{\kern0pt}\ Pn{\isacharunderscore}{\kern0pt}auto{\isacharparenleft}{\kern0pt}{\isasympi}{\isacharparenright}{\kern0pt}{\isacharbackquote}{\kern0pt}x\ {\isasymin}\ sep{\isacharunderscore}{\kern0pt}base{\isacharparenleft}{\kern0pt}x{\isadigit{0}}{\isacharparenright}{\kern0pt}{\isacharparenright}{\kern0pt}{\isachardoublequoteclose}\ \isakeyword{in}\ ed{\isacharunderscore}{\kern0pt}induction{\isacharcomma}{\kern0pt}\ rule\ impI{\isacharcomma}{\kern0pt}\ rule\ ballI{\isacharcomma}{\kern0pt}\ rename{\isacharunderscore}{\kern0pt}tac\ a\ x{\isadigit{0}}\ x{\isacharparenright}{\kern0pt}\isanewline
\ \ \ \ \isacommand{fix}\isamarkupfalse%
\ x{\isadigit{0}}\ x\ \isacommand{assume}\isamarkupfalse%
\ assms{\isadigit{1}}{\isacharcolon}{\kern0pt}\ {\isachardoublequoteopen}x{\isadigit{0}}\ {\isasymin}\ P{\isacharunderscore}{\kern0pt}names{\isachardoublequoteclose}\ {\isachardoublequoteopen}x\ {\isasymin}\ sep{\isacharunderscore}{\kern0pt}base{\isacharparenleft}{\kern0pt}x{\isadigit{0}}{\isacharparenright}{\kern0pt}{\isachardoublequoteclose}\ {\isachardoublequoteopen}{\isasymAnd}y{\isadigit{0}}{\isachardot}{\kern0pt}\ ed{\isacharparenleft}{\kern0pt}y{\isadigit{0}}{\isacharcomma}{\kern0pt}\ x{\isadigit{0}}{\isacharparenright}{\kern0pt}\ {\isasymLongrightarrow}\ y{\isadigit{0}}\ {\isasymin}\ P{\isacharunderscore}{\kern0pt}names\ {\isasymlongrightarrow}\ {\isacharparenleft}{\kern0pt}\ {\isasymforall}\ y\ {\isasymin}\ sep{\isacharunderscore}{\kern0pt}base{\isacharparenleft}{\kern0pt}y{\isadigit{0}}{\isacharparenright}{\kern0pt}{\isachardot}{\kern0pt}\ Pn{\isacharunderscore}{\kern0pt}auto{\isacharparenleft}{\kern0pt}{\isasympi}{\isacharparenright}{\kern0pt}\ {\isacharbackquote}{\kern0pt}\ y\ {\isasymin}\ sep{\isacharunderscore}{\kern0pt}base{\isacharparenleft}{\kern0pt}y{\isadigit{0}}{\isacharparenright}{\kern0pt}{\isacharparenright}{\kern0pt}{\isachardoublequoteclose}\isanewline
\isanewline
\ \ \ \ \isacommand{have}\isamarkupfalse%
\ xpname\ {\isacharcolon}{\kern0pt}\ {\isachardoublequoteopen}x\ {\isasymin}\ P{\isacharunderscore}{\kern0pt}names{\isachardoublequoteclose}\ \isacommand{using}\isamarkupfalse%
\ sep{\isacharunderscore}{\kern0pt}base{\isacharunderscore}{\kern0pt}subset\ assms{\isadigit{1}}\ \isacommand{by}\isamarkupfalse%
\ auto\ \isanewline
\ \ \ \ \isacommand{have}\isamarkupfalse%
\ pnautoeq\ {\isacharcolon}{\kern0pt}\ {\isachardoublequoteopen}Pn{\isacharunderscore}{\kern0pt}auto{\isacharparenleft}{\kern0pt}{\isasympi}{\isacharparenright}{\kern0pt}{\isacharbackquote}{\kern0pt}x\ {\isacharequal}{\kern0pt}\ {\isacharbraceleft}{\kern0pt}\ {\isacharless}{\kern0pt}Pn{\isacharunderscore}{\kern0pt}auto{\isacharparenleft}{\kern0pt}{\isasympi}{\isacharparenright}{\kern0pt}{\isacharbackquote}{\kern0pt}y{\isacharcomma}{\kern0pt}\ {\isasympi}{\isacharbackquote}{\kern0pt}p{\isachargreater}{\kern0pt}\ {\isachardot}{\kern0pt}\ {\isacharless}{\kern0pt}y{\isacharcomma}{\kern0pt}\ p{\isachargreater}{\kern0pt}\ {\isasymin}\ x\ {\isacharbraceright}{\kern0pt}{\isachardoublequoteclose}\ \isacommand{using}\isamarkupfalse%
\ Pn{\isacharunderscore}{\kern0pt}auto\ xpname\ \isacommand{by}\isamarkupfalse%
\ auto\isanewline
\ \ \ \ \isacommand{have}\isamarkupfalse%
\ deleq\ {\isacharcolon}{\kern0pt}\ {\isachardoublequoteopen}sep{\isacharunderscore}{\kern0pt}base{\isacharparenleft}{\kern0pt}x{\isadigit{0}}{\isacharparenright}{\kern0pt}\ {\isacharequal}{\kern0pt}\ Pow{\isacharparenleft}{\kern0pt}{\isasymUnion}RepFun{\isacharparenleft}{\kern0pt}domain{\isacharparenleft}{\kern0pt}x{\isadigit{0}}{\isacharparenright}{\kern0pt}{\isacharcomma}{\kern0pt}\ sep{\isacharunderscore}{\kern0pt}base{\isacharparenright}{\kern0pt}\ {\isasymtimes}\ P{\isacharparenright}{\kern0pt}\ {\isasyminter}\ M{\isachardoublequoteclose}\isanewline
\ \ \ \ \ \ \isacommand{using}\isamarkupfalse%
\ def{\isacharunderscore}{\kern0pt}sep{\isacharunderscore}{\kern0pt}base\ P{\isacharunderscore}{\kern0pt}name{\isacharunderscore}{\kern0pt}in{\isacharunderscore}{\kern0pt}M\ assms{\isadigit{1}}\ \isacommand{by}\isamarkupfalse%
\ auto\ \isanewline
\ \ \ \ \isacommand{have}\isamarkupfalse%
\ xsubset\ {\isacharcolon}{\kern0pt}\ {\isachardoublequoteopen}x\ {\isasymsubseteq}\ {\isasymUnion}RepFun{\isacharparenleft}{\kern0pt}domain{\isacharparenleft}{\kern0pt}x{\isadigit{0}}{\isacharparenright}{\kern0pt}{\isacharcomma}{\kern0pt}\ sep{\isacharunderscore}{\kern0pt}base{\isacharparenright}{\kern0pt}\ {\isasymtimes}\ P{\isachardoublequoteclose}\ \isacommand{using}\isamarkupfalse%
\ deleq\ assms{\isadigit{1}}\ \isacommand{by}\isamarkupfalse%
\ auto\ \ \isanewline
\isanewline
\ \ \ \ \isacommand{have}\isamarkupfalse%
\ domH{\isacharcolon}{\kern0pt}\ {\isachardoublequoteopen}{\isasymAnd}y{\isachardot}{\kern0pt}\ y\ {\isasymin}\ domain{\isacharparenleft}{\kern0pt}x{\isacharparenright}{\kern0pt}\ {\isasymLongrightarrow}\ Pn{\isacharunderscore}{\kern0pt}auto{\isacharparenleft}{\kern0pt}{\isasympi}{\isacharparenright}{\kern0pt}{\isacharbackquote}{\kern0pt}y\ {\isasymin}\ {\isasymUnion}RepFun{\isacharparenleft}{\kern0pt}domain{\isacharparenleft}{\kern0pt}x{\isadigit{0}}{\isacharparenright}{\kern0pt}{\isacharcomma}{\kern0pt}\ sep{\isacharunderscore}{\kern0pt}base{\isacharparenright}{\kern0pt}{\isachardoublequoteclose}\ \isanewline
\ \ \ \ \isacommand{proof}\isamarkupfalse%
\ {\isacharminus}{\kern0pt}\ \isanewline
\ \ \ \ \ \ \isacommand{fix}\isamarkupfalse%
\ y\ \isacommand{assume}\isamarkupfalse%
\ yindom\ {\isacharcolon}{\kern0pt}\ {\isachardoublequoteopen}y\ {\isasymin}\ domain{\isacharparenleft}{\kern0pt}x{\isacharparenright}{\kern0pt}{\isachardoublequoteclose}\ \isanewline
\ \ \ \ \ \ \isacommand{then}\isamarkupfalse%
\ \isacommand{obtain}\isamarkupfalse%
\ y{\isadigit{0}}\ \isakeyword{where}\ y{\isadigit{0}}H\ {\isacharcolon}{\kern0pt}\ {\isachardoublequoteopen}y{\isadigit{0}}\ {\isasymin}\ domain{\isacharparenleft}{\kern0pt}x{\isadigit{0}}{\isacharparenright}{\kern0pt}{\isachardoublequoteclose}\ {\isachardoublequoteopen}y\ {\isasymin}\ sep{\isacharunderscore}{\kern0pt}base{\isacharparenleft}{\kern0pt}y{\isadigit{0}}{\isacharparenright}{\kern0pt}{\isachardoublequoteclose}\ \isacommand{using}\isamarkupfalse%
\ assms\ xsubset\ \isacommand{by}\isamarkupfalse%
\ blast\isanewline
\ \ \ \ \ \ \isacommand{then}\isamarkupfalse%
\ \isacommand{have}\isamarkupfalse%
\ {\isachardoublequoteopen}Pn{\isacharunderscore}{\kern0pt}auto{\isacharparenleft}{\kern0pt}{\isasympi}{\isacharparenright}{\kern0pt}{\isacharbackquote}{\kern0pt}y\ {\isasymin}\ sep{\isacharunderscore}{\kern0pt}base{\isacharparenleft}{\kern0pt}y{\isadigit{0}}{\isacharparenright}{\kern0pt}{\isachardoublequoteclose}\ \isacommand{using}\isamarkupfalse%
\ assms{\isadigit{1}}\ P{\isacharunderscore}{\kern0pt}name{\isacharunderscore}{\kern0pt}domain{\isacharunderscore}{\kern0pt}P{\isacharunderscore}{\kern0pt}name\ ed{\isacharunderscore}{\kern0pt}def\ \isacommand{by}\isamarkupfalse%
\ auto\isanewline
\ \ \ \ \ \ \isacommand{then}\isamarkupfalse%
\ \isacommand{show}\isamarkupfalse%
\ {\isachardoublequoteopen}Pn{\isacharunderscore}{\kern0pt}auto{\isacharparenleft}{\kern0pt}{\isasympi}{\isacharparenright}{\kern0pt}{\isacharbackquote}{\kern0pt}y\ {\isasymin}\ {\isasymUnion}RepFun{\isacharparenleft}{\kern0pt}domain{\isacharparenleft}{\kern0pt}x{\isadigit{0}}{\isacharparenright}{\kern0pt}{\isacharcomma}{\kern0pt}\ sep{\isacharunderscore}{\kern0pt}base{\isacharparenright}{\kern0pt}{\isachardoublequoteclose}\ \isacommand{using}\isamarkupfalse%
\ y{\isadigit{0}}H\ \isacommand{by}\isamarkupfalse%
\ auto\ \isanewline
\ \ \ \ \isacommand{qed}\isamarkupfalse%
\isanewline
\isanewline
\ \ \ \ \isacommand{have}\isamarkupfalse%
\ H{\isacharcolon}{\kern0pt}\ {\isachardoublequoteopen}Pn{\isacharunderscore}{\kern0pt}auto{\isacharparenleft}{\kern0pt}{\isasympi}{\isacharparenright}{\kern0pt}{\isacharbackquote}{\kern0pt}x\ {\isasymsubseteq}\ {\isacharparenleft}{\kern0pt}{\isasymUnion}RepFun{\isacharparenleft}{\kern0pt}domain{\isacharparenleft}{\kern0pt}x{\isadigit{0}}{\isacharparenright}{\kern0pt}{\isacharcomma}{\kern0pt}\ sep{\isacharunderscore}{\kern0pt}base{\isacharparenright}{\kern0pt}\ {\isasymtimes}\ P{\isacharparenright}{\kern0pt}{\isachardoublequoteclose}\isanewline
\ \ \ \ \isacommand{proof}\isamarkupfalse%
{\isacharparenleft}{\kern0pt}rule\ subsetI{\isacharparenright}{\kern0pt}\isanewline
\ \ \ \ \ \ \isacommand{fix}\isamarkupfalse%
\ v\ \isacommand{assume}\isamarkupfalse%
\ {\isachardoublequoteopen}v\ {\isasymin}\ Pn{\isacharunderscore}{\kern0pt}auto{\isacharparenleft}{\kern0pt}{\isasympi}{\isacharparenright}{\kern0pt}{\isacharbackquote}{\kern0pt}x{\isachardoublequoteclose}\ \isanewline
\ \ \ \ \ \ \isacommand{then}\isamarkupfalse%
\ \isacommand{have}\isamarkupfalse%
\ {\isachardoublequoteopen}{\isasymexists}y\ p{\isachardot}{\kern0pt}\ {\isacharless}{\kern0pt}y{\isacharcomma}{\kern0pt}\ p{\isachargreater}{\kern0pt}\ {\isasymin}\ x\ {\isasymand}\ v\ {\isacharequal}{\kern0pt}\ {\isacharless}{\kern0pt}Pn{\isacharunderscore}{\kern0pt}auto{\isacharparenleft}{\kern0pt}{\isasympi}{\isacharparenright}{\kern0pt}{\isacharbackquote}{\kern0pt}y{\isacharcomma}{\kern0pt}\ {\isasympi}{\isacharbackquote}{\kern0pt}p{\isachargreater}{\kern0pt}{\isachardoublequoteclose}\ \isanewline
\ \ \ \ \ \ \ \ \isacommand{apply}\isamarkupfalse%
{\isacharparenleft}{\kern0pt}rule{\isacharunderscore}{\kern0pt}tac\ pair{\isacharunderscore}{\kern0pt}rel{\isacharunderscore}{\kern0pt}arg{\isacharparenright}{\kern0pt}\isanewline
\ \ \ \ \ \ \ \ \isacommand{using}\isamarkupfalse%
\ xpname\ relation{\isacharunderscore}{\kern0pt}P{\isacharunderscore}{\kern0pt}name\ pnautoeq\ \isanewline
\ \ \ \ \ \ \ \ \isacommand{by}\isamarkupfalse%
\ auto\isanewline
\ \ \ \ \ \ \isacommand{then}\isamarkupfalse%
\ \isacommand{obtain}\isamarkupfalse%
\ y\ p\ \isakeyword{where}\ ypH{\isacharcolon}{\kern0pt}\ {\isachardoublequoteopen}v\ {\isacharequal}{\kern0pt}\ {\isacharless}{\kern0pt}Pn{\isacharunderscore}{\kern0pt}auto{\isacharparenleft}{\kern0pt}{\isasympi}{\isacharparenright}{\kern0pt}{\isacharbackquote}{\kern0pt}y{\isacharcomma}{\kern0pt}\ {\isasympi}{\isacharbackquote}{\kern0pt}p{\isachargreater}{\kern0pt}{\isachardoublequoteclose}\ {\isachardoublequoteopen}{\isacharless}{\kern0pt}y{\isacharcomma}{\kern0pt}\ p{\isachargreater}{\kern0pt}\ {\isasymin}\ x{\isachardoublequoteclose}\ \isacommand{using}\isamarkupfalse%
\ pnautoeq\ \isacommand{by}\isamarkupfalse%
\ blast\isanewline
\ \ \ \ \ \ \isacommand{then}\isamarkupfalse%
\ \isacommand{have}\isamarkupfalse%
\ H\ {\isacharcolon}{\kern0pt}\ {\isachardoublequoteopen}Pn{\isacharunderscore}{\kern0pt}auto{\isacharparenleft}{\kern0pt}{\isasympi}{\isacharparenright}{\kern0pt}{\isacharbackquote}{\kern0pt}y\ {\isasymin}\ {\isasymUnion}RepFun{\isacharparenleft}{\kern0pt}domain{\isacharparenleft}{\kern0pt}x{\isadigit{0}}{\isacharparenright}{\kern0pt}{\isacharcomma}{\kern0pt}\ sep{\isacharunderscore}{\kern0pt}base{\isacharparenright}{\kern0pt}{\isachardoublequoteclose}\ \isacommand{using}\isamarkupfalse%
\ ypH\ domH\ \isacommand{by}\isamarkupfalse%
\ auto\isanewline
\ \ \ \ \ \ \isacommand{have}\isamarkupfalse%
\ {\isachardoublequoteopen}p\ {\isasymin}\ P{\isachardoublequoteclose}\ \isacommand{using}\isamarkupfalse%
\ xpname\ P{\isacharunderscore}{\kern0pt}name{\isacharunderscore}{\kern0pt}iff\ ypH\ \isacommand{by}\isamarkupfalse%
\ auto\ \isanewline
\ \ \ \ \ \ \isacommand{then}\isamarkupfalse%
\ \isacommand{have}\isamarkupfalse%
\ {\isachardoublequoteopen}{\isasympi}{\isacharbackquote}{\kern0pt}p\ {\isasymin}\ P{\isachardoublequoteclose}\ \isacommand{using}\isamarkupfalse%
\ assms\ P{\isacharunderscore}{\kern0pt}auto{\isacharunderscore}{\kern0pt}value\ P{\isacharunderscore}{\kern0pt}auto{\isacharunderscore}{\kern0pt}def\ \isacommand{by}\isamarkupfalse%
\ auto\ \isanewline
\ \ \ \ \ \ \isacommand{then}\isamarkupfalse%
\ \isacommand{show}\isamarkupfalse%
\ {\isachardoublequoteopen}v\ {\isasymin}\ {\isacharparenleft}{\kern0pt}{\isasymUnion}RepFun{\isacharparenleft}{\kern0pt}domain{\isacharparenleft}{\kern0pt}x{\isadigit{0}}{\isacharparenright}{\kern0pt}{\isacharcomma}{\kern0pt}\ sep{\isacharunderscore}{\kern0pt}base{\isacharparenright}{\kern0pt}\ {\isasymtimes}\ P{\isacharparenright}{\kern0pt}{\isachardoublequoteclose}\ \isacommand{using}\isamarkupfalse%
\ H\ ypH\ \isacommand{by}\isamarkupfalse%
\ auto\isanewline
\ \ \ \ \isacommand{qed}\isamarkupfalse%
\isanewline
\isanewline
\ \ \ \ \isacommand{show}\isamarkupfalse%
\ {\isachardoublequoteopen}Pn{\isacharunderscore}{\kern0pt}auto{\isacharparenleft}{\kern0pt}{\isasympi}{\isacharparenright}{\kern0pt}\ {\isacharbackquote}{\kern0pt}\ x\ {\isasymin}\ sep{\isacharunderscore}{\kern0pt}base{\isacharparenleft}{\kern0pt}x{\isadigit{0}}{\isacharparenright}{\kern0pt}{\isachardoublequoteclose}\isanewline
\ \ \ \ \ \ \isacommand{apply}\isamarkupfalse%
{\isacharparenleft}{\kern0pt}subst\ deleq{\isacharparenright}{\kern0pt}\isanewline
\ \ \ \ \ \ \isacommand{using}\isamarkupfalse%
\ H\ \isanewline
\ \ \ \ \ \ \isacommand{apply}\isamarkupfalse%
\ simp\isanewline
\ \ \ \ \ \ \isacommand{apply}\isamarkupfalse%
{\isacharparenleft}{\kern0pt}rule\ Pn{\isacharunderscore}{\kern0pt}auto{\isacharunderscore}{\kern0pt}value{\isacharunderscore}{\kern0pt}in{\isacharunderscore}{\kern0pt}M{\isacharparenright}{\kern0pt}\isanewline
\ \ \ \ \ \ \isacommand{using}\isamarkupfalse%
\ assms\ P{\isacharunderscore}{\kern0pt}auto{\isacharunderscore}{\kern0pt}def\ xpname\isanewline
\ \ \ \ \ \ \isacommand{by}\isamarkupfalse%
\ auto\isanewline
\ \ \isacommand{qed}\isamarkupfalse%
\isanewline
\isanewline
\ \ \isacommand{then}\isamarkupfalse%
\ \isacommand{show}\isamarkupfalse%
\ {\isacharquery}{\kern0pt}thesis\ \isacommand{using}\isamarkupfalse%
\ assms\ \isacommand{by}\isamarkupfalse%
\ auto\isanewline
\isacommand{qed}\isamarkupfalse%
%
\endisatagproof
{\isafoldproof}%
%
\isadelimproof
\isanewline
%
\endisadelimproof
\isanewline
\isacommand{lemma}\isamarkupfalse%
\ sep{\isacharunderscore}{\kern0pt}base{\isacharunderscore}{\kern0pt}closed{\isacharunderscore}{\kern0pt}under{\isacharunderscore}{\kern0pt}Pn{\isacharunderscore}{\kern0pt}auto{\isacharprime}{\kern0pt}\ {\isacharcolon}{\kern0pt}\ \isanewline
\ \ \isakeyword{fixes}\ x\ x{\isadigit{0}}\ {\isasympi}\isanewline
\ \ \isakeyword{assumes}\ {\isachardoublequoteopen}x\ {\isasymin}\ P{\isacharunderscore}{\kern0pt}names{\isachardoublequoteclose}\ {\isachardoublequoteopen}x{\isadigit{0}}\ {\isasymin}\ P{\isacharunderscore}{\kern0pt}names{\isachardoublequoteclose}\ {\isachardoublequoteopen}{\isasympi}\ {\isasymin}\ P{\isacharunderscore}{\kern0pt}auto{\isachardoublequoteclose}\ \isanewline
\ \ \isakeyword{shows}\ {\isachardoublequoteopen}x\ {\isasymin}\ sep{\isacharunderscore}{\kern0pt}base{\isacharparenleft}{\kern0pt}x{\isadigit{0}}{\isacharparenright}{\kern0pt}\ {\isasymlongleftrightarrow}\ Pn{\isacharunderscore}{\kern0pt}auto{\isacharparenleft}{\kern0pt}{\isasympi}{\isacharparenright}{\kern0pt}{\isacharbackquote}{\kern0pt}x\ {\isasymin}\ sep{\isacharunderscore}{\kern0pt}base{\isacharparenleft}{\kern0pt}x{\isadigit{0}}{\isacharparenright}{\kern0pt}{\isachardoublequoteclose}\ \isanewline
%
\isadelimproof
%
\endisadelimproof
%
\isatagproof
\isacommand{proof}\isamarkupfalse%
\ {\isacharparenleft}{\kern0pt}rule\ iffI{\isacharparenright}{\kern0pt}\ \isanewline
\ \ \isacommand{assume}\isamarkupfalse%
\ {\isachardoublequoteopen}x\ {\isasymin}\ sep{\isacharunderscore}{\kern0pt}base{\isacharparenleft}{\kern0pt}x{\isadigit{0}}{\isacharparenright}{\kern0pt}{\isachardoublequoteclose}\ \isanewline
\ \ \isacommand{then}\isamarkupfalse%
\ \isacommand{show}\isamarkupfalse%
\ {\isachardoublequoteopen}Pn{\isacharunderscore}{\kern0pt}auto{\isacharparenleft}{\kern0pt}{\isasympi}{\isacharparenright}{\kern0pt}{\isacharbackquote}{\kern0pt}x\ {\isasymin}\ sep{\isacharunderscore}{\kern0pt}base{\isacharparenleft}{\kern0pt}x{\isadigit{0}}{\isacharparenright}{\kern0pt}{\isachardoublequoteclose}\ \isanewline
\ \ \ \ \isacommand{apply}\isamarkupfalse%
{\isacharparenleft}{\kern0pt}rule{\isacharunderscore}{\kern0pt}tac\ sep{\isacharunderscore}{\kern0pt}base{\isacharunderscore}{\kern0pt}closed{\isacharunderscore}{\kern0pt}under{\isacharunderscore}{\kern0pt}Pn{\isacharunderscore}{\kern0pt}auto{\isacharparenright}{\kern0pt}\isanewline
\ \ \ \ \isacommand{using}\isamarkupfalse%
\ assms\ \isanewline
\ \ \ \ \isacommand{by}\isamarkupfalse%
\ auto\ \isanewline
\isacommand{next}\isamarkupfalse%
\ \isanewline
\ \ \isacommand{assume}\isamarkupfalse%
\ assm{\isadigit{1}}\ {\isacharcolon}{\kern0pt}\ {\isachardoublequoteopen}Pn{\isacharunderscore}{\kern0pt}auto{\isacharparenleft}{\kern0pt}{\isasympi}{\isacharparenright}{\kern0pt}\ {\isacharbackquote}{\kern0pt}\ x\ {\isasymin}\ sep{\isacharunderscore}{\kern0pt}base{\isacharparenleft}{\kern0pt}x{\isadigit{0}}{\isacharparenright}{\kern0pt}{\isachardoublequoteclose}\ \isanewline
\ \ \isacommand{have}\isamarkupfalse%
\ {\isachardoublequoteopen}converse{\isacharparenleft}{\kern0pt}{\isasympi}{\isacharparenright}{\kern0pt}\ {\isasymin}\ bij{\isacharparenleft}{\kern0pt}P{\isacharcomma}{\kern0pt}\ P{\isacharparenright}{\kern0pt}{\isachardoublequoteclose}\ \isanewline
\ \ \ \ \isacommand{apply}\isamarkupfalse%
{\isacharparenleft}{\kern0pt}rule\ bij{\isacharunderscore}{\kern0pt}converse{\isacharunderscore}{\kern0pt}bij{\isacharparenright}{\kern0pt}\ \isanewline
\ \ \ \ \isacommand{using}\isamarkupfalse%
\ assms\ P{\isacharunderscore}{\kern0pt}auto{\isacharunderscore}{\kern0pt}def\ is{\isacharunderscore}{\kern0pt}P{\isacharunderscore}{\kern0pt}auto{\isacharunderscore}{\kern0pt}def\ \isanewline
\ \ \ \ \isacommand{by}\isamarkupfalse%
\ auto\isanewline
\ \ \isacommand{then}\isamarkupfalse%
\ \isacommand{have}\isamarkupfalse%
\ H{\isacharcolon}{\kern0pt}\ {\isachardoublequoteopen}converse{\isacharparenleft}{\kern0pt}{\isasympi}{\isacharparenright}{\kern0pt}\ {\isasymin}\ P{\isacharunderscore}{\kern0pt}auto{\isachardoublequoteclose}\ \isanewline
\ \ \ \ \isacommand{unfolding}\isamarkupfalse%
\ P{\isacharunderscore}{\kern0pt}auto{\isacharunderscore}{\kern0pt}def\isanewline
\ \ \ \ \isacommand{apply}\isamarkupfalse%
\ simp\isanewline
\ \ \ \ \isacommand{apply}\isamarkupfalse%
{\isacharparenleft}{\kern0pt}rule\ conjI{\isacharparenright}{\kern0pt}\isanewline
\ \ \ \ \isacommand{using}\isamarkupfalse%
\ bij{\isacharunderscore}{\kern0pt}def\ inj{\isacharunderscore}{\kern0pt}def\ \isanewline
\ \ \ \ \ \isacommand{apply}\isamarkupfalse%
\ force\isanewline
\ \ \ \ \isacommand{using}\isamarkupfalse%
\ P{\isacharunderscore}{\kern0pt}auto{\isacharunderscore}{\kern0pt}converse\ assms\ P{\isacharunderscore}{\kern0pt}auto{\isacharunderscore}{\kern0pt}def\isanewline
\ \ \ \ \isacommand{by}\isamarkupfalse%
\ auto\ \isanewline
\ \ \isacommand{have}\isamarkupfalse%
\ H{\isadigit{1}}{\isacharcolon}{\kern0pt}\ {\isachardoublequoteopen}Pn{\isacharunderscore}{\kern0pt}auto{\isacharparenleft}{\kern0pt}converse{\isacharparenleft}{\kern0pt}{\isasympi}{\isacharparenright}{\kern0pt}{\isacharparenright}{\kern0pt}\ {\isacharbackquote}{\kern0pt}\ {\isacharparenleft}{\kern0pt}Pn{\isacharunderscore}{\kern0pt}auto{\isacharparenleft}{\kern0pt}{\isasympi}{\isacharparenright}{\kern0pt}{\isacharbackquote}{\kern0pt}x{\isacharparenright}{\kern0pt}\ {\isasymin}\ sep{\isacharunderscore}{\kern0pt}base{\isacharparenleft}{\kern0pt}x{\isadigit{0}}{\isacharparenright}{\kern0pt}{\isachardoublequoteclose}\ \isanewline
\ \ \ \ \isacommand{apply}\isamarkupfalse%
{\isacharparenleft}{\kern0pt}rule\ sep{\isacharunderscore}{\kern0pt}base{\isacharunderscore}{\kern0pt}closed{\isacharunderscore}{\kern0pt}under{\isacharunderscore}{\kern0pt}Pn{\isacharunderscore}{\kern0pt}auto{\isacharparenright}{\kern0pt}\isanewline
\ \ \ \ \isacommand{using}\isamarkupfalse%
\ assms\ assm{\isadigit{1}}\ H\ \isanewline
\ \ \ \ \isacommand{by}\isamarkupfalse%
\ auto\ \isanewline
\ \ \isanewline
\ \ \isacommand{have}\isamarkupfalse%
\ {\isachardoublequoteopen}Pn{\isacharunderscore}{\kern0pt}auto{\isacharparenleft}{\kern0pt}converse{\isacharparenleft}{\kern0pt}{\isasympi}{\isacharparenright}{\kern0pt}{\isacharparenright}{\kern0pt}\ {\isacharbackquote}{\kern0pt}\ {\isacharparenleft}{\kern0pt}Pn{\isacharunderscore}{\kern0pt}auto{\isacharparenleft}{\kern0pt}{\isasympi}{\isacharparenright}{\kern0pt}{\isacharbackquote}{\kern0pt}x{\isacharparenright}{\kern0pt}\ {\isacharequal}{\kern0pt}\ converse{\isacharparenleft}{\kern0pt}Pn{\isacharunderscore}{\kern0pt}auto{\isacharparenleft}{\kern0pt}{\isasympi}{\isacharparenright}{\kern0pt}{\isacharparenright}{\kern0pt}\ {\isacharbackquote}{\kern0pt}\ {\isacharparenleft}{\kern0pt}Pn{\isacharunderscore}{\kern0pt}auto{\isacharparenleft}{\kern0pt}{\isasympi}{\isacharparenright}{\kern0pt}{\isacharbackquote}{\kern0pt}x{\isacharparenright}{\kern0pt}{\isachardoublequoteclose}\ \isanewline
\ \ \ \ \isacommand{apply}\isamarkupfalse%
{\isacharparenleft}{\kern0pt}subst\ Pn{\isacharunderscore}{\kern0pt}auto{\isacharunderscore}{\kern0pt}converse{\isacharparenright}{\kern0pt}\isanewline
\ \ \ \ \isacommand{using}\isamarkupfalse%
\ assms\ P{\isacharunderscore}{\kern0pt}auto{\isacharunderscore}{\kern0pt}def\ \isanewline
\ \ \ \ \isacommand{by}\isamarkupfalse%
\ auto\isanewline
\ \ \isacommand{also}\isamarkupfalse%
\ \isacommand{have}\isamarkupfalse%
\ {\isachardoublequoteopen}{\isachardot}{\kern0pt}{\isachardot}{\kern0pt}{\isachardot}{\kern0pt}\ {\isacharequal}{\kern0pt}\ {\isacharparenleft}{\kern0pt}converse{\isacharparenleft}{\kern0pt}Pn{\isacharunderscore}{\kern0pt}auto{\isacharparenleft}{\kern0pt}{\isasympi}{\isacharparenright}{\kern0pt}{\isacharparenright}{\kern0pt}\ O\ Pn{\isacharunderscore}{\kern0pt}auto{\isacharparenleft}{\kern0pt}{\isasympi}{\isacharparenright}{\kern0pt}{\isacharparenright}{\kern0pt}\ {\isacharbackquote}{\kern0pt}\ x{\isachardoublequoteclose}\ \isanewline
\ \ \ \ \isacommand{apply}\isamarkupfalse%
{\isacharparenleft}{\kern0pt}rule\ eq{\isacharunderscore}{\kern0pt}flip{\isacharcomma}{\kern0pt}\ rule\ comp{\isacharunderscore}{\kern0pt}fun{\isacharunderscore}{\kern0pt}apply{\isacharparenright}{\kern0pt}\isanewline
\ \ \ \ \ \isacommand{apply}\isamarkupfalse%
{\isacharparenleft}{\kern0pt}rule\ Pn{\isacharunderscore}{\kern0pt}auto{\isacharunderscore}{\kern0pt}type{\isacharparenright}{\kern0pt}\isanewline
\ \ \ \ \isacommand{using}\isamarkupfalse%
\ assms\ P{\isacharunderscore}{\kern0pt}auto{\isacharunderscore}{\kern0pt}def\isanewline
\ \ \ \ \isacommand{by}\isamarkupfalse%
\ auto\ \isanewline
\ \ \isacommand{also}\isamarkupfalse%
\ \isacommand{have}\isamarkupfalse%
\ {\isachardoublequoteopen}{\isachardot}{\kern0pt}{\isachardot}{\kern0pt}{\isachardot}{\kern0pt}\ {\isacharequal}{\kern0pt}\ id{\isacharparenleft}{\kern0pt}P{\isacharunderscore}{\kern0pt}names{\isacharparenright}{\kern0pt}{\isacharbackquote}{\kern0pt}x{\isachardoublequoteclose}\ \isanewline
\ \ \ \ \isacommand{apply}\isamarkupfalse%
{\isacharparenleft}{\kern0pt}subst\ left{\isacharunderscore}{\kern0pt}comp{\isacharunderscore}{\kern0pt}inverse{\isacharparenright}{\kern0pt}\isanewline
\ \ \ \ \isacommand{using}\isamarkupfalse%
\ assms\ Pn{\isacharunderscore}{\kern0pt}auto{\isacharunderscore}{\kern0pt}bij\ bij{\isacharunderscore}{\kern0pt}is{\isacharunderscore}{\kern0pt}inj\ P{\isacharunderscore}{\kern0pt}auto{\isacharunderscore}{\kern0pt}def\ \isanewline
\ \ \ \ \isacommand{by}\isamarkupfalse%
\ auto\isanewline
\ \ \isacommand{also}\isamarkupfalse%
\ \isacommand{have}\isamarkupfalse%
\ {\isachardoublequoteopen}{\isachardot}{\kern0pt}{\isachardot}{\kern0pt}{\isachardot}{\kern0pt}\ {\isacharequal}{\kern0pt}\ x{\isachardoublequoteclose}\ \isacommand{using}\isamarkupfalse%
\ assms\ \isacommand{by}\isamarkupfalse%
\ auto\ \isanewline
\isanewline
\ \ \isacommand{finally}\isamarkupfalse%
\ \isacommand{show}\isamarkupfalse%
\ {\isachardoublequoteopen}x\ {\isasymin}\ sep{\isacharunderscore}{\kern0pt}base{\isacharparenleft}{\kern0pt}x{\isadigit{0}}{\isacharparenright}{\kern0pt}{\isachardoublequoteclose}\ \isacommand{using}\isamarkupfalse%
\ H{\isadigit{1}}\ \isacommand{by}\isamarkupfalse%
\ auto\isanewline
\isacommand{qed}\isamarkupfalse%
%
\endisatagproof
{\isafoldproof}%
%
\isadelimproof
\isanewline
%
\endisadelimproof
\isanewline
\isacommand{lemma}\isamarkupfalse%
\ strong{\isacharunderscore}{\kern0pt}replacement{\isacharunderscore}{\kern0pt}sep{\isacharunderscore}{\kern0pt}base\ {\isacharcolon}{\kern0pt}\ \isanewline
\ \ \isakeyword{shows}\ {\isachardoublequoteopen}strong{\isacharunderscore}{\kern0pt}replacement{\isacharparenleft}{\kern0pt}{\isacharhash}{\kern0pt}{\isacharhash}{\kern0pt}M{\isacharcomma}{\kern0pt}\ {\isasymlambda}x\ y{\isachardot}{\kern0pt}\ y\ {\isacharequal}{\kern0pt}\ sep{\isacharunderscore}{\kern0pt}base{\isacharparenleft}{\kern0pt}x{\isacharparenright}{\kern0pt}{\isacharparenright}{\kern0pt}{\isachardoublequoteclose}\ \isanewline
%
\isadelimproof
%
\endisadelimproof
%
\isatagproof
\isacommand{proof}\isamarkupfalse%
\ {\isacharminus}{\kern0pt}\isanewline
\ \ \isacommand{have}\isamarkupfalse%
\ H{\isacharcolon}{\kern0pt}\ {\isachardoublequoteopen}strong{\isacharunderscore}{\kern0pt}replacement{\isacharparenleft}{\kern0pt}{\isacharhash}{\kern0pt}{\isacharhash}{\kern0pt}M{\isacharcomma}{\kern0pt}\ {\isasymlambda}x\ y{\isachardot}{\kern0pt}\ sats{\isacharparenleft}{\kern0pt}M{\isacharcomma}{\kern0pt}\ is{\isacharunderscore}{\kern0pt}sep{\isacharunderscore}{\kern0pt}base{\isacharunderscore}{\kern0pt}fm{\isacharparenleft}{\kern0pt}{\isadigit{0}}{\isacharcomma}{\kern0pt}\ {\isadigit{2}}{\isacharcomma}{\kern0pt}\ {\isadigit{1}}{\isacharparenright}{\kern0pt}{\isacharcomma}{\kern0pt}\ {\isacharbrackleft}{\kern0pt}x{\isacharcomma}{\kern0pt}\ y{\isacharbrackright}{\kern0pt}\ {\isacharat}{\kern0pt}\ {\isacharbrackleft}{\kern0pt}P{\isacharbrackright}{\kern0pt}{\isacharparenright}{\kern0pt}{\isacharparenright}{\kern0pt}{\isachardoublequoteclose}\ \isanewline
\ \ \ \ \isacommand{apply}\isamarkupfalse%
{\isacharparenleft}{\kern0pt}rule\ replacement{\isacharunderscore}{\kern0pt}ax{\isacharparenright}{\kern0pt}\isanewline
\ \ \ \ \ \ \isacommand{apply}\isamarkupfalse%
{\isacharparenleft}{\kern0pt}simp\ add{\isacharcolon}{\kern0pt}\ is{\isacharunderscore}{\kern0pt}sep{\isacharunderscore}{\kern0pt}base{\isacharunderscore}{\kern0pt}fm{\isacharunderscore}{\kern0pt}def{\isacharcomma}{\kern0pt}\ rule\ is{\isacharunderscore}{\kern0pt}memrel{\isacharunderscore}{\kern0pt}wftrec{\isacharunderscore}{\kern0pt}fm{\isacharunderscore}{\kern0pt}type{\isacharcomma}{\kern0pt}\ simp\ add{\isacharcolon}{\kern0pt}\ Hsep{\isacharunderscore}{\kern0pt}base{\isacharunderscore}{\kern0pt}M{\isacharunderscore}{\kern0pt}fm{\isacharunderscore}{\kern0pt}def{\isacharparenright}{\kern0pt}\isanewline
\ \ \ \ \isacommand{using}\isamarkupfalse%
\ P{\isacharunderscore}{\kern0pt}in{\isacharunderscore}{\kern0pt}M\ \isanewline
\ \ \ \ \ \ \ \ \isacommand{apply}\isamarkupfalse%
\ auto{\isacharbrackleft}{\kern0pt}{\isadigit{4}}{\isacharbrackright}{\kern0pt}\isanewline
\ \ \ \ \isacommand{apply}\isamarkupfalse%
\ simp\isanewline
\ \ \ \ \isacommand{apply}\isamarkupfalse%
{\isacharparenleft}{\kern0pt}simp\ add{\isacharcolon}{\kern0pt}is{\isacharunderscore}{\kern0pt}sep{\isacharunderscore}{\kern0pt}base{\isacharunderscore}{\kern0pt}fm{\isacharunderscore}{\kern0pt}def{\isacharcomma}{\kern0pt}\ rule\ le{\isacharunderscore}{\kern0pt}trans{\isacharcomma}{\kern0pt}\ rule\ arity{\isacharunderscore}{\kern0pt}is{\isacharunderscore}{\kern0pt}memrel{\isacharunderscore}{\kern0pt}wftrec{\isacharunderscore}{\kern0pt}fm{\isacharparenright}{\kern0pt}\isanewline
\ \ \ \ \ \ \ \ \ \isacommand{apply}\isamarkupfalse%
{\isacharparenleft}{\kern0pt}simp\ add{\isacharcolon}{\kern0pt}\ Hsep{\isacharunderscore}{\kern0pt}base{\isacharunderscore}{\kern0pt}M{\isacharunderscore}{\kern0pt}fm{\isacharunderscore}{\kern0pt}def{\isacharcomma}{\kern0pt}\ simp\ add{\isacharcolon}{\kern0pt}Hsep{\isacharunderscore}{\kern0pt}base{\isacharunderscore}{\kern0pt}M{\isacharunderscore}{\kern0pt}fm{\isacharunderscore}{\kern0pt}def{\isacharparenright}{\kern0pt}\isanewline
\ \ \ \ \ \ \ \ \isacommand{apply}\isamarkupfalse%
{\isacharparenleft}{\kern0pt}simp\ del{\isacharcolon}{\kern0pt}FOL{\isacharunderscore}{\kern0pt}sats{\isacharunderscore}{\kern0pt}iff\ pair{\isacharunderscore}{\kern0pt}abs\ add{\isacharcolon}{\kern0pt}\ fm{\isacharunderscore}{\kern0pt}defs\ nat{\isacharunderscore}{\kern0pt}simp{\isacharunderscore}{\kern0pt}union{\isacharparenright}{\kern0pt}\isanewline
\ \ \ \ \ \ \ \isacommand{apply}\isamarkupfalse%
\ auto{\isacharbrackleft}{\kern0pt}{\isadigit{3}}{\isacharbrackright}{\kern0pt}\isanewline
\ \ \ \ \isacommand{apply}\isamarkupfalse%
{\isacharparenleft}{\kern0pt}rule\ Un{\isacharunderscore}{\kern0pt}least{\isacharunderscore}{\kern0pt}lt{\isacharparenright}{\kern0pt}{\isacharplus}{\kern0pt}\isanewline
\ \ \ \ \isacommand{by}\isamarkupfalse%
\ auto\isanewline
\ \ \isacommand{show}\isamarkupfalse%
\ {\isacharquery}{\kern0pt}thesis\isanewline
\ \ \ \ \isacommand{apply}\isamarkupfalse%
{\isacharparenleft}{\kern0pt}rule\ iffD{\isadigit{1}}{\isacharcomma}{\kern0pt}\ rule\ strong{\isacharunderscore}{\kern0pt}replacement{\isacharunderscore}{\kern0pt}cong{\isacharparenright}{\kern0pt}\isanewline
\ \ \ \ \ \isacommand{apply}\isamarkupfalse%
{\isacharparenleft}{\kern0pt}rename{\isacharunderscore}{\kern0pt}tac\ x\ y{\isacharcomma}{\kern0pt}\ rule{\isacharunderscore}{\kern0pt}tac\ env\ {\isacharequal}{\kern0pt}\ {\isachardoublequoteopen}{\isacharbrackleft}{\kern0pt}x{\isacharcomma}{\kern0pt}\ y{\isacharbrackright}{\kern0pt}\ {\isacharat}{\kern0pt}\ {\isacharbrackleft}{\kern0pt}P{\isacharbrackright}{\kern0pt}{\isachardoublequoteclose}\ \isakeyword{and}\ i{\isacharequal}{\kern0pt}{\isadigit{0}}\ \isakeyword{and}\ j{\isacharequal}{\kern0pt}{\isadigit{2}}\ \isakeyword{and}\ k{\isacharequal}{\kern0pt}{\isadigit{1}}\ \isakeyword{in}\ sats{\isacharunderscore}{\kern0pt}is{\isacharunderscore}{\kern0pt}sep{\isacharunderscore}{\kern0pt}base{\isacharunderscore}{\kern0pt}fm{\isacharunderscore}{\kern0pt}iff{\isacharparenright}{\kern0pt}\isanewline
\ \ \ \ \isacommand{using}\isamarkupfalse%
\ P{\isacharunderscore}{\kern0pt}in{\isacharunderscore}{\kern0pt}M\isanewline
\ \ \ \ \ \ \ \ \ \ \ \isacommand{apply}\isamarkupfalse%
\ auto{\isacharbrackleft}{\kern0pt}{\isadigit{8}}{\isacharbrackright}{\kern0pt}\isanewline
\ \ \ \ \isacommand{using}\isamarkupfalse%
\ H\ \isanewline
\ \ \ \ \isacommand{by}\isamarkupfalse%
\ auto\isanewline
\isacommand{qed}\isamarkupfalse%
%
\endisatagproof
{\isafoldproof}%
%
\isadelimproof
\isanewline
%
\endisadelimproof
\isanewline
\isacommand{lemma}\isamarkupfalse%
\ sep{\isacharunderscore}{\kern0pt}base{\isacharunderscore}{\kern0pt}repfun{\isacharunderscore}{\kern0pt}in{\isacharunderscore}{\kern0pt}M\ {\isacharcolon}{\kern0pt}\ \isanewline
\ \ \isakeyword{fixes}\ A\ \isanewline
\ \ \isakeyword{assumes}\ {\isachardoublequoteopen}A\ {\isasymin}\ M{\isachardoublequoteclose}\ \isanewline
\ \ \isakeyword{shows}\ {\isachardoublequoteopen}{\isacharbraceleft}{\kern0pt}\ sep{\isacharunderscore}{\kern0pt}base{\isacharparenleft}{\kern0pt}x{\isacharparenright}{\kern0pt}{\isachardot}{\kern0pt}\ x\ {\isasymin}\ A\ {\isacharbraceright}{\kern0pt}\ {\isasymin}\ M{\isachardoublequoteclose}\isanewline
%
\isadelimproof
\isanewline
\ \ %
\endisadelimproof
%
\isatagproof
\isacommand{apply}\isamarkupfalse%
{\isacharparenleft}{\kern0pt}rule\ to{\isacharunderscore}{\kern0pt}rin{\isacharcomma}{\kern0pt}\ rule\ RepFun{\isacharunderscore}{\kern0pt}closed{\isacharcomma}{\kern0pt}\ rule\ strong{\isacharunderscore}{\kern0pt}replacement{\isacharunderscore}{\kern0pt}sep{\isacharunderscore}{\kern0pt}base{\isacharcomma}{\kern0pt}\ simp\ add{\isacharcolon}{\kern0pt}assms{\isacharparenright}{\kern0pt}\isanewline
\ \ \isacommand{apply}\isamarkupfalse%
{\isacharparenleft}{\kern0pt}rule\ ballI{\isacharcomma}{\kern0pt}\ simp{\isacharcomma}{\kern0pt}\ rule\ sep{\isacharunderscore}{\kern0pt}base{\isacharunderscore}{\kern0pt}in{\isacharunderscore}{\kern0pt}M{\isacharparenright}{\kern0pt}\isanewline
\ \ \isacommand{using}\isamarkupfalse%
\ assms\ transM\ \isanewline
\ \ \isacommand{by}\isamarkupfalse%
\ auto%
\endisatagproof
{\isafoldproof}%
%
\isadelimproof
\isanewline
%
\endisadelimproof
\isanewline
\isacommand{definition}\isamarkupfalse%
\ sep{\isacharunderscore}{\kern0pt}Base\ \isakeyword{where}\ {\isachardoublequoteopen}sep{\isacharunderscore}{\kern0pt}Base{\isacharparenleft}{\kern0pt}x{\isacharparenright}{\kern0pt}\ \ {\isasymequiv}\ {\isasymUnion}{\isacharbraceleft}{\kern0pt}\ sep{\isacharunderscore}{\kern0pt}base{\isacharparenleft}{\kern0pt}y{\isacharparenright}{\kern0pt}{\isachardot}{\kern0pt}\ y\ {\isasymin}\ domain{\isacharparenleft}{\kern0pt}x{\isacharparenright}{\kern0pt}\ {\isacharbraceright}{\kern0pt}\ {\isasyminter}\ HS{\isachardoublequoteclose}\ \isanewline
\isanewline
\isacommand{lemma}\isamarkupfalse%
\ sep{\isacharunderscore}{\kern0pt}Base{\isacharunderscore}{\kern0pt}in{\isacharunderscore}{\kern0pt}M\ {\isacharcolon}{\kern0pt}\ \isanewline
\ \ \isakeyword{fixes}\ x\ \isanewline
\ \ \isakeyword{assumes}\ {\isachardoublequoteopen}x\ {\isasymin}\ M{\isachardoublequoteclose}\ \isanewline
\ \ \isakeyword{shows}\ {\isachardoublequoteopen}sep{\isacharunderscore}{\kern0pt}Base{\isacharparenleft}{\kern0pt}x{\isacharparenright}{\kern0pt}\ {\isasymin}\ M{\isachardoublequoteclose}\ \isanewline
%
\isadelimproof
\isanewline
\ \ %
\endisadelimproof
%
\isatagproof
\isacommand{unfolding}\isamarkupfalse%
\ sep{\isacharunderscore}{\kern0pt}Base{\isacharunderscore}{\kern0pt}def\ \isanewline
\ \ \isacommand{apply}\isamarkupfalse%
{\isacharparenleft}{\kern0pt}rule\ HS{\isacharunderscore}{\kern0pt}separation{\isacharcomma}{\kern0pt}\ rule\ to{\isacharunderscore}{\kern0pt}rin{\isacharcomma}{\kern0pt}\ rule\ Union{\isacharunderscore}{\kern0pt}closed{\isacharcomma}{\kern0pt}\ simp{\isacharparenright}{\kern0pt}\isanewline
\ \ \isacommand{apply}\isamarkupfalse%
{\isacharparenleft}{\kern0pt}rule\ sep{\isacharunderscore}{\kern0pt}base{\isacharunderscore}{\kern0pt}repfun{\isacharunderscore}{\kern0pt}in{\isacharunderscore}{\kern0pt}M{\isacharparenright}{\kern0pt}\isanewline
\ \ \isacommand{using}\isamarkupfalse%
\ assms\ domain{\isacharunderscore}{\kern0pt}closed\ \isanewline
\ \ \isacommand{by}\isamarkupfalse%
\ auto%
\endisatagproof
{\isafoldproof}%
%
\isadelimproof
\isanewline
%
\endisadelimproof
\isanewline
\isacommand{lemma}\isamarkupfalse%
\ domain{\isacharunderscore}{\kern0pt}in{\isacharunderscore}{\kern0pt}sep{\isacharunderscore}{\kern0pt}Base\ {\isacharcolon}{\kern0pt}\ {\isachardoublequoteopen}x\ {\isasymsubseteq}\ HS\ {\isasymtimes}\ P\ {\isasymLongrightarrow}\ y\ {\isasymin}\ domain{\isacharparenleft}{\kern0pt}x{\isacharparenright}{\kern0pt}\ {\isasymLongrightarrow}\ y\ {\isasymin}\ sep{\isacharunderscore}{\kern0pt}Base{\isacharparenleft}{\kern0pt}x{\isacharparenright}{\kern0pt}{\isachardoublequoteclose}\ \isanewline
%
\isadelimproof
\ \ %
\endisadelimproof
%
\isatagproof
\isacommand{unfolding}\isamarkupfalse%
\ sep{\isacharunderscore}{\kern0pt}Base{\isacharunderscore}{\kern0pt}def\ \isanewline
\ \ \isacommand{apply}\isamarkupfalse%
{\isacharparenleft}{\kern0pt}simp{\isacharcomma}{\kern0pt}\ rule\ conjI{\isacharparenright}{\kern0pt}\isanewline
\ \ \ \isacommand{apply}\isamarkupfalse%
{\isacharparenleft}{\kern0pt}rule{\isacharunderscore}{\kern0pt}tac\ x{\isacharequal}{\kern0pt}{\isachardoublequoteopen}y{\isachardoublequoteclose}\ \isakeyword{in}\ bexI{\isacharparenright}{\kern0pt}\ \isanewline
\ \ \ \ \isacommand{apply}\isamarkupfalse%
{\isacharparenleft}{\kern0pt}rule\ sep{\isacharunderscore}{\kern0pt}base{\isacharunderscore}{\kern0pt}in{\isacharparenright}{\kern0pt}\isanewline
\ \ \isacommand{using}\isamarkupfalse%
\ HS{\isacharunderscore}{\kern0pt}iff\ P{\isacharunderscore}{\kern0pt}name{\isacharunderscore}{\kern0pt}domain{\isacharunderscore}{\kern0pt}P{\isacharunderscore}{\kern0pt}name\isanewline
\ \ \ \ \isacommand{apply}\isamarkupfalse%
\ force\isanewline
\ \ \ \isacommand{apply}\isamarkupfalse%
\ simp\isanewline
\ \ \isacommand{using}\isamarkupfalse%
\ HS{\isacharunderscore}{\kern0pt}iff\ \isanewline
\ \ \isacommand{by}\isamarkupfalse%
\ auto%
\endisatagproof
{\isafoldproof}%
%
\isadelimproof
\isanewline
%
\endisadelimproof
\isanewline
\isacommand{lemma}\isamarkupfalse%
\ in{\isacharunderscore}{\kern0pt}sep{\isacharunderscore}{\kern0pt}Base{\isacharunderscore}{\kern0pt}iff\ {\isacharcolon}{\kern0pt}\ {\isachardoublequoteopen}v\ {\isasymin}\ sep{\isacharunderscore}{\kern0pt}Base{\isacharparenleft}{\kern0pt}x{\isacharparenright}{\kern0pt}\ {\isasymlongleftrightarrow}\ {\isacharparenleft}{\kern0pt}v\ {\isasymin}\ HS\ {\isasymand}\ {\isacharparenleft}{\kern0pt}{\isasymexists}u\ {\isasymin}\ domain{\isacharparenleft}{\kern0pt}x{\isacharparenright}{\kern0pt}{\isachardot}{\kern0pt}\ v\ {\isasymin}\ sep{\isacharunderscore}{\kern0pt}base{\isacharparenleft}{\kern0pt}u{\isacharparenright}{\kern0pt}{\isacharparenright}{\kern0pt}{\isacharparenright}{\kern0pt}{\isachardoublequoteclose}\ \isanewline
%
\isadelimproof
\ \ %
\endisadelimproof
%
\isatagproof
\isacommand{unfolding}\isamarkupfalse%
\ sep{\isacharunderscore}{\kern0pt}Base{\isacharunderscore}{\kern0pt}def\ \isanewline
\ \ \isacommand{by}\isamarkupfalse%
\ auto%
\endisatagproof
{\isafoldproof}%
%
\isadelimproof
\isanewline
%
\endisadelimproof
\isanewline
\isacommand{lemma}\isamarkupfalse%
\ sep{\isacharunderscore}{\kern0pt}Base{\isacharunderscore}{\kern0pt}Pn{\isacharunderscore}{\kern0pt}auto\ {\isacharcolon}{\kern0pt}\ \isanewline
\ \ \isakeyword{fixes}\ x{\isadigit{0}}\ {\isasympi}\ x\ \isanewline
\ \ \isakeyword{assumes}\ {\isachardoublequoteopen}x{\isadigit{0}}\ {\isasymin}\ P{\isacharunderscore}{\kern0pt}names{\isachardoublequoteclose}\ {\isachardoublequoteopen}{\isasympi}\ {\isasymin}\ {\isasymG}{\isachardoublequoteclose}\ {\isachardoublequoteopen}x\ {\isasymin}\ P{\isacharunderscore}{\kern0pt}names{\isachardoublequoteclose}\ \isanewline
\ \ \isakeyword{shows}\ {\isachardoublequoteopen}x\ {\isasymin}\ sep{\isacharunderscore}{\kern0pt}Base{\isacharparenleft}{\kern0pt}x{\isadigit{0}}{\isacharparenright}{\kern0pt}\ {\isasymlongleftrightarrow}\ Pn{\isacharunderscore}{\kern0pt}auto{\isacharparenleft}{\kern0pt}{\isasympi}{\isacharparenright}{\kern0pt}{\isacharbackquote}{\kern0pt}x\ {\isasymin}\ sep{\isacharunderscore}{\kern0pt}Base{\isacharparenleft}{\kern0pt}x{\isadigit{0}}{\isacharparenright}{\kern0pt}{\isachardoublequoteclose}\isanewline
%
\isadelimproof
%
\endisadelimproof
%
\isatagproof
\isacommand{proof}\isamarkupfalse%
\ {\isacharminus}{\kern0pt}\isanewline
\ \ \isacommand{have}\isamarkupfalse%
\ {\isachardoublequoteopen}x\ {\isasymin}\ sep{\isacharunderscore}{\kern0pt}Base{\isacharparenleft}{\kern0pt}x{\isadigit{0}}{\isacharparenright}{\kern0pt}\ {\isasymlongleftrightarrow}\ {\isacharparenleft}{\kern0pt}x\ {\isasymin}\ HS\ {\isasymand}\ {\isacharparenleft}{\kern0pt}{\isasymexists}u\ {\isasymin}\ domain{\isacharparenleft}{\kern0pt}x{\isadigit{0}}{\isacharparenright}{\kern0pt}{\isachardot}{\kern0pt}\ x\ {\isasymin}\ sep{\isacharunderscore}{\kern0pt}base{\isacharparenleft}{\kern0pt}u{\isacharparenright}{\kern0pt}{\isacharparenright}{\kern0pt}{\isacharparenright}{\kern0pt}{\isachardoublequoteclose}\ \isanewline
\ \ \ \ \isacommand{using}\isamarkupfalse%
\ in{\isacharunderscore}{\kern0pt}sep{\isacharunderscore}{\kern0pt}Base{\isacharunderscore}{\kern0pt}iff\ assms\ \isacommand{by}\isamarkupfalse%
\ auto\isanewline
\ \ \isacommand{also}\isamarkupfalse%
\ \isacommand{have}\isamarkupfalse%
\ {\isachardoublequoteopen}{\isachardot}{\kern0pt}{\isachardot}{\kern0pt}{\isachardot}{\kern0pt}\ {\isasymlongleftrightarrow}\ {\isacharparenleft}{\kern0pt}Pn{\isacharunderscore}{\kern0pt}auto{\isacharparenleft}{\kern0pt}{\isasympi}{\isacharparenright}{\kern0pt}{\isacharbackquote}{\kern0pt}x\ {\isasymin}\ HS\ {\isasymand}\ {\isacharparenleft}{\kern0pt}{\isasymexists}u\ {\isasymin}\ domain{\isacharparenleft}{\kern0pt}x{\isadigit{0}}{\isacharparenright}{\kern0pt}{\isachardot}{\kern0pt}\ Pn{\isacharunderscore}{\kern0pt}auto{\isacharparenleft}{\kern0pt}{\isasympi}{\isacharparenright}{\kern0pt}{\isacharbackquote}{\kern0pt}x\ {\isasymin}\ sep{\isacharunderscore}{\kern0pt}base{\isacharparenleft}{\kern0pt}u{\isacharparenright}{\kern0pt}{\isacharparenright}{\kern0pt}{\isacharparenright}{\kern0pt}{\isachardoublequoteclose}\ \isanewline
\ \ \ \ \isacommand{apply}\isamarkupfalse%
{\isacharparenleft}{\kern0pt}rule\ iff{\isacharunderscore}{\kern0pt}conjI{\isacharparenright}{\kern0pt}\isanewline
\ \ \ \ \ \isacommand{apply}\isamarkupfalse%
{\isacharparenleft}{\kern0pt}rule\ HS{\isacharunderscore}{\kern0pt}Pn{\isacharunderscore}{\kern0pt}auto{\isacharcomma}{\kern0pt}\ simp\ add{\isacharcolon}{\kern0pt}assms{\isacharcomma}{\kern0pt}\ simp\ add{\isacharcolon}{\kern0pt}assms{\isacharparenright}{\kern0pt}\isanewline
\ \ \ \ \isacommand{apply}\isamarkupfalse%
{\isacharparenleft}{\kern0pt}rule\ bex{\isacharunderscore}{\kern0pt}iff{\isacharparenright}{\kern0pt}\isanewline
\ \ \ \ \isacommand{apply}\isamarkupfalse%
{\isacharparenleft}{\kern0pt}rule\ sep{\isacharunderscore}{\kern0pt}base{\isacharunderscore}{\kern0pt}closed{\isacharunderscore}{\kern0pt}under{\isacharunderscore}{\kern0pt}Pn{\isacharunderscore}{\kern0pt}auto{\isacharprime}{\kern0pt}{\isacharparenright}{\kern0pt}\isanewline
\ \ \ \ \isacommand{using}\isamarkupfalse%
\ assms\ P{\isacharunderscore}{\kern0pt}name{\isacharunderscore}{\kern0pt}domain{\isacharunderscore}{\kern0pt}P{\isacharunderscore}{\kern0pt}name\ {\isasymG}{\isacharunderscore}{\kern0pt}P{\isacharunderscore}{\kern0pt}auto{\isacharunderscore}{\kern0pt}group\ is{\isacharunderscore}{\kern0pt}P{\isacharunderscore}{\kern0pt}auto{\isacharunderscore}{\kern0pt}group{\isacharunderscore}{\kern0pt}def\ P{\isacharunderscore}{\kern0pt}auto{\isacharunderscore}{\kern0pt}def\ \isanewline
\ \ \ \ \isacommand{by}\isamarkupfalse%
\ auto\isanewline
\ \ \isacommand{also}\isamarkupfalse%
\ \isacommand{have}\isamarkupfalse%
\ {\isachardoublequoteopen}{\isachardot}{\kern0pt}{\isachardot}{\kern0pt}{\isachardot}{\kern0pt}\ {\isasymlongleftrightarrow}\ Pn{\isacharunderscore}{\kern0pt}auto{\isacharparenleft}{\kern0pt}{\isasympi}{\isacharparenright}{\kern0pt}{\isacharbackquote}{\kern0pt}x\ {\isasymin}\ sep{\isacharunderscore}{\kern0pt}Base{\isacharparenleft}{\kern0pt}x{\isadigit{0}}{\isacharparenright}{\kern0pt}{\isachardoublequoteclose}\ \isanewline
\ \ \ \ \isacommand{apply}\isamarkupfalse%
{\isacharparenleft}{\kern0pt}rule\ iff{\isacharunderscore}{\kern0pt}flip{\isacharcomma}{\kern0pt}\ rule\ in{\isacharunderscore}{\kern0pt}sep{\isacharunderscore}{\kern0pt}Base{\isacharunderscore}{\kern0pt}iff{\isacharparenright}{\kern0pt}\isanewline
\ \ \ \ \isacommand{done}\isamarkupfalse%
\isanewline
\ \ \isacommand{finally}\isamarkupfalse%
\ \isacommand{show}\isamarkupfalse%
\ {\isachardoublequoteopen}x\ {\isasymin}\ sep{\isacharunderscore}{\kern0pt}Base{\isacharparenleft}{\kern0pt}x{\isadigit{0}}{\isacharparenright}{\kern0pt}\ {\isasymlongleftrightarrow}\ Pn{\isacharunderscore}{\kern0pt}auto{\isacharparenleft}{\kern0pt}{\isasympi}{\isacharparenright}{\kern0pt}\ {\isacharbackquote}{\kern0pt}\ x\ {\isasymin}\ sep{\isacharunderscore}{\kern0pt}Base{\isacharparenleft}{\kern0pt}x{\isadigit{0}}{\isacharparenright}{\kern0pt}\ {\isachardoublequoteclose}\ \isacommand{by}\isamarkupfalse%
\ simp\isanewline
\isacommand{qed}\isamarkupfalse%
%
\endisatagproof
{\isafoldproof}%
%
\isadelimproof
\isanewline
%
\endisadelimproof
\isanewline
\isacommand{lemma}\isamarkupfalse%
\ ex{\isacharunderscore}{\kern0pt}separation{\isacharunderscore}{\kern0pt}base\ {\isacharcolon}{\kern0pt}\ \isanewline
\ \ \isakeyword{fixes}\ X\ \isanewline
\ \ \isakeyword{assumes}\ {\isachardoublequoteopen}X\ {\isasymsubseteq}\ HS{\isachardoublequoteclose}\ {\isachardoublequoteopen}X\ {\isasymin}\ M{\isachardoublequoteclose}\ \isanewline
\ \ \isakeyword{shows}\ {\isachardoublequoteopen}{\isasymexists}S\ {\isasymin}\ SymExt{\isacharparenleft}{\kern0pt}G{\isacharparenright}{\kern0pt}{\isachardot}{\kern0pt}\ {\isacharbraceleft}{\kern0pt}\ val{\isacharparenleft}{\kern0pt}G{\isacharcomma}{\kern0pt}\ x{\isacharparenright}{\kern0pt}{\isachardot}{\kern0pt}\ x\ {\isasymin}\ X\ {\isacharbraceright}{\kern0pt}\ {\isasymsubseteq}\ S{\isachardoublequoteclose}\ \isanewline
%
\isadelimproof
%
\endisadelimproof
%
\isatagproof
\isacommand{proof}\isamarkupfalse%
\ {\isacharminus}{\kern0pt}\ \isanewline
\ \ \isacommand{define}\isamarkupfalse%
\ A\ \isakeyword{where}\ {\isachardoublequoteopen}A\ {\isasymequiv}\ X\ {\isasymtimes}\ P{\isachardoublequoteclose}\ \isanewline
\ \ \isacommand{define}\isamarkupfalse%
\ B\ \isakeyword{where}\ {\isachardoublequoteopen}B\ {\isasymequiv}\ sep{\isacharunderscore}{\kern0pt}Base{\isacharparenleft}{\kern0pt}A{\isacharparenright}{\kern0pt}\ {\isasymtimes}\ P{\isachardoublequoteclose}\ \isanewline
\isanewline
\ \ \isacommand{have}\isamarkupfalse%
\ Apname\ {\isacharcolon}{\kern0pt}\ {\isachardoublequoteopen}A\ {\isasymin}\ P{\isacharunderscore}{\kern0pt}names{\isachardoublequoteclose}\ \isanewline
\ \ \ \ \isacommand{apply}\isamarkupfalse%
{\isacharparenleft}{\kern0pt}rule\ iffD{\isadigit{2}}{\isacharcomma}{\kern0pt}\ rule\ P{\isacharunderscore}{\kern0pt}name{\isacharunderscore}{\kern0pt}iff{\isacharparenright}{\kern0pt}\isanewline
\ \ \ \ \isacommand{unfolding}\isamarkupfalse%
\ A{\isacharunderscore}{\kern0pt}def\isanewline
\ \ \ \ \isacommand{apply}\isamarkupfalse%
{\isacharparenleft}{\kern0pt}rule\ conjI{\isacharcomma}{\kern0pt}\ rule\ to{\isacharunderscore}{\kern0pt}rin{\isacharcomma}{\kern0pt}\ rule\ cartprod{\isacharunderscore}{\kern0pt}closed{\isacharparenright}{\kern0pt}\isanewline
\ \ \ \ \isacommand{using}\isamarkupfalse%
\ assms\ P{\isacharunderscore}{\kern0pt}in{\isacharunderscore}{\kern0pt}M\ HS{\isacharunderscore}{\kern0pt}iff\isanewline
\ \ \ \ \isacommand{by}\isamarkupfalse%
\ auto\isanewline
\isanewline
\ \ \isacommand{then}\isamarkupfalse%
\ \isacommand{have}\isamarkupfalse%
\ {\isachardoublequoteopen}sep{\isacharunderscore}{\kern0pt}Base{\isacharparenleft}{\kern0pt}A{\isacharparenright}{\kern0pt}\ {\isasymin}\ M{\isachardoublequoteclose}\ \isanewline
\ \ \ \ \isacommand{by}\isamarkupfalse%
{\isacharparenleft}{\kern0pt}rule{\isacharunderscore}{\kern0pt}tac\ sep{\isacharunderscore}{\kern0pt}Base{\isacharunderscore}{\kern0pt}in{\isacharunderscore}{\kern0pt}M{\isacharcomma}{\kern0pt}\ simp\ add{\isacharcolon}{\kern0pt}assms\ P{\isacharunderscore}{\kern0pt}name{\isacharunderscore}{\kern0pt}in{\isacharunderscore}{\kern0pt}M{\isacharparenright}{\kern0pt}\isanewline
\isanewline
\ \ \isacommand{then}\isamarkupfalse%
\ \isacommand{have}\isamarkupfalse%
\ pname\ {\isacharcolon}{\kern0pt}\ {\isachardoublequoteopen}sep{\isacharunderscore}{\kern0pt}Base{\isacharparenleft}{\kern0pt}A{\isacharparenright}{\kern0pt}\ {\isasymtimes}\ P\ {\isasymin}\ P{\isacharunderscore}{\kern0pt}names{\isachardoublequoteclose}\ \isanewline
\ \ \ \ \isacommand{apply}\isamarkupfalse%
{\isacharparenleft}{\kern0pt}rule{\isacharunderscore}{\kern0pt}tac\ iffD{\isadigit{2}}{\isacharparenright}{\kern0pt}\isanewline
\ \ \ \ \ \isacommand{apply}\isamarkupfalse%
{\isacharparenleft}{\kern0pt}rule\ P{\isacharunderscore}{\kern0pt}name{\isacharunderscore}{\kern0pt}iff{\isacharparenright}{\kern0pt}\isanewline
\ \ \ \ \isacommand{apply}\isamarkupfalse%
{\isacharparenleft}{\kern0pt}rule\ conjI{\isacharcomma}{\kern0pt}\ rule\ to{\isacharunderscore}{\kern0pt}rin{\isacharcomma}{\kern0pt}\ rule\ cartprod{\isacharunderscore}{\kern0pt}closed{\isacharcomma}{\kern0pt}\ simp{\isacharcomma}{\kern0pt}\ simp\ add{\isacharcolon}{\kern0pt}P{\isacharunderscore}{\kern0pt}in{\isacharunderscore}{\kern0pt}M{\isacharparenright}{\kern0pt}\isanewline
\ \ \ \ \isacommand{unfolding}\isamarkupfalse%
\ sep{\isacharunderscore}{\kern0pt}Base{\isacharunderscore}{\kern0pt}def\ \isanewline
\ \ \ \ \isacommand{using}\isamarkupfalse%
\ HS{\isacharunderscore}{\kern0pt}iff\ \isanewline
\ \ \ \ \isacommand{by}\isamarkupfalse%
\ auto\ \isanewline
\isanewline
\ \ \isacommand{have}\isamarkupfalse%
\ {\isachardoublequoteopen}{\isasymAnd}{\isasympi}{\isachardot}{\kern0pt}\ {\isasympi}\ {\isasymin}\ {\isasymG}\ {\isasymLongrightarrow}\ Pn{\isacharunderscore}{\kern0pt}auto{\isacharparenleft}{\kern0pt}{\isasympi}{\isacharparenright}{\kern0pt}{\isacharbackquote}{\kern0pt}{\isacharparenleft}{\kern0pt}sep{\isacharunderscore}{\kern0pt}Base{\isacharparenleft}{\kern0pt}A{\isacharparenright}{\kern0pt}\ {\isasymtimes}\ P{\isacharparenright}{\kern0pt}\ {\isacharequal}{\kern0pt}\ sep{\isacharunderscore}{\kern0pt}Base{\isacharparenleft}{\kern0pt}A{\isacharparenright}{\kern0pt}\ {\isasymtimes}\ P{\isachardoublequoteclose}\ \isanewline
\ \ \isacommand{proof}\isamarkupfalse%
\ {\isacharminus}{\kern0pt}\ \isanewline
\ \ \ \ \isacommand{fix}\isamarkupfalse%
\ {\isasympi}\isanewline
\ \ \ \ \isacommand{assume}\isamarkupfalse%
\ piin\ {\isacharcolon}{\kern0pt}\ {\isachardoublequoteopen}{\isasympi}\ {\isasymin}\ {\isasymG}{\isachardoublequoteclose}\isanewline
\ \ \ \ \isacommand{then}\isamarkupfalse%
\ \isacommand{have}\isamarkupfalse%
\ pipauto\ {\isacharcolon}{\kern0pt}\ {\isachardoublequoteopen}{\isasympi}\ {\isasymin}\ P{\isacharunderscore}{\kern0pt}auto{\isachardoublequoteclose}\ \isanewline
\ \ \ \ \ \ \isacommand{using}\isamarkupfalse%
\ {\isasymG}{\isacharunderscore}{\kern0pt}P{\isacharunderscore}{\kern0pt}auto{\isacharunderscore}{\kern0pt}group\ is{\isacharunderscore}{\kern0pt}P{\isacharunderscore}{\kern0pt}auto{\isacharunderscore}{\kern0pt}group{\isacharunderscore}{\kern0pt}def\ P{\isacharunderscore}{\kern0pt}auto{\isacharunderscore}{\kern0pt}def\isanewline
\ \ \ \ \ \ \isacommand{by}\isamarkupfalse%
\ auto\ \isanewline
\isanewline
\ \ \ \ \isacommand{have}\isamarkupfalse%
\ {\isachardoublequoteopen}Pn{\isacharunderscore}{\kern0pt}auto{\isacharparenleft}{\kern0pt}{\isasympi}{\isacharparenright}{\kern0pt}{\isacharbackquote}{\kern0pt}{\isacharparenleft}{\kern0pt}sep{\isacharunderscore}{\kern0pt}Base{\isacharparenleft}{\kern0pt}A{\isacharparenright}{\kern0pt}\ {\isasymtimes}\ P{\isacharparenright}{\kern0pt}\ {\isacharequal}{\kern0pt}\ {\isacharbraceleft}{\kern0pt}{\isasymlangle}Pn{\isacharunderscore}{\kern0pt}auto{\isacharparenleft}{\kern0pt}{\isasympi}{\isacharparenright}{\kern0pt}\ {\isacharbackquote}{\kern0pt}\ y{\isacharcomma}{\kern0pt}\ {\isasympi}\ {\isacharbackquote}{\kern0pt}\ p{\isasymrangle}\ {\isachardot}{\kern0pt}\ {\isasymlangle}y{\isacharcomma}{\kern0pt}p{\isasymrangle}\ {\isasymin}\ sep{\isacharunderscore}{\kern0pt}Base{\isacharparenleft}{\kern0pt}A{\isacharparenright}{\kern0pt}\ {\isasymtimes}\ P{\isacharbraceright}{\kern0pt}{\isachardoublequoteclose}\ \isanewline
\ \ \ \ \ \ \isacommand{apply}\isamarkupfalse%
{\isacharparenleft}{\kern0pt}subst\ Pn{\isacharunderscore}{\kern0pt}auto{\isacharparenright}{\kern0pt}\isanewline
\ \ \ \ \ \ \isacommand{using}\isamarkupfalse%
\ pname\isanewline
\ \ \ \ \ \ \isacommand{by}\isamarkupfalse%
\ auto\ \isanewline
\ \ \ \ \isacommand{also}\isamarkupfalse%
\ \isacommand{have}\isamarkupfalse%
\ {\isachardoublequoteopen}{\isachardot}{\kern0pt}{\isachardot}{\kern0pt}{\isachardot}{\kern0pt}\ {\isacharequal}{\kern0pt}\ sep{\isacharunderscore}{\kern0pt}Base{\isacharparenleft}{\kern0pt}A{\isacharparenright}{\kern0pt}\ {\isasymtimes}\ P{\isachardoublequoteclose}\ \isanewline
\ \ \ \ \isacommand{proof}\isamarkupfalse%
{\isacharparenleft}{\kern0pt}rule\ equality{\isacharunderscore}{\kern0pt}iffI{\isacharcomma}{\kern0pt}\ rule\ iffI{\isacharparenright}{\kern0pt}\isanewline
\ \ \ \ \ \ \isacommand{fix}\isamarkupfalse%
\ v\ \isacommand{assume}\isamarkupfalse%
\ {\isachardoublequoteopen}v\ {\isasymin}\ {\isacharbraceleft}{\kern0pt}{\isasymlangle}Pn{\isacharunderscore}{\kern0pt}auto{\isacharparenleft}{\kern0pt}{\isasympi}{\isacharparenright}{\kern0pt}\ {\isacharbackquote}{\kern0pt}\ y{\isacharcomma}{\kern0pt}\ {\isasympi}\ {\isacharbackquote}{\kern0pt}\ p{\isasymrangle}\ {\isachardot}{\kern0pt}\ {\isasymlangle}y{\isacharcomma}{\kern0pt}p{\isasymrangle}\ {\isasymin}\ sep{\isacharunderscore}{\kern0pt}Base{\isacharparenleft}{\kern0pt}A{\isacharparenright}{\kern0pt}\ {\isasymtimes}\ P{\isacharbraceright}{\kern0pt}{\isachardoublequoteclose}\ \isanewline
\ \ \ \ \ \ \isacommand{then}\isamarkupfalse%
\ \isacommand{obtain}\isamarkupfalse%
\ y\ p\ \isakeyword{where}\ ypH\ {\isacharcolon}{\kern0pt}\ {\isachardoublequoteopen}y\ {\isasymin}\ sep{\isacharunderscore}{\kern0pt}Base{\isacharparenleft}{\kern0pt}A{\isacharparenright}{\kern0pt}{\isachardoublequoteclose}\ {\isachardoublequoteopen}p\ {\isasymin}\ P{\isachardoublequoteclose}\ {\isachardoublequoteopen}v\ {\isacharequal}{\kern0pt}\ {\isasymlangle}Pn{\isacharunderscore}{\kern0pt}auto{\isacharparenleft}{\kern0pt}{\isasympi}{\isacharparenright}{\kern0pt}\ {\isacharbackquote}{\kern0pt}\ y{\isacharcomma}{\kern0pt}\ {\isasympi}\ {\isacharbackquote}{\kern0pt}\ p{\isasymrangle}{\isachardoublequoteclose}\ \isacommand{by}\isamarkupfalse%
\ auto\isanewline
\isanewline
\ \ \ \ \ \ \isacommand{have}\isamarkupfalse%
\ H{\isadigit{1}}{\isacharcolon}{\kern0pt}\ {\isachardoublequoteopen}Pn{\isacharunderscore}{\kern0pt}auto{\isacharparenleft}{\kern0pt}{\isasympi}{\isacharparenright}{\kern0pt}\ {\isacharbackquote}{\kern0pt}\ y\ {\isasymin}\ sep{\isacharunderscore}{\kern0pt}Base{\isacharparenleft}{\kern0pt}A{\isacharparenright}{\kern0pt}{\isachardoublequoteclose}\ \isanewline
\ \ \ \ \ \ \ \ \isacommand{apply}\isamarkupfalse%
{\isacharparenleft}{\kern0pt}rule\ iffD{\isadigit{1}}{\isacharcomma}{\kern0pt}\ rule\ sep{\isacharunderscore}{\kern0pt}Base{\isacharunderscore}{\kern0pt}Pn{\isacharunderscore}{\kern0pt}auto{\isacharparenright}{\kern0pt}\isanewline
\ \ \ \ \ \ \ \ \isacommand{using}\isamarkupfalse%
\ assms\ piin\ sep{\isacharunderscore}{\kern0pt}Base{\isacharunderscore}{\kern0pt}def\ HS{\isacharunderscore}{\kern0pt}iff\ ypH\ Apname\isanewline
\ \ \ \ \ \ \ \ \isacommand{by}\isamarkupfalse%
\ auto\isanewline
\ \ \ \ \ \ \isacommand{have}\isamarkupfalse%
\ H{\isadigit{2}}{\isacharcolon}{\kern0pt}\ {\isachardoublequoteopen}{\isasympi}{\isacharbackquote}{\kern0pt}p\ {\isasymin}\ P{\isachardoublequoteclose}\ \isanewline
\ \ \ \ \ \ \ \ \isacommand{apply}\isamarkupfalse%
{\isacharparenleft}{\kern0pt}rule\ P{\isacharunderscore}{\kern0pt}auto{\isacharunderscore}{\kern0pt}value{\isacharparenright}{\kern0pt}\isanewline
\ \ \ \ \ \ \ \ \isacommand{using}\isamarkupfalse%
\ pipauto\ P{\isacharunderscore}{\kern0pt}auto{\isacharunderscore}{\kern0pt}def\ ypH\isanewline
\ \ \ \ \ \ \ \ \isacommand{by}\isamarkupfalse%
\ auto\ \isanewline
\ \ \ \ \ \ \isacommand{show}\isamarkupfalse%
\ {\isachardoublequoteopen}v\ {\isasymin}\ sep{\isacharunderscore}{\kern0pt}Base{\isacharparenleft}{\kern0pt}A{\isacharparenright}{\kern0pt}\ {\isasymtimes}\ P{\isachardoublequoteclose}\ \isanewline
\ \ \ \ \ \ \ \ \isacommand{using}\isamarkupfalse%
\ H{\isadigit{1}}\ H{\isadigit{2}}\ ypH\ \isanewline
\ \ \ \ \ \ \ \ \isacommand{by}\isamarkupfalse%
\ auto\isanewline
\ \ \ \ \isacommand{next}\isamarkupfalse%
\ \isanewline
\ \ \ \ \ \ \isacommand{fix}\isamarkupfalse%
\ v\ \isacommand{assume}\isamarkupfalse%
\ {\isachardoublequoteopen}v\ {\isasymin}\ sep{\isacharunderscore}{\kern0pt}Base{\isacharparenleft}{\kern0pt}A{\isacharparenright}{\kern0pt}\ {\isasymtimes}\ P{\isachardoublequoteclose}\ \isanewline
\ \ \ \ \ \ \isacommand{then}\isamarkupfalse%
\ \isacommand{obtain}\isamarkupfalse%
\ y\ p\ \isakeyword{where}\ ypH\ {\isacharcolon}{\kern0pt}\ {\isachardoublequoteopen}y\ {\isasymin}\ sep{\isacharunderscore}{\kern0pt}Base{\isacharparenleft}{\kern0pt}A{\isacharparenright}{\kern0pt}{\isachardoublequoteclose}\ {\isachardoublequoteopen}p\ {\isasymin}\ P{\isachardoublequoteclose}\ {\isachardoublequoteopen}v\ {\isacharequal}{\kern0pt}\ {\isasymlangle}y{\isacharcomma}{\kern0pt}\ p{\isasymrangle}{\isachardoublequoteclose}\ \isacommand{by}\isamarkupfalse%
\ auto\isanewline
\isanewline
\ \ \ \ \ \ \isacommand{have}\isamarkupfalse%
\ ypname\ {\isacharcolon}{\kern0pt}\ {\isachardoublequoteopen}y\ {\isasymin}\ P{\isacharunderscore}{\kern0pt}names{\isachardoublequoteclose}\ \isacommand{using}\isamarkupfalse%
\ ypH\ sep{\isacharunderscore}{\kern0pt}Base{\isacharunderscore}{\kern0pt}def\ HS{\isacharunderscore}{\kern0pt}iff\ \isacommand{by}\isamarkupfalse%
\ auto\isanewline
\isanewline
\ \ \ \ \ \ \isacommand{have}\isamarkupfalse%
\ {\isachardoublequoteopen}Pn{\isacharunderscore}{\kern0pt}auto{\isacharparenleft}{\kern0pt}{\isasympi}{\isacharparenright}{\kern0pt}\ {\isasymin}\ surj{\isacharparenleft}{\kern0pt}P{\isacharunderscore}{\kern0pt}names{\isacharcomma}{\kern0pt}\ P{\isacharunderscore}{\kern0pt}names{\isacharparenright}{\kern0pt}{\isachardoublequoteclose}\ \isanewline
\ \ \ \ \ \ \ \ \isacommand{apply}\isamarkupfalse%
{\isacharparenleft}{\kern0pt}rule\ bij{\isacharunderscore}{\kern0pt}is{\isacharunderscore}{\kern0pt}surj{\isacharcomma}{\kern0pt}\ rule\ Pn{\isacharunderscore}{\kern0pt}auto{\isacharunderscore}{\kern0pt}bij{\isacharparenright}{\kern0pt}\isanewline
\ \ \ \ \ \ \ \ \isacommand{using}\isamarkupfalse%
\ pipauto\ P{\isacharunderscore}{\kern0pt}auto{\isacharunderscore}{\kern0pt}def\isanewline
\ \ \ \ \ \ \ \ \isacommand{by}\isamarkupfalse%
\ auto\ \isanewline
\ \ \ \ \ \ \isacommand{then}\isamarkupfalse%
\ \isacommand{obtain}\isamarkupfalse%
\ y{\isacharprime}{\kern0pt}\ \isakeyword{where}\ y{\isacharprime}{\kern0pt}H\ {\isacharcolon}{\kern0pt}\ {\isachardoublequoteopen}y\ {\isacharequal}{\kern0pt}\ Pn{\isacharunderscore}{\kern0pt}auto{\isacharparenleft}{\kern0pt}{\isasympi}{\isacharparenright}{\kern0pt}{\isacharbackquote}{\kern0pt}y{\isacharprime}{\kern0pt}{\isachardoublequoteclose}\ {\isachardoublequoteopen}y{\isacharprime}{\kern0pt}\ {\isasymin}\ P{\isacharunderscore}{\kern0pt}names{\isachardoublequoteclose}\ \isanewline
\ \ \ \ \ \ \ \ \isacommand{unfolding}\isamarkupfalse%
\ surj{\isacharunderscore}{\kern0pt}def\ \isacommand{using}\isamarkupfalse%
\ ypname\ \isacommand{by}\isamarkupfalse%
\ auto\ \isanewline
\ \ \ \ \ \ \isacommand{have}\isamarkupfalse%
\ y{\isacharprime}{\kern0pt}in\ {\isacharcolon}{\kern0pt}\ {\isachardoublequoteopen}y{\isacharprime}{\kern0pt}\ {\isasymin}\ sep{\isacharunderscore}{\kern0pt}Base{\isacharparenleft}{\kern0pt}A{\isacharparenright}{\kern0pt}{\isachardoublequoteclose}\ \isanewline
\ \ \ \ \ \ \ \ \isacommand{apply}\isamarkupfalse%
{\isacharparenleft}{\kern0pt}rule\ iffD{\isadigit{2}}{\isacharcomma}{\kern0pt}\ rule{\isacharunderscore}{\kern0pt}tac\ {\isasympi}{\isacharequal}{\kern0pt}{\isasympi}\ \isakeyword{in}\ sep{\isacharunderscore}{\kern0pt}Base{\isacharunderscore}{\kern0pt}Pn{\isacharunderscore}{\kern0pt}auto{\isacharparenright}{\kern0pt}\isanewline
\ \ \ \ \ \ \ \ \isacommand{using}\isamarkupfalse%
\ y{\isacharprime}{\kern0pt}H\ ypH\ piin\ assms\ Apname\isanewline
\ \ \ \ \ \ \ \ \isacommand{by}\isamarkupfalse%
\ auto\isanewline
\isanewline
\ \ \ \ \ \ \isacommand{have}\isamarkupfalse%
\ {\isachardoublequoteopen}{\isasympi}\ {\isasymin}\ surj{\isacharparenleft}{\kern0pt}P{\isacharcomma}{\kern0pt}\ P{\isacharparenright}{\kern0pt}{\isachardoublequoteclose}\ \isanewline
\ \ \ \ \ \ \ \ \isacommand{apply}\isamarkupfalse%
{\isacharparenleft}{\kern0pt}rule\ bij{\isacharunderscore}{\kern0pt}is{\isacharunderscore}{\kern0pt}surj{\isacharparenright}{\kern0pt}\isanewline
\ \ \ \ \ \ \ \ \isacommand{using}\isamarkupfalse%
\ pipauto\ P{\isacharunderscore}{\kern0pt}auto{\isacharunderscore}{\kern0pt}def\ is{\isacharunderscore}{\kern0pt}P{\isacharunderscore}{\kern0pt}auto{\isacharunderscore}{\kern0pt}def\isanewline
\ \ \ \ \ \ \ \ \isacommand{by}\isamarkupfalse%
\ auto\isanewline
\ \ \ \ \ \ \isacommand{then}\isamarkupfalse%
\ \isacommand{obtain}\isamarkupfalse%
\ p{\isacharprime}{\kern0pt}\ \isakeyword{where}\ p{\isacharprime}{\kern0pt}H\ {\isacharcolon}{\kern0pt}\ {\isachardoublequoteopen}p{\isacharprime}{\kern0pt}\ {\isasymin}\ P{\isachardoublequoteclose}\ {\isachardoublequoteopen}p\ {\isacharequal}{\kern0pt}\ {\isasympi}{\isacharbackquote}{\kern0pt}p{\isacharprime}{\kern0pt}{\isachardoublequoteclose}\ \isanewline
\ \ \ \ \ \ \ \ \isacommand{unfolding}\isamarkupfalse%
\ surj{\isacharunderscore}{\kern0pt}def\ \isacommand{using}\isamarkupfalse%
\ ypH\ \isacommand{by}\isamarkupfalse%
\ auto\ \isanewline
\isanewline
\ \ \ \ \ \ \isacommand{have}\isamarkupfalse%
\ {\isachardoublequoteopen}{\isacharless}{\kern0pt}Pn{\isacharunderscore}{\kern0pt}auto{\isacharparenleft}{\kern0pt}{\isasympi}{\isacharparenright}{\kern0pt}{\isacharbackquote}{\kern0pt}y{\isacharprime}{\kern0pt}{\isacharcomma}{\kern0pt}\ {\isasympi}{\isacharbackquote}{\kern0pt}p{\isacharprime}{\kern0pt}{\isachargreater}{\kern0pt}\ {\isasymin}\ {\isacharbraceleft}{\kern0pt}{\isasymlangle}Pn{\isacharunderscore}{\kern0pt}auto{\isacharparenleft}{\kern0pt}{\isasympi}{\isacharparenright}{\kern0pt}\ {\isacharbackquote}{\kern0pt}\ y{\isacharcomma}{\kern0pt}\ {\isasympi}\ {\isacharbackquote}{\kern0pt}\ p{\isasymrangle}\ {\isachardot}{\kern0pt}\ {\isasymlangle}y{\isacharcomma}{\kern0pt}p{\isasymrangle}\ {\isasymin}\ sep{\isacharunderscore}{\kern0pt}Base{\isacharparenleft}{\kern0pt}A{\isacharparenright}{\kern0pt}\ {\isasymtimes}\ P{\isacharbraceright}{\kern0pt}{\isachardoublequoteclose}\isanewline
\ \ \ \ \ \ \ \ \isacommand{using}\isamarkupfalse%
\ y{\isacharprime}{\kern0pt}H\ p{\isacharprime}{\kern0pt}H\ y{\isacharprime}{\kern0pt}in\ \isacommand{by}\isamarkupfalse%
\ auto\ \isanewline
\ \ \ \ \ \ \isacommand{then}\isamarkupfalse%
\ \isacommand{show}\isamarkupfalse%
\ {\isachardoublequoteopen}v\ {\isasymin}\ {\isacharbraceleft}{\kern0pt}{\isasymlangle}Pn{\isacharunderscore}{\kern0pt}auto{\isacharparenleft}{\kern0pt}{\isasympi}{\isacharparenright}{\kern0pt}\ {\isacharbackquote}{\kern0pt}\ y{\isacharcomma}{\kern0pt}\ {\isasympi}\ {\isacharbackquote}{\kern0pt}\ p{\isasymrangle}\ {\isachardot}{\kern0pt}\ {\isasymlangle}y{\isacharcomma}{\kern0pt}p{\isasymrangle}\ {\isasymin}\ sep{\isacharunderscore}{\kern0pt}Base{\isacharparenleft}{\kern0pt}A{\isacharparenright}{\kern0pt}\ {\isasymtimes}\ P{\isacharbraceright}{\kern0pt}{\isachardoublequoteclose}\ \isanewline
\ \ \ \ \ \ \ \ \isacommand{using}\isamarkupfalse%
\ p{\isacharprime}{\kern0pt}H\ y{\isacharprime}{\kern0pt}H\ ypH\ \isacommand{by}\isamarkupfalse%
\ auto\isanewline
\ \ \ \ \isacommand{qed}\isamarkupfalse%
\isanewline
\isanewline
\ \ \ \ \isacommand{finally}\isamarkupfalse%
\ \isacommand{show}\isamarkupfalse%
\ {\isachardoublequoteopen}Pn{\isacharunderscore}{\kern0pt}auto{\isacharparenleft}{\kern0pt}{\isasympi}{\isacharparenright}{\kern0pt}\ {\isacharbackquote}{\kern0pt}\ {\isacharparenleft}{\kern0pt}sep{\isacharunderscore}{\kern0pt}Base{\isacharparenleft}{\kern0pt}A{\isacharparenright}{\kern0pt}\ {\isasymtimes}\ P{\isacharparenright}{\kern0pt}\ {\isacharequal}{\kern0pt}\ sep{\isacharunderscore}{\kern0pt}Base{\isacharparenleft}{\kern0pt}A{\isacharparenright}{\kern0pt}\ {\isasymtimes}\ P\ {\isachardoublequoteclose}\ \isacommand{by}\isamarkupfalse%
\ simp\isanewline
\ \ \isacommand{qed}\isamarkupfalse%
\isanewline
\isanewline
\ \ \isacommand{then}\isamarkupfalse%
\ \isacommand{have}\isamarkupfalse%
\ {\isachardoublequoteopen}sym{\isacharparenleft}{\kern0pt}sep{\isacharunderscore}{\kern0pt}Base{\isacharparenleft}{\kern0pt}A{\isacharparenright}{\kern0pt}\ {\isasymtimes}\ P{\isacharparenright}{\kern0pt}\ {\isacharequal}{\kern0pt}\ {\isasymG}{\isachardoublequoteclose}\ \isanewline
\ \ \ \ \isacommand{unfolding}\isamarkupfalse%
\ sym{\isacharunderscore}{\kern0pt}def\isanewline
\ \ \ \ \isacommand{by}\isamarkupfalse%
\ blast\isanewline
\isanewline
\ \ \isacommand{then}\isamarkupfalse%
\ \isacommand{have}\isamarkupfalse%
\ symH\ {\isacharcolon}{\kern0pt}\ {\isachardoublequoteopen}symmetric{\isacharparenleft}{\kern0pt}sep{\isacharunderscore}{\kern0pt}Base{\isacharparenleft}{\kern0pt}A{\isacharparenright}{\kern0pt}\ {\isasymtimes}\ P{\isacharparenright}{\kern0pt}{\isachardoublequoteclose}\ \isanewline
\ \ \ \ \isacommand{unfolding}\isamarkupfalse%
\ symmetric{\isacharunderscore}{\kern0pt}def\isanewline
\ \ \ \ \isacommand{using}\isamarkupfalse%
\ {\isasymG}{\isacharunderscore}{\kern0pt}in{\isacharunderscore}{\kern0pt}{\isasymF}\isanewline
\ \ \ \ \isacommand{by}\isamarkupfalse%
\ auto\isanewline
\isanewline
\ \ \isacommand{have}\isamarkupfalse%
\ {\isachardoublequoteopen}sep{\isacharunderscore}{\kern0pt}Base{\isacharparenleft}{\kern0pt}A{\isacharparenright}{\kern0pt}\ {\isasymtimes}\ P\ {\isasymin}\ HS{\isachardoublequoteclose}\ \isanewline
\ \ \ \ \isacommand{apply}\isamarkupfalse%
{\isacharparenleft}{\kern0pt}rule\ iffD{\isadigit{2}}{\isacharcomma}{\kern0pt}\ rule\ HS{\isacharunderscore}{\kern0pt}iff{\isacharparenright}{\kern0pt}\isanewline
\ \ \ \ \isacommand{using}\isamarkupfalse%
\ pname\ symH\ sep{\isacharunderscore}{\kern0pt}Base{\isacharunderscore}{\kern0pt}def\isanewline
\ \ \ \ \isacommand{by}\isamarkupfalse%
\ auto\ \isanewline
\isanewline
\ \ \isacommand{then}\isamarkupfalse%
\ \isacommand{have}\isamarkupfalse%
\ insymext\ {\isacharcolon}{\kern0pt}\ {\isachardoublequoteopen}val{\isacharparenleft}{\kern0pt}G{\isacharcomma}{\kern0pt}\ sep{\isacharunderscore}{\kern0pt}Base{\isacharparenleft}{\kern0pt}A{\isacharparenright}{\kern0pt}\ {\isasymtimes}\ P{\isacharparenright}{\kern0pt}\ {\isasymin}\ SymExt{\isacharparenleft}{\kern0pt}G{\isacharparenright}{\kern0pt}{\isachardoublequoteclose}\ \isanewline
\ \ \ \ \isacommand{unfolding}\isamarkupfalse%
\ SymExt{\isacharunderscore}{\kern0pt}def\ \isacommand{by}\isamarkupfalse%
\ auto\ \isanewline
\isanewline
\ \ \isacommand{have}\isamarkupfalse%
\ Xsubset{\isacharcolon}{\kern0pt}\ {\isachardoublequoteopen}X\ {\isasymsubseteq}\ sep{\isacharunderscore}{\kern0pt}Base{\isacharparenleft}{\kern0pt}A{\isacharparenright}{\kern0pt}{\isachardoublequoteclose}\ \isanewline
\ \ \ \ \isacommand{unfolding}\isamarkupfalse%
\ A{\isacharunderscore}{\kern0pt}def\isanewline
\ \ \ \ \isacommand{apply}\isamarkupfalse%
{\isacharparenleft}{\kern0pt}rule\ subsetI{\isacharcomma}{\kern0pt}\ rule\ domain{\isacharunderscore}{\kern0pt}in{\isacharunderscore}{\kern0pt}sep{\isacharunderscore}{\kern0pt}Base{\isacharparenright}{\kern0pt}\isanewline
\ \ \ \ \isacommand{using}\isamarkupfalse%
\ assms\isanewline
\ \ \ \ \ \isacommand{apply}\isamarkupfalse%
\ force\isanewline
\ \ \ \ \isacommand{apply}\isamarkupfalse%
{\isacharparenleft}{\kern0pt}rule{\isacharunderscore}{\kern0pt}tac\ b{\isacharequal}{\kern0pt}one\ \isakeyword{in}\ domainI{\isacharparenright}{\kern0pt}\isanewline
\ \ \ \ \isacommand{using}\isamarkupfalse%
\ one{\isacharunderscore}{\kern0pt}in{\isacharunderscore}{\kern0pt}P\isanewline
\ \ \ \ \isacommand{by}\isamarkupfalse%
\ auto\ \isanewline
\isanewline
\ \ \isacommand{have}\isamarkupfalse%
\ subsetH{\isacharcolon}{\kern0pt}\ {\isachardoublequoteopen}{\isacharbraceleft}{\kern0pt}\ val{\isacharparenleft}{\kern0pt}G{\isacharcomma}{\kern0pt}\ x{\isacharparenright}{\kern0pt}{\isachardot}{\kern0pt}\ x\ {\isasymin}\ X\ {\isacharbraceright}{\kern0pt}\ {\isasymsubseteq}\ val{\isacharparenleft}{\kern0pt}G{\isacharcomma}{\kern0pt}\ sep{\isacharunderscore}{\kern0pt}Base{\isacharparenleft}{\kern0pt}A{\isacharparenright}{\kern0pt}\ {\isasymtimes}\ P{\isacharparenright}{\kern0pt}{\isachardoublequoteclose}\ \isanewline
\ \ \isacommand{proof}\isamarkupfalse%
{\isacharparenleft}{\kern0pt}rule\ subsetI{\isacharparenright}{\kern0pt}\isanewline
\ \ \ \ \isacommand{fix}\isamarkupfalse%
\ v\ \isacommand{assume}\isamarkupfalse%
\ {\isachardoublequoteopen}v\ {\isasymin}\ {\isacharbraceleft}{\kern0pt}\ val{\isacharparenleft}{\kern0pt}G{\isacharcomma}{\kern0pt}\ x{\isacharparenright}{\kern0pt}{\isachardot}{\kern0pt}\ x\ {\isasymin}\ X\ {\isacharbraceright}{\kern0pt}{\isachardoublequoteclose}\isanewline
\ \ \ \ \isacommand{then}\isamarkupfalse%
\ \isacommand{obtain}\isamarkupfalse%
\ x\ \isakeyword{where}\ xH\ {\isacharcolon}{\kern0pt}\ {\isachardoublequoteopen}v\ {\isacharequal}{\kern0pt}\ val{\isacharparenleft}{\kern0pt}G{\isacharcomma}{\kern0pt}\ x{\isacharparenright}{\kern0pt}{\isachardoublequoteclose}\ {\isachardoublequoteopen}x\ {\isasymin}\ X{\isachardoublequoteclose}\ \isacommand{by}\isamarkupfalse%
\ auto\ \isanewline
\ \ \ \ \isacommand{then}\isamarkupfalse%
\ \isacommand{have}\isamarkupfalse%
\ {\isachardoublequoteopen}val{\isacharparenleft}{\kern0pt}G{\isacharcomma}{\kern0pt}\ x{\isacharparenright}{\kern0pt}\ {\isasymin}\ val{\isacharparenleft}{\kern0pt}G{\isacharcomma}{\kern0pt}\ sep{\isacharunderscore}{\kern0pt}Base{\isacharparenleft}{\kern0pt}A{\isacharparenright}{\kern0pt}\ {\isasymtimes}\ P{\isacharparenright}{\kern0pt}{\isachardoublequoteclose}\ \isanewline
\ \ \ \ \ \ \isacommand{apply}\isamarkupfalse%
{\isacharparenleft}{\kern0pt}subst\ {\isacharparenleft}{\kern0pt}{\isadigit{2}}{\isacharparenright}{\kern0pt}\ def{\isacharunderscore}{\kern0pt}val{\isacharcomma}{\kern0pt}\ simp{\isacharparenright}{\kern0pt}\isanewline
\ \ \ \ \ \ \isacommand{apply}\isamarkupfalse%
{\isacharparenleft}{\kern0pt}rule{\isacharunderscore}{\kern0pt}tac\ x{\isacharequal}{\kern0pt}x\ \isakeyword{in}\ bexI{\isacharcomma}{\kern0pt}\ simp{\isacharcomma}{\kern0pt}\ rule\ conjI{\isacharparenright}{\kern0pt}\isanewline
\ \ \ \ \ \ \isacommand{using}\isamarkupfalse%
\ Xsubset\ \isanewline
\ \ \ \ \ \ \ \ \isacommand{apply}\isamarkupfalse%
\ force\isanewline
\ \ \ \ \ \ \ \isacommand{apply}\isamarkupfalse%
{\isacharparenleft}{\kern0pt}rule{\isacharunderscore}{\kern0pt}tac\ x{\isacharequal}{\kern0pt}one\ \isakeyword{in}\ bexI{\isacharparenright}{\kern0pt}\isanewline
\ \ \ \ \ \ \isacommand{using}\isamarkupfalse%
\ one{\isacharunderscore}{\kern0pt}in{\isacharunderscore}{\kern0pt}G\ one{\isacharunderscore}{\kern0pt}in{\isacharunderscore}{\kern0pt}P\ generic\ Xsubset\isanewline
\ \ \ \ \ \ \isacommand{by}\isamarkupfalse%
\ auto\isanewline
\ \ \ \ \isacommand{then}\isamarkupfalse%
\ \isacommand{show}\isamarkupfalse%
\ {\isachardoublequoteopen}v\ {\isasymin}\ val{\isacharparenleft}{\kern0pt}G{\isacharcomma}{\kern0pt}\ sep{\isacharunderscore}{\kern0pt}Base{\isacharparenleft}{\kern0pt}A{\isacharparenright}{\kern0pt}\ {\isasymtimes}\ P{\isacharparenright}{\kern0pt}{\isachardoublequoteclose}\ \isanewline
\ \ \ \ \ \ \isacommand{using}\isamarkupfalse%
\ xH\ \isanewline
\ \ \ \ \ \ \isacommand{by}\isamarkupfalse%
\ auto\ \isanewline
\ \ \isacommand{qed}\isamarkupfalse%
\isanewline
\isanewline
\ \ \isacommand{show}\isamarkupfalse%
\ {\isacharquery}{\kern0pt}thesis\ \isanewline
\ \ \ \ \isacommand{apply}\isamarkupfalse%
{\isacharparenleft}{\kern0pt}rule{\isacharunderscore}{\kern0pt}tac\ x{\isacharequal}{\kern0pt}{\isachardoublequoteopen}val{\isacharparenleft}{\kern0pt}G{\isacharcomma}{\kern0pt}\ sep{\isacharunderscore}{\kern0pt}Base{\isacharparenleft}{\kern0pt}A{\isacharparenright}{\kern0pt}\ {\isasymtimes}\ P{\isacharparenright}{\kern0pt}{\isachardoublequoteclose}\ \isakeyword{in}\ bexI{\isacharparenright}{\kern0pt}\isanewline
\ \ \ \ \ \isacommand{apply}\isamarkupfalse%
{\isacharparenleft}{\kern0pt}rule\ subsetH{\isacharcomma}{\kern0pt}\ rule\ insymext{\isacharparenright}{\kern0pt}\isanewline
\ \ \ \ \isacommand{done}\isamarkupfalse%
\isanewline
\isacommand{qed}\isamarkupfalse%
%
\endisatagproof
{\isafoldproof}%
%
\isadelimproof
\isanewline
%
\endisadelimproof
\isanewline
\isacommand{end}\isamarkupfalse%
\isanewline
%
\isadelimtheory
%
\endisadelimtheory
%
\isatagtheory
\isacommand{end}\isamarkupfalse%
%
\endisatagtheory
{\isafoldtheory}%
%
\isadelimtheory
%
\endisadelimtheory
%
\end{isabellebody}%
\endinput
%:%file=~/source/repos/ZF-notAC/code/SymExt_Separation_Base.thy%:%
%:%10=1%:%
%:%11=1%:%
%:%12=2%:%
%:%13=3%:%
%:%14=4%:%
%:%15=5%:%
%:%20=5%:%
%:%23=6%:%
%:%24=7%:%
%:%25=7%:%
%:%26=8%:%
%:%27=9%:%
%:%28=10%:%
%:%29=11%:%
%:%30=11%:%
%:%31=12%:%
%:%32=12%:%
%:%33=13%:%
%:%34=13%:%
%:%35=14%:%
%:%36=14%:%
%:%39=15%:%
%:%43=15%:%
%:%44=15%:%
%:%45=16%:%
%:%46=16%:%
%:%47=17%:%
%:%48=17%:%
%:%49=18%:%
%:%50=18%:%
%:%55=18%:%
%:%58=19%:%
%:%59=20%:%
%:%60=20%:%
%:%61=21%:%
%:%62=21%:%
%:%64=23%:%
%:%65=24%:%
%:%66=25%:%
%:%67=25%:%
%:%68=26%:%
%:%71=27%:%
%:%75=27%:%
%:%76=27%:%
%:%77=28%:%
%:%78=28%:%
%:%79=29%:%
%:%80=29%:%
%:%81=30%:%
%:%82=31%:%
%:%83=31%:%
%:%84=32%:%
%:%85=32%:%
%:%86=33%:%
%:%87=33%:%
%:%88=34%:%
%:%89=34%:%
%:%90=35%:%
%:%91=35%:%
%:%92=36%:%
%:%93=36%:%
%:%94=37%:%
%:%95=37%:%
%:%96=38%:%
%:%97=38%:%
%:%98=39%:%
%:%99=39%:%
%:%100=40%:%
%:%101=40%:%
%:%102=41%:%
%:%103=41%:%
%:%104=42%:%
%:%105=42%:%
%:%106=43%:%
%:%107=43%:%
%:%112=43%:%
%:%115=44%:%
%:%116=45%:%
%:%117=45%:%
%:%118=46%:%
%:%119=47%:%
%:%120=48%:%
%:%121=49%:%
%:%122=50%:%
%:%123=51%:%
%:%124=52%:%
%:%125=53%:%
%:%128=54%:%
%:%129=55%:%
%:%133=55%:%
%:%134=55%:%
%:%135=56%:%
%:%136=56%:%
%:%137=57%:%
%:%138=57%:%
%:%139=58%:%
%:%140=58%:%
%:%141=59%:%
%:%142=59%:%
%:%143=60%:%
%:%144=60%:%
%:%145=61%:%
%:%146=61%:%
%:%147=62%:%
%:%148=62%:%
%:%149=63%:%
%:%150=63%:%
%:%155=63%:%
%:%158=64%:%
%:%159=65%:%
%:%160=65%:%
%:%161=66%:%
%:%185=90%:%
%:%186=91%:%
%:%187=92%:%
%:%188=92%:%
%:%195=93%:%
%:%196=93%:%
%:%197=94%:%
%:%198=94%:%
%:%199=95%:%
%:%200=95%:%
%:%201=95%:%
%:%202=95%:%
%:%203=95%:%
%:%204=96%:%
%:%205=96%:%
%:%206=96%:%
%:%207=96%:%
%:%208=96%:%
%:%209=97%:%
%:%210=97%:%
%:%211=98%:%
%:%212=98%:%
%:%213=99%:%
%:%214=100%:%
%:%215=100%:%
%:%216=101%:%
%:%217=101%:%
%:%218=101%:%
%:%219=101%:%
%:%220=102%:%
%:%221=103%:%
%:%222=103%:%
%:%223=104%:%
%:%224=104%:%
%:%225=105%:%
%:%226=105%:%
%:%227=106%:%
%:%228=106%:%
%:%229=107%:%
%:%230=108%:%
%:%231=108%:%
%:%232=109%:%
%:%233=109%:%
%:%234=110%:%
%:%235=110%:%
%:%236=111%:%
%:%237=111%:%
%:%238=112%:%
%:%239=113%:%
%:%240=113%:%
%:%241=113%:%
%:%242=114%:%
%:%243=114%:%
%:%244=114%:%
%:%245=115%:%
%:%251=115%:%
%:%254=116%:%
%:%255=117%:%
%:%256=117%:%
%:%257=118%:%
%:%258=119%:%
%:%259=119%:%
%:%260=120%:%
%:%261=121%:%
%:%262=121%:%
%:%263=122%:%
%:%264=123%:%
%:%265=124%:%
%:%272=125%:%
%:%273=125%:%
%:%274=126%:%
%:%275=127%:%
%:%276=127%:%
%:%277=128%:%
%:%278=128%:%
%:%279=129%:%
%:%280=130%:%
%:%281=130%:%
%:%282=131%:%
%:%283=131%:%
%:%284=132%:%
%:%285=132%:%
%:%286=133%:%
%:%287=133%:%
%:%288=134%:%
%:%289=134%:%
%:%290=135%:%
%:%291=135%:%
%:%292=136%:%
%:%293=137%:%
%:%294=137%:%
%:%295=138%:%
%:%296=138%:%
%:%297=139%:%
%:%298=139%:%
%:%299=139%:%
%:%300=140%:%
%:%301=140%:%
%:%302=140%:%
%:%303=140%:%
%:%304=141%:%
%:%305=141%:%
%:%306=142%:%
%:%307=142%:%
%:%308=143%:%
%:%309=143%:%
%:%310=144%:%
%:%311=144%:%
%:%312=145%:%
%:%313=145%:%
%:%314=146%:%
%:%315=146%:%
%:%316=147%:%
%:%317=147%:%
%:%318=147%:%
%:%319=147%:%
%:%320=148%:%
%:%321=148%:%
%:%322=148%:%
%:%323=148%:%
%:%324=148%:%
%:%325=148%:%
%:%326=149%:%
%:%327=149%:%
%:%328=149%:%
%:%329=150%:%
%:%330=150%:%
%:%331=151%:%
%:%332=151%:%
%:%333=152%:%
%:%334=152%:%
%:%335=153%:%
%:%336=153%:%
%:%337=154%:%
%:%338=154%:%
%:%339=154%:%
%:%340=155%:%
%:%341=155%:%
%:%342=155%:%
%:%343=155%:%
%:%344=155%:%
%:%345=156%:%
%:%346=156%:%
%:%347=156%:%
%:%348=156%:%
%:%349=156%:%
%:%350=157%:%
%:%351=157%:%
%:%352=158%:%
%:%353=158%:%
%:%354=159%:%
%:%355=159%:%
%:%356=160%:%
%:%357=160%:%
%:%358=161%:%
%:%359=161%:%
%:%360=162%:%
%:%361=162%:%
%:%362=163%:%
%:%363=163%:%
%:%364=164%:%
%:%365=164%:%
%:%366=165%:%
%:%367=165%:%
%:%368=166%:%
%:%369=166%:%
%:%370=166%:%
%:%371=167%:%
%:%372=167%:%
%:%373=167%:%
%:%374=168%:%
%:%375=168%:%
%:%376=169%:%
%:%377=170%:%
%:%378=170%:%
%:%379=171%:%
%:%380=171%:%
%:%381=172%:%
%:%382=172%:%
%:%383=173%:%
%:%384=173%:%
%:%385=174%:%
%:%391=174%:%
%:%394=175%:%
%:%395=176%:%
%:%396=176%:%
%:%397=177%:%
%:%398=178%:%
%:%399=178%:%
%:%400=179%:%
%:%401=180%:%
%:%402=181%:%
%:%403=182%:%
%:%410=183%:%
%:%411=183%:%
%:%412=184%:%
%:%413=185%:%
%:%414=185%:%
%:%415=185%:%
%:%416=186%:%
%:%417=187%:%
%:%418=187%:%
%:%419=188%:%
%:%420=188%:%
%:%421=189%:%
%:%422=189%:%
%:%423=189%:%
%:%424=190%:%
%:%425=190%:%
%:%426=190%:%
%:%427=190%:%
%:%428=191%:%
%:%429=191%:%
%:%430=191%:%
%:%431=192%:%
%:%432=192%:%
%:%433=192%:%
%:%434=192%:%
%:%435=192%:%
%:%436=193%:%
%:%437=193%:%
%:%438=194%:%
%:%439=194%:%
%:%440=194%:%
%:%441=195%:%
%:%442=195%:%
%:%443=195%:%
%:%444=195%:%
%:%445=196%:%
%:%446=196%:%
%:%447=197%:%
%:%448=197%:%
%:%449=198%:%
%:%450=198%:%
%:%451=199%:%
%:%452=199%:%
%:%453=200%:%
%:%454=200%:%
%:%455=200%:%
%:%456=201%:%
%:%457=201%:%
%:%458=202%:%
%:%459=202%:%
%:%460=203%:%
%:%461=203%:%
%:%462=204%:%
%:%463=205%:%
%:%464=205%:%
%:%465=206%:%
%:%466=206%:%
%:%467=207%:%
%:%468=207%:%
%:%469=208%:%
%:%470=208%:%
%:%471=209%:%
%:%472=209%:%
%:%473=210%:%
%:%474=210%:%
%:%475=211%:%
%:%476=211%:%
%:%477=212%:%
%:%478=212%:%
%:%479=213%:%
%:%480=213%:%
%:%481=214%:%
%:%482=214%:%
%:%483=215%:%
%:%484=215%:%
%:%485=216%:%
%:%486=216%:%
%:%487=217%:%
%:%488=217%:%
%:%489=217%:%
%:%490=218%:%
%:%491=218%:%
%:%492=219%:%
%:%493=219%:%
%:%494=220%:%
%:%495=220%:%
%:%496=221%:%
%:%497=221%:%
%:%498=222%:%
%:%499=222%:%
%:%500=223%:%
%:%501=223%:%
%:%502=224%:%
%:%503=224%:%
%:%504=224%:%
%:%505=224%:%
%:%506=225%:%
%:%507=225%:%
%:%508=225%:%
%:%509=226%:%
%:%510=226%:%
%:%511=227%:%
%:%512=227%:%
%:%513=228%:%
%:%514=228%:%
%:%515=229%:%
%:%516=229%:%
%:%517=230%:%
%:%518=230%:%
%:%519=231%:%
%:%520=231%:%
%:%521=232%:%
%:%522=232%:%
%:%523=233%:%
%:%524=233%:%
%:%525=233%:%
%:%526=233%:%
%:%527=234%:%
%:%528=234%:%
%:%529=235%:%
%:%530=236%:%
%:%531=236%:%
%:%532=236%:%
%:%533=237%:%
%:%534=237%:%
%:%535=238%:%
%:%536=238%:%
%:%537=239%:%
%:%538=239%:%
%:%539=240%:%
%:%540=240%:%
%:%541=241%:%
%:%547=241%:%
%:%550=242%:%
%:%551=243%:%
%:%552=243%:%
%:%553=244%:%
%:%554=245%:%
%:%555=246%:%
%:%556=247%:%
%:%559=248%:%
%:%560=249%:%
%:%564=249%:%
%:%565=249%:%
%:%566=250%:%
%:%567=250%:%
%:%568=251%:%
%:%569=251%:%
%:%570=252%:%
%:%571=252%:%
%:%572=253%:%
%:%573=253%:%
%:%574=254%:%
%:%575=254%:%
%:%576=255%:%
%:%577=255%:%
%:%578=256%:%
%:%579=256%:%
%:%580=257%:%
%:%581=257%:%
%:%582=258%:%
%:%583=258%:%
%:%584=259%:%
%:%585=259%:%
%:%586=260%:%
%:%587=260%:%
%:%588=261%:%
%:%589=261%:%
%:%590=262%:%
%:%591=262%:%
%:%596=262%:%
%:%599=263%:%
%:%600=264%:%
%:%601=264%:%
%:%602=265%:%
%:%603=266%:%
%:%604=267%:%
%:%607=268%:%
%:%608=269%:%
%:%612=269%:%
%:%613=269%:%
%:%614=270%:%
%:%615=270%:%
%:%616=271%:%
%:%617=271%:%
%:%618=272%:%
%:%619=272%:%
%:%620=273%:%
%:%621=273%:%
%:%622=274%:%
%:%623=274%:%
%:%624=275%:%
%:%625=275%:%
%:%626=276%:%
%:%627=276%:%
%:%628=277%:%
%:%629=277%:%
%:%630=278%:%
%:%631=278%:%
%:%632=279%:%
%:%633=279%:%
%:%634=280%:%
%:%635=280%:%
%:%636=281%:%
%:%637=281%:%
%:%642=281%:%
%:%645=282%:%
%:%646=283%:%
%:%647=283%:%
%:%648=284%:%
%:%649=285%:%
%:%650=286%:%
%:%657=287%:%
%:%658=287%:%
%:%659=288%:%
%:%660=288%:%
%:%661=289%:%
%:%662=289%:%
%:%663=290%:%
%:%664=290%:%
%:%665=290%:%
%:%666=291%:%
%:%667=292%:%
%:%668=292%:%
%:%669=293%:%
%:%670=293%:%
%:%671=294%:%
%:%672=294%:%
%:%673=295%:%
%:%674=295%:%
%:%675=296%:%
%:%676=297%:%
%:%677=297%:%
%:%678=298%:%
%:%679=298%:%
%:%680=299%:%
%:%681=299%:%
%:%682=300%:%
%:%683=301%:%
%:%684=301%:%
%:%685=302%:%
%:%686=302%:%
%:%687=303%:%
%:%688=303%:%
%:%689=303%:%
%:%690=304%:%
%:%691=304%:%
%:%692=304%:%
%:%693=304%:%
%:%694=304%:%
%:%695=305%:%
%:%696=305%:%
%:%697=305%:%
%:%698=305%:%
%:%699=306%:%
%:%700=306%:%
%:%701=306%:%
%:%702=306%:%
%:%703=306%:%
%:%704=307%:%
%:%705=307%:%
%:%706=307%:%
%:%707=307%:%
%:%708=308%:%
%:%709=308%:%
%:%710=308%:%
%:%711=308%:%
%:%712=308%:%
%:%713=309%:%
%:%714=309%:%
%:%715=309%:%
%:%716=309%:%
%:%717=310%:%
%:%718=310%:%
%:%719=311%:%
%:%720=311%:%
%:%721=311%:%
%:%722=312%:%
%:%723=312%:%
%:%724=313%:%
%:%725=313%:%
%:%726=314%:%
%:%727=314%:%
%:%728=315%:%
%:%729=315%:%
%:%730=316%:%
%:%731=317%:%
%:%732=317%:%
%:%733=317%:%
%:%734=317%:%
%:%735=317%:%
%:%736=318%:%
%:%742=318%:%
%:%745=319%:%
%:%746=320%:%
%:%747=320%:%
%:%748=321%:%
%:%749=322%:%
%:%750=323%:%
%:%757=324%:%
%:%758=324%:%
%:%759=325%:%
%:%760=325%:%
%:%761=326%:%
%:%762=326%:%
%:%763=327%:%
%:%764=327%:%
%:%765=328%:%
%:%766=328%:%
%:%767=329%:%
%:%768=329%:%
%:%769=330%:%
%:%770=330%:%
%:%771=331%:%
%:%772=331%:%
%:%773=332%:%
%:%774=332%:%
%:%775=333%:%
%:%776=333%:%
%:%777=334%:%
%:%778=334%:%
%:%779=334%:%
%:%780=335%:%
%:%781=335%:%
%:%782=336%:%
%:%783=336%:%
%:%784=337%:%
%:%785=337%:%
%:%786=338%:%
%:%787=338%:%
%:%788=339%:%
%:%789=339%:%
%:%790=340%:%
%:%791=340%:%
%:%792=340%:%
%:%793=341%:%
%:%794=341%:%
%:%795=341%:%
%:%796=342%:%
%:%797=342%:%
%:%798=343%:%
%:%799=343%:%
%:%800=343%:%
%:%801=343%:%
%:%802=343%:%
%:%803=344%:%
%:%809=344%:%
%:%812=345%:%
%:%813=346%:%
%:%814=346%:%
%:%815=347%:%
%:%816=348%:%
%:%817=349%:%
%:%824=350%:%
%:%825=350%:%
%:%826=351%:%
%:%827=351%:%
%:%828=352%:%
%:%829=352%:%
%:%830=353%:%
%:%831=353%:%
%:%832=353%:%
%:%833=354%:%
%:%834=355%:%
%:%835=355%:%
%:%836=355%:%
%:%837=355%:%
%:%838=356%:%
%:%839=356%:%
%:%840=356%:%
%:%841=356%:%
%:%842=357%:%
%:%843=357%:%
%:%844=358%:%
%:%845=358%:%
%:%846=358%:%
%:%847=359%:%
%:%848=359%:%
%:%849=359%:%
%:%850=359%:%
%:%851=360%:%
%:%852=361%:%
%:%853=361%:%
%:%854=362%:%
%:%855=362%:%
%:%856=363%:%
%:%857=363%:%
%:%858=363%:%
%:%859=364%:%
%:%860=364%:%
%:%861=364%:%
%:%862=364%:%
%:%863=364%:%
%:%864=365%:%
%:%865=365%:%
%:%866=365%:%
%:%867=365%:%
%:%868=365%:%
%:%869=366%:%
%:%870=366%:%
%:%871=366%:%
%:%872=366%:%
%:%873=366%:%
%:%874=367%:%
%:%875=367%:%
%:%876=368%:%
%:%877=369%:%
%:%878=369%:%
%:%879=370%:%
%:%880=370%:%
%:%881=371%:%
%:%882=371%:%
%:%883=371%:%
%:%884=372%:%
%:%885=372%:%
%:%886=372%:%
%:%887=373%:%
%:%888=373%:%
%:%889=374%:%
%:%890=374%:%
%:%891=375%:%
%:%892=375%:%
%:%893=376%:%
%:%894=376%:%
%:%895=376%:%
%:%896=376%:%
%:%897=376%:%
%:%898=377%:%
%:%899=377%:%
%:%900=377%:%
%:%901=377%:%
%:%902=377%:%
%:%903=378%:%
%:%904=378%:%
%:%905=378%:%
%:%906=378%:%
%:%907=379%:%
%:%908=379%:%
%:%909=379%:%
%:%910=379%:%
%:%911=379%:%
%:%912=380%:%
%:%913=380%:%
%:%914=380%:%
%:%915=380%:%
%:%916=380%:%
%:%917=381%:%
%:%918=381%:%
%:%919=382%:%
%:%920=383%:%
%:%921=383%:%
%:%922=384%:%
%:%923=384%:%
%:%924=385%:%
%:%925=385%:%
%:%926=386%:%
%:%927=386%:%
%:%928=387%:%
%:%929=387%:%
%:%930=388%:%
%:%931=388%:%
%:%932=389%:%
%:%933=389%:%
%:%934=390%:%
%:%935=390%:%
%:%936=391%:%
%:%937=392%:%
%:%938=392%:%
%:%939=392%:%
%:%940=392%:%
%:%941=392%:%
%:%942=393%:%
%:%948=393%:%
%:%951=394%:%
%:%952=395%:%
%:%953=395%:%
%:%954=396%:%
%:%955=397%:%
%:%956=398%:%
%:%963=399%:%
%:%964=399%:%
%:%965=400%:%
%:%966=400%:%
%:%967=401%:%
%:%968=401%:%
%:%969=401%:%
%:%970=402%:%
%:%971=402%:%
%:%972=403%:%
%:%973=403%:%
%:%974=404%:%
%:%975=404%:%
%:%976=405%:%
%:%977=405%:%
%:%978=406%:%
%:%979=406%:%
%:%980=407%:%
%:%981=407%:%
%:%982=408%:%
%:%983=408%:%
%:%984=409%:%
%:%985=409%:%
%:%986=410%:%
%:%987=410%:%
%:%988=411%:%
%:%989=411%:%
%:%990=411%:%
%:%991=412%:%
%:%992=412%:%
%:%993=413%:%
%:%994=413%:%
%:%995=414%:%
%:%996=414%:%
%:%997=415%:%
%:%998=415%:%
%:%999=416%:%
%:%1000=416%:%
%:%1001=417%:%
%:%1002=417%:%
%:%1003=418%:%
%:%1004=418%:%
%:%1005=419%:%
%:%1006=419%:%
%:%1007=420%:%
%:%1008=420%:%
%:%1009=421%:%
%:%1010=421%:%
%:%1011=422%:%
%:%1012=422%:%
%:%1013=423%:%
%:%1014=424%:%
%:%1015=424%:%
%:%1016=425%:%
%:%1017=425%:%
%:%1018=426%:%
%:%1019=426%:%
%:%1020=427%:%
%:%1021=427%:%
%:%1022=428%:%
%:%1023=428%:%
%:%1024=428%:%
%:%1025=429%:%
%:%1026=429%:%
%:%1027=430%:%
%:%1028=430%:%
%:%1029=431%:%
%:%1030=431%:%
%:%1031=432%:%
%:%1032=432%:%
%:%1033=433%:%
%:%1034=433%:%
%:%1035=433%:%
%:%1036=434%:%
%:%1037=434%:%
%:%1038=435%:%
%:%1039=435%:%
%:%1040=436%:%
%:%1041=436%:%
%:%1042=437%:%
%:%1043=437%:%
%:%1044=437%:%
%:%1045=437%:%
%:%1046=437%:%
%:%1047=438%:%
%:%1048=439%:%
%:%1049=439%:%
%:%1050=439%:%
%:%1051=439%:%
%:%1052=439%:%
%:%1053=440%:%
%:%1059=440%:%
%:%1062=441%:%
%:%1063=442%:%
%:%1064=442%:%
%:%1065=443%:%
%:%1072=444%:%
%:%1073=444%:%
%:%1074=445%:%
%:%1075=445%:%
%:%1076=446%:%
%:%1077=446%:%
%:%1078=447%:%
%:%1079=447%:%
%:%1080=448%:%
%:%1081=448%:%
%:%1082=449%:%
%:%1083=449%:%
%:%1084=450%:%
%:%1085=450%:%
%:%1086=451%:%
%:%1087=451%:%
%:%1088=452%:%
%:%1089=452%:%
%:%1090=453%:%
%:%1091=453%:%
%:%1092=454%:%
%:%1093=454%:%
%:%1094=455%:%
%:%1095=455%:%
%:%1096=456%:%
%:%1097=456%:%
%:%1098=457%:%
%:%1099=457%:%
%:%1100=458%:%
%:%1101=458%:%
%:%1102=459%:%
%:%1103=459%:%
%:%1104=460%:%
%:%1105=460%:%
%:%1106=461%:%
%:%1107=461%:%
%:%1108=462%:%
%:%1109=462%:%
%:%1110=463%:%
%:%1111=463%:%
%:%1112=464%:%
%:%1118=464%:%
%:%1121=465%:%
%:%1122=466%:%
%:%1123=466%:%
%:%1124=467%:%
%:%1125=468%:%
%:%1126=469%:%
%:%1129=470%:%
%:%1130=471%:%
%:%1134=471%:%
%:%1135=471%:%
%:%1136=472%:%
%:%1137=472%:%
%:%1138=473%:%
%:%1139=473%:%
%:%1140=474%:%
%:%1141=474%:%
%:%1146=474%:%
%:%1149=475%:%
%:%1150=476%:%
%:%1151=476%:%
%:%1152=477%:%
%:%1153=478%:%
%:%1154=478%:%
%:%1155=479%:%
%:%1156=480%:%
%:%1157=481%:%
%:%1160=482%:%
%:%1161=483%:%
%:%1165=483%:%
%:%1166=483%:%
%:%1167=484%:%
%:%1168=484%:%
%:%1169=485%:%
%:%1170=485%:%
%:%1171=486%:%
%:%1172=486%:%
%:%1173=487%:%
%:%1174=487%:%
%:%1179=487%:%
%:%1182=488%:%
%:%1183=489%:%
%:%1184=489%:%
%:%1187=490%:%
%:%1191=490%:%
%:%1192=490%:%
%:%1193=491%:%
%:%1194=491%:%
%:%1195=492%:%
%:%1196=492%:%
%:%1197=493%:%
%:%1198=493%:%
%:%1199=494%:%
%:%1200=494%:%
%:%1201=495%:%
%:%1202=495%:%
%:%1203=496%:%
%:%1204=496%:%
%:%1205=497%:%
%:%1206=497%:%
%:%1207=498%:%
%:%1208=498%:%
%:%1213=498%:%
%:%1216=499%:%
%:%1217=500%:%
%:%1218=500%:%
%:%1221=501%:%
%:%1225=501%:%
%:%1226=501%:%
%:%1227=502%:%
%:%1228=502%:%
%:%1233=502%:%
%:%1236=503%:%
%:%1237=504%:%
%:%1238=504%:%
%:%1239=505%:%
%:%1240=506%:%
%:%1241=507%:%
%:%1248=508%:%
%:%1249=508%:%
%:%1250=509%:%
%:%1251=509%:%
%:%1252=510%:%
%:%1253=510%:%
%:%1254=510%:%
%:%1255=511%:%
%:%1256=511%:%
%:%1257=511%:%
%:%1258=512%:%
%:%1259=512%:%
%:%1260=513%:%
%:%1261=513%:%
%:%1262=514%:%
%:%1263=514%:%
%:%1264=515%:%
%:%1265=515%:%
%:%1266=516%:%
%:%1267=516%:%
%:%1268=517%:%
%:%1269=517%:%
%:%1270=518%:%
%:%1271=518%:%
%:%1272=518%:%
%:%1273=519%:%
%:%1274=519%:%
%:%1275=520%:%
%:%1276=520%:%
%:%1277=521%:%
%:%1278=521%:%
%:%1279=521%:%
%:%1280=521%:%
%:%1281=522%:%
%:%1287=522%:%
%:%1290=523%:%
%:%1291=524%:%
%:%1292=524%:%
%:%1293=525%:%
%:%1294=526%:%
%:%1295=527%:%
%:%1302=528%:%
%:%1303=528%:%
%:%1304=529%:%
%:%1305=529%:%
%:%1306=530%:%
%:%1307=530%:%
%:%1308=531%:%
%:%1309=532%:%
%:%1310=532%:%
%:%1311=533%:%
%:%1312=533%:%
%:%1313=534%:%
%:%1314=534%:%
%:%1315=535%:%
%:%1316=535%:%
%:%1317=536%:%
%:%1318=536%:%
%:%1319=537%:%
%:%1320=537%:%
%:%1321=538%:%
%:%1322=539%:%
%:%1323=539%:%
%:%1324=539%:%
%:%1325=540%:%
%:%1326=540%:%
%:%1327=541%:%
%:%1328=542%:%
%:%1329=542%:%
%:%1330=542%:%
%:%1331=543%:%
%:%1332=543%:%
%:%1333=544%:%
%:%1334=544%:%
%:%1335=545%:%
%:%1336=545%:%
%:%1337=546%:%
%:%1338=546%:%
%:%1339=547%:%
%:%1340=547%:%
%:%1341=548%:%
%:%1342=548%:%
%:%1343=549%:%
%:%1344=550%:%
%:%1345=550%:%
%:%1346=551%:%
%:%1347=551%:%
%:%1348=552%:%
%:%1349=552%:%
%:%1350=553%:%
%:%1351=553%:%
%:%1352=554%:%
%:%1353=554%:%
%:%1354=554%:%
%:%1355=555%:%
%:%1356=555%:%
%:%1357=556%:%
%:%1358=556%:%
%:%1359=557%:%
%:%1360=558%:%
%:%1361=558%:%
%:%1362=559%:%
%:%1363=559%:%
%:%1364=560%:%
%:%1365=560%:%
%:%1366=561%:%
%:%1367=561%:%
%:%1368=562%:%
%:%1369=562%:%
%:%1370=562%:%
%:%1371=563%:%
%:%1372=563%:%
%:%1373=564%:%
%:%1374=564%:%
%:%1375=564%:%
%:%1376=565%:%
%:%1377=565%:%
%:%1378=565%:%
%:%1379=565%:%
%:%1380=566%:%
%:%1381=567%:%
%:%1382=567%:%
%:%1383=568%:%
%:%1384=568%:%
%:%1385=569%:%
%:%1386=569%:%
%:%1387=570%:%
%:%1388=570%:%
%:%1389=571%:%
%:%1390=571%:%
%:%1391=572%:%
%:%1392=572%:%
%:%1393=573%:%
%:%1394=573%:%
%:%1395=574%:%
%:%1396=574%:%
%:%1397=575%:%
%:%1398=575%:%
%:%1399=576%:%
%:%1400=576%:%
%:%1401=577%:%
%:%1402=577%:%
%:%1403=578%:%
%:%1404=578%:%
%:%1405=579%:%
%:%1406=579%:%
%:%1407=579%:%
%:%1408=580%:%
%:%1409=580%:%
%:%1410=580%:%
%:%1411=580%:%
%:%1412=581%:%
%:%1413=582%:%
%:%1414=582%:%
%:%1415=582%:%
%:%1416=582%:%
%:%1417=583%:%
%:%1418=584%:%
%:%1419=584%:%
%:%1420=585%:%
%:%1421=585%:%
%:%1422=586%:%
%:%1423=586%:%
%:%1424=587%:%
%:%1425=587%:%
%:%1426=588%:%
%:%1427=588%:%
%:%1428=588%:%
%:%1429=589%:%
%:%1430=589%:%
%:%1431=589%:%
%:%1432=589%:%
%:%1433=590%:%
%:%1434=590%:%
%:%1435=591%:%
%:%1436=591%:%
%:%1437=592%:%
%:%1438=592%:%
%:%1439=593%:%
%:%1440=593%:%
%:%1441=594%:%
%:%1442=595%:%
%:%1443=595%:%
%:%1444=596%:%
%:%1445=596%:%
%:%1446=597%:%
%:%1447=597%:%
%:%1448=598%:%
%:%1449=598%:%
%:%1450=599%:%
%:%1451=599%:%
%:%1452=599%:%
%:%1453=600%:%
%:%1454=600%:%
%:%1455=600%:%
%:%1456=600%:%
%:%1457=601%:%
%:%1458=602%:%
%:%1459=602%:%
%:%1460=603%:%
%:%1461=603%:%
%:%1462=603%:%
%:%1463=604%:%
%:%1464=604%:%
%:%1465=604%:%
%:%1466=605%:%
%:%1467=605%:%
%:%1468=605%:%
%:%1469=606%:%
%:%1470=606%:%
%:%1471=607%:%
%:%1472=608%:%
%:%1473=608%:%
%:%1474=608%:%
%:%1475=608%:%
%:%1476=609%:%
%:%1477=609%:%
%:%1478=610%:%
%:%1479=611%:%
%:%1480=611%:%
%:%1481=611%:%
%:%1482=612%:%
%:%1483=612%:%
%:%1484=613%:%
%:%1485=613%:%
%:%1486=614%:%
%:%1487=615%:%
%:%1488=615%:%
%:%1489=615%:%
%:%1490=616%:%
%:%1491=616%:%
%:%1492=617%:%
%:%1493=617%:%
%:%1494=618%:%
%:%1495=618%:%
%:%1496=619%:%
%:%1497=620%:%
%:%1498=620%:%
%:%1499=621%:%
%:%1500=621%:%
%:%1501=622%:%
%:%1502=622%:%
%:%1503=623%:%
%:%1504=623%:%
%:%1505=624%:%
%:%1506=625%:%
%:%1507=625%:%
%:%1508=625%:%
%:%1509=626%:%
%:%1510=626%:%
%:%1511=626%:%
%:%1512=627%:%
%:%1513=628%:%
%:%1514=628%:%
%:%1515=629%:%
%:%1516=629%:%
%:%1517=630%:%
%:%1518=630%:%
%:%1519=631%:%
%:%1520=631%:%
%:%1521=632%:%
%:%1522=632%:%
%:%1523=633%:%
%:%1524=633%:%
%:%1525=634%:%
%:%1526=634%:%
%:%1527=635%:%
%:%1528=635%:%
%:%1529=636%:%
%:%1530=637%:%
%:%1531=637%:%
%:%1532=638%:%
%:%1533=638%:%
%:%1534=639%:%
%:%1535=639%:%
%:%1536=639%:%
%:%1537=640%:%
%:%1538=640%:%
%:%1539=640%:%
%:%1540=640%:%
%:%1541=641%:%
%:%1542=641%:%
%:%1543=641%:%
%:%1544=642%:%
%:%1545=642%:%
%:%1546=643%:%
%:%1547=643%:%
%:%1548=644%:%
%:%1549=644%:%
%:%1550=645%:%
%:%1551=645%:%
%:%1552=646%:%
%:%1553=646%:%
%:%1554=647%:%
%:%1555=647%:%
%:%1556=648%:%
%:%1557=648%:%
%:%1558=649%:%
%:%1559=649%:%
%:%1560=649%:%
%:%1561=650%:%
%:%1562=650%:%
%:%1563=651%:%
%:%1564=651%:%
%:%1565=652%:%
%:%1566=652%:%
%:%1567=653%:%
%:%1568=654%:%
%:%1569=654%:%
%:%1570=655%:%
%:%1571=655%:%
%:%1572=656%:%
%:%1573=656%:%
%:%1574=657%:%
%:%1575=657%:%
%:%1576=658%:%
%:%1582=658%:%
%:%1585=659%:%
%:%1586=660%:%
%:%1587=660%:%
%:%1594=661%:%

%
\begin{isabellebody}%
\setisabellecontext{SymExt{\isacharunderscore}{\kern0pt}Replacement}%
%
\isadelimtheory
%
\endisadelimtheory
%
\isatagtheory
\isacommand{theory}\isamarkupfalse%
\ SymExt{\isacharunderscore}{\kern0pt}Replacement\isanewline
\ \ \isakeyword{imports}\ \ \isanewline
\ \ \ \ SymExt{\isacharunderscore}{\kern0pt}Separation\isanewline
\ \ \ \ SymExt{\isacharunderscore}{\kern0pt}Separation{\isacharunderscore}{\kern0pt}Base\isanewline
\isakeyword{begin}%
\endisatagtheory
{\isafoldtheory}%
%
\isadelimtheory
\ \isanewline
%
\endisadelimtheory
\isanewline
\isacommand{context}\isamarkupfalse%
\ M{\isacharunderscore}{\kern0pt}symmetric{\isacharunderscore}{\kern0pt}system{\isacharunderscore}{\kern0pt}G{\isacharunderscore}{\kern0pt}generic\isanewline
\isakeyword{begin}\isanewline
\isanewline
\isacommand{definition}\isamarkupfalse%
\ is{\isacharunderscore}{\kern0pt}MVset{\isacharunderscore}{\kern0pt}fm\ \isakeyword{where}\ {\isachardoublequoteopen}is{\isacharunderscore}{\kern0pt}MVset{\isacharunderscore}{\kern0pt}fm{\isacharparenleft}{\kern0pt}a{\isacharcomma}{\kern0pt}\ V{\isacharparenright}{\kern0pt}\ {\isasymequiv}\ And{\isacharparenleft}{\kern0pt}ordinal{\isacharunderscore}{\kern0pt}fm{\isacharparenleft}{\kern0pt}a{\isacharparenright}{\kern0pt}{\isacharcomma}{\kern0pt}\ Exists{\isacharparenleft}{\kern0pt}And{\isacharparenleft}{\kern0pt}empty{\isacharunderscore}{\kern0pt}fm{\isacharparenleft}{\kern0pt}{\isadigit{0}}{\isacharparenright}{\kern0pt}{\isacharcomma}{\kern0pt}\ is{\isacharunderscore}{\kern0pt}transrec{\isacharunderscore}{\kern0pt}fm{\isacharparenleft}{\kern0pt}is{\isacharunderscore}{\kern0pt}HVfrom{\isacharunderscore}{\kern0pt}fm{\isacharparenleft}{\kern0pt}{\isadigit{8}}{\isacharcomma}{\kern0pt}\ {\isadigit{2}}{\isacharcomma}{\kern0pt}\ {\isadigit{1}}{\isacharcomma}{\kern0pt}\ {\isadigit{0}}{\isacharparenright}{\kern0pt}{\isacharcomma}{\kern0pt}\ succ{\isacharparenleft}{\kern0pt}a{\isacharparenright}{\kern0pt}{\isacharcomma}{\kern0pt}\ succ{\isacharparenleft}{\kern0pt}V{\isacharparenright}{\kern0pt}{\isacharparenright}{\kern0pt}{\isacharparenright}{\kern0pt}{\isacharparenright}{\kern0pt}{\isacharparenright}{\kern0pt}{\isachardoublequoteclose}\ \isanewline
\isanewline
\isacommand{lemma}\isamarkupfalse%
\ is{\isacharunderscore}{\kern0pt}MVset{\isacharunderscore}{\kern0pt}fm{\isacharunderscore}{\kern0pt}type\ {\isacharcolon}{\kern0pt}\ \isanewline
\ \ \isakeyword{fixes}\ i\ j\ \isanewline
\ \ \isakeyword{assumes}\ {\isachardoublequoteopen}i\ {\isasymin}\ nat{\isachardoublequoteclose}\ {\isachardoublequoteopen}j\ {\isasymin}\ nat{\isachardoublequoteclose}\ \isanewline
\ \ \isakeyword{shows}\ {\isachardoublequoteopen}is{\isacharunderscore}{\kern0pt}MVset{\isacharunderscore}{\kern0pt}fm{\isacharparenleft}{\kern0pt}i{\isacharcomma}{\kern0pt}\ j{\isacharparenright}{\kern0pt}\ {\isasymin}\ formula{\isachardoublequoteclose}\isanewline
%
\isadelimproof
\ \ %
\endisadelimproof
%
\isatagproof
\isacommand{unfolding}\isamarkupfalse%
\ is{\isacharunderscore}{\kern0pt}MVset{\isacharunderscore}{\kern0pt}fm{\isacharunderscore}{\kern0pt}def\isanewline
\ \ \isacommand{using}\isamarkupfalse%
\ assms\ \isanewline
\ \ \isacommand{by}\isamarkupfalse%
\ auto%
\endisatagproof
{\isafoldproof}%
%
\isadelimproof
\isanewline
%
\endisadelimproof
\isanewline
\isacommand{lemma}\isamarkupfalse%
\ arity{\isacharunderscore}{\kern0pt}is{\isacharunderscore}{\kern0pt}MVset{\isacharunderscore}{\kern0pt}fm\ {\isacharcolon}{\kern0pt}\ \isanewline
\ \ \isakeyword{fixes}\ i\ j\ \isanewline
\ \ \isakeyword{assumes}\ {\isachardoublequoteopen}i\ {\isasymin}\ nat{\isachardoublequoteclose}\ {\isachardoublequoteopen}j\ {\isasymin}\ nat{\isachardoublequoteclose}\ \isanewline
\ \ \isakeyword{shows}\ {\isachardoublequoteopen}arity{\isacharparenleft}{\kern0pt}is{\isacharunderscore}{\kern0pt}MVset{\isacharunderscore}{\kern0pt}fm{\isacharparenleft}{\kern0pt}i{\isacharcomma}{\kern0pt}\ j{\isacharparenright}{\kern0pt}{\isacharparenright}{\kern0pt}\ {\isasymle}\ succ{\isacharparenleft}{\kern0pt}i{\isacharparenright}{\kern0pt}\ {\isasymunion}\ succ{\isacharparenleft}{\kern0pt}j{\isacharparenright}{\kern0pt}{\isachardoublequoteclose}\ \isanewline
%
\isadelimproof
\ \ %
\endisadelimproof
%
\isatagproof
\isacommand{unfolding}\isamarkupfalse%
\ is{\isacharunderscore}{\kern0pt}MVset{\isacharunderscore}{\kern0pt}fm{\isacharunderscore}{\kern0pt}def\ \isanewline
\ \ \isacommand{apply}\isamarkupfalse%
\ simp\isanewline
\ \ \isacommand{apply}\isamarkupfalse%
{\isacharparenleft}{\kern0pt}subst\ arity{\isacharunderscore}{\kern0pt}ordinal{\isacharunderscore}{\kern0pt}fm{\isacharcomma}{\kern0pt}\ simp\ add{\isacharcolon}{\kern0pt}assms{\isacharparenright}{\kern0pt}\isanewline
\ \ \isacommand{using}\isamarkupfalse%
\ assms\isanewline
\ \ \isacommand{apply}\isamarkupfalse%
{\isacharparenleft}{\kern0pt}rule{\isacharunderscore}{\kern0pt}tac\ Un{\isacharunderscore}{\kern0pt}least{\isacharunderscore}{\kern0pt}lt{\isacharcomma}{\kern0pt}\ simp{\isacharcomma}{\kern0pt}\ rule{\isacharunderscore}{\kern0pt}tac\ ltI{\isacharcomma}{\kern0pt}\ simp{\isacharunderscore}{\kern0pt}all{\isacharparenright}{\kern0pt}\isanewline
\ \ \isacommand{apply}\isamarkupfalse%
{\isacharparenleft}{\kern0pt}rule\ pred{\isacharunderscore}{\kern0pt}le{\isacharcomma}{\kern0pt}\ simp{\isacharcomma}{\kern0pt}\ simp{\isacharcomma}{\kern0pt}\ rule{\isacharunderscore}{\kern0pt}tac\ Un{\isacharunderscore}{\kern0pt}least{\isacharunderscore}{\kern0pt}lt{\isacharcomma}{\kern0pt}\ subst\ arity{\isacharunderscore}{\kern0pt}empty{\isacharunderscore}{\kern0pt}fm{\isacharcomma}{\kern0pt}\ simp{\isacharunderscore}{\kern0pt}all{\isacharparenright}{\kern0pt}\isanewline
\ \ \isacommand{unfolding}\isamarkupfalse%
\ is{\isacharunderscore}{\kern0pt}transrec{\isacharunderscore}{\kern0pt}fm{\isacharunderscore}{\kern0pt}def\isanewline
\ \ \isacommand{apply}\isamarkupfalse%
\ simp\isanewline
\ \ \isacommand{apply}\isamarkupfalse%
{\isacharparenleft}{\kern0pt}rule\ pred{\isacharunderscore}{\kern0pt}le{\isacharcomma}{\kern0pt}\ simp{\isacharunderscore}{\kern0pt}all{\isacharparenright}{\kern0pt}{\isacharplus}{\kern0pt}\isanewline
\ \ \isacommand{apply}\isamarkupfalse%
{\isacharparenleft}{\kern0pt}subst\ arity{\isacharunderscore}{\kern0pt}upair{\isacharunderscore}{\kern0pt}fm{\isacharcomma}{\kern0pt}\ simp{\isacharunderscore}{\kern0pt}all{\isacharparenright}{\kern0pt}\isanewline
\ \ \isacommand{apply}\isamarkupfalse%
{\isacharparenleft}{\kern0pt}subst\ arity{\isacharunderscore}{\kern0pt}is{\isacharunderscore}{\kern0pt}eclose{\isacharunderscore}{\kern0pt}fm{\isacharcomma}{\kern0pt}\ simp{\isacharunderscore}{\kern0pt}all{\isacharparenright}{\kern0pt}\isanewline
\ \ \isacommand{apply}\isamarkupfalse%
{\isacharparenleft}{\kern0pt}subst\ arity{\isacharunderscore}{\kern0pt}Memrel{\isacharunderscore}{\kern0pt}fm{\isacharcomma}{\kern0pt}\ simp{\isacharunderscore}{\kern0pt}all{\isacharparenright}{\kern0pt}\isanewline
\ \ \isacommand{apply}\isamarkupfalse%
{\isacharparenleft}{\kern0pt}subst\ arity{\isacharunderscore}{\kern0pt}is{\isacharunderscore}{\kern0pt}wfrec{\isacharunderscore}{\kern0pt}fm{\isacharcomma}{\kern0pt}\ simp{\isacharunderscore}{\kern0pt}all{\isacharparenright}{\kern0pt}\isanewline
\ \ \isacommand{apply}\isamarkupfalse%
{\isacharparenleft}{\kern0pt}rule\ Un{\isacharunderscore}{\kern0pt}least{\isacharunderscore}{\kern0pt}lt{\isacharparenright}{\kern0pt}{\isacharplus}{\kern0pt}\isanewline
\ \ \ \ \isacommand{apply}\isamarkupfalse%
{\isacharparenleft}{\kern0pt}subst\ succ{\isacharunderscore}{\kern0pt}Un{\isacharunderscore}{\kern0pt}distrib{\isacharcomma}{\kern0pt}\ simp{\isacharunderscore}{\kern0pt}all{\isacharparenright}{\kern0pt}{\isacharplus}{\kern0pt}\isanewline
\ \ \ \isacommand{apply}\isamarkupfalse%
{\isacharparenleft}{\kern0pt}rule\ ltI{\isacharcomma}{\kern0pt}\ simp{\isacharcomma}{\kern0pt}\ simp{\isacharparenright}{\kern0pt}\isanewline
\ \ \isacommand{apply}\isamarkupfalse%
{\isacharparenleft}{\kern0pt}rule\ Un{\isacharunderscore}{\kern0pt}least{\isacharunderscore}{\kern0pt}lt{\isacharparenright}{\kern0pt}{\isacharplus}{\kern0pt}\isanewline
\ \ \ \ \isacommand{apply}\isamarkupfalse%
\ simp{\isacharunderscore}{\kern0pt}all\isanewline
\ \ \isacommand{apply}\isamarkupfalse%
{\isacharparenleft}{\kern0pt}rule\ Un{\isacharunderscore}{\kern0pt}least{\isacharunderscore}{\kern0pt}lt{\isacharparenright}{\kern0pt}{\isacharplus}{\kern0pt}\isanewline
\ \ \ \ \isacommand{apply}\isamarkupfalse%
\ simp{\isacharunderscore}{\kern0pt}all\ \ \ \ \ \ \ \isanewline
\ \ \isacommand{apply}\isamarkupfalse%
{\isacharparenleft}{\kern0pt}rule\ Un{\isacharunderscore}{\kern0pt}least{\isacharunderscore}{\kern0pt}lt{\isacharparenright}{\kern0pt}{\isacharplus}{\kern0pt}\isanewline
\ \ \ \ \ \isacommand{apply}\isamarkupfalse%
\ simp{\isacharunderscore}{\kern0pt}all\isanewline
\ \ \ \ \isacommand{apply}\isamarkupfalse%
{\isacharparenleft}{\kern0pt}rule\ ltI{\isacharcomma}{\kern0pt}\ simp{\isacharunderscore}{\kern0pt}all{\isacharparenright}{\kern0pt}{\isacharplus}{\kern0pt}\isanewline
\ \ \isacommand{apply}\isamarkupfalse%
{\isacharparenleft}{\kern0pt}rule\ ltD{\isacharparenright}{\kern0pt}\isanewline
\ \ \isacommand{apply}\isamarkupfalse%
{\isacharparenleft}{\kern0pt}rule\ pred{\isacharunderscore}{\kern0pt}le{\isacharcomma}{\kern0pt}\ simp{\isacharunderscore}{\kern0pt}all{\isacharparenright}{\kern0pt}{\isacharplus}{\kern0pt}\isanewline
\ \ \isacommand{unfolding}\isamarkupfalse%
\ is{\isacharunderscore}{\kern0pt}HVfrom{\isacharunderscore}{\kern0pt}fm{\isacharunderscore}{\kern0pt}def\ is{\isacharunderscore}{\kern0pt}powapply{\isacharunderscore}{\kern0pt}fm{\isacharunderscore}{\kern0pt}def\isanewline
\ \ \isacommand{apply}\isamarkupfalse%
{\isacharparenleft}{\kern0pt}simp\ del{\isacharcolon}{\kern0pt}FOL{\isacharunderscore}{\kern0pt}sats{\isacharunderscore}{\kern0pt}iff\ pair{\isacharunderscore}{\kern0pt}abs\ add{\isacharcolon}{\kern0pt}\ fm{\isacharunderscore}{\kern0pt}defs\ nat{\isacharunderscore}{\kern0pt}simp{\isacharunderscore}{\kern0pt}union{\isacharparenright}{\kern0pt}\ \isanewline
\ \ \isacommand{done}\isamarkupfalse%
%
\endisatagproof
{\isafoldproof}%
%
\isadelimproof
\isanewline
%
\endisadelimproof
\isanewline
\isacommand{lemma}\isamarkupfalse%
\ sats{\isacharunderscore}{\kern0pt}is{\isacharunderscore}{\kern0pt}MVset{\isacharunderscore}{\kern0pt}fm{\isacharunderscore}{\kern0pt}iff\ {\isacharcolon}{\kern0pt}\ \isanewline
\ \ \isakeyword{fixes}\ a\ V\ i\ j\ env\ \ \isanewline
\ \ \isakeyword{assumes}\ {\isachardoublequoteopen}env\ {\isasymin}\ list{\isacharparenleft}{\kern0pt}M{\isacharparenright}{\kern0pt}{\isachardoublequoteclose}\ {\isachardoublequoteopen}nth{\isacharparenleft}{\kern0pt}i{\isacharcomma}{\kern0pt}\ env{\isacharparenright}{\kern0pt}\ {\isacharequal}{\kern0pt}\ a{\isachardoublequoteclose}\ {\isachardoublequoteopen}nth{\isacharparenleft}{\kern0pt}j{\isacharcomma}{\kern0pt}\ env{\isacharparenright}{\kern0pt}\ {\isacharequal}{\kern0pt}\ V{\isachardoublequoteclose}\ {\isachardoublequoteopen}i\ {\isacharless}{\kern0pt}\ length{\isacharparenleft}{\kern0pt}env{\isacharparenright}{\kern0pt}{\isachardoublequoteclose}\ {\isachardoublequoteopen}j\ {\isacharless}{\kern0pt}\ length{\isacharparenleft}{\kern0pt}env{\isacharparenright}{\kern0pt}{\isachardoublequoteclose}\isanewline
\ \ \isakeyword{shows}\ {\isachardoublequoteopen}sats{\isacharparenleft}{\kern0pt}M{\isacharcomma}{\kern0pt}\ is{\isacharunderscore}{\kern0pt}MVset{\isacharunderscore}{\kern0pt}fm{\isacharparenleft}{\kern0pt}i{\isacharcomma}{\kern0pt}\ j{\isacharparenright}{\kern0pt}{\isacharcomma}{\kern0pt}\ env{\isacharparenright}{\kern0pt}\ {\isasymlongleftrightarrow}\ {\isacharparenleft}{\kern0pt}a\ {\isasymin}\ M\ {\isasymand}\ Ord{\isacharparenleft}{\kern0pt}a{\isacharparenright}{\kern0pt}{\isacharparenright}{\kern0pt}\ {\isasymand}\ V\ {\isacharequal}{\kern0pt}\ MVset{\isacharparenleft}{\kern0pt}a{\isacharparenright}{\kern0pt}{\isachardoublequoteclose}\ \isanewline
%
\isadelimproof
\isanewline
\ \ %
\endisadelimproof
%
\isatagproof
\isacommand{unfolding}\isamarkupfalse%
\ is{\isacharunderscore}{\kern0pt}MVset{\isacharunderscore}{\kern0pt}fm{\isacharunderscore}{\kern0pt}def\ \isanewline
\ \ \isacommand{apply}\isamarkupfalse%
{\isacharparenleft}{\kern0pt}rule\ iff{\isacharunderscore}{\kern0pt}trans{\isacharcomma}{\kern0pt}\ rule\ sats{\isacharunderscore}{\kern0pt}And{\isacharunderscore}{\kern0pt}iff{\isacharcomma}{\kern0pt}\ simp\ add{\isacharcolon}{\kern0pt}assms{\isacharcomma}{\kern0pt}rule\ iff{\isacharunderscore}{\kern0pt}conjI{\isadigit{2}}{\isacharparenright}{\kern0pt}\ \isanewline
\ \ \ \isacommand{apply}\isamarkupfalse%
{\isacharparenleft}{\kern0pt}rule\ iff{\isacharunderscore}{\kern0pt}trans{\isacharcomma}{\kern0pt}\ rule\ sats{\isacharunderscore}{\kern0pt}ordinal{\isacharunderscore}{\kern0pt}fm{\isacharparenright}{\kern0pt}\isanewline
\ \ \isacommand{using}\isamarkupfalse%
\ assms\ trans{\isacharunderscore}{\kern0pt}M\isanewline
\ \ \ \ \ \ \isacommand{apply}\isamarkupfalse%
\ auto{\isacharbrackleft}{\kern0pt}{\isadigit{4}}{\isacharbrackright}{\kern0pt}\isanewline
\ \ \isacommand{apply}\isamarkupfalse%
{\isacharparenleft}{\kern0pt}rule\ iff{\isacharunderscore}{\kern0pt}trans{\isacharcomma}{\kern0pt}\ rule\ iff{\isacharunderscore}{\kern0pt}flip{\isacharcomma}{\kern0pt}\ rule\ is{\isacharunderscore}{\kern0pt}Vset{\isacharunderscore}{\kern0pt}iff{\isacharunderscore}{\kern0pt}sats{\isacharparenright}{\kern0pt}\isanewline
\ \ \isacommand{using}\isamarkupfalse%
\ assms\ zero{\isacharunderscore}{\kern0pt}in{\isacharunderscore}{\kern0pt}M\ lt{\isacharunderscore}{\kern0pt}nat{\isacharunderscore}{\kern0pt}in{\isacharunderscore}{\kern0pt}nat\isanewline
\ \ \ \ \ \ \ \ \ \ \isacommand{apply}\isamarkupfalse%
\ auto{\isacharbrackleft}{\kern0pt}{\isadigit{8}}{\isacharbrackright}{\kern0pt}\isanewline
\ \ \isacommand{apply}\isamarkupfalse%
{\isacharparenleft}{\kern0pt}rule\ iff{\isacharunderscore}{\kern0pt}trans{\isacharcomma}{\kern0pt}\ rule\ Vset{\isacharunderscore}{\kern0pt}abs{\isacharparenright}{\kern0pt}\isanewline
\ \ \isacommand{using}\isamarkupfalse%
\ assms\ \isanewline
\ \ \ \ \ \isacommand{apply}\isamarkupfalse%
\ auto{\isacharbrackleft}{\kern0pt}{\isadigit{3}}{\isacharbrackright}{\kern0pt}\isanewline
\ \ \isacommand{unfolding}\isamarkupfalse%
\ MVset{\isacharunderscore}{\kern0pt}def\isanewline
\ \ \isacommand{by}\isamarkupfalse%
\ auto%
\endisatagproof
{\isafoldproof}%
%
\isadelimproof
\isanewline
%
\endisadelimproof
\isanewline
\isacommand{definition}\isamarkupfalse%
\ ren{\isacharunderscore}{\kern0pt}least{\isacharunderscore}{\kern0pt}index\ \isakeyword{where}\ {\isachardoublequoteopen}ren{\isacharunderscore}{\kern0pt}least{\isacharunderscore}{\kern0pt}index{\isacharparenleft}{\kern0pt}{\isasymphi}{\isacharparenright}{\kern0pt}\ {\isasymequiv}\isanewline
\ \ \ \ Exists{\isacharparenleft}{\kern0pt}Exists{\isacharparenleft}{\kern0pt}Exists{\isacharparenleft}{\kern0pt}Exists{\isacharparenleft}{\kern0pt}Exists{\isacharparenleft}{\kern0pt}Exists{\isacharparenleft}{\kern0pt}\isanewline
\ \ \ \ \ \ And{\isacharparenleft}{\kern0pt}Equal{\isacharparenleft}{\kern0pt}{\isadigit{0}}{\isacharcomma}{\kern0pt}\ {\isadigit{1}}{\isadigit{1}}{\isacharparenright}{\kern0pt}{\isacharcomma}{\kern0pt}\ \isanewline
\ \ \ \ \ \ And{\isacharparenleft}{\kern0pt}Equal{\isacharparenleft}{\kern0pt}{\isadigit{1}}{\isacharcomma}{\kern0pt}\ {\isadigit{1}}{\isadigit{4}}{\isacharparenright}{\kern0pt}{\isacharcomma}{\kern0pt}\ \isanewline
\ \ \ \ \ \ And{\isacharparenleft}{\kern0pt}Equal{\isacharparenleft}{\kern0pt}{\isadigit{2}}{\isacharcomma}{\kern0pt}\ {\isadigit{1}}{\isadigit{4}}{\isacharhash}{\kern0pt}{\isacharplus}{\kern0pt}{\isadigit{1}}{\isacharparenright}{\kern0pt}{\isacharcomma}{\kern0pt}\ \isanewline
\ \ \ \ \ \ And{\isacharparenleft}{\kern0pt}Equal{\isacharparenleft}{\kern0pt}{\isadigit{3}}{\isacharcomma}{\kern0pt}\ {\isadigit{1}}{\isadigit{4}}{\isacharhash}{\kern0pt}{\isacharplus}{\kern0pt}{\isadigit{2}}{\isacharparenright}{\kern0pt}{\isacharcomma}{\kern0pt}\ \isanewline
\ \ \ \ \ \ And{\isacharparenleft}{\kern0pt}Equal{\isacharparenleft}{\kern0pt}{\isadigit{4}}{\isacharcomma}{\kern0pt}\ {\isadigit{1}}{\isadigit{4}}{\isacharhash}{\kern0pt}{\isacharplus}{\kern0pt}{\isadigit{3}}{\isacharparenright}{\kern0pt}{\isacharcomma}{\kern0pt}\ \isanewline
\ \ \ \ \ \ And{\isacharparenleft}{\kern0pt}Equal{\isacharparenleft}{\kern0pt}{\isadigit{5}}{\isacharcomma}{\kern0pt}\ {\isadigit{1}}{\isadigit{0}}{\isacharparenright}{\kern0pt}{\isacharcomma}{\kern0pt}\ \isanewline
\ \ \ \ \ \ \ \ \ \ iterates{\isacharparenleft}{\kern0pt}{\isasymlambda}p{\isachardot}{\kern0pt}\ incr{\isacharunderscore}{\kern0pt}bv{\isacharparenleft}{\kern0pt}p{\isacharparenright}{\kern0pt}{\isacharbackquote}{\kern0pt}{\isadigit{7}}{\isacharcomma}{\kern0pt}\ {\isadigit{1}}{\isadigit{1}}{\isacharcomma}{\kern0pt}\ {\isasymphi}{\isacharparenright}{\kern0pt}{\isacharparenright}{\kern0pt}{\isacharparenright}{\kern0pt}{\isacharparenright}{\kern0pt}{\isacharparenright}{\kern0pt}{\isacharparenright}{\kern0pt}{\isacharparenright}{\kern0pt}{\isacharparenright}{\kern0pt}{\isacharparenright}{\kern0pt}{\isacharparenright}{\kern0pt}{\isacharparenright}{\kern0pt}{\isacharparenright}{\kern0pt}{\isacharparenright}{\kern0pt}{\isachardoublequoteclose}\ \isanewline
\isanewline
\isacommand{lemma}\isamarkupfalse%
\ sats{\isacharunderscore}{\kern0pt}ren{\isacharunderscore}{\kern0pt}least{\isacharunderscore}{\kern0pt}index{\isacharunderscore}{\kern0pt}iff\ {\isacharcolon}{\kern0pt}\ \isanewline
\ \ \isakeyword{assumes}\ {\isachardoublequoteopen}{\isasymphi}\ {\isasymin}\ formula{\isachardoublequoteclose}\ {\isachardoublequoteopen}{\isacharbrackleft}{\kern0pt}a{\isadigit{0}}{\isacharcomma}{\kern0pt}\ a{\isadigit{1}}{\isacharcomma}{\kern0pt}\ a{\isadigit{2}}{\isacharcomma}{\kern0pt}\ a{\isadigit{3}}{\isacharcomma}{\kern0pt}\ a{\isadigit{4}}{\isacharcomma}{\kern0pt}\ a{\isadigit{5}}{\isacharcomma}{\kern0pt}\ a{\isadigit{6}}{\isacharcomma}{\kern0pt}\ a{\isadigit{7}}{\isacharcomma}{\kern0pt}\ a{\isadigit{8}}{\isacharcomma}{\kern0pt}\ a{\isadigit{9}}{\isacharcomma}{\kern0pt}\ a{\isadigit{1}}{\isadigit{0}}{\isacharcomma}{\kern0pt}\ a{\isadigit{1}}{\isadigit{1}}{\isacharbrackright}{\kern0pt}\ {\isacharat}{\kern0pt}\ env\ {\isasymin}\ list{\isacharparenleft}{\kern0pt}M{\isacharparenright}{\kern0pt}{\isachardoublequoteclose}\isanewline
\ \ \isakeyword{shows}\ {\isachardoublequoteopen}sats{\isacharparenleft}{\kern0pt}M{\isacharcomma}{\kern0pt}\ ren{\isacharunderscore}{\kern0pt}least{\isacharunderscore}{\kern0pt}index{\isacharparenleft}{\kern0pt}{\isasymphi}{\isacharparenright}{\kern0pt}{\isacharcomma}{\kern0pt}\ {\isacharbrackleft}{\kern0pt}a{\isadigit{0}}{\isacharcomma}{\kern0pt}\ a{\isadigit{1}}{\isacharcomma}{\kern0pt}\ a{\isadigit{2}}{\isacharcomma}{\kern0pt}\ a{\isadigit{3}}{\isacharcomma}{\kern0pt}\ a{\isadigit{4}}{\isacharcomma}{\kern0pt}\ a{\isadigit{5}}{\isacharcomma}{\kern0pt}\ a{\isadigit{6}}{\isacharcomma}{\kern0pt}\ a{\isadigit{7}}{\isacharcomma}{\kern0pt}\ a{\isadigit{8}}{\isacharcomma}{\kern0pt}\ a{\isadigit{9}}{\isacharcomma}{\kern0pt}\ a{\isadigit{1}}{\isadigit{0}}{\isacharcomma}{\kern0pt}\ a{\isadigit{1}}{\isadigit{1}}{\isacharbrackright}{\kern0pt}\ {\isacharat}{\kern0pt}\ env{\isacharparenright}{\kern0pt}\ {\isasymlongleftrightarrow}\ \isanewline
\ \ \ \ \ \ \ \ \ sats{\isacharparenleft}{\kern0pt}M{\isacharcomma}{\kern0pt}\ {\isasymphi}{\isacharcomma}{\kern0pt}\ {\isacharbrackleft}{\kern0pt}a{\isadigit{5}}{\isacharcomma}{\kern0pt}\ a{\isadigit{8}}{\isacharcomma}{\kern0pt}\ a{\isadigit{9}}{\isacharcomma}{\kern0pt}\ a{\isadigit{1}}{\isadigit{0}}{\isacharcomma}{\kern0pt}\ a{\isadigit{1}}{\isadigit{1}}{\isacharcomma}{\kern0pt}\ a{\isadigit{4}}{\isacharcomma}{\kern0pt}\ a{\isadigit{0}}{\isacharbrackright}{\kern0pt}\ {\isacharat}{\kern0pt}\ env{\isacharparenright}{\kern0pt}{\isachardoublequoteclose}\ \isanewline
%
\isadelimproof
\ \ %
\endisadelimproof
%
\isatagproof
\isacommand{unfolding}\isamarkupfalse%
\ ren{\isacharunderscore}{\kern0pt}least{\isacharunderscore}{\kern0pt}index{\isacharunderscore}{\kern0pt}def\isanewline
\ \ \isacommand{using}\isamarkupfalse%
\ assms\isanewline
\ \ \isacommand{apply}\isamarkupfalse%
\ simp\isanewline
\ \ \isacommand{apply}\isamarkupfalse%
\ {\isacharparenleft}{\kern0pt}insert\ sats{\isacharunderscore}{\kern0pt}incr{\isacharunderscore}{\kern0pt}bv{\isacharunderscore}{\kern0pt}iff\ {\isacharbrackleft}{\kern0pt}of\ {\isacharunderscore}{\kern0pt}\ {\isacharunderscore}{\kern0pt}\ M\ {\isacharunderscore}{\kern0pt}\ {\isachardoublequoteopen}{\isacharbrackleft}{\kern0pt}a{\isadigit{5}}{\isacharcomma}{\kern0pt}\ a{\isadigit{8}}{\isacharcomma}{\kern0pt}\ a{\isadigit{9}}{\isacharcomma}{\kern0pt}\ a{\isadigit{1}}{\isadigit{0}}{\isacharcomma}{\kern0pt}\ a{\isadigit{1}}{\isadigit{1}}{\isacharcomma}{\kern0pt}\ a{\isadigit{4}}{\isacharcomma}{\kern0pt}\ a{\isadigit{0}}{\isacharbrackright}{\kern0pt}{\isachardoublequoteclose}{\isacharbrackright}{\kern0pt}{\isacharparenright}{\kern0pt}\isanewline
\ \ \isacommand{by}\isamarkupfalse%
\ simp%
\endisatagproof
{\isafoldproof}%
%
\isadelimproof
\isanewline
%
\endisadelimproof
\isanewline
\isacommand{lemma}\isamarkupfalse%
\ ren{\isacharunderscore}{\kern0pt}least{\isacharunderscore}{\kern0pt}index{\isacharunderscore}{\kern0pt}type\ {\isacharcolon}{\kern0pt}\ \isanewline
\ \ {\isachardoublequoteopen}{\isasymphi}\ {\isasymin}\ formula\ {\isasymLongrightarrow}\ ren{\isacharunderscore}{\kern0pt}least{\isacharunderscore}{\kern0pt}index{\isacharparenleft}{\kern0pt}{\isasymphi}{\isacharparenright}{\kern0pt}\ {\isasymin}\ formula{\isachardoublequoteclose}\ \isanewline
%
\isadelimproof
\ \ %
\endisadelimproof
%
\isatagproof
\isacommand{unfolding}\isamarkupfalse%
\ ren{\isacharunderscore}{\kern0pt}least{\isacharunderscore}{\kern0pt}index{\isacharunderscore}{\kern0pt}def\ \isanewline
\isanewline
\ \ \isacommand{apply}\isamarkupfalse%
{\isacharparenleft}{\kern0pt}subgoal{\isacharunderscore}{\kern0pt}tac\ {\isachardoublequoteopen}\ {\isacharparenleft}{\kern0pt}{\isasymlambda}p{\isachardot}{\kern0pt}\ incr{\isacharunderscore}{\kern0pt}bv{\isacharparenleft}{\kern0pt}p{\isacharparenright}{\kern0pt}\ {\isacharbackquote}{\kern0pt}\ {\isadigit{7}}{\isacharparenright}{\kern0pt}{\isacharcircum}{\kern0pt}{\isadigit{1}}{\isadigit{1}}\ {\isacharparenleft}{\kern0pt}{\isasymphi}{\isacharparenright}{\kern0pt}\ {\isasymin}\ formula{\isachardoublequoteclose}{\isacharparenright}{\kern0pt}\isanewline
\ \ \ \isacommand{apply}\isamarkupfalse%
\ force\ \isanewline
\ \ \isacommand{apply}\isamarkupfalse%
{\isacharparenleft}{\kern0pt}rule\ iterates{\isacharunderscore}{\kern0pt}type{\isacharparenright}{\kern0pt}\isanewline
\ \ \isacommand{by}\isamarkupfalse%
\ auto%
\endisatagproof
{\isafoldproof}%
%
\isadelimproof
\isanewline
%
\endisadelimproof
\isanewline
\isacommand{lemma}\isamarkupfalse%
\ arity{\isacharunderscore}{\kern0pt}ren{\isacharunderscore}{\kern0pt}least{\isacharunderscore}{\kern0pt}index\ {\isacharcolon}{\kern0pt}\ \isanewline
\ \ \isakeyword{fixes}\ {\isasymphi}\isanewline
\ \ \isakeyword{assumes}\ {\isachardoublequoteopen}{\isasymphi}\ {\isasymin}\ formula{\isachardoublequoteclose}\ {\isachardoublequoteopen}{\isadigit{0}}\ {\isacharless}{\kern0pt}\ arity{\isacharparenleft}{\kern0pt}{\isasymphi}{\isacharparenright}{\kern0pt}{\isachardoublequoteclose}\ \isanewline
\ \ \isakeyword{shows}\ {\isachardoublequoteopen}arity{\isacharparenleft}{\kern0pt}ren{\isacharunderscore}{\kern0pt}least{\isacharunderscore}{\kern0pt}index{\isacharparenleft}{\kern0pt}{\isasymphi}{\isacharparenright}{\kern0pt}{\isacharparenright}{\kern0pt}\ {\isasymle}\ {\isadigit{1}}{\isadigit{2}}\ {\isasymunion}\ {\isacharparenleft}{\kern0pt}{\isadigit{5}}\ {\isacharhash}{\kern0pt}{\isacharplus}{\kern0pt}\ arity{\isacharparenleft}{\kern0pt}{\isasymphi}{\isacharparenright}{\kern0pt}{\isacharparenright}{\kern0pt}{\isachardoublequoteclose}\ \isanewline
%
\isadelimproof
\isanewline
\ \ %
\endisadelimproof
%
\isatagproof
\isacommand{unfolding}\isamarkupfalse%
\ ren{\isacharunderscore}{\kern0pt}least{\isacharunderscore}{\kern0pt}index{\isacharunderscore}{\kern0pt}def\ \isanewline
\ \ \isacommand{apply}\isamarkupfalse%
{\isacharparenleft}{\kern0pt}subgoal{\isacharunderscore}{\kern0pt}tac\ {\isachardoublequoteopen}{\isacharparenleft}{\kern0pt}{\isasymlambda}p{\isachardot}{\kern0pt}\ incr{\isacharunderscore}{\kern0pt}bv{\isacharparenleft}{\kern0pt}p{\isacharparenright}{\kern0pt}\ {\isacharbackquote}{\kern0pt}\ {\isadigit{7}}{\isacharparenright}{\kern0pt}{\isacharcircum}{\kern0pt}{\isadigit{1}}{\isadigit{1}}\ {\isacharparenleft}{\kern0pt}{\isasymphi}{\isacharparenright}{\kern0pt}\ {\isasymin}\ formula{\isachardoublequoteclose}{\isacharparenright}{\kern0pt}\ \isanewline
\ \ \isacommand{using}\isamarkupfalse%
\ assms\isanewline
\ \ \ \isacommand{apply}\isamarkupfalse%
\ simp\isanewline
\ \ \ \isacommand{apply}\isamarkupfalse%
{\isacharparenleft}{\kern0pt}rule\ pred{\isacharunderscore}{\kern0pt}le{\isacharcomma}{\kern0pt}\ simp{\isacharcomma}{\kern0pt}\ simp{\isacharparenright}{\kern0pt}{\isacharplus}{\kern0pt}\isanewline
\ \ \ \isacommand{apply}\isamarkupfalse%
{\isacharparenleft}{\kern0pt}subst\ succ{\isacharunderscore}{\kern0pt}Un{\isacharunderscore}{\kern0pt}distrib{\isacharcomma}{\kern0pt}\ simp{\isacharcomma}{\kern0pt}\ simp{\isacharparenright}{\kern0pt}{\isacharplus}{\kern0pt}\isanewline
\ \ \ \isacommand{apply}\isamarkupfalse%
{\isacharparenleft}{\kern0pt}rule\ Un{\isacharunderscore}{\kern0pt}least{\isacharunderscore}{\kern0pt}lt{\isacharcomma}{\kern0pt}\ rule\ Un{\isacharunderscore}{\kern0pt}least{\isacharunderscore}{\kern0pt}lt{\isacharcomma}{\kern0pt}\ rule\ union{\isacharunderscore}{\kern0pt}lt{\isadigit{1}}{\isacharcomma}{\kern0pt}\ simp{\isacharcomma}{\kern0pt}\ simp{\isacharcomma}{\kern0pt}\ simp{\isacharcomma}{\kern0pt}\ simp{\isacharcomma}{\kern0pt}\ rule\ union{\isacharunderscore}{\kern0pt}lt{\isadigit{1}}{\isacharcomma}{\kern0pt}\ simp{\isacharcomma}{\kern0pt}\ simp{\isacharcomma}{\kern0pt}\ simp{\isacharcomma}{\kern0pt}\ simp{\isacharparenright}{\kern0pt}{\isacharplus}{\kern0pt}\isanewline
\ \ \ \isacommand{apply}\isamarkupfalse%
{\isacharparenleft}{\kern0pt}rule{\isacharunderscore}{\kern0pt}tac\ b{\isacharequal}{\kern0pt}{\isachardoublequoteopen}arity{\isacharparenleft}{\kern0pt}{\isacharparenleft}{\kern0pt}{\isasymlambda}p{\isachardot}{\kern0pt}\ incr{\isacharunderscore}{\kern0pt}bv{\isacharparenleft}{\kern0pt}p{\isacharparenright}{\kern0pt}\ {\isacharbackquote}{\kern0pt}\ {\isadigit{7}}{\isacharparenright}{\kern0pt}{\isacharcircum}{\kern0pt}{\isadigit{1}}{\isadigit{1}}\ {\isacharparenleft}{\kern0pt}{\isasymphi}{\isacharparenright}{\kern0pt}{\isacharparenright}{\kern0pt}{\isachardoublequoteclose}\ \isakeyword{in}\ le{\isacharunderscore}{\kern0pt}lt{\isacharunderscore}{\kern0pt}lt{\isacharcomma}{\kern0pt}\ simp{\isacharparenright}{\kern0pt}\isanewline
\ \ \ \isacommand{apply}\isamarkupfalse%
{\isacharparenleft}{\kern0pt}rule{\isacharunderscore}{\kern0pt}tac\ le{\isacharunderscore}{\kern0pt}lt{\isacharunderscore}{\kern0pt}lt{\isacharparenright}{\kern0pt}\isanewline
\ \ \ \ \isacommand{apply}\isamarkupfalse%
{\isacharparenleft}{\kern0pt}rule\ arity{\isacharunderscore}{\kern0pt}incr{\isacharunderscore}{\kern0pt}bv{\isacharunderscore}{\kern0pt}le{\isacharparenright}{\kern0pt}\isanewline
\ \ \ \ \ \ \isacommand{apply}\isamarkupfalse%
\ auto{\isacharbrackleft}{\kern0pt}{\isadigit{3}}{\isacharbrackright}{\kern0pt}\isanewline
\ \ \isacommand{apply}\isamarkupfalse%
{\isacharparenleft}{\kern0pt}rule\ union{\isacharunderscore}{\kern0pt}lt{\isadigit{2}}{\isacharcomma}{\kern0pt}\ simp{\isacharcomma}{\kern0pt}\ simp{\isacharcomma}{\kern0pt}\ simp{\isacharcomma}{\kern0pt}\ simp{\isacharparenright}{\kern0pt}\isanewline
\ \ \isacommand{apply}\isamarkupfalse%
{\isacharparenleft}{\kern0pt}rule\ iterates{\isacharunderscore}{\kern0pt}type{\isacharparenright}{\kern0pt}\isanewline
\ \ \isacommand{using}\isamarkupfalse%
\ assms\isanewline
\ \ \isacommand{by}\isamarkupfalse%
\ auto%
\endisatagproof
{\isafoldproof}%
%
\isadelimproof
\isanewline
%
\endisadelimproof
\isanewline
\isacommand{definition}\isamarkupfalse%
\ is{\isacharunderscore}{\kern0pt}least{\isacharunderscore}{\kern0pt}index{\isacharunderscore}{\kern0pt}of{\isacharunderscore}{\kern0pt}Vset{\isacharunderscore}{\kern0pt}contains{\isacharunderscore}{\kern0pt}witness{\isacharunderscore}{\kern0pt}fm\ \isakeyword{where}\isanewline
\ \ {\isachardoublequoteopen}is{\isacharunderscore}{\kern0pt}least{\isacharunderscore}{\kern0pt}index{\isacharunderscore}{\kern0pt}of{\isacharunderscore}{\kern0pt}Vset{\isacharunderscore}{\kern0pt}contains{\isacharunderscore}{\kern0pt}witness{\isacharunderscore}{\kern0pt}fm{\isacharparenleft}{\kern0pt}p{\isacharparenright}{\kern0pt}\ {\isasymequiv}\ \isanewline
\ \ \ \ Exists{\isacharparenleft}{\kern0pt}Exists{\isacharparenleft}{\kern0pt}Exists{\isacharparenleft}{\kern0pt}\isanewline
\ \ \ \ \ \ And{\isacharparenleft}{\kern0pt}fst{\isacharunderscore}{\kern0pt}fm{\isacharparenleft}{\kern0pt}{\isadigit{3}}{\isacharcomma}{\kern0pt}\ {\isadigit{1}}{\isacharparenright}{\kern0pt}{\isacharcomma}{\kern0pt}\ \isanewline
\ \ \ \ \ \ And{\isacharparenleft}{\kern0pt}snd{\isacharunderscore}{\kern0pt}fm{\isacharparenleft}{\kern0pt}{\isadigit{3}}{\isacharcomma}{\kern0pt}\ {\isadigit{2}}{\isacharparenright}{\kern0pt}{\isacharcomma}{\kern0pt}\ \isanewline
\ \ \ \ \ \ And{\isacharparenleft}{\kern0pt}least{\isacharunderscore}{\kern0pt}fm{\isacharparenleft}{\kern0pt}\isanewline
\ \ \ \ \ \ \ \ Exists{\isacharparenleft}{\kern0pt}Exists{\isacharparenleft}{\kern0pt}\isanewline
\ \ \ \ \ \ \ \ \ \ And{\isacharparenleft}{\kern0pt}is{\isacharunderscore}{\kern0pt}MVset{\isacharunderscore}{\kern0pt}fm{\isacharparenleft}{\kern0pt}{\isadigit{2}}{\isacharcomma}{\kern0pt}\ {\isadigit{1}}{\isacharparenright}{\kern0pt}{\isacharcomma}{\kern0pt}\isanewline
\ \ \ \ \ \ \ \ \ \ And{\isacharparenleft}{\kern0pt}Member{\isacharparenleft}{\kern0pt}{\isadigit{0}}{\isacharcomma}{\kern0pt}\ {\isadigit{1}}{\isacharparenright}{\kern0pt}{\isacharcomma}{\kern0pt}\ \isanewline
\ \ \ \ \ \ \ \ \ \ And{\isacharparenleft}{\kern0pt}is{\isacharunderscore}{\kern0pt}HS{\isacharunderscore}{\kern0pt}fm{\isacharparenleft}{\kern0pt}{\isadigit{1}}{\isadigit{1}}{\isacharcomma}{\kern0pt}\ {\isadigit{0}}{\isacharparenright}{\kern0pt}{\isacharcomma}{\kern0pt}\ ren{\isacharunderscore}{\kern0pt}least{\isacharunderscore}{\kern0pt}index{\isacharparenleft}{\kern0pt}forcesHS{\isacharparenleft}{\kern0pt}p{\isacharparenright}{\kern0pt}{\isacharparenright}{\kern0pt}{\isacharparenright}{\kern0pt}{\isacharparenright}{\kern0pt}{\isacharparenright}{\kern0pt}{\isacharparenright}{\kern0pt}{\isacharparenright}{\kern0pt}{\isacharcomma}{\kern0pt}\ \isanewline
\ \ \ \ \ \ {\isadigit{0}}{\isacharparenright}{\kern0pt}{\isacharcomma}{\kern0pt}\isanewline
\ \ \ \ \ \ pair{\isacharunderscore}{\kern0pt}fm{\isacharparenleft}{\kern0pt}{\isadigit{3}}{\isacharcomma}{\kern0pt}\ {\isadigit{0}}{\isacharcomma}{\kern0pt}\ {\isadigit{4}}{\isacharparenright}{\kern0pt}\ \isanewline
\ \ {\isacharparenright}{\kern0pt}{\isacharparenright}{\kern0pt}{\isacharparenright}{\kern0pt}{\isacharparenright}{\kern0pt}{\isacharparenright}{\kern0pt}{\isacharparenright}{\kern0pt}{\isachardoublequoteclose}\isanewline
\isanewline
\isanewline
\isanewline
\isanewline
\isacommand{lemma}\isamarkupfalse%
\ sats{\isacharunderscore}{\kern0pt}is{\isacharunderscore}{\kern0pt}least{\isacharunderscore}{\kern0pt}index{\isacharunderscore}{\kern0pt}of{\isacharunderscore}{\kern0pt}Vset{\isacharunderscore}{\kern0pt}contains{\isacharunderscore}{\kern0pt}witness{\isacharunderscore}{\kern0pt}fm{\isacharunderscore}{\kern0pt}iff\ {\isacharcolon}{\kern0pt}\ \isanewline
\ \ \isakeyword{fixes}\ {\isasymphi}\ u\ v\ env\isanewline
\ \ \isakeyword{assumes}\ {\isachardoublequoteopen}{\isasymphi}\ {\isasymin}\ formula{\isachardoublequoteclose}\ {\isachardoublequoteopen}u\ {\isasymin}\ M{\isachardoublequoteclose}\ {\isachardoublequoteopen}v\ {\isasymin}\ M{\isachardoublequoteclose}\ {\isachardoublequoteopen}env\ {\isasymin}\ list{\isacharparenleft}{\kern0pt}M{\isacharparenright}{\kern0pt}{\isachardoublequoteclose}\ \isanewline
\ \ \isakeyword{shows}\ {\isachardoublequoteopen}sats{\isacharparenleft}{\kern0pt}M{\isacharcomma}{\kern0pt}\ is{\isacharunderscore}{\kern0pt}least{\isacharunderscore}{\kern0pt}index{\isacharunderscore}{\kern0pt}of{\isacharunderscore}{\kern0pt}Vset{\isacharunderscore}{\kern0pt}contains{\isacharunderscore}{\kern0pt}witness{\isacharunderscore}{\kern0pt}fm{\isacharparenleft}{\kern0pt}{\isasymphi}{\isacharparenright}{\kern0pt}{\isacharcomma}{\kern0pt}\ {\isacharbrackleft}{\kern0pt}u{\isacharcomma}{\kern0pt}\ v{\isacharcomma}{\kern0pt}\ P{\isacharcomma}{\kern0pt}\ leq{\isacharcomma}{\kern0pt}\ one{\isacharcomma}{\kern0pt}\ {\isacharless}{\kern0pt}{\isasymF}{\isacharcomma}{\kern0pt}\ {\isasymG}{\isacharcomma}{\kern0pt}\ P{\isacharcomma}{\kern0pt}\ P{\isacharunderscore}{\kern0pt}auto{\isachargreater}{\kern0pt}{\isacharbrackright}{\kern0pt}\ {\isacharat}{\kern0pt}\ env{\isacharparenright}{\kern0pt}\ \isanewline
\ \ \ \ \ \ \ \ {\isasymlongleftrightarrow}\ v\ {\isacharequal}{\kern0pt}\ {\isacharless}{\kern0pt}u{\isacharcomma}{\kern0pt}\ {\isasymmu}\ a{\isachardot}{\kern0pt}\ a\ {\isasymin}\ M\ {\isasymand}\ Ord{\isacharparenleft}{\kern0pt}a{\isacharparenright}{\kern0pt}\ {\isasymand}\ {\isacharparenleft}{\kern0pt}{\isasymexists}z\ {\isasymin}\ MVset{\isacharparenleft}{\kern0pt}a{\isacharparenright}{\kern0pt}\ {\isasyminter}\ HS{\isachardot}{\kern0pt}\ snd{\isacharparenleft}{\kern0pt}u{\isacharparenright}{\kern0pt}\ {\isasymtturnstile}HS\ {\isasymphi}\ {\isacharparenleft}{\kern0pt}{\isacharbrackleft}{\kern0pt}fst{\isacharparenleft}{\kern0pt}u{\isacharparenright}{\kern0pt}{\isacharcomma}{\kern0pt}\ z{\isacharbrackright}{\kern0pt}\ {\isacharat}{\kern0pt}\ env{\isacharparenright}{\kern0pt}{\isacharparenright}{\kern0pt}{\isachargreater}{\kern0pt}{\isachardoublequoteclose}\ \ \isanewline
%
\isadelimproof
\isanewline
\ \ %
\endisadelimproof
%
\isatagproof
\isacommand{apply}\isamarkupfalse%
{\isacharparenleft}{\kern0pt}subgoal{\isacharunderscore}{\kern0pt}tac\ {\isachardoublequoteopen}P\ {\isasymin}\ M\ {\isasymand}\ leq\ {\isasymin}\ M\ {\isasymand}\ one\ {\isasymin}\ M\ {\isasymand}\ {\isacharless}{\kern0pt}{\isasymF}{\isacharcomma}{\kern0pt}\ {\isasymG}{\isacharcomma}{\kern0pt}\ P{\isacharcomma}{\kern0pt}\ P{\isacharunderscore}{\kern0pt}auto{\isachargreater}{\kern0pt}\ {\isasymin}\ M{\isachardoublequoteclose}{\isacharparenright}{\kern0pt}\isanewline
\ \ \isacommand{unfolding}\isamarkupfalse%
\ is{\isacharunderscore}{\kern0pt}least{\isacharunderscore}{\kern0pt}index{\isacharunderscore}{\kern0pt}of{\isacharunderscore}{\kern0pt}Vset{\isacharunderscore}{\kern0pt}contains{\isacharunderscore}{\kern0pt}witness{\isacharunderscore}{\kern0pt}fm{\isacharunderscore}{\kern0pt}def\isanewline
\ \ \isacommand{apply}\isamarkupfalse%
{\isacharparenleft}{\kern0pt}rule{\isacharunderscore}{\kern0pt}tac\ Q{\isacharequal}{\kern0pt}\isanewline
\ \ \ \ \ \ {\isachardoublequoteopen}{\isasymexists}p\ {\isasymin}\ M{\isachardot}{\kern0pt}\ \ {\isasymexists}y\ {\isasymin}\ M{\isachardot}{\kern0pt}\ {\isasymexists}a\ {\isasymin}\ M{\isachardot}{\kern0pt}\isanewline
\ \ \ \ \ \ \ \ \ \ y\ {\isacharequal}{\kern0pt}\ fst{\isacharparenleft}{\kern0pt}u{\isacharparenright}{\kern0pt}\ {\isasymand}\ p\ {\isacharequal}{\kern0pt}\ snd{\isacharparenleft}{\kern0pt}u{\isacharparenright}{\kern0pt}\ {\isasymand}\ \isanewline
\ \ \ \ \ \ \ \ \ \ a\ {\isacharequal}{\kern0pt}\ {\isacharparenleft}{\kern0pt}{\isasymmu}\ a\ {\isachardot}{\kern0pt}\ {\isasymexists}z\ {\isasymin}\ M{\isachardot}{\kern0pt}\ {\isasymexists}V\ {\isasymin}\ M{\isachardot}{\kern0pt}\ a\ {\isasymin}\ M\ {\isasymand}\ Ord{\isacharparenleft}{\kern0pt}a{\isacharparenright}{\kern0pt}\ {\isasymand}\ V\ {\isacharequal}{\kern0pt}\ MVset{\isacharparenleft}{\kern0pt}a{\isacharparenright}{\kern0pt}\ {\isasymand}\ z\ {\isasymin}\ V\ {\isasymand}\ z\ {\isasymin}\ HS\ {\isasymand}\ p\ {\isasymtturnstile}HS\ {\isasymphi}\ {\isacharbrackleft}{\kern0pt}y{\isacharcomma}{\kern0pt}\ z{\isacharbrackright}{\kern0pt}\ {\isacharat}{\kern0pt}\ env{\isacharparenright}{\kern0pt}\ {\isasymand}\ \isanewline
\ \ \ \ \ \ \ \ \ \ v\ {\isacharequal}{\kern0pt}\ {\isacharless}{\kern0pt}u{\isacharcomma}{\kern0pt}\ a{\isachargreater}{\kern0pt}{\isachardoublequoteclose}\ \isakeyword{in}\ iff{\isacharunderscore}{\kern0pt}trans{\isacharparenright}{\kern0pt}\ \isanewline
\ \ \ \ \isacommand{apply}\isamarkupfalse%
{\isacharparenleft}{\kern0pt}rule\ iff{\isacharunderscore}{\kern0pt}trans{\isacharcomma}{\kern0pt}\ rule\ sats{\isacharunderscore}{\kern0pt}Exists{\isacharunderscore}{\kern0pt}iff{\isacharcomma}{\kern0pt}\ simp\ add{\isacharcolon}{\kern0pt}assms{\isacharcomma}{\kern0pt}\ rule\ bex{\isacharunderscore}{\kern0pt}iff{\isacharparenright}{\kern0pt}{\isacharplus}{\kern0pt}\isanewline
\ \ \ \ \isacommand{apply}\isamarkupfalse%
{\isacharparenleft}{\kern0pt}rule\ iff{\isacharunderscore}{\kern0pt}trans{\isacharcomma}{\kern0pt}\ rule\ sats{\isacharunderscore}{\kern0pt}And{\isacharunderscore}{\kern0pt}iff{\isacharcomma}{\kern0pt}\ simp\ add{\isacharcolon}{\kern0pt}assms{\isacharcomma}{\kern0pt}\ rule\ iff{\isacharunderscore}{\kern0pt}conjI{\isadigit{2}}{\isacharparenright}{\kern0pt}\isanewline
\ \ \ \ \ \isacommand{apply}\isamarkupfalse%
{\isacharparenleft}{\kern0pt}rule\ iff{\isacharunderscore}{\kern0pt}trans{\isacharcomma}{\kern0pt}\ rule\ sats{\isacharunderscore}{\kern0pt}fst{\isacharunderscore}{\kern0pt}fm{\isacharcomma}{\kern0pt}\ simp{\isacharcomma}{\kern0pt}\ simp{\isacharcomma}{\kern0pt}\ simp\ add{\isacharcolon}{\kern0pt}assms{\isacharparenright}{\kern0pt}\isanewline
\ \ \isacommand{using}\isamarkupfalse%
\ assms\isanewline
\ \ \ \ \ \isacommand{apply}\isamarkupfalse%
\ simp\isanewline
\ \ \ \ \isacommand{apply}\isamarkupfalse%
{\isacharparenleft}{\kern0pt}rule\ iff{\isacharunderscore}{\kern0pt}trans{\isacharcomma}{\kern0pt}\ rule\ sats{\isacharunderscore}{\kern0pt}And{\isacharunderscore}{\kern0pt}iff{\isacharcomma}{\kern0pt}\ simp\ add{\isacharcolon}{\kern0pt}assms{\isacharcomma}{\kern0pt}\ rule\ iff{\isacharunderscore}{\kern0pt}conjI{\isadigit{2}}{\isacharparenright}{\kern0pt}\isanewline
\ \ \ \ \ \isacommand{apply}\isamarkupfalse%
{\isacharparenleft}{\kern0pt}rule\ iff{\isacharunderscore}{\kern0pt}trans{\isacharcomma}{\kern0pt}\ rule\ sats{\isacharunderscore}{\kern0pt}snd{\isacharunderscore}{\kern0pt}fm{\isacharcomma}{\kern0pt}\ simp{\isacharcomma}{\kern0pt}\ simp{\isacharcomma}{\kern0pt}\ simp\ add{\isacharcolon}{\kern0pt}assms{\isacharparenright}{\kern0pt}\isanewline
\ \ \isacommand{using}\isamarkupfalse%
\ assms\ \ \isanewline
\ \ \ \ \ \isacommand{apply}\isamarkupfalse%
\ simp\isanewline
\ \ \ \ \isacommand{apply}\isamarkupfalse%
{\isacharparenleft}{\kern0pt}rule\ iff{\isacharunderscore}{\kern0pt}trans{\isacharcomma}{\kern0pt}\ rule\ sats{\isacharunderscore}{\kern0pt}And{\isacharunderscore}{\kern0pt}iff{\isacharcomma}{\kern0pt}\ simp\ add{\isacharcolon}{\kern0pt}assms{\isacharcomma}{\kern0pt}\ rule\ iff{\isacharunderscore}{\kern0pt}conjI{\isadigit{2}}{\isacharparenright}{\kern0pt}\isanewline
\ \ \ \ \ \isacommand{apply}\isamarkupfalse%
{\isacharparenleft}{\kern0pt}rule\ iff{\isacharunderscore}{\kern0pt}trans{\isacharcomma}{\kern0pt}\ rule\ sats{\isacharunderscore}{\kern0pt}least{\isacharunderscore}{\kern0pt}fm{\isacharcomma}{\kern0pt}\ simp{\isacharcomma}{\kern0pt}\ simp{\isacharcomma}{\kern0pt}\ simp\ add{\isacharcolon}{\kern0pt}assms{\isacharcomma}{\kern0pt}\ simp\ add{\isacharcolon}{\kern0pt}zero{\isacharunderscore}{\kern0pt}in{\isacharunderscore}{\kern0pt}M{\isacharparenright}{\kern0pt}\ \isanewline
\ \ \ \ \ \isacommand{apply}\isamarkupfalse%
{\isacharparenleft}{\kern0pt}rule\ iff{\isacharunderscore}{\kern0pt}flip{\isacharcomma}{\kern0pt}\ rule\ iff{\isacharunderscore}{\kern0pt}trans{\isacharcomma}{\kern0pt}\ rule\ iff{\isacharunderscore}{\kern0pt}flip{\isacharcomma}{\kern0pt}\ rule\ least{\isacharunderscore}{\kern0pt}abs{\isacharparenright}{\kern0pt}\isanewline
\ \ \ \ \ \ \ \isacommand{apply}\isamarkupfalse%
\ auto{\isacharbrackleft}{\kern0pt}{\isadigit{2}}{\isacharbrackright}{\kern0pt}\isanewline
\ \ \ \ \ \isacommand{apply}\isamarkupfalse%
\ simp\ \isanewline
\ \ \ \ \ \isacommand{apply}\isamarkupfalse%
{\isacharparenleft}{\kern0pt}rule\ least{\isacharunderscore}{\kern0pt}cong{\isacharcomma}{\kern0pt}\ simp{\isacharcomma}{\kern0pt}\ rule\ iff{\isacharunderscore}{\kern0pt}flip{\isacharparenright}{\kern0pt}\isanewline
\ \ \ \ \ \isacommand{apply}\isamarkupfalse%
{\isacharparenleft}{\kern0pt}rule\ iff{\isacharunderscore}{\kern0pt}trans{\isacharcomma}{\kern0pt}\ rule\ sats{\isacharunderscore}{\kern0pt}Exists{\isacharunderscore}{\kern0pt}iff{\isacharcomma}{\kern0pt}\ simp\ add{\isacharcolon}{\kern0pt}assms{\isacharcomma}{\kern0pt}\ rule\ iff{\isacharunderscore}{\kern0pt}trans{\isacharcomma}{\kern0pt}\ rule\ bex{\isacharunderscore}{\kern0pt}iff{\isacharparenright}{\kern0pt}{\isacharplus}{\kern0pt}\isanewline
\ \ \ \ \ \ \ \isacommand{apply}\isamarkupfalse%
{\isacharparenleft}{\kern0pt}rule\ sats{\isacharunderscore}{\kern0pt}And{\isacharunderscore}{\kern0pt}iff{\isacharcomma}{\kern0pt}\ simp\ add{\isacharcolon}{\kern0pt}assms{\isacharcomma}{\kern0pt}\ rule\ bex{\isacharunderscore}{\kern0pt}iff{\isacharcomma}{\kern0pt}\ rule\ iff{\isacharunderscore}{\kern0pt}conjI{\isadigit{2}}{\isacharcomma}{\kern0pt}\ rule\ sats{\isacharunderscore}{\kern0pt}is{\isacharunderscore}{\kern0pt}MVset{\isacharunderscore}{\kern0pt}fm{\isacharunderscore}{\kern0pt}iff{\isacharcomma}{\kern0pt}\ simp\ add{\isacharcolon}{\kern0pt}assms{\isacharparenright}{\kern0pt}\isanewline
\ \ \isacommand{using}\isamarkupfalse%
\ assms\isanewline
\ \ \ \ \ \ \ \ \ \ \isacommand{apply}\isamarkupfalse%
\ auto{\isacharbrackleft}{\kern0pt}{\isadigit{5}}{\isacharbrackright}{\kern0pt}\isanewline
\ \ \ \ \ \isacommand{apply}\isamarkupfalse%
{\isacharparenleft}{\kern0pt}rule\ iff{\isacharunderscore}{\kern0pt}trans{\isacharcomma}{\kern0pt}\ rule\ bex{\isacharunderscore}{\kern0pt}iff{\isacharcomma}{\kern0pt}\ rule\ bex{\isacharunderscore}{\kern0pt}iff{\isacharcomma}{\kern0pt}\ rule\ iff{\isacharunderscore}{\kern0pt}conjI{\isadigit{2}}{\isacharcomma}{\kern0pt}\ simp{\isacharcomma}{\kern0pt}\ rule\ iff{\isacharunderscore}{\kern0pt}conjI{\isadigit{2}}{\isacharcomma}{\kern0pt}\ simp{\isacharcomma}{\kern0pt}\ rule\ iff{\isacharunderscore}{\kern0pt}conjI{\isadigit{2}}{\isacharparenright}{\kern0pt}\isanewline
\ \ \ \ \ \ \ \isacommand{apply}\isamarkupfalse%
{\isacharparenleft}{\kern0pt}rule\ sats{\isacharunderscore}{\kern0pt}is{\isacharunderscore}{\kern0pt}HS{\isacharunderscore}{\kern0pt}fm{\isacharunderscore}{\kern0pt}iff{\isacharcomma}{\kern0pt}\ simp\ add{\isacharcolon}{\kern0pt}assms{\isacharparenright}{\kern0pt}\isanewline
\ \ \isacommand{using}\isamarkupfalse%
\ assms\isanewline
\ \ \ \ \ \ \ \ \ \ \isacommand{apply}\isamarkupfalse%
\ auto{\isacharbrackleft}{\kern0pt}{\isadigit{5}}{\isacharbrackright}{\kern0pt}\isanewline
\ \ \isacommand{using}\isamarkupfalse%
\ MVset{\isacharunderscore}{\kern0pt}in{\isacharunderscore}{\kern0pt}M\isanewline
\ \ \ \ \ \isacommand{apply}\isamarkupfalse%
\ simp\isanewline
\ \ \ \ \ \isacommand{apply}\isamarkupfalse%
{\isacharparenleft}{\kern0pt}rule\ bex{\isacharunderscore}{\kern0pt}iff{\isacharcomma}{\kern0pt}\ rule\ iff{\isacharunderscore}{\kern0pt}conjI{\isadigit{2}}{\isacharcomma}{\kern0pt}\ simp{\isacharcomma}{\kern0pt}\ rule\ iff{\isacharunderscore}{\kern0pt}conjI{\isadigit{2}}{\isacharcomma}{\kern0pt}\ simp{\isacharparenright}{\kern0pt}\isanewline
\ \ \isacommand{apply}\isamarkupfalse%
{\isacharparenleft}{\kern0pt}rename{\isacharunderscore}{\kern0pt}tac\ p\ y\ l\ a\ z{\isacharcomma}{\kern0pt}\ rule{\isacharunderscore}{\kern0pt}tac\ b{\isacharequal}{\kern0pt}{\isachardoublequoteopen}Cons{\isacharparenleft}{\kern0pt}z{\isacharcomma}{\kern0pt}\ Cons{\isacharparenleft}{\kern0pt}MVset{\isacharparenleft}{\kern0pt}a{\isacharparenright}{\kern0pt}{\isacharcomma}{\kern0pt}\ Cons{\isacharparenleft}{\kern0pt}a{\isacharcomma}{\kern0pt}\ Cons{\isacharparenleft}{\kern0pt}l{\isacharcomma}{\kern0pt}\ Cons{\isacharparenleft}{\kern0pt}fst{\isacharparenleft}{\kern0pt}u{\isacharparenright}{\kern0pt}{\isacharcomma}{\kern0pt}\ Cons{\isacharparenleft}{\kern0pt}snd{\isacharparenleft}{\kern0pt}u{\isacharparenright}{\kern0pt}{\isacharcomma}{\kern0pt}\ Cons{\isacharparenleft}{\kern0pt}u{\isacharcomma}{\kern0pt}\ Cons{\isacharparenleft}{\kern0pt}v{\isacharcomma}{\kern0pt}\ Cons{\isacharparenleft}{\kern0pt}P{\isacharcomma}{\kern0pt}\ Cons{\isacharparenleft}{\kern0pt}leq{\isacharcomma}{\kern0pt}\ Cons{\isacharparenleft}{\kern0pt}one{\isacharcomma}{\kern0pt}\ Cons{\isacharparenleft}{\kern0pt}{\isasymlangle}{\isasymF}{\isacharcomma}{\kern0pt}\ {\isasymG}{\isacharcomma}{\kern0pt}\ P{\isacharcomma}{\kern0pt}\ P{\isacharunderscore}{\kern0pt}auto{\isasymrangle}{\isacharcomma}{\kern0pt}\ env{\isacharparenright}{\kern0pt}{\isacharparenright}{\kern0pt}{\isacharparenright}{\kern0pt}{\isacharparenright}{\kern0pt}{\isacharparenright}{\kern0pt}{\isacharparenright}{\kern0pt}{\isacharparenright}{\kern0pt}{\isacharparenright}{\kern0pt}{\isacharparenright}{\kern0pt}{\isacharparenright}{\kern0pt}{\isacharparenright}{\kern0pt}{\isacharparenright}{\kern0pt}{\isachardoublequoteclose}\ \isanewline
\ \ \ \ \ \ \ \ \ \ \ \ \ \ \ \ \ \ \ \ \ \ \ \ \ \ \ \ \ \ \ \ \ \ \ \isakeyword{and}\ a{\isacharequal}{\kern0pt}{\isachardoublequoteopen}{\isacharbrackleft}{\kern0pt}z{\isacharcomma}{\kern0pt}\ MVset{\isacharparenleft}{\kern0pt}a{\isacharparenright}{\kern0pt}{\isacharcomma}{\kern0pt}\ a{\isacharcomma}{\kern0pt}\ l{\isacharcomma}{\kern0pt}\ fst{\isacharparenleft}{\kern0pt}u{\isacharparenright}{\kern0pt}{\isacharcomma}{\kern0pt}\ snd{\isacharparenleft}{\kern0pt}u{\isacharparenright}{\kern0pt}{\isacharcomma}{\kern0pt}\ u{\isacharcomma}{\kern0pt}\ v{\isacharcomma}{\kern0pt}\ P{\isacharcomma}{\kern0pt}\ leq{\isacharcomma}{\kern0pt}\ one{\isacharcomma}{\kern0pt}\ {\isasymlangle}{\isasymF}{\isacharcomma}{\kern0pt}\ {\isasymG}{\isacharcomma}{\kern0pt}\ P{\isacharcomma}{\kern0pt}\ P{\isacharunderscore}{\kern0pt}auto{\isasymrangle}{\isacharbrackright}{\kern0pt}\ {\isacharat}{\kern0pt}\ env{\isachardoublequoteclose}\ \isakeyword{in}\ ssubst{\isacharcomma}{\kern0pt}\ simp{\isacharparenright}{\kern0pt}\isanewline
\ \ \ \ \ \isacommand{apply}\isamarkupfalse%
{\isacharparenleft}{\kern0pt}rule\ iff{\isacharunderscore}{\kern0pt}trans{\isacharcomma}{\kern0pt}\ rule\ sats{\isacharunderscore}{\kern0pt}ren{\isacharunderscore}{\kern0pt}least{\isacharunderscore}{\kern0pt}index{\isacharunderscore}{\kern0pt}iff{\isacharparenright}{\kern0pt}\isanewline
\ \ \ \ \ \ \ \isacommand{apply}\isamarkupfalse%
{\isacharparenleft}{\kern0pt}rule\ forcesHS{\isacharunderscore}{\kern0pt}type{\isacharparenright}{\kern0pt}\isanewline
\ \ \isacommand{using}\isamarkupfalse%
\ assms\ fst{\isacharunderscore}{\kern0pt}snd{\isacharunderscore}{\kern0pt}closed\ \isanewline
\ \ \ \ \ \ \ \isacommand{apply}\isamarkupfalse%
\ auto{\isacharbrackleft}{\kern0pt}{\isadigit{4}}{\isacharbrackright}{\kern0pt}\isanewline
\ \ \isacommand{using}\isamarkupfalse%
\ pair{\isacharunderscore}{\kern0pt}in{\isacharunderscore}{\kern0pt}M{\isacharunderscore}{\kern0pt}iff\ fst{\isacharunderscore}{\kern0pt}snd{\isacharunderscore}{\kern0pt}closed\ assms\isanewline
\ \ \ \isacommand{apply}\isamarkupfalse%
{\isacharparenleft}{\kern0pt}rule{\isacharunderscore}{\kern0pt}tac\ Q{\isacharequal}{\kern0pt}{\isachardoublequoteopen}{\isacharparenleft}{\kern0pt}{\isasymexists}p{\isasymin}M{\isachardot}{\kern0pt}\ {\isasymexists}y{\isasymin}M{\isachardot}{\kern0pt}\ {\isasymexists}a{\isasymin}M{\isachardot}{\kern0pt}\ y\ {\isacharequal}{\kern0pt}\ fst{\isacharparenleft}{\kern0pt}u{\isacharparenright}{\kern0pt}\ {\isasymand}\ p\ {\isacharequal}{\kern0pt}\ snd{\isacharparenleft}{\kern0pt}u{\isacharparenright}{\kern0pt}\ {\isasymand}\ a\ {\isacharequal}{\kern0pt}\ {\isacharparenleft}{\kern0pt}{\isasymmu}\ a{\isachardot}{\kern0pt}\ a\ {\isasymin}\ M\ {\isasymand}\ Ord{\isacharparenleft}{\kern0pt}a{\isacharparenright}{\kern0pt}\ {\isasymand}\ {\isacharparenleft}{\kern0pt}{\isasymexists}z\ {\isasymin}\ MVset{\isacharparenleft}{\kern0pt}a{\isacharparenright}{\kern0pt}\ {\isasyminter}\ HS{\isachardot}{\kern0pt}\ M{\isacharcomma}{\kern0pt}\ {\isacharbrackleft}{\kern0pt}p{\isacharcomma}{\kern0pt}\ P{\isacharcomma}{\kern0pt}\ leq{\isacharcomma}{\kern0pt}\ one{\isacharcomma}{\kern0pt}\ {\isasymlangle}{\isasymF}{\isacharcomma}{\kern0pt}\ {\isasymG}{\isacharcomma}{\kern0pt}\ P{\isacharcomma}{\kern0pt}\ P{\isacharunderscore}{\kern0pt}auto{\isasymrangle}{\isacharbrackright}{\kern0pt}\ {\isacharat}{\kern0pt}\ {\isacharbrackleft}{\kern0pt}y{\isacharcomma}{\kern0pt}\ z{\isacharbrackright}{\kern0pt}\ {\isacharat}{\kern0pt}\ env\ {\isasymTurnstile}\ forcesHS{\isacharparenleft}{\kern0pt}{\isasymphi}{\isacharparenright}{\kern0pt}{\isacharparenright}{\kern0pt}{\isacharparenright}{\kern0pt}\ {\isasymand}\ v\ {\isacharequal}{\kern0pt}\ {\isasymlangle}u{\isacharcomma}{\kern0pt}\ a{\isasymrangle}{\isacharparenright}{\kern0pt}{\isachardoublequoteclose}\ \isakeyword{in}\ iff{\isacharunderscore}{\kern0pt}trans{\isacharparenright}{\kern0pt}\isanewline
\ \ \ \ \isacommand{apply}\isamarkupfalse%
{\isacharparenleft}{\kern0pt}rule\ bex{\isacharunderscore}{\kern0pt}iff{\isacharparenright}{\kern0pt}{\isacharplus}{\kern0pt}\isanewline
\ \ \ \ \isacommand{apply}\isamarkupfalse%
{\isacharparenleft}{\kern0pt}rule\ iff{\isacharunderscore}{\kern0pt}conjI{\isadigit{2}}{\isacharcomma}{\kern0pt}\ simp{\isacharparenright}{\kern0pt}{\isacharplus}{\kern0pt}\isanewline
\ \ \ \ \ \isacommand{apply}\isamarkupfalse%
{\isacharparenleft}{\kern0pt}rule\ eq{\isacharunderscore}{\kern0pt}cong{\isacharcomma}{\kern0pt}\ simp{\isacharparenright}{\kern0pt}\isanewline
\ \ \ \ \ \isacommand{apply}\isamarkupfalse%
{\isacharparenleft}{\kern0pt}rule\ Least{\isacharunderscore}{\kern0pt}cong{\isacharparenright}{\kern0pt}\isanewline
\ \ \ \ \ \isacommand{apply}\isamarkupfalse%
{\isacharparenleft}{\kern0pt}rule\ iff{\isacharunderscore}{\kern0pt}conjI{\isadigit{2}}{\isacharcomma}{\kern0pt}\ simp{\isacharparenright}{\kern0pt}{\isacharplus}{\kern0pt}\isanewline
\ \ \isacommand{using}\isamarkupfalse%
\ MVset{\isacharunderscore}{\kern0pt}in{\isacharunderscore}{\kern0pt}M\isanewline
\ \ \ \ \ \isacommand{apply}\isamarkupfalse%
\ simp\isanewline
\ \ \ \ \ \isacommand{apply}\isamarkupfalse%
{\isacharparenleft}{\kern0pt}rule\ iffI{\isacharcomma}{\kern0pt}\ force{\isacharparenright}{\kern0pt}\isanewline
\ \ \ \ \ \isacommand{apply}\isamarkupfalse%
\ clarify\ \isanewline
\ \ \ \ \ \isacommand{apply}\isamarkupfalse%
{\isacharparenleft}{\kern0pt}rename{\isacharunderscore}{\kern0pt}tac\ p\ y\ a\ b\ z{\isacharcomma}{\kern0pt}\ rule{\isacharunderscore}{\kern0pt}tac\ x{\isacharequal}{\kern0pt}z\ \isakeyword{in}\ bexI{\isacharcomma}{\kern0pt}\ force{\isacharparenright}{\kern0pt}\isanewline
\ \ \isacommand{using}\isamarkupfalse%
\ MVset{\isacharunderscore}{\kern0pt}in{\isacharunderscore}{\kern0pt}M\ transM\ \isanewline
\ \ \ \ \ \isacommand{apply}\isamarkupfalse%
\ auto{\isacharbrackleft}{\kern0pt}{\isadigit{3}}{\isacharbrackright}{\kern0pt}\isanewline
\ \ \isacommand{using}\isamarkupfalse%
\ P{\isacharunderscore}{\kern0pt}in{\isacharunderscore}{\kern0pt}M\ leq{\isacharunderscore}{\kern0pt}in{\isacharunderscore}{\kern0pt}M\ one{\isacharunderscore}{\kern0pt}in{\isacharunderscore}{\kern0pt}M\ {\isasymF}{\isacharunderscore}{\kern0pt}in{\isacharunderscore}{\kern0pt}M\ {\isasymG}{\isacharunderscore}{\kern0pt}in{\isacharunderscore}{\kern0pt}M\ P{\isacharunderscore}{\kern0pt}auto{\isacharunderscore}{\kern0pt}in{\isacharunderscore}{\kern0pt}M\ pair{\isacharunderscore}{\kern0pt}in{\isacharunderscore}{\kern0pt}M{\isacharunderscore}{\kern0pt}iff\isanewline
\ \ \isacommand{by}\isamarkupfalse%
\ auto%
\endisatagproof
{\isafoldproof}%
%
\isadelimproof
\isanewline
%
\endisadelimproof
\isanewline
\isacommand{lemma}\isamarkupfalse%
\ is{\isacharunderscore}{\kern0pt}least{\isacharunderscore}{\kern0pt}index{\isacharunderscore}{\kern0pt}of{\isacharunderscore}{\kern0pt}Vset{\isacharunderscore}{\kern0pt}contains{\isacharunderscore}{\kern0pt}witness{\isacharunderscore}{\kern0pt}fm{\isacharunderscore}{\kern0pt}type\ {\isacharcolon}{\kern0pt}\ \isanewline
\ \ \isakeyword{fixes}\ {\isasymphi}\ \ \isanewline
\ \ \isakeyword{assumes}\ {\isachardoublequoteopen}{\isasymphi}\ {\isasymin}\ formula{\isachardoublequoteclose}\isanewline
\ \ \isakeyword{shows}\ {\isachardoublequoteopen}is{\isacharunderscore}{\kern0pt}least{\isacharunderscore}{\kern0pt}index{\isacharunderscore}{\kern0pt}of{\isacharunderscore}{\kern0pt}Vset{\isacharunderscore}{\kern0pt}contains{\isacharunderscore}{\kern0pt}witness{\isacharunderscore}{\kern0pt}fm{\isacharparenleft}{\kern0pt}{\isasymphi}{\isacharparenright}{\kern0pt}\ {\isasymin}\ formula{\isachardoublequoteclose}\ \isanewline
%
\isadelimproof
\isanewline
\ \ %
\endisadelimproof
%
\isatagproof
\isacommand{unfolding}\isamarkupfalse%
\ is{\isacharunderscore}{\kern0pt}least{\isacharunderscore}{\kern0pt}index{\isacharunderscore}{\kern0pt}of{\isacharunderscore}{\kern0pt}Vset{\isacharunderscore}{\kern0pt}contains{\isacharunderscore}{\kern0pt}witness{\isacharunderscore}{\kern0pt}fm{\isacharunderscore}{\kern0pt}def\isanewline
\ \ \isacommand{apply}\isamarkupfalse%
{\isacharparenleft}{\kern0pt}subgoal{\isacharunderscore}{\kern0pt}tac\ {\isachardoublequoteopen}fst{\isacharunderscore}{\kern0pt}fm{\isacharparenleft}{\kern0pt}{\isadigit{3}}{\isacharcomma}{\kern0pt}\ {\isadigit{1}}{\isacharparenright}{\kern0pt}\ {\isasymin}\ formula\ {\isasymand}\ snd{\isacharunderscore}{\kern0pt}fm{\isacharparenleft}{\kern0pt}{\isadigit{3}}{\isacharcomma}{\kern0pt}\ {\isadigit{2}}{\isacharparenright}{\kern0pt}\ {\isasymin}\ formula\ {\isasymand}\ is{\isacharunderscore}{\kern0pt}MVset{\isacharunderscore}{\kern0pt}fm{\isacharparenleft}{\kern0pt}{\isadigit{2}}{\isacharcomma}{\kern0pt}\ {\isadigit{1}}{\isacharparenright}{\kern0pt}\ {\isasymin}\ formula\ {\isasymand}\ is{\isacharunderscore}{\kern0pt}HS{\isacharunderscore}{\kern0pt}fm{\isacharparenleft}{\kern0pt}{\isadigit{1}}{\isadigit{1}}{\isacharcomma}{\kern0pt}\ {\isadigit{0}}{\isacharparenright}{\kern0pt}\ {\isasymin}\ formula\ {\isasymand}\ ren{\isacharunderscore}{\kern0pt}least{\isacharunderscore}{\kern0pt}index{\isacharparenleft}{\kern0pt}forcesHS{\isacharparenleft}{\kern0pt}{\isasymphi}{\isacharparenright}{\kern0pt}{\isacharparenright}{\kern0pt}\ {\isasymin}\ formula{\isachardoublequoteclose}{\isacharparenright}{\kern0pt}\isanewline
\ \ \ \isacommand{apply}\isamarkupfalse%
\ force\isanewline
\ \ \isacommand{apply}\isamarkupfalse%
{\isacharparenleft}{\kern0pt}rule\ conjI{\isacharcomma}{\kern0pt}\ simp\ add{\isacharcolon}{\kern0pt}fst{\isacharunderscore}{\kern0pt}fm{\isacharunderscore}{\kern0pt}def{\isacharparenright}{\kern0pt}\isanewline
\ \ \isacommand{apply}\isamarkupfalse%
{\isacharparenleft}{\kern0pt}rule\ conjI{\isacharcomma}{\kern0pt}\ simp\ add{\isacharcolon}{\kern0pt}snd{\isacharunderscore}{\kern0pt}fm{\isacharunderscore}{\kern0pt}def{\isacharparenright}{\kern0pt}\isanewline
\ \ \isacommand{apply}\isamarkupfalse%
{\isacharparenleft}{\kern0pt}rule\ conjI{\isacharcomma}{\kern0pt}\ rule\ is{\isacharunderscore}{\kern0pt}MVset{\isacharunderscore}{\kern0pt}fm{\isacharunderscore}{\kern0pt}type{\isacharcomma}{\kern0pt}\ simp{\isacharcomma}{\kern0pt}\ simp{\isacharcomma}{\kern0pt}\ rule\ conjI{\isacharparenright}{\kern0pt}\isanewline
\ \ \ \isacommand{apply}\isamarkupfalse%
{\isacharparenleft}{\kern0pt}rule\ is{\isacharunderscore}{\kern0pt}HS{\isacharunderscore}{\kern0pt}fm{\isacharunderscore}{\kern0pt}type{\isacharcomma}{\kern0pt}\ simp{\isacharcomma}{\kern0pt}\ simp{\isacharparenright}{\kern0pt}\isanewline
\ \ \isacommand{unfolding}\isamarkupfalse%
\ ren{\isacharunderscore}{\kern0pt}least{\isacharunderscore}{\kern0pt}index{\isacharunderscore}{\kern0pt}def\ \isanewline
\ \ \isacommand{apply}\isamarkupfalse%
{\isacharparenleft}{\kern0pt}rule\ Exists{\isacharunderscore}{\kern0pt}type{\isacharparenright}{\kern0pt}{\isacharplus}{\kern0pt}\isanewline
\ \ \isacommand{apply}\isamarkupfalse%
{\isacharparenleft}{\kern0pt}rule\ And{\isacharunderscore}{\kern0pt}type{\isacharcomma}{\kern0pt}\ simp{\isacharparenright}{\kern0pt}{\isacharplus}{\kern0pt}\isanewline
\ \ \isacommand{apply}\isamarkupfalse%
{\isacharparenleft}{\kern0pt}rule\ iterates{\isacharunderscore}{\kern0pt}type{\isacharparenright}{\kern0pt}\isanewline
\ \ \isacommand{using}\isamarkupfalse%
\ assms\ forcesHS{\isacharunderscore}{\kern0pt}type\ \isanewline
\ \ \ \ \isacommand{apply}\isamarkupfalse%
\ auto{\isacharbrackleft}{\kern0pt}{\isadigit{2}}{\isacharbrackright}{\kern0pt}\isanewline
\ \ \isacommand{apply}\isamarkupfalse%
{\isacharparenleft}{\kern0pt}rule\ function{\isacharunderscore}{\kern0pt}value{\isacharunderscore}{\kern0pt}in{\isacharparenright}{\kern0pt}\isanewline
\ \ \isacommand{using}\isamarkupfalse%
\ incr{\isacharunderscore}{\kern0pt}bv{\isacharunderscore}{\kern0pt}type\ \isanewline
\ \ \isacommand{by}\isamarkupfalse%
\ auto%
\endisatagproof
{\isafoldproof}%
%
\isadelimproof
\isanewline
%
\endisadelimproof
\isanewline
\isacommand{lemma}\isamarkupfalse%
\ arity{\isacharunderscore}{\kern0pt}least{\isacharunderscore}{\kern0pt}fm\ {\isacharcolon}{\kern0pt}\ \isanewline
\ \ \isakeyword{fixes}\ p\ i\isanewline
\ \ \isakeyword{assumes}\ {\isachardoublequoteopen}p\ {\isasymin}\ formula{\isachardoublequoteclose}\ {\isachardoublequoteopen}i\ {\isasymin}\ nat{\isachardoublequoteclose}\ \isanewline
\ \ \isakeyword{shows}\ {\isachardoublequoteopen}arity{\isacharparenleft}{\kern0pt}least{\isacharunderscore}{\kern0pt}fm{\isacharparenleft}{\kern0pt}p{\isacharcomma}{\kern0pt}\ i{\isacharparenright}{\kern0pt}{\isacharparenright}{\kern0pt}\ {\isasymle}\ pred{\isacharparenleft}{\kern0pt}arity{\isacharparenleft}{\kern0pt}p{\isacharparenright}{\kern0pt}{\isacharparenright}{\kern0pt}\ {\isasymunion}\ succ{\isacharparenleft}{\kern0pt}i{\isacharparenright}{\kern0pt}{\isachardoublequoteclose}\ \isanewline
%
\isadelimproof
\ \ %
\endisadelimproof
%
\isatagproof
\isacommand{unfolding}\isamarkupfalse%
\ least{\isacharunderscore}{\kern0pt}fm{\isacharunderscore}{\kern0pt}def\ \isanewline
\ \ \isacommand{apply}\isamarkupfalse%
\ simp\isanewline
\ \ \isacommand{apply}\isamarkupfalse%
{\isacharparenleft}{\kern0pt}subst\ arity{\isacharunderscore}{\kern0pt}ordinal{\isacharunderscore}{\kern0pt}fm{\isacharcomma}{\kern0pt}\ simp\ add{\isacharcolon}{\kern0pt}assms{\isacharparenright}{\kern0pt}\isanewline
\ \ \isacommand{apply}\isamarkupfalse%
{\isacharparenleft}{\kern0pt}subst\ arity{\isacharunderscore}{\kern0pt}empty{\isacharunderscore}{\kern0pt}fm{\isacharcomma}{\kern0pt}\ simp\ add{\isacharcolon}{\kern0pt}assms{\isacharparenright}{\kern0pt}\ \isanewline
\ \ \isacommand{using}\isamarkupfalse%
\ assms\isanewline
\ \ \isacommand{apply}\isamarkupfalse%
{\isacharparenleft}{\kern0pt}rule{\isacharunderscore}{\kern0pt}tac\ Un{\isacharunderscore}{\kern0pt}least{\isacharunderscore}{\kern0pt}lt{\isacharcomma}{\kern0pt}\ simp{\isacharcomma}{\kern0pt}\ rule{\isacharunderscore}{\kern0pt}tac\ ltI{\isacharcomma}{\kern0pt}\ simp{\isacharcomma}{\kern0pt}\ simp{\isacharparenright}{\kern0pt}{\isacharplus}{\kern0pt}\isanewline
\ \ \ \ \isacommand{apply}\isamarkupfalse%
{\isacharparenleft}{\kern0pt}rule\ ltD{\isacharcomma}{\kern0pt}\ rule\ Un{\isacharunderscore}{\kern0pt}least{\isacharunderscore}{\kern0pt}lt{\isacharcomma}{\kern0pt}\ simp{\isacharcomma}{\kern0pt}\ rule{\isacharunderscore}{\kern0pt}tac\ ltI{\isacharcomma}{\kern0pt}\ simp{\isacharcomma}{\kern0pt}\ simp{\isacharparenright}{\kern0pt}\isanewline
\ \ \ \ \isacommand{apply}\isamarkupfalse%
{\isacharparenleft}{\kern0pt}rule\ pred{\isacharunderscore}{\kern0pt}le{\isacharcomma}{\kern0pt}\ simp{\isacharcomma}{\kern0pt}\ simp{\isacharcomma}{\kern0pt}\ rule\ Un{\isacharunderscore}{\kern0pt}least{\isacharunderscore}{\kern0pt}lt{\isacharcomma}{\kern0pt}\ simp{\isacharparenright}{\kern0pt}\isanewline
\ \ \ \ \isacommand{apply}\isamarkupfalse%
{\isacharparenleft}{\kern0pt}rule{\isacharunderscore}{\kern0pt}tac\ n{\isacharequal}{\kern0pt}{\isachardoublequoteopen}arity{\isacharparenleft}{\kern0pt}p{\isacharparenright}{\kern0pt}{\isachardoublequoteclose}\ \isakeyword{in}\ natE{\isacharcomma}{\kern0pt}\ simp{\isacharcomma}{\kern0pt}\ simp{\isacharparenright}{\kern0pt}\isanewline
\ \ \ \ \isacommand{apply}\isamarkupfalse%
{\isacharparenleft}{\kern0pt}subst\ succ{\isacharunderscore}{\kern0pt}Un{\isacharunderscore}{\kern0pt}distrib{\isacharcomma}{\kern0pt}\ simp{\isacharcomma}{\kern0pt}\ simp{\isacharparenright}{\kern0pt}\isanewline
\ \ \ \ \isacommand{apply}\isamarkupfalse%
{\isacharparenleft}{\kern0pt}subst\ succ{\isacharunderscore}{\kern0pt}pred{\isacharunderscore}{\kern0pt}eq{\isacharcomma}{\kern0pt}\ simp{\isacharcomma}{\kern0pt}\ simp{\isacharparenright}{\kern0pt}\isanewline
\ \ \ \ \isacommand{apply}\isamarkupfalse%
{\isacharparenleft}{\kern0pt}subst\ succ{\isacharunderscore}{\kern0pt}Un{\isacharunderscore}{\kern0pt}distrib{\isacharcomma}{\kern0pt}\ simp{\isacharcomma}{\kern0pt}\ simp{\isacharparenright}{\kern0pt}\isanewline
\ \ \ \ \isacommand{apply}\isamarkupfalse%
{\isacharparenleft}{\kern0pt}rule\ ltI{\isacharcomma}{\kern0pt}\ simp{\isacharcomma}{\kern0pt}\ simp{\isacharcomma}{\kern0pt}\ simp{\isacharparenright}{\kern0pt}\isanewline
\ \ \isacommand{apply}\isamarkupfalse%
{\isacharparenleft}{\kern0pt}rule\ Un{\isacharunderscore}{\kern0pt}least{\isacharunderscore}{\kern0pt}lt{\isacharparenright}{\kern0pt}\isanewline
\ \ \ \isacommand{apply}\isamarkupfalse%
{\isacharparenleft}{\kern0pt}rule\ pred{\isacharunderscore}{\kern0pt}le{\isacharcomma}{\kern0pt}\ simp{\isacharunderscore}{\kern0pt}all{\isacharparenright}{\kern0pt}\isanewline
\ \ \ \isacommand{apply}\isamarkupfalse%
{\isacharparenleft}{\kern0pt}rule\ Un{\isacharunderscore}{\kern0pt}least{\isacharunderscore}{\kern0pt}lt{\isacharparenright}{\kern0pt}\isanewline
\ \ \ \ \isacommand{apply}\isamarkupfalse%
{\isacharparenleft}{\kern0pt}subst\ succ{\isacharunderscore}{\kern0pt}Un{\isacharunderscore}{\kern0pt}distrib{\isacharcomma}{\kern0pt}\ simp{\isacharcomma}{\kern0pt}\ simp{\isacharparenright}{\kern0pt}\isanewline
\ \ \ \ \isacommand{apply}\isamarkupfalse%
{\isacharparenleft}{\kern0pt}rule{\isacharunderscore}{\kern0pt}tac\ n{\isacharequal}{\kern0pt}{\isachardoublequoteopen}arity{\isacharparenleft}{\kern0pt}p{\isacharparenright}{\kern0pt}{\isachardoublequoteclose}\ \isakeyword{in}\ natE{\isacharcomma}{\kern0pt}\ simp{\isacharcomma}{\kern0pt}\ simp{\isacharparenright}{\kern0pt}\isanewline
\ \ \ \ \isacommand{apply}\isamarkupfalse%
{\isacharparenleft}{\kern0pt}subst\ succ{\isacharunderscore}{\kern0pt}Un{\isacharunderscore}{\kern0pt}distrib{\isacharcomma}{\kern0pt}\ simp{\isacharcomma}{\kern0pt}\ simp{\isacharparenright}{\kern0pt}\isanewline
\ \ \ \ \isacommand{apply}\isamarkupfalse%
{\isacharparenleft}{\kern0pt}subst\ succ{\isacharunderscore}{\kern0pt}pred{\isacharunderscore}{\kern0pt}eq{\isacharcomma}{\kern0pt}\ simp{\isacharcomma}{\kern0pt}\ simp{\isacharparenright}{\kern0pt}\isanewline
\ \ \ \ \isacommand{apply}\isamarkupfalse%
{\isacharparenleft}{\kern0pt}rule\ ltI{\isacharcomma}{\kern0pt}\ simp{\isacharcomma}{\kern0pt}\ simp{\isacharparenright}{\kern0pt}\isanewline
\ \ \ \isacommand{apply}\isamarkupfalse%
{\isacharparenleft}{\kern0pt}rule\ Un{\isacharunderscore}{\kern0pt}least{\isacharunderscore}{\kern0pt}lt{\isacharcomma}{\kern0pt}\ simp{\isacharcomma}{\kern0pt}\ simp{\isacharparenright}{\kern0pt}\isanewline
\ \ \ \isacommand{apply}\isamarkupfalse%
{\isacharparenleft}{\kern0pt}rule\ ltI{\isacharcomma}{\kern0pt}\ simp{\isacharcomma}{\kern0pt}\ simp{\isacharparenright}{\kern0pt}\isanewline
\ \ \isacommand{apply}\isamarkupfalse%
{\isacharparenleft}{\kern0pt}rule\ pred{\isacharunderscore}{\kern0pt}le{\isacharcomma}{\kern0pt}\ simp{\isacharcomma}{\kern0pt}\ simp{\isacharparenright}{\kern0pt}\isanewline
\ \ \isacommand{apply}\isamarkupfalse%
{\isacharparenleft}{\kern0pt}rule\ Un{\isacharunderscore}{\kern0pt}least{\isacharunderscore}{\kern0pt}lt{\isacharparenright}{\kern0pt}{\isacharplus}{\kern0pt}\isanewline
\ \ \ \ \isacommand{apply}\isamarkupfalse%
{\isacharparenleft}{\kern0pt}simp{\isacharparenright}{\kern0pt}\ \ \ \ \ \ \ \ \ \ \ \ \ \ \ \ \ \isanewline
\ \ \ \isacommand{apply}\isamarkupfalse%
{\isacharparenleft}{\kern0pt}rule\ Un{\isacharunderscore}{\kern0pt}least{\isacharunderscore}{\kern0pt}lt{\isacharparenright}{\kern0pt}{\isacharplus}{\kern0pt}\isanewline
\ \ \ \ \isacommand{apply}\isamarkupfalse%
\ {\isacharparenleft}{\kern0pt}simp{\isacharcomma}{\kern0pt}\ simp{\isacharcomma}{\kern0pt}\ rule\ ltI{\isacharcomma}{\kern0pt}\ simp{\isacharcomma}{\kern0pt}\ simp{\isacharparenright}{\kern0pt}\isanewline
\ \ \ \ \isacommand{apply}\isamarkupfalse%
{\isacharparenleft}{\kern0pt}rule{\isacharunderscore}{\kern0pt}tac\ n{\isacharequal}{\kern0pt}{\isachardoublequoteopen}arity{\isacharparenleft}{\kern0pt}p{\isacharparenright}{\kern0pt}{\isachardoublequoteclose}\ \isakeyword{in}\ natE{\isacharcomma}{\kern0pt}\ simp{\isacharcomma}{\kern0pt}\ simp{\isacharparenright}{\kern0pt}\isanewline
\ \ \ \ \isacommand{apply}\isamarkupfalse%
{\isacharparenleft}{\kern0pt}subst\ succ{\isacharunderscore}{\kern0pt}Un{\isacharunderscore}{\kern0pt}distrib{\isacharcomma}{\kern0pt}\ simp{\isacharcomma}{\kern0pt}\ simp{\isacharparenright}{\kern0pt}\isanewline
\ \ \ \ \isacommand{apply}\isamarkupfalse%
{\isacharparenleft}{\kern0pt}subst\ succ{\isacharunderscore}{\kern0pt}pred{\isacharunderscore}{\kern0pt}eq{\isacharcomma}{\kern0pt}\ simp{\isacharcomma}{\kern0pt}\ simp{\isacharparenright}{\kern0pt}\isanewline
\ \ \ \ \isacommand{apply}\isamarkupfalse%
{\isacharparenleft}{\kern0pt}rule\ ltI{\isacharparenright}{\kern0pt}\isanewline
\ \ \ \isacommand{apply}\isamarkupfalse%
{\isacharparenleft}{\kern0pt}subst\ succ{\isacharunderscore}{\kern0pt}Un{\isacharunderscore}{\kern0pt}distrib{\isacharcomma}{\kern0pt}\ simp{\isacharcomma}{\kern0pt}\ simp{\isacharcomma}{\kern0pt}\ simp{\isacharcomma}{\kern0pt}\ simp{\isacharparenright}{\kern0pt}\isanewline
\ \ \isacommand{done}\isamarkupfalse%
%
\endisatagproof
{\isafoldproof}%
%
\isadelimproof
\isanewline
%
\endisadelimproof
\isanewline
\isacommand{lemma}\isamarkupfalse%
\ arity{\isacharunderscore}{\kern0pt}is{\isacharunderscore}{\kern0pt}least{\isacharunderscore}{\kern0pt}index{\isacharunderscore}{\kern0pt}of{\isacharunderscore}{\kern0pt}Vset{\isacharunderscore}{\kern0pt}contains{\isacharunderscore}{\kern0pt}witness{\isacharunderscore}{\kern0pt}fm\ {\isacharcolon}{\kern0pt}\ \isanewline
\ \ \isakeyword{fixes}\ {\isasymphi}\ \ \isanewline
\ \ \isakeyword{assumes}\ {\isachardoublequoteopen}{\isasymphi}\ {\isasymin}\ formula{\isachardoublequoteclose}\isanewline
\ \ \isakeyword{shows}\ {\isachardoublequoteopen}arity{\isacharparenleft}{\kern0pt}is{\isacharunderscore}{\kern0pt}least{\isacharunderscore}{\kern0pt}index{\isacharunderscore}{\kern0pt}of{\isacharunderscore}{\kern0pt}Vset{\isacharunderscore}{\kern0pt}contains{\isacharunderscore}{\kern0pt}witness{\isacharunderscore}{\kern0pt}fm{\isacharparenleft}{\kern0pt}{\isasymphi}{\isacharparenright}{\kern0pt}{\isacharparenright}{\kern0pt}\ {\isasymle}\ {\isadigit{4}}\ {\isacharhash}{\kern0pt}{\isacharplus}{\kern0pt}\ {\isacharparenleft}{\kern0pt}{\isadigit{2}}\ {\isasymunion}\ arity{\isacharparenleft}{\kern0pt}{\isasymphi}{\isacharparenright}{\kern0pt}{\isacharparenright}{\kern0pt}{\isachardoublequoteclose}\ \isanewline
%
\isadelimproof
\isanewline
\ \ %
\endisadelimproof
%
\isatagproof
\isacommand{apply}\isamarkupfalse%
{\isacharparenleft}{\kern0pt}subgoal{\isacharunderscore}{\kern0pt}tac\ {\isachardoublequoteopen}fst{\isacharunderscore}{\kern0pt}fm{\isacharparenleft}{\kern0pt}{\isadigit{3}}{\isacharcomma}{\kern0pt}\ {\isadigit{1}}{\isacharparenright}{\kern0pt}\ {\isasymin}\ formula\ {\isasymand}\ snd{\isacharunderscore}{\kern0pt}fm{\isacharparenleft}{\kern0pt}{\isadigit{3}}{\isacharcomma}{\kern0pt}\ {\isadigit{2}}{\isacharparenright}{\kern0pt}\ {\isasymin}\ formula\ {\isasymand}\ is{\isacharunderscore}{\kern0pt}MVset{\isacharunderscore}{\kern0pt}fm{\isacharparenleft}{\kern0pt}{\isadigit{2}}{\isacharcomma}{\kern0pt}\ {\isadigit{1}}{\isacharparenright}{\kern0pt}\ {\isasymin}\ formula\ {\isasymand}\ is{\isacharunderscore}{\kern0pt}HS{\isacharunderscore}{\kern0pt}fm{\isacharparenleft}{\kern0pt}{\isadigit{1}}{\isadigit{1}}{\isacharcomma}{\kern0pt}\ {\isadigit{0}}{\isacharparenright}{\kern0pt}\ {\isasymin}\ formula\ {\isasymand}\ ren{\isacharunderscore}{\kern0pt}least{\isacharunderscore}{\kern0pt}index{\isacharparenleft}{\kern0pt}forcesHS{\isacharparenleft}{\kern0pt}{\isasymphi}{\isacharparenright}{\kern0pt}{\isacharparenright}{\kern0pt}\ {\isasymin}\ formula{\isachardoublequoteclose}{\isacharparenright}{\kern0pt}\isanewline
\ \ \isacommand{unfolding}\isamarkupfalse%
\ is{\isacharunderscore}{\kern0pt}least{\isacharunderscore}{\kern0pt}index{\isacharunderscore}{\kern0pt}of{\isacharunderscore}{\kern0pt}Vset{\isacharunderscore}{\kern0pt}contains{\isacharunderscore}{\kern0pt}witness{\isacharunderscore}{\kern0pt}fm{\isacharunderscore}{\kern0pt}def\isanewline
\ \ \ \isacommand{apply}\isamarkupfalse%
\ simp\isanewline
\ \ \ \isacommand{apply}\isamarkupfalse%
{\isacharparenleft}{\kern0pt}rule\ pred{\isacharunderscore}{\kern0pt}le{\isacharcomma}{\kern0pt}\ force{\isacharcomma}{\kern0pt}\ force{\isacharparenright}{\kern0pt}{\isacharplus}{\kern0pt}\isanewline
\ \ \ \isacommand{apply}\isamarkupfalse%
{\isacharparenleft}{\kern0pt}subst\ arity{\isacharunderscore}{\kern0pt}fst{\isacharunderscore}{\kern0pt}fm{\isacharcomma}{\kern0pt}\ simp{\isacharcomma}{\kern0pt}\ simp{\isacharparenright}{\kern0pt}\isanewline
\ \ \ \isacommand{apply}\isamarkupfalse%
{\isacharparenleft}{\kern0pt}subst\ arity{\isacharunderscore}{\kern0pt}snd{\isacharunderscore}{\kern0pt}fm{\isacharcomma}{\kern0pt}\ simp{\isacharcomma}{\kern0pt}\ simp{\isacharparenright}{\kern0pt}\isanewline
\ \ \ \isacommand{apply}\isamarkupfalse%
{\isacharparenleft}{\kern0pt}rule\ Un{\isacharunderscore}{\kern0pt}least{\isacharunderscore}{\kern0pt}lt{\isacharparenright}{\kern0pt}{\isacharplus}{\kern0pt}\isanewline
\ \ \ \ \ \isacommand{apply}\isamarkupfalse%
\ auto{\isacharbrackleft}{\kern0pt}{\isadigit{2}}{\isacharbrackright}{\kern0pt}\isanewline
\ \ \ \isacommand{apply}\isamarkupfalse%
{\isacharparenleft}{\kern0pt}rule\ Un{\isacharunderscore}{\kern0pt}least{\isacharunderscore}{\kern0pt}lt{\isacharparenright}{\kern0pt}{\isacharplus}{\kern0pt}\isanewline
\ \ \ \ \ \isacommand{apply}\isamarkupfalse%
\ auto{\isacharbrackleft}{\kern0pt}{\isadigit{2}}{\isacharbrackright}{\kern0pt}\isanewline
\ \ \ \isacommand{apply}\isamarkupfalse%
{\isacharparenleft}{\kern0pt}rule\ Un{\isacharunderscore}{\kern0pt}least{\isacharunderscore}{\kern0pt}lt{\isacharcomma}{\kern0pt}\ rule\ le{\isacharunderscore}{\kern0pt}trans{\isacharparenright}{\kern0pt}\ \isanewline
\ \ \ \ \ \isacommand{apply}\isamarkupfalse%
{\isacharparenleft}{\kern0pt}subst\ arity{\isacharunderscore}{\kern0pt}least{\isacharunderscore}{\kern0pt}fm{\isacharcomma}{\kern0pt}\ force{\isacharcomma}{\kern0pt}\ simp{\isacharcomma}{\kern0pt}\ simp{\isacharparenright}{\kern0pt}\isanewline
\ \ \ \ \isacommand{apply}\isamarkupfalse%
{\isacharparenleft}{\kern0pt}rule\ Un{\isacharunderscore}{\kern0pt}least{\isacharunderscore}{\kern0pt}lt{\isacharcomma}{\kern0pt}\ rule\ pred{\isacharunderscore}{\kern0pt}le{\isacharcomma}{\kern0pt}\ force{\isacharcomma}{\kern0pt}\ force{\isacharcomma}{\kern0pt}\ simp{\isacharparenright}{\kern0pt}\isanewline
\ \ \ \ \ \isacommand{apply}\isamarkupfalse%
{\isacharparenleft}{\kern0pt}rule\ pred{\isacharunderscore}{\kern0pt}le{\isacharcomma}{\kern0pt}\ force{\isacharcomma}{\kern0pt}\ force{\isacharparenright}{\kern0pt}\isanewline
\ \ \ \ \ \isacommand{apply}\isamarkupfalse%
{\isacharparenleft}{\kern0pt}rule\ pred{\isacharunderscore}{\kern0pt}le{\isacharcomma}{\kern0pt}\ force{\isacharcomma}{\kern0pt}\ force{\isacharparenright}{\kern0pt}{\isacharplus}{\kern0pt}\isanewline
\ \ \ \ \ \isacommand{apply}\isamarkupfalse%
{\isacharparenleft}{\kern0pt}rule\ Un{\isacharunderscore}{\kern0pt}least{\isacharunderscore}{\kern0pt}lt{\isacharcomma}{\kern0pt}\ rule\ le{\isacharunderscore}{\kern0pt}trans{\isacharcomma}{\kern0pt}\ rule\ arity{\isacharunderscore}{\kern0pt}is{\isacharunderscore}{\kern0pt}MVset{\isacharunderscore}{\kern0pt}fm{\isacharcomma}{\kern0pt}\ simp{\isacharcomma}{\kern0pt}\ simp{\isacharparenright}{\kern0pt}\isanewline
\ \ \ \ \ \ \isacommand{apply}\isamarkupfalse%
{\isacharparenleft}{\kern0pt}rule\ Un{\isacharunderscore}{\kern0pt}least{\isacharunderscore}{\kern0pt}lt{\isacharcomma}{\kern0pt}\ simp{\isacharcomma}{\kern0pt}\ simp{\isacharparenright}{\kern0pt}{\isacharplus}{\kern0pt}\isanewline
\ \ \ \ \ \isacommand{apply}\isamarkupfalse%
{\isacharparenleft}{\kern0pt}rule\ Un{\isacharunderscore}{\kern0pt}least{\isacharunderscore}{\kern0pt}lt{\isacharparenright}{\kern0pt}{\isacharplus}{\kern0pt}\isanewline
\ \ \ \ \ \ \ \isacommand{apply}\isamarkupfalse%
\ {\isacharparenleft}{\kern0pt}simp{\isacharcomma}{\kern0pt}\ simp{\isacharparenright}{\kern0pt}\isanewline
\ \ \ \ \ \isacommand{apply}\isamarkupfalse%
{\isacharparenleft}{\kern0pt}rule\ Un{\isacharunderscore}{\kern0pt}least{\isacharunderscore}{\kern0pt}lt{\isacharparenright}{\kern0pt}{\isacharplus}{\kern0pt}\isanewline
\ \ \ \ \ \ \isacommand{apply}\isamarkupfalse%
{\isacharparenleft}{\kern0pt}rule\ le{\isacharunderscore}{\kern0pt}trans{\isacharcomma}{\kern0pt}rule\ arity{\isacharunderscore}{\kern0pt}is{\isacharunderscore}{\kern0pt}HS{\isacharunderscore}{\kern0pt}fm{\isacharcomma}{\kern0pt}\ simp{\isacharcomma}{\kern0pt}\ simp{\isacharparenright}{\kern0pt}\isanewline
\ \ \ \ \ \ \isacommand{apply}\isamarkupfalse%
{\isacharparenleft}{\kern0pt}rule\ Un{\isacharunderscore}{\kern0pt}least{\isacharunderscore}{\kern0pt}lt{\isacharparenright}{\kern0pt}{\isacharplus}{\kern0pt}\isanewline
\ \ \isacommand{using}\isamarkupfalse%
\ assms\isanewline
\ \ \ \ \ \ \ \isacommand{apply}\isamarkupfalse%
\ auto{\isacharbrackleft}{\kern0pt}{\isadigit{2}}{\isacharbrackright}{\kern0pt}\isanewline
\ \ \isacommand{apply}\isamarkupfalse%
{\isacharparenleft}{\kern0pt}rule\ union{\isacharunderscore}{\kern0pt}lt{\isadigit{1}}{\isacharcomma}{\kern0pt}\ simp{\isacharcomma}{\kern0pt}\ simp{\isacharcomma}{\kern0pt}\ simp{\isacharcomma}{\kern0pt}\ simp{\isacharparenright}{\kern0pt}\isanewline
\ \ \ \ \ \isacommand{apply}\isamarkupfalse%
{\isacharparenleft}{\kern0pt}rule\ le{\isacharunderscore}{\kern0pt}trans{\isacharcomma}{\kern0pt}\ rule\ arity{\isacharunderscore}{\kern0pt}ren{\isacharunderscore}{\kern0pt}least{\isacharunderscore}{\kern0pt}index{\isacharparenright}{\kern0pt}\isanewline
\ \ \isacommand{using}\isamarkupfalse%
\ assms\ forcesHS{\isacharunderscore}{\kern0pt}type\ \ \isanewline
\ \ \ \ \ \ \ \isacommand{apply}\isamarkupfalse%
\ auto{\isacharbrackleft}{\kern0pt}{\isadigit{1}}{\isacharbrackright}{\kern0pt}\isanewline
\ \ \isacommand{using}\isamarkupfalse%
\ forcesHS{\isacharunderscore}{\kern0pt}def\isanewline
\ \ \ \ \ \ \isacommand{apply}\isamarkupfalse%
\ simp\isanewline
\ \ \ \ \ \ \isacommand{apply}\isamarkupfalse%
{\isacharparenleft}{\kern0pt}rule\ ltI{\isacharcomma}{\kern0pt}\ simp{\isacharparenright}{\kern0pt}\isanewline
\ \ \isacommand{using}\isamarkupfalse%
\ assms\ forcesHS{\isacharprime}{\kern0pt}{\isacharunderscore}{\kern0pt}type\ \isanewline
\ \ \ \ \ \ \isacommand{apply}\isamarkupfalse%
\ force\isanewline
\ \ \ \ \ \isacommand{apply}\isamarkupfalse%
\ simp\isanewline
\ \ \ \ \ \isacommand{apply}\isamarkupfalse%
{\isacharparenleft}{\kern0pt}rule\ Un{\isacharunderscore}{\kern0pt}least{\isacharunderscore}{\kern0pt}lt{\isacharparenright}{\kern0pt}\isanewline
\ \ \isacommand{using}\isamarkupfalse%
\ assms\isanewline
\ \ \ \ \ \ \isacommand{apply}\isamarkupfalse%
\ simp\isanewline
\ \ \ \ \ \ \isacommand{apply}\isamarkupfalse%
{\isacharparenleft}{\kern0pt}rule\ union{\isacharunderscore}{\kern0pt}lt{\isadigit{1}}{\isacharcomma}{\kern0pt}\ simp{\isacharcomma}{\kern0pt}\ simp{\isacharcomma}{\kern0pt}\ simp{\isacharcomma}{\kern0pt}\ simp{\isacharcomma}{\kern0pt}\ simp{\isacharparenright}{\kern0pt}\isanewline
\ \ \ \ \ \isacommand{apply}\isamarkupfalse%
{\isacharparenleft}{\kern0pt}insert\ forcesHS{\isacharunderscore}{\kern0pt}type{\isacharbrackleft}{\kern0pt}\isakeyword{where}\ {\isasymphi}{\isacharequal}{\kern0pt}{\isasymphi}{\isacharbrackright}{\kern0pt}{\isacharparenright}{\kern0pt}\isanewline
\ \ \isacommand{using}\isamarkupfalse%
\ assms\isanewline
\ \ \ \ \ \isacommand{apply}\isamarkupfalse%
\ simp\isanewline
\ \ \ \ \ \isacommand{apply}\isamarkupfalse%
{\isacharparenleft}{\kern0pt}rule\ le{\isacharunderscore}{\kern0pt}trans{\isacharcomma}{\kern0pt}\ rule\ arity{\isacharunderscore}{\kern0pt}forcesHS{\isacharparenright}{\kern0pt}\isanewline
\ \ \isacommand{using}\isamarkupfalse%
\ assms\isanewline
\ \ \ \ \ \ \isacommand{apply}\isamarkupfalse%
\ auto{\isacharbrackleft}{\kern0pt}{\isadigit{2}}{\isacharbrackright}{\kern0pt}\isanewline
\ \ \ \ \ \isacommand{apply}\isamarkupfalse%
{\isacharparenleft}{\kern0pt}rule\ max{\isacharunderscore}{\kern0pt}le{\isadigit{2}}{\isacharcomma}{\kern0pt}\ simp{\isacharcomma}{\kern0pt}\ simp{\isacharcomma}{\kern0pt}\ simp{\isacharparenright}{\kern0pt}\isanewline
\ \ \ \isacommand{apply}\isamarkupfalse%
{\isacharparenleft}{\kern0pt}subst\ arity{\isacharunderscore}{\kern0pt}pair{\isacharunderscore}{\kern0pt}fm{\isacharcomma}{\kern0pt}\ simp{\isacharcomma}{\kern0pt}\ simp{\isacharcomma}{\kern0pt}\ simp{\isacharcomma}{\kern0pt}\ simp{\isacharparenright}{\kern0pt}\isanewline
\ \ \ \isacommand{apply}\isamarkupfalse%
{\isacharparenleft}{\kern0pt}rule\ Un{\isacharunderscore}{\kern0pt}least{\isacharunderscore}{\kern0pt}lt{\isacharcomma}{\kern0pt}\ simp{\isacharparenright}{\kern0pt}{\isacharplus}{\kern0pt}\isanewline
\ \ \ \isacommand{apply}\isamarkupfalse%
\ simp\isanewline
\ \ \isacommand{using}\isamarkupfalse%
\ fst{\isacharunderscore}{\kern0pt}fm{\isacharunderscore}{\kern0pt}def\ snd{\isacharunderscore}{\kern0pt}fm{\isacharunderscore}{\kern0pt}def\ is{\isacharunderscore}{\kern0pt}MVset{\isacharunderscore}{\kern0pt}fm{\isacharunderscore}{\kern0pt}type\ is{\isacharunderscore}{\kern0pt}HS{\isacharunderscore}{\kern0pt}fm{\isacharunderscore}{\kern0pt}type\ ren{\isacharunderscore}{\kern0pt}least{\isacharunderscore}{\kern0pt}index{\isacharunderscore}{\kern0pt}type\ forcesHS{\isacharunderscore}{\kern0pt}type\ assms\isanewline
\ \ \isacommand{apply}\isamarkupfalse%
\ auto\isanewline
\ \ \isacommand{done}\isamarkupfalse%
%
\endisatagproof
{\isafoldproof}%
%
\isadelimproof
\isanewline
%
\endisadelimproof
\isanewline
\isacommand{definition}\isamarkupfalse%
\ least{\isacharunderscore}{\kern0pt}indexes{\isacharunderscore}{\kern0pt}of{\isacharunderscore}{\kern0pt}Vset{\isacharunderscore}{\kern0pt}contains{\isacharunderscore}{\kern0pt}witness\ \isakeyword{where}\ \isanewline
\ \ {\isachardoublequoteopen}least{\isacharunderscore}{\kern0pt}indexes{\isacharunderscore}{\kern0pt}of{\isacharunderscore}{\kern0pt}Vset{\isacharunderscore}{\kern0pt}contains{\isacharunderscore}{\kern0pt}witness{\isacharparenleft}{\kern0pt}{\isasymphi}{\isacharcomma}{\kern0pt}\ env{\isacharcomma}{\kern0pt}\ x{\isacharparenright}{\kern0pt}\ {\isasymequiv}\ {\isacharbraceleft}{\kern0pt}\ {\isacharless}{\kern0pt}{\isacharless}{\kern0pt}y{\isacharcomma}{\kern0pt}\ p{\isachargreater}{\kern0pt}{\isacharcomma}{\kern0pt}\ {\isacharparenleft}{\kern0pt}{\isasymmu}\ a{\isachardot}{\kern0pt}\ a\ {\isasymin}\ M\ {\isasymand}\ Ord{\isacharparenleft}{\kern0pt}a{\isacharparenright}{\kern0pt}\ {\isasymand}\ {\isacharparenleft}{\kern0pt}{\isasymexists}z\ {\isasymin}\ MVset{\isacharparenleft}{\kern0pt}a{\isacharparenright}{\kern0pt}\ {\isasyminter}\ HS{\isachardot}{\kern0pt}\ p\ {\isasymtturnstile}HS\ {\isasymphi}\ {\isacharparenleft}{\kern0pt}{\isacharbrackleft}{\kern0pt}y{\isacharcomma}{\kern0pt}\ z{\isacharbrackright}{\kern0pt}\ {\isacharat}{\kern0pt}\ env{\isacharparenright}{\kern0pt}{\isacharparenright}{\kern0pt}{\isacharparenright}{\kern0pt}{\isachargreater}{\kern0pt}{\isachardot}{\kern0pt}\ {\isacharless}{\kern0pt}y{\isacharcomma}{\kern0pt}\ p{\isachargreater}{\kern0pt}\ {\isasymin}\ domain{\isacharparenleft}{\kern0pt}x{\isacharparenright}{\kern0pt}\ {\isasymtimes}\ P\ {\isacharbraceright}{\kern0pt}{\isachardoublequoteclose}\isanewline
\isanewline
\isacommand{lemma}\isamarkupfalse%
\ least{\isacharunderscore}{\kern0pt}indexes{\isacharunderscore}{\kern0pt}of{\isacharunderscore}{\kern0pt}Vset{\isacharunderscore}{\kern0pt}contains{\isacharunderscore}{\kern0pt}witness{\isacharunderscore}{\kern0pt}in{\isacharunderscore}{\kern0pt}M\ {\isacharcolon}{\kern0pt}\ \isanewline
\ \ \isakeyword{fixes}\ {\isasymphi}\ env\ x\ \isanewline
\ \ \isakeyword{assumes}\ {\isachardoublequoteopen}{\isasymphi}\ {\isasymin}\ formula{\isachardoublequoteclose}\ {\isachardoublequoteopen}env\ {\isasymin}\ list{\isacharparenleft}{\kern0pt}M{\isacharparenright}{\kern0pt}{\isachardoublequoteclose}\ {\isachardoublequoteopen}arity{\isacharparenleft}{\kern0pt}{\isasymphi}{\isacharparenright}{\kern0pt}\ {\isasymle}\ {\isadigit{2}}\ {\isacharhash}{\kern0pt}{\isacharplus}{\kern0pt}\ length{\isacharparenleft}{\kern0pt}env{\isacharparenright}{\kern0pt}{\isachardoublequoteclose}\ {\isachardoublequoteopen}x\ {\isasymin}\ M{\isachardoublequoteclose}\ \isanewline
\ \ \isakeyword{shows}\ {\isachardoublequoteopen}least{\isacharunderscore}{\kern0pt}indexes{\isacharunderscore}{\kern0pt}of{\isacharunderscore}{\kern0pt}Vset{\isacharunderscore}{\kern0pt}contains{\isacharunderscore}{\kern0pt}witness{\isacharparenleft}{\kern0pt}{\isasymphi}{\isacharcomma}{\kern0pt}\ env{\isacharcomma}{\kern0pt}\ x{\isacharparenright}{\kern0pt}\ {\isasymin}\ M{\isachardoublequoteclose}\ \isanewline
%
\isadelimproof
%
\endisadelimproof
%
\isatagproof
\isacommand{proof}\isamarkupfalse%
\ {\isacharminus}{\kern0pt}\isanewline
\ \ \isacommand{have}\isamarkupfalse%
\ strep\ {\isacharcolon}{\kern0pt}\ {\isachardoublequoteopen}strong{\isacharunderscore}{\kern0pt}replacement{\isacharparenleft}{\kern0pt}{\isacharhash}{\kern0pt}{\isacharhash}{\kern0pt}M{\isacharcomma}{\kern0pt}\ {\isasymlambda}u\ v{\isachardot}{\kern0pt}\ sats{\isacharparenleft}{\kern0pt}M{\isacharcomma}{\kern0pt}\ is{\isacharunderscore}{\kern0pt}least{\isacharunderscore}{\kern0pt}index{\isacharunderscore}{\kern0pt}of{\isacharunderscore}{\kern0pt}Vset{\isacharunderscore}{\kern0pt}contains{\isacharunderscore}{\kern0pt}witness{\isacharunderscore}{\kern0pt}fm{\isacharparenleft}{\kern0pt}{\isasymphi}{\isacharparenright}{\kern0pt}{\isacharcomma}{\kern0pt}\ {\isacharbrackleft}{\kern0pt}u{\isacharcomma}{\kern0pt}\ v{\isacharbrackright}{\kern0pt}\ {\isacharat}{\kern0pt}\ {\isacharbrackleft}{\kern0pt}P{\isacharcomma}{\kern0pt}\ leq{\isacharcomma}{\kern0pt}\ one{\isacharcomma}{\kern0pt}\ {\isacharless}{\kern0pt}{\isasymF}{\isacharcomma}{\kern0pt}\ {\isasymG}{\isacharcomma}{\kern0pt}\ P{\isacharcomma}{\kern0pt}\ P{\isacharunderscore}{\kern0pt}auto{\isachargreater}{\kern0pt}{\isacharbrackright}{\kern0pt}\ {\isacharat}{\kern0pt}\ env{\isacharparenright}{\kern0pt}{\isacharparenright}{\kern0pt}{\isachardoublequoteclose}\ \isanewline
\ \ \ \ \isacommand{apply}\isamarkupfalse%
{\isacharparenleft}{\kern0pt}rule\ replacement{\isacharunderscore}{\kern0pt}ax{\isacharcomma}{\kern0pt}\ rule\ is{\isacharunderscore}{\kern0pt}least{\isacharunderscore}{\kern0pt}index{\isacharunderscore}{\kern0pt}of{\isacharunderscore}{\kern0pt}Vset{\isacharunderscore}{\kern0pt}contains{\isacharunderscore}{\kern0pt}witness{\isacharunderscore}{\kern0pt}fm{\isacharunderscore}{\kern0pt}type{\isacharparenright}{\kern0pt}\isanewline
\ \ \ \ \isacommand{using}\isamarkupfalse%
\ assms\ {\isasymF}{\isacharunderscore}{\kern0pt}in{\isacharunderscore}{\kern0pt}M\ {\isasymG}{\isacharunderscore}{\kern0pt}in{\isacharunderscore}{\kern0pt}M\ P{\isacharunderscore}{\kern0pt}in{\isacharunderscore}{\kern0pt}M\ P{\isacharunderscore}{\kern0pt}auto{\isacharunderscore}{\kern0pt}in{\isacharunderscore}{\kern0pt}M\ leq{\isacharunderscore}{\kern0pt}in{\isacharunderscore}{\kern0pt}M\ one{\isacharunderscore}{\kern0pt}in{\isacharunderscore}{\kern0pt}M\ pair{\isacharunderscore}{\kern0pt}in{\isacharunderscore}{\kern0pt}M{\isacharunderscore}{\kern0pt}iff\isanewline
\ \ \ \ \ \ \isacommand{apply}\isamarkupfalse%
\ auto{\isacharbrackleft}{\kern0pt}{\isadigit{2}}{\isacharbrackright}{\kern0pt}\isanewline
\ \ \ \ \isacommand{apply}\isamarkupfalse%
{\isacharparenleft}{\kern0pt}rule\ le{\isacharunderscore}{\kern0pt}trans{\isacharcomma}{\kern0pt}\ rule\ arity{\isacharunderscore}{\kern0pt}is{\isacharunderscore}{\kern0pt}least{\isacharunderscore}{\kern0pt}index{\isacharunderscore}{\kern0pt}of{\isacharunderscore}{\kern0pt}Vset{\isacharunderscore}{\kern0pt}contains{\isacharunderscore}{\kern0pt}witness{\isacharunderscore}{\kern0pt}fm{\isacharparenright}{\kern0pt}\isanewline
\ \ \ \ \isacommand{using}\isamarkupfalse%
\ assms\ Un{\isacharunderscore}{\kern0pt}least{\isacharunderscore}{\kern0pt}lt\isanewline
\ \ \ \ \ \ \isacommand{apply}\isamarkupfalse%
\ simp{\isacharunderscore}{\kern0pt}all\isanewline
\ \ \ \ \isacommand{done}\isamarkupfalse%
\isanewline
\isanewline
\ \ \isacommand{have}\isamarkupfalse%
\ strep{\isacharprime}{\kern0pt}\ {\isacharcolon}{\kern0pt}\ {\isachardoublequoteopen}strong{\isacharunderscore}{\kern0pt}replacement{\isacharparenleft}{\kern0pt}{\isacharhash}{\kern0pt}{\isacharhash}{\kern0pt}M{\isacharcomma}{\kern0pt}\ {\isasymlambda}u\ v{\isachardot}{\kern0pt}\ v\ {\isacharequal}{\kern0pt}\ {\isacharless}{\kern0pt}u{\isacharcomma}{\kern0pt}\ {\isacharparenleft}{\kern0pt}{\isasymmu}\ a{\isachardot}{\kern0pt}\ a\ {\isasymin}\ M\ {\isasymand}\ Ord{\isacharparenleft}{\kern0pt}a{\isacharparenright}{\kern0pt}\ {\isasymand}\ {\isacharparenleft}{\kern0pt}{\isasymexists}z\ {\isasymin}\ MVset{\isacharparenleft}{\kern0pt}a{\isacharparenright}{\kern0pt}\ {\isasyminter}\ HS{\isachardot}{\kern0pt}\ snd{\isacharparenleft}{\kern0pt}u{\isacharparenright}{\kern0pt}\ {\isasymtturnstile}HS\ {\isasymphi}\ {\isacharparenleft}{\kern0pt}{\isacharbrackleft}{\kern0pt}fst{\isacharparenleft}{\kern0pt}u{\isacharparenright}{\kern0pt}{\isacharcomma}{\kern0pt}\ z{\isacharbrackright}{\kern0pt}\ {\isacharat}{\kern0pt}\ env{\isacharparenright}{\kern0pt}{\isacharparenright}{\kern0pt}{\isacharparenright}{\kern0pt}{\isachargreater}{\kern0pt}{\isacharparenright}{\kern0pt}{\isachardoublequoteclose}\ \isanewline
\ \ \ \ \isacommand{apply}\isamarkupfalse%
{\isacharparenleft}{\kern0pt}rule\ iffD{\isadigit{1}}{\isacharcomma}{\kern0pt}\ rule\ strong{\isacharunderscore}{\kern0pt}replacement{\isacharunderscore}{\kern0pt}cong{\isacharparenright}{\kern0pt}\isanewline
\ \ \ \ \ \isacommand{apply}\isamarkupfalse%
{\isacharparenleft}{\kern0pt}rule\ sats{\isacharunderscore}{\kern0pt}is{\isacharunderscore}{\kern0pt}least{\isacharunderscore}{\kern0pt}index{\isacharunderscore}{\kern0pt}of{\isacharunderscore}{\kern0pt}Vset{\isacharunderscore}{\kern0pt}contains{\isacharunderscore}{\kern0pt}witness{\isacharunderscore}{\kern0pt}fm{\isacharunderscore}{\kern0pt}iff{\isacharparenright}{\kern0pt}\isanewline
\ \ \ \ \isacommand{using}\isamarkupfalse%
\ assms\ strep\ \isanewline
\ \ \ \ \isacommand{by}\isamarkupfalse%
\ auto\isanewline
\isanewline
\ \ \isacommand{have}\isamarkupfalse%
\ H\ {\isacharcolon}{\kern0pt}\ {\isachardoublequoteopen}{\isacharbraceleft}{\kern0pt}\ {\isacharless}{\kern0pt}u{\isacharcomma}{\kern0pt}\ {\isacharparenleft}{\kern0pt}{\isasymmu}\ a{\isachardot}{\kern0pt}\ a\ {\isasymin}\ M\ {\isasymand}\ Ord{\isacharparenleft}{\kern0pt}a{\isacharparenright}{\kern0pt}\ {\isasymand}\ {\isacharparenleft}{\kern0pt}{\isasymexists}z\ {\isasymin}\ MVset{\isacharparenleft}{\kern0pt}a{\isacharparenright}{\kern0pt}\ {\isasyminter}\ HS{\isachardot}{\kern0pt}\ snd{\isacharparenleft}{\kern0pt}u{\isacharparenright}{\kern0pt}\ {\isasymtturnstile}HS\ {\isasymphi}\ {\isacharparenleft}{\kern0pt}{\isacharbrackleft}{\kern0pt}fst{\isacharparenleft}{\kern0pt}u{\isacharparenright}{\kern0pt}{\isacharcomma}{\kern0pt}\ z{\isacharbrackright}{\kern0pt}\ {\isacharat}{\kern0pt}\ env{\isacharparenright}{\kern0pt}{\isacharparenright}{\kern0pt}{\isacharparenright}{\kern0pt}{\isachargreater}{\kern0pt}\ {\isachardot}{\kern0pt}\ u\ {\isasymin}\ domain{\isacharparenleft}{\kern0pt}x{\isacharparenright}{\kern0pt}\ {\isasymtimes}\ P\ {\isacharbraceright}{\kern0pt}\ {\isasymin}\ M{\isachardoublequoteclose}\ {\isacharparenleft}{\kern0pt}\isakeyword{is}\ {\isachardoublequoteopen}{\isacharquery}{\kern0pt}A\ {\isasymin}\ M{\isachardoublequoteclose}{\isacharparenright}{\kern0pt}\isanewline
\ \ \ \ \isacommand{apply}\isamarkupfalse%
{\isacharparenleft}{\kern0pt}subgoal{\isacharunderscore}{\kern0pt}tac\ {\isachardoublequoteopen}domain{\isacharparenleft}{\kern0pt}x{\isacharparenright}{\kern0pt}\ {\isasymtimes}\ P\ {\isasymin}\ M{\isachardoublequoteclose}{\isacharparenright}{\kern0pt}\isanewline
\ \ \ \ \isacommand{apply}\isamarkupfalse%
{\isacharparenleft}{\kern0pt}rule\ to{\isacharunderscore}{\kern0pt}rin{\isacharcomma}{\kern0pt}\ rule\ RepFun{\isacharunderscore}{\kern0pt}closed{\isacharcomma}{\kern0pt}\ rule\ strep{\isacharprime}{\kern0pt}{\isacharcomma}{\kern0pt}\ simp{\isacharparenright}{\kern0pt}\isanewline
\ \ \ \ \isacommand{apply}\isamarkupfalse%
{\isacharparenleft}{\kern0pt}rule\ ballI{\isacharcomma}{\kern0pt}\ rule\ pair{\isacharunderscore}{\kern0pt}in{\isacharunderscore}{\kern0pt}MI{\isacharcomma}{\kern0pt}\ rule\ conjI{\isacharparenright}{\kern0pt}\isanewline
\ \ \ \ \isacommand{using}\isamarkupfalse%
\ transM\ \isanewline
\ \ \ \ \ \ \isacommand{apply}\isamarkupfalse%
\ auto{\isacharbrackleft}{\kern0pt}{\isadigit{1}}{\isacharbrackright}{\kern0pt}\isanewline
\ \ \ \ \ \isacommand{apply}\isamarkupfalse%
{\isacharparenleft}{\kern0pt}rule\ Least{\isacharunderscore}{\kern0pt}closed{\isacharcomma}{\kern0pt}\ simp{\isacharparenright}{\kern0pt}\isanewline
\ \ \ \ \isacommand{using}\isamarkupfalse%
\ domain{\isacharunderscore}{\kern0pt}closed\ cartprod{\isacharunderscore}{\kern0pt}closed\ P{\isacharunderscore}{\kern0pt}in{\isacharunderscore}{\kern0pt}M\ assms\isanewline
\ \ \ \ \isacommand{by}\isamarkupfalse%
\ auto\isanewline
\isanewline
\ \ \isacommand{have}\isamarkupfalse%
\ {\isachardoublequoteopen}{\isacharquery}{\kern0pt}A\ {\isacharequal}{\kern0pt}\ least{\isacharunderscore}{\kern0pt}indexes{\isacharunderscore}{\kern0pt}of{\isacharunderscore}{\kern0pt}Vset{\isacharunderscore}{\kern0pt}contains{\isacharunderscore}{\kern0pt}witness{\isacharparenleft}{\kern0pt}{\isasymphi}{\isacharcomma}{\kern0pt}\ env{\isacharcomma}{\kern0pt}\ x{\isacharparenright}{\kern0pt}{\isachardoublequoteclose}\ \isanewline
\ \ \ \ \isacommand{unfolding}\isamarkupfalse%
\ least{\isacharunderscore}{\kern0pt}indexes{\isacharunderscore}{\kern0pt}of{\isacharunderscore}{\kern0pt}Vset{\isacharunderscore}{\kern0pt}contains{\isacharunderscore}{\kern0pt}witness{\isacharunderscore}{\kern0pt}def\isanewline
\ \ \ \ \isacommand{apply}\isamarkupfalse%
{\isacharparenleft}{\kern0pt}rule\ equality{\isacharunderscore}{\kern0pt}iffI{\isacharcomma}{\kern0pt}\ rule\ iffI{\isacharparenright}{\kern0pt}\isanewline
\ \ \ \ \ \isacommand{apply}\isamarkupfalse%
\ force\isanewline
\ \ \ \ \isacommand{apply}\isamarkupfalse%
\ force\isanewline
\ \ \ \ \isacommand{done}\isamarkupfalse%
\isanewline
\isanewline
\ \ \isacommand{then}\isamarkupfalse%
\ \isacommand{show}\isamarkupfalse%
\ {\isacharquery}{\kern0pt}thesis\ \isacommand{using}\isamarkupfalse%
\ H\ \isacommand{by}\isamarkupfalse%
\ auto\isanewline
\isacommand{qed}\isamarkupfalse%
%
\endisatagproof
{\isafoldproof}%
%
\isadelimproof
\isanewline
%
\endisadelimproof
\isanewline
\isacommand{lemma}\isamarkupfalse%
\ ex{\isacharunderscore}{\kern0pt}hs{\isacharunderscore}{\kern0pt}subset{\isacharunderscore}{\kern0pt}contains{\isacharunderscore}{\kern0pt}witnesses\ {\isacharcolon}{\kern0pt}\ \isanewline
\ \ \isakeyword{fixes}\ {\isasymphi}\ env\ x\ \isanewline
\ \ \isakeyword{assumes}\ {\isachardoublequoteopen}{\isasymphi}\ {\isasymin}\ formula{\isachardoublequoteclose}\ {\isachardoublequoteopen}env\ {\isasymin}\ list{\isacharparenleft}{\kern0pt}M{\isacharparenright}{\kern0pt}{\isachardoublequoteclose}\ {\isachardoublequoteopen}arity{\isacharparenleft}{\kern0pt}{\isasymphi}{\isacharparenright}{\kern0pt}\ {\isasymle}\ {\isadigit{2}}\ {\isacharhash}{\kern0pt}{\isacharplus}{\kern0pt}\ length{\isacharparenleft}{\kern0pt}env{\isacharparenright}{\kern0pt}{\isachardoublequoteclose}\ {\isachardoublequoteopen}x\ {\isasymin}\ M{\isachardoublequoteclose}\ \isanewline
\ \ \isakeyword{shows}\ {\isachardoublequoteopen}{\isasymexists}S{\isachardot}{\kern0pt}\ S\ {\isasymin}\ M\ {\isasymand}\ S\ {\isasymsubseteq}\ HS\ {\isasymand}\ {\isacharparenleft}{\kern0pt}{\isasymforall}p\ {\isasymin}\ G{\isachardot}{\kern0pt}\ {\isasymforall}y\ {\isasymin}\ domain{\isacharparenleft}{\kern0pt}x{\isacharparenright}{\kern0pt}{\isachardot}{\kern0pt}\ {\isacharparenleft}{\kern0pt}{\isasymexists}z\ {\isasymin}\ HS{\isachardot}{\kern0pt}\ p\ {\isasymtturnstile}HS\ {\isasymphi}\ {\isacharparenleft}{\kern0pt}{\isacharbrackleft}{\kern0pt}y{\isacharcomma}{\kern0pt}\ z{\isacharbrackright}{\kern0pt}\ {\isacharat}{\kern0pt}\ env{\isacharparenright}{\kern0pt}{\isacharparenright}{\kern0pt}\ {\isasymlongleftrightarrow}\ {\isacharparenleft}{\kern0pt}{\isasymexists}z\ {\isasymin}\ S{\isachardot}{\kern0pt}\ p\ {\isasymtturnstile}HS\ {\isasymphi}\ {\isacharparenleft}{\kern0pt}{\isacharbrackleft}{\kern0pt}y{\isacharcomma}{\kern0pt}\ z{\isacharbrackright}{\kern0pt}\ {\isacharat}{\kern0pt}\ env{\isacharparenright}{\kern0pt}{\isacharparenright}{\kern0pt}{\isacharparenright}{\kern0pt}{\isachardoublequoteclose}\ \isanewline
%
\isadelimproof
%
\endisadelimproof
%
\isatagproof
\isacommand{proof}\isamarkupfalse%
{\isacharminus}{\kern0pt}\ \ \isanewline
\ \ \isacommand{define}\isamarkupfalse%
\ A\ \isakeyword{where}\ {\isachardoublequoteopen}A\ {\isasymequiv}\ least{\isacharunderscore}{\kern0pt}indexes{\isacharunderscore}{\kern0pt}of{\isacharunderscore}{\kern0pt}Vset{\isacharunderscore}{\kern0pt}contains{\isacharunderscore}{\kern0pt}witness{\isacharparenleft}{\kern0pt}{\isasymphi}{\isacharcomma}{\kern0pt}\ env{\isacharcomma}{\kern0pt}\ x{\isacharparenright}{\kern0pt}{\isachardoublequoteclose}\ \isanewline
\ \ \isacommand{define}\isamarkupfalse%
\ contains{\isacharunderscore}{\kern0pt}witness\ \isakeyword{where}\ {\isachardoublequoteopen}contains{\isacharunderscore}{\kern0pt}witness\ {\isasymequiv}\ {\isasymlambda}y\ p\ a{\isachardot}{\kern0pt}\ a\ {\isasymin}\ M\ {\isasymand}\ Ord{\isacharparenleft}{\kern0pt}a{\isacharparenright}{\kern0pt}\ {\isasymand}\ {\isacharparenleft}{\kern0pt}{\isasymexists}z\ {\isasymin}\ MVset{\isacharparenleft}{\kern0pt}a{\isacharparenright}{\kern0pt}\ {\isasyminter}\ HS{\isachardot}{\kern0pt}\ p\ {\isasymtturnstile}HS\ {\isasymphi}\ {\isacharparenleft}{\kern0pt}{\isacharbrackleft}{\kern0pt}y{\isacharcomma}{\kern0pt}\ z{\isacharbrackright}{\kern0pt}\ {\isacharat}{\kern0pt}\ env{\isacharparenright}{\kern0pt}{\isacharparenright}{\kern0pt}{\isachardoublequoteclose}\ \isanewline
\isanewline
\ \ \isacommand{have}\isamarkupfalse%
\ contains{\isacharunderscore}{\kern0pt}witness{\isacharunderscore}{\kern0pt}mono\ {\isacharcolon}{\kern0pt}\ \isanewline
\ \ \ \ {\isachardoublequoteopen}{\isasymAnd}a\ b\ y\ p{\isachardot}{\kern0pt}\ Ord{\isacharparenleft}{\kern0pt}a{\isacharparenright}{\kern0pt}\ {\isasymLongrightarrow}\ Ord{\isacharparenleft}{\kern0pt}b{\isacharparenright}{\kern0pt}\ {\isasymLongrightarrow}\ b\ {\isasymin}\ M\ {\isasymLongrightarrow}\ a\ {\isasymle}\ b\ {\isasymLongrightarrow}\ contains{\isacharunderscore}{\kern0pt}witness{\isacharparenleft}{\kern0pt}y{\isacharcomma}{\kern0pt}\ p{\isacharcomma}{\kern0pt}\ a{\isacharparenright}{\kern0pt}\ {\isasymLongrightarrow}\ contains{\isacharunderscore}{\kern0pt}witness{\isacharparenleft}{\kern0pt}y{\isacharcomma}{\kern0pt}\ p{\isacharcomma}{\kern0pt}\ b{\isacharparenright}{\kern0pt}{\isachardoublequoteclose}\ \isanewline
\ \ \ \ \isacommand{unfolding}\isamarkupfalse%
\ contains{\isacharunderscore}{\kern0pt}witness{\isacharunderscore}{\kern0pt}def\ MVset{\isacharunderscore}{\kern0pt}def\ \isanewline
\ \ \ \ \isacommand{apply}\isamarkupfalse%
{\isacharparenleft}{\kern0pt}rename{\isacharunderscore}{\kern0pt}tac\ a\ b\ y\ p{\isacharcomma}{\kern0pt}\ subgoal{\isacharunderscore}{\kern0pt}tac\ {\isachardoublequoteopen}Vset{\isacharparenleft}{\kern0pt}a{\isacharparenright}{\kern0pt}\ {\isasymsubseteq}\ Vset{\isacharparenleft}{\kern0pt}b{\isacharparenright}{\kern0pt}{\isachardoublequoteclose}{\isacharparenright}{\kern0pt}\isanewline
\ \ \ \ \ \isacommand{apply}\isamarkupfalse%
\ force\ \isanewline
\ \ \ \ \isacommand{apply}\isamarkupfalse%
{\isacharparenleft}{\kern0pt}rule\ subsetI{\isacharparenright}{\kern0pt}\isanewline
\ \ \ \ \isacommand{apply}\isamarkupfalse%
{\isacharparenleft}{\kern0pt}rename{\isacharunderscore}{\kern0pt}tac\ a\ b\ y\ p\ x{\isacharcomma}{\kern0pt}\ subgoal{\isacharunderscore}{\kern0pt}tac\ {\isachardoublequoteopen}rank{\isacharparenleft}{\kern0pt}x{\isacharparenright}{\kern0pt}\ {\isacharless}{\kern0pt}\ a{\isachardoublequoteclose}{\isacharparenright}{\kern0pt}\isanewline
\ \ \ \ \isacommand{apply}\isamarkupfalse%
{\isacharparenleft}{\kern0pt}rule\ VsetI{\isacharparenright}{\kern0pt}\isanewline
\ \ \ \ \ \isacommand{apply}\isamarkupfalse%
{\isacharparenleft}{\kern0pt}rename{\isacharunderscore}{\kern0pt}tac\ a\ b\ y\ p\ x{\isacharcomma}{\kern0pt}\ rule{\isacharunderscore}{\kern0pt}tac\ b{\isacharequal}{\kern0pt}a\ \isakeyword{in}\ lt{\isacharunderscore}{\kern0pt}le{\isacharunderscore}{\kern0pt}lt{\isacharcomma}{\kern0pt}\ simp{\isacharcomma}{\kern0pt}\ simp{\isacharparenright}{\kern0pt}\isanewline
\ \ \ \ \isacommand{using}\isamarkupfalse%
\ Vset{\isacharunderscore}{\kern0pt}Ord{\isacharunderscore}{\kern0pt}rank{\isacharunderscore}{\kern0pt}iff\isanewline
\ \ \ \ \isacommand{by}\isamarkupfalse%
\ auto\isanewline
\isanewline
\ \ \isacommand{have}\isamarkupfalse%
\ Aeq{\isacharcolon}{\kern0pt}\ {\isachardoublequoteopen}A\ {\isacharequal}{\kern0pt}\ {\isacharbraceleft}{\kern0pt}\ {\isacharless}{\kern0pt}{\isacharless}{\kern0pt}y{\isacharcomma}{\kern0pt}\ p{\isachargreater}{\kern0pt}{\isacharcomma}{\kern0pt}\ {\isacharparenleft}{\kern0pt}{\isasymmu}\ a{\isachardot}{\kern0pt}\ contains{\isacharunderscore}{\kern0pt}witness{\isacharparenleft}{\kern0pt}y{\isacharcomma}{\kern0pt}\ p{\isacharcomma}{\kern0pt}\ a{\isacharparenright}{\kern0pt}{\isacharparenright}{\kern0pt}{\isachargreater}{\kern0pt}\ {\isachardot}{\kern0pt}\ {\isacharless}{\kern0pt}y{\isacharcomma}{\kern0pt}\ p{\isachargreater}{\kern0pt}\ {\isasymin}\ domain{\isacharparenleft}{\kern0pt}x{\isacharparenright}{\kern0pt}\ {\isasymtimes}\ P\ {\isacharbraceright}{\kern0pt}{\isachardoublequoteclose}\ \isanewline
\ \ \ \ \isacommand{unfolding}\isamarkupfalse%
\ contains{\isacharunderscore}{\kern0pt}witness{\isacharunderscore}{\kern0pt}def\ A{\isacharunderscore}{\kern0pt}def\ least{\isacharunderscore}{\kern0pt}indexes{\isacharunderscore}{\kern0pt}of{\isacharunderscore}{\kern0pt}Vset{\isacharunderscore}{\kern0pt}contains{\isacharunderscore}{\kern0pt}witness{\isacharunderscore}{\kern0pt}def\isanewline
\ \ \ \ \isacommand{by}\isamarkupfalse%
\ auto\isanewline
\isanewline
\ \ \isacommand{have}\isamarkupfalse%
\ {\isachardoublequoteopen}A\ {\isasymin}\ M{\isachardoublequoteclose}\ \isanewline
\ \ \ \ \isacommand{unfolding}\isamarkupfalse%
\ A{\isacharunderscore}{\kern0pt}def\ \isanewline
\ \ \ \ \isacommand{apply}\isamarkupfalse%
{\isacharparenleft}{\kern0pt}rule\ least{\isacharunderscore}{\kern0pt}indexes{\isacharunderscore}{\kern0pt}of{\isacharunderscore}{\kern0pt}Vset{\isacharunderscore}{\kern0pt}contains{\isacharunderscore}{\kern0pt}witness{\isacharunderscore}{\kern0pt}in{\isacharunderscore}{\kern0pt}M{\isacharparenright}{\kern0pt}\isanewline
\ \ \ \ \isacommand{using}\isamarkupfalse%
\ assms\isanewline
\ \ \ \ \isacommand{by}\isamarkupfalse%
\ auto\isanewline
\isanewline
\ \ \isacommand{then}\isamarkupfalse%
\ \isacommand{have}\isamarkupfalse%
\ {\isachardoublequoteopen}range{\isacharparenleft}{\kern0pt}A{\isacharparenright}{\kern0pt}\ {\isasymin}\ M{\isachardoublequoteclose}\ \isanewline
\ \ \ \ \isacommand{using}\isamarkupfalse%
\ range{\isacharunderscore}{\kern0pt}closed\ \isanewline
\ \ \ \ \isacommand{by}\isamarkupfalse%
\ auto\isanewline
\isanewline
\ \ \isacommand{define}\isamarkupfalse%
\ {\isasymalpha}\ \isakeyword{where}\ {\isachardoublequoteopen}{\isasymalpha}\ {\isasymequiv}\ {\isasymUnion}range{\isacharparenleft}{\kern0pt}A{\isacharparenright}{\kern0pt}{\isachardoublequoteclose}\ \isanewline
\isanewline
\ \ \isacommand{have}\isamarkupfalse%
\ {\isachardoublequoteopen}{\isasymAnd}a{\isachardot}{\kern0pt}\ a\ {\isasymin}\ range{\isacharparenleft}{\kern0pt}A{\isacharparenright}{\kern0pt}\ {\isasymLongrightarrow}\ Ord{\isacharparenleft}{\kern0pt}a{\isacharparenright}{\kern0pt}{\isachardoublequoteclose}\ \isanewline
\ \ \ \ \isacommand{unfolding}\isamarkupfalse%
\ A{\isacharunderscore}{\kern0pt}def\ least{\isacharunderscore}{\kern0pt}indexes{\isacharunderscore}{\kern0pt}of{\isacharunderscore}{\kern0pt}Vset{\isacharunderscore}{\kern0pt}contains{\isacharunderscore}{\kern0pt}witness{\isacharunderscore}{\kern0pt}def\isanewline
\ \ \ \ \isacommand{by}\isamarkupfalse%
\ auto\isanewline
\isanewline
\ \ \isacommand{then}\isamarkupfalse%
\ \isacommand{have}\isamarkupfalse%
\ {\isachardoublequoteopen}Ord{\isacharparenleft}{\kern0pt}{\isasymalpha}{\isacharparenright}{\kern0pt}{\isachardoublequoteclose}\ \isanewline
\ \ \ \ \isacommand{using}\isamarkupfalse%
\ Ord{\isacharunderscore}{\kern0pt}Union\ {\isasymalpha}{\isacharunderscore}{\kern0pt}def\isanewline
\ \ \ \ \isacommand{by}\isamarkupfalse%
\ simp\isanewline
\isanewline
\ \ \isacommand{have}\isamarkupfalse%
\ {\isachardoublequoteopen}{\isasymalpha}\ {\isasymin}\ M{\isachardoublequoteclose}\ \isacommand{using}\isamarkupfalse%
\ {\isacartoucheopen}range{\isacharparenleft}{\kern0pt}A{\isacharparenright}{\kern0pt}\ {\isasymin}\ M{\isacartoucheclose}\ Union{\isacharunderscore}{\kern0pt}closed\ {\isasymalpha}{\isacharunderscore}{\kern0pt}def\ \isacommand{by}\isamarkupfalse%
\ auto\isanewline
\isanewline
\ \ \isacommand{have}\isamarkupfalse%
\ H\ {\isacharcolon}{\kern0pt}\ {\isachardoublequoteopen}{\isasymAnd}y\ p{\isachardot}{\kern0pt}\ y\ {\isasymin}\ domain{\isacharparenleft}{\kern0pt}x{\isacharparenright}{\kern0pt}\ {\isasymLongrightarrow}\ p\ {\isasymin}\ P\ {\isasymLongrightarrow}\ {\isacharparenleft}{\kern0pt}{\isasymexists}z\ {\isasymin}\ HS{\isachardot}{\kern0pt}\ p\ {\isasymtturnstile}HS\ {\isasymphi}\ {\isacharparenleft}{\kern0pt}{\isacharbrackleft}{\kern0pt}y{\isacharcomma}{\kern0pt}\ z{\isacharbrackright}{\kern0pt}\ {\isacharat}{\kern0pt}\ env{\isacharparenright}{\kern0pt}{\isacharparenright}{\kern0pt}\ {\isasymLongrightarrow}\ {\isacharparenleft}{\kern0pt}{\isasymexists}z\ {\isasymin}\ MVset{\isacharparenleft}{\kern0pt}{\isasymalpha}{\isacharparenright}{\kern0pt}\ {\isasyminter}\ HS{\isachardot}{\kern0pt}\ p\ {\isasymtturnstile}HS\ {\isasymphi}\ {\isacharparenleft}{\kern0pt}{\isacharbrackleft}{\kern0pt}y{\isacharcomma}{\kern0pt}\ z{\isacharbrackright}{\kern0pt}\ {\isacharat}{\kern0pt}\ env{\isacharparenright}{\kern0pt}{\isacharparenright}{\kern0pt}{\isachardoublequoteclose}\isanewline
\ \ \isacommand{proof}\isamarkupfalse%
\ {\isacharminus}{\kern0pt}\ \isanewline
\ \ \ \ \isacommand{fix}\isamarkupfalse%
\ y\ p\ \isanewline
\ \ \ \ \isacommand{assume}\isamarkupfalse%
\ assms{\isadigit{1}}\ {\isacharcolon}{\kern0pt}\ {\isachardoublequoteopen}y\ {\isasymin}\ domain{\isacharparenleft}{\kern0pt}x{\isacharparenright}{\kern0pt}{\isachardoublequoteclose}\ {\isachardoublequoteopen}p\ {\isasymin}\ P{\isachardoublequoteclose}\ {\isachardoublequoteopen}{\isasymexists}z\ {\isasymin}\ HS{\isachardot}{\kern0pt}\ p\ {\isasymtturnstile}HS\ {\isasymphi}\ {\isacharparenleft}{\kern0pt}{\isacharbrackleft}{\kern0pt}y{\isacharcomma}{\kern0pt}\ z{\isacharbrackright}{\kern0pt}\ {\isacharat}{\kern0pt}\ env{\isacharparenright}{\kern0pt}{\isachardoublequoteclose}\ \isanewline
\isanewline
\ \ \ \ \isacommand{then}\isamarkupfalse%
\ \isacommand{obtain}\isamarkupfalse%
\ z\ \isakeyword{where}\ zH\ {\isacharcolon}{\kern0pt}\ {\isachardoublequoteopen}z\ {\isasymin}\ HS{\isachardoublequoteclose}\ {\isachardoublequoteopen}p\ {\isasymtturnstile}HS\ {\isasymphi}\ {\isacharparenleft}{\kern0pt}{\isacharbrackleft}{\kern0pt}y{\isacharcomma}{\kern0pt}\ z{\isacharbrackright}{\kern0pt}\ {\isacharat}{\kern0pt}\ env{\isacharparenright}{\kern0pt}{\isachardoublequoteclose}\ \isacommand{by}\isamarkupfalse%
\ auto\isanewline
\ \ \ \ \isacommand{have}\isamarkupfalse%
\ {\isachardoublequoteopen}contains{\isacharunderscore}{\kern0pt}witness{\isacharparenleft}{\kern0pt}y{\isacharcomma}{\kern0pt}\ p{\isacharcomma}{\kern0pt}\ succ{\isacharparenleft}{\kern0pt}rank{\isacharparenleft}{\kern0pt}z{\isacharparenright}{\kern0pt}{\isacharparenright}{\kern0pt}{\isacharparenright}{\kern0pt}{\isachardoublequoteclose}\ \isanewline
\ \ \ \ \ \ \isacommand{unfolding}\isamarkupfalse%
\ contains{\isacharunderscore}{\kern0pt}witness{\isacharunderscore}{\kern0pt}def\ \isanewline
\ \ \ \ \ \ \isacommand{apply}\isamarkupfalse%
{\isacharparenleft}{\kern0pt}insert\ HS{\isacharunderscore}{\kern0pt}iff\ P{\isacharunderscore}{\kern0pt}name{\isacharunderscore}{\kern0pt}in{\isacharunderscore}{\kern0pt}M\ rank{\isacharunderscore}{\kern0pt}closed\ succ{\isacharunderscore}{\kern0pt}in{\isacharunderscore}{\kern0pt}MI\ Ord{\isacharunderscore}{\kern0pt}rank\ zH{\isacharparenright}{\kern0pt}\isanewline
\ \ \ \ \ \ \isacommand{apply}\isamarkupfalse%
{\isacharparenleft}{\kern0pt}rule{\isacharunderscore}{\kern0pt}tac\ conjI{\isacharcomma}{\kern0pt}\ force{\isacharparenright}{\kern0pt}{\isacharplus}{\kern0pt}\isanewline
\ \ \ \ \ \ \isacommand{apply}\isamarkupfalse%
{\isacharparenleft}{\kern0pt}rule{\isacharunderscore}{\kern0pt}tac\ x{\isacharequal}{\kern0pt}z\ \isakeyword{in}\ bexI{\isacharparenright}{\kern0pt}\isanewline
\ \ \ \ \ \ \ \isacommand{apply}\isamarkupfalse%
\ simp\ \isanewline
\ \ \ \ \ \ \isacommand{apply}\isamarkupfalse%
\ simp\isanewline
\ \ \ \ \ \ \isacommand{apply}\isamarkupfalse%
{\isacharparenleft}{\kern0pt}rule\ MVsetI{\isacharparenright}{\kern0pt}\isanewline
\ \ \ \ \ \ \isacommand{by}\isamarkupfalse%
\ auto\isanewline
\ \ \ \ \isacommand{then}\isamarkupfalse%
\ \isacommand{have}\isamarkupfalse%
\ H{\isacharcolon}{\kern0pt}\ {\isachardoublequoteopen}contains{\isacharunderscore}{\kern0pt}witness{\isacharparenleft}{\kern0pt}y{\isacharcomma}{\kern0pt}\ p{\isacharcomma}{\kern0pt}\ {\isasymmu}\ a{\isachardot}{\kern0pt}\ contains{\isacharunderscore}{\kern0pt}witness{\isacharparenleft}{\kern0pt}y{\isacharcomma}{\kern0pt}\ p{\isacharcomma}{\kern0pt}\ a{\isacharparenright}{\kern0pt}{\isacharparenright}{\kern0pt}{\isachardoublequoteclose}\ \isanewline
\ \ \ \ \ \ \isacommand{apply}\isamarkupfalse%
{\isacharparenleft}{\kern0pt}rule{\isacharunderscore}{\kern0pt}tac\ LeastI{\isacharparenright}{\kern0pt}\isanewline
\ \ \ \ \ \ \isacommand{using}\isamarkupfalse%
\ Ord{\isacharunderscore}{\kern0pt}rank\isanewline
\ \ \ \ \ \ \isacommand{by}\isamarkupfalse%
\ auto\isanewline
\isanewline
\ \ \ \ \isacommand{have}\isamarkupfalse%
\ {\isachardoublequoteopen}{\isacharless}{\kern0pt}{\isacharless}{\kern0pt}y{\isacharcomma}{\kern0pt}\ p{\isachargreater}{\kern0pt}{\isacharcomma}{\kern0pt}\ {\isacharparenleft}{\kern0pt}{\isasymmu}\ a{\isachardot}{\kern0pt}\ contains{\isacharunderscore}{\kern0pt}witness{\isacharparenleft}{\kern0pt}y{\isacharcomma}{\kern0pt}\ p{\isacharcomma}{\kern0pt}\ a{\isacharparenright}{\kern0pt}{\isacharparenright}{\kern0pt}{\isachargreater}{\kern0pt}\ {\isasymin}\ A{\isachardoublequoteclose}\ \isanewline
\ \ \ \ \ \ \isacommand{unfolding}\isamarkupfalse%
\ A{\isacharunderscore}{\kern0pt}def\ contains{\isacharunderscore}{\kern0pt}witness{\isacharunderscore}{\kern0pt}def\ least{\isacharunderscore}{\kern0pt}indexes{\isacharunderscore}{\kern0pt}of{\isacharunderscore}{\kern0pt}Vset{\isacharunderscore}{\kern0pt}contains{\isacharunderscore}{\kern0pt}witness{\isacharunderscore}{\kern0pt}def\isanewline
\ \ \ \ \ \ \isacommand{using}\isamarkupfalse%
\ assms{\isadigit{1}}\ \isanewline
\ \ \ \ \ \ \isacommand{by}\isamarkupfalse%
\ auto\isanewline
\ \ \ \ \isacommand{then}\isamarkupfalse%
\ \isacommand{have}\isamarkupfalse%
\ {\isachardoublequoteopen}{\isacharparenleft}{\kern0pt}{\isasymmu}\ a{\isachardot}{\kern0pt}\ contains{\isacharunderscore}{\kern0pt}witness{\isacharparenleft}{\kern0pt}y{\isacharcomma}{\kern0pt}\ p{\isacharcomma}{\kern0pt}\ a{\isacharparenright}{\kern0pt}{\isacharparenright}{\kern0pt}\ {\isasymin}\ range{\isacharparenleft}{\kern0pt}A{\isacharparenright}{\kern0pt}{\isachardoublequoteclose}\ \isacommand{by}\isamarkupfalse%
\ auto\ \isanewline
\ \ \ \ \isacommand{then}\isamarkupfalse%
\ \isacommand{have}\isamarkupfalse%
\ {\isachardoublequoteopen}{\isacharparenleft}{\kern0pt}{\isasymmu}\ a{\isachardot}{\kern0pt}\ contains{\isacharunderscore}{\kern0pt}witness{\isacharparenleft}{\kern0pt}y{\isacharcomma}{\kern0pt}\ p{\isacharcomma}{\kern0pt}\ a{\isacharparenright}{\kern0pt}{\isacharparenright}{\kern0pt}\ {\isasymle}\ {\isasymalpha}{\isachardoublequoteclose}\ \isanewline
\ \ \ \ \ \ \isacommand{using}\isamarkupfalse%
\ {\isacartoucheopen}Ord{\isacharparenleft}{\kern0pt}{\isasymalpha}{\isacharparenright}{\kern0pt}{\isacartoucheclose}\isanewline
\ \ \ \ \ \ \isacommand{unfolding}\isamarkupfalse%
\ {\isasymalpha}{\isacharunderscore}{\kern0pt}def\isanewline
\ \ \ \ \ \ \isacommand{apply}\isamarkupfalse%
{\isacharparenleft}{\kern0pt}rule{\isacharunderscore}{\kern0pt}tac\ Union{\isacharunderscore}{\kern0pt}upper{\isacharunderscore}{\kern0pt}le{\isacharparenright}{\kern0pt}\isanewline
\ \ \ \ \ \ \isacommand{by}\isamarkupfalse%
\ auto\isanewline
\ \ \ \ \isacommand{then}\isamarkupfalse%
\ \isacommand{have}\isamarkupfalse%
\ H{\isacharprime}{\kern0pt}\ {\isacharcolon}{\kern0pt}\ {\isachardoublequoteopen}contains{\isacharunderscore}{\kern0pt}witness{\isacharparenleft}{\kern0pt}y{\isacharcomma}{\kern0pt}\ p{\isacharcomma}{\kern0pt}\ {\isasymalpha}{\isacharparenright}{\kern0pt}{\isachardoublequoteclose}\ \isanewline
\ \ \ \ \ \ \isacommand{apply}\isamarkupfalse%
{\isacharparenleft}{\kern0pt}rule{\isacharunderscore}{\kern0pt}tac\ contains{\isacharunderscore}{\kern0pt}witness{\isacharunderscore}{\kern0pt}mono{\isacharbrackleft}{\kern0pt}of\ {\isachardoublequoteopen}{\isacharparenleft}{\kern0pt}{\isasymmu}\ a{\isachardot}{\kern0pt}\ contains{\isacharunderscore}{\kern0pt}witness{\isacharparenleft}{\kern0pt}y{\isacharcomma}{\kern0pt}\ p{\isacharcomma}{\kern0pt}\ a{\isacharparenright}{\kern0pt}{\isacharparenright}{\kern0pt}{\isachardoublequoteclose}{\isacharbrackright}{\kern0pt}{\isacharparenright}{\kern0pt}\isanewline
\ \ \ \ \ \ \isacommand{using}\isamarkupfalse%
\ {\isacartoucheopen}Ord{\isacharparenleft}{\kern0pt}{\isasymalpha}{\isacharparenright}{\kern0pt}{\isacartoucheclose}\ {\isacartoucheopen}{\isasymalpha}\ {\isasymin}\ M{\isacartoucheclose}\ H\isanewline
\ \ \ \ \ \ \isacommand{by}\isamarkupfalse%
\ auto\isanewline
\isanewline
\ \ \ \ \isacommand{then}\isamarkupfalse%
\ \isacommand{show}\isamarkupfalse%
\ {\isachardoublequoteopen}{\isacharparenleft}{\kern0pt}{\isasymexists}z\ {\isasymin}\ MVset{\isacharparenleft}{\kern0pt}{\isasymalpha}{\isacharparenright}{\kern0pt}\ {\isasyminter}\ HS{\isachardot}{\kern0pt}\ p\ {\isasymtturnstile}HS\ {\isasymphi}\ {\isacharparenleft}{\kern0pt}{\isacharbrackleft}{\kern0pt}y{\isacharcomma}{\kern0pt}\ z{\isacharbrackright}{\kern0pt}\ {\isacharat}{\kern0pt}\ env{\isacharparenright}{\kern0pt}{\isacharparenright}{\kern0pt}{\isachardoublequoteclose}\ \isanewline
\ \ \ \ \ \ \isacommand{unfolding}\isamarkupfalse%
\ contains{\isacharunderscore}{\kern0pt}witness{\isacharunderscore}{\kern0pt}def\ \isanewline
\ \ \ \ \ \ \isacommand{by}\isamarkupfalse%
\ auto\isanewline
\ \ \isacommand{qed}\isamarkupfalse%
\isanewline
\isanewline
\ \ \isacommand{show}\isamarkupfalse%
\ {\isacharquery}{\kern0pt}thesis\ \isanewline
\ \ \ \ \isacommand{apply}\isamarkupfalse%
{\isacharparenleft}{\kern0pt}rule{\isacharunderscore}{\kern0pt}tac\ x{\isacharequal}{\kern0pt}{\isachardoublequoteopen}MVset{\isacharparenleft}{\kern0pt}{\isasymalpha}{\isacharparenright}{\kern0pt}\ {\isasyminter}\ HS{\isachardoublequoteclose}\ \isakeyword{in}\ exI{\isacharparenright}{\kern0pt}\isanewline
\ \ \ \ \isacommand{apply}\isamarkupfalse%
{\isacharparenleft}{\kern0pt}rule\ conjI{\isacharcomma}{\kern0pt}\ rule\ HS{\isacharunderscore}{\kern0pt}separation{\isacharparenright}{\kern0pt}\isanewline
\ \ \ \ \isacommand{apply}\isamarkupfalse%
{\isacharparenleft}{\kern0pt}rule\ MVset{\isacharunderscore}{\kern0pt}in{\isacharunderscore}{\kern0pt}M{\isacharparenright}{\kern0pt}\isanewline
\ \ \ \ \isacommand{using}\isamarkupfalse%
\ {\isacartoucheopen}Ord{\isacharparenleft}{\kern0pt}{\isasymalpha}{\isacharparenright}{\kern0pt}{\isacartoucheclose}\ {\isacartoucheopen}{\isasymalpha}\ {\isasymin}\ M{\isacartoucheclose}\isanewline
\ \ \ \ \ \ \isacommand{apply}\isamarkupfalse%
\ auto{\isacharbrackleft}{\kern0pt}{\isadigit{2}}{\isacharbrackright}{\kern0pt}\isanewline
\ \ \ \ \isacommand{apply}\isamarkupfalse%
{\isacharparenleft}{\kern0pt}rule\ conjI{\isacharcomma}{\kern0pt}\ force{\isacharparenright}{\kern0pt}\isanewline
\ \ \ \ \isacommand{apply}\isamarkupfalse%
{\isacharparenleft}{\kern0pt}rule\ ballI{\isacharparenright}{\kern0pt}{\isacharplus}{\kern0pt}\isanewline
\ \ \ \ \isacommand{apply}\isamarkupfalse%
{\isacharparenleft}{\kern0pt}rule\ iffI{\isacharparenright}{\kern0pt}\isanewline
\ \ \ \ \ \isacommand{apply}\isamarkupfalse%
{\isacharparenleft}{\kern0pt}rule\ H{\isacharparenright}{\kern0pt}\isanewline
\ \ \ \ \isacommand{using}\isamarkupfalse%
\ generic\ M{\isacharunderscore}{\kern0pt}genericD\ \isanewline
\ \ \ \ \isacommand{by}\isamarkupfalse%
\ auto\isanewline
\isacommand{qed}\isamarkupfalse%
%
\endisatagproof
{\isafoldproof}%
%
\isadelimproof
\isanewline
%
\endisadelimproof
\isanewline
\isacommand{lemma}\isamarkupfalse%
\ ex{\isacharunderscore}{\kern0pt}SymExt{\isacharunderscore}{\kern0pt}elem{\isacharunderscore}{\kern0pt}contains{\isacharunderscore}{\kern0pt}witnesses\ {\isacharcolon}{\kern0pt}\ \isanewline
\ \ \isakeyword{fixes}\ {\isasymphi}\ env\ x\ \isanewline
\ \ \isakeyword{assumes}\ {\isachardoublequoteopen}{\isasymphi}\ {\isasymin}\ formula{\isachardoublequoteclose}\ {\isachardoublequoteopen}env\ {\isasymin}\ list{\isacharparenleft}{\kern0pt}SymExt{\isacharparenleft}{\kern0pt}G{\isacharparenright}{\kern0pt}{\isacharparenright}{\kern0pt}{\isachardoublequoteclose}\ {\isachardoublequoteopen}arity{\isacharparenleft}{\kern0pt}{\isasymphi}{\isacharparenright}{\kern0pt}\ {\isasymle}\ {\isadigit{2}}\ {\isacharhash}{\kern0pt}{\isacharplus}{\kern0pt}\ length{\isacharparenleft}{\kern0pt}env{\isacharparenright}{\kern0pt}{\isachardoublequoteclose}\ {\isachardoublequoteopen}x\ {\isasymin}\ SymExt{\isacharparenleft}{\kern0pt}G{\isacharparenright}{\kern0pt}{\isachardoublequoteclose}\ \isanewline
\ \ \isakeyword{shows}\ {\isachardoublequoteopen}{\isasymexists}S\ {\isasymin}\ SymExt{\isacharparenleft}{\kern0pt}G{\isacharparenright}{\kern0pt}{\isachardot}{\kern0pt}\ {\isasymforall}y\ {\isasymin}\ x{\isachardot}{\kern0pt}\ {\isacharparenleft}{\kern0pt}{\isacharparenleft}{\kern0pt}{\isasymexists}z\ {\isasymin}\ SymExt{\isacharparenleft}{\kern0pt}G{\isacharparenright}{\kern0pt}{\isachardot}{\kern0pt}\ sats{\isacharparenleft}{\kern0pt}SymExt{\isacharparenleft}{\kern0pt}G{\isacharparenright}{\kern0pt}{\isacharcomma}{\kern0pt}\ {\isasymphi}{\isacharcomma}{\kern0pt}\ {\isacharbrackleft}{\kern0pt}y{\isacharcomma}{\kern0pt}\ z{\isacharbrackright}{\kern0pt}\ {\isacharat}{\kern0pt}\ env{\isacharparenright}{\kern0pt}{\isacharparenright}{\kern0pt}\ {\isasymlongleftrightarrow}\ {\isacharparenleft}{\kern0pt}{\isasymexists}z\ {\isasymin}\ S{\isachardot}{\kern0pt}\ sats{\isacharparenleft}{\kern0pt}SymExt{\isacharparenleft}{\kern0pt}G{\isacharparenright}{\kern0pt}{\isacharcomma}{\kern0pt}\ {\isasymphi}{\isacharcomma}{\kern0pt}\ {\isacharbrackleft}{\kern0pt}y{\isacharcomma}{\kern0pt}\ z{\isacharbrackright}{\kern0pt}\ {\isacharat}{\kern0pt}\ env{\isacharparenright}{\kern0pt}{\isacharparenright}{\kern0pt}{\isacharparenright}{\kern0pt}{\isachardoublequoteclose}\isanewline
%
\isadelimproof
%
\endisadelimproof
%
\isatagproof
\isacommand{proof}\isamarkupfalse%
\ {\isacharminus}{\kern0pt}\ \isanewline
\ \ \isacommand{obtain}\isamarkupfalse%
\ x{\isacharprime}{\kern0pt}\ \isakeyword{where}\ x{\isacharprime}{\kern0pt}H\ {\isacharcolon}{\kern0pt}\ {\isachardoublequoteopen}x{\isacharprime}{\kern0pt}\ {\isasymin}\ HS{\isachardoublequoteclose}\ {\isachardoublequoteopen}val{\isacharparenleft}{\kern0pt}G{\isacharcomma}{\kern0pt}\ x{\isacharprime}{\kern0pt}{\isacharparenright}{\kern0pt}\ {\isacharequal}{\kern0pt}\ x{\isachardoublequoteclose}\ \isacommand{using}\isamarkupfalse%
\ SymExt{\isacharunderscore}{\kern0pt}def\ assms\ \isacommand{by}\isamarkupfalse%
\ auto\isanewline
\ \ \isacommand{have}\isamarkupfalse%
\ {\isachardoublequoteopen}{\isasymexists}env{\isacharprime}{\kern0pt}\ {\isasymin}\ list{\isacharparenleft}{\kern0pt}HS{\isacharparenright}{\kern0pt}{\isachardot}{\kern0pt}\ map{\isacharparenleft}{\kern0pt}val{\isacharparenleft}{\kern0pt}G{\isacharparenright}{\kern0pt}{\isacharcomma}{\kern0pt}\ env{\isacharprime}{\kern0pt}{\isacharparenright}{\kern0pt}\ {\isacharequal}{\kern0pt}\ env{\isachardoublequoteclose}\ \isanewline
\ \ \ \ \isacommand{apply}\isamarkupfalse%
{\isacharparenleft}{\kern0pt}rule\ ex{\isacharunderscore}{\kern0pt}HS{\isacharunderscore}{\kern0pt}name{\isacharunderscore}{\kern0pt}list{\isacharparenright}{\kern0pt}\isanewline
\ \ \ \ \isacommand{using}\isamarkupfalse%
\ assms\isanewline
\ \ \ \ \isacommand{by}\isamarkupfalse%
\ auto\isanewline
\ \ \isacommand{then}\isamarkupfalse%
\ \isacommand{obtain}\isamarkupfalse%
\ env{\isacharprime}{\kern0pt}\ \isakeyword{where}\ env{\isacharprime}{\kern0pt}H{\isacharcolon}{\kern0pt}\ {\isachardoublequoteopen}env{\isacharprime}{\kern0pt}\ {\isasymin}\ list{\isacharparenleft}{\kern0pt}HS{\isacharparenright}{\kern0pt}{\isachardoublequoteclose}\ {\isachardoublequoteopen}map{\isacharparenleft}{\kern0pt}val{\isacharparenleft}{\kern0pt}G{\isacharparenright}{\kern0pt}{\isacharcomma}{\kern0pt}\ env{\isacharprime}{\kern0pt}{\isacharparenright}{\kern0pt}\ {\isacharequal}{\kern0pt}\ env{\isachardoublequoteclose}\ \isacommand{using}\isamarkupfalse%
\ assms\ \isacommand{by}\isamarkupfalse%
\ auto\isanewline
\ \ \isacommand{then}\isamarkupfalse%
\ \isacommand{have}\isamarkupfalse%
\ leneq\ {\isacharcolon}{\kern0pt}\ {\isachardoublequoteopen}length{\isacharparenleft}{\kern0pt}env{\isacharprime}{\kern0pt}{\isacharparenright}{\kern0pt}\ {\isacharequal}{\kern0pt}\ length{\isacharparenleft}{\kern0pt}env{\isacharparenright}{\kern0pt}{\isachardoublequoteclose}\ \isacommand{by}\isamarkupfalse%
\ auto\isanewline
\ \ \isacommand{have}\isamarkupfalse%
\ env{\isacharprime}{\kern0pt}in\ {\isacharcolon}{\kern0pt}\ {\isachardoublequoteopen}env{\isacharprime}{\kern0pt}\ {\isasymin}\ list{\isacharparenleft}{\kern0pt}M{\isacharparenright}{\kern0pt}{\isachardoublequoteclose}\ \isanewline
\ \ \ \ \isacommand{apply}\isamarkupfalse%
{\isacharparenleft}{\kern0pt}rule{\isacharunderscore}{\kern0pt}tac\ A{\isacharequal}{\kern0pt}{\isachardoublequoteopen}list{\isacharparenleft}{\kern0pt}HS{\isacharparenright}{\kern0pt}{\isachardoublequoteclose}\ \isakeyword{in}\ subsetD{\isacharcomma}{\kern0pt}\ rule\ list{\isacharunderscore}{\kern0pt}mono{\isacharparenright}{\kern0pt}\isanewline
\ \ \ \ \isacommand{using}\isamarkupfalse%
\ HS{\isacharunderscore}{\kern0pt}iff\ P{\isacharunderscore}{\kern0pt}name{\isacharunderscore}{\kern0pt}in{\isacharunderscore}{\kern0pt}M\ env{\isacharprime}{\kern0pt}H\isanewline
\ \ \ \ \isacommand{by}\isamarkupfalse%
\ auto\isanewline
\isanewline
\ \ \isacommand{have}\isamarkupfalse%
\ {\isachardoublequoteopen}{\isasymexists}S{\isachardot}{\kern0pt}\ S\ {\isasymin}\ M\ {\isasymand}\ S\ {\isasymsubseteq}\ HS\ {\isasymand}\ {\isacharparenleft}{\kern0pt}{\isasymforall}p\ {\isasymin}\ G{\isachardot}{\kern0pt}\ {\isasymforall}y\ {\isasymin}\ domain{\isacharparenleft}{\kern0pt}x{\isacharprime}{\kern0pt}{\isacharparenright}{\kern0pt}{\isachardot}{\kern0pt}\ {\isacharparenleft}{\kern0pt}{\isasymexists}z\ {\isasymin}\ HS{\isachardot}{\kern0pt}\ p\ {\isasymtturnstile}HS\ {\isasymphi}\ {\isacharparenleft}{\kern0pt}{\isacharbrackleft}{\kern0pt}y{\isacharcomma}{\kern0pt}\ z{\isacharbrackright}{\kern0pt}\ {\isacharat}{\kern0pt}\ env{\isacharprime}{\kern0pt}{\isacharparenright}{\kern0pt}{\isacharparenright}{\kern0pt}\ {\isasymlongleftrightarrow}\ {\isacharparenleft}{\kern0pt}{\isasymexists}z\ {\isasymin}\ S{\isachardot}{\kern0pt}\ p\ {\isasymtturnstile}HS\ {\isasymphi}\ {\isacharparenleft}{\kern0pt}{\isacharbrackleft}{\kern0pt}y{\isacharcomma}{\kern0pt}\ z{\isacharbrackright}{\kern0pt}\ {\isacharat}{\kern0pt}\ env{\isacharprime}{\kern0pt}{\isacharparenright}{\kern0pt}{\isacharparenright}{\kern0pt}{\isacharparenright}{\kern0pt}{\isachardoublequoteclose}\ \isanewline
\ \ \ \ \isacommand{apply}\isamarkupfalse%
{\isacharparenleft}{\kern0pt}rule\ ex{\isacharunderscore}{\kern0pt}hs{\isacharunderscore}{\kern0pt}subset{\isacharunderscore}{\kern0pt}contains{\isacharunderscore}{\kern0pt}witnesses{\isacharparenright}{\kern0pt}\isanewline
\ \ \ \ \isacommand{using}\isamarkupfalse%
\ assms\ env{\isacharprime}{\kern0pt}in\ HS{\isacharunderscore}{\kern0pt}iff\ P{\isacharunderscore}{\kern0pt}name{\isacharunderscore}{\kern0pt}in{\isacharunderscore}{\kern0pt}M\ app{\isacharunderscore}{\kern0pt}type\ x{\isacharprime}{\kern0pt}H\ leneq\isanewline
\ \ \ \ \isacommand{by}\isamarkupfalse%
\ auto\isanewline
\ \ \isacommand{then}\isamarkupfalse%
\ \isacommand{obtain}\isamarkupfalse%
\ S\ \isakeyword{where}\ SH{\isacharcolon}{\kern0pt}\ {\isachardoublequoteopen}S\ {\isasymin}\ M{\isachardoublequoteclose}\ {\isachardoublequoteopen}S\ {\isasymsubseteq}\ HS{\isachardoublequoteclose}\ {\isachardoublequoteopen}{\isacharparenleft}{\kern0pt}{\isasymforall}p\ {\isasymin}\ G{\isachardot}{\kern0pt}\ {\isasymforall}y\ {\isasymin}\ domain{\isacharparenleft}{\kern0pt}x{\isacharprime}{\kern0pt}{\isacharparenright}{\kern0pt}{\isachardot}{\kern0pt}\ {\isacharparenleft}{\kern0pt}{\isasymexists}z\ {\isasymin}\ HS{\isachardot}{\kern0pt}\ p\ {\isasymtturnstile}HS\ {\isasymphi}\ {\isacharparenleft}{\kern0pt}{\isacharbrackleft}{\kern0pt}y{\isacharcomma}{\kern0pt}\ z{\isacharbrackright}{\kern0pt}\ {\isacharat}{\kern0pt}\ env{\isacharprime}{\kern0pt}{\isacharparenright}{\kern0pt}{\isacharparenright}{\kern0pt}\ {\isasymlongleftrightarrow}\ {\isacharparenleft}{\kern0pt}{\isasymexists}z\ {\isasymin}\ S{\isachardot}{\kern0pt}\ p\ {\isasymtturnstile}HS\ {\isasymphi}\ {\isacharparenleft}{\kern0pt}{\isacharbrackleft}{\kern0pt}y{\isacharcomma}{\kern0pt}\ z{\isacharbrackright}{\kern0pt}\ {\isacharat}{\kern0pt}\ env{\isacharprime}{\kern0pt}{\isacharparenright}{\kern0pt}{\isacharparenright}{\kern0pt}{\isacharparenright}{\kern0pt}{\isachardoublequoteclose}\ \isacommand{by}\isamarkupfalse%
\ auto\isanewline
\ \ \isacommand{then}\isamarkupfalse%
\ \isacommand{have}\isamarkupfalse%
\ {\isachardoublequoteopen}{\isasymexists}T\ {\isasymin}\ SymExt{\isacharparenleft}{\kern0pt}G{\isacharparenright}{\kern0pt}{\isachardot}{\kern0pt}\ {\isacharbraceleft}{\kern0pt}\ val{\isacharparenleft}{\kern0pt}G{\isacharcomma}{\kern0pt}\ x{\isacharparenright}{\kern0pt}{\isachardot}{\kern0pt}\ x\ {\isasymin}\ S\ {\isacharbraceright}{\kern0pt}\ {\isasymsubseteq}\ T{\isachardoublequoteclose}\ \isanewline
\ \ \ \ \isacommand{apply}\isamarkupfalse%
{\isacharparenleft}{\kern0pt}rule{\isacharunderscore}{\kern0pt}tac\ ex{\isacharunderscore}{\kern0pt}separation{\isacharunderscore}{\kern0pt}base{\isacharparenright}{\kern0pt}\isanewline
\ \ \ \ \isacommand{by}\isamarkupfalse%
\ auto\isanewline
\ \ \isacommand{then}\isamarkupfalse%
\ \isacommand{obtain}\isamarkupfalse%
\ T\ \isakeyword{where}\ TH{\isacharcolon}{\kern0pt}\ {\isachardoublequoteopen}T\ {\isasymin}\ SymExt{\isacharparenleft}{\kern0pt}G{\isacharparenright}{\kern0pt}{\isachardoublequoteclose}\ {\isachardoublequoteopen}{\isacharbraceleft}{\kern0pt}\ val{\isacharparenleft}{\kern0pt}G{\isacharcomma}{\kern0pt}\ x{\isacharparenright}{\kern0pt}{\isachardot}{\kern0pt}\ x\ {\isasymin}\ S\ {\isacharbraceright}{\kern0pt}\ {\isasymsubseteq}\ T{\isachardoublequoteclose}\ \isacommand{by}\isamarkupfalse%
\ auto\ \isanewline
\isanewline
\ \ \isacommand{show}\isamarkupfalse%
\ {\isacharquery}{\kern0pt}thesis\isanewline
\ \ \isacommand{proof}\isamarkupfalse%
{\isacharparenleft}{\kern0pt}rule{\isacharunderscore}{\kern0pt}tac\ x{\isacharequal}{\kern0pt}T\ \isakeyword{in}\ bexI{\isacharcomma}{\kern0pt}\ rule\ ballI{\isacharparenright}{\kern0pt}\isanewline
\ \ \ \ \isacommand{fix}\isamarkupfalse%
\ y\ \isacommand{assume}\isamarkupfalse%
\ yin\ {\isacharcolon}{\kern0pt}\ {\isachardoublequoteopen}y\ {\isasymin}\ x{\isachardoublequoteclose}\ \isanewline
\ \ \ \ \isacommand{have}\isamarkupfalse%
\ {\isachardoublequoteopen}{\isasymexists}y{\isacharprime}{\kern0pt}\ {\isasymin}\ domain{\isacharparenleft}{\kern0pt}x{\isacharprime}{\kern0pt}{\isacharparenright}{\kern0pt}\ {\isachardot}{\kern0pt}\ {\isasymexists}p\ {\isasymin}\ G{\isachardot}{\kern0pt}\ val{\isacharparenleft}{\kern0pt}G{\isacharcomma}{\kern0pt}\ y{\isacharprime}{\kern0pt}{\isacharparenright}{\kern0pt}\ {\isacharequal}{\kern0pt}\ y\ {\isasymand}\ {\isacharless}{\kern0pt}y{\isacharprime}{\kern0pt}{\isacharcomma}{\kern0pt}\ p{\isachargreater}{\kern0pt}\ {\isasymin}\ x{\isacharprime}{\kern0pt}{\isachardoublequoteclose}\ \isanewline
\ \ \ \ \ \ \isacommand{apply}\isamarkupfalse%
{\isacharparenleft}{\kern0pt}rule{\isacharunderscore}{\kern0pt}tac\ P{\isacharequal}{\kern0pt}{\isachardoublequoteopen}y\ {\isasymin}\ val{\isacharparenleft}{\kern0pt}G{\isacharcomma}{\kern0pt}\ x{\isacharprime}{\kern0pt}{\isacharparenright}{\kern0pt}{\isachardoublequoteclose}\ \isakeyword{in}\ mp{\isacharparenright}{\kern0pt}\isanewline
\ \ \ \ \ \ \ \isacommand{apply}\isamarkupfalse%
{\isacharparenleft}{\kern0pt}subst\ def{\isacharunderscore}{\kern0pt}val{\isacharcomma}{\kern0pt}\ force{\isacharparenright}{\kern0pt}\isanewline
\ \ \ \ \ \ \isacommand{using}\isamarkupfalse%
\ yin\ x{\isacharprime}{\kern0pt}H\ \isanewline
\ \ \ \ \ \ \isacommand{by}\isamarkupfalse%
\ auto\isanewline
\ \ \ \ \isacommand{then}\isamarkupfalse%
\ \isacommand{obtain}\isamarkupfalse%
\ y{\isacharprime}{\kern0pt}\ p\ \isakeyword{where}\ y{\isacharprime}{\kern0pt}pH{\isacharcolon}{\kern0pt}\ {\isachardoublequoteopen}y{\isacharprime}{\kern0pt}\ {\isasymin}\ domain{\isacharparenleft}{\kern0pt}x{\isacharprime}{\kern0pt}{\isacharparenright}{\kern0pt}{\isachardoublequoteclose}\ {\isachardoublequoteopen}val{\isacharparenleft}{\kern0pt}G{\isacharcomma}{\kern0pt}\ y{\isacharprime}{\kern0pt}{\isacharparenright}{\kern0pt}\ {\isacharequal}{\kern0pt}\ y{\isachardoublequoteclose}\ {\isachardoublequoteopen}p\ {\isasymin}\ G{\isachardoublequoteclose}\ \isacommand{by}\isamarkupfalse%
\ auto\isanewline
\ \ \ \ \isacommand{then}\isamarkupfalse%
\ \isacommand{have}\isamarkupfalse%
\ y{\isacharprime}{\kern0pt}inHS\ {\isacharcolon}{\kern0pt}\ {\isachardoublequoteopen}y{\isacharprime}{\kern0pt}\ {\isasymin}\ HS{\isachardoublequoteclose}\ \isacommand{using}\isamarkupfalse%
\ x{\isacharprime}{\kern0pt}H\ HS{\isacharunderscore}{\kern0pt}iff\ \isacommand{by}\isamarkupfalse%
\ auto\isanewline
\isanewline
\ \ \ \ \isacommand{have}\isamarkupfalse%
\ {\isachardoublequoteopen}{\isacharparenleft}{\kern0pt}{\isasymexists}z{\isasymin}SymExt{\isacharparenleft}{\kern0pt}G{\isacharparenright}{\kern0pt}{\isachardot}{\kern0pt}\ SymExt{\isacharparenleft}{\kern0pt}G{\isacharparenright}{\kern0pt}{\isacharcomma}{\kern0pt}\ {\isacharbrackleft}{\kern0pt}y{\isacharcomma}{\kern0pt}\ z{\isacharbrackright}{\kern0pt}\ {\isacharat}{\kern0pt}\ env\ {\isasymTurnstile}\ {\isasymphi}{\isacharparenright}{\kern0pt}\ {\isasymlongleftrightarrow}\ {\isacharparenleft}{\kern0pt}{\isasymexists}z{\isacharprime}{\kern0pt}{\isasymin}HS{\isachardot}{\kern0pt}\ SymExt{\isacharparenleft}{\kern0pt}G{\isacharparenright}{\kern0pt}{\isacharcomma}{\kern0pt}\ map{\isacharparenleft}{\kern0pt}val{\isacharparenleft}{\kern0pt}G{\isacharparenright}{\kern0pt}{\isacharcomma}{\kern0pt}\ {\isacharbrackleft}{\kern0pt}y{\isacharprime}{\kern0pt}{\isacharcomma}{\kern0pt}\ z{\isacharprime}{\kern0pt}{\isacharbrackright}{\kern0pt}\ {\isacharat}{\kern0pt}\ env{\isacharprime}{\kern0pt}{\isacharparenright}{\kern0pt}\ {\isasymTurnstile}\ {\isasymphi}{\isacharparenright}{\kern0pt}{\isachardoublequoteclose}\isanewline
\ \ \ \ \ \ \isacommand{unfolding}\isamarkupfalse%
\ SymExt{\isacharunderscore}{\kern0pt}def\ \isanewline
\ \ \ \ \ \ \isacommand{using}\isamarkupfalse%
\ y{\isacharprime}{\kern0pt}pH\ env{\isacharprime}{\kern0pt}H\ \isanewline
\ \ \ \ \ \ \isacommand{by}\isamarkupfalse%
\ auto\isanewline
\ \ \ \ \isacommand{also}\isamarkupfalse%
\ \isacommand{have}\isamarkupfalse%
\ {\isachardoublequoteopen}{\isachardot}{\kern0pt}{\isachardot}{\kern0pt}{\isachardot}{\kern0pt}\ {\isasymlongleftrightarrow}\ {\isacharparenleft}{\kern0pt}{\isasymexists}z{\isacharprime}{\kern0pt}{\isasymin}HS{\isachardot}{\kern0pt}\ {\isasymexists}q\ {\isasymin}\ G{\isachardot}{\kern0pt}\ q\ {\isasymtturnstile}HS\ {\isasymphi}\ {\isacharbrackleft}{\kern0pt}y{\isacharprime}{\kern0pt}{\isacharcomma}{\kern0pt}\ z{\isacharprime}{\kern0pt}{\isacharbrackright}{\kern0pt}\ {\isacharat}{\kern0pt}\ env{\isacharprime}{\kern0pt}{\isacharparenright}{\kern0pt}{\isachardoublequoteclose}\isanewline
\ \ \ \ \ \ \isacommand{apply}\isamarkupfalse%
{\isacharparenleft}{\kern0pt}rule\ bex{\isacharunderscore}{\kern0pt}iff{\isacharcomma}{\kern0pt}\ rule\ iff{\isacharunderscore}{\kern0pt}flip{\isacharparenright}{\kern0pt}\isanewline
\ \ \ \ \ \ \isacommand{apply}\isamarkupfalse%
{\isacharparenleft}{\kern0pt}rule\ HS{\isacharunderscore}{\kern0pt}truth{\isacharunderscore}{\kern0pt}lemma{\isacharparenright}{\kern0pt}\isanewline
\ \ \ \ \ \ \isacommand{using}\isamarkupfalse%
\ assms\ generic\ y{\isacharprime}{\kern0pt}pH\ x{\isacharprime}{\kern0pt}H\ env{\isacharprime}{\kern0pt}H\ y{\isacharprime}{\kern0pt}inHS\ leneq\isanewline
\ \ \ \ \ \ \isacommand{by}\isamarkupfalse%
\ auto\isanewline
\ \ \ \ \isacommand{also}\isamarkupfalse%
\ \isacommand{have}\isamarkupfalse%
\ {\isachardoublequoteopen}{\isachardot}{\kern0pt}{\isachardot}{\kern0pt}{\isachardot}{\kern0pt}\ {\isasymlongleftrightarrow}\ {\isacharparenleft}{\kern0pt}{\isasymexists}q\ {\isasymin}\ G{\isachardot}{\kern0pt}\ {\isasymexists}z{\isacharprime}{\kern0pt}{\isasymin}HS{\isachardot}{\kern0pt}\ q\ {\isasymtturnstile}HS\ {\isasymphi}\ {\isacharbrackleft}{\kern0pt}y{\isacharprime}{\kern0pt}{\isacharcomma}{\kern0pt}\ z{\isacharprime}{\kern0pt}{\isacharbrackright}{\kern0pt}\ {\isacharat}{\kern0pt}\ env{\isacharprime}{\kern0pt}{\isacharparenright}{\kern0pt}{\isachardoublequoteclose}\ \isacommand{by}\isamarkupfalse%
\ auto\isanewline
\ \ \ \ \isacommand{also}\isamarkupfalse%
\ \isacommand{have}\isamarkupfalse%
\ {\isachardoublequoteopen}{\isachardot}{\kern0pt}{\isachardot}{\kern0pt}{\isachardot}{\kern0pt}\ {\isasymlongleftrightarrow}\ {\isacharparenleft}{\kern0pt}{\isasymexists}q\ {\isasymin}\ G{\isachardot}{\kern0pt}\ {\isasymexists}z{\isacharprime}{\kern0pt}\ {\isasymin}\ S{\isachardot}{\kern0pt}\ q\ {\isasymtturnstile}HS\ {\isasymphi}\ {\isacharparenleft}{\kern0pt}{\isacharbrackleft}{\kern0pt}y{\isacharprime}{\kern0pt}{\isacharcomma}{\kern0pt}\ z{\isacharprime}{\kern0pt}{\isacharbrackright}{\kern0pt}\ {\isacharat}{\kern0pt}\ env{\isacharprime}{\kern0pt}{\isacharparenright}{\kern0pt}{\isacharparenright}{\kern0pt}{\isachardoublequoteclose}\ \isanewline
\ \ \ \ \ \ \isacommand{apply}\isamarkupfalse%
{\isacharparenleft}{\kern0pt}rule\ bex{\isacharunderscore}{\kern0pt}iff{\isacharparenright}{\kern0pt}\isanewline
\ \ \ \ \ \ \isacommand{using}\isamarkupfalse%
\ y{\isacharprime}{\kern0pt}pH\ SH\isanewline
\ \ \ \ \ \ \isacommand{by}\isamarkupfalse%
\ auto\isanewline
\ \ \ \ \isacommand{also}\isamarkupfalse%
\ \isacommand{have}\isamarkupfalse%
\ {\isachardoublequoteopen}{\isachardot}{\kern0pt}{\isachardot}{\kern0pt}{\isachardot}{\kern0pt}\ {\isasymlongleftrightarrow}\ {\isacharparenleft}{\kern0pt}{\isasymexists}z{\isacharprime}{\kern0pt}\ {\isasymin}\ S{\isachardot}{\kern0pt}\ {\isasymexists}q\ {\isasymin}\ G{\isachardot}{\kern0pt}\ q\ {\isasymtturnstile}HS\ {\isasymphi}\ {\isacharparenleft}{\kern0pt}{\isacharbrackleft}{\kern0pt}y{\isacharprime}{\kern0pt}{\isacharcomma}{\kern0pt}\ z{\isacharprime}{\kern0pt}{\isacharbrackright}{\kern0pt}\ {\isacharat}{\kern0pt}\ env{\isacharprime}{\kern0pt}{\isacharparenright}{\kern0pt}{\isacharparenright}{\kern0pt}{\isachardoublequoteclose}\ \isacommand{by}\isamarkupfalse%
\ auto\isanewline
\ \ \ \ \isacommand{also}\isamarkupfalse%
\ \isacommand{have}\isamarkupfalse%
\ {\isachardoublequoteopen}{\isachardot}{\kern0pt}{\isachardot}{\kern0pt}{\isachardot}{\kern0pt}\ {\isasymlongleftrightarrow}\ {\isacharparenleft}{\kern0pt}{\isasymexists}z{\isacharprime}{\kern0pt}\ {\isasymin}\ S{\isachardot}{\kern0pt}\ sats{\isacharparenleft}{\kern0pt}SymExt{\isacharparenleft}{\kern0pt}G{\isacharparenright}{\kern0pt}{\isacharcomma}{\kern0pt}\ {\isasymphi}{\isacharcomma}{\kern0pt}\ map{\isacharparenleft}{\kern0pt}val{\isacharparenleft}{\kern0pt}G{\isacharparenright}{\kern0pt}{\isacharcomma}{\kern0pt}\ {\isacharbrackleft}{\kern0pt}y{\isacharprime}{\kern0pt}{\isacharcomma}{\kern0pt}\ z{\isacharprime}{\kern0pt}{\isacharbrackright}{\kern0pt}\ {\isacharat}{\kern0pt}\ env{\isacharprime}{\kern0pt}{\isacharparenright}{\kern0pt}{\isacharparenright}{\kern0pt}{\isacharparenright}{\kern0pt}{\isachardoublequoteclose}\ \isanewline
\ \ \ \ \ \ \isacommand{apply}\isamarkupfalse%
{\isacharparenleft}{\kern0pt}rule\ bex{\isacharunderscore}{\kern0pt}iff{\isacharcomma}{\kern0pt}\ rule\ HS{\isacharunderscore}{\kern0pt}truth{\isacharunderscore}{\kern0pt}lemma{\isacharparenright}{\kern0pt}\isanewline
\ \ \ \ \ \ \isacommand{using}\isamarkupfalse%
\ assms\ generic\ SH\ env{\isacharprime}{\kern0pt}H\ y{\isacharprime}{\kern0pt}inHS\isanewline
\ \ \ \ \ \ \isacommand{by}\isamarkupfalse%
\ auto\isanewline
\ \ \ \ \isacommand{also}\isamarkupfalse%
\ \isacommand{have}\isamarkupfalse%
\ {\isachardoublequoteopen}{\isachardot}{\kern0pt}{\isachardot}{\kern0pt}{\isachardot}{\kern0pt}\ {\isasymlongleftrightarrow}\ {\isacharparenleft}{\kern0pt}{\isasymexists}z{\isacharprime}{\kern0pt}\ {\isasymin}\ S{\isachardot}{\kern0pt}\ sats{\isacharparenleft}{\kern0pt}SymExt{\isacharparenleft}{\kern0pt}G{\isacharparenright}{\kern0pt}{\isacharcomma}{\kern0pt}\ {\isasymphi}{\isacharcomma}{\kern0pt}\ {\isacharbrackleft}{\kern0pt}y{\isacharcomma}{\kern0pt}\ val{\isacharparenleft}{\kern0pt}G{\isacharcomma}{\kern0pt}\ z{\isacharprime}{\kern0pt}{\isacharparenright}{\kern0pt}{\isacharbrackright}{\kern0pt}\ {\isacharat}{\kern0pt}\ env{\isacharparenright}{\kern0pt}{\isacharparenright}{\kern0pt}{\isachardoublequoteclose}\ \isacommand{using}\isamarkupfalse%
\ y{\isacharprime}{\kern0pt}pH\ env{\isacharprime}{\kern0pt}H\ \isacommand{by}\isamarkupfalse%
\ auto\isanewline
\ \ \ \ \isacommand{finally}\isamarkupfalse%
\ \isacommand{have}\isamarkupfalse%
\ {\isachardoublequoteopen}{\isacharparenleft}{\kern0pt}{\isasymexists}z{\isasymin}SymExt{\isacharparenleft}{\kern0pt}G{\isacharparenright}{\kern0pt}{\isachardot}{\kern0pt}\ SymExt{\isacharparenleft}{\kern0pt}G{\isacharparenright}{\kern0pt}{\isacharcomma}{\kern0pt}\ {\isacharbrackleft}{\kern0pt}y{\isacharcomma}{\kern0pt}\ z{\isacharbrackright}{\kern0pt}\ {\isacharat}{\kern0pt}\ env\ {\isasymTurnstile}\ {\isasymphi}{\isacharparenright}{\kern0pt}\ {\isasymlongleftrightarrow}\ {\isacharparenleft}{\kern0pt}{\isasymexists}z{\isacharprime}{\kern0pt}{\isasymin}S{\isachardot}{\kern0pt}\ SymExt{\isacharparenleft}{\kern0pt}G{\isacharparenright}{\kern0pt}{\isacharcomma}{\kern0pt}\ {\isacharbrackleft}{\kern0pt}y{\isacharcomma}{\kern0pt}\ val{\isacharparenleft}{\kern0pt}G{\isacharcomma}{\kern0pt}\ z{\isacharprime}{\kern0pt}{\isacharparenright}{\kern0pt}{\isacharbrackright}{\kern0pt}\ {\isacharat}{\kern0pt}\ env\ {\isasymTurnstile}\ {\isasymphi}{\isacharparenright}{\kern0pt}\ {\isachardoublequoteclose}\ \isacommand{by}\isamarkupfalse%
\ auto\isanewline
\isanewline
\ \ \ \ \isacommand{then}\isamarkupfalse%
\ \isacommand{show}\isamarkupfalse%
\ {\isachardoublequoteopen}{\isacharparenleft}{\kern0pt}{\isasymexists}z{\isasymin}SymExt{\isacharparenleft}{\kern0pt}G{\isacharparenright}{\kern0pt}{\isachardot}{\kern0pt}\ SymExt{\isacharparenleft}{\kern0pt}G{\isacharparenright}{\kern0pt}{\isacharcomma}{\kern0pt}\ {\isacharbrackleft}{\kern0pt}y{\isacharcomma}{\kern0pt}\ z{\isacharbrackright}{\kern0pt}\ {\isacharat}{\kern0pt}\ env\ {\isasymTurnstile}\ {\isasymphi}{\isacharparenright}{\kern0pt}\ {\isasymlongleftrightarrow}\ {\isacharparenleft}{\kern0pt}{\isasymexists}z{\isasymin}T{\isachardot}{\kern0pt}\ SymExt{\isacharparenleft}{\kern0pt}G{\isacharparenright}{\kern0pt}{\isacharcomma}{\kern0pt}\ {\isacharbrackleft}{\kern0pt}y{\isacharcomma}{\kern0pt}\ z{\isacharbrackright}{\kern0pt}\ {\isacharat}{\kern0pt}\ env\ {\isasymTurnstile}\ {\isasymphi}{\isacharparenright}{\kern0pt}{\isachardoublequoteclose}\isanewline
\ \ \ \ \ \ \isacommand{using}\isamarkupfalse%
\ TH\ SymExt{\isacharunderscore}{\kern0pt}trans\isanewline
\ \ \ \ \ \ \isacommand{by}\isamarkupfalse%
\ auto\isanewline
\ \ \isacommand{next}\isamarkupfalse%
\ \isanewline
\ \ \ \ \isacommand{show}\isamarkupfalse%
\ {\isachardoublequoteopen}T\ {\isasymin}\ SymExt{\isacharparenleft}{\kern0pt}G{\isacharparenright}{\kern0pt}{\isachardoublequoteclose}\ \isacommand{using}\isamarkupfalse%
\ TH\ \isacommand{by}\isamarkupfalse%
\ auto\isanewline
\ \ \isacommand{qed}\isamarkupfalse%
\isanewline
\isacommand{qed}\isamarkupfalse%
%
\endisatagproof
{\isafoldproof}%
%
\isadelimproof
\isanewline
%
\endisadelimproof
\isanewline
\isacommand{lemma}\isamarkupfalse%
\ SymExt{\isacharunderscore}{\kern0pt}replacement\ {\isacharcolon}{\kern0pt}\isanewline
\ \ \isakeyword{fixes}\ {\isasymphi}\ env\isanewline
\ \ \isakeyword{assumes}\ {\isachardoublequoteopen}{\isasymphi}\ {\isasymin}\ formula{\isachardoublequoteclose}\ {\isachardoublequoteopen}arity{\isacharparenleft}{\kern0pt}{\isasymphi}{\isacharparenright}{\kern0pt}\ {\isasymle}\ {\isadigit{2}}\ {\isacharhash}{\kern0pt}{\isacharplus}{\kern0pt}\ length{\isacharparenleft}{\kern0pt}env{\isacharparenright}{\kern0pt}{\isachardoublequoteclose}\ {\isachardoublequoteopen}env\ {\isasymin}\ list{\isacharparenleft}{\kern0pt}SymExt{\isacharparenleft}{\kern0pt}G{\isacharparenright}{\kern0pt}{\isacharparenright}{\kern0pt}{\isachardoublequoteclose}\ \isanewline
\ \ \isakeyword{shows}\ {\isachardoublequoteopen}strong{\isacharunderscore}{\kern0pt}replacement{\isacharparenleft}{\kern0pt}{\isacharhash}{\kern0pt}{\isacharhash}{\kern0pt}SymExt{\isacharparenleft}{\kern0pt}G{\isacharparenright}{\kern0pt}{\isacharcomma}{\kern0pt}\ {\isasymlambda}x\ y{\isachardot}{\kern0pt}\ sats{\isacharparenleft}{\kern0pt}SymExt{\isacharparenleft}{\kern0pt}G{\isacharparenright}{\kern0pt}{\isacharcomma}{\kern0pt}\ {\isasymphi}{\isacharcomma}{\kern0pt}\ {\isacharbrackleft}{\kern0pt}x{\isacharcomma}{\kern0pt}\ y{\isacharbrackright}{\kern0pt}\ {\isacharat}{\kern0pt}\ env{\isacharparenright}{\kern0pt}{\isacharparenright}{\kern0pt}{\isachardoublequoteclose}\ \isanewline
%
\isadelimproof
\isanewline
\ \ %
\endisadelimproof
%
\isatagproof
\isacommand{unfolding}\isamarkupfalse%
\ strong{\isacharunderscore}{\kern0pt}replacement{\isacharunderscore}{\kern0pt}def\ \isanewline
\isacommand{proof}\isamarkupfalse%
{\isacharparenleft}{\kern0pt}rule\ rallI{\isacharcomma}{\kern0pt}\ rule\ impI{\isacharparenright}{\kern0pt}\isanewline
\isanewline
\ \ \isacommand{fix}\isamarkupfalse%
\ X\isanewline
\ \ \isacommand{assume}\isamarkupfalse%
\ assms{\isadigit{1}}\ {\isacharcolon}{\kern0pt}\ {\isachardoublequoteopen}{\isacharparenleft}{\kern0pt}{\isacharhash}{\kern0pt}{\isacharhash}{\kern0pt}SymExt{\isacharparenleft}{\kern0pt}G{\isacharparenright}{\kern0pt}{\isacharparenright}{\kern0pt}{\isacharparenleft}{\kern0pt}X{\isacharparenright}{\kern0pt}{\isachardoublequoteclose}\ {\isachardoublequoteopen}univalent{\isacharparenleft}{\kern0pt}{\isacharhash}{\kern0pt}{\isacharhash}{\kern0pt}SymExt{\isacharparenleft}{\kern0pt}G{\isacharparenright}{\kern0pt}{\isacharcomma}{\kern0pt}\ X{\isacharcomma}{\kern0pt}\ {\isasymlambda}x\ y{\isachardot}{\kern0pt}\ {\isacharparenleft}{\kern0pt}SymExt{\isacharparenleft}{\kern0pt}G{\isacharparenright}{\kern0pt}{\isacharcomma}{\kern0pt}\ {\isacharbrackleft}{\kern0pt}x{\isacharcomma}{\kern0pt}\ y{\isacharbrackright}{\kern0pt}\ {\isacharat}{\kern0pt}\ env\ {\isasymTurnstile}\ {\isasymphi}{\isacharparenright}{\kern0pt}{\isacharparenright}{\kern0pt}{\isachardoublequoteclose}\isanewline
\isanewline
\ \ \isacommand{have}\isamarkupfalse%
\ {\isachardoublequoteopen}{\isasymexists}Y\ {\isasymin}\ SymExt{\isacharparenleft}{\kern0pt}G{\isacharparenright}{\kern0pt}{\isachardot}{\kern0pt}\ {\isasymforall}x\ {\isasymin}\ X{\isachardot}{\kern0pt}\ {\isacharparenleft}{\kern0pt}{\isacharparenleft}{\kern0pt}{\isasymexists}y\ {\isasymin}\ SymExt{\isacharparenleft}{\kern0pt}G{\isacharparenright}{\kern0pt}{\isachardot}{\kern0pt}\ sats{\isacharparenleft}{\kern0pt}SymExt{\isacharparenleft}{\kern0pt}G{\isacharparenright}{\kern0pt}{\isacharcomma}{\kern0pt}\ {\isasymphi}{\isacharcomma}{\kern0pt}\ {\isacharbrackleft}{\kern0pt}x{\isacharcomma}{\kern0pt}\ y{\isacharbrackright}{\kern0pt}\ {\isacharat}{\kern0pt}\ env{\isacharparenright}{\kern0pt}{\isacharparenright}{\kern0pt}\ {\isasymlongleftrightarrow}\ {\isacharparenleft}{\kern0pt}{\isasymexists}y\ {\isasymin}\ Y{\isachardot}{\kern0pt}\ sats{\isacharparenleft}{\kern0pt}SymExt{\isacharparenleft}{\kern0pt}G{\isacharparenright}{\kern0pt}{\isacharcomma}{\kern0pt}\ {\isasymphi}{\isacharcomma}{\kern0pt}\ {\isacharbrackleft}{\kern0pt}x{\isacharcomma}{\kern0pt}\ y{\isacharbrackright}{\kern0pt}\ {\isacharat}{\kern0pt}\ env{\isacharparenright}{\kern0pt}{\isacharparenright}{\kern0pt}{\isacharparenright}{\kern0pt}{\isachardoublequoteclose}\isanewline
\ \ \ \ \isacommand{apply}\isamarkupfalse%
{\isacharparenleft}{\kern0pt}rule\ ex{\isacharunderscore}{\kern0pt}SymExt{\isacharunderscore}{\kern0pt}elem{\isacharunderscore}{\kern0pt}contains{\isacharunderscore}{\kern0pt}witnesses{\isacharparenright}{\kern0pt}\isanewline
\ \ \ \ \isacommand{using}\isamarkupfalse%
\ assms\ assms{\isadigit{1}}\isanewline
\ \ \ \ \isacommand{by}\isamarkupfalse%
\ auto\isanewline
\ \ \isacommand{then}\isamarkupfalse%
\ \isacommand{obtain}\isamarkupfalse%
\ Y\ \isakeyword{where}\ YH\ {\isacharcolon}{\kern0pt}\ {\isachardoublequoteopen}Y\ {\isasymin}\ SymExt{\isacharparenleft}{\kern0pt}G{\isacharparenright}{\kern0pt}{\isachardoublequoteclose}\ {\isachardoublequoteopen}{\isasymforall}x\ {\isasymin}\ X{\isachardot}{\kern0pt}\ {\isacharparenleft}{\kern0pt}{\isacharparenleft}{\kern0pt}{\isasymexists}y\ {\isasymin}\ SymExt{\isacharparenleft}{\kern0pt}G{\isacharparenright}{\kern0pt}{\isachardot}{\kern0pt}\ sats{\isacharparenleft}{\kern0pt}SymExt{\isacharparenleft}{\kern0pt}G{\isacharparenright}{\kern0pt}{\isacharcomma}{\kern0pt}\ {\isasymphi}{\isacharcomma}{\kern0pt}\ {\isacharbrackleft}{\kern0pt}x{\isacharcomma}{\kern0pt}\ y{\isacharbrackright}{\kern0pt}\ {\isacharat}{\kern0pt}\ env{\isacharparenright}{\kern0pt}{\isacharparenright}{\kern0pt}\ {\isasymlongleftrightarrow}\ {\isacharparenleft}{\kern0pt}{\isasymexists}y\ {\isasymin}\ Y{\isachardot}{\kern0pt}\ sats{\isacharparenleft}{\kern0pt}SymExt{\isacharparenleft}{\kern0pt}G{\isacharparenright}{\kern0pt}{\isacharcomma}{\kern0pt}\ {\isasymphi}{\isacharcomma}{\kern0pt}\ {\isacharbrackleft}{\kern0pt}x{\isacharcomma}{\kern0pt}\ y{\isacharbrackright}{\kern0pt}\ {\isacharat}{\kern0pt}\ env{\isacharparenright}{\kern0pt}{\isacharparenright}{\kern0pt}{\isacharparenright}{\kern0pt}{\isachardoublequoteclose}\ \isacommand{by}\isamarkupfalse%
\ auto\isanewline
\isanewline
\ \ \isacommand{define}\isamarkupfalse%
\ {\isasympsi}\ \isakeyword{where}\ {\isachardoublequoteopen}{\isasympsi}\ {\isasymequiv}\ Exists{\isacharparenleft}{\kern0pt}And{\isacharparenleft}{\kern0pt}Member{\isacharparenleft}{\kern0pt}{\isadigit{0}}{\isacharcomma}{\kern0pt}\ {\isadigit{2}}\ {\isacharhash}{\kern0pt}{\isacharplus}{\kern0pt}\ length{\isacharparenleft}{\kern0pt}env{\isacharparenright}{\kern0pt}{\isacharparenright}{\kern0pt}{\isacharcomma}{\kern0pt}\ {\isasymphi}{\isacharparenright}{\kern0pt}{\isacharparenright}{\kern0pt}{\isachardoublequoteclose}\ \isanewline
\isanewline
\ \ \isacommand{have}\isamarkupfalse%
\ {\isasympsi}{\isacharunderscore}{\kern0pt}type\ {\isacharcolon}{\kern0pt}\ {\isachardoublequoteopen}{\isasympsi}\ {\isasymin}\ formula{\isachardoublequoteclose}\ \isanewline
\ \ \ \ \isacommand{unfolding}\isamarkupfalse%
\ {\isasympsi}{\isacharunderscore}{\kern0pt}def\ \isanewline
\ \ \ \ \isacommand{using}\isamarkupfalse%
\ assms\isanewline
\ \ \ \ \isacommand{by}\isamarkupfalse%
\ auto\isanewline
\isanewline
\ \ \isacommand{have}\isamarkupfalse%
\ arity{\isacharunderscore}{\kern0pt}{\isasympsi}\ {\isacharcolon}{\kern0pt}\ {\isachardoublequoteopen}arity{\isacharparenleft}{\kern0pt}{\isasympsi}{\isacharparenright}{\kern0pt}\ {\isasymle}\ {\isadigit{2}}\ {\isacharhash}{\kern0pt}{\isacharplus}{\kern0pt}\ length{\isacharparenleft}{\kern0pt}env{\isacharparenright}{\kern0pt}{\isachardoublequoteclose}\ \isanewline
\ \ \ \ \isacommand{unfolding}\isamarkupfalse%
\ {\isasympsi}{\isacharunderscore}{\kern0pt}def\isanewline
\ \ \ \ \isacommand{using}\isamarkupfalse%
\ assms\isanewline
\ \ \ \ \isacommand{apply}\isamarkupfalse%
\ simp\isanewline
\ \ \ \ \isacommand{apply}\isamarkupfalse%
{\isacharparenleft}{\kern0pt}rule\ pred{\isacharunderscore}{\kern0pt}le{\isacharparenright}{\kern0pt}\isanewline
\ \ \ \ \isacommand{using}\isamarkupfalse%
\ assms\isanewline
\ \ \ \ \isacommand{apply}\isamarkupfalse%
\ auto{\isacharbrackleft}{\kern0pt}{\isadigit{2}}{\isacharbrackright}{\kern0pt}\isanewline
\ \ \ \ \isacommand{apply}\isamarkupfalse%
{\isacharparenleft}{\kern0pt}rule\ Un{\isacharunderscore}{\kern0pt}least{\isacharunderscore}{\kern0pt}lt{\isacharparenright}{\kern0pt}{\isacharplus}{\kern0pt}\isanewline
\ \ \ \ \ \ \isacommand{apply}\isamarkupfalse%
\ auto{\isacharbrackleft}{\kern0pt}{\isadigit{2}}{\isacharbrackright}{\kern0pt}\isanewline
\ \ \ \ \isacommand{apply}\isamarkupfalse%
{\isacharparenleft}{\kern0pt}rule{\isacharunderscore}{\kern0pt}tac\ j{\isacharequal}{\kern0pt}{\isachardoublequoteopen}{\isadigit{2}}\ {\isacharhash}{\kern0pt}{\isacharplus}{\kern0pt}\ length{\isacharparenleft}{\kern0pt}env{\isacharparenright}{\kern0pt}{\isachardoublequoteclose}\ \isakeyword{in}\ le{\isacharunderscore}{\kern0pt}trans{\isacharparenright}{\kern0pt}\ \isanewline
\ \ \ \ \isacommand{by}\isamarkupfalse%
\ auto\isanewline
\isanewline
\ \ \isacommand{define}\isamarkupfalse%
\ U\ \isakeyword{where}\ {\isachardoublequoteopen}U\ {\isasymequiv}\ {\isacharbraceleft}{\kern0pt}\ y\ {\isasymin}\ Y\ {\isachardot}{\kern0pt}\ {\isasymexists}x\ {\isasymin}\ X{\isachardot}{\kern0pt}\ sats{\isacharparenleft}{\kern0pt}SymExt{\isacharparenleft}{\kern0pt}G{\isacharparenright}{\kern0pt}{\isacharcomma}{\kern0pt}\ {\isasymphi}{\isacharcomma}{\kern0pt}\ {\isacharbrackleft}{\kern0pt}x{\isacharcomma}{\kern0pt}\ y{\isacharbrackright}{\kern0pt}\ {\isacharat}{\kern0pt}\ env{\isacharparenright}{\kern0pt}\ {\isacharbraceright}{\kern0pt}{\isachardoublequoteclose}\ \isanewline
\isanewline
\ \ \isacommand{have}\isamarkupfalse%
\ {\isachardoublequoteopen}U\ {\isacharequal}{\kern0pt}\ {\isacharbraceleft}{\kern0pt}\ y\ {\isasymin}\ Y{\isachardot}{\kern0pt}\ {\isasymexists}x\ {\isasymin}\ X{\isachardot}{\kern0pt}\ sats{\isacharparenleft}{\kern0pt}SymExt{\isacharparenleft}{\kern0pt}G{\isacharparenright}{\kern0pt}{\isacharcomma}{\kern0pt}\ {\isasymphi}{\isacharcomma}{\kern0pt}\ {\isacharparenleft}{\kern0pt}{\isacharbrackleft}{\kern0pt}x{\isacharcomma}{\kern0pt}\ y{\isacharbrackright}{\kern0pt}\ {\isacharat}{\kern0pt}\ env{\isacharparenright}{\kern0pt}\ {\isacharat}{\kern0pt}\ {\isacharbrackleft}{\kern0pt}X{\isacharbrackright}{\kern0pt}{\isacharparenright}{\kern0pt}\ {\isacharbraceright}{\kern0pt}{\isachardoublequoteclose}\ \isanewline
\ \ \ \ \isacommand{unfolding}\isamarkupfalse%
\ U{\isacharunderscore}{\kern0pt}def\ \isanewline
\ \ \ \ \isacommand{apply}\isamarkupfalse%
{\isacharparenleft}{\kern0pt}rule\ iff{\isacharunderscore}{\kern0pt}eq{\isacharcomma}{\kern0pt}\ rule\ bex{\isacharunderscore}{\kern0pt}iff{\isacharcomma}{\kern0pt}\ rule\ iff{\isacharunderscore}{\kern0pt}flip{\isacharparenright}{\kern0pt}\isanewline
\ \ \ \ \isacommand{apply}\isamarkupfalse%
{\isacharparenleft}{\kern0pt}rule\ arity{\isacharunderscore}{\kern0pt}sats{\isacharunderscore}{\kern0pt}iff{\isacharparenright}{\kern0pt}\isanewline
\ \ \ \ \isacommand{using}\isamarkupfalse%
\ assms\ assms{\isadigit{1}}\ SymExt{\isacharunderscore}{\kern0pt}trans\ YH\isanewline
\ \ \ \ \isacommand{by}\isamarkupfalse%
\ auto\ \isanewline
\ \ \isacommand{also}\isamarkupfalse%
\ \isacommand{have}\isamarkupfalse%
\ {\isachardoublequoteopen}{\isachardot}{\kern0pt}{\isachardot}{\kern0pt}{\isachardot}{\kern0pt}\ {\isacharequal}{\kern0pt}\ {\isacharbraceleft}{\kern0pt}\ y\ {\isasymin}\ Y\ {\isachardot}{\kern0pt}\ sats{\isacharparenleft}{\kern0pt}SymExt{\isacharparenleft}{\kern0pt}G{\isacharparenright}{\kern0pt}{\isacharcomma}{\kern0pt}\ {\isasympsi}{\isacharcomma}{\kern0pt}\ {\isacharbrackleft}{\kern0pt}y{\isacharbrackright}{\kern0pt}\ {\isacharat}{\kern0pt}\ env\ {\isacharat}{\kern0pt}\ {\isacharbrackleft}{\kern0pt}X{\isacharbrackright}{\kern0pt}{\isacharparenright}{\kern0pt}\ {\isacharbraceright}{\kern0pt}{\isachardoublequoteclose}\ {\isacharparenleft}{\kern0pt}\isakeyword{is}\ {\isachardoublequoteopen}{\isacharunderscore}{\kern0pt}\ {\isacharequal}{\kern0pt}\ {\isacharquery}{\kern0pt}A{\isachardoublequoteclose}{\isacharparenright}{\kern0pt}\ \isanewline
\ \ \ \ \isacommand{unfolding}\isamarkupfalse%
\ U{\isacharunderscore}{\kern0pt}def\ {\isasympsi}{\isacharunderscore}{\kern0pt}def\isanewline
\ \ \ \ \isacommand{apply}\isamarkupfalse%
{\isacharparenleft}{\kern0pt}rule\ iff{\isacharunderscore}{\kern0pt}eq{\isacharparenright}{\kern0pt}\isanewline
\ \ \ \ \isacommand{apply}\isamarkupfalse%
{\isacharparenleft}{\kern0pt}rename{\isacharunderscore}{\kern0pt}tac\ y{\isacharcomma}{\kern0pt}\ subgoal{\isacharunderscore}{\kern0pt}tac\ {\isachardoublequoteopen}y\ {\isasymin}\ SymExt{\isacharparenleft}{\kern0pt}G{\isacharparenright}{\kern0pt}{\isachardoublequoteclose}{\isacharparenright}{\kern0pt}\isanewline
\ \ \ \ \isacommand{using}\isamarkupfalse%
\ assms\ assms{\isadigit{1}}\ \isanewline
\ \ \ \ \ \isacommand{apply}\isamarkupfalse%
\ simp\ \isanewline
\ \ \ \ \ \isacommand{apply}\isamarkupfalse%
{\isacharparenleft}{\kern0pt}subst\ nth{\isacharunderscore}{\kern0pt}append{\isacharcomma}{\kern0pt}\ simp{\isacharcomma}{\kern0pt}\ simp{\isacharcomma}{\kern0pt}\ subst\ if{\isacharunderscore}{\kern0pt}not{\isacharunderscore}{\kern0pt}P{\isacharcomma}{\kern0pt}\ force{\isacharparenright}{\kern0pt}\isanewline
\ \ \ \ \isacommand{using}\isamarkupfalse%
\ SymExt{\isacharunderscore}{\kern0pt}trans\ YH\ \isanewline
\ \ \ \ \isacommand{by}\isamarkupfalse%
\ auto\isanewline
\ \ \isacommand{finally}\isamarkupfalse%
\ \isacommand{have}\isamarkupfalse%
\ {\isachardoublequoteopen}U\ {\isacharequal}{\kern0pt}\ {\isacharbraceleft}{\kern0pt}\ y\ {\isasymin}\ Y\ {\isachardot}{\kern0pt}\ sats{\isacharparenleft}{\kern0pt}SymExt{\isacharparenleft}{\kern0pt}G{\isacharparenright}{\kern0pt}{\isacharcomma}{\kern0pt}\ {\isasympsi}{\isacharcomma}{\kern0pt}\ {\isacharbrackleft}{\kern0pt}y{\isacharbrackright}{\kern0pt}\ {\isacharat}{\kern0pt}\ env\ {\isacharat}{\kern0pt}\ {\isacharbrackleft}{\kern0pt}X{\isacharbrackright}{\kern0pt}{\isacharparenright}{\kern0pt}\ {\isacharbraceright}{\kern0pt}{\isachardoublequoteclose}\ {\isacharparenleft}{\kern0pt}\isakeyword{is}\ {\isachardoublequoteopen}{\isacharunderscore}{\kern0pt}\ {\isacharequal}{\kern0pt}\ {\isacharquery}{\kern0pt}A{\isachardoublequoteclose}{\isacharparenright}{\kern0pt}\ \isacommand{by}\isamarkupfalse%
\ simp\isanewline
\isanewline
\ \ \isacommand{have}\isamarkupfalse%
\ {\isachardoublequoteopen}{\isacharquery}{\kern0pt}A\ {\isasymin}\ SymExt{\isacharparenleft}{\kern0pt}G{\isacharparenright}{\kern0pt}{\isachardoublequoteclose}\ \isanewline
\ \ \ \ \isacommand{apply}\isamarkupfalse%
{\isacharparenleft}{\kern0pt}rule\ SymExt{\isacharunderscore}{\kern0pt}separation{\isacharparenright}{\kern0pt}\isanewline
\ \ \ \ \isacommand{using}\isamarkupfalse%
\ YH\ assms\ assms{\isadigit{1}}\ {\isasympsi}{\isacharunderscore}{\kern0pt}type\ arity{\isacharunderscore}{\kern0pt}{\isasympsi}\isanewline
\ \ \ \ \isacommand{by}\isamarkupfalse%
\ auto\isanewline
\ \ \isacommand{then}\isamarkupfalse%
\ \isacommand{have}\isamarkupfalse%
\ {\isachardoublequoteopen}U\ {\isasymin}\ SymExt{\isacharparenleft}{\kern0pt}G{\isacharparenright}{\kern0pt}{\isachardoublequoteclose}\ \isacommand{using}\isamarkupfalse%
\ {\isacartoucheopen}U\ {\isacharequal}{\kern0pt}\ {\isacharquery}{\kern0pt}A{\isacartoucheclose}\ \isacommand{by}\isamarkupfalse%
\ simp\isanewline
\isanewline
\ \ \isacommand{have}\isamarkupfalse%
\ {\isachardoublequoteopen}{\isasymAnd}y{\isachardot}{\kern0pt}\ y\ {\isasymin}\ SymExt{\isacharparenleft}{\kern0pt}G{\isacharparenright}{\kern0pt}\ {\isasymLongrightarrow}\ y\ {\isasymin}\ U\ {\isasymlongleftrightarrow}\ {\isacharparenleft}{\kern0pt}{\isasymexists}x\ {\isasymin}\ X{\isachardot}{\kern0pt}\ sats{\isacharparenleft}{\kern0pt}SymExt{\isacharparenleft}{\kern0pt}G{\isacharparenright}{\kern0pt}{\isacharcomma}{\kern0pt}\ {\isasymphi}{\isacharcomma}{\kern0pt}\ {\isacharbrackleft}{\kern0pt}x{\isacharcomma}{\kern0pt}\ y{\isacharbrackright}{\kern0pt}\ {\isacharat}{\kern0pt}\ env{\isacharparenright}{\kern0pt}{\isacharparenright}{\kern0pt}{\isachardoublequoteclose}\ \isanewline
\ \ \ \ \isacommand{unfolding}\isamarkupfalse%
\ U{\isacharunderscore}{\kern0pt}def\ \isanewline
\ \ \isacommand{proof}\isamarkupfalse%
{\isacharparenleft}{\kern0pt}rule\ iffI{\isacharcomma}{\kern0pt}\ force{\isacharparenright}{\kern0pt}\isanewline
\ \ \ \ \isacommand{fix}\isamarkupfalse%
\ y\ \isanewline
\ \ \ \ \isacommand{assume}\isamarkupfalse%
\ assms{\isadigit{2}}{\isacharcolon}{\kern0pt}\ {\isachardoublequoteopen}y\ {\isasymin}\ SymExt{\isacharparenleft}{\kern0pt}G{\isacharparenright}{\kern0pt}{\isachardoublequoteclose}\ {\isachardoublequoteopen}y\ {\isasymin}\ SymExt{\isacharparenleft}{\kern0pt}G{\isacharparenright}{\kern0pt}{\isachardoublequoteclose}\ {\isachardoublequoteopen}{\isasymexists}x{\isasymin}X{\isachardot}{\kern0pt}\ SymExt{\isacharparenleft}{\kern0pt}G{\isacharparenright}{\kern0pt}{\isacharcomma}{\kern0pt}\ {\isacharbrackleft}{\kern0pt}x{\isacharcomma}{\kern0pt}\ y{\isacharbrackright}{\kern0pt}\ {\isacharat}{\kern0pt}\ env\ {\isasymTurnstile}\ {\isasymphi}{\isachardoublequoteclose}\ \isanewline
\ \ \ \ \isacommand{then}\isamarkupfalse%
\ \isacommand{obtain}\isamarkupfalse%
\ x\ \isakeyword{where}\ xH{\isacharcolon}{\kern0pt}\ {\isachardoublequoteopen}x\ {\isasymin}\ X{\isachardoublequoteclose}\ {\isachardoublequoteopen}SymExt{\isacharparenleft}{\kern0pt}G{\isacharparenright}{\kern0pt}{\isacharcomma}{\kern0pt}\ {\isacharbrackleft}{\kern0pt}x{\isacharcomma}{\kern0pt}\ y{\isacharbrackright}{\kern0pt}\ {\isacharat}{\kern0pt}\ env\ {\isasymTurnstile}\ {\isasymphi}{\isachardoublequoteclose}\ \isacommand{by}\isamarkupfalse%
\ auto\isanewline
\isanewline
\ \ \ \ \isacommand{then}\isamarkupfalse%
\ \isacommand{have}\isamarkupfalse%
\ {\isachardoublequoteopen}{\isasymexists}y\ {\isasymin}\ SymExt{\isacharparenleft}{\kern0pt}G{\isacharparenright}{\kern0pt}{\isachardot}{\kern0pt}\ sats{\isacharparenleft}{\kern0pt}SymExt{\isacharparenleft}{\kern0pt}G{\isacharparenright}{\kern0pt}{\isacharcomma}{\kern0pt}\ {\isasymphi}{\isacharcomma}{\kern0pt}\ {\isacharbrackleft}{\kern0pt}x{\isacharcomma}{\kern0pt}\ y{\isacharbrackright}{\kern0pt}\ {\isacharat}{\kern0pt}\ env{\isacharparenright}{\kern0pt}{\isachardoublequoteclose}\ \isacommand{using}\isamarkupfalse%
\ assms{\isadigit{2}}\ xH\ \isacommand{by}\isamarkupfalse%
\ auto\isanewline
\ \ \ \ \isacommand{then}\isamarkupfalse%
\ \isacommand{have}\isamarkupfalse%
\ {\isachardoublequoteopen}{\isasymexists}y\ {\isasymin}\ Y{\isachardot}{\kern0pt}\ sats{\isacharparenleft}{\kern0pt}SymExt{\isacharparenleft}{\kern0pt}G{\isacharparenright}{\kern0pt}{\isacharcomma}{\kern0pt}\ {\isasymphi}{\isacharcomma}{\kern0pt}\ {\isacharbrackleft}{\kern0pt}x{\isacharcomma}{\kern0pt}\ y{\isacharbrackright}{\kern0pt}\ {\isacharat}{\kern0pt}\ env{\isacharparenright}{\kern0pt}{\isachardoublequoteclose}\ \isacommand{using}\isamarkupfalse%
\ YH\ xH\ \isacommand{by}\isamarkupfalse%
\ auto\isanewline
\ \ \ \ \isacommand{then}\isamarkupfalse%
\ \isacommand{obtain}\isamarkupfalse%
\ z\ \isakeyword{where}\ zH{\isacharcolon}{\kern0pt}\ {\isachardoublequoteopen}z\ {\isasymin}\ Y{\isachardoublequoteclose}\ {\isachardoublequoteopen}z\ {\isasymin}\ SymExt{\isacharparenleft}{\kern0pt}G{\isacharparenright}{\kern0pt}{\isachardoublequoteclose}\ {\isachardoublequoteopen}sats{\isacharparenleft}{\kern0pt}SymExt{\isacharparenleft}{\kern0pt}G{\isacharparenright}{\kern0pt}{\isacharcomma}{\kern0pt}\ {\isasymphi}{\isacharcomma}{\kern0pt}\ {\isacharbrackleft}{\kern0pt}x{\isacharcomma}{\kern0pt}\ z{\isacharbrackright}{\kern0pt}\ {\isacharat}{\kern0pt}\ env{\isacharparenright}{\kern0pt}{\isachardoublequoteclose}\ \isacommand{using}\isamarkupfalse%
\ YH\ SymExt{\isacharunderscore}{\kern0pt}trans\ \isacommand{by}\isamarkupfalse%
\ auto\isanewline
\isanewline
\ \ \ \ \isacommand{have}\isamarkupfalse%
\ univalentE\ {\isacharcolon}{\kern0pt}\ {\isachardoublequoteopen}{\isasymAnd}x\ y\ z{\isachardot}{\kern0pt}\ x\ {\isasymin}\ X\ {\isasymLongrightarrow}\ y\ {\isasymin}\ SymExt{\isacharparenleft}{\kern0pt}G{\isacharparenright}{\kern0pt}\ {\isasymLongrightarrow}\ z\ {\isasymin}\ SymExt{\isacharparenleft}{\kern0pt}G{\isacharparenright}{\kern0pt}\ {\isasymLongrightarrow}\ {\isacharparenleft}{\kern0pt}SymExt{\isacharparenleft}{\kern0pt}G{\isacharparenright}{\kern0pt}{\isacharcomma}{\kern0pt}\ {\isacharbrackleft}{\kern0pt}x{\isacharcomma}{\kern0pt}\ y{\isacharbrackright}{\kern0pt}\ {\isacharat}{\kern0pt}\ env\ {\isasymTurnstile}\ {\isasymphi}{\isacharparenright}{\kern0pt}\ {\isasymLongrightarrow}\ {\isacharparenleft}{\kern0pt}SymExt{\isacharparenleft}{\kern0pt}G{\isacharparenright}{\kern0pt}{\isacharcomma}{\kern0pt}\ {\isacharbrackleft}{\kern0pt}x{\isacharcomma}{\kern0pt}\ z{\isacharbrackright}{\kern0pt}\ {\isacharat}{\kern0pt}\ env\ {\isasymTurnstile}\ {\isasymphi}{\isacharparenright}{\kern0pt}\ {\isasymLongrightarrow}\ y\ {\isacharequal}{\kern0pt}\ z{\isachardoublequoteclose}\isanewline
\ \ \ \ \ \ \isacommand{using}\isamarkupfalse%
\ assms{\isadigit{1}}\ SymExt{\isacharunderscore}{\kern0pt}trans\ \isanewline
\ \ \ \ \ \ \isacommand{unfolding}\isamarkupfalse%
\ univalent{\isacharunderscore}{\kern0pt}def\ \isanewline
\ \ \ \ \ \ \isacommand{by}\isamarkupfalse%
\ auto\isanewline
\isanewline
\ \ \ \ \isacommand{have}\isamarkupfalse%
\ {\isachardoublequoteopen}y\ {\isacharequal}{\kern0pt}\ z{\isachardoublequoteclose}\ \isanewline
\ \ \ \ \ \ \isacommand{apply}\isamarkupfalse%
{\isacharparenleft}{\kern0pt}rule\ univalentE{\isacharparenright}{\kern0pt}\isanewline
\ \ \ \ \ \ \isacommand{using}\isamarkupfalse%
\ assms{\isadigit{1}}\ xH\ assms{\isadigit{2}}\ zH\ \isanewline
\ \ \ \ \ \ \isacommand{by}\isamarkupfalse%
\ auto\isanewline
\ \ \ \ \isacommand{then}\isamarkupfalse%
\ \isacommand{have}\isamarkupfalse%
\ {\isachardoublequoteopen}y\ {\isasymin}\ Y{\isachardoublequoteclose}\ \isacommand{using}\isamarkupfalse%
\ zH\ \isacommand{by}\isamarkupfalse%
\ auto\isanewline
\ \ \ \ \isacommand{then}\isamarkupfalse%
\ \isacommand{show}\isamarkupfalse%
\ {\isachardoublequoteopen}y\ {\isasymin}\ {\isacharbraceleft}{\kern0pt}y\ {\isasymin}\ Y\ {\isachardot}{\kern0pt}\ {\isasymexists}x{\isasymin}X{\isachardot}{\kern0pt}\ SymExt{\isacharparenleft}{\kern0pt}G{\isacharparenright}{\kern0pt}{\isacharcomma}{\kern0pt}\ {\isacharbrackleft}{\kern0pt}x{\isacharcomma}{\kern0pt}\ y{\isacharbrackright}{\kern0pt}\ {\isacharat}{\kern0pt}\ env\ {\isasymTurnstile}\ {\isasymphi}{\isacharbraceright}{\kern0pt}{\isachardoublequoteclose}\ \isanewline
\ \ \ \ \ \ \isacommand{using}\isamarkupfalse%
\ assms{\isadigit{2}}\ \isanewline
\ \ \ \ \ \ \isacommand{by}\isamarkupfalse%
\ auto\isanewline
\ \ \isacommand{qed}\isamarkupfalse%
\isanewline
\ \ \isacommand{then}\isamarkupfalse%
\ \isacommand{show}\isamarkupfalse%
\ {\isachardoublequoteopen}{\isasymexists}Y{\isacharbrackleft}{\kern0pt}{\isacharhash}{\kern0pt}{\isacharhash}{\kern0pt}SymExt{\isacharparenleft}{\kern0pt}G{\isacharparenright}{\kern0pt}{\isacharbrackright}{\kern0pt}{\isachardot}{\kern0pt}\ {\isasymforall}b{\isacharbrackleft}{\kern0pt}{\isacharhash}{\kern0pt}{\isacharhash}{\kern0pt}SymExt{\isacharparenleft}{\kern0pt}G{\isacharparenright}{\kern0pt}{\isacharbrackright}{\kern0pt}{\isachardot}{\kern0pt}\ b\ {\isasymin}\ Y\ {\isasymlongleftrightarrow}\ {\isacharparenleft}{\kern0pt}{\isasymexists}x{\isacharbrackleft}{\kern0pt}{\isacharhash}{\kern0pt}{\isacharhash}{\kern0pt}SymExt{\isacharparenleft}{\kern0pt}G{\isacharparenright}{\kern0pt}{\isacharbrackright}{\kern0pt}{\isachardot}{\kern0pt}\ x\ {\isasymin}\ X\ {\isasymand}\ SymExt{\isacharparenleft}{\kern0pt}G{\isacharparenright}{\kern0pt}{\isacharcomma}{\kern0pt}\ {\isacharbrackleft}{\kern0pt}x{\isacharcomma}{\kern0pt}\ b{\isacharbrackright}{\kern0pt}\ {\isacharat}{\kern0pt}\ env\ {\isasymTurnstile}\ {\isasymphi}{\isacharparenright}{\kern0pt}{\isachardoublequoteclose}\ \isanewline
\ \ \ \ \isacommand{apply}\isamarkupfalse%
{\isacharparenleft}{\kern0pt}rule{\isacharunderscore}{\kern0pt}tac\ x{\isacharequal}{\kern0pt}U\ \isakeyword{in}\ rexI{\isacharparenright}{\kern0pt}\isanewline
\ \ \ \ \ \isacommand{apply}\isamarkupfalse%
{\isacharparenleft}{\kern0pt}rule\ rallI{\isacharparenright}{\kern0pt}\isanewline
\ \ \ \ \isacommand{using}\isamarkupfalse%
\ assms{\isadigit{1}}\ SymExt{\isacharunderscore}{\kern0pt}trans\isanewline
\ \ \ \ \ \isacommand{apply}\isamarkupfalse%
\ force\ \isanewline
\ \ \ \ \isacommand{using}\isamarkupfalse%
\ {\isacartoucheopen}U\ {\isasymin}\ SymExt{\isacharparenleft}{\kern0pt}G{\isacharparenright}{\kern0pt}{\isacartoucheclose}\isanewline
\ \ \ \ \isacommand{by}\isamarkupfalse%
\ simp\isanewline
\isacommand{qed}\isamarkupfalse%
%
\endisatagproof
{\isafoldproof}%
%
\isadelimproof
\isanewline
%
\endisadelimproof
\isanewline
\isanewline
\isacommand{end}\isamarkupfalse%
\isanewline
%
\isadelimtheory
%
\endisadelimtheory
%
\isatagtheory
\isacommand{end}\isamarkupfalse%
%
\endisatagtheory
{\isafoldtheory}%
%
\isadelimtheory
%
\endisadelimtheory
%
\end{isabellebody}%
\endinput
%:%file=~/source/repos/ZF-notAC/code/SymExt_Replacement.thy%:%
%:%10=1%:%
%:%11=1%:%
%:%12=2%:%
%:%13=3%:%
%:%14=4%:%
%:%15=5%:%
%:%20=5%:%
%:%23=6%:%
%:%24=7%:%
%:%25=7%:%
%:%26=8%:%
%:%27=9%:%
%:%28=10%:%
%:%29=10%:%
%:%30=11%:%
%:%31=12%:%
%:%32=12%:%
%:%33=13%:%
%:%34=14%:%
%:%35=15%:%
%:%38=16%:%
%:%42=16%:%
%:%43=16%:%
%:%44=17%:%
%:%45=17%:%
%:%46=18%:%
%:%47=18%:%
%:%52=18%:%
%:%55=19%:%
%:%56=20%:%
%:%57=20%:%
%:%58=21%:%
%:%59=22%:%
%:%60=23%:%
%:%63=24%:%
%:%67=24%:%
%:%68=24%:%
%:%69=25%:%
%:%70=25%:%
%:%71=26%:%
%:%72=26%:%
%:%73=27%:%
%:%74=27%:%
%:%75=28%:%
%:%76=28%:%
%:%77=29%:%
%:%78=29%:%
%:%79=30%:%
%:%80=30%:%
%:%81=31%:%
%:%82=31%:%
%:%83=32%:%
%:%84=32%:%
%:%85=33%:%
%:%86=33%:%
%:%87=34%:%
%:%88=34%:%
%:%89=35%:%
%:%90=35%:%
%:%91=36%:%
%:%92=36%:%
%:%93=37%:%
%:%94=37%:%
%:%95=38%:%
%:%96=38%:%
%:%97=39%:%
%:%98=39%:%
%:%99=40%:%
%:%100=40%:%
%:%101=41%:%
%:%102=41%:%
%:%103=42%:%
%:%104=42%:%
%:%105=43%:%
%:%106=43%:%
%:%107=44%:%
%:%108=44%:%
%:%109=45%:%
%:%110=45%:%
%:%111=46%:%
%:%112=46%:%
%:%113=47%:%
%:%114=47%:%
%:%115=48%:%
%:%116=48%:%
%:%117=49%:%
%:%118=49%:%
%:%119=50%:%
%:%120=50%:%
%:%121=51%:%
%:%127=51%:%
%:%130=52%:%
%:%131=53%:%
%:%132=53%:%
%:%133=54%:%
%:%134=55%:%
%:%135=56%:%
%:%138=57%:%
%:%139=58%:%
%:%143=58%:%
%:%144=58%:%
%:%145=59%:%
%:%146=59%:%
%:%147=60%:%
%:%148=60%:%
%:%149=61%:%
%:%150=61%:%
%:%151=62%:%
%:%152=62%:%
%:%153=63%:%
%:%154=63%:%
%:%155=64%:%
%:%156=64%:%
%:%157=65%:%
%:%158=65%:%
%:%159=66%:%
%:%160=66%:%
%:%161=67%:%
%:%162=67%:%
%:%163=68%:%
%:%164=68%:%
%:%165=69%:%
%:%166=69%:%
%:%167=70%:%
%:%168=70%:%
%:%173=70%:%
%:%176=71%:%
%:%177=72%:%
%:%178=72%:%
%:%186=80%:%
%:%187=81%:%
%:%188=82%:%
%:%189=82%:%
%:%190=83%:%
%:%191=84%:%
%:%192=85%:%
%:%195=86%:%
%:%199=86%:%
%:%200=86%:%
%:%201=87%:%
%:%202=87%:%
%:%203=88%:%
%:%204=88%:%
%:%205=89%:%
%:%206=89%:%
%:%207=90%:%
%:%208=90%:%
%:%213=90%:%
%:%216=91%:%
%:%217=92%:%
%:%218=92%:%
%:%219=93%:%
%:%222=94%:%
%:%226=94%:%
%:%227=94%:%
%:%228=95%:%
%:%229=96%:%
%:%230=96%:%
%:%231=97%:%
%:%232=97%:%
%:%233=98%:%
%:%234=98%:%
%:%235=99%:%
%:%236=99%:%
%:%241=99%:%
%:%244=100%:%
%:%245=101%:%
%:%246=101%:%
%:%247=102%:%
%:%248=103%:%
%:%249=104%:%
%:%252=105%:%
%:%253=106%:%
%:%257=106%:%
%:%258=106%:%
%:%259=107%:%
%:%260=107%:%
%:%261=108%:%
%:%262=108%:%
%:%263=109%:%
%:%264=109%:%
%:%265=110%:%
%:%266=110%:%
%:%267=111%:%
%:%268=111%:%
%:%269=112%:%
%:%270=112%:%
%:%271=113%:%
%:%272=113%:%
%:%273=114%:%
%:%274=114%:%
%:%275=115%:%
%:%276=115%:%
%:%277=116%:%
%:%278=116%:%
%:%279=117%:%
%:%280=117%:%
%:%281=118%:%
%:%282=118%:%
%:%283=119%:%
%:%284=119%:%
%:%285=120%:%
%:%286=120%:%
%:%291=120%:%
%:%294=121%:%
%:%295=122%:%
%:%296=122%:%
%:%297=123%:%
%:%308=134%:%
%:%309=135%:%
%:%310=140%:%
%:%311=141%:%
%:%312=142%:%
%:%313=143%:%
%:%314=143%:%
%:%315=144%:%
%:%316=145%:%
%:%317=146%:%
%:%318=147%:%
%:%321=148%:%
%:%322=149%:%
%:%326=149%:%
%:%327=149%:%
%:%328=150%:%
%:%329=150%:%
%:%330=151%:%
%:%331=151%:%
%:%332=152%:%
%:%335=155%:%
%:%336=156%:%
%:%337=156%:%
%:%338=157%:%
%:%339=157%:%
%:%340=158%:%
%:%341=158%:%
%:%342=159%:%
%:%343=159%:%
%:%344=160%:%
%:%345=160%:%
%:%346=161%:%
%:%347=161%:%
%:%348=162%:%
%:%349=162%:%
%:%350=163%:%
%:%351=163%:%
%:%352=164%:%
%:%353=164%:%
%:%354=165%:%
%:%355=165%:%
%:%356=166%:%
%:%357=166%:%
%:%358=167%:%
%:%359=167%:%
%:%360=168%:%
%:%361=168%:%
%:%362=169%:%
%:%363=169%:%
%:%364=170%:%
%:%365=170%:%
%:%366=171%:%
%:%367=171%:%
%:%368=172%:%
%:%369=172%:%
%:%370=173%:%
%:%371=173%:%
%:%372=174%:%
%:%373=174%:%
%:%374=175%:%
%:%375=175%:%
%:%376=176%:%
%:%377=176%:%
%:%378=177%:%
%:%379=177%:%
%:%380=178%:%
%:%381=178%:%
%:%382=179%:%
%:%383=179%:%
%:%384=180%:%
%:%385=180%:%
%:%386=181%:%
%:%387=181%:%
%:%388=182%:%
%:%389=182%:%
%:%390=183%:%
%:%391=184%:%
%:%392=184%:%
%:%393=185%:%
%:%394=185%:%
%:%395=186%:%
%:%396=186%:%
%:%397=187%:%
%:%398=187%:%
%:%399=188%:%
%:%400=188%:%
%:%401=189%:%
%:%402=189%:%
%:%403=190%:%
%:%404=190%:%
%:%405=191%:%
%:%406=191%:%
%:%407=192%:%
%:%408=192%:%
%:%409=193%:%
%:%410=193%:%
%:%411=194%:%
%:%412=194%:%
%:%413=195%:%
%:%414=195%:%
%:%415=196%:%
%:%416=196%:%
%:%417=197%:%
%:%418=197%:%
%:%419=198%:%
%:%420=198%:%
%:%421=199%:%
%:%422=199%:%
%:%423=200%:%
%:%424=200%:%
%:%425=201%:%
%:%426=201%:%
%:%427=202%:%
%:%428=202%:%
%:%429=203%:%
%:%430=203%:%
%:%435=203%:%
%:%438=204%:%
%:%439=205%:%
%:%440=205%:%
%:%441=206%:%
%:%442=207%:%
%:%443=208%:%
%:%446=209%:%
%:%447=210%:%
%:%451=210%:%
%:%452=210%:%
%:%453=211%:%
%:%454=211%:%
%:%455=212%:%
%:%456=212%:%
%:%457=213%:%
%:%458=213%:%
%:%459=214%:%
%:%460=214%:%
%:%461=215%:%
%:%462=215%:%
%:%463=216%:%
%:%464=216%:%
%:%465=217%:%
%:%466=217%:%
%:%467=218%:%
%:%468=218%:%
%:%469=219%:%
%:%470=219%:%
%:%471=220%:%
%:%472=220%:%
%:%473=221%:%
%:%474=221%:%
%:%475=222%:%
%:%476=222%:%
%:%477=223%:%
%:%478=223%:%
%:%479=224%:%
%:%480=224%:%
%:%481=225%:%
%:%482=225%:%
%:%487=225%:%
%:%490=226%:%
%:%491=227%:%
%:%492=227%:%
%:%493=228%:%
%:%494=229%:%
%:%495=230%:%
%:%498=231%:%
%:%502=231%:%
%:%503=231%:%
%:%504=232%:%
%:%505=232%:%
%:%506=233%:%
%:%507=233%:%
%:%508=234%:%
%:%509=234%:%
%:%510=235%:%
%:%511=235%:%
%:%512=236%:%
%:%513=236%:%
%:%514=237%:%
%:%515=237%:%
%:%516=238%:%
%:%517=238%:%
%:%518=239%:%
%:%519=239%:%
%:%520=240%:%
%:%521=240%:%
%:%522=241%:%
%:%523=241%:%
%:%524=242%:%
%:%525=242%:%
%:%526=243%:%
%:%527=243%:%
%:%528=244%:%
%:%529=244%:%
%:%530=245%:%
%:%531=245%:%
%:%532=246%:%
%:%533=246%:%
%:%534=247%:%
%:%535=247%:%
%:%536=248%:%
%:%537=248%:%
%:%538=249%:%
%:%539=249%:%
%:%540=250%:%
%:%541=250%:%
%:%542=251%:%
%:%543=251%:%
%:%544=252%:%
%:%545=252%:%
%:%546=253%:%
%:%547=253%:%
%:%548=254%:%
%:%549=254%:%
%:%550=255%:%
%:%551=255%:%
%:%552=256%:%
%:%553=256%:%
%:%554=257%:%
%:%555=257%:%
%:%556=258%:%
%:%557=258%:%
%:%558=259%:%
%:%559=259%:%
%:%560=260%:%
%:%561=260%:%
%:%562=261%:%
%:%563=261%:%
%:%564=262%:%
%:%565=262%:%
%:%566=263%:%
%:%567=263%:%
%:%568=264%:%
%:%574=264%:%
%:%577=265%:%
%:%578=266%:%
%:%579=266%:%
%:%580=267%:%
%:%581=268%:%
%:%582=269%:%
%:%585=270%:%
%:%586=271%:%
%:%590=271%:%
%:%591=271%:%
%:%592=272%:%
%:%593=272%:%
%:%594=273%:%
%:%595=273%:%
%:%596=274%:%
%:%597=274%:%
%:%598=275%:%
%:%599=275%:%
%:%600=276%:%
%:%601=276%:%
%:%602=277%:%
%:%603=277%:%
%:%604=278%:%
%:%605=278%:%
%:%606=279%:%
%:%607=279%:%
%:%608=280%:%
%:%609=280%:%
%:%610=281%:%
%:%611=281%:%
%:%612=282%:%
%:%613=282%:%
%:%614=283%:%
%:%615=283%:%
%:%616=284%:%
%:%617=284%:%
%:%618=285%:%
%:%619=285%:%
%:%620=286%:%
%:%621=286%:%
%:%622=287%:%
%:%623=287%:%
%:%624=288%:%
%:%625=288%:%
%:%626=289%:%
%:%627=289%:%
%:%628=290%:%
%:%629=290%:%
%:%630=291%:%
%:%631=291%:%
%:%632=292%:%
%:%633=292%:%
%:%634=293%:%
%:%635=293%:%
%:%636=294%:%
%:%637=294%:%
%:%638=295%:%
%:%639=295%:%
%:%640=296%:%
%:%641=296%:%
%:%642=297%:%
%:%643=297%:%
%:%644=298%:%
%:%645=298%:%
%:%646=299%:%
%:%647=299%:%
%:%648=300%:%
%:%649=300%:%
%:%650=301%:%
%:%651=301%:%
%:%652=302%:%
%:%653=302%:%
%:%654=303%:%
%:%655=303%:%
%:%656=304%:%
%:%657=304%:%
%:%658=305%:%
%:%659=305%:%
%:%660=306%:%
%:%661=306%:%
%:%662=307%:%
%:%663=307%:%
%:%664=308%:%
%:%665=308%:%
%:%666=309%:%
%:%667=309%:%
%:%668=310%:%
%:%669=310%:%
%:%670=311%:%
%:%671=311%:%
%:%672=312%:%
%:%673=312%:%
%:%674=313%:%
%:%675=313%:%
%:%676=314%:%
%:%677=314%:%
%:%678=315%:%
%:%679=315%:%
%:%680=316%:%
%:%681=316%:%
%:%682=317%:%
%:%683=317%:%
%:%684=318%:%
%:%685=318%:%
%:%686=319%:%
%:%687=319%:%
%:%688=320%:%
%:%689=320%:%
%:%690=321%:%
%:%696=321%:%
%:%699=322%:%
%:%700=323%:%
%:%701=323%:%
%:%702=324%:%
%:%703=325%:%
%:%704=326%:%
%:%705=326%:%
%:%706=327%:%
%:%707=328%:%
%:%708=329%:%
%:%715=330%:%
%:%716=330%:%
%:%717=331%:%
%:%718=331%:%
%:%719=332%:%
%:%720=332%:%
%:%721=333%:%
%:%722=333%:%
%:%723=334%:%
%:%724=334%:%
%:%725=335%:%
%:%726=335%:%
%:%727=336%:%
%:%728=336%:%
%:%729=337%:%
%:%730=337%:%
%:%731=338%:%
%:%732=338%:%
%:%733=339%:%
%:%734=340%:%
%:%735=340%:%
%:%736=341%:%
%:%737=341%:%
%:%738=342%:%
%:%739=342%:%
%:%740=343%:%
%:%741=343%:%
%:%742=344%:%
%:%743=344%:%
%:%744=345%:%
%:%745=346%:%
%:%746=346%:%
%:%747=347%:%
%:%748=347%:%
%:%749=348%:%
%:%750=348%:%
%:%751=349%:%
%:%752=349%:%
%:%753=350%:%
%:%754=350%:%
%:%755=351%:%
%:%756=351%:%
%:%757=352%:%
%:%758=352%:%
%:%759=353%:%
%:%760=353%:%
%:%761=354%:%
%:%762=354%:%
%:%763=355%:%
%:%764=356%:%
%:%765=356%:%
%:%766=357%:%
%:%767=357%:%
%:%768=358%:%
%:%769=358%:%
%:%770=359%:%
%:%771=359%:%
%:%772=360%:%
%:%773=360%:%
%:%774=361%:%
%:%775=361%:%
%:%776=362%:%
%:%777=363%:%
%:%778=363%:%
%:%779=363%:%
%:%780=363%:%
%:%781=363%:%
%:%782=364%:%
%:%788=364%:%
%:%791=365%:%
%:%792=366%:%
%:%793=366%:%
%:%794=367%:%
%:%795=368%:%
%:%796=369%:%
%:%803=370%:%
%:%804=370%:%
%:%805=371%:%
%:%806=371%:%
%:%807=372%:%
%:%808=372%:%
%:%809=373%:%
%:%810=374%:%
%:%811=374%:%
%:%812=375%:%
%:%813=376%:%
%:%814=376%:%
%:%815=377%:%
%:%816=377%:%
%:%817=378%:%
%:%818=378%:%
%:%819=379%:%
%:%820=379%:%
%:%821=380%:%
%:%822=380%:%
%:%823=381%:%
%:%824=381%:%
%:%825=382%:%
%:%826=382%:%
%:%827=383%:%
%:%828=383%:%
%:%829=384%:%
%:%830=384%:%
%:%831=385%:%
%:%832=386%:%
%:%833=386%:%
%:%834=387%:%
%:%835=387%:%
%:%836=388%:%
%:%837=388%:%
%:%838=389%:%
%:%839=390%:%
%:%840=390%:%
%:%841=391%:%
%:%842=391%:%
%:%843=392%:%
%:%844=392%:%
%:%845=393%:%
%:%846=393%:%
%:%847=394%:%
%:%848=394%:%
%:%849=395%:%
%:%850=396%:%
%:%851=396%:%
%:%852=396%:%
%:%853=397%:%
%:%854=397%:%
%:%855=398%:%
%:%856=398%:%
%:%857=399%:%
%:%858=400%:%
%:%859=400%:%
%:%860=401%:%
%:%861=402%:%
%:%862=402%:%
%:%863=403%:%
%:%864=403%:%
%:%865=404%:%
%:%866=404%:%
%:%867=405%:%
%:%868=406%:%
%:%869=406%:%
%:%870=406%:%
%:%871=407%:%
%:%872=407%:%
%:%873=408%:%
%:%874=408%:%
%:%875=409%:%
%:%876=410%:%
%:%877=410%:%
%:%878=410%:%
%:%879=410%:%
%:%880=411%:%
%:%881=412%:%
%:%882=412%:%
%:%883=413%:%
%:%884=413%:%
%:%885=414%:%
%:%886=414%:%
%:%887=415%:%
%:%888=415%:%
%:%889=416%:%
%:%890=417%:%
%:%891=417%:%
%:%892=417%:%
%:%893=417%:%
%:%894=418%:%
%:%895=418%:%
%:%896=419%:%
%:%897=419%:%
%:%898=420%:%
%:%899=420%:%
%:%900=421%:%
%:%901=421%:%
%:%902=422%:%
%:%903=422%:%
%:%904=423%:%
%:%905=423%:%
%:%906=424%:%
%:%907=424%:%
%:%908=425%:%
%:%909=425%:%
%:%910=426%:%
%:%911=426%:%
%:%912=427%:%
%:%913=427%:%
%:%914=427%:%
%:%915=428%:%
%:%916=428%:%
%:%917=429%:%
%:%918=429%:%
%:%919=430%:%
%:%920=430%:%
%:%921=431%:%
%:%922=432%:%
%:%923=432%:%
%:%924=433%:%
%:%925=433%:%
%:%926=434%:%
%:%927=434%:%
%:%928=435%:%
%:%929=435%:%
%:%930=436%:%
%:%931=436%:%
%:%932=436%:%
%:%933=436%:%
%:%934=437%:%
%:%935=437%:%
%:%936=437%:%
%:%937=438%:%
%:%938=438%:%
%:%939=439%:%
%:%940=439%:%
%:%941=440%:%
%:%942=440%:%
%:%943=441%:%
%:%944=441%:%
%:%945=442%:%
%:%946=442%:%
%:%947=442%:%
%:%948=443%:%
%:%949=443%:%
%:%950=444%:%
%:%951=444%:%
%:%952=445%:%
%:%953=445%:%
%:%954=446%:%
%:%955=447%:%
%:%956=447%:%
%:%957=447%:%
%:%958=448%:%
%:%959=448%:%
%:%960=449%:%
%:%961=449%:%
%:%962=450%:%
%:%963=450%:%
%:%964=451%:%
%:%965=452%:%
%:%966=452%:%
%:%967=453%:%
%:%968=453%:%
%:%969=454%:%
%:%970=454%:%
%:%971=455%:%
%:%972=455%:%
%:%973=456%:%
%:%974=456%:%
%:%975=457%:%
%:%976=457%:%
%:%977=458%:%
%:%978=458%:%
%:%979=459%:%
%:%980=459%:%
%:%981=460%:%
%:%982=460%:%
%:%983=461%:%
%:%984=461%:%
%:%985=462%:%
%:%986=462%:%
%:%987=463%:%
%:%988=463%:%
%:%989=464%:%
%:%995=464%:%
%:%998=465%:%
%:%999=466%:%
%:%1000=466%:%
%:%1001=467%:%
%:%1002=468%:%
%:%1003=469%:%
%:%1010=470%:%
%:%1011=470%:%
%:%1012=471%:%
%:%1013=471%:%
%:%1014=471%:%
%:%1015=471%:%
%:%1016=472%:%
%:%1017=472%:%
%:%1018=473%:%
%:%1019=473%:%
%:%1020=474%:%
%:%1021=474%:%
%:%1022=475%:%
%:%1023=475%:%
%:%1024=476%:%
%:%1025=476%:%
%:%1026=476%:%
%:%1027=476%:%
%:%1028=476%:%
%:%1029=477%:%
%:%1030=477%:%
%:%1031=477%:%
%:%1032=477%:%
%:%1033=478%:%
%:%1034=478%:%
%:%1035=479%:%
%:%1036=479%:%
%:%1037=480%:%
%:%1038=480%:%
%:%1039=481%:%
%:%1040=481%:%
%:%1041=482%:%
%:%1042=483%:%
%:%1043=483%:%
%:%1044=484%:%
%:%1045=484%:%
%:%1046=485%:%
%:%1047=485%:%
%:%1048=486%:%
%:%1049=486%:%
%:%1050=487%:%
%:%1051=487%:%
%:%1052=487%:%
%:%1053=487%:%
%:%1054=488%:%
%:%1055=488%:%
%:%1056=488%:%
%:%1057=489%:%
%:%1058=489%:%
%:%1059=490%:%
%:%1060=490%:%
%:%1061=491%:%
%:%1062=491%:%
%:%1063=491%:%
%:%1064=491%:%
%:%1065=492%:%
%:%1066=493%:%
%:%1067=493%:%
%:%1068=494%:%
%:%1069=494%:%
%:%1070=495%:%
%:%1071=495%:%
%:%1072=495%:%
%:%1073=496%:%
%:%1074=496%:%
%:%1075=497%:%
%:%1076=497%:%
%:%1077=498%:%
%:%1078=498%:%
%:%1079=499%:%
%:%1080=499%:%
%:%1081=500%:%
%:%1082=500%:%
%:%1083=501%:%
%:%1084=501%:%
%:%1085=501%:%
%:%1086=501%:%
%:%1087=502%:%
%:%1088=502%:%
%:%1089=502%:%
%:%1090=502%:%
%:%1091=502%:%
%:%1092=503%:%
%:%1093=504%:%
%:%1094=504%:%
%:%1095=505%:%
%:%1096=505%:%
%:%1097=506%:%
%:%1098=506%:%
%:%1099=507%:%
%:%1100=507%:%
%:%1101=508%:%
%:%1102=508%:%
%:%1103=508%:%
%:%1104=509%:%
%:%1105=509%:%
%:%1106=510%:%
%:%1107=510%:%
%:%1108=511%:%
%:%1109=511%:%
%:%1110=512%:%
%:%1111=512%:%
%:%1112=513%:%
%:%1113=513%:%
%:%1114=513%:%
%:%1115=513%:%
%:%1116=514%:%
%:%1117=514%:%
%:%1118=514%:%
%:%1119=515%:%
%:%1120=515%:%
%:%1121=516%:%
%:%1122=516%:%
%:%1123=517%:%
%:%1124=517%:%
%:%1125=518%:%
%:%1126=518%:%
%:%1127=518%:%
%:%1128=518%:%
%:%1129=519%:%
%:%1130=519%:%
%:%1131=519%:%
%:%1132=520%:%
%:%1133=520%:%
%:%1134=521%:%
%:%1135=521%:%
%:%1136=522%:%
%:%1137=522%:%
%:%1138=523%:%
%:%1139=523%:%
%:%1140=523%:%
%:%1141=523%:%
%:%1142=523%:%
%:%1143=524%:%
%:%1144=524%:%
%:%1145=524%:%
%:%1146=524%:%
%:%1147=525%:%
%:%1148=526%:%
%:%1149=526%:%
%:%1150=526%:%
%:%1151=527%:%
%:%1152=527%:%
%:%1153=528%:%
%:%1154=528%:%
%:%1155=529%:%
%:%1156=529%:%
%:%1157=530%:%
%:%1158=530%:%
%:%1159=530%:%
%:%1160=530%:%
%:%1161=531%:%
%:%1162=531%:%
%:%1163=532%:%
%:%1169=532%:%
%:%1172=533%:%
%:%1173=534%:%
%:%1174=534%:%
%:%1175=535%:%
%:%1176=536%:%
%:%1177=537%:%
%:%1180=538%:%
%:%1181=539%:%
%:%1185=539%:%
%:%1186=539%:%
%:%1187=540%:%
%:%1188=540%:%
%:%1189=541%:%
%:%1190=542%:%
%:%1191=542%:%
%:%1192=543%:%
%:%1193=543%:%
%:%1194=544%:%
%:%1195=545%:%
%:%1196=545%:%
%:%1197=546%:%
%:%1198=546%:%
%:%1199=547%:%
%:%1200=547%:%
%:%1201=548%:%
%:%1202=548%:%
%:%1203=549%:%
%:%1204=549%:%
%:%1205=549%:%
%:%1206=549%:%
%:%1207=550%:%
%:%1208=551%:%
%:%1209=551%:%
%:%1210=552%:%
%:%1211=553%:%
%:%1212=553%:%
%:%1213=554%:%
%:%1214=554%:%
%:%1215=555%:%
%:%1216=555%:%
%:%1217=556%:%
%:%1218=556%:%
%:%1219=557%:%
%:%1220=558%:%
%:%1221=558%:%
%:%1222=559%:%
%:%1223=559%:%
%:%1224=560%:%
%:%1225=560%:%
%:%1226=561%:%
%:%1227=561%:%
%:%1228=562%:%
%:%1229=562%:%
%:%1230=563%:%
%:%1231=563%:%
%:%1232=564%:%
%:%1233=564%:%
%:%1234=565%:%
%:%1235=565%:%
%:%1236=566%:%
%:%1237=566%:%
%:%1238=567%:%
%:%1239=567%:%
%:%1240=568%:%
%:%1241=568%:%
%:%1242=569%:%
%:%1243=570%:%
%:%1244=570%:%
%:%1245=571%:%
%:%1246=572%:%
%:%1247=572%:%
%:%1248=573%:%
%:%1249=573%:%
%:%1250=574%:%
%:%1251=574%:%
%:%1252=575%:%
%:%1253=575%:%
%:%1254=576%:%
%:%1255=576%:%
%:%1256=577%:%
%:%1257=577%:%
%:%1258=578%:%
%:%1259=578%:%
%:%1260=578%:%
%:%1261=579%:%
%:%1262=579%:%
%:%1263=580%:%
%:%1264=580%:%
%:%1265=581%:%
%:%1266=581%:%
%:%1267=582%:%
%:%1268=582%:%
%:%1269=583%:%
%:%1270=583%:%
%:%1271=584%:%
%:%1272=584%:%
%:%1273=585%:%
%:%1274=585%:%
%:%1275=586%:%
%:%1276=586%:%
%:%1277=587%:%
%:%1278=587%:%
%:%1279=587%:%
%:%1280=587%:%
%:%1281=588%:%
%:%1282=589%:%
%:%1283=589%:%
%:%1284=590%:%
%:%1285=590%:%
%:%1286=591%:%
%:%1287=591%:%
%:%1288=592%:%
%:%1289=592%:%
%:%1290=593%:%
%:%1291=593%:%
%:%1292=593%:%
%:%1293=593%:%
%:%1294=593%:%
%:%1295=594%:%
%:%1296=595%:%
%:%1297=595%:%
%:%1298=596%:%
%:%1299=596%:%
%:%1300=597%:%
%:%1301=597%:%
%:%1302=598%:%
%:%1303=598%:%
%:%1304=599%:%
%:%1305=599%:%
%:%1306=600%:%
%:%1307=600%:%
%:%1308=600%:%
%:%1309=600%:%
%:%1310=601%:%
%:%1311=602%:%
%:%1312=602%:%
%:%1313=602%:%
%:%1314=602%:%
%:%1315=602%:%
%:%1316=603%:%
%:%1317=603%:%
%:%1318=603%:%
%:%1319=603%:%
%:%1320=603%:%
%:%1321=604%:%
%:%1322=604%:%
%:%1323=604%:%
%:%1324=604%:%
%:%1325=604%:%
%:%1326=605%:%
%:%1327=606%:%
%:%1328=606%:%
%:%1329=607%:%
%:%1330=607%:%
%:%1331=608%:%
%:%1332=608%:%
%:%1333=609%:%
%:%1334=609%:%
%:%1335=610%:%
%:%1336=611%:%
%:%1337=611%:%
%:%1338=612%:%
%:%1339=612%:%
%:%1340=613%:%
%:%1341=613%:%
%:%1342=614%:%
%:%1343=614%:%
%:%1344=615%:%
%:%1345=615%:%
%:%1346=615%:%
%:%1347=615%:%
%:%1348=615%:%
%:%1349=616%:%
%:%1350=616%:%
%:%1351=616%:%
%:%1352=617%:%
%:%1353=617%:%
%:%1354=618%:%
%:%1355=618%:%
%:%1356=619%:%
%:%1357=619%:%
%:%1358=620%:%
%:%1359=620%:%
%:%1360=620%:%
%:%1361=621%:%
%:%1362=621%:%
%:%1363=622%:%
%:%1364=622%:%
%:%1365=623%:%
%:%1366=623%:%
%:%1367=624%:%
%:%1368=624%:%
%:%1369=625%:%
%:%1370=625%:%
%:%1371=626%:%
%:%1372=626%:%
%:%1373=627%:%
%:%1379=627%:%
%:%1382=628%:%
%:%1383=629%:%
%:%1384=630%:%
%:%1385=630%:%
%:%1392=631%:%

%
\begin{isabellebody}%
\setisabellecontext{SymExt{\isacharunderscore}{\kern0pt}ZF}%
%
\isadelimtheory
%
\endisadelimtheory
%
\isatagtheory
\isacommand{theory}\isamarkupfalse%
\ SymExt{\isacharunderscore}{\kern0pt}ZF\isanewline
\ \ \isakeyword{imports}\ SymExt{\isacharunderscore}{\kern0pt}Replacement\isanewline
\isakeyword{begin}%
\endisatagtheory
{\isafoldtheory}%
%
\isadelimtheory
\isanewline
%
\endisadelimtheory
\isanewline
\isacommand{context}\isamarkupfalse%
\ M{\isacharunderscore}{\kern0pt}symmetric{\isacharunderscore}{\kern0pt}system{\isacharunderscore}{\kern0pt}G{\isacharunderscore}{\kern0pt}generic\ \isakeyword{begin}\ \isanewline
\isanewline
\isacommand{lemma}\isamarkupfalse%
\ zero{\isacharunderscore}{\kern0pt}in{\isacharunderscore}{\kern0pt}SymExt\ {\isacharcolon}{\kern0pt}\ {\isachardoublequoteopen}{\isadigit{0}}\ {\isasymin}\ SymExt{\isacharparenleft}{\kern0pt}G{\isacharparenright}{\kern0pt}{\isachardoublequoteclose}\ \isanewline
%
\isadelimproof
\ \ %
\endisadelimproof
%
\isatagproof
\isacommand{using}\isamarkupfalse%
\ zero{\isacharunderscore}{\kern0pt}in{\isacharunderscore}{\kern0pt}M\ M{\isacharunderscore}{\kern0pt}subset{\isacharunderscore}{\kern0pt}SymExt\ \isacommand{by}\isamarkupfalse%
\ auto%
\endisatagproof
{\isafoldproof}%
%
\isadelimproof
\isanewline
%
\endisadelimproof
\isanewline
\isacommand{lemma}\isamarkupfalse%
\ upair{\isacharunderscore}{\kern0pt}in{\isacharunderscore}{\kern0pt}SymExt\ {\isacharcolon}{\kern0pt}\ \isanewline
\ \ \isakeyword{fixes}\ a\ b\ \isanewline
\ \ \isakeyword{assumes}\ {\isachardoublequoteopen}a\ {\isasymin}\ SymExt{\isacharparenleft}{\kern0pt}G{\isacharparenright}{\kern0pt}{\isachardoublequoteclose}\ {\isachardoublequoteopen}b\ {\isasymin}\ SymExt{\isacharparenleft}{\kern0pt}G{\isacharparenright}{\kern0pt}{\isachardoublequoteclose}\ \isanewline
\ \ \isakeyword{shows}\ {\isachardoublequoteopen}{\isacharbraceleft}{\kern0pt}a{\isacharcomma}{\kern0pt}\ b{\isacharbraceright}{\kern0pt}\ {\isasymin}\ SymExt{\isacharparenleft}{\kern0pt}G{\isacharparenright}{\kern0pt}{\isachardoublequoteclose}\ \isanewline
%
\isadelimproof
%
\endisadelimproof
%
\isatagproof
\isacommand{proof}\isamarkupfalse%
\ {\isacharminus}{\kern0pt}\ \isanewline
\ \ \isacommand{obtain}\isamarkupfalse%
\ a{\isacharprime}{\kern0pt}\ b{\isacharprime}{\kern0pt}\ \isakeyword{where}\ a{\isacharprime}{\kern0pt}b{\isacharprime}{\kern0pt}H{\isacharcolon}{\kern0pt}\ {\isachardoublequoteopen}a{\isacharprime}{\kern0pt}\ {\isasymin}\ HS{\isachardoublequoteclose}\ {\isachardoublequoteopen}b{\isacharprime}{\kern0pt}\ {\isasymin}\ HS{\isachardoublequoteclose}\ {\isachardoublequoteopen}val{\isacharparenleft}{\kern0pt}G{\isacharcomma}{\kern0pt}\ a{\isacharprime}{\kern0pt}{\isacharparenright}{\kern0pt}\ {\isacharequal}{\kern0pt}\ a{\isachardoublequoteclose}\ {\isachardoublequoteopen}val{\isacharparenleft}{\kern0pt}G{\isacharcomma}{\kern0pt}\ b{\isacharprime}{\kern0pt}{\isacharparenright}{\kern0pt}\ {\isacharequal}{\kern0pt}\ b{\isachardoublequoteclose}\ \isanewline
\ \ \ \ \isacommand{using}\isamarkupfalse%
\ assms\ SymExt{\isacharunderscore}{\kern0pt}def\isanewline
\ \ \ \ \isacommand{by}\isamarkupfalse%
\ auto\isanewline
\ \ \isacommand{have}\isamarkupfalse%
\ {\isachardoublequoteopen}{\isasymexists}S\ {\isasymin}\ SymExt{\isacharparenleft}{\kern0pt}G{\isacharparenright}{\kern0pt}{\isachardot}{\kern0pt}\ {\isacharbraceleft}{\kern0pt}\ val{\isacharparenleft}{\kern0pt}G{\isacharcomma}{\kern0pt}\ x{\isacharparenright}{\kern0pt}{\isachardot}{\kern0pt}\ x\ {\isasymin}\ {\isacharbraceleft}{\kern0pt}a{\isacharprime}{\kern0pt}{\isacharcomma}{\kern0pt}\ b{\isacharprime}{\kern0pt}{\isacharbraceright}{\kern0pt}\ {\isacharbraceright}{\kern0pt}\ {\isasymsubseteq}\ S{\isachardoublequoteclose}\isanewline
\ \ \ \ \isacommand{apply}\isamarkupfalse%
{\isacharparenleft}{\kern0pt}rule\ ex{\isacharunderscore}{\kern0pt}separation{\isacharunderscore}{\kern0pt}base{\isacharparenright}{\kern0pt}\isanewline
\ \ \ \ \isacommand{apply}\isamarkupfalse%
{\isacharparenleft}{\kern0pt}subgoal{\isacharunderscore}{\kern0pt}tac\ {\isachardoublequoteopen}a{\isacharprime}{\kern0pt}\ {\isasymin}\ M\ {\isasymand}\ b{\isacharprime}{\kern0pt}\ {\isasymin}\ M{\isachardoublequoteclose}{\isacharparenright}{\kern0pt}\isanewline
\ \ \ \ \isacommand{using}\isamarkupfalse%
\ upair{\isacharunderscore}{\kern0pt}ax\ upair{\isacharunderscore}{\kern0pt}abs\ HS{\isacharunderscore}{\kern0pt}iff\ P{\isacharunderscore}{\kern0pt}name{\isacharunderscore}{\kern0pt}in{\isacharunderscore}{\kern0pt}M\ a{\isacharprime}{\kern0pt}b{\isacharprime}{\kern0pt}H\isanewline
\ \ \ \ \isacommand{unfolding}\isamarkupfalse%
\ upair{\isacharunderscore}{\kern0pt}ax{\isacharunderscore}{\kern0pt}def\ \isanewline
\ \ \ \ \isacommand{by}\isamarkupfalse%
\ auto\isanewline
\ \ \isacommand{then}\isamarkupfalse%
\ \isacommand{obtain}\isamarkupfalse%
\ S\ \isakeyword{where}\ SH\ {\isacharcolon}{\kern0pt}\ {\isachardoublequoteopen}S\ {\isasymin}\ SymExt{\isacharparenleft}{\kern0pt}G{\isacharparenright}{\kern0pt}{\isachardoublequoteclose}\ {\isachardoublequoteopen}{\isacharbraceleft}{\kern0pt}\ a{\isacharcomma}{\kern0pt}\ b\ {\isacharbraceright}{\kern0pt}\ {\isasymsubseteq}\ S{\isachardoublequoteclose}\ \isacommand{using}\isamarkupfalse%
\ a{\isacharprime}{\kern0pt}b{\isacharprime}{\kern0pt}H\ \isacommand{by}\isamarkupfalse%
\ auto\isanewline
\isanewline
\ \ \isacommand{define}\isamarkupfalse%
\ X\ \isakeyword{where}\ {\isachardoublequoteopen}X\ {\isasymequiv}\ {\isacharbraceleft}{\kern0pt}\ x\ {\isasymin}\ S{\isachardot}{\kern0pt}\ sats{\isacharparenleft}{\kern0pt}SymExt{\isacharparenleft}{\kern0pt}G{\isacharparenright}{\kern0pt}{\isacharcomma}{\kern0pt}\ Or{\isacharparenleft}{\kern0pt}Equal{\isacharparenleft}{\kern0pt}{\isadigit{0}}{\isacharcomma}{\kern0pt}\ {\isadigit{1}}{\isacharparenright}{\kern0pt}{\isacharcomma}{\kern0pt}\ Equal{\isacharparenleft}{\kern0pt}{\isadigit{0}}{\isacharcomma}{\kern0pt}\ {\isadigit{2}}{\isacharparenright}{\kern0pt}{\isacharparenright}{\kern0pt}{\isacharcomma}{\kern0pt}\ {\isacharbrackleft}{\kern0pt}x{\isacharbrackright}{\kern0pt}\ {\isacharat}{\kern0pt}\ {\isacharbrackleft}{\kern0pt}a{\isacharcomma}{\kern0pt}\ b{\isacharbrackright}{\kern0pt}\ {\isacharparenright}{\kern0pt}\ {\isacharbraceright}{\kern0pt}{\isachardoublequoteclose}\ \isanewline
\ \ \isacommand{have}\isamarkupfalse%
\ Xin\ {\isacharcolon}{\kern0pt}\ {\isachardoublequoteopen}X\ {\isasymin}\ SymExt{\isacharparenleft}{\kern0pt}G{\isacharparenright}{\kern0pt}{\isachardoublequoteclose}\ \isanewline
\ \ \ \ \isacommand{unfolding}\isamarkupfalse%
\ X{\isacharunderscore}{\kern0pt}def\ \isanewline
\ \ \ \ \isacommand{apply}\isamarkupfalse%
{\isacharparenleft}{\kern0pt}rule\ SymExt{\isacharunderscore}{\kern0pt}separation{\isacharparenright}{\kern0pt}\isanewline
\ \ \ \ \isacommand{using}\isamarkupfalse%
\ SH\ assms\ \isanewline
\ \ \ \ \ \ \ \isacommand{apply}\isamarkupfalse%
\ auto{\isacharbrackleft}{\kern0pt}{\isadigit{3}}{\isacharbrackright}{\kern0pt}\isanewline
\ \ \ \ \isacommand{apply}\isamarkupfalse%
{\isacharparenleft}{\kern0pt}simp\ del{\isacharcolon}{\kern0pt}FOL{\isacharunderscore}{\kern0pt}sats{\isacharunderscore}{\kern0pt}iff\ pair{\isacharunderscore}{\kern0pt}abs\ add{\isacharcolon}{\kern0pt}\ fm{\isacharunderscore}{\kern0pt}defs\ nat{\isacharunderscore}{\kern0pt}simp{\isacharunderscore}{\kern0pt}union{\isacharparenright}{\kern0pt}\ \isanewline
\ \ \ \ \isacommand{done}\isamarkupfalse%
\isanewline
\ \ \isacommand{have}\isamarkupfalse%
\ {\isachardoublequoteopen}X\ {\isacharequal}{\kern0pt}\ {\isacharbraceleft}{\kern0pt}\ x\ {\isasymin}\ S{\isachardot}{\kern0pt}\ x\ {\isacharequal}{\kern0pt}\ a\ {\isasymor}\ x\ {\isacharequal}{\kern0pt}\ b\ {\isacharbraceright}{\kern0pt}{\isachardoublequoteclose}\ \isanewline
\ \ \ \ \isacommand{unfolding}\isamarkupfalse%
\ X{\isacharunderscore}{\kern0pt}def\ \isanewline
\ \ \ \ \isacommand{apply}\isamarkupfalse%
{\isacharparenleft}{\kern0pt}rule\ iff{\isacharunderscore}{\kern0pt}eq{\isacharparenright}{\kern0pt}\isanewline
\ \ \ \ \isacommand{apply}\isamarkupfalse%
{\isacharparenleft}{\kern0pt}rename{\isacharunderscore}{\kern0pt}tac\ x{\isacharcomma}{\kern0pt}\ subgoal{\isacharunderscore}{\kern0pt}tac\ {\isachardoublequoteopen}x\ {\isasymin}\ SymExt{\isacharparenleft}{\kern0pt}G{\isacharparenright}{\kern0pt}{\isachardoublequoteclose}{\isacharparenright}{\kern0pt}\ \isanewline
\ \ \ \ \isacommand{using}\isamarkupfalse%
\ assms\ SymExt{\isacharunderscore}{\kern0pt}trans\ SH\isanewline
\ \ \ \ \isacommand{by}\isamarkupfalse%
\ auto\isanewline
\ \ \isacommand{also}\isamarkupfalse%
\ \isacommand{have}\isamarkupfalse%
\ {\isachardoublequoteopen}{\isachardot}{\kern0pt}{\isachardot}{\kern0pt}{\isachardot}{\kern0pt}\ {\isacharequal}{\kern0pt}\ {\isacharbraceleft}{\kern0pt}\ a{\isacharcomma}{\kern0pt}\ b\ {\isacharbraceright}{\kern0pt}{\isachardoublequoteclose}\ \isacommand{using}\isamarkupfalse%
\ SH\ \isacommand{by}\isamarkupfalse%
\ auto\isanewline
\ \ \isacommand{finally}\isamarkupfalse%
\ \isacommand{show}\isamarkupfalse%
\ {\isachardoublequoteopen}{\isacharbraceleft}{\kern0pt}a{\isacharcomma}{\kern0pt}\ b{\isacharbraceright}{\kern0pt}\ {\isasymin}\ SymExt{\isacharparenleft}{\kern0pt}G{\isacharparenright}{\kern0pt}{\isachardoublequoteclose}\ \isacommand{using}\isamarkupfalse%
\ Xin\ \isacommand{by}\isamarkupfalse%
\ auto\isanewline
\isacommand{qed}\isamarkupfalse%
%
\endisatagproof
{\isafoldproof}%
%
\isadelimproof
\isanewline
%
\endisadelimproof
\isanewline
\isacommand{lemma}\isamarkupfalse%
\ Union{\isacharunderscore}{\kern0pt}in{\isacharunderscore}{\kern0pt}SymExt\ {\isacharcolon}{\kern0pt}\ \isanewline
\ \ \isakeyword{fixes}\ x\ \isanewline
\ \ \isakeyword{assumes}\ {\isachardoublequoteopen}x\ {\isasymin}\ SymExt{\isacharparenleft}{\kern0pt}G{\isacharparenright}{\kern0pt}{\isachardoublequoteclose}\ \isanewline
\ \ \isakeyword{shows}\ {\isachardoublequoteopen}{\isasymUnion}x\ {\isasymin}\ SymExt{\isacharparenleft}{\kern0pt}G{\isacharparenright}{\kern0pt}{\isachardoublequoteclose}\ \isanewline
%
\isadelimproof
%
\endisadelimproof
%
\isatagproof
\isacommand{proof}\isamarkupfalse%
\ {\isacharminus}{\kern0pt}\ \isanewline
\ \ \isacommand{obtain}\isamarkupfalse%
\ x{\isacharprime}{\kern0pt}\ \isakeyword{where}\ x{\isacharprime}{\kern0pt}H{\isacharcolon}{\kern0pt}\ {\isachardoublequoteopen}x{\isacharprime}{\kern0pt}\ {\isasymin}\ HS{\isachardoublequoteclose}\ {\isachardoublequoteopen}val{\isacharparenleft}{\kern0pt}G{\isacharcomma}{\kern0pt}\ x{\isacharprime}{\kern0pt}{\isacharparenright}{\kern0pt}\ {\isacharequal}{\kern0pt}\ x{\isachardoublequoteclose}\ \isacommand{using}\isamarkupfalse%
\ assms\ SymExt{\isacharunderscore}{\kern0pt}def\ \isacommand{by}\isamarkupfalse%
\ auto\isanewline
\ \ \isacommand{define}\isamarkupfalse%
\ X\ \isakeyword{where}\ {\isachardoublequoteopen}X\ {\isasymequiv}\ domain{\isacharparenleft}{\kern0pt}{\isasymUnion}{\isacharparenleft}{\kern0pt}domain{\isacharparenleft}{\kern0pt}x{\isacharprime}{\kern0pt}{\isacharparenright}{\kern0pt}{\isacharparenright}{\kern0pt}{\isacharparenright}{\kern0pt}{\isachardoublequoteclose}\ \isanewline
\isanewline
\ \ \isacommand{have}\isamarkupfalse%
\ {\isachardoublequoteopen}X\ {\isasymsubseteq}\ HS{\isachardoublequoteclose}\ \isanewline
\ \ \isacommand{proof}\isamarkupfalse%
{\isacharparenleft}{\kern0pt}rule\ subsetI{\isacharparenright}{\kern0pt}\isanewline
\ \ \ \ \isacommand{fix}\isamarkupfalse%
\ z\ \isacommand{assume}\isamarkupfalse%
\ xin\ {\isacharcolon}{\kern0pt}\ {\isachardoublequoteopen}z\ {\isasymin}\ X{\isachardoublequoteclose}\ \isanewline
\ \ \ \ \isacommand{then}\isamarkupfalse%
\ \isacommand{obtain}\isamarkupfalse%
\ p\ \isakeyword{where}\ {\isachardoublequoteopen}{\isacharless}{\kern0pt}z{\isacharcomma}{\kern0pt}\ p{\isachargreater}{\kern0pt}\ {\isasymin}\ {\isasymUnion}{\isacharparenleft}{\kern0pt}domain{\isacharparenleft}{\kern0pt}x{\isacharprime}{\kern0pt}{\isacharparenright}{\kern0pt}{\isacharparenright}{\kern0pt}{\isachardoublequoteclose}\ \isacommand{using}\isamarkupfalse%
\ X{\isacharunderscore}{\kern0pt}def\ \isacommand{by}\isamarkupfalse%
\ auto\isanewline
\ \ \ \ \isacommand{then}\isamarkupfalse%
\ \isacommand{obtain}\isamarkupfalse%
\ y\ \isakeyword{where}\ yH{\isacharcolon}{\kern0pt}\ {\isachardoublequoteopen}y\ {\isasymin}\ domain{\isacharparenleft}{\kern0pt}x{\isacharprime}{\kern0pt}{\isacharparenright}{\kern0pt}{\isachardoublequoteclose}\ {\isachardoublequoteopen}{\isacharless}{\kern0pt}z{\isacharcomma}{\kern0pt}\ p{\isachargreater}{\kern0pt}\ {\isasymin}\ y{\isachardoublequoteclose}\ \isacommand{by}\isamarkupfalse%
\ auto\ \isanewline
\ \ \ \ \isacommand{then}\isamarkupfalse%
\ \isacommand{have}\isamarkupfalse%
\ {\isachardoublequoteopen}y\ {\isasymin}\ HS{\isachardoublequoteclose}\ \isacommand{using}\isamarkupfalse%
\ HS{\isacharunderscore}{\kern0pt}iff\ x{\isacharprime}{\kern0pt}H\ \isacommand{by}\isamarkupfalse%
\ auto\isanewline
\ \ \ \ \isacommand{then}\isamarkupfalse%
\ \isacommand{show}\isamarkupfalse%
\ {\isachardoublequoteopen}z\ {\isasymin}\ HS{\isachardoublequoteclose}\ \isacommand{using}\isamarkupfalse%
\ HS{\isacharunderscore}{\kern0pt}iff\ yH\ \isacommand{by}\isamarkupfalse%
\ auto\ \isanewline
\ \ \isacommand{qed}\isamarkupfalse%
\isanewline
\ \ \isacommand{then}\isamarkupfalse%
\ \isacommand{have}\isamarkupfalse%
\ {\isachardoublequoteopen}{\isasymexists}\ S\ {\isasymin}\ SymExt{\isacharparenleft}{\kern0pt}G{\isacharparenright}{\kern0pt}{\isachardot}{\kern0pt}\ \ {\isacharbraceleft}{\kern0pt}\ val{\isacharparenleft}{\kern0pt}G{\isacharcomma}{\kern0pt}\ x{\isacharparenright}{\kern0pt}{\isachardot}{\kern0pt}\ x\ {\isasymin}\ X\ {\isacharbraceright}{\kern0pt}\ {\isasymsubseteq}\ S{\isachardoublequoteclose}\ \isanewline
\ \ \ \ \isacommand{apply}\isamarkupfalse%
{\isacharparenleft}{\kern0pt}rule\ ex{\isacharunderscore}{\kern0pt}separation{\isacharunderscore}{\kern0pt}base{\isacharparenright}{\kern0pt}\isanewline
\ \ \ \ \isacommand{unfolding}\isamarkupfalse%
\ X{\isacharunderscore}{\kern0pt}def\isanewline
\ \ \ \ \isacommand{using}\isamarkupfalse%
\ domain{\isacharunderscore}{\kern0pt}closed\ Union{\isacharunderscore}{\kern0pt}closed\ HS{\isacharunderscore}{\kern0pt}iff\ P{\isacharunderscore}{\kern0pt}name{\isacharunderscore}{\kern0pt}in{\isacharunderscore}{\kern0pt}M\ x{\isacharprime}{\kern0pt}H\ \isanewline
\ \ \ \ \isacommand{by}\isamarkupfalse%
\ auto\isanewline
\ \ \isacommand{then}\isamarkupfalse%
\ \isacommand{obtain}\isamarkupfalse%
\ S\ \isakeyword{where}\ SH\ {\isacharcolon}{\kern0pt}\ {\isachardoublequoteopen}S\ {\isasymin}\ SymExt{\isacharparenleft}{\kern0pt}G{\isacharparenright}{\kern0pt}{\isachardoublequoteclose}\ {\isachardoublequoteopen}{\isacharbraceleft}{\kern0pt}\ val{\isacharparenleft}{\kern0pt}G{\isacharcomma}{\kern0pt}\ x{\isacharparenright}{\kern0pt}{\isachardot}{\kern0pt}\ x\ {\isasymin}\ X\ {\isacharbraceright}{\kern0pt}\ {\isasymsubseteq}\ S{\isachardoublequoteclose}\ \isacommand{by}\isamarkupfalse%
\ auto\isanewline
\ \ \isacommand{have}\isamarkupfalse%
\ subsetS\ {\isacharcolon}{\kern0pt}\ {\isachardoublequoteopen}{\isasymUnion}x\ {\isasymsubseteq}\ S{\isachardoublequoteclose}\ \isanewline
\ \ \isacommand{proof}\isamarkupfalse%
{\isacharparenleft}{\kern0pt}rule\ subsetI{\isacharparenright}{\kern0pt}\isanewline
\ \ \ \ \isacommand{fix}\isamarkupfalse%
\ z\ \isanewline
\ \ \ \ \isacommand{assume}\isamarkupfalse%
\ zin\ {\isacharcolon}{\kern0pt}\ {\isachardoublequoteopen}z\ {\isasymin}\ {\isasymUnion}x{\isachardoublequoteclose}\ \isanewline
\ \ \ \ \isacommand{then}\isamarkupfalse%
\ \isacommand{obtain}\isamarkupfalse%
\ y\ \isakeyword{where}\ yH\ {\isacharcolon}{\kern0pt}\ {\isachardoublequoteopen}z\ {\isasymin}\ y{\isachardoublequoteclose}\ {\isachardoublequoteopen}y\ {\isasymin}\ x{\isachardoublequoteclose}\ \isacommand{by}\isamarkupfalse%
\ auto\isanewline
\ \ \ \ \isacommand{then}\isamarkupfalse%
\ \isacommand{have}\isamarkupfalse%
\ yin\ {\isacharcolon}{\kern0pt}\ {\isachardoublequoteopen}y\ {\isasymin}\ val{\isacharparenleft}{\kern0pt}G{\isacharcomma}{\kern0pt}\ x{\isacharprime}{\kern0pt}{\isacharparenright}{\kern0pt}{\isachardoublequoteclose}\ \isacommand{using}\isamarkupfalse%
\ x{\isacharprime}{\kern0pt}H\ \isacommand{by}\isamarkupfalse%
\ auto\isanewline
\ \ \ \ \isacommand{then}\isamarkupfalse%
\ \isacommand{have}\isamarkupfalse%
\ {\isachardoublequoteopen}{\isasymexists}y{\isacharprime}{\kern0pt}\ {\isasymin}\ domain{\isacharparenleft}{\kern0pt}x{\isacharprime}{\kern0pt}{\isacharparenright}{\kern0pt}{\isachardot}{\kern0pt}\ val{\isacharparenleft}{\kern0pt}G{\isacharcomma}{\kern0pt}\ y{\isacharprime}{\kern0pt}{\isacharparenright}{\kern0pt}\ {\isacharequal}{\kern0pt}\ y{\isachardoublequoteclose}\ \isanewline
\ \ \ \ \ \ \isacommand{apply}\isamarkupfalse%
{\isacharparenleft}{\kern0pt}rule{\isacharunderscore}{\kern0pt}tac\ P{\isacharequal}{\kern0pt}{\isachardoublequoteopen}y\ {\isasymin}\ val{\isacharparenleft}{\kern0pt}G{\isacharcomma}{\kern0pt}\ x{\isacharprime}{\kern0pt}{\isacharparenright}{\kern0pt}{\isachardoublequoteclose}\ \isakeyword{in}\ mp{\isacharparenright}{\kern0pt}\isanewline
\ \ \ \ \ \ \isacommand{apply}\isamarkupfalse%
{\isacharparenleft}{\kern0pt}subst\ def{\isacharunderscore}{\kern0pt}val{\isacharcomma}{\kern0pt}\ force{\isacharparenright}{\kern0pt}\isanewline
\ \ \ \ \ \ \isacommand{using}\isamarkupfalse%
\ x{\isacharprime}{\kern0pt}H\ yH\ HS{\isacharunderscore}{\kern0pt}iff\isanewline
\ \ \ \ \ \ \isacommand{by}\isamarkupfalse%
\ auto\isanewline
\ \ \ \ \isacommand{then}\isamarkupfalse%
\ \isacommand{obtain}\isamarkupfalse%
\ y{\isacharprime}{\kern0pt}\ \isakeyword{where}\ y{\isacharprime}{\kern0pt}H{\isacharcolon}{\kern0pt}\ {\isachardoublequoteopen}y{\isacharprime}{\kern0pt}\ {\isasymin}\ domain{\isacharparenleft}{\kern0pt}x{\isacharprime}{\kern0pt}{\isacharparenright}{\kern0pt}{\isachardoublequoteclose}\ {\isachardoublequoteopen}val{\isacharparenleft}{\kern0pt}G{\isacharcomma}{\kern0pt}\ y{\isacharprime}{\kern0pt}{\isacharparenright}{\kern0pt}\ {\isacharequal}{\kern0pt}\ y{\isachardoublequoteclose}\ \isacommand{using}\isamarkupfalse%
\ yin\ \isacommand{by}\isamarkupfalse%
\ auto\ \isanewline
\ \ \ \ \isacommand{then}\isamarkupfalse%
\ \isacommand{have}\isamarkupfalse%
\ {\isachardoublequoteopen}{\isasymexists}z{\isacharprime}{\kern0pt}\ {\isasymin}\ domain{\isacharparenleft}{\kern0pt}y{\isacharprime}{\kern0pt}{\isacharparenright}{\kern0pt}{\isachardot}{\kern0pt}\ val{\isacharparenleft}{\kern0pt}G{\isacharcomma}{\kern0pt}\ z{\isacharprime}{\kern0pt}{\isacharparenright}{\kern0pt}\ {\isacharequal}{\kern0pt}\ z{\isachardoublequoteclose}\ \isanewline
\ \ \ \ \ \ \isacommand{apply}\isamarkupfalse%
{\isacharparenleft}{\kern0pt}rule{\isacharunderscore}{\kern0pt}tac\ P{\isacharequal}{\kern0pt}{\isachardoublequoteopen}z\ {\isasymin}\ val{\isacharparenleft}{\kern0pt}G{\isacharcomma}{\kern0pt}\ y{\isacharprime}{\kern0pt}{\isacharparenright}{\kern0pt}{\isachardoublequoteclose}\ \isakeyword{in}\ mp{\isacharparenright}{\kern0pt}\isanewline
\ \ \ \ \ \ \isacommand{apply}\isamarkupfalse%
{\isacharparenleft}{\kern0pt}subst\ def{\isacharunderscore}{\kern0pt}val{\isacharcomma}{\kern0pt}\ force{\isacharparenright}{\kern0pt}\isanewline
\ \ \ \ \ \ \isacommand{using}\isamarkupfalse%
\ y{\isacharprime}{\kern0pt}H\ HS{\isacharunderscore}{\kern0pt}iff\ x{\isacharprime}{\kern0pt}H\ yH\ \isanewline
\ \ \ \ \ \ \isacommand{by}\isamarkupfalse%
\ auto\isanewline
\ \ \ \ \isacommand{then}\isamarkupfalse%
\ \isacommand{obtain}\isamarkupfalse%
\ z{\isacharprime}{\kern0pt}\ p\ \isakeyword{where}\ z{\isacharprime}{\kern0pt}H\ {\isacharcolon}{\kern0pt}\ {\isachardoublequoteopen}val{\isacharparenleft}{\kern0pt}G{\isacharcomma}{\kern0pt}\ z{\isacharprime}{\kern0pt}{\isacharparenright}{\kern0pt}\ {\isacharequal}{\kern0pt}\ z{\isachardoublequoteclose}\ {\isachardoublequoteopen}{\isacharless}{\kern0pt}z{\isacharprime}{\kern0pt}{\isacharcomma}{\kern0pt}\ p{\isachargreater}{\kern0pt}\ {\isasymin}\ y{\isacharprime}{\kern0pt}{\isachardoublequoteclose}\ \isacommand{by}\isamarkupfalse%
\ auto\isanewline
\ \ \ \ \isacommand{then}\isamarkupfalse%
\ \isacommand{have}\isamarkupfalse%
\ {\isachardoublequoteopen}z{\isacharprime}{\kern0pt}\ {\isasymin}\ X{\isachardoublequoteclose}\ \isanewline
\ \ \ \ \ \ \isacommand{unfolding}\isamarkupfalse%
\ X{\isacharunderscore}{\kern0pt}def\ \isanewline
\ \ \ \ \ \ \isacommand{apply}\isamarkupfalse%
{\isacharparenleft}{\kern0pt}rule{\isacharunderscore}{\kern0pt}tac\ b{\isacharequal}{\kern0pt}p\ \isakeyword{in}\ domainI{\isacharparenright}{\kern0pt}\isanewline
\ \ \ \ \ \ \isacommand{apply}\isamarkupfalse%
{\isacharparenleft}{\kern0pt}rule{\isacharunderscore}{\kern0pt}tac\ B{\isacharequal}{\kern0pt}y{\isacharprime}{\kern0pt}\ \isakeyword{in}\ UnionI{\isacharparenright}{\kern0pt}\isanewline
\ \ \ \ \ \ \isacommand{using}\isamarkupfalse%
\ y{\isacharprime}{\kern0pt}H\ z{\isacharprime}{\kern0pt}H\ \isanewline
\ \ \ \ \ \ \isacommand{by}\isamarkupfalse%
\ auto\isanewline
\ \ \ \ \isacommand{then}\isamarkupfalse%
\ \isacommand{have}\isamarkupfalse%
\ {\isachardoublequoteopen}val{\isacharparenleft}{\kern0pt}G{\isacharcomma}{\kern0pt}\ z{\isacharprime}{\kern0pt}{\isacharparenright}{\kern0pt}\ {\isasymin}\ S{\isachardoublequoteclose}\ \isacommand{using}\isamarkupfalse%
\ SH\ \isacommand{by}\isamarkupfalse%
\ auto\isanewline
\ \ \ \ \isacommand{then}\isamarkupfalse%
\ \isacommand{show}\isamarkupfalse%
\ {\isachardoublequoteopen}z\ {\isasymin}\ S{\isachardoublequoteclose}\ \isacommand{using}\isamarkupfalse%
\ z{\isacharprime}{\kern0pt}H\ \isacommand{by}\isamarkupfalse%
\ auto\isanewline
\ \ \isacommand{qed}\isamarkupfalse%
\isanewline
\isanewline
\ \ \isacommand{define}\isamarkupfalse%
\ T\ \isakeyword{where}\ {\isachardoublequoteopen}T\ {\isasymequiv}\ {\isacharbraceleft}{\kern0pt}\ z\ {\isasymin}\ S{\isachardot}{\kern0pt}\ sats{\isacharparenleft}{\kern0pt}SymExt{\isacharparenleft}{\kern0pt}G{\isacharparenright}{\kern0pt}{\isacharcomma}{\kern0pt}\ Exists{\isacharparenleft}{\kern0pt}And{\isacharparenleft}{\kern0pt}Member{\isacharparenleft}{\kern0pt}{\isadigit{1}}{\isacharcomma}{\kern0pt}\ {\isadigit{0}}{\isacharparenright}{\kern0pt}{\isacharcomma}{\kern0pt}\ Member{\isacharparenleft}{\kern0pt}{\isadigit{0}}{\isacharcomma}{\kern0pt}\ {\isadigit{2}}{\isacharparenright}{\kern0pt}{\isacharparenright}{\kern0pt}{\isacharparenright}{\kern0pt}{\isacharcomma}{\kern0pt}\ {\isacharbrackleft}{\kern0pt}z{\isacharbrackright}{\kern0pt}\ {\isacharat}{\kern0pt}\ {\isacharbrackleft}{\kern0pt}x{\isacharbrackright}{\kern0pt}{\isacharparenright}{\kern0pt}\ {\isacharbraceright}{\kern0pt}{\isachardoublequoteclose}\ \isanewline
\ \ \isacommand{have}\isamarkupfalse%
\ Tin\ {\isacharcolon}{\kern0pt}\ {\isachardoublequoteopen}T\ {\isasymin}\ SymExt{\isacharparenleft}{\kern0pt}G{\isacharparenright}{\kern0pt}{\isachardoublequoteclose}\ \isanewline
\ \ \ \ \isacommand{unfolding}\isamarkupfalse%
\ T{\isacharunderscore}{\kern0pt}def\ \isanewline
\ \ \ \ \isacommand{apply}\isamarkupfalse%
{\isacharparenleft}{\kern0pt}rule\ SymExt{\isacharunderscore}{\kern0pt}separation{\isacharparenright}{\kern0pt}\isanewline
\ \ \ \ \isacommand{using}\isamarkupfalse%
\ SH\ assms\ \isanewline
\ \ \ \ \ \ \ \isacommand{apply}\isamarkupfalse%
\ auto{\isacharbrackleft}{\kern0pt}{\isadigit{3}}{\isacharbrackright}{\kern0pt}\isanewline
\ \ \ \ \isacommand{unfolding}\isamarkupfalse%
\ BExists{\isacharunderscore}{\kern0pt}def\ BExists{\isacharprime}{\kern0pt}{\isacharunderscore}{\kern0pt}def\ \isanewline
\ \ \ \ \isacommand{apply}\isamarkupfalse%
{\isacharparenleft}{\kern0pt}simp\ del{\isacharcolon}{\kern0pt}FOL{\isacharunderscore}{\kern0pt}sats{\isacharunderscore}{\kern0pt}iff\ pair{\isacharunderscore}{\kern0pt}abs\ add{\isacharcolon}{\kern0pt}\ fm{\isacharunderscore}{\kern0pt}defs\ nat{\isacharunderscore}{\kern0pt}simp{\isacharunderscore}{\kern0pt}union{\isacharparenright}{\kern0pt}\ \isanewline
\ \ \ \ \isacommand{done}\isamarkupfalse%
\isanewline
\isanewline
\ \ \isacommand{have}\isamarkupfalse%
\ {\isachardoublequoteopen}T\ {\isacharequal}{\kern0pt}\ {\isacharbraceleft}{\kern0pt}\ z\ {\isasymin}\ S{\isachardot}{\kern0pt}\ {\isasymexists}y\ {\isasymin}\ SymExt{\isacharparenleft}{\kern0pt}G{\isacharparenright}{\kern0pt}{\isachardot}{\kern0pt}\ y\ {\isasymin}\ x\ {\isasymand}\ z\ {\isasymin}\ y\ {\isacharbraceright}{\kern0pt}{\isachardoublequoteclose}\ \isanewline
\ \ \ \ \isacommand{unfolding}\isamarkupfalse%
\ T{\isacharunderscore}{\kern0pt}def\ \isanewline
\ \ \ \ \isacommand{apply}\isamarkupfalse%
{\isacharparenleft}{\kern0pt}rule\ iff{\isacharunderscore}{\kern0pt}eq{\isacharparenright}{\kern0pt}\isanewline
\ \ \ \ \isacommand{apply}\isamarkupfalse%
{\isacharparenleft}{\kern0pt}rename{\isacharunderscore}{\kern0pt}tac\ z{\isacharcomma}{\kern0pt}\ subgoal{\isacharunderscore}{\kern0pt}tac\ {\isachardoublequoteopen}z\ {\isasymin}\ SymExt{\isacharparenleft}{\kern0pt}G{\isacharparenright}{\kern0pt}{\isachardoublequoteclose}{\isacharparenright}{\kern0pt}\isanewline
\ \ \ \ \isacommand{unfolding}\isamarkupfalse%
\ BExists{\isacharunderscore}{\kern0pt}def\ BExists{\isacharprime}{\kern0pt}{\isacharunderscore}{\kern0pt}def\ \isanewline
\ \ \ \ \isacommand{using}\isamarkupfalse%
\ assms\ SymExt{\isacharunderscore}{\kern0pt}trans\ SH\ \isanewline
\ \ \ \ \isacommand{by}\isamarkupfalse%
\ auto\isanewline
\ \ \isacommand{also}\isamarkupfalse%
\ \isacommand{have}\isamarkupfalse%
\ {\isachardoublequoteopen}{\isachardot}{\kern0pt}{\isachardot}{\kern0pt}{\isachardot}{\kern0pt}\ {\isacharequal}{\kern0pt}\ {\isacharbraceleft}{\kern0pt}\ z\ {\isasymin}\ S{\isachardot}{\kern0pt}\ {\isasymexists}y\ {\isasymin}\ x{\isachardot}{\kern0pt}\ z\ {\isasymin}\ y\ {\isacharbraceright}{\kern0pt}{\isachardoublequoteclose}\ \isanewline
\ \ \ \ \isacommand{using}\isamarkupfalse%
\ SymExt{\isacharunderscore}{\kern0pt}trans\ assms\isanewline
\ \ \ \ \isacommand{by}\isamarkupfalse%
\ auto\isanewline
\ \ \isacommand{also}\isamarkupfalse%
\ \isacommand{have}\isamarkupfalse%
\ {\isachardoublequoteopen}{\isachardot}{\kern0pt}{\isachardot}{\kern0pt}{\isachardot}{\kern0pt}\ {\isacharequal}{\kern0pt}\ {\isacharbraceleft}{\kern0pt}\ z\ {\isasymin}\ S{\isachardot}{\kern0pt}\ z\ {\isasymin}\ {\isasymUnion}x\ {\isacharbraceright}{\kern0pt}{\isachardoublequoteclose}\ \isanewline
\ \ \ \ \isacommand{by}\isamarkupfalse%
\ auto\isanewline
\ \ \isacommand{also}\isamarkupfalse%
\ \isacommand{have}\isamarkupfalse%
\ {\isachardoublequoteopen}{\isachardot}{\kern0pt}{\isachardot}{\kern0pt}{\isachardot}{\kern0pt}\ {\isacharequal}{\kern0pt}\ {\isasymUnion}x{\isachardoublequoteclose}\ \isanewline
\ \ \ \ \isacommand{using}\isamarkupfalse%
\ subsetS\ \isanewline
\ \ \ \ \isacommand{by}\isamarkupfalse%
\ auto\isanewline
\ \ \isacommand{finally}\isamarkupfalse%
\ \isacommand{show}\isamarkupfalse%
\ {\isacharquery}{\kern0pt}thesis\ \isanewline
\ \ \ \ \isacommand{using}\isamarkupfalse%
\ Tin\ \isacommand{by}\isamarkupfalse%
\ auto\isanewline
\isacommand{qed}\isamarkupfalse%
%
\endisatagproof
{\isafoldproof}%
%
\isadelimproof
\isanewline
%
\endisadelimproof
\isanewline
\isacommand{lemma}\isamarkupfalse%
\ Un{\isacharunderscore}{\kern0pt}in{\isacharunderscore}{\kern0pt}SymExt\ {\isacharcolon}{\kern0pt}\ \isanewline
\ \ \isakeyword{fixes}\ a\ b\ \isanewline
\ \ \isakeyword{assumes}\ {\isachardoublequoteopen}a\ {\isasymin}\ SymExt{\isacharparenleft}{\kern0pt}G{\isacharparenright}{\kern0pt}{\isachardoublequoteclose}\ {\isachardoublequoteopen}b\ {\isasymin}\ SymExt{\isacharparenleft}{\kern0pt}G{\isacharparenright}{\kern0pt}{\isachardoublequoteclose}\ \isanewline
\ \ \isakeyword{shows}\ {\isachardoublequoteopen}a\ {\isasymunion}\ b\ {\isasymin}\ SymExt{\isacharparenleft}{\kern0pt}G{\isacharparenright}{\kern0pt}{\isachardoublequoteclose}\ \isanewline
%
\isadelimproof
\isanewline
\ \ %
\endisadelimproof
%
\isatagproof
\isacommand{apply}\isamarkupfalse%
{\isacharparenleft}{\kern0pt}rule{\isacharunderscore}{\kern0pt}tac\ b{\isacharequal}{\kern0pt}{\isachardoublequoteopen}a\ {\isasymunion}\ b{\isachardoublequoteclose}\ \isakeyword{and}\ a{\isacharequal}{\kern0pt}{\isachardoublequoteopen}{\isasymUnion}{\isacharbraceleft}{\kern0pt}a{\isacharcomma}{\kern0pt}\ b{\isacharbraceright}{\kern0pt}{\isachardoublequoteclose}\ \isakeyword{in}\ ssubst{\isacharcomma}{\kern0pt}\ simp{\isacharparenright}{\kern0pt}\isanewline
\ \ \isacommand{apply}\isamarkupfalse%
{\isacharparenleft}{\kern0pt}rule\ Union{\isacharunderscore}{\kern0pt}in{\isacharunderscore}{\kern0pt}SymExt{\isacharparenright}{\kern0pt}\isanewline
\ \ \isacommand{apply}\isamarkupfalse%
{\isacharparenleft}{\kern0pt}rule\ upair{\isacharunderscore}{\kern0pt}in{\isacharunderscore}{\kern0pt}SymExt{\isacharparenright}{\kern0pt}\isanewline
\ \ \isacommand{using}\isamarkupfalse%
\ assms\isanewline
\ \ \isacommand{by}\isamarkupfalse%
\ auto%
\endisatagproof
{\isafoldproof}%
%
\isadelimproof
\isanewline
%
\endisadelimproof
\isanewline
\isacommand{lemma}\isamarkupfalse%
\ powerset{\isacharunderscore}{\kern0pt}in{\isacharunderscore}{\kern0pt}SymExt\ {\isacharcolon}{\kern0pt}\ \isanewline
\ \ \isakeyword{fixes}\ x\ \isanewline
\ \ \isakeyword{assumes}\ {\isachardoublequoteopen}x\ {\isasymin}\ SymExt{\isacharparenleft}{\kern0pt}G{\isacharparenright}{\kern0pt}{\isachardoublequoteclose}\ \isanewline
\ \ \isakeyword{shows}\ {\isachardoublequoteopen}Pow{\isacharparenleft}{\kern0pt}x{\isacharparenright}{\kern0pt}\ {\isasyminter}\ SymExt{\isacharparenleft}{\kern0pt}G{\isacharparenright}{\kern0pt}\ {\isasymin}\ SymExt{\isacharparenleft}{\kern0pt}G{\isacharparenright}{\kern0pt}{\isachardoublequoteclose}\ \isanewline
%
\isadelimproof
%
\endisadelimproof
%
\isatagproof
\isacommand{proof}\isamarkupfalse%
\ {\isacharminus}{\kern0pt}\ \isanewline
\ \ \isacommand{have}\isamarkupfalse%
\ {\isachardoublequoteopen}x\ {\isasymin}\ M{\isacharbrackleft}{\kern0pt}G{\isacharbrackright}{\kern0pt}{\isachardoublequoteclose}\ \isacommand{using}\isamarkupfalse%
\ assms\ SymExt{\isacharunderscore}{\kern0pt}subset{\isacharunderscore}{\kern0pt}GenExt\ \isacommand{by}\isamarkupfalse%
\ auto\isanewline
\ \ \isacommand{then}\isamarkupfalse%
\ \isacommand{have}\isamarkupfalse%
\ {\isachardoublequoteopen}{\isasymexists}powx\ {\isasymin}\ M{\isacharbrackleft}{\kern0pt}G{\isacharbrackright}{\kern0pt}\ {\isachardot}{\kern0pt}\ powerset{\isacharparenleft}{\kern0pt}{\isacharhash}{\kern0pt}{\isacharhash}{\kern0pt}M{\isacharbrackleft}{\kern0pt}G{\isacharbrackright}{\kern0pt}{\isacharcomma}{\kern0pt}\ x{\isacharcomma}{\kern0pt}\ powx{\isacharparenright}{\kern0pt}{\isachardoublequoteclose}\ \isacommand{using}\isamarkupfalse%
\ power{\isacharunderscore}{\kern0pt}in{\isacharunderscore}{\kern0pt}MG\ power{\isacharunderscore}{\kern0pt}ax{\isacharunderscore}{\kern0pt}def\ \isacommand{by}\isamarkupfalse%
\ auto\isanewline
\ \ \isacommand{then}\isamarkupfalse%
\ \isacommand{obtain}\isamarkupfalse%
\ powx\ \isakeyword{where}\ powxH\ {\isacharcolon}{\kern0pt}\ {\isachardoublequoteopen}powx\ {\isasymin}\ M{\isacharbrackleft}{\kern0pt}G{\isacharbrackright}{\kern0pt}{\isachardoublequoteclose}\ {\isachardoublequoteopen}powerset{\isacharparenleft}{\kern0pt}{\isacharhash}{\kern0pt}{\isacharhash}{\kern0pt}M{\isacharbrackleft}{\kern0pt}G{\isacharbrackright}{\kern0pt}{\isacharcomma}{\kern0pt}\ x{\isacharcomma}{\kern0pt}\ powx{\isacharparenright}{\kern0pt}{\isachardoublequoteclose}\ \isacommand{by}\isamarkupfalse%
\ auto\isanewline
\ \ \isacommand{then}\isamarkupfalse%
\ \isacommand{have}\isamarkupfalse%
\ {\isachardoublequoteopen}{\isasymAnd}y{\isachardot}{\kern0pt}\ y\ {\isasymin}\ M{\isacharbrackleft}{\kern0pt}G{\isacharbrackright}{\kern0pt}\ {\isasymLongrightarrow}\ y\ {\isasymsubseteq}\ x\ {\isasymLongrightarrow}\ y\ {\isasymin}\ powx{\isachardoublequoteclose}\ \isanewline
\ \ \ \ \isacommand{apply}\isamarkupfalse%
{\isacharparenleft}{\kern0pt}rename{\isacharunderscore}{\kern0pt}tac\ y{\isacharcomma}{\kern0pt}\ subgoal{\isacharunderscore}{\kern0pt}tac\ {\isachardoublequoteopen}{\isacharparenleft}{\kern0pt}{\isasymforall}z{\isacharbrackleft}{\kern0pt}{\isacharhash}{\kern0pt}{\isacharhash}{\kern0pt}M{\isacharbrackleft}{\kern0pt}G{\isacharbrackright}{\kern0pt}{\isacharbrackright}{\kern0pt}{\isachardot}{\kern0pt}\ z\ {\isasymin}\ y\ {\isasymlongrightarrow}\ z\ {\isasymin}\ x{\isacharparenright}{\kern0pt}{\isachardoublequoteclose}{\isacharparenright}{\kern0pt}\isanewline
\ \ \ \ \isacommand{unfolding}\isamarkupfalse%
\ powerset{\isacharunderscore}{\kern0pt}def\ subset{\isacharunderscore}{\kern0pt}def\ \isanewline
\ \ \ \ \isacommand{by}\isamarkupfalse%
\ auto\isanewline
\ \ \isacommand{then}\isamarkupfalse%
\ \isacommand{have}\isamarkupfalse%
\ subsetpowx\ {\isacharcolon}{\kern0pt}\ {\isachardoublequoteopen}Pow{\isacharparenleft}{\kern0pt}x{\isacharparenright}{\kern0pt}\ {\isasyminter}\ M{\isacharbrackleft}{\kern0pt}G{\isacharbrackright}{\kern0pt}\ {\isasymsubseteq}\ powx{\isachardoublequoteclose}\ \isacommand{by}\isamarkupfalse%
\ auto\isanewline
\isanewline
\ \ \isacommand{obtain}\isamarkupfalse%
\ powx{\isacharprime}{\kern0pt}\ \isakeyword{where}\ powx{\isacharprime}{\kern0pt}H\ {\isacharcolon}{\kern0pt}\ {\isachardoublequoteopen}powx{\isacharprime}{\kern0pt}\ {\isasymin}\ M{\isachardoublequoteclose}\ {\isachardoublequoteopen}val{\isacharparenleft}{\kern0pt}G{\isacharcomma}{\kern0pt}\ powx{\isacharprime}{\kern0pt}{\isacharparenright}{\kern0pt}\ {\isacharequal}{\kern0pt}\ powx{\isachardoublequoteclose}\ \isanewline
\ \ \ \ \isacommand{using}\isamarkupfalse%
\ powxH\ GenExt{\isacharunderscore}{\kern0pt}def\ \isanewline
\ \ \ \ \isacommand{by}\isamarkupfalse%
\ auto\isanewline
\isanewline
\ \ \isacommand{have}\isamarkupfalse%
\ {\isachardoublequoteopen}{\isasymexists}S{\isachardot}{\kern0pt}\ S\ {\isasymin}\ M\ {\isasymand}\ S\ {\isasymsubseteq}\ HS\ {\isasymand}\ {\isacharparenleft}{\kern0pt}{\isasymforall}p{\isasymin}G{\isachardot}{\kern0pt}\ {\isasymforall}y{\isasymin}domain{\isacharparenleft}{\kern0pt}powx{\isacharprime}{\kern0pt}{\isacharparenright}{\kern0pt}{\isachardot}{\kern0pt}\ {\isacharparenleft}{\kern0pt}{\isasymexists}z{\isasymin}HS{\isachardot}{\kern0pt}\ p\ {\isasymtturnstile}HS\ Equal{\isacharparenleft}{\kern0pt}{\isadigit{0}}{\isacharcomma}{\kern0pt}\ {\isadigit{1}}{\isacharparenright}{\kern0pt}\ {\isacharbrackleft}{\kern0pt}y{\isacharcomma}{\kern0pt}\ z{\isacharbrackright}{\kern0pt}\ {\isacharat}{\kern0pt}\ {\isacharbrackleft}{\kern0pt}{\isacharbrackright}{\kern0pt}{\isacharparenright}{\kern0pt}\ {\isasymlongleftrightarrow}\ {\isacharparenleft}{\kern0pt}{\isasymexists}z{\isasymin}S{\isachardot}{\kern0pt}\ p\ {\isasymtturnstile}HS\ Equal{\isacharparenleft}{\kern0pt}{\isadigit{0}}{\isacharcomma}{\kern0pt}\ {\isadigit{1}}{\isacharparenright}{\kern0pt}\ {\isacharbrackleft}{\kern0pt}y{\isacharcomma}{\kern0pt}\ z{\isacharbrackright}{\kern0pt}\ {\isacharat}{\kern0pt}\ {\isacharbrackleft}{\kern0pt}{\isacharbrackright}{\kern0pt}{\isacharparenright}{\kern0pt}{\isacharparenright}{\kern0pt}{\isachardoublequoteclose}\ \isanewline
\ \ \ \ \isacommand{apply}\isamarkupfalse%
{\isacharparenleft}{\kern0pt}rule\ ex{\isacharunderscore}{\kern0pt}hs{\isacharunderscore}{\kern0pt}subset{\isacharunderscore}{\kern0pt}contains{\isacharunderscore}{\kern0pt}witnesses{\isacharparenright}{\kern0pt}\isanewline
\ \ \ \ \isacommand{using}\isamarkupfalse%
\ Un{\isacharunderscore}{\kern0pt}least{\isacharunderscore}{\kern0pt}lt\ powx{\isacharprime}{\kern0pt}H\isanewline
\ \ \ \ \isacommand{by}\isamarkupfalse%
\ auto\isanewline
\ \ \isacommand{then}\isamarkupfalse%
\ \isacommand{obtain}\isamarkupfalse%
\ S\ \isakeyword{where}\ SH\ {\isacharcolon}{\kern0pt}\ {\isachardoublequoteopen}S\ {\isasymin}\ M{\isachardoublequoteclose}\ {\isachardoublequoteopen}S\ {\isasymsubseteq}\ HS{\isachardoublequoteclose}\ {\isachardoublequoteopen}{\isasymforall}p{\isasymin}G{\isachardot}{\kern0pt}\ {\isasymforall}y{\isasymin}domain{\isacharparenleft}{\kern0pt}powx{\isacharprime}{\kern0pt}{\isacharparenright}{\kern0pt}{\isachardot}{\kern0pt}\ {\isacharparenleft}{\kern0pt}{\isasymexists}z{\isasymin}HS{\isachardot}{\kern0pt}\ p\ {\isasymtturnstile}HS\ Equal{\isacharparenleft}{\kern0pt}{\isadigit{0}}{\isacharcomma}{\kern0pt}\ {\isadigit{1}}{\isacharparenright}{\kern0pt}\ {\isacharbrackleft}{\kern0pt}y{\isacharcomma}{\kern0pt}\ z{\isacharbrackright}{\kern0pt}{\isacharparenright}{\kern0pt}\ {\isasymlongleftrightarrow}\ {\isacharparenleft}{\kern0pt}{\isasymexists}z{\isasymin}S{\isachardot}{\kern0pt}\ p\ {\isasymtturnstile}HS\ Equal{\isacharparenleft}{\kern0pt}{\isadigit{0}}{\isacharcomma}{\kern0pt}\ {\isadigit{1}}{\isacharparenright}{\kern0pt}\ {\isacharbrackleft}{\kern0pt}y{\isacharcomma}{\kern0pt}\ z{\isacharbrackright}{\kern0pt}{\isacharparenright}{\kern0pt}{\isachardoublequoteclose}\ \isanewline
\ \ \ \ \isacommand{by}\isamarkupfalse%
\ auto\isanewline
\ \ \isacommand{then}\isamarkupfalse%
\ \isacommand{have}\isamarkupfalse%
\ SE{\isacharcolon}{\kern0pt}\ {\isachardoublequoteopen}{\isasymAnd}p\ y{\isachardot}{\kern0pt}\ p\ {\isasymin}\ G\ {\isasymLongrightarrow}\ y\ {\isasymin}\ domain{\isacharparenleft}{\kern0pt}powx{\isacharprime}{\kern0pt}{\isacharparenright}{\kern0pt}\ {\isasymLongrightarrow}\ {\isacharparenleft}{\kern0pt}{\isasymexists}z{\isasymin}HS{\isachardot}{\kern0pt}\ p\ {\isasymtturnstile}HS\ Equal{\isacharparenleft}{\kern0pt}{\isadigit{0}}{\isacharcomma}{\kern0pt}\ {\isadigit{1}}{\isacharparenright}{\kern0pt}\ {\isacharbrackleft}{\kern0pt}y{\isacharcomma}{\kern0pt}\ z{\isacharbrackright}{\kern0pt}{\isacharparenright}{\kern0pt}\ {\isasymlongleftrightarrow}\ {\isacharparenleft}{\kern0pt}{\isasymexists}z{\isasymin}S{\isachardot}{\kern0pt}\ p\ {\isasymtturnstile}HS\ Equal{\isacharparenleft}{\kern0pt}{\isadigit{0}}{\isacharcomma}{\kern0pt}\ {\isadigit{1}}{\isacharparenright}{\kern0pt}\ {\isacharbrackleft}{\kern0pt}y{\isacharcomma}{\kern0pt}\ z{\isacharbrackright}{\kern0pt}{\isacharparenright}{\kern0pt}{\isachardoublequoteclose}\ \isacommand{by}\isamarkupfalse%
\ auto\isanewline
\isanewline
\ \ \isacommand{have}\isamarkupfalse%
\ {\isachardoublequoteopen}{\isasymexists}T\ {\isasymin}\ SymExt{\isacharparenleft}{\kern0pt}G{\isacharparenright}{\kern0pt}{\isachardot}{\kern0pt}\ {\isacharbraceleft}{\kern0pt}\ val{\isacharparenleft}{\kern0pt}G{\isacharcomma}{\kern0pt}\ x{\isacharparenright}{\kern0pt}{\isachardot}{\kern0pt}\ x\ {\isasymin}\ S\ {\isacharbraceright}{\kern0pt}\ {\isasymsubseteq}\ T{\isachardoublequoteclose}\isanewline
\ \ \ \ \isacommand{apply}\isamarkupfalse%
{\isacharparenleft}{\kern0pt}rule\ ex{\isacharunderscore}{\kern0pt}separation{\isacharunderscore}{\kern0pt}base{\isacharparenright}{\kern0pt}\isanewline
\ \ \ \ \isacommand{using}\isamarkupfalse%
\ SH\ \isanewline
\ \ \ \ \isacommand{by}\isamarkupfalse%
\ auto\isanewline
\ \ \isacommand{then}\isamarkupfalse%
\ \isacommand{obtain}\isamarkupfalse%
\ T\ \isakeyword{where}\ TH\ {\isacharcolon}{\kern0pt}\ {\isachardoublequoteopen}T\ {\isasymin}\ SymExt{\isacharparenleft}{\kern0pt}G{\isacharparenright}{\kern0pt}{\isachardoublequoteclose}\ {\isachardoublequoteopen}{\isacharbraceleft}{\kern0pt}\ val{\isacharparenleft}{\kern0pt}G{\isacharcomma}{\kern0pt}\ x{\isacharparenright}{\kern0pt}{\isachardot}{\kern0pt}\ x\ {\isasymin}\ S\ {\isacharbraceright}{\kern0pt}\ {\isasymsubseteq}\ T{\isachardoublequoteclose}\ \isacommand{by}\isamarkupfalse%
\ auto\isanewline
\isanewline
\ \ \isacommand{have}\isamarkupfalse%
\ subsetT{\isacharcolon}{\kern0pt}\ {\isachardoublequoteopen}Pow{\isacharparenleft}{\kern0pt}x{\isacharparenright}{\kern0pt}\ {\isasyminter}\ SymExt{\isacharparenleft}{\kern0pt}G{\isacharparenright}{\kern0pt}\ {\isasymsubseteq}\ T{\isachardoublequoteclose}\ \isanewline
\ \ \isacommand{proof}\isamarkupfalse%
\ {\isacharparenleft}{\kern0pt}rule\ subsetI{\isacharparenright}{\kern0pt}\isanewline
\ \ \ \ \isacommand{fix}\isamarkupfalse%
\ y\ \isacommand{assume}\isamarkupfalse%
\ yin{\isacharcolon}{\kern0pt}\ {\isachardoublequoteopen}y\ {\isasymin}\ Pow{\isacharparenleft}{\kern0pt}x{\isacharparenright}{\kern0pt}\ {\isasyminter}\ SymExt{\isacharparenleft}{\kern0pt}G{\isacharparenright}{\kern0pt}{\isachardoublequoteclose}\ \isanewline
\isanewline
\ \ \ \ \isacommand{then}\isamarkupfalse%
\ \isacommand{have}\isamarkupfalse%
\ {\isachardoublequoteopen}y\ {\isasymin}\ powx{\isachardoublequoteclose}\ \isacommand{using}\isamarkupfalse%
\ SymExt{\isacharunderscore}{\kern0pt}subset{\isacharunderscore}{\kern0pt}GenExt\ subsetpowx\ \isacommand{by}\isamarkupfalse%
\ auto\isanewline
\ \ \ \ \isacommand{then}\isamarkupfalse%
\ \isacommand{have}\isamarkupfalse%
\ {\isachardoublequoteopen}{\isasymexists}y{\isacharprime}{\kern0pt}{\isasymin}domain{\isacharparenleft}{\kern0pt}powx{\isacharprime}{\kern0pt}{\isacharparenright}{\kern0pt}{\isachardot}{\kern0pt}\ val{\isacharparenleft}{\kern0pt}G{\isacharcomma}{\kern0pt}\ y{\isacharprime}{\kern0pt}{\isacharparenright}{\kern0pt}\ {\isacharequal}{\kern0pt}\ y{\isachardoublequoteclose}\isanewline
\ \ \ \ \ \ \isacommand{apply}\isamarkupfalse%
{\isacharparenleft}{\kern0pt}rule{\isacharunderscore}{\kern0pt}tac\ P{\isacharequal}{\kern0pt}{\isachardoublequoteopen}powx\ {\isacharequal}{\kern0pt}\ val{\isacharparenleft}{\kern0pt}G{\isacharcomma}{\kern0pt}\ powx{\isacharprime}{\kern0pt}{\isacharparenright}{\kern0pt}{\isachardoublequoteclose}\ \isakeyword{in}\ mp{\isacharparenright}{\kern0pt}\isanewline
\ \ \ \ \ \ \isacommand{apply}\isamarkupfalse%
{\isacharparenleft}{\kern0pt}subst\ def{\isacharunderscore}{\kern0pt}val{\isacharcomma}{\kern0pt}\ force{\isacharparenright}{\kern0pt}\isanewline
\ \ \ \ \ \ \isacommand{using}\isamarkupfalse%
\ powx{\isacharprime}{\kern0pt}H\ \isanewline
\ \ \ \ \ \ \isacommand{by}\isamarkupfalse%
\ auto\isanewline
\ \ \ \ \isacommand{then}\isamarkupfalse%
\ \isacommand{obtain}\isamarkupfalse%
\ y{\isacharprime}{\kern0pt}\ \isakeyword{where}\ y{\isacharprime}{\kern0pt}H\ {\isacharcolon}{\kern0pt}\ {\isachardoublequoteopen}y{\isacharprime}{\kern0pt}\ {\isasymin}\ domain{\isacharparenleft}{\kern0pt}powx{\isacharprime}{\kern0pt}{\isacharparenright}{\kern0pt}{\isachardoublequoteclose}\ {\isachardoublequoteopen}val{\isacharparenleft}{\kern0pt}G{\isacharcomma}{\kern0pt}\ y{\isacharprime}{\kern0pt}{\isacharparenright}{\kern0pt}\ {\isacharequal}{\kern0pt}\ y{\isachardoublequoteclose}\ \isacommand{by}\isamarkupfalse%
\ auto\isanewline
\ \ \ \ \isacommand{then}\isamarkupfalse%
\ \isacommand{have}\isamarkupfalse%
\ y{\isacharprime}{\kern0pt}inM\ {\isacharcolon}{\kern0pt}\ {\isachardoublequoteopen}y{\isacharprime}{\kern0pt}\ {\isasymin}\ M{\isachardoublequoteclose}\ \isanewline
\ \ \ \ \ \ \isacommand{using}\isamarkupfalse%
\ powx{\isacharprime}{\kern0pt}H\ domain{\isacharunderscore}{\kern0pt}elem{\isacharunderscore}{\kern0pt}in{\isacharunderscore}{\kern0pt}M\ \isanewline
\ \ \ \ \ \ \isacommand{by}\isamarkupfalse%
\ auto\isanewline
\isanewline
\ \ \ \ \isacommand{obtain}\isamarkupfalse%
\ y{\isacharprime}{\kern0pt}{\isacharprime}{\kern0pt}\ \isakeyword{where}\ y{\isacharprime}{\kern0pt}{\isacharprime}{\kern0pt}H{\isacharcolon}{\kern0pt}\ {\isachardoublequoteopen}y{\isacharprime}{\kern0pt}{\isacharprime}{\kern0pt}\ {\isasymin}\ HS{\isachardoublequoteclose}\ {\isachardoublequoteopen}val{\isacharparenleft}{\kern0pt}G{\isacharcomma}{\kern0pt}\ y{\isacharprime}{\kern0pt}{\isacharprime}{\kern0pt}{\isacharparenright}{\kern0pt}\ {\isacharequal}{\kern0pt}\ y{\isachardoublequoteclose}\ \isacommand{using}\isamarkupfalse%
\ yin\ SymExt{\isacharunderscore}{\kern0pt}def\ \isacommand{by}\isamarkupfalse%
\ auto\isanewline
\isanewline
\ \ \ \ \isacommand{have}\isamarkupfalse%
\ {\isachardoublequoteopen}sats{\isacharparenleft}{\kern0pt}M{\isacharbrackleft}{\kern0pt}G{\isacharbrackright}{\kern0pt}{\isacharcomma}{\kern0pt}\ Equal{\isacharparenleft}{\kern0pt}{\isadigit{0}}{\isacharcomma}{\kern0pt}\ {\isadigit{1}}{\isacharparenright}{\kern0pt}{\isacharcomma}{\kern0pt}\ map{\isacharparenleft}{\kern0pt}val{\isacharparenleft}{\kern0pt}G{\isacharparenright}{\kern0pt}{\isacharcomma}{\kern0pt}\ {\isacharbrackleft}{\kern0pt}y{\isacharprime}{\kern0pt}{\isacharcomma}{\kern0pt}\ y{\isacharprime}{\kern0pt}{\isacharprime}{\kern0pt}{\isacharbrackright}{\kern0pt}{\isacharparenright}{\kern0pt}{\isacharparenright}{\kern0pt}{\isachardoublequoteclose}\ \isacommand{using}\isamarkupfalse%
\ y{\isacharprime}{\kern0pt}H\ y{\isacharprime}{\kern0pt}{\isacharprime}{\kern0pt}H\ yin\ SymExt{\isacharunderscore}{\kern0pt}subset{\isacharunderscore}{\kern0pt}GenExt\ \isacommand{by}\isamarkupfalse%
\ auto\isanewline
\ \ \ \ \isacommand{then}\isamarkupfalse%
\ \isacommand{have}\isamarkupfalse%
\ {\isachardoublequoteopen}{\isasymexists}p\ {\isasymin}\ G{\isachardot}{\kern0pt}\ p\ {\isasymtturnstile}\ Equal{\isacharparenleft}{\kern0pt}{\isadigit{0}}{\isacharcomma}{\kern0pt}\ {\isadigit{1}}{\isacharparenright}{\kern0pt}\ {\isacharbrackleft}{\kern0pt}y{\isacharprime}{\kern0pt}{\isacharcomma}{\kern0pt}\ y{\isacharprime}{\kern0pt}{\isacharprime}{\kern0pt}{\isacharbrackright}{\kern0pt}{\isachardoublequoteclose}\ \isanewline
\ \ \ \ \ \ \isacommand{apply}\isamarkupfalse%
{\isacharparenleft}{\kern0pt}rule{\isacharunderscore}{\kern0pt}tac\ iffD{\isadigit{2}}{\isacharparenright}{\kern0pt}\isanewline
\ \ \ \ \ \ \ \isacommand{apply}\isamarkupfalse%
{\isacharparenleft}{\kern0pt}rule\ truth{\isacharunderscore}{\kern0pt}lemma{\isacharparenright}{\kern0pt}\isanewline
\ \ \ \ \ \ \isacommand{using}\isamarkupfalse%
\ generic\ y{\isacharprime}{\kern0pt}inM\ y{\isacharprime}{\kern0pt}{\isacharprime}{\kern0pt}H\ HS{\isacharunderscore}{\kern0pt}iff\ P{\isacharunderscore}{\kern0pt}name{\isacharunderscore}{\kern0pt}in{\isacharunderscore}{\kern0pt}M\ Un{\isacharunderscore}{\kern0pt}least{\isacharunderscore}{\kern0pt}lt\isanewline
\ \ \ \ \ \ \isacommand{by}\isamarkupfalse%
\ auto\isanewline
\ \ \ \ \isacommand{then}\isamarkupfalse%
\ \isacommand{have}\isamarkupfalse%
\ {\isachardoublequoteopen}{\isasymexists}p\ {\isasymin}\ G{\isachardot}{\kern0pt}\ p\ {\isasymtturnstile}HS\ Equal{\isacharparenleft}{\kern0pt}{\isadigit{0}}{\isacharcomma}{\kern0pt}\ {\isadigit{1}}{\isacharparenright}{\kern0pt}\ {\isacharbrackleft}{\kern0pt}y{\isacharprime}{\kern0pt}{\isacharcomma}{\kern0pt}\ y{\isacharprime}{\kern0pt}{\isacharprime}{\kern0pt}{\isacharbrackright}{\kern0pt}{\isachardoublequoteclose}\ \isanewline
\ \ \ \ \ \ \isacommand{apply}\isamarkupfalse%
{\isacharparenleft}{\kern0pt}rule{\isacharunderscore}{\kern0pt}tac\ iffD{\isadigit{1}}{\isacharparenright}{\kern0pt}\isanewline
\ \ \ \ \ \ \ \isacommand{apply}\isamarkupfalse%
{\isacharparenleft}{\kern0pt}rule\ bex{\isacharunderscore}{\kern0pt}iff{\isacharcomma}{\kern0pt}\ rule\ ForcesHS{\isacharunderscore}{\kern0pt}Equal{\isacharparenright}{\kern0pt}\isanewline
\ \ \ \ \ \ \isacommand{using}\isamarkupfalse%
\ generic\ y{\isacharprime}{\kern0pt}inM\ y{\isacharprime}{\kern0pt}{\isacharprime}{\kern0pt}H\ HS{\isacharunderscore}{\kern0pt}iff\ P{\isacharunderscore}{\kern0pt}name{\isacharunderscore}{\kern0pt}in{\isacharunderscore}{\kern0pt}M\ Un{\isacharunderscore}{\kern0pt}least{\isacharunderscore}{\kern0pt}lt\ generic\ M{\isacharunderscore}{\kern0pt}genericD\ P{\isacharunderscore}{\kern0pt}in{\isacharunderscore}{\kern0pt}M\ transM\ \isanewline
\ \ \ \ \ \ \isacommand{by}\isamarkupfalse%
\ auto\isanewline
\ \ \ \ \isacommand{then}\isamarkupfalse%
\ \isacommand{have}\isamarkupfalse%
\ {\isachardoublequoteopen}{\isasymexists}p\ {\isasymin}\ G{\isachardot}{\kern0pt}\ {\isasymexists}y{\isacharprime}{\kern0pt}{\isacharprime}{\kern0pt}\ {\isasymin}\ HS{\isachardot}{\kern0pt}\ p\ {\isasymtturnstile}HS\ Equal{\isacharparenleft}{\kern0pt}{\isadigit{0}}{\isacharcomma}{\kern0pt}\ {\isadigit{1}}{\isacharparenright}{\kern0pt}\ {\isacharbrackleft}{\kern0pt}y{\isacharprime}{\kern0pt}{\isacharcomma}{\kern0pt}\ y{\isacharprime}{\kern0pt}{\isacharprime}{\kern0pt}{\isacharbrackright}{\kern0pt}{\isachardoublequoteclose}\ {\isacharparenleft}{\kern0pt}\isakeyword{is}\ {\isacharquery}{\kern0pt}A{\isacharparenright}{\kern0pt}\ \isacommand{using}\isamarkupfalse%
\ y{\isacharprime}{\kern0pt}{\isacharprime}{\kern0pt}H\ \isacommand{by}\isamarkupfalse%
\ auto\isanewline
\ \ \ \ \isacommand{then}\isamarkupfalse%
\ \isacommand{have}\isamarkupfalse%
\ {\isachardoublequoteopen}{\isasymexists}p\ {\isasymin}\ G{\isachardot}{\kern0pt}\ {\isasymexists}y{\isacharprime}{\kern0pt}{\isacharprime}{\kern0pt}\ {\isasymin}\ S{\isachardot}{\kern0pt}\ p\ {\isasymtturnstile}HS\ Equal{\isacharparenleft}{\kern0pt}{\isadigit{0}}{\isacharcomma}{\kern0pt}\ {\isadigit{1}}{\isacharparenright}{\kern0pt}\ {\isacharbrackleft}{\kern0pt}y{\isacharprime}{\kern0pt}{\isacharcomma}{\kern0pt}\ y{\isacharprime}{\kern0pt}{\isacharprime}{\kern0pt}{\isacharbrackright}{\kern0pt}{\isachardoublequoteclose}\isanewline
\ \ \ \ \ \ \isacommand{apply}\isamarkupfalse%
{\isacharparenleft}{\kern0pt}rule{\isacharunderscore}{\kern0pt}tac\ P{\isacharequal}{\kern0pt}{\isachardoublequoteopen}{\isacharquery}{\kern0pt}A{\isachardoublequoteclose}\ \isakeyword{in}\ iffD{\isadigit{1}}{\isacharparenright}{\kern0pt}\isanewline
\ \ \ \ \ \ \ \isacommand{apply}\isamarkupfalse%
{\isacharparenleft}{\kern0pt}rule\ bex{\isacharunderscore}{\kern0pt}iff{\isacharcomma}{\kern0pt}\ rule\ SE{\isacharparenright}{\kern0pt}\isanewline
\ \ \ \ \ \ \isacommand{using}\isamarkupfalse%
\ y{\isacharprime}{\kern0pt}H\ \isanewline
\ \ \ \ \ \ \isacommand{by}\isamarkupfalse%
\ auto\isanewline
\ \ \ \ \isacommand{then}\isamarkupfalse%
\ \isacommand{have}\isamarkupfalse%
\ {\isachardoublequoteopen}{\isasymexists}y{\isacharprime}{\kern0pt}{\isacharprime}{\kern0pt}{\isacharprime}{\kern0pt}\ {\isasymin}\ S{\isachardot}{\kern0pt}\ {\isasymexists}p\ {\isasymin}\ G{\isachardot}{\kern0pt}\ p\ {\isasymtturnstile}HS\ Equal{\isacharparenleft}{\kern0pt}{\isadigit{0}}{\isacharcomma}{\kern0pt}\ {\isadigit{1}}{\isacharparenright}{\kern0pt}\ {\isacharbrackleft}{\kern0pt}y{\isacharprime}{\kern0pt}{\isacharcomma}{\kern0pt}\ y{\isacharprime}{\kern0pt}{\isacharprime}{\kern0pt}{\isacharprime}{\kern0pt}{\isacharbrackright}{\kern0pt}{\isachardoublequoteclose}\ {\isacharparenleft}{\kern0pt}\isakeyword{is}\ {\isacharquery}{\kern0pt}B{\isacharparenright}{\kern0pt}\ \isacommand{by}\isamarkupfalse%
\ auto\isanewline
\ \ \ \ \isacommand{then}\isamarkupfalse%
\ \isacommand{have}\isamarkupfalse%
\ {\isachardoublequoteopen}{\isasymexists}y{\isacharprime}{\kern0pt}{\isacharprime}{\kern0pt}{\isacharprime}{\kern0pt}\ {\isasymin}\ S{\isachardot}{\kern0pt}\ {\isasymexists}p\ {\isasymin}\ G{\isachardot}{\kern0pt}\ p\ {\isasymtturnstile}\ Equal{\isacharparenleft}{\kern0pt}{\isadigit{0}}{\isacharcomma}{\kern0pt}\ {\isadigit{1}}{\isacharparenright}{\kern0pt}\ {\isacharbrackleft}{\kern0pt}y{\isacharprime}{\kern0pt}{\isacharcomma}{\kern0pt}\ y{\isacharprime}{\kern0pt}{\isacharprime}{\kern0pt}{\isacharprime}{\kern0pt}{\isacharbrackright}{\kern0pt}{\isachardoublequoteclose}\ {\isacharparenleft}{\kern0pt}\isakeyword{is}\ {\isacharquery}{\kern0pt}C{\isacharparenright}{\kern0pt}\isanewline
\ \ \ \ \ \ \isacommand{apply}\isamarkupfalse%
{\isacharparenleft}{\kern0pt}rule{\isacharunderscore}{\kern0pt}tac\ Q{\isacharequal}{\kern0pt}{\isachardoublequoteopen}{\isacharquery}{\kern0pt}B{\isachardoublequoteclose}\ \isakeyword{in}\ iffD{\isadigit{2}}{\isacharparenright}{\kern0pt}\isanewline
\ \ \ \ \ \ \ \isacommand{apply}\isamarkupfalse%
{\isacharparenleft}{\kern0pt}rule\ bex{\isacharunderscore}{\kern0pt}iff{\isacharparenright}{\kern0pt}{\isacharplus}{\kern0pt}\isanewline
\ \ \ \ \ \ \ \isacommand{apply}\isamarkupfalse%
{\isacharparenleft}{\kern0pt}rule\ ForcesHS{\isacharunderscore}{\kern0pt}Equal{\isacharparenright}{\kern0pt}\isanewline
\ \ \ \ \ \ \isacommand{using}\isamarkupfalse%
\ y{\isacharprime}{\kern0pt}H\ SH\ transM\ y{\isacharprime}{\kern0pt}inM\ generic\ M{\isacharunderscore}{\kern0pt}genericD\ P{\isacharunderscore}{\kern0pt}in{\isacharunderscore}{\kern0pt}M\isanewline
\ \ \ \ \ \ \isacommand{by}\isamarkupfalse%
\ auto\isanewline
\ \ \ \ \isacommand{then}\isamarkupfalse%
\ \isacommand{have}\isamarkupfalse%
\ {\isachardoublequoteopen}{\isasymexists}y{\isacharprime}{\kern0pt}{\isacharprime}{\kern0pt}{\isacharprime}{\kern0pt}\ {\isasymin}\ S{\isachardot}{\kern0pt}\ sats{\isacharparenleft}{\kern0pt}M{\isacharbrackleft}{\kern0pt}G{\isacharbrackright}{\kern0pt}{\isacharcomma}{\kern0pt}\ Equal{\isacharparenleft}{\kern0pt}{\isadigit{0}}{\isacharcomma}{\kern0pt}\ {\isadigit{1}}{\isacharparenright}{\kern0pt}{\isacharcomma}{\kern0pt}\ map{\isacharparenleft}{\kern0pt}val{\isacharparenleft}{\kern0pt}G{\isacharparenright}{\kern0pt}{\isacharcomma}{\kern0pt}\ {\isacharbrackleft}{\kern0pt}y{\isacharprime}{\kern0pt}{\isacharcomma}{\kern0pt}\ y{\isacharprime}{\kern0pt}{\isacharprime}{\kern0pt}{\isacharprime}{\kern0pt}{\isacharbrackright}{\kern0pt}{\isacharparenright}{\kern0pt}{\isacharparenright}{\kern0pt}{\isachardoublequoteclose}\isanewline
\ \ \ \ \ \ \isacommand{apply}\isamarkupfalse%
{\isacharparenleft}{\kern0pt}rule{\isacharunderscore}{\kern0pt}tac\ P{\isacharequal}{\kern0pt}{\isachardoublequoteopen}{\isacharquery}{\kern0pt}C{\isachardoublequoteclose}\ \isakeyword{in}\ iffD{\isadigit{1}}{\isacharparenright}{\kern0pt}\isanewline
\ \ \ \ \ \ \ \isacommand{apply}\isamarkupfalse%
{\isacharparenleft}{\kern0pt}rule\ bex{\isacharunderscore}{\kern0pt}iff{\isacharcomma}{\kern0pt}\ rule\ truth{\isacharunderscore}{\kern0pt}lemma{\isacharparenright}{\kern0pt}\isanewline
\ \ \ \ \ \ \isacommand{using}\isamarkupfalse%
\ generic\ y{\isacharprime}{\kern0pt}inM\ SH\ transM\ Un{\isacharunderscore}{\kern0pt}least{\isacharunderscore}{\kern0pt}lt\isanewline
\ \ \ \ \ \ \isacommand{by}\isamarkupfalse%
\ auto\isanewline
\ \ \ \ \isacommand{then}\isamarkupfalse%
\ \isacommand{obtain}\isamarkupfalse%
\ y{\isacharprime}{\kern0pt}{\isacharprime}{\kern0pt}{\isacharprime}{\kern0pt}\ \isakeyword{where}\ y{\isacharprime}{\kern0pt}{\isacharprime}{\kern0pt}{\isacharprime}{\kern0pt}H{\isacharcolon}{\kern0pt}\ {\isachardoublequoteopen}y{\isacharprime}{\kern0pt}{\isacharprime}{\kern0pt}{\isacharprime}{\kern0pt}\ {\isasymin}\ S{\isachardoublequoteclose}\ {\isachardoublequoteopen}sats{\isacharparenleft}{\kern0pt}M{\isacharbrackleft}{\kern0pt}G{\isacharbrackright}{\kern0pt}{\isacharcomma}{\kern0pt}\ Equal{\isacharparenleft}{\kern0pt}{\isadigit{0}}{\isacharcomma}{\kern0pt}\ {\isadigit{1}}{\isacharparenright}{\kern0pt}{\isacharcomma}{\kern0pt}\ {\isacharbrackleft}{\kern0pt}val{\isacharparenleft}{\kern0pt}G{\isacharcomma}{\kern0pt}\ y{\isacharprime}{\kern0pt}{\isacharparenright}{\kern0pt}{\isacharcomma}{\kern0pt}\ val{\isacharparenleft}{\kern0pt}G{\isacharcomma}{\kern0pt}\ y{\isacharprime}{\kern0pt}{\isacharprime}{\kern0pt}{\isacharprime}{\kern0pt}{\isacharparenright}{\kern0pt}{\isacharbrackright}{\kern0pt}{\isacharparenright}{\kern0pt}{\isachardoublequoteclose}\ \isacommand{by}\isamarkupfalse%
\ auto\isanewline
\isanewline
\ \ \ \ \isacommand{then}\isamarkupfalse%
\ \isacommand{have}\isamarkupfalse%
\ {\isachardoublequoteopen}val{\isacharparenleft}{\kern0pt}G{\isacharcomma}{\kern0pt}\ y{\isacharprime}{\kern0pt}{\isacharparenright}{\kern0pt}\ {\isacharequal}{\kern0pt}\ val{\isacharparenleft}{\kern0pt}G{\isacharcomma}{\kern0pt}\ y{\isacharprime}{\kern0pt}{\isacharprime}{\kern0pt}{\isacharprime}{\kern0pt}{\isacharparenright}{\kern0pt}{\isachardoublequoteclose}\isanewline
\ \ \ \ \ \ \isacommand{apply}\isamarkupfalse%
{\isacharparenleft}{\kern0pt}subgoal{\isacharunderscore}{\kern0pt}tac\ {\isachardoublequoteopen}val{\isacharparenleft}{\kern0pt}G{\isacharcomma}{\kern0pt}\ y{\isacharprime}{\kern0pt}{\isacharparenright}{\kern0pt}\ {\isasymin}\ M{\isacharbrackleft}{\kern0pt}G{\isacharbrackright}{\kern0pt}\ {\isasymand}\ val{\isacharparenleft}{\kern0pt}G{\isacharcomma}{\kern0pt}\ y{\isacharprime}{\kern0pt}{\isacharprime}{\kern0pt}{\isacharprime}{\kern0pt}{\isacharparenright}{\kern0pt}\ {\isasymin}\ M{\isacharbrackleft}{\kern0pt}G{\isacharbrackright}{\kern0pt}{\isachardoublequoteclose}{\isacharparenright}{\kern0pt}\isanewline
\ \ \ \ \ \ \ \isacommand{apply}\isamarkupfalse%
\ simp\isanewline
\ \ \ \ \ \ \isacommand{using}\isamarkupfalse%
\ GenExt{\isacharunderscore}{\kern0pt}def\ y{\isacharprime}{\kern0pt}inM\ y{\isacharprime}{\kern0pt}{\isacharprime}{\kern0pt}{\isacharprime}{\kern0pt}H\ SH\ transM\ \isanewline
\ \ \ \ \ \ \isacommand{by}\isamarkupfalse%
\ auto\isanewline
\ \ \ \ \isacommand{then}\isamarkupfalse%
\ \isacommand{have}\isamarkupfalse%
\ {\isachardoublequoteopen}val{\isacharparenleft}{\kern0pt}G{\isacharcomma}{\kern0pt}\ y{\isacharprime}{\kern0pt}{\isacharprime}{\kern0pt}{\isacharprime}{\kern0pt}{\isacharparenright}{\kern0pt}\ {\isacharequal}{\kern0pt}\ y{\isachardoublequoteclose}\ \isacommand{using}\isamarkupfalse%
\ y{\isacharprime}{\kern0pt}H\ \isacommand{by}\isamarkupfalse%
\ auto\isanewline
\ \ \ \ \isacommand{then}\isamarkupfalse%
\ \isacommand{show}\isamarkupfalse%
\ {\isachardoublequoteopen}y\ {\isasymin}\ T{\isachardoublequoteclose}\ \isacommand{using}\isamarkupfalse%
\ y{\isacharprime}{\kern0pt}{\isacharprime}{\kern0pt}{\isacharprime}{\kern0pt}H\ TH\ \isacommand{by}\isamarkupfalse%
\ auto\isanewline
\ \ \isacommand{qed}\isamarkupfalse%
\isanewline
\isanewline
\ \ \isacommand{define}\isamarkupfalse%
\ U\ \isakeyword{where}\ {\isachardoublequoteopen}U\ {\isasymequiv}\ {\isacharbraceleft}{\kern0pt}\ y\ {\isasymin}\ T{\isachardot}{\kern0pt}\ sats{\isacharparenleft}{\kern0pt}SymExt{\isacharparenleft}{\kern0pt}G{\isacharparenright}{\kern0pt}{\isacharcomma}{\kern0pt}\ subset{\isacharunderscore}{\kern0pt}fm{\isacharparenleft}{\kern0pt}{\isadigit{0}}{\isacharcomma}{\kern0pt}\ {\isadigit{1}}{\isacharparenright}{\kern0pt}{\isacharcomma}{\kern0pt}\ {\isacharbrackleft}{\kern0pt}y{\isacharbrackright}{\kern0pt}\ {\isacharat}{\kern0pt}\ {\isacharbrackleft}{\kern0pt}x{\isacharbrackright}{\kern0pt}{\isacharparenright}{\kern0pt}\ {\isacharbraceright}{\kern0pt}{\isachardoublequoteclose}\isanewline
\isanewline
\ \ \isacommand{have}\isamarkupfalse%
\ {\isachardoublequoteopen}U\ {\isasymin}\ SymExt{\isacharparenleft}{\kern0pt}G{\isacharparenright}{\kern0pt}{\isachardoublequoteclose}\ \isanewline
\ \ \ \ \isacommand{unfolding}\isamarkupfalse%
\ U{\isacharunderscore}{\kern0pt}def\isanewline
\ \ \ \ \isacommand{apply}\isamarkupfalse%
{\isacharparenleft}{\kern0pt}rule\ SymExt{\isacharunderscore}{\kern0pt}separation{\isacharparenright}{\kern0pt}\isanewline
\ \ \ \ \isacommand{using}\isamarkupfalse%
\ TH\ assms\ subset{\isacharunderscore}{\kern0pt}fm{\isacharunderscore}{\kern0pt}def\ \isanewline
\ \ \ \ \ \ \ \isacommand{apply}\isamarkupfalse%
\ auto{\isacharbrackleft}{\kern0pt}{\isadigit{3}}{\isacharbrackright}{\kern0pt}\isanewline
\ \ \ \ \isacommand{apply}\isamarkupfalse%
{\isacharparenleft}{\kern0pt}subst\ arity{\isacharunderscore}{\kern0pt}subset{\isacharunderscore}{\kern0pt}fm{\isacharparenright}{\kern0pt}\isanewline
\ \ \ \ \isacommand{using}\isamarkupfalse%
\ Un{\isacharunderscore}{\kern0pt}least{\isacharunderscore}{\kern0pt}lt\ \isanewline
\ \ \ \ \isacommand{by}\isamarkupfalse%
\ auto\isanewline
\isanewline
\ \ \isacommand{have}\isamarkupfalse%
\ {\isachardoublequoteopen}U\ {\isacharequal}{\kern0pt}\ {\isacharbraceleft}{\kern0pt}\ y\ {\isasymin}\ T{\isachardot}{\kern0pt}\ y\ {\isasymsubseteq}\ x\ {\isacharbraceright}{\kern0pt}{\isachardoublequoteclose}\ \isanewline
\ \ \ \ \isacommand{unfolding}\isamarkupfalse%
\ U{\isacharunderscore}{\kern0pt}def\isanewline
\ \ \ \ \isacommand{apply}\isamarkupfalse%
{\isacharparenleft}{\kern0pt}rule\ iff{\isacharunderscore}{\kern0pt}eq{\isacharparenright}{\kern0pt}\isanewline
\ \ \ \ \isacommand{apply}\isamarkupfalse%
{\isacharparenleft}{\kern0pt}rule\ iff{\isacharunderscore}{\kern0pt}trans{\isacharcomma}{\kern0pt}\ rule\ sats{\isacharunderscore}{\kern0pt}subset{\isacharunderscore}{\kern0pt}fm{\isacharparenright}{\kern0pt}\isanewline
\ \ \ \ \isacommand{using}\isamarkupfalse%
\ TH\ SymExt{\isacharunderscore}{\kern0pt}trans\ assms\ Transset{\isacharunderscore}{\kern0pt}SymExt\isanewline
\ \ \ \ \isacommand{by}\isamarkupfalse%
\ auto\isanewline
\ \ \isacommand{also}\isamarkupfalse%
\ \isacommand{have}\isamarkupfalse%
\ {\isachardoublequoteopen}{\isachardot}{\kern0pt}{\isachardot}{\kern0pt}{\isachardot}{\kern0pt}\ {\isacharequal}{\kern0pt}\ Pow{\isacharparenleft}{\kern0pt}x{\isacharparenright}{\kern0pt}\ {\isasyminter}\ SymExt{\isacharparenleft}{\kern0pt}G{\isacharparenright}{\kern0pt}{\isachardoublequoteclose}\ \isanewline
\ \ \ \ \isacommand{using}\isamarkupfalse%
\ subsetT\ TH\ SymExt{\isacharunderscore}{\kern0pt}trans\ \isanewline
\ \ \ \ \isacommand{by}\isamarkupfalse%
\ auto\isanewline
\ \ \isacommand{finally}\isamarkupfalse%
\ \isacommand{show}\isamarkupfalse%
\ {\isachardoublequoteopen}Pow{\isacharparenleft}{\kern0pt}x{\isacharparenright}{\kern0pt}\ {\isasyminter}\ SymExt{\isacharparenleft}{\kern0pt}G{\isacharparenright}{\kern0pt}\ {\isasymin}\ SymExt{\isacharparenleft}{\kern0pt}G{\isacharparenright}{\kern0pt}{\isachardoublequoteclose}\ \isanewline
\ \ \ \ \isacommand{using}\isamarkupfalse%
\ {\isacartoucheopen}U\ {\isasymin}\ SymExt{\isacharparenleft}{\kern0pt}G{\isacharparenright}{\kern0pt}{\isacartoucheclose}\ \isacommand{by}\isamarkupfalse%
\ auto\isanewline
\isacommand{qed}\isamarkupfalse%
%
\endisatagproof
{\isafoldproof}%
%
\isadelimproof
\isanewline
%
\endisadelimproof
\isanewline
\isacommand{lemma}\isamarkupfalse%
\ SymExt{\isacharunderscore}{\kern0pt}foundation{\isacharunderscore}{\kern0pt}ax\ {\isacharcolon}{\kern0pt}\ {\isachardoublequoteopen}foundation{\isacharunderscore}{\kern0pt}ax{\isacharparenleft}{\kern0pt}{\isacharhash}{\kern0pt}{\isacharhash}{\kern0pt}SymExt{\isacharparenleft}{\kern0pt}G{\isacharparenright}{\kern0pt}{\isacharparenright}{\kern0pt}{\isachardoublequoteclose}\ \isanewline
%
\isadelimproof
\ \ %
\endisadelimproof
%
\isatagproof
\isacommand{apply}\isamarkupfalse%
{\isacharparenleft}{\kern0pt}rule\ ccontr{\isacharparenright}{\kern0pt}\isanewline
\ \ \isacommand{apply}\isamarkupfalse%
{\isacharparenleft}{\kern0pt}subgoal{\isacharunderscore}{\kern0pt}tac\ {\isachardoublequoteopen}{\isasymnot}foundation{\isacharunderscore}{\kern0pt}ax{\isacharparenleft}{\kern0pt}{\isacharhash}{\kern0pt}{\isacharhash}{\kern0pt}M{\isacharbrackleft}{\kern0pt}G{\isacharbrackright}{\kern0pt}{\isacharparenright}{\kern0pt}{\isachardoublequoteclose}{\isacharparenright}{\kern0pt}\isanewline
\ \ \isacommand{using}\isamarkupfalse%
\ foundation{\isacharunderscore}{\kern0pt}in{\isacharunderscore}{\kern0pt}MG\ \isanewline
\ \ \ \isacommand{apply}\isamarkupfalse%
\ force\ \isanewline
\ \ \isacommand{unfolding}\isamarkupfalse%
\ foundation{\isacharunderscore}{\kern0pt}ax{\isacharunderscore}{\kern0pt}def\ \isanewline
\ \ \isacommand{apply}\isamarkupfalse%
\ clarsimp\isanewline
\ \ \isacommand{apply}\isamarkupfalse%
{\isacharparenleft}{\kern0pt}rename{\isacharunderscore}{\kern0pt}tac\ x\ y{\isacharcomma}{\kern0pt}\ rule{\isacharunderscore}{\kern0pt}tac\ x{\isacharequal}{\kern0pt}x\ \isakeyword{in}\ bexI{\isacharcomma}{\kern0pt}\ rule\ conjI{\isacharparenright}{\kern0pt}\isanewline
\ \ \ \ \isacommand{apply}\isamarkupfalse%
{\isacharparenleft}{\kern0pt}rename{\isacharunderscore}{\kern0pt}tac\ x\ y{\isacharcomma}{\kern0pt}\ rule{\isacharunderscore}{\kern0pt}tac\ x{\isacharequal}{\kern0pt}y\ \isakeyword{in}\ bexI{\isacharcomma}{\kern0pt}\ simp{\isacharparenright}{\kern0pt}\isanewline
\ \ \isacommand{using}\isamarkupfalse%
\ SymExt{\isacharunderscore}{\kern0pt}subset{\isacharunderscore}{\kern0pt}GenExt\ \isanewline
\ \ \ \ \isacommand{apply}\isamarkupfalse%
\ force\ \isanewline
\ \ \ \isacommand{apply}\isamarkupfalse%
{\isacharparenleft}{\kern0pt}rule\ ballI{\isacharparenright}{\kern0pt}\isanewline
\ \ \ \isacommand{apply}\isamarkupfalse%
{\isacharparenleft}{\kern0pt}rename{\isacharunderscore}{\kern0pt}tac\ x\ y\ y{\isacharprime}{\kern0pt}{\isacharcomma}{\kern0pt}\ case{\isacharunderscore}{\kern0pt}tac\ {\isachardoublequoteopen}y{\isacharprime}{\kern0pt}\ {\isasymnotin}\ x{\isachardoublequoteclose}{\isacharcomma}{\kern0pt}\ force{\isacharcomma}{\kern0pt}\ simp{\isacharparenright}{\kern0pt}\isanewline
\ \ \ \isacommand{apply}\isamarkupfalse%
{\isacharparenleft}{\kern0pt}rename{\isacharunderscore}{\kern0pt}tac\ x\ y\ y{\isacharprime}{\kern0pt}{\isacharcomma}{\kern0pt}\ subgoal{\isacharunderscore}{\kern0pt}tac\ {\isachardoublequoteopen}{\isacharparenleft}{\kern0pt}{\isasymexists}z{\isasymin}SymExt{\isacharparenleft}{\kern0pt}G{\isacharparenright}{\kern0pt}{\isachardot}{\kern0pt}\ z\ {\isasymin}\ x\ {\isasymand}\ z\ {\isasymin}\ y{\isacharprime}{\kern0pt}{\isacharparenright}{\kern0pt}{\isachardoublequoteclose}{\isacharparenright}{\kern0pt}\isanewline
\ \ \ \ \isacommand{apply}\isamarkupfalse%
\ clarsimp\isanewline
\ \ \ \ \isacommand{apply}\isamarkupfalse%
{\isacharparenleft}{\kern0pt}rename{\isacharunderscore}{\kern0pt}tac\ x\ y\ y{\isacharprime}{\kern0pt}\ z{\isacharcomma}{\kern0pt}\ rule{\isacharunderscore}{\kern0pt}tac\ x{\isacharequal}{\kern0pt}z\ \isakeyword{in}\ bexI{\isacharcomma}{\kern0pt}\ force{\isacharparenright}{\kern0pt}\isanewline
\ \ \isacommand{using}\isamarkupfalse%
\ SymExt{\isacharunderscore}{\kern0pt}subset{\isacharunderscore}{\kern0pt}GenExt\ SymExt{\isacharunderscore}{\kern0pt}trans\isanewline
\ \ \ \ \isacommand{apply}\isamarkupfalse%
\ force\isanewline
\ \ \isacommand{using}\isamarkupfalse%
\ SymExt{\isacharunderscore}{\kern0pt}trans\isanewline
\ \ \ \isacommand{apply}\isamarkupfalse%
\ force\isanewline
\ \ \isacommand{using}\isamarkupfalse%
\ SymExt{\isacharunderscore}{\kern0pt}subset{\isacharunderscore}{\kern0pt}GenExt\isanewline
\ \ \isacommand{by}\isamarkupfalse%
\ auto%
\endisatagproof
{\isafoldproof}%
%
\isadelimproof
\isanewline
%
\endisadelimproof
\isanewline
\isacommand{lemma}\isamarkupfalse%
\ succ{\isacharunderscore}{\kern0pt}in{\isacharunderscore}{\kern0pt}SymExt\ {\isacharcolon}{\kern0pt}\ \isanewline
\ \ \isakeyword{fixes}\ x\isanewline
\ \ \isakeyword{assumes}\ {\isachardoublequoteopen}x\ {\isasymin}\ SymExt{\isacharparenleft}{\kern0pt}G{\isacharparenright}{\kern0pt}{\isachardoublequoteclose}\ \isanewline
\ \ \isakeyword{shows}\ {\isachardoublequoteopen}succ{\isacharparenleft}{\kern0pt}x{\isacharparenright}{\kern0pt}\ {\isasymin}\ SymExt{\isacharparenleft}{\kern0pt}G{\isacharparenright}{\kern0pt}{\isachardoublequoteclose}\ \isanewline
%
\isadelimproof
%
\endisadelimproof
%
\isatagproof
\isacommand{proof}\isamarkupfalse%
\ {\isacharminus}{\kern0pt}\ \isanewline
\ \ \isacommand{have}\isamarkupfalse%
\ {\isachardoublequoteopen}succ{\isacharparenleft}{\kern0pt}x{\isacharparenright}{\kern0pt}\ {\isacharequal}{\kern0pt}\ {\isacharbraceleft}{\kern0pt}x{\isacharbraceright}{\kern0pt}\ {\isasymunion}\ x{\isachardoublequoteclose}\ \isacommand{by}\isamarkupfalse%
\ auto\isanewline
\ \ \isacommand{also}\isamarkupfalse%
\ \isacommand{have}\isamarkupfalse%
\ {\isachardoublequoteopen}{\isachardot}{\kern0pt}{\isachardot}{\kern0pt}{\isachardot}{\kern0pt}\ {\isacharequal}{\kern0pt}\ {\isasymUnion}\ {\isacharbraceleft}{\kern0pt}\ {\isacharbraceleft}{\kern0pt}x{\isacharcomma}{\kern0pt}\ x{\isacharbraceright}{\kern0pt}\ {\isacharcomma}{\kern0pt}\ x\ {\isacharbraceright}{\kern0pt}{\isachardoublequoteclose}\ \isacommand{by}\isamarkupfalse%
\ auto\isanewline
\ \ \isacommand{finally}\isamarkupfalse%
\ \isacommand{have}\isamarkupfalse%
\ H{\isacharcolon}{\kern0pt}\ {\isachardoublequoteopen}succ{\isacharparenleft}{\kern0pt}x{\isacharparenright}{\kern0pt}\ {\isacharequal}{\kern0pt}\ {\isasymUnion}\ {\isacharbraceleft}{\kern0pt}\ {\isacharbraceleft}{\kern0pt}x{\isacharcomma}{\kern0pt}\ x{\isacharbraceright}{\kern0pt}\ {\isacharcomma}{\kern0pt}\ x\ {\isacharbraceright}{\kern0pt}{\isachardoublequoteclose}\ \isacommand{by}\isamarkupfalse%
\ auto\isanewline
\isanewline
\ \ \isacommand{have}\isamarkupfalse%
\ {\isachardoublequoteopen}{\isasymUnion}\ {\isacharbraceleft}{\kern0pt}\ {\isacharbraceleft}{\kern0pt}x{\isacharcomma}{\kern0pt}\ x{\isacharbraceright}{\kern0pt}\ {\isacharcomma}{\kern0pt}\ x\ {\isacharbraceright}{\kern0pt}\ {\isasymin}\ SymExt{\isacharparenleft}{\kern0pt}G{\isacharparenright}{\kern0pt}{\isachardoublequoteclose}\ \isanewline
\ \ \ \ \isacommand{apply}\isamarkupfalse%
{\isacharparenleft}{\kern0pt}rule\ Union{\isacharunderscore}{\kern0pt}in{\isacharunderscore}{\kern0pt}SymExt{\isacharparenright}{\kern0pt}\isanewline
\ \ \ \ \isacommand{apply}\isamarkupfalse%
{\isacharparenleft}{\kern0pt}rule\ upair{\isacharunderscore}{\kern0pt}in{\isacharunderscore}{\kern0pt}SymExt{\isacharparenright}{\kern0pt}{\isacharplus}{\kern0pt}\isanewline
\ \ \ \ \isacommand{using}\isamarkupfalse%
\ assms\isanewline
\ \ \ \ \isacommand{by}\isamarkupfalse%
\ auto\isanewline
\ \ \isacommand{then}\isamarkupfalse%
\ \isacommand{show}\isamarkupfalse%
\ {\isacharquery}{\kern0pt}thesis\ \isacommand{using}\isamarkupfalse%
\ H\ \isacommand{by}\isamarkupfalse%
\ auto\isanewline
\isacommand{qed}\isamarkupfalse%
%
\endisatagproof
{\isafoldproof}%
%
\isadelimproof
\isanewline
%
\endisadelimproof
\isanewline
\isacommand{lemma}\isamarkupfalse%
\ SymExt{\isacharunderscore}{\kern0pt}infinity{\isacharunderscore}{\kern0pt}ax\ {\isacharcolon}{\kern0pt}\ {\isachardoublequoteopen}infinity{\isacharunderscore}{\kern0pt}ax{\isacharparenleft}{\kern0pt}{\isacharhash}{\kern0pt}{\isacharhash}{\kern0pt}SymExt{\isacharparenleft}{\kern0pt}G{\isacharparenright}{\kern0pt}{\isacharparenright}{\kern0pt}{\isachardoublequoteclose}\isanewline
%
\isadelimproof
\ \ %
\endisadelimproof
%
\isatagproof
\isacommand{unfolding}\isamarkupfalse%
\ infinity{\isacharunderscore}{\kern0pt}ax{\isacharunderscore}{\kern0pt}def\ \isanewline
\ \ \isacommand{apply}\isamarkupfalse%
{\isacharparenleft}{\kern0pt}rule{\isacharunderscore}{\kern0pt}tac\ x{\isacharequal}{\kern0pt}nat\ \isakeyword{in}\ rexI{\isacharcomma}{\kern0pt}\ rule\ conjI{\isacharparenright}{\kern0pt}\isanewline
\ \ \ \ \isacommand{apply}\isamarkupfalse%
{\isacharparenleft}{\kern0pt}rule{\isacharunderscore}{\kern0pt}tac\ x{\isacharequal}{\kern0pt}{\isadigit{0}}\ \isakeyword{in}\ rexI{\isacharparenright}{\kern0pt}\isanewline
\ \ \isacommand{unfolding}\isamarkupfalse%
\ empty{\isacharunderscore}{\kern0pt}def\ \isanewline
\ \ \isacommand{using}\isamarkupfalse%
\ zero{\isacharunderscore}{\kern0pt}in{\isacharunderscore}{\kern0pt}SymExt\ \isanewline
\ \ \ \ \ \isacommand{apply}\isamarkupfalse%
\ auto{\isacharbrackleft}{\kern0pt}{\isadigit{2}}{\isacharbrackright}{\kern0pt}\isanewline
\ \ \ \isacommand{apply}\isamarkupfalse%
{\isacharparenleft}{\kern0pt}rule\ rallI{\isacharcomma}{\kern0pt}\ rule\ impI{\isacharparenright}{\kern0pt}\isanewline
\ \ \ \isacommand{apply}\isamarkupfalse%
{\isacharparenleft}{\kern0pt}rename{\isacharunderscore}{\kern0pt}tac\ n{\isacharcomma}{\kern0pt}\ rule{\isacharunderscore}{\kern0pt}tac\ x{\isacharequal}{\kern0pt}{\isachardoublequoteopen}succ{\isacharparenleft}{\kern0pt}n{\isacharparenright}{\kern0pt}{\isachardoublequoteclose}\ \isakeyword{in}\ rexI{\isacharparenright}{\kern0pt}\isanewline
\ \ \isacommand{unfolding}\isamarkupfalse%
\ successor{\isacharunderscore}{\kern0pt}def\ is{\isacharunderscore}{\kern0pt}cons{\isacharunderscore}{\kern0pt}def\ \isanewline
\ \ \ \ \isacommand{apply}\isamarkupfalse%
{\isacharparenleft}{\kern0pt}rule\ conjI{\isacharcomma}{\kern0pt}\ rename{\isacharunderscore}{\kern0pt}tac\ n{\isacharcomma}{\kern0pt}\ rule{\isacharunderscore}{\kern0pt}tac\ x{\isacharequal}{\kern0pt}{\isachardoublequoteopen}{\isacharbraceleft}{\kern0pt}n{\isacharbraceright}{\kern0pt}{\isachardoublequoteclose}\ \isakeyword{in}\ rexI{\isacharcomma}{\kern0pt}\ rule\ conjI{\isacharparenright}{\kern0pt}\isanewline
\ \ \isacommand{unfolding}\isamarkupfalse%
\ upair{\isacharunderscore}{\kern0pt}def\ \isanewline
\ \ \ \ \ \ \ \isacommand{apply}\isamarkupfalse%
\ force\isanewline
\ \ \isacommand{unfolding}\isamarkupfalse%
\ union{\isacharunderscore}{\kern0pt}def\ \isanewline
\ \ \ \ \ \ \isacommand{apply}\isamarkupfalse%
\ force\ \isanewline
\ \ \ \ \ \isacommand{apply}\isamarkupfalse%
\ simp\isanewline
\ \ \ \ \ \isacommand{apply}\isamarkupfalse%
{\isacharparenleft}{\kern0pt}rename{\isacharunderscore}{\kern0pt}tac\ n{\isacharcomma}{\kern0pt}\ rule{\isacharunderscore}{\kern0pt}tac\ b{\isacharequal}{\kern0pt}{\isachardoublequoteopen}{\isacharbraceleft}{\kern0pt}n{\isacharbraceright}{\kern0pt}{\isachardoublequoteclose}\ \isakeyword{and}\ a{\isacharequal}{\kern0pt}{\isachardoublequoteopen}{\isacharbraceleft}{\kern0pt}n{\isacharcomma}{\kern0pt}\ n{\isacharbraceright}{\kern0pt}{\isachardoublequoteclose}\ \isakeyword{in}\ ssubst{\isacharcomma}{\kern0pt}\ force{\isacharparenright}{\kern0pt}\isanewline
\ \ \ \ \ \isacommand{apply}\isamarkupfalse%
\ {\isacharparenleft}{\kern0pt}rule\ upair{\isacharunderscore}{\kern0pt}in{\isacharunderscore}{\kern0pt}SymExt{\isacharparenright}{\kern0pt}\isanewline
\ \ \isacommand{using}\isamarkupfalse%
\ succ{\isacharunderscore}{\kern0pt}in{\isacharunderscore}{\kern0pt}SymExt\ \isanewline
\ \ \ \ \ \ \isacommand{apply}\isamarkupfalse%
\ auto{\isacharbrackleft}{\kern0pt}{\isadigit{4}}{\isacharbrackright}{\kern0pt}\isanewline
\ \ \isacommand{using}\isamarkupfalse%
\ nat{\isacharunderscore}{\kern0pt}in{\isacharunderscore}{\kern0pt}M\ M{\isacharunderscore}{\kern0pt}subset{\isacharunderscore}{\kern0pt}SymExt\ \isanewline
\ \ \isacommand{by}\isamarkupfalse%
\ auto%
\endisatagproof
{\isafoldproof}%
%
\isadelimproof
\isanewline
%
\endisadelimproof
\isanewline
\isacommand{lemma}\isamarkupfalse%
\ SymExt{\isacharunderscore}{\kern0pt}M{\isacharunderscore}{\kern0pt}ZF\ {\isacharcolon}{\kern0pt}\ {\isachardoublequoteopen}M{\isacharunderscore}{\kern0pt}ZF{\isacharparenleft}{\kern0pt}SymExt{\isacharparenleft}{\kern0pt}G{\isacharparenright}{\kern0pt}{\isacharparenright}{\kern0pt}{\isachardoublequoteclose}\ \isanewline
%
\isadelimproof
\ \ %
\endisadelimproof
%
\isatagproof
\isacommand{unfolding}\isamarkupfalse%
\ M{\isacharunderscore}{\kern0pt}ZF{\isacharunderscore}{\kern0pt}def\ \isanewline
\ \ \isacommand{apply}\isamarkupfalse%
{\isacharparenleft}{\kern0pt}rule\ conjI{\isacharparenright}{\kern0pt}{\isacharplus}{\kern0pt}\isanewline
\ \ \isacommand{unfolding}\isamarkupfalse%
\ upair{\isacharunderscore}{\kern0pt}ax{\isacharunderscore}{\kern0pt}def\ upair{\isacharunderscore}{\kern0pt}def\ \isanewline
\ \ \ \ \ \isacommand{apply}\isamarkupfalse%
{\isacharparenleft}{\kern0pt}rule\ rallI{\isacharparenright}{\kern0pt}{\isacharplus}{\kern0pt}\isanewline
\ \ \ \ \ \isacommand{apply}\isamarkupfalse%
{\isacharparenleft}{\kern0pt}rename{\isacharunderscore}{\kern0pt}tac\ x\ y{\isacharcomma}{\kern0pt}\ rule{\isacharunderscore}{\kern0pt}tac\ x{\isacharequal}{\kern0pt}{\isachardoublequoteopen}{\isacharbraceleft}{\kern0pt}x{\isacharcomma}{\kern0pt}\ y{\isacharbraceright}{\kern0pt}{\isachardoublequoteclose}\ \isakeyword{in}\ rexI{\isacharparenright}{\kern0pt}\isanewline
\ \ \ \ \ \ \isacommand{apply}\isamarkupfalse%
{\isacharparenleft}{\kern0pt}rule\ conjI{\isacharcomma}{\kern0pt}\ simp{\isacharcomma}{\kern0pt}\ rule\ conjI{\isacharcomma}{\kern0pt}\ simp{\isacharparenright}{\kern0pt}\isanewline
\ \ \ \ \ \ \isacommand{apply}\isamarkupfalse%
{\isacharparenleft}{\kern0pt}rule\ rallI{\isacharcomma}{\kern0pt}\ rule\ impI{\isacharcomma}{\kern0pt}\ force{\isacharcomma}{\kern0pt}\ simp{\isacharparenright}{\kern0pt}\isanewline
\ \ \ \ \ \isacommand{apply}\isamarkupfalse%
{\isacharparenleft}{\kern0pt}rule\ upair{\isacharunderscore}{\kern0pt}in{\isacharunderscore}{\kern0pt}SymExt{\isacharparenright}{\kern0pt}\isanewline
\ \ \ \ \ \ \isacommand{apply}\isamarkupfalse%
\ auto{\isacharbrackleft}{\kern0pt}{\isadigit{2}}{\isacharbrackright}{\kern0pt}\isanewline
\ \ \isacommand{unfolding}\isamarkupfalse%
\ Union{\isacharunderscore}{\kern0pt}ax{\isacharunderscore}{\kern0pt}def\ \isanewline
\ \ \ \ \isacommand{apply}\isamarkupfalse%
{\isacharparenleft}{\kern0pt}rule\ rallI{\isacharparenright}{\kern0pt}\isanewline
\ \ \ \ \isacommand{apply}\isamarkupfalse%
{\isacharparenleft}{\kern0pt}rename{\isacharunderscore}{\kern0pt}tac\ x{\isacharcomma}{\kern0pt}\ rule{\isacharunderscore}{\kern0pt}tac\ x{\isacharequal}{\kern0pt}{\isachardoublequoteopen}{\isasymUnion}x{\isachardoublequoteclose}\ \isakeyword{in}\ rexI{\isacharparenright}{\kern0pt}\isanewline
\ \ \isacommand{unfolding}\isamarkupfalse%
\ big{\isacharunderscore}{\kern0pt}union{\isacharunderscore}{\kern0pt}def\ \isanewline
\ \ \ \ \ \isacommand{apply}\isamarkupfalse%
{\isacharparenleft}{\kern0pt}rule\ rallI{\isacharparenright}{\kern0pt}\isanewline
\ \ \isacommand{using}\isamarkupfalse%
\ SymExt{\isacharunderscore}{\kern0pt}trans\ \isanewline
\ \ \ \ \ \isacommand{apply}\isamarkupfalse%
\ force\ \isanewline
\ \ \isacommand{using}\isamarkupfalse%
\ Union{\isacharunderscore}{\kern0pt}in{\isacharunderscore}{\kern0pt}SymExt\ \isanewline
\ \ \ \ \isacommand{apply}\isamarkupfalse%
\ force\ \isanewline
\ \ \ \isacommand{apply}\isamarkupfalse%
{\isacharparenleft}{\kern0pt}rule\ conjI{\isacharparenright}{\kern0pt}\isanewline
\ \ \isacommand{unfolding}\isamarkupfalse%
\ power{\isacharunderscore}{\kern0pt}ax{\isacharunderscore}{\kern0pt}def\ \isanewline
\ \ \ \ \isacommand{apply}\isamarkupfalse%
{\isacharparenleft}{\kern0pt}rule\ rallI{\isacharcomma}{\kern0pt}\ rename{\isacharunderscore}{\kern0pt}tac\ x{\isacharcomma}{\kern0pt}\ rule{\isacharunderscore}{\kern0pt}tac\ x{\isacharequal}{\kern0pt}{\isachardoublequoteopen}Pow{\isacharparenleft}{\kern0pt}x{\isacharparenright}{\kern0pt}\ {\isasyminter}\ SymExt{\isacharparenleft}{\kern0pt}G{\isacharparenright}{\kern0pt}{\isachardoublequoteclose}\ \isakeyword{in}\ rexI{\isacharparenright}{\kern0pt}\isanewline
\ \ \isacommand{unfolding}\isamarkupfalse%
\ powerset{\isacharunderscore}{\kern0pt}def\ subset{\isacharunderscore}{\kern0pt}def\ \isanewline
\ \ \ \ \ \isacommand{apply}\isamarkupfalse%
{\isacharparenleft}{\kern0pt}rule\ rallI{\isacharcomma}{\kern0pt}\ rule\ iffI{\isacharcomma}{\kern0pt}\ force{\isacharparenright}{\kern0pt}\isanewline
\ \ \ \ \ \isacommand{apply}\isamarkupfalse%
\ simp\isanewline
\ \ \ \ \ \isacommand{apply}\isamarkupfalse%
{\isacharparenleft}{\kern0pt}rule\ subsetI{\isacharparenright}{\kern0pt}\isanewline
\ \ \ \ \ \isacommand{apply}\isamarkupfalse%
{\isacharparenleft}{\kern0pt}rename{\isacharunderscore}{\kern0pt}tac\ x\ y\ z{\isacharcomma}{\kern0pt}\ subgoal{\isacharunderscore}{\kern0pt}tac\ {\isachardoublequoteopen}z\ {\isasymin}\ SymExt{\isacharparenleft}{\kern0pt}G{\isacharparenright}{\kern0pt}{\isachardoublequoteclose}{\isacharcomma}{\kern0pt}\ force{\isacharparenright}{\kern0pt}\isanewline
\ \ \isacommand{using}\isamarkupfalse%
\ SymExt{\isacharunderscore}{\kern0pt}trans\isanewline
\ \ \ \ \ \isacommand{apply}\isamarkupfalse%
{\isacharparenleft}{\kern0pt}rename{\isacharunderscore}{\kern0pt}tac\ x\ y\ z{\isacharcomma}{\kern0pt}\ rule{\isacharunderscore}{\kern0pt}tac\ x{\isacharequal}{\kern0pt}y\ \isakeyword{in}\ SymExt{\isacharunderscore}{\kern0pt}trans{\isacharcomma}{\kern0pt}\ force{\isacharcomma}{\kern0pt}\ force{\isacharparenright}{\kern0pt}\isanewline
\ \ \isacommand{using}\isamarkupfalse%
\ powerset{\isacharunderscore}{\kern0pt}in{\isacharunderscore}{\kern0pt}SymExt\ \isanewline
\ \ \ \ \isacommand{apply}\isamarkupfalse%
\ force\ \isanewline
\ \ \isacommand{unfolding}\isamarkupfalse%
\ extensionality{\isacharunderscore}{\kern0pt}def\isanewline
\ \ \ \isacommand{apply}\isamarkupfalse%
{\isacharparenleft}{\kern0pt}rule\ rallI{\isacharparenright}{\kern0pt}{\isacharplus}{\kern0pt}\isanewline
\ \ \ \isacommand{apply}\isamarkupfalse%
{\isacharparenleft}{\kern0pt}rule\ impI{\isacharparenright}{\kern0pt}\isanewline
\ \ \ \isacommand{apply}\isamarkupfalse%
{\isacharparenleft}{\kern0pt}rule\ equality{\isacharunderscore}{\kern0pt}iffI{\isacharcomma}{\kern0pt}\ rule\ iffI{\isacharparenright}{\kern0pt}\isanewline
\ \ \ \ \isacommand{apply}\isamarkupfalse%
{\isacharparenleft}{\kern0pt}rename{\isacharunderscore}{\kern0pt}tac\ x\ y\ z{\isacharcomma}{\kern0pt}\ subgoal{\isacharunderscore}{\kern0pt}tac\ {\isachardoublequoteopen}z\ {\isasymin}\ SymExt{\isacharparenleft}{\kern0pt}G{\isacharparenright}{\kern0pt}{\isachardoublequoteclose}{\isacharcomma}{\kern0pt}\ force{\isacharparenright}{\kern0pt}\isanewline
\ \ \ \ \isacommand{apply}\isamarkupfalse%
{\isacharparenleft}{\kern0pt}rename{\isacharunderscore}{\kern0pt}tac\ x\ y\ z{\isacharcomma}{\kern0pt}\ rule{\isacharunderscore}{\kern0pt}tac\ x{\isacharequal}{\kern0pt}x\ \isakeyword{in}\ SymExt{\isacharunderscore}{\kern0pt}trans{\isacharcomma}{\kern0pt}\ force{\isacharcomma}{\kern0pt}\ force{\isacharparenright}{\kern0pt}\isanewline
\ \ \ \ \isacommand{apply}\isamarkupfalse%
{\isacharparenleft}{\kern0pt}rename{\isacharunderscore}{\kern0pt}tac\ x\ y\ z{\isacharcomma}{\kern0pt}\ subgoal{\isacharunderscore}{\kern0pt}tac\ {\isachardoublequoteopen}z\ {\isasymin}\ SymExt{\isacharparenleft}{\kern0pt}G{\isacharparenright}{\kern0pt}{\isachardoublequoteclose}{\isacharcomma}{\kern0pt}\ force{\isacharparenright}{\kern0pt}\isanewline
\ \ \ \isacommand{apply}\isamarkupfalse%
{\isacharparenleft}{\kern0pt}rename{\isacharunderscore}{\kern0pt}tac\ x\ y\ z{\isacharcomma}{\kern0pt}\ rule{\isacharunderscore}{\kern0pt}tac\ x{\isacharequal}{\kern0pt}y\ \isakeyword{in}\ SymExt{\isacharunderscore}{\kern0pt}trans{\isacharcomma}{\kern0pt}\ force{\isacharcomma}{\kern0pt}\ force{\isacharparenright}{\kern0pt}\isanewline
\ \ \isacommand{apply}\isamarkupfalse%
{\isacharparenleft}{\kern0pt}rule\ conjI{\isacharparenright}{\kern0pt}{\isacharplus}{\kern0pt}\isanewline
\ \ \isacommand{using}\isamarkupfalse%
\ SymExt{\isacharunderscore}{\kern0pt}foundation{\isacharunderscore}{\kern0pt}ax\ \isanewline
\ \ \ \ \isacommand{apply}\isamarkupfalse%
\ force\ \isanewline
\ \ \ \isacommand{apply}\isamarkupfalse%
{\isacharparenleft}{\kern0pt}rule\ SymExt{\isacharunderscore}{\kern0pt}infinity{\isacharunderscore}{\kern0pt}ax{\isacharparenright}{\kern0pt}\isanewline
\ \ \isacommand{apply}\isamarkupfalse%
{\isacharparenleft}{\kern0pt}rule\ conjI{\isacharparenright}{\kern0pt}\isanewline
\ \ \ \isacommand{apply}\isamarkupfalse%
{\isacharparenleft}{\kern0pt}rule\ allI{\isacharparenright}{\kern0pt}{\isacharplus}{\kern0pt}\isanewline
\ \ \ \isacommand{apply}\isamarkupfalse%
{\isacharparenleft}{\kern0pt}rule\ impI{\isacharparenright}{\kern0pt}{\isacharplus}{\kern0pt}\isanewline
\ \ \isacommand{unfolding}\isamarkupfalse%
\ separation{\isacharunderscore}{\kern0pt}def\ \isanewline
\ \ \ \isacommand{apply}\isamarkupfalse%
{\isacharparenleft}{\kern0pt}rule\ rallI{\isacharparenright}{\kern0pt}\isanewline
\ \ \ \isacommand{apply}\isamarkupfalse%
{\isacharparenleft}{\kern0pt}rename{\isacharunderscore}{\kern0pt}tac\ {\isasymphi}\ env\ x{\isacharcomma}{\kern0pt}\ rule{\isacharunderscore}{\kern0pt}tac\ x{\isacharequal}{\kern0pt}{\isachardoublequoteopen}{\isacharbraceleft}{\kern0pt}\ y\ {\isasymin}\ x{\isachardot}{\kern0pt}\ sats{\isacharparenleft}{\kern0pt}SymExt{\isacharparenleft}{\kern0pt}G{\isacharparenright}{\kern0pt}{\isacharcomma}{\kern0pt}\ {\isasymphi}{\isacharcomma}{\kern0pt}\ {\isacharbrackleft}{\kern0pt}y{\isacharbrackright}{\kern0pt}{\isacharat}{\kern0pt}env{\isacharparenright}{\kern0pt}\ {\isacharbraceright}{\kern0pt}{\isachardoublequoteclose}\ \isakeyword{in}\ rexI{\isacharparenright}{\kern0pt}\isanewline
\ \ \ \ \isacommand{apply}\isamarkupfalse%
\ force\ \isanewline
\ \ \isacommand{using}\isamarkupfalse%
\ SymExt{\isacharunderscore}{\kern0pt}separation\isanewline
\ \ \ \isacommand{apply}\isamarkupfalse%
\ force\ \isanewline
\ \ \ \isacommand{apply}\isamarkupfalse%
{\isacharparenleft}{\kern0pt}rule\ allI{\isacharparenright}{\kern0pt}{\isacharplus}{\kern0pt}\isanewline
\ \ \ \isacommand{apply}\isamarkupfalse%
{\isacharparenleft}{\kern0pt}rule\ impI{\isacharparenright}{\kern0pt}{\isacharplus}{\kern0pt}\isanewline
\ \ \isacommand{using}\isamarkupfalse%
\ SymExt{\isacharunderscore}{\kern0pt}replacement\ \isanewline
\ \ \isacommand{by}\isamarkupfalse%
\ auto%
\endisatagproof
{\isafoldproof}%
%
\isadelimproof
\isanewline
%
\endisadelimproof
\isanewline
\isacommand{lemma}\isamarkupfalse%
\ SymExt{\isacharunderscore}{\kern0pt}M{\isacharunderscore}{\kern0pt}ZF{\isacharunderscore}{\kern0pt}trans\ {\isacharcolon}{\kern0pt}\ {\isachardoublequoteopen}M{\isacharunderscore}{\kern0pt}ZF{\isacharunderscore}{\kern0pt}trans{\isacharparenleft}{\kern0pt}SymExt{\isacharparenleft}{\kern0pt}G{\isacharparenright}{\kern0pt}{\isacharparenright}{\kern0pt}{\isachardoublequoteclose}\isanewline
%
\isadelimproof
\ \ %
\endisadelimproof
%
\isatagproof
\isacommand{unfolding}\isamarkupfalse%
\ M{\isacharunderscore}{\kern0pt}ZF{\isacharunderscore}{\kern0pt}trans{\isacharunderscore}{\kern0pt}def\ \isanewline
\ \ \isacommand{using}\isamarkupfalse%
\ SymExt{\isacharunderscore}{\kern0pt}M{\isacharunderscore}{\kern0pt}ZF\ M{\isacharunderscore}{\kern0pt}ZF{\isacharunderscore}{\kern0pt}def\ Transset{\isacharunderscore}{\kern0pt}SymExt\isanewline
\ \ \isacommand{by}\isamarkupfalse%
\ auto%
\endisatagproof
{\isafoldproof}%
%
\isadelimproof
\isanewline
%
\endisadelimproof
\isanewline
\isacommand{theorem}\isamarkupfalse%
\ SymExt{\isacharunderscore}{\kern0pt}sats{\isacharunderscore}{\kern0pt}ZF\ {\isacharcolon}{\kern0pt}\ {\isachardoublequoteopen}SymExt{\isacharparenleft}{\kern0pt}G{\isacharparenright}{\kern0pt}\ {\isasymTurnstile}\ ZF{\isachardoublequoteclose}\ \isanewline
%
\isadelimproof
\ \ %
\endisadelimproof
%
\isatagproof
\isacommand{using}\isamarkupfalse%
\ M{\isacharunderscore}{\kern0pt}ZF{\isacharunderscore}{\kern0pt}iff{\isacharunderscore}{\kern0pt}M{\isacharunderscore}{\kern0pt}satT\ SymExt{\isacharunderscore}{\kern0pt}M{\isacharunderscore}{\kern0pt}ZF\ \isacommand{by}\isamarkupfalse%
\ auto%
\endisatagproof
{\isafoldproof}%
%
\isadelimproof
\isanewline
%
\endisadelimproof
\isanewline
\isacommand{end}\isamarkupfalse%
\isanewline
%
\isadelimtheory
%
\endisadelimtheory
%
\isatagtheory
\isacommand{end}\isamarkupfalse%
%
\endisatagtheory
{\isafoldtheory}%
%
\isadelimtheory
%
\endisadelimtheory
%
\end{isabellebody}%
\endinput
%:%file=~/source/repos/ZF-notAC/code/SymExt_ZF.thy%:%
%:%10=1%:%
%:%11=1%:%
%:%12=2%:%
%:%13=3%:%
%:%18=3%:%
%:%21=4%:%
%:%22=5%:%
%:%23=5%:%
%:%24=6%:%
%:%25=7%:%
%:%26=7%:%
%:%29=8%:%
%:%33=8%:%
%:%34=8%:%
%:%35=8%:%
%:%40=8%:%
%:%43=9%:%
%:%44=10%:%
%:%45=10%:%
%:%46=11%:%
%:%47=12%:%
%:%48=13%:%
%:%55=14%:%
%:%56=14%:%
%:%57=15%:%
%:%58=15%:%
%:%59=16%:%
%:%60=16%:%
%:%61=17%:%
%:%62=17%:%
%:%63=18%:%
%:%64=18%:%
%:%65=19%:%
%:%66=19%:%
%:%67=20%:%
%:%68=20%:%
%:%69=21%:%
%:%70=21%:%
%:%71=22%:%
%:%72=22%:%
%:%73=23%:%
%:%74=23%:%
%:%75=24%:%
%:%76=24%:%
%:%77=24%:%
%:%78=24%:%
%:%79=24%:%
%:%80=25%:%
%:%81=26%:%
%:%82=26%:%
%:%83=27%:%
%:%84=27%:%
%:%85=28%:%
%:%86=28%:%
%:%87=29%:%
%:%88=29%:%
%:%89=30%:%
%:%90=30%:%
%:%91=31%:%
%:%92=31%:%
%:%93=32%:%
%:%94=32%:%
%:%95=33%:%
%:%96=33%:%
%:%97=34%:%
%:%98=34%:%
%:%99=35%:%
%:%100=35%:%
%:%101=36%:%
%:%102=36%:%
%:%103=37%:%
%:%104=37%:%
%:%105=38%:%
%:%106=38%:%
%:%107=39%:%
%:%108=39%:%
%:%109=40%:%
%:%110=40%:%
%:%111=40%:%
%:%112=40%:%
%:%113=40%:%
%:%114=41%:%
%:%115=41%:%
%:%116=41%:%
%:%117=41%:%
%:%118=41%:%
%:%119=42%:%
%:%125=42%:%
%:%128=43%:%
%:%129=44%:%
%:%130=44%:%
%:%131=45%:%
%:%132=46%:%
%:%133=47%:%
%:%140=48%:%
%:%141=48%:%
%:%142=49%:%
%:%143=49%:%
%:%144=49%:%
%:%145=49%:%
%:%146=50%:%
%:%147=50%:%
%:%148=51%:%
%:%149=52%:%
%:%150=52%:%
%:%151=53%:%
%:%152=53%:%
%:%153=54%:%
%:%154=54%:%
%:%155=54%:%
%:%156=55%:%
%:%157=55%:%
%:%158=55%:%
%:%159=55%:%
%:%160=55%:%
%:%161=56%:%
%:%162=56%:%
%:%163=56%:%
%:%164=56%:%
%:%165=57%:%
%:%166=57%:%
%:%167=57%:%
%:%168=57%:%
%:%169=57%:%
%:%170=58%:%
%:%171=58%:%
%:%172=58%:%
%:%173=58%:%
%:%174=58%:%
%:%175=59%:%
%:%176=59%:%
%:%177=60%:%
%:%178=60%:%
%:%179=60%:%
%:%180=61%:%
%:%181=61%:%
%:%182=62%:%
%:%183=62%:%
%:%184=63%:%
%:%185=63%:%
%:%186=64%:%
%:%187=64%:%
%:%188=65%:%
%:%189=65%:%
%:%190=65%:%
%:%191=65%:%
%:%192=66%:%
%:%193=66%:%
%:%194=67%:%
%:%195=67%:%
%:%196=68%:%
%:%197=68%:%
%:%198=69%:%
%:%199=69%:%
%:%200=70%:%
%:%201=70%:%
%:%202=70%:%
%:%203=70%:%
%:%204=71%:%
%:%205=71%:%
%:%206=71%:%
%:%207=71%:%
%:%208=71%:%
%:%209=72%:%
%:%210=72%:%
%:%211=72%:%
%:%212=73%:%
%:%213=73%:%
%:%214=74%:%
%:%215=74%:%
%:%216=75%:%
%:%217=75%:%
%:%218=76%:%
%:%219=76%:%
%:%220=77%:%
%:%221=77%:%
%:%222=77%:%
%:%223=77%:%
%:%224=77%:%
%:%225=78%:%
%:%226=78%:%
%:%227=78%:%
%:%228=79%:%
%:%229=79%:%
%:%230=80%:%
%:%231=80%:%
%:%232=81%:%
%:%233=81%:%
%:%234=82%:%
%:%235=82%:%
%:%236=83%:%
%:%237=83%:%
%:%238=83%:%
%:%239=83%:%
%:%240=84%:%
%:%241=84%:%
%:%242=84%:%
%:%243=85%:%
%:%244=85%:%
%:%245=86%:%
%:%246=86%:%
%:%247=87%:%
%:%248=87%:%
%:%249=88%:%
%:%250=88%:%
%:%251=89%:%
%:%252=89%:%
%:%253=90%:%
%:%254=90%:%
%:%255=90%:%
%:%256=90%:%
%:%257=90%:%
%:%258=91%:%
%:%259=91%:%
%:%260=91%:%
%:%261=91%:%
%:%262=91%:%
%:%263=92%:%
%:%264=92%:%
%:%265=93%:%
%:%266=94%:%
%:%267=94%:%
%:%268=95%:%
%:%269=95%:%
%:%270=96%:%
%:%271=96%:%
%:%272=97%:%
%:%273=97%:%
%:%274=98%:%
%:%275=98%:%
%:%276=99%:%
%:%277=99%:%
%:%278=100%:%
%:%279=100%:%
%:%280=101%:%
%:%281=101%:%
%:%282=102%:%
%:%283=102%:%
%:%284=103%:%
%:%285=104%:%
%:%286=104%:%
%:%287=105%:%
%:%288=105%:%
%:%289=106%:%
%:%290=106%:%
%:%291=107%:%
%:%292=107%:%
%:%293=108%:%
%:%294=108%:%
%:%295=109%:%
%:%296=109%:%
%:%297=110%:%
%:%298=110%:%
%:%299=111%:%
%:%300=111%:%
%:%301=111%:%
%:%302=112%:%
%:%303=112%:%
%:%304=113%:%
%:%305=113%:%
%:%306=114%:%
%:%307=114%:%
%:%308=114%:%
%:%309=115%:%
%:%310=115%:%
%:%311=116%:%
%:%312=116%:%
%:%313=116%:%
%:%314=117%:%
%:%315=117%:%
%:%316=118%:%
%:%317=118%:%
%:%318=119%:%
%:%319=119%:%
%:%320=119%:%
%:%321=120%:%
%:%322=120%:%
%:%323=120%:%
%:%324=121%:%
%:%330=121%:%
%:%333=122%:%
%:%334=123%:%
%:%335=123%:%
%:%336=124%:%
%:%337=125%:%
%:%338=126%:%
%:%341=127%:%
%:%342=128%:%
%:%346=128%:%
%:%347=128%:%
%:%348=129%:%
%:%349=129%:%
%:%350=130%:%
%:%351=130%:%
%:%352=131%:%
%:%353=131%:%
%:%354=132%:%
%:%355=132%:%
%:%360=132%:%
%:%363=133%:%
%:%364=134%:%
%:%365=134%:%
%:%366=135%:%
%:%367=136%:%
%:%368=137%:%
%:%375=138%:%
%:%376=138%:%
%:%377=139%:%
%:%378=139%:%
%:%379=139%:%
%:%380=139%:%
%:%381=140%:%
%:%382=140%:%
%:%383=140%:%
%:%384=140%:%
%:%385=140%:%
%:%386=141%:%
%:%387=141%:%
%:%388=141%:%
%:%389=141%:%
%:%390=142%:%
%:%391=142%:%
%:%392=142%:%
%:%393=143%:%
%:%394=143%:%
%:%395=144%:%
%:%396=144%:%
%:%397=145%:%
%:%398=145%:%
%:%399=146%:%
%:%400=146%:%
%:%401=146%:%
%:%402=146%:%
%:%403=147%:%
%:%404=148%:%
%:%405=148%:%
%:%406=149%:%
%:%407=149%:%
%:%408=150%:%
%:%409=150%:%
%:%410=151%:%
%:%411=152%:%
%:%412=152%:%
%:%413=153%:%
%:%414=153%:%
%:%415=154%:%
%:%416=154%:%
%:%417=155%:%
%:%418=155%:%
%:%419=156%:%
%:%420=156%:%
%:%421=156%:%
%:%422=157%:%
%:%423=157%:%
%:%424=158%:%
%:%425=158%:%
%:%426=158%:%
%:%427=158%:%
%:%428=159%:%
%:%429=160%:%
%:%430=160%:%
%:%431=161%:%
%:%432=161%:%
%:%433=162%:%
%:%434=162%:%
%:%435=163%:%
%:%436=163%:%
%:%437=164%:%
%:%438=164%:%
%:%439=164%:%
%:%440=164%:%
%:%441=165%:%
%:%442=166%:%
%:%443=166%:%
%:%444=167%:%
%:%445=167%:%
%:%446=168%:%
%:%447=168%:%
%:%448=168%:%
%:%449=169%:%
%:%450=170%:%
%:%451=170%:%
%:%452=170%:%
%:%453=170%:%
%:%454=170%:%
%:%455=171%:%
%:%456=171%:%
%:%457=171%:%
%:%458=172%:%
%:%459=172%:%
%:%460=173%:%
%:%461=173%:%
%:%462=174%:%
%:%463=174%:%
%:%464=175%:%
%:%465=175%:%
%:%466=176%:%
%:%467=176%:%
%:%468=176%:%
%:%469=176%:%
%:%470=177%:%
%:%471=177%:%
%:%472=177%:%
%:%473=178%:%
%:%474=178%:%
%:%475=179%:%
%:%476=179%:%
%:%477=180%:%
%:%478=181%:%
%:%479=181%:%
%:%480=181%:%
%:%481=181%:%
%:%482=182%:%
%:%483=183%:%
%:%484=183%:%
%:%485=183%:%
%:%486=183%:%
%:%487=184%:%
%:%488=184%:%
%:%489=184%:%
%:%490=185%:%
%:%491=185%:%
%:%492=186%:%
%:%493=186%:%
%:%494=187%:%
%:%495=187%:%
%:%496=188%:%
%:%497=188%:%
%:%498=189%:%
%:%499=189%:%
%:%500=189%:%
%:%501=190%:%
%:%502=190%:%
%:%503=191%:%
%:%504=191%:%
%:%505=192%:%
%:%506=192%:%
%:%507=193%:%
%:%508=193%:%
%:%509=194%:%
%:%510=194%:%
%:%511=194%:%
%:%512=194%:%
%:%513=194%:%
%:%514=195%:%
%:%515=195%:%
%:%516=195%:%
%:%517=196%:%
%:%518=196%:%
%:%519=197%:%
%:%520=197%:%
%:%521=198%:%
%:%522=198%:%
%:%523=199%:%
%:%524=199%:%
%:%525=200%:%
%:%526=200%:%
%:%527=200%:%
%:%528=200%:%
%:%529=201%:%
%:%530=201%:%
%:%531=201%:%
%:%532=202%:%
%:%533=202%:%
%:%534=203%:%
%:%535=203%:%
%:%536=204%:%
%:%537=204%:%
%:%538=205%:%
%:%539=205%:%
%:%540=206%:%
%:%541=206%:%
%:%542=207%:%
%:%543=207%:%
%:%544=207%:%
%:%545=208%:%
%:%546=208%:%
%:%547=209%:%
%:%548=209%:%
%:%549=210%:%
%:%550=210%:%
%:%551=211%:%
%:%552=211%:%
%:%553=212%:%
%:%554=212%:%
%:%555=212%:%
%:%556=212%:%
%:%557=213%:%
%:%558=214%:%
%:%559=214%:%
%:%560=214%:%
%:%561=215%:%
%:%562=215%:%
%:%563=216%:%
%:%564=216%:%
%:%565=217%:%
%:%566=217%:%
%:%567=218%:%
%:%568=218%:%
%:%569=219%:%
%:%570=219%:%
%:%571=219%:%
%:%572=219%:%
%:%573=219%:%
%:%574=220%:%
%:%575=220%:%
%:%576=220%:%
%:%577=220%:%
%:%578=220%:%
%:%579=221%:%
%:%580=221%:%
%:%581=222%:%
%:%582=223%:%
%:%583=223%:%
%:%584=224%:%
%:%585=225%:%
%:%586=225%:%
%:%587=226%:%
%:%588=226%:%
%:%589=227%:%
%:%590=227%:%
%:%591=228%:%
%:%592=228%:%
%:%593=229%:%
%:%594=229%:%
%:%595=230%:%
%:%596=230%:%
%:%597=231%:%
%:%598=231%:%
%:%599=232%:%
%:%600=232%:%
%:%601=233%:%
%:%602=234%:%
%:%603=234%:%
%:%604=235%:%
%:%605=235%:%
%:%606=236%:%
%:%607=236%:%
%:%608=237%:%
%:%609=237%:%
%:%610=238%:%
%:%611=238%:%
%:%612=239%:%
%:%613=239%:%
%:%614=240%:%
%:%615=240%:%
%:%616=240%:%
%:%617=241%:%
%:%618=241%:%
%:%619=242%:%
%:%620=242%:%
%:%621=243%:%
%:%622=243%:%
%:%623=243%:%
%:%624=244%:%
%:%625=244%:%
%:%626=244%:%
%:%627=245%:%
%:%633=245%:%
%:%636=246%:%
%:%637=247%:%
%:%638=247%:%
%:%641=248%:%
%:%645=248%:%
%:%646=248%:%
%:%647=249%:%
%:%648=249%:%
%:%649=250%:%
%:%650=250%:%
%:%651=251%:%
%:%652=251%:%
%:%653=252%:%
%:%654=252%:%
%:%655=253%:%
%:%656=253%:%
%:%657=254%:%
%:%658=254%:%
%:%659=255%:%
%:%660=255%:%
%:%661=256%:%
%:%662=256%:%
%:%663=257%:%
%:%664=257%:%
%:%665=258%:%
%:%666=258%:%
%:%667=259%:%
%:%668=259%:%
%:%669=260%:%
%:%670=260%:%
%:%671=261%:%
%:%672=261%:%
%:%673=262%:%
%:%674=262%:%
%:%675=263%:%
%:%676=263%:%
%:%677=264%:%
%:%678=264%:%
%:%679=265%:%
%:%680=265%:%
%:%681=266%:%
%:%682=266%:%
%:%683=267%:%
%:%684=267%:%
%:%685=268%:%
%:%686=268%:%
%:%691=268%:%
%:%694=269%:%
%:%695=270%:%
%:%696=270%:%
%:%697=271%:%
%:%698=272%:%
%:%699=273%:%
%:%706=274%:%
%:%707=274%:%
%:%708=275%:%
%:%709=275%:%
%:%710=275%:%
%:%711=276%:%
%:%712=276%:%
%:%713=276%:%
%:%714=276%:%
%:%715=277%:%
%:%716=277%:%
%:%717=277%:%
%:%718=277%:%
%:%719=278%:%
%:%720=279%:%
%:%721=279%:%
%:%722=280%:%
%:%723=280%:%
%:%724=281%:%
%:%725=281%:%
%:%726=282%:%
%:%727=282%:%
%:%728=283%:%
%:%729=283%:%
%:%730=284%:%
%:%731=284%:%
%:%732=284%:%
%:%733=284%:%
%:%734=284%:%
%:%735=285%:%
%:%741=285%:%
%:%744=286%:%
%:%745=287%:%
%:%746=287%:%
%:%749=288%:%
%:%753=288%:%
%:%754=288%:%
%:%755=289%:%
%:%756=289%:%
%:%757=290%:%
%:%758=290%:%
%:%759=291%:%
%:%760=291%:%
%:%761=292%:%
%:%762=292%:%
%:%763=293%:%
%:%764=293%:%
%:%765=294%:%
%:%766=294%:%
%:%767=295%:%
%:%768=295%:%
%:%769=296%:%
%:%770=296%:%
%:%771=297%:%
%:%772=297%:%
%:%773=298%:%
%:%774=298%:%
%:%775=299%:%
%:%776=299%:%
%:%777=300%:%
%:%778=300%:%
%:%779=301%:%
%:%780=301%:%
%:%781=302%:%
%:%782=302%:%
%:%783=303%:%
%:%784=303%:%
%:%785=304%:%
%:%786=304%:%
%:%787=305%:%
%:%788=305%:%
%:%789=306%:%
%:%790=306%:%
%:%791=307%:%
%:%792=307%:%
%:%793=308%:%
%:%794=308%:%
%:%799=308%:%
%:%802=309%:%
%:%803=310%:%
%:%804=310%:%
%:%807=311%:%
%:%811=311%:%
%:%812=311%:%
%:%813=312%:%
%:%814=312%:%
%:%815=313%:%
%:%816=313%:%
%:%817=314%:%
%:%818=314%:%
%:%819=315%:%
%:%820=315%:%
%:%821=316%:%
%:%822=316%:%
%:%823=317%:%
%:%824=317%:%
%:%825=318%:%
%:%826=318%:%
%:%827=319%:%
%:%828=319%:%
%:%829=320%:%
%:%830=320%:%
%:%831=321%:%
%:%832=321%:%
%:%833=322%:%
%:%834=322%:%
%:%835=323%:%
%:%836=323%:%
%:%837=324%:%
%:%838=324%:%
%:%839=325%:%
%:%840=325%:%
%:%841=326%:%
%:%842=326%:%
%:%843=327%:%
%:%844=327%:%
%:%845=328%:%
%:%846=328%:%
%:%847=329%:%
%:%848=329%:%
%:%849=330%:%
%:%850=330%:%
%:%851=331%:%
%:%852=331%:%
%:%853=332%:%
%:%854=332%:%
%:%855=333%:%
%:%856=333%:%
%:%857=334%:%
%:%858=334%:%
%:%859=335%:%
%:%860=335%:%
%:%861=336%:%
%:%862=336%:%
%:%863=337%:%
%:%864=337%:%
%:%865=338%:%
%:%866=338%:%
%:%867=339%:%
%:%868=339%:%
%:%869=340%:%
%:%870=340%:%
%:%871=341%:%
%:%872=341%:%
%:%873=342%:%
%:%874=342%:%
%:%875=343%:%
%:%876=343%:%
%:%877=344%:%
%:%878=344%:%
%:%879=345%:%
%:%880=345%:%
%:%881=346%:%
%:%882=346%:%
%:%883=347%:%
%:%884=347%:%
%:%885=348%:%
%:%886=348%:%
%:%887=349%:%
%:%888=349%:%
%:%889=350%:%
%:%890=350%:%
%:%891=351%:%
%:%892=351%:%
%:%893=352%:%
%:%894=352%:%
%:%895=353%:%
%:%896=353%:%
%:%897=354%:%
%:%898=354%:%
%:%899=355%:%
%:%900=355%:%
%:%901=356%:%
%:%902=356%:%
%:%903=357%:%
%:%904=357%:%
%:%905=358%:%
%:%906=358%:%
%:%907=359%:%
%:%908=359%:%
%:%909=360%:%
%:%910=360%:%
%:%911=361%:%
%:%912=361%:%
%:%913=362%:%
%:%914=362%:%
%:%915=363%:%
%:%916=363%:%
%:%917=364%:%
%:%918=364%:%
%:%919=365%:%
%:%920=365%:%
%:%925=365%:%
%:%928=366%:%
%:%929=367%:%
%:%930=367%:%
%:%933=368%:%
%:%937=368%:%
%:%938=368%:%
%:%939=369%:%
%:%940=369%:%
%:%941=370%:%
%:%942=370%:%
%:%947=370%:%
%:%950=371%:%
%:%951=372%:%
%:%952=372%:%
%:%955=373%:%
%:%959=373%:%
%:%960=373%:%
%:%961=373%:%
%:%966=373%:%
%:%969=374%:%
%:%970=375%:%
%:%971=375%:%
%:%978=376%:%

%
\begin{isabellebody}%
\setisabellecontext{Fn{\isacharunderscore}{\kern0pt}Perm{\isacharunderscore}{\kern0pt}Definition}%
%
\isadelimtheory
%
\endisadelimtheory
%
\isatagtheory
\isacommand{theory}\isamarkupfalse%
\ Fn{\isacharunderscore}{\kern0pt}Perm{\isacharunderscore}{\kern0pt}Definition\isanewline
\ \ \isakeyword{imports}\ SymExt{\isacharunderscore}{\kern0pt}ZF\ \isanewline
\isakeyword{begin}%
\endisatagtheory
{\isafoldtheory}%
%
\isadelimtheory
\ \isanewline
%
\endisadelimtheory
\isanewline
\isacommand{context}\isamarkupfalse%
\ M{\isacharunderscore}{\kern0pt}ctm\ \isakeyword{begin}\ \isanewline
\isanewline
\isacommand{definition}\isamarkupfalse%
\ finite{\isacharunderscore}{\kern0pt}M\ \isakeyword{where}\ {\isachardoublequoteopen}finite{\isacharunderscore}{\kern0pt}M{\isacharparenleft}{\kern0pt}x{\isacharparenright}{\kern0pt}\ {\isasymequiv}\ {\isasymexists}n\ {\isasymin}\ nat{\isachardot}{\kern0pt}\ inj{\isacharparenleft}{\kern0pt}x{\isacharcomma}{\kern0pt}\ n{\isacharparenright}{\kern0pt}\ {\isasyminter}\ M\ {\isasymnoteq}\ {\isadigit{0}}{\isachardoublequoteclose}\ \ \isanewline
\isacommand{definition}\isamarkupfalse%
\ finite{\isacharunderscore}{\kern0pt}M{\isacharunderscore}{\kern0pt}fm\ \isakeyword{where}\ {\isachardoublequoteopen}finite{\isacharunderscore}{\kern0pt}M{\isacharunderscore}{\kern0pt}fm{\isacharparenleft}{\kern0pt}N{\isacharcomma}{\kern0pt}\ x{\isacharparenright}{\kern0pt}\ {\isasymequiv}\ Exists{\isacharparenleft}{\kern0pt}Exists{\isacharparenleft}{\kern0pt}And{\isacharparenleft}{\kern0pt}Member{\isacharparenleft}{\kern0pt}{\isadigit{0}}{\isacharcomma}{\kern0pt}\ N{\isacharhash}{\kern0pt}{\isacharplus}{\kern0pt}{\isadigit{2}}{\isacharparenright}{\kern0pt}{\isacharcomma}{\kern0pt}\ injection{\isacharunderscore}{\kern0pt}fm{\isacharparenleft}{\kern0pt}x{\isacharhash}{\kern0pt}{\isacharplus}{\kern0pt}{\isadigit{2}}{\isacharcomma}{\kern0pt}\ {\isadigit{0}}{\isacharcomma}{\kern0pt}\ {\isadigit{1}}{\isacharparenright}{\kern0pt}{\isacharparenright}{\kern0pt}{\isacharparenright}{\kern0pt}{\isacharparenright}{\kern0pt}{\isachardoublequoteclose}\isanewline
\isanewline
\isacommand{lemma}\isamarkupfalse%
\ finite{\isacharunderscore}{\kern0pt}M{\isacharunderscore}{\kern0pt}fm{\isacharunderscore}{\kern0pt}type\ {\isacharcolon}{\kern0pt}\ \isanewline
\ \ \isakeyword{fixes}\ i\ j\ \isanewline
\ \ \isakeyword{assumes}\ {\isachardoublequoteopen}i\ {\isasymin}\ nat{\isachardoublequoteclose}\ {\isachardoublequoteopen}j\ {\isasymin}\ nat{\isachardoublequoteclose}\ \isanewline
\ \ \isakeyword{shows}\ {\isachardoublequoteopen}finite{\isacharunderscore}{\kern0pt}M{\isacharunderscore}{\kern0pt}fm{\isacharparenleft}{\kern0pt}i{\isacharcomma}{\kern0pt}\ j{\isacharparenright}{\kern0pt}\ {\isasymin}\ formula{\isachardoublequoteclose}\ \isanewline
%
\isadelimproof
\ \ %
\endisadelimproof
%
\isatagproof
\isacommand{unfolding}\isamarkupfalse%
\ finite{\isacharunderscore}{\kern0pt}M{\isacharunderscore}{\kern0pt}fm{\isacharunderscore}{\kern0pt}def\ \isanewline
\ \ \isacommand{using}\isamarkupfalse%
\ assms\isanewline
\ \ \isacommand{by}\isamarkupfalse%
\ auto%
\endisatagproof
{\isafoldproof}%
%
\isadelimproof
\isanewline
%
\endisadelimproof
\isanewline
\isacommand{lemma}\isamarkupfalse%
\ arity{\isacharunderscore}{\kern0pt}finite{\isacharunderscore}{\kern0pt}M{\isacharunderscore}{\kern0pt}fm\ {\isacharcolon}{\kern0pt}\ \isanewline
\ \ \isakeyword{fixes}\ i\ j\ \isanewline
\ \ \isakeyword{assumes}\ {\isachardoublequoteopen}i\ {\isasymin}\ nat{\isachardoublequoteclose}\ {\isachardoublequoteopen}j\ {\isasymin}\ nat{\isachardoublequoteclose}\ \isanewline
\ \ \isakeyword{shows}\ {\isachardoublequoteopen}arity{\isacharparenleft}{\kern0pt}finite{\isacharunderscore}{\kern0pt}M{\isacharunderscore}{\kern0pt}fm{\isacharparenleft}{\kern0pt}i{\isacharcomma}{\kern0pt}\ j{\isacharparenright}{\kern0pt}{\isacharparenright}{\kern0pt}\ {\isasymle}\ succ{\isacharparenleft}{\kern0pt}i{\isacharparenright}{\kern0pt}\ {\isasymunion}\ succ{\isacharparenleft}{\kern0pt}j{\isacharparenright}{\kern0pt}{\isachardoublequoteclose}\ \isanewline
%
\isadelimproof
\ \ %
\endisadelimproof
%
\isatagproof
\isacommand{unfolding}\isamarkupfalse%
\ finite{\isacharunderscore}{\kern0pt}M{\isacharunderscore}{\kern0pt}fm{\isacharunderscore}{\kern0pt}def\ bijection{\isacharunderscore}{\kern0pt}fm{\isacharunderscore}{\kern0pt}def\ injection{\isacharunderscore}{\kern0pt}fm{\isacharunderscore}{\kern0pt}def\ surjection{\isacharunderscore}{\kern0pt}fm{\isacharunderscore}{\kern0pt}def\ typed{\isacharunderscore}{\kern0pt}function{\isacharunderscore}{\kern0pt}fm{\isacharunderscore}{\kern0pt}def\isanewline
\ \ \isacommand{using}\isamarkupfalse%
\ assms\isanewline
\ \ \isacommand{apply}\isamarkupfalse%
\ {\isacharparenleft}{\kern0pt}simp\ del{\isacharcolon}{\kern0pt}FOL{\isacharunderscore}{\kern0pt}sats{\isacharunderscore}{\kern0pt}iff\ pair{\isacharunderscore}{\kern0pt}abs\ add{\isacharcolon}{\kern0pt}\ fm{\isacharunderscore}{\kern0pt}defs\ nat{\isacharunderscore}{\kern0pt}simp{\isacharunderscore}{\kern0pt}union{\isacharparenright}{\kern0pt}\isanewline
\ \ \isacommand{done}\isamarkupfalse%
%
\endisatagproof
{\isafoldproof}%
%
\isadelimproof
\isanewline
%
\endisadelimproof
\isanewline
\isacommand{lemma}\isamarkupfalse%
\ sats{\isacharunderscore}{\kern0pt}finite{\isacharunderscore}{\kern0pt}M{\isacharunderscore}{\kern0pt}fm{\isacharunderscore}{\kern0pt}iff\ {\isacharcolon}{\kern0pt}\ \isanewline
\ \ \isakeyword{fixes}\ env\ i\ j\ x\isanewline
\ \ \isakeyword{assumes}\ {\isachardoublequoteopen}env\ {\isasymin}\ list{\isacharparenleft}{\kern0pt}M{\isacharparenright}{\kern0pt}{\isachardoublequoteclose}\ {\isachardoublequoteopen}i\ {\isacharless}{\kern0pt}\ length{\isacharparenleft}{\kern0pt}env{\isacharparenright}{\kern0pt}{\isachardoublequoteclose}\ {\isachardoublequoteopen}j\ {\isacharless}{\kern0pt}\ length{\isacharparenleft}{\kern0pt}env{\isacharparenright}{\kern0pt}{\isachardoublequoteclose}\ {\isachardoublequoteopen}nth{\isacharparenleft}{\kern0pt}i{\isacharcomma}{\kern0pt}\ env{\isacharparenright}{\kern0pt}\ {\isacharequal}{\kern0pt}\ nat{\isachardoublequoteclose}\ {\isachardoublequoteopen}nth{\isacharparenleft}{\kern0pt}j{\isacharcomma}{\kern0pt}\ env{\isacharparenright}{\kern0pt}\ {\isacharequal}{\kern0pt}\ x{\isachardoublequoteclose}\isanewline
\ \ \isakeyword{shows}\ {\isachardoublequoteopen}sats{\isacharparenleft}{\kern0pt}M{\isacharcomma}{\kern0pt}\ finite{\isacharunderscore}{\kern0pt}M{\isacharunderscore}{\kern0pt}fm{\isacharparenleft}{\kern0pt}i{\isacharcomma}{\kern0pt}\ j{\isacharparenright}{\kern0pt}{\isacharcomma}{\kern0pt}\ env{\isacharparenright}{\kern0pt}\ {\isasymlongleftrightarrow}\ finite{\isacharunderscore}{\kern0pt}M{\isacharparenleft}{\kern0pt}x{\isacharparenright}{\kern0pt}{\isachardoublequoteclose}\ \isanewline
%
\isadelimproof
\isanewline
\ \ %
\endisadelimproof
%
\isatagproof
\isacommand{unfolding}\isamarkupfalse%
\ finite{\isacharunderscore}{\kern0pt}M{\isacharunderscore}{\kern0pt}fm{\isacharunderscore}{\kern0pt}def\ finite{\isacharunderscore}{\kern0pt}M{\isacharunderscore}{\kern0pt}def\isanewline
\ \ \isacommand{using}\isamarkupfalse%
\ assms\ nth{\isacharunderscore}{\kern0pt}type\ lt{\isacharunderscore}{\kern0pt}nat{\isacharunderscore}{\kern0pt}in{\isacharunderscore}{\kern0pt}nat\ nat{\isacharunderscore}{\kern0pt}in{\isacharunderscore}{\kern0pt}M\isanewline
\ \ \isacommand{apply}\isamarkupfalse%
\ simp\isanewline
\ \ \isacommand{apply}\isamarkupfalse%
{\isacharparenleft}{\kern0pt}rule\ iffI{\isacharcomma}{\kern0pt}\ clarsimp{\isacharparenright}{\kern0pt}\isanewline
\ \ \ \isacommand{apply}\isamarkupfalse%
{\isacharparenleft}{\kern0pt}rename{\isacharunderscore}{\kern0pt}tac\ f\ n{\isacharcomma}{\kern0pt}\ rule{\isacharunderscore}{\kern0pt}tac\ x{\isacharequal}{\kern0pt}n\ \isakeyword{in}\ bexI{\isacharcomma}{\kern0pt}\ force{\isacharcomma}{\kern0pt}\ force{\isacharparenright}{\kern0pt}\isanewline
\ \ \isacommand{apply}\isamarkupfalse%
\ clarsimp\isanewline
\ \ \isacommand{apply}\isamarkupfalse%
{\isacharparenleft}{\kern0pt}rule\ not{\isacharunderscore}{\kern0pt}emptyE{\isacharcomma}{\kern0pt}\ assumption{\isacharparenright}{\kern0pt}\isanewline
\ \ \isacommand{apply}\isamarkupfalse%
{\isacharparenleft}{\kern0pt}rename{\isacharunderscore}{\kern0pt}tac\ n\ f{\isacharcomma}{\kern0pt}\ rule{\isacharunderscore}{\kern0pt}tac\ x{\isacharequal}{\kern0pt}f\ \isakeyword{in}\ bexI{\isacharparenright}{\kern0pt}\isanewline
\ \ \ \isacommand{apply}\isamarkupfalse%
{\isacharparenleft}{\kern0pt}rename{\isacharunderscore}{\kern0pt}tac\ n\ f{\isacharcomma}{\kern0pt}\ rule{\isacharunderscore}{\kern0pt}tac\ x{\isacharequal}{\kern0pt}n\ \isakeyword{in}\ bexI{\isacharcomma}{\kern0pt}\ force{\isacharparenright}{\kern0pt}\isanewline
\ \ \isacommand{using}\isamarkupfalse%
\ nat{\isacharunderscore}{\kern0pt}in{\isacharunderscore}{\kern0pt}M\ transM\ \isanewline
\ \ \isacommand{by}\isamarkupfalse%
\ auto%
\endisatagproof
{\isafoldproof}%
%
\isadelimproof
\isanewline
%
\endisadelimproof
\isanewline
\isacommand{definition}\isamarkupfalse%
\ is{\isacharunderscore}{\kern0pt}finite{\isacharunderscore}{\kern0pt}function{\isacharunderscore}{\kern0pt}fm\ \isakeyword{where}\ \isanewline
\ \ {\isachardoublequoteopen}is{\isacharunderscore}{\kern0pt}finite{\isacharunderscore}{\kern0pt}function{\isacharunderscore}{\kern0pt}fm{\isacharparenleft}{\kern0pt}d{\isacharcomma}{\kern0pt}\ r{\isacharcomma}{\kern0pt}\ N{\isacharcomma}{\kern0pt}\ f{\isacharparenright}{\kern0pt}\ {\isasymequiv}\ \isanewline
\ \ \ \ And{\isacharparenleft}{\kern0pt}function{\isacharunderscore}{\kern0pt}fm{\isacharparenleft}{\kern0pt}f{\isacharparenright}{\kern0pt}{\isacharcomma}{\kern0pt}\ \isanewline
\ \ \ \ And{\isacharparenleft}{\kern0pt}Exists{\isacharparenleft}{\kern0pt}\isanewline
\ \ \ \ \ \ And{\isacharparenleft}{\kern0pt}domain{\isacharunderscore}{\kern0pt}fm{\isacharparenleft}{\kern0pt}f{\isacharhash}{\kern0pt}{\isacharplus}{\kern0pt}{\isadigit{1}}{\isacharcomma}{\kern0pt}\ {\isadigit{0}}{\isacharparenright}{\kern0pt}{\isacharcomma}{\kern0pt}\ \isanewline
\ \ \ \ \ \ And{\isacharparenleft}{\kern0pt}subset{\isacharunderscore}{\kern0pt}fm{\isacharparenleft}{\kern0pt}{\isadigit{0}}{\isacharcomma}{\kern0pt}\ d\ {\isacharhash}{\kern0pt}{\isacharplus}{\kern0pt}\ {\isadigit{1}}{\isacharparenright}{\kern0pt}{\isacharcomma}{\kern0pt}\ \isanewline
\ \ \ \ \ \ \ \ \ \ finite{\isacharunderscore}{\kern0pt}M{\isacharunderscore}{\kern0pt}fm{\isacharparenleft}{\kern0pt}N\ {\isacharhash}{\kern0pt}{\isacharplus}{\kern0pt}\ {\isadigit{1}}{\isacharcomma}{\kern0pt}\ {\isadigit{0}}{\isacharparenright}{\kern0pt}{\isacharparenright}{\kern0pt}{\isacharparenright}{\kern0pt}{\isacharparenright}{\kern0pt}{\isacharcomma}{\kern0pt}\ \isanewline
\ \ \ \ \ \ \ \ Exists{\isacharparenleft}{\kern0pt}And{\isacharparenleft}{\kern0pt}range{\isacharunderscore}{\kern0pt}fm{\isacharparenleft}{\kern0pt}f{\isacharhash}{\kern0pt}{\isacharplus}{\kern0pt}{\isadigit{1}}{\isacharcomma}{\kern0pt}\ {\isadigit{0}}{\isacharparenright}{\kern0pt}{\isacharcomma}{\kern0pt}\ subset{\isacharunderscore}{\kern0pt}fm{\isacharparenleft}{\kern0pt}{\isadigit{0}}{\isacharcomma}{\kern0pt}\ r{\isacharhash}{\kern0pt}{\isacharplus}{\kern0pt}{\isadigit{1}}{\isacharparenright}{\kern0pt}{\isacharparenright}{\kern0pt}{\isacharparenright}{\kern0pt}{\isacharparenright}{\kern0pt}{\isacharparenright}{\kern0pt}{\isachardoublequoteclose}\ \isanewline
\isanewline
\isacommand{lemma}\isamarkupfalse%
\ is{\isacharunderscore}{\kern0pt}finite{\isacharunderscore}{\kern0pt}function{\isacharunderscore}{\kern0pt}fm{\isacharunderscore}{\kern0pt}type\ {\isacharbrackleft}{\kern0pt}simp{\isacharbrackright}{\kern0pt}\ {\isacharcolon}{\kern0pt}\ \isanewline
\ \ \isakeyword{fixes}\ i\ j\ k\ l\ \isanewline
\ \ \isakeyword{assumes}\ {\isachardoublequoteopen}i\ {\isasymin}\ nat{\isachardoublequoteclose}\ {\isachardoublequoteopen}j\ {\isasymin}\ nat{\isachardoublequoteclose}\ {\isachardoublequoteopen}k\ {\isasymin}\ nat{\isachardoublequoteclose}\ {\isachardoublequoteopen}l\ {\isasymin}\ nat{\isachardoublequoteclose}\ \isanewline
\ \ \isakeyword{shows}\ {\isachardoublequoteopen}is{\isacharunderscore}{\kern0pt}finite{\isacharunderscore}{\kern0pt}function{\isacharunderscore}{\kern0pt}fm{\isacharparenleft}{\kern0pt}i{\isacharcomma}{\kern0pt}\ j{\isacharcomma}{\kern0pt}\ k{\isacharcomma}{\kern0pt}\ l{\isacharparenright}{\kern0pt}\ {\isasymin}\ formula{\isachardoublequoteclose}\isanewline
%
\isadelimproof
\ \ %
\endisadelimproof
%
\isatagproof
\isacommand{unfolding}\isamarkupfalse%
\ is{\isacharunderscore}{\kern0pt}finite{\isacharunderscore}{\kern0pt}function{\isacharunderscore}{\kern0pt}fm{\isacharunderscore}{\kern0pt}def\ \isanewline
\ \ \isacommand{apply}\isamarkupfalse%
{\isacharparenleft}{\kern0pt}subgoal{\isacharunderscore}{\kern0pt}tac\ {\isachardoublequoteopen}finite{\isacharunderscore}{\kern0pt}M{\isacharunderscore}{\kern0pt}fm{\isacharparenleft}{\kern0pt}k\ {\isacharhash}{\kern0pt}{\isacharplus}{\kern0pt}\ {\isadigit{1}}{\isacharcomma}{\kern0pt}\ {\isadigit{0}}{\isacharparenright}{\kern0pt}\ {\isasymin}\ formula{\isachardoublequoteclose}{\isacharparenright}{\kern0pt}\isanewline
\ \ \isacommand{using}\isamarkupfalse%
\ assms\ \isanewline
\ \ \ \isacommand{apply}\isamarkupfalse%
\ force\isanewline
\ \ \isacommand{using}\isamarkupfalse%
\ finite{\isacharunderscore}{\kern0pt}M{\isacharunderscore}{\kern0pt}fm{\isacharunderscore}{\kern0pt}type\isanewline
\ \ \isacommand{by}\isamarkupfalse%
\ auto%
\endisatagproof
{\isafoldproof}%
%
\isadelimproof
\isanewline
%
\endisadelimproof
\isanewline
\isacommand{lemma}\isamarkupfalse%
\ arity{\isacharunderscore}{\kern0pt}is{\isacharunderscore}{\kern0pt}finite{\isacharunderscore}{\kern0pt}function{\isacharunderscore}{\kern0pt}fm\ {\isacharcolon}{\kern0pt}\ \isanewline
\ \ \isakeyword{fixes}\ i\ j\ k\ l\ \isanewline
\ \ \isakeyword{assumes}\ {\isachardoublequoteopen}i\ {\isasymin}\ nat{\isachardoublequoteclose}\ {\isachardoublequoteopen}j\ {\isasymin}\ nat{\isachardoublequoteclose}\ {\isachardoublequoteopen}k\ {\isasymin}\ nat{\isachardoublequoteclose}\ {\isachardoublequoteopen}l\ {\isasymin}\ nat{\isachardoublequoteclose}\ \isanewline
\ \ \isakeyword{shows}\ {\isachardoublequoteopen}arity{\isacharparenleft}{\kern0pt}is{\isacharunderscore}{\kern0pt}finite{\isacharunderscore}{\kern0pt}function{\isacharunderscore}{\kern0pt}fm{\isacharparenleft}{\kern0pt}i{\isacharcomma}{\kern0pt}\ j{\isacharcomma}{\kern0pt}\ k{\isacharcomma}{\kern0pt}\ l{\isacharparenright}{\kern0pt}{\isacharparenright}{\kern0pt}\ {\isasymle}\ succ{\isacharparenleft}{\kern0pt}i{\isacharparenright}{\kern0pt}\ {\isasymunion}\ succ{\isacharparenleft}{\kern0pt}j{\isacharparenright}{\kern0pt}\ {\isasymunion}\ succ{\isacharparenleft}{\kern0pt}k{\isacharparenright}{\kern0pt}\ {\isasymunion}\ succ{\isacharparenleft}{\kern0pt}l{\isacharparenright}{\kern0pt}{\isachardoublequoteclose}\ \isanewline
%
\isadelimproof
\ \ %
\endisadelimproof
%
\isatagproof
\isacommand{unfolding}\isamarkupfalse%
\ is{\isacharunderscore}{\kern0pt}finite{\isacharunderscore}{\kern0pt}function{\isacharunderscore}{\kern0pt}fm{\isacharunderscore}{\kern0pt}def\ \isanewline
\ \ \isacommand{unfolding}\isamarkupfalse%
\ bijection{\isacharunderscore}{\kern0pt}fm{\isacharunderscore}{\kern0pt}def\ injection{\isacharunderscore}{\kern0pt}fm{\isacharunderscore}{\kern0pt}def\ surjection{\isacharunderscore}{\kern0pt}fm{\isacharunderscore}{\kern0pt}def\ typed{\isacharunderscore}{\kern0pt}function{\isacharunderscore}{\kern0pt}fm{\isacharunderscore}{\kern0pt}def\isanewline
\ \ \isacommand{apply}\isamarkupfalse%
{\isacharparenleft}{\kern0pt}subgoal{\isacharunderscore}{\kern0pt}tac\ {\isachardoublequoteopen}finite{\isacharunderscore}{\kern0pt}M{\isacharunderscore}{\kern0pt}fm{\isacharparenleft}{\kern0pt}k\ {\isacharhash}{\kern0pt}{\isacharplus}{\kern0pt}\ {\isadigit{1}}{\isacharcomma}{\kern0pt}\ {\isadigit{0}}{\isacharparenright}{\kern0pt}\ {\isasymin}\ formula{\isachardoublequoteclose}{\isacharparenright}{\kern0pt}\isanewline
\ \ \isacommand{using}\isamarkupfalse%
\ assms\isanewline
\ \ \ \isacommand{apply}\isamarkupfalse%
\ simp\isanewline
\ \ \ \isacommand{apply}\isamarkupfalse%
{\isacharparenleft}{\kern0pt}rule\ Un{\isacharunderscore}{\kern0pt}least{\isacharunderscore}{\kern0pt}lt{\isacharparenright}{\kern0pt}\isanewline
\ \ \ \ \isacommand{apply}\isamarkupfalse%
{\isacharparenleft}{\kern0pt}subst\ arity{\isacharunderscore}{\kern0pt}function{\isacharunderscore}{\kern0pt}fm{\isacharcomma}{\kern0pt}\ simp{\isacharparenright}{\kern0pt}\isanewline
\ \ \ \ \isacommand{apply}\isamarkupfalse%
{\isacharparenleft}{\kern0pt}subst\ succ{\isacharunderscore}{\kern0pt}Un{\isacharunderscore}{\kern0pt}distrib{\isacharcomma}{\kern0pt}\ simp{\isacharcomma}{\kern0pt}\ simp{\isacharparenright}{\kern0pt}{\isacharplus}{\kern0pt}\isanewline
\ \ \ \ \isacommand{apply}\isamarkupfalse%
{\isacharparenleft}{\kern0pt}rule\ ltI{\isacharcomma}{\kern0pt}\ simp{\isacharcomma}{\kern0pt}\ simp{\isacharparenright}{\kern0pt}\isanewline
\ \ \ \isacommand{apply}\isamarkupfalse%
{\isacharparenleft}{\kern0pt}rule\ Un{\isacharunderscore}{\kern0pt}least{\isacharunderscore}{\kern0pt}lt{\isacharparenright}{\kern0pt}\isanewline
\ \ \ \ \isacommand{apply}\isamarkupfalse%
{\isacharparenleft}{\kern0pt}rule\ pred{\isacharunderscore}{\kern0pt}le{\isacharcomma}{\kern0pt}\ simp{\isacharcomma}{\kern0pt}\ simp{\isacharparenright}{\kern0pt}{\isacharplus}{\kern0pt}\isanewline
\ \ \ \ \isacommand{apply}\isamarkupfalse%
{\isacharparenleft}{\kern0pt}rule\ Un{\isacharunderscore}{\kern0pt}least{\isacharunderscore}{\kern0pt}lt{\isacharparenright}{\kern0pt}\isanewline
\ \ \ \ \ \isacommand{apply}\isamarkupfalse%
{\isacharparenleft}{\kern0pt}subst\ arity{\isacharunderscore}{\kern0pt}domain{\isacharunderscore}{\kern0pt}fm{\isacharcomma}{\kern0pt}\ simp{\isacharcomma}{\kern0pt}\ simp{\isacharparenright}{\kern0pt}\isanewline
\ \ \ \ \ \isacommand{apply}\isamarkupfalse%
{\isacharparenleft}{\kern0pt}rule\ Un{\isacharunderscore}{\kern0pt}least{\isacharunderscore}{\kern0pt}lt{\isacharparenright}{\kern0pt}\isanewline
\ \ \ \ \ \ \isacommand{apply}\isamarkupfalse%
\ simp\isanewline
\ \ \ \ \ \ \isacommand{apply}\isamarkupfalse%
{\isacharparenleft}{\kern0pt}rule\ ltI{\isacharcomma}{\kern0pt}\ simp{\isacharcomma}{\kern0pt}\ simp{\isacharcomma}{\kern0pt}\ simp{\isacharparenright}{\kern0pt}\isanewline
\ \ \ \ \isacommand{apply}\isamarkupfalse%
{\isacharparenleft}{\kern0pt}rule\ Un{\isacharunderscore}{\kern0pt}least{\isacharunderscore}{\kern0pt}lt{\isacharparenright}{\kern0pt}{\isacharplus}{\kern0pt}\isanewline
\ \ \ \ \ \ \isacommand{apply}\isamarkupfalse%
\ {\isacharparenleft}{\kern0pt}simp{\isacharcomma}{\kern0pt}\ simp{\isacharparenright}{\kern0pt}\isanewline
\ \ \ \ \ \isacommand{apply}\isamarkupfalse%
{\isacharparenleft}{\kern0pt}rule\ ltI{\isacharcomma}{\kern0pt}\ simp{\isacharcomma}{\kern0pt}\ simp{\isacharparenright}{\kern0pt}\isanewline
\ \ \ \ \isacommand{apply}\isamarkupfalse%
{\isacharparenleft}{\kern0pt}rule\ le{\isacharunderscore}{\kern0pt}trans{\isacharcomma}{\kern0pt}\ rule\ arity{\isacharunderscore}{\kern0pt}finite{\isacharunderscore}{\kern0pt}M{\isacharunderscore}{\kern0pt}fm{\isacharcomma}{\kern0pt}\ simp{\isacharcomma}{\kern0pt}\ simp{\isacharparenright}{\kern0pt}\isanewline
\ \ \ \ \isacommand{apply}\isamarkupfalse%
{\isacharparenleft}{\kern0pt}rule\ Un{\isacharunderscore}{\kern0pt}least{\isacharunderscore}{\kern0pt}lt{\isacharparenright}{\kern0pt}{\isacharplus}{\kern0pt}\isanewline
\ \ \ \ \ \isacommand{apply}\isamarkupfalse%
\ simp\isanewline
\ \ \ \ \ \isacommand{apply}\isamarkupfalse%
{\isacharparenleft}{\kern0pt}rule\ ltI{\isacharcomma}{\kern0pt}\ simp{\isacharcomma}{\kern0pt}\ simp{\isacharcomma}{\kern0pt}\ simp{\isacharparenright}{\kern0pt}\isanewline
\ \ \ \isacommand{apply}\isamarkupfalse%
{\isacharparenleft}{\kern0pt}rule\ pred{\isacharunderscore}{\kern0pt}le{\isacharcomma}{\kern0pt}\ simp{\isacharcomma}{\kern0pt}\ simp{\isacharparenright}{\kern0pt}{\isacharplus}{\kern0pt}\isanewline
\ \ \ \isacommand{apply}\isamarkupfalse%
{\isacharparenleft}{\kern0pt}rule\ Un{\isacharunderscore}{\kern0pt}least{\isacharunderscore}{\kern0pt}lt{\isacharparenright}{\kern0pt}{\isacharplus}{\kern0pt}\isanewline
\ \ \ \ \isacommand{apply}\isamarkupfalse%
{\isacharparenleft}{\kern0pt}subst\ arity{\isacharunderscore}{\kern0pt}range{\isacharunderscore}{\kern0pt}fm{\isacharcomma}{\kern0pt}\ simp{\isacharcomma}{\kern0pt}\ simp{\isacharparenright}{\kern0pt}\isanewline
\ \ \ \ \isacommand{apply}\isamarkupfalse%
{\isacharparenleft}{\kern0pt}rule\ Un{\isacharunderscore}{\kern0pt}least{\isacharunderscore}{\kern0pt}lt{\isacharparenright}{\kern0pt}{\isacharplus}{\kern0pt}\isanewline
\ \ \ \ \ \isacommand{apply}\isamarkupfalse%
\ {\isacharparenleft}{\kern0pt}simp{\isacharcomma}{\kern0pt}\ rule\ ltI{\isacharcomma}{\kern0pt}\ simp{\isacharcomma}{\kern0pt}\ simp{\isacharcomma}{\kern0pt}\ simp{\isacharparenright}{\kern0pt}\isanewline
\ \ \ \isacommand{apply}\isamarkupfalse%
{\isacharparenleft}{\kern0pt}rule\ Un{\isacharunderscore}{\kern0pt}least{\isacharunderscore}{\kern0pt}lt{\isacharparenright}{\kern0pt}{\isacharplus}{\kern0pt}\isanewline
\ \ \ \ \isacommand{apply}\isamarkupfalse%
{\isacharparenleft}{\kern0pt}simp{\isacharcomma}{\kern0pt}\ simp{\isacharcomma}{\kern0pt}\ rule\ ltI{\isacharcomma}{\kern0pt}\ simp{\isacharcomma}{\kern0pt}\ simp{\isacharparenright}{\kern0pt}\isanewline
\ \ \isacommand{using}\isamarkupfalse%
\ finite{\isacharunderscore}{\kern0pt}M{\isacharunderscore}{\kern0pt}fm{\isacharunderscore}{\kern0pt}type\ assms\isanewline
\ \ \isacommand{by}\isamarkupfalse%
\ auto%
\endisatagproof
{\isafoldproof}%
%
\isadelimproof
\isanewline
%
\endisadelimproof
\isanewline
\isacommand{lemma}\isamarkupfalse%
\ sats{\isacharunderscore}{\kern0pt}is{\isacharunderscore}{\kern0pt}finite{\isacharunderscore}{\kern0pt}function{\isacharunderscore}{\kern0pt}fm\ {\isacharcolon}{\kern0pt}\ \isanewline
\ \ \isakeyword{fixes}\ d\ r\ f\ i\ j\ k\ l\ env\ \isanewline
\ \ \isakeyword{assumes}\ {\isachardoublequoteopen}env\ {\isasymin}\ list{\isacharparenleft}{\kern0pt}M{\isacharparenright}{\kern0pt}{\isachardoublequoteclose}\ {\isachardoublequoteopen}i\ {\isacharless}{\kern0pt}\ length{\isacharparenleft}{\kern0pt}env{\isacharparenright}{\kern0pt}{\isachardoublequoteclose}\ {\isachardoublequoteopen}j\ {\isacharless}{\kern0pt}\ length{\isacharparenleft}{\kern0pt}env{\isacharparenright}{\kern0pt}{\isachardoublequoteclose}\ {\isachardoublequoteopen}k\ {\isacharless}{\kern0pt}\ length{\isacharparenleft}{\kern0pt}env{\isacharparenright}{\kern0pt}{\isachardoublequoteclose}\ {\isachardoublequoteopen}l\ {\isacharless}{\kern0pt}\ length{\isacharparenleft}{\kern0pt}env{\isacharparenright}{\kern0pt}{\isachardoublequoteclose}\ \isanewline
\ \ \ \ \ \ \ \ \ \ {\isachardoublequoteopen}nth{\isacharparenleft}{\kern0pt}i{\isacharcomma}{\kern0pt}\ env{\isacharparenright}{\kern0pt}\ {\isacharequal}{\kern0pt}\ d{\isachardoublequoteclose}\ {\isachardoublequoteopen}nth{\isacharparenleft}{\kern0pt}j{\isacharcomma}{\kern0pt}\ env{\isacharparenright}{\kern0pt}\ {\isacharequal}{\kern0pt}\ r{\isachardoublequoteclose}\ {\isachardoublequoteopen}nth{\isacharparenleft}{\kern0pt}k{\isacharcomma}{\kern0pt}\ env{\isacharparenright}{\kern0pt}\ {\isacharequal}{\kern0pt}\ nat{\isachardoublequoteclose}\ {\isachardoublequoteopen}nth{\isacharparenleft}{\kern0pt}l{\isacharcomma}{\kern0pt}\ env{\isacharparenright}{\kern0pt}\ {\isacharequal}{\kern0pt}\ f{\isachardoublequoteclose}\ \isanewline
\ \ \isakeyword{shows}\ {\isachardoublequoteopen}sats{\isacharparenleft}{\kern0pt}M{\isacharcomma}{\kern0pt}\ is{\isacharunderscore}{\kern0pt}finite{\isacharunderscore}{\kern0pt}function{\isacharunderscore}{\kern0pt}fm{\isacharparenleft}{\kern0pt}i{\isacharcomma}{\kern0pt}\ j{\isacharcomma}{\kern0pt}\ k{\isacharcomma}{\kern0pt}\ l{\isacharparenright}{\kern0pt}{\isacharcomma}{\kern0pt}\ env{\isacharparenright}{\kern0pt}\ {\isasymlongleftrightarrow}\ \isanewline
\ \ \ \ \ \ \ \ \ {\isacharparenleft}{\kern0pt}function{\isacharparenleft}{\kern0pt}f{\isacharparenright}{\kern0pt}\ {\isasymand}\ domain{\isacharparenleft}{\kern0pt}f{\isacharparenright}{\kern0pt}\ {\isasymsubseteq}\ d\ {\isasymand}\ finite{\isacharunderscore}{\kern0pt}M{\isacharparenleft}{\kern0pt}domain{\isacharparenleft}{\kern0pt}f{\isacharparenright}{\kern0pt}{\isacharparenright}{\kern0pt}\ {\isasymand}\ range{\isacharparenleft}{\kern0pt}f{\isacharparenright}{\kern0pt}\ {\isasymsubseteq}\ r{\isacharparenright}{\kern0pt}{\isachardoublequoteclose}\ \isanewline
%
\isadelimproof
\ \ %
\endisadelimproof
%
\isatagproof
\isacommand{unfolding}\isamarkupfalse%
\ is{\isacharunderscore}{\kern0pt}finite{\isacharunderscore}{\kern0pt}function{\isacharunderscore}{\kern0pt}fm{\isacharunderscore}{\kern0pt}def\isanewline
\ \ \isacommand{apply}\isamarkupfalse%
{\isacharparenleft}{\kern0pt}rule{\isacharunderscore}{\kern0pt}tac\ Q{\isacharequal}{\kern0pt}{\isachardoublequoteopen}function{\isacharparenleft}{\kern0pt}f{\isacharparenright}{\kern0pt}\ {\isasymand}\ \isanewline
\ \ \ \ \ \ \ \ \ \ \ \ \ \ \ \ \ \ \ \ {\isacharparenleft}{\kern0pt}{\isasymexists}d{\isacharprime}{\kern0pt}\ {\isasymin}\ M{\isachardot}{\kern0pt}\ d{\isacharprime}{\kern0pt}\ {\isacharequal}{\kern0pt}\ domain{\isacharparenleft}{\kern0pt}f{\isacharparenright}{\kern0pt}\ {\isasymand}\ d{\isacharprime}{\kern0pt}\ {\isasymsubseteq}\ d\ {\isasymand}\ finite{\isacharunderscore}{\kern0pt}M{\isacharparenleft}{\kern0pt}d{\isacharprime}{\kern0pt}{\isacharparenright}{\kern0pt}{\isacharparenright}{\kern0pt}\ {\isasymand}\ \isanewline
\ \ \ \ \ \ \ \ \ \ \ \ \ \ \ \ \ \ \ \ {\isacharparenleft}{\kern0pt}{\isasymexists}r{\isacharprime}{\kern0pt}\ {\isasymin}\ M{\isachardot}{\kern0pt}\ r{\isacharprime}{\kern0pt}\ {\isacharequal}{\kern0pt}\ range{\isacharparenleft}{\kern0pt}f{\isacharparenright}{\kern0pt}\ {\isasymand}\ r{\isacharprime}{\kern0pt}\ {\isasymsubseteq}\ r{\isacharparenright}{\kern0pt}{\isachardoublequoteclose}\ \isakeyword{in}\ iff{\isacharunderscore}{\kern0pt}trans{\isacharparenright}{\kern0pt}\isanewline
\ \ \ \isacommand{apply}\isamarkupfalse%
{\isacharparenleft}{\kern0pt}rule\ iff{\isacharunderscore}{\kern0pt}trans{\isacharcomma}{\kern0pt}\ rule\ sats{\isacharunderscore}{\kern0pt}And{\isacharunderscore}{\kern0pt}iff{\isacharcomma}{\kern0pt}\ simp\ add{\isacharcolon}{\kern0pt}assms{\isacharparenright}{\kern0pt}\isanewline
\ \ \ \isacommand{apply}\isamarkupfalse%
{\isacharparenleft}{\kern0pt}rule\ iff{\isacharunderscore}{\kern0pt}conjI{\isadigit{2}}{\isacharparenright}{\kern0pt}\isanewline
\ \ \isacommand{using}\isamarkupfalse%
\ lt{\isacharunderscore}{\kern0pt}nat{\isacharunderscore}{\kern0pt}in{\isacharunderscore}{\kern0pt}nat\ assms\ nth{\isacharunderscore}{\kern0pt}type\isanewline
\ \ \ \ \isacommand{apply}\isamarkupfalse%
\ force\isanewline
\ \ \ \isacommand{apply}\isamarkupfalse%
{\isacharparenleft}{\kern0pt}rule\ iff{\isacharunderscore}{\kern0pt}trans{\isacharcomma}{\kern0pt}\ rule\ sats{\isacharunderscore}{\kern0pt}And{\isacharunderscore}{\kern0pt}iff{\isacharcomma}{\kern0pt}\ simp\ add{\isacharcolon}{\kern0pt}assms{\isacharcomma}{\kern0pt}\ rule\ iff{\isacharunderscore}{\kern0pt}conjI{\isadigit{2}}{\isacharparenright}{\kern0pt}\isanewline
\ \ \ \ \isacommand{apply}\isamarkupfalse%
{\isacharparenleft}{\kern0pt}rule\ iff{\isacharunderscore}{\kern0pt}trans{\isacharcomma}{\kern0pt}\ rule\ sats{\isacharunderscore}{\kern0pt}Exists{\isacharunderscore}{\kern0pt}iff{\isacharcomma}{\kern0pt}\ simp\ add{\isacharcolon}{\kern0pt}assms{\isacharcomma}{\kern0pt}\ rule\ bex{\isacharunderscore}{\kern0pt}iff{\isacharparenright}{\kern0pt}{\isacharplus}{\kern0pt}\isanewline
\ \ \ \ \isacommand{apply}\isamarkupfalse%
{\isacharparenleft}{\kern0pt}rule\ iff{\isacharunderscore}{\kern0pt}trans{\isacharcomma}{\kern0pt}\ rule\ sats{\isacharunderscore}{\kern0pt}And{\isacharunderscore}{\kern0pt}iff{\isacharcomma}{\kern0pt}\ simp\ add{\isacharcolon}{\kern0pt}assms{\isacharcomma}{\kern0pt}\ rule\ iff{\isacharunderscore}{\kern0pt}conjI{\isadigit{2}}{\isacharparenright}{\kern0pt}\isanewline
\ \ \isacommand{using}\isamarkupfalse%
\ lt{\isacharunderscore}{\kern0pt}nat{\isacharunderscore}{\kern0pt}in{\isacharunderscore}{\kern0pt}nat\ assms\ nth{\isacharunderscore}{\kern0pt}type\isanewline
\ \ \ \ \ \isacommand{apply}\isamarkupfalse%
\ force\isanewline
\ \ \ \ \isacommand{apply}\isamarkupfalse%
{\isacharparenleft}{\kern0pt}rule\ iff{\isacharunderscore}{\kern0pt}trans{\isacharcomma}{\kern0pt}\ rule\ sats{\isacharunderscore}{\kern0pt}And{\isacharunderscore}{\kern0pt}iff{\isacharcomma}{\kern0pt}\ simp\ add{\isacharcolon}{\kern0pt}assms{\isacharcomma}{\kern0pt}\ rule\ iff{\isacharunderscore}{\kern0pt}conjI{\isadigit{2}}{\isacharparenright}{\kern0pt}\isanewline
\ \ \ \ \ \isacommand{apply}\isamarkupfalse%
{\isacharparenleft}{\kern0pt}rule\ iff{\isacharunderscore}{\kern0pt}trans{\isacharcomma}{\kern0pt}\ rule\ sats{\isacharunderscore}{\kern0pt}subset{\isacharunderscore}{\kern0pt}fm{\isacharparenright}{\kern0pt}\isanewline
\ \ \isacommand{using}\isamarkupfalse%
\ lt{\isacharunderscore}{\kern0pt}nat{\isacharunderscore}{\kern0pt}in{\isacharunderscore}{\kern0pt}nat\ assms\ nth{\isacharunderscore}{\kern0pt}type\ M{\isacharunderscore}{\kern0pt}ctm{\isacharunderscore}{\kern0pt}axioms\ M{\isacharunderscore}{\kern0pt}ctm{\isacharunderscore}{\kern0pt}def\ M{\isacharunderscore}{\kern0pt}ctm{\isacharunderscore}{\kern0pt}axioms{\isacharunderscore}{\kern0pt}def\isanewline
\ \ \ \ \ \ \ \ \ \isacommand{apply}\isamarkupfalse%
\ auto{\isacharbrackleft}{\kern0pt}{\isadigit{5}}{\isacharbrackright}{\kern0pt}\isanewline
\ \ \ \ \isacommand{apply}\isamarkupfalse%
{\isacharparenleft}{\kern0pt}rule\ sats{\isacharunderscore}{\kern0pt}finite{\isacharunderscore}{\kern0pt}M{\isacharunderscore}{\kern0pt}fm{\isacharunderscore}{\kern0pt}iff{\isacharparenright}{\kern0pt}\isanewline
\ \ \isacommand{using}\isamarkupfalse%
\ lt{\isacharunderscore}{\kern0pt}nat{\isacharunderscore}{\kern0pt}in{\isacharunderscore}{\kern0pt}nat\ assms\ nth{\isacharunderscore}{\kern0pt}type\ M{\isacharunderscore}{\kern0pt}ctm{\isacharunderscore}{\kern0pt}axioms\ M{\isacharunderscore}{\kern0pt}ctm{\isacharunderscore}{\kern0pt}def\ M{\isacharunderscore}{\kern0pt}ctm{\isacharunderscore}{\kern0pt}axioms{\isacharunderscore}{\kern0pt}def\isanewline
\ \ \ \ \ \ \ \ \isacommand{apply}\isamarkupfalse%
\ auto{\isacharbrackleft}{\kern0pt}{\isadigit{5}}{\isacharbrackright}{\kern0pt}\isanewline
\ \ \ \isacommand{apply}\isamarkupfalse%
{\isacharparenleft}{\kern0pt}rule\ iff{\isacharunderscore}{\kern0pt}trans{\isacharcomma}{\kern0pt}\ rule\ sats{\isacharunderscore}{\kern0pt}Exists{\isacharunderscore}{\kern0pt}iff{\isacharcomma}{\kern0pt}\ simp\ add{\isacharcolon}{\kern0pt}assms{\isacharcomma}{\kern0pt}\ rule\ bex{\isacharunderscore}{\kern0pt}iff{\isacharparenright}{\kern0pt}\isanewline
\ \ \ \isacommand{apply}\isamarkupfalse%
{\isacharparenleft}{\kern0pt}rule\ iff{\isacharunderscore}{\kern0pt}trans{\isacharcomma}{\kern0pt}\ rule\ sats{\isacharunderscore}{\kern0pt}And{\isacharunderscore}{\kern0pt}iff{\isacharcomma}{\kern0pt}\ simp\ add{\isacharcolon}{\kern0pt}assms{\isacharcomma}{\kern0pt}\ rule\ iff{\isacharunderscore}{\kern0pt}conjI{\isadigit{2}}{\isacharparenright}{\kern0pt}\isanewline
\ \ \isacommand{using}\isamarkupfalse%
\ sats{\isacharunderscore}{\kern0pt}range{\isacharunderscore}{\kern0pt}fm\ lt{\isacharunderscore}{\kern0pt}nat{\isacharunderscore}{\kern0pt}in{\isacharunderscore}{\kern0pt}nat\ assms\ nth{\isacharunderscore}{\kern0pt}type\ \isanewline
\ \ \ \ \isacommand{apply}\isamarkupfalse%
\ auto{\isacharbrackleft}{\kern0pt}{\isadigit{1}}{\isacharbrackright}{\kern0pt}\isanewline
\ \ \isacommand{apply}\isamarkupfalse%
{\isacharparenleft}{\kern0pt}rule\ iff{\isacharunderscore}{\kern0pt}trans{\isacharcomma}{\kern0pt}\ rule\ sats{\isacharunderscore}{\kern0pt}subset{\isacharunderscore}{\kern0pt}fm{\isacharparenright}{\kern0pt}\isanewline
\ \ \isacommand{using}\isamarkupfalse%
\ sats{\isacharunderscore}{\kern0pt}range{\isacharunderscore}{\kern0pt}fm\ lt{\isacharunderscore}{\kern0pt}nat{\isacharunderscore}{\kern0pt}in{\isacharunderscore}{\kern0pt}nat\ assms\ nth{\isacharunderscore}{\kern0pt}type\ M{\isacharunderscore}{\kern0pt}ctm{\isacharunderscore}{\kern0pt}axioms\ M{\isacharunderscore}{\kern0pt}ctm{\isacharunderscore}{\kern0pt}def\ M{\isacharunderscore}{\kern0pt}ctm{\isacharunderscore}{\kern0pt}axioms{\isacharunderscore}{\kern0pt}def\isanewline
\ \ \ \ \ \ \ \isacommand{apply}\isamarkupfalse%
\ auto{\isacharbrackleft}{\kern0pt}{\isadigit{5}}{\isacharbrackright}{\kern0pt}\isanewline
\ \ \isacommand{apply}\isamarkupfalse%
{\isacharparenleft}{\kern0pt}rule\ iff{\isacharunderscore}{\kern0pt}conjI{\isadigit{2}}{\isacharcomma}{\kern0pt}\ simp{\isacharparenright}{\kern0pt}\isanewline
\ \ \isacommand{apply}\isamarkupfalse%
\ {\isacharparenleft}{\kern0pt}rule\ iffI{\isacharcomma}{\kern0pt}\ clarsimp{\isacharcomma}{\kern0pt}\ rule\ conjI{\isacharparenright}{\kern0pt}\isanewline
\ \ \isacommand{using}\isamarkupfalse%
\ assms\ nth{\isacharunderscore}{\kern0pt}type\ lt{\isacharunderscore}{\kern0pt}nat{\isacharunderscore}{\kern0pt}in{\isacharunderscore}{\kern0pt}nat\ domain{\isacharunderscore}{\kern0pt}closed\ range{\isacharunderscore}{\kern0pt}closed\isanewline
\ \ \isacommand{by}\isamarkupfalse%
\ auto%
\endisatagproof
{\isafoldproof}%
%
\isadelimproof
\isanewline
%
\endisadelimproof
\isanewline
\isacommand{definition}\isamarkupfalse%
\ Fn\ \isakeyword{where}\ {\isachardoublequoteopen}Fn\ {\isasymequiv}\ {\isacharbraceleft}{\kern0pt}\ f\ {\isasymin}\ Pow{\isacharparenleft}{\kern0pt}{\isacharparenleft}{\kern0pt}nat\ {\isasymtimes}\ nat{\isacharparenright}{\kern0pt}\ {\isasymtimes}\ {\isadigit{2}}{\isacharparenright}{\kern0pt}\ {\isasyminter}\ M{\isachardot}{\kern0pt}\ function{\isacharparenleft}{\kern0pt}f{\isacharparenright}{\kern0pt}\ {\isasymand}\ domain{\isacharparenleft}{\kern0pt}f{\isacharparenright}{\kern0pt}\ {\isasymsubseteq}\ nat\ {\isasymtimes}\ nat\ {\isasymand}\ finite{\isacharunderscore}{\kern0pt}M{\isacharparenleft}{\kern0pt}domain{\isacharparenleft}{\kern0pt}f{\isacharparenright}{\kern0pt}{\isacharparenright}{\kern0pt}\ {\isasymand}\ range{\isacharparenleft}{\kern0pt}f{\isacharparenright}{\kern0pt}\ {\isasymsubseteq}\ {\isadigit{2}}\ {\isacharbraceright}{\kern0pt}{\isachardoublequoteclose}\isanewline
\isanewline
\isacommand{lemma}\isamarkupfalse%
\ Fn{\isacharunderscore}{\kern0pt}in{\isacharunderscore}{\kern0pt}M\ {\isacharcolon}{\kern0pt}\ {\isachardoublequoteopen}Fn\ {\isasymin}\ M{\isachardoublequoteclose}\ \isanewline
%
\isadelimproof
%
\endisadelimproof
%
\isatagproof
\isacommand{proof}\isamarkupfalse%
\ {\isacharminus}{\kern0pt}\ \isanewline
\isanewline
\ \ \isacommand{define}\isamarkupfalse%
\ X\ \isakeyword{where}\ {\isachardoublequoteopen}X\ {\isasymequiv}\ {\isacharbraceleft}{\kern0pt}\ f\ {\isasymin}\ Pow{\isacharparenleft}{\kern0pt}{\isacharparenleft}{\kern0pt}nat\ {\isasymtimes}\ nat{\isacharparenright}{\kern0pt}\ {\isasymtimes}\ {\isadigit{2}}{\isacharparenright}{\kern0pt}\ {\isasyminter}\ M{\isachardot}{\kern0pt}\ sats{\isacharparenleft}{\kern0pt}M{\isacharcomma}{\kern0pt}\ is{\isacharunderscore}{\kern0pt}finite{\isacharunderscore}{\kern0pt}function{\isacharunderscore}{\kern0pt}fm{\isacharparenleft}{\kern0pt}{\isadigit{1}}{\isacharcomma}{\kern0pt}\ {\isadigit{2}}{\isacharcomma}{\kern0pt}\ {\isadigit{3}}{\isacharcomma}{\kern0pt}\ {\isadigit{0}}{\isacharparenright}{\kern0pt}{\isacharcomma}{\kern0pt}\ {\isacharbrackleft}{\kern0pt}f{\isacharbrackright}{\kern0pt}\ {\isacharat}{\kern0pt}\ {\isacharbrackleft}{\kern0pt}nat{\isasymtimes}nat{\isacharcomma}{\kern0pt}\ {\isadigit{2}}{\isacharcomma}{\kern0pt}\ nat{\isacharbrackright}{\kern0pt}{\isacharparenright}{\kern0pt}\ {\isacharbraceright}{\kern0pt}{\isachardoublequoteclose}\isanewline
\ \ \isacommand{have}\isamarkupfalse%
\ {\isachardoublequoteopen}X\ {\isasymin}\ M{\isachardoublequoteclose}\ \isanewline
\ \ \ \ \isacommand{unfolding}\isamarkupfalse%
\ X{\isacharunderscore}{\kern0pt}def\isanewline
\ \ \ \ \isacommand{apply}\isamarkupfalse%
{\isacharparenleft}{\kern0pt}rule\ separation{\isacharunderscore}{\kern0pt}notation{\isacharparenright}{\kern0pt}\isanewline
\ \ \ \ \ \isacommand{apply}\isamarkupfalse%
{\isacharparenleft}{\kern0pt}rule\ separation{\isacharunderscore}{\kern0pt}ax{\isacharcomma}{\kern0pt}\ rule\ is{\isacharunderscore}{\kern0pt}finite{\isacharunderscore}{\kern0pt}function{\isacharunderscore}{\kern0pt}fm{\isacharunderscore}{\kern0pt}type{\isacharparenright}{\kern0pt}\isanewline
\ \ \ \ \isacommand{using}\isamarkupfalse%
\ nat{\isacharunderscore}{\kern0pt}in{\isacharunderscore}{\kern0pt}M\ cartprod{\isacharunderscore}{\kern0pt}closed\ zero{\isacharunderscore}{\kern0pt}in{\isacharunderscore}{\kern0pt}M\ succ{\isacharunderscore}{\kern0pt}in{\isacharunderscore}{\kern0pt}MI\isanewline
\ \ \ \ \ \ \ \ \ \ \isacommand{apply}\isamarkupfalse%
\ auto{\isacharbrackleft}{\kern0pt}{\isadigit{5}}{\isacharbrackright}{\kern0pt}\isanewline
\ \ \ \ \ \isacommand{apply}\isamarkupfalse%
{\isacharparenleft}{\kern0pt}rule\ le{\isacharunderscore}{\kern0pt}trans{\isacharcomma}{\kern0pt}\ rule\ arity{\isacharunderscore}{\kern0pt}is{\isacharunderscore}{\kern0pt}finite{\isacharunderscore}{\kern0pt}function{\isacharunderscore}{\kern0pt}fm{\isacharparenright}{\kern0pt}\isanewline
\ \ \ \ \ \ \ \ \ \isacommand{apply}\isamarkupfalse%
\ auto{\isacharbrackleft}{\kern0pt}{\isadigit{4}}{\isacharbrackright}{\kern0pt}\isanewline
\ \ \ \ \ \isacommand{apply}\isamarkupfalse%
{\isacharparenleft}{\kern0pt}rule\ Un{\isacharunderscore}{\kern0pt}least{\isacharunderscore}{\kern0pt}lt{\isacharparenright}{\kern0pt}{\isacharplus}{\kern0pt}\isanewline
\ \ \ \ \ \ \ \ \isacommand{apply}\isamarkupfalse%
\ auto{\isacharbrackleft}{\kern0pt}{\isadigit{4}}{\isacharbrackright}{\kern0pt}\isanewline
\ \ \ \ \isacommand{apply}\isamarkupfalse%
{\isacharparenleft}{\kern0pt}rule\ M{\isacharunderscore}{\kern0pt}powerset{\isacharparenright}{\kern0pt}\isanewline
\ \ \ \ \isacommand{using}\isamarkupfalse%
\ cartprod{\isacharunderscore}{\kern0pt}closed\ nat{\isacharunderscore}{\kern0pt}in{\isacharunderscore}{\kern0pt}M\ zero{\isacharunderscore}{\kern0pt}in{\isacharunderscore}{\kern0pt}M\ succ{\isacharunderscore}{\kern0pt}in{\isacharunderscore}{\kern0pt}MI\isanewline
\ \ \ \ \isacommand{by}\isamarkupfalse%
\ auto\isanewline
\isanewline
\ \ \isacommand{have}\isamarkupfalse%
\ {\isachardoublequoteopen}X\ {\isacharequal}{\kern0pt}\ Fn{\isachardoublequoteclose}\ \isanewline
\ \ \ \ \isacommand{unfolding}\isamarkupfalse%
\ X{\isacharunderscore}{\kern0pt}def\ Fn{\isacharunderscore}{\kern0pt}def\ \isanewline
\ \ \ \ \isacommand{apply}\isamarkupfalse%
{\isacharparenleft}{\kern0pt}rule\ iff{\isacharunderscore}{\kern0pt}eq{\isacharparenright}{\kern0pt}\isanewline
\ \ \ \ \isacommand{apply}\isamarkupfalse%
{\isacharparenleft}{\kern0pt}rule\ sats{\isacharunderscore}{\kern0pt}is{\isacharunderscore}{\kern0pt}finite{\isacharunderscore}{\kern0pt}function{\isacharunderscore}{\kern0pt}fm{\isacharparenright}{\kern0pt}\isanewline
\ \ \ \ \isacommand{using}\isamarkupfalse%
\ nat{\isacharunderscore}{\kern0pt}in{\isacharunderscore}{\kern0pt}M\ cartprod{\isacharunderscore}{\kern0pt}closed\ zero{\isacharunderscore}{\kern0pt}in{\isacharunderscore}{\kern0pt}M\ succ{\isacharunderscore}{\kern0pt}in{\isacharunderscore}{\kern0pt}MI\isanewline
\ \ \ \ \isacommand{apply}\isamarkupfalse%
\ auto{\isacharbrackleft}{\kern0pt}{\isadigit{6}}{\isacharbrackright}{\kern0pt}\isanewline
\ \ \ \ \isacommand{by}\isamarkupfalse%
\ auto\isanewline
\ \ \isacommand{then}\isamarkupfalse%
\ \isacommand{show}\isamarkupfalse%
\ {\isacharquery}{\kern0pt}thesis\ \isacommand{using}\isamarkupfalse%
\ {\isacartoucheopen}X\ {\isasymin}\ M{\isacartoucheclose}\ \isacommand{by}\isamarkupfalse%
\ auto\isanewline
\isacommand{qed}\isamarkupfalse%
%
\endisatagproof
{\isafoldproof}%
%
\isadelimproof
\isanewline
%
\endisadelimproof
\isanewline
\isacommand{definition}\isamarkupfalse%
\ Fn{\isacharunderscore}{\kern0pt}leq\ \isakeyword{where}\ {\isachardoublequoteopen}Fn{\isacharunderscore}{\kern0pt}leq\ {\isasymequiv}\ {\isacharbraceleft}{\kern0pt}\ {\isacharless}{\kern0pt}f{\isacharcomma}{\kern0pt}\ g{\isachargreater}{\kern0pt}\ {\isasymin}\ Fn\ {\isasymtimes}\ Fn{\isachardot}{\kern0pt}\ g\ {\isasymsubseteq}\ f\ {\isacharbraceright}{\kern0pt}{\isachardoublequoteclose}\ \isanewline
\isanewline
\isacommand{lemma}\isamarkupfalse%
\ Fn{\isacharunderscore}{\kern0pt}leq{\isacharunderscore}{\kern0pt}in{\isacharunderscore}{\kern0pt}M\ {\isacharcolon}{\kern0pt}\ {\isachardoublequoteopen}Fn{\isacharunderscore}{\kern0pt}leq\ {\isasymin}\ M{\isachardoublequoteclose}\ \isanewline
%
\isadelimproof
%
\endisadelimproof
%
\isatagproof
\isacommand{proof}\isamarkupfalse%
\ {\isacharminus}{\kern0pt}\ \isanewline
\ \ \isacommand{define}\isamarkupfalse%
\ X\ \isakeyword{where}\ {\isachardoublequoteopen}X\ {\isasymequiv}\ {\isacharbraceleft}{\kern0pt}\ v\ {\isasymin}\ Fn\ {\isasymtimes}\ Fn\ {\isachardot}{\kern0pt}\ sats{\isacharparenleft}{\kern0pt}M{\isacharcomma}{\kern0pt}\ Exists{\isacharparenleft}{\kern0pt}Exists{\isacharparenleft}{\kern0pt}And{\isacharparenleft}{\kern0pt}pair{\isacharunderscore}{\kern0pt}fm{\isacharparenleft}{\kern0pt}{\isadigit{0}}{\isacharcomma}{\kern0pt}\ {\isadigit{1}}{\isacharcomma}{\kern0pt}\ {\isadigit{2}}{\isacharparenright}{\kern0pt}{\isacharcomma}{\kern0pt}\ subset{\isacharunderscore}{\kern0pt}fm{\isacharparenleft}{\kern0pt}{\isadigit{1}}{\isacharcomma}{\kern0pt}\ {\isadigit{0}}{\isacharparenright}{\kern0pt}{\isacharparenright}{\kern0pt}{\isacharparenright}{\kern0pt}{\isacharparenright}{\kern0pt}{\isacharcomma}{\kern0pt}\ {\isacharbrackleft}{\kern0pt}v{\isacharbrackright}{\kern0pt}\ {\isacharat}{\kern0pt}\ {\isacharbrackleft}{\kern0pt}{\isacharbrackright}{\kern0pt}{\isacharparenright}{\kern0pt}\ {\isacharbraceright}{\kern0pt}{\isachardoublequoteclose}\ \isanewline
\ \ \isacommand{have}\isamarkupfalse%
\ {\isachardoublequoteopen}X\ {\isasymin}\ M{\isachardoublequoteclose}\ \isanewline
\ \ \ \ \isacommand{unfolding}\isamarkupfalse%
\ X{\isacharunderscore}{\kern0pt}def\isanewline
\ \ \ \ \isacommand{apply}\isamarkupfalse%
{\isacharparenleft}{\kern0pt}rule\ separation{\isacharunderscore}{\kern0pt}notation{\isacharparenright}{\kern0pt}\isanewline
\ \ \ \ \ \isacommand{apply}\isamarkupfalse%
{\isacharparenleft}{\kern0pt}rule\ separation{\isacharunderscore}{\kern0pt}ax{\isacharparenright}{\kern0pt}\isanewline
\ \ \ \ \ \ \ \isacommand{apply}\isamarkupfalse%
\ auto{\isacharbrackleft}{\kern0pt}{\isadigit{2}}{\isacharbrackright}{\kern0pt}\isanewline
\ \ \ \ \ \isacommand{apply}\isamarkupfalse%
\ {\isacharparenleft}{\kern0pt}simp\ del{\isacharcolon}{\kern0pt}FOL{\isacharunderscore}{\kern0pt}sats{\isacharunderscore}{\kern0pt}iff\ pair{\isacharunderscore}{\kern0pt}abs\ add{\isacharcolon}{\kern0pt}\ fm{\isacharunderscore}{\kern0pt}defs\ nat{\isacharunderscore}{\kern0pt}simp{\isacharunderscore}{\kern0pt}union{\isacharparenright}{\kern0pt}\isanewline
\ \ \ \ \isacommand{using}\isamarkupfalse%
\ Fn{\isacharunderscore}{\kern0pt}in{\isacharunderscore}{\kern0pt}M\ cartprod{\isacharunderscore}{\kern0pt}closed\isanewline
\ \ \ \ \isacommand{by}\isamarkupfalse%
\ auto\isanewline
\ \ \isacommand{have}\isamarkupfalse%
\ {\isachardoublequoteopen}X\ {\isacharequal}{\kern0pt}\ {\isacharbraceleft}{\kern0pt}\ {\isacharless}{\kern0pt}f{\isacharcomma}{\kern0pt}\ g{\isachargreater}{\kern0pt}\ {\isasymin}\ Fn\ {\isasymtimes}\ Fn{\isachardot}{\kern0pt}\ g\ {\isasymsubseteq}\ f\ {\isacharbraceright}{\kern0pt}{\isachardoublequoteclose}\ \isanewline
\ \ \ \ \isacommand{unfolding}\isamarkupfalse%
\ X{\isacharunderscore}{\kern0pt}def\ \isanewline
\ \ \ \ \isacommand{apply}\isamarkupfalse%
{\isacharparenleft}{\kern0pt}rule\ equality{\isacharunderscore}{\kern0pt}iffI{\isacharcomma}{\kern0pt}\ rule\ iffI{\isacharparenright}{\kern0pt}\isanewline
\ \ \ \ \isacommand{using}\isamarkupfalse%
\ pair{\isacharunderscore}{\kern0pt}in{\isacharunderscore}{\kern0pt}M{\isacharunderscore}{\kern0pt}iff\ sats{\isacharunderscore}{\kern0pt}subset{\isacharunderscore}{\kern0pt}fm{\isacharprime}{\kern0pt}\ Fn{\isacharunderscore}{\kern0pt}in{\isacharunderscore}{\kern0pt}M\ transM\isanewline
\ \ \ \ \isacommand{by}\isamarkupfalse%
\ auto\isanewline
\ \ \isacommand{then}\isamarkupfalse%
\ \isacommand{show}\isamarkupfalse%
\ {\isacharquery}{\kern0pt}thesis\ \isacommand{using}\isamarkupfalse%
\ {\isacartoucheopen}X\ {\isasymin}\ M{\isacartoucheclose}\ Fn{\isacharunderscore}{\kern0pt}leq{\isacharunderscore}{\kern0pt}def\ \isacommand{by}\isamarkupfalse%
\ auto\isanewline
\isacommand{qed}\isamarkupfalse%
%
\endisatagproof
{\isafoldproof}%
%
\isadelimproof
\isanewline
%
\endisadelimproof
\isanewline
\isacommand{lemma}\isamarkupfalse%
\ Fn{\isacharunderscore}{\kern0pt}partial{\isacharunderscore}{\kern0pt}ord\ {\isacharcolon}{\kern0pt}\ {\isachardoublequoteopen}partial{\isacharunderscore}{\kern0pt}order{\isacharunderscore}{\kern0pt}on{\isacharparenleft}{\kern0pt}Fn{\isacharcomma}{\kern0pt}\ Fn{\isacharunderscore}{\kern0pt}leq{\isacharparenright}{\kern0pt}{\isachardoublequoteclose}\ \isanewline
%
\isadelimproof
\ \ %
\endisadelimproof
%
\isatagproof
\isacommand{unfolding}\isamarkupfalse%
\ partial{\isacharunderscore}{\kern0pt}order{\isacharunderscore}{\kern0pt}on{\isacharunderscore}{\kern0pt}def\ preorder{\isacharunderscore}{\kern0pt}on{\isacharunderscore}{\kern0pt}def\ \isanewline
\ \ \isacommand{apply}\isamarkupfalse%
{\isacharparenleft}{\kern0pt}rule\ conjI{\isacharparenright}{\kern0pt}{\isacharplus}{\kern0pt}\isanewline
\ \ \isacommand{unfolding}\isamarkupfalse%
\ refl{\isacharunderscore}{\kern0pt}def\ Fn{\isacharunderscore}{\kern0pt}leq{\isacharunderscore}{\kern0pt}def\ trans{\isacharunderscore}{\kern0pt}on{\isacharunderscore}{\kern0pt}def\ antisym{\isacharunderscore}{\kern0pt}def\isanewline
\ \ \isacommand{by}\isamarkupfalse%
\ auto%
\endisatagproof
{\isafoldproof}%
%
\isadelimproof
\isanewline
%
\endisadelimproof
\isanewline
\isacommand{lemma}\isamarkupfalse%
\ zero{\isacharunderscore}{\kern0pt}in{\isacharunderscore}{\kern0pt}Fn\ {\isacharcolon}{\kern0pt}\ {\isachardoublequoteopen}{\isadigit{0}}\ {\isasymin}\ Fn{\isachardoublequoteclose}\ \isanewline
%
\isadelimproof
\ \ %
\endisadelimproof
%
\isatagproof
\isacommand{apply}\isamarkupfalse%
{\isacharparenleft}{\kern0pt}subst\ Fn{\isacharunderscore}{\kern0pt}def{\isacharcomma}{\kern0pt}\ simp{\isacharcomma}{\kern0pt}\ rule\ conjI{\isacharcomma}{\kern0pt}\ rule\ zero{\isacharunderscore}{\kern0pt}in{\isacharunderscore}{\kern0pt}M{\isacharcomma}{\kern0pt}\ rule\ conjI{\isacharparenright}{\kern0pt}\isanewline
\ \ \ \isacommand{apply}\isamarkupfalse%
{\isacharparenleft}{\kern0pt}simp\ add{\isacharcolon}{\kern0pt}function{\isacharunderscore}{\kern0pt}def{\isacharparenright}{\kern0pt}\isanewline
\ \ \isacommand{unfolding}\isamarkupfalse%
\ finite{\isacharunderscore}{\kern0pt}M{\isacharunderscore}{\kern0pt}def\ \isanewline
\ \ \isacommand{apply}\isamarkupfalse%
{\isacharparenleft}{\kern0pt}rule{\isacharunderscore}{\kern0pt}tac\ x{\isacharequal}{\kern0pt}{\isadigit{0}}\ \isakeyword{in}\ bexI{\isacharcomma}{\kern0pt}\ rule{\isacharunderscore}{\kern0pt}tac\ a{\isacharequal}{\kern0pt}{\isachardoublequoteopen}{\isadigit{0}}{\isachardoublequoteclose}\ \isakeyword{in}\ not{\isacharunderscore}{\kern0pt}emptyI{\isacharparenright}{\kern0pt}\ \isanewline
\ \ \ \isacommand{apply}\isamarkupfalse%
\ simp\isanewline
\ \ \ \isacommand{apply}\isamarkupfalse%
{\isacharparenleft}{\kern0pt}rule\ conjI{\isacharparenright}{\kern0pt}\isanewline
\ \ \isacommand{unfolding}\isamarkupfalse%
\ inj{\isacharunderscore}{\kern0pt}def\isanewline
\ \ \isacommand{using}\isamarkupfalse%
\ zero{\isacharunderscore}{\kern0pt}in{\isacharunderscore}{\kern0pt}M\isanewline
\ \ \isacommand{by}\isamarkupfalse%
\ auto%
\endisatagproof
{\isafoldproof}%
%
\isadelimproof
\isanewline
%
\endisadelimproof
\isanewline
\isacommand{lemma}\isamarkupfalse%
\ Fn{\isacharunderscore}{\kern0pt}forcing{\isacharunderscore}{\kern0pt}notion\ {\isacharcolon}{\kern0pt}\ {\isachardoublequoteopen}forcing{\isacharunderscore}{\kern0pt}notion{\isacharparenleft}{\kern0pt}Fn{\isacharcomma}{\kern0pt}\ Fn{\isacharunderscore}{\kern0pt}leq{\isacharcomma}{\kern0pt}\ {\isadigit{0}}{\isacharparenright}{\kern0pt}{\isachardoublequoteclose}\isanewline
%
\isadelimproof
\ \ %
\endisadelimproof
%
\isatagproof
\isacommand{unfolding}\isamarkupfalse%
\ forcing{\isacharunderscore}{\kern0pt}notion{\isacharunderscore}{\kern0pt}def\isanewline
\ \ \isacommand{using}\isamarkupfalse%
\ Fn{\isacharunderscore}{\kern0pt}partial{\isacharunderscore}{\kern0pt}ord\ partial{\isacharunderscore}{\kern0pt}order{\isacharunderscore}{\kern0pt}on{\isacharunderscore}{\kern0pt}def\ zero{\isacharunderscore}{\kern0pt}in{\isacharunderscore}{\kern0pt}Fn\isanewline
\ \ \isacommand{apply}\isamarkupfalse%
\ simp\isanewline
\ \ \isacommand{unfolding}\isamarkupfalse%
\ Fn{\isacharunderscore}{\kern0pt}leq{\isacharunderscore}{\kern0pt}def\isanewline
\ \ \isacommand{apply}\isamarkupfalse%
{\isacharparenleft}{\kern0pt}rule\ ballI{\isacharcomma}{\kern0pt}\ simp{\isacharparenright}{\kern0pt}\isanewline
\ \ \isacommand{done}\isamarkupfalse%
%
\endisatagproof
{\isafoldproof}%
%
\isadelimproof
\isanewline
%
\endisadelimproof
\isanewline
\isacommand{lemma}\isamarkupfalse%
\ Fn{\isacharunderscore}{\kern0pt}forcing{\isacharunderscore}{\kern0pt}data{\isacharunderscore}{\kern0pt}partial\ {\isacharcolon}{\kern0pt}\ {\isachardoublequoteopen}forcing{\isacharunderscore}{\kern0pt}data{\isacharunderscore}{\kern0pt}partial{\isacharparenleft}{\kern0pt}Fn{\isacharcomma}{\kern0pt}\ Fn{\isacharunderscore}{\kern0pt}leq{\isacharcomma}{\kern0pt}\ {\isadigit{0}}{\isacharcomma}{\kern0pt}\ M{\isacharcomma}{\kern0pt}\ enum{\isacharparenright}{\kern0pt}{\isachardoublequoteclose}\ \isanewline
%
\isadelimproof
\ \ %
\endisadelimproof
%
\isatagproof
\isacommand{unfolding}\isamarkupfalse%
\ forcing{\isacharunderscore}{\kern0pt}data{\isacharunderscore}{\kern0pt}partial{\isacharunderscore}{\kern0pt}def\ forcing{\isacharunderscore}{\kern0pt}data{\isacharunderscore}{\kern0pt}def\ forcing{\isacharunderscore}{\kern0pt}data{\isacharunderscore}{\kern0pt}partial{\isacharunderscore}{\kern0pt}axioms{\isacharunderscore}{\kern0pt}def\ forcing{\isacharunderscore}{\kern0pt}data{\isacharunderscore}{\kern0pt}axioms{\isacharunderscore}{\kern0pt}def\ \isanewline
\ \ \isacommand{using}\isamarkupfalse%
\ Fn{\isacharunderscore}{\kern0pt}forcing{\isacharunderscore}{\kern0pt}notion\ M{\isacharunderscore}{\kern0pt}ctm{\isacharunderscore}{\kern0pt}axioms\ Fn{\isacharunderscore}{\kern0pt}in{\isacharunderscore}{\kern0pt}M\ Fn{\isacharunderscore}{\kern0pt}leq{\isacharunderscore}{\kern0pt}in{\isacharunderscore}{\kern0pt}M\ Fn{\isacharunderscore}{\kern0pt}leq{\isacharunderscore}{\kern0pt}def\ Fn{\isacharunderscore}{\kern0pt}partial{\isacharunderscore}{\kern0pt}ord\isanewline
\ \ \isacommand{by}\isamarkupfalse%
\ auto%
\endisatagproof
{\isafoldproof}%
%
\isadelimproof
\isanewline
%
\endisadelimproof
\isanewline
\isacommand{definition}\isamarkupfalse%
\ nat{\isacharunderscore}{\kern0pt}perms\ \isakeyword{where}\ {\isachardoublequoteopen}nat{\isacharunderscore}{\kern0pt}perms\ {\isasymequiv}\ bij{\isacharparenleft}{\kern0pt}nat{\isacharcomma}{\kern0pt}\ nat{\isacharparenright}{\kern0pt}\ {\isasyminter}\ M{\isachardoublequoteclose}\ \isanewline
\isanewline
\isacommand{lemma}\isamarkupfalse%
\ nat{\isacharunderscore}{\kern0pt}perms{\isacharunderscore}{\kern0pt}in{\isacharunderscore}{\kern0pt}M\ {\isacharcolon}{\kern0pt}\ {\isachardoublequoteopen}nat{\isacharunderscore}{\kern0pt}perms\ {\isasymin}\ M{\isachardoublequoteclose}\ \isanewline
%
\isadelimproof
%
\endisadelimproof
%
\isatagproof
\isacommand{proof}\isamarkupfalse%
\ {\isacharminus}{\kern0pt}\ \isanewline
\ \ \isacommand{define}\isamarkupfalse%
\ X\ \isakeyword{where}\ {\isachardoublequoteopen}X\ {\isasymequiv}\ {\isacharbraceleft}{\kern0pt}\ f\ {\isasymin}\ Pow{\isacharparenleft}{\kern0pt}nat\ {\isasymtimes}\ nat{\isacharparenright}{\kern0pt}\ {\isasyminter}\ M{\isachardot}{\kern0pt}\ sats{\isacharparenleft}{\kern0pt}M{\isacharcomma}{\kern0pt}\ bijection{\isacharunderscore}{\kern0pt}fm{\isacharparenleft}{\kern0pt}{\isadigit{1}}{\isacharcomma}{\kern0pt}\ {\isadigit{1}}{\isacharcomma}{\kern0pt}\ {\isadigit{0}}{\isacharparenright}{\kern0pt}{\isacharcomma}{\kern0pt}\ {\isacharbrackleft}{\kern0pt}f{\isacharbrackright}{\kern0pt}\ {\isacharat}{\kern0pt}\ {\isacharbrackleft}{\kern0pt}nat{\isacharbrackright}{\kern0pt}{\isacharparenright}{\kern0pt}\ {\isacharbraceright}{\kern0pt}{\isachardoublequoteclose}\ \isanewline
\isanewline
\ \ \isacommand{have}\isamarkupfalse%
\ XinM\ {\isacharcolon}{\kern0pt}\ {\isachardoublequoteopen}X\ {\isasymin}\ M{\isachardoublequoteclose}\ \isanewline
\ \ \ \ \isacommand{unfolding}\isamarkupfalse%
\ X{\isacharunderscore}{\kern0pt}def\isanewline
\ \ \ \ \isacommand{apply}\isamarkupfalse%
{\isacharparenleft}{\kern0pt}rule\ separation{\isacharunderscore}{\kern0pt}notation{\isacharparenright}{\kern0pt}\isanewline
\ \ \ \ \ \isacommand{apply}\isamarkupfalse%
{\isacharparenleft}{\kern0pt}rule\ separation{\isacharunderscore}{\kern0pt}ax{\isacharparenright}{\kern0pt}\isanewline
\ \ \ \ \isacommand{using}\isamarkupfalse%
\ nat{\isacharunderscore}{\kern0pt}in{\isacharunderscore}{\kern0pt}M\isanewline
\ \ \ \ \ \ \ \isacommand{apply}\isamarkupfalse%
\ auto{\isacharbrackleft}{\kern0pt}{\isadigit{2}}{\isacharbrackright}{\kern0pt}\isanewline
\ \ \ \ \isacommand{unfolding}\isamarkupfalse%
\ bijection{\isacharunderscore}{\kern0pt}fm{\isacharunderscore}{\kern0pt}def\ injection{\isacharunderscore}{\kern0pt}fm{\isacharunderscore}{\kern0pt}def\ surjection{\isacharunderscore}{\kern0pt}fm{\isacharunderscore}{\kern0pt}def\isanewline
\ \ \ \ \ \isacommand{apply}\isamarkupfalse%
\ {\isacharparenleft}{\kern0pt}simp\ del{\isacharcolon}{\kern0pt}FOL{\isacharunderscore}{\kern0pt}sats{\isacharunderscore}{\kern0pt}iff\ pair{\isacharunderscore}{\kern0pt}abs\ add{\isacharcolon}{\kern0pt}\ fm{\isacharunderscore}{\kern0pt}defs\ nat{\isacharunderscore}{\kern0pt}simp{\isacharunderscore}{\kern0pt}union{\isacharparenright}{\kern0pt}\isanewline
\ \ \ \ \isacommand{apply}\isamarkupfalse%
{\isacharparenleft}{\kern0pt}rule\ M{\isacharunderscore}{\kern0pt}powerset{\isacharparenright}{\kern0pt}\isanewline
\ \ \ \ \isacommand{using}\isamarkupfalse%
\ cartprod{\isacharunderscore}{\kern0pt}closed\ nat{\isacharunderscore}{\kern0pt}in{\isacharunderscore}{\kern0pt}M\isanewline
\ \ \ \ \isacommand{by}\isamarkupfalse%
\ auto\isanewline
\isanewline
\ \ \isacommand{have}\isamarkupfalse%
\ {\isachardoublequoteopen}X\ {\isacharequal}{\kern0pt}\ nat{\isacharunderscore}{\kern0pt}perms{\isachardoublequoteclose}\isanewline
\ \ \ \ \isacommand{unfolding}\isamarkupfalse%
\ X{\isacharunderscore}{\kern0pt}def\ nat{\isacharunderscore}{\kern0pt}perms{\isacharunderscore}{\kern0pt}def\ \isanewline
\ \ \ \ \isacommand{using}\isamarkupfalse%
\ nat{\isacharunderscore}{\kern0pt}in{\isacharunderscore}{\kern0pt}M\ bij{\isacharunderscore}{\kern0pt}def\ inj{\isacharunderscore}{\kern0pt}def\ Pi{\isacharunderscore}{\kern0pt}def\ \isanewline
\ \ \ \ \isacommand{by}\isamarkupfalse%
\ auto\isanewline
\ \ \isacommand{then}\isamarkupfalse%
\ \isacommand{show}\isamarkupfalse%
\ {\isachardoublequoteopen}nat{\isacharunderscore}{\kern0pt}perms\ {\isasymin}\ M{\isachardoublequoteclose}\ \isacommand{using}\isamarkupfalse%
\ XinM\ \isacommand{by}\isamarkupfalse%
\ auto\isanewline
\isacommand{qed}\isamarkupfalse%
%
\endisatagproof
{\isafoldproof}%
%
\isadelimproof
\isanewline
%
\endisadelimproof
\isanewline
\isacommand{definition}\isamarkupfalse%
\ Fn{\isacharunderscore}{\kern0pt}perm\ \isakeyword{where}\ {\isachardoublequoteopen}Fn{\isacharunderscore}{\kern0pt}perm{\isacharparenleft}{\kern0pt}f{\isacharcomma}{\kern0pt}\ p{\isacharparenright}{\kern0pt}\ {\isasymequiv}\ {\isacharbraceleft}{\kern0pt}\ {\isacharless}{\kern0pt}{\isacharless}{\kern0pt}f{\isacharbackquote}{\kern0pt}n{\isacharcomma}{\kern0pt}\ m{\isachargreater}{\kern0pt}{\isacharcomma}{\kern0pt}\ l{\isachargreater}{\kern0pt}\ {\isachardot}{\kern0pt}\ {\isacharless}{\kern0pt}{\isacharless}{\kern0pt}n{\isacharcomma}{\kern0pt}\ m{\isachargreater}{\kern0pt}{\isacharcomma}{\kern0pt}\ l{\isachargreater}{\kern0pt}\ {\isasymin}\ p\ {\isacharbraceright}{\kern0pt}{\isachardoublequoteclose}\ \isanewline
\isacommand{definition}\isamarkupfalse%
\ Fn{\isacharunderscore}{\kern0pt}perm{\isacharprime}{\kern0pt}\ \isakeyword{where}\ {\isachardoublequoteopen}Fn{\isacharunderscore}{\kern0pt}perm{\isacharprime}{\kern0pt}{\isacharparenleft}{\kern0pt}f{\isacharparenright}{\kern0pt}\ {\isasymequiv}\ {\isacharbraceleft}{\kern0pt}\ {\isacharless}{\kern0pt}p{\isacharcomma}{\kern0pt}\ Fn{\isacharunderscore}{\kern0pt}perm{\isacharparenleft}{\kern0pt}f{\isacharcomma}{\kern0pt}\ p{\isacharparenright}{\kern0pt}{\isachargreater}{\kern0pt}{\isachardot}{\kern0pt}\ p\ {\isasymin}\ Fn\ {\isacharbraceright}{\kern0pt}{\isachardoublequoteclose}\ \isanewline
\isacommand{definition}\isamarkupfalse%
\ Fn{\isacharunderscore}{\kern0pt}perms\ \isakeyword{where}\ {\isachardoublequoteopen}Fn{\isacharunderscore}{\kern0pt}perms\ {\isasymequiv}\ {\isacharbraceleft}{\kern0pt}\ Fn{\isacharunderscore}{\kern0pt}perm{\isacharprime}{\kern0pt}{\isacharparenleft}{\kern0pt}f{\isacharparenright}{\kern0pt}{\isachardot}{\kern0pt}\ f\ {\isasymin}\ nat{\isacharunderscore}{\kern0pt}perms\ {\isacharbraceright}{\kern0pt}{\isachardoublequoteclose}\ \isanewline
\isanewline
\isacommand{lemma}\isamarkupfalse%
\ Fn{\isacharunderscore}{\kern0pt}permE\ {\isacharcolon}{\kern0pt}\ \isanewline
\ \ \isakeyword{fixes}\ f\ p\ b\ \isanewline
\ \ \isakeyword{assumes}\ {\isachardoublequoteopen}p\ {\isasymin}\ Fn{\isachardoublequoteclose}\ {\isachardoublequoteopen}f\ {\isasymin}\ nat{\isacharunderscore}{\kern0pt}perms{\isachardoublequoteclose}\ {\isachardoublequoteopen}b\ {\isasymin}\ Fn{\isacharunderscore}{\kern0pt}perm{\isacharparenleft}{\kern0pt}f{\isacharcomma}{\kern0pt}\ p{\isacharparenright}{\kern0pt}{\isachardoublequoteclose}\ \isanewline
\ \ \isakeyword{shows}\ {\isachardoublequoteopen}{\isasymexists}n\ {\isasymin}\ nat{\isachardot}{\kern0pt}\ {\isasymexists}m\ {\isasymin}\ nat{\isachardot}{\kern0pt}\ {\isasymexists}l\ {\isasymin}\ {\isadigit{2}}{\isachardot}{\kern0pt}\ {\isacharless}{\kern0pt}{\isacharless}{\kern0pt}n{\isacharcomma}{\kern0pt}\ m{\isachargreater}{\kern0pt}{\isacharcomma}{\kern0pt}\ l{\isachargreater}{\kern0pt}\ {\isasymin}\ p\ {\isasymand}\ b\ {\isacharequal}{\kern0pt}\ {\isacharless}{\kern0pt}{\isacharless}{\kern0pt}f{\isacharbackquote}{\kern0pt}n{\isacharcomma}{\kern0pt}\ m{\isachargreater}{\kern0pt}{\isacharcomma}{\kern0pt}\ l{\isachargreater}{\kern0pt}{\isachardoublequoteclose}\ \isanewline
%
\isadelimproof
%
\endisadelimproof
%
\isatagproof
\isacommand{proof}\isamarkupfalse%
\ {\isacharminus}{\kern0pt}\ \isanewline
\ \ \isacommand{have}\isamarkupfalse%
\ H{\isacharcolon}{\kern0pt}\ {\isachardoublequoteopen}p\ {\isasymsubseteq}\ {\isacharparenleft}{\kern0pt}nat\ {\isasymtimes}\ nat{\isacharparenright}{\kern0pt}\ {\isasymtimes}\ {\isadigit{2}}{\isachardoublequoteclose}\ \isanewline
\ \ \ \ \isacommand{using}\isamarkupfalse%
\ assms\ Fn{\isacharunderscore}{\kern0pt}def\ \isanewline
\ \ \ \ \isacommand{by}\isamarkupfalse%
\ auto\isanewline
\ \ \isacommand{have}\isamarkupfalse%
\ {\isachardoublequoteopen}{\isasymexists}v\ {\isasymin}\ p{\isachardot}{\kern0pt}\ b\ {\isacharequal}{\kern0pt}\ {\isasymlangle}{\isasymlangle}f\ {\isacharbackquote}{\kern0pt}\ fst{\isacharparenleft}{\kern0pt}fst{\isacharparenleft}{\kern0pt}v{\isacharparenright}{\kern0pt}{\isacharparenright}{\kern0pt}{\isacharcomma}{\kern0pt}\ snd{\isacharparenleft}{\kern0pt}fst{\isacharparenleft}{\kern0pt}v{\isacharparenright}{\kern0pt}{\isacharparenright}{\kern0pt}{\isasymrangle}{\isacharcomma}{\kern0pt}\ snd{\isacharparenleft}{\kern0pt}v{\isacharparenright}{\kern0pt}{\isasymrangle}{\isachardoublequoteclose}\isanewline
\ \ \ \ \isacommand{using}\isamarkupfalse%
\ assms\isanewline
\ \ \ \ \isacommand{unfolding}\isamarkupfalse%
\ Fn{\isacharunderscore}{\kern0pt}perm{\isacharunderscore}{\kern0pt}def\ split{\isacharunderscore}{\kern0pt}def\isanewline
\ \ \ \ \isacommand{by}\isamarkupfalse%
\ auto\isanewline
\ \ \isacommand{then}\isamarkupfalse%
\ \isacommand{obtain}\isamarkupfalse%
\ v\ \isakeyword{where}\ vH{\isacharcolon}{\kern0pt}\ {\isachardoublequoteopen}v\ {\isasymin}\ p{\isachardoublequoteclose}\ {\isachardoublequoteopen}b\ {\isacharequal}{\kern0pt}\ {\isasymlangle}{\isasymlangle}f\ {\isacharbackquote}{\kern0pt}\ fst{\isacharparenleft}{\kern0pt}fst{\isacharparenleft}{\kern0pt}v{\isacharparenright}{\kern0pt}{\isacharparenright}{\kern0pt}{\isacharcomma}{\kern0pt}\ snd{\isacharparenleft}{\kern0pt}fst{\isacharparenleft}{\kern0pt}v{\isacharparenright}{\kern0pt}{\isacharparenright}{\kern0pt}{\isasymrangle}{\isacharcomma}{\kern0pt}\ snd{\isacharparenleft}{\kern0pt}v{\isacharparenright}{\kern0pt}{\isasymrangle}{\isachardoublequoteclose}\ \isacommand{by}\isamarkupfalse%
\ auto\isanewline
\ \ \isacommand{then}\isamarkupfalse%
\ \isacommand{obtain}\isamarkupfalse%
\ x\ y\ z\ \isakeyword{where}\ veq{\isacharcolon}{\kern0pt}\ {\isachardoublequoteopen}v\ {\isacharequal}{\kern0pt}\ {\isacharless}{\kern0pt}{\isacharless}{\kern0pt}x{\isacharcomma}{\kern0pt}\ y{\isachargreater}{\kern0pt}{\isacharcomma}{\kern0pt}\ z{\isachargreater}{\kern0pt}{\isachardoublequoteclose}\ {\isachardoublequoteopen}x\ {\isasymin}\ nat{\isachardoublequoteclose}\ {\isachardoublequoteopen}y\ {\isasymin}\ nat{\isachardoublequoteclose}\ {\isachardoublequoteopen}z\ {\isasymin}\ {\isadigit{2}}{\isachardoublequoteclose}\ \isacommand{using}\isamarkupfalse%
\ H\ \isacommand{by}\isamarkupfalse%
\ blast\isanewline
\ \ \isacommand{then}\isamarkupfalse%
\ \isacommand{have}\isamarkupfalse%
\ beq{\isacharcolon}{\kern0pt}\ {\isachardoublequoteopen}b\ {\isacharequal}{\kern0pt}\ {\isacharless}{\kern0pt}{\isacharless}{\kern0pt}f{\isacharbackquote}{\kern0pt}x{\isacharcomma}{\kern0pt}\ y{\isachargreater}{\kern0pt}{\isacharcomma}{\kern0pt}\ z{\isachargreater}{\kern0pt}{\isachardoublequoteclose}\ \isacommand{using}\isamarkupfalse%
\ vH\ \isacommand{by}\isamarkupfalse%
\ auto\isanewline
\ \ \isacommand{have}\isamarkupfalse%
\ {\isachardoublequoteopen}p\ {\isasymin}\ M{\isachardoublequoteclose}\ \isacommand{using}\isamarkupfalse%
\ Fn{\isacharunderscore}{\kern0pt}in{\isacharunderscore}{\kern0pt}M\ transM\ assms\ \isacommand{by}\isamarkupfalse%
\ auto\isanewline
\ \ \isacommand{then}\isamarkupfalse%
\ \isacommand{have}\isamarkupfalse%
\ {\isachardoublequoteopen}v\ {\isasymin}\ M{\isachardoublequoteclose}\ \isacommand{using}\isamarkupfalse%
\ transM\ vH\ \isacommand{by}\isamarkupfalse%
\ auto\isanewline
\ \ \isacommand{then}\isamarkupfalse%
\ \isacommand{have}\isamarkupfalse%
\ {\isachardoublequoteopen}x\ {\isasymin}\ nat\ {\isasymand}\ y\ {\isasymin}\ nat\ {\isasymand}\ z\ {\isasymin}\ {\isadigit{2}}{\isachardoublequoteclose}\ \isacommand{using}\isamarkupfalse%
\ veq\ pair{\isacharunderscore}{\kern0pt}in{\isacharunderscore}{\kern0pt}M{\isacharunderscore}{\kern0pt}iff\ \isacommand{by}\isamarkupfalse%
\ auto\isanewline
\ \ \isacommand{then}\isamarkupfalse%
\ \isacommand{show}\isamarkupfalse%
\ {\isacharquery}{\kern0pt}thesis\ \isacommand{using}\isamarkupfalse%
\ beq\ veq\ vH\ \isacommand{by}\isamarkupfalse%
\ auto\isanewline
\isacommand{qed}\isamarkupfalse%
%
\endisatagproof
{\isafoldproof}%
%
\isadelimproof
\isanewline
%
\endisadelimproof
\isanewline
\isanewline
\isacommand{definition}\isamarkupfalse%
\ is{\isacharunderscore}{\kern0pt}Fn{\isacharunderscore}{\kern0pt}perm{\isacharunderscore}{\kern0pt}elem{\isacharunderscore}{\kern0pt}fm\ \isakeyword{where}\ \isanewline
\ \ {\isachardoublequoteopen}is{\isacharunderscore}{\kern0pt}Fn{\isacharunderscore}{\kern0pt}perm{\isacharunderscore}{\kern0pt}elem{\isacharunderscore}{\kern0pt}fm{\isacharparenleft}{\kern0pt}f{\isacharcomma}{\kern0pt}\ a{\isacharcomma}{\kern0pt}\ v{\isacharparenright}{\kern0pt}\ {\isasymequiv}\ \isanewline
\ \ \ \ Exists{\isacharparenleft}{\kern0pt}Exists{\isacharparenleft}{\kern0pt}Exists{\isacharparenleft}{\kern0pt}Exists{\isacharparenleft}{\kern0pt}Exists{\isacharparenleft}{\kern0pt}Exists{\isacharparenleft}{\kern0pt}\isanewline
\ \ \ \ \ \ And{\isacharparenleft}{\kern0pt}pair{\isacharunderscore}{\kern0pt}fm{\isacharparenleft}{\kern0pt}{\isadigit{0}}{\isacharcomma}{\kern0pt}\ {\isadigit{1}}{\isacharcomma}{\kern0pt}\ {\isadigit{3}}{\isacharparenright}{\kern0pt}{\isacharcomma}{\kern0pt}\ \isanewline
\ \ \ \ \ \ And{\isacharparenleft}{\kern0pt}pair{\isacharunderscore}{\kern0pt}fm{\isacharparenleft}{\kern0pt}{\isadigit{3}}{\isacharcomma}{\kern0pt}\ {\isadigit{2}}{\isacharcomma}{\kern0pt}\ a\ {\isacharhash}{\kern0pt}{\isacharplus}{\kern0pt}\ {\isadigit{6}}{\isacharparenright}{\kern0pt}{\isacharcomma}{\kern0pt}\ \isanewline
\ \ \ \ \ \ And{\isacharparenleft}{\kern0pt}pair{\isacharunderscore}{\kern0pt}fm{\isacharparenleft}{\kern0pt}{\isadigit{4}}{\isacharcomma}{\kern0pt}\ {\isadigit{1}}{\isacharcomma}{\kern0pt}\ {\isadigit{5}}{\isacharparenright}{\kern0pt}{\isacharcomma}{\kern0pt}\ \isanewline
\ \ \ \ \ \ And{\isacharparenleft}{\kern0pt}fun{\isacharunderscore}{\kern0pt}apply{\isacharunderscore}{\kern0pt}fm{\isacharparenleft}{\kern0pt}f\ {\isacharhash}{\kern0pt}{\isacharplus}{\kern0pt}\ {\isadigit{6}}{\isacharcomma}{\kern0pt}\ {\isadigit{0}}{\isacharcomma}{\kern0pt}\ {\isadigit{4}}{\isacharparenright}{\kern0pt}{\isacharcomma}{\kern0pt}\ \isanewline
\ \ \ \ \ \ \ \ \ \ pair{\isacharunderscore}{\kern0pt}fm{\isacharparenleft}{\kern0pt}{\isadigit{5}}{\isacharcomma}{\kern0pt}\ {\isadigit{2}}{\isacharcomma}{\kern0pt}\ v\ {\isacharhash}{\kern0pt}{\isacharplus}{\kern0pt}\ {\isadigit{6}}{\isacharparenright}{\kern0pt}{\isacharparenright}{\kern0pt}{\isacharparenright}{\kern0pt}{\isacharparenright}{\kern0pt}{\isacharparenright}{\kern0pt}{\isacharparenright}{\kern0pt}{\isacharparenright}{\kern0pt}{\isacharparenright}{\kern0pt}{\isacharparenright}{\kern0pt}{\isacharparenright}{\kern0pt}{\isacharparenright}{\kern0pt}{\isachardoublequoteclose}\ \isanewline
\isanewline
\isacommand{lemma}\isamarkupfalse%
\ is{\isacharunderscore}{\kern0pt}Fn{\isacharunderscore}{\kern0pt}perm{\isacharunderscore}{\kern0pt}elem{\isacharunderscore}{\kern0pt}fm{\isacharunderscore}{\kern0pt}type\ {\isacharcolon}{\kern0pt}\ \isanewline
\ \ \isakeyword{fixes}\ i\ j\ k\isanewline
\ \ \isakeyword{assumes}\ {\isachardoublequoteopen}i\ {\isasymin}\ nat{\isachardoublequoteclose}\ {\isachardoublequoteopen}j\ {\isasymin}\ nat{\isachardoublequoteclose}\ {\isachardoublequoteopen}k\ {\isasymin}\ nat{\isachardoublequoteclose}\ \isanewline
\ \ \isakeyword{shows}\ {\isachardoublequoteopen}is{\isacharunderscore}{\kern0pt}Fn{\isacharunderscore}{\kern0pt}perm{\isacharunderscore}{\kern0pt}elem{\isacharunderscore}{\kern0pt}fm{\isacharparenleft}{\kern0pt}i{\isacharcomma}{\kern0pt}\ j{\isacharcomma}{\kern0pt}\ k{\isacharparenright}{\kern0pt}\ {\isasymin}\ formula{\isachardoublequoteclose}\ \isanewline
%
\isadelimproof
\ \ %
\endisadelimproof
%
\isatagproof
\isacommand{unfolding}\isamarkupfalse%
\ is{\isacharunderscore}{\kern0pt}Fn{\isacharunderscore}{\kern0pt}perm{\isacharunderscore}{\kern0pt}elem{\isacharunderscore}{\kern0pt}fm{\isacharunderscore}{\kern0pt}def\ \isanewline
\ \ \isacommand{using}\isamarkupfalse%
\ assms\isanewline
\ \ \isacommand{by}\isamarkupfalse%
\ auto%
\endisatagproof
{\isafoldproof}%
%
\isadelimproof
\isanewline
%
\endisadelimproof
\isanewline
\isacommand{lemma}\isamarkupfalse%
\ arity{\isacharunderscore}{\kern0pt}is{\isacharunderscore}{\kern0pt}Fn{\isacharunderscore}{\kern0pt}perm{\isacharunderscore}{\kern0pt}elem{\isacharunderscore}{\kern0pt}fm\ {\isacharcolon}{\kern0pt}\ \isanewline
\ \ \isakeyword{fixes}\ i\ j\ k\isanewline
\ \ \isakeyword{assumes}\ {\isachardoublequoteopen}i\ {\isasymin}\ nat{\isachardoublequoteclose}\ {\isachardoublequoteopen}j\ {\isasymin}\ nat{\isachardoublequoteclose}\ {\isachardoublequoteopen}k\ {\isasymin}\ nat{\isachardoublequoteclose}\ \isanewline
\ \ \isakeyword{shows}\ {\isachardoublequoteopen}arity{\isacharparenleft}{\kern0pt}is{\isacharunderscore}{\kern0pt}Fn{\isacharunderscore}{\kern0pt}perm{\isacharunderscore}{\kern0pt}elem{\isacharunderscore}{\kern0pt}fm{\isacharparenleft}{\kern0pt}i{\isacharcomma}{\kern0pt}\ j{\isacharcomma}{\kern0pt}\ k{\isacharparenright}{\kern0pt}{\isacharparenright}{\kern0pt}\ {\isasymle}\ succ{\isacharparenleft}{\kern0pt}i{\isacharparenright}{\kern0pt}\ {\isasymunion}\ succ{\isacharparenleft}{\kern0pt}j{\isacharparenright}{\kern0pt}\ {\isasymunion}\ succ{\isacharparenleft}{\kern0pt}k{\isacharparenright}{\kern0pt}{\isachardoublequoteclose}\isanewline
%
\isadelimproof
\ \ %
\endisadelimproof
%
\isatagproof
\isacommand{unfolding}\isamarkupfalse%
\ is{\isacharunderscore}{\kern0pt}Fn{\isacharunderscore}{\kern0pt}perm{\isacharunderscore}{\kern0pt}elem{\isacharunderscore}{\kern0pt}fm{\isacharunderscore}{\kern0pt}def\isanewline
\ \ \isacommand{using}\isamarkupfalse%
\ assms\isanewline
\ \ \isacommand{apply}\isamarkupfalse%
\ simp\isanewline
\ \ \isacommand{apply}\isamarkupfalse%
{\isacharparenleft}{\kern0pt}rule\ pred{\isacharunderscore}{\kern0pt}le{\isacharcomma}{\kern0pt}\ simp{\isacharcomma}{\kern0pt}\ simp{\isacharparenright}{\kern0pt}{\isacharplus}{\kern0pt}\isanewline
\ \ \isacommand{apply}\isamarkupfalse%
{\isacharparenleft}{\kern0pt}subst\ arity{\isacharunderscore}{\kern0pt}fun{\isacharunderscore}{\kern0pt}apply{\isacharunderscore}{\kern0pt}fm{\isacharcomma}{\kern0pt}\ simp{\isacharcomma}{\kern0pt}\ simp{\isacharcomma}{\kern0pt}\ simp{\isacharparenright}{\kern0pt}\isanewline
\ \ \isacommand{apply}\isamarkupfalse%
{\isacharparenleft}{\kern0pt}subst\ arity{\isacharunderscore}{\kern0pt}pair{\isacharunderscore}{\kern0pt}fm{\isacharcomma}{\kern0pt}\ simp{\isacharcomma}{\kern0pt}\ simp{\isacharcomma}{\kern0pt}\ simp{\isacharparenright}{\kern0pt}{\isacharplus}{\kern0pt}\isanewline
\ \ \isacommand{apply}\isamarkupfalse%
\ simp\isanewline
\ \ \isacommand{apply}\isamarkupfalse%
{\isacharparenleft}{\kern0pt}rule\ Un{\isacharunderscore}{\kern0pt}least{\isacharunderscore}{\kern0pt}lt{\isacharparenright}{\kern0pt}{\isacharplus}{\kern0pt}\isanewline
\ \ \ \ \isacommand{apply}\isamarkupfalse%
\ simp\isanewline
\ \ \ \isacommand{apply}\isamarkupfalse%
{\isacharparenleft}{\kern0pt}rule\ Un{\isacharunderscore}{\kern0pt}least{\isacharunderscore}{\kern0pt}lt{\isacharparenright}{\kern0pt}{\isacharplus}{\kern0pt}\isanewline
\ \ \ \ \isacommand{apply}\isamarkupfalse%
\ auto{\isacharbrackleft}{\kern0pt}{\isadigit{2}}{\isacharbrackright}{\kern0pt}\isanewline
\ \ \isacommand{apply}\isamarkupfalse%
{\isacharparenleft}{\kern0pt}rule\ Un{\isacharunderscore}{\kern0pt}least{\isacharunderscore}{\kern0pt}lt{\isacharparenright}{\kern0pt}{\isacharplus}{\kern0pt}\isanewline
\ \ \ \ \isacommand{apply}\isamarkupfalse%
\ simp\isanewline
\ \ \ \isacommand{apply}\isamarkupfalse%
{\isacharparenleft}{\kern0pt}rule\ Un{\isacharunderscore}{\kern0pt}least{\isacharunderscore}{\kern0pt}lt{\isacharparenright}{\kern0pt}{\isacharplus}{\kern0pt}\isanewline
\ \ \ \ \isacommand{apply}\isamarkupfalse%
\ {\isacharparenleft}{\kern0pt}simp{\isacharcomma}{\kern0pt}\ simp{\isacharparenright}{\kern0pt}\isanewline
\ \ \ \isacommand{apply}\isamarkupfalse%
{\isacharparenleft}{\kern0pt}rule\ ltI{\isacharcomma}{\kern0pt}\ simp{\isacharcomma}{\kern0pt}\ simp{\isacharparenright}{\kern0pt}\isanewline
\ \ \isacommand{apply}\isamarkupfalse%
{\isacharparenleft}{\kern0pt}rule\ Un{\isacharunderscore}{\kern0pt}least{\isacharunderscore}{\kern0pt}lt{\isacharparenright}{\kern0pt}{\isacharplus}{\kern0pt}\isanewline
\ \ \ \ \isacommand{apply}\isamarkupfalse%
\ simp\isanewline
\ \ \ \isacommand{apply}\isamarkupfalse%
{\isacharparenleft}{\kern0pt}rule\ Un{\isacharunderscore}{\kern0pt}least{\isacharunderscore}{\kern0pt}lt{\isacharparenright}{\kern0pt}{\isacharplus}{\kern0pt}\isanewline
\ \ \ \ \isacommand{apply}\isamarkupfalse%
\ auto{\isacharbrackleft}{\kern0pt}{\isadigit{2}}{\isacharbrackright}{\kern0pt}\isanewline
\ \ \isacommand{apply}\isamarkupfalse%
{\isacharparenleft}{\kern0pt}rule\ Un{\isacharunderscore}{\kern0pt}least{\isacharunderscore}{\kern0pt}lt{\isacharparenright}{\kern0pt}{\isacharplus}{\kern0pt}\isanewline
\ \ \ \ \ \isacommand{apply}\isamarkupfalse%
\ simp\isanewline
\ \ \ \ \ \isacommand{apply}\isamarkupfalse%
{\isacharparenleft}{\kern0pt}rule\ ltI{\isacharparenright}{\kern0pt}\isanewline
\ \ \ \ \ \ \isacommand{apply}\isamarkupfalse%
\ auto{\isacharbrackleft}{\kern0pt}{\isadigit{4}}{\isacharbrackright}{\kern0pt}\isanewline
\ \ \isacommand{apply}\isamarkupfalse%
{\isacharparenleft}{\kern0pt}rule\ Un{\isacharunderscore}{\kern0pt}least{\isacharunderscore}{\kern0pt}lt{\isacharparenright}{\kern0pt}{\isacharplus}{\kern0pt}\isanewline
\ \ \ \isacommand{apply}\isamarkupfalse%
\ simp\isanewline
\ \ \isacommand{apply}\isamarkupfalse%
{\isacharparenleft}{\kern0pt}rule\ Un{\isacharunderscore}{\kern0pt}least{\isacharunderscore}{\kern0pt}lt{\isacharparenright}{\kern0pt}{\isacharplus}{\kern0pt}\isanewline
\ \ \ \isacommand{apply}\isamarkupfalse%
\ {\isacharparenleft}{\kern0pt}simp{\isacharcomma}{\kern0pt}\ simp{\isacharparenright}{\kern0pt}\isanewline
\ \ \isacommand{apply}\isamarkupfalse%
{\isacharparenleft}{\kern0pt}rule\ ltI{\isacharcomma}{\kern0pt}\ simp{\isacharcomma}{\kern0pt}\ simp{\isacharparenright}{\kern0pt}\isanewline
\ \ \isacommand{done}\isamarkupfalse%
%
\endisatagproof
{\isafoldproof}%
%
\isadelimproof
\isanewline
%
\endisadelimproof
\isanewline
\isacommand{lemma}\isamarkupfalse%
\ sats{\isacharunderscore}{\kern0pt}is{\isacharunderscore}{\kern0pt}Fn{\isacharunderscore}{\kern0pt}perm{\isacharunderscore}{\kern0pt}elem{\isacharunderscore}{\kern0pt}fm{\isacharunderscore}{\kern0pt}iff\ {\isacharcolon}{\kern0pt}\ \isanewline
\ \ \isakeyword{fixes}\ env\ i\ j\ k\ f\ a\ v\isanewline
\ \ \isakeyword{assumes}\ {\isachardoublequoteopen}env\ {\isasymin}\ list{\isacharparenleft}{\kern0pt}M{\isacharparenright}{\kern0pt}{\isachardoublequoteclose}\ {\isachardoublequoteopen}i\ {\isacharless}{\kern0pt}\ length{\isacharparenleft}{\kern0pt}env{\isacharparenright}{\kern0pt}{\isachardoublequoteclose}\ {\isachardoublequoteopen}j\ {\isacharless}{\kern0pt}\ length{\isacharparenleft}{\kern0pt}env{\isacharparenright}{\kern0pt}{\isachardoublequoteclose}\ {\isachardoublequoteopen}k\ {\isacharless}{\kern0pt}\ length{\isacharparenleft}{\kern0pt}env{\isacharparenright}{\kern0pt}{\isachardoublequoteclose}\isanewline
\ \ \ \ \ \ \ \ \ \ {\isachardoublequoteopen}nth{\isacharparenleft}{\kern0pt}i{\isacharcomma}{\kern0pt}\ env{\isacharparenright}{\kern0pt}\ {\isacharequal}{\kern0pt}\ f{\isachardoublequoteclose}\ {\isachardoublequoteopen}nth{\isacharparenleft}{\kern0pt}j{\isacharcomma}{\kern0pt}\ env{\isacharparenright}{\kern0pt}\ {\isacharequal}{\kern0pt}\ a{\isachardoublequoteclose}\ {\isachardoublequoteopen}nth{\isacharparenleft}{\kern0pt}k{\isacharcomma}{\kern0pt}\ env{\isacharparenright}{\kern0pt}\ {\isacharequal}{\kern0pt}\ v{\isachardoublequoteclose}\ \isanewline
\ \ \isakeyword{shows}\ {\isachardoublequoteopen}sats{\isacharparenleft}{\kern0pt}M{\isacharcomma}{\kern0pt}\ is{\isacharunderscore}{\kern0pt}Fn{\isacharunderscore}{\kern0pt}perm{\isacharunderscore}{\kern0pt}elem{\isacharunderscore}{\kern0pt}fm{\isacharparenleft}{\kern0pt}i{\isacharcomma}{\kern0pt}\ j{\isacharcomma}{\kern0pt}\ k{\isacharparenright}{\kern0pt}{\isacharcomma}{\kern0pt}\ env{\isacharparenright}{\kern0pt}\ {\isasymlongleftrightarrow}\ {\isacharparenleft}{\kern0pt}{\isasymexists}n\ m\ l{\isachardot}{\kern0pt}\ {\isacharless}{\kern0pt}{\isacharless}{\kern0pt}n{\isacharcomma}{\kern0pt}\ m{\isachargreater}{\kern0pt}{\isacharcomma}{\kern0pt}\ l{\isachargreater}{\kern0pt}\ {\isacharequal}{\kern0pt}\ a\ {\isasymand}\ v\ {\isacharequal}{\kern0pt}\ {\isacharless}{\kern0pt}{\isacharless}{\kern0pt}f{\isacharbackquote}{\kern0pt}n{\isacharcomma}{\kern0pt}\ m{\isachargreater}{\kern0pt}{\isacharcomma}{\kern0pt}\ l{\isachargreater}{\kern0pt}{\isacharparenright}{\kern0pt}{\isachardoublequoteclose}\ \isanewline
%
\isadelimproof
\ \ %
\endisadelimproof
%
\isatagproof
\isacommand{unfolding}\isamarkupfalse%
\ is{\isacharunderscore}{\kern0pt}Fn{\isacharunderscore}{\kern0pt}perm{\isacharunderscore}{\kern0pt}elem{\isacharunderscore}{\kern0pt}fm{\isacharunderscore}{\kern0pt}def\ \isanewline
\ \ \isacommand{using}\isamarkupfalse%
\ assms\ pair{\isacharunderscore}{\kern0pt}in{\isacharunderscore}{\kern0pt}M{\isacharunderscore}{\kern0pt}iff\ nth{\isacharunderscore}{\kern0pt}type\ lt{\isacharunderscore}{\kern0pt}nat{\isacharunderscore}{\kern0pt}in{\isacharunderscore}{\kern0pt}nat\ \ \isanewline
\ \ \isacommand{apply}\isamarkupfalse%
\ clarsimp\isanewline
\ \ \isacommand{apply}\isamarkupfalse%
{\isacharparenleft}{\kern0pt}rule\ iffI{\isacharcomma}{\kern0pt}\ clarsimp{\isacharcomma}{\kern0pt}\ clarsimp{\isacharparenright}{\kern0pt}\isanewline
\ \ \isacommand{apply}\isamarkupfalse%
{\isacharparenleft}{\kern0pt}rename{\isacharunderscore}{\kern0pt}tac\ n\ m\ l{\isacharcomma}{\kern0pt}\ subgoal{\isacharunderscore}{\kern0pt}tac\ {\isachardoublequoteopen}n\ {\isasymin}\ M\ {\isasymand}\ m\ {\isasymin}\ M\ {\isasymand}\ l\ {\isasymin}\ M{\isachardoublequoteclose}{\isacharparenright}{\kern0pt}\isanewline
\ \ \isacommand{apply}\isamarkupfalse%
{\isacharparenleft}{\kern0pt}rename{\isacharunderscore}{\kern0pt}tac\ n\ m\ l{\isacharcomma}{\kern0pt}\ rule{\isacharunderscore}{\kern0pt}tac\ x{\isacharequal}{\kern0pt}{\isachardoublequoteopen}{\isacharless}{\kern0pt}f{\isacharbackquote}{\kern0pt}n{\isacharcomma}{\kern0pt}\ m{\isachargreater}{\kern0pt}{\isachardoublequoteclose}\ \isakeyword{in}\ bexI{\isacharparenright}{\kern0pt}\isanewline
\ \ \ \isacommand{apply}\isamarkupfalse%
{\isacharparenleft}{\kern0pt}rename{\isacharunderscore}{\kern0pt}tac\ n\ m\ l{\isacharcomma}{\kern0pt}\ rule{\isacharunderscore}{\kern0pt}tac\ x{\isacharequal}{\kern0pt}{\isachardoublequoteopen}f{\isacharbackquote}{\kern0pt}n{\isachardoublequoteclose}\ \isakeyword{in}\ bexI{\isacharparenright}{\kern0pt}\isanewline
\ \ \ \ \isacommand{apply}\isamarkupfalse%
{\isacharparenleft}{\kern0pt}rename{\isacharunderscore}{\kern0pt}tac\ n\ m\ l{\isacharcomma}{\kern0pt}\ rule{\isacharunderscore}{\kern0pt}tac\ x{\isacharequal}{\kern0pt}{\isachardoublequoteopen}{\isacharless}{\kern0pt}n{\isacharcomma}{\kern0pt}\ m{\isachargreater}{\kern0pt}{\isachardoublequoteclose}\ \isakeyword{in}\ bexI{\isacharparenright}{\kern0pt}\isanewline
\ \ \ \ \ \isacommand{apply}\isamarkupfalse%
{\isacharparenleft}{\kern0pt}rename{\isacharunderscore}{\kern0pt}tac\ n\ m\ l{\isacharcomma}{\kern0pt}\ rule{\isacharunderscore}{\kern0pt}tac\ x{\isacharequal}{\kern0pt}{\isachardoublequoteopen}l{\isachardoublequoteclose}\ \isakeyword{in}\ bexI{\isacharparenright}{\kern0pt}\isanewline
\ \ \ \ \ \ \isacommand{apply}\isamarkupfalse%
{\isacharparenleft}{\kern0pt}rename{\isacharunderscore}{\kern0pt}tac\ n\ m\ l{\isacharcomma}{\kern0pt}\ rule{\isacharunderscore}{\kern0pt}tac\ x{\isacharequal}{\kern0pt}{\isachardoublequoteopen}m{\isachardoublequoteclose}\ \isakeyword{in}\ bexI{\isacharparenright}{\kern0pt}\isanewline
\ \ \ \ \ \ \ \ \isacommand{apply}\isamarkupfalse%
{\isacharparenleft}{\kern0pt}rename{\isacharunderscore}{\kern0pt}tac\ n\ m\ l{\isacharcomma}{\kern0pt}\ rule{\isacharunderscore}{\kern0pt}tac\ x{\isacharequal}{\kern0pt}{\isachardoublequoteopen}n{\isachardoublequoteclose}\ \isakeyword{in}\ bexI{\isacharparenright}{\kern0pt}\isanewline
\ \ \isacommand{using}\isamarkupfalse%
\ pair{\isacharunderscore}{\kern0pt}in{\isacharunderscore}{\kern0pt}M{\isacharunderscore}{\kern0pt}iff\ apply{\isacharunderscore}{\kern0pt}closed\isanewline
\ \ \ \ \ \ \ \ \ \isacommand{apply}\isamarkupfalse%
\ auto{\isacharbrackleft}{\kern0pt}{\isadigit{7}}{\isacharbrackright}{\kern0pt}\isanewline
\ \ \isacommand{apply}\isamarkupfalse%
{\isacharparenleft}{\kern0pt}rename{\isacharunderscore}{\kern0pt}tac\ n\ m\ l{\isacharcomma}{\kern0pt}\ subgoal{\isacharunderscore}{\kern0pt}tac\ {\isachardoublequoteopen}{\isacharparenleft}{\kern0pt}{\isacharhash}{\kern0pt}{\isacharhash}{\kern0pt}M{\isacharparenright}{\kern0pt}{\isacharparenleft}{\kern0pt}{\isacharless}{\kern0pt}n{\isacharcomma}{\kern0pt}\ m{\isachargreater}{\kern0pt}{\isacharparenright}{\kern0pt}\ {\isasymand}\ {\isacharparenleft}{\kern0pt}{\isacharhash}{\kern0pt}{\isacharhash}{\kern0pt}M{\isacharparenright}{\kern0pt}{\isacharparenleft}{\kern0pt}l{\isacharparenright}{\kern0pt}{\isachardoublequoteclose}{\isacharparenright}{\kern0pt}\isanewline
\ \ \isacommand{using}\isamarkupfalse%
\ pair{\isacharunderscore}{\kern0pt}in{\isacharunderscore}{\kern0pt}M{\isacharunderscore}{\kern0pt}iff\ nth{\isacharunderscore}{\kern0pt}type\ \isanewline
\ \ \ \isacommand{apply}\isamarkupfalse%
\ force\isanewline
\ \ \isacommand{apply}\isamarkupfalse%
{\isacharparenleft}{\kern0pt}rule\ iffD{\isadigit{1}}{\isacharcomma}{\kern0pt}\ rule\ pair{\isacharunderscore}{\kern0pt}in{\isacharunderscore}{\kern0pt}M{\isacharunderscore}{\kern0pt}iff{\isacharparenright}{\kern0pt}\isanewline
\ \ \isacommand{using}\isamarkupfalse%
\ nth{\isacharunderscore}{\kern0pt}type\ \isanewline
\ \ \isacommand{by}\isamarkupfalse%
\ auto%
\endisatagproof
{\isafoldproof}%
%
\isadelimproof
\isanewline
%
\endisadelimproof
\isanewline
\isacommand{lemma}\isamarkupfalse%
\ Fn{\isacharunderscore}{\kern0pt}perm{\isacharunderscore}{\kern0pt}in{\isacharunderscore}{\kern0pt}M\ {\isacharcolon}{\kern0pt}\ \isanewline
\ \ \isakeyword{fixes}\ f\ p\ \isanewline
\ \ \isakeyword{assumes}\ {\isachardoublequoteopen}f\ {\isasymin}\ nat{\isacharunderscore}{\kern0pt}perms{\isachardoublequoteclose}\ {\isachardoublequoteopen}p\ {\isasymin}\ Fn{\isachardoublequoteclose}\ \isanewline
\ \ \isakeyword{shows}\ {\isachardoublequoteopen}Fn{\isacharunderscore}{\kern0pt}perm{\isacharparenleft}{\kern0pt}f{\isacharcomma}{\kern0pt}\ p{\isacharparenright}{\kern0pt}\ {\isasymin}\ M{\isachardoublequoteclose}\ \isanewline
%
\isadelimproof
\isanewline
%
\endisadelimproof
%
\isatagproof
\isacommand{proof}\isamarkupfalse%
\ {\isacharminus}{\kern0pt}\ \isanewline
\ \ \isacommand{define}\isamarkupfalse%
\ Q\ \isakeyword{where}\ {\isachardoublequoteopen}Q\ {\isasymequiv}\ {\isasymlambda}x\ y{\isachardot}{\kern0pt}\ {\isasymexists}n\ m\ l{\isachardot}{\kern0pt}\ {\isacharless}{\kern0pt}{\isacharless}{\kern0pt}n{\isacharcomma}{\kern0pt}\ m{\isachargreater}{\kern0pt}{\isacharcomma}{\kern0pt}\ l{\isachargreater}{\kern0pt}\ {\isacharequal}{\kern0pt}\ x\ {\isasymand}\ y\ {\isacharequal}{\kern0pt}\ {\isacharless}{\kern0pt}{\isacharless}{\kern0pt}f{\isacharbackquote}{\kern0pt}n{\isacharcomma}{\kern0pt}\ m{\isachargreater}{\kern0pt}{\isacharcomma}{\kern0pt}\ l{\isachargreater}{\kern0pt}{\isachardoublequoteclose}\ \isanewline
\isanewline
\ \ \isacommand{have}\isamarkupfalse%
\ {\isachardoublequoteopen}strong{\isacharunderscore}{\kern0pt}replacement{\isacharparenleft}{\kern0pt}{\isacharhash}{\kern0pt}{\isacharhash}{\kern0pt}M{\isacharcomma}{\kern0pt}\ {\isasymlambda}x\ y{\isachardot}{\kern0pt}\ sats{\isacharparenleft}{\kern0pt}M{\isacharcomma}{\kern0pt}\ is{\isacharunderscore}{\kern0pt}Fn{\isacharunderscore}{\kern0pt}perm{\isacharunderscore}{\kern0pt}elem{\isacharunderscore}{\kern0pt}fm{\isacharparenleft}{\kern0pt}{\isadigit{2}}{\isacharcomma}{\kern0pt}\ {\isadigit{0}}{\isacharcomma}{\kern0pt}\ {\isadigit{1}}{\isacharparenright}{\kern0pt}{\isacharcomma}{\kern0pt}\ {\isacharbrackleft}{\kern0pt}x{\isacharcomma}{\kern0pt}\ y{\isacharbrackright}{\kern0pt}\ {\isacharat}{\kern0pt}\ {\isacharbrackleft}{\kern0pt}f{\isacharbrackright}{\kern0pt}{\isacharparenright}{\kern0pt}{\isacharparenright}{\kern0pt}{\isachardoublequoteclose}\ {\isacharparenleft}{\kern0pt}\isakeyword{is}\ {\isacharquery}{\kern0pt}A{\isacharparenright}{\kern0pt}\isanewline
\ \ \ \ \isacommand{apply}\isamarkupfalse%
{\isacharparenleft}{\kern0pt}rule\ replacement{\isacharunderscore}{\kern0pt}ax{\isacharparenright}{\kern0pt}\isanewline
\ \ \ \ \isacommand{apply}\isamarkupfalse%
{\isacharparenleft}{\kern0pt}rule\ is{\isacharunderscore}{\kern0pt}Fn{\isacharunderscore}{\kern0pt}perm{\isacharunderscore}{\kern0pt}elem{\isacharunderscore}{\kern0pt}fm{\isacharunderscore}{\kern0pt}type{\isacharparenright}{\kern0pt}\isanewline
\ \ \ \ \isacommand{using}\isamarkupfalse%
\ assms\ nat{\isacharunderscore}{\kern0pt}perms{\isacharunderscore}{\kern0pt}in{\isacharunderscore}{\kern0pt}M\ transM\isanewline
\ \ \ \ \ \ \ \ \isacommand{apply}\isamarkupfalse%
\ auto{\isacharbrackleft}{\kern0pt}{\isadigit{4}}{\isacharbrackright}{\kern0pt}\isanewline
\ \ \ \ \isacommand{apply}\isamarkupfalse%
{\isacharparenleft}{\kern0pt}rule\ le{\isacharunderscore}{\kern0pt}trans{\isacharcomma}{\kern0pt}\ rule\ arity{\isacharunderscore}{\kern0pt}is{\isacharunderscore}{\kern0pt}Fn{\isacharunderscore}{\kern0pt}perm{\isacharunderscore}{\kern0pt}elem{\isacharunderscore}{\kern0pt}fm{\isacharparenright}{\kern0pt}\isanewline
\ \ \ \ \isacommand{using}\isamarkupfalse%
\ Un{\isacharunderscore}{\kern0pt}least{\isacharunderscore}{\kern0pt}lt\ \isanewline
\ \ \ \ \isacommand{by}\isamarkupfalse%
\ auto\isanewline
\ \ \isacommand{have}\isamarkupfalse%
\ {\isachardoublequoteopen}strong{\isacharunderscore}{\kern0pt}replacement{\isacharparenleft}{\kern0pt}{\isacharhash}{\kern0pt}{\isacharhash}{\kern0pt}M{\isacharcomma}{\kern0pt}\ {\isasymlambda}x\ y{\isachardot}{\kern0pt}\ Q{\isacharparenleft}{\kern0pt}x{\isacharcomma}{\kern0pt}\ y{\isacharparenright}{\kern0pt}{\isacharparenright}{\kern0pt}{\isachardoublequoteclose}\ \isanewline
\ \ \ \ \isacommand{unfolding}\isamarkupfalse%
\ Q{\isacharunderscore}{\kern0pt}def\isanewline
\ \ \ \ \isacommand{apply}\isamarkupfalse%
{\isacharparenleft}{\kern0pt}rule{\isacharunderscore}{\kern0pt}tac\ P{\isacharequal}{\kern0pt}{\isachardoublequoteopen}{\isacharquery}{\kern0pt}A{\isachardoublequoteclose}\ \isakeyword{in}\ iffD{\isadigit{1}}{\isacharparenright}{\kern0pt}\isanewline
\ \ \ \ \ \isacommand{apply}\isamarkupfalse%
{\isacharparenleft}{\kern0pt}rule\ strong{\isacharunderscore}{\kern0pt}replacement{\isacharunderscore}{\kern0pt}cong{\isacharparenright}{\kern0pt}\isanewline
\ \ \ \ \ \isacommand{apply}\isamarkupfalse%
{\isacharparenleft}{\kern0pt}rule\ sats{\isacharunderscore}{\kern0pt}is{\isacharunderscore}{\kern0pt}Fn{\isacharunderscore}{\kern0pt}perm{\isacharunderscore}{\kern0pt}elem{\isacharunderscore}{\kern0pt}fm{\isacharunderscore}{\kern0pt}iff{\isacharparenright}{\kern0pt}\isanewline
\ \ \ \ \isacommand{using}\isamarkupfalse%
\ assms\ {\isacartoucheopen}{\isacharquery}{\kern0pt}A{\isacartoucheclose}\ Fn{\isacharunderscore}{\kern0pt}in{\isacharunderscore}{\kern0pt}M\ transM\ nat{\isacharunderscore}{\kern0pt}perms{\isacharunderscore}{\kern0pt}in{\isacharunderscore}{\kern0pt}M\isanewline
\ \ \ \ \isacommand{by}\isamarkupfalse%
\ auto\isanewline
\ \ \isacommand{then}\isamarkupfalse%
\ \isacommand{have}\isamarkupfalse%
\ {\isachardoublequoteopen}{\isasymexists}Y{\isacharbrackleft}{\kern0pt}{\isacharhash}{\kern0pt}{\isacharhash}{\kern0pt}M{\isacharbrackright}{\kern0pt}{\isachardot}{\kern0pt}\ {\isasymforall}b{\isacharbrackleft}{\kern0pt}{\isacharhash}{\kern0pt}{\isacharhash}{\kern0pt}M{\isacharbrackright}{\kern0pt}{\isachardot}{\kern0pt}\ b\ {\isasymin}\ Y\ {\isasymlongleftrightarrow}\ {\isacharparenleft}{\kern0pt}{\isasymexists}x{\isacharbrackleft}{\kern0pt}{\isacharhash}{\kern0pt}{\isacharhash}{\kern0pt}M{\isacharbrackright}{\kern0pt}{\isachardot}{\kern0pt}\ x\ {\isasymin}\ p\ {\isasymand}\ Q{\isacharparenleft}{\kern0pt}x{\isacharcomma}{\kern0pt}\ b{\isacharparenright}{\kern0pt}{\isacharparenright}{\kern0pt}{\isachardoublequoteclose}\isanewline
\ \ \ \ \isacommand{apply}\isamarkupfalse%
{\isacharparenleft}{\kern0pt}rule\ strong{\isacharunderscore}{\kern0pt}replacementD{\isacharparenright}{\kern0pt}\isanewline
\ \ \ \ \isacommand{using}\isamarkupfalse%
\ assms\ univalent{\isacharunderscore}{\kern0pt}def\ Q{\isacharunderscore}{\kern0pt}def\ Fn{\isacharunderscore}{\kern0pt}in{\isacharunderscore}{\kern0pt}M\ transM\isanewline
\ \ \ \ \isacommand{by}\isamarkupfalse%
\ auto\isanewline
\ \ \isacommand{then}\isamarkupfalse%
\ \isacommand{obtain}\isamarkupfalse%
\ Y\ \isakeyword{where}\ YH{\isacharcolon}{\kern0pt}\ {\isachardoublequoteopen}Y\ {\isasymin}\ M{\isachardoublequoteclose}\ {\isachardoublequoteopen}{\isasymAnd}b{\isachardot}{\kern0pt}\ b\ {\isasymin}\ M\ {\isasymLongrightarrow}\ b\ {\isasymin}\ Y\ {\isasymlongleftrightarrow}\ {\isacharparenleft}{\kern0pt}{\isasymexists}x\ {\isasymin}\ M{\isachardot}{\kern0pt}\ x\ {\isasymin}\ p\ {\isasymand}\ Q{\isacharparenleft}{\kern0pt}x{\isacharcomma}{\kern0pt}\ b{\isacharparenright}{\kern0pt}{\isacharparenright}{\kern0pt}{\isachardoublequoteclose}\ \isacommand{by}\isamarkupfalse%
\ auto\isanewline
\isanewline
\ \ \isacommand{have}\isamarkupfalse%
\ {\isachardoublequoteopen}Y\ {\isacharequal}{\kern0pt}\ Fn{\isacharunderscore}{\kern0pt}perm{\isacharparenleft}{\kern0pt}f{\isacharcomma}{\kern0pt}\ p{\isacharparenright}{\kern0pt}{\isachardoublequoteclose}\isanewline
\ \ \isacommand{proof}\isamarkupfalse%
\ {\isacharparenleft}{\kern0pt}rule\ equality{\isacharunderscore}{\kern0pt}iffI{\isacharcomma}{\kern0pt}\ rule\ iffI{\isacharparenright}{\kern0pt}\isanewline
\ \ \ \ \isacommand{fix}\isamarkupfalse%
\ b\ \isanewline
\ \ \ \ \isacommand{assume}\isamarkupfalse%
\ bin{\isacharcolon}{\kern0pt}\ {\isachardoublequoteopen}b\ {\isasymin}\ Y{\isachardoublequoteclose}\ \isanewline
\ \ \ \ \isacommand{then}\isamarkupfalse%
\ \isacommand{have}\isamarkupfalse%
\ {\isachardoublequoteopen}b\ {\isasymin}\ M{\isachardoublequoteclose}\ \isacommand{using}\isamarkupfalse%
\ YH\ transM\ \isacommand{by}\isamarkupfalse%
\ auto\isanewline
\ \ \ \ \isacommand{then}\isamarkupfalse%
\ \isacommand{have}\isamarkupfalse%
\ {\isachardoublequoteopen}{\isasymexists}x\ {\isasymin}\ M{\isachardot}{\kern0pt}\ x\ {\isasymin}\ p\ {\isasymand}\ Q{\isacharparenleft}{\kern0pt}x{\isacharcomma}{\kern0pt}\ b{\isacharparenright}{\kern0pt}{\isachardoublequoteclose}\ \isacommand{using}\isamarkupfalse%
\ YH\ bin\ \isacommand{by}\isamarkupfalse%
\ auto\isanewline
\ \ \ \ \isacommand{then}\isamarkupfalse%
\ \isacommand{obtain}\isamarkupfalse%
\ x\ \isakeyword{where}\ xH\ {\isacharcolon}{\kern0pt}\ {\isachardoublequoteopen}x\ {\isasymin}\ M{\isachardoublequoteclose}\ {\isachardoublequoteopen}x\ {\isasymin}\ p{\isachardoublequoteclose}\ {\isachardoublequoteopen}Q{\isacharparenleft}{\kern0pt}x{\isacharcomma}{\kern0pt}\ b{\isacharparenright}{\kern0pt}{\isachardoublequoteclose}\ \isacommand{by}\isamarkupfalse%
\ auto\isanewline
\ \ \ \ \isacommand{then}\isamarkupfalse%
\ \isacommand{have}\isamarkupfalse%
\ {\isachardoublequoteopen}{\isasymexists}n\ m\ l{\isachardot}{\kern0pt}\ {\isacharless}{\kern0pt}{\isacharless}{\kern0pt}n{\isacharcomma}{\kern0pt}\ m{\isachargreater}{\kern0pt}{\isacharcomma}{\kern0pt}\ l{\isachargreater}{\kern0pt}\ {\isacharequal}{\kern0pt}\ x\ {\isasymand}\ b\ {\isacharequal}{\kern0pt}\ {\isacharless}{\kern0pt}{\isacharless}{\kern0pt}f{\isacharbackquote}{\kern0pt}n{\isacharcomma}{\kern0pt}\ m{\isachargreater}{\kern0pt}{\isacharcomma}{\kern0pt}\ l{\isachargreater}{\kern0pt}{\isachardoublequoteclose}\ \isacommand{unfolding}\isamarkupfalse%
\ Q{\isacharunderscore}{\kern0pt}def\ \isacommand{by}\isamarkupfalse%
\ auto\isanewline
\ \ \ \ \isacommand{then}\isamarkupfalse%
\ \isacommand{obtain}\isamarkupfalse%
\ n\ m\ l\ \isakeyword{where}\ H{\isacharcolon}{\kern0pt}\ {\isachardoublequoteopen}{\isacharless}{\kern0pt}{\isacharless}{\kern0pt}n{\isacharcomma}{\kern0pt}\ m{\isachargreater}{\kern0pt}{\isacharcomma}{\kern0pt}\ l{\isachargreater}{\kern0pt}\ {\isacharequal}{\kern0pt}\ x{\isachardoublequoteclose}\ {\isachardoublequoteopen}b\ {\isacharequal}{\kern0pt}\ {\isacharless}{\kern0pt}{\isacharless}{\kern0pt}f{\isacharbackquote}{\kern0pt}n{\isacharcomma}{\kern0pt}\ m{\isachargreater}{\kern0pt}{\isacharcomma}{\kern0pt}\ l{\isachargreater}{\kern0pt}{\isachardoublequoteclose}\ \isacommand{by}\isamarkupfalse%
\ auto\isanewline
\ \ \ \ \isacommand{then}\isamarkupfalse%
\ \isacommand{show}\isamarkupfalse%
\ {\isachardoublequoteopen}b\ {\isasymin}\ Fn{\isacharunderscore}{\kern0pt}perm{\isacharparenleft}{\kern0pt}f{\isacharcomma}{\kern0pt}\ p{\isacharparenright}{\kern0pt}{\isachardoublequoteclose}\isanewline
\ \ \ \ \ \ \isacommand{unfolding}\isamarkupfalse%
\ Fn{\isacharunderscore}{\kern0pt}perm{\isacharunderscore}{\kern0pt}def\ \isanewline
\ \ \ \ \ \ \isacommand{apply}\isamarkupfalse%
\ clarsimp\isanewline
\ \ \ \ \ \ \isacommand{apply}\isamarkupfalse%
{\isacharparenleft}{\kern0pt}rule{\isacharunderscore}{\kern0pt}tac\ x{\isacharequal}{\kern0pt}{\isachardoublequoteopen}{\isacharless}{\kern0pt}{\isacharless}{\kern0pt}n{\isacharcomma}{\kern0pt}\ m{\isachargreater}{\kern0pt}{\isacharcomma}{\kern0pt}\ l{\isachargreater}{\kern0pt}{\isachardoublequoteclose}\ \isakeyword{in}\ bexI{\isacharparenright}{\kern0pt}\isanewline
\ \ \ \ \ \ \isacommand{using}\isamarkupfalse%
\ xH\ H\isanewline
\ \ \ \ \ \ \isacommand{by}\isamarkupfalse%
\ auto\isanewline
\ \ \isacommand{next}\isamarkupfalse%
\ \isanewline
\ \ \ \ \isacommand{fix}\isamarkupfalse%
\ b\ \isanewline
\ \ \ \ \isacommand{assume}\isamarkupfalse%
\ bin\ {\isacharcolon}{\kern0pt}\ {\isachardoublequoteopen}b\ {\isasymin}\ Fn{\isacharunderscore}{\kern0pt}perm{\isacharparenleft}{\kern0pt}f{\isacharcomma}{\kern0pt}\ p{\isacharparenright}{\kern0pt}{\isachardoublequoteclose}\isanewline
\ \ \ \ \isacommand{then}\isamarkupfalse%
\ \isacommand{obtain}\isamarkupfalse%
\ n\ m\ l\ \isakeyword{where}\ H{\isacharcolon}{\kern0pt}\ {\isachardoublequoteopen}b\ {\isacharequal}{\kern0pt}\ {\isacharless}{\kern0pt}{\isacharless}{\kern0pt}f{\isacharbackquote}{\kern0pt}n{\isacharcomma}{\kern0pt}\ m{\isachargreater}{\kern0pt}{\isacharcomma}{\kern0pt}\ l{\isachargreater}{\kern0pt}{\isachardoublequoteclose}\ {\isachardoublequoteopen}{\isacharless}{\kern0pt}{\isacharless}{\kern0pt}n{\isacharcomma}{\kern0pt}\ m{\isachargreater}{\kern0pt}{\isacharcomma}{\kern0pt}\ l{\isachargreater}{\kern0pt}\ {\isasymin}\ p{\isachardoublequoteclose}\ {\isachardoublequoteopen}n\ {\isasymin}\ nat{\isachardoublequoteclose}\ {\isachardoublequoteopen}m\ {\isasymin}\ nat{\isachardoublequoteclose}\ {\isachardoublequoteopen}l\ {\isasymin}\ {\isadigit{2}}{\isachardoublequoteclose}\ \ \isacommand{using}\isamarkupfalse%
\ Fn{\isacharunderscore}{\kern0pt}permE\ assms\ \isacommand{by}\isamarkupfalse%
\ blast\isanewline
\ \ \ \ \isacommand{then}\isamarkupfalse%
\ \isacommand{have}\isamarkupfalse%
\ {\isachardoublequoteopen}Q{\isacharparenleft}{\kern0pt}{\isacharless}{\kern0pt}{\isacharless}{\kern0pt}n{\isacharcomma}{\kern0pt}\ m{\isachargreater}{\kern0pt}{\isacharcomma}{\kern0pt}\ l{\isachargreater}{\kern0pt}{\isacharcomma}{\kern0pt}\ b{\isacharparenright}{\kern0pt}{\isachardoublequoteclose}\isanewline
\ \ \ \ \ \ \isacommand{unfolding}\isamarkupfalse%
\ Q{\isacharunderscore}{\kern0pt}def\ \isanewline
\ \ \ \ \ \ \isacommand{by}\isamarkupfalse%
\ auto\isanewline
\ \ \ \ \isacommand{then}\isamarkupfalse%
\ \isacommand{have}\isamarkupfalse%
\ {\isachardoublequoteopen}{\isasymexists}x\ {\isasymin}\ M{\isachardot}{\kern0pt}\ x\ {\isasymin}\ p\ {\isasymand}\ Q{\isacharparenleft}{\kern0pt}x{\isacharcomma}{\kern0pt}\ b{\isacharparenright}{\kern0pt}{\isachardoublequoteclose}\ {\isacharparenleft}{\kern0pt}\isakeyword{is}\ {\isacharquery}{\kern0pt}A{\isacharparenright}{\kern0pt}\isanewline
\ \ \ \ \ \ \isacommand{apply}\isamarkupfalse%
{\isacharparenleft}{\kern0pt}rule{\isacharunderscore}{\kern0pt}tac\ x{\isacharequal}{\kern0pt}{\isachardoublequoteopen}{\isacharless}{\kern0pt}{\isacharless}{\kern0pt}n{\isacharcomma}{\kern0pt}\ m{\isachargreater}{\kern0pt}{\isacharcomma}{\kern0pt}\ l{\isachargreater}{\kern0pt}{\isachardoublequoteclose}\ \isakeyword{in}\ bexI{\isacharparenright}{\kern0pt}\isanewline
\ \ \ \ \ \ \isacommand{using}\isamarkupfalse%
\ H\ \isanewline
\ \ \ \ \ \ \ \isacommand{apply}\isamarkupfalse%
\ force\isanewline
\ \ \ \ \ \ \isacommand{apply}\isamarkupfalse%
{\isacharparenleft}{\kern0pt}rule\ to{\isacharunderscore}{\kern0pt}rin{\isacharcomma}{\kern0pt}\ rule\ pair{\isacharunderscore}{\kern0pt}in{\isacharunderscore}{\kern0pt}MI{\isacharcomma}{\kern0pt}\ rule\ conjI{\isacharcomma}{\kern0pt}\ rule\ pair{\isacharunderscore}{\kern0pt}in{\isacharunderscore}{\kern0pt}MI{\isacharparenright}{\kern0pt}\isanewline
\ \ \ \ \ \ \isacommand{using}\isamarkupfalse%
\ H\ nat{\isacharunderscore}{\kern0pt}in{\isacharunderscore}{\kern0pt}M\ zero{\isacharunderscore}{\kern0pt}in{\isacharunderscore}{\kern0pt}M\ succ{\isacharunderscore}{\kern0pt}in{\isacharunderscore}{\kern0pt}MI\ transM\isanewline
\ \ \ \ \ \ \isacommand{by}\isamarkupfalse%
\ auto\isanewline
\ \ \ \ \isacommand{have}\isamarkupfalse%
\ {\isachardoublequoteopen}{\isacharless}{\kern0pt}{\isacharless}{\kern0pt}f{\isacharbackquote}{\kern0pt}n{\isacharcomma}{\kern0pt}\ m{\isachargreater}{\kern0pt}{\isacharcomma}{\kern0pt}\ l{\isachargreater}{\kern0pt}\ {\isasymin}\ M{\isachardoublequoteclose}\ \isanewline
\ \ \ \ \ \ \isacommand{apply}\isamarkupfalse%
{\isacharparenleft}{\kern0pt}rule\ to{\isacharunderscore}{\kern0pt}rin{\isacharcomma}{\kern0pt}\ rule\ pair{\isacharunderscore}{\kern0pt}in{\isacharunderscore}{\kern0pt}MI{\isacharcomma}{\kern0pt}\ rule\ conjI{\isacharcomma}{\kern0pt}\ rule\ pair{\isacharunderscore}{\kern0pt}in{\isacharunderscore}{\kern0pt}MI{\isacharcomma}{\kern0pt}\ rule\ conjI{\isacharparenright}{\kern0pt}\isanewline
\ \ \ \ \ \ \ \ \isacommand{apply}\isamarkupfalse%
{\isacharparenleft}{\kern0pt}rule\ apply{\isacharunderscore}{\kern0pt}closed{\isacharparenright}{\kern0pt}\isanewline
\ \ \ \ \ \ \isacommand{using}\isamarkupfalse%
\ H\ nat{\isacharunderscore}{\kern0pt}in{\isacharunderscore}{\kern0pt}M\ zero{\isacharunderscore}{\kern0pt}in{\isacharunderscore}{\kern0pt}M\ succ{\isacharunderscore}{\kern0pt}in{\isacharunderscore}{\kern0pt}MI\ transM\ assms\ nat{\isacharunderscore}{\kern0pt}perms{\isacharunderscore}{\kern0pt}in{\isacharunderscore}{\kern0pt}M\isanewline
\ \ \ \ \ \ \isacommand{by}\isamarkupfalse%
\ auto\isanewline
\ \ \ \ \isacommand{then}\isamarkupfalse%
\ \isacommand{show}\isamarkupfalse%
\ {\isachardoublequoteopen}b\ {\isasymin}\ Y{\isachardoublequoteclose}\ \isacommand{using}\isamarkupfalse%
\ YH\ {\isacartoucheopen}{\isacharquery}{\kern0pt}A{\isacartoucheclose}\ H\ \isacommand{by}\isamarkupfalse%
\ auto\isanewline
\ \ \isacommand{qed}\isamarkupfalse%
\isanewline
\isanewline
\ \ \isacommand{then}\isamarkupfalse%
\ \isacommand{show}\isamarkupfalse%
\ {\isacharquery}{\kern0pt}thesis\ \isacommand{using}\isamarkupfalse%
\ YH\ \isacommand{by}\isamarkupfalse%
\ auto\isanewline
\isacommand{qed}\isamarkupfalse%
%
\endisatagproof
{\isafoldproof}%
%
\isadelimproof
\isanewline
%
\endisadelimproof
\isanewline
\isacommand{definition}\isamarkupfalse%
\ is{\isacharunderscore}{\kern0pt}Fn{\isacharunderscore}{\kern0pt}perm{\isacharunderscore}{\kern0pt}fm\ \isakeyword{where}\ \isanewline
\ \ {\isachardoublequoteopen}is{\isacharunderscore}{\kern0pt}Fn{\isacharunderscore}{\kern0pt}perm{\isacharunderscore}{\kern0pt}fm{\isacharparenleft}{\kern0pt}f{\isacharcomma}{\kern0pt}\ p{\isacharcomma}{\kern0pt}\ v{\isacharparenright}{\kern0pt}\ {\isasymequiv}\ Forall{\isacharparenleft}{\kern0pt}Iff{\isacharparenleft}{\kern0pt}Member{\isacharparenleft}{\kern0pt}{\isadigit{0}}{\isacharcomma}{\kern0pt}\ v{\isacharhash}{\kern0pt}{\isacharplus}{\kern0pt}{\isadigit{1}}{\isacharparenright}{\kern0pt}{\isacharcomma}{\kern0pt}\ Exists{\isacharparenleft}{\kern0pt}And{\isacharparenleft}{\kern0pt}Member{\isacharparenleft}{\kern0pt}{\isadigit{0}}{\isacharcomma}{\kern0pt}p{\isacharhash}{\kern0pt}{\isacharplus}{\kern0pt}{\isadigit{2}}{\isacharparenright}{\kern0pt}{\isacharcomma}{\kern0pt}is{\isacharunderscore}{\kern0pt}Fn{\isacharunderscore}{\kern0pt}perm{\isacharunderscore}{\kern0pt}elem{\isacharunderscore}{\kern0pt}fm{\isacharparenleft}{\kern0pt}f{\isacharhash}{\kern0pt}{\isacharplus}{\kern0pt}{\isadigit{2}}{\isacharcomma}{\kern0pt}\ {\isadigit{0}}{\isacharcomma}{\kern0pt}\ {\isadigit{1}}{\isacharparenright}{\kern0pt}{\isacharparenright}{\kern0pt}{\isacharparenright}{\kern0pt}{\isacharparenright}{\kern0pt}{\isacharparenright}{\kern0pt}{\isachardoublequoteclose}\ \isanewline
\isanewline
\isacommand{lemma}\isamarkupfalse%
\ is{\isacharunderscore}{\kern0pt}Fn{\isacharunderscore}{\kern0pt}perm{\isacharunderscore}{\kern0pt}fm{\isacharunderscore}{\kern0pt}type\ {\isacharcolon}{\kern0pt}\ \isanewline
\ \ \isakeyword{fixes}\ i\ j\ k\ \isanewline
\ \ \isakeyword{assumes}\ {\isachardoublequoteopen}i\ {\isasymin}\ nat{\isachardoublequoteclose}\ {\isachardoublequoteopen}j\ {\isasymin}\ nat{\isachardoublequoteclose}\ {\isachardoublequoteopen}k\ {\isasymin}\ nat{\isachardoublequoteclose}\ \isanewline
\ \ \isakeyword{shows}\ {\isachardoublequoteopen}is{\isacharunderscore}{\kern0pt}Fn{\isacharunderscore}{\kern0pt}perm{\isacharunderscore}{\kern0pt}fm{\isacharparenleft}{\kern0pt}i{\isacharcomma}{\kern0pt}\ j{\isacharcomma}{\kern0pt}\ k{\isacharparenright}{\kern0pt}\ {\isasymin}\ formula{\isachardoublequoteclose}\ \isanewline
%
\isadelimproof
\ \ %
\endisadelimproof
%
\isatagproof
\isacommand{unfolding}\isamarkupfalse%
\ is{\isacharunderscore}{\kern0pt}Fn{\isacharunderscore}{\kern0pt}perm{\isacharunderscore}{\kern0pt}fm{\isacharunderscore}{\kern0pt}def\isanewline
\ \ \isacommand{apply}\isamarkupfalse%
{\isacharparenleft}{\kern0pt}subgoal{\isacharunderscore}{\kern0pt}tac\ {\isachardoublequoteopen}is{\isacharunderscore}{\kern0pt}Fn{\isacharunderscore}{\kern0pt}perm{\isacharunderscore}{\kern0pt}elem{\isacharunderscore}{\kern0pt}fm{\isacharparenleft}{\kern0pt}i\ {\isacharhash}{\kern0pt}{\isacharplus}{\kern0pt}\ {\isadigit{2}}{\isacharcomma}{\kern0pt}\ {\isadigit{0}}{\isacharcomma}{\kern0pt}\ {\isadigit{1}}{\isacharparenright}{\kern0pt}\ {\isasymin}\ formula{\isachardoublequoteclose}{\isacharcomma}{\kern0pt}\ force{\isacharparenright}{\kern0pt}\isanewline
\ \ \isacommand{using}\isamarkupfalse%
\ is{\isacharunderscore}{\kern0pt}Fn{\isacharunderscore}{\kern0pt}perm{\isacharunderscore}{\kern0pt}elem{\isacharunderscore}{\kern0pt}fm{\isacharunderscore}{\kern0pt}type\ assms\isanewline
\ \ \isacommand{by}\isamarkupfalse%
\ force%
\endisatagproof
{\isafoldproof}%
%
\isadelimproof
\isanewline
%
\endisadelimproof
\isanewline
\isacommand{lemma}\isamarkupfalse%
\ arity{\isacharunderscore}{\kern0pt}is{\isacharunderscore}{\kern0pt}Fn{\isacharunderscore}{\kern0pt}perm{\isacharunderscore}{\kern0pt}fm\ {\isacharcolon}{\kern0pt}\ \isanewline
\ \ \isakeyword{fixes}\ i\ j\ k\ \isanewline
\ \ \isakeyword{assumes}\ {\isachardoublequoteopen}i\ {\isasymin}\ nat{\isachardoublequoteclose}\ {\isachardoublequoteopen}j\ {\isasymin}\ nat{\isachardoublequoteclose}\ {\isachardoublequoteopen}k\ {\isasymin}\ nat{\isachardoublequoteclose}\ \isanewline
\ \ \isakeyword{shows}\ {\isachardoublequoteopen}arity{\isacharparenleft}{\kern0pt}is{\isacharunderscore}{\kern0pt}Fn{\isacharunderscore}{\kern0pt}perm{\isacharunderscore}{\kern0pt}fm{\isacharparenleft}{\kern0pt}i{\isacharcomma}{\kern0pt}\ j{\isacharcomma}{\kern0pt}\ k{\isacharparenright}{\kern0pt}{\isacharparenright}{\kern0pt}\ {\isasymle}\ succ{\isacharparenleft}{\kern0pt}i{\isacharparenright}{\kern0pt}\ {\isasymunion}\ succ{\isacharparenleft}{\kern0pt}j{\isacharparenright}{\kern0pt}\ {\isasymunion}\ succ{\isacharparenleft}{\kern0pt}k{\isacharparenright}{\kern0pt}{\isachardoublequoteclose}\isanewline
%
\isadelimproof
\ \ %
\endisadelimproof
%
\isatagproof
\isacommand{unfolding}\isamarkupfalse%
\ is{\isacharunderscore}{\kern0pt}Fn{\isacharunderscore}{\kern0pt}perm{\isacharunderscore}{\kern0pt}fm{\isacharunderscore}{\kern0pt}def\ \isanewline
\ \ \isacommand{using}\isamarkupfalse%
\ assms\isanewline
\ \ \isacommand{apply}\isamarkupfalse%
{\isacharparenleft}{\kern0pt}subgoal{\isacharunderscore}{\kern0pt}tac\ {\isachardoublequoteopen}is{\isacharunderscore}{\kern0pt}Fn{\isacharunderscore}{\kern0pt}perm{\isacharunderscore}{\kern0pt}elem{\isacharunderscore}{\kern0pt}fm{\isacharparenleft}{\kern0pt}i\ {\isacharhash}{\kern0pt}{\isacharplus}{\kern0pt}\ {\isadigit{2}}{\isacharcomma}{\kern0pt}\ {\isadigit{0}}{\isacharcomma}{\kern0pt}\ {\isadigit{1}}{\isacharparenright}{\kern0pt}\ {\isasymin}\ formula{\isachardoublequoteclose}{\isacharparenright}{\kern0pt}\isanewline
\ \ \isacommand{apply}\isamarkupfalse%
\ simp\isanewline
\ \ \ \isacommand{apply}\isamarkupfalse%
{\isacharparenleft}{\kern0pt}rule\ pred{\isacharunderscore}{\kern0pt}le{\isacharcomma}{\kern0pt}\ simp{\isacharcomma}{\kern0pt}\ simp{\isacharparenright}{\kern0pt}\isanewline
\ \ \ \isacommand{apply}\isamarkupfalse%
{\isacharparenleft}{\kern0pt}rule\ Un{\isacharunderscore}{\kern0pt}least{\isacharunderscore}{\kern0pt}lt{\isacharparenright}{\kern0pt}{\isacharplus}{\kern0pt}\isanewline
\ \ \ \ \ \isacommand{apply}\isamarkupfalse%
\ auto{\isacharbrackleft}{\kern0pt}{\isadigit{1}}{\isacharbrackright}{\kern0pt}\isanewline
\ \ \ \ \isacommand{apply}\isamarkupfalse%
\ simp\isanewline
\ \ \ \ \isacommand{apply}\isamarkupfalse%
{\isacharparenleft}{\kern0pt}rule\ ltI{\isacharcomma}{\kern0pt}\ simp{\isacharcomma}{\kern0pt}\ simp{\isacharparenright}{\kern0pt}\isanewline
\ \ \ \isacommand{apply}\isamarkupfalse%
{\isacharparenleft}{\kern0pt}rule\ pred{\isacharunderscore}{\kern0pt}le{\isacharcomma}{\kern0pt}\ simp{\isacharcomma}{\kern0pt}\ simp{\isacharparenright}{\kern0pt}\isanewline
\ \ \ \isacommand{apply}\isamarkupfalse%
{\isacharparenleft}{\kern0pt}rule\ Un{\isacharunderscore}{\kern0pt}least{\isacharunderscore}{\kern0pt}lt{\isacharparenright}{\kern0pt}{\isacharplus}{\kern0pt}\isanewline
\ \ \ \ \ \isacommand{apply}\isamarkupfalse%
\ {\isacharparenleft}{\kern0pt}simp{\isacharcomma}{\kern0pt}\ simp{\isacharparenright}{\kern0pt}\isanewline
\ \ \ \ \isacommand{apply}\isamarkupfalse%
{\isacharparenleft}{\kern0pt}rule\ ltI{\isacharcomma}{\kern0pt}\ simp{\isacharcomma}{\kern0pt}\ simp{\isacharparenright}{\kern0pt}\isanewline
\ \ \ \isacommand{apply}\isamarkupfalse%
{\isacharparenleft}{\kern0pt}rule\ le{\isacharunderscore}{\kern0pt}trans{\isacharcomma}{\kern0pt}\ rule\ arity{\isacharunderscore}{\kern0pt}is{\isacharunderscore}{\kern0pt}Fn{\isacharunderscore}{\kern0pt}perm{\isacharunderscore}{\kern0pt}elem{\isacharunderscore}{\kern0pt}fm{\isacharparenright}{\kern0pt}\isanewline
\ \ \ \ \ \ \isacommand{apply}\isamarkupfalse%
\ auto{\isacharbrackleft}{\kern0pt}{\isadigit{3}}{\isacharbrackright}{\kern0pt}\isanewline
\ \ \ \isacommand{apply}\isamarkupfalse%
{\isacharparenleft}{\kern0pt}rule\ Un{\isacharunderscore}{\kern0pt}least{\isacharunderscore}{\kern0pt}lt{\isacharparenright}{\kern0pt}{\isacharplus}{\kern0pt}\isanewline
\ \ \ \ \ \isacommand{apply}\isamarkupfalse%
\ simp\isanewline
\ \ \ \ \ \isacommand{apply}\isamarkupfalse%
{\isacharparenleft}{\kern0pt}rule\ ltI{\isacharcomma}{\kern0pt}\ simp{\isacharcomma}{\kern0pt}\ simp{\isacharcomma}{\kern0pt}\ simp{\isacharcomma}{\kern0pt}\ simp{\isacharparenright}{\kern0pt}\isanewline
\ \ \isacommand{using}\isamarkupfalse%
\ is{\isacharunderscore}{\kern0pt}Fn{\isacharunderscore}{\kern0pt}perm{\isacharunderscore}{\kern0pt}elem{\isacharunderscore}{\kern0pt}fm{\isacharunderscore}{\kern0pt}type\ assms\isanewline
\ \ \isacommand{by}\isamarkupfalse%
\ force%
\endisatagproof
{\isafoldproof}%
%
\isadelimproof
\isanewline
%
\endisadelimproof
\isanewline
\isacommand{lemma}\isamarkupfalse%
\ sats{\isacharunderscore}{\kern0pt}is{\isacharunderscore}{\kern0pt}Fn{\isacharunderscore}{\kern0pt}perm{\isacharunderscore}{\kern0pt}fm{\isacharunderscore}{\kern0pt}iff\ {\isacharcolon}{\kern0pt}\ \isanewline
\ \ \isakeyword{fixes}\ env\ i\ j\ k\ f\ p\ v\ \isanewline
\ \ \isakeyword{assumes}\ {\isachardoublequoteopen}env\ {\isasymin}\ list{\isacharparenleft}{\kern0pt}M{\isacharparenright}{\kern0pt}{\isachardoublequoteclose}\ {\isachardoublequoteopen}i\ {\isacharless}{\kern0pt}\ length{\isacharparenleft}{\kern0pt}env{\isacharparenright}{\kern0pt}{\isachardoublequoteclose}\ {\isachardoublequoteopen}j\ {\isacharless}{\kern0pt}\ length{\isacharparenleft}{\kern0pt}env{\isacharparenright}{\kern0pt}{\isachardoublequoteclose}\ {\isachardoublequoteopen}k\ {\isacharless}{\kern0pt}\ length{\isacharparenleft}{\kern0pt}env{\isacharparenright}{\kern0pt}{\isachardoublequoteclose}\ \isanewline
\ \ \ \ \ \ \ \ \ \ {\isachardoublequoteopen}f\ {\isacharequal}{\kern0pt}\ nth{\isacharparenleft}{\kern0pt}i{\isacharcomma}{\kern0pt}\ env{\isacharparenright}{\kern0pt}{\isachardoublequoteclose}\ {\isachardoublequoteopen}p\ {\isacharequal}{\kern0pt}\ nth{\isacharparenleft}{\kern0pt}j{\isacharcomma}{\kern0pt}\ env{\isacharparenright}{\kern0pt}{\isachardoublequoteclose}\ {\isachardoublequoteopen}v\ {\isacharequal}{\kern0pt}\ nth{\isacharparenleft}{\kern0pt}k{\isacharcomma}{\kern0pt}\ env{\isacharparenright}{\kern0pt}{\isachardoublequoteclose}\ \isanewline
\ \ \ \ \ \ \ \ \ \ {\isachardoublequoteopen}p\ {\isasymin}\ Fn{\isachardoublequoteclose}\ {\isachardoublequoteopen}f\ {\isasymin}\ nat{\isacharunderscore}{\kern0pt}perms{\isachardoublequoteclose}\isanewline
\ \ \isakeyword{shows}\ {\isachardoublequoteopen}sats{\isacharparenleft}{\kern0pt}M{\isacharcomma}{\kern0pt}\ is{\isacharunderscore}{\kern0pt}Fn{\isacharunderscore}{\kern0pt}perm{\isacharunderscore}{\kern0pt}fm{\isacharparenleft}{\kern0pt}i{\isacharcomma}{\kern0pt}\ j{\isacharcomma}{\kern0pt}\ k{\isacharparenright}{\kern0pt}{\isacharcomma}{\kern0pt}\ env{\isacharparenright}{\kern0pt}\ {\isasymlongleftrightarrow}\ v\ {\isacharequal}{\kern0pt}\ Fn{\isacharunderscore}{\kern0pt}perm{\isacharparenleft}{\kern0pt}f{\isacharcomma}{\kern0pt}\ p{\isacharparenright}{\kern0pt}{\isachardoublequoteclose}\isanewline
%
\isadelimproof
%
\endisadelimproof
%
\isatagproof
\isacommand{proof}\isamarkupfalse%
\ {\isacharminus}{\kern0pt}\ \isanewline
\ \ \isacommand{have}\isamarkupfalse%
\ I{\isadigit{1}}{\isacharcolon}{\kern0pt}\ \ {\isachardoublequoteopen}sats{\isacharparenleft}{\kern0pt}M{\isacharcomma}{\kern0pt}\ is{\isacharunderscore}{\kern0pt}Fn{\isacharunderscore}{\kern0pt}perm{\isacharunderscore}{\kern0pt}fm{\isacharparenleft}{\kern0pt}i{\isacharcomma}{\kern0pt}\ j{\isacharcomma}{\kern0pt}\ k{\isacharparenright}{\kern0pt}{\isacharcomma}{\kern0pt}\ env{\isacharparenright}{\kern0pt}\ {\isasymlongleftrightarrow}\ {\isacharparenleft}{\kern0pt}{\isasymforall}x\ {\isasymin}\ M{\isachardot}{\kern0pt}\ {\isacharparenleft}{\kern0pt}x\ {\isasymin}\ v\ {\isasymlongleftrightarrow}\ {\isacharparenleft}{\kern0pt}{\isasymexists}a\ {\isasymin}\ M{\isachardot}{\kern0pt}\ a\ {\isasymin}\ p\ {\isasymand}\ {\isacharparenleft}{\kern0pt}{\isasymexists}n\ m\ l{\isachardot}{\kern0pt}\ {\isacharless}{\kern0pt}{\isacharless}{\kern0pt}n{\isacharcomma}{\kern0pt}\ m{\isachargreater}{\kern0pt}{\isacharcomma}{\kern0pt}\ l{\isachargreater}{\kern0pt}\ {\isacharequal}{\kern0pt}\ a\ {\isasymand}\ x\ {\isacharequal}{\kern0pt}\ {\isacharless}{\kern0pt}{\isacharless}{\kern0pt}f{\isacharbackquote}{\kern0pt}n{\isacharcomma}{\kern0pt}\ m{\isachargreater}{\kern0pt}{\isacharcomma}{\kern0pt}\ l{\isachargreater}{\kern0pt}{\isacharparenright}{\kern0pt}{\isacharparenright}{\kern0pt}{\isacharparenright}{\kern0pt}{\isacharparenright}{\kern0pt}{\isachardoublequoteclose}\isanewline
\ \ \ \ \isacommand{unfolding}\isamarkupfalse%
\ is{\isacharunderscore}{\kern0pt}Fn{\isacharunderscore}{\kern0pt}perm{\isacharunderscore}{\kern0pt}fm{\isacharunderscore}{\kern0pt}def\isanewline
\ \ \ \ \isacommand{using}\isamarkupfalse%
\ assms\ lt{\isacharunderscore}{\kern0pt}nat{\isacharunderscore}{\kern0pt}in{\isacharunderscore}{\kern0pt}nat\isanewline
\ \ \ \ \isacommand{apply}\isamarkupfalse%
\ simp\isanewline
\ \ \ \ \isacommand{apply}\isamarkupfalse%
{\isacharparenleft}{\kern0pt}rule\ ball{\isacharunderscore}{\kern0pt}iff{\isacharcomma}{\kern0pt}\ rule\ iff{\isacharunderscore}{\kern0pt}iff{\isacharcomma}{\kern0pt}\ simp{\isacharparenright}{\kern0pt}\isanewline
\ \ \ \ \isacommand{apply}\isamarkupfalse%
{\isacharparenleft}{\kern0pt}rule\ bex{\isacharunderscore}{\kern0pt}iff{\isacharcomma}{\kern0pt}\ rule\ iff{\isacharunderscore}{\kern0pt}conjI{\isadigit{2}}{\isacharcomma}{\kern0pt}\ simp{\isacharparenright}{\kern0pt}\isanewline
\ \ \ \ \isacommand{apply}\isamarkupfalse%
{\isacharparenleft}{\kern0pt}rule\ sats{\isacharunderscore}{\kern0pt}is{\isacharunderscore}{\kern0pt}Fn{\isacharunderscore}{\kern0pt}perm{\isacharunderscore}{\kern0pt}elem{\isacharunderscore}{\kern0pt}fm{\isacharunderscore}{\kern0pt}iff{\isacharparenright}{\kern0pt}\isanewline
\ \ \ \ \isacommand{by}\isamarkupfalse%
\ auto\isanewline
\ \ \isacommand{have}\isamarkupfalse%
\ I{\isadigit{2}}{\isacharcolon}{\kern0pt}\ {\isachardoublequoteopen}{\isachardot}{\kern0pt}{\isachardot}{\kern0pt}{\isachardot}{\kern0pt}\ {\isasymlongleftrightarrow}\ {\isacharparenleft}{\kern0pt}{\isasymforall}x{\isachardot}{\kern0pt}\ x\ {\isasymin}\ v\ {\isasymlongleftrightarrow}\ {\isacharparenleft}{\kern0pt}{\isasymexists}a\ {\isasymin}\ M{\isachardot}{\kern0pt}\ a\ {\isasymin}\ p\ {\isasymand}\ {\isacharparenleft}{\kern0pt}{\isasymexists}n\ m\ l{\isachardot}{\kern0pt}\ {\isacharless}{\kern0pt}{\isacharless}{\kern0pt}n{\isacharcomma}{\kern0pt}\ m{\isachargreater}{\kern0pt}{\isacharcomma}{\kern0pt}\ l{\isachargreater}{\kern0pt}\ {\isacharequal}{\kern0pt}\ a\ {\isasymand}\ x\ {\isacharequal}{\kern0pt}\ {\isacharless}{\kern0pt}{\isacharless}{\kern0pt}f{\isacharbackquote}{\kern0pt}n{\isacharcomma}{\kern0pt}\ m{\isachargreater}{\kern0pt}{\isacharcomma}{\kern0pt}\ l{\isachargreater}{\kern0pt}{\isacharparenright}{\kern0pt}{\isacharparenright}{\kern0pt}{\isacharparenright}{\kern0pt}{\isachardoublequoteclose}\isanewline
\ \ \ \ \isacommand{apply}\isamarkupfalse%
{\isacharparenleft}{\kern0pt}rule\ iffI{\isacharcomma}{\kern0pt}\ rule\ allI{\isacharcomma}{\kern0pt}\ rule\ iffI{\isacharparenright}{\kern0pt}\isanewline
\ \ \ \ \isacommand{using}\isamarkupfalse%
\ assms\ nth{\isacharunderscore}{\kern0pt}type\ lt{\isacharunderscore}{\kern0pt}nat{\isacharunderscore}{\kern0pt}in{\isacharunderscore}{\kern0pt}nat\ transM\isanewline
\ \ \ \ \ \ \isacommand{apply}\isamarkupfalse%
\ force\isanewline
\ \ \ \ \isacommand{apply}\isamarkupfalse%
{\isacharparenleft}{\kern0pt}rename{\isacharunderscore}{\kern0pt}tac\ x{\isacharcomma}{\kern0pt}\ subgoal{\isacharunderscore}{\kern0pt}tac\ {\isachardoublequoteopen}x\ {\isasymin}\ M{\isachardoublequoteclose}{\isacharcomma}{\kern0pt}\ force{\isacharparenright}{\kern0pt}\isanewline
\ \ \ \ \isacommand{using}\isamarkupfalse%
\ assms\ lt{\isacharunderscore}{\kern0pt}nat{\isacharunderscore}{\kern0pt}in{\isacharunderscore}{\kern0pt}nat\ nth{\isacharunderscore}{\kern0pt}type\ pair{\isacharunderscore}{\kern0pt}in{\isacharunderscore}{\kern0pt}M{\isacharunderscore}{\kern0pt}iff\ apply{\isacharunderscore}{\kern0pt}closed\isanewline
\ \ \ \ \isacommand{by}\isamarkupfalse%
\ auto\isanewline
\ \ \isacommand{have}\isamarkupfalse%
\ I{\isadigit{3}}{\isacharcolon}{\kern0pt}\ {\isachardoublequoteopen}{\isachardot}{\kern0pt}{\isachardot}{\kern0pt}{\isachardot}{\kern0pt}\ {\isasymlongleftrightarrow}\ {\isacharparenleft}{\kern0pt}{\isasymforall}x{\isachardot}{\kern0pt}\ x\ {\isasymin}\ v\ {\isasymlongleftrightarrow}\ {\isacharparenleft}{\kern0pt}{\isasymexists}a{\isachardot}{\kern0pt}\ a\ {\isasymin}\ p\ {\isasymand}\ {\isacharparenleft}{\kern0pt}{\isasymexists}n\ m\ l{\isachardot}{\kern0pt}\ {\isacharless}{\kern0pt}{\isacharless}{\kern0pt}n{\isacharcomma}{\kern0pt}\ m{\isachargreater}{\kern0pt}{\isacharcomma}{\kern0pt}\ l{\isachargreater}{\kern0pt}\ {\isacharequal}{\kern0pt}\ a\ {\isasymand}\ x\ {\isacharequal}{\kern0pt}\ {\isacharless}{\kern0pt}{\isacharless}{\kern0pt}f{\isacharbackquote}{\kern0pt}n{\isacharcomma}{\kern0pt}\ m{\isachargreater}{\kern0pt}{\isacharcomma}{\kern0pt}\ l{\isachargreater}{\kern0pt}{\isacharparenright}{\kern0pt}{\isacharparenright}{\kern0pt}{\isacharparenright}{\kern0pt}{\isachardoublequoteclose}\ \ \isanewline
\ \ \ \ \isacommand{apply}\isamarkupfalse%
{\isacharparenleft}{\kern0pt}rule\ all{\isacharunderscore}{\kern0pt}iff{\isacharcomma}{\kern0pt}\ rule\ iff{\isacharunderscore}{\kern0pt}iff{\isacharcomma}{\kern0pt}\ simp{\isacharcomma}{\kern0pt}\ rule\ iffI{\isacharcomma}{\kern0pt}\ force{\isacharcomma}{\kern0pt}\ clarify{\isacharparenright}{\kern0pt}\isanewline
\ \ \ \ \isacommand{using}\isamarkupfalse%
\ assms\ nth{\isacharunderscore}{\kern0pt}type\ lt{\isacharunderscore}{\kern0pt}nat{\isacharunderscore}{\kern0pt}in{\isacharunderscore}{\kern0pt}nat\ transM\ \isanewline
\ \ \ \ \isacommand{apply}\isamarkupfalse%
\ force\isanewline
\ \ \ \ \isacommand{done}\isamarkupfalse%
\ \isanewline
\ \ \isacommand{have}\isamarkupfalse%
\ I{\isadigit{4}}{\isacharcolon}{\kern0pt}\ {\isachardoublequoteopen}{\isachardot}{\kern0pt}{\isachardot}{\kern0pt}{\isachardot}{\kern0pt}\ {\isasymlongleftrightarrow}\ v\ {\isacharequal}{\kern0pt}\ Fn{\isacharunderscore}{\kern0pt}perm{\isacharparenleft}{\kern0pt}f{\isacharcomma}{\kern0pt}\ p{\isacharparenright}{\kern0pt}{\isachardoublequoteclose}\isanewline
\ \ \ \ \isacommand{apply}\isamarkupfalse%
{\isacharparenleft}{\kern0pt}rule\ iffI{\isacharcomma}{\kern0pt}\ rule\ equality{\isacharunderscore}{\kern0pt}iffI{\isacharcomma}{\kern0pt}\ rule\ iffI{\isacharparenright}{\kern0pt}\isanewline
\ \ \ \ \isacommand{using}\isamarkupfalse%
\ Fn{\isacharunderscore}{\kern0pt}perm{\isacharunderscore}{\kern0pt}def\isanewline
\ \ \ \ \ \ \isacommand{apply}\isamarkupfalse%
\ force\ \isanewline
\ \ \ \ \ \isacommand{apply}\isamarkupfalse%
{\isacharparenleft}{\kern0pt}rename{\isacharunderscore}{\kern0pt}tac\ x{\isacharcomma}{\kern0pt}\ subgoal{\isacharunderscore}{\kern0pt}tac\ {\isachardoublequoteopen}{\isasymexists}n\ {\isasymin}\ nat{\isachardot}{\kern0pt}\ {\isasymexists}m\ {\isasymin}\ nat{\isachardot}{\kern0pt}\ {\isasymexists}l\ {\isasymin}\ {\isadigit{2}}{\isachardot}{\kern0pt}\ {\isacharless}{\kern0pt}{\isacharless}{\kern0pt}n{\isacharcomma}{\kern0pt}\ m{\isachargreater}{\kern0pt}{\isacharcomma}{\kern0pt}\ l{\isachargreater}{\kern0pt}\ {\isasymin}\ p\ {\isasymand}\ x\ {\isacharequal}{\kern0pt}\ {\isacharless}{\kern0pt}{\isacharless}{\kern0pt}f{\isacharbackquote}{\kern0pt}n{\isacharcomma}{\kern0pt}\ m{\isachargreater}{\kern0pt}{\isacharcomma}{\kern0pt}\ l{\isachargreater}{\kern0pt}{\isachardoublequoteclose}{\isacharcomma}{\kern0pt}\ force{\isacharparenright}{\kern0pt}\isanewline
\ \ \ \ \ \isacommand{apply}\isamarkupfalse%
{\isacharparenleft}{\kern0pt}rule\ Fn{\isacharunderscore}{\kern0pt}permE{\isacharparenright}{\kern0pt}\isanewline
\ \ \ \ \isacommand{using}\isamarkupfalse%
\ assms\isanewline
\ \ \ \ \ \ \ \isacommand{apply}\isamarkupfalse%
\ auto{\isacharbrackleft}{\kern0pt}{\isadigit{3}}{\isacharbrackright}{\kern0pt}\isanewline
\ \ \ \ \isacommand{apply}\isamarkupfalse%
\ simp\isanewline
\ \ \ \ \isacommand{apply}\isamarkupfalse%
{\isacharparenleft}{\kern0pt}rule\ allI{\isacharcomma}{\kern0pt}\ rule\ iffI{\isacharparenright}{\kern0pt}\isanewline
\ \ \ \ \ \isacommand{apply}\isamarkupfalse%
{\isacharparenleft}{\kern0pt}rename{\isacharunderscore}{\kern0pt}tac\ x{\isacharcomma}{\kern0pt}\ subgoal{\isacharunderscore}{\kern0pt}tac\ {\isachardoublequoteopen}{\isasymexists}n\ {\isasymin}\ nat{\isachardot}{\kern0pt}\ {\isasymexists}m\ {\isasymin}\ nat{\isachardot}{\kern0pt}\ {\isasymexists}l\ {\isasymin}\ {\isadigit{2}}{\isachardot}{\kern0pt}\ {\isacharless}{\kern0pt}{\isacharless}{\kern0pt}n{\isacharcomma}{\kern0pt}\ m{\isachargreater}{\kern0pt}{\isacharcomma}{\kern0pt}\ l{\isachargreater}{\kern0pt}\ {\isasymin}\ p\ {\isasymand}\ x\ {\isacharequal}{\kern0pt}\ {\isacharless}{\kern0pt}{\isacharless}{\kern0pt}f{\isacharbackquote}{\kern0pt}n{\isacharcomma}{\kern0pt}\ m{\isachargreater}{\kern0pt}{\isacharcomma}{\kern0pt}\ l{\isachargreater}{\kern0pt}{\isachardoublequoteclose}{\isacharcomma}{\kern0pt}\ force{\isacharparenright}{\kern0pt}\isanewline
\ \ \ \ \ \isacommand{apply}\isamarkupfalse%
{\isacharparenleft}{\kern0pt}rule\ Fn{\isacharunderscore}{\kern0pt}permE{\isacharparenright}{\kern0pt}\isanewline
\ \ \ \ \isacommand{using}\isamarkupfalse%
\ assms\isanewline
\ \ \ \ \ \ \ \isacommand{apply}\isamarkupfalse%
\ auto{\isacharbrackleft}{\kern0pt}{\isadigit{3}}{\isacharbrackright}{\kern0pt}\isanewline
\ \ \ \ \isacommand{using}\isamarkupfalse%
\ Fn{\isacharunderscore}{\kern0pt}perm{\isacharunderscore}{\kern0pt}def\isanewline
\ \ \ \ \isacommand{apply}\isamarkupfalse%
\ force\ \isanewline
\ \ \ \ \isacommand{done}\isamarkupfalse%
\isanewline
\ \ \isacommand{show}\isamarkupfalse%
\ {\isacharquery}{\kern0pt}thesis\ \isacommand{using}\isamarkupfalse%
\ I{\isadigit{1}}\ I{\isadigit{2}}\ I{\isadigit{3}}\ I{\isadigit{4}}\ \isacommand{by}\isamarkupfalse%
\ auto\isanewline
\isacommand{qed}\isamarkupfalse%
%
\endisatagproof
{\isafoldproof}%
%
\isadelimproof
\isanewline
%
\endisadelimproof
\isanewline
\isacommand{definition}\isamarkupfalse%
\ is{\isacharunderscore}{\kern0pt}Fn{\isacharunderscore}{\kern0pt}perm{\isacharprime}{\kern0pt}{\isacharunderscore}{\kern0pt}elem{\isacharunderscore}{\kern0pt}fm\ \isakeyword{where}\ \isanewline
\ \ {\isachardoublequoteopen}is{\isacharunderscore}{\kern0pt}Fn{\isacharunderscore}{\kern0pt}perm{\isacharprime}{\kern0pt}{\isacharunderscore}{\kern0pt}elem{\isacharunderscore}{\kern0pt}fm{\isacharparenleft}{\kern0pt}f{\isacharcomma}{\kern0pt}\ p{\isacharcomma}{\kern0pt}\ v{\isacharparenright}{\kern0pt}\ {\isasymequiv}\ Exists{\isacharparenleft}{\kern0pt}And{\isacharparenleft}{\kern0pt}is{\isacharunderscore}{\kern0pt}Fn{\isacharunderscore}{\kern0pt}perm{\isacharunderscore}{\kern0pt}fm{\isacharparenleft}{\kern0pt}f\ {\isacharhash}{\kern0pt}{\isacharplus}{\kern0pt}\ {\isadigit{1}}{\isacharcomma}{\kern0pt}\ p\ {\isacharhash}{\kern0pt}{\isacharplus}{\kern0pt}\ {\isadigit{1}}{\isacharcomma}{\kern0pt}\ {\isadigit{0}}{\isacharparenright}{\kern0pt}{\isacharcomma}{\kern0pt}\ pair{\isacharunderscore}{\kern0pt}fm{\isacharparenleft}{\kern0pt}p\ {\isacharhash}{\kern0pt}{\isacharplus}{\kern0pt}\ {\isadigit{1}}{\isacharcomma}{\kern0pt}\ {\isadigit{0}}{\isacharcomma}{\kern0pt}\ v\ {\isacharhash}{\kern0pt}{\isacharplus}{\kern0pt}\ {\isadigit{1}}{\isacharparenright}{\kern0pt}{\isacharparenright}{\kern0pt}{\isacharparenright}{\kern0pt}{\isachardoublequoteclose}\isanewline
\isanewline
\isacommand{lemma}\isamarkupfalse%
\ is{\isacharunderscore}{\kern0pt}Fn{\isacharunderscore}{\kern0pt}perm{\isacharprime}{\kern0pt}{\isacharunderscore}{\kern0pt}elem{\isacharunderscore}{\kern0pt}fm{\isacharunderscore}{\kern0pt}type{\isacharcolon}{\kern0pt}\ \isanewline
\ \ \isakeyword{fixes}\ i\ j\ k\ \isanewline
\ \ \isakeyword{assumes}\ {\isachardoublequoteopen}i\ {\isasymin}\ nat{\isachardoublequoteclose}\ {\isachardoublequoteopen}j\ {\isasymin}\ nat{\isachardoublequoteclose}\ {\isachardoublequoteopen}k\ {\isasymin}\ nat{\isachardoublequoteclose}\ \isanewline
\ \ \isakeyword{shows}\ {\isachardoublequoteopen}is{\isacharunderscore}{\kern0pt}Fn{\isacharunderscore}{\kern0pt}perm{\isacharprime}{\kern0pt}{\isacharunderscore}{\kern0pt}elem{\isacharunderscore}{\kern0pt}fm{\isacharparenleft}{\kern0pt}i{\isacharcomma}{\kern0pt}\ j{\isacharcomma}{\kern0pt}\ k{\isacharparenright}{\kern0pt}\ {\isasymin}\ formula{\isachardoublequoteclose}\ \isanewline
%
\isadelimproof
\ \ %
\endisadelimproof
%
\isatagproof
\isacommand{unfolding}\isamarkupfalse%
\ is{\isacharunderscore}{\kern0pt}Fn{\isacharunderscore}{\kern0pt}perm{\isacharprime}{\kern0pt}{\isacharunderscore}{\kern0pt}elem{\isacharunderscore}{\kern0pt}fm{\isacharunderscore}{\kern0pt}def\isanewline
\ \ \isacommand{apply}\isamarkupfalse%
{\isacharparenleft}{\kern0pt}subgoal{\isacharunderscore}{\kern0pt}tac\ {\isachardoublequoteopen}is{\isacharunderscore}{\kern0pt}Fn{\isacharunderscore}{\kern0pt}perm{\isacharunderscore}{\kern0pt}fm{\isacharparenleft}{\kern0pt}i\ {\isacharhash}{\kern0pt}{\isacharplus}{\kern0pt}\ {\isadigit{1}}{\isacharcomma}{\kern0pt}\ j\ {\isacharhash}{\kern0pt}{\isacharplus}{\kern0pt}\ {\isadigit{1}}{\isacharcomma}{\kern0pt}\ {\isadigit{0}}{\isacharparenright}{\kern0pt}\ {\isasymin}\ formula{\isachardoublequoteclose}{\isacharparenright}{\kern0pt}\isanewline
\ \ \ \isacommand{apply}\isamarkupfalse%
\ force\ \isanewline
\ \ \isacommand{using}\isamarkupfalse%
\ is{\isacharunderscore}{\kern0pt}Fn{\isacharunderscore}{\kern0pt}perm{\isacharunderscore}{\kern0pt}fm{\isacharunderscore}{\kern0pt}type\ assms\isanewline
\ \ \isacommand{by}\isamarkupfalse%
\ auto%
\endisatagproof
{\isafoldproof}%
%
\isadelimproof
\isanewline
%
\endisadelimproof
\isanewline
\isacommand{lemma}\isamarkupfalse%
\ arity{\isacharunderscore}{\kern0pt}is{\isacharunderscore}{\kern0pt}Fn{\isacharunderscore}{\kern0pt}perm{\isacharprime}{\kern0pt}{\isacharunderscore}{\kern0pt}elem{\isacharunderscore}{\kern0pt}fm\ {\isacharcolon}{\kern0pt}\ \isanewline
\ \ \isakeyword{fixes}\ i\ j\ k\ \isanewline
\ \ \isakeyword{assumes}\ {\isachardoublequoteopen}i\ {\isasymin}\ nat{\isachardoublequoteclose}\ {\isachardoublequoteopen}j\ {\isasymin}\ nat{\isachardoublequoteclose}\ {\isachardoublequoteopen}k\ {\isasymin}\ nat{\isachardoublequoteclose}\ \isanewline
\ \ \isakeyword{shows}\ {\isachardoublequoteopen}arity{\isacharparenleft}{\kern0pt}is{\isacharunderscore}{\kern0pt}Fn{\isacharunderscore}{\kern0pt}perm{\isacharprime}{\kern0pt}{\isacharunderscore}{\kern0pt}elem{\isacharunderscore}{\kern0pt}fm{\isacharparenleft}{\kern0pt}i{\isacharcomma}{\kern0pt}\ j{\isacharcomma}{\kern0pt}\ k{\isacharparenright}{\kern0pt}{\isacharparenright}{\kern0pt}\ {\isasymle}\ succ{\isacharparenleft}{\kern0pt}i{\isacharparenright}{\kern0pt}\ {\isasymunion}\ succ{\isacharparenleft}{\kern0pt}j{\isacharparenright}{\kern0pt}\ {\isasymunion}\ succ{\isacharparenleft}{\kern0pt}k{\isacharparenright}{\kern0pt}{\isachardoublequoteclose}\isanewline
%
\isadelimproof
\isanewline
\ \ %
\endisadelimproof
%
\isatagproof
\isacommand{unfolding}\isamarkupfalse%
\ is{\isacharunderscore}{\kern0pt}Fn{\isacharunderscore}{\kern0pt}perm{\isacharprime}{\kern0pt}{\isacharunderscore}{\kern0pt}elem{\isacharunderscore}{\kern0pt}fm{\isacharunderscore}{\kern0pt}def\isanewline
\ \ \isacommand{apply}\isamarkupfalse%
{\isacharparenleft}{\kern0pt}subgoal{\isacharunderscore}{\kern0pt}tac\ {\isachardoublequoteopen}is{\isacharunderscore}{\kern0pt}Fn{\isacharunderscore}{\kern0pt}perm{\isacharunderscore}{\kern0pt}fm{\isacharparenleft}{\kern0pt}succ{\isacharparenleft}{\kern0pt}i{\isacharparenright}{\kern0pt}{\isacharcomma}{\kern0pt}\ succ{\isacharparenleft}{\kern0pt}j{\isacharparenright}{\kern0pt}{\isacharcomma}{\kern0pt}\ {\isadigit{0}}{\isacharparenright}{\kern0pt}\ {\isasymin}\ formula{\isachardoublequoteclose}{\isacharparenright}{\kern0pt}\isanewline
\ \ \isacommand{using}\isamarkupfalse%
\ assms\isanewline
\ \ \isacommand{apply}\isamarkupfalse%
\ simp\isanewline
\ \ \ \isacommand{apply}\isamarkupfalse%
{\isacharparenleft}{\kern0pt}rule\ pred{\isacharunderscore}{\kern0pt}le{\isacharcomma}{\kern0pt}\ simp{\isacharcomma}{\kern0pt}\ simp{\isacharparenright}{\kern0pt}\isanewline
\ \ \ \isacommand{apply}\isamarkupfalse%
{\isacharparenleft}{\kern0pt}rule\ Un{\isacharunderscore}{\kern0pt}least{\isacharunderscore}{\kern0pt}lt{\isacharcomma}{\kern0pt}\ rule\ le{\isacharunderscore}{\kern0pt}trans{\isacharcomma}{\kern0pt}\ rule\ arity{\isacharunderscore}{\kern0pt}is{\isacharunderscore}{\kern0pt}Fn{\isacharunderscore}{\kern0pt}perm{\isacharunderscore}{\kern0pt}fm{\isacharparenright}{\kern0pt}\isanewline
\ \ \ \ \ \ \ \isacommand{apply}\isamarkupfalse%
\ auto{\isacharbrackleft}{\kern0pt}{\isadigit{3}}{\isacharbrackright}{\kern0pt}\isanewline
\ \ \ \ \isacommand{apply}\isamarkupfalse%
\ {\isacharparenleft}{\kern0pt}simp\ del{\isacharcolon}{\kern0pt}FOL{\isacharunderscore}{\kern0pt}sats{\isacharunderscore}{\kern0pt}iff\ pair{\isacharunderscore}{\kern0pt}abs\ add{\isacharcolon}{\kern0pt}\ fm{\isacharunderscore}{\kern0pt}defs\ nat{\isacharunderscore}{\kern0pt}simp{\isacharunderscore}{\kern0pt}union{\isacharparenright}{\kern0pt}\isanewline
\ \ \ \isacommand{apply}\isamarkupfalse%
{\isacharparenleft}{\kern0pt}subst\ arity{\isacharunderscore}{\kern0pt}pair{\isacharunderscore}{\kern0pt}fm{\isacharparenright}{\kern0pt}\isanewline
\ \ \isacommand{apply}\isamarkupfalse%
\ auto{\isacharbrackleft}{\kern0pt}{\isadigit{3}}{\isacharbrackright}{\kern0pt}\isanewline
\ \ \ \isacommand{apply}\isamarkupfalse%
\ simp\isanewline
\ \ \ \isacommand{apply}\isamarkupfalse%
{\isacharparenleft}{\kern0pt}rule\ Un{\isacharunderscore}{\kern0pt}least{\isacharunderscore}{\kern0pt}lt{\isacharcomma}{\kern0pt}\ simp{\isacharcomma}{\kern0pt}\ rule\ ltI{\isacharcomma}{\kern0pt}\ simp{\isacharcomma}{\kern0pt}\ simp{\isacharparenright}{\kern0pt}\isanewline
\ \ \ \isacommand{apply}\isamarkupfalse%
{\isacharparenleft}{\kern0pt}rule\ Un{\isacharunderscore}{\kern0pt}least{\isacharunderscore}{\kern0pt}lt{\isacharcomma}{\kern0pt}\ simp{\isacharcomma}{\kern0pt}\ simp{\isacharcomma}{\kern0pt}\ rule\ ltI{\isacharcomma}{\kern0pt}\ simp{\isacharcomma}{\kern0pt}\ simp{\isacharparenright}{\kern0pt}\isanewline
\ \ \isacommand{apply}\isamarkupfalse%
{\isacharparenleft}{\kern0pt}rule\ is{\isacharunderscore}{\kern0pt}Fn{\isacharunderscore}{\kern0pt}perm{\isacharunderscore}{\kern0pt}fm{\isacharunderscore}{\kern0pt}type{\isacharparenright}{\kern0pt}\isanewline
\ \ \isacommand{using}\isamarkupfalse%
\ assms\isanewline
\ \ \isacommand{by}\isamarkupfalse%
\ auto%
\endisatagproof
{\isafoldproof}%
%
\isadelimproof
\isanewline
%
\endisadelimproof
\isanewline
\isacommand{lemma}\isamarkupfalse%
\ sats{\isacharunderscore}{\kern0pt}is{\isacharunderscore}{\kern0pt}Fn{\isacharunderscore}{\kern0pt}perm{\isacharprime}{\kern0pt}{\isacharunderscore}{\kern0pt}elem{\isacharunderscore}{\kern0pt}fm{\isacharunderscore}{\kern0pt}iff\ {\isacharcolon}{\kern0pt}\ \isanewline
\ \ \isakeyword{fixes}\ env\ i\ j\ k\ f\ p\ v\ \isanewline
\ \ \isakeyword{assumes}\ {\isachardoublequoteopen}env\ {\isasymin}\ list{\isacharparenleft}{\kern0pt}M{\isacharparenright}{\kern0pt}{\isachardoublequoteclose}\ {\isachardoublequoteopen}i\ {\isacharless}{\kern0pt}\ length{\isacharparenleft}{\kern0pt}env{\isacharparenright}{\kern0pt}{\isachardoublequoteclose}\ {\isachardoublequoteopen}j\ {\isacharless}{\kern0pt}\ length{\isacharparenleft}{\kern0pt}env{\isacharparenright}{\kern0pt}{\isachardoublequoteclose}\ {\isachardoublequoteopen}k\ {\isacharless}{\kern0pt}\ length{\isacharparenleft}{\kern0pt}env{\isacharparenright}{\kern0pt}{\isachardoublequoteclose}\ \isanewline
\ \ \ \ \ \ \ \ \ \ {\isachardoublequoteopen}f\ {\isacharequal}{\kern0pt}\ nth{\isacharparenleft}{\kern0pt}i{\isacharcomma}{\kern0pt}\ env{\isacharparenright}{\kern0pt}{\isachardoublequoteclose}\ {\isachardoublequoteopen}p\ {\isacharequal}{\kern0pt}\ nth{\isacharparenleft}{\kern0pt}j{\isacharcomma}{\kern0pt}\ env{\isacharparenright}{\kern0pt}{\isachardoublequoteclose}\ {\isachardoublequoteopen}v\ {\isacharequal}{\kern0pt}\ nth{\isacharparenleft}{\kern0pt}k{\isacharcomma}{\kern0pt}\ env{\isacharparenright}{\kern0pt}{\isachardoublequoteclose}\ \isanewline
\ \ \ \ \ \ \ \ \ \ {\isachardoublequoteopen}p\ {\isasymin}\ Fn{\isachardoublequoteclose}\ {\isachardoublequoteopen}f\ {\isasymin}\ nat{\isacharunderscore}{\kern0pt}perms{\isachardoublequoteclose}\isanewline
\ \ \isakeyword{shows}\ {\isachardoublequoteopen}sats{\isacharparenleft}{\kern0pt}M{\isacharcomma}{\kern0pt}\ is{\isacharunderscore}{\kern0pt}Fn{\isacharunderscore}{\kern0pt}perm{\isacharprime}{\kern0pt}{\isacharunderscore}{\kern0pt}elem{\isacharunderscore}{\kern0pt}fm{\isacharparenleft}{\kern0pt}i{\isacharcomma}{\kern0pt}\ j{\isacharcomma}{\kern0pt}\ k{\isacharparenright}{\kern0pt}{\isacharcomma}{\kern0pt}\ env{\isacharparenright}{\kern0pt}\ {\isasymlongleftrightarrow}\ v\ {\isacharequal}{\kern0pt}\ {\isacharless}{\kern0pt}p{\isacharcomma}{\kern0pt}\ Fn{\isacharunderscore}{\kern0pt}perm{\isacharparenleft}{\kern0pt}f{\isacharcomma}{\kern0pt}\ p{\isacharparenright}{\kern0pt}{\isachargreater}{\kern0pt}{\isachardoublequoteclose}\ \isanewline
%
\isadelimproof
%
\endisadelimproof
%
\isatagproof
\isacommand{proof}\isamarkupfalse%
\ {\isacharminus}{\kern0pt}\ \isanewline
\ \ \isacommand{have}\isamarkupfalse%
\ I{\isadigit{1}}{\isacharcolon}{\kern0pt}\ {\isachardoublequoteopen}sats{\isacharparenleft}{\kern0pt}M{\isacharcomma}{\kern0pt}\ is{\isacharunderscore}{\kern0pt}Fn{\isacharunderscore}{\kern0pt}perm{\isacharprime}{\kern0pt}{\isacharunderscore}{\kern0pt}elem{\isacharunderscore}{\kern0pt}fm{\isacharparenleft}{\kern0pt}i{\isacharcomma}{\kern0pt}\ j{\isacharcomma}{\kern0pt}\ k{\isacharparenright}{\kern0pt}{\isacharcomma}{\kern0pt}\ env{\isacharparenright}{\kern0pt}\ {\isasymlongleftrightarrow}\ {\isacharparenleft}{\kern0pt}{\isasymexists}u\ {\isasymin}\ M{\isachardot}{\kern0pt}\ u\ {\isacharequal}{\kern0pt}\ Fn{\isacharunderscore}{\kern0pt}perm{\isacharparenleft}{\kern0pt}f{\isacharcomma}{\kern0pt}\ p{\isacharparenright}{\kern0pt}\ {\isasymand}\ v\ {\isacharequal}{\kern0pt}\ {\isacharless}{\kern0pt}p{\isacharcomma}{\kern0pt}\ u{\isachargreater}{\kern0pt}{\isacharparenright}{\kern0pt}{\isachardoublequoteclose}\ \isanewline
\ \ \ \ \isacommand{unfolding}\isamarkupfalse%
\ is{\isacharunderscore}{\kern0pt}Fn{\isacharunderscore}{\kern0pt}perm{\isacharprime}{\kern0pt}{\isacharunderscore}{\kern0pt}elem{\isacharunderscore}{\kern0pt}fm{\isacharunderscore}{\kern0pt}def\isanewline
\ \ \ \ \isacommand{apply}\isamarkupfalse%
{\isacharparenleft}{\kern0pt}rule\ iff{\isacharunderscore}{\kern0pt}trans{\isacharcomma}{\kern0pt}\ rule\ sats{\isacharunderscore}{\kern0pt}Exists{\isacharunderscore}{\kern0pt}iff{\isacharcomma}{\kern0pt}\ simp\ add{\isacharcolon}{\kern0pt}assms{\isacharcomma}{\kern0pt}\ rule\ bex{\isacharunderscore}{\kern0pt}iff{\isacharparenright}{\kern0pt}\isanewline
\ \ \ \ \isacommand{apply}\isamarkupfalse%
{\isacharparenleft}{\kern0pt}rule\ iff{\isacharunderscore}{\kern0pt}trans{\isacharcomma}{\kern0pt}\ rule\ sats{\isacharunderscore}{\kern0pt}And{\isacharunderscore}{\kern0pt}iff{\isacharcomma}{\kern0pt}\ simp\ add{\isacharcolon}{\kern0pt}assms{\isacharcomma}{\kern0pt}\ rule\ iff{\isacharunderscore}{\kern0pt}conjI{\isadigit{2}}{\isacharparenright}{\kern0pt}\isanewline
\ \ \ \ \ \isacommand{apply}\isamarkupfalse%
{\isacharparenleft}{\kern0pt}rule\ sats{\isacharunderscore}{\kern0pt}is{\isacharunderscore}{\kern0pt}Fn{\isacharunderscore}{\kern0pt}perm{\isacharunderscore}{\kern0pt}fm{\isacharunderscore}{\kern0pt}iff{\isacharparenright}{\kern0pt}\isanewline
\ \ \ \ \isacommand{using}\isamarkupfalse%
\ assms\ nth{\isacharunderscore}{\kern0pt}type\ lt{\isacharunderscore}{\kern0pt}nat{\isacharunderscore}{\kern0pt}in{\isacharunderscore}{\kern0pt}nat\isanewline
\ \ \ \ \isacommand{by}\isamarkupfalse%
\ auto\isanewline
\ \ \isacommand{also}\isamarkupfalse%
\ \isacommand{have}\isamarkupfalse%
\ I{\isadigit{2}}{\isacharcolon}{\kern0pt}\ {\isachardoublequoteopen}{\isachardot}{\kern0pt}{\isachardot}{\kern0pt}{\isachardot}{\kern0pt}\ {\isasymlongleftrightarrow}\ v\ {\isacharequal}{\kern0pt}\ {\isacharless}{\kern0pt}p{\isacharcomma}{\kern0pt}\ Fn{\isacharunderscore}{\kern0pt}perm{\isacharparenleft}{\kern0pt}f{\isacharcomma}{\kern0pt}\ p{\isacharparenright}{\kern0pt}{\isachargreater}{\kern0pt}{\isachardoublequoteclose}\isanewline
\ \ \ \ \isacommand{using}\isamarkupfalse%
\ Fn{\isacharunderscore}{\kern0pt}perm{\isacharunderscore}{\kern0pt}in{\isacharunderscore}{\kern0pt}M\ assms\isanewline
\ \ \ \ \isacommand{by}\isamarkupfalse%
\ auto\isanewline
\ \ \isacommand{show}\isamarkupfalse%
\ {\isacharquery}{\kern0pt}thesis\ \isacommand{using}\isamarkupfalse%
\ I{\isadigit{1}}\ I{\isadigit{2}}\ \isacommand{by}\isamarkupfalse%
\ auto\isanewline
\isacommand{qed}\isamarkupfalse%
%
\endisatagproof
{\isafoldproof}%
%
\isadelimproof
\isanewline
%
\endisadelimproof
\isanewline
\isacommand{lemma}\isamarkupfalse%
\ Fn{\isacharunderscore}{\kern0pt}perm{\isacharprime}{\kern0pt}{\isacharunderscore}{\kern0pt}subset\ {\isacharcolon}{\kern0pt}\ \isanewline
\ \ \isakeyword{fixes}\ f\ \isanewline
\ \ \isakeyword{assumes}\ {\isachardoublequoteopen}f\ {\isasymin}\ nat{\isacharunderscore}{\kern0pt}perms{\isachardoublequoteclose}\ \isanewline
\ \ \isakeyword{shows}\ {\isachardoublequoteopen}Fn{\isacharunderscore}{\kern0pt}perm{\isacharprime}{\kern0pt}{\isacharparenleft}{\kern0pt}f{\isacharparenright}{\kern0pt}\ {\isasymsubseteq}\ Fn\ {\isasymtimes}\ {\isacharparenleft}{\kern0pt}Pow{\isacharparenleft}{\kern0pt}{\isacharparenleft}{\kern0pt}nat\ {\isasymtimes}\ nat{\isacharparenright}{\kern0pt}\ {\isasymtimes}\ {\isadigit{2}}{\isacharparenright}{\kern0pt}\ {\isasyminter}\ M{\isacharparenright}{\kern0pt}{\isachardoublequoteclose}\ \isanewline
%
\isadelimproof
\ \ %
\endisadelimproof
%
\isatagproof
\isacommand{unfolding}\isamarkupfalse%
\ Fn{\isacharunderscore}{\kern0pt}perm{\isacharprime}{\kern0pt}{\isacharunderscore}{\kern0pt}def\ \isanewline
\ \ \isacommand{apply}\isamarkupfalse%
\ {\isacharparenleft}{\kern0pt}rule\ subsetI{\isacharcomma}{\kern0pt}\ clarsimp{\isacharcomma}{\kern0pt}\ rule\ conjI{\isacharparenright}{\kern0pt}\isanewline
\ \ \ \isacommand{apply}\isamarkupfalse%
{\isacharparenleft}{\kern0pt}rule\ subsetI{\isacharcomma}{\kern0pt}\ rename{\isacharunderscore}{\kern0pt}tac\ p\ x{\isacharcomma}{\kern0pt}\ subgoal{\isacharunderscore}{\kern0pt}tac\ {\isachardoublequoteopen}{\isasymexists}n\ {\isasymin}\ nat{\isachardot}{\kern0pt}\ {\isasymexists}m\ {\isasymin}\ nat{\isachardot}{\kern0pt}\ {\isasymexists}l\ {\isasymin}\ {\isadigit{2}}{\isachardot}{\kern0pt}\ {\isacharless}{\kern0pt}{\isacharless}{\kern0pt}n{\isacharcomma}{\kern0pt}\ m{\isachargreater}{\kern0pt}{\isacharcomma}{\kern0pt}\ l{\isachargreater}{\kern0pt}\ {\isasymin}\ p\ {\isasymand}\ x\ {\isacharequal}{\kern0pt}\ {\isacharless}{\kern0pt}{\isacharless}{\kern0pt}f{\isacharbackquote}{\kern0pt}n{\isacharcomma}{\kern0pt}\ m{\isachargreater}{\kern0pt}{\isacharcomma}{\kern0pt}\ l{\isachargreater}{\kern0pt}{\isachardoublequoteclose}{\isacharparenright}{\kern0pt}\isanewline
\ \ \ \ \isacommand{apply}\isamarkupfalse%
{\isacharparenleft}{\kern0pt}clarify{\isacharcomma}{\kern0pt}\ rule\ function{\isacharunderscore}{\kern0pt}value{\isacharunderscore}{\kern0pt}in{\isacharparenright}{\kern0pt}\isanewline
\ \ \isacommand{using}\isamarkupfalse%
\ assms\ nat{\isacharunderscore}{\kern0pt}perms{\isacharunderscore}{\kern0pt}def\ bij{\isacharunderscore}{\kern0pt}def\ inj{\isacharunderscore}{\kern0pt}def\ \isanewline
\ \ \ \ \ \isacommand{apply}\isamarkupfalse%
\ auto{\isacharbrackleft}{\kern0pt}{\isadigit{2}}{\isacharbrackright}{\kern0pt}\isanewline
\ \ \ \isacommand{apply}\isamarkupfalse%
{\isacharparenleft}{\kern0pt}rule\ Fn{\isacharunderscore}{\kern0pt}permE{\isacharparenright}{\kern0pt}\isanewline
\ \ \isacommand{using}\isamarkupfalse%
\ assms\ Fn{\isacharunderscore}{\kern0pt}perm{\isacharunderscore}{\kern0pt}in{\isacharunderscore}{\kern0pt}M\isanewline
\ \ \isacommand{by}\isamarkupfalse%
\ auto%
\endisatagproof
{\isafoldproof}%
%
\isadelimproof
\isanewline
%
\endisadelimproof
\isanewline
\isacommand{lemma}\isamarkupfalse%
\ Fn{\isacharunderscore}{\kern0pt}perm{\isacharprime}{\kern0pt}{\isacharunderscore}{\kern0pt}in{\isacharunderscore}{\kern0pt}M\ {\isacharcolon}{\kern0pt}\ \isanewline
\ \ \isakeyword{fixes}\ f\ \isanewline
\ \ \isakeyword{assumes}\ {\isachardoublequoteopen}f\ {\isasymin}\ nat{\isacharunderscore}{\kern0pt}perms{\isachardoublequoteclose}\ \isanewline
\ \ \isakeyword{shows}\ {\isachardoublequoteopen}Fn{\isacharunderscore}{\kern0pt}perm{\isacharprime}{\kern0pt}{\isacharparenleft}{\kern0pt}f{\isacharparenright}{\kern0pt}\ {\isasymin}\ M{\isachardoublequoteclose}\ \isanewline
%
\isadelimproof
%
\endisadelimproof
%
\isatagproof
\isacommand{proof}\isamarkupfalse%
\ {\isacharminus}{\kern0pt}\ \ \ \ \isanewline
\isanewline
\ \ \isacommand{define}\isamarkupfalse%
\ X\ \isakeyword{where}\ {\isachardoublequoteopen}X\ {\isasymequiv}\ {\isacharbraceleft}{\kern0pt}\ v\ {\isasymin}\ Fn\ {\isasymtimes}\ {\isacharparenleft}{\kern0pt}Pow{\isacharparenleft}{\kern0pt}{\isacharparenleft}{\kern0pt}nat\ {\isasymtimes}\ nat{\isacharparenright}{\kern0pt}\ {\isasymtimes}\ {\isadigit{2}}{\isacharparenright}{\kern0pt}\ {\isasyminter}\ M{\isacharparenright}{\kern0pt}{\isachardot}{\kern0pt}\ sats{\isacharparenleft}{\kern0pt}M{\isacharcomma}{\kern0pt}\ Exists{\isacharparenleft}{\kern0pt}And{\isacharparenleft}{\kern0pt}Member{\isacharparenleft}{\kern0pt}{\isadigit{0}}{\isacharcomma}{\kern0pt}\ {\isadigit{2}}{\isacharparenright}{\kern0pt}{\isacharcomma}{\kern0pt}\ is{\isacharunderscore}{\kern0pt}Fn{\isacharunderscore}{\kern0pt}perm{\isacharprime}{\kern0pt}{\isacharunderscore}{\kern0pt}elem{\isacharunderscore}{\kern0pt}fm{\isacharparenleft}{\kern0pt}{\isadigit{3}}{\isacharcomma}{\kern0pt}\ {\isadigit{0}}{\isacharcomma}{\kern0pt}\ {\isadigit{1}}{\isacharparenright}{\kern0pt}{\isacharparenright}{\kern0pt}{\isacharparenright}{\kern0pt}{\isacharcomma}{\kern0pt}\ {\isacharbrackleft}{\kern0pt}v{\isacharbrackright}{\kern0pt}\ {\isacharat}{\kern0pt}\ {\isacharbrackleft}{\kern0pt}Fn{\isacharcomma}{\kern0pt}\ f{\isacharbrackright}{\kern0pt}{\isacharparenright}{\kern0pt}\ {\isacharbraceright}{\kern0pt}{\isachardoublequoteclose}\ \isanewline
\isanewline
\ \ \isacommand{have}\isamarkupfalse%
\ {\isachardoublequoteopen}X\ {\isasymin}\ M{\isachardoublequoteclose}\isanewline
\ \ \ \ \isacommand{apply}\isamarkupfalse%
{\isacharparenleft}{\kern0pt}subgoal{\isacharunderscore}{\kern0pt}tac\ {\isachardoublequoteopen}is{\isacharunderscore}{\kern0pt}Fn{\isacharunderscore}{\kern0pt}perm{\isacharprime}{\kern0pt}{\isacharunderscore}{\kern0pt}elem{\isacharunderscore}{\kern0pt}fm{\isacharparenleft}{\kern0pt}{\isadigit{3}}{\isacharcomma}{\kern0pt}\ {\isadigit{0}}{\isacharcomma}{\kern0pt}\ {\isadigit{1}}{\isacharparenright}{\kern0pt}\ {\isasymin}\ formula{\isachardoublequoteclose}{\isacharparenright}{\kern0pt}\isanewline
\ \ \ \ \isacommand{unfolding}\isamarkupfalse%
\ X{\isacharunderscore}{\kern0pt}def\isanewline
\ \ \ \ \isacommand{apply}\isamarkupfalse%
{\isacharparenleft}{\kern0pt}rule\ separation{\isacharunderscore}{\kern0pt}notation{\isacharcomma}{\kern0pt}\ rule\ separation{\isacharunderscore}{\kern0pt}ax{\isacharparenright}{\kern0pt}\isanewline
\ \ \ \ \isacommand{using}\isamarkupfalse%
\ assms\ nat{\isacharunderscore}{\kern0pt}perms{\isacharunderscore}{\kern0pt}in{\isacharunderscore}{\kern0pt}M\ transM\ Fn{\isacharunderscore}{\kern0pt}in{\isacharunderscore}{\kern0pt}M\isanewline
\ \ \ \ \ \ \ \ \isacommand{apply}\isamarkupfalse%
\ auto{\isacharbrackleft}{\kern0pt}{\isadigit{2}}{\isacharbrackright}{\kern0pt}\isanewline
\ \ \ \ \ \ \isacommand{apply}\isamarkupfalse%
\ simp\isanewline
\ \ \ \ \ \ \isacommand{apply}\isamarkupfalse%
{\isacharparenleft}{\kern0pt}rule\ pred{\isacharunderscore}{\kern0pt}le{\isacharcomma}{\kern0pt}\ simp{\isacharcomma}{\kern0pt}\ simp{\isacharparenright}{\kern0pt}\isanewline
\ \ \ \ \ \ \isacommand{apply}\isamarkupfalse%
{\isacharparenleft}{\kern0pt}rule\ Un{\isacharunderscore}{\kern0pt}least{\isacharunderscore}{\kern0pt}lt{\isacharparenright}{\kern0pt}{\isacharplus}{\kern0pt}\isanewline
\ \ \ \ \ \ \ \ \isacommand{apply}\isamarkupfalse%
\ auto{\isacharbrackleft}{\kern0pt}{\isadigit{3}}{\isacharbrackright}{\kern0pt}\isanewline
\ \ \ \ \ \ \isacommand{apply}\isamarkupfalse%
{\isacharparenleft}{\kern0pt}rule\ le{\isacharunderscore}{\kern0pt}trans{\isacharcomma}{\kern0pt}\ rule\ arity{\isacharunderscore}{\kern0pt}is{\isacharunderscore}{\kern0pt}Fn{\isacharunderscore}{\kern0pt}perm{\isacharprime}{\kern0pt}{\isacharunderscore}{\kern0pt}elem{\isacharunderscore}{\kern0pt}fm{\isacharparenright}{\kern0pt}\isanewline
\ \ \ \ \isacommand{using}\isamarkupfalse%
\ Un{\isacharunderscore}{\kern0pt}least{\isacharunderscore}{\kern0pt}lt\isanewline
\ \ \ \ \ \ \ \ \ \isacommand{apply}\isamarkupfalse%
\ auto{\isacharbrackleft}{\kern0pt}{\isadigit{4}}{\isacharbrackright}{\kern0pt}\isanewline
\ \ \ \ \ \isacommand{apply}\isamarkupfalse%
{\isacharparenleft}{\kern0pt}rule\ to{\isacharunderscore}{\kern0pt}rin{\isacharcomma}{\kern0pt}\ rule\ cartprod{\isacharunderscore}{\kern0pt}closed{\isacharcomma}{\kern0pt}\ simp\ add{\isacharcolon}{\kern0pt}Fn{\isacharunderscore}{\kern0pt}in{\isacharunderscore}{\kern0pt}M{\isacharcomma}{\kern0pt}\ simp{\isacharcomma}{\kern0pt}\ rule\ M{\isacharunderscore}{\kern0pt}powerset{\isacharparenright}{\kern0pt}\isanewline
\ \ \ \ \isacommand{using}\isamarkupfalse%
\ cartprod{\isacharunderscore}{\kern0pt}closed\ nat{\isacharunderscore}{\kern0pt}in{\isacharunderscore}{\kern0pt}M\ zero{\isacharunderscore}{\kern0pt}in{\isacharunderscore}{\kern0pt}M\ succ{\isacharunderscore}{\kern0pt}in{\isacharunderscore}{\kern0pt}MI\ is{\isacharunderscore}{\kern0pt}Fn{\isacharunderscore}{\kern0pt}perm{\isacharprime}{\kern0pt}{\isacharunderscore}{\kern0pt}elem{\isacharunderscore}{\kern0pt}fm{\isacharunderscore}{\kern0pt}type\isanewline
\ \ \ \ \isacommand{by}\isamarkupfalse%
\ auto\isanewline
\isanewline
\ \ \isacommand{have}\isamarkupfalse%
\ {\isachardoublequoteopen}X\ {\isacharequal}{\kern0pt}\ {\isacharbraceleft}{\kern0pt}\ v\ {\isasymin}\ Fn\ {\isasymtimes}\ {\isacharparenleft}{\kern0pt}Pow{\isacharparenleft}{\kern0pt}{\isacharparenleft}{\kern0pt}nat\ {\isasymtimes}\ nat{\isacharparenright}{\kern0pt}\ {\isasymtimes}\ {\isadigit{2}}{\isacharparenright}{\kern0pt}\ {\isasyminter}\ M{\isacharparenright}{\kern0pt}{\isachardot}{\kern0pt}\ {\isasymexists}p\ {\isasymin}\ M{\isachardot}{\kern0pt}\ p\ {\isasymin}\ Fn\ {\isasymand}\ v\ {\isacharequal}{\kern0pt}\ {\isacharless}{\kern0pt}p{\isacharcomma}{\kern0pt}\ Fn{\isacharunderscore}{\kern0pt}perm{\isacharparenleft}{\kern0pt}f{\isacharcomma}{\kern0pt}\ p{\isacharparenright}{\kern0pt}{\isachargreater}{\kern0pt}\ {\isacharbraceright}{\kern0pt}{\isachardoublequoteclose}\isanewline
\ \ \ \ \isacommand{unfolding}\isamarkupfalse%
\ X{\isacharunderscore}{\kern0pt}def\ \isanewline
\ \ \ \ \isacommand{apply}\isamarkupfalse%
{\isacharparenleft}{\kern0pt}rule\ iff{\isacharunderscore}{\kern0pt}eq{\isacharparenright}{\kern0pt}\isanewline
\ \ \ \ \isacommand{apply}\isamarkupfalse%
{\isacharparenleft}{\kern0pt}rule\ iff{\isacharunderscore}{\kern0pt}trans{\isacharcomma}{\kern0pt}\ rule\ sats{\isacharunderscore}{\kern0pt}Exists{\isacharunderscore}{\kern0pt}iff{\isacharparenright}{\kern0pt}\isanewline
\ \ \ \ \isacommand{using}\isamarkupfalse%
\ Fn{\isacharunderscore}{\kern0pt}in{\isacharunderscore}{\kern0pt}M\ transM\ nat{\isacharunderscore}{\kern0pt}perms{\isacharunderscore}{\kern0pt}in{\isacharunderscore}{\kern0pt}M\ assms\ pair{\isacharunderscore}{\kern0pt}in{\isacharunderscore}{\kern0pt}M{\isacharunderscore}{\kern0pt}iff\isanewline
\ \ \ \ \ \isacommand{apply}\isamarkupfalse%
\ auto{\isacharbrackleft}{\kern0pt}{\isadigit{1}}{\isacharbrackright}{\kern0pt}\isanewline
\ \ \ \ \isacommand{apply}\isamarkupfalse%
{\isacharparenleft}{\kern0pt}rule\ bex{\isacharunderscore}{\kern0pt}iff{\isacharcomma}{\kern0pt}\ rule\ iff{\isacharunderscore}{\kern0pt}trans{\isacharcomma}{\kern0pt}\ rule\ sats{\isacharunderscore}{\kern0pt}And{\isacharunderscore}{\kern0pt}iff{\isacharparenright}{\kern0pt}\isanewline
\ \ \ \ \isacommand{using}\isamarkupfalse%
\ Fn{\isacharunderscore}{\kern0pt}in{\isacharunderscore}{\kern0pt}M\ transM\ nat{\isacharunderscore}{\kern0pt}perms{\isacharunderscore}{\kern0pt}in{\isacharunderscore}{\kern0pt}M\ assms\ pair{\isacharunderscore}{\kern0pt}in{\isacharunderscore}{\kern0pt}M{\isacharunderscore}{\kern0pt}iff\isanewline
\ \ \ \ \ \isacommand{apply}\isamarkupfalse%
\ auto{\isacharbrackleft}{\kern0pt}{\isadigit{1}}{\isacharbrackright}{\kern0pt}\isanewline
\ \ \ \ \isacommand{apply}\isamarkupfalse%
{\isacharparenleft}{\kern0pt}rule\ iff{\isacharunderscore}{\kern0pt}conjI{\isadigit{2}}{\isacharparenright}{\kern0pt}\isanewline
\ \ \ \ \isacommand{using}\isamarkupfalse%
\ Fn{\isacharunderscore}{\kern0pt}in{\isacharunderscore}{\kern0pt}M\ transM\ nat{\isacharunderscore}{\kern0pt}perms{\isacharunderscore}{\kern0pt}in{\isacharunderscore}{\kern0pt}M\ assms\ pair{\isacharunderscore}{\kern0pt}in{\isacharunderscore}{\kern0pt}M{\isacharunderscore}{\kern0pt}iff\isanewline
\ \ \ \ \ \isacommand{apply}\isamarkupfalse%
\ auto{\isacharbrackleft}{\kern0pt}{\isadigit{1}}{\isacharbrackright}{\kern0pt}\isanewline
\ \ \ \ \isacommand{apply}\isamarkupfalse%
{\isacharparenleft}{\kern0pt}rule\ sats{\isacharunderscore}{\kern0pt}is{\isacharunderscore}{\kern0pt}Fn{\isacharunderscore}{\kern0pt}perm{\isacharprime}{\kern0pt}{\isacharunderscore}{\kern0pt}elem{\isacharunderscore}{\kern0pt}fm{\isacharunderscore}{\kern0pt}iff{\isacharparenright}{\kern0pt}\isanewline
\ \ \ \ \isacommand{using}\isamarkupfalse%
\ Fn{\isacharunderscore}{\kern0pt}in{\isacharunderscore}{\kern0pt}M\ transM\ nat{\isacharunderscore}{\kern0pt}perms{\isacharunderscore}{\kern0pt}in{\isacharunderscore}{\kern0pt}M\ assms\ pair{\isacharunderscore}{\kern0pt}in{\isacharunderscore}{\kern0pt}M{\isacharunderscore}{\kern0pt}iff\isanewline
\ \ \ \ \isacommand{by}\isamarkupfalse%
\ auto\isanewline
\isanewline
\ \ \isacommand{also}\isamarkupfalse%
\ \isacommand{have}\isamarkupfalse%
\ {\isachardoublequoteopen}{\isachardot}{\kern0pt}{\isachardot}{\kern0pt}{\isachardot}{\kern0pt}\ {\isacharequal}{\kern0pt}\ Fn{\isacharunderscore}{\kern0pt}perm{\isacharprime}{\kern0pt}{\isacharparenleft}{\kern0pt}f{\isacharparenright}{\kern0pt}{\isachardoublequoteclose}\ \isanewline
\ \ \ \ \isacommand{apply}\isamarkupfalse%
{\isacharparenleft}{\kern0pt}rule\ equality{\isacharunderscore}{\kern0pt}iffI{\isacharcomma}{\kern0pt}\ rule\ iffI{\isacharparenright}{\kern0pt}\isanewline
\ \ \ \ \isacommand{apply}\isamarkupfalse%
{\isacharparenleft}{\kern0pt}subst\ Fn{\isacharunderscore}{\kern0pt}perm{\isacharprime}{\kern0pt}{\isacharunderscore}{\kern0pt}def{\isacharparenright}{\kern0pt}\isanewline
\ \ \ \ \isacommand{using}\isamarkupfalse%
\ Fn{\isacharunderscore}{\kern0pt}perm{\isacharprime}{\kern0pt}{\isacharunderscore}{\kern0pt}subset\ assms\ Fn{\isacharunderscore}{\kern0pt}in{\isacharunderscore}{\kern0pt}M\ transM\ \isanewline
\ \ \ \ \isacommand{unfolding}\isamarkupfalse%
\ Fn{\isacharunderscore}{\kern0pt}perm{\isacharprime}{\kern0pt}{\isacharunderscore}{\kern0pt}def\isanewline
\ \ \ \ \isacommand{by}\isamarkupfalse%
\ auto\isanewline
\isanewline
\ \ \isacommand{finally}\isamarkupfalse%
\ \isacommand{show}\isamarkupfalse%
\ {\isacharquery}{\kern0pt}thesis\ \isacommand{using}\isamarkupfalse%
\ {\isacartoucheopen}X\ {\isasymin}\ M{\isacartoucheclose}\ \isacommand{by}\isamarkupfalse%
\ auto\isanewline
\isacommand{qed}\isamarkupfalse%
%
\endisatagproof
{\isafoldproof}%
%
\isadelimproof
\isanewline
%
\endisadelimproof
\isanewline
\isacommand{definition}\isamarkupfalse%
\ is{\isacharunderscore}{\kern0pt}Fn{\isacharunderscore}{\kern0pt}perm{\isacharprime}{\kern0pt}{\isacharunderscore}{\kern0pt}fm\ \isakeyword{where}\isanewline
\ \ {\isachardoublequoteopen}is{\isacharunderscore}{\kern0pt}Fn{\isacharunderscore}{\kern0pt}perm{\isacharprime}{\kern0pt}{\isacharunderscore}{\kern0pt}fm{\isacharparenleft}{\kern0pt}f{\isacharcomma}{\kern0pt}\ fn{\isacharcomma}{\kern0pt}\ v{\isacharparenright}{\kern0pt}\ {\isasymequiv}\ Forall{\isacharparenleft}{\kern0pt}Iff{\isacharparenleft}{\kern0pt}Member{\isacharparenleft}{\kern0pt}{\isadigit{0}}{\isacharcomma}{\kern0pt}\ v\ {\isacharhash}{\kern0pt}{\isacharplus}{\kern0pt}\ {\isadigit{1}}{\isacharparenright}{\kern0pt}{\isacharcomma}{\kern0pt}\ Exists{\isacharparenleft}{\kern0pt}And{\isacharparenleft}{\kern0pt}Member{\isacharparenleft}{\kern0pt}{\isadigit{0}}{\isacharcomma}{\kern0pt}\ fn{\isacharhash}{\kern0pt}{\isacharplus}{\kern0pt}{\isadigit{2}}{\isacharparenright}{\kern0pt}{\isacharcomma}{\kern0pt}\ is{\isacharunderscore}{\kern0pt}Fn{\isacharunderscore}{\kern0pt}perm{\isacharprime}{\kern0pt}{\isacharunderscore}{\kern0pt}elem{\isacharunderscore}{\kern0pt}fm{\isacharparenleft}{\kern0pt}f{\isacharhash}{\kern0pt}{\isacharplus}{\kern0pt}{\isadigit{2}}{\isacharcomma}{\kern0pt}\ {\isadigit{0}}{\isacharcomma}{\kern0pt}\ {\isadigit{1}}{\isacharparenright}{\kern0pt}{\isacharparenright}{\kern0pt}{\isacharparenright}{\kern0pt}{\isacharparenright}{\kern0pt}{\isacharparenright}{\kern0pt}{\isachardoublequoteclose}\ \isanewline
\isanewline
\isacommand{lemma}\isamarkupfalse%
\ is{\isacharunderscore}{\kern0pt}Fn{\isacharunderscore}{\kern0pt}perm{\isacharprime}{\kern0pt}{\isacharunderscore}{\kern0pt}fm{\isacharunderscore}{\kern0pt}type\ {\isacharcolon}{\kern0pt}\ \isanewline
\ \ \isakeyword{fixes}\ i\ j\ k\ \isanewline
\ \ \isakeyword{assumes}\ {\isachardoublequoteopen}i\ {\isasymin}\ nat{\isachardoublequoteclose}\ {\isachardoublequoteopen}j\ {\isasymin}\ nat{\isachardoublequoteclose}\ {\isachardoublequoteopen}k\ {\isasymin}\ nat{\isachardoublequoteclose}\ \isanewline
\ \ \isakeyword{shows}\ {\isachardoublequoteopen}is{\isacharunderscore}{\kern0pt}Fn{\isacharunderscore}{\kern0pt}perm{\isacharprime}{\kern0pt}{\isacharunderscore}{\kern0pt}fm{\isacharparenleft}{\kern0pt}i{\isacharcomma}{\kern0pt}\ j{\isacharcomma}{\kern0pt}\ k{\isacharparenright}{\kern0pt}\ {\isasymin}\ formula{\isachardoublequoteclose}\ \isanewline
%
\isadelimproof
\ \ %
\endisadelimproof
%
\isatagproof
\isacommand{unfolding}\isamarkupfalse%
\ is{\isacharunderscore}{\kern0pt}Fn{\isacharunderscore}{\kern0pt}perm{\isacharprime}{\kern0pt}{\isacharunderscore}{\kern0pt}fm{\isacharunderscore}{\kern0pt}def\ \isanewline
\ \ \isacommand{apply}\isamarkupfalse%
{\isacharparenleft}{\kern0pt}subgoal{\isacharunderscore}{\kern0pt}tac\ {\isachardoublequoteopen}is{\isacharunderscore}{\kern0pt}Fn{\isacharunderscore}{\kern0pt}perm{\isacharprime}{\kern0pt}{\isacharunderscore}{\kern0pt}elem{\isacharunderscore}{\kern0pt}fm{\isacharparenleft}{\kern0pt}i\ {\isacharhash}{\kern0pt}{\isacharplus}{\kern0pt}\ {\isadigit{2}}{\isacharcomma}{\kern0pt}\ {\isadigit{0}}{\isacharcomma}{\kern0pt}\ {\isadigit{1}}{\isacharparenright}{\kern0pt}\ {\isasymin}\ formula{\isachardoublequoteclose}{\isacharparenright}{\kern0pt}\isanewline
\ \ \isacommand{using}\isamarkupfalse%
\ assms\isanewline
\ \ \ \isacommand{apply}\isamarkupfalse%
\ force\ \isanewline
\ \ \isacommand{apply}\isamarkupfalse%
{\isacharparenleft}{\kern0pt}rule\ is{\isacharunderscore}{\kern0pt}Fn{\isacharunderscore}{\kern0pt}perm{\isacharprime}{\kern0pt}{\isacharunderscore}{\kern0pt}elem{\isacharunderscore}{\kern0pt}fm{\isacharunderscore}{\kern0pt}type{\isacharparenright}{\kern0pt}\isanewline
\ \ \isacommand{using}\isamarkupfalse%
\ assms\isanewline
\ \ \isacommand{by}\isamarkupfalse%
\ auto%
\endisatagproof
{\isafoldproof}%
%
\isadelimproof
\isanewline
%
\endisadelimproof
\isanewline
\isacommand{lemma}\isamarkupfalse%
\ arity{\isacharunderscore}{\kern0pt}is{\isacharunderscore}{\kern0pt}Fn{\isacharunderscore}{\kern0pt}perm{\isacharprime}{\kern0pt}{\isacharunderscore}{\kern0pt}fm\ {\isacharcolon}{\kern0pt}\ \isanewline
\ \ \isakeyword{fixes}\ i\ j\ k\ \isanewline
\ \ \isakeyword{assumes}\ {\isachardoublequoteopen}i\ {\isasymin}\ nat{\isachardoublequoteclose}\ {\isachardoublequoteopen}j\ {\isasymin}\ nat{\isachardoublequoteclose}\ {\isachardoublequoteopen}k\ {\isasymin}\ nat{\isachardoublequoteclose}\ \isanewline
\ \ \isakeyword{shows}\ {\isachardoublequoteopen}arity{\isacharparenleft}{\kern0pt}is{\isacharunderscore}{\kern0pt}Fn{\isacharunderscore}{\kern0pt}perm{\isacharprime}{\kern0pt}{\isacharunderscore}{\kern0pt}fm{\isacharparenleft}{\kern0pt}i{\isacharcomma}{\kern0pt}\ j{\isacharcomma}{\kern0pt}\ k{\isacharparenright}{\kern0pt}{\isacharparenright}{\kern0pt}\ {\isasymle}\ succ{\isacharparenleft}{\kern0pt}i{\isacharparenright}{\kern0pt}\ {\isasymunion}\ succ{\isacharparenleft}{\kern0pt}j{\isacharparenright}{\kern0pt}\ {\isasymunion}\ succ{\isacharparenleft}{\kern0pt}k{\isacharparenright}{\kern0pt}{\isachardoublequoteclose}\isanewline
%
\isadelimproof
\isanewline
\ \ %
\endisadelimproof
%
\isatagproof
\isacommand{unfolding}\isamarkupfalse%
\ is{\isacharunderscore}{\kern0pt}Fn{\isacharunderscore}{\kern0pt}perm{\isacharprime}{\kern0pt}{\isacharunderscore}{\kern0pt}fm{\isacharunderscore}{\kern0pt}def\isanewline
\ \ \isacommand{apply}\isamarkupfalse%
{\isacharparenleft}{\kern0pt}subgoal{\isacharunderscore}{\kern0pt}tac\ {\isachardoublequoteopen}is{\isacharunderscore}{\kern0pt}Fn{\isacharunderscore}{\kern0pt}perm{\isacharprime}{\kern0pt}{\isacharunderscore}{\kern0pt}elem{\isacharunderscore}{\kern0pt}fm{\isacharparenleft}{\kern0pt}i\ {\isacharhash}{\kern0pt}{\isacharplus}{\kern0pt}\ {\isadigit{2}}{\isacharcomma}{\kern0pt}\ {\isadigit{0}}{\isacharcomma}{\kern0pt}\ {\isadigit{1}}{\isacharparenright}{\kern0pt}\ {\isasymin}\ formula{\isachardoublequoteclose}{\isacharparenright}{\kern0pt}\isanewline
\ \ \isacommand{using}\isamarkupfalse%
\ assms\isanewline
\ \ \isacommand{apply}\isamarkupfalse%
\ simp\isanewline
\ \ \ \isacommand{apply}\isamarkupfalse%
{\isacharparenleft}{\kern0pt}rule\ pred{\isacharunderscore}{\kern0pt}le{\isacharcomma}{\kern0pt}\ simp{\isacharcomma}{\kern0pt}\ simp{\isacharparenright}{\kern0pt}\isanewline
\ \ \ \isacommand{apply}\isamarkupfalse%
{\isacharparenleft}{\kern0pt}rule\ Un{\isacharunderscore}{\kern0pt}least{\isacharunderscore}{\kern0pt}lt{\isacharparenright}{\kern0pt}{\isacharplus}{\kern0pt}\isanewline
\ \ \ \ \ \isacommand{apply}\isamarkupfalse%
\ {\isacharparenleft}{\kern0pt}simp{\isacharcomma}{\kern0pt}\ simp{\isacharparenright}{\kern0pt}\isanewline
\ \ \ \ \isacommand{apply}\isamarkupfalse%
{\isacharparenleft}{\kern0pt}rule\ ltI{\isacharcomma}{\kern0pt}\ simp{\isacharcomma}{\kern0pt}\ simp{\isacharparenright}{\kern0pt}\isanewline
\ \ \ \isacommand{apply}\isamarkupfalse%
{\isacharparenleft}{\kern0pt}rule\ pred{\isacharunderscore}{\kern0pt}le{\isacharcomma}{\kern0pt}\ simp{\isacharcomma}{\kern0pt}\ simp{\isacharparenright}{\kern0pt}\isanewline
\ \ \ \isacommand{apply}\isamarkupfalse%
{\isacharparenleft}{\kern0pt}rule\ Un{\isacharunderscore}{\kern0pt}least{\isacharunderscore}{\kern0pt}lt{\isacharparenright}{\kern0pt}{\isacharplus}{\kern0pt}\isanewline
\ \ \ \ \ \isacommand{apply}\isamarkupfalse%
\ {\isacharparenleft}{\kern0pt}simp{\isacharcomma}{\kern0pt}\ simp{\isacharparenright}{\kern0pt}\isanewline
\ \ \ \ \isacommand{apply}\isamarkupfalse%
{\isacharparenleft}{\kern0pt}rule\ ltI{\isacharcomma}{\kern0pt}\ simp{\isacharcomma}{\kern0pt}\ simp{\isacharparenright}{\kern0pt}\isanewline
\ \ \ \isacommand{apply}\isamarkupfalse%
{\isacharparenleft}{\kern0pt}rule\ le{\isacharunderscore}{\kern0pt}trans{\isacharcomma}{\kern0pt}\ rule\ arity{\isacharunderscore}{\kern0pt}is{\isacharunderscore}{\kern0pt}Fn{\isacharunderscore}{\kern0pt}perm{\isacharprime}{\kern0pt}{\isacharunderscore}{\kern0pt}elem{\isacharunderscore}{\kern0pt}fm{\isacharparenright}{\kern0pt}\isanewline
\ \ \isacommand{using}\isamarkupfalse%
\ assms\isanewline
\ \ \ \ \ \ \isacommand{apply}\isamarkupfalse%
\ auto{\isacharbrackleft}{\kern0pt}{\isadigit{3}}{\isacharbrackright}{\kern0pt}\isanewline
\ \ \ \isacommand{apply}\isamarkupfalse%
{\isacharparenleft}{\kern0pt}rule\ Un{\isacharunderscore}{\kern0pt}least{\isacharunderscore}{\kern0pt}lt{\isacharparenright}{\kern0pt}{\isacharplus}{\kern0pt}\isanewline
\ \ \ \ \ \isacommand{apply}\isamarkupfalse%
\ {\isacharparenleft}{\kern0pt}simp{\isacharcomma}{\kern0pt}\ rule\ ltI{\isacharcomma}{\kern0pt}\ simp{\isacharunderscore}{\kern0pt}all{\isacharparenright}{\kern0pt}\isanewline
\ \ \isacommand{apply}\isamarkupfalse%
{\isacharparenleft}{\kern0pt}rule\ is{\isacharunderscore}{\kern0pt}Fn{\isacharunderscore}{\kern0pt}perm{\isacharprime}{\kern0pt}{\isacharunderscore}{\kern0pt}elem{\isacharunderscore}{\kern0pt}fm{\isacharunderscore}{\kern0pt}type{\isacharparenright}{\kern0pt}\isanewline
\ \ \isacommand{using}\isamarkupfalse%
\ assms\isanewline
\ \ \isacommand{by}\isamarkupfalse%
\ auto%
\endisatagproof
{\isafoldproof}%
%
\isadelimproof
\isanewline
%
\endisadelimproof
\isanewline
\isacommand{lemma}\isamarkupfalse%
\ sats{\isacharunderscore}{\kern0pt}is{\isacharunderscore}{\kern0pt}Fn{\isacharunderscore}{\kern0pt}perm{\isacharprime}{\kern0pt}{\isacharunderscore}{\kern0pt}fm{\isacharunderscore}{\kern0pt}iff\ {\isacharcolon}{\kern0pt}\ \isanewline
\ \ \isakeyword{fixes}\ env\ i\ j\ k\ f\ v\ \isanewline
\ \ \isakeyword{assumes}\ {\isachardoublequoteopen}env\ {\isasymin}\ list{\isacharparenleft}{\kern0pt}M{\isacharparenright}{\kern0pt}{\isachardoublequoteclose}\ {\isachardoublequoteopen}i\ {\isacharless}{\kern0pt}\ length{\isacharparenleft}{\kern0pt}env{\isacharparenright}{\kern0pt}{\isachardoublequoteclose}\ {\isachardoublequoteopen}j\ {\isacharless}{\kern0pt}\ length{\isacharparenleft}{\kern0pt}env{\isacharparenright}{\kern0pt}{\isachardoublequoteclose}\ {\isachardoublequoteopen}k\ {\isacharless}{\kern0pt}\ length{\isacharparenleft}{\kern0pt}env{\isacharparenright}{\kern0pt}{\isachardoublequoteclose}\ \isanewline
\ \ \ \ \ \ \ \ \ \ {\isachardoublequoteopen}f\ {\isacharequal}{\kern0pt}\ nth{\isacharparenleft}{\kern0pt}i{\isacharcomma}{\kern0pt}\ env{\isacharparenright}{\kern0pt}{\isachardoublequoteclose}\ {\isachardoublequoteopen}Fn\ {\isacharequal}{\kern0pt}\ nth{\isacharparenleft}{\kern0pt}j{\isacharcomma}{\kern0pt}\ env{\isacharparenright}{\kern0pt}{\isachardoublequoteclose}\ {\isachardoublequoteopen}v\ {\isacharequal}{\kern0pt}\ nth{\isacharparenleft}{\kern0pt}k{\isacharcomma}{\kern0pt}\ env{\isacharparenright}{\kern0pt}{\isachardoublequoteclose}\ \isanewline
\ \ \ \ \ \ \ \ \ \ {\isachardoublequoteopen}f\ {\isasymin}\ nat{\isacharunderscore}{\kern0pt}perms{\isachardoublequoteclose}\isanewline
\ \ \isakeyword{shows}\ {\isachardoublequoteopen}sats{\isacharparenleft}{\kern0pt}M{\isacharcomma}{\kern0pt}\ is{\isacharunderscore}{\kern0pt}Fn{\isacharunderscore}{\kern0pt}perm{\isacharprime}{\kern0pt}{\isacharunderscore}{\kern0pt}fm{\isacharparenleft}{\kern0pt}i{\isacharcomma}{\kern0pt}\ j{\isacharcomma}{\kern0pt}\ k{\isacharparenright}{\kern0pt}{\isacharcomma}{\kern0pt}\ env{\isacharparenright}{\kern0pt}\ {\isasymlongleftrightarrow}\ v\ {\isacharequal}{\kern0pt}\ Fn{\isacharunderscore}{\kern0pt}perm{\isacharprime}{\kern0pt}{\isacharparenleft}{\kern0pt}f{\isacharparenright}{\kern0pt}{\isachardoublequoteclose}\isanewline
%
\isadelimproof
%
\endisadelimproof
%
\isatagproof
\isacommand{proof}\isamarkupfalse%
\ {\isacharminus}{\kern0pt}\ \isanewline
\ \ \isacommand{have}\isamarkupfalse%
\ I{\isadigit{1}}{\isacharcolon}{\kern0pt}\ {\isachardoublequoteopen}sats{\isacharparenleft}{\kern0pt}M{\isacharcomma}{\kern0pt}\ is{\isacharunderscore}{\kern0pt}Fn{\isacharunderscore}{\kern0pt}perm{\isacharprime}{\kern0pt}{\isacharunderscore}{\kern0pt}fm{\isacharparenleft}{\kern0pt}i{\isacharcomma}{\kern0pt}\ j{\isacharcomma}{\kern0pt}\ k{\isacharparenright}{\kern0pt}{\isacharcomma}{\kern0pt}\ env{\isacharparenright}{\kern0pt}\ {\isasymlongleftrightarrow}\ {\isacharparenleft}{\kern0pt}{\isasymforall}x\ {\isasymin}\ M{\isachardot}{\kern0pt}\ x\ {\isasymin}\ v\ {\isasymlongleftrightarrow}\ {\isacharparenleft}{\kern0pt}{\isasymexists}p\ {\isasymin}\ M{\isachardot}{\kern0pt}\ p\ {\isasymin}\ Fn\ {\isasymand}\ x\ {\isacharequal}{\kern0pt}\ {\isacharless}{\kern0pt}p{\isacharcomma}{\kern0pt}\ Fn{\isacharunderscore}{\kern0pt}perm{\isacharparenleft}{\kern0pt}f{\isacharcomma}{\kern0pt}\ p{\isacharparenright}{\kern0pt}{\isachargreater}{\kern0pt}{\isacharparenright}{\kern0pt}{\isacharparenright}{\kern0pt}{\isachardoublequoteclose}\ \isanewline
\ \ \ \ \isacommand{unfolding}\isamarkupfalse%
\ is{\isacharunderscore}{\kern0pt}Fn{\isacharunderscore}{\kern0pt}perm{\isacharprime}{\kern0pt}{\isacharunderscore}{\kern0pt}fm{\isacharunderscore}{\kern0pt}def\ \isanewline
\ \ \ \ \isacommand{apply}\isamarkupfalse%
{\isacharparenleft}{\kern0pt}rule\ iff{\isacharunderscore}{\kern0pt}trans{\isacharcomma}{\kern0pt}\ rule\ sats{\isacharunderscore}{\kern0pt}Forall{\isacharunderscore}{\kern0pt}iff{\isacharcomma}{\kern0pt}\ simp\ add{\isacharcolon}{\kern0pt}assms{\isacharcomma}{\kern0pt}\ rule\ ball{\isacharunderscore}{\kern0pt}iff{\isacharparenright}{\kern0pt}\isanewline
\ \ \ \ \isacommand{apply}\isamarkupfalse%
{\isacharparenleft}{\kern0pt}rule\ iff{\isacharunderscore}{\kern0pt}trans{\isacharcomma}{\kern0pt}\ rule\ sats{\isacharunderscore}{\kern0pt}Iff{\isacharunderscore}{\kern0pt}iff{\isacharcomma}{\kern0pt}\ simp\ add{\isacharcolon}{\kern0pt}assms{\isacharcomma}{\kern0pt}\ rule\ iff{\isacharunderscore}{\kern0pt}iff{\isacharparenright}{\kern0pt}\isanewline
\ \ \ \ \isacommand{using}\isamarkupfalse%
\ assms\ lt{\isacharunderscore}{\kern0pt}nat{\isacharunderscore}{\kern0pt}in{\isacharunderscore}{\kern0pt}nat\ nth{\isacharunderscore}{\kern0pt}type\isanewline
\ \ \ \ \ \isacommand{apply}\isamarkupfalse%
\ simp\isanewline
\ \ \ \ \isacommand{apply}\isamarkupfalse%
{\isacharparenleft}{\kern0pt}rule\ iff{\isacharunderscore}{\kern0pt}trans{\isacharcomma}{\kern0pt}\ rule\ sats{\isacharunderscore}{\kern0pt}Exists{\isacharunderscore}{\kern0pt}iff{\isacharcomma}{\kern0pt}\ simp\ add{\isacharcolon}{\kern0pt}assms{\isacharcomma}{\kern0pt}\ rule\ bex{\isacharunderscore}{\kern0pt}iff{\isacharparenright}{\kern0pt}\isanewline
\ \ \ \ \isacommand{apply}\isamarkupfalse%
{\isacharparenleft}{\kern0pt}rule\ iff{\isacharunderscore}{\kern0pt}trans{\isacharcomma}{\kern0pt}\ rule\ sats{\isacharunderscore}{\kern0pt}And{\isacharunderscore}{\kern0pt}iff{\isacharcomma}{\kern0pt}\ simp\ add{\isacharcolon}{\kern0pt}assms{\isacharcomma}{\kern0pt}\ rule\ iff{\isacharunderscore}{\kern0pt}conjI{\isadigit{2}}{\isacharparenright}{\kern0pt}\isanewline
\ \ \ \ \isacommand{using}\isamarkupfalse%
\ assms\ lt{\isacharunderscore}{\kern0pt}nat{\isacharunderscore}{\kern0pt}in{\isacharunderscore}{\kern0pt}nat\ nth{\isacharunderscore}{\kern0pt}type\isanewline
\ \ \ \ \ \isacommand{apply}\isamarkupfalse%
\ auto{\isacharbrackleft}{\kern0pt}{\isadigit{1}}{\isacharbrackright}{\kern0pt}\isanewline
\ \ \ \ \isacommand{apply}\isamarkupfalse%
{\isacharparenleft}{\kern0pt}rule\ sats{\isacharunderscore}{\kern0pt}is{\isacharunderscore}{\kern0pt}Fn{\isacharunderscore}{\kern0pt}perm{\isacharprime}{\kern0pt}{\isacharunderscore}{\kern0pt}elem{\isacharunderscore}{\kern0pt}fm{\isacharunderscore}{\kern0pt}iff{\isacharparenright}{\kern0pt}\isanewline
\ \ \ \ \isacommand{using}\isamarkupfalse%
\ assms\ lt{\isacharunderscore}{\kern0pt}nat{\isacharunderscore}{\kern0pt}in{\isacharunderscore}{\kern0pt}nat\ nth{\isacharunderscore}{\kern0pt}type\isanewline
\ \ \ \ \isacommand{by}\isamarkupfalse%
\ auto\isanewline
\ \ \isacommand{have}\isamarkupfalse%
\ I{\isadigit{2}}{\isacharcolon}{\kern0pt}\ {\isachardoublequoteopen}{\isachardot}{\kern0pt}{\isachardot}{\kern0pt}{\isachardot}{\kern0pt}\ {\isasymlongleftrightarrow}\ {\isacharparenleft}{\kern0pt}{\isasymforall}x{\isachardot}{\kern0pt}\ x\ {\isasymin}\ v\ {\isasymlongleftrightarrow}\ {\isacharparenleft}{\kern0pt}{\isasymexists}p\ {\isasymin}\ M{\isachardot}{\kern0pt}\ p\ {\isasymin}\ Fn\ {\isasymand}\ x\ {\isacharequal}{\kern0pt}\ {\isacharless}{\kern0pt}p{\isacharcomma}{\kern0pt}\ Fn{\isacharunderscore}{\kern0pt}perm{\isacharparenleft}{\kern0pt}f{\isacharcomma}{\kern0pt}\ p{\isacharparenright}{\kern0pt}{\isachargreater}{\kern0pt}{\isacharparenright}{\kern0pt}{\isacharparenright}{\kern0pt}{\isachardoublequoteclose}\ \isanewline
\ \ \ \ \isacommand{apply}\isamarkupfalse%
{\isacharparenleft}{\kern0pt}rule\ iffI{\isacharcomma}{\kern0pt}\ rule\ allI{\isacharcomma}{\kern0pt}\ rule\ iffI{\isacharparenright}{\kern0pt}\isanewline
\ \ \ \ \isacommand{using}\isamarkupfalse%
\ assms\ lt{\isacharunderscore}{\kern0pt}nat{\isacharunderscore}{\kern0pt}in{\isacharunderscore}{\kern0pt}nat\ nth{\isacharunderscore}{\kern0pt}type\ transM\isanewline
\ \ \ \ \ \ \isacommand{apply}\isamarkupfalse%
\ force\ \isanewline
\ \ \ \ \ \isacommand{apply}\isamarkupfalse%
\ {\isacharparenleft}{\kern0pt}rename{\isacharunderscore}{\kern0pt}tac\ x{\isacharcomma}{\kern0pt}\ subgoal{\isacharunderscore}{\kern0pt}tac\ {\isachardoublequoteopen}x\ {\isasymin}\ M{\isachardoublequoteclose}{\isacharcomma}{\kern0pt}\ force{\isacharparenright}{\kern0pt}\isanewline
\ \ \ \ \ \isacommand{apply}\isamarkupfalse%
\ clarsimp\isanewline
\ \ \ \ \isacommand{using}\isamarkupfalse%
\ Fn{\isacharunderscore}{\kern0pt}in{\isacharunderscore}{\kern0pt}M\ Fn{\isacharunderscore}{\kern0pt}perm{\isacharunderscore}{\kern0pt}in{\isacharunderscore}{\kern0pt}M\ assms\ pair{\isacharunderscore}{\kern0pt}in{\isacharunderscore}{\kern0pt}M{\isacharunderscore}{\kern0pt}iff\isanewline
\ \ \ \ \isacommand{by}\isamarkupfalse%
\ auto\isanewline
\ \ \isacommand{have}\isamarkupfalse%
\ I{\isadigit{3}}{\isacharcolon}{\kern0pt}\ {\isachardoublequoteopen}{\isachardot}{\kern0pt}{\isachardot}{\kern0pt}{\isachardot}{\kern0pt}\ {\isasymlongleftrightarrow}\ {\isacharparenleft}{\kern0pt}{\isasymforall}x{\isachardot}{\kern0pt}\ x\ {\isasymin}\ v\ {\isasymlongleftrightarrow}\ {\isacharparenleft}{\kern0pt}{\isasymexists}p\ {\isasymin}\ Fn{\isachardot}{\kern0pt}\ x\ {\isacharequal}{\kern0pt}\ {\isacharless}{\kern0pt}p{\isacharcomma}{\kern0pt}\ Fn{\isacharunderscore}{\kern0pt}perm{\isacharparenleft}{\kern0pt}f{\isacharcomma}{\kern0pt}\ p{\isacharparenright}{\kern0pt}{\isachargreater}{\kern0pt}{\isacharparenright}{\kern0pt}{\isacharparenright}{\kern0pt}{\isachardoublequoteclose}\ \isanewline
\ \ \ \ \isacommand{using}\isamarkupfalse%
\ Fn{\isacharunderscore}{\kern0pt}in{\isacharunderscore}{\kern0pt}M\ transM\ \isanewline
\ \ \ \ \isacommand{by}\isamarkupfalse%
\ force\ \isanewline
\ \ \isacommand{have}\isamarkupfalse%
\ I{\isadigit{4}}{\isacharcolon}{\kern0pt}\ {\isachardoublequoteopen}{\isachardot}{\kern0pt}{\isachardot}{\kern0pt}{\isachardot}{\kern0pt}\ {\isasymlongleftrightarrow}\ v\ {\isacharequal}{\kern0pt}\ Fn{\isacharunderscore}{\kern0pt}perm{\isacharprime}{\kern0pt}{\isacharparenleft}{\kern0pt}f{\isacharparenright}{\kern0pt}{\isachardoublequoteclose}\ \isanewline
\ \ \ \ \isacommand{unfolding}\isamarkupfalse%
\ Fn{\isacharunderscore}{\kern0pt}perm{\isacharprime}{\kern0pt}{\isacharunderscore}{\kern0pt}def\ \isanewline
\ \ \ \ \isacommand{by}\isamarkupfalse%
\ auto\isanewline
\ \ \isacommand{show}\isamarkupfalse%
\ {\isacharquery}{\kern0pt}thesis\isanewline
\ \ \ \ \isacommand{using}\isamarkupfalse%
\ I{\isadigit{1}}\ I{\isadigit{2}}\ I{\isadigit{3}}\ I{\isadigit{4}}\ \isanewline
\ \ \ \ \isacommand{by}\isamarkupfalse%
\ auto\ \isanewline
\isacommand{qed}\isamarkupfalse%
%
\endisatagproof
{\isafoldproof}%
%
\isadelimproof
\isanewline
%
\endisadelimproof
\isanewline
\isacommand{lemma}\isamarkupfalse%
\ Fn{\isacharunderscore}{\kern0pt}perms{\isacharunderscore}{\kern0pt}in{\isacharunderscore}{\kern0pt}M\ {\isacharcolon}{\kern0pt}\ {\isachardoublequoteopen}Fn{\isacharunderscore}{\kern0pt}perms\ {\isasymin}\ M{\isachardoublequoteclose}\ \isanewline
%
\isadelimproof
%
\endisadelimproof
%
\isatagproof
\isacommand{proof}\isamarkupfalse%
\ {\isacharminus}{\kern0pt}\ \isanewline
\ \ \isacommand{have}\isamarkupfalse%
\ {\isachardoublequoteopen}univalent{\isacharparenleft}{\kern0pt}{\isacharhash}{\kern0pt}{\isacharhash}{\kern0pt}M{\isacharcomma}{\kern0pt}\ nat{\isacharunderscore}{\kern0pt}perms{\isacharcomma}{\kern0pt}\ {\isasymlambda}x\ b{\isachardot}{\kern0pt}\ b\ {\isacharequal}{\kern0pt}\ Fn{\isacharunderscore}{\kern0pt}perm{\isacharprime}{\kern0pt}{\isacharparenleft}{\kern0pt}x{\isacharparenright}{\kern0pt}{\isacharparenright}{\kern0pt}{\isachardoublequoteclose}\ {\isacharparenleft}{\kern0pt}\isakeyword{is}\ {\isacharquery}{\kern0pt}B{\isacharparenright}{\kern0pt}\isanewline
\ \ \ \ \isacommand{unfolding}\isamarkupfalse%
\ univalent{\isacharunderscore}{\kern0pt}def\ \isanewline
\ \ \ \ \isacommand{by}\isamarkupfalse%
\ auto\isanewline
\ \ \isacommand{have}\isamarkupfalse%
\ {\isachardoublequoteopen}univalent{\isacharparenleft}{\kern0pt}{\isacharhash}{\kern0pt}{\isacharhash}{\kern0pt}M{\isacharcomma}{\kern0pt}\ nat{\isacharunderscore}{\kern0pt}perms{\isacharcomma}{\kern0pt}\ {\isasymlambda}x\ b{\isachardot}{\kern0pt}\ {\isacharparenleft}{\kern0pt}M{\isacharcomma}{\kern0pt}\ {\isacharbrackleft}{\kern0pt}x{\isacharcomma}{\kern0pt}\ b{\isacharbrackright}{\kern0pt}\ {\isacharat}{\kern0pt}\ {\isacharbrackleft}{\kern0pt}Fn{\isacharbrackright}{\kern0pt}\ {\isasymTurnstile}\ is{\isacharunderscore}{\kern0pt}Fn{\isacharunderscore}{\kern0pt}perm{\isacharprime}{\kern0pt}{\isacharunderscore}{\kern0pt}fm{\isacharparenleft}{\kern0pt}{\isadigit{0}}{\isacharcomma}{\kern0pt}\ {\isadigit{2}}{\isacharcomma}{\kern0pt}\ {\isadigit{1}}{\isacharparenright}{\kern0pt}{\isacharparenright}{\kern0pt}{\isacharparenright}{\kern0pt}{\isachardoublequoteclose}\ {\isacharparenleft}{\kern0pt}\isakeyword{is}\ {\isacharquery}{\kern0pt}C{\isacharparenright}{\kern0pt}\isanewline
\ \ \ \ \isacommand{apply}\isamarkupfalse%
{\isacharparenleft}{\kern0pt}rule{\isacharunderscore}{\kern0pt}tac\ Q{\isacharequal}{\kern0pt}{\isachardoublequoteopen}{\isacharquery}{\kern0pt}B{\isachardoublequoteclose}\ \isakeyword{in}\ iffD{\isadigit{2}}{\isacharparenright}{\kern0pt}\isanewline
\ \ \ \ \ \isacommand{apply}\isamarkupfalse%
{\isacharparenleft}{\kern0pt}rule\ univalent{\isacharunderscore}{\kern0pt}cong{\isacharcomma}{\kern0pt}\ simp{\isacharparenright}{\kern0pt}\isanewline
\ \ \ \ \ \isacommand{apply}\isamarkupfalse%
{\isacharparenleft}{\kern0pt}rule\ sats{\isacharunderscore}{\kern0pt}is{\isacharunderscore}{\kern0pt}Fn{\isacharunderscore}{\kern0pt}perm{\isacharprime}{\kern0pt}{\isacharunderscore}{\kern0pt}fm{\isacharunderscore}{\kern0pt}iff{\isacharparenright}{\kern0pt}\isanewline
\ \ \ \ \isacommand{using}\isamarkupfalse%
\ Fn{\isacharunderscore}{\kern0pt}in{\isacharunderscore}{\kern0pt}M\ {\isacartoucheopen}{\isacharquery}{\kern0pt}B{\isacartoucheclose}\isanewline
\ \ \ \ \isacommand{by}\isamarkupfalse%
\ auto\isanewline
\isanewline
\ \ \isacommand{have}\isamarkupfalse%
\ {\isachardoublequoteopen}strong{\isacharunderscore}{\kern0pt}replacement{\isacharparenleft}{\kern0pt}{\isacharhash}{\kern0pt}{\isacharhash}{\kern0pt}M{\isacharcomma}{\kern0pt}\ {\isasymlambda}x\ y{\isachardot}{\kern0pt}\ sats{\isacharparenleft}{\kern0pt}M{\isacharcomma}{\kern0pt}\ is{\isacharunderscore}{\kern0pt}Fn{\isacharunderscore}{\kern0pt}perm{\isacharprime}{\kern0pt}{\isacharunderscore}{\kern0pt}fm{\isacharparenleft}{\kern0pt}{\isadigit{0}}{\isacharcomma}{\kern0pt}\ {\isadigit{2}}{\isacharcomma}{\kern0pt}\ {\isadigit{1}}{\isacharparenright}{\kern0pt}{\isacharcomma}{\kern0pt}\ {\isacharbrackleft}{\kern0pt}x{\isacharcomma}{\kern0pt}\ y{\isacharbrackright}{\kern0pt}\ {\isacharat}{\kern0pt}\ {\isacharbrackleft}{\kern0pt}Fn{\isacharbrackright}{\kern0pt}{\isacharparenright}{\kern0pt}{\isacharparenright}{\kern0pt}{\isachardoublequoteclose}\isanewline
\ \ \ \ \isacommand{apply}\isamarkupfalse%
{\isacharparenleft}{\kern0pt}rule\ replacement{\isacharunderscore}{\kern0pt}ax{\isacharparenright}{\kern0pt}\isanewline
\ \ \ \ \ \ \isacommand{apply}\isamarkupfalse%
{\isacharparenleft}{\kern0pt}rule\ is{\isacharunderscore}{\kern0pt}Fn{\isacharunderscore}{\kern0pt}perm{\isacharprime}{\kern0pt}{\isacharunderscore}{\kern0pt}fm{\isacharunderscore}{\kern0pt}type{\isacharparenright}{\kern0pt}\isanewline
\ \ \ \ \isacommand{using}\isamarkupfalse%
\ Fn{\isacharunderscore}{\kern0pt}in{\isacharunderscore}{\kern0pt}M\isanewline
\ \ \ \ \ \ \ \ \isacommand{apply}\isamarkupfalse%
\ auto{\isacharbrackleft}{\kern0pt}{\isadigit{4}}{\isacharbrackright}{\kern0pt}\isanewline
\ \ \ \ \isacommand{apply}\isamarkupfalse%
{\isacharparenleft}{\kern0pt}rule\ le{\isacharunderscore}{\kern0pt}trans{\isacharcomma}{\kern0pt}\ rule\ arity{\isacharunderscore}{\kern0pt}is{\isacharunderscore}{\kern0pt}Fn{\isacharunderscore}{\kern0pt}perm{\isacharprime}{\kern0pt}{\isacharunderscore}{\kern0pt}fm{\isacharparenright}{\kern0pt}\isanewline
\ \ \ \ \isacommand{using}\isamarkupfalse%
\ Un{\isacharunderscore}{\kern0pt}least{\isacharunderscore}{\kern0pt}lt\isanewline
\ \ \ \ \isacommand{by}\isamarkupfalse%
\ auto\isanewline
\isanewline
\ \ \isacommand{then}\isamarkupfalse%
\ \isacommand{have}\isamarkupfalse%
\ {\isachardoublequoteopen}{\isasymexists}Y{\isacharbrackleft}{\kern0pt}{\isacharhash}{\kern0pt}{\isacharhash}{\kern0pt}M{\isacharbrackright}{\kern0pt}{\isachardot}{\kern0pt}\ {\isasymforall}b{\isacharbrackleft}{\kern0pt}{\isacharhash}{\kern0pt}{\isacharhash}{\kern0pt}M{\isacharbrackright}{\kern0pt}{\isachardot}{\kern0pt}\ b\ {\isasymin}\ Y\ {\isasymlongleftrightarrow}\ {\isacharparenleft}{\kern0pt}{\isasymexists}x{\isacharbrackleft}{\kern0pt}{\isacharhash}{\kern0pt}{\isacharhash}{\kern0pt}M{\isacharbrackright}{\kern0pt}{\isachardot}{\kern0pt}\ x\ {\isasymin}\ nat{\isacharunderscore}{\kern0pt}perms\ {\isasymand}\ sats{\isacharparenleft}{\kern0pt}M{\isacharcomma}{\kern0pt}\ is{\isacharunderscore}{\kern0pt}Fn{\isacharunderscore}{\kern0pt}perm{\isacharprime}{\kern0pt}{\isacharunderscore}{\kern0pt}fm{\isacharparenleft}{\kern0pt}{\isadigit{0}}{\isacharcomma}{\kern0pt}\ {\isadigit{2}}{\isacharcomma}{\kern0pt}\ {\isadigit{1}}{\isacharparenright}{\kern0pt}{\isacharcomma}{\kern0pt}\ {\isacharbrackleft}{\kern0pt}x{\isacharcomma}{\kern0pt}\ b{\isacharbrackright}{\kern0pt}\ {\isacharat}{\kern0pt}\ {\isacharbrackleft}{\kern0pt}Fn{\isacharbrackright}{\kern0pt}{\isacharparenright}{\kern0pt}{\isacharparenright}{\kern0pt}{\isachardoublequoteclose}\ \isanewline
\ \ \ \ \isacommand{apply}\isamarkupfalse%
{\isacharparenleft}{\kern0pt}rule\ strong{\isacharunderscore}{\kern0pt}replacementD{\isacharparenright}{\kern0pt}\isanewline
\ \ \ \ \isacommand{using}\isamarkupfalse%
\ nat{\isacharunderscore}{\kern0pt}perms{\isacharunderscore}{\kern0pt}in{\isacharunderscore}{\kern0pt}M\ {\isacartoucheopen}{\isacharquery}{\kern0pt}C{\isacartoucheclose}\isanewline
\ \ \ \ \isacommand{by}\isamarkupfalse%
\ auto\isanewline
\ \ \isacommand{then}\isamarkupfalse%
\ \isacommand{obtain}\isamarkupfalse%
\ Y\ \isakeyword{where}\ YH{\isacharcolon}{\kern0pt}\ {\isachardoublequoteopen}Y\ {\isasymin}\ M{\isachardoublequoteclose}\ {\isachardoublequoteopen}{\isasymAnd}b{\isachardot}{\kern0pt}\ b\ {\isasymin}\ M\ {\isasymLongrightarrow}\ b\ {\isasymin}\ Y\ {\isasymlongleftrightarrow}\ {\isacharparenleft}{\kern0pt}{\isasymexists}f\ {\isasymin}\ M{\isachardot}{\kern0pt}\ f\ {\isasymin}\ nat{\isacharunderscore}{\kern0pt}perms\ {\isasymand}\ sats{\isacharparenleft}{\kern0pt}M{\isacharcomma}{\kern0pt}\ is{\isacharunderscore}{\kern0pt}Fn{\isacharunderscore}{\kern0pt}perm{\isacharprime}{\kern0pt}{\isacharunderscore}{\kern0pt}fm{\isacharparenleft}{\kern0pt}{\isadigit{0}}{\isacharcomma}{\kern0pt}\ {\isadigit{2}}{\isacharcomma}{\kern0pt}\ {\isadigit{1}}{\isacharparenright}{\kern0pt}{\isacharcomma}{\kern0pt}\ {\isacharbrackleft}{\kern0pt}f{\isacharcomma}{\kern0pt}\ b{\isacharbrackright}{\kern0pt}\ {\isacharat}{\kern0pt}\ {\isacharbrackleft}{\kern0pt}Fn{\isacharbrackright}{\kern0pt}{\isacharparenright}{\kern0pt}{\isacharparenright}{\kern0pt}{\isachardoublequoteclose}\isanewline
\ \ \ \ \isacommand{by}\isamarkupfalse%
\ auto\isanewline
\ \ \isacommand{have}\isamarkupfalse%
\ {\isachardoublequoteopen}{\isasymAnd}b{\isachardot}{\kern0pt}\ b\ {\isasymin}\ M\ {\isasymLongrightarrow}\ b\ {\isasymin}\ Y\ {\isasymlongleftrightarrow}\ {\isacharparenleft}{\kern0pt}{\isasymexists}f\ {\isasymin}\ M{\isachardot}{\kern0pt}\ f\ {\isasymin}\ nat{\isacharunderscore}{\kern0pt}perms\ {\isasymand}\ b\ {\isacharequal}{\kern0pt}\ Fn{\isacharunderscore}{\kern0pt}perm{\isacharprime}{\kern0pt}{\isacharparenleft}{\kern0pt}f{\isacharparenright}{\kern0pt}{\isacharparenright}{\kern0pt}{\isachardoublequoteclose}\isanewline
\ \ \ \ \isacommand{apply}\isamarkupfalse%
{\isacharparenleft}{\kern0pt}rename{\isacharunderscore}{\kern0pt}tac\ b{\isacharcomma}{\kern0pt}\ rule{\isacharunderscore}{\kern0pt}tac\ Q{\isacharequal}{\kern0pt}{\isachardoublequoteopen}{\isacharparenleft}{\kern0pt}{\isasymexists}f\ {\isasymin}\ M{\isachardot}{\kern0pt}\ f\ {\isasymin}\ nat{\isacharunderscore}{\kern0pt}perms\ {\isasymand}\ sats{\isacharparenleft}{\kern0pt}M{\isacharcomma}{\kern0pt}\ is{\isacharunderscore}{\kern0pt}Fn{\isacharunderscore}{\kern0pt}perm{\isacharprime}{\kern0pt}{\isacharunderscore}{\kern0pt}fm{\isacharparenleft}{\kern0pt}{\isadigit{0}}{\isacharcomma}{\kern0pt}\ {\isadigit{2}}{\isacharcomma}{\kern0pt}\ {\isadigit{1}}{\isacharparenright}{\kern0pt}{\isacharcomma}{\kern0pt}\ {\isacharbrackleft}{\kern0pt}f{\isacharcomma}{\kern0pt}\ b{\isacharbrackright}{\kern0pt}\ {\isacharat}{\kern0pt}\ {\isacharbrackleft}{\kern0pt}Fn{\isacharbrackright}{\kern0pt}{\isacharparenright}{\kern0pt}{\isacharparenright}{\kern0pt}{\isachardoublequoteclose}\ \isakeyword{in}\ iff{\isacharunderscore}{\kern0pt}trans{\isacharparenright}{\kern0pt}\isanewline
\ \ \ \ \isacommand{using}\isamarkupfalse%
\ YH\ \isanewline
\ \ \ \ \ \isacommand{apply}\isamarkupfalse%
\ force\ \isanewline
\ \ \ \ \isacommand{apply}\isamarkupfalse%
{\isacharparenleft}{\kern0pt}rule\ bex{\isacharunderscore}{\kern0pt}iff{\isacharcomma}{\kern0pt}\ rule\ iff{\isacharunderscore}{\kern0pt}conjI{\isadigit{2}}{\isacharcomma}{\kern0pt}\ simp{\isacharparenright}{\kern0pt}\isanewline
\ \ \ \ \isacommand{apply}\isamarkupfalse%
{\isacharparenleft}{\kern0pt}rule\ sats{\isacharunderscore}{\kern0pt}is{\isacharunderscore}{\kern0pt}Fn{\isacharunderscore}{\kern0pt}perm{\isacharprime}{\kern0pt}{\isacharunderscore}{\kern0pt}fm{\isacharunderscore}{\kern0pt}iff{\isacharparenright}{\kern0pt}\isanewline
\ \ \ \ \isacommand{using}\isamarkupfalse%
\ Fn{\isacharunderscore}{\kern0pt}in{\isacharunderscore}{\kern0pt}M\ \isanewline
\ \ \ \ \isacommand{by}\isamarkupfalse%
\ auto\isanewline
\ \ \isacommand{then}\isamarkupfalse%
\ \isacommand{have}\isamarkupfalse%
\ {\isachardoublequoteopen}{\isasymAnd}b{\isachardot}{\kern0pt}\ b\ {\isasymin}\ Y\ {\isasymlongleftrightarrow}\ {\isacharparenleft}{\kern0pt}{\isasymexists}f{\isachardot}{\kern0pt}\ f\ {\isasymin}\ nat{\isacharunderscore}{\kern0pt}perms\ {\isasymand}\ b\ {\isacharequal}{\kern0pt}\ Fn{\isacharunderscore}{\kern0pt}perm{\isacharprime}{\kern0pt}{\isacharparenleft}{\kern0pt}f{\isacharparenright}{\kern0pt}{\isacharparenright}{\kern0pt}{\isachardoublequoteclose}\isanewline
\ \ \ \ \isacommand{apply}\isamarkupfalse%
{\isacharparenleft}{\kern0pt}rule{\isacharunderscore}{\kern0pt}tac\ iffI{\isacharparenright}{\kern0pt}\isanewline
\ \ \ \ \isacommand{using}\isamarkupfalse%
\ YH\ transM\ \isanewline
\ \ \ \ \ \isacommand{apply}\isamarkupfalse%
\ force\ \isanewline
\ \ \ \ \isacommand{apply}\isamarkupfalse%
{\isacharparenleft}{\kern0pt}rule\ exE{\isacharcomma}{\kern0pt}\ assumption{\isacharparenright}{\kern0pt}\isanewline
\ \ \ \ \isacommand{apply}\isamarkupfalse%
{\isacharparenleft}{\kern0pt}rename{\isacharunderscore}{\kern0pt}tac\ b\ f{\isacharcomma}{\kern0pt}\ subgoal{\isacharunderscore}{\kern0pt}tac\ {\isachardoublequoteopen}f\ {\isasymin}\ M\ {\isasymand}\ b\ {\isasymin}\ M{\isachardoublequoteclose}{\isacharcomma}{\kern0pt}\ force{\isacharparenright}{\kern0pt}\isanewline
\ \ \ \ \isacommand{using}\isamarkupfalse%
\ nat{\isacharunderscore}{\kern0pt}perms{\isacharunderscore}{\kern0pt}in{\isacharunderscore}{\kern0pt}M\ Fn{\isacharunderscore}{\kern0pt}perm{\isacharprime}{\kern0pt}{\isacharunderscore}{\kern0pt}in{\isacharunderscore}{\kern0pt}M\ transM\ pair{\isacharunderscore}{\kern0pt}in{\isacharunderscore}{\kern0pt}M{\isacharunderscore}{\kern0pt}iff\isanewline
\ \ \ \ \isacommand{by}\isamarkupfalse%
\ auto\ \isanewline
\ \ \isacommand{then}\isamarkupfalse%
\ \isacommand{have}\isamarkupfalse%
\ {\isachardoublequoteopen}Y\ {\isacharequal}{\kern0pt}\ Fn{\isacharunderscore}{\kern0pt}perms{\isachardoublequoteclose}\ \isanewline
\ \ \ \ \isacommand{unfolding}\isamarkupfalse%
\ Fn{\isacharunderscore}{\kern0pt}perms{\isacharunderscore}{\kern0pt}def\isanewline
\ \ \ \ \isacommand{by}\isamarkupfalse%
\ auto\isanewline
\ \ \isacommand{then}\isamarkupfalse%
\ \isacommand{show}\isamarkupfalse%
\ {\isacharquery}{\kern0pt}thesis\ \isacommand{using}\isamarkupfalse%
\ YH\ \isacommand{by}\isamarkupfalse%
\ auto\isanewline
\isacommand{qed}\isamarkupfalse%
%
\endisatagproof
{\isafoldproof}%
%
\isadelimproof
\ \isanewline
%
\endisadelimproof
\isanewline
\isacommand{end}\isamarkupfalse%
\isanewline
%
\isadelimtheory
%
\endisadelimtheory
%
\isatagtheory
\isacommand{end}\isamarkupfalse%
%
\endisatagtheory
{\isafoldtheory}%
%
\isadelimtheory
%
\endisadelimtheory
%
\end{isabellebody}%
\endinput
%:%file=~/source/repos/ZF-notAC/code/Fn_Perm_Definition.thy%:%
%:%10=1%:%
%:%11=1%:%
%:%12=2%:%
%:%13=3%:%
%:%18=3%:%
%:%21=4%:%
%:%22=5%:%
%:%23=5%:%
%:%24=6%:%
%:%25=7%:%
%:%26=7%:%
%:%27=8%:%
%:%28=8%:%
%:%29=9%:%
%:%30=10%:%
%:%31=10%:%
%:%32=11%:%
%:%33=12%:%
%:%34=13%:%
%:%37=14%:%
%:%41=14%:%
%:%42=14%:%
%:%43=15%:%
%:%44=15%:%
%:%45=16%:%
%:%46=16%:%
%:%51=16%:%
%:%54=17%:%
%:%55=18%:%
%:%56=18%:%
%:%57=19%:%
%:%58=20%:%
%:%59=21%:%
%:%62=22%:%
%:%66=22%:%
%:%67=22%:%
%:%68=23%:%
%:%69=23%:%
%:%70=24%:%
%:%71=24%:%
%:%72=25%:%
%:%78=25%:%
%:%81=26%:%
%:%82=27%:%
%:%83=27%:%
%:%84=28%:%
%:%85=29%:%
%:%86=30%:%
%:%89=31%:%
%:%90=32%:%
%:%94=32%:%
%:%95=32%:%
%:%96=33%:%
%:%97=33%:%
%:%98=34%:%
%:%99=34%:%
%:%100=35%:%
%:%101=35%:%
%:%102=36%:%
%:%103=36%:%
%:%104=37%:%
%:%105=37%:%
%:%106=38%:%
%:%107=38%:%
%:%108=39%:%
%:%109=39%:%
%:%110=40%:%
%:%111=40%:%
%:%112=41%:%
%:%113=41%:%
%:%114=42%:%
%:%115=42%:%
%:%120=42%:%
%:%123=43%:%
%:%124=44%:%
%:%125=44%:%
%:%126=45%:%
%:%132=51%:%
%:%133=52%:%
%:%134=53%:%
%:%135=53%:%
%:%136=54%:%
%:%137=55%:%
%:%138=56%:%
%:%141=57%:%
%:%145=57%:%
%:%146=57%:%
%:%147=58%:%
%:%148=58%:%
%:%149=59%:%
%:%150=59%:%
%:%151=60%:%
%:%152=60%:%
%:%153=61%:%
%:%154=61%:%
%:%155=62%:%
%:%156=62%:%
%:%161=62%:%
%:%164=63%:%
%:%165=64%:%
%:%166=64%:%
%:%167=65%:%
%:%168=66%:%
%:%169=67%:%
%:%172=68%:%
%:%176=68%:%
%:%177=68%:%
%:%178=69%:%
%:%179=69%:%
%:%180=70%:%
%:%181=70%:%
%:%182=71%:%
%:%183=71%:%
%:%184=72%:%
%:%185=72%:%
%:%186=73%:%
%:%187=73%:%
%:%188=74%:%
%:%189=74%:%
%:%190=75%:%
%:%191=75%:%
%:%192=76%:%
%:%193=76%:%
%:%194=77%:%
%:%195=77%:%
%:%196=78%:%
%:%197=78%:%
%:%198=79%:%
%:%199=79%:%
%:%200=80%:%
%:%201=80%:%
%:%202=81%:%
%:%203=81%:%
%:%204=82%:%
%:%205=82%:%
%:%206=83%:%
%:%207=83%:%
%:%208=84%:%
%:%209=84%:%
%:%210=85%:%
%:%211=85%:%
%:%212=86%:%
%:%213=86%:%
%:%214=87%:%
%:%215=87%:%
%:%216=88%:%
%:%217=88%:%
%:%218=89%:%
%:%219=89%:%
%:%220=90%:%
%:%221=90%:%
%:%222=91%:%
%:%223=91%:%
%:%224=92%:%
%:%225=92%:%
%:%226=93%:%
%:%227=93%:%
%:%228=94%:%
%:%229=94%:%
%:%230=95%:%
%:%231=95%:%
%:%232=96%:%
%:%233=96%:%
%:%234=97%:%
%:%235=97%:%
%:%236=98%:%
%:%237=98%:%
%:%238=99%:%
%:%239=99%:%
%:%244=99%:%
%:%247=100%:%
%:%248=101%:%
%:%249=101%:%
%:%250=102%:%
%:%251=103%:%
%:%252=104%:%
%:%253=105%:%
%:%254=106%:%
%:%257=107%:%
%:%261=107%:%
%:%262=107%:%
%:%263=108%:%
%:%264=108%:%
%:%266=110%:%
%:%267=111%:%
%:%268=111%:%
%:%269=112%:%
%:%270=112%:%
%:%271=113%:%
%:%272=113%:%
%:%273=114%:%
%:%274=114%:%
%:%275=115%:%
%:%276=115%:%
%:%277=116%:%
%:%278=116%:%
%:%279=117%:%
%:%280=117%:%
%:%281=118%:%
%:%282=118%:%
%:%283=119%:%
%:%284=119%:%
%:%285=120%:%
%:%286=120%:%
%:%287=121%:%
%:%288=121%:%
%:%289=122%:%
%:%290=122%:%
%:%291=123%:%
%:%292=123%:%
%:%293=124%:%
%:%294=124%:%
%:%295=125%:%
%:%296=125%:%
%:%297=126%:%
%:%298=126%:%
%:%299=127%:%
%:%300=127%:%
%:%301=128%:%
%:%302=128%:%
%:%303=129%:%
%:%304=129%:%
%:%305=130%:%
%:%306=130%:%
%:%307=131%:%
%:%308=131%:%
%:%309=132%:%
%:%310=132%:%
%:%311=133%:%
%:%312=133%:%
%:%313=134%:%
%:%314=134%:%
%:%315=135%:%
%:%316=135%:%
%:%317=136%:%
%:%318=136%:%
%:%319=137%:%
%:%320=137%:%
%:%325=137%:%
%:%328=138%:%
%:%329=139%:%
%:%330=139%:%
%:%331=140%:%
%:%332=141%:%
%:%333=141%:%
%:%340=142%:%
%:%341=142%:%
%:%342=143%:%
%:%343=144%:%
%:%344=144%:%
%:%345=145%:%
%:%346=145%:%
%:%347=146%:%
%:%348=146%:%
%:%349=147%:%
%:%350=147%:%
%:%351=148%:%
%:%352=148%:%
%:%353=149%:%
%:%354=149%:%
%:%355=150%:%
%:%356=150%:%
%:%357=151%:%
%:%358=151%:%
%:%359=152%:%
%:%360=152%:%
%:%361=153%:%
%:%362=153%:%
%:%363=154%:%
%:%364=154%:%
%:%365=155%:%
%:%366=155%:%
%:%367=156%:%
%:%368=156%:%
%:%369=157%:%
%:%370=157%:%
%:%371=158%:%
%:%372=159%:%
%:%373=159%:%
%:%374=160%:%
%:%375=160%:%
%:%376=161%:%
%:%377=161%:%
%:%378=162%:%
%:%379=162%:%
%:%380=163%:%
%:%381=163%:%
%:%382=164%:%
%:%383=164%:%
%:%384=165%:%
%:%385=165%:%
%:%386=166%:%
%:%387=166%:%
%:%388=166%:%
%:%389=166%:%
%:%390=166%:%
%:%391=167%:%
%:%397=167%:%
%:%400=168%:%
%:%401=169%:%
%:%402=169%:%
%:%403=170%:%
%:%404=171%:%
%:%405=171%:%
%:%412=172%:%
%:%413=172%:%
%:%414=173%:%
%:%415=173%:%
%:%416=174%:%
%:%417=174%:%
%:%418=175%:%
%:%419=175%:%
%:%420=176%:%
%:%421=176%:%
%:%422=177%:%
%:%423=177%:%
%:%424=178%:%
%:%425=178%:%
%:%426=179%:%
%:%427=179%:%
%:%428=180%:%
%:%429=180%:%
%:%430=181%:%
%:%431=181%:%
%:%432=182%:%
%:%433=182%:%
%:%434=183%:%
%:%435=183%:%
%:%436=184%:%
%:%437=184%:%
%:%438=185%:%
%:%439=185%:%
%:%440=186%:%
%:%441=186%:%
%:%442=187%:%
%:%443=187%:%
%:%444=187%:%
%:%445=187%:%
%:%446=187%:%
%:%447=188%:%
%:%453=188%:%
%:%456=189%:%
%:%457=190%:%
%:%458=190%:%
%:%461=191%:%
%:%465=191%:%
%:%466=191%:%
%:%467=192%:%
%:%468=192%:%
%:%469=193%:%
%:%470=193%:%
%:%471=194%:%
%:%472=194%:%
%:%477=194%:%
%:%480=195%:%
%:%481=196%:%
%:%482=196%:%
%:%485=197%:%
%:%489=197%:%
%:%490=197%:%
%:%491=198%:%
%:%492=198%:%
%:%493=199%:%
%:%494=199%:%
%:%495=200%:%
%:%496=200%:%
%:%497=201%:%
%:%498=201%:%
%:%499=202%:%
%:%500=202%:%
%:%501=203%:%
%:%502=203%:%
%:%503=204%:%
%:%504=204%:%
%:%505=205%:%
%:%506=205%:%
%:%511=205%:%
%:%514=206%:%
%:%515=207%:%
%:%516=207%:%
%:%519=208%:%
%:%523=208%:%
%:%524=208%:%
%:%525=209%:%
%:%526=209%:%
%:%527=210%:%
%:%528=210%:%
%:%529=211%:%
%:%530=211%:%
%:%531=212%:%
%:%532=212%:%
%:%533=213%:%
%:%539=213%:%
%:%542=214%:%
%:%543=215%:%
%:%544=215%:%
%:%547=216%:%
%:%551=216%:%
%:%552=216%:%
%:%553=217%:%
%:%554=217%:%
%:%555=218%:%
%:%556=218%:%
%:%561=218%:%
%:%564=219%:%
%:%565=220%:%
%:%566=220%:%
%:%567=221%:%
%:%568=222%:%
%:%569=222%:%
%:%576=223%:%
%:%577=223%:%
%:%578=224%:%
%:%579=224%:%
%:%580=225%:%
%:%581=226%:%
%:%582=226%:%
%:%583=227%:%
%:%584=227%:%
%:%585=228%:%
%:%586=228%:%
%:%587=229%:%
%:%588=229%:%
%:%589=230%:%
%:%590=230%:%
%:%591=231%:%
%:%592=231%:%
%:%593=232%:%
%:%594=232%:%
%:%595=233%:%
%:%596=233%:%
%:%597=234%:%
%:%598=234%:%
%:%599=235%:%
%:%600=235%:%
%:%601=236%:%
%:%602=236%:%
%:%603=237%:%
%:%604=238%:%
%:%605=238%:%
%:%606=239%:%
%:%607=239%:%
%:%608=240%:%
%:%609=240%:%
%:%610=241%:%
%:%611=241%:%
%:%612=242%:%
%:%613=242%:%
%:%614=242%:%
%:%615=242%:%
%:%616=242%:%
%:%617=243%:%
%:%623=243%:%
%:%626=244%:%
%:%627=245%:%
%:%628=245%:%
%:%629=246%:%
%:%630=246%:%
%:%631=247%:%
%:%632=247%:%
%:%633=248%:%
%:%634=249%:%
%:%635=249%:%
%:%636=250%:%
%:%637=251%:%
%:%638=252%:%
%:%645=253%:%
%:%646=253%:%
%:%647=254%:%
%:%648=254%:%
%:%649=255%:%
%:%650=255%:%
%:%651=256%:%
%:%652=256%:%
%:%653=257%:%
%:%654=257%:%
%:%655=258%:%
%:%656=258%:%
%:%657=259%:%
%:%658=259%:%
%:%659=260%:%
%:%660=260%:%
%:%661=261%:%
%:%662=261%:%
%:%663=261%:%
%:%664=261%:%
%:%665=262%:%
%:%666=262%:%
%:%667=262%:%
%:%668=262%:%
%:%669=262%:%
%:%670=263%:%
%:%671=263%:%
%:%672=263%:%
%:%673=263%:%
%:%674=263%:%
%:%675=264%:%
%:%676=264%:%
%:%677=264%:%
%:%678=264%:%
%:%679=265%:%
%:%680=265%:%
%:%681=265%:%
%:%682=265%:%
%:%683=265%:%
%:%684=266%:%
%:%685=266%:%
%:%686=266%:%
%:%687=266%:%
%:%688=266%:%
%:%689=267%:%
%:%690=267%:%
%:%691=267%:%
%:%692=267%:%
%:%693=267%:%
%:%694=268%:%
%:%700=268%:%
%:%703=269%:%
%:%704=270%:%
%:%705=271%:%
%:%706=271%:%
%:%707=272%:%
%:%713=278%:%
%:%714=279%:%
%:%715=280%:%
%:%716=280%:%
%:%717=281%:%
%:%718=282%:%
%:%719=283%:%
%:%722=284%:%
%:%726=284%:%
%:%727=284%:%
%:%728=285%:%
%:%729=285%:%
%:%730=286%:%
%:%731=286%:%
%:%736=286%:%
%:%739=287%:%
%:%740=288%:%
%:%741=288%:%
%:%742=289%:%
%:%743=290%:%
%:%744=291%:%
%:%747=292%:%
%:%751=292%:%
%:%752=292%:%
%:%753=293%:%
%:%754=293%:%
%:%755=294%:%
%:%756=294%:%
%:%757=295%:%
%:%758=295%:%
%:%759=296%:%
%:%760=296%:%
%:%761=297%:%
%:%762=297%:%
%:%763=298%:%
%:%764=298%:%
%:%765=299%:%
%:%766=299%:%
%:%767=300%:%
%:%768=300%:%
%:%769=301%:%
%:%770=301%:%
%:%771=302%:%
%:%772=302%:%
%:%773=303%:%
%:%774=303%:%
%:%775=304%:%
%:%776=304%:%
%:%777=305%:%
%:%778=305%:%
%:%779=306%:%
%:%780=306%:%
%:%781=307%:%
%:%782=307%:%
%:%783=308%:%
%:%784=308%:%
%:%785=309%:%
%:%786=309%:%
%:%787=310%:%
%:%788=310%:%
%:%789=311%:%
%:%790=311%:%
%:%791=312%:%
%:%792=312%:%
%:%793=313%:%
%:%794=313%:%
%:%795=314%:%
%:%796=314%:%
%:%797=315%:%
%:%798=315%:%
%:%799=316%:%
%:%800=316%:%
%:%801=317%:%
%:%802=317%:%
%:%803=318%:%
%:%804=318%:%
%:%805=319%:%
%:%806=319%:%
%:%807=320%:%
%:%808=320%:%
%:%809=321%:%
%:%815=321%:%
%:%818=322%:%
%:%819=323%:%
%:%820=323%:%
%:%821=324%:%
%:%822=325%:%
%:%823=326%:%
%:%824=327%:%
%:%827=328%:%
%:%831=328%:%
%:%832=328%:%
%:%833=329%:%
%:%834=329%:%
%:%835=330%:%
%:%836=330%:%
%:%837=331%:%
%:%838=331%:%
%:%839=332%:%
%:%840=332%:%
%:%841=333%:%
%:%842=333%:%
%:%843=334%:%
%:%844=334%:%
%:%845=335%:%
%:%846=335%:%
%:%847=336%:%
%:%848=336%:%
%:%849=337%:%
%:%850=337%:%
%:%851=338%:%
%:%852=338%:%
%:%853=339%:%
%:%854=339%:%
%:%855=340%:%
%:%856=340%:%
%:%857=341%:%
%:%858=341%:%
%:%859=342%:%
%:%860=342%:%
%:%861=343%:%
%:%862=343%:%
%:%863=344%:%
%:%864=344%:%
%:%865=345%:%
%:%866=345%:%
%:%867=346%:%
%:%868=346%:%
%:%873=346%:%
%:%876=347%:%
%:%877=348%:%
%:%878=348%:%
%:%879=349%:%
%:%880=350%:%
%:%881=351%:%
%:%884=352%:%
%:%889=353%:%
%:%890=353%:%
%:%891=354%:%
%:%892=354%:%
%:%893=355%:%
%:%894=356%:%
%:%895=356%:%
%:%896=357%:%
%:%897=357%:%
%:%898=358%:%
%:%899=358%:%
%:%900=359%:%
%:%901=359%:%
%:%902=360%:%
%:%903=360%:%
%:%904=361%:%
%:%905=361%:%
%:%906=362%:%
%:%907=362%:%
%:%908=363%:%
%:%909=363%:%
%:%910=364%:%
%:%911=364%:%
%:%912=365%:%
%:%913=365%:%
%:%914=366%:%
%:%915=366%:%
%:%916=367%:%
%:%917=367%:%
%:%918=368%:%
%:%919=368%:%
%:%920=369%:%
%:%921=369%:%
%:%922=370%:%
%:%923=370%:%
%:%924=371%:%
%:%925=371%:%
%:%926=371%:%
%:%927=372%:%
%:%928=372%:%
%:%929=373%:%
%:%930=373%:%
%:%931=374%:%
%:%932=374%:%
%:%933=375%:%
%:%934=375%:%
%:%935=375%:%
%:%936=375%:%
%:%937=376%:%
%:%938=377%:%
%:%939=377%:%
%:%940=378%:%
%:%941=378%:%
%:%942=379%:%
%:%943=379%:%
%:%944=380%:%
%:%945=380%:%
%:%946=381%:%
%:%947=381%:%
%:%948=381%:%
%:%949=381%:%
%:%950=381%:%
%:%951=382%:%
%:%952=382%:%
%:%953=382%:%
%:%954=382%:%
%:%955=382%:%
%:%956=383%:%
%:%957=383%:%
%:%958=383%:%
%:%959=383%:%
%:%960=384%:%
%:%961=384%:%
%:%962=384%:%
%:%963=384%:%
%:%964=384%:%
%:%965=385%:%
%:%966=385%:%
%:%967=385%:%
%:%968=385%:%
%:%969=386%:%
%:%970=386%:%
%:%971=386%:%
%:%972=387%:%
%:%973=387%:%
%:%974=388%:%
%:%975=388%:%
%:%976=389%:%
%:%977=389%:%
%:%978=390%:%
%:%979=390%:%
%:%980=391%:%
%:%981=391%:%
%:%982=392%:%
%:%983=392%:%
%:%984=393%:%
%:%985=393%:%
%:%986=394%:%
%:%987=394%:%
%:%988=395%:%
%:%989=395%:%
%:%990=395%:%
%:%991=395%:%
%:%992=395%:%
%:%993=396%:%
%:%994=396%:%
%:%995=396%:%
%:%996=397%:%
%:%997=397%:%
%:%998=398%:%
%:%999=398%:%
%:%1000=399%:%
%:%1001=399%:%
%:%1002=399%:%
%:%1003=400%:%
%:%1004=400%:%
%:%1005=401%:%
%:%1006=401%:%
%:%1007=402%:%
%:%1008=402%:%
%:%1009=403%:%
%:%1010=403%:%
%:%1011=404%:%
%:%1012=404%:%
%:%1013=405%:%
%:%1014=405%:%
%:%1015=406%:%
%:%1016=406%:%
%:%1017=407%:%
%:%1018=407%:%
%:%1019=408%:%
%:%1020=408%:%
%:%1021=409%:%
%:%1022=409%:%
%:%1023=410%:%
%:%1024=410%:%
%:%1025=411%:%
%:%1026=411%:%
%:%1027=411%:%
%:%1028=411%:%
%:%1029=411%:%
%:%1030=412%:%
%:%1031=412%:%
%:%1032=413%:%
%:%1033=414%:%
%:%1034=414%:%
%:%1035=414%:%
%:%1036=414%:%
%:%1037=414%:%
%:%1038=415%:%
%:%1044=415%:%
%:%1047=416%:%
%:%1048=417%:%
%:%1049=417%:%
%:%1050=418%:%
%:%1051=419%:%
%:%1052=420%:%
%:%1053=420%:%
%:%1054=421%:%
%:%1055=422%:%
%:%1056=423%:%
%:%1059=424%:%
%:%1063=424%:%
%:%1064=424%:%
%:%1065=425%:%
%:%1066=425%:%
%:%1067=426%:%
%:%1068=426%:%
%:%1069=427%:%
%:%1070=427%:%
%:%1075=427%:%
%:%1078=428%:%
%:%1079=429%:%
%:%1080=429%:%
%:%1081=430%:%
%:%1082=431%:%
%:%1083=432%:%
%:%1086=433%:%
%:%1090=433%:%
%:%1091=433%:%
%:%1092=434%:%
%:%1093=434%:%
%:%1094=435%:%
%:%1095=435%:%
%:%1096=436%:%
%:%1097=436%:%
%:%1098=437%:%
%:%1099=437%:%
%:%1100=438%:%
%:%1101=438%:%
%:%1102=439%:%
%:%1103=439%:%
%:%1104=440%:%
%:%1105=440%:%
%:%1106=441%:%
%:%1107=441%:%
%:%1108=442%:%
%:%1109=442%:%
%:%1110=443%:%
%:%1111=443%:%
%:%1112=444%:%
%:%1113=444%:%
%:%1114=445%:%
%:%1115=445%:%
%:%1116=446%:%
%:%1117=446%:%
%:%1118=447%:%
%:%1119=447%:%
%:%1120=448%:%
%:%1121=448%:%
%:%1122=449%:%
%:%1123=449%:%
%:%1124=450%:%
%:%1125=450%:%
%:%1126=451%:%
%:%1127=451%:%
%:%1128=452%:%
%:%1129=452%:%
%:%1134=452%:%
%:%1137=453%:%
%:%1138=454%:%
%:%1139=454%:%
%:%1140=455%:%
%:%1141=456%:%
%:%1142=457%:%
%:%1143=458%:%
%:%1144=459%:%
%:%1151=460%:%
%:%1152=460%:%
%:%1153=461%:%
%:%1154=461%:%
%:%1155=462%:%
%:%1156=462%:%
%:%1157=463%:%
%:%1158=463%:%
%:%1159=464%:%
%:%1160=464%:%
%:%1161=465%:%
%:%1162=465%:%
%:%1163=466%:%
%:%1164=466%:%
%:%1165=467%:%
%:%1166=467%:%
%:%1167=468%:%
%:%1168=468%:%
%:%1169=469%:%
%:%1170=469%:%
%:%1171=470%:%
%:%1172=470%:%
%:%1173=471%:%
%:%1174=471%:%
%:%1175=472%:%
%:%1176=472%:%
%:%1177=473%:%
%:%1178=473%:%
%:%1179=474%:%
%:%1180=474%:%
%:%1181=475%:%
%:%1182=475%:%
%:%1183=476%:%
%:%1184=476%:%
%:%1185=477%:%
%:%1186=477%:%
%:%1187=478%:%
%:%1188=478%:%
%:%1189=479%:%
%:%1190=479%:%
%:%1191=480%:%
%:%1192=480%:%
%:%1193=481%:%
%:%1194=481%:%
%:%1195=482%:%
%:%1196=482%:%
%:%1197=483%:%
%:%1198=483%:%
%:%1199=484%:%
%:%1200=484%:%
%:%1201=485%:%
%:%1202=485%:%
%:%1203=486%:%
%:%1204=486%:%
%:%1205=487%:%
%:%1206=487%:%
%:%1207=488%:%
%:%1208=488%:%
%:%1209=489%:%
%:%1210=489%:%
%:%1211=490%:%
%:%1212=490%:%
%:%1213=491%:%
%:%1214=491%:%
%:%1215=492%:%
%:%1216=492%:%
%:%1217=493%:%
%:%1218=493%:%
%:%1219=494%:%
%:%1220=494%:%
%:%1221=495%:%
%:%1222=495%:%
%:%1223=496%:%
%:%1224=496%:%
%:%1225=497%:%
%:%1226=497%:%
%:%1227=498%:%
%:%1228=498%:%
%:%1229=498%:%
%:%1230=498%:%
%:%1231=499%:%
%:%1237=499%:%
%:%1240=500%:%
%:%1241=501%:%
%:%1242=501%:%
%:%1243=502%:%
%:%1244=503%:%
%:%1245=504%:%
%:%1246=504%:%
%:%1247=505%:%
%:%1248=506%:%
%:%1249=507%:%
%:%1252=508%:%
%:%1256=508%:%
%:%1257=508%:%
%:%1258=509%:%
%:%1259=509%:%
%:%1260=510%:%
%:%1261=510%:%
%:%1262=511%:%
%:%1263=511%:%
%:%1264=512%:%
%:%1265=512%:%
%:%1270=512%:%
%:%1273=513%:%
%:%1274=514%:%
%:%1275=514%:%
%:%1276=515%:%
%:%1277=516%:%
%:%1278=517%:%
%:%1281=518%:%
%:%1282=519%:%
%:%1286=519%:%
%:%1287=519%:%
%:%1288=520%:%
%:%1289=520%:%
%:%1290=521%:%
%:%1291=521%:%
%:%1292=522%:%
%:%1293=522%:%
%:%1294=523%:%
%:%1295=523%:%
%:%1296=524%:%
%:%1297=524%:%
%:%1298=525%:%
%:%1299=525%:%
%:%1300=526%:%
%:%1301=526%:%
%:%1302=527%:%
%:%1303=527%:%
%:%1304=528%:%
%:%1305=528%:%
%:%1306=529%:%
%:%1307=529%:%
%:%1308=530%:%
%:%1309=530%:%
%:%1310=531%:%
%:%1311=531%:%
%:%1312=532%:%
%:%1313=532%:%
%:%1314=533%:%
%:%1315=533%:%
%:%1316=534%:%
%:%1317=534%:%
%:%1322=534%:%
%:%1325=535%:%
%:%1326=536%:%
%:%1327=536%:%
%:%1328=537%:%
%:%1329=538%:%
%:%1330=539%:%
%:%1331=540%:%
%:%1332=541%:%
%:%1339=542%:%
%:%1340=542%:%
%:%1341=543%:%
%:%1342=543%:%
%:%1343=544%:%
%:%1344=544%:%
%:%1345=545%:%
%:%1346=545%:%
%:%1347=546%:%
%:%1348=546%:%
%:%1349=547%:%
%:%1350=547%:%
%:%1351=548%:%
%:%1352=548%:%
%:%1353=549%:%
%:%1354=549%:%
%:%1355=550%:%
%:%1356=550%:%
%:%1357=550%:%
%:%1358=551%:%
%:%1359=551%:%
%:%1360=552%:%
%:%1361=552%:%
%:%1362=553%:%
%:%1363=553%:%
%:%1364=553%:%
%:%1365=553%:%
%:%1366=554%:%
%:%1372=554%:%
%:%1375=555%:%
%:%1376=556%:%
%:%1377=556%:%
%:%1378=557%:%
%:%1379=558%:%
%:%1380=559%:%
%:%1383=560%:%
%:%1387=560%:%
%:%1388=560%:%
%:%1389=561%:%
%:%1390=561%:%
%:%1391=562%:%
%:%1392=562%:%
%:%1393=563%:%
%:%1394=563%:%
%:%1395=564%:%
%:%1396=564%:%
%:%1397=565%:%
%:%1398=565%:%
%:%1399=566%:%
%:%1400=566%:%
%:%1401=567%:%
%:%1402=567%:%
%:%1403=568%:%
%:%1404=568%:%
%:%1409=568%:%
%:%1412=569%:%
%:%1413=570%:%
%:%1414=570%:%
%:%1415=571%:%
%:%1416=572%:%
%:%1417=573%:%
%:%1424=574%:%
%:%1425=574%:%
%:%1426=575%:%
%:%1427=576%:%
%:%1428=576%:%
%:%1429=577%:%
%:%1430=578%:%
%:%1431=578%:%
%:%1432=579%:%
%:%1433=579%:%
%:%1434=580%:%
%:%1435=580%:%
%:%1436=581%:%
%:%1437=581%:%
%:%1438=582%:%
%:%1439=582%:%
%:%1440=583%:%
%:%1441=583%:%
%:%1442=584%:%
%:%1443=584%:%
%:%1444=585%:%
%:%1445=585%:%
%:%1446=586%:%
%:%1447=586%:%
%:%1448=587%:%
%:%1449=587%:%
%:%1450=588%:%
%:%1451=588%:%
%:%1452=589%:%
%:%1453=589%:%
%:%1454=590%:%
%:%1455=590%:%
%:%1456=591%:%
%:%1457=591%:%
%:%1458=592%:%
%:%1459=592%:%
%:%1460=593%:%
%:%1461=593%:%
%:%1462=594%:%
%:%1463=595%:%
%:%1464=595%:%
%:%1465=596%:%
%:%1466=596%:%
%:%1467=597%:%
%:%1468=597%:%
%:%1469=598%:%
%:%1470=598%:%
%:%1471=599%:%
%:%1472=599%:%
%:%1473=600%:%
%:%1474=600%:%
%:%1475=601%:%
%:%1476=601%:%
%:%1477=602%:%
%:%1478=602%:%
%:%1479=603%:%
%:%1480=603%:%
%:%1481=604%:%
%:%1482=604%:%
%:%1483=605%:%
%:%1484=605%:%
%:%1485=606%:%
%:%1486=606%:%
%:%1487=607%:%
%:%1488=607%:%
%:%1489=608%:%
%:%1490=608%:%
%:%1491=609%:%
%:%1492=609%:%
%:%1493=610%:%
%:%1494=611%:%
%:%1495=611%:%
%:%1496=611%:%
%:%1497=612%:%
%:%1498=612%:%
%:%1499=613%:%
%:%1500=613%:%
%:%1501=614%:%
%:%1502=614%:%
%:%1503=615%:%
%:%1504=615%:%
%:%1505=616%:%
%:%1506=616%:%
%:%1507=617%:%
%:%1508=618%:%
%:%1509=618%:%
%:%1510=618%:%
%:%1511=618%:%
%:%1512=618%:%
%:%1513=619%:%
%:%1519=619%:%
%:%1522=620%:%
%:%1523=621%:%
%:%1524=621%:%
%:%1525=622%:%
%:%1526=623%:%
%:%1527=624%:%
%:%1528=624%:%
%:%1529=625%:%
%:%1530=626%:%
%:%1531=627%:%
%:%1534=628%:%
%:%1538=628%:%
%:%1539=628%:%
%:%1540=629%:%
%:%1541=629%:%
%:%1542=630%:%
%:%1543=630%:%
%:%1544=631%:%
%:%1545=631%:%
%:%1546=632%:%
%:%1547=632%:%
%:%1548=633%:%
%:%1549=633%:%
%:%1550=634%:%
%:%1551=634%:%
%:%1556=634%:%
%:%1559=635%:%
%:%1560=636%:%
%:%1561=636%:%
%:%1562=637%:%
%:%1563=638%:%
%:%1564=639%:%
%:%1567=640%:%
%:%1568=641%:%
%:%1572=641%:%
%:%1573=641%:%
%:%1574=642%:%
%:%1575=642%:%
%:%1576=643%:%
%:%1577=643%:%
%:%1578=644%:%
%:%1579=644%:%
%:%1580=645%:%
%:%1581=645%:%
%:%1582=646%:%
%:%1583=646%:%
%:%1584=647%:%
%:%1585=647%:%
%:%1586=648%:%
%:%1587=648%:%
%:%1588=649%:%
%:%1589=649%:%
%:%1590=650%:%
%:%1591=650%:%
%:%1592=651%:%
%:%1593=651%:%
%:%1594=652%:%
%:%1595=652%:%
%:%1596=653%:%
%:%1597=653%:%
%:%1598=654%:%
%:%1599=654%:%
%:%1600=655%:%
%:%1601=655%:%
%:%1602=656%:%
%:%1603=656%:%
%:%1604=657%:%
%:%1605=657%:%
%:%1606=658%:%
%:%1607=658%:%
%:%1608=659%:%
%:%1609=659%:%
%:%1610=660%:%
%:%1611=660%:%
%:%1616=660%:%
%:%1619=661%:%
%:%1620=662%:%
%:%1621=662%:%
%:%1622=663%:%
%:%1623=664%:%
%:%1624=665%:%
%:%1625=666%:%
%:%1626=667%:%
%:%1633=668%:%
%:%1634=668%:%
%:%1635=669%:%
%:%1636=669%:%
%:%1637=670%:%
%:%1638=670%:%
%:%1639=671%:%
%:%1640=671%:%
%:%1641=672%:%
%:%1642=672%:%
%:%1643=673%:%
%:%1644=673%:%
%:%1645=674%:%
%:%1646=674%:%
%:%1647=675%:%
%:%1648=675%:%
%:%1649=676%:%
%:%1650=676%:%
%:%1651=677%:%
%:%1652=677%:%
%:%1653=678%:%
%:%1654=678%:%
%:%1655=679%:%
%:%1656=679%:%
%:%1657=680%:%
%:%1658=680%:%
%:%1659=681%:%
%:%1660=681%:%
%:%1661=682%:%
%:%1662=682%:%
%:%1663=683%:%
%:%1664=683%:%
%:%1665=684%:%
%:%1666=684%:%
%:%1667=685%:%
%:%1668=685%:%
%:%1669=686%:%
%:%1670=686%:%
%:%1671=687%:%
%:%1672=687%:%
%:%1673=688%:%
%:%1674=688%:%
%:%1675=689%:%
%:%1676=689%:%
%:%1677=690%:%
%:%1678=690%:%
%:%1679=691%:%
%:%1680=691%:%
%:%1681=692%:%
%:%1682=692%:%
%:%1683=693%:%
%:%1684=693%:%
%:%1685=694%:%
%:%1686=694%:%
%:%1687=695%:%
%:%1688=695%:%
%:%1689=696%:%
%:%1690=696%:%
%:%1691=697%:%
%:%1692=697%:%
%:%1693=698%:%
%:%1694=698%:%
%:%1695=699%:%
%:%1701=699%:%
%:%1704=700%:%
%:%1705=701%:%
%:%1706=701%:%
%:%1713=702%:%
%:%1714=702%:%
%:%1715=703%:%
%:%1716=703%:%
%:%1717=704%:%
%:%1718=704%:%
%:%1719=705%:%
%:%1720=705%:%
%:%1721=706%:%
%:%1722=706%:%
%:%1723=707%:%
%:%1724=707%:%
%:%1725=708%:%
%:%1726=708%:%
%:%1727=709%:%
%:%1728=709%:%
%:%1729=710%:%
%:%1730=710%:%
%:%1731=711%:%
%:%1732=711%:%
%:%1733=712%:%
%:%1734=713%:%
%:%1735=713%:%
%:%1736=714%:%
%:%1737=714%:%
%:%1738=715%:%
%:%1739=715%:%
%:%1740=716%:%
%:%1741=716%:%
%:%1742=717%:%
%:%1743=717%:%
%:%1744=718%:%
%:%1745=718%:%
%:%1746=719%:%
%:%1747=719%:%
%:%1748=720%:%
%:%1749=720%:%
%:%1750=721%:%
%:%1751=722%:%
%:%1752=722%:%
%:%1753=722%:%
%:%1754=723%:%
%:%1755=723%:%
%:%1756=724%:%
%:%1757=724%:%
%:%1758=725%:%
%:%1759=725%:%
%:%1760=726%:%
%:%1761=726%:%
%:%1762=726%:%
%:%1763=727%:%
%:%1764=727%:%
%:%1765=728%:%
%:%1766=728%:%
%:%1767=729%:%
%:%1768=729%:%
%:%1769=730%:%
%:%1770=730%:%
%:%1771=731%:%
%:%1772=731%:%
%:%1773=732%:%
%:%1774=732%:%
%:%1775=733%:%
%:%1776=733%:%
%:%1777=734%:%
%:%1778=734%:%
%:%1779=735%:%
%:%1780=735%:%
%:%1781=736%:%
%:%1782=736%:%
%:%1783=736%:%
%:%1784=737%:%
%:%1785=737%:%
%:%1786=738%:%
%:%1787=738%:%
%:%1788=739%:%
%:%1789=739%:%
%:%1790=740%:%
%:%1791=740%:%
%:%1792=741%:%
%:%1793=741%:%
%:%1794=742%:%
%:%1795=742%:%
%:%1796=743%:%
%:%1797=743%:%
%:%1798=744%:%
%:%1799=744%:%
%:%1800=744%:%
%:%1801=745%:%
%:%1802=745%:%
%:%1803=746%:%
%:%1804=746%:%
%:%1805=747%:%
%:%1806=747%:%
%:%1807=747%:%
%:%1808=747%:%
%:%1809=747%:%
%:%1810=748%:%
%:%1816=748%:%
%:%1819=749%:%
%:%1820=750%:%
%:%1821=750%:%
%:%1828=751%:%

%
\begin{isabellebody}%
\setisabellecontext{Fn{\isacharunderscore}{\kern0pt}Perm{\isacharunderscore}{\kern0pt}Automorphism}%
%
\isadelimtheory
%
\endisadelimtheory
%
\isatagtheory
\isacommand{theory}\isamarkupfalse%
\ Fn{\isacharunderscore}{\kern0pt}Perm{\isacharunderscore}{\kern0pt}Automorphism\ \isanewline
\ \ \isakeyword{imports}\ Fn{\isacharunderscore}{\kern0pt}Perm{\isacharunderscore}{\kern0pt}Definition\ \isanewline
\isanewline
\isakeyword{begin}%
\endisatagtheory
{\isafoldtheory}%
%
\isadelimtheory
\isanewline
%
\endisadelimtheory
\isacommand{context}\isamarkupfalse%
\ M{\isacharunderscore}{\kern0pt}ctm\ \isakeyword{begin}\ \isanewline
\isanewline
\isacommand{lemma}\isamarkupfalse%
\ converse{\isacharunderscore}{\kern0pt}in{\isacharunderscore}{\kern0pt}nat{\isacharunderscore}{\kern0pt}perms\ {\isacharcolon}{\kern0pt}\ \isanewline
\ \ \isakeyword{fixes}\ f\ \isanewline
\ \ \isakeyword{assumes}\ {\isachardoublequoteopen}f\ {\isasymin}\ nat{\isacharunderscore}{\kern0pt}perms{\isachardoublequoteclose}\ \isanewline
\ \ \isakeyword{shows}\ {\isachardoublequoteopen}converse{\isacharparenleft}{\kern0pt}f{\isacharparenright}{\kern0pt}\ {\isasymin}\ nat{\isacharunderscore}{\kern0pt}perms{\isachardoublequoteclose}\ \isanewline
%
\isadelimproof
\ \ %
\endisadelimproof
%
\isatagproof
\isacommand{unfolding}\isamarkupfalse%
\ nat{\isacharunderscore}{\kern0pt}perms{\isacharunderscore}{\kern0pt}def\ \isanewline
\ \ \isacommand{using}\isamarkupfalse%
\ bij{\isacharunderscore}{\kern0pt}converse{\isacharunderscore}{\kern0pt}bij\ assms\ nat{\isacharunderscore}{\kern0pt}perms{\isacharunderscore}{\kern0pt}def\ nat{\isacharunderscore}{\kern0pt}perms{\isacharunderscore}{\kern0pt}in{\isacharunderscore}{\kern0pt}M\ transM\ converse{\isacharunderscore}{\kern0pt}closed\isanewline
\ \ \isacommand{by}\isamarkupfalse%
\ auto%
\endisatagproof
{\isafoldproof}%
%
\isadelimproof
\isanewline
%
\endisadelimproof
\isanewline
\isacommand{lemma}\isamarkupfalse%
\ composition{\isacharunderscore}{\kern0pt}in{\isacharunderscore}{\kern0pt}nat{\isacharunderscore}{\kern0pt}perms\ {\isacharcolon}{\kern0pt}\ \isanewline
\ \ \isakeyword{fixes}\ f\ g\ \isanewline
\ \ \isakeyword{assumes}\ {\isachardoublequoteopen}f\ {\isasymin}\ nat{\isacharunderscore}{\kern0pt}perms{\isachardoublequoteclose}\ {\isachardoublequoteopen}g\ {\isasymin}\ nat{\isacharunderscore}{\kern0pt}perms{\isachardoublequoteclose}\ \isanewline
\ \ \isakeyword{shows}\ {\isachardoublequoteopen}f\ O\ g\ {\isasymin}\ nat{\isacharunderscore}{\kern0pt}perms{\isachardoublequoteclose}\ \isanewline
%
\isadelimproof
\ \ %
\endisadelimproof
%
\isatagproof
\isacommand{using}\isamarkupfalse%
\ assms\ comp{\isacharunderscore}{\kern0pt}closed\ comp{\isacharunderscore}{\kern0pt}bij\isanewline
\ \ \isacommand{unfolding}\isamarkupfalse%
\ nat{\isacharunderscore}{\kern0pt}perms{\isacharunderscore}{\kern0pt}def\ \isanewline
\ \ \isacommand{by}\isamarkupfalse%
\ auto%
\endisatagproof
{\isafoldproof}%
%
\isadelimproof
\isanewline
%
\endisadelimproof
\isanewline
\isacommand{lemma}\isamarkupfalse%
\ Fn{\isacharunderscore}{\kern0pt}perm{\isacharunderscore}{\kern0pt}subset\ {\isacharcolon}{\kern0pt}\ \isanewline
\ \ \isakeyword{fixes}\ f\ p\ \isanewline
\ \ \isakeyword{assumes}\ {\isachardoublequoteopen}f\ {\isasymin}\ nat{\isacharunderscore}{\kern0pt}perms{\isachardoublequoteclose}\ {\isachardoublequoteopen}p\ {\isasymin}\ Fn{\isachardoublequoteclose}\ \isanewline
\ \ \isakeyword{shows}\ {\isachardoublequoteopen}Fn{\isacharunderscore}{\kern0pt}perm{\isacharparenleft}{\kern0pt}f{\isacharcomma}{\kern0pt}\ p{\isacharparenright}{\kern0pt}\ {\isasymsubseteq}\ {\isacharparenleft}{\kern0pt}nat\ {\isasymtimes}\ nat{\isacharparenright}{\kern0pt}\ {\isasymtimes}\ {\isadigit{2}}{\isachardoublequoteclose}\ \isanewline
%
\isadelimproof
%
\endisadelimproof
%
\isatagproof
\isacommand{proof}\isamarkupfalse%
{\isacharparenleft}{\kern0pt}rule\ subsetI{\isacharparenright}{\kern0pt}\isanewline
\ \ \isacommand{fix}\isamarkupfalse%
\ v\ \isacommand{assume}\isamarkupfalse%
\ vin\ {\isacharcolon}{\kern0pt}\ {\isachardoublequoteopen}v\ {\isasymin}\ Fn{\isacharunderscore}{\kern0pt}perm{\isacharparenleft}{\kern0pt}f{\isacharcomma}{\kern0pt}\ p{\isacharparenright}{\kern0pt}{\isachardoublequoteclose}\ \isanewline
\ \ \isacommand{have}\isamarkupfalse%
\ {\isachardoublequoteopen}{\isasymexists}n{\isasymin}nat{\isachardot}{\kern0pt}\ {\isasymexists}m{\isasymin}nat{\isachardot}{\kern0pt}\ {\isasymexists}l{\isasymin}{\isadigit{2}}{\isachardot}{\kern0pt}\ {\isasymlangle}{\isasymlangle}n{\isacharcomma}{\kern0pt}\ m{\isasymrangle}{\isacharcomma}{\kern0pt}\ l{\isasymrangle}\ {\isasymin}\ p\ {\isasymand}\ v\ {\isacharequal}{\kern0pt}\ {\isasymlangle}{\isasymlangle}f\ {\isacharbackquote}{\kern0pt}\ n{\isacharcomma}{\kern0pt}\ m{\isasymrangle}{\isacharcomma}{\kern0pt}\ l{\isasymrangle}{\isachardoublequoteclose}\isanewline
\ \ \ \ \isacommand{apply}\isamarkupfalse%
{\isacharparenleft}{\kern0pt}rule\ Fn{\isacharunderscore}{\kern0pt}permE{\isacharparenright}{\kern0pt}\isanewline
\ \ \ \ \isacommand{using}\isamarkupfalse%
\ assms\ vin\ \isanewline
\ \ \ \ \isacommand{by}\isamarkupfalse%
\ auto\isanewline
\ \ \isacommand{then}\isamarkupfalse%
\ \isacommand{obtain}\isamarkupfalse%
\ n\ m\ l\ \isakeyword{where}\ H{\isacharcolon}{\kern0pt}\ {\isachardoublequoteopen}n\ {\isasymin}\ nat{\isachardoublequoteclose}\ {\isachardoublequoteopen}m\ {\isasymin}\ nat{\isachardoublequoteclose}\ {\isachardoublequoteopen}l\ {\isasymin}\ {\isadigit{2}}{\isachardoublequoteclose}\ {\isachardoublequoteopen}{\isacharless}{\kern0pt}{\isacharless}{\kern0pt}n{\isacharcomma}{\kern0pt}\ m{\isachargreater}{\kern0pt}{\isacharcomma}{\kern0pt}\ l{\isachargreater}{\kern0pt}\ {\isasymin}\ p{\isachardoublequoteclose}\ {\isachardoublequoteopen}v\ {\isacharequal}{\kern0pt}\ {\isacharless}{\kern0pt}{\isacharless}{\kern0pt}f{\isacharbackquote}{\kern0pt}n{\isacharcomma}{\kern0pt}\ m{\isachargreater}{\kern0pt}{\isacharcomma}{\kern0pt}\ l{\isachargreater}{\kern0pt}{\isachardoublequoteclose}\ \isacommand{by}\isamarkupfalse%
\ auto\isanewline
\ \ \isacommand{have}\isamarkupfalse%
\ {\isachardoublequoteopen}f{\isacharbackquote}{\kern0pt}n\ {\isasymin}\ nat{\isachardoublequoteclose}\isanewline
\ \ \ \ \isacommand{apply}\isamarkupfalse%
{\isacharparenleft}{\kern0pt}rule\ function{\isacharunderscore}{\kern0pt}value{\isacharunderscore}{\kern0pt}in{\isacharparenright}{\kern0pt}\isanewline
\ \ \ \ \isacommand{using}\isamarkupfalse%
\ nat{\isacharunderscore}{\kern0pt}perms{\isacharunderscore}{\kern0pt}def\ bij{\isacharunderscore}{\kern0pt}def\ inj{\isacharunderscore}{\kern0pt}def\ H\ assms\isanewline
\ \ \ \ \isacommand{by}\isamarkupfalse%
\ auto\isanewline
\ \ \isacommand{then}\isamarkupfalse%
\ \isacommand{show}\isamarkupfalse%
\ {\isachardoublequoteopen}v\ {\isasymin}\ {\isacharparenleft}{\kern0pt}nat\ {\isasymtimes}\ nat{\isacharparenright}{\kern0pt}\ {\isasymtimes}\ {\isadigit{2}}{\isachardoublequoteclose}\ \isanewline
\ \ \ \ \isacommand{using}\isamarkupfalse%
\ H\ \isanewline
\ \ \ \ \isacommand{by}\isamarkupfalse%
\ auto\isanewline
\isacommand{qed}\isamarkupfalse%
%
\endisatagproof
{\isafoldproof}%
%
\isadelimproof
\isanewline
%
\endisadelimproof
\isanewline
\isacommand{lemma}\isamarkupfalse%
\ function{\isacharunderscore}{\kern0pt}Fn{\isacharunderscore}{\kern0pt}perm\ {\isacharcolon}{\kern0pt}\ \isanewline
\ \ \isakeyword{fixes}\ f\ p\ \isanewline
\ \ \isakeyword{assumes}\ {\isachardoublequoteopen}f\ {\isasymin}\ nat{\isacharunderscore}{\kern0pt}perms{\isachardoublequoteclose}\ {\isachardoublequoteopen}p\ {\isasymin}\ Fn{\isachardoublequoteclose}\ \isanewline
\ \ \isakeyword{shows}\ {\isachardoublequoteopen}function{\isacharparenleft}{\kern0pt}Fn{\isacharunderscore}{\kern0pt}perm{\isacharparenleft}{\kern0pt}f{\isacharcomma}{\kern0pt}\ p{\isacharparenright}{\kern0pt}{\isacharparenright}{\kern0pt}{\isachardoublequoteclose}\isanewline
%
\isadelimproof
\ \ %
\endisadelimproof
%
\isatagproof
\isacommand{unfolding}\isamarkupfalse%
\ function{\isacharunderscore}{\kern0pt}def\isanewline
\isacommand{proof}\isamarkupfalse%
{\isacharparenleft}{\kern0pt}rule\ allI{\isacharcomma}{\kern0pt}\ rule\ allI{\isacharcomma}{\kern0pt}\ rule\ impI{\isacharcomma}{\kern0pt}\ rule\ allI{\isacharcomma}{\kern0pt}\ rule\ impI{\isacharparenright}{\kern0pt}\isanewline
\ \ \isacommand{fix}\isamarkupfalse%
\ x\ y\ y{\isacharprime}{\kern0pt}\ \isanewline
\ \ \isacommand{assume}\isamarkupfalse%
\ assms{\isadigit{1}}\ {\isacharcolon}{\kern0pt}\ {\isachardoublequoteopen}{\isacharless}{\kern0pt}x{\isacharcomma}{\kern0pt}\ y{\isachargreater}{\kern0pt}\ {\isasymin}\ Fn{\isacharunderscore}{\kern0pt}perm{\isacharparenleft}{\kern0pt}f{\isacharcomma}{\kern0pt}\ p{\isacharparenright}{\kern0pt}{\isachardoublequoteclose}\ {\isachardoublequoteopen}{\isacharless}{\kern0pt}x{\isacharcomma}{\kern0pt}\ y{\isacharprime}{\kern0pt}{\isachargreater}{\kern0pt}\ {\isasymin}\ Fn{\isacharunderscore}{\kern0pt}perm{\isacharparenleft}{\kern0pt}f{\isacharcomma}{\kern0pt}\ p{\isacharparenright}{\kern0pt}{\isachardoublequoteclose}\isanewline
\ \ \isacommand{have}\isamarkupfalse%
\ {\isachardoublequoteopen}{\isasymexists}n\ {\isasymin}\ nat{\isachardot}{\kern0pt}\ {\isasymexists}m\ {\isasymin}\ nat{\isachardot}{\kern0pt}\ {\isasymexists}l\ {\isasymin}\ {\isadigit{2}}{\isachardot}{\kern0pt}\ {\isacharless}{\kern0pt}{\isacharless}{\kern0pt}n{\isacharcomma}{\kern0pt}\ m{\isachargreater}{\kern0pt}{\isacharcomma}{\kern0pt}\ l{\isachargreater}{\kern0pt}\ {\isasymin}\ p\ {\isasymand}\ {\isacharless}{\kern0pt}x{\isacharcomma}{\kern0pt}\ y{\isachargreater}{\kern0pt}\ {\isacharequal}{\kern0pt}\ {\isacharless}{\kern0pt}{\isacharless}{\kern0pt}f{\isacharbackquote}{\kern0pt}n{\isacharcomma}{\kern0pt}\ m{\isachargreater}{\kern0pt}{\isacharcomma}{\kern0pt}\ l{\isachargreater}{\kern0pt}{\isachardoublequoteclose}\ \isanewline
\ \ \ \ \isacommand{apply}\isamarkupfalse%
{\isacharparenleft}{\kern0pt}rule\ Fn{\isacharunderscore}{\kern0pt}permE{\isacharparenright}{\kern0pt}\isanewline
\ \ \ \ \isacommand{using}\isamarkupfalse%
\ assms\ assms{\isadigit{1}}\ \isanewline
\ \ \ \ \isacommand{by}\isamarkupfalse%
\ auto\isanewline
\ \ \isacommand{then}\isamarkupfalse%
\ \isacommand{obtain}\isamarkupfalse%
\ n\ m\ l\ \isakeyword{where}\ H{\isacharcolon}{\kern0pt}\ {\isachardoublequoteopen}n\ {\isasymin}\ nat{\isachardoublequoteclose}\ {\isachardoublequoteopen}{\isacharless}{\kern0pt}{\isacharless}{\kern0pt}n{\isacharcomma}{\kern0pt}\ m{\isachargreater}{\kern0pt}{\isacharcomma}{\kern0pt}\ l{\isachargreater}{\kern0pt}\ {\isasymin}\ p{\isachardoublequoteclose}\ {\isachardoublequoteopen}{\isacharless}{\kern0pt}x{\isacharcomma}{\kern0pt}\ y{\isachargreater}{\kern0pt}\ {\isacharequal}{\kern0pt}\ {\isacharless}{\kern0pt}{\isacharless}{\kern0pt}f{\isacharbackquote}{\kern0pt}n{\isacharcomma}{\kern0pt}\ m{\isachargreater}{\kern0pt}{\isacharcomma}{\kern0pt}\ l{\isachargreater}{\kern0pt}{\isachardoublequoteclose}\ \isacommand{by}\isamarkupfalse%
\ auto\isanewline
\isanewline
\ \ \isacommand{have}\isamarkupfalse%
\ {\isachardoublequoteopen}{\isasymexists}n{\isacharprime}{\kern0pt}\ {\isasymin}\ nat{\isachardot}{\kern0pt}\ {\isasymexists}m{\isacharprime}{\kern0pt}\ {\isasymin}\ nat{\isachardot}{\kern0pt}\ {\isasymexists}l{\isacharprime}{\kern0pt}\ {\isasymin}\ {\isadigit{2}}{\isachardot}{\kern0pt}\ {\isacharless}{\kern0pt}{\isacharless}{\kern0pt}n{\isacharprime}{\kern0pt}{\isacharcomma}{\kern0pt}\ m{\isacharprime}{\kern0pt}{\isachargreater}{\kern0pt}{\isacharcomma}{\kern0pt}\ l{\isacharprime}{\kern0pt}{\isachargreater}{\kern0pt}\ {\isasymin}\ p\ {\isasymand}\ {\isacharless}{\kern0pt}x{\isacharcomma}{\kern0pt}\ y{\isacharprime}{\kern0pt}{\isachargreater}{\kern0pt}\ {\isacharequal}{\kern0pt}\ {\isacharless}{\kern0pt}{\isacharless}{\kern0pt}f{\isacharbackquote}{\kern0pt}n{\isacharprime}{\kern0pt}{\isacharcomma}{\kern0pt}\ m{\isacharprime}{\kern0pt}{\isachargreater}{\kern0pt}{\isacharcomma}{\kern0pt}\ l{\isacharprime}{\kern0pt}{\isachargreater}{\kern0pt}{\isachardoublequoteclose}\ \isanewline
\ \ \ \ \isacommand{apply}\isamarkupfalse%
{\isacharparenleft}{\kern0pt}rule\ Fn{\isacharunderscore}{\kern0pt}permE{\isacharparenright}{\kern0pt}\isanewline
\ \ \ \ \isacommand{using}\isamarkupfalse%
\ assms\ assms{\isadigit{1}}\ \isanewline
\ \ \ \ \isacommand{by}\isamarkupfalse%
\ auto\isanewline
\ \ \isacommand{then}\isamarkupfalse%
\ \isacommand{obtain}\isamarkupfalse%
\ n{\isacharprime}{\kern0pt}\ m{\isacharprime}{\kern0pt}\ l{\isacharprime}{\kern0pt}\ \isakeyword{where}\ H{\isacharprime}{\kern0pt}{\isacharcolon}{\kern0pt}\ {\isachardoublequoteopen}n{\isacharprime}{\kern0pt}\ {\isasymin}\ nat{\isachardoublequoteclose}\ {\isachardoublequoteopen}{\isacharless}{\kern0pt}{\isacharless}{\kern0pt}n{\isacharprime}{\kern0pt}{\isacharcomma}{\kern0pt}\ m{\isacharprime}{\kern0pt}{\isachargreater}{\kern0pt}{\isacharcomma}{\kern0pt}\ l{\isacharprime}{\kern0pt}{\isachargreater}{\kern0pt}\ {\isasymin}\ p{\isachardoublequoteclose}\ {\isachardoublequoteopen}{\isacharless}{\kern0pt}x{\isacharcomma}{\kern0pt}\ y{\isacharprime}{\kern0pt}{\isachargreater}{\kern0pt}\ {\isacharequal}{\kern0pt}\ {\isacharless}{\kern0pt}{\isacharless}{\kern0pt}f{\isacharbackquote}{\kern0pt}n{\isacharprime}{\kern0pt}{\isacharcomma}{\kern0pt}\ m{\isacharprime}{\kern0pt}{\isachargreater}{\kern0pt}{\isacharcomma}{\kern0pt}\ l{\isacharprime}{\kern0pt}{\isachargreater}{\kern0pt}{\isachardoublequoteclose}\ \isacommand{by}\isamarkupfalse%
\ auto\isanewline
\isanewline
\ \ \isacommand{have}\isamarkupfalse%
\ eq{\isacharcolon}{\kern0pt}\ {\isachardoublequoteopen}f{\isacharbackquote}{\kern0pt}n\ {\isacharequal}{\kern0pt}\ f{\isacharbackquote}{\kern0pt}n{\isacharprime}{\kern0pt}\ {\isasymand}\ m\ {\isacharequal}{\kern0pt}\ m{\isacharprime}{\kern0pt}{\isachardoublequoteclose}\ \isacommand{using}\isamarkupfalse%
\ H\ H{\isacharprime}{\kern0pt}\ \isacommand{by}\isamarkupfalse%
\ auto\isanewline
\isanewline
\ \ \isacommand{have}\isamarkupfalse%
\ {\isachardoublequoteopen}f\ {\isasymin}\ inj{\isacharparenleft}{\kern0pt}nat{\isacharcomma}{\kern0pt}\ nat{\isacharparenright}{\kern0pt}{\isachardoublequoteclose}\ \isanewline
\ \ \ \ \isacommand{using}\isamarkupfalse%
\ assms\ nat{\isacharunderscore}{\kern0pt}perms{\isacharunderscore}{\kern0pt}def\ bij{\isacharunderscore}{\kern0pt}is{\isacharunderscore}{\kern0pt}inj\ \isanewline
\ \ \ \ \isacommand{by}\isamarkupfalse%
\ auto\isanewline
\ \ \isacommand{then}\isamarkupfalse%
\ \isacommand{have}\isamarkupfalse%
\ eq{\isacharprime}{\kern0pt}{\isacharcolon}{\kern0pt}\ {\isachardoublequoteopen}n\ {\isacharequal}{\kern0pt}\ n{\isacharprime}{\kern0pt}{\isachardoublequoteclose}\ \isacommand{using}\isamarkupfalse%
\ H\ H{\isacharprime}{\kern0pt}\ eq\ inj{\isacharunderscore}{\kern0pt}def\ \isacommand{by}\isamarkupfalse%
\ auto\isanewline
\isanewline
\ \ \isacommand{have}\isamarkupfalse%
\ {\isachardoublequoteopen}function{\isacharparenleft}{\kern0pt}p{\isacharparenright}{\kern0pt}{\isachardoublequoteclose}\ \isacommand{using}\isamarkupfalse%
\ assms\ Fn{\isacharunderscore}{\kern0pt}def\ \isacommand{by}\isamarkupfalse%
\ auto\isanewline
\ \ \isacommand{then}\isamarkupfalse%
\ \isacommand{have}\isamarkupfalse%
\ {\isachardoublequoteopen}l\ {\isacharequal}{\kern0pt}\ l{\isacharprime}{\kern0pt}{\isachardoublequoteclose}\ \isacommand{using}\isamarkupfalse%
\ function{\isacharunderscore}{\kern0pt}def\ H\ H{\isacharprime}{\kern0pt}\ eq\ eq{\isacharprime}{\kern0pt}\ \isacommand{by}\isamarkupfalse%
\ auto\isanewline
\ \ \isacommand{then}\isamarkupfalse%
\ \isacommand{show}\isamarkupfalse%
\ {\isachardoublequoteopen}y\ {\isacharequal}{\kern0pt}\ y{\isacharprime}{\kern0pt}{\isachardoublequoteclose}\ \isacommand{using}\isamarkupfalse%
\ H\ H{\isacharprime}{\kern0pt}\ \isacommand{by}\isamarkupfalse%
\ auto\isanewline
\isacommand{qed}\isamarkupfalse%
%
\endisatagproof
{\isafoldproof}%
%
\isadelimproof
\isanewline
%
\endisadelimproof
\isanewline
\isacommand{lemma}\isamarkupfalse%
\ domain{\isacharunderscore}{\kern0pt}Fn{\isacharunderscore}{\kern0pt}perm\ {\isacharcolon}{\kern0pt}\ \isanewline
\ \ \isakeyword{fixes}\ f\ p\ \isanewline
\ \ \isakeyword{assumes}\ {\isachardoublequoteopen}f\ {\isasymin}\ nat{\isacharunderscore}{\kern0pt}perms{\isachardoublequoteclose}\ {\isachardoublequoteopen}p\ {\isasymin}\ Fn{\isachardoublequoteclose}\ \isanewline
\ \ \isakeyword{shows}\ {\isachardoublequoteopen}domain{\isacharparenleft}{\kern0pt}Fn{\isacharunderscore}{\kern0pt}perm{\isacharparenleft}{\kern0pt}f{\isacharcomma}{\kern0pt}\ p{\isacharparenright}{\kern0pt}{\isacharparenright}{\kern0pt}\ {\isacharequal}{\kern0pt}\ {\isacharbraceleft}{\kern0pt}\ {\isacharless}{\kern0pt}f{\isacharbackquote}{\kern0pt}n{\isacharcomma}{\kern0pt}\ m{\isachargreater}{\kern0pt}{\isachardot}{\kern0pt}\ {\isacharless}{\kern0pt}n{\isacharcomma}{\kern0pt}\ m{\isachargreater}{\kern0pt}\ {\isasymin}\ domain{\isacharparenleft}{\kern0pt}p{\isacharparenright}{\kern0pt}\ {\isacharbraceright}{\kern0pt}{\isachardoublequoteclose}\ \isanewline
%
\isadelimproof
%
\endisadelimproof
%
\isatagproof
\isacommand{proof}\isamarkupfalse%
\ {\isacharparenleft}{\kern0pt}rule\ equality{\isacharunderscore}{\kern0pt}iffI{\isacharcomma}{\kern0pt}\ rule\ iffI{\isacharparenright}{\kern0pt}\ \isanewline
\ \ \isacommand{fix}\isamarkupfalse%
\ v\ \isacommand{assume}\isamarkupfalse%
\ vin\ {\isacharcolon}{\kern0pt}\ {\isachardoublequoteopen}v\ {\isasymin}\ domain{\isacharparenleft}{\kern0pt}Fn{\isacharunderscore}{\kern0pt}perm{\isacharparenleft}{\kern0pt}f{\isacharcomma}{\kern0pt}\ p{\isacharparenright}{\kern0pt}{\isacharparenright}{\kern0pt}{\isachardoublequoteclose}\ \isanewline
\ \ \isacommand{then}\isamarkupfalse%
\ \isacommand{obtain}\isamarkupfalse%
\ u\ \isakeyword{where}\ uH{\isacharcolon}{\kern0pt}\ {\isachardoublequoteopen}{\isacharless}{\kern0pt}v{\isacharcomma}{\kern0pt}\ u{\isachargreater}{\kern0pt}\ {\isasymin}\ Fn{\isacharunderscore}{\kern0pt}perm{\isacharparenleft}{\kern0pt}f{\isacharcomma}{\kern0pt}\ p{\isacharparenright}{\kern0pt}{\isachardoublequoteclose}\ \isacommand{by}\isamarkupfalse%
\ auto\isanewline
\ \ \isacommand{then}\isamarkupfalse%
\ \isacommand{have}\isamarkupfalse%
\ {\isachardoublequoteopen}{\isasymexists}n\ {\isasymin}\ nat{\isachardot}{\kern0pt}\ {\isasymexists}m\ {\isasymin}\ nat{\isachardot}{\kern0pt}\ {\isasymexists}l\ {\isasymin}\ {\isadigit{2}}{\isachardot}{\kern0pt}\ {\isacharless}{\kern0pt}{\isacharless}{\kern0pt}n{\isacharcomma}{\kern0pt}\ m{\isachargreater}{\kern0pt}{\isacharcomma}{\kern0pt}\ l{\isachargreater}{\kern0pt}\ {\isasymin}\ p\ {\isasymand}\ {\isacharless}{\kern0pt}v{\isacharcomma}{\kern0pt}\ u{\isachargreater}{\kern0pt}\ {\isacharequal}{\kern0pt}\ {\isacharless}{\kern0pt}{\isacharless}{\kern0pt}f{\isacharbackquote}{\kern0pt}n{\isacharcomma}{\kern0pt}\ m{\isachargreater}{\kern0pt}{\isacharcomma}{\kern0pt}\ l{\isachargreater}{\kern0pt}{\isachardoublequoteclose}\ \isanewline
\ \ \ \ \isacommand{apply}\isamarkupfalse%
{\isacharparenleft}{\kern0pt}rule{\isacharunderscore}{\kern0pt}tac\ Fn{\isacharunderscore}{\kern0pt}permE{\isacharparenright}{\kern0pt}\isanewline
\ \ \ \ \isacommand{using}\isamarkupfalse%
\ assms\ \ \isanewline
\ \ \ \ \isacommand{by}\isamarkupfalse%
\ auto\isanewline
\ \ \isacommand{then}\isamarkupfalse%
\ \isacommand{obtain}\isamarkupfalse%
\ n\ m\ \isakeyword{where}\ {\isachardoublequoteopen}n\ {\isasymin}\ nat{\isachardoublequoteclose}\ {\isachardoublequoteopen}m\ {\isasymin}\ nat{\isachardoublequoteclose}\ {\isachardoublequoteopen}{\isacharless}{\kern0pt}{\isacharless}{\kern0pt}n{\isacharcomma}{\kern0pt}\ m{\isachargreater}{\kern0pt}{\isacharcomma}{\kern0pt}\ u{\isachargreater}{\kern0pt}\ {\isasymin}\ p{\isachardoublequoteclose}\ {\isachardoublequoteopen}v\ {\isacharequal}{\kern0pt}\ {\isacharless}{\kern0pt}f{\isacharbackquote}{\kern0pt}n{\isacharcomma}{\kern0pt}\ m{\isachargreater}{\kern0pt}{\isachardoublequoteclose}\ \isacommand{by}\isamarkupfalse%
\ auto\isanewline
\ \ \isacommand{then}\isamarkupfalse%
\ \isacommand{show}\isamarkupfalse%
\ {\isachardoublequoteopen}v\ {\isasymin}\ {\isacharbraceleft}{\kern0pt}\ {\isacharless}{\kern0pt}f{\isacharbackquote}{\kern0pt}n{\isacharcomma}{\kern0pt}\ m{\isachargreater}{\kern0pt}{\isachardot}{\kern0pt}\ {\isacharless}{\kern0pt}n{\isacharcomma}{\kern0pt}\ m{\isachargreater}{\kern0pt}\ {\isasymin}\ domain{\isacharparenleft}{\kern0pt}p{\isacharparenright}{\kern0pt}\ {\isacharbraceright}{\kern0pt}{\isachardoublequoteclose}\ \isacommand{by}\isamarkupfalse%
\ force\isanewline
\isacommand{next}\isamarkupfalse%
\isanewline
\ \ \isacommand{fix}\isamarkupfalse%
\ v\ \isacommand{assume}\isamarkupfalse%
\ {\isachardoublequoteopen}v\ {\isasymin}\ {\isacharbraceleft}{\kern0pt}\ {\isacharless}{\kern0pt}f{\isacharbackquote}{\kern0pt}n{\isacharcomma}{\kern0pt}\ m{\isachargreater}{\kern0pt}{\isachardot}{\kern0pt}\ {\isacharless}{\kern0pt}n{\isacharcomma}{\kern0pt}\ m{\isachargreater}{\kern0pt}\ {\isasymin}\ domain{\isacharparenleft}{\kern0pt}p{\isacharparenright}{\kern0pt}\ {\isacharbraceright}{\kern0pt}{\isachardoublequoteclose}\ \isanewline
\ \ \isacommand{then}\isamarkupfalse%
\ \isacommand{have}\isamarkupfalse%
\ {\isachardoublequoteopen}{\isasymexists}n\ m{\isachardot}{\kern0pt}\ {\isasymlangle}n{\isacharcomma}{\kern0pt}\ m{\isasymrangle}\ {\isasymin}\ domain{\isacharparenleft}{\kern0pt}p{\isacharparenright}{\kern0pt}\ {\isasymand}\ v\ {\isacharequal}{\kern0pt}\ {\isacharless}{\kern0pt}f{\isacharbackquote}{\kern0pt}n{\isacharcomma}{\kern0pt}\ m{\isachargreater}{\kern0pt}{\isachardoublequoteclose}\ \ \isanewline
\ \ \ \ \isacommand{apply}\isamarkupfalse%
{\isacharparenleft}{\kern0pt}rule{\isacharunderscore}{\kern0pt}tac\ pair{\isacharunderscore}{\kern0pt}rel{\isacharunderscore}{\kern0pt}arg{\isacharparenright}{\kern0pt}\isanewline
\ \ \ \ \isacommand{using}\isamarkupfalse%
\ assms\ Fn{\isacharunderscore}{\kern0pt}def\ relation{\isacharunderscore}{\kern0pt}def\isanewline
\ \ \ \ \ \isacommand{apply}\isamarkupfalse%
\ force\isanewline
\ \ \ \ \isacommand{by}\isamarkupfalse%
\ auto\isanewline
\ \ \isacommand{then}\isamarkupfalse%
\ \isacommand{obtain}\isamarkupfalse%
\ n\ m\ \isakeyword{where}\ H{\isacharcolon}{\kern0pt}\ {\isachardoublequoteopen}{\isacharless}{\kern0pt}n{\isacharcomma}{\kern0pt}\ m{\isachargreater}{\kern0pt}\ {\isasymin}\ domain{\isacharparenleft}{\kern0pt}p{\isacharparenright}{\kern0pt}{\isachardoublequoteclose}\ {\isachardoublequoteopen}v\ {\isacharequal}{\kern0pt}\ {\isacharless}{\kern0pt}f{\isacharbackquote}{\kern0pt}n{\isacharcomma}{\kern0pt}\ m{\isachargreater}{\kern0pt}{\isachardoublequoteclose}\ \isacommand{by}\isamarkupfalse%
\ auto\isanewline
\ \ \isacommand{then}\isamarkupfalse%
\ \isacommand{obtain}\isamarkupfalse%
\ l\ \isakeyword{where}\ {\isachardoublequoteopen}{\isacharless}{\kern0pt}{\isacharless}{\kern0pt}n{\isacharcomma}{\kern0pt}\ m{\isachargreater}{\kern0pt}{\isacharcomma}{\kern0pt}\ l{\isachargreater}{\kern0pt}\ {\isasymin}\ p{\isachardoublequoteclose}\ \isacommand{by}\isamarkupfalse%
\ auto\isanewline
\ \ \isacommand{then}\isamarkupfalse%
\ \isacommand{have}\isamarkupfalse%
\ {\isachardoublequoteopen}{\isacharless}{\kern0pt}{\isacharless}{\kern0pt}f{\isacharbackquote}{\kern0pt}n{\isacharcomma}{\kern0pt}\ m{\isachargreater}{\kern0pt}{\isacharcomma}{\kern0pt}\ l{\isachargreater}{\kern0pt}\ {\isasymin}\ Fn{\isacharunderscore}{\kern0pt}perm{\isacharparenleft}{\kern0pt}f{\isacharcomma}{\kern0pt}\ p{\isacharparenright}{\kern0pt}{\isachardoublequoteclose}\ \isanewline
\ \ \ \ \isacommand{unfolding}\isamarkupfalse%
\ Fn{\isacharunderscore}{\kern0pt}perm{\isacharunderscore}{\kern0pt}def\isanewline
\ \ \ \ \isacommand{by}\isamarkupfalse%
\ force\isanewline
\ \ \isacommand{then}\isamarkupfalse%
\ \isacommand{show}\isamarkupfalse%
\ {\isachardoublequoteopen}v\ {\isasymin}\ domain{\isacharparenleft}{\kern0pt}Fn{\isacharunderscore}{\kern0pt}perm{\isacharparenleft}{\kern0pt}f{\isacharcomma}{\kern0pt}\ p{\isacharparenright}{\kern0pt}{\isacharparenright}{\kern0pt}{\isachardoublequoteclose}\ \isacommand{using}\isamarkupfalse%
\ H\ \isacommand{by}\isamarkupfalse%
\ auto\isanewline
\isacommand{qed}\isamarkupfalse%
%
\endisatagproof
{\isafoldproof}%
%
\isadelimproof
\isanewline
%
\endisadelimproof
\isanewline
\isacommand{lemma}\isamarkupfalse%
\ Fn{\isacharunderscore}{\kern0pt}perm{\isacharunderscore}{\kern0pt}in{\isacharunderscore}{\kern0pt}Fn\ {\isacharcolon}{\kern0pt}\ \isanewline
\ \ \isakeyword{fixes}\ f\ p\ \isanewline
\ \ \isakeyword{assumes}\ {\isachardoublequoteopen}f\ {\isasymin}\ nat{\isacharunderscore}{\kern0pt}perms{\isachardoublequoteclose}\ {\isachardoublequoteopen}p\ {\isasymin}\ Fn{\isachardoublequoteclose}\ \isanewline
\ \ \isakeyword{shows}\ {\isachardoublequoteopen}Fn{\isacharunderscore}{\kern0pt}perm{\isacharparenleft}{\kern0pt}f{\isacharcomma}{\kern0pt}\ p{\isacharparenright}{\kern0pt}\ {\isasymin}\ Fn{\isachardoublequoteclose}\ \isanewline
%
\isadelimproof
%
\endisadelimproof
%
\isatagproof
\isacommand{proof}\isamarkupfalse%
\ {\isacharminus}{\kern0pt}\ \isanewline
\isanewline
\ \ \isacommand{have}\isamarkupfalse%
\ {\isachardoublequoteopen}domain{\isacharparenleft}{\kern0pt}Fn{\isacharunderscore}{\kern0pt}perm{\isacharparenleft}{\kern0pt}f{\isacharcomma}{\kern0pt}\ p{\isacharparenright}{\kern0pt}{\isacharparenright}{\kern0pt}\ {\isasymsubseteq}\ nat\ {\isasymtimes}\ nat{\isachardoublequoteclose}\ {\isacharparenleft}{\kern0pt}\isakeyword{is}\ {\isacharquery}{\kern0pt}A{\isacharparenright}{\kern0pt}\isanewline
\ \ \isacommand{proof}\isamarkupfalse%
{\isacharparenleft}{\kern0pt}rule\ subsetI{\isacharparenright}{\kern0pt}\isanewline
\ \ \ \ \isacommand{fix}\isamarkupfalse%
\ v\ \isacommand{assume}\isamarkupfalse%
\ {\isachardoublequoteopen}v\ {\isasymin}\ domain{\isacharparenleft}{\kern0pt}Fn{\isacharunderscore}{\kern0pt}perm{\isacharparenleft}{\kern0pt}f{\isacharcomma}{\kern0pt}\ p{\isacharparenright}{\kern0pt}{\isacharparenright}{\kern0pt}{\isachardoublequoteclose}\ \isanewline
\ \ \ \ \isacommand{then}\isamarkupfalse%
\ \isacommand{obtain}\isamarkupfalse%
\ u\ \isakeyword{where}\ {\isachardoublequoteopen}{\isacharless}{\kern0pt}v{\isacharcomma}{\kern0pt}\ u{\isachargreater}{\kern0pt}\ {\isasymin}\ Fn{\isacharunderscore}{\kern0pt}perm{\isacharparenleft}{\kern0pt}f{\isacharcomma}{\kern0pt}\ p{\isacharparenright}{\kern0pt}{\isachardoublequoteclose}\ \isacommand{by}\isamarkupfalse%
\ auto\isanewline
\ \ \ \ \isacommand{then}\isamarkupfalse%
\ \isacommand{have}\isamarkupfalse%
\ {\isachardoublequoteopen}{\isasymexists}n\ {\isasymin}\ nat{\isachardot}{\kern0pt}\ {\isasymexists}m\ {\isasymin}\ nat{\isachardot}{\kern0pt}\ {\isasymexists}l\ {\isasymin}\ {\isadigit{2}}{\isachardot}{\kern0pt}\ {\isacharless}{\kern0pt}{\isacharless}{\kern0pt}n{\isacharcomma}{\kern0pt}\ m{\isachargreater}{\kern0pt}{\isacharcomma}{\kern0pt}\ l{\isachargreater}{\kern0pt}\ {\isasymin}\ p\ {\isasymand}\ {\isacharless}{\kern0pt}v{\isacharcomma}{\kern0pt}\ u{\isachargreater}{\kern0pt}\ {\isacharequal}{\kern0pt}\ {\isacharless}{\kern0pt}{\isacharless}{\kern0pt}f{\isacharbackquote}{\kern0pt}n{\isacharcomma}{\kern0pt}\ m{\isachargreater}{\kern0pt}{\isacharcomma}{\kern0pt}\ l{\isachargreater}{\kern0pt}{\isachardoublequoteclose}\ \isanewline
\ \ \ \ \ \ \isacommand{apply}\isamarkupfalse%
{\isacharparenleft}{\kern0pt}rule{\isacharunderscore}{\kern0pt}tac\ Fn{\isacharunderscore}{\kern0pt}permE{\isacharparenright}{\kern0pt}\isanewline
\ \ \ \ \ \ \isacommand{using}\isamarkupfalse%
\ assms\ \ \isanewline
\ \ \ \ \ \ \isacommand{by}\isamarkupfalse%
\ auto\isanewline
\ \ \ \ \isacommand{then}\isamarkupfalse%
\ \isacommand{obtain}\isamarkupfalse%
\ n\ m\ \isakeyword{where}\ H{\isacharcolon}{\kern0pt}\ {\isachardoublequoteopen}n\ {\isasymin}\ nat{\isachardoublequoteclose}\ {\isachardoublequoteopen}m\ {\isasymin}\ nat{\isachardoublequoteclose}\ {\isachardoublequoteopen}{\isacharless}{\kern0pt}{\isacharless}{\kern0pt}n{\isacharcomma}{\kern0pt}\ m{\isachargreater}{\kern0pt}{\isacharcomma}{\kern0pt}\ u{\isachargreater}{\kern0pt}\ {\isasymin}\ p{\isachardoublequoteclose}\ {\isachardoublequoteopen}{\isacharless}{\kern0pt}v{\isacharcomma}{\kern0pt}\ u{\isachargreater}{\kern0pt}\ {\isacharequal}{\kern0pt}\ {\isacharless}{\kern0pt}{\isacharless}{\kern0pt}f{\isacharbackquote}{\kern0pt}n{\isacharcomma}{\kern0pt}\ m{\isachargreater}{\kern0pt}{\isacharcomma}{\kern0pt}\ u{\isachargreater}{\kern0pt}{\isachardoublequoteclose}\ \isacommand{by}\isamarkupfalse%
\ auto\isanewline
\ \ \ \ \isacommand{then}\isamarkupfalse%
\ \isacommand{have}\isamarkupfalse%
\ {\isachardoublequoteopen}v\ {\isacharequal}{\kern0pt}\ {\isacharless}{\kern0pt}f{\isacharbackquote}{\kern0pt}n{\isacharcomma}{\kern0pt}\ m{\isachargreater}{\kern0pt}{\isachardoublequoteclose}\ \isacommand{by}\isamarkupfalse%
\ auto\isanewline
\ \ \ \ \isacommand{have}\isamarkupfalse%
\ {\isachardoublequoteopen}f{\isacharbackquote}{\kern0pt}n\ {\isasymin}\ nat{\isachardoublequoteclose}\ \isanewline
\ \ \ \ \ \ \isacommand{apply}\isamarkupfalse%
{\isacharparenleft}{\kern0pt}rule\ function{\isacharunderscore}{\kern0pt}value{\isacharunderscore}{\kern0pt}in{\isacharparenright}{\kern0pt}\isanewline
\ \ \ \ \ \ \isacommand{using}\isamarkupfalse%
\ assms\ nat{\isacharunderscore}{\kern0pt}perms{\isacharunderscore}{\kern0pt}def\ bij{\isacharunderscore}{\kern0pt}is{\isacharunderscore}{\kern0pt}inj\ inj{\isacharunderscore}{\kern0pt}def\ H\isanewline
\ \ \ \ \ \ \isacommand{by}\isamarkupfalse%
\ auto\isanewline
\ \ \ \ \isacommand{then}\isamarkupfalse%
\ \isacommand{show}\isamarkupfalse%
\ {\isachardoublequoteopen}v\ {\isasymin}\ nat\ {\isasymtimes}\ nat{\isachardoublequoteclose}\ \isacommand{using}\isamarkupfalse%
\ H\ \isacommand{by}\isamarkupfalse%
\ auto\isanewline
\ \ \isacommand{qed}\isamarkupfalse%
\isanewline
\ \isanewline
\ \ \isacommand{have}\isamarkupfalse%
\ {\isachardoublequoteopen}finite{\isacharunderscore}{\kern0pt}M{\isacharparenleft}{\kern0pt}domain{\isacharparenleft}{\kern0pt}Fn{\isacharunderscore}{\kern0pt}perm{\isacharparenleft}{\kern0pt}f{\isacharcomma}{\kern0pt}\ p{\isacharparenright}{\kern0pt}{\isacharparenright}{\kern0pt}{\isacharparenright}{\kern0pt}{\isachardoublequoteclose}\ {\isacharparenleft}{\kern0pt}\isakeyword{is}\ {\isacharquery}{\kern0pt}B{\isacharparenright}{\kern0pt}\isanewline
\ \ \isacommand{proof}\isamarkupfalse%
\ {\isacharminus}{\kern0pt}\ \isanewline
\ \ \ \ \isacommand{have}\isamarkupfalse%
\ {\isachardoublequoteopen}finite{\isacharunderscore}{\kern0pt}M{\isacharparenleft}{\kern0pt}domain{\isacharparenleft}{\kern0pt}p{\isacharparenright}{\kern0pt}{\isacharparenright}{\kern0pt}{\isachardoublequoteclose}\ \isacommand{using}\isamarkupfalse%
\ assms\ Fn{\isacharunderscore}{\kern0pt}def\ \isacommand{by}\isamarkupfalse%
\ auto\isanewline
\ \ \ \ \isacommand{then}\isamarkupfalse%
\ \isacommand{obtain}\isamarkupfalse%
\ g\ N\ \isakeyword{where}\ gNH{\isacharcolon}{\kern0pt}\ {\isachardoublequoteopen}g\ {\isasymin}\ inj{\isacharparenleft}{\kern0pt}domain{\isacharparenleft}{\kern0pt}p{\isacharparenright}{\kern0pt}{\isacharcomma}{\kern0pt}\ N{\isacharparenright}{\kern0pt}{\isachardoublequoteclose}\ {\isachardoublequoteopen}g\ {\isasymin}\ M{\isachardoublequoteclose}\ {\isachardoublequoteopen}N\ {\isasymin}\ nat{\isachardoublequoteclose}\ \isacommand{unfolding}\isamarkupfalse%
\ finite{\isacharunderscore}{\kern0pt}M{\isacharunderscore}{\kern0pt}def\ \isacommand{by}\isamarkupfalse%
\ force\isanewline
\isanewline
\ \ \ \ \isacommand{define}\isamarkupfalse%
\ h\ \isakeyword{where}\ {\isachardoublequoteopen}h\ {\isasymequiv}\ {\isacharbraceleft}{\kern0pt}\ {\isacharless}{\kern0pt}{\isacharless}{\kern0pt}f{\isacharbackquote}{\kern0pt}n{\isacharcomma}{\kern0pt}\ m{\isachargreater}{\kern0pt}{\isacharcomma}{\kern0pt}\ g{\isacharbackquote}{\kern0pt}{\isacharless}{\kern0pt}n{\isacharcomma}{\kern0pt}\ m{\isachargreater}{\kern0pt}{\isachargreater}{\kern0pt}{\isachardot}{\kern0pt}\ {\isacharless}{\kern0pt}n{\isacharcomma}{\kern0pt}\ m{\isachargreater}{\kern0pt}\ {\isasymin}\ domain{\isacharparenleft}{\kern0pt}p{\isacharparenright}{\kern0pt}\ {\isacharbraceright}{\kern0pt}{\isachardoublequoteclose}\isanewline
\isanewline
\ \ \ \ \isacommand{have}\isamarkupfalse%
\ hinM\ {\isacharcolon}{\kern0pt}\ {\isachardoublequoteopen}h\ {\isasymin}\ M{\isachardoublequoteclose}\ \isanewline
\ \ \ \ \isacommand{proof}\isamarkupfalse%
\ {\isacharminus}{\kern0pt}\ \isanewline
\ \ \ \ \ \ \isanewline
\ \ \ \ \ \ \isacommand{define}\isamarkupfalse%
\ {\isasymphi}\ \isakeyword{where}\ {\isachardoublequoteopen}{\isasymphi}\ {\isasymequiv}\ Exists{\isacharparenleft}{\kern0pt}Exists{\isacharparenleft}{\kern0pt}Exists{\isacharparenleft}{\kern0pt}Exists{\isacharparenleft}{\kern0pt}Exists{\isacharparenleft}{\kern0pt}Exists{\isacharparenleft}{\kern0pt}\isanewline
\ \ \ \ \ \ \ \ \ \ \ \ \ \ \ \ \ \ \ \ \ \ \ \ \ \ \ \ And{\isacharparenleft}{\kern0pt}pair{\isacharunderscore}{\kern0pt}fm{\isacharparenleft}{\kern0pt}{\isadigit{0}}{\isacharcomma}{\kern0pt}\ {\isadigit{1}}{\isacharcomma}{\kern0pt}\ {\isadigit{4}}{\isacharparenright}{\kern0pt}{\isacharcomma}{\kern0pt}\ \isanewline
\ \ \ \ \ \ \ \ \ \ \ \ \ \ \ \ \ \ \ \ \ \ \ \ \ \ \ \ And{\isacharparenleft}{\kern0pt}Member{\isacharparenleft}{\kern0pt}{\isadigit{4}}{\isacharcomma}{\kern0pt}\ {\isadigit{7}}{\isacharparenright}{\kern0pt}{\isacharcomma}{\kern0pt}\isanewline
\ \ \ \ \ \ \ \ \ \ \ \ \ \ \ \ \ \ \ \ \ \ \ \ \ \ \ \ And{\isacharparenleft}{\kern0pt}fun{\isacharunderscore}{\kern0pt}apply{\isacharunderscore}{\kern0pt}fm{\isacharparenleft}{\kern0pt}{\isadigit{8}}{\isacharcomma}{\kern0pt}\ {\isadigit{0}}{\isacharcomma}{\kern0pt}\ {\isadigit{2}}{\isacharparenright}{\kern0pt}{\isacharcomma}{\kern0pt}\ \isanewline
\ \ \ \ \ \ \ \ \ \ \ \ \ \ \ \ \ \ \ \ \ \ \ \ \ \ \ \ And{\isacharparenleft}{\kern0pt}pair{\isacharunderscore}{\kern0pt}fm{\isacharparenleft}{\kern0pt}{\isadigit{2}}{\isacharcomma}{\kern0pt}\ {\isadigit{1}}{\isacharcomma}{\kern0pt}\ {\isadigit{3}}{\isacharparenright}{\kern0pt}{\isacharcomma}{\kern0pt}\ \isanewline
\ \ \ \ \ \ \ \ \ \ \ \ \ \ \ \ \ \ \ \ \ \ \ \ \ \ \ \ And{\isacharparenleft}{\kern0pt}fun{\isacharunderscore}{\kern0pt}apply{\isacharunderscore}{\kern0pt}fm{\isacharparenleft}{\kern0pt}{\isadigit{9}}{\isacharcomma}{\kern0pt}\ {\isadigit{4}}{\isacharcomma}{\kern0pt}\ {\isadigit{5}}{\isacharparenright}{\kern0pt}{\isacharcomma}{\kern0pt}\ \isanewline
\ \ \ \ \ \ \ \ \ \ \ \ \ \ \ \ \ \ \ \ \ \ \ \ \ \ \ \ \ \ \ \ pair{\isacharunderscore}{\kern0pt}fm{\isacharparenleft}{\kern0pt}{\isadigit{3}}{\isacharcomma}{\kern0pt}\ {\isadigit{5}}{\isacharcomma}{\kern0pt}\ {\isadigit{6}}{\isacharparenright}{\kern0pt}{\isacharparenright}{\kern0pt}{\isacharparenright}{\kern0pt}{\isacharparenright}{\kern0pt}{\isacharparenright}{\kern0pt}{\isacharparenright}{\kern0pt}{\isacharparenright}{\kern0pt}{\isacharparenright}{\kern0pt}{\isacharparenright}{\kern0pt}{\isacharparenright}{\kern0pt}{\isacharparenright}{\kern0pt}{\isacharparenright}{\kern0pt}{\isachardoublequoteclose}\ \ \isanewline
\ \ \ \ \ \ \isacommand{define}\isamarkupfalse%
\ X\ \isakeyword{where}\ {\isachardoublequoteopen}X\ {\isasymequiv}\ {\isacharbraceleft}{\kern0pt}\ v\ {\isasymin}\ {\isacharparenleft}{\kern0pt}nat\ {\isasymtimes}\ nat{\isacharparenright}{\kern0pt}\ {\isasymtimes}\ N{\isachardot}{\kern0pt}\ sats{\isacharparenleft}{\kern0pt}M{\isacharcomma}{\kern0pt}\ {\isasymphi}{\isacharcomma}{\kern0pt}\ {\isacharbrackleft}{\kern0pt}v{\isacharbrackright}{\kern0pt}\ {\isacharat}{\kern0pt}\ {\isacharbrackleft}{\kern0pt}domain{\isacharparenleft}{\kern0pt}p{\isacharparenright}{\kern0pt}{\isacharcomma}{\kern0pt}\ f{\isacharcomma}{\kern0pt}\ g{\isacharbrackright}{\kern0pt}{\isacharparenright}{\kern0pt}\ {\isacharbraceright}{\kern0pt}{\isachardoublequoteclose}\isanewline
\isanewline
\ \ \ \ \ \ \isacommand{have}\isamarkupfalse%
\ XinM{\isacharcolon}{\kern0pt}\ {\isachardoublequoteopen}X\ {\isasymin}\ M{\isachardoublequoteclose}\ \isanewline
\ \ \ \ \ \ \ \ \isacommand{unfolding}\isamarkupfalse%
\ X{\isacharunderscore}{\kern0pt}def\isanewline
\ \ \ \ \ \ \ \ \isacommand{apply}\isamarkupfalse%
{\isacharparenleft}{\kern0pt}rule\ separation{\isacharunderscore}{\kern0pt}notation{\isacharcomma}{\kern0pt}\ rule\ separation{\isacharunderscore}{\kern0pt}ax{\isacharparenright}{\kern0pt}\isanewline
\ \ \ \ \ \ \ \ \isacommand{unfolding}\isamarkupfalse%
\ {\isasymphi}{\isacharunderscore}{\kern0pt}def\ \isanewline
\ \ \ \ \ \ \ \ \ \ \ \isacommand{apply}\isamarkupfalse%
\ simp\isanewline
\ \ \ \ \ \ \ \ \isacommand{using}\isamarkupfalse%
\ domain{\isacharunderscore}{\kern0pt}closed\ gNH\ assms\ Fn{\isacharunderscore}{\kern0pt}in{\isacharunderscore}{\kern0pt}M\ nat{\isacharunderscore}{\kern0pt}perms{\isacharunderscore}{\kern0pt}in{\isacharunderscore}{\kern0pt}M\ transM\ \isanewline
\ \ \ \ \ \ \ \ \ \ \isacommand{apply}\isamarkupfalse%
\ force\ \isanewline
\ \ \ \ \ \ \ \ \isacommand{apply}\isamarkupfalse%
\ simp\isanewline
\ \ \ \ \ \ \ \ \ \isacommand{apply}\isamarkupfalse%
\ {\isacharparenleft}{\kern0pt}simp\ del{\isacharcolon}{\kern0pt}FOL{\isacharunderscore}{\kern0pt}sats{\isacharunderscore}{\kern0pt}iff\ pair{\isacharunderscore}{\kern0pt}abs\ add{\isacharcolon}{\kern0pt}\ fm{\isacharunderscore}{\kern0pt}defs\ nat{\isacharunderscore}{\kern0pt}simp{\isacharunderscore}{\kern0pt}union{\isacharparenright}{\kern0pt}\isanewline
\ \ \ \ \ \ \ \ \isacommand{using}\isamarkupfalse%
\ cartprod{\isacharunderscore}{\kern0pt}closed\ gNH\ nat{\isacharunderscore}{\kern0pt}in{\isacharunderscore}{\kern0pt}M\ transM\ \isanewline
\ \ \ \ \ \ \ \ \isacommand{by}\isamarkupfalse%
\ auto\isanewline
\isanewline
\ \ \ \ \ \ \isacommand{have}\isamarkupfalse%
\ {\isachardoublequoteopen}X\ {\isacharequal}{\kern0pt}\ {\isacharbraceleft}{\kern0pt}\ v\ {\isasymin}\ {\isacharparenleft}{\kern0pt}nat\ {\isasymtimes}\ nat{\isacharparenright}{\kern0pt}\ {\isasymtimes}\ N{\isachardot}{\kern0pt}\ \isanewline
\ \ \ \ \ \ \ \ \ \ \ \ \ \ \ \ \ \ \ \ {\isasymexists}gnm\ {\isasymin}\ M{\isachardot}{\kern0pt}\ {\isasymexists}nm\ {\isasymin}\ M{\isachardot}{\kern0pt}\ {\isasymexists}fnm\ {\isasymin}\ M{\isachardot}{\kern0pt}\ {\isasymexists}fn\ {\isasymin}\ M{\isachardot}{\kern0pt}\ {\isasymexists}m\ {\isasymin}\ M{\isachardot}{\kern0pt}\ {\isasymexists}n\ {\isasymin}\ M{\isachardot}{\kern0pt}\ \isanewline
\ \ \ \ \ \ \ \ \ \ \ \ \ \ \ \ \ \ \ \ {\isacharless}{\kern0pt}n{\isacharcomma}{\kern0pt}\ m{\isachargreater}{\kern0pt}\ {\isacharequal}{\kern0pt}\ nm\ {\isasymand}\ nm\ {\isasymin}\ domain{\isacharparenleft}{\kern0pt}p{\isacharparenright}{\kern0pt}\ {\isasymand}\ f{\isacharbackquote}{\kern0pt}n\ {\isacharequal}{\kern0pt}\ fn\ {\isasymand}\ fnm\ {\isacharequal}{\kern0pt}\ {\isacharless}{\kern0pt}fn{\isacharcomma}{\kern0pt}\ m{\isachargreater}{\kern0pt}\ {\isasymand}\ gnm\ {\isacharequal}{\kern0pt}\ g{\isacharbackquote}{\kern0pt}nm\ {\isasymand}\ v\ {\isacharequal}{\kern0pt}\ {\isacharless}{\kern0pt}fnm{\isacharcomma}{\kern0pt}\ gnm{\isachargreater}{\kern0pt}\ {\isacharbraceright}{\kern0pt}{\isachardoublequoteclose}\ \isanewline
\isanewline
\ \ \ \ \ \ \ \ \isacommand{unfolding}\isamarkupfalse%
\ X{\isacharunderscore}{\kern0pt}def\ {\isasymphi}{\isacharunderscore}{\kern0pt}def\isanewline
\ \ \ \ \ \ \ \ \isacommand{apply}\isamarkupfalse%
{\isacharparenleft}{\kern0pt}rule\ iff{\isacharunderscore}{\kern0pt}eq{\isacharparenright}{\kern0pt}\isanewline
\ \ \ \ \ \ \ \ \isacommand{apply}\isamarkupfalse%
{\isacharparenleft}{\kern0pt}rename{\isacharunderscore}{\kern0pt}tac\ v{\isacharcomma}{\kern0pt}\ subgoal{\isacharunderscore}{\kern0pt}tac\ {\isachardoublequoteopen}v\ {\isasymin}\ M\ {\isasymand}\ domain{\isacharparenleft}{\kern0pt}p{\isacharparenright}{\kern0pt}\ {\isasymin}\ M\ {\isasymand}\ f\ {\isasymin}\ M\ {\isasymand}\ g\ {\isasymin}\ M\ {\isasymand}\ N\ {\isasymin}\ M{\isachardoublequoteclose}{\isacharparenright}{\kern0pt}\isanewline
\ \ \ \ \ \ \ \ \isacommand{using}\isamarkupfalse%
\ pair{\isacharunderscore}{\kern0pt}in{\isacharunderscore}{\kern0pt}M{\isacharunderscore}{\kern0pt}iff\ zero{\isacharunderscore}{\kern0pt}in{\isacharunderscore}{\kern0pt}M\ succ{\isacharunderscore}{\kern0pt}in{\isacharunderscore}{\kern0pt}MI\ gNH\ domain{\isacharunderscore}{\kern0pt}closed\ assms\ transM\ nat{\isacharunderscore}{\kern0pt}perms{\isacharunderscore}{\kern0pt}in{\isacharunderscore}{\kern0pt}M\ Fn{\isacharunderscore}{\kern0pt}in{\isacharunderscore}{\kern0pt}M\ nat{\isacharunderscore}{\kern0pt}in{\isacharunderscore}{\kern0pt}M\isanewline
\ \ \ \ \ \ \ \ \ \isacommand{apply}\isamarkupfalse%
\ simp\isanewline
\ \ \ \ \ \ \ \ \ \isacommand{apply}\isamarkupfalse%
{\isacharparenleft}{\kern0pt}rule\ bex{\isacharunderscore}{\kern0pt}iff{\isacharparenright}{\kern0pt}{\isacharplus}{\kern0pt}\isanewline
\ \ \ \ \ \ \ \ \isacommand{using}\isamarkupfalse%
\ pair{\isacharunderscore}{\kern0pt}in{\isacharunderscore}{\kern0pt}M{\isacharunderscore}{\kern0pt}iff\ zero{\isacharunderscore}{\kern0pt}in{\isacharunderscore}{\kern0pt}M\ succ{\isacharunderscore}{\kern0pt}in{\isacharunderscore}{\kern0pt}MI\ gNH\ domain{\isacharunderscore}{\kern0pt}closed\ assms\ transM\ nat{\isacharunderscore}{\kern0pt}perms{\isacharunderscore}{\kern0pt}in{\isacharunderscore}{\kern0pt}M\ Fn{\isacharunderscore}{\kern0pt}in{\isacharunderscore}{\kern0pt}M\ nat{\isacharunderscore}{\kern0pt}in{\isacharunderscore}{\kern0pt}M\isanewline
\ \ \ \ \ \ \ \ \isacommand{by}\isamarkupfalse%
\ auto\isanewline
\ \ \ \ \ \ \isacommand{also}\isamarkupfalse%
\ \isacommand{have}\isamarkupfalse%
\ {\isachardoublequoteopen}{\isachardot}{\kern0pt}{\isachardot}{\kern0pt}{\isachardot}{\kern0pt}\ {\isacharequal}{\kern0pt}\ {\isacharbraceleft}{\kern0pt}\ v\ {\isasymin}\ {\isacharparenleft}{\kern0pt}nat\ {\isasymtimes}\ nat{\isacharparenright}{\kern0pt}\ {\isasymtimes}\ N{\isachardot}{\kern0pt}\ {\isasymexists}n\ m{\isachardot}{\kern0pt}\ {\isacharless}{\kern0pt}n{\isacharcomma}{\kern0pt}\ m{\isachargreater}{\kern0pt}\ {\isasymin}\ domain{\isacharparenleft}{\kern0pt}p{\isacharparenright}{\kern0pt}\ {\isasymand}\ v\ {\isacharequal}{\kern0pt}\ {\isacharless}{\kern0pt}{\isacharless}{\kern0pt}f{\isacharbackquote}{\kern0pt}n{\isacharcomma}{\kern0pt}\ m{\isachargreater}{\kern0pt}{\isacharcomma}{\kern0pt}\ g{\isacharbackquote}{\kern0pt}{\isacharless}{\kern0pt}n{\isacharcomma}{\kern0pt}\ m{\isachargreater}{\kern0pt}{\isachargreater}{\kern0pt}\ {\isacharbraceright}{\kern0pt}{\isachardoublequoteclose}\ \isanewline
\ \ \ \ \ \ \ \ \isacommand{apply}\isamarkupfalse%
{\isacharparenleft}{\kern0pt}rule\ iff{\isacharunderscore}{\kern0pt}eq{\isacharcomma}{\kern0pt}\ rule\ iffI{\isacharcomma}{\kern0pt}\ force{\isacharcomma}{\kern0pt}\ clarsimp{\isacharparenright}{\kern0pt}\isanewline
\ \ \ \ \ \ \ \ \isacommand{apply}\isamarkupfalse%
{\isacharparenleft}{\kern0pt}rename{\isacharunderscore}{\kern0pt}tac\ n\ m\ l{\isacharcomma}{\kern0pt}\ subgoal{\isacharunderscore}{\kern0pt}tac\ {\isachardoublequoteopen}{\isacharless}{\kern0pt}{\isacharless}{\kern0pt}n{\isacharcomma}{\kern0pt}\ m{\isachargreater}{\kern0pt}{\isacharcomma}{\kern0pt}\ l{\isachargreater}{\kern0pt}\ {\isasymin}\ M{\isachardoublequoteclose}{\isacharparenright}{\kern0pt}\isanewline
\ \ \ \ \ \ \ \ \ \isacommand{apply}\isamarkupfalse%
{\isacharparenleft}{\kern0pt}rename{\isacharunderscore}{\kern0pt}tac\ n\ m\ l{\isacharcomma}{\kern0pt}\ rule{\isacharunderscore}{\kern0pt}tac\ x{\isacharequal}{\kern0pt}{\isachardoublequoteopen}g{\isacharbackquote}{\kern0pt}{\isacharless}{\kern0pt}n{\isacharcomma}{\kern0pt}\ m{\isachargreater}{\kern0pt}{\isachardoublequoteclose}\ \isakeyword{in}\ bexI{\isacharparenright}{\kern0pt}\isanewline
\ \ \ \ \ \ \ \ \ \ \isacommand{apply}\isamarkupfalse%
{\isacharparenleft}{\kern0pt}rename{\isacharunderscore}{\kern0pt}tac\ n\ m\ l{\isacharcomma}{\kern0pt}\ rule{\isacharunderscore}{\kern0pt}tac\ x{\isacharequal}{\kern0pt}{\isachardoublequoteopen}{\isacharless}{\kern0pt}n{\isacharcomma}{\kern0pt}\ m{\isachargreater}{\kern0pt}{\isachardoublequoteclose}\ \isakeyword{in}\ bexI{\isacharparenright}{\kern0pt}\isanewline
\ \ \ \ \ \ \ \ \ \ \ \isacommand{apply}\isamarkupfalse%
{\isacharparenleft}{\kern0pt}rename{\isacharunderscore}{\kern0pt}tac\ n\ m\ l{\isacharcomma}{\kern0pt}\ rule{\isacharunderscore}{\kern0pt}tac\ x{\isacharequal}{\kern0pt}{\isachardoublequoteopen}{\isacharless}{\kern0pt}f{\isacharbackquote}{\kern0pt}n{\isacharcomma}{\kern0pt}\ m{\isachargreater}{\kern0pt}{\isachardoublequoteclose}\ \isakeyword{in}\ bexI{\isacharparenright}{\kern0pt}\ \ \isanewline
\ \ \ \ \ \ \ \ \ \ \ \ \isacommand{apply}\isamarkupfalse%
{\isacharparenleft}{\kern0pt}rename{\isacharunderscore}{\kern0pt}tac\ n\ m\ l{\isacharcomma}{\kern0pt}\ rule{\isacharunderscore}{\kern0pt}tac\ x{\isacharequal}{\kern0pt}{\isachardoublequoteopen}f{\isacharbackquote}{\kern0pt}n{\isachardoublequoteclose}\ \isakeyword{in}\ bexI{\isacharparenright}{\kern0pt}\isanewline
\ \ \ \ \ \ \ \ \ \ \ \ \ \isacommand{apply}\isamarkupfalse%
{\isacharparenleft}{\kern0pt}rename{\isacharunderscore}{\kern0pt}tac\ n\ m\ l{\isacharcomma}{\kern0pt}\ rule{\isacharunderscore}{\kern0pt}tac\ x{\isacharequal}{\kern0pt}{\isachardoublequoteopen}m{\isachardoublequoteclose}\ \isakeyword{in}\ bexI{\isacharparenright}{\kern0pt}\isanewline
\ \ \ \ \ \ \ \ \ \ \ \ \ \ \isacommand{apply}\isamarkupfalse%
{\isacharparenleft}{\kern0pt}rename{\isacharunderscore}{\kern0pt}tac\ n\ m\ l{\isacharcomma}{\kern0pt}\ rule{\isacharunderscore}{\kern0pt}tac\ x{\isacharequal}{\kern0pt}{\isachardoublequoteopen}n{\isachardoublequoteclose}\ \isakeyword{in}\ bexI{\isacharcomma}{\kern0pt}\ force{\isacharparenright}{\kern0pt}\isanewline
\ \ \ \ \ \ \ \ \isacommand{using}\isamarkupfalse%
\ pair{\isacharunderscore}{\kern0pt}in{\isacharunderscore}{\kern0pt}M{\isacharunderscore}{\kern0pt}iff\ zero{\isacharunderscore}{\kern0pt}in{\isacharunderscore}{\kern0pt}M\ succ{\isacharunderscore}{\kern0pt}in{\isacharunderscore}{\kern0pt}MI\ gNH\ domain{\isacharunderscore}{\kern0pt}closed\ assms\ transM\ nat{\isacharunderscore}{\kern0pt}perms{\isacharunderscore}{\kern0pt}in{\isacharunderscore}{\kern0pt}M\ Fn{\isacharunderscore}{\kern0pt}in{\isacharunderscore}{\kern0pt}M\ nat{\isacharunderscore}{\kern0pt}in{\isacharunderscore}{\kern0pt}M\isanewline
\ \ \ \ \ \ \ \ \ \ \ \ \ \ \isacommand{apply}\isamarkupfalse%
\ auto{\isacharbrackleft}{\kern0pt}{\isadigit{6}}{\isacharbrackright}{\kern0pt}\isanewline
\ \ \ \ \ \ \ \ \isacommand{using}\isamarkupfalse%
\ assms\ Fn{\isacharunderscore}{\kern0pt}in{\isacharunderscore}{\kern0pt}M\ transM\isanewline
\ \ \ \ \ \ \ \ \isacommand{by}\isamarkupfalse%
\ auto\isanewline
\ \ \ \ \ \ \isacommand{also}\isamarkupfalse%
\ \isacommand{have}\isamarkupfalse%
\ {\isachardoublequoteopen}{\isachardot}{\kern0pt}{\isachardot}{\kern0pt}{\isachardot}{\kern0pt}\ {\isacharequal}{\kern0pt}\ h{\isachardoublequoteclose}\ \isanewline
\ \ \ \ \ \ \isacommand{proof}\isamarkupfalse%
\ {\isacharminus}{\kern0pt}\ \isanewline
\ \ \ \ \ \ \ \ \isacommand{have}\isamarkupfalse%
\ H{\isacharcolon}{\kern0pt}\ {\isachardoublequoteopen}h\ {\isasymsubseteq}\ {\isacharbraceleft}{\kern0pt}v\ {\isasymin}\ {\isacharparenleft}{\kern0pt}nat\ {\isasymtimes}\ nat{\isacharparenright}{\kern0pt}\ {\isasymtimes}\ N\ {\isachardot}{\kern0pt}\ {\isasymexists}n\ m{\isachardot}{\kern0pt}\ {\isasymlangle}n{\isacharcomma}{\kern0pt}\ m{\isasymrangle}\ {\isasymin}\ domain{\isacharparenleft}{\kern0pt}p{\isacharparenright}{\kern0pt}\ {\isasymand}\ v\ {\isacharequal}{\kern0pt}\ {\isasymlangle}{\isasymlangle}f\ {\isacharbackquote}{\kern0pt}\ n{\isacharcomma}{\kern0pt}\ m{\isasymrangle}{\isacharcomma}{\kern0pt}\ g\ {\isacharbackquote}{\kern0pt}\ {\isasymlangle}n{\isacharcomma}{\kern0pt}\ m{\isasymrangle}{\isasymrangle}{\isacharbraceright}{\kern0pt}{\isachardoublequoteclose}\ \isanewline
\ \ \ \ \ \ \ \ \isacommand{proof}\isamarkupfalse%
{\isacharparenleft}{\kern0pt}rule\ subsetI{\isacharparenright}{\kern0pt}\ \isanewline
\ \ \ \ \ \ \ \ \ \ \isacommand{fix}\isamarkupfalse%
\ v\ \isacommand{assume}\isamarkupfalse%
\ {\isachardoublequoteopen}v\ {\isasymin}\ h{\isachardoublequoteclose}\ \isanewline
\ \ \ \ \ \ \ \ \ \ \isacommand{then}\isamarkupfalse%
\ \isacommand{obtain}\isamarkupfalse%
\ n\ m\ \isakeyword{where}\ nmH{\isacharcolon}{\kern0pt}\ {\isachardoublequoteopen}{\isacharless}{\kern0pt}n{\isacharcomma}{\kern0pt}\ m{\isachargreater}{\kern0pt}\ {\isasymin}\ domain{\isacharparenleft}{\kern0pt}p{\isacharparenright}{\kern0pt}{\isachardoublequoteclose}\ {\isachardoublequoteopen}v\ {\isacharequal}{\kern0pt}\ {\isacharless}{\kern0pt}{\isacharless}{\kern0pt}f{\isacharbackquote}{\kern0pt}n{\isacharcomma}{\kern0pt}\ m{\isachargreater}{\kern0pt}{\isacharcomma}{\kern0pt}\ g{\isacharbackquote}{\kern0pt}{\isacharless}{\kern0pt}n{\isacharcomma}{\kern0pt}\ m{\isachargreater}{\kern0pt}{\isachargreater}{\kern0pt}{\isachardoublequoteclose}\ {\isachardoublequoteopen}n\ {\isasymin}\ nat{\isachardoublequoteclose}\ {\isachardoublequoteopen}m\ {\isasymin}\ nat{\isachardoublequoteclose}\ \ \isanewline
\ \ \ \ \ \ \ \ \ \ \ \ \isacommand{unfolding}\isamarkupfalse%
\ h{\isacharunderscore}{\kern0pt}def\isanewline
\ \ \ \ \ \ \ \ \ \ \ \ \isacommand{apply}\isamarkupfalse%
{\isacharparenleft}{\kern0pt}subgoal{\isacharunderscore}{\kern0pt}tac\ {\isachardoublequoteopen}domain{\isacharparenleft}{\kern0pt}p{\isacharparenright}{\kern0pt}\ {\isasymsubseteq}\ nat\ {\isasymtimes}\ nat{\isachardoublequoteclose}{\isacharcomma}{\kern0pt}\ force{\isacharparenright}{\kern0pt}\isanewline
\ \ \ \ \ \ \ \ \ \ \ \ \isacommand{using}\isamarkupfalse%
\ assms\ Fn{\isacharunderscore}{\kern0pt}def\isanewline
\ \ \ \ \ \ \ \ \ \ \ \ \isacommand{by}\isamarkupfalse%
\ auto\isanewline
\ \ \ \ \ \ \ \ \ \ \isacommand{have}\isamarkupfalse%
\ fnin{\isacharcolon}{\kern0pt}\ {\isachardoublequoteopen}f{\isacharbackquote}{\kern0pt}n\ {\isasymin}\ nat{\isachardoublequoteclose}\ \isanewline
\ \ \ \ \ \ \ \ \ \ \ \ \isacommand{apply}\isamarkupfalse%
{\isacharparenleft}{\kern0pt}rule\ function{\isacharunderscore}{\kern0pt}value{\isacharunderscore}{\kern0pt}in{\isacharparenright}{\kern0pt}\isanewline
\ \ \ \ \ \ \ \ \ \ \ \ \isacommand{using}\isamarkupfalse%
\ assms\ nmH\ nat{\isacharunderscore}{\kern0pt}perms{\isacharunderscore}{\kern0pt}def\ bij{\isacharunderscore}{\kern0pt}def\ inj{\isacharunderscore}{\kern0pt}def\ \isanewline
\ \ \ \ \ \ \ \ \ \ \ \ \isacommand{by}\isamarkupfalse%
\ auto\isanewline
\ \ \ \ \ \ \ \ \ \ \isacommand{then}\isamarkupfalse%
\ \isacommand{have}\isamarkupfalse%
\ {\isachardoublequoteopen}g{\isacharbackquote}{\kern0pt}{\isacharless}{\kern0pt}n{\isacharcomma}{\kern0pt}\ m{\isachargreater}{\kern0pt}\ {\isasymin}\ N{\isachardoublequoteclose}\ \isanewline
\ \ \ \ \ \ \ \ \ \ \ \ \isacommand{apply}\isamarkupfalse%
{\isacharparenleft}{\kern0pt}rule{\isacharunderscore}{\kern0pt}tac\ function{\isacharunderscore}{\kern0pt}value{\isacharunderscore}{\kern0pt}in{\isacharparenright}{\kern0pt}\isanewline
\ \ \ \ \ \ \ \ \ \ \ \ \isacommand{using}\isamarkupfalse%
\ gNH\ nmH\ inj{\isacharunderscore}{\kern0pt}def\isanewline
\ \ \ \ \ \ \ \ \ \ \ \ \isacommand{by}\isamarkupfalse%
\ auto\isanewline
\ \ \ \ \ \ \ \ \ \ \isacommand{then}\isamarkupfalse%
\ \isacommand{show}\isamarkupfalse%
\ {\isachardoublequoteopen}v\ {\isasymin}\ {\isacharbraceleft}{\kern0pt}v\ {\isasymin}\ {\isacharparenleft}{\kern0pt}nat\ {\isasymtimes}\ nat{\isacharparenright}{\kern0pt}\ {\isasymtimes}\ N\ {\isachardot}{\kern0pt}\ {\isasymexists}n\ m{\isachardot}{\kern0pt}\ {\isasymlangle}n{\isacharcomma}{\kern0pt}\ m{\isasymrangle}\ {\isasymin}\ domain{\isacharparenleft}{\kern0pt}p{\isacharparenright}{\kern0pt}\ {\isasymand}\ v\ {\isacharequal}{\kern0pt}\ {\isasymlangle}{\isasymlangle}f\ {\isacharbackquote}{\kern0pt}\ n{\isacharcomma}{\kern0pt}\ m{\isasymrangle}{\isacharcomma}{\kern0pt}\ g\ {\isacharbackquote}{\kern0pt}\ {\isasymlangle}n{\isacharcomma}{\kern0pt}\ m{\isasymrangle}{\isasymrangle}{\isacharbraceright}{\kern0pt}{\isachardoublequoteclose}\ \isanewline
\ \ \ \ \ \ \ \ \ \ \ \ \isacommand{using}\isamarkupfalse%
\ nmH\ fnin\ \isanewline
\ \ \ \ \ \ \ \ \ \ \ \ \isacommand{by}\isamarkupfalse%
\ auto\isanewline
\ \ \ \ \ \ \ \ \isacommand{qed}\isamarkupfalse%
\isanewline
\ \ \ \ \ \ \ \ \isacommand{show}\isamarkupfalse%
\ {\isachardoublequoteopen}{\isacharbraceleft}{\kern0pt}v\ {\isasymin}\ {\isacharparenleft}{\kern0pt}nat\ {\isasymtimes}\ nat{\isacharparenright}{\kern0pt}\ {\isasymtimes}\ N\ {\isachardot}{\kern0pt}\ {\isasymexists}n\ m{\isachardot}{\kern0pt}\ {\isasymlangle}n{\isacharcomma}{\kern0pt}\ m{\isasymrangle}\ {\isasymin}\ domain{\isacharparenleft}{\kern0pt}p{\isacharparenright}{\kern0pt}\ {\isasymand}\ v\ {\isacharequal}{\kern0pt}\ {\isasymlangle}{\isasymlangle}f\ {\isacharbackquote}{\kern0pt}\ n{\isacharcomma}{\kern0pt}\ m{\isasymrangle}{\isacharcomma}{\kern0pt}\ g\ {\isacharbackquote}{\kern0pt}\ {\isasymlangle}n{\isacharcomma}{\kern0pt}\ m{\isasymrangle}{\isasymrangle}{\isacharbraceright}{\kern0pt}\ {\isacharequal}{\kern0pt}\ h{\isachardoublequoteclose}\ \isanewline
\ \ \ \ \ \ \ \ \ \ \isacommand{apply}\isamarkupfalse%
{\isacharparenleft}{\kern0pt}rule{\isacharunderscore}{\kern0pt}tac\ equality{\isacharunderscore}{\kern0pt}iffI{\isacharcomma}{\kern0pt}\ rule{\isacharunderscore}{\kern0pt}tac\ iffI{\isacharparenright}{\kern0pt}\isanewline
\ \ \ \ \ \ \ \ \ \ \ \isacommand{apply}\isamarkupfalse%
{\isacharparenleft}{\kern0pt}subst\ h{\isacharunderscore}{\kern0pt}def{\isacharcomma}{\kern0pt}\ force{\isacharparenright}{\kern0pt}\isanewline
\ \ \ \ \ \ \ \ \ \ \isacommand{using}\isamarkupfalse%
\ H\ \isanewline
\ \ \ \ \ \ \ \ \ \ \isacommand{by}\isamarkupfalse%
\ blast\ \isanewline
\ \ \ \ \ \ \isacommand{qed}\isamarkupfalse%
\isanewline
\ \ \ \ \ \ \isacommand{finally}\isamarkupfalse%
\ \isacommand{show}\isamarkupfalse%
\ {\isachardoublequoteopen}h\ {\isasymin}\ M{\isachardoublequoteclose}\ \isacommand{using}\isamarkupfalse%
\ XinM\ \isacommand{by}\isamarkupfalse%
\ auto\isanewline
\ \ \ \ \isacommand{qed}\isamarkupfalse%
\isanewline
\ \ \ \ \ \ \ \ \isanewline
\ \ \ \ \isacommand{have}\isamarkupfalse%
\ htype{\isacharcolon}{\kern0pt}\ {\isachardoublequoteopen}h\ {\isasymin}\ domain{\isacharparenleft}{\kern0pt}Fn{\isacharunderscore}{\kern0pt}perm{\isacharparenleft}{\kern0pt}f{\isacharcomma}{\kern0pt}\ p{\isacharparenright}{\kern0pt}{\isacharparenright}{\kern0pt}\ {\isasymrightarrow}\ N{\isachardoublequoteclose}\isanewline
\ \ \ \ \isacommand{proof}\isamarkupfalse%
{\isacharparenleft}{\kern0pt}rule\ Pi{\isacharunderscore}{\kern0pt}memberI{\isacharparenright}{\kern0pt}\isanewline
\ \ \ \ \ \ \isacommand{show}\isamarkupfalse%
\ {\isachardoublequoteopen}relation{\isacharparenleft}{\kern0pt}h{\isacharparenright}{\kern0pt}{\isachardoublequoteclose}\ \isanewline
\ \ \ \ \ \ \ \ \isacommand{unfolding}\isamarkupfalse%
\ relation{\isacharunderscore}{\kern0pt}def\ h{\isacharunderscore}{\kern0pt}def\isanewline
\ \ \ \ \ \ \ \ \isacommand{apply}\isamarkupfalse%
{\isacharparenleft}{\kern0pt}subgoal{\isacharunderscore}{\kern0pt}tac\ {\isachardoublequoteopen}domain{\isacharparenleft}{\kern0pt}p{\isacharparenright}{\kern0pt}\ {\isasymsubseteq}\ nat\ {\isasymtimes}\ nat{\isachardoublequoteclose}{\isacharcomma}{\kern0pt}\ force{\isacharparenright}{\kern0pt}\isanewline
\ \ \ \ \ \ \ \ \isacommand{using}\isamarkupfalse%
\ assms\ Fn{\isacharunderscore}{\kern0pt}def\ \isanewline
\ \ \ \ \ \ \ \ \isacommand{by}\isamarkupfalse%
\ auto\isanewline
\ \ \ \ \isacommand{next}\isamarkupfalse%
\isanewline
\ \ \ \ \ \ \isacommand{show}\isamarkupfalse%
\ {\isachardoublequoteopen}function{\isacharparenleft}{\kern0pt}h{\isacharparenright}{\kern0pt}{\isachardoublequoteclose}\ \isanewline
\ \ \ \ \ \ \ \ \isacommand{unfolding}\isamarkupfalse%
\ function{\isacharunderscore}{\kern0pt}def\ \isanewline
\ \ \ \ \ \ \isacommand{proof}\isamarkupfalse%
{\isacharparenleft}{\kern0pt}rule\ allI{\isacharcomma}{\kern0pt}\ rule\ allI{\isacharcomma}{\kern0pt}\ rule\ impI{\isacharcomma}{\kern0pt}\ rule\ allI{\isacharcomma}{\kern0pt}\ rule\ impI{\isacharparenright}{\kern0pt}\isanewline
\ \ \ \ \ \ \ \ \isacommand{fix}\isamarkupfalse%
\ x\ y\ y{\isacharprime}{\kern0pt}\ \isanewline
\ \ \ \ \ \ \ \ \isacommand{assume}\isamarkupfalse%
\ assms{\isadigit{1}}{\isacharcolon}{\kern0pt}\ {\isachardoublequoteopen}{\isacharless}{\kern0pt}x{\isacharcomma}{\kern0pt}\ y{\isachargreater}{\kern0pt}\ {\isasymin}\ h{\isachardoublequoteclose}\ {\isachardoublequoteopen}{\isacharless}{\kern0pt}x{\isacharcomma}{\kern0pt}\ y{\isacharprime}{\kern0pt}{\isachargreater}{\kern0pt}\ {\isasymin}\ h{\isachardoublequoteclose}\ \isanewline
\ \ \ \ \ \ \ \ \isacommand{obtain}\isamarkupfalse%
\ n\ m\ \isakeyword{where}\ H{\isacharcolon}{\kern0pt}\ {\isachardoublequoteopen}n\ {\isasymin}\ nat{\isachardoublequoteclose}\ {\isachardoublequoteopen}m\ {\isasymin}\ nat{\isachardoublequoteclose}\ {\isachardoublequoteopen}x\ {\isacharequal}{\kern0pt}\ {\isacharless}{\kern0pt}f{\isacharbackquote}{\kern0pt}n{\isacharcomma}{\kern0pt}\ m{\isachargreater}{\kern0pt}{\isachardoublequoteclose}\ {\isachardoublequoteopen}y\ {\isacharequal}{\kern0pt}\ g{\isacharbackquote}{\kern0pt}{\isacharless}{\kern0pt}n{\isacharcomma}{\kern0pt}\ m{\isachargreater}{\kern0pt}{\isachardoublequoteclose}\ \isanewline
\ \ \ \ \ \ \ \ \ \ \isacommand{using}\isamarkupfalse%
\ h{\isacharunderscore}{\kern0pt}def\ assms{\isadigit{1}}\ \isanewline
\ \ \ \ \ \ \ \ \ \ \isacommand{apply}\isamarkupfalse%
{\isacharparenleft}{\kern0pt}subgoal{\isacharunderscore}{\kern0pt}tac\ {\isachardoublequoteopen}domain{\isacharparenleft}{\kern0pt}p{\isacharparenright}{\kern0pt}\ {\isasymsubseteq}\ nat\ {\isasymtimes}\ nat{\isachardoublequoteclose}{\isacharcomma}{\kern0pt}\ force{\isacharparenright}{\kern0pt}\isanewline
\ \ \ \ \ \ \ \ \ \ \isacommand{using}\isamarkupfalse%
\ assms\ Fn{\isacharunderscore}{\kern0pt}def\ \isanewline
\ \ \ \ \ \ \ \ \ \ \isacommand{by}\isamarkupfalse%
\ auto\isanewline
\ \ \ \ \ \ \ \ \isacommand{obtain}\isamarkupfalse%
\ n{\isacharprime}{\kern0pt}\ m{\isacharprime}{\kern0pt}\ \isakeyword{where}\ H{\isacharprime}{\kern0pt}{\isacharcolon}{\kern0pt}\ {\isachardoublequoteopen}n{\isacharprime}{\kern0pt}\ {\isasymin}\ nat{\isachardoublequoteclose}\ {\isachardoublequoteopen}m{\isacharprime}{\kern0pt}\ {\isasymin}\ nat{\isachardoublequoteclose}\ {\isachardoublequoteopen}x\ {\isacharequal}{\kern0pt}\ {\isacharless}{\kern0pt}f{\isacharbackquote}{\kern0pt}n{\isacharprime}{\kern0pt}{\isacharcomma}{\kern0pt}\ m{\isacharprime}{\kern0pt}{\isachargreater}{\kern0pt}{\isachardoublequoteclose}\ {\isachardoublequoteopen}y{\isacharprime}{\kern0pt}\ {\isacharequal}{\kern0pt}\ g{\isacharbackquote}{\kern0pt}{\isacharless}{\kern0pt}n{\isacharprime}{\kern0pt}{\isacharcomma}{\kern0pt}\ m{\isacharprime}{\kern0pt}{\isachargreater}{\kern0pt}{\isachardoublequoteclose}\isanewline
\ \ \ \ \ \ \ \ \ \ \isacommand{using}\isamarkupfalse%
\ h{\isacharunderscore}{\kern0pt}def\ assms{\isadigit{1}}\ \isanewline
\ \ \ \ \ \ \ \ \ \ \isacommand{apply}\isamarkupfalse%
{\isacharparenleft}{\kern0pt}subgoal{\isacharunderscore}{\kern0pt}tac\ {\isachardoublequoteopen}domain{\isacharparenleft}{\kern0pt}p{\isacharparenright}{\kern0pt}\ {\isasymsubseteq}\ nat\ {\isasymtimes}\ nat{\isachardoublequoteclose}{\isacharcomma}{\kern0pt}\ force{\isacharparenright}{\kern0pt}\isanewline
\ \ \ \ \ \ \ \ \ \ \isacommand{using}\isamarkupfalse%
\ assms\ Fn{\isacharunderscore}{\kern0pt}def\ \isanewline
\ \ \ \ \ \ \ \ \ \ \isacommand{by}\isamarkupfalse%
\ auto\isanewline
\ \ \ \ \ \ \ \ \isanewline
\ \ \ \ \ \ \ \ \isacommand{have}\isamarkupfalse%
\ {\isachardoublequoteopen}f\ {\isasymin}\ inj{\isacharparenleft}{\kern0pt}nat{\isacharcomma}{\kern0pt}\ nat{\isacharparenright}{\kern0pt}{\isachardoublequoteclose}\ \isanewline
\ \ \ \ \ \ \ \ \ \ \isacommand{using}\isamarkupfalse%
\ assms\ nat{\isacharunderscore}{\kern0pt}perms{\isacharunderscore}{\kern0pt}def\ bij{\isacharunderscore}{\kern0pt}is{\isacharunderscore}{\kern0pt}inj\ \isanewline
\ \ \ \ \ \ \ \ \ \ \isacommand{by}\isamarkupfalse%
\ auto\isanewline
\ \ \ \ \ \ \ \ \isacommand{then}\isamarkupfalse%
\ \isacommand{have}\isamarkupfalse%
\ {\isachardoublequoteopen}n\ {\isacharequal}{\kern0pt}\ n{\isacharprime}{\kern0pt}{\isachardoublequoteclose}\ \isacommand{using}\isamarkupfalse%
\ H\ H{\isacharprime}{\kern0pt}\ inj{\isacharunderscore}{\kern0pt}def\ \isacommand{by}\isamarkupfalse%
\ auto\isanewline
\ \ \ \ \ \ \ \ \isacommand{then}\isamarkupfalse%
\ \isacommand{show}\isamarkupfalse%
\ {\isachardoublequoteopen}y\ {\isacharequal}{\kern0pt}\ y{\isacharprime}{\kern0pt}{\isachardoublequoteclose}\ \isacommand{using}\isamarkupfalse%
\ H\ H{\isacharprime}{\kern0pt}\ \isacommand{by}\isamarkupfalse%
\ auto\isanewline
\ \ \ \ \ \ \isacommand{qed}\isamarkupfalse%
\isanewline
\ \ \ \ \isacommand{next}\isamarkupfalse%
\ \isanewline
\ \ \ \ \ \ \isacommand{have}\isamarkupfalse%
\ {\isachardoublequoteopen}domain{\isacharparenleft}{\kern0pt}Fn{\isacharunderscore}{\kern0pt}perm{\isacharparenleft}{\kern0pt}f{\isacharcomma}{\kern0pt}\ p{\isacharparenright}{\kern0pt}{\isacharparenright}{\kern0pt}\ {\isacharequal}{\kern0pt}\ {\isacharbraceleft}{\kern0pt}\ {\isacharless}{\kern0pt}f{\isacharbackquote}{\kern0pt}n{\isacharcomma}{\kern0pt}\ m{\isachargreater}{\kern0pt}{\isachardot}{\kern0pt}\ {\isacharless}{\kern0pt}n{\isacharcomma}{\kern0pt}\ m{\isachargreater}{\kern0pt}\ {\isasymin}\ domain{\isacharparenleft}{\kern0pt}p{\isacharparenright}{\kern0pt}\ {\isacharbraceright}{\kern0pt}{\isachardoublequoteclose}\ \isanewline
\ \ \ \ \ \ \ \ \isacommand{apply}\isamarkupfalse%
{\isacharparenleft}{\kern0pt}subst\ domain{\isacharunderscore}{\kern0pt}Fn{\isacharunderscore}{\kern0pt}perm{\isacharparenright}{\kern0pt}\isanewline
\ \ \ \ \ \ \ \ \isacommand{using}\isamarkupfalse%
\ assms\isanewline
\ \ \ \ \ \ \ \ \isacommand{by}\isamarkupfalse%
\ auto\isanewline
\ \ \ \ \ \ \isacommand{also}\isamarkupfalse%
\ \isacommand{have}\isamarkupfalse%
\ {\isachardoublequoteopen}{\isachardot}{\kern0pt}{\isachardot}{\kern0pt}{\isachardot}{\kern0pt}\ {\isacharequal}{\kern0pt}\ domain{\isacharparenleft}{\kern0pt}h{\isacharparenright}{\kern0pt}{\isachardoublequoteclose}\ \isanewline
\ \ \ \ \ \ \ \ \isacommand{unfolding}\isamarkupfalse%
\ h{\isacharunderscore}{\kern0pt}def\ \isanewline
\ \ \ \ \ \ \ \ \isacommand{apply}\isamarkupfalse%
{\isacharparenleft}{\kern0pt}subgoal{\isacharunderscore}{\kern0pt}tac\ {\isachardoublequoteopen}domain{\isacharparenleft}{\kern0pt}p{\isacharparenright}{\kern0pt}\ {\isasymsubseteq}\ nat\ {\isasymtimes}\ nat{\isachardoublequoteclose}{\isacharcomma}{\kern0pt}\ force{\isacharparenright}{\kern0pt}\isanewline
\ \ \ \ \ \ \ \ \isacommand{using}\isamarkupfalse%
\ assms\ Fn{\isacharunderscore}{\kern0pt}def\ \isanewline
\ \ \ \ \ \ \ \ \isacommand{by}\isamarkupfalse%
\ auto\isanewline
\ \ \ \ \ \ \isacommand{finally}\isamarkupfalse%
\ \isacommand{show}\isamarkupfalse%
\ {\isachardoublequoteopen}domain{\isacharparenleft}{\kern0pt}Fn{\isacharunderscore}{\kern0pt}perm{\isacharparenleft}{\kern0pt}f{\isacharcomma}{\kern0pt}\ p{\isacharparenright}{\kern0pt}{\isacharparenright}{\kern0pt}\ {\isacharequal}{\kern0pt}\ domain{\isacharparenleft}{\kern0pt}h{\isacharparenright}{\kern0pt}{\isachardoublequoteclose}\ \isacommand{by}\isamarkupfalse%
\ auto\isanewline
\ \ \ \ \isacommand{next}\isamarkupfalse%
\ \isanewline
\ \ \ \ \ \ \isacommand{show}\isamarkupfalse%
\ {\isachardoublequoteopen}range{\isacharparenleft}{\kern0pt}h{\isacharparenright}{\kern0pt}\ {\isasymsubseteq}\ N{\isachardoublequoteclose}\ \isanewline
\ \ \ \ \ \ \isacommand{proof}\isamarkupfalse%
\ {\isacharparenleft}{\kern0pt}rule\ subsetI{\isacharparenright}{\kern0pt}\isanewline
\ \ \ \ \ \ \ \ \isacommand{fix}\isamarkupfalse%
\ v\ \isacommand{assume}\isamarkupfalse%
\ {\isachardoublequoteopen}v\ {\isasymin}\ range{\isacharparenleft}{\kern0pt}h{\isacharparenright}{\kern0pt}{\isachardoublequoteclose}\ \isanewline
\ \ \ \ \ \ \ \ \isacommand{then}\isamarkupfalse%
\ \isacommand{obtain}\isamarkupfalse%
\ a\ b\ \isakeyword{where}\ abH{\isacharcolon}{\kern0pt}\ {\isachardoublequoteopen}v\ {\isacharequal}{\kern0pt}\ g{\isacharbackquote}{\kern0pt}{\isacharless}{\kern0pt}a{\isacharcomma}{\kern0pt}\ b{\isachargreater}{\kern0pt}{\isachardoublequoteclose}\ {\isachardoublequoteopen}a\ {\isasymin}\ nat{\isachardoublequoteclose}\ {\isachardoublequoteopen}b\ {\isasymin}\ nat{\isachardoublequoteclose}\ {\isachardoublequoteopen}{\isacharless}{\kern0pt}a{\isacharcomma}{\kern0pt}\ b{\isachargreater}{\kern0pt}\ {\isasymin}\ domain{\isacharparenleft}{\kern0pt}p{\isacharparenright}{\kern0pt}{\isachardoublequoteclose}\ \isanewline
\ \ \ \ \ \ \ \ \ \ \isacommand{using}\isamarkupfalse%
\ h{\isacharunderscore}{\kern0pt}def\ \isanewline
\ \ \ \ \ \ \ \ \ \ \isacommand{apply}\isamarkupfalse%
{\isacharparenleft}{\kern0pt}subgoal{\isacharunderscore}{\kern0pt}tac\ {\isachardoublequoteopen}domain{\isacharparenleft}{\kern0pt}p{\isacharparenright}{\kern0pt}\ {\isasymsubseteq}\ nat\ {\isasymtimes}\ nat{\isachardoublequoteclose}{\isacharcomma}{\kern0pt}\ force{\isacharparenright}{\kern0pt}\isanewline
\ \ \ \ \ \ \ \ \ \ \isacommand{using}\isamarkupfalse%
\ assms\ Fn{\isacharunderscore}{\kern0pt}def\ \isanewline
\ \ \ \ \ \ \ \ \ \ \isacommand{by}\isamarkupfalse%
\ auto\isanewline
\ \ \ \ \ \ \ \ \isacommand{then}\isamarkupfalse%
\ \isacommand{have}\isamarkupfalse%
\ {\isachardoublequoteopen}g{\isacharbackquote}{\kern0pt}{\isacharless}{\kern0pt}a{\isacharcomma}{\kern0pt}\ b{\isachargreater}{\kern0pt}\ {\isasymin}\ N{\isachardoublequoteclose}\ \isanewline
\ \ \ \ \ \ \ \ \ \ \isacommand{apply}\isamarkupfalse%
{\isacharparenleft}{\kern0pt}rule{\isacharunderscore}{\kern0pt}tac\ function{\isacharunderscore}{\kern0pt}value{\isacharunderscore}{\kern0pt}in{\isacharparenright}{\kern0pt}\isanewline
\ \ \ \ \ \ \ \ \ \ \isacommand{using}\isamarkupfalse%
\ gNH\ inj{\isacharunderscore}{\kern0pt}def\ \isanewline
\ \ \ \ \ \ \ \ \ \ \isacommand{by}\isamarkupfalse%
\ auto\isanewline
\ \ \ \ \ \ \ \ \isacommand{then}\isamarkupfalse%
\ \isacommand{show}\isamarkupfalse%
\ {\isachardoublequoteopen}v\ {\isasymin}\ N{\isachardoublequoteclose}\ \isacommand{using}\isamarkupfalse%
\ abH\ \isacommand{by}\isamarkupfalse%
\ auto\isanewline
\ \ \ \ \ \ \isacommand{qed}\isamarkupfalse%
\isanewline
\ \ \ \ \isacommand{qed}\isamarkupfalse%
\isanewline
\isanewline
\ \ \ \ \isacommand{have}\isamarkupfalse%
\ {\isachardoublequoteopen}{\isasymforall}x{\isasymin}domain{\isacharparenleft}{\kern0pt}Fn{\isacharunderscore}{\kern0pt}perm{\isacharparenleft}{\kern0pt}f{\isacharcomma}{\kern0pt}\ p{\isacharparenright}{\kern0pt}{\isacharparenright}{\kern0pt}{\isachardot}{\kern0pt}\ {\isasymforall}y{\isasymin}domain{\isacharparenleft}{\kern0pt}Fn{\isacharunderscore}{\kern0pt}perm{\isacharparenleft}{\kern0pt}f{\isacharcomma}{\kern0pt}\ p{\isacharparenright}{\kern0pt}{\isacharparenright}{\kern0pt}{\isachardot}{\kern0pt}\ h\ {\isacharbackquote}{\kern0pt}\ x\ {\isacharequal}{\kern0pt}\ h\ {\isacharbackquote}{\kern0pt}\ y\ {\isasymlongrightarrow}\ x\ {\isacharequal}{\kern0pt}\ y{\isachardoublequoteclose}\isanewline
\ \ \ \ \isacommand{proof}\isamarkupfalse%
{\isacharparenleft}{\kern0pt}rule\ ballI{\isacharcomma}{\kern0pt}\ rule\ ballI{\isacharcomma}{\kern0pt}\ rule\ impI{\isacharparenright}{\kern0pt}\isanewline
\ \ \ \ \ \ \isacommand{fix}\isamarkupfalse%
\ x\ y\ \isanewline
\ \ \ \ \ \ \isacommand{assume}\isamarkupfalse%
\ assms{\isadigit{1}}{\isacharcolon}{\kern0pt}\ {\isachardoublequoteopen}x\ {\isasymin}\ domain{\isacharparenleft}{\kern0pt}Fn{\isacharunderscore}{\kern0pt}perm{\isacharparenleft}{\kern0pt}f{\isacharcomma}{\kern0pt}\ p{\isacharparenright}{\kern0pt}{\isacharparenright}{\kern0pt}{\isachardoublequoteclose}\ {\isachardoublequoteopen}y\ {\isasymin}\ domain{\isacharparenleft}{\kern0pt}Fn{\isacharunderscore}{\kern0pt}perm{\isacharparenleft}{\kern0pt}f{\isacharcomma}{\kern0pt}\ p{\isacharparenright}{\kern0pt}{\isacharparenright}{\kern0pt}{\isachardoublequoteclose}\ {\isachardoublequoteopen}h\ {\isacharbackquote}{\kern0pt}\ x\ {\isacharequal}{\kern0pt}\ h\ {\isacharbackquote}{\kern0pt}\ y{\isachardoublequoteclose}\isanewline
\ \ \ \ \ \ \isacommand{obtain}\isamarkupfalse%
\ vx\ \isakeyword{where}\ vxH{\isacharcolon}{\kern0pt}\ {\isachardoublequoteopen}{\isacharless}{\kern0pt}x{\isacharcomma}{\kern0pt}\ vx{\isachargreater}{\kern0pt}\ {\isasymin}\ h{\isachardoublequoteclose}\ {\isachardoublequoteopen}vx\ {\isasymin}\ N{\isachardoublequoteclose}\ \isacommand{using}\isamarkupfalse%
\ assms{\isadigit{1}}\ htype\ Pi{\isacharunderscore}{\kern0pt}def\ \isacommand{by}\isamarkupfalse%
\ force\isanewline
\ \ \ \ \ \ \isacommand{obtain}\isamarkupfalse%
\ vy\ \isakeyword{where}\ vyH{\isacharcolon}{\kern0pt}\ {\isachardoublequoteopen}{\isacharless}{\kern0pt}y{\isacharcomma}{\kern0pt}\ vy{\isachargreater}{\kern0pt}\ {\isasymin}\ h{\isachardoublequoteclose}\ {\isachardoublequoteopen}vy\ {\isasymin}\ N{\isachardoublequoteclose}\ \isacommand{using}\isamarkupfalse%
\ assms{\isadigit{1}}\ htype\ Pi{\isacharunderscore}{\kern0pt}def\ \isacommand{by}\isamarkupfalse%
\ force\isanewline
\isanewline
\ \ \ \ \ \ \isacommand{have}\isamarkupfalse%
\ hxeq{\isacharcolon}{\kern0pt}\ {\isachardoublequoteopen}h{\isacharbackquote}{\kern0pt}x\ {\isacharequal}{\kern0pt}\ vx{\isachardoublequoteclose}\ \isanewline
\ \ \ \ \ \ \ \ \isacommand{apply}\isamarkupfalse%
{\isacharparenleft}{\kern0pt}rule\ function{\isacharunderscore}{\kern0pt}apply{\isacharunderscore}{\kern0pt}equality{\isacharparenright}{\kern0pt}\isanewline
\ \ \ \ \ \ \ \ \isacommand{using}\isamarkupfalse%
\ vxH\ htype\ Pi{\isacharunderscore}{\kern0pt}def\isanewline
\ \ \ \ \ \ \ \ \isacommand{by}\isamarkupfalse%
\ auto\isanewline
\ \ \ \ \ \ \isacommand{have}\isamarkupfalse%
\ hyeq{\isacharcolon}{\kern0pt}\ {\isachardoublequoteopen}h{\isacharbackquote}{\kern0pt}y\ {\isacharequal}{\kern0pt}\ vy{\isachardoublequoteclose}\ \isanewline
\ \ \ \ \ \ \ \ \isacommand{apply}\isamarkupfalse%
{\isacharparenleft}{\kern0pt}rule\ function{\isacharunderscore}{\kern0pt}apply{\isacharunderscore}{\kern0pt}equality{\isacharparenright}{\kern0pt}\isanewline
\ \ \ \ \ \ \ \ \isacommand{using}\isamarkupfalse%
\ vyH\ htype\ Pi{\isacharunderscore}{\kern0pt}def\isanewline
\ \ \ \ \ \ \ \ \isacommand{by}\isamarkupfalse%
\ auto\isanewline
\isanewline
\ \ \ \ \ \ \isacommand{obtain}\isamarkupfalse%
\ n\ m\ \isakeyword{where}\ nmH{\isacharcolon}{\kern0pt}\ {\isachardoublequoteopen}vx\ {\isacharequal}{\kern0pt}\ g{\isacharbackquote}{\kern0pt}{\isacharless}{\kern0pt}n{\isacharcomma}{\kern0pt}\ m{\isachargreater}{\kern0pt}{\isachardoublequoteclose}\ {\isachardoublequoteopen}n\ {\isasymin}\ nat{\isachardoublequoteclose}\ {\isachardoublequoteopen}m\ {\isasymin}\ nat{\isachardoublequoteclose}\ {\isachardoublequoteopen}{\isacharless}{\kern0pt}n{\isacharcomma}{\kern0pt}\ m{\isachargreater}{\kern0pt}\ {\isasymin}\ domain{\isacharparenleft}{\kern0pt}p{\isacharparenright}{\kern0pt}{\isachardoublequoteclose}\ {\isachardoublequoteopen}x\ {\isacharequal}{\kern0pt}\ {\isacharless}{\kern0pt}f{\isacharbackquote}{\kern0pt}n{\isacharcomma}{\kern0pt}\ m{\isachargreater}{\kern0pt}{\isachardoublequoteclose}\ \isanewline
\ \ \ \ \ \ \ \ \isacommand{using}\isamarkupfalse%
\ vxH\ \isacommand{unfolding}\isamarkupfalse%
\ h{\isacharunderscore}{\kern0pt}def\ \isanewline
\ \ \ \ \ \ \ \ \isacommand{apply}\isamarkupfalse%
{\isacharparenleft}{\kern0pt}subgoal{\isacharunderscore}{\kern0pt}tac\ {\isachardoublequoteopen}domain{\isacharparenleft}{\kern0pt}p{\isacharparenright}{\kern0pt}\ {\isasymsubseteq}\ nat\ {\isasymtimes}\ nat{\isachardoublequoteclose}{\isacharcomma}{\kern0pt}\ force{\isacharparenright}{\kern0pt}\isanewline
\ \ \ \ \ \ \ \ \isacommand{using}\isamarkupfalse%
\ assms\ Fn{\isacharunderscore}{\kern0pt}def\ \isanewline
\ \ \ \ \ \ \ \ \isacommand{by}\isamarkupfalse%
\ auto\isanewline
\ \ \ \ \ \ \isacommand{obtain}\isamarkupfalse%
\ n{\isacharprime}{\kern0pt}\ m{\isacharprime}{\kern0pt}\ \isakeyword{where}\ n{\isacharprime}{\kern0pt}m{\isacharprime}{\kern0pt}H{\isacharcolon}{\kern0pt}\ {\isachardoublequoteopen}vy\ {\isacharequal}{\kern0pt}\ g{\isacharbackquote}{\kern0pt}{\isacharless}{\kern0pt}n{\isacharprime}{\kern0pt}{\isacharcomma}{\kern0pt}\ m{\isacharprime}{\kern0pt}{\isachargreater}{\kern0pt}{\isachardoublequoteclose}\ {\isachardoublequoteopen}n{\isacharprime}{\kern0pt}\ {\isasymin}\ nat{\isachardoublequoteclose}\ {\isachardoublequoteopen}m{\isacharprime}{\kern0pt}\ {\isasymin}\ nat{\isachardoublequoteclose}\ {\isachardoublequoteopen}{\isacharless}{\kern0pt}n{\isacharprime}{\kern0pt}{\isacharcomma}{\kern0pt}\ m{\isacharprime}{\kern0pt}{\isachargreater}{\kern0pt}\ {\isasymin}\ domain{\isacharparenleft}{\kern0pt}p{\isacharparenright}{\kern0pt}{\isachardoublequoteclose}\ {\isachardoublequoteopen}y\ {\isacharequal}{\kern0pt}\ {\isacharless}{\kern0pt}f{\isacharbackquote}{\kern0pt}n{\isacharprime}{\kern0pt}{\isacharcomma}{\kern0pt}\ m{\isacharprime}{\kern0pt}{\isachargreater}{\kern0pt}{\isachardoublequoteclose}\ \isanewline
\ \ \ \ \ \ \ \ \isacommand{using}\isamarkupfalse%
\ vyH\ \isacommand{unfolding}\isamarkupfalse%
\ h{\isacharunderscore}{\kern0pt}def\ \isanewline
\ \ \ \ \ \ \ \ \isacommand{apply}\isamarkupfalse%
{\isacharparenleft}{\kern0pt}subgoal{\isacharunderscore}{\kern0pt}tac\ {\isachardoublequoteopen}domain{\isacharparenleft}{\kern0pt}p{\isacharparenright}{\kern0pt}\ {\isasymsubseteq}\ nat\ {\isasymtimes}\ nat{\isachardoublequoteclose}{\isacharcomma}{\kern0pt}\ force{\isacharparenright}{\kern0pt}\isanewline
\ \ \ \ \ \ \ \ \isacommand{using}\isamarkupfalse%
\ assms\ Fn{\isacharunderscore}{\kern0pt}def\ \isanewline
\ \ \ \ \ \ \ \ \isacommand{by}\isamarkupfalse%
\ auto\isanewline
\isanewline
\ \ \ \ \ \ \isacommand{have}\isamarkupfalse%
\ {\isachardoublequoteopen}{\isacharless}{\kern0pt}n{\isacharcomma}{\kern0pt}\ m{\isachargreater}{\kern0pt}\ {\isacharequal}{\kern0pt}\ {\isacharless}{\kern0pt}n{\isacharprime}{\kern0pt}{\isacharcomma}{\kern0pt}\ m{\isacharprime}{\kern0pt}{\isachargreater}{\kern0pt}{\isachardoublequoteclose}\ \isanewline
\ \ \ \ \ \ \ \ \isacommand{apply}\isamarkupfalse%
{\isacharparenleft}{\kern0pt}rule\ inj{\isacharunderscore}{\kern0pt}apply{\isacharunderscore}{\kern0pt}equality{\isacharparenright}{\kern0pt}\isanewline
\ \ \ \ \ \ \ \ \isacommand{using}\isamarkupfalse%
\ gNH\ assms{\isadigit{1}}\ nmH\ n{\isacharprime}{\kern0pt}m{\isacharprime}{\kern0pt}H\ hxeq\ hyeq\ \isanewline
\ \ \ \ \ \ \ \ \isacommand{by}\isamarkupfalse%
\ auto\isanewline
\ \ \ \ \ \ \isacommand{then}\isamarkupfalse%
\ \isacommand{show}\isamarkupfalse%
\ {\isachardoublequoteopen}x\ {\isacharequal}{\kern0pt}\ y{\isachardoublequoteclose}\ \isacommand{using}\isamarkupfalse%
\ nmH\ n{\isacharprime}{\kern0pt}m{\isacharprime}{\kern0pt}H\ \isacommand{by}\isamarkupfalse%
\ auto\isanewline
\ \ \ \ \isacommand{qed}\isamarkupfalse%
\isanewline
\isanewline
\ \ \ \ \isacommand{then}\isamarkupfalse%
\ \isacommand{show}\isamarkupfalse%
\ {\isachardoublequoteopen}finite{\isacharunderscore}{\kern0pt}M{\isacharparenleft}{\kern0pt}domain{\isacharparenleft}{\kern0pt}Fn{\isacharunderscore}{\kern0pt}perm{\isacharparenleft}{\kern0pt}f{\isacharcomma}{\kern0pt}\ p{\isacharparenright}{\kern0pt}{\isacharparenright}{\kern0pt}{\isacharparenright}{\kern0pt}{\isachardoublequoteclose}\ \isanewline
\ \ \ \ \ \ \isacommand{unfolding}\isamarkupfalse%
\ finite{\isacharunderscore}{\kern0pt}M{\isacharunderscore}{\kern0pt}def\ \isanewline
\ \ \ \ \ \ \isacommand{apply}\isamarkupfalse%
{\isacharparenleft}{\kern0pt}rule{\isacharunderscore}{\kern0pt}tac\ x{\isacharequal}{\kern0pt}N\ \isakeyword{in}\ bexI{\isacharparenright}{\kern0pt}\isanewline
\ \ \ \ \ \ \ \isacommand{apply}\isamarkupfalse%
{\isacharparenleft}{\kern0pt}rule{\isacharunderscore}{\kern0pt}tac\ a{\isacharequal}{\kern0pt}h\ \isakeyword{in}\ not{\isacharunderscore}{\kern0pt}emptyI{\isacharparenright}{\kern0pt}\isanewline
\ \ \ \ \ \ \isacommand{unfolding}\isamarkupfalse%
\ inj{\isacharunderscore}{\kern0pt}def\ \isanewline
\ \ \ \ \ \ \isacommand{using}\isamarkupfalse%
\ hinM\ htype\ gNH\ transM\isanewline
\ \ \ \ \ \ \isacommand{by}\isamarkupfalse%
\ auto\isanewline
\ \ \isacommand{qed}\isamarkupfalse%
\isanewline
\isanewline
\ \ \isacommand{have}\isamarkupfalse%
\ {\isachardoublequoteopen}range{\isacharparenleft}{\kern0pt}Fn{\isacharunderscore}{\kern0pt}perm{\isacharparenleft}{\kern0pt}f{\isacharcomma}{\kern0pt}\ p{\isacharparenright}{\kern0pt}{\isacharparenright}{\kern0pt}\ {\isasymsubseteq}\ {\isadigit{2}}{\isachardoublequoteclose}\ {\isacharparenleft}{\kern0pt}\isakeyword{is}\ {\isacharquery}{\kern0pt}C{\isacharparenright}{\kern0pt}\isanewline
\ \ \isacommand{proof}\isamarkupfalse%
\ {\isacharparenleft}{\kern0pt}rule\ subsetI{\isacharparenright}{\kern0pt}\isanewline
\ \ \ \ \isacommand{fix}\isamarkupfalse%
\ v\ \isacommand{assume}\isamarkupfalse%
\ {\isachardoublequoteopen}v\ {\isasymin}\ range{\isacharparenleft}{\kern0pt}Fn{\isacharunderscore}{\kern0pt}perm{\isacharparenleft}{\kern0pt}f{\isacharcomma}{\kern0pt}\ p{\isacharparenright}{\kern0pt}{\isacharparenright}{\kern0pt}{\isachardoublequoteclose}\ \isanewline
\ \ \ \ \isacommand{then}\isamarkupfalse%
\ \isacommand{obtain}\isamarkupfalse%
\ u\ \isakeyword{where}\ {\isachardoublequoteopen}{\isacharless}{\kern0pt}u{\isacharcomma}{\kern0pt}\ v{\isachargreater}{\kern0pt}\ {\isasymin}\ Fn{\isacharunderscore}{\kern0pt}perm{\isacharparenleft}{\kern0pt}f{\isacharcomma}{\kern0pt}\ p{\isacharparenright}{\kern0pt}{\isachardoublequoteclose}\ \isacommand{by}\isamarkupfalse%
\ auto\ \isacommand{thm}\isamarkupfalse%
\ Fn{\isacharunderscore}{\kern0pt}permE\isanewline
\ \ \ \ \isacommand{then}\isamarkupfalse%
\ \isacommand{have}\isamarkupfalse%
\ {\isachardoublequoteopen}{\isasymexists}n{\isasymin}nat{\isachardot}{\kern0pt}\ {\isasymexists}m{\isasymin}nat{\isachardot}{\kern0pt}\ {\isasymexists}l{\isasymin}{\isadigit{2}}{\isachardot}{\kern0pt}\ {\isasymlangle}{\isasymlangle}n{\isacharcomma}{\kern0pt}\ m{\isasymrangle}{\isacharcomma}{\kern0pt}\ l{\isasymrangle}\ {\isasymin}\ p\ {\isasymand}\ {\isacharless}{\kern0pt}u{\isacharcomma}{\kern0pt}\ v{\isachargreater}{\kern0pt}\ {\isacharequal}{\kern0pt}\ {\isasymlangle}{\isasymlangle}f\ {\isacharbackquote}{\kern0pt}\ n{\isacharcomma}{\kern0pt}\ m{\isasymrangle}{\isacharcomma}{\kern0pt}\ l{\isasymrangle}{\isachardoublequoteclose}\ \isanewline
\ \ \ \ \ \ \isacommand{apply}\isamarkupfalse%
{\isacharparenleft}{\kern0pt}rule{\isacharunderscore}{\kern0pt}tac\ Fn{\isacharunderscore}{\kern0pt}permE{\isacharparenright}{\kern0pt}\isanewline
\ \ \ \ \ \ \isacommand{using}\isamarkupfalse%
\ assms\isanewline
\ \ \ \ \ \ \isacommand{by}\isamarkupfalse%
\ auto\isanewline
\ \ \ \ \isacommand{then}\isamarkupfalse%
\ \isacommand{show}\isamarkupfalse%
\ {\isachardoublequoteopen}v\ {\isasymin}\ {\isadigit{2}}{\isachardoublequoteclose}\ \isacommand{by}\isamarkupfalse%
\ auto\isanewline
\ \ \isacommand{qed}\isamarkupfalse%
\isanewline
\isanewline
\ \ \isacommand{then}\isamarkupfalse%
\ \isacommand{show}\isamarkupfalse%
\ {\isacharquery}{\kern0pt}thesis\ \isanewline
\ \ \ \ \isacommand{unfolding}\isamarkupfalse%
\ Fn{\isacharunderscore}{\kern0pt}def\ \isanewline
\ \ \ \ \isacommand{using}\isamarkupfalse%
\ Fn{\isacharunderscore}{\kern0pt}perm{\isacharunderscore}{\kern0pt}in{\isacharunderscore}{\kern0pt}M\ Fn{\isacharunderscore}{\kern0pt}perm{\isacharunderscore}{\kern0pt}subset\ function{\isacharunderscore}{\kern0pt}Fn{\isacharunderscore}{\kern0pt}perm\ {\isacartoucheopen}{\isacharquery}{\kern0pt}A{\isacartoucheclose}\ {\isacartoucheopen}{\isacharquery}{\kern0pt}B{\isacartoucheclose}\ {\isacartoucheopen}{\isacharquery}{\kern0pt}C{\isacartoucheclose}\ assms\isanewline
\ \ \ \ \isacommand{by}\isamarkupfalse%
\ auto\isanewline
\isacommand{qed}\isamarkupfalse%
%
\endisatagproof
{\isafoldproof}%
%
\isadelimproof
\isanewline
%
\endisadelimproof
\isanewline
\isacommand{lemma}\isamarkupfalse%
\ Fn{\isacharunderscore}{\kern0pt}perm{\isacharunderscore}{\kern0pt}comp\ {\isacharcolon}{\kern0pt}\ \isanewline
\ \ \isakeyword{fixes}\ f\ f{\isacharprime}{\kern0pt}\ p\ \isanewline
\ \ \isakeyword{assumes}\ {\isachardoublequoteopen}f\ {\isasymin}\ nat{\isacharunderscore}{\kern0pt}perms{\isachardoublequoteclose}\ {\isachardoublequoteopen}f{\isacharprime}{\kern0pt}\ {\isasymin}\ nat{\isacharunderscore}{\kern0pt}perms{\isachardoublequoteclose}\ {\isachardoublequoteopen}p\ {\isasymin}\ Fn{\isachardoublequoteclose}\ \isanewline
\ \ \isakeyword{shows}\ {\isachardoublequoteopen}Fn{\isacharunderscore}{\kern0pt}perm{\isacharparenleft}{\kern0pt}f{\isacharprime}{\kern0pt}{\isacharcomma}{\kern0pt}\ Fn{\isacharunderscore}{\kern0pt}perm{\isacharparenleft}{\kern0pt}f{\isacharcomma}{\kern0pt}\ p{\isacharparenright}{\kern0pt}{\isacharparenright}{\kern0pt}\ {\isacharequal}{\kern0pt}\ Fn{\isacharunderscore}{\kern0pt}perm{\isacharparenleft}{\kern0pt}f{\isacharprime}{\kern0pt}\ O\ f{\isacharcomma}{\kern0pt}\ p{\isacharparenright}{\kern0pt}{\isachardoublequoteclose}\ \isanewline
%
\isadelimproof
%
\endisadelimproof
%
\isatagproof
\isacommand{proof}\isamarkupfalse%
\ {\isacharparenleft}{\kern0pt}rule\ equality{\isacharunderscore}{\kern0pt}iffI{\isacharparenright}{\kern0pt}\isanewline
\ \ \isacommand{fix}\isamarkupfalse%
\ v\isanewline
\ \ \isacommand{have}\isamarkupfalse%
\ I{\isadigit{1}}{\isacharcolon}{\kern0pt}\ {\isachardoublequoteopen}v\ {\isasymin}\ Fn{\isacharunderscore}{\kern0pt}perm{\isacharparenleft}{\kern0pt}f{\isacharprime}{\kern0pt}{\isacharcomma}{\kern0pt}\ Fn{\isacharunderscore}{\kern0pt}perm{\isacharparenleft}{\kern0pt}f{\isacharcomma}{\kern0pt}\ p{\isacharparenright}{\kern0pt}{\isacharparenright}{\kern0pt}\ {\isasymlongleftrightarrow}\ {\isacharparenleft}{\kern0pt}{\isasymexists}n\ {\isasymin}\ nat{\isachardot}{\kern0pt}\ {\isasymexists}m\ {\isasymin}\ nat{\isachardot}{\kern0pt}\ {\isasymexists}l\ {\isasymin}\ {\isadigit{2}}{\isachardot}{\kern0pt}\ {\isacharless}{\kern0pt}{\isacharless}{\kern0pt}n{\isacharcomma}{\kern0pt}\ m{\isachargreater}{\kern0pt}{\isacharcomma}{\kern0pt}\ l{\isachargreater}{\kern0pt}\ {\isasymin}\ Fn{\isacharunderscore}{\kern0pt}perm{\isacharparenleft}{\kern0pt}f{\isacharcomma}{\kern0pt}\ p{\isacharparenright}{\kern0pt}\ {\isasymand}\ v\ {\isacharequal}{\kern0pt}\ {\isacharless}{\kern0pt}{\isacharless}{\kern0pt}f{\isacharprime}{\kern0pt}{\isacharbackquote}{\kern0pt}n{\isacharcomma}{\kern0pt}\ m{\isachargreater}{\kern0pt}{\isacharcomma}{\kern0pt}\ l{\isachargreater}{\kern0pt}{\isacharparenright}{\kern0pt}{\isachardoublequoteclose}\ \isanewline
\ \ \ \ \isacommand{apply}\isamarkupfalse%
{\isacharparenleft}{\kern0pt}rule\ iffI{\isacharparenright}{\kern0pt}\ \isanewline
\ \ \ \ \ \isacommand{apply}\isamarkupfalse%
{\isacharparenleft}{\kern0pt}rule\ Fn{\isacharunderscore}{\kern0pt}permE{\isacharcomma}{\kern0pt}\ rule\ Fn{\isacharunderscore}{\kern0pt}perm{\isacharunderscore}{\kern0pt}in{\isacharunderscore}{\kern0pt}Fn{\isacharparenright}{\kern0pt}\isanewline
\ \ \ \ \isacommand{using}\isamarkupfalse%
\ assms\isanewline
\ \ \ \ \ \ \ \ \isacommand{apply}\isamarkupfalse%
\ auto{\isacharbrackleft}{\kern0pt}{\isadigit{4}}{\isacharbrackright}{\kern0pt}\isanewline
\ \ \ \ \isacommand{apply}\isamarkupfalse%
{\isacharparenleft}{\kern0pt}subst\ Fn{\isacharunderscore}{\kern0pt}perm{\isacharunderscore}{\kern0pt}def{\isacharcomma}{\kern0pt}\ force{\isacharparenright}{\kern0pt}\isanewline
\ \ \ \ \isacommand{done}\isamarkupfalse%
\isanewline
\ \ \isacommand{have}\isamarkupfalse%
\ I{\isadigit{2}}{\isacharcolon}{\kern0pt}\ {\isachardoublequoteopen}{\isachardot}{\kern0pt}{\isachardot}{\kern0pt}{\isachardot}{\kern0pt}\ {\isasymlongleftrightarrow}\ {\isacharparenleft}{\kern0pt}{\isasymexists}n\ {\isasymin}\ nat{\isachardot}{\kern0pt}\ {\isasymexists}m\ {\isasymin}\ nat{\isachardot}{\kern0pt}\ {\isasymexists}l\ {\isasymin}\ {\isadigit{2}}{\isachardot}{\kern0pt}\ {\isacharparenleft}{\kern0pt}{\isasymexists}n{\isacharprime}{\kern0pt}\ {\isasymin}\ nat{\isachardot}{\kern0pt}\ {\isasymexists}m{\isacharprime}{\kern0pt}\ {\isasymin}\ nat{\isachardot}{\kern0pt}\ {\isasymexists}l{\isacharprime}{\kern0pt}\ {\isasymin}\ {\isadigit{2}}{\isachardot}{\kern0pt}\ {\isacharless}{\kern0pt}{\isacharless}{\kern0pt}n{\isacharprime}{\kern0pt}{\isacharcomma}{\kern0pt}\ m{\isacharprime}{\kern0pt}{\isachargreater}{\kern0pt}{\isacharcomma}{\kern0pt}\ l{\isacharprime}{\kern0pt}{\isachargreater}{\kern0pt}\ {\isasymin}\ p\ {\isasymand}\ {\isacharless}{\kern0pt}{\isacharless}{\kern0pt}n{\isacharcomma}{\kern0pt}\ m{\isachargreater}{\kern0pt}{\isacharcomma}{\kern0pt}\ l{\isachargreater}{\kern0pt}\ {\isacharequal}{\kern0pt}\ {\isacharless}{\kern0pt}{\isacharless}{\kern0pt}f{\isacharbackquote}{\kern0pt}n{\isacharprime}{\kern0pt}{\isacharcomma}{\kern0pt}\ m{\isacharprime}{\kern0pt}{\isachargreater}{\kern0pt}{\isacharcomma}{\kern0pt}\ l{\isacharprime}{\kern0pt}{\isachargreater}{\kern0pt}{\isacharparenright}{\kern0pt}\ {\isasymand}\ v\ {\isacharequal}{\kern0pt}\ {\isacharless}{\kern0pt}{\isacharless}{\kern0pt}f{\isacharprime}{\kern0pt}{\isacharbackquote}{\kern0pt}n{\isacharcomma}{\kern0pt}\ m{\isachargreater}{\kern0pt}{\isacharcomma}{\kern0pt}\ l{\isachargreater}{\kern0pt}{\isacharparenright}{\kern0pt}{\isachardoublequoteclose}\ \isanewline
\ \ \ \ \isacommand{apply}\isamarkupfalse%
{\isacharparenleft}{\kern0pt}rule\ bex{\isacharunderscore}{\kern0pt}iff{\isacharparenright}{\kern0pt}{\isacharplus}{\kern0pt}\isanewline
\ \ \ \ \isacommand{apply}\isamarkupfalse%
{\isacharparenleft}{\kern0pt}rule\ iff{\isacharunderscore}{\kern0pt}conjI{\isadigit{2}}{\isacharcomma}{\kern0pt}\ rule\ iffI{\isacharcomma}{\kern0pt}\ rule\ Fn{\isacharunderscore}{\kern0pt}permE{\isacharparenright}{\kern0pt}\isanewline
\ \ \ \ \isacommand{using}\isamarkupfalse%
\ assms\ \isanewline
\ \ \ \ \ \ \ \ \isacommand{apply}\isamarkupfalse%
\ auto{\isacharbrackleft}{\kern0pt}{\isadigit{3}}{\isacharbrackright}{\kern0pt}\isanewline
\ \ \ \ \ \isacommand{apply}\isamarkupfalse%
{\isacharparenleft}{\kern0pt}subst\ Fn{\isacharunderscore}{\kern0pt}perm{\isacharunderscore}{\kern0pt}def{\isacharcomma}{\kern0pt}\ force{\isacharparenright}{\kern0pt}\isanewline
\ \ \ \ \isacommand{by}\isamarkupfalse%
\ auto\isanewline
\ \ \isacommand{have}\isamarkupfalse%
\ I{\isadigit{3}}{\isacharcolon}{\kern0pt}\ {\isachardoublequoteopen}{\isachardot}{\kern0pt}{\isachardot}{\kern0pt}{\isachardot}{\kern0pt}\ {\isasymlongleftrightarrow}\ {\isacharparenleft}{\kern0pt}{\isasymexists}m\ {\isasymin}\ nat{\isachardot}{\kern0pt}\ {\isasymexists}l\ {\isasymin}\ {\isadigit{2}}{\isachardot}{\kern0pt}\ {\isasymexists}n{\isacharprime}{\kern0pt}\ {\isasymin}\ nat{\isachardot}{\kern0pt}\ {\isasymexists}n\ {\isasymin}\ nat{\isachardot}{\kern0pt}\ f{\isacharbackquote}{\kern0pt}n{\isacharprime}{\kern0pt}\ {\isacharequal}{\kern0pt}\ n\ {\isasymand}\ {\isacharless}{\kern0pt}{\isacharless}{\kern0pt}n{\isacharprime}{\kern0pt}{\isacharcomma}{\kern0pt}\ m{\isachargreater}{\kern0pt}{\isacharcomma}{\kern0pt}\ l{\isachargreater}{\kern0pt}\ {\isasymin}\ p\ {\isasymand}\ v\ {\isacharequal}{\kern0pt}\ {\isacharless}{\kern0pt}{\isacharless}{\kern0pt}f{\isacharprime}{\kern0pt}{\isacharbackquote}{\kern0pt}n{\isacharcomma}{\kern0pt}\ m{\isachargreater}{\kern0pt}{\isacharcomma}{\kern0pt}\ l{\isachargreater}{\kern0pt}{\isacharparenright}{\kern0pt}{\isachardoublequoteclose}\ \isacommand{by}\isamarkupfalse%
\ auto\isanewline
\ \ \isacommand{have}\isamarkupfalse%
\ I{\isadigit{4}}{\isacharcolon}{\kern0pt}\ {\isachardoublequoteopen}{\isachardot}{\kern0pt}{\isachardot}{\kern0pt}{\isachardot}{\kern0pt}\ {\isasymlongleftrightarrow}\ {\isacharparenleft}{\kern0pt}{\isasymexists}m\ {\isasymin}\ nat{\isachardot}{\kern0pt}\ {\isasymexists}l\ {\isasymin}\ {\isadigit{2}}{\isachardot}{\kern0pt}\ {\isasymexists}n\ {\isasymin}\ nat{\isachardot}{\kern0pt}\ {\isacharless}{\kern0pt}{\isacharless}{\kern0pt}n{\isacharcomma}{\kern0pt}\ m{\isachargreater}{\kern0pt}{\isacharcomma}{\kern0pt}\ l{\isachargreater}{\kern0pt}\ {\isasymin}\ p\ {\isasymand}\ v\ {\isacharequal}{\kern0pt}\ {\isacharless}{\kern0pt}{\isacharless}{\kern0pt}f{\isacharprime}{\kern0pt}{\isacharbackquote}{\kern0pt}{\isacharparenleft}{\kern0pt}f{\isacharbackquote}{\kern0pt}n{\isacharparenright}{\kern0pt}{\isacharcomma}{\kern0pt}\ m{\isachargreater}{\kern0pt}{\isacharcomma}{\kern0pt}\ l{\isachargreater}{\kern0pt}{\isacharparenright}{\kern0pt}{\isachardoublequoteclose}\ \isanewline
\ \ \ \ \isacommand{apply}\isamarkupfalse%
{\isacharparenleft}{\kern0pt}rule\ bex{\isacharunderscore}{\kern0pt}iff{\isacharparenright}{\kern0pt}{\isacharplus}{\kern0pt}\isanewline
\ \ \ \ \isacommand{apply}\isamarkupfalse%
{\isacharparenleft}{\kern0pt}rule\ iffI{\isacharcomma}{\kern0pt}\ force{\isacharcomma}{\kern0pt}\ clarsimp{\isacharparenright}{\kern0pt}\isanewline
\ \ \ \ \isacommand{apply}\isamarkupfalse%
{\isacharparenleft}{\kern0pt}rule\ function{\isacharunderscore}{\kern0pt}value{\isacharunderscore}{\kern0pt}in{\isacharparenright}{\kern0pt}\isanewline
\ \ \ \ \isacommand{using}\isamarkupfalse%
\ assms\ nat{\isacharunderscore}{\kern0pt}perms{\isacharunderscore}{\kern0pt}def\ bij{\isacharunderscore}{\kern0pt}is{\isacharunderscore}{\kern0pt}inj\ inj{\isacharunderscore}{\kern0pt}def\ \isanewline
\ \ \ \ \isacommand{by}\isamarkupfalse%
\ auto\isanewline
\ \ \isacommand{have}\isamarkupfalse%
\ I{\isadigit{5}}{\isacharcolon}{\kern0pt}\ {\isachardoublequoteopen}{\isachardot}{\kern0pt}{\isachardot}{\kern0pt}{\isachardot}{\kern0pt}\ {\isasymlongleftrightarrow}\ {\isacharparenleft}{\kern0pt}{\isasymexists}n\ {\isasymin}\ nat{\isachardot}{\kern0pt}\ {\isasymexists}m\ {\isasymin}\ nat{\isachardot}{\kern0pt}\ {\isasymexists}l\ {\isasymin}\ {\isadigit{2}}{\isachardot}{\kern0pt}\ {\isacharless}{\kern0pt}{\isacharless}{\kern0pt}n{\isacharcomma}{\kern0pt}\ m{\isachargreater}{\kern0pt}{\isacharcomma}{\kern0pt}\ l{\isachargreater}{\kern0pt}\ {\isasymin}\ p\ {\isasymand}\ v\ {\isacharequal}{\kern0pt}\ {\isacharless}{\kern0pt}{\isacharless}{\kern0pt}f{\isacharprime}{\kern0pt}{\isacharbackquote}{\kern0pt}{\isacharparenleft}{\kern0pt}f{\isacharbackquote}{\kern0pt}n{\isacharparenright}{\kern0pt}{\isacharcomma}{\kern0pt}\ m{\isachargreater}{\kern0pt}{\isacharcomma}{\kern0pt}\ l{\isachargreater}{\kern0pt}{\isacharparenright}{\kern0pt}{\isachardoublequoteclose}\ \isacommand{by}\isamarkupfalse%
\ auto\isanewline
\ \ \isacommand{have}\isamarkupfalse%
\ I{\isadigit{6}}{\isacharcolon}{\kern0pt}\ {\isachardoublequoteopen}{\isachardot}{\kern0pt}{\isachardot}{\kern0pt}{\isachardot}{\kern0pt}\ {\isasymlongleftrightarrow}\ {\isacharparenleft}{\kern0pt}{\isasymexists}n\ {\isasymin}\ nat{\isachardot}{\kern0pt}\ {\isasymexists}m\ {\isasymin}\ nat{\isachardot}{\kern0pt}\ {\isasymexists}l\ {\isasymin}\ {\isadigit{2}}{\isachardot}{\kern0pt}\ {\isacharless}{\kern0pt}{\isacharless}{\kern0pt}n{\isacharcomma}{\kern0pt}\ m{\isachargreater}{\kern0pt}{\isacharcomma}{\kern0pt}\ l{\isachargreater}{\kern0pt}\ {\isasymin}\ p\ {\isasymand}\ v\ {\isacharequal}{\kern0pt}\ {\isacharless}{\kern0pt}{\isacharless}{\kern0pt}{\isacharparenleft}{\kern0pt}f{\isacharprime}{\kern0pt}\ O\ f{\isacharparenright}{\kern0pt}{\isacharbackquote}{\kern0pt}n{\isacharcomma}{\kern0pt}\ m{\isachargreater}{\kern0pt}{\isacharcomma}{\kern0pt}\ l{\isachargreater}{\kern0pt}{\isacharparenright}{\kern0pt}{\isachardoublequoteclose}\ \ \isanewline
\ \ \ \ \isacommand{apply}\isamarkupfalse%
{\isacharparenleft}{\kern0pt}rule\ bex{\isacharunderscore}{\kern0pt}iff{\isacharparenright}{\kern0pt}{\isacharplus}{\kern0pt}\isanewline
\ \ \ \ \isacommand{apply}\isamarkupfalse%
{\isacharparenleft}{\kern0pt}subst\ comp{\isacharunderscore}{\kern0pt}fun{\isacharunderscore}{\kern0pt}apply{\isacharparenright}{\kern0pt}\isanewline
\ \ \ \ \isacommand{using}\isamarkupfalse%
\ assms\ nat{\isacharunderscore}{\kern0pt}perms{\isacharunderscore}{\kern0pt}def\ bij{\isacharunderscore}{\kern0pt}is{\isacharunderscore}{\kern0pt}inj\ inj{\isacharunderscore}{\kern0pt}def\ \isanewline
\ \ \ \ \ \ \isacommand{apply}\isamarkupfalse%
\ force\isanewline
\ \ \ \ \isacommand{by}\isamarkupfalse%
\ auto\isanewline
\ \ \isacommand{have}\isamarkupfalse%
\ I{\isadigit{7}}{\isacharcolon}{\kern0pt}\ {\isachardoublequoteopen}{\isachardot}{\kern0pt}{\isachardot}{\kern0pt}{\isachardot}{\kern0pt}\ {\isasymlongleftrightarrow}\ v\ {\isasymin}\ Fn{\isacharunderscore}{\kern0pt}perm{\isacharparenleft}{\kern0pt}f{\isacharprime}{\kern0pt}\ O\ f{\isacharcomma}{\kern0pt}\ p{\isacharparenright}{\kern0pt}{\isachardoublequoteclose}\ \isanewline
\ \ \ \ \isacommand{apply}\isamarkupfalse%
{\isacharparenleft}{\kern0pt}rule\ iffI{\isacharcomma}{\kern0pt}\ subst\ Fn{\isacharunderscore}{\kern0pt}perm{\isacharunderscore}{\kern0pt}def{\isacharcomma}{\kern0pt}\ force{\isacharparenright}{\kern0pt}\isanewline
\ \ \ \ \isacommand{apply}\isamarkupfalse%
{\isacharparenleft}{\kern0pt}rule\ Fn{\isacharunderscore}{\kern0pt}permE{\isacharparenright}{\kern0pt}\isanewline
\ \ \ \ \isacommand{using}\isamarkupfalse%
\ assms\isanewline
\ \ \ \ \ \ \isacommand{apply}\isamarkupfalse%
\ simp\isanewline
\ \ \ \ \isacommand{unfolding}\isamarkupfalse%
\ nat{\isacharunderscore}{\kern0pt}perms{\isacharunderscore}{\kern0pt}def\isanewline
\ \ \ \ \ \isacommand{apply}\isamarkupfalse%
\ {\isacharparenleft}{\kern0pt}simp{\isacharcomma}{\kern0pt}\ rule\ conjI{\isacharcomma}{\kern0pt}\ rule\ comp{\isacharunderscore}{\kern0pt}bij{\isacharparenright}{\kern0pt}\isanewline
\ \ \ \ \isacommand{using}\isamarkupfalse%
\ assms\ nat{\isacharunderscore}{\kern0pt}perms{\isacharunderscore}{\kern0pt}def\ bij{\isacharunderscore}{\kern0pt}is{\isacharunderscore}{\kern0pt}inj\ inj{\isacharunderscore}{\kern0pt}def\ comp{\isacharunderscore}{\kern0pt}closed\isanewline
\ \ \ \ \isacommand{by}\isamarkupfalse%
\ auto\isanewline
\ \ \isacommand{show}\isamarkupfalse%
\ {\isachardoublequoteopen}v\ {\isasymin}\ Fn{\isacharunderscore}{\kern0pt}perm{\isacharparenleft}{\kern0pt}f{\isacharprime}{\kern0pt}{\isacharcomma}{\kern0pt}\ Fn{\isacharunderscore}{\kern0pt}perm{\isacharparenleft}{\kern0pt}f{\isacharcomma}{\kern0pt}\ p{\isacharparenright}{\kern0pt}{\isacharparenright}{\kern0pt}\ {\isasymlongleftrightarrow}\ v\ {\isasymin}\ Fn{\isacharunderscore}{\kern0pt}perm{\isacharparenleft}{\kern0pt}f{\isacharprime}{\kern0pt}\ O\ f{\isacharcomma}{\kern0pt}\ p{\isacharparenright}{\kern0pt}{\isachardoublequoteclose}\ \isacommand{using}\isamarkupfalse%
\ I{\isadigit{1}}\ I{\isadigit{2}}\ I{\isadigit{3}}\ I{\isadigit{4}}\ I{\isadigit{5}}\ I{\isadigit{6}}\ I{\isadigit{7}}\ \isacommand{by}\isamarkupfalse%
\ auto\isanewline
\isacommand{qed}\isamarkupfalse%
%
\endisatagproof
{\isafoldproof}%
%
\isadelimproof
\isanewline
%
\endisadelimproof
\isanewline
\isacommand{lemma}\isamarkupfalse%
\ Fn{\isacharunderscore}{\kern0pt}perm{\isacharprime}{\kern0pt}{\isacharunderscore}{\kern0pt}eq\ {\isacharcolon}{\kern0pt}\ \isanewline
\ \ \isakeyword{fixes}\ f\ p\ \isanewline
\ \ \isakeyword{assumes}\ {\isachardoublequoteopen}f\ {\isasymin}\ nat{\isacharunderscore}{\kern0pt}perms{\isachardoublequoteclose}\ {\isachardoublequoteopen}p\ {\isasymin}\ Fn{\isachardoublequoteclose}\ \ \isanewline
\ \ \isakeyword{shows}\ {\isachardoublequoteopen}Fn{\isacharunderscore}{\kern0pt}perm{\isacharprime}{\kern0pt}{\isacharparenleft}{\kern0pt}f{\isacharparenright}{\kern0pt}{\isacharbackquote}{\kern0pt}p\ {\isacharequal}{\kern0pt}\ Fn{\isacharunderscore}{\kern0pt}perm{\isacharparenleft}{\kern0pt}f{\isacharcomma}{\kern0pt}\ p{\isacharparenright}{\kern0pt}{\isachardoublequoteclose}\ \isanewline
%
\isadelimproof
\ \ %
\endisadelimproof
%
\isatagproof
\isacommand{apply}\isamarkupfalse%
{\isacharparenleft}{\kern0pt}rule\ function{\isacharunderscore}{\kern0pt}apply{\isacharunderscore}{\kern0pt}equality{\isacharparenright}{\kern0pt}\isanewline
\ \ \isacommand{unfolding}\isamarkupfalse%
\ Fn{\isacharunderscore}{\kern0pt}perm{\isacharprime}{\kern0pt}{\isacharunderscore}{\kern0pt}def\ \isanewline
\ \ \isacommand{using}\isamarkupfalse%
\ assms\ function{\isacharunderscore}{\kern0pt}def\isanewline
\ \ \isacommand{by}\isamarkupfalse%
\ auto%
\endisatagproof
{\isafoldproof}%
%
\isadelimproof
\isanewline
%
\endisadelimproof
\isanewline
\isacommand{lemma}\isamarkupfalse%
\ Fn{\isacharunderscore}{\kern0pt}perm{\isacharprime}{\kern0pt}{\isacharunderscore}{\kern0pt}value{\isacharunderscore}{\kern0pt}in\ {\isacharcolon}{\kern0pt}\ \isanewline
\ \ \isakeyword{fixes}\ f\ p\ \isanewline
\ \ \isakeyword{assumes}\ {\isachardoublequoteopen}f\ {\isasymin}\ nat{\isacharunderscore}{\kern0pt}perms{\isachardoublequoteclose}\ {\isachardoublequoteopen}p\ {\isasymin}\ Fn{\isachardoublequoteclose}\ \isanewline
\ \ \isakeyword{shows}\ {\isachardoublequoteopen}Fn{\isacharunderscore}{\kern0pt}perm{\isacharprime}{\kern0pt}{\isacharparenleft}{\kern0pt}f{\isacharparenright}{\kern0pt}{\isacharbackquote}{\kern0pt}p\ {\isasymin}\ Fn{\isachardoublequoteclose}\ \isanewline
%
\isadelimproof
\isanewline
\ \ %
\endisadelimproof
%
\isatagproof
\isacommand{using}\isamarkupfalse%
\ Fn{\isacharunderscore}{\kern0pt}perm{\isacharunderscore}{\kern0pt}in{\isacharunderscore}{\kern0pt}Fn\ Fn{\isacharunderscore}{\kern0pt}perm{\isacharprime}{\kern0pt}{\isacharunderscore}{\kern0pt}eq\ assms\isanewline
\ \ \isacommand{by}\isamarkupfalse%
\ auto%
\endisatagproof
{\isafoldproof}%
%
\isadelimproof
\isanewline
%
\endisadelimproof
\isanewline
\isacommand{lemma}\isamarkupfalse%
\ Fn{\isacharunderscore}{\kern0pt}perm{\isacharprime}{\kern0pt}{\isacharunderscore}{\kern0pt}type\ {\isacharcolon}{\kern0pt}\ \isanewline
\ \ \isakeyword{fixes}\ f\isanewline
\ \ \isakeyword{assumes}\ {\isachardoublequoteopen}f\ {\isasymin}\ nat{\isacharunderscore}{\kern0pt}perms{\isachardoublequoteclose}\ \isanewline
\ \ \isakeyword{shows}\ {\isachardoublequoteopen}Fn{\isacharunderscore}{\kern0pt}perm{\isacharprime}{\kern0pt}{\isacharparenleft}{\kern0pt}f{\isacharparenright}{\kern0pt}\ {\isasymin}\ Fn\ {\isasymrightarrow}\ Fn{\isachardoublequoteclose}\ \isanewline
%
\isadelimproof
\ \ %
\endisadelimproof
%
\isatagproof
\isacommand{apply}\isamarkupfalse%
{\isacharparenleft}{\kern0pt}rule\ Pi{\isacharunderscore}{\kern0pt}memberI{\isacharparenright}{\kern0pt}\isanewline
\ \ \isacommand{using}\isamarkupfalse%
\ relation{\isacharunderscore}{\kern0pt}def\ function{\isacharunderscore}{\kern0pt}def\ Fn{\isacharunderscore}{\kern0pt}perm{\isacharprime}{\kern0pt}{\isacharunderscore}{\kern0pt}def\ Fn{\isacharunderscore}{\kern0pt}perm{\isacharunderscore}{\kern0pt}in{\isacharunderscore}{\kern0pt}Fn\ assms\isanewline
\ \ \isacommand{by}\isamarkupfalse%
\ auto%
\endisatagproof
{\isafoldproof}%
%
\isadelimproof
\isanewline
%
\endisadelimproof
\isanewline
\isacommand{lemma}\isamarkupfalse%
\ domain{\isacharunderscore}{\kern0pt}Fn{\isacharunderscore}{\kern0pt}perm{\isacharprime}{\kern0pt}\ {\isacharcolon}{\kern0pt}\ \isanewline
\ \ \isakeyword{fixes}\ f\ \isanewline
\ \ \isakeyword{assumes}\ {\isachardoublequoteopen}f\ {\isasymin}\ nat{\isacharunderscore}{\kern0pt}perms{\isachardoublequoteclose}\ \isanewline
\ \ \isakeyword{shows}\ {\isachardoublequoteopen}domain{\isacharparenleft}{\kern0pt}Fn{\isacharunderscore}{\kern0pt}perm{\isacharprime}{\kern0pt}{\isacharparenleft}{\kern0pt}f{\isacharparenright}{\kern0pt}{\isacharparenright}{\kern0pt}\ {\isacharequal}{\kern0pt}\ Fn{\isachardoublequoteclose}\ \isanewline
%
\isadelimproof
\ \ %
\endisadelimproof
%
\isatagproof
\isacommand{unfolding}\isamarkupfalse%
\ Fn{\isacharunderscore}{\kern0pt}perm{\isacharprime}{\kern0pt}{\isacharunderscore}{\kern0pt}def\isanewline
\ \ \isacommand{by}\isamarkupfalse%
\ auto%
\endisatagproof
{\isafoldproof}%
%
\isadelimproof
\isanewline
%
\endisadelimproof
\isanewline
\isacommand{lemma}\isamarkupfalse%
\ Fn{\isacharunderscore}{\kern0pt}perm{\isacharunderscore}{\kern0pt}id\ {\isacharcolon}{\kern0pt}\ \isanewline
\ \ \isakeyword{fixes}\ p\ \isanewline
\ \ \isakeyword{assumes}\ {\isachardoublequoteopen}p\ {\isasymin}\ Fn{\isachardoublequoteclose}\ \isanewline
\ \ \isakeyword{shows}\ {\isachardoublequoteopen}Fn{\isacharunderscore}{\kern0pt}perm{\isacharparenleft}{\kern0pt}id{\isacharparenleft}{\kern0pt}nat{\isacharparenright}{\kern0pt}{\isacharcomma}{\kern0pt}\ p{\isacharparenright}{\kern0pt}\ {\isacharequal}{\kern0pt}\ p{\isachardoublequoteclose}\ \isanewline
%
\isadelimproof
\ \ %
\endisadelimproof
%
\isatagproof
\isacommand{unfolding}\isamarkupfalse%
\ Fn{\isacharunderscore}{\kern0pt}perm{\isacharunderscore}{\kern0pt}def\ \isanewline
\ \ \isacommand{apply}\isamarkupfalse%
{\isacharparenleft}{\kern0pt}subgoal{\isacharunderscore}{\kern0pt}tac\ {\isachardoublequoteopen}p\ {\isasymsubseteq}\ {\isacharparenleft}{\kern0pt}nat\ {\isasymtimes}\ nat{\isacharparenright}{\kern0pt}\ {\isasymtimes}\ {\isadigit{2}}{\isachardoublequoteclose}{\isacharcomma}{\kern0pt}\ force{\isacharparenright}{\kern0pt}\isanewline
\ \ \isacommand{using}\isamarkupfalse%
\ assms\ Fn{\isacharunderscore}{\kern0pt}def\isanewline
\ \ \isacommand{by}\isamarkupfalse%
\ auto%
\endisatagproof
{\isafoldproof}%
%
\isadelimproof
\isanewline
%
\endisadelimproof
\isanewline
\isacommand{lemma}\isamarkupfalse%
\ Fn{\isacharunderscore}{\kern0pt}perm{\isacharprime}{\kern0pt}{\isacharunderscore}{\kern0pt}id\ {\isacharcolon}{\kern0pt}\ \isanewline
\ \ \isakeyword{shows}\ {\isachardoublequoteopen}Fn{\isacharunderscore}{\kern0pt}perm{\isacharprime}{\kern0pt}{\isacharparenleft}{\kern0pt}id{\isacharparenleft}{\kern0pt}nat{\isacharparenright}{\kern0pt}{\isacharparenright}{\kern0pt}\ {\isacharequal}{\kern0pt}\ id{\isacharparenleft}{\kern0pt}Fn{\isacharparenright}{\kern0pt}{\isachardoublequoteclose}\ \isanewline
%
\isadelimproof
\ \ %
\endisadelimproof
%
\isatagproof
\isacommand{apply}\isamarkupfalse%
{\isacharparenleft}{\kern0pt}subgoal{\isacharunderscore}{\kern0pt}tac\ {\isachardoublequoteopen}id{\isacharparenleft}{\kern0pt}nat{\isacharparenright}{\kern0pt}\ {\isasymin}\ nat{\isacharunderscore}{\kern0pt}perms{\isachardoublequoteclose}{\isacharparenright}{\kern0pt}\isanewline
\ \ \ \isacommand{apply}\isamarkupfalse%
{\isacharparenleft}{\kern0pt}rule\ function{\isacharunderscore}{\kern0pt}eq{\isacharparenright}{\kern0pt}\isanewline
\ \ \isacommand{using}\isamarkupfalse%
\ relation{\isacharunderscore}{\kern0pt}def\ Fn{\isacharunderscore}{\kern0pt}perm{\isacharprime}{\kern0pt}{\isacharunderscore}{\kern0pt}def\ function{\isacharunderscore}{\kern0pt}def\isanewline
\ \ \ \ \ \ \ \ \isacommand{apply}\isamarkupfalse%
\ auto{\isacharbrackleft}{\kern0pt}{\isadigit{5}}{\isacharbrackright}{\kern0pt}\isanewline
\ \ \ \isacommand{apply}\isamarkupfalse%
{\isacharparenleft}{\kern0pt}rename{\isacharunderscore}{\kern0pt}tac\ x{\isacharcomma}{\kern0pt}\ rule{\isacharunderscore}{\kern0pt}tac\ P{\isacharequal}{\kern0pt}{\isachardoublequoteopen}x\ {\isasymin}\ domain{\isacharparenleft}{\kern0pt}Fn{\isacharunderscore}{\kern0pt}perm{\isacharprime}{\kern0pt}{\isacharparenleft}{\kern0pt}id{\isacharparenleft}{\kern0pt}nat{\isacharparenright}{\kern0pt}{\isacharparenright}{\kern0pt}{\isacharparenright}{\kern0pt}{\isachardoublequoteclose}\ \isakeyword{in}\ mp{\isacharparenright}{\kern0pt}\isanewline
\ \ \ \ \isacommand{apply}\isamarkupfalse%
{\isacharparenleft}{\kern0pt}subst\ domain{\isacharunderscore}{\kern0pt}Fn{\isacharunderscore}{\kern0pt}perm{\isacharprime}{\kern0pt}{\isacharcomma}{\kern0pt}\ simp{\isacharparenright}{\kern0pt}\isanewline
\ \ \ \ \isacommand{apply}\isamarkupfalse%
{\isacharparenleft}{\kern0pt}rule\ impI{\isacharcomma}{\kern0pt}\ subst\ Fn{\isacharunderscore}{\kern0pt}perm{\isacharprime}{\kern0pt}{\isacharunderscore}{\kern0pt}eq{\isacharcomma}{\kern0pt}\ simp{\isacharcomma}{\kern0pt}\ simp{\isacharparenright}{\kern0pt}\isanewline
\ \ \ \ \isacommand{apply}\isamarkupfalse%
{\isacharparenleft}{\kern0pt}subst\ Fn{\isacharunderscore}{\kern0pt}perm{\isacharunderscore}{\kern0pt}id{\isacharcomma}{\kern0pt}\ simp{\isacharcomma}{\kern0pt}\ rule\ sym{\isacharparenright}{\kern0pt}\isanewline
\ \ \ \ \isacommand{apply}\isamarkupfalse%
{\isacharparenleft}{\kern0pt}rule\ function{\isacharunderscore}{\kern0pt}apply{\isacharunderscore}{\kern0pt}equality{\isacharcomma}{\kern0pt}\ rule\ idI{\isacharcomma}{\kern0pt}\ simp{\isacharparenright}{\kern0pt}\isanewline
\ \ \ \ \isacommand{apply}\isamarkupfalse%
{\isacharparenleft}{\kern0pt}insert\ id{\isacharunderscore}{\kern0pt}type\ Pi{\isacharunderscore}{\kern0pt}def{\isacharcomma}{\kern0pt}\ force{\isacharcomma}{\kern0pt}\ simp{\isacharparenright}{\kern0pt}\isanewline
\ \ \isacommand{unfolding}\isamarkupfalse%
\ nat{\isacharunderscore}{\kern0pt}perms{\isacharunderscore}{\kern0pt}def\isanewline
\ \ \isacommand{apply}\isamarkupfalse%
{\isacharparenleft}{\kern0pt}simp{\isacharcomma}{\kern0pt}\ rule\ conjI{\isacharcomma}{\kern0pt}\ rule\ id{\isacharunderscore}{\kern0pt}bij{\isacharcomma}{\kern0pt}\ rule\ id{\isacharunderscore}{\kern0pt}closed{\isacharparenright}{\kern0pt}\isanewline
\ \ \isacommand{using}\isamarkupfalse%
\ nat{\isacharunderscore}{\kern0pt}in{\isacharunderscore}{\kern0pt}M\isanewline
\ \ \isacommand{by}\isamarkupfalse%
\ auto%
\endisatagproof
{\isafoldproof}%
%
\isadelimproof
\isanewline
%
\endisadelimproof
\isanewline
\isacommand{lemma}\isamarkupfalse%
\ Fn{\isacharunderscore}{\kern0pt}perm{\isacharprime}{\kern0pt}{\isacharunderscore}{\kern0pt}comp\ {\isacharcolon}{\kern0pt}\ \ \isanewline
\ \ \isakeyword{fixes}\ f\ f{\isacharprime}{\kern0pt}\isanewline
\ \ \isakeyword{assumes}\ {\isachardoublequoteopen}f\ {\isasymin}\ nat{\isacharunderscore}{\kern0pt}perms{\isachardoublequoteclose}\ {\isachardoublequoteopen}f{\isacharprime}{\kern0pt}\ {\isasymin}\ nat{\isacharunderscore}{\kern0pt}perms{\isachardoublequoteclose}\ \isanewline
\ \ \isakeyword{shows}\ {\isachardoublequoteopen}Fn{\isacharunderscore}{\kern0pt}perm{\isacharprime}{\kern0pt}{\isacharparenleft}{\kern0pt}f{\isacharparenright}{\kern0pt}\ O\ Fn{\isacharunderscore}{\kern0pt}perm{\isacharprime}{\kern0pt}{\isacharparenleft}{\kern0pt}f{\isacharprime}{\kern0pt}{\isacharparenright}{\kern0pt}\ {\isacharequal}{\kern0pt}\ Fn{\isacharunderscore}{\kern0pt}perm{\isacharprime}{\kern0pt}{\isacharparenleft}{\kern0pt}f\ O\ f{\isacharprime}{\kern0pt}{\isacharparenright}{\kern0pt}{\isachardoublequoteclose}\ \isanewline
%
\isadelimproof
\ \ %
\endisadelimproof
%
\isatagproof
\isacommand{apply}\isamarkupfalse%
{\isacharparenleft}{\kern0pt}subgoal{\isacharunderscore}{\kern0pt}tac\ {\isachardoublequoteopen}domain{\isacharparenleft}{\kern0pt}Fn{\isacharunderscore}{\kern0pt}perm{\isacharprime}{\kern0pt}{\isacharparenleft}{\kern0pt}f{\isacharparenright}{\kern0pt}\ O\ Fn{\isacharunderscore}{\kern0pt}perm{\isacharprime}{\kern0pt}{\isacharparenleft}{\kern0pt}f{\isacharprime}{\kern0pt}{\isacharparenright}{\kern0pt}{\isacharparenright}{\kern0pt}\ {\isacharequal}{\kern0pt}\ Fn{\isachardoublequoteclose}{\isacharparenright}{\kern0pt}\isanewline
\ \ \ \isacommand{apply}\isamarkupfalse%
{\isacharparenleft}{\kern0pt}rule\ function{\isacharunderscore}{\kern0pt}eq{\isacharparenright}{\kern0pt}\isanewline
\ \ \isacommand{using}\isamarkupfalse%
\ relation{\isacharunderscore}{\kern0pt}def\ Fn{\isacharunderscore}{\kern0pt}perm{\isacharprime}{\kern0pt}{\isacharunderscore}{\kern0pt}def\ comp{\isacharunderscore}{\kern0pt}def\ function{\isacharunderscore}{\kern0pt}def\isanewline
\ \ \ \ \ \ \ \ \isacommand{apply}\isamarkupfalse%
\ auto{\isacharbrackleft}{\kern0pt}{\isadigit{4}}{\isacharbrackright}{\kern0pt}\isanewline
\ \ \ \ \isacommand{apply}\isamarkupfalse%
{\isacharparenleft}{\kern0pt}subst\ domain{\isacharunderscore}{\kern0pt}Fn{\isacharunderscore}{\kern0pt}perm{\isacharprime}{\kern0pt}{\isacharcomma}{\kern0pt}\ subst\ nat{\isacharunderscore}{\kern0pt}perms{\isacharunderscore}{\kern0pt}def{\isacharcomma}{\kern0pt}\ simp{\isacharcomma}{\kern0pt}\ rule\ conjI{\isacharparenright}{\kern0pt}\isanewline
\ \ \ \ \ \ \isacommand{apply}\isamarkupfalse%
{\isacharparenleft}{\kern0pt}rule\ comp{\isacharunderscore}{\kern0pt}bij{\isacharparenright}{\kern0pt}\isanewline
\ \ \isacommand{using}\isamarkupfalse%
\ assms\ nat{\isacharunderscore}{\kern0pt}perms{\isacharunderscore}{\kern0pt}def\ comp{\isacharunderscore}{\kern0pt}closed\ \isanewline
\ \ \ \ \ \ \isacommand{apply}\isamarkupfalse%
\ auto{\isacharbrackleft}{\kern0pt}{\isadigit{4}}{\isacharbrackright}{\kern0pt}\isanewline
\ \ \ \isacommand{apply}\isamarkupfalse%
{\isacharparenleft}{\kern0pt}subst\ comp{\isacharunderscore}{\kern0pt}fun{\isacharunderscore}{\kern0pt}apply{\isacharcomma}{\kern0pt}\ rule\ Fn{\isacharunderscore}{\kern0pt}perm{\isacharprime}{\kern0pt}{\isacharunderscore}{\kern0pt}type{\isacharparenright}{\kern0pt}\isanewline
\ \ \isacommand{using}\isamarkupfalse%
\ assms\ \isanewline
\ \ \ \ \ \isacommand{apply}\isamarkupfalse%
\ auto{\isacharbrackleft}{\kern0pt}{\isadigit{2}}{\isacharbrackright}{\kern0pt}\isanewline
\ \ \ \isacommand{apply}\isamarkupfalse%
{\isacharparenleft}{\kern0pt}subst\ Fn{\isacharunderscore}{\kern0pt}perm{\isacharprime}{\kern0pt}{\isacharunderscore}{\kern0pt}eq{\isacharcomma}{\kern0pt}\ simp\ add{\isacharcolon}{\kern0pt}assms{\isacharcomma}{\kern0pt}\ rule\ function{\isacharunderscore}{\kern0pt}value{\isacharunderscore}{\kern0pt}in{\isacharcomma}{\kern0pt}\ rule\ Fn{\isacharunderscore}{\kern0pt}perm{\isacharprime}{\kern0pt}{\isacharunderscore}{\kern0pt}type{\isacharcomma}{\kern0pt}\ simp\ add{\isacharcolon}{\kern0pt}assms{\isacharcomma}{\kern0pt}\ simp{\isacharparenright}{\kern0pt}\isanewline
\ \ \ \isacommand{apply}\isamarkupfalse%
{\isacharparenleft}{\kern0pt}subst\ Fn{\isacharunderscore}{\kern0pt}perm{\isacharprime}{\kern0pt}{\isacharunderscore}{\kern0pt}eq{\isacharcomma}{\kern0pt}\ simp\ add{\isacharcolon}{\kern0pt}assms{\isacharcomma}{\kern0pt}\ simp{\isacharparenright}{\kern0pt}\isanewline
\ \ \ \isacommand{apply}\isamarkupfalse%
{\isacharparenleft}{\kern0pt}subst\ Fn{\isacharunderscore}{\kern0pt}perm{\isacharprime}{\kern0pt}{\isacharunderscore}{\kern0pt}eq{\isacharcomma}{\kern0pt}\ subst\ nat{\isacharunderscore}{\kern0pt}perms{\isacharunderscore}{\kern0pt}def{\isacharcomma}{\kern0pt}\ simp\ {\isacharcomma}{\kern0pt}rule\ conjI{\isacharparenright}{\kern0pt}\isanewline
\ \ \ \ \ \ \isacommand{apply}\isamarkupfalse%
{\isacharparenleft}{\kern0pt}rule\ comp{\isacharunderscore}{\kern0pt}bij{\isacharparenright}{\kern0pt}\isanewline
\ \ \isacommand{using}\isamarkupfalse%
\ assms\ nat{\isacharunderscore}{\kern0pt}perms{\isacharunderscore}{\kern0pt}def\ comp{\isacharunderscore}{\kern0pt}closed\ \isanewline
\ \ \ \ \ \ \ \isacommand{apply}\isamarkupfalse%
\ auto{\isacharbrackleft}{\kern0pt}{\isadigit{4}}{\isacharbrackright}{\kern0pt}\isanewline
\ \ \ \isacommand{apply}\isamarkupfalse%
{\isacharparenleft}{\kern0pt}rule\ Fn{\isacharunderscore}{\kern0pt}perm{\isacharunderscore}{\kern0pt}comp{\isacharparenright}{\kern0pt}\isanewline
\ \ \isacommand{using}\isamarkupfalse%
\ assms\isanewline
\ \ \ \ \ \isacommand{apply}\isamarkupfalse%
\ auto{\isacharbrackleft}{\kern0pt}{\isadigit{3}}{\isacharbrackright}{\kern0pt}\isanewline
\ \ \isacommand{apply}\isamarkupfalse%
{\isacharparenleft}{\kern0pt}subst\ domain{\isacharunderscore}{\kern0pt}comp{\isacharunderscore}{\kern0pt}eq{\isacharcomma}{\kern0pt}\ subst\ domain{\isacharunderscore}{\kern0pt}Fn{\isacharunderscore}{\kern0pt}perm{\isacharprime}{\kern0pt}{\isacharcomma}{\kern0pt}\ simp\ add{\isacharcolon}{\kern0pt}assms{\isacharparenright}{\kern0pt}\isanewline
\ \ \isacommand{using}\isamarkupfalse%
\ Fn{\isacharunderscore}{\kern0pt}perm{\isacharprime}{\kern0pt}{\isacharunderscore}{\kern0pt}type\ assms\ Pi{\isacharunderscore}{\kern0pt}def\ \isanewline
\ \ \ \isacommand{apply}\isamarkupfalse%
\ force\ \isanewline
\ \ \isacommand{using}\isamarkupfalse%
\ domain{\isacharunderscore}{\kern0pt}Fn{\isacharunderscore}{\kern0pt}perm{\isacharprime}{\kern0pt}\ assms\isanewline
\ \ \isacommand{by}\isamarkupfalse%
\ auto%
\endisatagproof
{\isafoldproof}%
%
\isadelimproof
\isanewline
%
\endisadelimproof
\isanewline
\isacommand{lemma}\isamarkupfalse%
\ Fn{\isacharunderscore}{\kern0pt}perm{\isacharprime}{\kern0pt}{\isacharunderscore}{\kern0pt}bij\ {\isacharcolon}{\kern0pt}\ \isanewline
\ \ \isakeyword{fixes}\ f\ \isanewline
\ \ \isakeyword{assumes}\ {\isachardoublequoteopen}f\ {\isasymin}\ nat{\isacharunderscore}{\kern0pt}perms{\isachardoublequoteclose}\isanewline
\ \ \isakeyword{shows}\ {\isachardoublequoteopen}Fn{\isacharunderscore}{\kern0pt}perm{\isacharprime}{\kern0pt}{\isacharparenleft}{\kern0pt}f{\isacharparenright}{\kern0pt}\ {\isasymin}\ bij{\isacharparenleft}{\kern0pt}Fn{\isacharcomma}{\kern0pt}\ Fn{\isacharparenright}{\kern0pt}{\isachardoublequoteclose}\ \isanewline
%
\isadelimproof
\ \ %
\endisadelimproof
%
\isatagproof
\isacommand{apply}\isamarkupfalse%
{\isacharparenleft}{\kern0pt}rule{\isacharunderscore}{\kern0pt}tac\ g{\isacharequal}{\kern0pt}{\isachardoublequoteopen}Fn{\isacharunderscore}{\kern0pt}perm{\isacharprime}{\kern0pt}{\isacharparenleft}{\kern0pt}converse{\isacharparenleft}{\kern0pt}f{\isacharparenright}{\kern0pt}{\isacharparenright}{\kern0pt}{\isachardoublequoteclose}\ \isakeyword{in}\ fg{\isacharunderscore}{\kern0pt}imp{\isacharunderscore}{\kern0pt}bijective{\isacharparenright}{\kern0pt}\isanewline
\ \ \ \ \ \isacommand{apply}\isamarkupfalse%
{\isacharparenleft}{\kern0pt}rule\ Fn{\isacharunderscore}{\kern0pt}perm{\isacharprime}{\kern0pt}{\isacharunderscore}{\kern0pt}type{\isacharcomma}{\kern0pt}\ simp\ add{\isacharcolon}{\kern0pt}assms{\isacharparenright}{\kern0pt}\isanewline
\ \ \ \ \isacommand{apply}\isamarkupfalse%
{\isacharparenleft}{\kern0pt}rule\ Fn{\isacharunderscore}{\kern0pt}perm{\isacharprime}{\kern0pt}{\isacharunderscore}{\kern0pt}type{\isacharcomma}{\kern0pt}\ rule\ converse{\isacharunderscore}{\kern0pt}in{\isacharunderscore}{\kern0pt}nat{\isacharunderscore}{\kern0pt}perms{\isacharcomma}{\kern0pt}\ simp\ add{\isacharcolon}{\kern0pt}assms{\isacharparenright}{\kern0pt}\isanewline
\ \ \ \isacommand{apply}\isamarkupfalse%
{\isacharparenleft}{\kern0pt}subst\ Fn{\isacharunderscore}{\kern0pt}perm{\isacharprime}{\kern0pt}{\isacharunderscore}{\kern0pt}comp{\isacharcomma}{\kern0pt}\ simp\ add{\isacharcolon}{\kern0pt}assms{\isacharcomma}{\kern0pt}\ rule\ converse{\isacharunderscore}{\kern0pt}in{\isacharunderscore}{\kern0pt}nat{\isacharunderscore}{\kern0pt}perms{\isacharcomma}{\kern0pt}\ simp\ add{\isacharcolon}{\kern0pt}assms{\isacharparenright}{\kern0pt}\isanewline
\ \ \ \isacommand{apply}\isamarkupfalse%
{\isacharparenleft}{\kern0pt}subst\ right{\isacharunderscore}{\kern0pt}comp{\isacharunderscore}{\kern0pt}inverse{\isacharcomma}{\kern0pt}\ rule\ bij{\isacharunderscore}{\kern0pt}is{\isacharunderscore}{\kern0pt}surj{\isacharparenright}{\kern0pt}\isanewline
\ \ \isacommand{using}\isamarkupfalse%
\ assms\ nat{\isacharunderscore}{\kern0pt}perms{\isacharunderscore}{\kern0pt}def\ Fn{\isacharunderscore}{\kern0pt}perm{\isacharunderscore}{\kern0pt}id\ Fn{\isacharunderscore}{\kern0pt}perm{\isacharprime}{\kern0pt}{\isacharunderscore}{\kern0pt}id\ \isanewline
\ \ \ \ \isacommand{apply}\isamarkupfalse%
\ auto{\isacharbrackleft}{\kern0pt}{\isadigit{2}}{\isacharbrackright}{\kern0pt}\isanewline
\ \ \isacommand{apply}\isamarkupfalse%
{\isacharparenleft}{\kern0pt}subst\ Fn{\isacharunderscore}{\kern0pt}perm{\isacharprime}{\kern0pt}{\isacharunderscore}{\kern0pt}comp{\isacharcomma}{\kern0pt}\ rule\ converse{\isacharunderscore}{\kern0pt}in{\isacharunderscore}{\kern0pt}nat{\isacharunderscore}{\kern0pt}perms{\isacharparenright}{\kern0pt}\isanewline
\ \ \isacommand{using}\isamarkupfalse%
\ assms\isanewline
\ \ \ \ \isacommand{apply}\isamarkupfalse%
\ auto{\isacharbrackleft}{\kern0pt}{\isadigit{2}}{\isacharbrackright}{\kern0pt}\isanewline
\ \ \isacommand{apply}\isamarkupfalse%
{\isacharparenleft}{\kern0pt}subst\ left{\isacharunderscore}{\kern0pt}comp{\isacharunderscore}{\kern0pt}inverse{\isacharcomma}{\kern0pt}\ rule\ bij{\isacharunderscore}{\kern0pt}is{\isacharunderscore}{\kern0pt}inj{\isacharparenright}{\kern0pt}\isanewline
\ \ \isacommand{using}\isamarkupfalse%
\ assms\ nat{\isacharunderscore}{\kern0pt}perms{\isacharunderscore}{\kern0pt}def\ Fn{\isacharunderscore}{\kern0pt}perm{\isacharunderscore}{\kern0pt}id\ Fn{\isacharunderscore}{\kern0pt}perm{\isacharprime}{\kern0pt}{\isacharunderscore}{\kern0pt}id\ \isanewline
\ \ \ \isacommand{apply}\isamarkupfalse%
\ auto{\isacharbrackleft}{\kern0pt}{\isadigit{2}}{\isacharbrackright}{\kern0pt}\isanewline
\ \ \isacommand{done}\isamarkupfalse%
%
\endisatagproof
{\isafoldproof}%
%
\isadelimproof
\isanewline
%
\endisadelimproof
\isanewline
\isacommand{lemma}\isamarkupfalse%
\ Fn{\isacharunderscore}{\kern0pt}perm{\isacharprime}{\kern0pt}{\isacharunderscore}{\kern0pt}converse\ {\isacharcolon}{\kern0pt}\ \isanewline
\ \ \isakeyword{fixes}\ f\ \isanewline
\ \ \isakeyword{assumes}\ {\isachardoublequoteopen}f\ {\isasymin}\ nat{\isacharunderscore}{\kern0pt}perms{\isachardoublequoteclose}\isanewline
\ \ \isakeyword{shows}\ {\isachardoublequoteopen}converse{\isacharparenleft}{\kern0pt}Fn{\isacharunderscore}{\kern0pt}perm{\isacharprime}{\kern0pt}{\isacharparenleft}{\kern0pt}f{\isacharparenright}{\kern0pt}{\isacharparenright}{\kern0pt}\ {\isacharequal}{\kern0pt}\ Fn{\isacharunderscore}{\kern0pt}perm{\isacharprime}{\kern0pt}{\isacharparenleft}{\kern0pt}converse{\isacharparenleft}{\kern0pt}f{\isacharparenright}{\kern0pt}{\isacharparenright}{\kern0pt}{\isachardoublequoteclose}\ \isanewline
%
\isadelimproof
%
\endisadelimproof
%
\isatagproof
\isacommand{proof}\isamarkupfalse%
\ {\isacharparenleft}{\kern0pt}rule\ equality{\isacharunderscore}{\kern0pt}iffI{\isacharparenright}{\kern0pt}\isanewline
\ \ \isacommand{fix}\isamarkupfalse%
\ v\isanewline
\isanewline
\ \ \isacommand{have}\isamarkupfalse%
\ {\isachardoublequoteopen}Fn{\isacharunderscore}{\kern0pt}perm{\isacharprime}{\kern0pt}{\isacharparenleft}{\kern0pt}converse{\isacharparenleft}{\kern0pt}f{\isacharparenright}{\kern0pt}{\isacharparenright}{\kern0pt}\ {\isasymin}\ surj{\isacharparenleft}{\kern0pt}Fn{\isacharcomma}{\kern0pt}\ Fn{\isacharparenright}{\kern0pt}{\isachardoublequoteclose}\ \isanewline
\ \ \ \ \isacommand{using}\isamarkupfalse%
\ Fn{\isacharunderscore}{\kern0pt}perm{\isacharprime}{\kern0pt}{\isacharunderscore}{\kern0pt}bij\ assms\ bij{\isacharunderscore}{\kern0pt}is{\isacharunderscore}{\kern0pt}surj\ converse{\isacharunderscore}{\kern0pt}in{\isacharunderscore}{\kern0pt}nat{\isacharunderscore}{\kern0pt}perms\isanewline
\ \ \ \ \isacommand{by}\isamarkupfalse%
\ force\ \isanewline
\ \ \isacommand{then}\isamarkupfalse%
\ \isacommand{have}\isamarkupfalse%
\ surjE{\isacharcolon}{\kern0pt}\ {\isachardoublequoteopen}{\isasymAnd}p{\isachardot}{\kern0pt}\ p\ {\isasymin}\ Fn\ {\isasymLongrightarrow}\ {\isasymexists}p{\isacharprime}{\kern0pt}\ {\isasymin}\ Fn{\isachardot}{\kern0pt}\ Fn{\isacharunderscore}{\kern0pt}perm{\isacharprime}{\kern0pt}{\isacharparenleft}{\kern0pt}converse{\isacharparenleft}{\kern0pt}f{\isacharparenright}{\kern0pt}{\isacharparenright}{\kern0pt}{\isacharbackquote}{\kern0pt}p{\isacharprime}{\kern0pt}\ {\isacharequal}{\kern0pt}\ p{\isachardoublequoteclose}\ \isacommand{using}\isamarkupfalse%
\ surj{\isacharunderscore}{\kern0pt}def\ Fn{\isacharunderscore}{\kern0pt}perm{\isacharunderscore}{\kern0pt}in{\isacharunderscore}{\kern0pt}Fn\ assms\ \isacommand{by}\isamarkupfalse%
\ auto\isanewline
\isanewline
\ \ \isacommand{have}\isamarkupfalse%
\ eq{\isacharcolon}{\kern0pt}\ {\isachardoublequoteopen}{\isasymAnd}p{\isachardot}{\kern0pt}\ p\ {\isasymin}\ Fn\ {\isasymLongrightarrow}\ Fn{\isacharunderscore}{\kern0pt}perm{\isacharprime}{\kern0pt}{\isacharparenleft}{\kern0pt}f{\isacharparenright}{\kern0pt}{\isacharbackquote}{\kern0pt}{\isacharparenleft}{\kern0pt}Fn{\isacharunderscore}{\kern0pt}perm{\isacharprime}{\kern0pt}{\isacharparenleft}{\kern0pt}converse{\isacharparenleft}{\kern0pt}f{\isacharparenright}{\kern0pt}{\isacharparenright}{\kern0pt}{\isacharbackquote}{\kern0pt}p{\isacharparenright}{\kern0pt}\ {\isacharequal}{\kern0pt}\ p{\isachardoublequoteclose}\ \isanewline
\ \ \ \ \isacommand{apply}\isamarkupfalse%
{\isacharparenleft}{\kern0pt}subst\ comp{\isacharunderscore}{\kern0pt}fun{\isacharunderscore}{\kern0pt}apply{\isacharbrackleft}{\kern0pt}symmetric{\isacharbrackright}{\kern0pt}{\isacharparenright}{\kern0pt}\isanewline
\ \ \ \ \ \ \isacommand{apply}\isamarkupfalse%
{\isacharparenleft}{\kern0pt}rule\ Fn{\isacharunderscore}{\kern0pt}perm{\isacharprime}{\kern0pt}{\isacharunderscore}{\kern0pt}type{\isacharparenright}{\kern0pt}\isanewline
\ \ \ \ \isacommand{using}\isamarkupfalse%
\ assms\ converse{\isacharunderscore}{\kern0pt}in{\isacharunderscore}{\kern0pt}nat{\isacharunderscore}{\kern0pt}perms\ \isanewline
\ \ \ \ \ \ \isacommand{apply}\isamarkupfalse%
\ auto{\isacharbrackleft}{\kern0pt}{\isadigit{2}}{\isacharbrackright}{\kern0pt}\isanewline
\ \ \ \ \isacommand{apply}\isamarkupfalse%
{\isacharparenleft}{\kern0pt}subst\ Fn{\isacharunderscore}{\kern0pt}perm{\isacharprime}{\kern0pt}{\isacharunderscore}{\kern0pt}comp{\isacharparenright}{\kern0pt}\isanewline
\ \ \ \ \isacommand{using}\isamarkupfalse%
\ assms\ converse{\isacharunderscore}{\kern0pt}in{\isacharunderscore}{\kern0pt}nat{\isacharunderscore}{\kern0pt}perms\ \isanewline
\ \ \ \ \ \ \isacommand{apply}\isamarkupfalse%
\ auto{\isacharbrackleft}{\kern0pt}{\isadigit{2}}{\isacharbrackright}{\kern0pt}\isanewline
\ \ \ \ \isacommand{apply}\isamarkupfalse%
{\isacharparenleft}{\kern0pt}subst\ right{\isacharunderscore}{\kern0pt}comp{\isacharunderscore}{\kern0pt}inverse{\isacharparenright}{\kern0pt}\isanewline
\ \ \ \ \isacommand{using}\isamarkupfalse%
\ nat{\isacharunderscore}{\kern0pt}perms{\isacharunderscore}{\kern0pt}def\ assms\ bij{\isacharunderscore}{\kern0pt}is{\isacharunderscore}{\kern0pt}surj\ \isanewline
\ \ \ \ \ \isacommand{apply}\isamarkupfalse%
\ auto{\isacharbrackleft}{\kern0pt}{\isadigit{1}}{\isacharbrackright}{\kern0pt}\isanewline
\ \ \ \ \isacommand{apply}\isamarkupfalse%
{\isacharparenleft}{\kern0pt}subst\ Fn{\isacharunderscore}{\kern0pt}perm{\isacharprime}{\kern0pt}{\isacharunderscore}{\kern0pt}id{\isacharparenright}{\kern0pt}\isanewline
\ \ \ \ \isacommand{apply}\isamarkupfalse%
{\isacharparenleft}{\kern0pt}rule\ function{\isacharunderscore}{\kern0pt}apply{\isacharunderscore}{\kern0pt}equality{\isacharcomma}{\kern0pt}\ simp{\isacharparenright}{\kern0pt}\isanewline
\ \ \ \ \isacommand{using}\isamarkupfalse%
\ id{\isacharunderscore}{\kern0pt}bij\ bij{\isacharunderscore}{\kern0pt}def\ inj{\isacharunderscore}{\kern0pt}def\ Pi{\isacharunderscore}{\kern0pt}def\isanewline
\ \ \ \ \isacommand{by}\isamarkupfalse%
\ auto\isanewline
\isanewline
\ \ \isacommand{have}\isamarkupfalse%
\ I{\isadigit{1}}{\isacharcolon}{\kern0pt}\ {\isachardoublequoteopen}v\ {\isasymin}\ converse{\isacharparenleft}{\kern0pt}Fn{\isacharunderscore}{\kern0pt}perm{\isacharprime}{\kern0pt}{\isacharparenleft}{\kern0pt}f{\isacharparenright}{\kern0pt}{\isacharparenright}{\kern0pt}\ {\isasymlongleftrightarrow}\ {\isacharparenleft}{\kern0pt}{\isasymexists}p\ {\isasymin}\ Fn{\isachardot}{\kern0pt}\ v\ {\isacharequal}{\kern0pt}\ {\isacharless}{\kern0pt}Fn{\isacharunderscore}{\kern0pt}perm{\isacharparenleft}{\kern0pt}f{\isacharcomma}{\kern0pt}\ p{\isacharparenright}{\kern0pt}{\isacharcomma}{\kern0pt}\ p{\isachargreater}{\kern0pt}{\isacharparenright}{\kern0pt}{\isachardoublequoteclose}\isanewline
\ \ \ \ \isacommand{using}\isamarkupfalse%
\ Fn{\isacharunderscore}{\kern0pt}perm{\isacharprime}{\kern0pt}{\isacharunderscore}{\kern0pt}type\ assms\ Pi{\isacharunderscore}{\kern0pt}def\ Fn{\isacharunderscore}{\kern0pt}perm{\isacharprime}{\kern0pt}{\isacharunderscore}{\kern0pt}def\isanewline
\ \ \ \ \isacommand{by}\isamarkupfalse%
\ force\isanewline
\ \ \isacommand{have}\isamarkupfalse%
\ I{\isadigit{2}}{\isacharcolon}{\kern0pt}\ {\isachardoublequoteopen}{\isachardot}{\kern0pt}{\isachardot}{\kern0pt}{\isachardot}{\kern0pt}\ {\isasymlongleftrightarrow}\ {\isacharparenleft}{\kern0pt}{\isasymexists}p\ {\isasymin}\ Fn{\isachardot}{\kern0pt}\ v\ {\isacharequal}{\kern0pt}\ {\isacharless}{\kern0pt}Fn{\isacharunderscore}{\kern0pt}perm{\isacharprime}{\kern0pt}{\isacharparenleft}{\kern0pt}f{\isacharparenright}{\kern0pt}{\isacharbackquote}{\kern0pt}p{\isacharcomma}{\kern0pt}\ p{\isachargreater}{\kern0pt}{\isacharparenright}{\kern0pt}{\isachardoublequoteclose}\isanewline
\ \ \ \ \isacommand{using}\isamarkupfalse%
\ assms\ Fn{\isacharunderscore}{\kern0pt}perm{\isacharprime}{\kern0pt}{\isacharunderscore}{\kern0pt}eq\ \isanewline
\ \ \ \ \isacommand{by}\isamarkupfalse%
\ auto\isanewline
\ \ \isacommand{have}\isamarkupfalse%
\ I{\isadigit{3}}{\isacharcolon}{\kern0pt}\ {\isachardoublequoteopen}{\isachardot}{\kern0pt}{\isachardot}{\kern0pt}{\isachardot}{\kern0pt}\ {\isasymlongleftrightarrow}\ {\isacharparenleft}{\kern0pt}{\isasymexists}p{\isacharprime}{\kern0pt}\ {\isasymin}\ Fn{\isachardot}{\kern0pt}\ v\ {\isacharequal}{\kern0pt}\ {\isacharless}{\kern0pt}Fn{\isacharunderscore}{\kern0pt}perm{\isacharprime}{\kern0pt}{\isacharparenleft}{\kern0pt}f{\isacharparenright}{\kern0pt}{\isacharbackquote}{\kern0pt}{\isacharparenleft}{\kern0pt}Fn{\isacharunderscore}{\kern0pt}perm{\isacharprime}{\kern0pt}{\isacharparenleft}{\kern0pt}converse{\isacharparenleft}{\kern0pt}f{\isacharparenright}{\kern0pt}{\isacharparenright}{\kern0pt}{\isacharbackquote}{\kern0pt}p{\isacharprime}{\kern0pt}{\isacharparenright}{\kern0pt}{\isacharcomma}{\kern0pt}\ Fn{\isacharunderscore}{\kern0pt}perm{\isacharprime}{\kern0pt}{\isacharparenleft}{\kern0pt}converse{\isacharparenleft}{\kern0pt}f{\isacharparenright}{\kern0pt}{\isacharparenright}{\kern0pt}{\isacharbackquote}{\kern0pt}p{\isacharprime}{\kern0pt}{\isachargreater}{\kern0pt}{\isacharparenright}{\kern0pt}{\isachardoublequoteclose}\isanewline
\ \ \ \ \isacommand{using}\isamarkupfalse%
\ converse{\isacharunderscore}{\kern0pt}in{\isacharunderscore}{\kern0pt}nat{\isacharunderscore}{\kern0pt}perms\ assms\ Fn{\isacharunderscore}{\kern0pt}perm{\isacharunderscore}{\kern0pt}in{\isacharunderscore}{\kern0pt}Fn\ Fn{\isacharunderscore}{\kern0pt}perm{\isacharprime}{\kern0pt}{\isacharunderscore}{\kern0pt}eq\ surjE\isanewline
\ \ \ \ \isacommand{by}\isamarkupfalse%
\ force\isanewline
\ \ \isacommand{have}\isamarkupfalse%
\ I{\isadigit{4}}{\isacharcolon}{\kern0pt}\ {\isachardoublequoteopen}{\isachardot}{\kern0pt}{\isachardot}{\kern0pt}{\isachardot}{\kern0pt}\ {\isasymlongleftrightarrow}\ {\isacharparenleft}{\kern0pt}{\isasymexists}p{\isacharprime}{\kern0pt}\ {\isasymin}\ Fn{\isachardot}{\kern0pt}\ v\ {\isacharequal}{\kern0pt}\ {\isacharless}{\kern0pt}p{\isacharprime}{\kern0pt}{\isacharcomma}{\kern0pt}\ Fn{\isacharunderscore}{\kern0pt}perm{\isacharprime}{\kern0pt}{\isacharparenleft}{\kern0pt}converse{\isacharparenleft}{\kern0pt}f{\isacharparenright}{\kern0pt}{\isacharparenright}{\kern0pt}{\isacharbackquote}{\kern0pt}p{\isacharprime}{\kern0pt}{\isachargreater}{\kern0pt}{\isacharparenright}{\kern0pt}{\isachardoublequoteclose}\isanewline
\ \ \ \ \isacommand{apply}\isamarkupfalse%
{\isacharparenleft}{\kern0pt}rule\ bex{\isacharunderscore}{\kern0pt}iff{\isacharparenright}{\kern0pt}\isanewline
\ \ \ \ \isacommand{apply}\isamarkupfalse%
{\isacharparenleft}{\kern0pt}subst\ comp{\isacharunderscore}{\kern0pt}fun{\isacharunderscore}{\kern0pt}apply\ {\isacharbrackleft}{\kern0pt}symmetric{\isacharbrackright}{\kern0pt}{\isacharparenright}{\kern0pt}\isanewline
\ \ \ \ \ \ \isacommand{apply}\isamarkupfalse%
{\isacharparenleft}{\kern0pt}rule\ Fn{\isacharunderscore}{\kern0pt}perm{\isacharprime}{\kern0pt}{\isacharunderscore}{\kern0pt}type{\isacharcomma}{\kern0pt}\ rule\ converse{\isacharunderscore}{\kern0pt}in{\isacharunderscore}{\kern0pt}nat{\isacharunderscore}{\kern0pt}perms{\isacharcomma}{\kern0pt}\ simp\ add{\isacharcolon}{\kern0pt}assms{\isacharcomma}{\kern0pt}\ simp{\isacharparenright}{\kern0pt}\isanewline
\ \ \ \ \isacommand{apply}\isamarkupfalse%
{\isacharparenleft}{\kern0pt}subst\ Fn{\isacharunderscore}{\kern0pt}perm{\isacharprime}{\kern0pt}{\isacharunderscore}{\kern0pt}comp{\isacharcomma}{\kern0pt}\ simp\ add{\isacharcolon}{\kern0pt}assms{\isacharcomma}{\kern0pt}\ rule\ converse{\isacharunderscore}{\kern0pt}in{\isacharunderscore}{\kern0pt}nat{\isacharunderscore}{\kern0pt}perms{\isacharcomma}{\kern0pt}\ simp\ add{\isacharcolon}{\kern0pt}assms{\isacharparenright}{\kern0pt}\isanewline
\ \ \ \ \isacommand{apply}\isamarkupfalse%
{\isacharparenleft}{\kern0pt}subst\ right{\isacharunderscore}{\kern0pt}comp{\isacharunderscore}{\kern0pt}inverse{\isacharparenright}{\kern0pt}\isanewline
\ \ \ \ \isacommand{using}\isamarkupfalse%
\ assms\ nat{\isacharunderscore}{\kern0pt}perms{\isacharunderscore}{\kern0pt}def\ bij{\isacharunderscore}{\kern0pt}is{\isacharunderscore}{\kern0pt}surj\ \isanewline
\ \ \ \ \ \isacommand{apply}\isamarkupfalse%
\ force\isanewline
\ \ \ \ \isacommand{apply}\isamarkupfalse%
{\isacharparenleft}{\kern0pt}subst\ Fn{\isacharunderscore}{\kern0pt}perm{\isacharprime}{\kern0pt}{\isacharunderscore}{\kern0pt}id{\isacharcomma}{\kern0pt}\ subst\ function{\isacharunderscore}{\kern0pt}apply{\isacharunderscore}{\kern0pt}equality{\isacharcomma}{\kern0pt}\ rule\ idI{\isacharcomma}{\kern0pt}\ simp{\isacharparenright}{\kern0pt}\isanewline
\ \ \ \ \isacommand{using}\isamarkupfalse%
\ id{\isacharunderscore}{\kern0pt}bij\ bij{\isacharunderscore}{\kern0pt}def\ inj{\isacharunderscore}{\kern0pt}def\ Pi{\isacharunderscore}{\kern0pt}def\ \isanewline
\ \ \ \ \isacommand{by}\isamarkupfalse%
\ auto\isanewline
\ \ \isacommand{have}\isamarkupfalse%
\ I{\isadigit{5}}{\isacharcolon}{\kern0pt}\ {\isachardoublequoteopen}{\isachardot}{\kern0pt}{\isachardot}{\kern0pt}{\isachardot}{\kern0pt}\ {\isasymlongleftrightarrow}\ {\isacharparenleft}{\kern0pt}{\isasymexists}p{\isacharprime}{\kern0pt}\ {\isasymin}\ Fn{\isachardot}{\kern0pt}\ v\ {\isacharequal}{\kern0pt}\ {\isacharless}{\kern0pt}p{\isacharprime}{\kern0pt}{\isacharcomma}{\kern0pt}\ Fn{\isacharunderscore}{\kern0pt}perm{\isacharparenleft}{\kern0pt}converse{\isacharparenleft}{\kern0pt}f{\isacharparenright}{\kern0pt}{\isacharcomma}{\kern0pt}\ p{\isacharprime}{\kern0pt}{\isacharparenright}{\kern0pt}{\isachargreater}{\kern0pt}{\isacharparenright}{\kern0pt}{\isachardoublequoteclose}\isanewline
\ \ \ \ \isacommand{using}\isamarkupfalse%
\ Fn{\isacharunderscore}{\kern0pt}perm{\isacharprime}{\kern0pt}{\isacharunderscore}{\kern0pt}type\ assms\ Pi{\isacharunderscore}{\kern0pt}def\ Fn{\isacharunderscore}{\kern0pt}perm{\isacharprime}{\kern0pt}{\isacharunderscore}{\kern0pt}def\ converse{\isacharunderscore}{\kern0pt}in{\isacharunderscore}{\kern0pt}nat{\isacharunderscore}{\kern0pt}perms\ Fn{\isacharunderscore}{\kern0pt}perm{\isacharprime}{\kern0pt}{\isacharunderscore}{\kern0pt}eq\isanewline
\ \ \ \ \isacommand{by}\isamarkupfalse%
\ force\isanewline
\ \ \isacommand{have}\isamarkupfalse%
\ I{\isadigit{6}}{\isacharcolon}{\kern0pt}\ {\isachardoublequoteopen}{\isachardot}{\kern0pt}{\isachardot}{\kern0pt}{\isachardot}{\kern0pt}\ {\isasymlongleftrightarrow}\ v\ {\isasymin}\ Fn{\isacharunderscore}{\kern0pt}perm{\isacharprime}{\kern0pt}{\isacharparenleft}{\kern0pt}converse{\isacharparenleft}{\kern0pt}f{\isacharparenright}{\kern0pt}{\isacharparenright}{\kern0pt}{\isachardoublequoteclose}\ \isanewline
\ \ \ \ \isacommand{using}\isamarkupfalse%
\ Fn{\isacharunderscore}{\kern0pt}perm{\isacharprime}{\kern0pt}{\isacharunderscore}{\kern0pt}type\ assms\ Pi{\isacharunderscore}{\kern0pt}def\ Fn{\isacharunderscore}{\kern0pt}perm{\isacharprime}{\kern0pt}{\isacharunderscore}{\kern0pt}def\ converse{\isacharunderscore}{\kern0pt}in{\isacharunderscore}{\kern0pt}nat{\isacharunderscore}{\kern0pt}perms\isanewline
\ \ \ \ \isacommand{by}\isamarkupfalse%
\ force\isanewline
\ \ \isacommand{show}\isamarkupfalse%
\ {\isachardoublequoteopen}v\ {\isasymin}\ converse{\isacharparenleft}{\kern0pt}Fn{\isacharunderscore}{\kern0pt}perm{\isacharprime}{\kern0pt}{\isacharparenleft}{\kern0pt}f{\isacharparenright}{\kern0pt}{\isacharparenright}{\kern0pt}\ {\isasymlongleftrightarrow}\ v\ {\isasymin}\ Fn{\isacharunderscore}{\kern0pt}perm{\isacharprime}{\kern0pt}{\isacharparenleft}{\kern0pt}converse{\isacharparenleft}{\kern0pt}f{\isacharparenright}{\kern0pt}{\isacharparenright}{\kern0pt}{\isachardoublequoteclose}\ \isanewline
\ \ \ \ \isacommand{using}\isamarkupfalse%
\ I{\isadigit{1}}\ I{\isadigit{2}}\ I{\isadigit{3}}\ I{\isadigit{4}}\ I{\isadigit{5}}\ I{\isadigit{6}}\ \isanewline
\ \ \ \ \isacommand{by}\isamarkupfalse%
\ auto\isanewline
\isacommand{qed}\isamarkupfalse%
%
\endisatagproof
{\isafoldproof}%
%
\isadelimproof
\isanewline
%
\endisadelimproof
\isanewline
\isacommand{lemma}\isamarkupfalse%
\ Fn{\isacharunderscore}{\kern0pt}perm{\isacharprime}{\kern0pt}{\isacharunderscore}{\kern0pt}preserves{\isacharunderscore}{\kern0pt}order\ {\isacharcolon}{\kern0pt}\ \isanewline
\ \ \isakeyword{fixes}\ f\ p\ p{\isacharprime}{\kern0pt}\ \isanewline
\ \ \isakeyword{assumes}\ {\isachardoublequoteopen}f\ {\isasymin}\ nat{\isacharunderscore}{\kern0pt}perms{\isachardoublequoteclose}\ {\isachardoublequoteopen}p\ {\isasymin}\ Fn{\isachardoublequoteclose}\ {\isachardoublequoteopen}p{\isacharprime}{\kern0pt}\ {\isasymin}\ Fn{\isachardoublequoteclose}\ {\isachardoublequoteopen}{\isacharless}{\kern0pt}p{\isacharcomma}{\kern0pt}\ p{\isacharprime}{\kern0pt}{\isachargreater}{\kern0pt}\ {\isasymin}\ Fn{\isacharunderscore}{\kern0pt}leq{\isachardoublequoteclose}\isanewline
\ \ \isakeyword{shows}\ {\isachardoublequoteopen}{\isacharless}{\kern0pt}Fn{\isacharunderscore}{\kern0pt}perm{\isacharprime}{\kern0pt}{\isacharparenleft}{\kern0pt}f{\isacharparenright}{\kern0pt}{\isacharbackquote}{\kern0pt}p{\isacharcomma}{\kern0pt}\ Fn{\isacharunderscore}{\kern0pt}perm{\isacharprime}{\kern0pt}{\isacharparenleft}{\kern0pt}f{\isacharparenright}{\kern0pt}{\isacharbackquote}{\kern0pt}p{\isacharprime}{\kern0pt}{\isachargreater}{\kern0pt}\ {\isasymin}\ Fn{\isacharunderscore}{\kern0pt}leq{\isachardoublequoteclose}\ \isanewline
%
\isadelimproof
%
\endisadelimproof
%
\isatagproof
\isacommand{proof}\isamarkupfalse%
\ {\isacharminus}{\kern0pt}\ \isanewline
\ \ \isacommand{have}\isamarkupfalse%
\ {\isachardoublequoteopen}p{\isacharprime}{\kern0pt}\ {\isasymsubseteq}\ p{\isachardoublequoteclose}\ \isacommand{using}\isamarkupfalse%
\ assms\ Fn{\isacharunderscore}{\kern0pt}leq{\isacharunderscore}{\kern0pt}def\ \isacommand{by}\isamarkupfalse%
\ auto\isanewline
\isanewline
\ \ \isacommand{have}\isamarkupfalse%
\ {\isachardoublequoteopen}Fn{\isacharunderscore}{\kern0pt}perm{\isacharprime}{\kern0pt}{\isacharparenleft}{\kern0pt}f{\isacharparenright}{\kern0pt}{\isacharbackquote}{\kern0pt}p{\isacharprime}{\kern0pt}\ {\isasymsubseteq}\ Fn{\isacharunderscore}{\kern0pt}perm{\isacharprime}{\kern0pt}{\isacharparenleft}{\kern0pt}f{\isacharparenright}{\kern0pt}{\isacharbackquote}{\kern0pt}p{\isachardoublequoteclose}\ \isanewline
\ \ \isacommand{proof}\isamarkupfalse%
{\isacharparenleft}{\kern0pt}rule\ subsetI{\isacharparenright}{\kern0pt}\isanewline
\ \ \ \ \isacommand{fix}\isamarkupfalse%
\ v\ \isanewline
\ \ \ \ \isacommand{assume}\isamarkupfalse%
\ {\isachardoublequoteopen}v\ {\isasymin}\ Fn{\isacharunderscore}{\kern0pt}perm{\isacharprime}{\kern0pt}{\isacharparenleft}{\kern0pt}f{\isacharparenright}{\kern0pt}{\isacharbackquote}{\kern0pt}p{\isacharprime}{\kern0pt}{\isachardoublequoteclose}\ \isanewline
\ \ \ \ \isacommand{then}\isamarkupfalse%
\ \isacommand{have}\isamarkupfalse%
\ {\isachardoublequoteopen}v\ {\isasymin}\ Fn{\isacharunderscore}{\kern0pt}perm{\isacharparenleft}{\kern0pt}f{\isacharcomma}{\kern0pt}\ p{\isacharprime}{\kern0pt}{\isacharparenright}{\kern0pt}{\isachardoublequoteclose}\ \isacommand{using}\isamarkupfalse%
\ Fn{\isacharunderscore}{\kern0pt}perm{\isacharprime}{\kern0pt}{\isacharunderscore}{\kern0pt}eq\ assms\ \isacommand{by}\isamarkupfalse%
\ auto\isanewline
\ \ \ \ \isacommand{then}\isamarkupfalse%
\ \isacommand{have}\isamarkupfalse%
\ {\isachardoublequoteopen}{\isasymexists}n\ {\isasymin}\ nat{\isachardot}{\kern0pt}\ {\isasymexists}m\ {\isasymin}\ nat{\isachardot}{\kern0pt}\ {\isasymexists}l\ {\isasymin}\ {\isadigit{2}}{\isachardot}{\kern0pt}\ {\isacharless}{\kern0pt}{\isacharless}{\kern0pt}n{\isacharcomma}{\kern0pt}\ m{\isachargreater}{\kern0pt}{\isacharcomma}{\kern0pt}\ l{\isachargreater}{\kern0pt}\ {\isasymin}\ p{\isacharprime}{\kern0pt}\ {\isasymand}\ v\ {\isacharequal}{\kern0pt}\ {\isacharless}{\kern0pt}{\isacharless}{\kern0pt}f{\isacharbackquote}{\kern0pt}n{\isacharcomma}{\kern0pt}\ m{\isachargreater}{\kern0pt}{\isacharcomma}{\kern0pt}\ l{\isachargreater}{\kern0pt}{\isachardoublequoteclose}\ \isanewline
\ \ \ \ \ \ \isacommand{apply}\isamarkupfalse%
{\isacharparenleft}{\kern0pt}rule{\isacharunderscore}{\kern0pt}tac\ Fn{\isacharunderscore}{\kern0pt}permE{\isacharparenright}{\kern0pt}\isanewline
\ \ \ \ \ \ \isacommand{using}\isamarkupfalse%
\ assms\isanewline
\ \ \ \ \ \ \isacommand{by}\isamarkupfalse%
\ auto\isanewline
\ \ \ \ \isacommand{then}\isamarkupfalse%
\ \isacommand{obtain}\isamarkupfalse%
\ n\ m\ l\ \isakeyword{where}\ H{\isacharcolon}{\kern0pt}\ {\isachardoublequoteopen}n\ {\isasymin}\ nat{\isachardoublequoteclose}\ {\isachardoublequoteopen}m\ {\isasymin}\ nat{\isachardoublequoteclose}\ {\isachardoublequoteopen}l\ {\isasymin}\ {\isadigit{2}}{\isachardoublequoteclose}\ {\isachardoublequoteopen}{\isacharless}{\kern0pt}{\isacharless}{\kern0pt}n{\isacharcomma}{\kern0pt}\ m{\isachargreater}{\kern0pt}{\isacharcomma}{\kern0pt}\ l{\isachargreater}{\kern0pt}\ {\isasymin}\ p{\isacharprime}{\kern0pt}{\isachardoublequoteclose}\ {\isachardoublequoteopen}v\ {\isacharequal}{\kern0pt}\ {\isacharless}{\kern0pt}{\isacharless}{\kern0pt}f{\isacharbackquote}{\kern0pt}n{\isacharcomma}{\kern0pt}\ m{\isachargreater}{\kern0pt}{\isacharcomma}{\kern0pt}\ l{\isachargreater}{\kern0pt}{\isachardoublequoteclose}\ \isacommand{by}\isamarkupfalse%
\ auto\isanewline
\ \ \ \ \isacommand{then}\isamarkupfalse%
\ \isacommand{have}\isamarkupfalse%
\ {\isachardoublequoteopen}{\isacharless}{\kern0pt}{\isacharless}{\kern0pt}n{\isacharcomma}{\kern0pt}\ m{\isachargreater}{\kern0pt}{\isacharcomma}{\kern0pt}\ l{\isachargreater}{\kern0pt}\ {\isasymin}\ p{\isachardoublequoteclose}\ \isacommand{using}\isamarkupfalse%
\ {\isacartoucheopen}p{\isacharprime}{\kern0pt}\ {\isasymsubseteq}\ p{\isacartoucheclose}\ \isacommand{by}\isamarkupfalse%
\ auto\isanewline
\ \ \ \ \isacommand{then}\isamarkupfalse%
\ \isacommand{have}\isamarkupfalse%
\ {\isachardoublequoteopen}{\isacharless}{\kern0pt}{\isacharless}{\kern0pt}f{\isacharbackquote}{\kern0pt}n{\isacharcomma}{\kern0pt}\ m{\isachargreater}{\kern0pt}{\isacharcomma}{\kern0pt}\ l{\isachargreater}{\kern0pt}\ {\isasymin}\ Fn{\isacharunderscore}{\kern0pt}perm{\isacharparenleft}{\kern0pt}f{\isacharcomma}{\kern0pt}\ p{\isacharparenright}{\kern0pt}{\isachardoublequoteclose}\ \isanewline
\ \ \ \ \ \ \isacommand{unfolding}\isamarkupfalse%
\ Fn{\isacharunderscore}{\kern0pt}perm{\isacharunderscore}{\kern0pt}def\isanewline
\ \ \ \ \ \ \isacommand{by}\isamarkupfalse%
\ force\isanewline
\ \ \ \ \isacommand{then}\isamarkupfalse%
\ \isacommand{have}\isamarkupfalse%
\ {\isachardoublequoteopen}{\isacharless}{\kern0pt}{\isacharless}{\kern0pt}f{\isacharbackquote}{\kern0pt}n{\isacharcomma}{\kern0pt}\ m{\isachargreater}{\kern0pt}{\isacharcomma}{\kern0pt}\ l{\isachargreater}{\kern0pt}\ {\isasymin}\ Fn{\isacharunderscore}{\kern0pt}perm{\isacharprime}{\kern0pt}{\isacharparenleft}{\kern0pt}f{\isacharparenright}{\kern0pt}{\isacharbackquote}{\kern0pt}p{\isachardoublequoteclose}\ \isanewline
\ \ \ \ \ \ \isacommand{apply}\isamarkupfalse%
{\isacharparenleft}{\kern0pt}subst\ Fn{\isacharunderscore}{\kern0pt}perm{\isacharprime}{\kern0pt}{\isacharunderscore}{\kern0pt}eq{\isacharparenright}{\kern0pt}\isanewline
\ \ \ \ \ \ \isacommand{using}\isamarkupfalse%
\ assms\isanewline
\ \ \ \ \ \ \isacommand{by}\isamarkupfalse%
\ auto\isanewline
\ \ \ \ \isacommand{then}\isamarkupfalse%
\ \isacommand{show}\isamarkupfalse%
\ {\isachardoublequoteopen}v\ {\isasymin}\ Fn{\isacharunderscore}{\kern0pt}perm{\isacharprime}{\kern0pt}{\isacharparenleft}{\kern0pt}f{\isacharparenright}{\kern0pt}{\isacharbackquote}{\kern0pt}p{\isachardoublequoteclose}\ \isacommand{using}\isamarkupfalse%
\ H\ \isacommand{by}\isamarkupfalse%
\ auto\isanewline
\ \ \isacommand{qed}\isamarkupfalse%
\isanewline
\ \ \isacommand{then}\isamarkupfalse%
\ \isacommand{show}\isamarkupfalse%
\ {\isacharquery}{\kern0pt}thesis\ \isanewline
\ \ \ \ \isacommand{unfolding}\isamarkupfalse%
\ Fn{\isacharunderscore}{\kern0pt}leq{\isacharunderscore}{\kern0pt}def\ \isanewline
\ \ \ \ \isacommand{apply}\isamarkupfalse%
\ simp\ \isanewline
\ \ \ \ \isacommand{apply}\isamarkupfalse%
{\isacharparenleft}{\kern0pt}rule\ conjI{\isacharcomma}{\kern0pt}\ rule\ function{\isacharunderscore}{\kern0pt}value{\isacharunderscore}{\kern0pt}in{\isacharparenright}{\kern0pt}\isanewline
\ \ \ \ \ \ \isacommand{apply}\isamarkupfalse%
{\isacharparenleft}{\kern0pt}rule\ Fn{\isacharunderscore}{\kern0pt}perm{\isacharprime}{\kern0pt}{\isacharunderscore}{\kern0pt}type{\isacharcomma}{\kern0pt}\ simp\ add{\isacharcolon}{\kern0pt}assms{\isacharcomma}{\kern0pt}\ simp\ add{\isacharcolon}{\kern0pt}assms{\isacharparenright}{\kern0pt}\isanewline
\ \ \ \ \isacommand{apply}\isamarkupfalse%
{\isacharparenleft}{\kern0pt}rule\ function{\isacharunderscore}{\kern0pt}value{\isacharunderscore}{\kern0pt}in{\isacharcomma}{\kern0pt}\ rule\ Fn{\isacharunderscore}{\kern0pt}perm{\isacharprime}{\kern0pt}{\isacharunderscore}{\kern0pt}type{\isacharcomma}{\kern0pt}\ simp\ add{\isacharcolon}{\kern0pt}assms{\isacharcomma}{\kern0pt}\ simp\ add{\isacharcolon}{\kern0pt}assms{\isacharparenright}{\kern0pt}\isanewline
\ \ \ \ \isacommand{done}\isamarkupfalse%
\isanewline
\isacommand{qed}\isamarkupfalse%
%
\endisatagproof
{\isafoldproof}%
%
\isadelimproof
\isanewline
%
\endisadelimproof
\isanewline
\isacommand{lemma}\isamarkupfalse%
\ Fn{\isacharunderscore}{\kern0pt}perm{\isacharprime}{\kern0pt}{\isacharunderscore}{\kern0pt}preserves{\isacharunderscore}{\kern0pt}order{\isacharprime}{\kern0pt}\ {\isacharcolon}{\kern0pt}\ \isanewline
\ \ \isakeyword{fixes}\ f\ p\ p{\isacharprime}{\kern0pt}\ \isanewline
\ \ \isakeyword{assumes}\ {\isachardoublequoteopen}f\ {\isasymin}\ nat{\isacharunderscore}{\kern0pt}perms{\isachardoublequoteclose}\ {\isachardoublequoteopen}p\ {\isasymin}\ Fn{\isachardoublequoteclose}\ {\isachardoublequoteopen}p{\isacharprime}{\kern0pt}\ {\isasymin}\ Fn{\isachardoublequoteclose}\ {\isachardoublequoteopen}{\isacharless}{\kern0pt}Fn{\isacharunderscore}{\kern0pt}perm{\isacharprime}{\kern0pt}{\isacharparenleft}{\kern0pt}f{\isacharparenright}{\kern0pt}{\isacharbackquote}{\kern0pt}p{\isacharcomma}{\kern0pt}\ Fn{\isacharunderscore}{\kern0pt}perm{\isacharprime}{\kern0pt}{\isacharparenleft}{\kern0pt}f{\isacharparenright}{\kern0pt}{\isacharbackquote}{\kern0pt}p{\isacharprime}{\kern0pt}{\isachargreater}{\kern0pt}\ {\isasymin}\ Fn{\isacharunderscore}{\kern0pt}leq{\isachardoublequoteclose}\isanewline
\ \ \isakeyword{shows}\ {\isachardoublequoteopen}{\isacharless}{\kern0pt}p{\isacharcomma}{\kern0pt}\ p{\isacharprime}{\kern0pt}{\isachargreater}{\kern0pt}\ {\isasymin}\ Fn{\isacharunderscore}{\kern0pt}leq{\isachardoublequoteclose}\ \isanewline
%
\isadelimproof
%
\endisadelimproof
%
\isatagproof
\isacommand{proof}\isamarkupfalse%
\ {\isacharminus}{\kern0pt}\ \isanewline
\ \ \isacommand{have}\isamarkupfalse%
\ {\isachardoublequoteopen}f\ {\isasymin}\ inj{\isacharparenleft}{\kern0pt}nat{\isacharcomma}{\kern0pt}\ nat{\isacharparenright}{\kern0pt}{\isachardoublequoteclose}\ \isacommand{using}\isamarkupfalse%
\ assms\ nat{\isacharunderscore}{\kern0pt}perms{\isacharunderscore}{\kern0pt}def\ bij{\isacharunderscore}{\kern0pt}is{\isacharunderscore}{\kern0pt}inj\ \isacommand{by}\isamarkupfalse%
\ auto\isanewline
\ \ \isacommand{then}\isamarkupfalse%
\ \isacommand{have}\isamarkupfalse%
\ injE{\isacharcolon}{\kern0pt}\ {\isachardoublequoteopen}{\isasymAnd}\ n\ m{\isachardot}{\kern0pt}\ n\ {\isasymin}\ nat\ {\isasymLongrightarrow}\ m\ {\isasymin}\ nat\ {\isasymLongrightarrow}\ f{\isacharbackquote}{\kern0pt}n\ {\isacharequal}{\kern0pt}\ f{\isacharbackquote}{\kern0pt}m\ {\isasymLongrightarrow}\ n\ {\isacharequal}{\kern0pt}\ m{\isachardoublequoteclose}\ \isanewline
\ \ \ \ \isacommand{using}\isamarkupfalse%
\ inj{\isacharunderscore}{\kern0pt}def\ \isanewline
\ \ \ \ \isacommand{by}\isamarkupfalse%
\ auto\isanewline
\isanewline
\ \ \isacommand{have}\isamarkupfalse%
\ subsetH{\isacharcolon}{\kern0pt}\ {\isachardoublequoteopen}Fn{\isacharunderscore}{\kern0pt}perm{\isacharprime}{\kern0pt}{\isacharparenleft}{\kern0pt}f{\isacharparenright}{\kern0pt}{\isacharbackquote}{\kern0pt}p{\isacharprime}{\kern0pt}\ {\isasymsubseteq}\ Fn{\isacharunderscore}{\kern0pt}perm{\isacharprime}{\kern0pt}{\isacharparenleft}{\kern0pt}f{\isacharparenright}{\kern0pt}{\isacharbackquote}{\kern0pt}p{\isachardoublequoteclose}\ \isanewline
\ \ \ \ \isacommand{using}\isamarkupfalse%
\ assms\ Fn{\isacharunderscore}{\kern0pt}leq{\isacharunderscore}{\kern0pt}def\isanewline
\ \ \ \ \isacommand{by}\isamarkupfalse%
\ auto\isanewline
\ \ \isacommand{have}\isamarkupfalse%
\ {\isachardoublequoteopen}p{\isacharprime}{\kern0pt}\ {\isasymsubseteq}\ p{\isachardoublequoteclose}\ \isanewline
\ \ \isacommand{proof}\isamarkupfalse%
\ {\isacharparenleft}{\kern0pt}rule\ subsetI{\isacharparenright}{\kern0pt}\isanewline
\ \ \ \ \isacommand{fix}\isamarkupfalse%
\ v\ \isacommand{assume}\isamarkupfalse%
\ vin\ {\isacharcolon}{\kern0pt}\ {\isachardoublequoteopen}v\ {\isasymin}\ p{\isacharprime}{\kern0pt}{\isachardoublequoteclose}\ \isanewline
\ \ \ \ \isacommand{then}\isamarkupfalse%
\ \isacommand{have}\isamarkupfalse%
\ {\isachardoublequoteopen}p{\isacharprime}{\kern0pt}\ {\isasymsubseteq}\ {\isacharparenleft}{\kern0pt}nat\ {\isasymtimes}\ nat{\isacharparenright}{\kern0pt}\ {\isasymtimes}\ {\isadigit{2}}{\isachardoublequoteclose}\isanewline
\ \ \ \ \ \ \isacommand{using}\isamarkupfalse%
\ assms\ Fn{\isacharunderscore}{\kern0pt}def\ \isanewline
\ \ \ \ \ \ \isacommand{by}\isamarkupfalse%
\ auto\isanewline
\ \ \ \ \isacommand{then}\isamarkupfalse%
\ \isacommand{have}\isamarkupfalse%
\ {\isachardoublequoteopen}{\isasymexists}n\ {\isasymin}\ nat{\isachardot}{\kern0pt}\ {\isasymexists}m\ {\isasymin}\ nat{\isachardot}{\kern0pt}\ {\isasymexists}l\ {\isasymin}\ {\isadigit{2}}{\isachardot}{\kern0pt}\ v\ {\isacharequal}{\kern0pt}\ {\isacharless}{\kern0pt}{\isacharless}{\kern0pt}n{\isacharcomma}{\kern0pt}\ m{\isachargreater}{\kern0pt}{\isacharcomma}{\kern0pt}\ l{\isachargreater}{\kern0pt}{\isachardoublequoteclose}\ \isanewline
\ \ \ \ \ \ \isacommand{using}\isamarkupfalse%
\ vin\ \isanewline
\ \ \ \ \ \ \isacommand{by}\isamarkupfalse%
\ auto\isanewline
\ \ \ \ \isacommand{then}\isamarkupfalse%
\ \isacommand{obtain}\isamarkupfalse%
\ n\ m\ l\ \isakeyword{where}\ H{\isacharcolon}{\kern0pt}\ {\isachardoublequoteopen}n\ {\isasymin}\ nat{\isachardoublequoteclose}\ {\isachardoublequoteopen}m\ {\isasymin}\ nat{\isachardoublequoteclose}\ {\isachardoublequoteopen}l\ {\isasymin}\ {\isadigit{2}}{\isachardoublequoteclose}\ {\isachardoublequoteopen}v\ {\isacharequal}{\kern0pt}\ {\isacharless}{\kern0pt}{\isacharless}{\kern0pt}n{\isacharcomma}{\kern0pt}\ m{\isachargreater}{\kern0pt}{\isacharcomma}{\kern0pt}\ l{\isachargreater}{\kern0pt}{\isachardoublequoteclose}\ \isanewline
\ \ \ \ \ \ \isacommand{using}\isamarkupfalse%
\ assms\ Fn{\isacharunderscore}{\kern0pt}def\ vin\ \isanewline
\ \ \ \ \ \ \isacommand{by}\isamarkupfalse%
\ force\isanewline
\ \ \ \ \isacommand{then}\isamarkupfalse%
\ \isacommand{have}\isamarkupfalse%
\ {\isachardoublequoteopen}{\isacharless}{\kern0pt}{\isacharless}{\kern0pt}f{\isacharbackquote}{\kern0pt}n{\isacharcomma}{\kern0pt}\ m{\isachargreater}{\kern0pt}{\isacharcomma}{\kern0pt}\ l{\isachargreater}{\kern0pt}\ {\isasymin}\ Fn{\isacharunderscore}{\kern0pt}perm{\isacharparenleft}{\kern0pt}f{\isacharcomma}{\kern0pt}\ p{\isacharprime}{\kern0pt}{\isacharparenright}{\kern0pt}{\isachardoublequoteclose}\ \isanewline
\ \ \ \ \ \ \isacommand{unfolding}\isamarkupfalse%
\ Fn{\isacharunderscore}{\kern0pt}perm{\isacharunderscore}{\kern0pt}def\isanewline
\ \ \ \ \ \ \isacommand{using}\isamarkupfalse%
\ vin\ H\ \isanewline
\ \ \ \ \ \ \isacommand{by}\isamarkupfalse%
\ force\isanewline
\ \ \ \ \isacommand{then}\isamarkupfalse%
\ \isacommand{have}\isamarkupfalse%
\ {\isachardoublequoteopen}{\isacharless}{\kern0pt}{\isacharless}{\kern0pt}f{\isacharbackquote}{\kern0pt}n{\isacharcomma}{\kern0pt}\ m{\isachargreater}{\kern0pt}{\isacharcomma}{\kern0pt}\ l{\isachargreater}{\kern0pt}\ {\isasymin}\ Fn{\isacharunderscore}{\kern0pt}perm{\isacharprime}{\kern0pt}{\isacharparenleft}{\kern0pt}f{\isacharparenright}{\kern0pt}{\isacharbackquote}{\kern0pt}p{\isacharprime}{\kern0pt}{\isachardoublequoteclose}\isanewline
\ \ \ \ \ \ \isacommand{apply}\isamarkupfalse%
{\isacharparenleft}{\kern0pt}subst\ Fn{\isacharunderscore}{\kern0pt}perm{\isacharprime}{\kern0pt}{\isacharunderscore}{\kern0pt}eq{\isacharparenright}{\kern0pt}\isanewline
\ \ \ \ \ \ \isacommand{using}\isamarkupfalse%
\ assms\ \isanewline
\ \ \ \ \ \ \isacommand{by}\isamarkupfalse%
\ auto\isanewline
\ \ \ \ \isacommand{then}\isamarkupfalse%
\ \isacommand{have}\isamarkupfalse%
\ {\isachardoublequoteopen}{\isacharless}{\kern0pt}{\isacharless}{\kern0pt}f{\isacharbackquote}{\kern0pt}n{\isacharcomma}{\kern0pt}\ m{\isachargreater}{\kern0pt}{\isacharcomma}{\kern0pt}\ l{\isachargreater}{\kern0pt}\ {\isasymin}\ Fn{\isacharunderscore}{\kern0pt}perm{\isacharprime}{\kern0pt}{\isacharparenleft}{\kern0pt}f{\isacharparenright}{\kern0pt}{\isacharbackquote}{\kern0pt}p{\isachardoublequoteclose}\ \isanewline
\ \ \ \ \ \ \isacommand{using}\isamarkupfalse%
\ subsetH\isanewline
\ \ \ \ \ \ \isacommand{by}\isamarkupfalse%
\ auto\isanewline
\ \ \ \ \isacommand{then}\isamarkupfalse%
\ \isacommand{have}\isamarkupfalse%
\ {\isachardoublequoteopen}{\isacharless}{\kern0pt}{\isacharless}{\kern0pt}f{\isacharbackquote}{\kern0pt}n{\isacharcomma}{\kern0pt}\ m{\isachargreater}{\kern0pt}{\isacharcomma}{\kern0pt}\ l{\isachargreater}{\kern0pt}\ {\isasymin}\ Fn{\isacharunderscore}{\kern0pt}perm{\isacharparenleft}{\kern0pt}f{\isacharcomma}{\kern0pt}\ p{\isacharparenright}{\kern0pt}{\isachardoublequoteclose}\isanewline
\ \ \ \ \ \ \isacommand{using}\isamarkupfalse%
\ Fn{\isacharunderscore}{\kern0pt}perm{\isacharprime}{\kern0pt}{\isacharunderscore}{\kern0pt}eq\ assms\isanewline
\ \ \ \ \ \ \isacommand{by}\isamarkupfalse%
\ auto\isanewline
\ \ \ \ \isacommand{then}\isamarkupfalse%
\ \isacommand{have}\isamarkupfalse%
\ {\isachardoublequoteopen}{\isasymexists}n{\isacharprime}{\kern0pt}\ {\isasymin}\ nat{\isachardot}{\kern0pt}\ {\isasymexists}m{\isacharprime}{\kern0pt}\ {\isasymin}\ nat{\isachardot}{\kern0pt}\ {\isasymexists}l{\isacharprime}{\kern0pt}\ {\isasymin}\ {\isadigit{2}}{\isachardot}{\kern0pt}\ {\isacharless}{\kern0pt}{\isacharless}{\kern0pt}n{\isacharprime}{\kern0pt}{\isacharcomma}{\kern0pt}\ m{\isacharprime}{\kern0pt}{\isachargreater}{\kern0pt}{\isacharcomma}{\kern0pt}\ l{\isacharprime}{\kern0pt}{\isachargreater}{\kern0pt}\ {\isasymin}\ p\ {\isasymand}\ {\isacharless}{\kern0pt}{\isacharless}{\kern0pt}f{\isacharbackquote}{\kern0pt}n{\isacharcomma}{\kern0pt}\ m{\isachargreater}{\kern0pt}{\isacharcomma}{\kern0pt}\ l{\isachargreater}{\kern0pt}\ {\isacharequal}{\kern0pt}\ {\isacharless}{\kern0pt}{\isacharless}{\kern0pt}f{\isacharbackquote}{\kern0pt}n{\isacharprime}{\kern0pt}{\isacharcomma}{\kern0pt}\ m{\isacharprime}{\kern0pt}{\isachargreater}{\kern0pt}{\isacharcomma}{\kern0pt}\ l{\isacharprime}{\kern0pt}{\isachargreater}{\kern0pt}{\isachardoublequoteclose}\ \isanewline
\ \ \ \ \ \ \isacommand{apply}\isamarkupfalse%
{\isacharparenleft}{\kern0pt}rule{\isacharunderscore}{\kern0pt}tac\ Fn{\isacharunderscore}{\kern0pt}permE{\isacharparenright}{\kern0pt}\isanewline
\ \ \ \ \ \ \isacommand{using}\isamarkupfalse%
\ assms\isanewline
\ \ \ \ \ \ \isacommand{by}\isamarkupfalse%
\ auto\isanewline
\ \ \ \ \isacommand{then}\isamarkupfalse%
\ \isacommand{have}\isamarkupfalse%
\ {\isachardoublequoteopen}{\isacharless}{\kern0pt}{\isacharless}{\kern0pt}n{\isacharcomma}{\kern0pt}\ m{\isachargreater}{\kern0pt}{\isacharcomma}{\kern0pt}\ l{\isachargreater}{\kern0pt}\ {\isasymin}\ p{\isachardoublequoteclose}\ \isanewline
\ \ \ \ \ \ \isacommand{using}\isamarkupfalse%
\ injE\ H\ \isanewline
\ \ \ \ \ \ \isacommand{by}\isamarkupfalse%
\ auto\isanewline
\ \ \ \ \isacommand{then}\isamarkupfalse%
\ \isacommand{show}\isamarkupfalse%
\ {\isachardoublequoteopen}v\ {\isasymin}\ p{\isachardoublequoteclose}\ \isanewline
\ \ \ \ \ \ \isacommand{using}\isamarkupfalse%
\ H\ \isacommand{by}\isamarkupfalse%
\ auto\isanewline
\ \ \isacommand{qed}\isamarkupfalse%
\isanewline
\ \ \isacommand{then}\isamarkupfalse%
\ \isacommand{show}\isamarkupfalse%
\ {\isacharquery}{\kern0pt}thesis\ \isanewline
\ \ \ \ \isacommand{unfolding}\isamarkupfalse%
\ Fn{\isacharunderscore}{\kern0pt}leq{\isacharunderscore}{\kern0pt}def\ \isanewline
\ \ \ \ \isacommand{using}\isamarkupfalse%
\ assms\isanewline
\ \ \ \ \isacommand{by}\isamarkupfalse%
\ auto\isanewline
\isacommand{qed}\isamarkupfalse%
%
\endisatagproof
{\isafoldproof}%
%
\isadelimproof
\isanewline
%
\endisadelimproof
\isanewline
\isacommand{lemma}\isamarkupfalse%
\ Fn{\isacharunderscore}{\kern0pt}perm{\isacharprime}{\kern0pt}{\isacharunderscore}{\kern0pt}is{\isacharunderscore}{\kern0pt}P{\isacharunderscore}{\kern0pt}auto\ {\isacharcolon}{\kern0pt}\ \isanewline
\ \ \isakeyword{fixes}\ f\ \isanewline
\ \ \isakeyword{assumes}\ {\isachardoublequoteopen}f\ {\isasymin}\ nat{\isacharunderscore}{\kern0pt}perms{\isachardoublequoteclose}\ \isanewline
\ \ \isakeyword{shows}\ {\isachardoublequoteopen}forcing{\isacharunderscore}{\kern0pt}data{\isacharunderscore}{\kern0pt}partial{\isachardot}{\kern0pt}is{\isacharunderscore}{\kern0pt}P{\isacharunderscore}{\kern0pt}auto{\isacharparenleft}{\kern0pt}Fn{\isacharcomma}{\kern0pt}\ Fn{\isacharunderscore}{\kern0pt}leq{\isacharcomma}{\kern0pt}\ M{\isacharcomma}{\kern0pt}\ Fn{\isacharunderscore}{\kern0pt}perm{\isacharprime}{\kern0pt}{\isacharparenleft}{\kern0pt}f{\isacharparenright}{\kern0pt}{\isacharparenright}{\kern0pt}{\isachardoublequoteclose}\isanewline
%
\isadelimproof
\isanewline
\ \ %
\endisadelimproof
%
\isatagproof
\isacommand{apply}\isamarkupfalse%
{\isacharparenleft}{\kern0pt}subst\ forcing{\isacharunderscore}{\kern0pt}data{\isacharunderscore}{\kern0pt}partial{\isachardot}{\kern0pt}is{\isacharunderscore}{\kern0pt}P{\isacharunderscore}{\kern0pt}auto{\isacharunderscore}{\kern0pt}def{\isacharparenright}{\kern0pt}\isanewline
\ \ \ \isacommand{apply}\isamarkupfalse%
{\isacharparenleft}{\kern0pt}rule\ Fn{\isacharunderscore}{\kern0pt}forcing{\isacharunderscore}{\kern0pt}data{\isacharunderscore}{\kern0pt}partial{\isacharparenright}{\kern0pt}\isanewline
\ \ \isacommand{apply}\isamarkupfalse%
{\isacharparenleft}{\kern0pt}rule\ conjI{\isacharcomma}{\kern0pt}\ rule\ Fn{\isacharunderscore}{\kern0pt}perm{\isacharprime}{\kern0pt}{\isacharunderscore}{\kern0pt}in{\isacharunderscore}{\kern0pt}M{\isacharcomma}{\kern0pt}\ simp\ add{\isacharcolon}{\kern0pt}assms{\isacharcomma}{\kern0pt}\ rule\ conjI{\isacharcomma}{\kern0pt}\ rule\ Fn{\isacharunderscore}{\kern0pt}perm{\isacharprime}{\kern0pt}{\isacharunderscore}{\kern0pt}bij{\isacharcomma}{\kern0pt}\ simp\ add{\isacharcolon}{\kern0pt}assms{\isacharparenright}{\kern0pt}\isanewline
\ \ \isacommand{apply}\isamarkupfalse%
{\isacharparenleft}{\kern0pt}rule\ ballI{\isacharparenright}{\kern0pt}{\isacharplus}{\kern0pt}\isanewline
\ \ \isacommand{apply}\isamarkupfalse%
{\isacharparenleft}{\kern0pt}rule\ iffI{\isacharparenright}{\kern0pt}\isanewline
\ \ \ \isacommand{apply}\isamarkupfalse%
{\isacharparenleft}{\kern0pt}rule\ Fn{\isacharunderscore}{\kern0pt}perm{\isacharprime}{\kern0pt}{\isacharunderscore}{\kern0pt}preserves{\isacharunderscore}{\kern0pt}order{\isacharparenright}{\kern0pt}\isanewline
\ \ \isacommand{using}\isamarkupfalse%
\ assms\isanewline
\ \ \ \ \ \ \isacommand{apply}\isamarkupfalse%
\ auto{\isacharbrackleft}{\kern0pt}{\isadigit{4}}{\isacharbrackright}{\kern0pt}\isanewline
\ \ \ \ \isacommand{apply}\isamarkupfalse%
{\isacharparenleft}{\kern0pt}rule\ Fn{\isacharunderscore}{\kern0pt}perm{\isacharprime}{\kern0pt}{\isacharunderscore}{\kern0pt}preserves{\isacharunderscore}{\kern0pt}order{\isacharprime}{\kern0pt}{\isacharparenright}{\kern0pt}\isanewline
\ \ \isacommand{using}\isamarkupfalse%
\ assms\isanewline
\ \ \isacommand{by}\isamarkupfalse%
\ auto%
\endisatagproof
{\isafoldproof}%
%
\isadelimproof
\isanewline
%
\endisadelimproof
\isanewline
\isanewline
\isacommand{end}\isamarkupfalse%
\isanewline
%
\isadelimtheory
%
\endisadelimtheory
%
\isatagtheory
\isacommand{end}\isamarkupfalse%
%
\endisatagtheory
{\isafoldtheory}%
%
\isadelimtheory
%
\endisadelimtheory
%
\end{isabellebody}%
\endinput
%:%file=~/source/repos/ZF-notAC/code/Fn_Perm_Automorphism.thy%:%
%:%10=1%:%
%:%11=1%:%
%:%12=2%:%
%:%13=3%:%
%:%14=4%:%
%:%19=4%:%
%:%22=5%:%
%:%23=5%:%
%:%24=6%:%
%:%25=7%:%
%:%26=7%:%
%:%27=8%:%
%:%28=9%:%
%:%29=10%:%
%:%32=11%:%
%:%36=11%:%
%:%37=11%:%
%:%38=12%:%
%:%39=12%:%
%:%40=13%:%
%:%41=13%:%
%:%46=13%:%
%:%49=14%:%
%:%50=15%:%
%:%51=15%:%
%:%52=16%:%
%:%53=17%:%
%:%54=18%:%
%:%57=19%:%
%:%61=19%:%
%:%62=19%:%
%:%63=20%:%
%:%64=20%:%
%:%65=21%:%
%:%66=21%:%
%:%71=21%:%
%:%74=22%:%
%:%75=23%:%
%:%76=23%:%
%:%77=24%:%
%:%78=25%:%
%:%79=26%:%
%:%86=27%:%
%:%87=27%:%
%:%88=28%:%
%:%89=28%:%
%:%90=28%:%
%:%91=29%:%
%:%92=29%:%
%:%93=30%:%
%:%94=30%:%
%:%95=31%:%
%:%96=31%:%
%:%97=32%:%
%:%98=32%:%
%:%99=33%:%
%:%100=33%:%
%:%101=33%:%
%:%102=33%:%
%:%103=34%:%
%:%104=34%:%
%:%105=35%:%
%:%106=35%:%
%:%107=36%:%
%:%108=36%:%
%:%109=37%:%
%:%110=37%:%
%:%111=38%:%
%:%112=38%:%
%:%113=38%:%
%:%114=39%:%
%:%115=39%:%
%:%116=40%:%
%:%117=40%:%
%:%118=41%:%
%:%124=41%:%
%:%127=42%:%
%:%128=43%:%
%:%129=43%:%
%:%130=44%:%
%:%131=45%:%
%:%132=46%:%
%:%135=47%:%
%:%139=47%:%
%:%140=47%:%
%:%141=48%:%
%:%142=48%:%
%:%143=49%:%
%:%144=49%:%
%:%145=50%:%
%:%146=50%:%
%:%147=51%:%
%:%148=51%:%
%:%149=52%:%
%:%150=52%:%
%:%151=53%:%
%:%152=53%:%
%:%153=54%:%
%:%154=54%:%
%:%155=55%:%
%:%156=55%:%
%:%157=55%:%
%:%158=55%:%
%:%159=56%:%
%:%160=57%:%
%:%161=57%:%
%:%162=58%:%
%:%163=58%:%
%:%164=59%:%
%:%165=59%:%
%:%166=60%:%
%:%167=60%:%
%:%168=61%:%
%:%169=61%:%
%:%170=61%:%
%:%171=61%:%
%:%172=62%:%
%:%173=63%:%
%:%174=63%:%
%:%175=63%:%
%:%176=63%:%
%:%177=64%:%
%:%178=65%:%
%:%179=65%:%
%:%180=66%:%
%:%181=66%:%
%:%182=67%:%
%:%183=67%:%
%:%184=68%:%
%:%185=68%:%
%:%186=68%:%
%:%187=68%:%
%:%188=68%:%
%:%189=69%:%
%:%190=70%:%
%:%191=70%:%
%:%192=70%:%
%:%193=70%:%
%:%194=71%:%
%:%195=71%:%
%:%196=71%:%
%:%197=71%:%
%:%198=71%:%
%:%199=72%:%
%:%200=72%:%
%:%201=72%:%
%:%202=72%:%
%:%203=72%:%
%:%204=73%:%
%:%210=73%:%
%:%213=74%:%
%:%214=75%:%
%:%215=75%:%
%:%216=76%:%
%:%217=77%:%
%:%218=78%:%
%:%225=79%:%
%:%226=79%:%
%:%227=80%:%
%:%228=80%:%
%:%229=80%:%
%:%230=81%:%
%:%231=81%:%
%:%232=81%:%
%:%233=81%:%
%:%234=82%:%
%:%235=82%:%
%:%236=82%:%
%:%237=83%:%
%:%238=83%:%
%:%239=84%:%
%:%240=84%:%
%:%241=85%:%
%:%242=85%:%
%:%243=86%:%
%:%244=86%:%
%:%245=86%:%
%:%246=86%:%
%:%247=87%:%
%:%248=87%:%
%:%249=87%:%
%:%250=87%:%
%:%251=88%:%
%:%252=88%:%
%:%253=89%:%
%:%254=89%:%
%:%255=89%:%
%:%256=90%:%
%:%257=90%:%
%:%258=90%:%
%:%259=91%:%
%:%260=91%:%
%:%261=92%:%
%:%262=92%:%
%:%263=93%:%
%:%264=93%:%
%:%265=94%:%
%:%266=94%:%
%:%267=95%:%
%:%268=95%:%
%:%269=95%:%
%:%270=95%:%
%:%271=96%:%
%:%272=96%:%
%:%273=96%:%
%:%274=96%:%
%:%275=97%:%
%:%276=97%:%
%:%277=97%:%
%:%278=98%:%
%:%279=98%:%
%:%280=99%:%
%:%281=99%:%
%:%282=100%:%
%:%283=100%:%
%:%284=100%:%
%:%285=100%:%
%:%286=100%:%
%:%287=101%:%
%:%293=101%:%
%:%296=102%:%
%:%297=103%:%
%:%298=103%:%
%:%299=104%:%
%:%300=105%:%
%:%301=106%:%
%:%308=107%:%
%:%309=107%:%
%:%310=108%:%
%:%311=109%:%
%:%312=109%:%
%:%313=110%:%
%:%314=110%:%
%:%315=111%:%
%:%316=111%:%
%:%317=111%:%
%:%318=112%:%
%:%319=112%:%
%:%320=112%:%
%:%321=112%:%
%:%322=113%:%
%:%323=113%:%
%:%324=113%:%
%:%325=114%:%
%:%326=114%:%
%:%327=115%:%
%:%328=115%:%
%:%329=116%:%
%:%330=116%:%
%:%331=117%:%
%:%332=117%:%
%:%333=117%:%
%:%334=117%:%
%:%335=118%:%
%:%336=118%:%
%:%337=118%:%
%:%338=118%:%
%:%339=119%:%
%:%340=119%:%
%:%341=120%:%
%:%342=120%:%
%:%343=121%:%
%:%344=121%:%
%:%345=122%:%
%:%346=122%:%
%:%347=123%:%
%:%348=123%:%
%:%349=123%:%
%:%350=123%:%
%:%351=123%:%
%:%352=124%:%
%:%353=124%:%
%:%354=125%:%
%:%355=126%:%
%:%356=126%:%
%:%357=127%:%
%:%358=127%:%
%:%359=128%:%
%:%360=128%:%
%:%361=128%:%
%:%362=128%:%
%:%363=129%:%
%:%364=129%:%
%:%365=129%:%
%:%366=129%:%
%:%367=129%:%
%:%368=130%:%
%:%369=131%:%
%:%370=131%:%
%:%371=132%:%
%:%372=133%:%
%:%373=133%:%
%:%374=134%:%
%:%375=134%:%
%:%376=135%:%
%:%377=136%:%
%:%378=136%:%
%:%384=142%:%
%:%385=143%:%
%:%386=143%:%
%:%387=144%:%
%:%388=145%:%
%:%389=145%:%
%:%390=146%:%
%:%391=146%:%
%:%392=147%:%
%:%393=147%:%
%:%394=148%:%
%:%395=148%:%
%:%396=149%:%
%:%397=149%:%
%:%398=150%:%
%:%399=150%:%
%:%400=151%:%
%:%401=151%:%
%:%402=152%:%
%:%403=152%:%
%:%404=153%:%
%:%405=153%:%
%:%406=154%:%
%:%407=154%:%
%:%408=155%:%
%:%409=155%:%
%:%410=156%:%
%:%411=157%:%
%:%412=157%:%
%:%414=159%:%
%:%415=160%:%
%:%416=161%:%
%:%417=161%:%
%:%418=162%:%
%:%419=162%:%
%:%420=163%:%
%:%421=163%:%
%:%422=164%:%
%:%423=164%:%
%:%424=165%:%
%:%425=165%:%
%:%426=166%:%
%:%427=166%:%
%:%428=167%:%
%:%429=167%:%
%:%430=168%:%
%:%431=168%:%
%:%432=169%:%
%:%433=169%:%
%:%434=169%:%
%:%435=170%:%
%:%436=170%:%
%:%437=171%:%
%:%438=171%:%
%:%439=172%:%
%:%440=172%:%
%:%441=173%:%
%:%442=173%:%
%:%443=174%:%
%:%444=174%:%
%:%445=175%:%
%:%446=175%:%
%:%447=176%:%
%:%448=176%:%
%:%449=177%:%
%:%450=177%:%
%:%451=178%:%
%:%452=178%:%
%:%453=179%:%
%:%454=179%:%
%:%455=180%:%
%:%456=180%:%
%:%457=181%:%
%:%458=181%:%
%:%459=182%:%
%:%460=182%:%
%:%461=182%:%
%:%462=183%:%
%:%463=183%:%
%:%464=184%:%
%:%465=184%:%
%:%466=185%:%
%:%467=185%:%
%:%468=186%:%
%:%469=186%:%
%:%470=186%:%
%:%471=187%:%
%:%472=187%:%
%:%473=187%:%
%:%474=188%:%
%:%475=188%:%
%:%476=189%:%
%:%477=189%:%
%:%478=190%:%
%:%479=190%:%
%:%480=191%:%
%:%481=191%:%
%:%482=192%:%
%:%483=192%:%
%:%484=193%:%
%:%485=193%:%
%:%486=194%:%
%:%487=194%:%
%:%488=195%:%
%:%489=195%:%
%:%490=196%:%
%:%491=196%:%
%:%492=196%:%
%:%493=197%:%
%:%494=197%:%
%:%495=198%:%
%:%496=198%:%
%:%497=199%:%
%:%498=199%:%
%:%499=200%:%
%:%500=200%:%
%:%501=200%:%
%:%502=201%:%
%:%503=201%:%
%:%504=202%:%
%:%505=202%:%
%:%506=203%:%
%:%507=203%:%
%:%508=204%:%
%:%509=204%:%
%:%510=205%:%
%:%511=205%:%
%:%512=206%:%
%:%513=206%:%
%:%514=207%:%
%:%515=207%:%
%:%516=208%:%
%:%517=208%:%
%:%518=209%:%
%:%519=209%:%
%:%520=210%:%
%:%521=210%:%
%:%522=210%:%
%:%523=210%:%
%:%524=210%:%
%:%525=211%:%
%:%526=211%:%
%:%527=212%:%
%:%528=213%:%
%:%529=213%:%
%:%530=214%:%
%:%531=214%:%
%:%532=215%:%
%:%533=215%:%
%:%534=216%:%
%:%535=216%:%
%:%536=217%:%
%:%537=217%:%
%:%538=218%:%
%:%539=218%:%
%:%540=219%:%
%:%541=219%:%
%:%542=220%:%
%:%543=220%:%
%:%544=221%:%
%:%545=221%:%
%:%546=222%:%
%:%547=222%:%
%:%548=223%:%
%:%549=223%:%
%:%550=224%:%
%:%551=224%:%
%:%552=225%:%
%:%553=225%:%
%:%554=226%:%
%:%555=226%:%
%:%556=227%:%
%:%557=227%:%
%:%558=228%:%
%:%559=228%:%
%:%560=229%:%
%:%561=229%:%
%:%562=230%:%
%:%563=230%:%
%:%564=231%:%
%:%565=231%:%
%:%566=232%:%
%:%567=232%:%
%:%568=233%:%
%:%569=233%:%
%:%570=234%:%
%:%571=234%:%
%:%572=235%:%
%:%573=235%:%
%:%574=236%:%
%:%575=237%:%
%:%576=237%:%
%:%577=238%:%
%:%578=238%:%
%:%579=239%:%
%:%580=239%:%
%:%581=240%:%
%:%582=240%:%
%:%583=240%:%
%:%584=240%:%
%:%585=240%:%
%:%586=241%:%
%:%587=241%:%
%:%588=241%:%
%:%589=241%:%
%:%590=241%:%
%:%591=242%:%
%:%592=242%:%
%:%593=243%:%
%:%594=243%:%
%:%595=244%:%
%:%596=244%:%
%:%597=245%:%
%:%598=245%:%
%:%599=246%:%
%:%600=246%:%
%:%601=247%:%
%:%602=247%:%
%:%603=248%:%
%:%604=248%:%
%:%605=248%:%
%:%606=249%:%
%:%607=249%:%
%:%608=250%:%
%:%609=250%:%
%:%610=251%:%
%:%611=251%:%
%:%612=252%:%
%:%613=252%:%
%:%614=253%:%
%:%615=253%:%
%:%616=253%:%
%:%617=253%:%
%:%618=254%:%
%:%619=254%:%
%:%620=255%:%
%:%621=255%:%
%:%622=256%:%
%:%623=256%:%
%:%624=257%:%
%:%625=257%:%
%:%626=257%:%
%:%627=258%:%
%:%628=258%:%
%:%629=258%:%
%:%630=259%:%
%:%631=259%:%
%:%632=260%:%
%:%633=260%:%
%:%634=261%:%
%:%635=261%:%
%:%636=262%:%
%:%637=262%:%
%:%638=263%:%
%:%639=263%:%
%:%640=263%:%
%:%641=264%:%
%:%642=264%:%
%:%643=265%:%
%:%644=265%:%
%:%645=266%:%
%:%646=266%:%
%:%647=267%:%
%:%648=267%:%
%:%649=267%:%
%:%650=267%:%
%:%651=267%:%
%:%652=268%:%
%:%653=268%:%
%:%654=269%:%
%:%655=269%:%
%:%656=270%:%
%:%657=271%:%
%:%658=271%:%
%:%659=272%:%
%:%660=272%:%
%:%661=273%:%
%:%662=273%:%
%:%663=274%:%
%:%664=274%:%
%:%665=275%:%
%:%666=275%:%
%:%667=275%:%
%:%668=275%:%
%:%669=276%:%
%:%670=276%:%
%:%671=276%:%
%:%672=276%:%
%:%673=277%:%
%:%674=278%:%
%:%675=278%:%
%:%676=279%:%
%:%677=279%:%
%:%678=280%:%
%:%679=280%:%
%:%680=281%:%
%:%681=281%:%
%:%682=282%:%
%:%683=282%:%
%:%684=283%:%
%:%685=283%:%
%:%686=284%:%
%:%687=284%:%
%:%688=285%:%
%:%689=285%:%
%:%690=286%:%
%:%691=287%:%
%:%692=287%:%
%:%693=288%:%
%:%694=288%:%
%:%695=288%:%
%:%696=289%:%
%:%697=289%:%
%:%698=290%:%
%:%699=290%:%
%:%700=291%:%
%:%701=291%:%
%:%702=292%:%
%:%703=292%:%
%:%704=293%:%
%:%705=293%:%
%:%706=293%:%
%:%707=294%:%
%:%708=294%:%
%:%709=295%:%
%:%710=295%:%
%:%711=296%:%
%:%712=296%:%
%:%713=297%:%
%:%714=298%:%
%:%715=298%:%
%:%716=299%:%
%:%717=299%:%
%:%718=300%:%
%:%719=300%:%
%:%720=301%:%
%:%721=301%:%
%:%722=302%:%
%:%723=302%:%
%:%724=302%:%
%:%725=302%:%
%:%726=302%:%
%:%727=303%:%
%:%728=303%:%
%:%729=304%:%
%:%730=305%:%
%:%731=305%:%
%:%732=305%:%
%:%733=306%:%
%:%734=306%:%
%:%735=307%:%
%:%736=307%:%
%:%737=308%:%
%:%738=308%:%
%:%739=309%:%
%:%740=309%:%
%:%741=310%:%
%:%742=310%:%
%:%743=311%:%
%:%744=311%:%
%:%745=312%:%
%:%746=312%:%
%:%747=313%:%
%:%748=314%:%
%:%749=314%:%
%:%750=315%:%
%:%751=315%:%
%:%752=316%:%
%:%753=316%:%
%:%754=316%:%
%:%755=317%:%
%:%756=317%:%
%:%757=317%:%
%:%758=317%:%
%:%759=317%:%
%:%760=318%:%
%:%761=318%:%
%:%762=318%:%
%:%763=319%:%
%:%764=319%:%
%:%765=320%:%
%:%766=320%:%
%:%767=321%:%
%:%768=321%:%
%:%769=322%:%
%:%770=322%:%
%:%771=322%:%
%:%772=322%:%
%:%773=323%:%
%:%774=323%:%
%:%775=324%:%
%:%776=325%:%
%:%777=325%:%
%:%778=325%:%
%:%779=326%:%
%:%780=326%:%
%:%781=327%:%
%:%782=327%:%
%:%783=328%:%
%:%784=328%:%
%:%785=329%:%
%:%791=329%:%
%:%794=330%:%
%:%795=331%:%
%:%796=331%:%
%:%797=332%:%
%:%798=333%:%
%:%799=334%:%
%:%806=335%:%
%:%807=335%:%
%:%808=336%:%
%:%809=336%:%
%:%810=337%:%
%:%811=337%:%
%:%812=338%:%
%:%813=338%:%
%:%814=339%:%
%:%815=339%:%
%:%816=340%:%
%:%817=340%:%
%:%818=341%:%
%:%819=341%:%
%:%820=342%:%
%:%821=342%:%
%:%822=343%:%
%:%823=343%:%
%:%824=344%:%
%:%825=344%:%
%:%826=345%:%
%:%827=345%:%
%:%828=346%:%
%:%829=346%:%
%:%830=347%:%
%:%831=347%:%
%:%832=348%:%
%:%833=348%:%
%:%834=349%:%
%:%835=349%:%
%:%836=350%:%
%:%837=350%:%
%:%838=351%:%
%:%839=351%:%
%:%840=351%:%
%:%841=352%:%
%:%842=352%:%
%:%843=353%:%
%:%844=353%:%
%:%845=354%:%
%:%846=354%:%
%:%847=355%:%
%:%848=355%:%
%:%849=356%:%
%:%850=356%:%
%:%851=357%:%
%:%852=357%:%
%:%853=358%:%
%:%854=358%:%
%:%855=358%:%
%:%856=359%:%
%:%857=359%:%
%:%858=360%:%
%:%859=360%:%
%:%860=361%:%
%:%861=361%:%
%:%862=362%:%
%:%863=362%:%
%:%864=363%:%
%:%865=363%:%
%:%866=364%:%
%:%867=364%:%
%:%868=365%:%
%:%869=365%:%
%:%870=366%:%
%:%871=366%:%
%:%872=367%:%
%:%873=367%:%
%:%874=368%:%
%:%875=368%:%
%:%876=369%:%
%:%877=369%:%
%:%878=370%:%
%:%879=370%:%
%:%880=371%:%
%:%881=371%:%
%:%882=372%:%
%:%883=372%:%
%:%884=373%:%
%:%885=373%:%
%:%886=374%:%
%:%887=374%:%
%:%888=374%:%
%:%889=374%:%
%:%890=375%:%
%:%896=375%:%
%:%899=376%:%
%:%900=377%:%
%:%901=377%:%
%:%902=378%:%
%:%903=379%:%
%:%904=380%:%
%:%907=381%:%
%:%911=381%:%
%:%912=381%:%
%:%913=382%:%
%:%914=382%:%
%:%915=383%:%
%:%916=383%:%
%:%917=384%:%
%:%918=384%:%
%:%923=384%:%
%:%926=385%:%
%:%927=386%:%
%:%928=386%:%
%:%929=387%:%
%:%930=388%:%
%:%931=389%:%
%:%934=390%:%
%:%935=391%:%
%:%939=391%:%
%:%940=391%:%
%:%941=392%:%
%:%942=392%:%
%:%947=392%:%
%:%950=393%:%
%:%951=394%:%
%:%952=394%:%
%:%953=395%:%
%:%954=396%:%
%:%955=397%:%
%:%958=398%:%
%:%962=398%:%
%:%963=398%:%
%:%964=399%:%
%:%965=399%:%
%:%966=400%:%
%:%967=400%:%
%:%972=400%:%
%:%975=401%:%
%:%976=402%:%
%:%977=402%:%
%:%978=403%:%
%:%979=404%:%
%:%980=405%:%
%:%983=406%:%
%:%987=406%:%
%:%988=406%:%
%:%989=407%:%
%:%990=407%:%
%:%995=407%:%
%:%998=408%:%
%:%999=409%:%
%:%1000=409%:%
%:%1001=410%:%
%:%1002=411%:%
%:%1003=412%:%
%:%1006=413%:%
%:%1010=413%:%
%:%1011=413%:%
%:%1012=414%:%
%:%1013=414%:%
%:%1014=415%:%
%:%1015=415%:%
%:%1016=416%:%
%:%1017=416%:%
%:%1022=416%:%
%:%1025=417%:%
%:%1026=418%:%
%:%1027=418%:%
%:%1028=419%:%
%:%1031=420%:%
%:%1035=420%:%
%:%1036=420%:%
%:%1037=421%:%
%:%1038=421%:%
%:%1039=422%:%
%:%1040=422%:%
%:%1041=423%:%
%:%1042=423%:%
%:%1043=424%:%
%:%1044=424%:%
%:%1045=425%:%
%:%1046=425%:%
%:%1047=426%:%
%:%1048=426%:%
%:%1049=427%:%
%:%1050=427%:%
%:%1051=428%:%
%:%1052=428%:%
%:%1053=429%:%
%:%1054=429%:%
%:%1055=430%:%
%:%1056=430%:%
%:%1057=431%:%
%:%1058=431%:%
%:%1059=432%:%
%:%1060=432%:%
%:%1061=433%:%
%:%1062=433%:%
%:%1067=433%:%
%:%1070=434%:%
%:%1071=435%:%
%:%1072=435%:%
%:%1073=436%:%
%:%1074=437%:%
%:%1075=438%:%
%:%1078=439%:%
%:%1082=439%:%
%:%1083=439%:%
%:%1084=440%:%
%:%1085=440%:%
%:%1086=441%:%
%:%1087=441%:%
%:%1088=442%:%
%:%1089=442%:%
%:%1090=443%:%
%:%1091=443%:%
%:%1092=444%:%
%:%1093=444%:%
%:%1094=445%:%
%:%1095=445%:%
%:%1096=446%:%
%:%1097=446%:%
%:%1098=447%:%
%:%1099=447%:%
%:%1100=448%:%
%:%1101=448%:%
%:%1102=449%:%
%:%1103=449%:%
%:%1104=450%:%
%:%1105=450%:%
%:%1106=451%:%
%:%1107=451%:%
%:%1108=452%:%
%:%1109=452%:%
%:%1110=453%:%
%:%1111=453%:%
%:%1112=454%:%
%:%1113=454%:%
%:%1114=455%:%
%:%1115=455%:%
%:%1116=456%:%
%:%1117=456%:%
%:%1118=457%:%
%:%1119=457%:%
%:%1120=458%:%
%:%1121=458%:%
%:%1122=459%:%
%:%1123=459%:%
%:%1124=460%:%
%:%1125=460%:%
%:%1126=461%:%
%:%1127=461%:%
%:%1128=462%:%
%:%1129=462%:%
%:%1130=463%:%
%:%1131=463%:%
%:%1136=463%:%
%:%1139=464%:%
%:%1140=465%:%
%:%1141=465%:%
%:%1142=466%:%
%:%1143=467%:%
%:%1144=468%:%
%:%1147=469%:%
%:%1151=469%:%
%:%1152=469%:%
%:%1153=470%:%
%:%1154=470%:%
%:%1155=471%:%
%:%1156=471%:%
%:%1157=472%:%
%:%1158=472%:%
%:%1159=473%:%
%:%1160=473%:%
%:%1161=474%:%
%:%1162=474%:%
%:%1163=475%:%
%:%1164=475%:%
%:%1165=476%:%
%:%1166=476%:%
%:%1167=477%:%
%:%1168=477%:%
%:%1169=478%:%
%:%1170=478%:%
%:%1171=479%:%
%:%1172=479%:%
%:%1173=480%:%
%:%1174=480%:%
%:%1175=481%:%
%:%1176=481%:%
%:%1177=482%:%
%:%1183=482%:%
%:%1186=483%:%
%:%1187=484%:%
%:%1188=484%:%
%:%1189=485%:%
%:%1190=486%:%
%:%1191=487%:%
%:%1198=488%:%
%:%1199=488%:%
%:%1200=489%:%
%:%1201=489%:%
%:%1202=490%:%
%:%1203=491%:%
%:%1204=491%:%
%:%1205=492%:%
%:%1206=492%:%
%:%1207=493%:%
%:%1208=493%:%
%:%1209=494%:%
%:%1210=494%:%
%:%1211=494%:%
%:%1212=494%:%
%:%1213=494%:%
%:%1214=495%:%
%:%1215=496%:%
%:%1216=496%:%
%:%1217=497%:%
%:%1218=497%:%
%:%1219=498%:%
%:%1220=498%:%
%:%1221=499%:%
%:%1222=499%:%
%:%1223=500%:%
%:%1224=500%:%
%:%1225=501%:%
%:%1226=501%:%
%:%1227=502%:%
%:%1228=502%:%
%:%1229=503%:%
%:%1230=503%:%
%:%1231=504%:%
%:%1232=504%:%
%:%1233=505%:%
%:%1234=505%:%
%:%1235=506%:%
%:%1236=506%:%
%:%1237=507%:%
%:%1238=507%:%
%:%1239=508%:%
%:%1240=508%:%
%:%1241=509%:%
%:%1242=509%:%
%:%1243=510%:%
%:%1244=510%:%
%:%1245=511%:%
%:%1246=512%:%
%:%1247=512%:%
%:%1248=513%:%
%:%1249=513%:%
%:%1250=514%:%
%:%1251=514%:%
%:%1252=515%:%
%:%1253=515%:%
%:%1254=516%:%
%:%1255=516%:%
%:%1256=517%:%
%:%1257=517%:%
%:%1258=518%:%
%:%1259=518%:%
%:%1260=519%:%
%:%1261=519%:%
%:%1262=520%:%
%:%1263=520%:%
%:%1264=521%:%
%:%1265=521%:%
%:%1266=522%:%
%:%1267=522%:%
%:%1268=523%:%
%:%1269=523%:%
%:%1270=524%:%
%:%1271=524%:%
%:%1272=525%:%
%:%1273=525%:%
%:%1274=526%:%
%:%1275=526%:%
%:%1276=527%:%
%:%1277=527%:%
%:%1278=528%:%
%:%1279=528%:%
%:%1280=529%:%
%:%1281=529%:%
%:%1282=530%:%
%:%1283=530%:%
%:%1284=531%:%
%:%1285=531%:%
%:%1286=532%:%
%:%1287=532%:%
%:%1288=533%:%
%:%1289=533%:%
%:%1290=534%:%
%:%1291=534%:%
%:%1292=535%:%
%:%1293=535%:%
%:%1294=536%:%
%:%1295=536%:%
%:%1296=537%:%
%:%1297=537%:%
%:%1298=538%:%
%:%1299=538%:%
%:%1300=539%:%
%:%1301=539%:%
%:%1302=540%:%
%:%1303=540%:%
%:%1304=541%:%
%:%1310=541%:%
%:%1313=542%:%
%:%1314=543%:%
%:%1315=543%:%
%:%1316=544%:%
%:%1317=545%:%
%:%1318=546%:%
%:%1325=547%:%
%:%1326=547%:%
%:%1327=548%:%
%:%1328=548%:%
%:%1329=548%:%
%:%1330=548%:%
%:%1331=549%:%
%:%1332=550%:%
%:%1333=550%:%
%:%1334=551%:%
%:%1335=551%:%
%:%1336=552%:%
%:%1337=552%:%
%:%1338=553%:%
%:%1339=553%:%
%:%1340=554%:%
%:%1341=554%:%
%:%1342=554%:%
%:%1343=554%:%
%:%1344=554%:%
%:%1345=555%:%
%:%1346=555%:%
%:%1347=555%:%
%:%1348=556%:%
%:%1349=556%:%
%:%1350=557%:%
%:%1351=557%:%
%:%1352=558%:%
%:%1353=558%:%
%:%1354=559%:%
%:%1355=559%:%
%:%1356=559%:%
%:%1357=559%:%
%:%1358=560%:%
%:%1359=560%:%
%:%1360=560%:%
%:%1361=560%:%
%:%1362=560%:%
%:%1363=561%:%
%:%1364=561%:%
%:%1365=561%:%
%:%1366=562%:%
%:%1367=562%:%
%:%1368=563%:%
%:%1369=563%:%
%:%1370=564%:%
%:%1371=564%:%
%:%1372=564%:%
%:%1373=565%:%
%:%1374=565%:%
%:%1375=566%:%
%:%1376=566%:%
%:%1377=567%:%
%:%1378=567%:%
%:%1379=568%:%
%:%1380=568%:%
%:%1381=568%:%
%:%1382=568%:%
%:%1383=568%:%
%:%1384=569%:%
%:%1385=569%:%
%:%1386=570%:%
%:%1387=570%:%
%:%1388=570%:%
%:%1389=571%:%
%:%1390=571%:%
%:%1391=572%:%
%:%1392=572%:%
%:%1393=573%:%
%:%1394=573%:%
%:%1395=574%:%
%:%1396=574%:%
%:%1397=575%:%
%:%1398=575%:%
%:%1399=576%:%
%:%1400=576%:%
%:%1401=577%:%
%:%1407=577%:%
%:%1410=578%:%
%:%1411=579%:%
%:%1412=579%:%
%:%1413=580%:%
%:%1414=581%:%
%:%1415=582%:%
%:%1422=583%:%
%:%1423=583%:%
%:%1424=584%:%
%:%1425=584%:%
%:%1426=584%:%
%:%1427=584%:%
%:%1428=585%:%
%:%1429=585%:%
%:%1430=585%:%
%:%1431=586%:%
%:%1432=586%:%
%:%1433=587%:%
%:%1434=587%:%
%:%1435=588%:%
%:%1436=589%:%
%:%1437=589%:%
%:%1438=590%:%
%:%1439=590%:%
%:%1440=591%:%
%:%1441=591%:%
%:%1442=592%:%
%:%1443=592%:%
%:%1444=593%:%
%:%1445=593%:%
%:%1446=594%:%
%:%1447=594%:%
%:%1448=594%:%
%:%1449=595%:%
%:%1450=595%:%
%:%1451=595%:%
%:%1452=596%:%
%:%1453=596%:%
%:%1454=597%:%
%:%1455=597%:%
%:%1456=598%:%
%:%1457=598%:%
%:%1458=598%:%
%:%1459=599%:%
%:%1460=599%:%
%:%1461=600%:%
%:%1462=600%:%
%:%1463=601%:%
%:%1464=601%:%
%:%1465=601%:%
%:%1466=602%:%
%:%1467=602%:%
%:%1468=603%:%
%:%1469=603%:%
%:%1470=604%:%
%:%1471=604%:%
%:%1472=604%:%
%:%1473=605%:%
%:%1474=605%:%
%:%1475=606%:%
%:%1476=606%:%
%:%1477=607%:%
%:%1478=607%:%
%:%1479=608%:%
%:%1480=608%:%
%:%1481=608%:%
%:%1482=609%:%
%:%1483=609%:%
%:%1484=610%:%
%:%1485=610%:%
%:%1486=611%:%
%:%1487=611%:%
%:%1488=612%:%
%:%1489=612%:%
%:%1490=612%:%
%:%1491=613%:%
%:%1492=613%:%
%:%1493=614%:%
%:%1494=614%:%
%:%1495=615%:%
%:%1496=615%:%
%:%1497=615%:%
%:%1498=616%:%
%:%1499=616%:%
%:%1500=617%:%
%:%1501=617%:%
%:%1502=618%:%
%:%1503=618%:%
%:%1504=618%:%
%:%1505=619%:%
%:%1506=619%:%
%:%1507=620%:%
%:%1508=620%:%
%:%1509=621%:%
%:%1510=621%:%
%:%1511=622%:%
%:%1512=622%:%
%:%1513=622%:%
%:%1514=623%:%
%:%1515=623%:%
%:%1516=624%:%
%:%1517=624%:%
%:%1518=625%:%
%:%1519=625%:%
%:%1520=625%:%
%:%1521=626%:%
%:%1522=626%:%
%:%1523=626%:%
%:%1524=627%:%
%:%1525=627%:%
%:%1526=628%:%
%:%1527=628%:%
%:%1528=628%:%
%:%1529=629%:%
%:%1530=629%:%
%:%1531=630%:%
%:%1532=630%:%
%:%1533=631%:%
%:%1534=631%:%
%:%1535=632%:%
%:%1541=632%:%
%:%1544=633%:%
%:%1545=634%:%
%:%1546=634%:%
%:%1547=635%:%
%:%1548=636%:%
%:%1549=637%:%
%:%1552=638%:%
%:%1553=639%:%
%:%1557=639%:%
%:%1558=639%:%
%:%1559=640%:%
%:%1560=640%:%
%:%1561=641%:%
%:%1562=641%:%
%:%1563=642%:%
%:%1564=642%:%
%:%1565=643%:%
%:%1566=643%:%
%:%1567=644%:%
%:%1568=644%:%
%:%1569=645%:%
%:%1570=645%:%
%:%1571=646%:%
%:%1572=646%:%
%:%1573=647%:%
%:%1574=647%:%
%:%1575=648%:%
%:%1576=648%:%
%:%1577=649%:%
%:%1578=649%:%
%:%1583=649%:%
%:%1586=650%:%
%:%1587=651%:%
%:%1588=652%:%
%:%1589=652%:%
%:%1596=653%:%

%
\begin{isabellebody}%
\setisabellecontext{Fn{\isacharunderscore}{\kern0pt}Perm{\isacharunderscore}{\kern0pt}Filter}%
%
\isadelimtheory
%
\endisadelimtheory
%
\isatagtheory
\isacommand{theory}\isamarkupfalse%
\ Fn{\isacharunderscore}{\kern0pt}Perm{\isacharunderscore}{\kern0pt}Filter\isanewline
\ \ \isakeyword{imports}\ Fn{\isacharunderscore}{\kern0pt}Perm{\isacharunderscore}{\kern0pt}Automorphism\isanewline
\isakeyword{begin}%
\endisatagtheory
{\isafoldtheory}%
%
\isadelimtheory
\ \isanewline
%
\endisadelimtheory
\isanewline
\isacommand{context}\isamarkupfalse%
\ M{\isacharunderscore}{\kern0pt}ctm\ \isakeyword{begin}\ \isanewline
\isanewline
\isacommand{lemma}\isamarkupfalse%
\ Fn{\isacharunderscore}{\kern0pt}perms{\isacharunderscore}{\kern0pt}group\ {\isacharcolon}{\kern0pt}\ {\isachardoublequoteopen}forcing{\isacharunderscore}{\kern0pt}data{\isacharunderscore}{\kern0pt}partial{\isachardot}{\kern0pt}is{\isacharunderscore}{\kern0pt}P{\isacharunderscore}{\kern0pt}auto{\isacharunderscore}{\kern0pt}group{\isacharparenleft}{\kern0pt}Fn{\isacharcomma}{\kern0pt}\ Fn{\isacharunderscore}{\kern0pt}leq{\isacharcomma}{\kern0pt}\ M{\isacharcomma}{\kern0pt}\ Fn{\isacharunderscore}{\kern0pt}perms{\isacharparenright}{\kern0pt}{\isachardoublequoteclose}\ \isanewline
%
\isadelimproof
\ \ %
\endisadelimproof
%
\isatagproof
\isacommand{apply}\isamarkupfalse%
{\isacharparenleft}{\kern0pt}subst\ forcing{\isacharunderscore}{\kern0pt}data{\isacharunderscore}{\kern0pt}partial{\isachardot}{\kern0pt}is{\isacharunderscore}{\kern0pt}P{\isacharunderscore}{\kern0pt}auto{\isacharunderscore}{\kern0pt}group{\isacharunderscore}{\kern0pt}def{\isacharparenright}{\kern0pt}\isanewline
\ \ \ \isacommand{apply}\isamarkupfalse%
{\isacharparenleft}{\kern0pt}rule\ Fn{\isacharunderscore}{\kern0pt}forcing{\isacharunderscore}{\kern0pt}data{\isacharunderscore}{\kern0pt}partial{\isacharcomma}{\kern0pt}\ rule\ conjI{\isacharparenright}{\kern0pt}\isanewline
\ \ \ \isacommand{apply}\isamarkupfalse%
{\isacharparenleft}{\kern0pt}subst\ Fn{\isacharunderscore}{\kern0pt}perms{\isacharunderscore}{\kern0pt}def{\isacharparenright}{\kern0pt}\isanewline
\ \ \isacommand{using}\isamarkupfalse%
\ Fn{\isacharunderscore}{\kern0pt}perm{\isacharprime}{\kern0pt}{\isacharunderscore}{\kern0pt}type\ Fn{\isacharunderscore}{\kern0pt}perm{\isacharprime}{\kern0pt}{\isacharunderscore}{\kern0pt}is{\isacharunderscore}{\kern0pt}P{\isacharunderscore}{\kern0pt}auto\isanewline
\ \ \ \isacommand{apply}\isamarkupfalse%
\ force\isanewline
\ \ \isacommand{apply}\isamarkupfalse%
{\isacharparenleft}{\kern0pt}rule\ conjI{\isacharparenright}{\kern0pt}\isanewline
\ \ \ \isacommand{apply}\isamarkupfalse%
{\isacharparenleft}{\kern0pt}subst\ Fn{\isacharunderscore}{\kern0pt}perms{\isacharunderscore}{\kern0pt}def{\isacharparenright}{\kern0pt}{\isacharplus}{\kern0pt}\isanewline
\ \ \ \isacommand{apply}\isamarkupfalse%
\ clarsimp\isanewline
\ \ \ \isacommand{apply}\isamarkupfalse%
{\isacharparenleft}{\kern0pt}rename{\isacharunderscore}{\kern0pt}tac\ f\ f{\isacharprime}{\kern0pt}{\isacharcomma}{\kern0pt}\ rule{\isacharunderscore}{\kern0pt}tac\ x{\isacharequal}{\kern0pt}{\isachardoublequoteopen}f\ O\ f{\isacharprime}{\kern0pt}{\isachardoublequoteclose}\ \isakeyword{in}\ bexI{\isacharparenright}{\kern0pt}\isanewline
\ \ \ \ \isacommand{apply}\isamarkupfalse%
{\isacharparenleft}{\kern0pt}rule\ Fn{\isacharunderscore}{\kern0pt}perm{\isacharprime}{\kern0pt}{\isacharunderscore}{\kern0pt}comp{\isacharparenright}{\kern0pt}\isanewline
\ \ \isacommand{using}\isamarkupfalse%
\ nat{\isacharunderscore}{\kern0pt}perms{\isacharunderscore}{\kern0pt}def\ comp{\isacharunderscore}{\kern0pt}closed\ comp{\isacharunderscore}{\kern0pt}bij\ \isanewline
\ \ \ \ \ \isacommand{apply}\isamarkupfalse%
\ auto{\isacharbrackleft}{\kern0pt}{\isadigit{3}}{\isacharbrackright}{\kern0pt}\isanewline
\ \ \isacommand{unfolding}\isamarkupfalse%
\ Fn{\isacharunderscore}{\kern0pt}perms{\isacharunderscore}{\kern0pt}def\ \isanewline
\ \ \isacommand{apply}\isamarkupfalse%
{\isacharparenleft}{\kern0pt}clarsimp{\isacharparenright}{\kern0pt}\isanewline
\ \ \isacommand{apply}\isamarkupfalse%
{\isacharparenleft}{\kern0pt}rename{\isacharunderscore}{\kern0pt}tac\ f{\isacharcomma}{\kern0pt}\ rule{\isacharunderscore}{\kern0pt}tac\ x{\isacharequal}{\kern0pt}{\isachardoublequoteopen}converse{\isacharparenleft}{\kern0pt}f{\isacharparenright}{\kern0pt}{\isachardoublequoteclose}\ \isakeyword{in}\ bexI{\isacharparenright}{\kern0pt}\isanewline
\ \ \ \isacommand{apply}\isamarkupfalse%
{\isacharparenleft}{\kern0pt}rule\ Fn{\isacharunderscore}{\kern0pt}perm{\isacharprime}{\kern0pt}{\isacharunderscore}{\kern0pt}converse{\isacharparenright}{\kern0pt}\isanewline
\ \ \isacommand{using}\isamarkupfalse%
\ converse{\isacharunderscore}{\kern0pt}in{\isacharunderscore}{\kern0pt}nat{\isacharunderscore}{\kern0pt}perms\isanewline
\ \ \isacommand{by}\isamarkupfalse%
\ auto%
\endisatagproof
{\isafoldproof}%
%
\isadelimproof
\isanewline
%
\endisadelimproof
\isanewline
\isacommand{definition}\isamarkupfalse%
\ Fix\ \isakeyword{where}\ {\isachardoublequoteopen}Fix{\isacharparenleft}{\kern0pt}E{\isacharparenright}{\kern0pt}\ {\isasymequiv}\ {\isacharbraceleft}{\kern0pt}\ Fn{\isacharunderscore}{\kern0pt}perm{\isacharprime}{\kern0pt}{\isacharparenleft}{\kern0pt}f{\isacharparenright}{\kern0pt}{\isachardot}{\kern0pt}{\isachardot}{\kern0pt}\ f\ {\isasymin}\ nat{\isacharunderscore}{\kern0pt}perms{\isacharcomma}{\kern0pt}\ {\isasymforall}n\ {\isasymin}\ E{\isachardot}{\kern0pt}\ f{\isacharbackquote}{\kern0pt}n\ {\isacharequal}{\kern0pt}\ n\ {\isacharbraceright}{\kern0pt}{\isachardoublequoteclose}\isanewline
\isanewline
\isacommand{definition}\isamarkupfalse%
\ is{\isacharunderscore}{\kern0pt}Fix{\isacharunderscore}{\kern0pt}elem{\isacharunderscore}{\kern0pt}fm\ \isakeyword{where}\ {\isachardoublequoteopen}is{\isacharunderscore}{\kern0pt}Fix{\isacharunderscore}{\kern0pt}elem{\isacharunderscore}{\kern0pt}fm{\isacharparenleft}{\kern0pt}perms{\isacharcomma}{\kern0pt}\ fn{\isacharcomma}{\kern0pt}\ E{\isacharcomma}{\kern0pt}\ v{\isacharparenright}{\kern0pt}\ {\isasymequiv}\ \isanewline
\ \ Exists{\isacharparenleft}{\kern0pt}And{\isacharparenleft}{\kern0pt}Member{\isacharparenleft}{\kern0pt}{\isadigit{0}}{\isacharcomma}{\kern0pt}\ perms{\isacharhash}{\kern0pt}{\isacharplus}{\kern0pt}{\isadigit{1}}{\isacharparenright}{\kern0pt}{\isacharcomma}{\kern0pt}\ And{\isacharparenleft}{\kern0pt}is{\isacharunderscore}{\kern0pt}Fn{\isacharunderscore}{\kern0pt}perm{\isacharprime}{\kern0pt}{\isacharunderscore}{\kern0pt}fm{\isacharparenleft}{\kern0pt}{\isadigit{0}}{\isacharcomma}{\kern0pt}\ fn{\isacharhash}{\kern0pt}{\isacharplus}{\kern0pt}{\isadigit{1}}{\isacharcomma}{\kern0pt}\ v{\isacharhash}{\kern0pt}{\isacharplus}{\kern0pt}{\isadigit{1}}{\isacharparenright}{\kern0pt}{\isacharcomma}{\kern0pt}\ Forall{\isacharparenleft}{\kern0pt}Implies{\isacharparenleft}{\kern0pt}Member{\isacharparenleft}{\kern0pt}{\isadigit{0}}{\isacharcomma}{\kern0pt}\ E{\isacharhash}{\kern0pt}{\isacharplus}{\kern0pt}{\isadigit{2}}{\isacharparenright}{\kern0pt}{\isacharcomma}{\kern0pt}\ fun{\isacharunderscore}{\kern0pt}apply{\isacharunderscore}{\kern0pt}fm{\isacharparenleft}{\kern0pt}{\isadigit{1}}{\isacharcomma}{\kern0pt}\ {\isadigit{0}}{\isacharcomma}{\kern0pt}\ {\isadigit{0}}{\isacharparenright}{\kern0pt}{\isacharparenright}{\kern0pt}{\isacharparenright}{\kern0pt}{\isacharparenright}{\kern0pt}{\isacharparenright}{\kern0pt}{\isacharparenright}{\kern0pt}{\isachardoublequoteclose}\ \ \isanewline
\isanewline
\isacommand{lemma}\isamarkupfalse%
\ is{\isacharunderscore}{\kern0pt}Fix{\isacharunderscore}{\kern0pt}elem{\isacharunderscore}{\kern0pt}fm{\isacharunderscore}{\kern0pt}type\ {\isacharcolon}{\kern0pt}\ \isanewline
\ \ \isakeyword{fixes}\ i\ j\ k\ l\ \isanewline
\ \ \isakeyword{assumes}\ {\isachardoublequoteopen}i\ {\isasymin}\ nat{\isachardoublequoteclose}\ {\isachardoublequoteopen}j\ {\isasymin}\ nat{\isachardoublequoteclose}\ {\isachardoublequoteopen}k\ {\isasymin}\ nat{\isachardoublequoteclose}\ {\isachardoublequoteopen}l\ {\isasymin}\ nat{\isachardoublequoteclose}\ \isanewline
\ \ \isakeyword{shows}\ {\isachardoublequoteopen}is{\isacharunderscore}{\kern0pt}Fix{\isacharunderscore}{\kern0pt}elem{\isacharunderscore}{\kern0pt}fm{\isacharparenleft}{\kern0pt}i{\isacharcomma}{\kern0pt}\ j{\isacharcomma}{\kern0pt}\ k{\isacharcomma}{\kern0pt}\ l{\isacharparenright}{\kern0pt}\ {\isasymin}\ formula{\isachardoublequoteclose}\ \isanewline
%
\isadelimproof
\ \ %
\endisadelimproof
%
\isatagproof
\isacommand{apply}\isamarkupfalse%
{\isacharparenleft}{\kern0pt}subgoal{\isacharunderscore}{\kern0pt}tac\ {\isachardoublequoteopen}is{\isacharunderscore}{\kern0pt}Fn{\isacharunderscore}{\kern0pt}perm{\isacharprime}{\kern0pt}{\isacharunderscore}{\kern0pt}fm{\isacharparenleft}{\kern0pt}{\isadigit{0}}{\isacharcomma}{\kern0pt}\ j\ {\isacharhash}{\kern0pt}{\isacharplus}{\kern0pt}\ {\isadigit{1}}{\isacharcomma}{\kern0pt}\ l{\isacharhash}{\kern0pt}{\isacharplus}{\kern0pt}{\isadigit{1}}{\isacharparenright}{\kern0pt}\ {\isasymin}\ formula{\isachardoublequoteclose}{\isacharparenright}{\kern0pt}\isanewline
\ \ \isacommand{unfolding}\isamarkupfalse%
\ is{\isacharunderscore}{\kern0pt}Fix{\isacharunderscore}{\kern0pt}elem{\isacharunderscore}{\kern0pt}fm{\isacharunderscore}{\kern0pt}def\isanewline
\ \ \isacommand{apply}\isamarkupfalse%
\ force\isanewline
\ \ \isacommand{using}\isamarkupfalse%
\ is{\isacharunderscore}{\kern0pt}Fn{\isacharunderscore}{\kern0pt}perm{\isacharprime}{\kern0pt}{\isacharunderscore}{\kern0pt}fm{\isacharunderscore}{\kern0pt}type\ assms\isanewline
\ \ \isacommand{by}\isamarkupfalse%
\ auto%
\endisatagproof
{\isafoldproof}%
%
\isadelimproof
\isanewline
%
\endisadelimproof
\isanewline
\isacommand{lemma}\isamarkupfalse%
\ arity{\isacharunderscore}{\kern0pt}is{\isacharunderscore}{\kern0pt}Fix{\isacharunderscore}{\kern0pt}elem{\isacharunderscore}{\kern0pt}fm\ {\isacharcolon}{\kern0pt}\ \isanewline
\ \ \isakeyword{fixes}\ i\ j\ k\ l\ \isanewline
\ \ \isakeyword{assumes}\ {\isachardoublequoteopen}i\ {\isasymin}\ nat{\isachardoublequoteclose}\ {\isachardoublequoteopen}j\ {\isasymin}\ nat{\isachardoublequoteclose}\ {\isachardoublequoteopen}k\ {\isasymin}\ nat{\isachardoublequoteclose}\ {\isachardoublequoteopen}l\ {\isasymin}\ nat{\isachardoublequoteclose}\ \isanewline
\ \ \isakeyword{shows}\ {\isachardoublequoteopen}arity{\isacharparenleft}{\kern0pt}is{\isacharunderscore}{\kern0pt}Fix{\isacharunderscore}{\kern0pt}elem{\isacharunderscore}{\kern0pt}fm{\isacharparenleft}{\kern0pt}i{\isacharcomma}{\kern0pt}\ j{\isacharcomma}{\kern0pt}\ k{\isacharcomma}{\kern0pt}\ l{\isacharparenright}{\kern0pt}{\isacharparenright}{\kern0pt}\ {\isasymle}\ succ{\isacharparenleft}{\kern0pt}i{\isacharparenright}{\kern0pt}\ {\isasymunion}\ succ{\isacharparenleft}{\kern0pt}j{\isacharparenright}{\kern0pt}\ {\isasymunion}\ succ{\isacharparenleft}{\kern0pt}k{\isacharparenright}{\kern0pt}\ {\isasymunion}\ succ{\isacharparenleft}{\kern0pt}l{\isacharparenright}{\kern0pt}{\isachardoublequoteclose}\isanewline
%
\isadelimproof
\ \ %
\endisadelimproof
%
\isatagproof
\isacommand{apply}\isamarkupfalse%
{\isacharparenleft}{\kern0pt}subgoal{\isacharunderscore}{\kern0pt}tac\ {\isachardoublequoteopen}is{\isacharunderscore}{\kern0pt}Fn{\isacharunderscore}{\kern0pt}perm{\isacharprime}{\kern0pt}{\isacharunderscore}{\kern0pt}fm{\isacharparenleft}{\kern0pt}{\isadigit{0}}{\isacharcomma}{\kern0pt}\ j\ {\isacharhash}{\kern0pt}{\isacharplus}{\kern0pt}\ {\isadigit{1}}{\isacharcomma}{\kern0pt}\ l{\isacharhash}{\kern0pt}{\isacharplus}{\kern0pt}{\isadigit{1}}{\isacharparenright}{\kern0pt}\ {\isasymin}\ formula{\isachardoublequoteclose}{\isacharparenright}{\kern0pt}\isanewline
\ \ \isacommand{unfolding}\isamarkupfalse%
\ is{\isacharunderscore}{\kern0pt}Fix{\isacharunderscore}{\kern0pt}elem{\isacharunderscore}{\kern0pt}fm{\isacharunderscore}{\kern0pt}def\ \isanewline
\ \ \isacommand{using}\isamarkupfalse%
\ assms\isanewline
\ \ \isacommand{apply}\isamarkupfalse%
\ simp\isanewline
\ \ \ \isacommand{apply}\isamarkupfalse%
{\isacharparenleft}{\kern0pt}rule\ pred{\isacharunderscore}{\kern0pt}le{\isacharcomma}{\kern0pt}\ simp{\isacharcomma}{\kern0pt}\ simp{\isacharparenright}{\kern0pt}\isanewline
\ \ \ \isacommand{apply}\isamarkupfalse%
{\isacharparenleft}{\kern0pt}rule\ Un{\isacharunderscore}{\kern0pt}least{\isacharunderscore}{\kern0pt}lt{\isacharparenright}{\kern0pt}{\isacharplus}{\kern0pt}\isanewline
\ \ \ \ \ \isacommand{apply}\isamarkupfalse%
{\isacharparenleft}{\kern0pt}simp{\isacharcomma}{\kern0pt}\ simp{\isacharparenright}{\kern0pt}\isanewline
\ \ \ \ \isacommand{apply}\isamarkupfalse%
{\isacharparenleft}{\kern0pt}rule\ ltI{\isacharcomma}{\kern0pt}\ simp{\isacharcomma}{\kern0pt}\ simp{\isacharparenright}{\kern0pt}\isanewline
\ \ \ \isacommand{apply}\isamarkupfalse%
{\isacharparenleft}{\kern0pt}rule\ Un{\isacharunderscore}{\kern0pt}least{\isacharunderscore}{\kern0pt}lt{\isacharparenright}{\kern0pt}{\isacharplus}{\kern0pt}\isanewline
\ \ \ \ \isacommand{apply}\isamarkupfalse%
{\isacharparenleft}{\kern0pt}rule\ le{\isacharunderscore}{\kern0pt}trans{\isacharcomma}{\kern0pt}\ rule\ arity{\isacharunderscore}{\kern0pt}is{\isacharunderscore}{\kern0pt}Fn{\isacharunderscore}{\kern0pt}perm{\isacharprime}{\kern0pt}{\isacharunderscore}{\kern0pt}fm{\isacharparenright}{\kern0pt}\isanewline
\ \ \ \ \ \ \ \isacommand{apply}\isamarkupfalse%
\ auto{\isacharbrackleft}{\kern0pt}{\isadigit{3}}{\isacharbrackright}{\kern0pt}\isanewline
\ \ \ \ \isacommand{apply}\isamarkupfalse%
{\isacharparenleft}{\kern0pt}rule\ Un{\isacharunderscore}{\kern0pt}least{\isacharunderscore}{\kern0pt}lt{\isacharparenright}{\kern0pt}{\isacharplus}{\kern0pt}\isanewline
\ \ \ \ \ \ \isacommand{apply}\isamarkupfalse%
\ {\isacharparenleft}{\kern0pt}simp{\isacharcomma}{\kern0pt}\ simp{\isacharparenright}{\kern0pt}\isanewline
\ \ \ \ \ \isacommand{apply}\isamarkupfalse%
{\isacharparenleft}{\kern0pt}rule\ ltI{\isacharcomma}{\kern0pt}\ simp{\isacharcomma}{\kern0pt}\ simp{\isacharcomma}{\kern0pt}\ simp{\isacharparenright}{\kern0pt}\isanewline
\ \ \isacommand{apply}\isamarkupfalse%
{\isacharparenleft}{\kern0pt}rule\ union{\isacharunderscore}{\kern0pt}lt{\isadigit{2}}{\isacharcomma}{\kern0pt}\ simp{\isacharcomma}{\kern0pt}\ simp{\isacharcomma}{\kern0pt}\ simp{\isacharcomma}{\kern0pt}\ simp{\isacharparenright}{\kern0pt}\isanewline
\ \ \ \isacommand{apply}\isamarkupfalse%
{\isacharparenleft}{\kern0pt}rule\ pred{\isacharunderscore}{\kern0pt}le{\isacharcomma}{\kern0pt}\ simp{\isacharcomma}{\kern0pt}\ simp{\isacharparenright}{\kern0pt}\isanewline
\ \ \ \isacommand{apply}\isamarkupfalse%
{\isacharparenleft}{\kern0pt}rule\ Un{\isacharunderscore}{\kern0pt}least{\isacharunderscore}{\kern0pt}lt{\isacharparenright}{\kern0pt}{\isacharplus}{\kern0pt}\isanewline
\ \ \ \ \ \isacommand{apply}\isamarkupfalse%
{\isacharparenleft}{\kern0pt}simp{\isacharcomma}{\kern0pt}\ simp{\isacharparenright}{\kern0pt}\isanewline
\ \ \ \ \isacommand{apply}\isamarkupfalse%
{\isacharparenleft}{\kern0pt}rule\ ltI{\isacharcomma}{\kern0pt}\ simp{\isacharcomma}{\kern0pt}\ simp{\isacharparenright}{\kern0pt}\isanewline
\ \ \ \isacommand{apply}\isamarkupfalse%
{\isacharparenleft}{\kern0pt}subst\ arity{\isacharunderscore}{\kern0pt}fun{\isacharunderscore}{\kern0pt}apply{\isacharunderscore}{\kern0pt}fm{\isacharcomma}{\kern0pt}\ simp{\isacharcomma}{\kern0pt}\ simp{\isacharparenright}{\kern0pt}\isanewline
\ \ \ \isacommand{apply}\isamarkupfalse%
{\isacharparenleft}{\kern0pt}rule\ Un{\isacharunderscore}{\kern0pt}least{\isacharunderscore}{\kern0pt}lt{\isacharparenright}{\kern0pt}{\isacharplus}{\kern0pt}\isanewline
\ \ \isacommand{using}\isamarkupfalse%
\ is{\isacharunderscore}{\kern0pt}Fn{\isacharunderscore}{\kern0pt}perm{\isacharprime}{\kern0pt}{\isacharunderscore}{\kern0pt}fm{\isacharunderscore}{\kern0pt}type\ assms\isanewline
\ \ \ \ \ \isacommand{apply}\isamarkupfalse%
\ auto\isanewline
\ \ \isacommand{done}\isamarkupfalse%
%
\endisatagproof
{\isafoldproof}%
%
\isadelimproof
\isanewline
%
\endisadelimproof
\isanewline
\isacommand{lemma}\isamarkupfalse%
\ sats{\isacharunderscore}{\kern0pt}is{\isacharunderscore}{\kern0pt}Fix{\isacharunderscore}{\kern0pt}elem{\isacharunderscore}{\kern0pt}fm{\isacharunderscore}{\kern0pt}iff\ {\isacharcolon}{\kern0pt}\ \isanewline
\ \ \isakeyword{fixes}\ env\ i\ j\ k\ l\ E\ v\ \isanewline
\ \ \isakeyword{assumes}\ {\isachardoublequoteopen}env\ {\isasymin}\ list{\isacharparenleft}{\kern0pt}M{\isacharparenright}{\kern0pt}{\isachardoublequoteclose}\ {\isachardoublequoteopen}i\ {\isacharless}{\kern0pt}\ length{\isacharparenleft}{\kern0pt}env{\isacharparenright}{\kern0pt}{\isachardoublequoteclose}\ {\isachardoublequoteopen}j\ {\isacharless}{\kern0pt}\ length{\isacharparenleft}{\kern0pt}env{\isacharparenright}{\kern0pt}{\isachardoublequoteclose}\ {\isachardoublequoteopen}k\ {\isacharless}{\kern0pt}\ length{\isacharparenleft}{\kern0pt}env{\isacharparenright}{\kern0pt}{\isachardoublequoteclose}\ {\isachardoublequoteopen}l\ {\isacharless}{\kern0pt}\ length{\isacharparenleft}{\kern0pt}env{\isacharparenright}{\kern0pt}{\isachardoublequoteclose}\isanewline
\ \ \ \ \ \ \ \ \ \ {\isachardoublequoteopen}nth{\isacharparenleft}{\kern0pt}i{\isacharcomma}{\kern0pt}\ env{\isacharparenright}{\kern0pt}\ {\isacharequal}{\kern0pt}\ nat{\isacharunderscore}{\kern0pt}perms{\isachardoublequoteclose}\ {\isachardoublequoteopen}nth{\isacharparenleft}{\kern0pt}j{\isacharcomma}{\kern0pt}\ env{\isacharparenright}{\kern0pt}\ {\isacharequal}{\kern0pt}\ Fn{\isachardoublequoteclose}\ {\isachardoublequoteopen}nth{\isacharparenleft}{\kern0pt}k{\isacharcomma}{\kern0pt}\ env{\isacharparenright}{\kern0pt}\ {\isacharequal}{\kern0pt}\ E{\isachardoublequoteclose}\ {\isachardoublequoteopen}nth{\isacharparenleft}{\kern0pt}l{\isacharcomma}{\kern0pt}\ env{\isacharparenright}{\kern0pt}\ {\isacharequal}{\kern0pt}\ v{\isachardoublequoteclose}\ \isanewline
\ \ \isakeyword{shows}\ {\isachardoublequoteopen}sats{\isacharparenleft}{\kern0pt}M{\isacharcomma}{\kern0pt}\ is{\isacharunderscore}{\kern0pt}Fix{\isacharunderscore}{\kern0pt}elem{\isacharunderscore}{\kern0pt}fm{\isacharparenleft}{\kern0pt}i{\isacharcomma}{\kern0pt}\ j{\isacharcomma}{\kern0pt}\ k{\isacharcomma}{\kern0pt}\ l{\isacharparenright}{\kern0pt}{\isacharcomma}{\kern0pt}\ env{\isacharparenright}{\kern0pt}\ {\isasymlongleftrightarrow}\ v\ {\isasymin}\ Fix{\isacharparenleft}{\kern0pt}E{\isacharparenright}{\kern0pt}{\isachardoublequoteclose}\ \isanewline
%
\isadelimproof
%
\endisadelimproof
%
\isatagproof
\isacommand{proof}\isamarkupfalse%
\ {\isacharminus}{\kern0pt}\ \isanewline
\ \ \isacommand{have}\isamarkupfalse%
\ I{\isadigit{1}}{\isacharcolon}{\kern0pt}{\isachardoublequoteopen}sats{\isacharparenleft}{\kern0pt}M{\isacharcomma}{\kern0pt}\ is{\isacharunderscore}{\kern0pt}Fix{\isacharunderscore}{\kern0pt}elem{\isacharunderscore}{\kern0pt}fm{\isacharparenleft}{\kern0pt}i{\isacharcomma}{\kern0pt}\ j{\isacharcomma}{\kern0pt}\ k{\isacharcomma}{\kern0pt}\ l{\isacharparenright}{\kern0pt}{\isacharcomma}{\kern0pt}\ env{\isacharparenright}{\kern0pt}\ {\isasymlongleftrightarrow}\ {\isacharparenleft}{\kern0pt}{\isasymexists}f\ {\isasymin}\ M{\isachardot}{\kern0pt}\ f\ {\isasymin}\ nat{\isacharunderscore}{\kern0pt}perms\ {\isasymand}\ v\ {\isacharequal}{\kern0pt}\ Fn{\isacharunderscore}{\kern0pt}perm{\isacharprime}{\kern0pt}{\isacharparenleft}{\kern0pt}f{\isacharparenright}{\kern0pt}\ {\isasymand}\ {\isacharparenleft}{\kern0pt}{\isasymforall}n\ {\isasymin}\ M{\isachardot}{\kern0pt}\ n\ {\isasymin}\ E\ {\isasymlongrightarrow}\ f{\isacharbackquote}{\kern0pt}n\ {\isacharequal}{\kern0pt}\ n{\isacharparenright}{\kern0pt}{\isacharparenright}{\kern0pt}{\isachardoublequoteclose}\ \isanewline
\ \ \ \ \isacommand{unfolding}\isamarkupfalse%
\ is{\isacharunderscore}{\kern0pt}Fix{\isacharunderscore}{\kern0pt}elem{\isacharunderscore}{\kern0pt}fm{\isacharunderscore}{\kern0pt}def\isanewline
\ \ \ \ \isacommand{apply}\isamarkupfalse%
{\isacharparenleft}{\kern0pt}rule\ iff{\isacharunderscore}{\kern0pt}trans{\isacharcomma}{\kern0pt}\ rule\ sats{\isacharunderscore}{\kern0pt}Exists{\isacharunderscore}{\kern0pt}iff{\isacharcomma}{\kern0pt}\ simp\ add{\isacharcolon}{\kern0pt}assms{\isacharcomma}{\kern0pt}\ rule\ bex{\isacharunderscore}{\kern0pt}iff{\isacharparenright}{\kern0pt}\isanewline
\ \ \ \ \isacommand{apply}\isamarkupfalse%
{\isacharparenleft}{\kern0pt}rule\ iff{\isacharunderscore}{\kern0pt}trans{\isacharcomma}{\kern0pt}\ rule\ sats{\isacharunderscore}{\kern0pt}And{\isacharunderscore}{\kern0pt}iff{\isacharcomma}{\kern0pt}\ simp\ add{\isacharcolon}{\kern0pt}assms{\isacharcomma}{\kern0pt}\ rule\ iff{\isacharunderscore}{\kern0pt}conjI{\isadigit{2}}{\isacharparenright}{\kern0pt}\isanewline
\ \ \ \ \isacommand{using}\isamarkupfalse%
\ assms\ nth{\isacharunderscore}{\kern0pt}type\ lt{\isacharunderscore}{\kern0pt}nat{\isacharunderscore}{\kern0pt}in{\isacharunderscore}{\kern0pt}nat\isanewline
\ \ \ \ \ \isacommand{apply}\isamarkupfalse%
\ simp\isanewline
\ \ \ \ \isacommand{apply}\isamarkupfalse%
{\isacharparenleft}{\kern0pt}rule\ iff{\isacharunderscore}{\kern0pt}trans{\isacharcomma}{\kern0pt}\ rule\ sats{\isacharunderscore}{\kern0pt}And{\isacharunderscore}{\kern0pt}iff{\isacharcomma}{\kern0pt}\ simp\ add{\isacharcolon}{\kern0pt}assms{\isacharcomma}{\kern0pt}\ rule\ iff{\isacharunderscore}{\kern0pt}conjI{\isadigit{2}}{\isacharparenright}{\kern0pt}\isanewline
\ \ \ \ \ \isacommand{apply}\isamarkupfalse%
{\isacharparenleft}{\kern0pt}rule\ sats{\isacharunderscore}{\kern0pt}is{\isacharunderscore}{\kern0pt}Fn{\isacharunderscore}{\kern0pt}perm{\isacharprime}{\kern0pt}{\isacharunderscore}{\kern0pt}fm{\isacharunderscore}{\kern0pt}iff{\isacharparenright}{\kern0pt}\isanewline
\ \ \ \ \isacommand{using}\isamarkupfalse%
\ assms\ nth{\isacharunderscore}{\kern0pt}type\ lt{\isacharunderscore}{\kern0pt}nat{\isacharunderscore}{\kern0pt}in{\isacharunderscore}{\kern0pt}nat\isanewline
\ \ \ \ \ \ \ \ \ \ \ \ \isacommand{apply}\isamarkupfalse%
\ auto{\isacharbrackleft}{\kern0pt}{\isadigit{8}}{\isacharbrackright}{\kern0pt}\isanewline
\ \ \ \ \isacommand{apply}\isamarkupfalse%
{\isacharparenleft}{\kern0pt}rule\ iff{\isacharunderscore}{\kern0pt}trans{\isacharcomma}{\kern0pt}\ rule\ sats{\isacharunderscore}{\kern0pt}Forall{\isacharunderscore}{\kern0pt}iff{\isacharcomma}{\kern0pt}\ simp\ add{\isacharcolon}{\kern0pt}assms{\isacharcomma}{\kern0pt}\ rule\ ball{\isacharunderscore}{\kern0pt}iff{\isacharparenright}{\kern0pt}\isanewline
\ \ \ \ \isacommand{apply}\isamarkupfalse%
{\isacharparenleft}{\kern0pt}rule\ iff{\isacharunderscore}{\kern0pt}trans{\isacharcomma}{\kern0pt}\ rule\ sats{\isacharunderscore}{\kern0pt}Implies{\isacharunderscore}{\kern0pt}iff{\isacharcomma}{\kern0pt}\ simp\ add{\isacharcolon}{\kern0pt}assms{\isacharcomma}{\kern0pt}\ rule\ imp{\isacharunderscore}{\kern0pt}iff{\isadigit{2}}{\isacharparenright}{\kern0pt}\isanewline
\ \ \ \ \isacommand{using}\isamarkupfalse%
\ assms\ nth{\isacharunderscore}{\kern0pt}type\ lt{\isacharunderscore}{\kern0pt}nat{\isacharunderscore}{\kern0pt}in{\isacharunderscore}{\kern0pt}nat\isanewline
\ \ \ \ \ \isacommand{apply}\isamarkupfalse%
\ simp\isanewline
\ \ \ \ \isacommand{apply}\isamarkupfalse%
{\isacharparenleft}{\kern0pt}rule\ iff{\isacharunderscore}{\kern0pt}trans{\isacharcomma}{\kern0pt}\ rule\ sats{\isacharunderscore}{\kern0pt}fun{\isacharunderscore}{\kern0pt}apply{\isacharunderscore}{\kern0pt}fm{\isacharparenright}{\kern0pt}\isanewline
\ \ \ \ \isacommand{using}\isamarkupfalse%
\ assms\isanewline
\ \ \ \ \isacommand{by}\isamarkupfalse%
\ auto\isanewline
\ \ \isacommand{have}\isamarkupfalse%
\ I{\isadigit{2}}{\isacharcolon}{\kern0pt}{\isachardoublequoteopen}{\isachardot}{\kern0pt}{\isachardot}{\kern0pt}{\isachardot}{\kern0pt}\ {\isasymlongleftrightarrow}\ {\isacharparenleft}{\kern0pt}{\isasymexists}f{\isachardot}{\kern0pt}\ f\ {\isasymin}\ nat{\isacharunderscore}{\kern0pt}perms\ {\isasymand}\ v\ {\isacharequal}{\kern0pt}\ Fn{\isacharunderscore}{\kern0pt}perm{\isacharprime}{\kern0pt}{\isacharparenleft}{\kern0pt}f{\isacharparenright}{\kern0pt}\ {\isasymand}\ {\isacharparenleft}{\kern0pt}{\isasymforall}n\ {\isasymin}\ M{\isachardot}{\kern0pt}\ n\ {\isasymin}\ E\ {\isasymlongrightarrow}\ f{\isacharbackquote}{\kern0pt}n\ {\isacharequal}{\kern0pt}\ n{\isacharparenright}{\kern0pt}{\isacharparenright}{\kern0pt}{\isachardoublequoteclose}\ \isanewline
\ \ \ \ \isacommand{using}\isamarkupfalse%
\ nat{\isacharunderscore}{\kern0pt}perms{\isacharunderscore}{\kern0pt}in{\isacharunderscore}{\kern0pt}M\ transM\isanewline
\ \ \ \ \isacommand{by}\isamarkupfalse%
\ auto\isanewline
\ \ \isacommand{have}\isamarkupfalse%
\ I{\isadigit{3}}{\isacharcolon}{\kern0pt}{\isachardoublequoteopen}{\isachardot}{\kern0pt}{\isachardot}{\kern0pt}{\isachardot}{\kern0pt}\ {\isasymlongleftrightarrow}\ {\isacharparenleft}{\kern0pt}{\isasymexists}f{\isachardot}{\kern0pt}\ f\ {\isasymin}\ nat{\isacharunderscore}{\kern0pt}perms\ {\isasymand}\ v\ {\isacharequal}{\kern0pt}\ Fn{\isacharunderscore}{\kern0pt}perm{\isacharprime}{\kern0pt}{\isacharparenleft}{\kern0pt}f{\isacharparenright}{\kern0pt}\ {\isasymand}\ {\isacharparenleft}{\kern0pt}{\isasymforall}n\ {\isasymin}\ E{\isachardot}{\kern0pt}\ f{\isacharbackquote}{\kern0pt}n\ {\isacharequal}{\kern0pt}\ n{\isacharparenright}{\kern0pt}{\isacharparenright}{\kern0pt}{\isachardoublequoteclose}\ \isanewline
\ \ \ \ \isacommand{using}\isamarkupfalse%
\ assms\ nth{\isacharunderscore}{\kern0pt}type\ lt{\isacharunderscore}{\kern0pt}nat{\isacharunderscore}{\kern0pt}in{\isacharunderscore}{\kern0pt}nat\ transM\isanewline
\ \ \ \ \isacommand{by}\isamarkupfalse%
\ auto\isanewline
\ \ \isacommand{have}\isamarkupfalse%
\ I{\isadigit{4}}{\isacharcolon}{\kern0pt}{\isachardoublequoteopen}{\isachardot}{\kern0pt}{\isachardot}{\kern0pt}{\isachardot}{\kern0pt}\ {\isasymlongleftrightarrow}\ v\ {\isasymin}\ Fix{\isacharparenleft}{\kern0pt}E{\isacharparenright}{\kern0pt}{\isachardoublequoteclose}\ \isanewline
\ \ \ \ \isacommand{unfolding}\isamarkupfalse%
\ Fix{\isacharunderscore}{\kern0pt}def\isanewline
\ \ \ \ \isacommand{by}\isamarkupfalse%
\ auto\isanewline
\ \ \isacommand{show}\isamarkupfalse%
\ {\isacharquery}{\kern0pt}thesis\ \isacommand{using}\isamarkupfalse%
\ I{\isadigit{1}}\ I{\isadigit{2}}\ I{\isadigit{3}}\ I{\isadigit{4}}\ \isacommand{by}\isamarkupfalse%
\ auto\isanewline
\isacommand{qed}\isamarkupfalse%
%
\endisatagproof
{\isafoldproof}%
%
\isadelimproof
\isanewline
%
\endisadelimproof
\isanewline
\isacommand{lemma}\isamarkupfalse%
\ Fix{\isacharunderscore}{\kern0pt}in{\isacharunderscore}{\kern0pt}M\ {\isacharcolon}{\kern0pt}\ \isanewline
\ \ \isakeyword{fixes}\ E\ \isanewline
\ \ \isakeyword{assumes}\ {\isachardoublequoteopen}E\ {\isasymin}\ M{\isachardoublequoteclose}\ \isanewline
\ \ \isakeyword{shows}\ {\isachardoublequoteopen}Fix{\isacharparenleft}{\kern0pt}E{\isacharparenright}{\kern0pt}\ {\isasymin}\ M{\isachardoublequoteclose}\ \isanewline
%
\isadelimproof
%
\endisadelimproof
%
\isatagproof
\isacommand{proof}\isamarkupfalse%
\ {\isacharminus}{\kern0pt}\ \isanewline
\ \ \isacommand{define}\isamarkupfalse%
\ X\ \isakeyword{where}\ {\isachardoublequoteopen}X\ {\isasymequiv}\ {\isacharbraceleft}{\kern0pt}\ {\isasympi}\ {\isasymin}\ Fn{\isacharunderscore}{\kern0pt}perms\ {\isachardot}{\kern0pt}\ sats{\isacharparenleft}{\kern0pt}M{\isacharcomma}{\kern0pt}\ is{\isacharunderscore}{\kern0pt}Fix{\isacharunderscore}{\kern0pt}elem{\isacharunderscore}{\kern0pt}fm{\isacharparenleft}{\kern0pt}{\isadigit{1}}{\isacharcomma}{\kern0pt}{\isadigit{2}}{\isacharcomma}{\kern0pt}{\isadigit{3}}{\isacharcomma}{\kern0pt}{\isadigit{0}}{\isacharparenright}{\kern0pt}{\isacharcomma}{\kern0pt}\ {\isacharbrackleft}{\kern0pt}{\isasympi}{\isacharbrackright}{\kern0pt}\ {\isacharat}{\kern0pt}\ {\isacharbrackleft}{\kern0pt}nat{\isacharunderscore}{\kern0pt}perms{\isacharcomma}{\kern0pt}\ Fn{\isacharcomma}{\kern0pt}\ E{\isacharbrackright}{\kern0pt}{\isacharparenright}{\kern0pt}\ {\isacharbraceright}{\kern0pt}{\isachardoublequoteclose}\isanewline
\isanewline
\ \ \isacommand{have}\isamarkupfalse%
\ {\isachardoublequoteopen}X\ {\isasymin}\ M{\isachardoublequoteclose}\ \isanewline
\ \ \ \ \isacommand{unfolding}\isamarkupfalse%
\ X{\isacharunderscore}{\kern0pt}def\isanewline
\ \ \ \ \isacommand{apply}\isamarkupfalse%
{\isacharparenleft}{\kern0pt}rule\ separation{\isacharunderscore}{\kern0pt}notation{\isacharcomma}{\kern0pt}\ rule\ separation{\isacharunderscore}{\kern0pt}ax{\isacharparenright}{\kern0pt}\isanewline
\ \ \ \ \ \ \ \isacommand{apply}\isamarkupfalse%
{\isacharparenleft}{\kern0pt}rule\ is{\isacharunderscore}{\kern0pt}Fix{\isacharunderscore}{\kern0pt}elem{\isacharunderscore}{\kern0pt}fm{\isacharunderscore}{\kern0pt}type{\isacharparenright}{\kern0pt}\isanewline
\ \ \ \ \isacommand{using}\isamarkupfalse%
\ assms\ nat{\isacharunderscore}{\kern0pt}perms{\isacharunderscore}{\kern0pt}in{\isacharunderscore}{\kern0pt}M\ Fn{\isacharunderscore}{\kern0pt}in{\isacharunderscore}{\kern0pt}M\ \isanewline
\ \ \ \ \ \ \ \ \ \ \isacommand{apply}\isamarkupfalse%
\ auto{\isacharbrackleft}{\kern0pt}{\isadigit{5}}{\isacharbrackright}{\kern0pt}\isanewline
\ \ \ \ \ \isacommand{apply}\isamarkupfalse%
{\isacharparenleft}{\kern0pt}rule\ le{\isacharunderscore}{\kern0pt}trans{\isacharcomma}{\kern0pt}\ rule\ local{\isachardot}{\kern0pt}arity{\isacharunderscore}{\kern0pt}is{\isacharunderscore}{\kern0pt}Fix{\isacharunderscore}{\kern0pt}elem{\isacharunderscore}{\kern0pt}fm{\isacharparenright}{\kern0pt}\isanewline
\ \ \ \ \isacommand{using}\isamarkupfalse%
\ Un{\isacharunderscore}{\kern0pt}least{\isacharunderscore}{\kern0pt}lt\ Fn{\isacharunderscore}{\kern0pt}perms{\isacharunderscore}{\kern0pt}in{\isacharunderscore}{\kern0pt}M\isanewline
\ \ \ \ \isacommand{by}\isamarkupfalse%
\ auto\isanewline
\ \ \isacommand{have}\isamarkupfalse%
\ {\isachardoublequoteopen}X\ {\isacharequal}{\kern0pt}\ {\isacharbraceleft}{\kern0pt}\ {\isasympi}\ {\isasymin}\ Fn{\isacharunderscore}{\kern0pt}perms\ {\isachardot}{\kern0pt}\ {\isasympi}\ {\isasymin}\ Fix{\isacharparenleft}{\kern0pt}E{\isacharparenright}{\kern0pt}\ {\isacharbraceright}{\kern0pt}{\isachardoublequoteclose}\isanewline
\ \ \ \ \isacommand{unfolding}\isamarkupfalse%
\ X{\isacharunderscore}{\kern0pt}def\isanewline
\ \ \ \ \isacommand{apply}\isamarkupfalse%
{\isacharparenleft}{\kern0pt}rule\ iff{\isacharunderscore}{\kern0pt}eq{\isacharcomma}{\kern0pt}\ rule\ sats{\isacharunderscore}{\kern0pt}is{\isacharunderscore}{\kern0pt}Fix{\isacharunderscore}{\kern0pt}elem{\isacharunderscore}{\kern0pt}fm{\isacharunderscore}{\kern0pt}iff{\isacharparenright}{\kern0pt}\isanewline
\ \ \ \ \isacommand{using}\isamarkupfalse%
\ Fn{\isacharunderscore}{\kern0pt}perms{\isacharunderscore}{\kern0pt}in{\isacharunderscore}{\kern0pt}M\ assms\ nat{\isacharunderscore}{\kern0pt}perms{\isacharunderscore}{\kern0pt}in{\isacharunderscore}{\kern0pt}M\ Fn{\isacharunderscore}{\kern0pt}in{\isacharunderscore}{\kern0pt}M\ transM\isanewline
\ \ \ \ \isacommand{by}\isamarkupfalse%
\ auto\isanewline
\ \ \isacommand{also}\isamarkupfalse%
\ \isacommand{have}\isamarkupfalse%
\ {\isachardoublequoteopen}{\isachardot}{\kern0pt}{\isachardot}{\kern0pt}{\isachardot}{\kern0pt}\ {\isacharequal}{\kern0pt}\ Fix{\isacharparenleft}{\kern0pt}E{\isacharparenright}{\kern0pt}{\isachardoublequoteclose}\ \isanewline
\ \ \ \ \isacommand{apply}\isamarkupfalse%
{\isacharparenleft}{\kern0pt}rule\ equalityI{\isacharcomma}{\kern0pt}\ force{\isacharparenright}{\kern0pt}\isanewline
\ \ \ \ \isacommand{unfolding}\isamarkupfalse%
\ Fix{\isacharunderscore}{\kern0pt}def\ Fn{\isacharunderscore}{\kern0pt}perms{\isacharunderscore}{\kern0pt}def\isanewline
\ \ \ \ \isacommand{by}\isamarkupfalse%
\ auto\isanewline
\ \ \isacommand{finally}\isamarkupfalse%
\ \isacommand{show}\isamarkupfalse%
\ {\isacharquery}{\kern0pt}thesis\ \isacommand{using}\isamarkupfalse%
\ {\isacartoucheopen}X\ {\isasymin}\ M{\isacartoucheclose}\ \isacommand{by}\isamarkupfalse%
\ auto\isanewline
\isacommand{qed}\isamarkupfalse%
%
\endisatagproof
{\isafoldproof}%
%
\isadelimproof
\isanewline
%
\endisadelimproof
\isanewline
\isacommand{definition}\isamarkupfalse%
\ is{\isacharunderscore}{\kern0pt}Fix{\isacharunderscore}{\kern0pt}fm\ \isakeyword{where}\ {\isachardoublequoteopen}is{\isacharunderscore}{\kern0pt}Fix{\isacharunderscore}{\kern0pt}fm{\isacharparenleft}{\kern0pt}perms{\isacharcomma}{\kern0pt}\ fn{\isacharcomma}{\kern0pt}\ E{\isacharcomma}{\kern0pt}\ v{\isacharparenright}{\kern0pt}\ {\isasymequiv}\ Forall{\isacharparenleft}{\kern0pt}Iff{\isacharparenleft}{\kern0pt}Member{\isacharparenleft}{\kern0pt}{\isadigit{0}}{\isacharcomma}{\kern0pt}\ v{\isacharhash}{\kern0pt}{\isacharplus}{\kern0pt}{\isadigit{1}}{\isacharparenright}{\kern0pt}{\isacharcomma}{\kern0pt}\ is{\isacharunderscore}{\kern0pt}Fix{\isacharunderscore}{\kern0pt}elem{\isacharunderscore}{\kern0pt}fm{\isacharparenleft}{\kern0pt}perms{\isacharhash}{\kern0pt}{\isacharplus}{\kern0pt}{\isadigit{1}}{\isacharcomma}{\kern0pt}\ fn{\isacharhash}{\kern0pt}{\isacharplus}{\kern0pt}{\isadigit{1}}{\isacharcomma}{\kern0pt}\ E{\isacharhash}{\kern0pt}{\isacharplus}{\kern0pt}{\isadigit{1}}{\isacharcomma}{\kern0pt}\ {\isadigit{0}}{\isacharparenright}{\kern0pt}{\isacharparenright}{\kern0pt}{\isacharparenright}{\kern0pt}{\isachardoublequoteclose}\ \isanewline
\isanewline
\isacommand{lemma}\isamarkupfalse%
\ is{\isacharunderscore}{\kern0pt}Fix{\isacharunderscore}{\kern0pt}fm{\isacharunderscore}{\kern0pt}type\ {\isacharcolon}{\kern0pt}\ \isanewline
\ \ \isakeyword{fixes}\ i\ j\ k\ l\ \isanewline
\ \ \isakeyword{assumes}\ {\isachardoublequoteopen}i\ {\isasymin}\ nat{\isachardoublequoteclose}\ {\isachardoublequoteopen}j\ {\isasymin}\ nat{\isachardoublequoteclose}\ {\isachardoublequoteopen}k\ {\isasymin}\ nat{\isachardoublequoteclose}\ {\isachardoublequoteopen}l\ {\isasymin}\ nat{\isachardoublequoteclose}\isanewline
\ \ \isakeyword{shows}\ {\isachardoublequoteopen}is{\isacharunderscore}{\kern0pt}Fix{\isacharunderscore}{\kern0pt}fm{\isacharparenleft}{\kern0pt}i{\isacharcomma}{\kern0pt}\ j{\isacharcomma}{\kern0pt}\ k{\isacharcomma}{\kern0pt}\ l{\isacharparenright}{\kern0pt}\ {\isasymin}\ formula{\isachardoublequoteclose}\ \isanewline
%
\isadelimproof
\isanewline
\ \ %
\endisadelimproof
%
\isatagproof
\isacommand{unfolding}\isamarkupfalse%
\ is{\isacharunderscore}{\kern0pt}Fix{\isacharunderscore}{\kern0pt}fm{\isacharunderscore}{\kern0pt}def\isanewline
\ \ \isacommand{apply}\isamarkupfalse%
{\isacharparenleft}{\kern0pt}subgoal{\isacharunderscore}{\kern0pt}tac\ {\isachardoublequoteopen}is{\isacharunderscore}{\kern0pt}Fix{\isacharunderscore}{\kern0pt}elem{\isacharunderscore}{\kern0pt}fm{\isacharparenleft}{\kern0pt}i\ {\isacharhash}{\kern0pt}{\isacharplus}{\kern0pt}\ {\isadigit{1}}{\isacharcomma}{\kern0pt}\ j\ {\isacharhash}{\kern0pt}{\isacharplus}{\kern0pt}\ {\isadigit{1}}{\isacharcomma}{\kern0pt}\ k\ {\isacharhash}{\kern0pt}{\isacharplus}{\kern0pt}\ {\isadigit{1}}{\isacharcomma}{\kern0pt}\ {\isadigit{0}}{\isacharparenright}{\kern0pt}\ {\isasymin}\ formula{\isachardoublequoteclose}{\isacharcomma}{\kern0pt}\ force{\isacharparenright}{\kern0pt}\isanewline
\ \ \isacommand{apply}\isamarkupfalse%
{\isacharparenleft}{\kern0pt}rule\ is{\isacharunderscore}{\kern0pt}Fix{\isacharunderscore}{\kern0pt}elem{\isacharunderscore}{\kern0pt}fm{\isacharunderscore}{\kern0pt}type{\isacharparenright}{\kern0pt}\isanewline
\ \ \isacommand{using}\isamarkupfalse%
\ assms\isanewline
\ \ \isacommand{by}\isamarkupfalse%
\ auto%
\endisatagproof
{\isafoldproof}%
%
\isadelimproof
\isanewline
%
\endisadelimproof
\isanewline
\isacommand{lemma}\isamarkupfalse%
\ arity{\isacharunderscore}{\kern0pt}is{\isacharunderscore}{\kern0pt}Fix{\isacharunderscore}{\kern0pt}fm\ {\isacharcolon}{\kern0pt}\ \isanewline
\ \ \isakeyword{fixes}\ i\ j\ k\ l\ \isanewline
\ \ \isakeyword{assumes}\ {\isachardoublequoteopen}i\ {\isasymin}\ nat{\isachardoublequoteclose}\ {\isachardoublequoteopen}j\ {\isasymin}\ nat{\isachardoublequoteclose}\ {\isachardoublequoteopen}k\ {\isasymin}\ nat{\isachardoublequoteclose}\ {\isachardoublequoteopen}l\ {\isasymin}\ nat{\isachardoublequoteclose}\isanewline
\ \ \isakeyword{shows}\ {\isachardoublequoteopen}arity{\isacharparenleft}{\kern0pt}is{\isacharunderscore}{\kern0pt}Fix{\isacharunderscore}{\kern0pt}fm{\isacharparenleft}{\kern0pt}i{\isacharcomma}{\kern0pt}\ j{\isacharcomma}{\kern0pt}\ k{\isacharcomma}{\kern0pt}\ l{\isacharparenright}{\kern0pt}{\isacharparenright}{\kern0pt}\ {\isasymle}\ succ{\isacharparenleft}{\kern0pt}i{\isacharparenright}{\kern0pt}\ {\isasymunion}\ succ{\isacharparenleft}{\kern0pt}j{\isacharparenright}{\kern0pt}\ {\isasymunion}\ succ{\isacharparenleft}{\kern0pt}k{\isacharparenright}{\kern0pt}\ {\isasymunion}\ succ{\isacharparenleft}{\kern0pt}l{\isacharparenright}{\kern0pt}{\isachardoublequoteclose}\isanewline
%
\isadelimproof
\isanewline
\ \ %
\endisadelimproof
%
\isatagproof
\isacommand{unfolding}\isamarkupfalse%
\ is{\isacharunderscore}{\kern0pt}Fix{\isacharunderscore}{\kern0pt}fm{\isacharunderscore}{\kern0pt}def\isanewline
\ \ \isacommand{apply}\isamarkupfalse%
{\isacharparenleft}{\kern0pt}subgoal{\isacharunderscore}{\kern0pt}tac\ {\isachardoublequoteopen}is{\isacharunderscore}{\kern0pt}Fix{\isacharunderscore}{\kern0pt}elem{\isacharunderscore}{\kern0pt}fm{\isacharparenleft}{\kern0pt}i\ {\isacharhash}{\kern0pt}{\isacharplus}{\kern0pt}\ {\isadigit{1}}{\isacharcomma}{\kern0pt}\ j\ {\isacharhash}{\kern0pt}{\isacharplus}{\kern0pt}\ {\isadigit{1}}{\isacharcomma}{\kern0pt}\ k\ {\isacharhash}{\kern0pt}{\isacharplus}{\kern0pt}\ {\isadigit{1}}{\isacharcomma}{\kern0pt}\ {\isadigit{0}}{\isacharparenright}{\kern0pt}\ {\isasymin}\ formula{\isachardoublequoteclose}{\isacharparenright}{\kern0pt}\isanewline
\ \ \isacommand{using}\isamarkupfalse%
\ assms\isanewline
\ \ \ \isacommand{apply}\isamarkupfalse%
\ simp\isanewline
\ \ \ \isacommand{apply}\isamarkupfalse%
{\isacharparenleft}{\kern0pt}rule\ pred{\isacharunderscore}{\kern0pt}le{\isacharcomma}{\kern0pt}\ simp{\isacharcomma}{\kern0pt}\ simp{\isacharparenright}{\kern0pt}\isanewline
\ \ \ \isacommand{apply}\isamarkupfalse%
{\isacharparenleft}{\kern0pt}rule\ Un{\isacharunderscore}{\kern0pt}least{\isacharunderscore}{\kern0pt}lt{\isacharparenright}{\kern0pt}{\isacharplus}{\kern0pt}\isanewline
\ \ \ \ \ \isacommand{apply}\isamarkupfalse%
\ {\isacharparenleft}{\kern0pt}simp{\isacharcomma}{\kern0pt}\ simp{\isacharparenright}{\kern0pt}\isanewline
\ \ \ \ \isacommand{apply}\isamarkupfalse%
{\isacharparenleft}{\kern0pt}rule\ ltI{\isacharcomma}{\kern0pt}\ simp{\isacharcomma}{\kern0pt}\ simp{\isacharparenright}{\kern0pt}\isanewline
\ \ \ \isacommand{apply}\isamarkupfalse%
{\isacharparenleft}{\kern0pt}rule\ le{\isacharunderscore}{\kern0pt}trans{\isacharcomma}{\kern0pt}\ rule\ arity{\isacharunderscore}{\kern0pt}is{\isacharunderscore}{\kern0pt}Fix{\isacharunderscore}{\kern0pt}elem{\isacharunderscore}{\kern0pt}fm{\isacharparenright}{\kern0pt}\isanewline
\ \ \ \ \ \ \ \isacommand{apply}\isamarkupfalse%
\ auto{\isacharbrackleft}{\kern0pt}{\isadigit{4}}{\isacharbrackright}{\kern0pt}\isanewline
\ \ \ \isacommand{apply}\isamarkupfalse%
{\isacharparenleft}{\kern0pt}rule\ Un{\isacharunderscore}{\kern0pt}least{\isacharunderscore}{\kern0pt}lt{\isacharparenright}{\kern0pt}{\isacharplus}{\kern0pt}\isanewline
\ \ \ \ \ \ \isacommand{apply}\isamarkupfalse%
{\isacharparenleft}{\kern0pt}simp{\isacharcomma}{\kern0pt}\ rule\ ltI{\isacharcomma}{\kern0pt}\ simp{\isacharcomma}{\kern0pt}\ simp{\isacharparenright}{\kern0pt}{\isacharplus}{\kern0pt}\isanewline
\ \ \ \isacommand{apply}\isamarkupfalse%
\ simp\isanewline
\ \ \isacommand{apply}\isamarkupfalse%
{\isacharparenleft}{\kern0pt}rule\ is{\isacharunderscore}{\kern0pt}Fix{\isacharunderscore}{\kern0pt}elem{\isacharunderscore}{\kern0pt}fm{\isacharunderscore}{\kern0pt}type{\isacharparenright}{\kern0pt}\isanewline
\ \ \isacommand{using}\isamarkupfalse%
\ assms\isanewline
\ \ \isacommand{by}\isamarkupfalse%
\ auto%
\endisatagproof
{\isafoldproof}%
%
\isadelimproof
\isanewline
%
\endisadelimproof
\isanewline
\isacommand{lemma}\isamarkupfalse%
\ sats{\isacharunderscore}{\kern0pt}is{\isacharunderscore}{\kern0pt}Fix{\isacharunderscore}{\kern0pt}fm{\isacharunderscore}{\kern0pt}iff\ {\isacharcolon}{\kern0pt}\ \isanewline
\ \ \isakeyword{fixes}\ env\ i\ j\ k\ l\ E\ v\ \isanewline
\ \ \isakeyword{assumes}\ {\isachardoublequoteopen}env\ {\isasymin}\ list{\isacharparenleft}{\kern0pt}M{\isacharparenright}{\kern0pt}{\isachardoublequoteclose}\ {\isachardoublequoteopen}i\ {\isacharless}{\kern0pt}\ length{\isacharparenleft}{\kern0pt}env{\isacharparenright}{\kern0pt}{\isachardoublequoteclose}\ {\isachardoublequoteopen}j\ {\isacharless}{\kern0pt}\ length{\isacharparenleft}{\kern0pt}env{\isacharparenright}{\kern0pt}{\isachardoublequoteclose}\ {\isachardoublequoteopen}k\ {\isacharless}{\kern0pt}\ length{\isacharparenleft}{\kern0pt}env{\isacharparenright}{\kern0pt}{\isachardoublequoteclose}\ {\isachardoublequoteopen}l\ {\isacharless}{\kern0pt}\ length{\isacharparenleft}{\kern0pt}env{\isacharparenright}{\kern0pt}{\isachardoublequoteclose}\isanewline
\ \ \ \ \ \ \ \ \ \ {\isachardoublequoteopen}nth{\isacharparenleft}{\kern0pt}i{\isacharcomma}{\kern0pt}\ env{\isacharparenright}{\kern0pt}\ {\isacharequal}{\kern0pt}\ nat{\isacharunderscore}{\kern0pt}perms{\isachardoublequoteclose}\ {\isachardoublequoteopen}nth{\isacharparenleft}{\kern0pt}j{\isacharcomma}{\kern0pt}\ env{\isacharparenright}{\kern0pt}\ {\isacharequal}{\kern0pt}\ Fn{\isachardoublequoteclose}\ {\isachardoublequoteopen}nth{\isacharparenleft}{\kern0pt}k{\isacharcomma}{\kern0pt}\ env{\isacharparenright}{\kern0pt}\ {\isacharequal}{\kern0pt}\ E{\isachardoublequoteclose}\ {\isachardoublequoteopen}nth{\isacharparenleft}{\kern0pt}l{\isacharcomma}{\kern0pt}\ env{\isacharparenright}{\kern0pt}\ {\isacharequal}{\kern0pt}\ v{\isachardoublequoteclose}\ \isanewline
\ \ \isakeyword{shows}\ {\isachardoublequoteopen}sats{\isacharparenleft}{\kern0pt}M{\isacharcomma}{\kern0pt}\ is{\isacharunderscore}{\kern0pt}Fix{\isacharunderscore}{\kern0pt}fm{\isacharparenleft}{\kern0pt}i{\isacharcomma}{\kern0pt}\ j{\isacharcomma}{\kern0pt}\ k{\isacharcomma}{\kern0pt}\ l{\isacharparenright}{\kern0pt}{\isacharcomma}{\kern0pt}\ env{\isacharparenright}{\kern0pt}\ {\isasymlongleftrightarrow}\ v\ {\isacharequal}{\kern0pt}\ Fix{\isacharparenleft}{\kern0pt}E{\isacharparenright}{\kern0pt}{\isachardoublequoteclose}\ \isanewline
%
\isadelimproof
%
\endisadelimproof
%
\isatagproof
\isacommand{proof}\isamarkupfalse%
{\isacharminus}{\kern0pt}\isanewline
\ \ \isacommand{have}\isamarkupfalse%
\ I{\isadigit{1}}{\isacharcolon}{\kern0pt}\ {\isachardoublequoteopen}sats{\isacharparenleft}{\kern0pt}M{\isacharcomma}{\kern0pt}\ is{\isacharunderscore}{\kern0pt}Fix{\isacharunderscore}{\kern0pt}fm{\isacharparenleft}{\kern0pt}i{\isacharcomma}{\kern0pt}\ j{\isacharcomma}{\kern0pt}\ k{\isacharcomma}{\kern0pt}\ l{\isacharparenright}{\kern0pt}{\isacharcomma}{\kern0pt}\ env{\isacharparenright}{\kern0pt}\ {\isasymlongleftrightarrow}\ {\isacharparenleft}{\kern0pt}{\isasymforall}u\ {\isasymin}\ M{\isachardot}{\kern0pt}\ u\ {\isasymin}\ v\ {\isasymlongleftrightarrow}\ u\ {\isasymin}\ Fix{\isacharparenleft}{\kern0pt}E{\isacharparenright}{\kern0pt}{\isacharparenright}{\kern0pt}{\isachardoublequoteclose}\ \isanewline
\ \ \ \ \isacommand{unfolding}\isamarkupfalse%
\ is{\isacharunderscore}{\kern0pt}Fix{\isacharunderscore}{\kern0pt}fm{\isacharunderscore}{\kern0pt}def\isanewline
\ \ \ \ \isacommand{apply}\isamarkupfalse%
{\isacharparenleft}{\kern0pt}rule\ iff{\isacharunderscore}{\kern0pt}trans{\isacharcomma}{\kern0pt}\ rule\ sats{\isacharunderscore}{\kern0pt}Forall{\isacharunderscore}{\kern0pt}iff{\isacharcomma}{\kern0pt}\ simp\ add{\isacharcolon}{\kern0pt}assms{\isacharcomma}{\kern0pt}\ rule\ ball{\isacharunderscore}{\kern0pt}iff{\isacharparenright}{\kern0pt}\isanewline
\ \ \ \ \isacommand{apply}\isamarkupfalse%
{\isacharparenleft}{\kern0pt}rule\ iff{\isacharunderscore}{\kern0pt}trans{\isacharcomma}{\kern0pt}\ rule\ sats{\isacharunderscore}{\kern0pt}Iff{\isacharunderscore}{\kern0pt}iff{\isacharcomma}{\kern0pt}\ simp\ add{\isacharcolon}{\kern0pt}assms{\isacharcomma}{\kern0pt}\ rule\ iff{\isacharunderscore}{\kern0pt}iff{\isacharparenright}{\kern0pt}\isanewline
\ \ \ \ \isacommand{using}\isamarkupfalse%
\ assms\ nth{\isacharunderscore}{\kern0pt}type\ lt{\isacharunderscore}{\kern0pt}nat{\isacharunderscore}{\kern0pt}in{\isacharunderscore}{\kern0pt}nat\isanewline
\ \ \ \ \ \isacommand{apply}\isamarkupfalse%
\ force\ \isanewline
\ \ \ \ \isacommand{apply}\isamarkupfalse%
{\isacharparenleft}{\kern0pt}rule\ sats{\isacharunderscore}{\kern0pt}is{\isacharunderscore}{\kern0pt}Fix{\isacharunderscore}{\kern0pt}elem{\isacharunderscore}{\kern0pt}fm{\isacharunderscore}{\kern0pt}iff{\isacharparenright}{\kern0pt}\isanewline
\ \ \ \ \isacommand{using}\isamarkupfalse%
\ assms\ nth{\isacharunderscore}{\kern0pt}type\ lt{\isacharunderscore}{\kern0pt}nat{\isacharunderscore}{\kern0pt}in{\isacharunderscore}{\kern0pt}nat\isanewline
\ \ \ \ \isacommand{by}\isamarkupfalse%
\ auto\isanewline
\ \ \isacommand{have}\isamarkupfalse%
\ I{\isadigit{2}}{\isacharcolon}{\kern0pt}\ {\isachardoublequoteopen}{\isachardot}{\kern0pt}{\isachardot}{\kern0pt}{\isachardot}{\kern0pt}\ {\isasymlongleftrightarrow}\ {\isacharparenleft}{\kern0pt}{\isasymforall}u{\isachardot}{\kern0pt}\ u\ {\isasymin}\ v\ {\isasymlongleftrightarrow}\ u\ {\isasymin}\ Fix{\isacharparenleft}{\kern0pt}E{\isacharparenright}{\kern0pt}{\isacharparenright}{\kern0pt}{\isachardoublequoteclose}\ \isanewline
\ \ \ \ \isacommand{apply}\isamarkupfalse%
{\isacharparenleft}{\kern0pt}rule\ iffI{\isacharparenright}{\kern0pt}\isanewline
\ \ \ \ \isacommand{using}\isamarkupfalse%
\ assms\ nth{\isacharunderscore}{\kern0pt}type\ lt{\isacharunderscore}{\kern0pt}nat{\isacharunderscore}{\kern0pt}in{\isacharunderscore}{\kern0pt}nat\ transM\ Fix{\isacharunderscore}{\kern0pt}in{\isacharunderscore}{\kern0pt}M\isanewline
\ \ \ \ \ \isacommand{apply}\isamarkupfalse%
\ force\ \isanewline
\ \ \ \ \isacommand{by}\isamarkupfalse%
\ auto\isanewline
\ \ \isacommand{have}\isamarkupfalse%
\ I{\isadigit{3}}{\isacharcolon}{\kern0pt}\ {\isachardoublequoteopen}{\isachardot}{\kern0pt}{\isachardot}{\kern0pt}{\isachardot}{\kern0pt}\ {\isasymlongleftrightarrow}\ v\ {\isacharequal}{\kern0pt}\ Fix{\isacharparenleft}{\kern0pt}E{\isacharparenright}{\kern0pt}{\isachardoublequoteclose}\ \isacommand{by}\isamarkupfalse%
\ auto\isanewline
\ \ \isacommand{show}\isamarkupfalse%
\ {\isacharquery}{\kern0pt}thesis\ \isacommand{using}\isamarkupfalse%
\ I{\isadigit{1}}\ I{\isadigit{2}}\ I{\isadigit{3}}\ \isacommand{by}\isamarkupfalse%
\ auto\isanewline
\isacommand{qed}\isamarkupfalse%
%
\endisatagproof
{\isafoldproof}%
%
\isadelimproof
\isanewline
%
\endisadelimproof
\isanewline
\isanewline
\isacommand{schematic{\isacharunderscore}{\kern0pt}goal}\isamarkupfalse%
\ converse{\isacharunderscore}{\kern0pt}fm{\isacharunderscore}{\kern0pt}auto{\isacharcolon}{\kern0pt}\isanewline
\ \ \isakeyword{assumes}\isanewline
\ \ \ \ {\isachardoublequoteopen}i\ {\isasymin}\ nat{\isachardoublequoteclose}\isanewline
\ \ \ \ {\isachardoublequoteopen}j\ {\isasymin}\ nat{\isachardoublequoteclose}\isanewline
\ \ \ \ {\isachardoublequoteopen}nth{\isacharparenleft}{\kern0pt}i{\isacharcomma}{\kern0pt}env{\isacharparenright}{\kern0pt}\ {\isacharequal}{\kern0pt}\ A{\isachardoublequoteclose}\isanewline
\ \ \ \ {\isachardoublequoteopen}nth{\isacharparenleft}{\kern0pt}j{\isacharcomma}{\kern0pt}env{\isacharparenright}{\kern0pt}\ {\isacharequal}{\kern0pt}\ B{\isachardoublequoteclose}\isanewline
\ \ \ \ {\isachardoublequoteopen}env\ {\isasymin}\ list{\isacharparenleft}{\kern0pt}M{\isacharparenright}{\kern0pt}{\isachardoublequoteclose}\ \isanewline
\ \ \isakeyword{shows}\ {\isachardoublequoteopen}is{\isacharunderscore}{\kern0pt}relation{\isacharparenleft}{\kern0pt}{\isacharhash}{\kern0pt}{\isacharhash}{\kern0pt}M{\isacharcomma}{\kern0pt}\ A{\isacharparenright}{\kern0pt}\ {\isasymand}\ is{\isacharunderscore}{\kern0pt}relation{\isacharparenleft}{\kern0pt}{\isacharhash}{\kern0pt}{\isacharhash}{\kern0pt}M{\isacharcomma}{\kern0pt}\ B{\isacharparenright}{\kern0pt}\ {\isasymand}\ {\isacharparenleft}{\kern0pt}{\isasymforall}x\ {\isasymin}\ M{\isachardot}{\kern0pt}\ {\isasymforall}y\ {\isasymin}\ M{\isachardot}{\kern0pt}\ {\isacharparenleft}{\kern0pt}{\isasymexists}z\ {\isasymin}\ M{\isachardot}{\kern0pt}\ pair{\isacharparenleft}{\kern0pt}{\isacharhash}{\kern0pt}{\isacharhash}{\kern0pt}M{\isacharcomma}{\kern0pt}\ x{\isacharcomma}{\kern0pt}\ y{\isacharcomma}{\kern0pt}\ z{\isacharparenright}{\kern0pt}\ {\isasymand}\ z\ {\isasymin}\ A{\isacharparenright}{\kern0pt}\ {\isasymlongleftrightarrow}\ {\isacharparenleft}{\kern0pt}{\isasymexists}w\ {\isasymin}\ M{\isachardot}{\kern0pt}\ pair{\isacharparenleft}{\kern0pt}{\isacharhash}{\kern0pt}{\isacharhash}{\kern0pt}M{\isacharcomma}{\kern0pt}\ y{\isacharcomma}{\kern0pt}\ x{\isacharcomma}{\kern0pt}\ w{\isacharparenright}{\kern0pt}\ {\isasymand}\ w\ {\isasymin}\ B{\isacharparenright}{\kern0pt}{\isacharparenright}{\kern0pt}\ {\isasymlongleftrightarrow}\ sats{\isacharparenleft}{\kern0pt}M{\isacharcomma}{\kern0pt}{\isacharquery}{\kern0pt}fm{\isacharparenleft}{\kern0pt}i{\isacharcomma}{\kern0pt}\ j{\isacharparenright}{\kern0pt}{\isacharcomma}{\kern0pt}env{\isacharparenright}{\kern0pt}{\isachardoublequoteclose}\ \isanewline
%
\isadelimproof
\ \ %
\endisadelimproof
%
\isatagproof
\isacommand{by}\isamarkupfalse%
\ {\isacharparenleft}{\kern0pt}insert\ assms\ {\isacharsemicolon}{\kern0pt}\ {\isacharparenleft}{\kern0pt}rule\ sep{\isacharunderscore}{\kern0pt}rules\ {\isacharbar}{\kern0pt}\ simp{\isacharparenright}{\kern0pt}{\isacharplus}{\kern0pt}{\isacharparenright}{\kern0pt}%
\endisatagproof
{\isafoldproof}%
%
\isadelimproof
\ \isanewline
%
\endisadelimproof
\isanewline
\isacommand{definition}\isamarkupfalse%
\ converse{\isacharunderscore}{\kern0pt}fm\ \isakeyword{where}\ {\isachardoublequoteopen}converse{\isacharunderscore}{\kern0pt}fm{\isacharparenleft}{\kern0pt}i{\isacharcomma}{\kern0pt}\ j{\isacharparenright}{\kern0pt}\ {\isasymequiv}\ And{\isacharparenleft}{\kern0pt}relation{\isacharunderscore}{\kern0pt}fm{\isacharparenleft}{\kern0pt}i{\isacharparenright}{\kern0pt}{\isacharcomma}{\kern0pt}\isanewline
\ \ \ \ \ \ \ \ \ \ \ \ \ \ \ \ \ \ \ \ \ \ \ \ \ \ \ \ \ \ \ \ \ \ \ \ \ \ \ \ \ \ \ \ \ \ \ \ \ \ \ \ And{\isacharparenleft}{\kern0pt}relation{\isacharunderscore}{\kern0pt}fm{\isacharparenleft}{\kern0pt}j{\isacharparenright}{\kern0pt}{\isacharcomma}{\kern0pt}\isanewline
\ \ \ \ \ \ \ \ \ \ \ \ \ \ \ \ \ \ \ \ \ \ \ \ \ \ \ \ \ \ \ \ \ \ \ \ \ \ \ \ \ \ \ \ \ \ \ \ \ \ \ \ \ \ \ \ Forall\isanewline
\ \ \ \ \ \ \ \ \ \ \ \ \ \ \ \ \ \ \ \ \ \ \ \ \ \ \ \ \ \ \ \ \ \ \ \ \ \ \ \ \ \ \ \ \ \ \ \ \ \ \ \ \ \ \ \ \ {\isacharparenleft}{\kern0pt}Forall\isanewline
\ \ \ \ \ \ \ \ \ \ \ \ \ \ \ \ \ \ \ \ \ \ \ \ \ \ \ \ \ \ \ \ \ \ \ \ \ \ \ \ \ \ \ \ \ \ \ \ \ \ \ \ \ \ \ \ \ \ \ {\isacharparenleft}{\kern0pt}Iff{\isacharparenleft}{\kern0pt}Exists{\isacharparenleft}{\kern0pt}And{\isacharparenleft}{\kern0pt}pair{\isacharunderscore}{\kern0pt}fm{\isacharparenleft}{\kern0pt}{\isadigit{2}}{\isacharcomma}{\kern0pt}\ {\isadigit{1}}{\isacharcomma}{\kern0pt}\ {\isadigit{0}}{\isacharparenright}{\kern0pt}{\isacharcomma}{\kern0pt}\ Member{\isacharparenleft}{\kern0pt}{\isadigit{0}}{\isacharcomma}{\kern0pt}\ succ{\isacharparenleft}{\kern0pt}succ{\isacharparenleft}{\kern0pt}succ{\isacharparenleft}{\kern0pt}i{\isacharparenright}{\kern0pt}{\isacharparenright}{\kern0pt}{\isacharparenright}{\kern0pt}{\isacharparenright}{\kern0pt}{\isacharparenright}{\kern0pt}{\isacharparenright}{\kern0pt}{\isacharcomma}{\kern0pt}\isanewline
\ \ \ \ \ \ \ \ \ \ \ \ \ \ \ \ \ \ \ \ \ \ \ \ \ \ \ \ \ \ \ \ \ \ \ \ \ \ \ \ \ \ \ \ \ \ \ \ \ \ \ \ \ \ \ \ \ \ \ \ \ \ \ \ Exists{\isacharparenleft}{\kern0pt}And{\isacharparenleft}{\kern0pt}pair{\isacharunderscore}{\kern0pt}fm{\isacharparenleft}{\kern0pt}{\isadigit{1}}{\isacharcomma}{\kern0pt}\ {\isadigit{2}}{\isacharcomma}{\kern0pt}\ {\isadigit{0}}{\isacharparenright}{\kern0pt}{\isacharcomma}{\kern0pt}\ Member{\isacharparenleft}{\kern0pt}{\isadigit{0}}{\isacharcomma}{\kern0pt}\ succ{\isacharparenleft}{\kern0pt}succ{\isacharparenleft}{\kern0pt}succ{\isacharparenleft}{\kern0pt}j{\isacharparenright}{\kern0pt}{\isacharparenright}{\kern0pt}{\isacharparenright}{\kern0pt}{\isacharparenright}{\kern0pt}{\isacharparenright}{\kern0pt}{\isacharparenright}{\kern0pt}{\isacharparenright}{\kern0pt}{\isacharparenright}{\kern0pt}{\isacharparenright}{\kern0pt}{\isacharparenright}{\kern0pt}{\isacharparenright}{\kern0pt}\ {\isachardoublequoteclose}\isanewline
\isanewline
\isacommand{lemma}\isamarkupfalse%
\ sats{\isacharunderscore}{\kern0pt}converse{\isacharunderscore}{\kern0pt}fm{\isacharunderscore}{\kern0pt}iff\ {\isacharcolon}{\kern0pt}\ \isanewline
\ \ \isakeyword{fixes}\ env\ i\ j\ A\ B\ \isanewline
\ \ \isakeyword{assumes}\ {\isachardoublequoteopen}env\ {\isasymin}\ list{\isacharparenleft}{\kern0pt}M{\isacharparenright}{\kern0pt}{\isachardoublequoteclose}\ {\isachardoublequoteopen}i\ {\isacharless}{\kern0pt}\ length{\isacharparenleft}{\kern0pt}env{\isacharparenright}{\kern0pt}{\isachardoublequoteclose}\ {\isachardoublequoteopen}j\ {\isacharless}{\kern0pt}\ length{\isacharparenleft}{\kern0pt}env{\isacharparenright}{\kern0pt}{\isachardoublequoteclose}\ {\isachardoublequoteopen}nth{\isacharparenleft}{\kern0pt}i{\isacharcomma}{\kern0pt}\ env{\isacharparenright}{\kern0pt}\ {\isacharequal}{\kern0pt}\ A{\isachardoublequoteclose}\ {\isachardoublequoteopen}nth{\isacharparenleft}{\kern0pt}j{\isacharcomma}{\kern0pt}\ env{\isacharparenright}{\kern0pt}\ {\isacharequal}{\kern0pt}\ B{\isachardoublequoteclose}\ \isanewline
\ \ \isakeyword{shows}\ {\isachardoublequoteopen}sats{\isacharparenleft}{\kern0pt}M{\isacharcomma}{\kern0pt}\ converse{\isacharunderscore}{\kern0pt}fm{\isacharparenleft}{\kern0pt}i{\isacharcomma}{\kern0pt}\ j{\isacharparenright}{\kern0pt}{\isacharcomma}{\kern0pt}\ env{\isacharparenright}{\kern0pt}\ {\isasymlongleftrightarrow}\ relation{\isacharparenleft}{\kern0pt}A{\isacharparenright}{\kern0pt}\ {\isasymand}\ relation{\isacharparenleft}{\kern0pt}B{\isacharparenright}{\kern0pt}\ {\isasymand}\ B\ {\isacharequal}{\kern0pt}\ converse{\isacharparenleft}{\kern0pt}A{\isacharparenright}{\kern0pt}{\isachardoublequoteclose}\ \isanewline
%
\isadelimproof
%
\endisadelimproof
%
\isatagproof
\isacommand{proof}\isamarkupfalse%
\ {\isacharminus}{\kern0pt}\ \isanewline
\ \ \isacommand{have}\isamarkupfalse%
\ I{\isadigit{1}}{\isacharcolon}{\kern0pt}\ {\isachardoublequoteopen}sats{\isacharparenleft}{\kern0pt}M{\isacharcomma}{\kern0pt}\ converse{\isacharunderscore}{\kern0pt}fm{\isacharparenleft}{\kern0pt}i{\isacharcomma}{\kern0pt}\ j{\isacharparenright}{\kern0pt}{\isacharcomma}{\kern0pt}\ env{\isacharparenright}{\kern0pt}\ {\isasymlongleftrightarrow}\ is{\isacharunderscore}{\kern0pt}relation{\isacharparenleft}{\kern0pt}{\isacharhash}{\kern0pt}{\isacharhash}{\kern0pt}M{\isacharcomma}{\kern0pt}\ A{\isacharparenright}{\kern0pt}\ {\isasymand}\ is{\isacharunderscore}{\kern0pt}relation{\isacharparenleft}{\kern0pt}{\isacharhash}{\kern0pt}{\isacharhash}{\kern0pt}M{\isacharcomma}{\kern0pt}\ B{\isacharparenright}{\kern0pt}\ {\isasymand}\ {\isacharparenleft}{\kern0pt}{\isasymforall}x\ {\isasymin}\ M{\isachardot}{\kern0pt}\ {\isasymforall}y\ {\isasymin}\ M{\isachardot}{\kern0pt}\ {\isacharparenleft}{\kern0pt}{\isasymexists}z\ {\isasymin}\ M{\isachardot}{\kern0pt}\ pair{\isacharparenleft}{\kern0pt}{\isacharhash}{\kern0pt}{\isacharhash}{\kern0pt}M{\isacharcomma}{\kern0pt}\ x{\isacharcomma}{\kern0pt}\ y{\isacharcomma}{\kern0pt}\ z{\isacharparenright}{\kern0pt}\ {\isasymand}\ z\ {\isasymin}\ A{\isacharparenright}{\kern0pt}\ {\isasymlongleftrightarrow}\ {\isacharparenleft}{\kern0pt}{\isasymexists}w\ {\isasymin}\ M{\isachardot}{\kern0pt}\ pair{\isacharparenleft}{\kern0pt}{\isacharhash}{\kern0pt}{\isacharhash}{\kern0pt}M{\isacharcomma}{\kern0pt}\ y{\isacharcomma}{\kern0pt}\ x{\isacharcomma}{\kern0pt}\ w{\isacharparenright}{\kern0pt}\ {\isasymand}\ w\ {\isasymin}\ B{\isacharparenright}{\kern0pt}{\isacharparenright}{\kern0pt}{\isachardoublequoteclose}\ \isanewline
\ \ \ \ \isacommand{unfolding}\isamarkupfalse%
\ converse{\isacharunderscore}{\kern0pt}fm{\isacharunderscore}{\kern0pt}def\ \isanewline
\ \ \ \ \isacommand{apply}\isamarkupfalse%
{\isacharparenleft}{\kern0pt}rule\ iff{\isacharunderscore}{\kern0pt}flip{\isacharcomma}{\kern0pt}\ rule\ converse{\isacharunderscore}{\kern0pt}fm{\isacharunderscore}{\kern0pt}auto{\isacharparenright}{\kern0pt}\isanewline
\ \ \ \ \isacommand{using}\isamarkupfalse%
\ assms\ lt{\isacharunderscore}{\kern0pt}nat{\isacharunderscore}{\kern0pt}in{\isacharunderscore}{\kern0pt}nat\ nth{\isacharunderscore}{\kern0pt}type\isanewline
\ \ \ \ \isacommand{by}\isamarkupfalse%
\ auto\isanewline
\ \ \isacommand{have}\isamarkupfalse%
\ I{\isadigit{2}}{\isacharcolon}{\kern0pt}\ {\isachardoublequoteopen}{\isachardot}{\kern0pt}{\isachardot}{\kern0pt}{\isachardot}{\kern0pt}\ {\isasymlongleftrightarrow}\ relation{\isacharparenleft}{\kern0pt}A{\isacharparenright}{\kern0pt}\ {\isasymand}\ relation{\isacharparenleft}{\kern0pt}B{\isacharparenright}{\kern0pt}\ {\isasymand}\ {\isacharparenleft}{\kern0pt}{\isasymforall}x\ {\isasymin}\ M{\isachardot}{\kern0pt}\ {\isasymforall}\ y\ {\isasymin}\ M{\isachardot}{\kern0pt}\ {\isacharless}{\kern0pt}x{\isacharcomma}{\kern0pt}\ y{\isachargreater}{\kern0pt}\ {\isasymin}\ A\ {\isasymlongleftrightarrow}\ {\isacharless}{\kern0pt}y{\isacharcomma}{\kern0pt}\ x{\isachargreater}{\kern0pt}\ {\isasymin}\ B{\isacharparenright}{\kern0pt}{\isachardoublequoteclose}\ \isanewline
\ \ \ \ \isacommand{using}\isamarkupfalse%
\ assms\ lt{\isacharunderscore}{\kern0pt}nat{\isacharunderscore}{\kern0pt}in{\isacharunderscore}{\kern0pt}nat\ nth{\isacharunderscore}{\kern0pt}type\ relation{\isacharunderscore}{\kern0pt}abs\ pair{\isacharunderscore}{\kern0pt}in{\isacharunderscore}{\kern0pt}M{\isacharunderscore}{\kern0pt}iff\isanewline
\ \ \ \ \isacommand{by}\isamarkupfalse%
\ auto\isanewline
\ \ \isacommand{have}\isamarkupfalse%
\ I{\isadigit{3}}{\isacharcolon}{\kern0pt}\ {\isachardoublequoteopen}{\isachardot}{\kern0pt}{\isachardot}{\kern0pt}{\isachardot}{\kern0pt}\ {\isasymlongleftrightarrow}\ relation{\isacharparenleft}{\kern0pt}A{\isacharparenright}{\kern0pt}\ {\isasymand}\ relation{\isacharparenleft}{\kern0pt}B{\isacharparenright}{\kern0pt}\ {\isasymand}\ {\isacharparenleft}{\kern0pt}{\isasymforall}x{\isachardot}{\kern0pt}\ {\isasymforall}y{\isachardot}{\kern0pt}\ {\isacharless}{\kern0pt}x{\isacharcomma}{\kern0pt}\ y{\isachargreater}{\kern0pt}\ {\isasymin}\ A\ {\isasymlongleftrightarrow}\ {\isacharless}{\kern0pt}y{\isacharcomma}{\kern0pt}\ x{\isachargreater}{\kern0pt}\ {\isasymin}\ B{\isacharparenright}{\kern0pt}{\isachardoublequoteclose}\ \isanewline
\ \ \ \ \isacommand{apply}\isamarkupfalse%
{\isacharparenleft}{\kern0pt}rule\ iffI{\isacharcomma}{\kern0pt}\ simp{\isacharcomma}{\kern0pt}\ rule\ allI{\isacharcomma}{\kern0pt}\ rule\ allI{\isacharcomma}{\kern0pt}\ rule\ iffI{\isacharparenright}{\kern0pt}\isanewline
\ \ \ \ \ \ \isacommand{apply}\isamarkupfalse%
{\isacharparenleft}{\kern0pt}rename{\isacharunderscore}{\kern0pt}tac\ x\ y{\isacharcomma}{\kern0pt}\ subgoal{\isacharunderscore}{\kern0pt}tac\ {\isachardoublequoteopen}{\isacharless}{\kern0pt}x{\isacharcomma}{\kern0pt}\ y{\isachargreater}{\kern0pt}\ {\isasymin}\ M{\isachardoublequoteclose}{\isacharparenright}{\kern0pt}\isanewline
\ \ \ \ \isacommand{using}\isamarkupfalse%
\ pair{\isacharunderscore}{\kern0pt}in{\isacharunderscore}{\kern0pt}M{\isacharunderscore}{\kern0pt}iff\isanewline
\ \ \ \ \ \ \ \isacommand{apply}\isamarkupfalse%
\ force\isanewline
\ \ \ \ \isacommand{using}\isamarkupfalse%
\ assms\ lt{\isacharunderscore}{\kern0pt}nat{\isacharunderscore}{\kern0pt}in{\isacharunderscore}{\kern0pt}nat\ nth{\isacharunderscore}{\kern0pt}type\ transM\ \isanewline
\ \ \ \ \ \ \isacommand{apply}\isamarkupfalse%
\ force\isanewline
\ \ \ \ \ \ \isacommand{apply}\isamarkupfalse%
{\isacharparenleft}{\kern0pt}rename{\isacharunderscore}{\kern0pt}tac\ x\ y{\isacharcomma}{\kern0pt}\ subgoal{\isacharunderscore}{\kern0pt}tac\ {\isachardoublequoteopen}{\isacharless}{\kern0pt}y{\isacharcomma}{\kern0pt}\ x{\isachargreater}{\kern0pt}\ {\isasymin}\ M{\isachardoublequoteclose}{\isacharparenright}{\kern0pt}\isanewline
\ \ \ \ \isacommand{using}\isamarkupfalse%
\ pair{\isacharunderscore}{\kern0pt}in{\isacharunderscore}{\kern0pt}M{\isacharunderscore}{\kern0pt}iff\isanewline
\ \ \ \ \ \ \ \isacommand{apply}\isamarkupfalse%
\ force\isanewline
\ \ \ \ \isacommand{using}\isamarkupfalse%
\ assms\ lt{\isacharunderscore}{\kern0pt}nat{\isacharunderscore}{\kern0pt}in{\isacharunderscore}{\kern0pt}nat\ nth{\isacharunderscore}{\kern0pt}type\ transM\ \isanewline
\ \ \ \ \ \isacommand{apply}\isamarkupfalse%
\ force\isanewline
\ \ \ \ \isacommand{by}\isamarkupfalse%
\ auto\isanewline
\ \ \isacommand{have}\isamarkupfalse%
\ I{\isadigit{4}}{\isacharcolon}{\kern0pt}\ {\isachardoublequoteopen}{\isachardot}{\kern0pt}{\isachardot}{\kern0pt}{\isachardot}{\kern0pt}\ {\isasymlongleftrightarrow}\ relation{\isacharparenleft}{\kern0pt}A{\isacharparenright}{\kern0pt}\ {\isasymand}\ relation{\isacharparenleft}{\kern0pt}B{\isacharparenright}{\kern0pt}\ {\isasymand}\ B\ {\isacharequal}{\kern0pt}\ converse{\isacharparenleft}{\kern0pt}A{\isacharparenright}{\kern0pt}{\isachardoublequoteclose}\isanewline
\ \ \ \ \isacommand{apply}\isamarkupfalse%
\ {\isacharparenleft}{\kern0pt}rule\ iffI{\isacharparenright}{\kern0pt}\isanewline
\ \ \ \ \ \isacommand{apply}\isamarkupfalse%
{\isacharparenleft}{\kern0pt}simp{\isacharcomma}{\kern0pt}\ rule\ equality{\isacharunderscore}{\kern0pt}iffI{\isacharcomma}{\kern0pt}\ rule\ iffI{\isacharparenright}{\kern0pt}\isanewline
\ \ \ \ \ \ \isacommand{apply}\isamarkupfalse%
{\isacharparenleft}{\kern0pt}rename{\isacharunderscore}{\kern0pt}tac\ x{\isacharcomma}{\kern0pt}\ subgoal{\isacharunderscore}{\kern0pt}tac\ {\isachardoublequoteopen}{\isasymexists}y\ z{\isachardot}{\kern0pt}\ x\ {\isacharequal}{\kern0pt}\ {\isacharless}{\kern0pt}y{\isacharcomma}{\kern0pt}\ z{\isachargreater}{\kern0pt}{\isachardoublequoteclose}{\isacharcomma}{\kern0pt}\ force{\isacharcomma}{\kern0pt}\ simp\ add{\isacharcolon}{\kern0pt}relation{\isacharunderscore}{\kern0pt}def{\isacharparenright}{\kern0pt}\isanewline
\ \ \ \ \ \isacommand{apply}\isamarkupfalse%
{\isacharparenleft}{\kern0pt}rename{\isacharunderscore}{\kern0pt}tac\ x{\isacharcomma}{\kern0pt}\ subgoal{\isacharunderscore}{\kern0pt}tac\ {\isachardoublequoteopen}{\isasymexists}y\ z{\isachardot}{\kern0pt}\ x\ {\isacharequal}{\kern0pt}\ {\isacharless}{\kern0pt}y{\isacharcomma}{\kern0pt}\ z{\isachargreater}{\kern0pt}{\isachardoublequoteclose}{\isacharcomma}{\kern0pt}\ force{\isacharcomma}{\kern0pt}\ simp\ add{\isacharcolon}{\kern0pt}relation{\isacharunderscore}{\kern0pt}def{\isacharcomma}{\kern0pt}\ force{\isacharparenright}{\kern0pt}\isanewline
\ \ \ \ \isacommand{by}\isamarkupfalse%
\ auto\isanewline
\ \ \isacommand{show}\isamarkupfalse%
\ {\isacharquery}{\kern0pt}thesis\ \isacommand{using}\isamarkupfalse%
\ I{\isadigit{1}}\ I{\isadigit{2}}\ I{\isadigit{3}}\ I{\isadigit{4}}\ \isacommand{by}\isamarkupfalse%
\ auto\ \isanewline
\isacommand{qed}\isamarkupfalse%
%
\endisatagproof
{\isafoldproof}%
%
\isadelimproof
\isanewline
%
\endisadelimproof
\ \ \ \isanewline
\isacommand{schematic{\isacharunderscore}{\kern0pt}goal}\isamarkupfalse%
\ closed{\isacharunderscore}{\kern0pt}under{\isacharunderscore}{\kern0pt}comp{\isacharunderscore}{\kern0pt}fm{\isacharunderscore}{\kern0pt}auto{\isacharcolon}{\kern0pt}\isanewline
\ \ \isakeyword{assumes}\isanewline
\ \ \ \ {\isachardoublequoteopen}i\ {\isasymin}\ nat{\isachardoublequoteclose}\isanewline
\ \ \ \ {\isachardoublequoteopen}nth{\isacharparenleft}{\kern0pt}i{\isacharcomma}{\kern0pt}env{\isacharparenright}{\kern0pt}\ {\isacharequal}{\kern0pt}\ A{\isachardoublequoteclose}\isanewline
\ \ \ \ {\isachardoublequoteopen}env\ {\isasymin}\ list{\isacharparenleft}{\kern0pt}M{\isacharparenright}{\kern0pt}{\isachardoublequoteclose}\ \isanewline
\ \isakeyword{shows}\ \isanewline
\ \ \ \ {\isachardoublequoteopen}{\isacharparenleft}{\kern0pt}{\isasymforall}x\ {\isasymin}\ M{\isachardot}{\kern0pt}\ {\isasymforall}y\ {\isasymin}\ M{\isachardot}{\kern0pt}\ x\ {\isasymin}\ A\ {\isasymlongrightarrow}\ y\ {\isasymin}\ A\ {\isasymlongrightarrow}\ {\isacharparenleft}{\kern0pt}{\isasymexists}z\ {\isasymin}\ M{\isachardot}{\kern0pt}\ composition{\isacharparenleft}{\kern0pt}{\isacharhash}{\kern0pt}{\isacharhash}{\kern0pt}M{\isacharcomma}{\kern0pt}\ x{\isacharcomma}{\kern0pt}\ y{\isacharcomma}{\kern0pt}\ z{\isacharparenright}{\kern0pt}\ {\isasymand}\ z\ {\isasymin}\ A{\isacharparenright}{\kern0pt}{\isacharparenright}{\kern0pt}\isanewline
\ \ \ \ \ {\isasymlongleftrightarrow}\ sats{\isacharparenleft}{\kern0pt}M{\isacharcomma}{\kern0pt}{\isacharquery}{\kern0pt}fm{\isacharparenleft}{\kern0pt}i{\isacharparenright}{\kern0pt}{\isacharcomma}{\kern0pt}env{\isacharparenright}{\kern0pt}{\isachardoublequoteclose}\ \isanewline
%
\isadelimproof
\ \ %
\endisadelimproof
%
\isatagproof
\isacommand{unfolding}\isamarkupfalse%
\ is{\isacharunderscore}{\kern0pt}converse{\isacharunderscore}{\kern0pt}def\ \isanewline
\ \ \isacommand{by}\isamarkupfalse%
\ {\isacharparenleft}{\kern0pt}insert\ assms\ {\isacharsemicolon}{\kern0pt}\ {\isacharparenleft}{\kern0pt}rule\ sep{\isacharunderscore}{\kern0pt}rules\ {\isacharbar}{\kern0pt}\ simp{\isacharparenright}{\kern0pt}{\isacharplus}{\kern0pt}{\isacharparenright}{\kern0pt}%
\endisatagproof
{\isafoldproof}%
%
\isadelimproof
\ \isanewline
%
\endisadelimproof
\isacommand{definition}\isamarkupfalse%
\ closed{\isacharunderscore}{\kern0pt}under{\isacharunderscore}{\kern0pt}comp{\isacharunderscore}{\kern0pt}and{\isacharunderscore}{\kern0pt}converse{\isacharunderscore}{\kern0pt}fm\ \isakeyword{where}\ \isanewline
\ \ {\isachardoublequoteopen}closed{\isacharunderscore}{\kern0pt}under{\isacharunderscore}{\kern0pt}comp{\isacharunderscore}{\kern0pt}and{\isacharunderscore}{\kern0pt}converse{\isacharunderscore}{\kern0pt}fm{\isacharparenleft}{\kern0pt}i{\isacharparenright}{\kern0pt}\ {\isasymequiv}\ \isanewline
\ \ \ \ \ \ And{\isacharparenleft}{\kern0pt}Forall{\isacharparenleft}{\kern0pt}Forall{\isacharparenleft}{\kern0pt}Implies{\isacharparenleft}{\kern0pt}Member{\isacharparenleft}{\kern0pt}{\isadigit{1}}{\isacharcomma}{\kern0pt}\ succ{\isacharparenleft}{\kern0pt}succ{\isacharparenleft}{\kern0pt}i{\isacharparenright}{\kern0pt}{\isacharparenright}{\kern0pt}{\isacharparenright}{\kern0pt}{\isacharcomma}{\kern0pt}\isanewline
\ \ \ \ \ \ \ \ \ \ \ \ \ \ \ \ \ \ \ \ \ \ \ \ Implies{\isacharparenleft}{\kern0pt}Member{\isacharparenleft}{\kern0pt}{\isadigit{0}}{\isacharcomma}{\kern0pt}\ succ{\isacharparenleft}{\kern0pt}succ{\isacharparenleft}{\kern0pt}i{\isacharparenright}{\kern0pt}{\isacharparenright}{\kern0pt}{\isacharparenright}{\kern0pt}{\isacharcomma}{\kern0pt}\ \isanewline
\ \ \ \ \ \ \ \ \ \ \ \ \ \ \ \ \ \ \ \ \ \ \ \ Exists{\isacharparenleft}{\kern0pt}And{\isacharparenleft}{\kern0pt}composition{\isacharunderscore}{\kern0pt}fm{\isacharparenleft}{\kern0pt}{\isadigit{2}}{\isacharcomma}{\kern0pt}\ {\isadigit{1}}{\isacharcomma}{\kern0pt}\ {\isadigit{0}}{\isacharparenright}{\kern0pt}{\isacharcomma}{\kern0pt}\ Member{\isacharparenleft}{\kern0pt}{\isadigit{0}}{\isacharcomma}{\kern0pt}\ succ{\isacharparenleft}{\kern0pt}succ{\isacharparenleft}{\kern0pt}succ{\isacharparenleft}{\kern0pt}i{\isacharparenright}{\kern0pt}{\isacharparenright}{\kern0pt}{\isacharparenright}{\kern0pt}{\isacharparenright}{\kern0pt}{\isacharparenright}{\kern0pt}{\isacharparenright}{\kern0pt}{\isacharparenright}{\kern0pt}{\isacharparenright}{\kern0pt}{\isacharparenright}{\kern0pt}{\isacharparenright}{\kern0pt}{\isacharcomma}{\kern0pt}\ \isanewline
\ \ \ \ \ \ \ \ \ \ Forall{\isacharparenleft}{\kern0pt}Implies{\isacharparenleft}{\kern0pt}Member{\isacharparenleft}{\kern0pt}{\isadigit{0}}{\isacharcomma}{\kern0pt}\ succ{\isacharparenleft}{\kern0pt}i{\isacharparenright}{\kern0pt}{\isacharparenright}{\kern0pt}{\isacharcomma}{\kern0pt}\ Exists{\isacharparenleft}{\kern0pt}And{\isacharparenleft}{\kern0pt}Member{\isacharparenleft}{\kern0pt}{\isadigit{0}}{\isacharcomma}{\kern0pt}\ succ{\isacharparenleft}{\kern0pt}succ{\isacharparenleft}{\kern0pt}i{\isacharparenright}{\kern0pt}{\isacharparenright}{\kern0pt}{\isacharparenright}{\kern0pt}{\isacharcomma}{\kern0pt}\ converse{\isacharunderscore}{\kern0pt}fm{\isacharparenleft}{\kern0pt}{\isadigit{1}}{\isacharcomma}{\kern0pt}\ {\isadigit{0}}{\isacharparenright}{\kern0pt}{\isacharparenright}{\kern0pt}{\isacharparenright}{\kern0pt}{\isacharparenright}{\kern0pt}{\isacharparenright}{\kern0pt}{\isacharparenright}{\kern0pt}{\isachardoublequoteclose}\ \isanewline
\isanewline
\isacommand{lemma}\isamarkupfalse%
\ sats{\isacharunderscore}{\kern0pt}closed{\isacharunderscore}{\kern0pt}under{\isacharunderscore}{\kern0pt}comp{\isacharunderscore}{\kern0pt}and{\isacharunderscore}{\kern0pt}converse{\isacharunderscore}{\kern0pt}fm{\isacharunderscore}{\kern0pt}iff\ {\isacharcolon}{\kern0pt}\ \isanewline
\ \ \isakeyword{fixes}\ env\ i\ A\isanewline
\ \ \isakeyword{assumes}\ {\isachardoublequoteopen}env\ {\isasymin}\ list{\isacharparenleft}{\kern0pt}M{\isacharparenright}{\kern0pt}{\isachardoublequoteclose}\ {\isachardoublequoteopen}i\ {\isacharless}{\kern0pt}\ length{\isacharparenleft}{\kern0pt}env{\isacharparenright}{\kern0pt}{\isachardoublequoteclose}\ {\isachardoublequoteopen}nth{\isacharparenleft}{\kern0pt}i{\isacharcomma}{\kern0pt}\ env{\isacharparenright}{\kern0pt}\ {\isacharequal}{\kern0pt}\ A{\isachardoublequoteclose}\ {\isachardoublequoteopen}{\isasymAnd}x{\isachardot}{\kern0pt}\ x\ {\isasymin}\ A\ {\isasymLongrightarrow}\ relation{\isacharparenleft}{\kern0pt}x{\isacharparenright}{\kern0pt}{\isachardoublequoteclose}\ \isanewline
\ \ \isakeyword{shows}\ {\isachardoublequoteopen}sats{\isacharparenleft}{\kern0pt}M{\isacharcomma}{\kern0pt}\ closed{\isacharunderscore}{\kern0pt}under{\isacharunderscore}{\kern0pt}comp{\isacharunderscore}{\kern0pt}and{\isacharunderscore}{\kern0pt}converse{\isacharunderscore}{\kern0pt}fm{\isacharparenleft}{\kern0pt}i{\isacharparenright}{\kern0pt}{\isacharcomma}{\kern0pt}\ env{\isacharparenright}{\kern0pt}\ {\isasymlongleftrightarrow}\ {\isacharparenleft}{\kern0pt}{\isasymforall}x\ {\isasymin}\ A{\isachardot}{\kern0pt}\ {\isasymforall}y\ {\isasymin}\ A{\isachardot}{\kern0pt}\ x\ O\ y\ {\isasymin}\ A{\isacharparenright}{\kern0pt}\ {\isasymand}\ {\isacharparenleft}{\kern0pt}{\isasymforall}x\ {\isasymin}\ A{\isachardot}{\kern0pt}\ converse{\isacharparenleft}{\kern0pt}x{\isacharparenright}{\kern0pt}\ {\isasymin}\ A{\isacharparenright}{\kern0pt}{\isachardoublequoteclose}\ \isanewline
%
\isadelimproof
\isanewline
%
\endisadelimproof
%
\isatagproof
\isacommand{proof}\isamarkupfalse%
\ {\isacharminus}{\kern0pt}\ \isanewline
\ \ \isacommand{have}\isamarkupfalse%
\ I{\isadigit{1}}{\isacharcolon}{\kern0pt}{\isachardoublequoteopen}sats{\isacharparenleft}{\kern0pt}M{\isacharcomma}{\kern0pt}\ closed{\isacharunderscore}{\kern0pt}under{\isacharunderscore}{\kern0pt}comp{\isacharunderscore}{\kern0pt}and{\isacharunderscore}{\kern0pt}converse{\isacharunderscore}{\kern0pt}fm{\isacharparenleft}{\kern0pt}i{\isacharparenright}{\kern0pt}{\isacharcomma}{\kern0pt}\ env{\isacharparenright}{\kern0pt}\ {\isasymlongleftrightarrow}\ \isanewline
\ \ \ \ \ \ \ \ {\isacharparenleft}{\kern0pt}{\isasymforall}x\ {\isasymin}\ M{\isachardot}{\kern0pt}\ {\isasymforall}y\ {\isasymin}\ M{\isachardot}{\kern0pt}\ x\ {\isasymin}\ A\ {\isasymlongrightarrow}\ y\ {\isasymin}\ A\ {\isasymlongrightarrow}\ {\isacharparenleft}{\kern0pt}{\isasymexists}z\ {\isasymin}\ M{\isachardot}{\kern0pt}\ composition{\isacharparenleft}{\kern0pt}{\isacharhash}{\kern0pt}{\isacharhash}{\kern0pt}M{\isacharcomma}{\kern0pt}\ x{\isacharcomma}{\kern0pt}\ y{\isacharcomma}{\kern0pt}\ z{\isacharparenright}{\kern0pt}\ {\isasymand}\ z\ {\isasymin}\ A{\isacharparenright}{\kern0pt}{\isacharparenright}{\kern0pt}\ {\isasymand}\ \isanewline
\ \ \ \ \ \ \ \ {\isacharparenleft}{\kern0pt}{\isasymforall}x\ {\isasymin}\ M{\isachardot}{\kern0pt}\ x\ {\isasymin}\ A\ {\isasymlongrightarrow}\ {\isacharparenleft}{\kern0pt}{\isasymexists}y\ {\isasymin}\ M{\isachardot}{\kern0pt}\ y\ {\isasymin}\ A\ {\isasymand}\ y\ {\isacharequal}{\kern0pt}\ converse{\isacharparenleft}{\kern0pt}x{\isacharparenright}{\kern0pt}{\isacharparenright}{\kern0pt}{\isacharparenright}{\kern0pt}{\isachardoublequoteclose}\isanewline
\ \ \ \ \isacommand{unfolding}\isamarkupfalse%
\ closed{\isacharunderscore}{\kern0pt}under{\isacharunderscore}{\kern0pt}comp{\isacharunderscore}{\kern0pt}and{\isacharunderscore}{\kern0pt}converse{\isacharunderscore}{\kern0pt}fm{\isacharunderscore}{\kern0pt}def\ \isanewline
\ \ \ \ \isacommand{apply}\isamarkupfalse%
{\isacharparenleft}{\kern0pt}rule\ iff{\isacharunderscore}{\kern0pt}trans{\isacharcomma}{\kern0pt}\ rule\ sats{\isacharunderscore}{\kern0pt}And{\isacharunderscore}{\kern0pt}iff{\isacharcomma}{\kern0pt}\ simp\ add{\isacharcolon}{\kern0pt}assms{\isacharcomma}{\kern0pt}\ rule\ iff{\isacharunderscore}{\kern0pt}conjI{\isacharparenright}{\kern0pt}\isanewline
\ \ \ \ \isacommand{apply}\isamarkupfalse%
{\isacharparenleft}{\kern0pt}rule\ iff{\isacharunderscore}{\kern0pt}flip{\isacharcomma}{\kern0pt}\ rule\ closed{\isacharunderscore}{\kern0pt}under{\isacharunderscore}{\kern0pt}comp{\isacharunderscore}{\kern0pt}fm{\isacharunderscore}{\kern0pt}auto{\isacharparenright}{\kern0pt}\isanewline
\ \ \ \ \isacommand{using}\isamarkupfalse%
\ assms\ lt{\isacharunderscore}{\kern0pt}nat{\isacharunderscore}{\kern0pt}in{\isacharunderscore}{\kern0pt}nat\ \isanewline
\ \ \ \ \ \ \ \isacommand{apply}\isamarkupfalse%
\ auto{\isacharbrackleft}{\kern0pt}{\isadigit{3}}{\isacharbrackright}{\kern0pt}\isanewline
\ \ \ \ \isacommand{apply}\isamarkupfalse%
{\isacharparenleft}{\kern0pt}rule\ iff{\isacharunderscore}{\kern0pt}trans{\isacharcomma}{\kern0pt}\ rule\ sats{\isacharunderscore}{\kern0pt}Forall{\isacharunderscore}{\kern0pt}iff{\isacharcomma}{\kern0pt}\ simp\ add{\isacharcolon}{\kern0pt}assms{\isacharcomma}{\kern0pt}\ rule\ ball{\isacharunderscore}{\kern0pt}iff{\isacharparenright}{\kern0pt}\isanewline
\ \ \ \ \isacommand{apply}\isamarkupfalse%
{\isacharparenleft}{\kern0pt}rule\ iff{\isacharunderscore}{\kern0pt}trans{\isacharcomma}{\kern0pt}\ rule\ sats{\isacharunderscore}{\kern0pt}Implies{\isacharunderscore}{\kern0pt}iff{\isacharcomma}{\kern0pt}\ simp\ add{\isacharcolon}{\kern0pt}assms{\isacharcomma}{\kern0pt}\ rule\ imp{\isacharunderscore}{\kern0pt}iff{\isadigit{2}}{\isacharparenright}{\kern0pt}\isanewline
\ \ \ \ \isacommand{using}\isamarkupfalse%
\ assms\ lt{\isacharunderscore}{\kern0pt}nat{\isacharunderscore}{\kern0pt}in{\isacharunderscore}{\kern0pt}nat\ \isanewline
\ \ \ \ \ \isacommand{apply}\isamarkupfalse%
\ simp\isanewline
\ \ \ \ \isacommand{apply}\isamarkupfalse%
{\isacharparenleft}{\kern0pt}rule\ iff{\isacharunderscore}{\kern0pt}trans{\isacharcomma}{\kern0pt}\ rule\ sats{\isacharunderscore}{\kern0pt}Exists{\isacharunderscore}{\kern0pt}iff{\isacharcomma}{\kern0pt}\ simp\ add{\isacharcolon}{\kern0pt}assms{\isacharcomma}{\kern0pt}\ rule\ bex{\isacharunderscore}{\kern0pt}iff{\isacharparenright}{\kern0pt}\isanewline
\ \ \ \ \isacommand{apply}\isamarkupfalse%
{\isacharparenleft}{\kern0pt}rule\ iff{\isacharunderscore}{\kern0pt}trans{\isacharcomma}{\kern0pt}\ rule\ sats{\isacharunderscore}{\kern0pt}And{\isacharunderscore}{\kern0pt}iff{\isacharcomma}{\kern0pt}\ simp\ add{\isacharcolon}{\kern0pt}assms{\isacharcomma}{\kern0pt}\ rule\ iff{\isacharunderscore}{\kern0pt}conjI{\isadigit{2}}{\isacharparenright}{\kern0pt}\isanewline
\ \ \ \ \isacommand{using}\isamarkupfalse%
\ assms\ lt{\isacharunderscore}{\kern0pt}nat{\isacharunderscore}{\kern0pt}in{\isacharunderscore}{\kern0pt}nat\ \isanewline
\ \ \ \ \ \isacommand{apply}\isamarkupfalse%
\ simp\isanewline
\ \ \ \ \isacommand{apply}\isamarkupfalse%
{\isacharparenleft}{\kern0pt}rule\ iff{\isacharunderscore}{\kern0pt}trans{\isacharcomma}{\kern0pt}\ rule\ sats{\isacharunderscore}{\kern0pt}converse{\isacharunderscore}{\kern0pt}fm{\isacharunderscore}{\kern0pt}iff{\isacharparenright}{\kern0pt}\ \ \isanewline
\ \ \ \ \isacommand{using}\isamarkupfalse%
\ assms\ lt{\isacharunderscore}{\kern0pt}nat{\isacharunderscore}{\kern0pt}in{\isacharunderscore}{\kern0pt}nat\isanewline
\ \ \ \ \isacommand{by}\isamarkupfalse%
\ auto\isanewline
\ \ \isacommand{also}\isamarkupfalse%
\ \isacommand{have}\isamarkupfalse%
\ I{\isadigit{2}}{\isacharcolon}{\kern0pt}\ {\isachardoublequoteopen}{\isachardot}{\kern0pt}{\isachardot}{\kern0pt}{\isachardot}{\kern0pt}\ {\isasymlongleftrightarrow}\ {\isacharparenleft}{\kern0pt}{\isasymforall}x\ {\isasymin}\ A{\isachardot}{\kern0pt}\ {\isasymforall}y\ {\isasymin}\ A{\isachardot}{\kern0pt}\ x\ O\ y\ {\isasymin}\ A{\isacharparenright}{\kern0pt}\ {\isasymand}\ {\isacharparenleft}{\kern0pt}{\isasymforall}x\ {\isasymin}\ A{\isachardot}{\kern0pt}\ converse{\isacharparenleft}{\kern0pt}x{\isacharparenright}{\kern0pt}\ {\isasymin}\ A{\isacharparenright}{\kern0pt}{\isachardoublequoteclose}\ \isanewline
\ \ \ \ \isacommand{apply}\isamarkupfalse%
{\isacharparenleft}{\kern0pt}rule\ iff{\isacharunderscore}{\kern0pt}conjI{\isacharparenright}{\kern0pt}\isanewline
\ \ \ \ \isacommand{using}\isamarkupfalse%
\ transM\ nth{\isacharunderscore}{\kern0pt}type\ lt{\isacharunderscore}{\kern0pt}nat{\isacharunderscore}{\kern0pt}in{\isacharunderscore}{\kern0pt}nat\ assms\ comp{\isacharunderscore}{\kern0pt}closed\isanewline
\ \ \ \ \ \isacommand{apply}\isamarkupfalse%
\ force\isanewline
\ \ \ \ \isacommand{using}\isamarkupfalse%
\ transM\ nth{\isacharunderscore}{\kern0pt}type\ lt{\isacharunderscore}{\kern0pt}nat{\isacharunderscore}{\kern0pt}in{\isacharunderscore}{\kern0pt}nat\ assms\ comp{\isacharunderscore}{\kern0pt}closed\ converse{\isacharunderscore}{\kern0pt}closed\ \isanewline
\ \ \ \ \isacommand{apply}\isamarkupfalse%
\ force\isanewline
\ \ \ \ \isacommand{done}\isamarkupfalse%
\isanewline
\ \ \isacommand{show}\isamarkupfalse%
\ {\isacharquery}{\kern0pt}thesis\ \isacommand{using}\isamarkupfalse%
\ I{\isadigit{1}}\ I{\isadigit{2}}\ \isacommand{by}\isamarkupfalse%
\ auto\isanewline
\isacommand{qed}\isamarkupfalse%
%
\endisatagproof
{\isafoldproof}%
%
\isadelimproof
\isanewline
%
\endisadelimproof
\isanewline
\isacommand{lemma}\isamarkupfalse%
\ closed{\isacharunderscore}{\kern0pt}under{\isacharunderscore}{\kern0pt}comp{\isacharunderscore}{\kern0pt}and{\isacharunderscore}{\kern0pt}converse{\isacharunderscore}{\kern0pt}fm{\isacharunderscore}{\kern0pt}type\ {\isacharcolon}{\kern0pt}\ \isanewline
\ \ \isakeyword{fixes}\ i\ \isanewline
\ \ \isakeyword{assumes}\ {\isachardoublequoteopen}i\ {\isasymin}\ nat{\isachardoublequoteclose}\ \isanewline
\ \ \isakeyword{shows}\ {\isachardoublequoteopen}closed{\isacharunderscore}{\kern0pt}under{\isacharunderscore}{\kern0pt}comp{\isacharunderscore}{\kern0pt}and{\isacharunderscore}{\kern0pt}converse{\isacharunderscore}{\kern0pt}fm{\isacharparenleft}{\kern0pt}i{\isacharparenright}{\kern0pt}\ {\isasymin}\ formula{\isachardoublequoteclose}\ \isanewline
%
\isadelimproof
\ \ %
\endisadelimproof
%
\isatagproof
\isacommand{unfolding}\isamarkupfalse%
\ closed{\isacharunderscore}{\kern0pt}under{\isacharunderscore}{\kern0pt}comp{\isacharunderscore}{\kern0pt}and{\isacharunderscore}{\kern0pt}converse{\isacharunderscore}{\kern0pt}fm{\isacharunderscore}{\kern0pt}def\ converse{\isacharunderscore}{\kern0pt}fm{\isacharunderscore}{\kern0pt}def\isanewline
\ \ \isacommand{using}\isamarkupfalse%
\ assms\isanewline
\ \ \isacommand{by}\isamarkupfalse%
\ auto%
\endisatagproof
{\isafoldproof}%
%
\isadelimproof
\isanewline
%
\endisadelimproof
\isanewline
\isacommand{lemma}\isamarkupfalse%
\ arity{\isacharunderscore}{\kern0pt}closed{\isacharunderscore}{\kern0pt}under{\isacharunderscore}{\kern0pt}comp{\isacharunderscore}{\kern0pt}and{\isacharunderscore}{\kern0pt}converse{\isacharunderscore}{\kern0pt}fm\ {\isacharcolon}{\kern0pt}\ \isanewline
\ \ \isakeyword{fixes}\ i\ \isanewline
\ \ \isakeyword{assumes}\ {\isachardoublequoteopen}i\ {\isasymin}\ nat{\isachardoublequoteclose}\ \isanewline
\ \ \isakeyword{shows}\ {\isachardoublequoteopen}arity{\isacharparenleft}{\kern0pt}closed{\isacharunderscore}{\kern0pt}under{\isacharunderscore}{\kern0pt}comp{\isacharunderscore}{\kern0pt}and{\isacharunderscore}{\kern0pt}converse{\isacharunderscore}{\kern0pt}fm{\isacharparenleft}{\kern0pt}i{\isacharparenright}{\kern0pt}{\isacharparenright}{\kern0pt}\ {\isasymle}\ succ{\isacharparenleft}{\kern0pt}i{\isacharparenright}{\kern0pt}{\isachardoublequoteclose}\ \isanewline
%
\isadelimproof
\ \ %
\endisadelimproof
%
\isatagproof
\isacommand{unfolding}\isamarkupfalse%
\ closed{\isacharunderscore}{\kern0pt}under{\isacharunderscore}{\kern0pt}comp{\isacharunderscore}{\kern0pt}and{\isacharunderscore}{\kern0pt}converse{\isacharunderscore}{\kern0pt}fm{\isacharunderscore}{\kern0pt}def\ converse{\isacharunderscore}{\kern0pt}fm{\isacharunderscore}{\kern0pt}def\isanewline
\ \ \isacommand{using}\isamarkupfalse%
\ assms\ \isanewline
\ \ \isacommand{apply}\isamarkupfalse%
\ {\isacharparenleft}{\kern0pt}simp\ del{\isacharcolon}{\kern0pt}FOL{\isacharunderscore}{\kern0pt}sats{\isacharunderscore}{\kern0pt}iff\ pair{\isacharunderscore}{\kern0pt}abs\ add{\isacharcolon}{\kern0pt}\ fm{\isacharunderscore}{\kern0pt}defs\ nat{\isacharunderscore}{\kern0pt}simp{\isacharunderscore}{\kern0pt}union{\isacharparenright}{\kern0pt}\isanewline
\ \ \isacommand{done}\isamarkupfalse%
%
\endisatagproof
{\isafoldproof}%
%
\isadelimproof
\isanewline
%
\endisadelimproof
\isanewline
\isanewline
\isacommand{definition}\isamarkupfalse%
\ Fn{\isacharunderscore}{\kern0pt}perms{\isacharunderscore}{\kern0pt}filter\ \isakeyword{where}\ {\isachardoublequoteopen}Fn{\isacharunderscore}{\kern0pt}perms{\isacharunderscore}{\kern0pt}filter\ {\isasymequiv}\ {\isacharbraceleft}{\kern0pt}\ H\ {\isasymin}\ forcing{\isacharunderscore}{\kern0pt}data{\isacharunderscore}{\kern0pt}partial{\isachardot}{\kern0pt}P{\isacharunderscore}{\kern0pt}auto{\isacharunderscore}{\kern0pt}subgroups{\isacharparenleft}{\kern0pt}Fn{\isacharcomma}{\kern0pt}\ Fn{\isacharunderscore}{\kern0pt}leq{\isacharcomma}{\kern0pt}\ M{\isacharcomma}{\kern0pt}\ Fn{\isacharunderscore}{\kern0pt}perms{\isacharparenright}{\kern0pt}{\isachardot}{\kern0pt}\ \ {\isasymexists}E\ {\isasymin}\ Pow{\isacharparenleft}{\kern0pt}nat{\isacharparenright}{\kern0pt}\ {\isasyminter}\ M{\isachardot}{\kern0pt}\ finite{\isacharunderscore}{\kern0pt}M{\isacharparenleft}{\kern0pt}E{\isacharparenright}{\kern0pt}\ {\isasymand}\ Fix{\isacharparenleft}{\kern0pt}E{\isacharparenright}{\kern0pt}\ {\isasymsubseteq}\ H\ {\isacharbraceright}{\kern0pt}{\isachardoublequoteclose}\ \isanewline
\isanewline
\isacommand{lemma}\isamarkupfalse%
\ Fn{\isacharunderscore}{\kern0pt}perms{\isacharunderscore}{\kern0pt}filter{\isacharunderscore}{\kern0pt}in{\isacharunderscore}{\kern0pt}M\ {\isacharcolon}{\kern0pt}\ {\isachardoublequoteopen}Fn{\isacharunderscore}{\kern0pt}perms{\isacharunderscore}{\kern0pt}filter\ {\isasymin}\ M{\isachardoublequoteclose}\ \isanewline
%
\isadelimproof
%
\endisadelimproof
%
\isatagproof
\isacommand{proof}\isamarkupfalse%
\ {\isacharminus}{\kern0pt}\ \isanewline
\ \ \isacommand{define}\isamarkupfalse%
\ X\ \isakeyword{where}\ {\isachardoublequoteopen}X\ {\isasymequiv}\ {\isacharbraceleft}{\kern0pt}\ H\ {\isasymin}\ Pow{\isacharparenleft}{\kern0pt}Fn{\isacharunderscore}{\kern0pt}perms{\isacharparenright}{\kern0pt}\ {\isasyminter}\ M{\isachardot}{\kern0pt}\ sats{\isacharparenleft}{\kern0pt}M{\isacharcomma}{\kern0pt}\ And{\isacharparenleft}{\kern0pt}closed{\isacharunderscore}{\kern0pt}under{\isacharunderscore}{\kern0pt}comp{\isacharunderscore}{\kern0pt}and{\isacharunderscore}{\kern0pt}converse{\isacharunderscore}{\kern0pt}fm{\isacharparenleft}{\kern0pt}{\isadigit{0}}{\isacharparenright}{\kern0pt}{\isacharcomma}{\kern0pt}\ Exists{\isacharparenleft}{\kern0pt}Exists{\isacharparenleft}{\kern0pt}And{\isacharparenleft}{\kern0pt}Member{\isacharparenleft}{\kern0pt}{\isadigit{0}}{\isacharcomma}{\kern0pt}\ {\isadigit{3}}{\isacharparenright}{\kern0pt}{\isacharcomma}{\kern0pt}\ And{\isacharparenleft}{\kern0pt}finite{\isacharunderscore}{\kern0pt}M{\isacharunderscore}{\kern0pt}fm{\isacharparenleft}{\kern0pt}{\isadigit{6}}{\isacharcomma}{\kern0pt}\ {\isadigit{0}}{\isacharparenright}{\kern0pt}{\isacharcomma}{\kern0pt}\ And{\isacharparenleft}{\kern0pt}is{\isacharunderscore}{\kern0pt}Fix{\isacharunderscore}{\kern0pt}fm{\isacharparenleft}{\kern0pt}{\isadigit{4}}{\isacharcomma}{\kern0pt}\ {\isadigit{5}}{\isacharcomma}{\kern0pt}\ {\isadigit{0}}{\isacharcomma}{\kern0pt}\ {\isadigit{1}}{\isacharparenright}{\kern0pt}{\isacharcomma}{\kern0pt}\ subset{\isacharunderscore}{\kern0pt}fm{\isacharparenleft}{\kern0pt}{\isadigit{1}}{\isacharcomma}{\kern0pt}\ {\isadigit{2}}{\isacharparenright}{\kern0pt}{\isacharparenright}{\kern0pt}{\isacharparenright}{\kern0pt}{\isacharparenright}{\kern0pt}{\isacharparenright}{\kern0pt}{\isacharparenright}{\kern0pt}{\isacharparenright}{\kern0pt}{\isacharcomma}{\kern0pt}\ {\isacharbrackleft}{\kern0pt}H{\isacharbrackright}{\kern0pt}\ {\isacharat}{\kern0pt}\ {\isacharbrackleft}{\kern0pt}Pow{\isacharparenleft}{\kern0pt}nat{\isacharparenright}{\kern0pt}{\isasyminter}M{\isacharcomma}{\kern0pt}\ nat{\isacharunderscore}{\kern0pt}perms{\isacharcomma}{\kern0pt}\ Fn{\isacharcomma}{\kern0pt}\ nat{\isacharbrackright}{\kern0pt}{\isacharparenright}{\kern0pt}\ {\isacharbraceright}{\kern0pt}{\isachardoublequoteclose}\ \isanewline
\isanewline
\ \ \isacommand{have}\isamarkupfalse%
\ XinM{\isacharcolon}{\kern0pt}\ {\isachardoublequoteopen}X\ {\isasymin}\ M{\isachardoublequoteclose}\ \isanewline
\ \ \ \ \isacommand{unfolding}\isamarkupfalse%
\ X{\isacharunderscore}{\kern0pt}def\isanewline
\ \ \ \ \ \ \ \isacommand{apply}\isamarkupfalse%
{\isacharparenleft}{\kern0pt}subgoal{\isacharunderscore}{\kern0pt}tac\ {\isachardoublequoteopen}finite{\isacharunderscore}{\kern0pt}M{\isacharunderscore}{\kern0pt}fm{\isacharparenleft}{\kern0pt}{\isadigit{6}}{\isacharcomma}{\kern0pt}\ {\isadigit{0}}{\isacharparenright}{\kern0pt}\ {\isasymin}\ formula\ {\isasymand}\ is{\isacharunderscore}{\kern0pt}Fix{\isacharunderscore}{\kern0pt}fm{\isacharparenleft}{\kern0pt}{\isadigit{4}}{\isacharcomma}{\kern0pt}\ {\isadigit{5}}{\isacharcomma}{\kern0pt}\ {\isadigit{0}}{\isacharcomma}{\kern0pt}\ {\isadigit{1}}{\isacharparenright}{\kern0pt}\ {\isasymin}\ formula\ {\isasymand}\ closed{\isacharunderscore}{\kern0pt}under{\isacharunderscore}{\kern0pt}comp{\isacharunderscore}{\kern0pt}and{\isacharunderscore}{\kern0pt}converse{\isacharunderscore}{\kern0pt}fm{\isacharparenleft}{\kern0pt}{\isadigit{0}}{\isacharparenright}{\kern0pt}\ {\isasymin}\ formula{\isachardoublequoteclose}{\isacharparenright}{\kern0pt}\isanewline
\ \ \ \ \isacommand{apply}\isamarkupfalse%
{\isacharparenleft}{\kern0pt}rule\ separation{\isacharunderscore}{\kern0pt}notation{\isacharcomma}{\kern0pt}\ rule\ separation{\isacharunderscore}{\kern0pt}ax{\isacharcomma}{\kern0pt}\ force{\isacharparenright}{\kern0pt}\isanewline
\ \ \ \ \isacommand{using}\isamarkupfalse%
\ nat{\isacharunderscore}{\kern0pt}in{\isacharunderscore}{\kern0pt}M\ nat{\isacharunderscore}{\kern0pt}perms{\isacharunderscore}{\kern0pt}in{\isacharunderscore}{\kern0pt}M\ Fn{\isacharunderscore}{\kern0pt}in{\isacharunderscore}{\kern0pt}M\ M{\isacharunderscore}{\kern0pt}powerset\ \isanewline
\ \ \ \ \ \ \isacommand{apply}\isamarkupfalse%
\ force\ \isanewline
\ \ \ \ \ \ \isacommand{apply}\isamarkupfalse%
\ simp\isanewline
\ \ \ \ \ \ \isacommand{apply}\isamarkupfalse%
{\isacharparenleft}{\kern0pt}rule\ Un{\isacharunderscore}{\kern0pt}least{\isacharunderscore}{\kern0pt}lt{\isacharcomma}{\kern0pt}\ rule\ le{\isacharunderscore}{\kern0pt}trans{\isacharcomma}{\kern0pt}\ rule\ arity{\isacharunderscore}{\kern0pt}closed{\isacharunderscore}{\kern0pt}under{\isacharunderscore}{\kern0pt}comp{\isacharunderscore}{\kern0pt}and{\isacharunderscore}{\kern0pt}converse{\isacharunderscore}{\kern0pt}fm{\isacharcomma}{\kern0pt}\ simp{\isacharcomma}{\kern0pt}\ simp{\isacharparenright}{\kern0pt}\isanewline
\ \ \ \ \ \ \isacommand{apply}\isamarkupfalse%
{\isacharparenleft}{\kern0pt}rule\ pred{\isacharunderscore}{\kern0pt}le{\isacharcomma}{\kern0pt}\ force{\isacharcomma}{\kern0pt}\ force{\isacharparenright}{\kern0pt}{\isacharplus}{\kern0pt}\isanewline
\ \ \ \ \ \ \isacommand{apply}\isamarkupfalse%
{\isacharparenleft}{\kern0pt}rule\ Un{\isacharunderscore}{\kern0pt}least{\isacharunderscore}{\kern0pt}lt{\isacharparenright}{\kern0pt}{\isacharplus}{\kern0pt}\isanewline
\ \ \ \ \ \ \ \ \isacommand{apply}\isamarkupfalse%
\ auto{\isacharbrackleft}{\kern0pt}{\isadigit{2}}{\isacharbrackright}{\kern0pt}\isanewline
\ \ \ \ \ \ \isacommand{apply}\isamarkupfalse%
{\isacharparenleft}{\kern0pt}rule\ Un{\isacharunderscore}{\kern0pt}least{\isacharunderscore}{\kern0pt}lt{\isacharparenright}{\kern0pt}\isanewline
\ \ \ \ \ \ \ \isacommand{apply}\isamarkupfalse%
{\isacharparenleft}{\kern0pt}rule\ le{\isacharunderscore}{\kern0pt}trans{\isacharcomma}{\kern0pt}\ rule\ arity{\isacharunderscore}{\kern0pt}finite{\isacharunderscore}{\kern0pt}M{\isacharunderscore}{\kern0pt}fm{\isacharparenright}{\kern0pt}\isanewline
\ \ \ \ \isacommand{using}\isamarkupfalse%
\ Un{\isacharunderscore}{\kern0pt}least{\isacharunderscore}{\kern0pt}lt\ \isanewline
\ \ \ \ \ \ \ \ \ \isacommand{apply}\isamarkupfalse%
\ auto{\isacharbrackleft}{\kern0pt}{\isadigit{3}}{\isacharbrackright}{\kern0pt}\isanewline
\ \ \ \ \ \ \isacommand{apply}\isamarkupfalse%
{\isacharparenleft}{\kern0pt}rule\ Un{\isacharunderscore}{\kern0pt}least{\isacharunderscore}{\kern0pt}lt{\isacharparenright}{\kern0pt}\isanewline
\ \ \ \ \ \ \ \isacommand{apply}\isamarkupfalse%
{\isacharparenleft}{\kern0pt}rule\ le{\isacharunderscore}{\kern0pt}trans{\isacharcomma}{\kern0pt}\ rule\ arity{\isacharunderscore}{\kern0pt}is{\isacharunderscore}{\kern0pt}Fix{\isacharunderscore}{\kern0pt}fm{\isacharparenright}{\kern0pt}\isanewline
\ \ \ \ \isacommand{using}\isamarkupfalse%
\ Un{\isacharunderscore}{\kern0pt}least{\isacharunderscore}{\kern0pt}lt\ Fn{\isacharunderscore}{\kern0pt}perms{\isacharunderscore}{\kern0pt}in{\isacharunderscore}{\kern0pt}M\ M{\isacharunderscore}{\kern0pt}powerset\ is{\isacharunderscore}{\kern0pt}Fix{\isacharunderscore}{\kern0pt}fm{\isacharunderscore}{\kern0pt}type\ finite{\isacharunderscore}{\kern0pt}M{\isacharunderscore}{\kern0pt}fm{\isacharunderscore}{\kern0pt}type\ closed{\isacharunderscore}{\kern0pt}under{\isacharunderscore}{\kern0pt}comp{\isacharunderscore}{\kern0pt}and{\isacharunderscore}{\kern0pt}converse{\isacharunderscore}{\kern0pt}fm{\isacharunderscore}{\kern0pt}type\isanewline
\ \ \ \ \ \ \ \ \ \ \ \isacommand{apply}\isamarkupfalse%
\ auto\isanewline
\ \ \ \ \isacommand{done}\isamarkupfalse%
\isanewline
\isanewline
\ \ \isacommand{have}\isamarkupfalse%
\ {\isachardoublequoteopen}X\ {\isacharequal}{\kern0pt}\ {\isacharbraceleft}{\kern0pt}\ H\ {\isasymin}\ Pow{\isacharparenleft}{\kern0pt}Fn{\isacharunderscore}{\kern0pt}perms{\isacharparenright}{\kern0pt}\ {\isasyminter}\ M{\isachardot}{\kern0pt}\ {\isacharparenleft}{\kern0pt}{\isacharparenleft}{\kern0pt}{\isasymforall}x\ {\isasymin}\ H{\isachardot}{\kern0pt}\ {\isasymforall}y\ {\isasymin}\ H{\isachardot}{\kern0pt}\ x\ O\ y\ {\isasymin}\ H{\isacharparenright}{\kern0pt}\ {\isasymand}\ {\isacharparenleft}{\kern0pt}{\isasymforall}x\ {\isasymin}\ H{\isachardot}{\kern0pt}\ converse{\isacharparenleft}{\kern0pt}x{\isacharparenright}{\kern0pt}\ {\isasymin}\ H{\isacharparenright}{\kern0pt}{\isacharparenright}{\kern0pt}\ {\isasymand}\ {\isacharparenleft}{\kern0pt}{\isasymexists}F\ {\isasymin}\ M{\isachardot}{\kern0pt}\ {\isasymexists}E\ {\isasymin}\ M{\isachardot}{\kern0pt}\ E\ {\isasymin}\ Pow{\isacharparenleft}{\kern0pt}nat{\isacharparenright}{\kern0pt}\ {\isasyminter}\ M\ {\isasymand}\ finite{\isacharunderscore}{\kern0pt}M{\isacharparenleft}{\kern0pt}E{\isacharparenright}{\kern0pt}\ {\isasymand}\ F\ {\isacharequal}{\kern0pt}\ Fix{\isacharparenleft}{\kern0pt}E{\isacharparenright}{\kern0pt}\ {\isasymand}\ F\ {\isasymsubseteq}\ H{\isacharparenright}{\kern0pt}\ {\isacharbraceright}{\kern0pt}{\isachardoublequoteclose}\ \isanewline
\ \ \ \ \isacommand{unfolding}\isamarkupfalse%
\ X{\isacharunderscore}{\kern0pt}def\isanewline
\ \ \ \ \isacommand{apply}\isamarkupfalse%
{\isacharparenleft}{\kern0pt}rule\ iff{\isacharunderscore}{\kern0pt}eq{\isacharparenright}{\kern0pt}\isanewline
\ \ \ \ \isacommand{apply}\isamarkupfalse%
{\isacharparenleft}{\kern0pt}insert\ M{\isacharunderscore}{\kern0pt}powerset\ Fn{\isacharunderscore}{\kern0pt}in{\isacharunderscore}{\kern0pt}M\ nat{\isacharunderscore}{\kern0pt}perms{\isacharunderscore}{\kern0pt}in{\isacharunderscore}{\kern0pt}M\ nat{\isacharunderscore}{\kern0pt}in{\isacharunderscore}{\kern0pt}M{\isacharparenright}{\kern0pt}\isanewline
\ \ \ \ \isacommand{apply}\isamarkupfalse%
{\isacharparenleft}{\kern0pt}rule\ iff{\isacharunderscore}{\kern0pt}trans{\isacharcomma}{\kern0pt}\ rule\ sats{\isacharunderscore}{\kern0pt}And{\isacharunderscore}{\kern0pt}iff{\isacharcomma}{\kern0pt}\ simp{\isacharcomma}{\kern0pt}\ rule\ iff{\isacharunderscore}{\kern0pt}conjI{\isacharparenright}{\kern0pt}\isanewline
\ \ \ \ \ \isacommand{apply}\isamarkupfalse%
{\isacharparenleft}{\kern0pt}rule\ sats{\isacharunderscore}{\kern0pt}closed{\isacharunderscore}{\kern0pt}under{\isacharunderscore}{\kern0pt}comp{\isacharunderscore}{\kern0pt}and{\isacharunderscore}{\kern0pt}converse{\isacharunderscore}{\kern0pt}fm{\isacharunderscore}{\kern0pt}iff{\isacharcomma}{\kern0pt}\ simp{\isacharcomma}{\kern0pt}\ simp{\isacharcomma}{\kern0pt}\ simp{\isacharparenright}{\kern0pt}\isanewline
\ \ \ \ \isacommand{using}\isamarkupfalse%
\ Fn{\isacharunderscore}{\kern0pt}perms{\isacharunderscore}{\kern0pt}def\ relation{\isacharunderscore}{\kern0pt}def\ Fn{\isacharunderscore}{\kern0pt}perm{\isacharprime}{\kern0pt}{\isacharunderscore}{\kern0pt}def\ \isanewline
\ \ \ \ \ \isacommand{apply}\isamarkupfalse%
\ force\isanewline
\ \ \ \ \isacommand{apply}\isamarkupfalse%
{\isacharparenleft}{\kern0pt}rule\ iff{\isacharunderscore}{\kern0pt}trans{\isacharcomma}{\kern0pt}\ rule\ sats{\isacharunderscore}{\kern0pt}Exists{\isacharunderscore}{\kern0pt}iff{\isacharcomma}{\kern0pt}\ force{\isacharcomma}{\kern0pt}\ rule\ bex{\isacharunderscore}{\kern0pt}iff{\isacharparenright}{\kern0pt}{\isacharplus}{\kern0pt}\isanewline
\ \ \ \ \isacommand{apply}\isamarkupfalse%
{\isacharparenleft}{\kern0pt}rule\ iff{\isacharunderscore}{\kern0pt}trans{\isacharcomma}{\kern0pt}\ rule\ sats{\isacharunderscore}{\kern0pt}And{\isacharunderscore}{\kern0pt}iff{\isacharcomma}{\kern0pt}\ force{\isacharcomma}{\kern0pt}\ rule\ iff{\isacharunderscore}{\kern0pt}conjI{\isadigit{2}}{\isacharcomma}{\kern0pt}\ simp{\isacharparenright}{\kern0pt}{\isacharplus}{\kern0pt}\isanewline
\ \ \ \ \isacommand{apply}\isamarkupfalse%
{\isacharparenleft}{\kern0pt}rule\ sats{\isacharunderscore}{\kern0pt}finite{\isacharunderscore}{\kern0pt}M{\isacharunderscore}{\kern0pt}fm{\isacharunderscore}{\kern0pt}iff{\isacharcomma}{\kern0pt}\ simp{\isacharcomma}{\kern0pt}\ simp{\isacharcomma}{\kern0pt}\ simp{\isacharcomma}{\kern0pt}\ simp{\isacharcomma}{\kern0pt}\ simp{\isacharparenright}{\kern0pt}\isanewline
\ \ \ \ \isacommand{apply}\isamarkupfalse%
{\isacharparenleft}{\kern0pt}rule\ iff{\isacharunderscore}{\kern0pt}trans{\isacharcomma}{\kern0pt}\ rule\ sats{\isacharunderscore}{\kern0pt}And{\isacharunderscore}{\kern0pt}iff{\isacharcomma}{\kern0pt}\ force{\isacharcomma}{\kern0pt}\ rule\ iff{\isacharunderscore}{\kern0pt}conjI{\isadigit{2}}{\isacharcomma}{\kern0pt}\ simp{\isacharparenright}{\kern0pt}{\isacharplus}{\kern0pt}\isanewline
\ \ \ \ \ \isacommand{apply}\isamarkupfalse%
{\isacharparenleft}{\kern0pt}rule\ sats{\isacharunderscore}{\kern0pt}is{\isacharunderscore}{\kern0pt}Fix{\isacharunderscore}{\kern0pt}fm{\isacharunderscore}{\kern0pt}iff{\isacharparenright}{\kern0pt}\isanewline
\ \ \ \ \ \ \ \ \ \ \ \ \ \isacommand{apply}\isamarkupfalse%
\ auto{\isacharbrackleft}{\kern0pt}{\isadigit{9}}{\isacharbrackright}{\kern0pt}\isanewline
\ \ \ \ \isacommand{apply}\isamarkupfalse%
{\isacharparenleft}{\kern0pt}rule\ iff{\isacharunderscore}{\kern0pt}trans{\isacharcomma}{\kern0pt}\ rule\ sats{\isacharunderscore}{\kern0pt}subset{\isacharunderscore}{\kern0pt}fm{\isacharparenright}{\kern0pt}\isanewline
\ \ \ \ \isacommand{using}\isamarkupfalse%
\ M{\isacharunderscore}{\kern0pt}ctm{\isacharunderscore}{\kern0pt}axioms\ M{\isacharunderscore}{\kern0pt}ctm{\isacharunderscore}{\kern0pt}def\ M{\isacharunderscore}{\kern0pt}ctm{\isacharunderscore}{\kern0pt}axioms{\isacharunderscore}{\kern0pt}def\ \isanewline
\ \ \ \ \isacommand{by}\isamarkupfalse%
\ auto\isanewline
\ \ \isacommand{also}\isamarkupfalse%
\ \isacommand{have}\isamarkupfalse%
\ {\isachardoublequoteopen}{\isachardot}{\kern0pt}{\isachardot}{\kern0pt}{\isachardot}{\kern0pt}\ {\isacharequal}{\kern0pt}\ {\isacharbraceleft}{\kern0pt}\ H\ {\isasymin}\ Pow{\isacharparenleft}{\kern0pt}Fn{\isacharunderscore}{\kern0pt}perms{\isacharparenright}{\kern0pt}\ {\isasyminter}\ M{\isachardot}{\kern0pt}\ {\isacharparenleft}{\kern0pt}{\isasymforall}x\ {\isasymin}\ H{\isachardot}{\kern0pt}\ {\isasymforall}y\ {\isasymin}\ H{\isachardot}{\kern0pt}\ x\ O\ y\ {\isasymin}\ H{\isacharparenright}{\kern0pt}\ {\isasymand}\ {\isacharparenleft}{\kern0pt}{\isasymforall}x\ {\isasymin}\ H{\isachardot}{\kern0pt}\ converse{\isacharparenleft}{\kern0pt}x{\isacharparenright}{\kern0pt}\ {\isasymin}\ H{\isacharparenright}{\kern0pt}\ {\isasymand}\ {\isacharparenleft}{\kern0pt}{\isasymexists}E\ {\isasymin}\ Pow{\isacharparenleft}{\kern0pt}nat{\isacharparenright}{\kern0pt}\ {\isasyminter}\ M{\isachardot}{\kern0pt}\ finite{\isacharunderscore}{\kern0pt}M{\isacharparenleft}{\kern0pt}E{\isacharparenright}{\kern0pt}\ {\isasymand}\ Fix{\isacharparenleft}{\kern0pt}E{\isacharparenright}{\kern0pt}\ {\isasymsubseteq}\ H{\isacharparenright}{\kern0pt}\ {\isacharbraceright}{\kern0pt}{\isachardoublequoteclose}\ \isanewline
\ \ \ \ \isacommand{apply}\isamarkupfalse%
{\isacharparenleft}{\kern0pt}rule\ iff{\isacharunderscore}{\kern0pt}eq{\isacharcomma}{\kern0pt}\ rule\ iffI{\isacharcomma}{\kern0pt}\ force{\isacharparenright}{\kern0pt}\isanewline
\ \ \ \ \isacommand{using}\isamarkupfalse%
\ Fix{\isacharunderscore}{\kern0pt}in{\isacharunderscore}{\kern0pt}M\isanewline
\ \ \ \ \isacommand{by}\isamarkupfalse%
\ auto\isanewline
\ \ \isacommand{also}\isamarkupfalse%
\ \isacommand{have}\isamarkupfalse%
\ {\isachardoublequoteopen}{\isachardot}{\kern0pt}{\isachardot}{\kern0pt}{\isachardot}{\kern0pt}\ {\isacharequal}{\kern0pt}\ Fn{\isacharunderscore}{\kern0pt}perms{\isacharunderscore}{\kern0pt}filter{\isachardoublequoteclose}\ \isanewline
\ \ \ \ \isacommand{unfolding}\isamarkupfalse%
\ Fn{\isacharunderscore}{\kern0pt}perms{\isacharunderscore}{\kern0pt}filter{\isacharunderscore}{\kern0pt}def\isanewline
\ \ \ \ \isacommand{apply}\isamarkupfalse%
{\isacharparenleft}{\kern0pt}subst\ forcing{\isacharunderscore}{\kern0pt}data{\isacharunderscore}{\kern0pt}partial{\isachardot}{\kern0pt}P{\isacharunderscore}{\kern0pt}auto{\isacharunderscore}{\kern0pt}subgroups{\isacharunderscore}{\kern0pt}def{\isacharparenright}{\kern0pt}\isanewline
\ \ \ \ \ \isacommand{apply}\isamarkupfalse%
{\isacharparenleft}{\kern0pt}rule\ Fn{\isacharunderscore}{\kern0pt}forcing{\isacharunderscore}{\kern0pt}data{\isacharunderscore}{\kern0pt}partial{\isacharcomma}{\kern0pt}\ simp{\isacharparenright}{\kern0pt}\isanewline
\ \ \ \ \isacommand{apply}\isamarkupfalse%
{\isacharparenleft}{\kern0pt}rule\ iff{\isacharunderscore}{\kern0pt}eq{\isacharparenright}{\kern0pt}\isanewline
\ \ \ \ \isacommand{apply}\isamarkupfalse%
{\isacharparenleft}{\kern0pt}subst\ forcing{\isacharunderscore}{\kern0pt}data{\isacharunderscore}{\kern0pt}partial{\isachardot}{\kern0pt}is{\isacharunderscore}{\kern0pt}P{\isacharunderscore}{\kern0pt}auto{\isacharunderscore}{\kern0pt}group{\isacharunderscore}{\kern0pt}def{\isacharparenright}{\kern0pt}\isanewline
\ \ \ \ \ \isacommand{apply}\isamarkupfalse%
{\isacharparenleft}{\kern0pt}rule\ Fn{\isacharunderscore}{\kern0pt}forcing{\isacharunderscore}{\kern0pt}data{\isacharunderscore}{\kern0pt}partial{\isacharcomma}{\kern0pt}\ rule\ iffI{\isacharparenright}{\kern0pt}\isanewline
\ \ \ \ \ \isacommand{apply}\isamarkupfalse%
{\isacharparenleft}{\kern0pt}rule\ conjI{\isacharparenright}{\kern0pt}{\isacharplus}{\kern0pt}\isanewline
\ \ \ \ \ \ \ \isacommand{apply}\isamarkupfalse%
{\isacharparenleft}{\kern0pt}rule\ subsetI{\isacharcomma}{\kern0pt}\ simp{\isacharparenright}{\kern0pt}\isanewline
\ \ \ \ \ \ \ \isacommand{apply}\isamarkupfalse%
{\isacharparenleft}{\kern0pt}rename{\isacharunderscore}{\kern0pt}tac\ H\ x{\isacharcomma}{\kern0pt}\ subgoal{\isacharunderscore}{\kern0pt}tac\ {\isachardoublequoteopen}{\isasymexists}f\ {\isasymin}\ nat{\isacharunderscore}{\kern0pt}perms{\isachardot}{\kern0pt}\ Fn{\isacharunderscore}{\kern0pt}perm{\isacharprime}{\kern0pt}{\isacharparenleft}{\kern0pt}f{\isacharparenright}{\kern0pt}\ {\isacharequal}{\kern0pt}\ x{\isachardoublequoteclose}{\isacharparenright}{\kern0pt}\isanewline
\ \ \ \ \isacommand{using}\isamarkupfalse%
\ Fn{\isacharunderscore}{\kern0pt}perm{\isacharprime}{\kern0pt}{\isacharunderscore}{\kern0pt}type\ Fn{\isacharunderscore}{\kern0pt}perm{\isacharprime}{\kern0pt}{\isacharunderscore}{\kern0pt}is{\isacharunderscore}{\kern0pt}P{\isacharunderscore}{\kern0pt}auto\ Fn{\isacharunderscore}{\kern0pt}perms{\isacharunderscore}{\kern0pt}def\isanewline
\ \ \ \ \isacommand{by}\isamarkupfalse%
\ auto\isanewline
\ \ \ \ \isanewline
\ \ \isacommand{finally}\isamarkupfalse%
\ \isacommand{show}\isamarkupfalse%
\ {\isacharquery}{\kern0pt}thesis\ \isacommand{using}\isamarkupfalse%
\ {\isacartoucheopen}X\ {\isasymin}\ M{\isacartoucheclose}\ \isacommand{by}\isamarkupfalse%
\ auto\isanewline
\isacommand{qed}\isamarkupfalse%
%
\endisatagproof
{\isafoldproof}%
%
\isadelimproof
\isanewline
%
\endisadelimproof
\isanewline
\isacommand{lemma}\isamarkupfalse%
\ Fn{\isacharunderscore}{\kern0pt}perms{\isacharunderscore}{\kern0pt}filter{\isacharunderscore}{\kern0pt}nonempty\ {\isacharcolon}{\kern0pt}\ {\isachardoublequoteopen}Fn{\isacharunderscore}{\kern0pt}perms{\isacharunderscore}{\kern0pt}filter\ {\isasymnoteq}\ {\isadigit{0}}{\isachardoublequoteclose}\ \isanewline
%
\isadelimproof
\ \ %
\endisadelimproof
%
\isatagproof
\isacommand{apply}\isamarkupfalse%
{\isacharparenleft}{\kern0pt}rule{\isacharunderscore}{\kern0pt}tac\ a{\isacharequal}{\kern0pt}{\isachardoublequoteopen}Fn{\isacharunderscore}{\kern0pt}perms{\isachardoublequoteclose}\ \isakeyword{in}\ not{\isacharunderscore}{\kern0pt}emptyI{\isacharparenright}{\kern0pt}\isanewline
\ \ \isacommand{unfolding}\isamarkupfalse%
\ Fn{\isacharunderscore}{\kern0pt}perms{\isacharunderscore}{\kern0pt}filter{\isacharunderscore}{\kern0pt}def\isanewline
\ \ \isacommand{apply}\isamarkupfalse%
\ simp\isanewline
\ \ \isacommand{apply}\isamarkupfalse%
{\isacharparenleft}{\kern0pt}rule\ conjI{\isacharcomma}{\kern0pt}\ subst\ forcing{\isacharunderscore}{\kern0pt}data{\isacharunderscore}{\kern0pt}partial{\isachardot}{\kern0pt}P{\isacharunderscore}{\kern0pt}auto{\isacharunderscore}{\kern0pt}subgroups{\isacharunderscore}{\kern0pt}def{\isacharcomma}{\kern0pt}\ rule\ Fn{\isacharunderscore}{\kern0pt}forcing{\isacharunderscore}{\kern0pt}data{\isacharunderscore}{\kern0pt}partial{\isacharparenright}{\kern0pt}\isanewline
\ \ \ \isacommand{apply}\isamarkupfalse%
\ {\isacharparenleft}{\kern0pt}simp{\isacharcomma}{\kern0pt}\ rule\ conjI{\isacharcomma}{\kern0pt}\ rule\ Fn{\isacharunderscore}{\kern0pt}perms{\isacharunderscore}{\kern0pt}in{\isacharunderscore}{\kern0pt}M{\isacharcomma}{\kern0pt}\ rule\ Fn{\isacharunderscore}{\kern0pt}perms{\isacharunderscore}{\kern0pt}group{\isacharparenright}{\kern0pt}\isanewline
\ \ \isacommand{apply}\isamarkupfalse%
{\isacharparenleft}{\kern0pt}rule{\isacharunderscore}{\kern0pt}tac\ x{\isacharequal}{\kern0pt}{\isadigit{0}}\ \isakeyword{in}\ bexI{\isacharcomma}{\kern0pt}\ rule\ conjI{\isacharcomma}{\kern0pt}\ simp\ add{\isacharcolon}{\kern0pt}finite{\isacharunderscore}{\kern0pt}M{\isacharunderscore}{\kern0pt}def{\isacharparenright}{\kern0pt}\isanewline
\ \ \ \ \isacommand{apply}\isamarkupfalse%
{\isacharparenleft}{\kern0pt}rule{\isacharunderscore}{\kern0pt}tac\ x{\isacharequal}{\kern0pt}{\isadigit{0}}\ \isakeyword{in}\ bexI{\isacharcomma}{\kern0pt}\ rule{\isacharunderscore}{\kern0pt}tac\ a{\isacharequal}{\kern0pt}{\isadigit{0}}\ \isakeyword{in}\ not{\isacharunderscore}{\kern0pt}emptyI{\isacharparenright}{\kern0pt}\isanewline
\ \ \isacommand{using}\isamarkupfalse%
\ inj{\isacharunderscore}{\kern0pt}def\ zero{\isacharunderscore}{\kern0pt}in{\isacharunderscore}{\kern0pt}M\isanewline
\ \ \ \ \ \isacommand{apply}\isamarkupfalse%
\ auto{\isacharbrackleft}{\kern0pt}{\isadigit{2}}{\isacharbrackright}{\kern0pt}\isanewline
\ \ \isacommand{unfolding}\isamarkupfalse%
\ Fix{\isacharunderscore}{\kern0pt}def\ Fn{\isacharunderscore}{\kern0pt}perms{\isacharunderscore}{\kern0pt}def\isanewline
\ \ \ \isacommand{apply}\isamarkupfalse%
\ force\isanewline
\ \ \isacommand{using}\isamarkupfalse%
\ zero{\isacharunderscore}{\kern0pt}in{\isacharunderscore}{\kern0pt}M\isanewline
\ \ \isacommand{by}\isamarkupfalse%
\ auto%
\endisatagproof
{\isafoldproof}%
%
\isadelimproof
\isanewline
%
\endisadelimproof
\isanewline
\isacommand{lemma}\isamarkupfalse%
\ Fix{\isacharunderscore}{\kern0pt}subset\ {\isacharcolon}{\kern0pt}\ \isanewline
\ \ \isakeyword{fixes}\ A\ B\ \isanewline
\ \ \isakeyword{assumes}\ {\isachardoublequoteopen}A\ {\isasymsubseteq}\ B{\isachardoublequoteclose}\ \isanewline
\ \ \isakeyword{shows}\ {\isachardoublequoteopen}Fix{\isacharparenleft}{\kern0pt}B{\isacharparenright}{\kern0pt}\ {\isasymsubseteq}\ Fix{\isacharparenleft}{\kern0pt}A{\isacharparenright}{\kern0pt}{\isachardoublequoteclose}\ \isanewline
%
\isadelimproof
\ \ %
\endisadelimproof
%
\isatagproof
\isacommand{unfolding}\isamarkupfalse%
\ Fix{\isacharunderscore}{\kern0pt}def\ \isanewline
\ \ \isacommand{using}\isamarkupfalse%
\ assms\isanewline
\ \ \isacommand{by}\isamarkupfalse%
\ force%
\endisatagproof
{\isafoldproof}%
%
\isadelimproof
\isanewline
%
\endisadelimproof
\isanewline
\isanewline
\isacommand{definition}\isamarkupfalse%
\ succ{\isacharunderscore}{\kern0pt}fun\ \isakeyword{where}\ {\isachardoublequoteopen}succ{\isacharunderscore}{\kern0pt}fun\ {\isasymequiv}\ {\isacharbraceleft}{\kern0pt}\ {\isacharless}{\kern0pt}n{\isacharcomma}{\kern0pt}\ succ{\isacharparenleft}{\kern0pt}n{\isacharparenright}{\kern0pt}{\isachargreater}{\kern0pt}\ {\isachardot}{\kern0pt}\ n\ {\isasymin}\ nat\ {\isacharbraceright}{\kern0pt}{\isachardoublequoteclose}\isanewline
\isanewline
\isacommand{lemma}\isamarkupfalse%
\ succ{\isacharunderscore}{\kern0pt}fun{\isacharunderscore}{\kern0pt}type\ {\isacharcolon}{\kern0pt}\ {\isachardoublequoteopen}succ{\isacharunderscore}{\kern0pt}fun\ {\isasymin}\ nat\ {\isasymrightarrow}\ nat{\isachardoublequoteclose}\ \isanewline
%
\isadelimproof
\ \ %
\endisadelimproof
%
\isatagproof
\isacommand{apply}\isamarkupfalse%
{\isacharparenleft}{\kern0pt}rule\ Pi{\isacharunderscore}{\kern0pt}memberI{\isacharparenright}{\kern0pt}\isanewline
\ \ \isacommand{unfolding}\isamarkupfalse%
\ relation{\isacharunderscore}{\kern0pt}def\ succ{\isacharunderscore}{\kern0pt}fun{\isacharunderscore}{\kern0pt}def\ function{\isacharunderscore}{\kern0pt}def\isanewline
\ \ \isacommand{by}\isamarkupfalse%
\ auto%
\endisatagproof
{\isafoldproof}%
%
\isadelimproof
\isanewline
%
\endisadelimproof
\isanewline
\isacommand{lemma}\isamarkupfalse%
\ succ{\isacharunderscore}{\kern0pt}fun{\isacharunderscore}{\kern0pt}in{\isacharunderscore}{\kern0pt}M\ {\isacharcolon}{\kern0pt}\ {\isachardoublequoteopen}succ{\isacharunderscore}{\kern0pt}fun\ {\isasymin}\ M{\isachardoublequoteclose}\isanewline
%
\isadelimproof
%
\endisadelimproof
%
\isatagproof
\isacommand{proof}\isamarkupfalse%
\ {\isacharminus}{\kern0pt}\ \isanewline
\ \ \isacommand{define}\isamarkupfalse%
\ X\ \isakeyword{where}\ {\isachardoublequoteopen}X\ {\isasymequiv}\ {\isacharbraceleft}{\kern0pt}\ v\ {\isasymin}\ nat\ {\isasymtimes}\ nat{\isachardot}{\kern0pt}\ sats{\isacharparenleft}{\kern0pt}M{\isacharcomma}{\kern0pt}\ Exists{\isacharparenleft}{\kern0pt}Exists{\isacharparenleft}{\kern0pt}And{\isacharparenleft}{\kern0pt}pair{\isacharunderscore}{\kern0pt}fm{\isacharparenleft}{\kern0pt}{\isadigit{0}}{\isacharcomma}{\kern0pt}\ {\isadigit{1}}{\isacharcomma}{\kern0pt}\ {\isadigit{2}}{\isacharparenright}{\kern0pt}{\isacharcomma}{\kern0pt}\ succ{\isacharunderscore}{\kern0pt}fm{\isacharparenleft}{\kern0pt}{\isadigit{0}}{\isacharcomma}{\kern0pt}\ {\isadigit{1}}{\isacharparenright}{\kern0pt}{\isacharparenright}{\kern0pt}{\isacharparenright}{\kern0pt}{\isacharparenright}{\kern0pt}{\isacharcomma}{\kern0pt}\ {\isacharbrackleft}{\kern0pt}v{\isacharbrackright}{\kern0pt}\ {\isacharat}{\kern0pt}\ {\isacharbrackleft}{\kern0pt}{\isacharbrackright}{\kern0pt}{\isacharparenright}{\kern0pt}\ {\isacharbraceright}{\kern0pt}{\isachardoublequoteclose}\isanewline
\ \ \isacommand{have}\isamarkupfalse%
\ {\isachardoublequoteopen}X\ {\isasymin}\ M{\isachardoublequoteclose}\ \isanewline
\ \ \ \ \isacommand{unfolding}\isamarkupfalse%
\ X{\isacharunderscore}{\kern0pt}def\isanewline
\ \ \ \ \isacommand{apply}\isamarkupfalse%
{\isacharparenleft}{\kern0pt}rule\ separation{\isacharunderscore}{\kern0pt}notation{\isacharparenright}{\kern0pt}\isanewline
\ \ \ \ \ \isacommand{apply}\isamarkupfalse%
{\isacharparenleft}{\kern0pt}rule\ separation{\isacharunderscore}{\kern0pt}ax{\isacharparenright}{\kern0pt}\isanewline
\ \ \ \ \ \ \ \isacommand{apply}\isamarkupfalse%
\ auto{\isacharbrackleft}{\kern0pt}{\isadigit{2}}{\isacharbrackright}{\kern0pt}\isanewline
\ \ \ \ \ \isacommand{apply}\isamarkupfalse%
\ {\isacharparenleft}{\kern0pt}simp\ del{\isacharcolon}{\kern0pt}FOL{\isacharunderscore}{\kern0pt}sats{\isacharunderscore}{\kern0pt}iff\ pair{\isacharunderscore}{\kern0pt}abs\ add{\isacharcolon}{\kern0pt}\ fm{\isacharunderscore}{\kern0pt}defs\ nat{\isacharunderscore}{\kern0pt}simp{\isacharunderscore}{\kern0pt}union{\isacharparenright}{\kern0pt}\isanewline
\ \ \ \ \isacommand{using}\isamarkupfalse%
\ nat{\isacharunderscore}{\kern0pt}in{\isacharunderscore}{\kern0pt}M\ cartprod{\isacharunderscore}{\kern0pt}closed\ \isanewline
\ \ \ \ \isacommand{by}\isamarkupfalse%
\ auto\isanewline
\ \ \isacommand{have}\isamarkupfalse%
\ {\isachardoublequoteopen}X\ {\isacharequal}{\kern0pt}\ {\isacharbraceleft}{\kern0pt}\ {\isacharless}{\kern0pt}n{\isacharcomma}{\kern0pt}\ succ{\isacharparenleft}{\kern0pt}n{\isacharparenright}{\kern0pt}{\isachargreater}{\kern0pt}\ {\isachardot}{\kern0pt}\ n\ {\isasymin}\ nat\ {\isacharbraceright}{\kern0pt}{\isachardoublequoteclose}\ \isanewline
\ \ \ \ \isacommand{unfolding}\isamarkupfalse%
\ X{\isacharunderscore}{\kern0pt}def\isanewline
\ \ \ \ \isacommand{apply}\isamarkupfalse%
{\isacharparenleft}{\kern0pt}rule\ equality{\isacharunderscore}{\kern0pt}iffI{\isacharcomma}{\kern0pt}\ rule\ iffI{\isacharparenright}{\kern0pt}\isanewline
\ \ \ \ \isacommand{using}\isamarkupfalse%
\ pair{\isacharunderscore}{\kern0pt}in{\isacharunderscore}{\kern0pt}M{\isacharunderscore}{\kern0pt}iff\ nat{\isacharunderscore}{\kern0pt}in{\isacharunderscore}{\kern0pt}M\ transM\isanewline
\ \ \ \ \isacommand{by}\isamarkupfalse%
\ auto\isanewline
\ \ \isacommand{then}\isamarkupfalse%
\ \isacommand{show}\isamarkupfalse%
\ {\isacharquery}{\kern0pt}thesis\ \isacommand{using}\isamarkupfalse%
\ succ{\isacharunderscore}{\kern0pt}fun{\isacharunderscore}{\kern0pt}def\ {\isacartoucheopen}X\ {\isasymin}\ M{\isacartoucheclose}\ \isacommand{by}\isamarkupfalse%
\ auto\isanewline
\isacommand{qed}\isamarkupfalse%
%
\endisatagproof
{\isafoldproof}%
%
\isadelimproof
\isanewline
%
\endisadelimproof
\isanewline
\isacommand{definition}\isamarkupfalse%
\ add{\isacharunderscore}{\kern0pt}fun\ \isakeyword{where}\ {\isachardoublequoteopen}add{\isacharunderscore}{\kern0pt}fun{\isacharparenleft}{\kern0pt}m{\isacharparenright}{\kern0pt}\ {\isasymequiv}\ {\isacharbraceleft}{\kern0pt}\ {\isacharless}{\kern0pt}n{\isacharcomma}{\kern0pt}\ n\ {\isacharhash}{\kern0pt}{\isacharplus}{\kern0pt}\ m{\isachargreater}{\kern0pt}{\isachardot}{\kern0pt}\ n\ {\isasymin}\ nat\ {\isacharbraceright}{\kern0pt}{\isachardoublequoteclose}\ \isanewline
\isanewline
\isacommand{lemma}\isamarkupfalse%
\ add{\isacharunderscore}{\kern0pt}fun{\isacharunderscore}{\kern0pt}type\ {\isacharcolon}{\kern0pt}\ {\isachardoublequoteopen}m\ {\isasymin}\ nat\ {\isasymLongrightarrow}\ add{\isacharunderscore}{\kern0pt}fun{\isacharparenleft}{\kern0pt}m{\isacharparenright}{\kern0pt}\ {\isasymin}\ nat\ {\isasymrightarrow}\ nat{\isachardoublequoteclose}\ \isanewline
%
\isadelimproof
\ \ %
\endisadelimproof
%
\isatagproof
\isacommand{apply}\isamarkupfalse%
{\isacharparenleft}{\kern0pt}rule\ Pi{\isacharunderscore}{\kern0pt}memberI{\isacharparenright}{\kern0pt}\isanewline
\ \ \isacommand{unfolding}\isamarkupfalse%
\ add{\isacharunderscore}{\kern0pt}fun{\isacharunderscore}{\kern0pt}def\ relation{\isacharunderscore}{\kern0pt}def\ function{\isacharunderscore}{\kern0pt}def\isanewline
\ \ \isacommand{by}\isamarkupfalse%
\ auto%
\endisatagproof
{\isafoldproof}%
%
\isadelimproof
\isanewline
%
\endisadelimproof
\isanewline
\isacommand{lemma}\isamarkupfalse%
\ add{\isacharunderscore}{\kern0pt}fun{\isacharunderscore}{\kern0pt}comp\ {\isacharcolon}{\kern0pt}\ \isanewline
\ \ \isakeyword{fixes}\ m\ \isanewline
\ \ \isakeyword{assumes}\ {\isachardoublequoteopen}m\ {\isasymin}\ nat{\isachardoublequoteclose}\ \isanewline
\ \ \isakeyword{shows}\ {\isachardoublequoteopen}add{\isacharunderscore}{\kern0pt}fun{\isacharparenleft}{\kern0pt}succ{\isacharparenleft}{\kern0pt}m{\isacharparenright}{\kern0pt}{\isacharparenright}{\kern0pt}\ {\isacharequal}{\kern0pt}\ add{\isacharunderscore}{\kern0pt}fun{\isacharparenleft}{\kern0pt}m{\isacharparenright}{\kern0pt}\ O\ succ{\isacharunderscore}{\kern0pt}fun{\isachardoublequoteclose}\ \isanewline
%
\isadelimproof
\isanewline
\ \ %
\endisadelimproof
%
\isatagproof
\isacommand{apply}\isamarkupfalse%
{\isacharparenleft}{\kern0pt}rule\ function{\isacharunderscore}{\kern0pt}eq{\isacharparenright}{\kern0pt}\isanewline
\ \ \isacommand{using}\isamarkupfalse%
\ relation{\isacharunderscore}{\kern0pt}def\ add{\isacharunderscore}{\kern0pt}fun{\isacharunderscore}{\kern0pt}def\ succ{\isacharunderscore}{\kern0pt}fun{\isacharunderscore}{\kern0pt}def\ comp{\isacharunderscore}{\kern0pt}def\ function{\isacharunderscore}{\kern0pt}def\isanewline
\ \ \ \ \ \ \ \isacommand{apply}\isamarkupfalse%
\ auto{\isacharbrackleft}{\kern0pt}{\isadigit{3}}{\isacharbrackright}{\kern0pt}\isanewline
\ \ \ \ \isacommand{apply}\isamarkupfalse%
{\isacharparenleft}{\kern0pt}subgoal{\isacharunderscore}{\kern0pt}tac\ {\isachardoublequoteopen}add{\isacharunderscore}{\kern0pt}fun{\isacharparenleft}{\kern0pt}m{\isacharparenright}{\kern0pt}\ O\ succ{\isacharunderscore}{\kern0pt}fun\ {\isasymin}\ nat\ {\isasymrightarrow}\ nat{\isachardoublequoteclose}{\isacharparenright}{\kern0pt}\isanewline
\ \ \isacommand{using}\isamarkupfalse%
\ Pi{\isacharunderscore}{\kern0pt}def\isanewline
\ \ \ \ \ \isacommand{apply}\isamarkupfalse%
\ force\isanewline
\ \ \ \ \isacommand{apply}\isamarkupfalse%
{\isacharparenleft}{\kern0pt}rule{\isacharunderscore}{\kern0pt}tac\ A{\isacharequal}{\kern0pt}nat\ \isakeyword{and}\ B{\isacharequal}{\kern0pt}nat\ \isakeyword{and}\ C{\isacharequal}{\kern0pt}nat\ \isakeyword{in}\ comp{\isacharunderscore}{\kern0pt}fun{\isacharparenright}{\kern0pt}\isanewline
\ \ \isacommand{using}\isamarkupfalse%
\ succ{\isacharunderscore}{\kern0pt}fun{\isacharunderscore}{\kern0pt}type\ add{\isacharunderscore}{\kern0pt}fun{\isacharunderscore}{\kern0pt}type\ assms\ \isanewline
\ \ \ \ \ \isacommand{apply}\isamarkupfalse%
\ auto{\isacharbrackleft}{\kern0pt}{\isadigit{2}}{\isacharbrackright}{\kern0pt}\isanewline
\ \ \ \isacommand{apply}\isamarkupfalse%
{\isacharparenleft}{\kern0pt}subst\ domain{\isacharunderscore}{\kern0pt}comp{\isacharunderscore}{\kern0pt}eq{\isacharcomma}{\kern0pt}\ subst\ succ{\isacharunderscore}{\kern0pt}fun{\isacharunderscore}{\kern0pt}def{\isacharcomma}{\kern0pt}\ subst\ add{\isacharunderscore}{\kern0pt}fun{\isacharunderscore}{\kern0pt}def{\isacharparenright}{\kern0pt}\isanewline
\ \ \isacommand{using}\isamarkupfalse%
\ succ{\isacharunderscore}{\kern0pt}fun{\isacharunderscore}{\kern0pt}def\ add{\isacharunderscore}{\kern0pt}fun{\isacharunderscore}{\kern0pt}def\ assms\isanewline
\ \ \ \ \isacommand{apply}\isamarkupfalse%
\ force\isanewline
\ \ \ \isacommand{apply}\isamarkupfalse%
{\isacharparenleft}{\kern0pt}subst\ succ{\isacharunderscore}{\kern0pt}fun{\isacharunderscore}{\kern0pt}def{\isacharcomma}{\kern0pt}\ subst\ add{\isacharunderscore}{\kern0pt}fun{\isacharunderscore}{\kern0pt}def{\isacharcomma}{\kern0pt}\ force{\isacharparenright}{\kern0pt}\isanewline
\ \ \isacommand{apply}\isamarkupfalse%
{\isacharparenleft}{\kern0pt}subst\ function{\isacharunderscore}{\kern0pt}apply{\isacharunderscore}{\kern0pt}equality{\isacharcomma}{\kern0pt}\ simp\ add{\isacharcolon}{\kern0pt}add{\isacharunderscore}{\kern0pt}fun{\isacharunderscore}{\kern0pt}def{\isacharcomma}{\kern0pt}\ force{\isacharparenright}{\kern0pt}\isanewline
\ \ \isacommand{using}\isamarkupfalse%
\ add{\isacharunderscore}{\kern0pt}fun{\isacharunderscore}{\kern0pt}type\ assms\ Pi{\isacharunderscore}{\kern0pt}def\isanewline
\ \ \ \isacommand{apply}\isamarkupfalse%
\ force\ \isanewline
\ \ \isacommand{apply}\isamarkupfalse%
{\isacharparenleft}{\kern0pt}subst\ comp{\isacharunderscore}{\kern0pt}fun{\isacharunderscore}{\kern0pt}apply{\isacharcomma}{\kern0pt}\ rule\ succ{\isacharunderscore}{\kern0pt}fun{\isacharunderscore}{\kern0pt}type{\isacharparenright}{\kern0pt}\isanewline
\ \ \isacommand{using}\isamarkupfalse%
\ add{\isacharunderscore}{\kern0pt}fun{\isacharunderscore}{\kern0pt}def\isanewline
\ \ \ \isacommand{apply}\isamarkupfalse%
\ force\isanewline
\ \ \isacommand{apply}\isamarkupfalse%
{\isacharparenleft}{\kern0pt}subst\ {\isacharparenleft}{\kern0pt}{\isadigit{2}}{\isacharparenright}{\kern0pt}function{\isacharunderscore}{\kern0pt}apply{\isacharunderscore}{\kern0pt}equality{\isacharcomma}{\kern0pt}\ simp\ add{\isacharcolon}{\kern0pt}succ{\isacharunderscore}{\kern0pt}fun{\isacharunderscore}{\kern0pt}def{\isacharparenright}{\kern0pt}\isanewline
\ \ \isacommand{using}\isamarkupfalse%
\ add{\isacharunderscore}{\kern0pt}fun{\isacharunderscore}{\kern0pt}def\ succ{\isacharunderscore}{\kern0pt}fun{\isacharunderscore}{\kern0pt}type\ Pi{\isacharunderscore}{\kern0pt}def\isanewline
\ \ \ \ \isacommand{apply}\isamarkupfalse%
\ auto{\isacharbrackleft}{\kern0pt}{\isadigit{2}}{\isacharbrackright}{\kern0pt}\isanewline
\ \ \isacommand{apply}\isamarkupfalse%
{\isacharparenleft}{\kern0pt}subst\ function{\isacharunderscore}{\kern0pt}apply{\isacharunderscore}{\kern0pt}equality{\isacharcomma}{\kern0pt}\ simp\ add{\isacharcolon}{\kern0pt}add{\isacharunderscore}{\kern0pt}fun{\isacharunderscore}{\kern0pt}def{\isacharcomma}{\kern0pt}\ force{\isacharparenright}{\kern0pt}\isanewline
\ \ \isacommand{using}\isamarkupfalse%
\ add{\isacharunderscore}{\kern0pt}fun{\isacharunderscore}{\kern0pt}type\ Pi{\isacharunderscore}{\kern0pt}def\ assms\isanewline
\ \ \isacommand{by}\isamarkupfalse%
\ auto%
\endisatagproof
{\isafoldproof}%
%
\isadelimproof
\isanewline
%
\endisadelimproof
\isanewline
\isacommand{lemma}\isamarkupfalse%
\ add{\isacharunderscore}{\kern0pt}fun{\isacharunderscore}{\kern0pt}in{\isacharunderscore}{\kern0pt}M\ {\isacharcolon}{\kern0pt}\ \isanewline
\ \ \isakeyword{fixes}\ m\ \isanewline
\ \ \isakeyword{assumes}\ {\isachardoublequoteopen}m\ {\isasymin}\ nat{\isachardoublequoteclose}\ \isanewline
\ \ \isakeyword{shows}\ {\isachardoublequoteopen}add{\isacharunderscore}{\kern0pt}fun{\isacharparenleft}{\kern0pt}m{\isacharparenright}{\kern0pt}\ {\isasymin}\ M{\isachardoublequoteclose}\ \isanewline
%
\isadelimproof
\ \ %
\endisadelimproof
%
\isatagproof
\isacommand{using}\isamarkupfalse%
\ assms\isanewline
\ \ \isacommand{apply}\isamarkupfalse%
{\isacharparenleft}{\kern0pt}induct{\isacharparenright}{\kern0pt}\isanewline
\ \ \isacommand{apply}\isamarkupfalse%
{\isacharparenleft}{\kern0pt}rule{\isacharunderscore}{\kern0pt}tac\ b\ {\isacharequal}{\kern0pt}\ {\isachardoublequoteopen}add{\isacharunderscore}{\kern0pt}fun{\isacharparenleft}{\kern0pt}{\isadigit{0}}{\isacharparenright}{\kern0pt}{\isachardoublequoteclose}\ \isakeyword{and}\ a{\isacharequal}{\kern0pt}{\isachardoublequoteopen}id{\isacharparenleft}{\kern0pt}nat{\isacharparenright}{\kern0pt}{\isachardoublequoteclose}\ \isakeyword{in}\ ssubst{\isacharparenright}{\kern0pt}\isanewline
\ \ \isacommand{using}\isamarkupfalse%
\ add{\isacharunderscore}{\kern0pt}fun{\isacharunderscore}{\kern0pt}def\ id{\isacharunderscore}{\kern0pt}def\ lam{\isacharunderscore}{\kern0pt}def\isanewline
\ \ \ \ \isacommand{apply}\isamarkupfalse%
\ force\ \isanewline
\ \ \isacommand{using}\isamarkupfalse%
\ id{\isacharunderscore}{\kern0pt}closed\ nat{\isacharunderscore}{\kern0pt}in{\isacharunderscore}{\kern0pt}M\ \isanewline
\ \ \ \isacommand{apply}\isamarkupfalse%
\ force\isanewline
\ \ \isacommand{apply}\isamarkupfalse%
{\isacharparenleft}{\kern0pt}subst\ add{\isacharunderscore}{\kern0pt}fun{\isacharunderscore}{\kern0pt}comp{\isacharparenright}{\kern0pt}\isanewline
\ \ \isacommand{using}\isamarkupfalse%
\ comp{\isacharunderscore}{\kern0pt}closed\ succ{\isacharunderscore}{\kern0pt}fun{\isacharunderscore}{\kern0pt}in{\isacharunderscore}{\kern0pt}M\ \isanewline
\ \ \isacommand{by}\isamarkupfalse%
\ auto%
\endisatagproof
{\isafoldproof}%
%
\isadelimproof
\isanewline
%
\endisadelimproof
\ \ \isanewline
\isacommand{schematic{\isacharunderscore}{\kern0pt}goal}\isamarkupfalse%
\ finite{\isacharunderscore}{\kern0pt}M{\isacharunderscore}{\kern0pt}union{\isacharunderscore}{\kern0pt}h{\isacharunderscore}{\kern0pt}elem{\isacharunderscore}{\kern0pt}fm{\isacharunderscore}{\kern0pt}auto{\isacharcolon}{\kern0pt}\isanewline
\ \ \isakeyword{assumes}\isanewline
\ \ \ \ {\isachardoublequoteopen}nth{\isacharparenleft}{\kern0pt}{\isadigit{0}}{\isacharcomma}{\kern0pt}env{\isacharparenright}{\kern0pt}\ {\isacharequal}{\kern0pt}\ v{\isachardoublequoteclose}\isanewline
\ \ \ \ {\isachardoublequoteopen}nth{\isacharparenleft}{\kern0pt}{\isadigit{1}}{\isacharcomma}{\kern0pt}env{\isacharparenright}{\kern0pt}\ {\isacharequal}{\kern0pt}\ A{\isachardoublequoteclose}\isanewline
\ \ \ \ {\isachardoublequoteopen}nth{\isacharparenleft}{\kern0pt}{\isadigit{2}}{\isacharcomma}{\kern0pt}env{\isacharparenright}{\kern0pt}\ {\isacharequal}{\kern0pt}\ f{\isachardoublequoteclose}\isanewline
\ \ \ \ {\isachardoublequoteopen}nth{\isacharparenleft}{\kern0pt}{\isadigit{3}}{\isacharcomma}{\kern0pt}env{\isacharparenright}{\kern0pt}\ {\isacharequal}{\kern0pt}\ g{\isachardoublequoteclose}\isanewline
\ \ \ \ {\isachardoublequoteopen}nth{\isacharparenleft}{\kern0pt}{\isadigit{4}}{\isacharcomma}{\kern0pt}env{\isacharparenright}{\kern0pt}\ {\isacharequal}{\kern0pt}\ addn{\isachardoublequoteclose}\isanewline
\ \ \ \ {\isachardoublequoteopen}env\ {\isasymin}\ list{\isacharparenleft}{\kern0pt}M{\isacharparenright}{\kern0pt}{\isachardoublequoteclose}\ \isanewline
\ \ \isakeyword{shows}\ {\isachardoublequoteopen}{\isacharparenleft}{\kern0pt}{\isacharparenleft}{\kern0pt}{\isasymexists}x\ {\isasymin}\ M{\isachardot}{\kern0pt}\ {\isasymexists}fx\ {\isasymin}\ M{\isachardot}{\kern0pt}\ x\ {\isasymin}\ A\ {\isasymand}\ fun{\isacharunderscore}{\kern0pt}apply{\isacharparenleft}{\kern0pt}{\isacharhash}{\kern0pt}{\isacharhash}{\kern0pt}M{\isacharcomma}{\kern0pt}\ f{\isacharcomma}{\kern0pt}\ x{\isacharcomma}{\kern0pt}\ fx{\isacharparenright}{\kern0pt}\ {\isasymand}\ pair{\isacharparenleft}{\kern0pt}{\isacharhash}{\kern0pt}{\isacharhash}{\kern0pt}M{\isacharcomma}{\kern0pt}\ x{\isacharcomma}{\kern0pt}\ fx{\isacharcomma}{\kern0pt}\ v{\isacharparenright}{\kern0pt}{\isacharparenright}{\kern0pt}\ {\isasymor}\ \isanewline
\ \ \ \ \ \ \ \ \ \ {\isacharparenleft}{\kern0pt}{\isasymexists}x\ {\isasymin}\ M{\isachardot}{\kern0pt}\ {\isasymexists}gx\ {\isasymin}\ M{\isachardot}{\kern0pt}\ {\isasymexists}gxn\ {\isasymin}\ M{\isachardot}{\kern0pt}\ x\ {\isasymnotin}\ A\ {\isasymand}\ fun{\isacharunderscore}{\kern0pt}apply{\isacharparenleft}{\kern0pt}{\isacharhash}{\kern0pt}{\isacharhash}{\kern0pt}M{\isacharcomma}{\kern0pt}\ g{\isacharcomma}{\kern0pt}\ x{\isacharcomma}{\kern0pt}\ gx{\isacharparenright}{\kern0pt}\ {\isasymand}\ fun{\isacharunderscore}{\kern0pt}apply{\isacharparenleft}{\kern0pt}{\isacharhash}{\kern0pt}{\isacharhash}{\kern0pt}M{\isacharcomma}{\kern0pt}\ addn{\isacharcomma}{\kern0pt}\ gx{\isacharcomma}{\kern0pt}\ gxn{\isacharparenright}{\kern0pt}\ {\isasymand}\ pair{\isacharparenleft}{\kern0pt}{\isacharhash}{\kern0pt}{\isacharhash}{\kern0pt}M{\isacharcomma}{\kern0pt}\ x{\isacharcomma}{\kern0pt}\ gxn{\isacharcomma}{\kern0pt}\ v{\isacharparenright}{\kern0pt}{\isacharparenright}{\kern0pt}{\isacharparenright}{\kern0pt}\ {\isasymlongleftrightarrow}\ sats{\isacharparenleft}{\kern0pt}M{\isacharcomma}{\kern0pt}{\isacharquery}{\kern0pt}fm{\isacharcomma}{\kern0pt}env{\isacharparenright}{\kern0pt}{\isachardoublequoteclose}\ \isanewline
%
\isadelimproof
\ \ %
\endisadelimproof
%
\isatagproof
\isacommand{by}\isamarkupfalse%
\ {\isacharparenleft}{\kern0pt}insert\ assms\ {\isacharsemicolon}{\kern0pt}\ {\isacharparenleft}{\kern0pt}rule\ sep{\isacharunderscore}{\kern0pt}rules\ {\isacharbar}{\kern0pt}\ simp{\isacharparenright}{\kern0pt}{\isacharplus}{\kern0pt}{\isacharparenright}{\kern0pt}%
\endisatagproof
{\isafoldproof}%
%
\isadelimproof
\ \isanewline
%
\endisadelimproof
\isanewline
\isacommand{lemma}\isamarkupfalse%
\ finite{\isacharunderscore}{\kern0pt}M{\isacharunderscore}{\kern0pt}implies{\isacharunderscore}{\kern0pt}Finite\ {\isacharcolon}{\kern0pt}\ \isanewline
\ \ \isakeyword{fixes}\ A\ \isanewline
\ \ \isakeyword{assumes}\ {\isachardoublequoteopen}finite{\isacharunderscore}{\kern0pt}M{\isacharparenleft}{\kern0pt}A{\isacharparenright}{\kern0pt}{\isachardoublequoteclose}\ \isanewline
\ \ \isakeyword{shows}\ {\isachardoublequoteopen}Finite{\isacharparenleft}{\kern0pt}A{\isacharparenright}{\kern0pt}{\isachardoublequoteclose}\ \isanewline
%
\isadelimproof
\isanewline
\ \ %
\endisadelimproof
%
\isatagproof
\isacommand{using}\isamarkupfalse%
\ assms\isanewline
\ \ \isacommand{unfolding}\isamarkupfalse%
\ finite{\isacharunderscore}{\kern0pt}M{\isacharunderscore}{\kern0pt}def\ \isanewline
\ \ \isacommand{apply}\isamarkupfalse%
\ clarsimp\isanewline
\ \ \isacommand{apply}\isamarkupfalse%
{\isacharparenleft}{\kern0pt}rename{\isacharunderscore}{\kern0pt}tac\ n{\isacharcomma}{\kern0pt}\ rule{\isacharunderscore}{\kern0pt}tac\ X{\isacharequal}{\kern0pt}n\ \isakeyword{in}\ lepoll{\isacharunderscore}{\kern0pt}Finite{\isacharparenright}{\kern0pt}\isanewline
\ \ \isacommand{using}\isamarkupfalse%
\ lepoll{\isacharunderscore}{\kern0pt}def\ nat{\isacharunderscore}{\kern0pt}into{\isacharunderscore}{\kern0pt}Finite\isanewline
\ \ \isacommand{by}\isamarkupfalse%
\ auto%
\endisatagproof
{\isafoldproof}%
%
\isadelimproof
\isanewline
%
\endisadelimproof
\isanewline
\isacommand{lemma}\isamarkupfalse%
\ finite{\isacharunderscore}{\kern0pt}M{\isacharunderscore}{\kern0pt}union\ {\isacharcolon}{\kern0pt}\ \isanewline
\ \ \isakeyword{fixes}\ A\ B\ \isanewline
\ \ \isakeyword{assumes}\ {\isachardoublequoteopen}A\ {\isasymin}\ M{\isachardoublequoteclose}\ {\isachardoublequoteopen}B\ {\isasymin}\ M{\isachardoublequoteclose}\ {\isachardoublequoteopen}finite{\isacharunderscore}{\kern0pt}M{\isacharparenleft}{\kern0pt}A{\isacharparenright}{\kern0pt}{\isachardoublequoteclose}\ {\isachardoublequoteopen}finite{\isacharunderscore}{\kern0pt}M{\isacharparenleft}{\kern0pt}B{\isacharparenright}{\kern0pt}{\isachardoublequoteclose}\ \isanewline
\ \ \isakeyword{shows}\ {\isachardoublequoteopen}finite{\isacharunderscore}{\kern0pt}M{\isacharparenleft}{\kern0pt}A\ {\isasymunion}\ B{\isacharparenright}{\kern0pt}{\isachardoublequoteclose}\isanewline
%
\isadelimproof
%
\endisadelimproof
%
\isatagproof
\isacommand{proof}\isamarkupfalse%
\ {\isacharminus}{\kern0pt}\ \isanewline
\ \ \isacommand{obtain}\isamarkupfalse%
\ n\ f\ \isakeyword{where}\ nfH{\isacharcolon}{\kern0pt}\ {\isachardoublequoteopen}n\ {\isasymin}\ nat{\isachardoublequoteclose}\ {\isachardoublequoteopen}f\ {\isasymin}\ M{\isachardoublequoteclose}\ {\isachardoublequoteopen}f\ {\isasymin}\ inj{\isacharparenleft}{\kern0pt}A{\isacharcomma}{\kern0pt}\ n{\isacharparenright}{\kern0pt}{\isachardoublequoteclose}\ \isacommand{using}\isamarkupfalse%
\ assms\ finite{\isacharunderscore}{\kern0pt}M{\isacharunderscore}{\kern0pt}def\ \isacommand{by}\isamarkupfalse%
\ force\isanewline
\ \ \isacommand{obtain}\isamarkupfalse%
\ m\ g\ \isakeyword{where}\ mgH{\isacharcolon}{\kern0pt}\ {\isachardoublequoteopen}m\ {\isasymin}\ nat{\isachardoublequoteclose}\ {\isachardoublequoteopen}g\ {\isasymin}\ M{\isachardoublequoteclose}\ {\isachardoublequoteopen}g\ {\isasymin}\ inj{\isacharparenleft}{\kern0pt}B{\isacharcomma}{\kern0pt}\ m{\isacharparenright}{\kern0pt}{\isachardoublequoteclose}\ \isacommand{using}\isamarkupfalse%
\ assms\ finite{\isacharunderscore}{\kern0pt}M{\isacharunderscore}{\kern0pt}def\ \isacommand{by}\isamarkupfalse%
\ force\isanewline
\isanewline
\ \ \isacommand{have}\isamarkupfalse%
\ subsetH\ {\isacharcolon}{\kern0pt}\ {\isachardoublequoteopen}n\ {\isasymsubseteq}\ n\ {\isacharhash}{\kern0pt}{\isacharplus}{\kern0pt}\ m\ {\isasymand}\ m\ {\isasymsubseteq}\ n\ {\isacharhash}{\kern0pt}{\isacharplus}{\kern0pt}\ m{\isachardoublequoteclose}\ \isanewline
\ \ \ \ \isacommand{apply}\isamarkupfalse%
{\isacharparenleft}{\kern0pt}rule\ conjI{\isacharcomma}{\kern0pt}\ rule\ subsetI{\isacharcomma}{\kern0pt}\ rule\ ltD{\isacharcomma}{\kern0pt}\ rule{\isacharunderscore}{\kern0pt}tac\ b{\isacharequal}{\kern0pt}n\ \isakeyword{in}\ lt{\isacharunderscore}{\kern0pt}le{\isacharunderscore}{\kern0pt}lt{\isacharcomma}{\kern0pt}\ rule\ ltI{\isacharcomma}{\kern0pt}\ simp{\isacharcomma}{\kern0pt}\ simp\ add{\isacharcolon}{\kern0pt}nfH{\isacharparenright}{\kern0pt}\isanewline
\ \ \ \ \isacommand{using}\isamarkupfalse%
\ nfH\ mgH\ \isanewline
\ \ \ \ \ \isacommand{apply}\isamarkupfalse%
\ force\ \isanewline
\ \ \ \ \isacommand{apply}\isamarkupfalse%
{\isacharparenleft}{\kern0pt}rule\ subsetI{\isacharcomma}{\kern0pt}\ rule\ ltD{\isacharcomma}{\kern0pt}\ rule{\isacharunderscore}{\kern0pt}tac\ b{\isacharequal}{\kern0pt}m\ \isakeyword{in}\ lt{\isacharunderscore}{\kern0pt}le{\isacharunderscore}{\kern0pt}lt{\isacharcomma}{\kern0pt}\ rule\ ltI{\isacharcomma}{\kern0pt}\ simp{\isacharcomma}{\kern0pt}\ simp\ add{\isacharcolon}{\kern0pt}mgH{\isacharparenright}{\kern0pt}\isanewline
\ \ \ \ \isacommand{using}\isamarkupfalse%
\ nfH\ mgH\ \isanewline
\ \ \ \ \isacommand{apply}\isamarkupfalse%
\ force\ \isanewline
\ \ \ \ \isacommand{done}\isamarkupfalse%
\ \isanewline
\isanewline
\ \ \isacommand{have}\isamarkupfalse%
\ fvn{\isacharcolon}{\kern0pt}\ {\isachardoublequoteopen}{\isasymAnd}x{\isachardot}{\kern0pt}\ x\ {\isasymin}\ A\ {\isasymLongrightarrow}\ f{\isacharbackquote}{\kern0pt}x\ {\isasymin}\ n{\isachardoublequoteclose}\ \isanewline
\ \ \ \ \isacommand{apply}\isamarkupfalse%
{\isacharparenleft}{\kern0pt}rule\ function{\isacharunderscore}{\kern0pt}value{\isacharunderscore}{\kern0pt}in{\isacharparenright}{\kern0pt}\ \isanewline
\ \ \ \ \isacommand{using}\isamarkupfalse%
\ nfH\ inj{\isacharunderscore}{\kern0pt}def\ \isanewline
\ \ \ \ \isacommand{by}\isamarkupfalse%
\ auto\isanewline
\ \ \isacommand{have}\isamarkupfalse%
\ gvm{\isacharcolon}{\kern0pt}\ {\isachardoublequoteopen}{\isasymAnd}x{\isachardot}{\kern0pt}\ x\ {\isasymin}\ B\ {\isasymLongrightarrow}\ g{\isacharbackquote}{\kern0pt}x\ {\isasymin}\ m{\isachardoublequoteclose}\ \isanewline
\ \ \ \ \isacommand{apply}\isamarkupfalse%
{\isacharparenleft}{\kern0pt}rule\ function{\isacharunderscore}{\kern0pt}value{\isacharunderscore}{\kern0pt}in{\isacharparenright}{\kern0pt}\ \isanewline
\ \ \ \ \isacommand{using}\isamarkupfalse%
\ mgH\ inj{\isacharunderscore}{\kern0pt}def\ \isanewline
\ \ \ \ \isacommand{by}\isamarkupfalse%
\ auto\isanewline
\ \ \isacommand{then}\isamarkupfalse%
\ \isacommand{have}\isamarkupfalse%
\ gvnat{\isacharcolon}{\kern0pt}\ {\isachardoublequoteopen}{\isasymAnd}x{\isachardot}{\kern0pt}\ x\ {\isasymin}\ B\ {\isasymLongrightarrow}\ g{\isacharbackquote}{\kern0pt}x\ {\isasymin}\ nat{\isachardoublequoteclose}\ \isanewline
\ \ \ \ \isacommand{apply}\isamarkupfalse%
{\isacharparenleft}{\kern0pt}rule{\isacharunderscore}{\kern0pt}tac\ ltD{\isacharparenright}{\kern0pt}\isanewline
\ \ \ \ \isacommand{apply}\isamarkupfalse%
{\isacharparenleft}{\kern0pt}rule{\isacharunderscore}{\kern0pt}tac\ j{\isacharequal}{\kern0pt}m\ \isakeyword{in}\ lt{\isacharunderscore}{\kern0pt}trans{\isacharcomma}{\kern0pt}\ rule\ ltI{\isacharparenright}{\kern0pt}\isanewline
\ \ \ \ \isacommand{using}\isamarkupfalse%
\ mgH\ ltI\isanewline
\ \ \ \ \isacommand{by}\isamarkupfalse%
\ auto\isanewline
\isanewline
\ \ \isacommand{define}\isamarkupfalse%
\ h\ \isakeyword{where}\ {\isachardoublequoteopen}h\ {\isasymequiv}\ {\isacharbraceleft}{\kern0pt}\ {\isacharless}{\kern0pt}a{\isacharcomma}{\kern0pt}\ if\ a\ {\isasymin}\ A\ then\ f{\isacharbackquote}{\kern0pt}a\ else\ n\ {\isacharhash}{\kern0pt}{\isacharplus}{\kern0pt}\ g{\isacharbackquote}{\kern0pt}a{\isachargreater}{\kern0pt}\ {\isachardot}{\kern0pt}\ a\ {\isasymin}\ A\ {\isasymunion}\ B\ {\isacharbraceright}{\kern0pt}{\isachardoublequoteclose}\ \isanewline
\isanewline
\ \ \isacommand{define}\isamarkupfalse%
\ hfm\ \isakeyword{where}\ {\isachardoublequoteopen}hfm\ {\isasymequiv}\ Or{\isacharparenleft}{\kern0pt}Exists{\isacharparenleft}{\kern0pt}Exists{\isacharparenleft}{\kern0pt}And{\isacharparenleft}{\kern0pt}Member{\isacharparenleft}{\kern0pt}{\isadigit{1}}{\isacharcomma}{\kern0pt}\ {\isadigit{3}}{\isacharparenright}{\kern0pt}{\isacharcomma}{\kern0pt}\ And{\isacharparenleft}{\kern0pt}fun{\isacharunderscore}{\kern0pt}apply{\isacharunderscore}{\kern0pt}fm{\isacharparenleft}{\kern0pt}{\isadigit{4}}{\isacharcomma}{\kern0pt}\ {\isadigit{1}}{\isacharcomma}{\kern0pt}\ {\isadigit{0}}{\isacharparenright}{\kern0pt}{\isacharcomma}{\kern0pt}\ pair{\isacharunderscore}{\kern0pt}fm{\isacharparenleft}{\kern0pt}{\isadigit{1}}{\isacharcomma}{\kern0pt}\ {\isadigit{0}}{\isacharcomma}{\kern0pt}\ {\isadigit{2}}{\isacharparenright}{\kern0pt}{\isacharparenright}{\kern0pt}{\isacharparenright}{\kern0pt}{\isacharparenright}{\kern0pt}{\isacharparenright}{\kern0pt}{\isacharcomma}{\kern0pt}\isanewline
\ \ \ \ \ \ \ \ \ \ \ \ \ \ \ \ \ \ \ \ \ \ \ \ \ \ \ \ Exists{\isacharparenleft}{\kern0pt}Exists{\isacharparenleft}{\kern0pt}Exists{\isacharparenleft}{\kern0pt}And{\isacharparenleft}{\kern0pt}Neg{\isacharparenleft}{\kern0pt}Member{\isacharparenleft}{\kern0pt}{\isadigit{2}}{\isacharcomma}{\kern0pt}\ {\isadigit{4}}{\isacharparenright}{\kern0pt}{\isacharparenright}{\kern0pt}{\isacharcomma}{\kern0pt}\ And{\isacharparenleft}{\kern0pt}fun{\isacharunderscore}{\kern0pt}apply{\isacharunderscore}{\kern0pt}fm{\isacharparenleft}{\kern0pt}{\isadigit{6}}{\isacharcomma}{\kern0pt}\ {\isadigit{2}}{\isacharcomma}{\kern0pt}\ {\isadigit{1}}{\isacharparenright}{\kern0pt}{\isacharcomma}{\kern0pt}\ And{\isacharparenleft}{\kern0pt}fun{\isacharunderscore}{\kern0pt}apply{\isacharunderscore}{\kern0pt}fm{\isacharparenleft}{\kern0pt}{\isadigit{7}}{\isacharcomma}{\kern0pt}\ {\isadigit{1}}{\isacharcomma}{\kern0pt}\ {\isadigit{0}}{\isacharparenright}{\kern0pt}{\isacharcomma}{\kern0pt}\ pair{\isacharunderscore}{\kern0pt}fm{\isacharparenleft}{\kern0pt}{\isadigit{2}}{\isacharcomma}{\kern0pt}\ {\isadigit{0}}{\isacharcomma}{\kern0pt}\ {\isadigit{3}}{\isacharparenright}{\kern0pt}{\isacharparenright}{\kern0pt}{\isacharparenright}{\kern0pt}{\isacharparenright}{\kern0pt}{\isacharparenright}{\kern0pt}{\isacharparenright}{\kern0pt}{\isacharparenright}{\kern0pt}{\isacharparenright}{\kern0pt}\ \ {\isachardoublequoteclose}\isanewline
\isanewline
\ \ \isacommand{have}\isamarkupfalse%
\ {\isachardoublequoteopen}{\isacharbraceleft}{\kern0pt}\ v\ {\isasymin}\ {\isacharparenleft}{\kern0pt}A\ {\isasymunion}\ B{\isacharparenright}{\kern0pt}\ {\isasymtimes}\ {\isacharparenleft}{\kern0pt}n\ {\isacharhash}{\kern0pt}{\isacharplus}{\kern0pt}\ m{\isacharparenright}{\kern0pt}{\isachardot}{\kern0pt}\ sats{\isacharparenleft}{\kern0pt}M{\isacharcomma}{\kern0pt}\ hfm{\isacharcomma}{\kern0pt}\ {\isacharbrackleft}{\kern0pt}v{\isacharbrackright}{\kern0pt}\ {\isacharat}{\kern0pt}\ {\isacharbrackleft}{\kern0pt}A{\isacharcomma}{\kern0pt}\ f{\isacharcomma}{\kern0pt}\ g{\isacharcomma}{\kern0pt}\ add{\isacharunderscore}{\kern0pt}fun{\isacharparenleft}{\kern0pt}n{\isacharparenright}{\kern0pt}{\isacharbrackright}{\kern0pt}{\isacharparenright}{\kern0pt}\ {\isacharbraceright}{\kern0pt}\ {\isacharequal}{\kern0pt}\ \isanewline
\ \ \ \ \ \ \ \ {\isacharbraceleft}{\kern0pt}\ v\ {\isasymin}\ {\isacharparenleft}{\kern0pt}A\ {\isasymunion}\ B{\isacharparenright}{\kern0pt}\ {\isasymtimes}\ {\isacharparenleft}{\kern0pt}n\ {\isacharhash}{\kern0pt}{\isacharplus}{\kern0pt}\ m{\isacharparenright}{\kern0pt}{\isachardot}{\kern0pt}\ {\isacharparenleft}{\kern0pt}{\isasymexists}x\ {\isasymin}\ M{\isachardot}{\kern0pt}\ {\isasymexists}fx\ {\isasymin}\ M{\isachardot}{\kern0pt}\ x\ {\isasymin}\ A\ {\isasymand}\ fun{\isacharunderscore}{\kern0pt}apply{\isacharparenleft}{\kern0pt}{\isacharhash}{\kern0pt}{\isacharhash}{\kern0pt}M{\isacharcomma}{\kern0pt}\ f{\isacharcomma}{\kern0pt}\ x{\isacharcomma}{\kern0pt}\ fx{\isacharparenright}{\kern0pt}\ {\isasymand}\ pair{\isacharparenleft}{\kern0pt}{\isacharhash}{\kern0pt}{\isacharhash}{\kern0pt}M{\isacharcomma}{\kern0pt}\ x{\isacharcomma}{\kern0pt}\ fx{\isacharcomma}{\kern0pt}\ v{\isacharparenright}{\kern0pt}{\isacharparenright}{\kern0pt}\ {\isasymor}\ \isanewline
\ \ \ \ \ \ \ \ \ \ \ \ \ \ \ \ \ \ \ \ \ \ \ \ \ \ \ \ \ \ \ \ \ \ \ {\isacharparenleft}{\kern0pt}{\isasymexists}x\ {\isasymin}\ M{\isachardot}{\kern0pt}\ {\isasymexists}gx\ {\isasymin}\ M{\isachardot}{\kern0pt}\ {\isasymexists}gxn\ {\isasymin}\ M{\isachardot}{\kern0pt}\ x\ {\isasymnotin}\ A\ {\isasymand}\ fun{\isacharunderscore}{\kern0pt}apply{\isacharparenleft}{\kern0pt}{\isacharhash}{\kern0pt}{\isacharhash}{\kern0pt}M{\isacharcomma}{\kern0pt}\ g{\isacharcomma}{\kern0pt}\ x{\isacharcomma}{\kern0pt}\ gx{\isacharparenright}{\kern0pt}\ {\isasymand}\ fun{\isacharunderscore}{\kern0pt}apply{\isacharparenleft}{\kern0pt}{\isacharhash}{\kern0pt}{\isacharhash}{\kern0pt}M{\isacharcomma}{\kern0pt}\ add{\isacharunderscore}{\kern0pt}fun{\isacharparenleft}{\kern0pt}n{\isacharparenright}{\kern0pt}{\isacharcomma}{\kern0pt}\ gx{\isacharcomma}{\kern0pt}\ gxn{\isacharparenright}{\kern0pt}\ {\isasymand}\ pair{\isacharparenleft}{\kern0pt}{\isacharhash}{\kern0pt}{\isacharhash}{\kern0pt}M{\isacharcomma}{\kern0pt}\ x{\isacharcomma}{\kern0pt}\ gxn{\isacharcomma}{\kern0pt}\ v{\isacharparenright}{\kern0pt}{\isacharparenright}{\kern0pt}\ {\isacharbraceright}{\kern0pt}{\isachardoublequoteclose}\ \isanewline
\ \ \ \ {\isacharparenleft}{\kern0pt}\isakeyword{is}\ {\isachardoublequoteopen}{\isacharquery}{\kern0pt}A\ {\isacharequal}{\kern0pt}\ {\isacharunderscore}{\kern0pt}{\isachardoublequoteclose}{\isacharparenright}{\kern0pt}\isanewline
\ \ \ \ \isanewline
\ \ \ \ \isacommand{unfolding}\isamarkupfalse%
\ hfm{\isacharunderscore}{\kern0pt}def\isanewline
\ \ \ \ \isacommand{apply}\isamarkupfalse%
{\isacharparenleft}{\kern0pt}rule\ iff{\isacharunderscore}{\kern0pt}eq{\isacharcomma}{\kern0pt}\ rule\ iff{\isacharunderscore}{\kern0pt}flip{\isacharcomma}{\kern0pt}\ rule\ finite{\isacharunderscore}{\kern0pt}M{\isacharunderscore}{\kern0pt}union{\isacharunderscore}{\kern0pt}h{\isacharunderscore}{\kern0pt}elem{\isacharunderscore}{\kern0pt}fm{\isacharunderscore}{\kern0pt}auto{\isacharparenright}{\kern0pt}\isanewline
\ \ \ \ \ \ \ \ \ \ \isacommand{apply}\isamarkupfalse%
\ auto{\isacharbrackleft}{\kern0pt}{\isadigit{6}}{\isacharbrackright}{\kern0pt}\isanewline
\ \ \ \ \isacommand{using}\isamarkupfalse%
\ assms\ nfH\ mgH\ add{\isacharunderscore}{\kern0pt}fun{\isacharunderscore}{\kern0pt}in{\isacharunderscore}{\kern0pt}M\ cartprod{\isacharunderscore}{\kern0pt}closed\ Un{\isacharunderscore}{\kern0pt}closed\ nat{\isacharunderscore}{\kern0pt}in{\isacharunderscore}{\kern0pt}M\ transM\ \isanewline
\ \ \ \ \isacommand{by}\isamarkupfalse%
\ auto\isanewline
\ \ \isacommand{also}\isamarkupfalse%
\ \isacommand{have}\isamarkupfalse%
\ {\isachardoublequoteopen}{\isachardot}{\kern0pt}{\isachardot}{\kern0pt}{\isachardot}{\kern0pt}\ {\isacharequal}{\kern0pt}\ {\isacharbraceleft}{\kern0pt}\ v\ {\isasymin}\ {\isacharparenleft}{\kern0pt}A\ {\isasymunion}\ B{\isacharparenright}{\kern0pt}\ {\isasymtimes}\ {\isacharparenleft}{\kern0pt}n\ {\isacharhash}{\kern0pt}{\isacharplus}{\kern0pt}\ m{\isacharparenright}{\kern0pt}{\isachardot}{\kern0pt}\ {\isacharparenleft}{\kern0pt}{\isasymexists}x\ {\isasymin}\ A{\isachardot}{\kern0pt}\ v\ {\isacharequal}{\kern0pt}\ {\isacharless}{\kern0pt}x{\isacharcomma}{\kern0pt}\ f{\isacharbackquote}{\kern0pt}x{\isachargreater}{\kern0pt}{\isacharparenright}{\kern0pt}\ {\isasymor}\ {\isacharparenleft}{\kern0pt}{\isasymexists}x\ {\isasymin}\ B{\isachardot}{\kern0pt}\ x\ {\isasymnotin}\ A\ {\isasymand}\ v\ {\isacharequal}{\kern0pt}\ {\isacharless}{\kern0pt}x{\isacharcomma}{\kern0pt}\ g{\isacharbackquote}{\kern0pt}x\ {\isacharhash}{\kern0pt}{\isacharplus}{\kern0pt}\ n{\isachargreater}{\kern0pt}{\isacharparenright}{\kern0pt}\ {\isacharbraceright}{\kern0pt}{\isachardoublequoteclose}\ \isanewline
\ \ \ \ \isacommand{apply}\isamarkupfalse%
{\isacharparenleft}{\kern0pt}rule\ iff{\isacharunderscore}{\kern0pt}eq{\isacharparenright}{\kern0pt}\isanewline
\ \ \ \ \isacommand{using}\isamarkupfalse%
\ assms\ nfH\ mgH\ transM\ add{\isacharunderscore}{\kern0pt}fun{\isacharunderscore}{\kern0pt}in{\isacharunderscore}{\kern0pt}M\ apply{\isacharunderscore}{\kern0pt}closed\ \ cartprod{\isacharunderscore}{\kern0pt}closed\ Un{\isacharunderscore}{\kern0pt}closed\ nat{\isacharunderscore}{\kern0pt}in{\isacharunderscore}{\kern0pt}M\isanewline
\ \ \ \ \isacommand{apply}\isamarkupfalse%
\ simp\isanewline
\ \ \ \ \isacommand{apply}\isamarkupfalse%
{\isacharparenleft}{\kern0pt}rule\ iff{\isacharunderscore}{\kern0pt}disjI{\isacharparenright}{\kern0pt}\isanewline
\ \ \ \ \isacommand{using}\isamarkupfalse%
\ assms\ transM\isanewline
\ \ \ \ \ \isacommand{apply}\isamarkupfalse%
\ auto{\isacharbrackleft}{\kern0pt}{\isadigit{1}}{\isacharbrackright}{\kern0pt}\isanewline
\ \ \ \ \isacommand{apply}\isamarkupfalse%
{\isacharparenleft}{\kern0pt}rule\ iffI{\isacharcomma}{\kern0pt}\ clarsimp{\isacharcomma}{\kern0pt}\ subst\ function{\isacharunderscore}{\kern0pt}apply{\isacharunderscore}{\kern0pt}equality{\isacharcomma}{\kern0pt}\ simp\ add{\isacharcolon}{\kern0pt}add{\isacharunderscore}{\kern0pt}fun{\isacharunderscore}{\kern0pt}def{\isacharcomma}{\kern0pt}\ rule\ conjI{\isacharparenright}{\kern0pt}\isanewline
\ \ \ \ \ \ \ \ \isacommand{apply}\isamarkupfalse%
{\isacharparenleft}{\kern0pt}rule\ gvnat{\isacharcomma}{\kern0pt}\ simp{\isacharcomma}{\kern0pt}\ force{\isacharparenright}{\kern0pt}\isanewline
\ \ \ \ \isacommand{using}\isamarkupfalse%
\ add{\isacharunderscore}{\kern0pt}fun{\isacharunderscore}{\kern0pt}type\ Pi{\isacharunderscore}{\kern0pt}def\isanewline
\ \ \ \ \ \ \isacommand{apply}\isamarkupfalse%
\ auto{\isacharbrackleft}{\kern0pt}{\isadigit{2}}{\isacharbrackright}{\kern0pt}\isanewline
\ \ \ \ \isacommand{apply}\isamarkupfalse%
\ clarsimp\isanewline
\ \ \ \ \isacommand{apply}\isamarkupfalse%
{\isacharparenleft}{\kern0pt}subst\ function{\isacharunderscore}{\kern0pt}apply{\isacharunderscore}{\kern0pt}equality{\isacharcomma}{\kern0pt}\ simp\ add{\isacharcolon}{\kern0pt}add{\isacharunderscore}{\kern0pt}fun{\isacharunderscore}{\kern0pt}def{\isacharcomma}{\kern0pt}\ rule\ conjI{\isacharparenright}{\kern0pt}\isanewline
\ \ \ \ \ \ \ \isacommand{apply}\isamarkupfalse%
{\isacharparenleft}{\kern0pt}rule\ gvnat{\isacharcomma}{\kern0pt}\ simp{\isacharcomma}{\kern0pt}\ force{\isacharparenright}{\kern0pt}\isanewline
\ \ \ \ \isacommand{using}\isamarkupfalse%
\ add{\isacharunderscore}{\kern0pt}fun{\isacharunderscore}{\kern0pt}type\ Pi{\isacharunderscore}{\kern0pt}def\isanewline
\ \ \ \ \ \isacommand{apply}\isamarkupfalse%
\ auto{\isacharbrackleft}{\kern0pt}{\isadigit{2}}{\isacharbrackright}{\kern0pt}\isanewline
\ \ \ \ \isacommand{done}\isamarkupfalse%
\isanewline
\ \ \isacommand{also}\isamarkupfalse%
\ \isacommand{have}\isamarkupfalse%
\ {\isachardoublequoteopen}{\isachardot}{\kern0pt}{\isachardot}{\kern0pt}{\isachardot}{\kern0pt}\ {\isacharequal}{\kern0pt}\ h{\isachardoublequoteclose}\ \isanewline
\ \ \ \ \isacommand{unfolding}\isamarkupfalse%
\ h{\isacharunderscore}{\kern0pt}def\isanewline
\ \ \ \ \isacommand{apply}\isamarkupfalse%
{\isacharparenleft}{\kern0pt}rule\ equality{\isacharunderscore}{\kern0pt}iffI{\isacharcomma}{\kern0pt}\ rule\ iffI{\isacharcomma}{\kern0pt}\ clarsimp{\isacharparenright}{\kern0pt}\isanewline
\ \ \ \ \ \isacommand{apply}\isamarkupfalse%
{\isacharparenleft}{\kern0pt}rule\ disjE{\isacharcomma}{\kern0pt}\ assumption{\isacharcomma}{\kern0pt}\ simp{\isacharparenright}{\kern0pt}\isanewline
\ \ \ \ \ \ \isacommand{apply}\isamarkupfalse%
{\isacharparenleft}{\kern0pt}rename{\isacharunderscore}{\kern0pt}tac\ x\ y{\isacharcomma}{\kern0pt}\ rule{\isacharunderscore}{\kern0pt}tac\ x{\isacharequal}{\kern0pt}x\ \isakeyword{in}\ bexI{\isacharcomma}{\kern0pt}\ clarsimp{\isacharcomma}{\kern0pt}\ force{\isacharparenright}{\kern0pt}\isanewline
\ \ \ \ \ \isacommand{apply}\isamarkupfalse%
{\isacharparenleft}{\kern0pt}rename{\isacharunderscore}{\kern0pt}tac\ x\ y{\isacharcomma}{\kern0pt}\ rule{\isacharunderscore}{\kern0pt}tac\ x{\isacharequal}{\kern0pt}x\ \isakeyword{in}\ bexI{\isacharcomma}{\kern0pt}\ force{\isacharcomma}{\kern0pt}\ force{\isacharparenright}{\kern0pt}\isanewline
\ \ \ \ \isacommand{apply}\isamarkupfalse%
\ clarsimp\isanewline
\ \ \ \ \isacommand{apply}\isamarkupfalse%
{\isacharparenleft}{\kern0pt}rule\ conjI{\isacharcomma}{\kern0pt}\ rule\ impI{\isacharparenright}{\kern0pt}\isanewline
\ \ \ \ \ \isacommand{apply}\isamarkupfalse%
{\isacharparenleft}{\kern0pt}rule\ ltD{\isacharcomma}{\kern0pt}\ rule{\isacharunderscore}{\kern0pt}tac\ b{\isacharequal}{\kern0pt}n\ \isakeyword{in}\ lt{\isacharunderscore}{\kern0pt}le{\isacharunderscore}{\kern0pt}lt{\isacharcomma}{\kern0pt}\ rule\ ltI{\isacharcomma}{\kern0pt}\ rule\ fvn{\isacharparenright}{\kern0pt}\isanewline
\ \ \ \ \isacommand{using}\isamarkupfalse%
\ nfH\ mgH\ \isanewline
\ \ \ \ \ \ \ \isacommand{apply}\isamarkupfalse%
\ auto{\isacharbrackleft}{\kern0pt}{\isadigit{3}}{\isacharbrackright}{\kern0pt}\isanewline
\ \ \ \ \isacommand{apply}\isamarkupfalse%
{\isacharparenleft}{\kern0pt}rule\ impI{\isacharcomma}{\kern0pt}\ rule\ conjI{\isacharcomma}{\kern0pt}\ rule\ ltD{\isacharcomma}{\kern0pt}\ rule\ add{\isacharunderscore}{\kern0pt}lt{\isacharunderscore}{\kern0pt}mono{\isadigit{2}}{\isacharparenright}{\kern0pt}\isanewline
\ \ \ \ \ \ \isacommand{apply}\isamarkupfalse%
{\isacharparenleft}{\kern0pt}rule\ ltI{\isacharcomma}{\kern0pt}\ rule\ gvm{\isacharparenright}{\kern0pt}\isanewline
\ \ \ \ \isacommand{using}\isamarkupfalse%
\ mgH\ \isanewline
\ \ \ \ \isacommand{by}\isamarkupfalse%
\ auto\isanewline
\isanewline
\ \ \isacommand{finally}\isamarkupfalse%
\ \isacommand{have}\isamarkupfalse%
\ {\isachardoublequoteopen}{\isacharquery}{\kern0pt}A\ {\isacharequal}{\kern0pt}\ h{\isachardoublequoteclose}\ \isacommand{by}\isamarkupfalse%
\ simp\isanewline
\isanewline
\ \ \isacommand{have}\isamarkupfalse%
\ {\isachardoublequoteopen}{\isacharquery}{\kern0pt}A\ {\isasymin}\ M{\isachardoublequoteclose}\ \isanewline
\ \ \ \ \isacommand{apply}\isamarkupfalse%
{\isacharparenleft}{\kern0pt}rule\ separation{\isacharunderscore}{\kern0pt}notation{\isacharcomma}{\kern0pt}\ rule\ separation{\isacharunderscore}{\kern0pt}ax{\isacharparenright}{\kern0pt}\isanewline
\ \ \ \ \isacommand{unfolding}\isamarkupfalse%
\ hfm{\isacharunderscore}{\kern0pt}def\isanewline
\ \ \ \ \ \ \ \isacommand{apply}\isamarkupfalse%
\ force\isanewline
\ \ \ \ \isacommand{using}\isamarkupfalse%
\ mgH\ nfH\ add{\isacharunderscore}{\kern0pt}fun{\isacharunderscore}{\kern0pt}in{\isacharunderscore}{\kern0pt}M\ assms\isanewline
\ \ \ \ \ \ \isacommand{apply}\isamarkupfalse%
\ force\ \isanewline
\ \ \ \ \ \isacommand{apply}\isamarkupfalse%
\ {\isacharparenleft}{\kern0pt}simp\ del{\isacharcolon}{\kern0pt}FOL{\isacharunderscore}{\kern0pt}sats{\isacharunderscore}{\kern0pt}iff\ pair{\isacharunderscore}{\kern0pt}abs\ add{\isacharcolon}{\kern0pt}\ fm{\isacharunderscore}{\kern0pt}defs\ nat{\isacharunderscore}{\kern0pt}simp{\isacharunderscore}{\kern0pt}union{\isacharparenright}{\kern0pt}\isanewline
\ \ \ \ \isacommand{using}\isamarkupfalse%
\ assms\ nfH\ mgH\ transM\ add{\isacharunderscore}{\kern0pt}fun{\isacharunderscore}{\kern0pt}in{\isacharunderscore}{\kern0pt}M\ apply{\isacharunderscore}{\kern0pt}closed\ \ cartprod{\isacharunderscore}{\kern0pt}closed\ Un{\isacharunderscore}{\kern0pt}closed\ nat{\isacharunderscore}{\kern0pt}in{\isacharunderscore}{\kern0pt}M\isanewline
\ \ \ \ \isacommand{apply}\isamarkupfalse%
\ simp\isanewline
\ \ \ \ \isacommand{done}\isamarkupfalse%
\isanewline
\ \ \isacommand{then}\isamarkupfalse%
\ \isacommand{have}\isamarkupfalse%
\ hinM{\isacharcolon}{\kern0pt}\ {\isachardoublequoteopen}h\ {\isasymin}\ M{\isachardoublequoteclose}\ \isacommand{using}\isamarkupfalse%
\ {\isacartoucheopen}{\isacharquery}{\kern0pt}A\ {\isacharequal}{\kern0pt}\ h{\isacartoucheclose}\ \isacommand{by}\isamarkupfalse%
\ auto\isanewline
\isanewline
\ \ \isacommand{have}\isamarkupfalse%
\ htype{\isacharcolon}{\kern0pt}\ {\isachardoublequoteopen}h\ {\isasymin}\ A\ {\isasymunion}\ B\ {\isasymrightarrow}\ n\ {\isacharhash}{\kern0pt}{\isacharplus}{\kern0pt}\ m{\isachardoublequoteclose}\isanewline
\ \ \ \ \isacommand{apply}\isamarkupfalse%
{\isacharparenleft}{\kern0pt}rule\ Pi{\isacharunderscore}{\kern0pt}memberI{\isacharparenright}{\kern0pt}\isanewline
\ \ \ \ \isacommand{using}\isamarkupfalse%
\ relation{\isacharunderscore}{\kern0pt}def\ function{\isacharunderscore}{\kern0pt}def\ h{\isacharunderscore}{\kern0pt}def\ \isanewline
\ \ \ \ \ \ \ \isacommand{apply}\isamarkupfalse%
\ auto{\isacharbrackleft}{\kern0pt}{\isadigit{3}}{\isacharbrackright}{\kern0pt}\isanewline
\ \ \ \ \isacommand{apply}\isamarkupfalse%
{\isacharparenleft}{\kern0pt}rule\ subsetI{\isacharparenright}{\kern0pt}\isanewline
\ \ \ \ \isacommand{unfolding}\isamarkupfalse%
\ h{\isacharunderscore}{\kern0pt}def\isanewline
\ \ \ \ \isacommand{apply}\isamarkupfalse%
\ clarsimp\isanewline
\ \ \ \ \isacommand{apply}\isamarkupfalse%
{\isacharparenleft}{\kern0pt}rule\ conjI{\isacharparenright}{\kern0pt}\isanewline
\ \ \ \ \ \isacommand{apply}\isamarkupfalse%
{\isacharparenleft}{\kern0pt}rule\ impI{\isacharcomma}{\kern0pt}\ rule{\isacharunderscore}{\kern0pt}tac\ A{\isacharequal}{\kern0pt}A\ \isakeyword{in}\ function{\isacharunderscore}{\kern0pt}value{\isacharunderscore}{\kern0pt}in{\isacharparenright}{\kern0pt}\isanewline
\ \ \ \ \ \ \isacommand{apply}\isamarkupfalse%
{\isacharparenleft}{\kern0pt}rule\ Pi{\isacharunderscore}{\kern0pt}memberI{\isacharparenright}{\kern0pt}\isanewline
\ \ \ \ \isacommand{using}\isamarkupfalse%
\ inj{\isacharunderscore}{\kern0pt}def\ relation{\isacharunderscore}{\kern0pt}def\ nfH\ Pi{\isacharunderscore}{\kern0pt}def\ subsetH\isanewline
\ \ \ \ \ \ \ \ \ \isacommand{apply}\isamarkupfalse%
\ auto{\isacharbrackleft}{\kern0pt}{\isadigit{5}}{\isacharbrackright}{\kern0pt}\isanewline
\ \ \ \ \isacommand{apply}\isamarkupfalse%
{\isacharparenleft}{\kern0pt}rule\ impI{\isacharcomma}{\kern0pt}\ rule\ ltD{\isacharparenright}{\kern0pt}\isanewline
\ \ \ \ \isacommand{using}\isamarkupfalse%
\ mgH\ gvnat\ \isanewline
\ \ \ \ \isacommand{apply}\isamarkupfalse%
\ simp\isanewline
\ \ \ \ \isacommand{apply}\isamarkupfalse%
{\isacharparenleft}{\kern0pt}rule\ add{\isacharunderscore}{\kern0pt}lt{\isacharunderscore}{\kern0pt}mono{\isadigit{2}}{\isacharcomma}{\kern0pt}\ rule\ ltI{\isacharparenright}{\kern0pt}\isanewline
\ \ \ \ \isacommand{using}\isamarkupfalse%
\ mgH\ gvm\ \isanewline
\ \ \ \ \isacommand{by}\isamarkupfalse%
\ auto\isanewline
\ \ \isacommand{have}\isamarkupfalse%
\ hinj{\isacharcolon}{\kern0pt}\ {\isachardoublequoteopen}{\isasymAnd}x\ y{\isachardot}{\kern0pt}\ x\ {\isasymin}\ A\ {\isasymunion}\ B\ {\isasymLongrightarrow}\ y\ {\isasymin}\ A\ {\isasymunion}\ B\ {\isasymLongrightarrow}\ h{\isacharbackquote}{\kern0pt}x\ {\isacharequal}{\kern0pt}\ h{\isacharbackquote}{\kern0pt}y\ {\isasymlongrightarrow}\ x\ {\isacharequal}{\kern0pt}\ y{\isachardoublequoteclose}\ \isanewline
\ \ \ \ \isacommand{apply}\isamarkupfalse%
{\isacharparenleft}{\kern0pt}subst\ function{\isacharunderscore}{\kern0pt}apply{\isacharunderscore}{\kern0pt}equality{\isacharparenright}{\kern0pt}\isanewline
\ \ \ \ \ \ \isacommand{apply}\isamarkupfalse%
{\isacharparenleft}{\kern0pt}simp\ add{\isacharcolon}{\kern0pt}h{\isacharunderscore}{\kern0pt}def{\isacharparenright}{\kern0pt}\isanewline
\ \ \ \ \isacommand{using}\isamarkupfalse%
\ htype\ Pi{\isacharunderscore}{\kern0pt}def\isanewline
\ \ \ \ \ \isacommand{apply}\isamarkupfalse%
\ force\isanewline
\ \ \ \ \isacommand{apply}\isamarkupfalse%
{\isacharparenleft}{\kern0pt}subst\ {\isacharparenleft}{\kern0pt}{\isadigit{3}}{\isacharparenright}{\kern0pt}\ function{\isacharunderscore}{\kern0pt}apply{\isacharunderscore}{\kern0pt}equality{\isacharparenright}{\kern0pt}\isanewline
\ \ \ \ \ \ \isacommand{apply}\isamarkupfalse%
{\isacharparenleft}{\kern0pt}simp\ add{\isacharcolon}{\kern0pt}h{\isacharunderscore}{\kern0pt}def{\isacharparenright}{\kern0pt}\isanewline
\ \ \ \ \isacommand{using}\isamarkupfalse%
\ htype\ Pi{\isacharunderscore}{\kern0pt}def\isanewline
\ \ \ \ \ \isacommand{apply}\isamarkupfalse%
\ force\isanewline
\ \ \ \ \isacommand{apply}\isamarkupfalse%
\ simp\isanewline
\ \ \ \ \isacommand{apply}\isamarkupfalse%
{\isacharparenleft}{\kern0pt}rule\ conjI{\isacharcomma}{\kern0pt}\ rule\ impI{\isacharparenright}{\kern0pt}{\isacharplus}{\kern0pt}\isanewline
\ \ \ \ \isacommand{using}\isamarkupfalse%
\ nfH\ inj{\isacharunderscore}{\kern0pt}def\ \isanewline
\ \ \ \ \ \ \isacommand{apply}\isamarkupfalse%
\ force\isanewline
\ \ \ \ \ \isacommand{apply}\isamarkupfalse%
{\isacharparenleft}{\kern0pt}rule\ impI{\isacharparenright}{\kern0pt}{\isacharplus}{\kern0pt}\isanewline
\ \ \ \ \ \isacommand{apply}\isamarkupfalse%
{\isacharparenleft}{\kern0pt}rename{\isacharunderscore}{\kern0pt}tac\ x\ y{\isacharcomma}{\kern0pt}\ subgoal{\isacharunderscore}{\kern0pt}tac\ {\isachardoublequoteopen}f{\isacharbackquote}{\kern0pt}x\ {\isacharless}{\kern0pt}\ n\ {\isacharhash}{\kern0pt}{\isacharplus}{\kern0pt}\ g{\isacharbackquote}{\kern0pt}y{\isachardoublequoteclose}{\isacharcomma}{\kern0pt}\ force{\isacharparenright}{\kern0pt}\isanewline
\ \ \ \ \ \isacommand{apply}\isamarkupfalse%
{\isacharparenleft}{\kern0pt}rule{\isacharunderscore}{\kern0pt}tac\ b{\isacharequal}{\kern0pt}n\ \isakeyword{in}\ lt{\isacharunderscore}{\kern0pt}le{\isacharunderscore}{\kern0pt}lt{\isacharcomma}{\kern0pt}\ rule\ ltI{\isacharparenright}{\kern0pt}\isanewline
\ \ \ \ \isacommand{using}\isamarkupfalse%
\ fvn\ nfH\ \isanewline
\ \ \ \ \ \ \ \isacommand{apply}\isamarkupfalse%
\ {\isacharparenleft}{\kern0pt}force{\isacharcomma}{\kern0pt}\ force{\isacharcomma}{\kern0pt}\ force{\isacharparenright}{\kern0pt}\isanewline
\ \ \ \ \isacommand{apply}\isamarkupfalse%
{\isacharparenleft}{\kern0pt}rule\ impI{\isacharcomma}{\kern0pt}\ rule\ conjI{\isacharparenright}{\kern0pt}{\isacharplus}{\kern0pt}\isanewline
\ \ \ \ \ \isacommand{apply}\isamarkupfalse%
{\isacharparenleft}{\kern0pt}rule\ impI{\isacharparenright}{\kern0pt}\isanewline
\ \ \ \ \isacommand{apply}\isamarkupfalse%
{\isacharparenleft}{\kern0pt}rename{\isacharunderscore}{\kern0pt}tac\ x\ y{\isacharcomma}{\kern0pt}\ subgoal{\isacharunderscore}{\kern0pt}tac\ {\isachardoublequoteopen}f{\isacharbackquote}{\kern0pt}y\ {\isacharless}{\kern0pt}\ n\ {\isacharhash}{\kern0pt}{\isacharplus}{\kern0pt}\ g{\isacharbackquote}{\kern0pt}x{\isachardoublequoteclose}{\isacharcomma}{\kern0pt}\ force{\isacharparenright}{\kern0pt}\isanewline
\ \ \ \ \ \isacommand{apply}\isamarkupfalse%
{\isacharparenleft}{\kern0pt}rule{\isacharunderscore}{\kern0pt}tac\ b{\isacharequal}{\kern0pt}n\ \isakeyword{in}\ lt{\isacharunderscore}{\kern0pt}le{\isacharunderscore}{\kern0pt}lt{\isacharcomma}{\kern0pt}\ rule\ ltI{\isacharparenright}{\kern0pt}\isanewline
\ \ \ \ \isacommand{using}\isamarkupfalse%
\ fvn\ nfH\ \isanewline
\ \ \ \ \ \ \ \isacommand{apply}\isamarkupfalse%
\ {\isacharparenleft}{\kern0pt}force{\isacharcomma}{\kern0pt}\ force{\isacharcomma}{\kern0pt}\ force{\isacharparenright}{\kern0pt}\isanewline
\ \ \ \ \isacommand{using}\isamarkupfalse%
\ gvnat\ inj{\isacharunderscore}{\kern0pt}def\ mgH\ \isanewline
\ \ \ \ \isacommand{by}\isamarkupfalse%
\ auto\isanewline
\isanewline
\ \ \isacommand{have}\isamarkupfalse%
\ {\isachardoublequoteopen}h\ {\isasymin}\ inj{\isacharparenleft}{\kern0pt}A\ {\isasymunion}\ B{\isacharcomma}{\kern0pt}\ n\ {\isacharhash}{\kern0pt}{\isacharplus}{\kern0pt}\ m{\isacharparenright}{\kern0pt}{\isachardoublequoteclose}\ \isacommand{using}\isamarkupfalse%
\ htype\ hinj\ inj{\isacharunderscore}{\kern0pt}def\ \isacommand{by}\isamarkupfalse%
\ auto\isanewline
\ \ \isacommand{then}\isamarkupfalse%
\ \isacommand{show}\isamarkupfalse%
\ {\isacharquery}{\kern0pt}thesis\ \isanewline
\ \ \ \ \isacommand{unfolding}\isamarkupfalse%
\ finite{\isacharunderscore}{\kern0pt}M{\isacharunderscore}{\kern0pt}def\ \isanewline
\ \ \ \ \isacommand{apply}\isamarkupfalse%
{\isacharparenleft}{\kern0pt}rule{\isacharunderscore}{\kern0pt}tac\ x{\isacharequal}{\kern0pt}{\isachardoublequoteopen}n{\isacharhash}{\kern0pt}{\isacharplus}{\kern0pt}m{\isachardoublequoteclose}\ \isakeyword{in}\ bexI{\isacharparenright}{\kern0pt}\isanewline
\ \ \ \ \isacommand{using}\isamarkupfalse%
\ hinM\ mgH\ nfH\isanewline
\ \ \ \ \isacommand{by}\isamarkupfalse%
\ auto\isanewline
\isacommand{qed}\isamarkupfalse%
%
\endisatagproof
{\isafoldproof}%
%
\isadelimproof
\isanewline
%
\endisadelimproof
\isanewline
\isacommand{lemma}\isamarkupfalse%
\ Fn{\isacharunderscore}{\kern0pt}perms{\isacharunderscore}{\kern0pt}filter{\isacharunderscore}{\kern0pt}subset\ {\isacharcolon}{\kern0pt}\ {\isachardoublequoteopen}Fn{\isacharunderscore}{\kern0pt}perms{\isacharunderscore}{\kern0pt}filter\ {\isasymsubseteq}\ Pow{\isacharparenleft}{\kern0pt}Fn{\isacharunderscore}{\kern0pt}perms{\isacharparenright}{\kern0pt}\ {\isasyminter}\ M{\isachardoublequoteclose}\isanewline
%
\isadelimproof
%
\endisadelimproof
%
\isatagproof
\isacommand{proof}\isamarkupfalse%
{\isacharparenleft}{\kern0pt}rule\ subsetI{\isacharparenright}{\kern0pt}\isanewline
\ \ \isacommand{fix}\isamarkupfalse%
\ x\ \isacommand{assume}\isamarkupfalse%
\ {\isachardoublequoteopen}x\ {\isasymin}\ Fn{\isacharunderscore}{\kern0pt}perms{\isacharunderscore}{\kern0pt}filter{\isachardoublequoteclose}\ \isanewline
\ \ \isacommand{then}\isamarkupfalse%
\ \isacommand{have}\isamarkupfalse%
\ {\isachardoublequoteopen}x\ {\isasymin}\ forcing{\isacharunderscore}{\kern0pt}data{\isacharunderscore}{\kern0pt}partial{\isachardot}{\kern0pt}P{\isacharunderscore}{\kern0pt}auto{\isacharunderscore}{\kern0pt}subgroups{\isacharparenleft}{\kern0pt}Fn{\isacharcomma}{\kern0pt}\ Fn{\isacharunderscore}{\kern0pt}leq{\isacharcomma}{\kern0pt}\ M{\isacharcomma}{\kern0pt}\ Fn{\isacharunderscore}{\kern0pt}perms{\isacharparenright}{\kern0pt}{\isachardoublequoteclose}\ {\isacharparenleft}{\kern0pt}\isakeyword{is}\ {\isacharquery}{\kern0pt}A{\isacharparenright}{\kern0pt}\isanewline
\ \ \ \ \isacommand{using}\isamarkupfalse%
\ Fn{\isacharunderscore}{\kern0pt}perms{\isacharunderscore}{\kern0pt}filter{\isacharunderscore}{\kern0pt}def\isanewline
\ \ \ \ \isacommand{by}\isamarkupfalse%
\ auto\isanewline
\ \ \isacommand{have}\isamarkupfalse%
\ {\isachardoublequoteopen}{\isacharquery}{\kern0pt}A\ {\isasymlongrightarrow}\ x\ {\isasymsubseteq}\ Fn{\isacharunderscore}{\kern0pt}perms\ {\isasymand}\ x\ {\isasymin}\ M{\isachardoublequoteclose}\ \isanewline
\ \ \ \ \isacommand{apply}\isamarkupfalse%
{\isacharparenleft}{\kern0pt}subst\ forcing{\isacharunderscore}{\kern0pt}data{\isacharunderscore}{\kern0pt}partial{\isachardot}{\kern0pt}P{\isacharunderscore}{\kern0pt}auto{\isacharunderscore}{\kern0pt}subgroups{\isacharunderscore}{\kern0pt}def{\isacharparenright}{\kern0pt}\isanewline
\ \ \ \ \ \isacommand{apply}\isamarkupfalse%
{\isacharparenleft}{\kern0pt}rule\ Fn{\isacharunderscore}{\kern0pt}forcing{\isacharunderscore}{\kern0pt}data{\isacharunderscore}{\kern0pt}partial{\isacharparenright}{\kern0pt}\isanewline
\ \ \ \ \isacommand{by}\isamarkupfalse%
\ auto\isanewline
\ \ \isacommand{then}\isamarkupfalse%
\ \isacommand{show}\isamarkupfalse%
\ {\isachardoublequoteopen}x\ {\isasymin}\ Pow{\isacharparenleft}{\kern0pt}Fn{\isacharunderscore}{\kern0pt}perms{\isacharparenright}{\kern0pt}\ {\isasyminter}\ M{\isachardoublequoteclose}\ \isanewline
\ \ \ \ \isacommand{using}\isamarkupfalse%
\ {\isacartoucheopen}{\isacharquery}{\kern0pt}A{\isacartoucheclose}\isanewline
\ \ \ \ \isacommand{by}\isamarkupfalse%
\ auto\isanewline
\isacommand{qed}\isamarkupfalse%
%
\endisatagproof
{\isafoldproof}%
%
\isadelimproof
\isanewline
%
\endisadelimproof
\isanewline
\isacommand{lemma}\isamarkupfalse%
\ Fn{\isacharunderscore}{\kern0pt}perms{\isacharunderscore}{\kern0pt}filter{\isacharunderscore}{\kern0pt}intersection\ {\isacharcolon}{\kern0pt}\isanewline
\ \ \isakeyword{fixes}\ A\ B\ \isanewline
\ \ \isakeyword{assumes}\ {\isachardoublequoteopen}A\ {\isasymin}\ Fn{\isacharunderscore}{\kern0pt}perms{\isacharunderscore}{\kern0pt}filter{\isachardoublequoteclose}\ {\isachardoublequoteopen}B\ {\isasymin}\ Fn{\isacharunderscore}{\kern0pt}perms{\isacharunderscore}{\kern0pt}filter{\isachardoublequoteclose}\ \isanewline
\ \ \isakeyword{shows}\ {\isachardoublequoteopen}A\ {\isasyminter}\ B\ {\isasymin}\ Fn{\isacharunderscore}{\kern0pt}perms{\isacharunderscore}{\kern0pt}filter{\isachardoublequoteclose}\ \isanewline
%
\isadelimproof
%
\endisadelimproof
%
\isatagproof
\isacommand{proof}\isamarkupfalse%
{\isacharminus}{\kern0pt}\ \isanewline
\ \ \isacommand{obtain}\isamarkupfalse%
\ E\ \isakeyword{where}\ EH{\isacharcolon}{\kern0pt}\ {\isachardoublequoteopen}E\ {\isasymin}\ Pow{\isacharparenleft}{\kern0pt}nat{\isacharparenright}{\kern0pt}\ {\isasyminter}\ M{\isachardoublequoteclose}\ {\isachardoublequoteopen}finite{\isacharunderscore}{\kern0pt}M{\isacharparenleft}{\kern0pt}E{\isacharparenright}{\kern0pt}{\isachardoublequoteclose}\ {\isachardoublequoteopen}Fix{\isacharparenleft}{\kern0pt}E{\isacharparenright}{\kern0pt}\ {\isasymsubseteq}\ A{\isachardoublequoteclose}\isanewline
\ \ \ \ \isacommand{using}\isamarkupfalse%
\ Fn{\isacharunderscore}{\kern0pt}perms{\isacharunderscore}{\kern0pt}filter{\isacharunderscore}{\kern0pt}def\ assms\isanewline
\ \ \ \ \isacommand{by}\isamarkupfalse%
\ auto\isanewline
\ \ \isacommand{obtain}\isamarkupfalse%
\ F\ \isakeyword{where}\ FH{\isacharcolon}{\kern0pt}\ {\isachardoublequoteopen}F\ {\isasymin}\ Pow{\isacharparenleft}{\kern0pt}nat{\isacharparenright}{\kern0pt}\ {\isasyminter}\ M{\isachardoublequoteclose}\ {\isachardoublequoteopen}finite{\isacharunderscore}{\kern0pt}M{\isacharparenleft}{\kern0pt}F{\isacharparenright}{\kern0pt}{\isachardoublequoteclose}\ {\isachardoublequoteopen}Fix{\isacharparenleft}{\kern0pt}F{\isacharparenright}{\kern0pt}\ {\isasymsubseteq}\ B{\isachardoublequoteclose}\isanewline
\ \ \ \ \isacommand{using}\isamarkupfalse%
\ Fn{\isacharunderscore}{\kern0pt}perms{\isacharunderscore}{\kern0pt}filter{\isacharunderscore}{\kern0pt}def\ assms\isanewline
\ \ \ \ \isacommand{by}\isamarkupfalse%
\ auto\isanewline
\isanewline
\ \ \isacommand{define}\isamarkupfalse%
\ G\ \isakeyword{where}\ {\isachardoublequoteopen}G\ {\isasymequiv}\ E\ {\isasymunion}\ F{\isachardoublequoteclose}\ \isanewline
\ \ \isacommand{have}\isamarkupfalse%
\ H{\isacharcolon}{\kern0pt}\ {\isachardoublequoteopen}Fix{\isacharparenleft}{\kern0pt}G{\isacharparenright}{\kern0pt}\ {\isasymsubseteq}\ A\ {\isasyminter}\ B{\isachardoublequoteclose}\ \isanewline
\ \ \isacommand{proof}\isamarkupfalse%
\ {\isacharminus}{\kern0pt}\isanewline
\ \ \ \ \isacommand{have}\isamarkupfalse%
\ {\isachardoublequoteopen}Fix{\isacharparenleft}{\kern0pt}G{\isacharparenright}{\kern0pt}\ {\isasymsubseteq}\ Fix{\isacharparenleft}{\kern0pt}E{\isacharparenright}{\kern0pt}\ {\isasyminter}\ Fix{\isacharparenleft}{\kern0pt}F{\isacharparenright}{\kern0pt}{\isachardoublequoteclose}\ \isanewline
\ \ \ \ \ \ \isacommand{using}\isamarkupfalse%
\ G{\isacharunderscore}{\kern0pt}def\ Fix{\isacharunderscore}{\kern0pt}subset\isanewline
\ \ \ \ \ \ \isacommand{by}\isamarkupfalse%
\ blast\isanewline
\ \ \ \ \isacommand{also}\isamarkupfalse%
\ \isacommand{have}\isamarkupfalse%
\ {\isachardoublequoteopen}{\isachardot}{\kern0pt}{\isachardot}{\kern0pt}{\isachardot}{\kern0pt}\ {\isasymsubseteq}\ A\ {\isasyminter}\ B{\isachardoublequoteclose}\ \isanewline
\ \ \ \ \ \ \isacommand{using}\isamarkupfalse%
\ EH\ FH\ \isanewline
\ \ \ \ \ \ \isacommand{by}\isamarkupfalse%
\ blast\isanewline
\ \ \ \ \isacommand{finally}\isamarkupfalse%
\ \isacommand{show}\isamarkupfalse%
\ {\isachardoublequoteopen}Fix{\isacharparenleft}{\kern0pt}G{\isacharparenright}{\kern0pt}\ {\isasymsubseteq}\ A\ {\isasyminter}\ B{\isachardoublequoteclose}\ \isacommand{by}\isamarkupfalse%
\ auto\isanewline
\ \ \isacommand{qed}\isamarkupfalse%
\isanewline
\isanewline
\ \ \isacommand{show}\isamarkupfalse%
\ {\isachardoublequoteopen}A\ {\isasyminter}\ B\ {\isasymin}\ Fn{\isacharunderscore}{\kern0pt}perms{\isacharunderscore}{\kern0pt}filter{\isachardoublequoteclose}\ \isanewline
\ \ \ \ \isacommand{unfolding}\isamarkupfalse%
\ Fn{\isacharunderscore}{\kern0pt}perms{\isacharunderscore}{\kern0pt}filter{\isacharunderscore}{\kern0pt}def\isanewline
\ \ \ \ \isacommand{apply}\isamarkupfalse%
\ simp\isanewline
\ \ \ \ \isacommand{apply}\isamarkupfalse%
{\isacharparenleft}{\kern0pt}rule\ conjI{\isacharcomma}{\kern0pt}\ subst\ forcing{\isacharunderscore}{\kern0pt}data{\isacharunderscore}{\kern0pt}partial{\isachardot}{\kern0pt}P{\isacharunderscore}{\kern0pt}auto{\isacharunderscore}{\kern0pt}subgroups{\isacharunderscore}{\kern0pt}def{\isacharcomma}{\kern0pt}\ rule\ Fn{\isacharunderscore}{\kern0pt}forcing{\isacharunderscore}{\kern0pt}data{\isacharunderscore}{\kern0pt}partial{\isacharparenright}{\kern0pt}\isanewline
\ \ \ \ \ \isacommand{apply}\isamarkupfalse%
\ simp\isanewline
\ \ \ \ \ \isacommand{apply}\isamarkupfalse%
{\isacharparenleft}{\kern0pt}rule\ conjI{\isacharparenright}{\kern0pt}\isanewline
\ \ \ \ \isacommand{using}\isamarkupfalse%
\ assms\ Fn{\isacharunderscore}{\kern0pt}perms{\isacharunderscore}{\kern0pt}filter{\isacharunderscore}{\kern0pt}subset\ \isanewline
\ \ \ \ \ \ \isacommand{apply}\isamarkupfalse%
\ force\isanewline
\ \ \ \ \ \isacommand{apply}\isamarkupfalse%
{\isacharparenleft}{\kern0pt}rule\ conjI{\isacharparenright}{\kern0pt}\isanewline
\ \ \ \ \isacommand{using}\isamarkupfalse%
\ assms\ int{\isacharunderscore}{\kern0pt}closed\ Fn{\isacharunderscore}{\kern0pt}perms{\isacharunderscore}{\kern0pt}filter{\isacharunderscore}{\kern0pt}in{\isacharunderscore}{\kern0pt}M\ transM\isanewline
\ \ \ \ \ \ \isacommand{apply}\isamarkupfalse%
\ force\isanewline
\ \ \ \ \ \isacommand{apply}\isamarkupfalse%
{\isacharparenleft}{\kern0pt}subst\ forcing{\isacharunderscore}{\kern0pt}data{\isacharunderscore}{\kern0pt}partial{\isachardot}{\kern0pt}is{\isacharunderscore}{\kern0pt}P{\isacharunderscore}{\kern0pt}auto{\isacharunderscore}{\kern0pt}group{\isacharunderscore}{\kern0pt}def{\isacharparenright}{\kern0pt}\isanewline
\ \ \ \ \ \ \isacommand{apply}\isamarkupfalse%
{\isacharparenleft}{\kern0pt}rule\ Fn{\isacharunderscore}{\kern0pt}forcing{\isacharunderscore}{\kern0pt}data{\isacharunderscore}{\kern0pt}partial{\isacharcomma}{\kern0pt}\ rule\ conjI{\isacharcomma}{\kern0pt}\ rule\ subsetI{\isacharcomma}{\kern0pt}\ simp{\isacharcomma}{\kern0pt}\ rule\ conjI{\isacharparenright}{\kern0pt}\isanewline
\ \ \ \ \isacommand{using}\isamarkupfalse%
\ assms\ Fn{\isacharunderscore}{\kern0pt}perms{\isacharunderscore}{\kern0pt}filter{\isacharunderscore}{\kern0pt}subset\ Fn{\isacharunderscore}{\kern0pt}perms{\isacharunderscore}{\kern0pt}def\ Fn{\isacharunderscore}{\kern0pt}perm{\isacharprime}{\kern0pt}{\isacharunderscore}{\kern0pt}type\ Fn{\isacharunderscore}{\kern0pt}perm{\isacharprime}{\kern0pt}{\isacharunderscore}{\kern0pt}is{\isacharunderscore}{\kern0pt}P{\isacharunderscore}{\kern0pt}auto\isanewline
\ \ \ \ \ \ \ \isacommand{apply}\isamarkupfalse%
\ auto{\isacharbrackleft}{\kern0pt}{\isadigit{2}}{\isacharbrackright}{\kern0pt}\isanewline
\ \ \ \ \ \isacommand{apply}\isamarkupfalse%
{\isacharparenleft}{\kern0pt}rule{\isacharunderscore}{\kern0pt}tac\ P{\isacharequal}{\kern0pt}{\isachardoublequoteopen}A\ {\isasymin}\ Fn{\isacharunderscore}{\kern0pt}perms{\isacharunderscore}{\kern0pt}filter\ {\isasymand}\ B\ {\isasymin}\ Fn{\isacharunderscore}{\kern0pt}perms{\isacharunderscore}{\kern0pt}filter{\isachardoublequoteclose}\ \isakeyword{in}\ mp{\isacharparenright}{\kern0pt}\ \isanewline
\ \ \ \ \ \ \isacommand{apply}\isamarkupfalse%
{\isacharparenleft}{\kern0pt}subst\ Fn{\isacharunderscore}{\kern0pt}perms{\isacharunderscore}{\kern0pt}filter{\isacharunderscore}{\kern0pt}def{\isacharcomma}{\kern0pt}\ subst\ forcing{\isacharunderscore}{\kern0pt}data{\isacharunderscore}{\kern0pt}partial{\isachardot}{\kern0pt}P{\isacharunderscore}{\kern0pt}auto{\isacharunderscore}{\kern0pt}subgroups{\isacharunderscore}{\kern0pt}def{\isacharcomma}{\kern0pt}\ rule\ Fn{\isacharunderscore}{\kern0pt}forcing{\isacharunderscore}{\kern0pt}data{\isacharunderscore}{\kern0pt}partial{\isacharparenright}{\kern0pt}{\isacharplus}{\kern0pt}\isanewline
\ \ \ \ \ \ \isacommand{apply}\isamarkupfalse%
\ simp\isanewline
\ \ \ \ \ \ \isacommand{apply}\isamarkupfalse%
{\isacharparenleft}{\kern0pt}subst\ forcing{\isacharunderscore}{\kern0pt}data{\isacharunderscore}{\kern0pt}partial{\isachardot}{\kern0pt}is{\isacharunderscore}{\kern0pt}P{\isacharunderscore}{\kern0pt}auto{\isacharunderscore}{\kern0pt}group{\isacharunderscore}{\kern0pt}def{\isacharcomma}{\kern0pt}\ rule\ Fn{\isacharunderscore}{\kern0pt}forcing{\isacharunderscore}{\kern0pt}data{\isacharunderscore}{\kern0pt}partial{\isacharparenright}{\kern0pt}{\isacharplus}{\kern0pt}\isanewline
\ \ \ \ \ \ \isacommand{apply}\isamarkupfalse%
\ force\isanewline
\ \ \ \ \isacommand{using}\isamarkupfalse%
\ assms\isanewline
\ \ \ \ \ \isacommand{apply}\isamarkupfalse%
\ force\isanewline
\ \ \ \ \isacommand{apply}\isamarkupfalse%
{\isacharparenleft}{\kern0pt}rule{\isacharunderscore}{\kern0pt}tac\ x{\isacharequal}{\kern0pt}G\ \isakeyword{in}\ bexI{\isacharparenright}{\kern0pt}\isanewline
\ \ \ \ \isacommand{using}\isamarkupfalse%
\ finite{\isacharunderscore}{\kern0pt}M{\isacharunderscore}{\kern0pt}union\ assms\ EH\ FH\ H\ Un{\isacharunderscore}{\kern0pt}closed\isanewline
\ \ \ \ \isacommand{unfolding}\isamarkupfalse%
\ G{\isacharunderscore}{\kern0pt}def\isanewline
\ \ \ \ \isacommand{by}\isamarkupfalse%
\ auto\isanewline
\isacommand{qed}\isamarkupfalse%
%
\endisatagproof
{\isafoldproof}%
%
\isadelimproof
\isanewline
%
\endisadelimproof
\isanewline
\isacommand{lemma}\isamarkupfalse%
\ Fn{\isacharunderscore}{\kern0pt}perms{\isacharunderscore}{\kern0pt}filter{\isacharunderscore}{\kern0pt}supergroup\ {\isacharcolon}{\kern0pt}\ \isanewline
\ \ \isakeyword{fixes}\ A\ B\isanewline
\ \ \isakeyword{assumes}\ {\isachardoublequoteopen}A\ {\isasymin}\ Fn{\isacharunderscore}{\kern0pt}perms{\isacharunderscore}{\kern0pt}filter{\isachardoublequoteclose}\ {\isachardoublequoteopen}B\ {\isasymin}\ forcing{\isacharunderscore}{\kern0pt}data{\isacharunderscore}{\kern0pt}partial{\isachardot}{\kern0pt}P{\isacharunderscore}{\kern0pt}auto{\isacharunderscore}{\kern0pt}subgroups{\isacharparenleft}{\kern0pt}Fn{\isacharcomma}{\kern0pt}\ Fn{\isacharunderscore}{\kern0pt}leq{\isacharcomma}{\kern0pt}\ M{\isacharcomma}{\kern0pt}\ Fn{\isacharunderscore}{\kern0pt}perms{\isacharparenright}{\kern0pt}{\isachardoublequoteclose}\ {\isachardoublequoteopen}A\ {\isasymsubseteq}\ B{\isachardoublequoteclose}\ \isanewline
\ \ \isakeyword{shows}\ {\isachardoublequoteopen}B\ {\isasymin}\ Fn{\isacharunderscore}{\kern0pt}perms{\isacharunderscore}{\kern0pt}filter{\isachardoublequoteclose}\ \isanewline
%
\isadelimproof
\ \ %
\endisadelimproof
%
\isatagproof
\isacommand{using}\isamarkupfalse%
\ assms\isanewline
\ \ \isacommand{unfolding}\isamarkupfalse%
\ Fn{\isacharunderscore}{\kern0pt}perms{\isacharunderscore}{\kern0pt}filter{\isacharunderscore}{\kern0pt}def\ Fn{\isacharunderscore}{\kern0pt}perms{\isacharunderscore}{\kern0pt}filter{\isacharunderscore}{\kern0pt}def\isanewline
\ \ \isacommand{by}\isamarkupfalse%
\ auto%
\endisatagproof
{\isafoldproof}%
%
\isadelimproof
\isanewline
%
\endisadelimproof
\isanewline
\isacommand{definition}\isamarkupfalse%
\ normal{\isacharunderscore}{\kern0pt}comp{\isacharunderscore}{\kern0pt}elem{\isacharunderscore}{\kern0pt}fm\ \isakeyword{where}\ \isanewline
\ \ {\isachardoublequoteopen}normal{\isacharunderscore}{\kern0pt}comp{\isacharunderscore}{\kern0pt}elem{\isacharunderscore}{\kern0pt}fm{\isacharparenleft}{\kern0pt}X{\isacharcomma}{\kern0pt}\ Y{\isacharcomma}{\kern0pt}\ F{\isacharparenright}{\kern0pt}\ {\isasymequiv}\ \isanewline
\ \ \ \ \ \ Exists{\isacharparenleft}{\kern0pt}Exists{\isacharparenleft}{\kern0pt}\isanewline
\ \ \ \ \ \ \ \ \ \ And{\isacharparenleft}{\kern0pt}converse{\isacharunderscore}{\kern0pt}fm{\isacharparenleft}{\kern0pt}F{\isacharhash}{\kern0pt}{\isacharplus}{\kern0pt}{\isadigit{2}}{\isacharcomma}{\kern0pt}\ {\isadigit{0}}{\isacharparenright}{\kern0pt}{\isacharcomma}{\kern0pt}\ \isanewline
\ \ \ \ \ \ \ \ \ \ And{\isacharparenleft}{\kern0pt}composition{\isacharunderscore}{\kern0pt}fm{\isacharparenleft}{\kern0pt}F{\isacharhash}{\kern0pt}{\isacharplus}{\kern0pt}{\isadigit{2}}{\isacharcomma}{\kern0pt}\ X{\isacharhash}{\kern0pt}{\isacharplus}{\kern0pt}{\isadigit{2}}{\isacharcomma}{\kern0pt}\ {\isadigit{1}}{\isacharparenright}{\kern0pt}{\isacharcomma}{\kern0pt}\ \isanewline
\ \ \ \ \ \ \ \ \ \ \ \ \ \ composition{\isacharunderscore}{\kern0pt}fm{\isacharparenleft}{\kern0pt}{\isadigit{1}}{\isacharcomma}{\kern0pt}\ {\isadigit{0}}{\isacharcomma}{\kern0pt}\ Y{\isacharhash}{\kern0pt}{\isacharplus}{\kern0pt}{\isadigit{2}}{\isacharparenright}{\kern0pt}{\isacharparenright}{\kern0pt}{\isacharparenright}{\kern0pt}{\isacharparenright}{\kern0pt}{\isacharparenright}{\kern0pt}{\isachardoublequoteclose}\ \isanewline
\isanewline
\isacommand{lemma}\isamarkupfalse%
\ sats{\isacharunderscore}{\kern0pt}normal{\isacharunderscore}{\kern0pt}comp{\isacharunderscore}{\kern0pt}elem{\isacharunderscore}{\kern0pt}fm\ {\isacharcolon}{\kern0pt}\ \isanewline
\ \ \isakeyword{fixes}\ i\ j\ k\ X\ Y\ H\ env\ \ \isanewline
\ \ \isakeyword{assumes}\ {\isachardoublequoteopen}i\ {\isacharless}{\kern0pt}\ length{\isacharparenleft}{\kern0pt}env{\isacharparenright}{\kern0pt}{\isachardoublequoteclose}\ {\isachardoublequoteopen}j\ {\isacharless}{\kern0pt}\ length{\isacharparenleft}{\kern0pt}env{\isacharparenright}{\kern0pt}{\isachardoublequoteclose}\ {\isachardoublequoteopen}k\ {\isacharless}{\kern0pt}\ length{\isacharparenleft}{\kern0pt}env{\isacharparenright}{\kern0pt}{\isachardoublequoteclose}\ \isanewline
\ \ \ \ \ \ \ \ \ \ {\isachardoublequoteopen}nth{\isacharparenleft}{\kern0pt}i{\isacharcomma}{\kern0pt}\ env{\isacharparenright}{\kern0pt}\ {\isacharequal}{\kern0pt}\ X{\isachardoublequoteclose}\ {\isachardoublequoteopen}nth{\isacharparenleft}{\kern0pt}j{\isacharcomma}{\kern0pt}\ env{\isacharparenright}{\kern0pt}\ {\isacharequal}{\kern0pt}\ Y{\isachardoublequoteclose}\ {\isachardoublequoteopen}nth{\isacharparenleft}{\kern0pt}k{\isacharcomma}{\kern0pt}\ env{\isacharparenright}{\kern0pt}\ {\isacharequal}{\kern0pt}\ F{\isachardoublequoteclose}\ \isanewline
\ \ \ \ \ \ \ \ \ \ {\isachardoublequoteopen}env\ {\isasymin}\ list{\isacharparenleft}{\kern0pt}M{\isacharparenright}{\kern0pt}{\isachardoublequoteclose}\ {\isachardoublequoteopen}relation{\isacharparenleft}{\kern0pt}F{\isacharparenright}{\kern0pt}{\isachardoublequoteclose}\ \isanewline
\ \ \isakeyword{shows}\ {\isachardoublequoteopen}sats{\isacharparenleft}{\kern0pt}M{\isacharcomma}{\kern0pt}\ normal{\isacharunderscore}{\kern0pt}comp{\isacharunderscore}{\kern0pt}elem{\isacharunderscore}{\kern0pt}fm{\isacharparenleft}{\kern0pt}i{\isacharcomma}{\kern0pt}\ j{\isacharcomma}{\kern0pt}\ k{\isacharparenright}{\kern0pt}{\isacharcomma}{\kern0pt}\ env{\isacharparenright}{\kern0pt}\ {\isasymlongleftrightarrow}\ Y\ {\isacharequal}{\kern0pt}\ F\ O\ X\ O\ converse{\isacharparenleft}{\kern0pt}F{\isacharparenright}{\kern0pt}{\isachardoublequoteclose}\ \isanewline
%
\isadelimproof
\ \ %
\endisadelimproof
%
\isatagproof
\isacommand{apply}\isamarkupfalse%
{\isacharparenleft}{\kern0pt}rule{\isacharunderscore}{\kern0pt}tac\ Q{\isacharequal}{\kern0pt}{\isachardoublequoteopen}{\isasymexists}FX\ {\isasymin}\ M{\isachardot}{\kern0pt}\ {\isasymexists}Finv\ {\isasymin}\ M{\isachardot}{\kern0pt}\ Finv\ {\isacharequal}{\kern0pt}\ converse{\isacharparenleft}{\kern0pt}F{\isacharparenright}{\kern0pt}\ {\isasymand}\ FX\ {\isacharequal}{\kern0pt}\ F\ O\ X\ {\isasymand}\ Y\ {\isacharequal}{\kern0pt}\ FX\ O\ Finv{\isachardoublequoteclose}\ \isakeyword{in}\ iff{\isacharunderscore}{\kern0pt}trans{\isacharparenright}{\kern0pt}\isanewline
\ \ \isacommand{unfolding}\isamarkupfalse%
\ normal{\isacharunderscore}{\kern0pt}comp{\isacharunderscore}{\kern0pt}elem{\isacharunderscore}{\kern0pt}fm{\isacharunderscore}{\kern0pt}def\ \isanewline
\ \ \isacommand{apply}\isamarkupfalse%
{\isacharparenleft}{\kern0pt}rule\ iff{\isacharunderscore}{\kern0pt}trans{\isacharcomma}{\kern0pt}\ rule\ sats{\isacharunderscore}{\kern0pt}Exists{\isacharunderscore}{\kern0pt}iff{\isacharcomma}{\kern0pt}\ simp\ add{\isacharcolon}{\kern0pt}assms{\isacharcomma}{\kern0pt}\ rule\ bex{\isacharunderscore}{\kern0pt}iff{\isacharparenright}{\kern0pt}{\isacharplus}{\kern0pt}\isanewline
\ \ \isacommand{apply}\isamarkupfalse%
{\isacharparenleft}{\kern0pt}rule\ iff{\isacharunderscore}{\kern0pt}trans{\isacharcomma}{\kern0pt}\ rule\ sats{\isacharunderscore}{\kern0pt}And{\isacharunderscore}{\kern0pt}iff{\isacharcomma}{\kern0pt}\ simp\ add{\isacharcolon}{\kern0pt}assms{\isacharcomma}{\kern0pt}\ rule\ iff{\isacharunderscore}{\kern0pt}conjI{\isadigit{2}}{\isacharparenright}{\kern0pt}\isanewline
\ \ \ \isacommand{apply}\isamarkupfalse%
{\isacharparenleft}{\kern0pt}rule\ iff{\isacharunderscore}{\kern0pt}trans{\isacharcomma}{\kern0pt}\ rule\ sats{\isacharunderscore}{\kern0pt}converse{\isacharunderscore}{\kern0pt}fm{\isacharunderscore}{\kern0pt}iff{\isacharparenright}{\kern0pt}\isanewline
\ \ \isacommand{using}\isamarkupfalse%
\ assms\ lt{\isacharunderscore}{\kern0pt}nat{\isacharunderscore}{\kern0pt}in{\isacharunderscore}{\kern0pt}nat\ length{\isacharunderscore}{\kern0pt}type\ relation{\isacharunderscore}{\kern0pt}def\ converse{\isacharunderscore}{\kern0pt}def\isanewline
\ \ \ \ \ \ \ \ \isacommand{apply}\isamarkupfalse%
\ auto{\isacharbrackleft}{\kern0pt}{\isadigit{6}}{\isacharbrackright}{\kern0pt}\isanewline
\ \ \isacommand{apply}\isamarkupfalse%
{\isacharparenleft}{\kern0pt}rule\ iff{\isacharunderscore}{\kern0pt}trans{\isacharcomma}{\kern0pt}\ rule\ sats{\isacharunderscore}{\kern0pt}And{\isacharunderscore}{\kern0pt}iff{\isacharcomma}{\kern0pt}\ simp\ add{\isacharcolon}{\kern0pt}assms{\isacharcomma}{\kern0pt}\ rule\ iff{\isacharunderscore}{\kern0pt}conjI{\isadigit{2}}{\isacharparenright}{\kern0pt}\isanewline
\ \ \isacommand{apply}\isamarkupfalse%
{\isacharparenleft}{\kern0pt}rule\ iff{\isacharunderscore}{\kern0pt}trans{\isacharcomma}{\kern0pt}\ rule\ sats{\isacharunderscore}{\kern0pt}composition{\isacharunderscore}{\kern0pt}fm{\isacharparenright}{\kern0pt}\isanewline
\ \ \isacommand{using}\isamarkupfalse%
\ assms\ lt{\isacharunderscore}{\kern0pt}nat{\isacharunderscore}{\kern0pt}in{\isacharunderscore}{\kern0pt}nat\ length{\isacharunderscore}{\kern0pt}type\isanewline
\ \ \ \ \ \ \ \isacommand{apply}\isamarkupfalse%
\ auto{\isacharbrackleft}{\kern0pt}{\isadigit{5}}{\isacharbrackright}{\kern0pt}\isanewline
\ \ \isacommand{apply}\isamarkupfalse%
{\isacharparenleft}{\kern0pt}rule\ iff{\isacharunderscore}{\kern0pt}trans{\isacharcomma}{\kern0pt}\ rule\ sats{\isacharunderscore}{\kern0pt}composition{\isacharunderscore}{\kern0pt}fm{\isacharparenright}{\kern0pt}\isanewline
\ \ \isacommand{using}\isamarkupfalse%
\ assms\ lt{\isacharunderscore}{\kern0pt}nat{\isacharunderscore}{\kern0pt}in{\isacharunderscore}{\kern0pt}nat\ length{\isacharunderscore}{\kern0pt}type\ comp{\isacharunderscore}{\kern0pt}closed\ \isanewline
\ \ \ \ \ \ \ \isacommand{apply}\isamarkupfalse%
\ auto{\isacharbrackleft}{\kern0pt}{\isadigit{5}}{\isacharbrackright}{\kern0pt}\isanewline
\ \ \isacommand{apply}\isamarkupfalse%
{\isacharparenleft}{\kern0pt}rule\ iffI{\isacharcomma}{\kern0pt}\ force{\isacharparenright}{\kern0pt}\isanewline
\ \ \isacommand{apply}\isamarkupfalse%
{\isacharparenleft}{\kern0pt}rule{\isacharunderscore}{\kern0pt}tac\ x{\isacharequal}{\kern0pt}{\isachardoublequoteopen}F\ O\ X{\isachardoublequoteclose}\ \isakeyword{in}\ bexI{\isacharparenright}{\kern0pt}\isanewline
\ \ \ \isacommand{apply}\isamarkupfalse%
{\isacharparenleft}{\kern0pt}rule{\isacharunderscore}{\kern0pt}tac\ x{\isacharequal}{\kern0pt}{\isachardoublequoteopen}converse{\isacharparenleft}{\kern0pt}F{\isacharparenright}{\kern0pt}{\isachardoublequoteclose}\ \isakeyword{in}\ bexI{\isacharparenright}{\kern0pt}\isanewline
\ \ \isacommand{using}\isamarkupfalse%
\ comp{\isacharunderscore}{\kern0pt}assoc\isanewline
\ \ \ \ \isacommand{apply}\isamarkupfalse%
\ simp\isanewline
\ \ \isacommand{using}\isamarkupfalse%
\ assms\ lt{\isacharunderscore}{\kern0pt}nat{\isacharunderscore}{\kern0pt}in{\isacharunderscore}{\kern0pt}nat\ nth{\isacharunderscore}{\kern0pt}type\ length{\isacharunderscore}{\kern0pt}type\ converse{\isacharunderscore}{\kern0pt}closed\ comp{\isacharunderscore}{\kern0pt}closed\isanewline
\ \ \ \isacommand{apply}\isamarkupfalse%
\ auto\isanewline
\ \ \isacommand{done}\isamarkupfalse%
%
\endisatagproof
{\isafoldproof}%
%
\isadelimproof
\isanewline
%
\endisadelimproof
\isanewline
\isacommand{lemma}\isamarkupfalse%
\ normal{\isacharunderscore}{\kern0pt}comp{\isacharunderscore}{\kern0pt}closed\ {\isacharcolon}{\kern0pt}\ \isanewline
\ \ \isakeyword{fixes}\ Y\ F\ \isanewline
\ \ \isakeyword{assumes}\ {\isachardoublequoteopen}Y\ {\isasymin}\ M{\isachardoublequoteclose}\ {\isachardoublequoteopen}F\ {\isasymin}\ M{\isachardoublequoteclose}\ {\isachardoublequoteopen}relation{\isacharparenleft}{\kern0pt}F{\isacharparenright}{\kern0pt}{\isachardoublequoteclose}\ \isanewline
\ \ \isakeyword{shows}\ {\isachardoublequoteopen}{\isacharbraceleft}{\kern0pt}\ F\ O\ X\ O\ converse{\isacharparenleft}{\kern0pt}F{\isacharparenright}{\kern0pt}{\isachardot}{\kern0pt}\ X\ {\isasymin}\ Y\ {\isacharbraceright}{\kern0pt}\ {\isasymin}\ M{\isachardoublequoteclose}\isanewline
%
\isadelimproof
\isanewline
\ \ %
\endisadelimproof
%
\isatagproof
\isacommand{apply}\isamarkupfalse%
{\isacharparenleft}{\kern0pt}rule\ to{\isacharunderscore}{\kern0pt}rin{\isacharcomma}{\kern0pt}\ rule\ RepFun{\isacharunderscore}{\kern0pt}closed{\isacharparenright}{\kern0pt}\isanewline
\ \ \ \ \isacommand{apply}\isamarkupfalse%
{\isacharparenleft}{\kern0pt}rule\ iffD{\isadigit{1}}{\isacharcomma}{\kern0pt}\ rule{\isacharunderscore}{\kern0pt}tac\ P{\isacharequal}{\kern0pt}{\isachardoublequoteopen}{\isasymlambda}X\ Y{\isachardot}{\kern0pt}\ sats{\isacharparenleft}{\kern0pt}M{\isacharcomma}{\kern0pt}\ normal{\isacharunderscore}{\kern0pt}comp{\isacharunderscore}{\kern0pt}elem{\isacharunderscore}{\kern0pt}fm{\isacharparenleft}{\kern0pt}{\isadigit{0}}{\isacharcomma}{\kern0pt}\ {\isadigit{1}}{\isacharcomma}{\kern0pt}\ {\isadigit{2}}{\isacharparenright}{\kern0pt}{\isacharcomma}{\kern0pt}\ {\isacharbrackleft}{\kern0pt}X{\isacharcomma}{\kern0pt}\ Y{\isacharbrackright}{\kern0pt}\ {\isacharat}{\kern0pt}\ {\isacharbrackleft}{\kern0pt}F{\isacharbrackright}{\kern0pt}{\isacharparenright}{\kern0pt}{\isachardoublequoteclose}\ \isakeyword{in}\ strong{\isacharunderscore}{\kern0pt}replacement{\isacharunderscore}{\kern0pt}cong{\isacharparenright}{\kern0pt}\isanewline
\ \ \ \ \ \isacommand{apply}\isamarkupfalse%
{\isacharparenleft}{\kern0pt}rule\ sats{\isacharunderscore}{\kern0pt}normal{\isacharunderscore}{\kern0pt}comp{\isacharunderscore}{\kern0pt}elem{\isacharunderscore}{\kern0pt}fm{\isacharparenright}{\kern0pt}\isanewline
\ \ \isacommand{using}\isamarkupfalse%
\ assms\isanewline
\ \ \ \ \ \ \ \ \ \ \ \ \isacommand{apply}\isamarkupfalse%
\ auto{\isacharbrackleft}{\kern0pt}{\isadigit{8}}{\isacharbrackright}{\kern0pt}\isanewline
\ \ \ \ \isacommand{apply}\isamarkupfalse%
{\isacharparenleft}{\kern0pt}rule\ replacement{\isacharunderscore}{\kern0pt}ax{\isacharparenright}{\kern0pt}\isanewline
\ \ \isacommand{unfolding}\isamarkupfalse%
\ normal{\isacharunderscore}{\kern0pt}comp{\isacharunderscore}{\kern0pt}elem{\isacharunderscore}{\kern0pt}fm{\isacharunderscore}{\kern0pt}def\ converse{\isacharunderscore}{\kern0pt}fm{\isacharunderscore}{\kern0pt}def\ composition{\isacharunderscore}{\kern0pt}fm{\isacharunderscore}{\kern0pt}def\isanewline
\ \ \isacommand{using}\isamarkupfalse%
\ assms\isanewline
\ \ \ \ \ \ \isacommand{apply}\isamarkupfalse%
\ auto{\isacharbrackleft}{\kern0pt}{\isadigit{2}}{\isacharbrackright}{\kern0pt}\isanewline
\ \ \isacommand{apply}\isamarkupfalse%
\ {\isacharparenleft}{\kern0pt}simp\ del{\isacharcolon}{\kern0pt}FOL{\isacharunderscore}{\kern0pt}sats{\isacharunderscore}{\kern0pt}iff\ pair{\isacharunderscore}{\kern0pt}abs\ add{\isacharcolon}{\kern0pt}\ fm{\isacharunderscore}{\kern0pt}defs\ nat{\isacharunderscore}{\kern0pt}simp{\isacharunderscore}{\kern0pt}union{\isacharparenright}{\kern0pt}\isanewline
\ \ \isacommand{using}\isamarkupfalse%
\ assms\ transM\ comp{\isacharunderscore}{\kern0pt}closed\ converse{\isacharunderscore}{\kern0pt}closed\isanewline
\ \ \isacommand{by}\isamarkupfalse%
\ auto%
\endisatagproof
{\isafoldproof}%
%
\isadelimproof
\isanewline
%
\endisadelimproof
\isanewline
\isacommand{lemma}\isamarkupfalse%
\ Fn{\isacharunderscore}{\kern0pt}perms{\isacharunderscore}{\kern0pt}filter{\isacharunderscore}{\kern0pt}normal\ {\isacharcolon}{\kern0pt}\ \isanewline
\ \ \isakeyword{fixes}\ H\ F\isanewline
\ \ \isakeyword{assumes}\ {\isachardoublequoteopen}H\ {\isasymin}\ Fn{\isacharunderscore}{\kern0pt}perms{\isacharunderscore}{\kern0pt}filter{\isachardoublequoteclose}\ {\isachardoublequoteopen}F\ {\isasymin}\ Fn{\isacharunderscore}{\kern0pt}perms{\isachardoublequoteclose}\ \isanewline
\ \ \isakeyword{shows}\ {\isachardoublequoteopen}{\isacharbraceleft}{\kern0pt}\ F\ O\ G\ O\ converse{\isacharparenleft}{\kern0pt}F{\isacharparenright}{\kern0pt}\ {\isachardot}{\kern0pt}\ G\ {\isasymin}\ H\ {\isacharbraceright}{\kern0pt}\ {\isasymin}\ Fn{\isacharunderscore}{\kern0pt}perms{\isacharunderscore}{\kern0pt}filter{\isachardoublequoteclose}\ \isanewline
%
\isadelimproof
%
\endisadelimproof
%
\isatagproof
\isacommand{proof}\isamarkupfalse%
\ {\isacharminus}{\kern0pt}\ \isanewline
\ \ \isacommand{obtain}\isamarkupfalse%
\ f\ \isakeyword{where}\ fH{\isacharcolon}{\kern0pt}\ {\isachardoublequoteopen}F\ {\isacharequal}{\kern0pt}\ Fn{\isacharunderscore}{\kern0pt}perm{\isacharprime}{\kern0pt}{\isacharparenleft}{\kern0pt}f{\isacharparenright}{\kern0pt}{\isachardoublequoteclose}\ {\isachardoublequoteopen}f\ {\isasymin}\ nat{\isacharunderscore}{\kern0pt}perms{\isachardoublequoteclose}\isanewline
\ \ \ \ \isacommand{using}\isamarkupfalse%
\ assms\ Fn{\isacharunderscore}{\kern0pt}perms{\isacharunderscore}{\kern0pt}def\ \isanewline
\ \ \ \ \isacommand{by}\isamarkupfalse%
\ force\isanewline
\isanewline
\ \ \isacommand{define}\isamarkupfalse%
\ X\ \isakeyword{where}\ {\isachardoublequoteopen}X\ {\isasymequiv}\ {\isacharbraceleft}{\kern0pt}\ F\ O\ G\ O\ converse{\isacharparenleft}{\kern0pt}F{\isacharparenright}{\kern0pt}\ {\isachardot}{\kern0pt}\ G\ {\isasymin}\ H\ {\isacharbraceright}{\kern0pt}{\isachardoublequoteclose}\ \isanewline
\isanewline
\ \ \isacommand{have}\isamarkupfalse%
\ XinM\ {\isacharcolon}{\kern0pt}\ {\isachardoublequoteopen}X\ {\isasymin}\ M{\isachardoublequoteclose}\ \isanewline
\ \ \ \ \isacommand{unfolding}\isamarkupfalse%
\ X{\isacharunderscore}{\kern0pt}def\isanewline
\ \ \ \ \isacommand{apply}\isamarkupfalse%
{\isacharparenleft}{\kern0pt}rule\ normal{\isacharunderscore}{\kern0pt}comp{\isacharunderscore}{\kern0pt}closed{\isacharparenright}{\kern0pt}\isanewline
\ \ \ \ \isacommand{using}\isamarkupfalse%
\ transM\ Fn{\isacharunderscore}{\kern0pt}perms{\isacharunderscore}{\kern0pt}in{\isacharunderscore}{\kern0pt}M\ Fn{\isacharunderscore}{\kern0pt}perms{\isacharunderscore}{\kern0pt}filter{\isacharunderscore}{\kern0pt}in{\isacharunderscore}{\kern0pt}M\ assms\ relation{\isacharunderscore}{\kern0pt}def\ Fn{\isacharunderscore}{\kern0pt}perms{\isacharunderscore}{\kern0pt}def\ Fn{\isacharunderscore}{\kern0pt}perm{\isacharprime}{\kern0pt}{\isacharunderscore}{\kern0pt}def\isanewline
\ \ \ \ \isacommand{by}\isamarkupfalse%
\ auto\isanewline
\isanewline
\ \ \isacommand{have}\isamarkupfalse%
\ Xsubset{\isacharcolon}{\kern0pt}\ {\isachardoublequoteopen}X\ {\isasymsubseteq}\ Fn{\isacharunderscore}{\kern0pt}perms{\isachardoublequoteclose}\ \isanewline
\ \ \isacommand{proof}\isamarkupfalse%
{\isacharparenleft}{\kern0pt}rule\ subsetI{\isacharparenright}{\kern0pt}\isanewline
\ \ \ \ \isacommand{fix}\isamarkupfalse%
\ x\ \isanewline
\ \ \ \ \isacommand{assume}\isamarkupfalse%
\ {\isachardoublequoteopen}x\ {\isasymin}\ X{\isachardoublequoteclose}\ \isanewline
\ \ \ \ \isacommand{then}\isamarkupfalse%
\ \isacommand{obtain}\isamarkupfalse%
\ G\ \isakeyword{where}\ xH{\isacharcolon}{\kern0pt}\ {\isachardoublequoteopen}x\ {\isacharequal}{\kern0pt}\ F\ O\ G\ O\ converse{\isacharparenleft}{\kern0pt}F{\isacharparenright}{\kern0pt}{\isachardoublequoteclose}\ {\isachardoublequoteopen}G\ {\isasymin}\ H{\isachardoublequoteclose}\ \isacommand{using}\isamarkupfalse%
\ X{\isacharunderscore}{\kern0pt}def\ \isacommand{by}\isamarkupfalse%
\ force\ \isanewline
\ \ \ \ \isacommand{then}\isamarkupfalse%
\ \isacommand{have}\isamarkupfalse%
\ {\isachardoublequoteopen}G\ {\isasymin}\ Fn{\isacharunderscore}{\kern0pt}perms{\isachardoublequoteclose}\ \isacommand{using}\isamarkupfalse%
\ assms\ Fn{\isacharunderscore}{\kern0pt}perms{\isacharunderscore}{\kern0pt}filter{\isacharunderscore}{\kern0pt}subset\ \isacommand{by}\isamarkupfalse%
\ force\ \isanewline
\ \ \ \ \isacommand{then}\isamarkupfalse%
\ \isacommand{obtain}\isamarkupfalse%
\ g\ \isakeyword{where}\ gH\ {\isacharcolon}{\kern0pt}\ {\isachardoublequoteopen}G\ {\isacharequal}{\kern0pt}\ Fn{\isacharunderscore}{\kern0pt}perm{\isacharprime}{\kern0pt}{\isacharparenleft}{\kern0pt}g{\isacharparenright}{\kern0pt}{\isachardoublequoteclose}\ {\isachardoublequoteopen}g\ {\isasymin}\ nat{\isacharunderscore}{\kern0pt}perms{\isachardoublequoteclose}\ \isacommand{using}\isamarkupfalse%
\ Fn{\isacharunderscore}{\kern0pt}perms{\isacharunderscore}{\kern0pt}def\ \isacommand{by}\isamarkupfalse%
\ force\ \isanewline
\ \ \ \ \isacommand{then}\isamarkupfalse%
\ \isacommand{have}\isamarkupfalse%
\ {\isachardoublequoteopen}x\ {\isacharequal}{\kern0pt}\ F\ O\ G\ O\ converse{\isacharparenleft}{\kern0pt}F{\isacharparenright}{\kern0pt}{\isachardoublequoteclose}\ \isacommand{using}\isamarkupfalse%
\ xH\ \isacommand{by}\isamarkupfalse%
\ auto\ \isanewline
\ \ \ \ \isacommand{also}\isamarkupfalse%
\ \isacommand{have}\isamarkupfalse%
\ {\isachardoublequoteopen}{\isachardot}{\kern0pt}{\isachardot}{\kern0pt}{\isachardot}{\kern0pt}\ {\isacharequal}{\kern0pt}\ {\isacharparenleft}{\kern0pt}Fn{\isacharunderscore}{\kern0pt}perm{\isacharprime}{\kern0pt}{\isacharparenleft}{\kern0pt}f{\isacharparenright}{\kern0pt}\ O\ Fn{\isacharunderscore}{\kern0pt}perm{\isacharprime}{\kern0pt}{\isacharparenleft}{\kern0pt}g{\isacharparenright}{\kern0pt}{\isacharparenright}{\kern0pt}\ O\ Fn{\isacharunderscore}{\kern0pt}perm{\isacharprime}{\kern0pt}{\isacharparenleft}{\kern0pt}converse{\isacharparenleft}{\kern0pt}f{\isacharparenright}{\kern0pt}{\isacharparenright}{\kern0pt}{\isachardoublequoteclose}\ \isanewline
\ \ \ \ \ \ \isacommand{using}\isamarkupfalse%
\ fH\ Fn{\isacharunderscore}{\kern0pt}perm{\isacharprime}{\kern0pt}{\isacharunderscore}{\kern0pt}converse\ gH\ \isacommand{by}\isamarkupfalse%
\ auto\isanewline
\ \ \ \ \isacommand{also}\isamarkupfalse%
\ \isacommand{have}\isamarkupfalse%
\ {\isachardoublequoteopen}{\isachardot}{\kern0pt}{\isachardot}{\kern0pt}{\isachardot}{\kern0pt}\ {\isacharequal}{\kern0pt}\ Fn{\isacharunderscore}{\kern0pt}perm{\isacharprime}{\kern0pt}{\isacharparenleft}{\kern0pt}f\ O\ g{\isacharparenright}{\kern0pt}\ O\ Fn{\isacharunderscore}{\kern0pt}perm{\isacharprime}{\kern0pt}{\isacharparenleft}{\kern0pt}converse{\isacharparenleft}{\kern0pt}f{\isacharparenright}{\kern0pt}{\isacharparenright}{\kern0pt}{\isachardoublequoteclose}\ \isanewline
\ \ \ \ \ \ \isacommand{apply}\isamarkupfalse%
{\isacharparenleft}{\kern0pt}subst\ Fn{\isacharunderscore}{\kern0pt}perm{\isacharprime}{\kern0pt}{\isacharunderscore}{\kern0pt}comp{\isacharparenright}{\kern0pt}\isanewline
\ \ \ \ \ \ \isacommand{using}\isamarkupfalse%
\ gH\ converse{\isacharunderscore}{\kern0pt}in{\isacharunderscore}{\kern0pt}nat{\isacharunderscore}{\kern0pt}perms\ fH\isanewline
\ \ \ \ \ \ \isacommand{by}\isamarkupfalse%
\ auto\isanewline
\ \ \ \ \isacommand{also}\isamarkupfalse%
\ \isacommand{have}\isamarkupfalse%
\ {\isachardoublequoteopen}{\isachardot}{\kern0pt}{\isachardot}{\kern0pt}{\isachardot}{\kern0pt}\ {\isacharequal}{\kern0pt}\ Fn{\isacharunderscore}{\kern0pt}perm{\isacharprime}{\kern0pt}{\isacharparenleft}{\kern0pt}f\ O\ g\ O\ converse{\isacharparenleft}{\kern0pt}f{\isacharparenright}{\kern0pt}{\isacharparenright}{\kern0pt}{\isachardoublequoteclose}\ \isanewline
\ \ \ \ \ \ \isacommand{apply}\isamarkupfalse%
{\isacharparenleft}{\kern0pt}subst\ Fn{\isacharunderscore}{\kern0pt}perm{\isacharprime}{\kern0pt}{\isacharunderscore}{\kern0pt}comp{\isacharparenright}{\kern0pt}\isanewline
\ \ \ \ \ \ \ \ \isacommand{apply}\isamarkupfalse%
{\isacharparenleft}{\kern0pt}simp\ add{\isacharcolon}{\kern0pt}nat{\isacharunderscore}{\kern0pt}perms{\isacharunderscore}{\kern0pt}def{\isacharparenright}{\kern0pt}\isanewline
\ \ \ \ \ \ \ \ \isacommand{apply}\isamarkupfalse%
{\isacharparenleft}{\kern0pt}rule\ conjI{\isacharcomma}{\kern0pt}\ rule\ comp{\isacharunderscore}{\kern0pt}bij{\isacharparenright}{\kern0pt}\isanewline
\ \ \ \ \ \ \isacommand{using}\isamarkupfalse%
\ gH\ fH\ nat{\isacharunderscore}{\kern0pt}perms{\isacharunderscore}{\kern0pt}def\ comp{\isacharunderscore}{\kern0pt}closed\ converse{\isacharunderscore}{\kern0pt}in{\isacharunderscore}{\kern0pt}nat{\isacharunderscore}{\kern0pt}perms\ comp{\isacharunderscore}{\kern0pt}assoc\isanewline
\ \ \ \ \ \ \isacommand{by}\isamarkupfalse%
\ auto\isanewline
\ \ \ \ \isacommand{finally}\isamarkupfalse%
\ \isacommand{show}\isamarkupfalse%
\ {\isachardoublequoteopen}x\ {\isasymin}\ Fn{\isacharunderscore}{\kern0pt}perms{\isachardoublequoteclose}\ \isanewline
\ \ \ \ \ \ \isacommand{unfolding}\isamarkupfalse%
\ Fn{\isacharunderscore}{\kern0pt}perms{\isacharunderscore}{\kern0pt}def\isanewline
\ \ \ \ \ \ \isacommand{apply}\isamarkupfalse%
\ simp\isanewline
\ \ \ \ \ \ \isacommand{apply}\isamarkupfalse%
{\isacharparenleft}{\kern0pt}rule{\isacharunderscore}{\kern0pt}tac\ x{\isacharequal}{\kern0pt}{\isachardoublequoteopen}f\ O\ g\ O\ converse{\isacharparenleft}{\kern0pt}f{\isacharparenright}{\kern0pt}{\isachardoublequoteclose}\ \isakeyword{in}\ bexI{\isacharparenright}{\kern0pt}\isanewline
\ \ \ \ \ \ \ \isacommand{apply}\isamarkupfalse%
\ simp\isanewline
\ \ \ \ \ \ \isacommand{unfolding}\isamarkupfalse%
\ nat{\isacharunderscore}{\kern0pt}perms{\isacharunderscore}{\kern0pt}def\ \isanewline
\ \ \ \ \ \ \isacommand{apply}\isamarkupfalse%
{\isacharparenleft}{\kern0pt}simp{\isacharcomma}{\kern0pt}\ rule\ conjI{\isacharparenright}{\kern0pt}\isanewline
\ \ \ \ \ \ \ \isacommand{apply}\isamarkupfalse%
{\isacharparenleft}{\kern0pt}rule\ comp{\isacharunderscore}{\kern0pt}bij{\isacharparenright}{\kern0pt}{\isacharplus}{\kern0pt}\isanewline
\ \ \ \ \ \ \isacommand{using}\isamarkupfalse%
\ converse{\isacharunderscore}{\kern0pt}in{\isacharunderscore}{\kern0pt}nat{\isacharunderscore}{\kern0pt}perms\ fH\ gH\ nat{\isacharunderscore}{\kern0pt}perms{\isacharunderscore}{\kern0pt}def\ comp{\isacharunderscore}{\kern0pt}closed\ converse{\isacharunderscore}{\kern0pt}closed\isanewline
\ \ \ \ \ \ \ \ \ \isacommand{apply}\isamarkupfalse%
\ auto{\isacharbrackleft}{\kern0pt}{\isadigit{4}}{\isacharbrackright}{\kern0pt}\isanewline
\ \ \ \ \ \ \isacommand{done}\isamarkupfalse%
\isanewline
\ \ \isacommand{qed}\isamarkupfalse%
\isanewline
\isanewline
\ \ \isacommand{have}\isamarkupfalse%
\ Xsubset{\isacharprime}{\kern0pt}{\isacharcolon}{\kern0pt}\ {\isachardoublequoteopen}X\ {\isasymsubseteq}\ bij{\isacharparenleft}{\kern0pt}Fn{\isacharcomma}{\kern0pt}\ Fn{\isacharparenright}{\kern0pt}{\isachardoublequoteclose}\ \isanewline
\ \ \isacommand{proof}\isamarkupfalse%
{\isacharparenleft}{\kern0pt}rule\ subsetI{\isacharparenright}{\kern0pt}\isanewline
\ \ \ \ \isacommand{fix}\isamarkupfalse%
\ x\ \isanewline
\ \ \ \ \isacommand{assume}\isamarkupfalse%
\ {\isachardoublequoteopen}x\ {\isasymin}\ X{\isachardoublequoteclose}\ \isanewline
\ \ \ \ \isacommand{then}\isamarkupfalse%
\ \isacommand{have}\isamarkupfalse%
\ {\isachardoublequoteopen}x\ {\isasymin}\ Fn{\isacharunderscore}{\kern0pt}perms{\isachardoublequoteclose}\ \isacommand{using}\isamarkupfalse%
\ Xsubset\ \isacommand{by}\isamarkupfalse%
\ auto\isanewline
\ \ \ \ \isacommand{then}\isamarkupfalse%
\ \isacommand{obtain}\isamarkupfalse%
\ f\ \isakeyword{where}\ fH\ {\isacharcolon}{\kern0pt}\ {\isachardoublequoteopen}f\ {\isasymin}\ nat{\isacharunderscore}{\kern0pt}perms{\isachardoublequoteclose}\ {\isachardoublequoteopen}x\ {\isacharequal}{\kern0pt}\ Fn{\isacharunderscore}{\kern0pt}perm{\isacharprime}{\kern0pt}{\isacharparenleft}{\kern0pt}f{\isacharparenright}{\kern0pt}{\isachardoublequoteclose}\ \isacommand{using}\isamarkupfalse%
\ Fn{\isacharunderscore}{\kern0pt}perms{\isacharunderscore}{\kern0pt}def\ \isacommand{by}\isamarkupfalse%
\ force\ \isanewline
\ \ \ \ \isacommand{then}\isamarkupfalse%
\ \isacommand{have}\isamarkupfalse%
\ {\isachardoublequoteopen}Fn{\isacharunderscore}{\kern0pt}perm{\isacharprime}{\kern0pt}{\isacharparenleft}{\kern0pt}f{\isacharparenright}{\kern0pt}\ {\isasymin}\ bij{\isacharparenleft}{\kern0pt}Fn{\isacharcomma}{\kern0pt}\ Fn{\isacharparenright}{\kern0pt}{\isachardoublequoteclose}\ \isacommand{by}\isamarkupfalse%
{\isacharparenleft}{\kern0pt}rule{\isacharunderscore}{\kern0pt}tac\ Fn{\isacharunderscore}{\kern0pt}perm{\isacharprime}{\kern0pt}{\isacharunderscore}{\kern0pt}bij{\isacharparenright}{\kern0pt}\isanewline
\ \ \ \ \isacommand{then}\isamarkupfalse%
\ \isacommand{show}\isamarkupfalse%
\ {\isachardoublequoteopen}x\ {\isasymin}\ bij{\isacharparenleft}{\kern0pt}Fn{\isacharcomma}{\kern0pt}\ Fn{\isacharparenright}{\kern0pt}{\isachardoublequoteclose}\ \isacommand{using}\isamarkupfalse%
\ fH\ \isacommand{by}\isamarkupfalse%
\ auto\isanewline
\ \ \isacommand{qed}\isamarkupfalse%
\isanewline
\isanewline
\ \ \isacommand{have}\isamarkupfalse%
\ Xsubset{\isacharprime}{\kern0pt}{\isacharprime}{\kern0pt}{\isacharcolon}{\kern0pt}\ {\isachardoublequoteopen}{\isasymAnd}x{\isachardot}{\kern0pt}\ x\ {\isasymin}\ X\ {\isasymLongrightarrow}\ forcing{\isacharunderscore}{\kern0pt}data{\isacharunderscore}{\kern0pt}partial{\isachardot}{\kern0pt}is{\isacharunderscore}{\kern0pt}P{\isacharunderscore}{\kern0pt}auto{\isacharparenleft}{\kern0pt}Fn{\isacharcomma}{\kern0pt}\ Fn{\isacharunderscore}{\kern0pt}leq{\isacharcomma}{\kern0pt}\ M{\isacharcomma}{\kern0pt}\ x{\isacharparenright}{\kern0pt}{\isachardoublequoteclose}\ \isanewline
\ \ \ \ \isacommand{apply}\isamarkupfalse%
{\isacharparenleft}{\kern0pt}subst\ forcing{\isacharunderscore}{\kern0pt}data{\isacharunderscore}{\kern0pt}partial{\isachardot}{\kern0pt}is{\isacharunderscore}{\kern0pt}P{\isacharunderscore}{\kern0pt}auto{\isacharunderscore}{\kern0pt}def{\isacharparenright}{\kern0pt}\isanewline
\ \ \ \ \ \isacommand{apply}\isamarkupfalse%
{\isacharparenleft}{\kern0pt}rule\ Fn{\isacharunderscore}{\kern0pt}forcing{\isacharunderscore}{\kern0pt}data{\isacharunderscore}{\kern0pt}partial{\isacharparenright}{\kern0pt}\isanewline
\ \ \ \ \isacommand{apply}\isamarkupfalse%
{\isacharparenleft}{\kern0pt}rule\ conjI{\isacharparenright}{\kern0pt}\isanewline
\ \ \ \ \isacommand{using}\isamarkupfalse%
\ transM\ XinM\ \isanewline
\ \ \ \ \ \isacommand{apply}\isamarkupfalse%
\ force\ \isanewline
\ \ \ \ \isacommand{apply}\isamarkupfalse%
{\isacharparenleft}{\kern0pt}rule\ conjI{\isacharparenright}{\kern0pt}\isanewline
\ \ \ \ \isacommand{using}\isamarkupfalse%
\ Xsubset{\isacharprime}{\kern0pt}\isanewline
\ \ \ \ \ \isacommand{apply}\isamarkupfalse%
\ force\isanewline
\ \ \ \ \isacommand{apply}\isamarkupfalse%
{\isacharparenleft}{\kern0pt}rename{\isacharunderscore}{\kern0pt}tac\ x{\isacharcomma}{\kern0pt}\ subgoal{\isacharunderscore}{\kern0pt}tac\ {\isachardoublequoteopen}{\isasymexists}f\ {\isasymin}\ nat{\isacharunderscore}{\kern0pt}perms{\isachardot}{\kern0pt}\ x\ {\isacharequal}{\kern0pt}\ Fn{\isacharunderscore}{\kern0pt}perm{\isacharprime}{\kern0pt}{\isacharparenleft}{\kern0pt}f{\isacharparenright}{\kern0pt}{\isachardoublequoteclose}{\isacharcomma}{\kern0pt}\ clarsimp{\isacharparenright}{\kern0pt}\isanewline
\ \ \ \ \isacommand{using}\isamarkupfalse%
\ Fn{\isacharunderscore}{\kern0pt}perm{\isacharprime}{\kern0pt}{\isacharunderscore}{\kern0pt}preserves{\isacharunderscore}{\kern0pt}order\ Fn{\isacharunderscore}{\kern0pt}perm{\isacharprime}{\kern0pt}{\isacharunderscore}{\kern0pt}preserves{\isacharunderscore}{\kern0pt}order{\isacharprime}{\kern0pt}\isanewline
\ \ \ \ \ \isacommand{apply}\isamarkupfalse%
\ force\isanewline
\ \ \ \ \isacommand{using}\isamarkupfalse%
\ Xsubset\ Fn{\isacharunderscore}{\kern0pt}perms{\isacharunderscore}{\kern0pt}def\isanewline
\ \ \ \ \isacommand{by}\isamarkupfalse%
\ auto\isanewline
\isanewline
\ \ \isacommand{have}\isamarkupfalse%
\ compin{\isacharcolon}{\kern0pt}\ {\isachardoublequoteopen}{\isasymAnd}A\ B{\isachardot}{\kern0pt}\ A\ {\isasymin}\ X\ {\isasymLongrightarrow}\ B\ {\isasymin}\ X\ {\isasymLongrightarrow}\ A\ O\ B\ {\isasymin}\ X{\isachardoublequoteclose}\isanewline
\ \ \isacommand{proof}\isamarkupfalse%
\ {\isacharminus}{\kern0pt}\ \isanewline
\ \ \ \ \isacommand{fix}\isamarkupfalse%
\ A\ B\ \isanewline
\ \ \ \ \isacommand{assume}\isamarkupfalse%
\ assms{\isadigit{1}}{\isacharcolon}{\kern0pt}\ {\isachardoublequoteopen}A\ {\isasymin}\ X{\isachardoublequoteclose}\ {\isachardoublequoteopen}B\ {\isasymin}\ X{\isachardoublequoteclose}\ \isanewline
\ \ \ \ \isacommand{obtain}\isamarkupfalse%
\ S\ \isakeyword{where}\ SH{\isacharcolon}{\kern0pt}\ {\isachardoublequoteopen}A\ {\isacharequal}{\kern0pt}\ F\ O\ S\ O\ converse{\isacharparenleft}{\kern0pt}F{\isacharparenright}{\kern0pt}{\isachardoublequoteclose}\ {\isachardoublequoteopen}S\ {\isasymin}\ H{\isachardoublequoteclose}\ \isanewline
\ \ \ \ \ \ \isacommand{using}\isamarkupfalse%
\ assms{\isadigit{1}}\ X{\isacharunderscore}{\kern0pt}def\ \isacommand{by}\isamarkupfalse%
\ auto\isanewline
\ \ \ \ \isacommand{obtain}\isamarkupfalse%
\ T\ \isakeyword{where}\ TH{\isacharcolon}{\kern0pt}\ {\isachardoublequoteopen}B\ {\isacharequal}{\kern0pt}\ F\ O\ T\ O\ converse{\isacharparenleft}{\kern0pt}F{\isacharparenright}{\kern0pt}{\isachardoublequoteclose}\ {\isachardoublequoteopen}T\ {\isasymin}\ H{\isachardoublequoteclose}\ \isanewline
\ \ \ \ \ \ \isacommand{using}\isamarkupfalse%
\ assms{\isadigit{1}}\ X{\isacharunderscore}{\kern0pt}def\ \isacommand{by}\isamarkupfalse%
\ auto\isanewline
\isanewline
\ \ \ \ \isacommand{have}\isamarkupfalse%
{\isachardoublequoteopen}S\ {\isasymin}\ Fn{\isacharunderscore}{\kern0pt}perms{\isachardoublequoteclose}\ \isacommand{using}\isamarkupfalse%
\ SH\ assms\ Fn{\isacharunderscore}{\kern0pt}perms{\isacharunderscore}{\kern0pt}filter{\isacharunderscore}{\kern0pt}subset\ \isacommand{by}\isamarkupfalse%
\ auto\isanewline
\ \ \ \ \isacommand{then}\isamarkupfalse%
\ \isacommand{have}\isamarkupfalse%
\ {\isachardoublequoteopen}S\ {\isasymin}\ bij{\isacharparenleft}{\kern0pt}Fn{\isacharcomma}{\kern0pt}\ Fn{\isacharparenright}{\kern0pt}{\isachardoublequoteclose}\ \isacommand{using}\isamarkupfalse%
\ Fn{\isacharunderscore}{\kern0pt}perms{\isacharunderscore}{\kern0pt}def\ Fn{\isacharunderscore}{\kern0pt}perm{\isacharprime}{\kern0pt}{\isacharunderscore}{\kern0pt}bij\ \isacommand{by}\isamarkupfalse%
\ auto\isanewline
\ \ \ \ \isacommand{then}\isamarkupfalse%
\ \isacommand{have}\isamarkupfalse%
\ Ssubset\ {\isacharcolon}{\kern0pt}\ {\isachardoublequoteopen}S\ {\isasymsubseteq}\ Fn\ {\isasymtimes}\ Fn{\isachardoublequoteclose}\ \isacommand{using}\isamarkupfalse%
\ bij{\isacharunderscore}{\kern0pt}def\ inj{\isacharunderscore}{\kern0pt}def\ Pi{\isacharunderscore}{\kern0pt}def\ \isacommand{by}\isamarkupfalse%
\ force\isanewline
\isanewline
\ \ \ \ \isacommand{have}\isamarkupfalse%
\ {\isachardoublequoteopen}A\ O\ B\ {\isacharequal}{\kern0pt}\ F\ O\ {\isacharparenleft}{\kern0pt}{\isacharparenleft}{\kern0pt}S\ O\ {\isacharparenleft}{\kern0pt}converse{\isacharparenleft}{\kern0pt}F{\isacharparenright}{\kern0pt}\ O\ F{\isacharparenright}{\kern0pt}{\isacharparenright}{\kern0pt}\ O\ T{\isacharparenright}{\kern0pt}\ O\ converse{\isacharparenleft}{\kern0pt}F{\isacharparenright}{\kern0pt}{\isachardoublequoteclose}\isanewline
\ \ \ \ \ \ \isacommand{using}\isamarkupfalse%
\ SH\ TH\ comp{\isacharunderscore}{\kern0pt}assoc\ \isacommand{by}\isamarkupfalse%
\ auto\isanewline
\ \ \ \ \isacommand{also}\isamarkupfalse%
\ \isacommand{have}\isamarkupfalse%
\ {\isachardoublequoteopen}{\isachardot}{\kern0pt}{\isachardot}{\kern0pt}{\isachardot}{\kern0pt}\ {\isacharequal}{\kern0pt}\ F\ O\ {\isacharparenleft}{\kern0pt}{\isacharparenleft}{\kern0pt}S\ O\ {\isacharparenleft}{\kern0pt}id{\isacharparenleft}{\kern0pt}Fn{\isacharparenright}{\kern0pt}{\isacharparenright}{\kern0pt}{\isacharparenright}{\kern0pt}\ O\ T{\isacharparenright}{\kern0pt}\ O\ converse{\isacharparenleft}{\kern0pt}F{\isacharparenright}{\kern0pt}{\isachardoublequoteclose}\isanewline
\ \ \ \ \ \ \isacommand{apply}\isamarkupfalse%
{\isacharparenleft}{\kern0pt}subst\ left{\isacharunderscore}{\kern0pt}comp{\isacharunderscore}{\kern0pt}inverse{\isacharcomma}{\kern0pt}\ rule\ bij{\isacharunderscore}{\kern0pt}is{\isacharunderscore}{\kern0pt}inj{\isacharparenright}{\kern0pt}\isanewline
\ \ \ \ \ \ \isacommand{using}\isamarkupfalse%
\ assms\ Fn{\isacharunderscore}{\kern0pt}perms{\isacharunderscore}{\kern0pt}def\ Fn{\isacharunderscore}{\kern0pt}perm{\isacharprime}{\kern0pt}{\isacharunderscore}{\kern0pt}bij\ \isanewline
\ \ \ \ \ \ \isacommand{by}\isamarkupfalse%
\ auto\isanewline
\ \ \ \ \isacommand{also}\isamarkupfalse%
\ \isacommand{have}\isamarkupfalse%
\ {\isachardoublequoteopen}{\isachardot}{\kern0pt}{\isachardot}{\kern0pt}{\isachardot}{\kern0pt}\ {\isacharequal}{\kern0pt}\ F\ O\ {\isacharparenleft}{\kern0pt}S\ O\ T{\isacharparenright}{\kern0pt}\ O\ converse{\isacharparenleft}{\kern0pt}F{\isacharparenright}{\kern0pt}{\isachardoublequoteclose}\isanewline
\ \ \ \ \ \ \isacommand{apply}\isamarkupfalse%
{\isacharparenleft}{\kern0pt}subst\ right{\isacharunderscore}{\kern0pt}comp{\isacharunderscore}{\kern0pt}id\ {\isacharbrackleft}{\kern0pt}\isakeyword{where}\ B{\isacharequal}{\kern0pt}Fn{\isacharbrackright}{\kern0pt}{\isacharparenright}{\kern0pt}\isanewline
\ \ \ \ \ \ \ \isacommand{apply}\isamarkupfalse%
{\isacharparenleft}{\kern0pt}rule\ Ssubset{\isacharcomma}{\kern0pt}\ simp{\isacharparenright}{\kern0pt}\isanewline
\ \ \ \ \ \ \isacommand{done}\isamarkupfalse%
\isanewline
\ \ \ \ \isacommand{finally}\isamarkupfalse%
\ \isacommand{show}\isamarkupfalse%
\ {\isachardoublequoteopen}A\ O\ B\ {\isasymin}\ X{\isachardoublequoteclose}\ \isanewline
\ \ \ \ \ \ \isacommand{apply}\isamarkupfalse%
\ {\isacharparenleft}{\kern0pt}simp\ add{\isacharcolon}{\kern0pt}X{\isacharunderscore}{\kern0pt}def{\isacharparenright}{\kern0pt}\isanewline
\ \ \ \ \ \ \isacommand{apply}\isamarkupfalse%
{\isacharparenleft}{\kern0pt}rule{\isacharunderscore}{\kern0pt}tac\ x{\isacharequal}{\kern0pt}{\isachardoublequoteopen}S\ O\ T{\isachardoublequoteclose}\ \isakeyword{in}\ bexI{\isacharcomma}{\kern0pt}\ simp{\isacharparenright}{\kern0pt}\isanewline
\ \ \ \ \ \ \isacommand{apply}\isamarkupfalse%
{\isacharparenleft}{\kern0pt}rule{\isacharunderscore}{\kern0pt}tac\ P{\isacharequal}{\kern0pt}{\isachardoublequoteopen}H\ {\isasymin}\ forcing{\isacharunderscore}{\kern0pt}data{\isacharunderscore}{\kern0pt}partial{\isachardot}{\kern0pt}P{\isacharunderscore}{\kern0pt}auto{\isacharunderscore}{\kern0pt}subgroups{\isacharparenleft}{\kern0pt}Fn{\isacharcomma}{\kern0pt}\ Fn{\isacharunderscore}{\kern0pt}leq{\isacharcomma}{\kern0pt}\ M{\isacharcomma}{\kern0pt}\ Fn{\isacharunderscore}{\kern0pt}perms{\isacharparenright}{\kern0pt}{\isachardoublequoteclose}\ \isakeyword{in}\ mp{\isacharparenright}{\kern0pt}\isanewline
\ \ \ \ \ \ \ \isacommand{apply}\isamarkupfalse%
{\isacharparenleft}{\kern0pt}subst\ forcing{\isacharunderscore}{\kern0pt}data{\isacharunderscore}{\kern0pt}partial{\isachardot}{\kern0pt}P{\isacharunderscore}{\kern0pt}auto{\isacharunderscore}{\kern0pt}subgroups{\isacharunderscore}{\kern0pt}def{\isacharcomma}{\kern0pt}\ rule\ Fn{\isacharunderscore}{\kern0pt}forcing{\isacharunderscore}{\kern0pt}data{\isacharunderscore}{\kern0pt}partial{\isacharcomma}{\kern0pt}\ simp{\isacharparenright}{\kern0pt}\isanewline
\ \ \ \ \ \ \ \isacommand{apply}\isamarkupfalse%
{\isacharparenleft}{\kern0pt}subst\ forcing{\isacharunderscore}{\kern0pt}data{\isacharunderscore}{\kern0pt}partial{\isachardot}{\kern0pt}is{\isacharunderscore}{\kern0pt}P{\isacharunderscore}{\kern0pt}auto{\isacharunderscore}{\kern0pt}group{\isacharunderscore}{\kern0pt}def{\isacharcomma}{\kern0pt}\ rule\ Fn{\isacharunderscore}{\kern0pt}forcing{\isacharunderscore}{\kern0pt}data{\isacharunderscore}{\kern0pt}partial{\isacharparenright}{\kern0pt}\isanewline
\ \ \ \ \ \ \isacommand{using}\isamarkupfalse%
\ SH\ TH\ \isanewline
\ \ \ \ \ \ \ \isacommand{apply}\isamarkupfalse%
\ force\isanewline
\ \ \ \ \ \ \isacommand{using}\isamarkupfalse%
\ assms\ Fn{\isacharunderscore}{\kern0pt}perms{\isacharunderscore}{\kern0pt}filter{\isacharunderscore}{\kern0pt}def\ \isanewline
\ \ \ \ \ \ \isacommand{by}\isamarkupfalse%
\ auto\isanewline
\ \ \isacommand{qed}\isamarkupfalse%
\isanewline
\isanewline
\ \ \isacommand{have}\isamarkupfalse%
\ conversein\ {\isacharcolon}{\kern0pt}\ {\isachardoublequoteopen}{\isasymAnd}A{\isachardot}{\kern0pt}\ A\ {\isasymin}\ X\ {\isasymLongrightarrow}\ converse{\isacharparenleft}{\kern0pt}A{\isacharparenright}{\kern0pt}\ {\isasymin}\ X{\isachardoublequoteclose}\ \isanewline
\ \ \isacommand{proof}\isamarkupfalse%
\ {\isacharminus}{\kern0pt}\ \isanewline
\ \ \ \ \isacommand{fix}\isamarkupfalse%
\ A\ \isanewline
\ \ \ \ \isacommand{assume}\isamarkupfalse%
\ assms{\isadigit{1}}\ {\isacharcolon}{\kern0pt}\ {\isachardoublequoteopen}A\ {\isasymin}\ X{\isachardoublequoteclose}\isanewline
\ \ \ \ \isacommand{obtain}\isamarkupfalse%
\ S\ \isakeyword{where}\ SH{\isacharcolon}{\kern0pt}\ {\isachardoublequoteopen}A\ {\isacharequal}{\kern0pt}\ F\ O\ S\ O\ converse{\isacharparenleft}{\kern0pt}F{\isacharparenright}{\kern0pt}{\isachardoublequoteclose}\ {\isachardoublequoteopen}S\ {\isasymin}\ H{\isachardoublequoteclose}\ \isanewline
\ \ \ \ \ \ \isacommand{using}\isamarkupfalse%
\ assms{\isadigit{1}}\ X{\isacharunderscore}{\kern0pt}def\ \isacommand{by}\isamarkupfalse%
\ auto\isanewline
\ \ \ \ \isacommand{then}\isamarkupfalse%
\ \isacommand{have}\isamarkupfalse%
\ {\isachardoublequoteopen}converse{\isacharparenleft}{\kern0pt}A{\isacharparenright}{\kern0pt}\ {\isacharequal}{\kern0pt}\ converse{\isacharparenleft}{\kern0pt}F\ O\ {\isacharparenleft}{\kern0pt}S\ O\ converse{\isacharparenleft}{\kern0pt}F{\isacharparenright}{\kern0pt}{\isacharparenright}{\kern0pt}{\isacharparenright}{\kern0pt}{\isachardoublequoteclose}\ \isacommand{by}\isamarkupfalse%
\ auto\ \isanewline
\ \ \ \ \isacommand{also}\isamarkupfalse%
\ \isacommand{have}\isamarkupfalse%
\ {\isachardoublequoteopen}{\isachardot}{\kern0pt}{\isachardot}{\kern0pt}{\isachardot}{\kern0pt}\ {\isacharequal}{\kern0pt}\ converse{\isacharparenleft}{\kern0pt}S\ O\ converse{\isacharparenleft}{\kern0pt}F{\isacharparenright}{\kern0pt}{\isacharparenright}{\kern0pt}\ O\ converse{\isacharparenleft}{\kern0pt}F{\isacharparenright}{\kern0pt}{\isachardoublequoteclose}\ \isacommand{by}\isamarkupfalse%
\ {\isacharparenleft}{\kern0pt}subst\ converse{\isacharunderscore}{\kern0pt}comp{\isacharcomma}{\kern0pt}\ simp{\isacharparenright}{\kern0pt}\isanewline
\ \ \ \ \isacommand{also}\isamarkupfalse%
\ \isacommand{have}\isamarkupfalse%
\ {\isachardoublequoteopen}{\isachardot}{\kern0pt}{\isachardot}{\kern0pt}{\isachardot}{\kern0pt}\ {\isacharequal}{\kern0pt}\ {\isacharparenleft}{\kern0pt}converse{\isacharparenleft}{\kern0pt}converse{\isacharparenleft}{\kern0pt}F{\isacharparenright}{\kern0pt}{\isacharparenright}{\kern0pt}\ O\ converse{\isacharparenleft}{\kern0pt}S{\isacharparenright}{\kern0pt}{\isacharparenright}{\kern0pt}\ O\ converse{\isacharparenleft}{\kern0pt}F{\isacharparenright}{\kern0pt}{\isachardoublequoteclose}\ \isacommand{by}\isamarkupfalse%
\ {\isacharparenleft}{\kern0pt}subst\ converse{\isacharunderscore}{\kern0pt}comp{\isacharcomma}{\kern0pt}\ simp{\isacharparenright}{\kern0pt}\isanewline
\ \ \ \ \isacommand{also}\isamarkupfalse%
\ \isacommand{have}\isamarkupfalse%
\ {\isachardoublequoteopen}{\isachardot}{\kern0pt}{\isachardot}{\kern0pt}{\isachardot}{\kern0pt}\ {\isacharequal}{\kern0pt}\ F\ O\ converse{\isacharparenleft}{\kern0pt}S{\isacharparenright}{\kern0pt}\ O\ converse{\isacharparenleft}{\kern0pt}F{\isacharparenright}{\kern0pt}{\isachardoublequoteclose}\ \isacommand{using}\isamarkupfalse%
\ comp{\isacharunderscore}{\kern0pt}assoc\ \isacommand{by}\isamarkupfalse%
\ auto\isanewline
\ \ \ \ \isacommand{finally}\isamarkupfalse%
\ \isacommand{show}\isamarkupfalse%
\ {\isachardoublequoteopen}converse{\isacharparenleft}{\kern0pt}A{\isacharparenright}{\kern0pt}\ {\isasymin}\ X{\isachardoublequoteclose}\ \isanewline
\ \ \ \ \ \ \isacommand{apply}\isamarkupfalse%
{\isacharparenleft}{\kern0pt}simp\ add{\isacharcolon}{\kern0pt}X{\isacharunderscore}{\kern0pt}def{\isacharparenright}{\kern0pt}\isanewline
\ \ \ \ \ \ \isacommand{apply}\isamarkupfalse%
{\isacharparenleft}{\kern0pt}rule{\isacharunderscore}{\kern0pt}tac\ x{\isacharequal}{\kern0pt}{\isachardoublequoteopen}converse{\isacharparenleft}{\kern0pt}S{\isacharparenright}{\kern0pt}{\isachardoublequoteclose}\ \isakeyword{in}\ bexI{\isacharcomma}{\kern0pt}\ simp{\isacharparenright}{\kern0pt}\isanewline
\ \ \ \ \ \ \isacommand{apply}\isamarkupfalse%
{\isacharparenleft}{\kern0pt}rule{\isacharunderscore}{\kern0pt}tac\ P{\isacharequal}{\kern0pt}{\isachardoublequoteopen}H\ {\isasymin}\ forcing{\isacharunderscore}{\kern0pt}data{\isacharunderscore}{\kern0pt}partial{\isachardot}{\kern0pt}P{\isacharunderscore}{\kern0pt}auto{\isacharunderscore}{\kern0pt}subgroups{\isacharparenleft}{\kern0pt}Fn{\isacharcomma}{\kern0pt}\ Fn{\isacharunderscore}{\kern0pt}leq{\isacharcomma}{\kern0pt}\ M{\isacharcomma}{\kern0pt}\ Fn{\isacharunderscore}{\kern0pt}perms{\isacharparenright}{\kern0pt}{\isachardoublequoteclose}\ \isakeyword{in}\ mp{\isacharparenright}{\kern0pt}\isanewline
\ \ \ \ \ \ \ \isacommand{apply}\isamarkupfalse%
{\isacharparenleft}{\kern0pt}subst\ forcing{\isacharunderscore}{\kern0pt}data{\isacharunderscore}{\kern0pt}partial{\isachardot}{\kern0pt}P{\isacharunderscore}{\kern0pt}auto{\isacharunderscore}{\kern0pt}subgroups{\isacharunderscore}{\kern0pt}def{\isacharcomma}{\kern0pt}\ rule\ Fn{\isacharunderscore}{\kern0pt}forcing{\isacharunderscore}{\kern0pt}data{\isacharunderscore}{\kern0pt}partial{\isacharcomma}{\kern0pt}\ simp{\isacharparenright}{\kern0pt}\isanewline
\ \ \ \ \ \ \ \isacommand{apply}\isamarkupfalse%
{\isacharparenleft}{\kern0pt}subst\ forcing{\isacharunderscore}{\kern0pt}data{\isacharunderscore}{\kern0pt}partial{\isachardot}{\kern0pt}is{\isacharunderscore}{\kern0pt}P{\isacharunderscore}{\kern0pt}auto{\isacharunderscore}{\kern0pt}group{\isacharunderscore}{\kern0pt}def{\isacharcomma}{\kern0pt}\ rule\ Fn{\isacharunderscore}{\kern0pt}forcing{\isacharunderscore}{\kern0pt}data{\isacharunderscore}{\kern0pt}partial{\isacharparenright}{\kern0pt}\isanewline
\ \ \ \ \ \ \isacommand{using}\isamarkupfalse%
\ SH\isanewline
\ \ \ \ \ \ \ \isacommand{apply}\isamarkupfalse%
\ force\isanewline
\ \ \ \ \ \ \isacommand{using}\isamarkupfalse%
\ assms\ Fn{\isacharunderscore}{\kern0pt}perms{\isacharunderscore}{\kern0pt}filter{\isacharunderscore}{\kern0pt}def\ \isanewline
\ \ \ \ \ \ \isacommand{by}\isamarkupfalse%
\ auto\isanewline
\ \ \isacommand{qed}\isamarkupfalse%
\isanewline
\isanewline
\ \ \isacommand{have}\isamarkupfalse%
\ Xgroup\ {\isacharcolon}{\kern0pt}\ {\isachardoublequoteopen}forcing{\isacharunderscore}{\kern0pt}data{\isacharunderscore}{\kern0pt}partial{\isachardot}{\kern0pt}is{\isacharunderscore}{\kern0pt}P{\isacharunderscore}{\kern0pt}auto{\isacharunderscore}{\kern0pt}group{\isacharparenleft}{\kern0pt}Fn{\isacharcomma}{\kern0pt}\ Fn{\isacharunderscore}{\kern0pt}leq{\isacharcomma}{\kern0pt}\ M{\isacharcomma}{\kern0pt}\ X{\isacharparenright}{\kern0pt}{\isachardoublequoteclose}\isanewline
\ \ \ \ \isacommand{apply}\isamarkupfalse%
{\isacharparenleft}{\kern0pt}subst\ forcing{\isacharunderscore}{\kern0pt}data{\isacharunderscore}{\kern0pt}partial{\isachardot}{\kern0pt}is{\isacharunderscore}{\kern0pt}P{\isacharunderscore}{\kern0pt}auto{\isacharunderscore}{\kern0pt}group{\isacharunderscore}{\kern0pt}def{\isacharparenright}{\kern0pt}\isanewline
\ \ \ \ \isacommand{apply}\isamarkupfalse%
{\isacharparenleft}{\kern0pt}rule\ Fn{\isacharunderscore}{\kern0pt}forcing{\isacharunderscore}{\kern0pt}data{\isacharunderscore}{\kern0pt}partial{\isacharparenright}{\kern0pt}\isanewline
\ \ \ \ \isacommand{using}\isamarkupfalse%
\ Xsubset{\isacharprime}{\kern0pt}\ Xsubset{\isacharprime}{\kern0pt}{\isacharprime}{\kern0pt}\ bij{\isacharunderscore}{\kern0pt}def\ inj{\isacharunderscore}{\kern0pt}def\ compin\ conversein\isanewline
\ \ \ \ \isacommand{by}\isamarkupfalse%
\ auto\isanewline
\isanewline
\ \ \isacommand{have}\isamarkupfalse%
\ Xsubgroup\ {\isacharcolon}{\kern0pt}{\isachardoublequoteopen}X\ {\isasymin}\ forcing{\isacharunderscore}{\kern0pt}data{\isacharunderscore}{\kern0pt}partial{\isachardot}{\kern0pt}P{\isacharunderscore}{\kern0pt}auto{\isacharunderscore}{\kern0pt}subgroups{\isacharparenleft}{\kern0pt}Fn{\isacharcomma}{\kern0pt}\ Fn{\isacharunderscore}{\kern0pt}leq{\isacharcomma}{\kern0pt}\ M{\isacharcomma}{\kern0pt}\ Fn{\isacharunderscore}{\kern0pt}perms{\isacharparenright}{\kern0pt}{\isachardoublequoteclose}\ \isanewline
\ \ \ \ \isacommand{apply}\isamarkupfalse%
{\isacharparenleft}{\kern0pt}subst\ forcing{\isacharunderscore}{\kern0pt}data{\isacharunderscore}{\kern0pt}partial{\isachardot}{\kern0pt}P{\isacharunderscore}{\kern0pt}auto{\isacharunderscore}{\kern0pt}subgroups{\isacharunderscore}{\kern0pt}def{\isacharparenright}{\kern0pt}\isanewline
\ \ \ \ \ \isacommand{apply}\isamarkupfalse%
{\isacharparenleft}{\kern0pt}rule\ Fn{\isacharunderscore}{\kern0pt}forcing{\isacharunderscore}{\kern0pt}data{\isacharunderscore}{\kern0pt}partial{\isacharparenright}{\kern0pt}\isanewline
\ \ \ \ \isacommand{using}\isamarkupfalse%
\ Xsubset\ XinM\ Xgroup\isanewline
\ \ \ \ \isacommand{by}\isamarkupfalse%
\ auto\isanewline
\isanewline
\ \ \isacommand{obtain}\isamarkupfalse%
\ E\ \isakeyword{where}\ EH\ {\isacharcolon}{\kern0pt}\ {\isachardoublequoteopen}E\ {\isasymin}\ Pow{\isacharparenleft}{\kern0pt}nat{\isacharparenright}{\kern0pt}\ {\isasyminter}\ M{\isachardoublequoteclose}\ {\isachardoublequoteopen}finite{\isacharunderscore}{\kern0pt}M{\isacharparenleft}{\kern0pt}E{\isacharparenright}{\kern0pt}{\isachardoublequoteclose}\ {\isachardoublequoteopen}Fix{\isacharparenleft}{\kern0pt}E{\isacharparenright}{\kern0pt}\ {\isasymsubseteq}\ H{\isachardoublequoteclose}\ \isanewline
\ \ \ \ \isacommand{using}\isamarkupfalse%
\ assms\ Fn{\isacharunderscore}{\kern0pt}perms{\isacharunderscore}{\kern0pt}filter{\isacharunderscore}{\kern0pt}def\isanewline
\ \ \ \ \isacommand{by}\isamarkupfalse%
\ auto\isanewline
\isanewline
\ \ \isacommand{then}\isamarkupfalse%
\ \isacommand{obtain}\isamarkupfalse%
\ m\ e\ \isakeyword{where}\ meH{\isacharcolon}{\kern0pt}\ {\isachardoublequoteopen}m\ {\isasymin}\ nat{\isachardoublequoteclose}\ {\isachardoublequoteopen}e\ {\isasymin}\ inj{\isacharparenleft}{\kern0pt}E{\isacharcomma}{\kern0pt}\ m{\isacharparenright}{\kern0pt}{\isachardoublequoteclose}\ {\isachardoublequoteopen}e\ {\isasymin}\ M{\isachardoublequoteclose}\ \isacommand{using}\isamarkupfalse%
\ finite{\isacharunderscore}{\kern0pt}M{\isacharunderscore}{\kern0pt}def\ \isacommand{by}\isamarkupfalse%
\ force\isanewline
\isanewline
\ \ \isacommand{have}\isamarkupfalse%
\ imgsubset{\isacharcolon}{\kern0pt}\ {\isachardoublequoteopen}f{\isacharbackquote}{\kern0pt}{\isacharbackquote}{\kern0pt}E\ {\isasymsubseteq}\ nat{\isachardoublequoteclose}\ \isanewline
\ \ \isacommand{proof}\isamarkupfalse%
\ {\isacharparenleft}{\kern0pt}rule\ subsetI{\isacharparenright}{\kern0pt}\isanewline
\ \ \ \ \isacommand{fix}\isamarkupfalse%
\ y\ \isacommand{assume}\isamarkupfalse%
\ {\isachardoublequoteopen}y\ {\isasymin}\ f{\isacharbackquote}{\kern0pt}{\isacharbackquote}{\kern0pt}E{\isachardoublequoteclose}\ \isanewline
\ \ \ \ \isacommand{then}\isamarkupfalse%
\ \isacommand{obtain}\isamarkupfalse%
\ x\ \isakeyword{where}\ {\isachardoublequoteopen}{\isacharless}{\kern0pt}x{\isacharcomma}{\kern0pt}\ y{\isachargreater}{\kern0pt}\ {\isasymin}\ f{\isachardoublequoteclose}\ \isacommand{by}\isamarkupfalse%
\ auto\isanewline
\ \ \ \ \isacommand{then}\isamarkupfalse%
\ \isacommand{show}\isamarkupfalse%
\ {\isachardoublequoteopen}y\ {\isasymin}\ nat{\isachardoublequoteclose}\ \isacommand{using}\isamarkupfalse%
\ fH\ nat{\isacharunderscore}{\kern0pt}perms{\isacharunderscore}{\kern0pt}def\ bij{\isacharunderscore}{\kern0pt}def\ inj{\isacharunderscore}{\kern0pt}def\ Pi{\isacharunderscore}{\kern0pt}def\ \isacommand{by}\isamarkupfalse%
\ auto\isanewline
\ \ \isacommand{qed}\isamarkupfalse%
\isanewline
\isanewline
\ \ \isacommand{define}\isamarkupfalse%
\ e{\isacharprime}{\kern0pt}\ \isakeyword{where}\ {\isachardoublequoteopen}e{\isacharprime}{\kern0pt}\ {\isasymequiv}\ e\ O\ converse{\isacharparenleft}{\kern0pt}restrict{\isacharparenleft}{\kern0pt}f{\isacharcomma}{\kern0pt}\ E{\isacharparenright}{\kern0pt}{\isacharparenright}{\kern0pt}{\isachardoublequoteclose}\ \isanewline
\ \ \isacommand{have}\isamarkupfalse%
\ e{\isacharprime}{\kern0pt}in\ {\isacharcolon}{\kern0pt}\ {\isachardoublequoteopen}e{\isacharprime}{\kern0pt}\ {\isasymin}\ M{\isachardoublequoteclose}\ \isacommand{using}\isamarkupfalse%
\ e{\isacharprime}{\kern0pt}{\isacharunderscore}{\kern0pt}def\ comp{\isacharunderscore}{\kern0pt}closed\ meH\ converse{\isacharunderscore}{\kern0pt}closed\ fH\ nat{\isacharunderscore}{\kern0pt}perms{\isacharunderscore}{\kern0pt}def\ restrict{\isacharunderscore}{\kern0pt}closed\ EH\ image{\isacharunderscore}{\kern0pt}closed\ \isacommand{by}\isamarkupfalse%
\ auto\isanewline
\isanewline
\ \ \isacommand{have}\isamarkupfalse%
\ {\isachardoublequoteopen}e{\isacharprime}{\kern0pt}\ {\isasymin}\ inj{\isacharparenleft}{\kern0pt}f{\isacharbackquote}{\kern0pt}{\isacharbackquote}{\kern0pt}E{\isacharcomma}{\kern0pt}\ m{\isacharparenright}{\kern0pt}{\isachardoublequoteclose}\ \isanewline
\ \ \ \ \isacommand{unfolding}\isamarkupfalse%
\ e{\isacharprime}{\kern0pt}{\isacharunderscore}{\kern0pt}def\isanewline
\ \ \ \ \isacommand{apply}\isamarkupfalse%
{\isacharparenleft}{\kern0pt}rule{\isacharunderscore}{\kern0pt}tac\ B{\isacharequal}{\kern0pt}E\ \isakeyword{in}\ comp{\isacharunderscore}{\kern0pt}inj{\isacharparenright}{\kern0pt}\isanewline
\ \ \ \ \ \isacommand{apply}\isamarkupfalse%
{\isacharparenleft}{\kern0pt}rule\ bij{\isacharunderscore}{\kern0pt}is{\isacharunderscore}{\kern0pt}inj{\isacharcomma}{\kern0pt}\ rule\ bij{\isacharunderscore}{\kern0pt}converse{\isacharunderscore}{\kern0pt}bij{\isacharparenright}{\kern0pt}\isanewline
\ \ \ \ \ \isacommand{apply}\isamarkupfalse%
{\isacharparenleft}{\kern0pt}rule\ restrict{\isacharunderscore}{\kern0pt}bij{\isacharcomma}{\kern0pt}\ rule\ bij{\isacharunderscore}{\kern0pt}is{\isacharunderscore}{\kern0pt}inj{\isacharparenright}{\kern0pt}\isanewline
\ \ \ \ \isacommand{using}\isamarkupfalse%
\ fH\ nat{\isacharunderscore}{\kern0pt}perms{\isacharunderscore}{\kern0pt}def\ EH\ meH\isanewline
\ \ \ \ \isacommand{by}\isamarkupfalse%
\ auto\isanewline
\isanewline
\ \ \isacommand{then}\isamarkupfalse%
\ \isacommand{have}\isamarkupfalse%
\ imgfinite{\isacharcolon}{\kern0pt}\ {\isachardoublequoteopen}finite{\isacharunderscore}{\kern0pt}M{\isacharparenleft}{\kern0pt}f{\isacharbackquote}{\kern0pt}{\isacharbackquote}{\kern0pt}E{\isacharparenright}{\kern0pt}{\isachardoublequoteclose}\ \isanewline
\ \ \ \ \isacommand{unfolding}\isamarkupfalse%
\ finite{\isacharunderscore}{\kern0pt}M{\isacharunderscore}{\kern0pt}def\isanewline
\ \ \ \ \isacommand{using}\isamarkupfalse%
\ meH\ e{\isacharprime}{\kern0pt}in\ \isanewline
\ \ \ \ \isacommand{by}\isamarkupfalse%
\ force\isanewline
\isanewline
\ \ \isacommand{have}\isamarkupfalse%
\ fixsubset{\isacharcolon}{\kern0pt}\ {\isachardoublequoteopen}Fix{\isacharparenleft}{\kern0pt}f{\isacharbackquote}{\kern0pt}{\isacharbackquote}{\kern0pt}E{\isacharparenright}{\kern0pt}\ {\isasymsubseteq}\ X{\isachardoublequoteclose}\ \isanewline
\ \ \isacommand{proof}\isamarkupfalse%
{\isacharparenleft}{\kern0pt}rule\ subsetI{\isacharparenright}{\kern0pt}\isanewline
\ \ \ \ \isacommand{fix}\isamarkupfalse%
\ A\ \isacommand{assume}\isamarkupfalse%
\ assms{\isadigit{1}}\ {\isacharcolon}{\kern0pt}\ {\isachardoublequoteopen}A\ {\isasymin}\ Fix{\isacharparenleft}{\kern0pt}f{\isacharbackquote}{\kern0pt}{\isacharbackquote}{\kern0pt}E{\isacharparenright}{\kern0pt}{\isachardoublequoteclose}\ \isanewline
\ \ \ \ \isacommand{then}\isamarkupfalse%
\ \isacommand{obtain}\isamarkupfalse%
\ a\ \isakeyword{where}\ aH{\isacharcolon}{\kern0pt}\ {\isachardoublequoteopen}A\ {\isacharequal}{\kern0pt}\ Fn{\isacharunderscore}{\kern0pt}perm{\isacharprime}{\kern0pt}{\isacharparenleft}{\kern0pt}a{\isacharparenright}{\kern0pt}{\isachardoublequoteclose}\ {\isachardoublequoteopen}a\ {\isasymin}\ nat{\isacharunderscore}{\kern0pt}perms{\isachardoublequoteclose}\ {\isachardoublequoteopen}{\isasymforall}n\ {\isasymin}\ f{\isacharbackquote}{\kern0pt}{\isacharbackquote}{\kern0pt}E{\isachardot}{\kern0pt}\ a{\isacharbackquote}{\kern0pt}n\ {\isacharequal}{\kern0pt}\ n{\isachardoublequoteclose}\ \isacommand{using}\isamarkupfalse%
\ Fix{\isacharunderscore}{\kern0pt}def\ \isacommand{by}\isamarkupfalse%
\ auto\isanewline
\ \ \isanewline
\ \ \ \ \isacommand{define}\isamarkupfalse%
\ g\ \isakeyword{where}\ {\isachardoublequoteopen}g\ {\isasymequiv}\ converse{\isacharparenleft}{\kern0pt}f{\isacharparenright}{\kern0pt}\ O\ a\ O\ f{\isachardoublequoteclose}\ \isanewline
\ \ \ \ \isacommand{have}\isamarkupfalse%
\ gin{\isacharcolon}{\kern0pt}\ {\isachardoublequoteopen}g\ {\isasymin}\ nat{\isacharunderscore}{\kern0pt}perms{\isachardoublequoteclose}\ \isanewline
\ \ \ \ \ \ \isacommand{unfolding}\isamarkupfalse%
\ g{\isacharunderscore}{\kern0pt}def\ \isanewline
\ \ \ \ \ \ \isacommand{using}\isamarkupfalse%
\ composition{\isacharunderscore}{\kern0pt}in{\isacharunderscore}{\kern0pt}nat{\isacharunderscore}{\kern0pt}perms\ fH\ aH\ converse{\isacharunderscore}{\kern0pt}in{\isacharunderscore}{\kern0pt}nat{\isacharunderscore}{\kern0pt}perms\isanewline
\ \ \ \ \ \ \isacommand{by}\isamarkupfalse%
\ auto\isanewline
\isanewline
\ \ \ \ \isacommand{have}\isamarkupfalse%
\ {\isachardoublequoteopen}{\isasymAnd}n{\isachardot}{\kern0pt}\ n\ {\isasymin}\ E\ {\isasymLongrightarrow}\ g{\isacharbackquote}{\kern0pt}n\ {\isacharequal}{\kern0pt}\ n{\isachardoublequoteclose}\isanewline
\ \ \ \ \isacommand{proof}\isamarkupfalse%
{\isacharminus}{\kern0pt}\ \ \isanewline
\ \ \ \ \ \ \isacommand{fix}\isamarkupfalse%
\ n\ \isacommand{assume}\isamarkupfalse%
\ nin\ {\isacharcolon}{\kern0pt}\ {\isachardoublequoteopen}n\ {\isasymin}\ E{\isachardoublequoteclose}\ \isanewline
\ \ \ \ \ \ \isacommand{have}\isamarkupfalse%
\ {\isachardoublequoteopen}g{\isacharbackquote}{\kern0pt}n\ {\isacharequal}{\kern0pt}\ {\isacharparenleft}{\kern0pt}{\isacharparenleft}{\kern0pt}converse{\isacharparenleft}{\kern0pt}f{\isacharparenright}{\kern0pt}\ O\ a{\isacharparenright}{\kern0pt}\ O\ f{\isacharparenright}{\kern0pt}\ {\isacharbackquote}{\kern0pt}\ n{\isachardoublequoteclose}\ \isacommand{using}\isamarkupfalse%
\ g{\isacharunderscore}{\kern0pt}def\ comp{\isacharunderscore}{\kern0pt}assoc\ \isacommand{by}\isamarkupfalse%
\ simp\isanewline
\ \ \ \ \ \ \isacommand{also}\isamarkupfalse%
\ \isacommand{have}\isamarkupfalse%
\ {\isachardoublequoteopen}{\isachardot}{\kern0pt}{\isachardot}{\kern0pt}{\isachardot}{\kern0pt}\ {\isacharequal}{\kern0pt}\ {\isacharparenleft}{\kern0pt}converse{\isacharparenleft}{\kern0pt}f{\isacharparenright}{\kern0pt}\ O\ a{\isacharparenright}{\kern0pt}\ {\isacharbackquote}{\kern0pt}\ {\isacharparenleft}{\kern0pt}f\ {\isacharbackquote}{\kern0pt}\ n{\isacharparenright}{\kern0pt}{\isachardoublequoteclose}\ \isanewline
\ \ \ \ \ \ \ \ \isacommand{apply}\isamarkupfalse%
{\isacharparenleft}{\kern0pt}subst\ comp{\isacharunderscore}{\kern0pt}fun{\isacharunderscore}{\kern0pt}apply{\isacharparenright}{\kern0pt}\isanewline
\ \ \ \ \ \ \ \ \isacommand{using}\isamarkupfalse%
\ fH\ nat{\isacharunderscore}{\kern0pt}perms{\isacharunderscore}{\kern0pt}def\ bij{\isacharunderscore}{\kern0pt}def\ inj{\isacharunderscore}{\kern0pt}def\ \isanewline
\ \ \ \ \ \ \ \ \ \ \isacommand{apply}\isamarkupfalse%
\ auto{\isacharbrackleft}{\kern0pt}{\isadigit{1}}{\isacharbrackright}{\kern0pt}\isanewline
\ \ \ \ \ \ \ \ \isacommand{using}\isamarkupfalse%
\ nin\ EH\ \isanewline
\ \ \ \ \ \ \ \ \ \isacommand{apply}\isamarkupfalse%
\ force\isanewline
\ \ \ \ \ \ \ \ \isacommand{by}\isamarkupfalse%
\ simp\isanewline
\ \ \ \ \ \ \isacommand{also}\isamarkupfalse%
\ \isacommand{have}\isamarkupfalse%
\ {\isachardoublequoteopen}{\isachardot}{\kern0pt}{\isachardot}{\kern0pt}{\isachardot}{\kern0pt}\ {\isacharequal}{\kern0pt}\ converse{\isacharparenleft}{\kern0pt}f{\isacharparenright}{\kern0pt}\ {\isacharbackquote}{\kern0pt}\ {\isacharparenleft}{\kern0pt}a\ {\isacharbackquote}{\kern0pt}\ {\isacharparenleft}{\kern0pt}f\ {\isacharbackquote}{\kern0pt}\ n{\isacharparenright}{\kern0pt}{\isacharparenright}{\kern0pt}{\isachardoublequoteclose}\isanewline
\ \ \ \ \ \ \ \ \isacommand{apply}\isamarkupfalse%
{\isacharparenleft}{\kern0pt}subst\ comp{\isacharunderscore}{\kern0pt}fun{\isacharunderscore}{\kern0pt}apply{\isacharparenright}{\kern0pt}\isanewline
\ \ \ \ \ \ \ \ \isacommand{using}\isamarkupfalse%
\ aH\ nat{\isacharunderscore}{\kern0pt}perms{\isacharunderscore}{\kern0pt}def\ bij{\isacharunderscore}{\kern0pt}def\ inj{\isacharunderscore}{\kern0pt}def\ \isanewline
\ \ \ \ \ \ \ \ \ \ \isacommand{apply}\isamarkupfalse%
\ auto{\isacharbrackleft}{\kern0pt}{\isadigit{1}}{\isacharbrackright}{\kern0pt}\isanewline
\ \ \ \ \ \ \ \ \ \isacommand{apply}\isamarkupfalse%
{\isacharparenleft}{\kern0pt}rule\ function{\isacharunderscore}{\kern0pt}value{\isacharunderscore}{\kern0pt}in{\isacharparenright}{\kern0pt}\isanewline
\ \ \ \ \ \ \ \ \isacommand{using}\isamarkupfalse%
\ fH\ nat{\isacharunderscore}{\kern0pt}perms{\isacharunderscore}{\kern0pt}def\ bij{\isacharunderscore}{\kern0pt}def\ inj{\isacharunderscore}{\kern0pt}def\ \isanewline
\ \ \ \ \ \ \ \ \ \ \isacommand{apply}\isamarkupfalse%
\ auto{\isacharbrackleft}{\kern0pt}{\isadigit{1}}{\isacharbrackright}{\kern0pt}\isanewline
\ \ \ \ \ \ \ \ \isacommand{using}\isamarkupfalse%
\ nin\ EH\ \isanewline
\ \ \ \ \ \ \ \ \ \isacommand{apply}\isamarkupfalse%
\ force\isanewline
\ \ \ \ \ \ \ \ \isacommand{by}\isamarkupfalse%
\ simp\isanewline
\ \ \ \ \ \ \isacommand{also}\isamarkupfalse%
\ \isacommand{have}\isamarkupfalse%
\ {\isachardoublequoteopen}{\isachardot}{\kern0pt}{\isachardot}{\kern0pt}{\isachardot}{\kern0pt}\ {\isacharequal}{\kern0pt}\ converse{\isacharparenleft}{\kern0pt}f{\isacharparenright}{\kern0pt}\ {\isacharbackquote}{\kern0pt}\ {\isacharparenleft}{\kern0pt}f\ {\isacharbackquote}{\kern0pt}\ n{\isacharparenright}{\kern0pt}{\isachardoublequoteclose}\ \isanewline
\ \ \ \ \ \ \ \ \isacommand{apply}\isamarkupfalse%
{\isacharparenleft}{\kern0pt}subgoal{\isacharunderscore}{\kern0pt}tac\ {\isachardoublequoteopen}f\ {\isacharbackquote}{\kern0pt}\ n\ {\isasymin}\ f\ {\isacharbackquote}{\kern0pt}{\isacharbackquote}{\kern0pt}\ E{\isachardoublequoteclose}{\isacharparenright}{\kern0pt}\isanewline
\ \ \ \ \ \ \ \ \isacommand{using}\isamarkupfalse%
\ aH\ \isanewline
\ \ \ \ \ \ \ \ \ \isacommand{apply}\isamarkupfalse%
\ force\ \isanewline
\ \ \ \ \ \ \ \ \isacommand{apply}\isamarkupfalse%
{\isacharparenleft}{\kern0pt}rule\ imageI{\isacharcomma}{\kern0pt}\ rule\ function{\isacharunderscore}{\kern0pt}apply{\isacharunderscore}{\kern0pt}Pair{\isacharparenright}{\kern0pt}\isanewline
\ \ \ \ \ \ \ \ \isacommand{using}\isamarkupfalse%
\ fH\ nat{\isacharunderscore}{\kern0pt}perms{\isacharunderscore}{\kern0pt}def\ bij{\isacharunderscore}{\kern0pt}def\ inj{\isacharunderscore}{\kern0pt}def\ Pi{\isacharunderscore}{\kern0pt}def\ nin\ EH\isanewline
\ \ \ \ \ \ \ \ \isacommand{by}\isamarkupfalse%
\ auto\isanewline
\ \ \ \ \ \ \isacommand{also}\isamarkupfalse%
\ \isacommand{have}\isamarkupfalse%
\ {\isachardoublequoteopen}{\isachardot}{\kern0pt}{\isachardot}{\kern0pt}{\isachardot}{\kern0pt}\ {\isacharequal}{\kern0pt}\ n{\isachardoublequoteclose}\isanewline
\ \ \ \ \ \ \ \ \isacommand{apply}\isamarkupfalse%
{\isacharparenleft}{\kern0pt}rule\ left{\isacharunderscore}{\kern0pt}inverse{\isacharparenright}{\kern0pt}\isanewline
\ \ \ \ \ \ \ \ \isacommand{using}\isamarkupfalse%
\ fH\ nat{\isacharunderscore}{\kern0pt}perms{\isacharunderscore}{\kern0pt}def\ bij{\isacharunderscore}{\kern0pt}def\isanewline
\ \ \ \ \ \ \ \ \ \isacommand{apply}\isamarkupfalse%
\ force\isanewline
\ \ \ \ \ \ \ \ \isacommand{using}\isamarkupfalse%
\ nin\ EH\isanewline
\ \ \ \ \ \ \ \ \isacommand{by}\isamarkupfalse%
\ auto\isanewline
\ \ \ \ \ \ \isacommand{finally}\isamarkupfalse%
\ \isacommand{show}\isamarkupfalse%
\ {\isachardoublequoteopen}g{\isacharbackquote}{\kern0pt}n\ {\isacharequal}{\kern0pt}\ n{\isachardoublequoteclose}\ \isacommand{by}\isamarkupfalse%
\ auto\isanewline
\ \ \ \ \isacommand{qed}\isamarkupfalse%
\isanewline
\ \ \ \ \ \ \ \ \isanewline
\ \ \ \ \isacommand{then}\isamarkupfalse%
\ \isacommand{have}\isamarkupfalse%
\ gfix\ {\isacharcolon}{\kern0pt}\ {\isachardoublequoteopen}Fn{\isacharunderscore}{\kern0pt}perm{\isacharprime}{\kern0pt}{\isacharparenleft}{\kern0pt}g{\isacharparenright}{\kern0pt}\ {\isasymin}\ Fix{\isacharparenleft}{\kern0pt}E{\isacharparenright}{\kern0pt}{\isachardoublequoteclose}\ \isanewline
\ \ \ \ \ \ \isacommand{unfolding}\isamarkupfalse%
\ Fix{\isacharunderscore}{\kern0pt}def\ \isanewline
\ \ \ \ \ \ \isacommand{using}\isamarkupfalse%
\ gin\isanewline
\ \ \ \ \ \ \isacommand{by}\isamarkupfalse%
\ auto\isanewline
\isanewline
\ \ \ \ \isacommand{have}\isamarkupfalse%
\ {\isachardoublequoteopen}f\ O\ g\ O\ converse{\isacharparenleft}{\kern0pt}f{\isacharparenright}{\kern0pt}\ {\isacharequal}{\kern0pt}\ {\isacharparenleft}{\kern0pt}f\ O\ converse{\isacharparenleft}{\kern0pt}f{\isacharparenright}{\kern0pt}{\isacharparenright}{\kern0pt}\ O\ a\ O\ {\isacharparenleft}{\kern0pt}f\ O\ converse{\isacharparenleft}{\kern0pt}f{\isacharparenright}{\kern0pt}{\isacharparenright}{\kern0pt}{\isachardoublequoteclose}\ \isanewline
\ \ \ \ \ \ \isacommand{unfolding}\isamarkupfalse%
\ g{\isacharunderscore}{\kern0pt}def\isanewline
\ \ \ \ \ \ \isacommand{using}\isamarkupfalse%
\ comp{\isacharunderscore}{\kern0pt}assoc\isanewline
\ \ \ \ \ \ \isacommand{by}\isamarkupfalse%
\ auto\isanewline
\ \ \ \ \isacommand{also}\isamarkupfalse%
\ \isacommand{have}\isamarkupfalse%
\ {\isachardoublequoteopen}{\isachardot}{\kern0pt}{\isachardot}{\kern0pt}{\isachardot}{\kern0pt}\ {\isacharequal}{\kern0pt}\ id{\isacharparenleft}{\kern0pt}nat{\isacharparenright}{\kern0pt}\ O\ {\isacharparenleft}{\kern0pt}a\ O\ id{\isacharparenleft}{\kern0pt}nat{\isacharparenright}{\kern0pt}{\isacharparenright}{\kern0pt}{\isachardoublequoteclose}\ \isanewline
\ \ \ \ \ \ \isacommand{apply}\isamarkupfalse%
{\isacharparenleft}{\kern0pt}subst\ right{\isacharunderscore}{\kern0pt}comp{\isacharunderscore}{\kern0pt}inverse{\isacharparenright}{\kern0pt}\isanewline
\ \ \ \ \ \ \isacommand{using}\isamarkupfalse%
\ fH\ nat{\isacharunderscore}{\kern0pt}perms{\isacharunderscore}{\kern0pt}def\ bij{\isacharunderscore}{\kern0pt}def\isanewline
\ \ \ \ \ \ \isacommand{apply}\isamarkupfalse%
\ force\isanewline
\ \ \ \ \ \ \isacommand{apply}\isamarkupfalse%
{\isacharparenleft}{\kern0pt}subst\ right{\isacharunderscore}{\kern0pt}comp{\isacharunderscore}{\kern0pt}inverse{\isacharparenright}{\kern0pt}\isanewline
\ \ \ \ \ \ \isacommand{using}\isamarkupfalse%
\ fH\ nat{\isacharunderscore}{\kern0pt}perms{\isacharunderscore}{\kern0pt}def\ bij{\isacharunderscore}{\kern0pt}def\isanewline
\ \ \ \ \ \ \isacommand{by}\isamarkupfalse%
\ auto\isanewline
\ \ \ \ \isacommand{also}\isamarkupfalse%
\ \isacommand{have}\isamarkupfalse%
\ {\isachardoublequoteopen}{\isachardot}{\kern0pt}{\isachardot}{\kern0pt}{\isachardot}{\kern0pt}\ {\isacharequal}{\kern0pt}\ id{\isacharparenleft}{\kern0pt}nat{\isacharparenright}{\kern0pt}\ O\ a{\isachardoublequoteclose}\ \isanewline
\ \ \ \ \ \ \isacommand{apply}\isamarkupfalse%
{\isacharparenleft}{\kern0pt}subst\ right{\isacharunderscore}{\kern0pt}comp{\isacharunderscore}{\kern0pt}id{\isacharparenright}{\kern0pt}\isanewline
\ \ \ \ \ \ \isacommand{using}\isamarkupfalse%
\ aH\ nat{\isacharunderscore}{\kern0pt}perms{\isacharunderscore}{\kern0pt}def\ bij{\isacharunderscore}{\kern0pt}def\ inj{\isacharunderscore}{\kern0pt}def\ Pi{\isacharunderscore}{\kern0pt}def\isanewline
\ \ \ \ \ \ \isacommand{by}\isamarkupfalse%
\ auto\isanewline
\ \ \ \ \isacommand{also}\isamarkupfalse%
\ \isacommand{have}\isamarkupfalse%
\ {\isachardoublequoteopen}{\isachardot}{\kern0pt}{\isachardot}{\kern0pt}{\isachardot}{\kern0pt}\ {\isacharequal}{\kern0pt}\ a{\isachardoublequoteclose}\isanewline
\ \ \ \ \ \ \isacommand{apply}\isamarkupfalse%
{\isacharparenleft}{\kern0pt}subst\ left{\isacharunderscore}{\kern0pt}comp{\isacharunderscore}{\kern0pt}id{\isacharparenright}{\kern0pt}\isanewline
\ \ \ \ \ \ \isacommand{using}\isamarkupfalse%
\ aH\ nat{\isacharunderscore}{\kern0pt}perms{\isacharunderscore}{\kern0pt}def\ bij{\isacharunderscore}{\kern0pt}def\ inj{\isacharunderscore}{\kern0pt}def\ Pi{\isacharunderscore}{\kern0pt}def\isanewline
\ \ \ \ \ \ \isacommand{by}\isamarkupfalse%
\ auto\isanewline
\ \ \ \ \isacommand{finally}\isamarkupfalse%
\ \isacommand{have}\isamarkupfalse%
\ {\isachardoublequoteopen}a\ {\isacharequal}{\kern0pt}\ f\ O\ g\ O\ converse{\isacharparenleft}{\kern0pt}f{\isacharparenright}{\kern0pt}{\isachardoublequoteclose}\ \isacommand{by}\isamarkupfalse%
\ simp\isanewline
\ \ \ \ \isacommand{then}\isamarkupfalse%
\ \isacommand{have}\isamarkupfalse%
\ {\isachardoublequoteopen}Fn{\isacharunderscore}{\kern0pt}perm{\isacharprime}{\kern0pt}{\isacharparenleft}{\kern0pt}a{\isacharparenright}{\kern0pt}\ {\isacharequal}{\kern0pt}\ Fn{\isacharunderscore}{\kern0pt}perm{\isacharprime}{\kern0pt}{\isacharparenleft}{\kern0pt}f\ O\ g\ O\ converse{\isacharparenleft}{\kern0pt}f{\isacharparenright}{\kern0pt}{\isacharparenright}{\kern0pt}{\isachardoublequoteclose}\ \isacommand{by}\isamarkupfalse%
\ simp\isanewline
\ \ \ \ \isacommand{then}\isamarkupfalse%
\ \isacommand{have}\isamarkupfalse%
\ {\isachardoublequoteopen}A\ {\isacharequal}{\kern0pt}\ Fn{\isacharunderscore}{\kern0pt}perm{\isacharprime}{\kern0pt}{\isacharparenleft}{\kern0pt}f\ O\ g\ O\ converse{\isacharparenleft}{\kern0pt}f{\isacharparenright}{\kern0pt}{\isacharparenright}{\kern0pt}{\isachardoublequoteclose}\ \isacommand{using}\isamarkupfalse%
\ aH\ \isacommand{by}\isamarkupfalse%
\ auto\isanewline
\ \ \ \ \isacommand{also}\isamarkupfalse%
\ \isacommand{have}\isamarkupfalse%
\ {\isachardoublequoteopen}{\isachardot}{\kern0pt}{\isachardot}{\kern0pt}{\isachardot}{\kern0pt}\ {\isacharequal}{\kern0pt}\ Fn{\isacharunderscore}{\kern0pt}perm{\isacharprime}{\kern0pt}{\isacharparenleft}{\kern0pt}f{\isacharparenright}{\kern0pt}\ O\ Fn{\isacharunderscore}{\kern0pt}perm{\isacharprime}{\kern0pt}{\isacharparenleft}{\kern0pt}g\ O\ converse{\isacharparenleft}{\kern0pt}f{\isacharparenright}{\kern0pt}{\isacharparenright}{\kern0pt}{\isachardoublequoteclose}\ \isanewline
\ \ \ \ \ \ \isacommand{apply}\isamarkupfalse%
{\isacharparenleft}{\kern0pt}subst\ Fn{\isacharunderscore}{\kern0pt}perm{\isacharprime}{\kern0pt}{\isacharunderscore}{\kern0pt}comp{\isacharparenright}{\kern0pt}\isanewline
\ \ \ \ \ \ \isacommand{using}\isamarkupfalse%
\ fH\ composition{\isacharunderscore}{\kern0pt}in{\isacharunderscore}{\kern0pt}nat{\isacharunderscore}{\kern0pt}perms\ converse{\isacharunderscore}{\kern0pt}in{\isacharunderscore}{\kern0pt}nat{\isacharunderscore}{\kern0pt}perms\ fH\ gin\isanewline
\ \ \ \ \ \ \isacommand{by}\isamarkupfalse%
\ auto\isanewline
\ \ \ \ \isacommand{also}\isamarkupfalse%
\ \isacommand{have}\isamarkupfalse%
\ {\isachardoublequoteopen}{\isachardot}{\kern0pt}{\isachardot}{\kern0pt}{\isachardot}{\kern0pt}\ {\isacharequal}{\kern0pt}\ Fn{\isacharunderscore}{\kern0pt}perm{\isacharprime}{\kern0pt}{\isacharparenleft}{\kern0pt}f{\isacharparenright}{\kern0pt}\ O\ Fn{\isacharunderscore}{\kern0pt}perm{\isacharprime}{\kern0pt}{\isacharparenleft}{\kern0pt}g{\isacharparenright}{\kern0pt}\ O\ Fn{\isacharunderscore}{\kern0pt}perm{\isacharprime}{\kern0pt}{\isacharparenleft}{\kern0pt}converse{\isacharparenleft}{\kern0pt}f{\isacharparenright}{\kern0pt}{\isacharparenright}{\kern0pt}{\isachardoublequoteclose}\ \isanewline
\ \ \ \ \ \ \isacommand{apply}\isamarkupfalse%
{\isacharparenleft}{\kern0pt}subst\ {\isacharparenleft}{\kern0pt}{\isadigit{2}}{\isacharparenright}{\kern0pt}\ Fn{\isacharunderscore}{\kern0pt}perm{\isacharprime}{\kern0pt}{\isacharunderscore}{\kern0pt}comp{\isacharparenright}{\kern0pt}\isanewline
\ \ \ \ \ \ \isacommand{using}\isamarkupfalse%
\ fH\ composition{\isacharunderscore}{\kern0pt}in{\isacharunderscore}{\kern0pt}nat{\isacharunderscore}{\kern0pt}perms\ converse{\isacharunderscore}{\kern0pt}in{\isacharunderscore}{\kern0pt}nat{\isacharunderscore}{\kern0pt}perms\ fH\ gin\isanewline
\ \ \ \ \ \ \isacommand{by}\isamarkupfalse%
\ auto\isanewline
\ \ \ \ \isacommand{also}\isamarkupfalse%
\ \isacommand{have}\isamarkupfalse%
\ {\isachardoublequoteopen}{\isachardot}{\kern0pt}{\isachardot}{\kern0pt}{\isachardot}{\kern0pt}\ {\isacharequal}{\kern0pt}\ Fn{\isacharunderscore}{\kern0pt}perm{\isacharprime}{\kern0pt}{\isacharparenleft}{\kern0pt}f{\isacharparenright}{\kern0pt}\ O\ Fn{\isacharunderscore}{\kern0pt}perm{\isacharprime}{\kern0pt}{\isacharparenleft}{\kern0pt}g{\isacharparenright}{\kern0pt}\ O\ converse{\isacharparenleft}{\kern0pt}Fn{\isacharunderscore}{\kern0pt}perm{\isacharprime}{\kern0pt}{\isacharparenleft}{\kern0pt}f{\isacharparenright}{\kern0pt}{\isacharparenright}{\kern0pt}{\isachardoublequoteclose}\ \isanewline
\ \ \ \ \ \ \isacommand{using}\isamarkupfalse%
\ Fn{\isacharunderscore}{\kern0pt}perm{\isacharprime}{\kern0pt}{\isacharunderscore}{\kern0pt}converse\ fH\ converse{\isacharunderscore}{\kern0pt}in{\isacharunderscore}{\kern0pt}nat{\isacharunderscore}{\kern0pt}perms\isanewline
\ \ \ \ \ \ \isacommand{by}\isamarkupfalse%
\ auto\isanewline
\ \ \ \ \isacommand{also}\isamarkupfalse%
\ \isacommand{have}\isamarkupfalse%
\ {\isachardoublequoteopen}{\isachardot}{\kern0pt}{\isachardot}{\kern0pt}{\isachardot}{\kern0pt}\ {\isacharequal}{\kern0pt}\ F\ O\ Fn{\isacharunderscore}{\kern0pt}perm{\isacharprime}{\kern0pt}{\isacharparenleft}{\kern0pt}g{\isacharparenright}{\kern0pt}\ O\ converse{\isacharparenleft}{\kern0pt}F{\isacharparenright}{\kern0pt}{\isachardoublequoteclose}\ \isacommand{using}\isamarkupfalse%
\ fH\ \isacommand{by}\isamarkupfalse%
\ auto\isanewline
\ \ \ \ \isacommand{finally}\isamarkupfalse%
\ \isacommand{show}\isamarkupfalse%
\ {\isachardoublequoteopen}A\ {\isasymin}\ X{\isachardoublequoteclose}\ \isanewline
\ \ \ \ \ \ \isacommand{apply}\isamarkupfalse%
{\isacharparenleft}{\kern0pt}simp\ add{\isacharcolon}{\kern0pt}X{\isacharunderscore}{\kern0pt}def{\isacharparenright}{\kern0pt}\isanewline
\ \ \ \ \ \ \isacommand{apply}\isamarkupfalse%
{\isacharparenleft}{\kern0pt}rule{\isacharunderscore}{\kern0pt}tac\ x{\isacharequal}{\kern0pt}{\isachardoublequoteopen}Fn{\isacharunderscore}{\kern0pt}perm{\isacharprime}{\kern0pt}{\isacharparenleft}{\kern0pt}g{\isacharparenright}{\kern0pt}{\isachardoublequoteclose}\ \isakeyword{in}\ bexI{\isacharcomma}{\kern0pt}\ simp{\isacharparenright}{\kern0pt}\isanewline
\ \ \ \ \ \ \isacommand{using}\isamarkupfalse%
\ gfix\ EH\ \isanewline
\ \ \ \ \ \ \isacommand{by}\isamarkupfalse%
\ auto\isanewline
\ \ \isacommand{qed}\isamarkupfalse%
\isanewline
\isanewline
\ \ \isacommand{have}\isamarkupfalse%
\ {\isachardoublequoteopen}X\ {\isasymin}\ Fn{\isacharunderscore}{\kern0pt}perms{\isacharunderscore}{\kern0pt}filter{\isachardoublequoteclose}\ \isanewline
\ \ \ \ \isacommand{unfolding}\isamarkupfalse%
\ Fn{\isacharunderscore}{\kern0pt}perms{\isacharunderscore}{\kern0pt}filter{\isacharunderscore}{\kern0pt}def\ \isanewline
\ \ \ \ \isacommand{using}\isamarkupfalse%
\ Xsubgroup\ imgsubset\ fH\ EH\ nat{\isacharunderscore}{\kern0pt}perms{\isacharunderscore}{\kern0pt}def\ image{\isacharunderscore}{\kern0pt}closed\ imgfinite\ fixsubset\isanewline
\ \ \ \ \isacommand{by}\isamarkupfalse%
\ force\isanewline
\isanewline
\ \ \isacommand{then}\isamarkupfalse%
\ \isacommand{show}\isamarkupfalse%
\ {\isacharquery}{\kern0pt}thesis\ \isanewline
\ \ \ \ \isacommand{using}\isamarkupfalse%
\ X{\isacharunderscore}{\kern0pt}def\ \isanewline
\ \ \ \ \isacommand{by}\isamarkupfalse%
\ auto\isanewline
\isacommand{qed}\isamarkupfalse%
%
\endisatagproof
{\isafoldproof}%
%
\isadelimproof
\isanewline
%
\endisadelimproof
\isanewline
\isacommand{interpretation}\isamarkupfalse%
\ forcing{\isacharunderscore}{\kern0pt}data{\isacharunderscore}{\kern0pt}partial\ {\isachardoublequoteopen}Fn{\isachardoublequoteclose}\ {\isachardoublequoteopen}Fn{\isacharunderscore}{\kern0pt}leq{\isachardoublequoteclose}\ {\isachardoublequoteopen}{\isadigit{0}}{\isachardoublequoteclose}\ {\isachardoublequoteopen}M{\isachardoublequoteclose}\ {\isachardoublequoteopen}enum{\isachardoublequoteclose}%
\isadelimproof
\ %
\endisadelimproof
%
\isatagproof
\isacommand{using}\isamarkupfalse%
\ Fn{\isacharunderscore}{\kern0pt}forcing{\isacharunderscore}{\kern0pt}data{\isacharunderscore}{\kern0pt}partial\ \isacommand{by}\isamarkupfalse%
\ auto%
\endisatagproof
{\isafoldproof}%
%
\isadelimproof
%
\endisadelimproof
\isanewline
\isanewline
\isacommand{lemma}\isamarkupfalse%
\ Fn{\isacharunderscore}{\kern0pt}M{\isacharunderscore}{\kern0pt}symmetric{\isacharunderscore}{\kern0pt}system\ {\isacharcolon}{\kern0pt}\ {\isachardoublequoteopen}M{\isacharunderscore}{\kern0pt}symmetric{\isacharunderscore}{\kern0pt}system{\isacharparenleft}{\kern0pt}Fn{\isacharcomma}{\kern0pt}\ Fn{\isacharunderscore}{\kern0pt}leq{\isacharcomma}{\kern0pt}\ {\isadigit{0}}{\isacharcomma}{\kern0pt}\ M{\isacharcomma}{\kern0pt}\ enum{\isacharcomma}{\kern0pt}\ Fn{\isacharunderscore}{\kern0pt}perms{\isacharcomma}{\kern0pt}\ Fn{\isacharunderscore}{\kern0pt}perms{\isacharunderscore}{\kern0pt}filter{\isacharparenright}{\kern0pt}{\isachardoublequoteclose}\ \isanewline
%
\isadelimproof
\ \ %
\endisadelimproof
%
\isatagproof
\isacommand{unfolding}\isamarkupfalse%
\ M{\isacharunderscore}{\kern0pt}symmetric{\isacharunderscore}{\kern0pt}system{\isacharunderscore}{\kern0pt}def\isanewline
\ \ \isacommand{apply}\isamarkupfalse%
{\isacharparenleft}{\kern0pt}rule\ conjI{\isacharparenright}{\kern0pt}\isanewline
\ \ \ \isacommand{apply}\isamarkupfalse%
{\isacharparenleft}{\kern0pt}rule\ forcing{\isacharunderscore}{\kern0pt}data{\isacharunderscore}{\kern0pt}partial{\isacharunderscore}{\kern0pt}axioms{\isacharparenright}{\kern0pt}\isanewline
\ \ \isacommand{unfolding}\isamarkupfalse%
\ M{\isacharunderscore}{\kern0pt}symmetric{\isacharunderscore}{\kern0pt}system{\isacharunderscore}{\kern0pt}axioms{\isacharunderscore}{\kern0pt}def\ \isanewline
\ \ \isacommand{using}\isamarkupfalse%
\ Fn{\isacharunderscore}{\kern0pt}perms{\isacharunderscore}{\kern0pt}in{\isacharunderscore}{\kern0pt}M\ Fn{\isacharunderscore}{\kern0pt}perms{\isacharunderscore}{\kern0pt}group\ Fn{\isacharunderscore}{\kern0pt}perms{\isacharunderscore}{\kern0pt}filter{\isacharunderscore}{\kern0pt}in{\isacharunderscore}{\kern0pt}M\ Fn{\isacharunderscore}{\kern0pt}perms{\isacharunderscore}{\kern0pt}filter{\isacharunderscore}{\kern0pt}def\isanewline
\ \ \ \ \ \ \ \ Fn{\isacharunderscore}{\kern0pt}perms{\isacharunderscore}{\kern0pt}filter{\isacharunderscore}{\kern0pt}nonempty\ Fn{\isacharunderscore}{\kern0pt}perms{\isacharunderscore}{\kern0pt}filter{\isacharunderscore}{\kern0pt}intersection\ \ \isanewline
\ \ \ \ \ \ \ \ Fn{\isacharunderscore}{\kern0pt}perms{\isacharunderscore}{\kern0pt}filter{\isacharunderscore}{\kern0pt}supergroup\ Fn{\isacharunderscore}{\kern0pt}perms{\isacharunderscore}{\kern0pt}filter{\isacharunderscore}{\kern0pt}normal\isanewline
\ \ \isacommand{by}\isamarkupfalse%
\ auto%
\endisatagproof
{\isafoldproof}%
%
\isadelimproof
\isanewline
%
\endisadelimproof
\isanewline
\isacommand{end}\isamarkupfalse%
\isanewline
%
\isadelimtheory
%
\endisadelimtheory
%
\isatagtheory
\isacommand{end}\isamarkupfalse%
%
\endisatagtheory
{\isafoldtheory}%
%
\isadelimtheory
%
\endisadelimtheory
%
\end{isabellebody}%
\endinput
%:%file=~/source/repos/ZF-notAC/code/Fn_Perm_Filter.thy%:%
%:%10=1%:%
%:%11=1%:%
%:%12=2%:%
%:%13=3%:%
%:%18=3%:%
%:%21=4%:%
%:%22=5%:%
%:%23=5%:%
%:%24=6%:%
%:%25=7%:%
%:%26=7%:%
%:%29=8%:%
%:%33=8%:%
%:%34=8%:%
%:%35=9%:%
%:%36=9%:%
%:%37=10%:%
%:%38=10%:%
%:%39=11%:%
%:%40=11%:%
%:%41=12%:%
%:%42=12%:%
%:%43=13%:%
%:%44=13%:%
%:%45=14%:%
%:%46=14%:%
%:%47=15%:%
%:%48=15%:%
%:%49=16%:%
%:%50=16%:%
%:%51=17%:%
%:%52=17%:%
%:%53=18%:%
%:%54=18%:%
%:%55=19%:%
%:%56=19%:%
%:%57=20%:%
%:%58=20%:%
%:%59=21%:%
%:%60=21%:%
%:%61=22%:%
%:%62=22%:%
%:%63=23%:%
%:%64=23%:%
%:%65=24%:%
%:%66=24%:%
%:%67=25%:%
%:%68=25%:%
%:%73=25%:%
%:%76=26%:%
%:%77=27%:%
%:%78=27%:%
%:%79=28%:%
%:%80=29%:%
%:%81=29%:%
%:%82=30%:%
%:%83=31%:%
%:%84=32%:%
%:%85=32%:%
%:%86=33%:%
%:%87=34%:%
%:%88=35%:%
%:%91=36%:%
%:%95=36%:%
%:%96=36%:%
%:%97=37%:%
%:%98=37%:%
%:%99=38%:%
%:%100=38%:%
%:%101=39%:%
%:%102=39%:%
%:%103=40%:%
%:%104=40%:%
%:%109=40%:%
%:%112=41%:%
%:%113=42%:%
%:%114=42%:%
%:%115=43%:%
%:%116=44%:%
%:%117=45%:%
%:%120=46%:%
%:%124=46%:%
%:%125=46%:%
%:%126=47%:%
%:%127=47%:%
%:%128=48%:%
%:%129=48%:%
%:%130=49%:%
%:%131=49%:%
%:%132=50%:%
%:%133=50%:%
%:%134=51%:%
%:%135=51%:%
%:%136=52%:%
%:%137=52%:%
%:%138=53%:%
%:%139=53%:%
%:%140=54%:%
%:%141=54%:%
%:%142=55%:%
%:%143=55%:%
%:%144=56%:%
%:%145=56%:%
%:%146=57%:%
%:%147=57%:%
%:%148=58%:%
%:%149=58%:%
%:%150=59%:%
%:%151=59%:%
%:%152=60%:%
%:%153=60%:%
%:%154=61%:%
%:%155=61%:%
%:%156=62%:%
%:%157=62%:%
%:%158=63%:%
%:%159=63%:%
%:%160=64%:%
%:%161=64%:%
%:%162=65%:%
%:%163=65%:%
%:%164=66%:%
%:%165=66%:%
%:%166=67%:%
%:%167=67%:%
%:%168=68%:%
%:%169=68%:%
%:%170=69%:%
%:%176=69%:%
%:%179=70%:%
%:%180=71%:%
%:%181=71%:%
%:%182=72%:%
%:%183=73%:%
%:%184=74%:%
%:%185=75%:%
%:%192=76%:%
%:%193=76%:%
%:%194=77%:%
%:%195=77%:%
%:%196=78%:%
%:%197=78%:%
%:%198=79%:%
%:%199=79%:%
%:%200=80%:%
%:%201=80%:%
%:%202=81%:%
%:%203=81%:%
%:%204=82%:%
%:%205=82%:%
%:%206=83%:%
%:%207=83%:%
%:%208=84%:%
%:%209=84%:%
%:%210=85%:%
%:%211=85%:%
%:%212=86%:%
%:%213=86%:%
%:%214=87%:%
%:%215=87%:%
%:%216=88%:%
%:%217=88%:%
%:%218=89%:%
%:%219=89%:%
%:%220=90%:%
%:%221=90%:%
%:%222=91%:%
%:%223=91%:%
%:%224=92%:%
%:%225=92%:%
%:%226=93%:%
%:%227=93%:%
%:%228=94%:%
%:%229=94%:%
%:%230=95%:%
%:%231=95%:%
%:%232=96%:%
%:%233=96%:%
%:%234=97%:%
%:%235=97%:%
%:%236=98%:%
%:%237=98%:%
%:%238=99%:%
%:%239=99%:%
%:%240=100%:%
%:%241=100%:%
%:%242=101%:%
%:%243=101%:%
%:%244=102%:%
%:%245=102%:%
%:%246=103%:%
%:%247=103%:%
%:%248=103%:%
%:%249=103%:%
%:%250=104%:%
%:%256=104%:%
%:%259=105%:%
%:%260=106%:%
%:%261=106%:%
%:%262=107%:%
%:%263=108%:%
%:%264=109%:%
%:%271=110%:%
%:%272=110%:%
%:%273=111%:%
%:%274=111%:%
%:%275=112%:%
%:%276=113%:%
%:%277=113%:%
%:%278=114%:%
%:%279=114%:%
%:%280=115%:%
%:%281=115%:%
%:%282=116%:%
%:%283=116%:%
%:%284=117%:%
%:%285=117%:%
%:%286=118%:%
%:%287=118%:%
%:%288=119%:%
%:%289=119%:%
%:%290=120%:%
%:%291=120%:%
%:%292=121%:%
%:%293=121%:%
%:%294=122%:%
%:%295=122%:%
%:%296=123%:%
%:%297=123%:%
%:%298=124%:%
%:%299=124%:%
%:%300=125%:%
%:%301=125%:%
%:%302=126%:%
%:%303=126%:%
%:%304=127%:%
%:%305=127%:%
%:%306=127%:%
%:%307=128%:%
%:%308=128%:%
%:%309=129%:%
%:%310=129%:%
%:%311=130%:%
%:%312=130%:%
%:%313=131%:%
%:%314=131%:%
%:%315=131%:%
%:%316=131%:%
%:%317=131%:%
%:%318=132%:%
%:%324=132%:%
%:%327=133%:%
%:%328=134%:%
%:%329=134%:%
%:%330=135%:%
%:%331=136%:%
%:%332=136%:%
%:%333=137%:%
%:%334=138%:%
%:%335=139%:%
%:%338=140%:%
%:%339=141%:%
%:%343=141%:%
%:%344=141%:%
%:%345=142%:%
%:%346=142%:%
%:%347=143%:%
%:%348=143%:%
%:%349=144%:%
%:%350=144%:%
%:%351=145%:%
%:%352=145%:%
%:%357=145%:%
%:%360=146%:%
%:%361=147%:%
%:%362=147%:%
%:%363=148%:%
%:%364=149%:%
%:%365=150%:%
%:%368=151%:%
%:%369=152%:%
%:%373=152%:%
%:%374=152%:%
%:%375=153%:%
%:%376=153%:%
%:%377=154%:%
%:%378=154%:%
%:%379=155%:%
%:%380=155%:%
%:%381=156%:%
%:%382=156%:%
%:%383=157%:%
%:%384=157%:%
%:%385=158%:%
%:%386=158%:%
%:%387=159%:%
%:%388=159%:%
%:%389=160%:%
%:%390=160%:%
%:%391=161%:%
%:%392=161%:%
%:%393=162%:%
%:%394=162%:%
%:%395=163%:%
%:%396=163%:%
%:%397=164%:%
%:%398=164%:%
%:%399=165%:%
%:%400=165%:%
%:%401=166%:%
%:%402=166%:%
%:%403=167%:%
%:%404=167%:%
%:%409=167%:%
%:%412=168%:%
%:%413=169%:%
%:%414=169%:%
%:%415=170%:%
%:%416=171%:%
%:%417=172%:%
%:%418=173%:%
%:%425=174%:%
%:%426=174%:%
%:%427=175%:%
%:%428=175%:%
%:%429=176%:%
%:%430=176%:%
%:%431=177%:%
%:%432=177%:%
%:%433=178%:%
%:%434=178%:%
%:%435=179%:%
%:%436=179%:%
%:%437=180%:%
%:%438=180%:%
%:%439=181%:%
%:%440=181%:%
%:%441=182%:%
%:%442=182%:%
%:%443=183%:%
%:%444=183%:%
%:%445=184%:%
%:%446=184%:%
%:%447=185%:%
%:%448=185%:%
%:%449=186%:%
%:%450=186%:%
%:%451=187%:%
%:%452=187%:%
%:%453=188%:%
%:%454=188%:%
%:%455=189%:%
%:%456=189%:%
%:%457=189%:%
%:%458=190%:%
%:%459=190%:%
%:%460=190%:%
%:%461=190%:%
%:%462=191%:%
%:%468=191%:%
%:%471=192%:%
%:%472=193%:%
%:%473=194%:%
%:%474=194%:%
%:%475=195%:%
%:%476=196%:%
%:%477=197%:%
%:%478=198%:%
%:%479=199%:%
%:%480=200%:%
%:%481=201%:%
%:%484=202%:%
%:%488=202%:%
%:%489=202%:%
%:%494=202%:%
%:%497=203%:%
%:%498=204%:%
%:%499=204%:%
%:%504=209%:%
%:%505=210%:%
%:%506=211%:%
%:%507=211%:%
%:%508=212%:%
%:%509=213%:%
%:%510=214%:%
%:%517=215%:%
%:%518=215%:%
%:%519=216%:%
%:%520=216%:%
%:%521=217%:%
%:%522=217%:%
%:%523=218%:%
%:%524=218%:%
%:%525=219%:%
%:%526=219%:%
%:%527=220%:%
%:%528=220%:%
%:%529=221%:%
%:%530=221%:%
%:%531=222%:%
%:%532=222%:%
%:%533=223%:%
%:%534=223%:%
%:%535=224%:%
%:%536=224%:%
%:%537=225%:%
%:%538=225%:%
%:%539=226%:%
%:%540=226%:%
%:%541=227%:%
%:%542=227%:%
%:%543=228%:%
%:%544=228%:%
%:%545=229%:%
%:%546=229%:%
%:%547=230%:%
%:%548=230%:%
%:%549=231%:%
%:%550=231%:%
%:%551=232%:%
%:%552=232%:%
%:%553=233%:%
%:%554=233%:%
%:%555=234%:%
%:%556=234%:%
%:%557=235%:%
%:%558=235%:%
%:%559=236%:%
%:%560=236%:%
%:%561=237%:%
%:%562=237%:%
%:%563=238%:%
%:%564=238%:%
%:%565=239%:%
%:%566=239%:%
%:%567=240%:%
%:%568=240%:%
%:%569=241%:%
%:%570=241%:%
%:%571=242%:%
%:%572=242%:%
%:%573=243%:%
%:%574=243%:%
%:%575=243%:%
%:%576=243%:%
%:%577=244%:%
%:%583=244%:%
%:%586=245%:%
%:%587=246%:%
%:%588=246%:%
%:%589=247%:%
%:%590=248%:%
%:%591=249%:%
%:%592=250%:%
%:%593=251%:%
%:%594=252%:%
%:%595=253%:%
%:%598=254%:%
%:%602=254%:%
%:%603=254%:%
%:%604=255%:%
%:%605=255%:%
%:%610=255%:%
%:%613=256%:%
%:%614=256%:%
%:%615=257%:%
%:%619=261%:%
%:%620=262%:%
%:%621=263%:%
%:%622=263%:%
%:%623=264%:%
%:%624=265%:%
%:%625=266%:%
%:%628=267%:%
%:%633=268%:%
%:%634=268%:%
%:%635=269%:%
%:%636=269%:%
%:%638=271%:%
%:%639=272%:%
%:%640=272%:%
%:%641=273%:%
%:%642=273%:%
%:%643=274%:%
%:%644=274%:%
%:%645=275%:%
%:%646=275%:%
%:%647=276%:%
%:%648=276%:%
%:%649=277%:%
%:%650=277%:%
%:%651=278%:%
%:%652=278%:%
%:%653=279%:%
%:%654=279%:%
%:%655=280%:%
%:%656=280%:%
%:%657=281%:%
%:%658=281%:%
%:%659=282%:%
%:%660=282%:%
%:%661=283%:%
%:%662=283%:%
%:%663=284%:%
%:%664=284%:%
%:%665=285%:%
%:%666=285%:%
%:%667=286%:%
%:%668=286%:%
%:%669=287%:%
%:%670=287%:%
%:%671=288%:%
%:%672=288%:%
%:%673=288%:%
%:%674=289%:%
%:%675=289%:%
%:%676=290%:%
%:%677=290%:%
%:%678=291%:%
%:%679=291%:%
%:%680=292%:%
%:%681=292%:%
%:%682=293%:%
%:%683=293%:%
%:%684=294%:%
%:%685=294%:%
%:%686=295%:%
%:%687=295%:%
%:%688=295%:%
%:%689=295%:%
%:%690=296%:%
%:%696=296%:%
%:%699=297%:%
%:%700=298%:%
%:%701=298%:%
%:%702=299%:%
%:%703=300%:%
%:%704=301%:%
%:%707=302%:%
%:%711=302%:%
%:%712=302%:%
%:%713=303%:%
%:%714=303%:%
%:%715=304%:%
%:%716=304%:%
%:%721=304%:%
%:%724=305%:%
%:%725=306%:%
%:%726=306%:%
%:%727=307%:%
%:%728=308%:%
%:%729=309%:%
%:%732=310%:%
%:%736=310%:%
%:%737=310%:%
%:%738=311%:%
%:%739=311%:%
%:%740=312%:%
%:%741=312%:%
%:%742=313%:%
%:%748=313%:%
%:%751=314%:%
%:%752=315%:%
%:%753=316%:%
%:%754=316%:%
%:%755=317%:%
%:%756=318%:%
%:%757=318%:%
%:%764=319%:%
%:%765=319%:%
%:%766=320%:%
%:%767=320%:%
%:%768=321%:%
%:%769=322%:%
%:%770=322%:%
%:%771=323%:%
%:%772=323%:%
%:%773=324%:%
%:%774=324%:%
%:%775=325%:%
%:%776=325%:%
%:%777=326%:%
%:%778=326%:%
%:%779=327%:%
%:%780=327%:%
%:%781=328%:%
%:%782=328%:%
%:%783=329%:%
%:%784=329%:%
%:%785=330%:%
%:%786=330%:%
%:%787=331%:%
%:%788=331%:%
%:%789=332%:%
%:%790=332%:%
%:%791=333%:%
%:%792=333%:%
%:%793=334%:%
%:%794=334%:%
%:%795=335%:%
%:%796=335%:%
%:%797=336%:%
%:%798=336%:%
%:%799=337%:%
%:%800=337%:%
%:%801=338%:%
%:%802=338%:%
%:%803=339%:%
%:%804=339%:%
%:%805=340%:%
%:%806=340%:%
%:%807=341%:%
%:%808=341%:%
%:%809=342%:%
%:%810=343%:%
%:%811=343%:%
%:%812=344%:%
%:%813=344%:%
%:%814=345%:%
%:%815=345%:%
%:%816=346%:%
%:%817=346%:%
%:%818=347%:%
%:%819=347%:%
%:%820=348%:%
%:%821=348%:%
%:%822=349%:%
%:%823=349%:%
%:%824=350%:%
%:%825=350%:%
%:%826=351%:%
%:%827=351%:%
%:%828=352%:%
%:%829=352%:%
%:%830=353%:%
%:%831=353%:%
%:%832=354%:%
%:%833=354%:%
%:%834=355%:%
%:%835=355%:%
%:%836=356%:%
%:%837=356%:%
%:%838=357%:%
%:%839=357%:%
%:%840=358%:%
%:%841=358%:%
%:%842=359%:%
%:%843=359%:%
%:%844=360%:%
%:%845=360%:%
%:%846=360%:%
%:%847=361%:%
%:%848=361%:%
%:%849=362%:%
%:%850=362%:%
%:%851=363%:%
%:%852=363%:%
%:%853=364%:%
%:%854=364%:%
%:%855=364%:%
%:%856=365%:%
%:%857=365%:%
%:%858=366%:%
%:%859=366%:%
%:%860=367%:%
%:%861=367%:%
%:%862=368%:%
%:%863=368%:%
%:%864=369%:%
%:%865=369%:%
%:%866=370%:%
%:%867=370%:%
%:%868=371%:%
%:%869=371%:%
%:%870=372%:%
%:%871=372%:%
%:%872=373%:%
%:%873=373%:%
%:%874=374%:%
%:%875=374%:%
%:%876=375%:%
%:%877=375%:%
%:%878=376%:%
%:%879=377%:%
%:%880=377%:%
%:%881=377%:%
%:%882=377%:%
%:%883=377%:%
%:%884=378%:%
%:%890=378%:%
%:%893=379%:%
%:%894=380%:%
%:%895=380%:%
%:%898=381%:%
%:%902=381%:%
%:%903=381%:%
%:%904=382%:%
%:%905=382%:%
%:%906=383%:%
%:%907=383%:%
%:%908=384%:%
%:%909=384%:%
%:%910=385%:%
%:%911=385%:%
%:%912=386%:%
%:%913=386%:%
%:%914=387%:%
%:%915=387%:%
%:%916=388%:%
%:%917=388%:%
%:%918=389%:%
%:%919=389%:%
%:%920=390%:%
%:%921=390%:%
%:%922=391%:%
%:%923=391%:%
%:%924=392%:%
%:%925=392%:%
%:%926=393%:%
%:%927=393%:%
%:%932=393%:%
%:%935=394%:%
%:%936=395%:%
%:%937=395%:%
%:%938=396%:%
%:%939=397%:%
%:%940=398%:%
%:%943=399%:%
%:%947=399%:%
%:%948=399%:%
%:%949=400%:%
%:%950=400%:%
%:%951=401%:%
%:%952=401%:%
%:%957=401%:%
%:%960=402%:%
%:%961=403%:%
%:%962=404%:%
%:%963=404%:%
%:%964=405%:%
%:%965=406%:%
%:%966=406%:%
%:%969=407%:%
%:%973=407%:%
%:%974=407%:%
%:%975=408%:%
%:%976=408%:%
%:%977=409%:%
%:%978=409%:%
%:%983=409%:%
%:%986=410%:%
%:%987=411%:%
%:%988=411%:%
%:%995=412%:%
%:%996=412%:%
%:%997=413%:%
%:%998=413%:%
%:%999=414%:%
%:%1000=414%:%
%:%1001=415%:%
%:%1002=415%:%
%:%1003=416%:%
%:%1004=416%:%
%:%1005=417%:%
%:%1006=417%:%
%:%1007=418%:%
%:%1008=418%:%
%:%1009=419%:%
%:%1010=419%:%
%:%1011=420%:%
%:%1012=420%:%
%:%1013=421%:%
%:%1014=421%:%
%:%1015=422%:%
%:%1016=422%:%
%:%1017=423%:%
%:%1018=423%:%
%:%1019=424%:%
%:%1020=424%:%
%:%1021=425%:%
%:%1022=425%:%
%:%1023=426%:%
%:%1024=426%:%
%:%1025=427%:%
%:%1026=427%:%
%:%1027=427%:%
%:%1028=427%:%
%:%1029=427%:%
%:%1030=428%:%
%:%1036=428%:%
%:%1039=429%:%
%:%1040=430%:%
%:%1041=430%:%
%:%1042=431%:%
%:%1043=432%:%
%:%1044=432%:%
%:%1047=433%:%
%:%1051=433%:%
%:%1052=433%:%
%:%1053=434%:%
%:%1054=434%:%
%:%1055=435%:%
%:%1056=435%:%
%:%1061=435%:%
%:%1064=436%:%
%:%1065=437%:%
%:%1066=437%:%
%:%1067=438%:%
%:%1068=439%:%
%:%1069=440%:%
%:%1072=441%:%
%:%1073=442%:%
%:%1077=442%:%
%:%1078=442%:%
%:%1079=443%:%
%:%1080=443%:%
%:%1081=444%:%
%:%1082=444%:%
%:%1083=445%:%
%:%1084=445%:%
%:%1085=446%:%
%:%1086=446%:%
%:%1087=447%:%
%:%1088=447%:%
%:%1089=448%:%
%:%1090=448%:%
%:%1091=449%:%
%:%1092=449%:%
%:%1093=450%:%
%:%1094=450%:%
%:%1095=451%:%
%:%1096=451%:%
%:%1097=452%:%
%:%1098=452%:%
%:%1099=453%:%
%:%1100=453%:%
%:%1101=454%:%
%:%1102=454%:%
%:%1103=455%:%
%:%1104=455%:%
%:%1105=456%:%
%:%1106=456%:%
%:%1107=457%:%
%:%1108=457%:%
%:%1109=458%:%
%:%1110=458%:%
%:%1111=459%:%
%:%1112=459%:%
%:%1113=460%:%
%:%1114=460%:%
%:%1115=461%:%
%:%1116=461%:%
%:%1117=462%:%
%:%1118=462%:%
%:%1119=463%:%
%:%1120=463%:%
%:%1121=464%:%
%:%1122=464%:%
%:%1123=465%:%
%:%1124=465%:%
%:%1125=466%:%
%:%1126=466%:%
%:%1131=466%:%
%:%1134=467%:%
%:%1135=468%:%
%:%1136=468%:%
%:%1137=469%:%
%:%1138=470%:%
%:%1139=471%:%
%:%1142=472%:%
%:%1146=472%:%
%:%1147=472%:%
%:%1148=473%:%
%:%1149=473%:%
%:%1150=474%:%
%:%1151=474%:%
%:%1152=475%:%
%:%1153=475%:%
%:%1154=476%:%
%:%1155=476%:%
%:%1156=477%:%
%:%1157=477%:%
%:%1158=478%:%
%:%1159=478%:%
%:%1160=479%:%
%:%1161=479%:%
%:%1162=480%:%
%:%1163=480%:%
%:%1164=481%:%
%:%1165=481%:%
%:%1170=481%:%
%:%1173=482%:%
%:%1174=483%:%
%:%1175=483%:%
%:%1176=484%:%
%:%1177=485%:%
%:%1178=486%:%
%:%1179=487%:%
%:%1180=488%:%
%:%1181=489%:%
%:%1182=490%:%
%:%1183=491%:%
%:%1184=492%:%
%:%1187=493%:%
%:%1191=493%:%
%:%1192=493%:%
%:%1197=493%:%
%:%1200=494%:%
%:%1201=495%:%
%:%1202=495%:%
%:%1203=496%:%
%:%1204=497%:%
%:%1205=498%:%
%:%1208=499%:%
%:%1209=500%:%
%:%1213=500%:%
%:%1214=500%:%
%:%1215=501%:%
%:%1216=501%:%
%:%1217=502%:%
%:%1218=502%:%
%:%1219=503%:%
%:%1220=503%:%
%:%1221=504%:%
%:%1222=504%:%
%:%1223=505%:%
%:%1224=505%:%
%:%1229=505%:%
%:%1232=506%:%
%:%1233=507%:%
%:%1234=507%:%
%:%1235=508%:%
%:%1236=509%:%
%:%1237=510%:%
%:%1244=511%:%
%:%1245=511%:%
%:%1246=512%:%
%:%1247=512%:%
%:%1248=512%:%
%:%1249=512%:%
%:%1250=513%:%
%:%1251=513%:%
%:%1252=513%:%
%:%1253=513%:%
%:%1254=514%:%
%:%1255=515%:%
%:%1256=515%:%
%:%1257=516%:%
%:%1258=516%:%
%:%1259=517%:%
%:%1260=517%:%
%:%1261=518%:%
%:%1262=518%:%
%:%1263=519%:%
%:%1264=519%:%
%:%1265=520%:%
%:%1266=520%:%
%:%1267=521%:%
%:%1268=521%:%
%:%1269=522%:%
%:%1270=522%:%
%:%1271=523%:%
%:%1272=524%:%
%:%1273=524%:%
%:%1274=525%:%
%:%1275=525%:%
%:%1276=526%:%
%:%1277=526%:%
%:%1278=527%:%
%:%1279=527%:%
%:%1280=528%:%
%:%1281=528%:%
%:%1282=529%:%
%:%1283=529%:%
%:%1284=530%:%
%:%1285=530%:%
%:%1286=531%:%
%:%1287=531%:%
%:%1288=532%:%
%:%1289=532%:%
%:%1290=532%:%
%:%1291=533%:%
%:%1292=533%:%
%:%1293=534%:%
%:%1294=534%:%
%:%1295=535%:%
%:%1296=535%:%
%:%1297=536%:%
%:%1298=536%:%
%:%1299=537%:%
%:%1300=538%:%
%:%1301=538%:%
%:%1302=539%:%
%:%1303=540%:%
%:%1304=540%:%
%:%1305=541%:%
%:%1306=542%:%
%:%1307=543%:%
%:%1308=543%:%
%:%1310=545%:%
%:%1311=546%:%
%:%1312=547%:%
%:%1313=548%:%
%:%1314=548%:%
%:%1315=549%:%
%:%1316=549%:%
%:%1317=550%:%
%:%1318=550%:%
%:%1319=551%:%
%:%1320=551%:%
%:%1321=552%:%
%:%1322=552%:%
%:%1323=553%:%
%:%1324=553%:%
%:%1325=553%:%
%:%1326=554%:%
%:%1327=554%:%
%:%1328=555%:%
%:%1329=555%:%
%:%1330=556%:%
%:%1331=556%:%
%:%1332=557%:%
%:%1333=557%:%
%:%1334=558%:%
%:%1335=558%:%
%:%1336=559%:%
%:%1337=559%:%
%:%1338=560%:%
%:%1339=560%:%
%:%1340=561%:%
%:%1341=561%:%
%:%1342=562%:%
%:%1343=562%:%
%:%1344=563%:%
%:%1345=563%:%
%:%1346=564%:%
%:%1347=564%:%
%:%1348=565%:%
%:%1349=565%:%
%:%1350=566%:%
%:%1351=566%:%
%:%1352=567%:%
%:%1353=567%:%
%:%1354=568%:%
%:%1355=568%:%
%:%1356=569%:%
%:%1357=569%:%
%:%1358=570%:%
%:%1359=570%:%
%:%1360=570%:%
%:%1361=571%:%
%:%1362=571%:%
%:%1363=572%:%
%:%1364=572%:%
%:%1365=573%:%
%:%1366=573%:%
%:%1367=574%:%
%:%1368=574%:%
%:%1369=575%:%
%:%1370=575%:%
%:%1371=576%:%
%:%1372=576%:%
%:%1373=577%:%
%:%1374=577%:%
%:%1375=578%:%
%:%1376=578%:%
%:%1377=579%:%
%:%1378=579%:%
%:%1379=580%:%
%:%1380=580%:%
%:%1381=581%:%
%:%1382=581%:%
%:%1383=582%:%
%:%1384=582%:%
%:%1385=583%:%
%:%1386=583%:%
%:%1387=584%:%
%:%1388=584%:%
%:%1389=585%:%
%:%1390=586%:%
%:%1391=586%:%
%:%1392=586%:%
%:%1393=586%:%
%:%1394=587%:%
%:%1395=588%:%
%:%1396=588%:%
%:%1397=589%:%
%:%1398=589%:%
%:%1399=590%:%
%:%1400=590%:%
%:%1401=591%:%
%:%1402=591%:%
%:%1403=592%:%
%:%1404=592%:%
%:%1405=593%:%
%:%1406=593%:%
%:%1407=594%:%
%:%1408=594%:%
%:%1409=595%:%
%:%1410=595%:%
%:%1411=596%:%
%:%1412=596%:%
%:%1413=597%:%
%:%1414=597%:%
%:%1415=598%:%
%:%1416=598%:%
%:%1417=598%:%
%:%1418=598%:%
%:%1419=598%:%
%:%1420=599%:%
%:%1421=600%:%
%:%1422=600%:%
%:%1423=601%:%
%:%1424=601%:%
%:%1425=602%:%
%:%1426=602%:%
%:%1427=603%:%
%:%1428=603%:%
%:%1429=604%:%
%:%1430=604%:%
%:%1431=605%:%
%:%1432=605%:%
%:%1433=606%:%
%:%1434=606%:%
%:%1435=607%:%
%:%1436=607%:%
%:%1437=608%:%
%:%1438=608%:%
%:%1439=609%:%
%:%1440=609%:%
%:%1441=610%:%
%:%1442=610%:%
%:%1443=611%:%
%:%1444=611%:%
%:%1445=612%:%
%:%1446=612%:%
%:%1447=613%:%
%:%1448=613%:%
%:%1449=614%:%
%:%1450=614%:%
%:%1451=615%:%
%:%1452=615%:%
%:%1453=616%:%
%:%1454=616%:%
%:%1455=617%:%
%:%1456=617%:%
%:%1457=618%:%
%:%1458=618%:%
%:%1459=619%:%
%:%1460=619%:%
%:%1461=620%:%
%:%1462=620%:%
%:%1463=621%:%
%:%1464=621%:%
%:%1465=622%:%
%:%1466=622%:%
%:%1467=623%:%
%:%1468=623%:%
%:%1469=624%:%
%:%1470=624%:%
%:%1471=625%:%
%:%1472=625%:%
%:%1473=626%:%
%:%1474=626%:%
%:%1475=627%:%
%:%1476=627%:%
%:%1477=628%:%
%:%1478=628%:%
%:%1479=629%:%
%:%1480=629%:%
%:%1481=630%:%
%:%1482=630%:%
%:%1483=631%:%
%:%1484=631%:%
%:%1485=632%:%
%:%1486=632%:%
%:%1487=633%:%
%:%1488=633%:%
%:%1489=634%:%
%:%1490=634%:%
%:%1491=635%:%
%:%1492=635%:%
%:%1493=636%:%
%:%1494=636%:%
%:%1495=637%:%
%:%1496=637%:%
%:%1497=638%:%
%:%1498=638%:%
%:%1499=639%:%
%:%1500=639%:%
%:%1501=640%:%
%:%1502=640%:%
%:%1503=641%:%
%:%1504=641%:%
%:%1505=642%:%
%:%1506=642%:%
%:%1507=643%:%
%:%1508=643%:%
%:%1509=644%:%
%:%1510=645%:%
%:%1511=645%:%
%:%1512=645%:%
%:%1513=645%:%
%:%1514=646%:%
%:%1515=646%:%
%:%1516=646%:%
%:%1517=647%:%
%:%1518=647%:%
%:%1519=648%:%
%:%1520=648%:%
%:%1521=649%:%
%:%1522=649%:%
%:%1523=650%:%
%:%1524=650%:%
%:%1525=651%:%
%:%1531=651%:%
%:%1534=652%:%
%:%1535=653%:%
%:%1536=653%:%
%:%1543=654%:%
%:%1544=654%:%
%:%1545=655%:%
%:%1546=655%:%
%:%1547=655%:%
%:%1548=656%:%
%:%1549=656%:%
%:%1550=656%:%
%:%1551=657%:%
%:%1552=657%:%
%:%1553=658%:%
%:%1554=658%:%
%:%1555=659%:%
%:%1556=659%:%
%:%1557=660%:%
%:%1558=660%:%
%:%1559=661%:%
%:%1560=661%:%
%:%1561=662%:%
%:%1562=662%:%
%:%1563=663%:%
%:%1564=663%:%
%:%1565=663%:%
%:%1566=664%:%
%:%1567=664%:%
%:%1568=665%:%
%:%1569=665%:%
%:%1570=666%:%
%:%1576=666%:%
%:%1579=667%:%
%:%1580=668%:%
%:%1581=668%:%
%:%1582=669%:%
%:%1583=670%:%
%:%1584=671%:%
%:%1591=672%:%
%:%1592=672%:%
%:%1593=673%:%
%:%1594=673%:%
%:%1595=674%:%
%:%1596=674%:%
%:%1597=675%:%
%:%1598=675%:%
%:%1599=676%:%
%:%1600=676%:%
%:%1601=677%:%
%:%1602=677%:%
%:%1603=678%:%
%:%1604=678%:%
%:%1605=679%:%
%:%1606=680%:%
%:%1607=680%:%
%:%1608=681%:%
%:%1609=681%:%
%:%1610=682%:%
%:%1611=682%:%
%:%1612=683%:%
%:%1613=683%:%
%:%1614=684%:%
%:%1615=684%:%
%:%1616=685%:%
%:%1617=685%:%
%:%1618=686%:%
%:%1619=686%:%
%:%1620=686%:%
%:%1621=687%:%
%:%1622=687%:%
%:%1623=688%:%
%:%1624=688%:%
%:%1625=689%:%
%:%1626=689%:%
%:%1627=689%:%
%:%1628=689%:%
%:%1629=690%:%
%:%1630=690%:%
%:%1631=691%:%
%:%1632=692%:%
%:%1633=692%:%
%:%1634=693%:%
%:%1635=693%:%
%:%1636=694%:%
%:%1637=694%:%
%:%1638=695%:%
%:%1639=695%:%
%:%1640=696%:%
%:%1641=696%:%
%:%1642=697%:%
%:%1643=697%:%
%:%1644=698%:%
%:%1645=698%:%
%:%1646=699%:%
%:%1647=699%:%
%:%1648=700%:%
%:%1649=700%:%
%:%1650=701%:%
%:%1651=701%:%
%:%1652=702%:%
%:%1653=702%:%
%:%1654=703%:%
%:%1655=703%:%
%:%1656=704%:%
%:%1657=704%:%
%:%1658=705%:%
%:%1659=705%:%
%:%1660=706%:%
%:%1661=706%:%
%:%1662=707%:%
%:%1663=707%:%
%:%1664=708%:%
%:%1665=708%:%
%:%1666=709%:%
%:%1667=709%:%
%:%1668=710%:%
%:%1669=710%:%
%:%1670=711%:%
%:%1671=711%:%
%:%1672=712%:%
%:%1673=712%:%
%:%1674=713%:%
%:%1675=713%:%
%:%1676=714%:%
%:%1677=714%:%
%:%1678=715%:%
%:%1679=715%:%
%:%1680=716%:%
%:%1681=716%:%
%:%1682=717%:%
%:%1683=717%:%
%:%1684=718%:%
%:%1690=718%:%
%:%1693=719%:%
%:%1694=720%:%
%:%1695=720%:%
%:%1696=721%:%
%:%1697=722%:%
%:%1698=723%:%
%:%1701=724%:%
%:%1705=724%:%
%:%1706=724%:%
%:%1707=725%:%
%:%1708=725%:%
%:%1709=726%:%
%:%1710=726%:%
%:%1715=726%:%
%:%1718=727%:%
%:%1719=728%:%
%:%1720=728%:%
%:%1721=729%:%
%:%1725=733%:%
%:%1726=734%:%
%:%1727=735%:%
%:%1728=735%:%
%:%1729=736%:%
%:%1730=737%:%
%:%1731=738%:%
%:%1732=739%:%
%:%1733=740%:%
%:%1736=741%:%
%:%1740=741%:%
%:%1741=741%:%
%:%1742=742%:%
%:%1743=742%:%
%:%1744=743%:%
%:%1745=743%:%
%:%1746=744%:%
%:%1747=744%:%
%:%1748=745%:%
%:%1749=745%:%
%:%1750=746%:%
%:%1751=746%:%
%:%1752=747%:%
%:%1753=747%:%
%:%1754=748%:%
%:%1755=748%:%
%:%1756=749%:%
%:%1757=749%:%
%:%1758=750%:%
%:%1759=750%:%
%:%1760=751%:%
%:%1761=751%:%
%:%1762=752%:%
%:%1763=752%:%
%:%1764=753%:%
%:%1765=753%:%
%:%1766=754%:%
%:%1767=754%:%
%:%1768=755%:%
%:%1769=755%:%
%:%1770=756%:%
%:%1771=756%:%
%:%1772=757%:%
%:%1773=757%:%
%:%1774=758%:%
%:%1775=758%:%
%:%1776=759%:%
%:%1777=759%:%
%:%1778=760%:%
%:%1779=760%:%
%:%1780=761%:%
%:%1781=761%:%
%:%1782=762%:%
%:%1788=762%:%
%:%1791=763%:%
%:%1792=764%:%
%:%1793=764%:%
%:%1794=765%:%
%:%1795=766%:%
%:%1796=767%:%
%:%1799=768%:%
%:%1800=769%:%
%:%1804=769%:%
%:%1805=769%:%
%:%1806=770%:%
%:%1807=770%:%
%:%1808=771%:%
%:%1809=771%:%
%:%1810=772%:%
%:%1811=772%:%
%:%1812=773%:%
%:%1813=773%:%
%:%1814=774%:%
%:%1815=774%:%
%:%1816=775%:%
%:%1817=775%:%
%:%1818=776%:%
%:%1819=776%:%
%:%1820=777%:%
%:%1821=777%:%
%:%1822=778%:%
%:%1823=778%:%
%:%1824=779%:%
%:%1825=779%:%
%:%1826=780%:%
%:%1827=780%:%
%:%1832=780%:%
%:%1835=781%:%
%:%1836=782%:%
%:%1837=782%:%
%:%1838=783%:%
%:%1839=784%:%
%:%1840=785%:%
%:%1847=786%:%
%:%1848=786%:%
%:%1849=787%:%
%:%1850=787%:%
%:%1851=788%:%
%:%1852=788%:%
%:%1853=789%:%
%:%1854=789%:%
%:%1855=790%:%
%:%1856=791%:%
%:%1857=791%:%
%:%1858=792%:%
%:%1859=793%:%
%:%1860=793%:%
%:%1861=794%:%
%:%1862=794%:%
%:%1863=795%:%
%:%1864=795%:%
%:%1865=796%:%
%:%1866=796%:%
%:%1867=797%:%
%:%1868=797%:%
%:%1869=798%:%
%:%1870=799%:%
%:%1871=799%:%
%:%1872=800%:%
%:%1873=800%:%
%:%1874=801%:%
%:%1875=801%:%
%:%1876=802%:%
%:%1877=802%:%
%:%1878=803%:%
%:%1879=803%:%
%:%1880=803%:%
%:%1881=803%:%
%:%1882=803%:%
%:%1883=804%:%
%:%1884=804%:%
%:%1885=804%:%
%:%1886=804%:%
%:%1887=804%:%
%:%1888=805%:%
%:%1889=805%:%
%:%1890=805%:%
%:%1891=805%:%
%:%1892=805%:%
%:%1893=806%:%
%:%1894=806%:%
%:%1895=806%:%
%:%1896=806%:%
%:%1897=806%:%
%:%1898=807%:%
%:%1899=807%:%
%:%1900=807%:%
%:%1901=808%:%
%:%1902=808%:%
%:%1903=808%:%
%:%1904=809%:%
%:%1905=809%:%
%:%1906=809%:%
%:%1907=810%:%
%:%1908=810%:%
%:%1909=811%:%
%:%1910=811%:%
%:%1911=812%:%
%:%1912=812%:%
%:%1913=813%:%
%:%1914=813%:%
%:%1915=813%:%
%:%1916=814%:%
%:%1917=814%:%
%:%1918=815%:%
%:%1919=815%:%
%:%1920=816%:%
%:%1921=816%:%
%:%1922=817%:%
%:%1923=817%:%
%:%1924=818%:%
%:%1925=818%:%
%:%1926=819%:%
%:%1927=819%:%
%:%1928=819%:%
%:%1929=820%:%
%:%1930=820%:%
%:%1931=821%:%
%:%1932=821%:%
%:%1933=822%:%
%:%1934=822%:%
%:%1935=823%:%
%:%1936=823%:%
%:%1937=824%:%
%:%1938=824%:%
%:%1939=825%:%
%:%1940=825%:%
%:%1941=826%:%
%:%1942=826%:%
%:%1943=827%:%
%:%1944=827%:%
%:%1945=828%:%
%:%1946=828%:%
%:%1947=829%:%
%:%1948=829%:%
%:%1949=830%:%
%:%1950=830%:%
%:%1951=831%:%
%:%1952=832%:%
%:%1953=832%:%
%:%1954=833%:%
%:%1955=833%:%
%:%1956=834%:%
%:%1957=834%:%
%:%1958=835%:%
%:%1959=835%:%
%:%1960=836%:%
%:%1961=836%:%
%:%1962=836%:%
%:%1963=836%:%
%:%1964=836%:%
%:%1965=837%:%
%:%1966=837%:%
%:%1967=837%:%
%:%1968=837%:%
%:%1969=837%:%
%:%1970=838%:%
%:%1971=838%:%
%:%1972=838%:%
%:%1973=838%:%
%:%1974=839%:%
%:%1975=839%:%
%:%1976=839%:%
%:%1977=839%:%
%:%1978=839%:%
%:%1979=840%:%
%:%1980=840%:%
%:%1981=841%:%
%:%1982=842%:%
%:%1983=842%:%
%:%1984=843%:%
%:%1985=843%:%
%:%1986=844%:%
%:%1987=844%:%
%:%1988=845%:%
%:%1989=845%:%
%:%1990=846%:%
%:%1991=846%:%
%:%1992=847%:%
%:%1993=847%:%
%:%1994=848%:%
%:%1995=848%:%
%:%1996=849%:%
%:%1997=849%:%
%:%1998=850%:%
%:%1999=850%:%
%:%2000=851%:%
%:%2001=851%:%
%:%2002=852%:%
%:%2003=852%:%
%:%2004=853%:%
%:%2005=853%:%
%:%2006=854%:%
%:%2007=854%:%
%:%2008=855%:%
%:%2009=855%:%
%:%2010=856%:%
%:%2011=857%:%
%:%2012=857%:%
%:%2013=858%:%
%:%2014=858%:%
%:%2015=859%:%
%:%2016=859%:%
%:%2017=860%:%
%:%2018=860%:%
%:%2019=861%:%
%:%2020=861%:%
%:%2021=862%:%
%:%2022=862%:%
%:%2023=862%:%
%:%2024=863%:%
%:%2025=863%:%
%:%2026=864%:%
%:%2027=864%:%
%:%2028=864%:%
%:%2029=865%:%
%:%2030=866%:%
%:%2031=866%:%
%:%2032=866%:%
%:%2033=866%:%
%:%2034=867%:%
%:%2035=867%:%
%:%2036=867%:%
%:%2037=867%:%
%:%2038=867%:%
%:%2039=868%:%
%:%2040=868%:%
%:%2041=868%:%
%:%2042=868%:%
%:%2043=868%:%
%:%2044=869%:%
%:%2045=870%:%
%:%2046=870%:%
%:%2047=871%:%
%:%2048=871%:%
%:%2049=871%:%
%:%2050=872%:%
%:%2051=872%:%
%:%2052=872%:%
%:%2053=873%:%
%:%2054=873%:%
%:%2055=874%:%
%:%2056=874%:%
%:%2057=875%:%
%:%2058=875%:%
%:%2059=876%:%
%:%2060=876%:%
%:%2061=876%:%
%:%2062=877%:%
%:%2063=877%:%
%:%2064=878%:%
%:%2065=878%:%
%:%2066=879%:%
%:%2067=879%:%
%:%2068=880%:%
%:%2069=880%:%
%:%2070=880%:%
%:%2071=881%:%
%:%2072=881%:%
%:%2073=882%:%
%:%2074=882%:%
%:%2075=883%:%
%:%2076=883%:%
%:%2077=884%:%
%:%2078=884%:%
%:%2079=885%:%
%:%2080=885%:%
%:%2081=886%:%
%:%2082=886%:%
%:%2083=887%:%
%:%2084=887%:%
%:%2085=888%:%
%:%2086=888%:%
%:%2087=889%:%
%:%2088=889%:%
%:%2089=890%:%
%:%2090=890%:%
%:%2091=891%:%
%:%2092=892%:%
%:%2093=892%:%
%:%2094=893%:%
%:%2095=893%:%
%:%2096=894%:%
%:%2097=894%:%
%:%2098=895%:%
%:%2099=895%:%
%:%2100=896%:%
%:%2101=896%:%
%:%2102=897%:%
%:%2103=897%:%
%:%2104=897%:%
%:%2105=898%:%
%:%2106=898%:%
%:%2107=898%:%
%:%2108=898%:%
%:%2109=899%:%
%:%2110=899%:%
%:%2111=899%:%
%:%2112=899%:%
%:%2113=900%:%
%:%2114=900%:%
%:%2115=900%:%
%:%2116=900%:%
%:%2117=901%:%
%:%2118=901%:%
%:%2119=901%:%
%:%2120=901%:%
%:%2121=901%:%
%:%2122=902%:%
%:%2123=902%:%
%:%2124=902%:%
%:%2125=903%:%
%:%2126=903%:%
%:%2127=904%:%
%:%2128=904%:%
%:%2129=905%:%
%:%2130=905%:%
%:%2131=906%:%
%:%2132=906%:%
%:%2133=907%:%
%:%2134=907%:%
%:%2135=908%:%
%:%2136=908%:%
%:%2137=909%:%
%:%2138=909%:%
%:%2139=910%:%
%:%2140=910%:%
%:%2141=911%:%
%:%2142=911%:%
%:%2143=912%:%
%:%2144=912%:%
%:%2145=913%:%
%:%2146=914%:%
%:%2147=914%:%
%:%2148=915%:%
%:%2149=915%:%
%:%2150=916%:%
%:%2151=916%:%
%:%2152=917%:%
%:%2153=917%:%
%:%2154=918%:%
%:%2155=918%:%
%:%2156=919%:%
%:%2157=920%:%
%:%2158=920%:%
%:%2159=921%:%
%:%2160=921%:%
%:%2161=922%:%
%:%2162=922%:%
%:%2163=923%:%
%:%2164=923%:%
%:%2165=924%:%
%:%2166=924%:%
%:%2167=925%:%
%:%2168=926%:%
%:%2169=926%:%
%:%2170=927%:%
%:%2171=927%:%
%:%2172=928%:%
%:%2173=928%:%
%:%2174=929%:%
%:%2175=930%:%
%:%2176=930%:%
%:%2177=930%:%
%:%2178=930%:%
%:%2179=930%:%
%:%2180=931%:%
%:%2181=932%:%
%:%2182=932%:%
%:%2183=933%:%
%:%2184=933%:%
%:%2185=934%:%
%:%2186=934%:%
%:%2187=934%:%
%:%2188=935%:%
%:%2189=935%:%
%:%2190=935%:%
%:%2191=935%:%
%:%2192=936%:%
%:%2193=936%:%
%:%2194=936%:%
%:%2195=936%:%
%:%2196=936%:%
%:%2197=937%:%
%:%2198=937%:%
%:%2199=938%:%
%:%2200=939%:%
%:%2201=939%:%
%:%2202=940%:%
%:%2203=940%:%
%:%2204=940%:%
%:%2205=940%:%
%:%2206=941%:%
%:%2207=942%:%
%:%2208=942%:%
%:%2209=943%:%
%:%2210=943%:%
%:%2211=944%:%
%:%2212=944%:%
%:%2213=945%:%
%:%2214=945%:%
%:%2215=946%:%
%:%2216=946%:%
%:%2217=947%:%
%:%2218=947%:%
%:%2219=948%:%
%:%2220=948%:%
%:%2221=949%:%
%:%2222=950%:%
%:%2223=950%:%
%:%2224=950%:%
%:%2225=951%:%
%:%2226=951%:%
%:%2227=952%:%
%:%2228=952%:%
%:%2229=953%:%
%:%2230=953%:%
%:%2231=954%:%
%:%2232=955%:%
%:%2233=955%:%
%:%2234=956%:%
%:%2235=956%:%
%:%2236=957%:%
%:%2237=957%:%
%:%2238=957%:%
%:%2239=958%:%
%:%2240=958%:%
%:%2241=958%:%
%:%2242=958%:%
%:%2243=958%:%
%:%2244=959%:%
%:%2245=960%:%
%:%2246=960%:%
%:%2247=961%:%
%:%2248=961%:%
%:%2249=962%:%
%:%2250=962%:%
%:%2251=963%:%
%:%2252=963%:%
%:%2253=964%:%
%:%2254=964%:%
%:%2255=965%:%
%:%2256=966%:%
%:%2257=966%:%
%:%2258=967%:%
%:%2259=967%:%
%:%2260=968%:%
%:%2261=968%:%
%:%2262=968%:%
%:%2263=969%:%
%:%2264=969%:%
%:%2265=969%:%
%:%2266=969%:%
%:%2267=970%:%
%:%2268=970%:%
%:%2269=970%:%
%:%2270=971%:%
%:%2271=971%:%
%:%2272=972%:%
%:%2273=972%:%
%:%2274=973%:%
%:%2275=973%:%
%:%2276=974%:%
%:%2277=974%:%
%:%2278=975%:%
%:%2279=975%:%
%:%2280=976%:%
%:%2281=976%:%
%:%2282=977%:%
%:%2283=977%:%
%:%2284=977%:%
%:%2285=978%:%
%:%2286=978%:%
%:%2287=979%:%
%:%2288=979%:%
%:%2289=980%:%
%:%2290=980%:%
%:%2291=981%:%
%:%2292=981%:%
%:%2293=982%:%
%:%2294=982%:%
%:%2295=983%:%
%:%2296=983%:%
%:%2297=984%:%
%:%2298=984%:%
%:%2299=985%:%
%:%2300=985%:%
%:%2301=986%:%
%:%2302=986%:%
%:%2303=987%:%
%:%2304=987%:%
%:%2305=987%:%
%:%2306=988%:%
%:%2307=988%:%
%:%2308=989%:%
%:%2309=989%:%
%:%2310=990%:%
%:%2311=990%:%
%:%2312=991%:%
%:%2313=991%:%
%:%2314=992%:%
%:%2315=992%:%
%:%2316=993%:%
%:%2317=993%:%
%:%2318=994%:%
%:%2319=994%:%
%:%2320=994%:%
%:%2321=995%:%
%:%2322=995%:%
%:%2323=996%:%
%:%2324=996%:%
%:%2325=997%:%
%:%2326=997%:%
%:%2327=998%:%
%:%2328=998%:%
%:%2329=999%:%
%:%2330=999%:%
%:%2331=1000%:%
%:%2332=1000%:%
%:%2333=1000%:%
%:%2334=1000%:%
%:%2335=1001%:%
%:%2336=1001%:%
%:%2337=1002%:%
%:%2338=1003%:%
%:%2339=1003%:%
%:%2340=1003%:%
%:%2341=1004%:%
%:%2342=1004%:%
%:%2343=1005%:%
%:%2344=1005%:%
%:%2345=1006%:%
%:%2346=1006%:%
%:%2347=1007%:%
%:%2348=1008%:%
%:%2349=1008%:%
%:%2350=1009%:%
%:%2351=1009%:%
%:%2352=1010%:%
%:%2353=1010%:%
%:%2354=1011%:%
%:%2355=1011%:%
%:%2356=1012%:%
%:%2357=1012%:%
%:%2358=1012%:%
%:%2359=1013%:%
%:%2360=1013%:%
%:%2361=1014%:%
%:%2362=1014%:%
%:%2363=1015%:%
%:%2364=1015%:%
%:%2365=1016%:%
%:%2366=1016%:%
%:%2367=1017%:%
%:%2368=1017%:%
%:%2369=1018%:%
%:%2370=1018%:%
%:%2371=1019%:%
%:%2372=1019%:%
%:%2373=1019%:%
%:%2374=1020%:%
%:%2375=1020%:%
%:%2376=1021%:%
%:%2377=1021%:%
%:%2378=1022%:%
%:%2379=1022%:%
%:%2380=1023%:%
%:%2381=1023%:%
%:%2382=1023%:%
%:%2383=1024%:%
%:%2384=1024%:%
%:%2385=1025%:%
%:%2386=1025%:%
%:%2387=1026%:%
%:%2388=1026%:%
%:%2389=1027%:%
%:%2390=1027%:%
%:%2391=1027%:%
%:%2392=1027%:%
%:%2393=1028%:%
%:%2394=1028%:%
%:%2395=1028%:%
%:%2396=1028%:%
%:%2397=1029%:%
%:%2398=1029%:%
%:%2399=1029%:%
%:%2400=1029%:%
%:%2401=1029%:%
%:%2402=1030%:%
%:%2403=1030%:%
%:%2404=1030%:%
%:%2405=1031%:%
%:%2406=1031%:%
%:%2407=1032%:%
%:%2408=1032%:%
%:%2409=1033%:%
%:%2410=1033%:%
%:%2411=1034%:%
%:%2412=1034%:%
%:%2413=1034%:%
%:%2414=1035%:%
%:%2415=1035%:%
%:%2416=1036%:%
%:%2417=1036%:%
%:%2418=1037%:%
%:%2419=1037%:%
%:%2420=1038%:%
%:%2421=1038%:%
%:%2422=1038%:%
%:%2423=1039%:%
%:%2424=1039%:%
%:%2425=1040%:%
%:%2426=1040%:%
%:%2427=1041%:%
%:%2428=1041%:%
%:%2429=1041%:%
%:%2430=1041%:%
%:%2431=1041%:%
%:%2432=1042%:%
%:%2433=1042%:%
%:%2434=1042%:%
%:%2435=1043%:%
%:%2436=1043%:%
%:%2437=1044%:%
%:%2438=1044%:%
%:%2439=1045%:%
%:%2440=1045%:%
%:%2441=1046%:%
%:%2442=1046%:%
%:%2443=1047%:%
%:%2444=1047%:%
%:%2445=1048%:%
%:%2446=1049%:%
%:%2447=1049%:%
%:%2448=1050%:%
%:%2449=1050%:%
%:%2450=1051%:%
%:%2451=1051%:%
%:%2452=1052%:%
%:%2453=1052%:%
%:%2454=1053%:%
%:%2455=1054%:%
%:%2456=1054%:%
%:%2457=1054%:%
%:%2458=1055%:%
%:%2459=1055%:%
%:%2460=1056%:%
%:%2461=1056%:%
%:%2462=1057%:%
%:%2468=1057%:%
%:%2471=1058%:%
%:%2472=1059%:%
%:%2473=1059%:%
%:%2475=1059%:%
%:%2479=1059%:%
%:%2480=1059%:%
%:%2481=1059%:%
%:%2488=1059%:%
%:%2489=1060%:%
%:%2490=1061%:%
%:%2491=1061%:%
%:%2494=1062%:%
%:%2498=1062%:%
%:%2499=1062%:%
%:%2500=1063%:%
%:%2501=1063%:%
%:%2502=1064%:%
%:%2503=1064%:%
%:%2504=1065%:%
%:%2505=1065%:%
%:%2506=1066%:%
%:%2507=1066%:%
%:%2508=1067%:%
%:%2509=1068%:%
%:%2510=1069%:%
%:%2511=1069%:%
%:%2516=1069%:%
%:%2519=1070%:%
%:%2520=1071%:%
%:%2521=1071%:%
%:%2528=1072%:%

%
\begin{isabellebody}%
\setisabellecontext{NotAC{\isacharunderscore}{\kern0pt}Binmap}%
%
\isadelimtheory
%
\endisadelimtheory
%
\isatagtheory
\isacommand{theory}\isamarkupfalse%
\ NotAC{\isacharunderscore}{\kern0pt}Binmap\isanewline
\ \ \isakeyword{imports}\ Fn{\isacharunderscore}{\kern0pt}Perm{\isacharunderscore}{\kern0pt}Filter\ HS{\isacharunderscore}{\kern0pt}Forces\isanewline
\isakeyword{begin}%
\endisatagtheory
{\isafoldtheory}%
%
\isadelimtheory
\ \isanewline
%
\endisadelimtheory
\isanewline
\isacommand{context}\isamarkupfalse%
\ M{\isacharunderscore}{\kern0pt}ctm\ \isakeyword{begin}\ \isanewline
\isanewline
\isacommand{interpretation}\isamarkupfalse%
\ M{\isacharunderscore}{\kern0pt}symmetric{\isacharunderscore}{\kern0pt}system\ {\isachardoublequoteopen}Fn{\isachardoublequoteclose}\ {\isachardoublequoteopen}Fn{\isacharunderscore}{\kern0pt}leq{\isachardoublequoteclose}\ {\isachardoublequoteopen}{\isadigit{0}}{\isachardoublequoteclose}\ {\isachardoublequoteopen}M{\isachardoublequoteclose}\ {\isachardoublequoteopen}enum{\isachardoublequoteclose}\ {\isachardoublequoteopen}Fn{\isacharunderscore}{\kern0pt}perms{\isachardoublequoteclose}\ {\isachardoublequoteopen}Fn{\isacharunderscore}{\kern0pt}perms{\isacharunderscore}{\kern0pt}filter{\isachardoublequoteclose}\isanewline
%
\isadelimproof
\ \ %
\endisadelimproof
%
\isatagproof
\isacommand{using}\isamarkupfalse%
\ Fn{\isacharunderscore}{\kern0pt}M{\isacharunderscore}{\kern0pt}symmetric{\isacharunderscore}{\kern0pt}system\ \isacommand{by}\isamarkupfalse%
\ auto%
\endisatagproof
{\isafoldproof}%
%
\isadelimproof
\isanewline
%
\endisadelimproof
\isanewline
\isacommand{definition}\isamarkupfalse%
\ binmap{\isacharunderscore}{\kern0pt}row{\isacharprime}{\kern0pt}\ \isakeyword{where}\ {\isachardoublequoteopen}binmap{\isacharunderscore}{\kern0pt}row{\isacharprime}{\kern0pt}{\isacharparenleft}{\kern0pt}n{\isacharparenright}{\kern0pt}\ {\isasymequiv}\ {\isacharbraceleft}{\kern0pt}\ x\ {\isasymin}\ domain{\isacharparenleft}{\kern0pt}check{\isacharparenleft}{\kern0pt}nat{\isacharparenright}{\kern0pt}{\isacharparenright}{\kern0pt}\ {\isasymtimes}\ Fn{\isachardot}{\kern0pt}\ {\isasymexists}F\ {\isasymin}\ Fn{\isachardot}{\kern0pt}\ {\isasymexists}m\ {\isasymin}\ nat{\isachardot}{\kern0pt}\ x\ {\isacharequal}{\kern0pt}\ {\isacharless}{\kern0pt}check{\isacharparenleft}{\kern0pt}m{\isacharparenright}{\kern0pt}{\isacharcomma}{\kern0pt}\ F{\isachargreater}{\kern0pt}\ {\isasymand}\ F{\isacharbackquote}{\kern0pt}{\isacharless}{\kern0pt}n{\isacharcomma}{\kern0pt}\ m{\isachargreater}{\kern0pt}\ {\isacharequal}{\kern0pt}\ {\isadigit{1}}\ {\isacharbraceright}{\kern0pt}{\isachardoublequoteclose}\ \isanewline
\isacommand{definition}\isamarkupfalse%
\ binmap{\isacharunderscore}{\kern0pt}row\ \isakeyword{where}\ {\isachardoublequoteopen}binmap{\isacharunderscore}{\kern0pt}row{\isacharparenleft}{\kern0pt}G{\isacharcomma}{\kern0pt}\ n{\isacharparenright}{\kern0pt}\ {\isasymequiv}\ {\isacharbraceleft}{\kern0pt}\ m\ {\isasymin}\ nat{\isachardot}{\kern0pt}\ {\isasymexists}p\ {\isasymin}\ G{\isachardot}{\kern0pt}\ p{\isacharbackquote}{\kern0pt}{\isacharless}{\kern0pt}n{\isacharcomma}{\kern0pt}\ m{\isachargreater}{\kern0pt}\ {\isacharequal}{\kern0pt}\ {\isadigit{1}}\ {\isacharbraceright}{\kern0pt}{\isachardoublequoteclose}\ \isanewline
\isacommand{definition}\isamarkupfalse%
\ binmap{\isacharprime}{\kern0pt}\ \isakeyword{where}\ {\isachardoublequoteopen}binmap{\isacharprime}{\kern0pt}\ {\isasymequiv}\ {\isacharbraceleft}{\kern0pt}\ binmap{\isacharunderscore}{\kern0pt}row{\isacharprime}{\kern0pt}{\isacharparenleft}{\kern0pt}n{\isacharparenright}{\kern0pt}{\isachardot}{\kern0pt}\ n\ {\isasymin}\ nat\ {\isacharbraceright}{\kern0pt}\ {\isasymtimes}\ {\isacharbraceleft}{\kern0pt}\ {\isadigit{0}}\ {\isacharbraceright}{\kern0pt}{\isachardoublequoteclose}\ \isanewline
\isacommand{definition}\isamarkupfalse%
\ binmap\ \isakeyword{where}\ {\isachardoublequoteopen}binmap{\isacharparenleft}{\kern0pt}G{\isacharparenright}{\kern0pt}\ {\isasymequiv}\ {\isacharbraceleft}{\kern0pt}\ binmap{\isacharunderscore}{\kern0pt}row{\isacharparenleft}{\kern0pt}G{\isacharcomma}{\kern0pt}\ n{\isacharparenright}{\kern0pt}{\isachardot}{\kern0pt}\ n\ {\isasymin}\ nat\ {\isacharbraceright}{\kern0pt}{\isachardoublequoteclose}\ \isanewline
\isanewline
\isanewline
\isacommand{lemma}\isamarkupfalse%
\ binmap{\isacharunderscore}{\kern0pt}row{\isacharprime}{\kern0pt}{\isacharunderscore}{\kern0pt}eq\ {\isacharcolon}{\kern0pt}\ \isanewline
\ \ \isakeyword{fixes}\ n\ f\ \isanewline
\ \ \isakeyword{assumes}\ {\isachardoublequoteopen}n\ {\isasymin}\ nat{\isachardoublequoteclose}\ {\isachardoublequoteopen}f\ {\isasymin}\ nat{\isacharunderscore}{\kern0pt}perms{\isachardoublequoteclose}\ \isanewline
\ \ \isakeyword{shows}\ {\isachardoublequoteopen}binmap{\isacharunderscore}{\kern0pt}row{\isacharprime}{\kern0pt}{\isacharparenleft}{\kern0pt}n{\isacharparenright}{\kern0pt}\ {\isacharequal}{\kern0pt}\ {\isacharbraceleft}{\kern0pt}\ {\isacharless}{\kern0pt}check{\isacharparenleft}{\kern0pt}m{\isacharparenright}{\kern0pt}{\isacharcomma}{\kern0pt}\ F{\isachargreater}{\kern0pt}{\isachardot}{\kern0pt}{\isachardot}{\kern0pt}\ {\isacharless}{\kern0pt}m{\isacharcomma}{\kern0pt}\ F{\isachargreater}{\kern0pt}\ {\isasymin}\ nat\ {\isasymtimes}\ Fn{\isacharcomma}{\kern0pt}\ F{\isacharbackquote}{\kern0pt}{\isacharless}{\kern0pt}n{\isacharcomma}{\kern0pt}\ m{\isachargreater}{\kern0pt}\ {\isacharequal}{\kern0pt}\ {\isadigit{1}}\ {\isacharbraceright}{\kern0pt}{\isachardoublequoteclose}\ {\isacharparenleft}{\kern0pt}\isakeyword{is}\ {\isachardoublequoteopen}{\isacharunderscore}{\kern0pt}\ {\isacharequal}{\kern0pt}\ {\isacharquery}{\kern0pt}A{\isachardoublequoteclose}{\isacharparenright}{\kern0pt}\isanewline
%
\isadelimproof
\isanewline
%
\endisadelimproof
%
\isatagproof
\isacommand{proof}\isamarkupfalse%
{\isacharparenleft}{\kern0pt}rule\ equality{\isacharunderscore}{\kern0pt}iffI{\isacharcomma}{\kern0pt}\ rule\ iffI{\isacharparenright}{\kern0pt}\isanewline
\ \ \isacommand{fix}\isamarkupfalse%
\ v\isanewline
\ \ \isacommand{assume}\isamarkupfalse%
\ vin\ {\isacharcolon}{\kern0pt}\ {\isachardoublequoteopen}v\ {\isasymin}\ binmap{\isacharunderscore}{\kern0pt}row{\isacharprime}{\kern0pt}{\isacharparenleft}{\kern0pt}n{\isacharparenright}{\kern0pt}{\isachardoublequoteclose}\ \isanewline
\ \ \isacommand{then}\isamarkupfalse%
\ \isacommand{obtain}\isamarkupfalse%
\ m\ F\ \isakeyword{where}\ {\isachardoublequoteopen}v\ {\isacharequal}{\kern0pt}\ {\isacharless}{\kern0pt}check{\isacharparenleft}{\kern0pt}m{\isacharparenright}{\kern0pt}{\isacharcomma}{\kern0pt}\ F{\isachargreater}{\kern0pt}{\isachardoublequoteclose}\ {\isachardoublequoteopen}m\ {\isasymin}\ nat{\isachardoublequoteclose}\ {\isachardoublequoteopen}F\ {\isasymin}\ Fn{\isachardoublequoteclose}\ {\isachardoublequoteopen}F{\isacharbackquote}{\kern0pt}{\isacharless}{\kern0pt}n{\isacharcomma}{\kern0pt}\ m{\isachargreater}{\kern0pt}\ {\isacharequal}{\kern0pt}\ {\isadigit{1}}{\isachardoublequoteclose}\ \isanewline
\ \ \ \ \isacommand{using}\isamarkupfalse%
\ binmap{\isacharunderscore}{\kern0pt}row{\isacharprime}{\kern0pt}{\isacharunderscore}{\kern0pt}def\ \isanewline
\ \ \ \ \isacommand{by}\isamarkupfalse%
\ force\ \isanewline
\ \ \isacommand{then}\isamarkupfalse%
\ \isacommand{show}\isamarkupfalse%
\ {\isachardoublequoteopen}v\ {\isasymin}\ {\isacharquery}{\kern0pt}A{\isachardoublequoteclose}\ \isanewline
\ \ \ \ \isacommand{by}\isamarkupfalse%
\ auto\isanewline
\isacommand{next}\isamarkupfalse%
\ \isanewline
\ \ \isacommand{fix}\isamarkupfalse%
\ v\ \isanewline
\ \ \isacommand{assume}\isamarkupfalse%
\ vin\ {\isacharcolon}{\kern0pt}\ {\isachardoublequoteopen}v\ {\isasymin}\ {\isacharbraceleft}{\kern0pt}\ {\isacharless}{\kern0pt}check{\isacharparenleft}{\kern0pt}m{\isacharparenright}{\kern0pt}{\isacharcomma}{\kern0pt}\ F{\isachargreater}{\kern0pt}{\isachardot}{\kern0pt}{\isachardot}{\kern0pt}\ {\isacharless}{\kern0pt}m{\isacharcomma}{\kern0pt}\ F{\isachargreater}{\kern0pt}\ {\isasymin}\ nat\ {\isasymtimes}\ Fn{\isacharcomma}{\kern0pt}\ F{\isacharbackquote}{\kern0pt}{\isacharless}{\kern0pt}n{\isacharcomma}{\kern0pt}\ m{\isachargreater}{\kern0pt}\ {\isacharequal}{\kern0pt}\ {\isadigit{1}}\ {\isacharbraceright}{\kern0pt}{\isachardoublequoteclose}\ \isanewline
\ \ \isacommand{then}\isamarkupfalse%
\ \isacommand{obtain}\isamarkupfalse%
\ m\ F\ \isakeyword{where}\ mFH{\isacharcolon}{\kern0pt}\ {\isachardoublequoteopen}v\ {\isacharequal}{\kern0pt}\ {\isacharless}{\kern0pt}check{\isacharparenleft}{\kern0pt}m{\isacharparenright}{\kern0pt}{\isacharcomma}{\kern0pt}\ F{\isachargreater}{\kern0pt}{\isachardoublequoteclose}\ {\isachardoublequoteopen}m\ {\isasymin}\ nat{\isachardoublequoteclose}\ {\isachardoublequoteopen}F\ {\isasymin}\ Fn{\isachardoublequoteclose}\ {\isachardoublequoteopen}F{\isacharbackquote}{\kern0pt}{\isacharless}{\kern0pt}n{\isacharcomma}{\kern0pt}\ m{\isachargreater}{\kern0pt}\ {\isacharequal}{\kern0pt}\ {\isadigit{1}}{\isachardoublequoteclose}\ \isanewline
\ \ \ \ \isacommand{using}\isamarkupfalse%
\ binmap{\isacharunderscore}{\kern0pt}row{\isacharprime}{\kern0pt}{\isacharunderscore}{\kern0pt}def\ \isanewline
\ \ \ \ \isacommand{by}\isamarkupfalse%
\ force\ \isanewline
\ \ \isacommand{then}\isamarkupfalse%
\ \isacommand{have}\isamarkupfalse%
\ {\isachardoublequoteopen}check{\isacharparenleft}{\kern0pt}m{\isacharparenright}{\kern0pt}\ {\isasymin}\ domain{\isacharparenleft}{\kern0pt}check{\isacharparenleft}{\kern0pt}nat{\isacharparenright}{\kern0pt}{\isacharparenright}{\kern0pt}{\isachardoublequoteclose}\ \isanewline
\ \ \ \ \isacommand{by}\isamarkupfalse%
{\isacharparenleft}{\kern0pt}subst\ {\isacharparenleft}{\kern0pt}{\isadigit{2}}{\isacharparenright}{\kern0pt}\ def{\isacharunderscore}{\kern0pt}check{\isacharcomma}{\kern0pt}\ force{\isacharparenright}{\kern0pt}\isanewline
\ \ \isacommand{then}\isamarkupfalse%
\ \isacommand{show}\isamarkupfalse%
\ {\isachardoublequoteopen}v\ {\isasymin}\ binmap{\isacharunderscore}{\kern0pt}row{\isacharprime}{\kern0pt}{\isacharparenleft}{\kern0pt}n{\isacharparenright}{\kern0pt}{\isachardoublequoteclose}\ \isanewline
\ \ \ \ \isacommand{using}\isamarkupfalse%
\ binmap{\isacharunderscore}{\kern0pt}row{\isacharprime}{\kern0pt}{\isacharunderscore}{\kern0pt}def\ mFH\ \isanewline
\ \ \ \ \isacommand{by}\isamarkupfalse%
\ force\ \isanewline
\isacommand{qed}\isamarkupfalse%
%
\endisatagproof
{\isafoldproof}%
%
\isadelimproof
\isanewline
%
\endisadelimproof
\isanewline
\isacommand{definition}\isamarkupfalse%
\ binmap{\isacharunderscore}{\kern0pt}row{\isacharprime}{\kern0pt}{\isacharunderscore}{\kern0pt}member{\isacharunderscore}{\kern0pt}fm\ \isakeyword{where}\ \isanewline
\ \ {\isachardoublequoteopen}binmap{\isacharunderscore}{\kern0pt}row{\isacharprime}{\kern0pt}{\isacharunderscore}{\kern0pt}member{\isacharunderscore}{\kern0pt}fm{\isacharparenleft}{\kern0pt}x{\isacharcomma}{\kern0pt}\ n{\isacharcomma}{\kern0pt}\ fn{\isacharcomma}{\kern0pt}\ N{\isacharparenright}{\kern0pt}\ {\isasymequiv}\ \isanewline
\ \ \ \ Exists{\isacharparenleft}{\kern0pt}Exists{\isacharparenleft}{\kern0pt}Exists{\isacharparenleft}{\kern0pt}Exists{\isacharparenleft}{\kern0pt}Exists{\isacharparenleft}{\kern0pt}Exists{\isacharparenleft}{\kern0pt}\isanewline
\ \ \ \ \ \ And{\isacharparenleft}{\kern0pt}empty{\isacharunderscore}{\kern0pt}fm{\isacharparenleft}{\kern0pt}{\isadigit{0}}{\isacharparenright}{\kern0pt}{\isacharcomma}{\kern0pt}\ \isanewline
\ \ \ \ \ \ And{\isacharparenleft}{\kern0pt}Member{\isacharparenleft}{\kern0pt}{\isadigit{1}}{\isacharcomma}{\kern0pt}\ fn\ {\isacharhash}{\kern0pt}{\isacharplus}{\kern0pt}\ {\isadigit{6}}{\isacharparenright}{\kern0pt}{\isacharcomma}{\kern0pt}\ \isanewline
\ \ \ \ \ \ And{\isacharparenleft}{\kern0pt}Member{\isacharparenleft}{\kern0pt}{\isadigit{2}}{\isacharcomma}{\kern0pt}\ N\ {\isacharhash}{\kern0pt}{\isacharplus}{\kern0pt}\ {\isadigit{6}}{\isacharparenright}{\kern0pt}{\isacharcomma}{\kern0pt}\ \isanewline
\ \ \ \ \ \ And{\isacharparenleft}{\kern0pt}check{\isacharunderscore}{\kern0pt}fm{\isacharparenleft}{\kern0pt}{\isadigit{2}}{\isacharcomma}{\kern0pt}\ {\isadigit{0}}{\isacharcomma}{\kern0pt}\ {\isadigit{3}}{\isacharparenright}{\kern0pt}{\isacharcomma}{\kern0pt}\ \isanewline
\ \ \ \ \ \ And{\isacharparenleft}{\kern0pt}pair{\isacharunderscore}{\kern0pt}fm{\isacharparenleft}{\kern0pt}{\isadigit{3}}{\isacharcomma}{\kern0pt}\ {\isadigit{1}}{\isacharcomma}{\kern0pt}\ x{\isacharhash}{\kern0pt}{\isacharplus}{\kern0pt}{\isadigit{6}}{\isacharparenright}{\kern0pt}{\isacharcomma}{\kern0pt}\ \isanewline
\ \ \ \ \ \ And{\isacharparenleft}{\kern0pt}pair{\isacharunderscore}{\kern0pt}fm{\isacharparenleft}{\kern0pt}n{\isacharhash}{\kern0pt}{\isacharplus}{\kern0pt}{\isadigit{6}}{\isacharcomma}{\kern0pt}\ {\isadigit{2}}{\isacharcomma}{\kern0pt}\ {\isadigit{4}}{\isacharparenright}{\kern0pt}{\isacharcomma}{\kern0pt}\ \isanewline
\ \ \ \ \ \ And{\isacharparenleft}{\kern0pt}fun{\isacharunderscore}{\kern0pt}apply{\isacharunderscore}{\kern0pt}fm{\isacharparenleft}{\kern0pt}{\isadigit{1}}{\isacharcomma}{\kern0pt}\ {\isadigit{4}}{\isacharcomma}{\kern0pt}\ {\isadigit{5}}{\isacharparenright}{\kern0pt}{\isacharcomma}{\kern0pt}\ \isanewline
\ \ \ \ \ \ \ \ \ \ is{\isacharunderscore}{\kern0pt}{\isadigit{1}}{\isacharunderscore}{\kern0pt}fm{\isacharparenleft}{\kern0pt}{\isadigit{5}}{\isacharparenright}{\kern0pt}{\isacharparenright}{\kern0pt}{\isacharparenright}{\kern0pt}{\isacharparenright}{\kern0pt}{\isacharparenright}{\kern0pt}{\isacharparenright}{\kern0pt}{\isacharparenright}{\kern0pt}{\isacharparenright}{\kern0pt}{\isacharparenright}{\kern0pt}{\isacharparenright}{\kern0pt}{\isacharparenright}{\kern0pt}{\isacharparenright}{\kern0pt}{\isacharparenright}{\kern0pt}{\isacharparenright}{\kern0pt}{\isachardoublequoteclose}\ \isanewline
\isanewline
\isacommand{lemma}\isamarkupfalse%
\ binmap{\isacharunderscore}{\kern0pt}row{\isacharprime}{\kern0pt}{\isacharunderscore}{\kern0pt}member{\isacharunderscore}{\kern0pt}fm{\isacharunderscore}{\kern0pt}type\ {\isacharcolon}{\kern0pt}\ \isanewline
\ \ \isakeyword{fixes}\ i\ j\ k\ l\ \isanewline
\ \ \isakeyword{assumes}\ {\isachardoublequoteopen}i\ {\isasymin}\ nat{\isachardoublequoteclose}\ {\isachardoublequoteopen}j\ {\isasymin}\ nat{\isachardoublequoteclose}\ {\isachardoublequoteopen}k\ {\isasymin}\ nat{\isachardoublequoteclose}\ {\isachardoublequoteopen}l\ {\isasymin}\ nat{\isachardoublequoteclose}\ \isanewline
\ \ \isakeyword{shows}\ {\isachardoublequoteopen}binmap{\isacharunderscore}{\kern0pt}row{\isacharprime}{\kern0pt}{\isacharunderscore}{\kern0pt}member{\isacharunderscore}{\kern0pt}fm{\isacharparenleft}{\kern0pt}i{\isacharcomma}{\kern0pt}\ j{\isacharcomma}{\kern0pt}\ k{\isacharcomma}{\kern0pt}\ l{\isacharparenright}{\kern0pt}\ {\isasymin}\ formula{\isachardoublequoteclose}\ \isanewline
%
\isadelimproof
\ \ \isanewline
\ \ %
\endisadelimproof
%
\isatagproof
\isacommand{apply}\isamarkupfalse%
{\isacharparenleft}{\kern0pt}subst\ binmap{\isacharunderscore}{\kern0pt}row{\isacharprime}{\kern0pt}{\isacharunderscore}{\kern0pt}member{\isacharunderscore}{\kern0pt}fm{\isacharunderscore}{\kern0pt}def{\isacharparenright}{\kern0pt}\isanewline
\ \ \isacommand{apply}\isamarkupfalse%
{\isacharparenleft}{\kern0pt}subgoal{\isacharunderscore}{\kern0pt}tac\ {\isachardoublequoteopen}check{\isacharunderscore}{\kern0pt}fm{\isacharparenleft}{\kern0pt}{\isadigit{2}}{\isacharcomma}{\kern0pt}\ {\isadigit{0}}{\isacharcomma}{\kern0pt}\ {\isadigit{3}}{\isacharparenright}{\kern0pt}\ {\isasymin}\ formula\ {\isasymand}\ is{\isacharunderscore}{\kern0pt}{\isadigit{1}}{\isacharunderscore}{\kern0pt}fm{\isacharparenleft}{\kern0pt}{\isadigit{5}}{\isacharparenright}{\kern0pt}\ {\isasymin}\ formula{\isachardoublequoteclose}{\isacharcomma}{\kern0pt}\ force{\isacharparenright}{\kern0pt}\isanewline
\ \ \isacommand{apply}\isamarkupfalse%
{\isacharparenleft}{\kern0pt}rule\ conjI{\isacharcomma}{\kern0pt}\ rule\ check{\isacharunderscore}{\kern0pt}fm{\isacharunderscore}{\kern0pt}type{\isacharparenright}{\kern0pt}\isanewline
\ \ \ \ \ \isacommand{apply}\isamarkupfalse%
\ auto{\isacharbrackleft}{\kern0pt}{\isadigit{3}}{\isacharbrackright}{\kern0pt}\isanewline
\ \ \isacommand{apply}\isamarkupfalse%
{\isacharparenleft}{\kern0pt}rule\ is{\isacharunderscore}{\kern0pt}{\isadigit{1}}{\isacharunderscore}{\kern0pt}fm{\isacharunderscore}{\kern0pt}type{\isacharcomma}{\kern0pt}\ force{\isacharparenright}{\kern0pt}\isanewline
\ \ \isacommand{done}\isamarkupfalse%
%
\endisatagproof
{\isafoldproof}%
%
\isadelimproof
\isanewline
%
\endisadelimproof
\isanewline
\isacommand{lemma}\isamarkupfalse%
\ arity{\isacharunderscore}{\kern0pt}binmap{\isacharunderscore}{\kern0pt}row{\isacharprime}{\kern0pt}{\isacharunderscore}{\kern0pt}member{\isacharunderscore}{\kern0pt}fm\ {\isacharcolon}{\kern0pt}\ \isanewline
\ \ \isakeyword{fixes}\ i\ j\ k\ l\ \isanewline
\ \ \isakeyword{assumes}\ {\isachardoublequoteopen}i\ {\isasymin}\ nat{\isachardoublequoteclose}\ {\isachardoublequoteopen}j\ {\isasymin}\ nat{\isachardoublequoteclose}\ {\isachardoublequoteopen}k\ {\isasymin}\ nat{\isachardoublequoteclose}\ {\isachardoublequoteopen}l\ {\isasymin}\ nat{\isachardoublequoteclose}\ \isanewline
\ \ \isakeyword{shows}\ {\isachardoublequoteopen}arity{\isacharparenleft}{\kern0pt}binmap{\isacharunderscore}{\kern0pt}row{\isacharprime}{\kern0pt}{\isacharunderscore}{\kern0pt}member{\isacharunderscore}{\kern0pt}fm{\isacharparenleft}{\kern0pt}i{\isacharcomma}{\kern0pt}\ j{\isacharcomma}{\kern0pt}\ k{\isacharcomma}{\kern0pt}\ l{\isacharparenright}{\kern0pt}{\isacharparenright}{\kern0pt}\ {\isasymle}\ succ{\isacharparenleft}{\kern0pt}i{\isacharparenright}{\kern0pt}\ {\isasymunion}\ succ{\isacharparenleft}{\kern0pt}j{\isacharparenright}{\kern0pt}\ {\isasymunion}\ succ{\isacharparenleft}{\kern0pt}k{\isacharparenright}{\kern0pt}\ {\isasymunion}\ succ{\isacharparenleft}{\kern0pt}l{\isacharparenright}{\kern0pt}{\isachardoublequoteclose}\ \isanewline
%
\isadelimproof
\isanewline
\ \ %
\endisadelimproof
%
\isatagproof
\isacommand{unfolding}\isamarkupfalse%
\ binmap{\isacharunderscore}{\kern0pt}row{\isacharprime}{\kern0pt}{\isacharunderscore}{\kern0pt}member{\isacharunderscore}{\kern0pt}fm{\isacharunderscore}{\kern0pt}def\ check{\isacharunderscore}{\kern0pt}fm{\isacharunderscore}{\kern0pt}def\ rcheck{\isacharunderscore}{\kern0pt}fm{\isacharunderscore}{\kern0pt}def\isanewline
\ \ \ \isacommand{apply}\isamarkupfalse%
\ simp\isanewline
\ \ \ \isacommand{apply}\isamarkupfalse%
{\isacharparenleft}{\kern0pt}subst\ arity{\isacharunderscore}{\kern0pt}tran{\isacharunderscore}{\kern0pt}closure{\isacharunderscore}{\kern0pt}fm{\isacharcomma}{\kern0pt}\ simp{\isacharcomma}{\kern0pt}\ simp{\isacharparenright}{\kern0pt}\isanewline
\ \ \isacommand{apply}\isamarkupfalse%
{\isacharparenleft}{\kern0pt}subst\ arity{\isacharunderscore}{\kern0pt}Memrel{\isacharunderscore}{\kern0pt}fm{\isacharcomma}{\kern0pt}\ simp{\isacharcomma}{\kern0pt}\ simp{\isacharparenright}{\kern0pt}\isanewline
\ \ \ \isacommand{apply}\isamarkupfalse%
{\isacharparenleft}{\kern0pt}subst\ arity{\isacharunderscore}{\kern0pt}empty{\isacharunderscore}{\kern0pt}fm{\isacharcomma}{\kern0pt}\ simp{\isacharparenright}{\kern0pt}\isanewline
\ \ \ \isacommand{apply}\isamarkupfalse%
{\isacharparenleft}{\kern0pt}subst\ arity{\isacharunderscore}{\kern0pt}singleton{\isacharunderscore}{\kern0pt}fm{\isacharcomma}{\kern0pt}\ simp{\isacharcomma}{\kern0pt}\ simp{\isacharparenright}{\kern0pt}\isanewline
\ \ \ \isacommand{apply}\isamarkupfalse%
{\isacharparenleft}{\kern0pt}subst\ arity{\isacharunderscore}{\kern0pt}is{\isacharunderscore}{\kern0pt}eclose{\isacharunderscore}{\kern0pt}fm{\isacharcomma}{\kern0pt}\ simp{\isacharcomma}{\kern0pt}\ simp{\isacharparenright}{\kern0pt}\isanewline
\ \ \ \isacommand{apply}\isamarkupfalse%
{\isacharparenleft}{\kern0pt}subst\ arity{\isacharunderscore}{\kern0pt}is{\isacharunderscore}{\kern0pt}wfrec{\isacharunderscore}{\kern0pt}fm\ {\isacharbrackleft}{\kern0pt}\isakeyword{where}\ i{\isacharequal}{\kern0pt}{\isadigit{7}}{\isacharbrackright}{\kern0pt}{\isacharcomma}{\kern0pt}\ simp{\isacharcomma}{\kern0pt}\ simp{\isacharcomma}{\kern0pt}\ simp{\isacharcomma}{\kern0pt}\ simp{\isacharcomma}{\kern0pt}\ simp{\isacharparenright}{\kern0pt}\isanewline
\ \ \isacommand{unfolding}\isamarkupfalse%
\ is{\isacharunderscore}{\kern0pt}Hcheck{\isacharunderscore}{\kern0pt}fm{\isacharunderscore}{\kern0pt}def\ Replace{\isacharunderscore}{\kern0pt}fm{\isacharunderscore}{\kern0pt}def\ PHcheck{\isacharunderscore}{\kern0pt}fm{\isacharunderscore}{\kern0pt}def\isanewline
\ \ \ \ \isacommand{apply}\isamarkupfalse%
\ simp\isanewline
\ \ \ \isacommand{apply}\isamarkupfalse%
\ {\isacharparenleft}{\kern0pt}simp\ del{\isacharcolon}{\kern0pt}FOL{\isacharunderscore}{\kern0pt}sats{\isacharunderscore}{\kern0pt}iff\ pair{\isacharunderscore}{\kern0pt}abs\ add{\isacharcolon}{\kern0pt}\ fm{\isacharunderscore}{\kern0pt}defs\ nat{\isacharunderscore}{\kern0pt}simp{\isacharunderscore}{\kern0pt}union{\isacharparenright}{\kern0pt}\isanewline
\ \ \isacommand{unfolding}\isamarkupfalse%
\ is{\isacharunderscore}{\kern0pt}{\isadigit{1}}{\isacharunderscore}{\kern0pt}fm{\isacharunderscore}{\kern0pt}def\isanewline
\ \ \isacommand{apply}\isamarkupfalse%
\ simp\isanewline
\ \ \isacommand{apply}\isamarkupfalse%
{\isacharparenleft}{\kern0pt}subst\ arity{\isacharunderscore}{\kern0pt}empty{\isacharunderscore}{\kern0pt}fm{\isacharcomma}{\kern0pt}\ simp\ add{\isacharcolon}{\kern0pt}assms{\isacharparenright}{\kern0pt}\isanewline
\ \ \isacommand{apply}\isamarkupfalse%
{\isacharparenleft}{\kern0pt}subst\ arity{\isacharunderscore}{\kern0pt}fun{\isacharunderscore}{\kern0pt}apply{\isacharunderscore}{\kern0pt}fm{\isacharcomma}{\kern0pt}\ simp{\isacharcomma}{\kern0pt}\ simp{\isacharcomma}{\kern0pt}\ simp{\isacharparenright}{\kern0pt}\isanewline
\ \ \isacommand{apply}\isamarkupfalse%
{\isacharparenleft}{\kern0pt}insert\ assms{\isacharparenright}{\kern0pt}\isanewline
\ \ \isacommand{apply}\isamarkupfalse%
{\isacharparenleft}{\kern0pt}rule\ pred{\isacharunderscore}{\kern0pt}le{\isacharcomma}{\kern0pt}\ simp{\isacharcomma}{\kern0pt}\ simp{\isacharparenright}{\kern0pt}{\isacharplus}{\kern0pt}\ \isanewline
\ \ \isacommand{apply}\isamarkupfalse%
{\isacharparenleft}{\kern0pt}subst\ succ{\isacharunderscore}{\kern0pt}Un{\isacharunderscore}{\kern0pt}distrib{\isacharcomma}{\kern0pt}\ simp{\isacharcomma}{\kern0pt}\ simp{\isacharparenright}{\kern0pt}{\isacharplus}{\kern0pt}\isanewline
\ \ \isacommand{apply}\isamarkupfalse%
{\isacharparenleft}{\kern0pt}rule\ Un{\isacharunderscore}{\kern0pt}least{\isacharunderscore}{\kern0pt}lt{\isacharcomma}{\kern0pt}\ rule\ Un{\isacharunderscore}{\kern0pt}upper{\isadigit{2}}{\isacharunderscore}{\kern0pt}lt{\isacharcomma}{\kern0pt}\ simp{\isacharcomma}{\kern0pt}\ simp{\isacharparenright}{\kern0pt}\isanewline
\ \ \isacommand{apply}\isamarkupfalse%
{\isacharparenleft}{\kern0pt}rule\ Un{\isacharunderscore}{\kern0pt}least{\isacharunderscore}{\kern0pt}lt{\isacharparenright}{\kern0pt}{\isacharplus}{\kern0pt}\isanewline
\ \ \ \ \isacommand{apply}\isamarkupfalse%
{\isacharparenleft}{\kern0pt}rule\ Un{\isacharunderscore}{\kern0pt}upper{\isadigit{2}}{\isacharunderscore}{\kern0pt}lt{\isacharcomma}{\kern0pt}\ simp{\isacharcomma}{\kern0pt}\ simp{\isacharparenright}{\kern0pt}\isanewline
\ \ \ \isacommand{apply}\isamarkupfalse%
{\isacharparenleft}{\kern0pt}rule\ Un{\isacharunderscore}{\kern0pt}upper{\isadigit{1}}{\isacharunderscore}{\kern0pt}lt{\isacharcomma}{\kern0pt}\ rule\ Un{\isacharunderscore}{\kern0pt}upper{\isadigit{2}}{\isacharunderscore}{\kern0pt}lt{\isacharcomma}{\kern0pt}\ simp{\isacharcomma}{\kern0pt}\ simp{\isacharcomma}{\kern0pt}\ simp{\isacharparenright}{\kern0pt}\isanewline
\ \ \isacommand{apply}\isamarkupfalse%
{\isacharparenleft}{\kern0pt}rule\ Un{\isacharunderscore}{\kern0pt}least{\isacharunderscore}{\kern0pt}lt{\isacharparenright}{\kern0pt}{\isacharplus}{\kern0pt}\isanewline
\ \ \ \ \isacommand{apply}\isamarkupfalse%
{\isacharparenleft}{\kern0pt}rule\ Un{\isacharunderscore}{\kern0pt}upper{\isadigit{2}}{\isacharunderscore}{\kern0pt}lt{\isacharcomma}{\kern0pt}\ simp{\isacharcomma}{\kern0pt}\ simp{\isacharparenright}{\kern0pt}{\isacharplus}{\kern0pt}\isanewline
\ \ \isacommand{apply}\isamarkupfalse%
\ {\isacharparenleft}{\kern0pt}simp\ del{\isacharcolon}{\kern0pt}FOL{\isacharunderscore}{\kern0pt}sats{\isacharunderscore}{\kern0pt}iff\ pair{\isacharunderscore}{\kern0pt}abs\ add{\isacharcolon}{\kern0pt}\ fm{\isacharunderscore}{\kern0pt}defs\ nat{\isacharunderscore}{\kern0pt}simp{\isacharunderscore}{\kern0pt}union{\isacharparenright}{\kern0pt}\isanewline
\ \ \isacommand{using}\isamarkupfalse%
\ le{\isacharunderscore}{\kern0pt}trans\ \isanewline
\ \ \ \isacommand{apply}\isamarkupfalse%
\ force\isanewline
\ \ \isacommand{done}\isamarkupfalse%
%
\endisatagproof
{\isafoldproof}%
%
\isadelimproof
\isanewline
%
\endisadelimproof
\isanewline
\isacommand{lemma}\isamarkupfalse%
\ sats{\isacharunderscore}{\kern0pt}binmap{\isacharunderscore}{\kern0pt}row{\isacharprime}{\kern0pt}{\isacharunderscore}{\kern0pt}member{\isacharunderscore}{\kern0pt}fm{\isacharunderscore}{\kern0pt}iff\ {\isacharcolon}{\kern0pt}\ \isanewline
\ \ \isakeyword{fixes}\ i\ j\ k\ l\ x\ n\ env\isanewline
\ \ \isakeyword{assumes}\ {\isachardoublequoteopen}i\ {\isacharless}{\kern0pt}\ length{\isacharparenleft}{\kern0pt}env{\isacharparenright}{\kern0pt}{\isachardoublequoteclose}\ {\isachardoublequoteopen}j\ {\isacharless}{\kern0pt}\ length{\isacharparenleft}{\kern0pt}env{\isacharparenright}{\kern0pt}{\isachardoublequoteclose}\ {\isachardoublequoteopen}k\ {\isacharless}{\kern0pt}\ length{\isacharparenleft}{\kern0pt}env{\isacharparenright}{\kern0pt}{\isachardoublequoteclose}\ {\isachardoublequoteopen}l\ {\isacharless}{\kern0pt}\ length{\isacharparenleft}{\kern0pt}env{\isacharparenright}{\kern0pt}{\isachardoublequoteclose}\ \isanewline
\ \ \ \ \ \ \ \ \ \ {\isachardoublequoteopen}x\ {\isacharequal}{\kern0pt}\ nth{\isacharparenleft}{\kern0pt}i{\isacharcomma}{\kern0pt}\ env{\isacharparenright}{\kern0pt}{\isachardoublequoteclose}\ {\isachardoublequoteopen}n\ {\isacharequal}{\kern0pt}\ nth{\isacharparenleft}{\kern0pt}j{\isacharcomma}{\kern0pt}\ env{\isacharparenright}{\kern0pt}{\isachardoublequoteclose}\ {\isachardoublequoteopen}Fn\ {\isacharequal}{\kern0pt}\ nth{\isacharparenleft}{\kern0pt}k{\isacharcomma}{\kern0pt}\ env{\isacharparenright}{\kern0pt}{\isachardoublequoteclose}\ {\isachardoublequoteopen}nat\ {\isacharequal}{\kern0pt}\ nth{\isacharparenleft}{\kern0pt}l{\isacharcomma}{\kern0pt}\ env{\isacharparenright}{\kern0pt}{\isachardoublequoteclose}\ \isanewline
\ \ \ \ \ \ \ \ \ \ {\isachardoublequoteopen}env\ {\isasymin}\ list{\isacharparenleft}{\kern0pt}M{\isacharparenright}{\kern0pt}{\isachardoublequoteclose}\isanewline
\ \ \isakeyword{shows}\ {\isachardoublequoteopen}sats{\isacharparenleft}{\kern0pt}M{\isacharcomma}{\kern0pt}\ binmap{\isacharunderscore}{\kern0pt}row{\isacharprime}{\kern0pt}{\isacharunderscore}{\kern0pt}member{\isacharunderscore}{\kern0pt}fm{\isacharparenleft}{\kern0pt}i{\isacharcomma}{\kern0pt}\ j{\isacharcomma}{\kern0pt}\ k{\isacharcomma}{\kern0pt}\ l{\isacharparenright}{\kern0pt}{\isacharcomma}{\kern0pt}\ env{\isacharparenright}{\kern0pt}\ {\isasymlongleftrightarrow}\ {\isacharparenleft}{\kern0pt}{\isasymexists}F\ {\isasymin}\ Fn{\isachardot}{\kern0pt}\ {\isasymexists}m\ {\isasymin}\ nat{\isachardot}{\kern0pt}\ x\ {\isacharequal}{\kern0pt}\ {\isacharless}{\kern0pt}check{\isacharparenleft}{\kern0pt}m{\isacharparenright}{\kern0pt}{\isacharcomma}{\kern0pt}\ F{\isachargreater}{\kern0pt}\ {\isasymand}\ F{\isacharbackquote}{\kern0pt}{\isacharless}{\kern0pt}n{\isacharcomma}{\kern0pt}\ m{\isachargreater}{\kern0pt}\ {\isacharequal}{\kern0pt}\ {\isadigit{1}}{\isacharparenright}{\kern0pt}{\isachardoublequoteclose}\ \isanewline
%
\isadelimproof
\isanewline
\ \ %
\endisadelimproof
%
\isatagproof
\isacommand{unfolding}\isamarkupfalse%
\ binmap{\isacharunderscore}{\kern0pt}row{\isacharprime}{\kern0pt}{\isacharunderscore}{\kern0pt}member{\isacharunderscore}{\kern0pt}fm{\isacharunderscore}{\kern0pt}def\ \isanewline
\ \ \isacommand{apply}\isamarkupfalse%
{\isacharparenleft}{\kern0pt}rule{\isacharunderscore}{\kern0pt}tac\ Q{\isacharequal}{\kern0pt}{\isachardoublequoteopen}{\isasymexists}Fnm\ {\isasymin}\ M{\isachardot}{\kern0pt}\ {\isasymexists}nm\ {\isasymin}\ M{\isachardot}{\kern0pt}\ {\isasymexists}checkm\ {\isasymin}\ M{\isachardot}{\kern0pt}\ {\isasymexists}m\ {\isasymin}\ M{\isachardot}{\kern0pt}\ {\isasymexists}F\ {\isasymin}\ M{\isachardot}{\kern0pt}\ {\isasymexists}zero\ {\isasymin}\ M{\isachardot}{\kern0pt}\ \isanewline
\ \ \ \ \ \ \ \ \ \ \ \ \ \ \ \ \ \ \ \ zero\ {\isacharequal}{\kern0pt}\ {\isadigit{0}}\ {\isasymand}\ F\ {\isasymin}\ Fn\ {\isasymand}\ m\ {\isasymin}\ nat\ {\isasymand}\ checkm\ {\isacharequal}{\kern0pt}\ check{\isacharparenleft}{\kern0pt}m{\isacharparenright}{\kern0pt}\ {\isasymand}\ x\ {\isacharequal}{\kern0pt}\ {\isacharless}{\kern0pt}checkm{\isacharcomma}{\kern0pt}\ F{\isachargreater}{\kern0pt}\ {\isasymand}\ nm\ {\isacharequal}{\kern0pt}\ {\isacharless}{\kern0pt}n{\isacharcomma}{\kern0pt}\ m{\isachargreater}{\kern0pt}\ {\isasymand}\ F{\isacharbackquote}{\kern0pt}nm\ {\isacharequal}{\kern0pt}\ Fnm\ {\isasymand}\ Fnm\ {\isacharequal}{\kern0pt}\ {\isadigit{1}}{\isachardoublequoteclose}\ \isakeyword{in}\ iff{\isacharunderscore}{\kern0pt}trans{\isacharparenright}{\kern0pt}\isanewline
\ \ \ \isacommand{apply}\isamarkupfalse%
{\isacharparenleft}{\kern0pt}rule\ iff{\isacharunderscore}{\kern0pt}trans{\isacharcomma}{\kern0pt}\ rule\ sats{\isacharunderscore}{\kern0pt}Exists{\isacharunderscore}{\kern0pt}iff{\isacharcomma}{\kern0pt}\ simp\ add{\isacharcolon}{\kern0pt}assms{\isacharcomma}{\kern0pt}\ rule\ bex{\isacharunderscore}{\kern0pt}iff{\isacharparenright}{\kern0pt}{\isacharplus}{\kern0pt}\isanewline
\ \ \ \isacommand{apply}\isamarkupfalse%
{\isacharparenleft}{\kern0pt}rule\ iff{\isacharunderscore}{\kern0pt}trans{\isacharcomma}{\kern0pt}\ rule\ sats{\isacharunderscore}{\kern0pt}And{\isacharunderscore}{\kern0pt}iff{\isacharcomma}{\kern0pt}\ simp\ add{\isacharcolon}{\kern0pt}assms{\isacharcomma}{\kern0pt}\ rule\ iff{\isacharunderscore}{\kern0pt}conjI{\isadigit{2}}{\isacharparenright}{\kern0pt}\isanewline
\ \ \ \ \isacommand{apply}\isamarkupfalse%
{\isacharparenleft}{\kern0pt}rule\ iff{\isacharunderscore}{\kern0pt}trans{\isacharcomma}{\kern0pt}\ rule\ sats{\isacharunderscore}{\kern0pt}empty{\isacharunderscore}{\kern0pt}fm{\isacharcomma}{\kern0pt}\ force{\isacharcomma}{\kern0pt}\ simp\ add{\isacharcolon}{\kern0pt}assms{\isacharcomma}{\kern0pt}\ force{\isacharparenright}{\kern0pt}\isanewline
\ \ \ \isacommand{apply}\isamarkupfalse%
{\isacharparenleft}{\kern0pt}rule\ iff{\isacharunderscore}{\kern0pt}trans{\isacharcomma}{\kern0pt}\ rule\ sats{\isacharunderscore}{\kern0pt}And{\isacharunderscore}{\kern0pt}iff{\isacharcomma}{\kern0pt}\ simp\ add{\isacharcolon}{\kern0pt}assms{\isacharcomma}{\kern0pt}\ rule\ iff{\isacharunderscore}{\kern0pt}conjI{\isadigit{2}}{\isacharparenright}{\kern0pt}\isanewline
\ \ \isacommand{using}\isamarkupfalse%
\ assms\ lt{\isacharunderscore}{\kern0pt}nat{\isacharunderscore}{\kern0pt}in{\isacharunderscore}{\kern0pt}nat\ length{\isacharunderscore}{\kern0pt}type\ \isanewline
\ \ \ \ \isacommand{apply}\isamarkupfalse%
\ force\isanewline
\ \ \ \isacommand{apply}\isamarkupfalse%
{\isacharparenleft}{\kern0pt}rule\ iff{\isacharunderscore}{\kern0pt}trans{\isacharcomma}{\kern0pt}\ rule\ sats{\isacharunderscore}{\kern0pt}And{\isacharunderscore}{\kern0pt}iff{\isacharcomma}{\kern0pt}\ simp\ add{\isacharcolon}{\kern0pt}assms{\isacharcomma}{\kern0pt}\ rule\ iff{\isacharunderscore}{\kern0pt}conjI{\isadigit{2}}{\isacharparenright}{\kern0pt}\isanewline
\ \ \isacommand{using}\isamarkupfalse%
\ assms\ lt{\isacharunderscore}{\kern0pt}nat{\isacharunderscore}{\kern0pt}in{\isacharunderscore}{\kern0pt}nat\ length{\isacharunderscore}{\kern0pt}type\ \isanewline
\ \ \ \ \isacommand{apply}\isamarkupfalse%
\ force\isanewline
\ \ \ \isacommand{apply}\isamarkupfalse%
{\isacharparenleft}{\kern0pt}rule\ iff{\isacharunderscore}{\kern0pt}trans{\isacharcomma}{\kern0pt}\ rule\ sats{\isacharunderscore}{\kern0pt}And{\isacharunderscore}{\kern0pt}iff{\isacharcomma}{\kern0pt}\ simp\ add{\isacharcolon}{\kern0pt}assms{\isacharcomma}{\kern0pt}\ rule\ iff{\isacharunderscore}{\kern0pt}conjI{\isadigit{2}}{\isacharparenright}{\kern0pt}\isanewline
\ \ \ \ \isacommand{apply}\isamarkupfalse%
{\isacharparenleft}{\kern0pt}rule\ iff{\isacharunderscore}{\kern0pt}trans{\isacharcomma}{\kern0pt}\ rule\ sats{\isacharunderscore}{\kern0pt}check{\isacharunderscore}{\kern0pt}fm{\isacharparenright}{\kern0pt}\isanewline
\ \ \isacommand{using}\isamarkupfalse%
\ assms\ check{\isacharunderscore}{\kern0pt}abs\isanewline
\ \ \ \ \ \ \ \ \ \ \ \isacommand{apply}\isamarkupfalse%
\ auto{\isacharbrackleft}{\kern0pt}{\isadigit{8}}{\isacharbrackright}{\kern0pt}\isanewline
\ \ \ \isacommand{apply}\isamarkupfalse%
{\isacharparenleft}{\kern0pt}rule\ iff{\isacharunderscore}{\kern0pt}trans{\isacharcomma}{\kern0pt}\ rule\ sats{\isacharunderscore}{\kern0pt}And{\isacharunderscore}{\kern0pt}iff{\isacharcomma}{\kern0pt}\ simp\ add{\isacharcolon}{\kern0pt}assms{\isacharcomma}{\kern0pt}\ rule\ iff{\isacharunderscore}{\kern0pt}conjI{\isadigit{2}}{\isacharparenright}{\kern0pt}\isanewline
\ \ \isacommand{using}\isamarkupfalse%
\ assms\ lt{\isacharunderscore}{\kern0pt}nat{\isacharunderscore}{\kern0pt}in{\isacharunderscore}{\kern0pt}nat\ length{\isacharunderscore}{\kern0pt}type\ \isanewline
\ \ \ \ \isacommand{apply}\isamarkupfalse%
\ force\isanewline
\ \ \ \isacommand{apply}\isamarkupfalse%
{\isacharparenleft}{\kern0pt}rule\ iff{\isacharunderscore}{\kern0pt}trans{\isacharcomma}{\kern0pt}\ rule\ sats{\isacharunderscore}{\kern0pt}And{\isacharunderscore}{\kern0pt}iff{\isacharcomma}{\kern0pt}\ simp\ add{\isacharcolon}{\kern0pt}assms{\isacharcomma}{\kern0pt}\ rule\ iff{\isacharunderscore}{\kern0pt}conjI{\isadigit{2}}{\isacharparenright}{\kern0pt}\isanewline
\ \ \isacommand{using}\isamarkupfalse%
\ assms\ lt{\isacharunderscore}{\kern0pt}nat{\isacharunderscore}{\kern0pt}in{\isacharunderscore}{\kern0pt}nat\ length{\isacharunderscore}{\kern0pt}type\ \isanewline
\ \ \ \ \isacommand{apply}\isamarkupfalse%
\ force\isanewline
\ \ \ \isacommand{apply}\isamarkupfalse%
{\isacharparenleft}{\kern0pt}rule\ iff{\isacharunderscore}{\kern0pt}trans{\isacharcomma}{\kern0pt}\ rule\ sats{\isacharunderscore}{\kern0pt}And{\isacharunderscore}{\kern0pt}iff{\isacharcomma}{\kern0pt}\ simp\ add{\isacharcolon}{\kern0pt}assms{\isacharcomma}{\kern0pt}\ rule\ iff{\isacharunderscore}{\kern0pt}conjI{\isadigit{2}}{\isacharparenright}{\kern0pt}\isanewline
\ \ \isacommand{using}\isamarkupfalse%
\ assms\ lt{\isacharunderscore}{\kern0pt}nat{\isacharunderscore}{\kern0pt}in{\isacharunderscore}{\kern0pt}nat\ length{\isacharunderscore}{\kern0pt}type\ \isanewline
\ \ \ \ \isacommand{apply}\isamarkupfalse%
\ force\isanewline
\ \ \ \isacommand{apply}\isamarkupfalse%
{\isacharparenleft}{\kern0pt}rule\ iff{\isacharunderscore}{\kern0pt}trans{\isacharcomma}{\kern0pt}\ rule\ sats{\isacharunderscore}{\kern0pt}is{\isacharunderscore}{\kern0pt}{\isadigit{1}}{\isacharunderscore}{\kern0pt}fm{\isacharunderscore}{\kern0pt}iff{\isacharparenright}{\kern0pt}\isanewline
\ \ \isacommand{using}\isamarkupfalse%
\ assms\ lt{\isacharunderscore}{\kern0pt}nat{\isacharunderscore}{\kern0pt}in{\isacharunderscore}{\kern0pt}nat\ length{\isacharunderscore}{\kern0pt}type\ zero{\isacharunderscore}{\kern0pt}in{\isacharunderscore}{\kern0pt}M\ transM\ \isanewline
\ \ \ \ \ \ \isacommand{apply}\isamarkupfalse%
\ auto{\isacharbrackleft}{\kern0pt}{\isadigit{4}}{\isacharbrackright}{\kern0pt}\isanewline
\ \ \isacommand{apply}\isamarkupfalse%
{\isacharparenleft}{\kern0pt}rule\ iffI{\isacharcomma}{\kern0pt}\ force{\isacharcomma}{\kern0pt}\ clarsimp{\isacharparenright}{\kern0pt}\isanewline
\ \ \isacommand{apply}\isamarkupfalse%
{\isacharparenleft}{\kern0pt}rule\ conjI{\isacharcomma}{\kern0pt}\ simp\ add{\isacharcolon}{\kern0pt}zero{\isacharunderscore}{\kern0pt}in{\isacharunderscore}{\kern0pt}M{\isacharparenright}{\kern0pt}\isanewline
\ \ \isacommand{apply}\isamarkupfalse%
{\isacharparenleft}{\kern0pt}rule\ conjI{\isacharparenright}{\kern0pt}\isanewline
\ \ \isacommand{using}\isamarkupfalse%
\ Fn{\isacharunderscore}{\kern0pt}in{\isacharunderscore}{\kern0pt}M\ transM\ \isanewline
\ \ \ \isacommand{apply}\isamarkupfalse%
\ force\isanewline
\ \ \isacommand{apply}\isamarkupfalse%
{\isacharparenleft}{\kern0pt}rename{\isacharunderscore}{\kern0pt}tac\ F\ m{\isacharcomma}{\kern0pt}\ rule{\isacharunderscore}{\kern0pt}tac\ x{\isacharequal}{\kern0pt}{\isachardoublequoteopen}F{\isacharbackquote}{\kern0pt}{\isacharless}{\kern0pt}n{\isacharcomma}{\kern0pt}\ m{\isachargreater}{\kern0pt}{\isachardoublequoteclose}\ \isakeyword{in}\ bexI{\isacharparenright}{\kern0pt}\isanewline
\ \ \ \isacommand{apply}\isamarkupfalse%
{\isacharparenleft}{\kern0pt}rename{\isacharunderscore}{\kern0pt}tac\ F\ m{\isacharcomma}{\kern0pt}\ rule{\isacharunderscore}{\kern0pt}tac\ x{\isacharequal}{\kern0pt}{\isachardoublequoteopen}{\isacharless}{\kern0pt}n{\isacharcomma}{\kern0pt}\ m{\isachargreater}{\kern0pt}{\isachardoublequoteclose}\ \isakeyword{in}\ bexI{\isacharparenright}{\kern0pt}\isanewline
\ \ \ \ \isacommand{apply}\isamarkupfalse%
{\isacharparenleft}{\kern0pt}rename{\isacharunderscore}{\kern0pt}tac\ F\ m{\isacharcomma}{\kern0pt}\ rule{\isacharunderscore}{\kern0pt}tac\ x{\isacharequal}{\kern0pt}{\isachardoublequoteopen}check{\isacharparenleft}{\kern0pt}m{\isacharparenright}{\kern0pt}{\isachardoublequoteclose}\ \isakeyword{in}\ bexI{\isacharparenright}{\kern0pt}\isanewline
\ \ \ \ \ \isacommand{apply}\isamarkupfalse%
{\isacharparenleft}{\kern0pt}rename{\isacharunderscore}{\kern0pt}tac\ F\ m{\isacharcomma}{\kern0pt}\ rule{\isacharunderscore}{\kern0pt}tac\ x{\isacharequal}{\kern0pt}m\ \isakeyword{in}\ bexI{\isacharcomma}{\kern0pt}\ force{\isacharparenright}{\kern0pt}\isanewline
\ \ \isacommand{using}\isamarkupfalse%
\ nat{\isacharunderscore}{\kern0pt}in{\isacharunderscore}{\kern0pt}M\ transM\ check{\isacharunderscore}{\kern0pt}in{\isacharunderscore}{\kern0pt}M\ pair{\isacharunderscore}{\kern0pt}in{\isacharunderscore}{\kern0pt}M{\isacharunderscore}{\kern0pt}iff\ assms\ lt{\isacharunderscore}{\kern0pt}nat{\isacharunderscore}{\kern0pt}in{\isacharunderscore}{\kern0pt}nat\ nth{\isacharunderscore}{\kern0pt}type\ apply{\isacharunderscore}{\kern0pt}closed\isanewline
\ \ \isacommand{by}\isamarkupfalse%
\ auto%
\endisatagproof
{\isafoldproof}%
%
\isadelimproof
\isanewline
%
\endisadelimproof
\isanewline
\isacommand{lemma}\isamarkupfalse%
\ binmap{\isacharunderscore}{\kern0pt}row{\isacharprime}{\kern0pt}{\isacharunderscore}{\kern0pt}in{\isacharunderscore}{\kern0pt}M\ {\isacharcolon}{\kern0pt}\ \isanewline
\ \ \isakeyword{fixes}\ n\ \isanewline
\ \ \isakeyword{assumes}\ {\isachardoublequoteopen}n\ {\isasymin}\ nat{\isachardoublequoteclose}\ \isanewline
\ \ \isakeyword{shows}\ {\isachardoublequoteopen}binmap{\isacharunderscore}{\kern0pt}row{\isacharprime}{\kern0pt}{\isacharparenleft}{\kern0pt}n{\isacharparenright}{\kern0pt}\ {\isasymin}\ M{\isachardoublequoteclose}\ \isanewline
%
\isadelimproof
%
\endisadelimproof
%
\isatagproof
\isacommand{proof}\isamarkupfalse%
\ {\isacharminus}{\kern0pt}\ \isanewline
\ \ \isacommand{define}\isamarkupfalse%
\ X\ \isakeyword{where}\ {\isachardoublequoteopen}X\ {\isasymequiv}\ {\isacharbraceleft}{\kern0pt}\ x\ {\isasymin}\ domain{\isacharparenleft}{\kern0pt}check{\isacharparenleft}{\kern0pt}nat{\isacharparenright}{\kern0pt}{\isacharparenright}{\kern0pt}\ {\isasymtimes}\ Fn{\isachardot}{\kern0pt}\ sats{\isacharparenleft}{\kern0pt}M{\isacharcomma}{\kern0pt}\ binmap{\isacharunderscore}{\kern0pt}row{\isacharprime}{\kern0pt}{\isacharunderscore}{\kern0pt}member{\isacharunderscore}{\kern0pt}fm{\isacharparenleft}{\kern0pt}{\isadigit{0}}{\isacharcomma}{\kern0pt}\ {\isadigit{1}}{\isacharcomma}{\kern0pt}\ {\isadigit{2}}{\isacharcomma}{\kern0pt}\ {\isadigit{3}}{\isacharparenright}{\kern0pt}{\isacharcomma}{\kern0pt}\ {\isacharbrackleft}{\kern0pt}x{\isacharbrackright}{\kern0pt}\ {\isacharat}{\kern0pt}\ {\isacharbrackleft}{\kern0pt}n{\isacharcomma}{\kern0pt}\ Fn{\isacharcomma}{\kern0pt}\ nat{\isacharbrackright}{\kern0pt}{\isacharparenright}{\kern0pt}\ {\isacharbraceright}{\kern0pt}{\isachardoublequoteclose}\ \isanewline
\isanewline
\ \ \isacommand{have}\isamarkupfalse%
\ {\isachardoublequoteopen}X\ {\isasymin}\ M{\isachardoublequoteclose}\ \isanewline
\ \ \ \ \isacommand{unfolding}\isamarkupfalse%
\ X{\isacharunderscore}{\kern0pt}def\ \isanewline
\ \ \ \ \isacommand{apply}\isamarkupfalse%
{\isacharparenleft}{\kern0pt}rule\ separation{\isacharunderscore}{\kern0pt}notation{\isacharparenright}{\kern0pt}\isanewline
\ \ \ \ \ \isacommand{apply}\isamarkupfalse%
{\isacharparenleft}{\kern0pt}rule\ separation{\isacharunderscore}{\kern0pt}ax{\isacharparenright}{\kern0pt}\isanewline
\ \ \ \ \ \ \ \isacommand{apply}\isamarkupfalse%
{\isacharparenleft}{\kern0pt}subst\ binmap{\isacharunderscore}{\kern0pt}row{\isacharprime}{\kern0pt}{\isacharunderscore}{\kern0pt}member{\isacharunderscore}{\kern0pt}fm{\isacharunderscore}{\kern0pt}def{\isacharparenright}{\kern0pt}\isanewline
\ \ \ \ \ \ \ \isacommand{apply}\isamarkupfalse%
{\isacharparenleft}{\kern0pt}subgoal{\isacharunderscore}{\kern0pt}tac\ {\isachardoublequoteopen}check{\isacharunderscore}{\kern0pt}fm{\isacharparenleft}{\kern0pt}{\isadigit{2}}{\isacharcomma}{\kern0pt}\ {\isadigit{0}}{\isacharcomma}{\kern0pt}\ {\isadigit{3}}{\isacharparenright}{\kern0pt}\ {\isasymin}\ formula\ {\isasymand}\ is{\isacharunderscore}{\kern0pt}{\isadigit{1}}{\isacharunderscore}{\kern0pt}fm{\isacharparenleft}{\kern0pt}{\isadigit{5}}{\isacharparenright}{\kern0pt}\ {\isasymin}\ formula{\isachardoublequoteclose}{\isacharcomma}{\kern0pt}\ force{\isacharparenright}{\kern0pt}\isanewline
\ \ \ \ \ \ \ \isacommand{apply}\isamarkupfalse%
{\isacharparenleft}{\kern0pt}rule\ conjI{\isacharcomma}{\kern0pt}\ rule\ check{\isacharunderscore}{\kern0pt}fm{\isacharunderscore}{\kern0pt}type{\isacharparenright}{\kern0pt}\isanewline
\ \ \ \ \ \ \ \ \ \ \isacommand{apply}\isamarkupfalse%
\ auto{\isacharbrackleft}{\kern0pt}{\isadigit{3}}{\isacharbrackright}{\kern0pt}\isanewline
\ \ \ \ \ \ \ \isacommand{apply}\isamarkupfalse%
{\isacharparenleft}{\kern0pt}rule\ is{\isacharunderscore}{\kern0pt}{\isadigit{1}}{\isacharunderscore}{\kern0pt}fm{\isacharunderscore}{\kern0pt}type{\isacharcomma}{\kern0pt}\ force{\isacharparenright}{\kern0pt}\isanewline
\ \ \ \ \isacommand{using}\isamarkupfalse%
\ assms\ nat{\isacharunderscore}{\kern0pt}in{\isacharunderscore}{\kern0pt}M\ transM\ Fn{\isacharunderscore}{\kern0pt}in{\isacharunderscore}{\kern0pt}M\ \isanewline
\ \ \ \ \ \ \isacommand{apply}\isamarkupfalse%
\ force\ \isanewline
\ \ \ \ \ \isacommand{apply}\isamarkupfalse%
{\isacharparenleft}{\kern0pt}rule\ le{\isacharunderscore}{\kern0pt}trans{\isacharcomma}{\kern0pt}\ rule\ arity{\isacharunderscore}{\kern0pt}binmap{\isacharunderscore}{\kern0pt}row{\isacharprime}{\kern0pt}{\isacharunderscore}{\kern0pt}member{\isacharunderscore}{\kern0pt}fm{\isacharparenright}{\kern0pt}\isanewline
\ \ \ \ \ \ \ \ \ \isacommand{apply}\isamarkupfalse%
\ auto{\isacharbrackleft}{\kern0pt}{\isadigit{4}}{\isacharbrackright}{\kern0pt}\isanewline
\ \ \ \ \ \isacommand{apply}\isamarkupfalse%
\ {\isacharparenleft}{\kern0pt}simp\ del{\isacharcolon}{\kern0pt}FOL{\isacharunderscore}{\kern0pt}sats{\isacharunderscore}{\kern0pt}iff\ pair{\isacharunderscore}{\kern0pt}abs\ add{\isacharcolon}{\kern0pt}\ fm{\isacharunderscore}{\kern0pt}defs\ nat{\isacharunderscore}{\kern0pt}simp{\isacharunderscore}{\kern0pt}union{\isacharparenright}{\kern0pt}\isanewline
\ \ \ \ \isacommand{using}\isamarkupfalse%
\ domain{\isacharunderscore}{\kern0pt}closed\ check{\isacharunderscore}{\kern0pt}in{\isacharunderscore}{\kern0pt}M\ nat{\isacharunderscore}{\kern0pt}in{\isacharunderscore}{\kern0pt}M\ Fn{\isacharunderscore}{\kern0pt}in{\isacharunderscore}{\kern0pt}M\ cartprod{\isacharunderscore}{\kern0pt}closed\isanewline
\ \ \ \ \isacommand{by}\isamarkupfalse%
\ auto\isanewline
\isanewline
\ \ \isacommand{have}\isamarkupfalse%
\ {\isachardoublequoteopen}X\ {\isacharequal}{\kern0pt}\ binmap{\isacharunderscore}{\kern0pt}row{\isacharprime}{\kern0pt}{\isacharparenleft}{\kern0pt}n{\isacharparenright}{\kern0pt}{\isachardoublequoteclose}\ \isanewline
\ \ \ \ \isacommand{unfolding}\isamarkupfalse%
\ X{\isacharunderscore}{\kern0pt}def\ binmap{\isacharunderscore}{\kern0pt}row{\isacharprime}{\kern0pt}{\isacharunderscore}{\kern0pt}def\ \isanewline
\ \ \ \ \isacommand{apply}\isamarkupfalse%
{\isacharparenleft}{\kern0pt}rule\ iff{\isacharunderscore}{\kern0pt}eq{\isacharparenright}{\kern0pt}\isanewline
\ \ \ \ \isacommand{apply}\isamarkupfalse%
{\isacharparenleft}{\kern0pt}rule\ sats{\isacharunderscore}{\kern0pt}binmap{\isacharunderscore}{\kern0pt}row{\isacharprime}{\kern0pt}{\isacharunderscore}{\kern0pt}member{\isacharunderscore}{\kern0pt}fm{\isacharunderscore}{\kern0pt}iff{\isacharparenright}{\kern0pt}\isanewline
\ \ \ \ \isacommand{using}\isamarkupfalse%
\ assms\ Fn{\isacharunderscore}{\kern0pt}in{\isacharunderscore}{\kern0pt}M\ nat{\isacharunderscore}{\kern0pt}in{\isacharunderscore}{\kern0pt}M\ pair{\isacharunderscore}{\kern0pt}in{\isacharunderscore}{\kern0pt}M{\isacharunderscore}{\kern0pt}iff\ transM\ check{\isacharunderscore}{\kern0pt}in{\isacharunderscore}{\kern0pt}M\isanewline
\ \ \ \ \ \ \ \ \ \ \ \ \isacommand{apply}\isamarkupfalse%
\ {\isacharparenleft}{\kern0pt}auto{\isacharcomma}{\kern0pt}\ blast{\isacharparenright}{\kern0pt}\isanewline
\ \ \ \ \isacommand{done}\isamarkupfalse%
\isanewline
\isanewline
\ \ \isacommand{then}\isamarkupfalse%
\ \isacommand{show}\isamarkupfalse%
\ {\isacharquery}{\kern0pt}thesis\ \isacommand{using}\isamarkupfalse%
\ {\isacartoucheopen}X\ {\isasymin}\ M{\isacartoucheclose}\ \isacommand{by}\isamarkupfalse%
\ auto\isanewline
\isacommand{qed}\isamarkupfalse%
%
\endisatagproof
{\isafoldproof}%
%
\isadelimproof
\ \isanewline
%
\endisadelimproof
\isanewline
\isacommand{definition}\isamarkupfalse%
\ is{\isacharunderscore}{\kern0pt}binmap{\isacharunderscore}{\kern0pt}row{\isacharprime}{\kern0pt}{\isacharunderscore}{\kern0pt}fm\ \isakeyword{where}\ \isanewline
\ \ {\isachardoublequoteopen}is{\isacharunderscore}{\kern0pt}binmap{\isacharunderscore}{\kern0pt}row{\isacharprime}{\kern0pt}{\isacharunderscore}{\kern0pt}fm{\isacharparenleft}{\kern0pt}y{\isacharcomma}{\kern0pt}\ n{\isacharcomma}{\kern0pt}\ checknatdom{\isacharcomma}{\kern0pt}\ fn{\isacharcomma}{\kern0pt}\ N{\isacharparenright}{\kern0pt}\ {\isasymequiv}\isanewline
\ \ \ \ Forall{\isacharparenleft}{\kern0pt}Iff{\isacharparenleft}{\kern0pt}Member{\isacharparenleft}{\kern0pt}{\isadigit{0}}{\isacharcomma}{\kern0pt}\ y{\isacharhash}{\kern0pt}{\isacharplus}{\kern0pt}{\isadigit{1}}{\isacharparenright}{\kern0pt}{\isacharcomma}{\kern0pt}\ \isanewline
\ \ \ \ \ \ And{\isacharparenleft}{\kern0pt}Exists{\isacharparenleft}{\kern0pt}And{\isacharparenleft}{\kern0pt}cartprod{\isacharunderscore}{\kern0pt}fm{\isacharparenleft}{\kern0pt}checknatdom{\isacharhash}{\kern0pt}{\isacharplus}{\kern0pt}{\isadigit{2}}{\isacharcomma}{\kern0pt}\ fn{\isacharhash}{\kern0pt}{\isacharplus}{\kern0pt}{\isadigit{2}}{\isacharcomma}{\kern0pt}\ {\isadigit{0}}{\isacharparenright}{\kern0pt}{\isacharcomma}{\kern0pt}\ Member{\isacharparenleft}{\kern0pt}{\isadigit{1}}{\isacharcomma}{\kern0pt}\ {\isadigit{0}}{\isacharparenright}{\kern0pt}{\isacharparenright}{\kern0pt}{\isacharparenright}{\kern0pt}{\isacharcomma}{\kern0pt}\ \ \isanewline
\ \ \ \ \ \ binmap{\isacharunderscore}{\kern0pt}row{\isacharprime}{\kern0pt}{\isacharunderscore}{\kern0pt}member{\isacharunderscore}{\kern0pt}fm{\isacharparenleft}{\kern0pt}{\isadigit{0}}{\isacharcomma}{\kern0pt}\ n{\isacharhash}{\kern0pt}{\isacharplus}{\kern0pt}{\isadigit{1}}{\isacharcomma}{\kern0pt}\ fn{\isacharhash}{\kern0pt}{\isacharplus}{\kern0pt}{\isadigit{1}}{\isacharcomma}{\kern0pt}\ N{\isacharhash}{\kern0pt}{\isacharplus}{\kern0pt}{\isadigit{1}}{\isacharparenright}{\kern0pt}{\isacharparenright}{\kern0pt}{\isacharparenright}{\kern0pt}{\isacharparenright}{\kern0pt}{\isachardoublequoteclose}\ \ \isanewline
\isanewline
\isacommand{lemma}\isamarkupfalse%
\ is{\isacharunderscore}{\kern0pt}binmap{\isacharunderscore}{\kern0pt}row{\isacharprime}{\kern0pt}{\isacharunderscore}{\kern0pt}fm{\isacharunderscore}{\kern0pt}type\ {\isacharcolon}{\kern0pt}\ \isanewline
\ \ \isakeyword{fixes}\ i\ j\ k\ l\ m\ \isanewline
\ \ \isakeyword{assumes}\ {\isachardoublequoteopen}i\ {\isasymin}\ nat{\isachardoublequoteclose}\ {\isachardoublequoteopen}j\ {\isasymin}\ nat{\isachardoublequoteclose}\ {\isachardoublequoteopen}k\ {\isasymin}\ nat{\isachardoublequoteclose}\ {\isachardoublequoteopen}l\ {\isasymin}\ nat{\isachardoublequoteclose}\ {\isachardoublequoteopen}m\ {\isasymin}\ nat{\isachardoublequoteclose}\ \isanewline
\ \ \isakeyword{shows}\ {\isachardoublequoteopen}is{\isacharunderscore}{\kern0pt}binmap{\isacharunderscore}{\kern0pt}row{\isacharprime}{\kern0pt}{\isacharunderscore}{\kern0pt}fm{\isacharparenleft}{\kern0pt}i{\isacharcomma}{\kern0pt}\ j{\isacharcomma}{\kern0pt}\ k{\isacharcomma}{\kern0pt}\ l{\isacharcomma}{\kern0pt}\ m{\isacharparenright}{\kern0pt}\ {\isasymin}\ formula{\isachardoublequoteclose}\isanewline
%
\isadelimproof
\ \ %
\endisadelimproof
%
\isatagproof
\isacommand{unfolding}\isamarkupfalse%
\ is{\isacharunderscore}{\kern0pt}binmap{\isacharunderscore}{\kern0pt}row{\isacharprime}{\kern0pt}{\isacharunderscore}{\kern0pt}fm{\isacharunderscore}{\kern0pt}def\isanewline
\ \ \isacommand{using}\isamarkupfalse%
\ assms\isanewline
\ \ \isacommand{apply}\isamarkupfalse%
\ simp\isanewline
\ \ \isacommand{apply}\isamarkupfalse%
{\isacharparenleft}{\kern0pt}subgoal{\isacharunderscore}{\kern0pt}tac\ {\isachardoublequoteopen}binmap{\isacharunderscore}{\kern0pt}row{\isacharprime}{\kern0pt}{\isacharunderscore}{\kern0pt}member{\isacharunderscore}{\kern0pt}fm{\isacharparenleft}{\kern0pt}{\isadigit{0}}{\isacharcomma}{\kern0pt}\ succ{\isacharparenleft}{\kern0pt}j{\isacharparenright}{\kern0pt}{\isacharcomma}{\kern0pt}\ succ{\isacharparenleft}{\kern0pt}l{\isacharparenright}{\kern0pt}{\isacharcomma}{\kern0pt}\ succ{\isacharparenleft}{\kern0pt}m{\isacharparenright}{\kern0pt}{\isacharparenright}{\kern0pt}\ {\isasymin}\ formula{\isachardoublequoteclose}{\isacharcomma}{\kern0pt}\ force{\isacharparenright}{\kern0pt}\isanewline
\ \ \isacommand{apply}\isamarkupfalse%
{\isacharparenleft}{\kern0pt}rule\ binmap{\isacharunderscore}{\kern0pt}row{\isacharprime}{\kern0pt}{\isacharunderscore}{\kern0pt}member{\isacharunderscore}{\kern0pt}fm{\isacharunderscore}{\kern0pt}type{\isacharparenright}{\kern0pt}\isanewline
\ \ \isacommand{by}\isamarkupfalse%
\ auto%
\endisatagproof
{\isafoldproof}%
%
\isadelimproof
\isanewline
%
\endisadelimproof
\isanewline
\isacommand{lemma}\isamarkupfalse%
\ arity{\isacharunderscore}{\kern0pt}is{\isacharunderscore}{\kern0pt}binmap{\isacharunderscore}{\kern0pt}row{\isacharprime}{\kern0pt}{\isacharunderscore}{\kern0pt}fm\ {\isacharcolon}{\kern0pt}\ \isanewline
\ \ \isakeyword{fixes}\ i\ j\ k\ l\ m\ \isanewline
\ \ \isakeyword{assumes}\ {\isachardoublequoteopen}i\ {\isasymin}\ nat{\isachardoublequoteclose}\ {\isachardoublequoteopen}j\ {\isasymin}\ nat{\isachardoublequoteclose}\ {\isachardoublequoteopen}k\ {\isasymin}\ nat{\isachardoublequoteclose}\ {\isachardoublequoteopen}l\ {\isasymin}\ nat{\isachardoublequoteclose}\ {\isachardoublequoteopen}m\ {\isasymin}\ nat{\isachardoublequoteclose}\ \isanewline
\ \ \isakeyword{shows}\ {\isachardoublequoteopen}arity{\isacharparenleft}{\kern0pt}is{\isacharunderscore}{\kern0pt}binmap{\isacharunderscore}{\kern0pt}row{\isacharprime}{\kern0pt}{\isacharunderscore}{\kern0pt}fm{\isacharparenleft}{\kern0pt}i{\isacharcomma}{\kern0pt}\ j{\isacharcomma}{\kern0pt}\ k{\isacharcomma}{\kern0pt}\ l{\isacharcomma}{\kern0pt}\ m{\isacharparenright}{\kern0pt}{\isacharparenright}{\kern0pt}\ {\isasymle}\ succ{\isacharparenleft}{\kern0pt}i{\isacharparenright}{\kern0pt}\ {\isasymunion}\ succ{\isacharparenleft}{\kern0pt}j{\isacharparenright}{\kern0pt}\ {\isasymunion}\ succ{\isacharparenleft}{\kern0pt}k{\isacharparenright}{\kern0pt}\ {\isasymunion}\ succ{\isacharparenleft}{\kern0pt}l{\isacharparenright}{\kern0pt}\ {\isasymunion}\ succ{\isacharparenleft}{\kern0pt}m{\isacharparenright}{\kern0pt}{\isachardoublequoteclose}\ \isanewline
%
\isadelimproof
\ \ %
\endisadelimproof
%
\isatagproof
\isacommand{unfolding}\isamarkupfalse%
\ is{\isacharunderscore}{\kern0pt}binmap{\isacharunderscore}{\kern0pt}row{\isacharprime}{\kern0pt}{\isacharunderscore}{\kern0pt}fm{\isacharunderscore}{\kern0pt}def\isanewline
\ \ \isacommand{apply}\isamarkupfalse%
\ simp\isanewline
\ \ \isacommand{apply}\isamarkupfalse%
{\isacharparenleft}{\kern0pt}insert\ assms\ binmap{\isacharunderscore}{\kern0pt}row{\isacharprime}{\kern0pt}{\isacharunderscore}{\kern0pt}member{\isacharunderscore}{\kern0pt}fm{\isacharunderscore}{\kern0pt}type\ {\isacharbrackleft}{\kern0pt}of\ {\isadigit{0}}\ {\isachardoublequoteopen}succ{\isacharparenleft}{\kern0pt}j{\isacharparenright}{\kern0pt}{\isachardoublequoteclose}\ {\isachardoublequoteopen}succ{\isacharparenleft}{\kern0pt}l{\isacharparenright}{\kern0pt}{\isachardoublequoteclose}\ {\isachardoublequoteopen}succ{\isacharparenleft}{\kern0pt}m{\isacharparenright}{\kern0pt}{\isachardoublequoteclose}{\isacharbrackright}{\kern0pt}{\isacharparenright}{\kern0pt}\isanewline
\ \ \isacommand{apply}\isamarkupfalse%
{\isacharparenleft}{\kern0pt}rule\ pred{\isacharunderscore}{\kern0pt}le{\isacharcomma}{\kern0pt}\ simp{\isacharcomma}{\kern0pt}\ simp{\isacharparenright}{\kern0pt}\isanewline
\ \ \isacommand{apply}\isamarkupfalse%
{\isacharparenleft}{\kern0pt}rule\ Un{\isacharunderscore}{\kern0pt}least{\isacharunderscore}{\kern0pt}lt{\isacharparenright}{\kern0pt}{\isacharplus}{\kern0pt}\isanewline
\ \ \ \ \isacommand{apply}\isamarkupfalse%
\ auto{\isacharbrackleft}{\kern0pt}{\isadigit{1}}{\isacharbrackright}{\kern0pt}\isanewline
\ \ \ \isacommand{apply}\isamarkupfalse%
{\isacharparenleft}{\kern0pt}subst\ succ{\isacharunderscore}{\kern0pt}Un{\isacharunderscore}{\kern0pt}distrib{\isacharcomma}{\kern0pt}\ simp{\isacharcomma}{\kern0pt}\ simp{\isacharparenright}{\kern0pt}{\isacharplus}{\kern0pt}\isanewline
\ \ \ \isacommand{apply}\isamarkupfalse%
{\isacharparenleft}{\kern0pt}rule\ Un{\isacharunderscore}{\kern0pt}upper{\isadigit{1}}{\isacharunderscore}{\kern0pt}lt{\isacharparenright}{\kern0pt}{\isacharplus}{\kern0pt}\isanewline
\ \ \ \ \ \ \ \isacommand{apply}\isamarkupfalse%
\ auto{\isacharbrackleft}{\kern0pt}{\isadigit{5}}{\isacharbrackright}{\kern0pt}\isanewline
\ \ \isacommand{apply}\isamarkupfalse%
{\isacharparenleft}{\kern0pt}rule\ Un{\isacharunderscore}{\kern0pt}least{\isacharunderscore}{\kern0pt}lt{\isacharparenright}{\kern0pt}{\isacharplus}{\kern0pt}\isanewline
\ \ \isacommand{apply}\isamarkupfalse%
{\isacharparenleft}{\kern0pt}subst\ arity{\isacharunderscore}{\kern0pt}cartprod{\isacharunderscore}{\kern0pt}fm{\isacharcomma}{\kern0pt}\ simp{\isacharcomma}{\kern0pt}\ simp{\isacharcomma}{\kern0pt}\ simp{\isacharparenright}{\kern0pt}\isanewline
\ \ \ \isacommand{apply}\isamarkupfalse%
{\isacharparenleft}{\kern0pt}rule\ pred{\isacharunderscore}{\kern0pt}le{\isacharcomma}{\kern0pt}\ simp{\isacharcomma}{\kern0pt}\ simp{\isacharparenright}{\kern0pt}\isanewline
\ \ \ \isacommand{apply}\isamarkupfalse%
{\isacharparenleft}{\kern0pt}rule\ Un{\isacharunderscore}{\kern0pt}least{\isacharunderscore}{\kern0pt}lt{\isacharparenright}{\kern0pt}{\isacharplus}{\kern0pt}\isanewline
\ \ \ \ \ \ \isacommand{apply}\isamarkupfalse%
{\isacharparenleft}{\kern0pt}subst\ succ{\isacharunderscore}{\kern0pt}Un{\isacharunderscore}{\kern0pt}distrib{\isacharcomma}{\kern0pt}\ simp{\isacharcomma}{\kern0pt}\ simp{\isacharparenright}{\kern0pt}{\isacharplus}{\kern0pt}\isanewline
\ \ \ \ \ \ \isacommand{apply}\isamarkupfalse%
{\isacharparenleft}{\kern0pt}rule\ Un{\isacharunderscore}{\kern0pt}upper{\isadigit{1}}{\isacharunderscore}{\kern0pt}lt{\isacharcomma}{\kern0pt}\ rule\ Un{\isacharunderscore}{\kern0pt}upper{\isadigit{1}}{\isacharunderscore}{\kern0pt}lt{\isacharcomma}{\kern0pt}\ rule\ Un{\isacharunderscore}{\kern0pt}upper{\isadigit{2}}{\isacharunderscore}{\kern0pt}lt{\isacharcomma}{\kern0pt}\ simp{\isacharcomma}{\kern0pt}\ simp{\isacharcomma}{\kern0pt}\ simp{\isacharcomma}{\kern0pt}\ simp{\isacharparenright}{\kern0pt}\isanewline
\ \ \ \ \ \ \isacommand{apply}\isamarkupfalse%
{\isacharparenleft}{\kern0pt}subst\ succ{\isacharunderscore}{\kern0pt}Un{\isacharunderscore}{\kern0pt}distrib{\isacharcomma}{\kern0pt}\ simp{\isacharcomma}{\kern0pt}\ simp{\isacharparenright}{\kern0pt}{\isacharplus}{\kern0pt}\isanewline
\ \ \ \ \ \isacommand{apply}\isamarkupfalse%
{\isacharparenleft}{\kern0pt}rule\ Un{\isacharunderscore}{\kern0pt}upper{\isadigit{1}}{\isacharunderscore}{\kern0pt}lt{\isacharcomma}{\kern0pt}\ rule\ Un{\isacharunderscore}{\kern0pt}upper{\isadigit{2}}{\isacharunderscore}{\kern0pt}lt{\isacharcomma}{\kern0pt}\ simp{\isacharcomma}{\kern0pt}\ simp{\isacharcomma}{\kern0pt}\ simp{\isacharcomma}{\kern0pt}\ simp{\isacharparenright}{\kern0pt}\isanewline
\ \ \ \isacommand{apply}\isamarkupfalse%
{\isacharparenleft}{\kern0pt}rule\ Un{\isacharunderscore}{\kern0pt}least{\isacharunderscore}{\kern0pt}lt{\isacharcomma}{\kern0pt}\ simp{\isacharcomma}{\kern0pt}\ simp{\isacharparenright}{\kern0pt}\isanewline
\ \ \isacommand{apply}\isamarkupfalse%
{\isacharparenleft}{\kern0pt}rule\ le{\isacharunderscore}{\kern0pt}trans{\isacharcomma}{\kern0pt}\ rule\ arity{\isacharunderscore}{\kern0pt}binmap{\isacharunderscore}{\kern0pt}row{\isacharprime}{\kern0pt}{\isacharunderscore}{\kern0pt}member{\isacharunderscore}{\kern0pt}fm{\isacharparenright}{\kern0pt}\isanewline
\ \ \ \ \ \ \isacommand{apply}\isamarkupfalse%
\ auto{\isacharbrackleft}{\kern0pt}{\isadigit{4}}{\isacharbrackright}{\kern0pt}\isanewline
\ \ \isacommand{apply}\isamarkupfalse%
{\isacharparenleft}{\kern0pt}rule\ Un{\isacharunderscore}{\kern0pt}least{\isacharunderscore}{\kern0pt}lt{\isacharparenright}{\kern0pt}{\isacharplus}{\kern0pt}\isanewline
\ \ \ \ \ \isacommand{apply}\isamarkupfalse%
\ simp\isanewline
\ \ \ \ \isacommand{apply}\isamarkupfalse%
\ simp\isanewline
\ \ \ \ \isacommand{apply}\isamarkupfalse%
{\isacharparenleft}{\kern0pt}rule\ ltI{\isacharcomma}{\kern0pt}\ simp{\isacharcomma}{\kern0pt}\ simp{\isacharcomma}{\kern0pt}\ simp{\isacharcomma}{\kern0pt}\ rule\ ltI{\isacharcomma}{\kern0pt}\ simp{\isacharcomma}{\kern0pt}\ simp{\isacharparenright}{\kern0pt}\isanewline
\ \ \isacommand{apply}\isamarkupfalse%
{\isacharparenleft}{\kern0pt}simp{\isacharcomma}{\kern0pt}\ rule\ ltI{\isacharcomma}{\kern0pt}\ simp{\isacharcomma}{\kern0pt}\ simp{\isacharparenright}{\kern0pt}\isanewline
\ \ \isacommand{done}\isamarkupfalse%
%
\endisatagproof
{\isafoldproof}%
%
\isadelimproof
\isanewline
%
\endisadelimproof
\isanewline
\isacommand{lemma}\isamarkupfalse%
\ sats{\isacharunderscore}{\kern0pt}is{\isacharunderscore}{\kern0pt}binmap{\isacharunderscore}{\kern0pt}row{\isacharprime}{\kern0pt}{\isacharunderscore}{\kern0pt}fm\ {\isacharcolon}{\kern0pt}\ \isanewline
\ \ \isakeyword{fixes}\ i\ j\ k\ l\ m\ y\ n\ env\ \isanewline
\ \ \isakeyword{assumes}\ {\isachardoublequoteopen}i\ {\isacharless}{\kern0pt}\ length{\isacharparenleft}{\kern0pt}env{\isacharparenright}{\kern0pt}{\isachardoublequoteclose}\ {\isachardoublequoteopen}j\ {\isacharless}{\kern0pt}\ length{\isacharparenleft}{\kern0pt}env{\isacharparenright}{\kern0pt}{\isachardoublequoteclose}\ {\isachardoublequoteopen}k\ {\isacharless}{\kern0pt}\ length{\isacharparenleft}{\kern0pt}env{\isacharparenright}{\kern0pt}{\isachardoublequoteclose}\ {\isachardoublequoteopen}l\ {\isacharless}{\kern0pt}\ length{\isacharparenleft}{\kern0pt}env{\isacharparenright}{\kern0pt}{\isachardoublequoteclose}\ {\isachardoublequoteopen}m\ {\isacharless}{\kern0pt}\ length{\isacharparenleft}{\kern0pt}env{\isacharparenright}{\kern0pt}{\isachardoublequoteclose}\isanewline
\ \ \ \ \ \ \ \ \ \ {\isachardoublequoteopen}nth{\isacharparenleft}{\kern0pt}i{\isacharcomma}{\kern0pt}\ env{\isacharparenright}{\kern0pt}\ {\isacharequal}{\kern0pt}\ y{\isachardoublequoteclose}\ {\isachardoublequoteopen}nth{\isacharparenleft}{\kern0pt}j{\isacharcomma}{\kern0pt}\ env{\isacharparenright}{\kern0pt}\ {\isacharequal}{\kern0pt}\ n{\isachardoublequoteclose}\ {\isachardoublequoteopen}nth{\isacharparenleft}{\kern0pt}k{\isacharcomma}{\kern0pt}\ env{\isacharparenright}{\kern0pt}\ {\isacharequal}{\kern0pt}\ domain{\isacharparenleft}{\kern0pt}check{\isacharparenleft}{\kern0pt}nat{\isacharparenright}{\kern0pt}{\isacharparenright}{\kern0pt}{\isachardoublequoteclose}\ {\isachardoublequoteopen}nth{\isacharparenleft}{\kern0pt}l{\isacharcomma}{\kern0pt}\ env{\isacharparenright}{\kern0pt}\ {\isacharequal}{\kern0pt}\ Fn{\isachardoublequoteclose}\ {\isachardoublequoteopen}nth{\isacharparenleft}{\kern0pt}m{\isacharcomma}{\kern0pt}\ env{\isacharparenright}{\kern0pt}\ {\isacharequal}{\kern0pt}\ nat{\isachardoublequoteclose}\ \isanewline
\ \ \ \ \ \ \ \ \ \ {\isachardoublequoteopen}env\ {\isasymin}\ list{\isacharparenleft}{\kern0pt}M{\isacharparenright}{\kern0pt}{\isachardoublequoteclose}\ {\isachardoublequoteopen}n\ {\isasymin}\ nat{\isachardoublequoteclose}\ \isanewline
\ \ \isakeyword{shows}\ {\isachardoublequoteopen}sats{\isacharparenleft}{\kern0pt}M{\isacharcomma}{\kern0pt}\ is{\isacharunderscore}{\kern0pt}binmap{\isacharunderscore}{\kern0pt}row{\isacharprime}{\kern0pt}{\isacharunderscore}{\kern0pt}fm{\isacharparenleft}{\kern0pt}i{\isacharcomma}{\kern0pt}\ j{\isacharcomma}{\kern0pt}\ k{\isacharcomma}{\kern0pt}\ l{\isacharcomma}{\kern0pt}\ m{\isacharparenright}{\kern0pt}{\isacharcomma}{\kern0pt}\ env{\isacharparenright}{\kern0pt}\ {\isasymlongleftrightarrow}\ y\ {\isacharequal}{\kern0pt}\ binmap{\isacharunderscore}{\kern0pt}row{\isacharprime}{\kern0pt}{\isacharparenleft}{\kern0pt}n{\isacharparenright}{\kern0pt}{\isachardoublequoteclose}\ \isanewline
%
\isadelimproof
\isanewline
\ \ %
\endisadelimproof
%
\isatagproof
\isacommand{apply}\isamarkupfalse%
{\isacharparenleft}{\kern0pt}rule{\isacharunderscore}{\kern0pt}tac\ Q{\isacharequal}{\kern0pt}{\isachardoublequoteopen}{\isasymforall}v\ {\isasymin}\ M{\isachardot}{\kern0pt}\ v\ {\isasymin}\ y\ {\isasymlongleftrightarrow}\ {\isacharparenleft}{\kern0pt}{\isacharparenleft}{\kern0pt}{\isasymexists}A\ {\isasymin}\ M{\isachardot}{\kern0pt}\ A\ {\isacharequal}{\kern0pt}\ domain{\isacharparenleft}{\kern0pt}check{\isacharparenleft}{\kern0pt}nat{\isacharparenright}{\kern0pt}{\isacharparenright}{\kern0pt}\ {\isasymtimes}\ Fn\ {\isasymand}\ v\ {\isasymin}\ A{\isacharparenright}{\kern0pt}\ {\isasymand}\ {\isacharparenleft}{\kern0pt}{\isasymexists}F\ {\isasymin}\ Fn{\isachardot}{\kern0pt}\ {\isasymexists}m\ {\isasymin}\ nat{\isachardot}{\kern0pt}\ v\ {\isacharequal}{\kern0pt}\ {\isacharless}{\kern0pt}check{\isacharparenleft}{\kern0pt}m{\isacharparenright}{\kern0pt}{\isacharcomma}{\kern0pt}\ F{\isachargreater}{\kern0pt}\ {\isasymand}\ F{\isacharbackquote}{\kern0pt}{\isacharless}{\kern0pt}n{\isacharcomma}{\kern0pt}\ m{\isachargreater}{\kern0pt}\ {\isacharequal}{\kern0pt}\ {\isadigit{1}}{\isacharparenright}{\kern0pt}{\isacharparenright}{\kern0pt}{\isachardoublequoteclose}\ \isakeyword{in}\ iff{\isacharunderscore}{\kern0pt}trans{\isacharparenright}{\kern0pt}\isanewline
\ \ \isacommand{unfolding}\isamarkupfalse%
\ is{\isacharunderscore}{\kern0pt}binmap{\isacharunderscore}{\kern0pt}row{\isacharprime}{\kern0pt}{\isacharunderscore}{\kern0pt}fm{\isacharunderscore}{\kern0pt}def\isanewline
\ \ \ \isacommand{apply}\isamarkupfalse%
{\isacharparenleft}{\kern0pt}rule\ iff{\isacharunderscore}{\kern0pt}trans{\isacharcomma}{\kern0pt}\ rule\ sats{\isacharunderscore}{\kern0pt}Forall{\isacharunderscore}{\kern0pt}iff{\isacharcomma}{\kern0pt}\ simp\ add{\isacharcolon}{\kern0pt}assms{\isacharcomma}{\kern0pt}\ rule\ ball{\isacharunderscore}{\kern0pt}iff{\isacharparenright}{\kern0pt}\isanewline
\ \ \ \isacommand{apply}\isamarkupfalse%
{\isacharparenleft}{\kern0pt}rule\ iff{\isacharunderscore}{\kern0pt}trans{\isacharcomma}{\kern0pt}\ rule\ sats{\isacharunderscore}{\kern0pt}Iff{\isacharunderscore}{\kern0pt}iff{\isacharcomma}{\kern0pt}\ simp\ add{\isacharcolon}{\kern0pt}assms{\isacharcomma}{\kern0pt}\ rule\ iff{\isacharunderscore}{\kern0pt}iff{\isacharparenright}{\kern0pt}\isanewline
\ \ \isacommand{using}\isamarkupfalse%
\ lt{\isacharunderscore}{\kern0pt}nat{\isacharunderscore}{\kern0pt}in{\isacharunderscore}{\kern0pt}nat\ length{\isacharunderscore}{\kern0pt}type\ assms\isanewline
\ \ \ \ \isacommand{apply}\isamarkupfalse%
\ force\ \isanewline
\ \ \ \isacommand{apply}\isamarkupfalse%
{\isacharparenleft}{\kern0pt}rule\ iff{\isacharunderscore}{\kern0pt}trans{\isacharcomma}{\kern0pt}\ rule\ sats{\isacharunderscore}{\kern0pt}And{\isacharunderscore}{\kern0pt}iff{\isacharcomma}{\kern0pt}\ simp\ add{\isacharcolon}{\kern0pt}assms{\isacharcomma}{\kern0pt}\ rule\ iff{\isacharunderscore}{\kern0pt}conjI{\isacharparenright}{\kern0pt}\isanewline
\ \ \ \ \isacommand{apply}\isamarkupfalse%
{\isacharparenleft}{\kern0pt}rule\ iff{\isacharunderscore}{\kern0pt}trans{\isacharcomma}{\kern0pt}\ rule\ sats{\isacharunderscore}{\kern0pt}Exists{\isacharunderscore}{\kern0pt}iff{\isacharcomma}{\kern0pt}\ simp\ add{\isacharcolon}{\kern0pt}assms{\isacharcomma}{\kern0pt}\ rule\ bex{\isacharunderscore}{\kern0pt}iff{\isacharparenright}{\kern0pt}\isanewline
\ \ \ \ \isacommand{apply}\isamarkupfalse%
{\isacharparenleft}{\kern0pt}rule\ iff{\isacharunderscore}{\kern0pt}trans{\isacharcomma}{\kern0pt}\ rule\ sats{\isacharunderscore}{\kern0pt}And{\isacharunderscore}{\kern0pt}iff{\isacharcomma}{\kern0pt}\ simp\ add{\isacharcolon}{\kern0pt}assms{\isacharparenright}{\kern0pt}\isanewline
\ \ \ \ \isacommand{apply}\isamarkupfalse%
{\isacharparenleft}{\kern0pt}rule\ iff{\isacharunderscore}{\kern0pt}conjI{\isadigit{2}}{\isacharcomma}{\kern0pt}\ rule\ iff{\isacharunderscore}{\kern0pt}trans{\isacharcomma}{\kern0pt}\ rule\ sats{\isacharunderscore}{\kern0pt}cartprod{\isacharunderscore}{\kern0pt}fm{\isacharparenright}{\kern0pt}\isanewline
\ \ \isacommand{using}\isamarkupfalse%
\ lt{\isacharunderscore}{\kern0pt}nat{\isacharunderscore}{\kern0pt}in{\isacharunderscore}{\kern0pt}nat\ length{\isacharunderscore}{\kern0pt}type\ assms\ nth{\isacharunderscore}{\kern0pt}type\isanewline
\ \ \ \ \ \ \ \ \ \isacommand{apply}\isamarkupfalse%
\ auto{\isacharbrackleft}{\kern0pt}{\isadigit{4}}{\isacharbrackright}{\kern0pt}\isanewline
\ \ \ \ \ \isacommand{apply}\isamarkupfalse%
{\isacharparenleft}{\kern0pt}rule\ iff{\isacharunderscore}{\kern0pt}trans{\isacharcomma}{\kern0pt}\ rule\ cartprod{\isacharunderscore}{\kern0pt}abs{\isacharparenright}{\kern0pt}\isanewline
\ \ \ \ \ \ \ \ \isacommand{apply}\isamarkupfalse%
\ simp\isanewline
\ \ \ \ \ \ \ \ \isacommand{apply}\isamarkupfalse%
{\isacharparenleft}{\kern0pt}rule\ nth{\isacharunderscore}{\kern0pt}type{\isacharparenright}{\kern0pt}\isanewline
\ \ \isacommand{using}\isamarkupfalse%
\ lt{\isacharunderscore}{\kern0pt}nat{\isacharunderscore}{\kern0pt}in{\isacharunderscore}{\kern0pt}nat\ length{\isacharunderscore}{\kern0pt}type\ assms\ nth{\isacharunderscore}{\kern0pt}type\isanewline
\ \ \ \ \ \ \ \ \ \isacommand{apply}\isamarkupfalse%
\ auto{\isacharbrackleft}{\kern0pt}{\isadigit{2}}{\isacharbrackright}{\kern0pt}\isanewline
\ \ \ \ \ \ \ \ \isacommand{apply}\isamarkupfalse%
\ simp\isanewline
\ \ \ \ \ \ \ \ \isacommand{apply}\isamarkupfalse%
{\isacharparenleft}{\kern0pt}rule\ nth{\isacharunderscore}{\kern0pt}type{\isacharparenright}{\kern0pt}\isanewline
\ \ \isacommand{using}\isamarkupfalse%
\ lt{\isacharunderscore}{\kern0pt}nat{\isacharunderscore}{\kern0pt}in{\isacharunderscore}{\kern0pt}nat\ length{\isacharunderscore}{\kern0pt}type\ assms\ nth{\isacharunderscore}{\kern0pt}type\isanewline
\ \ \ \ \ \ \ \ \isacommand{apply}\isamarkupfalse%
\ auto{\isacharbrackleft}{\kern0pt}{\isadigit{5}}{\isacharbrackright}{\kern0pt}\isanewline
\ \ \ \isacommand{apply}\isamarkupfalse%
{\isacharparenleft}{\kern0pt}rule\ sats{\isacharunderscore}{\kern0pt}binmap{\isacharunderscore}{\kern0pt}row{\isacharprime}{\kern0pt}{\isacharunderscore}{\kern0pt}member{\isacharunderscore}{\kern0pt}fm{\isacharunderscore}{\kern0pt}iff{\isacharparenright}{\kern0pt}\isanewline
\ \ \isacommand{using}\isamarkupfalse%
\ lt{\isacharunderscore}{\kern0pt}nat{\isacharunderscore}{\kern0pt}in{\isacharunderscore}{\kern0pt}nat\ length{\isacharunderscore}{\kern0pt}type\ assms\ nth{\isacharunderscore}{\kern0pt}type\isanewline
\ \ \ \ \ \ \ \ \ \ \ \isacommand{apply}\isamarkupfalse%
\ auto{\isacharbrackleft}{\kern0pt}{\isadigit{9}}{\isacharbrackright}{\kern0pt}\isanewline
\ \ \isacommand{apply}\isamarkupfalse%
{\isacharparenleft}{\kern0pt}rule{\isacharunderscore}{\kern0pt}tac\ Q{\isacharequal}{\kern0pt}{\isachardoublequoteopen}{\isasymforall}v\ {\isasymin}\ M{\isachardot}{\kern0pt}\ v\ {\isasymin}\ y\ {\isasymlongleftrightarrow}\ v\ {\isasymin}\ binmap{\isacharunderscore}{\kern0pt}row{\isacharprime}{\kern0pt}{\isacharparenleft}{\kern0pt}n{\isacharparenright}{\kern0pt}{\isachardoublequoteclose}\ \isakeyword{in}\ iff{\isacharunderscore}{\kern0pt}trans{\isacharparenright}{\kern0pt}\isanewline
\ \ \ \isacommand{apply}\isamarkupfalse%
{\isacharparenleft}{\kern0pt}rule\ ball{\isacharunderscore}{\kern0pt}iff{\isacharcomma}{\kern0pt}\ rule\ iff{\isacharunderscore}{\kern0pt}iff{\isacharcomma}{\kern0pt}\ simp{\isacharparenright}{\kern0pt}\isanewline
\ \ \ \isacommand{apply}\isamarkupfalse%
{\isacharparenleft}{\kern0pt}simp\ add{\isacharcolon}{\kern0pt}\ binmap{\isacharunderscore}{\kern0pt}row{\isacharprime}{\kern0pt}{\isacharunderscore}{\kern0pt}def{\isacharparenright}{\kern0pt}\isanewline
\ \ \ \isacommand{apply}\isamarkupfalse%
\ {\isacharparenleft}{\kern0pt}rule\ iffI{\isacharcomma}{\kern0pt}\ force{\isacharcomma}{\kern0pt}\ simp{\isacharparenright}{\kern0pt}\isanewline
\ \ \isacommand{using}\isamarkupfalse%
\ cartprod{\isacharunderscore}{\kern0pt}closed\ domain{\isacharunderscore}{\kern0pt}closed\ check{\isacharunderscore}{\kern0pt}in{\isacharunderscore}{\kern0pt}M\ nat{\isacharunderscore}{\kern0pt}in{\isacharunderscore}{\kern0pt}M\ Fn{\isacharunderscore}{\kern0pt}in{\isacharunderscore}{\kern0pt}M\ \isanewline
\ \ \ \isacommand{apply}\isamarkupfalse%
\ force\ \isanewline
\ \ \isacommand{apply}\isamarkupfalse%
{\isacharparenleft}{\kern0pt}rule\ iffI{\isacharcomma}{\kern0pt}\ rule\ equality{\isacharunderscore}{\kern0pt}iffI{\isacharcomma}{\kern0pt}\ rule\ iffI{\isacharparenright}{\kern0pt}\isanewline
\ \ \ \ \isacommand{apply}\isamarkupfalse%
{\isacharparenleft}{\kern0pt}rename{\isacharunderscore}{\kern0pt}tac\ x{\isacharcomma}{\kern0pt}\ subgoal{\isacharunderscore}{\kern0pt}tac\ {\isachardoublequoteopen}x\ {\isasymin}\ M{\isachardoublequoteclose}{\isacharcomma}{\kern0pt}\ force{\isacharparenright}{\kern0pt}\isanewline
\ \ \isacommand{using}\isamarkupfalse%
\ lt{\isacharunderscore}{\kern0pt}nat{\isacharunderscore}{\kern0pt}in{\isacharunderscore}{\kern0pt}nat\ transM\ nth{\isacharunderscore}{\kern0pt}type\ length{\isacharunderscore}{\kern0pt}type\ assms\ \isanewline
\ \ \ \ \isacommand{apply}\isamarkupfalse%
\ force\isanewline
\ \ \ \isacommand{apply}\isamarkupfalse%
{\isacharparenleft}{\kern0pt}rename{\isacharunderscore}{\kern0pt}tac\ x{\isacharcomma}{\kern0pt}\ subgoal{\isacharunderscore}{\kern0pt}tac\ {\isachardoublequoteopen}x\ {\isasymin}\ M{\isachardoublequoteclose}{\isacharcomma}{\kern0pt}\ force{\isacharparenright}{\kern0pt}\isanewline
\ \ \isacommand{using}\isamarkupfalse%
\ lt{\isacharunderscore}{\kern0pt}nat{\isacharunderscore}{\kern0pt}in{\isacharunderscore}{\kern0pt}nat\ transM\ nth{\isacharunderscore}{\kern0pt}type\ length{\isacharunderscore}{\kern0pt}type\ assms\ binmap{\isacharunderscore}{\kern0pt}row{\isacharprime}{\kern0pt}{\isacharunderscore}{\kern0pt}in{\isacharunderscore}{\kern0pt}M\isanewline
\ \ \ \isacommand{apply}\isamarkupfalse%
\ force\isanewline
\ \ \isacommand{by}\isamarkupfalse%
\ auto%
\endisatagproof
{\isafoldproof}%
%
\isadelimproof
\isanewline
%
\endisadelimproof
\isanewline
\isacommand{lemma}\isamarkupfalse%
\ binmap{\isacharprime}{\kern0pt}{\isacharunderscore}{\kern0pt}in{\isacharunderscore}{\kern0pt}M\ {\isacharcolon}{\kern0pt}\ {\isachardoublequoteopen}binmap{\isacharprime}{\kern0pt}\ {\isasymin}\ M{\isachardoublequoteclose}\ \isanewline
%
\isadelimproof
%
\endisadelimproof
%
\isatagproof
\isacommand{proof}\isamarkupfalse%
\ {\isacharminus}{\kern0pt}\ \isanewline
\ \ \isacommand{define}\isamarkupfalse%
\ X\ \isakeyword{where}\ {\isachardoublequoteopen}X\ {\isasymequiv}\ {\isacharbraceleft}{\kern0pt}\ x\ {\isasymin}\ Pow{\isacharparenleft}{\kern0pt}domain{\isacharparenleft}{\kern0pt}check{\isacharparenleft}{\kern0pt}nat{\isacharparenright}{\kern0pt}{\isacharparenright}{\kern0pt}\ {\isasymtimes}\ Fn{\isacharparenright}{\kern0pt}\ {\isasyminter}\ M{\isachardot}{\kern0pt}\ {\isasymexists}n\ {\isasymin}\ M{\isachardot}{\kern0pt}\ n\ {\isasymin}\ nat\ {\isasymand}\ x\ {\isacharequal}{\kern0pt}\ binmap{\isacharunderscore}{\kern0pt}row{\isacharprime}{\kern0pt}{\isacharparenleft}{\kern0pt}n{\isacharparenright}{\kern0pt}\ {\isacharbraceright}{\kern0pt}{\isachardoublequoteclose}\ \isanewline
\isanewline
\ \ \isacommand{have}\isamarkupfalse%
\ {\isachardoublequoteopen}X\ {\isacharequal}{\kern0pt}\ {\isacharbraceleft}{\kern0pt}\ x\ {\isasymin}\ Pow{\isacharparenleft}{\kern0pt}domain{\isacharparenleft}{\kern0pt}check{\isacharparenleft}{\kern0pt}nat{\isacharparenright}{\kern0pt}{\isacharparenright}{\kern0pt}\ {\isasymtimes}\ Fn{\isacharparenright}{\kern0pt}\ {\isasyminter}\ M{\isachardot}{\kern0pt}\ sats{\isacharparenleft}{\kern0pt}M{\isacharcomma}{\kern0pt}\ Exists{\isacharparenleft}{\kern0pt}And{\isacharparenleft}{\kern0pt}Member{\isacharparenleft}{\kern0pt}{\isadigit{0}}{\isacharcomma}{\kern0pt}\ {\isadigit{4}}{\isacharparenright}{\kern0pt}{\isacharcomma}{\kern0pt}\ is{\isacharunderscore}{\kern0pt}binmap{\isacharunderscore}{\kern0pt}row{\isacharprime}{\kern0pt}{\isacharunderscore}{\kern0pt}fm{\isacharparenleft}{\kern0pt}{\isadigit{1}}{\isacharcomma}{\kern0pt}{\isadigit{0}}{\isacharcomma}{\kern0pt}{\isadigit{2}}{\isacharcomma}{\kern0pt}{\isadigit{3}}{\isacharcomma}{\kern0pt}{\isadigit{4}}{\isacharparenright}{\kern0pt}{\isacharparenright}{\kern0pt}{\isacharparenright}{\kern0pt}{\isacharcomma}{\kern0pt}\ {\isacharbrackleft}{\kern0pt}x{\isacharbrackright}{\kern0pt}\ {\isacharat}{\kern0pt}\ {\isacharbrackleft}{\kern0pt}domain{\isacharparenleft}{\kern0pt}check{\isacharparenleft}{\kern0pt}nat{\isacharparenright}{\kern0pt}{\isacharparenright}{\kern0pt}{\isacharcomma}{\kern0pt}\ Fn{\isacharcomma}{\kern0pt}\ nat{\isacharbrackright}{\kern0pt}{\isacharparenright}{\kern0pt}\ {\isacharbraceright}{\kern0pt}{\isachardoublequoteclose}\isanewline
\ \ \ \ {\isacharparenleft}{\kern0pt}\isakeyword{is}\ {\isachardoublequoteopen}{\isacharunderscore}{\kern0pt}\ {\isacharequal}{\kern0pt}\ {\isacharquery}{\kern0pt}A{\isachardoublequoteclose}{\isacharparenright}{\kern0pt}\isanewline
\ \ \ \ \isanewline
\ \ \ \ \isacommand{unfolding}\isamarkupfalse%
\ X{\isacharunderscore}{\kern0pt}def\isanewline
\ \ \ \ \isacommand{apply}\isamarkupfalse%
{\isacharparenleft}{\kern0pt}rule\ iff{\isacharunderscore}{\kern0pt}eq{\isacharcomma}{\kern0pt}\ rule\ iff{\isacharunderscore}{\kern0pt}flip{\isacharparenright}{\kern0pt}\isanewline
\ \ \ \ \isacommand{apply}\isamarkupfalse%
{\isacharparenleft}{\kern0pt}rule\ iff{\isacharunderscore}{\kern0pt}trans{\isacharcomma}{\kern0pt}\ rule\ sats{\isacharunderscore}{\kern0pt}Exists{\isacharunderscore}{\kern0pt}iff{\isacharparenright}{\kern0pt}\isanewline
\ \ \ \ \isacommand{using}\isamarkupfalse%
\ domain{\isacharunderscore}{\kern0pt}closed\ check{\isacharunderscore}{\kern0pt}in{\isacharunderscore}{\kern0pt}M\ nat{\isacharunderscore}{\kern0pt}in{\isacharunderscore}{\kern0pt}M\ Fn{\isacharunderscore}{\kern0pt}in{\isacharunderscore}{\kern0pt}M\ transM\isanewline
\ \ \ \ \ \isacommand{apply}\isamarkupfalse%
\ force\isanewline
\ \ \ \ \isacommand{apply}\isamarkupfalse%
{\isacharparenleft}{\kern0pt}rule\ bex{\isacharunderscore}{\kern0pt}iff{\isacharcomma}{\kern0pt}\ rule\ iff{\isacharunderscore}{\kern0pt}trans{\isacharcomma}{\kern0pt}\ rule\ sats{\isacharunderscore}{\kern0pt}And{\isacharunderscore}{\kern0pt}iff{\isacharparenright}{\kern0pt}\isanewline
\ \ \ \ \isacommand{using}\isamarkupfalse%
\ domain{\isacharunderscore}{\kern0pt}closed\ check{\isacharunderscore}{\kern0pt}in{\isacharunderscore}{\kern0pt}M\ nat{\isacharunderscore}{\kern0pt}in{\isacharunderscore}{\kern0pt}M\ Fn{\isacharunderscore}{\kern0pt}in{\isacharunderscore}{\kern0pt}M\ transM\isanewline
\ \ \ \ \ \isacommand{apply}\isamarkupfalse%
\ force\isanewline
\ \ \ \ \isacommand{apply}\isamarkupfalse%
{\isacharparenleft}{\kern0pt}rule\ iff{\isacharunderscore}{\kern0pt}conjI{\isadigit{2}}{\isacharparenright}{\kern0pt}\isanewline
\ \ \ \ \isacommand{using}\isamarkupfalse%
\ domain{\isacharunderscore}{\kern0pt}closed\ check{\isacharunderscore}{\kern0pt}in{\isacharunderscore}{\kern0pt}M\ nat{\isacharunderscore}{\kern0pt}in{\isacharunderscore}{\kern0pt}M\ Fn{\isacharunderscore}{\kern0pt}in{\isacharunderscore}{\kern0pt}M\ transM\isanewline
\ \ \ \ \ \isacommand{apply}\isamarkupfalse%
\ force\isanewline
\ \ \ \ \isacommand{apply}\isamarkupfalse%
{\isacharparenleft}{\kern0pt}rule\ sats{\isacharunderscore}{\kern0pt}is{\isacharunderscore}{\kern0pt}binmap{\isacharunderscore}{\kern0pt}row{\isacharprime}{\kern0pt}{\isacharunderscore}{\kern0pt}fm{\isacharparenright}{\kern0pt}\isanewline
\ \ \ \ \isacommand{using}\isamarkupfalse%
\ domain{\isacharunderscore}{\kern0pt}closed\ check{\isacharunderscore}{\kern0pt}in{\isacharunderscore}{\kern0pt}M\ nat{\isacharunderscore}{\kern0pt}in{\isacharunderscore}{\kern0pt}M\ Fn{\isacharunderscore}{\kern0pt}in{\isacharunderscore}{\kern0pt}M\ transM\isanewline
\ \ \ \ \isacommand{by}\isamarkupfalse%
\ auto\isanewline
\ \ \isanewline
\ \ \isacommand{have}\isamarkupfalse%
\ {\isachardoublequoteopen}{\isacharquery}{\kern0pt}A\ {\isasymin}\ M{\isachardoublequoteclose}\ \isanewline
\ \ \ \ \isacommand{apply}\isamarkupfalse%
{\isacharparenleft}{\kern0pt}rule\ separation{\isacharunderscore}{\kern0pt}notation{\isacharparenright}{\kern0pt}\isanewline
\ \ \ \ \ \isacommand{apply}\isamarkupfalse%
{\isacharparenleft}{\kern0pt}rule\ separation{\isacharunderscore}{\kern0pt}ax{\isacharparenright}{\kern0pt}\isanewline
\ \ \ \ \ \ \ \isacommand{apply}\isamarkupfalse%
{\isacharparenleft}{\kern0pt}insert\ is{\isacharunderscore}{\kern0pt}binmap{\isacharunderscore}{\kern0pt}row{\isacharprime}{\kern0pt}{\isacharunderscore}{\kern0pt}fm{\isacharunderscore}{\kern0pt}type\ {\isacharbrackleft}{\kern0pt}of\ {\isadigit{1}}\ {\isadigit{0}}\ {\isadigit{2}}\ {\isadigit{3}}\ {\isadigit{4}}{\isacharbrackright}{\kern0pt}{\isacharcomma}{\kern0pt}\ force{\isacharparenright}{\kern0pt}\isanewline
\ \ \ \ \isacommand{using}\isamarkupfalse%
\ domain{\isacharunderscore}{\kern0pt}closed\ check{\isacharunderscore}{\kern0pt}in{\isacharunderscore}{\kern0pt}M\ nat{\isacharunderscore}{\kern0pt}in{\isacharunderscore}{\kern0pt}M\ Fn{\isacharunderscore}{\kern0pt}in{\isacharunderscore}{\kern0pt}M\ transM\isanewline
\ \ \ \ \ \ \isacommand{apply}\isamarkupfalse%
\ force\isanewline
\ \ \ \ \ \isacommand{apply}\isamarkupfalse%
\ simp\isanewline
\ \ \ \ \isacommand{apply}\isamarkupfalse%
{\isacharparenleft}{\kern0pt}rule\ pred{\isacharunderscore}{\kern0pt}le{\isacharcomma}{\kern0pt}\ simp{\isacharcomma}{\kern0pt}\ simp{\isacharparenright}{\kern0pt}\isanewline
\ \ \ \ \ \isacommand{apply}\isamarkupfalse%
{\isacharparenleft}{\kern0pt}rule\ Un{\isacharunderscore}{\kern0pt}least{\isacharunderscore}{\kern0pt}lt{\isacharparenright}{\kern0pt}{\isacharplus}{\kern0pt}\isanewline
\ \ \ \ \ \ \ \isacommand{apply}\isamarkupfalse%
\ auto{\isacharbrackleft}{\kern0pt}{\isadigit{2}}{\isacharbrackright}{\kern0pt}\isanewline
\ \ \ \ \ \isacommand{apply}\isamarkupfalse%
{\isacharparenleft}{\kern0pt}rule\ le{\isacharunderscore}{\kern0pt}trans{\isacharcomma}{\kern0pt}\ rule\ arity{\isacharunderscore}{\kern0pt}is{\isacharunderscore}{\kern0pt}binmap{\isacharunderscore}{\kern0pt}row{\isacharprime}{\kern0pt}{\isacharunderscore}{\kern0pt}fm{\isacharparenright}{\kern0pt}\isanewline
\ \ \ \ \ \ \ \ \ \ \isacommand{apply}\isamarkupfalse%
\ auto{\isacharbrackleft}{\kern0pt}{\isadigit{5}}{\isacharbrackright}{\kern0pt}\isanewline
\ \ \ \ \ \isacommand{apply}\isamarkupfalse%
\ {\isacharparenleft}{\kern0pt}simp\ del{\isacharcolon}{\kern0pt}FOL{\isacharunderscore}{\kern0pt}sats{\isacharunderscore}{\kern0pt}iff\ pair{\isacharunderscore}{\kern0pt}abs\ add{\isacharcolon}{\kern0pt}\ fm{\isacharunderscore}{\kern0pt}defs\ nat{\isacharunderscore}{\kern0pt}simp{\isacharunderscore}{\kern0pt}union{\isacharparenright}{\kern0pt}\isanewline
\ \ \ \ \isacommand{apply}\isamarkupfalse%
{\isacharparenleft}{\kern0pt}rule\ M{\isacharunderscore}{\kern0pt}powerset{\isacharparenright}{\kern0pt}\isanewline
\ \ \ \ \isacommand{using}\isamarkupfalse%
\ domain{\isacharunderscore}{\kern0pt}closed\ check{\isacharunderscore}{\kern0pt}in{\isacharunderscore}{\kern0pt}M\ nat{\isacharunderscore}{\kern0pt}in{\isacharunderscore}{\kern0pt}M\ Fn{\isacharunderscore}{\kern0pt}in{\isacharunderscore}{\kern0pt}M\ transM\ cartprod{\isacharunderscore}{\kern0pt}closed\isanewline
\ \ \ \ \isacommand{by}\isamarkupfalse%
\ auto\isanewline
\isanewline
\ \ \isacommand{then}\isamarkupfalse%
\ \isacommand{have}\isamarkupfalse%
\ {\isachardoublequoteopen}X\ {\isasymin}\ M{\isachardoublequoteclose}\ \isacommand{using}\isamarkupfalse%
\ {\isacartoucheopen}X\ {\isacharequal}{\kern0pt}\ {\isacharquery}{\kern0pt}A{\isacartoucheclose}\ \isacommand{by}\isamarkupfalse%
\ auto\isanewline
\isanewline
\ \ \isacommand{have}\isamarkupfalse%
\ {\isachardoublequoteopen}X\ {\isacharequal}{\kern0pt}\ {\isacharbraceleft}{\kern0pt}\ binmap{\isacharunderscore}{\kern0pt}row{\isacharprime}{\kern0pt}{\isacharparenleft}{\kern0pt}n{\isacharparenright}{\kern0pt}{\isachardot}{\kern0pt}\ n\ {\isasymin}\ nat\ {\isacharbraceright}{\kern0pt}{\isachardoublequoteclose}\ {\isacharparenleft}{\kern0pt}\isakeyword{is}\ {\isachardoublequoteopen}{\isacharunderscore}{\kern0pt}\ {\isacharequal}{\kern0pt}\ {\isacharquery}{\kern0pt}B{\isachardoublequoteclose}{\isacharparenright}{\kern0pt}\isanewline
\ \ \ \ \isacommand{unfolding}\isamarkupfalse%
\ X{\isacharunderscore}{\kern0pt}def\ \isanewline
\ \ \ \ \isacommand{apply}\isamarkupfalse%
{\isacharparenleft}{\kern0pt}rule\ equality{\isacharunderscore}{\kern0pt}iffI{\isacharcomma}{\kern0pt}\ rule\ iffI{\isacharcomma}{\kern0pt}\ force{\isacharcomma}{\kern0pt}\ clarsimp{\isacharparenright}{\kern0pt}\isanewline
\ \ \ \ \isacommand{apply}\isamarkupfalse%
{\isacharparenleft}{\kern0pt}rule\ conjI{\isacharcomma}{\kern0pt}\ simp\ add{\isacharcolon}{\kern0pt}binmap{\isacharunderscore}{\kern0pt}row{\isacharprime}{\kern0pt}{\isacharunderscore}{\kern0pt}def{\isacharcomma}{\kern0pt}\ force{\isacharparenright}{\kern0pt}\isanewline
\ \ \ \ \isacommand{apply}\isamarkupfalse%
{\isacharparenleft}{\kern0pt}rule\ conjI{\isacharcomma}{\kern0pt}\ rule\ binmap{\isacharunderscore}{\kern0pt}row{\isacharprime}{\kern0pt}{\isacharunderscore}{\kern0pt}in{\isacharunderscore}{\kern0pt}M{\isacharcomma}{\kern0pt}\ simp{\isacharparenright}{\kern0pt}\isanewline
\ \ \ \ \isacommand{using}\isamarkupfalse%
\ nat{\isacharunderscore}{\kern0pt}in{\isacharunderscore}{\kern0pt}M\ transM\ \isanewline
\ \ \ \ \isacommand{by}\isamarkupfalse%
\ auto\isanewline
\isanewline
\ \ \isacommand{then}\isamarkupfalse%
\ \isacommand{have}\isamarkupfalse%
\ {\isachardoublequoteopen}{\isacharquery}{\kern0pt}B\ {\isasymin}\ M{\isachardoublequoteclose}\ \isacommand{using}\isamarkupfalse%
\ {\isacartoucheopen}X\ {\isasymin}\ M{\isacartoucheclose}\ \isacommand{by}\isamarkupfalse%
\ auto\isanewline
\ \ \isacommand{then}\isamarkupfalse%
\ \isacommand{have}\isamarkupfalse%
\ {\isachardoublequoteopen}{\isacharquery}{\kern0pt}B\ {\isasymtimes}\ {\isacharbraceleft}{\kern0pt}\ {\isadigit{0}}\ {\isacharbraceright}{\kern0pt}\ {\isasymin}\ M{\isachardoublequoteclose}\ \isacommand{using}\isamarkupfalse%
\ singleton{\isacharunderscore}{\kern0pt}in{\isacharunderscore}{\kern0pt}M{\isacharunderscore}{\kern0pt}iff\ one{\isacharunderscore}{\kern0pt}in{\isacharunderscore}{\kern0pt}M\ cartprod{\isacharunderscore}{\kern0pt}closed\ \isacommand{by}\isamarkupfalse%
\ auto\isanewline
\ \ \isacommand{then}\isamarkupfalse%
\ \isacommand{show}\isamarkupfalse%
\ {\isacharquery}{\kern0pt}thesis\ \isanewline
\ \ \ \ \isacommand{using}\isamarkupfalse%
\ binmap{\isacharprime}{\kern0pt}{\isacharunderscore}{\kern0pt}def\isanewline
\ \ \ \ \isacommand{by}\isamarkupfalse%
\ auto\isanewline
\isacommand{qed}\isamarkupfalse%
%
\endisatagproof
{\isafoldproof}%
%
\isadelimproof
\isanewline
%
\endisadelimproof
\isanewline
\isacommand{lemma}\isamarkupfalse%
\ binmap{\isacharunderscore}{\kern0pt}row{\isacharprime}{\kern0pt}{\isacharunderscore}{\kern0pt}P{\isacharunderscore}{\kern0pt}name\ {\isacharcolon}{\kern0pt}\ \isanewline
\ \ \isakeyword{fixes}\ n\ \isanewline
\ \ \isakeyword{assumes}\ {\isachardoublequoteopen}n\ {\isasymin}\ nat{\isachardoublequoteclose}\ \ \isanewline
\ \ \isakeyword{shows}\ {\isachardoublequoteopen}binmap{\isacharunderscore}{\kern0pt}row{\isacharprime}{\kern0pt}{\isacharparenleft}{\kern0pt}n{\isacharparenright}{\kern0pt}\ {\isasymin}\ P{\isacharunderscore}{\kern0pt}names{\isachardoublequoteclose}\ \isanewline
%
\isadelimproof
\isanewline
\ \ %
\endisadelimproof
%
\isatagproof
\isacommand{apply}\isamarkupfalse%
{\isacharparenleft}{\kern0pt}rule\ iffD{\isadigit{2}}{\isacharcomma}{\kern0pt}\ rule\ P{\isacharunderscore}{\kern0pt}name{\isacharunderscore}{\kern0pt}iff{\isacharcomma}{\kern0pt}\ rule\ conjI{\isacharparenright}{\kern0pt}\isanewline
\ \ \ \isacommand{apply}\isamarkupfalse%
{\isacharparenleft}{\kern0pt}rule\ binmap{\isacharunderscore}{\kern0pt}row{\isacharprime}{\kern0pt}{\isacharunderscore}{\kern0pt}in{\isacharunderscore}{\kern0pt}M{\isacharcomma}{\kern0pt}\ simp\ add{\isacharcolon}{\kern0pt}assms{\isacharparenright}{\kern0pt}\isanewline
\ \ \isacommand{unfolding}\isamarkupfalse%
\ binmap{\isacharunderscore}{\kern0pt}row{\isacharprime}{\kern0pt}{\isacharunderscore}{\kern0pt}def\ \isanewline
\ \ \isacommand{using}\isamarkupfalse%
\ check{\isacharunderscore}{\kern0pt}P{\isacharunderscore}{\kern0pt}name\ nat{\isacharunderscore}{\kern0pt}in{\isacharunderscore}{\kern0pt}M\ P{\isacharunderscore}{\kern0pt}name{\isacharunderscore}{\kern0pt}domain{\isacharunderscore}{\kern0pt}P{\isacharunderscore}{\kern0pt}name\isanewline
\ \ \isacommand{by}\isamarkupfalse%
\ auto%
\endisatagproof
{\isafoldproof}%
%
\isadelimproof
\ \ \ \ \isanewline
%
\endisadelimproof
\isanewline
\isacommand{lemma}\isamarkupfalse%
\ binmap{\isacharunderscore}{\kern0pt}row{\isacharprime}{\kern0pt}{\isacharunderscore}{\kern0pt}pauto\ {\isacharcolon}{\kern0pt}\ \isanewline
\ \ \isakeyword{fixes}\ n\ f\ \isanewline
\ \ \isakeyword{assumes}\ {\isachardoublequoteopen}n\ {\isasymin}\ nat{\isachardoublequoteclose}\ {\isachardoublequoteopen}f\ {\isasymin}\ nat{\isacharunderscore}{\kern0pt}perms{\isachardoublequoteclose}\ \isanewline
\ \ \isakeyword{shows}\ {\isachardoublequoteopen}Pn{\isacharunderscore}{\kern0pt}auto{\isacharparenleft}{\kern0pt}Fn{\isacharunderscore}{\kern0pt}perm{\isacharprime}{\kern0pt}{\isacharparenleft}{\kern0pt}f{\isacharparenright}{\kern0pt}{\isacharparenright}{\kern0pt}{\isacharbackquote}{\kern0pt}binmap{\isacharunderscore}{\kern0pt}row{\isacharprime}{\kern0pt}{\isacharparenleft}{\kern0pt}n{\isacharparenright}{\kern0pt}\ {\isacharequal}{\kern0pt}\ binmap{\isacharunderscore}{\kern0pt}row{\isacharprime}{\kern0pt}{\isacharparenleft}{\kern0pt}f{\isacharbackquote}{\kern0pt}n{\isacharparenright}{\kern0pt}{\isachardoublequoteclose}\ {\isacharparenleft}{\kern0pt}\isakeyword{is}\ {\isachardoublequoteopen}{\isacharquery}{\kern0pt}A\ {\isacharequal}{\kern0pt}\ {\isacharquery}{\kern0pt}B{\isachardoublequoteclose}{\isacharparenright}{\kern0pt}\isanewline
%
\isadelimproof
%
\endisadelimproof
%
\isatagproof
\isacommand{proof}\isamarkupfalse%
\ {\isacharminus}{\kern0pt}\ \isanewline
\isanewline
\ \ \isacommand{have}\isamarkupfalse%
\ H{\isacharcolon}{\kern0pt}\ {\isachardoublequoteopen}{\isasymAnd}p\ m{\isachardot}{\kern0pt}\ p\ {\isasymin}\ Fn\ {\isasymLongrightarrow}\ m\ {\isasymin}\ nat\ {\isasymLongrightarrow}\ {\isacharless}{\kern0pt}n{\isacharcomma}{\kern0pt}\ m{\isachargreater}{\kern0pt}\ {\isasymin}\ domain{\isacharparenleft}{\kern0pt}p{\isacharparenright}{\kern0pt}\ {\isasymLongrightarrow}\ p{\isacharbackquote}{\kern0pt}{\isacharless}{\kern0pt}n{\isacharcomma}{\kern0pt}\ m{\isachargreater}{\kern0pt}\ {\isacharequal}{\kern0pt}\ {\isadigit{1}}\ {\isasymlongleftrightarrow}\ {\isacharparenleft}{\kern0pt}Fn{\isacharunderscore}{\kern0pt}perm{\isacharprime}{\kern0pt}{\isacharparenleft}{\kern0pt}f{\isacharparenright}{\kern0pt}{\isacharbackquote}{\kern0pt}p{\isacharparenright}{\kern0pt}{\isacharbackquote}{\kern0pt}{\isacharless}{\kern0pt}f{\isacharbackquote}{\kern0pt}n{\isacharcomma}{\kern0pt}\ m{\isachargreater}{\kern0pt}\ {\isacharequal}{\kern0pt}\ {\isadigit{1}}{\isachardoublequoteclose}\ \isanewline
\ \ \isacommand{proof}\isamarkupfalse%
\ {\isacharminus}{\kern0pt}\ \isanewline
\ \ \ \ \isacommand{fix}\isamarkupfalse%
\ p\ m\ \isanewline
\ \ \ \ \isacommand{assume}\isamarkupfalse%
\ assms{\isadigit{1}}{\isacharcolon}{\kern0pt}\ {\isachardoublequoteopen}p\ {\isasymin}\ Fn{\isachardoublequoteclose}\ {\isachardoublequoteopen}m\ {\isasymin}\ nat{\isachardoublequoteclose}\ {\isachardoublequoteopen}{\isacharless}{\kern0pt}n{\isacharcomma}{\kern0pt}\ m{\isachargreater}{\kern0pt}\ {\isasymin}\ domain{\isacharparenleft}{\kern0pt}p{\isacharparenright}{\kern0pt}{\isachardoublequoteclose}\ \isanewline
\isanewline
\ \ \ \ \isacommand{then}\isamarkupfalse%
\ \isacommand{obtain}\isamarkupfalse%
\ l\ \isakeyword{where}\ {\isachardoublequoteopen}{\isacharless}{\kern0pt}{\isacharless}{\kern0pt}n{\isacharcomma}{\kern0pt}\ m{\isachargreater}{\kern0pt}{\isacharcomma}{\kern0pt}\ l{\isachargreater}{\kern0pt}\ {\isasymin}\ p{\isachardoublequoteclose}\ \isacommand{by}\isamarkupfalse%
\ auto\isanewline
\ \ \ \ \isacommand{then}\isamarkupfalse%
\ \isacommand{have}\isamarkupfalse%
\ {\isachardoublequoteopen}{\isacharless}{\kern0pt}{\isacharless}{\kern0pt}f{\isacharbackquote}{\kern0pt}n{\isacharcomma}{\kern0pt}\ m{\isachargreater}{\kern0pt}{\isacharcomma}{\kern0pt}\ l{\isachargreater}{\kern0pt}\ {\isasymin}\ Fn{\isacharunderscore}{\kern0pt}perm{\isacharparenleft}{\kern0pt}f{\isacharcomma}{\kern0pt}\ p{\isacharparenright}{\kern0pt}{\isachardoublequoteclose}\ \isanewline
\ \ \ \ \ \ \isacommand{unfolding}\isamarkupfalse%
\ Fn{\isacharunderscore}{\kern0pt}perm{\isacharunderscore}{\kern0pt}def\ \isanewline
\ \ \ \ \ \ \isacommand{by}\isamarkupfalse%
\ force\ \isanewline
\ \ \ \ \isacommand{then}\isamarkupfalse%
\ \isacommand{have}\isamarkupfalse%
\ domin{\isacharcolon}{\kern0pt}\ \ {\isachardoublequoteopen}{\isacharless}{\kern0pt}f{\isacharbackquote}{\kern0pt}n{\isacharcomma}{\kern0pt}\ m{\isachargreater}{\kern0pt}\ {\isasymin}\ domain{\isacharparenleft}{\kern0pt}Fn{\isacharunderscore}{\kern0pt}perm{\isacharparenleft}{\kern0pt}f{\isacharcomma}{\kern0pt}\ p{\isacharparenright}{\kern0pt}{\isacharparenright}{\kern0pt}{\isachardoublequoteclose}\ \isacommand{by}\isamarkupfalse%
\ auto\isanewline
\isanewline
\ \ \ \ \isacommand{have}\isamarkupfalse%
\ {\isachardoublequoteopen}p{\isacharbackquote}{\kern0pt}{\isacharless}{\kern0pt}n{\isacharcomma}{\kern0pt}\ m{\isachargreater}{\kern0pt}\ {\isacharequal}{\kern0pt}\ {\isadigit{1}}\ {\isasymlongleftrightarrow}\ {\isacharless}{\kern0pt}{\isacharless}{\kern0pt}n{\isacharcomma}{\kern0pt}\ m{\isachargreater}{\kern0pt}{\isacharcomma}{\kern0pt}\ {\isadigit{1}}{\isachargreater}{\kern0pt}\ {\isasymin}\ p{\isachardoublequoteclose}\ \isanewline
\ \ \ \ \ \ \isacommand{apply}\isamarkupfalse%
{\isacharparenleft}{\kern0pt}rule\ iffI{\isacharparenright}{\kern0pt}\isanewline
\ \ \ \ \ \ \isacommand{apply}\isamarkupfalse%
{\isacharparenleft}{\kern0pt}rule{\isacharunderscore}{\kern0pt}tac\ b{\isacharequal}{\kern0pt}{\isadigit{1}}\ \isakeyword{and}\ a{\isacharequal}{\kern0pt}{\isachardoublequoteopen}p{\isacharbackquote}{\kern0pt}{\isacharless}{\kern0pt}n{\isacharcomma}{\kern0pt}m{\isachargreater}{\kern0pt}{\isachardoublequoteclose}\ \isakeyword{in}\ ssubst{\isacharcomma}{\kern0pt}\ simp{\isacharparenright}{\kern0pt}\isanewline
\ \ \ \ \ \ \ \isacommand{apply}\isamarkupfalse%
{\isacharparenleft}{\kern0pt}rule\ function{\isacharunderscore}{\kern0pt}apply{\isacharunderscore}{\kern0pt}Pair{\isacharparenright}{\kern0pt}\isanewline
\ \ \ \ \ \ \isacommand{using}\isamarkupfalse%
\ assms{\isadigit{1}}\ Fn{\isacharunderscore}{\kern0pt}def\ \isanewline
\ \ \ \ \ \ \ \ \isacommand{apply}\isamarkupfalse%
\ auto{\isacharbrackleft}{\kern0pt}{\isadigit{2}}{\isacharbrackright}{\kern0pt}\isanewline
\ \ \ \ \ \ \isacommand{apply}\isamarkupfalse%
{\isacharparenleft}{\kern0pt}rule\ function{\isacharunderscore}{\kern0pt}apply{\isacharunderscore}{\kern0pt}equality{\isacharparenright}{\kern0pt}\isanewline
\ \ \ \ \ \ \isacommand{using}\isamarkupfalse%
\ assms{\isadigit{1}}\ Fn{\isacharunderscore}{\kern0pt}def\ \isanewline
\ \ \ \ \ \ \ \isacommand{apply}\isamarkupfalse%
\ auto{\isacharbrackleft}{\kern0pt}{\isadigit{2}}{\isacharbrackright}{\kern0pt}\isanewline
\ \ \ \ \ \ \isacommand{done}\isamarkupfalse%
\isanewline
\ \ \ \ \isacommand{also}\isamarkupfalse%
\ \isacommand{have}\isamarkupfalse%
\ {\isachardoublequoteopen}{\isachardot}{\kern0pt}{\isachardot}{\kern0pt}{\isachardot}{\kern0pt}\ {\isasymlongleftrightarrow}\ {\isacharless}{\kern0pt}{\isacharless}{\kern0pt}f{\isacharbackquote}{\kern0pt}n{\isacharcomma}{\kern0pt}\ m{\isachargreater}{\kern0pt}{\isacharcomma}{\kern0pt}\ {\isadigit{1}}{\isachargreater}{\kern0pt}\ {\isasymin}\ Fn{\isacharunderscore}{\kern0pt}perm{\isacharparenleft}{\kern0pt}f{\isacharcomma}{\kern0pt}\ p{\isacharparenright}{\kern0pt}{\isachardoublequoteclose}\ \isanewline
\ \ \ \ \isacommand{proof}\isamarkupfalse%
{\isacharparenleft}{\kern0pt}rule\ iffI{\isacharparenright}{\kern0pt}\isanewline
\ \ \ \ \ \ \isacommand{assume}\isamarkupfalse%
\ {\isachardoublequoteopen}{\isasymlangle}{\isasymlangle}n{\isacharcomma}{\kern0pt}\ m{\isasymrangle}{\isacharcomma}{\kern0pt}\ {\isadigit{1}}{\isasymrangle}\ {\isasymin}\ p{\isachardoublequoteclose}\ \isanewline
\ \ \ \ \ \ \isacommand{then}\isamarkupfalse%
\ \isacommand{show}\isamarkupfalse%
\ {\isachardoublequoteopen}{\isasymlangle}{\isasymlangle}f\ {\isacharbackquote}{\kern0pt}\ n{\isacharcomma}{\kern0pt}\ m{\isasymrangle}{\isacharcomma}{\kern0pt}\ {\isadigit{1}}{\isasymrangle}\ {\isasymin}\ Fn{\isacharunderscore}{\kern0pt}perm{\isacharparenleft}{\kern0pt}f{\isacharcomma}{\kern0pt}\ p{\isacharparenright}{\kern0pt}{\isachardoublequoteclose}\ \isanewline
\ \ \ \ \ \ \ \ \isacommand{unfolding}\isamarkupfalse%
\ Fn{\isacharunderscore}{\kern0pt}perm{\isacharunderscore}{\kern0pt}def\ \isanewline
\ \ \ \ \ \ \ \ \isacommand{by}\isamarkupfalse%
\ force\ \isanewline
\ \ \ \ \isacommand{next}\isamarkupfalse%
\ \isanewline
\ \ \ \ \ \ \isacommand{assume}\isamarkupfalse%
\ {\isachardoublequoteopen}{\isasymlangle}{\isasymlangle}f\ {\isacharbackquote}{\kern0pt}\ n{\isacharcomma}{\kern0pt}\ m{\isasymrangle}{\isacharcomma}{\kern0pt}\ {\isadigit{1}}{\isasymrangle}\ {\isasymin}\ Fn{\isacharunderscore}{\kern0pt}perm{\isacharparenleft}{\kern0pt}f{\isacharcomma}{\kern0pt}\ p{\isacharparenright}{\kern0pt}{\isachardoublequoteclose}\ \isanewline
\ \ \ \ \ \ \isacommand{then}\isamarkupfalse%
\ \isacommand{have}\isamarkupfalse%
\ {\isachardoublequoteopen}{\isasymexists}n{\isacharprime}{\kern0pt}{\isasymin}nat{\isachardot}{\kern0pt}\ {\isasymexists}m{\isacharprime}{\kern0pt}{\isasymin}nat{\isachardot}{\kern0pt}\ {\isasymexists}l\ {\isasymin}\ {\isadigit{2}}{\isachardot}{\kern0pt}\ {\isasymlangle}{\isasymlangle}n{\isacharprime}{\kern0pt}{\isacharcomma}{\kern0pt}\ m{\isacharprime}{\kern0pt}{\isasymrangle}{\isacharcomma}{\kern0pt}\ l{\isasymrangle}\ {\isasymin}\ p\ {\isasymand}\ {\isasymlangle}{\isasymlangle}f\ {\isacharbackquote}{\kern0pt}\ n{\isacharcomma}{\kern0pt}\ m{\isasymrangle}{\isacharcomma}{\kern0pt}\ {\isadigit{1}}{\isasymrangle}\ {\isacharequal}{\kern0pt}\ {\isasymlangle}{\isasymlangle}f\ {\isacharbackquote}{\kern0pt}\ n{\isacharprime}{\kern0pt}{\isacharcomma}{\kern0pt}\ m{\isacharprime}{\kern0pt}{\isasymrangle}{\isacharcomma}{\kern0pt}\ l{\isasymrangle}{\isachardoublequoteclose}\ \isanewline
\ \ \ \ \ \ \ \ \isacommand{apply}\isamarkupfalse%
{\isacharparenleft}{\kern0pt}rule{\isacharunderscore}{\kern0pt}tac\ Fn{\isacharunderscore}{\kern0pt}permE{\isacharparenright}{\kern0pt}\isanewline
\ \ \ \ \ \ \ \ \isacommand{using}\isamarkupfalse%
\ assms{\isadigit{1}}\ assms\ \isanewline
\ \ \ \ \ \ \ \ \isacommand{by}\isamarkupfalse%
\ auto\isanewline
\ \ \ \ \ \ \isacommand{then}\isamarkupfalse%
\ \isacommand{obtain}\isamarkupfalse%
\ n{\isacharprime}{\kern0pt}\ \isakeyword{where}\ H{\isacharcolon}{\kern0pt}\ {\isachardoublequoteopen}n{\isacharprime}{\kern0pt}\ {\isasymin}\ nat{\isachardoublequoteclose}\ {\isachardoublequoteopen}{\isasymlangle}{\isasymlangle}n{\isacharprime}{\kern0pt}{\isacharcomma}{\kern0pt}\ m{\isasymrangle}{\isacharcomma}{\kern0pt}\ {\isadigit{1}}{\isasymrangle}\ {\isasymin}\ p{\isachardoublequoteclose}\ {\isachardoublequoteopen}{\isasymlangle}{\isasymlangle}f\ {\isacharbackquote}{\kern0pt}\ n{\isacharcomma}{\kern0pt}\ m{\isasymrangle}{\isacharcomma}{\kern0pt}\ {\isadigit{1}}{\isasymrangle}\ {\isacharequal}{\kern0pt}\ {\isasymlangle}{\isasymlangle}f\ {\isacharbackquote}{\kern0pt}\ n{\isacharprime}{\kern0pt}{\isacharcomma}{\kern0pt}\ m{\isasymrangle}{\isacharcomma}{\kern0pt}\ {\isadigit{1}}{\isasymrangle}{\isachardoublequoteclose}\ \isanewline
\ \ \ \ \ \ \ \ \isacommand{by}\isamarkupfalse%
\ force\ \isanewline
\ \ \ \ \ \ \isacommand{then}\isamarkupfalse%
\ \isacommand{have}\isamarkupfalse%
\ {\isachardoublequoteopen}f{\isacharbackquote}{\kern0pt}n\ {\isacharequal}{\kern0pt}\ f{\isacharbackquote}{\kern0pt}n{\isacharprime}{\kern0pt}{\isachardoublequoteclose}\ \isacommand{by}\isamarkupfalse%
\ auto\ \isanewline
\ \ \ \ \ \ \isacommand{then}\isamarkupfalse%
\ \isacommand{have}\isamarkupfalse%
\ {\isachardoublequoteopen}n\ {\isacharequal}{\kern0pt}\ n{\isacharprime}{\kern0pt}{\isachardoublequoteclose}\ \isacommand{using}\isamarkupfalse%
\ assms\ nat{\isacharunderscore}{\kern0pt}perms{\isacharunderscore}{\kern0pt}def\ bij{\isacharunderscore}{\kern0pt}def\ inj{\isacharunderscore}{\kern0pt}def\ H\ \isacommand{by}\isamarkupfalse%
\ force\isanewline
\ \ \ \ \ \ \isacommand{then}\isamarkupfalse%
\ \isacommand{show}\isamarkupfalse%
\ {\isachardoublequoteopen}{\isasymlangle}{\isasymlangle}n{\isacharcomma}{\kern0pt}\ m{\isasymrangle}{\isacharcomma}{\kern0pt}\ {\isadigit{1}}{\isasymrangle}\ {\isasymin}\ p{\isachardoublequoteclose}\ \isacommand{using}\isamarkupfalse%
\ H\ \isacommand{by}\isamarkupfalse%
\ auto\isanewline
\ \ \ \ \isacommand{qed}\isamarkupfalse%
\isanewline
\ \ \ \ \isacommand{also}\isamarkupfalse%
\ \isacommand{have}\isamarkupfalse%
\ {\isachardoublequoteopen}{\isachardot}{\kern0pt}{\isachardot}{\kern0pt}{\isachardot}{\kern0pt}\ {\isasymlongleftrightarrow}\ Fn{\isacharunderscore}{\kern0pt}perm{\isacharparenleft}{\kern0pt}f{\isacharcomma}{\kern0pt}\ p{\isacharparenright}{\kern0pt}{\isacharbackquote}{\kern0pt}{\isacharless}{\kern0pt}f{\isacharbackquote}{\kern0pt}n{\isacharcomma}{\kern0pt}\ m{\isachargreater}{\kern0pt}\ {\isacharequal}{\kern0pt}\ {\isadigit{1}}{\isachardoublequoteclose}\ \isanewline
\ \ \ \ \ \ \isacommand{apply}\isamarkupfalse%
{\isacharparenleft}{\kern0pt}rule\ iffI{\isacharcomma}{\kern0pt}\ rule\ function{\isacharunderscore}{\kern0pt}apply{\isacharunderscore}{\kern0pt}equality{\isacharcomma}{\kern0pt}\ simp{\isacharparenright}{\kern0pt}\isanewline
\ \ \ \ \ \ \ \isacommand{apply}\isamarkupfalse%
{\isacharparenleft}{\kern0pt}rule\ function{\isacharunderscore}{\kern0pt}Fn{\isacharunderscore}{\kern0pt}perm{\isacharparenright}{\kern0pt}\isanewline
\ \ \ \ \ \ \isacommand{using}\isamarkupfalse%
\ assms{\isadigit{1}}\ assms\isanewline
\ \ \ \ \ \ \ \ \isacommand{apply}\isamarkupfalse%
\ auto{\isacharbrackleft}{\kern0pt}{\isadigit{2}}{\isacharbrackright}{\kern0pt}\isanewline
\ \ \ \ \ \ \isacommand{apply}\isamarkupfalse%
{\isacharparenleft}{\kern0pt}rule{\isacharunderscore}{\kern0pt}tac\ b{\isacharequal}{\kern0pt}{\isadigit{1}}\ \isakeyword{and}\ a{\isacharequal}{\kern0pt}{\isachardoublequoteopen}Fn{\isacharunderscore}{\kern0pt}perm{\isacharparenleft}{\kern0pt}f{\isacharcomma}{\kern0pt}\ p{\isacharparenright}{\kern0pt}\ {\isacharbackquote}{\kern0pt}\ {\isasymlangle}f\ {\isacharbackquote}{\kern0pt}\ n{\isacharcomma}{\kern0pt}\ m{\isasymrangle}{\isachardoublequoteclose}\ \isakeyword{in}\ ssubst{\isacharcomma}{\kern0pt}\ simp{\isacharparenright}{\kern0pt}\isanewline
\ \ \ \ \ \ \isacommand{apply}\isamarkupfalse%
{\isacharparenleft}{\kern0pt}rule\ function{\isacharunderscore}{\kern0pt}apply{\isacharunderscore}{\kern0pt}Pair{\isacharparenright}{\kern0pt}\isanewline
\ \ \ \ \ \ \ \isacommand{apply}\isamarkupfalse%
{\isacharparenleft}{\kern0pt}rule\ function{\isacharunderscore}{\kern0pt}Fn{\isacharunderscore}{\kern0pt}perm{\isacharparenright}{\kern0pt}\isanewline
\ \ \ \ \ \ \isacommand{using}\isamarkupfalse%
\ assms{\isadigit{1}}\ assms\ domin\isanewline
\ \ \ \ \ \ \ \ \isacommand{apply}\isamarkupfalse%
\ auto{\isacharbrackleft}{\kern0pt}{\isadigit{3}}{\isacharbrackright}{\kern0pt}\isanewline
\ \ \ \ \ \ \isacommand{done}\isamarkupfalse%
\isanewline
\ \ \ \ \isacommand{also}\isamarkupfalse%
\ \isacommand{have}\isamarkupfalse%
\ {\isachardoublequoteopen}{\isachardot}{\kern0pt}{\isachardot}{\kern0pt}{\isachardot}{\kern0pt}\ {\isasymlongleftrightarrow}\ Fn{\isacharunderscore}{\kern0pt}perm{\isacharprime}{\kern0pt}{\isacharparenleft}{\kern0pt}f{\isacharparenright}{\kern0pt}\ {\isacharbackquote}{\kern0pt}\ p\ {\isacharbackquote}{\kern0pt}\ {\isasymlangle}f\ {\isacharbackquote}{\kern0pt}\ n{\isacharcomma}{\kern0pt}\ m{\isasymrangle}\ {\isacharequal}{\kern0pt}\ {\isadigit{1}}{\isachardoublequoteclose}\ \ \isanewline
\ \ \ \ \ \ \isacommand{apply}\isamarkupfalse%
{\isacharparenleft}{\kern0pt}subst\ Fn{\isacharunderscore}{\kern0pt}perm{\isacharprime}{\kern0pt}{\isacharunderscore}{\kern0pt}eq{\isacharparenright}{\kern0pt}\isanewline
\ \ \ \ \ \ \isacommand{using}\isamarkupfalse%
\ assms\ assms{\isadigit{1}}\isanewline
\ \ \ \ \ \ \isacommand{by}\isamarkupfalse%
\ auto\isanewline
\ \ \ \ \isacommand{finally}\isamarkupfalse%
\ \isacommand{show}\isamarkupfalse%
\ {\isachardoublequoteopen}p\ {\isacharbackquote}{\kern0pt}\ {\isasymlangle}n{\isacharcomma}{\kern0pt}\ m{\isasymrangle}\ {\isacharequal}{\kern0pt}\ {\isadigit{1}}\ {\isasymlongleftrightarrow}\ Fn{\isacharunderscore}{\kern0pt}perm{\isacharprime}{\kern0pt}{\isacharparenleft}{\kern0pt}f{\isacharparenright}{\kern0pt}\ {\isacharbackquote}{\kern0pt}\ p\ {\isacharbackquote}{\kern0pt}\ {\isasymlangle}f\ {\isacharbackquote}{\kern0pt}\ n{\isacharcomma}{\kern0pt}\ m{\isasymrangle}\ {\isacharequal}{\kern0pt}\ {\isadigit{1}}{\isachardoublequoteclose}\ \isacommand{by}\isamarkupfalse%
\ simp\isanewline
\ \ \isacommand{qed}\isamarkupfalse%
\isanewline
\isanewline
\ \ \isacommand{have}\isamarkupfalse%
\ apply{\isacharunderscore}{\kern0pt}not{\isadigit{0}}{\isacharunderscore}{\kern0pt}indom\ {\isacharcolon}{\kern0pt}\ {\isachardoublequoteopen}{\isasymAnd}f\ x{\isachardot}{\kern0pt}\ function{\isacharparenleft}{\kern0pt}f{\isacharparenright}{\kern0pt}\ {\isasymLongrightarrow}\ f{\isacharbackquote}{\kern0pt}x\ {\isasymnoteq}\ {\isadigit{0}}\ {\isasymLongrightarrow}\ x\ {\isasymin}\ domain{\isacharparenleft}{\kern0pt}f{\isacharparenright}{\kern0pt}{\isachardoublequoteclose}\ \isanewline
\ \ \ \ \isacommand{apply}\isamarkupfalse%
{\isacharparenleft}{\kern0pt}rule\ ccontr{\isacharparenright}{\kern0pt}\isanewline
\ \ \ \ \isacommand{apply}\isamarkupfalse%
{\isacharparenleft}{\kern0pt}rename{\isacharunderscore}{\kern0pt}tac\ f\ x{\isacharcomma}{\kern0pt}\ subgoal{\isacharunderscore}{\kern0pt}tac\ {\isachardoublequoteopen}f{\isacharbackquote}{\kern0pt}x\ {\isacharequal}{\kern0pt}\ {\isadigit{0}}{\isachardoublequoteclose}{\isacharcomma}{\kern0pt}\ simp{\isacharparenright}{\kern0pt}\isanewline
\ \ \ \ \isacommand{apply}\isamarkupfalse%
{\isacharparenleft}{\kern0pt}rule\ apply{\isacharunderscore}{\kern0pt}{\isadigit{0}}{\isacharparenright}{\kern0pt}\isanewline
\ \ \ \ \isacommand{by}\isamarkupfalse%
\ auto\isanewline
\isanewline
\ \ \isacommand{have}\isamarkupfalse%
\ indom\ {\isacharcolon}{\kern0pt}\ {\isachardoublequoteopen}{\isasymAnd}m\ p{\isachardot}{\kern0pt}\ m\ {\isasymin}\ nat\ {\isasymLongrightarrow}\ Fn{\isacharunderscore}{\kern0pt}perm{\isacharprime}{\kern0pt}{\isacharparenleft}{\kern0pt}f{\isacharparenright}{\kern0pt}\ {\isacharbackquote}{\kern0pt}\ p\ {\isasymin}\ Fn\ {\isasymLongrightarrow}\ Fn{\isacharunderscore}{\kern0pt}perm{\isacharprime}{\kern0pt}{\isacharparenleft}{\kern0pt}f{\isacharparenright}{\kern0pt}\ {\isacharbackquote}{\kern0pt}\ p\ {\isacharbackquote}{\kern0pt}\ {\isasymlangle}f\ {\isacharbackquote}{\kern0pt}\ n{\isacharcomma}{\kern0pt}\ m{\isasymrangle}\ {\isacharequal}{\kern0pt}\ {\isadigit{1}}\ {\isasymLongrightarrow}\ p\ {\isasymin}\ Fn\ {\isasymLongrightarrow}\ {\isasymlangle}n{\isacharcomma}{\kern0pt}\ m{\isasymrangle}\ {\isasymin}\ domain{\isacharparenleft}{\kern0pt}p{\isacharparenright}{\kern0pt}{\isachardoublequoteclose}\isanewline
\ \ \isacommand{proof}\isamarkupfalse%
\ {\isacharminus}{\kern0pt}\ \isanewline
\ \ \ \ \isacommand{fix}\isamarkupfalse%
\ m\ p\ \isanewline
\ \ \ \ \isacommand{assume}\isamarkupfalse%
\ assms{\isadigit{1}}\ {\isacharcolon}{\kern0pt}\ {\isachardoublequoteopen}m\ {\isasymin}\ nat{\isachardoublequoteclose}\ {\isachardoublequoteopen}Fn{\isacharunderscore}{\kern0pt}perm{\isacharprime}{\kern0pt}{\isacharparenleft}{\kern0pt}f{\isacharparenright}{\kern0pt}\ {\isacharbackquote}{\kern0pt}\ p\ {\isasymin}\ Fn{\isachardoublequoteclose}\ {\isachardoublequoteopen}Fn{\isacharunderscore}{\kern0pt}perm{\isacharprime}{\kern0pt}{\isacharparenleft}{\kern0pt}f{\isacharparenright}{\kern0pt}\ {\isacharbackquote}{\kern0pt}\ p\ {\isacharbackquote}{\kern0pt}\ {\isasymlangle}f\ {\isacharbackquote}{\kern0pt}\ n{\isacharcomma}{\kern0pt}\ m{\isasymrangle}\ {\isacharequal}{\kern0pt}\ {\isadigit{1}}{\isachardoublequoteclose}\ {\isachardoublequoteopen}p\ {\isasymin}\ Fn{\isachardoublequoteclose}\ \isanewline
\ \ \ \ \isacommand{have}\isamarkupfalse%
\ {\isachardoublequoteopen}{\isasymlangle}f\ {\isacharbackquote}{\kern0pt}\ n{\isacharcomma}{\kern0pt}\ m{\isasymrangle}\ {\isasymin}\ domain{\isacharparenleft}{\kern0pt}Fn{\isacharunderscore}{\kern0pt}perm{\isacharprime}{\kern0pt}{\isacharparenleft}{\kern0pt}f{\isacharparenright}{\kern0pt}\ {\isacharbackquote}{\kern0pt}\ p{\isacharparenright}{\kern0pt}{\isachardoublequoteclose}\ \isanewline
\ \ \ \ \ \ \isacommand{apply}\isamarkupfalse%
{\isacharparenleft}{\kern0pt}rule\ apply{\isacharunderscore}{\kern0pt}not{\isadigit{0}}{\isacharunderscore}{\kern0pt}indom{\isacharparenright}{\kern0pt}\isanewline
\ \ \ \ \ \ \ \isacommand{apply}\isamarkupfalse%
{\isacharparenleft}{\kern0pt}subgoal{\isacharunderscore}{\kern0pt}tac\ {\isachardoublequoteopen}Fn{\isacharunderscore}{\kern0pt}perm{\isacharprime}{\kern0pt}{\isacharparenleft}{\kern0pt}f{\isacharparenright}{\kern0pt}\ {\isacharbackquote}{\kern0pt}\ p\ {\isasymin}\ Fn{\isachardoublequoteclose}{\isacharparenright}{\kern0pt}\isanewline
\ \ \ \ \ \ \isacommand{using}\isamarkupfalse%
\ Fn{\isacharunderscore}{\kern0pt}def\ \isanewline
\ \ \ \ \ \ \ \ \isacommand{apply}\isamarkupfalse%
\ force\isanewline
\ \ \ \ \ \ \isacommand{apply}\isamarkupfalse%
{\isacharparenleft}{\kern0pt}rule\ function{\isacharunderscore}{\kern0pt}value{\isacharunderscore}{\kern0pt}in{\isacharcomma}{\kern0pt}\ rule\ Fn{\isacharunderscore}{\kern0pt}perm{\isacharprime}{\kern0pt}{\isacharunderscore}{\kern0pt}type{\isacharparenright}{\kern0pt}\isanewline
\ \ \ \ \ \ \isacommand{using}\isamarkupfalse%
\ assms\ assms{\isadigit{1}}\ \isanewline
\ \ \ \ \ \ \isacommand{by}\isamarkupfalse%
\ auto\isanewline
\ \ \ \ \isacommand{then}\isamarkupfalse%
\ \isacommand{obtain}\isamarkupfalse%
\ l\ \isakeyword{where}\ {\isachardoublequoteopen}{\isacharless}{\kern0pt}{\isacharless}{\kern0pt}f{\isacharbackquote}{\kern0pt}n{\isacharcomma}{\kern0pt}\ m{\isachargreater}{\kern0pt}{\isacharcomma}{\kern0pt}\ l{\isachargreater}{\kern0pt}\ {\isasymin}\ Fn{\isacharunderscore}{\kern0pt}perm{\isacharprime}{\kern0pt}{\isacharparenleft}{\kern0pt}f{\isacharparenright}{\kern0pt}\ {\isacharbackquote}{\kern0pt}\ p{\isachardoublequoteclose}\ \isacommand{by}\isamarkupfalse%
\ auto\isanewline
\ \ \ \ \isacommand{then}\isamarkupfalse%
\ \isacommand{have}\isamarkupfalse%
\ {\isachardoublequoteopen}{\isacharless}{\kern0pt}{\isacharless}{\kern0pt}f{\isacharbackquote}{\kern0pt}n{\isacharcomma}{\kern0pt}\ m{\isachargreater}{\kern0pt}{\isacharcomma}{\kern0pt}\ l{\isachargreater}{\kern0pt}\ {\isasymin}\ Fn{\isacharunderscore}{\kern0pt}perm{\isacharparenleft}{\kern0pt}f{\isacharcomma}{\kern0pt}\ p{\isacharparenright}{\kern0pt}{\isachardoublequoteclose}\ \isanewline
\ \ \ \ \ \ \isacommand{using}\isamarkupfalse%
\ Fn{\isacharunderscore}{\kern0pt}perm{\isacharprime}{\kern0pt}{\isacharunderscore}{\kern0pt}eq\ assms\ assms{\isadigit{1}}\isanewline
\ \ \ \ \ \ \isacommand{by}\isamarkupfalse%
\ auto\ \isanewline
\ \ \ \ \isacommand{then}\isamarkupfalse%
\ \isacommand{have}\isamarkupfalse%
\ {\isachardoublequoteopen}{\isasymexists}n{\isacharprime}{\kern0pt}{\isasymin}nat{\isachardot}{\kern0pt}\ {\isasymexists}m{\isacharprime}{\kern0pt}{\isasymin}nat{\isachardot}{\kern0pt}\ {\isasymexists}l{\isacharprime}{\kern0pt}\ {\isasymin}\ {\isadigit{2}}{\isachardot}{\kern0pt}\ {\isasymlangle}{\isasymlangle}n{\isacharprime}{\kern0pt}{\isacharcomma}{\kern0pt}\ m{\isacharprime}{\kern0pt}{\isasymrangle}{\isacharcomma}{\kern0pt}\ l{\isacharprime}{\kern0pt}{\isasymrangle}\ {\isasymin}\ p\ {\isasymand}\ {\isasymlangle}{\isasymlangle}f\ {\isacharbackquote}{\kern0pt}\ n{\isacharcomma}{\kern0pt}\ m{\isasymrangle}{\isacharcomma}{\kern0pt}\ l{\isasymrangle}\ {\isacharequal}{\kern0pt}\ {\isasymlangle}{\isasymlangle}f\ {\isacharbackquote}{\kern0pt}\ n{\isacharprime}{\kern0pt}{\isacharcomma}{\kern0pt}\ m{\isacharprime}{\kern0pt}{\isasymrangle}{\isacharcomma}{\kern0pt}\ l{\isacharprime}{\kern0pt}{\isasymrangle}{\isachardoublequoteclose}\ \isanewline
\ \ \ \ \ \ \ \ \isacommand{apply}\isamarkupfalse%
{\isacharparenleft}{\kern0pt}rule{\isacharunderscore}{\kern0pt}tac\ Fn{\isacharunderscore}{\kern0pt}permE{\isacharparenright}{\kern0pt}\isanewline
\ \ \ \ \ \ \ \ \isacommand{using}\isamarkupfalse%
\ assms{\isadigit{1}}\ assms\ \isanewline
\ \ \ \ \ \ \ \ \isacommand{by}\isamarkupfalse%
\ auto\isanewline
\ \ \ \ \isacommand{then}\isamarkupfalse%
\ \isacommand{obtain}\isamarkupfalse%
\ n{\isacharprime}{\kern0pt}\ \isakeyword{where}\ n{\isacharprime}{\kern0pt}H\ {\isacharcolon}{\kern0pt}\ {\isachardoublequoteopen}f{\isacharbackquote}{\kern0pt}n\ {\isacharequal}{\kern0pt}\ f{\isacharbackquote}{\kern0pt}n{\isacharprime}{\kern0pt}{\isachardoublequoteclose}\ {\isachardoublequoteopen}n{\isacharprime}{\kern0pt}\ {\isasymin}\ nat{\isachardoublequoteclose}\ {\isachardoublequoteopen}{\isacharless}{\kern0pt}{\isacharless}{\kern0pt}n{\isacharprime}{\kern0pt}{\isacharcomma}{\kern0pt}\ m{\isachargreater}{\kern0pt}{\isacharcomma}{\kern0pt}\ l{\isachargreater}{\kern0pt}\ {\isasymin}\ p{\isachardoublequoteclose}\ \isacommand{by}\isamarkupfalse%
\ auto\isanewline
\ \ \ \ \isacommand{then}\isamarkupfalse%
\ \isacommand{have}\isamarkupfalse%
\ {\isachardoublequoteopen}n\ {\isacharequal}{\kern0pt}\ n{\isacharprime}{\kern0pt}{\isachardoublequoteclose}\ \isacommand{using}\isamarkupfalse%
\ assms\ nat{\isacharunderscore}{\kern0pt}perms{\isacharunderscore}{\kern0pt}def\ bij{\isacharunderscore}{\kern0pt}def\ inj{\isacharunderscore}{\kern0pt}def\ \isacommand{by}\isamarkupfalse%
\ auto\isanewline
\ \ \ \ \isacommand{then}\isamarkupfalse%
\ \isacommand{show}\isamarkupfalse%
\ {\isachardoublequoteopen}{\isacharless}{\kern0pt}n{\isacharcomma}{\kern0pt}\ m{\isachargreater}{\kern0pt}\ {\isasymin}\ domain{\isacharparenleft}{\kern0pt}p{\isacharparenright}{\kern0pt}{\isachardoublequoteclose}\ \isacommand{using}\isamarkupfalse%
\ n{\isacharprime}{\kern0pt}H\ \isacommand{by}\isamarkupfalse%
\ auto\isanewline
\ \ \isacommand{qed}\isamarkupfalse%
\isanewline
\ \ \ \ \ \isanewline
\ \ \isacommand{have}\isamarkupfalse%
\ {\isachardoublequoteopen}{\isacharquery}{\kern0pt}A\ {\isacharequal}{\kern0pt}\ {\isacharbraceleft}{\kern0pt}\ {\isacharless}{\kern0pt}Pn{\isacharunderscore}{\kern0pt}auto{\isacharparenleft}{\kern0pt}Fn{\isacharunderscore}{\kern0pt}perm{\isacharprime}{\kern0pt}{\isacharparenleft}{\kern0pt}f{\isacharparenright}{\kern0pt}{\isacharparenright}{\kern0pt}{\isacharbackquote}{\kern0pt}y{\isacharcomma}{\kern0pt}\ Fn{\isacharunderscore}{\kern0pt}perm{\isacharprime}{\kern0pt}{\isacharparenleft}{\kern0pt}f{\isacharparenright}{\kern0pt}\ {\isacharbackquote}{\kern0pt}\ p{\isachargreater}{\kern0pt}\ {\isachardot}{\kern0pt}\ {\isacharless}{\kern0pt}y{\isacharcomma}{\kern0pt}\ p{\isachargreater}{\kern0pt}\ {\isasymin}\ binmap{\isacharunderscore}{\kern0pt}row{\isacharprime}{\kern0pt}{\isacharparenleft}{\kern0pt}n{\isacharparenright}{\kern0pt}\ {\isacharbraceright}{\kern0pt}{\isachardoublequoteclose}\ \isanewline
\ \ \ \ \isacommand{apply}\isamarkupfalse%
{\isacharparenleft}{\kern0pt}subst\ Pn{\isacharunderscore}{\kern0pt}auto{\isacharcomma}{\kern0pt}\ rule\ binmap{\isacharunderscore}{\kern0pt}row{\isacharprime}{\kern0pt}{\isacharunderscore}{\kern0pt}P{\isacharunderscore}{\kern0pt}name{\isacharparenright}{\kern0pt}\isanewline
\ \ \ \ \isacommand{using}\isamarkupfalse%
\ assms\isanewline
\ \ \ \ \isacommand{by}\isamarkupfalse%
\ auto\isanewline
\ \ \isacommand{also}\isamarkupfalse%
\ \isacommand{have}\isamarkupfalse%
\ {\isachardoublequoteopen}{\isachardot}{\kern0pt}{\isachardot}{\kern0pt}{\isachardot}{\kern0pt}\ {\isacharequal}{\kern0pt}\ {\isacharbraceleft}{\kern0pt}\ {\isacharless}{\kern0pt}Pn{\isacharunderscore}{\kern0pt}auto{\isacharparenleft}{\kern0pt}Fn{\isacharunderscore}{\kern0pt}perm{\isacharprime}{\kern0pt}{\isacharparenleft}{\kern0pt}f{\isacharparenright}{\kern0pt}{\isacharparenright}{\kern0pt}{\isacharbackquote}{\kern0pt}y{\isacharcomma}{\kern0pt}\ Fn{\isacharunderscore}{\kern0pt}perm{\isacharprime}{\kern0pt}{\isacharparenleft}{\kern0pt}f{\isacharparenright}{\kern0pt}\ {\isacharbackquote}{\kern0pt}\ p{\isachargreater}{\kern0pt}\ {\isachardot}{\kern0pt}\ {\isacharless}{\kern0pt}y{\isacharcomma}{\kern0pt}\ p{\isachargreater}{\kern0pt}\ {\isasymin}\ {\isacharbraceleft}{\kern0pt}\ {\isacharless}{\kern0pt}check{\isacharparenleft}{\kern0pt}m{\isacharparenright}{\kern0pt}{\isacharcomma}{\kern0pt}\ F{\isachargreater}{\kern0pt}{\isachardot}{\kern0pt}{\isachardot}{\kern0pt}\ {\isacharless}{\kern0pt}m{\isacharcomma}{\kern0pt}\ F{\isachargreater}{\kern0pt}\ {\isasymin}\ nat\ {\isasymtimes}\ Fn{\isacharcomma}{\kern0pt}\ F{\isacharbackquote}{\kern0pt}{\isacharless}{\kern0pt}n{\isacharcomma}{\kern0pt}\ m{\isachargreater}{\kern0pt}\ {\isacharequal}{\kern0pt}\ {\isadigit{1}}\ {\isacharbraceright}{\kern0pt}\ {\isacharbraceright}{\kern0pt}{\isachardoublequoteclose}\ \isanewline
\ \ \ \ \isacommand{apply}\isamarkupfalse%
{\isacharparenleft}{\kern0pt}subst\ binmap{\isacharunderscore}{\kern0pt}row{\isacharprime}{\kern0pt}{\isacharunderscore}{\kern0pt}eq{\isacharparenright}{\kern0pt}\isanewline
\ \ \ \ \isacommand{using}\isamarkupfalse%
\ assms\isanewline
\ \ \ \ \isacommand{by}\isamarkupfalse%
\ auto\isanewline
\ \ \isacommand{also}\isamarkupfalse%
\ \isacommand{have}\isamarkupfalse%
\ {\isachardoublequoteopen}{\isachardot}{\kern0pt}{\isachardot}{\kern0pt}{\isachardot}{\kern0pt}\ {\isacharequal}{\kern0pt}\ {\isacharbraceleft}{\kern0pt}\ {\isacharless}{\kern0pt}Pn{\isacharunderscore}{\kern0pt}auto{\isacharparenleft}{\kern0pt}Fn{\isacharunderscore}{\kern0pt}perm{\isacharprime}{\kern0pt}{\isacharparenleft}{\kern0pt}f{\isacharparenright}{\kern0pt}{\isacharparenright}{\kern0pt}{\isacharbackquote}{\kern0pt}check{\isacharparenleft}{\kern0pt}m{\isacharparenright}{\kern0pt}{\isacharcomma}{\kern0pt}\ Fn{\isacharunderscore}{\kern0pt}perm{\isacharprime}{\kern0pt}{\isacharparenleft}{\kern0pt}f{\isacharparenright}{\kern0pt}\ {\isacharbackquote}{\kern0pt}\ p{\isachargreater}{\kern0pt}{\isachardot}{\kern0pt}{\isachardot}{\kern0pt}\ {\isacharless}{\kern0pt}m{\isacharcomma}{\kern0pt}\ p{\isachargreater}{\kern0pt}\ {\isasymin}\ nat\ {\isasymtimes}\ Fn{\isacharcomma}{\kern0pt}\ p{\isacharbackquote}{\kern0pt}{\isacharless}{\kern0pt}n{\isacharcomma}{\kern0pt}\ m{\isachargreater}{\kern0pt}\ {\isacharequal}{\kern0pt}\ {\isadigit{1}}\ {\isacharbraceright}{\kern0pt}{\isachardoublequoteclose}\ \isanewline
\ \ \ \ \isacommand{by}\isamarkupfalse%
{\isacharparenleft}{\kern0pt}rule\ equality{\isacharunderscore}{\kern0pt}iffI{\isacharcomma}{\kern0pt}\ rule\ iffI{\isacharcomma}{\kern0pt}\ force{\isacharcomma}{\kern0pt}\ force{\isacharparenright}{\kern0pt}\ \isanewline
\ \ \isacommand{also}\isamarkupfalse%
\ \isacommand{have}\isamarkupfalse%
\ {\isachardoublequoteopen}{\isachardot}{\kern0pt}{\isachardot}{\kern0pt}{\isachardot}{\kern0pt}\ {\isacharequal}{\kern0pt}\ {\isacharbraceleft}{\kern0pt}\ {\isacharless}{\kern0pt}check{\isacharparenleft}{\kern0pt}m{\isacharparenright}{\kern0pt}{\isacharcomma}{\kern0pt}\ Fn{\isacharunderscore}{\kern0pt}perm{\isacharprime}{\kern0pt}{\isacharparenleft}{\kern0pt}f{\isacharparenright}{\kern0pt}\ {\isacharbackquote}{\kern0pt}\ p{\isachargreater}{\kern0pt}{\isachardot}{\kern0pt}{\isachardot}{\kern0pt}\ {\isacharless}{\kern0pt}m{\isacharcomma}{\kern0pt}\ p{\isachargreater}{\kern0pt}\ {\isasymin}\ nat\ {\isasymtimes}\ Fn{\isacharcomma}{\kern0pt}\ p{\isacharbackquote}{\kern0pt}{\isacharless}{\kern0pt}n{\isacharcomma}{\kern0pt}\ m{\isachargreater}{\kern0pt}\ {\isacharequal}{\kern0pt}\ {\isadigit{1}}\ {\isacharbraceright}{\kern0pt}{\isachardoublequoteclose}\ \isanewline
\ \ \ \ \isacommand{apply}\isamarkupfalse%
{\isacharparenleft}{\kern0pt}rule\ equality{\isacharunderscore}{\kern0pt}iffI{\isacharcomma}{\kern0pt}\ rule\ iffI{\isacharcomma}{\kern0pt}\ clarsimp{\isacharparenright}{\kern0pt}\isanewline
\ \ \ \ \ \isacommand{apply}\isamarkupfalse%
{\isacharparenleft}{\kern0pt}rename{\isacharunderscore}{\kern0pt}tac\ m\ F{\isacharcomma}{\kern0pt}\ rule{\isacharunderscore}{\kern0pt}tac\ x{\isacharequal}{\kern0pt}m\ \isakeyword{in}\ bexI{\isacharparenright}{\kern0pt}\isanewline
\ \ \ \ \ \ \isacommand{apply}\isamarkupfalse%
{\isacharparenleft}{\kern0pt}subst\ check{\isacharunderscore}{\kern0pt}fixpoint{\isacharcomma}{\kern0pt}\ rule\ Fn{\isacharunderscore}{\kern0pt}perm{\isacharprime}{\kern0pt}{\isacharunderscore}{\kern0pt}is{\isacharunderscore}{\kern0pt}P{\isacharunderscore}{\kern0pt}auto{\isacharparenright}{\kern0pt}\isanewline
\ \ \ \ \isacommand{using}\isamarkupfalse%
\ assms\ P{\isacharunderscore}{\kern0pt}auto{\isacharunderscore}{\kern0pt}def\ nat{\isacharunderscore}{\kern0pt}in{\isacharunderscore}{\kern0pt}M\ transM\ \isanewline
\ \ \ \ \ \ \ \ \isacommand{apply}\isamarkupfalse%
\ auto{\isacharbrackleft}{\kern0pt}{\isadigit{4}}{\isacharbrackright}{\kern0pt}\isanewline
\ \ \ \ \isacommand{apply}\isamarkupfalse%
\ clarsimp\isanewline
\ \ \ \ \ \isacommand{apply}\isamarkupfalse%
{\isacharparenleft}{\kern0pt}rename{\isacharunderscore}{\kern0pt}tac\ m\ F{\isacharcomma}{\kern0pt}\ rule{\isacharunderscore}{\kern0pt}tac\ x{\isacharequal}{\kern0pt}m\ \isakeyword{in}\ bexI{\isacharparenright}{\kern0pt}\isanewline
\ \ \ \ \ \ \isacommand{apply}\isamarkupfalse%
{\isacharparenleft}{\kern0pt}subst\ check{\isacharunderscore}{\kern0pt}fixpoint{\isacharcomma}{\kern0pt}\ rule\ Fn{\isacharunderscore}{\kern0pt}perm{\isacharprime}{\kern0pt}{\isacharunderscore}{\kern0pt}is{\isacharunderscore}{\kern0pt}P{\isacharunderscore}{\kern0pt}auto{\isacharparenright}{\kern0pt}\isanewline
\ \ \ \ \isacommand{using}\isamarkupfalse%
\ assms\ P{\isacharunderscore}{\kern0pt}auto{\isacharunderscore}{\kern0pt}def\ nat{\isacharunderscore}{\kern0pt}in{\isacharunderscore}{\kern0pt}M\ transM\ \isanewline
\ \ \ \ \isacommand{by}\isamarkupfalse%
\ auto\ \isanewline
\ \ \isacommand{also}\isamarkupfalse%
\ \isacommand{have}\isamarkupfalse%
\ {\isachardoublequoteopen}{\isachardot}{\kern0pt}{\isachardot}{\kern0pt}{\isachardot}{\kern0pt}\ {\isacharequal}{\kern0pt}\ {\isacharbraceleft}{\kern0pt}\ {\isacharless}{\kern0pt}check{\isacharparenleft}{\kern0pt}m{\isacharparenright}{\kern0pt}{\isacharcomma}{\kern0pt}\ p{\isachargreater}{\kern0pt}{\isachardot}{\kern0pt}{\isachardot}{\kern0pt}\ {\isacharless}{\kern0pt}m{\isacharcomma}{\kern0pt}\ p{\isachargreater}{\kern0pt}\ {\isasymin}\ nat\ {\isasymtimes}\ Fn{\isacharcomma}{\kern0pt}\ p{\isacharbackquote}{\kern0pt}{\isacharless}{\kern0pt}f{\isacharbackquote}{\kern0pt}n{\isacharcomma}{\kern0pt}\ m{\isachargreater}{\kern0pt}\ {\isacharequal}{\kern0pt}\ {\isadigit{1}}\ {\isacharbraceright}{\kern0pt}{\isachardoublequoteclose}\ \isanewline
\ \ \ \ \isacommand{apply}\isamarkupfalse%
{\isacharparenleft}{\kern0pt}rule\ equality{\isacharunderscore}{\kern0pt}iffI{\isacharcomma}{\kern0pt}\ rule\ iffI{\isacharcomma}{\kern0pt}\ clarsimp{\isacharparenright}{\kern0pt}\isanewline
\ \ \ \ \ \isacommand{apply}\isamarkupfalse%
{\isacharparenleft}{\kern0pt}rename{\isacharunderscore}{\kern0pt}tac\ m\ p{\isacharcomma}{\kern0pt}\ rule{\isacharunderscore}{\kern0pt}tac\ x{\isacharequal}{\kern0pt}m\ \isakeyword{in}\ bexI{\isacharparenright}{\kern0pt}\isanewline
\ \ \ \ \ \ \isacommand{apply}\isamarkupfalse%
{\isacharparenleft}{\kern0pt}rule\ conjI{\isacharcomma}{\kern0pt}\ simp{\isacharcomma}{\kern0pt}\ rule\ conjI{\isacharparenright}{\kern0pt}\isanewline
\ \ \ \ \ \ \ \isacommand{apply}\isamarkupfalse%
{\isacharparenleft}{\kern0pt}rule\ function{\isacharunderscore}{\kern0pt}value{\isacharunderscore}{\kern0pt}in{\isacharcomma}{\kern0pt}\ rule\ Fn{\isacharunderscore}{\kern0pt}perm{\isacharprime}{\kern0pt}{\isacharunderscore}{\kern0pt}type{\isacharcomma}{\kern0pt}\ simp\ add{\isacharcolon}{\kern0pt}assms{\isacharcomma}{\kern0pt}\ simp{\isacharparenright}{\kern0pt}\isanewline
\ \ \ \ \ \ \isacommand{apply}\isamarkupfalse%
{\isacharparenleft}{\kern0pt}rule\ iffD{\isadigit{1}}{\isacharcomma}{\kern0pt}\ rule\ H{\isacharcomma}{\kern0pt}\ simp{\isacharcomma}{\kern0pt}\ simp{\isacharparenright}{\kern0pt}\isanewline
\ \ \ \ \ \ \ \isacommand{apply}\isamarkupfalse%
{\isacharparenleft}{\kern0pt}rule\ apply{\isacharunderscore}{\kern0pt}not{\isadigit{0}}{\isacharunderscore}{\kern0pt}indom{\isacharparenright}{\kern0pt}\isanewline
\ \ \ \ \isacommand{using}\isamarkupfalse%
\ Fn{\isacharunderscore}{\kern0pt}def\ \isanewline
\ \ \ \ \ \ \ \ \isacommand{apply}\isamarkupfalse%
\ auto{\isacharbrackleft}{\kern0pt}{\isadigit{4}}{\isacharbrackright}{\kern0pt}\isanewline
\ \ \ \ \isacommand{apply}\isamarkupfalse%
\ clarsimp\isanewline
\ \ \ \ \isacommand{apply}\isamarkupfalse%
{\isacharparenleft}{\kern0pt}rename{\isacharunderscore}{\kern0pt}tac\ m\ p{\isacharcomma}{\kern0pt}\ rule{\isacharunderscore}{\kern0pt}tac\ x{\isacharequal}{\kern0pt}m\ \isakeyword{in}\ bexI{\isacharcomma}{\kern0pt}\ rule\ conjI{\isacharcomma}{\kern0pt}\ simp{\isacharparenright}{\kern0pt}\isanewline
\ \ \ \ \ \isacommand{apply}\isamarkupfalse%
{\isacharparenleft}{\kern0pt}rename{\isacharunderscore}{\kern0pt}tac\ m\ p{\isacharcomma}{\kern0pt}\ subgoal{\isacharunderscore}{\kern0pt}tac\ {\isachardoublequoteopen}{\isasymexists}y{\isasymin}Fn{\isachardot}{\kern0pt}\ p\ {\isacharequal}{\kern0pt}\ Fn{\isacharunderscore}{\kern0pt}perm{\isacharprime}{\kern0pt}{\isacharparenleft}{\kern0pt}f{\isacharparenright}{\kern0pt}\ {\isacharbackquote}{\kern0pt}\ y{\isachardoublequoteclose}{\isacharcomma}{\kern0pt}\ clarsimp{\isacharparenright}{\kern0pt}\isanewline
\ \ \ \ \ \ \isacommand{apply}\isamarkupfalse%
{\isacharparenleft}{\kern0pt}rename{\isacharunderscore}{\kern0pt}tac\ m\ p{\isacharcomma}{\kern0pt}\ rule{\isacharunderscore}{\kern0pt}tac\ x{\isacharequal}{\kern0pt}p\ \isakeyword{in}\ bexI{\isacharcomma}{\kern0pt}\ rule\ conjI{\isacharcomma}{\kern0pt}\ simp{\isacharparenright}{\kern0pt}\isanewline
\ \ \ \ \ \ \ \isacommand{apply}\isamarkupfalse%
{\isacharparenleft}{\kern0pt}rule\ iffD{\isadigit{2}}{\isacharcomma}{\kern0pt}\ rule\ H{\isacharparenright}{\kern0pt}\isanewline
\ \ \ \ \ \ \ \ \ \ \isacommand{apply}\isamarkupfalse%
\ auto{\isacharbrackleft}{\kern0pt}{\isadigit{2}}{\isacharbrackright}{\kern0pt}\isanewline
\ \ \ \ \ \ \ \ \isacommand{apply}\isamarkupfalse%
{\isacharparenleft}{\kern0pt}subgoal{\isacharunderscore}{\kern0pt}tac\ {\isachardoublequoteopen}{\isasymlangle}f\ {\isacharbackquote}{\kern0pt}\ n{\isacharcomma}{\kern0pt}\ m{\isasymrangle}\ {\isasymin}\ domain{\isacharparenleft}{\kern0pt}Fn{\isacharunderscore}{\kern0pt}perm{\isacharprime}{\kern0pt}{\isacharparenleft}{\kern0pt}f{\isacharparenright}{\kern0pt}\ {\isacharbackquote}{\kern0pt}\ p{\isacharparenright}{\kern0pt}{\isachardoublequoteclose}{\isacharparenright}{\kern0pt}\isanewline
\ \ \ \ \ \ \ \ \ \isacommand{apply}\isamarkupfalse%
{\isacharparenleft}{\kern0pt}rule\ indom{\isacharparenright}{\kern0pt}\isanewline
\ \ \ \ \ \ \ \ \ \ \ \ \isacommand{apply}\isamarkupfalse%
\ auto{\isacharbrackleft}{\kern0pt}{\isadigit{4}}{\isacharbrackright}{\kern0pt}\isanewline
\ \ \ \ \ \ \ \ \isacommand{apply}\isamarkupfalse%
{\isacharparenleft}{\kern0pt}rule\ apply{\isacharunderscore}{\kern0pt}not{\isadigit{0}}{\isacharunderscore}{\kern0pt}indom{\isacharparenright}{\kern0pt}\isanewline
\ \ \ \ \ \ \ \ \ \isacommand{apply}\isamarkupfalse%
{\isacharparenleft}{\kern0pt}subgoal{\isacharunderscore}{\kern0pt}tac\ {\isachardoublequoteopen}Fn{\isacharunderscore}{\kern0pt}perm{\isacharprime}{\kern0pt}{\isacharparenleft}{\kern0pt}f{\isacharparenright}{\kern0pt}\ {\isacharbackquote}{\kern0pt}\ p\ {\isasymin}\ Fn{\isachardoublequoteclose}{\isacharparenright}{\kern0pt}\isanewline
\ \ \ \ \isacommand{using}\isamarkupfalse%
\ Fn{\isacharunderscore}{\kern0pt}def\isanewline
\ \ \ \ \ \ \ \ \ \ \isacommand{apply}\isamarkupfalse%
\ auto{\isacharbrackleft}{\kern0pt}{\isadigit{5}}{\isacharbrackright}{\kern0pt}\isanewline
\ \ \ \ \isacommand{using}\isamarkupfalse%
\ Fn{\isacharunderscore}{\kern0pt}perm{\isacharprime}{\kern0pt}{\isacharunderscore}{\kern0pt}bij\ assms\ bij{\isacharunderscore}{\kern0pt}is{\isacharunderscore}{\kern0pt}surj\ surj{\isacharunderscore}{\kern0pt}def\isanewline
\ \ \ \ \ \isacommand{apply}\isamarkupfalse%
\ force\isanewline
\ \ \ \ \isacommand{by}\isamarkupfalse%
\ auto\isanewline
\ \ \isacommand{also}\isamarkupfalse%
\ \isacommand{have}\isamarkupfalse%
\ {\isachardoublequoteopen}{\isachardot}{\kern0pt}{\isachardot}{\kern0pt}{\isachardot}{\kern0pt}\ {\isacharequal}{\kern0pt}\ binmap{\isacharunderscore}{\kern0pt}row{\isacharprime}{\kern0pt}{\isacharparenleft}{\kern0pt}f{\isacharbackquote}{\kern0pt}n{\isacharparenright}{\kern0pt}{\isachardoublequoteclose}\ \isanewline
\ \ \ \ \isacommand{unfolding}\isamarkupfalse%
\ binmap{\isacharunderscore}{\kern0pt}row{\isacharprime}{\kern0pt}{\isacharunderscore}{\kern0pt}def\isanewline
\ \ \ \ \isacommand{apply}\isamarkupfalse%
{\isacharparenleft}{\kern0pt}rule\ equality{\isacharunderscore}{\kern0pt}iffI{\isacharcomma}{\kern0pt}\ rule\ iffI{\isacharparenright}{\kern0pt}\isanewline
\ \ \ \ \ \isacommand{apply}\isamarkupfalse%
\ clarsimp\isanewline
\ \ \ \ \ \isacommand{apply}\isamarkupfalse%
{\isacharparenleft}{\kern0pt}rule\ conjI{\isacharparenright}{\kern0pt}\isanewline
\ \ \ \ \ \ \isacommand{apply}\isamarkupfalse%
{\isacharparenleft}{\kern0pt}subst\ {\isacharparenleft}{\kern0pt}{\isadigit{2}}{\isacharparenright}{\kern0pt}\ def{\isacharunderscore}{\kern0pt}check{\isacharcomma}{\kern0pt}\ force{\isacharparenright}{\kern0pt}\isanewline
\ \ \ \ \ \isacommand{apply}\isamarkupfalse%
\ force\isanewline
\ \ \ \ \isacommand{apply}\isamarkupfalse%
\ force\isanewline
\ \ \ \ \isacommand{done}\isamarkupfalse%
\isanewline
\ \ \isacommand{finally}\isamarkupfalse%
\ \isacommand{show}\isamarkupfalse%
\ {\isacharquery}{\kern0pt}thesis\ \isacommand{by}\isamarkupfalse%
\ simp\isanewline
\isacommand{qed}\isamarkupfalse%
%
\endisatagproof
{\isafoldproof}%
%
\isadelimproof
\isanewline
%
\endisadelimproof
\isanewline
\isacommand{lemma}\isamarkupfalse%
\ binmap{\isacharunderscore}{\kern0pt}row{\isacharprime}{\kern0pt}{\isacharunderscore}{\kern0pt}symmetric\ {\isacharcolon}{\kern0pt}\ \isanewline
\ \ \isakeyword{fixes}\ n\ \isanewline
\ \ \isakeyword{assumes}\ {\isachardoublequoteopen}n\ {\isasymin}\ nat{\isachardoublequoteclose}\ \isanewline
\ \ \isakeyword{shows}\ {\isachardoublequoteopen}symmetric{\isacharparenleft}{\kern0pt}binmap{\isacharunderscore}{\kern0pt}row{\isacharprime}{\kern0pt}{\isacharparenleft}{\kern0pt}n{\isacharparenright}{\kern0pt}{\isacharparenright}{\kern0pt}{\isachardoublequoteclose}\ \isanewline
%
\isadelimproof
%
\endisadelimproof
%
\isatagproof
\isacommand{proof}\isamarkupfalse%
\ {\isacharminus}{\kern0pt}\ \isanewline
\ \ \isacommand{have}\isamarkupfalse%
\ H{\isacharcolon}{\kern0pt}\ {\isachardoublequoteopen}Fix{\isacharparenleft}{\kern0pt}{\isacharbraceleft}{\kern0pt}n{\isacharbraceright}{\kern0pt}{\isacharparenright}{\kern0pt}\ {\isasymsubseteq}\ sym{\isacharparenleft}{\kern0pt}binmap{\isacharunderscore}{\kern0pt}row{\isacharprime}{\kern0pt}{\isacharparenleft}{\kern0pt}n{\isacharparenright}{\kern0pt}{\isacharparenright}{\kern0pt}{\isachardoublequoteclose}\ \isanewline
\ \ \isacommand{proof}\isamarkupfalse%
{\isacharparenleft}{\kern0pt}rule\ subsetI{\isacharparenright}{\kern0pt}\isanewline
\ \ \ \ \isacommand{fix}\isamarkupfalse%
\ F\ \isacommand{assume}\isamarkupfalse%
\ Fin\ {\isacharcolon}{\kern0pt}\ {\isachardoublequoteopen}F\ {\isasymin}\ Fix{\isacharparenleft}{\kern0pt}{\isacharbraceleft}{\kern0pt}n{\isacharbraceright}{\kern0pt}{\isacharparenright}{\kern0pt}{\isachardoublequoteclose}\ \isanewline
\ \ \ \ \isacommand{then}\isamarkupfalse%
\ \isacommand{obtain}\isamarkupfalse%
\ f\ \isakeyword{where}\ fH{\isacharcolon}{\kern0pt}\ {\isachardoublequoteopen}Fn{\isacharunderscore}{\kern0pt}perm{\isacharprime}{\kern0pt}{\isacharparenleft}{\kern0pt}f{\isacharparenright}{\kern0pt}\ {\isacharequal}{\kern0pt}\ F{\isachardoublequoteclose}\ {\isachardoublequoteopen}f\ {\isasymin}\ nat{\isacharunderscore}{\kern0pt}perms{\isachardoublequoteclose}\ {\isachardoublequoteopen}f{\isacharbackquote}{\kern0pt}n\ {\isacharequal}{\kern0pt}\ n{\isachardoublequoteclose}\ \isanewline
\ \ \ \ \ \ \isacommand{unfolding}\isamarkupfalse%
\ Fix{\isacharunderscore}{\kern0pt}def\ \isanewline
\ \ \ \ \ \ \isacommand{by}\isamarkupfalse%
\ force\isanewline
\ \ \ \ \isacommand{have}\isamarkupfalse%
\ {\isachardoublequoteopen}Pn{\isacharunderscore}{\kern0pt}auto{\isacharparenleft}{\kern0pt}Fn{\isacharunderscore}{\kern0pt}perm{\isacharprime}{\kern0pt}{\isacharparenleft}{\kern0pt}f{\isacharparenright}{\kern0pt}{\isacharparenright}{\kern0pt}{\isacharbackquote}{\kern0pt}binmap{\isacharunderscore}{\kern0pt}row{\isacharprime}{\kern0pt}{\isacharparenleft}{\kern0pt}n{\isacharparenright}{\kern0pt}\ {\isacharequal}{\kern0pt}\ binmap{\isacharunderscore}{\kern0pt}row{\isacharprime}{\kern0pt}{\isacharparenleft}{\kern0pt}f{\isacharbackquote}{\kern0pt}n{\isacharparenright}{\kern0pt}{\isachardoublequoteclose}\ \isanewline
\ \ \ \ \ \ \isacommand{apply}\isamarkupfalse%
{\isacharparenleft}{\kern0pt}rule\ binmap{\isacharunderscore}{\kern0pt}row{\isacharprime}{\kern0pt}{\isacharunderscore}{\kern0pt}pauto{\isacharparenright}{\kern0pt}\isanewline
\ \ \ \ \ \ \isacommand{using}\isamarkupfalse%
\ fH\ assms\ \isanewline
\ \ \ \ \ \ \isacommand{by}\isamarkupfalse%
\ auto\isanewline
\ \ \ \ \isacommand{also}\isamarkupfalse%
\ \isacommand{have}\isamarkupfalse%
\ {\isachardoublequoteopen}{\isachardot}{\kern0pt}{\isachardot}{\kern0pt}{\isachardot}{\kern0pt}\ {\isacharequal}{\kern0pt}\ binmap{\isacharunderscore}{\kern0pt}row{\isacharprime}{\kern0pt}{\isacharparenleft}{\kern0pt}n{\isacharparenright}{\kern0pt}{\isachardoublequoteclose}\ \isacommand{using}\isamarkupfalse%
\ fH\ \isacommand{by}\isamarkupfalse%
\ auto\isanewline
\ \ \ \ \isacommand{finally}\isamarkupfalse%
\ \isacommand{show}\isamarkupfalse%
\ {\isachardoublequoteopen}F\ {\isasymin}\ sym{\isacharparenleft}{\kern0pt}binmap{\isacharunderscore}{\kern0pt}row{\isacharprime}{\kern0pt}{\isacharparenleft}{\kern0pt}n{\isacharparenright}{\kern0pt}{\isacharparenright}{\kern0pt}{\isachardoublequoteclose}\ \isanewline
\ \ \ \ \ \ \isacommand{unfolding}\isamarkupfalse%
\ sym{\isacharunderscore}{\kern0pt}def\ \isanewline
\ \ \ \ \ \ \isacommand{using}\isamarkupfalse%
\ fH\ Fn{\isacharunderscore}{\kern0pt}perms{\isacharunderscore}{\kern0pt}def\ assms\isanewline
\ \ \ \ \ \ \isacommand{by}\isamarkupfalse%
\ auto\isanewline
\ \ \isacommand{qed}\isamarkupfalse%
\isanewline
\isanewline
\ \ \isacommand{show}\isamarkupfalse%
\ {\isacharquery}{\kern0pt}thesis\ \isanewline
\ \ \ \ \isacommand{unfolding}\isamarkupfalse%
\ symmetric{\isacharunderscore}{\kern0pt}def\ \isanewline
\ \ \ \ \isacommand{unfolding}\isamarkupfalse%
\ Fn{\isacharunderscore}{\kern0pt}perms{\isacharunderscore}{\kern0pt}filter{\isacharunderscore}{\kern0pt}def\isanewline
\ \ \ \ \isacommand{apply}\isamarkupfalse%
\ simp\isanewline
\ \ \ \ \isacommand{apply}\isamarkupfalse%
{\isacharparenleft}{\kern0pt}rule\ conjI{\isacharcomma}{\kern0pt}\ rule\ sym{\isacharunderscore}{\kern0pt}P{\isacharunderscore}{\kern0pt}auto{\isacharunderscore}{\kern0pt}subgroup{\isacharparenright}{\kern0pt}\isanewline
\ \ \ \ \ \isacommand{apply}\isamarkupfalse%
{\isacharparenleft}{\kern0pt}rule\ binmap{\isacharunderscore}{\kern0pt}row{\isacharprime}{\kern0pt}{\isacharunderscore}{\kern0pt}P{\isacharunderscore}{\kern0pt}name{\isacharcomma}{\kern0pt}\ simp\ add{\isacharcolon}{\kern0pt}assms{\isacharparenright}{\kern0pt}\isanewline
\ \ \ \ \isacommand{apply}\isamarkupfalse%
{\isacharparenleft}{\kern0pt}rule{\isacharunderscore}{\kern0pt}tac\ x{\isacharequal}{\kern0pt}{\isachardoublequoteopen}{\isacharbraceleft}{\kern0pt}n{\isacharbraceright}{\kern0pt}{\isachardoublequoteclose}\ \isakeyword{in}\ bexI{\isacharcomma}{\kern0pt}\ rule\ conjI{\isacharparenright}{\kern0pt}\isanewline
\ \ \ \ \isacommand{unfolding}\isamarkupfalse%
\ finite{\isacharunderscore}{\kern0pt}M{\isacharunderscore}{\kern0pt}def\ \isanewline
\ \ \ \ \ \ \isacommand{apply}\isamarkupfalse%
{\isacharparenleft}{\kern0pt}rule{\isacharunderscore}{\kern0pt}tac\ x{\isacharequal}{\kern0pt}{\isadigit{1}}\ \isakeyword{in}\ bexI{\isacharcomma}{\kern0pt}\ rule{\isacharunderscore}{\kern0pt}tac\ a{\isacharequal}{\kern0pt}{\isachardoublequoteopen}{\isacharbraceleft}{\kern0pt}{\isacharless}{\kern0pt}n{\isacharcomma}{\kern0pt}\ {\isadigit{0}}{\isachargreater}{\kern0pt}{\isacharbraceright}{\kern0pt}{\isachardoublequoteclose}\ \isakeyword{in}\ not{\isacharunderscore}{\kern0pt}emptyI{\isacharcomma}{\kern0pt}\ simp{\isacharcomma}{\kern0pt}\ rule\ conjI{\isacharparenright}{\kern0pt}\isanewline
\ \ \ \ \isacommand{unfolding}\isamarkupfalse%
\ inj{\isacharunderscore}{\kern0pt}def\ \isanewline
\ \ \ \ \ \ \ \ \isacommand{apply}\isamarkupfalse%
\ simp\isanewline
\ \ \ \ \ \ \ \ \isacommand{apply}\isamarkupfalse%
{\isacharparenleft}{\kern0pt}rule\ Pi{\isacharunderscore}{\kern0pt}memberI{\isacharparenright}{\kern0pt}\isanewline
\ \ \ \ \isacommand{using}\isamarkupfalse%
\ relation{\isacharunderscore}{\kern0pt}def\ function{\isacharunderscore}{\kern0pt}def\ range{\isacharunderscore}{\kern0pt}def\ nat{\isacharunderscore}{\kern0pt}in{\isacharunderscore}{\kern0pt}M\ transM\ assms\ pair{\isacharunderscore}{\kern0pt}in{\isacharunderscore}{\kern0pt}M{\isacharunderscore}{\kern0pt}iff\ singleton{\isacharunderscore}{\kern0pt}in{\isacharunderscore}{\kern0pt}M{\isacharunderscore}{\kern0pt}iff\ H\isanewline
\ \ \ \ \isacommand{by}\isamarkupfalse%
\ auto\isanewline
\isacommand{qed}\isamarkupfalse%
%
\endisatagproof
{\isafoldproof}%
%
\isadelimproof
\isanewline
%
\endisadelimproof
\ \ \ \ \isanewline
\isacommand{lemma}\isamarkupfalse%
\ binmap{\isacharunderscore}{\kern0pt}row{\isacharprime}{\kern0pt}{\isacharunderscore}{\kern0pt}HS\ {\isacharcolon}{\kern0pt}\ \isanewline
\ \ \isakeyword{fixes}\ n\ \isanewline
\ \ \isakeyword{assumes}\ {\isachardoublequoteopen}n\ {\isasymin}\ nat{\isachardoublequoteclose}\ \isanewline
\ \ \isakeyword{shows}\ {\isachardoublequoteopen}binmap{\isacharunderscore}{\kern0pt}row{\isacharprime}{\kern0pt}{\isacharparenleft}{\kern0pt}n{\isacharparenright}{\kern0pt}\ {\isasymin}\ HS{\isachardoublequoteclose}\ \isanewline
%
\isadelimproof
\ \ %
\endisadelimproof
%
\isatagproof
\isacommand{apply}\isamarkupfalse%
{\isacharparenleft}{\kern0pt}rule\ iffD{\isadigit{2}}{\isacharcomma}{\kern0pt}\ rule\ HS{\isacharunderscore}{\kern0pt}iff{\isacharcomma}{\kern0pt}\ rule\ conjI{\isacharparenright}{\kern0pt}\isanewline
\ \ \ \isacommand{apply}\isamarkupfalse%
{\isacharparenleft}{\kern0pt}rule\ iffD{\isadigit{2}}{\isacharcomma}{\kern0pt}\ rule\ P{\isacharunderscore}{\kern0pt}name{\isacharunderscore}{\kern0pt}iff{\isacharcomma}{\kern0pt}\ rule\ conjI{\isacharparenright}{\kern0pt}\isanewline
\ \ \ \ \isacommand{apply}\isamarkupfalse%
{\isacharparenleft}{\kern0pt}rule\ binmap{\isacharunderscore}{\kern0pt}row{\isacharprime}{\kern0pt}{\isacharunderscore}{\kern0pt}in{\isacharunderscore}{\kern0pt}M{\isacharcomma}{\kern0pt}\ simp\ add{\isacharcolon}{\kern0pt}assms{\isacharparenright}{\kern0pt}\isanewline
\ \ \ \isacommand{apply}\isamarkupfalse%
{\isacharparenleft}{\kern0pt}simp\ add{\isacharcolon}{\kern0pt}\ binmap{\isacharunderscore}{\kern0pt}row{\isacharprime}{\kern0pt}{\isacharunderscore}{\kern0pt}def{\isacharparenright}{\kern0pt}\isanewline
\ \ \ \isacommand{apply}\isamarkupfalse%
{\isacharparenleft}{\kern0pt}subgoal{\isacharunderscore}{\kern0pt}tac\ {\isachardoublequoteopen}domain{\isacharparenleft}{\kern0pt}check{\isacharparenleft}{\kern0pt}nat{\isacharparenright}{\kern0pt}{\isacharparenright}{\kern0pt}\ {\isasymsubseteq}\ P{\isacharunderscore}{\kern0pt}names{\isachardoublequoteclose}{\isacharcomma}{\kern0pt}\ force{\isacharparenright}{\kern0pt}\isanewline
\ \ \isacommand{using}\isamarkupfalse%
\ check{\isacharunderscore}{\kern0pt}P{\isacharunderscore}{\kern0pt}name\ nat{\isacharunderscore}{\kern0pt}in{\isacharunderscore}{\kern0pt}M\ P{\isacharunderscore}{\kern0pt}name{\isacharunderscore}{\kern0pt}domain{\isacharunderscore}{\kern0pt}P{\isacharunderscore}{\kern0pt}name{\isacharprime}{\kern0pt}\isanewline
\ \ \ \isacommand{apply}\isamarkupfalse%
\ force\isanewline
\ \ \isacommand{apply}\isamarkupfalse%
{\isacharparenleft}{\kern0pt}rule\ conjI{\isacharcomma}{\kern0pt}\ simp\ add{\isacharcolon}{\kern0pt}\ binmap{\isacharunderscore}{\kern0pt}row{\isacharprime}{\kern0pt}{\isacharunderscore}{\kern0pt}def{\isacharparenright}{\kern0pt}\isanewline
\ \ \isacommand{using}\isamarkupfalse%
\ check{\isacharunderscore}{\kern0pt}in{\isacharunderscore}{\kern0pt}HS\ nat{\isacharunderscore}{\kern0pt}in{\isacharunderscore}{\kern0pt}M\ HS{\isacharunderscore}{\kern0pt}iff\ \isanewline
\ \ \ \isacommand{apply}\isamarkupfalse%
\ blast\isanewline
\ \ \isacommand{apply}\isamarkupfalse%
{\isacharparenleft}{\kern0pt}rule\ binmap{\isacharunderscore}{\kern0pt}row{\isacharprime}{\kern0pt}{\isacharunderscore}{\kern0pt}symmetric{\isacharparenright}{\kern0pt}\isanewline
\ \ \isacommand{using}\isamarkupfalse%
\ assms\isanewline
\ \ \isacommand{by}\isamarkupfalse%
\ auto%
\endisatagproof
{\isafoldproof}%
%
\isadelimproof
\isanewline
%
\endisadelimproof
\isanewline
\isacommand{lemma}\isamarkupfalse%
\ binmap{\isacharprime}{\kern0pt}{\isacharunderscore}{\kern0pt}in{\isacharunderscore}{\kern0pt}P{\isacharunderscore}{\kern0pt}name\ {\isacharcolon}{\kern0pt}\ \isanewline
\ \ \isakeyword{shows}\ {\isachardoublequoteopen}binmap{\isacharprime}{\kern0pt}\ {\isasymin}\ P{\isacharunderscore}{\kern0pt}names{\isachardoublequoteclose}\ \isanewline
%
\isadelimproof
\isanewline
\ \ %
\endisadelimproof
%
\isatagproof
\isacommand{apply}\isamarkupfalse%
{\isacharparenleft}{\kern0pt}rule\ iffD{\isadigit{2}}{\isacharcomma}{\kern0pt}\ rule\ P{\isacharunderscore}{\kern0pt}name{\isacharunderscore}{\kern0pt}iff{\isacharcomma}{\kern0pt}\ rule\ conjI{\isacharparenright}{\kern0pt}\isanewline
\ \ \ \isacommand{apply}\isamarkupfalse%
{\isacharparenleft}{\kern0pt}rule\ binmap{\isacharprime}{\kern0pt}{\isacharunderscore}{\kern0pt}in{\isacharunderscore}{\kern0pt}M{\isacharparenright}{\kern0pt}\isanewline
\ \ \isacommand{unfolding}\isamarkupfalse%
\ binmap{\isacharprime}{\kern0pt}{\isacharunderscore}{\kern0pt}def\ \isanewline
\ \ \isacommand{using}\isamarkupfalse%
\ zero{\isacharunderscore}{\kern0pt}in{\isacharunderscore}{\kern0pt}Fn\ binmap{\isacharunderscore}{\kern0pt}row{\isacharprime}{\kern0pt}{\isacharunderscore}{\kern0pt}P{\isacharunderscore}{\kern0pt}name\ \isanewline
\ \ \isacommand{by}\isamarkupfalse%
\ auto%
\endisatagproof
{\isafoldproof}%
%
\isadelimproof
\isanewline
%
\endisadelimproof
\isanewline
\isacommand{lemma}\isamarkupfalse%
\ pauto{\isacharunderscore}{\kern0pt}{\isadigit{0}}\ {\isacharcolon}{\kern0pt}\ \isanewline
\ \ \isakeyword{fixes}\ F\ \isanewline
\ \ \isakeyword{assumes}\ {\isachardoublequoteopen}F\ {\isasymin}\ Fn{\isacharunderscore}{\kern0pt}perms{\isachardoublequoteclose}\ \isanewline
\ \ \isakeyword{shows}\ {\isachardoublequoteopen}F{\isacharbackquote}{\kern0pt}{\isadigit{0}}\ {\isacharequal}{\kern0pt}\ {\isadigit{0}}{\isachardoublequoteclose}\ \isanewline
%
\isadelimproof
\ \ %
\endisadelimproof
%
\isatagproof
\isacommand{apply}\isamarkupfalse%
{\isacharparenleft}{\kern0pt}rule\ P{\isacharunderscore}{\kern0pt}auto{\isacharunderscore}{\kern0pt}preserves{\isacharunderscore}{\kern0pt}one{\isacharparenright}{\kern0pt}\isanewline
\ \ \isacommand{using}\isamarkupfalse%
\ Fn{\isacharunderscore}{\kern0pt}perm{\isacharprime}{\kern0pt}{\isacharunderscore}{\kern0pt}is{\isacharunderscore}{\kern0pt}P{\isacharunderscore}{\kern0pt}auto\ Fn{\isacharunderscore}{\kern0pt}perms{\isacharunderscore}{\kern0pt}def\ assms\isanewline
\ \ \isacommand{by}\isamarkupfalse%
\ force%
\endisatagproof
{\isafoldproof}%
%
\isadelimproof
\isanewline
%
\endisadelimproof
\isanewline
\isacommand{lemma}\isamarkupfalse%
\ binmap{\isacharprime}{\kern0pt}{\isacharunderscore}{\kern0pt}pnauto\ {\isacharcolon}{\kern0pt}\ \isanewline
\ \ \isakeyword{fixes}\ F\ \isanewline
\ \ \isakeyword{assumes}\ {\isachardoublequoteopen}F\ {\isasymin}\ Fn{\isacharunderscore}{\kern0pt}perms{\isachardoublequoteclose}\ \isanewline
\ \ \isakeyword{shows}\ {\isachardoublequoteopen}Pn{\isacharunderscore}{\kern0pt}auto{\isacharparenleft}{\kern0pt}F{\isacharparenright}{\kern0pt}{\isacharbackquote}{\kern0pt}binmap{\isacharprime}{\kern0pt}\ {\isacharequal}{\kern0pt}\ binmap{\isacharprime}{\kern0pt}{\isachardoublequoteclose}\ \isanewline
%
\isadelimproof
%
\endisadelimproof
%
\isatagproof
\isacommand{proof}\isamarkupfalse%
{\isacharparenleft}{\kern0pt}rule\ equality{\isacharunderscore}{\kern0pt}iffI{\isacharcomma}{\kern0pt}\ rule\ iffI{\isacharparenright}{\kern0pt}\isanewline
\ \ \isacommand{fix}\isamarkupfalse%
\ v\ \isanewline
\ \ \isacommand{assume}\isamarkupfalse%
\ vin\ {\isacharcolon}{\kern0pt}\ {\isachardoublequoteopen}v\ {\isasymin}\ Pn{\isacharunderscore}{\kern0pt}auto{\isacharparenleft}{\kern0pt}F{\isacharparenright}{\kern0pt}\ {\isacharbackquote}{\kern0pt}\ binmap{\isacharprime}{\kern0pt}{\isachardoublequoteclose}\isanewline
\isanewline
\ \ \isacommand{obtain}\isamarkupfalse%
\ f\ \isakeyword{where}\ fH{\isacharcolon}{\kern0pt}\ {\isachardoublequoteopen}F\ {\isacharequal}{\kern0pt}\ Fn{\isacharunderscore}{\kern0pt}perm{\isacharprime}{\kern0pt}{\isacharparenleft}{\kern0pt}f{\isacharparenright}{\kern0pt}{\isachardoublequoteclose}\ {\isachardoublequoteopen}f\ {\isasymin}\ nat{\isacharunderscore}{\kern0pt}perms{\isachardoublequoteclose}\ \isacommand{using}\isamarkupfalse%
\ assms\ Fn{\isacharunderscore}{\kern0pt}perms{\isacharunderscore}{\kern0pt}def\ \isacommand{by}\isamarkupfalse%
\ force\isanewline
\isanewline
\ \ \isacommand{have}\isamarkupfalse%
\ {\isachardoublequoteopen}Pn{\isacharunderscore}{\kern0pt}auto{\isacharparenleft}{\kern0pt}F{\isacharparenright}{\kern0pt}{\isacharbackquote}{\kern0pt}binmap{\isacharprime}{\kern0pt}\ {\isacharequal}{\kern0pt}\ {\isacharbraceleft}{\kern0pt}\ {\isacharless}{\kern0pt}Pn{\isacharunderscore}{\kern0pt}auto{\isacharparenleft}{\kern0pt}F{\isacharparenright}{\kern0pt}{\isacharbackquote}{\kern0pt}y{\isacharcomma}{\kern0pt}\ F{\isacharbackquote}{\kern0pt}p{\isachargreater}{\kern0pt}{\isachardot}{\kern0pt}\ {\isacharless}{\kern0pt}y{\isacharcomma}{\kern0pt}\ p{\isachargreater}{\kern0pt}\ {\isasymin}\ binmap{\isacharprime}{\kern0pt}\ {\isacharbraceright}{\kern0pt}{\isachardoublequoteclose}\ \isanewline
\ \ \ \ \isacommand{by}\isamarkupfalse%
{\isacharparenleft}{\kern0pt}rule\ Pn{\isacharunderscore}{\kern0pt}auto{\isacharcomma}{\kern0pt}\ rule\ binmap{\isacharprime}{\kern0pt}{\isacharunderscore}{\kern0pt}in{\isacharunderscore}{\kern0pt}P{\isacharunderscore}{\kern0pt}name{\isacharparenright}{\kern0pt}\isanewline
\ \ \isacommand{then}\isamarkupfalse%
\ \isacommand{have}\isamarkupfalse%
\ {\isachardoublequoteopen}v\ {\isasymin}\ {\isacharbraceleft}{\kern0pt}\ {\isacharless}{\kern0pt}Pn{\isacharunderscore}{\kern0pt}auto{\isacharparenleft}{\kern0pt}F{\isacharparenright}{\kern0pt}{\isacharbackquote}{\kern0pt}y{\isacharcomma}{\kern0pt}\ F{\isacharbackquote}{\kern0pt}p{\isachargreater}{\kern0pt}{\isachardot}{\kern0pt}\ {\isacharless}{\kern0pt}y{\isacharcomma}{\kern0pt}\ p{\isachargreater}{\kern0pt}\ {\isasymin}\ binmap{\isacharprime}{\kern0pt}\ {\isacharbraceright}{\kern0pt}{\isachardoublequoteclose}\ \isanewline
\ \ \ \ \isacommand{using}\isamarkupfalse%
\ vin\ \isanewline
\ \ \ \ \isacommand{by}\isamarkupfalse%
\ auto\isanewline
\ \ \isacommand{then}\isamarkupfalse%
\ \isacommand{have}\isamarkupfalse%
\ {\isachardoublequoteopen}{\isasymexists}y\ p{\isachardot}{\kern0pt}\ {\isacharless}{\kern0pt}y{\isacharcomma}{\kern0pt}\ p{\isachargreater}{\kern0pt}\ {\isasymin}\ binmap{\isacharprime}{\kern0pt}\ {\isasymand}\ v\ {\isacharequal}{\kern0pt}\ {\isacharless}{\kern0pt}Pn{\isacharunderscore}{\kern0pt}auto{\isacharparenleft}{\kern0pt}F{\isacharparenright}{\kern0pt}{\isacharbackquote}{\kern0pt}y{\isacharcomma}{\kern0pt}\ F{\isacharbackquote}{\kern0pt}p{\isachargreater}{\kern0pt}{\isachardoublequoteclose}\ \isanewline
\ \ \ \ \isacommand{apply}\isamarkupfalse%
{\isacharparenleft}{\kern0pt}rule{\isacharunderscore}{\kern0pt}tac\ pair{\isacharunderscore}{\kern0pt}rel{\isacharunderscore}{\kern0pt}arg{\isacharparenright}{\kern0pt}\isanewline
\ \ \ \ \isacommand{using}\isamarkupfalse%
\ relation{\isacharunderscore}{\kern0pt}def\ binmap{\isacharprime}{\kern0pt}{\isacharunderscore}{\kern0pt}def\ \isanewline
\ \ \ \ \isacommand{by}\isamarkupfalse%
\ auto\isanewline
\ \ \isacommand{then}\isamarkupfalse%
\ \isacommand{obtain}\isamarkupfalse%
\ m\ \isakeyword{where}\ mH{\isacharcolon}{\kern0pt}\ {\isachardoublequoteopen}{\isacharless}{\kern0pt}binmap{\isacharunderscore}{\kern0pt}row{\isacharprime}{\kern0pt}{\isacharparenleft}{\kern0pt}m{\isacharparenright}{\kern0pt}{\isacharcomma}{\kern0pt}\ {\isadigit{0}}{\isachargreater}{\kern0pt}\ {\isasymin}\ binmap{\isacharprime}{\kern0pt}{\isachardoublequoteclose}\ {\isachardoublequoteopen}v\ {\isacharequal}{\kern0pt}\ {\isacharless}{\kern0pt}Pn{\isacharunderscore}{\kern0pt}auto{\isacharparenleft}{\kern0pt}Fn{\isacharunderscore}{\kern0pt}perm{\isacharprime}{\kern0pt}{\isacharparenleft}{\kern0pt}f{\isacharparenright}{\kern0pt}{\isacharparenright}{\kern0pt}{\isacharbackquote}{\kern0pt}binmap{\isacharunderscore}{\kern0pt}row{\isacharprime}{\kern0pt}{\isacharparenleft}{\kern0pt}m{\isacharparenright}{\kern0pt}{\isacharcomma}{\kern0pt}\ F{\isacharbackquote}{\kern0pt}{\isadigit{0}}{\isachargreater}{\kern0pt}{\isachardoublequoteclose}\ {\isachardoublequoteopen}m\ {\isasymin}\ nat{\isachardoublequoteclose}\ \isanewline
\ \ \ \ \isacommand{using}\isamarkupfalse%
\ binmap{\isacharprime}{\kern0pt}{\isacharunderscore}{\kern0pt}def\ fH\isanewline
\ \ \ \ \isacommand{by}\isamarkupfalse%
\ force\isanewline
\ \ \isacommand{then}\isamarkupfalse%
\ \isacommand{have}\isamarkupfalse%
\ {\isachardoublequoteopen}v\ {\isacharequal}{\kern0pt}\ {\isacharless}{\kern0pt}binmap{\isacharunderscore}{\kern0pt}row{\isacharprime}{\kern0pt}{\isacharparenleft}{\kern0pt}f{\isacharbackquote}{\kern0pt}m{\isacharparenright}{\kern0pt}{\isacharcomma}{\kern0pt}\ {\isadigit{0}}{\isachargreater}{\kern0pt}{\isachardoublequoteclose}\isanewline
\ \ \ \ \isacommand{using}\isamarkupfalse%
\ pauto{\isacharunderscore}{\kern0pt}{\isadigit{0}}\ assms\ binmap{\isacharunderscore}{\kern0pt}row{\isacharprime}{\kern0pt}{\isacharunderscore}{\kern0pt}pauto\ fH\isanewline
\ \ \ \ \isacommand{by}\isamarkupfalse%
\ auto\isanewline
\ \ \isacommand{then}\isamarkupfalse%
\ \isacommand{show}\isamarkupfalse%
\ {\isachardoublequoteopen}v\ {\isasymin}\ binmap{\isacharprime}{\kern0pt}{\isachardoublequoteclose}\ \isanewline
\ \ \ \ \isacommand{unfolding}\isamarkupfalse%
\ binmap{\isacharprime}{\kern0pt}{\isacharunderscore}{\kern0pt}def\ \isanewline
\ \ \ \ \isacommand{apply}\isamarkupfalse%
{\isacharparenleft}{\kern0pt}subgoal{\isacharunderscore}{\kern0pt}tac\ {\isachardoublequoteopen}f{\isacharbackquote}{\kern0pt}m\ {\isasymin}\ nat{\isachardoublequoteclose}{\isacharcomma}{\kern0pt}\ force{\isacharparenright}{\kern0pt}\isanewline
\ \ \ \ \isacommand{apply}\isamarkupfalse%
{\isacharparenleft}{\kern0pt}rule\ function{\isacharunderscore}{\kern0pt}value{\isacharunderscore}{\kern0pt}in{\isacharparenright}{\kern0pt}\isanewline
\ \ \ \ \isacommand{using}\isamarkupfalse%
\ mH\ nat{\isacharunderscore}{\kern0pt}perms{\isacharunderscore}{\kern0pt}def\ bij{\isacharunderscore}{\kern0pt}def\ inj{\isacharunderscore}{\kern0pt}def\ fH\isanewline
\ \ \ \ \isacommand{by}\isamarkupfalse%
\ auto\isanewline
\isacommand{next}\isamarkupfalse%
\ \isanewline
\isanewline
\ \ \isacommand{obtain}\isamarkupfalse%
\ f\ \isakeyword{where}\ fH{\isacharcolon}{\kern0pt}\ {\isachardoublequoteopen}F\ {\isacharequal}{\kern0pt}\ Fn{\isacharunderscore}{\kern0pt}perm{\isacharprime}{\kern0pt}{\isacharparenleft}{\kern0pt}f{\isacharparenright}{\kern0pt}{\isachardoublequoteclose}\ {\isachardoublequoteopen}f\ {\isasymin}\ nat{\isacharunderscore}{\kern0pt}perms{\isachardoublequoteclose}\ \isacommand{using}\isamarkupfalse%
\ assms\ Fn{\isacharunderscore}{\kern0pt}perms{\isacharunderscore}{\kern0pt}def\ \isacommand{by}\isamarkupfalse%
\ force\isanewline
\isanewline
\ \ \isacommand{fix}\isamarkupfalse%
\ v\ \isanewline
\ \ \isacommand{assume}\isamarkupfalse%
\ {\isachardoublequoteopen}v\ {\isasymin}\ binmap{\isacharprime}{\kern0pt}{\isachardoublequoteclose}\ \isanewline
\ \ \isacommand{then}\isamarkupfalse%
\ \isacommand{obtain}\isamarkupfalse%
\ m\ \isakeyword{where}\ mH{\isacharcolon}{\kern0pt}\ {\isachardoublequoteopen}v\ {\isacharequal}{\kern0pt}\ {\isacharless}{\kern0pt}binmap{\isacharunderscore}{\kern0pt}row{\isacharprime}{\kern0pt}{\isacharparenleft}{\kern0pt}m{\isacharparenright}{\kern0pt}{\isacharcomma}{\kern0pt}\ {\isadigit{0}}{\isachargreater}{\kern0pt}{\isachardoublequoteclose}\ {\isachardoublequoteopen}m\ {\isasymin}\ nat{\isachardoublequoteclose}\ \isacommand{using}\isamarkupfalse%
\ binmap{\isacharprime}{\kern0pt}{\isacharunderscore}{\kern0pt}def\ \isacommand{by}\isamarkupfalse%
\ force\isanewline
\isanewline
\ \ \isacommand{have}\isamarkupfalse%
\ {\isachardoublequoteopen}f\ {\isasymin}\ surj{\isacharparenleft}{\kern0pt}nat{\isacharcomma}{\kern0pt}\ nat{\isacharparenright}{\kern0pt}{\isachardoublequoteclose}\ \isacommand{using}\isamarkupfalse%
\ fH\ nat{\isacharunderscore}{\kern0pt}perms{\isacharunderscore}{\kern0pt}def\ bij{\isacharunderscore}{\kern0pt}is{\isacharunderscore}{\kern0pt}surj\ \isacommand{by}\isamarkupfalse%
\ force\isanewline
\ \ \isacommand{then}\isamarkupfalse%
\ \isacommand{obtain}\isamarkupfalse%
\ m{\isacharprime}{\kern0pt}\ \isakeyword{where}\ {\isachardoublequoteopen}m\ {\isacharequal}{\kern0pt}\ f{\isacharbackquote}{\kern0pt}m{\isacharprime}{\kern0pt}{\isachardoublequoteclose}\ \isacommand{using}\isamarkupfalse%
\ surj{\isacharunderscore}{\kern0pt}def\ mH\ \isacommand{by}\isamarkupfalse%
\ auto\isanewline
\isanewline
\ \ \isacommand{have}\isamarkupfalse%
\ {\isachardoublequoteopen}Pn{\isacharunderscore}{\kern0pt}auto{\isacharparenleft}{\kern0pt}F{\isacharparenright}{\kern0pt}{\isacharbackquote}{\kern0pt}binmap{\isacharunderscore}{\kern0pt}row{\isacharprime}{\kern0pt}{\isacharparenleft}{\kern0pt}converse{\isacharparenleft}{\kern0pt}f{\isacharparenright}{\kern0pt}{\isacharbackquote}{\kern0pt}m{\isacharparenright}{\kern0pt}\ {\isacharequal}{\kern0pt}\ Pn{\isacharunderscore}{\kern0pt}auto{\isacharparenleft}{\kern0pt}Fn{\isacharunderscore}{\kern0pt}perm{\isacharprime}{\kern0pt}{\isacharparenleft}{\kern0pt}f{\isacharparenright}{\kern0pt}{\isacharparenright}{\kern0pt}{\isacharbackquote}{\kern0pt}binmap{\isacharunderscore}{\kern0pt}row{\isacharprime}{\kern0pt}{\isacharparenleft}{\kern0pt}converse{\isacharparenleft}{\kern0pt}f{\isacharparenright}{\kern0pt}{\isacharbackquote}{\kern0pt}m{\isacharparenright}{\kern0pt}{\isachardoublequoteclose}\ \isacommand{using}\isamarkupfalse%
\ fH\ \isacommand{by}\isamarkupfalse%
\ auto\isanewline
\ \ \isacommand{also}\isamarkupfalse%
\ \isacommand{have}\isamarkupfalse%
\ {\isachardoublequoteopen}{\isachardot}{\kern0pt}{\isachardot}{\kern0pt}{\isachardot}{\kern0pt}\ {\isacharequal}{\kern0pt}\ binmap{\isacharunderscore}{\kern0pt}row{\isacharprime}{\kern0pt}{\isacharparenleft}{\kern0pt}f{\isacharbackquote}{\kern0pt}{\isacharparenleft}{\kern0pt}converse{\isacharparenleft}{\kern0pt}f{\isacharparenright}{\kern0pt}{\isacharbackquote}{\kern0pt}m{\isacharparenright}{\kern0pt}{\isacharparenright}{\kern0pt}{\isachardoublequoteclose}\isanewline
\ \ \ \ \isacommand{apply}\isamarkupfalse%
{\isacharparenleft}{\kern0pt}rule\ binmap{\isacharunderscore}{\kern0pt}row{\isacharprime}{\kern0pt}{\isacharunderscore}{\kern0pt}pauto{\isacharparenright}{\kern0pt}\isanewline
\ \ \ \ \ \isacommand{apply}\isamarkupfalse%
{\isacharparenleft}{\kern0pt}rule\ function{\isacharunderscore}{\kern0pt}value{\isacharunderscore}{\kern0pt}in{\isacharparenright}{\kern0pt}\isanewline
\ \ \ \ \ \ \isacommand{apply}\isamarkupfalse%
{\isacharparenleft}{\kern0pt}insert\ bij{\isacharunderscore}{\kern0pt}converse{\isacharunderscore}{\kern0pt}bij\ {\isacharbrackleft}{\kern0pt}of\ f\ nat\ nat{\isacharbrackright}{\kern0pt}{\isacharparenright}{\kern0pt}\isanewline
\ \ \ \ \isacommand{using}\isamarkupfalse%
\ bij{\isacharunderscore}{\kern0pt}def\ inj{\isacharunderscore}{\kern0pt}def\ fH\ nat{\isacharunderscore}{\kern0pt}perms{\isacharunderscore}{\kern0pt}def\ mH\isanewline
\ \ \ \ \isacommand{by}\isamarkupfalse%
\ auto\isanewline
\ \ \isacommand{also}\isamarkupfalse%
\ \isacommand{have}\isamarkupfalse%
\ {\isachardoublequoteopen}{\isachardot}{\kern0pt}{\isachardot}{\kern0pt}{\isachardot}{\kern0pt}\ {\isacharequal}{\kern0pt}\ binmap{\isacharunderscore}{\kern0pt}row{\isacharprime}{\kern0pt}{\isacharparenleft}{\kern0pt}m{\isacharparenright}{\kern0pt}{\isachardoublequoteclose}\ \isanewline
\ \ \ \ \isacommand{apply}\isamarkupfalse%
{\isacharparenleft}{\kern0pt}subst\ right{\isacharunderscore}{\kern0pt}inverse{\isacharunderscore}{\kern0pt}bij{\isacharparenright}{\kern0pt}\isanewline
\ \ \ \ \isacommand{using}\isamarkupfalse%
\ fH\ nat{\isacharunderscore}{\kern0pt}perms{\isacharunderscore}{\kern0pt}def\ mH\ \isanewline
\ \ \ \ \isacommand{by}\isamarkupfalse%
\ auto\isanewline
\ \ \isacommand{finally}\isamarkupfalse%
\ \isacommand{have}\isamarkupfalse%
\ eq{\isacharcolon}{\kern0pt}\ {\isachardoublequoteopen}Pn{\isacharunderscore}{\kern0pt}auto{\isacharparenleft}{\kern0pt}F{\isacharparenright}{\kern0pt}\ {\isacharbackquote}{\kern0pt}\ binmap{\isacharunderscore}{\kern0pt}row{\isacharprime}{\kern0pt}{\isacharparenleft}{\kern0pt}converse{\isacharparenleft}{\kern0pt}f{\isacharparenright}{\kern0pt}\ {\isacharbackquote}{\kern0pt}\ m{\isacharparenright}{\kern0pt}\ {\isacharequal}{\kern0pt}\ binmap{\isacharunderscore}{\kern0pt}row{\isacharprime}{\kern0pt}{\isacharparenleft}{\kern0pt}m{\isacharparenright}{\kern0pt}{\isachardoublequoteclose}\ \isacommand{by}\isamarkupfalse%
\ simp\isanewline
\isanewline
\ \ \isacommand{have}\isamarkupfalse%
\ {\isachardoublequoteopen}{\isacharless}{\kern0pt}Pn{\isacharunderscore}{\kern0pt}auto{\isacharparenleft}{\kern0pt}F{\isacharparenright}{\kern0pt}\ {\isacharbackquote}{\kern0pt}\ binmap{\isacharunderscore}{\kern0pt}row{\isacharprime}{\kern0pt}{\isacharparenleft}{\kern0pt}converse{\isacharparenleft}{\kern0pt}f{\isacharparenright}{\kern0pt}\ {\isacharbackquote}{\kern0pt}\ m{\isacharparenright}{\kern0pt}{\isacharcomma}{\kern0pt}\ F{\isacharbackquote}{\kern0pt}{\isadigit{0}}{\isachargreater}{\kern0pt}\ {\isasymin}\ Pn{\isacharunderscore}{\kern0pt}auto{\isacharparenleft}{\kern0pt}F{\isacharparenright}{\kern0pt}\ {\isacharbackquote}{\kern0pt}\ binmap{\isacharprime}{\kern0pt}{\isachardoublequoteclose}\ \ \isanewline
\ \ \ \ \isacommand{apply}\isamarkupfalse%
{\isacharparenleft}{\kern0pt}subst\ {\isacharparenleft}{\kern0pt}{\isadigit{2}}{\isacharparenright}{\kern0pt}\ Pn{\isacharunderscore}{\kern0pt}auto{\isacharcomma}{\kern0pt}\ rule\ binmap{\isacharprime}{\kern0pt}{\isacharunderscore}{\kern0pt}in{\isacharunderscore}{\kern0pt}P{\isacharunderscore}{\kern0pt}name{\isacharparenright}{\kern0pt}\isanewline
\ \ \ \ \isacommand{apply}\isamarkupfalse%
{\isacharparenleft}{\kern0pt}rule\ pair{\isacharunderscore}{\kern0pt}relI{\isacharcomma}{\kern0pt}\ simp\ add{\isacharcolon}{\kern0pt}binmap{\isacharprime}{\kern0pt}{\isacharunderscore}{\kern0pt}def{\isacharparenright}{\kern0pt}\isanewline
\ \ \ \ \isacommand{apply}\isamarkupfalse%
{\isacharparenleft}{\kern0pt}rule{\isacharunderscore}{\kern0pt}tac\ x{\isacharequal}{\kern0pt}{\isachardoublequoteopen}converse{\isacharparenleft}{\kern0pt}f{\isacharparenright}{\kern0pt}{\isacharbackquote}{\kern0pt}m{\isachardoublequoteclose}\ \isakeyword{in}\ bexI{\isacharcomma}{\kern0pt}\ simp{\isacharparenright}{\kern0pt}\isanewline
\ \ \ \ \ \isacommand{apply}\isamarkupfalse%
{\isacharparenleft}{\kern0pt}rule\ function{\isacharunderscore}{\kern0pt}value{\isacharunderscore}{\kern0pt}in{\isacharparenright}{\kern0pt}\isanewline
\ \ \ \ \ \ \isacommand{apply}\isamarkupfalse%
{\isacharparenleft}{\kern0pt}insert\ bij{\isacharunderscore}{\kern0pt}converse{\isacharunderscore}{\kern0pt}bij\ {\isacharbrackleft}{\kern0pt}of\ f\ nat\ nat{\isacharbrackright}{\kern0pt}{\isacharparenright}{\kern0pt}\isanewline
\ \ \ \ \isacommand{using}\isamarkupfalse%
\ bij{\isacharunderscore}{\kern0pt}def\ inj{\isacharunderscore}{\kern0pt}def\ fH\ nat{\isacharunderscore}{\kern0pt}perms{\isacharunderscore}{\kern0pt}def\ mH\isanewline
\ \ \ \ \isacommand{by}\isamarkupfalse%
\ auto\isanewline
\ \ \isacommand{then}\isamarkupfalse%
\ \isacommand{have}\isamarkupfalse%
\ {\isachardoublequoteopen}{\isacharless}{\kern0pt}binmap{\isacharunderscore}{\kern0pt}row{\isacharprime}{\kern0pt}{\isacharparenleft}{\kern0pt}m{\isacharparenright}{\kern0pt}{\isacharcomma}{\kern0pt}\ {\isadigit{0}}{\isachargreater}{\kern0pt}\ {\isasymin}\ Pn{\isacharunderscore}{\kern0pt}auto{\isacharparenleft}{\kern0pt}F{\isacharparenright}{\kern0pt}\ {\isacharbackquote}{\kern0pt}\ binmap{\isacharprime}{\kern0pt}{\isachardoublequoteclose}\ \ \isanewline
\ \ \ \ \isacommand{using}\isamarkupfalse%
\ eq\ pauto{\isacharunderscore}{\kern0pt}{\isadigit{0}}\ assms\ \isanewline
\ \ \ \ \isacommand{by}\isamarkupfalse%
\ auto\isanewline
\ \ \isacommand{then}\isamarkupfalse%
\ \isacommand{show}\isamarkupfalse%
\ {\isachardoublequoteopen}v\ {\isasymin}\ Pn{\isacharunderscore}{\kern0pt}auto{\isacharparenleft}{\kern0pt}F{\isacharparenright}{\kern0pt}\ {\isacharbackquote}{\kern0pt}\ binmap{\isacharprime}{\kern0pt}{\isachardoublequoteclose}\ \ \isanewline
\ \ \ \ \isacommand{using}\isamarkupfalse%
\ mH\isanewline
\ \ \ \ \isacommand{by}\isamarkupfalse%
\ auto\isanewline
\isacommand{qed}\isamarkupfalse%
%
\endisatagproof
{\isafoldproof}%
%
\isadelimproof
\isanewline
%
\endisadelimproof
\isanewline
\isacommand{lemma}\isamarkupfalse%
\ sym{\isacharunderscore}{\kern0pt}binmap{\isacharprime}{\kern0pt}\ {\isacharcolon}{\kern0pt}\ {\isachardoublequoteopen}sym{\isacharparenleft}{\kern0pt}binmap{\isacharprime}{\kern0pt}{\isacharparenright}{\kern0pt}\ {\isacharequal}{\kern0pt}\ Fn{\isacharunderscore}{\kern0pt}perms{\isachardoublequoteclose}\ \isanewline
%
\isadelimproof
\ \ %
\endisadelimproof
%
\isatagproof
\isacommand{unfolding}\isamarkupfalse%
\ sym{\isacharunderscore}{\kern0pt}def\ \isanewline
\ \ \isacommand{using}\isamarkupfalse%
\ binmap{\isacharprime}{\kern0pt}{\isacharunderscore}{\kern0pt}pnauto\ \isanewline
\ \ \isacommand{by}\isamarkupfalse%
\ auto%
\endisatagproof
{\isafoldproof}%
%
\isadelimproof
\isanewline
%
\endisadelimproof
\isanewline
\isacommand{lemma}\isamarkupfalse%
\ binmap{\isacharprime}{\kern0pt}{\isacharunderscore}{\kern0pt}HS\ {\isacharcolon}{\kern0pt}\ {\isachardoublequoteopen}binmap{\isacharprime}{\kern0pt}\ {\isasymin}\ HS{\isachardoublequoteclose}\ \isanewline
%
\isadelimproof
\ \ %
\endisadelimproof
%
\isatagproof
\isacommand{apply}\isamarkupfalse%
{\isacharparenleft}{\kern0pt}rule\ iffD{\isadigit{2}}{\isacharcomma}{\kern0pt}\ rule\ HS{\isacharunderscore}{\kern0pt}iff{\isacharcomma}{\kern0pt}\ rule\ conjI{\isacharparenright}{\kern0pt}\isanewline
\ \ \ \isacommand{apply}\isamarkupfalse%
{\isacharparenleft}{\kern0pt}rule\ iffD{\isadigit{2}}{\isacharcomma}{\kern0pt}\ rule\ P{\isacharunderscore}{\kern0pt}name{\isacharunderscore}{\kern0pt}iff{\isacharcomma}{\kern0pt}\ rule\ conjI{\isacharparenright}{\kern0pt}\isanewline
\ \ \ \ \isacommand{apply}\isamarkupfalse%
{\isacharparenleft}{\kern0pt}rule\ binmap{\isacharprime}{\kern0pt}{\isacharunderscore}{\kern0pt}in{\isacharunderscore}{\kern0pt}M{\isacharparenright}{\kern0pt}\isanewline
\ \ \isacommand{using}\isamarkupfalse%
\ binmap{\isacharprime}{\kern0pt}{\isacharunderscore}{\kern0pt}def\ binmap{\isacharunderscore}{\kern0pt}row{\isacharprime}{\kern0pt}{\isacharunderscore}{\kern0pt}P{\isacharunderscore}{\kern0pt}name\ zero{\isacharunderscore}{\kern0pt}in{\isacharunderscore}{\kern0pt}Fn\ \isanewline
\ \ \ \isacommand{apply}\isamarkupfalse%
\ force\isanewline
\ \ \isacommand{apply}\isamarkupfalse%
{\isacharparenleft}{\kern0pt}rule\ conjI{\isacharparenright}{\kern0pt}\isanewline
\ \ \isacommand{using}\isamarkupfalse%
\ binmap{\isacharprime}{\kern0pt}{\isacharunderscore}{\kern0pt}def\ binmap{\isacharunderscore}{\kern0pt}row{\isacharprime}{\kern0pt}{\isacharunderscore}{\kern0pt}HS\ \isanewline
\ \ \ \isacommand{apply}\isamarkupfalse%
\ force\isanewline
\ \ \isacommand{unfolding}\isamarkupfalse%
\ symmetric{\isacharunderscore}{\kern0pt}def\ \isanewline
\ \ \isacommand{unfolding}\isamarkupfalse%
\ Fn{\isacharunderscore}{\kern0pt}perms{\isacharunderscore}{\kern0pt}filter{\isacharunderscore}{\kern0pt}def\isanewline
\ \ \isacommand{apply}\isamarkupfalse%
\ simp\isanewline
\ \ \isacommand{apply}\isamarkupfalse%
{\isacharparenleft}{\kern0pt}rule\ conjI{\isacharcomma}{\kern0pt}\ rule\ sym{\isacharunderscore}{\kern0pt}P{\isacharunderscore}{\kern0pt}auto{\isacharunderscore}{\kern0pt}subgroup{\isacharparenright}{\kern0pt}\isanewline
\ \ \ \isacommand{apply}\isamarkupfalse%
{\isacharparenleft}{\kern0pt}rule\ iffD{\isadigit{2}}{\isacharcomma}{\kern0pt}\ rule\ P{\isacharunderscore}{\kern0pt}name{\isacharunderscore}{\kern0pt}iff{\isacharcomma}{\kern0pt}\ rule\ conjI{\isacharparenright}{\kern0pt}\isanewline
\ \ \ \ \isacommand{apply}\isamarkupfalse%
{\isacharparenleft}{\kern0pt}rule\ binmap{\isacharprime}{\kern0pt}{\isacharunderscore}{\kern0pt}in{\isacharunderscore}{\kern0pt}M{\isacharparenright}{\kern0pt}\isanewline
\ \ \isacommand{using}\isamarkupfalse%
\ binmap{\isacharprime}{\kern0pt}{\isacharunderscore}{\kern0pt}def\ binmap{\isacharunderscore}{\kern0pt}row{\isacharprime}{\kern0pt}{\isacharunderscore}{\kern0pt}P{\isacharunderscore}{\kern0pt}name\ zero{\isacharunderscore}{\kern0pt}in{\isacharunderscore}{\kern0pt}Fn\ \isanewline
\ \ \ \isacommand{apply}\isamarkupfalse%
\ force\isanewline
\ \ \isacommand{apply}\isamarkupfalse%
{\isacharparenleft}{\kern0pt}rule{\isacharunderscore}{\kern0pt}tac\ x{\isacharequal}{\kern0pt}{\isadigit{0}}\ \isakeyword{in}\ bexI{\isacharparenright}{\kern0pt}\isanewline
\ \ \isacommand{unfolding}\isamarkupfalse%
\ finite{\isacharunderscore}{\kern0pt}M{\isacharunderscore}{\kern0pt}def\ \isanewline
\ \ \ \isacommand{apply}\isamarkupfalse%
{\isacharparenleft}{\kern0pt}rule\ conjI{\isacharcomma}{\kern0pt}\ rule{\isacharunderscore}{\kern0pt}tac\ x{\isacharequal}{\kern0pt}{\isadigit{0}}\ \isakeyword{in}\ bexI{\isacharparenright}{\kern0pt}\isanewline
\ \ \isacommand{apply}\isamarkupfalse%
{\isacharparenleft}{\kern0pt}rule{\isacharunderscore}{\kern0pt}tac\ a{\isacharequal}{\kern0pt}{\isadigit{0}}\ \isakeyword{in}\ not{\isacharunderscore}{\kern0pt}emptyI{\isacharparenright}{\kern0pt}\isanewline
\ \ \isacommand{using}\isamarkupfalse%
\ inj{\isacharunderscore}{\kern0pt}def\ zero{\isacharunderscore}{\kern0pt}in{\isacharunderscore}{\kern0pt}M\isanewline
\ \ \ \ \ \isacommand{apply}\isamarkupfalse%
\ auto{\isacharbrackleft}{\kern0pt}{\isadigit{2}}{\isacharbrackright}{\kern0pt}\isanewline
\ \ \ \isacommand{apply}\isamarkupfalse%
{\isacharparenleft}{\kern0pt}subst\ sym{\isacharunderscore}{\kern0pt}binmap{\isacharprime}{\kern0pt}{\isacharparenright}{\kern0pt}\isanewline
\ \ \isacommand{using}\isamarkupfalse%
\ Fix{\isacharunderscore}{\kern0pt}def\ Fn{\isacharunderscore}{\kern0pt}perms{\isacharunderscore}{\kern0pt}def\ zero{\isacharunderscore}{\kern0pt}in{\isacharunderscore}{\kern0pt}M\isanewline
\ \ \isacommand{by}\isamarkupfalse%
\ auto%
\endisatagproof
{\isafoldproof}%
%
\isadelimproof
\isanewline
%
\endisadelimproof
\isanewline
\isacommand{lemma}\isamarkupfalse%
\ finite{\isacharunderscore}{\kern0pt}M{\isacharunderscore}{\kern0pt}cons\ {\isacharcolon}{\kern0pt}\ \isanewline
\ \ \isakeyword{fixes}\ A\ x\ \isanewline
\ \ \isakeyword{assumes}\ {\isachardoublequoteopen}finite{\isacharunderscore}{\kern0pt}M{\isacharparenleft}{\kern0pt}A{\isacharparenright}{\kern0pt}{\isachardoublequoteclose}\ {\isachardoublequoteopen}x\ {\isasymin}\ M{\isachardoublequoteclose}\ {\isachardoublequoteopen}x\ {\isasymnotin}\ A{\isachardoublequoteclose}\ \ \isanewline
\ \ \isakeyword{shows}\ {\isachardoublequoteopen}finite{\isacharunderscore}{\kern0pt}M{\isacharparenleft}{\kern0pt}cons{\isacharparenleft}{\kern0pt}x{\isacharcomma}{\kern0pt}\ A{\isacharparenright}{\kern0pt}{\isacharparenright}{\kern0pt}{\isachardoublequoteclose}\ \isanewline
%
\isadelimproof
%
\endisadelimproof
%
\isatagproof
\isacommand{proof}\isamarkupfalse%
\ {\isacharminus}{\kern0pt}\ \isanewline
\isanewline
\ \ \isacommand{obtain}\isamarkupfalse%
\ n\ f\ \isakeyword{where}\ nfH{\isacharcolon}{\kern0pt}\ {\isachardoublequoteopen}n\ {\isasymin}\ nat{\isachardoublequoteclose}\ {\isachardoublequoteopen}f\ {\isasymin}\ inj{\isacharparenleft}{\kern0pt}A{\isacharcomma}{\kern0pt}\ n{\isacharparenright}{\kern0pt}{\isachardoublequoteclose}\ {\isachardoublequoteopen}f\ {\isasymin}\ M{\isachardoublequoteclose}\ \isacommand{using}\isamarkupfalse%
\ assms\ finite{\isacharunderscore}{\kern0pt}M{\isacharunderscore}{\kern0pt}def\ \isacommand{by}\isamarkupfalse%
\ force\isanewline
\isanewline
\ \ \isacommand{define}\isamarkupfalse%
\ g\ \isakeyword{where}\ {\isachardoublequoteopen}g\ {\isasymequiv}\ f\ {\isasymunion}\ {\isacharbraceleft}{\kern0pt}\ {\isacharless}{\kern0pt}x{\isacharcomma}{\kern0pt}\ n{\isachargreater}{\kern0pt}\ {\isacharbraceright}{\kern0pt}{\isachardoublequoteclose}\ \isanewline
\isanewline
\ \ \isacommand{have}\isamarkupfalse%
\ {\isachardoublequoteopen}{\isasymAnd}y{\isachardot}{\kern0pt}\ {\isacharless}{\kern0pt}x{\isacharcomma}{\kern0pt}\ y{\isachargreater}{\kern0pt}\ {\isasymnotin}\ f{\isachardoublequoteclose}\ \isanewline
\ \ \ \ \isacommand{using}\isamarkupfalse%
\ nfH\ inj{\isacharunderscore}{\kern0pt}def\ Pi{\isacharunderscore}{\kern0pt}def\ assms\isanewline
\ \ \ \ \isacommand{by}\isamarkupfalse%
\ force\isanewline
\ \ \isacommand{then}\isamarkupfalse%
\ \isacommand{have}\isamarkupfalse%
\ functionH{\isacharcolon}{\kern0pt}\ {\isachardoublequoteopen}function{\isacharparenleft}{\kern0pt}g{\isacharparenright}{\kern0pt}{\isachardoublequoteclose}\ \isanewline
\ \ \ \ \isacommand{unfolding}\isamarkupfalse%
\ function{\isacharunderscore}{\kern0pt}def\ g{\isacharunderscore}{\kern0pt}def\isanewline
\ \ \ \ \isacommand{using}\isamarkupfalse%
\ nfH\ inj{\isacharunderscore}{\kern0pt}def\ Pi{\isacharunderscore}{\kern0pt}def\ assms\ function{\isacharunderscore}{\kern0pt}def\isanewline
\ \ \ \ \isacommand{by}\isamarkupfalse%
\ force\isanewline
\isanewline
\ \ \isacommand{have}\isamarkupfalse%
\ {\isachardoublequoteopen}domain{\isacharparenleft}{\kern0pt}g{\isacharparenright}{\kern0pt}\ {\isacharequal}{\kern0pt}\ domain{\isacharparenleft}{\kern0pt}f{\isacharparenright}{\kern0pt}\ {\isasymunion}\ {\isacharbraceleft}{\kern0pt}x{\isacharbraceright}{\kern0pt}{\isachardoublequoteclose}\ \isanewline
\ \ \ \ \isacommand{unfolding}\isamarkupfalse%
\ g{\isacharunderscore}{\kern0pt}def\ \isanewline
\ \ \ \ \isacommand{by}\isamarkupfalse%
\ auto\isanewline
\ \ \isacommand{then}\isamarkupfalse%
\ \isacommand{have}\isamarkupfalse%
\ domH{\isacharcolon}{\kern0pt}\ {\isachardoublequoteopen}domain{\isacharparenleft}{\kern0pt}g{\isacharparenright}{\kern0pt}\ {\isacharequal}{\kern0pt}\ cons{\isacharparenleft}{\kern0pt}x{\isacharcomma}{\kern0pt}\ A{\isacharparenright}{\kern0pt}{\isachardoublequoteclose}\isanewline
\ \ \ \ \isacommand{using}\isamarkupfalse%
\ cons{\isacharunderscore}{\kern0pt}def\ assms\ nfH\ inj{\isacharunderscore}{\kern0pt}def\ Pi{\isacharunderscore}{\kern0pt}def\isanewline
\ \ \ \ \isacommand{by}\isamarkupfalse%
\ auto\isanewline
\isanewline
\ \ \isacommand{have}\isamarkupfalse%
\ ranH{\isacharcolon}{\kern0pt}\ {\isachardoublequoteopen}range{\isacharparenleft}{\kern0pt}g{\isacharparenright}{\kern0pt}\ {\isasymsubseteq}\ succ{\isacharparenleft}{\kern0pt}n{\isacharparenright}{\kern0pt}{\isachardoublequoteclose}\ \isanewline
\ \ \ \ \isacommand{unfolding}\isamarkupfalse%
\ g{\isacharunderscore}{\kern0pt}def\ \isanewline
\ \ \ \ \isacommand{using}\isamarkupfalse%
\ nfH\ inj{\isacharunderscore}{\kern0pt}def\ Pi{\isacharunderscore}{\kern0pt}def\isanewline
\ \ \ \ \isacommand{by}\isamarkupfalse%
\ auto\isanewline
\ \ \isanewline
\ \ \isacommand{have}\isamarkupfalse%
\ gpiH{\isacharcolon}{\kern0pt}\ {\isachardoublequoteopen}g\ {\isasymin}\ cons{\isacharparenleft}{\kern0pt}x{\isacharcomma}{\kern0pt}\ A{\isacharparenright}{\kern0pt}\ {\isasymrightarrow}\ succ{\isacharparenleft}{\kern0pt}n{\isacharparenright}{\kern0pt}{\isachardoublequoteclose}\ \isanewline
\ \ \ \ \isacommand{apply}\isamarkupfalse%
{\isacharparenleft}{\kern0pt}rule\ Pi{\isacharunderscore}{\kern0pt}memberI{\isacharparenright}{\kern0pt}\isanewline
\ \ \ \ \isacommand{using}\isamarkupfalse%
\ relation{\isacharunderscore}{\kern0pt}def\ g{\isacharunderscore}{\kern0pt}def\ nfH\ inj{\isacharunderscore}{\kern0pt}def\ Pi{\isacharunderscore}{\kern0pt}def\isanewline
\ \ \ \ \ \ \ \isacommand{apply}\isamarkupfalse%
\ force\isanewline
\ \ \ \ \isacommand{using}\isamarkupfalse%
\ functionH\ domH\ ranH\ \isanewline
\ \ \ \ \isacommand{by}\isamarkupfalse%
\ auto\isanewline
\isanewline
\ \ \isacommand{have}\isamarkupfalse%
\ appAH{\isacharcolon}{\kern0pt}\ {\isachardoublequoteopen}{\isasymAnd}a{\isachardot}{\kern0pt}\ a\ {\isasymin}\ A\ {\isasymLongrightarrow}\ g{\isacharbackquote}{\kern0pt}a\ {\isacharequal}{\kern0pt}\ f{\isacharbackquote}{\kern0pt}a{\isachardoublequoteclose}\ \isanewline
\ \ \ \ \isacommand{apply}\isamarkupfalse%
{\isacharparenleft}{\kern0pt}rule\ function{\isacharunderscore}{\kern0pt}apply{\isacharunderscore}{\kern0pt}equality{\isacharparenright}{\kern0pt}\isanewline
\ \ \ \ \isacommand{unfolding}\isamarkupfalse%
\ g{\isacharunderscore}{\kern0pt}def\ \isanewline
\ \ \ \ \ \isacommand{apply}\isamarkupfalse%
\ simp\isanewline
\ \ \ \ \ \isacommand{apply}\isamarkupfalse%
{\isacharparenleft}{\kern0pt}rule\ disjI{\isadigit{1}}{\isacharcomma}{\kern0pt}\ rule\ function{\isacharunderscore}{\kern0pt}apply{\isacharunderscore}{\kern0pt}Pair{\isacharparenright}{\kern0pt}\isanewline
\ \ \ \ \isacommand{using}\isamarkupfalse%
\ nfH\ inj{\isacharunderscore}{\kern0pt}def\ Pi{\isacharunderscore}{\kern0pt}def\ g{\isacharunderscore}{\kern0pt}def\ functionH\isanewline
\ \ \ \ \isacommand{by}\isamarkupfalse%
\ auto\isanewline
\isanewline
\ \ \isacommand{have}\isamarkupfalse%
\ appAH{\isacharprime}{\kern0pt}\ {\isacharcolon}{\kern0pt}\ {\isachardoublequoteopen}{\isasymAnd}a{\isachardot}{\kern0pt}\ a\ {\isasymin}\ A\ {\isasymLongrightarrow}\ g{\isacharbackquote}{\kern0pt}a\ {\isasymin}\ n{\isachardoublequoteclose}\ \isanewline
\ \ \isacommand{proof}\isamarkupfalse%
\ {\isacharminus}{\kern0pt}\ \isanewline
\ \ \ \ \isacommand{fix}\isamarkupfalse%
\ a\ \isanewline
\ \ \ \ \isacommand{assume}\isamarkupfalse%
\ ain\ {\isacharcolon}{\kern0pt}\ {\isachardoublequoteopen}a\ {\isasymin}\ A{\isachardoublequoteclose}\ \isanewline
\ \ \ \ \isacommand{then}\isamarkupfalse%
\ \isacommand{have}\isamarkupfalse%
\ {\isachardoublequoteopen}g{\isacharbackquote}{\kern0pt}a\ {\isacharequal}{\kern0pt}\ f{\isacharbackquote}{\kern0pt}a{\isachardoublequoteclose}\ \isacommand{using}\isamarkupfalse%
\ appAH\ \isacommand{by}\isamarkupfalse%
\ auto\isanewline
\ \ \ \ \isacommand{have}\isamarkupfalse%
\ {\isachardoublequoteopen}a\ {\isasymin}\ domain{\isacharparenleft}{\kern0pt}f{\isacharparenright}{\kern0pt}{\isachardoublequoteclose}\ \isacommand{using}\isamarkupfalse%
\ nfH\ inj{\isacharunderscore}{\kern0pt}def\ Pi{\isacharunderscore}{\kern0pt}def\ ain\ \isacommand{by}\isamarkupfalse%
\ auto\isanewline
\ \ \ \ \isacommand{then}\isamarkupfalse%
\ \isacommand{obtain}\isamarkupfalse%
\ m\ \isakeyword{where}\ mH{\isacharcolon}{\kern0pt}\ {\isachardoublequoteopen}{\isacharless}{\kern0pt}a{\isacharcomma}{\kern0pt}\ m{\isachargreater}{\kern0pt}\ {\isasymin}\ f{\isachardoublequoteclose}\ \isacommand{by}\isamarkupfalse%
\ auto\isanewline
\ \ \ \ \isacommand{then}\isamarkupfalse%
\ \isacommand{have}\isamarkupfalse%
\ {\isachardoublequoteopen}f{\isacharbackquote}{\kern0pt}a\ {\isacharequal}{\kern0pt}\ m{\isachardoublequoteclose}\ \isacommand{using}\isamarkupfalse%
\ function{\isacharunderscore}{\kern0pt}apply{\isacharunderscore}{\kern0pt}equality\ nfH\ inj{\isacharunderscore}{\kern0pt}def\ Pi{\isacharunderscore}{\kern0pt}def\ \isacommand{by}\isamarkupfalse%
\ auto\isanewline
\ \ \ \ \isacommand{have}\isamarkupfalse%
\ {\isachardoublequoteopen}m\ {\isasymin}\ n{\isachardoublequoteclose}\ \isacommand{using}\isamarkupfalse%
\ nfH\ inj{\isacharunderscore}{\kern0pt}def\ Pi{\isacharunderscore}{\kern0pt}def\ mH\ \isacommand{by}\isamarkupfalse%
\ auto\isanewline
\ \ \ \ \isacommand{then}\isamarkupfalse%
\ \isacommand{show}\isamarkupfalse%
\ {\isachardoublequoteopen}g{\isacharbackquote}{\kern0pt}a\ {\isasymin}\ n{\isachardoublequoteclose}\ \isacommand{using}\isamarkupfalse%
\ {\isacartoucheopen}f{\isacharbackquote}{\kern0pt}a\ {\isacharequal}{\kern0pt}\ m{\isacartoucheclose}\ {\isacartoucheopen}g{\isacharbackquote}{\kern0pt}a\ {\isacharequal}{\kern0pt}\ f{\isacharbackquote}{\kern0pt}a{\isacartoucheclose}\ \isacommand{by}\isamarkupfalse%
\ auto\isanewline
\ \ \isacommand{qed}\isamarkupfalse%
\ \ \ \ \isanewline
\isanewline
\ \ \isacommand{have}\isamarkupfalse%
\ appxH{\isacharcolon}{\kern0pt}\ {\isachardoublequoteopen}g{\isacharbackquote}{\kern0pt}x\ {\isacharequal}{\kern0pt}\ n{\isachardoublequoteclose}\ \isanewline
\ \ \ \ \isacommand{apply}\isamarkupfalse%
{\isacharparenleft}{\kern0pt}rule\ function{\isacharunderscore}{\kern0pt}apply{\isacharunderscore}{\kern0pt}equality{\isacharparenright}{\kern0pt}\isanewline
\ \ \ \ \isacommand{using}\isamarkupfalse%
\ g{\isacharunderscore}{\kern0pt}def\ functionH\isanewline
\ \ \ \ \isacommand{by}\isamarkupfalse%
\ auto\isanewline
\isanewline
\ \ \isacommand{have}\isamarkupfalse%
\ appneq\ {\isacharcolon}{\kern0pt}\ {\isachardoublequoteopen}{\isasymAnd}a{\isachardot}{\kern0pt}\ a\ {\isasymin}\ A\ {\isasymLongrightarrow}\ g{\isacharbackquote}{\kern0pt}a\ {\isasymnoteq}\ g{\isacharbackquote}{\kern0pt}x{\isachardoublequoteclose}\ \isanewline
\ \ \ \ \isacommand{apply}\isamarkupfalse%
{\isacharparenleft}{\kern0pt}rule\ ccontr{\isacharparenright}{\kern0pt}\isanewline
\ \ \ \ \isacommand{using}\isamarkupfalse%
\ appAH{\isacharprime}{\kern0pt}\ appxH\isanewline
\ \ \ \ \isacommand{apply}\isamarkupfalse%
\ simp\isanewline
\ \ \ \ \isacommand{apply}\isamarkupfalse%
{\isacharparenleft}{\kern0pt}subgoal{\isacharunderscore}{\kern0pt}tac\ {\isachardoublequoteopen}n\ {\isasymin}\ n{\isachardoublequoteclose}{\isacharcomma}{\kern0pt}\ rule\ mem{\isacharunderscore}{\kern0pt}irrefl{\isacharcomma}{\kern0pt}\ simp{\isacharparenright}{\kern0pt}\isanewline
\ \ \ \ \isacommand{by}\isamarkupfalse%
\ auto\isanewline
\isanewline
\ \ \isacommand{have}\isamarkupfalse%
\ appneq{\isacharprime}{\kern0pt}\ {\isacharcolon}{\kern0pt}\ {\isachardoublequoteopen}{\isasymAnd}a\ b{\isachardot}{\kern0pt}\ a\ {\isasymin}\ A\ {\isasymLongrightarrow}\ b\ {\isasymin}\ A\ {\isasymLongrightarrow}\ g{\isacharbackquote}{\kern0pt}a\ {\isacharequal}{\kern0pt}\ g{\isacharbackquote}{\kern0pt}b\ {\isasymLongrightarrow}\ a\ {\isacharequal}{\kern0pt}\ b{\isachardoublequoteclose}\ \isanewline
\ \ \ \ \isacommand{using}\isamarkupfalse%
\ appAH\ nfH\ inj{\isacharunderscore}{\kern0pt}def\ \isanewline
\ \ \ \ \isacommand{by}\isamarkupfalse%
\ force\isanewline
\isanewline
\ \ \isacommand{have}\isamarkupfalse%
\ {\isachardoublequoteopen}{\isasymAnd}a\ b{\isachardot}{\kern0pt}\ a\ {\isasymin}\ cons{\isacharparenleft}{\kern0pt}x{\isacharcomma}{\kern0pt}\ A{\isacharparenright}{\kern0pt}\ {\isasymLongrightarrow}\ b\ {\isasymin}\ cons{\isacharparenleft}{\kern0pt}x{\isacharcomma}{\kern0pt}\ A{\isacharparenright}{\kern0pt}\ {\isasymLongrightarrow}\ g{\isacharbackquote}{\kern0pt}a\ {\isacharequal}{\kern0pt}\ g{\isacharbackquote}{\kern0pt}b\ {\isasymLongrightarrow}\ a\ {\isacharequal}{\kern0pt}\ b{\isachardoublequoteclose}\isanewline
\ \ \ \ \isacommand{apply}\isamarkupfalse%
\ auto\isanewline
\ \ \ \ \isacommand{using}\isamarkupfalse%
\ appneq\isanewline
\ \ \ \ \isacommand{apply}\isamarkupfalse%
\ auto{\isacharbrackleft}{\kern0pt}{\isadigit{2}}{\isacharbrackright}{\kern0pt}\ \isanewline
\ \ \ \ \isacommand{apply}\isamarkupfalse%
{\isacharparenleft}{\kern0pt}rule\ appneq{\isacharprime}{\kern0pt}{\isacharparenright}{\kern0pt}\isanewline
\ \ \ \ \isacommand{by}\isamarkupfalse%
\ auto\isanewline
\ \ \isacommand{then}\isamarkupfalse%
\ \isacommand{have}\isamarkupfalse%
\ {\isachardoublequoteopen}g\ {\isasymin}\ inj{\isacharparenleft}{\kern0pt}cons{\isacharparenleft}{\kern0pt}x{\isacharcomma}{\kern0pt}\ A{\isacharparenright}{\kern0pt}{\isacharcomma}{\kern0pt}\ succ{\isacharparenleft}{\kern0pt}n{\isacharparenright}{\kern0pt}{\isacharparenright}{\kern0pt}{\isachardoublequoteclose}\ \isanewline
\ \ \ \ \isacommand{unfolding}\isamarkupfalse%
\ inj{\isacharunderscore}{\kern0pt}def\isanewline
\ \ \ \ \isacommand{using}\isamarkupfalse%
\ gpiH\ \isanewline
\ \ \ \ \isacommand{by}\isamarkupfalse%
\ auto\isanewline
\isanewline
\ \ \isacommand{then}\isamarkupfalse%
\ \isacommand{show}\isamarkupfalse%
\ {\isacharquery}{\kern0pt}thesis\ \isanewline
\ \ \ \ \isacommand{unfolding}\isamarkupfalse%
\ finite{\isacharunderscore}{\kern0pt}M{\isacharunderscore}{\kern0pt}def\ \isanewline
\ \ \ \ \isacommand{apply}\isamarkupfalse%
{\isacharparenleft}{\kern0pt}rule{\isacharunderscore}{\kern0pt}tac\ x{\isacharequal}{\kern0pt}{\isachardoublequoteopen}succ{\isacharparenleft}{\kern0pt}n{\isacharparenright}{\kern0pt}{\isachardoublequoteclose}\ \isakeyword{in}\ bexI{\isacharparenright}{\kern0pt}\isanewline
\ \ \ \ \ \isacommand{apply}\isamarkupfalse%
{\isacharparenleft}{\kern0pt}subgoal{\isacharunderscore}{\kern0pt}tac\ {\isachardoublequoteopen}g\ {\isasymin}\ M{\isachardoublequoteclose}{\isacharcomma}{\kern0pt}\ force{\isacharparenright}{\kern0pt}\isanewline
\ \ \ \ \isacommand{unfolding}\isamarkupfalse%
\ g{\isacharunderscore}{\kern0pt}def\isanewline
\ \ \ \ \isacommand{using}\isamarkupfalse%
\ nfH\ Un{\isacharunderscore}{\kern0pt}closed\ singleton{\isacharunderscore}{\kern0pt}in{\isacharunderscore}{\kern0pt}M{\isacharunderscore}{\kern0pt}iff\ pair{\isacharunderscore}{\kern0pt}in{\isacharunderscore}{\kern0pt}M{\isacharunderscore}{\kern0pt}iff\ assms\ nat{\isacharunderscore}{\kern0pt}in{\isacharunderscore}{\kern0pt}M\ transM\isanewline
\ \ \ \ \isacommand{by}\isamarkupfalse%
\ auto\ \ \isanewline
\isacommand{qed}\isamarkupfalse%
%
\endisatagproof
{\isafoldproof}%
%
\isadelimproof
\isanewline
%
\endisadelimproof
\isanewline
\isacommand{lemma}\isamarkupfalse%
\ generic{\isacharunderscore}{\kern0pt}filter{\isacharunderscore}{\kern0pt}contains{\isacharunderscore}{\kern0pt}max\ {\isacharcolon}{\kern0pt}\ \isanewline
\ \ \isakeyword{fixes}\ G\ \isanewline
\ \ \isakeyword{assumes}\ {\isachardoublequoteopen}M{\isacharunderscore}{\kern0pt}generic{\isacharparenleft}{\kern0pt}G{\isacharparenright}{\kern0pt}{\isachardoublequoteclose}\ \isanewline
\ \ \isakeyword{shows}\ {\isachardoublequoteopen}{\isadigit{0}}\ {\isasymin}\ G{\isachardoublequoteclose}\ \isanewline
%
\isadelimproof
%
\endisadelimproof
%
\isatagproof
\isacommand{proof}\isamarkupfalse%
\ {\isacharminus}{\kern0pt}\ \isanewline
\ \ \isacommand{thm}\isamarkupfalse%
\ M{\isacharunderscore}{\kern0pt}generic{\isacharunderscore}{\kern0pt}def\isanewline
\ \ filter{\isacharunderscore}{\kern0pt}def\isanewline
\isanewline
\ \ \isacommand{have}\isamarkupfalse%
\ {\isachardoublequoteopen}dense{\isacharparenleft}{\kern0pt}Fn{\isacharparenright}{\kern0pt}{\isachardoublequoteclose}\ \isanewline
\ \ \ \ \isacommand{unfolding}\isamarkupfalse%
\ dense{\isacharunderscore}{\kern0pt}def\ Fn{\isacharunderscore}{\kern0pt}leq{\isacharunderscore}{\kern0pt}def\isanewline
\ \ \ \ \isacommand{apply}\isamarkupfalse%
{\isacharparenleft}{\kern0pt}subst\ forcing{\isacharunderscore}{\kern0pt}notion{\isachardot}{\kern0pt}dense{\isacharunderscore}{\kern0pt}def{\isacharparenright}{\kern0pt}\isanewline
\ \ \ \ \isacommand{using}\isamarkupfalse%
\ Fn{\isacharunderscore}{\kern0pt}leq{\isacharunderscore}{\kern0pt}def\ forcing{\isacharunderscore}{\kern0pt}notion{\isacharunderscore}{\kern0pt}axioms\isanewline
\ \ \ \ \ \isacommand{apply}\isamarkupfalse%
\ force\isanewline
\ \ \ \ \isacommand{by}\isamarkupfalse%
\ auto\isanewline
\isanewline
\ \ \isacommand{then}\isamarkupfalse%
\ \isacommand{have}\isamarkupfalse%
\ {\isachardoublequoteopen}G\ {\isasyminter}\ Fn\ {\isasymnoteq}\ {\isadigit{0}}{\isachardoublequoteclose}\ \isacommand{using}\isamarkupfalse%
\ M{\isacharunderscore}{\kern0pt}generic{\isacharunderscore}{\kern0pt}def\ assms\ Fn{\isacharunderscore}{\kern0pt}in{\isacharunderscore}{\kern0pt}M\ \isacommand{by}\isamarkupfalse%
\ blast\isanewline
\ \ \isacommand{then}\isamarkupfalse%
\ \isacommand{obtain}\isamarkupfalse%
\ p\ \isakeyword{where}\ pH{\isacharcolon}{\kern0pt}\ {\isachardoublequoteopen}p\ {\isasymin}\ Fn{\isachardoublequoteclose}\ {\isachardoublequoteopen}p\ {\isasymin}\ G{\isachardoublequoteclose}\ \isacommand{by}\isamarkupfalse%
\ auto\isanewline
\isanewline
\ \ \isacommand{show}\isamarkupfalse%
\ {\isachardoublequoteopen}{\isadigit{0}}\ {\isasymin}\ G{\isachardoublequoteclose}\ \isanewline
\ \ \ \ \isacommand{apply}\isamarkupfalse%
{\isacharparenleft}{\kern0pt}rule\ M{\isacharunderscore}{\kern0pt}generic{\isacharunderscore}{\kern0pt}leqD{\isacharparenright}{\kern0pt}\isanewline
\ \ \ \ \isacommand{using}\isamarkupfalse%
\ assms\ pH\ zero{\isacharunderscore}{\kern0pt}in{\isacharunderscore}{\kern0pt}Fn\isanewline
\ \ \ \ \ \ \ \isacommand{apply}\isamarkupfalse%
\ auto{\isacharbrackleft}{\kern0pt}{\isadigit{3}}{\isacharbrackright}{\kern0pt}\isanewline
\ \ \ \ \isacommand{unfolding}\isamarkupfalse%
\ Fn{\isacharunderscore}{\kern0pt}leq{\isacharunderscore}{\kern0pt}def\isanewline
\ \ \ \ \isacommand{using}\isamarkupfalse%
\ zero{\isacharunderscore}{\kern0pt}in{\isacharunderscore}{\kern0pt}Fn\ pH\isanewline
\ \ \ \ \isacommand{by}\isamarkupfalse%
\ auto\isanewline
\isacommand{qed}\isamarkupfalse%
%
\endisatagproof
{\isafoldproof}%
%
\isadelimproof
\isanewline
%
\endisadelimproof
\ \ \ \isanewline
\isacommand{lemma}\isamarkupfalse%
\ binmap{\isacharunderscore}{\kern0pt}row{\isacharunderscore}{\kern0pt}eq\ {\isacharcolon}{\kern0pt}\ \isanewline
\ \ \isakeyword{fixes}\ n\ G\ \ \isanewline
\ \ \isakeyword{assumes}\ {\isachardoublequoteopen}n\ {\isasymin}\ nat{\isachardoublequoteclose}\ {\isachardoublequoteopen}M{\isacharunderscore}{\kern0pt}generic{\isacharparenleft}{\kern0pt}G{\isacharparenright}{\kern0pt}{\isachardoublequoteclose}\ \isanewline
\ \ \isakeyword{shows}\ {\isachardoublequoteopen}binmap{\isacharunderscore}{\kern0pt}row{\isacharparenleft}{\kern0pt}G{\isacharcomma}{\kern0pt}\ n{\isacharparenright}{\kern0pt}\ {\isacharequal}{\kern0pt}\ val{\isacharparenleft}{\kern0pt}G{\isacharcomma}{\kern0pt}\ binmap{\isacharunderscore}{\kern0pt}row{\isacharprime}{\kern0pt}{\isacharparenleft}{\kern0pt}n{\isacharparenright}{\kern0pt}{\isacharparenright}{\kern0pt}{\isachardoublequoteclose}\ \isanewline
%
\isadelimproof
%
\endisadelimproof
%
\isatagproof
\isacommand{proof}\isamarkupfalse%
{\isacharparenleft}{\kern0pt}rule\ equality{\isacharunderscore}{\kern0pt}iffI{\isacharcomma}{\kern0pt}\ rule\ iffI{\isacharparenright}{\kern0pt}\isanewline
\ \ \isacommand{fix}\isamarkupfalse%
\ v\ \isanewline
\ \ \isacommand{assume}\isamarkupfalse%
\ vin{\isacharcolon}{\kern0pt}\ {\isachardoublequoteopen}v\ {\isasymin}\ val{\isacharparenleft}{\kern0pt}G{\isacharcomma}{\kern0pt}\ binmap{\isacharunderscore}{\kern0pt}row{\isacharprime}{\kern0pt}{\isacharparenleft}{\kern0pt}n{\isacharparenright}{\kern0pt}{\isacharparenright}{\kern0pt}{\isachardoublequoteclose}\ \isanewline
\ \ \isacommand{have}\isamarkupfalse%
\ {\isachardoublequoteopen}v\ {\isasymin}\ {\isacharbraceleft}{\kern0pt}\ val{\isacharparenleft}{\kern0pt}G{\isacharcomma}{\kern0pt}\ m{\isacharparenright}{\kern0pt}\ {\isachardot}{\kern0pt}{\isachardot}{\kern0pt}\ m\ {\isasymin}\ domain{\isacharparenleft}{\kern0pt}binmap{\isacharunderscore}{\kern0pt}row{\isacharprime}{\kern0pt}{\isacharparenleft}{\kern0pt}n{\isacharparenright}{\kern0pt}{\isacharparenright}{\kern0pt}{\isacharcomma}{\kern0pt}\ {\isasymexists}p\ {\isasymin}\ Fn{\isachardot}{\kern0pt}\ {\isasymlangle}m{\isacharcomma}{\kern0pt}\ p{\isasymrangle}\ {\isasymin}\ binmap{\isacharunderscore}{\kern0pt}row{\isacharprime}{\kern0pt}{\isacharparenleft}{\kern0pt}n{\isacharparenright}{\kern0pt}\ {\isasymand}\ p\ {\isasymin}\ G\ {\isacharbraceright}{\kern0pt}{\isachardoublequoteclose}\isanewline
\ \ \ \ \isacommand{apply}\isamarkupfalse%
{\isacharparenleft}{\kern0pt}rule{\isacharunderscore}{\kern0pt}tac\ P{\isacharequal}{\kern0pt}{\isachardoublequoteopen}v\ {\isasymin}\ val{\isacharparenleft}{\kern0pt}G{\isacharcomma}{\kern0pt}\ binmap{\isacharunderscore}{\kern0pt}row{\isacharprime}{\kern0pt}{\isacharparenleft}{\kern0pt}n{\isacharparenright}{\kern0pt}{\isacharparenright}{\kern0pt}{\isachardoublequoteclose}\ \isakeyword{in}\ mp{\isacharparenright}{\kern0pt}\isanewline
\ \ \ \ \isacommand{apply}\isamarkupfalse%
{\isacharparenleft}{\kern0pt}subst\ def{\isacharunderscore}{\kern0pt}val{\isacharparenright}{\kern0pt}\isanewline
\ \ \ \ \isacommand{using}\isamarkupfalse%
\ vin\ \isanewline
\ \ \ \ \isacommand{by}\isamarkupfalse%
\ auto\isanewline
\ \ \isacommand{then}\isamarkupfalse%
\ \isacommand{obtain}\isamarkupfalse%
\ m\ p\ \isakeyword{where}\ mpH{\isacharcolon}{\kern0pt}\ {\isachardoublequoteopen}p\ {\isasymin}\ Fn{\isachardoublequoteclose}\ {\isachardoublequoteopen}m\ {\isasymin}\ nat{\isachardoublequoteclose}\ {\isachardoublequoteopen}{\isasymlangle}check{\isacharparenleft}{\kern0pt}m{\isacharparenright}{\kern0pt}{\isacharcomma}{\kern0pt}\ p{\isasymrangle}\ {\isasymin}\ binmap{\isacharunderscore}{\kern0pt}row{\isacharprime}{\kern0pt}{\isacharparenleft}{\kern0pt}n{\isacharparenright}{\kern0pt}{\isachardoublequoteclose}\ {\isachardoublequoteopen}p\ {\isasymin}\ G{\isachardoublequoteclose}\ {\isachardoublequoteopen}v\ {\isacharequal}{\kern0pt}\ val{\isacharparenleft}{\kern0pt}G{\isacharcomma}{\kern0pt}\ check{\isacharparenleft}{\kern0pt}m{\isacharparenright}{\kern0pt}{\isacharparenright}{\kern0pt}{\isachardoublequoteclose}\ {\isachardoublequoteopen}p{\isacharbackquote}{\kern0pt}{\isacharless}{\kern0pt}n{\isacharcomma}{\kern0pt}\ m{\isachargreater}{\kern0pt}\ {\isacharequal}{\kern0pt}\ {\isadigit{1}}{\isachardoublequoteclose}\ \isanewline
\ \ \ \ \isacommand{unfolding}\isamarkupfalse%
\ binmap{\isacharunderscore}{\kern0pt}row{\isacharprime}{\kern0pt}{\isacharunderscore}{\kern0pt}def\isanewline
\ \ \ \ \isacommand{by}\isamarkupfalse%
\ force\isanewline
\ \ \isanewline
\ \ \isacommand{have}\isamarkupfalse%
\ {\isachardoublequoteopen}val{\isacharparenleft}{\kern0pt}G{\isacharcomma}{\kern0pt}\ check{\isacharparenleft}{\kern0pt}m{\isacharparenright}{\kern0pt}{\isacharparenright}{\kern0pt}\ {\isacharequal}{\kern0pt}\ m{\isachardoublequoteclose}\isanewline
\ \ \ \ \isacommand{apply}\isamarkupfalse%
{\isacharparenleft}{\kern0pt}rule\ valcheck{\isacharparenright}{\kern0pt}\isanewline
\ \ \ \ \isacommand{using}\isamarkupfalse%
\ generic{\isacharunderscore}{\kern0pt}filter{\isacharunderscore}{\kern0pt}contains{\isacharunderscore}{\kern0pt}max\ zero{\isacharunderscore}{\kern0pt}in{\isacharunderscore}{\kern0pt}Fn\ assms\isanewline
\ \ \ \ \isacommand{by}\isamarkupfalse%
\ auto\isanewline
\ \ \isacommand{then}\isamarkupfalse%
\ \isacommand{have}\isamarkupfalse%
\ {\isachardoublequoteopen}v\ {\isacharequal}{\kern0pt}\ m{\isachardoublequoteclose}\ \isacommand{using}\isamarkupfalse%
\ mpH\ \isacommand{by}\isamarkupfalse%
\ auto\ \isanewline
\ \ \isacommand{then}\isamarkupfalse%
\ \isacommand{show}\isamarkupfalse%
\ {\isachardoublequoteopen}v\ {\isasymin}\ binmap{\isacharunderscore}{\kern0pt}row{\isacharparenleft}{\kern0pt}G{\isacharcomma}{\kern0pt}\ n{\isacharparenright}{\kern0pt}{\isachardoublequoteclose}\ \isanewline
\ \ \ \ \isacommand{unfolding}\isamarkupfalse%
\ binmap{\isacharunderscore}{\kern0pt}row{\isacharunderscore}{\kern0pt}def\ \isanewline
\ \ \ \ \isacommand{using}\isamarkupfalse%
\ mpH\ \isanewline
\ \ \ \ \isacommand{by}\isamarkupfalse%
\ auto\isanewline
\isacommand{next}\isamarkupfalse%
\ \isanewline
\ \ \isacommand{fix}\isamarkupfalse%
\ v\isanewline
\ \ \isacommand{assume}\isamarkupfalse%
\ {\isachardoublequoteopen}v\ {\isasymin}\ binmap{\isacharunderscore}{\kern0pt}row{\isacharparenleft}{\kern0pt}G{\isacharcomma}{\kern0pt}\ n{\isacharparenright}{\kern0pt}{\isachardoublequoteclose}\ \isanewline
\ \ \isacommand{then}\isamarkupfalse%
\ \isacommand{obtain}\isamarkupfalse%
\ m\ p\ \isakeyword{where}\ mpH{\isacharcolon}{\kern0pt}\ {\isachardoublequoteopen}m\ {\isasymin}\ nat{\isachardoublequoteclose}\ {\isachardoublequoteopen}p\ {\isasymin}\ G{\isachardoublequoteclose}\ {\isachardoublequoteopen}p{\isacharbackquote}{\kern0pt}{\isacharless}{\kern0pt}n{\isacharcomma}{\kern0pt}\ m{\isachargreater}{\kern0pt}\ {\isacharequal}{\kern0pt}\ {\isadigit{1}}{\isachardoublequoteclose}\ {\isachardoublequoteopen}v\ {\isacharequal}{\kern0pt}\ m{\isachardoublequoteclose}\ \isacommand{using}\isamarkupfalse%
\ binmap{\isacharunderscore}{\kern0pt}row{\isacharunderscore}{\kern0pt}def\ \isacommand{by}\isamarkupfalse%
\ auto\isanewline
\ \ \isacommand{then}\isamarkupfalse%
\ \isacommand{have}\isamarkupfalse%
\ {\isachardoublequoteopen}{\isacharless}{\kern0pt}check{\isacharparenleft}{\kern0pt}m{\isacharparenright}{\kern0pt}{\isacharcomma}{\kern0pt}\ p{\isachargreater}{\kern0pt}\ {\isasymin}\ binmap{\isacharunderscore}{\kern0pt}row{\isacharprime}{\kern0pt}{\isacharparenleft}{\kern0pt}n{\isacharparenright}{\kern0pt}{\isachardoublequoteclose}\ \isanewline
\ \ \ \ \isacommand{unfolding}\isamarkupfalse%
\ binmap{\isacharunderscore}{\kern0pt}row{\isacharprime}{\kern0pt}{\isacharunderscore}{\kern0pt}def\ \isanewline
\ \ \ \ \isacommand{apply}\isamarkupfalse%
\ auto\isanewline
\ \ \ \ \ \ \isacommand{apply}\isamarkupfalse%
{\isacharparenleft}{\kern0pt}subst\ {\isacharparenleft}{\kern0pt}{\isadigit{2}}{\isacharparenright}{\kern0pt}\ def{\isacharunderscore}{\kern0pt}check{\isacharparenright}{\kern0pt}\isanewline
\ \ \ \ \isacommand{using}\isamarkupfalse%
\ mpH\ M{\isacharunderscore}{\kern0pt}genericD\ assms\isanewline
\ \ \ \ \isacommand{by}\isamarkupfalse%
\ auto\isanewline
\ \ \isacommand{then}\isamarkupfalse%
\ \isacommand{have}\isamarkupfalse%
\ {\isachardoublequoteopen}val{\isacharparenleft}{\kern0pt}G{\isacharcomma}{\kern0pt}\ check{\isacharparenleft}{\kern0pt}m{\isacharparenright}{\kern0pt}{\isacharparenright}{\kern0pt}\ {\isasymin}\ {\isacharbraceleft}{\kern0pt}\ val{\isacharparenleft}{\kern0pt}G{\isacharcomma}{\kern0pt}\ m{\isacharparenright}{\kern0pt}\ {\isachardot}{\kern0pt}{\isachardot}{\kern0pt}\ m\ {\isasymin}\ domain{\isacharparenleft}{\kern0pt}binmap{\isacharunderscore}{\kern0pt}row{\isacharprime}{\kern0pt}{\isacharparenleft}{\kern0pt}n{\isacharparenright}{\kern0pt}{\isacharparenright}{\kern0pt}{\isacharcomma}{\kern0pt}\ {\isasymexists}p\ {\isasymin}\ Fn{\isachardot}{\kern0pt}\ {\isasymlangle}m{\isacharcomma}{\kern0pt}\ p{\isasymrangle}\ {\isasymin}\ binmap{\isacharunderscore}{\kern0pt}row{\isacharprime}{\kern0pt}{\isacharparenleft}{\kern0pt}n{\isacharparenright}{\kern0pt}\ {\isasymand}\ p\ {\isasymin}\ G\ {\isacharbraceright}{\kern0pt}{\isachardoublequoteclose}\isanewline
\ \ \ \ \isacommand{apply}\isamarkupfalse%
\ simp\isanewline
\ \ \ \ \isacommand{apply}\isamarkupfalse%
{\isacharparenleft}{\kern0pt}rule{\isacharunderscore}{\kern0pt}tac\ x{\isacharequal}{\kern0pt}{\isachardoublequoteopen}check{\isacharparenleft}{\kern0pt}m{\isacharparenright}{\kern0pt}{\isachardoublequoteclose}\ \isakeyword{in}\ bexI{\isacharparenright}{\kern0pt}\isanewline
\ \ \ \ \isacommand{using}\isamarkupfalse%
\ mpH\ M{\isacharunderscore}{\kern0pt}genericD\ assms\isanewline
\ \ \ \ \isacommand{by}\isamarkupfalse%
\ auto\isanewline
\ \ \isacommand{then}\isamarkupfalse%
\ \isacommand{have}\isamarkupfalse%
\ {\isachardoublequoteopen}val{\isacharparenleft}{\kern0pt}G{\isacharcomma}{\kern0pt}\ check{\isacharparenleft}{\kern0pt}m{\isacharparenright}{\kern0pt}{\isacharparenright}{\kern0pt}\ {\isasymin}\ val{\isacharparenleft}{\kern0pt}G{\isacharcomma}{\kern0pt}\ binmap{\isacharunderscore}{\kern0pt}row{\isacharprime}{\kern0pt}{\isacharparenleft}{\kern0pt}n{\isacharparenright}{\kern0pt}{\isacharparenright}{\kern0pt}{\isachardoublequoteclose}\ \isanewline
\ \ \ \ \isacommand{by}\isamarkupfalse%
{\isacharparenleft}{\kern0pt}subst\ {\isacharparenleft}{\kern0pt}{\isadigit{2}}{\isacharparenright}{\kern0pt}\ def{\isacharunderscore}{\kern0pt}val{\isacharcomma}{\kern0pt}\ force{\isacharparenright}{\kern0pt}\isanewline
\ \ \isacommand{then}\isamarkupfalse%
\ \isacommand{show}\isamarkupfalse%
\ {\isachardoublequoteopen}v\ {\isasymin}\ val{\isacharparenleft}{\kern0pt}G{\isacharcomma}{\kern0pt}\ binmap{\isacharunderscore}{\kern0pt}row{\isacharprime}{\kern0pt}{\isacharparenleft}{\kern0pt}n{\isacharparenright}{\kern0pt}{\isacharparenright}{\kern0pt}{\isachardoublequoteclose}\ \isanewline
\ \ \ \ \isacommand{apply}\isamarkupfalse%
{\isacharparenleft}{\kern0pt}subgoal{\isacharunderscore}{\kern0pt}tac\ {\isachardoublequoteopen}m\ {\isacharequal}{\kern0pt}\ val{\isacharparenleft}{\kern0pt}G{\isacharcomma}{\kern0pt}\ check{\isacharparenleft}{\kern0pt}m{\isacharparenright}{\kern0pt}{\isacharparenright}{\kern0pt}{\isachardoublequoteclose}{\isacharparenright}{\kern0pt}\isanewline
\ \ \ \ \isacommand{using}\isamarkupfalse%
\ mpH\ \isanewline
\ \ \ \ \ \isacommand{apply}\isamarkupfalse%
\ force\isanewline
\ \ \ \ \isacommand{apply}\isamarkupfalse%
{\isacharparenleft}{\kern0pt}rule\ sym{\isacharparenright}{\kern0pt}\isanewline
\ \ \ \ \isacommand{apply}\isamarkupfalse%
{\isacharparenleft}{\kern0pt}rule\ valcheck{\isacharparenright}{\kern0pt}\isanewline
\ \ \ \ \isacommand{using}\isamarkupfalse%
\ generic{\isacharunderscore}{\kern0pt}filter{\isacharunderscore}{\kern0pt}contains{\isacharunderscore}{\kern0pt}max\ zero{\isacharunderscore}{\kern0pt}in{\isacharunderscore}{\kern0pt}Fn\ assms\isanewline
\ \ \ \ \isacommand{by}\isamarkupfalse%
\ auto\isanewline
\isacommand{qed}\isamarkupfalse%
%
\endisatagproof
{\isafoldproof}%
%
\isadelimproof
\isanewline
%
\endisadelimproof
\isanewline
\isacommand{lemma}\isamarkupfalse%
\ binmap{\isacharunderscore}{\kern0pt}eq\ {\isacharcolon}{\kern0pt}\ \isanewline
\ \ \isakeyword{fixes}\ G\ \isanewline
\ \ \isakeyword{assumes}\ {\isachardoublequoteopen}M{\isacharunderscore}{\kern0pt}generic{\isacharparenleft}{\kern0pt}G{\isacharparenright}{\kern0pt}{\isachardoublequoteclose}\ \isanewline
\ \ \isakeyword{shows}\ {\isachardoublequoteopen}binmap{\isacharparenleft}{\kern0pt}G{\isacharparenright}{\kern0pt}\ {\isacharequal}{\kern0pt}\ val{\isacharparenleft}{\kern0pt}G{\isacharcomma}{\kern0pt}\ binmap{\isacharprime}{\kern0pt}{\isacharparenright}{\kern0pt}{\isachardoublequoteclose}\ \isanewline
%
\isadelimproof
\isanewline
\ \ %
\endisadelimproof
%
\isatagproof
\isacommand{apply}\isamarkupfalse%
{\isacharparenleft}{\kern0pt}subst\ def{\isacharunderscore}{\kern0pt}val{\isacharparenright}{\kern0pt}\isanewline
\ \ \isacommand{unfolding}\isamarkupfalse%
\ binmap{\isacharunderscore}{\kern0pt}def\isanewline
\ \ \isacommand{apply}\isamarkupfalse%
{\isacharparenleft}{\kern0pt}rule\ equality{\isacharunderscore}{\kern0pt}iffI{\isacharcomma}{\kern0pt}\ rule\ iffI{\isacharparenright}{\kern0pt}\isanewline
\ \ \ \isacommand{apply}\isamarkupfalse%
\ clarsimp\isanewline
\ \ \ \isacommand{apply}\isamarkupfalse%
{\isacharparenleft}{\kern0pt}rename{\isacharunderscore}{\kern0pt}tac\ n{\isacharcomma}{\kern0pt}\ rule{\isacharunderscore}{\kern0pt}tac\ x{\isacharequal}{\kern0pt}{\isachardoublequoteopen}binmap{\isacharunderscore}{\kern0pt}row{\isacharprime}{\kern0pt}{\isacharparenleft}{\kern0pt}n{\isacharparenright}{\kern0pt}{\isachardoublequoteclose}\ \isakeyword{in}\ bexI{\isacharparenright}{\kern0pt}\isanewline
\ \ \ \ \isacommand{apply}\isamarkupfalse%
{\isacharparenleft}{\kern0pt}rule\ conjI{\isacharparenright}{\kern0pt}\isanewline
\ \ \isacommand{using}\isamarkupfalse%
\ binmap{\isacharunderscore}{\kern0pt}row{\isacharunderscore}{\kern0pt}eq\ assms\isanewline
\ \ \ \ \ \isacommand{apply}\isamarkupfalse%
\ force\isanewline
\ \ \ \ \isacommand{apply}\isamarkupfalse%
{\isacharparenleft}{\kern0pt}rule{\isacharunderscore}{\kern0pt}tac\ x{\isacharequal}{\kern0pt}{\isadigit{0}}\ \isakeyword{in}\ bexI{\isacharparenright}{\kern0pt}\isanewline
\ \ \isacommand{using}\isamarkupfalse%
\ binmap{\isacharprime}{\kern0pt}{\isacharunderscore}{\kern0pt}def\ assms\ generic{\isacharunderscore}{\kern0pt}filter{\isacharunderscore}{\kern0pt}contains{\isacharunderscore}{\kern0pt}max\ zero{\isacharunderscore}{\kern0pt}in{\isacharunderscore}{\kern0pt}Fn\isanewline
\ \ \ \ \ \isacommand{apply}\isamarkupfalse%
\ auto{\isacharbrackleft}{\kern0pt}{\isadigit{3}}{\isacharbrackright}{\kern0pt}\isanewline
\ \ \isacommand{using}\isamarkupfalse%
\ binmap{\isacharprime}{\kern0pt}{\isacharunderscore}{\kern0pt}def\isanewline
\ \ \isacommand{apply}\isamarkupfalse%
\ clarsimp\isanewline
\ \ \isacommand{apply}\isamarkupfalse%
{\isacharparenleft}{\kern0pt}rename{\isacharunderscore}{\kern0pt}tac\ n\ m{\isacharcomma}{\kern0pt}\ rule{\isacharunderscore}{\kern0pt}tac\ x{\isacharequal}{\kern0pt}m\ \isakeyword{in}\ bexI{\isacharparenright}{\kern0pt}\isanewline
\ \ \isacommand{using}\isamarkupfalse%
\ binmap{\isacharunderscore}{\kern0pt}row{\isacharunderscore}{\kern0pt}eq\ assms\isanewline
\ \ \isacommand{by}\isamarkupfalse%
\ auto%
\endisatagproof
{\isafoldproof}%
%
\isadelimproof
\isanewline
%
\endisadelimproof
\isanewline
\isacommand{lemma}\isamarkupfalse%
\ Fn{\isacharunderscore}{\kern0pt}leq{\isacharunderscore}{\kern0pt}preserves{\isacharunderscore}{\kern0pt}value\ {\isacharcolon}{\kern0pt}\ \isanewline
\ \ \isakeyword{fixes}\ p\ q\ \isanewline
\ \ \isakeyword{assumes}\ {\isachardoublequoteopen}{\isacharless}{\kern0pt}q{\isacharcomma}{\kern0pt}\ p{\isachargreater}{\kern0pt}\ {\isasymin}\ Fn{\isacharunderscore}{\kern0pt}leq{\isachardoublequoteclose}\ {\isachardoublequoteopen}{\isacharless}{\kern0pt}m{\isacharcomma}{\kern0pt}\ n{\isachargreater}{\kern0pt}\ {\isasymin}\ domain{\isacharparenleft}{\kern0pt}p{\isacharparenright}{\kern0pt}{\isachardoublequoteclose}\ \isanewline
\ \ \isakeyword{shows}\ {\isachardoublequoteopen}q{\isacharbackquote}{\kern0pt}{\isacharless}{\kern0pt}m{\isacharcomma}{\kern0pt}\ n{\isachargreater}{\kern0pt}\ {\isacharequal}{\kern0pt}\ p{\isacharbackquote}{\kern0pt}{\isacharless}{\kern0pt}m{\isacharcomma}{\kern0pt}\ n{\isachargreater}{\kern0pt}{\isachardoublequoteclose}\ \isanewline
%
\isadelimproof
%
\endisadelimproof
%
\isatagproof
\isacommand{proof}\isamarkupfalse%
\ {\isacharminus}{\kern0pt}\ \isanewline
\ \ \isacommand{obtain}\isamarkupfalse%
\ v\ \isakeyword{where}\ vH{\isacharcolon}{\kern0pt}\ {\isachardoublequoteopen}{\isacharless}{\kern0pt}{\isacharless}{\kern0pt}m{\isacharcomma}{\kern0pt}\ n{\isachargreater}{\kern0pt}{\isacharcomma}{\kern0pt}\ v{\isachargreater}{\kern0pt}\ {\isasymin}\ p{\isachardoublequoteclose}\ \isacommand{using}\isamarkupfalse%
\ assms\ \isacommand{by}\isamarkupfalse%
\ auto\ \isanewline
\ \ \isacommand{then}\isamarkupfalse%
\ \isacommand{have}\isamarkupfalse%
\ vH{\isacharprime}{\kern0pt}{\isacharcolon}{\kern0pt}\ {\isachardoublequoteopen}{\isacharless}{\kern0pt}{\isacharless}{\kern0pt}m{\isacharcomma}{\kern0pt}\ n{\isachargreater}{\kern0pt}{\isacharcomma}{\kern0pt}\ v{\isachargreater}{\kern0pt}\ {\isasymin}\ q{\isachardoublequoteclose}\ \isacommand{using}\isamarkupfalse%
\ assms\ Fn{\isacharunderscore}{\kern0pt}leq{\isacharunderscore}{\kern0pt}def\ \isacommand{by}\isamarkupfalse%
\ auto\isanewline
\isanewline
\ \ \isacommand{have}\isamarkupfalse%
\ papp\ {\isacharcolon}{\kern0pt}\ {\isachardoublequoteopen}p{\isacharbackquote}{\kern0pt}{\isacharless}{\kern0pt}m{\isacharcomma}{\kern0pt}\ n{\isachargreater}{\kern0pt}\ {\isacharequal}{\kern0pt}\ v{\isachardoublequoteclose}\ \isanewline
\ \ \ \ \isacommand{apply}\isamarkupfalse%
{\isacharparenleft}{\kern0pt}rule\ function{\isacharunderscore}{\kern0pt}apply{\isacharunderscore}{\kern0pt}equality{\isacharparenright}{\kern0pt}\isanewline
\ \ \ \ \isacommand{using}\isamarkupfalse%
\ vH\ assms\ Fn{\isacharunderscore}{\kern0pt}leq{\isacharunderscore}{\kern0pt}def\ Fn{\isacharunderscore}{\kern0pt}def\ \isanewline
\ \ \ \ \isacommand{by}\isamarkupfalse%
\ auto\isanewline
\isanewline
\ \ \isacommand{have}\isamarkupfalse%
\ qapp\ {\isacharcolon}{\kern0pt}\ {\isachardoublequoteopen}q{\isacharbackquote}{\kern0pt}{\isacharless}{\kern0pt}m{\isacharcomma}{\kern0pt}\ n{\isachargreater}{\kern0pt}\ {\isacharequal}{\kern0pt}\ v{\isachardoublequoteclose}\ \isanewline
\ \ \ \ \isacommand{apply}\isamarkupfalse%
{\isacharparenleft}{\kern0pt}rule\ function{\isacharunderscore}{\kern0pt}apply{\isacharunderscore}{\kern0pt}equality{\isacharparenright}{\kern0pt}\isanewline
\ \ \ \ \isacommand{using}\isamarkupfalse%
\ vH\ assms\ Fn{\isacharunderscore}{\kern0pt}leq{\isacharunderscore}{\kern0pt}def\ Fn{\isacharunderscore}{\kern0pt}def\ \isanewline
\ \ \ \ \isacommand{by}\isamarkupfalse%
\ auto\isanewline
\isanewline
\ \ \isacommand{show}\isamarkupfalse%
\ {\isachardoublequoteopen}q{\isacharbackquote}{\kern0pt}{\isacharless}{\kern0pt}m{\isacharcomma}{\kern0pt}\ n{\isachargreater}{\kern0pt}\ {\isacharequal}{\kern0pt}\ p{\isacharbackquote}{\kern0pt}{\isacharless}{\kern0pt}m{\isacharcomma}{\kern0pt}\ n{\isachargreater}{\kern0pt}{\isachardoublequoteclose}\ \isacommand{using}\isamarkupfalse%
\ papp\ qapp\ \isacommand{by}\isamarkupfalse%
\ auto\ \isanewline
\isacommand{qed}\isamarkupfalse%
%
\endisatagproof
{\isafoldproof}%
%
\isadelimproof
\isanewline
%
\endisadelimproof
\isanewline
\isacommand{lemma}\isamarkupfalse%
\ Fn{\isacharunderscore}{\kern0pt}{\isadigit{1}}{\isacharunderscore}{\kern0pt}forces\ {\isacharcolon}{\kern0pt}\ \isanewline
\ \ \isakeyword{fixes}\ p\ n\ m\ \isanewline
\ \ \isakeyword{assumes}\ {\isachardoublequoteopen}p\ {\isasymin}\ Fn{\isachardoublequoteclose}\ {\isachardoublequoteopen}n\ {\isasymin}\ nat{\isachardoublequoteclose}\ {\isachardoublequoteopen}m\ {\isasymin}\ nat{\isachardoublequoteclose}\ {\isachardoublequoteopen}p{\isacharbackquote}{\kern0pt}{\isacharless}{\kern0pt}n{\isacharcomma}{\kern0pt}\ m{\isachargreater}{\kern0pt}\ {\isacharequal}{\kern0pt}\ {\isadigit{1}}{\isachardoublequoteclose}\ \isanewline
\ \ \isakeyword{shows}\ {\isachardoublequoteopen}{\isasymforall}G{\isachardot}{\kern0pt}\ M{\isacharunderscore}{\kern0pt}generic{\isacharparenleft}{\kern0pt}G{\isacharparenright}{\kern0pt}\ {\isasymand}\ p\ {\isasymin}\ G\ {\isasymlongrightarrow}\ SymExt{\isacharparenleft}{\kern0pt}G{\isacharparenright}{\kern0pt}{\isacharcomma}{\kern0pt}\ map{\isacharparenleft}{\kern0pt}val{\isacharparenleft}{\kern0pt}G{\isacharparenright}{\kern0pt}{\isacharcomma}{\kern0pt}\ {\isacharbrackleft}{\kern0pt}check{\isacharparenleft}{\kern0pt}m{\isacharparenright}{\kern0pt}{\isacharcomma}{\kern0pt}\ binmap{\isacharunderscore}{\kern0pt}row{\isacharprime}{\kern0pt}{\isacharparenleft}{\kern0pt}n{\isacharparenright}{\kern0pt}{\isacharbrackright}{\kern0pt}{\isacharparenright}{\kern0pt}\ {\isasymTurnstile}\ Member{\isacharparenleft}{\kern0pt}{\isadigit{0}}{\isacharcomma}{\kern0pt}\ {\isadigit{1}}{\isacharparenright}{\kern0pt}{\isachardoublequoteclose}\ \isanewline
%
\isadelimproof
\ \ %
\endisadelimproof
%
\isatagproof
\isacommand{apply}\isamarkupfalse%
{\isacharparenleft}{\kern0pt}clarsimp{\isacharparenright}{\kern0pt}\isanewline
\ \ \isacommand{apply}\isamarkupfalse%
{\isacharparenleft}{\kern0pt}rule\ iffD{\isadigit{2}}{\isacharcomma}{\kern0pt}\ rule\ sats{\isacharunderscore}{\kern0pt}Member{\isacharunderscore}{\kern0pt}iff{\isacharparenright}{\kern0pt}\isanewline
\ \ \ \isacommand{apply}\isamarkupfalse%
\ clarsimp\isanewline
\ \ \ \isacommand{apply}\isamarkupfalse%
{\isacharparenleft}{\kern0pt}rule\ conjI{\isacharcomma}{\kern0pt}\ subst\ SymExt{\isacharunderscore}{\kern0pt}def{\isacharparenright}{\kern0pt}\isanewline
\ \ \isacommand{using}\isamarkupfalse%
\ check{\isacharunderscore}{\kern0pt}in{\isacharunderscore}{\kern0pt}HS\ assms\ transM\ nat{\isacharunderscore}{\kern0pt}in{\isacharunderscore}{\kern0pt}M\ \isanewline
\ \ \ \ \isacommand{apply}\isamarkupfalse%
\ force\isanewline
\ \ \isacommand{using}\isamarkupfalse%
\ binmap{\isacharunderscore}{\kern0pt}row{\isacharprime}{\kern0pt}{\isacharunderscore}{\kern0pt}HS\ assms\ SymExt{\isacharunderscore}{\kern0pt}def\isanewline
\ \ \ \isacommand{apply}\isamarkupfalse%
\ force\isanewline
\ \ \isacommand{apply}\isamarkupfalse%
\ simp\isanewline
\ \ \isacommand{apply}\isamarkupfalse%
{\isacharparenleft}{\kern0pt}rename{\isacharunderscore}{\kern0pt}tac\ G{\isacharcomma}{\kern0pt}\ subgoal{\isacharunderscore}{\kern0pt}tac\ {\isachardoublequoteopen}val{\isacharparenleft}{\kern0pt}G{\isacharcomma}{\kern0pt}\ check{\isacharparenleft}{\kern0pt}m{\isacharparenright}{\kern0pt}{\isacharparenright}{\kern0pt}\ {\isacharequal}{\kern0pt}\ m\ {\isasymand}\ val{\isacharparenleft}{\kern0pt}G{\isacharcomma}{\kern0pt}\ binmap{\isacharunderscore}{\kern0pt}row{\isacharprime}{\kern0pt}{\isacharparenleft}{\kern0pt}n{\isacharparenright}{\kern0pt}{\isacharparenright}{\kern0pt}\ {\isacharequal}{\kern0pt}\ binmap{\isacharunderscore}{\kern0pt}row{\isacharparenleft}{\kern0pt}G{\isacharcomma}{\kern0pt}\ n{\isacharparenright}{\kern0pt}{\isachardoublequoteclose}{\isacharparenright}{\kern0pt}\isanewline
\ \ \ \isacommand{apply}\isamarkupfalse%
\ {\isacharparenleft}{\kern0pt}simp\ add{\isacharcolon}{\kern0pt}binmap{\isacharunderscore}{\kern0pt}row{\isacharunderscore}{\kern0pt}def{\isacharparenright}{\kern0pt}\isanewline
\ \ \isacommand{using}\isamarkupfalse%
\ assms\isanewline
\ \ \ \isacommand{apply}\isamarkupfalse%
\ force\isanewline
\ \ \isacommand{apply}\isamarkupfalse%
{\isacharparenleft}{\kern0pt}rule\ conjI{\isacharparenright}{\kern0pt}\isanewline
\ \ \ \isacommand{apply}\isamarkupfalse%
{\isacharparenleft}{\kern0pt}rule\ valcheck{\isacharparenright}{\kern0pt}\isanewline
\ \ \isacommand{using}\isamarkupfalse%
\ generic{\isacharunderscore}{\kern0pt}filter{\isacharunderscore}{\kern0pt}contains{\isacharunderscore}{\kern0pt}max\ zero{\isacharunderscore}{\kern0pt}in{\isacharunderscore}{\kern0pt}Fn\ assms\ binmap{\isacharunderscore}{\kern0pt}row{\isacharunderscore}{\kern0pt}eq\isanewline
\ \ \isacommand{by}\isamarkupfalse%
\ auto%
\endisatagproof
{\isafoldproof}%
%
\isadelimproof
\isanewline
%
\endisadelimproof
\isanewline
\isacommand{lemma}\isamarkupfalse%
\ Fn{\isacharunderscore}{\kern0pt}{\isadigit{0}}{\isacharunderscore}{\kern0pt}forces\ {\isacharcolon}{\kern0pt}\ \isanewline
\ \ \isakeyword{fixes}\ p\ n\ m\ \isanewline
\ \ \isakeyword{assumes}\ {\isachardoublequoteopen}p\ {\isasymin}\ Fn{\isachardoublequoteclose}\ {\isachardoublequoteopen}n\ {\isasymin}\ nat{\isachardoublequoteclose}\ {\isachardoublequoteopen}m\ {\isasymin}\ nat{\isachardoublequoteclose}\ {\isachardoublequoteopen}p{\isacharbackquote}{\kern0pt}{\isacharless}{\kern0pt}n{\isacharcomma}{\kern0pt}\ m{\isachargreater}{\kern0pt}\ {\isacharequal}{\kern0pt}\ {\isadigit{0}}{\isachardoublequoteclose}\ {\isachardoublequoteopen}{\isacharless}{\kern0pt}n\ {\isacharcomma}{\kern0pt}m{\isachargreater}{\kern0pt}\ {\isasymin}\ domain{\isacharparenleft}{\kern0pt}p{\isacharparenright}{\kern0pt}{\isachardoublequoteclose}\ \ \isanewline
\ \ \isakeyword{shows}\ {\isachardoublequoteopen}{\isasymforall}G{\isachardot}{\kern0pt}\ M{\isacharunderscore}{\kern0pt}generic{\isacharparenleft}{\kern0pt}G{\isacharparenright}{\kern0pt}\ {\isasymand}\ p\ {\isasymin}\ G\ {\isasymlongrightarrow}\ SymExt{\isacharparenleft}{\kern0pt}G{\isacharparenright}{\kern0pt}{\isacharcomma}{\kern0pt}\ map{\isacharparenleft}{\kern0pt}val{\isacharparenleft}{\kern0pt}G{\isacharparenright}{\kern0pt}{\isacharcomma}{\kern0pt}\ {\isacharbrackleft}{\kern0pt}check{\isacharparenleft}{\kern0pt}m{\isacharparenright}{\kern0pt}{\isacharcomma}{\kern0pt}\ binmap{\isacharunderscore}{\kern0pt}row{\isacharprime}{\kern0pt}{\isacharparenleft}{\kern0pt}n{\isacharparenright}{\kern0pt}{\isacharbrackright}{\kern0pt}{\isacharparenright}{\kern0pt}\ {\isasymTurnstile}\ Neg{\isacharparenleft}{\kern0pt}Member{\isacharparenleft}{\kern0pt}{\isadigit{0}}{\isacharcomma}{\kern0pt}\ {\isadigit{1}}{\isacharparenright}{\kern0pt}{\isacharparenright}{\kern0pt}{\isachardoublequoteclose}\ \isanewline
%
\isadelimproof
%
\endisadelimproof
%
\isatagproof
\isacommand{proof}\isamarkupfalse%
{\isacharparenleft}{\kern0pt}rule\ ccontr{\isacharparenright}{\kern0pt}\isanewline
\ \ \isacommand{assume}\isamarkupfalse%
\ {\isachardoublequoteopen}{\isasymnot}\ {\isacharparenleft}{\kern0pt}{\isasymforall}G{\isachardot}{\kern0pt}\ M{\isacharunderscore}{\kern0pt}generic{\isacharparenleft}{\kern0pt}G{\isacharparenright}{\kern0pt}\ {\isasymand}\ p\ {\isasymin}\ G\ {\isasymlongrightarrow}\ SymExt{\isacharparenleft}{\kern0pt}G{\isacharparenright}{\kern0pt}{\isacharcomma}{\kern0pt}\ map{\isacharparenleft}{\kern0pt}val{\isacharparenleft}{\kern0pt}G{\isacharparenright}{\kern0pt}{\isacharcomma}{\kern0pt}\ {\isacharbrackleft}{\kern0pt}check{\isacharparenleft}{\kern0pt}m{\isacharparenright}{\kern0pt}{\isacharcomma}{\kern0pt}\ binmap{\isacharunderscore}{\kern0pt}row{\isacharprime}{\kern0pt}{\isacharparenleft}{\kern0pt}n{\isacharparenright}{\kern0pt}{\isacharbrackright}{\kern0pt}{\isacharparenright}{\kern0pt}\ {\isasymTurnstile}\ Neg{\isacharparenleft}{\kern0pt}Member{\isacharparenleft}{\kern0pt}{\isadigit{0}}{\isacharcomma}{\kern0pt}\ {\isadigit{1}}{\isacharparenright}{\kern0pt}{\isacharparenright}{\kern0pt}{\isacharparenright}{\kern0pt}{\isachardoublequoteclose}\ \isanewline
\ \ \isacommand{then}\isamarkupfalse%
\ \isacommand{obtain}\isamarkupfalse%
\ G\ \isakeyword{where}\ GH{\isacharcolon}{\kern0pt}\ {\isachardoublequoteopen}M{\isacharunderscore}{\kern0pt}generic{\isacharparenleft}{\kern0pt}G{\isacharparenright}{\kern0pt}{\isachardoublequoteclose}\ {\isachardoublequoteopen}p\ {\isasymin}\ G{\isachardoublequoteclose}\ {\isachardoublequoteopen}{\isasymnot}{\isacharparenleft}{\kern0pt}SymExt{\isacharparenleft}{\kern0pt}G{\isacharparenright}{\kern0pt}{\isacharcomma}{\kern0pt}\ map{\isacharparenleft}{\kern0pt}val{\isacharparenleft}{\kern0pt}G{\isacharparenright}{\kern0pt}{\isacharcomma}{\kern0pt}\ {\isacharbrackleft}{\kern0pt}check{\isacharparenleft}{\kern0pt}m{\isacharparenright}{\kern0pt}{\isacharcomma}{\kern0pt}\ binmap{\isacharunderscore}{\kern0pt}row{\isacharprime}{\kern0pt}{\isacharparenleft}{\kern0pt}n{\isacharparenright}{\kern0pt}{\isacharbrackright}{\kern0pt}{\isacharparenright}{\kern0pt}\ {\isasymTurnstile}\ Neg{\isacharparenleft}{\kern0pt}Member{\isacharparenleft}{\kern0pt}{\isadigit{0}}{\isacharcomma}{\kern0pt}\ {\isadigit{1}}{\isacharparenright}{\kern0pt}{\isacharparenright}{\kern0pt}{\isacharparenright}{\kern0pt}{\isachardoublequoteclose}\isanewline
\ \ \ \ \isacommand{by}\isamarkupfalse%
\ auto\isanewline
\isanewline
\ \ \isacommand{have}\isamarkupfalse%
\ listin\ {\isacharcolon}{\kern0pt}\ {\isachardoublequoteopen}{\isacharbrackleft}{\kern0pt}check{\isacharparenleft}{\kern0pt}m{\isacharparenright}{\kern0pt}{\isacharcomma}{\kern0pt}\ binmap{\isacharunderscore}{\kern0pt}row{\isacharprime}{\kern0pt}{\isacharparenleft}{\kern0pt}n{\isacharparenright}{\kern0pt}{\isacharbrackright}{\kern0pt}\ {\isasymin}\ list{\isacharparenleft}{\kern0pt}HS{\isacharparenright}{\kern0pt}{\isachardoublequoteclose}\ \isanewline
\ \ \ \ \ \ \isacommand{apply}\isamarkupfalse%
\ simp\isanewline
\ \ \ \ \ \ \isacommand{apply}\isamarkupfalse%
{\isacharparenleft}{\kern0pt}rule\ conjI{\isacharparenright}{\kern0pt}\isanewline
\ \ \ \ \isacommand{using}\isamarkupfalse%
\ check{\isacharunderscore}{\kern0pt}in{\isacharunderscore}{\kern0pt}HS\ assms\ nat{\isacharunderscore}{\kern0pt}in{\isacharunderscore}{\kern0pt}M\ transM\isanewline
\ \ \ \ \ \ \ \isacommand{apply}\isamarkupfalse%
\ force\isanewline
\ \ \ \ \isacommand{using}\isamarkupfalse%
\ binmap{\isacharunderscore}{\kern0pt}row{\isacharprime}{\kern0pt}{\isacharunderscore}{\kern0pt}HS\ assms\isanewline
\ \ \ \ \isacommand{apply}\isamarkupfalse%
\ force\ \isanewline
\ \ \ \ \isacommand{done}\isamarkupfalse%
\isanewline
\isanewline
\ \ \isacommand{have}\isamarkupfalse%
\ mapin\ {\isacharcolon}{\kern0pt}\ {\isachardoublequoteopen}map{\isacharparenleft}{\kern0pt}val{\isacharparenleft}{\kern0pt}G{\isacharparenright}{\kern0pt}{\isacharcomma}{\kern0pt}\ {\isacharbrackleft}{\kern0pt}check{\isacharparenleft}{\kern0pt}m{\isacharparenright}{\kern0pt}{\isacharcomma}{\kern0pt}\ binmap{\isacharunderscore}{\kern0pt}row{\isacharprime}{\kern0pt}{\isacharparenleft}{\kern0pt}n{\isacharparenright}{\kern0pt}{\isacharbrackright}{\kern0pt}{\isacharparenright}{\kern0pt}\ {\isasymin}\ list{\isacharparenleft}{\kern0pt}SymExt{\isacharparenleft}{\kern0pt}G{\isacharparenright}{\kern0pt}{\isacharparenright}{\kern0pt}{\isachardoublequoteclose}\ \isanewline
\ \ \ \ \isacommand{unfolding}\isamarkupfalse%
\ SymExt{\isacharunderscore}{\kern0pt}def\ \isanewline
\ \ \ \ \isacommand{using}\isamarkupfalse%
\ listin\ \isanewline
\ \ \ \ \isacommand{by}\isamarkupfalse%
\ auto\isanewline
\isanewline
\ \ \isacommand{have}\isamarkupfalse%
\ {\isachardoublequoteopen}SymExt{\isacharparenleft}{\kern0pt}G{\isacharparenright}{\kern0pt}{\isacharcomma}{\kern0pt}\ map{\isacharparenleft}{\kern0pt}val{\isacharparenleft}{\kern0pt}G{\isacharparenright}{\kern0pt}{\isacharcomma}{\kern0pt}\ {\isacharbrackleft}{\kern0pt}check{\isacharparenleft}{\kern0pt}m{\isacharparenright}{\kern0pt}{\isacharcomma}{\kern0pt}\ binmap{\isacharunderscore}{\kern0pt}row{\isacharprime}{\kern0pt}{\isacharparenleft}{\kern0pt}n{\isacharparenright}{\kern0pt}{\isacharbrackright}{\kern0pt}{\isacharparenright}{\kern0pt}\ {\isasymTurnstile}\ Member{\isacharparenleft}{\kern0pt}{\isadigit{0}}{\isacharcomma}{\kern0pt}\ {\isadigit{1}}{\isacharparenright}{\kern0pt}{\isachardoublequoteclose}\isanewline
\ \ \ \ \isacommand{using}\isamarkupfalse%
\ mapin\ GH\ \isacommand{by}\isamarkupfalse%
\ auto\isanewline
\isanewline
\ \ \isacommand{then}\isamarkupfalse%
\ \isacommand{have}\isamarkupfalse%
\ {\isachardoublequoteopen}val{\isacharparenleft}{\kern0pt}G{\isacharcomma}{\kern0pt}\ check{\isacharparenleft}{\kern0pt}m{\isacharparenright}{\kern0pt}{\isacharparenright}{\kern0pt}\ {\isasymin}\ val{\isacharparenleft}{\kern0pt}G{\isacharcomma}{\kern0pt}\ binmap{\isacharunderscore}{\kern0pt}row{\isacharprime}{\kern0pt}{\isacharparenleft}{\kern0pt}n{\isacharparenright}{\kern0pt}{\isacharparenright}{\kern0pt}{\isachardoublequoteclose}\ \isanewline
\ \ \ \ \isacommand{using}\isamarkupfalse%
\ mapin\ \isanewline
\ \ \ \ \isacommand{by}\isamarkupfalse%
\ auto\isanewline
\ \ \isacommand{then}\isamarkupfalse%
\ \isacommand{have}\isamarkupfalse%
\ {\isachardoublequoteopen}m\ {\isasymin}\ binmap{\isacharunderscore}{\kern0pt}row{\isacharparenleft}{\kern0pt}G{\isacharcomma}{\kern0pt}\ n{\isacharparenright}{\kern0pt}{\isachardoublequoteclose}\ \isanewline
\ \ \ \ \isacommand{apply}\isamarkupfalse%
{\isacharparenleft}{\kern0pt}subgoal{\isacharunderscore}{\kern0pt}tac\ {\isachardoublequoteopen}val{\isacharparenleft}{\kern0pt}G{\isacharcomma}{\kern0pt}\ check{\isacharparenleft}{\kern0pt}m{\isacharparenright}{\kern0pt}{\isacharparenright}{\kern0pt}\ {\isacharequal}{\kern0pt}\ m\ {\isasymand}\ val{\isacharparenleft}{\kern0pt}G{\isacharcomma}{\kern0pt}\ binmap{\isacharunderscore}{\kern0pt}row{\isacharprime}{\kern0pt}{\isacharparenleft}{\kern0pt}n{\isacharparenright}{\kern0pt}{\isacharparenright}{\kern0pt}\ {\isacharequal}{\kern0pt}\ binmap{\isacharunderscore}{\kern0pt}row{\isacharparenleft}{\kern0pt}G{\isacharcomma}{\kern0pt}\ n{\isacharparenright}{\kern0pt}{\isachardoublequoteclose}{\isacharparenright}{\kern0pt}\isanewline
\ \ \ \ \ \isacommand{apply}\isamarkupfalse%
\ {\isacharparenleft}{\kern0pt}simp\ add{\isacharcolon}{\kern0pt}binmap{\isacharunderscore}{\kern0pt}row{\isacharunderscore}{\kern0pt}def{\isacharparenright}{\kern0pt}\isanewline
\ \ \ \ \isacommand{apply}\isamarkupfalse%
{\isacharparenleft}{\kern0pt}rule\ conjI{\isacharparenright}{\kern0pt}\isanewline
\ \ \ \ \ \isacommand{apply}\isamarkupfalse%
{\isacharparenleft}{\kern0pt}rule\ valcheck{\isacharparenright}{\kern0pt}\isanewline
\ \ \ \ \isacommand{using}\isamarkupfalse%
\ generic{\isacharunderscore}{\kern0pt}filter{\isacharunderscore}{\kern0pt}contains{\isacharunderscore}{\kern0pt}max\ zero{\isacharunderscore}{\kern0pt}in{\isacharunderscore}{\kern0pt}Fn\ assms\ binmap{\isacharunderscore}{\kern0pt}row{\isacharunderscore}{\kern0pt}eq\ GH\isanewline
\ \ \ \ \isacommand{by}\isamarkupfalse%
\ auto\isanewline
\ \ \isacommand{then}\isamarkupfalse%
\ \isacommand{obtain}\isamarkupfalse%
\ q\ \isakeyword{where}\ qH{\isacharcolon}{\kern0pt}\ {\isachardoublequoteopen}q\ {\isasymin}\ G{\isachardoublequoteclose}\ {\isachardoublequoteopen}q{\isacharbackquote}{\kern0pt}{\isacharless}{\kern0pt}n{\isacharcomma}{\kern0pt}\ m{\isachargreater}{\kern0pt}\ {\isacharequal}{\kern0pt}\ {\isadigit{1}}{\isachardoublequoteclose}\ \isanewline
\ \ \ \ \isacommand{using}\isamarkupfalse%
\ binmap{\isacharunderscore}{\kern0pt}row{\isacharunderscore}{\kern0pt}def\isanewline
\ \ \ \ \isacommand{by}\isamarkupfalse%
\ auto\isanewline
\isanewline
\ \ \isacommand{have}\isamarkupfalse%
\ domin\ {\isacharcolon}{\kern0pt}\ {\isachardoublequoteopen}{\isacharless}{\kern0pt}n{\isacharcomma}{\kern0pt}\ m{\isachargreater}{\kern0pt}\ {\isasymin}\ domain{\isacharparenleft}{\kern0pt}q{\isacharparenright}{\kern0pt}{\isachardoublequoteclose}\ \isanewline
\ \ \ \ \isacommand{apply}\isamarkupfalse%
{\isacharparenleft}{\kern0pt}rule\ ccontr{\isacharparenright}{\kern0pt}\isanewline
\ \ \ \ \isacommand{using}\isamarkupfalse%
\ qH\ apply{\isacharunderscore}{\kern0pt}{\isadigit{0}}\isanewline
\ \ \ \ \isacommand{by}\isamarkupfalse%
\ force\isanewline
\isanewline
\ \ \isacommand{have}\isamarkupfalse%
\ {\isachardoublequoteopen}{\isasymforall}p{\isasymin}G{\isachardot}{\kern0pt}\ {\isasymforall}q{\isasymin}G{\isachardot}{\kern0pt}\ compat{\isacharunderscore}{\kern0pt}in{\isacharparenleft}{\kern0pt}G{\isacharcomma}{\kern0pt}\ Fn{\isacharunderscore}{\kern0pt}leq{\isacharcomma}{\kern0pt}\ p{\isacharcomma}{\kern0pt}\ q{\isacharparenright}{\kern0pt}{\isachardoublequoteclose}\ \isanewline
\ \ \ \ \isacommand{using}\isamarkupfalse%
\ GH\ M{\isacharunderscore}{\kern0pt}generic{\isacharunderscore}{\kern0pt}def\ filter{\isacharunderscore}{\kern0pt}def\ \isacommand{by}\isamarkupfalse%
\ auto\isanewline
\ \ \isacommand{then}\isamarkupfalse%
\ \isacommand{obtain}\isamarkupfalse%
\ r\ \isakeyword{where}\ rH{\isacharcolon}{\kern0pt}\ {\isachardoublequoteopen}{\isacharless}{\kern0pt}r{\isacharcomma}{\kern0pt}\ p{\isachargreater}{\kern0pt}\ {\isasymin}\ Fn{\isacharunderscore}{\kern0pt}leq{\isachardoublequoteclose}\ {\isachardoublequoteopen}{\isacharless}{\kern0pt}r{\isacharcomma}{\kern0pt}\ q{\isachargreater}{\kern0pt}\ {\isasymin}\ Fn{\isacharunderscore}{\kern0pt}leq{\isachardoublequoteclose}\ {\isachardoublequoteopen}r\ {\isasymin}\ G{\isachardoublequoteclose}\ \isanewline
\ \ \ \ \isacommand{using}\isamarkupfalse%
\ compat{\isacharunderscore}{\kern0pt}in{\isacharunderscore}{\kern0pt}def\ qH\ GH\ \isacommand{by}\isamarkupfalse%
\ force\isanewline
\ \ \isacommand{then}\isamarkupfalse%
\ \isacommand{have}\isamarkupfalse%
\ {\isachardoublequoteopen}r{\isacharbackquote}{\kern0pt}{\isacharless}{\kern0pt}n{\isacharcomma}{\kern0pt}\ m{\isachargreater}{\kern0pt}\ {\isacharequal}{\kern0pt}\ q{\isacharbackquote}{\kern0pt}{\isacharless}{\kern0pt}n{\isacharcomma}{\kern0pt}\ m{\isachargreater}{\kern0pt}{\isachardoublequoteclose}\ \isanewline
\ \ \ \ \isacommand{using}\isamarkupfalse%
\ Fn{\isacharunderscore}{\kern0pt}leq{\isacharunderscore}{\kern0pt}preserves{\isacharunderscore}{\kern0pt}value\ domin\ \isanewline
\ \ \ \ \isacommand{by}\isamarkupfalse%
\ auto\isanewline
\ \ \isacommand{then}\isamarkupfalse%
\ \isacommand{have}\isamarkupfalse%
\ v{\isadigit{1}}{\isacharcolon}{\kern0pt}\ {\isachardoublequoteopen}r{\isacharbackquote}{\kern0pt}{\isacharless}{\kern0pt}n{\isacharcomma}{\kern0pt}\ m{\isachargreater}{\kern0pt}\ {\isacharequal}{\kern0pt}\ {\isadigit{1}}{\isachardoublequoteclose}\ \isacommand{using}\isamarkupfalse%
\ qH\ \isacommand{by}\isamarkupfalse%
\ auto\isanewline
\isanewline
\ \ \isacommand{have}\isamarkupfalse%
\ {\isachardoublequoteopen}r{\isacharbackquote}{\kern0pt}{\isacharless}{\kern0pt}n{\isacharcomma}{\kern0pt}\ m{\isachargreater}{\kern0pt}\ {\isacharequal}{\kern0pt}\ {\isadigit{0}}{\isachardoublequoteclose}\ \isanewline
\ \ \ \ \isacommand{using}\isamarkupfalse%
\ rH\ GH\ Fn{\isacharunderscore}{\kern0pt}leq{\isacharunderscore}{\kern0pt}preserves{\isacharunderscore}{\kern0pt}value\ assms\isanewline
\ \ \ \ \isacommand{by}\isamarkupfalse%
\ force\isanewline
\ \ \isacommand{then}\isamarkupfalse%
\ \isacommand{show}\isamarkupfalse%
\ {\isachardoublequoteopen}False{\isachardoublequoteclose}\ \isacommand{using}\isamarkupfalse%
\ v{\isadigit{1}}\ \isacommand{by}\isamarkupfalse%
\ auto\isanewline
\isacommand{qed}\isamarkupfalse%
%
\endisatagproof
{\isafoldproof}%
%
\isadelimproof
\ \isanewline
%
\endisadelimproof
\isanewline
\isacommand{lemma}\isamarkupfalse%
\ binmap{\isacharunderscore}{\kern0pt}row{\isacharprime}{\kern0pt}{\isacharunderscore}{\kern0pt}distinct{\isacharunderscore}{\kern0pt}helper\ {\isacharcolon}{\kern0pt}\ \isanewline
\ \ \isakeyword{fixes}\ n\ m\ p\isanewline
\ \ \isakeyword{assumes}\ {\isachardoublequoteopen}n\ {\isasymin}\ nat{\isachardoublequoteclose}\ {\isachardoublequoteopen}n{\isacharprime}{\kern0pt}\ {\isasymin}\ nat{\isachardoublequoteclose}\ {\isachardoublequoteopen}n\ {\isasymnoteq}\ n{\isacharprime}{\kern0pt}{\isachardoublequoteclose}\ {\isachardoublequoteopen}p\ {\isasymin}\ Fn{\isachardoublequoteclose}\ {\isachardoublequoteopen}ForcesHS{\isacharparenleft}{\kern0pt}p{\isacharcomma}{\kern0pt}\ Equal{\isacharparenleft}{\kern0pt}{\isadigit{0}}{\isacharcomma}{\kern0pt}\ {\isadigit{1}}{\isacharparenright}{\kern0pt}{\isacharcomma}{\kern0pt}\ {\isacharbrackleft}{\kern0pt}binmap{\isacharunderscore}{\kern0pt}row{\isacharprime}{\kern0pt}{\isacharparenleft}{\kern0pt}n{\isacharparenright}{\kern0pt}{\isacharcomma}{\kern0pt}\ binmap{\isacharunderscore}{\kern0pt}row{\isacharprime}{\kern0pt}{\isacharparenleft}{\kern0pt}n{\isacharprime}{\kern0pt}{\isacharparenright}{\kern0pt}{\isacharbrackright}{\kern0pt}{\isacharparenright}{\kern0pt}{\isachardoublequoteclose}\ \isanewline
\ \ \isakeyword{shows}\ False\ \isanewline
%
\isadelimproof
%
\endisadelimproof
%
\isatagproof
\isacommand{proof}\isamarkupfalse%
\ {\isacharminus}{\kern0pt}\ \isanewline
\ \ \isacommand{have}\isamarkupfalse%
\ H{\isacharcolon}{\kern0pt}\ {\isachardoublequoteopen}{\isasymAnd}A{\isachardot}{\kern0pt}\ A\ {\isasymsubseteq}\ nat\ {\isasymLongrightarrow}\ Finite{\isacharparenleft}{\kern0pt}A{\isacharparenright}{\kern0pt}\ {\isasymLongrightarrow}\ {\isasymexists}a\ {\isasymin}\ nat{\isachardot}{\kern0pt}\ a\ {\isasymnotin}\ A{\isachardoublequoteclose}\ \isanewline
\ \ \ \ \isacommand{apply}\isamarkupfalse%
{\isacharparenleft}{\kern0pt}rule\ ccontr{\isacharparenright}{\kern0pt}\isanewline
\ \ \ \ \isacommand{apply}\isamarkupfalse%
{\isacharparenleft}{\kern0pt}rename{\isacharunderscore}{\kern0pt}tac\ A{\isacharcomma}{\kern0pt}\ subgoal{\isacharunderscore}{\kern0pt}tac\ {\isachardoublequoteopen}A\ {\isacharequal}{\kern0pt}\ nat{\isachardoublequoteclose}{\isacharparenright}{\kern0pt}\isanewline
\ \ \ \ \isacommand{using}\isamarkupfalse%
\ nat{\isacharunderscore}{\kern0pt}not{\isacharunderscore}{\kern0pt}Finite\ \isanewline
\ \ \ \ \ \isacommand{apply}\isamarkupfalse%
\ force\ \isanewline
\ \ \ \ \isacommand{apply}\isamarkupfalse%
{\isacharparenleft}{\kern0pt}rule\ equalityI{\isacharparenright}{\kern0pt}\isanewline
\ \ \ \ \isacommand{by}\isamarkupfalse%
\ auto\isanewline
\isanewline
\ \ \isacommand{have}\isamarkupfalse%
\ {\isachardoublequoteopen}finite{\isacharunderscore}{\kern0pt}M{\isacharparenleft}{\kern0pt}domain{\isacharparenleft}{\kern0pt}p{\isacharparenright}{\kern0pt}{\isacharparenright}{\kern0pt}\ {\isasymand}\ domain{\isacharparenleft}{\kern0pt}p{\isacharparenright}{\kern0pt}\ {\isasymsubseteq}\ nat\ {\isasymtimes}\ nat{\isachardoublequoteclose}\ \isacommand{using}\isamarkupfalse%
\ assms\ Fn{\isacharunderscore}{\kern0pt}def\ \isacommand{by}\isamarkupfalse%
\ auto\isanewline
\isanewline
\ \ \isacommand{then}\isamarkupfalse%
\ \isacommand{have}\isamarkupfalse%
\ finitedom\ {\isacharcolon}{\kern0pt}\ {\isachardoublequoteopen}Finite{\isacharparenleft}{\kern0pt}domain{\isacharparenleft}{\kern0pt}p{\isacharparenright}{\kern0pt}{\isacharparenright}{\kern0pt}{\isachardoublequoteclose}\ \isanewline
\ \ \ \ \isacommand{unfolding}\isamarkupfalse%
\ finite{\isacharunderscore}{\kern0pt}M{\isacharunderscore}{\kern0pt}def\ \ \isanewline
\ \ \ \ \isacommand{apply}\isamarkupfalse%
\ clarsimp\isanewline
\ \ \ \ \isacommand{apply}\isamarkupfalse%
{\isacharparenleft}{\kern0pt}rename{\isacharunderscore}{\kern0pt}tac\ n{\isacharcomma}{\kern0pt}\ rule{\isacharunderscore}{\kern0pt}tac\ X{\isacharequal}{\kern0pt}n\ \isakeyword{in}\ lepoll{\isacharunderscore}{\kern0pt}Finite{\isacharparenright}{\kern0pt}\isanewline
\ \ \ \ \isacommand{unfolding}\isamarkupfalse%
\ lepoll{\isacharunderscore}{\kern0pt}def\isanewline
\ \ \ \ \ \isacommand{apply}\isamarkupfalse%
{\isacharparenleft}{\kern0pt}rule\ not{\isacharunderscore}{\kern0pt}emptyE{\isacharcomma}{\kern0pt}\ force{\isacharcomma}{\kern0pt}\ force{\isacharparenright}{\kern0pt}\isanewline
\ \ \ \ \isacommand{apply}\isamarkupfalse%
{\isacharparenleft}{\kern0pt}rule\ nat{\isacharunderscore}{\kern0pt}into{\isacharunderscore}{\kern0pt}Finite{\isacharcomma}{\kern0pt}\ simp{\isacharparenright}{\kern0pt}\isanewline
\ \ \ \ \isacommand{done}\isamarkupfalse%
\isanewline
\isanewline
\ \ \isacommand{have}\isamarkupfalse%
\ {\isachardoublequoteopen}{\isacharbraceleft}{\kern0pt}\ snd{\isacharparenleft}{\kern0pt}x{\isacharparenright}{\kern0pt}{\isachardot}{\kern0pt}\ x\ {\isasymin}\ domain{\isacharparenleft}{\kern0pt}p{\isacharparenright}{\kern0pt}\ {\isacharbraceright}{\kern0pt}\ {\isacharequal}{\kern0pt}\ range{\isacharparenleft}{\kern0pt}domain{\isacharparenleft}{\kern0pt}p{\isacharparenright}{\kern0pt}{\isacharparenright}{\kern0pt}{\isachardoublequoteclose}\ {\isacharparenleft}{\kern0pt}\isakeyword{is}\ {\isachardoublequoteopen}{\isacharquery}{\kern0pt}A\ {\isacharequal}{\kern0pt}\ {\isacharunderscore}{\kern0pt}{\isachardoublequoteclose}{\isacharparenright}{\kern0pt}\isanewline
\ \ \ \ \isacommand{apply}\isamarkupfalse%
{\isacharparenleft}{\kern0pt}rule\ equality{\isacharunderscore}{\kern0pt}iffI{\isacharcomma}{\kern0pt}\ rule\ iffI{\isacharparenright}{\kern0pt}\isanewline
\ \ \ \ \isacommand{using}\isamarkupfalse%
\ assms\ Fn{\isacharunderscore}{\kern0pt}def\isanewline
\ \ \ \ \ \isacommand{apply}\isamarkupfalse%
{\isacharparenleft}{\kern0pt}force{\isacharcomma}{\kern0pt}\ force{\isacharparenright}{\kern0pt}\isanewline
\ \ \ \ \isacommand{done}\isamarkupfalse%
\isanewline
\isanewline
\ \ \isacommand{have}\isamarkupfalse%
\ {\isachardoublequoteopen}Finite{\isacharparenleft}{\kern0pt}{\isacharquery}{\kern0pt}A{\isacharparenright}{\kern0pt}{\isachardoublequoteclose}\ \isanewline
\ \ \ \ \isacommand{apply}\isamarkupfalse%
{\isacharparenleft}{\kern0pt}rule\ Finite{\isacharunderscore}{\kern0pt}RepFun{\isacharcomma}{\kern0pt}\ rule\ finitedom{\isacharparenright}{\kern0pt}\isanewline
\ \ \ \ \isacommand{done}\isamarkupfalse%
\isanewline
\ \ \isacommand{then}\isamarkupfalse%
\ \isacommand{have}\isamarkupfalse%
\ {\isachardoublequoteopen}Finite{\isacharparenleft}{\kern0pt}range{\isacharparenleft}{\kern0pt}domain{\isacharparenleft}{\kern0pt}p{\isacharparenright}{\kern0pt}{\isacharparenright}{\kern0pt}{\isacharparenright}{\kern0pt}{\isachardoublequoteclose}\ \isacommand{using}\isamarkupfalse%
\ {\isacartoucheopen}{\isacharquery}{\kern0pt}A\ {\isacharequal}{\kern0pt}\ range{\isacharparenleft}{\kern0pt}domain{\isacharparenleft}{\kern0pt}p{\isacharparenright}{\kern0pt}{\isacharparenright}{\kern0pt}{\isacartoucheclose}\ \isacommand{by}\isamarkupfalse%
\ auto\isanewline
\ \ \isacommand{then}\isamarkupfalse%
\ \isacommand{have}\isamarkupfalse%
\ {\isachardoublequoteopen}{\isasymexists}x\ {\isasymin}\ nat{\isachardot}{\kern0pt}\ x\ {\isasymnotin}\ range{\isacharparenleft}{\kern0pt}domain{\isacharparenleft}{\kern0pt}p{\isacharparenright}{\kern0pt}{\isacharparenright}{\kern0pt}{\isachardoublequoteclose}\ \isanewline
\ \ \ \ \isacommand{apply}\isamarkupfalse%
{\isacharparenleft}{\kern0pt}rule{\isacharunderscore}{\kern0pt}tac\ H{\isacharparenright}{\kern0pt}\isanewline
\ \ \ \ \isacommand{using}\isamarkupfalse%
\ assms\ Fn{\isacharunderscore}{\kern0pt}def\ \isanewline
\ \ \ \ \isacommand{by}\isamarkupfalse%
\ auto\isanewline
\ \ \isacommand{then}\isamarkupfalse%
\ \isacommand{obtain}\isamarkupfalse%
\ m\ \isakeyword{where}\ mH{\isacharcolon}{\kern0pt}\ {\isachardoublequoteopen}m\ {\isasymin}\ nat{\isachardoublequoteclose}\ {\isachardoublequoteopen}m\ {\isasymnotin}\ range{\isacharparenleft}{\kern0pt}domain{\isacharparenleft}{\kern0pt}p{\isacharparenright}{\kern0pt}{\isacharparenright}{\kern0pt}{\isachardoublequoteclose}\ \isacommand{by}\isamarkupfalse%
\ auto\isanewline
\isanewline
\ \ \isacommand{define}\isamarkupfalse%
\ q\ \isakeyword{where}\ {\isachardoublequoteopen}q\ {\isasymequiv}\ p\ {\isasymunion}\ {\isacharbraceleft}{\kern0pt}\ {\isacharless}{\kern0pt}{\isacharless}{\kern0pt}n{\isacharcomma}{\kern0pt}\ m{\isachargreater}{\kern0pt}{\isacharcomma}{\kern0pt}\ {\isadigit{1}}{\isachargreater}{\kern0pt}\ {\isacharbraceright}{\kern0pt}\ {\isasymunion}\ {\isacharbraceleft}{\kern0pt}\ {\isacharless}{\kern0pt}{\isacharless}{\kern0pt}n{\isacharprime}{\kern0pt}{\isacharcomma}{\kern0pt}\ m{\isachargreater}{\kern0pt}{\isacharcomma}{\kern0pt}\ {\isadigit{0}}{\isachargreater}{\kern0pt}\ {\isacharbraceright}{\kern0pt}{\isachardoublequoteclose}\ \isanewline
\isanewline
\ \ \isacommand{have}\isamarkupfalse%
\ {\isachardoublequoteopen}domain{\isacharparenleft}{\kern0pt}q{\isacharparenright}{\kern0pt}\ {\isacharequal}{\kern0pt}\ domain{\isacharparenleft}{\kern0pt}p{\isacharparenright}{\kern0pt}\ {\isasymunion}\ {\isacharbraceleft}{\kern0pt}\ {\isacharless}{\kern0pt}n{\isacharcomma}{\kern0pt}\ m{\isachargreater}{\kern0pt}{\isacharcomma}{\kern0pt}\ {\isacharless}{\kern0pt}n{\isacharprime}{\kern0pt}{\isacharcomma}{\kern0pt}\ m{\isachargreater}{\kern0pt}\ {\isacharbraceright}{\kern0pt}{\isachardoublequoteclose}\ \isacommand{using}\isamarkupfalse%
\ q{\isacharunderscore}{\kern0pt}def\ \isacommand{by}\isamarkupfalse%
\ auto\isanewline
\ \ \isacommand{then}\isamarkupfalse%
\ \isacommand{have}\isamarkupfalse%
\ {\isachardoublequoteopen}domain{\isacharparenleft}{\kern0pt}q{\isacharparenright}{\kern0pt}\ {\isacharequal}{\kern0pt}\ cons{\isacharparenleft}{\kern0pt}{\isacharless}{\kern0pt}n{\isacharcomma}{\kern0pt}\ m{\isachargreater}{\kern0pt}{\isacharcomma}{\kern0pt}\ cons{\isacharparenleft}{\kern0pt}{\isacharless}{\kern0pt}n{\isacharprime}{\kern0pt}{\isacharcomma}{\kern0pt}\ m{\isachargreater}{\kern0pt}{\isacharcomma}{\kern0pt}\ domain{\isacharparenleft}{\kern0pt}p{\isacharparenright}{\kern0pt}{\isacharparenright}{\kern0pt}{\isacharparenright}{\kern0pt}{\isachardoublequoteclose}\ \isacommand{by}\isamarkupfalse%
\ auto\ \isanewline
\ \ \isacommand{then}\isamarkupfalse%
\ \isacommand{have}\isamarkupfalse%
\ domfin{\isacharcolon}{\kern0pt}\ {\isachardoublequoteopen}finite{\isacharunderscore}{\kern0pt}M{\isacharparenleft}{\kern0pt}domain{\isacharparenleft}{\kern0pt}q{\isacharparenright}{\kern0pt}{\isacharparenright}{\kern0pt}{\isachardoublequoteclose}\ \isanewline
\ \ \ \ \isacommand{apply}\isamarkupfalse%
\ simp\isanewline
\ \ \ \ \isacommand{apply}\isamarkupfalse%
{\isacharparenleft}{\kern0pt}rule\ finite{\isacharunderscore}{\kern0pt}M{\isacharunderscore}{\kern0pt}cons{\isacharparenright}{\kern0pt}{\isacharplus}{\kern0pt}\isanewline
\ \ \ \ \isacommand{using}\isamarkupfalse%
\ pair{\isacharunderscore}{\kern0pt}in{\isacharunderscore}{\kern0pt}M{\isacharunderscore}{\kern0pt}iff\ assms\ nat{\isacharunderscore}{\kern0pt}in{\isacharunderscore}{\kern0pt}M\ transM\ mH\ Fn{\isacharunderscore}{\kern0pt}def\isanewline
\ \ \ \ \isacommand{by}\isamarkupfalse%
\ auto\isanewline
\isanewline
\ \ \isacommand{have}\isamarkupfalse%
\ qinFn\ {\isacharcolon}{\kern0pt}\ {\isachardoublequoteopen}q\ {\isasymin}\ Fn{\isachardoublequoteclose}\ \isanewline
\ \ \ \ \isacommand{unfolding}\isamarkupfalse%
\ q{\isacharunderscore}{\kern0pt}def\ Fn{\isacharunderscore}{\kern0pt}def\ \ \isanewline
\ \ \ \ \isacommand{apply}\isamarkupfalse%
\ clarsimp\isanewline
\ \ \ \ \isacommand{apply}\isamarkupfalse%
{\isacharparenleft}{\kern0pt}rule\ conjI{\isacharparenright}{\kern0pt}\isanewline
\ \ \ \ \isacommand{using}\isamarkupfalse%
\ assms\ Fn{\isacharunderscore}{\kern0pt}def\ mH\ \isanewline
\ \ \ \ \ \isacommand{apply}\isamarkupfalse%
\ force\ \isanewline
\ \ \ \ \isacommand{apply}\isamarkupfalse%
{\isacharparenleft}{\kern0pt}rule\ conjI{\isacharparenright}{\kern0pt}\isanewline
\ \ \ \ \isacommand{using}\isamarkupfalse%
\ Fn{\isacharunderscore}{\kern0pt}def\ assms\ Un{\isacharunderscore}{\kern0pt}closed\ pair{\isacharunderscore}{\kern0pt}in{\isacharunderscore}{\kern0pt}M{\isacharunderscore}{\kern0pt}iff\ singleton{\isacharunderscore}{\kern0pt}in{\isacharunderscore}{\kern0pt}M{\isacharunderscore}{\kern0pt}iff\ nat{\isacharunderscore}{\kern0pt}in{\isacharunderscore}{\kern0pt}M\ transM\ zero{\isacharunderscore}{\kern0pt}in{\isacharunderscore}{\kern0pt}M\ mH\isanewline
\ \ \ \ \ \isacommand{apply}\isamarkupfalse%
\ force\isanewline
\ \ \ \ \isacommand{unfolding}\isamarkupfalse%
\ function{\isacharunderscore}{\kern0pt}def\isanewline
\ \ \ \ \isacommand{apply}\isamarkupfalse%
{\isacharparenleft}{\kern0pt}rule\ conjI{\isacharparenright}{\kern0pt}\isanewline
\ \ \ \ \ \isacommand{apply}\isamarkupfalse%
\ auto{\isacharbrackleft}{\kern0pt}{\isadigit{1}}{\isacharbrackright}{\kern0pt}\isanewline
\ \ \ \ \isacommand{using}\isamarkupfalse%
\ assms\ Fn{\isacharunderscore}{\kern0pt}def\ function{\isacharunderscore}{\kern0pt}def\ mH\isanewline
\ \ \ \ \ \ \ \ \ \ \ \isacommand{apply}\isamarkupfalse%
\ auto{\isacharbrackleft}{\kern0pt}{\isadigit{7}}{\isacharbrackright}{\kern0pt}\isanewline
\ \ \ \ \isacommand{apply}\isamarkupfalse%
{\isacharparenleft}{\kern0pt}rule\ conjI{\isacharparenright}{\kern0pt}\isanewline
\ \ \ \ \isacommand{using}\isamarkupfalse%
\ assms\ Fn{\isacharunderscore}{\kern0pt}def\ mH\isanewline
\ \ \ \ \ \isacommand{apply}\isamarkupfalse%
\ auto{\isacharbrackleft}{\kern0pt}{\isadigit{1}}{\isacharbrackright}{\kern0pt}\isanewline
\ \ \ \ \isacommand{apply}\isamarkupfalse%
{\isacharparenleft}{\kern0pt}rule\ conjI{\isacharparenright}{\kern0pt}\isanewline
\ \ \ \ \isacommand{using}\isamarkupfalse%
\ domfin\ q{\isacharunderscore}{\kern0pt}def\ \isanewline
\ \ \ \ \ \isacommand{apply}\isamarkupfalse%
\ force\isanewline
\ \ \ \ \isacommand{using}\isamarkupfalse%
\ assms\ Fn{\isacharunderscore}{\kern0pt}def\ mH\isanewline
\ \ \ \ \isacommand{by}\isamarkupfalse%
\ auto\isanewline
\isanewline
\ \ \isacommand{have}\isamarkupfalse%
\ appn\ {\isacharcolon}{\kern0pt}\ {\isachardoublequoteopen}q{\isacharbackquote}{\kern0pt}{\isacharless}{\kern0pt}n{\isacharcomma}{\kern0pt}\ m{\isachargreater}{\kern0pt}\ {\isacharequal}{\kern0pt}\ {\isadigit{1}}{\isachardoublequoteclose}\isanewline
\ \ \ \ \isacommand{apply}\isamarkupfalse%
{\isacharparenleft}{\kern0pt}rule\ function{\isacharunderscore}{\kern0pt}apply{\isacharunderscore}{\kern0pt}equality{\isacharparenright}{\kern0pt}\isanewline
\ \ \ \ \isacommand{using}\isamarkupfalse%
\ q{\isacharunderscore}{\kern0pt}def\ qinFn\ Fn{\isacharunderscore}{\kern0pt}def\ \isanewline
\ \ \ \ \isacommand{by}\isamarkupfalse%
\ auto\isanewline
\isanewline
\ \ \isacommand{have}\isamarkupfalse%
\ appn{\isacharprime}{\kern0pt}\ {\isacharcolon}{\kern0pt}\ {\isachardoublequoteopen}q{\isacharbackquote}{\kern0pt}{\isacharless}{\kern0pt}n{\isacharprime}{\kern0pt}{\isacharcomma}{\kern0pt}\ m{\isachargreater}{\kern0pt}\ {\isacharequal}{\kern0pt}\ {\isadigit{0}}{\isachardoublequoteclose}\ \isanewline
\ \ \ \ \isacommand{apply}\isamarkupfalse%
{\isacharparenleft}{\kern0pt}rule\ function{\isacharunderscore}{\kern0pt}apply{\isacharunderscore}{\kern0pt}equality{\isacharparenright}{\kern0pt}\isanewline
\ \ \ \ \isacommand{using}\isamarkupfalse%
\ q{\isacharunderscore}{\kern0pt}def\ qinFn\ Fn{\isacharunderscore}{\kern0pt}def\ \isanewline
\ \ \ \ \isacommand{by}\isamarkupfalse%
\ auto\ \ \ \ \isanewline
\isanewline
\ \ \isacommand{have}\isamarkupfalse%
\ listin\ {\isacharcolon}{\kern0pt}\ {\isachardoublequoteopen}{\isacharbrackleft}{\kern0pt}binmap{\isacharunderscore}{\kern0pt}row{\isacharprime}{\kern0pt}{\isacharparenleft}{\kern0pt}n{\isacharparenright}{\kern0pt}{\isacharcomma}{\kern0pt}\ binmap{\isacharunderscore}{\kern0pt}row{\isacharprime}{\kern0pt}{\isacharparenleft}{\kern0pt}n{\isacharprime}{\kern0pt}{\isacharparenright}{\kern0pt}{\isacharbrackright}{\kern0pt}\ {\isasymin}\ list{\isacharparenleft}{\kern0pt}HS{\isacharparenright}{\kern0pt}{\isachardoublequoteclose}\ \isanewline
\ \ \ \ \isacommand{using}\isamarkupfalse%
\ binmap{\isacharunderscore}{\kern0pt}row{\isacharprime}{\kern0pt}{\isacharunderscore}{\kern0pt}HS\ assms\isanewline
\ \ \ \ \isacommand{by}\isamarkupfalse%
\ auto\isanewline
\ \ \isacommand{then}\isamarkupfalse%
\ \isacommand{have}\isamarkupfalse%
\ mapin\ {\isacharcolon}{\kern0pt}\ {\isachardoublequoteopen}{\isasymAnd}G{\isachardot}{\kern0pt}\ map{\isacharparenleft}{\kern0pt}val{\isacharparenleft}{\kern0pt}G{\isacharparenright}{\kern0pt}{\isacharcomma}{\kern0pt}\ {\isacharbrackleft}{\kern0pt}binmap{\isacharunderscore}{\kern0pt}row{\isacharprime}{\kern0pt}{\isacharparenleft}{\kern0pt}n{\isacharparenright}{\kern0pt}{\isacharcomma}{\kern0pt}\ binmap{\isacharunderscore}{\kern0pt}row{\isacharprime}{\kern0pt}{\isacharparenleft}{\kern0pt}n{\isacharprime}{\kern0pt}{\isacharparenright}{\kern0pt}{\isacharbrackright}{\kern0pt}{\isacharparenright}{\kern0pt}\ {\isasymin}\ list{\isacharparenleft}{\kern0pt}SymExt{\isacharparenleft}{\kern0pt}G{\isacharparenright}{\kern0pt}{\isacharparenright}{\kern0pt}{\isachardoublequoteclose}\ \isanewline
\ \ \ \ \isacommand{unfolding}\isamarkupfalse%
\ SymExt{\isacharunderscore}{\kern0pt}def\isanewline
\ \ \ \ \isacommand{by}\isamarkupfalse%
\ auto\isanewline
\isanewline
\ \ \isacommand{have}\isamarkupfalse%
\ {\isachardoublequoteopen}{\isasymforall}G{\isachardot}{\kern0pt}\ M{\isacharunderscore}{\kern0pt}generic{\isacharparenleft}{\kern0pt}G{\isacharparenright}{\kern0pt}\ {\isasymand}\ q\ {\isasymin}\ G\ {\isasymlongrightarrow}\ SymExt{\isacharparenleft}{\kern0pt}G{\isacharparenright}{\kern0pt}{\isacharcomma}{\kern0pt}\ map{\isacharparenleft}{\kern0pt}val{\isacharparenleft}{\kern0pt}G{\isacharparenright}{\kern0pt}{\isacharcomma}{\kern0pt}\ {\isacharbrackleft}{\kern0pt}binmap{\isacharunderscore}{\kern0pt}row{\isacharprime}{\kern0pt}{\isacharparenleft}{\kern0pt}n{\isacharparenright}{\kern0pt}{\isacharcomma}{\kern0pt}\ binmap{\isacharunderscore}{\kern0pt}row{\isacharprime}{\kern0pt}{\isacharparenleft}{\kern0pt}n{\isacharprime}{\kern0pt}{\isacharparenright}{\kern0pt}{\isacharbrackright}{\kern0pt}{\isacharparenright}{\kern0pt}\ {\isasymTurnstile}\ Equal{\isacharparenleft}{\kern0pt}{\isadigit{0}}{\isacharcomma}{\kern0pt}\ {\isadigit{1}}{\isacharparenright}{\kern0pt}{\isachardoublequoteclose}\isanewline
\ \ \ \ \isacommand{apply}\isamarkupfalse%
{\isacharparenleft}{\kern0pt}rule\ iffD{\isadigit{1}}{\isacharcomma}{\kern0pt}\ rule\ definition{\isacharunderscore}{\kern0pt}of{\isacharunderscore}{\kern0pt}forcing{\isacharunderscore}{\kern0pt}HS{\isacharparenright}{\kern0pt}\isanewline
\ \ \ \ \isacommand{using}\isamarkupfalse%
\ assms\ binmap{\isacharunderscore}{\kern0pt}row{\isacharprime}{\kern0pt}{\isacharunderscore}{\kern0pt}HS\ qinFn\isanewline
\ \ \ \ \ \ \ \ \isacommand{apply}\isamarkupfalse%
\ auto{\isacharbrackleft}{\kern0pt}{\isadigit{3}}{\isacharbrackright}{\kern0pt}\isanewline
\ \ \ \ \ \isacommand{apply}\isamarkupfalse%
\ {\isacharparenleft}{\kern0pt}simp\ del{\isacharcolon}{\kern0pt}FOL{\isacharunderscore}{\kern0pt}sats{\isacharunderscore}{\kern0pt}iff\ pair{\isacharunderscore}{\kern0pt}abs\ add{\isacharcolon}{\kern0pt}\ fm{\isacharunderscore}{\kern0pt}defs\ nat{\isacharunderscore}{\kern0pt}simp{\isacharunderscore}{\kern0pt}union{\isacharparenright}{\kern0pt}\isanewline
\ \ \ \ \isacommand{apply}\isamarkupfalse%
{\isacharparenleft}{\kern0pt}rule{\isacharunderscore}{\kern0pt}tac\ p{\isacharequal}{\kern0pt}p\ \isakeyword{in}\ HS{\isacharunderscore}{\kern0pt}strengthening{\isacharunderscore}{\kern0pt}lemma{\isacharparenright}{\kern0pt}\isanewline
\ \ \ \ \isacommand{using}\isamarkupfalse%
\ assms\ qinFn\ Fn{\isacharunderscore}{\kern0pt}leq{\isacharunderscore}{\kern0pt}def\ q{\isacharunderscore}{\kern0pt}def\ listin\ HS{\isacharunderscore}{\kern0pt}iff\ P{\isacharunderscore}{\kern0pt}name{\isacharunderscore}{\kern0pt}in{\isacharunderscore}{\kern0pt}M\isanewline
\ \ \ \ \isacommand{apply}\isamarkupfalse%
\ auto{\isacharbrackleft}{\kern0pt}{\isadigit{5}}{\isacharbrackright}{\kern0pt}\isanewline
\ \ \ \ \ \isacommand{apply}\isamarkupfalse%
\ {\isacharparenleft}{\kern0pt}simp\ del{\isacharcolon}{\kern0pt}FOL{\isacharunderscore}{\kern0pt}sats{\isacharunderscore}{\kern0pt}iff\ pair{\isacharunderscore}{\kern0pt}abs\ add{\isacharcolon}{\kern0pt}\ fm{\isacharunderscore}{\kern0pt}defs\ nat{\isacharunderscore}{\kern0pt}simp{\isacharunderscore}{\kern0pt}union{\isacharparenright}{\kern0pt}\isanewline
\ \ \ \ \isacommand{using}\isamarkupfalse%
\ assms\isanewline
\ \ \ \ \isacommand{by}\isamarkupfalse%
\ auto\isanewline
\ \ \isacommand{then}\isamarkupfalse%
\ \isacommand{have}\isamarkupfalse%
\ eqH{\isacharcolon}{\kern0pt}\ {\isachardoublequoteopen}{\isasymforall}G{\isachardot}{\kern0pt}\ M{\isacharunderscore}{\kern0pt}generic{\isacharparenleft}{\kern0pt}G{\isacharparenright}{\kern0pt}\ {\isasymand}\ q\ {\isasymin}\ G\ {\isasymlongrightarrow}\ val{\isacharparenleft}{\kern0pt}G{\isacharcomma}{\kern0pt}\ binmap{\isacharunderscore}{\kern0pt}row{\isacharprime}{\kern0pt}{\isacharparenleft}{\kern0pt}n{\isacharparenright}{\kern0pt}{\isacharparenright}{\kern0pt}\ {\isacharequal}{\kern0pt}\ val{\isacharparenleft}{\kern0pt}G{\isacharcomma}{\kern0pt}\ binmap{\isacharunderscore}{\kern0pt}row{\isacharprime}{\kern0pt}{\isacharparenleft}{\kern0pt}n{\isacharprime}{\kern0pt}{\isacharparenright}{\kern0pt}{\isacharparenright}{\kern0pt}{\isachardoublequoteclose}\isanewline
\ \ \ \ \isacommand{using}\isamarkupfalse%
\ mapin\isanewline
\ \ \ \ \isacommand{by}\isamarkupfalse%
\ force\isanewline
\isanewline
\ \ \isacommand{have}\isamarkupfalse%
\ {\isachardoublequoteopen}{\isasymforall}G{\isachardot}{\kern0pt}\ M{\isacharunderscore}{\kern0pt}generic{\isacharparenleft}{\kern0pt}G{\isacharparenright}{\kern0pt}\ {\isasymand}\ q\ {\isasymin}\ G\ {\isasymlongrightarrow}\ SymExt{\isacharparenleft}{\kern0pt}G{\isacharparenright}{\kern0pt}{\isacharcomma}{\kern0pt}\ map{\isacharparenleft}{\kern0pt}val{\isacharparenleft}{\kern0pt}G{\isacharparenright}{\kern0pt}{\isacharcomma}{\kern0pt}\ {\isacharbrackleft}{\kern0pt}check{\isacharparenleft}{\kern0pt}m{\isacharparenright}{\kern0pt}{\isacharcomma}{\kern0pt}\ binmap{\isacharunderscore}{\kern0pt}row{\isacharprime}{\kern0pt}{\isacharparenleft}{\kern0pt}n{\isacharparenright}{\kern0pt}{\isacharbrackright}{\kern0pt}{\isacharparenright}{\kern0pt}\ {\isasymTurnstile}\ Member{\isacharparenleft}{\kern0pt}{\isadigit{0}}{\isacharcomma}{\kern0pt}\ {\isadigit{1}}{\isacharparenright}{\kern0pt}{\isachardoublequoteclose}\ \isanewline
\ \ \ \ \isacommand{apply}\isamarkupfalse%
{\isacharparenleft}{\kern0pt}rule\ Fn{\isacharunderscore}{\kern0pt}{\isadigit{1}}{\isacharunderscore}{\kern0pt}forces{\isacharparenright}{\kern0pt}\isanewline
\ \ \ \ \isacommand{using}\isamarkupfalse%
\ assms\ mH\ qinFn\ appn\isanewline
\ \ \ \ \isacommand{by}\isamarkupfalse%
\ auto\isanewline
\ \ \isacommand{then}\isamarkupfalse%
\ \isacommand{have}\isamarkupfalse%
\ memH{\isacharcolon}{\kern0pt}\ {\isachardoublequoteopen}{\isasymforall}G{\isachardot}{\kern0pt}\ M{\isacharunderscore}{\kern0pt}generic{\isacharparenleft}{\kern0pt}G{\isacharparenright}{\kern0pt}\ {\isasymand}\ q\ {\isasymin}\ G\ {\isasymlongrightarrow}\ val{\isacharparenleft}{\kern0pt}G{\isacharcomma}{\kern0pt}\ check{\isacharparenleft}{\kern0pt}m{\isacharparenright}{\kern0pt}{\isacharparenright}{\kern0pt}\ {\isasymin}\ val{\isacharparenleft}{\kern0pt}G{\isacharcomma}{\kern0pt}\ binmap{\isacharunderscore}{\kern0pt}row{\isacharprime}{\kern0pt}{\isacharparenleft}{\kern0pt}n{\isacharparenright}{\kern0pt}{\isacharparenright}{\kern0pt}{\isachardoublequoteclose}\isanewline
\ \ \ \ \isacommand{apply}\isamarkupfalse%
{\isacharparenleft}{\kern0pt}rule{\isacharunderscore}{\kern0pt}tac\ allI{\isacharparenright}{\kern0pt}\isanewline
\ \ \ \ \isacommand{apply}\isamarkupfalse%
{\isacharparenleft}{\kern0pt}rename{\isacharunderscore}{\kern0pt}tac\ G{\isacharcomma}{\kern0pt}\ subgoal{\isacharunderscore}{\kern0pt}tac\ {\isachardoublequoteopen}map{\isacharparenleft}{\kern0pt}val{\isacharparenleft}{\kern0pt}G{\isacharparenright}{\kern0pt}{\isacharcomma}{\kern0pt}\ {\isacharbrackleft}{\kern0pt}check{\isacharparenleft}{\kern0pt}m{\isacharparenright}{\kern0pt}{\isacharcomma}{\kern0pt}\ binmap{\isacharunderscore}{\kern0pt}row{\isacharprime}{\kern0pt}{\isacharparenleft}{\kern0pt}n{\isacharparenright}{\kern0pt}{\isacharbrackright}{\kern0pt}{\isacharparenright}{\kern0pt}\ {\isasymin}\ list{\isacharparenleft}{\kern0pt}SymExt{\isacharparenleft}{\kern0pt}G{\isacharparenright}{\kern0pt}{\isacharparenright}{\kern0pt}{\isachardoublequoteclose}{\isacharparenright}{\kern0pt}\isanewline
\ \ \ \ \ \isacommand{apply}\isamarkupfalse%
\ force\isanewline
\ \ \ \ \isacommand{unfolding}\isamarkupfalse%
\ SymExt{\isacharunderscore}{\kern0pt}def\isanewline
\ \ \ \ \isacommand{using}\isamarkupfalse%
\ check{\isacharunderscore}{\kern0pt}in{\isacharunderscore}{\kern0pt}HS\ listin\ mH\ nat{\isacharunderscore}{\kern0pt}in{\isacharunderscore}{\kern0pt}M\ transM\isanewline
\ \ \ \ \isacommand{by}\isamarkupfalse%
\ auto\isanewline
\isanewline
\ \ \isacommand{have}\isamarkupfalse%
\ {\isachardoublequoteopen}{\isasymforall}G{\isachardot}{\kern0pt}\ M{\isacharunderscore}{\kern0pt}generic{\isacharparenleft}{\kern0pt}G{\isacharparenright}{\kern0pt}\ {\isasymand}\ q\ {\isasymin}\ G\ {\isasymlongrightarrow}\ SymExt{\isacharparenleft}{\kern0pt}G{\isacharparenright}{\kern0pt}{\isacharcomma}{\kern0pt}\ map{\isacharparenleft}{\kern0pt}val{\isacharparenleft}{\kern0pt}G{\isacharparenright}{\kern0pt}{\isacharcomma}{\kern0pt}\ {\isacharbrackleft}{\kern0pt}check{\isacharparenleft}{\kern0pt}m{\isacharparenright}{\kern0pt}{\isacharcomma}{\kern0pt}\ binmap{\isacharunderscore}{\kern0pt}row{\isacharprime}{\kern0pt}{\isacharparenleft}{\kern0pt}n{\isacharprime}{\kern0pt}{\isacharparenright}{\kern0pt}{\isacharbrackright}{\kern0pt}{\isacharparenright}{\kern0pt}\ {\isasymTurnstile}\ Neg{\isacharparenleft}{\kern0pt}Member{\isacharparenleft}{\kern0pt}{\isadigit{0}}{\isacharcomma}{\kern0pt}\ {\isadigit{1}}{\isacharparenright}{\kern0pt}{\isacharparenright}{\kern0pt}{\isachardoublequoteclose}\isanewline
\ \ \ \ \isacommand{apply}\isamarkupfalse%
{\isacharparenleft}{\kern0pt}rule\ Fn{\isacharunderscore}{\kern0pt}{\isadigit{0}}{\isacharunderscore}{\kern0pt}forces{\isacharparenright}{\kern0pt}\isanewline
\ \ \ \ \isacommand{using}\isamarkupfalse%
\ assms\ mH\ qinFn\ appn{\isacharprime}{\kern0pt}\ q{\isacharunderscore}{\kern0pt}def\isanewline
\ \ \ \ \isacommand{by}\isamarkupfalse%
\ auto\isanewline
\ \ \isacommand{then}\isamarkupfalse%
\ \isacommand{have}\isamarkupfalse%
\ {\isachardoublequoteopen}{\isasymforall}G{\isachardot}{\kern0pt}\ M{\isacharunderscore}{\kern0pt}generic{\isacharparenleft}{\kern0pt}G{\isacharparenright}{\kern0pt}\ {\isasymand}\ q\ {\isasymin}\ G\ {\isasymlongrightarrow}\ val{\isacharparenleft}{\kern0pt}G{\isacharcomma}{\kern0pt}\ check{\isacharparenleft}{\kern0pt}m{\isacharparenright}{\kern0pt}{\isacharparenright}{\kern0pt}\ {\isasymnotin}\ val{\isacharparenleft}{\kern0pt}G{\isacharcomma}{\kern0pt}\ binmap{\isacharunderscore}{\kern0pt}row{\isacharprime}{\kern0pt}{\isacharparenleft}{\kern0pt}n{\isacharprime}{\kern0pt}{\isacharparenright}{\kern0pt}{\isacharparenright}{\kern0pt}{\isachardoublequoteclose}\isanewline
\ \ \ \ \isacommand{apply}\isamarkupfalse%
{\isacharparenleft}{\kern0pt}rule{\isacharunderscore}{\kern0pt}tac\ allI{\isacharparenright}{\kern0pt}\isanewline
\ \ \ \ \isacommand{apply}\isamarkupfalse%
{\isacharparenleft}{\kern0pt}rename{\isacharunderscore}{\kern0pt}tac\ G{\isacharcomma}{\kern0pt}\ subgoal{\isacharunderscore}{\kern0pt}tac\ {\isachardoublequoteopen}map{\isacharparenleft}{\kern0pt}val{\isacharparenleft}{\kern0pt}G{\isacharparenright}{\kern0pt}{\isacharcomma}{\kern0pt}\ {\isacharbrackleft}{\kern0pt}check{\isacharparenleft}{\kern0pt}m{\isacharparenright}{\kern0pt}{\isacharcomma}{\kern0pt}\ binmap{\isacharunderscore}{\kern0pt}row{\isacharprime}{\kern0pt}{\isacharparenleft}{\kern0pt}n{\isacharprime}{\kern0pt}{\isacharparenright}{\kern0pt}{\isacharbrackright}{\kern0pt}{\isacharparenright}{\kern0pt}\ {\isasymin}\ list{\isacharparenleft}{\kern0pt}SymExt{\isacharparenleft}{\kern0pt}G{\isacharparenright}{\kern0pt}{\isacharparenright}{\kern0pt}{\isachardoublequoteclose}{\isacharparenright}{\kern0pt}\isanewline
\ \ \ \ \ \isacommand{apply}\isamarkupfalse%
\ force\isanewline
\ \ \ \ \isacommand{unfolding}\isamarkupfalse%
\ SymExt{\isacharunderscore}{\kern0pt}def\isanewline
\ \ \ \ \isacommand{using}\isamarkupfalse%
\ check{\isacharunderscore}{\kern0pt}in{\isacharunderscore}{\kern0pt}HS\ listin\ mH\ nat{\isacharunderscore}{\kern0pt}in{\isacharunderscore}{\kern0pt}M\ transM\isanewline
\ \ \ \ \isacommand{by}\isamarkupfalse%
\ auto\isanewline
\ \ \isacommand{then}\isamarkupfalse%
\ \isacommand{have}\isamarkupfalse%
\ neqH{\isacharcolon}{\kern0pt}\ {\isachardoublequoteopen}{\isasymforall}G{\isachardot}{\kern0pt}\ M{\isacharunderscore}{\kern0pt}generic{\isacharparenleft}{\kern0pt}G{\isacharparenright}{\kern0pt}\ {\isasymand}\ q\ {\isasymin}\ G\ {\isasymlongrightarrow}\ val{\isacharparenleft}{\kern0pt}G{\isacharcomma}{\kern0pt}\ binmap{\isacharunderscore}{\kern0pt}row{\isacharprime}{\kern0pt}{\isacharparenleft}{\kern0pt}n{\isacharparenright}{\kern0pt}{\isacharparenright}{\kern0pt}\ {\isasymnoteq}\ val{\isacharparenleft}{\kern0pt}G{\isacharcomma}{\kern0pt}\ binmap{\isacharunderscore}{\kern0pt}row{\isacharprime}{\kern0pt}{\isacharparenleft}{\kern0pt}n{\isacharprime}{\kern0pt}{\isacharparenright}{\kern0pt}{\isacharparenright}{\kern0pt}{\isachardoublequoteclose}\isanewline
\ \ \ \ \isacommand{using}\isamarkupfalse%
\ memH\ \isacommand{by}\isamarkupfalse%
\ auto\isanewline
\ \ \isanewline
\ \ \isacommand{have}\isamarkupfalse%
\ {\isachardoublequoteopen}{\isasymexists}G{\isachardot}{\kern0pt}\ M{\isacharunderscore}{\kern0pt}generic{\isacharparenleft}{\kern0pt}G{\isacharparenright}{\kern0pt}\ {\isasymand}\ q\ {\isasymin}\ G{\isachardoublequoteclose}\ \isanewline
\ \ \ \ \isacommand{using}\isamarkupfalse%
\ generic{\isacharunderscore}{\kern0pt}filter{\isacharunderscore}{\kern0pt}existence\ qinFn\ \isanewline
\ \ \ \ \isacommand{by}\isamarkupfalse%
\ force\isanewline
\isanewline
\ \ \isacommand{then}\isamarkupfalse%
\ \isacommand{show}\isamarkupfalse%
\ False\ \isanewline
\ \ \ \ \isacommand{using}\isamarkupfalse%
\ eqH\ neqH\ \isanewline
\ \ \ \ \isacommand{by}\isamarkupfalse%
\ force\isanewline
\isacommand{qed}\isamarkupfalse%
%
\endisatagproof
{\isafoldproof}%
%
\isadelimproof
\isanewline
%
\endisadelimproof
\isanewline
\isacommand{lemma}\isamarkupfalse%
\ binmap{\isacharunderscore}{\kern0pt}row{\isacharprime}{\kern0pt}{\isacharunderscore}{\kern0pt}distinct\ {\isacharcolon}{\kern0pt}\ \isanewline
\ \ \isakeyword{fixes}\ n\ m\ \isanewline
\ \ \isakeyword{assumes}\ {\isachardoublequoteopen}n\ {\isasymin}\ nat{\isachardoublequoteclose}\ {\isachardoublequoteopen}m\ {\isasymin}\ nat{\isachardoublequoteclose}\ {\isachardoublequoteopen}n\ {\isasymnoteq}\ m{\isachardoublequoteclose}\ \isanewline
\ \ \isakeyword{shows}\ {\isachardoublequoteopen}ForcesHS{\isacharparenleft}{\kern0pt}{\isadigit{0}}{\isacharcomma}{\kern0pt}\ Neg{\isacharparenleft}{\kern0pt}Equal{\isacharparenleft}{\kern0pt}{\isadigit{0}}{\isacharcomma}{\kern0pt}\ {\isadigit{1}}{\isacharparenright}{\kern0pt}{\isacharparenright}{\kern0pt}{\isacharcomma}{\kern0pt}\ {\isacharbrackleft}{\kern0pt}binmap{\isacharunderscore}{\kern0pt}row{\isacharprime}{\kern0pt}{\isacharparenleft}{\kern0pt}n{\isacharparenright}{\kern0pt}{\isacharcomma}{\kern0pt}\ binmap{\isacharunderscore}{\kern0pt}row{\isacharprime}{\kern0pt}{\isacharparenleft}{\kern0pt}m{\isacharparenright}{\kern0pt}{\isacharbrackright}{\kern0pt}{\isacharparenright}{\kern0pt}{\isachardoublequoteclose}\ \isanewline
%
\isadelimproof
\isanewline
\ \ %
\endisadelimproof
%
\isatagproof
\isacommand{apply}\isamarkupfalse%
{\isacharparenleft}{\kern0pt}rule\ iffD{\isadigit{2}}{\isacharcomma}{\kern0pt}\ rule\ ForcesHS{\isacharunderscore}{\kern0pt}Neg{\isacharparenright}{\kern0pt}\isanewline
\ \ \isacommand{using}\isamarkupfalse%
\ Fn{\isacharunderscore}{\kern0pt}in{\isacharunderscore}{\kern0pt}M\ binmap{\isacharunderscore}{\kern0pt}row{\isacharprime}{\kern0pt}{\isacharunderscore}{\kern0pt}in{\isacharunderscore}{\kern0pt}M\ assms\ zero{\isacharunderscore}{\kern0pt}in{\isacharunderscore}{\kern0pt}Fn\isanewline
\ \ \ \ \ \isacommand{apply}\isamarkupfalse%
\ auto{\isacharbrackleft}{\kern0pt}{\isadigit{3}}{\isacharbrackright}{\kern0pt}\isanewline
\ \ \isacommand{apply}\isamarkupfalse%
{\isacharparenleft}{\kern0pt}rule\ ccontr{\isacharcomma}{\kern0pt}\ clarsimp{\isacharparenright}{\kern0pt}\isanewline
\ \ \isacommand{using}\isamarkupfalse%
\ binmap{\isacharunderscore}{\kern0pt}row{\isacharprime}{\kern0pt}{\isacharunderscore}{\kern0pt}distinct{\isacharunderscore}{\kern0pt}helper\ assms\isanewline
\ \ \isacommand{by}\isamarkupfalse%
\ auto%
\endisatagproof
{\isafoldproof}%
%
\isadelimproof
\isanewline
%
\endisadelimproof
\isanewline
\isacommand{lemma}\isamarkupfalse%
\ binmap{\isacharunderscore}{\kern0pt}row{\isacharunderscore}{\kern0pt}distinct\ {\isacharcolon}{\kern0pt}\ \isanewline
\ \ \isakeyword{fixes}\ G\ n\ m\ \isanewline
\ \ \isakeyword{assumes}\ {\isachardoublequoteopen}M{\isacharunderscore}{\kern0pt}generic{\isacharparenleft}{\kern0pt}G{\isacharparenright}{\kern0pt}{\isachardoublequoteclose}\ {\isachardoublequoteopen}n\ {\isasymin}\ nat{\isachardoublequoteclose}\ {\isachardoublequoteopen}m\ {\isasymin}\ nat{\isachardoublequoteclose}\ {\isachardoublequoteopen}n\ {\isasymnoteq}\ m{\isachardoublequoteclose}\ \isanewline
\ \ \isakeyword{shows}\ {\isachardoublequoteopen}binmap{\isacharunderscore}{\kern0pt}row{\isacharparenleft}{\kern0pt}G{\isacharcomma}{\kern0pt}\ n{\isacharparenright}{\kern0pt}\ {\isasymnoteq}\ binmap{\isacharunderscore}{\kern0pt}row{\isacharparenleft}{\kern0pt}G{\isacharcomma}{\kern0pt}\ m{\isacharparenright}{\kern0pt}{\isachardoublequoteclose}\ \isanewline
%
\isadelimproof
%
\endisadelimproof
%
\isatagproof
\isacommand{proof}\isamarkupfalse%
\ {\isacharminus}{\kern0pt}\ \isanewline
\isanewline
\ \ \isacommand{have}\isamarkupfalse%
\ listin\ {\isacharcolon}{\kern0pt}\ {\isachardoublequoteopen}{\isacharbrackleft}{\kern0pt}binmap{\isacharunderscore}{\kern0pt}row{\isacharprime}{\kern0pt}{\isacharparenleft}{\kern0pt}n{\isacharparenright}{\kern0pt}{\isacharcomma}{\kern0pt}\ binmap{\isacharunderscore}{\kern0pt}row{\isacharprime}{\kern0pt}{\isacharparenleft}{\kern0pt}m{\isacharparenright}{\kern0pt}{\isacharbrackright}{\kern0pt}\ {\isasymin}\ list{\isacharparenleft}{\kern0pt}HS{\isacharparenright}{\kern0pt}{\isachardoublequoteclose}\ \isanewline
\ \ \ \ \isacommand{using}\isamarkupfalse%
\ binmap{\isacharunderscore}{\kern0pt}row{\isacharprime}{\kern0pt}{\isacharunderscore}{\kern0pt}HS\ assms\isanewline
\ \ \ \ \isacommand{by}\isamarkupfalse%
\ auto\isanewline
\ \ \isanewline
\ \ \isacommand{have}\isamarkupfalse%
\ {\isachardoublequoteopen}ForcesHS{\isacharparenleft}{\kern0pt}{\isadigit{0}}{\isacharcomma}{\kern0pt}\ Neg{\isacharparenleft}{\kern0pt}Equal{\isacharparenleft}{\kern0pt}{\isadigit{0}}{\isacharcomma}{\kern0pt}\ {\isadigit{1}}{\isacharparenright}{\kern0pt}{\isacharparenright}{\kern0pt}{\isacharcomma}{\kern0pt}\ {\isacharbrackleft}{\kern0pt}binmap{\isacharunderscore}{\kern0pt}row{\isacharprime}{\kern0pt}{\isacharparenleft}{\kern0pt}n{\isacharparenright}{\kern0pt}{\isacharcomma}{\kern0pt}\ binmap{\isacharunderscore}{\kern0pt}row{\isacharprime}{\kern0pt}{\isacharparenleft}{\kern0pt}m{\isacharparenright}{\kern0pt}{\isacharbrackright}{\kern0pt}{\isacharparenright}{\kern0pt}{\isachardoublequoteclose}\isanewline
\ \ \ \ \isacommand{apply}\isamarkupfalse%
{\isacharparenleft}{\kern0pt}rule\ binmap{\isacharunderscore}{\kern0pt}row{\isacharprime}{\kern0pt}{\isacharunderscore}{\kern0pt}distinct{\isacharparenright}{\kern0pt}\isanewline
\ \ \ \ \isacommand{using}\isamarkupfalse%
\ assms\isanewline
\ \ \ \ \isacommand{by}\isamarkupfalse%
\ auto\isanewline
\ \ \isacommand{then}\isamarkupfalse%
\ \isacommand{have}\isamarkupfalse%
\ {\isachardoublequoteopen}{\isacharparenleft}{\kern0pt}{\isasymforall}H{\isachardot}{\kern0pt}\ M{\isacharunderscore}{\kern0pt}generic{\isacharparenleft}{\kern0pt}H{\isacharparenright}{\kern0pt}\ {\isasymand}\ {\isadigit{0}}{\isasymin}H\ \ {\isasymlongrightarrow}\ \ SymExt{\isacharparenleft}{\kern0pt}H{\isacharparenright}{\kern0pt}{\isacharcomma}{\kern0pt}\ map{\isacharparenleft}{\kern0pt}val{\isacharparenleft}{\kern0pt}H{\isacharparenright}{\kern0pt}{\isacharcomma}{\kern0pt}{\isacharbrackleft}{\kern0pt}binmap{\isacharunderscore}{\kern0pt}row{\isacharprime}{\kern0pt}{\isacharparenleft}{\kern0pt}n{\isacharparenright}{\kern0pt}{\isacharcomma}{\kern0pt}\ binmap{\isacharunderscore}{\kern0pt}row{\isacharprime}{\kern0pt}{\isacharparenleft}{\kern0pt}m{\isacharparenright}{\kern0pt}{\isacharbrackright}{\kern0pt}{\isacharparenright}{\kern0pt}\ {\isasymTurnstile}\ Neg{\isacharparenleft}{\kern0pt}Equal{\isacharparenleft}{\kern0pt}{\isadigit{0}}{\isacharcomma}{\kern0pt}\ {\isadigit{1}}{\isacharparenright}{\kern0pt}{\isacharparenright}{\kern0pt}{\isacharparenright}{\kern0pt}{\isachardoublequoteclose}\isanewline
\ \ \ \ \isacommand{apply}\isamarkupfalse%
{\isacharparenleft}{\kern0pt}rule{\isacharunderscore}{\kern0pt}tac\ iffD{\isadigit{1}}{\isacharparenright}{\kern0pt}\isanewline
\ \ \ \ \ \isacommand{apply}\isamarkupfalse%
{\isacharparenleft}{\kern0pt}rule\ definition{\isacharunderscore}{\kern0pt}of{\isacharunderscore}{\kern0pt}forcing{\isacharunderscore}{\kern0pt}HS{\isacharparenright}{\kern0pt}\isanewline
\ \ \ \ \isacommand{using}\isamarkupfalse%
\ zero{\isacharunderscore}{\kern0pt}in{\isacharunderscore}{\kern0pt}M\ assms\ zero{\isacharunderscore}{\kern0pt}in{\isacharunderscore}{\kern0pt}Fn\ listin\isanewline
\ \ \ \ \isacommand{apply}\isamarkupfalse%
\ auto{\isacharbrackleft}{\kern0pt}{\isadigit{4}}{\isacharbrackright}{\kern0pt}\isanewline
\ \ \ \ \ \isacommand{apply}\isamarkupfalse%
\ {\isacharparenleft}{\kern0pt}simp\ del{\isacharcolon}{\kern0pt}FOL{\isacharunderscore}{\kern0pt}sats{\isacharunderscore}{\kern0pt}iff\ pair{\isacharunderscore}{\kern0pt}abs\ add{\isacharcolon}{\kern0pt}\ fm{\isacharunderscore}{\kern0pt}defs\ nat{\isacharunderscore}{\kern0pt}simp{\isacharunderscore}{\kern0pt}union{\isacharparenright}{\kern0pt}\isanewline
\ \ \ \ \isacommand{by}\isamarkupfalse%
\ auto\isanewline
\ \ \isacommand{then}\isamarkupfalse%
\ \isacommand{have}\isamarkupfalse%
\ {\isachardoublequoteopen}SymExt{\isacharparenleft}{\kern0pt}G{\isacharparenright}{\kern0pt}{\isacharcomma}{\kern0pt}\ map{\isacharparenleft}{\kern0pt}val{\isacharparenleft}{\kern0pt}G{\isacharparenright}{\kern0pt}{\isacharcomma}{\kern0pt}{\isacharbrackleft}{\kern0pt}binmap{\isacharunderscore}{\kern0pt}row{\isacharprime}{\kern0pt}{\isacharparenleft}{\kern0pt}n{\isacharparenright}{\kern0pt}{\isacharcomma}{\kern0pt}\ binmap{\isacharunderscore}{\kern0pt}row{\isacharprime}{\kern0pt}{\isacharparenleft}{\kern0pt}m{\isacharparenright}{\kern0pt}{\isacharbrackright}{\kern0pt}{\isacharparenright}{\kern0pt}\ {\isasymTurnstile}\ Neg{\isacharparenleft}{\kern0pt}Equal{\isacharparenleft}{\kern0pt}{\isadigit{0}}{\isacharcomma}{\kern0pt}\ {\isadigit{1}}{\isacharparenright}{\kern0pt}{\isacharparenright}{\kern0pt}{\isachardoublequoteclose}\isanewline
\ \ \ \ \isacommand{using}\isamarkupfalse%
\ assms\ local{\isachardot}{\kern0pt}generic{\isacharunderscore}{\kern0pt}filter{\isacharunderscore}{\kern0pt}contains{\isacharunderscore}{\kern0pt}max\ \isanewline
\ \ \ \ \isacommand{by}\isamarkupfalse%
\ auto\isanewline
\ \ \isacommand{then}\isamarkupfalse%
\ \isacommand{have}\isamarkupfalse%
\ {\isachardoublequoteopen}val{\isacharparenleft}{\kern0pt}G{\isacharcomma}{\kern0pt}\ binmap{\isacharunderscore}{\kern0pt}row{\isacharprime}{\kern0pt}{\isacharparenleft}{\kern0pt}n{\isacharparenright}{\kern0pt}{\isacharparenright}{\kern0pt}\ {\isasymnoteq}\ val{\isacharparenleft}{\kern0pt}G{\isacharcomma}{\kern0pt}\ binmap{\isacharunderscore}{\kern0pt}row{\isacharprime}{\kern0pt}{\isacharparenleft}{\kern0pt}m{\isacharparenright}{\kern0pt}{\isacharparenright}{\kern0pt}{\isachardoublequoteclose}\ {\isacharparenleft}{\kern0pt}\isakeyword{is}\ {\isacharquery}{\kern0pt}A{\isacharparenright}{\kern0pt}\isanewline
\ \ \ \ \isacommand{apply}\isamarkupfalse%
{\isacharparenleft}{\kern0pt}subgoal{\isacharunderscore}{\kern0pt}tac\ {\isachardoublequoteopen}val{\isacharparenleft}{\kern0pt}G{\isacharcomma}{\kern0pt}\ binmap{\isacharunderscore}{\kern0pt}row{\isacharprime}{\kern0pt}{\isacharparenleft}{\kern0pt}n{\isacharparenright}{\kern0pt}{\isacharparenright}{\kern0pt}\ {\isasymin}\ SymExt{\isacharparenleft}{\kern0pt}G{\isacharparenright}{\kern0pt}\ {\isasymand}\ val{\isacharparenleft}{\kern0pt}G{\isacharcomma}{\kern0pt}\ binmap{\isacharunderscore}{\kern0pt}row{\isacharprime}{\kern0pt}{\isacharparenleft}{\kern0pt}m{\isacharparenright}{\kern0pt}{\isacharparenright}{\kern0pt}\ {\isasymin}\ SymExt{\isacharparenleft}{\kern0pt}G{\isacharparenright}{\kern0pt}{\isachardoublequoteclose}{\isacharparenright}{\kern0pt}\isanewline
\ \ \ \ \ \isacommand{apply}\isamarkupfalse%
\ force\isanewline
\ \ \ \ \isacommand{using}\isamarkupfalse%
\ SymExt{\isacharunderscore}{\kern0pt}def\ listin\isanewline
\ \ \ \ \isacommand{by}\isamarkupfalse%
\ auto\isanewline
\ \ \isacommand{then}\isamarkupfalse%
\ \isacommand{show}\isamarkupfalse%
\ rowneq{\isacharcolon}{\kern0pt}\ {\isachardoublequoteopen}binmap{\isacharunderscore}{\kern0pt}row{\isacharparenleft}{\kern0pt}G{\isacharcomma}{\kern0pt}\ n{\isacharparenright}{\kern0pt}\ {\isasymnoteq}\ binmap{\isacharunderscore}{\kern0pt}row{\isacharparenleft}{\kern0pt}G{\isacharcomma}{\kern0pt}\ m{\isacharparenright}{\kern0pt}{\isachardoublequoteclose}\ \isanewline
\ \ \ \ \isacommand{apply}\isamarkupfalse%
{\isacharparenleft}{\kern0pt}rule{\isacharunderscore}{\kern0pt}tac\ P{\isacharequal}{\kern0pt}{\isachardoublequoteopen}{\isacharquery}{\kern0pt}A{\isachardoublequoteclose}\ \isakeyword{in}\ iffD{\isadigit{1}}{\isacharparenright}{\kern0pt}\isanewline
\ \ \ \ \ \isacommand{apply}\isamarkupfalse%
{\isacharparenleft}{\kern0pt}subst\ binmap{\isacharunderscore}{\kern0pt}row{\isacharunderscore}{\kern0pt}eq{\isacharparenright}{\kern0pt}\isanewline
\ \ \ \ \isacommand{using}\isamarkupfalse%
\ assms\ \ \isanewline
\ \ \ \ \ \ \ \isacommand{apply}\isamarkupfalse%
\ auto{\isacharbrackleft}{\kern0pt}{\isadigit{2}}{\isacharbrackright}{\kern0pt}\isanewline
\ \ \ \ \ \isacommand{apply}\isamarkupfalse%
{\isacharparenleft}{\kern0pt}subst\ binmap{\isacharunderscore}{\kern0pt}row{\isacharunderscore}{\kern0pt}eq{\isacharparenright}{\kern0pt}\isanewline
\ \ \ \ \isacommand{using}\isamarkupfalse%
\ assms\ nat{\isacharunderscore}{\kern0pt}perms{\isacharunderscore}{\kern0pt}def\ bij{\isacharunderscore}{\kern0pt}def\ inj{\isacharunderscore}{\kern0pt}def\isanewline
\ \ \ \ \isacommand{by}\isamarkupfalse%
\ auto\isanewline
\isacommand{qed}\isamarkupfalse%
%
\endisatagproof
{\isafoldproof}%
%
\isadelimproof
\isanewline
%
\endisadelimproof
\isanewline
\isacommand{end}\isamarkupfalse%
\isanewline
%
\isadelimtheory
%
\endisadelimtheory
%
\isatagtheory
\isacommand{end}\isamarkupfalse%
%
\endisatagtheory
{\isafoldtheory}%
%
\isadelimtheory
%
\endisadelimtheory
%
\end{isabellebody}%
\endinput
%:%file=~/source/repos/ZF-notAC/code/NotAC_Binmap.thy%:%
%:%10=1%:%
%:%11=1%:%
%:%12=2%:%
%:%13=3%:%
%:%18=3%:%
%:%21=4%:%
%:%22=5%:%
%:%23=5%:%
%:%24=6%:%
%:%25=7%:%
%:%26=7%:%
%:%29=8%:%
%:%33=8%:%
%:%34=8%:%
%:%35=8%:%
%:%40=8%:%
%:%43=9%:%
%:%44=10%:%
%:%45=10%:%
%:%46=11%:%
%:%47=11%:%
%:%48=12%:%
%:%49=12%:%
%:%50=13%:%
%:%51=13%:%
%:%52=14%:%
%:%53=15%:%
%:%54=16%:%
%:%55=16%:%
%:%56=17%:%
%:%57=18%:%
%:%58=19%:%
%:%61=20%:%
%:%66=21%:%
%:%67=21%:%
%:%68=22%:%
%:%69=22%:%
%:%70=23%:%
%:%71=23%:%
%:%72=24%:%
%:%73=24%:%
%:%74=24%:%
%:%75=25%:%
%:%76=25%:%
%:%77=26%:%
%:%78=26%:%
%:%79=27%:%
%:%80=27%:%
%:%81=27%:%
%:%82=28%:%
%:%83=28%:%
%:%84=29%:%
%:%85=29%:%
%:%86=30%:%
%:%87=30%:%
%:%88=31%:%
%:%89=31%:%
%:%90=32%:%
%:%91=32%:%
%:%92=32%:%
%:%93=33%:%
%:%94=33%:%
%:%95=34%:%
%:%96=34%:%
%:%97=35%:%
%:%98=35%:%
%:%99=35%:%
%:%100=36%:%
%:%101=36%:%
%:%102=37%:%
%:%103=37%:%
%:%104=37%:%
%:%105=38%:%
%:%106=38%:%
%:%107=39%:%
%:%108=39%:%
%:%109=40%:%
%:%115=40%:%
%:%118=41%:%
%:%119=42%:%
%:%120=42%:%
%:%121=43%:%
%:%130=52%:%
%:%131=53%:%
%:%132=54%:%
%:%133=54%:%
%:%134=55%:%
%:%135=56%:%
%:%136=57%:%
%:%139=58%:%
%:%140=59%:%
%:%144=59%:%
%:%145=59%:%
%:%146=60%:%
%:%147=60%:%
%:%148=61%:%
%:%149=61%:%
%:%150=62%:%
%:%151=62%:%
%:%152=63%:%
%:%153=63%:%
%:%154=64%:%
%:%160=64%:%
%:%163=65%:%
%:%164=66%:%
%:%165=66%:%
%:%166=67%:%
%:%167=68%:%
%:%168=69%:%
%:%171=70%:%
%:%172=71%:%
%:%176=71%:%
%:%177=71%:%
%:%178=72%:%
%:%179=72%:%
%:%180=73%:%
%:%181=73%:%
%:%182=74%:%
%:%183=74%:%
%:%184=75%:%
%:%185=75%:%
%:%186=76%:%
%:%187=76%:%
%:%188=77%:%
%:%189=77%:%
%:%190=78%:%
%:%191=78%:%
%:%192=79%:%
%:%193=79%:%
%:%194=80%:%
%:%195=80%:%
%:%196=81%:%
%:%197=81%:%
%:%198=82%:%
%:%199=82%:%
%:%200=83%:%
%:%201=83%:%
%:%202=84%:%
%:%203=84%:%
%:%204=85%:%
%:%205=85%:%
%:%206=86%:%
%:%207=86%:%
%:%208=87%:%
%:%209=87%:%
%:%210=88%:%
%:%211=88%:%
%:%212=89%:%
%:%213=89%:%
%:%214=90%:%
%:%215=90%:%
%:%216=91%:%
%:%217=91%:%
%:%218=92%:%
%:%219=92%:%
%:%220=93%:%
%:%221=93%:%
%:%222=94%:%
%:%223=94%:%
%:%224=95%:%
%:%225=95%:%
%:%226=96%:%
%:%227=96%:%
%:%228=97%:%
%:%229=97%:%
%:%230=98%:%
%:%236=98%:%
%:%239=99%:%
%:%240=100%:%
%:%241=100%:%
%:%242=101%:%
%:%243=102%:%
%:%244=103%:%
%:%245=104%:%
%:%246=105%:%
%:%249=106%:%
%:%250=107%:%
%:%254=107%:%
%:%255=107%:%
%:%256=108%:%
%:%257=108%:%
%:%258=109%:%
%:%259=110%:%
%:%260=110%:%
%:%261=111%:%
%:%262=111%:%
%:%263=112%:%
%:%264=112%:%
%:%265=113%:%
%:%266=113%:%
%:%267=114%:%
%:%268=114%:%
%:%269=115%:%
%:%270=115%:%
%:%271=116%:%
%:%272=116%:%
%:%273=117%:%
%:%274=117%:%
%:%275=118%:%
%:%276=118%:%
%:%277=119%:%
%:%278=119%:%
%:%279=120%:%
%:%280=120%:%
%:%281=121%:%
%:%282=121%:%
%:%283=122%:%
%:%284=122%:%
%:%285=123%:%
%:%286=123%:%
%:%287=124%:%
%:%288=124%:%
%:%289=125%:%
%:%290=125%:%
%:%291=126%:%
%:%292=126%:%
%:%293=127%:%
%:%294=127%:%
%:%295=128%:%
%:%296=128%:%
%:%297=129%:%
%:%298=129%:%
%:%299=130%:%
%:%300=130%:%
%:%301=131%:%
%:%302=131%:%
%:%303=132%:%
%:%304=132%:%
%:%305=133%:%
%:%306=133%:%
%:%307=134%:%
%:%308=134%:%
%:%309=135%:%
%:%310=135%:%
%:%311=136%:%
%:%312=136%:%
%:%313=137%:%
%:%314=137%:%
%:%315=138%:%
%:%316=138%:%
%:%317=139%:%
%:%318=139%:%
%:%319=140%:%
%:%320=140%:%
%:%321=141%:%
%:%322=141%:%
%:%323=142%:%
%:%324=142%:%
%:%325=143%:%
%:%326=143%:%
%:%327=144%:%
%:%328=144%:%
%:%329=145%:%
%:%330=145%:%
%:%335=145%:%
%:%338=146%:%
%:%339=147%:%
%:%340=147%:%
%:%341=148%:%
%:%342=149%:%
%:%343=150%:%
%:%350=151%:%
%:%351=151%:%
%:%352=152%:%
%:%353=152%:%
%:%354=153%:%
%:%355=154%:%
%:%356=154%:%
%:%357=155%:%
%:%358=155%:%
%:%359=156%:%
%:%360=156%:%
%:%361=157%:%
%:%362=157%:%
%:%363=158%:%
%:%364=158%:%
%:%365=159%:%
%:%366=159%:%
%:%367=160%:%
%:%368=160%:%
%:%369=161%:%
%:%370=161%:%
%:%371=162%:%
%:%372=162%:%
%:%373=163%:%
%:%374=163%:%
%:%375=164%:%
%:%376=164%:%
%:%377=165%:%
%:%378=165%:%
%:%379=166%:%
%:%380=166%:%
%:%381=167%:%
%:%382=167%:%
%:%383=168%:%
%:%384=168%:%
%:%385=169%:%
%:%386=169%:%
%:%387=170%:%
%:%388=171%:%
%:%389=171%:%
%:%390=172%:%
%:%391=172%:%
%:%392=173%:%
%:%393=173%:%
%:%394=174%:%
%:%395=174%:%
%:%396=175%:%
%:%397=175%:%
%:%398=176%:%
%:%399=176%:%
%:%400=177%:%
%:%401=177%:%
%:%402=178%:%
%:%403=179%:%
%:%404=179%:%
%:%405=179%:%
%:%406=179%:%
%:%407=179%:%
%:%408=180%:%
%:%414=180%:%
%:%417=181%:%
%:%418=182%:%
%:%419=182%:%
%:%420=183%:%
%:%423=186%:%
%:%424=187%:%
%:%425=188%:%
%:%426=188%:%
%:%427=189%:%
%:%428=190%:%
%:%429=191%:%
%:%432=192%:%
%:%436=192%:%
%:%437=192%:%
%:%438=193%:%
%:%439=193%:%
%:%440=194%:%
%:%441=194%:%
%:%442=195%:%
%:%443=195%:%
%:%444=196%:%
%:%445=196%:%
%:%446=197%:%
%:%447=197%:%
%:%452=197%:%
%:%455=198%:%
%:%456=199%:%
%:%457=199%:%
%:%458=200%:%
%:%459=201%:%
%:%460=202%:%
%:%463=203%:%
%:%467=203%:%
%:%468=203%:%
%:%469=204%:%
%:%470=204%:%
%:%471=205%:%
%:%472=205%:%
%:%473=206%:%
%:%474=206%:%
%:%475=207%:%
%:%476=207%:%
%:%477=208%:%
%:%478=208%:%
%:%479=209%:%
%:%480=209%:%
%:%481=210%:%
%:%482=210%:%
%:%483=211%:%
%:%484=211%:%
%:%485=212%:%
%:%486=212%:%
%:%487=213%:%
%:%488=213%:%
%:%489=214%:%
%:%490=214%:%
%:%491=215%:%
%:%492=215%:%
%:%493=216%:%
%:%494=216%:%
%:%495=217%:%
%:%496=217%:%
%:%497=218%:%
%:%498=218%:%
%:%499=219%:%
%:%500=219%:%
%:%501=220%:%
%:%502=220%:%
%:%503=221%:%
%:%504=221%:%
%:%505=222%:%
%:%506=222%:%
%:%507=223%:%
%:%508=223%:%
%:%509=224%:%
%:%510=224%:%
%:%511=225%:%
%:%512=225%:%
%:%513=226%:%
%:%514=226%:%
%:%515=227%:%
%:%516=227%:%
%:%517=228%:%
%:%523=228%:%
%:%526=229%:%
%:%527=230%:%
%:%528=230%:%
%:%529=231%:%
%:%530=232%:%
%:%531=233%:%
%:%532=234%:%
%:%533=235%:%
%:%536=236%:%
%:%537=237%:%
%:%541=237%:%
%:%542=237%:%
%:%543=238%:%
%:%544=238%:%
%:%545=239%:%
%:%546=239%:%
%:%547=240%:%
%:%548=240%:%
%:%549=241%:%
%:%550=241%:%
%:%551=242%:%
%:%552=242%:%
%:%553=243%:%
%:%554=243%:%
%:%555=244%:%
%:%556=244%:%
%:%557=245%:%
%:%558=245%:%
%:%559=246%:%
%:%560=246%:%
%:%561=247%:%
%:%562=247%:%
%:%563=248%:%
%:%564=248%:%
%:%565=249%:%
%:%566=249%:%
%:%567=250%:%
%:%568=250%:%
%:%569=251%:%
%:%570=251%:%
%:%571=252%:%
%:%572=252%:%
%:%573=253%:%
%:%574=253%:%
%:%575=254%:%
%:%576=254%:%
%:%577=255%:%
%:%578=255%:%
%:%579=256%:%
%:%580=256%:%
%:%581=257%:%
%:%582=257%:%
%:%583=258%:%
%:%584=258%:%
%:%585=259%:%
%:%586=259%:%
%:%587=260%:%
%:%588=260%:%
%:%589=261%:%
%:%590=261%:%
%:%591=262%:%
%:%592=262%:%
%:%593=263%:%
%:%594=263%:%
%:%595=264%:%
%:%596=264%:%
%:%597=265%:%
%:%598=265%:%
%:%599=266%:%
%:%600=266%:%
%:%601=267%:%
%:%602=267%:%
%:%603=268%:%
%:%604=268%:%
%:%605=269%:%
%:%606=269%:%
%:%607=270%:%
%:%608=270%:%
%:%609=271%:%
%:%610=271%:%
%:%611=272%:%
%:%612=272%:%
%:%613=273%:%
%:%614=273%:%
%:%615=274%:%
%:%616=274%:%
%:%621=274%:%
%:%624=275%:%
%:%625=276%:%
%:%626=276%:%
%:%633=277%:%
%:%634=277%:%
%:%635=278%:%
%:%636=278%:%
%:%637=279%:%
%:%638=280%:%
%:%639=280%:%
%:%640=281%:%
%:%641=282%:%
%:%642=283%:%
%:%643=283%:%
%:%644=284%:%
%:%645=284%:%
%:%646=285%:%
%:%647=285%:%
%:%648=286%:%
%:%649=286%:%
%:%650=287%:%
%:%651=287%:%
%:%652=288%:%
%:%653=288%:%
%:%654=289%:%
%:%655=289%:%
%:%656=290%:%
%:%657=290%:%
%:%658=291%:%
%:%659=291%:%
%:%660=292%:%
%:%661=292%:%
%:%662=293%:%
%:%663=293%:%
%:%664=294%:%
%:%665=294%:%
%:%666=295%:%
%:%667=295%:%
%:%668=296%:%
%:%669=296%:%
%:%670=297%:%
%:%671=298%:%
%:%672=298%:%
%:%673=299%:%
%:%674=299%:%
%:%675=300%:%
%:%676=300%:%
%:%677=301%:%
%:%678=301%:%
%:%679=302%:%
%:%680=302%:%
%:%681=303%:%
%:%682=303%:%
%:%683=304%:%
%:%684=304%:%
%:%685=305%:%
%:%686=305%:%
%:%687=306%:%
%:%688=306%:%
%:%689=307%:%
%:%690=307%:%
%:%691=308%:%
%:%692=308%:%
%:%693=309%:%
%:%694=309%:%
%:%695=310%:%
%:%696=310%:%
%:%697=311%:%
%:%698=311%:%
%:%699=312%:%
%:%700=312%:%
%:%701=313%:%
%:%702=313%:%
%:%703=314%:%
%:%704=315%:%
%:%705=315%:%
%:%706=315%:%
%:%707=315%:%
%:%708=315%:%
%:%709=316%:%
%:%710=317%:%
%:%711=317%:%
%:%712=318%:%
%:%713=318%:%
%:%714=319%:%
%:%715=319%:%
%:%716=320%:%
%:%717=320%:%
%:%718=321%:%
%:%719=321%:%
%:%720=322%:%
%:%721=322%:%
%:%722=323%:%
%:%723=323%:%
%:%724=324%:%
%:%725=325%:%
%:%726=325%:%
%:%727=325%:%
%:%728=325%:%
%:%729=325%:%
%:%730=326%:%
%:%731=326%:%
%:%732=326%:%
%:%733=326%:%
%:%734=326%:%
%:%735=327%:%
%:%736=327%:%
%:%737=327%:%
%:%738=328%:%
%:%739=328%:%
%:%740=329%:%
%:%741=329%:%
%:%742=330%:%
%:%748=330%:%
%:%751=331%:%
%:%752=332%:%
%:%753=332%:%
%:%754=333%:%
%:%755=334%:%
%:%756=335%:%
%:%759=336%:%
%:%760=337%:%
%:%764=337%:%
%:%765=337%:%
%:%766=338%:%
%:%767=338%:%
%:%768=339%:%
%:%769=339%:%
%:%770=340%:%
%:%771=340%:%
%:%772=341%:%
%:%773=341%:%
%:%778=341%:%
%:%781=342%:%
%:%782=343%:%
%:%783=343%:%
%:%784=344%:%
%:%785=345%:%
%:%786=346%:%
%:%793=347%:%
%:%794=347%:%
%:%795=348%:%
%:%796=349%:%
%:%797=349%:%
%:%798=350%:%
%:%799=350%:%
%:%800=351%:%
%:%801=351%:%
%:%802=352%:%
%:%803=352%:%
%:%804=353%:%
%:%805=354%:%
%:%806=354%:%
%:%807=354%:%
%:%808=354%:%
%:%809=355%:%
%:%810=355%:%
%:%811=355%:%
%:%812=356%:%
%:%813=356%:%
%:%814=357%:%
%:%815=357%:%
%:%816=358%:%
%:%817=358%:%
%:%818=358%:%
%:%819=358%:%
%:%820=359%:%
%:%821=360%:%
%:%822=360%:%
%:%823=361%:%
%:%824=361%:%
%:%825=362%:%
%:%826=362%:%
%:%827=363%:%
%:%828=363%:%
%:%829=364%:%
%:%830=364%:%
%:%831=365%:%
%:%832=365%:%
%:%833=366%:%
%:%834=366%:%
%:%835=367%:%
%:%836=367%:%
%:%837=368%:%
%:%838=368%:%
%:%839=369%:%
%:%840=369%:%
%:%841=370%:%
%:%842=370%:%
%:%843=370%:%
%:%844=371%:%
%:%845=371%:%
%:%846=372%:%
%:%847=372%:%
%:%848=373%:%
%:%849=373%:%
%:%850=373%:%
%:%851=374%:%
%:%852=374%:%
%:%853=375%:%
%:%854=375%:%
%:%855=376%:%
%:%856=376%:%
%:%857=377%:%
%:%858=377%:%
%:%859=378%:%
%:%860=378%:%
%:%861=378%:%
%:%862=379%:%
%:%863=379%:%
%:%864=380%:%
%:%865=380%:%
%:%866=381%:%
%:%867=381%:%
%:%868=382%:%
%:%869=382%:%
%:%870=382%:%
%:%871=383%:%
%:%872=383%:%
%:%873=384%:%
%:%874=384%:%
%:%875=384%:%
%:%876=384%:%
%:%877=385%:%
%:%878=385%:%
%:%879=385%:%
%:%880=385%:%
%:%881=385%:%
%:%882=386%:%
%:%883=386%:%
%:%884=386%:%
%:%885=386%:%
%:%886=386%:%
%:%887=387%:%
%:%888=387%:%
%:%889=388%:%
%:%890=388%:%
%:%891=388%:%
%:%892=389%:%
%:%893=389%:%
%:%894=390%:%
%:%895=390%:%
%:%896=391%:%
%:%897=391%:%
%:%898=392%:%
%:%899=392%:%
%:%900=393%:%
%:%901=393%:%
%:%902=394%:%
%:%903=394%:%
%:%904=395%:%
%:%905=395%:%
%:%906=396%:%
%:%907=396%:%
%:%908=397%:%
%:%909=397%:%
%:%910=398%:%
%:%911=398%:%
%:%912=399%:%
%:%913=399%:%
%:%914=399%:%
%:%915=400%:%
%:%916=400%:%
%:%917=401%:%
%:%918=401%:%
%:%919=402%:%
%:%920=402%:%
%:%921=403%:%
%:%922=403%:%
%:%923=403%:%
%:%924=403%:%
%:%925=404%:%
%:%926=404%:%
%:%927=405%:%
%:%928=406%:%
%:%929=406%:%
%:%930=407%:%
%:%931=407%:%
%:%932=408%:%
%:%933=408%:%
%:%934=409%:%
%:%935=409%:%
%:%936=410%:%
%:%937=410%:%
%:%938=411%:%
%:%939=412%:%
%:%940=412%:%
%:%941=413%:%
%:%942=413%:%
%:%943=414%:%
%:%944=414%:%
%:%945=415%:%
%:%946=415%:%
%:%947=416%:%
%:%948=416%:%
%:%949=417%:%
%:%950=417%:%
%:%951=418%:%
%:%952=418%:%
%:%953=419%:%
%:%954=419%:%
%:%955=420%:%
%:%956=420%:%
%:%957=421%:%
%:%958=421%:%
%:%959=422%:%
%:%960=422%:%
%:%961=423%:%
%:%962=423%:%
%:%963=424%:%
%:%964=424%:%
%:%965=424%:%
%:%966=424%:%
%:%967=425%:%
%:%968=425%:%
%:%969=425%:%
%:%970=426%:%
%:%971=426%:%
%:%972=427%:%
%:%973=427%:%
%:%974=428%:%
%:%975=428%:%
%:%976=428%:%
%:%977=429%:%
%:%978=429%:%
%:%979=430%:%
%:%980=430%:%
%:%981=431%:%
%:%982=431%:%
%:%983=432%:%
%:%984=432%:%
%:%985=432%:%
%:%986=432%:%
%:%987=433%:%
%:%988=433%:%
%:%989=433%:%
%:%990=433%:%
%:%991=433%:%
%:%992=434%:%
%:%993=434%:%
%:%994=434%:%
%:%995=434%:%
%:%996=434%:%
%:%997=435%:%
%:%998=435%:%
%:%999=436%:%
%:%1000=437%:%
%:%1001=437%:%
%:%1002=438%:%
%:%1003=438%:%
%:%1004=439%:%
%:%1005=439%:%
%:%1006=440%:%
%:%1007=440%:%
%:%1008=441%:%
%:%1009=441%:%
%:%1010=441%:%
%:%1011=442%:%
%:%1012=442%:%
%:%1013=443%:%
%:%1014=443%:%
%:%1015=444%:%
%:%1016=444%:%
%:%1017=445%:%
%:%1018=445%:%
%:%1019=445%:%
%:%1020=446%:%
%:%1021=446%:%
%:%1022=447%:%
%:%1023=447%:%
%:%1024=447%:%
%:%1025=448%:%
%:%1026=448%:%
%:%1027=449%:%
%:%1028=449%:%
%:%1029=450%:%
%:%1030=450%:%
%:%1031=451%:%
%:%1032=451%:%
%:%1033=452%:%
%:%1034=452%:%
%:%1035=453%:%
%:%1036=453%:%
%:%1037=454%:%
%:%1038=454%:%
%:%1039=455%:%
%:%1040=455%:%
%:%1041=456%:%
%:%1042=456%:%
%:%1043=457%:%
%:%1044=457%:%
%:%1045=458%:%
%:%1046=458%:%
%:%1047=458%:%
%:%1048=459%:%
%:%1049=459%:%
%:%1050=460%:%
%:%1051=460%:%
%:%1052=461%:%
%:%1053=461%:%
%:%1054=462%:%
%:%1055=462%:%
%:%1056=463%:%
%:%1057=463%:%
%:%1058=464%:%
%:%1059=464%:%
%:%1060=465%:%
%:%1061=465%:%
%:%1062=466%:%
%:%1063=466%:%
%:%1064=467%:%
%:%1065=467%:%
%:%1066=468%:%
%:%1067=468%:%
%:%1068=469%:%
%:%1069=469%:%
%:%1070=470%:%
%:%1071=470%:%
%:%1072=471%:%
%:%1073=471%:%
%:%1074=472%:%
%:%1075=472%:%
%:%1076=473%:%
%:%1077=473%:%
%:%1078=474%:%
%:%1079=474%:%
%:%1080=475%:%
%:%1081=475%:%
%:%1082=476%:%
%:%1083=476%:%
%:%1084=477%:%
%:%1085=477%:%
%:%1086=478%:%
%:%1087=478%:%
%:%1088=479%:%
%:%1089=479%:%
%:%1090=480%:%
%:%1091=480%:%
%:%1092=481%:%
%:%1093=481%:%
%:%1094=482%:%
%:%1095=482%:%
%:%1096=483%:%
%:%1097=483%:%
%:%1098=483%:%
%:%1099=484%:%
%:%1100=484%:%
%:%1101=485%:%
%:%1102=485%:%
%:%1103=486%:%
%:%1104=486%:%
%:%1105=487%:%
%:%1106=487%:%
%:%1107=488%:%
%:%1108=488%:%
%:%1109=489%:%
%:%1110=489%:%
%:%1111=490%:%
%:%1112=490%:%
%:%1113=491%:%
%:%1114=491%:%
%:%1115=492%:%
%:%1116=492%:%
%:%1117=492%:%
%:%1118=492%:%
%:%1119=493%:%
%:%1125=493%:%
%:%1128=494%:%
%:%1129=495%:%
%:%1130=495%:%
%:%1131=496%:%
%:%1132=497%:%
%:%1133=498%:%
%:%1140=499%:%
%:%1141=499%:%
%:%1142=500%:%
%:%1143=500%:%
%:%1144=501%:%
%:%1145=501%:%
%:%1146=502%:%
%:%1147=502%:%
%:%1148=502%:%
%:%1149=503%:%
%:%1150=503%:%
%:%1151=503%:%
%:%1152=504%:%
%:%1153=504%:%
%:%1154=505%:%
%:%1155=505%:%
%:%1156=506%:%
%:%1157=506%:%
%:%1158=507%:%
%:%1159=507%:%
%:%1160=508%:%
%:%1161=508%:%
%:%1162=509%:%
%:%1163=509%:%
%:%1164=510%:%
%:%1165=510%:%
%:%1166=510%:%
%:%1167=510%:%
%:%1168=510%:%
%:%1169=511%:%
%:%1170=511%:%
%:%1171=511%:%
%:%1172=512%:%
%:%1173=512%:%
%:%1174=513%:%
%:%1175=513%:%
%:%1176=514%:%
%:%1177=514%:%
%:%1178=515%:%
%:%1179=515%:%
%:%1180=516%:%
%:%1181=517%:%
%:%1182=517%:%
%:%1183=518%:%
%:%1184=518%:%
%:%1185=519%:%
%:%1186=519%:%
%:%1187=520%:%
%:%1188=520%:%
%:%1189=521%:%
%:%1190=521%:%
%:%1191=522%:%
%:%1192=522%:%
%:%1193=523%:%
%:%1194=523%:%
%:%1195=524%:%
%:%1196=524%:%
%:%1197=525%:%
%:%1198=525%:%
%:%1199=526%:%
%:%1200=526%:%
%:%1201=527%:%
%:%1202=527%:%
%:%1203=528%:%
%:%1204=528%:%
%:%1205=529%:%
%:%1206=529%:%
%:%1207=530%:%
%:%1208=530%:%
%:%1209=531%:%
%:%1215=531%:%
%:%1218=532%:%
%:%1219=533%:%
%:%1220=533%:%
%:%1221=534%:%
%:%1222=535%:%
%:%1223=536%:%
%:%1226=537%:%
%:%1230=537%:%
%:%1231=537%:%
%:%1232=538%:%
%:%1233=538%:%
%:%1234=539%:%
%:%1235=539%:%
%:%1236=540%:%
%:%1237=540%:%
%:%1238=541%:%
%:%1239=541%:%
%:%1240=542%:%
%:%1241=542%:%
%:%1242=543%:%
%:%1243=543%:%
%:%1244=544%:%
%:%1245=544%:%
%:%1246=545%:%
%:%1247=545%:%
%:%1248=546%:%
%:%1249=546%:%
%:%1250=547%:%
%:%1251=547%:%
%:%1252=548%:%
%:%1253=548%:%
%:%1254=549%:%
%:%1255=549%:%
%:%1260=549%:%
%:%1263=550%:%
%:%1264=551%:%
%:%1265=551%:%
%:%1266=552%:%
%:%1269=553%:%
%:%1270=554%:%
%:%1274=554%:%
%:%1275=554%:%
%:%1276=555%:%
%:%1277=555%:%
%:%1278=556%:%
%:%1279=556%:%
%:%1280=557%:%
%:%1281=557%:%
%:%1282=558%:%
%:%1283=558%:%
%:%1288=558%:%
%:%1291=559%:%
%:%1292=560%:%
%:%1293=560%:%
%:%1294=561%:%
%:%1295=562%:%
%:%1296=563%:%
%:%1299=564%:%
%:%1303=564%:%
%:%1304=564%:%
%:%1305=565%:%
%:%1306=565%:%
%:%1307=566%:%
%:%1308=566%:%
%:%1313=566%:%
%:%1316=567%:%
%:%1317=568%:%
%:%1318=568%:%
%:%1319=569%:%
%:%1320=570%:%
%:%1321=571%:%
%:%1328=572%:%
%:%1329=572%:%
%:%1330=573%:%
%:%1331=573%:%
%:%1332=574%:%
%:%1333=574%:%
%:%1334=575%:%
%:%1335=576%:%
%:%1336=576%:%
%:%1337=576%:%
%:%1338=576%:%
%:%1339=577%:%
%:%1340=578%:%
%:%1341=578%:%
%:%1342=579%:%
%:%1343=579%:%
%:%1344=580%:%
%:%1345=580%:%
%:%1346=580%:%
%:%1347=581%:%
%:%1348=581%:%
%:%1349=582%:%
%:%1350=582%:%
%:%1351=583%:%
%:%1352=583%:%
%:%1353=583%:%
%:%1354=584%:%
%:%1355=584%:%
%:%1356=585%:%
%:%1357=585%:%
%:%1358=586%:%
%:%1359=586%:%
%:%1360=587%:%
%:%1361=587%:%
%:%1362=587%:%
%:%1363=588%:%
%:%1364=588%:%
%:%1365=589%:%
%:%1366=589%:%
%:%1367=590%:%
%:%1368=590%:%
%:%1369=590%:%
%:%1370=591%:%
%:%1371=591%:%
%:%1372=592%:%
%:%1373=592%:%
%:%1374=593%:%
%:%1375=593%:%
%:%1376=593%:%
%:%1377=594%:%
%:%1378=594%:%
%:%1379=595%:%
%:%1380=595%:%
%:%1381=596%:%
%:%1382=596%:%
%:%1383=597%:%
%:%1384=597%:%
%:%1385=598%:%
%:%1386=598%:%
%:%1387=599%:%
%:%1388=599%:%
%:%1389=600%:%
%:%1390=601%:%
%:%1391=601%:%
%:%1392=601%:%
%:%1393=601%:%
%:%1394=602%:%
%:%1395=603%:%
%:%1396=603%:%
%:%1397=604%:%
%:%1398=604%:%
%:%1399=605%:%
%:%1400=605%:%
%:%1401=605%:%
%:%1402=605%:%
%:%1403=605%:%
%:%1404=606%:%
%:%1405=607%:%
%:%1406=607%:%
%:%1407=607%:%
%:%1408=607%:%
%:%1409=608%:%
%:%1410=608%:%
%:%1411=608%:%
%:%1412=608%:%
%:%1413=608%:%
%:%1414=609%:%
%:%1415=610%:%
%:%1416=610%:%
%:%1417=610%:%
%:%1418=610%:%
%:%1419=611%:%
%:%1420=611%:%
%:%1421=611%:%
%:%1422=612%:%
%:%1423=612%:%
%:%1424=613%:%
%:%1425=613%:%
%:%1426=614%:%
%:%1427=614%:%
%:%1428=615%:%
%:%1429=615%:%
%:%1430=616%:%
%:%1431=616%:%
%:%1432=617%:%
%:%1433=617%:%
%:%1434=617%:%
%:%1435=618%:%
%:%1436=618%:%
%:%1437=619%:%
%:%1438=619%:%
%:%1439=620%:%
%:%1440=620%:%
%:%1441=621%:%
%:%1442=621%:%
%:%1443=621%:%
%:%1444=621%:%
%:%1445=622%:%
%:%1446=623%:%
%:%1447=623%:%
%:%1448=624%:%
%:%1449=624%:%
%:%1450=625%:%
%:%1451=625%:%
%:%1452=626%:%
%:%1453=626%:%
%:%1454=627%:%
%:%1455=627%:%
%:%1456=628%:%
%:%1457=628%:%
%:%1458=629%:%
%:%1459=629%:%
%:%1460=630%:%
%:%1461=630%:%
%:%1462=631%:%
%:%1463=631%:%
%:%1464=631%:%
%:%1465=632%:%
%:%1466=632%:%
%:%1467=633%:%
%:%1468=633%:%
%:%1469=634%:%
%:%1470=634%:%
%:%1471=634%:%
%:%1472=635%:%
%:%1473=635%:%
%:%1474=636%:%
%:%1475=636%:%
%:%1476=637%:%
%:%1482=637%:%
%:%1485=638%:%
%:%1486=639%:%
%:%1487=639%:%
%:%1490=640%:%
%:%1494=640%:%
%:%1495=640%:%
%:%1496=641%:%
%:%1497=641%:%
%:%1498=642%:%
%:%1499=642%:%
%:%1504=642%:%
%:%1507=643%:%
%:%1508=644%:%
%:%1509=644%:%
%:%1512=645%:%
%:%1516=645%:%
%:%1517=645%:%
%:%1518=646%:%
%:%1519=646%:%
%:%1520=647%:%
%:%1521=647%:%
%:%1522=648%:%
%:%1523=648%:%
%:%1524=649%:%
%:%1525=649%:%
%:%1526=650%:%
%:%1527=650%:%
%:%1528=651%:%
%:%1529=651%:%
%:%1530=652%:%
%:%1531=652%:%
%:%1532=653%:%
%:%1533=653%:%
%:%1534=654%:%
%:%1535=654%:%
%:%1536=655%:%
%:%1537=655%:%
%:%1538=656%:%
%:%1539=656%:%
%:%1540=657%:%
%:%1541=657%:%
%:%1542=658%:%
%:%1543=658%:%
%:%1544=659%:%
%:%1545=659%:%
%:%1546=660%:%
%:%1547=660%:%
%:%1548=661%:%
%:%1549=661%:%
%:%1550=662%:%
%:%1551=662%:%
%:%1552=663%:%
%:%1553=663%:%
%:%1554=664%:%
%:%1555=664%:%
%:%1556=665%:%
%:%1557=665%:%
%:%1558=666%:%
%:%1559=666%:%
%:%1560=667%:%
%:%1561=667%:%
%:%1562=668%:%
%:%1563=668%:%
%:%1564=669%:%
%:%1565=669%:%
%:%1570=669%:%
%:%1573=670%:%
%:%1574=671%:%
%:%1575=671%:%
%:%1576=672%:%
%:%1577=673%:%
%:%1578=674%:%
%:%1585=675%:%
%:%1586=675%:%
%:%1587=676%:%
%:%1588=677%:%
%:%1589=677%:%
%:%1590=677%:%
%:%1591=677%:%
%:%1592=678%:%
%:%1593=679%:%
%:%1594=679%:%
%:%1595=680%:%
%:%1596=681%:%
%:%1597=681%:%
%:%1598=682%:%
%:%1599=682%:%
%:%1600=683%:%
%:%1601=683%:%
%:%1602=684%:%
%:%1603=684%:%
%:%1604=684%:%
%:%1605=685%:%
%:%1606=685%:%
%:%1607=686%:%
%:%1608=686%:%
%:%1609=687%:%
%:%1610=687%:%
%:%1611=688%:%
%:%1612=689%:%
%:%1613=689%:%
%:%1614=690%:%
%:%1615=690%:%
%:%1616=691%:%
%:%1617=691%:%
%:%1618=692%:%
%:%1619=692%:%
%:%1620=692%:%
%:%1621=693%:%
%:%1622=693%:%
%:%1623=694%:%
%:%1624=694%:%
%:%1625=695%:%
%:%1626=696%:%
%:%1627=696%:%
%:%1628=697%:%
%:%1629=697%:%
%:%1630=698%:%
%:%1631=698%:%
%:%1632=699%:%
%:%1633=699%:%
%:%1634=700%:%
%:%1635=701%:%
%:%1636=701%:%
%:%1637=702%:%
%:%1638=702%:%
%:%1639=703%:%
%:%1640=703%:%
%:%1641=704%:%
%:%1642=704%:%
%:%1643=705%:%
%:%1644=705%:%
%:%1645=706%:%
%:%1646=706%:%
%:%1647=707%:%
%:%1648=708%:%
%:%1649=708%:%
%:%1650=709%:%
%:%1651=709%:%
%:%1652=710%:%
%:%1653=710%:%
%:%1654=711%:%
%:%1655=711%:%
%:%1656=712%:%
%:%1657=712%:%
%:%1658=713%:%
%:%1659=713%:%
%:%1660=714%:%
%:%1661=714%:%
%:%1662=715%:%
%:%1663=716%:%
%:%1664=716%:%
%:%1665=717%:%
%:%1666=717%:%
%:%1667=718%:%
%:%1668=718%:%
%:%1669=719%:%
%:%1670=719%:%
%:%1671=720%:%
%:%1672=720%:%
%:%1673=720%:%
%:%1674=720%:%
%:%1675=720%:%
%:%1676=721%:%
%:%1677=721%:%
%:%1678=721%:%
%:%1679=721%:%
%:%1680=722%:%
%:%1681=722%:%
%:%1682=722%:%
%:%1683=722%:%
%:%1684=723%:%
%:%1685=723%:%
%:%1686=723%:%
%:%1687=723%:%
%:%1688=723%:%
%:%1689=724%:%
%:%1690=724%:%
%:%1691=724%:%
%:%1692=724%:%
%:%1693=725%:%
%:%1694=725%:%
%:%1695=725%:%
%:%1696=725%:%
%:%1697=725%:%
%:%1698=726%:%
%:%1699=726%:%
%:%1700=727%:%
%:%1701=728%:%
%:%1702=728%:%
%:%1703=729%:%
%:%1704=729%:%
%:%1705=730%:%
%:%1706=730%:%
%:%1707=731%:%
%:%1708=731%:%
%:%1709=732%:%
%:%1710=733%:%
%:%1711=733%:%
%:%1712=734%:%
%:%1713=734%:%
%:%1714=735%:%
%:%1715=735%:%
%:%1716=736%:%
%:%1717=736%:%
%:%1718=737%:%
%:%1719=737%:%
%:%1720=738%:%
%:%1721=738%:%
%:%1722=739%:%
%:%1723=740%:%
%:%1724=740%:%
%:%1725=741%:%
%:%1726=741%:%
%:%1727=742%:%
%:%1728=742%:%
%:%1729=743%:%
%:%1730=744%:%
%:%1731=744%:%
%:%1732=745%:%
%:%1733=745%:%
%:%1734=746%:%
%:%1735=746%:%
%:%1736=747%:%
%:%1737=747%:%
%:%1738=748%:%
%:%1739=748%:%
%:%1740=749%:%
%:%1741=749%:%
%:%1742=750%:%
%:%1743=750%:%
%:%1744=750%:%
%:%1745=751%:%
%:%1746=751%:%
%:%1747=752%:%
%:%1748=752%:%
%:%1749=753%:%
%:%1750=753%:%
%:%1751=754%:%
%:%1752=755%:%
%:%1753=755%:%
%:%1754=755%:%
%:%1755=756%:%
%:%1756=756%:%
%:%1757=757%:%
%:%1758=757%:%
%:%1759=758%:%
%:%1760=758%:%
%:%1761=759%:%
%:%1762=759%:%
%:%1763=760%:%
%:%1764=760%:%
%:%1765=761%:%
%:%1766=761%:%
%:%1767=762%:%
%:%1773=762%:%
%:%1776=763%:%
%:%1777=764%:%
%:%1778=764%:%
%:%1779=765%:%
%:%1780=766%:%
%:%1781=767%:%
%:%1788=768%:%
%:%1789=768%:%
%:%1790=769%:%
%:%1791=769%:%
%:%1792=770%:%
%:%1793=771%:%
%:%1794=772%:%
%:%1795=772%:%
%:%1796=773%:%
%:%1797=773%:%
%:%1798=774%:%
%:%1799=774%:%
%:%1800=775%:%
%:%1801=775%:%
%:%1802=776%:%
%:%1803=776%:%
%:%1804=777%:%
%:%1805=777%:%
%:%1806=778%:%
%:%1807=779%:%
%:%1808=779%:%
%:%1809=779%:%
%:%1810=779%:%
%:%1811=779%:%
%:%1812=780%:%
%:%1813=780%:%
%:%1814=780%:%
%:%1815=780%:%
%:%1816=781%:%
%:%1817=782%:%
%:%1818=782%:%
%:%1819=783%:%
%:%1820=783%:%
%:%1821=784%:%
%:%1822=784%:%
%:%1823=785%:%
%:%1824=785%:%
%:%1825=786%:%
%:%1826=786%:%
%:%1827=787%:%
%:%1828=787%:%
%:%1829=788%:%
%:%1830=788%:%
%:%1831=789%:%
%:%1837=789%:%
%:%1840=790%:%
%:%1841=791%:%
%:%1842=791%:%
%:%1843=792%:%
%:%1844=793%:%
%:%1845=794%:%
%:%1852=795%:%
%:%1853=795%:%
%:%1854=796%:%
%:%1855=796%:%
%:%1856=797%:%
%:%1857=797%:%
%:%1858=798%:%
%:%1859=798%:%
%:%1860=799%:%
%:%1861=799%:%
%:%1862=800%:%
%:%1863=800%:%
%:%1864=801%:%
%:%1865=801%:%
%:%1866=802%:%
%:%1867=802%:%
%:%1868=803%:%
%:%1869=803%:%
%:%1870=803%:%
%:%1871=804%:%
%:%1872=804%:%
%:%1873=805%:%
%:%1874=805%:%
%:%1875=806%:%
%:%1876=807%:%
%:%1877=807%:%
%:%1878=808%:%
%:%1879=808%:%
%:%1880=809%:%
%:%1881=809%:%
%:%1882=810%:%
%:%1883=810%:%
%:%1884=811%:%
%:%1885=811%:%
%:%1886=811%:%
%:%1887=811%:%
%:%1888=811%:%
%:%1889=812%:%
%:%1890=812%:%
%:%1891=812%:%
%:%1892=813%:%
%:%1893=813%:%
%:%1894=814%:%
%:%1895=814%:%
%:%1896=815%:%
%:%1897=815%:%
%:%1898=816%:%
%:%1899=816%:%
%:%1900=817%:%
%:%1901=817%:%
%:%1902=818%:%
%:%1903=818%:%
%:%1904=819%:%
%:%1905=819%:%
%:%1906=819%:%
%:%1907=819%:%
%:%1908=819%:%
%:%1909=820%:%
%:%1910=820%:%
%:%1911=820%:%
%:%1912=821%:%
%:%1913=821%:%
%:%1914=822%:%
%:%1915=822%:%
%:%1916=823%:%
%:%1917=823%:%
%:%1918=824%:%
%:%1919=824%:%
%:%1920=825%:%
%:%1921=825%:%
%:%1922=826%:%
%:%1923=826%:%
%:%1924=826%:%
%:%1925=827%:%
%:%1926=827%:%
%:%1927=828%:%
%:%1928=828%:%
%:%1929=829%:%
%:%1930=829%:%
%:%1931=830%:%
%:%1932=830%:%
%:%1933=831%:%
%:%1934=831%:%
%:%1935=831%:%
%:%1936=832%:%
%:%1937=832%:%
%:%1938=833%:%
%:%1939=833%:%
%:%1940=833%:%
%:%1941=834%:%
%:%1942=834%:%
%:%1943=835%:%
%:%1944=835%:%
%:%1945=836%:%
%:%1946=836%:%
%:%1947=837%:%
%:%1948=837%:%
%:%1949=838%:%
%:%1950=838%:%
%:%1951=839%:%
%:%1952=839%:%
%:%1953=840%:%
%:%1954=840%:%
%:%1955=841%:%
%:%1961=841%:%
%:%1964=842%:%
%:%1965=843%:%
%:%1966=843%:%
%:%1967=844%:%
%:%1968=845%:%
%:%1969=846%:%
%:%1972=847%:%
%:%1973=848%:%
%:%1977=848%:%
%:%1978=848%:%
%:%1979=849%:%
%:%1980=849%:%
%:%1981=850%:%
%:%1982=850%:%
%:%1983=851%:%
%:%1984=851%:%
%:%1985=852%:%
%:%1986=852%:%
%:%1987=853%:%
%:%1988=853%:%
%:%1989=854%:%
%:%1990=854%:%
%:%1991=855%:%
%:%1992=855%:%
%:%1993=856%:%
%:%1994=856%:%
%:%1995=857%:%
%:%1996=857%:%
%:%1997=858%:%
%:%1998=858%:%
%:%1999=859%:%
%:%2000=859%:%
%:%2001=860%:%
%:%2002=860%:%
%:%2003=861%:%
%:%2004=861%:%
%:%2005=862%:%
%:%2006=862%:%
%:%2007=863%:%
%:%2008=863%:%
%:%2013=863%:%
%:%2016=864%:%
%:%2017=865%:%
%:%2018=865%:%
%:%2019=866%:%
%:%2020=867%:%
%:%2021=868%:%
%:%2028=869%:%
%:%2029=869%:%
%:%2030=870%:%
%:%2031=870%:%
%:%2032=870%:%
%:%2033=870%:%
%:%2034=871%:%
%:%2035=871%:%
%:%2036=871%:%
%:%2037=871%:%
%:%2038=871%:%
%:%2039=872%:%
%:%2040=873%:%
%:%2041=873%:%
%:%2042=874%:%
%:%2043=874%:%
%:%2044=875%:%
%:%2045=875%:%
%:%2046=876%:%
%:%2047=876%:%
%:%2048=877%:%
%:%2049=878%:%
%:%2050=878%:%
%:%2051=879%:%
%:%2052=879%:%
%:%2053=880%:%
%:%2054=880%:%
%:%2055=881%:%
%:%2056=881%:%
%:%2057=882%:%
%:%2058=883%:%
%:%2059=883%:%
%:%2060=883%:%
%:%2061=883%:%
%:%2062=884%:%
%:%2068=884%:%
%:%2071=885%:%
%:%2072=886%:%
%:%2073=886%:%
%:%2074=887%:%
%:%2075=888%:%
%:%2076=889%:%
%:%2079=890%:%
%:%2083=890%:%
%:%2084=890%:%
%:%2085=891%:%
%:%2086=891%:%
%:%2087=892%:%
%:%2088=892%:%
%:%2089=893%:%
%:%2090=893%:%
%:%2091=894%:%
%:%2092=894%:%
%:%2093=895%:%
%:%2094=895%:%
%:%2095=896%:%
%:%2096=896%:%
%:%2097=897%:%
%:%2098=897%:%
%:%2099=898%:%
%:%2100=898%:%
%:%2101=899%:%
%:%2102=899%:%
%:%2103=900%:%
%:%2104=900%:%
%:%2105=901%:%
%:%2106=901%:%
%:%2107=902%:%
%:%2108=902%:%
%:%2109=903%:%
%:%2110=903%:%
%:%2111=904%:%
%:%2112=904%:%
%:%2113=905%:%
%:%2114=905%:%
%:%2115=906%:%
%:%2116=906%:%
%:%2121=906%:%
%:%2124=907%:%
%:%2125=908%:%
%:%2126=908%:%
%:%2127=909%:%
%:%2128=910%:%
%:%2129=911%:%
%:%2136=912%:%
%:%2137=912%:%
%:%2138=913%:%
%:%2139=913%:%
%:%2140=914%:%
%:%2141=914%:%
%:%2142=914%:%
%:%2143=915%:%
%:%2144=915%:%
%:%2145=916%:%
%:%2146=917%:%
%:%2147=917%:%
%:%2148=918%:%
%:%2149=918%:%
%:%2150=919%:%
%:%2151=919%:%
%:%2152=920%:%
%:%2153=920%:%
%:%2154=921%:%
%:%2155=921%:%
%:%2156=922%:%
%:%2157=922%:%
%:%2158=923%:%
%:%2159=923%:%
%:%2160=924%:%
%:%2161=924%:%
%:%2162=925%:%
%:%2163=926%:%
%:%2164=926%:%
%:%2165=927%:%
%:%2166=927%:%
%:%2167=928%:%
%:%2168=928%:%
%:%2169=929%:%
%:%2170=929%:%
%:%2171=930%:%
%:%2172=931%:%
%:%2173=931%:%
%:%2174=932%:%
%:%2175=932%:%
%:%2176=932%:%
%:%2177=933%:%
%:%2178=934%:%
%:%2179=934%:%
%:%2180=934%:%
%:%2181=935%:%
%:%2182=935%:%
%:%2183=936%:%
%:%2184=936%:%
%:%2185=937%:%
%:%2186=937%:%
%:%2187=937%:%
%:%2188=938%:%
%:%2189=938%:%
%:%2190=939%:%
%:%2191=939%:%
%:%2192=940%:%
%:%2193=940%:%
%:%2194=941%:%
%:%2195=941%:%
%:%2196=942%:%
%:%2197=942%:%
%:%2198=943%:%
%:%2199=943%:%
%:%2200=944%:%
%:%2201=944%:%
%:%2202=944%:%
%:%2203=945%:%
%:%2204=945%:%
%:%2205=946%:%
%:%2206=946%:%
%:%2207=947%:%
%:%2208=948%:%
%:%2209=948%:%
%:%2210=949%:%
%:%2211=949%:%
%:%2212=950%:%
%:%2213=950%:%
%:%2214=951%:%
%:%2215=951%:%
%:%2216=952%:%
%:%2217=953%:%
%:%2218=953%:%
%:%2219=954%:%
%:%2220=954%:%
%:%2221=954%:%
%:%2222=955%:%
%:%2223=955%:%
%:%2224=955%:%
%:%2225=956%:%
%:%2226=956%:%
%:%2227=956%:%
%:%2228=957%:%
%:%2229=957%:%
%:%2230=957%:%
%:%2231=958%:%
%:%2232=958%:%
%:%2233=959%:%
%:%2234=959%:%
%:%2235=960%:%
%:%2236=960%:%
%:%2237=960%:%
%:%2238=960%:%
%:%2239=960%:%
%:%2240=961%:%
%:%2241=962%:%
%:%2242=962%:%
%:%2243=963%:%
%:%2244=963%:%
%:%2245=964%:%
%:%2246=964%:%
%:%2247=965%:%
%:%2248=965%:%
%:%2249=965%:%
%:%2250=965%:%
%:%2251=965%:%
%:%2252=966%:%
%:%2258=966%:%
%:%2261=967%:%
%:%2262=968%:%
%:%2263=968%:%
%:%2264=969%:%
%:%2265=970%:%
%:%2266=971%:%
%:%2273=972%:%
%:%2274=972%:%
%:%2275=973%:%
%:%2276=973%:%
%:%2277=974%:%
%:%2278=974%:%
%:%2279=975%:%
%:%2280=975%:%
%:%2281=976%:%
%:%2282=976%:%
%:%2283=977%:%
%:%2284=977%:%
%:%2285=978%:%
%:%2286=978%:%
%:%2287=979%:%
%:%2288=979%:%
%:%2289=980%:%
%:%2290=981%:%
%:%2291=981%:%
%:%2292=981%:%
%:%2293=981%:%
%:%2294=982%:%
%:%2295=983%:%
%:%2296=983%:%
%:%2297=983%:%
%:%2298=984%:%
%:%2299=984%:%
%:%2300=985%:%
%:%2301=985%:%
%:%2302=986%:%
%:%2303=986%:%
%:%2304=987%:%
%:%2305=987%:%
%:%2306=988%:%
%:%2307=988%:%
%:%2308=989%:%
%:%2309=989%:%
%:%2310=990%:%
%:%2311=990%:%
%:%2312=991%:%
%:%2313=992%:%
%:%2314=992%:%
%:%2315=993%:%
%:%2316=993%:%
%:%2317=994%:%
%:%2318=994%:%
%:%2319=995%:%
%:%2320=995%:%
%:%2321=996%:%
%:%2322=996%:%
%:%2323=997%:%
%:%2324=998%:%
%:%2325=998%:%
%:%2326=999%:%
%:%2327=999%:%
%:%2328=1000%:%
%:%2329=1000%:%
%:%2330=1001%:%
%:%2331=1001%:%
%:%2332=1001%:%
%:%2333=1001%:%
%:%2334=1001%:%
%:%2335=1002%:%
%:%2336=1002%:%
%:%2337=1002%:%
%:%2338=1003%:%
%:%2339=1003%:%
%:%2340=1004%:%
%:%2341=1004%:%
%:%2342=1005%:%
%:%2343=1005%:%
%:%2344=1006%:%
%:%2345=1006%:%
%:%2346=1006%:%
%:%2347=1006%:%
%:%2348=1007%:%
%:%2349=1008%:%
%:%2350=1008%:%
%:%2351=1009%:%
%:%2352=1010%:%
%:%2353=1010%:%
%:%2354=1010%:%
%:%2355=1010%:%
%:%2356=1011%:%
%:%2357=1011%:%
%:%2358=1011%:%
%:%2359=1011%:%
%:%2360=1012%:%
%:%2361=1012%:%
%:%2362=1012%:%
%:%2363=1013%:%
%:%2364=1013%:%
%:%2365=1014%:%
%:%2366=1014%:%
%:%2367=1015%:%
%:%2368=1015%:%
%:%2369=1016%:%
%:%2370=1016%:%
%:%2371=1017%:%
%:%2372=1018%:%
%:%2373=1018%:%
%:%2374=1019%:%
%:%2375=1019%:%
%:%2376=1020%:%
%:%2377=1020%:%
%:%2378=1021%:%
%:%2379=1021%:%
%:%2380=1022%:%
%:%2381=1022%:%
%:%2382=1023%:%
%:%2383=1023%:%
%:%2384=1024%:%
%:%2385=1024%:%
%:%2386=1025%:%
%:%2387=1025%:%
%:%2388=1026%:%
%:%2389=1026%:%
%:%2390=1027%:%
%:%2391=1027%:%
%:%2392=1028%:%
%:%2393=1028%:%
%:%2394=1029%:%
%:%2395=1029%:%
%:%2396=1030%:%
%:%2397=1030%:%
%:%2398=1031%:%
%:%2399=1031%:%
%:%2400=1032%:%
%:%2401=1032%:%
%:%2402=1033%:%
%:%2403=1033%:%
%:%2404=1034%:%
%:%2405=1034%:%
%:%2406=1035%:%
%:%2407=1035%:%
%:%2408=1036%:%
%:%2409=1036%:%
%:%2410=1037%:%
%:%2411=1037%:%
%:%2412=1038%:%
%:%2413=1038%:%
%:%2414=1039%:%
%:%2415=1039%:%
%:%2416=1040%:%
%:%2417=1041%:%
%:%2418=1041%:%
%:%2419=1042%:%
%:%2420=1042%:%
%:%2421=1043%:%
%:%2422=1043%:%
%:%2423=1044%:%
%:%2424=1044%:%
%:%2425=1045%:%
%:%2426=1046%:%
%:%2427=1046%:%
%:%2428=1047%:%
%:%2429=1047%:%
%:%2430=1048%:%
%:%2431=1048%:%
%:%2432=1049%:%
%:%2433=1049%:%
%:%2434=1050%:%
%:%2435=1051%:%
%:%2436=1051%:%
%:%2437=1052%:%
%:%2438=1052%:%
%:%2439=1053%:%
%:%2440=1053%:%
%:%2441=1054%:%
%:%2442=1054%:%
%:%2443=1054%:%
%:%2444=1055%:%
%:%2445=1055%:%
%:%2446=1056%:%
%:%2447=1056%:%
%:%2448=1057%:%
%:%2449=1058%:%
%:%2450=1058%:%
%:%2451=1059%:%
%:%2452=1059%:%
%:%2453=1060%:%
%:%2454=1060%:%
%:%2455=1061%:%
%:%2456=1061%:%
%:%2457=1062%:%
%:%2458=1062%:%
%:%2459=1063%:%
%:%2460=1063%:%
%:%2461=1064%:%
%:%2462=1064%:%
%:%2463=1065%:%
%:%2464=1065%:%
%:%2465=1066%:%
%:%2466=1066%:%
%:%2467=1067%:%
%:%2468=1067%:%
%:%2469=1068%:%
%:%2470=1068%:%
%:%2471=1069%:%
%:%2472=1069%:%
%:%2473=1069%:%
%:%2474=1070%:%
%:%2475=1070%:%
%:%2476=1071%:%
%:%2477=1071%:%
%:%2478=1072%:%
%:%2479=1073%:%
%:%2480=1073%:%
%:%2481=1074%:%
%:%2482=1074%:%
%:%2483=1075%:%
%:%2484=1075%:%
%:%2485=1076%:%
%:%2486=1076%:%
%:%2487=1077%:%
%:%2488=1077%:%
%:%2489=1077%:%
%:%2490=1078%:%
%:%2491=1078%:%
%:%2492=1079%:%
%:%2493=1079%:%
%:%2494=1080%:%
%:%2495=1080%:%
%:%2496=1081%:%
%:%2497=1081%:%
%:%2498=1082%:%
%:%2499=1082%:%
%:%2500=1083%:%
%:%2501=1083%:%
%:%2502=1084%:%
%:%2503=1085%:%
%:%2504=1085%:%
%:%2505=1086%:%
%:%2506=1086%:%
%:%2507=1087%:%
%:%2508=1087%:%
%:%2509=1088%:%
%:%2510=1088%:%
%:%2511=1089%:%
%:%2512=1089%:%
%:%2513=1089%:%
%:%2514=1090%:%
%:%2515=1090%:%
%:%2516=1091%:%
%:%2517=1091%:%
%:%2518=1092%:%
%:%2519=1092%:%
%:%2520=1093%:%
%:%2521=1093%:%
%:%2522=1094%:%
%:%2523=1094%:%
%:%2524=1095%:%
%:%2525=1095%:%
%:%2526=1096%:%
%:%2527=1096%:%
%:%2528=1096%:%
%:%2529=1097%:%
%:%2530=1097%:%
%:%2531=1097%:%
%:%2532=1098%:%
%:%2533=1099%:%
%:%2534=1099%:%
%:%2535=1100%:%
%:%2536=1100%:%
%:%2537=1101%:%
%:%2538=1101%:%
%:%2539=1102%:%
%:%2540=1103%:%
%:%2541=1103%:%
%:%2542=1103%:%
%:%2543=1104%:%
%:%2544=1104%:%
%:%2545=1105%:%
%:%2546=1105%:%
%:%2547=1106%:%
%:%2553=1106%:%
%:%2556=1107%:%
%:%2557=1108%:%
%:%2558=1108%:%
%:%2559=1109%:%
%:%2560=1110%:%
%:%2561=1111%:%
%:%2564=1112%:%
%:%2565=1113%:%
%:%2569=1113%:%
%:%2570=1113%:%
%:%2571=1114%:%
%:%2572=1114%:%
%:%2573=1115%:%
%:%2574=1115%:%
%:%2575=1116%:%
%:%2576=1116%:%
%:%2577=1117%:%
%:%2578=1117%:%
%:%2579=1118%:%
%:%2580=1118%:%
%:%2585=1118%:%
%:%2588=1119%:%
%:%2589=1120%:%
%:%2590=1120%:%
%:%2591=1121%:%
%:%2592=1122%:%
%:%2593=1123%:%
%:%2600=1124%:%
%:%2601=1124%:%
%:%2602=1125%:%
%:%2603=1126%:%
%:%2604=1126%:%
%:%2605=1127%:%
%:%2606=1127%:%
%:%2607=1128%:%
%:%2608=1128%:%
%:%2609=1129%:%
%:%2610=1130%:%
%:%2611=1130%:%
%:%2612=1131%:%
%:%2613=1131%:%
%:%2614=1132%:%
%:%2615=1132%:%
%:%2616=1133%:%
%:%2617=1133%:%
%:%2618=1134%:%
%:%2619=1134%:%
%:%2620=1134%:%
%:%2621=1135%:%
%:%2622=1135%:%
%:%2623=1136%:%
%:%2624=1136%:%
%:%2625=1137%:%
%:%2626=1137%:%
%:%2627=1138%:%
%:%2628=1138%:%
%:%2629=1139%:%
%:%2630=1139%:%
%:%2631=1140%:%
%:%2632=1140%:%
%:%2633=1141%:%
%:%2634=1141%:%
%:%2635=1141%:%
%:%2636=1142%:%
%:%2637=1142%:%
%:%2638=1143%:%
%:%2639=1143%:%
%:%2640=1144%:%
%:%2641=1144%:%
%:%2642=1144%:%
%:%2643=1145%:%
%:%2644=1145%:%
%:%2645=1146%:%
%:%2646=1146%:%
%:%2647=1147%:%
%:%2648=1147%:%
%:%2649=1148%:%
%:%2650=1148%:%
%:%2651=1149%:%
%:%2652=1149%:%
%:%2653=1149%:%
%:%2654=1150%:%
%:%2655=1150%:%
%:%2656=1151%:%
%:%2657=1151%:%
%:%2658=1152%:%
%:%2659=1152%:%
%:%2660=1153%:%
%:%2661=1153%:%
%:%2662=1154%:%
%:%2663=1154%:%
%:%2664=1155%:%
%:%2665=1155%:%
%:%2666=1156%:%
%:%2667=1156%:%
%:%2668=1157%:%
%:%2674=1157%:%
%:%2677=1158%:%
%:%2678=1159%:%
%:%2679=1159%:%
%:%2686=1160%:%

%
\begin{isabellebody}%
\setisabellecontext{NotAC{\isacharunderscore}{\kern0pt}Inj}%
%
\isadelimtheory
%
\endisadelimtheory
%
\isatagtheory
\isacommand{theory}\isamarkupfalse%
\ NotAC{\isacharunderscore}{\kern0pt}Inj\ \isanewline
\ \ \isakeyword{imports}\ Utilities\isanewline
\isakeyword{begin}%
\endisatagtheory
{\isafoldtheory}%
%
\isadelimtheory
\isanewline
%
\endisadelimtheory
\isanewline
\isacommand{context}\isamarkupfalse%
\ M{\isacharunderscore}{\kern0pt}ZF{\isacharunderscore}{\kern0pt}trans\ \isakeyword{begin}\isanewline
\isanewline
\isanewline
\isacommand{schematic{\isacharunderscore}{\kern0pt}goal}\isamarkupfalse%
\ well{\isacharunderscore}{\kern0pt}ord{\isacharunderscore}{\kern0pt}iso{\isacharunderscore}{\kern0pt}separation{\isacharunderscore}{\kern0pt}fm{\isacharunderscore}{\kern0pt}auto{\isacharcolon}{\kern0pt}\isanewline
\ \ \isakeyword{assumes}\isanewline
\ \ \ \ {\isachardoublequoteopen}x\ {\isasymin}\ M{\isachardoublequoteclose}\ {\isachardoublequoteopen}A\ {\isasymin}\ M{\isachardoublequoteclose}\ {\isachardoublequoteopen}f\ {\isasymin}\ M{\isachardoublequoteclose}\ {\isachardoublequoteopen}r\ {\isasymin}\ M{\isachardoublequoteclose}\ \isanewline
\ \ \isakeyword{shows}\isanewline
\ \ \ \ {\isachardoublequoteopen}{\isacharparenleft}{\kern0pt}x{\isasymin}A\ {\isasymlongrightarrow}\ {\isacharparenleft}{\kern0pt}{\isasymexists}y\ {\isasymin}\ M{\isachardot}{\kern0pt}\ {\isacharparenleft}{\kern0pt}{\isasymexists}p\ {\isasymin}\ M{\isachardot}{\kern0pt}\ fun{\isacharunderscore}{\kern0pt}apply{\isacharparenleft}{\kern0pt}{\isacharhash}{\kern0pt}{\isacharhash}{\kern0pt}M{\isacharcomma}{\kern0pt}f{\isacharcomma}{\kern0pt}x{\isacharcomma}{\kern0pt}y{\isacharparenright}{\kern0pt}\ {\isasymand}\ pair{\isacharparenleft}{\kern0pt}{\isacharhash}{\kern0pt}{\isacharhash}{\kern0pt}M{\isacharcomma}{\kern0pt}y{\isacharcomma}{\kern0pt}x{\isacharcomma}{\kern0pt}p{\isacharparenright}{\kern0pt}\ {\isasymand}\ p\ {\isasymin}\ r{\isacharparenright}{\kern0pt}{\isacharparenright}{\kern0pt}\ {\isasymand}\ order{\isacharunderscore}{\kern0pt}isomorphism{\isacharparenleft}{\kern0pt}{\isacharhash}{\kern0pt}{\isacharhash}{\kern0pt}M{\isacharcomma}{\kern0pt}\ x{\isacharcomma}{\kern0pt}\ x{\isacharcomma}{\kern0pt}x{\isacharcomma}{\kern0pt}x{\isacharcomma}{\kern0pt}x{\isacharparenright}{\kern0pt}\ {\isacharparenright}{\kern0pt}\ {\isasymlongleftrightarrow}\ sats{\isacharparenleft}{\kern0pt}M{\isacharcomma}{\kern0pt}{\isacharquery}{\kern0pt}fm{\isacharcomma}{\kern0pt}{\isacharbrackleft}{\kern0pt}x{\isacharcomma}{\kern0pt}\ A{\isacharcomma}{\kern0pt}\ f{\isacharcomma}{\kern0pt}\ r{\isacharbrackright}{\kern0pt}{\isacharparenright}{\kern0pt}{\isachardoublequoteclose}\ \isanewline
%
\isadelimproof
\ \ %
\endisadelimproof
%
\isatagproof
\isacommand{by}\isamarkupfalse%
\ {\isacharparenleft}{\kern0pt}insert\ assms\ {\isacharsemicolon}{\kern0pt}\ {\isacharparenleft}{\kern0pt}rule\ sep{\isacharunderscore}{\kern0pt}rules\ {\isacharbar}{\kern0pt}\ simp{\isacharparenright}{\kern0pt}{\isacharplus}{\kern0pt}{\isacharparenright}{\kern0pt}%
\endisatagproof
{\isafoldproof}%
%
\isadelimproof
\ \isanewline
%
\endisadelimproof
\isanewline
\isacommand{definition}\isamarkupfalse%
\ well{\isacharunderscore}{\kern0pt}ord{\isacharunderscore}{\kern0pt}iso{\isacharunderscore}{\kern0pt}separation{\isacharunderscore}{\kern0pt}fm\ \isakeyword{where}\ {\isachardoublequoteopen}well{\isacharunderscore}{\kern0pt}ord{\isacharunderscore}{\kern0pt}iso{\isacharunderscore}{\kern0pt}separation{\isacharunderscore}{\kern0pt}fm\ {\isasymequiv}\ Implies{\isacharparenleft}{\kern0pt}Member{\isacharparenleft}{\kern0pt}{\isadigit{0}}{\isacharcomma}{\kern0pt}\ {\isadigit{1}}{\isacharparenright}{\kern0pt}{\isacharcomma}{\kern0pt}\ Exists{\isacharparenleft}{\kern0pt}Exists{\isacharparenleft}{\kern0pt}And{\isacharparenleft}{\kern0pt}fun{\isacharunderscore}{\kern0pt}apply{\isacharunderscore}{\kern0pt}fm{\isacharparenleft}{\kern0pt}{\isadigit{4}}{\isacharcomma}{\kern0pt}\ {\isadigit{2}}{\isacharcomma}{\kern0pt}\ {\isadigit{1}}{\isacharparenright}{\kern0pt}{\isacharcomma}{\kern0pt}\ And{\isacharparenleft}{\kern0pt}pair{\isacharunderscore}{\kern0pt}fm{\isacharparenleft}{\kern0pt}{\isadigit{1}}{\isacharcomma}{\kern0pt}\ {\isadigit{2}}{\isacharcomma}{\kern0pt}\ {\isadigit{0}}{\isacharparenright}{\kern0pt}{\isacharcomma}{\kern0pt}\ Member{\isacharparenleft}{\kern0pt}{\isadigit{0}}{\isacharcomma}{\kern0pt}\ {\isadigit{5}}{\isacharparenright}{\kern0pt}{\isacharparenright}{\kern0pt}{\isacharparenright}{\kern0pt}{\isacharparenright}{\kern0pt}{\isacharparenright}{\kern0pt}{\isacharparenright}{\kern0pt}{\isachardoublequoteclose}\isanewline
\isanewline
\isacommand{lemma}\isamarkupfalse%
\ well{\isacharunderscore}{\kern0pt}ord{\isacharunderscore}{\kern0pt}iso{\isacharunderscore}{\kern0pt}separation\ {\isacharcolon}{\kern0pt}\ \isanewline
\ \ \isakeyword{fixes}\ A\ f\ r\ \isanewline
\ \ \isakeyword{assumes}\ {\isachardoublequoteopen}A\ {\isasymin}\ M{\isachardoublequoteclose}\ {\isachardoublequoteopen}f\ {\isasymin}\ M{\isachardoublequoteclose}\ {\isachardoublequoteopen}r\ {\isasymin}\ M{\isachardoublequoteclose}\ \isanewline
\ \ \isakeyword{shows}\ {\isachardoublequoteopen}separation{\isacharparenleft}{\kern0pt}{\isacharhash}{\kern0pt}{\isacharhash}{\kern0pt}M{\isacharcomma}{\kern0pt}\ {\isasymlambda}x{\isachardot}{\kern0pt}\ x{\isasymin}A\ {\isasymlongrightarrow}\ {\isacharparenleft}{\kern0pt}{\isasymexists}y{\isacharbrackleft}{\kern0pt}{\isacharhash}{\kern0pt}{\isacharhash}{\kern0pt}M{\isacharbrackright}{\kern0pt}{\isachardot}{\kern0pt}\ {\isacharparenleft}{\kern0pt}{\isasymexists}p{\isacharbrackleft}{\kern0pt}{\isacharhash}{\kern0pt}{\isacharhash}{\kern0pt}M{\isacharbrackright}{\kern0pt}{\isachardot}{\kern0pt}\ fun{\isacharunderscore}{\kern0pt}apply{\isacharparenleft}{\kern0pt}{\isacharhash}{\kern0pt}{\isacharhash}{\kern0pt}M{\isacharcomma}{\kern0pt}f{\isacharcomma}{\kern0pt}x{\isacharcomma}{\kern0pt}y{\isacharparenright}{\kern0pt}\ {\isasymand}\ pair{\isacharparenleft}{\kern0pt}{\isacharhash}{\kern0pt}{\isacharhash}{\kern0pt}M{\isacharcomma}{\kern0pt}y{\isacharcomma}{\kern0pt}x{\isacharcomma}{\kern0pt}p{\isacharparenright}{\kern0pt}\ {\isasymand}\ p\ {\isasymin}\ r{\isacharparenright}{\kern0pt}{\isacharparenright}{\kern0pt}{\isacharparenright}{\kern0pt}{\isachardoublequoteclose}\isanewline
%
\isadelimproof
\isanewline
\ \ %
\endisadelimproof
%
\isatagproof
\isacommand{apply}\isamarkupfalse%
{\isacharparenleft}{\kern0pt}rule{\isacharunderscore}{\kern0pt}tac\ P{\isacharequal}{\kern0pt}{\isachardoublequoteopen}separation{\isacharparenleft}{\kern0pt}{\isacharhash}{\kern0pt}{\isacharhash}{\kern0pt}M{\isacharcomma}{\kern0pt}\ {\isasymlambda}x{\isachardot}{\kern0pt}\ x{\isasymin}A\ {\isasymlongrightarrow}\ {\isacharparenleft}{\kern0pt}{\isasymexists}y\ {\isasymin}\ M{\isachardot}{\kern0pt}\ {\isacharparenleft}{\kern0pt}{\isasymexists}p\ {\isasymin}\ M{\isachardot}{\kern0pt}\ fun{\isacharunderscore}{\kern0pt}apply{\isacharparenleft}{\kern0pt}{\isacharhash}{\kern0pt}{\isacharhash}{\kern0pt}M{\isacharcomma}{\kern0pt}f{\isacharcomma}{\kern0pt}x{\isacharcomma}{\kern0pt}y{\isacharparenright}{\kern0pt}\ {\isasymand}\ pair{\isacharparenleft}{\kern0pt}{\isacharhash}{\kern0pt}{\isacharhash}{\kern0pt}M{\isacharcomma}{\kern0pt}y{\isacharcomma}{\kern0pt}x{\isacharcomma}{\kern0pt}p{\isacharparenright}{\kern0pt}\ {\isasymand}\ p\ {\isasymin}\ r{\isacharparenright}{\kern0pt}{\isacharparenright}{\kern0pt}{\isacharparenright}{\kern0pt}{\isachardoublequoteclose}\ \isakeyword{in}\ iffD{\isadigit{1}}{\isacharparenright}{\kern0pt}\isanewline
\ \ \ \isacommand{apply}\isamarkupfalse%
{\isacharparenleft}{\kern0pt}rule\ separation{\isacharunderscore}{\kern0pt}cong{\isacharparenright}{\kern0pt}\isanewline
\ \ \isacommand{using}\isamarkupfalse%
\ assms\isanewline
\ \ \ \isacommand{apply}\isamarkupfalse%
\ force\isanewline
\ \ \isacommand{apply}\isamarkupfalse%
{\isacharparenleft}{\kern0pt}rule{\isacharunderscore}{\kern0pt}tac\ P{\isacharequal}{\kern0pt}{\isachardoublequoteopen}separation{\isacharparenleft}{\kern0pt}{\isacharhash}{\kern0pt}{\isacharhash}{\kern0pt}M{\isacharcomma}{\kern0pt}\ {\isasymlambda}x{\isachardot}{\kern0pt}\ sats{\isacharparenleft}{\kern0pt}M{\isacharcomma}{\kern0pt}\ well{\isacharunderscore}{\kern0pt}ord{\isacharunderscore}{\kern0pt}iso{\isacharunderscore}{\kern0pt}separation{\isacharunderscore}{\kern0pt}fm{\isacharcomma}{\kern0pt}\ {\isacharbrackleft}{\kern0pt}x{\isacharbrackright}{\kern0pt}\ {\isacharat}{\kern0pt}\ {\isacharbrackleft}{\kern0pt}A{\isacharcomma}{\kern0pt}\ f{\isacharcomma}{\kern0pt}\ r{\isacharbrackright}{\kern0pt}{\isacharparenright}{\kern0pt}{\isacharparenright}{\kern0pt}{\isachardoublequoteclose}\ \isakeyword{in}\ iffD{\isadigit{1}}{\isacharparenright}{\kern0pt}\isanewline
\ \ \isacommand{apply}\isamarkupfalse%
{\isacharparenleft}{\kern0pt}rule\ separation{\isacharunderscore}{\kern0pt}cong{\isacharparenright}{\kern0pt}\isanewline
\ \ \isacommand{unfolding}\isamarkupfalse%
\ well{\isacharunderscore}{\kern0pt}ord{\isacharunderscore}{\kern0pt}iso{\isacharunderscore}{\kern0pt}separation{\isacharunderscore}{\kern0pt}fm{\isacharunderscore}{\kern0pt}def\isanewline
\ \ \ \isacommand{apply}\isamarkupfalse%
{\isacharparenleft}{\kern0pt}rule\ iff{\isacharunderscore}{\kern0pt}flip{\isacharparenright}{\kern0pt}\isanewline
\ \ \isacommand{using}\isamarkupfalse%
\ well{\isacharunderscore}{\kern0pt}ord{\isacharunderscore}{\kern0pt}iso{\isacharunderscore}{\kern0pt}separation{\isacharunderscore}{\kern0pt}fm{\isacharunderscore}{\kern0pt}auto\ assms\isanewline
\ \ \ \isacommand{apply}\isamarkupfalse%
\ force\isanewline
\ \ \isacommand{apply}\isamarkupfalse%
{\isacharparenleft}{\kern0pt}rule\ separation{\isacharunderscore}{\kern0pt}ax{\isacharparenright}{\kern0pt}\isanewline
\ \ \isacommand{using}\isamarkupfalse%
\ assms\isanewline
\ \ \isacommand{apply}\isamarkupfalse%
\ auto{\isacharbrackleft}{\kern0pt}{\isadigit{2}}{\isacharbrackright}{\kern0pt}\isanewline
\ \ \isacommand{apply}\isamarkupfalse%
\ {\isacharparenleft}{\kern0pt}simp\ del{\isacharcolon}{\kern0pt}FOL{\isacharunderscore}{\kern0pt}sats{\isacharunderscore}{\kern0pt}iff\ pair{\isacharunderscore}{\kern0pt}abs\ add{\isacharcolon}{\kern0pt}\ fm{\isacharunderscore}{\kern0pt}defs\ nat{\isacharunderscore}{\kern0pt}simp{\isacharunderscore}{\kern0pt}union{\isacharparenright}{\kern0pt}\isanewline
\ \ \isacommand{done}\isamarkupfalse%
%
\endisatagproof
{\isafoldproof}%
%
\isadelimproof
\isanewline
%
\endisadelimproof
\isanewline
\isacommand{schematic{\isacharunderscore}{\kern0pt}goal}\isamarkupfalse%
\ obase{\isacharunderscore}{\kern0pt}separation{\isacharunderscore}{\kern0pt}fm{\isacharunderscore}{\kern0pt}auto{\isacharcolon}{\kern0pt}\isanewline
\ \ \isakeyword{assumes}\isanewline
\ \ \ \ {\isachardoublequoteopen}a\ {\isasymin}\ M{\isachardoublequoteclose}\ {\isachardoublequoteopen}A\ {\isasymin}\ M{\isachardoublequoteclose}\ {\isachardoublequoteopen}r\ {\isasymin}\ M{\isachardoublequoteclose}\ \isanewline
\ \ \isakeyword{shows}\isanewline
\ \ \ \ {\isachardoublequoteopen}{\isacharparenleft}{\kern0pt}{\isasymexists}x\ {\isasymin}\ M{\isachardot}{\kern0pt}\ {\isasymexists}g\ {\isasymin}\ M{\isachardot}{\kern0pt}\ {\isasymexists}mx\ {\isasymin}\ M{\isachardot}{\kern0pt}\ {\isasymexists}par\ {\isasymin}\ M{\isachardot}{\kern0pt}\isanewline
\ \ \ \ \ \ \ \ \ \ \ \ \ ordinal{\isacharparenleft}{\kern0pt}{\isacharhash}{\kern0pt}{\isacharhash}{\kern0pt}M{\isacharcomma}{\kern0pt}x{\isacharparenright}{\kern0pt}\ {\isasymand}\ membership{\isacharparenleft}{\kern0pt}{\isacharhash}{\kern0pt}{\isacharhash}{\kern0pt}M{\isacharcomma}{\kern0pt}x{\isacharcomma}{\kern0pt}mx{\isacharparenright}{\kern0pt}\ {\isasymand}\ pred{\isacharunderscore}{\kern0pt}set{\isacharparenleft}{\kern0pt}{\isacharhash}{\kern0pt}{\isacharhash}{\kern0pt}M{\isacharcomma}{\kern0pt}A{\isacharcomma}{\kern0pt}a{\isacharcomma}{\kern0pt}r{\isacharcomma}{\kern0pt}par{\isacharparenright}{\kern0pt}\ {\isasymand}\isanewline
\ \ \ \ \ \ \ \ \ \ \ \ \ order{\isacharunderscore}{\kern0pt}isomorphism{\isacharparenleft}{\kern0pt}{\isacharhash}{\kern0pt}{\isacharhash}{\kern0pt}M{\isacharcomma}{\kern0pt}par{\isacharcomma}{\kern0pt}r{\isacharcomma}{\kern0pt}x{\isacharcomma}{\kern0pt}mx{\isacharcomma}{\kern0pt}g{\isacharparenright}{\kern0pt}{\isacharparenright}{\kern0pt}\ {\isasymlongleftrightarrow}\ sats{\isacharparenleft}{\kern0pt}M{\isacharcomma}{\kern0pt}{\isacharquery}{\kern0pt}fm{\isacharcomma}{\kern0pt}{\isacharbrackleft}{\kern0pt}a{\isacharcomma}{\kern0pt}\ A{\isacharcomma}{\kern0pt}\ r{\isacharbrackright}{\kern0pt}{\isacharparenright}{\kern0pt}{\isachardoublequoteclose}\ \isanewline
%
\isadelimproof
\ \ %
\endisadelimproof
%
\isatagproof
\isacommand{by}\isamarkupfalse%
\ {\isacharparenleft}{\kern0pt}insert\ assms\ {\isacharsemicolon}{\kern0pt}\ {\isacharparenleft}{\kern0pt}rule\ sep{\isacharunderscore}{\kern0pt}rules\ {\isacharbar}{\kern0pt}\ simp{\isacharparenright}{\kern0pt}{\isacharplus}{\kern0pt}{\isacharparenright}{\kern0pt}%
\endisatagproof
{\isafoldproof}%
%
\isadelimproof
\ \isanewline
%
\endisadelimproof
\isanewline
\isacommand{definition}\isamarkupfalse%
\ obase{\isacharunderscore}{\kern0pt}separation{\isacharunderscore}{\kern0pt}fm\ \isakeyword{where}\ {\isachardoublequoteopen}obase{\isacharunderscore}{\kern0pt}separation{\isacharunderscore}{\kern0pt}fm\ {\isasymequiv}\ Exists\isanewline
\ \ \ \ \ \ \ \ \ \ \ \ \ {\isacharparenleft}{\kern0pt}Exists\isanewline
\ \ \ \ \ \ \ \ \ \ \ \ \ \ \ {\isacharparenleft}{\kern0pt}Exists\isanewline
\ \ \ \ \ \ \ \ \ \ \ \ \ \ \ \ \ {\isacharparenleft}{\kern0pt}Exists\isanewline
\ \ \ \ \ \ \ \ \ \ \ \ \ \ \ \ \ \ \ {\isacharparenleft}{\kern0pt}And{\isacharparenleft}{\kern0pt}ordinal{\isacharunderscore}{\kern0pt}fm{\isacharparenleft}{\kern0pt}{\isadigit{3}}{\isacharparenright}{\kern0pt}{\isacharcomma}{\kern0pt}\ And{\isacharparenleft}{\kern0pt}Memrel{\isacharunderscore}{\kern0pt}fm{\isacharparenleft}{\kern0pt}{\isadigit{3}}{\isacharcomma}{\kern0pt}\ {\isadigit{1}}{\isacharparenright}{\kern0pt}{\isacharcomma}{\kern0pt}\ And{\isacharparenleft}{\kern0pt}pred{\isacharunderscore}{\kern0pt}set{\isacharunderscore}{\kern0pt}fm{\isacharparenleft}{\kern0pt}{\isadigit{5}}{\isacharcomma}{\kern0pt}\ {\isadigit{4}}{\isacharcomma}{\kern0pt}\ {\isadigit{6}}{\isacharcomma}{\kern0pt}\ {\isadigit{0}}{\isacharparenright}{\kern0pt}{\isacharcomma}{\kern0pt}\ order{\isacharunderscore}{\kern0pt}isomorphism{\isacharunderscore}{\kern0pt}fm{\isacharparenleft}{\kern0pt}{\isadigit{0}}{\isacharcomma}{\kern0pt}\ {\isadigit{6}}{\isacharcomma}{\kern0pt}\ {\isadigit{3}}{\isacharcomma}{\kern0pt}\ {\isadigit{1}}{\isacharcomma}{\kern0pt}\ {\isadigit{2}}{\isacharparenright}{\kern0pt}{\isacharparenright}{\kern0pt}{\isacharparenright}{\kern0pt}{\isacharparenright}{\kern0pt}{\isacharparenright}{\kern0pt}{\isacharparenright}{\kern0pt}{\isacharparenright}{\kern0pt}{\isacharparenright}{\kern0pt}{\isachardoublequoteclose}\isanewline
\isanewline
\isacommand{lemma}\isamarkupfalse%
\ obase{\isacharunderscore}{\kern0pt}separation\ {\isacharcolon}{\kern0pt}\ \isanewline
\ \ \isakeyword{fixes}\ A\ r\ \isanewline
\ \ \isakeyword{assumes}\ {\isachardoublequoteopen}A\ {\isasymin}\ M{\isachardoublequoteclose}\ {\isachardoublequoteopen}r\ {\isasymin}\ M{\isachardoublequoteclose}\ \isanewline
\ \ \isakeyword{shows}\ {\isachardoublequoteopen}separation{\isacharparenleft}{\kern0pt}{\isacharhash}{\kern0pt}{\isacharhash}{\kern0pt}M{\isacharcomma}{\kern0pt}\ {\isasymlambda}a{\isachardot}{\kern0pt}\ {\isasymexists}x{\isacharbrackleft}{\kern0pt}{\isacharhash}{\kern0pt}{\isacharhash}{\kern0pt}M{\isacharbrackright}{\kern0pt}{\isachardot}{\kern0pt}\ {\isasymexists}g{\isacharbrackleft}{\kern0pt}{\isacharhash}{\kern0pt}{\isacharhash}{\kern0pt}M{\isacharbrackright}{\kern0pt}{\isachardot}{\kern0pt}\ {\isasymexists}mx{\isacharbrackleft}{\kern0pt}{\isacharhash}{\kern0pt}{\isacharhash}{\kern0pt}M{\isacharbrackright}{\kern0pt}{\isachardot}{\kern0pt}\ {\isasymexists}par{\isacharbrackleft}{\kern0pt}{\isacharhash}{\kern0pt}{\isacharhash}{\kern0pt}M{\isacharbrackright}{\kern0pt}{\isachardot}{\kern0pt}\isanewline
\ \ \ \ \ \ \ \ \ \ \ \ \ ordinal{\isacharparenleft}{\kern0pt}{\isacharhash}{\kern0pt}{\isacharhash}{\kern0pt}M{\isacharcomma}{\kern0pt}x{\isacharparenright}{\kern0pt}\ {\isasymand}\ membership{\isacharparenleft}{\kern0pt}{\isacharhash}{\kern0pt}{\isacharhash}{\kern0pt}M{\isacharcomma}{\kern0pt}x{\isacharcomma}{\kern0pt}mx{\isacharparenright}{\kern0pt}\ {\isasymand}\ pred{\isacharunderscore}{\kern0pt}set{\isacharparenleft}{\kern0pt}{\isacharhash}{\kern0pt}{\isacharhash}{\kern0pt}M{\isacharcomma}{\kern0pt}A{\isacharcomma}{\kern0pt}a{\isacharcomma}{\kern0pt}r{\isacharcomma}{\kern0pt}par{\isacharparenright}{\kern0pt}\ {\isasymand}\isanewline
\ \ \ \ \ \ \ \ \ \ \ \ \ order{\isacharunderscore}{\kern0pt}isomorphism{\isacharparenleft}{\kern0pt}{\isacharhash}{\kern0pt}{\isacharhash}{\kern0pt}M{\isacharcomma}{\kern0pt}par{\isacharcomma}{\kern0pt}r{\isacharcomma}{\kern0pt}x{\isacharcomma}{\kern0pt}mx{\isacharcomma}{\kern0pt}g{\isacharparenright}{\kern0pt}{\isacharparenright}{\kern0pt}{\isachardoublequoteclose}\isanewline
%
\isadelimproof
\isanewline
\ \ %
\endisadelimproof
%
\isatagproof
\isacommand{apply}\isamarkupfalse%
{\isacharparenleft}{\kern0pt}rule{\isacharunderscore}{\kern0pt}tac\ P{\isacharequal}{\kern0pt}{\isachardoublequoteopen}separation{\isacharparenleft}{\kern0pt}{\isacharhash}{\kern0pt}{\isacharhash}{\kern0pt}M{\isacharcomma}{\kern0pt}\ {\isasymlambda}a{\isachardot}{\kern0pt}\ {\isacharparenleft}{\kern0pt}{\isasymexists}x\ {\isasymin}\ M{\isachardot}{\kern0pt}\ {\isasymexists}g\ {\isasymin}\ M{\isachardot}{\kern0pt}\ {\isasymexists}mx\ {\isasymin}\ M{\isachardot}{\kern0pt}\ {\isasymexists}par\ {\isasymin}\ M{\isachardot}{\kern0pt}\isanewline
\ \ \ \ \ \ \ \ \ \ \ \ \ ordinal{\isacharparenleft}{\kern0pt}{\isacharhash}{\kern0pt}{\isacharhash}{\kern0pt}M{\isacharcomma}{\kern0pt}x{\isacharparenright}{\kern0pt}\ {\isasymand}\ membership{\isacharparenleft}{\kern0pt}{\isacharhash}{\kern0pt}{\isacharhash}{\kern0pt}M{\isacharcomma}{\kern0pt}x{\isacharcomma}{\kern0pt}mx{\isacharparenright}{\kern0pt}\ {\isasymand}\ pred{\isacharunderscore}{\kern0pt}set{\isacharparenleft}{\kern0pt}{\isacharhash}{\kern0pt}{\isacharhash}{\kern0pt}M{\isacharcomma}{\kern0pt}A{\isacharcomma}{\kern0pt}a{\isacharcomma}{\kern0pt}r{\isacharcomma}{\kern0pt}par{\isacharparenright}{\kern0pt}\ {\isasymand}\isanewline
\ \ \ \ \ \ \ \ \ \ \ \ \ order{\isacharunderscore}{\kern0pt}isomorphism{\isacharparenleft}{\kern0pt}{\isacharhash}{\kern0pt}{\isacharhash}{\kern0pt}M{\isacharcomma}{\kern0pt}par{\isacharcomma}{\kern0pt}r{\isacharcomma}{\kern0pt}x{\isacharcomma}{\kern0pt}mx{\isacharcomma}{\kern0pt}g{\isacharparenright}{\kern0pt}{\isacharparenright}{\kern0pt}{\isacharparenright}{\kern0pt}{\isachardoublequoteclose}\ \isakeyword{in}\ iffD{\isadigit{1}}{\isacharparenright}{\kern0pt}\isanewline
\ \ \ \isacommand{apply}\isamarkupfalse%
{\isacharparenleft}{\kern0pt}rule\ separation{\isacharunderscore}{\kern0pt}cong{\isacharparenright}{\kern0pt}\isanewline
\ \ \isacommand{using}\isamarkupfalse%
\ assms\isanewline
\ \ \ \isacommand{apply}\isamarkupfalse%
\ force\isanewline
\ \ \isacommand{apply}\isamarkupfalse%
{\isacharparenleft}{\kern0pt}rule{\isacharunderscore}{\kern0pt}tac\ P{\isacharequal}{\kern0pt}{\isachardoublequoteopen}separation{\isacharparenleft}{\kern0pt}{\isacharhash}{\kern0pt}{\isacharhash}{\kern0pt}M{\isacharcomma}{\kern0pt}\ {\isasymlambda}a{\isachardot}{\kern0pt}\ sats{\isacharparenleft}{\kern0pt}M{\isacharcomma}{\kern0pt}\ obase{\isacharunderscore}{\kern0pt}separation{\isacharunderscore}{\kern0pt}fm{\isacharcomma}{\kern0pt}\ {\isacharbrackleft}{\kern0pt}a{\isacharbrackright}{\kern0pt}\ {\isacharat}{\kern0pt}\ {\isacharbrackleft}{\kern0pt}A{\isacharcomma}{\kern0pt}\ r{\isacharbrackright}{\kern0pt}{\isacharparenright}{\kern0pt}{\isacharparenright}{\kern0pt}{\isachardoublequoteclose}\ \isakeyword{in}\ iffD{\isadigit{1}}{\isacharparenright}{\kern0pt}\isanewline
\ \ \isacommand{apply}\isamarkupfalse%
{\isacharparenleft}{\kern0pt}rule\ separation{\isacharunderscore}{\kern0pt}cong{\isacharparenright}{\kern0pt}\isanewline
\ \ \isacommand{unfolding}\isamarkupfalse%
\ obase{\isacharunderscore}{\kern0pt}separation{\isacharunderscore}{\kern0pt}fm{\isacharunderscore}{\kern0pt}def\isanewline
\ \ \ \isacommand{apply}\isamarkupfalse%
{\isacharparenleft}{\kern0pt}rule\ iff{\isacharunderscore}{\kern0pt}flip{\isacharparenright}{\kern0pt}\isanewline
\ \ \isacommand{using}\isamarkupfalse%
\ obase{\isacharunderscore}{\kern0pt}separation{\isacharunderscore}{\kern0pt}fm{\isacharunderscore}{\kern0pt}auto\ assms\isanewline
\ \ \ \isacommand{apply}\isamarkupfalse%
\ force\isanewline
\ \ \isacommand{apply}\isamarkupfalse%
{\isacharparenleft}{\kern0pt}rule\ separation{\isacharunderscore}{\kern0pt}ax{\isacharparenright}{\kern0pt}\isanewline
\ \ \isacommand{using}\isamarkupfalse%
\ assms\isanewline
\ \ \ \ \isacommand{apply}\isamarkupfalse%
\ auto{\isacharbrackleft}{\kern0pt}{\isadigit{2}}{\isacharbrackright}{\kern0pt}\isanewline
\ \ \isacommand{unfolding}\isamarkupfalse%
\ order{\isacharunderscore}{\kern0pt}isomorphism{\isacharunderscore}{\kern0pt}fm{\isacharunderscore}{\kern0pt}def\ pred{\isacharunderscore}{\kern0pt}set{\isacharunderscore}{\kern0pt}fm{\isacharunderscore}{\kern0pt}def\ Memrel{\isacharunderscore}{\kern0pt}fm{\isacharunderscore}{\kern0pt}def\ bijection{\isacharunderscore}{\kern0pt}fm{\isacharunderscore}{\kern0pt}def\isanewline
\ \ \isacommand{apply}\isamarkupfalse%
\ {\isacharparenleft}{\kern0pt}simp\ del{\isacharcolon}{\kern0pt}FOL{\isacharunderscore}{\kern0pt}sats{\isacharunderscore}{\kern0pt}iff\ pair{\isacharunderscore}{\kern0pt}abs\ add{\isacharcolon}{\kern0pt}\ fm{\isacharunderscore}{\kern0pt}defs\ nat{\isacharunderscore}{\kern0pt}simp{\isacharunderscore}{\kern0pt}union{\isacharparenright}{\kern0pt}\isanewline
\ \ \isacommand{unfolding}\isamarkupfalse%
\ injection{\isacharunderscore}{\kern0pt}fm{\isacharunderscore}{\kern0pt}def\ surjection{\isacharunderscore}{\kern0pt}fm{\isacharunderscore}{\kern0pt}def\isanewline
\ \ \isacommand{apply}\isamarkupfalse%
\ {\isacharparenleft}{\kern0pt}simp\ del{\isacharcolon}{\kern0pt}FOL{\isacharunderscore}{\kern0pt}sats{\isacharunderscore}{\kern0pt}iff\ pair{\isacharunderscore}{\kern0pt}abs\ add{\isacharcolon}{\kern0pt}\ fm{\isacharunderscore}{\kern0pt}defs\ nat{\isacharunderscore}{\kern0pt}simp{\isacharunderscore}{\kern0pt}union{\isacharparenright}{\kern0pt}\isanewline
\ \ \isacommand{done}\isamarkupfalse%
%
\endisatagproof
{\isafoldproof}%
%
\isadelimproof
\isanewline
%
\endisadelimproof
\isanewline
\isanewline
\isacommand{schematic{\isacharunderscore}{\kern0pt}goal}\isamarkupfalse%
\ obase{\isacharunderscore}{\kern0pt}equals{\isacharunderscore}{\kern0pt}separation{\isacharunderscore}{\kern0pt}fm{\isacharunderscore}{\kern0pt}auto{\isacharcolon}{\kern0pt}\isanewline
\ \ \isakeyword{assumes}\isanewline
\ \ \ \ {\isachardoublequoteopen}x\ {\isasymin}\ M{\isachardoublequoteclose}\ {\isachardoublequoteopen}A\ {\isasymin}\ M{\isachardoublequoteclose}\ {\isachardoublequoteopen}r\ {\isasymin}\ M{\isachardoublequoteclose}\ \isanewline
\ \ \isakeyword{shows}\isanewline
\ \ \ \ {\isachardoublequoteopen}{\isacharparenleft}{\kern0pt}x{\isasymin}A\ {\isasymlongrightarrow}\ {\isasymnot}{\isacharparenleft}{\kern0pt}{\isasymexists}y\ {\isasymin}\ M{\isachardot}{\kern0pt}\ {\isasymexists}g\ {\isasymin}\ M{\isachardot}{\kern0pt}\isanewline
\ \ \ \ \ \ \ \ ordinal{\isacharparenleft}{\kern0pt}{\isacharhash}{\kern0pt}{\isacharhash}{\kern0pt}M{\isacharcomma}{\kern0pt}y{\isacharparenright}{\kern0pt}\ {\isasymand}\ {\isacharparenleft}{\kern0pt}{\isasymexists}my\ {\isasymin}\ M{\isachardot}{\kern0pt}\ {\isasymexists}pxr\ {\isasymin}\ M{\isachardot}{\kern0pt}\isanewline
\ \ \ \ \ \ \ \ membership{\isacharparenleft}{\kern0pt}{\isacharhash}{\kern0pt}{\isacharhash}{\kern0pt}M{\isacharcomma}{\kern0pt}y{\isacharcomma}{\kern0pt}my{\isacharparenright}{\kern0pt}\ {\isasymand}\ pred{\isacharunderscore}{\kern0pt}set{\isacharparenleft}{\kern0pt}{\isacharhash}{\kern0pt}{\isacharhash}{\kern0pt}M{\isacharcomma}{\kern0pt}A{\isacharcomma}{\kern0pt}x{\isacharcomma}{\kern0pt}r{\isacharcomma}{\kern0pt}pxr{\isacharparenright}{\kern0pt}\ {\isasymand}\isanewline
\ \ \ \ \ \ \ \ order{\isacharunderscore}{\kern0pt}isomorphism{\isacharparenleft}{\kern0pt}{\isacharhash}{\kern0pt}{\isacharhash}{\kern0pt}M{\isacharcomma}{\kern0pt}pxr{\isacharcomma}{\kern0pt}r{\isacharcomma}{\kern0pt}y{\isacharcomma}{\kern0pt}my{\isacharcomma}{\kern0pt}g{\isacharparenright}{\kern0pt}{\isacharparenright}{\kern0pt}{\isacharparenright}{\kern0pt}{\isacharparenright}{\kern0pt}\ {\isasymlongleftrightarrow}\ sats{\isacharparenleft}{\kern0pt}M{\isacharcomma}{\kern0pt}{\isacharquery}{\kern0pt}fm{\isacharcomma}{\kern0pt}{\isacharbrackleft}{\kern0pt}x{\isacharcomma}{\kern0pt}\ A{\isacharcomma}{\kern0pt}\ r{\isacharbrackright}{\kern0pt}{\isacharparenright}{\kern0pt}{\isachardoublequoteclose}\ \isanewline
%
\isadelimproof
\ \ %
\endisadelimproof
%
\isatagproof
\isacommand{by}\isamarkupfalse%
\ {\isacharparenleft}{\kern0pt}insert\ assms\ {\isacharsemicolon}{\kern0pt}\ {\isacharparenleft}{\kern0pt}rule\ sep{\isacharunderscore}{\kern0pt}rules\ {\isacharbar}{\kern0pt}\ simp{\isacharparenright}{\kern0pt}{\isacharplus}{\kern0pt}{\isacharparenright}{\kern0pt}%
\endisatagproof
{\isafoldproof}%
%
\isadelimproof
\ \isanewline
%
\endisadelimproof
\isanewline
\isacommand{definition}\isamarkupfalse%
\ obase{\isacharunderscore}{\kern0pt}equals{\isacharunderscore}{\kern0pt}separation{\isacharunderscore}{\kern0pt}fm\ \isakeyword{where}\ {\isachardoublequoteopen}obase{\isacharunderscore}{\kern0pt}equals{\isacharunderscore}{\kern0pt}separation{\isacharunderscore}{\kern0pt}fm\ {\isasymequiv}\ \isanewline
\ \ \ \ \ \ \ \ \ \ Implies\isanewline
\ \ \ \ \ \ \ \ \ \ \ \ \ {\isacharparenleft}{\kern0pt}Member{\isacharparenleft}{\kern0pt}{\isadigit{0}}{\isacharcomma}{\kern0pt}\ {\isadigit{1}}{\isacharparenright}{\kern0pt}{\isacharcomma}{\kern0pt}\isanewline
\ \ \ \ \ \ \ \ \ \ \ \ \ \ Neg{\isacharparenleft}{\kern0pt}Exists\isanewline
\ \ \ \ \ \ \ \ \ \ \ \ \ \ \ \ \ \ \ {\isacharparenleft}{\kern0pt}Exists\isanewline
\ \ \ \ \ \ \ \ \ \ \ \ \ \ \ \ \ \ \ \ \ {\isacharparenleft}{\kern0pt}And{\isacharparenleft}{\kern0pt}ordinal{\isacharunderscore}{\kern0pt}fm{\isacharparenleft}{\kern0pt}{\isadigit{1}}{\isacharparenright}{\kern0pt}{\isacharcomma}{\kern0pt}\isanewline
\ \ \ \ \ \ \ \ \ \ \ \ \ \ \ \ \ \ \ \ \ \ \ \ \ \ Exists{\isacharparenleft}{\kern0pt}Exists{\isacharparenleft}{\kern0pt}And{\isacharparenleft}{\kern0pt}Memrel{\isacharunderscore}{\kern0pt}fm{\isacharparenleft}{\kern0pt}{\isadigit{3}}{\isacharcomma}{\kern0pt}\ {\isadigit{1}}{\isacharparenright}{\kern0pt}{\isacharcomma}{\kern0pt}\ And{\isacharparenleft}{\kern0pt}pred{\isacharunderscore}{\kern0pt}set{\isacharunderscore}{\kern0pt}fm{\isacharparenleft}{\kern0pt}{\isadigit{5}}{\isacharcomma}{\kern0pt}\ {\isadigit{4}}{\isacharcomma}{\kern0pt}\ {\isadigit{6}}{\isacharcomma}{\kern0pt}\ {\isadigit{0}}{\isacharparenright}{\kern0pt}{\isacharcomma}{\kern0pt}\ order{\isacharunderscore}{\kern0pt}isomorphism{\isacharunderscore}{\kern0pt}fm{\isacharparenleft}{\kern0pt}{\isadigit{0}}{\isacharcomma}{\kern0pt}\ {\isadigit{6}}{\isacharcomma}{\kern0pt}\ {\isadigit{3}}{\isacharcomma}{\kern0pt}\ {\isadigit{1}}{\isacharcomma}{\kern0pt}\ {\isadigit{2}}{\isacharparenright}{\kern0pt}{\isacharparenright}{\kern0pt}{\isacharparenright}{\kern0pt}{\isacharparenright}{\kern0pt}{\isacharparenright}{\kern0pt}{\isacharparenright}{\kern0pt}{\isacharparenright}{\kern0pt}{\isacharparenright}{\kern0pt}{\isacharparenright}{\kern0pt}{\isacharparenright}{\kern0pt}{\isachardoublequoteclose}\isanewline
\isanewline
\isacommand{lemma}\isamarkupfalse%
\ obase{\isacharunderscore}{\kern0pt}equals{\isacharunderscore}{\kern0pt}separation\ {\isacharcolon}{\kern0pt}\ \isanewline
\ \ \isakeyword{fixes}\ A\ r\ \isanewline
\ \ \isakeyword{assumes}\ {\isachardoublequoteopen}A\ {\isasymin}\ M{\isachardoublequoteclose}\ {\isachardoublequoteopen}r\ {\isasymin}\ M{\isachardoublequoteclose}\ \isanewline
\ \ \isakeyword{shows}\ {\isachardoublequoteopen}separation\ {\isacharparenleft}{\kern0pt}{\isacharhash}{\kern0pt}{\isacharhash}{\kern0pt}M{\isacharcomma}{\kern0pt}\ {\isasymlambda}x{\isachardot}{\kern0pt}\ x{\isasymin}A\ {\isasymlongrightarrow}\ {\isasymnot}{\isacharparenleft}{\kern0pt}{\isasymexists}y{\isacharbrackleft}{\kern0pt}{\isacharhash}{\kern0pt}{\isacharhash}{\kern0pt}M{\isacharbrackright}{\kern0pt}{\isachardot}{\kern0pt}\ {\isasymexists}g{\isacharbrackleft}{\kern0pt}{\isacharhash}{\kern0pt}{\isacharhash}{\kern0pt}M{\isacharbrackright}{\kern0pt}{\isachardot}{\kern0pt}\isanewline
\ \ \ \ \ \ \ \ \ \ \ \ \ \ \ \ \ \ \ \ \ \ \ \ \ \ \ \ \ \ ordinal{\isacharparenleft}{\kern0pt}{\isacharhash}{\kern0pt}{\isacharhash}{\kern0pt}M{\isacharcomma}{\kern0pt}y{\isacharparenright}{\kern0pt}\ {\isasymand}\ {\isacharparenleft}{\kern0pt}{\isasymexists}my{\isacharbrackleft}{\kern0pt}{\isacharhash}{\kern0pt}{\isacharhash}{\kern0pt}M{\isacharbrackright}{\kern0pt}{\isachardot}{\kern0pt}\ {\isasymexists}pxr{\isacharbrackleft}{\kern0pt}{\isacharhash}{\kern0pt}{\isacharhash}{\kern0pt}M{\isacharbrackright}{\kern0pt}{\isachardot}{\kern0pt}\isanewline
\ \ \ \ \ \ \ \ \ \ \ \ \ \ \ \ \ \ \ \ \ \ \ \ \ \ \ \ \ \ membership{\isacharparenleft}{\kern0pt}{\isacharhash}{\kern0pt}{\isacharhash}{\kern0pt}M{\isacharcomma}{\kern0pt}y{\isacharcomma}{\kern0pt}my{\isacharparenright}{\kern0pt}\ {\isasymand}\ pred{\isacharunderscore}{\kern0pt}set{\isacharparenleft}{\kern0pt}{\isacharhash}{\kern0pt}{\isacharhash}{\kern0pt}M{\isacharcomma}{\kern0pt}A{\isacharcomma}{\kern0pt}x{\isacharcomma}{\kern0pt}r{\isacharcomma}{\kern0pt}pxr{\isacharparenright}{\kern0pt}\ {\isasymand}\isanewline
\ \ \ \ \ \ \ \ \ \ \ \ \ \ \ \ \ \ \ \ \ \ \ \ \ \ \ \ \ \ order{\isacharunderscore}{\kern0pt}isomorphism{\isacharparenleft}{\kern0pt}{\isacharhash}{\kern0pt}{\isacharhash}{\kern0pt}M{\isacharcomma}{\kern0pt}pxr{\isacharcomma}{\kern0pt}r{\isacharcomma}{\kern0pt}y{\isacharcomma}{\kern0pt}my{\isacharcomma}{\kern0pt}g{\isacharparenright}{\kern0pt}{\isacharparenright}{\kern0pt}{\isacharparenright}{\kern0pt}{\isacharparenright}{\kern0pt}{\isachardoublequoteclose}\isanewline
%
\isadelimproof
\isanewline
\ \ %
\endisadelimproof
%
\isatagproof
\isacommand{apply}\isamarkupfalse%
{\isacharparenleft}{\kern0pt}rule{\isacharunderscore}{\kern0pt}tac\ P{\isacharequal}{\kern0pt}{\isachardoublequoteopen}separation{\isacharparenleft}{\kern0pt}{\isacharhash}{\kern0pt}{\isacharhash}{\kern0pt}M{\isacharcomma}{\kern0pt}\ {\isasymlambda}x{\isachardot}{\kern0pt}\ {\isacharparenleft}{\kern0pt}x{\isasymin}A\ {\isasymlongrightarrow}\ {\isasymnot}{\isacharparenleft}{\kern0pt}{\isasymexists}y\ {\isasymin}\ M{\isachardot}{\kern0pt}\ {\isasymexists}g\ {\isasymin}\ M{\isachardot}{\kern0pt}\isanewline
\ \ \ \ \ \ \ \ ordinal{\isacharparenleft}{\kern0pt}{\isacharhash}{\kern0pt}{\isacharhash}{\kern0pt}M{\isacharcomma}{\kern0pt}y{\isacharparenright}{\kern0pt}\ {\isasymand}\ {\isacharparenleft}{\kern0pt}{\isasymexists}my\ {\isasymin}\ M{\isachardot}{\kern0pt}\ {\isasymexists}pxr\ {\isasymin}\ M{\isachardot}{\kern0pt}\isanewline
\ \ \ \ \ \ \ \ membership{\isacharparenleft}{\kern0pt}{\isacharhash}{\kern0pt}{\isacharhash}{\kern0pt}M{\isacharcomma}{\kern0pt}y{\isacharcomma}{\kern0pt}my{\isacharparenright}{\kern0pt}\ {\isasymand}\ pred{\isacharunderscore}{\kern0pt}set{\isacharparenleft}{\kern0pt}{\isacharhash}{\kern0pt}{\isacharhash}{\kern0pt}M{\isacharcomma}{\kern0pt}A{\isacharcomma}{\kern0pt}x{\isacharcomma}{\kern0pt}r{\isacharcomma}{\kern0pt}pxr{\isacharparenright}{\kern0pt}\ {\isasymand}\isanewline
\ \ \ \ \ \ \ \ order{\isacharunderscore}{\kern0pt}isomorphism{\isacharparenleft}{\kern0pt}{\isacharhash}{\kern0pt}{\isacharhash}{\kern0pt}M{\isacharcomma}{\kern0pt}pxr{\isacharcomma}{\kern0pt}r{\isacharcomma}{\kern0pt}y{\isacharcomma}{\kern0pt}my{\isacharcomma}{\kern0pt}g{\isacharparenright}{\kern0pt}{\isacharparenright}{\kern0pt}{\isacharparenright}{\kern0pt}{\isacharparenright}{\kern0pt}{\isacharparenright}{\kern0pt}{\isachardoublequoteclose}\ \isakeyword{in}\ iffD{\isadigit{1}}{\isacharparenright}{\kern0pt}\isanewline
\ \ \ \isacommand{apply}\isamarkupfalse%
{\isacharparenleft}{\kern0pt}rule\ separation{\isacharunderscore}{\kern0pt}cong{\isacharparenright}{\kern0pt}\isanewline
\ \ \isacommand{using}\isamarkupfalse%
\ assms\isanewline
\ \ \ \isacommand{apply}\isamarkupfalse%
\ force\isanewline
\ \ \isacommand{apply}\isamarkupfalse%
{\isacharparenleft}{\kern0pt}rule{\isacharunderscore}{\kern0pt}tac\ P{\isacharequal}{\kern0pt}{\isachardoublequoteopen}separation{\isacharparenleft}{\kern0pt}{\isacharhash}{\kern0pt}{\isacharhash}{\kern0pt}M{\isacharcomma}{\kern0pt}\ {\isasymlambda}a{\isachardot}{\kern0pt}\ sats{\isacharparenleft}{\kern0pt}M{\isacharcomma}{\kern0pt}\ obase{\isacharunderscore}{\kern0pt}equals{\isacharunderscore}{\kern0pt}separation{\isacharunderscore}{\kern0pt}fm{\isacharcomma}{\kern0pt}\ {\isacharbrackleft}{\kern0pt}a{\isacharbrackright}{\kern0pt}\ {\isacharat}{\kern0pt}\ {\isacharbrackleft}{\kern0pt}A{\isacharcomma}{\kern0pt}\ r{\isacharbrackright}{\kern0pt}{\isacharparenright}{\kern0pt}{\isacharparenright}{\kern0pt}{\isachardoublequoteclose}\ \isakeyword{in}\ iffD{\isadigit{1}}{\isacharparenright}{\kern0pt}\isanewline
\ \ \isacommand{apply}\isamarkupfalse%
{\isacharparenleft}{\kern0pt}rule\ separation{\isacharunderscore}{\kern0pt}cong{\isacharparenright}{\kern0pt}\isanewline
\ \ \isacommand{unfolding}\isamarkupfalse%
\ obase{\isacharunderscore}{\kern0pt}equals{\isacharunderscore}{\kern0pt}separation{\isacharunderscore}{\kern0pt}fm{\isacharunderscore}{\kern0pt}def\isanewline
\ \ \ \isacommand{apply}\isamarkupfalse%
{\isacharparenleft}{\kern0pt}rule\ iff{\isacharunderscore}{\kern0pt}flip{\isacharparenright}{\kern0pt}\isanewline
\ \ \isacommand{using}\isamarkupfalse%
\ obase{\isacharunderscore}{\kern0pt}equals{\isacharunderscore}{\kern0pt}separation{\isacharunderscore}{\kern0pt}fm{\isacharunderscore}{\kern0pt}auto\ assms\isanewline
\ \ \ \isacommand{apply}\isamarkupfalse%
\ force\isanewline
\ \ \isacommand{apply}\isamarkupfalse%
{\isacharparenleft}{\kern0pt}rule\ separation{\isacharunderscore}{\kern0pt}ax{\isacharparenright}{\kern0pt}\isanewline
\ \ \isacommand{using}\isamarkupfalse%
\ assms\isanewline
\ \ \ \ \isacommand{apply}\isamarkupfalse%
\ auto{\isacharbrackleft}{\kern0pt}{\isadigit{2}}{\isacharbrackright}{\kern0pt}\isanewline
\ \ \isacommand{unfolding}\isamarkupfalse%
\ order{\isacharunderscore}{\kern0pt}isomorphism{\isacharunderscore}{\kern0pt}fm{\isacharunderscore}{\kern0pt}def\ pred{\isacharunderscore}{\kern0pt}set{\isacharunderscore}{\kern0pt}fm{\isacharunderscore}{\kern0pt}def\ Memrel{\isacharunderscore}{\kern0pt}fm{\isacharunderscore}{\kern0pt}def\ bijection{\isacharunderscore}{\kern0pt}fm{\isacharunderscore}{\kern0pt}def\isanewline
\ \ \isacommand{apply}\isamarkupfalse%
\ {\isacharparenleft}{\kern0pt}simp\ del{\isacharcolon}{\kern0pt}FOL{\isacharunderscore}{\kern0pt}sats{\isacharunderscore}{\kern0pt}iff\ pair{\isacharunderscore}{\kern0pt}abs\ add{\isacharcolon}{\kern0pt}\ fm{\isacharunderscore}{\kern0pt}defs\ nat{\isacharunderscore}{\kern0pt}simp{\isacharunderscore}{\kern0pt}union{\isacharparenright}{\kern0pt}\isanewline
\ \ \isacommand{unfolding}\isamarkupfalse%
\ injection{\isacharunderscore}{\kern0pt}fm{\isacharunderscore}{\kern0pt}def\ surjection{\isacharunderscore}{\kern0pt}fm{\isacharunderscore}{\kern0pt}def\isanewline
\ \ \isacommand{apply}\isamarkupfalse%
\ {\isacharparenleft}{\kern0pt}simp\ del{\isacharcolon}{\kern0pt}FOL{\isacharunderscore}{\kern0pt}sats{\isacharunderscore}{\kern0pt}iff\ pair{\isacharunderscore}{\kern0pt}abs\ add{\isacharcolon}{\kern0pt}\ fm{\isacharunderscore}{\kern0pt}defs\ nat{\isacharunderscore}{\kern0pt}simp{\isacharunderscore}{\kern0pt}union{\isacharparenright}{\kern0pt}\isanewline
\ \ \isacommand{done}\isamarkupfalse%
%
\endisatagproof
{\isafoldproof}%
%
\isadelimproof
\isanewline
%
\endisadelimproof
\isanewline
\isacommand{schematic{\isacharunderscore}{\kern0pt}goal}\isamarkupfalse%
\ omap{\isacharunderscore}{\kern0pt}replacement{\isacharunderscore}{\kern0pt}fm{\isacharunderscore}{\kern0pt}auto{\isacharcolon}{\kern0pt}\isanewline
\ \ \isakeyword{assumes}\isanewline
\ \ \ \ {\isachardoublequoteopen}a\ {\isasymin}\ M{\isachardoublequoteclose}\ {\isachardoublequoteopen}z\ {\isasymin}\ M{\isachardoublequoteclose}\ {\isachardoublequoteopen}A\ {\isasymin}\ M{\isachardoublequoteclose}\ {\isachardoublequoteopen}r\ {\isasymin}\ M{\isachardoublequoteclose}\ \isanewline
\ \ \isakeyword{shows}\isanewline
\ \ \ \ {\isachardoublequoteopen}{\isacharparenleft}{\kern0pt}{\isasymexists}x\ {\isasymin}\ M{\isachardot}{\kern0pt}\ {\isasymexists}g\ {\isasymin}\ M{\isachardot}{\kern0pt}\ {\isasymexists}mx\ {\isasymin}\ M{\isachardot}{\kern0pt}\ {\isasymexists}par\ {\isasymin}\ M{\isachardot}{\kern0pt}\isanewline
\ \ \ \ \ \ \ \ \ \ \ \ \ ordinal{\isacharparenleft}{\kern0pt}{\isacharhash}{\kern0pt}{\isacharhash}{\kern0pt}M{\isacharcomma}{\kern0pt}x{\isacharparenright}{\kern0pt}\ {\isasymand}\ pair{\isacharparenleft}{\kern0pt}{\isacharhash}{\kern0pt}{\isacharhash}{\kern0pt}M{\isacharcomma}{\kern0pt}a{\isacharcomma}{\kern0pt}x{\isacharcomma}{\kern0pt}z{\isacharparenright}{\kern0pt}\ {\isasymand}\ membership{\isacharparenleft}{\kern0pt}{\isacharhash}{\kern0pt}{\isacharhash}{\kern0pt}M{\isacharcomma}{\kern0pt}x{\isacharcomma}{\kern0pt}mx{\isacharparenright}{\kern0pt}\ {\isasymand}\isanewline
\ \ \ \ \ \ \ \ \ \ \ \ \ pred{\isacharunderscore}{\kern0pt}set{\isacharparenleft}{\kern0pt}{\isacharhash}{\kern0pt}{\isacharhash}{\kern0pt}M{\isacharcomma}{\kern0pt}A{\isacharcomma}{\kern0pt}a{\isacharcomma}{\kern0pt}r{\isacharcomma}{\kern0pt}par{\isacharparenright}{\kern0pt}\ {\isasymand}\ order{\isacharunderscore}{\kern0pt}isomorphism{\isacharparenleft}{\kern0pt}{\isacharhash}{\kern0pt}{\isacharhash}{\kern0pt}M{\isacharcomma}{\kern0pt}par{\isacharcomma}{\kern0pt}r{\isacharcomma}{\kern0pt}x{\isacharcomma}{\kern0pt}mx{\isacharcomma}{\kern0pt}g{\isacharparenright}{\kern0pt}{\isacharparenright}{\kern0pt}\ {\isasymlongleftrightarrow}\ sats{\isacharparenleft}{\kern0pt}M{\isacharcomma}{\kern0pt}{\isacharquery}{\kern0pt}fm{\isacharcomma}{\kern0pt}{\isacharbrackleft}{\kern0pt}a{\isacharcomma}{\kern0pt}\ z{\isacharcomma}{\kern0pt}\ A{\isacharcomma}{\kern0pt}\ r{\isacharbrackright}{\kern0pt}{\isacharparenright}{\kern0pt}{\isachardoublequoteclose}\ \isanewline
%
\isadelimproof
\ \ %
\endisadelimproof
%
\isatagproof
\isacommand{by}\isamarkupfalse%
\ {\isacharparenleft}{\kern0pt}insert\ assms\ {\isacharsemicolon}{\kern0pt}\ {\isacharparenleft}{\kern0pt}rule\ sep{\isacharunderscore}{\kern0pt}rules\ {\isacharbar}{\kern0pt}\ simp{\isacharparenright}{\kern0pt}{\isacharplus}{\kern0pt}{\isacharparenright}{\kern0pt}%
\endisatagproof
{\isafoldproof}%
%
\isadelimproof
\ \isanewline
%
\endisadelimproof
\isanewline
\isacommand{definition}\isamarkupfalse%
\ omap{\isacharunderscore}{\kern0pt}replacement{\isacharunderscore}{\kern0pt}fm\ \isakeyword{where}\ {\isachardoublequoteopen}omap{\isacharunderscore}{\kern0pt}replacement{\isacharunderscore}{\kern0pt}fm\ {\isasymequiv}\ Exists\isanewline
\ \ \ \ \ \ \ \ \ \ \ \ \ {\isacharparenleft}{\kern0pt}Exists\isanewline
\ \ \ \ \ \ \ \ \ \ \ \ \ \ \ {\isacharparenleft}{\kern0pt}Exists\isanewline
\ \ \ \ \ \ \ \ \ \ \ \ \ \ \ \ \ {\isacharparenleft}{\kern0pt}Exists\isanewline
\ \ \ \ \ \ \ \ \ \ \ \ \ \ \ \ \ \ \ {\isacharparenleft}{\kern0pt}And{\isacharparenleft}{\kern0pt}ordinal{\isacharunderscore}{\kern0pt}fm{\isacharparenleft}{\kern0pt}{\isadigit{3}}{\isacharparenright}{\kern0pt}{\isacharcomma}{\kern0pt}\isanewline
\ \ \ \ \ \ \ \ \ \ \ \ \ \ \ \ \ \ \ \ \ \ \ \ And{\isacharparenleft}{\kern0pt}pair{\isacharunderscore}{\kern0pt}fm{\isacharparenleft}{\kern0pt}{\isadigit{4}}{\isacharcomma}{\kern0pt}\ {\isadigit{3}}{\isacharcomma}{\kern0pt}\ {\isadigit{5}}{\isacharparenright}{\kern0pt}{\isacharcomma}{\kern0pt}\isanewline
\ \ \ \ \ \ \ \ \ \ \ \ \ \ \ \ \ \ \ \ \ \ \ \ \ \ \ \ And{\isacharparenleft}{\kern0pt}Memrel{\isacharunderscore}{\kern0pt}fm{\isacharparenleft}{\kern0pt}{\isadigit{3}}{\isacharcomma}{\kern0pt}\ {\isadigit{1}}{\isacharparenright}{\kern0pt}{\isacharcomma}{\kern0pt}\ And{\isacharparenleft}{\kern0pt}pred{\isacharunderscore}{\kern0pt}set{\isacharunderscore}{\kern0pt}fm{\isacharparenleft}{\kern0pt}{\isadigit{6}}{\isacharcomma}{\kern0pt}\ {\isadigit{4}}{\isacharcomma}{\kern0pt}\ {\isadigit{7}}{\isacharcomma}{\kern0pt}\ {\isadigit{0}}{\isacharparenright}{\kern0pt}{\isacharcomma}{\kern0pt}\ order{\isacharunderscore}{\kern0pt}isomorphism{\isacharunderscore}{\kern0pt}fm{\isacharparenleft}{\kern0pt}{\isadigit{0}}{\isacharcomma}{\kern0pt}\ {\isadigit{7}}{\isacharcomma}{\kern0pt}\ {\isadigit{3}}{\isacharcomma}{\kern0pt}\ {\isadigit{1}}{\isacharcomma}{\kern0pt}\ {\isadigit{2}}{\isacharparenright}{\kern0pt}{\isacharparenright}{\kern0pt}{\isacharparenright}{\kern0pt}{\isacharparenright}{\kern0pt}{\isacharparenright}{\kern0pt}{\isacharparenright}{\kern0pt}{\isacharparenright}{\kern0pt}{\isacharparenright}{\kern0pt}{\isacharparenright}{\kern0pt}\ {\isachardoublequoteclose}\ \isanewline
\isanewline
\isacommand{lemma}\isamarkupfalse%
\ omap{\isacharunderscore}{\kern0pt}replacement\ {\isacharcolon}{\kern0pt}\ \isanewline
\ \ \isakeyword{fixes}\ A\ r\ \isanewline
\ \ \isakeyword{assumes}\ {\isachardoublequoteopen}A\ {\isasymin}\ M{\isachardoublequoteclose}\ {\isachardoublequoteopen}r\ {\isasymin}\ M{\isachardoublequoteclose}\ \isanewline
\ \ \isakeyword{shows}\ {\isachardoublequoteopen}strong{\isacharunderscore}{\kern0pt}replacement{\isacharparenleft}{\kern0pt}{\isacharhash}{\kern0pt}{\isacharhash}{\kern0pt}M{\isacharcomma}{\kern0pt}\isanewline
\ \ \ \ \ \ \ \ \ \ \ \ \ {\isasymlambda}a\ z{\isachardot}{\kern0pt}\ {\isasymexists}x{\isacharbrackleft}{\kern0pt}{\isacharhash}{\kern0pt}{\isacharhash}{\kern0pt}M{\isacharbrackright}{\kern0pt}{\isachardot}{\kern0pt}\ {\isasymexists}g{\isacharbrackleft}{\kern0pt}{\isacharhash}{\kern0pt}{\isacharhash}{\kern0pt}M{\isacharbrackright}{\kern0pt}{\isachardot}{\kern0pt}\ {\isasymexists}mx{\isacharbrackleft}{\kern0pt}{\isacharhash}{\kern0pt}{\isacharhash}{\kern0pt}M{\isacharbrackright}{\kern0pt}{\isachardot}{\kern0pt}\ {\isasymexists}par{\isacharbrackleft}{\kern0pt}{\isacharhash}{\kern0pt}{\isacharhash}{\kern0pt}M{\isacharbrackright}{\kern0pt}{\isachardot}{\kern0pt}\isanewline
\ \ \ \ \ \ \ \ \ \ \ \ \ ordinal{\isacharparenleft}{\kern0pt}{\isacharhash}{\kern0pt}{\isacharhash}{\kern0pt}M{\isacharcomma}{\kern0pt}x{\isacharparenright}{\kern0pt}\ {\isasymand}\ pair{\isacharparenleft}{\kern0pt}{\isacharhash}{\kern0pt}{\isacharhash}{\kern0pt}M{\isacharcomma}{\kern0pt}a{\isacharcomma}{\kern0pt}x{\isacharcomma}{\kern0pt}z{\isacharparenright}{\kern0pt}\ {\isasymand}\ membership{\isacharparenleft}{\kern0pt}{\isacharhash}{\kern0pt}{\isacharhash}{\kern0pt}M{\isacharcomma}{\kern0pt}x{\isacharcomma}{\kern0pt}mx{\isacharparenright}{\kern0pt}\ {\isasymand}\isanewline
\ \ \ \ \ \ \ \ \ \ \ \ \ pred{\isacharunderscore}{\kern0pt}set{\isacharparenleft}{\kern0pt}{\isacharhash}{\kern0pt}{\isacharhash}{\kern0pt}M{\isacharcomma}{\kern0pt}A{\isacharcomma}{\kern0pt}a{\isacharcomma}{\kern0pt}r{\isacharcomma}{\kern0pt}par{\isacharparenright}{\kern0pt}\ {\isasymand}\ order{\isacharunderscore}{\kern0pt}isomorphism{\isacharparenleft}{\kern0pt}{\isacharhash}{\kern0pt}{\isacharhash}{\kern0pt}M{\isacharcomma}{\kern0pt}par{\isacharcomma}{\kern0pt}r{\isacharcomma}{\kern0pt}x{\isacharcomma}{\kern0pt}mx{\isacharcomma}{\kern0pt}g{\isacharparenright}{\kern0pt}{\isacharparenright}{\kern0pt}{\isachardoublequoteclose}\isanewline
%
\isadelimproof
\isanewline
\ \ %
\endisadelimproof
%
\isatagproof
\isacommand{apply}\isamarkupfalse%
{\isacharparenleft}{\kern0pt}rule{\isacharunderscore}{\kern0pt}tac\ P{\isacharequal}{\kern0pt}{\isachardoublequoteopen}strong{\isacharunderscore}{\kern0pt}replacement{\isacharparenleft}{\kern0pt}{\isacharhash}{\kern0pt}{\isacharhash}{\kern0pt}M{\isacharcomma}{\kern0pt}\ {\isasymlambda}a\ z{\isachardot}{\kern0pt}\ {\isasymexists}x\ {\isasymin}\ M{\isachardot}{\kern0pt}\ {\isasymexists}g\ {\isasymin}\ M{\isachardot}{\kern0pt}\ {\isasymexists}mx\ {\isasymin}\ M{\isachardot}{\kern0pt}\ {\isasymexists}par\ {\isasymin}\ M{\isachardot}{\kern0pt}\isanewline
\ \ \ \ \ \ \ \ \ \ \ \ \ ordinal{\isacharparenleft}{\kern0pt}{\isacharhash}{\kern0pt}{\isacharhash}{\kern0pt}M{\isacharcomma}{\kern0pt}x{\isacharparenright}{\kern0pt}\ {\isasymand}\ pair{\isacharparenleft}{\kern0pt}{\isacharhash}{\kern0pt}{\isacharhash}{\kern0pt}M{\isacharcomma}{\kern0pt}a{\isacharcomma}{\kern0pt}x{\isacharcomma}{\kern0pt}z{\isacharparenright}{\kern0pt}\ {\isasymand}\ membership{\isacharparenleft}{\kern0pt}{\isacharhash}{\kern0pt}{\isacharhash}{\kern0pt}M{\isacharcomma}{\kern0pt}x{\isacharcomma}{\kern0pt}mx{\isacharparenright}{\kern0pt}\ {\isasymand}\isanewline
\ \ \ \ \ \ \ \ \ \ \ \ \ pred{\isacharunderscore}{\kern0pt}set{\isacharparenleft}{\kern0pt}{\isacharhash}{\kern0pt}{\isacharhash}{\kern0pt}M{\isacharcomma}{\kern0pt}A{\isacharcomma}{\kern0pt}a{\isacharcomma}{\kern0pt}r{\isacharcomma}{\kern0pt}par{\isacharparenright}{\kern0pt}\ {\isasymand}\ order{\isacharunderscore}{\kern0pt}isomorphism{\isacharparenleft}{\kern0pt}{\isacharhash}{\kern0pt}{\isacharhash}{\kern0pt}M{\isacharcomma}{\kern0pt}par{\isacharcomma}{\kern0pt}r{\isacharcomma}{\kern0pt}x{\isacharcomma}{\kern0pt}mx{\isacharcomma}{\kern0pt}g{\isacharparenright}{\kern0pt}{\isacharparenright}{\kern0pt}{\isachardoublequoteclose}\ \isakeyword{in}\ iffD{\isadigit{1}}{\isacharparenright}{\kern0pt}\isanewline
\ \ \ \isacommand{apply}\isamarkupfalse%
{\isacharparenleft}{\kern0pt}rule\ strong{\isacharunderscore}{\kern0pt}replacement{\isacharunderscore}{\kern0pt}cong{\isacharparenright}{\kern0pt}\isanewline
\ \ \isacommand{using}\isamarkupfalse%
\ assms\isanewline
\ \ \ \isacommand{apply}\isamarkupfalse%
\ force\isanewline
\ \ \isacommand{apply}\isamarkupfalse%
{\isacharparenleft}{\kern0pt}rule{\isacharunderscore}{\kern0pt}tac\ P{\isacharequal}{\kern0pt}{\isachardoublequoteopen}strong{\isacharunderscore}{\kern0pt}replacement{\isacharparenleft}{\kern0pt}{\isacharhash}{\kern0pt}{\isacharhash}{\kern0pt}M{\isacharcomma}{\kern0pt}\ {\isasymlambda}a\ z{\isachardot}{\kern0pt}\ sats{\isacharparenleft}{\kern0pt}M{\isacharcomma}{\kern0pt}\ omap{\isacharunderscore}{\kern0pt}replacement{\isacharunderscore}{\kern0pt}fm{\isacharcomma}{\kern0pt}\ {\isacharbrackleft}{\kern0pt}a{\isacharcomma}{\kern0pt}\ z{\isacharbrackright}{\kern0pt}\ {\isacharat}{\kern0pt}\ {\isacharbrackleft}{\kern0pt}A{\isacharcomma}{\kern0pt}\ r{\isacharbrackright}{\kern0pt}{\isacharparenright}{\kern0pt}{\isacharparenright}{\kern0pt}{\isachardoublequoteclose}\ \isakeyword{in}\ iffD{\isadigit{1}}{\isacharparenright}{\kern0pt}\isanewline
\ \ \isacommand{apply}\isamarkupfalse%
{\isacharparenleft}{\kern0pt}rule\ strong{\isacharunderscore}{\kern0pt}replacement{\isacharunderscore}{\kern0pt}cong{\isacharparenright}{\kern0pt}\isanewline
\ \ \isacommand{unfolding}\isamarkupfalse%
\ omap{\isacharunderscore}{\kern0pt}replacement{\isacharunderscore}{\kern0pt}fm{\isacharunderscore}{\kern0pt}def\isanewline
\ \ \ \isacommand{apply}\isamarkupfalse%
{\isacharparenleft}{\kern0pt}rule\ iff{\isacharunderscore}{\kern0pt}flip{\isacharparenright}{\kern0pt}\isanewline
\ \ \isacommand{using}\isamarkupfalse%
\ omap{\isacharunderscore}{\kern0pt}replacement{\isacharunderscore}{\kern0pt}fm{\isacharunderscore}{\kern0pt}auto\ assms\isanewline
\ \ \ \isacommand{apply}\isamarkupfalse%
\ force\isanewline
\ \ \isacommand{apply}\isamarkupfalse%
{\isacharparenleft}{\kern0pt}rule\ replacement{\isacharunderscore}{\kern0pt}ax{\isacharparenright}{\kern0pt}\isanewline
\ \ \isacommand{using}\isamarkupfalse%
\ assms\isanewline
\ \ \ \ \isacommand{apply}\isamarkupfalse%
\ auto{\isacharbrackleft}{\kern0pt}{\isadigit{2}}{\isacharbrackright}{\kern0pt}\isanewline
\ \ \isacommand{unfolding}\isamarkupfalse%
\ omap{\isacharunderscore}{\kern0pt}replacement{\isacharunderscore}{\kern0pt}fm{\isacharunderscore}{\kern0pt}def\ pred{\isacharunderscore}{\kern0pt}set{\isacharunderscore}{\kern0pt}fm{\isacharunderscore}{\kern0pt}def\ Memrel{\isacharunderscore}{\kern0pt}fm{\isacharunderscore}{\kern0pt}def\ bijection{\isacharunderscore}{\kern0pt}fm{\isacharunderscore}{\kern0pt}def\isanewline
\ \ \isacommand{apply}\isamarkupfalse%
\ {\isacharparenleft}{\kern0pt}simp\ del{\isacharcolon}{\kern0pt}FOL{\isacharunderscore}{\kern0pt}sats{\isacharunderscore}{\kern0pt}iff\ pair{\isacharunderscore}{\kern0pt}abs\ add{\isacharcolon}{\kern0pt}\ fm{\isacharunderscore}{\kern0pt}defs\ nat{\isacharunderscore}{\kern0pt}simp{\isacharunderscore}{\kern0pt}union{\isacharparenright}{\kern0pt}\isanewline
\ \ \isacommand{unfolding}\isamarkupfalse%
\ order{\isacharunderscore}{\kern0pt}isomorphism{\isacharunderscore}{\kern0pt}fm{\isacharunderscore}{\kern0pt}def\ bijection{\isacharunderscore}{\kern0pt}fm{\isacharunderscore}{\kern0pt}def\ injection{\isacharunderscore}{\kern0pt}fm{\isacharunderscore}{\kern0pt}def\ surjection{\isacharunderscore}{\kern0pt}fm{\isacharunderscore}{\kern0pt}def\isanewline
\ \ \isacommand{apply}\isamarkupfalse%
\ {\isacharparenleft}{\kern0pt}simp\ del{\isacharcolon}{\kern0pt}FOL{\isacharunderscore}{\kern0pt}sats{\isacharunderscore}{\kern0pt}iff\ pair{\isacharunderscore}{\kern0pt}abs\ add{\isacharcolon}{\kern0pt}\ fm{\isacharunderscore}{\kern0pt}defs\ nat{\isacharunderscore}{\kern0pt}simp{\isacharunderscore}{\kern0pt}union{\isacharparenright}{\kern0pt}\isanewline
\ \ \isacommand{done}\isamarkupfalse%
%
\endisatagproof
{\isafoldproof}%
%
\isadelimproof
\isanewline
%
\endisadelimproof
\isanewline
\isacommand{lemma}\isamarkupfalse%
\ mordertype\ {\isacharcolon}{\kern0pt}\ {\isachardoublequoteopen}M{\isacharunderscore}{\kern0pt}ordertype{\isacharparenleft}{\kern0pt}{\isacharhash}{\kern0pt}{\isacharhash}{\kern0pt}M{\isacharparenright}{\kern0pt}{\isachardoublequoteclose}\ \isanewline
%
\isadelimproof
\ \ %
\endisadelimproof
%
\isatagproof
\isacommand{unfolding}\isamarkupfalse%
\ M{\isacharunderscore}{\kern0pt}ordertype{\isacharunderscore}{\kern0pt}def\isanewline
\ \ \isacommand{using}\isamarkupfalse%
\ mbasic\isanewline
\ \ \isacommand{apply}\isamarkupfalse%
\ simp\isanewline
\ \ \isacommand{unfolding}\isamarkupfalse%
\ M{\isacharunderscore}{\kern0pt}ordertype{\isacharunderscore}{\kern0pt}axioms{\isacharunderscore}{\kern0pt}def\isanewline
\ \ \isacommand{using}\isamarkupfalse%
\ well{\isacharunderscore}{\kern0pt}ord{\isacharunderscore}{\kern0pt}iso{\isacharunderscore}{\kern0pt}separation\ obase{\isacharunderscore}{\kern0pt}separation\ obase{\isacharunderscore}{\kern0pt}equals{\isacharunderscore}{\kern0pt}separation\ omap{\isacharunderscore}{\kern0pt}replacement\isanewline
\ \ \isacommand{by}\isamarkupfalse%
\ auto%
\endisatagproof
{\isafoldproof}%
%
\isadelimproof
\isanewline
%
\endisadelimproof
\isanewline
\isacommand{lemma}\isamarkupfalse%
\ wellorder{\isacharunderscore}{\kern0pt}induces{\isacharunderscore}{\kern0pt}injection\ {\isacharcolon}{\kern0pt}\ \isanewline
\ \ \isakeyword{fixes}\ r\ A\isanewline
\ \ \isakeyword{assumes}\ {\isachardoublequoteopen}nat\ {\isasymlesssim}\ A{\isachardoublequoteclose}\ {\isachardoublequoteopen}wellordered{\isacharparenleft}{\kern0pt}{\isacharhash}{\kern0pt}{\isacharhash}{\kern0pt}M{\isacharcomma}{\kern0pt}\ A{\isacharcomma}{\kern0pt}\ r{\isacharparenright}{\kern0pt}{\isachardoublequoteclose}\ {\isachardoublequoteopen}r\ {\isasymin}\ M{\isachardoublequoteclose}\ {\isachardoublequoteopen}A\ {\isasymin}\ M{\isachardoublequoteclose}\ \isanewline
\ \ \isakeyword{shows}\ {\isachardoublequoteopen}{\isasymexists}f\ {\isasymin}\ M{\isachardot}{\kern0pt}\ f\ {\isasymin}\ inj{\isacharparenleft}{\kern0pt}nat{\isacharcomma}{\kern0pt}\ A{\isacharparenright}{\kern0pt}{\isachardoublequoteclose}\ \isanewline
%
\isadelimproof
%
\endisadelimproof
%
\isatagproof
\isacommand{proof}\isamarkupfalse%
\ {\isacharminus}{\kern0pt}\ \isanewline
\ \ \isacommand{have}\isamarkupfalse%
\ {\isachardoublequoteopen}{\isasymexists}f{\isacharbrackleft}{\kern0pt}{\isacharhash}{\kern0pt}{\isacharhash}{\kern0pt}M{\isacharbrackright}{\kern0pt}{\isachardot}{\kern0pt}\ {\isacharparenleft}{\kern0pt}{\isasymexists}i{\isacharbrackleft}{\kern0pt}{\isacharhash}{\kern0pt}{\isacharhash}{\kern0pt}M{\isacharbrackright}{\kern0pt}{\isachardot}{\kern0pt}\ Ord{\isacharparenleft}{\kern0pt}i{\isacharparenright}{\kern0pt}\ {\isasymand}\ f\ {\isasymin}\ ord{\isacharunderscore}{\kern0pt}iso{\isacharparenleft}{\kern0pt}A{\isacharcomma}{\kern0pt}\ r{\isacharcomma}{\kern0pt}\ i{\isacharcomma}{\kern0pt}\ Memrel{\isacharparenleft}{\kern0pt}i{\isacharparenright}{\kern0pt}{\isacharparenright}{\kern0pt}{\isacharparenright}{\kern0pt}{\isachardoublequoteclose}\isanewline
\ \ \ \ \isacommand{apply}\isamarkupfalse%
{\isacharparenleft}{\kern0pt}rule\ M{\isacharunderscore}{\kern0pt}ordertype{\isachardot}{\kern0pt}ordertype{\isacharunderscore}{\kern0pt}exists{\isacharparenright}{\kern0pt}\isanewline
\ \ \ \ \isacommand{using}\isamarkupfalse%
\ mordertype\ assms\isanewline
\ \ \ \ \isacommand{by}\isamarkupfalse%
\ auto\isanewline
\ \ \isacommand{then}\isamarkupfalse%
\ \isacommand{obtain}\isamarkupfalse%
\ f\ i\ \isakeyword{where}\ fiH{\isacharcolon}{\kern0pt}\ {\isachardoublequoteopen}f\ {\isasymin}\ M{\isachardoublequoteclose}\ {\isachardoublequoteopen}i\ {\isasymin}\ M{\isachardoublequoteclose}\ {\isachardoublequoteopen}Ord{\isacharparenleft}{\kern0pt}i{\isacharparenright}{\kern0pt}{\isachardoublequoteclose}\ {\isachardoublequoteopen}f\ {\isasymin}\ ord{\isacharunderscore}{\kern0pt}iso{\isacharparenleft}{\kern0pt}A{\isacharcomma}{\kern0pt}\ r{\isacharcomma}{\kern0pt}\ i{\isacharcomma}{\kern0pt}\ Memrel{\isacharparenleft}{\kern0pt}i{\isacharparenright}{\kern0pt}{\isacharparenright}{\kern0pt}{\isachardoublequoteclose}\isanewline
\ \ \ \ \isacommand{by}\isamarkupfalse%
\ auto\isanewline
\ \ \isacommand{then}\isamarkupfalse%
\ \isacommand{have}\isamarkupfalse%
\ fbij{\isacharcolon}{\kern0pt}\ {\isachardoublequoteopen}f\ {\isasymin}\ bij{\isacharparenleft}{\kern0pt}A{\isacharcomma}{\kern0pt}\ i{\isacharparenright}{\kern0pt}{\isachardoublequoteclose}\ \isacommand{using}\isamarkupfalse%
\ ord{\isacharunderscore}{\kern0pt}iso{\isacharunderscore}{\kern0pt}def\ \isacommand{by}\isamarkupfalse%
\ auto\isanewline
\ \ \isacommand{then}\isamarkupfalse%
\ \isacommand{have}\isamarkupfalse%
\ {\isachardoublequoteopen}A\ {\isasymapprox}\ i{\isachardoublequoteclose}\ \isacommand{using}\isamarkupfalse%
\ eqpoll{\isacharunderscore}{\kern0pt}def\ \isacommand{by}\isamarkupfalse%
\ auto\isanewline
\ \ \isacommand{then}\isamarkupfalse%
\ \isacommand{have}\isamarkupfalse%
\ {\isachardoublequoteopen}A\ {\isasymlesssim}\ i{\isachardoublequoteclose}\ \isacommand{using}\isamarkupfalse%
\ eqpoll{\isacharunderscore}{\kern0pt}imp{\isacharunderscore}{\kern0pt}lepoll\ \isacommand{by}\isamarkupfalse%
\ auto\isanewline
\ \ \isacommand{then}\isamarkupfalse%
\ \isacommand{have}\isamarkupfalse%
\ {\isachardoublequoteopen}nat\ {\isasymlesssim}\ i{\isachardoublequoteclose}\ \isanewline
\ \ \ \ \isacommand{apply}\isamarkupfalse%
{\isacharparenleft}{\kern0pt}rule{\isacharunderscore}{\kern0pt}tac\ Y{\isacharequal}{\kern0pt}{\isachardoublequoteopen}A{\isachardoublequoteclose}\ \isakeyword{in}\ lepoll{\isacharunderscore}{\kern0pt}trans{\isacharparenright}{\kern0pt}\isanewline
\ \ \ \ \isacommand{using}\isamarkupfalse%
\ assms\isanewline
\ \ \ \ \isacommand{by}\isamarkupfalse%
\ auto\isanewline
\ \ \isacommand{then}\isamarkupfalse%
\ \isacommand{have}\isamarkupfalse%
\ {\isachardoublequoteopen}{\isacharbar}{\kern0pt}nat{\isacharbar}{\kern0pt}\ {\isasymle}\ i{\isachardoublequoteclose}\ \isanewline
\ \ \ \ \isacommand{apply}\isamarkupfalse%
{\isacharparenleft}{\kern0pt}rule{\isacharunderscore}{\kern0pt}tac\ lepoll{\isacharunderscore}{\kern0pt}cardinal{\isacharunderscore}{\kern0pt}le{\isacharparenright}{\kern0pt}\isanewline
\ \ \ \ \isacommand{using}\isamarkupfalse%
\ fiH\ \isanewline
\ \ \ \ \isacommand{by}\isamarkupfalse%
\ auto\isanewline
\ \ \isacommand{then}\isamarkupfalse%
\ \isacommand{have}\isamarkupfalse%
\ {\isachardoublequoteopen}nat\ {\isasymle}\ i{\isachardoublequoteclose}\ \isanewline
\ \ \ \ \isacommand{using}\isamarkupfalse%
\ Card{\isacharunderscore}{\kern0pt}nat\ Card{\isacharunderscore}{\kern0pt}def\isanewline
\ \ \ \ \isacommand{by}\isamarkupfalse%
\ auto\isanewline
\ \ \isacommand{then}\isamarkupfalse%
\ \isacommand{have}\isamarkupfalse%
\ {\isachardoublequoteopen}nat\ {\isasymsubseteq}\ i{\isachardoublequoteclose}\ \isanewline
\ \ \ \ \isacommand{apply}\isamarkupfalse%
{\isacharparenleft}{\kern0pt}rule{\isacharunderscore}{\kern0pt}tac\ subsetI{\isacharparenright}{\kern0pt}\isanewline
\ \ \ \ \isacommand{apply}\isamarkupfalse%
{\isacharparenleft}{\kern0pt}rule\ ltD{\isacharparenright}{\kern0pt}\isanewline
\ \ \ \ \isacommand{apply}\isamarkupfalse%
{\isacharparenleft}{\kern0pt}rule{\isacharunderscore}{\kern0pt}tac\ b{\isacharequal}{\kern0pt}nat\ \isakeyword{in}\ lt{\isacharunderscore}{\kern0pt}le{\isacharunderscore}{\kern0pt}lt{\isacharparenright}{\kern0pt}\isanewline
\ \ \ \ \ \isacommand{apply}\isamarkupfalse%
{\isacharparenleft}{\kern0pt}rule\ ltI{\isacharparenright}{\kern0pt}\isanewline
\ \ \ \ \isacommand{by}\isamarkupfalse%
\ auto\isanewline
\isanewline
\ \ \isacommand{have}\isamarkupfalse%
\ {\isachardoublequoteopen}converse{\isacharparenleft}{\kern0pt}f{\isacharparenright}{\kern0pt}\ {\isasymin}\ M{\isachardoublequoteclose}\ \isacommand{using}\isamarkupfalse%
\ converse{\isacharunderscore}{\kern0pt}closed\ fiH\ \isacommand{by}\isamarkupfalse%
\ auto\isanewline
\ \ \isacommand{have}\isamarkupfalse%
\ {\isachardoublequoteopen}converse{\isacharparenleft}{\kern0pt}f{\isacharparenright}{\kern0pt}\ {\isasymin}\ bij{\isacharparenleft}{\kern0pt}i{\isacharcomma}{\kern0pt}\ A{\isacharparenright}{\kern0pt}{\isachardoublequoteclose}\ \isacommand{using}\isamarkupfalse%
\ bij{\isacharunderscore}{\kern0pt}converse{\isacharunderscore}{\kern0pt}bij\ fbij\ \isacommand{by}\isamarkupfalse%
\ auto\ \isanewline
\isanewline
\ \ \isacommand{then}\isamarkupfalse%
\ \isacommand{have}\isamarkupfalse%
\ {\isachardoublequoteopen}restrict{\isacharparenleft}{\kern0pt}converse{\isacharparenleft}{\kern0pt}f{\isacharparenright}{\kern0pt}{\isacharcomma}{\kern0pt}\ nat{\isacharparenright}{\kern0pt}\ {\isasymin}\ inj{\isacharparenleft}{\kern0pt}nat{\isacharcomma}{\kern0pt}\ A{\isacharparenright}{\kern0pt}{\isachardoublequoteclose}\ \isanewline
\ \ \ \ \isacommand{apply}\isamarkupfalse%
{\isacharparenleft}{\kern0pt}rule{\isacharunderscore}{\kern0pt}tac\ A{\isacharequal}{\kern0pt}i\ \isakeyword{in}\ restrict{\isacharunderscore}{\kern0pt}inj{\isacharparenright}{\kern0pt}\isanewline
\ \ \ \ \isacommand{using}\isamarkupfalse%
\ fbij\ bij{\isacharunderscore}{\kern0pt}is{\isacharunderscore}{\kern0pt}inj\ {\isacartoucheopen}nat\ {\isasymsubseteq}\ i{\isacartoucheclose}\isanewline
\ \ \ \ \isacommand{by}\isamarkupfalse%
\ auto\isanewline
\ \ \isacommand{then}\isamarkupfalse%
\ \isacommand{show}\isamarkupfalse%
\ {\isacharquery}{\kern0pt}thesis\ \isanewline
\ \ \ \ \isacommand{apply}\isamarkupfalse%
{\isacharparenleft}{\kern0pt}rule{\isacharunderscore}{\kern0pt}tac\ x{\isacharequal}{\kern0pt}{\isachardoublequoteopen}restrict{\isacharparenleft}{\kern0pt}converse{\isacharparenleft}{\kern0pt}f{\isacharparenright}{\kern0pt}{\isacharcomma}{\kern0pt}\ nat{\isacharparenright}{\kern0pt}{\isachardoublequoteclose}\ \isakeyword{in}\ bexI{\isacharcomma}{\kern0pt}\ simp{\isacharparenright}{\kern0pt}\isanewline
\ \ \ \ \isacommand{apply}\isamarkupfalse%
{\isacharparenleft}{\kern0pt}subgoal{\isacharunderscore}{\kern0pt}tac\ {\isachardoublequoteopen}{\isacharparenleft}{\kern0pt}{\isacharhash}{\kern0pt}{\isacharhash}{\kern0pt}M{\isacharparenright}{\kern0pt}{\isacharparenleft}{\kern0pt}restrict{\isacharparenleft}{\kern0pt}converse{\isacharparenleft}{\kern0pt}f{\isacharparenright}{\kern0pt}{\isacharcomma}{\kern0pt}\ nat{\isacharparenright}{\kern0pt}{\isacharparenright}{\kern0pt}{\isachardoublequoteclose}{\isacharparenright}{\kern0pt}\isanewline
\ \ \ \ \isacommand{apply}\isamarkupfalse%
{\isacharparenleft}{\kern0pt}force{\isacharcomma}{\kern0pt}\ rule\ restrict{\isacharunderscore}{\kern0pt}closed{\isacharparenright}{\kern0pt}\isanewline
\ \ \ \ \isacommand{using}\isamarkupfalse%
\ {\isacartoucheopen}converse{\isacharparenleft}{\kern0pt}f{\isacharparenright}{\kern0pt}\ {\isasymin}\ M{\isacartoucheclose}\ nat{\isacharunderscore}{\kern0pt}in{\isacharunderscore}{\kern0pt}M\isanewline
\ \ \ \ \isacommand{by}\isamarkupfalse%
\ auto\isanewline
\isacommand{qed}\isamarkupfalse%
%
\endisatagproof
{\isafoldproof}%
%
\isadelimproof
\isanewline
%
\endisadelimproof
\ \ \ \ \ \ \isanewline
\isacommand{end}\isamarkupfalse%
\ \isanewline
%
\isadelimtheory
%
\endisadelimtheory
%
\isatagtheory
\isacommand{end}\isamarkupfalse%
%
\endisatagtheory
{\isafoldtheory}%
%
\isadelimtheory
%
\endisadelimtheory
%
\end{isabellebody}%
\endinput
%:%file=~/source/repos/ZF-notAC/code/NotAC_Inj.thy%:%
%:%10=1%:%
%:%11=1%:%
%:%12=2%:%
%:%13=3%:%
%:%18=3%:%
%:%21=4%:%
%:%22=5%:%
%:%23=5%:%
%:%24=6%:%
%:%25=7%:%
%:%26=8%:%
%:%27=8%:%
%:%28=9%:%
%:%29=10%:%
%:%30=11%:%
%:%31=12%:%
%:%34=13%:%
%:%38=13%:%
%:%39=13%:%
%:%44=13%:%
%:%47=14%:%
%:%48=15%:%
%:%49=15%:%
%:%50=16%:%
%:%51=17%:%
%:%52=17%:%
%:%53=18%:%
%:%54=19%:%
%:%55=20%:%
%:%58=21%:%
%:%59=22%:%
%:%63=22%:%
%:%64=22%:%
%:%65=23%:%
%:%66=23%:%
%:%67=24%:%
%:%68=24%:%
%:%69=25%:%
%:%70=25%:%
%:%71=26%:%
%:%72=26%:%
%:%73=27%:%
%:%74=27%:%
%:%75=28%:%
%:%76=28%:%
%:%77=29%:%
%:%78=29%:%
%:%79=30%:%
%:%80=30%:%
%:%81=31%:%
%:%82=31%:%
%:%83=32%:%
%:%84=32%:%
%:%85=33%:%
%:%86=33%:%
%:%87=34%:%
%:%88=34%:%
%:%89=35%:%
%:%90=35%:%
%:%91=36%:%
%:%97=36%:%
%:%100=37%:%
%:%101=38%:%
%:%102=38%:%
%:%103=39%:%
%:%104=40%:%
%:%105=41%:%
%:%106=42%:%
%:%108=44%:%
%:%111=45%:%
%:%115=45%:%
%:%116=45%:%
%:%121=45%:%
%:%124=46%:%
%:%125=47%:%
%:%126=47%:%
%:%130=51%:%
%:%131=52%:%
%:%132=53%:%
%:%133=53%:%
%:%134=54%:%
%:%135=55%:%
%:%136=56%:%
%:%138=58%:%
%:%141=59%:%
%:%142=60%:%
%:%146=60%:%
%:%147=60%:%
%:%149=62%:%
%:%150=63%:%
%:%151=63%:%
%:%152=64%:%
%:%153=64%:%
%:%154=65%:%
%:%155=65%:%
%:%156=66%:%
%:%157=66%:%
%:%158=67%:%
%:%159=67%:%
%:%160=68%:%
%:%161=68%:%
%:%162=69%:%
%:%163=69%:%
%:%164=70%:%
%:%165=70%:%
%:%166=71%:%
%:%167=71%:%
%:%168=72%:%
%:%169=72%:%
%:%170=73%:%
%:%171=73%:%
%:%172=74%:%
%:%173=74%:%
%:%174=75%:%
%:%175=75%:%
%:%176=76%:%
%:%177=76%:%
%:%178=77%:%
%:%179=77%:%
%:%180=78%:%
%:%181=78%:%
%:%182=79%:%
%:%188=79%:%
%:%191=80%:%
%:%192=81%:%
%:%193=82%:%
%:%194=82%:%
%:%195=83%:%
%:%196=84%:%
%:%197=85%:%
%:%198=86%:%
%:%201=89%:%
%:%204=90%:%
%:%208=90%:%
%:%209=90%:%
%:%214=90%:%
%:%217=91%:%
%:%218=92%:%
%:%219=92%:%
%:%225=98%:%
%:%226=99%:%
%:%227=100%:%
%:%228=100%:%
%:%229=101%:%
%:%230=102%:%
%:%231=103%:%
%:%234=106%:%
%:%237=107%:%
%:%238=108%:%
%:%242=108%:%
%:%243=108%:%
%:%246=111%:%
%:%247=112%:%
%:%248=112%:%
%:%249=113%:%
%:%250=113%:%
%:%251=114%:%
%:%252=114%:%
%:%253=115%:%
%:%254=115%:%
%:%255=116%:%
%:%256=116%:%
%:%257=117%:%
%:%258=117%:%
%:%259=118%:%
%:%260=118%:%
%:%261=119%:%
%:%262=119%:%
%:%263=120%:%
%:%264=120%:%
%:%265=121%:%
%:%266=121%:%
%:%267=122%:%
%:%268=122%:%
%:%269=123%:%
%:%270=123%:%
%:%271=124%:%
%:%272=124%:%
%:%273=125%:%
%:%274=125%:%
%:%275=126%:%
%:%276=126%:%
%:%277=127%:%
%:%278=127%:%
%:%279=128%:%
%:%285=128%:%
%:%288=129%:%
%:%289=130%:%
%:%290=130%:%
%:%291=131%:%
%:%292=132%:%
%:%293=133%:%
%:%294=134%:%
%:%296=136%:%
%:%299=137%:%
%:%303=137%:%
%:%304=137%:%
%:%309=137%:%
%:%312=138%:%
%:%313=139%:%
%:%314=139%:%
%:%320=145%:%
%:%321=146%:%
%:%322=147%:%
%:%323=147%:%
%:%324=148%:%
%:%325=149%:%
%:%326=150%:%
%:%329=153%:%
%:%332=154%:%
%:%333=155%:%
%:%337=155%:%
%:%338=155%:%
%:%340=157%:%
%:%341=158%:%
%:%342=158%:%
%:%343=159%:%
%:%344=159%:%
%:%345=160%:%
%:%346=160%:%
%:%347=161%:%
%:%348=161%:%
%:%349=162%:%
%:%350=162%:%
%:%351=163%:%
%:%352=163%:%
%:%353=164%:%
%:%354=164%:%
%:%355=165%:%
%:%356=165%:%
%:%357=166%:%
%:%358=166%:%
%:%359=167%:%
%:%360=167%:%
%:%361=168%:%
%:%362=168%:%
%:%363=169%:%
%:%364=169%:%
%:%365=170%:%
%:%366=170%:%
%:%367=171%:%
%:%368=171%:%
%:%369=172%:%
%:%370=172%:%
%:%371=173%:%
%:%372=173%:%
%:%373=174%:%
%:%379=174%:%
%:%382=175%:%
%:%383=176%:%
%:%384=176%:%
%:%387=177%:%
%:%391=177%:%
%:%392=177%:%
%:%393=178%:%
%:%394=178%:%
%:%395=179%:%
%:%396=179%:%
%:%397=180%:%
%:%398=180%:%
%:%399=181%:%
%:%400=181%:%
%:%401=182%:%
%:%402=182%:%
%:%407=182%:%
%:%410=183%:%
%:%411=184%:%
%:%412=184%:%
%:%413=185%:%
%:%414=186%:%
%:%415=187%:%
%:%422=188%:%
%:%423=188%:%
%:%424=189%:%
%:%425=189%:%
%:%426=190%:%
%:%427=190%:%
%:%428=191%:%
%:%429=191%:%
%:%430=192%:%
%:%431=192%:%
%:%432=193%:%
%:%433=193%:%
%:%434=193%:%
%:%435=194%:%
%:%436=194%:%
%:%437=195%:%
%:%438=195%:%
%:%439=195%:%
%:%440=195%:%
%:%441=195%:%
%:%442=196%:%
%:%443=196%:%
%:%444=196%:%
%:%445=196%:%
%:%446=196%:%
%:%447=197%:%
%:%448=197%:%
%:%449=197%:%
%:%450=197%:%
%:%451=197%:%
%:%452=198%:%
%:%453=198%:%
%:%454=198%:%
%:%455=199%:%
%:%456=199%:%
%:%457=200%:%
%:%458=200%:%
%:%459=201%:%
%:%460=201%:%
%:%461=202%:%
%:%462=202%:%
%:%463=202%:%
%:%464=203%:%
%:%465=203%:%
%:%466=204%:%
%:%467=204%:%
%:%468=205%:%
%:%469=205%:%
%:%470=206%:%
%:%471=206%:%
%:%472=206%:%
%:%473=207%:%
%:%474=207%:%
%:%475=208%:%
%:%476=208%:%
%:%477=209%:%
%:%478=209%:%
%:%479=209%:%
%:%480=210%:%
%:%481=210%:%
%:%482=211%:%
%:%483=211%:%
%:%484=212%:%
%:%485=212%:%
%:%486=213%:%
%:%487=213%:%
%:%488=214%:%
%:%489=214%:%
%:%490=215%:%
%:%491=216%:%
%:%492=216%:%
%:%493=216%:%
%:%494=216%:%
%:%495=217%:%
%:%496=217%:%
%:%497=217%:%
%:%498=217%:%
%:%499=218%:%
%:%500=219%:%
%:%501=219%:%
%:%502=219%:%
%:%503=220%:%
%:%504=220%:%
%:%505=221%:%
%:%506=221%:%
%:%507=222%:%
%:%508=222%:%
%:%509=223%:%
%:%510=223%:%
%:%511=223%:%
%:%512=224%:%
%:%513=224%:%
%:%514=225%:%
%:%515=225%:%
%:%516=226%:%
%:%517=226%:%
%:%518=227%:%
%:%519=227%:%
%:%520=228%:%
%:%521=228%:%
%:%522=229%:%
%:%528=229%:%
%:%531=230%:%
%:%532=231%:%
%:%533=231%:%
%:%540=232%:%

%
\begin{isabellebody}%
\setisabellecontext{NotAC{\isacharunderscore}{\kern0pt}Proof}%
%
\isadelimtheory
%
\endisadelimtheory
%
\isatagtheory
\isacommand{theory}\isamarkupfalse%
\ NotAC{\isacharunderscore}{\kern0pt}Proof\ \isanewline
\ \ \isakeyword{imports}\ \isanewline
\ \ \ \ NotAC{\isacharunderscore}{\kern0pt}Binmap\ \isanewline
\ \ \ \ NotAC{\isacharunderscore}{\kern0pt}Inj\isanewline
\isakeyword{begin}%
\endisatagtheory
{\isafoldtheory}%
%
\isadelimtheory
\ \isanewline
%
\endisadelimtheory
\isanewline
\isacommand{context}\isamarkupfalse%
\ M{\isacharunderscore}{\kern0pt}ctm\ \isakeyword{begin}\ \isanewline
\isanewline
\isacommand{interpretation}\isamarkupfalse%
\ M{\isacharunderscore}{\kern0pt}symmetric{\isacharunderscore}{\kern0pt}system\ {\isachardoublequoteopen}Fn{\isachardoublequoteclose}\ {\isachardoublequoteopen}Fn{\isacharunderscore}{\kern0pt}leq{\isachardoublequoteclose}\ {\isachardoublequoteopen}{\isadigit{0}}{\isachardoublequoteclose}\ {\isachardoublequoteopen}M{\isachardoublequoteclose}\ {\isachardoublequoteopen}enum{\isachardoublequoteclose}\ {\isachardoublequoteopen}Fn{\isacharunderscore}{\kern0pt}perms{\isachardoublequoteclose}\ {\isachardoublequoteopen}Fn{\isacharunderscore}{\kern0pt}perms{\isacharunderscore}{\kern0pt}filter{\isachardoublequoteclose}\isanewline
%
\isadelimproof
\ \ %
\endisadelimproof
%
\isatagproof
\isacommand{using}\isamarkupfalse%
\ Fn{\isacharunderscore}{\kern0pt}M{\isacharunderscore}{\kern0pt}symmetric{\isacharunderscore}{\kern0pt}system\isanewline
\ \ \isacommand{by}\isamarkupfalse%
\ auto%
\endisatagproof
{\isafoldproof}%
%
\isadelimproof
\isanewline
%
\endisadelimproof
\isanewline
\isacommand{lemma}\isamarkupfalse%
\ no{\isacharunderscore}{\kern0pt}injection\ {\isacharcolon}{\kern0pt}\ \isanewline
\ \ \isakeyword{fixes}\ F{\isacharprime}{\kern0pt}\ p{\isadigit{0}}\isanewline
\ \ \isakeyword{assumes}\ {\isachardoublequoteopen}F{\isacharprime}{\kern0pt}\ {\isasymin}\ HS{\isachardoublequoteclose}\ {\isachardoublequoteopen}p{\isadigit{0}}\ {\isasymin}\ Fn{\isachardoublequoteclose}\ \ \isanewline
\ \ \ \ \ \ \ \ \ \ {\isachardoublequoteopen}ForcesHS{\isacharparenleft}{\kern0pt}p{\isadigit{0}}{\isacharcomma}{\kern0pt}\ injection{\isacharunderscore}{\kern0pt}fm{\isacharparenleft}{\kern0pt}{\isadigit{0}}{\isacharcomma}{\kern0pt}\ {\isadigit{1}}{\isacharcomma}{\kern0pt}\ {\isadigit{2}}{\isacharparenright}{\kern0pt}{\isacharcomma}{\kern0pt}\ {\isacharbrackleft}{\kern0pt}check{\isacharparenleft}{\kern0pt}nat{\isacharparenright}{\kern0pt}{\isacharcomma}{\kern0pt}\ binmap{\isacharprime}{\kern0pt}{\isacharcomma}{\kern0pt}\ F{\isacharprime}{\kern0pt}{\isacharbrackright}{\kern0pt}{\isacharparenright}{\kern0pt}{\isachardoublequoteclose}\ \isanewline
\ \ \isakeyword{shows}\ False\isanewline
%
\isadelimproof
%
\endisadelimproof
%
\isatagproof
\isacommand{proof}\isamarkupfalse%
\ {\isacharminus}{\kern0pt}\ \isanewline
\isanewline
\ \ \isacommand{define}\isamarkupfalse%
\ {\isasymphi}\ \isakeyword{where}\ {\isachardoublequoteopen}{\isasymphi}\ {\isasymequiv}\ Exists{\isacharparenleft}{\kern0pt}And{\isacharparenleft}{\kern0pt}fun{\isacharunderscore}{\kern0pt}apply{\isacharunderscore}{\kern0pt}fm{\isacharparenleft}{\kern0pt}{\isadigit{1}}{\isacharcomma}{\kern0pt}\ {\isadigit{2}}{\isacharcomma}{\kern0pt}\ {\isadigit{0}}{\isacharparenright}{\kern0pt}{\isacharcomma}{\kern0pt}\ Equal{\isacharparenleft}{\kern0pt}{\isadigit{0}}{\isacharcomma}{\kern0pt}\ {\isadigit{3}}{\isacharparenright}{\kern0pt}{\isacharparenright}{\kern0pt}{\isacharparenright}{\kern0pt}{\isachardoublequoteclose}\ \isanewline
\isanewline
\ \ \isacommand{have}\isamarkupfalse%
\ sats{\isacharunderscore}{\kern0pt}{\isasymphi}{\isacharunderscore}{\kern0pt}iff\ {\isacharcolon}{\kern0pt}\ {\isachardoublequoteopen}{\isasymAnd}G\ a\ b\ c{\isachardot}{\kern0pt}\ M{\isacharunderscore}{\kern0pt}generic{\isacharparenleft}{\kern0pt}G{\isacharparenright}{\kern0pt}\ {\isasymLongrightarrow}\ {\isacharbraceleft}{\kern0pt}a{\isacharcomma}{\kern0pt}\ b{\isacharcomma}{\kern0pt}\ c{\isacharbraceright}{\kern0pt}\ {\isasymsubseteq}\ SymExt{\isacharparenleft}{\kern0pt}G{\isacharparenright}{\kern0pt}\ {\isasymLongrightarrow}\ {\isacharparenleft}{\kern0pt}SymExt{\isacharparenleft}{\kern0pt}G{\isacharparenright}{\kern0pt}{\isacharcomma}{\kern0pt}\ {\isacharbrackleft}{\kern0pt}a{\isacharcomma}{\kern0pt}\ b{\isacharcomma}{\kern0pt}\ c{\isacharbrackright}{\kern0pt}\ {\isasymTurnstile}\ {\isasymphi}{\isacharparenright}{\kern0pt}\ {\isasymlongleftrightarrow}\ a{\isacharbackquote}{\kern0pt}b\ {\isacharequal}{\kern0pt}\ c{\isachardoublequoteclose}\isanewline
\ \ \isacommand{proof}\isamarkupfalse%
\ {\isacharminus}{\kern0pt}\ \isanewline
\ \ \ \ \isacommand{fix}\isamarkupfalse%
\ G\ a\ b\ c\ \isanewline
\ \ \ \ \isacommand{assume}\isamarkupfalse%
\ assms{\isadigit{1}}{\isacharcolon}{\kern0pt}\ {\isachardoublequoteopen}M{\isacharunderscore}{\kern0pt}generic{\isacharparenleft}{\kern0pt}G{\isacharparenright}{\kern0pt}{\isachardoublequoteclose}\ {\isachardoublequoteopen}{\isacharbraceleft}{\kern0pt}a{\isacharcomma}{\kern0pt}\ b{\isacharcomma}{\kern0pt}\ c{\isacharbraceright}{\kern0pt}\ {\isasymsubseteq}\ SymExt{\isacharparenleft}{\kern0pt}G{\isacharparenright}{\kern0pt}{\isachardoublequoteclose}\ \isanewline
\ \ \isanewline
\ \ \ \ \isacommand{interpret}\isamarkupfalse%
\ M{\isacharunderscore}{\kern0pt}symmetric{\isacharunderscore}{\kern0pt}system{\isacharunderscore}{\kern0pt}G{\isacharunderscore}{\kern0pt}generic\ \ {\isachardoublequoteopen}Fn{\isachardoublequoteclose}\ {\isachardoublequoteopen}Fn{\isacharunderscore}{\kern0pt}leq{\isachardoublequoteclose}\ {\isachardoublequoteopen}{\isadigit{0}}{\isachardoublequoteclose}\ {\isachardoublequoteopen}M{\isachardoublequoteclose}\ {\isachardoublequoteopen}enum{\isachardoublequoteclose}\ {\isachardoublequoteopen}Fn{\isacharunderscore}{\kern0pt}perms{\isachardoublequoteclose}\ {\isachardoublequoteopen}Fn{\isacharunderscore}{\kern0pt}perms{\isacharunderscore}{\kern0pt}filter{\isachardoublequoteclose}\ G\ \isanewline
\ \ \ \ \ \ \isacommand{unfolding}\isamarkupfalse%
\ M{\isacharunderscore}{\kern0pt}symmetric{\isacharunderscore}{\kern0pt}system{\isacharunderscore}{\kern0pt}G{\isacharunderscore}{\kern0pt}generic{\isacharunderscore}{\kern0pt}def\isanewline
\ \ \ \ \ \ \isacommand{apply}\isamarkupfalse%
{\isacharparenleft}{\kern0pt}rule\ conjI{\isacharparenright}{\kern0pt}\isanewline
\ \ \ \ \ \ \isacommand{using}\isamarkupfalse%
\ M{\isacharunderscore}{\kern0pt}symmetric{\isacharunderscore}{\kern0pt}system{\isacharunderscore}{\kern0pt}axioms\ \isanewline
\ \ \ \ \ \ \ \isacommand{apply}\isamarkupfalse%
\ force\isanewline
\ \ \ \ \ \ \isacommand{unfolding}\isamarkupfalse%
\ G{\isacharunderscore}{\kern0pt}generic{\isacharunderscore}{\kern0pt}def\isanewline
\ \ \ \ \ \ \isacommand{apply}\isamarkupfalse%
{\isacharparenleft}{\kern0pt}rule\ conjI{\isacharparenright}{\kern0pt}\isanewline
\ \ \ \ \ \ \isacommand{using}\isamarkupfalse%
\ forcing{\isacharunderscore}{\kern0pt}data{\isacharunderscore}{\kern0pt}axioms\isanewline
\ \ \ \ \ \ \ \isacommand{apply}\isamarkupfalse%
\ force\isanewline
\ \ \ \ \ \ \isacommand{unfolding}\isamarkupfalse%
\ G{\isacharunderscore}{\kern0pt}generic{\isacharunderscore}{\kern0pt}axioms{\isacharunderscore}{\kern0pt}def\isanewline
\ \ \ \ \ \ \isacommand{using}\isamarkupfalse%
\ assms{\isadigit{1}}\isanewline
\ \ \ \ \ \ \isacommand{by}\isamarkupfalse%
\ auto\isanewline
\isanewline
\ \ \ \ \isacommand{show}\isamarkupfalse%
\ {\isachardoublequoteopen}{\isacharparenleft}{\kern0pt}SymExt{\isacharparenleft}{\kern0pt}G{\isacharparenright}{\kern0pt}{\isacharcomma}{\kern0pt}\ {\isacharbrackleft}{\kern0pt}a{\isacharcomma}{\kern0pt}\ b{\isacharcomma}{\kern0pt}\ c{\isacharbrackright}{\kern0pt}\ {\isasymTurnstile}\ {\isasymphi}{\isacharparenright}{\kern0pt}\ {\isasymlongleftrightarrow}\ a{\isacharbackquote}{\kern0pt}b\ {\isacharequal}{\kern0pt}\ c{\isachardoublequoteclose}\isanewline
\ \ \ \ \ \ \isacommand{apply}\isamarkupfalse%
{\isacharparenleft}{\kern0pt}rule{\isacharunderscore}{\kern0pt}tac\ Q{\isacharequal}{\kern0pt}{\isachardoublequoteopen}{\isasymexists}d\ {\isasymin}\ SymExt{\isacharparenleft}{\kern0pt}G{\isacharparenright}{\kern0pt}{\isachardot}{\kern0pt}\ a{\isacharbackquote}{\kern0pt}b\ {\isacharequal}{\kern0pt}\ d\ {\isasymand}\ d\ {\isacharequal}{\kern0pt}\ c{\isachardoublequoteclose}\ \isakeyword{in}\ iff{\isacharunderscore}{\kern0pt}trans{\isacharparenright}{\kern0pt}\isanewline
\ \ \ \ \ \ \isacommand{unfolding}\isamarkupfalse%
\ {\isasymphi}{\isacharunderscore}{\kern0pt}def\isanewline
\ \ \ \ \ \ \ \isacommand{apply}\isamarkupfalse%
{\isacharparenleft}{\kern0pt}rule\ iff{\isacharunderscore}{\kern0pt}trans{\isacharcomma}{\kern0pt}\ rule\ sats{\isacharunderscore}{\kern0pt}Exists{\isacharunderscore}{\kern0pt}iff{\isacharparenright}{\kern0pt}\isanewline
\ \ \ \ \ \ \isacommand{using}\isamarkupfalse%
\ assms{\isadigit{1}}\isanewline
\ \ \ \ \ \ \ \ \isacommand{apply}\isamarkupfalse%
\ force\isanewline
\ \ \ \ \ \ \ \isacommand{apply}\isamarkupfalse%
{\isacharparenleft}{\kern0pt}rule\ bex{\isacharunderscore}{\kern0pt}iff{\isacharparenright}{\kern0pt}\isanewline
\ \ \ \ \ \ \ \isacommand{apply}\isamarkupfalse%
{\isacharparenleft}{\kern0pt}rule\ iff{\isacharunderscore}{\kern0pt}trans{\isacharcomma}{\kern0pt}\ rule\ sats{\isacharunderscore}{\kern0pt}And{\isacharunderscore}{\kern0pt}iff{\isacharparenright}{\kern0pt}\isanewline
\ \ \ \ \ \ \isacommand{using}\isamarkupfalse%
\ assms{\isadigit{1}}\isanewline
\ \ \ \ \ \ \ \ \isacommand{apply}\isamarkupfalse%
\ force\isanewline
\ \ \ \ \ \ \ \isacommand{apply}\isamarkupfalse%
{\isacharparenleft}{\kern0pt}rule\ iff{\isacharunderscore}{\kern0pt}conjI{\isacharparenright}{\kern0pt}\isanewline
\ \ \ \ \ \ \ \ \isacommand{apply}\isamarkupfalse%
{\isacharparenleft}{\kern0pt}rule\ iff{\isacharunderscore}{\kern0pt}trans{\isacharcomma}{\kern0pt}\ rule\ sats{\isacharunderscore}{\kern0pt}fun{\isacharunderscore}{\kern0pt}apply{\isacharunderscore}{\kern0pt}fm{\isacharparenright}{\kern0pt}\isanewline
\ \ \ \ \ \ \isacommand{using}\isamarkupfalse%
\ assms{\isadigit{1}}\isanewline
\ \ \ \ \ \ \ \ \ \ \ \ \isacommand{apply}\isamarkupfalse%
\ auto{\isacharbrackleft}{\kern0pt}{\isadigit{4}}{\isacharbrackright}{\kern0pt}\isanewline
\ \ \ \ \ \ \ \ \isacommand{apply}\isamarkupfalse%
\ simp\isanewline
\ \ \ \ \ \ \ \ \isacommand{apply}\isamarkupfalse%
{\isacharparenleft}{\kern0pt}rule\ M{\isacharunderscore}{\kern0pt}basic{\isachardot}{\kern0pt}apply{\isacharunderscore}{\kern0pt}abs{\isacharparenright}{\kern0pt}\isanewline
\ \ \ \ \ \ \isacommand{using}\isamarkupfalse%
\ M{\isacharunderscore}{\kern0pt}ZF{\isacharunderscore}{\kern0pt}trans{\isachardot}{\kern0pt}mbasic\ SymExt{\isacharunderscore}{\kern0pt}M{\isacharunderscore}{\kern0pt}ZF{\isacharunderscore}{\kern0pt}trans\ assms{\isadigit{1}}\isanewline
\ \ \ \ \ \ \ \ \ \ \ \isacommand{apply}\isamarkupfalse%
\ auto{\isacharbrackleft}{\kern0pt}{\isadigit{6}}{\isacharbrackright}{\kern0pt}\isanewline
\ \ \ \ \ \ \isacommand{done}\isamarkupfalse%
\isanewline
\ \ \isacommand{qed}\isamarkupfalse%
\isanewline
\isanewline
\ \ \isacommand{obtain}\isamarkupfalse%
\ G\ \isakeyword{where}\ GH{\isacharcolon}{\kern0pt}\ {\isachardoublequoteopen}M{\isacharunderscore}{\kern0pt}generic{\isacharparenleft}{\kern0pt}G{\isacharparenright}{\kern0pt}{\isachardoublequoteclose}\ {\isachardoublequoteopen}p{\isadigit{0}}\ {\isasymin}\ G{\isachardoublequoteclose}\ \isanewline
\ \ \ \ \isacommand{using}\isamarkupfalse%
\ assms\ generic{\isacharunderscore}{\kern0pt}filter{\isacharunderscore}{\kern0pt}existence\isanewline
\ \ \ \ \isacommand{by}\isamarkupfalse%
\ auto\isanewline
\isanewline
\ \ \isacommand{interpret}\isamarkupfalse%
\ M{\isacharunderscore}{\kern0pt}symmetric{\isacharunderscore}{\kern0pt}system{\isacharunderscore}{\kern0pt}G{\isacharunderscore}{\kern0pt}generic\ \ {\isachardoublequoteopen}Fn{\isachardoublequoteclose}\ {\isachardoublequoteopen}Fn{\isacharunderscore}{\kern0pt}leq{\isachardoublequoteclose}\ {\isachardoublequoteopen}{\isadigit{0}}{\isachardoublequoteclose}\ {\isachardoublequoteopen}M{\isachardoublequoteclose}\ {\isachardoublequoteopen}enum{\isachardoublequoteclose}\ {\isachardoublequoteopen}Fn{\isacharunderscore}{\kern0pt}perms{\isachardoublequoteclose}\ {\isachardoublequoteopen}Fn{\isacharunderscore}{\kern0pt}perms{\isacharunderscore}{\kern0pt}filter{\isachardoublequoteclose}\ G\ \isanewline
\ \ \ \ \isacommand{unfolding}\isamarkupfalse%
\ M{\isacharunderscore}{\kern0pt}symmetric{\isacharunderscore}{\kern0pt}system{\isacharunderscore}{\kern0pt}G{\isacharunderscore}{\kern0pt}generic{\isacharunderscore}{\kern0pt}def\isanewline
\ \ \ \ \isacommand{apply}\isamarkupfalse%
{\isacharparenleft}{\kern0pt}rule\ conjI{\isacharparenright}{\kern0pt}\isanewline
\ \ \ \ \isacommand{using}\isamarkupfalse%
\ M{\isacharunderscore}{\kern0pt}symmetric{\isacharunderscore}{\kern0pt}system{\isacharunderscore}{\kern0pt}axioms\ \isanewline
\ \ \ \ \ \isacommand{apply}\isamarkupfalse%
\ force\isanewline
\ \ \ \ \isacommand{unfolding}\isamarkupfalse%
\ G{\isacharunderscore}{\kern0pt}generic{\isacharunderscore}{\kern0pt}def\isanewline
\ \ \ \ \isacommand{apply}\isamarkupfalse%
{\isacharparenleft}{\kern0pt}rule\ conjI{\isacharparenright}{\kern0pt}\isanewline
\ \ \ \ \isacommand{using}\isamarkupfalse%
\ forcing{\isacharunderscore}{\kern0pt}data{\isacharunderscore}{\kern0pt}axioms\isanewline
\ \ \ \ \ \isacommand{apply}\isamarkupfalse%
\ force\isanewline
\ \ \ \ \isacommand{unfolding}\isamarkupfalse%
\ G{\isacharunderscore}{\kern0pt}generic{\isacharunderscore}{\kern0pt}axioms{\isacharunderscore}{\kern0pt}def\isanewline
\ \ \ \ \isacommand{using}\isamarkupfalse%
\ assms\ GH\isanewline
\ \ \ \ \isacommand{by}\isamarkupfalse%
\ auto\isanewline
\isanewline
\ \ \isacommand{have}\isamarkupfalse%
\ Un{\isacharunderscore}{\kern0pt}subset\ {\isacharcolon}{\kern0pt}\ {\isachardoublequoteopen}{\isasymAnd}A\ B\ C{\isachardot}{\kern0pt}\ A\ {\isasymsubseteq}\ C\ {\isasymLongrightarrow}\ B\ {\isasymsubseteq}\ C\ {\isasymLongrightarrow}\ A\ {\isasymunion}\ B\ {\isasymsubseteq}\ C{\isachardoublequoteclose}\ \isacommand{by}\isamarkupfalse%
\ auto\isanewline
\isanewline
\ \ \isacommand{define}\isamarkupfalse%
\ F\ \isakeyword{where}\ {\isachardoublequoteopen}F\ {\isasymequiv}\ val{\isacharparenleft}{\kern0pt}G{\isacharcomma}{\kern0pt}\ F{\isacharprime}{\kern0pt}{\isacharparenright}{\kern0pt}{\isachardoublequoteclose}\ \isanewline
\isanewline
\ \ \isacommand{have}\isamarkupfalse%
\ p{\isadigit{0}}inFn\ {\isacharcolon}{\kern0pt}{\isachardoublequoteopen}p{\isadigit{0}}\ {\isasymin}\ Fn{\isachardoublequoteclose}\ \isacommand{using}\isamarkupfalse%
\ M{\isacharunderscore}{\kern0pt}genericD\ assms\ \isacommand{by}\isamarkupfalse%
\ auto\isanewline
\isanewline
\ \ \isacommand{obtain}\isamarkupfalse%
\ e\ \isakeyword{where}\ eH{\isacharcolon}{\kern0pt}\ {\isachardoublequoteopen}Fix{\isacharparenleft}{\kern0pt}e{\isacharparenright}{\kern0pt}\ {\isasymsubseteq}\ sym{\isacharparenleft}{\kern0pt}F{\isacharprime}{\kern0pt}{\isacharparenright}{\kern0pt}{\isachardoublequoteclose}\ {\isachardoublequoteopen}e\ {\isasymsubseteq}\ nat{\isachardoublequoteclose}\ {\isachardoublequoteopen}finite{\isacharunderscore}{\kern0pt}M{\isacharparenleft}{\kern0pt}e{\isacharparenright}{\kern0pt}{\isachardoublequoteclose}\ \ \isanewline
\ \ \ \ \isacommand{using}\isamarkupfalse%
\ assms\ HS{\isacharunderscore}{\kern0pt}iff\ symmetric{\isacharunderscore}{\kern0pt}def\ Fn{\isacharunderscore}{\kern0pt}perms{\isacharunderscore}{\kern0pt}filter{\isacharunderscore}{\kern0pt}def\ \isanewline
\ \ \ \ \isacommand{by}\isamarkupfalse%
\ force\isanewline
\ \ \isacommand{then}\isamarkupfalse%
\ \isacommand{have}\isamarkupfalse%
\ e{\isacharunderscore}{\kern0pt}Finite\ {\isacharcolon}{\kern0pt}\ {\isachardoublequoteopen}Finite{\isacharparenleft}{\kern0pt}e{\isacharparenright}{\kern0pt}{\isachardoublequoteclose}\ \isanewline
\ \ \ \ \isacommand{using}\isamarkupfalse%
\ finite{\isacharunderscore}{\kern0pt}M{\isacharunderscore}{\kern0pt}implies{\isacharunderscore}{\kern0pt}Finite\isanewline
\ \ \ \ \isacommand{by}\isamarkupfalse%
\ auto\isanewline
\isanewline
\ \ \isacommand{have}\isamarkupfalse%
\ listin\ {\isacharcolon}{\kern0pt}\ {\isachardoublequoteopen}{\isacharbrackleft}{\kern0pt}check{\isacharparenleft}{\kern0pt}nat{\isacharparenright}{\kern0pt}{\isacharcomma}{\kern0pt}\ binmap{\isacharprime}{\kern0pt}{\isacharcomma}{\kern0pt}\ F{\isacharprime}{\kern0pt}{\isacharbrackright}{\kern0pt}\ {\isasymin}\ list{\isacharparenleft}{\kern0pt}HS{\isacharparenright}{\kern0pt}{\isachardoublequoteclose}\ \isanewline
\ \ \ \ \isacommand{using}\isamarkupfalse%
\ check{\isacharunderscore}{\kern0pt}in{\isacharunderscore}{\kern0pt}HS\ nat{\isacharunderscore}{\kern0pt}in{\isacharunderscore}{\kern0pt}M\ assms\ binmap{\isacharprime}{\kern0pt}{\isacharunderscore}{\kern0pt}HS\isanewline
\ \ \ \ \isacommand{by}\isamarkupfalse%
\ auto\isanewline
\ \ \isacommand{then}\isamarkupfalse%
\ \isacommand{have}\isamarkupfalse%
\ mapin\ {\isacharcolon}{\kern0pt}\ {\isachardoublequoteopen}map{\isacharparenleft}{\kern0pt}val{\isacharparenleft}{\kern0pt}G{\isacharparenright}{\kern0pt}{\isacharcomma}{\kern0pt}\ {\isacharbrackleft}{\kern0pt}check{\isacharparenleft}{\kern0pt}nat{\isacharparenright}{\kern0pt}{\isacharcomma}{\kern0pt}\ binmap{\isacharprime}{\kern0pt}{\isacharcomma}{\kern0pt}\ F{\isacharprime}{\kern0pt}{\isacharbrackright}{\kern0pt}{\isacharparenright}{\kern0pt}\ {\isasymin}\ list{\isacharparenleft}{\kern0pt}SymExt{\isacharparenleft}{\kern0pt}G{\isacharparenright}{\kern0pt}{\isacharparenright}{\kern0pt}{\isachardoublequoteclose}\ \isanewline
\ \ \ \ \isacommand{apply}\isamarkupfalse%
\ simp\isanewline
\ \ \ \ \isacommand{using}\isamarkupfalse%
\ SymExt{\isacharunderscore}{\kern0pt}def\isanewline
\ \ \ \ \isacommand{by}\isamarkupfalse%
\ auto\isanewline
\isanewline
\ \ \isacommand{have}\isamarkupfalse%
\ {\isachardoublequoteopen}SymExt{\isacharparenleft}{\kern0pt}G{\isacharparenright}{\kern0pt}{\isacharcomma}{\kern0pt}\ map{\isacharparenleft}{\kern0pt}val{\isacharparenleft}{\kern0pt}G{\isacharparenright}{\kern0pt}{\isacharcomma}{\kern0pt}\ {\isacharbrackleft}{\kern0pt}check{\isacharparenleft}{\kern0pt}nat{\isacharparenright}{\kern0pt}{\isacharcomma}{\kern0pt}\ binmap{\isacharprime}{\kern0pt}{\isacharcomma}{\kern0pt}\ F{\isacharprime}{\kern0pt}{\isacharbrackright}{\kern0pt}{\isacharparenright}{\kern0pt}\ {\isasymTurnstile}\ injection{\isacharunderscore}{\kern0pt}fm{\isacharparenleft}{\kern0pt}{\isadigit{0}}{\isacharcomma}{\kern0pt}\ {\isadigit{1}}{\isacharcomma}{\kern0pt}\ {\isadigit{2}}{\isacharparenright}{\kern0pt}{\isachardoublequoteclose}\ \isanewline
\ \ \ \ \isacommand{apply}\isamarkupfalse%
{\isacharparenleft}{\kern0pt}rule\ iffD{\isadigit{1}}{\isacharcomma}{\kern0pt}\ rule\ HS{\isacharunderscore}{\kern0pt}truth{\isacharunderscore}{\kern0pt}lemma{\isacharparenright}{\kern0pt}\isanewline
\ \ \ \ \isacommand{using}\isamarkupfalse%
\ assms\ GH\ check{\isacharunderscore}{\kern0pt}in{\isacharunderscore}{\kern0pt}HS\ nat{\isacharunderscore}{\kern0pt}in{\isacharunderscore}{\kern0pt}M\ binmap{\isacharprime}{\kern0pt}{\isacharunderscore}{\kern0pt}HS\isanewline
\ \ \ \ \ \ \ \ \isacommand{apply}\isamarkupfalse%
\ auto{\isacharbrackleft}{\kern0pt}{\isadigit{3}}{\isacharbrackright}{\kern0pt}\isanewline
\ \ \ \ \ \isacommand{apply}\isamarkupfalse%
{\isacharparenleft}{\kern0pt}subst\ injection{\isacharunderscore}{\kern0pt}fm{\isacharunderscore}{\kern0pt}def{\isacharparenright}{\kern0pt}\isanewline
\ \ \ \ \ \isacommand{apply}\isamarkupfalse%
\ {\isacharparenleft}{\kern0pt}simp\ del{\isacharcolon}{\kern0pt}FOL{\isacharunderscore}{\kern0pt}sats{\isacharunderscore}{\kern0pt}iff\ pair{\isacharunderscore}{\kern0pt}abs\ add{\isacharcolon}{\kern0pt}\ fm{\isacharunderscore}{\kern0pt}defs\ nat{\isacharunderscore}{\kern0pt}simp{\isacharunderscore}{\kern0pt}union{\isacharparenright}{\kern0pt}\isanewline
\ \ \ \ \isacommand{using}\isamarkupfalse%
\ assms\ GH\isanewline
\ \ \ \ \isacommand{by}\isamarkupfalse%
\ auto\isanewline
\ \ \isacommand{then}\isamarkupfalse%
\ \isacommand{have}\isamarkupfalse%
\ {\isachardoublequoteopen}injection{\isacharparenleft}{\kern0pt}{\isacharhash}{\kern0pt}{\isacharhash}{\kern0pt}SymExt{\isacharparenleft}{\kern0pt}G{\isacharparenright}{\kern0pt}{\isacharcomma}{\kern0pt}\ val{\isacharparenleft}{\kern0pt}G{\isacharcomma}{\kern0pt}\ check{\isacharparenleft}{\kern0pt}nat{\isacharparenright}{\kern0pt}{\isacharparenright}{\kern0pt}{\isacharcomma}{\kern0pt}\ val{\isacharparenleft}{\kern0pt}G{\isacharcomma}{\kern0pt}\ binmap{\isacharprime}{\kern0pt}{\isacharparenright}{\kern0pt}{\isacharcomma}{\kern0pt}\ F{\isacharparenright}{\kern0pt}{\isachardoublequoteclose}\ \isanewline
\ \ \ \ \isacommand{using}\isamarkupfalse%
\ sats{\isacharunderscore}{\kern0pt}bijection{\isacharunderscore}{\kern0pt}fm\ mapin\ F{\isacharunderscore}{\kern0pt}def\isanewline
\ \ \ \ \isacommand{by}\isamarkupfalse%
\ auto\isanewline
\ \ \isacommand{then}\isamarkupfalse%
\ \isacommand{have}\isamarkupfalse%
\ {\isachardoublequoteopen}injection{\isacharparenleft}{\kern0pt}{\isacharhash}{\kern0pt}{\isacharhash}{\kern0pt}SymExt{\isacharparenleft}{\kern0pt}G{\isacharparenright}{\kern0pt}{\isacharcomma}{\kern0pt}\ nat{\isacharcomma}{\kern0pt}\ binmap{\isacharparenleft}{\kern0pt}G{\isacharparenright}{\kern0pt}{\isacharcomma}{\kern0pt}\ F{\isacharparenright}{\kern0pt}{\isachardoublequoteclose}\ \isanewline
\ \ \ \ \isacommand{apply}\isamarkupfalse%
{\isacharparenleft}{\kern0pt}subgoal{\isacharunderscore}{\kern0pt}tac\ {\isachardoublequoteopen}val{\isacharparenleft}{\kern0pt}G{\isacharcomma}{\kern0pt}\ check{\isacharparenleft}{\kern0pt}nat{\isacharparenright}{\kern0pt}{\isacharparenright}{\kern0pt}\ {\isacharequal}{\kern0pt}\ nat\ {\isasymand}\ val{\isacharparenleft}{\kern0pt}G{\isacharcomma}{\kern0pt}\ binmap{\isacharprime}{\kern0pt}{\isacharparenright}{\kern0pt}\ {\isacharequal}{\kern0pt}\ binmap{\isacharparenleft}{\kern0pt}G{\isacharparenright}{\kern0pt}{\isachardoublequoteclose}{\isacharparenright}{\kern0pt}\isanewline
\ \ \ \ \ \isacommand{apply}\isamarkupfalse%
\ force\isanewline
\ \ \ \ \isacommand{apply}\isamarkupfalse%
{\isacharparenleft}{\kern0pt}rule\ conjI{\isacharparenright}{\kern0pt}\isanewline
\ \ \ \ \ \isacommand{apply}\isamarkupfalse%
{\isacharparenleft}{\kern0pt}rule\ valcheck{\isacharparenright}{\kern0pt}\isanewline
\ \ \ \ \isacommand{using}\isamarkupfalse%
\ generic{\isacharunderscore}{\kern0pt}filter{\isacharunderscore}{\kern0pt}contains{\isacharunderscore}{\kern0pt}max\ assms\ GH\ zero{\isacharunderscore}{\kern0pt}in{\isacharunderscore}{\kern0pt}Fn\ binmap{\isacharunderscore}{\kern0pt}eq\isanewline
\ \ \ \ \isacommand{by}\isamarkupfalse%
\ auto\isanewline
\ \ \isacommand{then}\isamarkupfalse%
\ \isacommand{have}\isamarkupfalse%
\ Finj{\isacharcolon}{\kern0pt}\ {\isachardoublequoteopen}F\ {\isasymin}\ inj{\isacharparenleft}{\kern0pt}nat{\isacharcomma}{\kern0pt}\ binmap{\isacharparenleft}{\kern0pt}G{\isacharparenright}{\kern0pt}{\isacharparenright}{\kern0pt}{\isachardoublequoteclose}\isanewline
\ \ \ \ \isacommand{apply}\isamarkupfalse%
{\isacharparenleft}{\kern0pt}rule{\isacharunderscore}{\kern0pt}tac\ iffD{\isadigit{1}}{\isacharparenright}{\kern0pt}\isanewline
\ \ \ \ \ \isacommand{apply}\isamarkupfalse%
{\isacharparenleft}{\kern0pt}rule\ M{\isacharunderscore}{\kern0pt}basic{\isachardot}{\kern0pt}injection{\isacharunderscore}{\kern0pt}abs{\isacharparenright}{\kern0pt}\isanewline
\ \ \ \ \isacommand{using}\isamarkupfalse%
\ SymExt{\isacharunderscore}{\kern0pt}M{\isacharunderscore}{\kern0pt}ZF{\isacharunderscore}{\kern0pt}trans\ M{\isacharunderscore}{\kern0pt}ZF{\isacharunderscore}{\kern0pt}trans{\isachardot}{\kern0pt}mbasic\isanewline
\ \ \ \ \ \ \ \ \isacommand{apply}\isamarkupfalse%
\ force\isanewline
\ \ \ \ \isacommand{using}\isamarkupfalse%
\ GH\ M{\isacharunderscore}{\kern0pt}subset{\isacharunderscore}{\kern0pt}SymExt\ nat{\isacharunderscore}{\kern0pt}in{\isacharunderscore}{\kern0pt}M\ binmap{\isacharunderscore}{\kern0pt}eq\ assms\ SymExt{\isacharunderscore}{\kern0pt}def\ binmap{\isacharprime}{\kern0pt}{\isacharunderscore}{\kern0pt}HS\ F{\isacharunderscore}{\kern0pt}def\isanewline
\ \ \ \ \isacommand{by}\isamarkupfalse%
\ auto\isanewline
\isanewline
\ \ \isacommand{have}\isamarkupfalse%
\ {\isachardoublequoteopen}F\ {\isasymin}\ bij{\isacharparenleft}{\kern0pt}nat{\isacharcomma}{\kern0pt}\ range{\isacharparenleft}{\kern0pt}F{\isacharparenright}{\kern0pt}{\isacharparenright}{\kern0pt}{\isachardoublequoteclose}\ \isanewline
\ \ \ \ \isacommand{apply}\isamarkupfalse%
{\isacharparenleft}{\kern0pt}rule\ inj{\isacharunderscore}{\kern0pt}bij{\isacharunderscore}{\kern0pt}range{\isacharparenright}{\kern0pt}\isanewline
\ \ \ \ \isacommand{using}\isamarkupfalse%
\ Finj\ \isanewline
\ \ \ \ \isacommand{by}\isamarkupfalse%
\ auto\isanewline
\ \ \isacommand{then}\isamarkupfalse%
\ \isacommand{have}\isamarkupfalse%
\ rangeinfinite{\isacharcolon}{\kern0pt}\ {\isachardoublequoteopen}{\isasymnot}Finite{\isacharparenleft}{\kern0pt}range{\isacharparenleft}{\kern0pt}F{\isacharparenright}{\kern0pt}{\isacharparenright}{\kern0pt}{\isachardoublequoteclose}\ \isanewline
\ \ \ \ \isacommand{apply}\isamarkupfalse%
{\isacharparenleft}{\kern0pt}rule{\isacharunderscore}{\kern0pt}tac\ ccontr{\isacharparenright}{\kern0pt}\isanewline
\ \ \ \ \isacommand{apply}\isamarkupfalse%
{\isacharparenleft}{\kern0pt}subgoal{\isacharunderscore}{\kern0pt}tac\ {\isachardoublequoteopen}Finite{\isacharparenleft}{\kern0pt}nat{\isacharparenright}{\kern0pt}{\isachardoublequoteclose}{\isacharparenright}{\kern0pt}\isanewline
\ \ \ \ \isacommand{using}\isamarkupfalse%
\ nat{\isacharunderscore}{\kern0pt}not{\isacharunderscore}{\kern0pt}Finite\isanewline
\ \ \ \ \ \isacommand{apply}\isamarkupfalse%
\ simp\isanewline
\ \ \ \ \isacommand{apply}\isamarkupfalse%
{\isacharparenleft}{\kern0pt}rule\ iffD{\isadigit{2}}{\isacharcomma}{\kern0pt}\ rule{\isacharunderscore}{\kern0pt}tac\ B{\isacharequal}{\kern0pt}{\isachardoublequoteopen}range{\isacharparenleft}{\kern0pt}F{\isacharparenright}{\kern0pt}{\isachardoublequoteclose}\ \isakeyword{in}\ eqpoll{\isacharunderscore}{\kern0pt}imp{\isacharunderscore}{\kern0pt}Finite{\isacharunderscore}{\kern0pt}iff{\isacharparenright}{\kern0pt}\isanewline
\ \ \ \ \isacommand{using}\isamarkupfalse%
\ eqpoll{\isacharunderscore}{\kern0pt}def\ \isanewline
\ \ \ \ \isacommand{by}\isamarkupfalse%
\ auto\isanewline
\ \ \isacommand{have}\isamarkupfalse%
\ {\isachardoublequoteopen}{\isasymnot}Finite{\isacharparenleft}{\kern0pt}{\isacharbraceleft}{\kern0pt}\ n\ {\isasymin}\ nat{\isachardot}{\kern0pt}\ binmap{\isacharunderscore}{\kern0pt}row{\isacharparenleft}{\kern0pt}G{\isacharcomma}{\kern0pt}\ n{\isacharparenright}{\kern0pt}\ {\isasymin}\ range{\isacharparenleft}{\kern0pt}F{\isacharparenright}{\kern0pt}\ {\isacharbraceright}{\kern0pt}{\isacharparenright}{\kern0pt}{\isachardoublequoteclose}\ {\isacharparenleft}{\kern0pt}\isakeyword{is}\ {\isachardoublequoteopen}{\isasymnot}Finite{\isacharparenleft}{\kern0pt}{\isacharquery}{\kern0pt}RangeIndexes{\isacharparenright}{\kern0pt}{\isachardoublequoteclose}{\isacharparenright}{\kern0pt}\isanewline
\ \ \isacommand{proof}\isamarkupfalse%
\ {\isacharparenleft}{\kern0pt}rule\ ccontr{\isacharparenright}{\kern0pt}\isanewline
\ \ \ \ \isacommand{assume}\isamarkupfalse%
\ {\isachardoublequoteopen}{\isasymnot}\ {\isasymnot}\ Finite{\isacharparenleft}{\kern0pt}{\isacharquery}{\kern0pt}RangeIndexes{\isacharparenright}{\kern0pt}{\isachardoublequoteclose}\ \isanewline
\ \ \ \ \isacommand{then}\isamarkupfalse%
\ \isacommand{have}\isamarkupfalse%
\ H{\isacharcolon}{\kern0pt}\ {\isachardoublequoteopen}Finite{\isacharparenleft}{\kern0pt}{\isacharbraceleft}{\kern0pt}\ binmap{\isacharunderscore}{\kern0pt}row{\isacharparenleft}{\kern0pt}G{\isacharcomma}{\kern0pt}\ n{\isacharparenright}{\kern0pt}{\isachardot}{\kern0pt}\ n\ {\isasymin}\ {\isacharquery}{\kern0pt}RangeIndexes\ {\isacharbraceright}{\kern0pt}{\isacharparenright}{\kern0pt}{\isachardoublequoteclose}\ \isanewline
\ \ \ \ \ \ \isacommand{apply}\isamarkupfalse%
{\isacharparenleft}{\kern0pt}rule{\isacharunderscore}{\kern0pt}tac\ Finite{\isacharunderscore}{\kern0pt}RepFun{\isacharparenright}{\kern0pt}\isanewline
\ \ \ \ \ \ \isacommand{by}\isamarkupfalse%
\ auto\isanewline
\ \ \ \ \isacommand{have}\isamarkupfalse%
\ {\isachardoublequoteopen}{\isacharbraceleft}{\kern0pt}\ binmap{\isacharunderscore}{\kern0pt}row{\isacharparenleft}{\kern0pt}G{\isacharcomma}{\kern0pt}\ n{\isacharparenright}{\kern0pt}{\isachardot}{\kern0pt}\ n\ {\isasymin}\ {\isacharquery}{\kern0pt}RangeIndexes\ {\isacharbraceright}{\kern0pt}\ {\isacharequal}{\kern0pt}\ range{\isacharparenleft}{\kern0pt}F{\isacharparenright}{\kern0pt}{\isachardoublequoteclose}\ \isanewline
\ \ \ \ \isacommand{proof}\isamarkupfalse%
{\isacharparenleft}{\kern0pt}rule\ equality{\isacharunderscore}{\kern0pt}iffI{\isacharcomma}{\kern0pt}\ rule\ iffI{\isacharcomma}{\kern0pt}\ force{\isacharcomma}{\kern0pt}\ clarsimp{\isacharparenright}{\kern0pt}\isanewline
\ \ \ \ \ \ \isacommand{fix}\isamarkupfalse%
\ n\ x\ \isanewline
\ \ \ \ \ \ \isacommand{assume}\isamarkupfalse%
\ {\isachardoublequoteopen}{\isacharless}{\kern0pt}n{\isacharcomma}{\kern0pt}\ x{\isachargreater}{\kern0pt}\ {\isasymin}\ F{\isachardoublequoteclose}\ \isanewline
\ \ \ \ \ \ \isacommand{then}\isamarkupfalse%
\ \isacommand{obtain}\isamarkupfalse%
\ m\ \isakeyword{where}\ {\isachardoublequoteopen}m\ {\isasymin}\ nat{\isachardoublequoteclose}\ {\isachardoublequoteopen}x\ {\isacharequal}{\kern0pt}\ binmap{\isacharunderscore}{\kern0pt}row{\isacharparenleft}{\kern0pt}G{\isacharcomma}{\kern0pt}\ m{\isacharparenright}{\kern0pt}{\isachardoublequoteclose}\ \isacommand{using}\isamarkupfalse%
\ Finj\ inj{\isacharunderscore}{\kern0pt}def\ Pi{\isacharunderscore}{\kern0pt}def\ binmap{\isacharunderscore}{\kern0pt}def\ \isacommand{by}\isamarkupfalse%
\ auto\isanewline
\ \ \ \ \ \ \isacommand{then}\isamarkupfalse%
\ \isacommand{show}\isamarkupfalse%
\ {\isachardoublequoteopen}{\isasymexists}m{\isasymin}nat{\isachardot}{\kern0pt}\ binmap{\isacharunderscore}{\kern0pt}row{\isacharparenleft}{\kern0pt}G{\isacharcomma}{\kern0pt}\ m{\isacharparenright}{\kern0pt}\ {\isasymin}\ range{\isacharparenleft}{\kern0pt}F{\isacharparenright}{\kern0pt}\ {\isasymand}\ x\ {\isacharequal}{\kern0pt}\ binmap{\isacharunderscore}{\kern0pt}row{\isacharparenleft}{\kern0pt}G{\isacharcomma}{\kern0pt}\ m{\isacharparenright}{\kern0pt}{\isachardoublequoteclose}\isanewline
\ \ \ \ \ \ \ \ \isacommand{using}\isamarkupfalse%
\ {\isacartoucheopen}{\isacharless}{\kern0pt}n{\isacharcomma}{\kern0pt}\ x{\isachargreater}{\kern0pt}\ {\isasymin}\ F{\isacartoucheclose}\isanewline
\ \ \ \ \ \ \ \ \isacommand{by}\isamarkupfalse%
\ auto\isanewline
\ \ \ \ \isacommand{qed}\isamarkupfalse%
\isanewline
\ \ \ \ \isacommand{then}\isamarkupfalse%
\ \isacommand{have}\isamarkupfalse%
\ {\isachardoublequoteopen}Finite{\isacharparenleft}{\kern0pt}range{\isacharparenleft}{\kern0pt}F{\isacharparenright}{\kern0pt}{\isacharparenright}{\kern0pt}{\isachardoublequoteclose}\ \isanewline
\ \ \ \ \ \ \isacommand{using}\isamarkupfalse%
\ H\ \isacommand{by}\isamarkupfalse%
\ auto\isanewline
\ \ \ \ \isacommand{then}\isamarkupfalse%
\ \isacommand{show}\isamarkupfalse%
\ False\ \isacommand{using}\isamarkupfalse%
\ rangeinfinite\ \isacommand{by}\isamarkupfalse%
\ auto\isanewline
\ \ \isacommand{qed}\isamarkupfalse%
\isanewline
\isanewline
\ \ \isacommand{then}\isamarkupfalse%
\ \isacommand{have}\isamarkupfalse%
\ {\isachardoublequoteopen}{\isacharquery}{\kern0pt}RangeIndexes\ {\isacharminus}{\kern0pt}\ e\ {\isasymnoteq}\ {\isadigit{0}}{\isachardoublequoteclose}\ \isanewline
\ \ \ \ \isacommand{apply}\isamarkupfalse%
{\isacharparenleft}{\kern0pt}subgoal{\isacharunderscore}{\kern0pt}tac\ {\isachardoublequoteopen}{\isasymnot}Finite{\isacharparenleft}{\kern0pt}{\isacharquery}{\kern0pt}RangeIndexes\ {\isacharminus}{\kern0pt}\ e{\isacharparenright}{\kern0pt}{\isachardoublequoteclose}{\isacharparenright}{\kern0pt}\isanewline
\ \ \ \ \ \isacommand{apply}\isamarkupfalse%
{\isacharparenleft}{\kern0pt}rule\ ccontr{\isacharparenright}{\kern0pt}\isanewline
\ \ \ \ \isacommand{using}\isamarkupfalse%
\ Finite{\isacharunderscore}{\kern0pt}{\isadigit{0}}\isanewline
\ \ \ \ \ \isacommand{apply}\isamarkupfalse%
\ force\isanewline
\ \ \ \ \isacommand{apply}\isamarkupfalse%
{\isacharparenleft}{\kern0pt}rule\ ccontr{\isacharcomma}{\kern0pt}\ simp{\isacharparenright}{\kern0pt}\isanewline
\ \ \ \ \isacommand{apply}\isamarkupfalse%
{\isacharparenleft}{\kern0pt}insert\ Diff{\isacharunderscore}{\kern0pt}Finite\ {\isacharbrackleft}{\kern0pt}of\ e\ {\isacharquery}{\kern0pt}RangeIndexes{\isacharbrackright}{\kern0pt}{\isacharparenright}{\kern0pt}\isanewline
\ \ \ \ \isacommand{using}\isamarkupfalse%
\ e{\isacharunderscore}{\kern0pt}Finite\ \isanewline
\ \ \ \ \isacommand{by}\isamarkupfalse%
\ auto\isanewline
\ \ \isacommand{then}\isamarkupfalse%
\ \isacommand{obtain}\isamarkupfalse%
\ n\ \isakeyword{where}\ nH{\isacharcolon}{\kern0pt}\ {\isachardoublequoteopen}n\ {\isasymnotin}\ e{\isachardoublequoteclose}\ {\isachardoublequoteopen}n\ {\isasymin}\ {\isacharquery}{\kern0pt}RangeIndexes{\isachardoublequoteclose}\ {\isachardoublequoteopen}n\ {\isasymin}\ nat{\isachardoublequoteclose}\ \isacommand{using}\isamarkupfalse%
\ e{\isacharunderscore}{\kern0pt}Finite\ eH\ \isacommand{by}\isamarkupfalse%
\ blast\isanewline
\ \ \isacommand{then}\isamarkupfalse%
\ \isacommand{obtain}\isamarkupfalse%
\ i\ \isakeyword{where}\ iH{\isacharcolon}{\kern0pt}\ {\isachardoublequoteopen}{\isacharless}{\kern0pt}i{\isacharcomma}{\kern0pt}\ binmap{\isacharunderscore}{\kern0pt}row{\isacharparenleft}{\kern0pt}G{\isacharcomma}{\kern0pt}\ n{\isacharparenright}{\kern0pt}{\isachargreater}{\kern0pt}\ {\isasymin}\ F{\isachardoublequoteclose}\ {\isachardoublequoteopen}i\ {\isasymin}\ nat{\isachardoublequoteclose}\ \isacommand{using}\isamarkupfalse%
\ Finj\ inj{\isacharunderscore}{\kern0pt}def\ Pi{\isacharunderscore}{\kern0pt}def\ \isacommand{by}\isamarkupfalse%
\ auto\isanewline
\ \ \isacommand{then}\isamarkupfalse%
\ \isacommand{have}\isamarkupfalse%
\ iH{\isacharprime}{\kern0pt}{\isacharcolon}{\kern0pt}\ {\isachardoublequoteopen}F{\isacharbackquote}{\kern0pt}i\ {\isacharequal}{\kern0pt}\ binmap{\isacharunderscore}{\kern0pt}row{\isacharparenleft}{\kern0pt}G{\isacharcomma}{\kern0pt}\ n{\isacharparenright}{\kern0pt}{\isachardoublequoteclose}\ \isanewline
\ \ \ \ \isacommand{apply}\isamarkupfalse%
{\isacharparenleft}{\kern0pt}rule{\isacharunderscore}{\kern0pt}tac\ function{\isacharunderscore}{\kern0pt}apply{\isacharunderscore}{\kern0pt}equality{\isacharparenright}{\kern0pt}\isanewline
\ \ \ \ \isacommand{using}\isamarkupfalse%
\ Finj\ inj{\isacharunderscore}{\kern0pt}def\ Pi{\isacharunderscore}{\kern0pt}def\isanewline
\ \ \ \ \isacommand{by}\isamarkupfalse%
\ auto\isanewline
\isanewline
\ \ \isacommand{have}\isamarkupfalse%
\ binmapin\ {\isacharcolon}{\kern0pt}\ {\isachardoublequoteopen}binmap{\isacharparenleft}{\kern0pt}G{\isacharparenright}{\kern0pt}\ {\isasymin}\ SymExt{\isacharparenleft}{\kern0pt}G{\isacharparenright}{\kern0pt}{\isachardoublequoteclose}\ \isanewline
\ \ \ \ \ \ \isacommand{apply}\isamarkupfalse%
{\isacharparenleft}{\kern0pt}subst\ binmap{\isacharunderscore}{\kern0pt}eq{\isacharparenright}{\kern0pt}\isanewline
\ \ \ \ \isacommand{using}\isamarkupfalse%
\ assms\ mapin\ GH\isanewline
\ \ \ \ \isacommand{by}\isamarkupfalse%
\ auto\isanewline
\isanewline
\ \ \isacommand{have}\isamarkupfalse%
\ rowin\ {\isacharcolon}{\kern0pt}\ {\isachardoublequoteopen}binmap{\isacharunderscore}{\kern0pt}row{\isacharparenleft}{\kern0pt}G{\isacharcomma}{\kern0pt}\ n{\isacharparenright}{\kern0pt}\ {\isasymin}\ SymExt{\isacharparenleft}{\kern0pt}G{\isacharparenright}{\kern0pt}{\isachardoublequoteclose}\ \isanewline
\ \ \ \ \isacommand{unfolding}\isamarkupfalse%
\ SymExt{\isacharunderscore}{\kern0pt}def\ \isanewline
\ \ \ \ \isacommand{apply}\isamarkupfalse%
{\isacharparenleft}{\kern0pt}subst\ binmap{\isacharunderscore}{\kern0pt}row{\isacharunderscore}{\kern0pt}eq{\isacharparenright}{\kern0pt}\isanewline
\ \ \ \ \isacommand{using}\isamarkupfalse%
\ nH\ assms\ GH\isanewline
\ \ \ \ \ \ \isacommand{apply}\isamarkupfalse%
\ auto{\isacharbrackleft}{\kern0pt}{\isadigit{2}}{\isacharbrackright}{\kern0pt}\isanewline
\ \ \ \ \isacommand{apply}\isamarkupfalse%
\ simp\isanewline
\ \ \ \ \isacommand{apply}\isamarkupfalse%
{\isacharparenleft}{\kern0pt}rule{\isacharunderscore}{\kern0pt}tac\ x{\isacharequal}{\kern0pt}{\isachardoublequoteopen}binmap{\isacharunderscore}{\kern0pt}row{\isacharprime}{\kern0pt}{\isacharparenleft}{\kern0pt}n{\isacharparenright}{\kern0pt}{\isachardoublequoteclose}\ \isakeyword{in}\ bexI{\isacharparenright}{\kern0pt}\isanewline
\ \ \ \ \isacommand{using}\isamarkupfalse%
\ binmap{\isacharunderscore}{\kern0pt}row{\isacharprime}{\kern0pt}{\isacharunderscore}{\kern0pt}HS\ nH\isanewline
\ \ \ \ \isacommand{by}\isamarkupfalse%
\ auto\isanewline
\isanewline
\ \ \isacommand{then}\isamarkupfalse%
\ \isacommand{have}\isamarkupfalse%
\ listin{\isacharprime}{\kern0pt}\ {\isacharcolon}{\kern0pt}\ {\isachardoublequoteopen}{\isacharbrackleft}{\kern0pt}F{\isacharcomma}{\kern0pt}\ i{\isacharcomma}{\kern0pt}\ binmap{\isacharunderscore}{\kern0pt}row{\isacharparenleft}{\kern0pt}G{\isacharcomma}{\kern0pt}\ n{\isacharparenright}{\kern0pt}{\isacharbrackright}{\kern0pt}\ {\isasymin}\ list{\isacharparenleft}{\kern0pt}SymExt{\isacharparenleft}{\kern0pt}G{\isacharparenright}{\kern0pt}{\isacharparenright}{\kern0pt}{\isachardoublequoteclose}\ \isanewline
\ \ \ \ \isacommand{apply}\isamarkupfalse%
\ auto\isanewline
\ \ \ \ \isacommand{using}\isamarkupfalse%
\ F{\isacharunderscore}{\kern0pt}def\ SymExt{\isacharunderscore}{\kern0pt}def\ assms\isanewline
\ \ \ \ \ \isacommand{apply}\isamarkupfalse%
\ force\isanewline
\ \ \ \ \isacommand{using}\isamarkupfalse%
\ iH\ transM\ nat{\isacharunderscore}{\kern0pt}in{\isacharunderscore}{\kern0pt}M\ M{\isacharunderscore}{\kern0pt}subset{\isacharunderscore}{\kern0pt}SymExt\isanewline
\ \ \ \ \isacommand{by}\isamarkupfalse%
\ force\isanewline
\isanewline
\ \ \isacommand{have}\isamarkupfalse%
\ mapeq{\isacharcolon}{\kern0pt}\ {\isachardoublequoteopen}map{\isacharparenleft}{\kern0pt}val{\isacharparenleft}{\kern0pt}G{\isacharparenright}{\kern0pt}{\isacharcomma}{\kern0pt}\ {\isacharbrackleft}{\kern0pt}F{\isacharprime}{\kern0pt}{\isacharcomma}{\kern0pt}\ check{\isacharparenleft}{\kern0pt}i{\isacharparenright}{\kern0pt}{\isacharcomma}{\kern0pt}\ binmap{\isacharunderscore}{\kern0pt}row{\isacharprime}{\kern0pt}{\isacharparenleft}{\kern0pt}n{\isacharparenright}{\kern0pt}{\isacharbrackright}{\kern0pt}{\isacharparenright}{\kern0pt}\ {\isacharequal}{\kern0pt}\ {\isacharbrackleft}{\kern0pt}F{\isacharcomma}{\kern0pt}\ i{\isacharcomma}{\kern0pt}\ binmap{\isacharunderscore}{\kern0pt}row{\isacharparenleft}{\kern0pt}G{\isacharcomma}{\kern0pt}\ n{\isacharparenright}{\kern0pt}{\isacharbrackright}{\kern0pt}{\isachardoublequoteclose}\isanewline
\ \ \ \ \isacommand{using}\isamarkupfalse%
\ F{\isacharunderscore}{\kern0pt}def\isanewline
\ \ \ \ \isacommand{apply}\isamarkupfalse%
\ auto\isanewline
\ \ \ \ \ \isacommand{apply}\isamarkupfalse%
{\isacharparenleft}{\kern0pt}rule\ valcheck{\isacharparenright}{\kern0pt}\isanewline
\ \ \ \ \isacommand{using}\isamarkupfalse%
\ generic{\isacharunderscore}{\kern0pt}filter{\isacharunderscore}{\kern0pt}contains{\isacharunderscore}{\kern0pt}max\ assms\ zero{\isacharunderscore}{\kern0pt}in{\isacharunderscore}{\kern0pt}Fn\ binmap{\isacharunderscore}{\kern0pt}row{\isacharunderscore}{\kern0pt}eq\ nH\ GH\isanewline
\ \ \ \ \isacommand{by}\isamarkupfalse%
\ auto\isanewline
\isanewline
\ \ \isacommand{have}\isamarkupfalse%
\ {\isachardoublequoteopen}F{\isacharbackquote}{\kern0pt}i\ {\isasymin}\ binmap{\isacharparenleft}{\kern0pt}G{\isacharparenright}{\kern0pt}{\isachardoublequoteclose}\ \isanewline
\ \ \ \ \isacommand{apply}\isamarkupfalse%
{\isacharparenleft}{\kern0pt}rule\ function{\isacharunderscore}{\kern0pt}value{\isacharunderscore}{\kern0pt}in{\isacharparenright}{\kern0pt}\isanewline
\ \ \ \ \isacommand{using}\isamarkupfalse%
\ Finj\ bij{\isacharunderscore}{\kern0pt}def\ inj{\isacharunderscore}{\kern0pt}def\ iH\isanewline
\ \ \ \ \isacommand{by}\isamarkupfalse%
\ auto\isanewline
\ \ \isacommand{then}\isamarkupfalse%
\ \isacommand{have}\isamarkupfalse%
\ appin\ {\isacharcolon}{\kern0pt}\ {\isachardoublequoteopen}F{\isacharbackquote}{\kern0pt}i\ {\isasymin}\ SymExt{\isacharparenleft}{\kern0pt}G{\isacharparenright}{\kern0pt}{\isachardoublequoteclose}\ \isanewline
\ \ \ \ \isacommand{using}\isamarkupfalse%
\ mapin\ binmapin\ SymExt{\isacharunderscore}{\kern0pt}trans\isanewline
\ \ \ \ \isacommand{by}\isamarkupfalse%
\ auto\isanewline
\isanewline
\ \ \isacommand{have}\isamarkupfalse%
\ {\isachardoublequoteopen}sats{\isacharparenleft}{\kern0pt}SymExt{\isacharparenleft}{\kern0pt}G{\isacharparenright}{\kern0pt}{\isacharcomma}{\kern0pt}\ {\isasymphi}{\isacharcomma}{\kern0pt}\ {\isacharbrackleft}{\kern0pt}F{\isacharcomma}{\kern0pt}\ i{\isacharcomma}{\kern0pt}\ binmap{\isacharunderscore}{\kern0pt}row{\isacharparenleft}{\kern0pt}G{\isacharcomma}{\kern0pt}\ n{\isacharparenright}{\kern0pt}{\isacharbrackright}{\kern0pt}{\isacharparenright}{\kern0pt}{\isachardoublequoteclose}\isanewline
\ \ \ \ \isacommand{apply}\isamarkupfalse%
{\isacharparenleft}{\kern0pt}rule\ iffD{\isadigit{2}}\ {\isacharcomma}{\kern0pt}\ rule\ sats{\isacharunderscore}{\kern0pt}{\isasymphi}{\isacharunderscore}{\kern0pt}iff{\isacharparenright}{\kern0pt}\isanewline
\ \ \ \ \isacommand{using}\isamarkupfalse%
\ GH\ listin{\isacharprime}{\kern0pt}\ iH\ iH{\isacharprime}{\kern0pt}\isanewline
\ \ \ \ \isacommand{by}\isamarkupfalse%
\ auto\isanewline
\ \ \isacommand{then}\isamarkupfalse%
\ \isacommand{have}\isamarkupfalse%
\ {\isachardoublequoteopen}{\isasymexists}q\ {\isasymin}\ G{\isachardot}{\kern0pt}\ ForcesHS{\isacharparenleft}{\kern0pt}q{\isacharcomma}{\kern0pt}\ {\isasymphi}{\isacharcomma}{\kern0pt}\ {\isacharbrackleft}{\kern0pt}F{\isacharprime}{\kern0pt}{\isacharcomma}{\kern0pt}\ check{\isacharparenleft}{\kern0pt}i{\isacharparenright}{\kern0pt}{\isacharcomma}{\kern0pt}\ binmap{\isacharunderscore}{\kern0pt}row{\isacharprime}{\kern0pt}{\isacharparenleft}{\kern0pt}n{\isacharparenright}{\kern0pt}{\isacharbrackright}{\kern0pt}{\isacharparenright}{\kern0pt}{\isachardoublequoteclose}\ \isanewline
\ \ \ \ \isacommand{apply}\isamarkupfalse%
{\isacharparenleft}{\kern0pt}rule{\isacharunderscore}{\kern0pt}tac\ iffD{\isadigit{2}}{\isacharparenright}{\kern0pt}\isanewline
\ \ \ \ \ \isacommand{apply}\isamarkupfalse%
{\isacharparenleft}{\kern0pt}rule\ HS{\isacharunderscore}{\kern0pt}truth{\isacharunderscore}{\kern0pt}lemma{\isacharparenright}{\kern0pt}\isanewline
\ \ \ \ \isacommand{unfolding}\isamarkupfalse%
\ {\isasymphi}{\isacharunderscore}{\kern0pt}def\isanewline
\ \ \ \ \isacommand{using}\isamarkupfalse%
\ assms\ GH\isanewline
\ \ \ \ \ \ \ \ \isacommand{apply}\isamarkupfalse%
\ auto{\isacharbrackleft}{\kern0pt}{\isadigit{2}}{\isacharbrackright}{\kern0pt}\isanewline
\ \ \ \ \isacommand{using}\isamarkupfalse%
\ iH\ nat{\isacharunderscore}{\kern0pt}in{\isacharunderscore}{\kern0pt}M\ transM\ check{\isacharunderscore}{\kern0pt}in{\isacharunderscore}{\kern0pt}HS\ binmap{\isacharunderscore}{\kern0pt}row{\isacharprime}{\kern0pt}{\isacharunderscore}{\kern0pt}HS\ nH\ assms\ GH\isanewline
\ \ \ \ \ \ \isacommand{apply}\isamarkupfalse%
\ force\isanewline
\ \ \ \ \ \isacommand{apply}\isamarkupfalse%
\ {\isacharparenleft}{\kern0pt}simp\ del{\isacharcolon}{\kern0pt}FOL{\isacharunderscore}{\kern0pt}sats{\isacharunderscore}{\kern0pt}iff\ pair{\isacharunderscore}{\kern0pt}abs\ add{\isacharcolon}{\kern0pt}\ fm{\isacharunderscore}{\kern0pt}defs\ nat{\isacharunderscore}{\kern0pt}simp{\isacharunderscore}{\kern0pt}union{\isacharparenright}{\kern0pt}\isanewline
\ \ \ \ \isacommand{using}\isamarkupfalse%
\ mapeq\ GH\isanewline
\ \ \ \ \isacommand{by}\isamarkupfalse%
\ auto\isanewline
\ \ \isacommand{then}\isamarkupfalse%
\ \isacommand{obtain}\isamarkupfalse%
\ p\ \isakeyword{where}\ pH{\isacharcolon}{\kern0pt}\ {\isachardoublequoteopen}p\ {\isasymin}\ G{\isachardoublequoteclose}\ {\isachardoublequoteopen}ForcesHS{\isacharparenleft}{\kern0pt}p{\isacharcomma}{\kern0pt}\ {\isasymphi}{\isacharcomma}{\kern0pt}\ {\isacharbrackleft}{\kern0pt}F{\isacharprime}{\kern0pt}{\isacharcomma}{\kern0pt}\ check{\isacharparenleft}{\kern0pt}i{\isacharparenright}{\kern0pt}{\isacharcomma}{\kern0pt}\ binmap{\isacharunderscore}{\kern0pt}row{\isacharprime}{\kern0pt}{\isacharparenleft}{\kern0pt}n{\isacharparenright}{\kern0pt}{\isacharbrackright}{\kern0pt}{\isacharparenright}{\kern0pt}{\isachardoublequoteclose}\ \isacommand{by}\isamarkupfalse%
\ auto\isanewline
\isanewline
\ \ \isacommand{have}\isamarkupfalse%
\ pinFn\ {\isacharcolon}{\kern0pt}\ {\isachardoublequoteopen}p\ {\isasymin}\ Fn{\isachardoublequoteclose}\ \isacommand{using}\isamarkupfalse%
\ Fn{\isacharunderscore}{\kern0pt}def\ M{\isacharunderscore}{\kern0pt}genericD\ pH\ assms\ GH\ \isacommand{by}\isamarkupfalse%
\ blast\ \isanewline
\isanewline
\ \ \isacommand{have}\isamarkupfalse%
\ domain{\isacharunderscore}{\kern0pt}Finite\ {\isacharcolon}{\kern0pt}\ {\isachardoublequoteopen}Finite{\isacharparenleft}{\kern0pt}domain{\isacharparenleft}{\kern0pt}p{\isacharparenright}{\kern0pt}{\isacharparenright}{\kern0pt}{\isachardoublequoteclose}\ \isanewline
\ \ \ \ \isacommand{using}\isamarkupfalse%
\ pinFn\ Fn{\isacharunderscore}{\kern0pt}def\ finite{\isacharunderscore}{\kern0pt}M{\isacharunderscore}{\kern0pt}implies{\isacharunderscore}{\kern0pt}Finite\isanewline
\ \ \ \ \isacommand{by}\isamarkupfalse%
\ auto\isanewline
\isanewline
\ \ \isacommand{have}\isamarkupfalse%
\ {\isachardoublequoteopen}{\isacharbraceleft}{\kern0pt}\ fst{\isacharparenleft}{\kern0pt}x{\isacharparenright}{\kern0pt}{\isachardot}{\kern0pt}\ x\ {\isasymin}\ domain{\isacharparenleft}{\kern0pt}p{\isacharparenright}{\kern0pt}\ {\isacharbraceright}{\kern0pt}\ {\isacharequal}{\kern0pt}\ domain{\isacharparenleft}{\kern0pt}domain{\isacharparenleft}{\kern0pt}p{\isacharparenright}{\kern0pt}{\isacharparenright}{\kern0pt}{\isachardoublequoteclose}\ {\isacharparenleft}{\kern0pt}\isakeyword{is}\ {\isachardoublequoteopen}{\isacharquery}{\kern0pt}A\ {\isacharequal}{\kern0pt}\ {\isacharunderscore}{\kern0pt}{\isachardoublequoteclose}{\isacharparenright}{\kern0pt}\isanewline
\ \ \ \ \isacommand{apply}\isamarkupfalse%
{\isacharparenleft}{\kern0pt}rule\ equality{\isacharunderscore}{\kern0pt}iffI{\isacharcomma}{\kern0pt}\ rule\ iffI{\isacharparenright}{\kern0pt}\isanewline
\ \ \ \ \isacommand{using}\isamarkupfalse%
\ assms\ Fn{\isacharunderscore}{\kern0pt}def\ pinFn\ GH\isanewline
\ \ \ \ \ \isacommand{apply}\isamarkupfalse%
\ force\isanewline
\ \ \ \ \isacommand{apply}\isamarkupfalse%
\ clarsimp\isanewline
\ \ \ \ \isacommand{apply}\isamarkupfalse%
{\isacharparenleft}{\kern0pt}rename{\isacharunderscore}{\kern0pt}tac\ a\ b\ c{\isacharcomma}{\kern0pt}\ rule{\isacharunderscore}{\kern0pt}tac\ x{\isacharequal}{\kern0pt}{\isachardoublequoteopen}{\isacharless}{\kern0pt}a{\isacharcomma}{\kern0pt}\ b{\isachargreater}{\kern0pt}{\isachardoublequoteclose}\ \isakeyword{in}\ bexI{\isacharparenright}{\kern0pt}\isanewline
\ \ \ \ \isacommand{by}\isamarkupfalse%
\ auto\isanewline
\ \ \isacommand{have}\isamarkupfalse%
\ {\isachardoublequoteopen}Finite{\isacharparenleft}{\kern0pt}{\isacharquery}{\kern0pt}A{\isacharparenright}{\kern0pt}{\isachardoublequoteclose}\ \isanewline
\ \ \ \ \isacommand{apply}\isamarkupfalse%
{\isacharparenleft}{\kern0pt}rule\ Finite{\isacharunderscore}{\kern0pt}RepFun{\isacharcomma}{\kern0pt}\ rule\ domain{\isacharunderscore}{\kern0pt}Finite{\isacharparenright}{\kern0pt}\isanewline
\ \ \ \ \isacommand{done}\isamarkupfalse%
\isanewline
\ \ \isacommand{then}\isamarkupfalse%
\ \isacommand{have}\isamarkupfalse%
\ {\isachardoublequoteopen}Finite{\isacharparenleft}{\kern0pt}domain{\isacharparenleft}{\kern0pt}domain{\isacharparenleft}{\kern0pt}p{\isacharparenright}{\kern0pt}{\isacharparenright}{\kern0pt}{\isacharparenright}{\kern0pt}{\isachardoublequoteclose}\ \isacommand{using}\isamarkupfalse%
\ {\isacartoucheopen}{\isacharquery}{\kern0pt}A\ {\isacharequal}{\kern0pt}\ domain{\isacharparenleft}{\kern0pt}domain{\isacharparenleft}{\kern0pt}p{\isacharparenright}{\kern0pt}{\isacharparenright}{\kern0pt}{\isacartoucheclose}\ \isacommand{by}\isamarkupfalse%
\ auto\isanewline
\ \ \isacommand{then}\isamarkupfalse%
\ \isacommand{have}\isamarkupfalse%
\ {\isachardoublequoteopen}Finite{\isacharparenleft}{\kern0pt}e\ {\isasymunion}\ domain{\isacharparenleft}{\kern0pt}domain{\isacharparenleft}{\kern0pt}p{\isacharparenright}{\kern0pt}{\isacharparenright}{\kern0pt}\ {\isasymunion}\ {\isacharbraceleft}{\kern0pt}n{\isacharbraceright}{\kern0pt}{\isacharparenright}{\kern0pt}{\isachardoublequoteclose}\ {\isacharparenleft}{\kern0pt}\isakeyword{is}\ {\isachardoublequoteopen}Finite{\isacharparenleft}{\kern0pt}{\isacharquery}{\kern0pt}B{\isacharparenright}{\kern0pt}{\isachardoublequoteclose}{\isacharparenright}{\kern0pt}\isanewline
\ \ \ \ \isacommand{using}\isamarkupfalse%
\ Finite{\isacharunderscore}{\kern0pt}Un\ e{\isacharunderscore}{\kern0pt}Finite\ \isanewline
\ \ \ \ \isacommand{by}\isamarkupfalse%
\ auto\isanewline
\isanewline
\ \ \isacommand{have}\isamarkupfalse%
\ {\isachardoublequoteopen}{\isacharquery}{\kern0pt}B\ {\isasymsubseteq}\ nat{\isachardoublequoteclose}\ \isanewline
\ \ \ \ \isacommand{using}\isamarkupfalse%
\ eH\ assms\ pinFn\ Fn{\isacharunderscore}{\kern0pt}def\ nH\isanewline
\ \ \ \ \isacommand{by}\isamarkupfalse%
\ force\ \ \isanewline
\isanewline
\ \ \isacommand{have}\isamarkupfalse%
\ ex{\isacharunderscore}{\kern0pt}remain{\isacharcolon}{\kern0pt}\ {\isachardoublequoteopen}{\isasymAnd}A{\isachardot}{\kern0pt}\ A\ {\isasymsubseteq}\ nat\ {\isasymLongrightarrow}\ Finite{\isacharparenleft}{\kern0pt}A{\isacharparenright}{\kern0pt}\ {\isasymLongrightarrow}\ {\isasymexists}a\ {\isasymin}\ nat{\isachardot}{\kern0pt}\ a\ {\isasymnotin}\ A{\isachardoublequoteclose}\ \isanewline
\ \ \ \ \isacommand{apply}\isamarkupfalse%
{\isacharparenleft}{\kern0pt}rule\ ccontr{\isacharparenright}{\kern0pt}\isanewline
\ \ \ \ \isacommand{apply}\isamarkupfalse%
{\isacharparenleft}{\kern0pt}rename{\isacharunderscore}{\kern0pt}tac\ A{\isacharcomma}{\kern0pt}\ subgoal{\isacharunderscore}{\kern0pt}tac\ {\isachardoublequoteopen}A\ {\isacharequal}{\kern0pt}\ nat{\isachardoublequoteclose}{\isacharparenright}{\kern0pt}\isanewline
\ \ \ \ \isacommand{using}\isamarkupfalse%
\ nat{\isacharunderscore}{\kern0pt}not{\isacharunderscore}{\kern0pt}Finite\ \isanewline
\ \ \ \ \ \isacommand{apply}\isamarkupfalse%
\ force\ \isanewline
\ \ \ \ \isacommand{apply}\isamarkupfalse%
{\isacharparenleft}{\kern0pt}rule\ equalityI{\isacharparenright}{\kern0pt}\isanewline
\ \ \ \ \isacommand{by}\isamarkupfalse%
\ auto\isanewline
\ \ \isacommand{obtain}\isamarkupfalse%
\ n{\isacharprime}{\kern0pt}\ \isakeyword{where}\ n{\isacharprime}{\kern0pt}H{\isacharcolon}{\kern0pt}\ {\isachardoublequoteopen}n{\isacharprime}{\kern0pt}\ {\isasymnotin}\ {\isacharquery}{\kern0pt}B{\isachardoublequoteclose}\ {\isachardoublequoteopen}n{\isacharprime}{\kern0pt}\ {\isasymin}\ nat{\isachardoublequoteclose}\ \isanewline
\ \ \ \ \isacommand{apply}\isamarkupfalse%
{\isacharparenleft}{\kern0pt}insert\ ex{\isacharunderscore}{\kern0pt}remain\ {\isacharbrackleft}{\kern0pt}of\ {\isacharquery}{\kern0pt}B{\isacharbrackright}{\kern0pt}{\isacharparenright}{\kern0pt}\isanewline
\ \ \ \ \isacommand{using}\isamarkupfalse%
\ {\isacartoucheopen}{\isacharquery}{\kern0pt}B\ {\isasymsubseteq}\ nat{\isacartoucheclose}\ {\isacartoucheopen}Finite{\isacharparenleft}{\kern0pt}{\isacharquery}{\kern0pt}B{\isacharparenright}{\kern0pt}{\isacartoucheclose}\isanewline
\ \ \ \ \isacommand{by}\isamarkupfalse%
\ blast\isanewline
\isanewline
\ \ \isacommand{define}\isamarkupfalse%
\ f\ \isakeyword{where}\ {\isachardoublequoteopen}f\ {\isasymequiv}\ {\isacharbraceleft}{\kern0pt}\ {\isacharless}{\kern0pt}x{\isacharcomma}{\kern0pt}\ x{\isachargreater}{\kern0pt}\ {\isachardot}{\kern0pt}{\isachardot}{\kern0pt}\ x\ {\isasymin}\ nat{\isacharcomma}{\kern0pt}\ x\ {\isasymnoteq}\ n\ {\isasymand}\ x\ {\isasymnoteq}\ n{\isacharprime}{\kern0pt}{\isacharbraceright}{\kern0pt}\ {\isasymunion}\ {\isacharbraceleft}{\kern0pt}\ {\isacharless}{\kern0pt}n{\isacharcomma}{\kern0pt}\ n{\isacharprime}{\kern0pt}{\isachargreater}{\kern0pt}{\isacharcomma}{\kern0pt}\ {\isacharless}{\kern0pt}n{\isacharprime}{\kern0pt}{\isacharcomma}{\kern0pt}\ n{\isachargreater}{\kern0pt}\ {\isacharbraceright}{\kern0pt}{\isachardoublequoteclose}\ \isanewline
\isanewline
\ \ \isacommand{have}\isamarkupfalse%
\ fn\ {\isacharcolon}{\kern0pt}\ {\isachardoublequoteopen}f{\isacharbackquote}{\kern0pt}n\ {\isacharequal}{\kern0pt}\ n{\isacharprime}{\kern0pt}{\isachardoublequoteclose}\ \isanewline
\ \ \ \ \isacommand{apply}\isamarkupfalse%
{\isacharparenleft}{\kern0pt}rule\ function{\isacharunderscore}{\kern0pt}apply{\isacharunderscore}{\kern0pt}equality{\isacharparenright}{\kern0pt}\isanewline
\ \ \ \ \isacommand{using}\isamarkupfalse%
\ f{\isacharunderscore}{\kern0pt}def\ function{\isacharunderscore}{\kern0pt}def\isanewline
\ \ \ \ \isacommand{by}\isamarkupfalse%
\ auto\isanewline
\ \ \isacommand{have}\isamarkupfalse%
\ fn{\isacharprime}{\kern0pt}\ {\isacharcolon}{\kern0pt}\ {\isachardoublequoteopen}f{\isacharbackquote}{\kern0pt}n{\isacharprime}{\kern0pt}\ {\isacharequal}{\kern0pt}\ n{\isachardoublequoteclose}\ \isanewline
\ \ \ \ \isacommand{apply}\isamarkupfalse%
{\isacharparenleft}{\kern0pt}rule\ function{\isacharunderscore}{\kern0pt}apply{\isacharunderscore}{\kern0pt}equality{\isacharparenright}{\kern0pt}\isanewline
\ \ \ \ \isacommand{using}\isamarkupfalse%
\ f{\isacharunderscore}{\kern0pt}def\ function{\isacharunderscore}{\kern0pt}def\isanewline
\ \ \ \ \isacommand{by}\isamarkupfalse%
\ auto\isanewline
\ \ \isacommand{have}\isamarkupfalse%
\ fx\ {\isacharcolon}{\kern0pt}\ {\isachardoublequoteopen}{\isasymAnd}x{\isachardot}{\kern0pt}\ x\ {\isasymin}\ nat\ {\isasymLongrightarrow}\ x\ {\isasymnoteq}\ n\ {\isasymLongrightarrow}\ x\ {\isasymnoteq}\ n{\isacharprime}{\kern0pt}\ {\isasymLongrightarrow}\ f{\isacharbackquote}{\kern0pt}x\ {\isacharequal}{\kern0pt}\ x{\isachardoublequoteclose}\ \isanewline
\ \ \ \ \isacommand{apply}\isamarkupfalse%
{\isacharparenleft}{\kern0pt}rule\ function{\isacharunderscore}{\kern0pt}apply{\isacharunderscore}{\kern0pt}equality{\isacharparenright}{\kern0pt}\isanewline
\ \ \ \ \isacommand{using}\isamarkupfalse%
\ f{\isacharunderscore}{\kern0pt}def\ function{\isacharunderscore}{\kern0pt}def\ \isanewline
\ \ \ \ \isacommand{by}\isamarkupfalse%
\ auto\isanewline
\isanewline
\ \ \isacommand{have}\isamarkupfalse%
\ finj\ {\isacharcolon}{\kern0pt}\ {\isachardoublequoteopen}{\isasymAnd}a\ b{\isachardot}{\kern0pt}\ a\ {\isasymin}\ nat\ {\isasymLongrightarrow}\ b\ {\isasymin}\ nat\ {\isasymLongrightarrow}\ a\ {\isasymnoteq}\ b\ {\isasymLongrightarrow}\ f{\isacharbackquote}{\kern0pt}a\ {\isasymnoteq}\ f{\isacharbackquote}{\kern0pt}b{\isachardoublequoteclose}\ \isanewline
\ \ \ \ \isacommand{apply}\isamarkupfalse%
{\isacharparenleft}{\kern0pt}rename{\isacharunderscore}{\kern0pt}tac\ a\ b{\isacharcomma}{\kern0pt}\ case{\isacharunderscore}{\kern0pt}tac\ {\isachardoublequoteopen}a\ {\isasymin}\ {\isacharbraceleft}{\kern0pt}n{\isacharcomma}{\kern0pt}\ n{\isacharprime}{\kern0pt}{\isacharbraceright}{\kern0pt}{\isachardoublequoteclose}{\isacharparenright}{\kern0pt}\ \isanewline
\ \ \ \ \ \isacommand{apply}\isamarkupfalse%
{\isacharparenleft}{\kern0pt}rename{\isacharunderscore}{\kern0pt}tac\ a\ b{\isacharcomma}{\kern0pt}\ case{\isacharunderscore}{\kern0pt}tac\ {\isachardoublequoteopen}b\ {\isasymin}\ {\isacharbraceleft}{\kern0pt}n{\isacharcomma}{\kern0pt}\ n{\isacharprime}{\kern0pt}{\isacharbraceright}{\kern0pt}{\isachardoublequoteclose}{\isacharparenright}{\kern0pt}\isanewline
\ \ \ \ \isacommand{using}\isamarkupfalse%
\ fn\ fn{\isacharprime}{\kern0pt}\ \isanewline
\ \ \ \ \ \ \isacommand{apply}\isamarkupfalse%
\ force\isanewline
\ \ \ \ \isacommand{using}\isamarkupfalse%
\ fn\ fn{\isacharprime}{\kern0pt}\ fx\isanewline
\ \ \ \ \ \ \isacommand{apply}\isamarkupfalse%
\ force\ \isanewline
\ \ \ \ \ \isacommand{apply}\isamarkupfalse%
{\isacharparenleft}{\kern0pt}rename{\isacharunderscore}{\kern0pt}tac\ a\ b{\isacharcomma}{\kern0pt}\ case{\isacharunderscore}{\kern0pt}tac\ {\isachardoublequoteopen}b\ {\isasymin}\ {\isacharbraceleft}{\kern0pt}n{\isacharcomma}{\kern0pt}\ n{\isacharprime}{\kern0pt}{\isacharbraceright}{\kern0pt}{\isachardoublequoteclose}{\isacharparenright}{\kern0pt}\isanewline
\ \ \ \ \isacommand{using}\isamarkupfalse%
\ fn\ fn{\isacharprime}{\kern0pt}\ fx\isanewline
\ \ \ \ \ \ \isacommand{apply}\isamarkupfalse%
\ force\isanewline
\ \ \ \ \isacommand{using}\isamarkupfalse%
\ fn\ fn{\isacharprime}{\kern0pt}\ fx\isanewline
\ \ \ \ \isacommand{apply}\isamarkupfalse%
\ force\isanewline
\ \ \ \ \isacommand{done}\isamarkupfalse%
\isanewline
\isanewline
\ \ \isacommand{define}\isamarkupfalse%
\ {\isasympsi}\ \isakeyword{where}\ {\isachardoublequoteopen}{\isasympsi}\ {\isasymequiv}\ Exists{\isacharparenleft}{\kern0pt}Exists{\isacharparenleft}{\kern0pt}And{\isacharparenleft}{\kern0pt}pair{\isacharunderscore}{\kern0pt}fm{\isacharparenleft}{\kern0pt}{\isadigit{0}}{\isacharcomma}{\kern0pt}\ {\isadigit{1}}{\isacharcomma}{\kern0pt}\ {\isadigit{2}}{\isacharparenright}{\kern0pt}{\isacharcomma}{\kern0pt}\ \isanewline
\ \ \ \ \ \ \ \ \ \ \ \ \ \ \ \ \ \ \ \ \ \ \ \ Or{\isacharparenleft}{\kern0pt}Exists{\isacharparenleft}{\kern0pt}And{\isacharparenleft}{\kern0pt}Member{\isacharparenleft}{\kern0pt}{\isadigit{0}}{\isacharcomma}{\kern0pt}\ {\isadigit{4}}{\isacharparenright}{\kern0pt}{\isacharcomma}{\kern0pt}\ And{\isacharparenleft}{\kern0pt}Equal{\isacharparenleft}{\kern0pt}{\isadigit{0}}{\isacharcomma}{\kern0pt}\ {\isadigit{1}}{\isacharparenright}{\kern0pt}{\isacharcomma}{\kern0pt}\ And{\isacharparenleft}{\kern0pt}Equal{\isacharparenleft}{\kern0pt}{\isadigit{0}}{\isacharcomma}{\kern0pt}\ {\isadigit{2}}{\isacharparenright}{\kern0pt}{\isacharcomma}{\kern0pt}\ And{\isacharparenleft}{\kern0pt}Neg{\isacharparenleft}{\kern0pt}Equal{\isacharparenleft}{\kern0pt}{\isadigit{0}}{\isacharcomma}{\kern0pt}\ {\isadigit{5}}{\isacharparenright}{\kern0pt}{\isacharparenright}{\kern0pt}{\isacharcomma}{\kern0pt}\ Neg{\isacharparenleft}{\kern0pt}Equal{\isacharparenleft}{\kern0pt}{\isadigit{0}}{\isacharcomma}{\kern0pt}\ {\isadigit{6}}{\isacharparenright}{\kern0pt}{\isacharparenright}{\kern0pt}{\isacharparenright}{\kern0pt}{\isacharparenright}{\kern0pt}{\isacharparenright}{\kern0pt}{\isacharparenright}{\kern0pt}{\isacharparenright}{\kern0pt}{\isacharcomma}{\kern0pt}\ \isanewline
\ \ \ \ \ \ \ \ \ \ \ \ \ \ \ \ \ \ \ \ \ \ \ \ Or{\isacharparenleft}{\kern0pt}And{\isacharparenleft}{\kern0pt}Equal{\isacharparenleft}{\kern0pt}{\isadigit{0}}{\isacharcomma}{\kern0pt}\ {\isadigit{4}}{\isacharparenright}{\kern0pt}{\isacharcomma}{\kern0pt}\ Equal{\isacharparenleft}{\kern0pt}{\isadigit{1}}{\isacharcomma}{\kern0pt}\ {\isadigit{5}}{\isacharparenright}{\kern0pt}{\isacharparenright}{\kern0pt}{\isacharcomma}{\kern0pt}\ \isanewline
\ \ \ \ \ \ \ \ \ \ \ \ \ \ \ \ \ \ \ \ \ \ \ \ \ \ \ And{\isacharparenleft}{\kern0pt}Equal{\isacharparenleft}{\kern0pt}{\isadigit{0}}{\isacharcomma}{\kern0pt}\ {\isadigit{5}}{\isacharparenright}{\kern0pt}{\isacharcomma}{\kern0pt}\ Equal{\isacharparenleft}{\kern0pt}{\isadigit{1}}{\isacharcomma}{\kern0pt}\ {\isadigit{4}}{\isacharparenright}{\kern0pt}{\isacharparenright}{\kern0pt}{\isacharparenright}{\kern0pt}{\isacharparenright}{\kern0pt}{\isacharparenright}{\kern0pt}{\isacharparenright}{\kern0pt}{\isacharparenright}{\kern0pt}{\isachardoublequoteclose}\ \isanewline
\isanewline
\ \ \isacommand{have}\isamarkupfalse%
\ {\isachardoublequoteopen}{\isasymAnd}v{\isachardot}{\kern0pt}\ v\ {\isasymin}\ M\ {\isasymLongrightarrow}\ sats{\isacharparenleft}{\kern0pt}M{\isacharcomma}{\kern0pt}\ {\isasympsi}{\isacharcomma}{\kern0pt}\ {\isacharbrackleft}{\kern0pt}v{\isacharcomma}{\kern0pt}\ nat{\isacharcomma}{\kern0pt}\ n{\isacharcomma}{\kern0pt}\ n{\isacharprime}{\kern0pt}{\isacharbrackright}{\kern0pt}{\isacharparenright}{\kern0pt}\ {\isasymlongleftrightarrow}\ v\ {\isasymin}\ f{\isachardoublequoteclose}\ \isanewline
\ \ \ \ \isanewline
\ \ \ \ \isacommand{apply}\isamarkupfalse%
{\isacharparenleft}{\kern0pt}rule{\isacharunderscore}{\kern0pt}tac\ Q{\isacharequal}{\kern0pt}\isanewline
\ \ \ \ \ \ {\isachardoublequoteopen}{\isacharparenleft}{\kern0pt}{\isasymexists}b\ {\isasymin}\ M{\isachardot}{\kern0pt}\ {\isasymexists}a\ {\isasymin}\ M{\isachardot}{\kern0pt}\ {\isacharparenleft}{\kern0pt}{\isacharless}{\kern0pt}a{\isacharcomma}{\kern0pt}\ b{\isachargreater}{\kern0pt}\ {\isacharequal}{\kern0pt}\ v\ {\isasymand}\ \isanewline
\ \ \ \ \ \ \ \ {\isacharparenleft}{\kern0pt}{\isacharparenleft}{\kern0pt}{\isasymexists}x\ {\isasymin}\ M{\isachardot}{\kern0pt}\ x\ {\isasymin}\ nat\ {\isasymand}\ x\ {\isacharequal}{\kern0pt}\ a\ {\isasymand}\ x\ {\isacharequal}{\kern0pt}\ b\ {\isasymand}\ x\ {\isasymnoteq}\ n\ {\isasymand}\ x\ {\isasymnoteq}\ n{\isacharprime}{\kern0pt}{\isacharparenright}{\kern0pt}\ {\isasymor}\ \isanewline
\ \ \ \ \ \ \ \ \ {\isacharparenleft}{\kern0pt}a\ {\isacharequal}{\kern0pt}\ n\ {\isasymand}\ b\ {\isacharequal}{\kern0pt}\ n{\isacharprime}{\kern0pt}{\isacharparenright}{\kern0pt}\ {\isasymor}\ \isanewline
\ \ \ \ \ \ \ \ \ {\isacharparenleft}{\kern0pt}a\ {\isacharequal}{\kern0pt}\ n{\isacharprime}{\kern0pt}\ {\isasymand}\ b\ {\isacharequal}{\kern0pt}\ n{\isacharparenright}{\kern0pt}{\isacharparenright}{\kern0pt}{\isacharparenright}{\kern0pt}{\isacharparenright}{\kern0pt}{\isachardoublequoteclose}\ \isakeyword{in}\ iff{\isacharunderscore}{\kern0pt}trans{\isacharparenright}{\kern0pt}\isanewline
\ \ \ \ \isacommand{unfolding}\isamarkupfalse%
\ {\isasympsi}{\isacharunderscore}{\kern0pt}def\isanewline
\ \ \ \ \isacommand{apply}\isamarkupfalse%
{\isacharparenleft}{\kern0pt}subgoal{\isacharunderscore}{\kern0pt}tac\ {\isachardoublequoteopen}{\isacharbrackleft}{\kern0pt}nat{\isacharcomma}{\kern0pt}\ n{\isacharcomma}{\kern0pt}\ n{\isacharprime}{\kern0pt}{\isacharbrackright}{\kern0pt}\ {\isasymin}\ list{\isacharparenleft}{\kern0pt}M{\isacharparenright}{\kern0pt}{\isachardoublequoteclose}{\isacharparenright}{\kern0pt}\isanewline
\ \ \ \ \isacommand{apply}\isamarkupfalse%
{\isacharparenleft}{\kern0pt}rule\ iff{\isacharunderscore}{\kern0pt}trans{\isacharcomma}{\kern0pt}\ rule\ sats{\isacharunderscore}{\kern0pt}Exists{\isacharunderscore}{\kern0pt}iff{\isacharcomma}{\kern0pt}\ simp{\isacharcomma}{\kern0pt}\ rule\ bex{\isacharunderscore}{\kern0pt}iff{\isacharparenright}{\kern0pt}{\isacharplus}{\kern0pt}\isanewline
\ \ \ \ \isacommand{apply}\isamarkupfalse%
{\isacharparenleft}{\kern0pt}rule\ iff{\isacharunderscore}{\kern0pt}trans{\isacharcomma}{\kern0pt}\ rule\ sats{\isacharunderscore}{\kern0pt}And{\isacharunderscore}{\kern0pt}iff{\isacharcomma}{\kern0pt}\ simp{\isacharcomma}{\kern0pt}\ rule\ iff{\isacharunderscore}{\kern0pt}conjI{\isacharcomma}{\kern0pt}\ force{\isacharparenright}{\kern0pt}\isanewline
\ \ \ \ \isacommand{apply}\isamarkupfalse%
{\isacharparenleft}{\kern0pt}rule\ iff{\isacharunderscore}{\kern0pt}trans{\isacharcomma}{\kern0pt}\ rule\ sats{\isacharunderscore}{\kern0pt}Or{\isacharunderscore}{\kern0pt}iff{\isacharcomma}{\kern0pt}\ simp{\isacharcomma}{\kern0pt}\ rule\ iff{\isacharunderscore}{\kern0pt}disjI{\isacharparenright}{\kern0pt}\isanewline
\ \ \ \ \ \ \isacommand{apply}\isamarkupfalse%
{\isacharparenleft}{\kern0pt}rule\ iff{\isacharunderscore}{\kern0pt}trans{\isacharcomma}{\kern0pt}\ rule\ sats{\isacharunderscore}{\kern0pt}Exists{\isacharunderscore}{\kern0pt}iff{\isacharcomma}{\kern0pt}\ simp{\isacharcomma}{\kern0pt}\ rule\ bex{\isacharunderscore}{\kern0pt}iff{\isacharparenright}{\kern0pt}\isanewline
\ \ \ \ \ \ \isacommand{apply}\isamarkupfalse%
{\isacharparenleft}{\kern0pt}rule\ iff{\isacharunderscore}{\kern0pt}trans{\isacharcomma}{\kern0pt}\ rule\ sats{\isacharunderscore}{\kern0pt}And{\isacharunderscore}{\kern0pt}iff{\isacharcomma}{\kern0pt}\ simp{\isacharcomma}{\kern0pt}\ rule\ iff{\isacharunderscore}{\kern0pt}conjI{\isacharcomma}{\kern0pt}\ simp{\isacharparenright}{\kern0pt}{\isacharplus}{\kern0pt}\isanewline
\ \ \ \ \ \ \isacommand{apply}\isamarkupfalse%
\ simp\isanewline
\ \ \ \ \ \isacommand{apply}\isamarkupfalse%
{\isacharparenleft}{\kern0pt}rule\ iff{\isacharunderscore}{\kern0pt}trans{\isacharcomma}{\kern0pt}\ rule\ sats{\isacharunderscore}{\kern0pt}Or{\isacharunderscore}{\kern0pt}iff{\isacharcomma}{\kern0pt}\ simp{\isacharcomma}{\kern0pt}\ rule\ iff{\isacharunderscore}{\kern0pt}disjI{\isacharparenright}{\kern0pt}\isanewline
\ \ \ \ \ \ \isacommand{apply}\isamarkupfalse%
{\isacharparenleft}{\kern0pt}rule\ iff{\isacharunderscore}{\kern0pt}trans{\isacharcomma}{\kern0pt}\ rule\ sats{\isacharunderscore}{\kern0pt}And{\isacharunderscore}{\kern0pt}iff{\isacharcomma}{\kern0pt}\ simp{\isacharcomma}{\kern0pt}\ rule\ iff{\isacharunderscore}{\kern0pt}conjI{\isacharcomma}{\kern0pt}\ simp{\isacharparenright}{\kern0pt}{\isacharplus}{\kern0pt}\isanewline
\ \ \ \ \ \ \isacommand{apply}\isamarkupfalse%
\ simp\isanewline
\ \ \ \ \ \isacommand{apply}\isamarkupfalse%
{\isacharparenleft}{\kern0pt}rule\ iff{\isacharunderscore}{\kern0pt}trans{\isacharcomma}{\kern0pt}\ rule\ sats{\isacharunderscore}{\kern0pt}And{\isacharunderscore}{\kern0pt}iff{\isacharcomma}{\kern0pt}\ simp{\isacharcomma}{\kern0pt}\ rule\ iff{\isacharunderscore}{\kern0pt}conjI{\isacharcomma}{\kern0pt}\ simp{\isacharparenright}{\kern0pt}{\isacharplus}{\kern0pt}\isanewline
\ \ \ \ \isacommand{using}\isamarkupfalse%
\ nat{\isacharunderscore}{\kern0pt}in{\isacharunderscore}{\kern0pt}M\ transM\ nH\ n{\isacharprime}{\kern0pt}H\ \isanewline
\ \ \ \ \ \ \isacommand{apply}\isamarkupfalse%
\ auto{\isacharbrackleft}{\kern0pt}{\isadigit{2}}{\isacharbrackright}{\kern0pt}\isanewline
\ \ \ \ \isacommand{unfolding}\isamarkupfalse%
\ f{\isacharunderscore}{\kern0pt}def\isanewline
\ \ \ \ \isacommand{apply}\isamarkupfalse%
{\isacharparenleft}{\kern0pt}rule\ iffI{\isacharparenright}{\kern0pt}\isanewline
\ \ \ \ \ \isacommand{apply}\isamarkupfalse%
\ force\isanewline
\ \ \ \ \isacommand{using}\isamarkupfalse%
\ pair{\isacharunderscore}{\kern0pt}in{\isacharunderscore}{\kern0pt}M{\isacharunderscore}{\kern0pt}iff\isanewline
\ \ \ \ \isacommand{by}\isamarkupfalse%
\ auto\isanewline
\ \ \isacommand{then}\isamarkupfalse%
\ \isacommand{have}\isamarkupfalse%
\ {\isachardoublequoteopen}f\ {\isasymin}\ M{\isachardoublequoteclose}\ \isanewline
\ \ \ \ \isacommand{apply}\isamarkupfalse%
{\isacharparenleft}{\kern0pt}rule{\isacharunderscore}{\kern0pt}tac\ a{\isacharequal}{\kern0pt}{\isachardoublequoteopen}{\isacharbraceleft}{\kern0pt}\ x\ {\isasymin}\ nat\ {\isasymtimes}\ nat{\isachardot}{\kern0pt}\ x\ {\isasymin}\ f\ {\isacharbraceright}{\kern0pt}{\isachardoublequoteclose}\ \isakeyword{and}\ b{\isacharequal}{\kern0pt}f\ \isakeyword{in}\ ssubst{\isacharparenright}{\kern0pt}\isanewline
\ \ \ \ \isacommand{using}\isamarkupfalse%
\ f{\isacharunderscore}{\kern0pt}def\ nH\ n{\isacharprime}{\kern0pt}H\isanewline
\ \ \ \ \ \isacommand{apply}\isamarkupfalse%
\ auto{\isacharbrackleft}{\kern0pt}{\isadigit{1}}{\isacharbrackright}{\kern0pt}\isanewline
\ \ \ \ \isacommand{apply}\isamarkupfalse%
{\isacharparenleft}{\kern0pt}rule{\isacharunderscore}{\kern0pt}tac\ a{\isacharequal}{\kern0pt}{\isachardoublequoteopen}{\isacharbraceleft}{\kern0pt}\ x\ {\isasymin}\ nat\ {\isasymtimes}\ nat{\isachardot}{\kern0pt}\ sats{\isacharparenleft}{\kern0pt}M{\isacharcomma}{\kern0pt}\ {\isasympsi}{\isacharcomma}{\kern0pt}\ {\isacharbrackleft}{\kern0pt}x{\isacharbrackright}{\kern0pt}\ {\isacharat}{\kern0pt}\ {\isacharbrackleft}{\kern0pt}nat{\isacharcomma}{\kern0pt}\ n{\isacharcomma}{\kern0pt}\ n{\isacharprime}{\kern0pt}{\isacharbrackright}{\kern0pt}{\isacharparenright}{\kern0pt}\ {\isacharbraceright}{\kern0pt}{\isachardoublequoteclose}\ \isakeyword{and}\ b{\isacharequal}{\kern0pt}{\isachardoublequoteopen}{\isacharbraceleft}{\kern0pt}\ x\ {\isasymin}\ nat\ {\isasymtimes}\ nat{\isachardot}{\kern0pt}\ x\ {\isasymin}\ f\ {\isacharbraceright}{\kern0pt}{\isachardoublequoteclose}\ \isakeyword{in}\ ssubst{\isacharparenright}{\kern0pt}\isanewline
\ \ \ \ \ \isacommand{apply}\isamarkupfalse%
{\isacharparenleft}{\kern0pt}rule\ iff{\isacharunderscore}{\kern0pt}eq{\isacharparenright}{\kern0pt}\isanewline
\ \ \ \ \isacommand{using}\isamarkupfalse%
\ nat{\isacharunderscore}{\kern0pt}in{\isacharunderscore}{\kern0pt}M\ cartprod{\isacharunderscore}{\kern0pt}closed\ transM\isanewline
\ \ \ \ \ \isacommand{apply}\isamarkupfalse%
\ force\isanewline
\ \ \ \ \isacommand{apply}\isamarkupfalse%
{\isacharparenleft}{\kern0pt}rule\ separation{\isacharunderscore}{\kern0pt}notation{\isacharcomma}{\kern0pt}\ rule\ separation{\isacharunderscore}{\kern0pt}ax{\isacharparenright}{\kern0pt}\isanewline
\ \ \ \ \isacommand{unfolding}\isamarkupfalse%
\ {\isasympsi}{\isacharunderscore}{\kern0pt}def\isanewline
\ \ \ \ \ \ \ \isacommand{apply}\isamarkupfalse%
\ force\isanewline
\ \ \ \ \isacommand{using}\isamarkupfalse%
\ nat{\isacharunderscore}{\kern0pt}in{\isacharunderscore}{\kern0pt}M\ transM\ nH\ n{\isacharprime}{\kern0pt}H\ \isanewline
\ \ \ \ \ \ \isacommand{apply}\isamarkupfalse%
\ auto{\isacharbrackleft}{\kern0pt}{\isadigit{1}}{\isacharbrackright}{\kern0pt}\isanewline
\ \ \ \ \ \isacommand{apply}\isamarkupfalse%
\ {\isacharparenleft}{\kern0pt}simp\ del{\isacharcolon}{\kern0pt}FOL{\isacharunderscore}{\kern0pt}sats{\isacharunderscore}{\kern0pt}iff\ pair{\isacharunderscore}{\kern0pt}abs\ add{\isacharcolon}{\kern0pt}\ fm{\isacharunderscore}{\kern0pt}defs\ nat{\isacharunderscore}{\kern0pt}simp{\isacharunderscore}{\kern0pt}union{\isacharparenright}{\kern0pt}\isanewline
\ \ \ \ \isacommand{using}\isamarkupfalse%
\ nat{\isacharunderscore}{\kern0pt}in{\isacharunderscore}{\kern0pt}M\ cartprod{\isacharunderscore}{\kern0pt}closed\ transM\isanewline
\ \ \ \ \ \isacommand{apply}\isamarkupfalse%
\ force\isanewline
\ \ \ \ \isacommand{done}\isamarkupfalse%
\isanewline
\ \ \isacommand{then}\isamarkupfalse%
\ \isacommand{have}\isamarkupfalse%
\ fin{\isacharcolon}{\kern0pt}\ {\isachardoublequoteopen}f\ {\isasymin}\ nat{\isacharunderscore}{\kern0pt}perms{\isachardoublequoteclose}\ \isanewline
\ \ \ \ \isacommand{unfolding}\isamarkupfalse%
\ nat{\isacharunderscore}{\kern0pt}perms{\isacharunderscore}{\kern0pt}def\ bij{\isacharunderscore}{\kern0pt}def\ inj{\isacharunderscore}{\kern0pt}def\ surj{\isacharunderscore}{\kern0pt}def\isanewline
\ \ \ \ \isacommand{apply}\isamarkupfalse%
\ simp\isanewline
\ \ \ \ \isacommand{apply}\isamarkupfalse%
{\isacharparenleft}{\kern0pt}rule\ conjI{\isacharparenright}{\kern0pt}\isanewline
\ \ \ \ \ \isacommand{apply}\isamarkupfalse%
{\isacharparenleft}{\kern0pt}rule\ Pi{\isacharunderscore}{\kern0pt}memberI{\isacharparenright}{\kern0pt}\isanewline
\ \ \ \ \isacommand{using}\isamarkupfalse%
\ f{\isacharunderscore}{\kern0pt}def\ relation{\isacharunderscore}{\kern0pt}def\ function{\isacharunderscore}{\kern0pt}def\isanewline
\ \ \ \ \ \ \ \ \isacommand{apply}\isamarkupfalse%
\ auto{\isacharbrackleft}{\kern0pt}{\isadigit{2}}{\isacharbrackright}{\kern0pt}\isanewline
\ \ \ \ \ \ \isacommand{apply}\isamarkupfalse%
{\isacharparenleft}{\kern0pt}rule\ equality{\isacharunderscore}{\kern0pt}iffI{\isacharcomma}{\kern0pt}\ rule\ iffI{\isacharparenright}{\kern0pt}\isanewline
\ \ \ \ \isacommand{using}\isamarkupfalse%
\ f{\isacharunderscore}{\kern0pt}def\ nH\ n{\isacharprime}{\kern0pt}H\isanewline
\ \ \ \ \ \ \ \isacommand{apply}\isamarkupfalse%
\ auto{\isacharbrackleft}{\kern0pt}{\isadigit{3}}{\isacharbrackright}{\kern0pt}\isanewline
\ \ \ \ \isacommand{apply}\isamarkupfalse%
{\isacharparenleft}{\kern0pt}rule\ conjI{\isacharparenright}{\kern0pt}\isanewline
\ \ \ \ \ \isacommand{apply}\isamarkupfalse%
\ clarsimp\isanewline
\ \ \ \ \isacommand{using}\isamarkupfalse%
\ finj\isanewline
\ \ \ \ \ \isacommand{apply}\isamarkupfalse%
\ force\isanewline
\ \ \ \ \isacommand{apply}\isamarkupfalse%
{\isacharparenleft}{\kern0pt}rule\ ballI{\isacharparenright}{\kern0pt}\isanewline
\ \ \ \ \isacommand{apply}\isamarkupfalse%
{\isacharparenleft}{\kern0pt}rename{\isacharunderscore}{\kern0pt}tac\ x{\isacharcomma}{\kern0pt}\ case{\isacharunderscore}{\kern0pt}tac\ {\isachardoublequoteopen}x\ {\isasymin}\ {\isacharbraceleft}{\kern0pt}n{\isacharcomma}{\kern0pt}\ n{\isacharprime}{\kern0pt}{\isacharbraceright}{\kern0pt}{\isachardoublequoteclose}{\isacharparenright}{\kern0pt}\isanewline
\ \ \ \ \isacommand{using}\isamarkupfalse%
\ fx\ fn\ fn{\isacharprime}{\kern0pt}\ nH\ n{\isacharprime}{\kern0pt}H\isanewline
\ \ \ \ \isacommand{by}\isamarkupfalse%
\ auto\isanewline
\ \isanewline
\ \ \isacommand{define}\isamarkupfalse%
\ {\isasympi}\ \isakeyword{where}\ {\isachardoublequoteopen}{\isasympi}\ {\isasymequiv}\ Fn{\isacharunderscore}{\kern0pt}perm{\isacharprime}{\kern0pt}{\isacharparenleft}{\kern0pt}f{\isacharparenright}{\kern0pt}{\isachardoublequoteclose}\ \isanewline
\isanewline
\ \ \isacommand{have}\isamarkupfalse%
\ {\isachardoublequoteopen}{\isasymAnd}a\ b\ c{\isachardot}{\kern0pt}\ {\isacharless}{\kern0pt}a{\isacharcomma}{\kern0pt}\ b{\isachargreater}{\kern0pt}\ {\isasymin}\ p\ {\isasymLongrightarrow}\ {\isacharless}{\kern0pt}a{\isacharcomma}{\kern0pt}\ c{\isachargreater}{\kern0pt}\ {\isasymin}\ Fn{\isacharunderscore}{\kern0pt}perm{\isacharparenleft}{\kern0pt}f{\isacharcomma}{\kern0pt}\ p{\isacharparenright}{\kern0pt}\ {\isasymLongrightarrow}\ b\ {\isacharequal}{\kern0pt}\ c{\isachardoublequoteclose}\ \isanewline
\ \ \isacommand{proof}\isamarkupfalse%
\ {\isacharminus}{\kern0pt}\ \isanewline
\ \ \ \ \isacommand{fix}\isamarkupfalse%
\ a\ b\ c\ \isanewline
\ \ \ \ \isacommand{assume}\isamarkupfalse%
\ assms{\isadigit{1}}\ {\isacharcolon}{\kern0pt}\ {\isachardoublequoteopen}{\isacharless}{\kern0pt}a{\isacharcomma}{\kern0pt}\ b{\isachargreater}{\kern0pt}\ {\isasymin}\ p{\isachardoublequoteclose}\ {\isachardoublequoteopen}{\isacharless}{\kern0pt}a{\isacharcomma}{\kern0pt}\ c{\isachargreater}{\kern0pt}\ {\isasymin}\ Fn{\isacharunderscore}{\kern0pt}perm{\isacharparenleft}{\kern0pt}f{\isacharcomma}{\kern0pt}\ p{\isacharparenright}{\kern0pt}{\isachardoublequoteclose}\ \isanewline
\isanewline
\ \ \ \ \isacommand{have}\isamarkupfalse%
\ {\isachardoublequoteopen}{\isasymexists}n{\isasymin}nat{\isachardot}{\kern0pt}\ {\isasymexists}m{\isasymin}nat{\isachardot}{\kern0pt}\ {\isasymexists}l{\isasymin}{\isadigit{2}}{\isachardot}{\kern0pt}\ {\isasymlangle}{\isasymlangle}n{\isacharcomma}{\kern0pt}\ m{\isasymrangle}{\isacharcomma}{\kern0pt}\ l{\isasymrangle}\ {\isasymin}\ p\ {\isasymand}\ {\isacharless}{\kern0pt}a{\isacharcomma}{\kern0pt}\ c{\isachargreater}{\kern0pt}\ {\isacharequal}{\kern0pt}\ {\isasymlangle}{\isasymlangle}f\ {\isacharbackquote}{\kern0pt}\ n{\isacharcomma}{\kern0pt}\ m{\isasymrangle}{\isacharcomma}{\kern0pt}\ l{\isasymrangle}{\isachardoublequoteclose}\isanewline
\ \ \ \ \ \ \isacommand{apply}\isamarkupfalse%
{\isacharparenleft}{\kern0pt}rule\ Fn{\isacharunderscore}{\kern0pt}permE{\isacharparenright}{\kern0pt}\isanewline
\ \ \ \ \ \ \isacommand{using}\isamarkupfalse%
\ assms{\isadigit{1}}\ pinFn\ fin\ \isanewline
\ \ \ \ \ \ \isacommand{by}\isamarkupfalse%
\ auto\isanewline
\ \ \ \ \isacommand{then}\isamarkupfalse%
\ \isacommand{obtain}\isamarkupfalse%
\ x\ y\ z\ \isakeyword{where}\ xyzH{\isacharcolon}{\kern0pt}\ {\isachardoublequoteopen}a\ {\isacharequal}{\kern0pt}\ {\isacharless}{\kern0pt}x{\isacharcomma}{\kern0pt}\ y{\isachargreater}{\kern0pt}{\isachardoublequoteclose}\ {\isachardoublequoteopen}x\ {\isacharequal}{\kern0pt}\ f{\isacharbackquote}{\kern0pt}z{\isachardoublequoteclose}\ {\isachardoublequoteopen}{\isacharless}{\kern0pt}{\isacharless}{\kern0pt}z{\isacharcomma}{\kern0pt}\ y{\isachargreater}{\kern0pt}{\isacharcomma}{\kern0pt}\ c{\isachargreater}{\kern0pt}\ {\isasymin}\ p{\isachardoublequoteclose}\ {\isachardoublequoteopen}x\ {\isasymin}\ nat{\isachardoublequoteclose}\ {\isachardoublequoteopen}y\ {\isasymin}\ nat{\isachardoublequoteclose}\ {\isachardoublequoteopen}z\ {\isasymin}\ nat{\isachardoublequoteclose}\ \isanewline
\ \ \ \ \ \ \isacommand{using}\isamarkupfalse%
\ assms{\isadigit{1}}\ pinFn\ Fn{\isacharunderscore}{\kern0pt}def\ \isanewline
\ \ \ \ \ \ \isacommand{by}\isamarkupfalse%
\ auto\isanewline
\isanewline
\ \ \ \ \isacommand{show}\isamarkupfalse%
\ {\isachardoublequoteopen}b\ {\isacharequal}{\kern0pt}\ c{\isachardoublequoteclose}\ \isanewline
\ \ \ \ \isacommand{proof}\isamarkupfalse%
{\isacharparenleft}{\kern0pt}cases\ {\isachardoublequoteopen}z\ {\isacharequal}{\kern0pt}\ n{\isachardoublequoteclose}{\isacharparenright}{\kern0pt}\isanewline
\ \ \ \ \ \ \isacommand{case}\isamarkupfalse%
\ True\ \isanewline
\ \ \ \ \ \ \isacommand{then}\isamarkupfalse%
\ \isacommand{have}\isamarkupfalse%
\ {\isachardoublequoteopen}x\ {\isacharequal}{\kern0pt}\ n{\isacharprime}{\kern0pt}{\isachardoublequoteclose}\ \isacommand{using}\isamarkupfalse%
\ xyzH\ fn\ \isacommand{by}\isamarkupfalse%
\ auto\isanewline
\ \ \ \ \ \ \isacommand{then}\isamarkupfalse%
\ \isacommand{have}\isamarkupfalse%
\ {\isachardoublequoteopen}n{\isacharprime}{\kern0pt}\ {\isasymin}\ domain{\isacharparenleft}{\kern0pt}domain{\isacharparenleft}{\kern0pt}p{\isacharparenright}{\kern0pt}{\isacharparenright}{\kern0pt}{\isachardoublequoteclose}\ \isacommand{using}\isamarkupfalse%
\ xyzH\ assms{\isadigit{1}}\ \isacommand{by}\isamarkupfalse%
\ auto\ \ \ \ \ \isanewline
\ \ \ \ \ \ \isacommand{then}\isamarkupfalse%
\ \isacommand{show}\isamarkupfalse%
\ {\isacharquery}{\kern0pt}thesis\ \isacommand{using}\isamarkupfalse%
\ n{\isacharprime}{\kern0pt}H\ \isacommand{by}\isamarkupfalse%
\ auto\isanewline
\ \ \ \ \isacommand{next}\isamarkupfalse%
\isanewline
\ \ \ \ \ \ \isacommand{case}\isamarkupfalse%
\ False\isanewline
\ \ \ \ \ \ \isacommand{then}\isamarkupfalse%
\ \isacommand{show}\isamarkupfalse%
\ {\isacharquery}{\kern0pt}thesis\isanewline
\ \ \ \ \ \ \isacommand{proof}\isamarkupfalse%
{\isacharparenleft}{\kern0pt}cases\ {\isachardoublequoteopen}z\ {\isacharequal}{\kern0pt}\ n{\isacharprime}{\kern0pt}{\isachardoublequoteclose}{\isacharparenright}{\kern0pt}\isanewline
\ \ \ \ \ \ \ \ \isacommand{case}\isamarkupfalse%
\ True\isanewline
\ \ \ \ \ \ \ \ \isacommand{then}\isamarkupfalse%
\ \isacommand{have}\isamarkupfalse%
\ {\isachardoublequoteopen}n{\isacharprime}{\kern0pt}\ {\isasymin}\ domain{\isacharparenleft}{\kern0pt}domain{\isacharparenleft}{\kern0pt}p{\isacharparenright}{\kern0pt}{\isacharparenright}{\kern0pt}{\isachardoublequoteclose}\ \isacommand{using}\isamarkupfalse%
\ xyzH\ assms{\isadigit{1}}\ \isacommand{by}\isamarkupfalse%
\ auto\ \ \ \ \ \isanewline
\ \ \ \ \ \ \ \ \isacommand{then}\isamarkupfalse%
\ \isacommand{show}\isamarkupfalse%
\ {\isacharquery}{\kern0pt}thesis\ \isacommand{using}\isamarkupfalse%
\ n{\isacharprime}{\kern0pt}H\ \isacommand{by}\isamarkupfalse%
\ auto\isanewline
\ \ \ \ \ \ \isacommand{next}\isamarkupfalse%
\isanewline
\ \ \ \ \ \ \ \ \isacommand{case}\isamarkupfalse%
\ False\isanewline
\ \ \ \ \ \ \ \ \isacommand{then}\isamarkupfalse%
\ \isacommand{have}\isamarkupfalse%
\ {\isachardoublequoteopen}z\ {\isasymnoteq}\ n\ {\isasymand}\ z\ {\isasymnoteq}\ n{\isacharprime}{\kern0pt}{\isachardoublequoteclose}\ \isacommand{using}\isamarkupfalse%
\ {\isacartoucheopen}z\ {\isasymnoteq}\ n{\isacartoucheclose}\ \isacommand{by}\isamarkupfalse%
\ auto\isanewline
\ \ \ \ \ \ \ \ \isacommand{then}\isamarkupfalse%
\ \isacommand{have}\isamarkupfalse%
\ {\isachardoublequoteopen}f{\isacharbackquote}{\kern0pt}z\ {\isacharequal}{\kern0pt}\ z{\isachardoublequoteclose}\ \isacommand{using}\isamarkupfalse%
\ fx\ xyzH\ \isacommand{by}\isamarkupfalse%
\ auto\isanewline
\ \ \ \ \ \ \ \ \isacommand{then}\isamarkupfalse%
\ \isacommand{have}\isamarkupfalse%
\ {\isachardoublequoteopen}x\ {\isacharequal}{\kern0pt}\ z{\isachardoublequoteclose}\ \isacommand{using}\isamarkupfalse%
\ xyzH\ \isacommand{by}\isamarkupfalse%
\ auto\isanewline
\ \ \ \ \ \ \ \ \isacommand{then}\isamarkupfalse%
\ \isacommand{have}\isamarkupfalse%
\ {\isachardoublequoteopen}{\isacharless}{\kern0pt}a{\isacharcomma}{\kern0pt}\ b{\isachargreater}{\kern0pt}\ {\isasymin}\ p\ {\isasymand}\ {\isacharless}{\kern0pt}a{\isacharcomma}{\kern0pt}\ c{\isachargreater}{\kern0pt}\ {\isasymin}\ p{\isachardoublequoteclose}\ \isacommand{using}\isamarkupfalse%
\ xyzH\ assms{\isadigit{1}}\ \isacommand{by}\isamarkupfalse%
\ auto\isanewline
\ \ \ \ \ \ \ \ \isacommand{then}\isamarkupfalse%
\ \isacommand{show}\isamarkupfalse%
\ {\isacharquery}{\kern0pt}thesis\ \isacommand{using}\isamarkupfalse%
\ assms{\isadigit{1}}\ pinFn\ Fn{\isacharunderscore}{\kern0pt}def\ function{\isacharunderscore}{\kern0pt}def\ \isacommand{by}\isamarkupfalse%
\ auto\isanewline
\ \ \ \ \ \ \isacommand{qed}\isamarkupfalse%
\isanewline
\ \ \ \ \isacommand{qed}\isamarkupfalse%
\isanewline
\ \ \isacommand{qed}\isamarkupfalse%
\isanewline
\ \ \isacommand{then}\isamarkupfalse%
\ \isacommand{have}\isamarkupfalse%
\ compatH\ {\isacharcolon}{\kern0pt}\ {\isachardoublequoteopen}compat{\isacharparenleft}{\kern0pt}p{\isacharcomma}{\kern0pt}\ Fn{\isacharunderscore}{\kern0pt}perm{\isacharparenleft}{\kern0pt}f{\isacharcomma}{\kern0pt}\ p{\isacharparenright}{\kern0pt}{\isacharparenright}{\kern0pt}{\isachardoublequoteclose}\ \isanewline
\ \ \ \ \isacommand{unfolding}\isamarkupfalse%
\ compat{\isacharunderscore}{\kern0pt}def\ compat{\isacharunderscore}{\kern0pt}in{\isacharunderscore}{\kern0pt}def\ \isanewline
\ \ \ \ \isacommand{apply}\isamarkupfalse%
{\isacharparenleft}{\kern0pt}subgoal{\isacharunderscore}{\kern0pt}tac\ {\isachardoublequoteopen}p\ {\isasymunion}\ Fn{\isacharunderscore}{\kern0pt}perm{\isacharparenleft}{\kern0pt}f{\isacharcomma}{\kern0pt}\ p{\isacharparenright}{\kern0pt}\ {\isasymin}\ Fn{\isachardoublequoteclose}{\isacharparenright}{\kern0pt}\isanewline
\ \ \ \ \ \isacommand{apply}\isamarkupfalse%
{\isacharparenleft}{\kern0pt}rule{\isacharunderscore}{\kern0pt}tac\ x{\isacharequal}{\kern0pt}{\isachardoublequoteopen}p\ {\isasymunion}\ Fn{\isacharunderscore}{\kern0pt}perm{\isacharparenleft}{\kern0pt}f{\isacharcomma}{\kern0pt}\ p{\isacharparenright}{\kern0pt}{\isachardoublequoteclose}\ \isakeyword{in}\ bexI{\isacharparenright}{\kern0pt}\isanewline
\ \ \ \ \isacommand{unfolding}\isamarkupfalse%
\ Fn{\isacharunderscore}{\kern0pt}leq{\isacharunderscore}{\kern0pt}def\isanewline
\ \ \ \ \isacommand{using}\isamarkupfalse%
\ pinFn\ Fn{\isacharunderscore}{\kern0pt}perm{\isacharunderscore}{\kern0pt}in{\isacharunderscore}{\kern0pt}Fn\ fin\ \isanewline
\ \ \ \ \ \ \isacommand{apply}\isamarkupfalse%
\ force\isanewline
\ \ \ \ \ \isacommand{apply}\isamarkupfalse%
\ force\isanewline
\ \ \ \ \isacommand{unfolding}\isamarkupfalse%
\ Fn{\isacharunderscore}{\kern0pt}def\ \isanewline
\ \ \ \ \isacommand{apply}\isamarkupfalse%
\ simp\isanewline
\ \ \ \ \isacommand{apply}\isamarkupfalse%
{\isacharparenleft}{\kern0pt}rule\ conjI{\isacharparenright}{\kern0pt}\isanewline
\ \ \ \ \ \isacommand{apply}\isamarkupfalse%
{\isacharparenleft}{\kern0pt}rule\ subsetI{\isacharcomma}{\kern0pt}\ clarsimp{\isacharparenright}{\kern0pt}\isanewline
\ \ \ \ \ \isacommand{apply}\isamarkupfalse%
\ auto{\isacharbrackleft}{\kern0pt}{\isadigit{1}}{\isacharbrackright}{\kern0pt}\isanewline
\ \ \ \ \isacommand{using}\isamarkupfalse%
\ pinFn\ Fn{\isacharunderscore}{\kern0pt}def\isanewline
\ \ \ \ \ \ \isacommand{apply}\isamarkupfalse%
\ force\isanewline
\ \ \ \ \isacommand{apply}\isamarkupfalse%
{\isacharparenleft}{\kern0pt}insert\ Fn{\isacharunderscore}{\kern0pt}perm{\isacharunderscore}{\kern0pt}subset\ {\isacharbrackleft}{\kern0pt}of\ f\ p{\isacharbrackright}{\kern0pt}{\isacharparenright}{\kern0pt}\isanewline
\ \ \ \ \isacommand{using}\isamarkupfalse%
\ pinFn\ Fn{\isacharunderscore}{\kern0pt}def\ fin\isanewline
\ \ \ \ \ \isacommand{apply}\isamarkupfalse%
\ force\isanewline
\ \ \ \ \isacommand{apply}\isamarkupfalse%
{\isacharparenleft}{\kern0pt}rule\ conjI{\isacharparenright}{\kern0pt}\isanewline
\ \ \ \ \isacommand{using}\isamarkupfalse%
\ Un{\isacharunderscore}{\kern0pt}closed\ pinFn\ Fn{\isacharunderscore}{\kern0pt}def\ Fn{\isacharunderscore}{\kern0pt}perm{\isacharunderscore}{\kern0pt}in{\isacharunderscore}{\kern0pt}M\ fin\ \isanewline
\ \ \ \ \ \isacommand{apply}\isamarkupfalse%
\ force\ \isanewline
\ \ \ \ \isacommand{apply}\isamarkupfalse%
{\isacharparenleft}{\kern0pt}rule\ conjI{\isacharparenright}{\kern0pt}\isanewline
\ \ \ \ \ \isacommand{apply}\isamarkupfalse%
{\isacharparenleft}{\kern0pt}simp\ add{\isacharcolon}{\kern0pt}function{\isacharunderscore}{\kern0pt}def{\isacharcomma}{\kern0pt}\ clarsimp{\isacharparenright}{\kern0pt}\isanewline
\ \ \ \ \ \isacommand{apply}\isamarkupfalse%
\ auto{\isacharbrackleft}{\kern0pt}{\isadigit{1}}{\isacharbrackright}{\kern0pt}\isanewline
\ \ \ \ \isacommand{using}\isamarkupfalse%
\ pinFn\ Fn{\isacharunderscore}{\kern0pt}def\ function{\isacharunderscore}{\kern0pt}def\isanewline
\ \ \ \ \ \ \isacommand{apply}\isamarkupfalse%
\ force\isanewline
\ \ \ \ \ \isacommand{apply}\isamarkupfalse%
{\isacharparenleft}{\kern0pt}insert\ function{\isacharunderscore}{\kern0pt}Fn{\isacharunderscore}{\kern0pt}perm\ {\isacharbrackleft}{\kern0pt}of\ f\ p{\isacharbrackright}{\kern0pt}{\isacharparenright}{\kern0pt}\isanewline
\ \ \ \ \isacommand{using}\isamarkupfalse%
\ function{\isacharunderscore}{\kern0pt}def\ fin\ pinFn\isanewline
\ \ \ \ \ \isacommand{apply}\isamarkupfalse%
\ force\ \isanewline
\ \ \ \ \isacommand{apply}\isamarkupfalse%
{\isacharparenleft}{\kern0pt}rule\ conjI{\isacharparenright}{\kern0pt}\isanewline
\ \ \ \ \ \isacommand{apply}\isamarkupfalse%
\ auto{\isacharbrackleft}{\kern0pt}{\isadigit{1}}{\isacharbrackright}{\kern0pt}\isanewline
\ \ \ \ \isacommand{using}\isamarkupfalse%
\ Fn{\isacharunderscore}{\kern0pt}def\ pinFn\ fin\isanewline
\ \ \ \ \ \ \isacommand{apply}\isamarkupfalse%
\ {\isacharparenleft}{\kern0pt}force{\isacharcomma}{\kern0pt}\ blast{\isacharparenright}{\kern0pt}\isanewline
\ \ \ \ \isacommand{apply}\isamarkupfalse%
{\isacharparenleft}{\kern0pt}rule\ conjI{\isacharparenright}{\kern0pt}\isanewline
\ \ \ \ \ \isacommand{apply}\isamarkupfalse%
{\isacharparenleft}{\kern0pt}rule\ finite{\isacharunderscore}{\kern0pt}M{\isacharunderscore}{\kern0pt}union{\isacharparenright}{\kern0pt}\isanewline
\ \ \ \ \isacommand{using}\isamarkupfalse%
\ Fn{\isacharunderscore}{\kern0pt}in{\isacharunderscore}{\kern0pt}M\ transM\ domain{\isacharunderscore}{\kern0pt}closed\ Fn{\isacharunderscore}{\kern0pt}perm{\isacharunderscore}{\kern0pt}in{\isacharunderscore}{\kern0pt}M\ pinFn\ fin\ Fn{\isacharunderscore}{\kern0pt}def\ Fn{\isacharunderscore}{\kern0pt}perm{\isacharunderscore}{\kern0pt}in{\isacharunderscore}{\kern0pt}Fn\isanewline
\ \ \ \ \ \ \ \ \isacommand{apply}\isamarkupfalse%
\ auto{\isacharbrackleft}{\kern0pt}{\isadigit{4}}{\isacharbrackright}{\kern0pt}\isanewline
\ \ \ \ \isacommand{apply}\isamarkupfalse%
{\isacharparenleft}{\kern0pt}rule\ Un{\isacharunderscore}{\kern0pt}subset{\isacharparenright}{\kern0pt}\isanewline
\ \ \ \ \isacommand{using}\isamarkupfalse%
\ pinFn\ Fn{\isacharunderscore}{\kern0pt}def\ fin\ \isanewline
\ \ \ \ \ \isacommand{apply}\isamarkupfalse%
\ auto{\isacharbrackleft}{\kern0pt}{\isadigit{1}}{\isacharbrackright}{\kern0pt}\isanewline
\ \ \ \ \isacommand{using}\isamarkupfalse%
\ pinFn\ fin\ Fn{\isacharunderscore}{\kern0pt}perm{\isacharunderscore}{\kern0pt}subset\ \isanewline
\ \ \ \ \isacommand{apply}\isamarkupfalse%
\ blast\isanewline
\ \ \ \ \isacommand{done}\isamarkupfalse%
\isanewline
\isanewline
\ \ \isacommand{have}\isamarkupfalse%
\ piinfix\ {\isacharcolon}{\kern0pt}\ {\isachardoublequoteopen}{\isasympi}\ {\isasymin}\ Fix{\isacharparenleft}{\kern0pt}e{\isacharparenright}{\kern0pt}{\isachardoublequoteclose}\ \isanewline
\ \ \ \ \isacommand{unfolding}\isamarkupfalse%
\ Fix{\isacharunderscore}{\kern0pt}def\isanewline
\ \ \ \ \isacommand{apply}\isamarkupfalse%
\ clarsimp\isanewline
\ \ \ \ \isacommand{apply}\isamarkupfalse%
{\isacharparenleft}{\kern0pt}rule{\isacharunderscore}{\kern0pt}tac\ x{\isacharequal}{\kern0pt}f\ \isakeyword{in}\ bexI{\isacharcomma}{\kern0pt}\ rule\ conjI{\isacharparenright}{\kern0pt}\isanewline
\ \ \ \ \isacommand{using}\isamarkupfalse%
\ {\isasympi}{\isacharunderscore}{\kern0pt}def\ \isanewline
\ \ \ \ \ \ \isacommand{apply}\isamarkupfalse%
\ simp\isanewline
\ \ \ \ \ \isacommand{apply}\isamarkupfalse%
\ clarsimp\isanewline
\ \ \ \ \ \isacommand{apply}\isamarkupfalse%
{\isacharparenleft}{\kern0pt}rule\ function{\isacharunderscore}{\kern0pt}apply{\isacharunderscore}{\kern0pt}equality{\isacharparenright}{\kern0pt}\isanewline
\ \ \ \ \isacommand{apply}\isamarkupfalse%
{\isacharparenleft}{\kern0pt}rename{\isacharunderscore}{\kern0pt}tac\ x{\isacharcomma}{\kern0pt}\ case{\isacharunderscore}{\kern0pt}tac\ {\isachardoublequoteopen}x\ {\isasymin}\ {\isacharbraceleft}{\kern0pt}n{\isacharcomma}{\kern0pt}\ n{\isacharprime}{\kern0pt}{\isacharbraceright}{\kern0pt}{\isachardoublequoteclose}{\isacharparenright}{\kern0pt}\isanewline
\ \ \ \ \isacommand{using}\isamarkupfalse%
\ f{\isacharunderscore}{\kern0pt}def\ nH\ n{\isacharprime}{\kern0pt}H\ eH\ \isanewline
\ \ \ \ \ \ \ \isacommand{apply}\isamarkupfalse%
\ auto{\isacharbrackleft}{\kern0pt}{\isadigit{2}}{\isacharbrackright}{\kern0pt}\isanewline
\ \ \ \ \isacommand{using}\isamarkupfalse%
\ fin\ nat{\isacharunderscore}{\kern0pt}perms{\isacharunderscore}{\kern0pt}def\ bij{\isacharunderscore}{\kern0pt}def\ inj{\isacharunderscore}{\kern0pt}def\ Pi{\isacharunderscore}{\kern0pt}def\isanewline
\ \ \ \ \isacommand{by}\isamarkupfalse%
\ auto\isanewline
\isanewline
\ \ \isacommand{have}\isamarkupfalse%
\ fnneq\ {\isacharcolon}{\kern0pt}\ {\isachardoublequoteopen}f{\isacharbackquote}{\kern0pt}n\ {\isasymnoteq}\ n{\isachardoublequoteclose}\isanewline
\ \ \ \ \isacommand{apply}\isamarkupfalse%
{\isacharparenleft}{\kern0pt}rule\ ccontr{\isacharcomma}{\kern0pt}\ simp{\isacharparenright}{\kern0pt}\isanewline
\ \ \ \ \isacommand{using}\isamarkupfalse%
\ fn\ n{\isacharprime}{\kern0pt}H\isanewline
\ \ \ \ \isacommand{by}\isamarkupfalse%
\ auto\isanewline
\isanewline
\ \ \isacommand{have}\isamarkupfalse%
\ pi{\isacharunderscore}{\kern0pt}preserves{\isacharunderscore}{\kern0pt}F{\isacharprime}{\kern0pt}\ {\isacharcolon}{\kern0pt}\ {\isachardoublequoteopen}Pn{\isacharunderscore}{\kern0pt}auto{\isacharparenleft}{\kern0pt}{\isasympi}{\isacharparenright}{\kern0pt}{\isacharbackquote}{\kern0pt}F{\isacharprime}{\kern0pt}\ {\isacharequal}{\kern0pt}\ F{\isacharprime}{\kern0pt}{\isachardoublequoteclose}\ \isanewline
\ \ \ \ \isacommand{using}\isamarkupfalse%
\ piinfix\ eH\ sym{\isacharunderscore}{\kern0pt}def\ \isanewline
\ \ \ \ \isacommand{by}\isamarkupfalse%
\ auto\isanewline
\isanewline
\ \ \isacommand{have}\isamarkupfalse%
\ mapeq\ {\isacharcolon}{\kern0pt}\ {\isachardoublequoteopen}map{\isacharparenleft}{\kern0pt}{\isasymlambda}x{\isachardot}{\kern0pt}\ Pn{\isacharunderscore}{\kern0pt}auto{\isacharparenleft}{\kern0pt}{\isasympi}{\isacharparenright}{\kern0pt}{\isacharbackquote}{\kern0pt}x{\isacharcomma}{\kern0pt}\ {\isacharbrackleft}{\kern0pt}F{\isacharprime}{\kern0pt}{\isacharcomma}{\kern0pt}\ check{\isacharparenleft}{\kern0pt}i{\isacharparenright}{\kern0pt}{\isacharcomma}{\kern0pt}\ binmap{\isacharunderscore}{\kern0pt}row{\isacharprime}{\kern0pt}{\isacharparenleft}{\kern0pt}n{\isacharparenright}{\kern0pt}{\isacharbrackright}{\kern0pt}{\isacharparenright}{\kern0pt}\ {\isacharequal}{\kern0pt}\ {\isacharbrackleft}{\kern0pt}F{\isacharprime}{\kern0pt}{\isacharcomma}{\kern0pt}\ check{\isacharparenleft}{\kern0pt}i{\isacharparenright}{\kern0pt}{\isacharcomma}{\kern0pt}\ binmap{\isacharunderscore}{\kern0pt}row{\isacharprime}{\kern0pt}{\isacharparenleft}{\kern0pt}f{\isacharbackquote}{\kern0pt}n{\isacharparenright}{\kern0pt}{\isacharbrackright}{\kern0pt}{\isachardoublequoteclose}\ \isanewline
\ \ \ \ \isacommand{apply}\isamarkupfalse%
\ auto\isanewline
\ \ \ \ \ \ \isacommand{apply}\isamarkupfalse%
{\isacharparenleft}{\kern0pt}rule\ pi{\isacharunderscore}{\kern0pt}preserves{\isacharunderscore}{\kern0pt}F{\isacharprime}{\kern0pt}{\isacharparenright}{\kern0pt}\isanewline
\ \ \ \ \ \isacommand{apply}\isamarkupfalse%
{\isacharparenleft}{\kern0pt}rule\ check{\isacharunderscore}{\kern0pt}fixpoint{\isacharparenright}{\kern0pt}\isanewline
\ \ \ \ \isacommand{using}\isamarkupfalse%
\ {\isasympi}{\isacharunderscore}{\kern0pt}def\ fin\ Fn{\isacharunderscore}{\kern0pt}perm{\isacharprime}{\kern0pt}{\isacharunderscore}{\kern0pt}is{\isacharunderscore}{\kern0pt}P{\isacharunderscore}{\kern0pt}auto\ iH\ nat{\isacharunderscore}{\kern0pt}in{\isacharunderscore}{\kern0pt}M\ transM\isanewline
\ \ \ \ \ \ \isacommand{apply}\isamarkupfalse%
\ auto{\isacharbrackleft}{\kern0pt}{\isadigit{2}}{\isacharbrackright}{\kern0pt}\isanewline
\ \ \ \ \isacommand{unfolding}\isamarkupfalse%
\ {\isasympi}{\isacharunderscore}{\kern0pt}def\isanewline
\ \ \ \ \isacommand{apply}\isamarkupfalse%
{\isacharparenleft}{\kern0pt}rule\ binmap{\isacharunderscore}{\kern0pt}row{\isacharprime}{\kern0pt}{\isacharunderscore}{\kern0pt}pauto{\isacharparenright}{\kern0pt}\isanewline
\ \ \ \ \isacommand{using}\isamarkupfalse%
\ nH\ fin\isanewline
\ \ \ \ \isacommand{by}\isamarkupfalse%
\ auto\isanewline
\isanewline
\ \ \isacommand{have}\isamarkupfalse%
\ {\isachardoublequoteopen}ForcesHS{\isacharparenleft}{\kern0pt}{\isasympi}{\isacharbackquote}{\kern0pt}p{\isacharcomma}{\kern0pt}\ {\isasymphi}{\isacharcomma}{\kern0pt}\ map{\isacharparenleft}{\kern0pt}{\isasymlambda}x{\isachardot}{\kern0pt}\ Pn{\isacharunderscore}{\kern0pt}auto{\isacharparenleft}{\kern0pt}{\isasympi}{\isacharparenright}{\kern0pt}{\isacharbackquote}{\kern0pt}x{\isacharcomma}{\kern0pt}\ {\isacharbrackleft}{\kern0pt}F{\isacharprime}{\kern0pt}{\isacharcomma}{\kern0pt}\ check{\isacharparenleft}{\kern0pt}i{\isacharparenright}{\kern0pt}{\isacharcomma}{\kern0pt}\ binmap{\isacharunderscore}{\kern0pt}row{\isacharprime}{\kern0pt}{\isacharparenleft}{\kern0pt}n{\isacharparenright}{\kern0pt}{\isacharbrackright}{\kern0pt}{\isacharparenright}{\kern0pt}{\isacharparenright}{\kern0pt}{\isachardoublequoteclose}\ \isanewline
\ \ \ \ \isacommand{apply}\isamarkupfalse%
{\isacharparenleft}{\kern0pt}rule\ iffD{\isadigit{1}}{\isacharcomma}{\kern0pt}\ rule\ symmetry{\isacharunderscore}{\kern0pt}lemma{\isacharparenright}{\kern0pt}\isanewline
\ \ \ \ \isacommand{using}\isamarkupfalse%
\ {\isasymphi}{\isacharunderscore}{\kern0pt}def\ {\isasympi}{\isacharunderscore}{\kern0pt}def\ Fn{\isacharunderscore}{\kern0pt}perms{\isacharunderscore}{\kern0pt}def\ fin\ Fn{\isacharunderscore}{\kern0pt}perm{\isacharprime}{\kern0pt}{\isacharunderscore}{\kern0pt}is{\isacharunderscore}{\kern0pt}P{\isacharunderscore}{\kern0pt}auto\ \isanewline
\ \ \ \ \ \ \ \ \ \ \isacommand{apply}\isamarkupfalse%
\ auto{\isacharbrackleft}{\kern0pt}{\isadigit{3}}{\isacharbrackright}{\kern0pt}\isanewline
\ \ \ \ \ \ \ \isacommand{apply}\isamarkupfalse%
\ auto{\isacharbrackleft}{\kern0pt}{\isadigit{1}}{\isacharbrackright}{\kern0pt}\isanewline
\ \ \ \ \isacommand{using}\isamarkupfalse%
\ listin\ check{\isacharunderscore}{\kern0pt}in{\isacharunderscore}{\kern0pt}HS\ iH\ nat{\isacharunderscore}{\kern0pt}in{\isacharunderscore}{\kern0pt}M\ transM\ \isanewline
\ \ \ \ \ \ \ \ \ \isacommand{apply}\isamarkupfalse%
\ auto{\isacharbrackleft}{\kern0pt}{\isadigit{2}}{\isacharbrackright}{\kern0pt}\isanewline
\ \ \ \ \ \ \ \isacommand{apply}\isamarkupfalse%
{\isacharparenleft}{\kern0pt}rule\ binmap{\isacharunderscore}{\kern0pt}row{\isacharprime}{\kern0pt}{\isacharunderscore}{\kern0pt}HS{\isacharparenright}{\kern0pt}\isanewline
\ \ \ \ \isacommand{using}\isamarkupfalse%
\ nH\ \isanewline
\ \ \ \ \ \ \ \isacommand{apply}\isamarkupfalse%
\ simp\isanewline
\ \ \ \ \ \ \isacommand{apply}\isamarkupfalse%
{\isacharparenleft}{\kern0pt}simp\ add{\isacharcolon}{\kern0pt}{\isasymphi}{\isacharunderscore}{\kern0pt}def{\isacharparenright}{\kern0pt}\isanewline
\ \ \ \ \ \ \isacommand{apply}\isamarkupfalse%
\ {\isacharparenleft}{\kern0pt}simp\ del{\isacharcolon}{\kern0pt}FOL{\isacharunderscore}{\kern0pt}sats{\isacharunderscore}{\kern0pt}iff\ pair{\isacharunderscore}{\kern0pt}abs\ add{\isacharcolon}{\kern0pt}\ fm{\isacharunderscore}{\kern0pt}defs\ nat{\isacharunderscore}{\kern0pt}simp{\isacharunderscore}{\kern0pt}union{\isacharparenright}{\kern0pt}\isanewline
\ \ \ \ \isacommand{using}\isamarkupfalse%
\ pinFn\ pH\ \isanewline
\ \ \ \ \isacommand{by}\isamarkupfalse%
\ auto\isanewline
\ \ \isacommand{then}\isamarkupfalse%
\ \isacommand{have}\isamarkupfalse%
\ piForces{\isacharcolon}{\kern0pt}\ {\isachardoublequoteopen}ForcesHS{\isacharparenleft}{\kern0pt}{\isasympi}{\isacharbackquote}{\kern0pt}p{\isacharcomma}{\kern0pt}\ {\isasymphi}{\isacharcomma}{\kern0pt}\ {\isacharbrackleft}{\kern0pt}F{\isacharprime}{\kern0pt}{\isacharcomma}{\kern0pt}\ check{\isacharparenleft}{\kern0pt}i{\isacharparenright}{\kern0pt}{\isacharcomma}{\kern0pt}\ binmap{\isacharunderscore}{\kern0pt}row{\isacharprime}{\kern0pt}{\isacharparenleft}{\kern0pt}f{\isacharbackquote}{\kern0pt}n{\isacharparenright}{\kern0pt}{\isacharbrackright}{\kern0pt}{\isacharparenright}{\kern0pt}{\isachardoublequoteclose}\ \ \isanewline
\ \ \ \ \isacommand{using}\isamarkupfalse%
\ mapeq\ \isanewline
\ \ \ \ \isacommand{by}\isamarkupfalse%
\ auto\isanewline
\isanewline
\ \ \isacommand{obtain}\isamarkupfalse%
\ q\ \isakeyword{where}\ qH\ {\isacharcolon}{\kern0pt}\ {\isachardoublequoteopen}{\isacharless}{\kern0pt}q{\isacharcomma}{\kern0pt}\ p{\isachargreater}{\kern0pt}\ {\isasymin}\ Fn{\isacharunderscore}{\kern0pt}leq{\isachardoublequoteclose}\ {\isachardoublequoteopen}{\isacharless}{\kern0pt}q{\isacharcomma}{\kern0pt}\ {\isasympi}{\isacharbackquote}{\kern0pt}p{\isachargreater}{\kern0pt}\ {\isasymin}\ Fn{\isacharunderscore}{\kern0pt}leq{\isachardoublequoteclose}\ {\isachardoublequoteopen}q\ {\isasymin}\ Fn{\isachardoublequoteclose}\ \isanewline
\ \ \ \ \isacommand{using}\isamarkupfalse%
\ compatH\ compat{\isacharunderscore}{\kern0pt}def\ {\isasympi}{\isacharunderscore}{\kern0pt}def\ Fn{\isacharunderscore}{\kern0pt}perm{\isacharprime}{\kern0pt}{\isacharunderscore}{\kern0pt}eq\ fin\ pinFn\ \isanewline
\ \ \ \ \isacommand{by}\isamarkupfalse%
\ force\isanewline
\isanewline
\ \ \isacommand{have}\isamarkupfalse%
\ listin{\isacharprime}{\kern0pt}{\isacharprime}{\kern0pt}\ {\isacharcolon}{\kern0pt}\ {\isachardoublequoteopen}{\isacharbrackleft}{\kern0pt}F{\isacharprime}{\kern0pt}{\isacharcomma}{\kern0pt}\ check{\isacharparenleft}{\kern0pt}i{\isacharparenright}{\kern0pt}{\isacharcomma}{\kern0pt}\ binmap{\isacharunderscore}{\kern0pt}row{\isacharprime}{\kern0pt}{\isacharparenleft}{\kern0pt}n{\isacharparenright}{\kern0pt}{\isacharbrackright}{\kern0pt}\ {\isasymin}\ list{\isacharparenleft}{\kern0pt}HS{\isacharparenright}{\kern0pt}{\isachardoublequoteclose}\ \isanewline
\ \ \ \ \isacommand{apply}\isamarkupfalse%
\ auto{\isacharbrackleft}{\kern0pt}{\isadigit{1}}{\isacharbrackright}{\kern0pt}\isanewline
\ \ \ \ \isacommand{using}\isamarkupfalse%
\ assms\ HS{\isacharunderscore}{\kern0pt}iff\ P{\isacharunderscore}{\kern0pt}name{\isacharunderscore}{\kern0pt}in{\isacharunderscore}{\kern0pt}M\ \isanewline
\ \ \ \ \ \ \isacommand{apply}\isamarkupfalse%
\ auto{\isacharbrackleft}{\kern0pt}{\isadigit{1}}{\isacharbrackright}{\kern0pt}\isanewline
\ \ \ \ \ \isacommand{apply}\isamarkupfalse%
{\isacharparenleft}{\kern0pt}rule\ check{\isacharunderscore}{\kern0pt}in{\isacharunderscore}{\kern0pt}HS{\isacharparenright}{\kern0pt}\isanewline
\ \ \ \ \isacommand{using}\isamarkupfalse%
\ iH\ nat{\isacharunderscore}{\kern0pt}in{\isacharunderscore}{\kern0pt}M\ transM\ \isanewline
\ \ \ \ \ \isacommand{apply}\isamarkupfalse%
\ force\isanewline
\ \ \ \ \isacommand{apply}\isamarkupfalse%
{\isacharparenleft}{\kern0pt}rule\ binmap{\isacharunderscore}{\kern0pt}row{\isacharprime}{\kern0pt}{\isacharunderscore}{\kern0pt}HS{\isacharparenright}{\kern0pt}\isanewline
\ \ \ \ \isacommand{using}\isamarkupfalse%
\ nH\isanewline
\ \ \ \ \isacommand{by}\isamarkupfalse%
\ auto\isanewline
\isanewline
\ \ \isacommand{have}\isamarkupfalse%
\ listin{\isacharprime}{\kern0pt}{\isacharprime}{\kern0pt}{\isacharprime}{\kern0pt}\ {\isacharcolon}{\kern0pt}\ {\isachardoublequoteopen}{\isacharbrackleft}{\kern0pt}F{\isacharprime}{\kern0pt}{\isacharcomma}{\kern0pt}\ check{\isacharparenleft}{\kern0pt}i{\isacharparenright}{\kern0pt}{\isacharcomma}{\kern0pt}\ binmap{\isacharunderscore}{\kern0pt}row{\isacharprime}{\kern0pt}{\isacharparenleft}{\kern0pt}f{\isacharbackquote}{\kern0pt}n{\isacharparenright}{\kern0pt}{\isacharbrackright}{\kern0pt}\ {\isasymin}\ list{\isacharparenleft}{\kern0pt}HS{\isacharparenright}{\kern0pt}{\isachardoublequoteclose}\ \isanewline
\ \ \ \ \isacommand{apply}\isamarkupfalse%
\ auto{\isacharbrackleft}{\kern0pt}{\isadigit{1}}{\isacharbrackright}{\kern0pt}\isanewline
\ \ \ \ \isacommand{using}\isamarkupfalse%
\ assms\ HS{\isacharunderscore}{\kern0pt}iff\ P{\isacharunderscore}{\kern0pt}name{\isacharunderscore}{\kern0pt}in{\isacharunderscore}{\kern0pt}M\ \isanewline
\ \ \ \ \ \ \isacommand{apply}\isamarkupfalse%
\ auto{\isacharbrackleft}{\kern0pt}{\isadigit{1}}{\isacharbrackright}{\kern0pt}\isanewline
\ \ \ \ \ \isacommand{apply}\isamarkupfalse%
{\isacharparenleft}{\kern0pt}rule\ check{\isacharunderscore}{\kern0pt}in{\isacharunderscore}{\kern0pt}HS{\isacharparenright}{\kern0pt}\isanewline
\ \ \ \ \isacommand{using}\isamarkupfalse%
\ iH\ nat{\isacharunderscore}{\kern0pt}in{\isacharunderscore}{\kern0pt}M\ transM\ \isanewline
\ \ \ \ \ \isacommand{apply}\isamarkupfalse%
\ force\isanewline
\ \ \ \ \isacommand{apply}\isamarkupfalse%
{\isacharparenleft}{\kern0pt}rule\ binmap{\isacharunderscore}{\kern0pt}row{\isacharprime}{\kern0pt}{\isacharunderscore}{\kern0pt}HS{\isacharparenright}{\kern0pt}\isanewline
\ \ \ \ \isacommand{apply}\isamarkupfalse%
{\isacharparenleft}{\kern0pt}rule\ function{\isacharunderscore}{\kern0pt}value{\isacharunderscore}{\kern0pt}in{\isacharparenright}{\kern0pt}\isanewline
\ \ \ \ \isacommand{using}\isamarkupfalse%
\ nH\ fin\ nat{\isacharunderscore}{\kern0pt}perms{\isacharunderscore}{\kern0pt}def\ bij{\isacharunderscore}{\kern0pt}def\ inj{\isacharunderscore}{\kern0pt}def\isanewline
\ \ \ \ \isacommand{by}\isamarkupfalse%
\ auto\isanewline
\isanewline
\ \ \isacommand{have}\isamarkupfalse%
\ {\isachardoublequoteopen}ForcesHS{\isacharparenleft}{\kern0pt}q{\isacharcomma}{\kern0pt}\ {\isasymphi}{\isacharcomma}{\kern0pt}\ {\isacharbrackleft}{\kern0pt}F{\isacharprime}{\kern0pt}{\isacharcomma}{\kern0pt}\ check{\isacharparenleft}{\kern0pt}i{\isacharparenright}{\kern0pt}{\isacharcomma}{\kern0pt}\ binmap{\isacharunderscore}{\kern0pt}row{\isacharprime}{\kern0pt}{\isacharparenleft}{\kern0pt}n{\isacharparenright}{\kern0pt}{\isacharbrackright}{\kern0pt}{\isacharparenright}{\kern0pt}{\isachardoublequoteclose}\isanewline
\ \ \ \ \isacommand{apply}\isamarkupfalse%
{\isacharparenleft}{\kern0pt}rule{\isacharunderscore}{\kern0pt}tac\ p{\isacharequal}{\kern0pt}{\isachardoublequoteopen}p{\isachardoublequoteclose}\ \isakeyword{in}\ HS{\isacharunderscore}{\kern0pt}strengthening{\isacharunderscore}{\kern0pt}lemma{\isacharparenright}{\kern0pt}\isanewline
\ \ \ \ \isacommand{using}\isamarkupfalse%
\ pinFn\ qH\ {\isasymphi}{\isacharunderscore}{\kern0pt}def\ \isanewline
\ \ \ \ \ \ \ \ \ \ \isacommand{apply}\isamarkupfalse%
\ auto{\isacharbrackleft}{\kern0pt}{\isadigit{4}}{\isacharbrackright}{\kern0pt}\isanewline
\ \ \ \ \ \ \isacommand{apply}\isamarkupfalse%
{\isacharparenleft}{\kern0pt}rule{\isacharunderscore}{\kern0pt}tac\ A{\isacharequal}{\kern0pt}{\isachardoublequoteopen}list{\isacharparenleft}{\kern0pt}HS{\isacharparenright}{\kern0pt}{\isachardoublequoteclose}\ \isakeyword{in}\ subsetD{\isacharcomma}{\kern0pt}\ rule\ list{\isacharunderscore}{\kern0pt}mono{\isacharparenright}{\kern0pt}\isanewline
\ \ \ \ \isacommand{using}\isamarkupfalse%
\ HS{\isacharunderscore}{\kern0pt}iff\ P{\isacharunderscore}{\kern0pt}name{\isacharunderscore}{\kern0pt}in{\isacharunderscore}{\kern0pt}M\ listin{\isacharprime}{\kern0pt}{\isacharprime}{\kern0pt}\isanewline
\ \ \ \ \ \ \ \isacommand{apply}\isamarkupfalse%
\ {\isacharparenleft}{\kern0pt}force{\isacharcomma}{\kern0pt}\ force{\isacharparenright}{\kern0pt}\isanewline
\ \ \ \ \isacommand{using}\isamarkupfalse%
\ {\isasymphi}{\isacharunderscore}{\kern0pt}def\isanewline
\ \ \ \ \ \isacommand{apply}\isamarkupfalse%
\ {\isacharparenleft}{\kern0pt}simp\ del{\isacharcolon}{\kern0pt}FOL{\isacharunderscore}{\kern0pt}sats{\isacharunderscore}{\kern0pt}iff\ pair{\isacharunderscore}{\kern0pt}abs\ add{\isacharcolon}{\kern0pt}\ fm{\isacharunderscore}{\kern0pt}defs\ nat{\isacharunderscore}{\kern0pt}simp{\isacharunderscore}{\kern0pt}union{\isacharparenright}{\kern0pt}\isanewline
\ \ \ \ \isacommand{using}\isamarkupfalse%
\ pH\isanewline
\ \ \ \ \isacommand{by}\isamarkupfalse%
\ auto\isanewline
\ \ \isacommand{then}\isamarkupfalse%
\ \isacommand{have}\isamarkupfalse%
\ ForcesH{\isadigit{1}}{\isacharcolon}{\kern0pt}\ {\isachardoublequoteopen}{\isacharparenleft}{\kern0pt}{\isasymforall}H{\isachardot}{\kern0pt}\ M{\isacharunderscore}{\kern0pt}generic{\isacharparenleft}{\kern0pt}H{\isacharparenright}{\kern0pt}\ {\isasymand}\ q{\isasymin}H\ \ {\isasymlongrightarrow}\ \ SymExt{\isacharparenleft}{\kern0pt}H{\isacharparenright}{\kern0pt}{\isacharcomma}{\kern0pt}\ map{\isacharparenleft}{\kern0pt}val{\isacharparenleft}{\kern0pt}H{\isacharparenright}{\kern0pt}{\isacharcomma}{\kern0pt}{\isacharbrackleft}{\kern0pt}F{\isacharprime}{\kern0pt}{\isacharcomma}{\kern0pt}\ check{\isacharparenleft}{\kern0pt}i{\isacharparenright}{\kern0pt}{\isacharcomma}{\kern0pt}\ binmap{\isacharunderscore}{\kern0pt}row{\isacharprime}{\kern0pt}{\isacharparenleft}{\kern0pt}n{\isacharparenright}{\kern0pt}{\isacharbrackright}{\kern0pt}{\isacharparenright}{\kern0pt}\ {\isasymTurnstile}\ {\isasymphi}{\isacharparenright}{\kern0pt}{\isachardoublequoteclose}\isanewline
\ \ \ \ \isacommand{apply}\isamarkupfalse%
{\isacharparenleft}{\kern0pt}rule{\isacharunderscore}{\kern0pt}tac\ iffD{\isadigit{1}}{\isacharparenright}{\kern0pt}\isanewline
\ \ \ \ \ \isacommand{apply}\isamarkupfalse%
{\isacharparenleft}{\kern0pt}rule\ definition{\isacharunderscore}{\kern0pt}of{\isacharunderscore}{\kern0pt}forcing{\isacharunderscore}{\kern0pt}HS{\isacharparenright}{\kern0pt}\isanewline
\ \ \ \ \isacommand{using}\isamarkupfalse%
\ pinFn\ {\isasymphi}{\isacharunderscore}{\kern0pt}def\ listin{\isacharprime}{\kern0pt}{\isacharprime}{\kern0pt}\ qH\isanewline
\ \ \ \ \ \ \ \ \isacommand{apply}\isamarkupfalse%
\ auto{\isacharbrackleft}{\kern0pt}{\isadigit{3}}{\isacharbrackright}{\kern0pt}\isanewline
\ \ \ \ \ \isacommand{apply}\isamarkupfalse%
{\isacharparenleft}{\kern0pt}simp\ add{\isacharcolon}{\kern0pt}{\isasymphi}{\isacharunderscore}{\kern0pt}def{\isacharparenright}{\kern0pt}\isanewline
\ \ \ \ \ \isacommand{apply}\isamarkupfalse%
\ {\isacharparenleft}{\kern0pt}simp\ del{\isacharcolon}{\kern0pt}FOL{\isacharunderscore}{\kern0pt}sats{\isacharunderscore}{\kern0pt}iff\ pair{\isacharunderscore}{\kern0pt}abs\ add{\isacharcolon}{\kern0pt}\ fm{\isacharunderscore}{\kern0pt}defs\ nat{\isacharunderscore}{\kern0pt}simp{\isacharunderscore}{\kern0pt}union{\isacharparenright}{\kern0pt}\isanewline
\ \ \ \ \isacommand{by}\isamarkupfalse%
\ auto\isanewline
\isanewline
\ \ \isacommand{have}\isamarkupfalse%
\ {\isachardoublequoteopen}ForcesHS{\isacharparenleft}{\kern0pt}q{\isacharcomma}{\kern0pt}\ {\isasymphi}{\isacharcomma}{\kern0pt}\ {\isacharbrackleft}{\kern0pt}F{\isacharprime}{\kern0pt}{\isacharcomma}{\kern0pt}\ check{\isacharparenleft}{\kern0pt}i{\isacharparenright}{\kern0pt}{\isacharcomma}{\kern0pt}\ binmap{\isacharunderscore}{\kern0pt}row{\isacharprime}{\kern0pt}{\isacharparenleft}{\kern0pt}f{\isacharbackquote}{\kern0pt}n{\isacharparenright}{\kern0pt}{\isacharbrackright}{\kern0pt}{\isacharparenright}{\kern0pt}{\isachardoublequoteclose}\isanewline
\ \ \ \ \isacommand{apply}\isamarkupfalse%
{\isacharparenleft}{\kern0pt}rule{\isacharunderscore}{\kern0pt}tac\ p{\isacharequal}{\kern0pt}{\isachardoublequoteopen}{\isasympi}{\isacharbackquote}{\kern0pt}p{\isachardoublequoteclose}\ \isakeyword{in}\ HS{\isacharunderscore}{\kern0pt}strengthening{\isacharunderscore}{\kern0pt}lemma{\isacharparenright}{\kern0pt}\isanewline
\ \ \ \ \ \ \ \ \ \ \isacommand{apply}\isamarkupfalse%
{\isacharparenleft}{\kern0pt}rule\ P{\isacharunderscore}{\kern0pt}auto{\isacharunderscore}{\kern0pt}value{\isacharparenright}{\kern0pt}\isanewline
\ \ \ \ \isacommand{using}\isamarkupfalse%
\ {\isasympi}{\isacharunderscore}{\kern0pt}def\ Fn{\isacharunderscore}{\kern0pt}perm{\isacharprime}{\kern0pt}{\isacharunderscore}{\kern0pt}is{\isacharunderscore}{\kern0pt}P{\isacharunderscore}{\kern0pt}auto\ fin\ \isanewline
\ \ \ \ \ \ \ \ \ \ \ \isacommand{apply}\isamarkupfalse%
\ force\isanewline
\ \ \ \ \isacommand{using}\isamarkupfalse%
\ pinFn\ qH\ {\isasymphi}{\isacharunderscore}{\kern0pt}def\ \isanewline
\ \ \ \ \ \ \ \ \ \ \isacommand{apply}\isamarkupfalse%
\ auto{\isacharbrackleft}{\kern0pt}{\isadigit{4}}{\isacharbrackright}{\kern0pt}\isanewline
\ \ \ \ \ \ \isacommand{apply}\isamarkupfalse%
\ auto{\isacharbrackleft}{\kern0pt}{\isadigit{1}}{\isacharbrackright}{\kern0pt}\isanewline
\ \ \ \ \isacommand{using}\isamarkupfalse%
\ assms\ HS{\isacharunderscore}{\kern0pt}iff\ P{\isacharunderscore}{\kern0pt}name{\isacharunderscore}{\kern0pt}in{\isacharunderscore}{\kern0pt}M\ check{\isacharunderscore}{\kern0pt}in{\isacharunderscore}{\kern0pt}M\ iH\ nat{\isacharunderscore}{\kern0pt}in{\isacharunderscore}{\kern0pt}M\ transM\isanewline
\ \ \ \ \ \ \ \ \isacommand{apply}\isamarkupfalse%
\ auto{\isacharbrackleft}{\kern0pt}{\isadigit{2}}{\isacharbrackright}{\kern0pt}\isanewline
\ \ \ \ \ \ \isacommand{apply}\isamarkupfalse%
{\isacharparenleft}{\kern0pt}rule\ binmap{\isacharunderscore}{\kern0pt}row{\isacharprime}{\kern0pt}{\isacharunderscore}{\kern0pt}in{\isacharunderscore}{\kern0pt}M{\isacharparenright}{\kern0pt}\isanewline
\ \ \ \ \ \ \isacommand{apply}\isamarkupfalse%
{\isacharparenleft}{\kern0pt}rule\ function{\isacharunderscore}{\kern0pt}value{\isacharunderscore}{\kern0pt}in{\isacharparenright}{\kern0pt}\isanewline
\ \ \ \ \isacommand{using}\isamarkupfalse%
\ fin\ nat{\isacharunderscore}{\kern0pt}perms{\isacharunderscore}{\kern0pt}def\ bij{\isacharunderscore}{\kern0pt}def\ inj{\isacharunderscore}{\kern0pt}def\ nH\isanewline
\ \ \ \ \ \ \ \isacommand{apply}\isamarkupfalse%
\ {\isacharparenleft}{\kern0pt}force{\isacharcomma}{\kern0pt}\ force{\isacharparenright}{\kern0pt}\isanewline
\ \ \ \ \isacommand{using}\isamarkupfalse%
\ {\isasymphi}{\isacharunderscore}{\kern0pt}def\isanewline
\ \ \ \ \ \isacommand{apply}\isamarkupfalse%
\ {\isacharparenleft}{\kern0pt}simp\ del{\isacharcolon}{\kern0pt}FOL{\isacharunderscore}{\kern0pt}sats{\isacharunderscore}{\kern0pt}iff\ pair{\isacharunderscore}{\kern0pt}abs\ add{\isacharcolon}{\kern0pt}\ fm{\isacharunderscore}{\kern0pt}defs\ nat{\isacharunderscore}{\kern0pt}simp{\isacharunderscore}{\kern0pt}union{\isacharparenright}{\kern0pt}\isanewline
\ \ \ \ \isacommand{using}\isamarkupfalse%
\ piForces\isanewline
\ \ \ \ \isacommand{by}\isamarkupfalse%
\ auto\isanewline
\ \ \isacommand{then}\isamarkupfalse%
\ \isacommand{have}\isamarkupfalse%
\ ForcesH{\isadigit{2}}{\isacharcolon}{\kern0pt}\ {\isachardoublequoteopen}{\isacharparenleft}{\kern0pt}{\isasymforall}H{\isachardot}{\kern0pt}\ M{\isacharunderscore}{\kern0pt}generic{\isacharparenleft}{\kern0pt}H{\isacharparenright}{\kern0pt}\ {\isasymand}\ q{\isasymin}H\ \ {\isasymlongrightarrow}\ \ SymExt{\isacharparenleft}{\kern0pt}H{\isacharparenright}{\kern0pt}{\isacharcomma}{\kern0pt}\ map{\isacharparenleft}{\kern0pt}val{\isacharparenleft}{\kern0pt}H{\isacharparenright}{\kern0pt}{\isacharcomma}{\kern0pt}{\isacharbrackleft}{\kern0pt}F{\isacharprime}{\kern0pt}{\isacharcomma}{\kern0pt}\ check{\isacharparenleft}{\kern0pt}i{\isacharparenright}{\kern0pt}{\isacharcomma}{\kern0pt}\ binmap{\isacharunderscore}{\kern0pt}row{\isacharprime}{\kern0pt}{\isacharparenleft}{\kern0pt}f{\isacharbackquote}{\kern0pt}n{\isacharparenright}{\kern0pt}{\isacharbrackright}{\kern0pt}{\isacharparenright}{\kern0pt}\ {\isasymTurnstile}\ {\isasymphi}{\isacharparenright}{\kern0pt}{\isachardoublequoteclose}\isanewline
\ \ \ \ \isacommand{apply}\isamarkupfalse%
{\isacharparenleft}{\kern0pt}rule{\isacharunderscore}{\kern0pt}tac\ iffD{\isadigit{1}}{\isacharparenright}{\kern0pt}\isanewline
\ \ \ \ \ \isacommand{apply}\isamarkupfalse%
{\isacharparenleft}{\kern0pt}rule\ definition{\isacharunderscore}{\kern0pt}of{\isacharunderscore}{\kern0pt}forcing{\isacharunderscore}{\kern0pt}HS{\isacharparenright}{\kern0pt}\isanewline
\ \ \ \ \isacommand{using}\isamarkupfalse%
\ Fn{\isacharunderscore}{\kern0pt}perm{\isacharprime}{\kern0pt}{\isacharunderscore}{\kern0pt}is{\isacharunderscore}{\kern0pt}P{\isacharunderscore}{\kern0pt}auto\ {\isasympi}{\isacharunderscore}{\kern0pt}def\ fin\ qH\ {\isasymphi}{\isacharunderscore}{\kern0pt}def\ listin{\isacharprime}{\kern0pt}{\isacharprime}{\kern0pt}{\isacharprime}{\kern0pt}\isanewline
\ \ \ \ \ \ \ \ \ \isacommand{apply}\isamarkupfalse%
\ auto{\isacharbrackleft}{\kern0pt}{\isadigit{4}}{\isacharbrackright}{\kern0pt}\isanewline
\ \ \ \ \ \isacommand{apply}\isamarkupfalse%
{\isacharparenleft}{\kern0pt}simp\ add{\isacharcolon}{\kern0pt}{\isasymphi}{\isacharunderscore}{\kern0pt}def{\isacharparenright}{\kern0pt}\isanewline
\ \ \ \ \ \isacommand{apply}\isamarkupfalse%
\ {\isacharparenleft}{\kern0pt}simp\ del{\isacharcolon}{\kern0pt}FOL{\isacharunderscore}{\kern0pt}sats{\isacharunderscore}{\kern0pt}iff\ pair{\isacharunderscore}{\kern0pt}abs\ add{\isacharcolon}{\kern0pt}\ fm{\isacharunderscore}{\kern0pt}defs\ nat{\isacharunderscore}{\kern0pt}simp{\isacharunderscore}{\kern0pt}union{\isacharparenright}{\kern0pt}\isanewline
\ \ \ \ \isacommand{by}\isamarkupfalse%
\ auto\isanewline
\isanewline
\ \ \isacommand{obtain}\isamarkupfalse%
\ H\ \isakeyword{where}\ HH{\isacharcolon}{\kern0pt}\ {\isachardoublequoteopen}M{\isacharunderscore}{\kern0pt}generic{\isacharparenleft}{\kern0pt}H{\isacharparenright}{\kern0pt}{\isachardoublequoteclose}\ {\isachardoublequoteopen}q\ {\isasymin}\ H{\isachardoublequoteclose}\ \isanewline
\ \ \ \ \isacommand{using}\isamarkupfalse%
\ generic{\isacharunderscore}{\kern0pt}filter{\isacharunderscore}{\kern0pt}existence\ qH\ \isanewline
\ \ \ \ \isacommand{by}\isamarkupfalse%
\ auto\isanewline
\ \ \isanewline
\ \ \isacommand{define}\isamarkupfalse%
\ F\ \isakeyword{where}\ {\isachardoublequoteopen}F\ {\isasymequiv}\ val{\isacharparenleft}{\kern0pt}H{\isacharcomma}{\kern0pt}\ F{\isacharprime}{\kern0pt}{\isacharparenright}{\kern0pt}{\isachardoublequoteclose}\ \isanewline
\isanewline
\ \ \isacommand{have}\isamarkupfalse%
\ {\isachardoublequoteopen}SymExt{\isacharparenleft}{\kern0pt}H{\isacharparenright}{\kern0pt}{\isacharcomma}{\kern0pt}\ map{\isacharparenleft}{\kern0pt}val{\isacharparenleft}{\kern0pt}H{\isacharparenright}{\kern0pt}{\isacharcomma}{\kern0pt}\ {\isacharbrackleft}{\kern0pt}F{\isacharprime}{\kern0pt}{\isacharcomma}{\kern0pt}\ check{\isacharparenleft}{\kern0pt}i{\isacharparenright}{\kern0pt}{\isacharcomma}{\kern0pt}\ binmap{\isacharunderscore}{\kern0pt}row{\isacharprime}{\kern0pt}{\isacharparenleft}{\kern0pt}n{\isacharparenright}{\kern0pt}{\isacharbrackright}{\kern0pt}{\isacharparenright}{\kern0pt}\ {\isasymTurnstile}\ {\isasymphi}{\isachardoublequoteclose}\ \isanewline
\ \ \ \ \isacommand{using}\isamarkupfalse%
\ ForcesH{\isadigit{1}}\ HH\isanewline
\ \ \ \ \isacommand{by}\isamarkupfalse%
\ auto\isanewline
\ \ \isacommand{then}\isamarkupfalse%
\ \isacommand{have}\isamarkupfalse%
\ {\isachardoublequoteopen}val{\isacharparenleft}{\kern0pt}H{\isacharcomma}{\kern0pt}\ F{\isacharprime}{\kern0pt}{\isacharparenright}{\kern0pt}{\isacharbackquote}{\kern0pt}val{\isacharparenleft}{\kern0pt}H{\isacharcomma}{\kern0pt}\ check{\isacharparenleft}{\kern0pt}i{\isacharparenright}{\kern0pt}{\isacharparenright}{\kern0pt}\ {\isacharequal}{\kern0pt}\ val{\isacharparenleft}{\kern0pt}H{\isacharcomma}{\kern0pt}\ binmap{\isacharunderscore}{\kern0pt}row{\isacharprime}{\kern0pt}{\isacharparenleft}{\kern0pt}n{\isacharparenright}{\kern0pt}{\isacharparenright}{\kern0pt}{\isachardoublequoteclose}\ {\isacharparenleft}{\kern0pt}\isakeyword{is}\ {\isacharquery}{\kern0pt}A{\isacharparenright}{\kern0pt}\isanewline
\ \ \ \ \isacommand{apply}\isamarkupfalse%
{\isacharparenleft}{\kern0pt}rule{\isacharunderscore}{\kern0pt}tac\ iffD{\isadigit{1}}{\isacharcomma}{\kern0pt}\ rule{\isacharunderscore}{\kern0pt}tac\ sats{\isacharunderscore}{\kern0pt}{\isasymphi}{\isacharunderscore}{\kern0pt}iff{\isacharparenright}{\kern0pt}\isanewline
\ \ \ \ \isacommand{using}\isamarkupfalse%
\ HH\ \isanewline
\ \ \ \ \ \ \isacommand{apply}\isamarkupfalse%
\ simp\isanewline
\ \ \ \ \ \isacommand{apply}\isamarkupfalse%
\ auto{\isacharbrackleft}{\kern0pt}{\isadigit{1}}{\isacharbrackright}{\kern0pt}\isanewline
\ \ \ \ \isacommand{using}\isamarkupfalse%
\ SymExt{\isacharunderscore}{\kern0pt}def\ listin{\isacharprime}{\kern0pt}{\isacharprime}{\kern0pt}\isanewline
\ \ \ \ \isacommand{by}\isamarkupfalse%
\ auto\isanewline
\ \ \isacommand{then}\isamarkupfalse%
\ \isacommand{have}\isamarkupfalse%
\ Fieq{\isadigit{1}}{\isacharcolon}{\kern0pt}\ {\isachardoublequoteopen}F{\isacharbackquote}{\kern0pt}i\ {\isacharequal}{\kern0pt}\ binmap{\isacharunderscore}{\kern0pt}row{\isacharparenleft}{\kern0pt}H{\isacharcomma}{\kern0pt}\ n{\isacharparenright}{\kern0pt}{\isachardoublequoteclose}\ \isanewline
\ \ \ \ \isacommand{apply}\isamarkupfalse%
{\isacharparenleft}{\kern0pt}rule{\isacharunderscore}{\kern0pt}tac\ P{\isacharequal}{\kern0pt}{\isachardoublequoteopen}{\isacharquery}{\kern0pt}A{\isachardoublequoteclose}\ \isakeyword{in}\ iffD{\isadigit{1}}{\isacharparenright}{\kern0pt}\isanewline
\ \ \ \ \ \isacommand{apply}\isamarkupfalse%
{\isacharparenleft}{\kern0pt}subst\ F{\isacharunderscore}{\kern0pt}def{\isacharcomma}{\kern0pt}\ subst\ valcheck{\isacharparenright}{\kern0pt}\isanewline
\ \ \ \ \isacommand{using}\isamarkupfalse%
\ HH\ generic{\isacharunderscore}{\kern0pt}filter{\isacharunderscore}{\kern0pt}contains{\isacharunderscore}{\kern0pt}max\ zero{\isacharunderscore}{\kern0pt}in{\isacharunderscore}{\kern0pt}Fn\isanewline
\ \ \ \ \ \ \ \isacommand{apply}\isamarkupfalse%
\ auto{\isacharbrackleft}{\kern0pt}{\isadigit{2}}{\isacharbrackright}{\kern0pt}\isanewline
\ \ \ \ \ \isacommand{apply}\isamarkupfalse%
{\isacharparenleft}{\kern0pt}subst\ binmap{\isacharunderscore}{\kern0pt}row{\isacharunderscore}{\kern0pt}eq{\isacharparenright}{\kern0pt}\isanewline
\ \ \ \ \isacommand{using}\isamarkupfalse%
\ nH\ HH\ \isanewline
\ \ \ \ \isacommand{by}\isamarkupfalse%
\ auto\isanewline
\isanewline
\ \ \isacommand{have}\isamarkupfalse%
\ {\isachardoublequoteopen}SymExt{\isacharparenleft}{\kern0pt}H{\isacharparenright}{\kern0pt}{\isacharcomma}{\kern0pt}\ map{\isacharparenleft}{\kern0pt}val{\isacharparenleft}{\kern0pt}H{\isacharparenright}{\kern0pt}{\isacharcomma}{\kern0pt}\ {\isacharbrackleft}{\kern0pt}F{\isacharprime}{\kern0pt}{\isacharcomma}{\kern0pt}\ check{\isacharparenleft}{\kern0pt}i{\isacharparenright}{\kern0pt}{\isacharcomma}{\kern0pt}\ binmap{\isacharunderscore}{\kern0pt}row{\isacharprime}{\kern0pt}{\isacharparenleft}{\kern0pt}f{\isacharbackquote}{\kern0pt}n{\isacharparenright}{\kern0pt}{\isacharbrackright}{\kern0pt}{\isacharparenright}{\kern0pt}\ {\isasymTurnstile}\ {\isasymphi}{\isachardoublequoteclose}\isanewline
\ \ \ \ \isacommand{using}\isamarkupfalse%
\ ForcesH{\isadigit{2}}\ HH\ \isanewline
\ \ \ \ \isacommand{by}\isamarkupfalse%
\ auto\isanewline
\ \ \isacommand{then}\isamarkupfalse%
\ \isacommand{have}\isamarkupfalse%
\ {\isachardoublequoteopen}val{\isacharparenleft}{\kern0pt}H{\isacharcomma}{\kern0pt}\ F{\isacharprime}{\kern0pt}{\isacharparenright}{\kern0pt}{\isacharbackquote}{\kern0pt}val{\isacharparenleft}{\kern0pt}H{\isacharcomma}{\kern0pt}\ check{\isacharparenleft}{\kern0pt}i{\isacharparenright}{\kern0pt}{\isacharparenright}{\kern0pt}\ {\isacharequal}{\kern0pt}\ val{\isacharparenleft}{\kern0pt}H{\isacharcomma}{\kern0pt}\ binmap{\isacharunderscore}{\kern0pt}row{\isacharprime}{\kern0pt}{\isacharparenleft}{\kern0pt}f{\isacharbackquote}{\kern0pt}n{\isacharparenright}{\kern0pt}{\isacharparenright}{\kern0pt}{\isachardoublequoteclose}\ {\isacharparenleft}{\kern0pt}\isakeyword{is}\ {\isacharquery}{\kern0pt}A{\isacharparenright}{\kern0pt}\isanewline
\ \ \ \ \isacommand{apply}\isamarkupfalse%
{\isacharparenleft}{\kern0pt}rule{\isacharunderscore}{\kern0pt}tac\ iffD{\isadigit{1}}{\isacharcomma}{\kern0pt}\ rule{\isacharunderscore}{\kern0pt}tac\ sats{\isacharunderscore}{\kern0pt}{\isasymphi}{\isacharunderscore}{\kern0pt}iff{\isacharparenright}{\kern0pt}\isanewline
\ \ \ \ \isacommand{using}\isamarkupfalse%
\ HH\ \isanewline
\ \ \ \ \ \ \isacommand{apply}\isamarkupfalse%
\ simp\isanewline
\ \ \ \ \ \isacommand{apply}\isamarkupfalse%
\ auto{\isacharbrackleft}{\kern0pt}{\isadigit{1}}{\isacharbrackright}{\kern0pt}\isanewline
\ \ \ \ \isacommand{using}\isamarkupfalse%
\ SymExt{\isacharunderscore}{\kern0pt}def\ listin{\isacharprime}{\kern0pt}{\isacharprime}{\kern0pt}{\isacharprime}{\kern0pt}\isanewline
\ \ \ \ \isacommand{by}\isamarkupfalse%
\ auto\isanewline
\ \ \isacommand{then}\isamarkupfalse%
\ \isacommand{have}\isamarkupfalse%
\ Fieq{\isadigit{2}}{\isacharcolon}{\kern0pt}\ {\isachardoublequoteopen}F{\isacharbackquote}{\kern0pt}i\ {\isacharequal}{\kern0pt}\ binmap{\isacharunderscore}{\kern0pt}row{\isacharparenleft}{\kern0pt}H{\isacharcomma}{\kern0pt}\ f{\isacharbackquote}{\kern0pt}n{\isacharparenright}{\kern0pt}{\isachardoublequoteclose}\ \isanewline
\ \ \ \ \isacommand{apply}\isamarkupfalse%
{\isacharparenleft}{\kern0pt}rule{\isacharunderscore}{\kern0pt}tac\ P{\isacharequal}{\kern0pt}{\isachardoublequoteopen}{\isacharquery}{\kern0pt}A{\isachardoublequoteclose}\ \isakeyword{in}\ iffD{\isadigit{1}}{\isacharparenright}{\kern0pt}\isanewline
\ \ \ \ \ \isacommand{apply}\isamarkupfalse%
{\isacharparenleft}{\kern0pt}subst\ F{\isacharunderscore}{\kern0pt}def{\isacharcomma}{\kern0pt}\ subst\ valcheck{\isacharparenright}{\kern0pt}\isanewline
\ \ \ \ \isacommand{using}\isamarkupfalse%
\ HH\ generic{\isacharunderscore}{\kern0pt}filter{\isacharunderscore}{\kern0pt}contains{\isacharunderscore}{\kern0pt}max\ zero{\isacharunderscore}{\kern0pt}in{\isacharunderscore}{\kern0pt}Fn\isanewline
\ \ \ \ \ \ \ \isacommand{apply}\isamarkupfalse%
\ auto{\isacharbrackleft}{\kern0pt}{\isadigit{2}}{\isacharbrackright}{\kern0pt}\isanewline
\ \ \ \ \ \isacommand{apply}\isamarkupfalse%
{\isacharparenleft}{\kern0pt}subst\ binmap{\isacharunderscore}{\kern0pt}row{\isacharunderscore}{\kern0pt}eq{\isacharparenright}{\kern0pt}\isanewline
\ \ \ \ \ \ \ \isacommand{apply}\isamarkupfalse%
{\isacharparenleft}{\kern0pt}rule\ function{\isacharunderscore}{\kern0pt}value{\isacharunderscore}{\kern0pt}in{\isacharparenright}{\kern0pt}\isanewline
\ \ \ \ \isacommand{using}\isamarkupfalse%
\ fin\ nat{\isacharunderscore}{\kern0pt}perms{\isacharunderscore}{\kern0pt}def\ nH\ HH\ bij{\isacharunderscore}{\kern0pt}def\ inj{\isacharunderscore}{\kern0pt}def\isanewline
\ \ \ \ \isacommand{by}\isamarkupfalse%
\ auto\isanewline
\ \ \ \ \isanewline
\ \ \isacommand{have}\isamarkupfalse%
\ roweq\ {\isacharcolon}{\kern0pt}\ {\isachardoublequoteopen}binmap{\isacharunderscore}{\kern0pt}row{\isacharparenleft}{\kern0pt}H{\isacharcomma}{\kern0pt}\ n{\isacharparenright}{\kern0pt}\ {\isacharequal}{\kern0pt}\ binmap{\isacharunderscore}{\kern0pt}row{\isacharparenleft}{\kern0pt}H{\isacharcomma}{\kern0pt}\ f{\isacharbackquote}{\kern0pt}n{\isacharparenright}{\kern0pt}{\isachardoublequoteclose}\ \isacommand{using}\isamarkupfalse%
\ Fieq{\isadigit{1}}\ Fieq{\isadigit{2}}\ \isacommand{by}\isamarkupfalse%
\ auto\isanewline
\isanewline
\ \ \isacommand{have}\isamarkupfalse%
\ rowneq{\isacharcolon}{\kern0pt}\ {\isachardoublequoteopen}binmap{\isacharunderscore}{\kern0pt}row{\isacharparenleft}{\kern0pt}H{\isacharcomma}{\kern0pt}\ n{\isacharparenright}{\kern0pt}\ {\isasymnoteq}\ binmap{\isacharunderscore}{\kern0pt}row{\isacharparenleft}{\kern0pt}H{\isacharcomma}{\kern0pt}\ f{\isacharbackquote}{\kern0pt}n{\isacharparenright}{\kern0pt}{\isachardoublequoteclose}\ \isanewline
\ \ \ \ \isacommand{apply}\isamarkupfalse%
{\isacharparenleft}{\kern0pt}rule\ binmap{\isacharunderscore}{\kern0pt}row{\isacharunderscore}{\kern0pt}distinct{\isacharparenright}{\kern0pt}\isanewline
\ \ \ \ \isacommand{using}\isamarkupfalse%
\ HH\ nH\ \isanewline
\ \ \ \ \ \ \ \isacommand{apply}\isamarkupfalse%
\ auto{\isacharbrackleft}{\kern0pt}{\isadigit{2}}{\isacharbrackright}{\kern0pt}\isanewline
\ \ \ \ \ \isacommand{apply}\isamarkupfalse%
{\isacharparenleft}{\kern0pt}rule\ function{\isacharunderscore}{\kern0pt}value{\isacharunderscore}{\kern0pt}in{\isacharparenright}{\kern0pt}\isanewline
\ \ \ \ \isacommand{using}\isamarkupfalse%
\ HH\ nH\ fin\ nat{\isacharunderscore}{\kern0pt}perms{\isacharunderscore}{\kern0pt}def\ bij{\isacharunderscore}{\kern0pt}def\ inj{\isacharunderscore}{\kern0pt}def\ \isanewline
\ \ \ \ \ \ \isacommand{apply}\isamarkupfalse%
\ auto{\isacharbrackleft}{\kern0pt}{\isadigit{2}}{\isacharbrackright}{\kern0pt}\isanewline
\ \ \ \ \isacommand{using}\isamarkupfalse%
\ fnneq\isanewline
\ \ \ \ \isacommand{by}\isamarkupfalse%
\ auto\isanewline
\isanewline
\ \ \isacommand{show}\isamarkupfalse%
\ False\isanewline
\ \ \ \ \isacommand{using}\isamarkupfalse%
\ roweq\ rowneq\isanewline
\ \ \ \ \isacommand{by}\isamarkupfalse%
\ auto\isanewline
\isacommand{qed}\isamarkupfalse%
%
\endisatagproof
{\isafoldproof}%
%
\isadelimproof
\isanewline
%
\endisadelimproof
\isanewline
\isacommand{lemma}\isamarkupfalse%
\ no{\isacharunderscore}{\kern0pt}wellorder\ {\isacharcolon}{\kern0pt}\ \isanewline
\ \ \isakeyword{fixes}\ r\ G\isanewline
\ \ \isakeyword{assumes}\ {\isachardoublequoteopen}M{\isacharunderscore}{\kern0pt}generic{\isacharparenleft}{\kern0pt}G{\isacharparenright}{\kern0pt}{\isachardoublequoteclose}\ {\isachardoublequoteopen}wellordered{\isacharparenleft}{\kern0pt}{\isacharhash}{\kern0pt}{\isacharhash}{\kern0pt}SymExt{\isacharparenleft}{\kern0pt}G{\isacharparenright}{\kern0pt}{\isacharcomma}{\kern0pt}\ binmap{\isacharparenleft}{\kern0pt}G{\isacharparenright}{\kern0pt}{\isacharcomma}{\kern0pt}\ r{\isacharparenright}{\kern0pt}{\isachardoublequoteclose}\ {\isachardoublequoteopen}r\ {\isasymin}\ SymExt{\isacharparenleft}{\kern0pt}G{\isacharparenright}{\kern0pt}{\isachardoublequoteclose}\ \isanewline
\ \ \isakeyword{shows}\ False\isanewline
%
\isadelimproof
%
\endisadelimproof
%
\isatagproof
\isacommand{proof}\isamarkupfalse%
\ {\isacharminus}{\kern0pt}\ \isanewline
\isanewline
\ \ \isacommand{interpret}\isamarkupfalse%
\ M{\isacharunderscore}{\kern0pt}symmetric{\isacharunderscore}{\kern0pt}system{\isacharunderscore}{\kern0pt}G{\isacharunderscore}{\kern0pt}generic\ \ {\isachardoublequoteopen}Fn{\isachardoublequoteclose}\ {\isachardoublequoteopen}Fn{\isacharunderscore}{\kern0pt}leq{\isachardoublequoteclose}\ {\isachardoublequoteopen}{\isadigit{0}}{\isachardoublequoteclose}\ {\isachardoublequoteopen}M{\isachardoublequoteclose}\ {\isachardoublequoteopen}enum{\isachardoublequoteclose}\ {\isachardoublequoteopen}Fn{\isacharunderscore}{\kern0pt}perms{\isachardoublequoteclose}\ {\isachardoublequoteopen}Fn{\isacharunderscore}{\kern0pt}perms{\isacharunderscore}{\kern0pt}filter{\isachardoublequoteclose}\ G\ \isanewline
\ \ \ \ \isacommand{unfolding}\isamarkupfalse%
\ M{\isacharunderscore}{\kern0pt}symmetric{\isacharunderscore}{\kern0pt}system{\isacharunderscore}{\kern0pt}G{\isacharunderscore}{\kern0pt}generic{\isacharunderscore}{\kern0pt}def\isanewline
\ \ \ \ \isacommand{apply}\isamarkupfalse%
{\isacharparenleft}{\kern0pt}rule\ conjI{\isacharparenright}{\kern0pt}\isanewline
\ \ \ \ \isacommand{using}\isamarkupfalse%
\ M{\isacharunderscore}{\kern0pt}symmetric{\isacharunderscore}{\kern0pt}system{\isacharunderscore}{\kern0pt}axioms\ \isanewline
\ \ \ \ \ \isacommand{apply}\isamarkupfalse%
\ force\isanewline
\ \ \ \ \isacommand{unfolding}\isamarkupfalse%
\ G{\isacharunderscore}{\kern0pt}generic{\isacharunderscore}{\kern0pt}def\isanewline
\ \ \ \ \isacommand{apply}\isamarkupfalse%
{\isacharparenleft}{\kern0pt}rule\ conjI{\isacharparenright}{\kern0pt}\isanewline
\ \ \ \ \isacommand{using}\isamarkupfalse%
\ forcing{\isacharunderscore}{\kern0pt}data{\isacharunderscore}{\kern0pt}axioms\isanewline
\ \ \ \ \ \isacommand{apply}\isamarkupfalse%
\ force\isanewline
\ \ \ \ \isacommand{unfolding}\isamarkupfalse%
\ G{\isacharunderscore}{\kern0pt}generic{\isacharunderscore}{\kern0pt}axioms{\isacharunderscore}{\kern0pt}def\isanewline
\ \ \ \ \isacommand{using}\isamarkupfalse%
\ assms\isanewline
\ \ \ \ \isacommand{by}\isamarkupfalse%
\ auto\isanewline
\isanewline
\isanewline
\ \ \isacommand{define}\isamarkupfalse%
\ f\ \isakeyword{where}\ {\isachardoublequoteopen}f\ {\isasymequiv}\ {\isacharbraceleft}{\kern0pt}\ {\isacharless}{\kern0pt}n{\isacharcomma}{\kern0pt}\ binmap{\isacharunderscore}{\kern0pt}row{\isacharparenleft}{\kern0pt}G{\isacharcomma}{\kern0pt}\ n{\isacharparenright}{\kern0pt}{\isachargreater}{\kern0pt}{\isachardot}{\kern0pt}\ n\ {\isasymin}\ nat\ {\isacharbraceright}{\kern0pt}{\isachardoublequoteclose}\ \isanewline
\ \ \isacommand{have}\isamarkupfalse%
\ fpi{\isacharcolon}{\kern0pt}\ {\isachardoublequoteopen}f\ {\isasymin}\ nat\ {\isasymrightarrow}\ binmap{\isacharparenleft}{\kern0pt}G{\isacharparenright}{\kern0pt}{\isachardoublequoteclose}\ \isanewline
\ \ \ \ \isacommand{apply}\isamarkupfalse%
{\isacharparenleft}{\kern0pt}rule\ Pi{\isacharunderscore}{\kern0pt}memberI{\isacharparenright}{\kern0pt}\isanewline
\ \ \ \ \isacommand{using}\isamarkupfalse%
\ f{\isacharunderscore}{\kern0pt}def\ relation{\isacharunderscore}{\kern0pt}def\ function{\isacharunderscore}{\kern0pt}def\ binmap{\isacharunderscore}{\kern0pt}def\isanewline
\ \ \ \ \isacommand{by}\isamarkupfalse%
\ auto\isanewline
\ \ \isacommand{have}\isamarkupfalse%
\ {\isachardoublequoteopen}{\isasymAnd}m\ n{\isachardot}{\kern0pt}\ m\ {\isasymin}\ nat\ {\isasymLongrightarrow}\ n\ {\isasymin}\ nat\ {\isasymLongrightarrow}\ m\ {\isasymnoteq}\ n\ {\isasymLongrightarrow}\ f{\isacharbackquote}{\kern0pt}m\ {\isasymnoteq}\ f{\isacharbackquote}{\kern0pt}n{\isachardoublequoteclose}\ \isanewline
\ \ \isacommand{proof}\isamarkupfalse%
{\isacharminus}{\kern0pt}\ \isanewline
\ \ \ \ \isacommand{fix}\isamarkupfalse%
\ m\ n\ \isanewline
\ \ \ \ \isacommand{assume}\isamarkupfalse%
\ assms{\isadigit{1}}{\isacharcolon}{\kern0pt}\ {\isachardoublequoteopen}m\ {\isasymin}\ nat{\isachardoublequoteclose}\ {\isachardoublequoteopen}n\ {\isasymin}\ nat{\isachardoublequoteclose}\ {\isachardoublequoteopen}m\ {\isasymnoteq}\ n{\isachardoublequoteclose}\ \isanewline
\ \ \ \ \isacommand{have}\isamarkupfalse%
\ fm{\isacharcolon}{\kern0pt}\ {\isachardoublequoteopen}f{\isacharbackquote}{\kern0pt}m\ {\isacharequal}{\kern0pt}\ binmap{\isacharunderscore}{\kern0pt}row{\isacharparenleft}{\kern0pt}G{\isacharcomma}{\kern0pt}\ m{\isacharparenright}{\kern0pt}{\isachardoublequoteclose}\ \isanewline
\ \ \ \ \ \ \isacommand{apply}\isamarkupfalse%
{\isacharparenleft}{\kern0pt}rule\ function{\isacharunderscore}{\kern0pt}apply{\isacharunderscore}{\kern0pt}equality{\isacharparenright}{\kern0pt}\isanewline
\ \ \ \ \ \ \isacommand{using}\isamarkupfalse%
\ f{\isacharunderscore}{\kern0pt}def\ assms{\isadigit{1}}\ fpi\ Pi{\isacharunderscore}{\kern0pt}def\isanewline
\ \ \ \ \ \ \isacommand{by}\isamarkupfalse%
\ auto\isanewline
\isanewline
\ \ \ \ \isacommand{have}\isamarkupfalse%
\ fn{\isacharcolon}{\kern0pt}\ {\isachardoublequoteopen}f{\isacharbackquote}{\kern0pt}n\ {\isacharequal}{\kern0pt}\ binmap{\isacharunderscore}{\kern0pt}row{\isacharparenleft}{\kern0pt}G{\isacharcomma}{\kern0pt}\ n{\isacharparenright}{\kern0pt}{\isachardoublequoteclose}\ \isanewline
\ \ \ \ \ \ \isacommand{apply}\isamarkupfalse%
{\isacharparenleft}{\kern0pt}rule\ function{\isacharunderscore}{\kern0pt}apply{\isacharunderscore}{\kern0pt}equality{\isacharparenright}{\kern0pt}\isanewline
\ \ \ \ \ \ \isacommand{using}\isamarkupfalse%
\ f{\isacharunderscore}{\kern0pt}def\ assms{\isadigit{1}}\ fpi\ Pi{\isacharunderscore}{\kern0pt}def\isanewline
\ \ \ \ \ \ \isacommand{by}\isamarkupfalse%
\ auto\isanewline
\isanewline
\ \ \ \ \isacommand{have}\isamarkupfalse%
\ {\isachardoublequoteopen}binmap{\isacharunderscore}{\kern0pt}row{\isacharparenleft}{\kern0pt}G{\isacharcomma}{\kern0pt}\ m{\isacharparenright}{\kern0pt}\ {\isasymnoteq}\ binmap{\isacharunderscore}{\kern0pt}row{\isacharparenleft}{\kern0pt}G{\isacharcomma}{\kern0pt}\ n{\isacharparenright}{\kern0pt}{\isachardoublequoteclose}\ \isanewline
\ \ \ \ \ \ \isacommand{apply}\isamarkupfalse%
{\isacharparenleft}{\kern0pt}rule\ binmap{\isacharunderscore}{\kern0pt}row{\isacharunderscore}{\kern0pt}distinct{\isacharparenright}{\kern0pt}\isanewline
\ \ \ \ \ \ \isacommand{using}\isamarkupfalse%
\ assms\ assms{\isadigit{1}}\isanewline
\ \ \ \ \ \ \isacommand{by}\isamarkupfalse%
\ auto\isanewline
\ \ \ \ \isacommand{then}\isamarkupfalse%
\ \isacommand{show}\isamarkupfalse%
\ {\isachardoublequoteopen}f{\isacharbackquote}{\kern0pt}m\ {\isasymnoteq}\ f{\isacharbackquote}{\kern0pt}n{\isachardoublequoteclose}\ \isanewline
\ \ \ \ \ \ \isacommand{using}\isamarkupfalse%
\ fm\ fn\ \isanewline
\ \ \ \ \ \ \isacommand{by}\isamarkupfalse%
\ auto\isanewline
\ \ \isacommand{qed}\isamarkupfalse%
\isanewline
\ \ \isacommand{then}\isamarkupfalse%
\ \isacommand{have}\isamarkupfalse%
\ {\isachardoublequoteopen}f\ {\isasymin}\ inj{\isacharparenleft}{\kern0pt}nat{\isacharcomma}{\kern0pt}\ binmap{\isacharparenleft}{\kern0pt}G{\isacharparenright}{\kern0pt}{\isacharparenright}{\kern0pt}{\isachardoublequoteclose}\ \isanewline
\ \ \ \ \isacommand{unfolding}\isamarkupfalse%
\ inj{\isacharunderscore}{\kern0pt}def\ \isanewline
\ \ \ \ \isacommand{using}\isamarkupfalse%
\ fpi\isanewline
\ \ \ \ \isacommand{by}\isamarkupfalse%
\ auto\isanewline
\ \ \isacommand{then}\isamarkupfalse%
\ \isacommand{have}\isamarkupfalse%
\ natle{\isacharcolon}{\kern0pt}\ {\isachardoublequoteopen}nat\ {\isasymlesssim}\ binmap{\isacharparenleft}{\kern0pt}G{\isacharparenright}{\kern0pt}{\isachardoublequoteclose}\ \isacommand{using}\isamarkupfalse%
\ lepoll{\isacharunderscore}{\kern0pt}def\ \isacommand{by}\isamarkupfalse%
\ auto\isanewline
\ \ \isanewline
\ \ \isacommand{have}\isamarkupfalse%
\ {\isachardoublequoteopen}{\isasymexists}g{\isasymin}SymExt{\isacharparenleft}{\kern0pt}G{\isacharparenright}{\kern0pt}{\isachardot}{\kern0pt}\ g\ {\isasymin}\ inj{\isacharparenleft}{\kern0pt}nat{\isacharcomma}{\kern0pt}\ binmap{\isacharparenleft}{\kern0pt}G{\isacharparenright}{\kern0pt}{\isacharparenright}{\kern0pt}{\isachardoublequoteclose}\isanewline
\ \ \ \ \isacommand{apply}\isamarkupfalse%
{\isacharparenleft}{\kern0pt}rule\ M{\isacharunderscore}{\kern0pt}ZF{\isacharunderscore}{\kern0pt}trans{\isachardot}{\kern0pt}wellorder{\isacharunderscore}{\kern0pt}induces{\isacharunderscore}{\kern0pt}injection{\isacharparenright}{\kern0pt}\isanewline
\ \ \ \ \isacommand{using}\isamarkupfalse%
\ SymExt{\isacharunderscore}{\kern0pt}M{\isacharunderscore}{\kern0pt}ZF{\isacharunderscore}{\kern0pt}trans\ natle\ assms\ \isanewline
\ \ \ \ \ \ \ \ \isacommand{apply}\isamarkupfalse%
\ auto{\isacharbrackleft}{\kern0pt}{\isadigit{4}}{\isacharbrackright}{\kern0pt}\isanewline
\ \ \ \ \isacommand{apply}\isamarkupfalse%
{\isacharparenleft}{\kern0pt}subst\ binmap{\isacharunderscore}{\kern0pt}eq{\isacharcomma}{\kern0pt}\ simp\ add{\isacharcolon}{\kern0pt}assms{\isacharparenright}{\kern0pt}\isanewline
\ \ \ \ \isacommand{unfolding}\isamarkupfalse%
\ SymExt{\isacharunderscore}{\kern0pt}def\isanewline
\ \ \ \ \isacommand{using}\isamarkupfalse%
\ binmap{\isacharprime}{\kern0pt}{\isacharunderscore}{\kern0pt}HS\isanewline
\ \ \ \ \isacommand{by}\isamarkupfalse%
\ auto\isanewline
\ \ \isacommand{then}\isamarkupfalse%
\ \isacommand{obtain}\isamarkupfalse%
\ g\ \isakeyword{where}\ gH\ {\isacharcolon}{\kern0pt}\ {\isachardoublequoteopen}g\ {\isasymin}\ SymExt{\isacharparenleft}{\kern0pt}G{\isacharparenright}{\kern0pt}{\isachardoublequoteclose}\ {\isachardoublequoteopen}g\ {\isasymin}\ inj{\isacharparenleft}{\kern0pt}nat{\isacharcomma}{\kern0pt}\ binmap{\isacharparenleft}{\kern0pt}G{\isacharparenright}{\kern0pt}{\isacharparenright}{\kern0pt}{\isachardoublequoteclose}\ \isacommand{by}\isamarkupfalse%
\ auto\isanewline
\isanewline
\ \ \isacommand{then}\isamarkupfalse%
\ \isacommand{obtain}\isamarkupfalse%
\ g{\isacharprime}{\kern0pt}\ \isakeyword{where}\ g{\isacharprime}{\kern0pt}H{\isacharcolon}{\kern0pt}\ {\isachardoublequoteopen}val{\isacharparenleft}{\kern0pt}G{\isacharcomma}{\kern0pt}\ g{\isacharprime}{\kern0pt}{\isacharparenright}{\kern0pt}\ {\isacharequal}{\kern0pt}\ g{\isachardoublequoteclose}\ {\isachardoublequoteopen}g{\isacharprime}{\kern0pt}\ {\isasymin}\ HS{\isachardoublequoteclose}\ \isacommand{using}\isamarkupfalse%
\ SymExt{\isacharunderscore}{\kern0pt}def\ \isacommand{by}\isamarkupfalse%
\ auto\isanewline
\ \ \isacommand{have}\isamarkupfalse%
\ checknatH{\isacharcolon}{\kern0pt}\ {\isachardoublequoteopen}val{\isacharparenleft}{\kern0pt}G{\isacharcomma}{\kern0pt}\ check{\isacharparenleft}{\kern0pt}nat{\isacharparenright}{\kern0pt}{\isacharparenright}{\kern0pt}\ {\isacharequal}{\kern0pt}\ nat{\isachardoublequoteclose}\ \isanewline
\ \ \ \ \isacommand{apply}\isamarkupfalse%
{\isacharparenleft}{\kern0pt}rule\ valcheck{\isacharparenright}{\kern0pt}\isanewline
\ \ \ \ \isacommand{using}\isamarkupfalse%
\ generic{\isacharunderscore}{\kern0pt}filter{\isacharunderscore}{\kern0pt}contains{\isacharunderscore}{\kern0pt}max\ assms\ zero{\isacharunderscore}{\kern0pt}in{\isacharunderscore}{\kern0pt}Fn\isanewline
\ \ \ \ \isacommand{by}\isamarkupfalse%
\ auto\isanewline
\ \ \isacommand{have}\isamarkupfalse%
\ binmap{\isacharprime}{\kern0pt}H\ {\isacharcolon}{\kern0pt}\ {\isachardoublequoteopen}val{\isacharparenleft}{\kern0pt}G{\isacharcomma}{\kern0pt}\ binmap{\isacharprime}{\kern0pt}{\isacharparenright}{\kern0pt}\ {\isacharequal}{\kern0pt}\ binmap{\isacharparenleft}{\kern0pt}G{\isacharparenright}{\kern0pt}{\isachardoublequoteclose}\ \isanewline
\ \ \ \ \isacommand{using}\isamarkupfalse%
\ assms\ binmap{\isacharunderscore}{\kern0pt}eq\isanewline
\ \ \ \ \isacommand{by}\isamarkupfalse%
\ auto\isanewline
\isanewline
\ \ \isacommand{have}\isamarkupfalse%
\ mapeq\ {\isacharcolon}{\kern0pt}\ {\isachardoublequoteopen}map{\isacharparenleft}{\kern0pt}val{\isacharparenleft}{\kern0pt}G{\isacharparenright}{\kern0pt}{\isacharcomma}{\kern0pt}\ {\isacharbrackleft}{\kern0pt}check{\isacharparenleft}{\kern0pt}nat{\isacharparenright}{\kern0pt}{\isacharcomma}{\kern0pt}\ binmap{\isacharprime}{\kern0pt}{\isacharcomma}{\kern0pt}\ g{\isacharprime}{\kern0pt}{\isacharbrackright}{\kern0pt}{\isacharparenright}{\kern0pt}\ {\isacharequal}{\kern0pt}\ {\isacharbrackleft}{\kern0pt}nat{\isacharcomma}{\kern0pt}\ binmap{\isacharparenleft}{\kern0pt}G{\isacharparenright}{\kern0pt}{\isacharcomma}{\kern0pt}\ g{\isacharbrackright}{\kern0pt}{\isachardoublequoteclose}\ \isanewline
\ \ \ \ \isacommand{using}\isamarkupfalse%
\ g{\isacharprime}{\kern0pt}H\ checknatH\ binmap{\isacharprime}{\kern0pt}H\isanewline
\ \ \ \ \isacommand{by}\isamarkupfalse%
\ auto\isanewline
\isanewline
\ \ \isacommand{have}\isamarkupfalse%
\ {\isachardoublequoteopen}SymExt{\isacharparenleft}{\kern0pt}G{\isacharparenright}{\kern0pt}{\isacharcomma}{\kern0pt}\ {\isacharbrackleft}{\kern0pt}nat{\isacharcomma}{\kern0pt}\ binmap{\isacharparenleft}{\kern0pt}G{\isacharparenright}{\kern0pt}{\isacharcomma}{\kern0pt}\ g{\isacharbrackright}{\kern0pt}\ {\isasymTurnstile}\ injection{\isacharunderscore}{\kern0pt}fm{\isacharparenleft}{\kern0pt}{\isadigit{0}}{\isacharcomma}{\kern0pt}{\isadigit{1}}{\isacharcomma}{\kern0pt}{\isadigit{2}}{\isacharparenright}{\kern0pt}{\isachardoublequoteclose}\ \isanewline
\ \ \ \ \isacommand{apply}\isamarkupfalse%
{\isacharparenleft}{\kern0pt}rule\ iffD{\isadigit{2}}{\isacharcomma}{\kern0pt}\ rule\ sats{\isacharunderscore}{\kern0pt}injection{\isacharunderscore}{\kern0pt}fm{\isacharparenright}{\kern0pt}\isanewline
\ \ \ \ \ \ \ \ \isacommand{apply}\isamarkupfalse%
\ auto{\isacharbrackleft}{\kern0pt}{\isadigit{4}}{\isacharbrackright}{\kern0pt}\isanewline
\ \ \ \ \isacommand{using}\isamarkupfalse%
\ nat{\isacharunderscore}{\kern0pt}in{\isacharunderscore}{\kern0pt}M\ M{\isacharunderscore}{\kern0pt}subset{\isacharunderscore}{\kern0pt}SymExt\ binmap{\isacharprime}{\kern0pt}{\isacharunderscore}{\kern0pt}HS\ SymExt{\isacharunderscore}{\kern0pt}def\ gH\ binmap{\isacharunderscore}{\kern0pt}eq\ assms\isanewline
\ \ \ \ \ \ \ \isacommand{apply}\isamarkupfalse%
\ auto{\isacharbrackleft}{\kern0pt}{\isadigit{3}}{\isacharbrackright}{\kern0pt}\isanewline
\ \ \ \ \isacommand{apply}\isamarkupfalse%
\ simp\isanewline
\ \ \ \ \isacommand{apply}\isamarkupfalse%
{\isacharparenleft}{\kern0pt}rule\ iffD{\isadigit{2}}{\isacharcomma}{\kern0pt}\ rule\ M{\isacharunderscore}{\kern0pt}basic{\isachardot}{\kern0pt}injection{\isacharunderscore}{\kern0pt}abs{\isacharparenright}{\kern0pt}\isanewline
\ \ \ \ \isacommand{using}\isamarkupfalse%
\ nat{\isacharunderscore}{\kern0pt}in{\isacharunderscore}{\kern0pt}M\ M{\isacharunderscore}{\kern0pt}subset{\isacharunderscore}{\kern0pt}SymExt\ gH\ binmap{\isacharprime}{\kern0pt}{\isacharunderscore}{\kern0pt}HS\ SymExt{\isacharunderscore}{\kern0pt}def\ gH\ binmap{\isacharunderscore}{\kern0pt}eq\ assms\ M{\isacharunderscore}{\kern0pt}ZF{\isacharunderscore}{\kern0pt}trans{\isachardot}{\kern0pt}mbasic\ SymExt{\isacharunderscore}{\kern0pt}M{\isacharunderscore}{\kern0pt}ZF{\isacharunderscore}{\kern0pt}trans\isanewline
\ \ \ \ \isacommand{by}\isamarkupfalse%
\ auto\isanewline
\ \ \isacommand{then}\isamarkupfalse%
\ \isacommand{have}\isamarkupfalse%
\ {\isachardoublequoteopen}{\isasymexists}p\ {\isasymin}\ G{\isachardot}{\kern0pt}\ {\isacharparenleft}{\kern0pt}ForcesHS{\isacharparenleft}{\kern0pt}p{\isacharcomma}{\kern0pt}\ injection{\isacharunderscore}{\kern0pt}fm{\isacharparenleft}{\kern0pt}{\isadigit{0}}{\isacharcomma}{\kern0pt}{\isadigit{1}}{\isacharcomma}{\kern0pt}{\isadigit{2}}{\isacharparenright}{\kern0pt}{\isacharcomma}{\kern0pt}\ {\isacharbrackleft}{\kern0pt}check{\isacharparenleft}{\kern0pt}nat{\isacharparenright}{\kern0pt}{\isacharcomma}{\kern0pt}\ binmap{\isacharprime}{\kern0pt}{\isacharcomma}{\kern0pt}\ g{\isacharprime}{\kern0pt}{\isacharbrackright}{\kern0pt}{\isacharparenright}{\kern0pt}{\isacharparenright}{\kern0pt}{\isachardoublequoteclose}\ \isanewline
\ \ \ \ \isacommand{apply}\isamarkupfalse%
{\isacharparenleft}{\kern0pt}rule{\isacharunderscore}{\kern0pt}tac\ iffD{\isadigit{2}}{\isacharparenright}{\kern0pt}\isanewline
\ \ \ \ \ \isacommand{apply}\isamarkupfalse%
{\isacharparenleft}{\kern0pt}rule\ HS{\isacharunderscore}{\kern0pt}truth{\isacharunderscore}{\kern0pt}lemma{\isacharparenright}{\kern0pt}\isanewline
\ \ \ \ \isacommand{using}\isamarkupfalse%
\ assms\ check{\isacharunderscore}{\kern0pt}in{\isacharunderscore}{\kern0pt}HS\ nat{\isacharunderscore}{\kern0pt}in{\isacharunderscore}{\kern0pt}M\ binmap{\isacharprime}{\kern0pt}{\isacharunderscore}{\kern0pt}HS\ g{\isacharprime}{\kern0pt}H\isanewline
\ \ \ \ \ \ \ \ \isacommand{apply}\isamarkupfalse%
\ auto{\isacharbrackleft}{\kern0pt}{\isadigit{3}}{\isacharbrackright}{\kern0pt}\isanewline
\ \ \ \ \ \isacommand{apply}\isamarkupfalse%
{\isacharparenleft}{\kern0pt}simp\ add{\isacharcolon}{\kern0pt}injection{\isacharunderscore}{\kern0pt}fm{\isacharunderscore}{\kern0pt}def{\isacharparenright}{\kern0pt}\isanewline
\ \ \ \ \ \isacommand{apply}\isamarkupfalse%
\ {\isacharparenleft}{\kern0pt}simp\ del{\isacharcolon}{\kern0pt}FOL{\isacharunderscore}{\kern0pt}sats{\isacharunderscore}{\kern0pt}iff\ pair{\isacharunderscore}{\kern0pt}abs\ add{\isacharcolon}{\kern0pt}\ fm{\isacharunderscore}{\kern0pt}defs\ nat{\isacharunderscore}{\kern0pt}simp{\isacharunderscore}{\kern0pt}union{\isacharparenright}{\kern0pt}\isanewline
\ \ \ \ \isacommand{using}\isamarkupfalse%
\ mapeq\isanewline
\ \ \ \ \isacommand{by}\isamarkupfalse%
\ auto\isanewline
\ \ \isacommand{then}\isamarkupfalse%
\ \isacommand{obtain}\isamarkupfalse%
\ p\ \isakeyword{where}\ {\isachardoublequoteopen}p\ {\isasymin}\ G{\isachardoublequoteclose}\ {\isachardoublequoteopen}ForcesHS{\isacharparenleft}{\kern0pt}p{\isacharcomma}{\kern0pt}\ injection{\isacharunderscore}{\kern0pt}fm{\isacharparenleft}{\kern0pt}{\isadigit{0}}{\isacharcomma}{\kern0pt}{\isadigit{1}}{\isacharcomma}{\kern0pt}{\isadigit{2}}{\isacharparenright}{\kern0pt}{\isacharcomma}{\kern0pt}\ {\isacharbrackleft}{\kern0pt}check{\isacharparenleft}{\kern0pt}nat{\isacharparenright}{\kern0pt}{\isacharcomma}{\kern0pt}\ binmap{\isacharprime}{\kern0pt}{\isacharcomma}{\kern0pt}\ g{\isacharprime}{\kern0pt}{\isacharbrackright}{\kern0pt}{\isacharparenright}{\kern0pt}{\isachardoublequoteclose}\ \isacommand{by}\isamarkupfalse%
\ auto\isanewline
\ \ \isacommand{then}\isamarkupfalse%
\ \isacommand{show}\isamarkupfalse%
\ False\ \isanewline
\ \ \ \ \isacommand{apply}\isamarkupfalse%
{\isacharparenleft}{\kern0pt}rule{\isacharunderscore}{\kern0pt}tac\ F{\isacharprime}{\kern0pt}{\isacharequal}{\kern0pt}g{\isacharprime}{\kern0pt}\ \isakeyword{in}\ no{\isacharunderscore}{\kern0pt}injection{\isacharparenright}{\kern0pt}\isanewline
\ \ \ \ \isacommand{using}\isamarkupfalse%
\ g{\isacharprime}{\kern0pt}H\ M{\isacharunderscore}{\kern0pt}genericD\ assms\ \isanewline
\ \ \ \ \isacommand{by}\isamarkupfalse%
\ auto\isanewline
\isacommand{qed}\isamarkupfalse%
%
\endisatagproof
{\isafoldproof}%
%
\isadelimproof
\isanewline
%
\endisadelimproof
\isanewline
\isanewline
\isanewline
\isanewline
\ \ \isanewline
\isanewline
\isanewline
\isacommand{end}\isamarkupfalse%
\isanewline
%
\isadelimtheory
%
\endisadelimtheory
%
\isatagtheory
\isacommand{end}\isamarkupfalse%
%
\endisatagtheory
{\isafoldtheory}%
%
\isadelimtheory
%
\endisadelimtheory
%
\end{isabellebody}%
\endinput
%:%file=~/source/repos/ZF-notAC/code/NotAC_Proof.thy%:%
%:%10=1%:%
%:%11=1%:%
%:%12=2%:%
%:%13=3%:%
%:%14=4%:%
%:%15=5%:%
%:%20=5%:%
%:%23=6%:%
%:%24=7%:%
%:%25=7%:%
%:%26=8%:%
%:%27=9%:%
%:%28=9%:%
%:%31=10%:%
%:%35=10%:%
%:%36=10%:%
%:%37=11%:%
%:%38=11%:%
%:%43=11%:%
%:%46=12%:%
%:%47=13%:%
%:%48=13%:%
%:%49=14%:%
%:%50=15%:%
%:%51=16%:%
%:%52=17%:%
%:%59=18%:%
%:%60=18%:%
%:%61=19%:%
%:%62=20%:%
%:%63=20%:%
%:%64=21%:%
%:%65=22%:%
%:%66=22%:%
%:%67=23%:%
%:%68=23%:%
%:%69=24%:%
%:%70=24%:%
%:%71=25%:%
%:%72=25%:%
%:%73=26%:%
%:%74=27%:%
%:%75=27%:%
%:%76=28%:%
%:%77=28%:%
%:%78=29%:%
%:%79=29%:%
%:%80=30%:%
%:%81=30%:%
%:%82=31%:%
%:%83=31%:%
%:%84=32%:%
%:%85=32%:%
%:%86=33%:%
%:%87=33%:%
%:%88=34%:%
%:%89=34%:%
%:%90=35%:%
%:%91=35%:%
%:%92=36%:%
%:%93=36%:%
%:%94=37%:%
%:%95=37%:%
%:%96=38%:%
%:%97=38%:%
%:%98=39%:%
%:%99=40%:%
%:%100=40%:%
%:%101=41%:%
%:%102=41%:%
%:%103=42%:%
%:%104=42%:%
%:%105=43%:%
%:%106=43%:%
%:%107=44%:%
%:%108=44%:%
%:%109=45%:%
%:%110=45%:%
%:%111=46%:%
%:%112=46%:%
%:%113=47%:%
%:%114=47%:%
%:%115=48%:%
%:%116=48%:%
%:%117=49%:%
%:%118=49%:%
%:%119=50%:%
%:%120=50%:%
%:%121=51%:%
%:%122=51%:%
%:%123=52%:%
%:%124=52%:%
%:%125=53%:%
%:%126=53%:%
%:%127=54%:%
%:%128=54%:%
%:%129=55%:%
%:%130=55%:%
%:%131=56%:%
%:%132=56%:%
%:%133=57%:%
%:%134=57%:%
%:%135=58%:%
%:%136=58%:%
%:%137=59%:%
%:%138=59%:%
%:%139=60%:%
%:%140=61%:%
%:%141=61%:%
%:%142=62%:%
%:%143=62%:%
%:%144=63%:%
%:%145=63%:%
%:%146=64%:%
%:%147=65%:%
%:%148=65%:%
%:%149=66%:%
%:%150=66%:%
%:%151=67%:%
%:%152=67%:%
%:%153=68%:%
%:%154=68%:%
%:%155=69%:%
%:%156=69%:%
%:%157=70%:%
%:%158=70%:%
%:%159=71%:%
%:%160=71%:%
%:%161=72%:%
%:%162=72%:%
%:%163=73%:%
%:%164=73%:%
%:%165=74%:%
%:%166=74%:%
%:%167=75%:%
%:%168=75%:%
%:%169=76%:%
%:%170=76%:%
%:%171=77%:%
%:%172=78%:%
%:%173=78%:%
%:%174=78%:%
%:%175=79%:%
%:%176=80%:%
%:%177=80%:%
%:%178=81%:%
%:%179=82%:%
%:%180=82%:%
%:%181=82%:%
%:%182=82%:%
%:%183=83%:%
%:%184=84%:%
%:%185=84%:%
%:%186=85%:%
%:%187=85%:%
%:%188=86%:%
%:%189=86%:%
%:%190=87%:%
%:%191=87%:%
%:%192=87%:%
%:%193=88%:%
%:%194=88%:%
%:%195=89%:%
%:%196=89%:%
%:%197=90%:%
%:%198=91%:%
%:%199=91%:%
%:%200=92%:%
%:%201=92%:%
%:%202=93%:%
%:%203=93%:%
%:%204=94%:%
%:%205=94%:%
%:%206=94%:%
%:%207=95%:%
%:%208=95%:%
%:%209=96%:%
%:%210=96%:%
%:%211=97%:%
%:%212=97%:%
%:%213=98%:%
%:%214=99%:%
%:%215=99%:%
%:%216=100%:%
%:%217=100%:%
%:%218=101%:%
%:%219=101%:%
%:%220=102%:%
%:%221=102%:%
%:%222=103%:%
%:%223=103%:%
%:%224=104%:%
%:%225=104%:%
%:%226=105%:%
%:%227=105%:%
%:%228=106%:%
%:%229=106%:%
%:%230=107%:%
%:%231=107%:%
%:%232=107%:%
%:%233=108%:%
%:%234=108%:%
%:%235=109%:%
%:%236=109%:%
%:%237=110%:%
%:%238=110%:%
%:%239=110%:%
%:%240=111%:%
%:%241=111%:%
%:%242=112%:%
%:%243=112%:%
%:%244=113%:%
%:%245=113%:%
%:%246=114%:%
%:%247=114%:%
%:%248=115%:%
%:%249=115%:%
%:%250=116%:%
%:%251=116%:%
%:%252=117%:%
%:%253=117%:%
%:%254=117%:%
%:%255=118%:%
%:%256=118%:%
%:%257=119%:%
%:%258=119%:%
%:%259=120%:%
%:%260=120%:%
%:%261=121%:%
%:%262=121%:%
%:%263=122%:%
%:%264=122%:%
%:%265=123%:%
%:%266=123%:%
%:%267=124%:%
%:%268=125%:%
%:%269=125%:%
%:%270=126%:%
%:%271=126%:%
%:%272=127%:%
%:%273=127%:%
%:%274=128%:%
%:%275=128%:%
%:%276=129%:%
%:%277=129%:%
%:%278=129%:%
%:%279=130%:%
%:%280=130%:%
%:%281=131%:%
%:%282=131%:%
%:%283=132%:%
%:%284=132%:%
%:%285=133%:%
%:%286=133%:%
%:%287=134%:%
%:%288=134%:%
%:%289=135%:%
%:%290=135%:%
%:%291=136%:%
%:%292=136%:%
%:%293=137%:%
%:%294=137%:%
%:%295=138%:%
%:%296=138%:%
%:%297=139%:%
%:%298=139%:%
%:%299=140%:%
%:%300=140%:%
%:%301=140%:%
%:%302=141%:%
%:%303=141%:%
%:%304=142%:%
%:%305=142%:%
%:%306=143%:%
%:%307=143%:%
%:%308=144%:%
%:%309=144%:%
%:%310=145%:%
%:%311=145%:%
%:%312=146%:%
%:%313=146%:%
%:%314=147%:%
%:%315=147%:%
%:%316=147%:%
%:%317=147%:%
%:%318=147%:%
%:%319=148%:%
%:%320=148%:%
%:%321=148%:%
%:%322=149%:%
%:%323=149%:%
%:%324=150%:%
%:%325=150%:%
%:%326=151%:%
%:%327=151%:%
%:%328=152%:%
%:%329=152%:%
%:%330=152%:%
%:%331=153%:%
%:%332=153%:%
%:%333=153%:%
%:%334=154%:%
%:%335=154%:%
%:%336=154%:%
%:%337=154%:%
%:%338=154%:%
%:%339=155%:%
%:%340=155%:%
%:%341=156%:%
%:%342=157%:%
%:%343=157%:%
%:%344=157%:%
%:%345=158%:%
%:%346=158%:%
%:%347=159%:%
%:%348=159%:%
%:%349=160%:%
%:%350=160%:%
%:%351=161%:%
%:%352=161%:%
%:%353=162%:%
%:%354=162%:%
%:%355=163%:%
%:%356=163%:%
%:%357=164%:%
%:%358=164%:%
%:%359=165%:%
%:%360=165%:%
%:%361=166%:%
%:%362=166%:%
%:%363=166%:%
%:%364=166%:%
%:%365=166%:%
%:%366=167%:%
%:%367=167%:%
%:%368=167%:%
%:%369=167%:%
%:%370=167%:%
%:%371=168%:%
%:%372=168%:%
%:%373=168%:%
%:%374=169%:%
%:%375=169%:%
%:%376=170%:%
%:%377=170%:%
%:%378=171%:%
%:%379=171%:%
%:%380=172%:%
%:%381=173%:%
%:%382=173%:%
%:%383=174%:%
%:%384=174%:%
%:%385=175%:%
%:%386=175%:%
%:%387=176%:%
%:%388=176%:%
%:%389=177%:%
%:%390=178%:%
%:%391=178%:%
%:%392=179%:%
%:%393=179%:%
%:%394=180%:%
%:%395=180%:%
%:%396=181%:%
%:%397=181%:%
%:%398=182%:%
%:%399=182%:%
%:%400=183%:%
%:%401=183%:%
%:%402=184%:%
%:%403=184%:%
%:%404=185%:%
%:%405=185%:%
%:%406=186%:%
%:%407=186%:%
%:%408=187%:%
%:%409=188%:%
%:%410=188%:%
%:%411=188%:%
%:%412=189%:%
%:%413=189%:%
%:%414=190%:%
%:%415=190%:%
%:%416=191%:%
%:%417=191%:%
%:%418=192%:%
%:%419=192%:%
%:%420=193%:%
%:%421=193%:%
%:%422=194%:%
%:%423=195%:%
%:%424=195%:%
%:%425=196%:%
%:%426=196%:%
%:%427=197%:%
%:%428=197%:%
%:%429=198%:%
%:%430=198%:%
%:%431=199%:%
%:%432=199%:%
%:%433=200%:%
%:%434=200%:%
%:%435=201%:%
%:%436=202%:%
%:%437=202%:%
%:%438=203%:%
%:%439=203%:%
%:%440=204%:%
%:%441=204%:%
%:%442=205%:%
%:%443=205%:%
%:%444=206%:%
%:%445=206%:%
%:%446=206%:%
%:%447=207%:%
%:%448=207%:%
%:%449=208%:%
%:%450=208%:%
%:%451=209%:%
%:%452=210%:%
%:%453=210%:%
%:%454=211%:%
%:%455=211%:%
%:%456=212%:%
%:%457=212%:%
%:%458=213%:%
%:%459=213%:%
%:%460=214%:%
%:%461=214%:%
%:%462=214%:%
%:%463=215%:%
%:%464=215%:%
%:%465=216%:%
%:%466=216%:%
%:%467=217%:%
%:%468=217%:%
%:%469=218%:%
%:%470=218%:%
%:%471=219%:%
%:%472=219%:%
%:%473=220%:%
%:%474=220%:%
%:%475=221%:%
%:%476=221%:%
%:%477=222%:%
%:%478=222%:%
%:%479=223%:%
%:%480=223%:%
%:%481=224%:%
%:%482=224%:%
%:%483=225%:%
%:%484=225%:%
%:%485=225%:%
%:%486=225%:%
%:%487=226%:%
%:%488=227%:%
%:%489=227%:%
%:%490=227%:%
%:%491=227%:%
%:%492=228%:%
%:%493=229%:%
%:%494=229%:%
%:%495=230%:%
%:%496=230%:%
%:%497=231%:%
%:%498=231%:%
%:%499=232%:%
%:%500=233%:%
%:%501=233%:%
%:%502=234%:%
%:%503=234%:%
%:%504=235%:%
%:%505=235%:%
%:%506=236%:%
%:%507=236%:%
%:%508=237%:%
%:%509=237%:%
%:%510=238%:%
%:%511=238%:%
%:%512=239%:%
%:%513=239%:%
%:%514=240%:%
%:%515=240%:%
%:%516=241%:%
%:%517=241%:%
%:%518=242%:%
%:%519=242%:%
%:%520=243%:%
%:%521=243%:%
%:%522=243%:%
%:%523=243%:%
%:%524=243%:%
%:%525=244%:%
%:%526=244%:%
%:%527=244%:%
%:%528=245%:%
%:%529=245%:%
%:%530=246%:%
%:%531=246%:%
%:%532=247%:%
%:%533=248%:%
%:%534=248%:%
%:%535=249%:%
%:%536=249%:%
%:%537=250%:%
%:%538=250%:%
%:%539=251%:%
%:%540=252%:%
%:%541=252%:%
%:%542=253%:%
%:%543=253%:%
%:%544=254%:%
%:%545=254%:%
%:%546=255%:%
%:%547=255%:%
%:%548=256%:%
%:%549=256%:%
%:%550=257%:%
%:%551=257%:%
%:%552=258%:%
%:%553=258%:%
%:%554=259%:%
%:%555=259%:%
%:%556=260%:%
%:%557=260%:%
%:%558=261%:%
%:%559=261%:%
%:%560=262%:%
%:%561=262%:%
%:%562=263%:%
%:%563=264%:%
%:%564=264%:%
%:%565=265%:%
%:%566=266%:%
%:%567=266%:%
%:%568=267%:%
%:%569=267%:%
%:%570=268%:%
%:%571=268%:%
%:%572=269%:%
%:%573=269%:%
%:%574=270%:%
%:%575=270%:%
%:%576=271%:%
%:%577=271%:%
%:%578=272%:%
%:%579=272%:%
%:%580=273%:%
%:%581=273%:%
%:%582=274%:%
%:%583=274%:%
%:%584=275%:%
%:%585=275%:%
%:%586=276%:%
%:%587=276%:%
%:%588=277%:%
%:%589=277%:%
%:%590=278%:%
%:%591=279%:%
%:%592=279%:%
%:%593=280%:%
%:%594=280%:%
%:%595=281%:%
%:%596=281%:%
%:%597=282%:%
%:%598=282%:%
%:%599=283%:%
%:%600=283%:%
%:%601=284%:%
%:%602=284%:%
%:%603=285%:%
%:%604=285%:%
%:%605=286%:%
%:%606=286%:%
%:%607=287%:%
%:%608=287%:%
%:%609=288%:%
%:%610=288%:%
%:%611=289%:%
%:%612=289%:%
%:%613=290%:%
%:%614=290%:%
%:%615=291%:%
%:%616=291%:%
%:%617=292%:%
%:%618=293%:%
%:%619=293%:%
%:%622=296%:%
%:%623=297%:%
%:%624=298%:%
%:%625=298%:%
%:%626=299%:%
%:%627=300%:%
%:%628=300%:%
%:%629=301%:%
%:%632=304%:%
%:%633=305%:%
%:%634=305%:%
%:%635=306%:%
%:%636=306%:%
%:%637=307%:%
%:%638=307%:%
%:%639=308%:%
%:%640=308%:%
%:%641=309%:%
%:%642=309%:%
%:%643=310%:%
%:%644=310%:%
%:%645=311%:%
%:%646=311%:%
%:%647=312%:%
%:%648=312%:%
%:%649=313%:%
%:%650=313%:%
%:%651=314%:%
%:%652=314%:%
%:%653=315%:%
%:%654=315%:%
%:%655=316%:%
%:%656=316%:%
%:%657=317%:%
%:%658=317%:%
%:%659=318%:%
%:%660=318%:%
%:%661=319%:%
%:%662=319%:%
%:%663=320%:%
%:%664=320%:%
%:%665=321%:%
%:%666=321%:%
%:%667=322%:%
%:%668=322%:%
%:%669=323%:%
%:%670=323%:%
%:%671=324%:%
%:%672=324%:%
%:%673=324%:%
%:%674=325%:%
%:%675=325%:%
%:%676=326%:%
%:%677=326%:%
%:%678=327%:%
%:%679=327%:%
%:%680=328%:%
%:%681=328%:%
%:%682=329%:%
%:%683=329%:%
%:%684=330%:%
%:%685=330%:%
%:%686=331%:%
%:%687=331%:%
%:%688=332%:%
%:%689=332%:%
%:%690=333%:%
%:%691=333%:%
%:%692=334%:%
%:%693=334%:%
%:%694=335%:%
%:%695=335%:%
%:%696=336%:%
%:%697=336%:%
%:%698=337%:%
%:%699=337%:%
%:%700=338%:%
%:%701=338%:%
%:%702=339%:%
%:%703=339%:%
%:%704=340%:%
%:%705=340%:%
%:%706=341%:%
%:%707=341%:%
%:%708=341%:%
%:%709=342%:%
%:%710=342%:%
%:%711=343%:%
%:%712=343%:%
%:%713=344%:%
%:%714=344%:%
%:%715=345%:%
%:%716=345%:%
%:%717=346%:%
%:%718=346%:%
%:%719=347%:%
%:%720=347%:%
%:%721=348%:%
%:%722=348%:%
%:%723=349%:%
%:%724=349%:%
%:%725=350%:%
%:%726=350%:%
%:%727=351%:%
%:%728=351%:%
%:%729=352%:%
%:%730=352%:%
%:%731=353%:%
%:%732=353%:%
%:%733=354%:%
%:%734=354%:%
%:%735=355%:%
%:%736=355%:%
%:%737=356%:%
%:%738=356%:%
%:%739=357%:%
%:%740=357%:%
%:%741=358%:%
%:%742=358%:%
%:%743=359%:%
%:%744=360%:%
%:%745=360%:%
%:%746=361%:%
%:%747=362%:%
%:%748=362%:%
%:%749=363%:%
%:%750=363%:%
%:%751=364%:%
%:%752=364%:%
%:%753=365%:%
%:%754=365%:%
%:%755=366%:%
%:%756=367%:%
%:%757=367%:%
%:%758=368%:%
%:%759=368%:%
%:%760=369%:%
%:%761=369%:%
%:%762=370%:%
%:%763=370%:%
%:%764=371%:%
%:%765=371%:%
%:%766=371%:%
%:%767=372%:%
%:%768=372%:%
%:%769=373%:%
%:%770=373%:%
%:%771=374%:%
%:%772=375%:%
%:%773=375%:%
%:%774=376%:%
%:%775=376%:%
%:%776=377%:%
%:%777=377%:%
%:%778=378%:%
%:%779=378%:%
%:%780=378%:%
%:%781=378%:%
%:%782=378%:%
%:%783=379%:%
%:%784=379%:%
%:%785=379%:%
%:%786=379%:%
%:%787=379%:%
%:%788=380%:%
%:%789=380%:%
%:%790=380%:%
%:%791=380%:%
%:%792=380%:%
%:%793=381%:%
%:%794=381%:%
%:%795=382%:%
%:%796=382%:%
%:%797=383%:%
%:%798=383%:%
%:%799=383%:%
%:%800=384%:%
%:%801=384%:%
%:%802=385%:%
%:%803=385%:%
%:%804=386%:%
%:%805=386%:%
%:%806=386%:%
%:%807=386%:%
%:%808=386%:%
%:%809=387%:%
%:%810=387%:%
%:%811=387%:%
%:%812=387%:%
%:%813=387%:%
%:%814=388%:%
%:%815=388%:%
%:%816=389%:%
%:%817=389%:%
%:%818=390%:%
%:%819=390%:%
%:%820=390%:%
%:%821=390%:%
%:%822=390%:%
%:%823=391%:%
%:%824=391%:%
%:%825=391%:%
%:%826=391%:%
%:%827=391%:%
%:%828=392%:%
%:%829=392%:%
%:%830=392%:%
%:%831=392%:%
%:%832=392%:%
%:%833=393%:%
%:%834=393%:%
%:%835=393%:%
%:%836=393%:%
%:%837=393%:%
%:%838=394%:%
%:%839=394%:%
%:%840=394%:%
%:%841=394%:%
%:%842=394%:%
%:%843=395%:%
%:%844=395%:%
%:%845=396%:%
%:%846=396%:%
%:%847=397%:%
%:%848=397%:%
%:%849=398%:%
%:%850=398%:%
%:%851=398%:%
%:%852=399%:%
%:%853=399%:%
%:%854=400%:%
%:%855=400%:%
%:%856=401%:%
%:%857=401%:%
%:%858=402%:%
%:%859=402%:%
%:%860=403%:%
%:%861=403%:%
%:%862=404%:%
%:%863=404%:%
%:%864=405%:%
%:%865=405%:%
%:%866=406%:%
%:%867=406%:%
%:%868=407%:%
%:%869=407%:%
%:%870=408%:%
%:%871=408%:%
%:%872=409%:%
%:%873=409%:%
%:%874=410%:%
%:%875=410%:%
%:%876=411%:%
%:%877=411%:%
%:%878=412%:%
%:%879=412%:%
%:%880=413%:%
%:%881=413%:%
%:%882=414%:%
%:%883=414%:%
%:%884=415%:%
%:%885=415%:%
%:%886=416%:%
%:%887=416%:%
%:%888=417%:%
%:%889=417%:%
%:%890=418%:%
%:%891=418%:%
%:%892=419%:%
%:%893=419%:%
%:%894=420%:%
%:%895=420%:%
%:%896=421%:%
%:%897=421%:%
%:%898=422%:%
%:%899=422%:%
%:%900=423%:%
%:%901=423%:%
%:%902=424%:%
%:%903=424%:%
%:%904=425%:%
%:%905=425%:%
%:%906=426%:%
%:%907=426%:%
%:%908=427%:%
%:%909=427%:%
%:%910=428%:%
%:%911=428%:%
%:%912=429%:%
%:%913=429%:%
%:%914=430%:%
%:%915=430%:%
%:%916=431%:%
%:%917=431%:%
%:%918=432%:%
%:%919=432%:%
%:%920=433%:%
%:%921=433%:%
%:%922=434%:%
%:%923=434%:%
%:%924=435%:%
%:%925=435%:%
%:%926=436%:%
%:%927=436%:%
%:%928=437%:%
%:%929=437%:%
%:%930=438%:%
%:%931=438%:%
%:%932=439%:%
%:%933=439%:%
%:%934=440%:%
%:%935=440%:%
%:%936=441%:%
%:%937=442%:%
%:%938=442%:%
%:%939=443%:%
%:%940=443%:%
%:%941=444%:%
%:%942=444%:%
%:%943=445%:%
%:%944=445%:%
%:%945=446%:%
%:%946=446%:%
%:%947=447%:%
%:%948=447%:%
%:%949=448%:%
%:%950=448%:%
%:%951=449%:%
%:%952=449%:%
%:%953=450%:%
%:%954=450%:%
%:%955=451%:%
%:%956=451%:%
%:%957=452%:%
%:%958=452%:%
%:%959=453%:%
%:%960=453%:%
%:%961=454%:%
%:%962=454%:%
%:%963=455%:%
%:%964=456%:%
%:%965=456%:%
%:%966=457%:%
%:%967=457%:%
%:%968=458%:%
%:%969=458%:%
%:%970=459%:%
%:%971=459%:%
%:%972=460%:%
%:%973=461%:%
%:%974=461%:%
%:%975=462%:%
%:%976=462%:%
%:%977=463%:%
%:%978=463%:%
%:%979=464%:%
%:%980=465%:%
%:%981=465%:%
%:%982=466%:%
%:%983=466%:%
%:%984=467%:%
%:%985=467%:%
%:%986=468%:%
%:%987=468%:%
%:%988=469%:%
%:%989=469%:%
%:%990=470%:%
%:%991=470%:%
%:%992=471%:%
%:%993=471%:%
%:%994=472%:%
%:%995=472%:%
%:%996=473%:%
%:%997=473%:%
%:%998=474%:%
%:%999=474%:%
%:%1000=475%:%
%:%1001=476%:%
%:%1002=476%:%
%:%1003=477%:%
%:%1004=477%:%
%:%1005=478%:%
%:%1006=478%:%
%:%1007=479%:%
%:%1008=479%:%
%:%1009=480%:%
%:%1010=480%:%
%:%1011=481%:%
%:%1012=481%:%
%:%1013=482%:%
%:%1014=482%:%
%:%1015=483%:%
%:%1016=483%:%
%:%1017=484%:%
%:%1018=484%:%
%:%1019=485%:%
%:%1020=485%:%
%:%1021=486%:%
%:%1022=486%:%
%:%1023=487%:%
%:%1024=487%:%
%:%1025=488%:%
%:%1026=488%:%
%:%1027=489%:%
%:%1028=489%:%
%:%1029=490%:%
%:%1030=490%:%
%:%1031=490%:%
%:%1032=491%:%
%:%1033=491%:%
%:%1034=492%:%
%:%1035=492%:%
%:%1036=493%:%
%:%1037=494%:%
%:%1038=494%:%
%:%1039=495%:%
%:%1040=495%:%
%:%1041=496%:%
%:%1042=496%:%
%:%1043=497%:%
%:%1044=498%:%
%:%1045=498%:%
%:%1046=499%:%
%:%1047=499%:%
%:%1048=500%:%
%:%1049=500%:%
%:%1050=501%:%
%:%1051=501%:%
%:%1052=502%:%
%:%1053=502%:%
%:%1054=503%:%
%:%1055=503%:%
%:%1056=504%:%
%:%1057=504%:%
%:%1058=505%:%
%:%1059=505%:%
%:%1060=506%:%
%:%1061=506%:%
%:%1062=507%:%
%:%1063=507%:%
%:%1064=508%:%
%:%1065=509%:%
%:%1066=509%:%
%:%1067=510%:%
%:%1068=510%:%
%:%1069=511%:%
%:%1070=511%:%
%:%1071=512%:%
%:%1072=512%:%
%:%1073=513%:%
%:%1074=513%:%
%:%1075=514%:%
%:%1076=514%:%
%:%1077=515%:%
%:%1078=515%:%
%:%1079=516%:%
%:%1080=516%:%
%:%1081=517%:%
%:%1082=517%:%
%:%1083=518%:%
%:%1084=518%:%
%:%1085=519%:%
%:%1086=519%:%
%:%1087=520%:%
%:%1088=521%:%
%:%1089=521%:%
%:%1090=522%:%
%:%1091=522%:%
%:%1092=523%:%
%:%1093=523%:%
%:%1094=524%:%
%:%1095=524%:%
%:%1096=525%:%
%:%1097=525%:%
%:%1098=526%:%
%:%1099=526%:%
%:%1100=527%:%
%:%1101=527%:%
%:%1102=528%:%
%:%1103=528%:%
%:%1104=529%:%
%:%1105=529%:%
%:%1106=530%:%
%:%1107=530%:%
%:%1108=531%:%
%:%1109=531%:%
%:%1110=532%:%
%:%1111=532%:%
%:%1112=532%:%
%:%1113=533%:%
%:%1114=533%:%
%:%1115=534%:%
%:%1116=534%:%
%:%1117=535%:%
%:%1118=535%:%
%:%1119=536%:%
%:%1120=536%:%
%:%1121=537%:%
%:%1122=537%:%
%:%1123=538%:%
%:%1124=538%:%
%:%1125=539%:%
%:%1126=539%:%
%:%1127=540%:%
%:%1128=541%:%
%:%1129=541%:%
%:%1130=542%:%
%:%1131=542%:%
%:%1132=543%:%
%:%1133=543%:%
%:%1134=544%:%
%:%1135=544%:%
%:%1136=545%:%
%:%1137=545%:%
%:%1138=546%:%
%:%1139=546%:%
%:%1140=547%:%
%:%1141=547%:%
%:%1142=548%:%
%:%1143=548%:%
%:%1144=549%:%
%:%1145=549%:%
%:%1146=550%:%
%:%1147=550%:%
%:%1148=551%:%
%:%1149=551%:%
%:%1150=552%:%
%:%1151=552%:%
%:%1152=553%:%
%:%1153=553%:%
%:%1154=554%:%
%:%1155=554%:%
%:%1156=555%:%
%:%1157=555%:%
%:%1158=556%:%
%:%1159=556%:%
%:%1160=557%:%
%:%1161=557%:%
%:%1162=558%:%
%:%1163=558%:%
%:%1164=559%:%
%:%1165=559%:%
%:%1166=559%:%
%:%1167=560%:%
%:%1168=560%:%
%:%1169=561%:%
%:%1170=561%:%
%:%1171=562%:%
%:%1172=562%:%
%:%1173=563%:%
%:%1174=563%:%
%:%1175=564%:%
%:%1176=564%:%
%:%1177=565%:%
%:%1178=565%:%
%:%1179=566%:%
%:%1180=566%:%
%:%1181=567%:%
%:%1182=568%:%
%:%1183=568%:%
%:%1184=569%:%
%:%1185=569%:%
%:%1186=570%:%
%:%1187=570%:%
%:%1188=571%:%
%:%1189=572%:%
%:%1190=572%:%
%:%1191=573%:%
%:%1192=574%:%
%:%1193=574%:%
%:%1194=575%:%
%:%1195=575%:%
%:%1196=576%:%
%:%1197=576%:%
%:%1198=577%:%
%:%1199=577%:%
%:%1200=577%:%
%:%1201=578%:%
%:%1202=578%:%
%:%1203=579%:%
%:%1204=579%:%
%:%1205=580%:%
%:%1206=580%:%
%:%1207=581%:%
%:%1208=581%:%
%:%1209=582%:%
%:%1210=582%:%
%:%1211=583%:%
%:%1212=583%:%
%:%1213=584%:%
%:%1214=584%:%
%:%1215=584%:%
%:%1216=585%:%
%:%1217=585%:%
%:%1218=586%:%
%:%1219=586%:%
%:%1220=587%:%
%:%1221=587%:%
%:%1222=588%:%
%:%1223=588%:%
%:%1224=589%:%
%:%1225=589%:%
%:%1226=590%:%
%:%1227=590%:%
%:%1228=591%:%
%:%1229=591%:%
%:%1230=592%:%
%:%1231=593%:%
%:%1232=593%:%
%:%1233=594%:%
%:%1234=594%:%
%:%1235=595%:%
%:%1236=595%:%
%:%1237=596%:%
%:%1238=596%:%
%:%1239=596%:%
%:%1240=597%:%
%:%1241=597%:%
%:%1242=598%:%
%:%1243=598%:%
%:%1244=599%:%
%:%1245=599%:%
%:%1246=600%:%
%:%1247=600%:%
%:%1248=601%:%
%:%1249=601%:%
%:%1250=602%:%
%:%1251=602%:%
%:%1252=603%:%
%:%1253=603%:%
%:%1254=603%:%
%:%1255=604%:%
%:%1256=604%:%
%:%1257=605%:%
%:%1258=605%:%
%:%1259=606%:%
%:%1260=606%:%
%:%1261=607%:%
%:%1262=607%:%
%:%1263=608%:%
%:%1264=608%:%
%:%1265=609%:%
%:%1266=609%:%
%:%1267=610%:%
%:%1268=610%:%
%:%1269=611%:%
%:%1270=611%:%
%:%1271=612%:%
%:%1272=613%:%
%:%1273=613%:%
%:%1274=613%:%
%:%1275=613%:%
%:%1276=614%:%
%:%1277=615%:%
%:%1278=615%:%
%:%1279=616%:%
%:%1280=616%:%
%:%1281=617%:%
%:%1282=617%:%
%:%1283=618%:%
%:%1284=618%:%
%:%1285=619%:%
%:%1286=619%:%
%:%1287=620%:%
%:%1288=620%:%
%:%1289=621%:%
%:%1290=621%:%
%:%1291=622%:%
%:%1292=622%:%
%:%1293=623%:%
%:%1294=623%:%
%:%1295=624%:%
%:%1296=625%:%
%:%1297=625%:%
%:%1298=626%:%
%:%1299=626%:%
%:%1300=627%:%
%:%1301=627%:%
%:%1302=628%:%
%:%1308=628%:%
%:%1311=629%:%
%:%1312=630%:%
%:%1313=630%:%
%:%1314=631%:%
%:%1315=632%:%
%:%1316=633%:%
%:%1323=634%:%
%:%1324=634%:%
%:%1325=635%:%
%:%1326=636%:%
%:%1327=636%:%
%:%1328=637%:%
%:%1329=637%:%
%:%1330=638%:%
%:%1331=638%:%
%:%1332=639%:%
%:%1333=639%:%
%:%1334=640%:%
%:%1335=640%:%
%:%1336=641%:%
%:%1337=641%:%
%:%1338=642%:%
%:%1339=642%:%
%:%1340=643%:%
%:%1341=643%:%
%:%1342=644%:%
%:%1343=644%:%
%:%1344=645%:%
%:%1345=645%:%
%:%1346=646%:%
%:%1347=646%:%
%:%1348=647%:%
%:%1349=647%:%
%:%1350=648%:%
%:%1351=649%:%
%:%1352=650%:%
%:%1353=650%:%
%:%1354=651%:%
%:%1355=651%:%
%:%1356=652%:%
%:%1357=652%:%
%:%1358=653%:%
%:%1359=653%:%
%:%1360=654%:%
%:%1361=654%:%
%:%1362=655%:%
%:%1363=655%:%
%:%1364=656%:%
%:%1365=656%:%
%:%1366=657%:%
%:%1367=657%:%
%:%1368=658%:%
%:%1369=658%:%
%:%1370=659%:%
%:%1371=659%:%
%:%1372=660%:%
%:%1373=660%:%
%:%1374=661%:%
%:%1375=661%:%
%:%1376=662%:%
%:%1377=662%:%
%:%1378=663%:%
%:%1379=664%:%
%:%1380=664%:%
%:%1381=665%:%
%:%1382=665%:%
%:%1383=666%:%
%:%1384=666%:%
%:%1385=667%:%
%:%1386=667%:%
%:%1387=668%:%
%:%1388=669%:%
%:%1389=669%:%
%:%1390=670%:%
%:%1391=670%:%
%:%1392=671%:%
%:%1393=671%:%
%:%1394=672%:%
%:%1395=672%:%
%:%1396=673%:%
%:%1397=673%:%
%:%1398=673%:%
%:%1399=674%:%
%:%1400=674%:%
%:%1401=675%:%
%:%1402=675%:%
%:%1403=676%:%
%:%1404=676%:%
%:%1405=677%:%
%:%1406=677%:%
%:%1407=677%:%
%:%1408=678%:%
%:%1409=678%:%
%:%1410=679%:%
%:%1411=679%:%
%:%1412=680%:%
%:%1413=680%:%
%:%1414=681%:%
%:%1415=681%:%
%:%1416=681%:%
%:%1417=681%:%
%:%1418=681%:%
%:%1419=682%:%
%:%1420=683%:%
%:%1421=683%:%
%:%1422=684%:%
%:%1423=684%:%
%:%1424=685%:%
%:%1425=685%:%
%:%1426=686%:%
%:%1427=686%:%
%:%1428=687%:%
%:%1429=687%:%
%:%1430=688%:%
%:%1431=688%:%
%:%1432=689%:%
%:%1433=689%:%
%:%1434=690%:%
%:%1435=690%:%
%:%1436=691%:%
%:%1437=691%:%
%:%1438=691%:%
%:%1439=691%:%
%:%1440=692%:%
%:%1441=693%:%
%:%1442=693%:%
%:%1443=693%:%
%:%1444=693%:%
%:%1445=693%:%
%:%1446=694%:%
%:%1447=694%:%
%:%1448=695%:%
%:%1449=695%:%
%:%1450=696%:%
%:%1451=696%:%
%:%1452=697%:%
%:%1453=697%:%
%:%1454=698%:%
%:%1455=698%:%
%:%1456=699%:%
%:%1457=699%:%
%:%1458=700%:%
%:%1459=700%:%
%:%1460=701%:%
%:%1461=702%:%
%:%1462=702%:%
%:%1463=703%:%
%:%1464=703%:%
%:%1465=704%:%
%:%1466=704%:%
%:%1467=705%:%
%:%1468=706%:%
%:%1469=706%:%
%:%1470=707%:%
%:%1471=707%:%
%:%1472=708%:%
%:%1473=708%:%
%:%1474=709%:%
%:%1475=709%:%
%:%1476=710%:%
%:%1477=710%:%
%:%1478=711%:%
%:%1479=711%:%
%:%1480=712%:%
%:%1481=712%:%
%:%1482=713%:%
%:%1483=713%:%
%:%1484=714%:%
%:%1485=714%:%
%:%1486=715%:%
%:%1487=715%:%
%:%1488=715%:%
%:%1489=716%:%
%:%1490=716%:%
%:%1491=717%:%
%:%1492=717%:%
%:%1493=718%:%
%:%1494=718%:%
%:%1495=719%:%
%:%1496=719%:%
%:%1497=720%:%
%:%1498=720%:%
%:%1499=721%:%
%:%1500=721%:%
%:%1501=722%:%
%:%1502=722%:%
%:%1503=723%:%
%:%1504=723%:%
%:%1505=724%:%
%:%1506=724%:%
%:%1507=724%:%
%:%1508=724%:%
%:%1509=725%:%
%:%1510=725%:%
%:%1511=725%:%
%:%1512=726%:%
%:%1513=726%:%
%:%1514=727%:%
%:%1515=727%:%
%:%1516=728%:%
%:%1517=728%:%
%:%1518=729%:%
%:%1524=729%:%
%:%1527=730%:%
%:%1528=731%:%
%:%1529=732%:%
%:%1530=733%:%
%:%1531=734%:%
%:%1532=735%:%
%:%1533=736%:%
%:%1534=737%:%
%:%1535=737%:%
%:%1542=738%:%

%
\begin{isabellebody}%
\setisabellecontext{ZF{\isacharunderscore}{\kern0pt}notAC}%
%
\isadelimtheory
%
\endisadelimtheory
%
\isatagtheory
\isacommand{theory}\isamarkupfalse%
\ ZF{\isacharunderscore}{\kern0pt}notAC\ \isanewline
\ \ \isakeyword{imports}\ NotAC{\isacharunderscore}{\kern0pt}Proof\ \isanewline
\isakeyword{begin}%
\endisatagtheory
{\isafoldtheory}%
%
\isadelimtheory
\isanewline
%
\endisadelimtheory
\isanewline
\isanewline
\isacommand{theorem}\isamarkupfalse%
\ ZF{\isacharunderscore}{\kern0pt}notAC{\isacharunderscore}{\kern0pt}main{\isacharunderscore}{\kern0pt}theorem\ {\isacharcolon}{\kern0pt}\isanewline
\ \ \isakeyword{fixes}\ M\ \isanewline
\ \ \isakeyword{assumes}\ {\isachardoublequoteopen}nat\ {\isasymapprox}\ M{\isachardoublequoteclose}\ {\isachardoublequoteopen}M\ {\isasymTurnstile}\ ZF{\isachardoublequoteclose}\ {\isachardoublequoteopen}Transset{\isacharparenleft}{\kern0pt}M{\isacharparenright}{\kern0pt}{\isachardoublequoteclose}\ \isanewline
\ \ \isakeyword{shows}\ {\isachardoublequoteopen}{\isasymexists}N{\isachardot}{\kern0pt}\ N\ {\isasymTurnstile}\ ZF\ {\isasymand}\ {\isasymnot}{\isacharparenleft}{\kern0pt}{\isasymforall}A\ {\isasymin}\ N{\isachardot}{\kern0pt}\ {\isasymexists}R\ {\isasymin}\ N{\isachardot}{\kern0pt}\ wellordered{\isacharparenleft}{\kern0pt}{\isacharhash}{\kern0pt}{\isacharhash}{\kern0pt}N{\isacharcomma}{\kern0pt}\ A{\isacharcomma}{\kern0pt}\ R{\isacharparenright}{\kern0pt}{\isacharparenright}{\kern0pt}{\isachardoublequoteclose}\ \isanewline
%
\isadelimproof
%
\endisadelimproof
%
\isatagproof
\isacommand{proof}\isamarkupfalse%
\ {\isacharminus}{\kern0pt}\ \isanewline
\isanewline
\ \ \isacommand{obtain}\isamarkupfalse%
\ enum\ \isakeyword{where}\ {\isachardoublequoteopen}enum\ {\isasymin}\ bij{\isacharparenleft}{\kern0pt}nat{\isacharcomma}{\kern0pt}\ M{\isacharparenright}{\kern0pt}{\isachardoublequoteclose}\ \isacommand{using}\isamarkupfalse%
\ assms\ eqpoll{\isacharunderscore}{\kern0pt}def\ \isacommand{by}\isamarkupfalse%
\ auto\isanewline
\ \ \isacommand{then}\isamarkupfalse%
\ \isacommand{interpret}\isamarkupfalse%
\ M{\isacharunderscore}{\kern0pt}ctm\ M\ enum\ \isanewline
\ \ \ \ \isacommand{unfolding}\isamarkupfalse%
\ M{\isacharunderscore}{\kern0pt}ctm{\isacharunderscore}{\kern0pt}def\ M{\isacharunderscore}{\kern0pt}ctm{\isacharunderscore}{\kern0pt}axioms{\isacharunderscore}{\kern0pt}def\isanewline
\ \ \ \ \isacommand{using}\isamarkupfalse%
\ M{\isacharunderscore}{\kern0pt}ZF{\isacharunderscore}{\kern0pt}iff{\isacharunderscore}{\kern0pt}M{\isacharunderscore}{\kern0pt}satT\ assms\ \isanewline
\ \ \ \ \isacommand{by}\isamarkupfalse%
\ auto\isanewline
\ \ \isacommand{interpret}\isamarkupfalse%
\ M{\isacharunderscore}{\kern0pt}symmetric{\isacharunderscore}{\kern0pt}system\ Fn\ Fn{\isacharunderscore}{\kern0pt}leq\ {\isadigit{0}}\ M\ enum\ Fn{\isacharunderscore}{\kern0pt}perms\ Fn{\isacharunderscore}{\kern0pt}perms{\isacharunderscore}{\kern0pt}filter\ \ \isanewline
\ \ \ \ \isacommand{using}\isamarkupfalse%
\ Fn{\isacharunderscore}{\kern0pt}M{\isacharunderscore}{\kern0pt}symmetric{\isacharunderscore}{\kern0pt}system\isanewline
\ \ \ \ \isacommand{by}\isamarkupfalse%
\ auto\isanewline
\isanewline
\ \ \isacommand{obtain}\isamarkupfalse%
\ G\ \isakeyword{where}\ GH{\isacharcolon}{\kern0pt}\ {\isachardoublequoteopen}M{\isacharunderscore}{\kern0pt}generic{\isacharparenleft}{\kern0pt}G{\isacharparenright}{\kern0pt}{\isachardoublequoteclose}\ \isanewline
\ \ \ \ \isacommand{using}\isamarkupfalse%
\ generic{\isacharunderscore}{\kern0pt}filter{\isacharunderscore}{\kern0pt}existence\ zero{\isacharunderscore}{\kern0pt}in{\isacharunderscore}{\kern0pt}Fn\ \isanewline
\ \ \ \ \isacommand{by}\isamarkupfalse%
\ auto\isanewline
\ \ \isacommand{then}\isamarkupfalse%
\ \isacommand{interpret}\isamarkupfalse%
\ M{\isacharunderscore}{\kern0pt}symmetric{\isacharunderscore}{\kern0pt}system{\isacharunderscore}{\kern0pt}G{\isacharunderscore}{\kern0pt}generic\ Fn\ Fn{\isacharunderscore}{\kern0pt}leq\ {\isadigit{0}}\ M\ enum\ Fn{\isacharunderscore}{\kern0pt}perms\ Fn{\isacharunderscore}{\kern0pt}perms{\isacharunderscore}{\kern0pt}filter\ G\isanewline
\ \ \ \ \isacommand{unfolding}\isamarkupfalse%
\ M{\isacharunderscore}{\kern0pt}symmetric{\isacharunderscore}{\kern0pt}system{\isacharunderscore}{\kern0pt}G{\isacharunderscore}{\kern0pt}generic{\isacharunderscore}{\kern0pt}def\isanewline
\ \ \ \ \isacommand{using}\isamarkupfalse%
\ M{\isacharunderscore}{\kern0pt}symmetric{\isacharunderscore}{\kern0pt}system{\isacharunderscore}{\kern0pt}axioms\ G{\isacharunderscore}{\kern0pt}generic{\isacharunderscore}{\kern0pt}def\ forcing{\isacharunderscore}{\kern0pt}data{\isacharunderscore}{\kern0pt}axioms\ G{\isacharunderscore}{\kern0pt}generic{\isacharunderscore}{\kern0pt}axioms{\isacharunderscore}{\kern0pt}def\ \isanewline
\ \ \ \ \isacommand{by}\isamarkupfalse%
\ auto\isanewline
\isanewline
\ \ \isacommand{define}\isamarkupfalse%
\ N\ \isakeyword{where}\ {\isachardoublequoteopen}N\ {\isasymequiv}\ SymExt{\isacharparenleft}{\kern0pt}G{\isacharparenright}{\kern0pt}{\isachardoublequoteclose}\ \isanewline
\isanewline
\ \ \isacommand{have}\isamarkupfalse%
\ N{\isacharunderscore}{\kern0pt}ZF\ {\isacharcolon}{\kern0pt}\ {\isachardoublequoteopen}N\ {\isasymTurnstile}\ ZF{\isachardoublequoteclose}\ \isanewline
\ \ \ \ \isacommand{using}\isamarkupfalse%
\ N{\isacharunderscore}{\kern0pt}def\ SymExt{\isacharunderscore}{\kern0pt}M{\isacharunderscore}{\kern0pt}ZF\ M{\isacharunderscore}{\kern0pt}ZF{\isacharunderscore}{\kern0pt}iff{\isacharunderscore}{\kern0pt}M{\isacharunderscore}{\kern0pt}satT\ \isanewline
\ \ \ \ \isacommand{by}\isamarkupfalse%
\ auto\isanewline
\isanewline
\ \ \isacommand{have}\isamarkupfalse%
\ {\isachardoublequoteopen}{\isasymexists}A\ {\isasymin}\ N{\isachardot}{\kern0pt}\ {\isasymforall}R\ {\isasymin}\ N{\isachardot}{\kern0pt}\ {\isasymnot}wellordered{\isacharparenleft}{\kern0pt}{\isacharhash}{\kern0pt}{\isacharhash}{\kern0pt}N{\isacharcomma}{\kern0pt}\ A{\isacharcomma}{\kern0pt}\ R{\isacharparenright}{\kern0pt}{\isachardoublequoteclose}\isanewline
\ \ \ \ \isacommand{apply}\isamarkupfalse%
{\isacharparenleft}{\kern0pt}rule{\isacharunderscore}{\kern0pt}tac\ x{\isacharequal}{\kern0pt}{\isachardoublequoteopen}binmap{\isacharparenleft}{\kern0pt}G{\isacharparenright}{\kern0pt}{\isachardoublequoteclose}\ \isakeyword{in}\ bexI{\isacharparenright}{\kern0pt}\isanewline
\ \ \ \ \ \isacommand{apply}\isamarkupfalse%
{\isacharparenleft}{\kern0pt}rule\ ballI{\isacharcomma}{\kern0pt}\ rule\ ccontr{\isacharparenright}{\kern0pt}\isanewline
\ \ \ \ \ \isacommand{apply}\isamarkupfalse%
{\isacharparenleft}{\kern0pt}rule\ no{\isacharunderscore}{\kern0pt}wellorder{\isacharparenright}{\kern0pt}\isanewline
\ \ \ \ \isacommand{using}\isamarkupfalse%
\ GH\ N{\isacharunderscore}{\kern0pt}def\ SymExt{\isacharunderscore}{\kern0pt}def\ binmap{\isacharunderscore}{\kern0pt}eq\ binmap{\isacharprime}{\kern0pt}{\isacharunderscore}{\kern0pt}HS\isanewline
\ \ \ \ \isacommand{by}\isamarkupfalse%
\ auto\ \isanewline
\ \ \isacommand{then}\isamarkupfalse%
\ \isacommand{show}\isamarkupfalse%
\ {\isacharquery}{\kern0pt}thesis\ \isacommand{using}\isamarkupfalse%
\ N{\isacharunderscore}{\kern0pt}ZF\ \isacommand{by}\isamarkupfalse%
\ auto\isanewline
\isacommand{qed}\isamarkupfalse%
%
\endisatagproof
{\isafoldproof}%
%
\isadelimproof
\isanewline
%
\endisadelimproof
\isanewline
\isanewline
%
\isadelimtheory
\isanewline
%
\endisadelimtheory
%
\isatagtheory
\isacommand{end}\isamarkupfalse%
%
\endisatagtheory
{\isafoldtheory}%
%
\isadelimtheory
%
\endisadelimtheory
%
\end{isabellebody}%
\endinput
%:%file=~/source/repos/ZF-notAC/code/ZF_notAC.thy%:%
%:%10=1%:%
%:%11=1%:%
%:%12=2%:%
%:%13=3%:%
%:%18=3%:%
%:%21=4%:%
%:%22=5%:%
%:%23=6%:%
%:%24=6%:%
%:%25=7%:%
%:%26=8%:%
%:%27=9%:%
%:%34=10%:%
%:%35=10%:%
%:%36=11%:%
%:%37=12%:%
%:%38=12%:%
%:%39=12%:%
%:%40=12%:%
%:%41=13%:%
%:%42=13%:%
%:%43=13%:%
%:%44=14%:%
%:%45=14%:%
%:%46=15%:%
%:%47=15%:%
%:%48=16%:%
%:%49=16%:%
%:%50=17%:%
%:%51=17%:%
%:%52=18%:%
%:%53=18%:%
%:%54=19%:%
%:%55=19%:%
%:%56=20%:%
%:%57=21%:%
%:%58=21%:%
%:%59=22%:%
%:%60=22%:%
%:%61=23%:%
%:%62=23%:%
%:%63=24%:%
%:%64=24%:%
%:%65=24%:%
%:%66=25%:%
%:%67=25%:%
%:%68=26%:%
%:%69=26%:%
%:%70=27%:%
%:%71=27%:%
%:%72=28%:%
%:%73=29%:%
%:%74=29%:%
%:%75=30%:%
%:%76=31%:%
%:%77=31%:%
%:%78=32%:%
%:%79=32%:%
%:%80=33%:%
%:%81=33%:%
%:%82=34%:%
%:%83=35%:%
%:%84=35%:%
%:%85=36%:%
%:%86=36%:%
%:%87=37%:%
%:%88=37%:%
%:%89=38%:%
%:%90=38%:%
%:%91=39%:%
%:%92=39%:%
%:%93=40%:%
%:%94=40%:%
%:%95=41%:%
%:%96=41%:%
%:%97=41%:%
%:%98=41%:%
%:%99=41%:%
%:%100=42%:%
%:%106=42%:%
%:%109=43%:%
%:%110=44%:%
%:%113=45%:%
%:%118=46%:%



% optional bibliography
\bibliographystyle{root}
\bibliography{root}

\end{document}

%%% Local Variables:
%%% mode: latex
%%% TeX-master: t
%%% End:
